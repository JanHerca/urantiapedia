\begin{document}


\title{يوسف وأسينات}

\chapter{1}

\par \textit{اعتراف وصلاة أسينات، ابنة بنتفرس الكاهن}

\par \textit{يُبحث عن أسيناث للزواج من قبل ابن الملك والعديد من الآخرين.}

\par 1 في السنة الأولى من الرخاء، في الشهر الثاني، في اليوم الخامس من الشهر، أرسل فرعون يوسف ليطوف بجميع أرض مصر؛ وفي الشهر الرابع من السنة الأولى، في اليوم الثامن عشر من الشهر،

\par 2 وصل يوسف إلى حدود هليوبوليس،

\par 3 وكان يجمع قمح تلك البلاد كرمل البحر.

\par 4 وكان في تلك المدينة رجل اسمه بنتفرس، وهو كاهن من أهليوبوليس ومرزبان فرعون، ورئيس جميع مرزبان وأمراء فرعون

\par 5 وكان هذا الرجل غنيًا جدًا وحكيمًا ووديعًا جدًا، وكان أيضًا مستشارًا لفرعون، لأنه كان حكيمًا أكثر من جميع أمراء فرعون

\par 6 وكانت له ابنة عذراء اسمها أسنات، عمرها ثماني عشرة سنة، طويلة وجميلة، وجميلة المنظر تفوق كل عذراء على الأرض

\par 7 وأما أسنات نفسها فلم تكن تشبه العذارى بنات المصريين، بل كانت في كل شيء تشبه بنات العبرانيين،

\par 8 طويلة مثل سارة، ووسيمة مثل ريبيكا، وجميلة مثل راشيل؛

\par 9 وانتشر صيت جمالها في كل تلك الأرض وإلى أقاصي العالم، حتى أن جميع أبناء الأمراء والولاة رغبوا في خطبتها، بل وأبناء الملوك أيضًا، كل الشبان والأقوياء،

\par 10 وكان بينهم نزاع عظيم بسببها، وحاولوا القتال فيما بينهم

\par 11 فسمع بها أيضًا ابن فرعون البكر، فطلب من أبيه أن يزوجها له

\par 12 ويقول له: "أعطني يا أبي، أسنات، ابنة بنتفرس، أول رجل من هليوبوليس زوجة".

\par 13 فقال له أبوه فرعون: لماذا تطلب زوجة أدنى منك وأنت ملك كل هذه الأرض؟

\par 14 بل هوذا ابنة يهوياقيم ملك موآب مخطوبة لك، وهي ملكة وجميلة المنظر جدًا. فخذ هذه لنفسك زوجة

\chapter{2}

\par \textit{يُوصف البرج الذي تعيش فيه أسنات.}

\par 1 لكن أسنات استهزأت بكل رجل واحتقرته، وكانت متباهية ومتغطرسة، ولم يرَها أحد قط، لأن بنتفريس كان لديه في منزله برج مجاور، عظيم وعالي للغاية،

\par 2 وفوق البرج كان هناك علية تحتوي على عشر غرف.

\par 3 "وكانت الغرفة الأولى عظيمة وجميلة جداً ومفروشة بحجارة أرجوانية، وكانت جدرانها مكسوة بحجارة كريمة ملونة،

\par 4 وكان سقف تلك الغرفة من ذهب. وفي تلك الغرفة نُصبت آلهة مصرية، لا يُحصى عددها، من ذهب وفضة.

\par 5 وكانت أسنات تعبد جميع أولئك وكانت تخافهم وتقدم لهم الذبائح كل يوم.

\par 6 واحتوت الغرفة الثانية أيضًا على جميع زينة أسنات وصناديقها،

\par 7 وكان فيها ذهب، وفضة كثيرة، وملابس منسوجة بالذهب لا تُحصى، وأحجار كريمة مختارة وثمينة،

\par 8 وثيابًا من كتان، وكل زينة عذريتها كانت هناك

\par 9 وكانت الغرفة الثالثة مخزن أسنات، وتحتوي على كل خيرات الأرض

\par 10 والغرف السبع المتبقية التي سكنتها العذارى السبع اللواتي كن يخدمن أسنات،

\par 11 كان لكل منهما حجرة واحدة، لأنهما كانا من نفس العمر، وُلدا في نفس الليلة مع أسنات، وكانت تحبهما كثيرًا؛ وكانا أيضًا جميلين للغاية كنجوم السماء، ولم يتحدث معهما رجل قط أو طفل ذكر

\par 12 وكان لغرفة أسنات الكبيرة حيث كانت تُحفظ عذريتها ثلاث نوافذ؛

\par 13 وكانت النافذة الأولى كبيرة جدًا، تطل على الفناء من جهة الشرق؛ والثانية تطل على الجنوب، والثالثة تطل على الشارع

\par 14 وكان سرير من ذهب في الغرفة ينظر نحو الشرق؛

\par 15 وكان السرير مفروشا بأرجوان منسوج بذهب، وكان السرير منسوجا من قرمز وقرمز وكتان ناعم.

\par 16 على هذا السرير، نامت أسيناث وحدها، ولم يجلس عليه رجل أو امرأة أخرى قط

\par 17 وكان هناك أيضًا فناء كبير مُلاصق للبيت من جميع الجهات، وسور عالٍ جدًا حول الفناء مبني من حجارة كبيرة مستطيلة الشكل؛

\par 18 وكان هناك أيضًا أربعة أبواب في الدار مُغشّاة بالحديد، وكان يحرس كل منها ثمانية عشر شابًا قويًا مُسلّحًا؛

\par 19 وكانت مغروسة أيضًا على طول السور أشجار جميلة من كل نوع، وكلها مثمرة، وثمارها ناضجة، لأنه كان موسم الحصاد؛

\par 20 وكان هناك أيضًا ينبوع ماء غني ينبع من يمين الفناء نفسه؛ وتحت الينبوع كان هناك صهريج كبير يستقبل مياه ذلك الينبوع، ومنه كان يجري، كما كان، نهر عبر وسط الفناء، وكان يسقي جميع أشجار ذلك الفناء

\chapter{3}

\textit{يوسف يعلن عن سكه لبنتفريس.}

\par 1 وحدث في السنة الأولى من سبع سني الشبع، في الشهر الرابع، في اليوم الثامن والعشرين من الشهر، أن يوسف جاء إلى حدود هليوبوليس ليجمع قمح تلك المنطقة.

\par 2 ولما اقترب يوسف من تلك المدينة، أرسل اثني عشر رجلاً أمامه إلى بنتفريس، كاهن هليوبوليس، قائلاً: "سآتي إليك اليوم، لأنه وقت الظهر ووقت العشاء،

\par 3 وهناك حرارة شمس شديدة، ولكي أتبرد تحت سقف بيتك

\par 4 فلما سمع بنتفرس هذه الأمور، فرح فرحًا عظيمًا جدًا، وقال:

\par 5 «تبارك الرب إله يوسف، لأن سيدي يوسف قد حسبني أهلاً». فدعا بنتفرس القائم على بيته وقال له:

\par 6 «أسرعي وجهزي بيتي، وجهزي عشاءً عظيمًا، لأن يوسف، قديس الله، يأتي إلينا اليوم.»

\par 7 ولما سمعت أسنات أن أباها وأمها قد خرجا من حيازة ميراثهما،

\par 8 فرحت فرحًا عظيمًا وقالت: "سأذهب وأرى أبي وأمي، لأنهما خرجا من ملك ميراثنا" (لأنه كان موسم الحصاد).

\par 9 فأسرعت أسنات إلى مخدعها حيث كانت ثيابها، وارتدت ثوبًا من كتان ناعم مصنوع من قرمز منسوج بذهب، وتنطقت بمنطقة من ذهب، وأساور حول يديها، ووضعت حول قدميها أحزمة من ذهب،

\par 10 وصاغت حول عنقها حلية ثمينة وأحجارًا كريمة، مزخرفة من جميع الجوانب، كما نقشت عليها أسماء آلهة المصريين في كل مكان، سواء على الأساور أو الأحجار؛

\par 11 ووضعت أيضًا تاجًا على رأسها ولفت إكليلًا حول صدغيها وغطت رأسها بعباءة

\chapter{4}

\textit{يقترح بنتفريس أن يزوج أسنات ليوسف.}

\par 1 ثم أسرعت ونزلت الدرج من عليتها وجاءت إلى أبيها وأمها وقبلتهما

\par 2 وفرح بنتفرس وزوجته بابنتهما أسنات فرحًا عظيمًا، لأنهما رأياها مزينة ومُزينة كعروس الله

\par 3 وأخرجوا كل الخير الذي أحضروه من ممتلكات ميراثهم وأعطوه لابنتهم؛

\par 4 وفرحَت أسنات بكل الأشياء الطيبة، بفاكهة الصيف المتأخرة والعنب والتمر والحمام والتوت والتين، لأنها كانت جميعها حسنة المذاق ولذيذة المذاق

\par 5 فقال بنتفرس لابنته أسنات: "يا ابنتي." فقالت: "ها أنا ذا يا سيدي."

\par 6 فقال لها: اجلسي بيننا فأكلمك بكلامي. فجلست بين أبيها وأمها.

\par 7 فأمسك بنتفريس أبوها بيدها اليمنى وقبلها بحنان وقال: يا ابنتي العزيزة. فقالت له: ها أنا ذا يا سيدي أبي.

\par 8 فقال لها بنتفريس: هوذا يوسف، قديس الله، يأتي إلينا اليوم، وهو حاكم على كل أرض مصر، وقد جعله الملك فرعون حاكمًا على كل أرضنا وملكًا، وهو نفسه يعطي القمح لكل هذه البلاد، وينقذها من المجاعة القادمة.

\par 9 وهذا يوسف رجل عابد لله، وحكيم وعذراء مثلك اليوم، ورجل مقتدر في الحكمة والمعرفة، وروح الله عليه ونعمة الرب فيه.

\par 10 تعالي يا حبيبتي، وسأجعلك زوجة له، وستكونين له عروسًا، وهو سيكون عريسك إلى الأبد.

\par 11 ولما سمعت أسنات هذا الكلام من أبيها، عرقت كثيرا على وجهها، وغضبت غضبا عظيما،

\par 12 ونظرت شزرًا بعينيها إلى أبيها وقالت: "لماذا يا سيدي أبي تتكلم بهذه الكلمات؟ هل تريد أن تسلمني أسيرة لأجنبي وهارب ومباع؟"

\par 13 أليس هذا ابن الراعي من أرض كنعان؟ وقد تركه هو أيضًا

\par 14 أليس هذا هو الذي اضطجع مع سيدته، فألقاه سيده في سجن الظلمة، فأخرجه فرعون من السجن، لأنه فسر حلمه كما تفسره أيضًا عجوزات المصريين؟

\par 15 كلا، بل سأتزوج من الابن البكر للملك، لأنه هو نفسه ملك كل الأرض

\par 16 عندما سمع بنتفرس هذه الأمور، خجل من أن يتحدث أكثر مع ابنته أسنات عن يوسف، لأنها أجابته بتباهي وغضب

\chapter{5}

\textit{يصل يوسف إلى منزل بنتفريس.}

\par 1 وإذا شاب من عبيد بنتفريس قد قفز إلى الداخل وقال له:

\par 2 «هوذا يوسف واقف أمام أبواب دارنا». فلما سمعت أسنات هذا الكلام، هربت من وجه أبيها وأمها وصعدت إلى العلية، ودخلت حجرتها ووقفت عند النافذة الكبيرة المطلة على الشرق لترى يوسف قادمًا إلى بيت أبيها

\par 3 فخرج بنتفرس وزوجته وجميع عشيرتهم وعبيدهم لاستقبال يوسف

\par 4 ولما انفتحت أبواب الدار التي تتجه شرقًا، دخل يوسف جالسًا في مركبة فرعون الثانية؛

\par 5 وكانت أربعة خيول مربوطة بنيران بيضاء كالثلج، ذات لجام من ذهب، وكانت المركبة كلها مصنوعة من ذهب خالص

\par 6 وكان يوسف يرتدي قميصًا أبيضَ نادرًا، والرداء الذي يُلف عليه كان أرجوانيًا، مصنوعًا من كتان ناعم منسوج بذهب، وعلى رأسه إكليل من ذهب، وحول إكليله اثنا عشر حجرًا مختارًا، وفوق الحجارة اثنا عشر شعاعًا من ذهب،

\par 7 وفي يده اليمنى عصا ملكية، وفيها غصن زيتون ممدود، وعليه ثمر كثير

\par 8 فلما دخل يوسف إلى الدار وأُغلقت أبوابها،

\par 9 وبقي كل رجل وامرأة غريبين خارج الدار، لأن حراس البوابات اقتربوا وأغلقوا الأبواب،

\par 10 جاء بنتفرس وزوجته وجميع عشيرتهما ما عدا ابنتهما أسنات، وسجدوا ليوسف على وجوههم على الأرض

\par 11 فنزل يوسف من مركبته وسلم عليهم بيده

\chapter{6}

\textit{أسنات ترى يوسف من النافذة.}

\par 1 ولما رأت أسنات يوسف، انتابها وخز شديد في نفسها وانسحق قلبها،

\par 2 فانفكت ركبتاها وارتجف جسدها كله، وخافت خوفًا عظيمًا، ثم تأوهت وقالت في قلبها: "ويل لي يا بائسة! أين أذهب الآن أنا البائسة؟ أو أين أختفي عن وجهه؟ أو كيف يراني يوسف ابن الله، وقد تكلمت عنه بسوء؟ ويل لي يا بائسة!"

\par 3 إلى أين أذهب وأختبئ، لأنه هو نفسه يرى كل مخبأ، ويعلم كل شيء، ولا يفلت منه شيء خفي بسبب النور العظيم الذي فيه؟

\par 4 والآن فليرحمني إله يوسف لأني تكلمت عليه بسوء عن جهل

\par 5 فماذا أتبع الآن، أنا الشقي؟ ألم أقل: «يوسف قادم ابن الراعي من أرض كنعان»؟ والآن فقد جاء إلينا في مركبته كالشمس من السماء، ودخل بيتنا اليوم، وهو يضيء فيه كالنور على الأرض

\par 6 ولكني جاهل وجريء، لأني احتقرته وتكلمت عنه كلامًا شريرًا، ولم أعلم أن يوسف ابن الله

\par 7 فمن من الرجال سينجب مثل هذا الجمال، أو أي رحم امرأة سيلد مثل هذا النور؟ أنا بائس وحمقاء، لأني تكلمت بكلمات شريرة مع أبي

\par 8 فالآن فليعطني أبي ليوسف خادمةً وأمةً بالحري، فأكون له عبدًا إلى الأبد

\chapter{7}

\par \textit{يرى جوزيف أسينات عند النافذة.}

\par 1 فدخل يوسف بيت بنتفرس وجلس على كرسي، فغسلوا رجليه، وأعدّوا له مائدة على انفراد، لأن يوسف لم يأكل مع المصريين، لأن ذلك كان رجسًا عنده.

\par 2 فرفع يوسف نظره فرأى أسنات تطل، فقال لبنتفرس: "من هذه المرأة التي تقف في العلية بجانب النافذة؟ فلتخرج من هذا البيت."

\par 3 فخاف يوسف وقال: «لئلا تضايقني هي أيضاً». لأن جميع نساء وبنات الرؤساء والولاة في كل أرض مصر كن يضايقنه لكي يضطهدن معه.

\par 4 ولكن كثيرات من نساء وبنات المصريين أيضا، كل من رأى يوسف، حزنن بسبب جماله.

\par 5 "والرسل الذين أرسلتهم النساء إليه بالذهب والفضة والهدايا الثمينة، ردهم يوسف بالتهديد والإهانة قائلاً: لا أخطئ أمام الرب الإله ووجه أبي إسرائيل."

\par 6 "فإن الله كان أمام يوسف دائماً، وكان يتذكر دائماً أوامر أبيه، وكان يعقوب كثيراً ما يتكلم ويوعظ ابنه يوسف وجميع أبنائه قائلاً: "احفظوا أنفسكم أيها الأولاد من المرأة الأجنبية، لئلا تصاحبوها، لأن شركتها خراب وهلاك".

\par 7 فقال يوسف لتخرج تلك المرأة من هذا البيت.

\par 8 فقال له بنتفريس: يا سيدي، إن المرأة التي رأيتها واقفة في العلية ليست غريبة، بل هي ابنتنا، وهي تكره كل رجل، ولم يرَها أي رجل آخر سواك اليوم فقط؛

\par 9 وإذا أردت يا سيدي، فسوف تأتي وتحدثك، لأن ابنتنا بمثابة أختك

\par 10 ففرح يوسف فرحًا عظيمًا جدًا، لأن بنتفرس قال: "إنها عذراء تبغض كل رجل".

\par 11 فقال يوسف لبنتفرس وامرأته: "إن كانت ابنتكما، وهي عذراء، فلتأتِ، لأنها أختي، وأنا أحبها من اليوم كأختي."

\chapter{8}


\par \textit{يوسف يبارك أسنات.}

\par 1 ثم صعدت أمها إلى العلية وأحضرت أسنات إلى يوسف، فقال لها بنتفريس: "قبلي أخاك، لأنه هو أيضًا عذراء مثلك اليوم، ويكره كل امرأة غريبة كما تكرهين كل رجل غريب".

\par 2 فقالت أسنات ليوسف: "السلام عليك يا رب، مبارك من الله العلي". فقال لها يوسف: "يباركك الله الذي يحيي كل شيء يا صبية".

\par 3 ثم قال بنتفريس لابنته أسنات: "تعالي وقبّلي أخاك."

\par 4 عندما تقدمت أسنات لتقبيل يوسف، مدّ يوسف يده اليمنى ووضعها على صدرها بين ثدييها (لأن ثدييها كانا بارزين بالفعل مثل التفاحتين الجميلتين)، وقال يوسف:

\par 5 «لا يليق بإنسان يعبد الله، ويبارك بفمه الإله الحي، ويأكل خبز الحياة المبارك، ويشرب كأس الخلود المبارك، ويُدهن بمسحة عدم الفساد المباركة، أن يقبل امرأة غريبة تبارك بفمها أصنامًا ميتة وصماء، وتأكل من مائدتها خبز الخنق، وتشرب من سكائبها كأس الخداع، وتُدهن بمسحة الهلاك؛

\par 6 أما الإنسان الذي يعبد الله فسيقبل أمه والأخت المولودة من أمه والأخت المولودة من سبطه وامرأته التي تجلس معه في فراشه، الذين يباركون الله الحي بأفواههم

\par 7 وكذلك أيضًا لا يليق بامرأة تعبد الله أن تقبل رجلاً أجنبيًا، لأن ذلك رجس في عيني الرب الإله

\par 8 فلما سمعت أسنات هذه الكلمات من يوسف، حزنت بشدة وتأوهت، وبينما كانت تنظر إلى يوسف وعيناها مفتوحتان، امتلأتا بالدموع

\par 9 فلما رآها يوسف تبكي أشفق عليها جداً لأنه كان وديعاً ورحيماً ويخاف الرب.

\par 10 ثم رفع يده اليمنى فوق رأسها وقال:

"الرب إله أبي إسرائيل، الإله العلي القدير،
الذي أحيا كل الأشياء ودعا من الظلمة إلى النور
ومن الخطأ إلى الحقيقة ومن الموت إلى الحياة،
بارك هذه العذراء أيضًا،

\par 11 وأحييها وجددها بروحك القدوس.
"فلتأكل خبز حياتك وتشرب كأس بركتك،
وأحصها مع شعبك الذي اخترته قبل أن يخلق كل شيء،
ودعوها تدخل راحتك التي أعددتها لمختاريك،
"ودعها تحيا في حياتك الأبدية إلى الأبد."

\chapter{9}

\par \textit{تتقاعد أسيناث ويستعد جوزيف للمغادرة.}

\par 1 ففرحت أسنات ببركة يوسف فرحًا عظيمًا. ثم أسرعت وصعدت إلى عليتها وحدها، وسقطت على فراشها مريضة، إذ كان في ذلك فرحها وحزنها وخوفها الشديد. وصبَّ عليها عرقٌ لا ينقطع حين سمعت هذه الكلمات من يوسف، وكلمها باسم الله العلي.

\par 2 ثم بكت بكاءً عظيمًا ومريرًا، ورجعت تائبة عن آلهتها التي اعتادت أن تعبدها، وعن الأصنام التي احتقرتها، وانتظرت حلول المساء

\par 3 فأكل يوسف وشرب، وأمر خديمه أن يربطوا الخيل إلى مركباتها، وأن يطوفوا في كل الأرض

\par 4 فقال بنتفريس ليوسف: "ليبُت سيدي هنا اليوم، وفي الصباح تمضي في طريقك."

\par 5 فقال يوسف: "لا، بل سأذهب اليوم، لأن هذا هو اليوم الذي بدأ فيه الله خلق جميع مخلوقاته، وفي اليوم الثامن أعود إليكم أيضًا وأبيت هنا."

\chapter{10}

\par \textit{أسيناث ترفض الآلهة المصرية وتذل نفسها.}

\par 1 ولما خرج يوسف من البيت ذهب بنتفرس أيضا وكل عشيرته إلى ميراثهم،

\par 2 وبقيت أسنات وحدها مع العذارى السبع، بلا حركة وتبكي حتى غربت الشمس؛ ولم تأكل خبزًا ولم تشرب ماءً، ولكن بينما كان الجميع نائمين كانت هي وحدها مستيقظة تبكي وتضرب صدرها بيدها كثيرًا.

\par 3 وبعد هذه الأمور نهضت أسنات من فراشها ونزلت بهدوء على الدرج من العلية، وعندما وصلت إلى البوابة وجدت البوابة نائمة مع أطفالها؛

\par 4 فأسرعت وأزالت من الباب غطاء الستارة الجلدي وملأته بالرماد وحملته إلى العلية ووضعته على الأرض.

\par 5 ثم أغلقت الباب بإحكام وثبتته بالمزلاج الحديدي من الجانب، وتأوهت أنينًا عظيمًا وبكاءً عظيمًا جدًا.

\par 6 أما العذراء التي أحبتها أسنات أكثر من جميع العذارى، فلما سمعت أنينها أسرعت وجاءت إلى الباب بعد أن أيقظت العذارى الأخريات أيضًا، فوجدته مغلقًا

\par 7 ولما استمعت إلى أنين وبكاء أسنات، قالت لها وهي واقفة في الخارج: "ما الأمر يا سيدتي، ولماذا أنتِ حزينة؟ وما الذي يزعجك؟"

\par 8 افتحي لنا ودعينا نراكِ." فقالت لها أسنات وهي محبوسة في الداخل: "لقد أصابني ألم شديد وشديد في رأسي، وأنا مستريحة في فراشي، ولا أستطيع النهوض وفتح الباب لكِ، لأني ضعيفة في جميع أعضائي. فاذهب كل واحدة منكن إلى حجرتها ونم، ودعوني أهدأ."

\par 9 ولما ذهبت العذارى كل واحدة إلى حجرتها، قامت أسنات وفتحت باب حجرتها بهدوء، وذهبت إلى حجرتها الثانية حيث كانت صناديق زينتها، وفتحت صندوقها وأخذت قميصًا أسود داكنًا، فلبسته وحزنت عندما مات أخيها البكر.

\par 10 بعد أن أخذت هذا القميص، حملته إلى حجرتها، وأغلقت الباب مرة أخرى بإحكام، ووضعت المزلاج على الجانب

\par 11 ثم خلعت أسنات رداءها الملكي، وارتدت ثوب الحداد، وحلت حزامها الذهبي، وتنطقت بحبل، وخلعت التاج، أي العمامة، عن رأسها، وكذلك الإكليل، وسلاسل يديها وقدميها كلها موضوعة على الأرض

\par 12 ثم تأخذ ثوبها المُختار ومنطقة الذهب والتاج وتاجها، وتطرحها من النافذة المُطلة نحو الشمال، إلى الفقراء

\par 13 ثم أخذت جميع آلهتها التي في حجرتها، آلهة الذهب والفضة التي لم يكن لها عدد، وكسرتها إلى قطع صغيرة، وألقتها من النافذة للفقراء والمتسولين

\par 14 فأخذت أسنات أيضًا عشاءها الملكي، والمسمّنات، والسمك، ولحم البقرة، وجميع ذبائح آلهتها، وأواني خمر السَّكْب، وألقتها كلها من خلال النافذة التي كانت تطل على الشمال لتكون طعامًا للكلاب

\par 15 وبعد هذه الأشياء، أخذت الغطاء الجلدي الذي يحتوي على الرماد وسكبته على الأرض؛

\par 16 فأخذت مسحًا وشدّت حقويها، وحلّت أيضًا شبكة شعر رأسها ونثرت الرماد على رأسها، ونثرت جمرًا أيضًا على الأرض،

\par 17 وسقطت على الجمر وظلت تضرب صدرها بيديها باستمرار وتبكي طوال الليل مع الأنين حتى الصباح

\par 18 ولما استيقظت أسنات في الصباح ورأت، وإذا الجمر تحتها كالطين من دموعها،

\par 19 سقطت مرة أخرى على وجهها فوق الجمر حتى غربت الشمس.

\par 20 ففعلت أسنات ذلك سبعة أيام ولم تذق شيئا.

\chapter{11}

\par \textit{تقرر أسينات أن تصلي إلى إله العبرانيين.}

\par 1 وفي اليوم الثامن، عندما طلع الفجر وكانت الطيور تغرد والكلاب تنبح على المارة، رفعت أسنات رأسها قليلًا عن الأرض والجمر الذي كانت تجلس عليه، لأنها كانت منهكة للغاية وفقدت قوة أطرافها من شدة إذلالها؛

\par 2 لأن أسنات كانت قد أصبحت متعبة وضعيفة وكانت قوتها تضعف، وعندها التفتت نحو الحائط، جالسة تحت النافذة المطلة على الشرق؛

\par 3 ووضعت رأسها على صدرها، ولفت أصابع يديها على ركبتها اليمنى؛

\par 4 وكان فمها مغلقًا، ولم تفتحه طوال الأيام السبعة والليالي السبع من إذلالها

\par 5 فقالت في قلبها دون أن تفتح فمها: ماذا أفعل أنا المتواضعة، أو إلى أين أذهب؟ ومع من ألجأ بعد الآن؟ أو إلى من أتحدث، العذراء اليتيمة والموحشة المهجورة من الجميع والمكروهة؟

\par 6 لقد أصبح الجميع يكرهونني الآن، ومن بينهم حتى أبي وأمي، لأني كرهت الآلهة ببغض، وتخلصت منهم، وأعطيتهم للفقراء ليهلكهم البشر. لأن أبي وأمي قالا: "أسينات ليست ابنتنا".

\par 7 لكن جميع أقاربي أيضًا كرهوني، وكرهوا جميع البشر، لأني سلّمت آلهتهم للهلاك. وقد كرهت كل إنسان وكل من خطبني، والآن في هذا الإذلال الذي أصابني، كرهني الجميع، وهم يفرحون بمحنتي

\par 8 لكن الرب إله يوسف القدير يبغض كل من يعبد الأصنام، لأنه إله غيور ومرعب، كما سمعت، على كل من يعبد آلهة غريبة؛ لذلك أبغضني أنا أيضًا، لأني عبدت أصنامًا ميتة وصماء وباركتها

\par 9 ولكنني الآن قد تجنبت ذبائحهم، وأصبح فمي غريبًا عن مائدتهم، وليس لدي الشجاعة لأدعو الرب إله السماء، العلي والقدير يوسف القدير، لأن فمي قد تلوث بذبائح الأصنام

\par 10 ولكني سمعت كثيرين يقولون إن إله العبرانيين هو إله حق، وإله حي، وإله رحيم، ورحيم، وطويل الأناة، ورحيم، ووديع، ولا يحسب خطيئة الإنسان المتواضع، وخاصة من يخطئ عن جهل، ولا يوبخ على الإثم في وقت ضيق الإنسان المتضايق

\par 11 وبناءً على ذلك، أنا أيضًا المتواضع، سأكون جريئًا وسألجأ إليه وألتجئ إليه وأعترف له بجميع خطاياي وأسكب طلبتي أمامه، فيرحم بؤسي

\par 12 فمن يدري إن كان سيرى إذلالي وخراب روحي فيشفق عليّ، ويرى أيضًا يتم بؤسي وعذريتي فيدافع عني؟

\par 13 لأنه، كما أسمع، هو نفسه أب للأيتام، وعزاء للمتألمين، ومعين للمضطهدين. ولكن على كل حال، أنا أيضًا المتواضع، سأتشجع وأصرخ إليه

\par 14 ثم نهضت أسنات عن الحائط الذي كانت جالسة فيه، ورفعت نفسها على ركبتيها نحو الشرق، ووجهت عينيها نحو السماء، وفتحت فمها وقالت لله:

\chapter{12}

\par \textit{صلاة أسيناث}

\par 1 صلاة أسيناث واعترافها:

\par 2 «يا رب إله الصالحين، الذي خلق الدهور وأحيا كل شيء،
الذي أعطى نسمة الحياة لكل خليقتك،
الذي أخرج الأشياء غير المرئية إلى النور،
الذي خلق كل الأشياء وأظهر الأشياء التي لم تظهر،

\par 3 الذي رفع السماء وأسس الأرض على المياه،
الذي ثبت الحجارة الكبيرة على هاوية الماء،
الذين لن يغرقوا بل يعملون مشيئتك إلى النهاية،
لأنك أنت يا رب، قلت الكلمة فجاءت كل الأشياء إلى الوجود، وكلمتك يا رب هي حياة كل مخلوقاتك، إليك ألجأ،

\par 4 يا رب إلهي، من الآن فصاعدًا، إليك أصرخ يا رب،
ولك أعترف بخطاياي، إليك أسكب طلبتي يا سيدي،
وأكشف لك آثامي.

\par 5 اعفُ عني يا رب، اعفُ عني، لأني ارتكبتُ في حقك خطايا كثيرة،
ارتكبتُ الإثم والفجور،
لقد تكلمت بأشياء لا يمكن النطق بها، وهي شريرة في عينيك.
لقد تنجست فمي يا رب من ذبائح أصنام المصريين.
ومن مائدة آلهتهم:

\par 6 أخطأتُ يا رب، أخطأتُ أمامَكَ، بمعرفةٍ وجهلٍ
لقد فعلت الشر عندما عبدت الأصنام الميتة والصماء،
ولست أهلاً أن أفتح فمي لك يا رب.

\par 7 أنا أسنات البائسة ابنة بنتفريس الكاهن، العذراء والملكة،
من كان ذات يوم فخورًا ومتغطرسًا وازدهر في ثروات والدي فوق كل الرجال،
لكن الآن أصبح يتيما ووحيدا ومتروكا من قبل جميع الناس.
إليك ألجأ يا رب، وإليك أرفع التماسي،
وإليك أصرخ.

\par 8 نجني من الذين يطاردونني يا سيدي، قبل أن يقبضوا عليّ؛
فكما يهرب الطفل إلى أبيه وأمه خوفًا من شخص ما،
ويمد أبوه يديه ويأخذه إلى صدره،
هكذا أيضًا يا رب، مد يديك الطاهرتين والرهيبتين عليّ كأب محب للأطفال،
وأنقذني من يد العدو فوق الحسي.

\par 9 هوذا الأسد القديم المتوحش القاسي يطاردني،
لأنه أبو آلهة المصريين،
وآلهة المهووسين بالأصنام هم أبناؤه،
وقد أصبحت أكرههم وأتخلص منهم،
لأنهم أبناء الأسد،
فطرحت عني جميع آلهة المصريين وأزلتها.
والأسد، أو أبوه إبليس، في غضبه عليّ يحاول أن يبتلعني.

\par 10 وأنت يا رب، نجني من يديه،
فأُنْقَذُ من فمه،
لئلا يمزقني ويلقيني في لهيب النار،
وألقتني النار في العاصفة،
والعاصفة تتغلب عليّ في الظلمة وتدفعني إلى أعماق البحر،
والوحش العظيم الذي من الأزل يبتلعني،
وأنا أهلك إلى الأبد.

\par 11 نجني يا رب قبل أن تأتي عليّ كل هذه الأمور؛
نجني يا سيدي، أنا الموحش والعاجز،
لأن أبي وأمي أنكراني وقالا:
"آسيناث ليست ابنتنا"
لأني حطمت آلهتهم ودمرتهم.
كأنني كرهتهم تمامًا. والآن أنا يتيمٌّ وحزين، وليس لي رجاءٌ سواك.

\par 12 يا رب، ولا ملجأ آخر سوى رحمتك، يا صديق البشر،
لأنك أنت وحدك أبا الأيتام، وبطل المضطهدين، ومعين المنكوبين.
ارحمني يا رب، واحفظني طاهرًا وبكرًا.
المهجور واليتيم، من أجلك أنت فقط.
يا رب، أنت أب لطيف وطيب ولطيف.
فأي أب لطيف وصالح مثلك يا رب؟
فها هي كل بيوت أبي بنتفرس
التي أعطاني إياها ميراثًا هي إلى حين وانقضاء.
"ولكن بيوت ميراثك يا رب هي غير فاسدة وأبدية."

\chapter{13}

\par \textit{صلاة أسيناث (تابع).}

\par 1 يا رب، زُرْ ذُلَّتي، وارحمْ يُتْمي، وارحمْني أنا المُتألِّم. ها أنا يا سيِّدي، هربتُ من الجميع، ولجأتُ إليكَ يا رفيقَ البشرِ.

\par 2 ها أنا ذا قد تركتُ كل خيرات الأرض ولجأتُ إليك يا رب، في المسوح والرماد، عاريًا ومنعزلًا

\par 3 ها أنا ذا الآن قد خلعت ثوبي الملكي من كتان وقرمز منسوج بذهب، ولبست ثوب حداد أسود. ها أنا ذا قد حللت منطقتي الذهبية وألقيتها عني، وتمنطقت بحبل ومسح

\par 4 ها أنا ذا! ألقيتُ تاجي وقلنسوتي من رأسي، ونثرتُ على نفسي الجمر،

\par 5 هوذا! أرضية غرفتي التي كانت مرصوفة بأحجار متعددة الألوان وأرجوانية، والتي كانت تُبلل بالمراهم وتُجفف بقطع من الكتان اللامع، أصبحت الآن مبللة بدموعي، وقد أُهينت لأنها مُنثرة بالرماد

\par 6 هوذا يا سيدي، من الجمر ودموعي قد تشكل طين كثير في غرفتي كما في طريق واسع

\par 7 انظر يا سيدي، عشائي الملكي واللحوم التي قدمتها للكلاب

\par 8 ها أنا أيضًا يا سيدي، صمتُ سبعة أيام وسبع ليالٍ ولم آكل خبزًا ولم أشرب ماءً، وفمي جافٌّ كعجينة، ولساني كقرن، وشفتاي كشقفة فخار، ووجهي قد تقلص، وعيناي كلتاهما من ذرف الدموع

\par 9 لكنك أنت يا رب إلهي، نجني من جهلي الكثير، واغفر لي ذلك، فقد ضللت، كوني عذراء جاهلة. هوذا! الآن عرفت أن جميع الآلهة التي كنت أعبدها من قبل في جهل كانت أصنامًا صماء ميتة، وقد حطمتها وأسلمتها لتُداس من قبل جميع البشر، ونهبها اللصوص، وهي من الذهب والفضة، وإليك التجأت. أيها الرب الإله، الرحيم الوحيد وصديق البشر

\par 10 سامحني يا رب، لأني ارتكبت خطايا كثيرة ضدك عن جهل، وتكلمت بكلمات تجديفية ضد سيدي يوسف، ولم أعلم أيها البائس أنه ابنك. يا رب، لأن الأشرار حسدًا قالوا لي: يوسف ابن راعٍ من أرض كنعان، وأنا البائس صدقتهم وضللتهم، واستهزأت به وتكلمت عنه بأشياء شريرة، ولم أعلم أنه ابنك

\par 11 فمن من البشر أنجب أو سينجب مثل هذا الجمال؟ أو من مثله، حكيم وعظيم مثل يوسف الجميل؟ إلا إليك يا رب، أستودعه، لأني أحبه أكثر من نفسي.

\par 12 احفظه آمنًا في حكمة نعمتك، وأودعني عنده أمةً وأمةً، لأغسل قدميه وأرتب فراشه وأخدمه وأخدمه، وسأكون أمةً له طوال حياتي

\chapter{14}

\par \textit{رئيس الملائكة ميخائيل يزور أسينات.}

\par 1 ولما فرغت أسنات من الاعتراف للرب، إذا بنجم الصبح قد طلع من السماء في المشرق.

\par 2 ورأت أسنات ذلك ففرحت وقالت: "هل سمع الرب الإله صلاتي إذن؟ لأن هذا النجم رسول ومبشر بنور اليوم العظيم."

\par 3 وإذا السماء قد انشقت بفعل نجم الصبح، وظهر نور عظيم لا يوصف

\par 4 ولما رأت أسنات ذلك، سقطت على وجهها فوق الجمر، وفي الحال جاء إليها رجل من السماء، يُرسل أشعة من نور، ووقف فوق رأسها. وبينما كانت مُستلقية على وجهها، قال لها الملاك الإلهي: "أسنات، قفي."

\par 5 فقالت: «من هو الذي دعاني وباب غرفتي مغلق والبرج مرتفع، فكيف دخل غرفتي؟»

\par 6 فناداها ثانيةً قائلًا: «أسنات، أسنات». فقالت: «هأنذا يا سيدي، أخبرني من أنت».

\par 7 فقال: «أنا رئيس قواد الرب الإله وقائد كل جند العلي. قم وقف على قدميك لأُكلمك بكلامي».

\par 8 فرفعت وجهها ونظرت، وإذا رجل في كل شيء يشبه يوسف، في رداء وإكليل وعصا ملكية،

\par 9 إلا أن وجهه كان كالبرق، وعيناه كنور الشمس، وشعر رأسه كلهب نار مشعل مشتعل، ويداه وقدماه كالحديد اللامع من نار، إذ كان شرر يخرج من يديه وقدميه

\par 10 فلما رأت أسنات هذه الأمور خافت وسقطت على وجهها، غير قادرة حتى على الوقوف على قدميها، لأنها خافت بشدة وارتجفت جميع أطرافها

\par 11 فقال لها الرجل: «ثقي يا أسنات، ولا تخافي، بل قومي وقفي على قدميكِ لأُكلِّمكِ بكلامي».

\par 12 فقامت أسنات ووقفت على رجليها، وقال لها الملاك:

\par 13 «ادخلي إلى حجرتك الثانية دون عائق، واخلعي ​​القميص الأسود الذي ترتدينه، واخلعي ​​المسح عن حقويك، وانفضي الرماد عن رأسك، واغسلي وجهك ويديك بماء طاهر، والبسِ رداءً أبيض لم يُمس، وشدّي حقويك بحزام العذرية اللامع المزدوج،

\par 14 وتعالَ إليَّ أيضًا، فأُكَلِّمكَ بالكلامِ المُرسَلِ إليكَ من قِبَلِ الرَّبِّ

\par 15 "فأسرعت أسنات ودخلت حجرتها الثانية التي فيها صناديق زينتها وفتحت صندوقها وأخذت ثوبا أبيض ناعما غير ملبوس ولبسته بعد أن خلعت الثوب الأسود أولا،

\par 16 ونزعت أيضًا الحبل والمسح عن حقويها، ونطقت بمنطقة عذريتها المزدوجة اللامعة، منطقة على حقويها ومنطقة أخرى على صدرها

\par 17 ونفضت أيضًا الجمر عن رأسها وغسلت يديها ووجهها بماء طاهر، وأخذت رداءً جميلاً وجميلاً وغطت به رأسها

\chapter{15}

\par \textit{يخبر ميخائيل أسنات أنها ستكون زوجة يوسف.}

\par 1 وعندئذٍ أتت إلى رئيس القبطان الإلهي ووقفت أمامه، فقال لها ملاك الرب: "انزعي الآن الرداء عن رأسك، فأنتِ اليوم عذراء طاهرة، ورأسكِ كرأس شاب."

\par 2 وأخذته أسنات من على رأسها. وقال لها الملاك الإلهي مرة أخرى: "تشجعي يا أسنات، أيتها العذراء الطاهرة، لأنه هوذا الرب الإله قد سمع كل كلمات اعترافك وصلاتك، وقد رأى أيضًا إذلال وبؤس أيام امتناعك السبعة، لأنه من دموعك تشكل طين كثير أمام وجهك على هذا الجمر."

\par 3 لذلك، تشجعي يا أسنات، العذراء الطاهرة، لأنه هوذا اسمك قد كُتب في سفر الحياة ولن يُمحى إلى الأبد

\par 4 ولكن من هذا اليوم ستتجدد وتُعاد صياغتك وتُحيى من جديد، وستأكل خبز الحياة المبارك وتشرب كأسًا مملوءة خلودًا، وتُمسح بمسحة عدم الفساد المباركة

\par 5 تشجعي يا أسنات، العذراء الطاهرة، هوذا الرب الإله قد أعطاكِ اليوم ليوسف عروسًا، وهو سيكون عرسكِ إلى الأبد

\par 6 ولن يُدعى اسمك بعد الآن أسنات، بل سيكون اسمك مدينة الملجأ، لأنه فيك ستلتمس أمم كثيرة ملجأً، وسيبيتون تحت جناحيك، وستجد أمم كثيرة مأوىً في طريقك، وعلى أسوارك سيُحفظ أولئك الذين يتمسكون بالله العلي من خلال التوبة آمنين؛

\par 7 لأن التوبة هي ابنة العلي، وهي نفسها تتوسل إلى الله العلي من أجلك في كل ساعة ومن أجل كل من يتوب، لأنه أبو التوبة،

\par 8 وهي نفسها مُكمِّلة جميع العذارى ومشرفة عليها، تُحبكم حبًا جمًا وتتوسل إلى العلي من أجلكم في كل ساعة، ولكل من يتوب ستوفر له مكان راحة في السماوات، وتجدد كل من تاب. والتوبة جميلة جدًا، عذراء طاهرة ولطيفة ووديعة؛ ولذلك يحبها الله العلي، ويحترمها جميع الملائكة، وأنا أحبها حبًا جمًا، لأنها هي أيضًا أختي، وكما تُحبكم أيها العذارى، فأنا أيضًا أحبكم

\par 9 وها أنا ذاهب إلى يوسف وأقول له كل هذا الكلام عنك، فيأتي إليك اليوم ويراك ويفرح بك ويحبك ويكون عريسك، وتكونين عروسه الحبيبة إلى الأبد.

\par 10 لذلك، اسمعي لي يا أسنات، والبسِ رداء العرس، الرداء القديم الأول الذي لا يزال موضوعًا في مخدعك منذ القدم، وضعي حولك أيضًا كل ما تختارينه من زينتك، وتجملي كعروس صالحة وجهزي نفسكِ للقائه؛

\par 11 ها هوذا هو يأتي إليكِ اليوم، وسيراكِ ويفرح

\par 12 ولما فرغ ملاك الرب في صورة إنسان من التكلم بهذه الكلمات لأسنات، فرحت فرحًا عظيمًا بكل ما قاله،

\par 13 وسقطت على وجهها على الأرض، وسجدت أمام قدميه وقالت له: "مبارك الرب إلهك الذي أرسلك لإنقاذي من الظلمة وإحضاري من أساسات الهاوية نفسها إلى النور، ومبارك اسمك إلى الأبد. فإن وجدت نعمة يا سيدي في عينيك وعلمت أنك ستفعل جميع الكلمات التي قلتها لي حتى تكتمل، فلتتحدث إليك أمتك." فقال لها الملاك: "استمري."

\par 14 فقالت: «أرجوك يا سيدي، اجلس قليلًا على هذا السرير، لأن هذا السرير طاهر وغير دنس، إذ لم يجلس عليه رجل أو امرأة قط، وسأضع أمامك مائدة وخبزًا، فتأكل، وسأحضر لك أيضًا خمرًا عتيقة وجيدة، تفوح رائحتها في السماء، فتشرب منها، ثم تمضي في طريقك.» فقال لها: «أسرعي وأتي به سريعًا».

\chapter{16}

\par \textit{أسيناث تجد قرص عسل في مخزنها.}

\par 1 فأسرعت أسنات ووضعت أمامه مائدة فارغة، وبينما كانت تحضر الخبز، قال لها الملاك الإلهي: "أحضري لي أيضًا قرص عسل". فوقفت في حيرة وحزنت لعدم وجود قرص نحل في مخزنها. فقال لها الملاك الإلهي: "لماذا تقفين ساكنة؟"

\par 2 فقالت: «يا سيدي، سأرسل غلامًا إلى الضواحي، لأن امتلاك ميراثنا قريب، فيأتي ويحضر واحدًا من هناك سريعًا، وأضعه أمامك».

\par 3 قال لها الملاك الإلهي: "ادخلي مخزنك وستجدين قرص نحل موضوعًا على الطاولة؛ خذيه وأتي به إلى هنا." فقالت: "يا رب، لا يوجد قرص نحل في مخزني." فقال: "اذهبي وستجدين."

\par 4 فدخلت أسنات مخزنها فوجدت قرص عسل موضوعًا على المائدة. وكان القرص كبيرًا وأبيض كالثلج ومملوءًا عسلًا، وكان ذلك العسل كندى السماء، ورائحته كرائحة حياة. فتعجبت أسنات وقالت في نفسها: "هل هذا القرص من فم هذا الرجل؟"

\par 5 فأخذت أسنات ذلك الشهد وأتت به ووضعته على المائدة، فقال لها الملاك: لماذا قلتِ ليس في بيتي شهد عسل، وها أنتِ أتيتني به؟

\par 6 فقالت: "يا رب، لم أضع شهد عسل في بيتي قط، إلا كما قلت، هكذا صار. هل خرج هذا من فمك؟ لأن رائحته كرائحة المرهم."

\par 7 فابتسم الرجل لفهم المرأة. ثم دعاها إليه، وعندما أتت، مدّ يده اليمنى وأمسك برأسها، وعندما هزّ رأسها بيده اليمنى، خافت أسنات من يد الملاك بشدة، لأن الشرر كان يخرج من يديه كالحديد المشتعل، ولذلك كانت طوال الوقت تحدق في يد الملاك بخوف شديد وترتجف

\par 8 فابتسم وقال: "طوبى لكِ يا أسنات، لأن أسرار الله التي لا تُوصف قد كُشفت لكِ؛ وطوبى لكل من يلتصق بالرب الإله بالتوبة، لأنهم سيأكلون من هذا المشط، لأن هذا المشط هو روح الحياة، وهذا ما صنعته نحل جنة البهجة من ندى ورود الحياة التي في جنة الله وكل زهرة، ومنه يأكل الملائكة وجميع مختاري الله وجميع أبناء العلي، وكل من يأكل منه لن يموت إلى الأبد."

\par 9 ثم مدّ الملاك الإلهي يده اليمنى وأخذ قطعة صغيرة من الشهد وأكلها، ووضع بيده ما تبقى في فم أسنات وقال لها: "كُلي"، فأكلت. فقال لها الملاك: "ها قد أكلتِ الآن خبز الحياة وشربتِ كأس الخلود ومُسحتِ بمسحة عدم الفساد؛

\par 10 هوذا اليوم يُنتج لحمك أزهار الحياة من ينبوع العلي، وستُسمَّن عظامك مثل أرز فردوس نعيم الله، وستدعمك قوى لا تعرف الكلل؛

\par 11 وبناءً على ذلك، لن يرى شبابك شيخوخة، ولن يذبل جمالك إلى الأبد، بل ستكونين كمدينة أم مسورة للجميع

\par 12 وحرض الملاك الشاهد، فنهضت نحل كثيرة من خلايا ذلك الشاهد، وكانت الخلايا لا تُحصى، عشرات الآلاف وعشرات الآلاف وآلاف الآلاف

\par 13 وكانت النحلات أيضًا بيضاء كالثلج، وأجنحتها كالأرجوان والقرمزي، وكانت لها لسعات حادة، ولم تؤذِ أحدًا

\par 14 ثم أحاطت كل تلك النحلات بأسينات من قدميها إلى رأسها، ونهضت نحلات عظيمة أخرى مثل ملكاتها من الخلايا، ودارت حول وجهها وعلى شفتيها، وصنعت مشطًا على فمها وعلى شفتيها مثل المشط الموضوع أمام الملاك، وأكلت كل تلك النحلات من المشط الذي على فم أسينات

\par 15 فقال الملاك للنحل: "اذهبوا الآن إلى مكانكم."

\par 16 ثم نهضت جميع النحلات وطارت وصعدت إلى السماء، أما من أرادت إيذاء أسينات، فسقطت جميعها على الأرض وماتت. ثم مد الملاك عصاه على النحلات الميتة.

\par 17 وقال لهم: «قوموا واذهبوا أنتم أيضًا إلى أماكنكم». فقامت جميع النحلات الميتة وذهبت إلى الدار المجاورة لبيت أسنات، وأقامت على الأشجار المثمرة

\chapter{17}

\par \textit{ميخائيل يغادر.}

\par 1 فقال الملاك لأسنات: "هل رأيتِ هذا الأمر؟" فقالت: "نعم يا سيدي، لقد رأيت كل هذه الأشياء."

\par 2 قال لها الملاك الإلهي: "هكذا تكون كل كلماتي التي كلمتك بها اليوم."

\par 3 ثم مدّ ملاك الرب يده اليمنى للمرة الثالثة ولمس طرف المشط، فخرجت نار في الحال من المائدة والتهمت المشط، أما المائدة فلم تصب بأذى

\par 4 وعندما انبعثت رائحة طيبة من حرق المشط وامتلأت الغرفة، قالت أسنات للملاك الإلهي: "يا رب، لدي سبع عذارى نشأن معي منذ صباي وولدن في ليلة واحدة معي، ويخدمنني، وأنا أحبهن جميعًا كأخواتي. سأدعوهن وستباركهن أيضًا، كما تباركني."

\par 5 فقال لها الملاك: "ادعُوهنّ". فدعت أسنات العذارى السبع وأوقفتهن أمام الملاك، فقال لهن الملاك: "يبارككن الرب الإله العلي، وتكونن أعمدة ملجأ لسبع مدن، ويسكن عليكم جميع مختاري تلك المدينة الساكنين معًا إلى الأبد".

\par 6 وبعد هذه الأمور، قال الملاك "الإلهي" لأسينات: "أزيلي هذه المائدة". وعندما استدارت أسينات لتزيل المائدة، اختفى على الفور عن عينيها، فرأت أسينات ما يشبه عربة بأربعة خيول متجهة شرقًا نحو السماء، وكانت العربة كلهب نار، والخيول كالبرق، وكان الملاك واقفًا فوق تلك العربة

\par 7 ثم قالت أسنات: "أنا غبية وحمقاء، أيتها المتواضعة، لأني تكلمت كما لو أن رجلاً دخل غرفتي من السماء، ولم أكن أعلم أن الله دخلها؛ وها هو ذا الآن يعود إلى السماء إلى مكانه." وقالت في نفسها: "كن كريمًا يا رب مع أمتك، وارحم أمتك، لأني من جهتي، تكلمت بأمور طائشة أمامك عن جهل."

\chapter{18}

\par \textit{تغير وجه أسنات.}

\par 1 وبينما كانت أسنات لا تزال تتحدث بهذه الكلمات في نفسها، إذا بشاب، أحد عبيد يوسف، يقول: «يوسف، رجل الله الجبار، يأتي إليكم اليوم».

\par 2 فدعت أسنات في الحال من كان على بيتها وقالت له: أسرع وأعد بيتي وأعد غداء جيدا، لأن يوسف رجل الله العظيم يأتي إلينا اليوم.

\par 3 فلما رآها رئيس البيت (لأن وجهها قد انكمش من شدة البكاء والحزن والامتناع لمدة سبعة أيام) حزن وبكى، فأمسك بيدها اليمنى وقبلها بحنان وقال: "ما لكِ يا سيدتي حتى انكمش وجهك هكذا؟" فقالت: "لقد ثقل رأسي وطار النوم من عيني". ثم مضى رئيس البيت وأعد البيت والطعام.

\par 4 فتذكرت أسنات كلام الملاك وأوامره، وأسرعت ودخلت حجرتها الثانية، حيث كانت صناديق زينتها، وفتحت صندوقها الكبير وأخرجت رداءها الأول كالبرق لتنظر إليه وتلبسه، وتنطقت أيضًا بمنطقة لامعة وملوكية من ذهب وأحجار كريمة،

\par 5 ووضعت على يديها أساور من ذهب، وعلى قدميها أحزمة من ذهب، وحلية ثمينة حول عنقها، وإكليلًا من ذهب وضعته حول رأسها. وعلى الإكليل كما في مقدمته حجر ياقوت أزرق كبير، وحول الحجر الكبير ستة أحجار ثمينة، وغطت رأسها بعباءة رائعة جدًا. ولما تذكرت أسنات كلام مشرف بيتها، لأنه قال لها إن وجهها قد انكمش، حزنت حزنًا شديدًا، وتأوهت وقالت: "ويل لي أيتها المسكينة، لأن وجهي قد انكمش. سيراني يوسف هكذا، وسأكون بلا قيمة لديه."

\par 6 فقالت لأمتها: "أتيني بماء نقي من النبع". فلما أحضرته، صبته في المغسلة، وانحنت لتغسل وجهها، فرأت وجهها يلمع كالشمس، وعينيها كنجم الصبح حين يشرق، وخديها كنجم السماء، وشفتيها كورد أحمر، وكان شعر رأسها كالكرمة التي تتفتح بين ثماره في فردوس الله، وعنقها كشجرة سرو متعددة الألوان. فلما رأت أسنات هذه الأشياء، تعجبت من المنظر وفرحت فرحًا عظيمًا ولم تغسل وجهها، لأنها قالت: "لئلا أغسل هذا الجمال العظيم والجميل".

\par 7 ثم عاد مشرف منزلها ليخبرها: "كل ما أمرتِ به قد تم". ولما رآها، خاف خوفًا شديدًا واستولى عليه ارتعاش طويل، وسقط عند قدميها وبدأ يقول: "ما هذا يا سيدتي؟ ما هذا الجمال الذي يحيط بكِ العظيم والعجيب؟ هل اختاركِ الرب إله السماء عروسًا لابنه يوسف؟"

\chapter{19}

\par \textit{يعود جوزيف وتستقبله أسيناث.}

\par 1 وبينما هن يتكلمن بهذا، جاء غلام وقال لأسنات: «هوذا يوسف واقف أمام أبواب دارنا». فأسرعت أسنات ونزلت من عليتها مع العذارى السبع لاستقبال يوسف، ووقفت في رواق بيتها.

\par 2 ولما دخل يوسف الدار، كانت الأبواب مغلقة، وبقي الغرباء في الخارج. فخرجت أسنات من الرواق لاستقبال يوسف، فلما رآها أعجب بجمالها، وقال لها: "من أنتِ يا فتاة؟ أخبريني سريعًا".

\par 3 فقالت له: "أنا يا سيدي، أمتك أسنات. جميع الأصنام التي نبذتها عني فتبددت. وجاء إليّ اليوم رجل من السماء وأعطاني خبز الحياة، فأكلت وشربت كأسًا مباركة، وقال لي: لقد زوجتك ليوسف، وهو يكون عريسك إلى الأبد، ولن يُدعى اسمك أسنات، بل تُدعى مدينة الملجأ، وسيملك الرب الإله على أمم كثيرة، ومن خلالك يلجؤون إلى الله العلي." فقال الرجل: "سأذهب أيضًا إلى يوسف لأقول له في أذنيه هذا الكلام عنك. والآن أنت تعلم يا سيدي إن كان ذلك الرجل قد جاء إليك، وإن كان قد كلمك عني."

\par 4 ثم قال يوسف لأسنات: "مباركة أنتِ يا امرأة، يا من الله العلي، ومبارك اسمكِ إلى الأبد، لأن الرب الإله قد أسس أسواركِ، وسيسكن بنو الله الحي في مدينتكِ ملجئكِ، وسيملك الرب الإله عليهم إلى الأبد. لأن ذلك الرجل جاء إليّ اليوم من السماء وقال لي هذه الكلمات عنكِ. والآن تعالي إليّ هنا أيتها العذراء الطاهرة، ولماذا تقفين بعيدة؟"

\par 5 ثم مدّ يوسف يديه واحتضن أسنات، وأسنات يوسف، وقبلا بعضهما البعض زمانًا طويلًا، وعادا كلاهما إلى روحهما. فقبّل يوسف أسنات وأعطاها روح الحياة، ثم في المرة الثانية أعطاها روح الحكمة، وفي المرة الثالثة قبلها بحنان وأعطاها روح الحق

\chapter{20}

\par \textit{يعود بنتفريس ويرغب في خطبة أسنات ليوسف، لكن يوسف يقرر أن يطلب يدها من فرعون.}

\par 1 وبعد أن تشبثا ببعضهما البعض لفترة طويلة وشبكا سلاسل أيديهما، قالت أسنات ليوسف: "تعال يا سيدي وادخل بيتنا، لأنني من جهتي قد أعددت لنا بيتًا وعشاءً كبيرًا."

\par 2 فأمسكت بيده اليمنى وأدخلته إلى بيتها وأجلسته على كرسي بنتفريس أبيها، وأحضرت ماءً ليغسل قدميه. وقال يوسف: «لتأتِ واحدة من العذارى وتغسل قدمي».

\par 3 فقالت له أسنات: "لا يا سيدي، فأنت من الآن سيدي وأنا أمتك. ولماذا تطلب هذا أن تغسل عذراء أخرى قدميك؟ لأن قدميك قدمي، ويديك يدي، ونفسك نفسي، ولا يغسل رجليك غيري." فألزمته وغسلت ثيابه.

\par 4 فأمسك يوسف بيدها اليمنى وقبلها قبلة حارة وقبلت أسنات رأسه قبلة حارة ثم أجلسها عن يمينه.

\par 5 فجاء أبوها وأمها وجميع عشيرتها من ميراثهم، فرأوها جالسة مع يوسف لابسة ثوب العرس. فتعجبوا من جمالها وفرحوا ومجدوا الله الذي يحيي الموتى. وبعد ذلك أكلوا وشربوا.

\par 6 وبعد أن ابتهج الجميع، قال بنتفرس ليوسف: "غدًا سأدعو جميع الأمراء والولاة في كل أرض مصر، وسأصنع لك عرسًا، وستتزوج ابنتي أسنات."

\par 7 فقال يوسف: "سأذهب غدًا إلى فرعون الملك، لأنه هو أبي وقد عيّنني حاكمًا على كل هذه الأرض، وسأكلمه بشأن أسنات، فيعطيني إياها زوجة." فقال له بنتفرس: "اذهب بسلام."

\chapter{21}

\par \textit{زواج يوسف وأسينات.}

\par 1 ومكث يوسف ذلك اليوم عند بنتفرس، ولم يدخل إلى أسنات، لأنه كان يقول: "لا يليق برجل يعبد الله أن يضاجع امرأته قبل زواجه". فبكر يوسف وذهب إلى فرعون وقال له: "أعطني أسنات، ابنة بنتفرس، كاهن هليوبوليس، زوجة". ففرح فرعون فرحًا عظيمًا، وقال ليوسف: "أليس هذه مخطوبة لك منذ الأزل؟ فلتكن لك زوجة من الآن وإلى الأبد".

\par 2 ثم أرسل فرعون ودعا بنتفرس، فأحضرت بنتفرس أسنات وأوقفتها أمام فرعون

\par 3 فلما رآها فرعون تعجب من جمالها وقال: «يباركك الرب إله يوسف يا بنتي، ويبقى جمالك هذا إلى الأبد، لأن الرب إله يوسف اختارك له عروسًا، لأن يوسف كابن العلي، وستُدعين عروسه من الآن وإلى الأبد».

\par 4 وبعد هذه الأمور، أخذ فرعون يوسف وأسنات، ووضع على رأسيهما أكاليل من ذهب كانت في بيته منذ القديم ومنذ الأزل، وأقام فرعون أسنات عن يمين يوسف. ووضع فرعون يديه على رأسيهما وقال:

\par 5 «يباركك الرب الإله العلي ويكثرك ويعظمك ويمجدك إلى الأبد.»

\par 6 ثم حولهما فرعون وجهيهما، وقربهما فمًا إلى فم، فقبلا بعضهما بعضًا. وأقام فرعون عرسًا ليوسف، ووليمة عظيمة وشربًا كثيرًا سبعة أيام، وجمع كل حكام مصر وجميع ملوك الأمم،

\par 7 "وبعد أن نادى في أرض مصر قائلاً: كل رجل يعمل عملاً في الأيام السبعة لزواج يوسف وأسنات يموت موتاً."

\par 8 وفيما كان العرس قائمًا، وعند انتهاء العشاء، دخل يوسف على أسنات، فحبلت أسنات من يوسف وولدت منسى وأفرايم أخاه في بيت يوسف

\chapter{22}

\par \textit{تم تقديم أسينات إلى يعقوب.}


\par 1 وبعد أن مضت سنوات الوفرة السبع، بدأت سنوات المجاعة السبع تأتي.

\par 2 ولما سمع يعقوب عن يوسف ابنه، جاء إلى مصر هو وجميع عشيرته في السنة الثانية من الجوع، في الشهر الثاني، في الحادي والعشرين من الشهر، وسكن في جشم

\par 3 فقالت أسنات ليوسف: «أذهب وأرى أباك، لأن أباك إسرائيل كأبي وإلهي». فقال لها يوسف: «تذهبين معي وترين أبي».

\par 4 وجاء يوسف وأسنات إلى يعقوب في أرض جشم، فاستقبلهما إخوة يوسف وسجدوا لهما على وجوههما على الأرض. فدخلا كلاهما على يعقوب، وكان يعقوب جالسًا على فراشه، وكان هو نفسه شيخًا طاعنًا في السن. فلما رأته أسنات، تعجبت من جماله، لأن يعقوب كان جميل المنظر جدًا، وشيخوخته كشاب جميل، وكان رأسه كله أبيض كالثلج، وشعر رأسه كله كثيف جدًا، ولحيته بيضاء تصل إلى صدره، وعيناه مبتهجتان لامعتان، وأوتاره ومنكباه وذراعاه كملاك، وفخذيه وساقيه وقدميه كجبار. فلما رأته أسنات هكذا، تعجبت وسقطت وسجدت على وجهها على الأرض فقال يعقوب ليوسف: «أهذه كنتي امرأتك؟ مباركة من الله العلي».

\par 5 ثم دعا يعقوب أسنات إليه وباركها وقبلها قبلة دافئة. فمدت أسنات يديها وأمسكت بعنق يعقوب وتعلقت بعنقه وقبلته قبلة دافئة

\par 6 وبعد ذلك أكلوا وشربوا.

\par 7 فذهب يوسف وأسنات إلى بيتهما، وقادهما شمعون ولاوي ابنا ليئة وحدهما، ولم يشاركهما في قيادتهما ابنا بلة وزلفة جاريتا ليئة وراحيل، لأنهما كانا يحسدانهما ويبغضانهما. وكان لاوي عن يمين أسنات، وشمعون عن يسارها.

\par 8 فأمسكت أسنات بيد لاوي، لأنها أحبته أكثر من جميع إخوة يوسف، وكان نبيًا وعابدًا لله ومتقيًا للرب. لأنه كان رجلاً فاهمًا ونبيًا للعلي، وقد رأى رسائل مكتوبة في السماء وقرأها وكشفها لأسنات سرًا. لأن لاوي نفسه أحب أسنات كثيرًا ورأى موضع راحتها في الأعالي

\chapter{23}

\par \textit{يحاول ابن فرعون حث شمعون ولاوي على قتل يوسف.}


\par 1 وحدث بينما كان يوسف وأسنات يمران وهما ذاهبان إلى يعقوب، رآهما ابن فرعون البكر من على السور،

\par 2 فلما رأى أسنات، جنّ عليها من فرط جمالها. فأرسل ابن فرعون رسلاً ودعا إليه شمعون ولاوي، فلما أتيا ووقفا أمامه،

\par 3 فقال لهم ابن فرعون البكر: أما أنا فأعلم أنكم اليوم أعظم من كل الناس في الأرض، وبيدكم اليمنى خربت مدينة السكيميين، وبسيفيكم قتلتم ثلاثين ألف مقاتل.

\par 4 وسأتخذكم اليوم رفقاء لي، وأعطيكم ذهبًا وفضة كثيرين، وخدمًا وإماءً وبيوتًا وميراثًا عظيمًا، وتناصرونني وتُحسنون إليّ؛ لأني تلقيت ازدراءً كبيرًا من أخيكم يوسف، لأنه هو نفسه اتخذ أسنات زوجة، وهذه المرأة كانت مخطوبة لي منذ القدم

\par 5 والآن تعالوا معي، وسأقاتل يوسف لأقتله بسيفي، وسأتزوج أسنات، وستكونون لي كأخوة وأصدقاء مخلصين

\par 6 ولكن إن لم تسمعوا لكلامي، فسأقتلكم بسيفي. وبعد أن قال هذا، استل سيفه وأراه لهم

\par 7 وكان شمعون رجلاً جريئًا ومقدامًا، ففكر في أن يضع يده اليمنى على مقبض سيفه ويسحبه من غمده ويضرب ابن فرعون لأنه تكلم معهم بكلام قاسٍ

\par 8 فرأى لاوي فكر قلبه، لأنه كان نبيًا، فداس بقدمه على قدم شمعون اليمنى وضغط عليها، طالبًا منه أن يكف عن غضبه

\par 9 وكان لاوي يقول لشمعون بهدوء: "لماذا أنت غاضب على هذا الرجل؟ نحن رجال نعبد الله، وليس من حقنا أن نكافئ شرًا بشر."

\par 10 فقال لاوي لابن فرعون صراحةً ووداعة قلب: "لماذا يتكلم سيدنا بهذه الكلمات؟ نحن رجال نعبد الله، وأبونا خليل الله العلي، وأخونا كابن لله. فكيف نفعل هذا الأمر الشرير، أن نخطئ أمام إلهنا وأبينا إسرائيل وأمام أخينا يوسف؟ والآن اسمع كلامي. لا يليق بإنسان يعبد الله أن يؤذي أحدًا بأي شكل من الأشكال؛ وإذا أراد أحد أن يؤذي إنسانًا يعبد الله، فإن ذلك الإنسان الذي يعبد الله لا ينتقم منه، لأنه ليس في يديه سيف

\par 11 واحذر أن تتكلم بهذه الكلمات بعد الآن عن أخينا يوسف

\par 12 ولكن إن استمررت في مشورتك الشريرة، فها هي سيوفنا مسلولة عليك

\par 13 ثم استلّ شمعون ولاوي سيوفهما من غمديهما وقالا: "أرأيت الآن هذين السيفين؟ بهذين السيفين عاقب الرب ازدراء السكيميين الذي ازدروا به بني إسرائيل عن طريق أختنا دينة التي نجّسها شكيم بن حمور."

\par 14 فلما رأى ابن فرعون السيوف مسلولة، خاف خوفًا شديدًا وارتعد في كل جسده، لأنها كانت تلمع كلهب نار، وكلت عيناه، وسقط على وجهه على الأرض تحت أقدامهم

\par 15 ثم مدّ لاوي يده اليمنى وأمسك به قائلاً: قم ولا تخف، ولكن احذر أن تتكلم بعد بكلمتين رديئتين على أخينا يوسف.

\par 16 فخرج كل من شمعون ولاوي من أمام وجهه.

\chapter{24}

ابن فرعون يتآمر مع دان وجاد لقتل يوسف والاستيلاء على أسنات.


\par 1 ثم استمر ابن فرعون في الخوف والحزن لأنه خاف من إخوة يوسف، ومرة ​​أخرى أصيب بالجنون الشديد بسبب جمال أسنات، وحزن بشدة

\par 2 ثم قال خدمه في أذنه: "هوذا! أبناء بلة وأبناء زلفة، جاريتا ليئة وراحيل، زوجتي يعقوب، في عداوة شديدة ليوسف وأسنات ويبغضونهما. سيكون هؤلاء معك في كل شيء حسب مشيئتك."

\par 3 فللوقت أرسل ابن فرعون رسلًا ودعاهم، فجاءوا إليه في الساعة الأولى من الليل، ووقفوا أمامه، فقال لهم: «لقد تعلمت من كثيرين أنكم أقوياء».

\par 4 فقال له دان وجاد، الأخوان الأكبران: «فليُكلِّم سيدي الآن خدامه بما يريد، حتى يسمع خدامك فنفعل حسب مشيئتك».

\par 5 ففرح ابن فرعون فرحًا عظيمًا جدًا، وقال لعبيده: «انصرفوا عني قليلًا، فإن لي كلامًا سريًّا مع هؤلاء الرجال». فانصرفوا جميعًا

\par 6 فكذب ابن فرعون وقال لهم: هوذا البركة والموت أمام وجوهكم، فتختارون البركة على الموت،

\par 7 لأنكم رجال أقوياء ولن تموتوا كالنساء، بل كونوا شجعانًا وانتقموا لأنفسكم من أعدائكم

\par 8 لأني سمعت يوسف أخاك يقول لفرعون أبي: «ليس دان وجاد ونفتاليم وأشير إخوتي، بل أبناء إماء أبي».

\par 9 لذلك أنتظر موت أبي، وسأمحوهم من الأرض مع كل نسلهم، لئلا يرثوا معنا، لأنهم أبناء الإماء

\par 10 فإن هؤلاء باعوني أيضًا للإسماعيليين، وسأرد لهم ما فعلوه بي من الشر. أبي وحده يموت

\par 11 فأثنى عليه أبي فرعون على هذه الأمور وقال له: «لقد أحسنت يا بني. لذلك، خذ مني رجالاً أشداء، وتصرف معهم كما فعلوا بك، وأنا أكون لك عونًا».

\par 12 ولما سمع دان وجاد هذه الأمور من ابن فرعون، اضطربا كثيرًا وحزنا حزنًا شديدًا، وقالا له: "يا رب، أعنّا؛ فنحن من الآن فصاعدًا عبيدك وعبيدك وسنموت معك". فقال ابن فرعون: "سأكون لكم معينًا إذا سمعتم أنتم أيضًا لكلامي". فقالا له: "مرنا بما تريد، وسنفعل حسب مشيئتك".

\par 13 فقال لهم ابن فرعون: «أقتل أبي فرعون هذه الليلة، لأن فرعون كأبي يوسف، وقد قال له إنه سيساعدكم ضدكم. فاقتلوا يوسف، فأتزوج أسنات، وتكونون إخوتي ووارثين معي كل أموالي. افعلوا هذا الأمر فقط».

\par 14 فقال له دان وجاد: "نحن خدامك اليوم، وسنفعل كل ما أمرتنا به. وقد سمعنا يوسف يقول لأسنات: اذهبي غدًا إلى نصيب ميراثنا، فهو موسم الحصاد". فأرسل ستمائة رجل جبار للحرب معها وخمسين متقدمًا. فالآن اسمعنا، وسنكلم سيدنا." وكلموه بكل ما في سرهم

\par 15 ثم أعطى ابن فرعون لكل واحد من الإخوة الأربعة خمسمائة رجل، وجعلهم رؤساءً وقادةً لهم

\par 16 فقال له دان وجاد: «نحن خدامك اليوم، وسنفعل كل ما أمرتنا به، وسنخرج ليلًا ونتربص في الوادي، ونختبئ في غابة القصب

\par 17 وخذ معك خمسين رجلاً من رماة السهام على الخيول واذهب أمامنا مسافة طويلة، وستأتي أسنات وتسقط في أيدينا، وسنقضي على الرجال الذين معها،

\par 18 وستهرب هي نفسها بمركبتها وتقع في يديك، وستفعل بها ما تشتهي نفسك؛

\par 19 وبعد هذه الأمور سنقتل يوسف أيضًا وهو حزين على أسنات. وكذلك سنقتل أولاده أيضًا أمام عينيه

\par 20 فلما سمع ابن فرعون البكر هذه الأمور، فرح فرحًا عظيمًا، وأرسلهم ومعهم ألفي رجل محارب. ولما وصلوا إلى الوادي اختبأوا في غابة القصب، وانقسموا إلى أربع فرق، وتمركزوا على الجانب الآخر من الوادي، كما في الجزء الأمامي: خمسمائة رجل على هذا الجانب من الطريق وعلى ذلك الجانب، وكذلك بقي الباقون على الجانب القريب من الوادي

\par 21 وأقاموا هم أيضًا في غابة القصب، خمسمائة رجل على جانبي الطريق، وكان بينهم طريق واسع وواسع


\chapter{25}

\par \textit{يذهب ابن فرعون لقتل أبيه، لكن لا يُسمح له بذلك. يحتج نفتاليم وآشير لدى دان وجاد على المؤامرة.}


\par 1 فقام ابن فرعون في تلك الليلة وجاء إلى مخدع أبيه ليقتله بالسيف. فمنعه حراس أبيه من الدخول إلى أبيه وقالوا له: "بماذا تأمر يا سيدي؟"

\par 2 فقال لهم ابن فرعون: «أريد أن أرى أبي، لأني ذاهب لأجمع ثمار كرمي الجديد».

\par 3 فقال له الحراس: إن أبوك كان يتألم ويسهر الليل كله، والآن استراح، وقال لنا لا يدخل عليه أحد حتى لو كان ابني البكر.

\par 4 فلما سمع ذلك انصرف غاضبًا، وفي الحال أخذ خمسين فارسًا من الرماة، وانصرف أمامهم كما قال له دان وجاد

\par 5 وكلم الأخوان الأصغران نفتاليم وأشير أخويه الأكبر دان وجاد قائلين: "لماذا تعودان إلى الشر ضد أبيكما إسرائيل وضد أخيكما يوسف؟ والله يحفظه كحدقة عين. هوذا!

\par 6 ألم تبيعوا يوسف مرة واحدة؟ وهو اليوم ملك على كل أرض مصر ومانح الطعام

\par 7 والآن، إن أردتم أن تعملوا الشر ضده مرة أخرى، فسوف يصرخ إلى العلي، فيرسل نارًا من السماء فتلتهمكم، ويقاتلكم ملائكة الله

\par 8 فغضب عليهما الإخوة الأكبر وقالوا: "وأنموت كالنساء؟ حاشا لنا." فخرجوا للقاء يوسف وأسنات

\chapter{26}

\par \textit{المتآمرون يقتلون حراس أسيناث وتهرب.}


\par 1 فقامت أسنات في الصباح وقالت ليوسف: "أنا ذاهبة إلى امتلاك ميراثنا كما قلت، لكن نفسي خائفة جدًا لأنك مفارقني."

\par 2 فقال لها يوسف: «ثقي ولا تخافي، بل اذهبي فرحة، لا ترتعبي من أحد، لأن الرب معك، وهو يحفظك كحدقة عين من كل شر.»

\par 3 وأخرج لأعطي طعامي وأعطي جميع رجال المدينة، فلا يهلك أحد من الجوع في أرض مصر

\par 4 ثم ذهبت أسنات في طريقها، ويوسف لأجل إعطائه الطعام.

\par 5 ولما وصلت أسنات إلى موضع الوادي مع الستمائة رجل، خرج الذين كانوا مع ابن فرعون فجأة من مكمنهم واشتبكوا مع الذين كانوا مع أسنات، فقتلوهم جميعا بالسيووف، وقتلوا كل من سبقوها.

\par 6 لكن أسنات هربت بمركبتها. فعرف لاوي بن ليئة كل هذه الأمور بالروح كأنه نبي، فأخبر إخوته بخطر أسنات.

\par 7 فأخذ كل واحد منهم سيفه على فخذه ودروعه على ذراعيه ورماحه في يده اليمنى وطارد أسنات بسرعة كبيرة.

\par 8 وبينما أسنات تهرب من أمامها إذا ابن فرعون قد لقيها وخمسون فارساً معه. فلما رأته أسنات خافت خوفاً عظيماً وارتعدت ودعت باسم الرب إلهها.

\chapter{27}

\par \textit{قُتل الرجال مع ابن فرعون وأولئك الذين مع دان وجاد؛ وفرّ الإخوة الأربعة إلى الوادي وسُلبت سيوفهم من أيديهم.}


\par 1 وكان بنيامين جالسا معها في المركبة عن اليمين، وكان بنيامين غلامًا قوي البنية ابن تسع عشرة سنة.

\par 2 وكان عليه جمال وقوة لا توصف كشبل أسد، وكان أيضًا يخشى الله خوفًا شديدًا

\par 3 فقفز بنيامين من المركبة، وأخذ حجرًا مستديرًا من الوادي وملأ يده، ورماه على ابن فرعون، فضرب صدغه الأيسر، وجرحه جرحًا بليغًا، فسقط عن جواده على الأرض نصف ميت

\par 4 وعندئذٍ، صعد بنيامين إلى صخرة، وقال لسائق مركبة أسنات: "أعطني حجارة من الوادي". فأعطاه خمسين حجرًا

\par 5 فألقى بنيامين الحجارة فقتل الخمسين رجلاً الذين كانوا مع ابن فرعون، حتى غاصت جميع الحجارة في معابدهم

\par 6 فتبع بنو ليئة، رأوبين وشمعون، ولاوي ويهوذا، ويساكر وزبولون، الرجال الذين كانوا يتربصون لأسنات، وهجموا عليهم بغتة وقتلوهم جميعًا، فقتل الستة رجال ألفين وستة وسبعين رجلاً

\par 7 فهرب ابنا بلة وزلفة من أمامهم وقالا: "لقد هلكنا على أيدي إخوتنا، ومات ابن فرعون أيضًا على يد بنيامين الغلام، وهلك كل من كان معه على يد بنيامين الغلام. لذلك، تعالوا فلنقتل أسنات وبنيامين ونهرب إلى غابة هذا القصب."

\par 8 فجاءوا إلى أسنات وهم يحملون سيوفهم مسلولة ملطخة بالدماء. فلما رأتهم أسنات خافت خوفًا شديدًا وقالت: "أيها الرب الإله، الذي أحييني وأنقذني من الأصنام وفساد الموت، كما قلت لي أن نفسي ستحيا إلى الأبد، نجني الآن أيضًا من هؤلاء الأشرار." فسمع الرب الإله صوت أسنات، وفي الحال سقطت سيوف الأعداء من أيديهم على الأرض وتحولت إلى رماد

\chapter{28}

\par \textit{نجا دان وجاد بفضل توسلات أسيناث.}


\par 1 "وأبناء بلّا وزلفا، عندما رأوا المعجزة الغريبة التي حدثت، خافوا وقالوا: ""الرب يحارب ضدنا نيابة عن أسنات""."

\par 2 ثم سقطوا على وجوههم إلى الأرض وسجدوا لأسنات وقالوا: "ارحمنا يا عبيدك، فأنت سيدتنا وملكتنا. لقد ارتكبنا شرورًا ضدك وضد أخينا يوسف، لكن الرب كافأنا حسب أعمالنا.

\par 3 لذلك، نحن عبيدك، نسألك أن ترحمنا نحن المساكين والبؤساء، وأن تنقذنا من أيدي إخوتنا، لأنهم سينتقمون لك من سيوفهم التي نصبوها ضدنا. فارحم عبيدك يا ​​سيدتي أمامهم.

\par 4 فقال لهم أسنات: تشجعوا ولا تخافوا من إخوتكم، لأنهم هم أنفسهم رجال يعبدون الله ويخافون الرب.

\par 5 بل اذهبوا إلى غابة القصب حتى أُرضيهم عنكم وأُخمد غضبهم على الجرائم العظيمة التي تجرأتم على ارتكابها بحقهم. ولينظر الرب ويحكم بيني وبينكم.

\par 6 فهرب دان وجاد إلى غابة القصب، فجاء إخوتهما بنو ليئة يركضون نحوهما كالغزلان مسرعين

\par 7 فنزلت أسنات عن المركبة التي كانت تسترها وأعطتهم يدها اليمنى بالدموع،

\par 8 فخرّوا وسجدوا لها على الأرض وبكوا بصوت عظيم

\par 9 واستمروا في طلب إخوتهم أبناء الجواري ليقتلوهم

\par 10 فقالت لهم أسنات: "أرجوكم، ارحموا إخوتكم، ولا تجازوهم شرًا بشر. لأن الرب أنقذني منهم وحطم خناجرهم وسيوفهم من أيديهم، وها هم قد ذابوا واحترقوا رمادًا على الأرض كالشمع من أمام النار،

\par 11 وهذا يكفينا أن الرب يقاتلهم عنا. لذلك تحفظون إخوتكم، لأنهم إخوتكم ودم أبيكم إسرائيل

\par 12 فقال لها سمعان: "لماذا تتكلم سيدتنا بكلام طيب من أجل أعدائها؟ كلا، بل سنقطعهم إربًا إربًا بسيوفهم،

\par 13 لأنهم تدبروا أمورًا شريرة على أخينا يوسف وأبينا إسرائيل وعليكِ سيدتنا اليوم

\par 14 ثم مدت أسنات يدها اليمنى ولمست لحية شمعون وقبلته بحنان وقالت: "لا تجازِ يا أخي جارك شرًا بشر، لأن الرب سينتقم له من هذا الحقد. أنت تعلم أنهم إخوتك وذرية أبيك إسرائيل، وقد هربوا من بعيد من أمام وجهك. فاغفر لهم."

\par 15 ثم تقدم لاوي إليها وقبل يدها اليمنى بحنان، لأنه كان يعلم أنها تريد أن تنقذ الرجال من غضب إخوتهم حتى لا يقتلوهم. وكانوا هم أنفسهم قريبين في غابة القصب. وعلم لاوي أخوه ذلك، فلم يخبر إخوته، لأنه خاف أن يقتلوا إخوتهم في غضبهم

\chapter{29}

\par \textit{مات ابن فرعون. مات فرعون أيضًا وخلفه يوسف.}

\par 1 فقام ابن فرعون عن الأرض وجلس وبصق دمًا من فمه، لأن الدم كان يسيل من صدغه إلى فمه

\par 2 فركض بنيامين إليه وأخذ سيفه واستله من غمد ابن فرعون (لأن بنيامين لم يكن يحمل سيفًا على فخذه) وأراد أن يضرب ابن فرعون على صدره

\par 3 فركض إليه لاوي وأمسك بيده وقال: «لا تفعل هذا يا أخي، فنحن أناس نعبد الله، وليس من اللائق بعبد الله أن يجازي شرًا بشر، ولا أن يدوس على ساقط، ولا أن يسحق عدوه حتى الموت. والآن رد السيف إلى مكانه.

\par 4 وتعالوا وساعدوني، ودعونا نشفيه من هذا الجرح. وإن عاش، فسيكون صديقنا، وسيكون أبوه فرعون أبانا

\par 5 ثم رفع لاوي ابن فرعون من الأرض وغسل الدم عن وجهه وربط على جرحه وأركبه على فرسه وذهب به إلى أبيه فرعون،

\par 6 يروي له كل ما حدث ووقع.

\par 7 فقام فرعون عن كرسيه وسجد للاوي على الأرض وباركه.

\par 8 ثم لما انقضى اليوم الثالث، مات ابن فرعون متأثرًا بالحجر الذي جرحه به بنيامين

\par 9 وحزن فرعون على ابنه البكر حزنًا شديدًا،

\par 10 ومن ثم مرض فرعون ومات عن عمر 109 سنوات، وترك تاجه ليوسف الجميل.

\par 11 وملك يوسف وحده في مصر ثماني وأربعين سنة. وبعد ذلك رد يوسف الإكليل إلى ابن فرعون الأصغر، الذي كان يرضع عندما مات فرعون الشيخ

\par 12 وكان يوسف من ذلك الحين أباً لابن فرعون الأصغر في مصر حتى وفاته، يُمجِّد الله ويُسبِّحه


\end{document}