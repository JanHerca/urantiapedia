\begin{document}

\title{ارميا}


\chapter{1}

\par 1 كَلاَمُ إِرْمِيَا بْنِ حِلْقِيَّا مِنَ الْكَهَنَةِ الَّذِينَ فِي عَنَاثُوثَ فِي أَرْضِ بِنْيَامِينَ
\par 2 الَّذِي كَانَتْ كَلِمَةُ الرَّبِّ إِلَيْهِ فِي أَيَّامِ يُوشِيَّا بْنِ آمُونَ مَلِكِ يَهُوذَا فِي السَّنَةِ الثَّالِثَةِ عَشَرَةَ مِنْ مُلْكِهِ.
\par 3 وَكَانَتْ فِي أَيَّامِ يَهُويَاقِيمَ بْنِ يُوشِيَّا مَلِكِ يَهُوذَا إِلَى تَمَامِ السَّنَةِ الْحَادِيَةِ عَشَرَةَ لِصِدْقِيَّا بْنِ يُوشِيَّا مَلِكِ يَهُوذَا إِلَى سَبْيِ أُورُشَلِيمَ فِي الشَّهْرِ الْخَامِسِ.
\par 4 فَكَانَتْ كَلِمَةُ الرَّبِّ إِلَيَّ:
\par 5 [قَبْلَمَا صَوَّرْتُكَ فِي الْبَطْنِ عَرَفْتُكَ وَقَبْلَمَا خَرَجْتَ مِنَ الرَّحِمِ قَدَّسْتُكَ. جَعَلْتُكَ نَبِيّاً لِلشُّعُوبِ].
\par 6 فَقُلْتُ: [آهِ يَا سَيِّدُ الرَّبُّ إِنِّي لاَ أَعْرِفُ أَنْ أَتَكَلَّمَ لأَنِّي وَلَدٌ].
\par 7 فَقَالَ الرَّبُّ لِي: [لاَ تَقُلْ إِنِّي وَلَدٌ لأَنَّكَ إِلَى كُلِّ مَنْ أُرْسِلُكَ إِلَيْهِ تَذْهَبُ وَتَتَكَلَّمُ بِكُلِّ مَا آمُرُكَ بِهِ.
\par 8 لاَ تَخَفْ مِنْ وُجُوهِهِمْ لأَنِّي أَنَا مَعَكَ لأُنْقِذَكَ يَقُولُ الرَّبُّ].
\par 9 وَمَدَّ الرَّبُّ يَدَهُ وَلَمَسَ فَمِي وَقَالَ الرَّبُّ لِي: [هَا قَدْ جَعَلْتُ كَلاَمِي فِي فَمِكَ.
\par 10 اُنْظُرْ! قَدْ وَكَّلْتُكَ هَذَا الْيَوْمَ عَلَى الشُّعُوبِ وَعَلَى الْمَمَالِكِ لِتَقْلَعَ وَتَهْدِمَ وَتُهْلِكَ وَتَنْقُضَ وَتَبْنِيَ وَتَغْرِسَ].
\par 11 ثُمَّ صَارَتْ كَلِمَةُ الرَّبِّ إِلَيَّ: [مَاذَا أَنْتَ رَاءٍ يَا إِرْمِيَا؟] فَقُلْتُ: [أَنَا رَاءٍ قَضِيبَ لَوْزٍ].
\par 12 فَقَالَ الرَّبُّ لِي: [أَحْسَنْتَ الرُّؤْيَةَ لأَنِّي أَنَا سَاهِرٌ عَلَى كَلِمَتِي لأُجْرِيَهَا].
\par 13 ثُمَّ صَارَتْ كَلِمَةُ الرَّبِّ إِلَيَّ ثَانِيَةً: [مَاذَا أَنْتَ رَاءٍ؟] فَقُلْتُ: [إِنِّي رَاءٍ قِدْراً مَنْفُوخَةً وَوَجْهُهَا مِنْ جِهَةِ الشِّمَالِ].
\par 14 فَقَالَ الرَّبُّ لِي: [مِنَ الشِّمَالِ يَنْفَتِحُ الشَّرُّ عَلَى كُلِّ سُكَّانِ الأَرْضِ.
\par 15 لأَنِّي هَئَنَذَا دَاعٍ كُلَّ عَشَائِرِ مَمَالِكِ الشِّمَالِ يَقُولُ الرَّبُّ فَيَأْتُونَ وَيَضَعُونَ كُلُّ وَاحِدٍ كُرْسِيَّهُ فِي مَدْخَلِ أَبْوَابِ أُورُشَلِيمَ وَعَلَى كُلِّ أَسْوَارِهَا حَوَالَيْهَا وَعَلَى كُلِّ مُدُنِ يَهُوذَا.
\par 16 وَأُقِيمُ دَعْوَايَ عَلَى كُلِّ شَرِّهِمْ لأَنَّهُمْ تَرَكُونِي وَبَخَّرُوا لِآلِهَةٍ أُخْرَى وَسَجَدُوا لأَعْمَالِ أَيْدِيهِمْ.
\par 17 [أَمَّا أَنْتَ فَنَطِّقْ حَقَوَيْكَ وَقُمْ وَكَلِّمْهُمْ بِكُلِّ مَا آمُرُكَ بِهِ. لاَ تَرْتَعْ مِنْ وُجُوهِهِمْ لِئَلاَّ أُرِيعَكَ أَمَامَهُمْ.
\par 18 هَئَنَذَا قَدْ جَعَلْتُكَ الْيَوْمَ مَدِينَةً حَصِينَةً وَعَمُودَ حَدِيدٍ وَأَسْوَارَ نُحَاسٍ عَلَى كُلِّ الأَرْضِ لِمُلُوكِ يَهُوذَا وَلِرُؤَسَائِهَا وَلِكَهَنَتِهَا وَلِشَعْبِ الأَرْضِ.
\par 19 فَيُحَارِبُونَكَ وَلاَ يَقْدِرُونَ عَلَيْكَ لأَنِّي أَنَا مَعَكَ يَقُولُ الرَّبُّ لأُنْقِذَكَ].

\chapter{2}

\par 1 وَصَارَتْ إِلَيَّ كَلِمَةُ الرَّبِّ:
\par 2 [اذْهَبْ وَنَادِ فِي أُذُنَيْ أُورُشَلِيمَ: هَكَذَا قَالَ الرَّبُّ: قَدْ ذَكَرْتُ لَكِ غَيْرَةَ صِبَاكِ مَحَبَّةَ خِطْبَتِكِ ذِهَابَكِ وَرَائِي فِي الْبَرِّيَّةِ فِي أَرْضٍ غَيْرِ مَزْرُوعَةٍ.
\par 3 إِسْرَائِيلُ قُدْسٌ لِلرَّبِّ أَوَائِلُ غَلَّتِهِ. كُلُّ آكِلِيهِ يَأْثَمُونَ. شَرٌّ يَأْتِي عَلَيْهِمْ يَقُولُ الرَّبُّ].
\par 4 اِسْمَعُوا كَلِمَةَ الرَّبِّ يَا بَيْتَ يَعْقُوبَ وَكُلَّ عَشَائِرِ بَيْتِ إِسْرَائِيلَ.
\par 5 هَكَذَا قَالَ الرَّبُّ: [مَاذَا وَجَدَ فِيَّ آبَاؤُكُمْ مِنْ جَوْرٍ حَتَّى ابْتَعَدُوا عَنِّي وَسَارُوا وَرَاءَ الْبَاطِلِ وَصَارُوا بَاطِلاً؟
\par 6 وَلَمْ يَقُولُوا: أَيْنَ هُوَ الرَّبُّ الَّذِي أَصْعَدَنَا مِنْ أَرْضِ مِصْرَ الَّذِي سَارَ بِنَا فِي الْبَرِّيَّةِ فِي أَرْضِ قَفْرٍ وَحُفَرٍ فِي أَرْضِ يُبُوسَةٍ وَظِلِّ الْمَوْتِ فِي أَرْضٍ لَمْ يَعْبُرْهَا رَجُلٌ وَلَمْ يَسْكُنْهَا إِنْسَانٌ؟
\par 7 وَأَتَيْتُ بِكُمْ إِلَى أَرْضِ بَسَاتِينَ لِتَأْكُلُوا ثَمَرَهَا وَخَيْرَهَا. فَأَتَيْتُمْ وَنَجَّسْتُمْ أَرْضِي وَجَعَلْتُمْ مِيرَاثِي رِجْساً.
\par 8 اَلْكَهَنَةُ لَمْ يَقُولُوا: أَيْنَ هُوَ الرَّبُّ؟ وَأَهْلُ الشَّرِيعَةِ لَمْ يَعْرِفُونِي وَالرُّعَاةُ عَصُوا عَلَيَّ وَالأَنْبِيَاءُ تَنَبَّأُوا بِبَعْلٍ وَذَهَبُوا وَرَاءَ مَا لاَ يَنْفَعُ.
\par 9 [لِذَلِكَ أُخَاصِمُكُمْ بَعْدُ يَقُولُ الرَّبُّ وَبَنِي بَنِيكُمْ أُخَاصِمُ.
\par 10 فَاعْبُرُوا جَزَائِرَ كِتِّيمَ وَانْظُرُوا وَأَرْسِلُوا إِلَى قِيدَارَ وَانْتَبِهُوا جِدّاً وَانْظُرُوا: هَلْ صَارَ مِثْلُ هَذَا؟
\par 11 هَلْ بَدَلَتْ أُمّةٌ آلِهَةً وَهِيَ لَيْسَتْ آلِهَةً؟ أَمَّا شَعْبِي فَقَدْ بَدَلَ مَجْدَهُ بِمَا لاَ يَنْفَعُ!
\par 12 اِبْهَتِي أَيَّتُهَا السَّمَاوَاتُ مِنْ هَذَا وَاقْشَعِرِّي وَتَحَيَّرِي جِدّاً يَقُولُ الرَّبُّ.
\par 13 لأَنَّ شَعْبِي عَمِلَ شَرَّيْنِ: تَرَكُونِي أَنَا يَنْبُوعَ الْمِيَاهِ الْحَيَّةِ لِيَنْقُرُوا لأَنْفُسِهِمْ آبَاراً آبَاراً مُشَقَّقَةً لاَ تَضْبُطُ مَاءً.
\par 14 [أَعَبْدٌ إِسْرَائِيلُ أَوْ مَوْلُودُ الْبَيْتِ هُوَ؟ لِمَاذَا صَارَ غَنِيمَةً؟
\par 15 زَمْجَرَتْ عَلَيْهِ الأَشْبَالُ. أَطْلَقَتْ صَوْتَهَا وَجَعَلَتْ أَرْضَهُ خَرِبَةً. أُحْرِقَتْ مُدُنُهُ فَلاَ سَاكِنَ.
\par 16 وَبَنُو نُوفَ وَتَحْفَنِيسَ قَدْ شَجُّوا هَامَتَكِ.
\par 17 أَمَا صَنَعْتِ هَذَا بِنَفْسِكِ إِذْ تَرَكْتِ الرَّبَّ إِلَهَكِ حِينَمَا كَانَ مُسَيِّرَكِ فِي الطَّرِيقِ؟
\par 18 وَالآنَ مَا لَكِ وَطَرِيقَ مِصْرَ لِشُرْبِ مِيَاهِ شِيحُورَ وَمَا لَكِ وَطَرِيقَ أَشُّورَ لِشُرْبِ مِيَاهِ النَّهْرِ؟
\par 19 يُوَبِّخُكِ شَرُّكِ وَعِصْيَانُكِ يُؤَدِّبُكِ. فَاعْلَمِي وَانْظُرِي أَنَّ تَرْكَكِ الرَّبَّ إِلَهَكِ شَرٌّ وَمُرٌّ وَأَنَّ خَشْيَتِي لَيْسَتْ فِيكِ يَقُولُ السَّيِّدُ رَبُّ الْجُنُودِ.
\par 20 [لأَنَّهُ مُنْذُ الْقَدِيمِ كَسَرْتُ نِيرَكِ وَقَطَعْتُ قُيُودَكِ وَقُلْتِ: لاَ أَتَعَبَّدُ. لأَنَّكِ عَلَى كُلِّ أَكَمَةٍ عَالِيَةٍ وَتَحْتَ كُلِّ شَجَرَةٍ خَضْرَاءَ أَنْتِ اضْطَجَعْتِ زَانِيَةً!
\par 21 وَأَنَا قَدْ غَرَسْتُكِ كَرْمَةَ سُورَقَ زَرْعَ حَقٍّ كُلَّهَا. فَكَيْفَ تَحَوَّلْتِ لِي سُرُوغَ جَفْنَةٍ غَرِيبَةٍ؟
\par 22 فَإِنَّكِ وَإِنِ اغْتَسَلْتِ بِنَطْرُونٍ وَأَكْثَرْتِ لِنَفْسِكِ الأَشْنَانَ فَقَدْ نُقِشَ إِثْمُكِ أَمَامِي يَقُولُ السَّيِّدُ الرَّبُّ
\par 23 كَيْفَ تَقُولِينَ: لَمْ أَتَنَجَّسْ. وَرَاءَ بَعْلِيمَ لَمْ أَذْهَبْ؟ انْظُرِي طَرِيقَكِ فِي الْوَادِي. إِعْرِفِي مَا عَمِلْتِ يَا نَاقَةً خَفِيفَةً ضَبِعَةً فِي طُرُقِهَا!
\par 24 يَا أَتَانَ الْفَرَاءِ قَدْ تَعَوَّدَتِ الْبَرِّيَّةَ! فِي شَهْوَةِ نَفْسِهَا تَسْتَنْشِقُ الرِّيحَ. عِنْدَ ضَبَعِهَا مَنْ يَرُدُّهَا؟ كُلُّ طَالِبِيهَا لاَ يُعْيُونَ. فِي شَهْرِهَا يَجِدُونَهَا.
\par 25 اِحْفَظِي رِجْلَكِ مِنَ الْحَفَا وَحَلْقَكِ مِنَ الظَّمَإِ. فَقُلْتِ: بَاطِلٌ! لاَ! لأَنِّي قَدْ أَحْبَبْتُ الْغُرَبَاءَ وَوَرَاءَهُمْ أَذْهَبُ.
\par 26 كَخِزْيِ السَّارِقِ إِذَا وُجِدَ هَكَذَا خِزْيُ بَيْتِ إِسْرَائِيلَ هُمْ وَمُلُوكُهُمْ وَرُؤَسَاؤُهُمْ وَكَهَنَتُهُمْ وَأَنْبِيَاؤُهُمْ
\par 27 قَائِلِينَ لِلْعُودِ: أَنْتَ أَبِي وَلِلْحَجَرِ: أَنْتَ وَلَدْتَنِي. لأَنَّهُمْ حَوَّلُوا نَحْوِي الْقَفَا لاَ الْوَجْهَ وَفِي وَقْتِ بَلِيَّتِهِمْ يَقُولُونَ: قُمْ وَخَلِّصْنَا.
\par 28 فَأَيْنَ آلِهَتُكَ الَّتِي صَنَعْتَ لِنَفْسِكَ؟ فَلْيَقُومُوا إِنْ كَانُوا يُخَلِّصُونَكَ فِي وَقْتِ بَلِيَّتِكَ. لأَنَّهُ عَلَى عَدَدِ مُدُنِكَ صَارَتْ آلِهَتُكَ يَا يَهُوذَا.
\par 29 لِمَاذَا تُخَاصِمُونَنِي؟ كُلُّكُمْ عَصَيْتُمُونِي يَقُولُ الرَّبُّ.
\par 30 لِبَاطِلٍ ضَرَبْتُ بَنِيكُمْ. لَمْ يَقْبَلُوا تَأْدِيباً. أَكَلَ سَيْفُكُمْ أَنْبِيَاءَكُمْ كَأَسَدٍ مُهْلِكٍ.
\par 31 [أَنْتُمْ أَيُّهَا الْجِيلُ انْظُرُوا كَلِمَةَ الرَّبِّ. هَلْ صِرْتُ بَرِّيَّةً لإِسْرَائِيلَ أَوْ أَرْضَ ظَلاَمٍ دَامِسٍ؟ لِمَاذَا قَالَ شَعْبِي: قَدْ شَرَدْنَا. لاَ نَجِيءُ إِلَيْكَ بَعْدُ؟
\par 32 هَلْ تَنْسَى عَذْرَاءٌ زِينَتَهَا أَوْ عَرُوسٌ مَنَاطِقَهَا؟ أَمَّا شَعْبِي فَقَدْ نَسِيَنِي أَيَّاماً بِلاَ عَدَدٍ.
\par 33 لِمَاذَا تُحَسِّنِينَ طَرِيقَكِ لِتَطْلُبِي الْمَحَبَّةَ؟ لِذَلِكَ عَلَّمْتِ الشِّرِّيرَاتِ أَيْضاً طُرُقَكِ.
\par 34 أَيْضاً فِي أَذْيَالِكِ وُجِدَ دَمُ نُفُوسِ الْمَسَاكِينِ الأَزْكِيَاءِ. لاَ بِالنَّقْبِ وَجَدْتُهُ بَلْ عَلَى كُلِّ هَذِهِ.
\par 35 وَتَقُولِينَ: لأَنِّي تَبَرَّأْتُ ارْتَدَّ غَضَبُهُ عَنِّي حَقّاً. هَئَنَذَا أُحَاكِمُكِ لأَنَّكِ قُلْتِ: لَمْ أُخْطِئْ.
\par 36 لِمَاذَا تَرْكُضِينَ لِتَبْدُلِي طَرِيقَكِ؟ مِنْ مِصْرَ أَيْضاً تَخْزِينَ كَمَا خَزِيتِ مِنْ أَشُّورَ.
\par 37 مِنْ هُنَا أَيْضاً تَخْرُجِينَ وَيَدَاكِ عَلَى رَأْسِكِ لأَنَّ الرَّبَّ قَدْ رَفَضَ ثِقَاتِكِ فَلاَ تَنْجَحِينَ فِيهَا.

\chapter{3}

\par 1 [يَسْأَلُونَ: إِذَا طَلَّقَ رَجُلٌ امْرَأَتَهُ فَانْطَلَقَتْ مِنْ عِنْدِهِ وَصَارَتْ لِرَجُلٍ آخَرَ فَهَلْ يَرْجِعُ إِلَيْهَا بَعْدُ؟ - أَلاَ تَتَنَجَّسُ تِلْكَ الأَرْضُ نَجَاسَةً؟ أَمَّا أَنْتِ فَقَدْ زَنَيْتِ بِأَصْحَابٍ كَثِيرِينَ! لَكِنِ ارْجِعِي إِلَيَّ يَقُولُ الرَّبُّ.
\par 2 اِرْفَعِي عَيْنَيْكِ إِلَى الْهِضَابِ وَانْظُرِي أَيْنَ لَمْ تُضَاجَعِي! فِي الطُّرُقَاتِ جَلَسْتِ لَهُمْ كَأَعْرَابِيٍّ فِي الْبَرِّيَّةِ وَنَجَّسْتِ الأَرْضَ بِزِنَاكِ وَبِشَرِّكِ.
\par 3 فَامْتَنَعَ الْغَيْثُ وَلَمْ يَكُنْ مَطَرٌ مُتَأَخِّرٌ. وَجَبْهَةُ امْرَأَةٍ زَانِيَةٍ كَانَتْ لَكِ. أَبَيْتِ أَنْ تَخْجَلِي.
\par 4 أَلَسْتِ مِنَ الآنَ تَدْعِينَنِي: يَا أَبِي أَلِيفُ صِبَايَ أَنْتَ.
\par 5 هَلْ يَحْقِدُ إِلَى الدَّهْرِ أَوْ يَحْفَظُ غَضَبَهُ إِلَى الأَبَدِ؟ هَا قَدْ تَكَلَّمْتِ وَعَمِلْتِ شُرُوراً وَاسْتَطَعْتِ!].
\par 6 وَقَالَ الرَّبُّ لِي فِي أَيَّامِ يُوشِيَّا الْمَلِكِ: [هَلْ رَأَيْتَ مَا فَعَلَتِ الْعَاصِيَةُ إِسْرَائِيلُ؟ انْطَلَقَتْ إِلَى كُلِّ جَبَلٍ عَالٍ وَإِلَى كُلِّ شَجَرَةٍ خَضْرَاءَ وَزَنَتْ هُنَاكَ.
\par 7 فَقُلْتُ بَعْدَ مَا فَعَلَتْ كُلَّ هَذِهِ: ارْجِعِي إِلَيَّ. فَلَمْ تَرْجِعْ. فَرَأَتْ أُخْتُهَا الْخَائِنَةُ يَهُوذَا.
\par 8 فَرَأَيْتُ أَنَّهُ لأَجْلِ كُلِّ الأَسْبَابِ إِذْ زَنَتِ الْعَاصِيَةُ إِسْرَائِيلُ فَطَلَّقْتُهَا وَأَعْطَيْتُهَا كِتَابَ طَلاَقِهَا لَمْ تَخَفِ الْخَائِنَةُ يَهُوذَا أُخْتُهَا بَلْ مَضَتْ وَزَنَتْ هِيَ أَيْضاً.
\par 9 وَكَانَ مِنْ هَوَانِ زِنَاهَا أَنَّهَا نَجَّسَتِ الأَرْضَ وَزَنَتْ مَعَ الْحَجَرِ وَمَعَ الشَّجَرِ.
\par 10 وَفِي كُلِّ هَذَا أَيْضاً لَمْ تَرْجِعْ إِلَيَّ أُخْتُهَا الْخَائِنَةُ يَهُوذَا بِكُلِّ قَلْبِهَا بَلْ بِالْكَذِبِ]. يَقُولُ الرَّبُّ.
\par 11 فَقَالَ الرَّبُّ لِي: [قَدْ بَرَّرَتْ نَفْسَهَا الْعَاصِيَةُ إِسْرَائِيلُ أَكْثَرَ مِنَ الْخَائِنَةِ يَهُوذَا].
\par 12 [اِذْهَبْ وَنَادِ بِهَذِهِ الْكَلِمَاتِ نَحْوَ الشِّمَالِ وَقُلِ: ارْجِعِي أَيَّتُهَا الْعَاصِيَةُ إِسْرَائِيلُ يَقُولُ الرَّبُّ. لاَ أُوقِعُ غَضَبِي بِكُمْ لأَنِّي رَأُوفٌ يَقُولُ الرَّبُّ. لاَ أَحْقِدُ إِلَى الأَبَدِ.
\par 13 اِعْرِفِي فَقَطْ إِثْمَكِ أَنَّكِ إِلَى الرَّبِّ إِلَهِكِ أَذْنَبْتِ وَفَرَّقْتِ طُرُقَكِ لِلْغُرَبَاءِ تَحْتَ كُلِّ شَجَرَةٍ خَضْرَاءَ وَلِصَوْتِي لَمْ تَسْمَعُوا]. يَقُولُ الرَّبُّ.
\par 14 [اِرْجِعُوا أَيُّهَا الْبَنُونَ الْعُصَاةُ يَقُولُ الرَّبُّ لأَنِّي سُدْتُ عَلَيْكُمْ فَآخُذَكُمْ وَاحِداً مِنَ الْمَدِينَةِ وَاثْنَيْنِ مِنَ الْعَشِيرَةِ وَآتِي بِكُمْ إِلَى صِهْيَوْنَ
\par 15 وَأُعْطِيكُمْ رُعَاةً حَسَبَ قَلْبِي فَيَرْعُونَكُمْ بِالْمَعْرِفَةِ وَالْفَهْمِ.
\par 16 وَيَكُونُ إِذْ تَكْثُرُونَ وَتُثْمِرُونَ فِي الأَرْضِ فِي تِلْكَ الأَيَّامِ يَقُولُ الرَّبُّ أَنَّهُمْ لاَ يَقُولُونَ بَعْدُ: تَابُوتَ عَهْدِ الرَّبِّ وَلاَ يَخْطُرُ عَلَى بَالٍ وَلاَ يَذْكُرُونَهُ وَلاَ يَتَعَهَّدُونَهُ وَلاَ يُصْنَعُ بَعْدُ.
\par 17 فِي ذَلِكَ الزَّمَانِ يُسَمُّونَ أُورُشَلِيمَ كُرْسِيَّ الرَّبِّ وَيَجْتَمِعُ إِلَيْهَا كُلُّ الأُمَمِ إِلَى اسْمِ الرَّبِّ إِلَى أُورُشَلِيمَ وَلاَ يَذْهَبُونَ بَعْدُ وَرَاءَ عِنَادِ قَلْبِهِمِ الشِّرِّيرِ.
\par 18 فِي تِلْكَ الأَيَّامِ يَذْهَبُ بَيْتُ يَهُوذَا مَعَ بَيْتِ إِسْرَائِيلَ وَيَأْتِيَانِ مَعاً مِنْ أَرْضِ الشِّمَالِ إِلَى الأَرْضِ الَّتِي مَلَّكْتُ آبَاءَكُمْ إِيَّاهَا.
\par 19 وَأَنَا قُلْتُ: كَيْفَ أَضَعُكِ بَيْنَ الْبَنِينَ وَأُعْطِيكِ أَرْضاً شَهِيَّةً مِيرَاثَ مَجْدِ أَمْجَادِ الأُمَمِ؟ وَقُلْتُ: تَدْعِينَنِي يَا أَبِي وَمِنْ وَرَائِي لاَ تَرْجِعِينَ.
\par 20 [حَقّاً إِنَّهُ كَمَا تَخُونُ الْمَرْأَةُ قَرِينَهَا هَكَذَا خُنْتُمُونِي يَا بَيْتَ إِسْرَائِيلَ يَقُولُ الرَّبُّ.
\par 21 سُمِعَ صَوْتٌ عَلَى الْهِضَابِ بُكَاءُ تَضَرُّعَاتِ بَنِي إِسْرَائِيلَ. لأَنَّهُمْ عَوَّجُوا طَرِيقَهُمْ. نَسُوا الرَّبَّ إِلَهَهُمْ.
\par 22 اِرْجِعُوا أَيُّهَا الْبَنُونَ الْعُصَاةُ فَأَشْفِيَ عِصْيَانَكُمْ]. [هَا قَدْ أَتَيْنَا إِلَيْكَ لأَنَّكَ أَنْتَ الرَّبُّ إِلَهُنَا.
\par 23 حَقّاً بَاطِلَةٌ هِيَ الآكَامُ ثَرْوَةُ الْجِبَالِ. حَقّاً بِالرَّبِّ إِلَهِنَا خَلاَصُ إِسْرَائِيلَ.
\par 24 وَقَدْ أَكَلَ الْخِزْيُ تَعَبَ آبَائِنَا مُنْذُ صِبَانَا - غَنَمَهُمْ وَبَقَرَهُمْ بَنِيهِمْ وَبَنَاتِهِمْ.
\par 25 نَضْطَجِعُ فِي خِزْيِنَا وَيُغَطِّينَا خَجَلُنَا لأَنَّنَا إِلَى الرَّبِّ إِلَهِنَا أَخْطَأْنَا نَحْنُ وَآبَاؤُنَا مُنْذُ صِبَانَا إِلَى هَذَا الْيَوْمِ وَلَمْ نَسْمَعْ لِصَوْتِ الرَّبِّ إِلَهِنَا].

\chapter{4}

\par 1 [إِنْ رَجَعْتَ يَا إِسْرَائِيلُ يَقُولُ الرَّبُّ إِنْ رَجَعْتَ إِلَيَّ وَإِنْ نَزَعْتَ مَكْرُهَاتِكَ مِنْ أَمَامِي فَلاَ تَتِيهُ.
\par 2 وَإِنْ حَلَفْتَ: حَيٌّ هُوَ الرَّبُّ بِالْحَقِّ وَالْعَدْلِ وَالْبِرِّ فَتَتَبَرَّكُ الشُّعُوبُ بِهِ وَبِهِ يَفْتَخِرُونَ].
\par 3 لأَنَّهُ هَكَذَا قَالَ الرَّبُّ لِرِجَالِ يَهُوذَا وَلأُورُشَلِيمَ: [احْرُثُوا لأَنْفُسِكُمْ حَرْثاً وَلاَ تَزْرَعُوا فِي الأَشْوَاكِ.
\par 4 اِخْتَتِنُوا لِلرَّبِّ وَانْزِعُوا غُرَلَ قُلُوبِكُمْ يَا رِجَالَ يَهُوذَا وَسُكَّانَ أُورُشَلِيمَ لِئَلاَّ يَخْرُجَ كَنَارٍ غَيْظِي فَيُحْرِقَ وَلَيْسَ مَنْ يُطْفِئُ بِسَبَبِ شَرِّ أَعْمَالِكُمْ.
\par 5 أَخْبِرُوا فِي يَهُوذَا وَسَمِّعُوا فِي أُورُشَلِيمَ وَقُولُوا: اضْرِبُوا بِالْبُوقِ فِي الأَرْضِ. نَادُوا بِصَوْتٍ عَالٍ وَقُولُوا: اجْتَمِعُوا فَلْنَدْخُلِ الْمُدُنَ الْحَصِينَةَ.
\par 6 اِرْفَعُوا الرَّايَةَ نَحْوَ صِهْيَوْنَ. احْتَمُوا. لاَ تَقِفُوا. لأَنِّي آتِي بِشَرٍّ مِنَ الشِّمَالِ وَكَسْرٍ عَظِيمٍ.
\par 7 قَدْ صَعِدَ الأَسَدُ مِنْ غَابَتِهِ وَزَحَفَ مُهْلِكُ الأُمَمِ. خَرَجَ مِنْ مَكَانِهِ لِيَجْعَلَ أَرْضَكِ خَرَاباً. تُخْرَبُ مُدُنُكِ فَلاَ سَاكِنَ.
\par 8 مِنْ أَجْلِ ذَلِكَ تَنَطَّقُوا بِمُسُوحٍ. الْطُمُوا وَوَلْوِلُوا لأَنَّهُ لَمْ يَرْتَدَّ حُمُوُّ غَضَبِ الرَّبِّ عَنَّا.
\par 9 وَيَكُونُ فِي ذَلِكَ الْيَوْمِ يَقُولُ الرَّبُّ أَنَّ قَلْبَ الْمَلِكِ يُعْدَمُ وَقُلُوبَ الرُّؤَسَاءِ. وَتَتَحَيَّرُ الْكَهَنَةُ وَتَتَعَجَّبُ الأَنْبِيَاءُ].
\par 10 فَقُلْتُ: [آهِ يَا سَيِّدُ الرَّبُّ حَقّاً إِنَّكَ خِدَاعاً خَادَعْتَ هَذَا الشَّعْبَ وَأُورُشَلِيمَ قَائِلاً: يَكُونُ لَكُمْ سَلاَمٌ وَقَدْ بَلَغَ السَّيْفُ النَّفْسَ].
\par 11 فِي ذَلِكَ الزَّمَانِ يُقَالُ لِهَذَا الشَّعْبِ وَلأُورُشَلِيمَ: [رِيحٌ لاَفِحَةٌ مِنَ الْهَِضَابِ فِي الْبَرِّيَّةِ نَحْوَ بِنْتِ شَعْبِي لاَ لِلتَّذْرِيَةِ وَلاَ لِلتَّنْقِيَةِ.
\par 12 رِيحٌ أَشَدُّ تَأْتِي لِي مِنْ هَذِهِ. الآنَ أَنَا أَيْضاً أُحَاكِمُهُمْ].
\par 13 [هُوَذَا كَسَحَابٍ يَصْعَدُ وَكَزَوْبَعَةٍ مَرْكَبَاتُهُ. أَسْرَعُ مِنَ النُّسُورِ خَيْلُهُ. وَيْلٌ لَنَا لأَنَّنَا قَدْ أُخْرِبْنَا].
\par 14 اِغْسِلِي مِنَ الشَّرِّ قَلْبَكِ يَا أُورُشَلِيمُ لِتُخَلَّصِي. إِلَى مَتَى تَبِيتُ فِي وَسَطِكِ أَفْكَارُكِ الْبَاطِلَةُ؟
\par 15 لأَنَّ صَوْتاً يُخْبِرُ مِنْ دَانَ وَيُسْمَعُ بِبَلِيَّةٍ مِنْ جَبَلِ أَفْرَايِمَ:
\par 16 [اُذْكُرُوا لِلأُمَمِ. انْظُرُوا. أَسْمِعُوا عَلَى أُورُشَلِيمَ. الْمُحَاصِرُونَ آتُونَ مِنْ أَرْضٍ بَعِيدَةٍ فَيُطْلِقُونَ عَلَى مُدُنِ يَهُوذَا صَوْتَهُمْ.
\par 17 كَحَارِسِي حَقْلٍ صَارُوا عَلَيْهَا حَوَالَيْهَا لأَنَّهَا تَمَرَّدَتْ عَلَيَّ يَقُولُ الرَّبُّ.
\par 18 طَرِيقُكِ وَأَعْمَالُكِ صَنَعَتْ هَذِهِ لَكِ. هَذَا شَرُّكِ. فَإِنَّهُ مُرٌّ. فَإِنَّهُ قَدْ بَلَغَ قَلْبَكِ].
\par 19 أَحْشَائِي أَحْشَائِي! تُوجِعُنِي جُدْرَانُ قَلْبِي. يَئِنُّ فِيَّ قَلْبِي. لاَ أَسْتَطِيعُ السُّكُوتَ. لأَنَّكِ سَمِعْتِ يَا نَفْسِي صَوْتَ الْبُوقِ وَهُتَافَ الْحَرْبِ.
\par 20 بِكَسْرٍ عَلَى كَسْرٍ نُودِيَ لأَنَّهُ قَدْ خَرِبَتْ كُلُّ الأَرْضِ. بَغْتَةً خَرِبَتْ خِيَامِي وَشُقَقِي فِي لَحْظَةٍ.
\par 21 حَتَّى مَتَى أَرَى الرَّايَةَ وَأَسْمَعُ صَوْتَ الْبُوقِ؟
\par 22 لأَنَّ شَعْبِي أَحْمَقُ. إِيَّايَ لَمْ يَعْرِفُوا. هُمْ بَنُونَ جَاهِلُونَ وَهُمْ غَيْرُ فَاهِمِينَ. هُمْ حُكَمَاءُ فِي عَمَلِ الشَّرِّ وَلِعَمَلِ الصَّالِحِ مَا يَفْهَمُونَ.
\par 23 نَظَرْتُ إِلَى الأَرْضِ وَإِذَا هِيَ خَرِبَةٌ وَخَالِيَةٌ وَإِلَى السَّمَاوَاتِ فَلاَ نُورَ لَهَا.
\par 24 نَظَرْتُ إِلَى الْجِبَالِ وَإِذَا هِيَ تَرْتَجِفُ وَكُلُّ الآكَامِ تَقَلْقَلَتْ.
\par 25 نَظَرْتُ وَإِذَا لاَ إِنْسَانَ وَكُلُّ طُيُورِ السَّمَاءِ هَرَبَتْ.
\par 26 نَظَرْتُ وَإِذَا الْبُسْتَانُ بَرِّيَّةٌ وَكُلُّ مُدُنِهَا نُقِضَتْ مِنْ وَجْهِ الرَّبِّ مِنْ وَجْهِ حُمُوِّ غَضَبِهِ.
\par 27 لأَنَّهُ هَكَذَا قَالَ الرَّبُّ: [ خَرَاباً تَكُونُ كُلُّ الأَرْضِ وَلَكِنَّنِي لاَ أُفْنِيهَا.
\par 28 مِنْ أَجْلِ ذَلِكَ تَنُوحُ الأَرْضُ وَتُظْلِمُ السَّمَاوَاتُ مِنْ فَوْقُ مِنْ أَجْلِ أَنِّي قَدْ تَكَلَّمْتُ. قَصَدْتُ وَلاَ أَنْدَمُ وَلاَ أَرْجِعُ عَنْهُ.
\par 29 مِنْ صَوْتِ الْفَارِسِ وَرَامِي الْقَوْسِ كُلُّ الْمَدِينَةِ هَارِبَةٌ. دَخَلُوا الْغَابَاتِ وَصَعِدُوا عَلَى الصُّخُورِ. كُلُّ الْمُدُنِ مَتْرُوكَةٌ وَلاَ إِنْسَانَ سَاكِنٌ فِيهَا.
\par 30 وَأَنْتِ أَيَّتُهَا الْخَرِبَةُ مَاذَا تَعْمَلِينَ؟ إِذَا لَبِسْتِ قِرْمِزاً إِذَا تَزَيَّنْتِ بِزِينَةٍ مِنْ ذَهَبٍ إِذَا كَحَّلْتِ بِالأُثْمُدِ عَيْنَيْكِ فَبَاطِلاً تُحَسِّنِينَ ذَاتَكِ فَقَدْ رَذَلَكِ الْعَاشِقُونَ. يَطْلُبُونَ نَفْسَكِ.
\par 31 لأَنِّي سَمِعْتُ صَوْتاً كَمَاخِضَةٍ ضِيقاً مِثْلَ ضِيقِ بِكْرِيَّةٍ. صَوْتَ ابْنَةِ صِهْيَوْنَ تَزْفِرُ. تَبْسُطُ يَدَيْهَا قَائِلَةً: وَيْلٌ لِي لأَنَّ نَفْسِي قَدْ أُغْمِيَ عَلَيْهَا بِسَبَبِ الْقَاتِلِينَ].

\chapter{5}

\par 1 طُوفُوا فِي شَوَارِعِ أُورُشَلِيمَ وَانْظُرُوا وَاعْرِفُوا وَفَتِّشُوا فِي سَاحَاتِهَا. هَلْ تَجِدُونَ إِنْسَاناً أَوْ يُوجَدُ عَامِلٌ بِالْعَدْلِ طَالِبُ الْحَقِّ فَأَصْفَحَ عَنْهَا؟
\par 2 وَإِنْ قَالُوا: [حَيٌّ هُوَ الرَّبُّ] فَإِنَّهُمْ يَحْلِفُونَ بِالْكَذِبِ!
\par 3 يَا رَبُّ أَلَيْسَتْ عَيْنَاكَ عَلَى الْحَقِّ؟ ضَرَبْتَهُمْ فَلَمْ يَتَوَجَّعُوا. أَفْنَيْتَهُمْ وَأَبُوا قُبُولَ التَّأْدِيبِ. صَلَّبُوا وُجُوهَهُمْ أَكْثَرَ مِنَ الصَّخْرِ. أَبُوا الرُّجُوعَ.
\par 4 أَمَّا أَنَا فَقُلْتُ: إِنَّمَا هُمْ مَسَاكِينُ. قَدْ جَهِلُوا لأَنَّهُمْ لَمْ يَعْرِفُوا طَرِيقَ الرَّبِّ قَضَاءَ إِلَهِهِمْ.
\par 5 أَنْطَلِقُ إِلَى الْعُظَمَاءِ وَأُكَلِّمُهُمْ لأَنَّهُمْ عَرَفُوا طَرِيقَ الرَّبِّ قَضَاءَ إِلَهِهِمْ. أَمَّا هُمْ فَقَدْ كَسَرُوا النِّيرَ جَمِيعاً وَقَطَعُوا الرُّبُطَ.
\par 6 مِنْ أَجْلِ ذَلِكَ يَضْرِبُهُمُ الأَسَدُ مِنَ الْوَعْرِ. ذِئْبُ الْمَسَاءِ يُهْلِكُهُمْ. يَكْمُنُ النَّمِرُ حَوْلَ مُدُنِهِمْ. كُلُّ مَنْ خَرَجَ مِنْهَا يُفْتَرَسُ لأَنَّ ذُنُوبَهُمْ كَثُرَتْ. تَعَاظَمَتْ مَعَاصِيهِمْ!
\par 7 [كَيْفَ أَصْفَحُ لَكِ عَنْ هَذِهِ؟ بَنُوكِ تَرَكُونِي وَحَلَفُوا بِمَا لَيْسَتْ آلِهَةً. وَلَمَّا أَشْبَعْتُهُمْ زَنُوا وَفِي بَيْتِ زَانِيَةٍ تَزَاحَمُوا.
\par 8 صَارُوا حُصُناً مَعْلُوفَةً سَائِبَةً. صَهَلُوا كُلُّ وَاحِدٍ عَلَى امْرَأَةِ صَاحِبِهِ.
\par 9 أَمَا أُعَاقِبُ عَلَى هَذَا يَقُولُ الرَّبُّ؟ أَوَ مَا تَنْتَقِمُ نَفْسِي مِنْ أُمَّةٍ كَهَذِهِ؟
\par 10 [اِصْعَدُوا عَلَى أَسْوَارِهَا وَاخْرِبُوا وَلَكِنْ لاَ تُفْنُوهَا. انْزِعُوا أَفْنَانَهَا لأَنَّهَا لَيْسَتْ لِلرَّبِّ.
\par 11 لأَنَّهُ خِيَانَةً خَانَنِي بَيْتُ إِسْرَائِيلَ وَبَيْتُ يَهُوذَا يَقُولُ الرَّبُّ.
\par 12 جَحَدُوا الرَّبَّ وَقَالُوا: لَيْسَ هُوَ وَلاَ يَأْتِي عَلَيْنَا شَرٌّ وَلاَ نَرَى سَيْفاً وَلاَ جُوعاً.
\par 13 وَالأَنْبِيَاءُ يَصِيرُونَ رِيحاً وَالْكَلِمَةُ لَيْسَتْ فِيهِمْ. هَكَذَا يُصْنَعُ بِهِمْ.
\par 14 لِذَلِكَ هَكَذَا قَالَ الرَّبُّ إِلَهُ الْجُنُودِ: مِنْ أَجْلِ أَنَّكُمْ تَتَكَلَّمُونَ بِهَذِهِ الْكَلِمَةِ هَئَنَذَا جَاعِلٌ كَلاَمِي فِي فَمِكَ نَاراً وَهَذَا الشَّعْبَ حَطَباً فَتَأْكُلُهُمْ.
\par 15 هَئَنَذَا أَجْلِبُ عَلَيْكُمْ أُمَّةً مِنْ بُعْدٍ يَا بَيْتَ إِسْرَائِيلَ يَقُولُ الرَّبُّ. أُمَّةً قَوِيَّةً. أُمَّةً مُنْذُ الْقَدِيمِ. أُمَّةً لاَ تَعْرِفُ لِسَانَهَا وَلاَ تَفْهَمُ مَا تَتَكَلَّمُ بِهِ.
\par 16 جُعْبَتُهُمْ كَقَبْرٍ مَفْتُوحٍ. كُلُّهُمْ جَبَابِرَةٌ.
\par 17 فَيَأْكُلُونَ حَصَادَكَ وَخُبْزَكَ الَّذِي يَأْكُلُهُ بَنُوكَ وَبَنَاتُكَ. يَأْكُلُونَ غَنَمَكَ وَبَقَرَكَ. يَأْكُلُونَ جَفْنَتَكَ وَتِينَكَ. يُهْلِكُونَ بِالسَّيْفِ مُدُنَكَ الْحَصِينَةَ الَّتِي أَنْتَ مُتَّكِلٌ عَلَيْهَا.
\par 18 وَأَيْضاً فِي تِلْكَ الأَيَّامِ يَقُولُ الرَّبُّ لاَ أُفْنِيكُمْ].
\par 19 [وَيَكُونُ حِينَ تَقُولُونَ: لِمَاذَا صَنَعَ الرَّبُّ إِلَهُنَا بِنَا كُلَّ هَذِهِ؟ تَقُولُ لَهُمْ: كَمَا أَنَّكُمْ تَرَكْتُمُونِي وَعَبَدْتُمْ آلِهَةً غَرِيبَةً فِي أَرْضِكُمْ هَكَذَا تَعْبُدُونَ الْغُرَبَاءَ فِي أَرْضٍ لَيْسَتْ لَكُمْ.
\par 20 أَخْبِرُوا بِهَذَا فِي بَيْتِ يَعْقُوبَ وَأَسْمِعُوا بِهِ فِي يَهُوذَا قَائِلِينَ:
\par 21 اِسْمَعْ هَذَا أَيُّهَا الشَّعْبُ الْجَاهِلُ وَالْعَدِيمُ الْفَهْمِ الَّذِينَ لَهُمْ أَعْيُنٌ وَلاَ يُبْصِرُونَ. لَهُمْ آذَانٌ وَلاَ يَسْمَعُونَ.
\par 22 أَإِيَّايَ لاَ تَخْشُونَ يَقُولُ الرَّبُّ؟ أَوَلاَ تَرْتَعِدُونَ مِنْ وَجْهِي أَنَا الَّذِي وَضَعْتُ الرَّمْلَ تُخُوماً لِلْبَحْرِ فَرِيضَةً أَبَدِيَّةً لاَ يَتَعَدَّاهَا فَتَتَلاَطَمُ وَلاَ تَسْتَطِيعُ وَتَعِجُّ أَمْوَاجُهُ وَلاَ تَتَجَاوَزُهَا.
\par 23 وَصَارَ لِهَذَا الشَّعْبِ قَلْبٌ عَاصٍ وَمُتَمَرِّدٌ. عَصُوا وَمَضُوا.
\par 24 وَلَمْ يَقُولُوا بِقُلُوبِهِمْ: لِنَخَفِ الرَّبَّ إِلَهَنَا الَّذِي يُعْطِي الْمَطَرَ الْمُبَكِّرَ وَالْمُتَأَخِّرَ فِي وَقْتِهِ. يَحْفَظُ لَنَا أَسَابِيعَ الْحَصَادِ الْمَفْرُوضَةَ.
\par 25 [آثَامُكُمْ عَكَسَتْ هَذِهِ وَخَطَايَاكُمْ مَنَعَتِ الْخَيْرَ عَنْكُمْ.
\par 26 لأَنَّهُ وُجِدَ فِي شَعْبِي أَشْرَارٌ يَرْصُدُونَ كَمُنْحَنٍ مِنَ الْقَانِصِينَ. يَنْصِبُونَ أَشْرَاكاً يُمْسِكُونَ النَّاسَ.
\par 27 مِثْلَ قَفَصٍ مَلآنٍ طُيُوراً هَكَذَا بُيُوتُهُمْ مَلآنَةٌ مَكْراً. مِنْ أَجْلِ ذَلِكَ عَظُمُوا وَاسْتَغْنُوا.
\par 28 سَمِنُوا. لَمَعُوا. أَيْضاً تَجَاوَزُوا فِي أُمُورِ الشَّرِّ. لَمْ يَقْضُوا فِي دَعْوَى الْيَتِيمِ. وَقَدْ نَجَحُوا. وَبِحَقِّ الْمَسَاكِينِ لَمْ يَقْضُوا.
\par 29 أَفَلأَجْلِ هَذِهِ لاَ أُعَاقِبُ يَقُولُ الرَّبُّ؟ أَوَلاَ تَنْتَقِمُ نَفْسِي مِنْ أُمَّةٍ كَهَذِهِ؟
\par 30 [صَارَ فِي الأَرْضِ دَهَشٌ وَقَشْعَرِيرَةٌ.
\par 31 اَلأَنْبِيَاءُ يَتَنَبَّأُونَ بِالْكَذِبِ وَالْكَهَنَةُ تَحْكُمُ عَلَى أَيْدِيهِمْ وَشَعْبِي هَكَذَا أَحَبَّ. وَمَاذَا تَعْمَلُونَ فِي آخِرَتِهَا؟

\chapter{6}

\par 1 [اُهْرُبُوا يَا بَنِي بِنْيَامِينَ مِنْ وَسَطِ أُورُشَلِيمَ وَاضْرِبُوا بِالْبُوقِ فِي تَقُوعَ وَعَلَى بَيْتِ هَكَّارِيمَ ارْفَعُوا عَلَمَ نَارٍ لأَنَّ الشَّرَّ أَشْرَفَ مِنَ الشِّمَالِ وَكَسْرٌ عَظِيمٌ.
\par 2 اَلْجَمِيلَةُ اللَّطِيفَةُ ابْنَةُ صِهْيَوْنَ أُهْلِكُهَا.
\par 3 إِلَيْهَا تَأْتِي الرُّعَاةُ وَقُطْعَانُهُمْ. يَنْصِبُونَ عِنْدَهَا خِيَاماً حَوَالَيْهَا. يَرْعُونَ كُلُّ وَاحِدٍ فِي مَكَانِهِ.
\par 4 قَدِّسُوا عَلَيْهَا حَرْباً. قُومُوا فَنَصْعَدَ في الظَّهِيرَةِ. وَيْلٌ لَنَا لأَنَّ النَّهَارَ مَالَ لأَنَّ ظِلاَلَ الْمَسَاءِ امْتَدَّتْ.
\par 5 قُومُوا فَنَصْعَدَ في اللَّيْلِ وَنَهْدِمَ قُصُورَهَا].
\par 6 لأَنَّهُ هَكَذَا قَالَ رَبُّ الْجُنُودِ: [اقْطَعُوا أَشْجَاراً. أَقِيمُوا حَوْلَ أُورُشَلِيمَ مِتْرَسَةً. هِيَ الْمَدِينَةُ الْمُعَاقَبَةُ. كُلُّهَا ظُلْمٌ فِي وَسَطِهَا.
\par 7 كَمَا تُنْبِعُ الْعَيْنُ مِيَاهَهَا هَكَذَا تُنْبِعُ هِيَ شَرَّهَا. ظُلْمٌ وَخَطْفٌ يُسْمَعُ فِيهَا. أَمَامِي دَائِماً مَرَضٌ وَضَرْبٌ.
\par 8 تَأَدَّبِي يَا أُورُشَلِيمُ لِئَلاَّ تَجْفُوكِ نَفْسِي. لِئَلاَّ أَجْعَلَكِ خَرَاباً أَرْضاً غَيْرَ مَسْكُونَةٍ].
\par 9 هَكَذَا قَالَ رَبُّ الْجُنُودِ: [تَعْلِيلاً يُعَلِّلُونَ كَجَفْنَةٍ بَقِيَّةَ إِسْرَائِيلَ. رُدَّ يَدَكَ كَقَاطِفٍ إِلَى السِّلاَلِ.
\par 10 مَنْ أُكَلِّمُهُمْ وَأُنْذِرُهُمْ فَيَسْمَعُوا؟ هَا إِنَّ أُذْنَهُمْ غَلْفَاءُ فَلاَ يَقْدِرُونَ أَنْ يَصْغُوا. هَا إِنَّ كَلِمَةَ الرَّبِّ صَارَتْ لَهُمْ عَاراً. لاَ يُسَرُّونَ بِهَا.
\par 11 فَامْتَلَأْتُ مِنْ غَيْظِ الرَّبِّ. مَلِلْتُ الطَّاقَةَ. أَسْكُبُهُ عَلَى الأَطْفَالِ فِي الْخَارِجِ وَعَلَى مَجْلِسِ الشُّبَّانِ مَعاً لأَنَّ الرَّجُلَ وَالْمَرْأَةَ يُؤْخَذَانِ كِلاَهُمَا وَالشَّيْخَ مَعَ الْمُمْتَلِئِ أَيَّاماً.
\par 12 وَتَتَحَوَّلُ بُيُوتُهُمْ إِلَى آخَرِينَ الْحُقُولُ وَالنِّسَاءُ مَعاً لأَنِّي أَمُدُّ يَدِي عَلَى سُكَّانِ الأَرْضِ يَقُولُ الرَّبُّ.
\par 13 لأَنَّهُمْ مِنْ صَغِيرِهِمْ إِلَى كَبِيرِهِمْ كُلُّ وَاحِدٍ مُولَعٌ بِالرِّبْحِ وَمِنَ النَّبِيِّ إِلَى الْكَاهِنِ كُلُّ وَاحِدٍ يَعْمَلُ بِالْكَذِبِ.
\par 14 وَيَشْفُونَ كَسْرَ بِنْتِ شَعْبِي عَلَى عَثَمٍ قَائِلِينَ: سَلاَمٌ سَلاَمٌ وَلاَ سَلاَمَ.
\par 15 هَلْ خَزُوا لأَنَّهُمْ عَمِلُوا رِجْساً؟ بَلْ لَمْ يَخْزُوا خِزْياً وَلَمْ يَعْرِفُوا الْخَجَلَ. لِذَلِكَ يَسْقُطُونَ بَيْنَ السَّاقِطِينَ. فِي وَقْتِ مُعَاقَبَتِهِمْ يَعْثِرُونَ قَالَ الرَّبُّ].
\par 16 هَكَذَا قَالَ الرَّبُّ: [قِفُوا عَلَى الطُّرُقِ وَانْظُرُوا وَاسْأَلُوا عَنِ السُّبُلِ الْقَدِيمَةِ: أَيْنَ هُوَ الطَّرِيقُ الصَّالِحُ؟ وَسِيرُوا فِيهِ فَتَجِدُوا رَاحَةً لِنُفُوسِكُمْ. وَلَكِنَّهُمْ قَالُوا: لاَ نَسِيرُ فِيهِ!
\par 17 وَأَقَمْتُ عَلَيْكُمْ رُقَبَاءَ قَائِلِينَ: اصْغُوا لِصَوْتِ الْبُوقِ. فَقَالُوا: لاَ نَصْغَى!
\par 18 لِذَلِكَ اسْمَعُوا يَا أَيُّهَا الشُّعُوبُ وَاعْرِفِي أَيَّتُهَا الْجَمَاعَةُ مَا هُوَ بَيْنَهُمْ.
\par 19 اِسْمَعِي أَيَّتُهَا الأَرْضُ: هَئَنَذَا جَالِبٌ شَرّاً عَلَى هَذَا الشَّعْبِ ثَمَرَ أَفْكَارِهِمْ لأَنَّهُمْ لَمْ يَصْغُوا لِكَلاَمِي وَشَرِيعَتِي رَفَضُوهَا.
\par 20 لِمَاذَا يَأْتِي لِي اللُّبَانُ مِنْ شَبَا وَقَصَبُ الذَّرِيرَةِ مِنْ أَرْضٍ بَعِيدَةٍ؟ مُحْرَقَاتُكُمْ غَيْرُ مَقْبُولَةٍ وَذَبَائِحُكُمْ لاَ تَلُذُّ لِي.
\par 21 لِذَلِكَ هَكَذَا قَالَ الرَّبُّ: هَئَنَذَا جَاعِلٌ لِهَذَا الشَّعْبِ مَعْثَرَاتٍ فَيَعْثُرُ بِهَا الآبَاءُ وَالأَبْنَاءُ مَعاً. الْجَارُ وَصَاحِبُهُ يَبِيدَانِ.
\par 22 هَكَذَا قَالَ الرَّبُّ: هُوَذَا شَعْبٌ قَادِمٌ مِنْ أَرْضِ الشِّمَالِ وَأُمَّةٌ عَظِيمَةٌ تَقُومُ مِنْ أَقَاصِي الأَرْضِ.
\par 23 تُمْسِكُ الْقَوْسَ وَالرُّمْحَ. هِيَ قَاسِيَةٌ لاَ تَرْحَمُ. صَوْتُهَا كَالْبَحْرِ يَعِجُّ وَعَلَى خَيْلٍ تَرْكَبُ مُصْطَفَّةً كَإِنْسَانٍ لِمُحَارَبَتِكِ يَا ابْنَةَ صِهْيَوْنَ].
\par 24 سَمِعْنَا خَبَرَهَا. ارْتَخَتْ أَيْدِينَا. أَمْسَكَنَا ضِيقٌ وَوَجَعٌ كَالْمَاخِضِ.
\par 25 لاَ تَخْرُجُوا إِلَى الْحَقْلِ وَفِي الطَّرِيقِ لاَ تَمْشُوا لأَنَّ سَيْفَ الْعَدُوِّ خَوْفٌ مِنْ كُلِّ جِهَةٍ.
\par 26 يَا ابْنَةَ شَعْبِي تَنَطَّقِي بِمِسْحٍ وَتَمَرَّغِي فِي الرَّمَادِ. نَوْحَ وَحِيدٍ اصْنَعِي لِنَفْسِكِ مَنَاحَةً مُرَّةً لأَنَّ الْمُخَرِّبَ يَأْتِي عَلَيْنَا بَغْتَةً.
\par 27 [قَدْ جَعَلْتُكَ بُرْجاً فِي شَعْبِي حِصْناً لِتَعْرِفَ وَتَمْتَحِنَ طَرِيقَهُ.
\par 28 كُلُّهُمْ عُصَاةٌ مُتَمَرِّدُونَ سَاعُونَ فِي الْوِشَايَةِ. هُمْ نُحَاسٌ وَحَدِيدٌ. كُلُّهُمْ مُفْسِدُونَ.
\par 29 اِحْتَرَقَ الْمِنْفَاخُ مِنَ النَّارِ. فَنِيَ الرِّصَاصُ. بَاطِلاً صَاغَ الصَّائِغُ وَالأَشْرَارُ لاَ يُفْرَزُونَ.
\par 30 فِضَّةً مَرْفُوضَةً يُدْعَوْنَ. لأَنَّ الرَّبَّ قَدْ رَفَضَهُمْ].

\chapter{7}

\par 1 اَلْكَلِمَةُ الَّتِي صَارَتْ إِلَى إِرْمِيَا مِنْ الرَّبِّ:
\par 2 [قِفْ فِي بَابِ بَيْتِ الرَّبِّ وَنَادِ هُنَاكَ بِهَذِهِ الْكَلِمَةِ: اسْمَعُوا كَلِمَةَ الرَّبِّ يَا جَمِيعَ يَهُوذَا الدَّاخِلِينَ فِي هَذِهِ الأَبْوَابِ لِتَسْجُدُوا لِلرَّبِّ.
\par 3 هَكَذَا قَالَ رَبُّ الْجُنُودِ إِلَهُ إِسْرَائِيلَ: أَصْلِحُوا طُرُقَكُمْ وَأَعْمَالَكُمْ فَأُسْكِنَكُمْ فِي هَذَا الْمَوْضِعِ.
\par 4 لاَ تَتَّكِلُوا عَلَى كَلاَمِ الْكَذِبِ قَائِلِينَ: هَيْكَلُ الرَّبِّ هَيْكَلُ الرَّبِّ هَيْكَلُ الرَّبِّ هُوَ!
\par 5 لأَنَّكُمْ إِنْ أَصْلَحْتُمْ إِصْلاَحاً طُرُقَكُمْ وَأَعْمَالَكُمْ إِنْ أَجْرَيْتُمْ عَدْلاً بَيْنَ الإِنْسَانِ وَصَاحِبِهِ
\par 6 إِنْ لَمْ تَظْلِمُوا الْغَرِيبَ وَالْيَتِيمَ وَالأَرْمَلَةَ وَلَمْ تَسْفِكُوا دَماً زَكِيّاً فِي هَذَا الْمَوْضِعِ وَلَمْ تَسِيرُوا وَرَاءَ آلِهَةٍ أُخْرَى لأَذَائِكُمْ
\par 7 فَإِنِّي أُسْكِنُكُمْ فِي هَذَا الْمَوْضِعِ فِي الأَرْضِ الَّتِي أَعْطَيْتُ لِآبَائِكُمْ مِنَ الأَزَلِ وَإِلَى الأَبَدِ.
\par 8 [هَا إِنَّكُمْ مُتَّكِلُونَ عَلَى كَلاَمِ الْكَذِبِ الَّذِي لاَ يَنْفَعُ.
\par 9 أَتَسْرِقُونَ وَتَقْتُلُونَ وَتَزْنُونَ وَتَحْلِفُونَ كَذِباً وَتُبَخِّرُونَ لِلْبَعْلِ وَتَسِيرُونَ وَرَاءَ آلِهَةٍ أُخْرَى لَمْ تَعْرِفُوهَا
\par 10 ثُمَّ تَأْتُونَ وَتَقِفُونَ أَمَامِي فِي هَذَا الْبَيْتِ الَّذِي دُعِيَ بِاسْمِي عَلَيْهِ وَتَقُولُونَ: قَدْ أُنْقِذْنَا. حَتَّى تَعْمَلُوا كُلَّ هَذِهِ الرَّجَاسَاتِ.
\par 11 هَلْ صَارَ هَذَا الْبَيْتُ الَّذِي دُعِيَ بِاسْمِي عَلَيْهِ مَغَارَةَ لُصُوصٍ فِي أَعْيُنِكُمْ؟ هَئَنَذَا أَيْضاً قَدْ رَأَيْتُ يَقُولُ الرَّبُّ.
\par 12 لَكِنِ اذْهَبُوا إِلَى مَوْضِعِي الَّذِي فِي شِيلُوهَ الَّذِي أَسْكَنْتُ فِيهِ اسْمِي أَوَّلاً وَانْظُرُوا مَا صَنَعْتُ بِهِ مِنْ أَجْلِ شَرِّ شَعْبِي إِسْرَائِيلَ.
\par 13 وَالآنَ مِنْ أَجْلِ عَمَلِكُمْ هَذِهِ الأَعْمَالَ يَقُولُ الرَّبُّ وَقَدْ كَلَّمْتُكُمْ مُبَكِّراً وَمُكَلِّماً فَلَمْ تَسْمَعُوا وَدَعَوْتُكُمْ فَلَمْ تُجِيبُوا
\par 14 أَصْنَعُ بِالْبَيْتِ الَّذِي دُعِيَ بِاسْمِي عَلَيْهِ الَّذِي أَنْتُمْ مُتَّكِلُونَ عَلَيْهِ وَبِالْمَوْضِعِ الَّذِي أَعْطَيْتُكُمْ وَآبَاءَكُمْ إِيَّاهُ كَمَا صَنَعْتُ بِشِيلُوهَ.
\par 15 وَأَطْرَحُكُمْ مِنْ أَمَامِي كَمَا طَرَحْتُ كُلَّ إِخْوَتِكُمْ كُلَّ نَسْلِ أَفْرَايِمَ.
\par 16 وَأَنْتَ فَلاَ تُصَلِّ لأَجْلِ هَذَا الشَّعْبِ وَلاَ تَرْفَعْ لأَجْلِهِمْ دُعَاءً وَلاَ صَلاَةً وَلاَ تُلِحَّ عَلَيَّ لأَنِّي لاَ أَسْمَعُك.
\par 17 [أَمَا تَرَى مَاذَا يَعْمَلُونَ فِي مُدُنِ يَهُوذَا وَفِي شَوَارِعِ أُورُشَلِيمَ؟
\par 18 الأَبْنَاءُ يَلْتَقِطُونَ حَطَباً وَالآبَاءُ يُوقِدُونَ النَّارَ وَالنِّسَاءُ يَعْجِنَّ الْعَجِينَ لِيَصْنَعْنَ كَعْكاً لِمَلِكَةِ السَّمَاوَاتِ وَلِسَكْبِ سَكَائِبَ لِآلِهَةٍ أُخْرَى لِيُغِيظُونِي.
\par 19 أَفَإِيَّايَ يُغِيظُونَ يَقُولُ الرَّبُّ؟ أَلَيْسَ أَنْفُسَهُمْ لأَجْلِ خِزْيِ وُجُوهِهِمْ؟].
\par 20 لِذَلِكَ هَكَذَا قَالَ السَّيِّدُ الرَّبُّ: [هَا غَضَبِي وَغَيْظِي يَنْسَكِبَانِ عَلَى هَذَا الْمَوْضِعِ عَلَى النَّاسِ وَعَلَى الْبَهَائِمِ وَعَلَى شَجَرِ الْحَقْلِ وَعَلَى ثَمَرِ الأَرْضِ فَيَتَّقِدَانِ وَلاَ يَنْطَفِئَانِ]
\par 21 هَكَذَا قَالَ رَبُّ الْجُنُودِ إِلَهُ إِسْرَائِيلَ: [ضُمُّوا مُحْرَقَاتِكُمْ إِلَى ذَبَائِحِكُمْ وَكُلُوا لَحْماً.
\par 22 لأَنِّي لَمْ أُكَلِّمْ آبَاءَكُمْ وَلاَ أَوْصَيْتُهُمْ يَوْمَ أَخْرَجْتُهُمْ مِنْ أَرْضِ مِصْرَ مِنْ جِهَةِ مُحْرَقَةٍ وَذَبِيحَةٍ.
\par 23 بَلْ إِنَّمَا أَوْصَيْتُهُمْ بِهَذَا الأَمْرِ: اسْمَعُوا صَوْتِي فَأَكُونَ لَكُمْ إِلَهاً وَأَنْتُمْ تَكُونُونَ لِي شَعْباً وَسِيرُوا فِي كُلِّ الطَّرِيقِ الَّذِي أُوصِيكُمْ بِهِ لِيُحْسَنَ إِلَيْكُمْ.
\par 24 فَلَمْ يَسْمَعُوا وَلَمْ يَمِيلُوا أُذْنَهُمْ بَلْ سَارُوا فِي مَشُورَاتِ وَعِنَادِ قَلْبِهِمِ الشِّرِّير وَأَعْطُوا الْقَفَا لاَ الْوَجْهَ.
\par 25 فَمِنَ الْيَوْمِ الَّذِي خَرَجَ فِيهِ آبَاؤُكُمْ مِنْ أَرْضِ مِصْرَ إِلَى هَذَا الْيَوْمِ أَرْسَلْتُ إِلَيْكُمْ كُلَّ عَبِيدِي الأَنْبِيَاءِ مُبَكِّراً كُلَّ يَوْمٍ وَمُرْسِلاً
\par 26 فَلَمْ يَسْمَعُوا لِي وَلَمْ يَمِيلُوا أُذُنَهُمْ بَلْ صَلَّبُوا رِقَابَهُمْ. أَسَاءُوا أَكْثَرَ مِنْ آبَائِهِمْ.
\par 27 فَتُكَلِّمُهُمْ بِكُلِّ هَذِهِ الْكَلِمَاتِ وَلاَ يَسْمَعُونَ لَكَ وَتَدْعُوهُمْ وَلاَ يُجِيبُونَكَ.
\par 28 فَتَقُولُ لَهُمْ: هَذِهِ هِيَ الأُمَّةُ الَّتِي لَمْ تَسْمَعْ لِصَوْتِ الرَّبِّ إِلَهِهَا وَلَمْ تَقْبَلْ تَأْدِيباً. بَادَ الْحَقُّ وَقُطِعَ عَنْ أَفْوَاهِهِمْ.
\par 29 [جُزِّي شَعْرَكِ وَاطْرَحِيهِ وَارْفَعِي عَلَى الْهِضَابِ مَرْثَاةً لأَنَّ الرَّبَّ قَدْ رَفَضَ وَرَذَلَ جِيلَ رِجْزِهِ.
\par 30 لأَنَّ بَنِي يَهُوذَا قَدْ عَمِلُوا الشَّرَّ فِي عَيْنَيَّ يَقُولُ الرَّبُّ. وَضَعُوا مَكْرُهَاتِهِمْ فِي الْبَيْتِ الَّذِي دُعِيَ بِاسْمِي لِيُنَجِّسُوهُ.
\par 31 وَبَنُوا مُرْتَفَعَاتِ تُوفَةَ الَّتِي فِي وَادِي ابْنِ هِنُّومَ لِيُحْرِقُوا بَنِيهِمْ وَبَنَاتِهِمْ بِالنَّارِ الَّذِي لَمْ آمُرْ بِهِ وَلاَ صَعِدَ عَلَى قَلْبِي.
\par 32 [لِذَلِكَ هَا هِيَ أَيَّامٌ تَأْتِي يَقُولُ الرَّبُّ وَلاَ يُسَمَّى بَعْدُ تُوفَةُ وَلاَ وَادِي ابْنِ هِنُّومَ بَلْ وَادِي الْقَتْلِ. وَيَدْفِنُونَ فِي تُوفَةَ حَتَّى لاَ يَكُونَ مَوْضِعٌ.
\par 33 وَتَصِيرُ جُثَثُ هَذَا الشَّعْبِ أَكْلاً لِطُيُورِ السَّمَاءِ وَلِوُحُوشِ الأَرْضِ وَلاَ مُزْعِجَ.
\par 34 وَأُبَطِّلُ مِنْ مُدُنِ يَهُوذَا وَمِنْ شَوَارِعِ أُورُشَلِيمَ صَوْتَ الطَّرَبِ وَصَوْتَ الْفَرَحِ صَوْتَ الْعَرِيسِ وَصَوْتَ الْعَرُوسِ لأَنَّ الأَرْضَ تَصِيرُ خَرَاباً].

\chapter{8}

\par 1 [فِي ذَلِكَ الزَّمَانِ يَقُولُ الرَّبُّ يُخْرِجُونَ عِظَامَ مُلُوكِ يَهُوذَا وَعِظَامَ رُؤَسَائِهِ وَعِظَامَ الْكَهَنَةِ وَعِظَامَ الأَنْبِيَاءِ وَعِظَامَ سُكَّانِ أُورُشَلِيمَ مِنْ قُبُورِهِمْ
\par 2 وَيَبْسُطُونَهَا لِلشَّمْسِ وَلِلْقَمَرِ وَلِكُلِّ جُنُودِ السَّمَاوَاتِ الَّتِي أَحَبُّوهَا وَالَّتِي عَبَدُوهَا وَالَّتِي سَارُوا وَرَاءَهَا وَالَّتِي اسْتَشَارُوهَا وَالَّتِي سَجَدُوا لَهَا. لاَ تُجْمَعُ وَلاَ تُدْفَنُ بَلْ تَكُونُ دِمْنَةً عَلَى وَجْهِ الأَرْضِ.
\par 3 وَيُخْتَارُ الْمَوْتُ عَلَى الْحَيَاةِ عِنْدَ كُلِّ الْبَقِيَّةِ الْبَاقِيَةِ مِنْ هَذِهِ الْعَشِيرَةِ الشِّرِّيرَةِ الْبَاقِيَةِ فِي كُلِّ الأَمَاكِنِ الَّتِي طَرَدْتُهُمْ إِلَيْهَا يَقُولُ رَبُّ الْجُنُودِ].
\par 4 وَتَقُولُ لَهُمْ هَكَذَا قَالَ الرَّبُّ: [هَلْ يَسْقُطُونَ وَلاَ يَقُومُونَ أَوْ يَرْتَدُّ أَحَدٌ وَلاَ يَرْجِعُ؟
\par 5 فَلِمَاذَا ارْتَدَّ هَذَا الشَّعْبُ فِي أُورُشَلِيمَ ارْتِدَاداً دَائِماً؟ تَمَسَّكُوا بِالْمَكْرِ. أَبُوا أَنْ يَرْجِعُوا.
\par 6 صَغَيْتُ وَسَمِعْتُ. بِغَيْرِ الْمُسْتَقِيمِ يَتَكَلَّمُونَ. لَيْسَ أَحَدٌ يَتُوبُ عَنْ شَرِّهِ قَائِلاً: مَاذَا عَمِلْتُ؟ كُلُّ وَاحِدٍ رَجَعَ إِلَى مَسْرَاهُ كَفَرَسٍ ثَائِرٍ فِي الْحَرْبِ.
\par 7 بَلِ اللَّقْلَقُ فِي السَّمَاوَاتِ يَعْرِفُ مِيعَادَهُ وَالْيَمَامَةُ وَالسُّنُوْنَةُ الْمُزَقْزِقَةُ حَفِظَتَا وَقْتَ مَجِيئِهِمَا. أَمَّا شَعْبِي فَلَمْ يَعْرِفْ قَضَاءَ الرَّبِّ.
\par 8 كَيْفَ تَقُولُونَ: نَحْنُ حُكَمَاءُ وَشَرِيعَةُ الرَّبِّ مَعَنَا؟ حَقّاً إِنَّهُ إِلَى الْكَذِبِ حَوَّلَهَا قَلَمُ الْكَتَبَةِ الْكَاذِبُ.
\par 9 خَزِيَ الْحُكَمَاءُ. ارْتَاعُوا وَأُخِذُوا. هَا قَدْ رَفَضُوا كَلِمَةَ الرَّبِّ فَأَيَّةُ حِكْمَةٍ لَهُمْ؟
\par 10 لِذَلِكَ أُعْطِي نِسَاءَهُمْ لِآخَرِينَ وَحُقُولَهُمْ لِمَالِكِينَ لأَنَّهُمْ مِنَ الصَّغِيرِ إِلَى الْكَبِيرِ كُلُّ وَاحِدٍ مُولَعٌ بِالرِّبْحِ مِنَ النَّبِيِّ إِلَى الْكَاهِنِ كُلُّ وَاحِدٍ يَعْمَلُ بِالْكَذِبِ.
\par 11 وَيَشْفُونَ كَسْرَ بِنْتِ شَعْبِي عَلَى عَثَمٍ قَائِلِينَ: سَلاَمٌ سَلاَمٌ! وَلاَ سَلاَمَ.
\par 12 هَلْ خَزُوا لأَنَّهُمْ عَمِلُوا رِجْساً؟ بَلْ لَمْ يَخْزُوا خِزْياً وَلَمْ يَعْرِفُوا الْخَجَلَ! لِذَلِكَ يَسْقُطُونَ بَيْنَ السَّاقِطِينَ. فِي وَقْتِ مُعَاقَبَتِهِمْ يَعْثُرُونَ قَالَ الرَّبُّ].
\par 13 [نَزْعاً أَنْزِعُهُمْ يَقُولُ الرَّبُّ. لاَ عِنَبَ فِي الْجَفْنَةِ وَلاَ تِينَ فِي التِّينَةِ وَالْوَرَقُ ذَبُلَ وَأُعْطِيهِمْ مَا يَزُولُ عَنْهُمْ.
\par 14 لِمَاذَا نَحْنُ جُلُوسٌ؟ اجْتَمِعُوا فَلْنَدْخُلْ إِلَى الْمُدُنِ الْحَصِينَةِ وَنَصْمُتْ هُنَاكَ. لأَنَّ الرَّبَّ إِلَهَنَا قَدْ أَصْمَتَنَا وَأَسْقَانَا مَاءَ الْعَلْقَمِ لأَنَّنَا قَدْ أَخْطَأْنَا إِلَى الرَّبِّ.
\par 15 اِنْتَظَرْنَا السَّلاَمَ وَلَمْ يَكُنْ خَيْرٌ وَزَمَانَ الشِّفَاءِ وَإِذَا رُعْبٌ.
\par 16 مِنْ دَانَ سُمِعَتْ حَمْحَمَةُ خَيْلِهِ. عِنْدَ صَوْتِ صَهِيلِ جِيَادِهِ ارْتَجَفَتْ كُلُّ الأَرْضِ. فَأَتُوا وَأَكَلُوا الأَرْضَ وَمِلأَهَا الْمَدِينَةَ وَالسَّاكِنِينَ فِيهَا.
\par 17 لأَنِّي هَئَنَذَا مُرْسِلٌ عَلَيْكُمْ حَيَّاتٍ أَفَاعِيَ لاَ تُرْقَى فَتَلْدَغُكُمْ يَقُولُ الرَّبُّ].
\par 18 مَنْ مُفَرِّجٌ عَنِّي الْحُزْنَ؟ قَلْبِي فِيَّ سَقِيمٌ.
\par 19 هُوَذَا صَوْتُ اسْتِغَاثَةِ بِنْتِ شَعْبِي مِنْ أَرْضٍ بَعِيدَةٍ. أَلَعَلَّ الرَّبَّ لَيْسَ فِي صِهْيَوْنَ أَوْ مَلِكَهَا لَيْسَ فِيهَا؟ لِمَاذَا أَغَاظُونِي بِمَنْحُوتَاتِهِمْ بِأَبَاطِيلَ غَرِيبَةٍ؟
\par 20 مَضَى الْحَصَادُ انْتَهَى الصَّيْفُ وَنَحْنُ لَمْ نَخْلُصْ!
\par 21 مِنْ أَجْلِ سَحْقِ بِنْتِ شَعْبِي انْسَحَقْتُ. حَزِنْتُ. أَخَذَتْنِي دَهْشَةٌ.
\par 22 أَلَيْسَ بَلَسَانٌ فِي جِلْعَادَ أَمْ لَيْسَ هُنَاكَ طَبِيبٌ؟ فَلِمَاذَا لَمْ تُعْصَبْ بِنْتُ شَعْبِي؟

\chapter{9}

\par 1 يَا لَيْتَ رَأْسِي مَاءٌ وَعَيْنَيَّ يَنْبُوعُ دُمُوعٍ فَأَبْكِيَ نَهَاراً وَلَيْلاً قَتْلَى بِنْتِ شَعْبِي.
\par 2 يَا لَيْتَ لِي فِي الْبَرِّيَّةِ مَبِيتَ مُسَافِرِينَ فَأَتْرُكَ شَعْبِي وَأَنْطَلِقَ مِنْ عِنْدِهِمْ لأَنَّهُمْ جَمِيعاً زُنَاةٌ جَمَاعَةُ خَائِنِينَ.
\par 3 يَمُدُّونَ أَلْسِنَتَهُمْ كَقِسِيِّهِمْ لِلْكَذِبِ. لاَ لِلْحَقِّ قَوُوا فِي الأَرْضِ. لأَنَّهُمْ خَرَجُوا مِنْ شَرٍّ إِلَى شَرٍّ وَإِيَّايَ لَمْ يَعْرِفُوا يَقُولُ الرَّبُّ.
\par 4 اِحْتَرِزُوا كُلُّ وَاحِدٍ مِنْ صَاحِبِهِ وَعَلَى كُلِّ أَخٍ لاَ تَتَّكِلُوا لأَنَّ كُلَّ أَخٍ يَعْقِبُ عَقِباً وَكُلَّ صَاحِبٍ يَسْعَى فِي الْوِشَايَةِ.
\par 5 وَيَخْتِلُ الإِنْسَانُ صَاحِبَهُ وَلاَ يَتَكَلَّمُونَ بِالْحَقِّ. عَلَّمُوا أَلْسِنَتَهُمُ التَّكَلُّمَ بِالْكَذِبِ وَتَعِبُوا فِي الاِفْتِرَاءِ.
\par 6 مَسْكَنُكَ فِي وَسَطِ الْمَكْرِ. بِالْمَكْرِ أَبُوا أَنْ يَعْرِفُونِي يَقُولُ الرَّبُّ.
\par 7 لِذَلِكَ هَكَذَا قَالَ رَبُّ الْجُنُودِ: [هَئَنَذَا أُنَقِّيهِمْ وَأَمْتَحِنُهُمْ. لأَنِّي مَاذَا أَعْمَلُ مِنْ أَجْلِ بِنْتِ شَعْبِي؟
\par 8 لِسَانُهُمْ سَهْمٌ قَتَّالٌ يَتَكَلَّمُ بِالْغِشِّ. بِفَمِهِ يُكَلِّمُ صَاحِبَهُ بِسَلاَمٍ وَفِي قَلْبِهِ يَضَعُ لَهُ كَمِيناً.
\par 9 أَفَمَا أُعَاقِبُهُمْ عَلَى هَذِهِ يَقُولُ الرَّبُّ؟ أَمْ لاَ تَنْتَقِمُ نَفْسِي مِنْ أُمَّةٍ كَهَذِهِ؟].
\par 10 عَلَى الْجِبَالِ أَرْفَعُ بُكَاءً وَمَرْثَاةً وَعَلَى مَرَاعِي الْبَرِّيَّةِ نَدْباً لأَنَّهَا احْتَرَقَتْ فَلاَ إِنْسَانَ عَابِرٌ وَلاَ يُسْمَعُ صَوْتُ الْمَاشِيَةِ. مِنْ طَيْرِ السَّمَاوَاتِ إِلَى الْبَهَائِمِ هَرَبَتْ مَضَتْ.
\par 11 وَأَجْعَلُ أُورُشَلِيمَ رُجَماً وَمَأْوَى بَنَاتِ آوَى وَمُدُنَ يَهُوذَا أَجْعَلُهَا خَرَاباً بِلاَ سَاكِنٍ.
\par 12 مَنْ هُوَ الإِنْسَانُ الْحَكِيمُ الَّذِي يَفْهَمُ هَذِهِ وَالَّذِي كَلَّمَهُ فَمُ الرَّبِّ فَيُخْبِرُ بِهَا؟ لِمَاذَا بَادَتِ الأَرْضُ وَاحْتَرَقَتْ كَبَرِّيَّةٍ بِلاَ عَابِرٍ؟
\par 13 فَقَالَ الرَّبُّ: [عَلَى تَرْكِهِمْ شَرِيعَتِي الَّتِي جَعَلْتُهَا أَمَامَهُمْ وَلَمْ يَسْمَعُوا لِصَوْتِي وَلَمْ يَسْلُكُوا بِهَا.
\par 14 بَلْ سَلَكُوا وَرَاءَ عِنَادِ قُلُوبِهِمْ وَوَرَاءَ الْبَعْلِيمِ الَّتِي عَلَّمَهُمْ إِيَّاهَا آبَاؤُهُمْ.
\par 15 لِذَلِكَ هَكَذَا قَالَ رَبُّ الْجُنُودِ إِلَهُ إِسْرَائِيلَ: هَئَنَذَا أُطْعِمُ هَذَا الشَّعْبَ أَفْسَنْتِيناً وَأَسْقِيهِمْ مَاءَ الْعَلْقَمِ
\par 16 وَأُبَدِّدُهُمْ فِي أُمَمٍ لَمْ يَعْرِفُوهَا هُمْ وَلاَ آبَاؤُهُمْ وَأُطْلِقُ وَرَاءَهُمُ السَّيْفَ حَتَّى أُفْنِيَهُمْ].
\par 17 هَكَذَا قَالَ رَبُّ الْجُنُودِ: [تَأَمَّلُوا وَادْعُوا النَّادِبَاتِ فَيَأْتِينَ وَأَرْسِلُوا إِلَى الْحَكِيمَاتِ فَيُقْبِلْنَ
\par 18 وَيُسْرِعْنَ وَيَرْفَعْنَ عَلَيْنَا مَرْثَاةً فَتَذْرِفَ أَعْيُنُنَا دُمُوعاً وَتَفِيضَ أَجْفَانُنَا مَاءً.
\par 19 لأَنَّ صَوْتَ رِثَايَةٍ سُمِعَ مِنْ صِهْيَوْنَ: كَيْفَ أُهْلِكْنَا؟ خَزِينَا جِدّاً لأَنَّنَا تَرَكْنَا الأَرْضَ لأَنَّهُمْ هَدَمُوا مَسَاكِنَنَا.
\par 20 بَلِ اسْمَعْنَ أَيَّتُهَا النِّسَاءُ كَلِمَةَ الرَّبِّ وَلْتَقْبَلْ آذَانُكُنَّ كَلِمَةَ فَمِهِ وَعَلِّمْنَ بَنَاتِكُنَّ الرِّثَايَةَ وَالْمَرْأَةُ صَاحِبَتَهَا النَّدْبَ!
\par 21 لأَنَّ الْمَوْتَ طَلَعَ إِلَى كُوانَا دَخَلَ قُصُورَنَا لِيَقْطَعَ الأَطْفَالَ مِنْ خَارِجٍ وَالشُّبَّانَ مِنَ السَّاحَاتِ.
\par 22 تَكَلَّمَ. هَكَذَا يَقُولُ الرَّبُّ. وَتَسْقُطُ جُثَّةُ الإِنْسَانِ كَدِمْنَةٍ عَلَى وَجْهِ الْحَقْلِ وَكَقَبْضَةٍ وَرَاءَ الْحَاصِدِ وَلَيْسَ مَنْ يَجْمَعُ!].
\par 23 هَكَذَا قَالَ الرَّبُّ: [لاَ يَفْتَخِرَنَّ الْحَكِيمُ بِحِكْمَتِهِ وَلاَ يَفْتَخِرِ الْجَبَّارُ بِجَبَرُوتِهِ وَلاَ يَفْتَخِرِ الْغَنِيُّ بِغِنَاهُ.
\par 24 بَلْ بِهَذَا لِيَفْتَخِرَنَّ الْمُفْتَخِرُ: بِأَنَّهُ يَفْهَمُ وَيَعْرِفُنِي أَنِّي أَنَا الرَّبُّ الصَّانِعُ رَحْمَةً وَقَضَاءً وَعَدْلاً فِي الأَرْضِ لأَنِّي بِهَذِهِ أُسَرُّ يَقُولُ الرَّبُّ.
\par 25 [هَا أَيَّامٌ تَأْتِي يَقُولُ الرَّبُّ وَأُعَاقِبُ كُلَّ مَخْتُونٍ وَأَغْلَفَ.
\par 26 مِصْرَ وَيَهُوذَا وَأَدُومَ وَبَنِي عَمُّونَ وَمُوآبَ وَكُلَّ مَقْصُوصِي الشَّعْرِ مُسْتَدِيراً السَّاكِنِينَ فِي الْبَرِّيَّةِ لأَنَّ كُلَّ الأُمَمِ غُلْفٌ وَكُلَّ بَيْتِ إِسْرَائِيلَ غُلْفُ الْقُلُوبِ].

\chapter{10}

\par 1 اِسْمَعُوا الْكَلِمَةَ الَّتِي تَكَلَّمَ بِهَا الرَّبُّ عَلَيْكُمْ يَا بَيْتَ إِسْرَائِيلَ.
\par 2 هَكَذَا قَالَ الرَّبُّ: [لاَ تَتَعَلَّمُوا طَرِيقَ الأُمَمِ وَمِنْ آيَاتِ السَّمَاوَاتِ لاَ تَرْتَعِبُوا لأَنَّ الأُمَمَ تَرْتَعِبُ مِنْهَا.
\par 3 لأَنَّ فَرَائِضَ الأُمَمِ بَاطِلَةٌ. لأَنَّهَا شَجَرَةٌ يَقْطَعُونَهَا مِنَ الْوَعْرِ. صَنْعَةُ يَدَيْ نَجَّارٍ بِالْقَدُومِ.
\par 4 بِالْفِضَّةِ وَالذَّهَبِ يُزَيِّنُونَهَا وَبِالْمَسَامِيرِ وَالْمَطَارِقِ يُشَدِّدُونَهَا فَلاَ تَتَحَرَّكُ.
\par 5 هِيَ كَاللَّعِينِ فِي مَقْثَأَةٍ فَلاَ تَتَكَلَّمُ! تُحْمَلُ حَمْلاً لأَنَّهَا لاَ تَمْشِي! لاَ تَخَافُوهَا لأَنَّهَا لاَ تَضُرُّ وَلاَ فِيهَا أَنْ تَصْنَعَ خَيْراً].
\par 6 لاَ مِثْلَ لَكَ يَا رَبُّ! عَظِيمٌ أَنْتَ وَعَظِيمٌ اسْمُكَ فِي الْجَبَرُوتِ.
\par 7 مَنْ لاَ يَخَافُكَ يَا مَلِكَ الشُّعُوبِ؟ لأَنَّهُ بِكَ يَلِيقُ. لأَنَّهُ فِي جَمِيعِ حُكَمَاءِ الشُّعُوبِ وَفِي كُلِّ مَمَالِكِهِمْ لَيْسَ مِثْلَكَ.
\par 8 بَلِدُوا وَحَمِقُوا مَعاً. أَدَبُ أَبَاطِيلَ هُوَ الْخَشَبُ.
\par 9 فِضَّةٌ مُطَرَّقَةٌ تُجْلَبُ مِنْ تَرْشِيشَ وَذَهَبٌ مِنْ أُوفَازَ صَنْعَةُ صَانِعٍ وَيَدَيْ صَائِغٍ. أَسْمَانْجُونِيٌّ وَأُرْجُوَانٌ لِبَاسُهَا. كُلُّهَا صَنْعَةُ حُكَمَاءَ.
\par 10 أَمَّا الرَّبُّ الإِلَهُ فَحَقٌّ. هُوَ إِلَهٌ حَيٌّ وَمَلِكٌ أَبَدِيٌّ. مِنْ سُخْطِهِ تَرْتَعِدُ الأَرْضُ وَلاَ تَطِيقُ الأُمَمُ غَضَبَهُ.
\par 11 هَكَذَا تَقُولُونَ لَهُمْ: [الآلِهَةُ الَّتِي لَمْ تَصْنَعِ السَّمَاوَاتِ وَالأَرْضَ تَبِيدُ مِنَ الأَرْضِ وَمِنْ تَحْتِ هَذِهِ السَّمَاوَاتِ.
\par 12 صَانِعُ الأَرْضِ بِقُوَّتِهِ مُؤَسِّسُ الْمَسْكُونَةِ بِحِكْمَتِهِ وَبِفَهْمِهِ بَسَطَ السَّمَاوَاتِ.
\par 13 إِذَا أَعْطَى قَوْلاً تَكُونُ كَثْرَةُ مِيَاهٍ فِي السَّمَاوَاتِ وَيُصْعِدُ السَّحَابَ مِنْ أَقَاصِي الأَرْضِ. صَنَعَ بُرُوقاً لِلْمَطَرِ وَأَخْرَجَ الرِّيحَ مِنْ خَزَائِنِهِ.
\par 14 بَلِدَ كُلُّ إِنْسَانٍ مِنْ مَعْرِفَتِهِ. خَزِيَ كُلُّ صَائِغٍ مِنَ التِّمْثَالِ لأَنَّ مَسْبُوكَهُ كَذِبٌ وَلاَ رُوحَ فِيهِ.
\par 15 هِيَ بَاطِلَةٌ صَنْعَةُ الأَضَالِيلِ. فِي وَقْتِ عِقَابِهَا تَبِيدُ.
\par 16 لَيْسَ كَهَذِهِ نَصِيبُ يَعْقُوبَ. لأَنَّهُ مُصَوِّرُ الْجَمِيعِ وَإِسْرَائِيلُ قَضِيبُ مِيرَاثِهِ. رَبُّ الْجُنُودِ اسْمُهُ.
\par 17 اِجْمَعِي مِنَ الأَرْضِ حُزَمَكِ أَيَّتُهَا السَّاكِنَةُ فِي الْحِصَارِ.
\par 18 لأَنَّهُ هَكَذَا قَالَ الرَّبُّ: [هَئَنَذَا رَامٍ مِنْ مِقْلاَعٍ سُكَّانَ الأَرْضِ هَذِهِ الْمَرَّةَ وَأُضَيِّقُ عَلَيْهِمْ لِكَيْ يَشْعُرُوا].
\par 19 وَيْلٌ لِي مِنْ أَجْلِ سَحْقِي! ضَرْبَتِي عَدِيمَةُ الشِّفَاءِ! فَقُلْتُ: إِنَّمَا هَذِهِ مُصِيبَةٌ فَأَحْتَمِلُهَا.
\par 20 خَيْمَتِي خَرِبَتْ وَكُلُّ أَطْنَابِي قُطِعَتْ. بَنِيَّ خَرَجُوا عَنِّي وَلَيْسُوا. لَيْسَ مَنْ يَبْسُطُ بَعْدُ خَيْمَتِي وَيُقِيمُ شُقَقِي.
\par 21 لأَنَّ الرُّعَاةَ بَلِدُوا وَالرَّبَّ لَمْ يَطْلُبُوا. مِنْ أَجْلِ ذَلِكَ لَمْ يَنْجَحُوا وَكُلُّ رَعِيَّتِهِمْ تَبَدَّدَتْ.
\par 22 هُوَذَا صَوْتُ خَبَرٍ جَاءَ وَاضْطِرَابٌ عَظِيمٌ مِنْ أَرْضِ الشِّمَالِ لِجَعْلِ مُدُنِ يَهُوذَا خَرَاباً مَأْوَى بَنَاتِ آوَى.
\par 23 عَرَفْتُ يَا رَبُّ أَنَّهُ لَيْسَ لِلإِنْسَانِ طَرِيقُهُ. لَيْسَ لإِنْسَانٍ يَمْشِي أَنْ يَهْدِيَ خَطَوَاتِهِ.
\par 24 أَدِّبْنِي يَا رَبُّ وَلَكِنْ بِالْحَقِّ لاَ بِغَضَبِكَ لِئَلاَّ تُفْنِيَنِي.
\par 25 اُسْكُبْ غَضَبَكَ عَلَى الأُمَمِ الَّتِي لَمْ تَعْرِفْكَ وَعَلَى الْعَشَائِرِ الَّتِي لَمْ تَدْعُ بِاسْمِكَ. لأَنَّهُمْ أَكَلُوا يَعْقُوبَ. أَكَلُوهُ وَأَفْنُوهُ وَأَخْرَبُوا مَسْكَنَهُ.

\chapter{11}

\par 1 اَلْكَلاَمُ الَّذِي صَارَ إِلَى إِرْمِيَا مِنْ الرَّبِّ:
\par 2 [اسْمَعُوا كَلاَمَ هَذَا الْعَهْدِ وَكَلِّمُوا رِجَالَ يَهُوذَا وَسُكَّانَ أُورُشَلِيمَ.
\par 3 فَتَقُولُ لَهُمْ: هَكَذَا قَالَ الرَّبُّ إِلَهُ إِسْرَائِيلَ: مَلْعُونٌ الإِنْسَانُ الَّذِي لاَ يَسْمَعُ كَلاَمَ هَذَا الْعَهْدِ
\par 4 الَّذِي أَمَرْتُ بِهِ آبَاءَكُمْ يَوْمَ أَخْرَجْتُهُمْ مِنْ أَرْضِ مِصْرَ مِنْ كُورِ الْحَدِيدِ قَائِلاً: اسْمَعُوا صَوْتِي وَاعْمَلُوا بِهِ حَسَبَ كُلِّ مَا آمُرُكُمْ بِهِ فَتَكُونُوا لِي شَعْباً وَأَنَا أَكُونُ لَكُمْ إِلَهاً
\par 5 لأُقِيمَ الْحَلْفَ الَّذِي حَلَفْتُ لِآبَائِكُمْ أَنْ أُعْطِيَهُمْ أَرْضاً تَفِيضُ لَبَناً وَعَسَلاً كَهَذَا الْيَوْمِ]. فَأَجَبْتُ: [آمِينَ يَا رَبُّ].
\par 6 فَقَالَ الرَّبُّ لِي: [نَادِ بِكُلِّ هَذَا الْكَلاَمِ فِي مُدُنِ يَهُوذَا وَفِي شَوَارِعِ أُورُشَلِيمَ: اسْمَعُوا كَلاَمَ هَذَا الْعَهْدِ وَاعْمَلُوا بِهِ.
\par 7 لأَنِّي أَشْهَدْتُ عَلَى آبَائِكُمْ إِشْهَاداً يَوْمَ أَصْعَدْتُهُمْ مِنْ أَرْضِ مِصْرَ إِلَى هَذَا الْيَوْمِ مُبْكِراً وَمُشْهِداً قَائِلاً: اسْمَعُوا صَوْتِي.
\par 8 فَلَمْ يَسْمَعُوا وَلَمْ يُمِيلُوا أُذُنَهُمْ بَلْ سَلَكُوا كُلُّ وَاحِدٍ فِي عِنَادِ قَلْبِهِ الشِّرِّيرِ. فَجَلَبْتُ عَلَيْهِمْ كُلَّ كَلاَمِ هَذَا الْعَهْدِ الَّذِي أَمَرْتُهُمْ أَنْ يَصْنَعُوهُ وَلَمْ يَصْنَعُوهُ].
\par 9 وَقَالَ الرَّبُّ لِي: [تُوجَدُ فِتْنَةٌ بَيْنَ رِجَالِ يَهُوذَا وَسُكَّانِ أُورُشَلِيمَ.
\par 10 قَدْ رَجَعُوا إِلَى آثَامِ آبَائِهِمِ الأَوَّلِينَ الَّذِينَ أَبُوا أَنْ يَسْمَعُوا كَلاَمِي وَقَدْ ذَهَبُوا وَرَاءَ آلِهَةٍ أُخْرَى لِيَعْبُدُوهَا. قَدْ نَقَضَ بَيْتُ إِسْرَائِيلَ وَبَيْتُ يَهُوذَا عَهْدِي الَّذِي قَطَعْتُهُ مَعَ آبَائِهِمْ.
\par 11 لِذَلِكَ هَكَذَا قَالَ الرَّبُّ: هَئَنَذَا جَالِبٌ عَلَيْهِمْ شَرّاً لاَ يَسْتَطِيعُونَ أَنْ يَخْرُجُوا مِنْهُ وَيَصْرُخُونَ إِلَيَّ فَلاَ أَسْمَعُ لَهُمْ.
\par 12 فَيَنْطَلِقُ مُدُنُ يَهُوذَا وَسُكَّانُ أُورُشَلِيمَ وَيَصْرُخُونَ إِلَى الآلِهَةِ الَّتِي يُبَخِّرُونَ لَهَا فَلَنْ تُخَلِّصَهُمْ فِي وَقْتِ بَلِيَّتِهِمْ.
\par 13 لأَنَّهُ بِعَدَدِ مُدُنِكَ صَارَتْ آلِهَتُكَ يَا يَهُوذَا وَبِعَدَدِ شَوَارِعِ أُورُشَلِيمَ وَضَعْتُمْ مَذَابِحَ لِلْخِزْيِ مَذَابِحَ لِلتَّبْخِيرِ لِلْبَعْلِ.
\par 14 وَأَنْتَ فَلاَ تُصَلِّ لأَجْلِ هَذَا الشَّعْبِ وَلاَ تَرْفَعْ لأَجْلِهِمْ دُعَاءً وَلاَ صَلاَةً لأَنِّي لاَ أَسْمَعُ فِي وَقْتِ صُرَاخِهِمْ إِلَيَّ مِنْ قِبَلِ بَلِيَّتِهِمْ.
\par 15 [مَا لِحَبِيبَتِي فِي بَيْتِي؟ قَدْ عَمِلَتْ فَظَائِعَ كَثِيرَةً وَاللَّحْمُ الْمُقَدَّسُ قَدْ عَبَرَ عَنْكِ. إِذَا صَنَعْتِ الشَّرَّ حِينَئِذٍ تَبْتَهِجِينَ.
\par 16 دَعَا الرَّبُّ اسْمَكِ: زَيْتُونَةً خَضْرَاءَ ذَاتَ ثَمَرٍ جَمِيلِ الصُّورَةِ. بِصَوْتِ ضَجَّةٍ عَظِيمَةٍ أَوْقَدَ نَاراً عَلَيْهَا فَانْكَسَرَتْ أَغْصَانُهَا.
\par 17 وَرَبُّ الْجُنُودِ غَارِسُكِ قَدْ تَكَلَّمَ عَلَيْكِ شَرّاً مِنْ أَجْلِ شَرِّ بَيْتِ إِسْرَائِيلَ وَبَيْتِ يَهُوذَا الَّذِي صَنَعُوهُ ضِدَّ أَنْفُسِهِمْ لِيُغِيظُونِي بِتَبْخِيرِهِمْ لِلْبَعْلِ].
\par 18 وَالرَّبُّ عَرَّفَنِي فَعَرَفْتُ. حِينَئِذٍ أَرَيْتَنِي أَفْعَالَهُمْ.
\par 19 وَأَنَا كَخَرُوفٍ دَاجِنٍ يُسَاقُ إِلَى الذَّبْحِ وَلَمْ أَعْلَمْ أَنَّهُمْ فَكَّرُوا عَلَيَّ أَفْكَاراً قَائِلِينَ: [لِنُهْلِكِ الشَّجَرَةَ بِثَمَرِهَا وَنَقْطَعْهُ مِنْ أَرْضِ الأَحْيَاءِ فَلاَ يُذْكَرَ بَعْدُ اسْمُهُ].
\par 20 فَيَا رَبَّ الْجُنُودِ الْقَاضِيَ الْعَدْلَ فَاحِصَ الْكُلَى وَالْقَلْبِ دَعْنِي أَرَى انْتِقَامَكَ مِنْهُمْ لأَنِّي لَكَ كَشَفْتُ دَعْوَايَ.
\par 21 لِذَلِكَ هَكَذَا قَالَ الرَّبُّ عَنْ أَهْلِ عَنَاثُوثَ الَّذِينَ يَطْلُبُونَ نَفْسَكَ قَائِلِينَ: [لاَ تَتَنَبَّأْ بِاسْمِ الرَّبِّ فَلاَ تَمُوتَ بِيَدِنَا].
\par 22 لِذَلِكَ هَكَذَا قَالَ رَبُّ الْجُنُودِ: [هَئَنَذَا أُعَاقِبُهُمْ. يَمُوتُ الشُّبَّانُ بِالسَّيْفِ وَيَمُوتُ بَنُوهُمْ وَبَنَاتُهُمْ بِالْجُوعِ.
\par 23 وَلاَ تَكُونُ لَهُمْ بَقِيَّةٌ لأَنِّي أَجْلِبُ شَرّاً عَلَى أَهْلِ عَنَاثُوثَ سَنَةَ عِقَابِهِمْ].

\chapter{12}

\par 1 أَبَرُّ أَنْتَ يَا رَبُّ مِنْ أَنْ أُخَاصِمَكَ. لَكِنْ أُكَلِّمُكَ مِنْ جِهَةِ أَحْكَامِكَ. لِمَاذَا تَنْجَحُ طَرِيقُ الأَشْرَارِ؟ اطْمَأَنَّ كُلُّ الْغَادِرِينَ غَدْراً.
\par 2 غَرَسْتَهُمْ فَأَصَّلُوا. نَمُوا وَأَثْمَرُوا ثَمَراً. أَنْتَ قَرِيبٌ فِي فَمِهِمْ وَبَعِيدٌ مِنْ كُلاَهُمْ.
\par 3 وَأَنْتَ يَا رَبُّ عَرَفْتَنِي. رَأَيْتَنِي وَاخْتَبَرْتَ قَلْبِي مِنْ جِهَتِكَ. افْرِزْهُمْ كَغَنَمٍ لِلذَّبْحِ وَخَصِّصْهُمْ لِيَوْمِ الْقَتْلِ.
\par 4 حَتَّى مَتَى تَنُوحُ الأَرْضُ وَيَيْبَسُ عُشْبُ كُلِّ الْحَقْلِ؟ مِنْ شَرِّ السَّاكِنِينَ فِيهَا فَنِيَتِ الْبَهَائِمُ وَالطُّيُورُ لأَنَّهُمْ قَالُوا: [لاَ يَرَى آخِرَتَنَا].
\par 5 إِنْ جَرَيْتَ مَعَ الْمُشَاةِ فَأَتْعَبُوكَ فَكَيْفَ تُبَارِي الْخَيْلَ؟ وَإِنْ كُنْتَ مُنْبَطِحاً فِي أَرْضِ السَّلاَمِ فَكَيْفَ تَعْمَلُ فِي كِبْرِيَاءِ الأُرْدُنِّ؟
\par 6 لأَنَّ إِخْوَتَكَ أَنْفُسَهُمْ وَبَيْتَ أَبِيكَ قَدْ غَادَرُوكَ هُمْ أَيْضاً. هُمْ أَيْضاً نَادُوا وَرَاءَكَ بِصَوْتٍ عَالٍ. لاَ تَأْتَمِنْهُمْ إِذَا كَلَّمُوكَ بِالْخَيْرِ.
\par 7 [قَدْ تَرَكْتُ بَيْتِي. رَفَضْتُ مِيرَاثِي. دَفَعْتُ حَبِيبَةَ نَفْسِي لِيَدِ أَعْدَائِهَا.
\par 8 صَارَ لِي مِيرَاثِي كَأَسَدٍ فِي الْوَعْرِ. نَطَقَ عَلَيَّ بِصَوْتِهِ. مِنْ أَجْلِ ذَلِكَ أَبْغَضْتُهُ.
\par 9 جَارِحَةٌ ضَبُعٌ مِيرَاثِي لِي. الْجَوَارِحُ حَوَالَيْهِ عَلَيْهِ. هَلُمَّ اجْمَعُوا كُلَّ حَيَوَانِ الْحَقْلِ. ايتُوا بِهَا لِلأَكْلِ.
\par 10 رُعَاةٌ كَثِيرُونَ أَفْسَدُوا كَرْمِي دَاسُوا نَصِيبِي. جَعَلُوا نَصِيبِي الْمُشْتَهَى بَرِّيَّةً خَرِبَةً.
\par 11 جَعَلُوهُ خَرَاباً يَنُوحُ عَلَيَّ وَهُوَ خَرِبٌ. خَرِبَتْ كُلُّ الأَرْضِ لأَنَّهُ لاَ أَحَدَ يَضَعُ فِي قَلْبِهِ.
\par 12 عَلَى جَمِيعِ الرَّوَابِي فِي الْبَرِّيَّةِ أَتَى النَّاهِبُونَ لأَنَّ سَيْفاً لِلرَّبِّ يَأْكُلُ مِنْ أَقْصَى الأَرْضِ إِلَى أَقْصَى الأَرْضِ. لَيْسَ سَلاَمٌ لأَحَدٍ مِنَ الْبَشَرِ.
\par 13 زَرَعُوا حِنْطَةً وَحَصَدُوا شَوْكاً. أَعْيُوا وَلَمْ يَنْتَفِعُوا بَلْ خَزُوا مِنْ غَلاَّتِكُمْ مِنْ حُمُوِّ غَضَبِ الرَّبِّ].
\par 14 هَكَذَا قَالَ الرَّبُّ عَلَى جَمِيعِ جِيرَانِي الأَشْرَارِ الَّذِينَ يَلْمِسُونَ الْمِيرَاثَ الَّذِي أَوْرَثْتُهُ لِشَعْبِي إِسْرَائِيلَ: [هَئَنَذَا أَقْتَلِعُهُمْ عَنْ أَرْضِهِمْ وَأَقْتَلِعُ بَيْتَ يَهُوذَا مِنْ وَسَطِهِمْ.
\par 15 وَيَكُونُ بَعْدَ اقْتِلاَعِي إِيَّاهُمْ أَنِّي أَرْجِعُ فَأَرْحَمُهُمْ وَأَرُدُّهُمْ كُلَّ وَاحِدٍ إِلَى مِيرَاثِهِ وَكُلَّ وَاحِدٍ إِلَى أَرْضِهِ.
\par 16 وَيَكُونُ إِذَا تَعَلَّمُوا عِلْماً طُرُقَ شَعْبِي أَنِّي أَحْلِفُ بِاسْمِي: حَيٌّ هُوَ الرَّبُّ كَمَا عَلَّمُوا شَعْبِي أَنْ يَحْلِفُوا بِبَعْلٍ أَنَّهُمْ يُبْنَوْنَ فِي وَسَطِ شَعْبِي.
\par 17 وَإِنْ لَمْ يَسْمَعُوا فَإِنِّي أَقْتَلِعُ تِلْكَ الأُمَّةَ اقْتِلاَعاً وَأُبِيدُهَا يَقُولُ الرَّبُّ].

\chapter{13}

\par 1 هَكَذَا قَالَ الرَّبُّ لِي: [اذْهَبْ وَاشْتَرِ لِنَفْسِكَ مِنْطَقَةً مِنْ كَتَّانٍ وَضَعْهَا عَلَى حَقَوَيْكَ وَلاَ تُدْخِلْهَا فِي الْمَاءِ].
\par 2 فَاشْتَرَيْتُ الْمِنْطَقَةَ كَقَوْلِ الرَّبِّ وَوَضَعْتُهَا عَلَى حَقَوَيَّ.
\par 3 فَصَارَ كَلاَمُ الرَّبِّ إِلَيَّ ثَانِيَةً:
\par 4 [خُذِ الْمِنْطَقَةَ الَّتِي اشْتَرَيْتَهَا الَّتِي هِيَ عَلَى حَقَوَيْكَ وَقُمِ انْطَلِقْ إِلَى الْفُرَاتِ وَاطْمُرْهَا هُنَاكَ فِي شَقِّ صَخْرٍ].
\par 5 فَانْطَلَقْتُ وَطَمَرْتُهَا عِنْدَ الْفُرَاتِ كَمَا أَمَرَنِي الرَّبُّ.
\par 6 وَكَانَ بَعْدَ أَيَّامٍ كَثِيرَةٍ أَنَّ الرَّبَّ قَالَ لِي: [قُمِ انْطَلِقْ إِلَى الْفُرَاتِ وَخُذْ مِنْ هُنَاكَ الْمِنْطَقَةَ الَّتِي أَمَرْتُكَ أَنْ تَطْمُرَهَا هُنَاكَ].
\par 7 فَانْطَلَقْتُ إِلَى الْفُرَاتِ وَحَفَرْتُ وَأَخَذْتُ الْمِنْطَقَةَ مِنَ الْمَوْضِعِ الَّذِي طَمَرْتُهَا فِيهِ. وَإِذَا بِالْمِنْطَقَةِ قَدْ فَسَدَتْ. لاَ تَصْلُحُ لِشَيْءٍ.
\par 8 فَصَارَ كَلاَمُ الرَّبِّ إِلَيَّ:
\par 9 [هَكَذَا قَالَ الرَّبُّ: هَكَذَا أُفْسِدُ كِبْرِيَاءَ يَهُوذَا وَكِبْرِيَاءَ أُورُشَلِيمَ الْعَظِيمَةِ.
\par 10 هَذَا الشَّعْبُ الشِّرِّيرُ الَّذِي يَأْبَى أَنْ يَسْمَعَ كَلاَمِي الَّذِي يَسْلُكُ فِي عِنَادِ قَلْبِهِ وَيَسِيرُ وَرَاءَ آلِهَةٍ أُخْرَى لِيَعْبُدَهَا وَيَسْجُدَ لَهَا يَصِيرُ كَهَذِهِ الْمِنْطَقَةِ الَّتِي لاَ تَصْلَُحُ لِشَيْءٍ.
\par 11 لأَنَّهُ كَمَا تَلْتَصِقُ الْمِنْطَقَةُ بِحَقَوَيِ الإِنْسَانِ هَكَذَا أَلْصَقْتُ بِنَفْسِي كُلَّ بَيْتِ إِسْرَائِيلَ وَكُلَّ بَيْتِ يَهُوذَا يَقُولُ الرَّبُّ لِيَكُونُوا لِي شَعْباً وَاسْماً وَفَخْراً وَمَجْداً وَلَكِنَّهُمْ لَمْ يَسْمَعُوا.
\par 12 فَتَقُولُ لَهُمْ هَذِهِ الْكَلِمَةَ: هَكَذَا قَالَ الرَّبُّ إِلَهُ إِسْرَائِيلَ: كُلُّ زِقٍّ يَمْتَلِئُ خَمْراً. فَيَقُولُونَ لَكَ: أَمَا نَعْرِفُ مَعْرِفَةً أَنَّ كُلَّ زِقٍّ يَمْتَلِئُ خَمْراً؟
\par 13 فَتَقُولُ لَهُمْ: هَكَذَا قَالَ الرَّبُّ: هَئَنَذَا أَمْلَأُ كُلَّ سُكَّانِ هَذِهِ الأَرْضِ وَالْمُلُوكَ الْجَالِسِينَ لِدَاوُدَ عَلَى كُرْسِيِّهِ وَالْكَهَنَةَ وَالأَنْبِيَاءَ وَكُلَّ سُكَّانِ أُورُشَلِيمَ سُكْراً.
\par 14 وَأُحَطِّمُهُمُ الْوَاحِدَ عَلَى أَخِيهِ الآبَاءَ وَالأَبْنَاءَ مَعاً يَقُولُ الرَّبُّ. لاَ أُشْفِقُ وَلاَ أَتَرَأَّفُ وَلاَ أَرْحَمُ مِنْ إِهْلاَكِهِمْ].
\par 15 اِسْمَعُوا وَاصْغُوا. لاَ تَتَعَظَّمُوا لأَنَّ الرَّبَّ تَكَلَّمَ.
\par 16 أَعْطُوا الرَّبَّ إِلَهَكُمْ مَجْداً قَبْلَ أَنْ يَجْعَلَ ظَلاَماً وَقَبْلَمَا تَعْثُرُ أَرْجُلُكُمْ عَلَى جِبَالِ الْعَتَمَةِ فَتَنْتَظِرُونَ نُوراً فَيَجْعَلُهُ ظِلَّ مَوْتٍ وَظَلاَماً دَامِساً.
\par 17 وَإِنْ لَمْ تَسْمَعُوا ذَلِكَ فَإِنَّ نَفْسِي تَبْكِي فِي أَمَاكِنَ مُسْتَتِرَةً مِنْ أَجْلِ الْكِبْرِيَاءِ وَتَبْكِي عَيْنَيَّ بُكَاءً وَتَذْرِفُ الدُّمُوعَ لأَنَّهُ قَدْ سُبِيَ قَطِيعُ الرَّبِّ.
\par 18 قُلْ لِلْمَلِكِ وَلِلْمَلِكَةِ: [اتَّضِعَا وَاجْلِسَا لأَنَّهُ قَدْ هَبَطَ عَنْ رَأْسَيْكُمَا تَاجُ مَجْدِكُمَا].
\par 19 أُغْلِقَتْ مُدُنُ الْجَنُوبِ وَلَيْسَ مَنْ يَفْتَحُ. سُبِيَتْ يَهُوذَا كُلُّهَا. سُبِيَتْ بِالتَّمَامِ.
\par 20 اِرْفَعُوا أَعْيُنَكُمْ وَانْظُرُوا الْمُقْبِلِينَ مِنَ الشِّمَالِ. أَيْنَ الْقَطِيعُ الَّذِي أُعْطِيَ لَكِ غَنَمُ مَجْدِكِ؟
\par 21 مَاذَا تَقُولِينَ حِينَ يُعَاقِبُكِ وَقَدْ عَلَّمْتِهِمْ عَلَى نَفْسِكِ قُوَّاداً لِلرِّيَاسَةِ؟ أَمَا تَأْخُذُكِ الأَوْجَاعُ كَامْرَأَةٍ مَاخِضٍ؟
\par 22 وَإِنْ قُلْتِ فِي قَلْبِكِ: لِمَاذَا أَصَابَتْنِي هَذِهِ؟ - لأَجْلِ عَظَمَةِ إِثْمِكِ هُتِكَ ذَيْلاَكِ وَانْكَشَفَ عُنْفاً عَقِبَاكِ.
\par 23 هَلْ يُغَيِّرُ الْكُوشِيُّ جِلْدَهُ أَوِ النَّمِرُ رُقَطَهُ؟ فَأَنْتُمْ أَيْضاً تَقْدِرُونَ أَنْ تَصْنَعُوا خَيْراً أَيُّهَا الْمُتَعَلِّمُونَ الشَّرَّ!
\par 24 فَأُبَدِّدُهُمْ كَقَشٍّ يَعْبُرُ مَعَ رِيحِ الْبَرِّيَّةِ.
\par 25 هَذِهِ قُرْعَتُكِ النَّصِيبُ الْمَكِيلُ لَكِ مِنْ عِنْدِي يَقُولُ الرَّبُّ لأَنَّكِ نَسِيتِنِي وَاتَّكَلْتِ عَلَى الْكَذِبِ.
\par 26 فَأَنَا أَيْضاً أَرْفَعُ ذَيْلَيْكِ عَلَى وَجْهِكِ فَيُرَى خِزْيُكِ.
\par 27 فِسْقُكِ وَصَهِيلُكِ وَرَذَالَةُ زِنَاكِ عَلَى الآكَامِ فِي الْحَقْلِ. قَدْ رَأَيْتُ مَكْرُهَاتِكِ. وَيْلٌ لَكِ يَا أُورُشَلِيمُ! لاَ تَطْهُرِينَ. حَتَّى مَتَى بَعْدُ؟

\chapter{14}

\par 1 كَلِمَةُ الرَّبِّ الَّتِي صَارَتْ إِلَى إِرْمِيَا مِنْ جِهَةِ الْقَحْطِ:
\par 2 [نَاحَتْ يَهُوذَا وَأَبْوَابُهَا ذَبُلَتْ. حَزِنَتْ إِلَى الأَرْضِ وَصَعِدَ عَوِيلُ أُورُشَلِيمَ.
\par 3 وَأَشْرَافُهُمْ أَرْسَلُوا أَصَاغِرَهُمْ لِلْمَاءِ. أَتُوا إِلَى الأَجْبَابِ فَلَمْ يَجِدُوا مَاءً. رَجَعُوا بِآنِيَتِهِمْ فَارِغَةً. خَزُوا وَخَجِلُوا وَغَطُّوا رُؤُوسَهُمْ
\par 4 مِنْ أَجْلِ أَنَّ الأَرْضَ قَدْ تَشَقَّقَتْ. لأَنَّهُ لَمْ يَكُنْ مَطَرٌ عَلَى الأَرْضِ خَزِيَ الْفَلاَّحُونَ. غَطُّوا رُؤُوسَهُمْ.
\par 5 حَتَّى أَنَّ الإِيَّلَةَ أَيْضاً فِي الْحَقْلِ وَلَدَتْ وَتَرَكَتْ لأَنَّهُ لَمْ يَكُنْ كَلَأٌ.
\par 6 الْفَرَاءُ وَقَفَتْ عَلَى الْهِضَابِ تَسْتَنْشِقُ الرِّيحَ مِثْلَ بَنَاتِ آوَى. كَلَّتْ عُيُونُهَا لأَنَّهُ لَيْسَ عُشْبٌ].
\par 7 وَإِنْ تَكُنْ آثَامُنَا تَشْهَدُ عَلَيْنَا يَا رَبُّ فَاعْمَلْ لأَجْلِ اسْمِكَ. لأَنَّ مَعَاصِيَنَا كَثُرَتْ. إِلَيْكَ أَخْطَأْنَا.
\par 8 يَا رَجَاءَ إِسْرَائِيلَ مُخَلِّصَهُ فِي زَمَانِ الضِّيقِ لِمَاذَا تَكُونُ كَغَرِيبٍ فِي الأَرْضِ وَكَمُسَافِرٍ يَمِيلُ لِيَبِيتَ؟
\par 9 لِمَاذَا تَكُونُ كَإِنْسَانٍ قَدْ تَحَيَّرَ كَجَبَّارٍ لاَ يَسْتَطِيعُ أَنْ يُخَلِّصَ؟ وَأَنْتَ فِي وَسَطِنَا يَا رَبُّ وَقَدْ دُعِينَا بِاسْمِكَ. لاَ تَتْرُكْنَا!
\par 10 هَكَذَا قَالَ الرَّبُّ لِهَذَا الشَّعْبِ: [هَكَذَا أَحَبُّوا أَنْ يَجُولُوا. لَمْ يَمْنَعُوا أَرْجُلَهُمْ فَالرَّبُّ لَمْ يَقْبَلْهُمْ. الآنَ يَذْكُرُ إِثْمَهُمْ وَيُعَاقِبُ خَطَايَاهُمْ].
\par 11 وَقَالَ الرَّبُّ لِي: [لاَ تُصَلِّ لأَجْلِ هَذَا الشَّعْبِ لِلْخَيْرِ.
\par 12 حِينَ يَصُومُونَ لاَ أَسْمَعُ صُرَاخَهُمْ وَحِينَ يُصْعِدُونَ مُحْرَقَةً وَتَقْدِمَةً لاَ أَقْبَلُهُمْ بَلْ بِالسَّيْفِ وَالْجُوعِ وَالْوَبَإِ أَنَا أُفْنِيهِمْ].
\par 13 فَقُلْتُ: [آهِ أَيُّهَا السَّيِّدُ الرَّبُّ! هُوَذَا الأَنْبِيَاءُ يَقُولُونَ لَهُمْ لاَ تَرُونَ سَيْفاً وَلاَ يَكُونُ لَكُمْ جُوعٌ بَلْ سَلاَماً ثَابِتاً أُعْطِيكُمْ فِي هَذَا الْمَوْضِعِ].
\par 14 فَقَالَ الرَّبُّ لِي: [بِالْكَذِبِ يَتَنَبَّأُ الأَنْبِيَاءُ بِاسْمِي. لَمْ أُرْسِلْهُمْ وَلاَ أَمَرْتُهُمْ وَلاَ كَلَّمْتُهُمْ. بِرُؤْيَا كَاذِبَةٍ وَعِرَافَةٍ وَبَاطِلٍ وَمَكْرِ قُلُوبِهِمْ هُمْ يَتَنَبَّأُونَ لَكُمْ].
\par 15 لِذَلِكَ هَكَذَا قَالَ الرَّبُّ عَنِ الأَنْبِيَاءِ الَّذِينَ يَتَنَبَّأُونَ بِاسْمِي وَأَنَا لَمْ أُرْسِلْهُمْ وَهُمْ يَقُولُونَ: [لاَ يَكُونُ سَيْفٌ وَلاَ جُوعٌ فِي هَذِهِ الأَرْضِ]: [لِلسَّيْفِ وَالْجُوعِ يَفْنَى أُولَئِكَ الأَنْبِيَاءُ.
\par 16 وَالشَّعْبُ الَّذِي يَتَنَبَّأُونَ لَهُ يَكُونُ مَطْرُوحاً فِي شَوَارِعِ أُورُشَلِيمَ مِنْ جَرَى الْجُوعِ وَالسَّيْفِ وَلَيْسَ مَنْ يَدْفِنُهُمْ هُمْ وَنِسَاءَهُمْ وَبَنِيهِمْ وَبَنَاتِهِمْ وَأَسْكُبُ عَلَيْهِمْ شَرَّهُمْ].
\par 17 وَتَقُولُ لَهُمْ هَذِهِ الْكَلِمَةَ: [لِتَذْرِفْ عَيْنَايَ دُمُوعاً لَيْلاً وَنَهَاراً وَلاَ تَكُفَّا لأَنَّ الْعَذْرَاءَ بِنْتَ شَعْبِي سُحِقَتْ سَحْقاً عَظِيماً بِضَرْبَةٍ مُوجِعَةٍ جِدّاً.
\par 18 إِذَا خَرَجْتُ إِلَى الْحَقْلِ فَإِذَا الْقَتْلَى بِالسَّيْفِ. وَإِذَا دَخَلْتُ الْمَدِينَةَ فَإِذَا الْمَرْضَى بِالْجُوعِ لأَنَّ النَّبِيَّ وَالْكَاهِنَ كِلَيْهِمَا يَطُوفَانِ فِي الأَرْضِ وَلاَ يَعْرِفَانِ شَيْئاً].
\par 19 هَلْ رَفَضْتَ يَهُوذَا رَفْضاً أَوْ كَرِهَتْ نَفْسُكَ صِهْيَوْنَ؟ لِمَاذَا ضَرَبْتَنَا وَلاَ شِفَاءَ لَنَا؟ انْتَظَرْنَا السَّلاَمَ فَلَمْ يَكُنْ خَيْرٌ وَزَمَانَ الشِّفَاءِ فَإِذَا رُعْبٌ.
\par 20 قَدْ عَرَفْنَا يَا رَبُّ شَرَّنَا إِثْمَ آبَائِنَا لأَنَّنَا قَدْ أَخْطَأْنَا إِلَيْكَ.
\par 21 لاَ تَرْفُضْ لأَجْلِ اسْمِكَ. لاَ تَهِنْ كُرْسِيَّ مَجْدِكَ. اذْكُرْ. لاَ تَنْقُضْ عَهْدَكَ مَعَنَا.
\par 22 هَلْ يُوجَدُ فِي أَبَاطِيلِ الأُمَمِ مَنْ يُمْطِرُ أَوْ هَلْ تُعْطِي السَّمَاوَاتُ وَابِلاً؟ أَمَا أَنْتَ هُوَ الرَّبُّ إِلَهُنَا؟ فَنَرْجُوكَ لأَنَّكَ أَنْتَ صَنَعْتَ كُلَّ هَذِهِ.

\chapter{15}

\par 1 ثُمَّ قَالَ الرَّبُّ لِي: [وَإِنْ وَقَفَ مُوسَى وَصَمُوئِيلُ أَمَامِي لاَ تَكُونُ نَفْسِي نَحْوَ هَذَا الشَّعْبِ. اطْرَحْهُمْ مِنْ أَمَامِي فَيَخْرُجُوا.
\par 2 وَيَكُونُ إِذَا قَالُوا لَكَ: إِلَى أَيْنَ نَخْرُجُ؟ أَنَّكَ تَقُولُ لَهُمْ: هَكَذَا قَالَ الرَّبُّ: الَّذِينَ لِلْمَوْتِ فَإِلَى الْمَوْتِ وَالَّذِينَ لِلسَّيْفِ فَإِلَى السَّيْفِ وَالَّذِينَ لِلْجُوعِ فَإِلَى الْجُوعِ وَالَّذِينَ لِلسَّبْيِ فَإِلَى السَّبْيِ.
\par 3 وَأُوَكِّلُ عَلَيْهِمْ أَرْبَعَةَ أَنْوَاعٍ يَقُولُ الرَّبُّ: السَّيْفَ لِلْقَتْلِ وَالْكِلاَبَ لِلسَّحْبِ وَطُيُورَ السَّمَاءِ وَوُحُوشَ الأَرْضِ لِلأَكْلِ وَالإِهْلاَكِ.
\par 4 وَأَدْفَعُهُمْ لِلْقَلَقِ فِي كُلِّ مَمَالِكِ الأَرْضِ مِنْ أَجْلِ مَنَسَّى بْنِ حَزَقِيَّا مَلِكِ يَهُوذَا مِنْ أَجْلِ مَا صَنَعَ فِي أُورُشَلِيمَ.
\par 5 فَمَنْ يُشْفِقُ عَلَيْكِ يَا أُورُشَلِيمُ وَمَنْ يُعَزِّيكِ وَمَنْ يَمِيلُ لِيَسْأَلَ عَنْ سَلاَمَتِكِ؟
\par 6 أَنْتِ تَرَكْتِنِي يَقُولُ الرَّبُّ. إِلَى الْوَرَاءِ سِرْتِ. فَأَمُدُّ يَدِي عَلَيْكِ وَأُهْلِكُكِ. مَلِلْتُ مِنَ النَّدَامَةِ.
\par 7 وَأُذْرِيهِمْ بِمِذْرَاةٍ فِي أَبْوَابِ الأَرْضِ. أُثْكِلُ وَأُبِيدُ شَعْبِي. لَمْ يَرْجِعُوا عَنْ طُرُقِهِمْ.
\par 8 كَثُرَتْ لِي أَرَامِلُهُمْ أَكْثَرَ مِنْ رَمْلِ الْبِحَارِ. جَلَبْتُ عَلَيْهِمْ عَلَى أُمِّ الشُّبَّانِ نَاهِباً فِي الظَّهِيرَةِ. أَوْقَعْتُ عَلَيْهَا بَغْتَةً رَعْدَةً وَرُعُبَاتٍ.
\par 9 ذَبُلَتْ وَالِدَةُ السَّبْعَةِ. أَسْلَمَتْ نَفْسَهَا. غَرَبَتْ شَمْسُهَا إِذْ بَعْدُ نَهَارٌ. خَزِيَتْ وَخَجِلَتْ. أَمَّا بَقِيَّتُهُمْ فَلِلسَّيْفِ أَدْفَعُهَا أَمَامَ أَعْدَائِهِمْ يَقُولُ الرَّبُّ].
\par 10 وَيْلٌ لِي يَا أُمِّي لأَنَّكِ وَلَدْتِنِي إِنْسَانَ خِصَامٍ وَإِنْسَانَ نِزَاعٍ لِكُلِّ الأَرْضِ. لَمْ أَقْرِضْ وَلاَ أَقْرَضُونِي وَكُلُّ وَاحِدٍ يَلْعَنُنِي.
\par 11 قَالَ الرَّبُّ: [إِنِّي أَحُلُّكَ لِلْخَيْرِ. إِنِّي أَجْعَلُ الْعَدُوَّ يَتَضَرَّعُ إِلَيْكَ فِي وَقْتِ الشَّرِّ وَفِي وَقْتِ الضِّيقِ.
\par 12 هَلْ يَكْسِرُ الْحَدِيدُ الْحَدِيدَ الَّذِي مِنَ الشِّمَالِ وَالنُّحَاسَ؟
\par 13 ثَرْوَتُكَ وَخَزَائِنُكَ أَدْفَعُهَا لِلنَّهْبِ لاَ بِثَمَنٍ بَلْ بِكُلِّ خَطَايَاكَ وَفِي كُلِّ تُخُومِكَ.
\par 14 وَأُعَبِّرُكَ مَعَ أَعْدَائِكَ فِي أَرْضٍ لَمْ تَعْرِفْهَا لأَنَّ نَاراً قَدْ أُشْعِلَتْ بِغَضَبِي تُوقَدُ عَلَيْكُمْ].
\par 15 أَنْتَ يَا رَبُّ عَرَفْتَ. اذْكُرْنِي وَتَعَهَّدْنِي وَانْتَقِمْ لِي مِنْ مُضْطَهِدِيَّ. بِطُولِ أَنَاتِكَ لاَ تَأْخُذْنِي. اعْرِفِ احْتِمَالِي الْعَارَ لأَجْلِكَ.
\par 16 وُجِدَ كَلاَمُكَ فَأَكَلْتُهُ فَكَانَ كَلاَمُكَ لِي لِلْفَرَحِ وَلِبَهْجَةِ قَلْبِي لأَنِّي دُعِيتُ بِاسْمِكَ يَا رَبُّ إِلَهَ الْجُنُودِ.
\par 17 لَمْ أَجْلِسْ فِي مَحْفَلِ الْمَازِحِينَ مُبْتَهِجاً. مِنْ أَجْلِ يَدِكَ جَلَسْتُ وَحْدِي لأَنَّكَ قَدْ مَلَأْتَنِي غَضَباً.
\par 18 لِمَاذَا كَانَ وَجَعِي دَائِماً وَجُرْحِي عَدِيمَ الشِّفَاءِ يَأْبَى أَنْ يُشْفَى؟ أَتَكُونُ لِي مِثْلَ كَاذِبٍ مِثْلَ مِيَاهٍ غَيْرِ دَائِمَةٍ؟
\par 19 لِذَلِكَ هَكَذَا قَالَ الرَّبُّ: [إِنْ رَجَعْتَ أُرَجِّعْكَ فَتَقِفُ أَمَامِي. وَإِذَا أَخْرَجْتَ الثَّمِينَ مِنَ الْمَرْذُولِ فَمِثْلَ فَمِي تَكُونُ. هُمْ يَرْجِعُونَ إِلَيْكَ وَأَنْتَ لاَ تَرْجِعُ إِلَيْهِمْ.
\par 20 وَأَجْعَلُكَ لِهَذَا الشَّعْبِ سُورَ نُحَاسٍ حَصِيناً فَيُحَارِبُونَكَ وَلاَ يَقْدِرُونَ عَلَيْكَ لأَنِّي مَعَكَ لأُخَلِّصَكَ وَأُنْقِذَكَ يَقُولُ الرَّبُّ.
\par 21 فَأُنْقِذُكَ مِنْ يَدِ الأَشْرَارِ وَأَفْدِيكَ مِنْ كَفِّ الْعُتَاةِ].

\chapter{16}

\par 1 ثُمَّ صَارَ إِلَيَّ كَلاَمُ الرَّبِّ:
\par 2 [لاَ تَتَّخِذْ لِنَفْسِكَ امْرَأَةً وَلاَ يَكُنْ لَكَ بَنُونَ وَلاَ بَنَاتٌ فِي هَذَا الْمَوْضِعِ.
\par 3 لأَنَّهُ هَكَذَا قَالَ الرَّبُّ عَنِ الْبَنِينَ وَعَنِ الْبَنَاتِ الْمَوْلُودِينَ فِي هَذَا الْمَوْضِعِ وَعَنْ أُمَّهَاتِهِمِ اللَّوَاتِي وَلَدْنَهُمْ وَعَنْ آبَائِهِمِ الَّذِينَ وَلَدُوهُمْ فِي هَذِهِ الأَرْضِ:
\par 4 مِيتَاتِ أَمْرَاضٍ يَمُوتُونَ. لاَ يُنْدَبُونَ وَلاَ يُدْفَنُونَ بَلْ يَكُونُونَ دِمْنَةً عَلَى وَجْهِ الأَرْضِ وَبِالسَّيْفِ وَالْجُوعِ يَفْنُونَ وَتَكُونُ جُثَثُهُمْ أَكْلاً لِطُيُورِ السَّمَاءِ وَلِوُحُوشِ الأَرْضِ.
\par 5 لأَنَّهُ هَكَذَا قَالَ الرَّبُّ: لاَ تَدْخُلْ بَيْتَ النَّوْحِ وَلاَ تَمْضِ لِلنَّدْبِ وَلاَ تُعَزِّهِمْ لأَنِّي نَزَعْتُ سَلاَمِي مِنْ هَذَا الشَّعْبِ يَقُولُ الرَّبُّ - الإِحْسَانَ وَالْمَرَاحِمَ.
\par 6 فَيَمُوتُ الْكِبَارُ وَالصِّغَارُ فِي هَذِهِ الأَرْضِ. لاَ يُدْفَنُونَ وَلاَ يَنْدُبُونَهُمْ وَلاَ يَخْمِشُونَ أَنْفُسَهُمْ وَلاَ يَجْعَلُونَ قَرَعَةً مِنْ أَجْلِهِمْ.
\par 7 وَلاَ يَكْسِرُونَ خُبْزاً فِي الْمَنَاحَةِ لِيُعَزُّوهُمْ عَنْ مَيِّتٍ وَلاَ يَسْقُونَهُمْ كَأْسَ التَّعْزِيَةِ عَنْ أَبٍ أَوْ أُمٍّ.
\par 8 وَلاَ تَدْخُلْ بَيْتَ الْوَلِيمَةِ لِتَجْلِسَ مَعَهُمْ لِلأَكْلِ وَالشُِّرْبِ.
\par 9 لأَنَّهُ هَكَذَا قَالَ رَبُّ الْجُنُودِ إِلَهُ إِسْرَائِيلَ: هَئَنَذَا مُبَطِّلٌ مِنْ هَذَا الْمَوْضِعِ أَمَامَ أَعْيُنِكُمْ وَفِي أَيَّامِكُمْ صَوْتَ الطَّرَبِ وَصَوْتَ الْفَرَحِ صَوْتَ الْعَرِيسِ وَصَوْتَ الْعَرُوسِ.
\par 10 [وَيَكُونُ حِينَ تُخْبِرُ هَذَا الشَّعْبَ بِكُلِّ هَذِهِ الأُمُورِ أَنَّهُمْ يَقُولُونَ لَكَ: لِمَاذَا تَكَلَّمَ الرَّبُّ عَلَيْنَا بِكُلِّ هَذَا الشَّرِّ الْعَظِيمِ فَمَا هُوَ ذَنْبُنَا وَمَا هِيَ خَطِيَّتُنَا الَّتِي أَخْطَأْنَاهَا إِلَى الرَّبِّ إِلَهِنَا؟
\par 11 فَتَقُولُ لَهُمْ: مِنْ أَجْلِ أَنَّ آبَاءَكُمْ قَدْ تَرَكُونِي يَقُولُ الرَّبُّ وَذَهَبُوا وَرَاءَ آلِهَةٍ أُخْرَى وَعَبَدُوهَا وَسَجَدُوا لَهَا وَإِيَّايَ تَرَكُوا وَشَرِيعَتِي لَمْ يَحْفَظُوهَا.
\par 12 وَأَنْتُمْ أَسَأْتُمْ فِي عَمَلِكُمْ أَكْثَرَ مِنْ آبَائِكُمْ. وَهَا أَنْتُمْ ذَاهِبُونَ كُلُّ وَاحِدٍ وَرَاءَ عِنَادِ قَلْبِهِ الشِّرِّيرِ حَتَّى لاَ تَسْمَعُوا لِي.
\par 13 فَأَطْرُدُكُمْ مِنْ هَذِهِ الأَرْضِ إِلَى أَرْضٍ لَمْ تَعْرِفُوهَا أَنْتُمْ وَلاَ آبَاؤُكُمْ فَتَعْبُدُونَ هُنَاكَ آلِهَةً أُخْرَى نَهَاراً وَلَيْلاً حَيْثُ لاَ أُعْطِيكُمْ نِعْمَةً.
\par 14 [لِذَلِكَ هَا أَيَّامٌ تَأْتِي يَقُولُ الرَّبُّ وَلاَ يُقَالُ بَعْدُ: حَيٌّ هُوَ الرَّبُّ الَّذِي أَصْعَدَ بَنِي إِسْرَائِيلَ مِنْ أَرْضِ مِصْرَ
\par 15 بَل:ْ حَيٌّ هُوَ الرَّبُّ الَّذِي أَصْعَدَ بَنِي إِسْرَائِيلَ مِنْ أَرْضِ الشِّمَالِ وَمِنْ جَمِيعِ الأَرَاضِي الَّتِي طَرَدَهُمْ إِلَيْهَا. فَأُرْجِعُهُمْ إِلَى أَرْضِهِمِ الَّتِي أَعْطَيْتُ آبَاءَهُمْ إِيَّاهَا.
\par 16 هَئَنَذَا أُرْسِلُ إِلَى صَيَّادِينَ كَثِيرِينَ يَقُولُ الرَّبُّ فَيَصْطَادُونَهُمْ ثُمَّ بَعْدَ ذَلِكَ أُرْسِلُ إِلَى كَثِيرِينَ مِنَ الْقَانِصِينَ فَيَقْتَنِصُونَهُمْ عَنْ كُلِّ جَبَلٍ وَعَنْ كُلِّ أَكَمَةٍ وَمِنْ شُقُوقِ الصُّخُورِ.
\par 17 لأَنَّ عَيْنَيَّ عَلَى كُلِّ طُرُقِهِمْ. لَمْ تَسْتَتِرْ عَنْ وَجْهِي وَلَمْ يَخْتَفِ إِثْمُهُمْ مِنْ أَمَامِ عَيْنَيَّ.
\par 18 وَأُعَاقِبُ أَوَّلاً إِثْمَهُمْ وَخَطِيَّتَهُمْ ضِعْفَيْنِ لأَنَّهُمْ دَنَّسُوا أَرْضِي وَبِجُثَثِ مَكْرُهَاتِهِمْ وَرَجَاسَاتِهِمْ قَدْ مَلَأُوا مِيرَاثِي].
\par 19 يَا رَبُّ عِزِّي وَحِصْنِي وَمَلْجَإِي فِي يَوْمِ الضِّيقِ إِلَيْكَ تَأْتِي الأُمَمُ مِنْ أَطْرَافِ الأَرْضِ وَيَقُولُونَ: [إِنَّمَا وَرَثَ آبَاؤُنَا كَذِباً وَأَبَاطِيلَ وَمَا لاَ مَنْفَعَةَ فِيهِ].
\par 20 هَلْ يَصْنَعُ الإِنْسَانُ لِنَفْسِهِ آلِهَةً وَهِيَ لَيْسَتْ آلِهَةً؟
\par 21 [لِذَلِكَ هَئَنَذَا أُعَرِّفُهُمْ هَذِهِ الْمَرَّةَ يَدِي وَجَبَرُوتِي فَيَعْرِفُونَ أَنَّ اسْمِي يَهْوَهُ].

\chapter{17}

\par 1 خَطِيَّةُ يَهُوذَا مَكْتُوبَةٌ بِقَلَمٍ مِنْ حَدِيدٍ بِرَأْسٍ مِنَ الْمَاسِ مَنْقُوشَةٌ عَلَى لَوْحِ قَلْبِهِمْ وَعَلَى قُرُونِ مَذَابِحِكُمْ.
\par 2 كَذِكْرِ بَنِيهِمْ مَذَابِحَهُمْ وَسَوَارِيَهُمْ عِنْدَ أَشْجَارٍ خُضْرٍ عَلَى آكَامٍ مُرْتَفِعَةٍ.
\par 3 يَا جَبَلِي فِي الْحَقْلِ أَجْعَلُ ثَرْوَتَكَ كُلَّ خَزَائِنِكَ لِلنَّهْبِ وَمُرْتَفَعَاتِكَ لِلْخَطِيَّةِ فِي كُلِّ تُخُومِكَ.
\par 4 وَتَتَبَرَّأُ وَبِنَفْسِكَ عَنْ مِيرَاثِكَ الَّذِي أَعْطَيْتُكَ إِيَّاهُ وَأَجْعَلُكَ تَخْدِمُ أَعْدَاءَكَ فِي أَرْضٍ لَمْ تَعْرِفْهَا لأَنَّكُمْ قَدْ أَضْرَمْتُمْ نَاراً بِغَضَبِي تَتَّقِدُ إِلَى الأَبَدِ.
\par 5 هَكَذَا قَالَ الرَّبُّ: [مَلْعُونٌ الرَّجُلُ الَّذِي يَتَّكِلُ عَلَى الإِنْسَانِ وَيَجْعَلُ الْبَشَرَ ذِرَاعَهُ وَعَنِ الرَّبِّ يَحِيدُ قَلْبُهُ.
\par 6 وَيَكُونُ مِثْلَ الْعَرْعَرِ فِي الْبَادِيَةِ وَلاَ يَرَى إِذَا جَاءَ الْخَيْرُ بَلْ يَسْكُنُ الْحَرَّةَ فِي الْبَرِّيَّةِ أَرْضاً سَبِخَةً وَغَيْرَ مَسْكُونَةٍ.
\par 7 مُبَارَكٌ الرَّجُلُ الَّذِي يَتَّكِلُ عَلَى الرَّبِّ وَكَانَ الرَّبُّ مُتَّكَلَهُ
\par 8 فَإِنَّهُ يَكُونُ كَشَجَرَةٍ مَغْرُوسَةٍ عَلَى مِيَاهٍ وَعَلَى نَهْرٍ تَمُدُّ أُصُولَهَا وَلاَ تَرَى إِذَا جَاءَ الْحَرُّ وَيَكُونُ وَرَقُهَا أَخْضَرَ وَفِي سَنَةِ الْقَحْطِ لاَ تَخَافُ وَلاَ تَكُفُّ عَنِ الإِثْمَارِ.
\par 9 [اَلْقَلْبُ أَخْدَعُ مِنْ كُلِّ شَيْءٍ وَهُوَ نَجِيسٌ مَنْ يَعْرِفُهُ!
\par 10 أَنَا الرَّبُّ فَاحِصُ الْقَلْبِ مُخْتَبِرُ الْكُلَى لأُعْطِيَ كُلَّ وَاحِدٍ حَسَبَ طُرُقِهِ حَسَبَ ثَمَرِ أَعْمَالِهِ.
\par 11 حَجَلَةٌ تَحْضُنُ مَا لَمْ تَبِضْ مُحَصِّلُ الْغِنَى بِغَيْرِ حَقٍّ. فِي نَُِصْفِ أَيَّامِهِ يَتْرُكُهُ وَفِي آخِرَتِهِ يَكُونُ أَحْمَقَ!]
\par 12 كُرْسِيُّ مَجْدٍ مُرْتَفِعٌ مِنَ الاِبْتِدَاءِ هُوَ مَوْضِعُ مَقْدِسِنَا.
\par 13 أَيُّهَا الرَّبُّ رَجَاءُ إِسْرَائِيلَ كُلُّ الَّذِينَ يَتْرُكُونَكَ يَخْزُونَ. [الْحَائِدُونَ عَنِّي فِي التُّرَابِ يُكْتَبُونَ لأَنَّهُمْ تَرَكُوا الرَّبَّ يَنْبُوعَ الْمِيَاهِ الْحَيَّةِ].
\par 14 اِشْفِنِي يَا رَبُّ فَأُشْفَى. خَلِّصْنِي فَأُخَلَّصَ لأَنَّكَ أَنْتَ تَسْبِيحَتِي.
\par 15 هَا هُمْ يَقُولُونَ لِي: [أَيْنَ هِيَ كَلِمَةُ الرَّبِّ؟ لِتَأْتِ!]
\par 16 أَمَّا أَنَا فَلَمْ أَعْتَزِلْ عَنْ أَنْ أَكُونَ رَاعِياً وَرَاءَكَ وَلاَ اشْتَهَيْتُ يَوْمَ الْبَلِيَّةِ. أَنْتَ عَرَفْتَ. مَا خَرَجَ مِنْ شَفَتَيَّ كَانَ مُقَابِلَ وَجْهِكَ.
\par 17 لاَ تَكُنْ لِي رُعْباً. أَنْتَ مَلْجَإِي فِي يَوْمِ الشَّرِّ.
\par 18 لِيَخْزَ طَارِدِيَّ وَلاَ أَخْزَ أَنَا. لِيَرْتَعِبُوا هُمْ وَلاَ أَرْتَعِبْ أَنَا. إِجْلِبْ عَلَيْهِمْ يَوْمَ الشَّرِّ وَاسْحَقْهُمْ سَحْقاً مُضَاعَفاً.
\par 19 هَكَذَا قَالَ الرَّبُّ لِي: [اذْهَبْ وَقِفْ فِي بَابِ بَنِي الشَّعْبِ الَّذِي يَدْخُلُ مِنْهُ مُلُوكُ يَهُوذَا وَيَخْرُجُونَ مِنْهُ وَفِي كُلِّ أَبْوَابِ أُورُشَلِيمَ
\par 20 وَقُلْ لَهُمُ: اسْمَعُوا كَلِمَةَ الرَّبِّ يَا مُلُوكَ يَهُوذَا وَكُلَّ يَهُوذَا وَكُلَّ سُكَّانِ أُورُشَلِيمَ الدَّاخِلِينَ مِنْ هَذِهِ الأَبْوَابِ.
\par 21 هَكَذَا قَالَ الرَّبُّ: تَحَفَّظُوا بِأَنْفُسِكُمْ وَلاَ تَحْمِلُوا حِمْلاً يَوْمَ السَّبْتِ وَلاَ تُدْخِلُوهُ فِي أَبْوَابِ أُورُشَلِيمَ
\par 22 وَلاَ تُخْرِجُوا حِمْلاً مِنْ بُيُوتِكُمْ يَوْمَ السَّبْتِ وَلاَ تَعْمَلُوا شُغْلاً مَا بَلْ قَدِّسُوا يَوْمَ السَّبْتِ كَمَا أَمَرْتُ آبَاءَكُمْ.
\par 23 فَلَمْ يَسْمَعُوا وَلَمْ يَمِيلُوا أُذُنَهُمْ بَلْ قَسُّوا أَعْنَاقَهُمْ لِئَلاَّ يَسْمَعُوا وَلِئَلاَّ يَقْبَلُوا تَأْدِيباً.
\par 24 وَيَكُونُ إِذَا سَمِعْتُمْ لِي سَمْعاً يَقُولُ الرَّبُّ وَلَمْ تُدْخِلُوا حِمْلاً فِي أَبْوَابِ هَذِهِ الْمَدِينَةِ يَوْمَ السَّبْتِ بَلْ قَدَّسْتُمْ يَوْمَ السَّبْتِ وَلَمْ تَعْمَلُوا فِيهِ شُغْلاً مَا
\par 25 أَنَّهُ يَدْخُلُ فِي أَبْوَابِ هَذِهِ الْمَدِينَةِ مُلُوكٌ وَرُؤَسَاءُ جَالِسُونَ عَلَى كُرْسِيِّ دَاوُدَ رَاكِبُونَ فِي مَرْكَبَاتٍ وَعَلَى خَيْلٍ هُمْ وَرُؤَسَاؤُهُمْ رِجَالُ يَهُوذَا وَسُكَّانُ أُورُشَلِيمَ وَتُسْكَنُ هَذِهِ الْمَدِينَةُ إِلَى الأَبَدِ.
\par 26 وَيَأْتُونَ مِنْ مُدُنِ يَهُوذَا وَمِنْ حَوَالَيْ أُورُشَلِيمَ وَمِنْ أَرْضِ بِنْيَامِينَ وَمِنَ السَّهْلِ وَمِنَ الْجِبَالِ وَمِنَ الْجَنُوبِ يَأْتُونَ بِمُحْرَقَاتٍ وَذَبَائِحَ وَتَقْدِمَاتٍ وَلُبَانٍ وَيَدْخُلُونَ بِذَبَائِحِ شُكْرٍ إِلَى بَيْتِ الرَّبِّ.
\par 27 وَلَكِنْ إِنْ لَمْ تَسْمَعُوا لِي لِتُقَدِّسُوا يَوْمَ السَّبْتِ لِكَيْلاَ تَحْمِلُوا حِمْلاً وَلاَ تُدْخِلُوهُ فِي أَبْوَابِ أُورُشَلِيمَ يَوْمَ السَّبْتِ فَإِنِّي أُشْعِلُ نَاراً فِي أَبْوَابِهَا فَتَأْكُلُ قُصُورَ أُورُشَلِيمَ وَلاَ تَنْطَفِئُ].

\chapter{18}

\par 1 الْكَلاَمُ الَّذِي صَارَ إِلَى إِرْمِيَا مِنْ الرَّبِّ:
\par 2 [قُمِ انْزِلْ إِلَى بَيْتِ الْفَخَّارِيِّ وَهُنَاكَ أُسْمِعُكَ كَلاَمِي].
\par 3 فَنَزَلْتُ إِلَى بَيْتِ الْفَخَّارِيِّ وَإِذَا هُوَ يَصْنَعُ عَمَلاً عَلَى الدُّولاَبِ.
\par 4 فَفَسَدَ الْوِعَاءُ الَّذِي كَانَ يَصْنَعُهُ مِنَ الطِّينِ بِيَدِ الْفَخَّارِيِّ فَعَادَ وَعَمِلَهُ وِعَاءً آخَرَ كَمَا حَسُنَ فِي عَيْنَيِ الْفَخَّارِيِّ أَنْ يَصْنَعَهُ.
\par 5 فَصَارَ إِلَيَّ كَلاَمُ الرَّبِّ:
\par 6 [أَمَا أَسْتَطِيعُ أَنْ أَصْنَعَ بِكُمْ كَهَذَا الْفَخَّارِيِّ يَا بَيْتَ إِسْرَائِيلَ يَقُولُ الرَّبُّ؟ هُوَذَا كَالطِّينِ بِيَدِ الْفَخَّارِيِّ أَنْتُمْ هَكَذَا بِيَدِي يَا بَيْتَ إِسْرَائِيلَ.
\par 7 تَارَةً أَتَكَلَّمُ عَلَى أُمَّةٍ وَعَلَى مَمْلَكَةٍ بِالْقَلْعِ وَالْهَدْمِ وَالإِهْلاَكِ
\par 8 فَتَرْجِعُ تِلْكَ الأُمَّةُ الَّتِي تَكَلَّمْتُ عَلَيْهَا عَنْ شَرِّهَا فَأَنْدَمُ عَنِ الشَّرِّ الَّذِي قَصَدْتُ أَنْ أَصْنَعَهُ بِهَا.
\par 9 وَتَارَةً أَتَكَلَّمُ عَلَى أُمَّةٍ وَعَلَى مَمْلَكَةٍ بِالْبِنَاءِ وَالْغَرْسِ
\par 10 فَتَفْعَلُ الشَّرَّ فِي عَيْنَيَّ فَلاَ تَسْمَعُ لِصَوْتِي فَأَنْدَمُ عَنِ الْخَيْرِ الَّذِي قُلْتُ إِنِّي أُحْسِنُ إِلَيْهَا بِهِ.
\par 11 [فَالآنَ قُلْ لِرِجَالِ يَهُوذَا وَسُكَّانِ أُورُشَلِيمَ: هَكَذَا قَالَ الرَّبُّ: هَئَنَذَا مُصْدِرٌ عَلَيْكُمْ شَرّاً وَقَاصِدٌ عَلَيْكُمْ قَصْداً. فَارْجِعُوا كُلُّ وَاحِدٍ عَنْ طَرِيقِهِ الرَّدِيءِ وَأَصْلِحُوا طُرُقَكُمْ وَأَعْمَالَكُمْ].
\par 12 فَقَالُوا: [بَاطِلٌ! لأَنَّنَا نَسْعَى وَرَاءَ أَفْكَارِنَا وَكُلُّ وَاحِدٍ يَعْمَلُ حَسَبَ عِنَادِ قَلْبِهِ الرَّدِيءِ].
\par 13 لِذَلِكَ هَكَذَا قَالَ الرَّبُّ: [اسْأَلُوا بَيْنَ الأُمَمِ. مَنْ سَمِعَ كَهَذِهِ؟ مَا يُقْشَعَرُّ مِنْهُ جِدّاً عَمِلَتْ عَذْرَاءُ إِسْرَائِيلَ.
\par 14 هَلْ يَخْلُو صَخْرُ حَقْلِي مِنْ ثَلْجِ لُبْنَانَ؟ أَوْ هَلْ تَنْشَفُ الْمِيَاهُ الْمُنْفَجِرَةُ الْبَارِدَةُ الْجَارِيَةُ؟
\par 15 لأَنَّ شَعْبِي قَدْ نَسِينِي! بَخَّرُوا لِلْبَاطِلِ وَقَدْ أَعْثَرُوهُمْ فِي طُرُقِهِمْ فِي السُّبُلِ الْقَدِيمَةِ لِيَسْلُكُوا فِي شُعَبٍ فِي طَرِيقٍ غَيْرِ مُسَهَّلٍ
\par 16 لِتُجْعَلْ أَرْضُهُمْ خَرَاباً وَصَفِيراً أَبَدِيّاً. كُلُّ مَارٍّ فِيهَا يَدْهَشُ وَيَنْغِضُ رَأْسَهُ.
\par 17 كَرِيحٍ شَرْقِيَّةٍ أُبَدِّدُهُمْ أَمَامَ الْعَدُوِّ. أُرِيهِمِ الْقَفَا لاَ الْوَجْهَ فِي يَوْمِ مُصِيبَتِهِمْ].
\par 18 فَقَالُوا: [هَلُمَّ فَنُفَكِّرُ عَلَى إِرْمِيَا أَفْكَاراً لأَنَّ الشَّرِيعَةَ لاَ تَبِيدُ عَنِ الْكَاهِنِ وَلاَ الْمَشُورَةَ عَنِ الْحَكِيمِ وَلاَ الْكَلِمَةَ عَنِ النَّبِيِّ. هَلُمَّ فَنَضْرِبُهُ بِاللِّسَانِ وَلِكُلِّ كَلاَمِهِ لاَ نُصْغِ].
\par 19 أَصْغِ لِي يَا رَبُّ وَاسْمَعْ صَوْتَ أَخْصَامِي.
\par 20 هَلْ يُجَازَى عَنْ خَيْرٍ بِشَرٍّ؟ لأَنَّهُمْ حَفَرُوا حُفْرَةً لِنَفْسِي. اذْكُرْ وُقُوفِي أَمَامَكَ لأَتَكَلَّمَ عَنْهُمْ بِالْخَيْرِ لأَرُدَّ غَضَبَكَ عَنْهُمْ.
\par 21 لِذَلِكَ سَلِّمْ بَنِيهِمْ لِلْجُوعِ وَادْفَعْهُمْ لِيَدِ السَّيْفِ فَتَصِيرَ نِسَاؤُهُمْ ثَكَالَى وَأَرَامِلَ وَتَصِيرَ رِجَالُهُمْ قَتْلَى الْمَوْتِ وَشُبَّانُهُمْ مَضْرُوبِي السَّيْفِ فِي الْحَرْبِ.
\par 22 لِيُسْمَعْ صِيَاحٌ مِنْ بُيُوتِهِمْ إِذْ تَجْلِبُ عَلَيْهِمْ جَيْشاً بَغْتَةً. لأَنَّهُمْ حَفَرُوا حُفْرَةً لِيُمْسِكُونِي وَطَمَرُوا فِخَاخاً لِرِجْلَيَّ.
\par 23 وَأَنْتَ يَا رَبُّ عَرَفْتَ كُلَّ مَشُورَتِهِمْ عَلَيَّ لِلْمَوْتِ. لاَ تَصْفَحْ عَنْ إِثْمِهِمْ وَلاَ تَمْحُ خَطِيَّتَهُمْ مِنْ أَمَامِكَ بَلْ لِيَكُونُوا مُتَعَثِّرِينَ أَمَامَكَ. فِي وَقْتِ غَضَبِكَ عَامِلْهُمْ.

\chapter{19}

\par 1 هَكَذَا قَالَ الرَّبُّ: [اذْهَبْ وَاشْتَرِ إِبْرِيقَ فَخَّارِيٍّ مِنْ خَزَفٍ وَخُذْ مِنْ شُيُوخِ الشَّعْبِ وَمِنْ شُيُوخِ الْكَهَنَةِ
\par 2 وَاخْرُجْ إِلَى وَادِي ابْنِ هِنُّومَ الَّذِي عِنْدَ مَدْخَلِ بَابِ الْفَخَّارِ وَنَادِ هُنَاكَ بِالْكَلِمَاتِ الَّتِي أُكَلِّمُكَ بِهَا.
\par 3 وَقُلِ: اسْمَعُوا كَلِمَةَ الرَّبِّ يَا مُلُوكَ يَهُوذَا وَسُكَّانَ أُورُشَلِيمَ. هَكَذَا قَالَ رَبُّ الْجُنُودِ إِلَهُ إِسْرَائِيلَ: هَئَنَذَا جَالِبٌ عَلَى هَذَا الْمَوْضِعِ شَرّاً كُلُّ مَنْ سَمِعَ بِهِ تَطِنُّ أُذُنَاهُ.
\par 4 مِنْ أَجْلِ أَنَّهُمْ تَرَكُونِي وَأَنْكَرُوا هَذَا الْمَوْضِعَ وَبَخَّرُوا فِيهِ لِآلِهَةٍ أُخْرَى لَمْ يَعْرِفُوهَا هُمْ وَلاَ آبَاؤُهُمْ وَلاَ مُلُوكُ يَهُوذَا وَمَلَأُوا هَذَا الْمَوْضِعَ مِنْ دَمِ الأَزْكِيَاءِ
\par 5 وَبَنُوا مُرْتَفَعَاتٍ لِلْبَعْلِ لِيُحْرِقُوا أَوْلاَدَهُمْ بِالنَّارِ مُحْرَقَاتٍ لِلْبَعْلِ الَّذِي لَمْ أُوصِ وَلاَ تَكَلَّمْتُ بِهِ وَلاَ صَعِدَ عَلَى قَلْبِي.
\par 6 لِذَلِكَ هَا أَيَّامٌ تَأْتِي يَقُولُ الرَّبُّ وَلاَ يُدْعَى بَعْدُ هَذَا الْمَوْضِعُ تُوفَةَ وَلاَ وَادِي ابْنِ هِنُّومَ بَلْ وَادِي الْقَتْلِ.
\par 7 وَأَنْقُضُ مَشُورَةَ يَهُوذَا وَأُورُشَلِيمَ فِي هَذَا الْمَوْضِعِ وَأَجْعَلُهُمْ يَسْقُطُونَ بِالسَّيْفِ أَمَامَ أَعْدَائِهِمْ وَبِيَدِ طَالِبِي نُفُوسِهِمْ وَأَجْعَلُ جُثَثَهُمْ أَكْلاً لِطُيُورِ السَّمَاءِ وَلِوُحُوشِ الأَرْضِ.
\par 8 وَأَجْعَلُ هَذِهِ الْمَدِينَةَ لِلدَّهَشِ وَالصَّفِيرِ. كُلُّ عَابِرٍ بِهَا يَدْهَشُ وَيَصْفِرُ مِنْ أَجْلِ كُلِّ ضَرَبَاتِهَا.
\par 9 وَأُطْعِمُهُمْ لَحْمَ بَنِيهِمْ وَلَحْمَ بَنَاتِهِمْ فَيَأْكُلُونَ كُلُّ وَاحِدٍ لَحْمَ صَاحِبِهِ فِي الْحِصَارِ وَالضِّيقِ الَّذِي يُضَايِقُهُمْ بِهِ أَعْدَاؤُهُمْ وَطَالِبُو نُفُوسِهِمْ.
\par 10 ثُمَّ تَكْسِرُ الإِبْرِيقَ أَمَامَ أَعْيُنِ الْقَوْمِ الَّذِينَ يَسِيرُونَ مَعَكَ
\par 11 وَتَقُولُ لَهُمْ: هَكَذَا قَالَ رَبُّ الْجُنُودِ: هَكَذَا أَكْسِرُ هَذَا الشَّعْبَ وَهَذِهِ الْمَدِينَةَ كَمَا يُكْسَرُ وِعَاءُ الْفَخَّارِيِّ بِحَيْثُ لاَ يُمْكِنُ جَبْرُهُ بَعْدُ وَفِي تُوفَةَ يُدْفَنُونَ حَتَّى لاَ يَكُونَ مَوْضِعٌ لِلدَّفْنِ.
\par 12 هَكَذَا أَصْنَعُ لِهَذَا الْمَوْضِعِ يَقُولُ الرَّبُّ وَلِسُكَّانِهِ وَأَجْعَلُ هَذِهِ الْمَدِينَةَ مِثْلَ تُوفَةَ.
\par 13 وَتَكُونُ بُيُوتُ أُورُشَلِيمَ وَبُيُوتُ مُلُوكِ يَهُوذَا كَمَوْضِعِ تُوفَةَ نَجِسَةً كُلُّ الْبُيُوتِ الَّتِي بَخَّرُوا عَلَى سُطُوحِهَا لِكُلِّ جُنْدِ السَّمَاءِ وَسَكَبُوا سَكَائِبَ لِآلِهَةٍ أُخْرَى].
\par 14 ثُمَّ جَاءَ إِرْمِيَا مِنْ تُوفَةَ الَّتِي أَرْسَلَهُ الرَّبُّ إِلَيْهَا لِيَتَنَبَّأَ وَوَقَفَ فِي دَارِ بَيْتِ الرَّبِّ وَقَالَ لِكُلِّ الشَّعْبِ:
\par 15 [هَكَذَا قَالَ رَبُّ الْجُنُودِ إِلَهُ إِسْرَائِيلَ: هَئَنَذَا جَالِبٌ عَلَى هَذِهِ الْمَدِينَةِ وَعَلَى كُلِّ قُرَاهَا كُلَّ الشَّرِّ الَّذِي تَكَلَّمْتُ بِهِ عَلَيْهَا لأَنَّهُمْ صَلَّبُوا رِقَابَهُمْ فَلَمْ يَسْمَعُوا لِكَلاَمِي].

\chapter{20}

\par 1 وَسَمِعَ فَشْحُورُ بْنُ إِمِّيرَ الْكَاهِنُ (وَهُوَ نَاظِرٌ أَوَّلٌ فِي بَيْتِ الرَّبِّ) إِرْمِيَا يَتَنَبَّأُ بِهَذِهِ الْكَلِمَاتِ.
\par 2 فَضَرَبَ فَشْحُورُ إِرْمِيَا النَّبِيَّ وَجَعَلَهُ فِي الْمِقْطَرَةِ الَّتِي فِي بَابِ بِنْيَامِينَ الأَعْلَى الَّذِي عِنْدَ بَيْتِ الرَّبِّ.
\par 3 وَكَانَ فِي الْغَدِ أَنَّ فَشْحُورَ أَخْرَجَ إِرْمِيَا مِنَ الْمِقْطَرَةِ. فَقَالَ لَهُ إِرْمِيَا: [لَمْ يَدْعُ الرَّبُّ اسْمَكَ فَشْحُورَ بَلْ مَجُورَ مِسَّابِيبَ.
\par 4 لأَنَّهُ هَكَذَا قَالَ الرَّبُّ: هَئَنَذَا أَجْعَلُكَ خَوْفاً لِنَفْسِكَ وَلِكُلِّ مُحِبِّيكَ فَيَسْقُطُونَ بِسَيْفِ أَعْدَائِهِمْ وَعَيْنَاكَ تَنْظُرَانِ وَأَدْفَعُ كُلَّ يَهُوذَا لِيَدِ مَلِكِ بَابِلَ فَيَسْبِيهِمْ إِلَى بَابِلَ وَيَضْرِبُهُمْ بِالسَّيْفِ.
\par 5 وَأَدْفَعُ كُلَّ ثَرْوَةِ هَذِهِ الْمَدِينَةِ وَكُلَّ تَعَبِهَا وَكُلَّ مُثَمَّنَاتِهَا وَكُلَّ خَزَائِنِ مُلُوكِ يَهُوذَا أَدْفَعُهَا لِيَدِ أَعْدَائِهِمْ فَيَغْنَمُونَهَا وَيَأْخُذُونَهَا وَيُحْضِرُونَهَا إِلَى بَابِلَ.
\par 6 وَأَنْتَ يَا فَشْحُورُ وَكُلُّ سُكَّانِ بَيْتِكَ تَذْهَبُونَ فِي السَّبْيِ وَتَأْتِي إِلَى بَابِلَ وَهُنَاكَ تَمُوتُ وَهُنَاكَ تُدْفَنُ أَنْتَ وَكُلُّ مُحِبِّيكَ الَّذِينَ تَنَبَّأْتَ لَهُمْ بِالْكَذِبِ].
\par 7 قَدْ أَقْنَعْتَنِي يَا رَبُّ فَاقْتَنَعْتُ وَأَلْحَحْتَ عَلَيَّ فَغَلَبْتَ. صِرْتُ لِلضِّحْكِ كُلَّ النَّهَارِ. كُلُّ وَاحِدٍ اسْتَهْزَأَ بِي.
\par 8 لأَنِّي كُلَّمَا تَكَلَّمْتُ صَرَخْتُ. نَادَيْتُ: [ظُلْمٌ وَاغْتِصَابٌ!] لأَنَّ كَلِمَةَ الرَّبِّ صَارَتْ لِي لِلْعَارِ وَلِلسُّخْرَةِ كُلَّ النَّهَارِ.
\par 9 فَقُلْتُ: [لاَ أَذْكُرُهُ وَلاَ أَنْطِقُ بَعْدُ بِاسْمِهِ]. فَكَانَ فِي قَلْبِي كَنَارٍ مُحْرِقَةٍ مَحْصُورَةٍ فِي عِظَامِي فَمَلِلْتُ مِنَ الإِمْسَاكِ وَلَمْ أَسْتَطِعْ.
\par 10 لأَنِّي سَمِعْتُ مَذَمَّةً مِنْ كَثِيرِينَ. خَوْفٌ مِنْ كُلِّ جَانِبٍ. يَقُولُونَ: [اشْتَكُوا فَنَشْتَكِيَ عَلَيْهِ]. كُلُّ أَصْحَابِي يُرَاقِبُونَ ظَلْعِي قَائِلِينَ: [لَعَلَّهُ يُطْغَى فَنَقْدِرَ عَلَيْهِ وَنَنْتَقِمَ مِنْهُ].
\par 11 وَلَكِنَّ الرَّبَّ مَعِي كَجَبَّارٍ قَدِيرٍ. مِنْ أَجْلِ ذَلِكَ يَعْثُرُ مُضْطَهِدِيَّ وَلاَ يَقْدِرُونَ. خَزُوا جِدّاً لأَنَّهُمْ لَمْ يَنْجَحُوا خِزْياً أَبَدِيّاً لاَ يُنْسَى.
\par 12 فَيَا رَبَّ الْجُنُودِ مُخْتَبِرَ الصِّدِّيقِ نَاظِرَ الْكُلَى وَالْقَلْبِ دَعْنِي أَرَى نَقْمَتَكَ مِنْهُمْ لأَنِّي لَكَ كَشَفْتُ دَعْوَايَ.
\par 13 رَنِّمُوا لِلرَّبِّ. سَبِّحُوا الرَّبَّ لأَنَّهُ قَدْ أَنْقَذَ نَفْسَ الْمِسْكِينِ مِنْ يَدِ الأَشْرَارِ.
\par 14 مَلْعُونٌ الْيَوْمُ الَّذِي وُلِدْتُ فِيهِ! الْيَوْمُ الَّذِي وَلَدَتْنِي فِيهِ أُمِّي لاَ يَكُنْ مُبَارَكاً!
\par 15 مَلْعُونٌ الإِنْسَانُ الَّذِي بَشَّرَ أَبِي قَائِلاً: [قَدْ وُلِدَ لَكَ ابْنٌ] مُفَرِّحاً إِيَّاهُ فَرَحاً.
\par 16 وَلْيَكُنْ ذَلِكَ الإِنْسَانُ كَالْمُدُنِ الَّتِي قَلَبَهَا الرَّبُّ وَلَمْ يَنْدَمْ فَيَسْمَعَ صِيَاحاً فِي الصَّبَاحِ وَجَلَبَةً فِي وَقْتِ الظَّهِيرَةِ.
\par 17 لأَنَّهُ لَمْ يَقْتُلْنِي مِنَ الرَّحِمِ فَكَانَتْ لِي أُمِّي قَبْرِي وَرَحِمُهَا حُبْلَى إِلَى الأَبَدِ.
\par 18 لِمَاذَا خَرَجْتُ مِنَ الرَّحِمِ لأَرَى تَعَباً وَحُزْناً فَتَفْنَى بِالْخِزْيِ أَيَّامِي؟.

\chapter{21}

\par 1 اَلْكَلاَمُ الَّذِي صَارَ إِلَى إِرْمِيَا مِنْ الرَّبِّ حِينَ أَرْسَلَ إِلَيْهِ الْمَلِكُ صِدْقِيَّا فَشْحُورَ بْنَ مَلْكِيَّا وَصَفَنْيَا بْنَ مَعْسِيَّا الْكَاهِنَ قَائِلاً:
\par 2 [اسْأَلِ الرَّبَّ مِنْ أَجْلِنَا لأَنَّ نَبُوخَذْنَصَّرَ مَلِكَ بَابِلَ يُحَارِبُنَا. لَعَلَّ الرَّبَّ يَصْنَعُ مَعَنَا حَسَبَ كُلِّ عَجَائِبِهِ فَيَصْعَدَ عَنَّا].
\par 3 فَقَالَ لَهُمَا إِرْمِيَا: [هَكَذَا تَقُولاَنِ لِصِدْقِيَّا:
\par 4 هَكَذَا قَالَ الرَّبُّ إِلَهُ إِسْرَائِيلَ: هَئَنَذَا أَرُدُّ أَدَوَاتِ الْحَرْبِ الَّتِي بِيَدِكُمُ الَّتِي أَنْتُمْ مُحَارِبُونَ بِهَا مَلِكَ بَابِلَ وَالْكِلْدَانِيِّينَ الَّذِينَ يُحَاصِرُونَكُمْ خَارِجَ السُّورِ وَأَجْمَعُهُمْ فِي وَسَطِ هَذِهِ الْمَدِينَةِ.
\par 5 وَأَنَا أُحَارِبُكُمْ بِيَدٍ مَمْدُودَةٍ وَبِذِرَاعٍ شَدِيدَةٍ وَبِغَضَبٍ وَحُمُوٍّ وَغَيْظٍ عَظِيمٍ.
\par 6 وَأَضْرِبُ سُكَّانَ هَذِهِ الْمَدِينَةِ النَّاسَ وَالْبَهَائِمَ مَعاً. بِوَبَأٍ عَظِيمٍ يَمُوتُونَ.
\par 7 ثُمَّ بَعْدَ ذَلِكَ قَالَ الرَّبُّ: أَدْفَعُ صِدْقِيَّا مَلِكَ يَهُوذَا وَعَبِيدَهُ وَالشَّعْبَ وَالْبَاقِينَ فِي هَذِهِ الْمَدِينَةِ مِنَ الْوَبَإِ وَالسَّيْفِ وَالْجُوعِ لِيَدِ نَبُوخَذْنَصَّرَ مَلِكِ بَابِلَ وَلِيَدِ أَعْدَائِهِمْ وَلِيَدِ طَالِبِي نُفُوسِهِمْ فَيَضْرِبُهُمْ بِحَدِّ السَّيْفِ. لاَ يَتَرَأَّفُ عَلَيْهِمْ وَلاَ يُشْفِقُ وَلاَ يَرْحَمُ].
\par 8 [وَتَقُولُ لِهَذَا الشَّعْبِ: هَكَذَا قَالَ الرَّبُّ. هَئَنَذَا أَجْعَلُ أَمَامَكُمْ طَرِيقَ الْحَيَاةِ وَطَرِيقَ الْمَوْتِ.
\par 9 الَّذِي يُقِيمُ فِي هَذِهِ الْمَدِينَةِ يَمُوتُ بِالسَّيْفِ وَالْجُوعِ وَالْوَبَإِ. وَالَّذِي يَخْرُجُ وَيَسْقُطُ إِلَى الْكِلْدَانِيِّينَ الَّذِينَ يُحَاصِرُونَكُمْ يَحْيَا وَتَصِيرُ نَفْسُهُ لَهُ غَنِيمَةً.
\par 10 لأَنِّي قَدْ جَعَلْتُ وَجْهِي عَلَى هَذِهِ الْمَدِينَةِ لِلشَّرِّ لاَ لِلْخَيْرِ يَقُولُ الرَّبُّ. لِيَدِ مَلِكِ بَابِلَ تُدْفَعُ فَيُحْرِقُهَا بِالنَّارِ.
\par 11 [وَلِبَيْتِ مَلِكِ يَهُوذَا تَقُولُ: اسْمَعُوا كَلِمَةَ الرَّبِّ.
\par 12 يَا بَيْتَ دَاوُدَ هَكَذَا قَالَ الرَّبُّ: اقْضُوا فِي الصَّبَاحِ عَدْلاً وَأَنْقِذُوا الْمَغْصُوبَ مِنْ يَدِ الظَّالِمِ لِئَلاَّ يَخْرُجَ كَنَارٍ غَضَبِي فَيُحْرِقَ وَلَيْسَ مَنْ يُطْفِئُ مِنْ أَجْلِ شَرِّ أَعْمَالِكُمْ.
\par 13 هَئَنَذَا ضِدُّكِ يَا سَاكِنَةَ الْعُمْقِ صَخْرَةَ السَّهْلِ يَقُولُ الرَّبُّ. الَّذِينَ يَقُولُونَ: مَنْ يَنْزِلُ عَلَيْنَا وَمَنْ يَدْخُلُ إِلَى مَنَازِلِنَا؟
\par 14 وَلَكِنَّنِي أُعَاقِبُكُمْ حَسَبَ ثَمَرِ أَعْمَالِكُمْ يَقُولُ الرَّبُّ وَأُشْعِلُ نَاراً فِي وَعْرِهِ فَتَأْكُلُ مَا حَوَالَيْهَا].

\chapter{22}

\par 1 هَكَذَا قَالَ الرَّبُّ: [انْزِلْ إِلَى بَيْتِ مَلِكِ يَهُوذَا وَتَكَلَّمْ هُنَاكَ بِهَذِهِ الْكَلِمَةِ
\par 2 وَقُلِ: اسْمَعْ كَلِمَةَ الرَّبِّ يَا مَلِكَ يَهُوذَا الْجَالِسَ عَلَى كُرْسِيِّ دَاوُدَ أَنْتَ وَعَبِيدُكَ وَشَعْبُكَ الدَّاخِلِينَ فِي هَذِهِ الأَبْوَابِ.
\par 3 هَكَذَا قَالَ الرَّبُّ: أَجْرُوا حَقّاً وَعَدْلاً وَأَنْقِذُوا الْمَغْصُوبَ مِنْ يَدِ الظَّالِمِ وَالْغَرِيبَ وَالْيَتِيمَ وَالأَرْمَلَةَ لاَ تَضْطَهِدُوا وَلاَ تَظْلِمُوا وَلاَ تَسْفُكُوا دَماً زَكِيّاً فِي هَذَا الْمَوْضِعِ.
\par 4 لأَنَّكُمْ إِنْ فَعَلْتُمْ هَذَا الأَمْرَ يَدْخُلُ فِي أَبْوَابِ هَذَا الْبَيْتِ مُلُوكٌ جَالِسُونَ لِدَاوُدَ عَلَى كُرْسِيِّهِ رَاكِبِينَ فِي مَرْكَبَاتٍ وَعَلَى خَيْلٍ. هُوَ وَعَبِيدُهُ وَشَعْبُهُ.
\par 5 وَإِنْ لَمْ تَسْمَعُوا لِهَذِهِ الْكَلِمَاتِ فَقَدْ أَقْسَمْتُ بِنَفْسِي يَقُولُ الرَّبُّ إِنَّ هَذَا الْبَيْتَ يَكُونُ خَرَاباً.
\par 6 لأَنَّهُ هَكَذَا قَالَ الرَّبُّ عَنْ بَيْتِ مَلِكِ يَهُوذَا: جِلْعَادٌ أَنْتَ لِي. رَأْسٌ مِنْ لُبْنَانَ. إِنِّي أَجْعَلُكَ بَرِّيَّةً مُدُناً غَيْرَ مَسْكُونَةٍ.
\par 7 وَأُقَدِّسُ عَلَيْكَ مُهْلِكِينَ كُلَّ وَاحِدٍ وَآلاَتِهِ فَيَقْطَعُونَ خِيَارَ أَرْزِكَ وَيُلْقُونَهُ فِي النَّارِ.
\par 8 وَيَعْبُرُ أُمَمٌ كَثِيرَةٌ فِي هَذِهِ الْمَدِينَةِ وَيَقُولُونَ الْوَاحِدُ لِصَاحِبِهِ: لِمَاذَا فَعَلَ الرَّبُّ مِثْلَ هَذَا لِهَذِهِ الْمَدِينَةِ الْعَظِيمَةِ؟
\par 9 فَيَقُولُونَ: مِنْ أَجْلِ أَنَّهُمْ تَرَكُوا عَهْدَ الرَّبِّ إِلَهِهِمْ وَسَجَدُوا لِآلِهَةٍ أُخْرَى وَعَبَدُوهَا].
\par 10 لاَ تَبْكُوا مَيِّتاً وَلاَ تَنْدُبُوهُ. ابْكُوا ابْكُوا مَنْ يَمْضِي لأَنَّهُ لاَ يَرْجِعُ بَعْدُ فَيَرَى أَرْضَ مِيلاَدِهِ.
\par 11 لأَنَّهُ هَكَذَا قَالَ الرَّبُّ عَنْ شَلُّومَ بْنِ يُوشِيَّا مَلِكِ يَهُوذَا الْمَالِكِ عِوَضاً عَنْ يُوشِيَّا أَبِيهِ: [الَّذِي خَرَجَ مِنْ هَذَا الْمَوْضِعِ لاَ يَرْجِعُ إِلَيْهِ بَعْدُ.
\par 12 بَلْ فِي الْمَوْضِعِ الَّذِي سَبُوهُ إِلَيْهِ يَمُوتُ. وَهَذِهِ الأَرْضُ لاَ يَرَاهَا بَعْدُ].
\par 13 وَيْلٌ لِمَنْ يَبْنِي بَيْتَهُ بِغَيْرِ عَدْلٍ وَعَلاَلِيَهُ بِغَيْرِ حَقٍّ الَّذِي يَسْتَخْدِمُ صَاحِبَهُ مَجَّاناً وَلاَ يُعْطِيهِ أُجْرَتَهُ.
\par 14 الْقَائِلُ: [أَبْنِي لِنَفْسِي بَيْتاً وَسِيعاً وَعَلاَلِيَ فَسِيحَةً] وَيَشُقُّ لِنَفْسِهِ كُوىً وَيَسْقِفُ بِأَرْزٍ وَيَدْهُنُ بِمُغْرَةٍ.
\par 15 هَلْ تَمْلِكُ لأَنَّكَ أَنْتَ تُحَاذِي الأَرْزَ؟ أَمَا أَكَلَ أَبُوكَ وَشَرِبَ وَأَجْرَى حَقّاً وَعَدْلاً؟ حِينَئِذٍ كَانَ لَهُ خَيْرٌ.
\par 16 قَضَى قَضَاءَ الْفَقِيرِ وَالْمِسْكِينِ حِينَئِذٍ كَانَ خَيْرٌ. أَلَيْسَ ذَلِكَ مَعْرِفَتِي يَقُولُ الرَّبُّ؟
\par 17 لأَنَّ عَيْنَيْكَ وَقَلْبَكَ لَيْسَتْ إِلاَّ عَلَى خَطْفِكَ وَعَلَى الدَّمِ الزَّكِيِّ لِتَسْفُكَهُ وَعَلَى الاِغْتِصَابِ وَالظُّلْمِ لِتَعْمَلَهُمَا.
\par 18 لِذَلِكَ هَكَذَا قَالَ الرَّبُّ عَنْ يَهُويَاقِيمَ بْنِ يُوشِيَّا مَلِكِ يَهُوذَا: [ لاَ يَنْدُبُونَهُ قَائِلِينَ: آهِ يَا أَخِي أَوْ آهِ يَا أُخْتِي! لاَ يَنْدُبُونَهُ قَائِلِينَ: آهِ يَا سَيِّدُ أَوْ آهِ يَا جَلاَلَهُ!
\par 19 يُدْفَنُ دَفْنَ حِمَارٍ مَسْحُوباً وَمَطْرُوحاً بَعِيداً عَنْ أَبْوَابِ أُورُشَلِيمَ].
\par 20 اِصْعَدِي عَلَى لُبْنَانَ وَاصْرُخِي وَفِي بَاشَانَ أَطْلِقِي صَوْتَكِ وَاصْرُخِي مِنْ عَبَارِيمَ لأَنَّهُ قَدْ سُحِقَ كُلُّ مُحِبِّيكِ.
\par 21 تَكَلَّمْتُ إِلَيْكِ فِي رَاحَتِكِ. قُلْتِ: [لاَ أَسْمَعُ]. هَذَا طَرِيقُكِ مُنْذُ صِبَاكِ أَنَّكِ لاَ تَسْمَعِينَ لِصَوْتِي.
\par 22 كُلُّ رُعَاتِكِ تَرْعَاهُمُ الرِّيحُ وَمُحِبُّوكِ يَذْهَبُونَ إِلَى السَّبْيِ. فَحِينَئِذٍ تَخْزِينَ وَتَخْجَلِينَ لأَجْلِ كُلِّ شَرِّكِ.
\par 23 أَيَّتُهَا السَّاكِنَةُ فِي لُبْنَانَ الْمُعَشِّشَةُ فِي الأَرْزِ كَمْ يُشْفِقُ عَلَيْكِ عِنْدَ إِتْيَانِ الْمُخَاضِ عَلَيْكِ وَجَعِ كَوَالِدَةٍ!
\par 24 حَيٌّ أَنَا يَقُولُ الرَّبُّ وَلَوْ كَانَ كُنْيَاهُو بْنُ يَهُويَاقِيمَ مَلِكُ يَهُوذَا خَاتِماً عَلَى يَدِي الْيُمْنَى فَإِنِّي مِنْ هُنَاكَ أَنْزِعُكَ
\par 25 وَأُسَلِّمُكَ لِيَدِ طَالِبِي نَفْسِكَ وَلِيَدِ الَّذِينَ تَخَافُ مِنْهُمْ وَلِيَدِ نَبُوخَذْنَصَّرَ مَلِكِ بَابِلَ وَلِيَدِ الْكِلْدَانِيِّينَ.
\par 26 وَأَطْرَحُكَ وَأُمَّكَ الَّتِي وَلَدَتْكَ إِلَى أَرْضٍ أُخْرَى لَمْ تُولَدَا فِيهَا وَهُنَاكَ تَمُوتَانِ.
\par 27 أَمَّا الأَرْضُ الَّتِي يَشْتَاقَانِ إِلَى الرُّجُوعِ إِلَيْهَا فَلاَ يَرْجِعَانِ إِلَيْهَا.
\par 28 هَلْ هَذَا الرَّجُلُ [كُنْيَاهُو] وِعَاءُ خَزَفٍ مُهَانٍ مَكْسُورٍ أَوْ إِنَاءٌ لَيْسَتْ فِيهِ مَسَرَّةٌ؟ لِمَاذَا طُرِحَ هُوَ وَنَسْلُهُ وَأُلْقُوا إِلَى أَرْضٍ لَمْ يَعْرِفُوهَا؟
\par 29 يَا أَرْضُ يَا أَرْضُ يَا أَرْضُ اسْمَعِي كَلِمَةَ الرَّبِّ!
\par 30 هَكَذَا قَالَ الرَّبُّ: [اكْتُبُوا هَذَا الرَّجُلَ عَقِيماً رَجُلاً لاَ يَنْجَحُ فِي أَيَّامِهِ لأَنَّهُ لاَ يَنْجَحُ مِنْ نَسْلِهِ أَحَدٌ جَالِساً عَلَى كُرْسِيِّ دَاوُدَ وَحَاكِماً بَعْدُ فِي يَهُوذَا].

\chapter{23}

\par 1 وَيْلٌ لِلرُّعَاةِ الَّذِينَ يُهْلِكُونَ وَيُبَدِّدُونَ غَنَمَ رَعِيَّتِي يَقُولُ الرَّبُّ.
\par 2 لِذَلِكَ هَكَذَا قَالَ الرَّبُّ إِلَهُ إِسْرَائِيلَ عَنِ الرُّعَاةِ الَّذِينَ يَرْعُونَ شَعْبِي: [أَنْتُمْ بَدَّدْتُمْ غَنَمِي وَطَرَدْتُمُوهَا وَلَمْ تَتَعَهَّدُوهَا. هَئَنَذَا أُعَاقِبُكُمْ عَلَى شَرِّ أَعْمَالِكُمْ يَقُولُ الرَّبُّ.
\par 3 وَأَنَا أَجْمَعُ بَقِيَّةَ غَنَمِي مِنْ جَمِيعِ الأَرَاضِي الَّتِي طَرَدْتُهَا إِلَيْهَا وَأَرُدُّهَا إِلَى مَرَابِضِهَا فَتُثْمِرُ وَتَكْثُرُ.
\par 4 وَأُقِيمُ عَلَيْهَا رُعَاةً يَرْعُونَهَا فَلاَ تَخَافُ بَعْدُ وَلاَ تَرْتَعِدُ وَلاَ تُفْقَدُ يَقُولُ الرَّبُّ].
\par 5 [هَا أَيَّامٌ تَأْتِي يَقُولُ الرَّبُّ وَأُقِيمُ لِدَاوُدَ غُصْنَ بِرٍّ فَيَمْلِكُ مَلِكٌ وَيَنْجَحُ وَيُجْرِي حَقّاً وَعَدْلاً فِي الأَرْضِ.
\par 6 فِي أَيَّامِهِ يُخَلَّصُ يَهُوذَا وَيَسْكُنُ إِسْرَائِيلُ آمِناً وَهَذَا هُوَ اسْمُهُ الَّذِي يَدْعُونَهُ بِهِ: الرَّبُّ بِرُّنَا.
\par 7 لِذَلِكَ هَا أَيَّامٌ تَأْتِي يَقُولُ الرَّبُّ وَلاَ يَقُولُونَ بَعْدُ: حَيٌّ هُوَ الرَّبُّ الَّذِي أَصْعَدَ بَنِي إِسْرَائِيلَ مِنْ أَرْضِ مِصْرَ.
\par 8 بَلْ: حَيٌّ هُوَ الرَّبُّ الَّذِي أَصْعَدَ وَأَتَى بِنَسْلِ بَيْتِ إِسْرَائِيلَ مِنْ أَرْضِ الشِّمَالِ وَمِنْ جَمِيعِ الأَرَاضِي الَّتِي طَرَدْتُهُمْ إِلَيْهَا فَيَسْكُنُونَ فِي أَرْضِهِمْ].
\par 9 فِي الأَنْبِيَاءِ - انْسَحَقَ قَلْبِي فِي وَسَطِي. ارْتَخَتْ كُلُّ عِظَامِي. صِرْتُ كَإِنْسَانٍ سَكْرَانَ وَمِثْلَ رَجُلٍ غَلَبَتْهُ الْخَمْرُ مِنْ أَجْلِ الرَّبِّ وَمِنْ أَجْلِ كَلاَمِ قُدْسِهِ.
\par 10 لأَنَّ الأَرْضَ امْتَلَأَتْ مِنَ الْفَاسِقِينَ. لأَنَّهُ مِنْ أَجْلِ اللَّعْنِ نَاحَتِ الأَرْضُ. جَفَّتْ مَرَاعِي الْبَرِّيَّةِ وَصَارَ سَعْيُهُمْ لِلشَّرِّ وَجَبَرُوتُهُمْ لِلْبَاطِلِ.
\par 11 لأَنَّ الأَنْبِيَاءَ وَالْكَهَنَةَ تَنَجَّسُوا جَمِيعاً بَلْ فِي بَيْتِي وَجَدْتُ شَرَّهُمْ يَقُولُ الرَّبُّ.
\par 12 لِذَلِكَ يَكُونُ طَرِيقُهُمْ لَهُمْ كَمَزَالِقَ فِي ظَلاَمٍ دَامِسٍ فَيُطْرَدُونَ وَيَسْقُطُونَ فِيهَا لأَنِّي أَجْلِبُ عَلَيْهِمْ شَرّاً سَنَةَ عِقَابِهِمْ يَقُولُ الرَّبُّ.
\par 13 وَقَدْ رَأَيْتُ فِي أَنْبِيَاءِ السَّامِرَةِ حَمَاقَةً. تَنَبَّأُوا بِالْبَعْلِ وَأَضَلُّوا شَعْبِي إِسْرَائِيلَ.
\par 14 وَفِي أَنْبِيَاءِ أُورُشَلِيمَ رَأَيْتُ مَا يُقْشَعَرُّ مِنْهُ. يَفْسِقُونَ وَيَسْلُكُونَ بِالْكَذِبِ وَيُشَدِّدُونَ أَيَادِيَ فَاعِلِي الشَّرِّ حَتَّى لاَ يَرْجِعُوا الْوَاحِدُ عَنْ شَرِّهِ. صَارُوا لِي كُلُّهُمْ كَسَدُومَ وَسُكَّانُهَا كَعَمُورَةَ.
\par 15 لِذَلِكَ هَكَذَا قَالَ رَبُّ الْجُنُودِ عَنِ الأَنْبِيَاءِ: هَئَنَذَا أُطْعِمُهُمْ أَفْسَنْتِيناً وَأَسْقِيهِمْ مَاءَ الْعَلْقَمِ لأَنَّهُ مِنْ عِنْدِ أَنْبِيَاءِ أُورُشَلِيمَ خَرَجَ نِفَاقٌ فِي كُلِّ الأَرْضِ.
\par 16 هَكَذَا قَالَ رَبُّ الْجُنُودِ: لاَ تَسْمَعُوا لِكَلاَمِ الأَنْبِيَاءِ الَّذِينَ يَتَنَبَّأُونَ لَكُمْ فَإِنَّهُمْ يَجْعَلُونَكُمْ بَاطِلاً. يَتَكَلَّمُونَ بِرُؤْيَا قَلْبِهِمْ لاَ عَنْ فَمِ الرَّبِّ.
\par 17 قَائِلِينَ قَوْلاً لِمُحْتَقِرِيَّ: [قَالَ الرَّبُّ: يَكُونُ لَكُمْ سَلاَمٌ!] وَيَقُولُونَ لِكُلِّ مَنْ يَسِيرُ فِي عِنَادِ قَلْبِهِ: [لاَ يَأْتِي عَلَيْكُمْ شَرٌّ].
\par 18 لأَنَّهُ مَنْ وَقَفَ فِي مَجْلِسِ الرَّبِّ وَرَأَى وَسَمِعَ كَلِمَتَهُ؟ مَنْ أَصْغَى لِكَلِمَتِهِ وَسَمِعَ؟
\par 19 هَا زَوْبَعَةُ الرَّبِّ. غَيْظٌ يَخْرُجُ وَنَوْءٌ هَائِجٌ. عَلَى رُؤُوسِ الأَشْرَارِ يَثُورُ.
\par 20 لاَ يَرْتَدُّ غَضَبُ الرَّبِّ حَتَّى يُجْرِيَ وَيُقِيمَ مَقَاصِدَ قَلْبِهِ. فِي آخِرِ الأَيَّامِ تَفْهَمُونَ فَهْماً.
\par 21 [لَمْ أُرْسِلِ الأَنْبِيَاءَ بَلْ هُمْ جَرُوا. لَمْ أَتَكَلَّمْ مَعَهُمْ بَلْ هُمْ تَنَبَّأُوا.
\par 22 وَلَوْ وَقَفُوا فِي مَجْلِسِي لَأَخْبَرُوا شَعْبِي بِكَلاَمِي وَرَدُّوهُمْ عَنْ طَرِيقِهِمِ الرَّدِيءِ وَعَنْ شَرِّ أَعْمَالِهِمْ.
\par 23 أَلَعَلِّي إِلَهٌ مِنْ قَرِيبٍ يَقُولُ الرَّبُّ وَلَسْتُ إِلَهاً مِنْ بَعِيدٍ.
\par 24 إِذَا اخْتَبَأَ إِنْسَانٌ فِي أَمَاكِنَ مُسْتَتِرَةٍ أَفَمَا أَرَاهُ أَنَا يَقُولُ الرَّبُّ؟ أَمَا أَمْلَأُ أَنَا السَّمَاوَاتِ وَالأَرْضَ يَقُولُ الرَّبُّ؟
\par 25 قَدْ سَمِعْتُ مَا قَالَهُ الأَنْبِيَاءُ الَّذِينَ تَنَبَّأُوا بِاسْمِي بِالْكَذِبِ قَائِلِينَ: حَلُمْتُ حَلُمْتُ.
\par 26 حَتَّى مَتَى يُوجَدُ فِي قَلْبِ الأَنْبِيَاءِ الْمُتَنَبِّئِينَ بِالْكَذِبِ؟ بَلْ هُمْ أَنْبِيَاءُ خِدَاعِ قَلْبِهِمِ!
\par 27 الَّذِينَ يُفَكِّرُونَ أَنْ يُنَسُّوا شَعْبِي اسْمِي بِأَحْلاَمِهِمِ الَّتِي يَقُصُّونَهَا الرَّجُلُ عَلَى صَاحِبِهِ كَمَا نَسِيَ آبَاؤُهُمُ اسْمِي لأَجْلِ الْبَعْلِ.
\par 28 اَلنَّبِيُّ الَّذِي مَعَهُ حُلْمٌ فَلْيَقُصَّ حُلْماً وَالَّذِي مَعَهُ كَلِمَتِي فَلْيَتَكَلَّمْ بِكَلِمَتِي بِالْحَقِّ. مَا لِلتِّبْنِ مَعَ الْحِنْطَةِ يَقُولُ الرَّبُّ؟
\par 29 أَلَيْسَتْ هَكَذَا كَلِمَتِي كَنَارٍ يَقُولُ الرَّبُّ وَكَمِطْرَقَةٍ تُحَطِّمُ الصَّخْرَ؟
\par 30 لِذَلِكَ هَئَنَذَا عَلَى الأَنْبِيَاءِ يَقُولُ الرَّبُّ الَّذِينَ يَسْرِقُونَ كَلِمَتِي بَعْضُهُمْ مِنْ بَعْضٍ.
\par 31 هَئَنَذَا عَلَى الأَنْبِيَاءِ يَقُولُ الرَّبُّ الَّذِينَ يَأْخُذُونَ لِسَانَهُمْ وَيَقُولُونَ: قَالَ.
\par 32 هَئَنَذَا عَلَى الَّذِينَ يَتَنَبَّأُونَ بِأَحْلاَمٍ كَاذِبَةٍ يَقُولُ الرَّبُّ الَّذِينَ يَقُصُّونَهَا وَيُضِلُّونَ شَعْبِي بِأَكَاذِيبِهِمْ وَمُفَاخَرَاتِهِمْ وَأَنَا لَمْ أُرْسِلْهُمْ وَلاَ أَمَرْتُهُمْ. فَلَمْ يُفِيدُوا هَذَا الشَّعْبَ فَائِدَةً يَقُولُ الرَّبُّ].
\par 33 وَإِذَا سَأَلَكَ هَذَا الشَّعْبُ أَوْ نَبِيٌّ أَوْ كَاهِنٌ: [مَا وَحْيُ الرَّبِّ؟] فَقُلْ لَهُمْ: [أَيُّ وَحْيٍ؟ إِنِّي أَرْفُضُكُمْ - هُوَ قَوْلُ الرَّبِّ.
\par 34 فَالنَّبِيُّ أَوِ الْكَاهِنُ أَوِ الشَّعْبُ الَّذِي يَقُولُ: وَحْيُ الرَّبِّ - أُعَاقِبُ ذَلِكَ الرَّجُلَ وَبَيْتَهُ.
\par 35 هَكَذَا تَقُولُونَ الرَّجُلُ لِصَاحِبِهِ وَالرَّجُلُ لأَخِيهِ: بِمَاذَا أَجَابَ الرَّبُّ وَمَاذَا تَكَلَّمَ بِهِ الرَّبُّ؟
\par 36 أَمَّا وَحْيُ الرَّبِّ فَلاَ تَذْكُرُوهُ بَعْدُ لأَنَّ كَلِمَةَ كُلِّ إِنْسَانٍ تَكُونُ وَحْيَهُ إِذْ قَدْ حَرَّفْتُمْ كَلاَمَ الإِلَهِ الْحَيِّ رَبِّ الْجُنُودِ إِلَهِنَا.
\par 37 هَكَذَا تَقُولُ لِلنَّبِيِّ: بِمَاذَا أَجَابَكَ الرَّبُّ وَمَاذَا تَكَلَّمَ بِهِ الرَّبُّ؟
\par 38 وَإِذَا كُنْتُمْ تَقُولُونَ: وَحْيُ الرَّبِّ - فَلِذَلِكَ هَكَذَا قَالَ الرَّبُّ: مِنْ أَجْلِ قَوْلِكُمْ هَذِهِ الْكَلِمَةَ: وَحْيُ الرَّبِّ وَقَدْ أَرْسَلْتُ إِلَيْكُمْ قَائِلاً لاَ تَقُولُوا: وَحْيُ الرَّبِّ
\par 39 لِذَلِكَ هَئَنَذَا أَنْسَاكُمْ نِسْيَاناً وَأَرْفُضُكُمْ مِنْ أَمَامِ وَجْهِي أَنْتُمْ وَالْمَدِينَةَ الَّتِي أَعْطَيْتُكُمْ وَآبَاءَكُمْ إِيَّاهَا.
\par 40 وَأَجْعَلُ عَلَيْكُمْ عَاراً أَبَدِيّاً وَخِزْياً أَبَدِيّاً لاَ يُنْسَى].

\chapter{24}

\par 1 أَرَانِي الرَّبُّ وَإِذَا سَلَّتَا تِينٍ مَوْضُوعَتَانِ أَمَامَ هَيْكَلِ الرَّبِّ بَعْدَ مَا سَبَى نَبُوخَذْنَصَّرُ مَلِكُ بَابِلَ يَكُنْيَا بْنَ يَهُويَاقِيمَ مَلِكَ يَهُوذَا وَرُؤَسَاءَ يَهُوذَا وَالنَّجَّارِينَ وَالْحَدَّادِينَ مِنْ أُورُشَلِيمَ وَأَتَى بِهِمْ إِلَى بَابِلَ.
\par 2 فِي السَّلَّةِ الْوَاحِدَةِ تِينٌ جَيِّدٌ جِدّاً مِثْلُ التِّينِ الْبَاكُورِيِّ وَفِي السَّلَّةِ الأُخْرَى تِينٌ رَدِيءٌ جِدّاً لاَ يُؤْكَلُ مِنْ رَدَاءَتِهِ.
\par 3 فَقَالَ لِي الرَّبُّ: [مَاذَا أَنْتَ رَاءٍ يَا إِرْمِيَا؟] فَقُلْتُ: [تِيناً. التِّينُ الْجَيِّدُ جَيِّدٌ جِدّاً وَالتِّينُ الرَّدِيءُ رَدِيءٌ جِدّاً لاَ يُؤْكَلُ مِنْ رَدَاءَتِهِ].
\par 4 ثُمَّ صَارَ كَلاَمُ الرَّبِّ إِلَيَّ:
\par 5 [هَكَذَا قَالَ الرَّبُّ إِلَهُ إِسْرَائِيلَ: كَهَذَا التِّينِ الْجَيِّدِ هَكَذَا أَنْظُرُ إِلَى سَبْيِ يَهُوذَا الَّذِي أَرْسَلْتُهُ مِنْ هَذَا الْمَوْضِعِ إِلَى أَرْضِ الْكَلْدَانِيِّينَ لِلْخَيْرِ.
\par 6 وَأَجْعَلُ عَيْنَيَّ عَلَيْهِمْ لِلْخَيْرِ وَأُرْجِعُهُمْ إِلَى هَذِهِ الأَرْضِ وَأَبْنِيهِمْ وَلاَ أَهْدِمُهُمْ وَأَغْرِسُهُمْ وَلاَ أَقْلَعُهُمْ.
\par 7 وَأُعْطِيهِمْ قَلْباً لِيَعْرِفُونِي أَنِّي أَنَا الرَّبُّ فَيَكُونُوا لِي شَعْباً وَأَنَا أَكُونُ لَهُمْ إِلَهاً لأَنَّهُمْ يَرْجِعُونَ إِلَيَّ بِكُلِّ قَلْبِهِمْ.
\par 8 [وَكَالتِّينِ الرَّدِيءِ الَّذِي لاَ يُؤْكَلُ مِنْ رَدَاءَتِهِ. هَكَذَا قَالَ الرَّبُّ. هَكَذَا أَجْعَلُ صِدْقِيَّا مَلِكَ يَهُوذَا وَرُؤَسَاءَهُ وَبَقِيَّةَ أُورُشَلِيمَ الْبَاقِيَةَ فِي هَذِهِ الأَرْضِ وَالسَّاكِنَةَ فِي أَرْضِ مِصْرَ.
\par 9 وَأُسَلِّمُهُمْ لِلْقَلَقِ وَالشَّرِّ فِي جَمِيعِ مَمَالِكِ الأَرْضِ عَاراً وَمَثَلاً وَهُزْأَةً وَلَعْنَةً فِي جَمِيعِ الْمَوَاضِعِ الَّتِي أَطْرُدُهُمْ إِلَيْهَا.
\par 10 وَأُرْسِلُ عَلَيْهِمِ السَّيْفَ وَالْجُوعَ وَالْوَبَأَ حَتَّى يَفْنُوا عَنْ وَجْهِ الأَرْضِ الَّتِي أَعْطَيْتُهُمْ وَآبَاءَهُمْ إِيَّاهَا].

\chapter{25}

\par 1 اَلْكَلاَمُ الَّذِي صَارَ إِلَى إِرْمِيَا عَنْ كُلِّ شَعْبِ يَهُوذَا فِي السَّنَةِ الرَّابِعَةِ لِيَهُويَاقِيمَ بْنِ يُوشِيَّا مَلِكِ يَهُوذَا (هِيَ السَّنَةُ الأُولَى لِنَبُوخَذْنَصَّرَ مَلِكِ بَابِلَ)
\par 2 الَّذِي تَكَلَّمَ بِهِ إِرْمِيَا النَّبِيُّ عَلَى كُلِّ شَعْبِ يَهُوذَا وَعَلَى كُلِّ سُكَّانِ أُورُشَلِيمَ:
\par 3 [مِنَ السَّنَةِ الثَّالِثَةِ عَشَرَةَ لِيُوشِيَّا بْنِ آمُونَ مَلِكِ يَهُوذَا إِلَى هَذَا الْيَوْمِ هَذِهِ الثَّلاَثِ وَالْعِشْرِينَ سَنَةً صَارَتْ كَلِمَةُ الرَّبِّ إِلَيَّ فَكَلَّمْتُكُمْ مُبَكِّراً وَمُكَلِّماً فَلَمْ تَسْمَعُوا.
\par 4 وَقَدْ أَرْسَلَ الرَّبُّ إِلَيْكُمْ كُلَّ عَبِيدِهِ الأَنْبِيَاءِ مُبَكِّراً وَمُرْسِلاً فَلَمْ تَسْمَعُوا وَلَمْ تَمِيلُوا أُذُنَكُمْ لِلسَّمْعِ
\par 5 قَائِلِينَ: ارْجِعُوا كُلُّ وَاحِدٍ عَنْ طَرِيقِهِ الرَّدِيءِ وَعَنْ شَرِّ أَعْمَالِكُمْ وَاسْكُنُوا فِي الأَرْضِ الَّتِي أَعْطَاكُمُ الرَّبُّ إِيَّاهَا وَآبَاءَكُمْ مِنَ الأَزَلِ وَإِلَى الأَبَدِ.
\par 6 وَلاَ تَسْلُكُوا وَرَاءَ آلِهَةٍ أُخْرَى لِتَعْبُدُوهَا وَتَسْجُدُوا لَهَا وَلاَ تَغِيظُونِي بِعَمَلِ أَيْدِيكُمْ فَلاَ أُسِيءَ إِلَيْكُمْ.
\par 7 فَلَمْ تَسْمَعُوا لِي يَقُولُ الرَّبُّ لِتَغِيظُونِي بِعَمَلِ أَيْدِيكُمْ شَرّاً لَكُمْ.
\par 8 [لِذَلِكَ هَكَذَا قَالَ رَبُّ الْجُنُودِ: مِنْ أَجْلِ أَنَّكُمْ لَمْ تَسْمَعُوا لِكَلاَمِي
\par 9 هَئَنَذَا أُرْسِلُ فَآخُذُ كُلَّ عَشَائِرِ الشِّمَالِ يَقُولُ الرَّبُّ وَإِلَى نَبُوخَذْنَصَّرَ عَبْدِي مَلِكِ بَابِلَ وَآتِي بِهِمْ عَلَى هَذِهِ الأَرْضِ وَعَلَى كُلِّ سُكَّانِهَا وَعَلَى كُلِّ هَذِهِ الشُّعُوبِ حَوَالَيْهَا فَأُحَرِّمُهُمْ وَأَجْعَلُهُمْ دَهَشاً وَصَفِيراً وَخِرَباً أَبَدِيَّةً.
\par 10 وَأُبِيدُ مِنْهُمْ صَوْتَ الطَّرَبِ وَصَوْتَ الْفَرَحِ صَوْتَ الْعَرِيسِ وَصَوْتَ الْعَرُوسِ صَوْتَ الأَرْحِيَةِ وَنُورَ السِّرَاجِ.
\par 11 وَتَصِيرُ كُلُّ هَذِهِ الأَرْضِ خَرَاباً وَدَهَشاً وَتَخْدِمُ هَذِهِ الشُّعُوبُ مَلِكَ بَابِلَ سَبْعِينَ سَنَةً.
\par 12 [وَيَكُونُ عِنْدَ تَمَامِ السَّبْعِينَ سَنَةً أَنِّي أُعَاقِبُ مَلِكَ بَابِلَ وَتِلْكَ الأُمَّةَ يَقُولُ الرَّبُّ عَلَى إِثْمِهِمْ وَأَرْضَ الْكِلْدَانِيِّينَ وَأَجْعَلُهَا خِرَباً أَبَدِيَّةً.
\par 13 وَأَجْلِبُ عَلَى تِلْكَ الأَرْضِ كُلَّ كَلاَمِي الَّذِي تَكَلَّمْتُ بِهِ عَلَيْهَا كُلَّ مَا كُتِبَ فِي هَذَا السِّفْرِ الَّذِي تَنَبَّأَ بِهِ إِرْمِيَا عَلَى كُلِّ الشُّعُوبِ.
\par 14 لأَنَّهُ قَدِ اسْتَعْبَدَهُمْ أَيْضاً أُمَمٌ كَثِيرَةٌ وَمُلُوكٌ عِظَامٌ فَأُجَازِيهِمْ حَسَبَ أَعْمَالِهِمْ وَحَسَبَ عَمَلِ أَيَادِيهِمْ].
\par 15 لأَنَّهُ هَكَذَا قَالَ لِي الرَّبُّ إِلَهُ إِسْرَائِيلَ: [خُذْ كَأْسَ خَمْرِ هَذَا السَّخَطِ مِنْ يَدِي وَاسْقِ جَمِيعَ الشُّعُوبِ الَّذِينَ أُرْسِلُكَ أَنَا إِلَيْهِمْ إِيَّاهَا.
\par 16 فَيَشْرَبُوا وَيَتَرَنَّحُوا وَيَتَجَنَّنُوا مِنْ أَجْلِ السَّيْفِ الَّذِي أُرْسِلُهُ أَنَا بَيْنَهُمْ].
\par 17 فَأَخَذْتُ الْكَأْسَ مِنْ يَدِ الرَّبِّ وَسَقَيْتُ كُلَّ الشُّعُوبِ الَّذِينَ أَرْسَلَنِي الرَّبُّ إِلَيْهِمْ.
\par 18 أُورُشَلِيمَ وَمُدُنَ يَهُوذَا وَمُلُوكَهَا وَرُؤَسَاءَهَا لِجَعْلِهَا خَرَاباً وَدَهَشاً وَصَفِيراً وَلَعْنَةً كَهَذَا الْيَوْمِ.
\par 19 وَفِرْعَوْنَ مَلِكَ مِصْرَ وَعَبِيدَهُ وَرُؤَسَاءَهُ وَكُلَّ شَعْبِهِ.
\par 20 وَكُلَّ اللَّفِيفِ وَكُلَّ مُلُوكِ أَرْضِ عُوصَ وَكُلَّ مُلُوكِ أَرْضِ فِلِسْطِينَ وَأَشْقَلُونَ وَغَزَّةَ وَعَقْرُونَ وَبَقِيَّةَ أَشْدُودَ
\par 21 وَأَدُومَ وَمُوآبَ وَبَنِي عَمُّونَ
\par 22 وَكُلَّ مُلُوكِ صُورَ وَكُلَّ مُلُوكِ صَيْدُونَ وَمُلُوكِ الْجَزَائِرِ الَّتِي فِي عَبْرِ الْبَحْرِ
\par 23 وَدَدَانَ وَتَيْمَاءَ وَبُوزَ وَكُلَّ مَقْصُوصِي الشَّعْرِ مُسْتَدِيراً
\par 24 وَكُلَّ مُلُوكِ الْعَرَبِ وَكُلَّ مُلُوكِ اللَّفِيفِ السَّاكِنِينَ فِي الْبَرِّيَّةِ
\par 25 وَكُلَّ مُلُوكِ زِمْرِي وَكُلَّ مُلُوكِ عِيلاَمَ وَكُلَّ مُلُوكِ مَادِي
\par 26 وَكُلَّ مُلُوكِ الشِّمَالِ الْقَرِيبِينَ وَالْبَعِيدِينَ كُلَّ وَاحِدٍ مَعَ أَخِيهِ وَكُلَّ مَمَالِكِ الأَرْضِ الَّتِي عَلَى وَجْهِ الأَرْضِ. وَمَلِكُ شِيشَكَ يَشْرَبُ بَعْدَهُمْ.
\par 27 وَتَقُولُ لَهُمْ: [هَكَذَا قَالَ رَبُّ الْجُنُودِ إِلَهُ إِسْرَائِيلَ: اشْرَبُوا وَاسْكَرُوا وَتَقَيَّأُوا وَاسْقُطُوا وَلاَ تَقُومُوا مِنْ أَجْلِ السَّيْفِ الَّذِي أُرْسِلُهُ أَنَا بَيْنَكُمْ.
\par 28 وَيَكُونُ إِذَا أَبُوا أَنْ يَأْخُذُوا الْكَأْسَ مِنْ يَدِكَ لِيَشْرَبُوا أَنَّكَ تَقُولُ لَهُمْ: هَكَذَا قَالَ رَبُّ الْجُنُودِ: تَشْرَبُونَ شُرْباً.
\par 29 لأَنِّي هَئَنَذَا أَبْتَدِئُ أُسِيءُ إِلَى الْمَدِينَةِ الَّتِي دُعِيَ اسْمِي عَلَيْهَا فَهَلْ تَتَبَرَّأُونَ أَنْتُمْ؟ لاَ تَتَبَرَّأُونَ لأَنِّي أَنَا أَدْعُو السَّيْفَ عَلَى كُلِّ سُكَّانِ الأَرْضِ يَقُولُ رَبُّ الْجُنُودِ.
\par 30 وَأَنْتَ فَتَنَبَّأْ عَلَيْهِمْ بِكُلِّ هَذَا الْكَلاَمِ وَقُلْ لَهُمْ: الرَّبُّ مِنَ الْعَلاَءِ يُزَمْجِرُ وَمِنْ مَسْكَنِ قُدْسِهِ يُطْلِقُ صَوْتَهُ يَزْأَرُ زَئِيراً عَلَى مَسْكَنِهِ بِهُتَافٍ كَالدَّائِسِينَ يَصْرُخُ ضِدَّ كُلِّ سُكَّانِ الأَرْضِ.
\par 31 بَلَغَ الضَّجِيجُ إِلَى أَطْرَافِ الأَرْضِ لأَنَّ لِلرَّبِّ خُصُومَةً مَعَ الشُّعُوبِ. هُوَ يُحَاكِمُ كُلَّ ذِي جَسَدٍ. يَدْفَعُ الأَشْرَارَ لِلسَّيْفِ يَقُولُ الرَّبُّ.
\par 32 هَكَذَا قَالَ رَبُّ الْجُنُودِ: هُوَذَا الشَّرُّ يَخْرُجُ مِنْ أُمَّةٍ إِلَى أُمَّةٍ وَيَنْهَضُ نَوْءٌ عَظِيمٌ مِنْ أَطْرَافِ الأَرْضِ.
\par 33 وَتَكُونُ قَتْلَى الرَّبِّ فِي ذَلِكَ الْيَوْمِ مِنْ أَقْصَاءِ الأَرْضِ إِلَى أَقْصَاءِ الأَرْضِ. لاَ يُنْدَبُونَ وَلاَ يُضَمُّونَ وَلاَ يُدْفَنُونَ. يَكُونُونَ دِمْنَةً عَلَى وَجْهِ الأَرْضِ].
\par 34 وَلْوِلُوا أَيُّهَا الرُّعَاةُ وَاصْرُخُوا وَتَمَرَّغُوا يَا رُؤَسَاءَ الْغَنَمِ لأَنَّ أَيَّامَكُمْ قَدْ كَمَلَتْ لِلذَّبْحِ. وَأُبَدِّدُكُمْ فَتَسْقُطُونَ كَإِنَاءٍ شَهِيٍّ.
\par 35 وَيَبِيدُ الْمَنَاصُ عَنِ الرُّعَاةِ وَالنَّجَاةُ عَنْ رُؤَسَاءِ الْغَنَمِ.
\par 36 صَوْتُ صُرَاخِ الرُّعَاةِ وَوَلْوَلَةِ رُؤَسَاءِ الْغَنَمِ. لأَنَّ الرَّبَّ قَدْ أَهْلَكَ مَرْعَاهُمْ.
\par 37 وَبَادَتْ مَرَاعِي السَّلاَمِ مِنْ أَجْلِ حُمُوِّ غَضَبِ الرَّبِّ.
\par 38 تَرَكَ كَشِبْلٍ عِيصَهُ لأَنَّ أَرْضَهُمْ صَارَتْ خَرَاباً مِنْ أَجْلِ الظَّالِمِ وَمِنْ أَجْلِ حُمُوِّ غَضَبِهِ.

\chapter{26}

\par 1 فِي ابْتِدَاءِ مُلْكِ يَهُويَاقِيمَ بْنِ يُوشِيَّا مَلِكِ يَهُوذَا صَارَ هَذَا الْكَلاَمُ مِنْ الرَّبِّ:
\par 2 [هَكَذَا قَالَ الرَّبُّ: قِفْ فِي دَارِ بَيْتِ الرَّبِّ وَتَكَلَّمْ عَلَى كُلِّ مُدُنِ يَهُوذَا الْقَادِمَةِ لِلسُّجُودِ فِي بَيْتِ الرَّبِّ بِكُلِّ الْكَلاَمِ الَّذِي أَوْصَيْتُكَ أَنْ تَتَكَلَّمَ بِهِ إِلَيْهِمْ. لاَ تُنَقِّصْ كَلِمَةً.
\par 3 لَعَلَّهُمْ يَسْمَعُونَ وَيَرْجِعُونَ كُلُّ وَاحِدٍ عَنْ طَرِيقِهِ الشِّرِّيرِ فَأَنْدَمَ عَنِ الشَّرِّ الَّذِي قَصَدْتُ أَنْ أَصْنَعَهُ بِهِمْ مِنْ أَجْلِ شَرِّ أَعْمَالِهِمْ.
\par 4 وَتَقُولُ لَهُمْ هَكَذَا قَالَ الرَّبُّ: إِنْ لَمْ تَسْمَعُوا لِي لِتَسْلُكُوا فِي شَرِيعَتِي الَّتِي جَعَلْتُهَا أَمَامَكُمْ
\par 5 لِتَسْمَعُوا لِكَلاَمِ عَبِيدِي الأَنْبِيَاءِ الَّذِينَ أَرْسَلْتُهُمْ أَنَا إِلَيْكُمْ مُبَكِّراً وَمُرْسِلاً إِيَّاهُمْ فَلَمْ تَسْمَعُوا.
\par 6 أَجْعَلُ هَذَا الْبَيْتَ كَشِيلُوهَ وَهَذِهِ الْمَدِينَةُ أَجْعَلُهَا لَعْنَةً لِكُلِّ شُعُوبِ الأَرْضِ].
\par 7 وَسَمِعَ الْكَهَنَةُ وَالأَنْبِيَاءُ وَكُلُّ الشَّعْبِ إِرْمِيَا يَتَكَلَّمُ بِهَذَا الْكَلاَمِ فِي بَيْتِ الرَّبِّ.
\par 8 وَكَانَ لَمَّا فَرَغَ إِرْمِيَا مِنَ التَّكَلُّمِ بِكُلِّ مَا أَوْصَاهُ الرَّبُّ أَنْ يُكَلِّمَ كُلَّ الشَّعْبِ بِهِ أَنَّ الْكَهَنَةَ وَالأَنْبِيَاءَ وَكُلَّ الشَّعْبِ أَمْسَكُوهُ قَائِلِينَ: [تَمُوتُ مَوْتاً!
\par 9 لِمَاذَا تَنَبَّأْتَ بِاسْمِ الرَّبِّ قَائِلاً مِثْلَ شِيلُوهَ يَكُونُ هَذَا الْبَيْتُ وَهَذِهِ الْمَدِينَةُ تَكُونُ خَرِبَةً بِلاَ سَاكِنٍ؟] وَاجْتَمَعَ كُلُّ الشَّعْبِ عَلَى إِرْمِيَا فِي بَيْتِ الرَّبِّ.
\par 10 فَلَمَّا سَمِعَ رُؤَسَاءُ يَهُوذَا بِهَذِهِ الأُمُورِ صَعِدُوا مِنْ بَيْتِ الْمَلِكِ إِلَى بَيْتِ الرَّبِّ وَجَلَسُوا فِي مَدْخَلِ بَابِ الرَّبِّ الْجَدِيدِ.
\par 11 فَتَكَلَّمَ الْكَهَنَةُ وَالأَنْبِيَاءُ مَعَ الرُّؤَسَاءِ وَكُلِّ الشَّعْبِ قَائِلِينَ: [حَقُّ الْمَوْتِ عَلَى هَذَا الرَّجُلِ لأَنَّهُ قَدْ تَنَبَّأَ عَلَى هَذِهِ الْمَدِينَةِ كَمَا سَمِعْتُمْ بِآذَانِكُمْ].
\par 12 فَكَلَّمَ إِرْمِيَا كُلَّ الرُّؤَسَاءِ وَكُلَّ الشَّعْبِ قَائِلاً: [الرَّبُّ أَرْسَلَنِي لأَتَنَبَّأَ عَلَى هَذَا الْبَيْتِ وَعَلَى هَذِهِ الْمَدِينَةِ بِكُلِّ الْكَلاَمِ الَّذِي سَمِعْتُمُوهُ.
\par 13 فَالآنَ أَصْلِحُوا طُرُقَكُمْ وَأَعْمَالَكُمْ وَاسْمَعُوا لِصَوْتِ الرَّبِّ إِلَهِكُمْ فَيَنْدَمَ الرَّبُّ عَنِ الشَّرِّ الَّذِي تَكَلَّمَ بِهِ عَلَيْكُمْ.
\par 14 أَمَّا أَنَا فَهَئَنَذَا بِيَدِكُمُ. اصْنَعُوا بِي كَمَا هُوَ حَسَنٌ وَمُسْتَقِيمٌ فِي أَعْيُنِكُمْ.
\par 15 لَكِنِ اعْلَمُوا عِلْماً أَنَّكُمْ إِنْ قَتَلْتُمُونِي تَجْعَلُونَ دَماً زَكِيّاً عَلَى أَنْفُسِكُمْ وَعَلَى هَذِهِ الْمَدِينَةِ وَعَلَى سُكَّانِهَا لأَنَّهُ حَقّاً قَدْ أَرْسَلَنِي الرَّبُّ إِلَيْكُمْ لأَتَكَلَّمَ فِي آذَانِكُمْ بِكُلِّ هَذَا الْكَلاَمِ].
\par 16 فَقَالَتِ الرُّؤَسَاءُ وَكُلُّ الشَّعْبِ لِلْكَهَنَةِ وَالأَنْبِيَاءِ: [لَيْسَ عَلَى هَذَا الرَّجُلِ حَقُّ الْمَوْتِ لأَنَّهُ إِنَّمَا كَلَّمَنَا بِاسْمِ الرَّبِّ إِلَهِنَا].
\par 17 فَقَامَ أُنَاسٌ مِنْ شُيُوخِ الأَرْضِ وَقَالُوا لِكُلِّ جَمَاعَةِ الشَّعْبِ:
\par 18 [إِنَّ مِيخَا الْمُورَشْتِيَّ تَنَبَّأَ فِي أَيَّامِ حَزَقِيَّا مَلِكِ يَهُوذَا وَقَالَ لِكُلِّ شَعْبِ يَهُوذَا: هَكَذَا قَالَ رَبُّ الْجُنُودِ: إِنَّ صِهْيَوْنَ تُفْلَحُ كَحَقْلٍ وَتَصِيرُ أُورُشَلِيمُ خِرَباً وَجَبَلُ الْبَيْتِ شَوَامِخَ وَعْرٍ.
\par 19 هَلْ قَتْلاً قَتَلَهُ حَزَقِيَّا مَلِكُ يَهُوذَا وَكُلُّ يَهُوذَا؟ أَلَمْ يَخَفِ الرَّبَّ وَطَلَبَ وَجْهَ الرَّبِّ فَنَدِمَ الرَّبُّ عَنِ الشَّرِّ الَّذِي تَكَلَّمَ بِهِ عَلَيْهِمْ؟ فَنَحْنُ عَامِلُونَ شَرّاً عَظِيماً ضِدَّ أَنْفُسِنَا].
\par 20 وَقَدْ كَانَ رَجُلٌ أَيْضاً يَتَنَبَّأُ بِاسْمِ الرَّبِّ أُورِيَّا بْنُ شِمْعِيَا مِن قَرْيَةِ يَعَارِيمَ فَتَنَبَّأَ عَلَى هَذِهِ الْمَدِينَةِ وَعَلَى هَذِهِ الأَرْضِ بِكُلِّ كَلاَمِ إِرْمِيَا.
\par 21 وَلَمَّا سَمِعَ الْمَلِكُ يَهُويَاقِيمُ وَكُلُّ أَبْطَالِهِ وَكُلُّ الرُّؤَسَاءِ كَلاَمَهُ طَلَبَ الْمَلِكُ أَنْ يَقْتُلَهُ. فَلَمَّا سَمِعَ أُورِيَّا خَافَ وَهَرَبَ وَأَتَى إِلَى مِصْرَ.
\par 22 فَأَرْسَلَ الْمَلِكُ يَهُويَاقِيمُ أُنَاساً إِلَى مِصْرَ أَلْنَاثَانَ بْنَ عَكْبُورَ وَرِجَالاً مَعَهُ إِلَى مِصْرَ
\par 23 فَأَخْرَجُوا أُورِيَّا مِنْ مِصْرَ وَأَتُوا بِهِ إِلَى الْمَلِكِ يَهُويَاقِيمَ فَضَرَبَهُ بِالسَّيْفِ وَطَرَحَ جُثَّتَهُ فِي قُبُورِ بَنِي الشَّعْبِ.
\par 24 وَلَكِنَّ يَدَ أَخِيقَامَ بْنِ شَافَانَ كَانَتْ مَعَ إِرْمِيَا حَتَّى لاَ يُدْفَعَ لِيَدِ الشَّعْبِ لِيَقْتُلُوهُ.

\chapter{27}

\par 1 فِي ابْتِدَاءِ مُلْكِ يَهُويَاقِيمَ بْنِ يُوشِيَّا مَلِكِ يَهُوذَا صَارَ هَذَا الْكَلاَمُ إِلَى إِرْمِيَا مِنْ الرَّبِّ:
\par 2 [هَكَذَا قَالَ الرَّبُّ لِي. اصْنَعْ لِنَفْسِكَ رُبُطاً وَأَنْيَاراً وَاجْعَلْهَا عَلَى عُنْقِكَ
\par 3 وَأَرْسِلْهَا إِلَى مَلِكِ أَدُومَ وَإِلَى مَلِكِ مُوآبَ وَإِلَى مَلِكِ بَنِي عَمُّونَ وَإِلَى مَلِكِ صُورَ وَإِلَى مَلِكِ صَيْدُونَ بِيَدِ الرُّسُلِ الْقَادِمِينَ إِلَى أُورُشَلِيمَ إِلَى صِدْقِيَّا مَلِكِ يَهُوذَا.
\par 4 وَأَوْصِهِمْ إِلَى سَادَتِهِمْ قَائِلاً: هَكَذَا قَالَ رَبُّ الْجُنُودِ إِلَهُ إِسْرَائِيلَ: هَكَذَا تَقُولُونَ لِسَادَتِكُمْ:
\par 5 إِنِّي أَنَا صَنَعْتُ الأَرْضَ وَالإِنْسَانَ وَالْحَيَوَانَ الَّذِي عَلَى وَجْهِ الأَرْضِ بِقُوَّتِي الْعَظِيمَةِ وَبِذِرَاعِي الْمَمْدُودَةِ وَأَعْطَيْتُهَا لِمَنْ حَسُنَ فِي عَيْنَيَّ.
\par 6 وَالآنَ قَدْ دَفَعْتُ كُلَّ هَذِهِ الأَرَاضِي لِيَدِ نَبُوخَذْنَصَّرَ مَلِكِ بَابِلَ عَبْدِي وَأَعْطَيْتُهُ أَيْضاً حَيَوَانَ الْحَقْلِ لِيَخْدِمَهُ.
\par 7 فَتَخْدِمُهُ كُلُّ الشُّعُوبِ وَابْنَهُ وَابْنَ ابْنِهِ حَتَّى يَأْتِيَ وَقْتُ أَرْضِهِ أَيْضاً فَتَسْتَخْدِمُهُ شُعُوبٌ كَثِيرَةٌ وَمُلُوكٌ عِظَامٌ.
\par 8 وَيَكُونُ أَنَّ الأُمَّةَ أَوِ الْمَمْلَكَةَ الَّتِي لاَ تَخْدِمُ نَبُوخَذْنَصَّرَ مَلِكَ بَابَِلَ وَالَّتِي لاَ تَجْعَلُ عُنُقَهَا تَحْتَ نِيرِ مَلِكِ بَابَِلَ إِنِّي أُعَاقِبُ تِلْكَ الأُمَّةَ بِالسَّيْفِ وَالْجُوعِ وَالْوَبَإِ يَقُولُ الرَّبُّ حَتَّى أُفْنِيَهَا بِيَدِهِ.
\par 9 فَلاَ تَسْمَعُوا أَنْتُمْ لأَنْبِيَائِكُمْ وَعَرَّافِيكُمْ وَحَالِمِيكُمْ وَعَائِفِيكُمْ وَسَحَرَتِكُمُ الَّذِينَ يَقُولُونَ لَكُمْ: [لاَ تَخْدِمُوا مَلِكَ بَابَِلَ.
\par 10 لأَنَّهُمْ إِنَّمَا يَتَنَبَّأُونَ لَكُمْ بِالْكَذِبِ لِيُبْعِدُوكُمْ مِنْ أَرْضِكُمْ وَلأَطْرُدَكُمْ فَتَهْلِكُوا.
\par 11 وَالأُمَّةُ الَّتِي تُدْخِلُ عُنُقَهَا تَحْتَ نِيرِ مَلِكِ بَابِلَ وَتَخْدِمُهُ أَجْعَلُهَا تَسْتَقِرُّ فِي أَرْضِهَا يَقُولُ الرَّبُّ وَتَعْمَلُهَا وَتَسْكُنُ بِهَا].
\par 12 وَكَلَّمْتُ صِدْقِيَّا مَلِكَ يَهُوذَا بِكُلِّ هَذَا الْكَلاَمِ قَائِلاً: [أَدْخِلُوا أَعْنَاقَكُمْ تَحْتَ نِيرِ مَلِكِ بَابِلَ وَاخْدِمُوهُ وَشَعْبَهُ وَاحْيُوا.
\par 13 لِمَاذَا تَمُوتُونَ أَنْتَ وَشَعْبُكَ بِالسَّيْفِ بِالْجُوعِ وَالْوَبَإِ كَمَا تَكَلَّمَ الرَّبُّ عَنِ الأُمَّةِ الَّتِي لاَ تَخْدِمُ مَلِكَ بَابَِلَ؟
\par 14 فَلاَ تَسْمَعُوا لِكَلاَمِ الأَنْبِيَاءِ الَّذِينَ يَقُولُونَ لَكُمْ: لاَ تَخْدِمُوا مَلِكَ بَابِلَ لأَنَّهُمْ إِنَّمَا يَتَنَبَّأُونَ لَكُمْ بِالْكَذِبِ.
\par 15 لأَنِّي لَمْ أُرْسِلْهُمْ يَقُولُ الرَّبُّ بَلْ هُمْ يَتَنَبَّأُونَ بِاسْمِي بِالْكَذِبِ لأَطْرُدَكُمْ فَتَهْلِكُوا أَنْتُمْ وَالأَنْبِيَاءُ الَّذِينَ يَتَنَبَّأُونَ لَكُمْ].
\par 16 وَكَلَّمْتُ الْكَهَنَةَ وَكُلَّ هَذَا الشَّعْبِ: [هَكَذَا قَالَ الرَّبُّ: لاَ تَسْمَعُوا لِكَلاَمِ أَنْبِيَائِكُمُ الَّذِينَ يَتَنَبَّأُونَ لَكُمْ قَائِلِينَ: هَا آنِيَةُ بَيْتِ الرَّبِّ سَتُرَدُّ سَرِيعاً مِنْ بَابِلَ. لأَنَّهُمْ إِنَّمَا يَتَنَبَّأُونَ لَكُمْ بِالْكَذِبِ.
\par 17 لاَ تَسْمَعُوا لَهُمْ. اخْدِمُوا مَلِكَ بَابِلَ وَاحْيُوا. لِمَاذَا تَصِيرُ هَذِهِ الْمَدِينَةُ خَرِبَةً؟
\par 18 فَإِنْ كَانُوا أَنْبِيَاءَ وَإِنْ كَانَتْ كَلِمَةُ الرَّبِّ مَعَهُمْ فَلْيَتَوَسَّلُوا إِلَى رَبِّ الْجُنُودِ لِكَيْ لاَ تَذْهَبَ إِلَى بَابِلَ الآنِيَةُ الْبَاقِيَةُ فِي بَيْتِ الرَّبِّ وَبَيْتِ مَلِكِ يَهُوذَا وَفِي أُورُشَلِيمَ.
\par 19 [لأَنَّهُ هَكَذَا قَالَ رَبُّ الْجُنُودِ عَنِ الأَعْمِدَةِ وَعَنِ الْبَحْرِ وَعَنِ الْقَوَاعِدِ وَعَنْ سَائِرِ الآنِيَةِ الْبَاقِيَةِ فِي هَذِهِ الْمَدِينَةِ
\par 20 الَّتِي لَمْ يَأْخُذْهَا نَبُوخَذْنَصَّرُ مَلِكُ بَابِلَ عِنْدَ سَبْيِهِ يَكُنْيَا بْنَ يَهُويَاقِيمَ مَلِكَ يَهُوذَا مِنْ أُورُشَلِيمَ إِلَى بَابِلَ وَكُلَّ أَشْرَافِ يَهُوذَا وَأُورُشَلِيمَ.
\par 21 إِنَّهُ هَكَذَا قَالَ رَبُّ الْجُنُودِ إِلَهُ إِسْرَائِيلَ عَنِ الآنِيَةِ الْبَاقِيَةِ فِي بَيْتِ الرَّبِّ وَبَيْتِ مَلِكِ يَهُوذَا وَفِي أُورُشَلِيمَ:
\par 22 يُؤْتَى بِهَا إِلَى بَابِلَ وَتَكُونُ هُنَاكَ إِلَى يَوْمِ افْتِقَادِي إِيَّاهَا يَقُولُ الرَّبُّ فَأُصْعِدُهَا وَأَرُدُّهَا إِلَى هَذَا الْمَوْضِعِ].

\chapter{28}

\par 1 وَحَدَثَ فِي تِلْكَ السَّنَةِ فِي ابْتِدَاءِ مُلْكِ صِدْقِيَّا مَلِكِ يَهُوذَا فِي السَّنَةِ الرَّابِعَةِ فِي الشَّهْرِ الْخَامِسِ أَنَّ حَنَنِيَّا بْنَ عَزُورَ النَّبِيَّ الَّذِي مِنْ جِبْعُونَ قَالَ لِي فِي بَيْتِ الرَّبِّ أَمَامَ الْكَهَنَةِ وَكُلِّ الشَّعْبِ:
\par 2 [هَكَذَا قَالَ رَبُّ الْجُنُودِ إِلَهُ إِسْرَائِيلَ: قَدْ كَسَرْتُ نِيرَ مَلِكِ بَابِلَ.
\par 3 فِي سَنَتَيْنِ مِنَ الزَّمَانِ أَرُدُّ إِلَى هَذَا الْمَوْضِعِ كُلَّ آنِيَةِ بَيْتِ الرَّبِّ الَّتِي أَخَذَهَا نَبُوخَذْنَصَّرُ مَلِكُ بَابِلَ مِنْ هَذَا الْمَوْضِعِ وَذَهَبَ بِهَا إِلَى بَابِلَ.
\par 4 وَأَرُدُّ إِلَى هَذَا الْمَوْضِعِ يَكُنْيَا بْنَ يَهُويَاقِيمَ مَلِكَ يَهُوذَا وَكُلَّ سَبْيِ يَهُوذَا الَّذِينَ ذَهَبُوا إِلَى بَابِلَ يَقُولُ الرَّبُّ لأَنِّي أَكْسِرُ نِيرَ مَلِكِ بَابِلَ].
\par 5 فَكَلَّمَ إِرْمِيَا النَّبِيُّ حَنَنِيَّا النَّبِيَّ أَمَامَ الْكَهَنَةِ وَأَمَامَ كُلِّ الشَّعْبِ الْوَاقِفِينَ فِي بَيْتِ الرَّبِّ
\par 6 وَقَالَ: [آمِينَ. هَكَذَا لِيَصْنَعِ الرَّبُّ. لِيُقِمِ الرَّبُّ كَلاَمَكَ الَّذِي تَنَبَّأْتَ بِهِ فَيَرُدَّ آنِيَةَ بَيْتِ الرَّبِّ وَكُلَّ السَّبْيِ مِنْ بَابِلَ إِلَى هَذَا الْمَوْضِعِ.
\par 7 وَلَكِنِ اسْمَعْ هَذِهِ الْكَلِمَةَ الَّتِي أَتَكَلَّمُ أَنَا بِهَا فِي أُذُنَيْكَ وَفِي آذَانِ كُلِّ الشَّعْبِ.
\par 8 إِنَّ الأَنْبِيَاءَ الَّذِينَ كَانُوا قَبْلِي وَقَبْلَكَ مُنْذُ الْقَدِيمِ وَتَنَبَّأُوا عَلَى أَرَاضٍ كَثِيرَةٍ وَعَلَى مَمَالِكَ عَظِيمَةٍ بِالْحَرْبِ وَالشَّرِّ وَالْوَبَإِ.
\par 9 النَّبِيُّ الَّذِي تَنَبَّأَ بِالسَّلاَمِ فَعِنْدَ حُصُولِ كَلِمَةِ النَّبِيِّ عُرِفَ ذَلِكَ النَّبِيُّ أَنَّ الرَّبَّ قَدْ أَرْسَلَهُ حَقّاً].
\par 10 ثُمَّ أَخَذَ حَنَنِيَّا النَّبِيُّ النِّيرَ عَنْ عُنُقِ إِرْمِيَا النَّبِيِّ وَكَسَرَهُ.
\par 11 وَقَالَ حَنَنِيَّا أَمَامَ كُلِّ الشَّعْبِ: [هَكَذَا قَالَ الرَّبُّ: هَكَذَا أَكْسِرُ نِيرَ نَبُوخَذْنَصَّرَ مَلِكِ بَابِلَ فِي سَنَتَيْنِ مِنَ الزَّمَانِ عَنْ عُنُقِ كُلِّ الشُّعُوبِ]. وَانْطَلَقَ إِرْمِيَا النَّبِيُّ فِي سَبِيلِهِ.
\par 12 ثُمَّ صَارَ كَلاَمُ الرَّبِّ إِلَى إِرْمِيَا النَّبِيِّ بَعْدَ مَا كَسَرَ حَنَنِيَّا النَّبِيُّ النِّيرَ عَنْ عُنُقِ إِرْمِيَا النَّبِيِّ:
\par 13 [اذْهَبْ وَقُلْ لِحَنَنِيَّا: هَكَذَا قَالَ الرَّبُّ: قَدْ كَسَرْتَ أَنْيَارَ الْخَشَبِ وَعَمِلْتَ عِوَضاً عَنْهَا أَنْيَاراً مِنْ حَدِيدٍ.
\par 14 لأَنَّهُ هَكَذَا قَالَ رَبُّ الْجُنُودِ إِلَهُ إِسْرَائِيلَ: قَدْ جَعَلْتُ نِيراً مِنْ حَدِيدٍ عَلَى عُنُقِ كُلِّ هَؤُلاَءِ الشُّعُوبِ لِيَخْدِمُوا نَبُوخَذْنَصَّرَ مَلِكَ بَابِلَ فَيَخْدِمُونَهُ وَقَدْ أَعْطَيْتُهُ أَيْضاً حَيَوَانَ الْحَقْلِ].
\par 15 فَقَالَ إِرْمِيَا النَّبِيُّ لِحَنَنِيَّا النَّبِيِّ: [اسْمَعْ يَا حَنَنِيَّا. إِنَّ الرَّبَّ لَمْ يُرْسِلْكَ وَأَنْتَ قَدْ جَعَلْتَ هَذَا الشَّعْبَ يَتَّكِلُ عَلَى الْكَذِبِ.
\par 16 لِذَلِكَ هَكَذَا قَالَ الرَّبُّ: هَئَنَذَا طَارِدُكَ عَنْ وَجْهِ الأَرْضِ. هَذِهِ السَّنَةَ تَمُوتُ لأَنَّكَ تَكَلَّمْتَ بِعِصْيَانٍ عَلَى الرَّبِّ].
\par 17 فَمَاتَ حَنَنِيَّا النَّبِيُّ فِي تِلْكَ السَّنَةِ فِي الشَّهْرِ السَّابِعِ.

\chapter{29}

\par 1 هَذَا كَلاَمُ الرِّسَالَةِ الَّتِي أَرْسَلَهَا إِرْمِيَا النَّبِيُّ مِنْ أُورُشَلِيمَ إِلَى بَقِيَّةِ شُيُوخِ السَّبْيِ وَإِلَى الْكَهَنَةِ وَالأَنْبِيَاءِ وَإِلَى كُلِّ الشَّعْبِ الَّذِينَ سَبَاهُمْ نَبُوخَذْنَصَّرُ مِنْ أُورُشَلِيمَ إِلَى بَابِلَ
\par 2 بَعْدَ خُرُوجِ يَكُنْيَا الْمَلِكِ وَالْمَلِكَةِ وَالْخِصْيَانِ وَرُؤَسَاءِ يَهُوذَا وَأُورُشَلِيمَ وَالنَّجَّارِينَ وَالْحَدَّادِينَ مِنْ أُورُشَلِيمَ
\par 3 بِيَدِ أَلْعَاسَةَ بْنِ شَافَانَ وَجَمَرْيَا بْنِ حِلْقِيَّا اللَّذَيْنِ أَرْسَلَهُمَا صِدْقِيَّا مَلِكُ يَهُوذَا إِلَى نَبُوخَذْنَصَّرَ مَلِكِ بَابِلَ إِلَى بَابِلَ قَائِلاً:
\par 4 [هَكَذَا قَالَ رَبُّ الْجُنُودِ إِلَهُ إِسْرَائِيلَ لِكُلِّ السَّبْيِ الَّذِي سَبَيْتُهُ مِنْ أُورُشَلِيمَ إِلَى بَابِلَ.
\par 5 اِبْنُوا بُيُوتاً وَاسْكُنُوا وَاغْرِسُوا جَنَّاتٍ وَكُلُوا ثَمَرَهَا.
\par 6 خُذُوا نِسَاءً وَلِدُوا بَنِينَ وَبَنَاتٍ وَخُذُوا لِبَنِيكُمْ نِسَاءً وَأَعْطُوا بَنَاتِكُمْ لِرِجَالٍ فَيَلِدْنَ بَنِينَ وَبَنَاتٍ وَاكْثُرُوا هُنَاكَ وَلاَ تَقِلُّوا.
\par 7 وَاطْلُبُوا سَلاَمَ الْمَدِينَةِ الَّتِي سَبَيْتُكُمْ إِلَيْهَا وَصَلُّوا لأَجْلِهَا إِلَى الرَّبِّ لأَنَّهُ بِسَلاَمِهَا يَكُونُ لَكُمْ سَلاَمٌ.
\par 8 لأَنَّهُ هَكَذَا قَالَ رَبُّ الْجُنُودِ إِلَهُ إِسْرَائِيلَ: لاَ تَغِشَّكُمْ أَنْبِيَاؤُكُمُ الَّذِينَ فِي وَسَطِكُمْ وَعَرَّافُوكُمْ وَلاَ تَسْمَعُوا لأَحْلاَمِكُمُ الَّتِي تَتَحَلَّمُونَهَا.
\par 9 لأَنَّهُمْ إِنَّمَا يَتَنَبَّأُونَ لَكُمْ بِاسْمِي بِالْكَذِبِ. أَنَا لَمْ أُرْسِلْهُمْ يَقُولُ الرَّبُّ.
\par 10 [لأَنَّهُ هَكَذَا قَالَ الرَّبُّ. إِنِّي عِنْدَ تَمَامِ سَبْعِينَ سَنَةً لِبَابِلَ أَتَعَهَّدُكُمْ وَأُقِيمُ لَكُمْ كَلاَمِي الصَّالِحَ بِرَدِّكُمْ إِلَى هَذَا الْمَوْضِعِ.
\par 11 لأَنِّي عَرَفْتُ الأَفْكَارَ الَّتِي أَنَا مُفْتَكِرٌ بِهَا عَنْكُمْ يَقُولُ الرَّبُّ أَفْكَارَ سَلاَمٍ لاَ شَرٍّ لأُعْطِيَكُمْ آخِرَةً وَرَجَاءً.
\par 12 فَتَدْعُونَنِي وَتَذْهَبُونَ وَتُصَلُّونَ إِلَيَّ فَأَسْمَعُ لَكُمْ.
\par 13 وَتَطْلُبُونَنِي فَتَجِدُونَنِي إِذْ تَطْلُبُونَنِي بِكُلِّ قَلْبِكُمْ.
\par 14 فَأُوجَدُ لَكُمْ يَقُولُ الرَّبُّ وَأَرُدُّ سَبْيَكُمْ وَأَجْمَعُكُمْ مِنْ كُلِّ الأُمَمِ وَمِنْ كُلِّ الْمَوَاضِعِ الَّتِي طَرَدْتُكُمْ إِلَيْهَا يَقُولُ الرَّبُّ وَأَرُدُّكُمْ إِلَى الْمَوْضِعِ الَّذِي سَبَيْتُكُمْ مِنْهُ.
\par 15 [لأَنَّكُمْ قُلْتُمْ: قَدْ أَقَامَ لَنَا الرَّبُّ أَنْبِيَاءَ فِي بَابِلَ
\par 16 فَهَكَذَا قَالَ الرَّبُّ لِلْمَلِكِ الْجَالِسِ عَلَى كُرْسِيِّ دَاوُدَ وَلِكُلِّ الشَّعْبِ الْجَالِسِ فِي هَذِهِ الْمَدِينَةِ إِخْوَتِكُمُ الَّذِينَ لَمْ يَخْرُجُوا مَعَكُمْ فِي السَّبْيِ:
\par 17 هَكَذَا قَالَ رَبُّ الْجُنُودِ. هَئَنَذَا أُرْسِلُ عَلَيْهِمِ السَّيْفَ وَالْجُوعَ وَالْوَبَأَ وَأَجْعَلُهُمْ كَتِينٍ رَدِيءٍ لاَ يُؤْكَلُ مِنَ الرَّدَاءَةِ.
\par 18 وَأُلْحِقُهُمْ بِالسَّيْفِ وَالْجُوعِ وَالْوَبَإِ وَأَجْعَلُهُمْ قَلَقاً لِكُلِّ مَمَالِكِ الأَرْضِ حِلْفاً وَدَهَشاً وَصَفِيراً وَعَاراً فِي جَمِيعِ الأُمَمِ الَّذِينَ طَرَدْتُهُمْ إِلَيْهِمْ
\par 19 مِنْ أَجْلِ أَنَّهُمْ لَمْ يَسْمَعُوا لِكَلاَمِي يَقُولُ الرَّبُّ إِذْ أَرْسَلْتُ إِلَيْهِمْ عَبِيدِي الأَنْبِيَاءَ مُبَكِّراً وَمُرْسِلاً وَلَمْ تَسْمَعُوا يَقُولُ الرَّبُّ.
\par 20 [وَأَنْتُمْ فَاسْمَعُوا كَلِمَةَ الرَّبِّ يَا جَمِيعَ السَّبْيِ الَّذِينَ أَرْسَلْتُهُمْ مِنْ أُورُشَلِيمَ إِلَى بَابِلَ.
\par 21 هَكَذَا قَالَ رَبُّ الْجُنُودِ إِلَهُ إِسْرَائِيلَ عَنْ أَخْآبَ بْنِ قُولاَيَا وَعَنْ صِدْقِيَّا بْنِ مَعْسِيَّا اللَّذَيْنِ يَتَنَبَّئَانِ لَكُمْ بِاسْمِي بِالْكَذِبِ. هَئَنَذَا أَدْفَعُهُمَا لِيَدِ نَبُوخَذْنَصَّرَ مَلِكِ بَابِلَ فَيَقْتُلُهُمَا أَمَامَ عُيُونِكُمْ.
\par 22 وَتُؤْخَذُ مِنْهُمَا لَعْنَةٌ لِكُلِّ سَبْيِ يَهُوذَا الَّذِينَ فِي بَابِلَ فَيُقَالُ: يَجْعَلُكَ الرَّبُّ مِثْلَ صِدْقِيَّا وَمِثْلَ أَخْآبَ اللَّذَيْنِ قَلاَهُمَا مَلِكُ بَابِلَ بِالنَّارِ.
\par 23 مِنْ أَجْلِ أَنَّهُمَا عَمِلاَ قَبِيحاً فِي إِسْرَائِيلَ وَزَنَيَا بِنِسَاءِ أَصْحَابِهِمَا وَتَكَلَّمَا بِاسْمِي كَلاَماً كَاذِباً لَمْ أُوصِهِمَا بِهِ وَأَنَا الْعَارِفُ وَالشَّاهِدُ يَقُولُ الرَّبُّ.
\par 24 [وَقُلْ لِشَمَعْيَا النِّحْلاَمِيِّ:
\par 25 هَكَذَا تَكَلَّمَ رَبُّ الْجُنُودِ إِلَهُ إِسْرَائِيلَ: مِنْ أَجْلِ أَنَّكَ أَرْسَلْتَ رَسَائِلَ بِاسْمِكَ إِلَى كُلِّ الشَّعْبِ الَّذِي فِي أُورُشَلِيمَ وَإِلَى صَفَنْيَا بْنِ مَعْسِيَّا الْكَاهِنِ وَإِلَى كُلِّ الْكَهَنَةِ قَائِلاً:
\par 26 قَدْ جَعَلَكَ الرَّبُّ كَاهِناً عِوَضاً عَنْ يَهُويَادَاعَ الْكَاهِنِ لِتَكُونُوا وُكَلاَءَ فِي بَيْتِ الرَّبِّ لِكُلِّ رَجُلٍ مَجْنُونٍ وَمُتَنَبِّئٍ فَتَدْفَعُهُ إِلَى الْمِقْطَرَةِ وَالْقُيُودِ.
\par 27 وَالآنَ لِمَاذَا لَمْ تَزْجُرْ إِرْمِيَا الْعَنَاثُوثِيَّ الْمُتَنَبِّئَ لَكُمْ.
\par 28 لأَنَّهُ لِذَلِكَ أَرْسَلَ إِلَيْنَا إِلَى بَابِلَ قَائِلاً إِنَّهَا مُسْتَطِيلَةٌ. ابْنُوا بُيُوتاً وَاسْكُنُوا وَاغْرِسُوا جَنَّاتٍ وَكُلُوا ثَمَرَهَا].
\par 29 فَقَرَأَ صَفَنْيَا الْكَاهِنُ هَذِهِ الرِّسَالَةَ فِي أُذُنَيْ إِرْمِيَا النَّبِيِّ.
\par 30 ثُمَّ صَارَ كَلاَمُ الرَّبِّ إِلَى إِرْمِيَا:
\par 31 [أَرْسِلْ إِلَى كُلِّ السَّبْيِ قَائِلاً: هَكَذَا قَالَ الرَّبُّ لِشَمَعْيَا النِّحْلاَمِيِّ: مِنْ أَجْلِ أَنَّ شَمَعْيَا قَدْ تَنَبَّأَ لَكُمْ وَأَنَا لَمْ أُرْسِلْهُ وَجَعَلَكُمْ تَتَّكِلُونَ عَلَى الْكَذِبِ.
\par 32 لِذَلِكَ هَكَذَا قَالَ الرَّبُّ. هَئَنَذَا أُعَاقِبُ شَمَعْيَا النِّحْلاَمِيَّ وَنَسْلَهُ. لاَ يَكُونُ لَهُ إِنْسَانٌ يَجْلِسُ فِي وَسَطِ هَذَا الشَّعْبِ وَلاَ يَرَى الْخَيْرَ الَّذِي سَأَصْنَعُهُ لِشَعْبِي يَقُولُ الرَّبُّ لأَنَّهُ تَكَلَّمَ بِعِصْيَانٍ عَلَى الرَّبِّ].

\chapter{30}

\par 1 اَلْكَلاَمُ الَّذِي صَارَ إِلَى إِرْمِيَا مِنْ الرَّبِّ:
\par 2 [هَكَذَا تَكَلَّمَ الرَّبُّ إِلَهُ إِسْرَائِيلَ: اكْتُبْ كُلَّ الْكَلاَمِ الَّذِي تَكَلَّمْتُ بِهِ إِلَيْكَ فِي سِفْرٍ
\par 3 لأَنَّهُ هَا أَيَّامٌ تَأْتِي وَأَرُدُّ سَبْيَ شَعْبِي إِسْرَائِيلَ وَيَهُوذَا وَأُرْجِعُهُمْ إِلَى الأَرْضِ الَّتِي أَعْطَيْتُ آبَاءَهُمْ إِيَّاهَا فَيَمْتَلِكُونَهَا].
\par 4 فَهَذَا هُوَ الْكَلاَمُ الَّذِي تَكَلَّمَ بِهِ الرَّبُّ عَنْ إِسْرَائِيلَ وَعَنْ يَهُوذَا:
\par 5 [لأَنَّهُ هَكَذَا قَالَ الرَّبُّ: صَوْتَ ارْتِعَادٍ سَمِعْنَا. خَوْفٌ وَلاَ سَلاَمٌ.
\par 6 اِسْأَلُوا وَانْظُرُوا إِنْ كَانَ ذَكَرٌ يَضَعُ! لِمَاذَا أَرَى كُلَّ رَجُلٍ يَدَاهُ عَلَى حَقَوَيْهِ كَمَاخِضٍ وَتَحَوَّلَ كُلُّ وَجْهٍ إِلَى صُفْرَةٍ؟
\par 7 آهِ! لأَنَّ ذَلِكَ الْيَوْمَ عَظِيمٌ وَلَيْسَ مِثْلُهُ. وَهُوَ وَقْتُ ضِيقٍ عَلَى يَعْقُوبَ وَلَكِنَّهُ سَيُخَلَّصُ مِنْهُ.
\par 8 وَيَكُونُ فِي ذَلِكَ الْيَوْمِ يَقُولُ رَبُّ الْجُنُودِ أَنِّي أَكْسِرُ نِيرَهُ عَنْ عُنُقِكَ وَأَقْطَعُ رُبُطَكَ وَلاَ يَسْتَعْبِدُهُ بَعْدُ الْغُرَبَاءُ
\par 9 بَلْ يَخْدِمُونَ الرَّبَّ إِلَهَهُمْ وَدَاوُدَ مَلِكَهُمُ الَّذِي أُقِيمُهُ لَهُمْ.
\par 10 [أَمَّا أَنْتَ يَا عَبْدِي يَعْقُوبَ فَلاَ تَخَفْ يَقُولُ الرَّبُّ وَلاَ تَرْتَعِبْ يَا إِسْرَائِيلُ لأَنِّي هَئَنَذَا أُخَلِّصُكَ مِنْ بَعِيدٍ وَنَسْلَكَ مِنْ أَرْضِ سَبْيِهِ فَيَرْجِعُ يَعْقُوبُ وَيَطْمَئِنُّ وَيَسْتَرِيحُ وَلاَ مُزْعِجَ.
\par 11 لأَنِّي أَنَا مَعَكَ يَقُولُ الرَّبُّ لأُخَلِّصَكَ. وَإِنْ أَفْنَيْتُ جَمِيعَ الأُمَمِ الَّذِينَ بَدَّدْتُكَ إِلَيْهِمْ فَأَنْتَ لاَ أُفْنِيكَ بَلْ أُؤَدِّبُكَ بِالْحَقِّ وَلاَ أُبَرِّئُكَ تَبْرِئَةً.
\par 12 لأَنَّهُ هَكَذَا قَالَ الرَّبُّ: كَسْرُكِ عَدِيمُ الْجَبْرِ وَجُرْحُكِ عُضَالٌ.
\par 13 لَيْسَ مَنْ يَقْضِي حَاجَتَكِ لِلْعَصْرِ. لَيْسَ لَكِ عَقَاقِيرُ رِفَادَةٍ.
\par 14 قَدْ نَسِيَكِ كُلُّ مُحِبِّيكِ. إِيَّاكِ لَمْ يَطْلُبُوا لأَنِّي ضَرَبْتُكِ ضَرْبَةَ عَدُوٍّ تَأْدِيبَ قَاسٍ لأَنَّ إِثْمَكِ قَدْ كَثُرَ وَخَطَايَاكِ تَعَاظَمَتْ.
\par 15 مَا بَالُكِ تَصْرُخِينَ بِسَبَبِ كَسْرِكِ؟ جُرْحُكِ عَدِيمُ الْبَرْءِ لأَنَّ إِثْمَكِ قَدْ كَثُرَ وَخَطَايَاكِ تَعَاظَمَتْ قَدْ صَنَعْتُ هَذِهِ بِكِ.
\par 16 لِذَلِكَ يُؤْكَلُ كُلُّ آكِلِيكِ وَيَذْهَبُ كُلُّ أَعْدَائِكِ قَاطِبَةً إِلَى السَّبْيِ وَيَكُونُ كُلُّ سَالِبِيكِ سَلْباً وَأَدْفَعُ كُلَّ نَاهِبِيكِ لِلنَّهْبِ.
\par 17 لأَنِّي أَرْفُدُكِ وَأَشْفِيكِ مِنْ جُرُوحِكِ يَقُولُ الرَّبُّ. لأَنَّهُمْ قَدْ دَعُوكِ مَنْفِيَّةَ صِهْيَوْنَ الَّتِي لاَ سَائِلَ عَنْهَا.]
\par 18 هَكَذَا قَالَ الرَّبُّ: [هَئَنَذَا أَرُدُّ سَبْيَ خِيَامِ يَعْقُوبَ وَأَرْحَمُ مَسَاكِنَهُ وَتُبْنَى الْمَدِينَةُ عَلَى تَلِّهَا وَالْقَصْرُ يُسْكَنُ عَلَى عَادَتِهِ.
\par 19 وَيَخْرُجُ مِنْهُمُ الْحَمْدُ وَصَوْتُ اللاَّعِبِينَ وَأُكَثِّرُهُمْ وَلاَ يَقِلُّونَ وَأُعَظِّمُهُمْ وَلاَ يَصْغُرُونَ.
\par 20 وَيَكُونُ بَنُوهُمْ كَمَا فِي الْقَدِيمِ وَجَمَاعَتُهُمْ تَثْبُتُ أَمَامِي وَأُعَاقِبُ كُلَّ مُضَايِقِيهِمْ.
\par 21 وَيَكُونُ حَاكِمُهُمْ مِنْهُمْ وَيَخْرُجُ وَالِيهِمْ مِنْ وَسَطِهِمْ وَأُقَرِّبُهُ فَيَدْنُو إِلَيَّ لأَنَّهُ مَنْ هُوَ هَذَا الَّذِي أَرْهَنَ قَلْبَهُ لِيَدْنُوَ إِلَيَّ يَقُولُ الرَّبُّ؟
\par 22 وَتَكُونُونَ لِي شَعْباً وَأَنَا أَكُونُ لَكُمْ إِلَهاً].
\par 23 هُوَذَا زَوْبَعَةُ الرَّبِّ تَخْرُجُ بِغَضَبٍ نَوْءٌ جَارِفٌ. عَلَى رَأْسِ الأَشْرَارِ يَثُورُ.
\par 24 لاَ يَرْتَدُّ حُمُوُّ غَضَبِ الرَّبِّ حَتَّى يَفْعَلَ وَحَتَّى يُقِيمَ مَقَاصِدَ قَلْبِهِ. فِي آخِرِ الأَيَّامِ تَفْهَمُونَهَا.

\chapter{31}

\par 1 [فِي ذَلِكَ الزَّمَانِ يَقُولُ الرَّبُّ أَكُونُ إِلَهاً لِكُلِّ عَشَائِرِ إِسْرَائِيلَ وَهُمْ يَكُونُونَ لِي شَعْباً].
\par 2 هَكَذَا قَالَ الرَّبُّ: [قَدْ وَجَدَ نِعْمَةً فِي الْبَرِّيَّةِ الشَّعْبُ الْبَاقِي عَنِ السَّيْفِ إِسْرَائِيلُ حِينَ سِرْتُ لأُرِيحَهُ].
\par 3 تَرَاءَى لِي الرَّبُّ مِنْ بَعِيدٍ: [وَمَحَبَّةً أَبَدِيَّةً أَحْبَبْتُكِ مِنْ أَجْلِ ذَلِكَ أَدَمْتُ لَكِ الرَّحْمَةَ.
\par 4 سَأَبْنِيكِ بَعْدُ فَتُبْنَيْنَ يَا عَذْرَاءَ إِسْرَائِيلَ. تَتَزَيَّنِينَ بَعْدُ بِدُفُوفِكِ وَتَخْرُجِينَ فِي رَقْصِ اللاَّعِبِينَ.
\par 5 تَغْرِسِينَ بَعْدُ كُرُوماً فِي جِبَالِ السَّامِرَةِ. يَغْرِسُ الْغَارِسُونَ وَيَبْتَكِرُونَ.
\par 6 لأَنَّهُ يَكُونُ يَوْمٌ يُنَادِي فِيهِ النَّوَاطِيرُ فِي جِبَالِ أَفْرَايِمَ: قُومُوا فَنَصْعَدَ إِلَى صِهْيَوْنَ إِلَى الرَّبِّ إِلَهِنَا.
\par 7 لأَنَّهُ هَكَذَا قَالَ الرَّبُّ: رَنِّمُوا لِيَعْقُوبَ فَرَحاً وَاهْتِفُوا بِرَأْسِ الشُّعُوبِ. سَمِّعُوا سَبِّحُوا وَقُولُوا: خَلِّصْ يَا رَبُّ شَعْبَكَ بَقِيَّةَ إِسْرَائِيلَ.
\par 8 هَئَنَذَا آتِي بِهِمْ مِنْ أَرْضِ الشِّمَالِ وَأَجْمَعُهُمْ مِنْ أَطْرَافِ الأَرْضِ. بَيْنَهُمُ الأَعْمَى وَالأَعْرَجُ الْحُبْلَى وَالْمَاخِضُ مَعاً. جَمْعٌ عَظِيمٌ يَرْجِعُ إِلَى هُنَا.
\par 9 بِالْبُكَاءِ يَأْتُونَ وَبِالتَّضَرُّعَاتِ أَقُودُهُمْ. أُسَيِّرُهُمْ إِلَى أَنْهَارِ مَاءٍ فِي طَرِيقٍ مُسْتَقِيمَةٍ لاَ يَعْثُرُونَ فِيهَا. لأَنِّي صِرْتُ لإِسْرَائِيلَ أَباً وَأَفْرَايِمُ هُوَ بِكْرِي].
\par 10 اِسْمَعُوا كَلِمَةَ الرَّبِّ أَيُّهَا الأُمَمُ وَأَخْبِرُوا فِي الْجَزَائِرِ الْبَعِيدَةِ وَقُولُوا: [مُبَدِّدُ إِسْرَائِيلَ يَجْمَعُهُ وَيَحْرُسُهُ كَرَاعٍ قَطِيعَهُ].
\par 11 لأَنَّ الرَّبَّ فَدَى يَعْقُوبَ وَفَكَّهُ مِنْ يَدِ الَّذِي هُوَ أَقْوَى مِنْهُ.
\par 12 فَيَأْتُونَ وَيُرَنِّمُونَ فِي مُرْتَفَعِ صِهْيَوْنَ وَيَجْرُونَ إِلَى جُودِ الرَّبِّ عَلَى الْحِنْطَةِ وَعَلَى الْخَمْرِ وَعَلَى الزَّيْتِ وَعَلَى أَبْنَاءِ الْغَنَمِ وَالْبَقَرِ. وَتَكُونُ نَفْسُهُمْ كَجَنَّةٍ رَيَّا وَلاَ يَعُودُونَ يَذُوبُونَ بَعْدُ.
\par 13 حِينَئِذٍ تَفْرَحُ الْعَذْرَاءُ بِالرَّقْصِ وَالشُّبَّانُ وَالشُّيُوخُ مَعاً. وَأُحَوِّلُ نَوْحَهُمْ إِلَى طَرَبٍ وَأُعَزِّيهِمْ وَأُفَرِّحُهُمْ مِنْ حُزْنِهِمْ.
\par 14 وَأُرْوِي نَفْسَ الْكَهَنَةِ مِنَ الدَّسَمِ وَيَشْبَعُ شَعْبِي مِنْ جُودِهِ يَقُولُ الرَّبُّ.
\par 15 هَكَذَا قَالَ الرَّبُّ: [صَوْتٌ سُمِعَ فِي الرَّامَةِ نَوْحٌ بُكَاءٌ مُرٌّ. رَاحِيلُ تَبْكِي عَلَى أَوْلاَدِهَا وَتَأْبَى أَنْ تَتَعَزَّى عَنْ أَوْلاَدِهَا لأَنَّهُمْ لَيْسُوا بِمَوْجُودِينَ].
\par 16 هَكَذَا قَالَ الرَّبُّ: [امْنَعِي صَوْتَكِ عَنِ الْبُكَاءِ وَعَيْنَيْكِ عَنِ الدُّمُوعِ لأَنَّهُ يُوجَدُ جَزَاءٌ لِعَمَلِكِ يَقُولُ الرَّبُّ. فَيَرْجِعُونَ مِنْ أَرْضِ الْعَدُوِّ.
\par 17 وَيُوجَدُ رَجَاءٌ لِآخِرَتِكِ يَقُولُ الرَّبُّ]. فَيَرْجِعُ الأَبْنَاءُ إِلَى تُخُمِهِمْ.
\par 18 سَمْعاً سَمِعْتُ أَفْرَايِمَ يَنْتَحِبُ: [أَدَّبْتَنِي فَتَأَدَّبْتُ كَعِجْلٍ غَيْرِ مَرُوضٍ. تَوِّبْنِي فَأَتُوبَ لأَنَّكَ أَنْتَ الرَّبُّ إِلَهِي.
\par 19 لأَنِّي بَعْدَ رُجُوعِي نَدِمْتُ وَبَعْدَ تَعَلُّمِي صَفَقْتُ عَلَى فَخْذِي. خَزِيتُ وَخَجِلْتُ لأَنِّي قَدْ حَمَلْتُ عَارَ صِبَايَ].
\par 20 هَلْ أَفْرَايِمُ ابْنٌ عَزِيزٌ لَدَيَّ أَوْ وَلَدٌ مُسِرٌّ؟ لأَنِّي كُلَّمَا تَكَلَّمْتُ بِهِ أَذْكُرُهُ بَعْدُ ذِكْراً. مِنْ أَجْلِ ذَلِكَ حَنَّتْ أَحْشَائِي إِلَيْهِ. رَحْمَةً أَرْحَمُهُ يَقُولُ الرَّبُّ.
\par 21 اِنْصِبِي لِنَفْسِكِ صُوًى. اجْعَلِي لِنَفْسِكِ أَنْصَاباً. اجْعَلِي قَلْبَكِ نَحْوَ السِّكَّةِ الطَّرِيقِ الَّتِي ذَهَبْتِ فِيهَا. ارْجِعِي يَا عَذْرَاءَ إِسْرَائِيلَ. ارْجِعِي إِلَى مُدُنِكِ هَذِهِ.
\par 22 حَتَّى مَتَى تَطُوفِينَ أَيَّتُهَا الْبِنْتُ الْمُرْتَدَّةُ؟ لأَنَّ الرَّبَّ قَدْ خَلَقَ شَيْئاً حَدِيثاً فِي الأَرْضِ. أُنْثَى تُحِيطُ بِرَجُلٍ.
\par 23 هَكَذَا قَالَ رَبُّ الْجُنُودِ إِلَهُ إِسْرَائِيلَ: [سَيَقُولُونَ بَعْدُ هَذِهِ الْكَلِمَةَ فِي أَرْضِ يَهُوذَا وَفِي مُدُنِهَا عِنْدَمَا أَرُدُّ سَبْيَهُمْ: يُبَارِكُكَ الرَّبُّ يَا مَسْكَنَ الْبِرِّ يَا أَيُّهَا الْجَبَلُ الْمُقَدَّسُ.
\par 24 فَيَسْكُنُ فِيهِ يَهُوذَا وَكُلُّ مُدُنِهِ مَعاً الْفَلاَّحُونَ وَالَّذِينَ يُسَرِّحُونَ الْقُطْعَانَ.
\par 25 لأَنِّي أَرْوَيْتُ النَّفْسَ الْمُعْيِيَةَ وَمَلَأْتُ كُلَّ نَفْسٍ ذَائِبَةٍ.
\par 26 عَلَى ذَلِكَ اسْتَيْقَظْتُ وَنَظَرْتُ وَلَذَّ لِي نَوْمِي].
\par 27 هَا أَيَّامٌ تَأْتِي يَقُولُ الرَّبُّ وَأَزْرَعُ بَيْتَ إِسْرَائِيلَ وَبَيْتَ يَهُوذَا بِزَرْعِ إِنْسَانٍ وَزَرْعِ حَيَوَانٍ.
\par 28 وَيَكُونُ كَمَا سَهِرْتُ عَلَيْهِمْ لِلاِقْتِلاَعِ وَالْهَدْمِ وَالْقَرْضِ وَالإِهْلاَكِ وَالأَذَى كَذَلِكَ أَسْهَرُ عَلَيْهِمْ لِلْبِنَاءِ وَالْغَرْسِ يَقُولُ الرَّبُّ.
\par 29 فِي تِلْكَ الأَيَّامِ لاَ يَقُولُونَ بَعْدُ: [الآبَاءُ أَكَلُوا حِصْرِماً وَأَسْنَانُ الأَبْنَاءِ ضَرِسَتْ].
\par 30 بَلْ: [كُلُّ وَاحِدٍ يَمُوتُ بِذَنْبِهِ]. كُلُّ إِنْسَانٍ يَأْكُلُ الْحِصْرِمَ تَضْرَسُ أَسْنَانُهُ.
\par 31 هَا أَيَّامٌ تَأْتِي يَقُولُ الرَّبُّ وَأَقْطَعُ مَعَ بَيْتِ إِسْرَائِيلَ وَمَعَ بَيْتِ يَهُوذَا عَهْداً جَدِيداً.
\par 32 لَيْسَ كَالْعَهْدِ الَّذِي قَطَعْتُهُ مَعَ آبَائِهِمْ يَوْمَ أَمْسَكْتُهُمْ بِيَدِهِمْ لأُخْرِجَهُمْ مِنْ أَرْضِ مِصْرَ حِينَ نَقَضُوا عَهْدِي فَرَفَضْتُهُمْ يَقُولُ الرَّبُّ.
\par 33 بَلْ هَذَا هُوَ الْعَهْدُ الَّذِي أَقْطَعُهُ مَعَ بَيْتِ إِسْرَائِيلَ بَعْدَ تِلْكَ الأَيَّامِ يَقُولُ الرَّبُّ: أَجْعَلُ شَرِيعَتِي فِي دَاخِلِهِمْ وَأَكْتُبُهَا عَلَى قُلُوبِهِمْ وَأَكُونُ لَهُمْ إِلَهاً وَهُمْ يَكُونُونَ لِي شَعْباً.
\par 34 وَلاَ يُعَلِّمُونَ بَعْدُ كُلُّ وَاحِدٍ صَاحِبَهُ وَكُلُّ وَاحِدٍ أَخَاهُ قَائِلِينَ: [اعْرِفُوا الرَّبَّ] لأَنَّهُمْ كُلَّهُمْ سَيَعْرِفُونَنِي مِنْ صَغِيرِهِمْ إِلَى كَبِيرِهِمْ يَقُولُ الرَّبُّ. لأَنِّي أَصْفَحُ عَنْ إِثْمِهِمْ وَلاَ أَذْكُرُ خَطِيَّتَهُمْ بَعْدُ.
\par 35 هَكَذَا قَالَ الرَّبُّ الْجَاعِلُ الشَّمْسَ لِلإِضَاءَةِ نَهَاراً وَفَرَائِضَ الْقَمَرِ وَالنُّجُومِ لِلإِضَاءَةِ لَيْلاً الزَّاجِرُ الْبَحْرَ حِينَ تَعِجُّ أَمْوَاجُهُ رَبُّ الْجُنُودِ اسْمُهُ:
\par 36 [إِنْ كَانَتْ هَذِهِ الْفَرَائِضُ تَزُولُ مِنْ أَمَامِي يَقُولُ الرَّبُّ فَإِنَّ نَسْلَ إِسْرَائِيلَ أَيْضاً يَكُفُّ مِنْ أَنْ يَكُونَ أُمَّةً أَمَامِي كُلَّ الأَيَّامِ.
\par 37 هَكَذَا قَالَ الرَّبُّ: إِنْ كَانَتِ السَّمَاوَاتُ تُقَاسُ مِنْ فَوْقُ وَتُفْحَصُ أَسَاسَاتُ الأَرْضِ مِنْ أَسْفَلُ فَإِنِّي أَنَا أَيْضاً أَرْفُضُ كُلَّ نَسْلِ إِسْرَائِيلَ مِنْ أَجْلِ كُلِّ مَا عَمِلُوا يَقُولُ الرَّبُّ].
\par 38 هَا أَيَّامٌ تَأْتِي يَقُولُ الرَّبُّ وَتُبْنَى الْمَدِينَةُ لِلرَّبِّ مِنْ بُرْجِ حَنَنْئِيلَ إِلَى بَابِ الزَّاوِيَةِ
\par 39 وَيَخْرُجُ بَعْدُ خَيْطُ الْقِيَاسِ مُقَابِلَهُ عَلَى أَكَمَةِ جَارِبَ وَيَسْتَدِيرُ إِلَى جَوْعَةَ
\par 40 وَيَكُونُ كُلُّ وَادِي الْجُثَثِ وَالرَّمَادِ وَكُلُّ الْحُقُولِ إِلَى وَادِي قَدْرُونَ إِلَى زَاوِيَةِ بَابِ الْخَيْلِ شَرْقاً قُدْساً لِلرَّبِّ. لاَ تُقْلَعُ وَلاَ تُهْدَمُ إِلَى الأَبَدِ.

\chapter{32}

\par 1 اَلْكَلِمَةُ الَّتِي صَارَتْ إِلَى إِرْمِيَا مِنْ الرَّبِّ فِي السَّنَةِ الْعَاشِرَةِ لِصِدْقِيَّا مَلِكِ يَهُوذَا (هِيَ السَّنَةُ الثَّامِنَةُ عَشَرَةَ لِنَبُوخَذْنَصَّرَ)
\par 2 وَكَانَ حِينَئِذٍ جَيْشُ مَلِكِ بَابِلَ يُحَاصِرُ أُورُشَلِيمَ وَكَانَ إِرْمِيَا النَّبِيُّ مَحْبُوساً فِي دَارِ السِّجْنِ الَّذِي فِي بَيْتِ مَلِكِ يَهُوذَا
\par 3 لأَنَّ صِدْقِيَّا مَلِكَ يَهُوذَا حَبَسَهُ قَائِلاً: [لِمَاذَا تَنَبَّأْتَ قَائِلاً هَكَذَا قَالَ الرَّبُّ: هَئَنَذَا أَدْفَعُ هَذِهِ الْمَدِينَةَ لِيَدِ مَلِكِ بَابِلَ فَيَأْخُذُهَا؟
\par 4 وَصِدْقِيَّا مَلِكُ يَهُوذَا لاَ يُفْلِتُ مِنْ يَدِ الْكِلْدَانِيِّينَ بَلْ إِنَّمَا يُدْفَعُ لِيَدِ مَلِكِ بَابِلَ وَيُكَلِّمُهُ فَماً لِفَمٍ وَعَيْنَاهُ تَرَيَانِ عَيْنَيْهِ
\par 5 وَيَسِيرُ بِصِدْقِيَّا إِلَى بَابِلَ فَيَكُونُ هُنَاكَ حَتَّى أَفْتَقِدَهُ يَقُولُ الرَّبُّ. إِنْ حَارَبْتُمُ الْكِلْدَانِيِّينَ لاَ تَنْجَحُونَ].
\par 6 فَقَالَ إِرْمِيَا: [كَلِمَةُ الرَّبِّ صَارَتْ إِلَيَّ قَائِلَةً:
\par 7 هُوَذَا حَنَمْئِيلُ بْنُ شَلُّومَ عَمِّكَ يَأْتِي إِلَيْكَ قَائِلاً: اشْتَرِ لِنَفْسِكَ حَقْلِي الَّذِي فِي عَنَاثُوثَ لأَنَّ لَكَ حَقَّ الْفِكَاكِ لِلشِّرَاءِ].
\par 8 فَجَاءَ إِلَيَّ حَنَمْئِيلُ ابْنُ عَمِّي حَسَبَ كَلِمَةِ الرَّبِّ إِلَى دَارِ السِّجْنِ وَقَالَ لِي: [اشْتَرِ حَقْلِي الَّذِي فِي عَنَاثُوثَ الَّذِي فِي أَرْضِ بِنْيَامِينَ لأَنَّ لَكَ حَقَّ الإِرْثِ وَلَكَ الْفِكَاكُ. اشْتَرِهِ لِنَفْسِكَ]. فَعَرَفْتُ أَنَّهَا كَلِمَةُ الرَّبِّ.
\par 9 فَاشْتَرَيْتُ مِنْ حَنَمْئِيلَ ابْنِ عَمِّي الْحَقْلَ الَّذِي فِي عَنَاثُوثَ وَوَزَنْتُ لَهُ سَبْعَةَ عَشَرَ شَاقِلاً مِنَ الْفِضَّةِ.
\par 10 وَكَتَبْتُهُ فِي صَكٍّ وَخَتَمْتُ وَأَشْهَدْتُ شُهُوداً وَوَزَنْتُ الْفِضَّةَ بِمَوَازِينَ.
\par 11 وَأَخَذْتُ صَكَّ الشِّرَاءِ الْمَخْتُومَ حَسَبَ الْوَصِيَّةِ وَالْفَرِيضَةِ وَالْمَفْتُوحَ
\par 12 وَسَلَّمْتُ صَكَّ الشِّرَاءِ لِبَارُوخَ بْنِ نِيرِيَّا بْنِ مَحْسِيَا أَمَامَ حَنَمْئِيلَ ابْنِ عَمِّي وَأَمَامَ الشُّهُودِ الَّذِينَ أَمْضُوا صَكَّ الشِّرَاءِ أَمَامَ كُلِّ الْيَهُودِ الْجَالِسِينَ فِي دَارِ السِّجْنِ.
\par 13 وَأَوْصَيْتُ بَارُوخَ أَمَامَهُمْ قَائِلاً:
\par 14 [هَكَذَا قَالَ رَبُّ الْجُنُودِ إِلَهُ إِسْرَائِيلَ: خُذْ هَذَيْنِ الصَّكَّيْنِ صَكَّ الشِّرَاءِ هَذَا الْمَخْتُومَ وَالصَّكَّ الْمَفْتُوحَ هَذَا وَاجْعَلْهُمَا فِي إِنَاءٍ مِنْ خَزَفٍ لِيَبْقَيَا أَيَّاماً كَثِيرَةً.
\par 15 لأَنَّهُ هَكَذَا قَالَ رَبُّ الْجُنُودِ إِلَهُ إِسْرَائِيلَ: سَيَشْتَرُونَ بَعْدُ بُيُوتاً وَحُقُولاً وَكُرُوماً فِي هَذِهِ الأَرْضِ].
\par 16 ثُمَّ صَلَّيْتُ إِلَى الرَّبِّ بَعْدَ تَسْلِيمِ صَكِّ الشِّرَاءِ لِبَارُوخَ بْنِ نِيرِيَّا:
\par 17 [آهِ أَيُّهَا السَّيِّدُ الرَّبُّ هَا إِنَّكَ قَدْ صَنَعْتَ السَّمَاوَاتِ وَالأَرْضَ بِقُوَّتِكَ الْعَظِيمَةِ وَبِذِرَاعِكَ الْمَمْدُودَةِ. لاَ يَعْسُرُ عَلَيْكَ شَيْءٌ.
\par 18 صَانِعُ الإِحْسَانِ لأُلُوفٍ وَمُجَازِي ذَنْبِ الآبَاءِ فِي حِضْنِ بَنِيهِمْ بَعْدَهُمُ الإِلَهُ الْعَظِيمُ الْجَبَّارُ رَبُّ الْجُنُودِ اسْمُهُ
\par 19 عَظِيمٌ فِي الْمَشُورَةِ وَقَادِرٌ فِي الْعَمَلِ الَّذِي عَيْنَاكَ مَفْتُوحَتَانِ عَلَى كُلِّ طُرُقِ بَنِي آدَمَ لِتُعْطِيَ كُلَّ وَاحِدٍ حَسَبَ طُرُقِهِ وَحَسَبَ ثَمَرِ أَعْمَالِهِ.
\par 20 الَّذِي جَعَلْتَ آيَاتٍ وَعَجَائِبَ فِي أَرْضِ مِصْرَ إِلَى هَذَا الْيَوْمِ وَفِي إِسْرَائِيلَ وَفِي النَّاسِ وَجَعَلْتَ لِنَفْسِكَ اسْماً كَهَذَا الْيَوْمِ
\par 21 وَأَخْرَجْتَ شَعْبَكَ إِسْرَائِيلَ مِنْ أَرْضِ مِصْرَ بِآيَاتٍ وَعَجَائِبَ وَبِيَدٍ شَدِيدَةٍ وَذِرَاعٍ مَمْدُودَةٍ وَمَخَافَةٍ عَظِيمَةٍ
\par 22 وَأَعْطَيْتَهُمْ هَذِهِ الأَرْضَ الَّتِي حَلَفْتَ لِآبَائِهِمْ أَنْ تُعْطِيَهُمْ إِيَّاهَا أَرْضاً تَفِيضُ لَبَناً وَعَسَلاً.
\par 23 فَأَتُوا وَامْتَلَكُوهَا وَلَمْ يَسْمَعُوا لِصَوْتِكَ وَلاَ سَارُوا فِي شَرِيعَتِكَ. كُلُّ مَا أَوْصَيْتَهُمْ أَنْ يَعْمَلُوهُ لَمْ يَعْمَلُوهُ فَأَوْقَعْتَ بِهِمْ كُلَّ هَذَا الشَّرِّ.
\par 24 هَا الْمَتَارِسُ! قَدْ أَتُوا إِلَى الْمَدِينَةِ لِيَأْخُذُوهَا وَقَدْ دُفِعَتِ الْمَدِينَةُ لِيَدِ الْكِلْدَانِيِّينَ الَّذِينَ يُحَارِبُونَهَا بِسَبَبِ السَّيْفِ وَالْجُوعِ وَالْوَبَإِ وَمَا تَكَلَّمْتَ بِهِ فَقَدْ حَدَثَ وَهَا أَنْتَ نَاظِرٌ.
\par 25 وَقَدْ قُلْتَ أَنْتَ لِي أَيُّهَا السَّيِّدُ الرَّبُّ: اشْتَرِ لِنَفْسِكَ الْحَقْلَ بِفِضَّةٍ وَأَشْهِدْ شُهُوداً وَقَدْ دُفِعَتِ الْمَدِينَةُ لِيَدِ الْكِلْدَانِيِّينَ].
\par 26 ثُمَّ صَارَتْ كَلِمَةُ الرَّبِّ إِلَى إِرْمِيَا:
\par 27 [هَئَنَذَا الرَّبُّ إِلَهُ كُلِّ ذِي جَسَدٍ. هَلْ يَعْسُرُ عَلَيَّ أَمْرٌ مَا؟
\par 28 لِذَلِكَ هَكَذَا قَالَ الرَّبُّ. هَئَنَذَا أَدْفَعُ هَذِهِ الْمَدِينَةَ لِيَدِ الْكِلْدَانِيِّينَ وَلِيَدِ نَبُوخَذْنَصَّرَ مَلِكِ بَابِلَ فَيَأْخُذُهَا.
\par 29 فَيَأْتِي الْكِلْدَانِيُّونَ الَّذِينَ يُحَارِبُونَ هَذِهِ الْمَدِينَةَ فَيُشْعِلُونَ هَذِهِ الْمَدِينَةَ بِالنَّارِ وَيُحْرِقُونَهَا وَالْبُيُوتَ الَّتِي بَخَّرُوا عَلَى سُطُوحِهَا لِلْبَعْلِ وَسَكَبُوا سَكَائِبَ لِآلِهَةٍ أُخْرَى لِيُغِيظُونِي.
\par 30 لأَنَّ بَنِي إِسْرَائِيلَ وَبَنِي يَهُوذَا صَنَعُوا الشَّرَّ فِي عَيْنَيَّ مُنْذُ صِبَاهُمْ. لأَنَّ بَنِي إِسْرَائِيلَ أَغَاظُونِي بِعَمَلِ أَيْدِيهِمْ يَقُولُ الرَّبُّ.
\par 31 لأَنَّ هَذِهِ الْمَدِينَةَ قَدْ صَارَتْ لِي لِغَضَبِي وَلِغَيْظِي مِنَ الْيَوْمِ الَّذِي فِيهِ بَنُوهَا إِلَى هَذَا الْيَوْمِ لأَنْزِعَهَا مِنْ أَمَامِ وَجْهِي
\par 32 مِنْ أَجْلِ كُلِّ شَرِّ بَنِي إِسْرَائِيلَ وَبَنِي يَهُوذَا الَّذِي عَمِلُوهُ لِيُغِيظُونِي بِهِ هُمْ وَمُلُوكُهُمْ وَرُؤَسَاؤُهُمْ وَكَهَنَتُهُمْ وَأَنْبِيَاؤُهُمْ وَرِجَالُ يَهُوذَا وَسُكَّانُ أُورُشَلِيمَ.
\par 33 وَقَدْ حَوَّلُوا لِي الْقَفَا لاَ الْوَجْهَ وَقَدْ عَلَّمْتُهُمْ مُبَكِّراً وَمُعَلِّماً وَلَكِنَّهُمْ لَمْ يَسْمَعُوا لِيَقْبَلُوا أَدَباً.
\par 34 بَلْ وَضَعُوا مَكْرُهَاتِهِمْ فِي الْبَيْتِ الَّذِي دُعِيَ بِاسْمِي لِيُنَجِّسُوهُ.
\par 35 وَبَنُوا الْمُرْتَفِعَاتِ لِلْبَعْلِ الَّتِي فِي وَادِي ابْنِ هِنُّومَ لِيُجِيزُوا بَنِيهِمْ وَبَنَاتِهِمْ فِي النَّارِ لِمُولَكَ الأَمْرَ الَّذِي لَمْ أُوصِهِمْ بِهِ وَلاَ صَعِدَ عَلَى قَلْبِي لِيَعْمَلُوا هَذَا الرِّجْسَ لِيَجْعَلُوا يَهُوذَا يُخْطِئُ.
\par 36 [وَالآنَ لِذَلِكَ هَكَذَا قَالَ الرَّبُّ إِلَهُ إِسْرَائِيلَ عَنْ هَذِهِ الْمَدِينَةِ الَّتِي تَقُولُونَ إِنَّهَا قَدْ دُفِعَتْ لِيَدِ مَلِكِ بَابِلَ بِالسَّيْفِ وَالْجُوعِ وَالْوَبَإِ.
\par 37 هَئَنَذَا أَجْمَعُهُمْ مِنْ كُلِّ الأَرَاضِي الَّتِي طَرَدْتُهُمْ إِلَيْهَا بِغَضَبِي وَغَيْظِي وَبَسَخْطٍ عَظِيمٍ وَأَرُدُّهُمْ إِلَى هَذَا الْمَوْضِعِ وَأُسَكِّنُهُمْ آمِنِينَ.
\par 38 وَيَكُونُونَ لِي شَعْباً وَأَنَا أَكُونُ لَهُمْ إِلَهاً.
\par 39 وَأُعْطِيهِمْ قَلْباً وَاحِداً وَطَرِيقاً وَاحِداً لِيَخَافُونِي كُلَّ الأَيَّامِ لِخَيْرِهِمْ وَخَيْرِ أَوْلاَدِهِمْ بَعْدَهُمْ.
\par 40 وَأَقْطَعُ لَهُمْ عَهْداً أَبَدِيّاً أَنِّي لاَ أَرْجِعُ عَنْهُمْ لأُحْسِنَ إِلَيْهِمْ وَأَجْعَلُ مَخَافَتِي فِي قُلُوبِهِمْ فَلاَ يَحِيدُونَ عَنِّي.
\par 41 وَأَفْرَحُ بِهِمْ لأُحْسِنَ إِلَيْهِمْ وَأَغْرِسَهُمْ فِي هَذِهِ الأَرْضِ بِالأَمَانَةِ بِكُلِّ قَلْبِي وَبِكُلِّ نَفْسِي.
\par 42 لأَنَّهُ هَكَذَا قَالَ الرَّبُّ. كَمَا جَلَبْتُ عَلَى هَذَا الشَّعْبِ كُلَّ هَذَا الشَّرِّ الْعَظِيمِ هَكَذَا أَجْلِبُ أَنَا عَلَيْهِمْ كُلَّ الْخَيْرِ الَّذِي تَكَلَّمْتُ بِهِ إِلَيْهِمْ.
\par 43 فَتُشْتَرَى الْحُقُولُ فِي هَذِهِ الأَرْضِ الَّتِي تَقُولُونَ إِنَّهَا خَرِبَةٌ بِلاَ إِنْسَانٍ وَبِلاَ حَيَوَانٍ وَقَدْ دُفِعَتْ لِيَدِ الْكِلْدَانِيِّينَ.
\par 44 يَشْتَرُونَ الْحُقُولَ بِفِضَّةٍ وَيَكْتُبُونَ ذَلِكَ فِي صُكُوكٍ وَيَخْتِمُونَ وَيُشْهِدُونَ شُهُوداً فِي أَرْضِ بِنْيَامِينَ وَحَوَالَيْ أُورُشَلِيمَ وَفِي مُدُنِ يَهُوذَا وَمُدُنِ الْجَبَلِ وَمُدُنِ السَّهْلِ وَمُدُنِ الْجَنُوبِ لأَنِّي أَرُدُّ سَبْيَهُمْ يَقُولُ الرَّبُّ].

\chapter{33}

\par 1 ثُمَّ صَارَتْ كَلِمَةُ الرَّبِّ إِلَى إِرْمِيَا ثَانِيَةً وَهُوَ مَحْبُوسٌ بَعْدُ فِي دَارِ السِّجْنِ:
\par 2 [هَكَذَا قَالَ الرَّبُّ صَانِعُهَا الرَّبُّ مُصَوِّرُهَا لِيُثَبِّتَهَا يَهْوَهُ اسْمُهُ:
\par 3 اُدْعُنِي فَأُجِيبَكَ وَأُخْبِرَكَ بِعَظَائِمَ وَعَوَائِصَ لَمْ تَعْرِفْهَا.
\par 4 لأَنَّهُ هَكَذَا قَالَ الرَّبُّ إِلَهُ إِسْرَائِيلَ عَنْ بُيُوتِ هَذِهِ الْمَدِينَةِ وَعَنْ بُيُوتِ مُلُوكِ يَهُوذَا الَّتِي هُدِمَتْ لِلْمَتَارِيسِ وَالْمَجَانِيقِ:
\par 5 يَأْتُونَ لِيُحَارِبُوا الْكِلْدَانِيِّينَ وَيَمْلَأُوهَا مِنْ جِيَفِ النَّاسِ الَّذِينَ ضَرَبْتُهُمْ بِغَضَبِي وَغَيْظِي وَالَّذِينَ سَتَرْتُ وَجْهِي عَنْ هَذِهِ الْمَدِينَةِ لأَجْلِ كُلِّ شَرِّهِمْ.
\par 6 هَئَنَذَا أَضَعُ عَلَيْهَا رِفَادَةً وَعِلاَجاً وَأَشْفِيهِمْ وَأُعْلِنُ لَهُمْ كَثْرَةَ السَّلاَمِ وَالأَمَانَةِ.
\par 7 وَأَرُدُّ سَبْيَ يَهُوذَا وَسَبْيَ إِسْرَائِيلَ وَأَبْنِيهِمْ كَالأَوَّلِ.
\par 8 وَأُطَهِّرُهُمْ مِنْ كُلِّ إِثْمِهِمِ الَّذِي أَخْطَأُوا بِهِ إِلَيَّ وَأَغْفِرُ كُلَّ ذُنُوبِهِمِ الَّتِي أَخْطَأُوا بِهَا إِلَيَّ وَالَّتِي عَصُوا بِهَا عَلَيَّ.
\par 9 فَتَكُونُ لِيَ اسْمَ فَرَحٍ لِلتَّسْبِيحِ وَلِلزِّينَةِ لَدَى كُلِّ أُمَمِ الأَرْضِ الَّذِينَ يَسْمَعُونَ بِكُلِّ الْخَيْرِ الَّذِي أَصْنَعُهُ مَعَهُمْ فَيَخَافُونَ وَيَرْتَعِدُونَ مِنْ أَجْلِ كُلِّ الْخَيْرِ وَمِنْ أَجْلِ كُلِّ السَّلاَمِ الَّذِي أَصْنَعُهُ لَهَا.
\par 10 هَكَذَا قَالَ الرَّبُّ: سَيُسْمَعُ بَعْدُ فِي هَذَا الْمَوْضِعِ الَّذِي تَقُولُونَ إِنَّهُ خَرِبٌ بِلاَ إِنْسَانٍ وَبِلاَ حَيَوَانٍ فِي مُدُنِ يَهُوذَا وَفِي شَوَارِعِ أُورُشَلِيمَ الْخَرِبَةِ بِلاَ إِنْسَانٍ وَلاَ سَاكِنٍ وَلاَ بَهِيمَةٍ
\par 11 صَوْتُ الطَّرَبِ وَصَوْتُ الْفَرَحِ صَوْتُ الْعَرِيسِ وَصَوْتُ الْعَرُوسِ صَوْتُ الْقَائِلِينَ: احْمَدُوا رَبَّ الْجُنُودِ لأَنَّ الرَّبَّ صَالِحٌ لأَنَّ إِلَى الأَبَدِ رَحْمَتَهُ. صَوْتُ الَّذِينَ يَأْتُونَ بِذَبِيحَةِ الشُّكْرِ إِلَى بَيْتِ الرَّبِّ لأَنِّي أَرُدُّ سَبْيَ الأَرْضِ كَالأَوَّلِ يَقُولُ الرَّبُّ.
\par 12 هَكَذَا قَالَ رَبُّ الْجُنُودِ: سَيَكُونُ بَعْدُ فِي هَذَا الْمَوْضِعِ الْخَرِبِ بِلاَ إِنْسَانٍ وَلاَ بَهِيمَةٍ وَفِي كُلِّ مُدُنِهِ مَسْكَنُ الرُّعَاةِ الْمُرْبِضِينَ الْغَنَمَ.
\par 13 فِي مُدُنِ الْجَبَلِ وَمُدُنِ السَّهْلِ وَمُدُنِ الْجَنُوبِ وَفِي أَرْضِ بِنْيَامِينَ وَحَوَالَيْ أُورُشَلِيمَ وَفِي مُدُنِ يَهُوذَا تَمُرُّ أَيْضاً الْغَنَمُ تَحْتَ يَدَيِ الْمُحْصِي يَقُولُ الرَّبُّ.
\par 14 [هَا أَيَّامٌ تَأْتِي يَقُولُ الرَّبُّ وَأُقِيمُ الْكَلِمَةَ الصَّالِحَةَ الَّتِي تَكَلَّمْتُ بِهَا إِلَى بَيْتِ إِسْرَائِيلَ وَإِلَى بَيْتِ يَهُوذَا.
\par 15 فِي تِلْكَ الأَيَّامِ وَفِي ذَلِكَ الزَّمَانِ أُنْبِتُ لِدَاوُدَ غُصْنَ الْبِرِّ فَيُجْرِي عَدْلاً وَبِرّاً فِي الأَرْضِ.
\par 16 فِي تِلْكَ الأَيَّامِ يَخْلُصُ يَهُوذَا وَتَسْكُنُ أُورُشَلِيمُ آمِنَةً وَهَذَا مَا تَتَسَمَّى بِهِ [الرَّبُّ بِرُّنَا].
\par 17 لأَنَّهُ هَكَذَا قَالَ الرَّبُّ: لاَ يَنْقَطِعُ لِدَاوُدَ إِنْسَانٌ يَجْلِسُ عَلَى كُرْسِيِّ بَيْتِ إِسْرَائِيلَ
\par 18 وَلاَ يَنْقَطِعُ لِلْكَهَنَةِ اللاَّوِيِّينَ إِنْسَانٌ مِنْ أَمَامِي يُصْعِدُ مُحْرَقَةً وَيُحْرِقُ تَقْدِمَةً وَيُهَيِّئُ ذَبِيحَةً كُلَّ الأَيَّامِ].
\par 19 ثُمَّ صَارَتْ كَلِمَةُ الرَّبِّ إِلَى إِرْمِيَا:
\par 20 [هَكَذَا قَالَ الرَّبُّ: إِنْ نَقَضْتُمْ عَهْدِي مَعَ النَّهَارِ وَعَهْدِي مَعَ اللَّيْلِ حَتَّى لاَ يَكُونَ نَهَارٌ وَلاَ لَيْلٌ فِي وَقْتِهِمَا
\par 21 فَإِنَّ عَهْدِي أَيْضاً مَعَ دَاوُدَ عَبْدِي يُنْقَضُ فَلاَ يَكُونُ لَهُ ابْنٌ مَالِكاً عَلَى كُرْسِيِّهِ وَمَعَ اللاَّوِيِّينَ الْكَهَنَةِ خَادِمِيَّ.
\par 22 كَمَا أَنَّ جُنْدَ السَّمَاوَاتِ لاَ يُعَدُّ وَرَمْلَ الْبَحْرِ لاَ يُحْصَى هَكَذَا أُكَثِّرُ نَسْلَ دَاوُدَ عَبْدِي وَاللاَّوِيِّينَ خَادِمِيَّ].
\par 23 ثُمَّ صَارَتْ كَلِمَةُ الرَّبِّ إِلَى إِرْمِيَا:
\par 24 [أَمَا تَرَى مَا تَكَلَّمَ بِهِ هَذَا الشَّعْبُ: إِنَّ الْعَشِيرَتَيْنِ اللَّتَيْنِ اخْتَارَهُمَا الرَّبُّ قَدْ رَفَضَهُمَا. فَقَدِ احْتَقَرُوا شَعْبِي حَتَّى لاَ يَكُونُوا بَعْدُ أُمَّةً أَمَامَهُمْ.
\par 25 هَكَذَا قَالَ الرَّبُّ: إِنْ كُنْتُ لَمْ أَجْعَلْ عَهْدِي مَعَ النَّهَارِ وَاللَّيْلِ فَرَائِضَ السَّمَاوَاتِ وَالأَرْضِ
\par 26 فَإِنِّي أَيْضاً أَرْفُضُ نَسْلَ يَعْقُوبَ وَدَاوُدَ عَبْدِي فَلاَ آخُذُ مِنْ نَسْلِهِ حُكَّاماً لِنَسْلِ إِبْرَاهِيمَ وَإِسْحَاقَ وَيَعْقُوبَ لأَنِّي أَرُدُّ سَبْيَهُمْ وَأَرْحَمُهُمْ].

\chapter{34}

\par 1 اَلْكَلِمَةُ الَّتِي صَارَتْ إِلَى إِرْمِيَا مِنْ الرَّبِّ حِينَ كَانَ نَبُوخَذْنَصَّرُ مَلِكُ بَابِلَ وَكُلُّ جَيْشِهِ وَكُلُّ مَمَالِكِ أَرَاضِي سُلْطَانِ يَدِهِ وَكُلُّ الشُّعُوبِ يُحَارِبُونَ أُورُشَلِيمَ وَكُلُّ مُدُنِهَا:
\par 2 [هَكَذَا قَالَ الرَّبُّ إِلَهُ إِسْرَائِيلَ: اذْهَبْ وَقُلْ لِصِدْقِيَّا مَلِكِ يَهُوذَا: هَكَذَا قَالَ الرَّبُّ: هَئَنَذَا أَدْفَعُ هَذِهِ الْمَدِينَةَ لِيَدِ مَلِكِ بَابِلَ فَيُحْرِقُهَا بِالنَّارِ.
\par 3 وَأَنْتَ لاَ تُفْلِتُ مِنْ يَدِهِ بَلْ تُمْسَكُ إِمْسَاكاً وَتُدْفَعُ لِيَدِهِ وَتَرَى عَيْنَاكَ عَيْنَيْ مَلِكِ بَابِلَ وَتُكَلِّمُهُ فَماً لِفَمٍ وَتَذْهَبُ إِلَى بَابِلَ.
\par 4 وَلَكِنِ اسْمَعْ كَلِمَةَ الرَّبِّ يَا صِدْقِيَّا مَلِكَ يَهُوذَا. هَكَذَا قَالَ الرَّبُّ مِنْ جِهَتِكَ: لاَ تَمُوتُ بِالسَّيْفِ.
\par 5 بِسَلاَمٍ تَمُوتُ وَبِإِحْرَاقِ آبَائِكَ الْمُلُوكِ الأَوَّلِينَ الَّذِينَ كَانُوا قَبْلَكَ هَكَذَا يُحْرِقُونَ لَكَ وَيَنْدُبُونَكَ قَائِلِينَ: آهِ يَا سَيِّدُ. لأَنِّي أَنَا تَكَلَّمْتُ بِالْكَلِمَةِ يَقُولُ الرَّبُّ].
\par 6 فَكَلَّمَ إِرْمِيَا النَّبِيُّ صِدْقِيَّا مَلِكَ يَهُوذَا بِكُلِّ هَذَا الْكَلاَمِ فِي أُورُشَلِيمَ
\par 7 إِذْ كَانَ جَيْشُ مَلِكِ بَابِلَ يُحَارِبُ أُورُشَلِيمَ وَكُلَّ مُدُنِ يَهُوذَا الْبَاقِيَةِ: لَخِيشَ وَعَزِيقَةَ. لأَنَّ هَاتَيْنِ بَقِيَتَا فِي مُدُنِ يَهُوذَا مَدِينَتَيْنِ حَصِينَتَيْنِ.
\par 8 الْكَلِمَةُ الَّتِي صَارَتْ إِلَى إِرْمِيَا مِنْ الرَّبِّ بَعْدَ قَطْعِ الْمَلِكِ صِدْقِيَّا عَهْداً مَعَ كُلِّ الشَّعْبِ الَّذِي فِي أُورُشَلِيمَ لِيُنَادُوا بِالْعِتْقِ
\par 9 أَنْ يُطْلِقَ كُلُّ وَاحِدٍ عَبْدَهُ وَكُلُّ وَاحِدٍ أَمَتَهُ الْعِبْرَانِيَّ وَالْعِبْرَانِيَّةَ حُرَّيْنِ حَتَّى لاَ يَسْتَعْبِدَهُمَا (أَيْ أَخَوَيْهِ الْيَهُودِيَّيْنِ) أَحَدٌ.
\par 10 فَلَمَّا سَمِعَ كُلُّ الرُّؤَسَاءِ وَكُلُّ الشَّعْبِ الَّذِينَ دَخَلُوا فِي الْعَهْدِ أَنْ يُطْلِقُوا كُلُّ وَاحِدٍ عَبْدَهُ وَكُلُّ وَاحِدٍ أَمَتَهُ حُرَّيْنِ وَلاَ يَسْتَعْبِدُوهُمَا بَعْدُ أَطَاعُوا وَأَطْلَقُوا.
\par 11 وَلَكِنَّهُمْ عَادُوا بَعْدَ ذَلِكَ فَأَرْجَعُوا الْعَبِيدَ وَالإِمَاءَ الَّذِينَ أَطْلَقُوهُمْ أَحْرَاراً وَأَخْضَعُوهُمْ عَبِيداً وَإِمَاءً.
\par 12 فَصَارَتْ كَلِمَةُ الرَّبِّ إِلَى إِرْمِيَا:
\par 13 [هَكَذَا قَالَ الرَّبُّ إِلَهُ إِسْرَائِيلَ: أَنَا قَطَعْتُ عَهْداً مَعَ آبَائِكُمْ يَوْمَ أَخْرَجْتُهُمْ مِنْ أَرْضِ مِصْرَ مِنْ بَيْتِ الْعَبِيدِ قَائِلاً:
\par 14 فِي نِهَايَةِ سَبْعِ سِنِينَ تُطْلِقُونَ كُلُّ وَاحِدٍ أَخَاهُ الْعِبْرَانِيَّ الَّذِي بِيعَ لَكَ وَخَدَمَكَ سِتَّ سِنِينَ فَتُطْلِقُهُ حُرّاً مِنْ عِنْدِكَ. وَلَكِنْ لَمْ يَسْمَعْ آبَاؤُكُمْ لِي وَلاَ أَمَالُوا أُذُنَهُمْ.
\par 15 وَقَدْ رَجَعْتُمْ أَنْتُمُ الْيَوْمَ وَفَعَلْتُمْ مَا هُوَ مُسْتَقِيمٌ فِي عَيْنَيَّ مُنَادِينَ بِالْعِتْقِ كُلُّ وَاحِدٍ إِلَى صَاحِبِهِ وَقَطَعْتُمْ عَهْداً أَمَامِي فِي الْبَيْتِ الَّذِي دُعِيَ بِاسْمِي.
\par 16 ثُمَّ عُدْتُمْ وَدَنَّسْتُمُ اسْمِي وَأَرْجَعْتُمْ كُلُّ وَاحِدٍ عَبْدَهُ وَكُلُّ وَاحِدٍ أَمَتَهُ الَّذِينَ أَطْلَقْتُمُوهُم أَحْرَاراً لأَنْفُسِهِمْ وَأَخْضَعْتُمُوهُمْ لِيَكُونُوا لَكُمْ عَبِيداً وَإِمَاءً.
\par 17 لِذَلِكَ هَكَذَا قَالَ الرَّبُّ: أَنْتُمْ لَمْ تَسْمَعُوا لِي لِتُنَادُوا بِالْعِتْقِ كُلُّ وَاحِدٍ إِلَى أَخِيهِ وَكُلُّ وَاحِدٍ إِلَى صَاحِبِهِ. هَئَنَذَا أُنَادِي لَكُمْ بِالْعِتْقِ يَقُولُ الرَّبُّ لِلسَّيْفِ وَالْوَبَإِ وَالْجُوعِ وَأَجْعَلُكُمْ قَلَقاً لِكُلِّ مَمَالِكِ الأَرْضِ.
\par 18 وَأَدْفَعُ النَّاسَ الَّذِينَ تَعَدُّوا عَهْدِي الَّذِينَ لَمْ يُقِيمُوا كَلاَمَ الْعَهْدِ الَّذِي قَطَعُوهُ أَمَامِي. الْعِجْلَ الَّذِي قَطَعُوهُ إِلَى اثْنَيْنِ وَجَازُوا بَيْنَ قِطْعَتَيْهِ.
\par 19 رُؤَسَاءَ يَهُوذَا وَرُؤَسَاءَ أُورُشَلِيمَ الْخِصْيَانَ وَالْكَهَنَةَ وَكُلَّ شَعْبِ الأَرْضِ الَّذِينَ جَازُوا بَيْنَ قِطْعَتَيِ الْعِجْلِ
\par 20 أَدْفَعُهُمْ لِيَدِ أَعْدَائِهِمْ وَلِيَدِ طَالِبِي نُفُوسِهِمْ فَتَكُونُ جُثَثُهُمْ أَكْلاً لِطُيُورِ السَّمَاءِ وَوُحُوشِ الأَرْضِ.
\par 21 وَأَدْفَعُ صِدْقِيَّا مَلِكَ يَهُوذَا وَرُؤَسَاءَهُ لِيَدِ أَعْدَائِهِمْ وَلِيَدِ طَالِبِي نُفُوسِهِمْ وَلِيَدِ جَيْشِ مَلِكِ بَابَِلَ الَّذِينَ صَعِدُوا عَنْكُمْ.
\par 22 هَئَنَذَا آمُرُ يَقُولُ الرَّبُّ وَأَرُدُّهُمْ إِلَى هَذِهِ الْمَدِينَةِ فَيُحَارِبُونَهَا وَيَأْخُذُونَهَا وَيُحْرِقُونَهَا بِالنَّارِ وَأَجْعَلُ مُدُنَ يَهُوذَا خَرِبَةً بِلاَ سَاكِنٍ].

\chapter{35}

\par 1 اَلْكَلِمَةُ الَّتِي صَارَتْ إِلَى إِرْمِيَا مِنْ الرَّبِّ فِي أَيَّامِ يَهُويَاقِيمَ بْنِ يُوشِيَّا مَلِكِ يَهُوذَا:
\par 2 [اِذْهَبْ إِلَى بَيْتِ الرَّكَابِيِّينَ وَكَلِّمْهُمْ وَادْخُلْ بِهِمْ إِلَى بَيْتِ الرَّبِّ إِلَى أَحَدِ الْمَخَادِعِ وَاسْقِهِمْ خَمْراً].
\par 3 فَأَخَذْتُ يَزَنْيَا بْنَ إِرْمِيَا بْنِ حَبْصِينِيَا وَإِخْوَتَهُ وَكُلَّ بَنِيهِ وَكُلَّ بَيْتِ الرَّكَابِيِّينَ
\par 4 وَدَخَلْتُ بِهِمْ إِلَى بَيْتِ الرَّبِّ إِلَى مِخْدَعِ بَنِي حَانَانَ بْنِ يَجَدْلِيَا رَجُلِ اللَّهِ الَّذِي بِجَانِبِ مِخْدَعِ الرُّؤَسَاءِ الَّذِي فَوْقَ مِخْدَعِ مَعْسِيَّا بْنِ شَلُّومَ حَارِسِ الْبَابِ.
\par 5 وَجَعَلْتُ أَمَامَ بَنِي بَيْتِ الرَّكَابِيِّينَ طَاسَاتٍ مَلآنَةً خَمْراً وَأَقْدَاحاً وَقُلْتُ لَهُمُ: [اشْرَبُوا خَمْراً].
\par 6 فَقَالُوا: [لاَ نَشْرَبُ خَمْراً لأَنَّ يُونَادَابَ بْنَ رَكَابَ أَبَانَا أَوْصَانَا قَائِلاً: لاَ تَشْرَبُوا خَمْراً أَنْتُمْ وَلاَ بَنُوكُمْ إِلَى الأَبَدِ.
\par 7 وَلاَ تَبْنُوا بَيْتاً وَلاَ تَزْرَعُوا زَرْعاً وَلاَ تَغْرِسُوا كَرْماً وَلاَ تَكُنْ لَكُمْ بَلِ اسْكُنُوا فِي الْخِيَامِ كُلَّ أَيَّامِكُمْ لِتَحْيُوا أَيَّاماً كَثِيرَةً عَلَى وَجْهِ الأَرْضِ الَّتِي أَنْتُمْ مُتَغَرِّبُونَ فِيهَا.
\par 8 فَسَمِعْنَا لِصَوْتِ يُونَادَابَ بْنِ رَكَابَ أَبِينَا فِي كُلِّ مَا أَوْصَانَا بِهِ أَنْ لاَ نَشْرَبَ خَمْراً كُلَّ أَيَّامِنَا نَحْنُ وَنِسَاؤُنَا وَبَنُونَا وَبَنَاتُنَا
\par 9 وَأَنْ لاَ نَبْنِيَ بُيُوتاً لِسُكْنَانَا وَأَنْ لاَ يَكُونَ لَنَا كَرْمٌ وَلاَ حَقْلٌ وَلاَ زَرْعٌ.
\par 10 فَسَكَنَّا فِي الْخِيَامِ وَسَمِعْنَا وَعَمِلْنَا حَسَبَ كُلِّ مَا أَوْصَانَا بِهِ يُونَادَابُ أَبُونَا.
\par 11 وَلَكِنْ كَانَ لَمَّا صَعِدَ نَبُوخَذْنَصَّرُ مَلِكُ بَابِلَ إِلَى الأَرْضِ أَنَّنَا قُلْنَا هَلُمَّ فَنَدْخُلُ إِلَى أُورُشَلِيمَ مِنْ وَجْهِ جَيْشِ الْكِلْدَانِيِّينَ وَمِنْ وَجْهِ جَيْشِ الأَرَامِيِّينَ. فَسَكَنَّا فِي أُورُشَلِيمَ].
\par 12 ثُمَّ صَارَتْ كَلِمَةُ الرَّبِّ إِلَى إِرْمِيَا:
\par 13 [هَكَذَا قَالَ رَبُّ الْجُنُودِ إِلَهُ إِسْرَائِيلَ: اذْهَبْ وَقُلْ لِرِجَالِ يَهُوذَا وَسُكَّانِ أُورُشَلِيمَ: أَمَا تَقْبَلُونَ تَأْدِيباً لِتَسْمَعُوا كَلاَمِي يَقُولُ الرَّبُّ؟
\par 14 قَدْ أُقِيمَ كَلاَمُ يُونَادَابَ بْنِ رَكَابَ الَّذِي أَوْصَى بِهِ بَنِيهِ أَنْ لاَ يَشْرَبُوا خَمْراً فَلَمْ يَشْرَبُوا إِلَى هَذَا الْيَوْمِ لأَنَّهُمْ سَمِعُوا وَصِيَّةَ أَبِيهِمْ. وَأَنَا قَدْ كَلَّمْتُكُمْ مُبَكِّراً وَمُكَلِّماً وَلَمْ تَسْمَعُوا لِي.
\par 15 وَقَدْ أَرْسَلْتُ إِلَيْكُمْ كُلَّ عَبِيدِي الأَنْبِيَاءِ مُبَكِّراً وَمُرْسِلاً قَائِلاً: ارْجِعُوا كُلُّ وَاحِدٍ عَنْ طَرِيقِهِ الرَّدِيئَةِ وَأَصْلِحُوا أَعْمَالَكُمْ وَلاَ تَذْهَبُوا وَرَاءَ آلِهَةٍ أُخْرَى لِتَعْبُدُوهَا فَتَسْكُنُوا فِي الأَرْضِ الَّتِي أَعْطَيْتُكُمْ وَآبَاءَكُمْ. فَلَمْ تُمِيلُوا أُذُنَكُمْ وَلاَ سَمِعْتُمْ لِي.
\par 16 لأَنَّ بَنِي يُونَادَابَ بْنِ رَكَابَ قَدْ أَقَامُوا وَصِيَّةَ أَبِيهِمِ الَّتِي أَوْصَاهُمْ بِهَا. أَمَّا هَذَا الشَّعْبُ فَلَمْ يَسْمَعْ لِي
\par 17 لِذَلِكَ هَكَذَا قَالَ الرَّبُّ إِلَهُ الْجُنُودِ إِلَهُ إِسْرَائِيلَ: هَئَنَذَا أَجْلِبُ عَلَى يَهُوذَا وَعَلَى كُلِّ سُكَّانِ أُورُشَلِيمَ كُلَّ الشَّرِّ الَّذِي تَكَلَّمْتُ بِهِ عَلَيْهِمْ لأَنِّي كَلَّمْتُهُمْ فَلَمْ يَسْمَعُوا وَدَعَوْتُهُمْ فَلَمْ يُجِيبُوا].
\par 18 وَقَالَ إِرْمِيَا لِبَيْتِ الرَّكَابِيِّينَ: [هَكَذَا قَالَ رَبُّ الْجُنُودِ إِلَهُ إِسْرَائِيلَ. مِنْ أَجْلِ أَنَّكُمْ سَمِعْتُمْ لِوَصِيَّةِ يُونَادَابَ أَبِيكُمْ وَحَفِظْتُمْ كُلَّ وَصَايَاهُ وَعَمِلْتُمْ حَسَبَ كُلِّ مَا أَوْصَاكُمْ بِهِ
\par 19 لِذَلِكَ هَكَذَا قَالَ رَبُّ الْجُنُودِ إِلَهُ إِسْرَائِيلَ: لاَ يَنْقَطِعُ لِيُونَادَابَ بْنِ رَكَابَ إِنْسَانٌ يَقِفُ أَمَامِي كُلَّ الأَيَّامِ].

\chapter{36}

\par 1 وَكَانَ فِي السَّنَةِ الرَّابِعَةِ لِيَهُويَاقِيمَ بْنِ يُوشِيَّا مَلِكِ يَهُوذَا أَنَّ هَذِهِ الْكَلِمَةَ صَارَتْ إِلَى إِرْمِيَا مِنْ الرَّبِّ:
\par 2 [خُذْ لِنَفْسِكَ دَرْجَ سِفْرٍ وَاكْتُبْ فِيهِ كُلَّ الْكَلاَمِ الَّذِي كَلَّمْتُكَ بِهِ عَلَى إِسْرَائِيلَ وَعَلَى يَهُوذَا وَعَلَى كُلِّ الشُّعُوبِ مِنَ الْيَوْمِ الَّذِي كَلَّمْتُكَ فِيهِ مِنْ أَيَّامِ يُوشِيَّا إِلَى هَذَا الْيَوْمِ.
\par 3 لَعَلَّ بَيْتَ يَهُوذَا يَسْمَعُونَ كُلَّ الشَّرِّ الَّذِي أَنَا مُفَكِّرٌ أَنْ أَصْنَعَهُ بِهِمْ فَيَرْجِعُوا كُلُّ وَاحِدٍ عَنْ طَرِيقِهِ الرَّدِيءِ فَأَغْفِرَ ذَنْبَهُمْ وَخَطِيَّتَهُمْ].
\par 4 فَدَعَا إِرْمِيَا بَارُوخَ بْنَ نِيرِيَّا فَكَتَبَ بَارُوخُ عَنْ فَمِ إِرْمِيَا كُلَّ كَلاَمِ الرَّبِّ الَّذِي كَلَّمَهُ بِهِ فِي دَرْجِ السِّفْرِ.
\par 5 وَأَوْصَى إِرْمِيَا بَارُوخَ قَائِلاً: [أَنَا مَحْبُوسٌ لاَ أَقْدِرُ أَنْ أَدْخُلَ بَيْتَ الرَّبِّ.
\par 6 فَادْخُلْ أَنْتَ وَاقْرَأْ فِي الدَّرْجِ الَّذِي كَتَبْتَ عَنْ فَمِي كُلَّ كَلاَمِ الرَّبِّ فِي آذَانِ الشَّعْبِ فِي بَيْتِ الرَّبِّ فِي يَوْمِ الصَّوْمِ وَاقْرَأْهُ أَيْضاً فِي آذَانِ كُلِّ يَهُوذَا الْقَادِمِينَ مِنْ مُدُنِهِمْ.
\par 7 لَعَلَّ تَضَرُّعَهُمْ يَقَعُ أَمَامَ الرَّبِّ فَيَرْجِعُوا كُلُّ وَاحِدٍ عَنْ طَرِيقِهِ الرَّدِيءِ لأَنَّهُ عَظِيمٌ الْغَضَبُ وَالْغَيْظُ اللَّذَانِ تَكَلَّمَ بِهِمَا الرَّبُّ عَلَى هَذَا الشَّعْبِ].
\par 8 فَفَعَلَ بَارُوخُ بْنُ نِيرِيَّا حَسَبَ كُلِّ مَا أَوْصَاهُ بِهِ إِرْمِيَا النَّبِيُّ بِقِرَاءَتِهِ فِي السِّفْرِ كَلاَمَ الرَّبِّ فِي بَيْتِ الرَّبِّ.
\par 9 وَكَانَ فِي السَّنَةِ الْخَامِسَةِ لِيَهُويَاقِيمَ بْنِ يُوشِيَّا مَلِكِ يَهُوذَا فِي الشَّهْرِ التَّاسِعِ أَنَّهُمْ نَادُوا لِصَوْمٍ أَمَامَ الرَّبِّ كُلَّ الشَّعْبِ فِي أُورُشَلِيمَ وَكُلَّ الشَّعْبِ الْقَادِمِينَ مِنْ مُدُنِ يَهُوذَا إِلَى أُورُشَلِيمَ.
\par 10 فَقَرَأَ بَارُوخُ فِي السِّفْرِ كَلاَمَ إِرْمِيَا فِي بَيْتِ الرَّبِّ فِي مِخْدَعِ جَمَرْيَا بْنِ شَافَانَ الْكَاتِبِ فِي الدَّارِ الْعُلْيَا فِي مَدْخَلِ بَابِ بَيْتِ الرَّبِّ الْجَدِيدِ فِي آذَانِ كُلِّ الشَّعْبِ.
\par 11 فَلَمَّا سَمِعَ مِيخَايَا بْنُ جَمَرْيَا بْنِ شَافَانَ كُلَّ كَلاَمِ الرَّبِّ مِنَ السِّفْرِ
\par 12 نَزَلَ إِلَى بَيْتِ الْمَلِكِ إِلَى مِخْدَعِ الْكَاتِبِ وَإِذَا كُلُّ الرُّؤَسَاءِ جُلُوسٌ هُنَاكَ. أَلِيشَامَاعُ الْكَاتِبُ وَدَلاَيَا بْنُ شَمَعْيَا وَأَلْنَاثَانُ بْنُ عَكْبُورَ وَجَمَرْيَا بْنُ شَافَانَ وَصِدْقِيَّا بْنُ حَنَنِيَّا وَكُلُّ الرُّؤَسَاءِ.
\par 13 فَأَخْبَرَهُمْ مِيخَايَا بِكُلِّ الْكَلاَمِ الَّذِي سَمِعَهُ عِنْدَمَا قَرَأَ بَارُوخُ السِّفْرَ فِي آذَانِ الشَّعْبِ.
\par 14 فَأَرْسَلَ كُلُّ الرُّؤَسَاءِ إِلَى بَارُوخَ يَهُودِيَ بْنَ نَثَنْيَا بْنِ شَلَمْيَا بْنِ كُوشِي قَائِلِينَ: [الدَّرْجُ الَّذِي قَرَأْتَ فِيهِ فِي آذَانِ الشَّعْبِ خُذْهُ بِيَدِكَ وَتَعَالَ]. فَأَخَذَ بَارُوخُ بْنُ نِيرِيَّا الدَّرْجَ بِيَدِهِ وَأَتَى إِلَيْهِمْ.
\par 15 فَقَالُوا لَهُ: [اجْلِسْ وَاقْرَأْهُ فِي آذَانِنَا]. فَقَرَأَ بَارُوخُ فِي آذَانِهِمْ.
\par 16 فَكَانَ لَمَّا سَمِعُوا كُلَّ الْكَلاَمِ أَنَّهُمْ خَافُوا نَاظِرِينَ بَعْضُهُمْ إِلَى بَعْضٍ وَقَالُوا لِبَارُوخَ: [إِخْبَاراً نُخْبِرُ الْمَلِكَ بِكُلِّ هَذَا الْكَلاَمِ].
\par 17 ثُمَّ سَأَلُوا بَارُوخَ: [أَخْبِرْنَا كَيْفَ كَتَبْتَ كُلَّ هَذَا الْكَلاَمِ عَنْ فَمِهِ؟]
\par 18 فَقَالَ لَهُمْ بَارُوخُ: [بِفَمِهِ كَانَ يَقْرَأُ لِي كُلَّ هَذَا الْكَلاَمِ وَأَنَا كُنْتُ أَكْتُبُ فِي السِّفْرِ بِالْحِبْرِ].
\par 19 فَقَالَ الرُّؤَسَاءُ لِبَارُوخَ: [اذْهَبْ وَاخْتَبِئْ أَنْتَ وَإِرْمِيَا وَلاَ يَعْلَمُ إِنْسَانٌ أَيْنَ أَنْتُمَا].
\par 20 ثُمَّ دَخَلُوا إِلَى الْمَلِكِ إِلَى الدَّارِ وَأَوْدَعُوا الدَّرْجَ فِي مِخْدَعِ أَلِيشَامَاعَ الْكَاتِبِ وَأَخْبَرُوا فِي أُذُنَيِ الْمَلِكِ بِكُلِّ الْكَلاَمِ.
\par 21 فَأَرْسَلَ الْمَلِكُ يَهُودِيَ لِيَأْخُذَ الدَّرْجَ فَأَخَذَهُ مِنْ مِخْدَعِ أَلِيشَامَاعَ الْكَاتِبِ وَقَرَأَهُ يَهُودِي فِي أُذُنَيِ الْمَلِكِ وَفِي آذَانِ كُلِّ الرُّؤَسَاءِ الْوَاقِفِينَ لَدَى الْمَلِكِ.
\par 22 وَكَانَ الْمَلِكُ جَالِساً فِي بَيْتِ الشِّتَاءِ فِي الشَّهْرِ التَّاسِعِ وَالْكَانُونُ قُدَّامَهُ مُتَّقِدٌ.
\par 23 وَكَانَ لَمَّا قَرَأَ يَهُودِي ثَلاَثَةَ شُطُورٍ أَوْ أَرْبَعَةً أَنَّ الْمَلِكْ شَقَّهُ بِمِبْرَاةِ الْكَاتِبِ وَأَلْقَاهُ إِلَى النَّارِ الَّتِي فِي الْكَانُونِ حَتَّى فَنِيَ كُلُّ الدَّرْجِ فِي النَّارِ الَّتِي فِي الْكَانُونِ.
\par 24 وَلَمْ يَخَفِ الْمَلِكُ وَلاَ كُلُّ عَبِيدِهِ السَّامِعِينَ كُلَّ هَذَا الْكَلاَمِ وَلاَ شَقَّقُوا ثِيَابَهُمْ.
\par 25 وَلَكِنَّ أَلْنَاثَانَ وَدَلاَيَا وَجَمَرْيَا تَرَجُّوا الْمَلِكَ أَنْ لاَ يُحْرِقَ الدَّرْجَ فَلَمْ يَسْمَعْ لَهُمْ.
\par 26 بَلْ أَمَرَ الْمَلِكُ يَرْحَمْئِيلَ ابْنَ الْمَلِكِ وَسَرَايَا بْنَ عَزَرْئِيلَ وَشَلَمْيَا بْنَ عَبْدِئِيلَ أَنْ يَقْبِضُوا عَلَى بَارُوخَ الْكَاتِبِ وَإِرْمِيَا النَّبِيِّ وَلَكِنَّ الرَّبَّ خَبَّأَهُمَا.
\par 27 ثُمَّ صَارَتْ كَلِمَةُ الرَّبِّ إِلَى إِرْمِيَا بَعْدَ إِحْرَاقِ الْمَلِكِ الدَّرْجَ وَالْكَلاَمَ الَّذِي كَتَبَهُ بَارُوخُ عَنْ فَمِ إِرْمِيَا:
\par 28 [عُدْ فَخُذْ لِنَفْسِكَ دَرْجاً آخَرَ وَاكْتُبْ فِيهِ كُلَّ الْكَلاَمِ الأَوَّلِ الَّذِي كَانَ فِي الدَّرْجِ الأَوَّلِ الَّذِي أَحْرَقَهُ يَهُويَاقِيمُ مَلِكُ يَهُوذَا
\par 29 وَقُلْ لِيَهُويَاقِيمَ مَلِكِ يَهُوذَا: هَكَذَا قَالَ الرَّبُّ: أَنْتَ قَدْ أَحْرَقْتَ ذَلِكَ الدَّرْجَ قَائِلاً: لِمَاذَا كَتَبْتَ فِيهِ: مَجِيئاً يَجِيءُ مَلِكُ بَابِلَ وَيُهْلِكُ هَذِهِ الأَرْضَ وَيُلاَشِي مِنْهَا الإِنْسَانَ وَالْحَيَوَانَ؟
\par 30 لِذَلِكَ هَكَذَا قَالَ الرَّبُّ عَنْ يَهُويَاقِيمَ مَلِكِ يَهُوذَا: لاَ يَكُونُ لَهُ جَالِسٌ عَلَى كُرْسِيِّ دَاوُدَ وَتَكُونُ جُثَّتُهُ مَطْرُوحَةً لِلْحَرِّ نَهَاراً وَلِلْبَرْدِ لَيْلاً.
\par 31 وَأُعَاقِبُهُ وَنَسْلَهُ وَعَبِيدَهُ عَلَى إِثْمِهِمْ وَأَجْلِبُ عَلَيْهِمْ وَعَلَى سُكَّانِ أُورُشَلِيمَ وَعَلَى رِجَالِ يَهُوذَا كُلَّ الشَّرِّ الَّذِي كَلَّمْتُهُمْ عَنْهُ وَلَمْ يَسْمَعُوا].
\par 32 فَأَخَذَ إِرْمِيَا دَرْجاً آخَرَ وَدَفَعَهُ لِبَارُوخَ بْنِ نِيرِيَّا الْكَاتِبِ فَكَتَبَ فِيهِ عَنْ فَمِ إِرْمِيَا كُلَّ كَلاَمِ السِّفْرِ الَّذِي أَحْرَقَهُ يَهُويَاقِيمُ مَلِكُ يَهُوذَا بِالنَّارِ وَزِيدَ عَلَيْهِ أَيْضاً كَلاَمٌ كَثِيرٌ مِثْلُهُ.

\chapter{37}

\par 1 وَمَلَكَ الْمَلِكُ صِدْقِيَّا بْنُ يُوشِيَّا مَكَانَ كُنْيَاهُو بْنِ يَهُويَاقِيمَ الَّذِي مَلَّكَهُ نَبُوخَذْنَصَّرُ مَلِكُ بَابِلَ فِي أَرْضِ يَهُوذَا.
\par 2 وَلَمْ يَسْمَعْ هُوَ وَلاَ عَبِيدُهُ وَلاَ شَعْبُ الأَرْضِ لِكَلاَمِ الرَّبِّ الَّذِي تَكَلَّمَ بِهِ عَنْ يَدِ إِرْمِيَا النَّبِيِّ.
\par 3 وَأَرْسَلَ الْمَلِكُ صِدْقِيَّا يَهُوخَلَ بْنَ شَلَمْيَا وَصَفَنْيَا بْنَ مَعْسِيَّا الْكَاهِنَ إِلَى إِرْمِيَا النَّبِيِّ قَائِلاً: [صَلِّ لأَجْلِنَا إِلَى الرَّبِّ إِلَهِنَا].
\par 4 وَكَانَ إِرْمِيَا يَدْخُلُ وَيَخْرُجُ فِي وَسَطِ الشَّعْبِ إِذْ لَمْ يَكُونُوا قَدْ جَعَلُوهُ فِي بَيْتِ السِّجْنِ.
\par 5 وَخَرَجَ جَيْشُ فِرْعَوْنَ مِنْ مِصْرَ. فَلَمَّا سَمِعَ الْكِلْدَانِيُّونَ الْمُحَاصِرُونَ أُورُشَلِيمَ بِخَبَرِهِمْ صَعِدُوا عَنْ أُورُشَلِيمَ.
\par 6 فَصَارَتْ كَلِمَةُ الرَّبِّ إِلَى إِرْمِيَا النَّبِيِّ:
\par 7 [هَكَذَا قَالَ الرَّبُّ إِلَهُ إِسْرَائِيلَ: هَكَذَا تَقُولُونَ لِمَلِكِ يَهُوذَا الَّذِي أَرْسَلَكُمْ إِلَيَّ لِتَسْتَشِيرُونِي: هَا إِنَّ جَيْشَ فِرْعَوْنَ الْخَارِجَ إِلَيْكُمْ لِمُسَاعَدَتِكُمْ يَرْجِعُ إِلَى أَرْضِهِ إِلَى مِصْرَ.
\par 8 وَيَرْجِعُ الْكِلْدَانِيُّونَ وَيُحَارِبُونَ هَذِهِ الْمَدِينَةَ وَيَأْخُذُونَهَا وَيُحْرِقُونَهَا بِالنَّارِ.
\par 9 هَكَذَا قَالَ الرَّبُّ: لاَ تَخْدَعُوا أَنْفُسَكُمْ قَائِلِينَ إِنَّ الْكِلْدَانِيِّينَ سَيَذْهَبُونَ عَنَّا لأَنَّهُمْ لاَ يَذْهَبُونَ.
\par 10 لأَنَّكُمْ وَإِنْ ضَرَبْتُمْ كُلَّ جَيْشِ الْكِلْدَانِيِّينَ الَّذِينَ يُحَارِبُونَكُمْ وَبَقِيَ مِنْهُمْ رِجَالٌ قَدْ طُعِنُوا فَإِنَّهُمْ يَقُومُونَ كُلُّ وَاحِدٍ فِي خَيْمَتِهِ وَيُحْرِقُونَ هَذِهِ الْمَدِينَةَ بِالنَّارِ].
\par 11 وَكَانَ لَمَّا أُصْعِدَ جَيْشُ الْكِلْدَانِيِّينَ عَنْ أُورُشَلِيمَ مِنْ وَجْهِ جَيْشِ فِرْعَوْنَ
\par 12 أَنَّ إِرْمِيَا خَرَجَ مِنْ أُورُشَلِيمَ لِيَنْطَلِقَ إِلَى أَرْضِ بِنْيَامِينَ لِيَنْسَابَ مِنْ هُنَاكَ فِي وَسَطِ الشَّعْبِ.
\par 13 وَفِيمَا هُوَ فِي بَابِ بِنْيَامِينَ إِذَا هُنَاكَ نَاظِرُ الْحُرَّاسِ اسْمُهُ يَرْئِيَّا بْنُ شَلَمْيَا بْنُ حَنَنِيَّا فَقَبَضَ عَلَى إِرْمِيَا النَّبِيِّ قَائِلاً: [إِنَّكَ تَقَعُ لِلْكِلْدَانِيِّينَ].
\par 14 فَقَالَ إِرْمِيَا: [كَذِبٌ! لاَ أَقَعُ لِلْكِلْدَانِيِّينَ]. وَلَمْ يَسْمَعْ لَهُ فَقَبَضَ يَرْئِيَّا عَلَى إِرْمِيَا وَأَتَى بِهِ إِلَى الرُّؤَسَاءِ.
\par 15 فَغَضِبَ الرُّؤَسَاءُ عَلَى إِرْمِيَا وَضَرَبُوهُ وَجَعَلُوهُ فِي بَيْتِ السِّجْنِ فِي بَيْتِ يُونَاثَانَ الْكَاتِبِ لأَنَّهُمْ جَعَلُوهُ بَيْتَ السِّجْنِ.
\par 16 فَلَمَّا دَخَلَ إِرْمِيَا إِلَى بَيْتِ الْجُبِّ وَإِلَى الْمُقَبَّبَاتِ أَقَامَ إِرْمِيَا هُنَاكَ أَيَّاماً كَثِيرَةً.
\par 17 ثُمَّ أَرْسَلَ الْمَلِكُ صِدْقِيَّا وَأَخَذَهُ وَسَأَلَهُ الْمَلِكُ فِي بَيْتِهِ سِرّاً: [هَلْ تُوجَدُ كَلِمَةٌ مِنْ قِبَلِ الرَّبِّ؟] فَقَالَ إِرْمِيَا: [تُوجَدُ. إِنَّكَ تُدْفَعُ لِيَدِ مَلِكِ بَابِلَ].
\par 18 ثُمَّ قَالَ إِرْمِيَا لِلْمَلِكِ صِدْقِيَّا: [مَا هِيَ خَطِيَّتِي إِلَيْكَ وَإِلَى عَبِيدِكَ وَإِلَى هَذَا الشَّعْبِ حَتَّى جَعَلْتُمُونِي فِي بَيْتِ السِّجْنِ؟
\par 19 فَأَيْنَ أَنْبِيَاؤُكُمُ الَّذِينَ تَنَبَّأُوا لَكُمْ قَائِلِينَ لاَ يَأْتِي مَلِكُ بَابِلَ عَلَيْكُمْ وَلاَ عَلَى هَذِهِ الأَرْضِ.
\par 20 فَالآنَ اسْمَعْ يَا سَيِّدِي الْمَلِكَ. لِيَقَعْ تَضَرُّعِي أَمَامَكَ وَلاَ تَرُدَّنِي إِلَى بَيْتِ يُونَاثَانَ الْكَاتِبِ فَلاَ أَمُوتَ هُنَاكَ].
\par 21 فَأَمَرَ الْمَلِكُ صِدْقِيَّا أَنْ يَضَعُوا إِرْمِيَا فِي دَارِ السِّجْنِ وَأَنْ يُعْطَى رَغِيفَ خُبْزٍ كُلَّ يَوْمٍ مِنْ سُوقِ الْخَبَّازِينَ حَتَّى يَنْفُدَ كُلُّ الْخُبْزِ مِنَ الْمَدِينَةِ. فَأَقَامَ إِرْمِيَا فِي دَارِ السِّجْنِ.

\chapter{38}

\par 1 وَسَمِعَ شَفَطْيَا بْنُ مَتَّانَ وَجَدَلْيَا بْنُ فَشْحُورَ وَيُوخَلُ بْنُ شَلَمْيَا وَفَشْحُورُ بْنُ مَلْكِيَّا الْكَلاَمَ الَّذِي كَانَ إِرْمِيَا يُكَلِّمُ بِهِ كُلَّ الشَّعْبِ قَائِلاً:
\par 2 [هَكَذَا قَالَ الرَّبُّ: الَّذِي يُقِيمُ فِي هَذِهِ الْمَدِينَةِ يَمُوتُ بِالسَّيْفِ وَالْجُوعِ وَالْوَبَإِ. أَمَّا الَّذِي يَخْرُجُ إِلَى الْكِلْدَانِيِّينَ فَإِنَّهُ يَحْيَا وَتَكُونُ لَهُ نَفْسُهُ غَنِيمَةً فَيَحْيَا.
\par 3 هَكَذَا قَالَ الرَّبُّ: هَذِهِ الْمَدِينَةُ سَتُدْفَعُ دَفْعاً لِيَدِ جَيْشِ مَلِكِ بَابِلَ فَيَأْخُذُهَا].
\par 4 فَقَالَ الرُّؤَسَاءُ لِلْمَلِكِ: [لِيُقْتَلْ هَذَا الرَّجُلُ لأَنَّهُ بِذَلِكَ يُضْعِفُ أَيَادِيَ رِجَالِ الْحَرْبِ الْبَاقِينَ فِي هَذِهِ الْمَدِينَةِ وَأَيَادِيَ كُلِّ الشَّعْبِ إِذْ يُكَلِّمُهُمْ بِمِثْلِ هَذَا الْكَلاَمِ. لأَنَّ هَذَا الرَّجُلَ لاَ يَطْلُبُ السَّلاَمَ لِهَذَا الشَّعْبِ بَلِ الشَّرَّ].
\par 5 فَقَالَ الْمَلِكُ صِدْقِيَّا: [هَا هُوَ بِيَدِكُمْ لأَنَّ الْمَلِكَ لاَ يَقْدِرُ عَلَيْكُمْ فِي شَيْءٍ].
\par 6 فَأَخَذُوا إِرْمِيَا وَأَلْقُوهُ فِي جُبِّ مَلْكِيَّا ابْنِ الْمَلِكِ الَّذِي فِي دَارِ السِّجْنِ وَدَلُّوا إِرْمِيَا بِحِبَالٍ. وَلَمْ يَكُنْ فِي الْجُبِّ مَاءٌ بَلْ وَحْلٌ فَغَاصَ إِرْمِيَا فِي الْوَحْلِ.
\par 7 فَلَمَّا سَمِعَ عَبْدَ مَلِكُ الْكُوشِيُّ رَجُلٌ خَصِيٌّ وَهُوَ فِي بَيْتِ الْمَلِكِ أَنَّهُمْ جَعَلُوا إِرْمِيَا فِي الْجُبِّ وَالْمَلِكُ جَالِسٌ فِي بَابِ بِنْيَامِينَ
\par 8 خَرَجَ عَبْدَ مَلِكُ مِنْ بَيْتِ الْمَلِكِ وَقَالَ لِلْمَلِكِ:
\par 9 [يَا سَيِّدِي الْمَلِكَ قَدْ أَسَاءَ هَؤُلاَءِ الرِّجَالُ فِي كُلِّ مَا فَعَلُوا بِإِرْمِيَا النَّبِيِّ الَّذِي طَرَحُوهُ فِي الْجُبِّ فَإِنَّهُ يَمُوتُ فِي مَكَانِهِ بِسَبَبِ الْجُوعِ لأَنَّهُ لَيْسَ بَعْدُ خُبْزٌ فِي الْمَدِينَةِ].
\par 10 فَأَمَرَ الْمَلِكُ عَبْدَ مَلِكَ الْكُوشِيَّ: [خُذْ مَعَكَ مِنْ هُنَا ثَلاَثِينَ رَجُلاً وَأَطْلِعْ إِرْمِيَا مِنَ الْجُبِّ قَبْلَمَا يَمُوتُ].
\par 11 فَأَخَذَ عَبْدَ مَلِكُ الرِّجَالَ مَعَهُ وَدَخَلَ إِلَى بَيْتِ الْمَلِكِ إِلَى أَسْفَلِ الْمَخْزَنِ وَأَخَذَ مِنْ هُنَاكَ ثِيَاباً رَثَّةً وَمَلاَبِسَ بَالِيَةً وَدَلاَّهَا إِلَى إِرْمِيَا إِلَى الْجُبِّ بِحِبَالٍ.
\par 12 وَقَالَ عَبْدَ مَلِكُ الْكُوشِيُّ لإِرْمِيَا: [ضَعِ الثِّيَابَ الرَّثَّةَ وَالْمَلاَبِسَ الْبَالِيَةَ تَحْتَ إِبْطَيْكَ تَحْتَ الْحِبَالِ. فَفَعَلَ إِرْمِيَا كَذَلِكَ.
\par 13 فَجَذَبُوا إِرْمِيَا بِالْحِبَالِ وَأَطْلَعُوهُ مِنَ الْجُبِّ. فَأَقَامَ إِرْمِيَا فِي دَارِ السِّجْنِ.
\par 14 فَأَرْسَلَ الْمَلِكُ صِدْقِيَّا وَأَخَذَ إِرْمِيَا النَّبِيَّ إِلَيْهِ إِلَى الْمَدْخَلِ الثَّالِثِ الَّذِي فِي بَيْتِ الرَّبِّ وَقَالَ الْمَلِكُ لإِرْمِيَا: [أَنَا أَسْأَلُكَ عَنْ أَمْرٍ. لاَ تُخْفِ عَنِّي شَيْئاً].
\par 15 فَقَالَ إِرْمِيَا لِصِدْقِيَّا: [إِذَا أَخْبَرْتُكَ أَفَمَا تَقْتُلُنِي قَتْلاً؟ وَإِذَا أَشَرْتُ عَلَيْكَ فَلاَ تَسْمَعُ لِي!]
\par 16 فَحَلَفَ الْمَلِكُ صِدْقِيَّا لإِرْمِيَا سِرّاً: [حَيٌّ هُوَ الرَّبُّ الَّذِي صَنَعَ لَنَا هَذِهِ النَّفْسَ إِنِّي لاَ أَقْتُلُكَ وَلاَ أَدْفَعُكَ لِيَدِ هَؤُلاَءِ الرِّجَالِ الَّذِينَ يَطْلُبُونَ نَفْسَكَ].
\par 17 فَقَالَ إِرْمِيَا لِصِدْقِيَّا: [هَكَذَا قَالَ الرَّبُّ إِلَهُ الْجُنُودِ إِلَهُ إِسْرَائِيلَ: إِنْ كُنْتَ تَخْرُجُ خُرُوجاً إِلَى رُؤَسَاءِ مَلِكِ بَابِلَ تَحْيَا نَفْسُكَ وَلاَ تُحْرَقُ هَذِهِ الْمَدِينَةُ بِالنَّارِ بَلْ تَحْيَا أَنْتَ وَبَيْتُكَ.
\par 18 وَلَكِنْ إِنْ كُنْتَ لاَ تَخْرُجُ إِلَى رُؤَسَاءِ مَلِكِ بَابِلَ تُدْفَعُ هَذِهِ الْمَدِينَةُ لِيَدِ الْكِلْدَانِيِّينَ فَيُحْرِقُونَهَا بِالنَّارِ وَأَنْتَ لاَ تُفْلِتُ مِنْ يَدِهِمْ].
\par 19 فَقَالَ صِدْقِيَّا الْمَلِكُ لإِرْمِيَا: [إِنِّي أَخَافُ مِنَ الْيَهُودِ الَّذِينَ قَدْ سَقَطُوا لِلْكِلْدَانِيِّينَ لِئَلاَّ يَدْفَعُونِي لِيَدِهِمْ فَيَزْدَرُوا بِي].
\par 20 فَقَالَ إِرْمِيَا: [لاَ يَدْفَعُونَكَ. اسْمَعْ لِصَوْتِ الرَّبِّ فِي مَا أُكَلِّمُكَ أَنَا بِهِ فَيُحْسَنَ إِلَيْكَ وَتَحْيَا نَفْسُكَ.
\par 21 وَإِنْ كُنْتَ تَأْبَى الْخُرُوجَ فَهَذِهِ هِيَ الْكَلِمَةُ الَّتِي أَرَانِي الرَّبُّ إِيَّاهَا.
\par 22 هَا كُلُّ النِّسَاءِ اللَّوَاتِي بَقِينَ فِي بَيْتِ مَلِكِ يَهُوذَا يُخْرَجْنَ إِلَى رُؤَسَاءِ مَلِكِ بَابِلَ وَهُنَّ يَقُلْنَ قَدْ خَدَعَكَ وَقَدِرَ عَلَيْكَ مُسَالِمُوكَ. غَاصَتْ فِي الْحَمْأَةِ رِجْلاَكَ وَارْتَدَّتَا إِلَى الْوَرَاءِ.
\par 23 وَيُخْرِجُونَ كُلَّ نِسَائِكَ وَبَنِيكَ إِلَى الْكِلْدَانِيِّينَ وَأَنْتَ لاَ تُفْلِتُ مِنْ يَدِهِمْ لأَنَّكَ أَنْتَ تُمْسَكُ بِيَدِ مَلِكِ بَابِلَ وَهَذِهِ الْمَدِينَةُ تُحْرَقُ بِالنَّارِ].
\par 24 فَقَالَ صِدْقِيَّا لإِرْمِيَا: [لاَ يَعْلَمْ أَحَدٌ بِهَذَا الْكَلاَمِ فَلاَ تَمُوتَ.
\par 25 وَإِذَا سَمِعَ الرُّؤَسَاءُ أَنِّي كَلَّمْتُكَ وَأَتُوا إِلَيْكَ وَقَالُوا لَكَ: أَخْبِرْنَا بِمَاذَا كَلَّمْتَ الْمَلِكَ لاَ تُخْفِ عَنَّا فَلاَ نَقْتُلَكَ وَمَاذَا قَالَ لَكَ الْمَلِكُ؟
\par 26 فَقُلْ لَهُمْ: إِنِّي أَلْقَيْتُ تَضَرُّعِي أَمَامَ الْمَلِكِ حَتَّى لاَ يَرُدَّنِي إِلَى بَيْتِ يُونَاثَانَ لأَمُوتَ هُنَاكَ].
\par 27 فَأَتَى كُلُّ الرُّؤَسَاءِ إِلَى إِرْمِيَا وَسَأَلُوهُ فَأَخْبَرَهُمْ حَسَبَ كُلِّ هَذَا الْكَلاَمِ الَّذِي أَوْصَاهُ بِهِ الْمَلِكُ فَسَكَتُوا عَنْهُ لأَنَّ الأَمْرَ لَمْ يُسْمَعْ.
\par 28 فَأَقَامَ إِرْمِيَا فِي دَارِ السِّجْنِ إِلَى الْيَوْمِ الَّذِي أُخِذَتْ فِيهِ أُورُشَلِيمَ.

\chapter{39}

\par 1 وَلَمَّا أُخِذَتْ أُورُشَلِيمَ فِي السَّنَةِ التَّاسِعَةِ لِصِدْقِيَّا مَلِكِ يَهُوذَا فِي الشَّهْرِ الْعَاشِرِ أَتَى نَبُوخَذْنَصَّرُ مَلِكُ بَابِلَ وَكُلُّ جَيْشِهِ إِلَى أُورُشَلِيمَ وَحَاصَرُوهَا.
\par 2 وَفِي السَّنَةِ الْحَادِيَةِ عَشَرَةَ لِصِدْقِيَّا فِي الشَّهْرِ الرَّابِعِ فِي تَاسِعِ الشَّهْرِ فُتِحَتِ الْمَدِينَةُ.
\par 3 وَدَخَلَ كُلُّ رُؤَسَاءِ مَلِكِ بَابِلَ وَجَلَسُوا فِي الْبَابِ الأَوْسَطِ: نَرْجَلَ شَرَاصَرُ وَسَمْجَرْ نَبُو وَسَرْسَخِيمُ رَئِيسُ الْخِصْيَانِ وَنَرْجَلَ شَرَاصَرُ رَئِيسُ الْمَجُوسِ وَكُلُّ بَقِيَّةِ رُؤَسَاءِ مَلِكِ بَابِلَ.
\par 4 فَلَمَّا رَآهُمْ صِدْقِيَّا مَلِكُ يَهُوذَا وَكُلُّ رِجَالِ الْحَرْبِ هَرَبُوا وَخَرَجُوا لَيْلاً مِنَ الْمَدِينَةِ فِي طَرِيقِ جَنَّةِ الْمَلِكِ مِنَ الْبَابِ بَيْنَ السُّورَيْنِ وَخَرَجَ هُوَ فِي طَرِيقِ الْعَرَبَةِ.
\par 5 فَسَعَى جَيْشُ الْكِلْدَانِيِّينَ وَرَاءَهُمْ فَأَدْرَكُوا صِدْقِيَّا فِي عَرَبَاتِ أَرِيحَا فَأَخَذُوهُ وَأَصْعَدُوهُ إِلَى نَبُوخَذْنَصَّرَ مَلِكِ بَابِلَ إِلَى رَبْلَةَ فِي أَرْضِ حَمَاةَ فَكَلَّمَهُ بِالْقَضَاءِ عَلَيْهِ.
\par 6 فَقَتَلَ مَلِكُ بَابِلَ بَنِي صِدْقِيَّا فِي رَبْلَةَ أَمَامَ عَيْنَيْهِ وَقَتَلَ مَلِكُ بَابِلَ كُلَّ أَشْرَافِ يَهُوذَا.
\par 7 وَأَعْمَى عَيْنَيْ صِدْقِيَّا وَقَيَّدَهُ بِسَلاَسِلِ نُحَاسٍ لِيَأْتِيَ بِهِ إِلَى بَابِلَ.
\par 8 أَمَّا بَيْتُ الْمَلِكِ وَبُيُوتُ الشَّعْبِ فَأَحْرَقَهَا الْكِلْدَانِيُّونَ بِالنَّارِ وَنَقَضُوا أَسْوَارَ أُورُشَلِيمَ.
\par 9 وَبَقِيَّةُ الشَّعْبِ الَّذِينَ بَقُوا فِي الْمَدِينَةِ وَالْهَارِبُونَ الَّذِينَ سَقَطُوا لَهُ وَبَقِيَّةُ الشَّعْبِ الَّذِينَ بَقُوا سَبَاهُمْ نَبُوزَرَادَانُ رَئِيسُ الشُّرَطِ إِلَى بَابِلَ.
\par 10 وَلَكِنَّ بَعْضَ الشَّعْبِ الْفُقَرَاءَ الَّذِينَ لَمْ يَكُنْ لَهُمْ شَيْءٌ تَرَكَهُمْ نَبُوزَرَادَانُ رَئِيسُ الشُّرَطِ فِي أَرْضِ يَهُوذَا وَأَعْطَاهُمْ كُرُوماً وَحُقُولاً فِي ذَلِكَ الْيَوْمِ.
\par 11 وَأَوْصَى نَبُوخَذْنَصَّرُ مَلِكُ بَابِلَ عَلَى إِرْمِيَا نَبُوزَرَادَانَ رَئِيسَ الشُّرَطِ قَائِلاً:
\par 12 [خُذْهُ وَضَعْ عَيْنَيْكَ عَلَيْهِ وَلاَ تَفْعَلْ بِهِ شَيْئاً رَدِيئاً بَلْ كَمَا يُكَلِّمُكَ هَكَذَا افْعَلْ مَعَهُ].
\par 13 فَأَرْسَلَ نَبُوزَرَادَانُ رَئِيسُ الشُّرَطِ وَنَبُوشَزْبَانُ رَئِيسُ الْخِصْيَانِ وَنَرْجَلُ شَرَاصَرُ رَئِيسُ الْمَجُوسِ وَكُلُّ رُؤَسَاءِ مَلِكِ بَابِلَ
\par 14 أَرْسَلُوا فَأَخَذُوا إِرْمِيَا مِنْ دَارِ السِّجْنِ وَأَسْلَمُوهُ لِجَدَلْيَا بْنِ أَخِيقَامَ بْنِ شَافَانَ لِيَخْرُجَ بِهِ إِلَى الْبَيْتِ. فَسَكَنَ بَيْنَ الشَّعْبِ.
\par 15 وَصَارَتْ كَلِمَةُ الرَّبِّ إِلَى إِرْمِيَا إِذْ كَانَ مَحْبُوساً فِي دَارِ السِّجْنِ:
\par 16 [اذْهَبْ وَقُلْ لِعَبْدَ مَلِكَ الْكُوشِيِّ هَكَذَا قَالَ رَبُّ الْجُنُودِ إِلَهُ إِسْرَائِيلَ: هَئَنَذَا جَالِبٌ كَلاَمِي عَلَى هَذِهِ الْمَدِينَةِ لِلشَّرِّ لاَ لِلْخَيْرِ فَيَحْدُثُ أَمَامَكَ فِي ذَلِكَ الْيَوْمِ.
\par 17 وَلَكِنَّنِي أُنْقِذُكَ فِي ذَلِكَ الْيَوْمِ يَقُولُ الرَّبُّ فَلاَ تُسْلَمُ لِيَدِ النَّاسِ الَّذِينَ أَنْتَ خَائِفٌ مِنْهُمْ.
\par 18 بَلْ إِنَّمَا أُنَجِّيكَ نَجَاةً فَلاَ تَسْقُطُ بِالسَّيْفِ بَلْ تَكُونُ لَكَ نَفْسُكَ غَنِيمَةً لأَنَّكَ قَدْ تَوَكَّلْتَ عَلَيَّ يَقُولُ الرَّبُّ].

\chapter{40}

\par 1 اَلْكَلِمَةُ الَّتِي صَارَتْ إِلَى إِرْمِيَا مِنْ الرَّبِّ بَعْدَ مَا أَرْسَلَهُ نَبُوزَرَادَانُ رَئِيسُ الشُّرَطِ مِنَ الرَّامَةِ إِذْ أَخَذَهُ وَهُوَ مُقَيَّدٌ بِالسَّلاَسِلِ فِي وَسَطِ كُلِّ سَبْيِ أُورُشَلِيمَ وَيَهُوذَا الَّذِينَ سُبُوا إِلَى بَابِلَ.
\par 2 فَأَخَذَ رَئِيسُ الشُّرَطِ إِرْمِيَا وَقَالَ لَهُ: [إِنَّ الرَّبَّ إِلَهَكَ قَدْ تَكَلَّمَ بِهَذَا الشَّرِّ عَلَى هَذَا الْمَوْضِعِ.
\par 3 فَجَلَبَ الرَّبُّ وَفَعَلَ كَمَا تَكَلَّمَ لأَنَّكُمْ قَدْ أَخْطَأْتُمْ إِلَى الرَّبِّ وَلَمْ تَسْمَعُوا لِصَوْتِهِ فَحَدَثَ لَكُمْ هَذَا الأَمْرُ.
\par 4 فَالآنَ هَئَنَذَا أَحُلُّكَ الْيَوْمَ مِنَ الْقُيُودِ الَّتِي عَلَى يَدِكَ. فَإِنْ حَسُنَ فِي عَيْنَيْكَ أَنْ تَأْتِيَ مَعِي إِلَى بَابِلَ فَتَعَالَ فَأَجْعَلُ عَيْنَيَّ عَلَيْكَ. وَإِنْ قَبُحَ فِي عَيْنَيْكَ أَنْ تَأْتِيَ مَعِي إِلَى بَابِلَ فَامْتَنِعْ. انْظُرْ. كُلُّ الأَرْضِ هِيَ أَمَامَكَ فَحَيْثُمَا حَسُنَ وَكَانَ مُسْتَقِيماً فِي عَيْنَيْكَ أَنْ تَنْطَلِقَ فَانْطَلِقْ إِلَى هُنَاكَ].
\par 5 وَإِذْ كَانَ لَمْ يَرْجِعْ بَعْدُ قَالَ: [ارْجِعْ إِلَى جَدَلْيَا بْنِ أَخِيقَامَ بْنِ شَافَانَ الَّذِي أَقَامَهُ مَلِكُ بَابِلَ عَلَى مُدُنِ يَهُوذَا وَأَقِمْ عِنْدَهُ فِي وَسَطِ الشَّعْبِ وَانْطَلِقْ إِلَى حَيْثُ كَانَ مُسْتَقِيماً فِي عَيْنَيْكَ أَنْ تَنْطَلِقَ]. وَأَعْطَاهُ رَئِيسُ الشُّرَطِ زَاداً وَهَدِيَّةً وَأَطْلَقَهُ.
\par 6 فَجَاءَ إِرْمِيَا إِلَى جَدَلْيَا بْنِ أَخِيقَامَ إِلَى الْمِصْفَاةِ وَأَقَامَ عِنْدَهُ فِي وَسَطِ الشَّعْبِ الْبَاقِينَ فِي الأَرْضِ.
\par 7 فَلَمَّا سَمِعَ كُلُّ رُؤَسَاءِ الْجُيُوشِ الَّذِينَ فِي الْحَقْلِ هُمْ وَرِجَالُهُمْ أَنَّ مَلِكَ بَابِلَ قَدْ أَقَامَ جَدَلْيَا بْنَ أَخِيقَامَ عَلَى الأَرْضِ وَأَنَّهُ وَكَّلَهُ عَلَى الرِّجَالِ وَالنِّسَاءِ وَالأَطْفَالِ وَعَلَى فُقَرَاءِ الأَرْضِ الَّذِينَ لَمْ يُسْبُوا إِلَى بَابِلَ
\par 8 أَتَى إِلَى جَدَلْيَا إِلَى الْمِصْفَاةِ إِسْمَاعِيلُ بْنُ نَثَنْيَا وَيُوحَانَانُ وَيُونَاثَانُ ابْنَا قَارِيحَ وَسَرَايَا بْنُ تَنْحُومَثَ وَبَنُو عِيفَايَ النَّطُوفَاتِيُّ وَيَزَنْيَا ابْنُ الْمَعْكِيِّ هُمْ وَرِجَالُهُمْ.
\par 9 فَحَلَفَ لَهُمْ جَدَلْيَا بْنُ أَخِيقَامَ بْنِ شَافَانَ وَلِرِجَالِهِمْ قَائِلاً: [لاَ تَخَافُوا مِنْ أَنْ تَخْدِمُوا الْكِلْدَانِيِّينَ. اسْكُنُوا فِي الأَرْضِ وَاخْدِمُوا مَلِكَ بَابِلَ فَيُحْسَنَ إِلَيْكُمْ.
\par 10 أَمَّا أَنَا فَهَئَنَذَا سَاكِنٌ فِي الْمِصْفَاةِ لأَقِفَ أَمَامَ الْكِلْدَانِيِّينَ الَّذِينَ يَأْتُونَ إِلَيْنَا. أَمَّا أَنْتُمْ فَاجْمَعُوا خَمْراً وَتِيناً وَزَيْتاً وَضَعُوا فِي أَوْعِيَتِكُمْ وَاسْكُنُوا فِي مُدُنِكُمُ الَّتِي أَخَذْتُمُوهَا].
\par 11 وَكَذَلِكَ كُلُّ الْيَهُودِ الَّذِينَ فِي مُوآبَ وَبَيْنَ بَنِي عَمُّونَ وَفِي أَدُومَ وَالَّذِينَ فِي كُلِّ الأَرَاضِي سَمِعُوا أَنَّ مَلِكَ بَابِلَ قَدْ جَعَلَ بَقِيَّةً لِيَهُوذَا وَقَدْ أَقَامَ عَلَيْهِمْ جَدَلْيَا بْنَ أَخِيقَامَ بْنِ شَافَانَ
\par 12 فَرَجَعَ كُلُّ الْيَهُودِ مِنْ كُلِّ الْمَوَاضِعِ الَّتِي طُوِّحُوا إِلَيْهَا وَأَتُوا إِلَى أَرْضِ يَهُوذَا إِلَى جَدَلْيَا إِلَى الْمِصْفَاةِ وَجَمَعُوا خَمْراً وَتِيناً كَثِيراً جِدّاً.
\par 13 ثُمَّ إِنَّ يُوحَانَانَ بْنَ قَارِيحَ وَكُلَّ رُؤَسَاءِ الْجُيُوشِ الَّذِينَ فِي الْحَقْلِ أَتُوا إِلَى جَدَلْيَا إِلَى الْمِصْفَاةِ
\par 14 وَقَالُوا لَهُ: [أَتَعْلَمُ عِلْماً أَنَّ بَعْلِيسَ مَلِكَ بَنِي عَمُّونَ قَدْ أَرْسَلَ إِسْمَاعِيلَ بْنَ نَثَنْيَا لِيَقْتُلَكَ؟] فَلَمْ يُصَدِّقْهُمْ جَدَلْيَا بْنُ أَخِيقَامَ.
\par 15 فَكَلَّمَ يُوحَانَانُ بْنُ قَارِيحَ جَدَلْيَا سِرّاً فِي الْمِصْفَاةِ قَائِلاً: [ دَعْنِي أَنْطَلِقْ وَأَضْرِبْ إِسْمَاعِيلَ بْنَ نَثَنْيَا وَلاَ يَعْلَمُ إِنْسَانٌ. لِمَاذَا يَقْتُلُكَ فَيَتَبَدَّدَ كُلُّ يَهُوذَا الْمُجْتَمِعُ إِلَيْكَ وَتَهْلِكَ بَقِيَّةُ يَهُوذَا؟]
\par 16 فَقَالَ جَدَلْيَا بْنُ أَخِيقَامَ لِيُوحَانَانَ بْنِ قَارِيحَ: [لاَ تَفْعَلْ هَذَا الأَمْرَ لأَنَّكَ إِنَّمَا تَتَكَلَّمُ بِالْكَذِبِ عَنْ إِسْمَاعِيلَ].

\chapter{41}

\par 1 وَكَانَ فِي الشَّهْرِ السَّابِعِ أَنَّ إِسْمَاعِيلَ بْنَ نَثَنْيَا بْنِ أَلِيشَامَاعَ مِنَ النَّسْلِ الْمُلُوكِيِّ جَاءَ هُوَ وَعُظَمَاءُ الْمَلِكِ وَعَشَرَةُ رِجَالٍ مَعَهُ إِلَى جَدَلْيَا بْنِ أَخِيقَامَ إِلَى الْمِصْفَاةِ وَأَكَلُوا هُنَاكَ خُبْزاً مَعاً فِي الْمِصْفَاةِ.
\par 2 فَقَامَ إِسْمَاعِيلُ بْنُ نَثَنْيَا وَالْعَشَرَةُ الرِّجَالِ الَّذِينَ كَانُوا مَعَهُ وَضَرَبُوا جَدَلْيَا بْنَ أَخِيقَامَ بْنِ شَافَانَ بِالسَّيْفِ فَقَتَلُوهُ هَذَا الَّذِي أَقَامَهُ مَلِكُ بَابِلَ عَلَى الأَرْضِ.
\par 3 وَكُلُّ الْيَهُودِ الَّذِينَ كَانُوا مَعَ جَدَلْيَا فِي الْمِصْفَاةِ وَالْكِلْدَانِيُّونَ الَّذِينَ وُجِدُوا هُنَاكَ وَرِجَالُ الْحَرْبِ ضَرَبَهُمْ إِسْمَاعِيلُ.
\par 4 وَكَانَ فِي الْيَوْمِ الثَّانِي بَعْدَ قَتْلِهِ جَدَلْيَا وَلَمْ يَعْلَمْ إِنْسَانٌ
\par 5 أَنَّ رِجَالاً أَتُوا مِنْ شَكِيمَ وَمِنْ شِيلُوَ وَمِنَ السَّامِرَةِ ثَمَانِينَ رَجُلاً مَحْلُوقِي اللُّحَى وَمُشَقَّقِي الثِّيَابِ وَمُخَمَّشِينَ وَبِيَدِهِمْ تَقْدِمَةٌ وَلُبَانٌ لِيُدْخِلُوهُمَا إِلَى بَيْتِ الرَّبِّ.
\par 6 فَخَرَجَ إِسْمَاعِيلُ بْنُ نَثَنْيَا لِلِقَائِهِمْ مِنَ الْمِصْفَاةِ سَائِراً وَبَاكِياً. فَكَانَ لَمَّا لَقِيَهُمْ أَنَّهُ قَالَ لَهُمْ: [هَلُمَّ إِلَى جَدَلْيَا بْنِ أَخِيقَامَ].
\par 7 فَكَانَ لَمَّا أَتُوا إِلَى وَسَطِ الْمَدِينَةِ أَنَّ إِسْمَاعِيلَ بْنَ نَثَنْيَا قَتَلَهُمْ وَأَلْقَاهُمْ إِلَى وَسَطِ الْجُبِّ هُوَ وَالرِّجَالُ الَّذِينَ مَعَهُ.
\par 8 وَلَكِنْ وُجِدَ فِيهِمْ عَشَرَةُ رِجَالٍ قَالُوا لإِسْمَاعِيلَ: [لاَ تَقْتُلْنَا لأَنَّهُ يُوجَدُ لَنَا خَزَائِنُ فِي الْحَقْلِ قَمْحٌ وَشَعِيرٌ وَزَيْتٌ وَعَسَلٌ]. فَامْتَنَعَ وَلَمْ يَقْتُلْهُمْ بَيْنَ إِخْوَتِهِمْ.
\par 9 فَالْجُبُّ الَّذِي طَرَحَ فِيهِ إِسْمَاعِيلُ كُلَّ جُثَثِ الرِّجَالِ الَّذِينَ قَتَلَهُمْ بِسَبَبِ جَدَلْيَا هُوَ الَّذِي صَنَعَهُ الْمَلِكُ آسَا مِنْ وَجْهِ بَعْشَا مَلِكِ إِسْرَائِيلَ. فَمَلَأَهُ إِسْمَاعِيلُ بْنُ نَثَنْيَا مِنَ الْقَتْلَى.
\par 10 فَسَبَى إِسْمَاعِيلُ كُلَّ بَقِيَّةِ الشَّعْبِ الَّذِينَ فِي الْمِصْفَاةِ بَنَاتِ الْمَلِكِ وَكُلَّ الشَّعْبِ الَّذِي بَقِيَ فِي الْمِصْفَاةِ الَّذِينَ أَقَامَ عَلَيْهِمْ نَبُوزَرَادَانُ رَئِيسُ الشُّرَطِ جَدَلْيَا بْنَ أَخِيقَامَ سَبَاهُمْ إِسْمَاعِيلُ بْنُ نَثَنْيَا وَذَهَبَ لِيَعْبُرَ إِلَى بَنِي عَمُّونَ.
\par 11 فَلَمَّا سَمِعَ يُوحَانَانُ بْنُ قَارِيحَ وَكُلُّ رُؤَسَاءِ الْجُيُوشِ الَّذِينَ مَعَهُ بِكُلِّ الشَّرِّ الَّذِي فَعَلَهُ إِسْمَاعِيلُ بْنُ نَثَنْيَا
\par 12 أَخَذُوا كُلَّ الرِّجَالِ وَسَارُوا لِيُحَارِبُوا إِسْمَاعِيلَ بْنَ نَثَنْيَا فَوَجَدُوهُ عِنْدَ الْمِيَاهِ الْكَثِيرَةِ الَّتِي فِي جِبْعُونَ.
\par 13 وَلَمَّا رَأَى كُلُّ الشَّعْبِ الَّذِي مَعَ إِسْمَاعِيلَ يُوحَانَانَ بْنَ قَارِيحَ وَكُلَّ رُؤَسَاءِ الْجُيُوشِ الَّذِينَ مَعَهُمْ فَرِحُوا.
\par 14 فَدَارَ كُلُّ الشَّعْبِ الَّذِي سَبَاهُ إِسْمَاعِيلُ مِنَ الْمِصْفَاةِ وَرَجَعُوا وَسَارُوا إِلَى يُوحَانَانَ بْنِ قَارِيحَ.
\par 15 أَمَّا إِسْمَاعِيلُ بْنُ نَثَنْيَا فَهَرَبَ بِثَمَانِيَةِ رِجَالٍ مِنْ وَجْهِ يُوحَانَانَ وَسَارَ إِلَى بَنِي عَمُّونَ.
\par 16 فَأَخَذَ يُوحَانَانُ بْنُ قَارِيحَ وَكُلُّ رُؤَسَاءِ الْجُيُوشِ الَّذِينَ مَعَهُ كُلَّ بَقِيَّةِ الشَّعْبِ الَّذِينَ اسْتَرَدَّهُمْ مِنْ إِسْمَاعِيلَ بْنِ نَثَنْيَا مِنَ الْمِصْفَاةِ بَعْدَ قَتْلِ جَدَلْيَا بْنِ أَخِيقَامَ رِجَالَ الْحَرْبِ الْمُقْتَدِرِينَ وَالنِّسَاءَ وَالأَطْفَالَ وَالْخِصْيَانَ الَّذِينَ اسْتَرَدَّهُمْ مِنْ جِبْعُونَ
\par 17 فَسَارُوا وَأَقَامُوا فِي جَيْرُوتَ كِمْهَامَ الَّتِي بِجَانِبِ بَيْتِ لَحْمٍ لِيَسِيرُوا وَيَدْخُلُوا مِصْرَ
\par 18 مِنْ وَجْهِ الْكِلْدَانِيِّينَ لأَنَّهُمْ كَانُوا خَائِفِينَ مِنْهُمْ لأَنَّ إِسْمَاعِيلَ بْنَ نَثَنْيَا كَانَ قَدْ ضَرَبَ جَدَلْيَا بْنَ أَخِيقَامَ الَّذِي أَقَامَهُ مَلِكُ بَابِلَ عَلَى الأَرْضِ.

\chapter{42}

\par 1 فَتَقَدَّمَ كُلُّ رُؤَسَاءِ الْجُيُوشِ وَيُوحَانَانُ بْنُ قَارِيحَ وَيَزَنْيَا بْنُ هُوشَعْيَا وَكُلُّ الشَّعْبِ مِنَ الصَّغِيرِ إِلَى الْكَبِيرِ
\par 2 وَقَالُوا لإِرْمِيَا النَّبِيِّ: [لَيْتَ تَضَرُّعَنَا يَقَعُ أَمَامَكَ فَتُصَلِّيَ لأَجْلِنَا إِلَى الرَّبِّ إِلَهِكَ لأَجْلِ كُلِّ هَذِهِ الْبَقِيَّةِ. لأَنَّنَا قَدْ بَقِينَا قَلِيلِينَ مِنْ كَثِيرِينَ كَمَا تَرَانَا عَيْنَاكَ.
\par 3 فَيُخْبِرُنَا الرَّبُّ إِلَهُكَ عَنِ الطَّرِيقِ الَّذِي نَسِيرُ فِيهِ وَالأَمْرِ الَّذِي نَفْعَلُهُ].
\par 4 فَقَالَ لَهُمْ إِرْمِيَا النَّبِيُّ: [قَدْ سَمِعْتُ. هَئَنَذَا أُصَلِّي إِلَى الرَّبِّ إِلَهِكُمْ كَقَوْلِكُمْ وَيَكُونُ أَنَّ كُلَّ الْكَلاَمِ الَّذِي يُجِيبُكُمُ الرَّبُّ أُخْبِرُكُمْ بِهِ. لاَ أَمْنَعُ عَنْكُمْ شَيْئاً].
\par 5 فَقَالُوا لإِرْمِيَا: [لِيَكُنِ الرَّبُّ بَيْنَنَا شَاهِداً صَادِقاً وَأَمِيناً إِنَّنَا نَفْعَلُ حَسَبَ كُلِّ أَمْرٍ يُرْسِلُكَ بِهِ الرَّبُّ إِلَهُكَ إِلَيْنَا
\par 6 إِنْ خَيْراً وَإِنْ شَرّاً. فَإِنَّنَا نَسْمَعُ لِصَوْتِ الرَّبِّ إِلَهِكَ الَّذِي نَحْنُ مُرْسِلُوكَ إِلَيْهِ لِيُحْسَنَ إِلَيْنَا إِذَا سَمِعْنَا لِصَوْتِ الرَّبِّ إِلَهِنَا].
\par 7 وَكَانَ بَعْدَ عَشَرَةِ أَيَّامٍ أَنَّ كَلِمَةَ الرَّبِّ صَارَتْ إِلَى إِرْمِيَا.
\par 8 فَدَعَا يُوحَانَانَ بْنَ قَارِيحَ وَكُلَّ رُؤَسَاءِ الْجُيُوشِ الَّذِينَ مَعَهُ وَكُلَّ الشَّعْبِ مِنَ الصَّغِيرِ إِلَى الْكَبِيرِ
\par 9 وَقَالَ لَهُمْ: [هَكَذَا قَالَ الرَّبُّ إِلَهُ إِسْرَائِيلَ الَّذِي أَرْسَلْتُمُونِي إِلَيْهِ لأُلْقِيَ تَضَرُّعَكُمْ أَمَامَهُ.
\par 10 إِنْ كُنْتُمْ تَسْكُنُونَ فِي هَذِهِ الأَرْضِ فَإِنِّي أَبْنِيكُمْ وَلاَ أَنْقُضُكُمْ وَأَغْرِسُكُمْ وَلاَ أَقْتَلِعُكُمْ. لأَنِّي نَدِمْتُ عَنِ الشَّرِّ الَّذِي صَنَعْتُهُ بِكُمْ.
\par 11 لاَ تَخَافُوا مَلِكَ بَابِلَ الَّذِي أَنْتُمْ خَائِفُوهُ. لاَ تَخَافُوهُ يَقُولُ الرَّبُّ لأَنِّي أَنَا مَعَكُمْ لأُخَلِّصَكُمْ وَأُنْقِذَكُمْ مِنْ يَدِهِ.
\par 12 وَأُعْطِيَكُمْ نِعْمَةً فَيَرْحَمُكُمْ وَيَرُدُّكُمْ إِلَى أَرْضِكُمْ.
\par 13 [وَإِنْ قُلْتُمْ لاَ نَسْكُنُ فِي هَذِهِ الأَرْضِ وَلَمْ تَسْمَعُوا لِصَوْتِ الرَّبِّ إِلَهِكُمْ
\par 14 قَائِلِينَ: لاَ بَلْ إِلَى أَرْضِ مِصْرَ نَذْهَبُ حَيْثُ لاَ نَرَى حَرْباً وَلاَ نَسْمَعُ صَوْتَ بُوقٍ وَلاَ نَجُوعُ لِلْخُبْزِ وَهُنَاكَ نَسْكُنُ.
\par 15 فَالآنَ لِذَلِكَ اسْمَعُوا كَلِمَةَ الرَّبِّ يَا بَقِيَّةَ يَهُوذَا: هَكَذَا قَالَ رَبُّ الْجُنُودِ إِلَهُ إِسْرَائِيلَ: إِنْ كُنْتُمْ تَجْعَلُونَ وُجُوهَكُمْ لِلدُّخُولِ إِلَى مِصْرَ وَتَذْهَبُونَ لِتَتَغَرَّبُوا هُنَاكَ
\par 16 يَحْدُثُ أَنَّ السَّيْفَ الَّذِي أَنْتُمْ خَائِفُونَ مِنْهُ يُدْرِكُكُمْ هُنَاكَ فِي أَرْضِ مِصْرَ وَالْجُوعَ الَّذِي أَنْتُمْ خَائِفُونَ مِنْهُ يَلْحَقُكُمْ هُنَاكَ فِي مِصْرَ فَتَمُوتُونَ هُنَاكَ.
\par 17 وَيَكُونُ أَنَّ كُلَّ الرِّجَالِ الَّذِينَ جَعَلُوا وُجُوهَهُمْ لِلدُّخُولِ إِلَى مِصْرَ لِيَتَغَرَّبُوا هُنَاكَ يَمُوتُونَ بِالسَّيْفِ وَالْجُوعِ وَالْوَبَإِ وَلاَ يَكُونُ مِنْهُمْ بَاقٍ وَلاَ نَاجٍ مِنَ الشَّرِّ الَّذِي أَجْلِبُهُ أَنَا عَلَيْهِمْ.
\par 18 لأَنَّهُمْ هَكَذَا قَالَ رَبُّ الْجُنُودِ إِلَهُ إِسْرَائِيلَ: كَمَا انْسَكَبَ غَضَبِي وَغَيْظِي عَلَى سُكَّانِ أُورُشَلِيمَ هَكَذَا يَنْسَكِبُ غَيْظِي عَلَيْكُمْ عِنْدَ دُخُولِكُمْ إِلَى مِصْرَ فَتَصِيرُونَ حَلَفاً وَدَهَشاً وَلَعْنَةً وَعَاراً وَلاَ تَرُونَ بَعْدُ هَذَا الْمَوْضِعَ.
\par 19 [قَدْ تَكَلَّمَ الرَّبُّ عَلَيْكُمْ يَا بَقِيَّةَ يَهُوذَا لاَ تَدْخُلُوا مِصْرَ. اعْلَمُوا عِلْماً أَنِّي قَدْ أَنْذَرْتُكُمُ الْيَوْمَ.
\par 20 لأَنَّكُمْ قَدْ خَدَعْتُمْ أَنْفُسَكُمْ إِذْ أَرْسَلْتُمُونِي إِلَى الرَّبِّ إِلَهِكُمْ قَائِلِينَ: صَلِّ لأَجْلِنَا إِلَى الرَّبِّ إِلَهِنَا وَحَسَبَ كُلِّ مَا يَقُولُهُ الرَّبُّ إِلَهُنَا هَكَذَا أَخْبِرْنَا فَنَفْعَلَ.
\par 21 فَقَدْ أَخْبَرْتُكُمُ الْيَوْمَ فَلَمْ تَسْمَعُوا لِصَوْتِ الرَّبِّ إِلَهِكُمْ وَلاَ لِشَيْءٍ مِمَّا أَرْسَلَنِي بِهِ إِلَيْكُمْ.
\par 22 فَالآنَ اعْلَمُوا عِلْماً أَنَّكُمْ تَمُوتُونَ بِالسَّيْفِ وَالْجُوعِ وَالْوَبَإِ فِي الْمَوْضِعِ الَّذِي ابْتَغَيْتُمْ أَنْ تَدْخُلُوهُ لِتَتَغَرَّبُوا فِيهِ].

\chapter{43}

\par 1 وَكَانَ لَمَّا فَرَغَ إِرْمِيَا مِنْ أَنْ كَلَّمَ كُلَّ الشَّعْبِ بِكُلِّ كَلاَمِ الرَّبِّ إِلَهِهِمِ الَّذِي أَرْسَلَهُ الرَّبُّ إِلَهُهُمْ إِلَيْهِمْ
\par 2 أَنَّ عَزَرْيَا بْنَ هُوشَعْيَا وَيُوحَانَانَ بْنَ قَارِيحَ وَكُلَّ الرِّجَالِ الْمُتَكَبِّرِينَ قَالُوا لإِرْمِيَا: [أَنْتَ مُتَكَلِّمٌ بِالْكَذِبِ! لَمْ يُرْسِلْكَ الرَّبُّ إِلَهُنَا لِتَقُولَ لاَ تَذْهَبُوا إِلَى مِصْرَ لِتَتَغَرَّبُوا هُنَاكَ
\par 3 بَلْ بَارُوخُ بْنُ نِيرِيَّا مُهَيِّجُكَ عَلَيْنَا لِتَدْفَعَنَا لِيَدِ الْكِلْدَانِيِّينَ لِيَقْتُلُونَا وَيَسْبُونَا إِلَى بَابِلَ].
\par 4 فَلَمْ يَسْمَعْ يُوحَانَانُ بْنُ قَارِيحَ وَكُلُّ رُؤَسَاءِ الْجُيُوشِ وَكُلُّ الشَّعْبِ لِصَوْتِ الرَّبِّ بِالإِقَامَةِ فِي أَرْضِ يَهُوذَا
\par 5 بَلْ أَخَذَ يُوحَانَانُ بْنُ قَارِيحَ وَكُلُّ رُؤَسَاءِ الْجُيُوشِ كُلَّ بَقِيَّةِ يَهُوذَا الَّذِينَ رَجَعُوا مِنْ كُلِّ الأُمَمِ الَّذِينَ طُوِّحُوا إِلَيْهِمْ لِيَتَغَرَّبُوا فِي أَرْضِ يَهُوذَا
\par 6 الرِّجَالَ وَالنِّسَاءَ وَالأَطْفَالَ وَبَنَاتِ الْمَلِكِ وَكُلَّ الأَنْفُسِ الَّذِينَ تَرَكَهُمْ نَبُوزَرَادَانُ رَئِيسُ الشُّرَطِ مَعَ جَدَلْيَا بْنِ أَخِيقَامَ بْنِ شَافَانَ وَإِرْمِيَا النَّبِيِّ وَبَارُوخَ بْنِ نِيرِيَّا
\par 7 فَجَاءُوا إِلَى أَرْضِ مِصْرَ لأَنَّهُمْ لَمْ يَسْمَعُوا لِصَوْتِ الرَّبِّ وَأَتُوا إِلَى تَحْفَنِيسَ.
\par 8 ثُمَّ صَارَتْ كَلِمَةُ الرَّبِّ إِلَى إِرْمِيَا فِي تَحْفَنِيسَ قَائِلَةً:
\par 9 [خُذْ بِيَدِكَ حِجَارَةً كَبِيرَةً وَاطْمُرْهَا فِي الْمِلاَطِ فِي الْمَلْبِنِ الَّذِي عِنْدَ بَابِ بَيْتِ فِرْعَوْنَ فِي تَحْفَنِيسَ أَمَامَ رِجَالٍ يَهُودٍ
\par 10 وَقُلْ لَهُمْ: هَكَذَا قَالَ رَبُّ الْجُنُودِ إِلَهُ إِسْرَائِيلَ: هَئَنَذَا أُرْسِلُ وَآخُذُ نَبُوخَذْنَصَّرَ مَلِكَ بَابِلَ عَبْدِي وَأَضَعُ كُرْسِيَّهُ فَوْقَ هَذِهِ الْحِجَارَةِ الَّتِي طَمَرْتُهَا فَيُبْسِطُ دِيبَاجَهُ عَلَيْهَا.
\par 11 وَيَأْتِي وَيَضْرِبُ أَرْضَ مِصْرَ الَّذِي لِلْمَوْتِ فَلِلْمَوْتِ وَالَّذِي لِلسَّبْيِ فَلِلسَّبْيِ وَالَّذِي لِلسَّيْفِ فَلِلسَّيْفِ.
\par 12 وَأُوقِدُ نَاراً فِي بُيُوتِ آلِهَةِ مِصْرَ فَيُحْرِقُهَا وَيَسْبِيهَا وَيَلْبَسُ أَرْضَ مِصْرَ كَمَا يَلْبَسُ الرَّاعِي رِدَاءَهُ ثُمَّ يَخْرُجُ مِنْ هُنَاكَ بِسَلاَمٍ.
\par 13 وَيَكْسِرُ أَنْصَابَ بَيْتَ شَمْسٍ الَّتِي فِي أَرْضِ مِصْرَ وَيُحْرِقُ بُيُوتَ آلِهَةِ مِصْرَ بِالنَّارِ].

\chapter{44}

\par 1 اَلْكَلِمَةُ الَّتِي صَارَتْ إِلَى إِرْمِيَا مِنْ جِهَةِ كُلِّ الْيَهُودِ السَّاكِنِينَ فِي أَرْضِ مِصْرَ السَّاكِنِينَ فِي مَجْدَلَ وَفِي تَحْفَنِيسَ وَفِي نُوفَ وَفِي أَرْضِ فَتْرُوسَ:
\par 2 [هَكَذَا قَالَ رَبُّ الْجُنُودِ إِلَهُ إِسْرَائِيلَ: أَنْتُمْ رَأَيْتُمْ كُلَّ الشَّرِّ الَّذِي جَلَبْتُهُ عَلَى أُورُشَلِيمَ وَعَلَى كُلِّ مُدُنِ يَهُوذَا فَهَا هِيَ خَرِبَةٌ هَذَا الْيَوْمَ وَلَيْسَ فِيهَا سَاكِنٌ
\par 3 مِنْ أَجْلِ شَرِّهِمِ الَّذِي فَعَلُوهُ لِيُغِيظُونِي إِذْ ذَهَبُوا لِيُبَخِّرُوا وَيَعْبُدُوا آلِهَةً أُخْرَى لَمْ يَعْرِفُوهَا هُمْ وَلاَ أَنْتُمْ وَلاَ آبَاؤُكُمْ.
\par 4 فَأَرْسَلْتُ إِلَيْكُمْ كُلَّ عَبِيدِي الأَنْبِيَاءِ مُبَكِّراً وَمُرْسِلاً قَائِلاً: لاَ تَفْعَلُوا أَمْرَ هَذَا الرِّجْسِ الَّذِي أَبْغَضْتُهُ.
\par 5 فَلَمْ يَسْمَعُوا وَلاَ أَمَالُوا أُذْنَهُمْ لِيَرْجِعُوا عَنْ شَرِّهِمْ فَلاَ يُبَخِّرُوا لِآلِهَةٍ أُخْرَى.
\par 6 فَانْسَكَبَ غَيْظِي وَغَضَبي وَاشْتَعَلاَ في مُدُنِ يَهُوذَا وَفِي شَوَارِعِ أُورُشَلِيمَ فَصَارَتْ خَرِبَةً مُقْفِرَةً كَهَذَا الْيَوْمِ.
\par 7 فَالآنَ هَكَذَا قَالَ الرَّبُّ إِلَهُ الْجُنُودِ إِلَهُ إِسْرَائِيلَ: لِمَاذَا أَنْتُمْ فَاعِلُونَ شَرّاً عَظِيماً ضِدَّ أَنْفُسِكُمْ لاِنْقِرَاضِكُمْ رِجَالاً وَنِسَاءً أَطْفَالاً وَرُضَّعاً مِنْ وَسَطِ يَهُوذَا وَلاَ تَبْقَى لَكُمْ بَقِيَّةٌ.
\par 8 لإِغَاظَتِي بِأَعْمَالِ أَيَادِيكُمْ إِذْ تُبَخِّرُونَ لِآلِهَةٍ أُخْرَى فِي أَرْضِ مِصْرَ الَّتِي أَتَيْتُمْ إِلَيْهَا لِتَتَغَرَّبُوا فِيهَا لِكَيْ تَنْقَرِضُوا وَتَصِيرُوا لَعْنَةً وَعَاراً بَيْنَ كُلِّ أُمَمِ الأَرْضِ.
\par 9 هَلْ نَسِيتُمْ شُرُورَ آبَائِكُمْ وَشُرُورَ مُلُوكِ يَهُوذَا وَشُرُورَ نِسَائِهِمْ وَشُرُورَكُمْ وَشُرُورَ نِسَائِكُمُ الَّتِي فُعِلَتْ فِي أَرْضِ يَهُوذَا وَفِي شَوَارِعِ أُورُشَلِيمَ؟
\par 10 لَمْ يُذَلُّوا إِلَى هَذَا الْيَوْمِ وَلاَ خَافُوا وَلاَ سَلَكُوا فِي شَرِيعَتِي وَفَرَائِضِي الَّتِي جَعَلْتُهَا أَمَامَكُمْ وَأَمَامَ آبَائِكُمْ.
\par 11 [لِذَلِكَ هَكَذَا قَالَ رَبُّ الْجُنُودِ إِلَهُ إِسْرَائِيلَ: هَئَنَذَا أَجْعَلُ وَجْهِي عَلَيْكُمْ لِلشَّرِّ وَلأَقْرِضَ كُلَّ يَهُوذَا.
\par 12 وَآخُذُ بَقِيَّةَ يَهُوذَا الَّذِينَ جَعَلُوا وُجُوهَهُمْ لِلدُّخُولِ إِلَى أَرْضِ مِصْرَ لِيَتَغَرَّبُوا هُنَاكَ فَيَفْنُونَ كُلُّهُمْ فِي أَرْضِ مِصْرَ. يَسْقُطُونَ بِالسَّيْفِ وَبِالْجُوعِ. يَفْنُونَ مِنَ الصَّغِيرِ إِلَى الْكَبِيرِ بِالسَّيْفِ وَالْجُوعِ. يَمُوتُونَ وَيَصِيرُونَ حَلْفاً وَدَهَشاً وَلَعْنَةً وَعَاراً.
\par 13 وَأُعَاقِبُ الَّذِينَ يَسْكُنُونَ فِي أَرْضِ مِصْرَ كَمَا عَاقَبْتُ أُورُشَلِيمَ بِالسَّيْفِ وَالْجُوعِ وَالْوَبَإِ.
\par 14 وَلاَ يَكُونُ نَاجٍ وَلاَ بَاقٍ لِبَقِيَّةِ يَهُوذَا الآتِينَ لِيَتَغَرَّبُوا هُنَاكَ فِي أَرْضِ مِصْرَ لِيَرْجِعُوا إِلَى أَرْضِ يَهُوذَا الَّتِي يَشْتَاقُونَ إِلَى الرُّجُوعِ لأَجْلِ السَّكَنِ فِيهَا لأَنَّهُ لاَ يَرْجِعُ مِنْهُمْ إِلاَّ الْمُنْفَلِتُونَ].
\par 15 فَأَجَابَ إِرْمِيَا كُلُّ الرِّجَالِ الَّذِينَ عَرَفُوا أَنَّ نِسَاءَهُمْ يُبَخِّرْنَ لِآلِهَةٍ أُخْرَى وَكُلُّ النِّسَاءِ الْوَاقِفَاتِ مَحْفَلٌ كَبِيرٌ وَكُلُّ الشَّعْبِ السَّاكِنِ فِي أَرْضِ مِصْرَ فِي فَتْرُوسَ:
\par 16 [إِنَّنَا لاَ نَسْمَعُ لَكَ الْكَلِمَةَ الَّتِي كَلَّمْتَنَا بِهَا بِاسْمِ الرَّبِّ
\par 17 بَلْ سَنَعْمَلُ كُلَّ أَمْرٍ خَرَجَ مِنْ فَمِنَا فَنُبَخِّرُ لِمَلِكَةِ السَّمَاوَاتِ وَنَسْكُبُ لَهَا سَكَائِبَ. كَمَا فَعَلْنَا نَحْنُ وَآبَاؤُنَا وَمُلُوكُنَا وَرُؤَسَاؤُنَا فِي أَرْضِ يَهُوذَا وَفِي شَوَارِعِ أُورُشَلِيمَ فَشَبِعْنَا خُبْزاً وَكُنَّا بِخَيْرٍ وَلَمْ نَرَ شَرّاً.
\par 18 وَلَكِنْ مِنْ حِينَ كَفَفْنَا عَنِ التَّبْخِيرِ لِمَلِكَةِ السَّمَاوَاتِ وَسَكْبِ سَكَائِبَ لَهَا احْتَجْنَا إِلَى كُلٍّ وَفَنِينَا بِالسَّيْفِ وَالْجُوعِ.
\par 19 وَإِذْ كُنَّا نُبَخِّرُ لِمَلِكَةِ السَّمَاوَاتِ وَنَسْكُبُ لَهَا سَكَائِبَ فَهَلْ بِدُونِ رِجَالِنَا كُنَّا نَصْنَعُ لَهَا كَعْكاً لِنَعْبُدَهَا وَنَسْكُبُ لَهَا السَّكَائِبَ؟].
\par 20 فَقَالَ إِرْمِيَا لِكُلِّ الشَّعْبِ الرِّجَالِ وَالنِّسَاءِ الَّذِينَ جَاوَبُوهُ بِهَذَا الْكَلاَمِ:
\par 21 [أَلَيْسَ الْبَخُورُ الَّذِي بَخَّرْتُمُوهُ فِي مُدُنِ يَهُوذَا وَفِي شَوَارِعِ أُورُشَلِيمَ أَنْتُمْ وَآبَاؤُكُمْ وَمُلُوكُكُمْ وَرُؤَسَاؤُكُمْ وَشَعْبُ الأَرْضِ هُوَ الَّذِي ذَكَرَهُ الرَّبُّ وَصَعِدَ عَلَى قَلْبِهِ.
\par 22 وَلَمْ يَسْتَطِعِ الرَّبُّ أَنْ يَحْتَمِلَ بَعْدُ مِنْ أَجْلِ شَرِّ أَعْمَالِكُمْ مِنْ أَجْلِ الرَّجَاسَاتِ الَّتِي فَعَلْتُمْ فَصَارَتْ أَرْضُكُمْ خَرِبَةً وَدَهَشاً وَلَعْنَةً بِلاَ سَاكِنٍ كَهَذَا الْيَوْمِ.
\par 23 مِنْ أَجْلِ أَنَّكُمْ قَدْ بَخَّرْتُمْ وَأَخْطَأْتُمْ إِلَى الرَّبِّ وَلَمْ تَسْمَعُوا لِصَوْتِ الرَّبِّ وَلَمْ تَسْلُكُوا فِي شَرِيعَتِهِ وَفَرَائِضِهِ وَشَهَادَاتِهِ مِنْ أَجْلِ ذَلِكُمْ قَدْ أَصَابَكُمْ هَذَا الشَّرُّ كَهَذَا الْيَوْمِ].
\par 24 ثُمَّ قَالَ إِرْمِيَا لِكُلِّ الشَّعْبِ وَلِكُلِّ النِّسَاءِ: [اسْمَعُوا كَلِمَةَ الرَّبِّ يَا جَمِيعَ يَهُوذَا الَّذِينَ فِي أَرْضِ مِصْرَ.
\par 25 هَكَذَا تَكَلَّمَ رَبُّ الْجُنُودِ إِلَهُ إِسْرَائِيلَ: أَنْتُمْ وَنِسَاؤُكُمْ تَكَلَّمْتُمْ بِفَمِكُمْ وَأَكْمَلْتُمْ بِأَيَادِيكُمْ قَائِلِينَ: إِنَّنَا إِنَّمَا نُتَمِّمُ نُذُورَنَا الَّتِي نَذَرْنَاهَا أَنْ نُبَخِّرَ لِمَلِكَةِ السَّمَاوَاتِ وَنَسْكُبُ لَهَا سَكَائِبَ فَإِنَّهُنَّ يُقِمْنَ نُذُورَكُمْ وَيُتَمِّمْنَ نُذُورَكُمْ.
\par 26 لِذَلِكَ اسْمَعُوا كَلِمَةَ الرَّبِّ يَا جَمِيعَ يَهُوذَا السَّاكِنِينَ فِي أَرْضِ مِصْرَ. هَئَنَذَا قَدْ حَلَفْتُ بِاسْمِي الْعَظِيمِ قَالَ الرَّبُّ إِنَّ اسْمِي لَنْ يُسَمَّى بَعْدُ بِفَمِ إِنْسَانٍ مَا مِنْ يَهُوذَا فِي كُلِّ أَرْضِ مِصْرَ قَائِلاً: حَيٌّ السَّيِّدُ الرَّبُّ.
\par 27 هَئَنَذَا أَسْهَرُ عَلَيْهِمْ لِلشَّرِّ لاَ لِلْخَيْرِ فَيَفْنَى كُلُّ رِجَالِ يَهُوذَا الَّذِينَ فِي أَرْضِ مِصْرَ بِالسَّيْفِ وَالْجُوعِ حَتَّى يَتَلاَشُوا.
\par 28 وَالنَّاجُونَ مِنَ السَّيْفِ يَرْجِعُونَ مِنْ أَرْضِ مِصْرَ إِلَى أَرْضِ يَهُوذَا نَفَراً قَلِيلاً فَيَعْلَمُ كُلُّ بَقِيَّةِ يَهُوذَا الَّذِينَ أَتُوا إِلَى أَرْضِ مِصْرَ لِيَتَغَرَّبُوا فِيهَا كَلِمَةَ أَيِّنَا تَقُومُ.
\par 29 [وَهَذِهِ هِيَ الْعَلاَمَةُ لَكُمْ يَقُولُ الرَّبُّ: إِنِّي أُعَاقِبُكُمْ فِي هَذَا الْمَوْضِعِ لِتَعْلَمُوا أَنَّهُ لاَ بُدَّ أَنْ يَقُومَ كَلاَمِي عَلَيْكُمْ لِلشَّرِّ.
\par 30 هَكَذَا قَالَ الرَّبُّ. هَئَنَذَا أَدْفَعُ فِرْعَوْنَ حَفْرَعَ مَلِكَ مِصْرَ لِيَدِ أَعْدَائِهِ وَلِيَدِ طَالِبِي نَفْسِهِ كَمَا دَفَعْتُ صِدْقِيَّا مَلِكَ يَهُوذَا لِيَدِ نَبُوخَذْنَصَّرَ مَلِكِ بَابِلَ عَدُوِّهِ وَطَالِبِ نَفْسِهِ].

\chapter{45}

\par 1 اَلْكَلِمَةُ الَّتِي تَكَلَّمَ بِهَا إِرْمِيَا النَّبِيُّ إِلَى بَارُوخَ بْنِ نِيرِيَّا عِنْدَ كَتَابَتِهِ هَذَا الْكَلاَمَ فِي سِفْرٍ عَنْ فَمِ إِرْمِيَا فِي السَّنَةِ الرَّابِعَةِ لِيَهُويَاقِيمَ بْنِ يُوشِيَّا مَلِكِ يَهُوذَا:
\par 2 [هَكَذَا قَالَ الرَّبُّ إِلَهُ إِسْرَائِيلَ لَكَ يَا بَارُوخُ.
\par 3 قَدْ قُلْتَ: وَيْلٌ لِي لأَنَّ الرَّبَّ قَدْ زَادَ حُزْناً عَلَى أَلَمِي. قَدْ غُشِيَ عَلَيَّ في تَنَهُّدِي وَلَمْ أَجِدْ رَاحَةً.
\par 4 [هَكَذَا تَقُولُ لَهُ: هَكَذَا قَالَ الرَّبُّ: هَئَنَذَا أَهْدِمُ مَا بَنَيْتُهُ وَأَقْتَلِعُ مَا غَرَسْتُهُ وَكُلَّ هَذِهِ الأَرْضِ.
\par 5 وَأَنْتَ فَهَلْ تَطْلُبُ لِنَفْسِكَ أُمُوراً عَظِيمَةً؟ لاَ تَطْلُبُ! لأَنِّي هَئَنَذَا جَالِبٌ شَرّاً عَلَى كُلِّ ذِي جَسَدٍ يَقُولُ الرَّبُّ وَأُعْطِيكَ نَفْسَكَ غَنِيمَةً فِي كُلِّ الْمَوَاضِعِ الَّتِي تَسِيرُ إِلَيْهَا].

\chapter{46}

\par 1 كَلِمَةُ الرَّبِّ الَّتِي صَارَتْ إِلَى إِرْمِيَا النَّبِيِّ عَنِ الأُمَمِ:
\par 2 عَنْ مِصْرَ. عَنْ جَيْشِ فِرْعَوْنَ نَخُو مَلِكِ مِصْرَ الَّذِي كَانَ عَلَى نَهْرِ الْفُرَاتِ فِي كَرْكَمِيشَ الَّذِي ضَرَبَهُ نَبُوخَذْنَصَّرُ مَلِكُ بَابِلَ فِي السَّنَةِ الرَّابِعَةِ لِيَهُويَاقِيمَ بْنِ يُوشِيَّا مَلِكِ يَهُوذَا:
\par 3 [أَعِدُّوا الْمِجَنَّ وَالتُّرْسَ وَتَقَدَّمُوا لِلْحَرْبِ.
\par 4 أَسْرِجُوا الْخَيْلَ وَاصْعَدُوا أَيُّهَا الْفُرْسَانُ وَانْتَصِبُوا بِالْخُوَذِ. اصْقِلُوا الرِّمَاحَ. الْبِسُوا الدُّرُوعَ.
\par 5 لِمَاذَا أَرَاهُمْ مُرْتَعِبِينَ وَمُدْبِرِينَ إِلَى الْوَرَاءِ وَقَدْ تَحَطَّمَتْ أَبْطَالُهُمْ وَفَرُّوا هَارِبِينَ وَلَمْ يَلْتَفِتُوا؟ الْخَوْفُ حَوَالَيْهِمْ يَقُولُ الرَّبُّ.
\par 6 الْخَفِيفُ لاَ يَنُوصُ وَالْبَطَلُ لاَ يَنْجُو. فِي الشِّمَالِ بِجَانِبِ نَهْرِ الْفُرَاتِ عَثَرُوا وَسَقَطُوا.
\par 7 مَنْ هَذَا الصَّاعِدُ كَالنِّيلِ كَأَنْهَارٍ تَتَلاَطَمُ أَمْوَاهُهَا؟
\par 8 تَصْعَدُ مِصْرُ كَالنِّيلِ وَكَأَنْهَارٍ تَتَلاَطَمُ الْمِيَاهُ. فَيَقُولُ: أَصْعَدُ وَأُغَطِّي الأَرْضَ. أُهْلِكُ الْمَدِينَةَ وَالسَّاكِنِينَ فِيهَا.
\par 9 اصْعَدِي أَيَّتُهَا الْخَيْلُ وَهِيجِي أَيَّتُهَا الْمَرْكَبَاتُ وَلْتَخْرُجِ الأَبْطَالُ. كُوشُ وَفُوطُ الْقَابِضَانِ الْمِجَنَّ وَاللُّودِيُّونَ الْقَابِضُونَ وَالْمَادُّونَ الْقَوْسَ.
\par 10 فَهَذَا الْيَوْمُ لِلسَّيِّدِ رَبِّ الْجُنُودِ يَوْمُ نَقْمَةٍ لِلاِنْتِقَامِ مِنْ مُبْغِضِيهِ فَيَأْكُلُ السَّيْفُ وَيَشْبَعُ وَيَرْتَوِي مِنْ دَمِهِمْ. لأَنَّ لِلسَّيِّدِ رَبِّ الْجُنُودِ ذَبِيحَةً فِي أَرْضِ الشِّمَالِ عِنْدَ نَهْرِ الْفُرَاتِ.
\par 11 اصْعَدِي إِلَى جِلْعَادَ وَخُذِي بَلَسَاناً يَا عَذْرَاءُ بِنْتَ مِصْرَ. بَاطِلاً تُكَثِّرِينَ الْعَقَاقِيرَ. لاَ رِفَادَةَ لَكِ.
\par 12 قَدْ سَمِعَتِ الأُمَمُ بِخِزْيِكِ وَقَدْ مَلَأَ الأَرْضَ عَوِيلُكِ لأَنَّ بَطَلاً يَصْدِمُ بَطَلاً فَيَسْقُطَانِ كِلاَهُمَا مَعاً].
\par 13 اَلْكَلِمَةُ الَّتِي تَكَلَّمَ بِهَا الرَّبُّ إِلَى إِرْمِيَا النَّبِيِّ فِي مَجِيءِ نَبُوخَذْنَصَّرَ مَلِكِ بَابِلَ لِيَضْرِبَ أَرْضَ مِصْرَ:
\par 14 [أَخْبِرُوا فِي مِصْرَ وَأَسْمِعُوا فِي مَجْدَلَ وَأَسْمِعُوا فِي نُوفَ وَفِي تَحْفَنِيسَ. قُولُوا انْتَصِبْ وَتَهَيَّأْ لأَنَّ السَّيْفَ يَأْكُلُ حَوَالَيْكَ.
\par 15 لِمَاذَا انْطَرَحَ مُقْتَدِرُوكَ؟ لاَ يَقِفُونَ لأَنَّ الرَّبَّ قَدْ طَرَحَهُمْ!
\par 16 كَثَّرَ الْعَاثِرِينَ حَتَّى يَسْقُطَ الْوَاحِدُ عَلَى صَاحِبِهِ وَيَقُولُوا: قُومُوا فَنَرْجِعَ إِلَى شَعْبِنَا وَإِلَى أَرْضِ مِيلاَدِنَا مِنْ وَجْهِ السَّيْفِ الصَّارِمِ.
\par 17 قَدْ نَادُوا هُنَاكَ: فِرْعَوْنُ مَلِكُ مِصْرَ هَالِكٌ. قَدْ فَاتَ الْمِيعَادُ.
\par 18 حَيٌّ أَنَا يَقُولُ الْمَلِكُ رَبُّ الْجُنُودِ اسْمُهُ كَتَابُورٍ بَيْنَ الْجِبَالِ وَكَكَرْمَلٍ عِنْدَ الْبَحْرِ يَأْتِي.
\par 19 اِصْنَعِي لِنَفْسِكِ أُهْبَةَ جَلاَءٍ أَيَّتُهَا الْبِنْتُ السَّاكِنَةُ مِصْرَ لأَنَّ نُوفَ تَصِيرُ خَرِبَةً وَتُحْرَقُ فَلاَ سَاكِنَ.
\par 20 مِصْرُ عِجْلَةٌ حَسَنَةٌ جِدّاً. الْهَلاَكُ مِنَ الشِّمَالِ جَاءَ جَاءَ.
\par 21 أَيْضاً مُسْتَأْجَرُوهَا فِي وَسَطِهَا كَعُجُولِ صِيرَةٍ. لأَنَّهُمْ هُمْ أَيْضاً يَرْتَدُّونَ يَهْرُبُونَ مَعاً. لَمْ يَقِفُوا لأَنَّ يَوْمَ هَلاَكِهِمْ أَتَى عَلَيْهِمْ وَقْتَ عِقَابِهِمْ.
\par 22 صَوْتُهَا يَمْشِي كَحَيَّةٍ لأَنَّهُمْ يَسِيرُونَ بِجَيْشٍ وَقَدْ جَاءُوا إِلَيْهَا بِالْفُؤُوسِ كَمُحْتَطِبِي حَطَبٍ.
\par 23 يَقْطَعُونَ وَعْرَهَا يَقُولُ الرَّبُّ وَإِنْ يَكُنْ لاَ يُحْصَى لأَنَّهُمْ قَدْ كَثُرُوا أَكْثَرَ مِنَ الْجَرَادِ وَلاَ عَدَدَ لَهُمْ.
\par 24 قَدْ أُخْزِيَتْ بِنْتُ مِصْرَ وَدُفِعَتْ لِيَدِ شَعْبِ الشِّمَالِ.
\par 25 قَالَ رَبُّ الْجُنُودِ إِلَهُ إِسْرَائِيلَ: هَئَنَذَا أُعَاقِبُ أَمُونَ نُوَ وَفِرْعَوْنَ وَمِصْرَ وَآلِهَتَهَا وَمُلُوكَهَا فِرْعَوْنَ وَالْمُتَوَكِّلِينَ عَلَيْهِ.
\par 26 وَأَدْفَعُهُمْ لِيَدِ طَالِبِي نُفُوسِهِمْ وَلِيَدِ نَبُوخَذْنَصَّرَ مَلِكِ بَابِلَ وَلِيَدِ عَبِيدِهِ. ثُمَّ بَعْدَ ذَلِكَ تُسْكَنُ كَالأَيَّامِ الْقَدِيمَةِ يَقُولُ الرَّبُّ.
\par 27 [وَأَنْتَ فَلاَ تَخَفْ يَا عَبْدِي يَعْقُوبُ وَلاَ تَرْتَعِبْ يَا إِسْرَائِيلُ لأَنِّي هَئَنَذَا أُخَلِّصُكَ مِنْ بَعِيدٍ وَنَسْلَكَ مِنْ أَرْضِ سَبْيِهِمْ فَيَرْجِعُ يَعْقُوبُ وَيَطْمَئِنُّ وَيَسْتَرِيحُ وَلاَ مُخِيفٌ.
\par 28 أَمَّا أَنْتَ يَا عَبْدِي يَعْقُوبُ فَلاَ تَخَفْ لأَنِّي أَنَا مَعَكَ لأَنِّي أُفْنِي كُلَّ الأُمَمِ الَّذِينَ بَدَّدْتُكَ إِلَيْهِمْ. أَمَّا أَنْتَ فَلاَ أُفْنِيكَ بَلْ أُؤَدِّبُكَ بِالْحَقِّ وَلاَ أُبَرِّئُكَ تَبْرِئَةً].

\chapter{47}

\par 1 كَلِمَةُ الرَّبِّ الَّتِي صَارَتْ إِلَى إِرْمِيَا النَّبِيِّ عَنِ الْفِلِسْطِينِيِّينَ قَبْلَ ضَرْبِ فِرْعَوْنَ غَزَّةَ:
\par 2 [هَكَذَا قَالَ الرَّبُّ هَا مِيَاهٌ تَصْعَدُ مِنَ الشِّمَالِ وَتَكُونُ سَيْلاً جَارِفاً فَتُغَشِّي الأَرْضَ وَمِلأَهَا الْمَدِينَةَ وَالسَّاكِنِينَ فِيهَا فَيَصْرُخُ النَّاسُ وَيُوَلْوِلُ كُلُّ سُكَّانِ الأَرْضِ.
\par 3 مِنْ صَوْتِ قَرْعِ حَوَافِرِ أَقْوِيَائِهِ مِنْ صَرِيرِ مَرْكَبَاتِهِ وَصَرِيفِ بَكَرَاتِهِ لاَ تَلْتَفِتُ الآبَاءُ إِلَى الْبَنِينَ بِسَبَبِ ارْتِخَاءِ الأَيَادِي.
\par 4 بِسَبَبِ الْيَوْمِ الآتِي لِهَلاَكِ كُلِّ الْفِلِسْطِينِيِّينَ لِيَنْقَرِضَ مِنْ صُورَ وَصَيْدُونَ كُلُّ بَقِيَّةٍ تُعِينُ لأَنَّ الرَّبَّ يُهْلِكُ الْفِلِسْطِينِيِّينَ بَقِيَّةَ جَزِيرَةِ كَفْتُورَ.
\par 5 أَتَى الصُّلْعُ عَلَى غَزَّةَ. أُهْلِكَتْ أَشْقَلُونُ مَعَ بَقِيَّةِ وَطَائِهِمْ. حَتَّى مَتَى تَخْمِشِينَ نَفْسَكِ.
\par 6 آهِ يَا سَيْفَ الرَّبِّ حَتَّى مَتَى لاَ تَسْتَرِيحُ؟ انْضَمَّ إِلَى غِمْدِكَ! اهْدَأْ وَاسْكُنْ.
\par 7 كَيْفَ يَسْتَرِيحُ وَالرَّبُّ قَدْ أَوْصَاهُ؟ عَلَى أَشْقَلُونَ وَعَلَى سَاحِلِ الْبَحْرِ هُنَاكَ وَاعَدَهُ!]

\chapter{48}

\par 1 عَنْ مُوآبَ: [هَكَذَا قَالَ رَبُّ الْجُنُودِ إِلَهُ إِسْرَائِيلَ: وَيْلٌ لِنَبُو لأَنَّهَا قَدْ خَرِبَتْ. خَزِيَتْ وَأُخِذَتْ قَرْيَتَايِمُ. خَزِيَتْ مِسْجَابُ وَارْتَعَبَتْ.
\par 2 لَيْسَ مَوْجُوداً بَعْدُ فَخْرُ مُوآبَ. فِي حَشْبُونَ فَكَّرُوا عَلَيْهَا شَرّاً. هَلُمَّ فَنَقْرِضُهَا مِنْ أَنْ تَكُونَ أُمَّةً. وَأَنْتِ أَيْضاً يَا مَدْمِينُ تُصَمِّينَ وَيَذْهَبُ وَرَاءَكِ السَّيْفُ.
\par 3 صَوْتُ صِيَاحٍ مِنْ حُورُونَايِمَ. هَلاَكٌ وَسَحْقٌ عَظِيمٌ.
\par 4 قَدْ حُطِّمَتْ مُوآبُ وَأَسْمَعَ صِغَارُهَا صُرَاخاً.
\par 5 لأَنَّهُ فِي عَقَبَةِ لُوحِيتَ يَصْعَدُ بُكَاءٌ عَلَى بُكَاءٍ لأَنَّهُ فِي مُنْحَدَرِ حُورُونَايِمَ سَمِعَ الأَعْدَاءُ صُرَاخَ انْكِسَارٍ.
\par 6 اهْرُبُوا نَجُّوا أَنْفُسَكُمْ وَكُونُوا كَعَرْعَرٍ فِي الْبَرِّيَّةِ.
\par 7 [فَمِنْ أَجْلِ اتِّكَالِكِ عَلَى أَعْمَالِكِ وَعَلَى خَزَائِنِكِ سَتُؤْخَذِينَ أَنْتِ أَيْضاً وَيَخْرُجُ كَمُوشُ إِلَى السَّبْيِ كَهَنَتُهُ ورُؤَسَاؤُهُ مَعاً.
\par 8 وَيَأْتِي الْمُهْلِكُ إِلَى كُلِّ مَدِينَةٍ فَلاَ تُفْلِتُ مَدِينَةٌ فَيَبِيدُ الْوَطَاءُ وَيَهْلِكُ السَّهْلُ كَمَا قَالَ الرَّبُّ.
\par 9 أَعْطُوا مُوآبَ جَنَاحاً لأَنَّهَا تَخْرُجُ طَائِرَةً وَتَصِيرُ مُدُنُهَا خَرِبَةً بِلاَ سَاكِنٍ فِيهَا.
\par 10 مَلْعُونٌ مَنْ يَعْمَلُ عَمَلَ الرَّبِّ بِرِخَاءٍ وَمَلْعُونٌ مَنْ يَمْنَعُ سَيْفَهُ عَنِ الدَّمِ.
\par 11 [مُسْتَرِيحٌ مُوآبُ مُنْذُ صِبَاهُ وَهُوَ مُسْتَقِرٌّ عَلَى دُرْدِيِّهِ وَلَمْ يُفْرَغْ مِنْ إِنَاءٍ إِلَى إِنَاءٍ وَلَمْ يَذْهَبْ إِلَى السَّبْيِ. لِذَلِكَ بَقِيَ طَعْمُهُ فِيهِ وَرَائِحَتُهُ لَمْ تَتَغَيَّرْ.
\par 12 لِذَلِكَ هَا أَيَّامٌ تَأْتِي يَقُولُ الرَّبُّ وَأُرْسِلُ إِلَيْهِ مُصْغِينَ فَيُصْغُونَهُ وَيُفَرِّغُونَ آنِيَتَهُ وَيَكْسِرُونَ أَوْعِيَتَهُمْ.
\par 13 فَيَخْجَلُ مُوآبُ مِنْ كَمُوشَ كَمَا خَجِلَ بَيْتُ إِسْرَائِيلَ مِنْ بَيْتِ إِيلَ مُتَّكَلِهِمْ.
\par 14 [كَيْفَ تَقُولُونَ نَحْنُ جَبَابِرَةٌ وَرِجَالُ قُوَّةٍ لِلْحَرْبِ؟
\par 15 أُهْلِكَتْ مُوآبُ وَصَعِدَتْ مُدُنُهَا وَخِيَارُ مُنْتَخَبِيهَا. نَزَلُوا لِلْقَتْلِ يَقُولُ الْمَلِكُ رَبُّ الْجُنُودِ اسْمُهُ.
\par 16 قَرِيبٌ مَجِيءُ هَلاَكِ مُوآبَ وَبَلِيَّتُهَا مُسْرِعَةٌ جِدّاً.
\par 17 اُنْدُبُوهَا يَا جَمِيعَ الَّذِينَ حَوَالَيْهَا وَكُلَّ الْعَارِفِينَ اسْمَهَا قُولُوا: كَيْفَ انْكَسَرَ قَضِيبُ الْعِزِّ عَصَا الْجَلاَلِ؟
\par 18 اِنْزِلِي مِنَ الْمَجْدِ اجْلِسِي فِي الظَّمَاءِ أَيَّتُهَا السَّاكِنَةُ بِنْتَ دِيبُونَ لأَنَّ مُهْلِكَ مُوآبَ قَدْ صَعِدَ إِلَيْكِ وَأَهْلَكَ حُصُونَكِ.
\par 19 قِفِي عَلَى الطَّرِيقِ وَتَطَلَّعِي يَا سَاكِنَةَ عَرُوعِيرَ. اسْأَلِي الْهَارِبَ وَالنَّاجِيَةَ. قُولِي: مَاذَا حَدَثَ؟
\par 20 قَدْ خَزِيَ مُوآبُ لأَنَّهُ قَدْ نُقِضَ. وَلْوِلُوا وَاصْرُخُوا. أَخْبِرُوا فِي أَرْنُونَ أَنَّ مُوآبَ قَدْ أُهْلِكَ.
\par 21 وَقَدْ جَاءَ الْقَضَاءُ عَلَى أَرْضِ السَّهْلِ عَلَى حُولُونَ وَعَلَى يَهْصَةَ وَعَلَى مَيْفَعَةَ
\par 22 وَعَلَى دِيبُونَ وَعَلَى نَبُو وَعَلَى بَيْتَِ دَبْلَتَايِمَ
\par 23 وَعَلَى قَرْيَتَايِمَ وَعَلَى بَيْتَِ جَامُولَ وَعَلَى بَيْتَِ مَعُونَ
\par 24 وَعَلَى قَرْيُوتَ وَعَلَى بُصْرَةَ وَعَلَى كُلِّ مُدُنِ أَرْضِ مُوآبَ الْبَعِيدَةِ وَالْقَرِيبَةِ.
\par 25 عُضِبَ قَرْنُ مُوآبَ وَتَحَطَّمَتْ ذِرَاعُهُ يَقُولُ الرَّبُّ.
\par 26 [أَسْكِرُوهُ لأَنَّهُ قَدْ تَعَاظَمَ عَلَى الرَّبِّ فَيَتَمَرَّغَ مُوآبُ فِي قُيَائِهِ وَهُوَ أَيْضاً يَكُونُ ضِحْكَةً.
\par 27 أَفَمَا كَانَ إِسْرَائِيلُ ضِحْكَةً لَكَ؟ هَلْ وُجِدَ بَيْنَ اللُّصُوصِ حَتَّى أَنَّكَ كُلَّمَا كُنْتَ تَتَكَلَّمُ بِهِ كُنْتَ تَنْغَضُ الرَّأْسَ؟
\par 28 خَلُّوا الْمُدُنَ وَاسْكُنُوا فِي الصَّخْرِ يَا سُكَّانَ مُوآبَ وَكُونُوا كَحَمَامَةٍ تُعَشِّشُ فِي جَوَانِبِ فَمِ الْحُفْرَةِ.
\par 29 قَدْ سَمِعْنَا بِكِبْرِيَاءِ مُوآبَ. هُوَ مُتَكَبِّرٌ جِدّاً. بِعَظَمَتِهِ وَبِكِبْرِيَائِهِ وَجَلاَلِهِ وَارْتِفَاعِ قَلْبِهِ.
\par 30 أَنَا عَرَفْتُ سَخَطَهُ يَقُولُ الرَّبُّ إِنَّهُ بَاطِلٌ. أَكَاذِيبُهُ فَعَلَتْ بَاطِلاً.
\par 31 مِنْ أَجْلِ ذَلِكَ أُوَلْوِلُ عَلَى مُوآبَ وَعَلَى مُوآبَ كُلِّهِ أَصْرُخُ. يُؤَنُّ عَلَى رِجَالِ قِيرَ حَارِسَ.
\par 32 أَبْكِي عَلَيْكِ بُكَاءَ يَعْزِيرَ يَا جَفْنَةَ سَبْمَةَ. قَدْ عَبَرَتْ قُضْبَانُكِ الْبَحْرَ وَصَلَتْ إِلَى بَحْرِ يَعْزِيرَ. وَقَعَ الْمُهْلِكُ عَلَى جَنَاكِ وَعَلَى قِطَافِكِ.
\par 33 وَنُزِعَ الْفَرَحُ وَالطَّرَبُ مِنَ الْبُسْتَانِ وَمِنْ أَرْضِ مُوآبَ. وَقَدْ أُبْطِلَتِ الْخَمْرُ مِنَ الْمَعَاصِرِ. لاَ يُدَاسُ بِهُتَافٍ. جَلَبَةٌ لاَ هُتَافٌ.
\par 34 قَدْ أَطْلَقُوا صَوْتَهُمْ مِنْ صُرَاخِ حَشْبُونَ إِلَى أَلْعَالَةَ إِلَى يَاهَصَ مِنْ صُوغَرَ إِلَى حُورُونَايِمَ كَعِجْلَةٍ ثُلاَثِيَّةٍ لأَنَّ مِيَاهَ نِمْرِيمَ أَيْضاً تَصِيرُ خَرِبَةً.
\par 35 وَأُبَطِّلُ مِنْ مُوآبَ يَقُولُ الرَّبُّ مَنْ يُصْعِدُ فِي مُرْتَفَعَةٍ وَمَنْ يُبَخِّرُ لِآلِهَتِهِ.
\par 36 مِنْ أَجْلِ ذَلِكَ يُصَوِّتُ قَلْبِي لِمُوآبَ كَنَايٍ وَيُصَوِّتُ قَلْبِي لِرِجَالِ قِيرَ حَارِسَ كَنَايٍ لأَنَّ الثَّرْوَةَ الَّتِي اكْتَسَبُوهَا قَدْ بَادَتْ.
\par 37 لأَنَّ كُلَّ رَأْسٍ أَقْرَعُ وَكُلَّ لِحْيَةٍ مَجْزُوزَةٌ وَعَلَى كُلِّ الأَيَادِي خُمُوشٌ وَعَلَى الأَحْقَاءِ مُسُوحٌ.
\par 38 عَلَى كُلِّ سُطُوحِ مُوآبَ وَفِي شَوَارِعِهَا كُلِّهَا نَوْحٌ لأَنِّي قَدْ حَطَمْتُ مُوآبَ كَإِنَاءٍ لاَ مَسَرَّةَ بِهِ يَقُولُ الرَّبُّ.
\par 39 يُوَلْوِلُونَ قَائِلِينَ: كَيْفَ نُقِضَتْ كَيْفَ حَوَّلَتْ مُوآبُ قَفَاهَا بِخِزْيٍ؟ فَقَدْ صَارَتْ مُوآبُ ضِحْكَةً وَرُعْباً لِكُلِّ مَنْ حَوَالَيْهَا.
\par 40 لأَنَّهُ هَكَذَا قَالَ الرَّبُّ: هَا هُوَ يَطِيرُ كَنَسْرٍ وَيَبْسُطُ جَنَاحَيْهِ عَلَى مُوآبَ.
\par 41 قَدْ أُخِذَتْ قَرْيُوتُ وَأُمْسِكَتِ الْحَصِينَاتُ وَسَيَكُونُ قَلْبُ جَبَابِرَةِ مُوآبَ فِي ذَلِكَ الْيَوْمِ كَقَلْبِ امْرَأَةٍ مَاخِضٍ.
\par 42 وَيَهْلِكُ مُوآبُ عَنْ أَنْ يَكُونَ شَعْباً لأَنَّهُ قَدْ تَعَاظَمَ عَلَى الرَّبِّ.
\par 43 خَوْفٌ وَحُفْرَةٌ وَفَخٌّ عَلَيْكَ يَا سَاكِنَ مُوآبَ يَقُولُ الرَّبُّ.
\par 44 الَّذِي يَهْرُبُ مِنْ وَجْهِ الْخَوْفِ يَسْقُطُ فِي الْحُفْرَةِ وَالَّذِي يَصْعَدُ مِنَ الْحُفْرَةِ يَعْلَقُ فِي الْفَخِّ لأَنِّي أَجْلِبُ عَلَى مُوآبَ سَنَةَ عِقَابِهِمْ يَقُولُ الرَّبُّ.
\par 45 فِي ظِلِّ حَشْبُونَ وَقَفَ الْهَارِبُونَ بِلاَ قُوَّةٍ. لأَنَّهُ قَدْ خَرَجَتْ نَارٌ مِنْ حَشْبُونَ وَلَهِيبٌ مِنْ وَسَطِ سِيحُونَ فَأَكَلَتْ زَاوِيَةَ مُوآبَ وَهَامَةَ بَنِي الْوَغَى.
\par 46 وَيْلٌ لَكَ يَا مُوآبُ. بَادَ شَعْبُ كَمُوشَ لأَنَّ بَنِيكَ قَدْ أُخِذُوا إِلَى السَّبْيِ وَبَنَاتِكَ إِلَى الْجَلاَءِ.
\par 47 وَلَكِنَّنِي أَرُدُّ سَبْيَ مُوآبَ فِي آخِرِ الأَيَّامِ يَقُولُ الرَّبُّ]. إِلَى هُنَا قَضَاءُ مُوآبَ.

\chapter{49}

\par 1 عَنْ بَنِي عَمُّونَ: [هَكَذَا قَالَ الرَّبُّ. أَلَيْسَ لإِسْرَائِيلَ بَنُونَ أَوْ لاَ وَارِثٌ لَهُ؟ لِمَاذَا يَرِثُ مَلِكُهُمْ جَادَ وَشَعْبُهُ يَسْكُنُ فِي مُدُنِهِ؟
\par 2 لِذَلِكَ هَا أَيَّامٌ تَأْتِي يَقُولُ الرَّبُّ وَأُسْمِعُ فِي رَبَّةِ بَنِي عَمُّونَ جَلَبَةَ حَرْبٍ وَتَصِيرُ تَلاًّ خَرِباً وَتُحْرَقُ بَنَاتُهَا بِالنَّارِ فَيَرِثُ إِسْرَائِيلُ الَّذِينَ وَرِثُوهُ يَقُولُ الرَّبُّ.
\par 3 وَلْوِلِي يَا حَشْبُونُ لأَنَّ عَايَ قَدْ خَرِبَتْ. اصْرُخْنَ يَا بَنَاتِ رَبَّةَ. تَنَطَّقْنَ بِمُسُوحٍ. انْدُبْنَ وَطَوِّفْنَ بَيْنَ الْجُدْرَانِ لأَنَّ مَلِكَهُمْ يَذْهَبُ إِلَى السَّبْيِ هُوَ وَكَهَنَتُهُ وَرُؤَسَاؤُهُ مَعاً.
\par 4 مَا بَالُكِ تَفْتَخِرِينَ بِالأَوْطِئَةِ؟ قَدْ فَاضَ وَطَاؤُكِ دَماً أَيَّتُهَا الْبِنْتُ الْمُرْتَدَّةُ وَالْمُتَوَكِّلَةُ عَلَى خَزَائِنِهَا قَائِلَةً: مَنْ يَأْتِي إِلَيَّ؟
\par 5 هَئَنَذَا أَجْلِبُ عَلَيْكِ خَوْفاً يَقُولُ السَّيِّدُ رَبُّ الْجُنُودِ مِنْ جَمِيعِ الَّذِينَ حَوَالَيْكِ وَتُطْرَدُونَ كُلُّ وَاحِدٍ إِلَى مَا أَمَامَهُ وَلَيْسَ مَنْ يَجْمَعُ التَّائِهِينَ.
\par 6 ثُمَّ بَعْدَ ذَلِكَ أَرُدُّ سَبْيَ بَنِي عَمُّونَ يَقُولُ الرَّبُّ].
\par 7 عَنْ أَدُومَ: [هَكَذَا قَالَ رَبُّ الْجُنُودِ. أَلاَ حِكْمَةَ بَعْدُ فِي تَيْمَانَ؟ هَلْ بَادَتِ الْمَشُورَةُ مِنَ الْفُهَمَاءِ؟ هَلْ فَرَغَتْ حِكْمَتُهُمْ؟
\par 8 اُهْرُبُوا. الْتَفِتُوا. تَعَمَّقُوا فِي السَّكَنِ يَا سُكَّانَ دَدَانَ. لأَنِّي قَدْ جَلَبْتُ عَلَيْهِ بَلِيَّةَ عِيسُو حِينَ عَاقَبْتُهُ.
\par 9 لَوْ أَتَاكَ الْقَاطِفُونَ أَفَمَا كَانُوا يَتْرُكُونَ عُلاَلَةً أَوِ اللُّصُوصُ لَيْلاً أَفَمَا كَانُوا يُهْلِكُونَ مَا يَكْفِيهِمْ؟
\par 10 وَلَكِنَّنِي جَرَّدْتُ عِيسُوَ وَكَشَفْتُ مُسْتَتَرَاتِهِ فَلاَ يَسْتَطِيعُ أَنْ يَخْتَبِئَ. هَلَكَ نَسْلُهُ وَإِخْوَتُهُ وَجِيرَانُهُ فَلاَ يُوجَدُ.
\par 11 اُتْرُكْ أَيْتَامَكَ أَنَا أُحْيِيهِمْ وَأَرَامِلُكَ عَلَيَّ لِيَتَوَكَّلْنَ.
\par 12 لأَنَّهُ هَكَذَا قَالَ الرَّبُّ: هَا إِنَّ الَّذِينَ لاَ حَقَّ لَهُمْ أَنْ يَشْرَبُوا الْكَأْسَ قَدْ شَرِبُوا فَهَلْ أَنْتَ تَتَبَرَّأُ تَبَرُّؤاً؟ لاَ تَتَبَرَّأ! بَلْ إِنَّمَا تَشْرَبُ شُرْباً.
\par 13 لأَنِّي بِذَاتِي حَلَفْتُ يَقُولُ الرَّبُّ إِنَّ بُصْرَةَ تَكُونُ دَهَشاً وَعَاراً وَخَرَاباً وَلَعْنَةً وَكُلَُّ مُدُنِهَا تَكُونُ خِرَباً أَبَدِيَّةً.
\par 14 قَدْ سَمِعْتُ خَبَراً مِنْ قِبَلِ الرَّبِّ وَأُرْسِلَ رَسُولٌ إِلَى الأُمَمِ قَائِلاً: [تَجَمَّعُوا وَتَعَالُوا عَلَيْهَا وَقُومُوا لِلْحَرْبِ.
\par 15 لأَنِّي هَا قَدْ جَعَلْتُكَ صَغِيراً بَيْنَ الشُّعُوبِ وَمُحْتَقَراً بَيْنَ النَّاسِ.
\par 16 قَدْ غَرَّكَ تَخْوِيفُكَ كِبْرِيَاءُ قَلْبِكَ يَا سَاكِنُ فِي مَحَاجِئِ الصَّخْرِ الْمَاسِكَ مُرْتَفَعِ الأَكَمَةِ. وَإِنْ رَفَعْتَ كَنَسْرٍ عُشَّكَ فَمِنْ هُنَاكَ أُحْدِرُكَ يَقُولُ الرَّبُّ.
\par 17 وَتَصِيرُ أَدُومُ عَجَباً. كُلُّ مَارٍّ بِهَا يَتَعَجَّبُ وَيَصْفِرُ بِسَبَبِ كُلِّ ضَرَبَاتِهَا!
\par 18 كَانْقِلاَبِ سَدُومَ وَعَمُورَةَ وَمُجَاوَرَاتِهِمَا يَقُولُ الرَّبُّ لاَ يَسْكُنُ هُنَاكَ إِنْسَانٌ وَلاَ يَتَغَرَّبُ فِيهَا ابْنُ آدَمَ.
\par 19 هُوَذَا يَصْعَدُ كَأَسَدٍ مِنْ كِبْرِيَاءِ الأُرْدُنِّ إِلَى مَرْعًى دَائِمٍ. لأَنِّي أَغْمِزُ وَأَجْعَلُهُ يَرْكُضُ عَنْهُ. فَمَنْ هُوَ مُنْتَخَبٌ فَأُقِيمَهُ عَلَيْهِ؟ لأَنَّهُ مَنْ مِثْلِي وَمَنْ يُحَاكِمُنِي وَمَنْ هُوَ الرَّاعِي الَّذِي يَقِفُ أَمَامِي؟
\par 20 لِذَلِكَ اسْمَعُوا مَشُورَةَ الرَّبِّ الَّتِي قَضَى بِهَا عَلَى أَدُومَ وَأَفْكَارَهُ الَّتِي افْتَكَرَ بِهَا عَلَى سُكَّانِ تِيمَانَ. إِنَّ صِغَارَ الْغَنَمِ تَسْحَبُهُمْ. إِنَّهُ يَخْرِبُ مَسْكَنَهُمْ عَلَيْهِمْ.
\par 21 مِنْ صَوْتِ سُقُوطِهِمْ رَجَفَتِ الأَرْضُ. صَرْخَةٌ سُمِعَ صَوْتُهَا فِي بَحْرِ سُوفَ.
\par 22 هُوَذَا كَنَسْرٍ يَرْتَفِعُ وَيَطِيرُ وَيَبْسُطُ جَنَاحَيْهِ عَلَى بُصْرَةَ وَيَكُونُ قَلْبُ جَبَابِرَةِ أَدُومَ فِي ذَلِكَ الْيَوْمِ كَقَلْبِ امْرَأَةٍ مَاخِضٍ].
\par 23 عَنْ دِمَشْقَ: [خَزِيَتْ حَمَاةُ وَأَرْفَادُ. قَدْ ذَابُوا لأَنَّهُمْ قَدْ سَمِعُوا خَبَراً رَدِيئاً. فِي الْبَحْرِ اضْطِرَابٌ لاَ يَسْتَطِيعُ الْهُدُوءَ.
\par 24 ارْتَخَتْ دِمَشْقُ وَالْتَفَتَتْ لِلْهَرَبِ. أَمْسَكَتْهَا الرَِّعْدَةُ وَأَخَذَهَا الضِّيقُ وَالأَوْجَاعُ كَمَاخِضٍ.
\par 25 كَيْفَ لَمْ تُتْرَكِ الْمَدِينَةُ الشَّهِيرَةُ قَرْيَةُ فَرَحِي؟
\par 26 لِذَلِكَ تَسْقُطُ شُبَّانُهَا فِي شَوَارِعِهَا وَتَهْلِكُ كُلُّ رِجَالِ الْحَرْبِ فِي ذَلِكَ الْيَوْمِ يَقُولُ رَبُّ الْجُنُودِ.
\par 27 وَأُشْعِلُ نَاراً فِي سُورِ دِمَشْقَ فَتَأْكُلُ قُصُورَ بَنْهَدَدَ].
\par 28 عَنْ قِيدَارَ وَعَنْ مَمَالِكِ حَاصُورَ الَّتِي ضَرَبَهَا نَبُوخَذْنَصَّرُ مَلِكُ بَابِلَ: [هَكَذَا قَالَ الرَّبُّ. قُومُوا اصْعَدُوا إِلَى قِيدَارَ. اخْرِبُوا بَنِي الْمَشْرِقِ.
\par 29 يَأْخُذُونَ خِيَامَهُمْ وَغَنَمَهُمْ وَيَأْخُذُونَ لأَنْفُسِهِمْ شُقَقَهُمْ وَكُلَّ آنِيَتِهِمْ وَجِمَالَهُمْ وَيُنَادُونَ إِلَيْهِمِ: الْخَوْفَ مِنْ كُلِّ جَانِبٍ.
\par 30 [اُهْرُبُوا. انْهَزِمُوا جِدّاً. تَعَمَّقُوا فِي السَّكَنِ يَا سُكَّانَ حَاصُورَ يَقُولُ الرَّبُّ لأَنَّ نَبُوخَذْنَصَّرَ مَلِكَ بَابِلَ قَدْ أَشَارَ عَلَيْكُمْ مَشُورَةً وَفَكَّرَ عَلَيْكُمْ فِكْراً.
\par 31 قُومُوا اصْعَدُوا إِلَى أُمَّةٍ مُطْمَئِنَّةٍ سَاكِنَةٍ آمِنَةٍ يَقُولُ الرَّبُّ لاَ مَصَارِيعَ وَلاَ عَوَارِضَ لَهَا. تَسْكُنُ وَحْدَهَا.
\par 32 وَتَكُونُ جِمَالُهُمْ نَهْباً وَكَثْرَةُ مَاشِيَتِهِمْ غَنِيمَةً وَأُذْرِي لِكُلِّ رِيحٍ مَقْصُوصِي الشَّعْرِ مُسْتَدِيراً وَآتِي بِهَلاَكِهِمْ مِنْ كُلِّ جِهَاتِهِ يَقُولُ الرَّبُّ.
\par 33 وَتَكُونُ حَاصُورُ مَسْكَنَ بَنَاتِ آوَى وَخَرِبَةً إِلَى الأَبَدِ. لاَ يَسْكُنُ هُنَاكَ إِنْسَانٌ وَلاَ يَتَغَرَّبُ فِيهَا ابْنُ آدَمَ].
\par 34 كَلِمَةُ الرَّبِّ الَّتِي صَارَتْ إِلَى إِرْمِيَا النَّبِيِّ عَلَى عِيلاَمَ فِي ابْتِدَاءِ مُلْكِ صِدْقِيَّا مَلِكِ يَهُوذَا:
\par 35 [هَكَذَا قَالَ رَبُّ الْجُنُودِ: هَئَنَذَا أُحَطِّمُ قَوْسَ عِيلاَمَ أَوَّلَ قُوَّتِهِمْ.
\par 36 وَأَجْلِبُ عَلَى عِيلاَمَ أَرْبَعَ رِيَاحٍ مِنْ أَرْبَعَةِ أَطْرَافِ السَّمَاءِ وَأُذَرِّيهِمْ لِكُلِّ هَذِهِ الرِّيَاحِ وَلاَ تَكُونُ أُمَّةٌ إِلاَّ وَيَأْتِي إِلَيْهَا مَنْفِيُّو عِيلاَمَ.
\par 37 وَأَجْعَلُ الْعِيلاَمِيِّينَ يَرْتَعِبُونَ أَمَامَ أَعْدَائِهِمْ وَأَمَامَ طَالِبِي نُفُوسِهِمْ وَأَجْلِبُ عَلَيْهِمْ شَرّاً حُمُوَّ غَضَبِي يَقُولُ الرَّبُّ. وَأُرْسِلُ وَرَاءَهُمُ السَّيْفَ حَتَّى أَُفْنِيَهُمْ.
\par 38 وَأَضَعُ كُرْسِيِّي فِي عِيلاَمَ وَأُبِيدُ مِنْ هُنَاكَ الْمَلِكَ وَالرُّؤَسَاءَ يَقُولُ الرَّبُّ.
\par 39 [وَيَكُونُ فِي آخِرِ الأَيَّامِ أَنِّي أَرُدُّ سَبْيَ عِيلاَمَ يَقُولُ الرَّبُّ].

\chapter{50}

\par 1 اَلْكَلِمَةُ الَّتِي تَكَلَّمَ بِهَا الرَّبُّ عَنْ بَابِلَ وَعَنْ أَرْضِ الْكِلْدَانِيِّينَ عَلَى يَدِ إِرْمِيَا النَّبِيِّ:
\par 2 [أَخْبِرُوا فِي الشُّعُوبِ وَأَسْمِعُوا وَارْفَعُوا رَايَةً. أَسْمِعُوا لاَ تُخْفُوا. قُولُوا: أُخِذَتْ بَابِلُ. خَزِيَ بِيلُ. انْسَحَقَ مَرُودَخُ. خَزِيَتْ أَوْثَانُهَا انْسَحَقَتْ أَصْنَامُهَا.
\par 3 لأَنَّهُ قَدْ طَلَعَتْ عَلَيْهَا أُمَّةٌ مِنَ الشِّمَالِ تَجْعَلُ أَرْضَهَا خَرِبَةً فَلاَ يَكُونُ فِيهَا سَاكِنٌ. مِنْ إِنْسَانٍ إِلَى حَيَوَانٍ هَرَبُوا وَذَهَبُوا.
\par 4 [فِي تِلْكَ الأَيَّامِ وَفِي ذَلِكَ الزَّمَانِ يَقُولُ الرَّبُّ يَأْتِي بَنُو إِسْرَائِيلَ هُمْ وَبَنُو يَهُوذَا مَعاً. يَسِيرُونَ سَيْراً وَيَبْكُونَ وَيَطْلُبُونَ الرَّبَّ إِلَهَهُمْ.
\par 5 يَسْأَلُونَ عَنْ طَرِيقِ صِهْيَوْنَ وَوُجُوهُهُمْ إِلَى هُنَاكَ قَائِلِينَ: هَلُمَّ فَنَلْصَقُ بِالرَّبِّ بِعَهْدٍ أَبَدِيٍّ لاَ يُنْسَى.
\par 6 كَانَ شَعْبِي خِرَافاً ضَالَّةً قَدْ أَضَلَّتْهُمْ رُعَاتُهُمْ. عَلَى الْجِبَالِ أَتَاهُوهُمْ. سَارُوا مِنْ جَبَلٍ إِلَى أَكَمَةٍ. نَسُوا مَرْبِضَهُمْ.
\par 7 كُلُّ الَّذِينَ وَجَدُوهُمْ أَكَلُوهُمْ وَقَالَ مُبْغِضُوهُمْ: لاَ نُذْنِبُ مِنْ أَجْلِ أَنَّهُمْ أَخْطَأُوا إِلَى الرَّبِّ مَسْكَنِ الْبِرِّ وَرَجَاءِ آبَائِهِمِ الرَّبِّ.
\par 8 اُهْرُبُوا مِنْ وَسَطِ بَابِلَ وَاخْرُجُوا مِنْ أَرْضِ الْكِلْدَانِيِّينَ وَكُونُوا مِثْلَ كَرَارِيزَ أَمَامَ الْغَنَمِ.
\par 9 [لأَنِّي هَئَنَذَا أُوقِظُ وَأُصْعِدُ عَلَى بَابِلَ جُمْهُورَ شُعُوبٍ عَظِيمَةٍ مِنْ أَرْضِ الشِّمَالِ فَيَصْطَفُّونَ عَلَيْهَا. مِنْ هُنَاكَ تُؤْخَذُ. نِبَالُهُمْ كَبَطَلٍ مُهْلِكٍ لاَ يَرْجِعُ فَارِغاً.
\par 10 وَتَكُونُ أَرْضُ الْكِلْدَانِيِّينَ غَنِيمَةً. كُلُّ مُغْتَنِمِيهَا يَشْبَعُونَ يَقُولُ الرَّبُّ.
\par 11 لأَنَّكُمْ قَدْ فَرِحْتُمْ وَشَمِتُّمْ يَا نَاهِبِي مِيرَاثِي وَقَفَزْتُمْ كَعِجْلَةٍ فِي الْكَلَإِ وَصَهَلْتُمْ كَخَيْلٍ
\par 12 تَخْزَى أُمُّكُمْ جِدّاً. تَخْجَلُ الَّتِي وَلَدَتْكُمْ. هَا آخِرَةُ الشُّعُوبِ بَرِّيَّةٌ وَأَرْضٌ نَاشِفَةٌ وَقَفْرٌ.
\par 13 بِسَبَبِ سَخَطِ الرَّبِّ لاَ تُسْكَنُ بَلْ تَصِيرُ خَرِبَةً بِالتَّمَامِ. كُلُّ مَارٍّ بِبَابِلَ يَتَعَجَّبُ وَيَصْفِرُ بِسَبَبِ كُلِّ ضَرَبَاتِهَا.
\par 14 اِصْطَفُّوا عَلَى بَابِلَ حَوَالَيْهَا يَا جَمِيعَ الَّذِينَ يَنْزِعُونَ فِي الْقَوْسِ. ارْمُوا عَلَيْهَا. لاَ تُوَفِّرُوا السِّهَامَ لأَنَّهَا قَدْ أَخْطَأَتْ إِلَى الرَّبِّ.
\par 15 اهْتِفُوا عَلَيْهَا حَوَالَيْهَا. قَدْ أَعْطَتْ يَدَهَا. سَقَطَتْ أُسُسُهَا نُقِضَتْ أَسْوَارُهَا. لأَنَّهَا نَقْمَةُ الرَّبِّ هِيَ فَانْتَقِمُوا مِنْهَا. كَمَا فَعَلَتِ افْعَلُوا بِهَا.
\par 16 اقْطَعُوا الزَّارِعَ مِنْ بَابِلَ وَمَاسِكَ الْمِنْجَلِ فِي وَقْتِ الْحَصَادِ. مِنْ وَجْهِ السَّيْفِ الْقَاسِي يَرْجِعُونَ كُلُّ وَاحِدٍ إِلَى شَعْبِهِ وَيَهْرُبُونَ كُلُّ وَاحِدٍ إِلَى أَرْضِهِ.
\par 17 [إِسْرَائِيلُ غَنَمٌ مُتَبَدِّدَةٌ. قَدْ طَرَدَتْهُ السِّبَاعُ. أَوَّلاً أَكَلَهُ مَلِكُ أَشُّورَ ثُمَّ هَذَا الأَخِيرُ نَبُوخَذْنَصَّرُ مَلِكُ بَابِلَ هَرَسَ عِظَامَهُ.
\par 18 لِذَلِكَ هَكَذَا قَالَ رَبُّ الْجُنُودِ إِلَهُ إِسْرَائِيلَ: هَئَنَذَا أُعَاقِبُ مَلِكَ بَابِلَ وَأَرْضَهُ كَمَا عَاقَبْتُ مَلِكَ أَشُّورَ.
\par 19 وَأَرُدُّ إِسْرَائِيلَ إِلَى مَسْكَنِهِ فَيَرْعَى كَرْمَلَ وَبَاشَانَ وَفِي جَبَلِ أَفْرَايِمَ وَجِلْعَادَ تَشْبَعُ نَفْسُهُ.
\par 20 فِي تِلْكَ الأَيَّامِ وَفِي ذَلِكَ الزَّمَانِ يَقُولُ الرَّبُّ يُطْلَبُ إِثْمُ إِسْرَائِيلَ فَلاَ يَكُونُ وَخَطِيَّةُ يَهُوذَا فَلاَ تُوجَدُ لأَنِّي أَغْفِرُ لِمَنْ أُبْقِيهِ.
\par 21 [اِصْعَدْ عَلَى أَرْضِ مِرَاثَايِمَ. عَلَيْهَا وَعَلَى سُكَّانِ فَقُودَ. اخْرِبْ وَحَرِّمْ وَرَاءَهُمْ يَقُولُ الرَّبُّ وَافْعَلْ حَسَبَ كُلِّ مَا أَمَرْتُكَ بِهِ.
\par 22 صَوْتُ حَرْبٍ فِي الأَرْضِ وَانْحِطَامٌ عَظِيمٌ.
\par 23 كَيْفَ قُطِعَتْ وَتَحَطَّمَتْ مِطْرَقَةُ كُلِّ الأَرْضِ؟ كَيْفَ صَارَتْ بَابِلُ خَرِبَةً بَيْنَ الشُّعُوبِ؟
\par 24 قَدْ نَصَبْتُ لَكِ شَرَكاً فَعَلِقْتِ يَا بَابِلُ وَأَنْتِ لَمْ تَعْرِفِي! قَدْ وُجِدْتِ وَأُمْسِكْتِ لأَنَّكِ قَدْ خَاصَمْتِ الرَّبَّ.
\par 25 فَتَحَ الرَّبُّ خِزَانَتَهُ وَأَخْرَجَ آلاَتِ رَجَزِهِ لأَنَّ لِلسَّيِّدِ رَبِّ الْجُنُودِ عَمَلاً فِي أَرْضِ الْكِلْدَانِيِّينَ.
\par 26 هَلُمَّ إِلَيْهَا مِنَ الأَقْصَى. افْتَحُوا أَهْرَاءَهَا. كَوِّمُوهَا عِرَاماً وَحَرِّمُوهَا وَلاَ تَكُنْ لَهَا بَقِيَّةٌ.
\par 27 أَهْلِكُوا كُلَّ عُجُولِهَا. لِتَنْزِلْ لِلذَّبْحِ. وَيْلٌ لَهُمْ لأَنَّهُ قَدْ أَتَى يَوْمُهُمْ زَمَانُ عِقَابِهِمْ.
\par 28 صَوْتُ هَارِبِينَ وَنَاجِينَ مِنْ أَرْضِ بَابِلَ لِيُخْبِرُوا فِي صِهْيَوْنَ بِنَقْمَةِ الرَّبِّ إِلَهِنَا نَقْمَةِ هَيْكَلِهِ.
\par 29 اُدْعُوا إِلَى بَابِلَ أَصْحَابَ الْقِسِيِّ. لِيَنْزِلْ عَلَيْهَا كُلُّ مَنْ يَنْزِعُ فِي الْقَوْسِ حَوَالَيْهَا. لاَ يَكُنْ نَاجٍ. كَافِئُوهَا نَظِيرَ عَمَلِهَا. افْعَلُوا بِهَا حَسَبَ كُلِّ مَا فَعَلَتْ. لأَنَّهَا بَغَتْ عَلَى الرَّبِّ عَلَى قُدُّوسِ إِسْرَائِيلَ.
\par 30 لِذَلِكَ يَسْقُطُ شُبَّانُهَا فِي الشَّوَارِعِ وَكُلُّ رِجَالِ حَرْبِهَا يَهْلِكُونَ فِي ذَلِكَ الْيَوْمِ يَقُولُ الرَّبُّ.
\par 31 هَئَنَذَا عَلَيْكِ أَيَّتُهَا الْبَاغِيَةُ يَقُولُ السَّيِّدُ رَبُّ الْجُنُودِ لأَنَّهُ قَد أَتَى يَوْمُكِ حِينَ عِقَابِي إِيَّاكِ.
\par 32 فَيَعْثُرُ الْبَاغِي وَيَسقُطُ ولاَ يَكُونُ لَهُ مَنْ يُقِيمُهُ وَأُشْعِلُ نَاراً في مُدُنِهِ فَتَأْكُلُ كُلَّ مَا حَوَالَيْهَا.
\par 33 [هَكَذَا قَالَ رَبُّ الْجُنُودِ: إِنَّ بَنِي إِسْرَائِيلَ وَبَنِي يَهُوذَا مَعاً مَظْلُومُونَ وَكُلُّ الَّذِينَ سَبُوهُمْ أَمْسَكُوهُمْ. أَبُوا أَنْ يُطْلِقُوهُمْ.
\par 34 وَلِيُّهُمْ قَوِيٌّ. رَبُّ الْجُنُودِ اسْمُهُ. يُقِيمُ دَعْوَاهُمْ لِيُرِيحَ الأَرْضَ وَيُزْعِجَ سُكَّانَ بَابِلَ.
\par 35 سَيْفٌ عَلَى الْكِلْدَانِيِّينَ يَقُولُ الرَّبُّ وَعَلَى سُكَّانِ بَابِلَ وَعَلَى رُؤَسَائِهَا وَعَلَى حُكَمَائِهَا.
\par 36 سَيْفٌ عَلَى الْمُخَادِعِينَ فَيَصِيرُونَ حُمُقاً. سَيْفٌ عَلَى أَبْطَالِهَا فَيَرْتَعِبُونَ.
\par 37 سَيْفٌ عَلَى خَيْلِهَا وَعَلَى مَرْكَبَاتِهَا وَعَلَى كُلِّ اللَّفِيفِ الَّذِي فِي وَسَطِهَا فَيَصِيرُونَ نِسَاءً. سَيْفٌ عَلَى خَزَائِنِهَا فَتُنْهَبُ.
\par 38 حَرٌّ عَلَى مِيَاهِهَا فَتَنْشَفُ لأَنَّهَا أَرْضُ مَنْحُوتَاتٍ هِيَ وَبِالأَصْنَامِ تُجَنُّ.
\par 39 لِذَلِكَ تَسْكُنُ وُحُوشُ الْقَفْرِ مَعَ بَنَاتِ آوَى وَتَسْكُنُ فِيهَا رِعَالُ النَّعَامِ وَلاَ تُسْكَنُ بَعْدُ إِلَى الأَبَدِ وَلاَ تُعْمَرُ إِلَى دَوْرٍ فَدَوْرٍ.
\par 40 كَقَلْبِ اللَّهِ سَدُومَ وَعَمُورَةَ وَمُجَاوَرَاتِهَا يَقُولُ الرَّبُّ لاَ يَسْكُنُ هُنَاكَ إِنْسَانٌ وَلاَ يَتَغَرَّبُ فِيهَا ابْنُ آدَمَ.
\par 41 هُوَذَا شَعْبٌ مُقْبِلٌ مِنَ الشِّمَالِ وَأُمَّةٌ عَظِيمَةٌ وَيُوقَظُ مُلُوكٌ كَثِيرُونَ مِنْ أَقَاصِي الأَرْضِ.
\par 42 يُمْسِكُونَ الْقَوْسَ وَالرُّمْحَ. هُمْ قُسَاةٌ لاَ يَرْحَمُونَ. صَوْتُهُمْ يَعِجُّ كَبَحْرٍ وَعَلَى خَيْلٍ يَرْكَبُونَ مُصْطَفِّينَ كَرَجُلٍ وَاحِدٍ لِمُحَارَبَتِكِ يَا بِنْتَ بَابِلَ.
\par 43 سَمِعَ مَلِكُ بَابِلَ خَبَرَهُمْ فَارْتَخَتْ يَدَاهُ. أَخَذَتْهُ الضِّيقَةُ وَالْوَجَعُ كَمَاخِضٍ.
\par 44 هَا هُوَ يَصْعَدُ كَأَسَدٍ مِنْ كِبْرِيَاءِ الأُرْدُنِّ إِلَى مَرْعًى دَائِمٍ. لأَنِّي أَغْمِزُ وَأَجْعَلُهُمْ يَرْكُضُونَ عَنْهُ. فَمَنْ هُوَ مُنْتَخَبٌ فَأُقِيمَهُ عَلَيْهِ؟ لأَنَّهُ مَنْ مِثْلِي وَمَنْ يُحَاكِمُنِي وَمَنْ هُوَ الرَّاعِي الَّذِي يَقِفُ أَمَامِي؟
\par 45 لِذَلِكَ اسْمَعُوا مَشُورَةَ الرَّبِّ الَّتِي قَضَى بِهَا عَلَى بَابِلَ وَأَفْكَارَهُ الَّتِي افْتَكَرَ بِهَا عَلَى أَرْضِ الْكِلْدَانِيِّينَ. إِنَّ صِغَارَ الْغَنَمِ تَسْحَبُهُمْ. إِنَّهُ يَخْرِبُ مَسْكَنَهُمْ عَلَيْهِمْ.
\par 46 مِنَ الْقَوْلِ أُخِذَتْ بَابِلُ. رَجَفَتِ الأَرْضُ وَسُمِعَ صُرَاخٌ فِي الشُّعُوبِ].

\chapter{51}

\par 1 هَكَذَا قَالَ الرَّبُّ: [هَئَنَذَا أُوقِظُ عَلَى بَابِلَ وَعَلَى السَّاكِنِينَ فِي وَسَطِ الْقَائِمِينَ عَلَيَّ رِيحاً مُهْلِكَةً.
\par 2 وَأُرْسِلُ إِلَى بَابِلَ مُذَرِّينَ فَيُذَرُّونَهَا وَيُفَرِّغُونَ أَرْضَهَا لأَنَّهُمْ يَكُونُونَ عَلَيْهَا مِنْ كُلِّ جِهَةٍ فِي يَوْمِ الشَّرِّ.
\par 3 عَلَى النَّازِعِ فِي قَوْسِهِ فَلْيَنْزِعِ النَّازِعُ وَعَلَى الْمُفْتَخِرِ بِدِرْعِهِ فَلاَ تُشْفِقُوا عَلَى مُنْتَخَبِيهَا بَلْ حَرِّمُوا كُلَّ جُنْدِهَا.
\par 4 فَتَسْقُطَ الْقَتْلَى فِي أَرْضِ الْكِلْدَانِيِّينَ وَالْمَطْعُونُونَ فِي شَوَارِعِهَا.
\par 5 لأَنَّ إِسْرَائِيلَ وَيَهُوذَا لَيْسَا بِمَقْطُوعَيْنِ عَنْ إِلَهِهِمَا عَنْ رَبِّ الْجُنُودِ وَإِنْ تَكُنْ أَرْضُهُمَا مَلآنَةً إِثْماً عَلَى قُدُّوسِ إِسْرَائِيلَ.
\par 6 اهْرُبُوا مِنْ وَسَطِ بَابِلَ وَانْجُوا كُلُّ وَاحِدٍ بِنَفْسِهِ. لاَ تَهْلِكُوا بِذَنْبِهَا لأَنَّ هَذَا زَمَانُ انْتِقَامِ الرَّبِّ. هُوَ يُؤَدِّي لَهَا جَزَاءَهَا.
\par 7 بَابِلُ كَأْسُ ذَهَبٍ بِيَدِ الرَّبِّ تُسْكِرُ كُلَّ الأَرْضِ. مِنْ خَمْرِهَا شَرِبَتِ الشُّعُوبُ. مِنْ أَجْلِ ذَلِكَ جُنَّتِ الشُّعُوبُ.
\par 8 سَقَطَتْ بَابِلُ بَغْتَةً وَتَحَطَّمَتْ. وَلْوِلُوا عَلَيْهَا. خُذُوا بَلَسَاناً لِجُرْحِهَا لَعَلَّهَا تُشْفَى.
\par 9 دَاوَيْنَا بَابِلَ فَلَمْ تُشْفَ. دَعُوهَا وَلْنَذْهَبْ كُلُّ وَاحِدٍ إِلَى أَرْضِهِ لأَنَّ قَضَاءَهَا وَصَلَ إِلَى السَّمَاءِ وَارْتَفَعَ إِلَى السَّحَابِ.
\par 10 قَدْ أَخْرَجَ الرَّبُّ بِرَّنَا. هَلُمَّ فَنَقُصُّ فِي صِهْيَوْنَ عَمَلَ الرَّبِّ إِلَهِنَا.
\par 11 سُنُّوا السِّهَامَ. أَعِدُّوا الأَتْرَاسَ. قَدْ أَيْقَظَ الرَّبُّ رُوحَ مُلُوكِ مَادِي لأَنَّ قَصْدَهُ عَلَى بَابِلَ أَنْ يُهْلِكَهَا. لأَنَّهُ نَقْمَةُ الرَّبِّ. نَقْمَةُ هَيْكَلِهِ.
\par 12 عَلَى أَسْوَارِ بَابِلَ ارْفَعُوا الرَّايَةَ. شَدِّدُوا الْحِرَاسَةَ. أَقِيمُوا الْحُرَّاسَ. أَعِدُّوا الْكَمِينَ لأَنَّ الرَّبَّ قَدْ قَصَدَ وَأَيْضاً فَعَلَ مَا تَكَلَّمَ بِهِ عَلَى سُكَّانِ بَابِلَ.
\par 13 أَيَّتُهَا السَّاكِنَةُ عَلَى مِيَاهٍ كَثِيرَةٍ الْوَافِرَةُ الْخَزَائِنِ قَدْ أَتَتْ آخِرَتُكِ كَيْلُ اغْتِصَابِكِ.
\par 14 قَدْ حَلَفَ رَبُّ الْجُنُودِ بِنَفْسِهِ: إِنِّي لَأَمْلَأَنَّكِ أُنَاساً كَالْغَوْغَاءِ فَيَرْفَعُونَ عَلَيْكِ جَلَبَةً.
\par 15 [صَانِعُ الأَرْضِ بِقُوَّتِهِ وَمُؤَسِّسُ الْمَسْكُونَةِ بِحِكْمَتِهِ وبِفَهْمِهِ مَدَّ السَّمَاوَاتِ.
\par 16 إِذَا أَعْطَى قَوْلاً تَكُونُ كَثْرَةُ مِيَاهٍ فِي السَّمَاوَاتِ وَيُصْعِدُ السَّحَابَ مِنْ أَقَاصِي الأَرْضِ. صَنَعَ بُرُوقاً لِلْمَطَرِ وَأَخْرَجَ الرِّيحَ مِنْ خَزَائِنِهِ.
\par 17 بَلِدَ كُلُّ إِنْسَانٍ بِمَعْرِفَتِهِ. خَزِيَ كُلُّ صَائِغٍ مِنَ التِّمْثَالِ لأَنَّ مَسْبُوكَهُ كَذِبٌ وَلاَ رُوحٍ فِيهِ.
\par 18 هِيَ بَاطِلَةٌ صَنْعَةُ الأَضَالِيلِ. فِي وَقْتِ عِقَابِهَا تَبِيدُ.
\par 19 لَيْسَ كَهَذِهِ نَصِيبُ يَعْقُوبَ لأَنَّهُ مُصَوِّرُ الْجَمِيعِ وَقَضِيبُ مِيرَاثِهِ رَبُّ الْجُنُودِ اسْمُهُ.
\par 20 أَنْتَ لِي فَأْسٌ وَأَدَوَاتُ حَرْبٍ فَأَسْحَقُ بِكَ الأُمَمَ وَأُهْلِكُ بِكَ الْمَمَالِكَ
\par 21 وَأُكَسِّرُ بِكَ الْفَرَسَ وَرَاكِبَهُ وَأَسْحَقُ بِكَ الْمَرْكَبَةَ وَرَاكِبَهَا
\par 22 وَأَسْحَقُ بِكَ الرَّجُلَ وَالْمَرْأَةَ وَأَسْحَقُ بِكَ الشَّيْخَ وَالْفَتَى وَأَسْحَقُ بِكَ الْغُلاَمَ وَالْعَذْرَاءَ
\par 23 وَأَسْحَقُ بِكَ الرَّاعِيَ وَقَطِيعَهُ وَأَسْحَقُ بِكَ الْفَلاَّحَ وَفَدَّانَهُ وَأَسْحَقُ بِكَ الْوُلاَةَ وَالْحُكَّامَ.
\par 24 وَأُكَافِئُ بَابِلَ وَكُلَّ سُكَّانِ أَرْضِ الْكِلْدَانِيِّينَ عَلَى كُلِّ شَرِّهِمِ الَّذِي فَعَلُوهُ فِي صِهْيَوْنَ أَمَامَ عُيُونِكُمْ يَقُولُ الرَّبُّ.
\par 25 هَئَنَذَا عَلَيْكَ أَيُّهَا الْجَبَلُ الْمُهْلِكُ يَقُولُ الرَّبُّ الْمُهْلِكُ كُلَّ الأَرْضِ فَأَمُدُّ يَدِي عَلَيْكَ وَأُدَحْرِجُكَ عَنِ الصُّخُورِ وَأَجْعَلُكَ جَبَلاً مُحْرَقاً
\par 26 فَلاَ يَأْخُذُونَ مِنْكَ حَجَراً لِزَاوِيَةٍ وَلاَ حَجَراً لأُسُسٍ بَلْ تَكُونُ خَرَاباً إِلَى الأَبَدِ يَقُولُ الرَّبُّ.
\par 27 [اِرْفَعُوا الرَّايَةَ فِي الأَرْضِ. اضْرِبُوا بِالْبُوقِ فِي الشُّعُوبِ. قَدِّسُوا عَلَيْهَا الأُمَمَ. نَادُوا عَلَيْهَا مَمَالِكَ أَرَارَاطَ وَمِنِّي وَأَشْكَنَازَ. أَقِيمُوا عَلَيْهَا قَائِداً. أَصْعِدُوا الْخَيْلَ كَغَوْغَاءَ مُقْشَعِرَّةٍ.
\par 28 قَدِّسُوا عَلَيْهَا الشُّعُوبَ مُلُوكَ مَادِي وُلاَتَهَا وَكُلَّ حُكَّامِهَا وَكُلَّ أَرْضِ سُلْطَانِهَا.
\par 29 فَتَرْتَجِفُ الأَرْضُ وَتَتَوَجَّعُ لأَنَّ أَفْكَارَ الرَّبِّ تَقُومُ عَلَى بَابِلَ لِيَجْعَلَ أَرْضَ بَابِلَ خَرَاباً بِلاَ سَاكِنٍ.
\par 30 كَفَّ جَبَابِرَةُ بَابِلَ عَنِ الْحَرْبِ وَجَلَسُوا فِي الْحُصُونِ. نَضَبَتْ شَجَاعَتُهُمْ. صَارُوا نِسَاءً. حَرَقُوا مَسَاكِنَهَا. تَحَطَّمَتْ عَوَارِضُهَا.
\par 31 يَرْكُضُ عَدَّاءٌ لِلِقَاءِ عَدَّاءٍ وَمُخْبِرٌ لِلِقَاءِ مُخْبِرٍ لِيُخْبِرَ مَلِكَ بَابِلَ بِأَنَّ مَدِينَتَهُ قَدْ أُخِذَتْ عَنْ أَقْصَى
\par 32 وَأَنَّ الْمَعَابِرَ قَدْ أُمْسِكَتْ وَالْقَصَبَ أَحْرَقُوهُ بِالنَّارِ وَرِجَالُ الْحَرْبِ اضْطَرَبَتْ.
\par 33 لأَنَّهُ هَكَذَا قَالَ رَبُّ الْجُنُودِ إِلَهُ إِسْرَائِيلَ إِنَّ بِنْتَ بَابِلَ كَبَيْدَرٍ وَقْتَ دَوْسِهِ. بَعْدَ قَلِيلٍ يَأْتِي عَلَيْهَا وَقْتُ الْحَصَادِ].
\par 34 [أَكَلَنِي أَفْنَانِي نَبُوخَذْنَصَّرُ مَلِكُ بَابِلَ. جَعَلَنِي إِنَاءً فَارِغاً. ابْتَلَعَنِي كَتِنِّينٍ وَمَلَأَ جَوْفَهُ مِنْ نِعَمِي. طَوَّحَنِي.
\par 35 ظُلْمِي وَلَحْمِي عَلَى بَابِلَ تَقُولُ سَاكِنَةُ صِهْيَوْنَ وَدَمِي عَلَى سُكَّانِ أَرْضِ الْكِلْدَانِيِّينَ تَقُولُ أُورُشَلِيمُ.
\par 36 لِذَلِكَ هَكَذَا قَالَ الرَّبُّ: هَئَنَذَا أُخَاصِمُ خُصُومَتَكِ وَأَنْتَقِمُ نَقْمَتَكِ وَأُنَشِّفُ بَحْرَهَا وَأُجَفِّفُ يَنْبُوعَهَا.
\par 37 وَتَكُونُ بَابِلُ كُوَماً وَمَأْوَى بَنَاتِ آوَى وَدَهَشاً وَصَفِيراً بِلاَ سَاكِنٍ.
\par 38 يُزَمْجِرُونَ مَعاً كَأَشْبَالٍ. يَزْأَرُونَ كَجِرَاءِ أُسُودٍ.
\par 39 عِنْدَ حَرَارَتِهِمْ أُعِدُّ لَهُمْ شَرَاباً وَأُسْكِرُهُمْ لِيَفْرَحُوا وَيَنَامُوا نَوْماً أَبَدِيّاً وَلاَ يَسْتَيْقِظُوا يَقُولُ الرَّبُّ.
\par 40 أُنَزِّلُهُمْ كَخِرَافٍ لِلذَّبْحِ وَكَكِبَاشٍ مَعَ أَعْتِدَةٍ.
\par 41 كَيْفَ أُخِذَتْ شِيشَكُ وَأُمْسِكَتْ فَخْرُ كُلِّ الأَرْضِ؟ كَيْفَ صَارَتْ بَابِلُ دَهَشاً فِي الشُّعُوبِ؟
\par 42 طَلَعَ الْبَحْرُ عَلَى بَابِلَ فَتَغَطَّتْ بِكَثْرَةِ أَمْوَاجِهِ.
\par 43 صَارَتْ مُدُنُهَا خَرَاباً أَرْضاً نَاشِفَةً وَقَفْراً أَرْضاً لاَ يَسْكُنُ فِيهَا إِنْسَانٌ وَلاَ يَعْبُرُ فِيهَا ابْنُ آدَمَ.
\par 44 وَأُعَاقِبُ بِيلَ فِي بَابِلَ وَأُخْرِجُ مِنْ فَمِهِ مَا ابْتَلَعَهُ فَلاَ تَجْرِي إِلَيْهِ الشُّعُوبُ بَعْدُ وَيَسْقُطُ سُورُ بَابِلَ أَيْضاً.
\par 45 اُخْرُجُوا مِنْ وَسَطِهَا يَا شَعْبِي وَلْيُنَجِّ كُلُّ وَاحِدٍ نَفْسَهُ مِنْ حُمُوِّ غَضَبِ الرَّبِّ.
\par 46 وَلاَ يَضْعُفْ قَلْبُكُمْ فَتَخَافُوا مِنَ الْخَبَرِ الَّذِي سُمِعَ فِي الأَرْضِ فَإِنَّهُ يَأْتِي خَبَرٌ فِي هَذِهِ السَّنَةِ ثُمَّ بَعْدَهُ فِي السَّنَةِ الأُخْرَى خَبَرٌ وَظُلْمٌ فِي الأَرْضِ مُتَسَلِّطٌ عَلَى مُتَسَلِّطٍ.
\par 47 لِذَلِكَ هَا أَيَّامٌ تَأْتِي وَأُعَاقِبُ مَنْحُوتَاتِ بَابِلَ فَتَخْزَى كُلُّ أَرْضِهَا وَتَسْقُطُ كُلُّ قَتْلاَهَا فِي وَسَطِهَا.
\par 48 فَتَهْتِفُ عَلَى بَابِلَ السَّمَاوَاتُ وَالأَرْضُ وَكُلُّ مَا فِيهَا لأَنَّ النَّاهِبِينَ يَأْتُونَ عَلَيْهَا مِنَ الشِّمَالِ يَقُولُ الرَّبُّ.
\par 49 كَمَا أَسْقَطَتْ بَابِلُ قَتْلَى إِسْرَائِيلَ تَسْقُطُ أَيْضاً قَتْلَى بَابِلَ فِي كُلِّ الأَرْضِ.
\par 50 أَيُّهَا النَّاجُونَ مِنَ السَّيْفِ اذْهَبُوا. لاَ تَقِفُوا. اذْكُرُوا الرَّبَّ مِنْ بَعِيدٍ وَلْتَخْطُرْ أُورُشَلِيمُ بِبَالِكُمْ.
\par 51 قَدْ خَزِينَا لأَنَّنَا قَدْ سَمِعْنَا عَاراً. غَطَّى الْخَجَلُ وُجُوهَنَا لأَنَّ الْغُرَبَاءَ قَدْ دَخَلُوا مَقَادِسَ بَيْتِ الرَّبِّ.
\par 52 لِذَلِكَ هَا أَيَّامٌ تَأْتِي يَقُولُ الرَّبُّ وَأُعَاقِبُ مَنْحُوتَاتِهَا وَيَتَنَهَّدُ الْجَرْحَى فِي كُلِّ أَرْضِهَا.
\par 53 فَلَوْ صَعِدَتْ بَابِلُ إِلَى السَّمَاوَاتِ وَلَوْ حَصَّنَتْ عَلْيَاءَ عِزِّهَا فَمِنْ عِنْدِي يَأْتِي عَلَيْهَا النَّاهِبُونَ يَقُولُ الرَّبُّ.
\par 54 [صَوْتُ صُرَاخٍ مِنْ بَابِلَ وَانْحِطَامٌ عَظِيمٌ مِنْ أَرْضِ الْكِلْدَانِيِّينَ
\par 55 لأَنَّ الرَّبَّ مُخْرِبٌ بَابِلَ وَقَدْ أَبَادَ مِنْهَا الصَّوْتَ الْعَظِيمَ وَقَدْ عَجَّتْ أَمْوَاجُهُمْ كَمِيَاهٍ كَثِيرَةٍ وَأُطْلِقَ ضَجِيجُ صَوْتِهِمْ.
\par 56 لأَنَّهُ جَاءَ عَلَى بَابِلَ الْمُخْرِبُ وَأُخِذَ جَبَابِرَتُهَا وَتَحَطَّمَتْ قِسِيُّهُمْ لأَنَّ الرَّبَّ إِلَهُ مُجَازَاةٍ يُكَافِئُ مُكَافَأَةً.
\par 57 وَأُسْكِرُ رُؤَسَاءَهَا وَحُكَمَاءَهَا وَوُلاَتَهَا وَحُكَّامَهَا وَأَبْطَالَهَا فَيَنَامُونَ نَوْماً أَبَدِيّاً وَلاَ يَسْتَيْقِظُونَ يَقُولُ الْمَلِكُ رَبُّ الْجُنُودِ اسْمُهُ.
\par 58 هَكَذَا قَالَ رَبُّ الْجُنُودِ إِنَّ أَسْوَارَ بَابِلَ الْعَرِيضَةَ تُدَمَّرُ تَدْمِيراً وَأَبْوَابُهَا الشَّامِخَةَ تُحْرَقُ بِالنَّارِ فَتَتْعَبُ الشُّعُوبُ لِلْبَاطِلِ وَالْقَبَائِلُ لِلنَّارِ حَتَّى تَعْيَا].
\par 59 اَلأَمْرُ الَّذِي أَوْصَى بِهِ إِرْمِيَا النَّبِيُّ سَرَايَا بْنَ نِيرِيَّا بْنِ مَحْسِيَّا عِنْدَ ذَهَابِهِ مَعَ صِدْقِيَّا مَلِكِ يَهُوذَا إِلَى بَابِلَ فِي السَّنَةِ الرَّابِعَةِ لِمُلْكِهِ. (وَكَانَ سَرَايَا رَئِيسَ الْمَحَلَّةِ)
\par 60 فَكَتَبَ إِرْمِيَا كُلَّ الشَّرِّ الآتِي عَلَى بَابِلَ فِي سِفْرٍ وَاحِدٍ كُلَّ هَذَا الْكَلاَمِ الْمَكْتُوبِ عَلَى بَابِلَ
\par 61 وَقَالَ إِرْمِيَا لِسَرَايَا: [إِذَا دَخَلْتَ إِلَى بَابِلَ وَنَظَرْتَ وَقَرَأْتَ كُلَّ هَذَا الْكَلاَمِ
\par 62 فَقُلْ: أَنْتَ يَا رَبُّ قَدْ تَكَلَّمْتَ عَلَى هَذَا الْمَوْضِعِ لِتَقْرِضَهُ حَتَّى لاَ يَكُونَ فِيهِ سَاكِنٌ مِنَ النَّاسِ إِلَى الْبَهَائِمِ بَلْ يَكُونُ خِرَباً أَبَدِيَّةً.
\par 63 وَيَكُونُ إِذَا فَرَغْتَ مِنْ قِرَاءَةِ هَذَا السِّفْرِ أَنَّكَ تَرْبُطُ بِهِ حَجَراً وَتَطْرَحُهُ إِلَى وَسَطِ الْفُرَاتِ
\par 64 وَتَقُولُ: هَكَذَا تَغْرَقُ بَابِلُ وَلاَ تَقُومُ مِنَ الشَّرِّ الَّذِي أَنَا جَالِبُهُ عَلَيْهَا وَيَعْيُونَ]. إِلَى هُنَا كَلاَمُ إِرْمِيَا.

\chapter{52}

\par 1 كَانَ صِدْقِيَّا ابْنَ إِحْدَى وَعِشْرِينَ سَنَةً حِينَ مَلَكَ وَمَلَكَ إِحْدَى عَشَرَةَ سَنَةً فِي أُورُشَلِيمَ وَاسْمُ أُمِّهِ حَمِيطَلُ بِنْتُ إِرْمِيَا مِنْ لِبْنَةَ.
\par 2 وَعَمِلَ الشَّرَّ فِي عَيْنَيِ الرَّبِّ حَسَبَ كُلِّ مَا عَمِلَ يَهُويَاقِيمُ.
\par 3 لأَنَّهُ لأَجْلِ غَضَبِ الرَّبِّ عَلَى أُورُشَلِيمَ وَيَهُوذَا حَتَّى طَرَحَهُمْ مِنْ أَمَامِ وَجْهِهِ. وَكَانَ أَنَّ صِدْقِيَّا تَمَرَّدَ عَلَى مَلِكِ بَابِلَ.
\par 4 وَفِي السَّنَةِ التَّاسِعَةِ لِمُلْكِهِ فِي الشَّهْرِ الْعَاشِرِ فِي عَاشِرِ الشَّهْرِ جَاءَ نَبُوخَذْنَصَّرُ مَلِكُ بَابِلَ هُوَ وَكُلُّ جَيْشِهِ عَلَى أُورُشَلِيمَ وَنَزَلُوا عَلَيْهَا وَبَنُوا عَلَيْهَا أَبْرَاجاً حَوَالَيْهَا.
\par 5 فَدَخَلَتِ الْمَدِينَةُ فِي الْحِصَارِ إِلَى السَّنَةِ الْحَادِيَةِ عَشَرَةَ لِلْمَلِكِ صِدْقِيَّا.
\par 6 فِي الشَّهْرِ الرَّابِعِ فِي تَاسِعِ الشَّهْرِ اشْتَدَّ الْجُوعُ فِي الْمَدِينَةِ وَلَمْ يَكُنْ خُبْزٌ لِشَعْبِ الأَرْضِ.
\par 7 فَثُغِرَتِ الْمَدِينَةُ وَهَرَبَ كُلُّ رِجَالِ الْقِتَالِ وَخَرَجُوا مِنَ الْمَدِينَةِ لَيْلاً فِي طَرِيقِ الْبَابِ بَيْنَ السُّورَيْنِ اللَّذَيْنِ عِنْدَ جَنَّةِ الْمَلِكِ وَالْكِلْدَانِيُّونَ عِنْدَ الْمَدِينَةِ حَوَالَيْهَا فَذَهَبُوا فِي طَرِيقِ الْبَرِّيَّةِ.
\par 8 فَتَبِعَتْ جُيُوشُ الْكِلْدَانِيِّينَ الْمَلِكَ فَأَدْرَكُوا صِدْقِيَّا فِي بَرِّيَّةِ أَرِيحَا وَتَفَرَّقَ كُلُّ جَيْشِهِ عَنْهُ.
\par 9 فَأَخَذُوا الْمَلِكَ وَأَصْعَدُوهُ إِلَى مَلِكِ بَابِلَ إِلَى رَبْلَةَ فِي أَرْضِ حَمَاةَ فَكَلَّمَهُ بِالْقَضَاءِ عَلَيْهِ.
\par 10 فَقَتَلَ مَلِكُ بَابِلَ بَنِي صِدْقِيَّا أَمَامَ عَيْنَيْهِ وَقَتَلَ أَيْضاً كُلَّ رُؤَسَاءِ يَهُوذَا فِي رَبْلَةَ
\par 11 وَأَعْمَى عَيْنَيْ صِدْقِيَّا وَقَيَّدَهُ بِسِلْسِلَتَيْنِ مِنْ نُحَاسٍ وَجَاءَ بِهِ مَلِكُ بَابِلَ إِلَى بَابِلَ وَجَعَلَهُ فِي السِّجْنِ إِلَى يَوْمِ وَفَاتِهِ.
\par 12 وَفِي الشَّهْرِ الْخَامِسِ فِي عَاشِرِ الشَّهْرِ (وَهِيَ السَّنَةُ التَّاسِعَةُ عَشَرَةَ لِلْمَلِكِ نَبُوخَذْنَصَّرَ مَلِكِ بَابِلَ) جَاءَ نَبُوزَرَادَانُ رَئِيسُ الشُّرَطِ الَّذِي كَانَ يَقِفُ أَمَامَ مَلِكِ بَابِلَ إِلَى أُورُشَلِيمَ
\par 13 وَأَحْرَقَ بَيْتَ الرَّبِّ وَبَيْتَ الْمَلِكِ وَكُلَّ بُيُوتِ أُورُشَلِيمَ وَكُلَّ بُيُوتِ الْعُظَمَاءِ. أَحْرَقَهَا بِالنَّارِ
\par 14 وَكُلَّ أَسْوَارِ أُورُشَلِيمَ مُسْتَدِيراً هَدَمَهَا كُلُّ جَيْشِ الْكِلْدَانِيِّينَ الَّذِي مَعَ رَئِيسِ الشُّرَطِ.
\par 15 وَسَبَى نَبُوزَرَادَانُ رَئِيسُ الشُّرَطِ بَعْضاً مِنْ فُقَرَاءِ الشَّعْبِ وَبَقِيَّةَ الشَّعْبِ الَّذِينَ بَقُوا فِي الْمَدِينَةِ وَالْهَارِبِينَ الَّذِينَ سَقَطُوا إِلَى مَلِكِ بَابِلَ وَبَقِيَّةَ الْجُمْهُورِ.
\par 16 وَلَكِنَّ نَبُوزَرَادَانَ رَئِيسَ الشُّرَطِ أَبْقَى مِنْ مَسَاكِينِ الأَرْضِ كَرَّامِينَ وَفَلاَّحِينَ.
\par 17 وَكَسَّرَ الْكِلْدَانِيُّونَ أَعْمِدَةَ النُّحَاسِ الَّتِي لِبَيْتِ الرَّبِّ وَالْقَوَاعِدَ وَبَحْرَ النُّحَاسِ الَّذِي فِي بَيْتِ الرَّبِّ وَحَمَلُوا كُلَّ نُحَاسِهَا إِلَى بَابِلَ.
\par 18 وَأَخَذُوا الْقُدُورَ وَالرُّفُوشَ وَالْمَقَاصَّ وَالْمَنَاضِحَ وَالصُّحُونَ وَكُلَّ آنِيَةِ النُّحَاسِ الَّتِي كَانُوا يَخْدِمُونَ بِهَا.
\par 19 وَأَخَذَ رَئِيسُ الشُّرَطِ الطُّسُوسَ وَالْمَجَامِرَ وَالْمَنَاضِحَ وَالْقُدُورَ وَالْمَنَائِرَ وَالصُّحُونَ وَالأَقْدَاحَ مَا كَانَ مِنْ ذَهَبٍ فَالذَّهَبَ وَمَا كَانَ مِنْ فِضَّةٍ فَالْفِضَّةَ.
\par 20 وَالْعَمُودَيْنِ وَالْبَحْرَ الْوَاحِدَ وَالاِثْنَيْ عَشَرَ ثَوْراً مِنْ نُحَاسٍ الَّتِي تَحْتَ الْقَوَاعِدِ الَّتِي عَمِلَهَا الْمَلِكُ سُلَيْمَانُ لِبَيْتِ الرَّبِّ. لَمْ يَكُنْ وَزْنٌ لِنُحَاسِ كُلِّ هَذِهِ الأَدَوَاتِ.
\par 21 أَمَّا الْعَمُودَانِ فَكَانَ طُولُ الْعَمُودِ الْوَاحِدِ ثَمَانِيَ عَشَرَةَ ذِرَاعاً وَخَيْطٌ اثْنَتَا عَشَرَةَ ذِرَاعاً يُحِيطُ بِهِ وَغِلَظُهُ أَرْبَعُ أَصَابِعَ وَهُوَ أَجْوَفُ.
\par 22 وَعَلَيْهِ تَاجٌ مِنْ نُحَاسٍ ارْتِفَاعُ التَّاجِ الْوَاحِدِ خَمْسُ أَذْرُعٍ. وَعَلَى التَّاجِ حَوَالَيْهِ شَبَكَةٌ وَرُمَّانَاتُ الْكُلُّ مِنْ نُحَاسٍ. وَمِثْلُ ذَلِكَ لِلْعَمُودِ الثَّانِي وَالرُّمَّانَاتِ.
\par 23 وَكَانَتِ الرُّمَّانَاتُ سِتّاً وَتِسْعِينَ لِلْجَانِبِ. كُلُّ الرُّمَّانَاتِ مِئَةٌ عَلَى الشَّبَكَةِ حَوَالَيْهَا.
\par 24 وَأَخَذَ رَئِيسُ الشُّرَطِ سَرَايَا الْكَاهِنَ الأَوَّلَ وَصَفَنْيَا الْكَاهِنَ الثَّانِي وَحَارِسِي الْبَابِ الثَّلاَثَةَ.
\par 25 وَأَخَذَ مِنَ الْمَدِينَةِ خَصِيّاً وَاحِداً كَانَ وَكِيلاً عَلَى رِجَالِ الْحَرْبِ وَسَبْعَةَ رِجَالٍ مِنَ الَّذِينَ يَنْظُرُونَ وَجْهَ الْمَلِكِ الَّذِينَ وُجِدُوا فِي الْمَدِينَةِ وَكَاتِبَ رَئِيسِ الْجُنْدِ الَّذِي كَانَ يَجْمَعُ شَعْبَ الأَرْضِ لِلتَّجَنُّدِ وَسِتِّينَ رَجُلاً مِنْ شَعْبِ الأَرْضِ الَّذِينَ وُجِدُوا فِي وَسَطِ الْمَدِينَةِ.
\par 26 أَخَذَهُمْ نَبُوزَرَادَانُ رَئِيسُ الشُّرَطِ وَسَارَ بِهِمْ إِلَى مَلِكِ بَابِلَ إِلَى رَبْلَةَ
\par 27 فَضَرَبَهُمْ مَلِكُ بَابِلَ وَقَتَلَهُمْ فِي رَبْلَةَ فِي أَرْضِ حَمَاةَ. فَسُبِيَ يَهُوذَا مِنْ أَرْضِهِ.
\par 28 هَذَا هُوَ الشَّعْبُ الَّذِي سَبَاهُ نَبُوخَذْنَصَّرُ فِي السَّنَةِ السَّابِعَةِ. مِنَ الْيَهُودِ ثَلاَثَةُ آلاَفٍ وَثَلاَثَةٌ وَعِشْرُونَ.
\par 29 وَفِي السَّنَةِ الثَّامِنَةِ عَشَرَةَ لِنَبُوخَذْنَصَّرَ سُبِيَ مِنْ أُورُشَلِيمَ ثَمَانُ مِئَةٍ وَاثْنَانِ وَثَلاَثُونَ نَفْساً.
\par 30 فِي السَّنَةِ الثَّالِثَةِ وَالْعِشْرِينَ لِنَبُوخَذْنَصَّرَ سَبَى نَبُوزَرَادَانُ رَئِيسُ الشُّرَطِ مِنَ الْيَهُودِ سَبْعَ مِئَةٍ وَخَمْساً وَأَرْبَعِينَ نَفْساً. جُمْلَةُ النُّفُوسِ أَرْبَعَةُ آلاَفٍ وَسِتُّ مِئَةٍ.
\par 31 وَفِي السَّنَةِ السَّابِعَةِ وَالثَّلاَثِينَ لِسَبْيِ يَهُويَاكِينَ فِي الشَّهْرِ الثَّانِي عَشَرَ فِي الْخَامِسِ وَالْعِشْرِينَ مِنَ الشَّهْرِ رَفَعَ أَوِيلُ مَرُودَخُ مَلِكُ بَابِلَ فِي سَنَةِ تَمَلُّكِهِ رَأْسَ يَهُويَاكِينَ مَلِكِ يَهُوذَا وَأَخْرَجَهُ مِنَ السِّجْنِ
\par 32 وَكَلَّمَهُ بِخَيْرٍ وَجَعَلَ كُرْسِيَّهُ فَوْقَ كَرَاسِيِّ الْمُلُوكِ الَّذِينَ مَعَهُ فِي بَابِلَ.
\par 33 وَغَيَّرَ ثِيَابَ سِجْنِهِ وَكَانَ يَأْكُلُ دَائِماً الْخُبْزَ أَمَامَهُ كُلَّ أَيَّامِ حَيَاتِهِ.
\par 34 وَوَظِيفَتُهُ وَظِيفَةٌ دَائِمَةٌ تُعْطَى لَهُ مِنْ عِنْدِ مَلِكِ بَابِلَ أَمْرَ كُلِّ يَوْمٍ بِيَوْمِهِ إِلَى يَوْمِ وَفَاتِهِ كُلَّ أَيَّامِ حَيَاتِهِ.

\end{document}