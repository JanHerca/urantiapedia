\begin{document}

\title{2 ملوك}


\chapter{1}

\par 1 وَعَصَى مُوآبُ عَلَى إِسْرَائِيلَ بَعْدَ وَفَاةِ أَخْآبَ.
\par 2 وَسَقَطَ أَخَزْيَا مِنَ الْكُوَّةِ الَّتِي فِي عُلِّيَّتِهِ الَّتِي فِي السَّامِرَةِ فَمَرِضَ، وَأَرْسَلَ رُسُلاً وَقَالَ لَهُمُ: [اذْهَبُوا اسْأَلُوا بَعْلَ زَبُوبَ إِلَهَ عَقْرُونَ إِنْ كُنْتُ أَبْرَأُ مِنْ هَذَا الْمَرَضِ].
\par 3 فَقَالَ مَلاَكُ الرَّبِّ لإِيلِيَّا التِّشْبِيِّ قُمِ: [اصْعَدْ لِلِقَاءِ رُسُلِ مَلِكِ السَّامِرَةِ وَقُلْ لَهُمْ: هَلْ لأَنَّهُ لاَ يُوجَدُ فِي إِسْرَائِيلَ إِلَهٌ، تَذْهَبُونَ لِتَسْأَلُوا بَعْلَ زَبُوبَ إِلَهَ عَقْرُونَ؟
\par 4 فَلِذَلِكَ هَكَذَا قَالَ الرَّبُّ: إِنَّ السَّرِيرَ الَّذِي صَعِدْتَ عَلَيْهِ لاَ تَنْزِلُ عَنْهُ بَلْ مَوْتاً تَمُوتُ]. فَانْطَلَقَ إِيلِيَّا.
\par 5 وَرَجَعَ الرُّسُلُ إِلَيْهِ. فَقَالَ لَهُمْ: [لِمَاذَا رَجَعْتُمْ؟]
\par 6 فَقَالُوا لَهُ: [صَعِدَ رَجُلٌ لِلِقَائِنَا وَقَالَ لَنَا: اذْهَبُوا رَاجِعِينَ إِلَى الْمَلِكِ الَّذِي أَرْسَلَكُمْ وَقُولُوا لَهُ: هَكَذَا قَالَ الرَّبُّ: هَلْ لأَنَّهُ لاَ يُوجَدُ فِي إِسْرَائِيلَ إِلَهٌ أَرْسَلْتَ لِتَسْأَلَ بَعْلَ زَبُوبَ إِلَهَ عَقْرُونَ؟ لِذَلِكَ السَّرِيرُ الَّذِي صَعِدْتَ عَلَيْهِ لاَ تَنْزِلُ عَنْهُ بَلْ مَوْتاً تَمُوتُ].
\par 7 فَقَالَ لَهُمْ: [مَا هِيَ هَيْئَةُ الرَّجُلِ الَّذِي صَعِدَ لِلِقَائِكُمْ وَكَلَّمَكُمْ بِهَذَا الْكَلاَمِ؟]
\par 8 فَقَالُوا لَهُ: [إِنَّهُ رَجُلٌ أَشْعَرُ مُتَنَطِّقٌ بِمِنْطَقَةٍ مِنْ جِلْدٍ عَلَى حَقْوَيْهِ]. فَقَالَ: [هُوَ إِيلِيَّا التِّشْبِيُّ].
\par 9 فَأَرْسَلَ إِلَيْهِ رَئِيسَ خَمْسِينَ مَعَ الْخَمْسِينَ الَّذِينَ لَهُ، فَصَعِدَ إِلَيْهِ وَإِذَا هُوَ جَالِسٌ عَلَى رَأْسِ الْجَبَلِ. فَقَالَ لَهُ: [يَا رَجُلَ اللَّهِ، الْمَلِكُ يَقُولُ انْزِلْ].
\par 10 فَأَجَابَ إِيلِيَّا رَئِيسَ الْخَمْسِينَ: [إِنْ كُنْتُ أَنَا رَجُلَ اللَّهِ فَلْتَنْزِلْ نَارٌ مِنَ السَّمَاءِ وَتَأْكُلْكَ أَنْتَ وَالْخَمْسِينَ الَّذِينَ لَكَ]. فَنَزَلَتْ نَارٌ مِنَ السَّمَاءِ وَأَكَلَتْهُ هُوَ وَالْخَمْسِينَ الَّذِينَ لَهُ.
\par 11 ثُمَّ عَادَ وَأَرْسَلَ إِلَيْهِ رَئِيسَ خَمْسِينَ آخَرَ وَالْخَمْسِينَ الَّذِينَ لَهُ. فَقَالَ لَهُ: [يَا رَجُلَ اللَّهِ، هَكَذَا يَقُولُ الْمَلِكُ: أَسْرِعْ وَانْزِلْ].
\par 12 فَأَجَابَ إِيلِيَّا: [إِنْ كُنْتُ أَنَا رَجُلَ اللَّهِ فَلْتَنْزِلْ نَارٌ مِنَ السَّمَاءِ وَتَأْكُلْكَ أَنْتَ وَالْخَمْسِينَ الَّذِينَ لَكَ]. فَنَزَلَتْ نَارُ اللَّهِ مِنَ السَّمَاءِ وَأَكَلَتْهُ هُوَ وَالْخَمْسِينَ الَّذِينَ لَهُ.
\par 13 ثُمَّ عَادَ فَأَرْسَلَ رَئِيسَ خَمْسِينَ ثَالِثاً وَالْخَمْسِينَ الَّذِينَ لَهُ. فَصَعِدَ رَئِيسُ الْخَمْسِينَ الثَّالِثُ وَجَاءَ وَجَثَا عَلَى رُكْبَتَيْهِ أمَامَ إِيلِيَّا، وَتَضَرَّعَ إِلَيْهِ وَقَالَ لَهُ: [يَا رَجُلَ اللَّهِ، لِتُكْرَمْ نَفْسِي وَأَنْفُسُ عَبِيدِكَ هَؤُلاَءِ الْخَمْسِينَ فِي عَيْنَيْكَ.
\par 14 هُوَذَا قَدْ نَزَلَتْ نَارٌ مِنَ السَّمَاءِ وَأَكَلَتْ رَئِيسَيِ الْخَمْسِينَيْنِ الأَوَّلَيْنِ وَخَمْسِينَيْهِمَا، وَالآنَ فَلْتُكْرَمْ نَفْسِي فِي عَيْنَيْكَ].
\par 15 فَقَالَ مَلاَكُ الرَّبِّ لإِيلِيَّا: [انْزِلْ مَعَهُ. لاَ تَخَفْ مِنْهُ]. فَقَامَ وَنَزَلَ مَعَهُ إِلَى الْمَلِكِ.
\par 16 وَقَالَ لَهُ: [هَكَذَا قَالَ الرَّبُّ: مِنْ أَجْلِ أَنَّكَ أَرْسَلْتَ رُسُلاً لِتَسْأَلَ بَعْلَ زَبُوبَ إِلَهَ عَقْرُونَ، فَهَلْ لاَ يُوجَدُ فِي إِسْرَائِيلَ إِلَهٌ لِتَسْأَلَ عَنْ كَلاَمِهِ! لِذَلِكَ السَّرِيرُ الَّذِي صَعِدْتَ عَلَيْهِ لاَ تَنْزِلُ عَنْهُ بَلْ مَوْتاً تَمُوتُ].
\par 17 فَمَاتَ حَسَبَ كَلاَمِ الرَّبِّ الَّذِي تَكَلَّمَ بِهِ إِيلِيَّا. وَمَلَكَ يُورَامُ عِوَضاً عَنْهُ فِي السَّنَةِ الثَّانِيَةِ لِيَهُورَامَ بْنِ يَهُوشَافَاطَ مَلِكِ يَهُوذَا، لأَنَّهُ لَمْ يَكُنْ لَهُ ابْنٌ.
\par 18 وَبَقِيَّةُ أُمُورِ أَخَزْيَا الَّتِي عَمِلَ مَكْتُوبَةٌ فِي سِفْرِ أَخْبَارِ الأَيَّامِ لِمُلُوكِ إِسْرَائِيلَ.

\chapter{2}

\par 1 وَكَانَ عِنْدَ إِصْعَادِ الرَّبِّ إِيلِيَّا فِي الْعَاصِفَةِ إِلَى السَّمَاءِ أَنَّ إِيلِيَّا وَأَلِيشَعَ ذَهَبَا مِنَ الْجِلْجَالِ.
\par 2 فَقَالَ إِيلِيَّا لأَلِيشَعَ: [امْكُثْ هُنَا لأَنَّ الرَّبَّ قَدْ أَرْسَلَنِي إِلَى بَيْتِ إِيلَ]. فَقَالَ أَلِيشَعُ: [حَيٌّ هُوَ الرَّبُّ وَحَيَّةٌ هِيَ نَفْسُكَ إِنِّي لاَ أَتْرُكُكَ]. وَنَزَلاَ إِلَى بَيْتِ إِيلَ.
\par 3 فَخَرَجَ بَنُو الأَنْبِيَاءِ الَّذِينَ فِي بَيْتِ إِيلَ إِلَى أَلِيشَعَ وَقَالُوا لَهُ: [أَتَعْلَمُ أَنَّهُ الْيَوْمَ يَأْخُذُ الرَّبُّ سَيِّدَكَ مِنْ عَلَى رَأْسِكَ؟] فَقَالَ: [نَعَمْ، إِنِّي أَعْلَمُ فَاصْمُتُوا].
\par 4 ثُمَّ قَالَ لَهُ إِيلِيَّا: [يَا أَلِيشَعُ، امْكُثْ هُنَا لأَنَّ الرَّبَّ قَدْ أَرْسَلَنِي إِلَى أَرِيحَا]. فَقَالَ: [حَيٌّ هُوَ الرَّبُّ وَحَيَّةٌ هِيَ نَفْسُكَ إِنِّي لاَ أَتْرُكُكَ]. وَأَتَيَا إِلَى أَرِيحَا.
\par 5 فَتَقَدَّمَ بَنُو الأَنْبِيَاءِ الَّذِينَ فِي أَرِيحَا إِلَى أَلِيشَعَ وَقَالُوا لَهُ: [أَتَعْلَمُ أَنَّهُ الْيَوْمَ يَأْخُذُ الرَّبُّ سَيِّدَكَ مِنْ عَلَى رَأْسِكَ؟] فَقَالَ: [نَعَمْ، إِنِّي أَعْلَمُ فَاصْمُتُوا].
\par 6 ثُمَّ قَالَ لَهُ إِيلِيَّا: [أُمْكُثْ هُنَا لأَنَّ الرَّبَّ قَدْ أَرْسَلَنِي إِلَى الأُرْدُنِّ]. فَقَالَ: [حَيٌّ هُوَ الرَّبُّ وَحَيَّةٌ هِيَ نَفْسُكَ إِنِّي لاَ أَتْرُكُكَ]. وَانْطَلَقَا كِلاَهُمَا.
\par 7 فَذَهَبَ خَمْسُونَ رَجُلاً مِنْ بَنِي الأَنْبِيَاءِ وَوَقَفُوا قُبَالَتَهُمَا مِنْ بَعِيدٍ. وَوَقَفَ كِلاَهُمَا بِجَانِبِ الأُرْدُنِّ.
\par 8 وَأَخَذَ إِيلِيَّا رِدَاءَهُ وَلَفَّهُ وَضَرَبَ الْمَاءَ، فَانْفَلَقَ إِلَى هُنَا وَهُنَاكَ، فَعَبَرَا كِلاَهُمَا فِي الْيَبَسِ.
\par 9 وَلَمَّا عَبَرَا قَالَ إِيلِيَّا لأَلِيشَعَ: [اطْلُبْ مَاذَا أَفْعَلُ لَكَ قَبْلَ أَنْ أُوخَذَ مِنْكَ]. فَقَالَ أَلِيشَعُ: [لِيَكُنْ نَصِيبُ اثْنَيْنِ مِنْ رُوحِكَ عَلَيَّ].
\par 10 فَقَالَ: [صَعَّبْتَ السُّؤَالَ. فَإِنْ رَأَيْتَنِي أُوخَذُ مِنْكَ يَكُونُ لَكَ كَذَلِكَ، وَإِلاَّ فَلاَ يَكُونُ].
\par 11 وَفِيمَا هُمَا يَسِيرَانِ وَيَتَكَلَّمَانِ إِذَا مَرْكَبَةٌ مِنْ نَارٍ وَخَيْلٌ مِنْ نَارٍ فَصَلَتْ بَيْنَهُمَا، فَصَعِدَ إِيلِيَّا فِي الْعَاصِفَةِ إِلَى السَّمَاءِ.
\par 12 وَكَانَ أَلِيشَعُ يَرَى وَهُوَ يَصْرُخُ: [يَا أَبِي يَا أَبِي، مَرْكَبَةَ إِسْرَائِيلَ وَفُرْسَانَهَا!] وَلَمْ يَرَهُ بَعْدُ. فَأَمْسَكَ ثِيَابَهُ وَمَزَّقَهَا قِطْعَتَيْنِ،
\par 13 وَرَفَعَ رِدَاءَ إِيلِيَّا الَّذِي سَقَطَ عَنْهُ، وَرَجَعَ وَوَقَفَ عَلَى شَاطِئِ الأُرْدُنِّ.
\par 14 فَأَخَذَ رِدَاءَ إِيلِيَّا الَّذِي سَقَطَ عَنْهُ وَضَرَبَ الْمَاءَ وَقَالَ: [أَيْنَ هُوَ الرَّبُّ إِلَهُ إِيلِيَّا؟] ثُمَّ ضَرَبَ الْمَاءَ أَيْضاً فَانْفَلَقَ إِلَى هُنَا وَهُنَاكَ، فَعَبَرَ أَلِيشَعُ.
\par 15 وَلَمَّا رَآهُ بَنُو الأَنْبِيَاءِ الَّذِينَ فِي أَرِيحَا قُبَالَتَهُ قَالُوا: [قَدِ اسْتَقَرَّتْ رُوحُ إِيلِيَّا عَلَى أَلِيشَعَ]. فَجَاءُوا لِلِقَائِهِ وَسَجَدُوا لَهُ إِلَى الأَرْضِ.
\par 16 وَقَالُوا لَهُ: [هُوَذَا مَعَ عَبِيدِكَ خَمْسُونَ رَجُلاً ذَوُو بَأْسٍ، فَدَعْهُمْ يَذْهَبُونَ وَيُفَتِّشُونَ عَلَى سَيِّدِكَ، لِئَلاَّ يَكُونَ قَدْ حَمَلَهُ رُوحُ الرَّبِّ وَطَرَحَهُ عَلَى أَحَدِ الْجِبَالِ أَوْ فِي أَحَدِ الأَوْدِيَةِ]. فَقَالَ: [لاَ تُرْسِلُوا].
\par 17 فَأَلَحُّوا عَلَيْهِ حَتَّى خَجِلَ وَقَالَ: [أَرْسِلُوا]. فَأَرْسَلُوا خَمْسِينَ رَجُلاً، فَفَتَّشُوا ثَلاَثَةَ أَيَّامٍ وَلَمْ يَجِدُوهُ.
\par 18 وَلَمَّا رَجَعُوا إِلَيْهِ وَهُوَ مَاكِثٌ فِي أَرِيحَا قَالَ لَهُمْ: [أَمَا قُلْتُ لَكُمْ لاَ تَذْهَبُوا؟].
\par 19 وَقَالَ رِجَالُ الْمَدِينَةِ لأَلِيشَعَ: [هُوَذَا مَوْقِعُ الْمَدِينَةِ حَسَنٌ كَمَا يَرَى سَيِّدِي، وَأَمَّا الْمِيَاهُ فَرَدِيئَةٌ وَالأَرْضُ مُجْدِبَةٌ].
\par 20 فَقَالَ: [ائْتُونِي بِصَحْنٍ جَدِيدٍ وَضَعُوا فِيهِ مِلْحاً]. فَأَتُوهُ بِهِ.
\par 21 فَخَرَجَ إِلَى نَبْعِ الْمَاءِ وَطَرَحَ فِيهِ الْمِلْحَ وَقَالَ: [هَكَذَا قَالَ الرَّبُّ: قَدْ أَبْرَأْتُ هَذِهِ الْمِيَاهَ. لاَ يَكُونُ فِيهَا أَيْضاً مَوْتٌ وَلاَ جَدْبٌ].
\par 22 فَبَرِئَتِ الْمِيَاهُ إِلَى هَذَا الْيَوْمِ حَسَبَ قَوْلِ أَلِيشَعَ الَّذِي نَطَقَ بِهِ.
\par 23 ثُمَّ صَعِدَ مِنْ هُنَاكَ إِلَى بَيْتِ إِيلَ. وَفِيمَا هُوَ صَاعِدٌ فِي الطَّرِيقِ إِذَا بِصِبْيَانٍ صِغَارٍ خَرَجُوا مِنَ الْمَدِينَةِ وَسَخِرُوا مِنْهُ وَقَالُوا لَهُ: [اصْعَدْ يَا أَقْرَعُ! اصْعَدْ يَا أَقْرَعُ!]
\par 24 فَالْتَفَتَ إِلَى وَرَائِهِ وَنَظَرَ إِلَيْهِمْ وَلَعَنَهُمْ بِاسْمِ الرَّبِّ. فَخَرَجَتْ دُبَّتَانِ مِنَ الْوَعْرِ وَافْتَرَسَتَا مِنْهُمُ اثْنَيْنِ وَأَرْبَعِينَ وَلَداً.
\par 25 وَذَهَبَ مِنْ هُنَاكَ إِلَى جَبَلِ الْكَرْمَلِ، وَمِنْ هُنَاكَ رَجَعَ إِلَى السَّامِرَةِ.

\chapter{3}

\par 1 وَمَلَكَ يُورَامُ بْنُ أَخْآبَ عَلَى إِسْرَائِيلَ فِي السَّامِرَةِ، فِي السَّنَةِ الثَّامِنَةَ عَشَرَةَ لِيَهُوشَافَاطَ مَلِكِ يَهُوذَا. مَلَكَ اثْنَتَيْ عَشَرَةَ سَنَةً.
\par 2 وَعَمِلَ الشَّرَّ فِي عَيْنَيِ الرَّبِّ، وَلَكِنْ لَيْسَ كَأَبِيهِ وَأُمِّهِ، فَإِنَّهُ أَزَالَ تِمْثَالَ الْبَعْلِ الَّذِي عَمِلَهُ أَبُوهُ.
\par 3 إِلاَّ أَنَّهُ لَصِقَ بِخَطَايَا يَرُبْعَامَ بْنِ نَبَاطَ الَّذِي جَعَلَ إِسْرَائِيلَ يُخْطِئُ. لَمْ يَحِدْ عَنْهَا.
\par 4 وَكَانَ مِيشَعُ مَلِكُ مُوآبَ صَاحِبَ مَوَاشٍ، فَأَدَّى لِمَلِكِ إِسْرَائِيلَ مِئَةَ أَلْفِ خَرُوفٍ وَمِئَةَ أَلْفِ كَبْشٍ بِصُوفِهَا.
\par 5 وَعِنْدَ مَوْتِ أَخْآبَ عَصَى مَلِكُ مُوآبَ عَلَى مَلِكِ إِسْرَائِيلَ.
\par 6 وَخَرَجَ الْمَلِكُ يُورَامُ فِي ذَلِكَ الْيَوْمِ مِنَ السَّامِرَةِ وَعَدَّ كُلَّ إِسْرَائِيلَ،
\par 7 وَذَهَبَ وَأَرْسَلَ إِلَى يَهُوشَافَاطَ مَلِكِ يَهُوذَا يَقُولُ: [قَدْ عَصَى عَلَيَّ مَلِكُ مُوآبَ. فَهَلْ تَذْهَبُ مَعِي إِلَى مُوآبَ لِلْحَرْبِ؟] فَقَالَ: [أَصْعَدُ. مَثَلِي مَثَلُكَ. شَعْبِي كَشَعْبِكَ وَخَيْلِي كَخَيْلِكَ].
\par 8 فَقَالَ: [مِنْ أَيِّ طَرِيقٍ نَصْعَدُ؟]. فَقَالَ: [مِنْ طَرِيقِ بَرِّيَّةِ أَدُومَ].
\par 9 فَذَهَبَ مَلِكُ إِسْرَائِيلَ وَمَلِكُ يَهُوذَا وَمَلِكُ أَدُومَ وَدَارُوا مَسِيرَةَ سَبْعَةِ أَيَّامٍ. وَلَمْ يَكُنْ مَاءٌ لِلْجَيْشِ وَالْبَهَائِمِ الَّتِي تَبِعَتْهُمْ.
\par 10 فَقَالَ مَلِكُ إِسْرَائِيلَ: [آهِ عَلَى أَنَّ الرَّبَّ قَدْ دَعَا هَؤُلاَءِ الثَّلاَثَةَ الْمُلُوكِ لِيَدْفَعَهُمْ إِلَى يَدِ مُوآبَ!]
\par 11 فَقَالَ يَهُوشَافَاطُ: [أَلَيْسَ هُنَا نَبِيٌّ لِلرَّبِّ فَنَسْأَلَ الرَّبَّ بِهِ؟] فَأَجَابَ وَاحِدٌ مِنْ عَبِيدِ مَلِكِ إِسْرَائِيلَ وَقَالَ: [هُنَا أَلِيشَعُ بْنُ شَافَاطَ الَّذِي كَانَ يَصُبُّ مَاءً عَلَى يَدَيْ إِيلِيَّا].
\par 12 فَقَالَ يَهُوشَافَاطُ: [عِنْدَهُ كَلاَمُ الرَّبِّ]. فَنَزَلَ إِلَيْهِ مَلِكُ إِسْرَائِيلَ وَيَهُوشَافَاطُ وَمَلِكُ أَدُومَ.
\par 13 فَقَالَ أَلِيشَعُ لِمَلِكِ إِسْرَائِيلَ: [مَا لِي وَلَكَ! اذْهَبْ إِلَى أَنْبِيَاءِ أَبِيكَ وَإِلَى أَنْبِيَاءِ أُمِّكَ]. فَقَالَ لَهُ مَلِكُ إِسْرَائِيلَ: [كَلاَّ. لأَنَّ الرَّبَّ قَدْ دَعَا هَؤُلاَءِ الثَّلاَثَةَ الْمُلُوكِ لِيَدْفَعَهُمْ إِلَى يَدِ مُوآبَ].
\par 14 فَقَالَ أَلِيشَعُ: [حَيٌّ هُوَ رَبُّ الْجُنُودِ الَّذِي أَنَا وَاقِفٌ أَمَامَهُ إِنَّهُ لَوْلاَ أَنِّي رَافِعٌ وَجْهَ يَهُوشَافَاطَ مَلِكِ يَهُوذَا لَمَا كُنْتُ أَنْظُرُ إِلَيْكَ وَلاَ أَرَاكَ.
\par 15 وَالآنَ فَأْتُونِي بِعَوَّادٍ]. وَلَمَّا ضَرَبَ الْعَوَّادُ بِالْعُودِ كَانَتْ عَلَيْهِ يَدُ الرَّبِّ
\par 16 فَقَالَ: [هَكَذَا قَالَ الرَّبُّ: اجْعَلُوا هَذَا الْوَادِيَ حُفَراً حُفَراً.
\par 17 لأَنَّهُ هَكَذَا قَالَ الرَّبُّ: لاَ تَرُونَ رِيحاً وَلاَ تَرُونَ مَطَراً وَهَذَا الْوَادِي يَمْتَلِئُ مَاءً، فَتَشْرَبُونَ أَنْتُمْ وَمَاشِيَتُكُمْ وَبَهَائِمُكُمْ.
\par 18 وَذَلِكَ يَسِيرٌ فِي عَيْنَيِ الرَّبِّ، فَيَدْفَعُ مُوآبَ إِلَى أَيْدِيكُمْ.
\par 19 فَتَضْرِبُونَ كُلَّ مَدِينَةٍ مُحَصَّنَةٍ وَكُلَّ مَدِينَةٍ مُخْتَارَةٍ وَتَقْطَعُونَ كُلَّ شَجَرَةٍ طَيِّبَةٍ وَتَطُمُّونَ جَمِيعَ عُيُونِ الْمَاءِ وَتُفْسِدُونَ كُلَّ حَقْلَةٍ جَيِّدَةٍ بِالْحِجَارَةِ].
\par 20 وَفِي الصَّبَاحِ عِنْدَ إِصْعَادِ التَّقْدِمَةِ إِذَا مِيَاهٌ آتِيَةٌ عَنْ طَرِيقِ أَدُومَ، فَامْتَلَأَتِ الأَرْضُ مَاءً.
\par 21 وَلَمَّا سَمِعَ كُلُّ الْمُوآبِيِّينَ أَنَّ الْمُلُوكَ قَدْ صَعِدُوا لِمُحَارَبَتِهِمْ جَمَعُوا كُلَّ مُتَقَلِّدِي السِّلاَحِ فَمَا فَوْقُ، وَوَقَفُوا عَلَى التُّخُمِ.
\par 22 وَبَكَّرُوا صَبَاحاً وَالشَّمْسُ أَشْرَقَتْ عَلَى الْمِيَاهِ، وَرَأَى الْمُوآبِيُّونَ مُقَابِلَهُمُ الْمِيَاهَ حَمْرَاءَ كَالدَّمِ.
\par 23 فَقَالُوا: [هَذَا دَمٌ! قَدْ تَحَارَبَ الْمُلُوكُ وَضَرَبَ بَعْضُهُمْ بَعْضاً، وَالآنَ فَإِلَى النَّهْبِ يَا مُوآبُ].
\par 24 وَأَتُوا إِلَى مَحَلَّةِ إِسْرَائِيلَ، فَقَامَ إِسْرَائِيلُ وَضَرَبُوا الْمُوآبِيِّينَ فَهَرَبُوا مِنْ أَمَامِهِمْ، فَدَخَلُوهَا وَهُمْ يَضْرِبُونَ الْمُوآبِيِّينَ.
\par 25 وَهَدَمُوا الْمُدُنَ، وَكَانَ كُلُّ وَاحِدٍ يُلْقِي حَجَرَهُ فِي كُلِّ حَقْلَةٍ جَيِّدَةٍ حَتَّى مَلَأُوهَا، وَطَمُّوا جَمِيعَ عُيُونِ الْمَاءِ وَقَطَعُوا كُلَّ شَجَرَةٍ طَيِّبَةٍ. وَلَكِنَّهُمْ أَبْقُوا فِي [قِيرِ حَارِسَةَ] حِجَارَتَهَا. وَاسْتَدَارَ أَصْحَابُ الْمَقَالِيعِ وَضَرَبُوهَا.
\par 26 فَلَمَّا رَأَى مَلِكُ مُوآبَ أَنَّ الْحَرْبَ قَدِ اشْتَدَّتْ عَلَيْهِ أَخَذَ مَعَهُ سَبْعَ مِئَةِ رَجُلٍ مُسْتَلِّي السُّيُوفِ لِيَشُقُّوا إِلَى مَلِكِ أَدُومَ، فَلَمْ يَقْدِرُوا.
\par 27 فَأَخَذَ ابْنَهُ الْبِكْرَ الَّذِي كَانَ مَلَكَ عِوَضاً عَنْهُ وَأَصْعَدَهُ مُحْرَقَةً عَلَى السُّورِ. فَكَانَ غَيْظٌ عَظِيمٌ عَلَى إِسْرَائِيلَ. فَانْصَرَفُوا عَنْهُ وَرَجَعُوا إِلَى أَرْضِهِمْ.

\chapter{4}

\par 1 وَصَرَخَتْ إِلَى أَلِيشَعَ امْرَأَةٌ مِنْ نِسَاءِ بَنِي الأَنْبِيَاءِ قَائِلَةً: [إِنَّ عَبْدَكَ زَوْجِي قَدْ مَاتَ، وَأَنْتَ تَعْلَمُ أَنَّ عَبْدَكَ كَانَ يَخَافُ الرَّبَّ. فَأَتَى الْمُرَابِي لِيَأْخُذَ وَلَدَيَّ لَهُ عَبْدَيْنِ].
\par 2 فَقَالَ لَهَا أَلِيشَعُ: [مَاذَا أَصْنَعُ لَكِ؟ أَخْبِرِينِي مَاذَا لَكِ فِي الْبَيْتِ]. فَقَالَتْ: [لَيْسَ لِجَارِيَتِكَ شَيْءٌ فِي الْبَيْتِ إِلاَّ دُهْنَةَ زَيْتٍ].
\par 3 فَقَالَ: [اذْهَبِي اسْتَعِيرِي لِنَفْسِكِ أَوْعِيَةً مِنْ خَارِجٍ مِنْ عِنْدِ جَمِيعِ جِيرَانِكِ، أَوْعِيَةً فَارِغَةً. لاَ تُقَلِّلِي.
\par 4 ثُمَّ ادْخُلِي وَأَغْلِقِي الْبَابَ عَلَى نَفْسِكِ وَعَلَى بَنِيكِ، وَصُبِّي فِي جَمِيعِ هَذِهِ الأَوْعِيَةِ، وَمَا امْتَلَأَ انْقُلِيهِ].
\par 5 فَذَهَبَتْ مِنْ عِنْدِهِ وَأَغْلَقَتِ الْبَابَ عَلَى نَفْسِهَا وَعَلَى بَنِيهَا. فَكَانُوا هُمْ يُقَدِّمُونَ لَهَا الأَوْعِيَةَ وَهِيَ تَصُبُّ.
\par 6 وَلَمَّا امْتَلَأَتِ الأَوْعِيَةُ قَالَتْ لاِبْنِهَا: [قَدِّمْ لِي أَيْضاً وِعَاءً]. فَقَالَ لَهَا: [لاَ يُوجَدُ بَعْدُ وِعَاءٌ]. فَوَقَفَ الزَّيْتُ.
\par 7 فَأَتَتْ وَأَخْبَرَتْ رَجُلَ اللَّهِ فَقَالَ: [اذْهَبِي بِيعِي الزَّيْتَ وَأَوْفِي دَيْنَكِ وَعِيشِي أَنْتِ وَبَنُوكِ بِمَا بَقِيَ].
\par 8 وَفِي ذَاتِ يَوْمٍ عَبَرَ أَلِيشَعُ إِلَى شُونَمَ. وَكَانَتْ هُنَاكَ امْرَأَةٌ عَظِيمَةٌ فَأَمْسَكَتْهُ لِيَأْكُلَ خُبْزاً. وَكَانَ كُلَّمَا عَبَرَ يَمِيلُ إِلَى هُنَاكَ لِيَأْكُلَ خُبْزاً.
\par 9 فَقَالَتْ لِرَجُلِهَا: [قَدْ عَلِمْتُ أَنَّهُ رَجُلَ اللَّهِ مُقَدَّسٌ الَّذِي يَمُرُّ عَلَيْنَا دَائِماً.
\par 10 فَلْنَعْمَلْ عُلِّيَّةً عَلَى الْحَائِطِ صَغِيرَةً وَنَضَعْ لَهُ هُنَاكَ سَرِيراً وَخِوَاناً وَكُرْسِيّاً وَمَنَارَةً، حَتَّى إِذَا جَاءَ إِلَيْنَا يَمِيلُ إِلَيْهَا].
\par 11 وَفِي ذَاتِ يَوْمٍ جَاءَ إِلَى هُنَاكَ وَمَالَ إِلَى الْعُلِّيَّةِ وَاضْطَجَعَ فِيهَا.
\par 12 فَقَالَ لِجِيحَزِي غُلاَمِهِ: [ادْعُ هَذِهِ الشُّونَمِيَّةَ]. فَدَعَاهَا فَوَقَفَتْ أَمَامَهُ.
\par 13 فَقَالَ لَهُ: [قُلْ لَهَا: هُوَذَا قَدِ انْزَعَجْتِ بِسَبَبِنَا كُلَّ هَذَا الاِنْزِعَاجِ، فَمَاذَا يُصْنَعُ لَكِ؟ هَلْ لَكِ مَا يُتَكَلَّمُ بِهِ إِلَى الْمَلِكِ أَوْ إِلَى رَئِيسِ الْجَيْشِ؟] فَقَالَتْ: [إِنَّمَا أَنَا سَاكِنَةٌ فِي وَسَطِ شَعْبِي].
\par 14 ثُمَّ قَالَ: [فَمَاذَا يُصْنَعُ لَهَا؟] فَقَالَ جِيحَزِي: [إِنَّهُ لَيْسَ لَهَا ابْنٌ وَرَجُلُهَا قَدْ شَاخَ].
\par 15 فَقَالَ: [ادْعُهَا]. فَدَعَاهَا فَوَقَفَتْ فِي الْبَابِ.
\par 16 فَقَالَ: [فِي هَذَا الْمِيعَادِ نَحْوَ زَمَانِ الْحَيَاةِ تَحْتَضِنِينَ ابْناً]. فَقَالَتْ: [لاَ يَا سَيِّدِي رَجُلَ اللَّهِ! لاَ تَكْذِبْ عَلَى جَارِيَتِكَ!].
\par 17 فَحَبِلَتِ الْمَرْأَةُ وَوَلَدَتِ ابْناً فِي ذَلِكَ الْمِيعَادِ نَحْوَ زَمَانِ الْحَيَاةِ كَمَا قَالَ لَهَا أَلِيشَعُ.
\par 18 وَكَبِرَ الْوَلَدُ. وَفِي ذَاتِ يَوْمٍ خَرَجَ إِلَى أَبِيهِ إِلَى الْحَصَّادِينَ.
\par 19 وَقَالَ لأَبِيهِ: [رَأْسِي رَأْسِي]. فَقَالَ لِلْغُلاَمِ: [احْمِلْهُ إِلَى أُمِّهِ].
\par 20 فَحَمَلَهُ وَأَتَى بِهِ إِلَى أُمِّهِ، فَجَلَسَ عَلَى رُكْبَتَيْهَا إِلَى الظُّهْرِ وَمَاتَ.
\par 21 فَصَعِدَتْ وَأَضْجَعَتْهُ عَلَى سَرِيرِ رَجُلِ اللَّهِ وَأَغْلَقَتْ عَلَيْهِ وَخَرَجَتْ.
\par 22 وَنَادَتْ رَجُلَهَا وَقَالَتْ: [أَرْسِلْ لِي وَاحِداً مِنَ الْغِلْمَانِ وَإِحْدَى الأُتُنِ فَأَجْرِيَ إِلَى رَجُلِ اللَّهِ وَأَرْجِعَ].
\par 23 فَقَالَ: [لِمَاذَا تَذْهَبِينَ إِلَيْهِ الْيَوْمَ؟ لاَ رَأْسُ شَهْرٍ وَلاَ سَبْتٌ]. فَقَالَتْ: [سَلاَمٌ].
\par 24 وَشَدَّتْ عَلَى الأَتَانِ، وَقَالَتْ لِغُلاَمِهَا: [سُقْ وَسِرْ وَلاَ تَتَعَوَّقْ لأَجْلِي فِي الرُّكُوبِ إِنْ لَمْ أَقُلْ لَكَ].
\par 25 وَانْطَلَقَتْ حَتَّى جَاءَتْ إِلَى رَجُلِ اللَّهِ إِلَى جَبَلِ الْكَرْمَلِ. فَلَمَّا رَآهَا رَجُلُ اللَّهِ مِنْ بَعِيدٍ قَالَ لِجِيحَزِي غُلاَمِهِ: [هُوَذَا تِلْكَ الشُّونَمِيَّةُ.
\par 26 اُرْكُضِ الآنَ لِلِقَائِهَا وَقُلْ لَهَا: أَسَلاَمٌ لَكِ؟ أَسَلاَمٌ لِزَوْجِكِ؟ أَسَلاَمٌ لِلْوَلَدِ؟] فَقَالَتْ: [سَلاَمٌ].
\par 27 فَلَمَّا جَاءَتْ إِلَى رَجُلِ اللَّهِ إِلَى الْجَبَلِ أَمْسَكَتْ رِجْلَيْهِ. فَتَقَدَّمَ جِيحَزِي لِيَدْفَعَهَا. فَقَالَ رَجُلُ اللَّهِ: [دَعْهَا لأَنَّ نَفْسَهَا مُرَّةٌ فِيهَا وَالرَّبُّ كَتَمَ الأَمْرَ عَنِّي وَلَمْ يُخْبِرْنِي].
\par 28 فَقَالَتْ: [هَلْ طَلَبْتُ ابْناً مِنْ سَيِّدِي؟ أَلَمْ أَقُلْ لاَ تَخْدَعْنِي؟]
\par 29 فَقَالَ لِجِيحَزِي: [أُشْدُدْ حَقَوَيْكَ وَخُذْ عُكَّازِي بِيَدِكَ وَانْطَلِقْ، وَإِذَا صَادَفْتَ أَحَداً فَلاَ تُبَارِكْهُ، وَإِنْ بَارَكَكَ أَحَدٌ فَلاَ تُجِبْهُ. وَضَعْ عُكَّازِي عَلَى وَجْهِ الصَّبِيِّ].
\par 30 فَقَالَتْ أُمُّ الصَّبِيِّ: [حَيٌّ هُوَ الرَّبُّ وَحَيَّةٌ هِيَ نَفْسُكَ إِنِّي لاَ أَتْرُكُكَ]. فَقَامَ وَتَبِعَهَا.
\par 31 وَجَازَ جِيحَزِي قُدَّامَهُمَا وَوَضَعَ الْعُكَّازَ عَلَى وَجْهِ الصَّبِيِّ فَلَمْ يَكُنْ صَوْتٌ وَلاَ مُصْغٍ. فَرَجَعَ لِلِقَائِهِ وَأَخْبَرَهُ قَائِلاً: [لَمْ يَنْتَبِهِ الصَّبِيُّ].
\par 32 وَدَخَلَ أَلِيشَعُ الْبَيْتَ وَإِذَا بِالصَّبِيِّ مَيِّتٌ وَمُضْطَجِعٌ عَلَى سَرِيرِهِ.
\par 33 فَدَخَلَ وَأَغْلَقَ الْبَابَ عَلَى نَفْسَيْهِمَا كِلَيْهِمَا وَصَلَّى إِلَى الرَّبِّ.
\par 34 ثُمَّ صَعِدَ وَاضْطَجَعَ فَوْقَ الصَّبِيِّ وَوَضَعَ فَمَهُ عَلَى فَمِهِ وَعَيْنَيْهِ عَلَى عَيْنَيْهِ وَيَدَيْهِ عَلَى يَدَيْهِ، وَتَمَدَّدَ عَلَيْهِ فَسَخُِنَ جَسَدُ الْوَلَدِ.
\par 35 ثُمَّ عَادَ وَتَمَشَّى فِي الْبَيْتِ تَارَةً إِلَى هُنَا وَتَارَةً إِلَى هُنَاكَ، وَصَعِدَ وَتَمَدَّدَ عَلَيْهِ فَعَطَسَ الصَّبِيُّ سَبْعَ مَرَّاتٍ ثُمَّ فَتَحَ الصَّبِيُّ عَيْنَيْهِ.
\par 36 فَدَعَا جِيحَزِي وَقَالَ: [ادْعُ هَذِهِ الشُّونَمِيَّةَ] فَدَعَاهَا. وَلَمَّا دَخَلَتْ إِلَيْهِ قَالَ: [احْمِلِي ابْنَكِ].
\par 37 فَأَتَتْ وَسَقَطَتْ عَلَى رِجْلَيْهِ وَسَجَدَتْ إِلَى الأَرْضِ، ثُمَّ حَمَلَتِ ابْنَهَا وَخَرَجَتْ.
\par 38 وَرَجَعَ أَلِيشَعُ إِلَى الْجِلْجَالِ. وَكَانَ جُوعٌ فِي الأَرْضِ وَكَانَ بَنُو الأَنْبِيَاءِ جُلُوساً أَمَامَهُ. فَقَالَ لِغُلاَمِهِ: [هَيِّئِ الْقِدْرَ الْكَبِيرَةَ وَاسْلُقْ سَلِيقَةً لِبَنِي الأَنْبِيَاءِ].
\par 39 وَخَرَجَ وَاحِدٌ إِلَى الْحَقْلِ لِيَلْتَقِطَ بُقُولاً، فَوَجَدَ يَقْطِيناً بَرِّيّاً، فَالْتَقَطَ مِنْهُ قُثَّاءً بَرِّيّاً مِلْءَ ثَوْبِهِ، وَأَتَى وَقَطَّعَهُ فِي قِدْرِ السَّلِيقَةِ، لأَنَّهُمْ لَمْ يَعْرِفُوا.
\par 40 وَصَبُّوا لِلْقَوْمِ لِيَأْكُلُوا. وَفِيمَا هُمْ يَأْكُلُونَ مِنَ السَّلِيقَةِ صَرَخُوا: [فِي الْقِدْرِ مَوْتٌ يَا رَجُلَ اللَّهِ!] وَلَمْ يَسْتَطِيعُوا أَنْ يَأْكُلُوا.
\par 41 فَقَالَ: [هَاتُوا دَقِيقاً]. فَأَلْقَاهُ فِي الْقِدْرِ وَقَالَ: [صُبَّ لِلْقَوْمِ فَيَأْكُلُوا]. فَكَأَنَّهُ لَمْ يَكُنْ شَيْءٌ رَدِيءٌ فِي الْقِدْرِ.
\par 42 وَجَاءَ رَجُلٌ مِنْ بَعْلِ شَلِيشَةَ وَأَحْضَرَ لِرَجُلِ اللَّهِ خُبْزَ بَاكُورَةٍ عِشْرِينَ رَغِيفاً مِنْ شَعِيرٍ وَسَوِيقاً فِي جِرَابِهِ. فَقَالَ: [أَعْطِ الشَّعْبَ لِيَأْكُلُوا].
\par 43 فَقَالَ خَادِمُهُ: [مَاذَا؟ هَلْ أَجْعَلُ هَذَا أَمَامَ مِئَةِ رَجُلٍ!] فَقَالَ: [أَعْطِ الشَّعْبَ فَيَأْكُلُوا، لأَنَّهُ هَكَذَا قَالَ الرَّبُّ: يَأْكُلُونَ وَيَفْضُلُ عَنْهُمْ].
\par 44 فَجَعَلَ أَمَامَهُمْ فَأَكَلُوا، وَفَضَلَ عَنْهُمْ حَسَبَ قَوْلِ الرَّبِّ.

\chapter{5}

\par 1 وَكَانَ نُعْمَانُ رَئِيسُ جَيْشِ مَلِكِ أَرَامَ رَجُلاً عَظِيماً عِنْدَ سَيِّدِهِ مَرْفُوعَ الْوَجْهِ، لأَنَّهُ عَنْ يَدِهِ أَعْطَى الرَّبُّ خَلاَصاً لأَرَامَ. وَكَانَ الرَّجُلُ جَبَّارَ بَأْسٍ، أَبْرَصَ.
\par 2 وَكَانَ الأَرَامِيُّونَ قَدْ خَرَجُوا غُزَاةً فَسَبُوا مِنْ أَرْضِ إِسْرَائِيلَ فَتَاةً صَغِيرَةً فَكَانَتْ بَيْنَ يَدَيِ امْرَأَةِ نُعْمَانَ.
\par 3 فَقَالَتْ لِمَوْلاَتِهَا: [يَا لَيْتَ سَيِّدِي أَمَامَ النَّبِيِّ الَّذِي فِي السَّامِرَةِ، فَإِنَّهُ كَانَ يَشْفِيهِ مِنْ بَرَصِهِ].
\par 4 فَدَخَلَ وَأَخْبَرَ سَيِّدَهُ: [كَذَا وَكَذَا قَالَتِ الْجَارِيَةُ الَّتِي مِنْ أَرْضِ إِسْرَائِيلَ].
\par 5 فَقَالَ مَلِكُ أَرَامَ: [انْطَلِقْ ذَاهِباً فَأُرْسِلَ كِتَاباً إِلَى مَلِكِ إِسْرَائِيلَ]. فَذَهَبَ وَأَخَذَ بِيَدِهِ عَشَرَ وَزَنَاتٍ مِنَ الْفِضَّةِ، وَسِتَّةَ آلاَفِ شَاقِلٍ مِنَ الذَّهَبِ، وَعَشَرَ حُلَلٍ مِنَ الثِّيَابِ.
\par 6 وَأَتَى بِالْكِتَابِ إِلَى مَلِكِ إِسْرَائِيلَ يَقُولُ فِيهِ: [فَالآنَ عِنْدَ وُصُولِ هَذَا الْكِتَابِ إِلَيْكَ، هُوَذَا قَدْ أَرْسَلْتُ إِلَيْكَ نُعْمَانَ عَبْدِي فَاشْفِهِ مِنْ بَرَصِهِ].
\par 7 فَلَمَّا قَرَأَ مَلِكُ إِسْرَائِيلَ الْكِتَابَ مَزَّقَ ثِيَابَهُ وَقَالَ: [هَلْ أَنَا اللَّهُ لِكَيْ أُمِيتَ وَأُحْيِيَ، حَتَّى إِنَّ هَذَا يُرْسِلُ إِلَيَّ أَنْ أَشْفِيَ رَجُلاً مِنْ بَرَصِهِ؟ فَاعْلَمُوا وَانْظُرُوا أَنَّهُ إِنَّمَا يَتَعَرَّضُ لِي!].
\par 8 وَلَمَّا سَمِعَ أَلِيشَعُ رَجُلُ اللَّهِ أَنَّ مَلِكَ إِسْرَائِيلَ قَدْ مَزَّقَ ثِيَابَهُ، أَرْسَلَ إِلَى الْمَلِكِ يَقُولُ: [لِمَاذَا مَزَّقْتَ ثِيَابَكَ؟ لِيَأْتِ إِلَيَّ فَيَعْلَمَ أَنَّهُ يُوجَدُ نَبِيٌّ فِي إِسْرَائِيلَ].
\par 9 فَجَاءَ نُعْمَانُ بِخَيْلِهِ وَمَرْكَبَاتِهِ وَوَقَفَ عِنْدَ بَابِ بَيْتِ أَلِيشَعَ.
\par 10 فَأَرْسَلَ إِلَيْهِ أَلِيشَعُ رَسُولاً يَقُولُ: [اذْهَبْ وَاغْتَسِلْ سَبْعَ مَرَّاتٍ فِي الأُرْدُنِّ فَيَرْجِعَ لَحْمُكَ إِلَيْكَ وَتَطْهُرَ].
\par 11 فَغَضِبَ نُعْمَانُ وَمَضَى وَقَالَ: [هُوَذَا قُلْتُ إِنَّهُ يَخْرُجُ إِلَيَّ وَيَقِفُ وَيَدْعُو بِاسْمِ الرَّبِّ إِلَهِهِ وَيُرَدِّدُ يَدَهُ فَوْقَ الْمَوْضِعِ فَيَشْفِي الأَبْرَصَ!
\par 12 أَلَيْسَ أَبَانَةُ وَفَرْفَرُ نَهْرَا دِمَشْقَ أَحْسَنَ مِنْ جَمِيعِ مِيَاهِ إِسْرَائِيلَ؟ أَمَا كُنْتُ أَغْتَسِلُ بِهِمَا فَأَطْهُرَ؟] وَرَجَعَ وَمَضَى بِغَيْظٍ.
\par 13 فَتَقَدَّمَ عَبِيدُهُ وَقَالُوا: [يَا أَبَانَا، لَوْ قَالَ لَكَ النَّبِيُّ أَمْراً عَظِيماً أَمَا كُنْتَ تَعْمَلُهُ، فَكَمْ بِالْحَرِيِّ إِذْ قَالَ لَكَ: اغْتَسِلْ وَاطْهُرْ؟].
\par 14 فَنَزَلَ وَغَطَسَ فِي الأُرْدُنِّ سَبْعَ مَرَّاتٍ حَسَبَ قَوْلِ رَجُلِ اللَّهِ، فَرَجَعَ لَحْمُهُ كَلَحْمِ صَبِيٍّ صَغِيرٍ وَطَهُرَ.
\par 15 فَرَجَعَ إِلَى رَجُلِ اللَّهِ هُوَ وَكُلُّ جَيْشِهِ وَدَخَلَ وَوَقَفَ أَمَامَهُ وَقَالَ: [هُوَذَا قَدْ عَرَفْتُ أَنَّهُ لَيْسَ إِلَهٌ فِي كُلِّ الأَرْضِ إِلاَّ فِي إِسْرَائِيلَ. وَالآنَ فَخُذْ بَرَكَةً مِنْ عَبْدِكَ].
\par 16 فَقَالَ: [حَيٌّ هُوَ الرَّبُّ الَّذِي أَنَا وَاقِفٌ أَمَامَهُ إِنِّي لاَ آخُذُ]. وَأَلَحَّ عَلَيْهِ أَنْ يَأْخُذَ فَأَبَى.
\par 17 فَقَالَ نُعْمَانُ: [أَمَا يُعْطَى لِعَبْدِكَ حِمْلُ بَغْلَيْنِ مِنَ التُّرَابِ، لأَنَّهُ لاَ يُقَرِّبُ بَعْدُ عَبْدُكَ مُحْرَقَةً وَلاَ ذَبِيحَةً لِآلِهَةٍ أُخْرَى بَلْ لِلرَّبِّ.
\par 18 عَنْ هَذَا الأَمْرِ يَصْفَحُ الرَّبُّ لِعَبْدِكَ: عِنْدَ دُخُولِ سَيِّدِي إِلَى بَيْتِ رِمُّونَ لِيَسْجُدَ هُنَاكَ وَيَسْتَنِدَ عَلَى يَدِي فَأَسْجُدُ فِي بَيْتِ رِمُّونَ، فَعِنْدَ سُجُودِي فِي بَيْتِ رِمُّونَ يَصْفَحُ الرَّبُّ لِعَبْدِكَ عَنْ هَذَا الأَمْرِ].
\par 19 فَقَالَ لَهُ: [امْضِ بِسَلاَمٍ]. وَلَمَّا مَضَى مِنْ عِنْدِهِ مَسَافَةً مِنَ الأَرْضِ
\par 20 قَالَ جِيحَزِي غُلاَمُ أَلِيشَعَ رَجُلِ اللَّهِ: [هُوَذَا سَيِّدِي قَدِ امْتَنَعَ عَنْ أَنْ يَأْخُذَ مِنْ يَدِ نُعْمَانَ الأَرَامِيِّ هَذَا مَا أَحْضَرَهُ. حَيٌّ هُوَ الرَّبُّ إِنِّي أَجْرِي وَرَاءَهُ وَآخُذُ مِنْهُ شَيْئاً].
\par 21 فَسَارَ جِيحَزِي وَرَاءَ نُعْمَانَ. وَلَمَّا رَآهُ نُعْمَانُ رَاكِضاً وَرَاءَهُ نَزَلَ عَنِ الْمَرْكَبَةِ لِلِقَائِهِ وَقَالَ: [أَسَلاَمٌ؟]
\par 22 فَقَالَ: [سَلاَمٌ. إِنَّ سَيِّدِي قَدْ أَرْسَلَنِي قَائِلاً: هُوَذَا فِي هَذَا الْوَقْتِ قَدْ جَاءَ إِلَيَّ غُلاَمَانِ مِنْ جَبَلِ أَفْرَايِمَ مِنْ بَنِي الأَنْبِيَاءِ، فَأَعْطِهِمَا وَزْنَةَ فِضَّةٍ وَحُلَّتَيْ ثِيَابٍ].
\par 23 فَقَالَ نُعْمَانُ: [اقْبَلْ وَخُذْ وَزْنَتَيْنِ]. وَأَلَحَّ عَلَيْهِ وَصَرَّ وَزْنَتَيْ فِضَّةٍ فِي كِيسَيْنِ وَحُلَّتَيِ الثِّيَابِ وَدَفَعَهَا لِغُلاَمَيْهِ فَحَمَلاَهَا قُدَّامَهُ.
\par 24 وَلَمَّا وَصَلَ إِلَى الأَكَمَةِ أَخَذَهَا مِنْ أَيْدِيهِمَا وَأَوْدَعَهَا فِي الْبَيْتِ وَأَطْلَقَ الرَّجُلَيْنِ فَانْطَلَقَا.
\par 25 وَأَمَّا هُوَ فَدَخَلَ وَوَقَفَ أَمَامَ سَيِّدِهِ. فَقَالَ لَهُ أَلِيشَعُ: [مِنْ أَيْنَ يَا جِيحَزِي؟] فَقَالَ: [لَمْ يَذْهَبْ عَبْدُكَ إِلَى هُنَا أَوْ هُنَاكَ].
\par 26 فَقَالَ لَهُ: [أَلَمْ يَذْهَبْ قَلْبِي حِينَ رَجَعَ الرَّجُلُ مِنْ مَرْكَبَتِهِ لِلِقَائِكَ؟ أَهُوَ وَقْتٌ لأَخْذِ الْفِضَّةِ وَلأَخْذِ ثِيَابٍ وَزَيْتُونٍ وَكُرُومٍ وَغَنَمٍ وَبَقَرٍ وَعَبِيدٍ وَجَوَارٍ؟
\par 27 فَبَرَصُ نُعْمَانَ يَلْصَقُ بِكَ وَبِنَسْلِكَ إِلَى الأَبَدِ!] فَخَرَجَ مِنْ أَمَامِهِ أَبْرَصَ كَالثَّلْجِ.

\chapter{6}

\par 1 وَقَالَ بَنُو الأَنْبِيَاءِ لأَلِيشَعَ: [هُوَذَا الْمَوْضِعُ الَّذِي نَحْنُ مُقِيمُونَ فِيهِ أَمَامَكَ ضَيِّقٌ عَلَيْنَا.
\par 2 فَلْنَذْهَبْ إِلَى الأُرْدُنِّ وَنَأْخُذْ مِنْ هُنَاكَ كُلُّ وَاحِدٍ خَشَبَةً، وَنَعْمَلْ لأَنْفُسِنَا هُنَاكَ مَوْضِعاً لِنُقِيمَ فِيهِ]. فَقَالَ: [اذْهَبُوا].
\par 3 فَقَالَ وَاحِدٌ: [اقْبَلْ وَاذْهَبْ مَعَ عَبِيدِكَ]. فَقَالَ: [إِنِّي أَذْهَبُ].
\par 4 فَانْطَلَقَ مَعَهُمْ. وَلَمَّا وَصَلُوا إِلَى الأُرْدُنِّ قَطَعُوا خَشَباً.
\par 5 وَإِذْ كَانَ وَاحِدٌ يَقْطَعُ خَشَبَةً وَقَعَ الْحَدِيدُ فِي الْمَاءِ. فَصَرَخَ: [آهِ يَا سَيِّدِي لأَنَّهُ عَارِيَةٌ!]
\par 6 فَقَالَ رَجُلُ اللَّهِ: [أَيْنَ سَقَطَ؟] فَأَرَاهُ الْمَوْضِعَ، فَقَطَعَ عُوداً وَأَلْقَاهُ هُنَاكَ، فَطَفَا الْحَدِيدُ.
\par 7 فَقَالَ: [ارْفَعْهُ لِنَفْسِكَ]. فَمَدَّ يَدَهُ وَأَخَذَهُ.
\par 8 وَأَمَّا مَلِكُ أَرَامَ فَكَانَ يُحَارِبُ إِسْرَائِيلَ، وَتَآمَرَ مَعَ عَبِيدِهِ قَائِلاً: [فِي الْمَكَانِ الْفُلاَنِيِّ تَكُونُ مَحَلَّتِي].
\par 9 فَأَرْسَلَ رَجُلُ اللَّهِ إِلَى مَلِكِ إِسْرَائِيلَ يَقُولُ: [احْذَرْ مِنْ أَنْ تَعْبُرَ بِهَذَا الْمَوْضِعِ، لأَنَّ الأَرَامِيِّينَ حَالُّونَ هُنَاكَ].
\par 10 فَأَرْسَلَ مَلِكُ إِسْرَائِيلَ إِلَى الْمَوْضِعِ الَّذِي قَالَ لَهُ عَنْهُ رَجُلُ اللَّهِ وَحَذَّرَهُ مِنْهُ وَتَحَفَّظَ هُنَاكَ، لاَ مَرَّةً وَلاَ مَرَّتَيْنِ.
\par 11 فَاضْطَرَبَ قَلْبُ مَلِكِ أَرَامَ مِنْ هَذَا الأَمْرِ، وَدَعَا عَبِيدَهُ وَقَالَ لَهُمْ: [أَمَا تُخْبِرُونَنِي مَنْ مِنَّا هُوَ لِمَلِكِ إِسْرَائِيلَ!]
\par 12 فَقَالَ وَاحِدٌ مِنْ عَبِيدِهِ: [لَيْسَ هَكَذَا يَا سَيِّدِي الْمَلِكَ. وَلَكِنَّ أَلِيشَعَ النَّبِيَّ الَّذِي فِي إِسْرَائِيلَ يُخْبِرُ مَلِكَ إِسْرَائِيلَ بِالأُمُورِ الَّتِي تَتَكَلَّمُ بِهَا فِي مِخْدَعِكِ].
\par 13 فَقَالَ: [اذْهَبُوا وَانْظُرُوا أَيْنَ هُوَ فَأُرْسِلَ وَآخُذَهُ]. فَأُخْبِرَ: [هُوَ فِي دُوثَانَ].
\par 14 فَأَرْسَلَ إِلَى هُنَاكَ خَيْلاً وَمَرْكَبَاتٍ وَجَيْشاً ثَقِيلاً، وَجَاءُوا لَيْلاً وَأَحَاطُوا بِالْمَدِينَةِ.
\par 15 فَبَكَّرَ خَادِمُ رَجُلِ اللَّهِ وَقَامَ وَخَرَجَ وَإِذَا جَيْشٌ مُحِيطٌ بِالْمَدِينَةِ وَخَيْلٌ وَمَرْكَبَاتٌ. فَقَالَ غُلاَمُهُ لَهُ: [آهِ يَا سَيِّدِي! كَيْفَ نَعْمَلُ؟]
\par 16 فَقَالَ: [لاَ تَخَفْ، لأَنَّ الَّذِينَ مَعَنَا أَكْثَرُ مِنَ الَّذِينَ مَعَهُمْ].
\par 17 وَصَلَّى أَلِيشَعُ وَقَالَ: [يَا رَبُّ، افْتَحْ عَيْنَيْهِ فَيُبْصِرَ]. فَفَتَحَ الرَّبُّ عَيْنَيِ الْغُلاَمِ فَأَبْصَرَ، وَإِذَا الْجَبَلُ مَمْلُوءٌ خَيْلاً وَمَرْكَبَاتِ نَارٍ حَوْلَ أَلِيشَعَ.
\par 18 وَلَمَّا نَزَلُوا إِلَيْهِ صَلَّى أَلِيشَعُ إِلَى الرَّبِّ: [اضْرِبْ هَؤُلاَءِ الأُمَمَ بِالْعَمَى]. فَضَرَبَهُمْ بِالْعَمَى كَقَوْلِ أَلِيشَعَ.
\par 19 فَقَالَ لَهُمْ أَلِيشَعُ: [لَيْسَتْ هَذِهِ هِيَ الطَّرِيقَ، وَلاَ هَذِهِ هِيَ الْمَدِينَةَ. اتْبَعُونِي فَأَسِيرَ بِكُمْ إِلَى الرَّجُلِ الَّذِي تُفَتِّشُونَ عَلَيْهِ]. فَسَارَ بِهِمْ إِلَى السَّامِرَةِ.
\par 20 فَلَمَّا دَخَلُوا السَّامِرَةَ قَالَ أَلِيشَعُ: [يَا رَبُّ افْتَحْ أَعْيُنَ هَؤُلاَءِ فَيُبْصِرُوا]. فَفَتَحَ الرَّبُّ أَعْيُنَهُمْ فَأَبْصَرُوا وَإِذَا هُمْ فِي وَسَطِ السَّامِرَةِ.
\par 21 فَقَالَ مَلِكُ إِسْرَائِيلَ لأَلِيشَعَ لَمَّا رَآهُمْ: [هَلْ أَضْرِبُ؟ هَلْ أَضْرِبُ يَا أَبِي؟]
\par 22 فَقَالَ: [لاَ تَضْرِبْ! تَضْرِبُ الَّذِينَ سَبَيْتَهُمْ بِسَيْفِكَ وَبِقَوْسِكَ. ضَعْ خُبْزاً وَمَاءً أَمَامَهُمْ فَيَأْكُلُوا وَيَشْرَبُوا ثُمَّ يَنْطَلِقُوا إِلَى سَيِّدِهِمْ].
\par 23 فَأَوْلَمَ لَهُمْ وَلِيمَةً عَظِيمَةً فَأَكَلُوا وَشَرِبُوا، ثُمَّ أَطْلَقَهُمْ فَانْطَلَقُوا إِلَى سَيِّدِهِمْ. وَلَمْ تَعُدْ أَيْضاً جُيُوشُ أَرَامَ تَدْخُلُ إِلَى أَرْضِ إِسْرَائِيلَ.
\par 24 وَكَانَ بَعْدَ ذَلِكَ أَنَّ بَنْهَدَدَ مَلِكَ أَرَامَ جَمَعَ كُلَّ جَيْشِهِ وَصَعِدَ فَحَاصَرَ السَّامِرَةَ.
\par 25 وَكَانَ جُوعٌ شَدِيدٌ فِي السَّامِرَةِ. وَهُمْ حَاصَرُوهَا حَتَّى صَارَ رَأْسُ الْحِمَارِ بِثَمَانِينَ مِنَ الْفِضَّةِ وَرُبْعُ الْقَابِ مِنْ زِبْلِ الْحَمَامِ بِخَمْسٍ مِنَ الْفِضَّةِ.
\par 26 وَبَيْنَمَا كَانَ مَلِكُ إِسْرَائِيلَ جَائِزاً عَلَى السُّورِ صَرَخَتِ امْرَأَةٌ إِلَيْهِ: [خَلِّصْ يَا سَيِّدِي الْمَلِكَ].
\par 27 فَقَالَ: [لاَ! يُخَلِّصْكِ الرَّبُّ. مِنْ أَيْنَ أُخَلِّصُكِ؟ أَمِنَ الْبَيْدَرِ أَوْ مِنَ الْمِعْصَرَةِ؟]
\par 28 ثُمَّ قَالَ لَهَا الْمَلِكُ: [مَا لَكِ؟] فَقَالَتْ: [هَذِهِ الْمَرْأَةُ قَالَتْ لِي: هَاتِي ابْنَكِ فَنَأْكُلَهُ الْيَوْمَ ثُمَّ نَأْكُلَ ابْنِي غَداً.
\par 29 فَسَلَقْنَا ابْنِي وَأَكَلْنَاهُ. ثُمَّ قُلْتُ لَهَا فِي الْيَوْمِ الآخَرِ: هَاتِي ابْنَكِ فَنَأْكُلَهُ فَخَبَّأَتِ ابْنَهَا].
\par 30 فَلَمَّا سَمِعَ الْمَلِكُ كَلاَمَ الْمَرْأَةِ مَزَّقَ ثِيَابَهُ وَهُوَ مُجْتَازٌ عَلَى السُّورِ، فَنَظَرَ الشَّعْبُ وَإِذَا مِسْحٌ مِنْ دَاخِلٍ عَلَى جَسَدِهِ.
\par 31 فَقَالَ: [هَكَذَا يَصْنَعُ لِي اللَّهُ وَهَكَذَا يَزِيدُ إِنْ قَامَ رَأْسُ أَلِيشَعَ بْنِ شَافَاطَ عَلَيْهِ الْيَوْمَ].
\par 32 وَكَانَ أَلِيشَعُ جَالِساً فِي بَيْتِهِ وَالشُّيُوخُ جُلُوساً عِنْدَهُ. فَأَرْسَلَ رَجُلاً مِنْ أَمَامِهِ. وَقَبْلَمَا أَتَى الرَّسُولُ إِلَيْهِ قَالَ لِلشُّيُوخِ: [هَلْ رَأَيْتُمْ أَنَّ ابْنَ الْقَاتِلِ هَذَا قَدْ أَرْسَلَ لِيَقْطَعَ رَأْسِي؟ انْظُرُوا إِذَا جَاءَ الرَّسُولُ فَأَغْلِقُوا الْبَابَ وَاحْصُرُوهُ عِنْدَ الْبَابِ. أَلَيْسَ صَوْتُ قَدَمَيْ سَيِّدِهِ وَرَاءَهُ؟].
\par 33 وَبَيْنَمَا هُوَ يُكَلِّمُهُمْ إِذَا بِالرَّسُولِ نَازِلٌ إِلَيْهِ. فَقَالَ: [هُوَذَا هَذَا الشَّرُّ هُوَ مِنْ قِبَلِ الرَّبِّ. مَاذَا أَنْتَظِرُ مِنَ الرَّبِّ بَعْدُ؟].

\chapter{7}

\par 1 وَقَالَ أَلِيشَعُ: [اسْمَعُوا كَلاَمَ الرَّبِّ. هَكَذَا قَالَ الرَّبُّ: فِي مِثْلِ هَذَا الْوَقْتِ غَداً تَكُونُ كَيْلَةُ الدَّقِيقِ بِشَاقِلٍ وَكَيْلَتَا الشَّعِيرِ بِشَاقِلٍ فِي بَابِ السَّامِرَةِ].
\par 2 وَإِنَّ جُنْدِيّاً لِلْمَلِكِ كَانَ يَسْتَنِدُ عَلَى يَدِهِ قَالَ لِرَجُلِ اللَّهِ: [هُوَذَا الرَّبُّ يَصْنَعُ كُوًى فِي السَّمَاءِ! هَلْ يَكُونُ هَذَا الأَمْرُ؟] فَقَالَ: [إِنَّكَ تَرَى بِعَيْنَيْكَ وَلَكِنْ لاَ تَأْكُلُ مِنْهُ].
\par 3 وَكَانَ أَرْبَعَةُ رِجَالٍ بُرْصٍ عِنْدَ مَدْخَلِ الْبَابِ. فَقَالَ أَحَدُهُمْ لِصَاحِبِهِ: [لِمَاذَا نَحْنُ جَالِسُونَ هُنَا حَتَّى نَمُوتَ؟
\par 4 إِذَا قُلْنَا نَدْخُلُ الْمَدِينَةَ، فَالْجُوعُ فِي الْمَدِينَةِ فَنَمُوتُ فِيهَا. وَإِذَا جَلَسْنَا هُنَا نَمُوتُ. فَالآنَ هَلُمَّ نَسْقُطْ إِلَى مَحَلَّةِ الأَرَامِيِّينَ، فَإِنِ اسْتَحْيُونَا حَيِينَا وَإِنْ قَتَلُونَا مُتْنَا].
\par 5 فَقَامُوا فِي الْعِشَاءِ لِيَذْهَبُوا إِلَى مَحَلَّةِ الأَرَامِيِّينَ. فَجَاءُوا إِلَى آخِرِ مَحَلَّةِ الأَرَامِيِّينَ فَلَمْ يَكُنْ هُنَاكَ أَحَدٌ.
\par 6 فَإِنَّ الرَّبَّ أَسْمَعَ جَيْشَ الأَرَامِيِّينَ صَوْتَ مَرْكَبَاتٍ وَصَوْتَ خَيْلٍ، صَوْتَ جَيْشٍ عَظِيمٍ. فَقَالُوا الْوَاحِدُ لأَخِيهِ: [هُوَذَا مَلِكُ إِسْرَائِيلَ قَدِ اسْتَأْجَرَ ضِدَّنَا مُلُوكَ الْحِثِّيِّينَ وَمُلُوكَ الْمِصْرِيِّينَ لِيَأْتُوا عَلَيْنَا].
\par 7 فَقَامُوا وَهَرَبُوا فِي الْعِشَاءِ وَتَرَكُوا خِيَامَهُمْ وَخَيْلَهُمْ وَحَمِيرَهُمُ، وَالْمَحَلَّةَ كَمَا هِيَ، وَهَرَبُوا لِنَجَاةِ أَنْفُسِهِمْ.
\par 8 وَجَاءَ هَؤُلاَءِ الْبُرْصُ إِلَى آخِرِ الْمَحَلَّةِ وَدَخَلُوا خَيْمَةً وَاحِدَةً، فَأَكَلُوا وَشَرِبُوا وَحَمَلُوا مِنْهَا فِضَّةً وَذَهَباً وَثِيَاباً وَمَضُوا وَطَمَرُوهَا. ثُمَّ رَجَعُوا وَدَخَلُوا خَيْمَةً أُخْرَى وَحَمَلُوا مِنْهَا وَمَضُوا وَطَمَرُوا.
\par 9 ثُمَّ قَالَ بَعْضُهُمْ لِبَعْضٍ: [لَسْنَا عَامِلِينَ حَسَناً. هَذَا الْيَوْمُ هُوَ يَوْمُ بِشَارَةٍ وَنَحْنُ سَاكِتُونَ! فَإِنِ انْتَظَرْنَا إِلَى ضُوءِ الصَّبَاحِ يُصَادِفُنَا شَرٌّ. فَهَلُمَّ الآنَ نَدْخُلْ وَنُخْبِرْ بَيْتَ الْمَلِكِ].
\par 10 فَجَاءُوا وَدَعُوا بَوَّابَ الْمَدِينَةِ وَأَخْبَرُوهُ: [إِنَّنَا دَخَلْنَا مَحَلَّةَ الأَرَامِيِّينَ فَلَمْ يَكُنْ هُنَاكَ أَحَدٌ وَلاَ صَوْتُ إِنْسَانٍ، وَلَكِنْ خَيْلٌ مَرْبُوطَةٌ وَحَمِيرٌ مَرْبُوطَةٌ وَخِيَامٌ كَمَا هِيَ].
\par 11 فَدَعَا الْبَوَّابِينَ فَأَخْبَرُوا بَيْتَ الْمَلِكِ دَاخِلاً.
\par 12 فَقَامَ الْمَلِكُ لَيْلاً وَقَالَ لِعَبِيدِهِ: [لَأُخْبِرَنَّكُمْ مَا فَعَلَ لَنَا الأَرَامِيُّونَ. عَلِمُوا أَنَّنَا جِيَاعٌ فَخَرَجُوا مِنَ الْمَحَلَّةِ لِيَخْتَبِئُوا فِي حَقْلٍ قَائِلِينَ: إِذَا خَرَجُوا مِنَ الْمَدِينَةِ قَبَضْنَا عَلَيْهِمْ أَحْيَاءً وَدَخَلْنَا الْمَدِينَةَ].
\par 13 فَأَجَابَ وَاحِدٌ مِنْ عَبِيدِهِ: [فَلْيَأْخُذُوا خَمْسَةً مِنَ الْخَيْلِ الْبَاقِيَةِ الَّتِي بَقِيَتْ فِيهَا. هِيَ نَظِيرُ كُلِّ جُمْهُورِ إِسْرَائِيلَ الَّذِينَ بَقُوا بِهَا، أَوْ هِيَ نَظِيرُ كُلِّ جُمْهُورِ إِسْرَائِيلَ الَّذِينَ فَنُوا. فَنُرْسِلُ وَنَرَى].
\par 14 فَأَخَذُوا مَرْكَبَتَيْ خَيْلٍ. وَأَرْسَلَ الْمَلِكُ وَرَاءَ جَيْشِ الأَرَامِيِّينَ قَائِلاً: [اذْهَبُوا وَانْظُرُوا].
\par 15 فَانْطَلَقُوا وَرَاءَهُمْ إِلَى الأُرْدُنِّ، وَإِذَا كُلُّ الطَّرِيقِ مَلآنٌ ثِيَاباً وَآنِيَةً قَدْ طَرَحَهَا الأَرَامِيُّونَ مِنْ عَجَلَتِهِمْ. فَرَجَعَ الرُّسُلُ وَأَخْبَرُوا الْمَلِكَ.
\par 16 فَخَرَجَ الشَّعْبُ وَنَهَبُوا مَحَلَّةَ الأَرَامِيِّينَ. فَكَانَتْ كَيْلَةُ الدَّقِيقِ بِشَاقِلٍ وَكَيْلَتَا الشَّعِيرِ بِشَاقِلٍ حَسَبَ كَلاَمِ الرَّبِّ.
\par 17 وَأَقَامَ الْمَلِكُ عَلَى الْبَابِ الْجُنْدِيَّ الَّذِي كَانَ يَسْتَنِدُ عَلَى يَدِهِ، فَدَاسَهُ الشَّعْبُ فِي الْبَابِ فَمَاتَ كَمَا قَالَ رَجُلُ اللَّهِ الَّذِي تَكَلَّمَ عِنْدَ نُزُولِ الْمَلِكِ إِلَيْهِ.
\par 18 فَإِنَّهُ لَمَّا قَالَ رَجُلُ اللَّهِ لِلْمَلِكِ: [كَيْلَتَا شَعِيرٍ بِشَاقِلٍ وَكَيْلَةُ دَقِيقٍ بِشَاقِلٍ تَكُونُ فِي مِثْلِ هَذَا الْوَقْتِ غَداً فِي بَابِ السَّامِرَةِ]
\par 19 أَجَابَ الْجُنْدِيُّ رَجُلَ اللَّهِ: [هُوَذَا الرَّبُّ يَصْنَعُ كُوًى فِي السَّمَاءِ! هَلْ يَكُونُ مِثْلَ هَذَا الأَمْرِ؟] قَالَ: [إِنَّكَ تَرَى بِعَيْنَيْكَ وَلَكِنَّكَ لاَ تَأْكُلُ مِنْهُ].
\par 20 فَكَانَ لَهُ كَذَلِكَ. دَاسَهُ الشَّعْبُ فِي الْبَابِ فَمَاتَ.

\chapter{8}

\par 1 وَقَالَ أَلِيشَعُ لِلْمَرْأَةَ الَّتِي أَحْيَا ابْنَهَا: [قُومِي وَانْطَلِقِي أَنْتِ وَبَيْتُكِ وَتَغَرَّبِي حَيْثُمَا تَتَغَرَّبِي. لأَنَّ الرَّبَّ قَدْ دَعَا بِجُوعٍ فَيَأْتِي أَيْضاً عَلَى الأَرْضِ سَبْعَ سِنِينٍ].
\par 2 فَقَامَتِ الْمَرْأَةُ وَفَعَلَتْ حَسَبَ كَلاَمِ رَجُلِ اللَّهِ، وَانْطَلَقَتْ هِيَ وَبَيْتُهَا وَتَغَرَّبَتْ فِي أَرْضِ الْفِلِسْطِينِيِّينَ سَبْعَ سِنِينٍ
\par 3 وَفِي نِهَايَةِ السِّنِينِ السَّبْعِ رَجَعَتِ الْمَرْأَةُ مِنْ أَرْضِ الْفِلِسْطِينِيِّينَ وَخَرَجَتْ لِتَصْرُخَ إِلَى الْمَلِكِ لأَجْلِ بَيْتِهَا وَحَقْلِهَا.
\par 4 وَقَالَ الْمَلِكُ لِجِيحَزِي غُلاَمِ رَجُلِ اللَّهِ: [قُصَّ عَلَيَّ جَمِيعَ الْعَظَائِمِ الَّتِي فَعَلَهَا أَلِيشَعُ].
\par 5 وَفِيمَا هُوَ يَقُصُّ عَلَى الْمَلِكِ كَيْفَ أَنَّهُ أَحْيَا الْمَيِّتَ إِذَا بِالْمَرْأَةِ الَّتِي أَحْيَا ابْنَهَا تَصْرُخُ إِلَى الْمَلِكِ لأَجْلِ بَيْتِهَا وَحَقْلِهَا. فَقَالَ جِيحَزِي: [يَا سَيِّدِي الْمَلِكَ، هَذِهِ هِيَ الْمَرْأَةُ وَهَذَا هُوَ ابْنُهَا الَّذِي أَحْيَاهُ أَلِيشَعُ].
\par 6 فَسَأَلَ الْمَلِكُ الْمَرْأَةَ فَقَصَّتْ عَلَيْهِ ذَلِكَ، فَأَعْطَاهَا الْمَلِكُ خَصِيّاً قَائِلاً: [أَرْجِعْ كُلَّ مَا لَهَا وَجَمِيعَ غَلاَّتِ الْحَقْلِ مِنْ حِينَِ تَرَكَتِ الأَرْضَ إِلَى الآنَ].
\par 7 وَجَاءَ أَلِيشَعُ إِلَى دِمَشْقَ. وَكَانَ بَنْهَدَدُ مَلِكُ أَرَامَ مَرِيضاً، فَقِيلَ لَهُ: [قَدْ جَاءَ رَجُلُ اللَّهِ إِلَى هُنَا].
\par 8 فَقَالَ الْمَلِكُ لِحَزَائِيلَ: [خُذْ بِيَدِكَ هَدِيَّةً وَاذْهَبْ لاِسْتِقْبَالِ رَجُلِ اللَّهِ، وَاسْأَلِ الرَّبَّ بِهِ: هَلْ أَشْفَى مِنْ مَرَضِي هَذَا].
\par 9 فَذَهَبَ حَزَائِيلُ لاِسْتِقْبَالِهِ وَأَخَذَ هَدِيَّةً بِيَدِهِ، وَمِنْ كُلِّ خَيْرَاتِ دِمَشْقَ حِمْلَ أَرْبَعِينَ جَمَلاً وَجَاءَ وَوَقَفَ أَمَامَهُ وَقَالَ: [إِنَّ ابْنَكَ بَنْهَدَدَ مَلِكَ أَرَامَ قَدْ أَرْسَلَنِي إِلَيْكَ قَائِلاً: هَلْ أُشْفَى مِنْ مَرَضِي هَذَا؟]
\par 10 فَقَالَ لَهُ أَلِيشَعُ: [اذْهَبْ وَقُلْ لَهُ شِفَاءً تُشْفَى. وَقَدْ أَرَانِي الرَّبُّ أَنَّهُ يَمُوتُ مَوْتاً].
\par 11 فَجَعَلَ نَظَرَهُ عَلَيْهِ وَثَبَّتَهُ حَتَّى خَجِلَ. فَبَكَى رَجُلُ اللَّهِ.
\par 12 فَقَالَ حَزَائِيلُ: [لِمَاذَا يَبْكِي سَيِّدِي؟] فَقَالَ: [لأَنِّي عَلِمْتُ مَا سَتَفْعَلُهُ بِبَنِي إِسْرَائِيلَ مِنَ الشَّرِّ، فَإِنَّكَ تُطْلِقُ النَّارَ فِي حُصُونِهِمْ وَتَقْتُلُ شُبَّانَهُمْ بِالسَّيْفِ وَتُحَطِّمُ أَطْفَالَهُمْ وَتَشُقُّ حَوَامِلَهُمْ].
\par 13 فَقَالَ حَزَائِيلُ: [وَمَنْ هُوَ عَبْدُكَ الْكَلْبُ حَتَّى يَفْعَلَ هَذَا الأَمْرَ الْعَظِيمَ؟] فَقَالَ أَلِيشَعُ: [قَدْ أَرَانِي الرَّبُّ إِيَّاكَ مَلِكاً عَلَى أَرَامَ].
\par 14 فَانْطَلَقَ مِنْ عِنْدِ أَلِيشَعَ وَدَخَلَ إِلَى سَيِّدِهِ فَسَأَلَهُ: [مَاذَا قَالَ لَكَ أَلِيشَعُ؟] فَقَالَ: [قَالَ لِي إِنَّكَ تَحْيَا].
\par 15 وَفِي الْغَدِ أَخَذَ اللِّبْدَةَ وَغَمَسَهَا بِالْمَاءِ وَنَشَرَهَا عَلَى وَجْهِهِ وَمَاتَ، وَمَلَكَ حَزَائِيلُ عِوَضاً عَنْهُ.
\par 16 وَفِي السَّنَةِ الْخَامِسَةِ لِيُورَامَ بْنِ أَخْآبَ مَلِكِ إِسْرَائِيلَ وَيَهُوشَافَاطَ مَلِكِ يَهُوذَا، مَلَكَ يَهُورَامُ بْنُ يَهُوشَافَاطَ مَلِكِ يَهُوذَا.
\par 17 كَانَ ابْنَ اثْنَتَيْنِ وَثَلاَثِينَ سَنَةً حِينَ مَلَكَ، وَمَلَكَ ثَمَانِي سِنِينٍَ فِي أُورُشَلِيمَ.
\par 18 وَسَارَ فِي طَرِيقِ مُلُوكِ إِسْرَائِيلَ كَمَا فَعَلَ بَيْتُ أَخْآبَ، لأَنَّ بِنْتَ أَخْآبَ كَانَتْ لَهُ امْرَأَةً. وَعَمِلَ الشَّرَّ فِي عَيْنَيِ الرَّبِّ.
\par 19 وَلَمْ يَشَإِ الرَّبُّ أَنْ يُبِيدَ يَهُوذَا مِنْ أَجْلِ دَاوُدَ عَبْدِهِ، كَمَا قَالَ إِنَّهُ يُعْطِيهِ سِرَاجاً وَلِبَنِيهِ كُلَّ الأَيَّامِ.
\par 20 فِي أَيَّامِهِ عَصَى أَدُومُ عَلَى يَهُوذَا وَمَلَّكُوا عَلَى أَنْفُسِهِمْ مَلِكاً.
\par 21 وَعَبَرَ يَهورَامُ إِلَى صَعِيرَ وَجَمِيعُ الْمَرْكَبَاتِ مَعَهُ، وَقَامَ لَيْلاً وَضَرَبَ أَدُومَ الْمُحِيطَ بِهِ وَرُؤَسَاءَ الْمَرْكَبَاتِ. وَهَرَبَ الشَّعْبُ إِلَى خِيَامِهِمْ.
\par 22 وَعَصَى أَدُومُ عَلَى يَهُوذَا إِلَى هَذَا الْيَوْمِ. حِينَئِذٍ عَصَتْ لِبْنَةُ فِي ذَلِكَ الْوَقْتِ.
\par 23 وَبَقِيَّةُ أُمُورِ يَهُورَامَ وَكُلُّ مَا صَنَعَ مَكْتُوبَةٌ فِي سِفْرِ أَخْبَارِ الأَيَّامِ لِمُلُوكِ يَهُوذَا.
\par 24 وَاضْطَجَعَ يَهُورَامُ مَعَ آبَائِهِ، وَدُفِنَ مَعَ آبَائِهِ فِي مَدِينَةِ دَاوُدَ، وَمَلَكَ أَخَزْيَا ابْنُهُ عِوَضاً عَنْهُ.
\par 25 فِي السَّنَةِ الثَّانِيَةَ عَشَرَةَ لِيُورَامَ بْنِ أَخْآبَ مَلِكِ إِسْرَائِيلَ، مَلَكَ أَخَزْيَا بْنُ يَهُورَامَ مَلِكِ يَهُوذَا.
\par 26 وَكَانَ أَخَزْيَا ابْنَ اثْنَتَيْنِ وَعِشْرِينَ سَنَةً حِينَ مَلَكَ، وَمَلَكَ سَنَةً وَاحِدَةً فِي أُورُشَلِيمَ. وَاسْمُ أُمِّهِ عَثَلْيَا بِنْتُ عُمْرِي مَلِكِ إِسْرَائِيلَ.
\par 27 وَسَارَ فِي طَرِيقِ بَيْتِ أَخْآبَ، وَعَمِلَ الشَّرَّ فِي عَيْنَيِ الرَّبِّ كَبَيْتِ أَخْآبَ لأَنَّهُ كَانَ صِهْرَ بَيْتِ أَخْآبَ.
\par 28 وَانْطَلَقَ مَعَ يُورَامَ بْنِ أَخْآبَ لِمُقَاتَلَةِ حَزَائِيلَ مَلِكِ أَرَامَ فِي رَامُوتِ جِلْعَادَ، فَضَرَبَ الأَرَامِيُّونَ يُورَامَ.
\par 29 فَرَجَعَ يُورَامُ الْمَلِكُ لِيَبْرَأَ فِي يَزْرَعِيلَ مِنَ الْجُرُوحِ الَّتِي جَرَحَهُ بِهَا الأَرَامِيُّونَ فِي رَامُوتَ عِنْدَ مُقَاتَلَتِهِ حَزَائِيلَ مَلِكَ أَرَامَ. وَنَزَلَ أَخَزْيَا بْنُ يَهُورَامَ مَلِكُ يَهُوذَا لِيَرَى يُورَامَ بْنَ أَخْآبَ فِي يَزْرَعِيلَ لأَنَّهُ كَانَ مَرِيضاً.

\chapter{9}

\par 1 وَدَعَا أَلِيشَعُ النَّبِيُّ وَاحِداً مِنْ بَنِي الأَنْبِيَاءِ وَقَالَ لَهُ: [شُدَّ حَقَوَيْكَ وَخُذْ قِنِّينَةَ الدُّهْنِ هَذِهِ بِيَدِكَ وَاذْهَبْ إِلَى رَامُوتَ جِلْعَادَ.
\par 2 وَإِذَا وَصَلْتَهَا فَانْظُرْ هُنَاكَ يَاهُوَ بْنَ يَهُوشَافَاطَ بْنَ نِمْشِي وَادْخُلْ وَأَقِمْهُ مِنْ وَسَطِ إِخْوَتِهِ وَادْخُلْ بِهِ إِلَى مِخْدَعٍ دَاخِلَ مِخْدَعٍ
\par 3 ثُمَّ خُذْ قِنِّينَةَ الدُّهْنِ وَصُبَّ عَلَى رَأْسِهِ وَقُلْ: هَكَذَا قَالَ الرَّبُّ: قَدْ مَسَحْتُكَ مَلِكاً عَلَى إِسْرَائِيلَ. ثُمَّ افْتَحِ الْبَابَ وَاهْرُبْ وَلاَ تَنْتَظِرْ].
\par 4 فَانْطَلَقَ النَّبِيُّ إِلَى رَامُوتَ جِلْعَادَ
\par 5 وَدَخَلَ وَإِذَا قُوَّادُ الْجَيْشِ جُلُوسٌ. فَقَالَ: [لِي كَلاَمٌ مَعَكَ يَا قَائِدُ]. فَقَالَ يَاهُو: [مَعَ مَنْ مِنَّا كُلِّنَا]. فَقَالَ: [مَعَكَ أَيُّهَا الْقَائِدُ].
\par 6 فَقَامَ وَدَخَلَ الْبَيْتَ، فَصَبَّ الدُّهْنَ عَلَى رَأْسِهِ وَقَالَ لَهُ: [هَكَذَا قَالَ الرَّبُّ إِلَهُ إِسْرَائِيلَ: قَدْ مَسَحْتُكَ مَلِكاً عَلَى شَعْبِ الرَّبِّ إِسْرَائِيلَ،
\par 7 فَتَضْرِبُ بَيْتَ أَخْآبَ سَيِّدِكَ. وَأَنْتَقِمُ لِدِمَاءِ عَبِيدِيَ الأَنْبِيَاءِ وَدِمَاءِ جَمِيعِ عَبِيدِ الرَّبِّ مِنْ يَدِ إِيزَابَلَ.
\par 8 فَيَبِيدُ كُلُّ بَيْتِ أَخْآبَ، وَأَسْتَأْصِلُ لأَخْآبَ كُلَّ ذَكَرٍ وَمَحْجُوزٍ وَمُطْلَقٍ فِي إِسْرَائِيلَ.
\par 9 وَأَجْعَلُ بَيْتَ أَخْآبَ كَبَيْتِ يَرُبْعَامَ بْنِ نَبَاطَ وَكَبَيْتِ بَعْشَا بْنِ أَخِيَّا.
\par 10 وَتَأْكُلُ الْكِلاَبُ إِيزَابَلَ فِي حَقْلِ يَزْرَعِيلَ وَلَيْسَ مَنْ يَدْفِنُهَا]. ثُمَّ فَتَحَ الْبَابَ وَهَرَبَ.
\par 11 وَأَمَّا يَاهُو فَخَرَجَ إِلَى عَبِيدِ سَيِّدِهِ فَقِيلَ لَهُ: [أَسَلاَمٌ؟ لِمَاذَا جَاءَ هَذَا الْمَجْنُونُ إِلَيْكَ؟] فَقَالَ لَهُمْ: [أَنْتُمْ تَعْرِفُونَ الرَّجُلَ وَكَلاَمَهُ].
\par 12 فَقَالُوا: [كَذِبٌ. فَأَخْبِرْنَا]. فَقَالَ: [بِكَذَا وَكَذَا قَالَ لِي: هَكَذَا قَالَ الرَّبُّ: قَدْ مَسَحْتُكَ مَلِكاً عَلَى إِسْرَائِيلَ].
\par 13 فَبَادَرَ كُلُّ وَاحِدٍ وَأَخَذَ ثَوْبَهُ وَوَضَعَهُ تَحْتَهُ عَلَى الدَّرَجِ نَفْسِهِ، وَضَرَبُوا بِالْبُوقِ وَقَالُوا: [قَدْ مَلَكَ يَاهُو].
\par 14 وَعَصَى يَاهُو بْنُ يَهُوشَافَاطَ بْنِ نِمْشِي عَلَى يُورَامَ. وَكَانَ يُورَامُ يُحَافِظُ عَلَى رَامُوتَ جِلْعَادَ هُوَ وَكُلُّ إِسْرَائِيلَ مِنْ حَزَائِيلَ مَلِكِ أَرَامَ.
\par 15 وَرَجَعَ يُورَامُ الْمَلِكُ لِيَبْرَأَ فِي يَزْرَعِيلَ مِنَ الْجُرُوحِ الَّتِي ضَرَبَهُ بِهَا الأَرَامِيُّونَ حِينَ قَاتَلَ حَزَائِيلَ مَلِكَ أَرَامَ. فَقَالَ يَاهُو: [إِنْ كَانَ فِي أَنْفُسِكُمْ، لاَ يَخْرُجْ مُنْهَزِمٌ مِنَ الْمَدِينَةِ لِيَنْطَلِقَ فَيُخْبِرَ فِي يَزْرَعِيلَ].
\par 16 وَرَكِبَ يَاهُو وَذَهَبَ إِلَى يَزْرَعِيلَ، لأَنَّ يُورَامَ كَانَ مُضْطَجِعاً هُنَاكَ. وَنَزَلَ أَخَزْيَا مَلِكُ يَهُوذَا لِيَرَى يُورَامَ.
\par 17 وَكَانَ الرَّقِيبُ وَاقِفاً عَلَى الْبُرْجِ فِي يَزْرَعِيلَ، فَرَأَى جَمَاعَةَ يَاهُو عِنْدَ إِقْبَالِهِ، فَقَالَ: [إِنِّي أَرَى جَمَاعَةً]. فَقَالَ يُورَامُ: [خُذْ فَارِساً وَأَرْسِلْهُ لِلِقَائِهِمْ فَيَقُولَ: [أَسَلاَمٌ؟]
\par 18 فَذَهَبَ رَاكِبُ الْفَرَسِ لِلِقَائِهِ وَقَالَ: [هَكَذَا يَقُولُ الْمَلِكُ: أَسَلاَمٌ؟] فَقَالَ يَاهُو: [مَا لَكَ وَلِلسَّلاَمِ؟ دُرْ إِلَى وَرَائِي!] فَقَالَ الرَّقِيبُ: [قَدْ وَصَلَ الرَّسُولُ إِلَيْهِمْ وَلَمْ يَرْجِعْ].
\par 19 فَأَرْسَلَ رَاكِبَ فَرَسٍ ثَانِياً. فَلَمَّا وَصَلَ إِلَيْهِمْ قَالَ: [هَكَذَا يَقُولُ الْمَلِكُ: أَسَلاَمٌ؟] فَقَالَ يَاهُو: [مَا لَكَ وَلِلسَّلاَمِ؟ دُرْ إِلَى وَرَائِي].
\par 20 فَقَالَ الرَّقِيبُ: [قَدْ وَصَلَ إِلَيْهِمْ وَلَمْ يَرْجِعْ. وَالسَّوْقُ كَسَوْقِ يَاهُوَ بْنِ نِمْشِي، لأَنَّهُ يَسُوقُ بِجُنُونٍ].
\par 21 فَقَالَ يُورَامُ: [اشْدُدْ]. فَشُدَّتْ مَرْكَبَتُهُ، وَخَرَجَ يُورَامُ مَلِكُ إِسْرَائِيلَ وَأَخَزْيَا مَلِكُ يَهُوذَا، كُلُّ وَاحِدٍ فِي مَرْكَبَتِهِ، خَرَجَا لِلِقَاءِ يَاهُو. فَصَادَفَاهُ عِنْدَ حَقْلَةِ نَابُوتَ الْيَزْرَعِيلِيِّ.
\par 22 فَلَمَّا رَأَى يُورَامُ يَاهُوَ قَالَ: [أَسَلاَمٌ يَا يَاهُو؟] فَقَالَ: [أَيُّ سَلاَمٍ مَا دَامَ زِنَى إِيزَابَلَ أُمِّكَ وَسِحْرُهَا الْكَثِيرُ؟]
\par 23 فَرَدَّ يَهُورَامُ يَدَيْهِ وَهَرَبَ وَقَالَ لأَخَزْيَا: [خِيَانَةً يَا أَخَزْيَا!]
\par 24 فَقَبَضَ يَاهُو بِيَدِهِ عَلَى الْقَوْسِ وَضَرَبَ يُورَامَ بَيْنَ ذِرَاعَيْهِ، فَخَرَجَ السَّهْمُ مِنْ قَلْبِهِ فَسَقَطَ فِي مَرْكَبَتِهِ.
\par 25 وَقَالَ لِبِدْقَرَ ثَالِثِهِ: [ارْفَعْهُ وَأَلْقِهِ فِي حِصَّةِ حَقْلِ نَابُوتَ الْيَزْرَعِيلِيِّ. وَاذْكُرْ كَيْفَ إِذْ رَكِبْتُ أَنَا وَإِيَّاكَ مَعاً وَرَاءَ أَخْآبَ أَبِيهِ جَعَلَ الرَّبُّ عَلَيْهِ هَذَا الْحُكْمَ.
\par 26 أَلَمْ أَرَ أَمْساً دَمَ نَابُوتَ وَدِمَاءَ بَنِيهِ يَقُولُ الرَّبُّ، فَأُجَازِيكَ فِي هَذِهِ الْحَقْلَةِ يَقُولُ الرَّبُّ. فَالآنَ ارْفَعْهُ وَأَلْقِهِ فِي الْحَقْلَةِ حَسَبَ قَوْلِ الرَّبِّ].
\par 27 وَلَمَّا رَأَى ذَلِكَ أَخَزْيَا مَلِكُ يَهُوذَا هَرَبَ فِي طَرِيقِ بَيْتِ الْبُسْتَانِ، فَطَارَدَهُ يَاهُو وَقَالَ: [اضْرِبُوهُ]. فَضَرَبُوهُ أَيْضاً فِي الْمَرْكَبَةِ فِي عَقَبَةِ جُورَ الَّتِي عِنْدَ يِبْلَعَامَ. فَهَرَبَ إِلَى مَجِدُّو وَمَاتَ هُنَاكَ.
\par 28 فَأَرْكَبَهُ عَبِيدُهُ إِلَى أُورُشَلِيمَ وَدَفَنُوهُ فِي قَبْرِهِ مَعَ آبَائِهِ فِي مَدِينَةِ دَاوُدَ.
\par 29 فِي السَّنَةِ الْحَادِيَةَ عَشَرَةَ لِيُورَامَ بْنِ أَخْآبَ، مَلَكَ أَخَزْيَا عَلَى يَهُوذَا.
\par 30 فَجَاءَ يَاهُو إِلَى يَزْرَعِيلَ. وَلَمَّا سَمِعَتْ إِيزَابَلُ كَحَّلَتْ بِالأُثْمُدِ عَيْنَيْهَا وَزَيَّنَتْ رَأْسَهَا وَتَطَلَّعَتْ مِنْ كُوَّةٍ.
\par 31 وَعِنْدَ دُخُولِ يَاهُو الْبَابَ قَالَتْ: [أَسَلاَمٌ لِزِمْرِي قَاتِلِ سَيِّدِهِ؟]
\par 32 فَرَفَعَ وَجْهَهُ نَحْوَ الْكُوَّةِ وَقَالَ: [مَنْ مَعِي؟ مَنْ؟] فَأَشْرَفَ عَلَيْهِ اثْنَانِ أَوْ ثَلاَثَةٌ مِنَ الْخِصْيَانِ.
\par 33 فَقَالَ: [اطْرَحُوهَا]. فَطَرَحُوهَا، فَسَالَ مِنْ دَمِهَا عَلَى الْحَائِطِ وَعَلَى الْخَيْلِ فَدَاسَهَا.
\par 34 وَدَخَلَ وَأَكَلَ وَشَرِبَ ثُمَّ قَالَ: [افْتَقِدُوا هَذِهِ الْمَلْعُونَةَ وَادْفِنُوهَا لأَنَّهَا بِنْتُ مَلِكٍ].
\par 35 وَلَمَّا مَضُوا لِيَدْفِنُوهَا لَمْ يَجِدُوا مِنْهَا إِلاَّ الْجُمْجُمَةَ وَالرِّجْلَيْنِ وَكَفَّيِ الْيَدَيْنِ.
\par 36 فَرَجَعُوا وَأَخْبَرُوهُ. فَقَالَ: [إِنَّهُ كَلاَمُ الرَّبِّ الَّذِي تَكَلَّمَ بِهِ عَنْ يَدِ عَبْدِهِ إِيلِيَّا التِّشْبِيِّ قَائِلاً: فِي حَقْلِ يَزْرَعِيلَ تَأْكُلُ الْكِلاَبُ لَحْمَ إِيزَابَلَ.
\par 37 وَتَكُونُ جُثَّةُ إِيزَابَلَ كَدِمْنَةٍ عَلَى وَجْهِ الْحَقْلِ فِي قِسْمِ يَزْرَعِيلَ حَتَّى لاَ يَقُولُوا هَذِهِ إِيزَابَلُ].

\chapter{10}

\par 1 وَكَانَ لأَخْآبَ سَبْعُونَ ابْناً فِي السَّامِرَةِ. فَكَتَبَ يَاهُو رَسَائِلَ وَأَرْسَلَهَا إِلَى السَّامِرَةِ إِلَى رُؤَسَاءِ يَزْرَعِيلَ الشُّيُوخِ وَإِلَى مُرَبِّي أَخْآبَ قَائِلاً:
\par 2 [فَالآنَ عِنْدَ وُصُولِ هَذِهِ الرِّسَالَةِ إِلَيْكُمْ، إِذْ عِنْدَكُمْ بَنُو سَيِّدِكُمْ وَعِنْدَكُمْ مَرْكَبَاتٌ وَخَيْلٌ وَمَدِينَةٌ مُحَصَّنَةٌ وَسِلاَحٌ،
\par 3 انْظُرُوا الأَفْضَلَ وَالأَصْلَحَ مِنْ بَنِي سَيِّدِكُمْ وَاجْعَلُوهُ عَلَى كُرْسِيِّ أَبِيهِ وَحَارِبُوا عَنْ بَيْتِ سَيِّدِكُمْ].
\par 4 فَخَافُوا جِدّاً جِدّاً وَقَالُوا: [هُوَذَا مَلِكَانِ لَمْ يَقِفَا أَمَامَهُ، فَكَيْفَ نَقِفُ نَحْنُ؟]
\par 5 فَأَرْسَلَ الَّذِي عَلَى الْبَيْتِ وَالَّذِي عَلَى الْمَدِينَةِ وَالشُّيُوخُ وَالْمُرَبُّونَ إِلَى يَاهُو قَائِلِينَ: [عَبِيدُكَ نَحْنُ، وَكُلَّ مَا قُلْتَ لَنَا نَفْعَلُهُ. لاَ نُمَلِّكُ أَحَداً. مَا يَحْسُنُ فِي عَيْنَيْكَ فَافْعَلْهُ].
\par 6 فَكَتَبَ إِلَيْهِمْ رِسَالَةً ثَانِيَةً قَائِلاً: [إِنْ كُنْتُمْ لِي وَسَمِعْتُمْ لِقَوْلِي، فَخُذُوا رُؤُوسَ الرِّجَالِ بَنِي سَيِّدِكُمْ وَتَعَالُوا إِلَيَّ فِي نَحْوِ هَذَا الْوَقْتِ غَداً إِلَى يَزْرَعِيلَ]. وَبَنُو الْمَلِكِ سَبْعُونَ رَجُلاً كَانُوا مَعَ عُظَمَاءِ الْمَدِينَةِ الَّذِينَ رَبُّوهُمْ.
\par 7 فَلَمَّا وَصَلَتِ الرِّسَالَةُ إِلَيْهِمْ أَخَذُوا بَنِي الْمَلِكِ وَقَتَلُوا سَبْعِينَ رَجُلاً وَوَضَعُوا رُؤُوسَهُمْ فِي سِلاَلٍ وَأَرْسَلُوهَا إِلَيْهِ إِلَى يَزْرَعِيلَ.
\par 8 فَجَاءَ الرَّسُولُ وَأَخْبَرَهُ: [قَدْ أَتُوا بِرُؤُوسِ بَنِي الْمَلِكِ]. فَقَالَ: [اجْعَلُوهَا كُومَتَيْنِ فِي مَدْخَلِ الْبَابِ إِلَى الصَّبَاحِ].
\par 9 وَفِي الصَّبَاحِ خَرَجَ وَوَقَفَ وَقَالَ لِجَمِيعِ الشَّعْبِ: [أَنْتُمْ أَبْرِيَاءُ. هَئَنَذَا قَدْ عَصَيْتُ عَلَى سَيِّدِي وَقَتَلْتُهُ، وَلَكِنْ مَنْ قَتَلَ كُلَّ هَؤُلاَءِ؟
\par 10 فَاعْلَمُوا الآنَ أَنَّهُ لاَ يَسْقُطُ مِنْ كَلاَمِ الرَّبِّ إِلَى الأَرْضِ الَّذِي تَكَلَّمَ بِهِ الرَّبُّ عَلَى بَيْتِ أَخْآبَ، وَقَدْ فَعَلَ الرَّبُّ مَا تَكَلَّمَ بِهِ عَنْ يَدِ عَبْدِهِ إِيلِيَّا].
\par 11 وَقَتَلَ يَاهُو كُلَّ الَّذِينَ بَقُوا لِبَيْتِ أَخْآبَ فِي يَزْرَعِيلَ وَكُلَّ عُظَمَائِهِ وَمَعَارِفِهِ وَكَهَنَتِهِ حَتَّى لَمْ يُبْقِ لَهُ شَارِداً.
\par 12 ثُمَّ قَامَ وَجَاءَ سَائِراً إِلَى السَّامِرَةِ. وَإِذْ كَانَ عِنْدَ بَيْتِ عَقْدِ الرُّعَاةِ فِي الطَّرِيقِ
\par 13 صَادَفَ يَاهُو إِخْوَةَ أَخَزْيَا مَلِكِ يَهُوذَا. فَقَالَ: [مَنْ أَنْتُمْ؟] فَقَالُوا: [نَحْنُ إِخْوَةُ أَخَزْيَا، وَنَحْنُ نَازِلُونَ لِنُسَلِّمَ عَلَى بَنِي الْمَلِكِ وَبَنِي الْمَلِكَةِ].
\par 14 فَقَالَ: [أَمْسِكُوهُمْ أَحْيَاءً]. فَأَمْسَكُوهُمْ أَحْيَاءً وَقَتَلُوهُمْ عِنْدَ بِئْرِ بَيْتِ عَقْدٍ، اثْنَيْنِ وَأَرْبَعِينَ رَجُلاً وَلَمْ يُبْقِ مِنْهُمْ أَحَداً.
\par 15 ثُمَّ انْطَلَقَ مِنْ هُنَاكَ فَصَادَفَ يَهُونَادَابَ بْنَ رَكَابٍ يُلاَقِيهِ، فَبَارَكَهُ وَقَالَ لَهُ: [هَلْ قَلْبُكَ مُسْتَقِيمٌ نَظِيرُ قَلْبِي مَعَ قَلْبِكَ؟] فَقَالَ يَهُونَادَابُ: [نَعَمْ]. فَقَالَ: [هَاتِ يَدَكَ]. فَأَعْطَاهُ يَدَهُ، فَأَصْعَدَهُ إِلَيْهِ إِلَى الْمَرْكَبَةِ.
\par 16 وَقَالَ: [هَلُمَّ مَعِي وَانْظُرْ غَيْرَتِي لِلرَّبِّ]. وَأَرْكَبَهُ مَعَهُ فِي مَرْكَبَتِهِ.
\par 17 وَجَاءَ إِلَى السَّامِرَةِ، وَقَتَلَ جَمِيعَ الَّذِينَ بَقُوا لأَخْآبَ فِي السَّامِرَةِ حَتَّى أَفْنَاهُ، حَسَبَ كَلاَمِ الرَّبِّ الَّذِي كَلَّمَ بِهِ إِيلِيَّا.
\par 18 ثُمَّ جَمَعَ يَاهُو كُلَّ الشَّعْبِ وَقَالَ لَهُمْ: [إِنَّ أَخْآبَ قَدْ عَبَدَ الْبَعْلَ قَلِيلاً، وَأَمَّا يَاهُو فَإِنَّهُ يَعْبُدُهُ كَثِيراً.
\par 19 وَالآنَ فَادْعُوا إِلَيَّ جَمِيعَ أَنْبِيَاءِ الْبَعْلِ وَكُلَّ عَابِدِيهِ وَكُلَّ كَهَنَتِهِ. لاَ يُفْقَدْ أَحَدٌ، لأَنَّ لِي ذَبِيحَةً عَظِيمَةً لِلْبَعْلِ. كُلُّ مَنْ فُقِدَ لاَ يَعِيشُ]. وَقَدْ فَعَلَ يَاهُو بِمَكْرٍ لِيُفْنِيَ عَبَدَةَ الْبَعْلِ.
\par 20 وَقَالَ يَاهُو: [قَدِّسُوا اعْتِكَافاً لِلْبَعْلِ]. فَنَادُوا بِهِ.
\par 21 وَأَرْسَلَ يَاهُو فِي كُلِّ إِسْرَائِيلَ فَأَتَى جَمِيعُ عَبَدَةِ الْبَعْلِ وَلَمْ يَبْقَ أَحَدٌ إِلاَّ أَتَى، وَدَخَلُوا بَيْتَ الْبَعْلِ فَامْتَلَأَ بَيْتُ الْبَعْلِ مِنْ جَانِبٍ إِلَى جَانِبٍ.
\par 22 فَقَالَ لِلَّذِي عَلَى الْمَلاَبِسِ: [أَخْرِجْ مَلاَبِسَ لِكُلِّ عَبَدَةِ الْبَعْلِ]. فَأَخْرَجَ لَهُمْ مَلاَبِسَ.
\par 23 وَدَخَلَ يَاهُو وَيَهُونَادَابُ بْنُ رَكَابٍ إِلَى بَيْتِ الْبَعْلِ. فَقَالَ لِعَبَدَةِ الْبَعْلِ: [فَتِّشُوا وَانْظُرُوا لِئَلاَّ يَكُونَ مَعَكُمْ هَهُنَا أَحَدٌ مِنْ عَبِيدِ الرَّبِّ، وَلَكِنَّ عَبَدَةَ الْبَعْلِ وَحْدَهُمْ].
\par 24 وَدَخَلُوا لِيُقَرِّبُوا ذَبَائِحَ وَمُحْرَقَاتٍ. وَأَمَّا يَاهُو فَأَقَامَ خَارِجاً ثَمَانِينَ رَجُلاً وَقَالَ: [الرَّجُلُ الَّذِي يَنْجُو مِنَ الرِّجَالِ الَّذِينَ أَتَيْتُ بِهِمْ إِلَى أَيْدِيكُمْ تَكُونُ أَنْفُسُكُمْ بَدَلَ نَفْسِهِ].
\par 25 وَلَمَّا انْتَهُوا مِنْ تَقْرِيبِ الْمُحْرَقَةِ قَالَ يَاهُو لِلسُّعَاةِ وَالثَّوَالِثِ: [ادْخُلُوا اضْرِبُوهُمْ. لاَ يَخْرُجْ أَحَدٌ]. فَضَرَبُوهُمْ بِحَدِّ السَّيْفِ، وَطَرَحَهُمُ السُّعَاةُ وَالثَّوَالِثُ. وَسَارُوا إِلَى مَدِينَةِ بَيْتِ الْبَعْلِ
\par 26 وَأَخْرَجُوا تَمَاثِيلَ بَيْتِ الْبَعْلِ وَأَحْرَقُوهَا،
\par 27 وَكَسَّرُوا تِمْثَالَ الْبَعْلِ وَهَدَمُوا بَيْتَ الْبَعْلِ وَجَعَلُوهُ مَزْبَلَةً إِلَى هَذَا الْيَوْمِ.
\par 28 وَاسْتَأْصَلَ يَاهُو الْبَعْلَ مِنْ إِسْرَائِيلَ.
\par 29 وَلَكِنَّ خَطَايَا يَرُبْعَامَ بْنِ نَبَاطَ الَّذِي جَعَلَ إِسْرَائِيلَ يُخْطِئُ لَمْ يَحِدْ يَاهُو عَنْهَا (أَيْ عُجُولِ الذَّهَبِ الَّتِي فِي بَيْتِ إِيلَ وَالَّتِي فِي دَانَ).
\par 30 وَقَالَ الرَّبُّ لِيَاهُو: [مِنْ أَجْلِ أَنَّكَ قَدْ أَحْسَنْتَ بِعَمَلِ مَا هُوَ مُسْتَقِيمٌ فِي عَيْنَيَّ، وَحَسَبَ كُلِّ مَا بِقَلْبِي فَعَلْتَ بِبَيْتِ أَخْآبَ، فَأَبْنَاؤُكَ إِلَى الْجِيلِ الرَّابِعِ يَجْلِسُونَ عَلَى كُرْسِيِّ إِسْرَائِيلَ].
\par 31 وَلَكِنْ يَاهُو لَمْ يَتَحَفَّظْ لِلسُّلُوكِ فِي شَرِيعَةِ الرَّبِّ إِلَهِ إِسْرَائِيلَ مِنْ كُلِّ قَلْبِهِ. لَمْ يَحِدْ عَنْ خَطَايَا يَرُبْعَامَ الَّذِي جَعَلَ إِسْرَائِيلَ يُخْطِئُ.
\par 32 فِي تِلْكَ الأَيَّامِ ابْتَدَأَ الرَّبُّ يَقُصُّ إِسْرَائِيلَ. فَضَرَبَهُمْ حَزَائِيلُ فِي جَمِيعِ تُخُومِ إِسْرَائِيلَ
\par 33 مِنَ الأُرْدُنِّ لِجِهَةِ مَشْرِقِ الشَّمْسِ، جَمِيعَ أَرْضِ جِلْعَادَ الْجَادِيِّينَ وَالرَّأُوبَيْنِيِّينَ وَالْمَنَسِّيِّينَ مِنْ عَرُوعِيرَ الَّتِي عَلَى وَادِي أَرْنُونَ وَجِلْعَادَ وَبَاشَانَ.
\par 34 وَبَقِيَّةُ أُمُورِ يَاهُو وَكُلُّ مَا عَمَلَ وَكُلُّ جَبَرُوتِهِ مَكْتُوبَةٌ فِي سِفْرِ أَخْبَارِ الأَيَّامِ لِمُلُوكِ إِسْرَائِيلَ.
\par 35 وَاضْطَجَعَ يَاهُو مَعَ آبَائِهِ فَدَفَنُوهُ فِي السَّامِرَةِ، وَمَلَكَ يَهُوأَحَازُ ابْنُهُ عِوَضاً عَنْهُ.
\par 36 وَكَانَتِ الأَيَّامُ الَّتِي مَلَكَ فِيهَا يَاهُو عَلَى إِسْرَائِيلَ فِي السَّامِرَةِ ثَمَانِياً وَعِشْرِينَ سَنَةً.

\chapter{11}

\par 1 فَلَمَّا رَأَتْ عَثَلْيَا أُمُّ أَخَزْيَا أَنَّ ابْنَهَا قَدْ مَاتَ، قَامَتْ فَأَبَادَتْ جَمِيعَ النَّسْلِ الْمَلِكِيِّ.
\par 2 فَأَخَذَتْ يَهُوشَبَعُ بِنْتُ الْمَلِكِ يَهُورَامَ (أُخْتُ أَخَزْيَا) يَهُوآشَ بْنَ أَخَزْيَا وَسَرِقَتْهُ مِنْ وَسَطِ بَنِي الْمَلِكِ الَّذِينَ قُتِلُوا هُوَ وَمُرْضِعَتَهُ مِنْ مِخْدَعِ السَّرِيرِ، وَخَبَّأُوهُ مِنْ وَجْهِ عَثَلْيَا فَلَمْ يُقْتَلْ.
\par 3 وَكَانَ مَعَهَا فِي بَيْتِ الرَّبِّ مُخْتَبِئاً سِتَّ سِنِينَ وَعَثَلْيَا مَالِكَةٌ عَلَى الأَرْضِ.
\par 4 وَفِي السَّنَةِ السَّابِعَةِ أَرْسَلَ يَهُويَادَاعُ فَأَخَذَ رُؤَسَاءَ مِئَاتِ الْجَلاَّدِينَ وَالسُّعَاةِ وَأَدْخَلَهُمْ إِلَيْهِ إِلَى بَيْتِ الرَّبِّ، وَقَطَعَ مَعَهُمْ عَهْداً وَاسْتَحْلَفَهُمْ فِي بَيْتِ الرَّبِّ وَأَرَاهُمُ ابْنَ الْمَلِكِ.
\par 5 وَأَمَرَهُمْ: [هَذَا مَا تَفْعَلُونَهُ. الثُّلْثُ مِنْكُمُ الَّذِينَ يَدْخُلُونَ فِي السَّبْتِ يَحْرُسُونَ حِرَاسَةَ بَيْتِ الْمَلِكِ،
\par 6 وَالثُّلْثُ عَلَى بَابِ سُورٍ، وَالثُّلْثُ عَلَى الْبَابِ وَرَاءَ السُّعَاةِ. فَتَحْرُسُونَ حِرَاسَةَ الْبَيْتِ لِلصَّدِّ.
\par 7 وَالْفِرْقَتَانِ مِنْكُمْ، جَمِيعُ الْخَارِجِينَ فِي السَّبْتِ، يَحْرُسُونَ حِرَاسَةَ بَيْتِ الرَّبِّ حَوْلَ الْمَلِكِ.
\par 8 وَتُحِيطُونَ بِالْمَلِكِ حَوَالَيْهِ، كُلُّ وَاحِدٍ سِلاَحُهُ بِيَدِهِ. وَمَنْ دَخَلَ الصُّفُوفَ يُقْتَلُ. وَكُونُوا مَعَ الْمَلِكِ فِي خُرُوجِهِ وَدُخُولِهِ.
\par 9 فَفَعَلَ رُؤَسَاءُ الْمِئَاتِ حَسَبَ كُلِّ مَا أَمَرَ بِهِ يَهُويَادَاعُ الْكَاهِنُ، وَأَخَذُوا كُلُّ وَاحِدٍ رِجَالَهُ الدَّاخِلِينَ فِي السَّبْتِ مَعَ الْخَارِجِينَ فِي السَّبْتِ وَجَاءُوا إِلَى يَهُويَادَاعَ الْكَاهِنِ.
\par 10 فَأَعْطَى الْكَاهِنُ لِرُؤَسَاءِ الْمِئَاتِ الْحِرَابَ وَالأَتْرَاسَ الَّتِي لِلْمَلِكِ دَاوُدَ الَّتِي فِي بَيْتِ الرَّبِّ.
\par 11 وَوَقَفَ السُّعَاةُ كُلُّ وَاحِدٍ سِلاَحُهُ بِيَدِهِ مِنْ جَانِبِ الْبَيْتِ الأَيْمَنِ إِلَى جَانِبِ الْبَيْتِ الأَيْسَرِ حَوْلَ الْمَذْبَحِ وَالْبَيْتِ، حَوْلَ الْمَلِكِ مُسْتَدِيرِينَ.
\par 12 وَأَخْرَجَ ابْنَ الْمَلِكِ وَوَضَعَ عَلَيْهِ التَّاجَ وَأَعْطَاهُ الشَّهَادَةَ، فَمَلَّكُوهُ وَمَسَحُوهُ وَصَفَّقُوا وَقَالُوا: [لِيَحْيَ الْمَلِكُ]
\par 13 وَلَمَّا سَمِعَتْ عَثَلْيَا صَوْتَ السُّعَاةِ وَالشَّعْبِ دَخَلَتْ إِلَى الشَّعْبِ إِلَى بَيْتِ الرَّبِّ،
\par 14 وَنَظَرَتْ وَإِذَا الْمَلِكُ وَاقِفٌ عَلَى الْمِنْبَرِ حَسَبَ الْعَادَةِ، وَالرُّؤَسَاءُ وَنَافِخُو الأَبْوَاقِ بِجَانِبِ الْمَلِكِ، وَكُلُّ شَعْبِ الأَرْضِ يَفْرَحُونَ وَيَضْرِبُونَ بِالأَبْوَاقِ. فَشَقَّتْ عَثَلْيَا ثِيَابَهَا وَصَرَخَتْ: [خِيَانَةٌ خِيَانَةٌ!]
\par 15 فَأَمَرَ يَهُويَادَاعُ الْكَاهِنُ رُؤَسَاءَ الْمِئَاتِ قُوَّادَ الْجَيْشِ: [أَخْرِجُوهَا إِلَى خَارِجِ الصُّفُوفِ، وَالَّذِي يَتْبَعُهَا اقْتُلُوهُ بِالسَّيْفِ]. لأَنَّ الْكَاهِنَ قَالَ: [لاَ تُقْتَلُ فِي بَيْتِ الرَّبِّ].
\par 16 فَأَلْقُوا عَلَيْهَا الأَيَادِيَ، وَمَضَتْ فِي طَرِيقِ مَدْخَلِ الْخَيْلِ إِلَى بَيْتِ الْمَلِكِ وَقُتِلَتْ هُنَاكَ.
\par 17 وَقَطَعَ يَهُويَادَاعُ عَهْداً بَيْنَ الرَّبِّ وَبَيْنَ الْمَلِكِ وَالشَّعْبِ لِيَكُونُوا شَعْباً لِلرَّبِّ، وَبَيْنَ الْمَلِكِ وَالشَّعْبِ.
\par 18 وَدَخَلَ جَمِيعُ شَعْبِ الأَرْضِ إِلَى بَيْتِ الْبَعْلِ وَهَدَمُوا مَذَابِحَهُ وَكَسَّرُوا تَمَاثِيلَهُ تَمَاماً، وَقَتَلُوا مَتَّانَ كَاهِنَ الْبَعْلِ أَمَامَ الْمَذَابِحِ. وَجَعَلَ الْكَاهِنُ نُظَّاراً عَلَى بَيْتِ الرَّبِّ.
\par 19 وَأَخَذَ رُؤَسَاءَ الْمِئَاتِ وَالْجَلاَّدِينَ وَالسُّعَاةَ وَكُلَّ شَعْبِ الأَرْضِ، فَأَنْزَلُوا الْمَلِكَ مِنْ بَيْتِ الرَّبِّ وَأَتُوا فِي طَرِيقِ بَابِ السُّعَاةِ إِلَى بَيْتِ الْمَلِكِ، فَجَلَسَ عَلَى كُرْسِيِّ الْمُلُوكِ.
\par 20 وَفَرِحَ جَمِيعُ شَعْبِ الأَرْضِ. وَاسْتَرَاحَتِ الْمَدِينَةُ. وَقَتَلُوا عَثَلْيَا بِالسَّيْفِ عِنْدَ بَيْتِ الْمَلِكِ.
\par 21 كَانَ يَهُوآشُ ابْنَ سَبْعِ سِنِينَ حِينَ مَلَكَ.

\chapter{12}

\par 1 فِي السَّنَةِ السَّابِعَةِ لِيَاهُو، مَلَكَ يَهُوآشُ. مَلَكَ أَرْبَعِينَ سَنَةً فِي أُورُشَلِيمَ. وَاسْمُ أُمِّهِ ظَبْيَةُ، مِنْ بِئْرِ سَبْعٍ.
\par 2 وَعَمِلَ يَهُوآشُ مَا هُوَ مُسْتَقِيمٌ فِي عَيْنَيِ الرَّبِّ كُلَّ أَيَّامِهِ الَّتِي فِيهَا عَلَّمَهُ يَهُويَادَاعُ الْكَاهِنُ.
\par 3 إِلاَّ أَنَّ الْمُرْتَفَعَاتِ لَمْ تُنْتَزَعْ، بَلْ كَانَ الشَّعْبُ لاَ يَزَالُونَ يَذْبَحُونَ وَيُوقِدُونَ عَلَى الْمُرْتَفَعَاتِ.
\par 4 وَقَالَ يَهُوآشُ لِلْكَهَنَةِ: [جَمِيعُ فِضَّةِ الأَقْدَاسِ الَّتِي أُدْخِلَتْ إِلَى بَيْتِ الرَّبِّ، الْفِضَّةُ الرَّائِجَةُ، فِضَّةُ كُلِّ وَاحِدٍ حَسَبَ النُّفُوسِ الْمُقَوَّمَةِ، كُلُّ فِضَّةٍ يَخْطُرُ بِبَالِ إِنْسَانٍ أَنْ يُدْخِلَهَا إِلَى بَيْتِ الرَّبِّ،
\par 5 لِيَأْخُذَهَا الْكَهَنَةُ لأَنْفُسِهِمْ كُلُّ وَاحِدٍ مِنْ عِنْدِ صَاحِبِهِ وَهُمْ يُرَمِّمُونَ مَا تَهَدَّمَ مِنَ الْبَيْتِ، كُلَّ مَا وُجِدَ فِيهِ مُتَهَدِّماً].
\par 6 وَفِي السَّنَةِ الثَّالِثَةِ وَالْعِشْرِينَ لِلْمَلِكِ يَهُوآشَ لَمْ تَكُنِ الْكَهَنَةُ رَمَّمُوا مَا تَهَدَّمَ مِنَ الْبَيْتِ.
\par 7 فَدَعَا الْمَلِكُ يَهُوآشُ يَهُويَادَاعَ الْكَاهِنَ وَالْكَهَنَةَ وَسَأَلَهُمْ: [لِمَاذَا لَمْ تُرَمِّمُوا مَا تَهَدَّمَ مِنَ الْبَيْتِ؟ فَالآنَ لاَ تَأْخُذُوا فِضَّةً مِنْ عِنْدِ أَصْحَابِكُمْ، بَلِ اجْعَلُوهَا لِمَا تَهَدَّمَ مِنَ الْبَيْتِ].
\par 8 فَوَافَقَ الْكَهَنَةُ عَلَى أَنْ لاَ يَأْخُذُوا فِضَّةً مِنَ الشَّعْبِ وَلاَ يُرَمِّمُوا مَا تَهَدَّمَ مِنَ الْبَيْتِ.
\par 9 فَأَخَذَ يَهُويَادَاعُ الْكَاهِنُ صُنْدُوقاً وَثَقَبَ ثَقْباً فِي غِطَائِهِ، وَجَعَلَهُ بِجَانِبِ الْمَذْبَحِ عَنِ الْيَمِينِ عِنْدَ دُخُولِ الإِنْسَانِ إِلَى بَيْتِ الرَّبِّ. وَالْكَهَنَةُ حَارِسُو الْبَابِ جَعَلُوا فِيهِ كُلَّ الْفِضَّةِ الْمُدْخَلَةِ إِلَى بَيْتِ الرَّبِّ.
\par 10 وَكَانَ لَمَّا رَأَوُا الْفِضَّةَ قَدْ كَثُرَتْ فِي الصُّنْدُوقِ أَنَّهُ صَعِدَ كَاتِبُ الْمَلِكِ وَالْكَاهِنُ الْعَظِيمُ وَصَرُّوا وَحَسَبُوا الْفِضَّةَ الْمَوْجُودَةَ فِي بَيْتِ الرَّبِّ.
\par 11 وَدَفَعُوا الْفِضَّةَ الْمَحْسُوبَةَ إِلَى أَيْدِي عَامِلِي الشُّغْلِ الْمُوَكَّلِينَ عَلَى بَيْتِ الرَّبِّ، وَأَنْفَقُوهَا لِلنَّجَّارِينَ وَالْبَنَّائِينَ الْعَامِلِينَ فِي بَيْتِ الرَّبِّ،
\par 12 وَلِبَنَّائِي الْحِيطَانِ وَنَحَّاتِي الْحِجَارَةِ وَلِشِرَاءِ الأَخْشَابِ وَالْحِجَارَةِ الْمَنْحُوتَةِ لِتَرْمِيمِ مَا تَهَدَّمَ مِنْ بَيْتِ الرَّبِّ وَلِكُلِّ مَا يُنْفَقُ عَلَى الْبَيْتِ لِتَرْمِيمِهِ.
\par 13 إِلاَّ أَنَّهُ لَمْ يُعْمَلْ لِبَيْتِ الرَّبِّ طُسُوسُ فِضَّةٍ وَلاَ مِقَصَّاتٌ وَلاَ مَنَاضِحُ وَلاَ أَبْوَاقٌ، كُلُّ آنِيَةِ الذَّهَبِ وَآنِيَةِ الْفِضَّةِ مِنَ الْفِضَّةِ الدَّاخِلَةِ إِلَى بَيْتِ الرَّبِّ،
\par 14 بَلْ كَانُوا يَدْفَعُونَهَا لِعَامِلِي الشُّغْلِ، فَكَانُوا يُرَمِّمُونَ بِهَا بَيْتَ الرَّبِّ.
\par 15 وَلَمْ يُحَاسِبُوا الرِّجَالَ الَّذِينَ سَلَّمُوهُمُ الْفِضَّةَ بِأَيْدِيهِمْ لِيُعْطُوهَا لِعَامِلِي الشُّغْلِ لأَنَّهُمْ كَانُوا يَعْمَلُونَ بِأَمَانَةٍ.
\par 16 وَأَمَّا فِضَّةُ ذَبِيحَةِ الإِثْمِ وَفِضَّةُ ذَبِيحَةِ الْخَطِيَّةِ فَلَمْ تُدْخَلْ إِلَى بَيْتِ الرَّبِّ، بَلْ كَانَتْ لِلْكَهَنَةِ.
\par 17 حِينَئِذٍ صَعِدَ حَزَائِيلُ مَلِكُ أَرَامَ وَحَارَبَ جَتَّ وَأَخَذَهَا. ثُمَّ حَوَّلَ حَزَائِيلُ وَجْهَهُ لِيَصْعَدَ إِلَى أُورُشَلِيمَ.
\par 18 فَأَخَذَ يَهُوآشُ مَلِكُ يَهُوذَا جَمِيعَ الأَقْدَاسِ الَّتِي قَدَّسَهَا يَهُوشَافَاطُ وَيَهُورَامُ وَأَخَزْيَا آبَاؤُهُ مُلُوكُ يَهُوذَا، وَأَقْدَاسَهُ وَكُلَّ الذَّهَبِ الْمَوْجُودِ فِي خَزَائِنِ بَيْتِ الرَّبِّ وَبَيْتِ الْمَلِكِ، وَأَرْسَلَهَا إِلَى حَزَائِيلَ مَلِكِ أَرَامَ فَصَعِدَ عَنْ أُورُشَلِيمَ.
\par 19 وَبَقِيَّةُ أُمُورِ يَهُوآشَ وَكُلُّ مَا عَمِلَ مَكْتُوبَةٌ فِي سِفْرِ أَخْبَارِ الأَيَّامِ لِمُلُوكِ يَهُوذَا.
\par 20 وَقَامَ عَبِيدُهُ وَفَتَنُوا فِتْنَةً وَقَتَلُوا يَهُوآشَ فِي بَيْتِ الْقَلْعَةِ حَيْثُ يَنْزِلُ إِلَى سَلَّى.
\par 21 لأَنَّ يُوزَاكَارَ بْنَ شِمْعَةَ وَيَهُوزَابَادَ بْنَ شُومِيرَ عَبْدَيْهِ ضَرَبَاهُ فَمَاتَ، فَدَفَنُوهُ مَعَ آبَائِهِ فِي مَدِينَةِ دَاوُدَ، وَمَلَكَ أَمَصْيَا ابْنُهُ عِوَضاً عَنْهُ.

\chapter{13}

\par 1 فِي السَّنَةِ الثَّالِثَةِ وَالْعِشْرِينَ لِيَهُوآشَ بْنِ أَخَزْيَا مَلِكِ يَهُوذَا، مَلَكَ يَهُوأَحَازُ بْنُ يَاهُو عَلَى إِسْرَائِيلَ فِي السَّامِرَةِ سَبْعَ عَشَرَةَ سَنَةً.
\par 2 وَعَمِلَ الشَّرَّ فِي عَيْنَيِ الرَّبِّ، وَسَارَ وَرَاءَ خَطَايَا يَرُبْعَامَ بْنِ نَبَاطَ الَّذِي جَعَلَ إِسْرَائِيلَ يُخْطِئُ. لَمْ يَحِدْ عَنْهَا.
\par 3 فَحَمِيَ غَضَبُ الرَّبِّ عَلَى إِسْرَائِيلَ فَدَفَعَهُمْ لِيَدِ حَزَائِيلَ مَلِكِ أَرَامَ، وَلِيَدِ بَنْهَدَدَ بْنِ حَزَائِيلَ كُلَّ الأَيَّامِ.
\par 4 وَتَضَرَّعَ يَهُوأَحَازُ إِلَى وَجْهِ الرَّبِّ، فَسَمِعَ لَهُ الرَّبُّ لأَنَّهُ رَأَى ضِيقَ إِسْرَائِيلَ، لأَنَّ مَلِكَ أَرَامَ ضَايَقَهُمْ.
\par 5 وَأَعْطَى الرَّبُّ إِسْرَائِيلَ مُخَلِّصاً، فَخَرَجُوا مِنْ تَحْتِ يَدِ الأَرَامِيِّينَ. وَأَقَامَ بَنُو إِسْرَائِيلَ فِي خِيَامِهِمْ كَأَمْسِ وَمَا قَبْلَهُ.
\par 6 وَلَكِنَّهُمْ لَمْ يَحِيدُوا عَنْ خَطَايَا بَيْتِ يَرُبْعَامَ الَّذِي جَعَلَ إِسْرَائِيلَ يُخْطِئُ بَلْ سَارُوا بِهَا، وَوَقَفَتِ السَّارِيَةُ أَيْضاً فِي السَّامِرَةِ.
\par 7 لأَنَّهُ لَمْ يُبْقِ لِيَهُوأَحَازَ شَعْباً إِلاَّ خَمْسِينَ فَارِساً وَعَشَرَ مَرْكَبَاتٍ وَعَشَرَةَ آلاَفِ رَاجِلٍ لأَنَّ مَلِكَ أَرَامَ أَفْنَاهُمْ وَوَضَعَهُمْ كَالتُّرَابِ لِلدَّوْسِ.
\par 8 وَبَقِيَّةُ أُمُورِ يَهُوأَحَازَ وَكُلُّ مَا عَمِلَ وَجَبَرُوتُهُ مَكْتُوبَةٌ فِي سِفْرِ أَخْبَارِ الأَيَّامِ لِمُلُوكِ إِسْرَائِيلَ.
\par 9 ثُمَّ اضْطَجَعَ يَهُوأَحَازُ مَعَ آبَائِهِ، فَدَفَنُوهُ فِي السَّامِرَةِ، وَمَلَكَ يُوآشُ ابْنُهُ عِوَضاً عَنْهُ.
\par 10 فِي السَّنَةِ السَّابِعَةِ وَالثَّلاَثِينَ لِيَهُوآشَ مَلِكِ يَهُوذَا، مَلَكَ يُوآشُ بْنُ يَهُوأَحَازَ عَلَى إِسْرَائِيلَ فِي السَّامِرَةِ سِتَّ عَشَرَةَ سَنَةً.
\par 11 وَعَمِلَ الشَّرَّ فِي عَيْنَيِ الرَّبِّ وَلَمْ يَحِدْ عَنْ جَمِيعِ خَطَايَا يَرُبْعَامَ بْنِ نَبَاطَ الَّذِي جَعَلَ إِسْرَائِيلَ يُخْطِئُ، بَلْ سَارَ بِهَا.
\par 12 وَبَقِيَّةُ أُمُورِ يُوآشَ وَكُلُّ مَا عَمِلَ وَجَبَرُوتُهُ وَكَيْفَ حَارَبَ أَمَصْيَا مَلِكَ يَهُوذَا مَكْتُوبَةٌ فِي سِفْرِ أَخْبَارِ الأَيَّامِ لِمُلُوكِ إِسْرَائِيلَ.
\par 13 ثُمَّ اضْطَجَعَ يُوآشُ مَعَ آبَائِهِ، وَجَلَسَ يَرُبْعَامُ عَلَى كُرْسِيِّهِ. وَدُفِنَ يُوآشُ فِي السَّامِرَةِ مَعَ مُلُوكِ إِسْرَائِيلَ.
\par 14 وَمَرِضَ أَلِيشَعُ مَرَضَهُ الَّذِي مَاتَ بِهِ، فَنَزَلَ إِلَيْهِ يُوآشُ مَلِكُ إِسْرَائِيلَ وَبَكَى عَلَى وَجْهِهِ وَقَالَ: [يَا أَبِي يَا أَبِي، يَا مَرْكَبَةَ إِسْرَائِيلَ وَفُرْسَانَهَا].
\par 15 فَقَالَ لَهُ أَلِيشَعُ: [خُذْ قَوْساً وَسِهَاماً]. فَأَخَذَ لِنَفْسِهِ قَوْساً وَسِهَاماً.
\par 16 ثُمَّ قَالَ لِمَلِكِ إِسْرَائِيلَ: [رَكِّبْ يَدَكَ عَلَى الْقَوْسِ]. فَرَكَّبَ يَدَهُ، ثُمَّ وَضَعَ أَلِيشَعُ يَدَهُ عَلَى يَدَيِ الْمَلِكِ
\par 17 وَقَالَ: [افْتَحِ الْكُوَّةَ لِجِهَةِ الشَّرْقِ]. فَفَتَحَهَا. فَقَالَ أَلِيشَعُ: [ارْمِ]. فَرَمَى. فَقَالَ: [سَهْمُ خَلاَصٍ لِلرَّبِّ وَسَهْمُ خَلاَصٍ مِنْ أَرَامَ، فَإِنَّكَ تَضْرِبُ أَرَامَ فِي أَفِيقَ إِلَى الْفَنَاءِ].
\par 18 ثُمَّ قَالَ: [خُذِ السِّهَامَ]. فَأَخَذَهَا. ثُمَّ قَالَ لِمَلِكِ إِسْرَائِيلَ: [اضْرِبْ عَلَى الأَرْضِ]. فَضَرَبَ ثَلاَثَ مَرَّاتٍ وَوَقَفَ.
\par 19 فَغَضِبَ عَلَيْهِ رَجُلُ اللَّهِ وَقَالَ: [لَوْ ضَرَبْتَ خَمْسَ أَوْ سِتَّ مَرَّاتٍ حِينَئِذٍ ضَرَبْتَ أَرَامَ إِلَى الْفَنَاءِ. وَأَمَّا الآنَ فَإِنَّكَ إِنَّمَا تَضْرِبُ أَرَامَ ثَلاَثَ مَرَّاتٍ].
\par 20 وَمَاتَ أَلِيشَعُ فَدَفَنُوهُ. وَكَانَ غُزَاةُ مُوآبَ تَدْخُلُ عَلَى الأَرْضِ عِنْدَ دُخُولِ السَّنَةِ.
\par 21 وَفِيمَا كَانُوا يَدْفِنُونَ رَجُلاً إِذَا بِهِمْ قَدْ رَأَوُا الْغُزَاةَ، فَطَرَحُوا الرَّجُلَ فِي قَبْرِ أَلِيشَعَ. فَلَمَّا نَزَلَ الرَّجُلُ وَمَسَّ عِظَامَ أَلِيشَعَ عَاشَ وَقَامَ عَلَى رِجْلَيْهِ.
\par 22 وَأَمَّا حَزَائِيلُ مَلِكُ أَرَامَ فَضَايَقَ إِسْرَائِيلَ كُلَّ أَيَّامِ يَهُوأَحَازَ،
\par 23 فَحَنَّ الرَّبُّ عَلَيْهِمْ وَرَحِمَهُمْ وَالْتَفَتَ إِلَيْهِمْ لأَجْلِ عَهْدِهِ مَعَ إِبْرَاهِيمَ وَإِسْحَاقَ وَيَعْقُوبَ، وَلَمْ يَشَأْ أَنْ يَسْتَأْصِلَهُمْ، وَلَمْ يَطْرَحْهُمْ عَنْ وَجْهِهِ حَتَّى الآنَ.
\par 24 ثُمَّ مَاتَ حَزَائِيلُ مَلِكُ أَرَامَ، وَمَلَكَ بَنْهَدَدُ ابْنُهُ عِوَضاً عَنْهُ.
\par 25 فَعَادَ يُوآشُ بْنُ يَهُوأَحَازَ وَأَخَذَ الْمُدُنَ مِنْ يَدِ بَنْهَدَدَ بْنِ حَزَائِيلَ الَّتِي أَخَذَهَا مِنْ يَدِ يَهُوأَحَازَ أَبِيهِ بِالْحَرْبِ. ضَرَبَهُ يُوآشُ ثَلاَثَ مَرَّاتٍ وَاسْتَرَدَّ مُدُنَ إِسْرَائِيلَ.

\chapter{14}

\par 1 فِي السَّنَةِ الثَّانِيَةِ لِيُوآشَ بْنِ يَهُوأَحَازَ مَلِكِ إِسْرَائِيلَ، مَلَكَ أَمَصْيَا بْنُ يَهُوآشَ مَلِكِ يَهُوذَا.
\par 2 كَانَ ابْنَ خَمْسٍ وَعِشْرِينَ سَنَةً حِينَ مَلَكَ. وَمَلَكَ تِسْعاً وَعِشْرِينَ سَنَةً فِي أُورُشَلِيمَ. وَاسْمُ أُمِّهِ يَهُوعَدَّانُ مِنْ أُورُشَلِيمَ.
\par 3 وَعَمِلَ مَا هُوَ مُسْتَقِيمٌ فِي عَيْنَيِ الرَّبِّ وَلَكِنْ لَيْسَ كَدَاوُدَ أَبِيهِ. عَمِلَ حَسَبَ كُلِّ مَا عَمِلَ يَهُوآشُ أَبُوهُ.
\par 4 إِلاَّ أَنَّ الْمُرْتَفَعَاتِ لَمْ تُنْتَزَعْ، بَلْ كَانَ الشَّعْبُ لاَ يَزَالُونَ يَذْبَحُونَ وَيُوقِدُونَ عَلَى الْمُرْتَفَعَاتِ.
\par 5 وَلَمَّا تَثَبَّتَتِ الْمَمْلَكَةُ بِيَدِهِ قَتَلَ عَبِيدَهُ الَّذِينَ قَتَلُوا الْمَلِكَ أَبَاهُ.
\par 6 وَلَكِنَّهُ لَمْ يَقْتُلْ أَبْنَاءَ الْقَاتِلِينَ حَسَبَ مَا هُوَ مَكْتُوبٌ فِي سِفْرِ شَرِيعَةِ مُوسَى، حَيْثُ أَمَرَ الرَّبُّ: [لاَ يُقْتَلُ الآبَاءُ مِنْ أَجْلِ الْبَنِينَ، وَالْبَنُونَ لاَ يُقْتَلُونَ مِنْ أَجْلِ الآبَاءِ. إِنَّمَا كُلُّ إِنْسَانٍ يُقْتَلُ بِخَطِيَّتِهِ].
\par 7 هُوَ قَتَلَ مِنْ أَدُومَ فِي وَادِي الْمِلْحِ عَشَرَةَ آلاَفٍ، وَأَخَذَ سَالِعَ بِالْحَرْبِ، وَدَعَا اسْمَهَا يَقْتَئِيلَ إِلَى هَذَا الْيَوْمِ.
\par 8 حِينَئِذٍ أَرْسَلَ أَمَصْيَا رُسُلاً إِلَى يُوآشَ بْنِ يَهُوأَحَازَ بْنِ يَاهُو مَلِكِ إِسْرَائِيلَ قَائِلاً: [هَلُمَّ نَتَرَاءَ مُواجَهَةً].
\par 9 فَأَرْسَلَ يُوآشُ مَلِكُ إِسْرَائِيلَ إِلَى أَمَصْيَا مَلِكِ يَهُوذَا قَائِلاً: [الْعَوْسَجُ الَّذِي فِي لُبْنَانَ أَرْسَلَ إِلَى الأَرْزِ الَّذِي فِي لُبْنَانَ يَقُولُ: أَعْطِ ابْنَتَكَ لاِبْنِي امْرَأَةً. فَعَبَرَ حَيَوَانٌ بَرِّيٌّ كَانَ فِي لُبْنَانَ وَدَاسَ الْعَوْسَجَ.
\par 10 إِنَّكَ قَدْ ضَرَبْتَ أَدُومَ فَرَفَعَكَ قَلْبُكَ. تَمَجَّدْ وَأَقِمْ فِي بَيْتِكَ. وَلِمَاذَا تَهْجِمُ عَلَى الشَّرِّ فَتَسْقُطَ أَنْتَ وَيَهُوذَا مَعَكَ؟]
\par 11 فَلَمْ يَسْمَعْ أَمَصْيَا. فَصَعِدَ يُوآشُ مَلِكُ إِسْرَائِيلَ وَتَرَاءَيَا مُواجَهَةً، هُوَ وَأَمَصْيَا مَلِكُ يَهُوذَا فِي بَيْتِ شَمْسٍ الَّتِي لِيَهُوذَا.
\par 12 فَانْهَزَمَ يَهُوذَا أَمَامَ إِسْرَائِيلَ وَهَرَبُوا كُلُّ وَاحِدٍ إِلَى خَيْمَتِهِ.
\par 13 وَأَمَّا أَمَصْيَا مَلِكُ يَهُوذَا ابْنُ يُوآشَ بْنِ أَخَزْيَا فَأَمْسَكَهُ يُوآشُ مَلِكُ إِسْرَائِيلَ فِي بَيْتِ شَمْسٍ، وَجَاءَ إِلَى أُورُشَلِيمَ وَهَدَمَ سُورَ أُورُشَلِيمَ مِنْ بَابِ أَفْرَايِمَ إِلَى بَابِ الزَّاوِيَةِ، أَرْبَعَ مِئَةِ ذِرَاعٍ.
\par 14 وَأَخَذَ كُلَّ الذَّهَبِ وَالْفِضَّةِ وَجَمِيعَ الآنِيَةِ الْمَوْجُودَةِ فِي بَيْتِ الرَّبِّ وَفِي خَزَائِنِ بَيْتِ الْمَلِكِ وَالرُّهَنَاءَ وَرَجَعَ إِلَى السَّامِرَةِ.
\par 15 وَبَقِيَّةُ أُمُورِ يُوآشَ الَّتِي عَمِلَ وَجَبَرُوتُهُ وَكَيْفَ حَارَبَ أَمَصْيَا مَلِكَ يَهُوذَا مَكْتُوبَةٌ فِي سِفْرِ أَخْبَارِ الأَيَّامِ لِمُلُوكِ إِسْرَائِيلَ.
\par 16 ثُمَّ اضْطَجَعَ يُوآشُ مَعَ آبَائِهِ، وَدُفِنَ فِي السَّامِرَةِ مَعَ مُلُوكِ إِسْرَائِيلَ. وَمَلَكَ يَرُبْعَامُ ابْنُهُ عِوَضاً عَنْهُ.
\par 17 وَعَاشَ أَمَصْيَا بْنُ يَهُوآشَ مَلِكُ يَهُوذَا بَعْدَ وَفَاةِ يُوآشَ بْنِ يَهُوأَحَازَ مَلِكِ إِسْرَائِيلَ خَمْسَ عَشَرَةَ سَنَةً.
\par 18 وَبَقِيَّةُ أُمُورِ أَمَصْيَا مَكْتُوبَةٌ فِي سِفْرِ أَخْبَارِ الأَيَّامِ لِمُلُوكِ يَهُوذَا.
\par 19 وَفَتَنُوا عَلَيْهِ فِتْنَةً فِي أُورُشَلِيمَ، فَهَرَبَ إِلَى لَخِيشَ، فَأَرْسَلُوا وَرَاءَهُ إِلَى لَخِيشَ وَقَتَلُوهُ هُنَاكَ.
\par 20 وَحَمَلُوهُ عَلَى الْخَيْلِ فَدُفِنَ فِي أُورُشَلِيمَ مَعَ آبَائِهِ فِي مَدِينَةِ دَاوُدَ.
\par 21 وَأَخَذَ كُلُّ شَعْبِ يَهُوذَا عَزَرْيَا وَهُوَ ابْنُ سِتَّ عَشَرَةَ سَنَةً وَمَلَّكُوهُ عِوَضاً عَنْ أَبِيهِ أَمَصْيَا.
\par 22 هُوَ بَنَى أَيْلَةَ وَاسْتَرَدَّهَا لِيَهُوذَا بَعْدَ اضْطِجَاعِ الْمَلِكِ مَعَ آبَائِهِ.
\par 23 فِي السَّنَةِ الْخَامِسَةَ عَشَرَةَ لأَمَصْيَا بْنِ يَهُوآشَ مَلِكِ يَهُوذَا، مَلَكَ يَرُبْعَامُ بْنُ يُوآشَ مَلِكِ إِسْرَائِيلَ فِي السَّامِرَةِ إِحْدَى وَأَرْبَعِينَ سَنَةً.
\par 24 وَعَمِلَ الشَّرَّ فِي عَيْنَيِ الرَّبِّ. لَمْ يَحِدْ عَنْ شَيْءٍ مِنْ خَطَايَا يَرُبْعَامَ بْنِ نَبَاطَ الَّذِي جَعَلَ إِسْرَائِيلَ يُخْطِئُ.
\par 25 هُوَ رَدَّ تُخُمَ إِسْرَائِيلَ مِنْ مَدْخَلِ حَمَاةَ إِلَى بَحْرِ الْعَرَبَةِ حَسَبَ كَلاَمِ الرَّبِّ إِلَهِ إِسْرَائِيلَ الَّذِي تَكَلَّمَ بِهِ عَنْ يَدِ عَبْدِهِ يُونَانَ بْنِ أَمِتَّايَ النَّبِيِّ الَّذِي مِنْ جَتَّ حَافَرَ.
\par 26 لأَنَّ الرَّبَّ رَأَى ضِيقَ إِسْرَائِيلَ مُرّاً جِدّاً. لأَنَّهُ لَمْ يَكُنْ مَحْجُوزٌ وَلاَ مُطْلَقٌ وَلَيْسَ مُعِينٌ لإِسْرَائِيلَ.
\par 27 وَلَمْ يَتَكَلَّمِ الرَّبُّ بِمَحْوِ اسْمِ إِسْرَائِيلَ مِنْ تَحْتِ السَّمَاءِ، فَخَلَّصَهُمْ بِيَدِ يَرُبْعَامَ ابْنِ يُوآشَ.
\par 28 وَبَقِيَّةُ أُمُورِ يَرُبْعَامَ وَكُلُّ مَا عَمِلَ، وَجَبَرُوتُهُ كَيْفَ حَارَبَ وَكَيْفَ اسْتَرْجَعَ إِلَى إِسْرَائِيلَ دِمَشْقَ وَحَمَاةَ الَّتِي لِيَهُوذَا مَكْتُوبَةٌ فِي سِفْرِ أَخْبَارِ الأَيَّامِ لِمُلُوكِ إِسْرَائِيلَ.
\par 29 ثُمَّ اضْطَجَعَ يَرُبْعَامُ مَعَ آبَائِهِ مَعَ مُلُوكِ إِسْرَائِيلَ، وَمَلَكَ زَكَرِيَّا ابْنُهُ عِوَضاً عَنْهُ.

\chapter{15}

\par 1 فِي السَّنَةِ السَّابِعَةِ وَالْعِشْرِينَ لِيَرُبْعَامَ مَلِكِ إِسْرَائِيلَ، مَلَكَ عَزَرْيَا بْنُ أَمَصْيَا مَلِكِ يَهُوذَا.
\par 2 كَانَ ابْنَ سِتَّ عَشَرَةَ سَنَةً حِينَ مَلَكَ، وَمَلَكَ اثْنَتَيْنِ وَخَمْسِينَ سَنَةً فِي أُورُشَلِيمَ، وَاسْمُ أُمِّهِ يَكُلْيَا مِنْ أُورُشَلِيمَ.
\par 3 وَعَمِلَ مَا هُوَ مُسْتَقِيمٌ فِي عَيْنَيِ الرَّبِّ حَسَبَ كُلِّ مَا عَمِلَ أَمَصْيَا أَبُوهُ.
\par 4 وَلَكِنِ الْمُرْتَفَعَاتُ لَمْ تُنْتَزَعْ، بَلْ كَانَ الشَّعْبُ لاَ يَزَالُونَ يَذْبَحُونَ وَيُوقِدُونَ عَلَى الْمُرْتَفَعَاتِ.
\par 5 وَضَرَبَ الرَّبُّ الْمَلِكَ فَكَانَ أَبْرَصَ إِلَى يَوْمِ وَفَاتِهِ، وَأَقَامَ فِي بَيْتِ الْمَرَضِ. وَكَانَ يُوثَامُ ابْنُ الْمَلِكِ عَلَى الْبَيْتِ يَحْكُمُ عَلَى شَعْبِ الأَرْضِ.
\par 6 وَبَقِيَّةُ أُمُورِ عَزَرْيَا وَكُلُّ مَا عَمِلَ مَكْتُوبَةٌ فِي سِفْرِ أَخْبَارِ الأَيَّامِ لِمُلُوكِ يَهُوذَا.
\par 7 ثُمَّ اضْطَجَعَ عَزَرْيَا مَعَ آبَائِهِ، فَدَفَنُوهُ مَعَ آبَائِهِ فِي مَدِينَةِ دَاوُدَ، وَمَلَكَ يُوثَامُ ابْنُهُ عِوَضاً عَنْهُ.
\par 8 فِي السَّنَةِ الثَّامِنَةِ وَالثَّلاَثِينَ لِعَزَرْيَا مَلِكِ يَهُوذَا، مَلَكَ زَكَرِيَّا بْنُ يَرُبْعَامَ عَلَى إِسْرَائِيلَ فِي السَّامِرَةِ سِتَّةَ أَشْهُرٍ.
\par 9 وَعَمِلَ الشَّرَّ فِي عَيْنَيِ الرَّبِّ كَمَا عَمِلَ آبَاؤُهُ. لَمْ يَحِدْ عَنْ خَطَايَا يَرُبْعَامَ بْنِ نَبَاطَ الَّذِي جَعَلَ إِسْرَائِيلَ يُخْطِئُ.
\par 10 فَفَتَنَ عَلَيْهِ شَلُّومُ بْنُ يَابِيشَ وَضَرَبَهُ أَمَامَ الشَّعْبِ فَقَتَلَهُ، وَمَلَكَ عِوَضاً عَنْهُ.
\par 11 وَبَقِيَّةُ أُمُورِ زَكَرِيَّا مَكْتُوبَةٌ فِي سِفْرِ أَخْبَارِ الأَيَّامِ لِمُلُوكِ إِسْرَائِيلَ.
\par 12 ذَلِكَ كَلاَمُ الرَّبِّ الَّذِي كَلَّمَ بِهِ يَاهُوَ قَائِلاً: [بَنُو الْجِيلِ الرَّابِعِ يَجْلِسُونَ لَكَ عَلَى كُرْسِيِّ إِسْرَائِيلَ]. وَهَكَذَا كَانَ.
\par 13 شَلُّومُ بْنُ يَابِيشَ مَلَكَ فِي السَّنَةِ التَّاسِعَةِ وَالثَّلاَثِينَ لِعُزِّيَّا مَلِكِ يَهُوذَا، وَمَلَكَ شَهْرَ أَيَّامٍ فِي السَّامِرَةِ.
\par 14 وَصَعِدَ مَنَحِيمُ بْنُ جَادِي مِنْ تِرْصَةَ وَجَاءَ إِلَى السَّامِرَةِ، وَضَرَبَ شَلُّومَ بْنَ يَابِيشَ فِي السَّامِرَةِ فَقَتَلَهُ وَمَلَكَ عِوَضاً عَنْهُ.
\par 15 وَبَقِيَّةُ أُمُورِ شَلُّومَ وَفِتْنَتُهُ الَّتِي فَتَنَهَا مَكْتُوبَةٌ فِي سِفْرِ أَخْبَارِ الأَيَّامِ لِمُلُوكِ إِسْرَائِيلَ.
\par 16 حِينَئِذٍ ضَرَبَ مَنَحِيمُ تَفْصَحَ وَكُلَّ مَا بِهَا وَتُخُومَهَا مِنْ تِرْصَةَ. لأَنَّهُمْ لَمْ يَفْتَحُوا لَهُ. ضَرَبَهَا وَشَقَّ جَمِيعَ حَوَامِلِهَا.
\par 17 فِي السَّنَةِ التَّاسِعَةِ وَالثَّلاَثِينَ لِعَزَرْيَا مَلِكِ يَهُوذَا، مَلَكَ مَنَحِيمُ بْنُ جَادِي عَلَى إِسْرَائِيلَ فِي السَّامِرَةِ عَشَرَ سِنِينَ
\par 18 وَعَمِلَ الشَّرَّ فِي عَيْنَيِ الرَّبِّ. لَمْ يَحِدْ عَنْ خَطَايَا يَرُبْعَامَ بْنِ نَبَاطَ الَّذِي جَعَلَ إِسْرَائِيلَ يُخْطِئُ كُلَّ أَيَّامِهِ.
\par 19 فَجَاءَ فُولُ مَلِكُ أَشُّورَ عَلَى الأَرْضِ، فَأَعْطَى مَنَحِيمُ لِفُولَ أَلْفَ وَزْنَةٍ مِنَ الْفِضَّةِ لِتَكُونَ يَدَاهُ مَعَهُ لِيُثَبِّتَ الْمَمْلَكَةَ فِي يَدِهِ.
\par 20 وَوَضَعَ مَنَحِيمُ الْفِضَّةَ عَلَى إِسْرَائِيلَ عَلَى جَمِيعِ جَبَابِرَةِ الْبَأْسِ لِيَدْفَعَ لِمَلِكِ أَشُّورَ خَمْسِينَ شَاقِلَ فِضَّةٍ عَلَى كُلِّ رَجُلٍ. فَرَجَعَ مَلِكُ أَشُّورَ وَلَمْ يُقِمْ هُنَاكَ فِي الأَرْضِ.
\par 21 وَبَقِيَّةُ أُمُورِ مَنَحِيمَ وَكُلُّ مَا عَمِلَ مَكْتُوبَةٌ فِي سِفْرِ أَخْبَارِ الأَيَّامِ لِمُلُوكِ إِسْرَائِيلَ.
\par 22 ثُمَّ اضْطَجَعَ مَنَحِيمُ مَعَ آبَائِهِ، وَمَلَكَ فَقَحْيَا ابْنُهُ عِوَضاً عَنْهُ.
\par 23 فِي السَّنَةِ الْخَمْسِينَ لِعَزَرْيَا مَلِكِ يَهُوذَا مَلَكَ فَقَحْيَا بْنُ مَنَحِيمَ عَلَى إِسْرَائِيلَ فِي السَّامِرَةِ سَنَتَيْنِ.
\par 24 وَعَمِلَ الشَّرَّ فِي عَيْنَيِ الرَّبِّ. لَمْ يَحِدْ عَنْ خَطَايَا يَرُبْعَامَ بْنِ نَبَاطَ الَّذِي جَعَلَ إِسْرَائِيلَ يُخْطِئُ.
\par 25 فَفَتَنَ عَلَيْهِ فَقْحُ بْنُ رَمَلْيَا ثَالِثُهُ، وَضَرَبَهُ فِي السَّامِرَةِ فِي قَصْرِ بَيْتِ الْمَلِكِ مَعَ أَرْجُوبَ وَمَعَ أَرْيَةَ وَمَعَهُ خَمْسُونَ رَجُلاً مِنْ بَنِي الْجِلْعَادِيِّينَ. قَتَلَهُ وَمَلَكَ عِوَضاً عَنْهُ.
\par 26 وَبَقِيَّةُ أُمُورِ فَقَحْيَا وَكُلُّ مَا عَمِلَ مَكْتُوبَةٌ فِي سِفْرِ أَخْبَارِ الأَيَّامِ لِمُلُوكِ إِسْرَائِيلَ.
\par 27 فِي السَّنَةِ الثَّانِيَةِ وَالْخَمْسِينَ لِعَزَرْيَا مَلِكِ يَهُوذَا، مَلَكَ فَقْحُ بْنُ رَمَلْيَا عَلَى إِسْرَائِيلَ فِي السَّامِرَةِ عِشْرِينَ سَنَةً.
\par 28 وَعَمِلَ الشَّرَّ فِي عَيْنَيِ الرَّبِّ. لَمْ يَحِدْ عَنْ خَطَايَا يَرُبْعَامَ بْنِ نَبَاطَ الَّذِي جَعَلَ إِسْرَائِيلَ يُخْطِئُ.
\par 29 فِي أَيَّامِ فَقْحٍ مَلِكِ إِسْرَائِيلَ، جَاءَ تَغْلَثَ فَلاَسِرُ مَلِكُ أَشُّورَ وَأَخَذَ عُيُونَ وَآبَلَ بَيْتِ مَعْكَةَ وَيَانُوحَ وَقَادِشَ وَحَاصُورَ وَجِلْعَادَ وَالْجَلِيلَ وَكُلَّ أَرْضِ نَفْتَالِي، وَسَبَاهُمْ إِلَى أَشُّورَ.
\par 30 وَفَتَنَ هُوشَعُ بْنُ أَيْلَةَ عَلَى فَقْحَ بْنِ رَمَلْيَا وَضَرَبَهُ فَقَتَلَهُ، وَمَلَكَ عِوَضاً عَنْهُ فِي السَّنَةِ الْعِشْرِينَ لِيُوثَامَ بْنِ عُزِّيَّا.
\par 31 وَبَقِيَّةُ أُمُورِ فَقْحٍ وَكُلُّ مَا عَمِلَ مَكْتُوبَةٌ فِي سِفْرِ أَخْبَارِ الأَيَّامِ لِمُلُوكِ إِسْرَائِيلَ.
\par 32 فِي السَّنَةِ الثَّانِيَةِ لِفَقْحَ بْنِ رَمَلْيَا مَلِكِ إِسْرَائِيلَ مَلَكَ يُوثَامُ بْنُ عُزِّيَّا مَلِكِ يَهُوذَا.
\par 33 كَانَ ابْنَ خَمْسٍ وَعِشْرِينَ سَنَةً حِينَ مَلَكَ، وَمَلَكَ سِتَّ عَشَرَةَ سَنَةً فِي أُورُشَلِيمَ. وَاسْمُ أُمِّهِ يَرُوشَا ابْنَةُ صَادُوقَ.
\par 34 وَعَمِلَ مَا هُوَ مُسْتَقِيمٌ فِي عَيْنَيِ الرَّبِّ. عَمِلَ حَسَبَ كُلِّ مَا عَمِلَ عُزِّيَّا أَبُوهُ.
\par 35 إِلاَّ أَنَّ الْمُرْتَفَعَاتِ لَمْ تُنْتَزَعْ، بَلْ كَانَ الشَّعْبُ لاَ يَزَالُونَ يَذْبَحُونَ وَيُوقِدُونَ عَلَى الْمُرْتَفَعَاتِ. هُوَ بَنَى الْبَابَ الأَعْلَى لِبَيْتِ الرَّبِّ.
\par 36 وَبَقِيَّةُ أُمُورِ يُوثَامَ وَكُلُّ مَا عَمِلَ مَكْتُوبَةٌ فِي سِفْرِ أَخْبَارِ الأَيَّامِ لِمُلُوكِ يَهُوذَا.
\par 37 فِي تِلْكَ الأَيَّامِ ابْتَدَأَ الرَّبُّ يُرْسِلُ عَلَى يَهُوذَا رَصِينَ مَلِكَ أَرَامَ وَفَقْحَ بْنَ رَمَلْيَا.
\par 38 وَاضْطَجَعَ يُوثَامُ مَعَ آبَائِهِ وَدُفِنَ مَعَ آبَائِهِ فِي مَدِينَةِ دَاوُدَ أَبِيهِ، وَمَلَكَ آحَازُ ابْنُهُ عِوَضاً عَنْهُ.

\chapter{16}

\par 1 فِي السَّنَةِ السَّابِعَةَ عَشْرَةَ لِفَقْحَ بْنِ رَمَلْيَا، مَلَكَ آحَازُ بْنُ يُوثَامَ مَلِكِ يَهُوذَا.
\par 2 كَانَ آحَازُ ابْنَ عِشْرِينَ سَنَةً حِينَ مَلَكَ، وَمَلَكَ سِتَّ عَشَرَةَ سَنَةً فِي أُورُشَلِيمَ. وَلَمْ يَعْمَلِ الْمُسْتَقِيمَ فِي عَيْنَيِ الرَّبِّ إِلَهِهِ كَدَاوُدَ أَبِيهِ،
\par 3 بَلْ سَارَ فِي طَرِيقِ مُلُوكِ إِسْرَائِيلَ حَتَّى إِنَّهُ عَبَّرَ ابْنَهُ فِي النَّارِ حَسَبَ أَرْجَاسِ الأُمَمِ الَّذِينَ طَرَدَهُمُ الرَّبُّ مِنْ أَمَامِ بَنِي إِسْرَائِيلَ.
\par 4 وَذَبَحَ وَأَوْقَدَ عَلَى الْمُرْتَفَعَاتِ وَعَلَى التِّلاَلِ وَتَحْتَ كُلِّ شَجَرَةٍ خَضْرَاءَ.
\par 5 حِينَئِذٍ صَعِدَ رَصِينُ مَلِكُ أَرَامَ وَفَقْحُ بْنُ رَمَلْيَا مَلِكُ إِسْرَائِيلَ إِلَى أُورُشَلِيمَ لِلْمُحَارَبَةِ، فَحَاصَرُوا آحَازَ وَلَمْ يَقْدِرُوا أَنْ يَغْلِبُوهُ.
\par 6 فِي ذَلِكَ الْوَقْتِ أَرْجَعَ رَصِينُ مَلِكُ أَرَامَ أَيْلَةَ لِلأَرَامِيِّينَ، وَطَرَدَ الْيَهُودَ مِنْ أَيْلَةَ. وَجَاءَ الأَرَامِيُّونَ إِلَى أَيْلَةَ وَأَقَامُوا هُنَاكَ إِلَى هَذَا الْيَوْمِ.
\par 7 وَأَرْسَلَ آحَازُ رُسُلاً إِلَى تَغْلَثَ فَلاَسِرَ مَلِكِ أَشُّورَ قَائِلاً: [أَنَا عَبْدُكَ وَابْنُكَ. اصْعَدْ وَخَلِّصْنِي مِنْ يَدِ مَلِكِ أَرَامَ وَمِنْ يَدِ مَلِكِ إِسْرَائِيلَ الْقَائِمَيْنِ عَلَيَّ].
\par 8 فَأَخَذَ آحَازُ الْفِضَّةَ وَالذَّهَبَ الْمَوْجُودَةَ فِي بَيْتِ الرَّبِّ وَفِي خَزَائِنِ بَيْتِ الْمَلِكِ وَأَرْسَلَهَا إِلَى مَلِكِ أَشُّورَ هَدِيَّةً.
\par 9 فَسَمِعَ لَهُ مَلِكُ أَشُّورَ، وَصَعِدَ مَلِكُ أَشُّورَ إِلَى دِمَشْقَ وَأَخَذَهَا وَسَبَاهَا إِلَى قِيرَ، وَقَتَلَ رَصِينَ.
\par 10 وَسَارَ الْمَلِكُ آحَازُ لِلِقَاءِ تَغْلَثَ فَلاَسِرَ مَلِكِ أَشُّورَ إِلَى دِمَشْقَ، وَرَأَى الْمَذْبَحَ الَّذِي فِي دِمَشْقَ. وَأَرْسَلَ الْمَلِكُ آحَازُ إِلَى أُورِيَّا الْكَاهِنِ شِبْهَ الْمَذْبَحِ وَشَكْلَهُ حَسَبَ كُلِّ صِنَاعَتِهِ.
\par 11 فَبَنَى أُورِيَّا الْكَاهِنُ مَذْبَحاً. حَسَبَ كُلِّ مَا أَرْسَلَ الْمَلِكُ آحَازُ مِنْ دِمَشْقَ كَذَلِكَ عَمِلَ أُورِيَّا الْكَاهِنُ رَيْثَمَا جَاءَ الْمَلِكُ آحَازُ مِنْ دِمَشْقَ.
\par 12 فَلَمَّا قَدِمَ الْمَلِكُ مِنْ دِمَشْقَ رَأَى الْمَلِكُ الْمَذْبَحَ، فَتَقَدَّمَ الْمَلِكُ إِلَى الْمَذْبَحِ وَأَصْعَدَ عَلَيْهِ،
\par 13 وَأَوْقَدَ مُحْرَقَتَهُ وَتَقْدِمَتَهُ وَسَكَبَ سَكِيبَهُ، وَرَشَّ دَمَ ذَبِيحَةِ السَّلاَمَةِ الَّتِي لَهُ عَلَى الْمَذْبَحِ.
\par 14 وَمَذْبَحُ النُّحَاسِ الَّذِي أَمَامَ الرَّبِّ قَدَّمَهُ مِنْ أَمَامِ الْبَيْتِ مِنْ بَيْنِ الْمَذْبَحِ وَبَيْتِ الرَّبِّ، وَجَعَلَهُ عَلَى جَانِبِ الْمَذْبَحِ الشِّمَالِيِّ.
\par 15 وَأَمَرَ الْمَلِكُ آحَازُ أُورِيَّا الْكَاهِنَ: [عَلَى الْمَذْبَحِ الْعَظِيمِ أَوْقِدْ مُحْرَقَةَ الصَّبَاحِ وَتَقْدِمَةَ الْمَسَاءِ وَمُحْرَقَةَ الْمَلِكِ وَتَقْدِمَتَهُ، مَعَ مُحْرَقَةِ كُلِّ شَعْبِ الأَرْضِ وَتَقْدِمَتِهِمْ وَسَكَائِبِهِمْ، وَرُشَّ عَلَيْهِ كُلَّ دَمِ مُحْرَقَةٍ وَكُلَّ دَمِ ذَبِيحَةٍ. وَمَذْبَحُ النُّحَاسِ يَكُونُ لِي لِلسُّؤَالِ].
\par 16 فَعَمِلَ أُورِيَّا الْكَاهِنُ حَسَبَ كُلِّ مَا أَمَرَ بِهِ الْمَلِكُ آحَازُ.
\par 17 وَقَطَعَ الْمَلِكُ آحَازُ أَتْرَاسَ الْقَوَاعِدِ وَرَفَعَ عَنْهَا الْمِرْحَضَةَ، وَأَنْزَلَ الْبَحْرَ عَنْ ثِيرَانِ النُّحَاسِ الَّتِي تَحْتَهُ وَجَعَلَهُ عَلَى رَصِيفٍ مِنْ حِجَارَةٍ.
\par 18 وَرِوَاقَ السَّبْتِ الَّذِي بَنُوهُ فِي الْبَيْتِ، وَمَدْخَلَ الْمَلِكِ مِنْ خَارِجٍ غَيَّرَهُ فِي بَيْتِ الرَّبِّ مِنْ أَجْلِ مَلِكِ أَشُّورَ.
\par 19 وَبَقِيَّةُ أُمُورِ آحَازَ الَّتِي عَمِلَ مَكْتُوبَةٌ فِي سِفْرِ أَخْبَارِ الأَيَّامِ لِمُلُوكِ يَهُوذَا.
\par 20 ثُمَّ اضْطَجَعَ آحَازُ مَعَ آبَائِهِ، وَدُفِنَ مَعَ آبَائِهِ فِي مَدِينَةِ دَاوُدَ، وَمَلَكَ حَزَقِيَّا ابْنُهُ عِوَضاً عَنْهُ.

\chapter{17}

\par 1 فِي السَّنَةِ الثَّانِيَةَ عَشَرَةَ لآحَازَ مَلِكِ يَهُوذَا، مَلَكَ هُوشَعُ بْنُ أَيْلَةَ فِي السَّامِرَةِ عَلَى إِسْرَائِيلَ تِسْعَ سِنِينَ.
\par 2 وَعَمِلَ الشَّرَّ فِي عَيْنَيِ الرَّبِّ، وَلَكِنْ لَيْسَ كَمُلُوكِ إِسْرَائِيلَ الَّذِينَ كَانُوا قَبْلَهُ.
\par 3 وَصَعِدَ عَلَيْهِ شَلْمَنْأَسَرُ مَلِكُ أَشُّورَ فَصَارَ لَهُ هُوشَعُ عَبْداً وَدَفَعَ لَهُ جِزْيَةً.
\par 4 وَوَجَدَ مَلِكُ أَشُّورَ فِي هُوشَعَ خِيَانَةً، لأَنَّهُ أَرْسَلَ رُسُلاً إِلَى سَوَا مَلِكِ مِصْرَ وَلَمْ يُؤَدِّ جِزْيَةً إِلَى مَلِكِ أَشُّورَ حَسَبَ كُلِّ سَنَةٍ، فَقَبَضَ عَلَيْهِ مَلِكُ أَشُّورَ وَأَوْثَقَهُ فِي السِّجْنِ.
\par 5 وَصَعِدَ مَلِكُ أَشُّورَ عَلَى كُلِّ الأَرْضِ، وَصَعِدَ إِلَى السَّامِرَةِ وَحَاصَرَهَا ثَلاَثَ سِنِينَ.
\par 6 فِي السَّنَةِ التَّاسِعَةِ لِهُوشَعَ أَخَذَ مَلِكُ أَشُّورَ السَّامِرَةَ، وَسَبَى إِسْرَائِيلَ إِلَى أَشُّورَ وَأَسْكَنَهُمْ فِي حَلَحَ وَخَابُورَ نَهْرِ جُوزَانَ وَفِي مُدُنِ مَادِي.
\par 7 وَكَانَ أَنَّ بَنِي إِسْرَائِيلَ أَخْطَأُوا إِلَى الرَّبِّ إِلَهِهِمِ الَّذِي أَصْعَدَهُمْ مِنْ أَرْضِ مِصْرَ مِنْ تَحْتِ يَدِ فِرْعَوْنَ مَلِكِ مِصْرَ وَاتَّقُوا آلِهَةً أُخْرَى،
\par 8 وَسَلَكُوا حَسَبَ فَرَائِضِ الأُمَمِ الَّذِينَ طَرَدَهُمُ الرَّبُّ مِنْ أَمَامِ بَنِي إِسْرَائِيلَ وَمُلُوكِ إِسْرَائِيلَ الَّذِينَ أَقَامُوهُمْ.
\par 9 وَعَمِلَ بَنُو إِسْرَائِيلَ سِرّاً ضِدَّ الرَّبِّ إِلَهِهِمْ أُمُوراً لَيْسَتْ بِمُسْتَقِيمَةٍ، وَبَنُوا لأَنْفُسِهِمْ مُرْتَفَعَاتٍ فِي جَمِيعِ مُدُنِهِمْ مِنْ بُرْجِ النَّوَاطِيرِ إِلَى الْمَدِينَةِ الْمُحَصَّنَةِ.
\par 10 وَأَقَامُوا لأَنْفُسِهِمْ أَنْصَاباً وَسَوَارِيَ عَلَى كُلِّ تَلٍّ عَالٍ وَتَحْتَ كُلِّ شَجَرَةٍ خَضْرَاءَ.
\par 11 وَأَوْقَدُوا هُنَاكَ عَلَى جَمِيعِ الْمُرْتَفَعَاتِ مِثْلَ الأُمَمِ الَّذِينَ سَاقَهُمُ الرَّبُّ مِنْ أَمَامِهِمْ، وَعَمِلُوا أُمُوراً قَبِيحَةً لإِغَاظَةِ الرَّبِّ.
\par 12 وَعَبَدُوا الأَصْنَامَ الَّتِي قَالَ الرَّبُّ لَهُمْ عَنْهَا: [لاَ تَعْمَلُوا هَذَا الأَمْرَ].
\par 13 وَأَشْهَدَ الرَّبُّ عَلَى إِسْرَائِيلَ وَعَلَى يَهُوذَا عَنْ يَدِ جَمِيعِ الأَنْبِيَاءِ وَكُلِّ رَاءٍ قَائِلاً: [ارْجِعُوا عَنْ طُرُقِكُمُ الرَّدِيئَةِ وَاحْفَظُوا وَصَايَايَ فَرَائِضِي حَسَبَ كُلِّ الشَّرِيعَةِ الَّتِي أَوْصَيْتُ بِهَا آبَاءَكُمْ، وَالَّتِي أَرْسَلْتُهَا إِلَيْكُمْ عَنْ يَدِ عَبِيدِي الأَنْبِيَاءِ].
\par 14 فَلَمْ يَسْمَعُوا بَلْ صَلَّبُوا أَقْفِيَتَهُمْ كَأَقْفِيَةِ آبَائِهِمِ الَّذِينَ لَمْ يُؤْمِنُوا بِالرَّبِّ إِلَهِهِمْ.
\par 15 وَرَفَضُوا فَرَائِضَهُ وَعَهْدَهُ الَّذِي قَطَعَهُ مَعَ آبَائِهِمْ وَشَهَادَاتِهِ الَّتِي شَهِدَ بِهَا عَلَيْهِمْ، وَسَارُوا وَرَاءَ الْبَاطِلِ، وَصَارُوا بَاطِلاً وَرَاءَ الأُمَمِ الَّذِينَ حَوْلَهُمُ، الَّذِينَ أَمَرَهُمُ الرَّبُّ أَنْ لاَ يَعْمَلُوا مِثْلَهُمْ.
\par 16 وَتَرَكُوا جَمِيعَ وَصَايَا الرَّبِّ إِلَهِهِمْ وَعَمِلُوا لأَنْفُسِهِمْ مَسْبُوكَاتٍ عِجْلَيْنِ، وَعَمِلُوا سَوَارِيَ وَسَجَدُوا لِجَمِيعِ جُنْدِ السَّمَاءِ، وَعَبَدُوا الْبَعْلَ.
\par 17 وَعَبَّرُوا بَنِيهِمْ وَبَنَاتِهِمْ فِي النَّارِ، وَعَرَفُوا عِرَافَةً وَتَفَاءَلُوا، وَبَاعُوا أَنْفُسَهُمْ لِعَمَلِ الشَّرِّ فِي عَيْنَيِ الرَّبِّ لإِغَاظَتِهِ.
\par 18 فَغَضِبَ الرَّبُّ جِدّاً عَلَى إِسْرَائِيلَ وَنَحَّاهُمْ مِنْ أَمَامِهِ، وَلَمْ يَبْقَ إِلاَّ سِبْطُ يَهُوذَا وَحْدَهُ.
\par 19 وَيَهُوذَا أَيْضاً لَمْ يَحْفَظُوا وَصَايَا الرَّبِّ إِلَهِهِمْ بَلْ سَلَكُوا فِي فَرَائِضِ إِسْرَائِيلَ الَّتِي عَمِلُوهَا.
\par 20 فَرَذَلَ الرَّبُّ كُلَّ نَسْلِ إِسْرَائِيلَ، وَأَذَلَّهُمْ وَدَفَعَهُمْ لِيَدِ نَاهِبِينَ حَتَّى طَرَحَهُمْ مِنْ أَمَامِهِ،
\par 21 لأَنَّهُ شَقَّ إِسْرَائِيلَ عَنْ بَيْتِ دَاوُدَ، فَمَلَّكُوا يَرُبْعَامَ بْنَ نَبَاطَ، فَأَبْعَدَ يَرُبْعَامُ إِسْرَائِيلَ مِنْ وَرَاءِ الرَّبِّ وَجَعَلَهُمْ يُخْطِئُونَ خَطِيَّةً عَظِيمَةً.
\par 22 وَسَلَكَ بَنُو إِسْرَائِيلَ فِي جَمِيعِ خَطَايَا يَرُبْعَامَ الَّتِي عَمِلَ. لَمْ يَحِيدُوا عَنْهَا
\par 23 حَتَّى نَحَّى الرَّبُّ إِسْرَائِيلَ مِنْ أَمَامِهِ كَمَا تَكَلَّمَ عَنْ يَدِ جَمِيعِ عَبِيدِهِ الأَنْبِيَاءِ، فَسُبِيَ إِسْرَائِيلُ مِنْ أَرْضِهِ إِلَى أَشُّورَ إِلَى هَذَا الْيَوْمِ.
\par 24 وَأَتَى مَلِكُ أَشُّورَ بِقَوْمٍ مِنْ بَابِلَ وَكُوثَ وَعَوَّا وَحَمَاةَ وَسَفَرْوَايِمَ وَأَسْكَنَهُمْ فِي مُدُنِ السَّامِرَةِ عِوَضاً عَنْ بَنِي إِسْرَائِيلَ، فَامْتَلَكُوا السَّامِرَةَ وَسَكَنُوا فِي مُدُنِهَا.
\par 25 وَكَانَ فِي ابْتِدَاءِ سَكَنِهِمْ هُنَاكَ أَنَّهُمْ لَمْ يَتَّقُوا الرَّبَّ، فَأَرْسَلَ الرَّبُّ عَلَيْهِمِ السِّبَاعَ فَكَانَتْ تَقْتُلُ مِنْهُمْ.
\par 26 فَقَالُوا لِمَلِكِ أَشُّورَ: [إِنَّ الأُمَمَ الَّذِينَ سَبَيْتَهُمْ وَأَسْكَنْتَهُمْ فِي مُدُنِ السَّامِرَةِ لاَ يَعْرِفُونَ قَضَاءَ إِلَهِ الأَرْضِ، فَأَرْسَلَ عَلَيْهِمِ السِّبَاعَ فَهِيَ تَقْتُلُهُمْ لأَنَّهُمْ لاَ يَعْرِفُونَ قَضَاءَ إِلَهِ الأَرْضِ].
\par 27 فَأَمَرَ مَلِكُ أَشُّورَ: [ابْعَثُوا إِلَى هُنَاكَ وَاحِداً مِنَ الْكَهَنَةِ الَّذِينَ سَبَيْتُمُوهُمْ مِنْ هُنَاكَ فَيَذْهَبَ وَيَسْكُنَ هُنَاكَ وَيُعَلِّمَهُمْ قَضَاءَ إِلَهِ الأَرْضِ].
\par 28 فَأَتَى وَاحِدٌ مِنَ الْكَهَنَةِ الَّذِينَ سَبُوهُمْ مِنَ السَّامِرَةِ وَسَكَنَ فِي بَيْتِ إِيلَ وَعَلَّمَهُمْ كَيْفَ يَتَّقُونَ الرَّبَّ.
\par 29 فَكَانَتْ كُلُّ أُمَّةٍ تَعْمَلُ آلِهَتَهَا وَوَضَعُوهَا فِي بُيُوتِ الْمُرْتَفَعَاتِ الَّتِي عَمِلَهَا السَّامِرِيُّونَ، كُلُّ أُمَّةٍ فِي مُدُنِهَا الَّتِي سَكَنَتْ فِيهَا.
\par 30 فَعَمِلَ أَهْلُ بَابِلَ سُكُّوتَ بَنُوثَ، وَأَهْلُ كُوثَ عَمِلُوا نَرْجَلَ، وَأَهْلُ حَمَاةَ عَمِلُوا أَشِيمَا،
\par 31 وَالْعُوِّيُّونَ عَمِلُوا نِبْحَزَ وَتَرْتَاقَ، وَالسَّفَرْوَايِمِيُّونَ كَانُوا يُحْرِقُونَ بَنِيهِمْ بِالنَّارِ لأَدْرَمَّلَكَ وَعَنَمَّلَكَ إِلَهَيْ سَفَرْوَايِمَ.
\par 32 فَكَانُوا يَتَّقُونَ الرَّبَّ وَيَعْمَلُونَ لأَنْفُسِهِمْ مِنْ بَيْنِهِمْ كَهَنَةَ مُرْتَفَعَاتٍ يُقَرِّبُونَ لأَجْلِهِمْ فِي بُيُوتِ الْمُرْتَفَعَاتِ.
\par 33 كَانُوا يَتَّقُونَ الرَّبَّ وَيَعْبُدُونَ آلِهتَهُمْ كَعَادَةِ الأُمَمِ الَّذِينَ سَبُوهُمْ مِنْ بَيْنِهِمْ
\par 34 إِلَى هَذَا الْيَوْمِ يَعْمَلُونَ كَعَادَاتِهِمِ الأُوَلِ. لاَ يَتَّقُونَ الرَّبَّ وَلاَ يَعْمَلُونَ حَسَبَ فَرَائِضِهِمْ وَعَوَائِدِهِمْ وَلاَ حَسَبَ الشَّرِيعَةِ وَالْوَصِيَّةِ الَّتِي أَمَرَ بِهَا الرَّبُّ بَنِي يَعْقُوبَ (الَّذِي جَعَلَ اسْمَهُ إِسْرَائِيلَ).
\par 35 وَقَطَعَ الرَّبُّ مَعَهُمْ عَهْداً وَأَمَرَهُمْ: [لاَ تَتَّقُوا آلِهَةً أُخْرَى وَلاَ تَسْجُدُوا لَهَا وَلاَ تَعْبُدُوهَا وَلاَ تَذْبَحُوا لَهَا.
\par 36 بَلْ إِنَّمَا اتَّقُوا الرَّبَّ الَّذِي أَصْعَدَكُمْ مِنْ أَرْضِ مِصْرَ بِقُوَّةٍ عَظِيمَةٍ وَذِرَاعٍ مَمْدُودَةٍ، وَلَهُ اسْجُدُوا وَلَهُ اذْبَحُوا.
\par 37 وَاحْفَظُوا الْفَرَائِضَ وَالأَحْكَامَ وَالشَّرِيعَةَ وَالْوَصِيَّةَ الَّتِي كَتَبَهَا لَكُمْ لِتَعْمَلُوا بِهَا كُلَّ الأَيَّامِ، وَلاَ تَتَّقُوا آلِهَةً أُخْرَى.
\par 38 وَلاَ تَنْسُوا الْعَهْدَ الَّذِي قَطَعْتُهُ مَعَكُمْ وَلاَ تَتَّقُوا آلِهَةً أُخْرَى.
\par 39 بَلْ إِنَّمَا اتَّقُوا الرَّبَّ إِلَهَكُمْ وَهُوَ يُنْقِذُكُمْ مِنْ أَيْدِي جَمِيعِ أَعْدَائِكُمْ].
\par 40 فَلَمْ يَسْمَعُوا بَلْ عَمِلُوا حَسَبَ عَادَتِهِمِ الأُولَى.
\par 41 فَكَانَ هَؤُلاَءِ الأُمَمُ يَتَّقُونَ الرَّبَّ وَيَعْبُدُونَ تَمَاثِيلَهُمْ، وَأَيْضاً بَنُوهُمْ وَبَنُو بَنِيهِمْ. فَكَمَا عَمِلَ آبَاؤُهُمْ هَكَذَا هُمْ عَامِلُونَ إِلَى هَذَا الْيَوْمِ.

\chapter{18}

\par 1 وَفِي السَّنَةِ الثَّالِثَةِ لِهُوشَعَ بْنِ أَيْلَةَ مَلِكِ إِسْرَائِيلَ مَلَكَ حَزَقِيَّا بْنُ آحَازَ مَلِكِ يَهُوذَا.
\par 2 كَانَ ابْنَ خَمْسٍ وَعِشْرِينَ سَنَةً حِينَ مَلَكَ، وَمَلَكَ تِسْعاً وَعِشْرِينَ سَنَةً فِي أُورُشَلِيمَ. وَاسْمُ أُمِّهِ أَبِي ابْنَةُ زَكَرِيَّا.
\par 3 وَعَمِلَ الْمُسْتَقِيمَ فِي عَيْنَيِ الرَّبِّ حَسَبَ كُلِّ مَا عَمِلَ دَاوُدُ أَبُوهُ.
\par 4 هُوَ أَزَالَ الْمُرْتَفَعَاتِ، وَكَسَّرَ التَّمَاثِيلَ، وَقَطَّعَ السَّوَارِيَ، وَسَحَقَ حَيَّةَ النُّحَاسِ الَّتِي عَمِلَهَا مُوسَى لأَنَّ بَنِي إِسْرَائِيلَ كَانُوا إِلَى تِلْكَ الأَيَّامِ يُوقِدُونَ لَهَا وَدَعُوهَا [نَحُشْتَانَ].
\par 5 عَلَى الرَّبِّ إِلَهِ إِسْرَائِيلَ اتَّكَلَ، وَبَعْدَهُ لَمْ يَكُنْ مِثْلُهُ فِي جَمِيعِ مُلُوكِ يَهُوذَا وَلاَ فِي الَّذِينَ كَانُوا قَبْلَهُ.
\par 6 وَالْتَصَقَ بِالرَّبِّ وَلَمْ يَحِدْ عَنْهُ بَلْ حَفِظَ وَصَايَاهُ الَّتِي أَمَرَ بِهَا الرَّبُّ مُوسَى.
\par 7 وَكَانَ الرَّبُّ مَعَهُ، وَحَيْثُمَا كَانَ يَخْرُجُ كَانَ يَنْجَحُ. وَعَصَى عَلَى مَلِكِ أَشُّورَ وَلَمْ يَخْضَعْ لَهُ.
\par 8 هُوَ ضَرَبَ الْفِلِسْطِينِيِّينَ إِلَى غَزَّةَ وَتُخُومِهَا مِنْ بُرْجِ النَّوَاطِيرِ إِلَى الْمَدِينَةِ الْمُحَصَّنَةِ.
\par 9 وَفِي السَّنَةِ الرَّابِعَةِ لِلْمَلِكِ حَزَقِيَّا، وَهِيَ السَّنَةُ السَّابِعَةُ لِهُوشَعَ بْنِ أَيْلَةَ مَلِكِ إِسْرَائِيلَ، صَعِدَ شَلْمَنْأَسَرُ مَلِكُ أَشُّورَ عَلَى السَّامِرَةِ وَحَاصَرَهَا.
\par 10 وَأَخَذُوهَا فِي نِهَايَةِ ثَلاَثِ سِنِينَ. فَفِي السَّنَةِ السَّادِسَةِ لِحَزَقِيَّا، وَهِيَ السَّنَةُ التَّاسِعَةُ لِهُوشَعَ مَلِكِ إِسْرَائِيلَ، أُخِذَتِ السَّامِرَةُ.
\par 11 وَسَبَى مَلِكُ أَشُّورَ إِسْرَائِيلَ إِلَى أَشُّورَ، وَوَضَعَهُمْ فِي حَلَحَ وَخَابُورَ نَهْرِ جُوزَانَ وَفِي مُدُنِ مَادِي
\par 12 لأَنَّهُمْ لَمْ يَسْمَعُوا لِصَوْتِ الرَّبِّ إِلَهِهِمْ، بَلْ تَجَاوَزُوا عَهْدَهُ وَكُلَّ مَا أَمَرَ بِهِ مُوسَى عَبْدُ الرَّبِّ، فَلَمْ يَسْمَعُوا وَلَمْ يَعْمَلُوا.
\par 13 وَفِي السَّنَةِ الرَّابِعَةَ عَشَرَةَ لِلْمَلِكِ حَزَقِيَّا صَعِدَ سَنْحَارِيبُ مَلِكُ أَشُّورَ عَلَى جَمِيعِ مُدُنِ يَهُوذَا الْحَصِينَةِ وَأَخَذَهَا.
\par 14 وَأَرْسَلَ حَزَقِيَّا مَلِكُ يَهُوذَا إِلَى مَلِكِ أَشُّورَ إِلَى لَخِيشَ يَقُولُ: [قَدْ أَخْطَأْتُ. ارْجِعْ عَنِّي، وَمَهْمَا جَعَلْتَ عَلَيَّ حَمَلْتُهُ]. فَوَضَعَ مَلِكُ أَشُّورَ عَلَى حَزَقِيَّا مَلِكِ يَهُوذَا ثَلاَثَ مِئَةِ وَزْنَةٍ مِنَ الْفِضَّةِ وَثَلاَثِينَ وَزْنَةً مِنَ الذَّهَبِ.
\par 15 فَدَفَعَ حَزَقِيَّا جَمِيعَ الْفِضَّةِ الْمَوْجُودَةِ فِي بَيْتِ الرَّبِّ وَفِي خَزَائِنِ بَيْتِ الْمَلِكِ.
\par 16 فِي ذَلِكَ الزَّمَانِ قَشَّرَ حَزَقِيَّا الذَّهَبَ عَنْ أَبْوَابِ هَيْكَلِ الرَّبِّ وَالدَّعَائِمِ الَّتِي كَانَ قَدْ غَشَّاهَا حَزَقِيَّا مَلِكُ يَهُوذَا، وَدَفَعَهُ لِمَلِكِ أَشُّورَ.
\par 17 وَأَرْسَلَ مَلِكُ أَشُّورَ تَرْتَانَ وَرَبْسَارِيسَ وَرَبْشَاقَى مِنْ لَخِيشَ إِلَى الْمَلِكِ حَزَقِيَّا بِجَيْشٍ عَظِيمٍ إِلَى أُورُشَلِيمَ، فَصَعِدُوا وَأَتُوا إِلَى أُورُشَلِيمَ. وَلَمَّا صَعِدُوا جَاءُوا وَوَقَفُوا عِنْدَ قَنَاةِ الْبِرْكَةِ الْعُلْيَا الَّتِي فِي طَرِيقِ حَقْلِ الْقَصَّارِ.
\par 18 وَدَعُوا الْمَلِكَ، فَخَرَجَ إِلَيْهِمْ أَلِيَاقِيمُ بْنُ حِلْقِيَّا الَّذِي عَلَى الْبَيْتِ وَشِبْنَةُ الْكَاتِبُ وَيُواخُ بْنُ آسَافَ الْمُسَجِّلُ.
\par 19 فَقَالَ لَهُمْ رَبْشَاقَى: [قُولُوا لِحَزَقِيَّا: هَكَذَا يَقُولُ الْمَلِكُ الْعَظِيمُ مَلِكُ أَشُّورَ. مَا الاِتِّكَالُ الَّذِي اتَّكَلْتَ؟
\par 20 قُلْتَ إِنَّمَا كَلاَمُ الشَّفَتَيْنِ هُوَ مَشُورَةٌ وَبَأْسٌ لِلْحَرْبِ. وَالآنَ عَلَى مَنِ اتَّكَلْتَ حَتَّى عَصَيْتَ عَلَيَّ؟
\par 21 فَالآنَ هُوَذَا قَدِ اتَّكَلْتَ عَلَى عُكَّازِ هَذِهِ الْقَصَبَةِ الْمَرْضُوضَةِ، عَلَى مِصْرَ، الَّتِي إِذَا تَوَكَّأَ أَحَدٌ عَلَيْهَا دَخَلَتْ فِي كَفِّهِ وَثَقَبَتْهَا! هَكَذَا هُوَ فِرْعَوْنُ مَلِكُ مِصْرَ لِجَمِيعِ الْمُتَّكِلِينَ عَلَيْهِ.
\par 22 وَإِذَا قُلْتُمْ لِي: عَلَى الرَّبِّ إِلَهِنَا اتَّكَلْنَا، أَفَلَيْسَ هُوَ الَّذِي أَزَالَ حَزَقِيَّا مُرْتَفَعَاتِهِ وَمَذَابِحَهُ، وَقَالَ لِيَهُوذَا وَلِأُورُشَلِيمَ: أَمَامَ هَذَا الْمَذْبَحِ تَسْجُدُونَ فِي أُورُشَلِيمَ؟
\par 23 وَالآنَ رَاهِنْ سَيِّدِي مَلِكَ أَشُّورَ فَأُعْطِيَكَ أَلْفَيْ فَرَسٍ إِنْ كُنْتَ تَقْدِرُ أَنْ تَجْعَلَ عَلَيْهَا رَاكِبِينَ.
\par 24 فَكَيْفَ تَرُدُّ وَجْهَ وَالٍ وَاحِدٍ مِنْ عَبِيدِ سَيِّدِي الصِّغَارِ وَتَتَّكِلُ عَلَى مِصْرَ لأَجْلِ مَرْكَبَاتٍ وَفُرْسَانٍ؟
\par 25 وَالآنَ هَلْ بِدُونِ الرَّبِّ صَعِدْتُ عَلَى هَذَا الْمَوْضِعِ لأَخْرِبَهُ؟ الرَّبُّ قَالَ لِي اصْعَدْ عَلَى هَذِهِ الأَرْضِ وَاخْرِبْهَا].
\par 26 فَقَالَ أَلِيَاقِيمُ بْنُ حِلْقِيَّا وَشِبْنَةُ وَيُواخُ لِرَبْشَاقَى: [كَلِّمْ عَبِيدَكَ بِالأَرَامِيِّ لأَنَّنَا نَفْهَمُهُ وَلاَ تُكَلِّمْنَا بِالْيَهُودِيِّ فِي مَسَامِعِ الشَّعْبِ الَّذِينَ عَلَى السُّورِ].
\par 27 فَقَالَ لَهُمْ رَبْشَاقَى: [هَلْ إِلَى سَيِّدِكَ وَإِلَيْكَ أَرْسَلَنِي سَيِّدِي لأَتَكَلَّمَ بِهَذَا الْكَلاَمِ؟ أَلَيْسَ إِلَى الرِّجَالِ الْجَالِسِينَ عَلَى السُّورِ لِيَأْكُلُوا عَذِرَتَهُمْ وَيَشْرَبُوا بَوْلَهُمْ مَعَكُمْ؟]
\par 28 ثُمَّ وَقَفَ رَبْشَاقَى وَنَادَى بِصَوْتٍ عَظِيمٍ بِالْيَهُودِيِّ: [اسْمَعُوا كَلاَمَ الْمَلِكِ الْعَظِيمِ مَلِكِ أَشُّورَ.
\par 29 هَكَذَا يَقُولُ الْمَلِكُ: لاَ يَخْدَعْكُمْ حَزَقِيَّا لأَنَّهُ لاَ يَقْدِرُ أَنْ يُنْقِذَكُمْ مِنْ يَدِهِ.
\par 30 وَلاَ يَجْعَلْكُمْ حَزَقِيَّا تَتَّكِلُونَ عَلَى الرَّبِّ قَائِلاً: إِنْقَاذاً يُنْقِذُنَا الرَّبُّ وَلاَ تُدْفَعُ هَذِهِ الْمَدِينَةُ إِلَى يَدِ مَلِكِ أَشُّورَ.
\par 31 لاَ تَسْمَعُوا لِحَزَقِيَّا. لأَنَّهُ هَكَذَا يَقُولُ مَلِكُ أَشُّورَ: اعْقِدُوا مَعِي صُلْحاً وَاخْرُجُوا إِلَيَّ، وَكُلُوا كُلُّ وَاحِدٍ مِنْ جَفْنَتِهِ وَكُلُّ وَاحِدٍ مِنْ تِينَتِهِ وَاشْرَبُوا كُلُّ وَاحِدٍ مَاءَ بِئْرِهِ
\par 32 حَتَّى آتِيَ وَآخُذَكُمْ إِلَى أَرْضٍ كَأَرْضِكُمْ، أَرْضَ حِنْطَةٍ وَخَمْرٍ، أَرْضَ خُبْزٍ وَكُرُومٍ، أَرْضَ زَيْتُونٍ وَعَسَلٍ وَاحْيُوا وَلاَ تَمُوتُوا. وَلاَ تَسْمَعُوا لِحَزَقِيَّا لأَنَّهُ يَغُرُّكُمْ قَائِلاً: الرَّبُّ يُنْقِذُنَا.
\par 33 هَلْ أَنْقَذَ آلِهَةُ الأُمَمِ كُلُّ وَاحِدٍ أَرْضَهُ مِنْ يَدِ مَلِكِ أَشُّورَ؟
\par 34 أَيْنَ آلِهَةُ حَمَاةَ وَأَرْفَادَ؟ أَيْنَ آلِهَةُ سَفَرْوَايِمَ وَهَيْنَعَ وَعِوَّا. هَلْ أَنْقَذُوا السَّامِرَةَ مِنْ يَدِي؟
\par 35 مَنْ مِنْ كُلِّ آلِهَةِ الأَرَاضِي أَنْقَذَ أَرْضَهُمْ مِنْ يَدِي حَتَّى يُنْقِذَ الرَّبُّ أُورُشَلِيمَ مِنْ يَدِي؟].
\par 36 فَسَكَتَ الشَّعْبُ وَلَمْ يُجِيبُوهُ بِكَلِمَةٍ، لأَنَّ أَمْرَ الْمَلِكِ كَانَ: [لاَ تُجِيبُوهُ].
\par 37 فَجَاءَ أَلِيَاقِيمُ بْنُ حِلْقِيَّا الَّذِي عَلَى الْبَيْتِ وَشِبْنَةُ الْكَاتِبُ وَيُوَاخُ بْنُ آسَافَ الْمُسَجِّلُ إِلَى حَزَقِيَّا وَثِيَابُهُمْ مُمَزَّقَةٌ، فَأَخْبَرُوهُ بِكَلاَمِ رَبْشَاقَى.

\chapter{19}

\par 1 فَلَمَّا سَمِعَ الْمَلِكُ حَزَقِيَّا ذَلِكَ مَزَّقَ ثِيَابَهُ وَتَغَطَّى بِمِسْحٍ وَدَخَلَ بَيْتَ الرَّبِّ.
\par 2 وَأَرْسَلَ أَلِيَاقِيمَ الَّذِي عَلَى الْبَيْتِ وَشِبْنَةَ الْكَاتِبَ وَشُيُوخَ الْكَهَنَةِ مُتَغَطِّينَ بِمِسْحٍ إِلَى إِشَعْيَاءَ النَّبِيِّ ابْنِ آمُوصَ.
\par 3 فَقَالُوا لَهُ: [هَكَذَا يَقُولُ حَزَقِيَّا: هَذَا الْيَوْمُ يَوْمُ شِدَّةٍ وَتَأْدِيبٍ وَإِهَانَةٍ، لأَنَّ الأَجِنَّةَ قَدْ دَنَتْ إِلَى الْمَوْلِدِ وَلاَ قُوَّةَ لِلْوِلاَدَةِ!
\par 4 لَعَلَّ الرَّبَّ إِلَهَكَ يَسْمَعُ جَمِيعَ كَلاَمِ رَبْشَاقَى الَّذِي أَرْسَلَهُ مَلِكُ أَشُّورَ سَيِّدُهُ لِيُعَيِّرَ الإِلَهَ الْحَيَّ، فَيُوَبِّخَ عَلَى الْكَلاَمِ الَّذِي سَمِعَهُ الرَّبُّ إِلَهُكَ. فَارْفَعْ صَلاَةً مِنْ أَجْلِ الْبَقِيَّةِ الْمَوْجُودَةِ].
\par 5 فَجَاءَ عَبِيدُ الْمَلِكِ حَزَقِيَّا إِلَى إِشَعْيَاءَ،
\par 6 فَقَالَ لَهُمْ إِشَعْيَاءَ: [هَكَذَا تَقُولُونَ لِسَيِّدِكُمْ: هَكَذَا قَالَ الرَّبُّ: لاَ تَخَفْ بِسَبَبِ الْكَلاَمِ الَّذِي سَمِعْتَهُ الَّذِي جَدَّفَ عَلَيَّ بِهِ غِلْمَانُ مَلِكِ أَشُّورَ.
\par 7 هَئَنَذَا أَجْعَلُ فِيهِ رُوحاً فَيَسْمَعُ خَبَراً وَيَرْجِعُ إِلَى أَرْضِهِ، وَأُسْقِطُهُ بِالسَّيْفِ فِي أَرْضِهِ].
\par 8 فَرَجَعَ رَبْشَاقَى وَوَجَدَ مَلِكَ أَشُّورَ يُحَارِبُ لِبْنَةَ، لأَنَّهُ سَمِعَ أَنَّهُ ارْتَحَلَ عَنْ لَخِيشَ.
\par 9 وَسَمِعَ عَنْ تِرْهَاقَةَ مَلِكِ كُوشٍَ قَوْلاً: [قَدْ خَرَجَ لِيُحَارِبَكَ]. فَعَادَ وَأَرْسَلَ رُسُلاً إِلَى حَزَقِيَّا قَائِلاً:
\par 10 [هَكَذَا تُكَلِّمُونَ حَزَقِيَّا مَلِكَ يَهُوذَا قَائِلِينَ: لاَ يَخْدَعْكَ إِلَهُكَ الَّذِي أَنْتَ مُتَّكِلٌ عَلَيْهِ قَائِلاً: لاَ تُدْفَعُ أُورُشَلِيمُ إِلَى يَدِ مَلِكِ أَشُّورَ.
\par 11 إِنَّكَ قَدْ سَمِعْتَ مَا فَعَلَ مُلُوكُ أَشُّورَ بِجَمِيعِ الأَرَاضِي لإِهْلاَكِهَا، وَهَلْ تَنْجُو أَنْتَ؟
\par 12 هَلْ أَنْقَذَتْ آلِهَةُ الأُمَمِ هَؤُلاَءِ الَّذِينَ أَهْلَكَهُمْ آبَائِي جُوزَانَ وَحَارَانَ وَرَصْفَ وَبَنِي عَدْنَ الَّذِينَ فِي تَلاَسَّارَ؟
\par 13 أَيْنَ مَلِكُ حَمَاةَ وَمَلِكُ أَرْفَادَ وَمَلِكُ مَدِينَةِ سَفْرَوَايِمَ وَهَيْنَعَ وَعِوَّا؟]
\par 14 فَأَخَذَ حَزَقِيَّا الرَّسَائِلَ مِنْ أَيْدِي الرُّسُلِ وَقَرَأَهَا، ثُمَّ صَعِدَ إِلَى بَيْتِ الرَّبِّ، وَنَشَرَهَا أَمَامَ الرَّبِّ.
\par 15 وَصَلَّى حَزَقِيَّا أَمَامَ الرَّبِّ وَقَالَ: [أَيُّهَا الرَّبُّ إِلَهُ إِسْرَائِيلَ، الْجَالِسُ فَوْقَ الْكَرُوبِيمَ، أَنْتَ هُوَ الإِلَهُ وَحْدَكَ لِكُلِّ مَمَالِكِ الأَرْضِ. أَنْتَ صَنَعْتَ السَّمَاءَ وَالأَرْضَ.
\par 16 أَمِلْ يَا رَبُّ أُذُنَكَ وَاسْمَعْ. افْتَحْ يَا رَبُّ عَيْنَيْكَ وَانْظُرْ، وَاسْمَعْ كَلاَمَ سَنْحَارِيبَ الَّذِي أَرْسَلَهُ لِيُعَيِّرَ اللَّهَ الْحَيَّ.
\par 17 حَقّاً يَا رَبُّ إِنَّ مُلُوكَ أَشُّورَ قَدْ خَرَّبُوا الأُمَمَ وَأَرَاضِيَهُمْ
\par 18 وَدَفَعُوا آلِهَتَهُمْ إِلَى النَّارِ. وَلأَنَّهُمْ لَيْسُوا آلِهَةً، بَلْ صَنْعَةُ أَيْدِي النَّاسِ: خَشَبٌ وَحَجَرٌ، فَأَبَادُوهُمْ.
\par 19 وَالآنَ أَيُّهَا الرَّبُّ إِلَهُنَا خَلِّصْنَا مِنْ يَدِهِ، فَتَعْلَمَ مَمَالِكُ الأَرْضِ كُلُّهَا أَنَّكَ أَنْتَ الرَّبُّ الإِلَهُ وَحْدَكَ].
\par 20 فَأَرْسَلَ إِشَعْيَاءُ بْنُ آمُوصَ إِلَى حَزَقِيَّا قَائِلاً: [هَكَذَا قَالَ الرَّبُّ إِلَهُ إِسْرَائِيلَ الَّذِي صَلَّيْتَ إِلَيْهِ مِنْ جِهَةِ سَنْحَارِيبَ مَلِكِ أَشُّورَ: قَدْ سَمِعْتُ.
\par 21 هَذَا هُوَ الْكَلاَمُ الَّذِي تَكَلَّمَ بِهِ الرَّبُّ عَلَيْهِ: احْتَقَرَتْكَ وَاسْتَهْزَأَتْ بِكَ الْعَذْرَاءُ ابْنَةُ صِهْيَوْنَ. وَنَحْوَكَ أَنْغَضَتِ ابْنَةُ أُورُشَلِيمَ رَأْسَهَا.
\par 22 مَنْ عَيَّرْتَ وَجَدَّفْتَ، وَعَلَى مَنْ عَلَّيْتَ صَوْتاً، وَقَدْ رَفَعْتَ إِلَى الْعَلاَءِ عَيْنَيْكَ عَلَى قُدُّوسِ إِسْرَائِيلَ؟
\par 23 عَلَى يَدِ رُسُلِكَ عَيَّرْتَ السَّيِّدَ، وَقُلْتَ: بِكَثْرَةِ مَرْكَبَاتِي قَدْ صَعِدْتُ إِلَى عُلْوِ الْجِبَالِ إِلَى عِقَابِ لُبْنَانَ وَأَقْطَعُ أَرْزَهُ الطَّوِيلَ وَأَفْضَلَ سَرْوِهِ، وَأَدْخُلُ أَقْصَى عُلْوِهِ وَعْرَ كَرْمَلِهِ.
\par 24 أَنَا قَدْ حَفَرْتُ وَشَرِبْتُ مِيَاهاً غَرِيبَةً، وَأُنَشِّفُ بِأَسْفَلِ قَدَمَيَّ جَمِيعَ خُلْجَانِ مِصْرَ.
\par 25 أَلَمْ تَسْمَعْ؟ مُنْذُ الْبَعِيدِ صَنَعْتُهُ، مُنْذُ الأَيَّامِ الْقَدِيمَةِ صَوَّرْتُهُ. الآنَ أَتَيْتُ بِهِ. فَتَكُونُ لِتَخْرِيبِ مُدُنٍ مُحَصَّنَةٍ حَتَّى تَصِيرَ رَوَابِيَ خَرِبَةً.
\par 26 فَسُكَّانُهَا قِصَارُ الأَيْدِي قَدِ ارْتَاعُوا وَخَجِلُوا. صَارُوا كَعُشْبِ الْحَقْلِ وَكَالنَّبَاتِ الأَخْضَرِ، كَحَشِيشِ السُّطُوحِ وَكَمَلْفُوحٍ قَبْلَ نُمُوِّهِ.
\par 27 وَلَكِنِّي عَالِمٌ بِجُلُوسِكَ وَخُرُوجِكَ وَدُخُولِكَ وَهَيَجَانِكَ عَلَيَّ.
\par 28 لأَنَّ هَيَجَانَكَ عَلَيَّ وَعَجْرَفَتَكَ قَدْ صَعِدَا إِلَى أُذُنَيَّ أَضَعُ خِزَامَتِي فِي أَنْفِكَ وَلِجَامِي فِي شَفَتَيْكَ، وَأَرُدُّكَ فِي الطَّرِيقِ الَّذِي جِئْتَ فِيهِ.
\par 29 [وَهَذِهِ لَكَ عَلاَمَةٌ: تَأْكُلُونَ هَذِهِ السَّنَةَ زِرِّيعاً، وَفِي السَّنَةِ الثَّانِيَةِ خِلْفَةً. وَأَمَّا السَّنَةُ الثَّالِثَةُ فَفِيهَا تَزْرَعُونَ وَتَحْصِدُونَ وَتَغْرِسُونَ كُرُوماً وَتَأْكُلُونَ أَثْمَارَهَا.
\par 30 وَيَعُودُ النَّاجُونَ مِنْ بَيْتِ يَهُوذَا، الْبَاقُونَ، يَتَأَصَّلُونَ إِلَى أَسْفَلٍ وَيَصْنَعُونَ ثَمَراً إِلَى مَا فَوْقُ.
\par 31 لأَنَّهُ مِنْ أُورُشَلِيمَ تَخْرُجُ الْبَقِيَّةُ وَالنَّاجُونَ مِنْ جَبَلِ صِهْيَوْنَ. غَيْرَةُ رَبِّ الْجُنُودِ تَصْنَعُ هَذَا.
\par 32 [لِذَلِكَ هَكَذَا قَالَ الرَّبُّ عَنْ مَلِكِ أَشُّورَ: لاَ يَدْخُلُ هَذِهِ الْمَدِينَةَ وَلاَ يَرْمِي هُنَاكَ سَهْماً وَلاَ يَتَقَدَّمُ عَلَيْهَا بِتُرْسٍ وَلاَ يُقِيمُ عَلَيْهَا مِتْرَسَةً.
\par 33 فِي الطَّرِيقِ الَّذِي جَاءَ فِيهِ يَرْجِعُ، وَإِلَى هَذِهِ الْمَدِينَةِ لاَ يَدْخُلُ، يَقُولُ الرَّبُّ.
\par 34 وَأُحَامِي عَنْ هَذِهِ الْمَدِينَةِ لِأُخَلِّصَهَا مِنْ أَجْلِ نَفْسِي وَمِنْ أَجْلِ دَاوُدَ عَبْدِي].
\par 35 وَكَانَ فِي تِلْكَ اللَّيْلَةِ أَنَّ مَلاَكَ الرَّبِّ خَرَجَ وَضَرَبَ مِنْ جَيْشِ أَشُّورَ مِئَةَ أَلْفٍ وَخَمْسَةً وَثَمَانِينَ أَلْفاً. وَلَمَّا بَكَّرُوا صَبَاحاً إِذَا هُمْ جَمِيعاً جُثَثٌ مَيِّتَةٌ.
\par 36 فَانْصَرَفَ سَنْحَارِيبُ مَلِكُ أَشُّورَ وَذَهَبَ رَاجِعاً وَأَقَامَ فِي نِينَوَى.
\par 37 وَفِيمَا هُوَ سَاجِدٌ فِي بَيْتِ نِسْرُوخَ إِلَهِهِ ضَرَبَهُ أَدْرَمَّلَكُ وَشَرَآصَرُ ابْنَاهُ بِالسَّيْفِ، وَنَجَوَا إِلَى أَرْضِ أَرَارَاطَ. وَمَلَكَ أَسَرْحَدُّونُ ابْنُهُ عِوَضاً عَنْهُ.

\chapter{20}

\par 1 فِي تِلْكَ الأَيَّامِ مَرِضَ حَزَقِيَّا لِلْمَوْتِ. فَجَاءَ إِلَيْهِ إِشَعْيَاءُ بْنُ آمُوصَ النَّبِيُّ وَقَالَ لَهُ: [هَكَذَا قَالَ الرَّبُّ: أَوْصِ بَيْتَكَ لأَنَّكَ تَمُوتُ وَلاَ تَعِيشُ].
\par 2 فَوَجَّهَ وَجْهَهُ إِلَى الْحَائِطِ وَصَلَّى إِلَى الرَّبِّ:
\par 3 [آهِ يَا رَبُّ، اذْكُرْ كَيْفَ سِرْتُ أَمَامَكَ بِالأَمَانَةِ وَبِقَلْبٍ سَلِيمٍ وَفَعَلْتُ الْحَسَنَ فِي عَيْنَيْكَ]. وَبَكَى حَزَقِيَّا بُكَاءً عَظِيماً.
\par 4 وَلَمْ يَخْرُجْ إِشَعْيَاءُ إِلَى الْمَدِينَةِ الْوُسْطَى حَتَّى كَانَ كَلاَمُ الرَّبِّ إِلَيْهِ:
\par 5 [ارْجِعْ وَقُلْ لِحَزَقِيَّا رَئِيسِ شَعْبِي: هَكَذَا قَالَ الرَّبُّ إِلَهُ دَاوُدَ أَبِيكَ: قَدْ سَمِعْتُ صَلاَتَكَ. قَدْ رَأَيْتُ دُمُوعَكَ. هَئَنَذَا أَشْفِيكَ. فِي الْيَوْمِ الثَّالِثِ تَصْعَدُ إِلَى بَيْتِ الرَّبِّ.
\par 6 وَأَزِيدُ عَلَى أَيَّامِكَ خَمْسَ عَشَرَةَ سَنَةً، وَأُنْقِذُكَ مِنْ يَدِ مَلِكِ أَشُّورَ مَعَ هَذِهِ الْمَدِينَةِ، وَأُحَامِي عَنْ هَذِهِ الْمَدِينَةِ مِنْ أَجْلِ نَفْسِي وَمِنْ أَجْلِ دَاوُدَ عَبْدِي].
\par 7 فَقَالَ إِشَعْيَاءُ: [خُذُوا قُرْصَ تِينٍ]. فَأَخَذُوهَا وَوَضَعُوهَا عَلَى الدَّبْلِ فَبَرِئَ.
\par 8 وَقَالَ حَزَقِيَّا لإِشَعْيَاءَ: [مَا الْعَلاَمَةُ أَنَّ الرَّبَّ يَشْفِينِي فَأَصْعَدَ فِي الْيَوْمِ الثَّالِثِ إِلَى بَيْتِ الرَّبِّ؟]
\par 9 فَقَالَ إِشَعْيَاءُ: [هَذِهِ لَكَ عَلاَمَةٌ مِنْ قِبَلِ الرَّبِّ عَلَى أَنَّ الرَّبَّ يَفْعَلُ الأَمْرَ الَّذِي تَكَلَّمَ بِهِ: هَلْ يَسِيرُ الظِّلُّ عَشَرَ دَرَجَاتٍ أَوْ يَرْجِعُ عَشَرَ دَرَجَاتٍ؟].
\par 10 فَقَالَ حَزَقِيَّا: [إِنَّهُ يَسِيرٌ عَلَى الظِّلِّ أَنْ يَمْتَدَّ عَشَرَ دَرَجَاتٍ. لاَ! بَلْ يَرْجِعُ الظِّلُّ إِلَى الْوَرَاءِ عَشَرَ دَرَجَاتٍ!]
\par 11 فَدَعَا إِشَعْيَاءُ النَّبِيُّ الرَّبَّ، فَأَرْجَعَ الظِّلَّ بِالدَّرَجَاتِ الَّتِي نَزَلَ بِهَا بِدَرَجَاتِ آحَازَ عَشْرَ دَرَجَاتٍ إِلَى الْوَرَاءِ.
\par 12 فِي ذَلِكَ الزَّمَانِ أَرْسَلَ بَرُودَخُ بَلاَدَانُ بْنُ بَلاَدَانَ مَلِكُ بَابِلَ رَسَائِلَ وَهَدِيَّةً إِلَى حَزَقِيَّا، لأَنَّهُ سَمِعَ أَنَّ حَزَقِيَّا قَدْ مَرِضَ
\par 13 فَسَمِعَ لَهُمْ حَزَقِيَّا وَأَرَاهُمْ كُلَّ بَيْتِ ذَخَائِرِهِ، وَالْفِضَّةَ وَالذَّهَبَ وَالأَطْيَابَ وَالزَّيْتَ الطَّيِّبَ، وَكُلَّ بَيْتِ أَسْلِحَتِهِ وَكُلَّ مَا وُجِدَ فِي خَزَائِنِهِ. لَمْ يَكُنْ شَيْءٌ لَمْ يُرِهِمْ إِيَّاهُ حَزَقِيَّا فِي بَيْتِهِ وَفِي كُلِّ سَلْطَنَتِهِ.
\par 14 فَجَاءَ إِشَعْيَاءُ النَّبِيُّ إِلَى الْمَلِكِ حَزَقِيَّا وَقَالَ لَهُ: [مَاذَا قَالَ هَؤُلاَءِ الرِّجَالُ، وَمِنْ أَيْنَ جَاءُوا إِلَيْكَ؟] فَقَالَ حَزَقِيَّا: [جَاءُوا مِنْ أَرْضٍ بَعِيدَةٍ، مِنْ بَابِلَ،
\par 15 فَقَالَ: [مَاذَا رَأُوا فِي بَيْتِكَ؟] فَقَالَ حَزَقِيَّا: [رَأُوا كُلَّ مَا فِي بَيْتِي. لَيْسَ فِي خَزَائِنِي شَيْءٌ لَمْ أُرِهِمْ إِيَّاهُ].
\par 16 فَقَالَ إِشَعْيَاءُ لِحَزَقِيَّا: [اسْمَعْ قَوْلَ الرَّبِّ:
\par 17 هُوَذَا تَأْتِي أَيَّامٌ يُحْمَلُ فِيهَا كُلُّ مَا فِي بَيْتِكَ وَمَا ذَخَرَهُ آبَاؤُكَ إِلَى هَذَا الْيَوْمِ إِلَى بَابِلَ. لاَ يُتْرَكُ شَيْءٌ، يَقُولُ الرَّبُّ.
\par 18 وَيُؤْخَذُ مِنْ بَنِيكَ الَّذِينَ يَخْرُجُونَ مِنْكَ الَّذِينَ تَلِدُهُمْ، فَيَكُونُونَ خِصْيَاناً فِي قَصْرِ مَلِكِ بَابِلَ].
\par 19 فَقَالَ حَزَقِيَّا لإِشَعْيَاءَ: [جَيِّدٌ هُوَ قَوْلُ الرَّبِّ الَّذِي تَكَلَّمْتَ بِهِ]. ثُمَّ قَالَ: [فَكَيْفَ لاَ، إِنْ يَكُنْ سَلاَمٌ وَأَمَانٌ فِي أَيَّامِي؟]
\par 20 وَبَقِيَّةُ أُمُورِ حَزَقِيَّا وَكُلُّ جَبَرُوتِهِ، وَكَيْفَ عَمِلَ الْبِرْكَةَ وَالْقَنَاةَ وَأَدْخَلَ الْمَاءَ إِلَى الْمَدِينَةِ مَكْتُوبَةٌ فِي سِفْرِ أَخْبَارِ الأَيَّامِ لِمُلُوكِ يَهُوذَا.
\par 21 ثُمَّ اضْطَجَعَ حَزَقِيَّا مَعَ آبَائِهِ، وَمَلَكَ مَنَسَّى ابْنُهُ عِوَضاً عَنْهُ.

\chapter{21}

\par 1 كَانَ مَنَسَّى ابْنَ اثْنَتَيْ عَشَرَةَ سَنَةً حِينَ مَلَكَ، وَمَلَكَ خَمْساً وَخَمْسِينَ سَنَةً فِي أُورُشَلِيمَ. وَاسْمُ أُمِّهِ حَفْصِيبَةُ.
\par 2 وَعَمِلَ الشَّرَّ فِي عَيْنَيِ الرَّبِّ حَسَبَ رَجَاسَاتِ الأُمَمِ الَّذِينَ طَرَدَهُمُ الرَّبُّ مِنْ أَمَامِ بَنِي إِسْرَائِيلَ.
\par 3 وَعَادَ فَبَنَى الْمُرْتَفَعَاتِ الَّتِي أَبَادَهَا حَزَقِيَّا أَبُوهُ، وَأَقَامَ مَذَابِحَ لِلْبَعْلِ وَعَمِلَ سَارِيَةً كَمَا عَمِلَ أَخْآبُ مَلِكُ إِسْرَائِيلَ، وَسَجَدَ لِكُلِّ جُنْدِ السَّمَاءِ وَعَبَدَهَا.
\par 4 وَبَنَى مَذَابِحَ فِي بَيْتِ الرَّبِّ الَّذِي قَالَ الرَّبُّ عَنْهُ: [فِي أُورُشَلِيمَ أَضَعُ اسْمِي].
\par 5 وَبَنَى مَذَابِحَ لِكُلِّ جُنْدِ السَّمَاءِ فِي دَارَيْ بَيْتِ الرَّبِّ.
\par 6 وَعَبَّرَ ابْنَهُ فِي النَّارِ، وَعَافَ وَتَفَاءَلَ وَاسْتَخْدَمَ جَانّاً وَتَوَابِعَ، وَأَكْثَرَ عَمَلَ الشَّرِّ فِي عَيْنَيِ الرَّبِّ لإِغَاظَتِهِ.
\par 7 وَوَضَعَ تِمْثَالَ السَّارِيَةِ الَّتِي عَمِلَ فِي الْبَيْتِ الَّذِي قَالَ الرَّبُّ عَنْهُ لِدَاوُدَ وَسُلَيْمَانَ ابْنِهِ: [فِي هَذَا الْبَيْتِ وَفِي أُورُشَلِيمَ الَّتِي اخْتَرْتُ مِنْ جَمِيعِ أَسْبَاطِ إِسْرَائِيلَ أَضَعُ اسْمِي إِلَى الأَبَدِ.
\par 8 وَلاَ أَعُودُ أُزَحْزِحُ رِجْلَ إِسْرَائِيلَ مِنَ الأَرْضِ الَّتِي أَعْطَيْتُ لِآبَائِهِمْ، وَذَلِكَ إِذَا حَفِظُوا وَعَمِلُوا حَسَبَ كُلِّ مَا أَوْصَيْتُهُمْ بِهِ وَكُلَّ الشَّرِيعَةِ الَّتِي أَمَرَهُمْ بِهَا عَبْدِي مُوسَى].
\par 9 فَلَمْ يَسْمَعُوا بَلْ أَضَلَّهُمْ مَنَسَّى لِيَعْمَلُوا مَا هُوَ أَقْبَحُ مِنَ الأُمَمِ الَّذِينَ طَرَدَهُمُ الرَّبُّ مِنْ أَمَامِ بَنِي إِسْرَائِيلَ.
\par 10 وَقَالَ الرَّبُّ عَنْ يَدِ عَبِيدِهِ الأَنْبِيَاءِ:
\par 11 [مِنْ أَجْلِ أَنَّ مَنَسَّى مَلِكَ يَهُوذَا قَدْ عَمِلَ هَذِهِ الأَرْجَاسَ، وَأَسَاءَ أَكْثَرَ مِنْ جَمِيعِ الَّذِي عَمِلَهُ الأَمُورِيُّونَ الَّذِينَ قَبْلَهُ، وَجَعَلَ أَيْضاً يَهُوذَا يُخْطِئُ بِأَصْنَامِهِ
\par 12 لِذَلِكَ هَكَذَا قَالَ الرَّبُّ إِلَهُ إِسْرَائِيلَ: هَئَنَذَا جَالِبٌ شَرّاً عَلَى أُورُشَلِيمَ وَيَهُوذَا حَتَّى أَنَّ كُلَّ مَنْ يَسْمَعُ بِهِ تَطِنُّ أُذُنَاهُ.
\par 13 وَأَمُدُّ عَلَى أُورُشَلِيمَ خَيْطَ السَّامِرَةِ وَمِطْمَارَ بَيْتِ أَخْآبَ، وَأَمْسَحُ أُورُشَلِيمَ كَمَا يَمْسَحُ وَاحِدٌ الصَّحْنَ. يَمْسَحُهُ وَيَقْلِبُهُ عَلَى وَجْهِهِ.
\par 14 وَأَرْفُضُ بَقِيَّةَ مِيرَاثِي، وَأَدْفَعُهُمْ إِلَى أَيْدِي أَعْدَائِهِمْ، فَيَكُونُونَ غَنِيمَةً وَنَهْباً لِجَمِيعِ أَعْدَائِهِمْ،
\par 15 لأَنَّهُمْ عَمِلُوا الشَّرَّ فِي عَيْنَيَّ، وَصَارُوا يُغِيظُونَنِي مِنَ الْيَوْمِ الَّذِي فِيهِ خَرَجَ آبَاؤُهُمْ مِنْ مِصْرَ إِلَى هَذَا الْيَوْمِ].
\par 16 وَسَفَكَ أَيْضاً مَنَسَّى دَماً بَرِيئاً كَثِيراً جِدّاً حَتَّى مَلَأَ أُورُشَلِيمَ مِنَ الْجَانِبِ إِلَى الْجَانِبِ، فَضْلاً عَنْ خَطِيَّتِهِ الَّتِي بِهَا جَعَلَ يَهُوذَا يُخْطِئُ بِعَمَلِ الشَّرِّ فِي عَيْنَيِ الرَّبِّ.
\par 17 وَبَقِيَّةُ أُمُورِ مَنَسَّى وَكُلُّ مَا عَمِلَ، وَخَطِيَّتُهُ الَّتِي أَخْطَأَ بِهَا مَكْتُوبَةٌ فِي سِفْرِ أَخْبَارِ الأَيَّامِ لِمُلُوكِ يَهُوذَا.
\par 18 ثُمَّ اضْطَجَعَ مَنَسَّى مَعَ آبَائِهِ وَدُفِنَ فِي بُسْتَانِ بَيْتِهِ فِي بُسْتَانِ عُزَّا، وَمَلَكَ آمُونُ ابْنُهُ عِوَضاً عَنْهُ.
\par 19 كَانَ آمُونُ ابْنَ اثْنَتَيْنِ وَعِشْرِينَ سَنَةً حِينَ مَلَكَ، وَمَلَكَ سَنَتَيْنِ فِي أُورُشَلِيمَ. وَاسْمُ أُمِّهِ مَشُلَّمَةُ بِنْتُ حَارُوصَ مِنْ يَطْبَةَ.
\par 20 وَعَمِلَ الشَّرَّ فِي عَيْنَيِ الرَّبِّ كَمَا عَمِلَ مَنَسَّى أَبُوهُ.
\par 21 وَسَلَكَ فِي كُلِّ الطَّرِيقِ الَّذِي سَلَكَ فِيهِ أَبُوهُ، وَعَبَدَ الأَصْنَامَ الَّتِي عَبَدَهَا أَبُوهُ وَسَجَدَ لَهَا.
\par 22 وَتَرَكَ الرَّبَّ إِلَهَ آبَائِهِ وَلَمْ يَسْلُكْ فِي طَرِيقِ الرَّبِّ.
\par 23 وَفَتَنَ عَبِيدُ آمُونَ عَلَيْهِ فَقَتَلُوا الْمَلِكَ فِي بَيْتِهِ.
\par 24 فَضَرَبَ كُلُّ شَعْبِ الأَرْضِ جَمِيعَ الْفَاتِنِينَ عَلَى الْمَلِكِ آمُونَ، وَمَلَّكَ شَعْبُ الأَرْضِ يُوشِيَّا ابْنَهُ عِوَضاً عَنْهُ.
\par 25 وَبَقِيَّةُ أُمُورِ آمُونَ الَّتِي عَمِلَ مَكْتُوبَةٌ فِي سِفْرِ أَخْبَارِ الأَيَّامِ لِمُلُوكِ يَهُوذَا.
\par 26 وَدُفِنَ فِي قَبْرِهِ فِي بُسْتَانِ عُزَّا، وَمَلَكَ يُوشِيَّا ابْنُهُ عِوَضاً عَنْهُ.

\chapter{22}

\par 1 كَانَ يُوشِيَّا ابْنَ ثَمَانِ سِنِينٍ حِينَ مَلَكَ، وَمَلَكَ إِحْدَى وَثَلاَثِينَ سَنَةً فِي أُورُشَلِيمَ. وَاسْمُ أُمِّهِ يَدْيَدَةُ بِنْتُ عَدَايَةَ مِنْ بُصْقَةَ.
\par 2 وَعَمِلَ الْمُسْتَقِيمَ فِي عَيْنَيِ الرَّبِّ وَسَارَ فِي جَمِيعِ طَرِيقِ دَاوُدَ أَبِيهِ. وَلَمْ يَحِدْ يَمِيناً وَلاَ شِمَالاً.
\par 3 وَفِي السَّنَةِ الثَّامِنَةَ عَشَرَةَ لِلْمَلِكِ يُوشِيَّا أَرْسَلَ الْمَلِكُ شَافَانَ بْنَ أَصَلْيَا بْنِ مَشُلاَّمَ الْكَاتِبَ إِلَى بَيْتِ الرَّبِّ قَائِلاً:
\par 4 [اصْعَدْ إِلَى حِلْقِيَّا الْكَاهِنِ الْعَظِيمِ فَيَحْسِبَ الْفِضَّةَ الْمُدْخَلَةَ إِلَى بَيْتِ الرَّبِّ الَّتِي جَمَعَهَا حَارِسُو الْبَابِ مِنَ الشَّعْبِ،
\par 5 فَيَدْفَعُوهَا لِيَدِ عَامِلِي الشُّغْلِ الْمُوَكَّلِينَ بِبَيْتِ الرَّبِّ، وَيَدْفَعُوهَا إِلَى عَامِلِي الشُّغْلِ الَّذِي فِي بَيْتِ الرَّبِّ لِتَرْمِيمِ ثُلَمِ الْبَيْتِ:
\par 6 لِلنَّجَّارِينَ وَالْبَنَّائِينَ وَالنَّحَّاتِينَ، وَلِشِرَاءِ أَخْشَابٍ وَحِجَارَةٍ مَنْحُوتَةٍ لأَجْلِ تَرْمِيمِ الْبَيْتِ].
\par 7 إِلاَّ أَنَّهُمْ لَمْ يُحَاسَبُوا بِالْفِضَّةِ الْمَدْفُوعَةِ لأَيْدِيهِمْ لأَنَّهُمْ إِنَّمَا عَمِلُوا بِأَمَانَةٍ.
\par 8 فَقَالَ حِلْقِيَّا الْكَاهِنُ الْعَظِيمُ لِشَافَانَ الْكَاتِبِ: [قَدْ وَجَدْتُ سِفْرَ الشَّرِيعَةِ فِي بَيْتِ الرَّبِّ]. وَسَلَّمَ حِلْقِيَّا السِّفْرَ لِشَافَانَ فَقَرَأَهُ.
\par 9 وَجَاءَ شَافَانُ الْكَاتِبُ إِلَى الْمَلِكِ وَقَالَ: [قَدْ أَفْرَغَ عَبِيدُكَ الْفِضَّةَ الْمَوْجُودَةَ فِي الْبَيْتِ وَدَفَعُوهَا إِلَى يَدِ عَامِلِي الشُّغْلِ وُكَلاَءِ بَيْتِ الرَّبِّ].
\par 10 وَأَخْبَرَ شَافَانُ الْكَاتِبُ الْمَلِكَ: [قَدْ أَعْطَانِي حِلْقِيَّا الْكَاهِنُ سِفْراً]. وَقَرَأَهُ شَافَانُ أَمَامَ الْمَلِكِ.
\par 11 فَلَمَّا سَمِعَ الْمَلِكُ كَلاَمَ سِفْرِ الشَّرِيعَةِ مَزَّقَ ثِيَابَهُ.
\par 12 وَأَمَرَ الْمَلِكُ حِلْقِيَّا الْكَاهِنَ وَأَخِيقَامَ بْنَ شَافَانَ وَعَكْبُورَ بْنَ مِيخَا وَشَافَانَ الْكَاتِبَ وَعَسَايَا عَبْدَ الْمَلِكِ:
\par 13 [اذْهَبُوا اسْأَلُوا الرَّبَّ لأَجْلِي وَلأَجْلِ الشَّعْبِ وَلأَجْلِ كُلِّ يَهُوذَا مِنْ جِهَةِ كَلاَمِ هَذَا السِّفْرِ الَّذِي وُجِدَ. لأَنَّهُ عَظِيمٌ هُوَ غَضَبُ الرَّبِّ الَّذِي اشْتَعَلَ عَلَيْنَا مِنْ أَجْلِ أَنَّ آبَاءَنَا لَمْ يَسْمَعُوا لِكَلاَمِ هَذَا السِّفْرِ لِيَعْمَلُوا حَسَبَ كُلِّ مَا هُوَ مَكْتُوبٌ عَلَيْنَا].
\par 14 فَذَهَبَ حِلْقِيَّا الْكَاهِنُ وَأَخِيقَامُ وَعَكْبُورُ وَشَافَانُ وَعَسَايَا إِلَى خَلْدَةَ النَّبِيَّةِ، امْرَأَةِ شَلُّومَ بْنِ تِقْوَةَ بْنِ حَرْحَسَ حَارِسِ الثِّيَابِ. وَهِيَ سَاكِنَةٌ فِي أُورُشَلِيمَ فِي الْقِسْمِ الثَّانِي وَكَلَّمُوهَا.
\par 15 فَقَالَتْ لَهُمْ: [هَكَذَا قَالَ الرَّبُّ إِلَهُ إِسْرَائِيلَ. قُولُوا لِلرَّجُلِ الَّذِي أَرْسَلَكُمْ إِلَيَّ:
\par 16 هَكَذَا قَالَ الرَّبُّ: هَئَنَذَا جَالِبٌ شَرّاً عَلَى هَذَا الْمَوْضِعِ وَعَلَى سُكَّانِهِ، كُلَّ كَلاَمِ السِّفْرِ الَّذِي قَرَأَهُ مَلِكُ يَهُوذَا،
\par 17 مِنْ أَجْلِ أَنَّهُمْ تَرَكُونِي وَأَوْقَدُوا لِآلِهَةٍ أُخْرَى لِيُغِيظُونِي بِكُلِّ عَمَلِ أَيْدِيهِمْ، فَيَشْتَعِلُ غَضَبِي عَلَى هَذَا الْمَوْضِعِ وَلاَ يَنْطَفِئُ.
\par 18 وَأَمَّا مَلِكُ يَهُوذَا الَّذِي أَرْسَلَكُمْ لِتَسْأَلُوا الرَّبَّ فَهَكَذَا تَقُولُونَ لَهُ: هَكَذَا قَالَ الرَّبُّ إِلَهُ إِسْرَائِيلَ مِنْ جِهَةِ الْكَلاَمِ الَّذِي سَمِعْتَ:
\par 19 مِنْ أَجْلِ أَنَّهُ قَدْ رَقَّ قَلْبُكَ وَتَوَاضَعْتَ أَمَامَ الرَّبِّ حِينَ سَمِعْتَ مَا تَكَلَّمْتُ بِهِ عَلَى هَذَا الْمَوْضِعِ وَعَلَى سُكَّانِهِ أَنَّهُمْ يَصِيرُونَ دَهَشاً وَلَعْنَةً، وَمَزَّقْتَ ثِيَابَكَ وَبَكَيْتَ أَمَامِي. قَدْ سَمِعْتُ أَنَا أَيْضاً يَقُولُ الرَّبُّ.
\par 20 لِذَلِكَ هَئَنَذَا أَضُمُّكَ إِلَى آبَائِكَ فَتُضَمُّ إِلَى قَبْرِكَ بِسَلاَمٍ، وَلاَ تَرَى عَيْنَاكَ كُلَّ الشَّرِّ الَّذِي أَنَا جَالِبُهُ عَلَى هَذَا الْمَوْضِعِ]. فَرَدُّوا عَلَى الْمَلِكِ جَوَاباً.

\chapter{23}

\par 1 وَأَرْسَلَ الْمَلِكُ، فَجَمَعُوا إِلَيْهِ كُلَّ شُيُوخِ يَهُوذَا وَأُورُشَلِيمَ.
\par 2 وَصَعِدَ الْمَلِكُ إِلَى بَيْتِ الرَّبِّ وَجَمِيعُ رِجَالِ يَهُوذَا وَكُلُّ سُكَّانِ أُورُشَلِيمَ مَعَهُ، وَالْكَهَنَةُ وَالأَنْبِيَاءُ وَكُلُّ الشَّعْبِ مِنَ الصَّغِيرِ إِلَى الْكَبِيرِ، وَقَرَأَ فِي آذَانِهِمْ كُلَّ كَلاَمِ سِفْرِ الشَّرِيعَةِ الَّذِي وُجِدَ فِي بَيْتِ الرَّبِّ.
\par 3 وَوَقَفَ الْمَلِكُ عَلَى الْمِنْبَرِ وَقَطَعَ عَهْداً أَمَامَ الرَّبِّ لِلذَّهَابِ وَرَاءَ الرَّبِّ وَلِحِفْظِ وَصَايَاهُ وَشَهَادَاتِهِ وَفَرَائِضِهِ بِكُلِّ الْقَلْبِ وَكُلِّ النَّفْسِ، لإِقَامَةِ كَلاَمِ هَذَا الْعَهْدِ الْمَكْتُوبِ فِي هَذَا السِّفْرِ. وَوَقَفَ جَمِيعُ الشَّعْبِ عِنْدَ الْعَهْدِ.
\par 4 وَأَمَرَ الْمَلِكُ حِلْقِيَّا الْكَاهِنَ الْعَظِيمَ وَكَهَنَةَ الْفِرْقَةِ الثَّانِيَةِ وَحُرَّاسَ الْبَابِ أَنْ يُخْرِجُوا مِنْ هَيْكَلِ الرَّبِّ جَمِيعَ الآنِيَةِ الْمَصْنُوعَةِ لِلْبَعْلِ وَلِلسَّارِيَةِ وَلِكُلِّ أَجْنَادِ السَّمَاءِ، وَأَحْرَقَهَا خَارِجَ أُورُشَلِيمَ فِي حُقُولِ قَدْرُونَ، وَحَمَلَ رَمَادَهَا إِلَى بَيْتِ إِيلَ.
\par 5 وَلاَشَى كَهَنَةَ الأَصْنَامِ الَّذِينَ جَعَلَهُمْ مُلُوكُ يَهُوذَا لِيُوقِدُوا عَلَى الْمُرْتَفَعَاتِ فِي مُدُنِ يَهُوذَا وَمَا يُحِيطُ بِأُورُشَلِيمَ، وَالَّذِينَ يُوقِدُونَ لِلْبَعْلِ: لِلشَّمْسِ وَالْقَمَرِ وَالْمَنَازِلِ، وَلِكُلِّ أَجْنَادِ السَّمَاءِ.
\par 6 وَأَخْرَجَ السَّارِيَةَ مِنْ بَيْتِ الرَّبِّ خَارِجَ أُورُشَلِيمَ إِلَى وَادِي قَدْرُونَ وَأَحْرَقَهَا فِي وَادِي قَدْرُونَ، وَدَقَّهَا إِلَى أَنْ صَارَتْ غُبَاراً، وَذَرَّى الْغُبَارَ عَلَى قُبُورِ عَامَّةِ الشَّعْبِ.
\par 7 وَهَدَمَ بُيُوتَ الْمَأْبُونِينَ الَّتِي عِنْدَ بَيْتِ الرَّبِّ حَيْثُ كَانَتِ النِّسَاءُ يَنْسِجْنَ بُيُوتاً لِلسَّارِيَةِ.
\par 8 وَجَاءَ بِجَمِيعِ الْكَهَنَةِ مِنْ مُدُنِ يَهُوذَا، وَنَجَّسَ الْمُرْتَفَعَاتِ حَيْثُ كَانَ الْكَهَنَةُ يُوقِدُونَ مِنْ جَبْعَ إِلَى بِئْرِ سَبْعٍ، وَهَدَمَ مُرْتَفَعَاتِ الأَبْوَابِ الَّتِي عِنْدَ مَدْخَلِ بَابِ يَشُوعَ رَئِيسِ الْمَدِينَةِ الَّتِي عَنِ الْيَسَارِ فِي بَابِ الْمَدِينَةِ.
\par 9 إِلاَّ أَنَّ كَهَنَةَ الْمُرْتَفَعَاتِ لَمْ يَصْعَدُوا إِلَى مَذْبَحِ الرَّبِّ فِي أُورُشَلِيمَ بَلْ أَكَلُوا فَطِيراً بَيْنَ إِخْوَتِهِمْ.
\par 10 وَنَجَّسَ تُوفَةَ الَّتِي فِي وَادِي بَنِي هِنُّومَ لِكَيْ لاَ يُعَبِّرَ أَحَدٌ ابْنَهُ أَوِ ابْنَتَهُ فِي النَّارِ لِمُولَكَ.
\par 11 وَأَبَادَ الْخَيْلَ الَّتِي أَعْطَاهَا مُلُوكُ يَهُوذَا لِلشَّمْسِ عِنْدَ مَدْخَلِ بَيْتِ الرَّبِّ عِنْدَ مِخْدَعِ نَثْنَمْلَكَ الْخَصِيِّ الَّذِي فِي الأَرْوِقَةِ، وَمَرْكَبَاتُِ الشَّمْسِ أَحْرَقَهَا بِالنَّارِ.
\par 12 وَالْمَذَابِحُ الَّتِي عَلَى سَطْحِ عُلِّيَّةِ آحَازَ الَّتِي عَمِلَهَا مُلُوكُ يَهُوذَا، وَالْمَذَابِحُ الَّتِي عَمِلَهَا مَنَسَّى فِي دَارَيْ بَيْتِ الرَّبِّ، هَدَمَهَا الْمَلِكُ. وَرَكَضَ مِنْ هُنَاكَ وَذَرَّى غُبَارَهَا فِي وَادِي قَدْرُونَ.
\par 13 وَالْمُرْتَفَعَاتُ الَّتِي قُبَالَةَ أُورُشَلِيمَ الَّتِي عَنْ يَمِينِ جَبَلِ الْهَلاَكِ الَّتِي بَنَاهَا سُلَيْمَانُ مَلِكُ إِسْرَائِيلَ لِعَشْتُورَثَ رَجَاسَةِ الصَّيْدُونِيِّينَ، وَلِكَمُوشَ رَجَاسَةِ الْمُوآبِيِّينَ، وَلِمَلْكُومَ كَرَاهَةِ بَنِي عَمُّونَ، نَجَّسَهَا الْمَلِكُ.
\par 14 وَكَسَّرَ التَّمَاثِيلَ وَقَطَّعَ السَّوَارِيَ وَمَلَأَ مَكَانَهَا مِنْ عِظَامِ النَّاسِ.
\par 15 وَكَذَلِكَ الْمَذْبَحُ الَّذِي فِي بَيْتِ إِيلَ فِي الْمُرْتَفَعَةِ الَّتِي عَمِلَهَا يَرُبْعَامُ بْنُ نَبَاطَ الَّذِي جَعَلَ إِسْرَائِيلَ يُخْطِئُ، فَذَانِكَ الْمَذْبَحُ وَالْمُرْتَفَعَةُ هَدَمَهُمَا وَأَحْرَقَ الْمُرْتَفَعَةَ وَسَحَقَهَا حَتَّى صَارَتْ غُبَاراً، وَأَحْرَقَ السَّارِيَةَ.
\par 16 وَالْتَفَتَ يُوشِيَّا فَرَأَى الْقُبُورَ الَّتِي هُنَاكَ فِي الْجَبَلِ، فَأَرْسَلَ وَأَخَذَ الْعِظَامَ مِنَ الْقُبُورِ وَأَحْرَقَهَا عَلَى الْمَذْبَحِ وَنَجَّسَهُ حَسَبَ كَلاَمِ الرَّبِّ الَّذِي نَادَى بِهِ رَجُلُ اللَّهِ الَّذِي نَادَى بِهَذَا الْكَلاَمِ.
\par 17 وَقَالَ: [مَا هَذِهِ الصُّوَّةُ الَّتِي أَرَى؟] فَقَالَ لَهُ رِجَالُ الْمَدِينَةِ: [هِيَ قَبْرُ رَجُلِ اللَّهِ الَّذِي جَاءَ مِنْ يَهُوذَا وَنَادَى بِهَذِهِ الأُمُورِ الَّتِي عَمِلْتَ عَلَى مَذْبَحِ بَيْتِ إِيلَ].
\par 18 فَقَالَ: [دَعُوهُ. لاَ يُحَرِّكَنَّ أَحَدٌ عِظَامَهُ]. فَتَرَكُوا عِظَامَهُ وَعِظَامَ النَّبِيِّ الَّذِي جَاءَ مِنَ السَّامِرَةِ.
\par 19 وَكَذَا جَمِيعُ بُيُوتِ الْمُرْتَفَعَاتِ الَّتِي فِي مُدُنِ السَّامِرَةِ الَّتِي عَمِلَهَا مُلُوكُ إِسْرَائِيلَ لِلإِغَاظَةِ أَزَالَهَا يُوشِيَّا، وَعَمِلَ بِهَا حَسَبَ جَمِيعِ الأَعْمَالِ الَّتِي عَمِلَهَا فِي بَيْتِ إِيلَ.
\par 20 وَذَبَحَ جَمِيعَ كَهَنَةِ الْمُرْتَفَعَاتِ الَّتِي هُنَاكَ عَلَى الْمَذَابِحِ، وَأَحْرَقَ عِظَامَ النَّاسِ عَلَيْهَا، ثُمَّ رَجَعَ إِلَى أُورُشَلِيمَ.
\par 21 وَأَمَرَ الْمَلِكُ جَمِيعَ الشَّعْبِ: [اعْمَلُوا فِصْحاً لِلرَّبِّ إِلَهِكُمْ كَمَا هُوَ مَكْتُوبٌ فِي سِفْرِ الْعَهْدِ هَذَا].
\par 22 إِنَّهُ لَمْ يُعْمَلْ مِثْلُ هَذَا الْفِصْحِ مُنْذُ أَيَّامِ الْقُضَاةِ الَّذِينَ حَكَمُوا عَلَى إِسْرَائِيلَ، وَلاَ فِي كُلِّ أَيَّامِ مُلُوكِ إِسْرَائِيلَ وَمُلُوكِ يَهُوذَا.
\par 23 وَلَكِنْ فِي السَّنَةِ الثَّامِنَةَ عَشَرَةَ لِلْمَلِكِ يُوشِيَّا عُمِلَ هَذَا الْفِصْحُ لِلرَّبِّ فِي أُورُشَلِيمَ.
\par 24 وَكَذَلِكَ السَّحَرَةُ وَالْعَرَّافُونَ وَالتَّرَافِيمُ وَالأَصْنَامُ وَجَمِيعُ الرَّجَاسَاتِ الَّتِي رُئِيَتْ فِي أَرْضِ يَهُوذَا وَفِي أُورُشَلِيمَ أَبَادَهَا يُوشِيَّا لِيُقِيمَ كَلاَمَ الشَّرِيعَةِ الْمَكْتُوبَ فِي السِّفْرِ الَّذِي وَجَدَهُ حِلْقِيَّا الْكَاهِنُ فِي بَيْتِ الرَّبِّ.
\par 25 وَلَمْ يَكُنْ قَبْلَهُ مَلِكٌ مِثْلُهُ قَدْ رَجَعَ إِلَى الرَّبِّ بِكُلِّ قَلْبِهِ وَكُلِّ نَفْسِهِ وَكُلِّ قُوَّتِهِ حَسَبَ كُلِّ شَرِيعَةِ مُوسَى، وَبَعْدَهُ لَمْ يَقُمْ مِثْلُهُ.
\par 26 وَلَكِنَّ الرَّبَّ لَمْ يَرْجِعْ عَنْ حُمُوِّ غَضَبِهِ الْعَظِيمِ، لأَنَّ غَضَبَهُ حَمِيَ عَلَى يَهُوذَا مِنْ أَجْلِ جَمِيعِ الإِغَاظَاتِ الَّتِي أَغَاظَهُ إِيَّاهَا مَنَسَّى.
\par 27 فَقَالَ الرَّبُّ: [إِنِّي أَنْزِعُ يَهُوذَا أَيْضاً مِنْ أَمَامِي كَمَا نَزَعْتُ إِسْرَائِيلَ، وَأَرْفُضُ هَذِهِ الْمَدِينَةَ الَّتِي اخْتَرْتُهَا أُورُشَلِيمَ وَالْبَيْتَ الَّذِي قُلْتُ يَكُونُ اسْمِي فِيهِ].
\par 28 وَبَقِيَّةُ أُمُورِ يُوشِيَّا وَكُلُّ مَا عَمِلَ مَكْتُوبَةٌ فِي سِفْرِ أَخْبَارِ الأَيَّامِ لِمُلُوكِ يَهُوذَا.
\par 29 فِي أَيَّامِهِ صَعِدَ فِرْعَوْنُ نَخُو مَلِكُ مِصْرَ عَلَى مَلِكِ أَشُّورَ إِلَى نَهْرِ الْفُرَاتِ. فَصَعِدَ الْمَلِكُ يُوشِيَّا لِلِقَائِهِ، فَقَتَلَهُ فِي مَجِدُّو حِينَ رَآهُ.
\par 30 وَأَرْكَبَهُ عَبِيدُهُ مَيِّتاً مِنْ مَجِدُّو وَجَاءُوا بِهِ إِلَى أُورُشَلِيمَ وَدَفَنُوهُ فِي قَبْرِهِ. فَأَخَذَ شَعْبُ الأَرْضِ يَهُوآحَازَ بْنَ يُوشِيَّا وَمَسَحُوهُ وَمَلَّكُوهُ عِوَضاً عَنْ أَبِيهِ.
\par 31 كَانَ يَهُوآحَازُ ابْنَ ثَلاَثٍ وَعِشْرِينَ سَنَةً حِينَ مَلَكَ، وَمَلَكَ ثَلاَثَةَ أَشْهُرٍ فِي أُورُشَلِيمَ، وَاسْمُ أُمِّهِ حَمُوطَلُ بِنْتُ إِرْمِيَا مِنْ لِبْنَةَ.
\par 32 فَعَمِلَ الشَّرَّ فِي عَيْنَيِ الرَّبِّ حَسَبَ كُلِّ مَا عَمِلَهُ آبَاؤُهُ.
\par 33 وَأَسَرَهُ فِرْعَوْنُ نَخُو فِي رَبْلَةَ فِي أَرْضِ حَمَاةَ لِئَلاَّ يَمْلِكَ فِي أُورُشَلِيمَ، وَغَرَّمَ الأَرْضَ بِمِئَةِ وَزْنَةٍ مِنَ الْفِضَّةِ وَوَزْنَةٍ مِنَ الذَّهَبِ.
\par 34 وَمَلَّكَ فِرْعَوْنُ نَخُو أَلِيَاقِيمَ بْنَ يُوشِيَّا عِوَضاً عَنْ يُوشِيَّا أَبِيهِ، وَغَيَّرَ اسْمَهُ إِلَى يَهُويَاقِيمَ، وَأَخَذَ يَهُوآحَازَ وَجَاءَ إِلَى مِصْرَ فَمَاتَ هُنَاكَ.
\par 35 وَدَفَعَ يَهُويَاقِيمُ الْفِضَّةَ وَالذَّهَبَ لِفِرْعَوْنَ، إِلاَّ أَنَّهُ قَوَّمَ الأَرْضَ لِدَفْعِ الْفِضَّةِ بِأَمْرِ فِرْعَوْنَ. كُلَّ وَاحِدٍ حَسَبَ تَقْوِيمِهِ. فَطَالَبَ شَعْبَ الأَرْضِ بِالْفِضَّةِ وَالذَّهَبِ لِيَدْفَعَ لِفِرْعَوْنَ نَخُو.
\par 36 كَانَ يَهُويَاقِيمُ ابْنَ خَمْسٍ وَعِشْرِينَ سَنَةً حِينَ مَلَكَ. وَمَلَكَ إِحْدَى عَشَرَةَ سَنَةً فِي أُورُشَلِيمَ. وَاسْمُ أُمِّهِ زَبِيدَةُ بِنْتُ فِدَايَةَ مِنْ رُومَةَ.
\par 37 وَعَمِلَ الشَّرَّ فِي عَيْنَيِ الرَّبِّ حَسَبَ كُلِّ مَا عَمِلَ آبَاؤُهُ.

\chapter{24}

\par 1 فِي أَيَّامِهِ صَعِدَ نَبُوخَذْنَصَّرُ مَلِكُ بَابِلَ، فَكَانَ لَهُ يَهُويَاقِيمُ عَبْداً ثَلاَثَ سِنِينَ. ثُمَّ عَادَ فَتَمَرَّدَ عَلَيْهِ.
\par 2 فَأَرْسَلَ الرَّبُّ عَلَيْهِ غُزَاةَ الْكَلْدَانِيِّينَ وَغُزَاةَ الأَرَامِيِّينَ وَغُزَاةَ الْمُوآبِيِّينَ وَغُزَاةَ بَنِي عَمُّونَ وَأَرْسَلَهُمْ عَلَى يَهُوذَا لِيُبِيدَهَا حَسَبَ كَلاَمِ الرَّبِّ الَّذِي تَكَلَّمَ بِهِ عَنْ يَدِ عَبِيدِهِ الأَنْبِيَاءِ.
\par 3 إِنَّ ذَلِكَ كَانَ حَسَبَ كَلاَمِ الرَّبِّ عَلَى يَهُوذَا لِيَنْزِعَهُمْ مِنْ أَمَامِهِ لأَجْلِ خَطَايَا مَنَسَّى حَسَبَ كُلِّ مَا عَمِلَ.
\par 4 وَكَذَلِكَ لأَجْلِ الدَّمِ الْبَرِيءِ الَّذِي سَفَكَهُ، لأَنَّهُ مَلَأَ أُورُشَلِيمَ دَماً بَرِيئاً، وَلَمْ يَشَإِ الرَّبُّ أَنْ يَغْفِرَ.
\par 5 وَبَقِيَّةُ أُمُورِ يَهُويَاقِيمَ وَكُلُّ مَا عَمِلَ مَكْتُوبَةٌ فِي سِفْرِ أَخْبَارِ الأَيَّامِ لِمُلُوكِ يَهُوذَا.
\par 6 ثُمَّ اضْطَجَعَ يَهُويَاقِيمُ مَعَ آبَائِهِ، وَمَلَكَ يَهُويَاكِينُ ابْنُهُ عِوَضاً عَنْهُ.
\par 7 وَلَمْ يَعُدْ أَيْضاً مَلِكُ مِصْرَ يَخْرُجُ مِنْ أَرْضِهِ لأَنَّ مَلِكَ بَابِلَ أَخَذَ مِنْ نَهْرِ مِصْرَ إِلَى نَهْرِ الْفُرَاتِ كُلَّ مَا كَانَ لِمَلِكِ مِصْرَ.
\par 8 كَانَ يَهُويَاكِينُ ابْنَ ثَمَانِي عَشَرَةَ سَنَةً حِينَ مَلَكَ، وَمَلَكَ ثَلاَثَةَ أَشْهُرٍ فِي أُورُشَلِيمَ. وَاسْمُ أُمِّهِ نَحُوشْتَا بِنْتُ أَلْنَاثَانَ مِنْ أُورُشَلِيمَ.
\par 9 وَعَمِلَ الشَّرَّ فِي عَيْنَيِ الرَّبِّ حَسَبَ كُلِّ مَا عَمِلَ أَبُوهُ.
\par 10 فِي ذَلِكَ الزَّمَانِ صَعِدَ عَبِيدُ نَبُوخَذْنَصَّرَ مَلِكِ بَابِلَ إِلَى أُورُشَلِيمَ، فَدَخَلَتِ الْمَدِينَةُ تَحْتَ الْحِصَارِ.
\par 11 وَجَاءَ نَبُوخَذْنَصَّرُ مَلِكُ بَابِلَ عَلَى الْمَدِينَةِ وَكَانَ عَبِيدُهُ يُحَاصِرُونَهَا.
\par 12 فَخَرَجَ يَهُويَاكِينُ مَلِكُ يَهُوذَا إِلَى مَلِكِ بَابِلَ هُوَ وَأُمُّهُ وَعَبِيدُهُ وَرُؤَسَاؤُهُ وَخِصْيَانُهُ، وَأَخَذَهُ مَلِكُ بَابِلَ فِي السَّنَةِ الثَّامِنَةِ مِنْ مُلْكِهِ.
\par 13 وَأَخْرَجَ مِنْ هُنَاكَ جَمِيعَ خَزَائِنِ بَيْتِ الرَّبِّ وَخَزَائِنِ بَيْتِ الْمَلِكِ، وَكَسَّرَ كُلَّ آنِيَةِ الذَّهَبِ الَّتِي عَمِلَهَا سُلَيْمَانُ مَلِكُ إِسْرَائِيلَ فِي هَيْكَلِ الرَّبِّ، كَمَا تَكَلَّمَ الرَّبُّ.
\par 14 وَسَبَى كُلَّ أُورُشَلِيمَ وَكُلَّ الرُّؤَسَاءِ وَجَمِيعَ جَبَابِرَةِ الْبَأْسِ، عَشَرَةَ آلاَفِ مَسْبِيٍّ، وَجَمِيعَ الصُّنَّاعِ وَالأَقْيَانِ. لَمْ يَبْقَ أَحَدٌ إِلاَّ مَسَاكِينُ شَعْبِ الأَرْضِ.
\par 15 وَسَبَى يَهُويَاكِينَ إِلَى بَابِلَ. وَأُمَّ الْمَلِكِ وَنِسَاءَ الْمَلِكِ وَخِصْيَانَهُ وَأَقْوِيَاءَ الأَرْضِ سَبَاهُمْ مِنْ أُورُشَلِيمَ إِلَى بَابِلَ.
\par 16 وَجَمِيعُ أَصْحَابِ الْبَأْسِ، سَبْعَةُ آلاَفٍ، وَالصُّنَّاعُ وَالأَقْيَانُ أَلْفٌ، وَجَمِيعُ الأَبْطَالِ أَهْلِ الْحَرْبِ، سَبَاهُمْ مَلِكُ بَابِلَ إِلَى بَابِلَ.
\par 17 وَمَلَّكَ مَلِكُ بَابِلَ مَتَّنِيَّا عَمَّهُ عِوَضاً عَنْهُ، وَغَيَّرَ اسْمَهُ إِلَى صِدْقِيَّا.
\par 18 كَانَ صِدْقِيَّا ابْنَ إِحْدَى وَعِشْرِينَ سَنَةً حِينَ مَلَكَ. وَمَلَكَ إِحْدَى عَشَرَةَ سَنَةً فِي أُورُشَلِيمَ، وَاسْمُ أُمِّهِ حَمِّيطَلُ بِنْتُ إِرْمِيَا مِنْ لِبْنَةَ.
\par 19 وَعَمِلَ الشَّرَّ فِي عَيْنَيِ الرَّبِّ حَسَبَ كُلِّ مَا عَمِلَ يَهُويَاقِيمُ.
\par 20 لأَنَّهُ لأَجْلِ غَضَبِ الرَّبِّ عَلَى أُورُشَلِيمَ وَعَلَى يَهُوذَا حَتَّى طَرَحَهُمْ مِنْ أَمَامِ وَجْهِهِ كَانَ أَنَّ صِدْقِيَّا تَمَرَّدَ عَلَى مَلِكِ بَابِلَ.

\chapter{25}

\par 1 وَفِي السَّنَةِ التَّاسِعَةِ لِمُلْكِهِ فِي الشَّهْرِ الْعَاشِرِ فِي عَاشِرِ الشَّهْرِ، جَاءَ نَبُوخَذْنَصَّرُ مَلِكُ بَابِلَ هُوَ وَكُلُّ جَيْشِهِ عَلَى أُورُشَلِيمَ وَنَزَلَ عَلَيْهَا، وَبَنُوا عَلَيْهَا أَبْرَاجاً حَوْلَهَا.
\par 2 وَدَخَلَتِ الْمَدِينَةُ تَحْتَ الْحِصَارِ إِلَى السَّنَةِ الْحَادِيَةَ عَشَرَةَ لِلْمَلِكِ صِدْقِيَّا.
\par 3 فِي تَاسِعِ الشَّهْرِ اشْتَدَّ الْجُوعُ فِي الْمَدِينَةِ وَلَمْ يَكُنْ خُبْزٌ لِشَعْبِ الأَرْضِ.
\par 4 فَثُغِرَتِ الْمَدِينَةُ وَهَرَبَ جَمِيعُ رِجَالِ الْقِتَالِ لَيْلاً مِنْ طَرِيقِ الْبَابِ بَيْنَ السُّورَيْنِ اللَّذَيْنِ نَحْوَ جَنَّةِ الْمَلِكِ. وَكَانَ الْكِلْدَانِيُّونَ حَوْلَ الْمَدِينَةِ مُسْتَدِيرِينَ. فَذَهَبُوا فِي طَرِيقِ الْبَرِّيَّةِ.
\par 5 فَتَبِعَتْ جُيُوشُ الْكِلْدَانِيِّينَ الْمَلِكَ فَأَدْرَكُوهُ فِي بَرِّيَّةِ أَرِيحَا، وَتَفَرَّقَتْ جَمِيعُ جُيُوشِهِ عَنْهُ.
\par 6 فَأَخَذُوا الْمَلِكَ وَأَصْعَدُوهُ إِلَى مَلِكِ بَابِلَ إِلَى رَبْلَةَ وَكَلَّمُوهُ بِالْقَضَاءِ عَلَيْهِ.
\par 7 وَقَتَلُوا بَنِي صِدْقِيَّا أَمَامَ عَيْنَيْهِ، وَقَلَعُوا عَيْنَيْ صِدْقِيَّا وَقَيَّدُوهُ بِسِلْسِلَتَيْنِ مِنْ نُحَاسٍ وَجَاءُوا بِهِ إِلَى بَابِلَ.
\par 8 وَفِي الشَّهْرِ الْخَامِسِ فِي سَابِعِ الشَّهْرِ، وَهِيَ السَّنَةُ التَّاسِعَةَ عَشَرَةَ لِلْمَلِكِ نَبُوخَذْنَصَّرَ مَلِكِ بَابِلَ، جَاءَ نَبُوزَرَادَانُ رَئِيسُ الشُّرَطِ عَبْدُ مَلِكِ بَابِلَ إِلَى أُورُشَلِيمَ،
\par 9 وَأَحْرَقَ بَيْتَ الرَّبِّ وَبَيْتَ الْمَلِكِ. وَكُلَّ بُيُوتِ أُورُشَلِيمَ وَكُلَّ بُيُوتِ الْعُظَمَاءِ أَحْرَقَهَا بِالنَّارِ
\par 10 وَجَمِيعُ أَسْوَارِ أُورُشَلِيمَ مُسْتَدِيراً هَدَمَهَا كُلُّ جُيُوشِ الْكِلْدَانِيِّينَ الَّذِينَ مَعَ رَئِيسِ الشُّرَطِ.
\par 11 وَبَقِيَّةُ الشَّعْبِ الَّذِينَ بَقُوا فِي الْمَدِينَةِ وَالْهَارِبُونَ الَّذِينَ هَرَبُوا إِلَى مَلِكِ بَابِلَ وَبَقِيَّةُ الْجُمْهُورِ سَبَاهُمْ نَبُوزَرَادَانُ رَئِيسُ الشُّرَطِ.
\par 12 وَلَكِنَّ رَئِيسَ الشُّرَطِ أَبْقَى مِنْ مَسَاكِينِ الأَرْضِ كَرَّامِينَ وَفَلاَّحِينَ.
\par 13 وَأَعْمِدَةَ النُّحَاسِ الَّتِي فِي بَيْتِ الرَّبِّ وَالْقَوَاعِدَ وَبَحْرَ النُّحَاسِ الَّذِي فِي بَيْتِ الرَّبِّ كَسَّرَهَا الْكِلْدَانِيُّونَ وَحَمَلُوا نُحَاسَهَا إِلَى بَابِلَ.
\par 14 وَالْقُدُورَ وَالرُّفُوشَ وَالْمَقَاصَّ وَالصُّحُونَ وَجَمِيعَ آنِيَةِ النُّحَاسِ الَّتِي كَانُوا يَخْدِمُونَ بِهَا أَخَذُوهَا
\par 15 وَالْمَجَامِرَ وَالْمَنَاضِحَ. مَا كَانَ مِنْ ذَهَبٍ فَالذَّهَبُ، وَمَا كَانَ مِنْ فِضَّةٍ فَالْفِضَّةُ، أَخَذَهَا رَئِيسُ الشُّرَطِ.
\par 16 وَالْعَمُودَانِ وَالْبَحْرُ الْوَاحِدُ وَالْقَوَاعِدُ الَّتِي عَمِلَهَا سُلَيْمَانُ لِبَيْتِ الرَّبِّ، لَمْ يَكُنْ وَزْنٌ لِنُحَاسِ كُلِّ هَذِهِ الأَدَوَاتِ.
\par 17 ثَمَانِي عَشَرَةَ ذِرَاعاً ارْتِفَاعُ الْعَمُودِ الْوَاحِدِ، وَعَلَيْهِ تَاجٌ مِنْ نُحَاسٍ وَارْتِفَاعُ التَّاجِ ثَلاَثُ أَذْرُعٍ وَالشَّبَكَةُ وَالرُّمَّانَاتُ الَّتِي عَلَى التَّاجِ مُسْتَدِيرَةً جَمِيعُهَا مِنْ نُحَاسٍ. وَكَانَ لِلْعَمُودِ الثَّانِي مِثْلُ هَذِهِ عَلَى الشَّبَكَةِ.
\par 18 وَأَخَذَ رَئِيسُ الشُّرَطِ سَرَايَا الْكَاهِنَ الرَّئِيسَ، وَصَفَنْيَا الْكَاهِنَ الثَّانِيَ، وَحَارِسِي الْبَابِ الثَّلاَثَةَ.
\par 19 وَمِنَ الْمَدِينَةِ أَخَذَ خَصِيّاً وَاحِداً كَانَ وَكِيلاً عَلَى رِجَالِ الْحَرْبِ، وَخَمْسَةَ رِجَالٍ مِنَ الَّذِينَ يَنْظُرُونَ وَجْهَ الْمَلِكِ الَّذِينَ وُجِدُوا فِي الْمَدِينَةِ، وَكَاتِبَ رَئِيسِ الْجُنْدِ الَّذِي كَانَ يَجْمَعُ شَعْبَ الأَرْضِ، وَسِتِّينَ رَجُلاً مِنْ شَعْبِ الأَرْضِ الْمَوْجُودِينَ فِي الْمَدِينَةِ
\par 20 وَأَخَذَهُمْ نَبُوزَرَادَانُ رَئِيسُ الشُّرَطِ وَسَارَ بِهِمْ إِلَى مَلِكِ بَابِلَ إِلَى رَبْلَةَ.
\par 21 فَضَرَبَهُمْ مَلِكُ بَابِلَ وَقَتَلَهُمْ فِي رَبْلَةَ فِي أَرْضِ حَمَاةَ. فَسُبِيَ يَهُوذَا مِنْ أَرْضِهِ.
\par 22 وَأَمَّا الشَّعْبُ الَّذِي بَقِيَ فِي أَرْضِ يَهُوذَا الَّذِينَ أَبْقَاهُمْ نَبُوخَذْنَصَّرُ مَلِكُ بَابِلَ، فَوَكَّلَ عَلَيْهِمْ جَدَلْيَا بْنَ أَخِيقَامَ بْنِ شَافَانَ.
\par 23 وَلَمَّا سَمِعَ جَمِيعُ رُؤَسَاءِ الْجُيُوشِ هُمْ وَرِجَالُهُمْ أَنَّ مَلِكَ بَابِلَ قَدْ وَكَّلَ جَدَلْيَا أَتُوا إِلَى جَدَلْيَا إِلَى الْمِصْفَاةِ، وَهُمْ إِسْمَاعِيلُ بْنُ نَثَنْيَا وَيُوحَنَانُ بْنُ قَارِيحَ وَسَرَايَا بْنُ تَنْحُومَثَ النَّطُوفَاتِيِّ وَيَازَنْيَا ابْنُ الْمَعْكِيِّ هُمْ وَرِجَالُهُمْ.
\par 24 وَحَلَفَ جَدَلْيَا لَهُمْ وَلِرِجَالِهِمْ وَقَالَ لَهُمْ: [لاَ تَخَافُوا مِنْ عَبِيدِ الْكِلْدَانِيِّينَ. اسْكُنُوا الأَرْضَ وَتَعَبَّدُوا لِمَلِكِ بَابِلَ فَيَكُونَ لَكُمْ خَيْرٌ].
\par 25 وَفِي الشَّهْرِ السَّابِعِ جَاءَ إِسْمَاعِيلُ بْنُ نَثَنْيَا بْنِ أَلِيشَمَعَ مِنَ النَّسْلِ الْمَلِكِيِّ وَعَشَرَةُ رِجَالٍ مَعَهُ وَضَرَبُوا جَدَلْيَا فَمَاتَ، وَأَيْضاً الْيَهُودُ وَالْكِلْدَانِيِّينَ الَّذِينَ مَعَهُ فِي الْمِصْفَاةِ.
\par 26 فَقَامَ جَمِيعُ الشَّعْبِ مِنَ الصَّغِيرِ إِلَى الْكَبِيرِ وَرُؤَسَاءُ الْجُيُوشِ وَجَاءُوا إِلَى مِصْرَ، لأَنَّهُمْ خَافُوا مِنَ الْكِلْدَانِيِّينَ.
\par 27 وَفِي السَّنَةِ السَّابِعَةِ وَالثَّلاَثِينَ لِسَبْيِ يَهُويَاكِينَ مَلِكِ يَهُوذَا، فِي الشَّهْرِ الثَّانِي عَشَرَ فِي السَّابِعِ وَالْعِشْرِينَ مِنَ الشَّهْرِ، رَفَعَ أَوِيلُ مَرُودَخُ مَلِكُ بَابِلَ فِي سَنَةِ تَمَلُّكِهِ رَأْسَ يَهُويَاكِينَ مَلِكِ يَهُوذَا مِنَ السِّجْنِ
\par 28 وَكَلَّمَهُ بِخَيْرٍ، وَجَعَلَ كُرْسِيَّهُ فَوْقَ كَرَاسِيِّ الْمُلُوكِ الَّذِينَ مَعَهُ فِي بَابِلَ.
\par 29 وَغَيَّرَ ثِيَابَ سِجْنِهِ. وَكَانَ يَأْكُلُ دَائِماً الْخُبْزَ أَمَامَهُ كُلَّ أَيَّامِ حَيَاتِهِ.
\par 30 وَوَظِيفَتُهُ وَظِيفَةٌ دَائِمَةٌ تُعْطَى لَهُ مِنْ عِنْدِ الْمَلِكِ أَمْرُ كُلِّ يَوْمٍ بِيَوْمِهِ كُلَّ أَيَّامِ حَيَاتِهِ.

\end{document}