\begin{document}

\title{صفنيا}


\chapter{1}

\par 1 كَلِمَةُ الرَّبِّ الَّتِي صَارَتْ إِلَى صَفَنْيَا بْنِ كُوشِي بْنِ جَدَلْيَا بْنِ أَمَرْيَا بْنِ حَزَقِيَّا, فِي أَيَّامِ يُوشِيَّا بْنِ آمُونَ مَلِكِ يَهُوذَا:
\par 2 [نَزْعاً أَنْزَعُ الْكُلَّ عَنْ وَجْهِ الأَرْضِ, يَقُولُ الرَّبُّ.
\par 3 أَنْزِعُ الإِنْسَانَ وَالْحَيَوَانَ. أَنْزِعُ طُيُورَ السَّمَاءِ وَسَمَكَ الْبَحْرِ, وَالْمَعَاثِرَ مَعَ الأَشْرَارِ, وَأَقْطَعُ الإِنْسَانَ عَنْ وَجْهِ الأَرْضِ, يَقُولُ الرَّبُّ.
\par 4 وَأَمُدُّ يَدِي عَلَى يَهُوذَا وَعَلَى كُلِّ سُكَّانِ أُورُشَلِيمَ, وَأَقْطَعُ مِنْ هَذَا الْمَكَانِ بَقِيَّةَ الْبَعْلِ, اسْمَ الْكَمَارِيمِ, مَعَ الْكَهَنَةِ,
\par 5 وَالسَّاجِدِينَ عَلَى السُّطُوحِ لِجُنْدِ السَّمَاءِ, وَالسَّاجِدِينَ الْحَالِفِينَ بِالرَّبِّ, وَالْحَالِفِينَ بِمَلْكُومَ,
\par 6 وَالْمُرْتَدِّينَ مِنْ وَرَاءِ الرَّبِّ, وَالَّذِينَ لَمْ يَطْلُبُوا الرَّبَّ وَلاَ سَأَلُوا عَنْهُ].
\par 7 اُسْكُتْ قُدَّامَ السَّيِّدِ الرَّبِّ, لأَنَّ يَوْمَ الرَّبِّ قَرِيبٌ. لأَنَّ الرَّبَّ قَدْ أَعَدَّ ذَبِيحَةً. قَدَّسَ مَدْعُوِّيهِ.
\par 8 [وَيَكُونُ فِي يَوْمِ ذَبِيحَةِ الرَّبِّ أَنِّي أُعَاقِبُ الرُّؤَسَاءَ وَبَنِي الْمَلِكِ وَجَمِيعَ اللاَّبِسِينَ لِبَاساً غَرِيباً.
\par 9 وَفِي ذَلِكَ الْيَوْمِ أُعَاقِبُ كُلَّ الَّذِينَ يَقْفِزُونَ مِنْ فَوْقِ الْعَتَبَةِ, الَّذِينَ يَمْلأُونَ بَيْتَ سَيِّدِهِمْ ظُلْماً وَغِشّاً.
\par 10 وَيَكُونُ فِي ذَلِكَ الْيَوْمِ, يَقُولُ الرَّبُّ, صَوْتُ صُرَاخٍ مِنْ بَابِ السَّمَكِ, وَوَلْوَلَةٌ مِنَ الْقِسْمِ الثَّانِي, وَكَسْرٌ عَظِيمٌ مِنَ الآكَامِ.
\par 11 وَلْوِلُوا يَا سُكَّانَ مَكْتِيشَ لأَنَّ كُلَّ شَعْبِ كَنْعَانَ بَادَ. انْقَطَعَ كُلُّ الْحَامِلِينَ الْفِضَّةَ.
\par 12 وَيَكُونُ فِي ذَلِكَ الْوَقْتِ أَنِّي أُفَتِّشُ أُورُشَلِيمَ بِالسُّرُجِ, وَأُعَاقِبُ الرِّجَالَ الْجَامِدِينَ عَلَى دُرْدِيِّهِمِ, الْقَائِلِينَ فِي قُلُوبِهِمْ: إِنَّ الرَّبَّ لاَ يُحْسِنُ وَلاَ يُسِيءُ.
\par 13 فَتَكُونُ ثَرْوَتُهُمْ غَنِيمَةً وَبُيُوتُهُمْ خَرَاباً, وَيَبْنُونَ بُيُوتاً وَلاَ يَسْكُنُونَهَا, وَيَغْرِسُونَ كُرُوماً وَلاَ يَشْرَبُونَ خَمْرَهَا].
\par 14 قَرِيبٌ يَوْمُ الرَّبِّ الْعَظِيمِ. قَرِيبٌ وَسَرِيعٌ جِدّاً. صَوْتُ يَوْمِ الرّبِّ. يَصْرُخُ حِينَئِذٍ الْجَبَّارُ مُرّاً.
\par 15 ذَلِكَ الْيَوْمُ يَوْمُ سَخَطٍ. يَوْمُ ضِيقٍ وَشِدَّةٍ. يَوْمُ خَرَابٍ وَدَمَارٍ. يَوْمُ ظَلاَمٍ وَقَتَامٍ. يَوْمُ سَحَابٍ وَضَبَابٍ.
\par 16 يَوْمُ بُوقٍ وَهُتَافٍ علَى الْمُدُنِ الْمُحَصَّنَةِ وَعَلَى الشُّرُفِ الرَّفِيعَةِ.
\par 17 [وَأُضَايِقُ النَّاسَ فَيَمْشُونَ كَالْعُمْيِ, لأَنَّهُمْ أَخْطَأُوا إِلَى الرَّبِّ, فَيُسْفَحُ دَمُهُمْ كَالتُّرَابِ وَلَحْمُهُمْ كَالْجِلَّةِ].
\par 18 لاَ فِضَّتُهُمْ وَلاَ ذَهَبُهُمْ يَسْتَطِيعُ إِنْقَاذَهُمْ في يَوْمِ غَضَبِ الرَّبِّ, بَلْ بِنَارِ غَيْرَتِهِ تُؤْكَلُ الأَرْضُ كُلُّهَا, لأَنَّهُ يَصْنَعُ فَنَاءً بَاغِتاً لِكُلِّ سُكَّانِ الأَرْضِ.

\chapter{2}

\par 1 تَجَمَّعِي وَاجْتَمِعِي يَا أَيَّتُهَا الأُمَّةُ غَيْرُ الْمُسْتَحِيَةِ.
\par 2 قَبْلَ وِلاَدَةِ الْقَضَاءِ. كَالْعُصَافَةِ عَبَرَ الْيَوْمُ. قَبْلَ أَنْ يَأْتِيَ عَلَيْكُمْ حُمُوُّ غَضَبِ الرَّبِّ. قَبْلَ أَنْ يَأْتِيَ عَلَيْكُمْ يَوْمُ سَخَطِ الرَّبِّ.
\par 3 أُطْلُبُوا الرَّبَّ يَا جَمِيعَ بَائِسِي الأَرْضِ الَّذِينَ فَعَلُوا حُكْمَهُ. اطْلُبُوا الْبِرَّ. اطْلُبُوا التَّوَاضُعَ. لَعَلَّكُمْ تُسْتَرُونَ فِي يَوْمِ سَخَطِ الرَّبِّ.
\par 4 لأَنَّ غَزَّةَ تَكُونُ مَتْرُوكَةً, وَأَشْقَلُونَ لِلْخَرَابِ. أَشْدُودُ عِنْدَ الظَّهِيرَةِ يَطْرُدُونَهَا, وَعَقْرُونُ تُسْتَأْصَلُ.
\par 5 وَيْلٌ لِسُكَّانِ سَاحِلِ الْبَحْرِ أُمَّةِ الْكَرِيتِيِّينَ. كَلِمَةُ الرَّبِّ عَلَيْكُمْ: [يَا كَنْعَانُ أَرْضَ الْفِلِسْطِينِيِّينَ, إِنِّي أَخْرِبُكِ بِلاَ سَاكِنٍ].
\par 6 وَيَكُونُ سَاحِلُ الْبَحْرِ مَرْعًى بِآبَارٍ لِلرُّعَاةِ وَحَظَائِرَ لِلْغَنَمِ.
\par 7 وَيَكُونُ السَّاحِلُ لِبَقِيَّةِ بَيْتِ يَهُوذَا. عَلَيْهِ يَرْعُونَ. فِي بُيُوتِ أَشْقَلُونَ عِنْدَ الْمَسَاءِ يَرْبُضُونَ, لأَنَّ الرَّبَّ إِلَهَهُمْ يَتَعَهَّدُهُمْ وَيَرُدُّ سَبْيَهُمْ.
\par 8 [قَدْ سَمِعْتُ تَعْيِيرَ مُوآبَ وَتَجَادِيفَ بَنِي عَمُّونَ الَّتِي بِهَا عَيَّرُوا شَعْبِي, وَتَعَظَّمُوا عَلَى تُخُمِهِمْ.
\par 9 فَلِذَلِكَ حَيٌّ أَنَا, يَقُولُ رَبُّ الْجُنُودِ إِلَهُ إِسْرَائِيلَ, إِنَّ مُوآبَ تَكُونُ كَسَدُومَ وَبَنِي عَمُّونَ كَعَمُورَةَ, مِلْكَ الْقَرِيصِ, وَحُفْرَةَ مِلْحٍ, وَخَرَاباً إِلَى الأَبَدِ. تَنْهَبُهُمْ بَقِيَّةُ شَعْبِي, وَبَقِيَّةُ أُمَّتِي تَمْتَلِكُهُمْ].
\par 10 هَذَا لَهُمْ عِوَضُ تَكَبُّرِهِمْ, لأَنَّهُمْ عَيَّرُوا وَتَعَظَّمُوا عَلَى شَعْبِ رَبِّ الْجُنُودِ.
\par 11 الرَّبُّ مُخِيفٌ إِلَيْهِمْ, لأَنَّهُ يُهْزِلُ جَمِيعَ آلِهَةِ الأَرْضِ, فَسَيَسْجُدُ لَهُ النَّاسُ, كُلُّ وَاحِدٍ مِنْ مَكَانِهِ, كُلُّ جَزَائِرِ الأُمَمِ.
\par 12 [وَأَنْتُمْ يَا أَيُّهَا الْكُوشِيُّونَ. قَتْلَى سَيْفِي هُمْ].
\par 13 وَيَمُدُّ يَدَهُ عَلَى الشِّمَالِ وَيُبِيدُ أَشُّورَ, وَيَجْعَلُ نِينَوَى خَرَاباً يَابِسَةً كَالْقَفْرِ.
\par 14 فَتَرْبُضُ فِي وَسَطِهَا الْقُطْعَانُ, كُلُّ طَوَائِفِ الْحَيَوَانِ. الْقُوقُ أَيْضاً وَالْقُنْفُذُ يَأْوِيَانِ إِلَى تِيجَانِ عُمُدِهَا. صَوْتٌ يَنْعِبُ فِي الْكُوى. خَرَابٌ عَلَى الأَعْتَابِ. لأَنَّهُ قَدْ تَعَرَّى أَرْزِيُّهَا.
\par 15 هَذِهِ هِيَ الْمَدِينَةُ الْمُبْتَهِجَةُ السَّاكِنَةُ مُطْمَئِنَّةً الْقَائِلَةُ فِي قَلْبِهَا: [أَنَا وَلَيْسَ غَيْرِي]. كَيْفَ صَارَتْ خَرَاباً, مَرْبِضاً لِلْحَيَوَانِ! كُلُّ عَابِرٍ بِهَا يَصْفِرُ وَيَهُزُّ يَدَهُ.

\chapter{3}

\par 1 وَيْلٌ لِلْمُتَمَرِّدَةِ الْمُنَجَّسَةِ, الْمَدِينَةِ الْجَائِرَةِ.
\par 2 لَمْ تَسْمَعِ الصَّوْتَ. لَمْ تَقْبَلِ التَّأْدِيبَ. لَمْ تَتَّكِلْ عَلَى الرَّبِّ. لَمْ تَتَقَرَّبْ إِلَى إِلَهِهَا.
\par 3 رُؤَسَاؤُهَا فِي وَسَطِهَا أُسُودٌ زَائِرَةٌ. قُضَاتُهَا ذِئَابُ مَسَاءٍ لاَ يُبْقُونَ شَيْئاً إِلَى الصَّبَاحِ.
\par 4 أَنْبِيَاؤُهَا مُتَفَاخِرُونَ, أَهْلُ غُدْرَاتٍ. كَهَنَتُهَا نَجَّسُوا الْقُدْسَ. خَالَفُوا الشَّرِيعَةَ.
\par 5 اَلرَّبُّ عَادِلٌ فِي وَسَطِهَا لاَ يَفْعَلُ ظُلْماً. غَدَاةً غَدَاةً يُبْرِزُ حُكْمَهُ إِلَى النُّورِ. لاَ يَتَعَذَّرُ. أَمَّا الظَّالِمُ فَلاَ يَعْرِفُ الْخِزْيَ.
\par 6 [قَطَعْتُ أُمَماً. خَرَّبْتُ شُرُفَاتِهِمْ. أَقْفَرْتُ أَسْوَاقَهُمْ بِلاَ عَابِرٍ. دُمِّرَتْ مُدُنُهُمْ بِلاَ إِنْسَانٍ, بِغَيْرِ سَاكِنٍ.
\par 7 فَقُلْتُ: إِنَّكِ لِتَخْشَيْنَنِي. تَقْبَلِينَ التَّأْدِيبَ. فَلاَ يَنْقَطِعُ مَسْكَنُهَا حَسَبَ كُلِّ مَا عَيَّنْتُهُ عَلَيْهَا. لَكِنْ بَكَّرُوا وَأَفْسَدُوا جَمِيعَ أَعْمَالِهَا.
\par 8 [لِذَلِكَ فَانْتَظِرُونِي يَقُولُ الرَّبُّ. إِلَى يَوْمِي أَقُومُ إِلَى السَّلْبِ, لأَنَّ حُكْمِي هُوَ بِجَمْعِ الأُمَمِ وَحَشْرِ الْمَمَالِكِ, لأَصُبَّ عَلَيْهِمْ سَخَطِي, كُلَّ حُمُوِّ غَضَبِي. لأَنَّهُ بِنَارِ غَيْرَتِي تُؤْكَلُ كُلُّ الأَرْضِ.
\par 9 لأَنِّي حِينَئِذٍ أُحَوِّلُ الشُّعُوبَ إِلَى شَفَةٍ نَقِيَّةٍ, لِيَدْعُوا كُلُّهُمْ بِاسْمِ الرَّبِّ, لِيَعْبُدُوهُ بِكَتِفٍ وَاحِدَةٍ.
\par 10 مِنْ عَبْرِ أَنْهَارِ كُوشٍ الْمُتَضَرِّعُونَ إِلَيَّ, مُتَبَدِّدِيَّ, يُقَدِّمُونَ تَقْدِمَتِي.
\par 11 فِي ذَلِكَ الْيَوْمِ لاَ تَخْزِينَ مِنْ كُلِّ أَعْمَالِكِ الَّتِي تَعَدَّيْتِ بِهَا عَلَيَّ. لأَنِّي حِينَئِذٍ أَنْزِعُ مِنْ وَسَطِكِ مُبْتَهِجِي كِبْرِيَائِكِ, وَلَنْ تَعُودِي بَعْدُ إِلَى التَّكَبُّرِ فِي جَبَلِ قُدْسِي.
\par 12 وَأُبْقِي فِي وَسَطِكِ شَعْباً بَائِساً وَمِسْكِيناً, فَيَتَوَكَّلُونَ عَلَى اسْمِ الرَّبِّ.
\par 13 بَقِيَّةُ إِسْرَائِيلَ لاَ يَفْعَلُونَ إِثْماً وَلاَ يَتَكَلَّمُونَ بِالْكَذِبِ وَلاَ يُوجَدُ فِي أَفْوَاهِهِمْ لِسَانُ غِشٍّ, لأَنَّهُمْ يَرْعُونَ وَيَرْبُضُونَ وَلاَ مُخِيفَ].
\par 14 تَرَنَّمِي يَا ابْنَةَ صِهْيَوْنَ. اهْتِفْ يَا إِسْرَائِيلُ. افْرَحِي وَابْتَهِجِي بِكُلِّ قَلْبِكِ يَا ابْنَةَ أُورُشَلِيمَ.
\par 15 قَدْ نَزَعَ الرَّبُّ الأَقْضِيَةَ عَلَيْكِ. أَزَالَ عَدُوَّكِ. مَلِكُ إِسْرَائِيلَ الرَّبُّ فِي وَسَطِكِ. لاَ تَنْظُرِينَ بَعْدُ شَرّاً.
\par 16 فِي ذَلِكَ الْيَوْمِ يُقَالُ لِأُورُشَلِيمَ: [لاَ تَخَافِي يَا صِهْيَوْنُ. لاَ تَرْتَخِ يَدَاكِ.
\par 17 الرَّبُّ إِلَهُكِ فِي وَسَطِكِ جَبَّارٌ يُخَلِّصُ. يَبْتَهِجُ بِكِ فَرَحاً. يَسْكُتُ فِي مَحَبَّتِهِ. يَبْتَهِجُ بِكِ بِتَرَنُّمٍ].
\par 18 [أَجْمَعُ الْمَحْزُونِينَ عَلَى الْمَوْسِمِ. كَانُوا مِنْكِ. حَامِلِينَ عَلَيْهَا الْعَارَ.
\par 19 هَئَنَذَا فِي ذَلِكَ الْيَوْمِ أُعَامِلُ كُلَّ مُذَلِّلِيكِ, وَأُخَلِّصُ الظَّالِعَةَ, وَأَجْمَعُ الْمَنْفِيَّةَ, وَأَجْعَلُهُمْ تَسْبِيحَةً وَاسْماً فِي كُلِّ أَرْضِ خِزْيِهِمْ,
\par 20 فِي الْوَقْتِ الَّذِي فِيهِ آتِي بِكُمْ وَفِي وَقْتِ جَمْعِي إِيَّاكُمْ. لأَنِّي أُصَيِّرُكُمُ اسْماً وَتَسْبِيحَةً فِي شُعُوبِ الأَرْضِ كُلِّهَا, حِينَ أَرُدُّ مَسْبِيِّيكُمْ قُدَّامَ أَعْيُنِكُمْ]. قَالَ الرَّبُّ.

\end{document}