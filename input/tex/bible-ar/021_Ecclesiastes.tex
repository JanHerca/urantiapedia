\begin{document}

\title{جامعة}


\chapter{1}

\par 1 كَلاَمُ الْجَامِعَةِ ابْنِ دَاوُدَ الْمَلِكِ فِي أُورُشَلِيمَ:
\par 2 «بَاطِلُ الأَبَاطِيلِ» قَالَ الْجَامِعَةُ. «بَاطِلُ الأَبَاطِيلِ الْكُلُّ بَاطِلٌ».
\par 3 مَا الْفَائِدَةُ لِلإِنْسَانِ مِنْ كُلِّ تَعَبِهِ الَّذِي يَتْعَبُهُ تَحْتَ الشَّمْسِ؟
\par 4 دَوْرٌ يَمْضِي وَدَوْرٌ يَجِيءُ وَالأَرْضُ قَائِمَةٌ إِلَى الأَبَدِ.
\par 5 وَالشَّمْسُ تُشْرِقُ وَالشَّمْسُ تَغْرُبُ وَتُسْرِعُ إِلَى مَوْضِعِهَا حَيْثُ تُشْرِقُ.
\par 6 اَلرِّيحُ تَذْهَبُ إِلَى الْجَنُوبِ وَتَدُورُ إِلَى الشِّمَالِ. تَذْهَبُ دَائِرَةً دَوَرَاناً وَإِلَى مَدَارَاتِهَا تَرْجِعُ الرِّيحُ.
\par 7 كُلُّ الأَنْهَارِ تَجْرِي إِلَى الْبَحْرِ وَالْبَحْرُ لَيْسَ بِمَلآنَ. إِلَى الْمَكَانِ الَّذِي جَرَتْ مِنْهُ الأَنْهَارُ إِلَى هُنَاكَ تَذْهَبُ رَاجِعَةً.
\par 8 كُلُّ الْكَلاَمِ يَقْصُرُ. لاَ يَسْتَطِيعُ الإِنْسَانُ أَنْ يُخْبِرَ بِالْكُلِّ. الْعَيْنُ لاَ تَشْبَعُ مِنَ النَّظَرِ وَالأُذُنُ لاَ تَمْتَلِئُ مِنَ السَّمْعِ.
\par 9 مَا كَانَ فَهُوَ مَا يَكُونُ وَالَّذِي صُنِعَ فَهُوَ الَّذِي يُصْنَعُ. فَلَيْسَ تَحْتَ الشَّمْسِ جَدِيدٌ.
\par 10 إِنْ وُجِدَ شَيْءٌ يُقَالُ عَنْهُ: «انْظُرْ. هَذَا جَدِيدٌ!» فَهُوَ مُنْذُ زَمَانٍ كَانَ فِي الدُّهُورِ الَّتِي كَانَتْ قَبْلَنَا.
\par 11 لَيْسَ ذِكْرٌ لِلأَوَّلِينَ. وَالآخِرُونَ أَيْضاً الَّذِينَ سَيَكُونُونَ لاَ يَكُونُ لَهُمْ ذِكْرٌ عِنْدَ الَّذِينَ يَكُونُونَ بَعْدَهُمْ.
\par 12 أَنَا الْجَامِعَةُ كُنْتُ مَلِكاً عَلَى إِسْرَائِيلَ فِي أُورُشَلِيمَ.
\par 13 وَوَجَّهْتُ قَلْبِي لِلسُّؤَالِ وَالتَّفْتِيشِ بِالْحِكْمَةِ عَنْ كُلِّ مَا عُمِلَ تَحْتَ السَّمَاوَاتِ. هُوَ عَنَاءٌ رَدِيءٌ جَعَلَهُ اللَّهُ لِبَنِي الْبَشَرِ لِيَعْنُوا فِيهِ.
\par 14 رَأَيْتُ كُلَّ الأَعْمَالِ الَّتِي عُمِلَتْ تَحْتَ الشَّمْسِ فَإِذَا الْكُلُّ بَاطِلٌ وَقَبْضُ الرِّيحِ.
\par 15 اَلأَعْوَجُ لاَ يُمْكِنُ أَنْ يُقَوَّمَ وَالنَّقْصُ لاَ يُمْكِنُ أَنْ يُجْبَرَ.
\par 16 أَنَا نَاجَيْتُ قَلْبِي قَائِلاً: «هَا أَنَا قَدْ عَظُمْتُ وَازْدَدْتُ حِكْمَةً أَكْثَرَ مِنْ كُلِّ مَنْ كَانَ قَبْلِي عَلَى أُورُشَلِيمَ وَقَدْ رَأَى قَلْبِي كَثِيراً مِنَ الْحِكْمَةِ وَالْمَعْرِفَةِ».
\par 17 وَوَجَّهْتُ قَلْبِي لِمَعْرِفَةِ الْحِكْمَةِ وَلِمَعْرِفَةِ الْحَمَاقَةِ وَالْجَهْلِ. فَعَرَفْتُ أَنَّ هَذَا أَيْضاً قَبْضُ الرِّيحِ.
\par 18 لأَنَّ فِي كَثْرَةِ الْحِكْمَةِ كَثْرَةُ الْغَمِّ وَالَّذِي يَزِيدُ عِلْماً يَزِيدُ حُزْناً.

\chapter{2}

\par 1 قُلْتُ أَنَا فِي قَلْبِي: «هَلُمَّ أَمْتَحِنُكَ بِالْفَرَحِ فَتَرَى خَيْراً». وَإِذَا هَذَا أَيْضاً بَاطِلٌ.
\par 2 لِلضَّحْكِ قُلْتُ: «مَجْنُونٌ» وَلِلْفَرَحِ: «مَاذَا يَفْعَلُ؟»
\par 3 اِفْتَكَرْتُ فِي قَلْبِي أَنْ أُعَلِّلَ جَسَدِي بِالْخَمْرِ وَقَلْبِي يَلْهَجُ بِالْحِكْمَةِ وَأَنْ آخُذَ بِالْحَمَاقَةِ حَتَّى أَرَى مَا هُوَ الْخَيْرُ لِبَنِي الْبَشَرِ حَتَّى يَفْعَلُوهُ تَحْتَ السَّمَاوَاتِ مُدَّةَ أَيَّامِ حَيَاتِهِمْ.
\par 4 فَعَظَّمْتُ عَمَلِي. بَنَيْتُ لِنَفْسِي بُيُوتاً غَرَسْتُ لِنَفْسِي كُرُوماً.
\par 5 عَمِلْتُ لِنَفْسِي جَنَّاتٍ وَفَرَادِيسَ وَغَرَسْتُ فِيهَا أَشْجَاراً مِنْ كُلِّ نَوْعِ ثَمَرٍ.
\par 6 عَمِلْتُ لِنَفْسِي بِرَكَ مِيَاهٍ لِتُسْقَى بِهَا الْمَغَارِسُ الْمُنْبِتَةُ الشَّجَرَ.
\par 7 قَنِيتُ عَبِيداً وَجَوَارِيَ وَكَانَ لِي وُلْدَانُ الْبَيْتِ. وَكَانَتْ لِي أَيْضاً قِنْيَةُ بَقَرٍ وَغَنَمٍ أَكْثَرَ مِنْ جَمِيعِ الَّذِينَ كَانُوا فِي أُورُشَلِيمَ قَبْلِي.
\par 8 جَمَعْتُ لِنَفْسِي أَيْضاً فِضَّةً وَذَهَباً وَخُصُوصِيَّاتِ الْمُلُوكِ وَالْبُلْدَانِ. اتَّخَذْتُ لِنَفْسِي مُغَنِّينَ وَمُغَنِّيَاتٍ وَتَنَعُّمَاتِ بَنِي الْبَشَرِ سَيِّدَةً وَسَيِّدَاتٍ.
\par 9 فَعَظُمْتُ وَازْدَدْتُ أَكْثَرَ مِنْ جَمِيعِ الَّذِينَ كَانُوا قَبْلِي فِي أُورُشَلِيمَ وَبَقِيَتْ أَيْضاً حِكْمَتِي مَعِي.
\par 10 وَمَهْمَا اشْتَهَتْهُ عَيْنَايَ لَمْ أُمْسِكْهُ عَنْهُمَا. لَمْ أَمْنَعْ قَلْبِي مِنْ كُلِّ فَرَحٍ لأَنَّ قَلْبِي فَرِحَ بِكُلِّ تَعَبِي. وَهَذَا كَانَ نَصِيبِي مِنْ كُلِّ تَعَبِي.
\par 11 ثُمَّ الْتَفَتُّ أَنَا إِلَى كُلِّ أَعْمَالِي الَّتِي عَمِلَتْهَا يَدَايَ وَإِلَى التَّعَبِ الَّذِي تَعِبْتُهُ فِي عَمَلِهِ فَإِذَا الْكُلُّ بَاطِلٌ وَقَبْضُ الرِّيحِ وَلاَ مَنْفَعَةَ تَحْتَ الشَّمْسِ!
\par 12 ثُمَّ الْتَفَتُّ لأَنْظُرَ الْحِكْمَةَ وَالْحَمَاقَةَ وَالْجَهْلَ. فَمَا الإِنْسَانُ الَّذِي يَأْتِي وَرَاءَ الْمَلِكِ الَّذِي قَدْ نَصَبُوهُ مُنْذُ زَمَانٍ؟
\par 13 فَرَأَيْتُ أَنَّ لِلْحِكْمَةِ مَنْفَعَةً أَكْثَرَ مِنَ الْجَهْلِ كَمَا أَنَّ لِلنُّورِ مَنْفَعَةً أَكْثَرَ مِنَ الظُّلْمَةِ.
\par 14 اَلْحَكِيمُ عَيْنَاهُ فِي رَأْسِهِ. أَمَّا الْجَاهِلُ فَيَسْلُكُ فِي الظَّلاَمِ. وَعَرَفْتُ أَنَا أَيْضاً أَنَّ حَادِثَةً وَاحِدَةً تَحْدُثُ لِكِلَيْهِمَا.
\par 15 فَقُلْتُ فِي قَلْبِي: «كَمَا يَحْدُثُ لِلْجَاهِلِ كَذَلِكَ يَحْدُثُ أَيْضاً لِي أَنَا. وَإِذْ ذَاكَ فَلِمَاذَا أَنَا أَوْفَرُ حِكْمَةً؟» فَقُلْتُ فِي قَلْبِي: «هَذَا أَيْضاً بَاطِلٌ!»
\par 16 لأَنَّهُ لَيْسَ ذِكْرٌ لِلْحَكِيمِ وَلاَ لِلْجَاهِلِ إِلَى الأَبَدِ. كَمَا مُنْذُ زَمَانٍ كَذَا الأَيَّامُ الآتِيَةُ: الْكُلُّ يُنْسَى. وَكَيْفَ يَمُوتُ الْحَكِيمُ؟ كَالْجَاهِلِ!
\par 17 فَكَرِهْتُ الْحَيَاةَ. لأَنَّهُ رَدِيءٌ عِنْدِي الْعَمَلُ الَّذِي عُمِلَ تَحْتَ الشَّمْسِ لأَنَّ الْكُلَّ بَاطِلٌ وَقَبْضُ الرِّيحِ.
\par 18 فَكَرِهْتُ كُلَّ تَعَبِي الَّذِي تَعِبْتُ فِيهِ تَحْتَ الشَّمْسِ حَيْثُ أَتْرُكُهُ لِلإِنْسَانِ الَّذِي يَكُونُ بَعْدِي.
\par 19 وَمَنْ يَعْلَمُ هَلْ يَكُونُ حَكِيماً أَوْ جَاهِلاً وَيَسْتَوْلِي عَلَى كُلِّ تَعَبِي الَّذِي تَعِبْتُ فِيهِ وَأَظْهَرْتُ فِيهِ حِكْمَتِي تَحْتَ الشَّمْسِ؟ هَذَا أَيْضاً بَاطِلٌ!
\par 20 فَتَحَوَّلْتُ لِكَيْ أَجْعَلَ قَلْبِي يَيْأَسُ مِنْ كُلِّ التَّعَبِ الَّذِي تَعِبْتُ فِيهِ تَحْتَ الشَّمْسِ.
\par 21 لأَنَّهُ قَدْ يَكُونُ إِنْسَانٌ تَعَبُهُ بِالْحِكْمَةِ وَالْمَعْرِفَةِ وَبِالْفَلاَحِ فَيَتْرُكُهُ نَصِيباً لإِنْسَانٍ لَمْ يَتْعَبْ فِيهِ. هَذَا أَيْضاً بَاطِلٌ وَشَرٌّ عَظِيمٌ.
\par 22 لأَنَّهُ مَاذَا لِلإِنْسَانِ مِنْ كُلِّ تَعَبِهِ وَمِنِ اجْتِهَادِ قَلْبِهِ الَّذِي تَعِبَ فِيهِ تَحْتَ الشَّمْسِ؟
\par 23 لأَنَّ كُلَّ أَيَّامِهِ أَحْزَانٌ وَعَمَلَهُ غَمٌّ. أَيْضاً بِاللَّيْلِ لاَ يَسْتَرِيحُ قَلْبُهُ. هَذَا أَيْضاً بَاطِلٌ هُوَ.
\par 24 لَيْسَ لِلإِنْسَانِ خَيْرٌ مِنْ أَنْ يَأْكُلَ وَيَشْرَبَ وَيُرِيَ نَفْسَهُ خَيْراً فِي تَعَبِهِ. رَأَيْتُ هَذَا أَيْضاً أَنَّهُ مِنْ يَدِ اللَّهِ.
\par 25 لأَنَّهُ مَنْ يَأْكُلُ وَمَنْ يَلْتَذُّ غَيْرِي؟
\par 26 لأَنَّهُ يُؤْتِي الإِنْسَانَ الصَّالِحَ قُدَّامَهُ حِكْمَةً وَمَعْرِفَةً وَفَرَحاً. أَمَّا الْخَاطِئُ فَيُعْطِيهِ شُغْلَ الْجَمْعِ وَالتَّكْوِيمِ لِيُعْطِيَ لِلصَّالِحِ قُدَّامَ اللَّهِ! هَذَا أَيْضاً بَاطِلٌ وَقَبْضُ الرِّيحِ.

\chapter{3}

\par 1 لِكُلِّ شَيْءٍ زَمَانٌ وَلِكُلِّ أَمْرٍ تَحْتَ السَّمَاوَاتِ وَقْتٌ.
\par 2 لِلْوِلاَدَةِ وَقْتٌ وَلِلْمَوْتِ وَقْتٌ. لِلْغَرْسِ وَقْتٌ وَلِقَلْعِ الْمَغْرُوسِ وَقْتٌ.
\par 3 لِلْقَتْلِ وَقْتٌ وَلِلشِّفَاءِ وَقْتٌ. لِلْهَدْمِ وَقْتٌ وَلِلْبِنَاءِ وَقْتٌ.
\par 4 لِلْبُكَاءِ وَقْتٌ وَلِلضِّحْكِ وَقْتٌ. لِلنَّوْحِ وَقْتٌ وَلِلرَّقْصِ وَقْتٌ.
\par 5 لِتَفْرِيقِ الْحِجَارَةِ وَقْتٌ وَلِجَمْعِ الْحِجَارَةِ وَقْتٌ. لِلْمُعَانَقَةِ وَقْتٌ وَلِلاِنْفِصَالِ عَنِ الْمُعَانَقَةِ وَقْتٌ.
\par 6 لِلْكَسْبِ وَقْتٌ وَلِلْخَسَارَةِ وَقْتٌ. لِلصِّيَانَةِ وَقْتٌ وَلِلطَّرْحِ وَقْتٌ.
\par 7 لِلتَّمْزِيقِ وَقْتٌ وَلِلتَّخْيِيطِ وَقْتٌ. لِلسُّكُوتِ وَقْتٌ وَلِلتَّكَلُّمِ وَقْتٌ.
\par 8 لِلْحُبِّ وَقْتٌ وَلِلْبُغْضَةِ وَقْتٌ. لِلْحَرْبِ وَقْتٌ وَلِلصُّلْحِ وَقْتٌ.
\par 9 فَأَيُّ مَنْفَعَةٍ لِمَنْ يَتْعَبُ مِمَّا يَتْعَبُ بِهِ!
\par 10 قَدْ رَأَيْتُ الشُّغْلَ الَّذِي أَعْطَاهُ اللَّهُ بَنِي الْبَشَرِ لِيَشْتَغِلُوا بِهِ.
\par 11 صَنَعَ الْكُلَّ حَسَناً فِي وَقْتِهِ وَأَيْضاً جَعَلَ الأَبَدِيَّةَ فِي قَلْبِهِمِ الَّتِي بِلاَهَا لاَ يُدْرِكُ الإِنْسَانُ الْعَمَلَ الَّذِي يَعْمَلُهُ اللَّهُ مِنَ الْبِدَايَةِ إِلَى النِّهَايَةِ.
\par 12 عَرَفْتُ أَنَّهُ لَيْسَ لَهُمْ خَيْرٌ إِلاَّ أَنْ يَفْرَحُوا وَيَفْعَلُوا خَيْراً فِي حَيَاتِهِمْ.
\par 13 وَأَيْضاً أَنْ يَأْكُلَ كُلُّ إِنْسَانٍ وَيَشْرَبَ وَيَرَى خَيْراً مِنْ كُلِّ تَعَبِهِ فَهُوَ عَطِيَّةُ اللَّهِ.
\par 14 قَدْ عَرَفْتُ أَنَّ كُلَّ مَا يَعْمَلُهُ اللَّهُ أَنَّهُ يَكُونُ إِلَى الأَبَدِ. لاَ شَيْءَ يُزَادُ عَلَيْهِ وَلاَ شَيْءَ يُنْقَصُ مِنْهُ وَأَنَّ اللَّهَ عَمِلَهُ حَتَّى يَخَافُوا أَمَامَهُ.
\par 15 مَا كَانَ فَمِنَ الْقِدَمِ هُوَ. وَمَا يَكُونُ فَمِنَ الْقِدَمِ قَدْ كَانَ. وَاللَّهُ يَطْلُبُ مَا قَدْ مَضَى.
\par 16 وَأَيْضاً رَأَيْتُ تَحْتَ الشَّمْسِ: مَوْضِعَ الْحَقِّ هُنَاكَ الظُّلْمُ وَمَوْضِعَ الْعَدْلِ هُنَاكَ الْجَوْرُ!
\par 17 فَقُلْتُ فِي قَلْبِي: «اللَّهُ يَدِينُ الصِّدِّيقَ وَالشِّرِّيرَ. لأَنَّ لِكُلِّ أَمْرٍ وَلِكُلِّ عَمَلٍ وَقْتاً هُنَاكَ».
\par 18 قُلْتُ فِي قَلْبِي: «مِنْ جِهَةِ أُمُورِ بَنِي الْبَشَرِ إِنَّ اللَّهَ يَمْتَحِنُهُمْ لِيُرِيَهُمْ أَنَّهُ كَمَا الْبَهِيمَةِ هَكَذَا هُمْ».
\par 19 لأَنَّ مَا يَحْدُثُ لِبَنِي الْبَشَرِ يَحْدُثُ لِلْبَهِيمَةِ وَحَادِثَةٌ وَاحِدَةٌ لَهُمْ. مَوْتُ هَذَا كَمَوْتِ ذَاكَ وَنَسَمَةٌ وَاحِدَةٌ لِلْكُلِّ. فَلَيْسَ لِلإِنْسَانِ مَزِيَّةٌ عَلَى الْبَهِيمَةِ لأَنَّ كِلَيْهِمَا بَاطِلٌ.
\par 20 يَذْهَبُ كِلاَهُمَا إِلَى مَكَانٍ وَاحِدٍ. كَانَ كِلاَهُمَا مِنَ التُّرَابِ وَإِلَى التُّرَابِ يَعُودُ كِلاَهُمَا.
\par 21 مَنْ يَعْلَمُ رُوحَ بَنِي الْبَشَرِ هَلْ هِيَ تَصْعَدُ إِلَى فَوْقٍ وَرُوحَ الْبَهِيمَةِ هَلْ هِيَ تَنْزِلُ إِلَى أَسْفَلَ إِلَى الأَرْضِ؟
\par 22 فَرَأَيْتُ أَنَّهُ لاَ شَيْءَ خَيْرٌ مِنْ أَنْ يَفْرَحَ الإِنْسَانُ بِأَعْمَالِهِ لأَنَّ ذَلِكَ نَصِيبَهُ. لأَنَّهُ مَنْ يَأْتِي بِهِ لِيَرَى مَا سَيَكُونُ بَعْدَهُ؟

\chapter{4}

\par 1 ثُمَّ رَجَعْتُ وَرَأَيْتُ كُلَّ الْمَظَالِمِ الَّتِي تُجْرَى تَحْتَ الشَّمْسِ فَهُوَذَا دُمُوعُ الْمَظْلُومِينَ وَلاَ مُعَزٍّ لَهُمْ وَمِنْ يَدِ ظَالِمِيهِمْ قَهْرٌ. أَمَّا هُمْ فَلاَ مُعَزٍّ لَهُمْ.
\par 2 فَغَبَطْتُ أَنَا الأَمْوَاتَ الَّذِينَ قَدْ مَاتُوا مُنْذُ زَمَانٍ أَكْثَرَ مِنَ الأَحْيَاءِ الَّذِينَ هُمْ عَائِشُونَ بَعْدُ.
\par 3 وَخَيْرٌ مِنْ كِلَيْهِمَا الَّذِي لَمْ يُولَدْ بَعْدُ الَّذِي لَمْ يَرَ الْعَمَلَ الرَّدِيءَ الَّذِي عُمِلَ تَحْتَ الشَّمْسِ!
\par 4 وَرَأَيْتُ كُلَّ التَّعَبِ وَكُلَّ فَلاَحِ عَمَلٍ أَنَّهُ حَسَدُ الإِنْسَانِ مِنْ قَرِيبِهِ! وَهَذَا أَيْضاً بَاطِلٌ وَقَبْضُ الرِّيحِ.
\par 5 اَلْكَسْلاَنُ يَأْكُلُ لَحْمَهُ وَهُوَ طَاوٍ يَدَيْهِ.
\par 6 حُفْنَةُ رَاحَةٍ خَيْرٌ مِنْ حُفْنَتَيْ تَعَبٍ وَقَبْضُ الرِّيحِ.
\par 7 ثُمَّ عُدْتُ وَرَأَيْتُ بَاطِلاً تَحْتَ الشَّمْسِ:
\par 8 يُوجَدُ وَاحِدٌ وَلاَ ثَانِيَ لَهُ وَلَيْسَ لَهُ ابْنٌ وَلاَ أَخٌ وَلاَ نِهَايَةَ لِكُلِّ تَعَبِهِ وَلاَ تَشْبَعُ عَيْنُهُ مِنَ الْغِنَى. فَلِمَنْ أَتْعَبُ أَنَا وَأُحَرِّمُ نَفْسِي الْخَيْرَ؟ هَذَا أَيْضاً بَاطِلٌ وَأَمْرٌ رَدِيءٌ هُوَ.
\par 9 اِثْنَانِ خَيْرٌ مِنْ وَاحِدٍ لأَنَّ لَهُمَا أُجْرَةً لِتَعَبِهِمَا صَالِحَةً.
\par 10 لأَنَّهُ إِنْ وَقَعَ أَحَدُهُمَا يُقِيمُهُ رَفِيقُهُ. وَوَيْلٌ لِمَنْ هُوَ وَحْدَهُ إِنْ وَقَعَ إِذْ لَيْسَ ثَانٍ لِيُقِيمَهُ.
\par 11 أَيْضاً إِنِ اضْطَجَعَ اثْنَانِ يَكُونُ لَهُمَا دِفْءٌ. أَمَّا الْوَحْدُ فَكَيْفَ يَدْفَأُ؟
\par 12 وَإِنْ غَلَبَ أَحَدٌ عَلَى الْوَاحِدِ يَقِفُ مُقَابِلَهُ الاِثْنَانِ وَالْخَيْطُ الْمَثْلُوثُ لاَ يَنْقَطِعُ سَرِيعاً.
\par 13 وَلَدٌ فَقِيرٌ وَحَكِيمٌ خَيْرٌ مِنْ مَلِكٍ شَيْخٍ جَاهِلٍ الَّذِي لاَ يَعْرِفُ أَنْ يُحَذَّرَ بَعْدُ.
\par 14 لأَنَّهُ مِنَ السِّجْنِ خَرَجَ إِلَى الْمُلْكِ وَالْمَوْلُودُ مَلِكاً قَدْ يَفْتَقِرُ.
\par 15 رَأَيْتُ كُلَّ الأَحْيَاءِ السَّائِرِينَ تَحْتَ الشَّمْسِ مَعَ الْوَلَدِ الثَّانِي الَّذِي يَقُومُ عِوَضاً عَنْهُ.
\par 16 لاَ نِهَايَةَ لِكُلِّ الشَّعْبِ لِكُلِّ الَّذِينَ كَانَ أَمَامَهُمْ. أَيْضاً الْمُتَأَخِّرُونَ لاَ يَفْرَحُونَ بِهِ. فَهَذَا أَيْضاً بَاطِلٌ وَقَبْضُ الرِّيحِ!

\chapter{5}

\par 1 اِحْفَظْ قَدَمَكَ حِينَ تَذْهَبُ إِلَى بَيْتِ اللَّهِ فَالاِسْتِمَاعُ أَقْرَبُ مِنْ تَقْدِيمِ ذَبِيحَةِ الْجُهَّالِ لأَنَّهُمْ لاَ يُبَالُونَ بِفَعْلِ الشَّرِّ.
\par 2 لاَ تَسْتَعْجِلْ فَمَكَ وَلاَ يُسْرِعْ قَلْبُكَ إِلَى نُطْقِ كَلاَمٍ قُدَّامَ اللَّهِ. لأَنَّ اللَّهَ فِي السَّمَاوَاتِ وَأَنْتَ عَلَى الأَرْضِ فَلِذَلِكَ لِتَكُنْ كَلِمَاتُكَ قَلِيلَةً.
\par 3 لأَنَّ الْحُلْمَ يَأْتِي مِنْ كَثْرَةِ الشُّغْلِ وَقَوْلَ الْجَهْلِ مِنْ كَثْرَةِ الْكَلاَمِ.
\par 4 إِذَا نَذَرْتَ نَذْراً لِلَّهِ فَلاَ تَتَأَخَّرْ عَنِ الْوَفَاءِ بِهِ. لأَنَّهُ لاَ يُسَرُّ بِالْجُهَّالِ. فَأَوْفِ بِمَا نَذَرْتَهُ.
\par 5 أَنْ لاَ تَنْذُرُ خَيْرٌ مِنْ أَنْ تَنْذُرَ وَلاَ تَفِيَ.
\par 6 لاَ تَدَعْ فَمَكَ يَجْعَلُ جَسَدَكَ يُخْطِئُ. وَلاَ تَقُلْ قُدَّامَ الْمَلاَكِ: «إِنَّهُ سَهْوٌ». لِمَاذَا يَغْضَبُ اللَّهُ عَلَى قَوْلِكَ وَيُفْسِدُ عَمَلَ يَدَيْكَ؟
\par 7 لأَنَّ ذَلِكَ مِنْ كَثْرَةِ الأَحْلاَمِ وَالأَبَاطِيلِ وَكَثْرَةِ الْكَلاَمِ. وَلَكِنِ اخْشَ اللَّهَ.
\par 8 إِنْ رَأَيْتَ ظُلْمَ الْفَقِيرِ وَنَزْعَ الْحَقِّ وَالْعَدْلِ فِي الْبِلاَدِ فَلاَ تَرْتَعْ مِنَ الأَمْرِ لأَنَّ فَوْقَ الْعَالِي عَالِياً يُلاَحِظُ وَالأَعْلَى فَوْقَهُمَا.
\par 9 وَمَنْفَعَةُ الأَرْضِ لِلْكُلِّ. الْمَلِكُ مَخْدُومٌ مِنَ الْحَقْلِ.
\par 10 مَنْ يُحِبُّ الْفِضَّةَ لاَ يَشْبَعُ مِنَ الْفِضَّةِ وَمَنْ يُحِبُّ الثَّرْوَةَ لاَ يَشْبَعُ مِنْ دَخْلٍ. هَذَا أَيْضاً بَاطِلٌ.
\par 11 إِذَا كَثُرَتِ الْخَيْرَاتُ كَثُرَ الَّذِينَ يَأْكُلُونَهَا وَأَيُّ مَنْفَعَةٍ لِصَاحِبِهَا إِلاَّ رُؤْيَتَهَا بِعَيْنَيْهِ؟
\par 12 نَوْمُ الْمُشْتَغِلِ حُلْوٌ إِنْ أَكَلَ قَلِيلاً أَوْ كَثِيراً وَوَفْرُ الْغَنِيِّ لاَ يُرِيحُهُ حَتَّى يَنَامَ.
\par 13 يُوجَدُ شَرٌّ خَبِيثٌ رَأَيْتُهُ تَحْتَ الشَّمْسِ: ثَرْوَةٌ مَصُونَةٌ لِصَاحِبِهَا لِضَرَرِهِ.
\par 14 فَهَلَكَتْ تِلْكَ الثَّرْوَةُ بِأَمْرٍ سَيِّئٍ ثُمَّ وَلَدَ ابْناً وَمَا بِيَدِهِ شَيْءٌ.
\par 15 كَمَا خَرَجَ مِنْ بَطْنِ أُمِّهِ عُرْيَاناً يَرْجِعُ ذَاهِباً كَمَا جَاءَ وَلاَ يَأْخُذُ شَيْئاً مِنْ تَعَبِهِ فَيَذْهَبُ بِهِ فِي يَدِهِ.
\par 16 وَهَذَا أَيْضاً مَصِيبَةٌ رَدِيئَةٌ. فِي كُلِّ شَيْءٍ كَمَا جَاءَ هَكَذَا يَذْهَبُ فَأَيَّةُ مَنْفَعَةٍ لَهُ لِلَّذِي تَعِبَ لِلرِّيحِ؟
\par 17 أَيْضاً يَأْكُلُ كُلَّ أَيَّامِهِ فِي الظَّلاَمِ وَيَغْتَمُّ كَثِيراً مَعَ حُزْنٍ وَغَيْظٍ.
\par 18 هُوَذَا الَّذِي رَأَيْتُهُ أَنَا خَيْراً الَّذِي هُوَ حَسَنٌ: أَنْ يَأْكُلَ الإِنْسَانُ وَيَشْرَبَ وَيَرَى خَيْراً مِنْ كُلِّ تَعَبِهِ الَّذِي يَتْعَبُ فِيهِ تَحْتَ الشَّمْسِ مُدَّةَ أَيَّامِ حَيَاتِهِ الَّتِي أَعْطَاهُ اللَّهُ إِيَّاهَا لأَنَّهُ نَصِيبُهُ.
\par 19 أَيْضاً كُلُّ إِنْسَانٍ أَعْطَاهُ اللَّهُ غِنًى وَمَالاً وَسَلَّطَهُ عَلَيْهِ حَتَّى يَأْكُلَ مِنْهُ وَيَأْخُذَ نَصِيبَهُ وَيَفْرَحَ بِتَعَبِهِ فَهَذَا هُوَ عَطِيَّةُ اللَّهِ.
\par 20 لأَنَّهُ لاَ يَذْكُرُ أَيَّامَ حَيَاتِهِ كَثِيراً لأَنَّ اللَّهَ مُلْهِيهِ بِفَرَحِ قَلْبِهِ.

\chapter{6}

\par 1 يُوجَدُ شَرٌّ قَدْ رَأَيْتُهُ تَحْتَ الشَّمْسِ وَهُوَ كَثِيرٌ بَيْنَ النَّاسِ:
\par 2 رَجُلٌ أَعْطَاهُ اللَّهُ غِنًى وَمَالاً وَكَرَامَةً وَلَيْسَ لِنَفْسِهِ عَوَزٌ مِنْ كُلِّ مَا يَشْتَهِيهِ وَلَمْ يُعْطِهِ اللَّهُ اسْتِطَاعَةً عَلَى أَنْ يَأْكُلَ مِنْهُ بَلْ يَأْكُلُهُ إِنْسَانٌ غَرِيبٌ. هَذَا بَاطِلٌ وَمُصِيبَةٌ رَدِيئَةٌ هُوَ.
\par 3 إِنْ وَلَدَ إِنْسَانٌ مِئَةً وَعَاشَ سِنِينَ كَثِيرَةً حَتَّى تَصِيرَ أَيَّامُ سِنِيهِ كَثِيرَةً وَلَمْ تَشْبَعْ نَفْسُهُ مِنَ الْخَيْرِ وَلَيْسَ لَهُ أَيْضاً دَفْنٌ فَأَقُولُ: «إِنَّ السِّقْطَ خَيْرٌ مِنْهُ».
\par 4 لأَنَّهُ فِي الْبَاطِلِ يَجِيءُ وَفِي الظَّلاَمِ يَذْهَبُ وَاسْمُهُ يُغَطَّى بِالظَّلاَمِ.
\par 5 وَأَيْضاً لَمْ يَرَ الشَّمْسَ وَلَمْ يَعْلَمْ. فَهَذَا لَهُ رَاحَةٌ أَكْثَرُ مِنْ ذَاكَ.
\par 6 وَإِنْ عَاشَ أَلْفَ سَنَةٍ مُضَاعَفَةً وَلَمْ يَرَ خَيْراً أَلَيْسَ إِلَى مَوْضِعٍ وَاحِدٍ يَذْهَبُ الْجَمِيعُ؟
\par 7 كُلُّ تَعَبِ الإِنْسَانِ لِفَمِهِ وَمَعَ ذَلِكَ فَالنَّفْسُ لاَ تَمْتَلِئُ.
\par 8 لأَنَّهُ مَاذَا يَبْقَى لِلْحَكِيمِ أَكْثَرَ مِنَ الْجَاهِلِ. مَاذَا لِلْفَقِيرِ الْعَارِفِ السُّلُوكَ أَمَامَ الأَحْيَاءِ؟
\par 9 رُؤْيَةُ الْعُيُونِ خَيْرٌ مِنْ شَهْوَةِ النَّفْسِ. هَذَا أَيْضاً بَاطِلٌ وَقَبْضُ الرِّيحِ.
\par 10 الَّذِي كَانَ فَقَدْ دُعِيَ بِاسْمٍ مُنْذُ زَمَانٍ وَهُوَ مَعْرُوفٌ أَنَّهُ إِنْسَانٌ وَلاَ يَسْتَطِيعُ أَنْ يُخَاصِمَ مَنْ هُوَ أَقْوَى مِنْهُ.
\par 11 لأَنَّهُ تُوجَدُ أُمُورٌ كَثِيرَةٌ تَزِيدُ الْبَاطِلَ. فَأَيُّ فَضْلٍ لِلإِنْسَانِ؟
\par 12 لأَنَّهُ مَنْ يَعْرِفُ مَا هُوَ خَيْرٌ لِلإِنْسَانِ فِي الْحَيَاةِ مُدَّةَ أَيَّامِ حَيَاةِ بَاطِلِهِ الَّتِي يَقْضِيهَا كَالظِّلِّ؟ لأَنَّهُ مَنْ يُخْبِرُ الإِنْسَانَ بِمَا يَكُونُ بَعْدَهُ تَحْتَ الشَّمْسِ؟

\chapter{7}

\par 1 اَلصِّيتُ خَيْرٌ مِنَ الدُّهْنِ الطَّيِّبِ وَيَوْمُ الْمَمَاتِ خَيْرٌ مِنْ يَوْمِ الْوِلاَدَةِ.
\par 2 اَلذِّهَابُ إِلَى بَيْتِ النَّوْحِ خَيْرٌ مِنَ الذِّهَابِ إِلَى بَيْتِ الْوَلِيمَةِ لأَنَّ ذَاكَ نِهَايَةُ كُلِّ إِنْسَانٍ وَالْحَيُّ يَضَعُهُ فِي قَلْبِهِ.
\par 3 اَلْحُزْنُ خَيْرٌ مِنَ الضَّحِكِ لأَنَّهُ بِكَآبَةِ الْوَجْهِ يُصْلَحُ الْقَلْبُ.
\par 4 قَلْبُ الْحُكَمَاءِ فِي بَيْتِ النَّوْحِ وَقَلْبُ الْجُهَّالِ فِي بَيْتِ الْفَرَحِ.
\par 5 سَمْعُ الاِنْتِهَارِ مِنَ الْحَكِيمِ خَيْرٌ لِلإِنْسَانِ مِنْ سَمْعِ غِنَاءِ الْجُهَّالِ
\par 6 لأَنَّهُ كَصَوْتِ الشَّوْكِ تَحْتَ الْقِدْرِ هَكَذَا ضِحْكُ الْجُهَّالِ. هَذَا أَيْضاً بَاطِلٌ.
\par 7 لأَنَّ الظُّلْمَ يُحَمِّقُ الْحَكِيمَ وَالْعَطِيَّةَ تُفْسِدُ الْقَلْبَ.
\par 8 نِهَايَةُ أَمْرٍ خَيْرٌ مِنْ بَدَايَتِهِ. طُولُ الرُّوحِ خَيْرٌ مِنْ تَكَبُّرِ الرُّوحِ.
\par 9 لاَ تُسْرِعْ بِرُوحِكَ إِلَى الْغَضَبِ لأَنَّ الْغَضَبَ يَسْتَقِرُّ فِي حِضْنِ الْجُهَّالِ.
\par 10 لاَ تَقُلْ: «لِمَاذَا كَانَتِ الأَيَّامُ الأُولَى خَيْراً مِنْ هَذِهِ؟» لأَنَّهُ لَيْسَ عَنْ حِكْمَةٍ تَسْأَلُ عَنْ هَذَا.
\par 11 اَلْحِكْمَةُ صَالِحَةٌ مِثْلُ الْمِيرَاثِ بَلْ أَفْضَلُ لِنَاظِرِي الشَّمْسِ.
\par 12 لأَنَّ الَّذِي فِي ظِلِّ الْحِكْمَةِ هُوَ فِي ظِلِّ الْفِضَّةِ وَفَضْلُ الْمَعْرِفَةِ هُوَ أَنَّ الْحِكْمَةَ تُحْيِي أَصْحَابَهَا.
\par 13 اُنْظُرْ عَمَلَ اللَّهِ لأَنَّهُ مَنْ يَقْدِرُ عَلَى تَقْوِيمِ مَا قَدْ عَوَّجَهُ؟
\par 14 فِي يَوْمِ الْخَيْرِ كُنْ بِخَيْرٍ وَفِي يَوْمِ الشَّرِّ اعْتَبِرْ. إِنَّ اللَّهَ جَعَلَ هَذَا مَعَ ذَاكَ لِكَيْلاَ يَجِدَ الإِنْسَانُ شَيْئاً بَعْدَهُ.
\par 15 قَدْ رَأَيْتُ الْكُلَّ فِي أَيَّامِ بُطْلِي. قَدْ يَكُونُ بَارٌّ يَبِيدُ فِي بِرِّهِ وَقَدْ يَكُونُ شِرِّيرٌ يَطُولُ فِي شَرِّهِ.
\par 16 لاَ تَكُنْ بَارّاً كَثِيراً وَلاَ تَكُنْ حَكِيماً بِزِيَادَةٍ. لِمَاذَا تَخْرِبُ نَفْسَكَ؟
\par 17 لاَ تَكُنْ شِرِّيراً كَثِيراً وَلاَ تَكُنْ جَاهِلاً. لِمَاذَا تَمُوتُ فِي غَيْرِ وَقْتِكَ؟
\par 18 حَسَنٌ أَنْ تَتَمَسَّكَ بِهَذَا وَأَيْضاً أَنْ لاَ تَرْخِيَ يَدَكَ عَنْ ذَاكَ لأَنَّ مُتَّقِيَ اللَّهِ يَخْرُجُ مِنْهُمَا كِلَيْهِمَا.
\par 19 اَلْحِكْمَةُ تُقَوِّي الْحَكِيمَ أَكْثَرَ مِنْ عَشَرَةِ مُسَلَِّطِينَ الَّذِينَ هُمْ فِي الْمَدِينَةِ.
\par 20 لأَنَّهُ لاَ إِنْسَانٌ صِدِّيقٌ فِي الأَرْضِ يَعْمَلُ صَلاَحاً وَلاَ يُخْطِئُ.
\par 21 أَيْضاً لاَ تَضَعْ قَلْبَكَ عَلَى كُلِّ الْكَلاَمِ الَّذِي يُقَالُ لِئَلاَّ تَسْمَعَ عَبْدَكَ يَسِبُّكَ.
\par 22 لأَنَّ قَلْبَكَ أَيْضاً يَعْلَمُ أَنَّكَ أَنْتَ كَذَلِكَ مِرَاراً كَثِيرَةً سَبَبْتَ آخَرِينَ.
\par 23 كُلُّ هَذَا امْتَحَنْتُهُ بِالْحِكْمَةِ. قُلْتُ: «أَكُونُ حَكِيماً». أَمَّا هِيَ فَبَعِيدَةٌ عَنِّي.
\par 24 بَعِيدٌ مَا كَانَ بَعِيداً وَالْعَمِيقُ الْعَمِيقُ مَنْ يَجِدُهُ؟
\par 25 دُرْتُ أَنَا وَقَلْبِي لأَعْلَمَ وَلأَبْحَثَ وَلأَطْلُبَ حِكْمَةً وَعَقْلاً وَلأَعْرِفَ الشَّرَّ أَنَّهُ جَهَالَةٌ وَالْحَمَاقَةَ أَنَّهَا جُنُونٌ.
\par 26 فَوَجَدْتُ أَمَرَّ مِنَ الْمَوْتِ: الْمَرْأَةَ الَّتِي هِيَ شِبَاكٌ وَقَلْبُهَا أَشْرَاكٌ وَيَدَاهَا قُيُودٌ. الصَّالِحُ قُدَّامَ اللَّهِ يَنْجُو مِنْهَا. أَمَّا الْخَاطِئُ فَيُؤْخَذُ بِهَا.
\par 27 «اُنْظُرْ. هَذَا وَجَدْتُهُ» قَالَ الْجَامِعَةُ: «وَاحِدَةً فَوَاحِدَةً لأَجِدَ النَّتِيجَةَ
\par 28 الَّتِي لَمْ تَزَلْ نَفْسِي تَطْلُبُهَا فَلَمْ أَجِدْهَا. رَجُلاً وَاحِداً بَيْنَ أَلْفٍ وَجَدْتُ. أَمَّا امْرَأَةً فَبَيْنَ كُلِّ أُولَئِكَ لَمْ أَجِدْ!
\par 29 اُنْظُرْ. هَذَا وَجَدْتُ فَقَطْ: أَنَّ اللَّهَ صَنَعَ الإِنْسَانَ مُسْتَقِيماً أَمَّا هُمْ فَطَلَبُوا اخْتِرَاعَاتٍ كَثِيرَةً».

\chapter{8}

\par 1 مَنْ كَالْحَكِيمِ وَمَنْ يَفْهَمُ تَفْسِيرَ أَمْرٍ؟ حِكْمَةُ الإِنْسَانِ تُنِيرُ وَجْهَهُ وَصَلاَبَةُ وَجْهِهِ تَتَغَيَّرُ.
\par 2 أَنَا أَقُولُ: «احْفَظْ أَمْرَ الْمَلِكِ وَذَاكَ بِسَبَبِ يَمِينِ اللَّهِ.
\par 3 لاَ تَعْجَلْ إِلَى الذَّهَابِ مِنْ وَجْهِهِ. لاَ تَقِفْ فِي أَمْرٍ شَاقٍّ لأَنَّهُ يَفْعَلُ كُلَّ مَا شَاءَ».
\par 4 حَيْثُ تَكُونُ كَلِمَةُ الْمَلِكِ فَهُنَاكَ سُلْطَانٌ. وَمَنْ يَقُولُ لَهُ: «مَاذَا تَفْعَلُ؟»
\par 5 حَافِظُ الْوَصِيَّةِ لاَ يَشْعُرُ بِأَمْرٍ شَاقٍّ وَقَلْبُ الْحَكِيمِ يَعْرِفُ الْوَقْتَ وَالْحُكْمَ.
\par 6 لأَنَّ لِكُلِّ أَمْرٍ وَقْتاً وَحُكْماً. لأَنَّ شَرَّ الإِنْسَانِ عَظِيمٌ عَلَيْهِ
\par 7 لأَنَّهُ لاَ يَعْلَمُ مَا سَيَكُونُ. لأَنَّهُ مَنْ يُخْبِرُهُ كَيْفَ يَكُونُ؟
\par 8 لَيْسَ لإِنْسَانٍ سُلْطَانٌ عَلَى الرُّوحِ لِيُمْسِكَ الرُّوحَ وَلاَ سُلْطَانٌ عَلَى يَوْمِ الْمَوْتِ وَلاَ تَخْلِيَةٌ فِي الْحَرْبِ وَلاَ يُنَجِّي الشَّرُّ أَصْحَابَهُ.
\par 9 كُلُّ هَذَا رَأَيْتُهُ إِذْ وَجَّهْتُ قَلْبِي لِكُلِّ عَمَلٍ عُمِلَ تَحْتَ الشَّمْسِ وَقْتَمَا يَتَسَلَّطُ إِنْسَانٌ عَلَى إِنْسَانٍ لِضَرَرِ نَفْسِهِ.
\par 10 وَهَكَذَا رَأَيْتُ أَشْرَاراً يُدْفَنُونَ وَضُمُّوا وَالَّذِينَ عَمِلُوا بِالْحَقِّ ذَهَبُوا مِنْ مَكَانِ الْقُدْسِ وَنُسُوا فِي الْمَدِينَةِ. هَذَا أَيْضاً بَاطِلٌ.
\par 11 لأَنَّ الْقَضَاءَ عَلَى الْعَمَلِ الرَّدِيءِ لاَ يُجْرَى سَرِيعاً فَلِذَلِكَ قَدِ امْتَلَأَ قَلْبُ بَنِي الْبَشَرِ فِيهِمْ لِفَعْلِ الشَّرِّ.
\par 12 اَلْخَاطِئُ وَإِنْ عَمِلَ شَرّاً مِئَةَ مَرَّةٍ وَطَالَتْ أَيَّامُهُ إِلاَّ أَنِّي أَعْلَمُ أَنَّهُ يَكُونُ خَيْرٌ لِلْمُتَّقِينَ اللَّهَ الَّذِينَ يَخَافُونَ قُدَّامَهُ.
\par 13 وَلاَ يَكُونُ خَيْرٌ لِلشِّرِّيرِ وَكَالظِّلِّ لاَ يُطِيلُ أَيَّامَهُ لأَنَّهُ لاَ يَخْشَى قُدَّامَ اللَّهِ.
\par 14 يُوجَدُ بَاطِلٌ يُجْرَى عَلَى الأَرْضِ: أَنْ يُوجَدَ صِدِّيقُونَ يُصِيبُهُمْ مِثْلُ عَمَلِ الأَشْرَارِ وَيُوجَدُ أَشْرَارٌ يُصِيبُهُمْ مِثْلُ عَمَلِ الصِّدِّيقِينَ. فَقُلْتُ: «إِنَّ هَذَا أَيْضاً بَاطِلٌ».
\par 15 فَمَدَحْتُ الْفَرَحَ لأَنَّهُ لَيْسَ لِلإِنْسَانِ خَيْرٌ تَحْتَ الشَّمْسِ إِلاَّ أَنْ يَأْكُلَ وَيَشْرَبَ وَيَفْرَحَ وَهَذَا يَبْقَى لَهُ فِي تَعَبِهِ مُدَّةَ أَيَّامِ حَيَاتِهِ الَّتِي يُعْطِيهِ اللَّهُ إِيَّاهَا تَحْتَ الشَّمْسِ.
\par 16 لَمَّا وَجَّهْتُ قَلْبِي لأَعْرِفَ الْحِكْمَةَ وَأَنْظُرَ الْعَمَلَ الَّذِي عُمِلَ عَلَى الأَرْضِ وَأَنَّهُ نَهَاراً وَلَيْلاً لاَ يَرَى النَّوْمَ بِعَيْنَيْهِ
\par 17 رَأَيْتُ كُلَّ عَمَلِ اللَّهِ أَنَّ الإِنْسَانَ لاَ يَسْتَطِيعُ أَنْ يَجِدَ الْعَمَلَ الَّذِي عُمِلَ تَحْتَ الشَّمْسِ. مَهْمَا تَعِبَ الإِنْسَانُ فِي الطَّلَبِ فَلاَ يَجِدُهُ وَالْحَكِيمُ أَيْضاً - وَإِنْ قَالَ بِمَعْرِفَتِهِ - لاَ يَقْدِرُ أَنْ يَجِدَهُ.

\chapter{9}

\par 1 لأَنَّ هَذَا كُلَّهُ جَعَلْتُهُ فِي قَلْبِي وَامْتَحَنْتُ هَذَا كُلَّهُ: أَنَّ الصِّدِّيقِينَ وَالْحُكَمَاءَ وَأَعْمَالَهُمْ فِي يَدِ اللَّهِ. الإِنْسَانُ لاَ يَعْلَمُ حُبّاً وَلاَ بُغْضاً. الْكُلُّ أَمَامَهُمُ.
\par 2 الْكُلُّ عَلَى مَا لِلْكُلِّ. حَادِثَةٌ وَاحِدَةٌ لِلصِّدِّيقِ وَلِلشِّرِّيرِ لِلصَّالِحِ وَلِلطَّاهِرِ وَلِلنَّجِسِ. لِلذَّابِحِ وَلِلَّذِي لاَ يَذْبَحُ. كَالصَّالِحِ الْخَاطِئُ. الْحَالِفُ كَالَّذِي يَخَافُ الْحَلْفَ.
\par 3 هَذَا أَشَرُّ كُلِّ مَا عُمِلَ تَحْتَ الشَّمْسِ: أَنَّ حَادِثَةً وَاحِدَةً لِلْجَمِيعِ. وَأَيْضاً قَلْبُ بَنِي الْبَشَرِ مَلآنُ مِنَ الشَّرِّ وَالْحَمَاقَةُ فِي قَلْبِهِمْ وَهُمْ أَحْيَاءٌ وَبَعْدَ ذَلِكَ يَذْهَبُونَ إِلَى الأَمْوَاتِ.
\par 4 لأَنَّهُ مَنْ يُسْتَثْنَى؟ لِكُلِّ الأَحْيَاءِ يُوجَدُ رَجَاءٌ فَإِنَّ الْكَلْبَ الْحَيَّ خَيْرٌ مِنَ الأَسَدِ الْمَيِّتِ.
\par 5 لأَنَّ الأَحْيَاءَ يَعْلَمُونَ أَنَّهُمْ سَيَمُوتُونَ أَمَّا الْمَوْتَى فَلاَ يَعْلَمُونَ شَيْئاً وَلَيْسَ لَهُمْ أَجْرٌ بَعْدُ لأَنَّ ذِكْرَهُمْ نُسِيَ.
\par 6 وَمَحَبَّتُهُمْ وَبُغْضَتُهُمْ وَحَسَدُهُمْ هَلَكَتْ مُنْذُ زَمَانٍ وَلاَ نَصِيبَ لَهُمْ بَعْدُ إِلَى الأَبَدِ فِي كُلِّ مَا عُمِلَ تَحْتَ الشَّمْسِ.
\par 7 اِذْهَبْ كُلْ خُبْزَكَ بِفَرَحٍ وَاشْرَبْ خَمْرَكَ بِقَلْبٍ طَيِّبٍ لأَنَّ اللَّهَ مُنْذُ زَمَانٍ قَدْ رَضِيَ عَمَلَكَ.
\par 8 لِتَكُنْ ثِيَابُكَ فِي كُلِّ حِينٍ بَيْضَاءَ وَلاَ يُعْوِزْ رَأْسَكَ الدُّهْنُ.
\par 9 اِلْتَذَّ عَيْشاً مَعَ الْمَرْأَةِ الَّتِي أَحْبَبْتَهَا كُلَّ أَيَّامِ حَيَاةِ بَاطِلِكَ الَّتِي أَعْطَاكَ إِيَّاهَا تَحْتَ الشَّمْسِ كُلَّ أَيَّامِ بَاطِلِكَ لأَنَّ ذَلِكَ نَصِيبُكَ فِي الْحَيَاةِ وَفِي تَعَبِكَ الَّذِي تَتْعَبُهُ تَحْتَ الشَّمْسِ.
\par 10 كُلُّ مَا تَجِدُهُ يَدُكَ لِتَفْعَلَهُ فَافْعَلْهُ بِقُوَّتِكَ لأَنَّهُ لَيْسَ مِنْ عَمَلٍ وَلاَ اخْتِرَاعٍ وَلاَ مَعْرِفَةٍ وَلاَ حِكْمَةٍ فِي الْهَاوِيَةِ الَّتِي أَنْتَ ذَاهِبٌ إِلَيْهَا.
\par 11 فَعُدْتُ وَرَأَيْتُ تَحْتَ الشَّمْسِ أَنَّ السَّعْيَ لَيْسَ لِلْخَفِيفِ وَلاَ الْحَرْبَ لِلأَقْوِيَاءِ وَلاَ الْخُبْزَ لِلْحُكَمَاءِ وَلاَ الْغِنَى لِلْفُهَمَاءِ وَلاَ النِّعْمَةَ لِذَوِي الْمَعْرِفَةِ لأَنَّهُ الْوَقْتُ وَالْعَرَضُ يُلاَقِيَانِهِمْ كَافَّةً.
\par 12 لأَنَّ الإِنْسَانَ أَيْضاً لاَ يَعْرِفُ وَقْتَهُ. كَالأَسْمَاكِ الَّتِي تُؤْخَذُ بِشَبَكَةٍ مُهْلِكَةٍ وَكَالْعَصَافِيرِ الَّتِي تُؤْخَذُ بِالشَّرَكِ كَذَلِكَ تُقْتَنَصُ بَنُو الْبَشَرِ فِي وَقْتِ شَرٍّ إِذْ يَقَعُ عَلَيْهِمْ بَغْتَةً.
\par 13 هَذِهِ الْحِكْمَةُ رَأَيْتُهَا أَيْضاً تَحْتَ الشَّمْسِ وَهِيَ عَظِيمَةٌ عِنْدِي.
\par 14 مَدِينَةٌ صَغِيرَةٌ فِيهَا أُنَاسٌ قَلِيلُونَ. فَجَاءَ عَلَيْهَا مَلِكٌ عَظِيمٌ وَحَاصَرَهَا وَبَنَى عَلَيْهَا أَبْرَاجاً عَظِيمَةً.
\par 15 وَوُجِدَ فِيهَا رَجُلٌ مِسْكِينٌ حَكِيمٌ فَنَجَّى هُوَ الْمَدِينَةَ بِحِكْمَتِهِ. وَمَا أَحَدٌ ذَكَرَ ذَلِكَ الرَّجُلَ الْمِسْكِينَ!
\par 16 فَقُلْتُ: «الْحِكْمَةُ خَيْرٌ مِنَ الْقُوَّةِ». أَمَّا حِكْمَةُ الْمِسْكِينِ فَمُحْتَقَرَةٌ وَكَلاَمُهُ لاَ يُسْمَعُ.
\par 17 كَلِمَاتُ الْحُكَمَاءِ تُسْمَعُ فِي الْهُدُوءِ أَكْثَرَ مِنْ صُرَاخِ الْمُتَسَلِّطِ بَيْنَ الْجُهَّالِ.
\par 18 اَلْحِكْمَةُ خَيْرٌ مِنْ أَدَوَاتِ الْحَرْبِ. أَمَّا خَاطِئٌ وَاحِدٌ فَيُفْسِدُ خَيْراً جَزِيلاً.

\chapter{10}

\par 1 اَلذُّبَابُ الْمَيِّتُ يُنَتِّنُ وَيُخَمِّرُ طِيبَ الْعَطَّارِ. جَهَالَةٌ قَلِيلَةٌ أَثْقَلُ مِنَ الْحِكْمَةِ وَمِنَ الْكَرَامَةِ.
\par 2 قَلْبُ الْحَكِيمِ عَنْ يَمِينِهِ وَقَلْبُ الْجَاهِلِ عَنْ يَسَارِهِ!
\par 3 أَيْضاً إِذَا مَشَى الْجَاهِلُ فِي الطَّرِيقِ يَنْقُصُ فَهْمُهُ وَيَقُولُ لِكُلِّ وَاحِدٍ إِنَّهُ جَاهِلٌ!
\par 4 إِنْ صَعِدَتْ عَلَيْكَ رُوحُ الْمُتَسَلِّطِ فَلاَ تَتْرُكْ مَكَانَكَ لأَنَّ الْهُدُوءَ يُسَكِّنُ خَطَايَا عَظِيمَةً.
\par 5 يُوجَدُ شَرٌّ رَأَيْتُهُ تَحْتَ الشَّمْسِ كَسَهْوٍ صَادِرٍ مِنْ قِبَلِ الْمُتَسَلِّطِ.
\par 6 الْجَهَالَةُ جُعِلَتْ فِي مَعَالِي كَثِيرَةٍ وَالأَغْنِيَاءُ يَجْلِسُونَ فِي السَّافِلِ.
\par 7 قَدْ رَأَيْتُ عَبِيداً عَلَى الْخَيْلِ وَرُؤَسَاءَ مَاشِينَ عَلَى الأَرْضِ كَالْعَبِيدِ.
\par 8 مَنْ يَحْفُرُ هُوَّةً يَقَعُ فِيهَا وَمَنْ يَنْقُضُ جِدَاراً تَلْدَغُهُ حَيَّةٌ.
\par 9 مَنْ يَقْلَعُ حِجَارَةً يُوجَعُ بِهَا. مَنْ يُشَقِّقُ حَطَباً يَكُونُ فِي خَطَرٍ مِنْهُ.
\par 10 إِنْ كَلَّ الْحَدِيدُ وَلَمْ يُسَنِّنْ هُوَ حَدَّهُ فَلْيَزِدِ الْقُوَّةَ. أَمَّا الْحِكْمَةُ فَنَافِعَةٌ لِلإِنْجَاحِ.
\par 11 إِنْ لَدَغَتِ الْحَيَّةُ بِلاَ رُقْيَةٍ فَلاَ مَنْفَعَةَ لِلرَّاقِي.
\par 12 كَلِمَاتُ فَمِ الْحَكِيمِ نِعْمَةٌ وَشَفَتَا الْجَاهِلِ تَبْتَلِعَانِهِ.
\par 13 اِبْتِدَاءُ كَلاَمِ فَمِهِ جَهَالَةٌ وَآخِرُ فَمِهِ جُنُونٌ رَدِيءٌ.
\par 14 وَالْجَاهِلُ يُكَثِّرُ الْكَلاَمَ. لاَ يَعْلَمُ إِنْسَانٌ مَا يَكُونُ. وَمَنْ يُخْبِرُهُ مَاذَا يَصِيرُ بَعْدَهُ؟
\par 15 تَعَبُ الْجُهَلاَءِ يُعْيِيهِمْ لأَنَّهُ لاَ يَعْلَمُ كَيْفَ يَذْهَبُ إِلَى الْمَدِينَةِ
\par 16 وَيْلٌ لَكِ أَيَّتُهَا الأَرْضُ إِذَا كَانَ مَلِكُكِ وَلَداً وَرُؤَسَاؤُكِ يَأْكُلُونَ فِي الصَّبَاحِ.
\par 17 طُوبَى لَكِ أَيَّتُهَا الأَرْضُ إِذَا كَانَ مَلِكُكِ ابْنَ شُرَفَاءَ وَرُؤَسَاؤُكِ يَأْكُلُونَ فِي الْوَقْتِ لِلْقُوَّةِ لاَ لِلسَُّكْرِ.
\par 18 بِالْكَسَلِ الْكَثِيرِ يَهْبِطُ السَّقْفُ وَبِتَدَلِّي الْيَدَيْنِ يَكِفُ الْبَيْتُ.
\par 19 لِلضِّحْكِ يَعْمَلُونَ وَلِيمَةً وَالْخَمْرُ تُفَرِّحُ الْعَيْشَ. أَمَّا الْفِضَّةُ فَتُحَصِّلُ الْكُلَّ.
\par 20 لاَ تَسُبَّ الْمَلِكَ وَلاَ فِي فِكْرِكَ وَلاَ تَسُبَّ الْغَنِيَّ فِي مَضْجَعِكَ لأَنَّ طَيْرَ السَّمَاءِ يَنْقُلُ الصَّوْتَ وَذُو الْجَنَاحِ يُخْبِرُ بِالأَمْرِ.

\chapter{11}

\par 1 اِرْمِ خُبْزَكَ عَلَى وَجْهِ الْمِيَاهِ فَإِنَّكَ تَجِدُهُ بَعْدَ أَيَّامٍ كَثِيرَةٍ.
\par 2 أَعْطِ نَصِيباً لِسَبْعَةٍ وَلِثَمَانِيَةٍ أَيْضاً لأَنَّكَ لَسْتَ تَعْلَمُ أَيَّ شَرٍّ يَكُونُ عَلَى الأَرْضِ.
\par 3 إِذَا امْتَلَأَتِ السُّحُبُ مَطَراً تُرِيقُهُ عَلَى الأَرْضِ. وَإِذَا وَقَعَتِ الشَّجَرَةُ نَحْوَ الْجَنُوبِ أَوْ نَحْوَ الشِّمَالِ فَفِي الْمَوْضِعِ حَيْثُ تَقَعُ الشَّجَرَةُ هُنَاكَ تَكُونُ.
\par 4 مَنْ يَرْصُدُ الرِّيحَ لاَ يَزْرَعُ وَمَنْ يُرَاقِبُ السُّحُبَ لاَ يَحْصُدُ.
\par 5 كَمَا أَنَّكَ لَسْتَ تَعْلَمُ مَا هِيَ طَرِيقُ الرِّيحِ وَلاَ كَيْفَ الْعِظَامُ فِي بَطْنِ الْحُبْلَى كَذَلِكَ لاَ تَعْلَمُ أَعْمَالَ اللَّهِ الَّذِي يَصْنَعُ الْجَمِيعَ.
\par 6 فِي الصَّبَاحِ ازْرَعْ زَرْعَكَ وَفِي الْمَسَاءِ لاَ تَرْخِ يَدَكَ لأَنَّكَ لاَ تَعْلَمُ أَيُّهُمَا يَنْمُو هَذَا أَوْ ذَاكَ أَوْ أَنْ يَكُونَ كِلاَهُمَا جَيِّدَيْنِ سَوَاءً.
\par 7 اَلنُّورُ حُلْوٌ وَخَيْرٌ لِلْعَيْنَيْنِ أَنْ تَنْظُرَا الشَّمْسَ.
\par 8 لأَنَّهُ إِنْ عَاشَ الإِنْسَانُ سِنِينَ كَثِيرَةً فَلْيَفْرَحْ فِيهَا كُلِّهَا وَلْيَتَذَكَّرْ أَيَّامَ الظُّلْمَةِ لأَنَّهَا تَكُونُ كَثِيرَةً. كُلُّ مَا يَأْتِي بَاطِلٌ.
\par 9 اِفْرَحْ أَيُّهَا الشَّابُّ في حَدَاثَتِكَ وَلْيَسُرَّكَ قَلْبُكَ فِي أَيَّامِ شَبَابِكَ وَاسْلُكْ فِي طَرِيقِ قَلْبِكَ وَبِمَرْأَى عَيْنَيْكَ وَاعْلَمْ أَنَّهُ عَلَى هَذِهِ الأُمُورِ كُلِّهَا يَأْتِي بِكَ اللَّهُ إِلَى الدَّيْنُونَةِ.
\par 10 فَانْزِعِ الْغَمَّ مِنْ قَلْبِكَ وَأَبْعِدِ الشَّرَّ عَنْ لَحْمِكَ لأَنَّ الْحَدَاثَةَ وَالشَّبَابَ بَاطِلاَنِ.

\chapter{12}

\par 1 فَاذْكُرْ خَالِقَكَ فِي أَيَّامِ شَبَابِكَ قَبْلَ أَنْ تَأْتِيَ أَيَّامُ الشَّرِّ أَوْ تَجِيءَ السِّنِينَ إِذْ تَقُولُ: «لَيْسَ لِي فِيهَا سُرُورٌ».
\par 2 قَبْلَ مَا تَظْلُمُ الشَّمْسُ وَالنُّورُ وَالْقَمَرُ وَالنُّجُومُ وَتَرْجِعُ السُّحُبُ بَعْدَ الْمَطَرِ.
\par 3 فِي يَوْمٍ يَتَزَعْزَعُ فِيهِ حَفَظَةُ الْبَيْتِ وَتَتَلَوَّى رِجَالُ الْقُوَّةِ وَتَبْطُلُ الطَّوَاحِنُ لأَنَّهَا قَلَّتْ وَتُظْلِمُ النَّوَاظِرُ مِنَ الشَّبَابِيكِ.
\par 4 وَتُغْلَقُ الأَبْوَابُ فِي السُّوقِ. حِينَ يَنْخَفِضُ صَوْتُ الْمِطْحَنَةِ وَيَقُومُ لِصَوْتِ الْعُصْفُورِ وَتُحَطُّ كُلُّ بَنَاتِ الْغِنَاءِ.
\par 5 وَأَيْضاً يَخَافُونَ مِنَ الْعَالِي وَفِي الطَّرِيقِ أَهْوَالٌ وَاللَّوْزُ يُزْهِرُ وَالْجُنْدُبُ يُسْتَثْقَلُ وَالشَّهْوَةُ تَبْطُلُ. لأَنَّ الإِنْسَانَ ذَاهِبٌ إِلَى بَيْتِهِ الأَبَدِيِّ وَالنَّادِبُونَ يَطُوفُونَ فِي السُّوقِ.
\par 6 قَبْلَ مَا يَنْفَصِمُ حَبْلُ الْفِضَّةِ أَوْ يَنْسَحِقُ كُوزُ الذَّهَبِ أَوْ تَنْكَسِرُ الْجَرَّةُ عَلَى الْعَيْنِ أَوْ تَنْقَصِفُ الْبَكَرَةُ عِنْدَ الْبِئْرِ.
\par 7 فَيَرْجِعُ التُّرَابُ إِلَى الأَرْضِ كَمَا كَانَ وَتَرْجِعُ الرُّوحُ إِلَى اللَّهِ الَّذِي أَعْطَاهَا.
\par 8 «بَاطِلُ الأَبَاطِيلِ» قَالَ الْجَامِعَةُ: «الْكُلُّ بَاطِلٌ».
\par 9 بَقِيَ أَنَّ الْجَامِعَةَ كَانَ حَكِيماً وَأَيْضاً عَلَّمَ الشَّعْبَ عِلْماً وَوَزَنَ وَبَحَثَ وَأَتْقَنَ أَمْثَالاً كَثِيرَةً.
\par 10 اَلْجَامِعَةُ طَلَبَ أَنْ يَجِدَ كَلِمَاتٍ مُسِرَّةً مَكْتُوبَةً بِالاِسْتِقَامَةِ كَلِمَاتِ حَقٍّ.
\par 11 كَلاَمُ الْحُكَمَاءِ كَالْمَنَاخِسِ وَكَأَوْتَادٍ مُنْغَرِزَةٍ أَرْبَابُ الْجَمَاعَاتِ قَدْ أُعْطِيَتْ مِنْ رَاعٍ وَاحِدٍ.
\par 12 وَبَقِيَ فَمِنْ هَذَا يَا ابْنِي تَحَذَّرْ: لِعَمَلِ كُتُبٍ كَثِيرَةٍ لاَ نِهَايَةَ وَالدَّرْسُ الْكَثِيرُ تَعَبٌ لِلْجَسَدِ.
\par 13 فَلْنَسْمَعْ خِتَامَ الأَمْرِ كُلِّهِ: اتَّقِ اللَّهَ وَاحْفَظْ وَصَايَاهُ لأَنَّ هَذَا هُوَ الإِنْسَانُ كُلُّهُ.
\par 14 لأَنَّ اللَّهَ يُحْضِرُ كُلَّ عَمَلٍ إِلَى الدَّيْنُونَةِ عَلَى كُلِّ خَفِيٍّ إِنْ كَانَ خَيْراً أَوْ شَرّاً.


\end{document}