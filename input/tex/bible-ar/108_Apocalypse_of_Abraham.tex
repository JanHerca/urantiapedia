\begin{document}

\title{رؤيا إبراهيم}

\chapter{1}

\par سفر رؤيا إبراهيم بن تارح بن ناحور بن سروج بن روج (رعو) بن أرفكشاد بن سام بن نوح بن لامك بن متوشالح بن حنوك بن يارد (أراد).

\par \textit{تحول إبراهيم عن عبادة الأصنام <br> (الفصول من الأول إلى الثامن).}

\par 1 في اليوم الذي خططت فيه آلهة أبي تارح وآلهة ناحور أخيه، عندما كنت أبحث عن من هو الإله القدير حقًا - أنا إبراهيم، في الوقت الذي وقع فيه الأمر في قرعتي، عندما أديتُ خدمات (ذبائح) أبي تارح لآلهته من الخشب والحجر والذهب والفضة والنحاس والحديد؛ دخلتُ هيكلهم للخدمة، ووجدتُ الإله الذي كان اسمه مرومة (الذي كان) منحوتًا من الحجر، ساقطًا إلى الأمام عند قدمي إله الحديد ناحون. وحدث، عندما رأيتُ ذلك، أن قلبي تحير، وفكرتُ في نفسي أنني لن أتمكن من إعادته إلى مكانه، أنا إبراهيم، وحدي، لأنه كان ثقيلًا، كونه من حجر كبير، وخرجتُ وأخبرتُ أبي بذلك ودخل معي، وعندما حركناه كلانا (الإله) للأمام حتى نتمكن من إعادته إلى مكانه، سقط رأسه عنه وأنا لا أزال ممسكًا برأسه. وحدث أنه عندما رأى أبي أن رأس المرومات قد سقط منه، قال لي: "إبراهيم!" فقلت: "ها أنا ذا". وقال لي: "أحضر لي فأسًا من الصغار من المنزل". فأحضرته إليه .. فقطع مرومات آخر من حجر آخر، بدون رأس، والرأس الذي أُلقي من المرومات وضعه عليه، وحطم بقية المرومات.

\chapter{2}

\par 1 فصنع خمسة آلهة أخرى، وأعطاني إياها، وأمرني أن أبيعها خارجًا في ساحة المدينة. وسرجت حمار أبي، ووضعتها عليه، وذهبت نحو النزل لأبيعها. وإذا بتجار من فندنا في سوريا يسافرون على جمالهم إلى مصر للتجارة. وتحدثت معهم. وأطلق أحد جمالهم أنينًا، فخاف الحمار وقفز بعيدًا وأزعج الآلهة، فتم تحطيم ثلاثة منها، ونجا اثنان ولما رأى الآراميون أن لديّ آلهة، قالوا لي: "لماذا لم تخبرنا [أن لديك آلهة؟ كنا سنشتريها] قبل أن يسمع الحمار صوت الجمل، فلا تضيع. أعطنا، على أي حال، الآلهة الباقية، وسنعطيك الثمن المناسب للآلهة المكسورة، وكذلك الآلهة المحفوظة." فقد كنتُ قلقًا في قلبي كيف يمكنني إحضار ثمن الشراء لأبي، والآلهة الثلاثة المكسورة رميتها في ماء نهر جور، الذي كان في ذلك المكان، فغاصت في الأعماق، ولم يبقَ منها شيء.

\chapter{3}

\par 1 وبينما كنتُ سائرًا في الطريق، تاه قلبي وتشتت ذهني. فقلتُ في نفسي: [ما هذا العمل الشرير الذي يفعله أبي؟ أليس هو إله آلهته، إذ هم موجودون بفضل أزاميله ومخارطه وحكمته، أليس من اللائق أن يعبدوا أبي، إذ هم من صنعه؟ ما هذا الضلال الذي خدع به أبي في أعماله؟] ها هو ذا مروماث قد سقط ولم يستطع النهوض في هيكله، ولم أستطع أنا وحدي أن أحركه حتى جاء أبي، وحركناه نحن الاثنين؛ وإذ كنا ضعفاء جدًا، سقط رأسه عنه، فوضعه (أي أبي) على حجر آخر لإله آخر صنعه بلا رأس. وتحطمت الآلهة الخمسة الأخرى عن الحمار، فلم تستطع مساعدة نفسها ولا إيذاء الحمار، لأنه حطمها؛ ولم تخرج شظاياها المكسورة من النهر. فقلت في قلبي: "إذا كان الأمر كذلك، فكيف يستطيع ميرومات، إله أبي، الذي له رأس من حجر آخر، وهو مصنوع من حجر آخر، أن ينقذ رجلاً، أو يستمع إلى صلاة رجل ويكافئه؟"

\chapter{4}

\par 1 وبينما كنت أفكر هكذا، وصلت إلى منزل أبي؛ وبعد أن سقيت الحمار، وجهزت له التبن، أحضرت الفضة وسلمتها إلى يد أبي تارح. فلما رآها فرح، وقال: "مبارك أنت يا إبراهيم آلهتي، لأنك جلبت ثمن الآلهة، فلم يذهب عملي سدىً." فأجبته وقلت له: "اسمع يا أبي تارح! مباركة آلهتك، لأنك أنت إلههم، لأنك أنت خلقتهم؛ لأن بركتهم هي الخراب، وقوتهم باطلة. من لم يساعد نفسه، فكيف سيساعدك أو يباركك؟ لقد كنت لطيفًا معك في هذا الأمر، لأنني (باستخدام) ذكائي، أحضرت لك المال للآلهة المكسورة." ولما سمع الخبر، غضب عليّ بشدة، لأني تكلمت بكلمات قاسية ضد آلهته

\chapter{5}

\par 1 أما أنا، فقد فكرت في غضب أبي، فخرجت؛ [وبعد أن خرجت، صرخ أبي قائلًا: "إبراهيم!" فقلت: "ها أنا ذا". فقال: "خذ واجمع شظايا الخشب الذي صنعت منه آلهة من خشب الصنوبر قبل أن تأتي؛ وجهّز لي طعام العشاء". وحدث أنه عندما جمعت شظايا الخشب، وجدت تحتها إلهًا صغيرًا كان ملقى بين الأغصان عن يساري، وكان مكتوبًا على جبهته: الإله باريسات. ولم أخبر أبي أنني وجدت الإله الخشبي باريسات تحت الشظايا. وحدث، عندما وضعت الشظايا في النار، حتى أتمكن من إعداد الطعام لأبي - عندما خرجت لأسأل عن الطعام، وضعت باريسات أمام النار المشتعلة، وقلت له مهددًا: "انتبه جيدًا يا باريسات، [أن] النار لا تنطفئ حتى أعود؛ ومع ذلك، إذا خمدت، فانفخ فيها حتى تحترق مرة أخرى." وخرجت وأنجزت غرضي. وعندما رجعت وجدت باريسات ساقطًا على ظهره، وقدميه محاطتان بالنار ومحترقتان بشكل رهيب. انفجرت في نوبة من الضحك، وقلت في نفسي: "حقًا يا باريسات، يمكنك إشعال النار وطهي الطعام!" وحدث، بينما كنت أتحدث (هكذا) في ضحكي، احترق (أي باريسات) تدريجيًا بالنار وتحول إلى رماد. وأحضرت الطعام إلى والدي، فأكل. وأعطيته خمرًا وحليبًا، ففرح وبارك إلهه مرومات. وقلت له: "يا أبا تارح، لا تبارك إلهك مرومات ولا تمدحه، بل امدح إلهك باريسات، لأنه، من شدة حبه لك، ألقى بنفسه في النار ليطبخ طعامك!". فقال لي: "وأين هو الآن؟" [فقلت:] "لقد احترق حتى تحول إلى رماد في شدة النار وتحول إلى تراب." وقال: "عظيمة هي قوة باريسات! سأصنع آخر اليوم، وغدًا سيُعدّ طعامي."

\chapter{6}

\par 1 فلما سمعتُ أنا إبراهيم هذه الكلمات من أبي، ضحكتُ في نفسي وتنهدتُ حزنًا وغضبًا، وقلتُ: "كيف إذن يكون ما صنعه - التماثيل المصنّعة - عونًا لأبي؟ أم يكون الجسد خاضعًا لروحه، والنفس للروح، والروح للجهل والحماقة؟" وقلتُ: "ينبغي أن أتحمل الشر مرةً واحدة. لذا سأوجّه عقلي إلى ما هو نقي، وأُفصح له عن أفكاري." [و] أجبت وقلت: "يا أبا تارح، أيًا كان من هؤلاء الذي تمجده كإله، فأنت أحمق في عقلك. انظر إلى آلهة أخيك أورا، التي تقف في الهيكل المقدس، هي أكثر استحقاقًا للتكريم من [آلهةك هذه. لأن انظر إلى زوكايوس، إله أخيك أورون، هو أكثر استحقاقًا للتكريم من إلهك ميروماث، لأنه مصنوع من الذهب الذي يقدره الناس كثيرًا، وعندما يكبر في السن سيتم إعادة تشكيله؛ ولكن إذا تغير إلهك ميروماث أو انكسر، فلن يتجدد، لأنه حجر؛ وهذا هو الحال أيضًا مع الإله جوافون [الذي يقف مع زوكايوس على الآلهة الأخرى - فكم هو أكثر استحقاقًا للتكريم من الإله باريسات، المصنوع من الخشب، بينما هو مصنوع من الفضة! كيف أصبح، من خلال تكيف الإنسان، ذا قيمة للمظهر الخارجي! لكن إلهك باريسات، عندما كان لا يزال، قبل أن يكون "مُعَدَّ، مُسْتَأْصِلاً مِنْ عَلَى الأَرْضِ، كَانَ عَظِيمًا وَعَجَبًا بِمَجْدِ الأَغْصَانِ وَالأَزْهَارِ، قَطَعْتَهُ بِالْفَأْسِ، وَبِفَضْلِ فَنَارِكَ تَحَوَّلَ إِلَهًا. وَهَا! شَحْنَتُهُ قَدْ ذَبَلَتْ وَهَلَكَتْ، وَسَقَطَ مِنْ الْعَالِي إِلَى الأَرْضِ، وَرَاجَعَ مِنْ الْعَظِيمِ إِلَى الْصِّغَارِ، وَتَخَلَّتْ ظُهُورُ مَحَلَّتِهِ، وَأُحْرِقَ بَارِيشَاتُ بِالنَّارِ وَتَحَوَّلَ إِلَى رَمَادٍ وَلَمْ يَعُدْ هُوَ بَعْدُ؛" وَقُلْتَ: "سَأُعِدُّ الْيَوْمَ غَيْرَهِ لِتَهْيِئَةِ طَعَامِي!" "قَدْ هَلَكَ إِلَى الْهَلَاكِ!"

\chapter{7}

\par 1 انظروا، النار أحق بالتقدير من كل شيء مخلوق، لأن حتى ما لا يخضع يخضع لها، والأشياء سريعة التلف تُسخر منها بلهيبها. لكن الماء أحق بالتقدير، لأنه يقهر النار ويُشبع الأرض. لكن حتى هو لا أسميه إلهًا، لأنه خاضع للأرض التي يميل الماء تحتها. لكني أسمي الأرض أحق بالتقدير بكثير، لأنها تتغلب على طبيعة (وملء) الماء. حتى هي (أي الأرض)، لا أسميها إلهًا، لأنها هي أيضًا تجففها الشمس، [و] مُخصصة للإنسان ليُزرع. [أسمي الشمس أحق بالتقدير من الأرض]، لأنها بأشعتها تُنير العالم كله والأجواء المختلفة. [لكن] حتى هو لا أسميه إلهًا، لأنه في الليل وبالسحب يُحجب مساره. ولا أسمي القمر أو النجوم "إلهًا، لأنهم أيضًا في موسمهم يحجبون نورهم في الليل. [لكن] اسمع [هذا] يا تارح أبي؛ لأني سأعرفك بالإله الذي صنع كل شيء، لا هؤلاء الذين نعتبرهم آلهة. فمن هو إذن؟ أو ما هو؟"

\par 2 الذي كسا السماوات باللون القرمزي، وجعل الشمس ذهبية، والقمر منيرًا، ومعه النجوم؛

\par 3 وجعل الأرض يابسة في وسط مياه كثيرة،

\par 4 وأدخلتك في ... واختبرتني في حيرة أفكاري

\par 5 «ولكن ليُظهِر الله نفسه لنا من خلاله!»

\chapter{8}

\par 1 وحدث بينما كنت أكلم أبي تارح في دار بيتي، إذا بصوت جبار من السماء ينزل في سحابة نارية، قائلاً وينادي: "إبراهيم، إبراهيم!" فقلت: "ها أنا ذا". فقال: "أنت تبحث في فهم قلبك عن إله الآلهة والخالق: أنا هو: اخرج من عند أبيك تارح، واخرج من البيت، لئلا تُقتل أنت أيضًا في خطايا بيت أبيك". فخرجت. وحدث عندما خرجت، أنه قبل أن أنجح في الخروج من أمام باب الدار، جاء صوت رعد [عظيم] وأحرقه هو وبيته وكل ما في بيته، حتى الأرض، أربعين ذراعًا

\chapter{9}

\par \textit{يتلقى إبراهيم أمرًا إلهيًا بتقديم ذبيحة بعد أربعين يومًا استعدادًا للوحي الإلهي (الإصحاح التاسع؛ راجع تكوين 15).}

\par 1 ثم جاءني صوت يتكلم مرتين: "إبراهيم، إبراهيم!" فقلت: "ها أنا ذا!" وقال: "ها أنا ذا؛ لا تخف، فأنا أمام العالمين، وإله قدير خلق نور العالم. أنا درع عليك، وأنا معينك. اذهب، خذ لي عجلة صغيرة عمرها ثلاث سنوات، وعنزة عمرها ثلاث سنوات، وكبشًا عمره ثلاث سنوات، ويمامة وحمامة، وأحضر لي ذبيحة طاهرة. وفي هذه الذبيحة سأضع أمامك الدهور (القادمة)، وأعلمك بما هو محفوظ، وسترى أشياء عظيمة لم ترها (حتى الآن)؛ لأنك أحببت أن تفحصني، وقد سميتك صديقي. ولكن امتنع عن كل طعام يخرج من النار، وعن شرب الخمر، وعن الدهن (بزيتك)، أربعين يومًا،" ثم ضع لي الذبيحة التي أمرتك بها، في المكان الذي سأريك إياه، على جبل عالٍ، و هناك سأريك العصور التي تم إنشاؤها وتأسيسها وصنعها وتجديدها بكلمتي، وسأعرفك بما سيحدث فيها على أولئك الذين فعلوا الشر ومارسوا البر في جيل البشر.

\chapter{10}

\par \textit{يتوجه إبراهيم، بتوجيه من الملاك ياعويل، إلى جبل حوريب، في رحلة مدتها أربعون يومًا، لتقديم الذبيحة (الفصول من العاشر إلى الثاني عشر).}

\par 1 ولما سمعتُ صوتَ مَن كلَّمني بهذه الكلمات، نظرتُ هنا وهناك، فإذا بي لا أتنفسُ إنسانًا، فارتاع روحي، وهربت مني نفسي، وصرتُ كالحجر، وسقطتُ على الأرض، إذ لم تعد لي قوةٌ على الوقوف عليها. وبينما كنتُ لا أزال مُستلقيًا ووجهي على الأرض، سمعتُ صوت القدوس يقول: "اذهب يا يعوئيل، وباسمي الذي لا يُنطق به، أنشِئْني هذا الإنسان، وقوّيه (حتى يتعافى) من ارتجافه". وجاء الملاك الذي أرسله إليّ في صورة إنسان، وأمسك بيدي اليمنى، وأقامني على قدميّ، وقال لي: "قم يا [إبراهيم] خليل الله الذي يحبك؛ لا تدع رعشة الإنسان تسيطر عليك! لأني ها أنا ذا قد أُرسلت إليك لأقويك وأباركك باسم الله الذي يحبك، خالق السماوات والأرض. لا تخف وأسرع إليه. أنا من يحرك ما معي في الامتداد السابع على السماء، قوة بفضل الاسم الذي لا يُنطق به الذي يسكن فيّ. أنا الذي أُعطيت لكبح، وفقًا لأمره، الهجوم المهدد للكائنات الحية من الكروبيم ضد بعضها البعض، وتعليم أولئك الذين يحملونه ترنيمة الساعة السابعة من ليل الإنسان. أنا مُكلّف بكبح ليفياثان، لأني عرضة لهجوم وتهديد كل زاحف. [أنا الذي كُلِّف بحل الجحيم، لتدمير من يحدق في الموتى.] أنا الذي كُلِّف بإحراق منزل أبيك معه، لأنه أظهر احترامًا للموتى (الأصنام). لقد أُرسِلتُ لأباركك الآن، والأرض التي أعدَّها لك الأبدي، الذي دعوته، ومن أجلك سلكتُ طريقي على الأرض. انهض يا إبراهيم! اذهب دون خوف؛ كن سعيدًا وابتهج؛ وأنا معك! لأن الأبدي قد أعدَّ لك شرفًا أبديًا. اذهب، وأتمم الذبائح المطلوبة. ها أنا ذا! لقد عُيِّنتُ لأكون معك ومع الجيل المُعَدَّ (للخروج) منك؛ ومعي يباركك ميخائيل إلى الأبد. كن سعيدًا، اذهب!

\chapter{11}

\par 1 فقمت ورأيت الذي أمسك بيدي اليمنى وأقامني على قدمي. وكان منظر جسمه كالياقوت الأزرق، ومنظر وجهه كالزبرجد، وشعر رأسه كالثلج، والعمامة على رأسه كمنظر قوس قزح، ولباس ثيابه كالأرجوان، وكان في يده اليمنى صولجان من ذهب. فقال لي: «يا إبراهيم!» فقلت: «ها أنا عبدك». فقال: «لا يرهبك نظري ولا كلامي، لئلا تضطرب نفسك. تعال معي، فأذهب معك حتى الذبيحة الظاهرة، وبعد الذبيحة غير الظاهرة إلى الأبد. تشجع وتعالي!»

\chapter{12}

\par 1 وذهبنا، نحن الاثنان معًا، أربعين يومًا وليلة، ولم آكل خبزًا، ولم أشرب ماءً، لأن طعامي كان رؤية الملاك الذي كان معي، وكلامه - كان شرابي. ووصلنا إلى جبل الله، حوريب المجيد. وقلت للملاك: "يا مُنشِد الأبدي! ها! ليس معي ذبيحة، ولا أعرف مكان مذبح على الجبل: كيف يمكنني تقديم ذبيحة؟" فقال لي: "انظر حولك!" والتفت حولي، وإذا! كانت تتبعنا جميع الذبائح المحددة - العجلة الصغيرة، والماعز، والكبش، واليمامة، والحمامة. وقال لي الملاك: "إبراهيم!" فقلت: "ها أنا ذا". "وقال لي: "اذبح كل هؤلاء، وقسم الحيوانات إلى نصفين، واحد ضد الآخر، ولكن الطيور لا تفصل؛ و('لكن') أعطها للرجال، الذين سأريكم، الواقفين بجانبك، لأن هؤلاء هم المذبح على الجبل، لتقديم ذبيحة للرب؛ ولكن أعطني اليمام والحمامة، لأني سأصعد على أجنحة الطائر، لكي أريك في السماء، وعلى الأرض، وفي البحر، وفي الهاوية، وفي العالم السفلي، وفي جنة عدن، وفي أنهارها وفي ملء العالم كله ودائرته - ستنظر (فيهم) جميعًا."

\chapter{13}

\par \textit{أتم إبراهيم الذبيحة، بتوجيه من الملاك، ورفض أن يصرفه عزازيل عن هدفه (الفصول 13-14).}

\par 1 ففعلتُ كل شيءٍ بحسب أمر الملاك، وأعطيتُ الملائكة الذين أتوا إلينا الحيوانات المقسمة، لكن الملاك أخذ الطيور. وانتظرتُ ذبيحة المساء. وطار طائرٌ نجسٌ على الجثث، فطردته. وكلمني الطائر النجس وقال: "ماذا تفعل يا إبراهيم على المرتفعات المقدسة، حيث لا يأكل أحدٌ ولا يشرب، وليس عليها طعام بشري، بل يأكلون كل شيء بالنار، وسيحرقونك. اترك الرجل الذي معك واهرب؛ لأنه إذا صعدت إلى المرتفعات فسوف يهلكوك". وحدث عندما رأيت الطائر يتكلم، قلت للملاك: "ما هذا يا سيدي؟" فقال: "هذا كفر، هذا عزازيل". فقال له: "عارٌ عليك يا عزازيل! لأن نصيب إبراهيم في السماء، أما نصيبك على الأرض. لأنك اخترت وأحببت هذا المكان ليكون مسكنًا لنجاستك، لذلك جعلك الرب الأبدي القدير ساكنًا على الأرض ومن خلالك كل روح شريرة من الأكاذيب، ومن خلالك الغضب والتجارب لأجيال من الرجال الأشرار؛ لأن الله الأبدي القدير لم يسمح بأن تكون أجساد الصالحين في يدك، حتى تضمن بذلك حياة الصالحين وهلاك النجسين. اسمع يا صديقي، ابتعد عني بالخجل. لأنه لم يُعط لك أن تلعب دور المُجرب فيما يتعلق بجميع الصالحين. ابتعد عن هذا الرجل! لا يمكنك أن تضله، لأنه عدو لك ولأولئك الذين يتبعونك ويحبون ما تريد. لأنه هوذا الثوب الذي كان لك في السماء قد خصص له، والموت الذي كان له قد انتقل إليك.

\chapter{14}

\par 1 قال لي الملاك: ["يا إبراهيم!" فقلت: "ها أنا عبدك." فقال: "اعلم من الآن فصاعدًا أن الأزلي قد اختارك، الذي تحبه؛ تشجع واستخدم هذه السلطة، بقدر ما آمرك، ضد من يشوه الحقيقة؛ ألا أستطيع أن أخزيه من بدد أسرار السماء على الأرض وتمرد على القدير؟] قل له: كن أنت جمرًا مشتعلًا في أتون الأرض؛ اذهب يا عزازيل، إلى الأجزاء النائية من الأرض؛ [لأن ميراثك هو على أولئك الذين يعيشون معك المولودين مع النجوم والسحب، مع البشر الذين أنت نصيبهم، والذين من خلال وجودك يوجدون؛ وعداوتك هي التبرير. لهذا السبب بهلاكك تختفي عني." فنطقتُ بالكلمات التي علّمني إياها الملاك. فقال: «يا إبراهيم!» فقلتُ: «ها أنا عبدك.»

\par 2 "فقال لي الملاك: لا تجيبه، لأن الله أعطاه سلطانًا (حرفيًا: إرادة) على من يجيبونه." [ثم كلمني الملاك مرة ثانية وقال: "والآن، مهما تكلم معك، فلا تجيبه، لئلا يكون لإرادته مجرى حر فيك، لأن الأزلي القدير أعطاه وزنًا وإرادة؛ لا تجيبه." ففعلت ما أمرني به الملاك؛] ومهما تكلم معي، لم أجبه بشيء على الإطلاق.

\chapter{15}

\par \textit{إبراهيم والملاك يصعدان على أجنحة الطير إلى السماء (الفصلان 15-16).}

\par 1 وحدث عند غروب الشمس، إذا بدخان كدخان أتون. وصعد الملائكة الذين معهم نصيب الذبيحة من أعلى أتون الدخان. فأخذني الملاك بيدي اليمنى وأجلسني على الجناح الأيمن للحمامة، وجلس على الجناح الأيسر لليمامة التي لم تُذبح ولم تُقسم. وحملني إلى أطراف النار المشتعلة [وصعدنا كما لو كنا نسير في رياح عاتية إلى السماء التي كانت مثبتة على السطح. ورأيت في الهواء على العُلو الذي صعدنا إليه نورًا قويًا، كان من المستحيل وصفه، وإذا في هذا النور نار مشتعلة بشدة للناس، كثير من الناس بمظهر ذكوري، جميعهم (باستمرار) يتغيرون في المظهر والشكل، يركضون ويتحولون، ويعبدون ويصرخون بصوت كلمات لم أعرفها

\chapter{16}

\par 1 فقلت للملاك: "لماذا أحضرتني إلى هنا الآن، فأنا لا أستطيع الرؤية الآن، لأني قد ضعفت بالفعل، وروحي تفارقني؟" فقال لي: "ابقَ بجانبي؛ لا تخف! والذي تراه يأتي إلينا مباشرة بصوت عظيم من القداسة - هذا هو الأزلي الذي يحبك؛ لكنك لا تستطيع أن تراه). ولكن لا تدع روحك تضعف [بسبب الصراخ العالي]، لأني معك، أُقويك."

\chapter{17}

\par \textit{إبراهيم، بعد أن علمه الملاك، ينطق بالترنيمة السماوية ويصلي من أجل التنوير (الفصل السابع عشر).}

\par 1 وبينما هو يتكلم، إذا بنار قد حاصرتنا، وصوت في النار كصوت مياه كثيرة، كصوت البحر في هيجانه. فأحنى الملاك رأسه معي وسجد. تمنيتُ أن أسقط على الأرض، فالمرتفع الذي كنا واقفين عليه [يرتفع تارة]، ثم يتدحرج تارة أخرى.

\par 2 فقال: "يا إبراهيم، اسجد فقط، وغنِّ الأغنية التي علمتك إياها"؛ لأنه لم تكن هناك أرض تسقط عليها. فسجدت فقط، وغنِّيت الأغنية التي علمني إياها. وقال: "رنِّم بلا انقطاع". وترتل، وهو أيضًا أنشد معي الأغنية:

\par 3 أبدي، عظيم، قدوس، إيل،
\par    الله وحده - الأسمى!
\par 4 أنت الذي أنت من أصل ذاتي، غير قابل للفساد، بلا دنس،
\par 5 غير مخلوق، بلا دنس، خالد،
\par     مكتمل ذاتيًا، مضيء ذاتيًا؛
\par 6 بلا أب، بلا أم، غير مولود، مرتفع، ناري!
\par 7 محب للناس، خير، كريم،
\par 8 غيور عليّ ورحيم جدًا؛
\par 9 إيلي، أي إلهي—
\par 10 الصباؤوت الأبدي القدير القدوس،
\par 11 يا إلهي المجيد، يا إلهي، يا إلهي، يا إلهي، يا يعوئيل!
\par 12 أنت هو الذي أحبته روحي!
\par 13 الحامي الأبدي، المتألق كالنار،
\par 14 صوته كالرعد،
\par 15 الذي يشبه البرق، يرى كل شيء،
\par 16 من يتقبل صلوات من يكرمك؟
\par 17 [ويعرض عن طلبات من يحرجونه بإحراج استفزازاتهم،
\par 18 الذي يُذيب تشويشات العالم التي تنشأ من الأشرار والأبرار في العصر الفاسد، ويُجدد عصر الأبرار!]
\par 19 أنت، أيها النور، تشرق أمام نور
\par 20 صباح الخير على مخلوقاتك
\par 21 [حتى يصير نهارًا على الأرض]
\par 22 وفي مساكنك السماوية لا يوجد
\par     الحاجة إلى أي ضوء آخر
\par 23 من ذلك الروعة التي لا توصف من
\par     أنوار وجهك
\par 24 اقبل دعائي [وارضَ به]،
\par 25 وكذلك أيضًا الذبيحة التي أعددتها
\par    أنت من خلالي الذي بحثت عنك!
\par 26 اقبلني بقبول، وأرني، وعلمني،
\par 27 وأعلم عبدك كما وعدتني.

\chapter{18}

\par \textit{رؤية إبراهيم للعرش الإلهي (الفصل الثامن عشر).}

\par 1 وبينما كنت لا أزال أقرأ الأغنية، ارتفع فم النار الذي كان على السطح إلى الأعلى. وسمعت صوتًا مثل هدير البحر؛ ولم يتوقف بسبب وفرة النار الغنية. وبينما ارتفعت النار، صاعدة إلى العلو، رأيت تحت النار عرشًا من نار، وحوله كل شيء يرون، يرددون الأغنية، وتحت العرش أربعة كائنات حية نارية تغني، وكان مظهرها واحدًا، كل واحد منهم بأربعة وجوه. وكان هذا مظهر وجوههم، أسد، ورجل، وثور، ونسر: أربعة رؤوس [كانت على أجسادهم] [بحيث كان للمخلوقات الأربعة ستة عشر وجهًا] وكان لكل منهم ستة أجنحة؛ من أكتافهم، [وجوانبهم] وأحقادهم. "وبجناحيهما من أكتافهم غطوا وجوههم، وبالجناحين اللذين خرجا من أحقائهم غطوا أقدامهم، أما الجناحين الأوسطين فقد بسطاهما للطيران إلى الأمام. ولما انتهوا من الغناء نظروا إلى بعضهم البعض وهدد بعضهم بعضًا. وحدث عندما رأى الملاك الذي كان معي أنهم يهددون بعضهم بعضًا، أنه تركني وذهب راكضًا إليهم وحول وجه كل كائن حي عن الوجه المواجه له في الحال، حتى لا يروا وجوههم تهدد بعضها بعضًا. وعلمهم أغنية السلام التي لها أصلها [في الأبدي].

\par 2 وبينما كنت واقفًا وحدي أنظر، رأيت خلف الحيوانات مركبة ذات عجلات نارية، كل عجلة مليئة بالعيون حولها؛ وفوق العجلات عرش؛ رأيته، وكان مغطى بالنار، وأحاطت به نار من حوله، وإذا بنار لا توصف تحيط بجيش من النار. وسمعت صوتها المقدس كصوت إنسان

\chapter{19}

\par \textit{الله يكشف لإبراهيم قدرات السماء (الفصل التاسع عشر).}

\par 1 وجاءني صوت من وسط النار قائلًا: "إبراهيم، إبراهيم!" فقلت: "ها أنا ذا!" فقال: "انظر إلى الجِبال التي تحت الجلد الذي أنت عليه (الآن)، وانظر كيف لا يوجد على جِبال واحد غير الذي طلبته أو أحبك." وبينما كان لا يزال يتكلم، إذا بالجِبال قد انفتحت، وتحتي السماوات. ورأيت على الجلد السابع الذي كنت واقفًا عليه نارًا ممتدة على نطاق واسع، ونورًا، وندى، وجمعًا من الملائكة، وقوة مجد غير مرئية على الكائنات الحية التي رأيتها؛ ولكن لم أرَ أي كائن آخر هناك

\par 2 ونظرت من الجبل الذي كنت أقف فيه [إلى الأسفل] إلى السماء السادسة، ورأيت هناك جمعًا من الملائكة، من الروح (الطاهرة)، بلا أجساد، ينفذون أوامر الملائكة النارية الذين كانوا على السماء الثامنة، بينما كنت واقفًا معلقًا فوقهم. وإذا على هذه السماء لم تكن هناك قوى أخرى من (أي) شكل آخر، إلا ملائكة من الروح (الطاهرة)، مثل القوة التي رأيتها على السماء السابعة. وأمر بإزالة السماء السادسة. ورأيت هناك، على السماء الخامسة، قوى النجوم التي تنفذ الأوامر الموضوعة عليها، وعناصر الأرض تطيعها.

\chapter{20}

\par \textit{وعد البذرة (الفصل العشرون).}

\par 1 وقال لي الأبدي القدير: "إبراهيم، إبراهيم!" فقلت: "ها أنا ذا". [وقال:] "انظر من فوق إلى النجوم التي تحتك، وأحصها [لي]، وأعرفني بعددها". فقلت: "متى أستطيع؟ فأنا إنسان [من تراب ورماد]". فقال لي: "كما أن عدد النجوم وقوتها (هكذا) أجعل نسلك أمة وشعبًا، مميزًا لي في ميراثي مع عزازيل".

\par 2 فقلت: "أيها الأبدي القدير! ليتكلم عبدك أمامك، ولا يشتعل غضبك على مختارك! هوذا عزازيل قد شتمني قبل أن تصعدني. فكيف إذن، وهو ليس الآن أمامك، جعلت نفسك معه؟"

\chapter{21}

\par \textit{رؤية الخطيئة والجنة: مرآة العالم (الفصل الحادي والعشرون).}

\par 1 وقال لي: "انظر الآن، تحت قدميك إلى السماء وافهم الخلق المرسوم في هذا الامتداد، والمخلوقات الموجودة عليه، والعصر المعد وفقًا له". ورأيت تحت [سطح القدمين، ورأيت تحت] السماء السادسة وما فيها، ثم الأرض وثمارها، وما كان يحركها وكائناتها الحية: وقوة رجالها، وإثم نفوسهم، وأعمالهم الصالحة [وبدايات أعمالهم]، والمناطق السفلى وهلاكها، والهاوية وعذاباتها. رأيت هناك البحر وجزره، ووحوشه وأسماكه، وليفياثان وسلطانه، ومعسكره، وكهوفه، والعالم الذي يقع عليه، وحركاته، وتدمير العالم بسببه. رأيت هناك الجداول وارتفاع مياهها، وتعرجاتها. ورأيتُ هناك جنة عدن وثمارها، ومنبع النهر الذي ينبع منها، وأشجارها وأزهارها، ومن أحسنوا إليها. ورأيتُ فيها طعامهم ونعيمهم. ورأيتُ هناك جمعًا غفيرًا - رجالًا ونساءً وأطفالًا [نصفهم على يمين الصورة] ونصفهم على يسارها.

\chapter{22}

\par \textit{سقوط الإنسان وتكملته (الفصول XXI-XXV.)}

\par 1 فقلتُ: يا أبديا، يا قدير! ما هذه الصورة للمخلوقات؟ فقال لي: هذه مشيئتي فيمن هم في مجلس العالم (الإلهي)، وقد بدت لي مقبولة، ثم أمرتهم بها بعد ذلك بكلمتي. وكان ما عزمت عليه، قد خُطط له مسبقًا في هذه (الصورة)، وظهر أمامي قبل أن يُخلق، كما رأيت.

\par 2 فقلتُ: "يا رب، أيها القدير والأبدي! من هم الناس في هذه الصورة على هذا الجانب وعلى ذاك؟" فقال لي: "هؤلاء الذين على الجانب الأيسر هم جموع الشعوب التي كانت موجودة سابقًا والتي هي بعدك مُقدّرة، بعضها للدينونة والاستعادة، والبعض الآخر للانتقام والدمار في نهاية العالم. أما هؤلاء الذين على الجانب الأيمن من الصورة، فهم الشعب المخصص لي من الشعوب مع عزازيل. هؤلاء هم الذين عينتهم ليولدوا منك ويُدعون شعبي."

\chapter{23}

\par 1 «الآن انظر مرة أخرى في الصورة، من هو الذي أغوى حواء وما هي ثمرة الشجرة، [وسوف] تعرف ما سيكون، وكيف سيكون لنسلك بين الناس في نهاية أيام هذا العصر، وبقدر ما لا تستطيع أن تفهمه سأخبرك به، لأنك مرضي في نظري، وسأخبرك بما هو محفوظ في قلبي.»

\par 2 ونظرت إلى الصورة، وركضت عيناي إلى جانب جنة عدن. ورأيت هناك رجلاً عظيم القامة ومخيفًا في العرض، لا مثيل له في المظهر، يعانق امرأة، كانت أيضًا تقترب من شكل الرجل وشكله. وكانا واقفين تحت شجرة (جنة) عدن، وكانت ثمرة هذه الشجرة مثل منظر عنب الكرمة، وخلف الشجرة كان واقفًا كما لو كان ثعبانًا في الشكل، له أيادٍ وأقدام مثل أيدي وأقدام الرجل، وأجنحة على كتفيه، ستة على الجانب الأيمن وستة على الجانب الأيسر، وكانا يمسكان عنب الشجرة بأيديهما، وكلاهما كان يأكل منه الذي رأيته يعانق.

\par 3 فقلت: "من هؤلاء المحتضنون، أو من هذا الذي بينهم، أو ما هي الفاكهة التي يأكلونها، أيها الأبدي العظيم؟"

\par 4 وقال: "هذا هو عالم البشر، وهذا آدم، وهذه رغبتهم على الأرض، وهذه حواء؛ ولكن الذي بينهما يمثل الكفر، وبدايتهم (في الطريق) إلى الهلاك، حتى عزازيل."

\par 5 فقلت: "أيها الأبدي القدير! لماذا أعطيت لهذه القوة أن تهلك جيل البشر بأعمالهم على الأرض؟"

\par 6 وقال لي: «إن الذين يريدون الشر - وكم أبغضته في الذين يفعلونه! - أعطيته عليهم سلطانًا، وأن يكون محبوبًا منهم».

\par 7 فأجبتُ وقلتُ: أيها الأبديُّ القدير، لماذا أردتَ أن تُريدَ الشرَّ في قلوبِ البشر، وأنتَ غاضبٌ على ما أردتَ، على مَن يفعلُ ما لا يُناسبُ مشورتَك؟

\chapter{24}

\par 1 وقال لي: "غاضبًا على الأمم بسببك، وبسبب أهل بيتك الذين سيُفرّقون بعدك، كما ترى في الصورة العبء (المصير) الذي (يُلقى) عليهم - وسأخبرك بما سيكون، وكم سيكون، في الأيام الأخيرة. انظر الآن إلى كل شيء في الصورة."

\par 2 ونظرتُ فرأيتُ هناك ما كان أمامي في الخليقة؛ رأيتُ آدم وحواء موجودين معه، ومعهما الخصم الماكر، وقابيل الذي تصرف بشكل غير قانوني من خلال الخصم، وهابيل المذبوح، والدمار الذي جلبه وتسبب به من خلال الشخص الخارج عن القانون. رأيتُ هناك أيضًا النجاسة، وأولئك الذين يشتهونها، ودنسها، وغيرتهم، ونار فسادهم في أسفل الأرض. رأيتُ هناك السرقة، وأولئك الذين يسارعون وراءها، والترتيب [لقصاصهم، حكم المحكمة الكبرى]. رأيتُ هناك رجالًا عراة، جباههم على بعضها البعض، وعارهم، وعاطفتهم التي (كانت لديهم) ضد بعضهم البعض، وقصاصهم. رأيتُ هناك الشهوة، وفي يدها رأس كل نوع من أنواع الفوضى [وازدرائها وإهدارها المخصص للهلاك].

\chapter{25}

\par 1 رأيت هناك شبه صنم الغيرة، على شبه صنم خشبي كما كان أبي يصنع، وكان تمثاله من نحاس لامع، وأمامه رجل، وكان يسجد له، وأمامه مذبح، وعليه صبي مذبوح أمام الصنم

\par 2 فقلت له: ما هذا الصنم، أو ما المذبح، أو من هم الذين يُذبحون، أو من هو المُذبَح؟ أو ما هو الهيكل الذي أراه جميل الصنع، وجماله كالمجد الذي تحت عرشك؟

\par 3 وقال: "اسمع يا إبراهيم. هذا الذي تراه، الهيكل والمذبح والجمال، هو فكرتي عن كهنوت اسمي المجيد، الذي فيه تسكن كل صلاة من صلوات البشر، وقيام الملوك والأنبياء، وأي ذبيحة أطلب تقديمها لي من بين شعبي الذين سيأتون من جيلك. لكن التمثال الذي رأيته هو غضبي الذي به يغضبني الناس الذين سيخرجون من أجلي منك. أما الرجل الذي رأيته يذبح، فهو الذي يحرض على تقديم ذبائح قاتلة، وهي شاهد لي على يوم القيامة، حتى في بداية الخليقة."

\chapter{26}

\par \textit{لماذا تُباح الخطيئة (الفصل السادس والعشرون)؟}

\par 1 فقلت: "أيها الأبدي القدير! لماذا قررت أن يكون الأمر كذلك، ثم أعلنت معرفته؟"

\par 2 فقال لي: اسمع يا إبراهيم، وافهم ما أقول لك، وأجبني على أسئلتي. لماذا لم يصغِ أبوك تارح لصوتك، ولم يكف عن عبادة الأصنام الشيطانية حتى هلك هو وأهل بيته معه؟

\par 3 فقلت: «يا أيها الأبدي [العظيم]! (كان ذلك) لأنه لم يختر أن يستمع إليّ؛ لكنني أيضًا لم أتبع أعماله.»

\par 4 وقال [لي]: "اسمع يا إبراهيم. كما أن مشورة أبيك فيه، وكما أن مشورتك فيك، كذلك مشورة إرادتي فيّ جاهزة للأيام القادمة، قبل أن تعرفها، أو (تستطيع) أن ترى بعينيك ما هو مستقبلها. كيف سيكون نسلك، انظر في الصورة."

\chapter{27}

\par \textit{رؤية الدينونة والخلاص (الفصل السابع والعشرون).}

\par 1 ونظرت فرأيت: إذا بالصورة قد تأرجحت، وخرج من جانبها الأيسر شعب وثني، ونهبوا من كان على الجانب الأيمن، رجالاً ونساءً وأطفالاً: [ذبحوا بعضهم]، واستولوا على آخرين. إذا بهم يركضون نحوهم من أربعة مداخل، وأحرقوا الهيكل بالنار، ونهبوا المقدسات التي فيه

\par 2 فقلت: "أيها الأبدي! هوذا الناس (الذين ينبعون) مني، الذين قبلتهم، تنهبهم جحافل الوثنيين، ويقتلون بعضهم، بينما يتمسكون بالآخرين كغرباء، وأحرقوا الهيكل بالنار، ونهبوا [ودمروا] الأشياء الجميلة فيه. أيها الأبدي العظيم! إذا كان الأمر كذلك، فلماذا مزقت قلبي الآن، ولماذا يكون الأمر كذلك؟"

\par 3 فقال لي: اسمع يا إبراهيم، ما رأيته سيحدث بسبب نسلك الذين أغضبوني بسبب التمثال الذي رأيته، وبسبب المذبحة البشرية في الصورة، بسبب الغيرة في الهيكل، وكما رأيت كذلك يكون.

\par 4 فقلتُ: «يا أيها الأبدي القدير! لتُمضِ الآن أعمال الشرور، بل أرني الذين أتمّوا الوصايا، أعمال برّه. فإنك أنت قادر على هذا».

\par 5 وقال لي: «إن زمن الصالحين يقابلهم أولًا من خلال القداسة (الصادرة) من الملوك والحكام الصالحين الذين خلقتهم في البداية ليحكموا بينهم. ولكن من هؤلاء يخرج رجال يهتمون بمصالحهم، كما أعلمتك ورأيت.»

\chapter{28}

\par \textit{كم من الوقت؟ (الفصول 28-29).}

\par 1 فأجبتُ وقلتُ: "أيها القدير، [الأبدي] المُقدَّس بقوتك! استجب لطلبي، [لأنك من أجل هذا رفعتني إلى هنا - وأرني]. كما رفعتني إلى علوّك، هكذا أعلمني [هذا]، يا حبيبك، بقدر ما أسأل - هل سيحدث لهم ما رأيته طويلًا؟"

\par 2 وأراني جمعًا من قومه، وقال لي: «بسببهم، كما رأيت، سأُستفزّ منهم، وفي هذه الأيام سيُجازى على أفعالهم. ولكن في اليوم الرابع من مئة عام وساعة واحدة من العمر - وهي مئة عام - سيكون ذلك في شقاء بين الأمم [ولكن ساعة واحدة في رحمة واحتقار، كما هو الحال بين الأمم]».

\chapter{29}

\par 1 فقلت: يا صمد!؟ وكم ساعة من الدهر؟

\par 2 وقال: "لقد عيّنتُ اثنتي عشرة سنة لهذا العصر الفاجر ليحكم بين الوثنيين وفي نسلك؛ وإلى نهاية الزمان سيكون الأمر كما رأيت. وأنتَ تُقدّر وتُدرك وتُنظر في الصورة."

\par 3 ونظرتُ فرأيتُ رجلاً خارجاً من يسار الأمم، وخرج رجال ونساء وأطفال من جانب الأمم، جيوشٌ كثيرة، وسجدوا له. وبينما كنتُ لا أزال أنظر، إذ خرج كثيرون من اليمين، وشتم بعضهم ذلك الرجل، وضربه آخرون، وسجد له آخرون. ورأيتُ كيف سجد له هؤلاء، فركض عزازيل وسجد له، وقبل وجهه، والتفت ووقف خلفه

\par 4 فقلت: يا إتيمال، أيها الجبار! من هو الرجل الذي يُهان ويُضرب، والذي يعبده الوثنيون مع عزازيل؟

\par 5 فأجاب وقال: "اسمع يا إبراهيم! الرجل الذي رأيته يُهان ويُضرب ثم يُعبد مرة أخرى - ما هو الفرج؟ (الممنوح) من الوثنيين للشعب الذي ينبثق منك، في الأيام الأخيرة، في هذه الساعة الثانية عشرة من عصر الكفر. ولكن في السنة الثانية عشرة من عمري الأخير، سأقيم هذا الرجل من جيلك، الذي رأيته (يخرج) من شعبي؛ سيتبعه الجميع، وسينضم إليه أولئك الذين أدعوهم، (حتى) أولئك الذين يغيرون في نصائحهم. وأولئك الذين رأيتهم يخرجون من الجانب الأيسر من الصورة - المعنى هو: سيكون هناك الكثير من الوثنيين الذين يعلقون آمالهم عليه؛ وأما أولئك الذين رأيتهم من نسلك على الجانب الأيمن، بعضهم يُهينه ويضربه، والبعض الآخر يعبده - فسيشعر الكثير منهم بالإهانة منه. ومع ذلك، فهو يختبر أولئك الذين عبدوه من نسلك، في تلك الساعة الثانية عشرة من النهاية، بهدف "تقصير عصر الإلحاد."

\par 6 قبل أن يبدأ عصر الصالحين، ستحل دينونتي على الأمم الخارجة عن القانون من خلال شعب نسلك الذين فُرِّزوا لي. في تلك الأيام، سأنزل على جميع مخلوقات الأرض عشر ضربات، من خلال الشقاء والمرض وتنهد حزن نفوسهم. هكذا سأنزل على أجيال البشر الذين يعيشون عليها بسبب استفزاز وفساد مخلوقاتها، الذين يستفزونني به. وحينئذٍ سيبقى الصالحون من نسلك، بالعدد الذي أخفيته، يسارعون في مجد اسمي إلى المكان المُعد لهم مسبقًا، الذي رأيته مُدمرًا في الصورة؛ وسيعيشون ويتأسسون بذبائح وعطايا البر والحق في عصر الصالحين، وسيفرحون بي دائمًا؛ وسيُهلكون من أهلكهم، وسيُهينون من أهانهم.

\par 7 «ومن الذين شتموهم سيبصقون في وجوههم، مستهزئين بي، بينما ينظرون إليّ (الأبرار) فرحين، فرحين مع شعبي، ومتقبلين العائدين إليّ [بالتوبة]».

\par 8 «انظر يا إبراهيم، ما رأيت، واسمع ما سمعت، و[اعلم تمامًا] ما عرفت. اذهب إلى ميراثك، وها أنا معك إلى الأبد.»

\chapter{30}

\par \textit{عقاب الوثنيين وجمع إسرائيل (الفصول 30-31).}

\par 1 ولكن بينما كان لا يزال يتكلم، وجدت نفسي على الأرض. وقلت: "أيها الأبدي، [القدير]، لم أعد في المجد الذي كنت فيه (عندما) كنت في الأعالي، وما تاقت نفسي لفهمه في قلبي لا أفهمه."

\par 2 فقال لي: «سأخبرك بما يشتهيه قلبك، لأنك طلبت أن ترى الضربات العشر التي أعددتها للأمم، وأعددتها مسبقًا عند مرور الساعة الثانية عشرة من عمر الأرض. اسمع ما أخبرك به، فيحدث: الأولى هي ألم شديد؛ الثانية: حريق مدن كثيرة؛ الثالثة: دمار ووباء الحيوانات؛ الرابعة: جوع العالم كله وشعبه؛ الخامسة: دمار حكامه، دمار بالزلزال والسيف؛ السادسة: تكاثر البَرَد والثلج؛ السابعة: ستكون الوحوش البرية قبرهم؛ الثامنة: الجوع والوباء يتناوبان مع دمارهم؛ التاسعة: عقاب بالسيف والهروب في محنة؛ العاشرة: رعد وأصوات وزلزال مدمر.»

\chapter{31}

\par 1 «وبعد ذلك سأنفخ في البوق من الهواء، وأرسل مختاري، وفي يده كل قوتي، مقياسًا واحدًا؛ وهذا سيستدعي شعبي المحتقر من بين الأمم، وسأحرق بالنار أولئك الذين أهانوهم والذين حكموا بينهم في (هذا) العصر.»

\par 2 وسأُسلِّم الذين غطوني بالسخرية إلى لعنة العصر القادم؛ وقد أعددتهم ليكونوا طعامًا لنار الجحيم وللطيران المستمر ذهابًا وإيابًا في الهواء في العالم السفلي تحت الأرض [الجسد الممتلئ بالديدان]. لأنهم سيرون بر الخالق - أولئك الذين اختاروا أن يفعلوا مشيئتي، والذين التزموا بوصاياي علنًا، (وسيفرحون) فرحًا بسقوط الرجال الذين ما زالوا، الذين اتبعوا الأصنام وجرائم القتل. لأنهم سيتعفنون في جسد الدودة الشريرة عزازيل، ويحترقون بنار لسان عزازيل؛ لأني رجوت أن يأتوا إليّ، ولا يكونوا قد أحبوا الإله الغريب ومجدوه، ولا يلتزموا بمن لم يُقدَّر لهم، بل (بدلاً من ذلك) تركوا الرب القدير.

\chapter{32}

\par \textit{الخاتمة (الفصل الثاني والثلاثون)}

\par 1 "لذلك اسمع يا إبراهيم وانظر هوذا جيلك السابع يذهب معك ويخرجون إلى أرض غريبة ويستعبدونهم ويعاملونهم بالسوء كما لو كانوا في ساعة من ساعات عصر الكفر ولكن الأمة التي يستعبدون لها أنا سأدينها."

\end{document}