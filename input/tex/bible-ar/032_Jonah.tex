\begin{document}

\title{يونان}


\chapter{1}

\par 1 وَصَارَ قَوْلُ الرَّبِّ إِلَى يُونَانَ بْنِ أَمِتَّايَ:
\par 2 «قُمِ اذْهَبْ إِلَى نِينَوَى الْمَدِينَةِ الْعَظِيمَةِ وَنَادِ عَلَيْهَا لأَنَّهُ قَدْ صَعِدَ شَرُّهُمْ أَمَامِي».
\par 3 فَقَامَ يُونَانُ لِيَهْرُبَ إِلَى تَرْشِيشَ مِنْ وَجْهِ الرَّبِّ فَنَزَلَ إِلَى يَافَا وَوَجَدَ سَفِينَةً ذَاهِبَةً إِلَى تَرْشِيشَ فَدَفَعَ أُجْرَتَهَا وَنَزَلَ فِيهَا لِيَذْهَبَ مَعَهُمْ إِلَى تَرْشِيشَ مِنْ وَجْهِ الرَّبِّ.
\par 4 فَأَرْسَلَ الرَّبُّ رِيحاً شَدِيدَةً إِلَى الْبَحْرِ فَحَدَثَ نَوْءٌ عَظِيمٌ فِي الْبَحْرِ حَتَّى كَادَتِ السَّفِينَةُ تَنْكَسِرُ.
\par 5 فَخَافَ الْمَلاَّحُونَ وَصَرَخُوا كُلُّ وَاحِدٍ إِلَى إِلَهِهِ وَطَرَحُوا الأَمْتِعَةَ الَّتِي فِي السَّفِينَةِ إِلَى الْبَحْرِ لِيُخَفِّفُوا عَنْهُمْ. وَأَمَّا يُونَانُ فَكَانَ قَدْ نَزَلَ إِلَى جَوْفِ السَّفِينَةِ وَاضْطَجَعَ وَنَامَ نَوْماً ثَقِيلاً.
\par 6 فَجَاءَ إِلَيْهِ رَئِيسُ النُّوتِيَّةِ وَقَالَ لَهُ: «مَا لَكَ نَائِماً؟ قُمِ اصْرُخْ إِلَى إِلَهِكَ عَسَى أَنْ يَفْتَكِرَ الإِلَهُ فِينَا فَلاَ نَهْلِكَ».
\par 7 وَقَالَ بَعْضُهُمْ لِبَعْضٍ: «هَلُمَّ نُلْقِي قُرَعاً لِنَعْرِفَ بِسَبَبِ مَنْ هَذِهِ الْبَلِيَّةُ». فَأَلْقُوا قُرَعاً فَوَقَعَتِ الْقُرْعَةُ عَلَى يُونَانَ.
\par 8 فَقَالُوا لَهُ: «أَخْبِرْنَا بِسَبَبِ مَنْ هَذِهِ الْمُصِيبَةُ عَلَيْنَا؟ مَا هُوَ عَمَلُكَ؟ وَمِنْ أَيْنَ أَتَيْتَ؟ مَا هِيَ أَرْضُكَ وَمِنْ أَيِّ شَعْبٍ أَنْتَ؟»
\par 9 فَقَالَ لَهُمْ: «أَنَا عِبْرَانِيٌّ وَأَنَا خَائِفٌ مِنَ الرَّبِّ إِلَهِ السَّمَاءِ الَّذِي صَنَعَ الْبَحْرَ وَالْبَرَّ».
\par 10 فَخَافَ الرِّجَالُ خَوْفاً عَظِيماً وَقَالُوا لَهُ: «لِمَاذَا فَعَلْتَ هَذَا؟» فَإِنَّ الرِّجَالَ عَرَفُوا أَنَّهُ هَارِبٌ مِنْ وَجْهِ الرَّبِّ لأَنَّهُ أَخْبَرَهُمْ.
\par 11 فَقَالُوا لَهُ: «مَاذَا نَصْنَعُ بِكَ لِيَسْكُنَ الْبَحْرُ عَنَّا؟» لأَنَّ الْبَحْرَ كَانَ يَزْدَادُ اضْطِرَاباً.
\par 12 فَقَالَ لَهُمْ: «خُذُونِي وَاطْرَحُونِي فِي الْبَحْرِ فَيَسْكُنَ الْبَحْرُ عَنْكُمْ لأَنَّنِي عَالِمٌ أَنَّهُ بِسَبَبِي هَذَا النَّوْءُ الْعَظِيمُ عَلَيْكُمْ».
\par 13 وَلَكِنَّ الرِّجَالَ جَذَّفُوا لِيُرَجِّعُوا السَّفِينَةَ إِلَى الْبَرِّ فَلَمْ يَسْتَطِيعُوا لأَنَّ الْبَحْرَ كَانَ يَزْدَادُ اضْطِرَاباً عَلَيْهِمْ.
\par 14 فَصَرَخُوا إِلَى الرَّبِّ: «آهِ يَا رَبُّ لاَ نَهْلِكْ مِنْ أَجْلِ نَفْسِ هَذَا الرَّجُلِ وَلاَ تَجْعَلْ عَلَيْنَا دَماً بَرِيئاً لأَنَّكَ يَا رَبُّ فَعَلْتَ كَمَا شِئْتَ».
\par 15 ثُمَّ أَخَذُوا يُونَانَ وَطَرَحُوهُ فِي الْبَحْرِ فَوَقَفَ الْبَحْرُ عَنْ هَيَجَانِهِ.
\par 16 فَخَافَ الرِّجَالُ مِنَ الرَّبِّ خَوْفاً عَظِيماً وَذَبَحُوا ذَبِيحَةً لِلرَّبِّ وَنَذَرُوا نُذُوراً.
\par 17 وَأَمَّا الرَّبُّ فَأَعَدَّ حُوتاً عَظِيماً لِيَبْتَلِعَ يُونَانَ. فَكَانَ يُونَانُ فِي جَوْفِ الْحُوتِ ثَلاَثَةَ أَيَّامٍ وَثَلاَثَ لَيَالٍ.

\chapter{2}

\par 1 فَصَلَّى يُونَانُ إِلَى الرَّبِّ إِلَهِهِ مِنْ جَوْفِ الْحُوتِ
\par 2 وَقَالَ: «دَعَوْتُ مِنْ ضِيقِي الرَّبَّ فَاسْتَجَابَنِي. صَرَخْتُ مِنْ جَوْفِ الْهَاوِيَةِ فَسَمِعْتَ صَوْتِي.
\par 3 لأَنَّكَ طَرَحْتَنِي فِي الْعُمْقِ فِي قَلْبِ الْبِحَارِ. فَأَحَاطَ بِي نَهْرٌ. جَازَتْ فَوْقِي جَمِيعُ تَيَّارَاتِكَ وَلُجَجِكَ.
\par 4 فَقُلْتُ: قَدْ طُرِدْتُ مِنْ أَمَامِ عَيْنَيْكَ. وَلَكِنَّنِي أَعُودُ أَنْظُرُ إِلَى هَيْكَلِ قُدْسِكَ.
\par 5 قَدِ اكْتَنَفَتْنِي مِيَاهٌ إِلَى النَّفْسِ. أَحَاطَ بِي غَمْرٌ. الْتَفَّ عُشْبُ الْبَحْرِ بِرَأْسِي.
\par 6 نَزَلْتُ إِلَى أَسَافِلِ الْجِبَالِ. مَغَالِيقُ الأَرْضِ عَلَيَّ إِلَى الأَبَدِ. ثُمَّ أَصْعَدْتَ مِنَ الْوَهْدَةِ حَيَاتِي أَيُّهَا الرَّبُّ إِلَهِي.
\par 7 حِينَ أَعْيَتْ فِيَّ نَفْسِي ذَكَرْتُ الرَّبَّ فَجَاءَتْ إِلَيْكَ صَلاَتِي إِلَى هَيْكَلِ قُدْسِكَ.
\par 8 اَلَّذِينَ يُرَاعُونَ أَبَاطِيلَ كَاذِبَةً يَتْرُكُونَ نِعْمَتَهُمْ.
\par 9 أَمَّا أَنَا فَبِصَوْتِ الْحَمْدِ أَذْبَحُ لَكَ وَأُوفِي بِمَا نَذَرْتُهُ. لِلرَّبِّ الْخَلاَصُ».
\par 10 وَأَمَرَ الرَّبُّ الْحُوتَ فَقَذَفَ يُونَانَ إِلَى الْبَرِّ.

\chapter{3}

\par 1 ثُمَّ صَارَ قَوْلُ الرَّبِّ إِلَى يُونَانَ ثَانِيَةً:
\par 2 «قُمِ اذْهَبْ إِلَى نِينَوَى الْمَدِينَةِ الْعَظِيمَةِ وَنَادِ لَهَا الْمُنَادَاةَ الَّتِي أَنَا مُكَلِّمُكَ بِهَا».
\par 3 فَقَامَ يُونَانُ وَذَهَبَ إِلَى نِينَوَى بِحَسَبِ قَوْلِ الرَّبِّ. أَمَّا نِينَوَى فَكَانَتْ مَدِينَةً عَظِيمَةً لِلَّهِ مَسِيرَةَ ثَلاَثَةِ أَيَّامٍ.
\par 4 فَابْتَدَأَ يُونَانُ يَدْخُلُ الْمَدِينَةَ مَسِيرَةَ يَوْمٍ وَاحِدٍ وَنَادَى: «بَعْدَ أَرْبَعِينَ يَوْماً تَنْقَلِبُ نِينَوَى».
\par 5 فَآمَنَ أَهْلُ نِينَوَى بِاللَّهِ وَنَادُوا بِصَوْمٍ وَلَبِسُوا مُسُوحاً مِنْ كَبِيرِهِمْ إِلَى صَغِيرِهِمْ.
\par 6 وَبَلَغَ الأَمْرُ مَلِكَ نِينَوَى فَقَامَ عَنْ كُرْسِيِّهِ وَخَلَعَ رِدَاءَهُ عَنْهُ وَتَغَطَّى بِمِسْحٍ وَجَلَسَ عَلَى الرَّمَادِ.
\par 7 وَنُودِيَ فِي نِينَوَى عَنْ أَمْرِ الْمَلِكِ وَعُظَمَائِهِ: «لاَ تَذُقِ النَّاسُ وَلاَ الْبَهَائِمُ وَلاَ الْبَقَرُ وَلاَ الْغَنَمُ شَيْئاً. لاَ تَرْعَ وَلاَ تَشْرَبْ مَاءً.
\par 8 وَلْيَتَغَطَّ بِمُسُوحٍ النَّاسُ وَالْبَهَائِمُ وَيَصْرُخُوا إِلَى اللَّهِ بِشِدَّةٍ وَيَرْجِعُوا كُلُّ وَاحِدٍ عَنْ طَرِيقِهِ الرَّدِيئَةِ وَعَنِ الظُّلْمِ الَّذِي فِي أَيْدِيهِمْ
\par 9 لَعَلَّ اللَّهَ يَعُودُ وَيَنْدَمُ وَيَرْجِعُ عَنْ حُمُوِّ غَضَبِهِ فَلاَ نَهْلِكَ».
\par 10 فَلَمَّا رَأَى اللَّهُ أَعْمَالَهُمْ أَنَّهُمْ رَجَعُوا عَنْ طَرِيقِهِمِ الرَّدِيئَةِ نَدِمَ اللَّهُ عَلَى الشَّرِّ الَّذِي تَكَلَّمَ أَنْ يَصْنَعَهُ بِهِمْ فَلَمْ يَصْنَعْهُ. اَلأَصْحَاحُ الرَّابِعُ

\chapter{4}

\par 1 فَغَمَّ ذَلِكَ يُونَانَ غَمّاً شَدِيداً فَاغْتَاظَ
\par 2 وَصَلَّى إِلَى الرَّبِّ: «آهِ يَا رَبُّ أَلَيْسَ هَذَا كَلاَمِي إِذْ كُنْتُ بَعْدُ فِي أَرْضِي؟ لِذَلِكَ بَادَرْتُ إِلَى الْهَرَبِ إِلَى تَرْشِيشَ لأَنِّي عَلِمْتُ أَنَّكَ إِلَهٌ رَأُوفٌ وَرَحِيمٌ بَطِيءُ الْغَضَبِ وَكَثِيرُ الرَّحْمَةِ وَنَادِمٌ عَلَى الشَّرِّ.
\par 3 فَالآنَ يَا رَبُّ خُذْ نَفْسِي مِنِّي لأَنَّ مَوْتِي خَيْرٌ مِنْ حَيَاتِي».
\par 4 فَقَالَ الرَّبُّ: «هَلِ اغْتَظْتَ بِالصَّوَابِ؟».
\par 5 وَخَرَجَ يُونَانُ مِنَ الْمَدِينَةِ وَجَلَسَ شَرْقِيَّ الْمَدِينَةِ وَصَنَعَ لِنَفْسِهِ هُنَاكَ مَظَلَّةً وَجَلَسَ تَحْتَهَا فِي الظِّلِّ حَتَّى يَرَى مَاذَا يَحْدُثُ فِي الْمَدِينَةِ.
\par 6 فَأَعَدَّ الرَّبُّ الإِلَهُ يَقْطِينَةً فَارْتَفَعَتْ فَوْقَ يُونَانَ لِتَكُونَ ظِلاًّ عَلَى رَأْسِهِ لِيُخَلِّصَهُ مِنْ غَمِّهِ. فَفَرِحَ يُونَانُ مِنْ أَجْلِ الْيَقْطِينَةِ فَرَحاً عَظِيماً.
\par 7 ثُمَّ أَعَدَّ اللَّهُ دُودَةً عِنْدَ طُلُوعِ الْفَجْرِ في الْغَدِ فَضَرَبَتِ الْيَقْطِينَةَ فَيَبِسَتْ.
\par 8 وَحَدَثَ عِنْدَ طُلُوعِ الشَّمْسِ أَنَّ اللَّهَ أَعَدَّ رِيحاً شَرْقِيَّةً حَارَّةً فَضَرَبَتِ الشَّمْسُ عَلَى رَأْسِ يُونَانَ فَذَبُلَ فَطَلَبَ لِنَفْسِهِ الْمَوْتَ وَقَالَ: «مَوْتِي خَيْرٌ مِنْ حَيَاتِي».
\par 9 فَقَالَ اللَّهُ لِيُونَانَ: «هَلِ اغْتَظْتَ بِالصَّوَابِ مِنْ أَجْلِ الْيَقْطِينَةِ؟» فَقَالَ: «اغْتَظْتُ بِالصَّوَابِ حَتَّى الْمَوْتِ».
\par 10 فَقَالَ الرَّبُّ: «أَنْتَ شَفِقْتَ عَلَى الْيَقْطِينَةِ الَّتِي لَمْ تَتْعَبْ فِيهَا وَلاَ رَبَّيْتَهَا الَّتِي بِنْتَ لَيْلَةٍ كَانَتْ وَبِنْتَ لَيْلَةٍ هَلَكَتْ.
\par 11 أَفَلاَ أَشْفَقُ أَنَا عَلَى نِينَوَى الْمَدِينَةِ الْعَظِيمَةِ الَّتِي يُوجَدُ فِيهَا أَكْثَرُ مِنِ اثْنَتَيْ عَشَرَةَ رَبْوَةً مِنَ النَّاسِ الَّذِينَ لاَ يَعْرِفُونَ يَمِينَهُمْ مِنْ شِمَالِهِمْ وَبَهَائِمُ كَثِيرَةٌ!».

\end{document}