\begin{document}

\title{باروخ}


\chapter{1}

\par 1 "وهذه هي كلمات السفر الذي كتبه باروخ بن نيريا بن معسيا بن صدقيا بن أسعديا بن حلقيا في بابل."
\par 2 وفي السنة الخامسة وفي اليوم السابع من الشهر حين أخذ الكلدانيون أورشليم وأحرقوها بالنار.
\par 3 فقرأ باروخ كلام هذا السفر في مسامع يكنيا بن يواقيم ملك يهوذا وفي آذان كل الشعب الذين جاءوا ليسمعوا السفر.
\par 4 وفي مسامع العظماء وبني الملك وفي مسامع الشيوخ وكل الشعب من الصغير إلى الكبير كل سكان بابل عند نهر سود.
\par 5 ثم بكوا وصاموا وصلوا أمام الرب.
\par 6 كما جمعوا أيضًا مبلغًا من المال حسب قدرة كل رجل:
\par 7 وأرسلوها إلى أورشليم إلى يواقيم رئيس الكهنة ابن حلقيا بن شالوم وإلى الكهنة وإلى كل الشعب الذين وجدوا معه في أورشليم.
\par 8 وفي ذلك الوقت حين استلم آنية بيت الرب التي أخرجت من الهيكل ليردها إلى أرض يهوذا في اليوم العاشر من شهر سيوان، وهي آنية فضة صنعها صدقيا بن يوشيا ملك يهوذا،
\par 9 وبعد ذلك، سبى نبوخذناصر ملك بابل يكنيا والرؤساء والأسرى والأبطال وشعب الأرض من أورشليم، وجاء بهم إلى بابل.
\par 10 فقالوا هوذا قد أرسلنا لكم فضة لتشتروا محرقات وذبائح خطية وبخورا وتصنعوا المن وتقدموه على مذبح الرب إلهنا.
\par 11 وصلوا من أجل حياة نبوخذنصر ملك بابل، وحياة بلتازار ابنه، لكي تكون أيامهما على الأرض كأيام السماء.
\par 12 "ويعطينا الرب قوة وينير عيوننا فنعيش في ظل نبوخذنصر ملك بابل وفي ظل بلتازار ابنه ونخدمهما أياما كثيرة ونجد نعمة في عيونهما."
\par 13 وصلي لأجلنا أيضاً إلى الرب إلهنا، لأننا أخطأنا إلى الرب إلهنا، ولم يرتد غضب الرب وغضبه عنا إلى هذا اليوم.
\par 14 "وتقرأون هذا الكتاب الذي أرسلناه إليكم للاعتراف في بيت الرب في الأعياد والمواسم."
\par 15 وتقولون للرب إلهنا البر، وأما لنا فخزي الوجوه كما حدث اليوم لأهل يهوذا ولسكان أورشليم.
\par 16 وإلى ملوكنا وأمرائنا وكهنتنا وأنبيائنا وآباءنا.
\par 17 لأننا أخطأنا أمام الرب،
\par 18 وعصوا أمره ولم يسمعوا لصوت الرب إلهنا لكي يسلكوا في الوصايا التي أوصانا بها علانية.
\par 19 ومنذ اليوم الذي أخرج فيه الرب آباءنا من أرض مصر إلى هذا اليوم، عصينا الرب إلهنا، وقصرنا في عدم سماع صوته.
\par 20 "لذلك التصقت بنا الشرور واللعنة التي عيّنها الرب عن يد موسى عبده في الوقت الذي أخرج فيه آباءنا من أرض مصر ليعطينا أرضاً تفيض لبنا وعسلاً كما نرى هذا اليوم."
\par 21 ولكننا لم نسمع لصوت الرب إلهنا حسب كل أقوال الأنبياء الذين أرسلهم إلينا.
\par 22 "ولكن كل إنسان سار وراء تصورات قلبه الشريرة ليعبد آلهة أخرى وليعمل الشر في عيني الرب إلهنا."

\chapter{2}

\par 1 "لذلك أكمل الرب كلامه الذي تكلم به علينا وعلى قضاتنا الذين حكموا إسرائيل وعلى ملوكنا وعلى رؤسائنا وعلى رجال إسرائيل ويهوذا."
\par 2 ليجلب علينا ضربات عظيمة لم تكن مثلها تحت السماء كلها، كما حدث في أورشليم حسب ما هو مكتوب في ناموس موسى.
\par 3 أن يأكل الإنسان لحم ابنه ولحم ابنته.
\par 4 وأسلمهم خاضعين لكل الممالك التي حولنا، ليكونوا عارا وخراباً بين جميع الشعوب حولنا حيث بددهم الرب.
\par 5 وهكذا انحطنا ولم نرتفع لأننا أخطأنا إلى الرب إلهنا ولم نطيع صوته.
\par 6 للرب إلهنا البر، وأما نحن ولآبائنا فالخزي واضح كما في هذا اليوم.
\par 7 لأنه قد جاءت علينا جميع هذه الضربات التي تكلم بها الرب علينا.
\par 8 ولكننا لم نصلي أمام الرب لكي يصرف كل واحد منا عن أفكار قلبه الشرير.
\par 9 لذلك راقب الرب الشر علينا، فجلبه الرب علينا، لأن الرب بار في جميع أعماله التي أوصانا بها.
\par 10 ولكننا لم نسمع لصوته، لنسلك في وصايا الرب التي وضعها أمامنا.
\par 11 والآن يا رب إله إسرائيل الذي أخرجت شعبك من أرض مصر بيد شديدة وذراع عالية وآيات وعجائب وقوة عظيمة وجعلت لنفسك اسما كما في هذا اليوم.
\par 12 يا رب إلهنا، لقد أخطأنا، وفعلنا الشر، وتعاملنا بالظلم في جميع أحكامك.
\par 13 "فلترتفع غضبك عنا، لأننا بقينا عددا قليلا بين الأمم الذين شتتنا بينهم."
\par 14 اسمع يا رب صلواتنا وتضرعاتنا، وأنقذنا من أجلك، وأعطنا نعمة في عيون الذين طردونا.
\par 15 لكي تعلم كل الأرض أنك أنت الرب إلهنا، لأنه باسمك دعي إسرائيل وعشيرته.
\par 16 يا رب، انظر من بيت قدسك وأنظر إلينا. أمل أذنك يا رب واسمعنا.
\par 17 افتح عينيك وانظر، فإن الأموات الذين في القبور، الذين تُنزع نفوسهم من أجسادهم، لن يعطوا الرب مديحًا ولا برا.
\par 18 "وأما النفس المتضايقة جداً، والتي تمشي منحنية ضعيفة، والعين التي تكل، والنفس الجائعة، فسوف يعطونك الحمد والبر، يا رب."
\par 19 لذلك لا نرفع دعاءنا الذليل أمامك، أيها الرب إلهنا، من أجل بر آبائنا وملوكنا.
\par 20 لأنك أرسلت غضبك وغضبك علينا كما تكلمت عن يد عبيدك الأنبياء قائلا:
\par 21 هكذا قال الرب: انحنوا على أكتافهم لخدمة ملك بابل، فتثبتوا في الأرض التي أعطيت لآبائكم.
\par 22 ولكن إن لم تسمعوا صوت الرب لتخدموا ملك بابل،
\par 23 وأبطل من مدن يهوذا ومن خارج أورشليم صوت الفرح وصوت الفرح صوت العريس وصوت العروس وتكون كل الأرض خرابا من السكان.
\par 24 ولكننا لم نسمع لصوتك لنخدم ملك بابل. لذلك أكملت الكلام الذي تكلمت به على يد عبيدك الأنبياء أن عظام ملوكنا وعظام آبائنا يجب أن تؤخذ من مكانها.
\par 25 وإذا هم مطروحون في حر النهار وفي صقيع الليل، ويموتون في بؤس عظيم من الجوع والسيف والطاعون.
\par 26 "وأنت خربت البيت الذي دعي باسمك كما هو ظاهر اليوم لأجل شرور بيت إسرائيل وبيت يهوذا."
\par 27 يا رب إلهنا، لقد تعاملت معنا حسب كل جودك، وحسب كل رحمتك العظيمة،
\par 28 كما تكلمت على لسان عبدك موسى يوم أمرته أن يكتب الشريعة أمام بني إسرائيل قائلا:
\par 29 وإن لم تسمعوا لصوتي، فإن هذا الجمع الكثير جداً يتحول إلى عدد قليل بين الأمم، حيث أشتتهم.
\par 30 لأني علمت أنهم لا يسمعون لي لأنهم شعب صلب الرقبة. ولكن في أرض سبيهم يذكرون أنفسهم.
\par 31 فيعلمون أني أنا الرب إلههم وأعطيهم قلبا وآذانا ليسمعوا.
\par 32 فيسبحونني في أرض سبيهم ويتذكرون اسمي.
\par 33 ويرجعون عن رقابهم القاسية وعن أعمالهم الشريرة، فيتذكرون طريق آبائهم الذين أخطأوا أمام الرب.
\par 34 وأرجعهم إلى الأرض التي أقسمت لآبائهم إبراهيم وإسحق ويعقوب فيكونون لها سادة وأكثرهم فلا يقلون.
\par 35 وأقطع لهم عهدا أبديا لأكون لهم إلها ويكونون لي شعبا ولا أطرد شعبي إسرائيل بعد من الأرض التي أعطيتهم إياها.

\chapter{3}

\par 1 يا رب القوات، إله إسرائيل، النفس المضطربة والروح المضطربة تصرخ إليك.
\par 2 اسمع يا رب وارحم، أنت رحيم، وارحمنا لأننا أخطأنا أمامك.
\par 3 لأنك أنت تدوم إلى الأبد، ونحن نهلك إلى الأبد.
\par 4 يا رب الجنود، أنت إله إسرائيل، اسمع الآن صلوات بني إسرائيل الأموات وأولادهم الذين أخطأوا أمامك ولم يسمعوا لصوت إلههم، لأن هذا هو سبب لصق هذه الضربات بنا.
\par 5 لا تذكر ذنوب آبائنا بل فكر في قدرتك واسمك الآن في هذا الوقت.
\par 6 لأنك أنت الرب إلهنا وإياك يا رب نحمد.
\par 7 ولذلك جعلت خوفك في قلوبنا لكي ندعو باسمك ونحمدك في سبينا لأننا تذكرنا كل ذنوب آبائنا الذين خطئوا أمامك.
\par 8 هوذا نحن اليوم بعد في سبينا الذي شتتنا فيه، عارا ولعنة، وعقابا حسب كل ذنوب آبائنا الذين ارتدوا عن الرب إلهنا.
\par 9 اسمع يا إسرائيل وصايا الحياة. أنصت لتفهم الحكمة.
\par 10 كيف يحدث يا إسرائيل أنك في أرض أعدائك، وأنك شاخست في أرض غريبة، وأنك نجست مع الأموات،
\par 11 حتى تحسب مع الذين يهبطون إلى الهاوية؟
\par 12 لقد تركت نبع الحكمة.
\par 13 لأنه لو كنت قد سلكت في طريق الله، لسكنت في سلام إلى الأبد.
\par 14 تعلم أين الحكمة وأين القوة وأين الفهم لكي تعرف أيضا أين طول الأيام والحياة وأين نور العيون والسلام.
\par 15 من اكتشف مكانها أو من دخل كنوزها؟
\par 16 أين أصبح أمراء الأمم، وأولئك الذين حكموا الوحوش على الأرض؛
\par 17 الذين كانوا يتسلون مع طيور السماء، والذين كانوا يجمعون الفضة والذهب، مما يتوكل عليه الناس، ولم يجعلوا من جمعهم نهاية؟
\par 18 أما أولئك الذين صنعوا الفضة، وكانوا حريصين للغاية، وأعمالهم غير قابلة للبحث،
\par 19 لقد اختفوا ونزلوا إلى القبر، وقام آخرون في مكانهم.
\par 20 "أبصر الشباب النور وسكنوا الأرض، ولكنهم لم يعرفوا طريق المعرفة،
\par 21 ولم يفهموا سبلها ولم يدركوها. كان أبناؤهم بعيدين عن ذلك الطريق.
\par 22 ولم يسمع به في كنعان، ولم ير في تيمان.
\par 23 الأجاريون الذين يطلبون الحكمة على الأرض، وتجار ميران وتيمان، ومؤلفو الخرافات، والباحثون عن الفهم؛ لم يعرف أحد منهم طريق الحكمة، أو يتذكر مساراتها.
\par 24 يا إسرائيل ما أعظم بيت الله وما أوسع مكان ملكه.
\par 25 عظيم، وليس له نهاية، مرتفع، ولا يمكن قياسه.
\par 26 كان هناك عمالقة مشهورين منذ البداية، وكانوا ذوي مكانة عظيمة، وخبراء في الحرب.
\par 27 أولئك لم يخترهم الرب ولم يعطهم طريق المعرفة.
\par 28 ولكنهم هلكوا لأنه لم تكن لهم الحكمة، وهلكوا بسبب جهلهم.
\par 29 من صعد إلى السماء فأخذها وأنزلها من السحاب؟
\par 30 من عبر البحر ووجدها ويأتي بها مقابل الذهب الخالص؟
\par 31 لا أحد يعرف طريقها، ولا يفكر في مسارها.
\par 32 "ولكن من يعرف كل شيء يعرفها، وقد وجدها بفهمه. الذي أعد الأرض إلى الأبد ملأها دواباً."
\par 33 من يرسل النور فيذهب، ويدعوه مرة أخرى، فيطيعه بخوف.
\par 34 النجوم أشرقت في محارسها وفرحّت، عندما دعاها قالت: هنا نحن، وهكذا بفرح أضاءت لصانعها.
\par 35 هذا هو إلهنا، ولا يُحسب أحدٌ غيره في مقارنته.
\par 36 وقد وجد كل طريق المعرفة وأعطاه ليعقوب عبده ولإسرائيل حبيبه.
\par 37 وبعد ذلك ظهر على الأرض وتحدث مع البشر.

\chapter{4}

\par 1 هذا هو كتاب وصايا الله والشريعة التي إلى الأبد. كل من يحفظه سيحيون، ولكن من يتركه يموت.
\par 2 ارجع يا يعقوب وامسكها، وامش في حضرة نورها لكي تستنير.
\par 3 لا تعطي شرفك لآخر، ولا الأشياء التي تنفعك لأمة غريبة.
\par 4 يا إسرائيل طوبى لنا، لأن الأمور المرضية أمام الله أصبحت معروفة لنا.
\par 5 تشجعوا يا شعبي، يا تذكار إسرائيل.
\par 6 لقد بيعتم للأمم ليس للهلاك، بل لأنكم أغضبتم الله فأسلمتم إلى الأعداء.
\par 7 لأنكم أغضبتم صانعكم بأن ذبحتم للشياطين وليس لله.
\par 8 لقد نسيتم الإله الأزلي الذي رباكم، وأحزنتم أورشليم التي رضعتكم.
\par 9 لأنها حين رأت غضب الله مقبلاً عليكم قالت: اسمعوا يا سكان صهيون. قد جلب الله عليّ حزناً عظيماً.
\par 10 لأني رأيت سبي أبنائي وبناتي الذي جلبه عليهم الأبد.
\par 11 لقد أطعمتهم بفرح، ولكنني ودعتهم بالبكاء والنحيب.
\par 12 لا يفرح بي أحد، أنا الأرملة المهجورة من كثيرين، التي تركت خرابا من أجل خطايا أولادي، لأنهم حادوا عن ناموس الله.
\par 13 ولم يعرفوا فرائضه، ولم يسلكوا في طرق وصاياه، ولم يسلكوا في سبل التأديب في بره.
\par 14 "ليأتِ سكان صهيون، ويذكروا سبي بنيّ وبناتي الذي جلبه عليهم الرب."
\par 15 لأنه جلب عليهم أمة من بعيد، أمة وقحة ولغتها غريبة، لا تحترم شيخاً ولا تشفق على طفل.
\par 16 لقد أخذوا أبناء الأرملة الأعزاء، وتركوها وحيدة بلا بنات.
\par 17 ولكن ماذا يمكنني أن أساعدك؟
\par 18 لأن الذي جلب عليكم هذه الضربات هو ينقذكم من أيدي أعدائكم.
\par 19 اذهبوا يا أبنائي، اذهبوا لأني قد تركت خرابا.
\par 20 قد خلعت ثوب السلام ولبست مسح صلاتي وأصرخ إلى الأبد في أيامي.
\par 21 تشجعوا يا أبنائي واصرخوا إلى الرب فينقذكم من قوة وأيدي الأعداء.
\par 22 لأن رجائي هو في الأبدي أنه يخلصكم، وقد جاءني فرح من القدوس بسبب الرحمة التي ستأتي إليكم قريبًا من الأزلي مخلصنا.
\par 23 لأني أرسلتكم بالحزن والبكاء، ولكن الله سيعيدكم لي بفرح وسرور إلى الأبد.
\par 24 كما رأى جيران صهيون الآن سبيكم، كذلك سيرون قريبا خلاصكم من عند إلهنا الذي يأتي عليكم بمجد عظيم وبهاء الأبدية.
\par 25 أولادي، اصبروا على الغضب الذي نزل عليكم من الله، لأن عدوكم اضطهدكم، ولكنكم سترون هلاكه قريبًا، وتدوسون على رقبته.
\par 26 لقد سلكت أحبائي طرقًا وعرة، وأخذوا بعيدًا مثل قطيع أسره الأعداء.
\par 27 تعزوا يا أبنائي واصرخوا إلى الله، لأنكم ستذكرون من جلب عليكم هذه الأشياء.
\par 28 فكما كنتم تفكرون في الابتعاد عن الله، كذلك أيضاً إذا رجعتم فاطلبوه عشر مرات أيضاً.
\par 29 لأن الذي جلب عليكم هذه الضربات هو الذي يجلب لكم الفرح الأبدي مع خلاصكم.
\par 30 "خذي قلباً طيباً يا أورشليم، لأن الذي أعطاك هذا الاسم هو الذي يعزيك."
\par 31 ويل للذين ضايقوك وفرحوا بسقوطك.
\par 32 ويل للمدن التي كان بنوك يخدمونها، ويل لتلك التي قبلت بنيك.
\par 33 فكما فرحت بخرابك وفرح بسقوطك، كذلك ستحزن على خرابها.
\par 34 لأني سأزيل فرح جمهورها الكثير، ويتحول كبرياؤها إلى حزن.
\par 35 لأنها ستأتي عليها نار من الأزل طويلة الأمد، وستسكنها الشياطين إلى الأبد.
\par 36 يا أورشليم، انظري حولك نحو المشرق، وانظري الفرح الذي يأتي إليك من الله.
\par 37 هوذا أبناؤك الذين أرسلتهم يأتون، يأتون مجتمعين من المشرق إلى المغرب بكلمة القدوس، فرحين بمجد الله.

\chapter{5}

\par 1 اخلعي ​​يا أورشليم ثوب الحزن والضيق، والبسي زينة المجد الذي من عند الله إلى الأبد.
\par 2 إلبسي ثوبين من البر الذي من الله، وضعي على رأسك إكليلاً من مجد الأبدي.
\par 3 لأن الله سيظهر سطوعك لكل الأرض التي تحت السماء.
\par 4 لأن اسمك يدعى من قبل الله إلى الأبد، سلام البر ومجد عبادة الله.
\par 5 انهضي يا أورشليم وقفي في الأعالي وانظري نحو المشرق وانظري أولادك مجتمعين من المغرب إلى المشرق بكلمة القدوس فرحين بذكر الله.
\par 6 فإنهم ارتحلوا عنك مشياً على الأقدام، واقتادهم أعداؤهم. ولكن الله يحضرهم إليك رافعين في المجد، كأبناء الملكوت.
\par 7 "لأن الله قد أمر أن يهدم كل جبل مرتفع وكل ضفاف طويلة الأمد، وأن يردم الوديان، لكي يسوي الأرض، لكي يسير إسرائيل بسلام في مجد الله،
\par 8 وأيضاً فإن الغابات وكل شجرة طيبة تظلل إسرائيل حسب أمر الله.
\par 9 لأن الله يقود إسرائيل بفرح في نور مجده بالرحمة والبر اللذين من عنده.

\end{document}