\begin{document}

\title{وصية رأوبين}

\chapter{1}

\par \textit{رأوبين، الابن البكر ليعقوب وليئة. ينصح الرجل ذو الخبرة ضد الزنا ويشير إلى الطرق التي يكون الرجال أكثر عرضة للوقوع في الخطأ بها.}

\par 1 نسخة وصية رأوبين، حتى الوصايا التي أعطاها لأبنائه قبل وفاته في السنة المائة والخامسة والعشرين من حياته

\par 2 بعد عامين من وفاة يوسف أخيه، عندما مرض رأوبين، اجتمع أبناؤه وأبناء أبنائه لزيارته

\par 3 فقال لهم: يا أولادي، ها أنا أموت وأذهب في طريق آبائي

\par 4 فرأى هناك يهوذا وجاد وأشير إخوته، فقال لهم: قموني لأخبر إخوتي وأولادي بما أخفيته في قلبي، لأني الآن قد مضيت

\par 5 فقام وقبلهم وقال لهم: اسمعوا يا إخوتي، وأنتم يا أبنائي، أصغوا إلى رأوبين أبيكم في الوصايا التي أوصيكم بها

\par 6 وها أنا أشهد عليكم اليوم إله السماء أنكم لا تسلكون في خطايا الشباب والزنى التي سُكِبتُ فيها، ونجست فراش أبي يعقوب

\par 7 وأقول لكم إنه ضربني بضربة رديئة في حقوي سبعة أشهر، ولولا أن أبي يعقوب صلى من أجلي إلى الرب، لكان الرب قد أهلكني

\par 8 لأني كنت في الثلاثين من عمري حين فعلت الشر أمام الرب، ومرضت سبعة أشهر حتى الموت

\par 9 وبعد ذلك تبت بعزمٍ من نفسي لمدة سبع سنوات أمام الرب

\par 10 ولم أشرب خمرًا ولا مسكرا، ولم يدخل لحم فمي، ولم آكل طعامًا شهيًا، بل حزنت على خطيئتي لأنها كانت عظيمة لم تكن مثلها في إسرائيل

\par 11 والآن اسمعوني يا أبنائي، ما رأيته بشأن أرواح الخداع السبعة، عندما تبت

\par 12 لذلك، تُعيَّن سبعة أرواح ضد الإنسان، وهم القادة في أعمال الشباب

\par 13 ويُعطى له سبعة أرواح أخرى عند خلقه، لكي يتم من خلالها كل عمل إنساني

\par 14 الأول هو روح الحياة، الذي به يُخلق دستور الإنسان

\par 15 والثانية هي حاسة البصر، التي تنشأ بها الرغبة.

\par 16 أما الحاسة الثالثة فهي حاسة السمع التي يأتي معها التعليم.

\par 17 الرابعة هي حاسة الشم، التي تُعطى من خلالها التذوقات لاستنشاق الهواء والتنفس

\par 18 الخامسة هي قوة الكلام، التي بها تأتي المعرفة.

\par 19 السادس هو حاسة التذوق، وبها يحصل أكل اللحوم والأشربة، وبها تحصل القوة، فإن في الطعام أساس القوة.

\par 20 السابعة هي قوة الإنجاب والجماع، والتي من خلالها تدخل الخطايا من خلال حب اللذة

\par 21 ولذلك فهي الأخيرة في ترتيب الخلق، والأولى في ترتيب الشباب، لأنها مليئة بالجهل، وتقود الشباب كأعمى إلى حفرة، وكوحش إلى هاوية.

\par 22 إلى جانب كل هذه، هناك روح ثامنة للنوم، والتي تُحدث بها غيبوبة الطبيعة وغيبوبة الموت

\par 23 مع هذه الأرواح تمتزج أرواح الخطأ.

\par 24 أولاً، روح الزنا متمركزة في الطبيعة وفي الحواس؛

\par 25 الثاني، روح الشراهة في البطن؛

\par 26 ثالثها روح القتال في الكبد والمرارة.

\par 27 الرابع هو روح الخضوع والخداع، حتى يبدو المرء جميلًا من خلال الاهتمام الرسمي

\par 28 الخامس هو روح الكبرياء، وهو أن يكون الإنسان متباهيًا ومتكبرًا

\par 29 السادس هو روح الكذب، والهلاك والغيرة على ممارسة الخداع، والإخفاء عن الأقارب والأصدقاء

\par 30 السابع هو روح الظلم، الذي معه السرقات وأعمال الجشع، لكي يُشبع الإنسان رغبة قلبه؛ لأن الظلم يعمل مع الأرواح الأخرى عن طريق أخذ الهدايا

\par 31 ومع كل هذا، تنضم روح النوم، وهي روح الخطأ والخيال

\par 32 وهكذا يهلك كل شاب، إذ يظلم عقله عن الحق، ولا يفهم شريعة الله، ولا يطيع نصائح آبائه، كما حدث لي أيضًا في شبابي

\par 33 والآن يا أبنائي، أحبوا الحق، فهو يحفظكم: اسمعوا كلام رأوبين أبيكم

\par 34 لا تُعر وجه المرأة اهتمامًا،

\par 35 ولا تُخالط زوجة رجل آخر،

\par 36 ولا تتدخلوا في شؤون النساء.

\par 37 لأنه لو لم أرَ بلهة تستحم في مكان مغطى، لما وقعت في هذا الإثم العظيم

\par 38 لأن عقلي، إذ أخذ في التفكير بعُري المرأة، لم يسمح لي بالنوم حتى ارتكبت ذلك الشيء البغيض

\par 39 فبينما ذهب يعقوب أبونا إلى إسحاق أبيه، ونحن في عدر بقرب أفراتة في بيت لحم، سكرت بلهة ونامت عارية في مخدعها

\par 40 فلما دخلت ورأيت عريها، ارتكبت الفجور دون أن تشعر به، وتركتها نائمة

\par 41 وفي الحال كشف ملاك الله لأبي عن كفري، فجاء وناح عليّ، ولم يلمسها بعد ذلك

\chapter{2}

\par \textit{يواصل روبن سرد تجاربه ونصائحه الجيدة.}

\par 1 لذلك يا أبنائي، لا تهتموا بجمال النساء، ولا تهتموا بشؤونهن؛ بل امشوا بقلب بسيط في مخافة الرب، واجتهدوا في الأعمال الصالحة، وفي الدراسة، وفي رعاية قطعانكم، حتى يعطيكم الرب زوجة يشاء، فلا تتألموا كما تألمت أنا

\par 2 لأنه حتى وفاة والدي، لم تكن لدي الجرأة على النظر في وجهه، أو التحدث إلى أي من إخوتي، بسبب العار

\par 3 وحتى الآن لا يزال ضميري يسبب لي الألم بسبب عدم تقواي.

\par 4 ومع ذلك، فقد عزاني أبي كثيرًا، وصلى من أجلي إلى الرب، لكي يعبر عني غضب الرب، كما أظهر الرب

\par 5 ومنذ ذلك الحين وحتى الآن، كنتُ على حذر ولم أخطئ.

\par 6 لذلك، يا أبنائي، أقول لكم: احفظوا كل ما أوصيكم به، ولن تخطئوا.

\par 7 لأن خطيئة الزنا حفرة للنفس، فهي تفصلها عن الله، وتقربها إلى الأصنام، لأنها تخدع العقل والفهم، وتهبط الشباب إلى الجحيم قبل أوانهم

\par 8 لأن الزنا أهلك كثيرين، لأنه سواء كان الإنسان شيخًا أو شريفًا، أو غنيًا أو فقيرًا، فإنه يجلب على نفسه العار عند بني البشر والسخرية عند بليعار

\par 9 فإنكم سمعتم عن يوسف كيف امتنع عن المرأة، وطهر أفكاره من كل زنا، ووجد نعمة في أعين الله والناس

\par 10 لأن المرأة المصرية فعلت به أشياء كثيرة، واستدعت سحرة، وقدمت له جرعات حب، لكن عزم روحه لم يسمح برغبة شريرة

\par 11 لذلك أنقذه إله آبائك من كل شر ومن كل موت خفي

\par 12 لأنه إن لم يغلب الزنا عقلك، فلن يغلبك بليار أيضًا

\par 13 لأن النساء شريرات يا أولادي؛ ولأنهن لا يملكن سلطانًا أو قوة على الرجل، فإنهن يستخدمن الحيل من خلال الانجذابات الخارجية، ليجذبنه ​​إليهن

\par 14 ومن لم يتمكنوا من سحره بجاذبات خارجية، غلبوه بالمكر

\par 15 علاوة على ذلك، أخبرني ملاك الرب، وعلمني، عنهن، أن النساء يغلب عليهن روح الزنا أكثر من الرجال، وفي قلوبهن يتآمرن على الرجال؛ وبزينتهن يخدعن عقولهن أولًا، وبنظرة العين يغرسن السم، ثم من خلال الفعل المنجز يأخذنهن أسرى

\par 16 لأن المرأة لا تستطيع أن تجبر الرجل علانية، لكنها تخدعه بتصرفات عاهرة

\par 17 فاهربوا إذن من الزنا يا أبنائي، وأوصوا زوجاتكم وبناتكم ألا يزينّ رؤوسهن ووجوههن لخداع العقول، لأن كل امرأة تستخدم هذه الحيل ستكون محجوزة للعقاب الأبدي

\par 18 لأنهم هكذا أغووا المراقبين الذين كانوا قبل الطوفان؛ إذ كانوا ينظرون إليهم باستمرار، اشتهواهم، وتصوروا الفعل في أذهانهم؛ إذ غيروا أنفسهم إلى شكل رجال، وظهروا لهم عندما كانوا مع أزواجهم

\par 19 والنساء اللواتي اشتهين أشكالهن في أذهانهن، أنجبن عمالقة، لأن المراقبين ظهروا لهن وكأنهم يصلون حتى إلى السماء

\par 20 فاحذروا الزنا، وإن أردتم أن تكونوا طاهري البال، فاحفظوا حواسك من كل امرأة

\par 21 وأوصِ النساء كذلك ألا يخالطن الرجال، ليكونوا طاهرات عقولهن

\par 22 فإن الاجتماعات المستمرة، حتى لو لم يتم ارتكاب العمل الشرير، هي بالنسبة لهم مرض لا يمكن علاجه، وبالنسبة لنا تدمير بليار وعار أبدي.

\par 23 لأنه ليس في الزنا فهم ولا تقوى، وكل غيرة تسكن في شهوتها

\par 24 لذلك أقول لكم: إنكم ستغارون من بني لاوي، وستطلبون أن ترتفعوا عليهم، ولكنكم لن تستطيعوا

\par 25 لأن الله سينتقم لهم، وأنتم تموتون موتًا شريرًا. لأن الله أعطى لاوي السيادة، وليهوذا معه، ولي أيضًا، ولدان ويوسف، لنكون حكامًا

\par 26 لذلك آمرك أن تسمع للاوي، لأنه سيعرف شريعة الرب، وسيُعطي أحكامًا للقضاء، وسيذبح عن كل إسرائيل إلى انقضاء الأزمنة، مثل رئيس الكهنة الممسوح الذي تكلم عنه الرب

\par 27 أقسم عليكم بإله السماء أن تفعلوا الحق كل واحد مع قريبه، وأن يكنّ كل واحد المحبة لأخيه

\par 28 واقتربوا من لاوي بتواضع القلب، لكي تأخذوا بركة من فمه

\par 29 لأنه يبارك إسرائيل ويهوذا، لأن الرب اختاره ملكًا على كل الأمة

\par 30 وانحني أمام نسله، لأنه سيموت من أجلنا في حروب ظاهرة وغير ظاهرة، وسيكون بينكم ملكًا أبديًا

\par 31 ومات رأوبين بعد أن أوصى بنيه بهذه الوصايا. فوضعوه في نعش حتى أصعدوه من مصر، ودفنوه في حبرون في المغارة التي كان أبوه فيها

\par \textit{الحواشي}

\par \textit{223:1 انظر الكتاب الثاني من قصة آدم وحواء، الفصل العشرون}


\end{document}