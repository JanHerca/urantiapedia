\begin{document}

\title{المكابيين الخامس}


\chapter{1}

\par \textit{محاولة هيليودوروس للاستيلاء على الخزانة. عُيّن عير من قبل ملوك اليونان}

\par 1 وقد أمر ملوك الأمم اليونانية بإرسال مبالغ كبيرة من المال إلى المدينة المقدسة كل عام، وتسليمها للكهنة، ليضيفوها إلى خزانة بيت الله، فضةً للمتصدقين [الأيتام] والأرامل

\par 2 وكان سلوقس ملكًا في مقدونية، وكان له صديق، أحد قواده، يُدعى هليودورس. أُرسل هذا الرجل لنهب الخزانة، وأخذ ما فيها من مال

\par 3 عندما انتشر هذا الأمر في الخارج، تسبب في حزن شديد بين المواطنين؛ وكانوا يخشون أن يمضي هيليودوروس إلى أبعد من ذلك؛

\par 4 حيث لم تكن لديهم سلطة كافية لمنعه من تنفيذ أوامره

\par 5 لذلك لجأوا جميعًا إلى الله طلبًا للمساعدة، وفرضوا صومًا عامًا، وتضرعوا بتواضع، وركوع، وبكاء شديد؛

\par 6 يلبسون المسوح، ويتدحرجون في الرماد، مع أونيا رئيس الكهنة والرؤساء الآخرين، والشيوخ، حتى عامة الشعب، والنساء والأطفال

\par 7 وفي الغد، جاء هليودورس إلى بيت الله مع قافلة من أتباعه، ودخل البيت مع جنوده، وكان هو نفسه راكبًا على جواد، وكان يبحث عن المال

\par 8 فأرسل الله العظيم الصالح عليه صوتًا عظيمًا رهيبًا، فرأى رجلاً مسلحًا بأسلحة حرب، راكبًا على فرس كبير، ويتقدم نحوه:

\par 9 فاستولى عليه الخوف والارتجاف، فتقدم إليه ذلك الرجل، وجذبه عن سرجه، وضربه بعنف على الأرض

\par 10 فأصيب برعب شديد، وفقد صوابه، فأصبح أخرس

\par 11 فلما رأى خدامه ما حل به، ولم يستطيعوا أن يروا أحدًا فعل به هذه الأشياء، حملوه بكل عجل إلى بيته

\par 12 وبقي عدة أيام لا يتكلم ولا يتناول طعامًا

\par 13 لذلك ذهب كبار أصدقائه إلى أونيا الكاهن، متوسلين إليه أن يرضيه، وأن يتوسل إلى الإله العظيم الصالح ألا يعاقبه

\par 14 ففعل ذلك أونياس، فشُفي هليودورس من مرضه.

\par 15 فرأى في رؤياه الشخص الذي رآه في المقدس يأمره أن يذهب إلى أونياس الكاهن ويسلم عليه ويكرمه تكريماً لائقاً، ويقول له إن الله العظيم الصالح قد سمع صلاته وشفاه بناء على طلب أونياس.

\par 16 فأسرع هليودورس إلى أونيا الكاهن، فسقط فسلم عليه، وأعطاه نقودًا من أنواع مختلفة، طالبًا منه أن يضيفها إلى ما في الخزانة

\par 17 ثم انطلق من أورشليم إلى بلاد مقدونية، وأخبر الملك سلوقس بما حدث له، وتوسل إليه ألا يفرض عليه أن يكون نائباً له في أورشليم.

\par 18 لذلك تعجب الملك من الأمور التي ذكرها له هليودورس، وأمره بنشرها للعالم

\par 19 وحرص على إبعاد رجاله عن القدس، وزاد من الهدايا التي اعتاد إرسالها إليها سنويًا، بسبب ما حل بهليودورس

\par 20 وزاد الملوك على الفضة التي أمروا بإعطائها للكهنة، ليُنفقوا على الأيتام والأرامل، وكذلك على ما يُنفق على الذبائح

\chapter{2}

\par \textit{تاريخ ترجمة الكتب الأربعة والعشرين من اللغة العبرية إلى اللغة اليونانية، لبطليموس ملك مصر.}

\par 1 كان هناك رجل مقدوني اسمه بطليموس، موهوب بالمعرفة والفهم، ولما كان يقيم في مصر، جعله المصريون ملكًا على بلاد مصر

\par 2 لذلك، ولأنه كان مسكونًا برغبة في طلب المعرفة المتنوعة، جمع جميع كتب الحكماء من كل حدب وصوب

\par 3 وإذ كان حريصًا على الحصول على "الأربعة والعشرين كتابًا"، كتب إلى رئيس الكهنة في أورشليم أن يرسل إليه سبعين شيخًا من بين الماهرين في تلك الكتب؛ وأرسل إلى الكاهن رسالة مع هدية

\par 4 فلما وصلت رسالة الملك إلى الكاهن، اختار سبعين رجلاً من أهل العلم، وأرسلهم مع رجل اسمه أليعازار، كان متفوقًا في الدين والعلم والمعرفة، فانطلق إلى مصر

\par 5 ولما أُبلغ الملك بقدومهم، أمر بإعداد سبعين مسكنًا، واستضافة الرجال فيها

\par 6 كما أمر بتعيين سكرتير لكل واحد منهم، ليقوم بتدوين تفسير هذه الكتب باللغة والخط اليونانيين

\par 7 كما نهى أن يتواصل أيٌّ من هؤلاء مع أيٍّ من رفاقه؛ خشية أن يتفقوا معًا على إجراء أي تغيير في تلك الكتب

\par 8 فأخذ الأمناء من كل واحد منهم ترجمة "الكتب الأربعة والعشرين".

\par 9 ولما فرغت من الترجمات، أحضرها ألعازار إلى الملك، وقارنها أمامه، فوجدت أنها متفقة

\par 10 ففرح الملك فرحًا شديدًا، وأمر بتقسيم مبلغ كبير من المال بين الحضور. لكن أليعازار خاناس كافأهم بمكافأة سخية

\par 11 وأطلق في ذلك اليوم أيضًا كل أسير وُجد في مصر من سبط يهوذا وبنيامين، ليرجعوا إلى أرضهم سورية

\par 12 كان عددهم حوالي مائة وثلاثين ألفًا.

\par 13 ثم أمر أن تقسم بينهم أموال، حتى صار لكل واحد منهم دينارا واحدا، فأخذوها وانصرفوا إلى أرضهم.

\par 14 ثم أمر بصنع مائدة عظيمة من أنقى أنواع الذهب، بحيث تكون كبيرة بما يكفي لاحتواء تمثيل لكامل أرض مصر، وصورة لنهر النيل، من بداية مجراه إلى نهايته في مصر، مع تفرعاته المختلفة عبر البلاد، وكيف يغسل الأرض كلها

\par 15 كما أمر بتجهيز المائدة بالعديد من الأحجار الكريمة.

\par 16 فصنعت هذه المائدة وأكملت نقشها ورصعت بالحجارة الكريمة وأتيت إلى مدينة أورشليم هدية للبيت العظيم.

\par 17 ولما وصل سالمًا، وُضع في المنزل، وفقًا لأمر الملك. وفي الحقيقة، لا يُحسب للناس مثله أبدًا، لجمال الصور، وإتقان الصنعة

\chapter{3}

\par \textit{تاريخ اليهود. رواية لما حل باليهود في عهد الملك أنطيوخس؛ والمعارك التي دارت بينهم وبين قادته؛ وإلى أي مدى وصل في النهاية.}

\par 1 كان رجل من ملوك مقدونيا يُدعى أنطيوخس، وكان من بين أعماله هذا:

\par 2 أنه عندما توفي بطليموس ملك مصر المذكور أعلاه، ذهب مع جيوشه لمهاجمة بطليموس الثاني. وبعد أن غزا بطليموس وقتله، استولى على بلاده مصر، واستولى عليها

\par 3 من هنا، ومع ازدياد قوة شئونه، أخضع جزءًا كبيرًا من الأرض؛ فأطاعه ملك فارس وآخرون

\par 4 لذلك ارتفع قلبه، وانتفخ بالكبرياء، فأمر أن تُصنع صور على صورته، لكي يسجد لها الناس، لتمجيده وإكرامه

\par 5 وعندما صُنعت هذه، أرسل رسلًا إلى جميع مناطق إمبراطوريته، يأمرهم بعبادتها وتبجيلها. وافقت الأمم على هذه الأوامر، خوفًا ورهبةً من طغيانه

\par 6 كان في ذلك الوقت في اليهودية ثلاثة رجال، أسوأ البشر على الإطلاق؛ وكان لكل واحد منهم، كما هو الحال، صلة بالرذيلة نفسها. كان اسم أحد هؤلاء الثلاثة مينيلوس؛ والثاني سمعان؛ والثالث ألكيمس

\par 7 وفي ذلك الوقت ظهرت صور معينة، شاهدها أهل أورشليم في الهواء لمدة أربعين يومًا: كانت صور رجال يركبون على خيول نارية يتقاتلون مع بعضهم البعض

\par 8 فذهب أولئك الرجال الأشرار إلى أنطيوخس، ليحصلوا منه على بعض السلطة، حتى يتمكنوا بسهولة من ارتكاب كل ما يريدون، من الزنا ونهب أموال الناس؛ وباختصار، ليتمكنوا من السيطرة على الباقين وإخضاعهم. فقالوا له:

\par 9 "أيها الملك، قد ظهر في الجو فوق أورشليم فرسان ناريون يتقاتلون، ولذلك فرح العبرانيون قائلين: «إن هذا ينبئ بموت الملك أنطيوخس».

\par 10 صدق الملك هذه الكلمات، فامتلأ غضبًا، وسار إلى القدس في أقصر وقت ممكن؛ ووصل إلى الأمة دون أن يكون مُنذرًا على الإطلاق بقدومه

\par 11 وهاجم رجاله السكان وقتلوهم بالسيف، محدثين قتلاً عظيماً، وجرحوا كثيرين، وسبوا جمعاً غفيراً

\par 12 لكن بعض الهاربين فروا إلى الجبال والغابات، حيث استمروا لفترة طويلة، يتغذون على الأعشاب

\par 13 بعد ذلك، قرر أنطيوخس مغادرة البلاد.

\par 14 ولكن الشر الذي صنعه بالأمة لم يكفيه، فترك بدلاً منه رجلاً اسمه فيلكس، وأمره أن يجبر اليهود على عبادة صورته، وأكل لحم الخنزير.

\par 15 ففعل فيلكس ذلك، إذ أرسل إلى الشعب ليطيعوا الملك في الأمور التي أمره بها

\par 16 لكنهم رفضوا أن يفعلوا ما دعوا إليه، لذلك قتل جمعًا كبيرًا منهم، وحفظ أولئك الأشرار وعائلاتهم، ورفع كرامتهم

\chapter{4}

\par \textit{تاريخ وفاة ألعازار الكاهن}

\par 1 "ثم قُبض على إليعازار الذي ذهب مع الأطباء إلى بطليموس، وكان آنذاك رجلاً شيخاً كبيراً في السن، تسعين عاماً، ووُضع أمام فيلكس؛

\par 2 الذي قال له: «يا أليعازار، أنت حقًا رجل حكيم وفطن، وقد أحببتك سنينًا طويلة، ولذلك لا أتمنى موتك

\par 3 فأطيعوا الملك، واعبدوا تمثاله، وكلوا من ذبائحه، وانطلقوا آمنين

\par 4 فأجابه أليعازار: "لن أتخلى عن طاعتي لله من أجل طاعة الملك".

\par 5 فتقدم فيلكس وهمس له: "احرص على أن ترسل في طلب أحد ليحضر لك لحمًا من قرابينك التي تضعها على مائدتي،

\par 6 وكل من ذلك أمام الناس، ليعلموا أنك أطعت الملك، فتنجو بحياتك دون أن يلحق بدينك أي ضرر

\par 7 أجابه أليعازار: "أنا لا أطيع الله تحت أي نوع من الاحتيال، بل سأتحمل عنفكم هذا. لأني رجل شيخ في التسعين من عمره، وقد ضعفت عظامي، وذبل جسدي

\par 8 إذا تحملتُ بروح شجاعة تلك العذابات، التي يتراجع عنها حتى أشجع الشباب خوفًا؛ فإن شعبي وشباب أمتي سيقلدونني بشجاعة، وسيقولون:

\par 9 «كيف لا نستطيع أن نتحمل الآلام التي تحملها من هو أقل منا قوةً، وأقل منا جسدًا وعظامًا؟»

\par 10 وهذا سيكون أفضل بالنسبة لي من أن أخدعهم بالتظاهر بالطاعة للملك.

\par 11 فإنهم سيقولون حينئذ: إذا كان ذلك الشيخ الهرم، الحكيم الفطن، متمسكًا بالحياة ومتغلبًا على آلام الأمور المؤقتة، متخليًا عن دينه؛ فإنه سيكون من حقنا حقًا ما كان من حقه، لأنه رجل شيخ حكيم، ويجب علينا أن نتبعه.

\par 12 لذلك أُفضّل أن أموت، تاركًا لهم ثباتًا في الدين وصبرًا على الطغيان؛ على أن أعيش؛ بعد أن أضعف ثباتهم في طاعة ربهم واتباع أوامره؛ حتى "يُجعلوا من خلالي سعداء، لا تعساء".

\par 13 فلما سمع فيلكس عزم أليعازار، غضب عليه بشدة، وأمر بتعذيبه بأنواع مختلفة من التعذيب، حتى دخل في صراع مميت يائسًا، وقال:

\par 14 «أنت يا الله تعلم أنني كان بإمكاني أن أنقذ نفسي من المشاكل التي وقعت فيها، من خلال طاعة شخص آخر غيرك

\par 15 مع ذلك، لم أفعل هذا؛ بل فضّلتُ طاعتك، واعتبرتُ كل العنف المُقدّم لي نورًا، من أجل الثبات في طاعتك

\par 16 والآن، أنا أقلل من شأن الأمور التي حدثت لي وفقًا لرضاك، وأدعمها قدر استطاعتي

\par 17 لذلك أدعوك أن تقبل مني هذا، وأن تمتني قبل أن أضعف في التحمل

\par 18 فاستجاب الله لدعائه، فمات في الحال.

\par 19 ولكنه ترك شعبه ملتزمين بعبادة إلههم، ومتمتعين بالقوة والتحمل في الدين، والصبر على مواجهة التجارب التي تنتظرهم.

\chapter{5}

\par \textit{تاريخ وفاة الإخوة السبعة.}

\par 1 وبعد ذلك قبض على سبعة إخوة وأمهم وأرسلوهم إلى الملك لأنه لم يكن قد ابتعد بعد عن أورشليم.

\par 2 ولما أُحضِروا إلى الملك، أُحضِر أحدهم إلى حضرته، فأمره بالتخلي عن دينه:

\par 3 لكنه رفض وقال له: "إذا كنت تعتقد أنك ستعلمنا الحقيقة لأول مرة، فالأمر ليس كذلك:

\par 4 لأن الحقيقة هي ما تعلمناه من آبائنا، والذي ألزمنا أنفسنا به باعتناق عبادة الله وحده، ومراعاة الشريعة باستمرار؛ ولن نحيد عن هذا بأي حال من الأحوال

\par 5 فغضب الملك أنطيخوس من هذه الكلمات، وأمر بإحضار مقلاة من حديد ووضعها على النار

\par 6 ثم أمر بقطع لسان الشاب، وقطع يديه ورجليه، وسلخ جلد رأسه، ووضعه في المقلاة. ففعلوا به كذلك

\par 7 ثم أمر بإحضار مرجل نحاسي كبير ووضعه فوق النار، وألقي فيه بقية جسده

\par 8 ولما قارب الرجل على الموت أمر بإزالة النار عنه لكي يتعذب مدة أطول، وكان يقصد بهذه الأفعال إرهاب أمه وإخوته.

\par 9 ولكن في الواقع، فقد منحهم بذلك شجاعة وقوة إضافيتين، للحفاظ على دينهم بثبات، وتحمل كل تلك العذابات التي يمكن أن يفرضها عليهم الطغيان

\par 10 فلما مات الأول، أُحضر الثاني أمامه، فقال له بعض الخدم: «أطع الأوامر التي يأمرك بها الملك، لئلا تهلك كما هلك أخوك».

\par 11 فأجاب: «لستُ أضعف روحًا من أخي، ولا أتأخر عنه في الإيمان. هات نارك وسيفك، ولا تنقص شيئًا مما فعلته بأخي». ففعلوا به كما فعلوا بأخيه

\par 12 ونادى الملك وقال له: "اسمع يا وحش القسوة على البشر، واعلم أنك لن تكسب منا شيئًا سوى أجسادنا؛ لكنك لن تحصل على أرواحنا بأي حال من الأحوال؛ وستذهب هذه قريبًا إلى خالقها،

\par 13 الذين سيعيدهم إلى أجسادهم، عندما يُحيي أموات أمته وقتلى شعبه

\par 14 وأُخرج الثالث، فأشار بيده وقال للملك: "لماذا تُخيفنا أيها العدو؟

\par 15 اعلموا أن هذا أُرسل علينا من السماء، ونحن نجتازه أيضًا، شاكرين لله، ومنه نرجو أجرنا

\par 16 وأُعجب الملك ومن وقفوا بالقرب منه بشجاعة الشاب، وثبات عقله، وحسن حديثه. ثم أصدر أوامره، فقُتل

\par 17 وأُخرج الرابع، فقال: "في سبيل دين الله، نبيع أرواحنا، ونؤجرها، لنطلب الأجر منه، في ذلك اليوم الذي لا عذر لكم فيه في الدينونة، ولا تقدرون على تحمل عذاباتكم."

\par 18 أمر الملك فقتله.

\par 19 فأخرج الخامس وقال له: لا تظن في نفسك أن الله تركنا بسبب ما أرسله علينا.

\par 20 ولكن إرادته حقًا هي أن يُظهر لنا الشرف والمحبة من خلال هذه الأشياء؛ وسوف ينتقم لنا منك ومن نسلك

\par 21 وأمر الملك فقُتل.

\par 22 وأخرج السادس فقال: إني أعترف بذنوبي لله، ولكني أومن أنها ستغفر لي بموته.

\par 23 لكنكم الآن خالفتم الله بقتلكم من يعتنقون دينه، وسيجازيكم أعمالكم، ويستأصلكم من أرضه. فأمر به فقتل

\par 24 وأُخرج السابع، وكان صبيًا.

\par 25 ثم نهضت أمه بلا خوف ولا تأثر، ونظرت إلى جثث أطفالها:

\par 26 فقالت: يا أبنائي، لا أعرف كيف حملتُ كل واحد منكم حين حملتُ به. ولم تكن لي القدرة على أن أمنحه نسمة، أو أن أخرجه إلى نور هذا العالم، أو أن أمنحه الشجاعة والفهم.

\par 27 ولكن الله العظيم الصالح نفسه شكَّله حسب إرادته، وأعطاه صورة حسب مسرته

\par 28 وأتى به إلى العالم بقدرته، وجعل له أجلاً، وأحكامًا صالحة، وشريعة دينية كما يرضيه

\par 29 لكنكم الآن قد بِعتم لله أجسادكم التي شكَّلها بنفسه، وأرواحكم التي خلقها، ورضيتم بأحكامه التي قضى بها

\par 30 لذلك، طوبى لكم في الأشياء التي حصلتم عليها بسعادة؛ وطوبى لكم في الأشياء التي "انتصرتم فيها".

\par 31 ظن أنطيوخس، عندما رآها تنهض، أنها فعلت ذلك بسبب خوفها الشديد على طفلها؛ وظن تمامًا أنها على وشك أن تأمره بطاعة الملك، حتى لا يهلك كما هلك إخوته

\par 32 فلما سمع كلامها، خجل واحمرّ خجلاً، وأمر بإحضار الصبي إليه، لكي يحثّه ويقنعه بحب الحياة، ويثنيه عن الموت

\par 33 لئلا يُنظر إلى جميع هؤلاء على أنهم يعارضون سلطته، فيتبعهم كثيرون غيرهم

\par 34 فلما أُحضِر إليه، حثّه بالكلام، ووعده بالغنى، وأقسم له أنه سيجعله خليفة على نفسه

\par 35 ولكن عندما لم يتأثر الصبي إطلاقًا بكلماته، ولم يُعرها اهتمامًا، التفت الملك إلى أمه وقال لها:

\par 36 «أيتها المرأة السعيدة، اشفقي على ابنك هذا، الذي لم يبقَ على قيد الحياة إلا أنت؛ وحثيه على الامتثال لأوامري، والنجاة من تلك المعاناة التي حلت بإخوته.»

\par 37 فقالت: «أحضروه إلى هنا لأُحَثَّه بكلام الله».

\par 38 ولما أحضروه إليها، انزوت عن الجمع، ثم قبلته، وضحكت ساخرة مما قاله لها أنطيوخس:

\par 39 ثم قال له: يا بني، تعال الآن، أطعني، لأني ولدتك، وأرضعتك، وربيتك، وعلمتك الدين الإلهي

\par 40 انظر الآن إلى السماء، والأرض، والماء، والنار؛ وافهم أن الإله الحقيقي الوحيد هو الذي خلق هذه الأشياء بنفسه؛ وصنع الإنسان من لحم ودم، الذي يعيش لفترة قصيرة، ثم يموت

\par 41 لذلك اتقوا الإله الحقيقي الذي لا يموت، وأطيعوا الكائن الحقيقي،

\par 42 الذي لا يغيّر وعوده، ولا تخف من هذا العملاق المجرد، ومُت من أجل دين الله كما مات إخوتك

\par 43 لأنه لو استطعت يا بني أن ترى مسكنهم الكريم، ونور مسكنهم، والمجد الذي بلغوه، فلن تطيق عدم اتباعهم

\par 44 وفي الحقيقة، آمل أيضًا أن يُعِدّني الله العظيم الصالح، وأن أتبعكم عن كثب

\par 45 ثم قال الغلام: «اعلموا أني أطيع الله جيدًا، ولن أطيع أوامر أنطيوخس. لذلك، لا تتأخروا عن دعوني أتبع إخوتي؛ لا تمنعوني من المغادرة إلى المكان الذي ذهبوا إليه.»

\par 46 ثم قال للملك: "ويل لك من الله! إلى أين تهرب منه؟ إلى أين تلجأ؟ أو إلى من تتوسل حتى لا ينتقم منك؟"

\par 47 لقد أحسنتَ إلينا حقًا، عندما كنتَ تنوي أن تُسيء إلينا: لقد أسأتَ إلى نفسك، ودمرتَها، بينما كنتَ تظنُّ أنك ستُحسن إليها

\par 48 نحن الآن في طريقنا إلى حياة لن يتبعها الموت أبدًا؛ وسنعيش في نور لن يمحوه الظلام أبدًا

\par 49 لكن مسكنكم سيكون في المناطق الجهنمية، مع عقوبات شديدة من الله

\par 50 وأنا أثق أن غضب الله سيزول عن شعبه، بسبب ما عانيناه من أجلهم:

\par 51 بل سيعذبكم في هذه الدنيا، ويوصلكم إلى موتة بائسة، وبعد ذلك سترحلون إلى عذاب أبدي

\par 52 فغضب أنطيوخس، لما رأى أن الغلام يعارض سلطته، فأمر بتعذيبه أشد من تعذيب إخوته. فتم ذلك، ومات

\par 53 لكن أمهم توسلت إلى الله، وتوسلت إليه أن تتبع ابنيها، فماتت في الحال

\par 54 ثم انطلق أنطيوخس إلى بلده مقدونية، وكتب إلى فيلكس وإلى الولاة الآخرين في سورية أن يقتلوا جميع اليهود، ما عدا من يعتنق دينه

\par 55 فأطاع عبيده أمره، فقتلوا جمعًا من الرجال

\chapter{6}

\par \textit{تاريخ متثيا الكاهن الأعظم، ابن يوحنان، وهو ابن حصماي الكاهن}

\par 1 هرب رجل اسمه متتيا بن يوحنان إلى أحد الجبال المحصنة. فهرب إليه الرجال المتفرقون، واختبأ بعضهم في أماكن منعزلة

\par 2 ولكن بعد أن رحل أنطيوخس إلى مسافة أبعد من البلاد، أرسل متثيا ابنه يهوذا سرًا إلى مدن يهوذا؛

\par 3 ليشهد لهم بصحته وصحة شعبه، وليتمنى أن يأتي إليه كل من ألهمته الشجاعة وسعة الصدر والحماسة للدين ولزوجاتهم وأطفالهم

\par 4 فخرج إليه أناس من كبار طبقات الشعب الذين كانوا قد تخلفوا، فلما جاءوا إليه قالوا لهم:

\par 5 «لم يبقَ لنا إلا الدعاء إلى الله، والثقة به، ومحاربة أعدائنا، عسى الله أن يمنحنا العون والنصر عليهم.»

\par 6 فوافق الشعب على رأي متثيا، وعملوا به

\par 7 فأخبر فيلكس، فزحف إليهم بجيش عظيم.

\par 8 وبلغه، أثناء مسيرته، أن نحو ألف من شعب اليهود، رجالًا ونساءً مختلطين، كانوا مجتمعين معًا، ويسكنون في كهف معين، حتى يتمكنوا من الحفاظ على طريقتهم الخاصة في العبادة

\par 9 ثم توجه إليهم مع بعض جنوده، وأرسل رؤساء رجاله مع بقية الجيش لمحاربة متثيا

\par 10 وطلب فيلكس من الذين في المغارة أن يخرجوا إليه ويقبلوا الدخول في دينه، فرفضوا

\par 11 عندها هددهم بأنه سيضع دخانًا تحتهم، فاحتملوا ذلك ولم يخرجوا إليه، فوضع دخانًا تحتهم، فماتوا جميعًا

\par 12 ولما كان قادة جيشه يزحفون على متثيا، وجاءوا إليه وهو مستعد للقتال؛

\par 13 ذهب إليه أحد القادة، من ذوي الدم النبيل، واقترح عليه طاعة الملك، وألا يعارض سلطته؛ حتى يعيش هو ومن معه، ولا يهلكوا

\par 14 فقال له: "أنا أطيع الله الملك الحقيقي، ولكن أطع ملكك وافعل ما تراه مناسبًا." ثم كف عن الكلام

\par 15 فبدأوا ينصبون له فخاخًا.

\par 16 فجاء رجل من أشرّ اليهود الذين كانوا معهم، وحرّضهم على الزحف عليه وإعداد حرب

\par 17 فاندفع متثياس نحوه بسيفه المسلول، وقطع رأس اليهودي. ثم ضرب القائد الذي كان اليهودي يخاطبه، فقتله هو أيضًا

\par 18 فلما رأى أصحاب متثياس ما فعله، أسرعوا إليه، فاندفعوا إلى معسكر العدو، وقتلوا منهم أعدادًا كثيرة، وهزموهم. وبعد ذلك، طاردوا الهاربين حتى قتلوا جميعهم

\par 19 بعد ذلك، نفخ متثيا في البوق، ونادى بحملة على فيلكس. فدخل هو ورفاقه أرض يهوذا، واستولوا على مدن كثيرة جدًا منها

\par 20 وأراحهم الله العلي على يديه من قواد أنطيوخس، فرجعوا إلى عبادتهم، وتراجعت جيوش أعدائهم من أمامهم

\chapter{7}

\par \textit{رواية وفاة متثيا، وأعمال يهوذا ابنه بعده.}

\par 1 فمرض متثيا. ولما قاربت وفاته، دعا أبناءه الخمسة وقال لهم:

\par 2 «أعلم يقينًا أن حروبًا كثيرة وعظيمة ستُشعل في أرض يهوذا، من أجل [أو بسبب] تلك الأمور التي من أجلها حرضنا الله العظيم الصالح على خوض حرب ضد أعدائنا

\par 3 ولكني أوصيكم أن تخافوا الله، وتتوكلوا عليه، وتكونوا غيورين على الشريعة، وعلى الأقداس، وعلى الشعب أيضًا؛

\par 4 وأعدوا أنفسكم لخوض الحرب ضد أعدائها، ولا تخافوا الموت، لأنه بلا شك مكتوب على جميع الناس.

\par 5 لذلك، إذا نصركم الله، فقد نلتم على الفور ما كنتم تتوقون إليه: ولكن إذا سقطتم، فلا يُعد ذلك خسارة لكم في نظره

\par 6 ومات متثيا ودُفن، ففعل بنوه كما أمرهم، واتفقوا على أن يجعلوا يهوذا أخاهم قائدًا عليهم

\par 7 وكان يهوذا أخوهم أفضلهم جميعًا مشورةً وأشجعهم قوةً

\par 8 فأرسل فيلكس جيشًا عليهم بقيادة رجل يُدعى سيرون، فهزمه يهوذا ورفاقه، وقتل عددًا كبيرًا

\par 9 فذاع صيت يهوذا، وازداد في آذان الناس جدًا، وخافه كل الأمم التي حوله خوفًا شديدًا

\par 10 فأُخبر الملك أنطيوخس بما فعله متثيا وابنه يهوذا

\par 11 وصل خبر ذلك أيضًا إلى ملك الفرس، فخدع أنطيوخس، وتخلى عن صداقته، واقتدى بيهوذا

\par 12 مما سبب لأنطيوخس قدرًا كبيرًا من القلق، فاستدعى إليه أحد ضباط بيته يُدعى ليسياس، وهو رجل قوي وشجاع، وقال له:

\par 13 لقد عزمتُ الآن على الذهاب إلى بلاد فارس لخوض الحرب، وأودُّ أن أترك ابني مكاني، وأن آخذ معي نصف جيشي، وأترك ​​الباقي مع ابني

\par 14 وها أنا قد أعطيتك حكم ابني، وحكم الرجال الذين أتركهم معه

\par 15 وأنتم تعلمون حقًا ما فعله متثيا ويهوذا بأصدقائي ورعيتي

\par 16 لذلك، أرسل واحدًا ليقود جيشًا قويًا إلى أرض يهوذا، وأمره بمهاجمة أرض يهوذا بالسيف، واقتلاعهم، وهدم مساكنهم، وإتلاف كل أثر لهم

\par 17 ثم انطلق أنطيوخس إلى بلاد فارس.

\par 18 فأعدّ ليسياس ثلاثة قوادٍ أشداء شجعان، ماهرين في الحرب، منهم بطليموس، ونيكانور، وجورجياس.

\par 19 وأرسل معهم أربعين ألف جندي مختار وسبعة آلاف فارس. وأمرهم أيضًا بإحضار جيش من السوريين والفلسطينيين، وأمرهم باستئصال اليهود تمامًا

\par 20 فساروا ومعهم جماعة من التجار، ليبيعوا لهم الأسرى الذين كانوا على وشك الحصول عليهم من بين اليهود

\par 21 فبلغ الخبر يهوذا بن متثيا، فذهب إلى بيت الإله العظيم الصالح

\par 22 وجمع رجاله، وأمرهم بالصوم والدعاء والصلوات إلى الله العظيم الصالح؛ وأمرهم أن يتوسلوا إليه من أجل النصر على أعدائهم؛ وهو ما فعلوه

\par 23 وبعد ذلك جمع يهوذا رجاله، وجعل على كل ألف رئيسًا، وعلى كل مائة رئيسًا، وعلى كل خمسين رئيسًا، وعلى كل عشرة رئيسًا.

\par 24 ثم أمر بنشر النداء بالبوق في جميع أنحاء جيشه، بأن كل من كان خائفًا، ومن أمر الله بتسريحه من الجيش، يجب أن يعود إلى منزله

\par 25 فعاد عدد كبير، وبقي معهم سبعة آلاف رجل من الأقوياء والشجاعين، ماهرين في الحروب ومعتادين عليها، ولم يهرب منهم أحد قط، وساروا ضد أعدائهم

\par 26 ولكن عندما اقتربوا منهم، صلى يهوذا إلى ربه، متوسلاً إليه أن يصرف عنه شر عدوه، وأن يساعده وينصره

\par 27 ثم أمر الكهنة بنفخ الأبواق، ففعلوا، ودعا جميع رجاله الله، واندفعوا على جيش نكانور

\par 28 فأعطاهم الله نصرًا عليهم، فهزموه ورجاله، فقتلوا منهم تسعة آلاف رجل، وتشتت الباقون

\par 29 فعاد يهوذا وجماعته إلى معسكر نكانور، ونهبوا ما فيه، ونهبوا أموال التجار كثيرة جدًا، وأرسلوها لتُقسم على المرضى

\par 30 وقعت هذه المعركة في اليوم السادس من الأسبوع، ولذلك بقي يهوذا ورجاله في نفس المكان حتى انقضى يوم السبت

\par 31 ثم ساروا نحو بطليموس وجورجياس، فوجدوهما وهزموهما، وانتصروا عليهما، وقتلوا عشرين ألفًا من جنودهما

\par 32 فهرب بطليموس وجورجياس، فطاردهما يهوذا وجماعته، إلا أنه لم يستطع اللحاق بهما، لأنهما اتخذا مدينة ذات صنمين، وتحصنا فيها ببقية جيشهما

\par 33 فهاجم يهوذا فيلكس، فانهزم أمامه. فطارده يهوذا. فجاء إلى بيت قريب، ودخله وأغلق الأبواب، لأنه كان بيتًا محصنًا

\par 34 فأمر يهوذا فأشعل النار، فاحترق البيت، وأُحرق فيلكس فيه. فانتقم منه يهوذا لأجل هليازار والآخرين الذين قتلهم فيلكس

\par 35 بعد ذلك عاد الناس إلى القتلى، وأخذوا غنائمهم ودروعهم؛ أما أفضل الغنائم فأرسلوها إلى الأرض المقدسة

\par 36 لكن نيكانور غادر متنكرًا، وعاد إلى ليسياس، وأخبره بكل ما حدث له ولرفاقه

\chapter{8}

\par \textit{قصة عودة أنطيوخس، ودخوله أرض يهوذا، والمرض الذي أصابه، ومات بسببه في رحلته.}

\par 1 لكن أنطيوخس عاد من بلاد فارس طائرًا، وقد تفكك جيشه

\par 2 ولما علم بما حدث لجيشه الذي أرسله ليسياس، ولجميع رجاله، خرج بجيش عظيم، سائرًا إلى أرض يهوذا

\par 3 وعندما وصل في طريقه إلى منتصف رحلته، ضرب الله جيشه بأسلحة عظيمة للغاية:

\par 4 لكن هذا لم يمنعه من رحلته؛ بل أصرّ عليها، متلفظًا بكل أنواع الوقاحة على الله، قائلًا إنه لا أحد يستطيع أن يثنيه، أو يمنعه من تحقيق أهدافه المحددة

\par 5 لذلك ضربه الإله العظيم الصالح أيضًا بقرحات أصابت جسده كله، لكنه مع ذلك لم يكف ولم يمتنع عن رحلته؛

\par 6 بل كان أكثر امتلاءً بالغضب، واشتعلت فيه رغبة شديدة في الحصول على ما عزم عليه، وتنفيذ قراره

\par 7 وكان في جيشه فيلة كثيرة جدًا. فحدث أن هرب أحدها وأصدر زئيرًا، فركضت الخيول التي كانت تجر السرير الذي كان أنطيوخس يرقد عليه، وألقته خارجًا

\par 8 ولأنه كان سمينًا وبدينًا، كانت أطرافه مصابة بكدمات، وبعض مفاصله مخلوعة

\par 9 واشتدت رائحة قرحه الكريهة، التي كانت تُصدر بالفعل رائحة كريهة، لدرجة أنه لم يعد بإمكانه هو نفسه تحملها، ولا حتى أولئك الذين اقتربوا منه

\par 10 فلما سقط، حمله عبيده على أكتافهم. ولكن لما اشتدت الرائحة الكريهة، طرحوه أرضًا ومضوا بعيدًا

\par 11 لذلك، إذ أدرك الشرور التي أحاطت به، آمن يقينًا أن كل ذلك العقاب قد نزل عليه من الله العظيم الصالح؛ بسبب الأذى والطغيان الذي مارسه تجاه العبرانيين، وسفك دمائهم ظلمًا

\par 12 فخاف فتوجه إلى الله، واعترف بخطاياه، وقال: "يا الله، أنا أستحق حقًا ما أنزلته عليّ، وأنت عادل في أحكامك

\par 13 أنت تُذلّ المرتفع، وتُحطّ المُتكبر. أما أنت فلك العظمة والجلال والجلال والبأس

\par 14 إنني أعترف حقًا بأنني ظلمتُ الناس، وتصرفتُ وحكمتُ ضدهم ظلمًا

\par 15 أدعوك يا الله أن تغفر لي هذا الخطأ، وامح خطيئتي، وامنحني عافيتي، وسيكون همي أن أملأ خزانة بيتك بالذهب والفضة

\par 16 وأن أُفرش أرض بيت قدسك بثياب أرجوانية، وأن أختتن، وأن أُعلن في كل مملكتي أنك أنت وحدك الإله الحقيقي، لا شريك لك، وأنه لا إله غيرك

\par 17 لكن الله لم يسمع صلواته، ولم يقبل دعاءه، بل ازدادت عليه ضيقاته حتى فرغ أمعاؤه، وازدادت قرحاته حتى تساقط لحمه من جسده

\par 18 ثم مات ودُفن مكانه. وملك مكانه ابنه، واسمه أوباتور

\chapter{9}

\par \textit{تاريخ أيام التكريس الثمانية}

\par 1 ولما هزم يهوذا بطليموس ونيكانور وجورجياس وقتل رجالهم، رجع هو وجيشه إلى بلاد البيت المقدس.

\par 2 وأمر بهدم جميع المذابح التي أمر أنطيوخس ببنائها:

\par 3 وأزال جميع الأصنام التي في المقدس، وبنوا مذبحًا جديدًا، وأمر بتقديم ذبائح عليه

\par 4 وصلوا أيضًا إلى الإله العظيم الصالح، أن يُخرج النار المقدسة التي قد تبقى على المذبح:

\par 5 وخرجت نار من بعض حجارة المذبح، فأحرقت الحطب والذبائح، وبقيت النار منه على المذبح حتى سبي الثالث

\par 6 ثم احتفلوا بعيد المذبح الجديد ثمانية أيام، ابتداءً من اليوم الخامس والعشرين من شهر كسلو

\par 7 ثم وضعوا الخبز على مائدة بيت الله، وأوقدوا سرج المنارة

\par 8 وفي كل يوم من هذه الأيام الثمانية، اجتمعوا معًا للصلاة والتسبيح، بل وجعلوها فريضة لكل سنة قادمة

\chapter{10}

\par \textit{تاريخ معارك يهوذا مع جورجياس وبطليموس}

\par 1 وبعد أيام التجديد، سار يهوذا إلى بلاد الأدوميين، إلى جبل سارة، لأن جرجياس كان يقيم هناك

\par 2 فخرج جورجياس لملاقاته بجيش عظيم، وكانت معارك شديدة بينهم، فسقط من رجال جورجياس عشرون ألفًا.

\par 3 فهرب جورجياس إلى بطليموس إلى أرض الغرب، لأن أنطيوخس كان قد جعله والياً على تلك البلاد، وكان يقيم هناك، وأخبره بما حدث له.

\par 4 فخرج بطليموس بجيش كان فيه مائة وعشرون ألف رجل من مقدونية والشرق.

\par 5 ثم سار حتى وصل إلى بلاد جعارش (أي جلعاد) وما حولها، فقتل من اليهود أعداداً كثيرة.

\par 6 فكتبوا إلى يهوذا يخبرونه بما حدث لهم، ويتوسلون إليه أن يأتي ويهزم بطليموس ويطرده عنهم.

\par 7 "فوصلته رسالتهم في نفس الوقت الذي وصلته فيه رسالة من سكان جبل الجليل الأول أيضاً، تخبره كيف اتحد عليهم المقدونيون الذين في صور وصيدا، وهاجموهم وقتلوا العديد منهم."

\par 8 وبعد أن قرأ يهوذا الرسالتين، استدعى رجاله، وأطلعهم على مضمون الرسالتين، وعيّن صوماً وطلبة.

\par 9 بعد ذلك، أمر أخاه سمعان أن يأخذ معه ثلاثة آلاف رجل من اليهود، وأن يسير بأقصى سرعة إلى جبل الجليل، «ويقضي على المقدونيين الذين هناك

\par 10 فذهب سمعان. وأما يهوذا فأسرع لملاقاة بطليموس.

\par 11 فهاجم سمعان المقدونيين بغتة، فقتل منهم ثمانية آلاف رجل، وأراح الجليليين.

\par 12 وأما يهوذا فسار حتى وصل إلى جرجياس وبطليموس، فضغط عليهما وحاصرهما، والتقى الجيشان، وكانت بينهما معارك شديدة جداً.

\par 13 لأن بطليموس كان على رأس جيش كبير من الرجال الأقوياء والشجعان. أما يهوذا فكان برفقته فرقة صغيرة جدًا:

\par 14 ومع ذلك، وبما أن الأشخاص الذين كانوا معه كانوا من أشجع وأقوى القوات، فقد قاوم بثبات، واستمرت المعركة بينهم طويلاً، وأصبحت شديدة للغاية

\par 15 لذلك نادى يهوذا الإله العظيم الصالح، وطلب مساعدته

\par 16 وروى أنه رأى خمسة فرسان شبان، ثلاثة منهم قاتلوا جيش بطليموس، واثنان وقفا بالقرب منه

\par 17 الذين عندما نظر إليهم بانتباه، بدا لهم أنهم ملائكة الله

\par 18 لذلك تعزى قلبه وقلوب رفاقه، وشنوا هجمات متكررة على العدو، فهزموهم وقتلوا منهم حشودًا كبيرة

\par 19 وكان عدد القتلى من جيش بطليموس، من بداية هذه المعركة إلى نهايتها، عشرين ألفًا وخمسمائة

\par 20 بعد هذه الأمور، هرب بطليموس ورجاله إلى ساحل البحر، بينما طاردهم يهوذا، وقتل منهم من أمسك

\par 21 أما بطليموس فهرب إلى غزة وأقام هناك، فجاء إليه رجال الحليسام

\par 22 فزحف يهوذا عليهم، فلما وجدهم هزمهم، فتشتت رجال بطليموس، أما هو فهرب إلى غزة، وتحصن هناك

\par 23 وطارد رجال يهوذا الجثة الهاربة، وقتلوا منهم أعدادًا كبيرة. وسار يهوذا والرجال الذين معه مباشرةً إلى غزة، ونصب معسكره وحاصرها

\par 24 فرجع إليه رجال يهوذا، وصعد الذين بقوا من جيش بطليموس إلى الحصن، وشتموا يهوذا كثيرًا

\par 25 واستمر القتال بينهم وبين جيش يهوذا خمسة أيام. ولما جاء اليوم الخامس، استمر الشعب في شتم يهوذا وسب دينه:

\par 26 عندها غضب عشرون من رجال يهوذا، فأخذوا دروعًا بأيديهم اليسرى، وسيوفًا بأيديهم اليمنى، ومعهم رجل يحمل سلمًا صنعوه، وساروا حتى وصلوا إلى السور

\par 27 ووقف ثمانية عشر منهم وألقوا السهام على من كانوا على السور، وأسرع اثنان إلى السور، ورفعا السلم وصعدا عليه

\par 28 لكن بعض الذين كانوا هناك، لما أدركوا أنهم صعدوا، وأن رفاقهم قد تبعوهم، ونزلوا أيضًا من السور إلى المدينة، نزلوا من السور خلفهم، فهزمهم رجال يهوذا، وقتلوا أعدادًا كبيرة من أعدائهم

\par 29 أما جيش يهوذا فتقدم نحو باب المدينة، فبدأ العشرون يركضون نحو الباب ليفتحوه، لكنهم طُردوا من هناك بشدة، ولذلك صرخوا بصوت عالٍ

\par 30 فعرف يهوذا ورجاله أنهم قد اقتربوا من الباب، فاشتد القتال خارج الباب وداخله.

\par 31 فهاجم يهوذا ورجاله الباب بالنار، فسقط، وهلك الشعب، وأُلقي القبض على الرجال الذين شتموا يهوذا، فأمر بإحضارهم وحرقهم

\par 32 ثم أمر بضرب المدينة ضربًا مبرحًا بالسيف، واستمرت المذبحة فيها يومين، ثم أُحرقت بالنار

\par 33 لكن بطليموس هرب، ولم يُسمع عنه خبر في ذلك الوقت، لأنه غيّر ملابسه، واختبأ في إحدى الحفر، ولم يُسمع عنه أي خبر

\par 34 فأُخذ أخويه وأُتيا إلى يهوذا، فأمر بقطع رأسيهما

\par 35 بعد ذلك دخل أرض المقدس، ومعه غنيمة وفيرة، وصلى هو ورفاقه فيها، شاكرين الله على النعم التي نالوها


\chapter{11}

\par \textit{علاقة المعركة بين يهوذا وليسياس قائد أوباتور، بعد وفاة الملك أنطيوخس}

\par 1 كان اسم أنطيوخس، الذي ذُكر أعلاه، هو إبيفانيوس: لكن اسم ابنه الذي حكم بعده كان أوباتور، الذي كان يُدعى أيضًا أنطيوخس

\par 2 ولما دارت معارك يهوذا مع هؤلاء القادة، كتبوا في هذا الشأن إلى يوباتور؛ الذي أرسل مع ليسياس، ابن عمه، جيشًا كبيرًا، كان فيه ثمانون ألف فارس وثمانون فيلاً

\par 3 فجاءوا إلى مدينة تُدعى بيت نير، ونزلوا حولها وحاصروها لأنها كانت مدينة عظيمة، وكان فيها شعب كثير

\par 4 ورفع ليسياس حولها المؤن، وبدأ يحاصر السكان:

\par 5 ولما قيل ليهوذا، خرج هو ورفاقه إلى جبال حصينة؛

\par 6 وأقاموا هناك لئلا إذا بقوا في مدينة، فيأتي ليسياس ويحاصرها، فيتغلب عليهم

\par 7 فجمع يهوذا رفاقه، وقرر أن يسير معهم إلى معسكر ليسياس، بعد أن يذهبوا إلى بيت الله ويقدموا الذبائح فيه؛

\par 8 يتوسلون إلى الله العظيم الصالح أن يصرف عنهم شر أعدائهم، وأن يمنحهم النصر عليهم: وهو ما فعلوه

\par 9 بعد ذلك، ساروا من منطقة البيت المقدس إلى بيت نِير. لأنهم كانوا يعتزمون مهاجمة الجيش فجأة، وهزيمته دون قتال

\par 10 يقول الناس الآن إنه ظهر ليهوذا شخص ما بين السماء والأرض، راكبًا على حصان ناري، وفي يده رمح كبير، ضرب به جيش الأمم

\par 11 فأعطاهم ما رأوه شجاعةً ومعنوياتٍ إضافية. فأسرعوا وهاجموا الجيش، وقتلوا أعدادًا كبيرةً من رجاله

\par 12 ولهذا السبب اضطرب جيش العدو ووقع في حالة من الفوضى الشديدة، وهرب كله في فوضى.

\par 13 فضغط عليهم سيف يهوذا وجماعته بشدة، فقتل منهم أحد عشر ألف راجل، وألفًا وستمائة فارس

\par 14 طُرد ليسياس أيضًا مع رفاقه إلى مكان بعيد، حيث بقي في أمان

\par 15 وأرسل إلى يهوذا يطلب منه أن يخضع للملك، محتفظًا بدينه ودين شعبه:

\par 16 الذي وافقه يهوذا في هذا الأمر، إلى أن أمكن كتابة كلمة إلى الملك، وتلقي رد بموافقته على ذلك

\par 17 وكتب يهوذا في هذا الشأن: وكتب ليسياس أيضًا إلى الملك، يُخبره بما حدث، وما لديه من دليل على قوة الأمة اليهودية وشجاعتها؛

\par 18 وأن استمرار الحروب معهم سيؤدي إلى إبادة رجاله، كما أُبيد هؤلاء المذكورون سابقًا: أخبره أيضًا بموافقتهم، وانتظاره حتى يتلقى رسالة ليخبره بما يجب عليه فعله

\par 19 فأجاب الملك أنه بدا له من الصواب أن يُصالح أمة اليهود، ويزيل تلك العقبة المتعلقة بممارسة دينهم: لأن هذا الأمر نفسه هو الذي حرضهم على الثورات، وعلى الهجمات التي شنت على أسلافه

\par 20 وأمره أيضًا أن يعقد معهم معاهدة صلح وطاعة؛ حتى لا تُوضع في طريقهم أي عقبات في أمر الدين

\par 21 وكتب أيضًا إلى يهوذا، وإلى جميع اليهود الذين كانوا في أرض يهوذا، بهذا المعنى: واستمر هذا السلام بينهم لبعض الوقت


\chapter{12}

\par \textit{سرد لبداية قوة الرومان، وتوسع إمبراطوريتهم.}

\par 1 في هذا الوقت نفسه، الذي كنا نتحدث عنه، بدأت شؤون الرومان تتعاظم: لكي يتمم الإله العظيم الصالح ما تنبأ به دانيال النبي (عليه السلام) بشأن الإمبراطورية الرابعة

\par 2 كان هناك أيضًا في ذلك الوقت ملكٌ كريمٌ جدًا في أفريقيا، اسمه أنيبال. وكان المقر الملكي لإمبراطوريته قرطاج. وقد عزم على الاستيلاء على مملكة الرومان:

\par 3 لذلك اتحدوا لمعارضته، وكثرت الحروب بينهم، حتى خاضوا ثماني عشرة معركة في غضون عشر سنوات، ولم يتمكنوا من طرده من بلادهم، بسبب جيشه وشعبه الذي لا يحصى عدده

\par 4 لذلك قرروا جمع قوة كبيرة مختارة من أشجع قواتهم وجيوشهم، ومهاجمة أنيبال بالحرب، والمثابرة حتى يصرفوا قواته عنهم

\par 5 وهو ما فعلوه حقًا: ووضعوا على رأس جيوشهم رجلين مشهورين؛ اسم أحدهما إيميليوس، والآخر فارو

\par 6 فقابله أنيبال فقاتله، فقتل من جيشه تسعين ألف رجل، ومن جيش أنيبال أربعين ألف رجل.

\par 7 لكن فارو هرب إلى مدينة كبيرة وقوية جدًا تُدعى فينوسيا. لم يطارده أنيبال، بل سار إلى روما ليستولي عليها، ويبقى هناك

\par 8 فظل أمامها ثمانية أيام، وبدأ في بناء منازل مقابلها؛

\par 9 وعندما رأى المواطنون ذلك، فكروا في إبرام صلح ومعاهدة معه، والتنازل عن البلاد

\par 10 ولكن كان بينهم شاب اسمه سكيبيو (لأن الرومان في ذلك الوقت كانوا بلا ملك، وكانت إدارة شؤونهم بأكملها تُعهد إلى ثلاثمائة وعشرين رجلاً، وكان يرأسهم شخص يُدعى كبير السن أو شيخ).

\par 11 لذلك جاء سكيبيو إلى هؤلاء، وأقنعهم بعدم الثقة بأنيبال أو الخضوع له. فأجابوا بأنهم لا يثقون به، لكنهم غير قادرين على مقاومته

\par 12 قال له: إن بلاد أفريقيا خالية تمامًا من الجنود، لأنهم جميعًا هنا مع أنيبال: أعطني إذن فرقة من الرجال المختارين، حتى أتمكن من الذهاب إلى أفريقيا

\par 13 وسأقوم بمآثر فيه، بحيث عندما تصل إليه أخبارها، ربما يترككم، وتتحررون منه، وتكونون في سلام: وبعد استعادة مواردكم وتعزيزها، إذا استعد للعودة، ستتمكنون من مقاومته

\par 14 وبدت لهم نصيحة سكيبيو صحيحة، فسلموه ثلاثين ألفًا من أشجع رجالهم

\par 15 ثم توجه إلى أفريقيا. والتقى به أسدروبال، شقيق أنيبال، وقاتله، فهزمه سكيبيو، وقطع رأسه، وأخذها مع بقية الغنائم، وعاد إلى روما

\par 16 ثم صعد على السور، ونادى أنيبال، وقال: كيف ستستطيع أن تنتصر على بلادنا هذه، وأنت غير قادر على طردي من أرضك التي ذهبت إليها: لقد أهلكتها، وقتلت أخاك، وأزلت رأسه

\par 17 ثم ألقى الرأس إليه. فلما أحضروه إلى أنيبال وتعرف عليه، ازداد غضبه وغضبه على الشعب، وأقسم أنه لن يغادر حتى يستولي على روما

\par 18 لكن المواطنين، لسحبه منهم وإبقائه تحت السيطرة، تشاوروا لإعادة سكيبيو لمحاصرة قرطاج ومهاجمتها

\par 19 وعاد سكيبيو بجيشه إلى أفريقيا، ونصبوا معسكرهم حول قرطاج، وحاصروها حصارًا شديدًا

\par 20 لذلك كتب السكان إلى أنيبال قائلين: أنت تطمع في أرض غريبة، لا تعلم إن كنت ستستطيع الاستيلاء عليها أم لا، ولكن قد جاء إلى بلدك من يسعى للاستيلاء عليها

\par 21 لذلك، إذا تأخرت في المجيء، فسوف نسلم له البلاد، وسنتنازل عن عائلتك وكل ما تملك وكنوزك؛ حتى لا نتعرض نحن وممتلكاتنا لأذى.

\par 22 ولما وصلت إليه هذه الرسالة، غادر روما وأسرع حتى وصل إلى أفريقيا:

\par 23 فتقدم سكيبيو والتقى به، وخاض معه معركة شرسة ثلاث مرات، وقُتل من رجاله خمسون ألفًا

\par 24 ولكن أنيبال، بعد أن هُزم، انسحب إلى أرض مصر؛ فطارده سكيبيو، وأسره، وعاد إلى أفريقيا

\par 25 وعندما كان هناك، ازدرى أنيبال أن يراه الأفارقة؛ لذلك تناول السم ومات

\par 26 وفاز سكيبيو ببلاد أفريقيا، واستولى على جميع ممتلكات أنيبال وخدمه وكنوزه

\par 27 وبهذه الطريقة ازدادت شهرة الرومان، وبدأت قوتهم منذ ذلك الوقت تتزايد

\chapter{13}

\par \textit{سرد لرسالة الرومان إلى يهوذا، والمعاهدة التي عُقدت بينهما.}

\par 1 «من الشيخ وثلاثمائة وعشرين حاكمًا، إلى يهوذا قائد الجيش، وإلى اليهود

\par 2 الصحة والعافية لك. لقد سمعنا بالفعل عن انتصاراتك وشجاعتك وقدرتك على التحمل في الحرب؛ وهذا ما نفرح به. كما علمنا أنك قد دخلت في اتفاقية مع أنطيوخس

\par 3 نكتب إليكم بهذا المعنى، أن تكونوا أصدقاء لنا، لا لليونانيين الذين أساءوا إليكم. علاوة على ذلك، ننوي الذهاب إلى أنطاكية، وشن حرب على سكانها

\par 4 لذلك، سارع بتعريفنا بمن أنت على عداوة معه، ومن تربطك به صداقة؛ حتى نتمكن من التصرف وفقًا لذلك

\par 5 نسخة المعاهدة. "هذه هي المعاهدة التي عقدها الشيخ وثلاثمائة وعشرون حاكمًا مع يهوذا، قائد الجيش، واليهود؛ لكي ينضموا إلى الرومان، ولكي يكون الرومان واليهود على رأي واحد في الحروب والانتصارات إلى الأبد

\par 6 الآن، إذا اندلعت حرب على الرومان، فسيساعدهم يهوذا وشعبه، ولن يقدموا أي مساعدة لأعداء الرومان، سواء بالمؤن أو بأي نوع من الأسلحة

\par 7 وعندما تأتي الحرب على اليهود، سيساعدهم الرومان بكل ما أوتوا من قوة، ولن يقدموا أي مساعدة لأعدائهم بأي شكل من الأشكال

\par 8 وكما أن اليهود مرتبطون بالرومان، فكذلك الرومان مرتبطون باليهود، دون أي زيادة أو نقصان

\par 9 فقبل يهوذا وقومه ذلك، وصمدت المعاهدة، واستمرت بينهم وبين الرومان زمانًا طويلًا

\chapter{14}

\par \textit{سرد للمعركة التي دارت بين يهوذا وبطليموس وجورجياس.}

\par 1 بعد ذلك، جمع بطليموس مئة وعشرين ألف رجل وألف فارس، وطاردوا يهوذا. فالتقى به يهوذا في عشرة آلاف رجل، وهزمه، فقُتل كثيرون من رجال بطليموس.

\par 2 فتوسل إلى يهوذا، وتوسل إليه بتواضع أن يدعه ينجو بحياته، وحلف أنه لن يحاربه بعد الآن، وأنه سوف يظهر اللطف لليهود الذين في كل بلادهم.

\par 3 فشفق عليه يهوذا وأطلقه، والتزم بطليموس بقسمه

\par 4 فجمع جورجياس ثلاثة آلاف رجل من جبل سارة (أي من أدوم) وأربعمائة فارس، والتقى يهوذا، وقتل قائد جيشه وبعض رجاله

\par 5 ثم تقدم يهوذا ورجاله نحوهم، فانهزم جورجياس، وقُتل أو هرب معظم جيشه، وفُتح عنه، ولم يُسمع عنه أي خبر، ولكن قيل إنه سقط في المعركة

\chapter{15}

\par \textit{سرد لنقض المعاهدة التي عقدها أنطيوخس مع يهوذا، ومسيرته (مع ليسياس ابن عمه) مع جيش عظيم، وحروبه.}

\par 1 ولكن عندما وصل الخبر إلى أنطيوخس أوباتور بأن أعمال يهوذا قد اكتسبت قوة، والانتصارات التي حققها، غضب بشدة؛

\par 2 ونقض المعاهدة التي عقدها مع يهوذا، وجمع جيشًا كبيرًا، كان فيه اثنان وعشرون فيلاً:

\par 3 فسار مع ليسياس ابن عمه إلى بلاد يهوذا، متوجهين إلى مدينة بيت نعر، فنزل أمامها وحاصرها.

\par 4 ولما أخبر يهوذا بذلك اجتمع هو وجميع شيوخ بني إسرائيل وصلوا إلى الإله العظيم الصالح وقدموا ذبائح كثيرة.

\par 5 ولما كمل ذلك ذهب يهوذا مع رؤساء قواته، ودخلوا المحلة ليلا، وهاجموها بغتة، فقتل من العدو أربعة آلاف رجل وفيلاً واحداً، ثم رجع إلى محلته حتى طلع الفجر.

\par 6 ثم انسحب كل جيش، واشتدت المعركة بينهم

\par 7 فرأى يهوذا أحد الفيلة المزينة بزخارف ذهبية، فظن أن الملك جالس عليه. فدعا رجاله وقال لهم: من منكم يخرج ويقتل هذا الفيل؟

\par 8 فخرج شاب من عبيده يُدعى أليعازار، واندفع على صف العدو، قاتلًا عن اليمين واليسار، حتى انحرف الرجال عن نظره

\par 9 فتقدم حتى وصل إلى الفيل، فزحف تحته، فشق بطنه، فسقط عليه الفيل ومات. فلما رأى الملك ذلك، أمر بالانسحاب، فتم ذلك

\par 10 وكان عدد الرجال من ذوي الرتب العليا الذين قُتلوا في ذلك اليوم في المعركة ثمانمائة رجل، بالإضافة إلى من قُتلوا من عامة الناس، ومن قُتلوا أثناء الليل

\par 11 فأخبر الملك أن رجلاً من أصدقائه اسمه فيليب قد تمرد عليه، وأن ديمتريوس بن سلوقس خرج من رومية بجيش عظيم من الرومان، مقصداً أن يأخذ المملكة من يده.

\par 12 فارتعب بشدة، فأرسل إلى يهوذا ليُصلح بينهما، فوافق يهوذا، وأقسم له أنطيوخس وليسياس ابن عمه أنهما لن يُحارباه بعد الآن

\par 13 وأظهر الملك مبلغًا كبيرًا من المال، وأعطاه ليهوذا هديةً لبيت الله

\par 14 وأمر الملك أيضًا بالقبض على مينيلوس، أحد الرجال الثلاثة الأشرار الذين جلبوا الشر على اليهود في أيام أنطيوخس أبيه، وأمر برفعه إلى برج عالٍ وإلقائه من هناك على الفور، وهو ما تم

\par 15 لأن الملك كان يقصد بذلك إرضاء اليهود، لأن هذا الرجل كان أحد أعدائهم الرئيسيين، وقد قتل أعدادًا كبيرة منهم

\chapter{16}

\par \textit{تاريخ وصول ديمتريوس بن سلوقس إلى أنطاكية، وهزيمته لإوباتور.}

\par 1 بعد هذه الأمور، سار الملك أوباتور إلى بلاد مقدونيا، ثم عاد إلى أنطاكية

\par 2 الذي هاجمه ديمتريوس بجيش من الرومان، وهزمه وقتله، مع ليسياس ابن عمه، وملك في أنطاكية

\par 3 فذهب إليه ألكيمس، زعيم أولئك الرجال الثلاثة الأشرار؛ الذي دخل إلى حضرته وسجد أمامه وبكى بكاءً شديدًا وقال:

\par 4 «أيها الملك، لقد قتل يهوذا ورفاقه أعدادًا كبيرة منا؛ لأننا، بعد أن تركنا دينهم، اعتنقنا دين الملك. لذلك، أيها الملك، ساعدنا عليهم، وانتقم لنا منهم.»

\par 5 ثم جعل اليهود يذهبون إليه، وأثار غضبه، واقترح عليهم أشياء قد تثير ديمتريوس، وتثير غضبه، ليجهز جيشًا لهزيمة يهوذا

\par 6 فأطاعه الملك، فأرسل إليه قائدًا يُدعى نيكانور، على رأس جيش عظيم وإمدادات وفيرة من أسلحة الحرب

\par 7 ولما جاء نيكانور إلى الأرض المقدسة، أرسل رسلًا إلى يهوذا ليأتوا إليه، ولم يُخبر أنه جاء ليغزو الأمة،

\par 8 لكنه ذكر أنه جاء فقط بسبب السلام الذي عُقد بينه وبين الأمة، وأنهم أيضًا كانوا تحت طاعة الرومان

\par 9 فخرج إليه يهوذا ومعه عدد من رجاله، وكانوا ذوي قوة وشجاعة، وأمرهم ألا يبتعدوا عنه لئلا ينصب له ديمتريوس فخًا

\par 10 فلما التقى ديمتريوس، سلم عليه، ووضع لكل منهما مقعد، فجلسا، وكان ديمتريوس يتحدث معه كما يشاء. وبعد ذلك دخل كل واحد منهما خيمة نصبها له الجند

\par 11 وانصرف نيكانور ويهوذا إلى المدينة المقدسة، وسكنا هناك معًا، ونشأت بينهما صداقة وطيدة

\par 12 ولما علم ألكيموس بذلك ذهب إلى ديمتريوس وأهابه على يهوذا، وأقنعه بكتابة أمر إلى نكانور بإرسال يهوذا إليه مقيدًا بالسلاسل.

\par 13 فبلغ الخبر يهوذا، فخرج من المدينة ليلاً ومضى إلى سبسطية، وأرسل إلى أصحابه أن يأتوا إليه

\par 14 ولما وصلوا، نفخ في البوق، وأمرهم أن يستعدوا لمهاجمة نيكانور

\par 15 لكن نيكانور بحث عن يهوذا باجتهاد كبير، ولم يستطع أن يعرف عنه شيئًا

\par 16 لذلك ذهب إلى بيت الله، وطلب من الكهنة أن يسلموه إليه، لكي يرسله مقيدًا بالسلاسل إلى الملك. لكنهم أقسموا أنه لم يدخل بيت الله

\par 17 وعندئذٍ شتمهم وبيت الله، وتحدث بوقاحة عن الهيكل، وهدد بهدمه من أساساته، ثم انصرف غاضبًا. كما حرص على تفتيش جميع بيوت المدينة المقدسة

\par 18 وبالمثل، أرسل رجاله إلى منزل رجلٍ فاضل، كان قد أُلقي القبض عليه في زمن أنطيوخس، وتعرض لتعذيبٍ شديد؛ ولكن بعد وفاة أنطيوخس، زاد اليهود من سلطته وأكرموه كثيرًا

\par 19 ولما جاء إليه رسل نكانور، خاف أن يلقى نفس المعاملة التي تلقاها من أنطيوخس، لذلك وضع يديه على نفسه

\par 20 عندما أُخبر يهوذا بذلك، حزن بشدة وتضايق كثيرًا، وأرسل إلى نيكانور قائلًا: "لا تبحث عني في المدينة، لأني لست هناك. لذلك اخرج إليّ، حتى نلتقي إما في السهول أو في الجبال، كما تشاء."

\par 21 فخرج إليه نكانور، فاستقبله يهوذا بهذه الكلمات: "يا الله، أنت الذي أهلكت جيش الملك سنحاريب، وكان في الحقيقة أعظم من هذا الرجل، في الشهرة، وفي الإمبراطورية، وفي كثرة جيشه

\par 22 وأنقذت حزقيا ملك يهوذا منه، عندما توكل عليك وصلى إليك. نجنا، يا الله، من شره، وانصرنا عليه

\par 23 ثم جهز نفسه للمعركة، وتقدم إلى نيكانور قائلًا: "اعتنِ بنفسك، إليك أتيت."

\par 24 فأدار نكانور ظهره وهرب، فضربه يهوذا الذي كان يطارده على كتفيه، فشقّهما، فانهزم رجاله

\par 25 فسقط منهم في ذلك اليوم ثلاثون ألفًا، وخرج سكان المدن وقتلوهم، ولم يُبقوا منهم أحدًا

\par 26 وقرروا أن يكون ذلك اليوم كل عام يومًا للشكر لله العظيم الصالح، ويومًا للفرح، وللولائم، وللشرب. [هكذا انتهى الكتاب الثاني من ترجمة العبرانيين.]

\chapter{17}

\par \textit{سرد لموت يهوذا}

\par 1 ولكن عندما جاء نفس الوقت تقريبًا من العام، خرج بكيديس مع ثلاثين ألفًا من أشجع المقدونيين؛

\par 2 فصادف يهوذا دون أن يصل إليه أي خبر بذلك، عندما كان في مدينة تُدعى لاليس، ومعه ثلاثة آلاف رجل:

\par 3 لذلك هرب معظم الذين كانوا معه، وبقي معه ثمانمائة رجل، وإخوته شمعون ويوناثان

\par 4 لكن أولئك الذين بقوا مع يهوذا كانوا الأقوى والأشجع، والذين تحملوا الكثير بالفعل في المعارك العديدة التي خاضها

\par 5 فخرج يهوذا وجماعته لملاقاة بكيديس وجيشه.

\par 6 ثم قسم بكيديس جيشه، فجعل خمسة عشر ألفًا عن يمين يهوذا وجماعته، وخمسة عشر ألفًا عن يسارهم.

\par 7 ثم صرخ كل فريق على يهوذا وجماعته. فنظروا إلى بعضهم البعض باهتمام، فأدركوا أن أقوى وأشجع جيوش العدو كانت على اليمين، ووجدوا أن بكيديس نفسه كان هناك بينهم

\par 8 كذلك قسّم يهوذا جماعته، فأخذ أشجعهم معه، وأعطى الباقي لإخوته. ثم هاجم الذين على اليمين، فقتل هو وجماعته نحو ألفي رجل

\par 9 ثم رأى باكيديس، فوجّه نظره وخطواته نحوه، وقتل جميع أشجع الرجال الذين كانوا حوله

\par 10 وتحمل هو ورفاقه الجموع التي كانت تضغط عليه، وأسقطوا معظمهم أرضًا، ثم اقترب من بكيديس

\par 11 الذي لما رآه بكيديس قادمًا نحوه كالأسد، ملوحًا في يده بسيف كبير ملطخ بالدماء، خاف منه خوفًا شديدًا، وارتجف، وهرب من أمامه

\par 12 فطارده يهوذا وجماعته، فقتلوا قومه بالسيف، فقتلوا أكثر أولئك الخمسة عشر ألفًا، وهرب بكيديس حتى إلى أشدود

\par 13 فتبعه الخمسة عشر ألفًا الذين كانوا عن يسار يهوذا، وهاجموا يهوذا، وكان قد وصل إليه في ذلك الوقت إخوته والذين كانوا معهم، وقد تعبوا جدًا

\par 14 فاندفع أولئك الخمسة عشر ألفًا نحوهم، ودارت بينهم وبين يهوذا معركة عظيمة جدًا، فسقط من الجانبين عدد من القتلى، كان من بينهم يهوذا

\par 15 الذي حمله إخوته ودفنوه بجانب قبر متتيا أبيه، [رحمهم الله]، وبكي عليه بنو إسرائيل سنينًا كثيرة

\par 16 وكانت مدة ملكه سبع سنوات، وخلفه أخوه يوناثان في الحكم

\chapter{18}

\par \textit{تاريخ يونثان بن متثيا}

\par 1 وخلف يونثان أخاه، وذهب إلى الأردن في عدد قليل من الرجال، فسمع بكيديس فسار إليه في جيش عظيم.

\par 2 ولما رآه يوناتان، عبر رجاله الأردن سابحا، فتبعهم بكيديس وجيشه، وحاصروهم.

\par 3 فهاجم يوناتان بكيديس، فلما تخلى الرجال عن يوناتان خرج هو وجماعته من وسطهم وذهبوا إلى بئر سبع.

\par 4 وانضم إليه أخوه شمعون، وأقاموا هناك، وأصلحوا ما سقط من الحصون، وتحصنوا هناك

\par 5 فسار إليهم بكيديس وحاصرهم، وخرج إليه يوناثان وأخوه والذين معهم ليلاً، فقتلوا من جيشه أعدادًا كثيرة، وأحرقوا المكابس والمجانيق

\par 6 فتشتت جيشه، وهرب بكيديس إلى البرية. فطارده يوناثان وشمعون والرجال الذين معه، فقبضوا عليه

\par 7 الذي لما رأى يوناثان، علم أن موته قريب، لذلك أعلن السلام مع يوناثان، وأقسم أنه لن يحاربه بعد الآن ، بل سيعيد جميع الأسرى الذين أخذهم من جيش يهوذا

\par 8 فأعطاه يوناثان يده وانصرف عنه، ولم تكن بينهما حرب بعد ذلك. وبعد ذلك بقليل مات يوناثان، فخلفه أخوه شمعون


\chapter{19}

\par \textit{تاريخ سمعان بن متتيا}

\par 1 ثم تولى الملك شمعون بن متتيا، وجمع كل الذين بقوا من جيش يهوذا.

\par 2 وازدهرت أموره، وقهر كل من مارس العداء ضد اليهود بعد وفاة أخيه يهوذا، وحسن سلوكه مع شعبه، ونظمت أمور بلاده على النحو الصحيح

\par 3 لماذا هاجمه أنطيوخس؟ وكذلك ديمتريوس بن سلوقس، وأرسل جيشًا عظيمًا ضده

\par 4 ولما التقى به خرج شمعون وابناه، وقسم جيشه إلى قسمين، قسم احتفظ به لنفسه، وأعطى القسم الآخر لبنيه.

\par 5 ثم انطلق هو والذين معه إلى الجيش، وأرسل ابنيه وأتباعهما في طريق أخرى، وعين معهما لمهاجمة الجيش في وقت معين.

\par 6 وبعد ذلك التقى بجيش أنطيوخس وهاجمه وبدأ يتغلب عليه. وجاء ابناه عندما بدأت المعركة، واشتدت المعركة، وجاءا من مؤخرة الجيش.

\par 7 "فأما جيش أنطيوخس فقد انقسم بين جيشين، ولم ينجُ منهم رجل واحد، ولم يرجع أنطيوخس بعد ذلك إلى حرب شمعون."

\par 8 وكان السلام والهدوء بين اليهود كل أيام شمعون، وكانت مدة ملكه سنتين.

\par 9 ثم انقض عليه بطليموس صهره وقتله في وليمة كان حاضرًا فيها. وقبض على امرأته وولديه. ووُضع ابن شمعون، واسمه هيركانوس، مكان أبيه

\par [هنا ينتهي التاريخ كما ورد في الكتابين المرفقين عادة بكتبنا المقدسة.]

\chapter{20}

\par \textit{تاريخ هيركانوس بن شمعون}

\par 1 وكان شمعون قد جعل ابنه يوحنان رئيساً وهو بعد حي، وجمع إليه جيشاً كثيراً جداً، وأرسله لقتل رجل خرج للقتال، اسمه هيركانوس.

\par 2 كان رجلاً ذا شهرة عظيمة، قوي القوة، وذو سيادة عريقة

\par 3 الذي واجهه يوناثان وهزمه، ولذلك سمى شمعون ابنه يوحنان هيركانوس؛ بسبب قتله هيركانوس وانتصاره عليه

\par 4 ولكن عندما سمع هيركانوس أن بطليموس قتل والده، خاف من بطليموس وهرب إلى غزة، وطارده بطليموس مع العديد من الأتباع

\par 5 لكن أهل غزة ساعدوا هيركانوس، وأغلقوا أبواب مدينتهم، ومنعوا بطليموس من الوصول إلى هيركانوس

\par 6 وعاد بطليموس، وذهب إلى داجون، ومعه أم هيركانوس وأخويه. وكان لداجون في ذلك الوقت قلعة حصينة

\par 7 لكن هيركانوس ذهب إلى البيت المقدس، وقدم القرابين، وخلف والده. وجمع جيشًا كبيرًا وذهب لمهاجمة بطليموس. لذلك أغلق بطليموس بوابة داجون في وجهه وجماعته، وتحصن بها

\par 8 فحاصره هيركانوس، وصنع كبشًا من حديد ليهدم السور ويفتحه، واستمرت المعركة بينهما طويلًا،

\par 9 وانتصر هيركانوس على بطليموس، وصعد إلى قرب القلعة، وكاد أن يستولي عليها

\par 10 فلما رأى بطليموس ذلك، أمر بإحضار أم هيركانوس وأخويه إلى السور وتعذيبهم أشد التعذيب، وهو ما حدث لهم

\par 11 ولكن هيركانوس، عندما رأى ذلك، وقف ساكنًا، وخاف أن يُقتلوا، فكف عن القتال

\par 12 نادت عليه أمه وقالت: "يا بني، لا تدع حبك وبرِّك لي ولإخوتك يحركك، بل فضِّلهم على أبيك:

\par 13 ولا تضعف رغبتك في الانتقام له بسبب أسرنا؛ بل طالب بالتعويض عن حقوق أبيك وأبي، بأقصى ما في وسعك

\par 14 لكن ما تخشاه علينا من ذلك الطاغية، سيفعله بنا حتمًا في كل الأحوال: لذلك واصل حصارك دون أي توقف

\par 15 فلما سمع هيركانوس كلام أمه، حث على الحصار، ولذلك زاد بطليموس من تعذيب أمه وإخوته، وأقسم أنه سيطردهم من القلعة كلما اقترب هيركانوس من السور

\par 16 لذلك خشي هيركانوس أن يكون سببًا في موتهم، فعاد إلى معسكره، واستمر في "حصار بطليموس".

\par 17 وحدث أن عيد المظال كان قريبًا، فذهب هيركانوس إلى مدينة البيت المقدس، ليكون حاضرًا في العيد والاحتفال والذبائح

\par 18 ولما علم بطليموس أنه ذهب إلى المدينة المقدسة وأنه محتجز هناك، قبض على أم هيركانوس وإخوته وقتلهم، ثم هرب إلى مكان لم يستطع هيركانوس أن يصل إليه.



\chapter{21}

\par \textit{تاريخ صعود أنطيوخس إلى مدينة البيت المقدس لمحاربة هيركانوس.}

\par 1 ولما سمع أنطيوخس أن سمعان قد مات، جمع جيشًا وسار حتى وصل إلى مدينة البيت المقدس

\par 2 ونزل حولها وحاصرها، عازمًا على الاستيلاء عليها بالقوة، لكنه لم يستطع، بسبب ارتفاع الأسوار ومتانتها، وكثرة المحاربين الذين كانوا فيها

\par 3 ولكن بمشيئة الله مُنع من الفوز بها: لأنه كان قد لجأ إلى الجانب الشمالي من المدينة، وبنى هناك مائة وثلاثين برجًا مقابل السور؛

\par 4 وجعل الرجال يركبونها، ليقاتلوا أولئك الذين يحاولون الصعود إلى أسوار المدينة

\par 5 كما عيّن رجالاً ليحفروا الأرض في مكان معين، حتى وصلوا إلى أساس السور، فلما وجدوه من خشب، أحرقوه بالنار، فسقط جزء كبير جداً من السور

\par 6 فقاومهم رجال هيركانوس، ومنعوهم من الدخول، وظلوا يحرسون الجزء المُخرب؛

\par 7 وخرج هيركانوس مع معظم رجاله المقاتلين ضد جيش أنطيوخس، وهزمهم هزيمة نكراء

\par 8 وانهزم أنطيوخس ورجاله، فطاردهم هيركانوس مع جنوده حتى طردوهم من المدينة

\par 9 ثم رجعوا إلى الأبراج التي بناها أنطيوخس، وهدموها، وأقاموا في المدينة وما حولها

\par 10 ونزل أنطيوخس في مكان بعيد عن مدينة بيت الله نحو غلوتين

\par 11 وعند اقتراب عيد المظال، أرسل هيركانوس إليه سفراء ليُبرموا هدنة حتى انتهاء العيد، فمنحه إياها، وأرسل ذبائح وذهبًا وفضة إلى بيت الله

\par 12 وأمر هيركانوس الكهنة أن يتسلموا ما أرسله أنطيوخس، ففعلوا ذلك

\par 13 ولما رأى هيركانوس والكهنة احترام أنطيوخس لهيكل الله، أرسل إليه سفراء ليُفاوضوه على السلام

\par 14 فوافق أنطيوخس على ذلك، وذهب إلى أورشليم، وعندما التقى به هيركانوس، دخلا المدينة معًا

\par 15 وأقام هيراكانوس وليمة لأنطيوخس وأمرائه، فأكلوا وشربوا معًا، وأهدى له ثلاثمائة وزنة من الذهب

\par 16 واتفق كل واحد منهم مع صاحبه على الصلح وتقديم المساعدة، وانصرف أنطيوخس إلى بلده

\par 17 "ولكن يروى أن هيركانوس فتح الخزانة التي صنعها بعض ملوك أبناء داود، [عليه السلام]، وأخرج منها مبلغًا كبيرًا من المال، وترك مثله فيها، وأعادها إلى حالتها السابقة من السرية.

\par 18 ثم بنى وأصلح الجزء الذي سقط من السور، واهتم بعناية براحة ومصلحة رعيته، وتصرف معهم باستقامة

\par 19 ولما جاء أنطيوخس إلى بلاده، عزم على أن يذهب ويحارب ملك فارس، لأنه ثار منذ عهد أنطيوخس الأول:

\par 20 وأرسل سفراء إلى هيركانوس ليذهب إليه، فذهب هيركانوس معه، وانصرف إلى بلاد فارس

\par 21 والتقى به جيش من الفرس وقاتلوه، فهزمهم أنطيوخس وقتلهم بحد السيف

\par 22 ثم أقام في المكان الذي كان فيه، وأقام بناءً رائعًا، ليكون نصبًا تذكاريًا له في بلادهم

\par 23 وبعد مدة، تقدم للقاء ملك الفرس، وتخلف هيركانوس بسبب السبت الذي أعقبه يوم الخمسين مباشرة

\par 24 والتقى ملك فارس وأنطيوخس، ودارت بينهما معارك عظيمة جدًا، قُتل فيها أنطيوخس وكثيرون من جيشه

\par 25 ولما وصل خبر ذلك إلى هيريكانوس، سار إلى بلاد سوريا، وفي رحلته حاصر حلبوس:

\par 26 فاستسلم له المواطنون، ودفعوا له الجزية، فانصرف عنهم، ورجع إلى المدينة المقدسة، وأقام فيها بضعة أيام

\par 27 ثم انطلق إلى بلاد السامرة، وحارب نيابوليس، لكن أهلها منعوه من دخولها

\par 28 وهدم كل ما كان لهم من أبنية على جبل إيزابل، والهيكل، وذلك بعد مئتي سنة من بناء سنبلط السامري له. كما قتل الكهنة الذين في سبسطية

\par 29 وسار إلى بلاد أدوميزا، أي جبال سارة، فاستسلموا له: الذين اشترط عليهم أن يختنوا ويعتنقوا دين التوراة (أو الشريعة الموسوية).

\par 30 فوافقوه، وختنوا، وصاروا يهودًا، وثبتوا على هذه العادة حتى خراب البيت الثاني

\par 31 وذهب هيريانوس إلى جميع الأمم المحيطة، فخضعوا جميعًا له، وفي الوقت نفسه دخلوا في اتفاقية سلام وطاعة

\par 32 كما أرسل سفراء إلى الرومان، يشرح لهم تجديد العهد الذي كان بينهم

\par 33 فلما جاء سفراؤه إلى الرومان، أكرموهم، وجعلوا لهم كرسيًا من الكرامة، واهتموا بالسفارة التي جاءوا من أجلها، وأرسلوا أعمالهم، وأجابوا على رسالته

\chapter{22}

\textit{نسخة رسالة الرومان إلى هيركانوس}

\par 1 «من الشيخ، وولاته الثلاثمائة والعشرين، إلى هيركانوس ملك يهوذا، الصحة

\par 2 لقد وصلتنا رسالتكم الآن، وقد فرحنا بقراءتها؛ وقد استفسرنا من سفرائكم عن حالة شؤونكم

\par 3 كما اعترفنا بمكانتهم المرموقة في العلم والانضباط الأخلاقي والفضائل؛ وكرمناهم، وجعلناهم يجلسون في حضرة شيخنا:

\par 4 الذي كان حريصًا على إنجاز جميع أعمالهم، وأصدر أمرًا بإعادة جميع المدن التي استولى عليها أنطيوخس بالقوة إليكم؛

\par 5 وأن يُزال كل ما يعيق ممارسة دينكم، وأن يُبطل كل ما أصدره أنطيوخس ضدكم

\par 6 كما أمر بأن تظل جميع المدن التي استولى عليها وفية لكم؛ كما أصدر أوامره برسائل إلى جميع مقاطعاته، بأن يُعامل سفراؤكم باحترام وتكريم

\par 7 علاوة على ذلك، فقد أرسل معهم سفيرًا إليكم اسمه كينزيوس، حاملاً رسالة، وقد عهد إليه أيضًا بسفارة، لكي يتعامل معكم شخصيًا

\par 8 لذلك عندما وصلت هذه الرسالة من الرومان إلى هيركانوس، بدأ يُلقب ملكًا، وكان يُدعى سابقًا رئيس كهنة: وهكذا اتحدت فيه الكرامات الملكية والكهنوتية

\par 9 وكان أول من دُعي ملكًا بين رؤساء اليهود في زمن البيت الثاني


\chapter{23}

\par \textit{تاريخ حروب هيركانوس مع السامريين}

\par 1 ثم سار هيركانوس إلى سبسطية، وحاصر السامريين الذين فيها مدة طويلة، حتى أوقعهم في ضيق شديد حتى اضطروا إلى أكل كل أنواع الجيف.

\par 2 ومع ذلك فقد تحملوا ذلك بصبر، خائفين من سيفه، ومعتمدين على المقدونيين والمصريين، الذين توسلوا لمساعدتهم.

\par 3 وفي هذه الأثناء يأتي الصوم الكبير، حيث يجب على هيركانوس أن يكون حاضراً في البيت المقدس، لتقديم الذبائح في ذلك اليوم.

\par 4 ولذلك استبدل ابنيه أنتيجونوس وأريستوبولوس بقائدي الجيش، وترك لهما الأوامر بمحاصرة السامريين وإجبارهم على الاستسلام.

\par 5 وكذلك أمر الجيش أن يطيعوا بنيه وينفذوا أوامرهم، ثم انطلق إلى مدينة البيت المقدس.

\par 6 ثم سار أنطيوخس المقدوني لمساعدة سكان سبسطية، وأُبلغ ذلك إلى ابني هيركانوس.

\par 7 الذي بعد أن استبدل قائداً بآخر لحصار سبسطية، ذهب للقاء أنطيوخس، الذي واجهوه وهزموه، وعادوا إلى سبسطية.

\par 8 وخرج من مصر أيضًا ليثراس، ابن الملكة كليوباترا، لمساعدة السامريين.

\par 9 وعندما وصل الخبر إلى هيركانوس، ذهب لمقابلته، بعد أن انقضى الاحتفال، وعندما التقى به واجهه بشراسة شديدة، وقتل العديد من رجاله.

\par 10 وهرب ليثراس، ولم يعد المصريون بعد ذلك ليقدموا المساعدة للسامريين.

\par 11 وعاد الملك هيركانيس إلى سبسطية، وضغط عليها بشدة، حتى استولى عليها بالسيف، وقتل من بقي من أهلها، ودمرها تدميرًا، وهدم أسوارها

\chapter{24}

\par \textit{تاريخ ليثراس ابن كليوباترا، وخروجه ضد والدته في مصر.}

\par 1 ثار ليثراس ابن كليغباترا، بعد أن أصبح قويًا في ممتلكاته ورجاله، على كليوباترا والدته؛ وكان كبار رجال المملكة من مساعديه

\par 2 لذلك، أرسلت كليوباترا في طلب يهوديين، أحدهما يُدعى خليسياس والآخر حنانيا، ووضعتهما على رأس أمراء مصر الذين بقوا في صفها، وجعلتهما قائدين للجيش المصري

\par 3 لقد أداروا جميع الأمور بشكل جيد مع عامة الناس، وأداروا شؤون الإمبراطورية بحكمة. أرسلتهم كليوباترا للقتال مع ليثراس؛

\par 4 الذين ذهبوا إليه للحرب، وهزموه، وهرب رجاله. فهرب إلى قبرص، وبقي هناك مع قلة ممن انضموا إليه

\chapter{25}

\par \textit{سرد للطوائف اليهودية في هذا الوقت.}

\par 1 في ذلك الوقت، كانت هناك ثلاث طوائف بين اليهود. الأولى، الفريسيون، أي "المنفصلون"، أو المتدينون؛

\par 2 الذين كان حكمهم هو الحفاظ على كل ما هو موجود في القانون، وفقًا لتفسيرات آبائهم.

\par 3 والثانية: الصدوقيون، وهم أتباع رجل من العلماء اسمه صادوق.

\par 4 الذي كان حكمه هو الحفاظ على الأمور الموجودة في نص الناموس، والتي يوجد دليل عليها في الكتاب المقدس نفسه؛ ولكن ليس ما ليس موجودًا في النص، ولا يتم إثباته من خلاله.

\par 5 أما الطائفة الثالثة فكانت طائفة الحسدانيين، أو أولئك الذين درسوا الفضائل: لكن مؤلف هذا الكتاب لم يذكر حكمهم، ولا نعرف عنهم إلا ما يُكتشف من خلال اسمهم:

\par 6 لأنهم كرّسوا أنفسهم لممارسات تقترب من الفضائل الأبرز؛ أي أن يختاروا من بين هاتين القاعدتين الأخريين ما هو الأكثر أمانًا في الاعتقاد والأكثر يقينًا وحذرًا

\par 7 كان هيركانوس في البداية أحد الفريسيين؛ وبعد ذلك ذهب إلى الصدوقيين؛

\par 8 لأن واحدًا من الفريسيين قال له: لا يجوز لك أن تكون رئيس كهنة، لأن أمك كانت أسيرة قبل أن تلدك في أيام أنطيوخس. ولكن لا يليق أن يكون ابن أسير رئيس كهنة

\par 9 وجرى هذا الحديث أمام رؤساء الفريسيين، مما أدى إلى انتقاله إلى ولاية الصدوقيين.

\par 10 وكان الصدوقيون في عداوة مع الفريسيين، ولذلك كانوا يخاصمون بعضهم بعضًا، وتمكنوا منه حتى قتل عددًا كبيرًا من الفريسيين

\par 11 ولقد وصلت المشكلة إلى حد كبير، حيث استمرت الحروب والشرور الكثيرة بينهم لفترة طويلة من الزمن.

\chapter{26}

\par \textit{رواية وفاة هيركانوس، وفترة حكمه}

\par 1 كان لهيركانوس ثلاثة أبناء، وهم: أنتيجونوس، وأريستوبولوس، والإسكندر

\par 2 وكان هيركانوس يحب أنتيجونوس وأريستوبولوس، لكن الإسكندر كان مكروهًا له

\par 3 وفي وقت رأى فيه في المنام، أن أبنائه سيحكمون الإسكندر بعد وفاته، وهذا سبب له قلقًا

\par 4 ولم يرَ من المناسب، في حياته، أن يُقيم أيًا من الأبناء الذين أحبهم، بسبب رؤيته؛

\par 5 ولا أن يعين الإسكندر ملكًا، لأنه كان يكرهه.. لذلك أرجأ الأمر؛ حتى يتخذ الأمر بعد وفاته المنحى الذي يرضي الله العظيم الصالح

\par 6 كان اليهود، في زمن أبيه وأعمامه، متحدين في المودة تجاههم، ومستعدين لطاعتهم، بسبب قهرهم لأعدائهم، والأعمال البطولية الرائعة التي قاموا بها

\par 7 واستمروا أيضًا متحدين في محبتهم لهيركانوس؛ حتى ارتكب مذبحة الفريسيين، واقتلاع اليهود، والحروب الأهلية بسبب الدين

\par 8 ومن هنا نشأت عداوات دائمة، وشرور لا تنقطع، وجرائم قتل كثيرة. وهذا هو سبب كراهية الكثيرين لهيركانوس

\par 9 وكانت مدة ملكه إحدى وثلاثين سنة، ومات.

\chapter{27}

\par \textit{تاريخ أرستوبولوس بن هيركانوس}

\par 1 وبعد أن مات هيركانوس، خلفه ابنه أريستوبولس على العرش؛ الذي أظهر الغطرسة والكبرياء والقوة؛ ووضع على رأسه تاجًا كبيرًا، احتقارًا لتاج الكهنوت المقدس.

\par 2 وكان يميل بمودة نحو أخيه أنتيجونوس، الذي كان يفضله على كل أصدقائه: لكنه احتفظ بأخيه في السجن، وكذلك أمه، بسبب حبها للإسكندر.

\par 3 "فأرسل أخاه أنتيجونوس فحاربه وانتصر عليه هو وجميع أنصاره وجيشه، فهزمهم ورجع إلى مدينة البيت المقدس. وكان هذا في حينه أريستوبولس مريضاً."

\par 4 "ولما كان أنتيجونوس في طريقه إلى المدينة، أُبلغ بمرض أخيه، فدخل المدينة وذهب إلى بيت الله ليشكر الله على رحمته في خلاصه من العدو، وليتوسل إلى الله العظيم الصالح أن يعيد الصحة إلى أخيه."

\par 5 ولذلك ذهب بعض الذين كانوا أعداء أنتيجونوس وكارهينه إلى أريستوبولوس وقالوا:

\par 6 لقد تم نقل خبر مرضك إلى أخيك، وها هو قادم مع أنصاره مسلحًا، وقد ذهب الآن إلى الحرم ليصنع لنفسه أصدقاء، حتى يأتي عليك فجأة ويقتلك.

\par 7 وكان الملك أريستوبولس خائفًا من اتخاذ أي خطوة متسرعة ضد أخيه فيما يتعلق بما قيل له، حتى يعرف صحة الخبر.

\par 8 لذلك أمر جميع حاشيته بأن يرابطوا مسلحين في مكان معين، بحيث لا يستطيع أي شخص يأتي إلى قصره أن يحيد عنه.

\par 9 كما أمر بأن يُعلن علنًا أنه لا يجوز لأحد يحمل سلاحًا من أي نوع أن يدخل إلى الملك إلى الديوان دون أن يكون مختبئًا

\par 10 بعد ذلك، أرسل إلى أنتيجونوس، يأمره بالمجيء إليه: وعندها خلع أنتيجونوس سلاحه طاعةً للملك

\par 11 في هذه الأثناء، يأتي إليه رسول من زوجة أخيه أريستوبولوس (التي كانت تكرهه)، ويقول له:

\par 12 يقول لك الملك: "لقد سمعت الآن عن جمال لباسك عندما دخلت المدينة، وأرغب في رؤيتك وأنت ترتدي هذا الثوب؛ لذا تعال إليّ بهذه الهيئة، حتى أكون سعيدًا برؤيتك."

\par 13 ولم يشك أنتيجونوس في أن هذه الرسالة كانت من الملك، كما أفاد الرسول؛

\par 14 وأنه لا يريد أن يضعه على قدم المساواة مع الآخرين فيما يتعلق بإلقاء أسلحتهم: فذهب إليه بتلك الطريقة واللباس

\par 15 ولما وصل إلى المكان الذي أمر الملك أريستوبولس رجاله بالتمركز فيه، وأوامر بقتل أي شخص يأتي إلى هناك مسلحًا؛

\par 16 وعندما رآه الرجال يرتدي ذراعيه، اندفعوا نحوه وقتلوه على الفور، وسال دمه على الرصيف الرخامي في ذلك المكان

\par 17 واشتدت صرخة الرجال، وتضاعف بكاؤهم ونحيبهم، حزنًا على موت أنتيجونوس، على جماله، وأناقة حديثه، ومآثره

\par 18 فلما سمع الملك ضجيج الرجال، استفسر عن الأمر، فوجد أن أنتيجونوس قد قُتل؛

\par 19 مما تسبب له في حزن شديد، سواء بسبب المودة التي كان يكنها له، أو لأنه لم يستحق هذا المصير: وأدرك أن فخًا قد نُصب لأخيه:

\par 20 فصرخ بصوت عالٍ وبكى بكاءً شديدًا، وضرب صدره ضربًا مبرحًا، حتى انفجرت بعض أوعية صدره، وسال الدم من فمه

\par 21 لكن أتاه خدمه ورئيس أصدقائه، يعزونه ويهدئونه ويهدئونه، لكي يكبحوه عن هذا الفعل؛

\par 22 خوفًا من أن يموت، لأنه كان ضعيفًا، وكان على وشك الموت بسبب ما فعله بالفعل

\par 23 وأخذوا طستًا من ذهب ليجمعوا فيه الدم الذي انسكب من فمه

\par 24 وأرسلوا الطست، الذي فيه الدم، بواسطة أحد الحاضرين إلى طبيب، ليراه، وينصح بما يجب فعله له

\par 25 وذهبت الصفحة مع الطست، وعندما وصلت إلى المكان الذي قُتل فيه أنتيجونوس، وكان دمه يسيل، انزلقت الصفحة وسقطت، وسفك دم الملك الذي كان في الطست على دم أخيه المقتول

\par 26 فعاد الرقيب بالحوض وأخبر رجال الحاشية بما حدث، فسبوه وشتموه، بينما برر هو نفسه وأقسم أنه لم يفعل هذا عمداً أو طوعاً.

\par 27 ولكن عندما سمع الملك يتشاجرون، طلب أن يُخبر بما يقولون، فأمسكوا ألسنتهم. ولكن عندما هددهم، أخبروه

\par 28 ثم قال: «المجد للقاضي العادل، الذي سفك دم الظالم على دم المظلوم».

\par 29 ثم تأوه ومات في الحال. وكانت مدة حكمه سنة كاملة

\par 30 وكان جميع رعيته ينوحون عليه، لأنه كان نبيلًا، منتصرًا، كريمًا، وملك أخوه الإسكندر مكانه

\chapter{28}

\par \textit{رواية الإسكندر بن هيركانوس}

\par 1 وبعد أن مات أريستوبولس، أُطلق سراح شقيقه الإسكندر من قيوده، وأُخرج من السجن، وتولى العرش.

\par 2 كان حاكم مدينة أكي (وهي بطليموس) قد تمرد، وأرسل رسلًا إلى ليثراس ابن كليوباترا، طالبًا منه مساعدته، وأن يأخذه تحت حمايته؛

\par 3 لكنه رفض لفترة طويلة، خوفًا من تكرار الأشياء التي عانى منها سابقًا من هيركانوس

\par 4 لكن الرسول شجّعه بالمساعدات التي وعد بها سيد صور وصيدا وآخرين. وسار ليثراس مع ثلاثين ألف رجل

\par 5 وأُبلغ الإسكندر بذلك، فسبقه في بطليموس وهاجمها، وأغلق مواطنو بطليموس الباب في وجهه، وحاولوا منعه من الدخول

\par 6 لذلك ضيّق عليهم الإسكندر، واستمر في محاصرتهم؛ حتى أُبلغ بزحف ليثراس: ثم انسحب من أمامهم، وكان ليثراس وجنوده في متناول اليد

\par 7 كان بين مواطني بطليموس رجل عجوز ذو سلطة معترف بها، أقنع المواطنين بعدم السماح لليثراس بدخول مدينتهم، أو بطاعته، لأنه كان من ديانة مختلفة

\par 8 وقال لهم أيضًا: «سيكون من الأفضل لكم في كل شيء الخضوع للإسكندر، الذي هو من نفس الدين، من الخضوع لليثراس». ولم يتوقف حتى وافقوا على آرائه

\par 9 ومنعوا ليثراس من دخول بطليموس، رافضين الخضوع له. وكان ليثراس حائرًا في أمره، ولم يتشاور فيما هو الأفضل له أن يفعله

\par 10 وأُخبر ملك صيدا بذلك، فأرسل إليه رسلًا ليساعده في الحرب ضد الإسكندر، إما ليهزموه، أو ليحتلوا بعض مدنه، وبالتالي يعاقبوه؛

\par 11 وهكذا قد يعود ليثراس إلى بلده، بعد القيام بأعمال قد تجعله هائلاً؛ والتي في الحقيقة ستكون أكثر فائدة له من العودة دون تحقيق هدفه

\par 12 وأخبر الإسكندر بذلك، فأرسل إلى ليثراس سفارة مشرفة مع هدية ثمينة للغاية، واقترح عليه عدم مساعدة ملك صيدا.

\par 13 وقبل ليثراس هدية الإسكندر، ووافق على طلبه.

\par 14 "فسار الإسكندر إلى صيدا وحارب سيدها، فغلبه الله عليه، فقتل من رجاله أعداداً كثيرة، وبعد أن هزمه استولى على بلاده."

\par 15 بعد ذلك، أرسل الإسكندر رسلًا إلى كليوباترا، ليأمرها بالقدوم مع جيش ضد ليثراس ابنها؛ وأنه سيسير هو أيضًا بجيشه ضده، ويسلمه إليها سجينًا

\par 16 وعندما علم ليثراس بذلك، انطلق إلى جبل الجليل، وقتل أعدادًا كبيرة من السكان، وأسر عشرة آلاف أسير، كما قُتل عدد كبير من رجاله

\par 17 ومن هناك سار حتى وصل إلى الأردن، وخيم هناك، حتى يتمكن رجاله وخيوله من الراحة، وبعد ذلك يمكنه السير إلى القدس لمحاربة الإسكندر

\par 18 أُخبر الإسكندر بذلك، فخرج لمواجهته بخمسين ألف رجل، منهم ستة آلاف يحملون دروعًا من نحاس، ويقال إن كل واحد منهم كان قادرًا على مقاومة أي عدد من الرجال

\par 19 وهاجمه عند الأردن، واشتبك معه هناك؛ لكنه لم يحقق النصر، لأنه وثق برجاله، ووضع ثقته في عددهم

\par 20 ولكن مع ليثراس كان هناك رجال ماهرون جدًا في المعارك وفي تشكيل الجيوش؛ نصحوه بتقسيم قواته إلى قسمين، بحيث يكون أحدهما مع ليثراس وفرقته مستعدًا للمعركة، ويكون القسم الآخر مع قائد آخر من فرقتهم

\par 21 وقاتل حتى الظهر، فقُتل عدد كبير من رجاله

\par 22 وتقدم صديقه، مع بقية الجيش الذي كان معه، والذي كانت قوته لا تزال كاملة، ضد الإسكندر ورجاله، الذين كانوا قد غلبهم التعب في ذلك الوقت:

\par 23 ففعل بهم ما شاء، وقتل منهم جموعًا كثيرة، وهرب الإسكندر والرجال الذين بقوا معه إلى مدينة البيت المقدس

\par 24 غادر ليثراس أيضًا نحو المساء إلى بلدة قريبة، وبالصدفة التقت به بعض النساء اليهوديات مع أطفالهن؛

\par 25 وأمر بقتل بعض الأطفال، وتقطيع لحومهم، متظاهرًا بوجود بعض في جيشه يتغذون على لحوم البشر؛ قاصدًا بهذه الأفعال إثارة رعب قواته في سكان البلاد

\par 26 بعد ذلك، جاءت كليوباترا؛ التي التقى بها الإسكندر، وأخبرها بما فعله ليثراس بجيشه، وعيّنها للذهاب معها بحثًا عنه

\par 27 بعد أن أُخبر ليثراس بذلك، هرب إلى مكان كان محطة لسفنه، وصعد على متنها وعاد إلى قبرص، وعادت كليوباترا إلى مصر

\par 28 "ولكن في نهاية العام زحف الإسكندر نحو غزة، لأن رئيسها تمرد عليه، وأرسل إلى ملك من ملوك العرب اسمه الحارثاسي لمساعدته، فوافق على ذلك وزحف نحو غزة.

\par 29 أُخبر الإسكندر بذلك، فترك بعض رجاله أمام غزة، وسار نحو حرستا، واشتبك معه، وهزمه

\par 30 ثم عاد إلى غزة، فاستلقى عليها متألمًا، فاستقبلها في نهاية عام

\par 31 لكن سبب أخذه لها كان شقيق ذلك الزعيم، الذي هاجمه فجأة وقتله

\par 32 عندما سعى المواطنون لقتله، جمع أصدقاءه، وذهب إلى بوابة المدينة، وخاطب الإسكندر، متوسلاً إليه أن يدخل المدينة بعد أن يضمن حياته وحياة أصدقائه؛

\par 33 الذي وعده به الإسكندر، دخل غزة، وقتل سكانها، وهدم الهيكل الذي كان فيها، وأحرق الصنم المذهب الذي كان في الهيكل

\par 34 وبعد ذلك انطلق إلى مدينة البيت المقدس، واحتفل هناك بعيد المظال

\par 35 ولما انقضت الوليمة، استعد لمواجهة هارتاس، الذي واجهه، وقتل عددًا كبيرًا من رجاله:

\par 36 وكانت شؤون هارتاس متوترة للغاية ومشلولة، وكان يخشى انقراضه التام. لذلك رفع دعوى قضائية ضد الإسكندر من أجل حياته، وأخضعه ودفع له الجزية

\par 37 فانصرف الإسكندر عنه، وسار إلى حماة وصور، فاستولى عليهما، وأخذ الجزية من أهلها، ثم عاد إلى مدينة البيت المقدس

\chapter{29}

\par \textit{سرد للمعارك التي دارت بين الفريسيين والصدوقيين.}

\par 1 وبعد ذلك نشأت الشرور بين الفريسيين والصدوقيين، واستمرت لمدة ست سنوات

\par 2 وساعد الإسكندر الصدوقيين على الفريسيين، فقتل منهم خمسين ألفًا في ست سنوات

\par 3 لذلك، بين هاتين الطائفتين، انحدرت حالة الأمور إلى الدمار التام، وتأكدت عداوتهم تمامًا

\par 4 فأرسل الإسكندر في طلب شيوخ كل طائفة، وتحدث إليهم بلطف، ونصحهم بالمصالحة

\par 5 فأجابوه: «في الحقيقة، أنت، في رأينا، تستحق الموت، لكثرة الدم البريء الذي سفكت. لذلك، لا يكون بيننا إلا السيف».

\par 6 ثم بعد ذلك، بدأوا يُظهرون عداوتهم جهارًا، فأرسلوا رسلًا إلى ديمتريوس ملك مقدونيا، لكي يأتي إليهم بجيش؛

\par 7 واعدًا إياهم بأنهم سيساعدونه ضد الإسكندر وحزبه، وسيخضعون العبرانيين للمقدونيين. فسار إليهم ديمتريوس بجيش كبير

\par 8 وقد أُخبر بذلك أيضًا الإسكندر، فأرسل شخصًا لاستئجار ستة آلاف مقدوني، فانضموا إلى قواته وتقدموا ضد ديمتريوس

\par 9 وذهب كثيرون من اليهود الفريسيين إلى ديمتريوس.

\par 10 فأرسل ديمتريوس سرًا أشخاصًا إلى أولئك المقدونيين الذين كانوا مع الإسكندر، ليغويهم عنه، لكنهم لم يسمعوا له

\par 11 أرسل الإسكندر أيضًا رجالًا سرًا إلى اليهود الذين كانوا مع ديمتريوس، ليجذبهم إلى جانبه؛ لكن هؤلاء أيضًا لم يفعلوا ما أراد منهم

\par 12 والتقى الإسكندر وديمتريوس، وخاضا معركة، فسقط فيها جميع رجال الإسكندر، وهرب هو وحده إلى أرض يهوذا

\par 13 ولكن عندما سمع رجاله ذلك، همسوا أنه نجا بسلام، واكتشفوا المكان الذي كان فيه؛

\par 14 اجتمع إليه نحو ستة آلاف رجل من أشجع بني إسرائيل، وانضم إليه كثيرون ممن ثاروا على ديمتريوس

\par 15 وبعد ذلك توافد عليه الرجال من كل جانب، فعاد ليُقاتل ديمتريوس بقوة عظيمة، وهزمه، ثم عاد ديمتريوس إلى بلده

\par 16 وزحف الإسكندر لمواجهته إلى أنطاكية، وحاصرها ثلاث سنوات. وعندما خرج ديمتريوس للقتال، هزمه الإسكندر وقتله

\par 17 وخرج من المدينة، ورجع إلى أورشليم إلى أهله، الذين عظموه وأكرموه وأثنوا عليه لأنه هزم أعدائه

\par 18 فوافق اليهود على الخضوع له، فاطمأن قلبه، وأرسل جيوشه ضد جميع أعدائه، فهزمهم وانتصر عليهم

\par 19 وامتلك أيضًا جبال سارة، وبلاد عمون، وموآب، وبلاد الفلسطينيين، وجميع الأنحاء التي كانت في أيدي العرب الذين حاربوه، حتى حدود البرية

\par 20 واستقامت أمور مملكته، وجعل شعبه وبلاده في أمان

\chapter{30}

\par \textit{رواية وفاة الإسكندر بن هيركانوس}

\par 1 بعد ذلك، مرض الملك الإسكندر بحمى الربع، لمدة ثلاث سنوات كاملة

\par 2 ولكن عندما ثار عليه حاكم مدينة يُدعى راغبة، قاد إلى هناك جيشًا قويًا، وأخذ معه زوجته وعائلته، وحاصر المدينة

\par 3 ولكن عندما كاد أن يُؤخذ، ازداد مرضه وتراجعت قوته؛ وفقدت زوجته، التي كانت تُدعى ألكسندرا، كل أمل في شفائه

\par 4 قال الذي صعد إليه: "أنت تعلم الآن ما الفرق بينك وبين الفريسيين. وابناك صغيران، وأنا امرأة، ولن نستطيع مقاومتهما إطلاقًا. فماذا تنصحني وإياهما؟"

\par 5 قال لها: "نصيحتي لكِ هي أن تصمدي ضد المدينة حتى يتم الاستيلاء عليها، وهو ما سيكون قريبًا

\par 6 وعندما يتم الفوز بها، قم بإقامة حكومتها كما تم تأسيس المدن الأخرى

\par 7 ولكن تجاه كل هؤلاء الناس، تظاهر بأنني مريض؛ وأي شيء تفعله، تظاهر بأنك تفعله بناءً على اقتراحي؛ واكشف موتي لأولئك الخدم الذين يمكنك الاعتماد عليهم.

\par 8 ومتى انتهيتم من هذه الأمور، فادخلوا مدينة البيت المقدس، وقد جففتم جسدي وحنطتموه بالأطياب، واملأوا المكان الذي أضطجع فيه بطيب كثير، حتى لا تخرج مني رائحة كريهة

\par 9 وعندما تستقر أمور البلاد، اذهب من هناك، ولفّني بكمية وفيرة من العطور، واحملني إلى القصر، كما لو كنت مريضًا:

\par 10 ومتى كنت هناك، فأرسل إلى وجوه الفريسيين، ومتى جاؤوا، فأكرمهم، وكلمهم كلامًا حسنًا

\par 11 فقل: لقد مات الإسكندر بالفعل، وها أنا أسلمه لك، فافعل به ما تراه مناسبًا، ومن الآن فصاعدًا سأتصرف معك كما تشاء

\par 12 لأنكِ إن فعلتِ هذا، فأنا أعلم جيدًا أنهم لن يفعلوا بي وبك شيئًا إلا ما هو خير؛ وسيتبعهم الشعب، وستُرتب أموركِ على ما يرام بعد وفاتي، وستملكين بأمان حتى يكبر ابناكِ

\par 13 بعد ذلك، مات الإسكندر، وأخفت زوجته موته، وعندما تم الاستيلاء على المدينة، عادت إلى أورشليم، وأرسلت في طلب رؤساء الفريسيين، وخاطبتهم كما نصحها الإسكندر

\par 14 فأجابوها أن الإسكندر كان ملكهم، وأنهم كانوا شعبه؛ وتحدثوا إليها بكل مودة، ووعدوها بوضعها على رأس حكومتهم

\par 15 ثم خرجوا وجمعوا رجالاً، وأخذوا جثمان الإسكندر، وحملوه إلى دفنه بشكل رائع، وأرسلوا في طلب رجال لتنصيب ألكسندرا ملكة، وقد عُينت بموافقتهم

\par 16 وكانت سنوات ملك الإسكندر سبعًا وعشرين.

\chapter{31}

\par \textit{تاريخ الملكة ألكسندرا}

\par 1 وفي أثناء حكم ألكسندرا، استدعت رؤساء الفريسيين، وأمرتهم أن يكتبوا إلى جميع الذين فروا من طائفتهم إلى مصر ومناطق أخرى، في أيام هيركانوس والإسكندر، أن يعودوا إلى أرض يهوذا

\par 2 وأظهرت لهم ميلها الإيجابي تجاههم، ولم تعارض طقوسهم، ولم تمنع احتفالاتهم، كما منعها الإسكندر وهيركانوس

\par 3 كما أطلقت سراح جميع المعتقلين في السجن.

\par 4 فاجتمعوا من كل ناحية، وامتنع الصدوقيون عن استخدام أي عنف ضدهم.

\par 5 واستقرت أمورهم، وتحسنت أحوالهم بزوال الخلافات

\par 6 ولكن عندما كبر هيركانوس وأريستوبولوس، ابنا الإسكندر، عينت الملكة هيركانوس كاهنًا أعظم، لأنه كان وديعًا، ولطيفًا، وأمينًا:

\par 7 ولكنها جعلت أرستوبولس قائداً للجيش لأنه كان قوياً شجاعاً وذو روح عالية، وأعطته أيضاً جيش الصدوقيين، ولكنها لم ترَ أنه من اللائق أن تجعله ملكاً لأنه كان لا يزال صبياً.

\par 8 علاوة على ذلك، أرسلت إلى جميع الذين دفعوا الجزية للإسكندر، وأخذت أبناء ملوكهم، الذين احتجزتهم بالقرب منها كرهائن؛ واستمروا في طاعتها دون انقطاع، يدفعون الجزية كل عام

\par 9 وسارت باستقامة مع شعبها، توزع العدل، وتأمر شعبها بفعل الشيء نفسه. ولذلك كان هناك سلام دائم بين الطرفين، ونالت رضاهم

\chapter{32}

\par \textit{سرد للأمور التي فعلها الفريسيون بالصدوقيين في زمن الإسكندرية.}

\par 1 كان بين الصدوقيين زعيم، رقّاه الإسكندر، يُدعى ديوجين، الذي كان قد حثّه سابقًا على قتل ثمانمائة رجل من الفريسيين

\par 2 لذلك جاء رؤساء الفريسيين إلى الإسكندرية، وذكروها بما فعله ديوجين، وطلبوا منها الإذن بقتله، فأذنت لهم. ولما أذنت لهم، قتلوا معه كثيرين من الصدوقيين

\par 3 فأخذ الصدوقيون ذلك على محمل الجد، فذهبوا إلى أرسطوبولس، وأخذوه معهم، وذهبوا إلى الملكة، وقالوا لها:

\par 4 أنت تعلم مدى فظاعة الأمور وثقلها التي مررنا بها، والحروب والمعارك العديدة التي خضناها لمساعدة الإسكندر وأبيه هيركانوس

\par 5 لذلك لم يكن من اللائق أن ندوس على حقوقنا، وأن نرفع يد أعدائنا علينا، وأن نحط من كرامتنا؛

\par 6 لن يخفى أمرٌ من هذا القبيل على هارتاس وغيره من أعدائك؛ الذين اختبروا شجاعتنا، ولم يتمكنوا من مقاومتنا، وامتلأت قلوبهم بالخوف منا

\par 7 لذلك، عندما يدركون ما فعلتموه بنا، سيتخيلون أن قلوبنا تُدبّر خططًا ضدكم؛ وعندما يتم التحقق منها، ثقوا أنها ستخدعكم

\par 8 ولا نحتمل أن يقتلنا الفريسيون كالغنم.

\par 9 "فإما أن تكف شرورهم عنا، أو تسمح لنا بالخروج من المدينة إلى بعض مدن يهوذا."

\par 10 فقالت لهما: «افعلا هذا لكي لا يُزعجكما».

\par 11 وخرج الصدوقيون من المدينة، وذهب رؤساؤهم مع رجال الحرب الذين انضموا إليهم، وذهبوا مع مواشيهم إلى مدن يهوذا التي اختاروها، وسكنوا بها

\par 12 وانضم إليهم أهل الخير (أي الحسنيين).

\chapter{33}

\par \textit{رواية وفاة ألكسندرا}

\par 1 بعد هذه الأمور، أصيبت ألكسندرا بمرض، ماتت بسببه

\par 2 ولما كاد اليأس يغلب على شفائها، خرج ابنها أريستوبولس من أورشليم ليلاً، وكان برفقته خادمه.

\par 3 ثم ذهب إلى جبعات إلى رجل من زعماء الصدوقيين، أحد أصدقائه.

\par 4 فأخذه معه، وسار إلى المدن التي يسكنها الصدوقيون، وكشف لهم عن نواياه، وحثهم على الخروج معه، وأن يكونوا حلفاءه في الحرب ضد أخيه والفريسيين، وأن يجعلوه ملكًا

\par 5 الذين وافقوا عليهم، خدعوا ألكسندرا علانية، وجمعوا رجالًا من ربع قدم للانضمام إلى أريستوبولوس

\par 6 ولما بلغ خبر هذه الأمور هيركانوس بن ألكسندرا، رئيس الكهنة، وشيوخ الفريسيين، ذهبوا إلى ألكسندرا، وهي مريضة، وأخبروها بالأمر؛

\par 7 يضغطون عليها الخوف الشديد الذي كان لديهم عليها وعلى ابنها هيركانوس من أريستوبولس ومن كانوا معه

\par 8 فأجابت: «أنا قريبة من الموت حقًا، لذا من الأنسب والأنفع لي أن أهتم بشؤوني الخاصة؛ فماذا يمكنني أن أفعل إذًا وأنا في هذه الحالة؟»

\par 9 لكن رجالي، وأمتعتي، وأسلحتي، معكم وفي أيديكم؛ لذا نظموا العمل كما ترونه صحيحًا، متوسلين إلى الله بالعون في أموركم، واطلبوا منه النجاة. ثم ماتت

\par 10 كان مقدار عمرها ثلاثًا وسبعين عامًا، ومدة حكمها تسع سنوات،

\chapter{34}

\par \textit{رواية هجوم أريستوبولوس على أخيه هيركانوس، بعد وفاة ألكسندرا.}

\par 1 عندما غادر أرسطوبولس أورشليم في أيام الإسكندرية، ترك زوجته وأولاده في أورشليم

\par 2 ولكن عندما وصل خبر رحيله إلى ألكسندرا، حبستهم في منزل معين، ووضعت عليهم حارسًا

\par 3 ولكن عندما ماتت ألكسندرا، دعاهم هيركانوس إليه، وتصرف معهم بلطف، واعتنى بهم؛ حتى يتمكنوا من إنقاذه من أخيه، إن أمكنه أن يغلبه

\par 4 ثم خرج أرستوبولس بجيش عظيم إلى الأردن، وخرج هيركانوس لملاقاته بجيش من الفريسيين.

\par 5 ولما التقى الجيشان، قُتل عدد كبير من جيش هيركانوس، وهرب هيركانوس وبقية جيشه.

\par 6 فقتل أريستوبولس وجنوده كل من وجدوه إلا الذين استسلموا.

\par 7 ثم تراجع هيركانوس إلى المدينة المقدسة، حيث وصل إليها أريستوبولس وجيشه، وحاصرها من كل جانب بخيامه، وحاول تدمير الحصن بالمكر.

\par 8 فخرج إليه شيوخ يهوذا وشيوخ الكهنة، ومنعوه من فعل ما كان ينوي فعله، وطلبوا منه أن يزيل من قلبه كل شعور عدائي تجاه أخيه، فوافق على هذا الاقتراح.

\par 9 ثم اتفقا على أن يكون أرستوبولس ملكاً على يهوذا، وهركانوس رئيساً للكهنة في بيت الله، ويكون في المرتبة التالية للملك.

\par 10 فوافق أرستوبولس على هذه الشروط، ودخل المدينة، والتقى بأخيه في بيت الله، وحلف الاثنان يمينًا معًا على التصديق على تلك الشروط التي اتفق عليها الشيوخ.

\par 11 وهكذا أصبح أريستوبولوس ملكًا، وكان هيركانوس يليه في المرتبة

\par 12 وكان الناس في سلام، وانتظمت أمور هذين الأخوين، وأصبح حال شعبهما وبلدهما في طمأنينة

\chapter{35}

\par \textit{رواية أنتيباتر (أي هيرودس الملك)، والفتن والمعارك التي أشعلها بين هيركانوس وأريستوبولوس.}

\par 1 كان رجل من اليهود، من أبناء بعض الذين صعدوا من بابل مع عزرا الكاهن، اسمه أنتيباتر

\par 2 وكان حكيمًا، فطنًا، حاد الذكاء، شجاعًا، رفيع الشأن، حسن الخلق، لطيفًا، مهذبًا؛ وكان غنيًا، يملك بيوتًا وأملاكًا ومواشي كثيرة

\par 3 هذا الرجل عيَّنه الملك الإسكندر حاكمًا على بلاد الأدوميين، ومن هناك تزوج منها، وأنجب منها أربعة أبناء، وهم: فاسيلوس، وهيرودس الذي حكم يهوذا، وفيروراس، ويوسيفوس

\par 4 بعد ذلك، بعد أن نُقل من جبال سارة، أي بلاد الأدوميين، في أيام الإسكندر، سكن في مدينة البيت المقدس:

\par 5 وكان هيركانوس يحبه، وكان يميل إليه كثيرًا: لذلك سعى أريستوبولوس لقتله؛ وهو ما لم ينجح فيه

\par 6 لذلك كان أنتيباتر خائفًا للغاية من أريستوبولوس، ولهذا السبب بدأ في التآمر سرًا ضد مملكة أريستوبولوس

\par 7 فذهب إلى كبار رجال المملكة، وبعد أن حصل منهم على تعهد بالحفاظ على السرية فيما يتعلق بالأمور التي كان على وشك إبلاغها،

\par 8 بدأ يتحدث إليهم عن حياة أرسطوبولس المشينة، وطغيانه، وكفره، وإراقة الدماء التي تسبب فيها، واغتصابه للعرش؛ الأمر الذي كان أخوه الأكبر أحق به

\par 9 ثم أمرهم بالحذر من الإله العظيم الصالح، ما لم يسلبوا يد الطاغية الحاكمة، ويعيدوا ما كان مستحقًا لملكهم الشرعي

\par 10 ولم يبقَ أحد من كبار الرجال، لم يتجاوزه، ويميل إلى الخضوع لهيركانوس، ويغويهم عن طاعتهم لأريستوبولوس، إذ لم يكن هيركانوس على علم بأي شيء من هذا القبيل

\par 11 لكن أنتيباتر نسب إليه كل هذا، لأنه لم يرغب في إخباره قبل أن يثبت الأمر.

\par 12 وبعد أن حسم هذه المسألة مع الشعب ذهب إلى هيركانوس وقال له:

\par 13 إن أخاك يخاف منك بشدة، لأنه يرى أن حالته لن تكون آمنة وأنت على قيد الحياة؛ ولذلك فهو يبحث عن فرصة لقتلك، ولن يدعك تعيش

\par 14 لكن هيركانوس لم يُصدّقه، بسبب طيبة قلبه وصدقه. ولذلك كرّر أنتيباتر هذا الحديث عليه مرارًا وتكرارًا

\par 15 كما أعطى مبالغ كبيرة من المال للأشخاص الذين وضع هيركانوس ثقته فيهم، واتفق معهم على أن يخبروه بأشياء مماثلة لما ذكره أنتيباتر؛

\par 16 مع الحرص فقط على ألا يتخيل أنهم يعرفون أن أنتيباتر كان يتحدث إليه حول هذا الموضوع

\par 17 فصدق هيركانوس كلامهم، ودُفع إلى وضع خطة يمكن من خلالها أن يتحرر من أخيه

\par 18 لذلك عندما تحدث إليه أنتيباتر مرة أخرى عن الأمر، أخبره أن حقيقة كلماته قد اتضحت له الآن، وأنه يعلم أنه نصحه جيدًا؛ وطلب مشورته في هذا الأمر

\par 19 ونصحه أنتيباتر بالخروج من المدينة إلى شخص يمكنه أن يثق به، وقد يكون قادرًا على مساعدته وإعانته

\par 20 وذهب أنتيباتر إلى هارتام، واتفق معه على أن يستقبل هيركانوس كضيف عند مجيئه، لأنه كان يخشى السكنى مع أخيه

\par 21 عندها فرح هارتام، ودخل في الخطة، واتفق مع أنتيباتر على أنه لن يسلم هيركانوس وأنتيباتر لأعدائهما بأي حال من الأحوال، وأنه سيساعدهما ويحميهما

\par 22 ثم رجع إلى أورشليم، وأخبر هيركانوس بما فعل، وكيف اتفق مع هرتام على ذهابهم إليه

\par 23 لذلك خرجا كلاهما من المدينة ليلًا، وذهبا إلى هارتام، وأقاما معه بعض الوقت

\par 24 ثم بدأ أنتيباتر في إقناع هارتام بقيادة جيش مع هيركانوس، لهزيمة أخيه أريستوبولوس والقبض عليه

\par 25 لكن هارتام رفض تنفيذ هذه الخطة، خوفًا من أنه لا يملك القوة لمقاومة أريستوبولوس

\par 26 لكن أنتيباتر لم يكف عن إظهار سهولة التعامل مع أريستوبولوس، وحثه على ذلك بحجج تتعلق بالكنز الذي سيكتسبه، وعظمة المجد الذي سيكتسبه، والذكرى التي سيتركها وراءه

\par 27 حتى وافق على الزحف؛ ولكن بشرط أن يعيد إليه هيركانوس أي مدن وبلدات تابعة له كان والده الإسكندر قد سلبها

\par 28 وافق هيركانوس على المعاهدة وأكملها، وسار هارتام (وهيركانوس معه) مع خمسين ألف فارس وجندي مشاة، متجهًا نحو بلاد يهوذا: فخرج أريستوبولوس وقاتلهم

\par 29 ولما اشتد القتال، ذهب كثير من جيش أريستوبولوس إلى هيركانوس

\par 30 ما إن أدرك أريستوبولوس ذلك، حتى أمر بالانسحاب، وعاد إلى معسكره، خوفًا من أن ينسحب جيشه بأكمله تدريجيًا إلى العدو، وبالتالي يُؤخذ هو نفسه أسيرًا

\par 31 ولكن عندما حل الليل، غادر أريستوبولس المخيم وحده، وذهب إلى المدينة المقدسة

\par 32 وعندما علم الجيش برحيله عند طلوع الفجر، انضم معظمهم إلى هيركانوس، وتفرق الباقون وذهبوا في طريقهم

\par 33 وأما هيركانوس وهارتام وأنتيباتر، فقد ذهبوا مباشرة إلى مدينة البيت المقدس، ومعهم جيش كبير؛

\par 34 ووجدوا أريستوبولوس مستعدًا بالفعل للحصار؛ لأنه أغلق أبواب المدينة، ووضع رجالًا على الأسوار للدفاع عنها

\par 35 ونزل هيركانوس وهرتام بقواتهما على المدينة وحاصراها


\chapter{36}

\par \textit{تاريخ جنيوس، قائد جيش الرومان.}

\par 1 وحدث أن غنوس، قائد جيش الرومان، خرج لقتال تيركانس الأرمني:

\par 2 لأن مواطني دمشق، وحماة وحلب، وبقية سوريا الذين ينتمون إلى الأرمن، قد تمردوا مؤخرًا ضد الرومان:

\par 3 ولهذا السبب أرسل غنوس سكوروس إلى دمشق وأراضيها للاستيلاء عليها؛ وهو الأمر الذي روى لأريستوبولوس وهيركانوس

\par 4 لذلك أرسل أريستوبولوس سفراء إلى سكوروس، ومالًا كثيرًا، طالبًا منه أن يأتي إليه بجيش، ويساعده ضد هيركانوس

\par 5 أرسل هيركانوس أيضًا سفراء إليه، طالبًا مساعدته ضد أريستوبولوس؛ لكنه لم يرسل إليه هدية

\par 6 لكن سكوروس رفض الذهاب إلى أي منهما: لكنه كتب إلى هارتام، يأمره بالانسحاب مع جيشه من مدينة البيت المقدس، ونهى عن مساعدة هيركانوس ضد أخيه؛

\par 7 وهدد بأنه سيدخل بلاده بجيش من الرومان والسوريين ما لم يطع

\par 8 عندما وصلت هذه الرسالة إلى هارتام، انسحب على الفور من المدينة:

\par 9 تراجع هيركانوس أيضًا، فطارده أريستوبولس مع عدد من جنوده، وأدركهم واشتبك معهم، وقُتل عدد كبير من العرب في تلك المعركة، وعدد كبير جدًا من اليهود، وعاد أريستوبولس إلى المدينة المقدسة

\par 10 في هذه الأثناء، وصل غنوس إلى دمشق؛ فأرسل إليه أريستوبولوس، عن طريق رجل يُدعى نيكوميديس، حديقةً وكرومًا من الذهب، يزن مجموعها خمسمائة وزنة، مع هدية ثمينة للغاية؛ وتوسل إليه أن يساعده ضد هيركانوس

\par 11 أرسل هيركانوس أيضًا أنتيباتر إلى بومبي، مع طلب مماثل.

\par 12 وكان بومبي (الذي هو جنيوس) يميل إلى مساعدة أريستوبولوس.

\par 13 وعندما رأى أنتيباتر ذلك، انتهز فرصة التحدث مع بومبي على انفراد، وقال له:

\par 14 «في الحقيقة، لا داعي لإعادة الهدية التي تلقيتها من أريستوبولوس إليه، حتى لو لم تساعده؛

\par 15 ومع ذلك، يعرض عليك هيركانوس ضعف هذا المبلغ: ولن يتمكن أريستوبولوس من إخضاع اليهود لك، ولكن هيركانوس سيفعل هذا

\par 16 وافترض بومبي أن الأمر كما قال أنتيباتر، وفرح لأنه اعتقد أنه يستطيع إخضاع اليهود لسلطانه

\par 17 ولذلك قال لأنتيباتر: سأساعد صديقك ضد أريستوبولس، على الرغم من أنني قد أتظاهر بمساعدته ضدك، حتى يأتمنني على نفسه.

\par 18 لأني متأكد من أنه بمجرد أن يكتشف أنني أساعد أخاه ضده، سيتظاهر بالكذب مع جميع رجاله، وسيهتم بنفسه، وسيتأخر عمله لفترة أطول

\par 19 لكنني سأرسل في طلبه، وسأذهب معه إلى المدينة المقدسة، ثم سأعمل على أن يحصل صديقك على حقه؛ ولكن بشرط أن يدفع لنا جزية سنوية

\par 20 رسول أرسطوبولس. بعد ذلك، أرسل في طلب نيقوميدس، وقال له: "اذهب إلى سيدك، وأخبره أنني قد وافقت على طلبه؛ واحمل إليه رسالتي، وقل له إنه يجب أن يأتي إليّ بسرعة دون تأخير، لأني أنتظره."

\par 21 وكتب رسالة إلى أريستوبولوس، وهذه نسخة منها:

\par 22 "من جنيوس، قائد جيش الرومان، إلى الملك أريستوبولوس، وريث العرش والكهنوت الأعظم، لك الصحة والعافية.

\par 23 لقد وصلت حديقتك وكرمة الذهب الخاصة بك؛ وقد استلمتهما، وأرسلتهما إلى "الشيوخ والحكام"؛ وقد قبلوها "ووضعوها في الهيكل!" في روما، شاكرين لك

\par 24 وقد كتبوا لي أيضًا أن أساعدك وأجعلك ملكًا على اليهود

\par 25 لذلك، إذا كنت ترى أنه من المناسب "أن تأتي إليّ بكل سرعة، لأصعد معك إلى المدينة المقدسة، وألبي رغباتك، فسأفعل ذلك".

\par 26 وانصرف نيقوميدس إلى أرسطوبولس حاملاً رسالة غنوس. وعاد أنتيباتر إلى هيركانوس وأخبره بوعد غنوس، ونصحه بالذهاب إلى دمشق

\par 27 فذهب هيركانوس إلى دمشق. وذهب أريستوبولوس أيضًا. والتقيا في دمشق في قاعة بومبي (أي غنوس)؛ وقال أنتيباتر وشيوخ اليهود لغنوس:

\par 28 «اعلم، أيها القائد الجليل، أن هذا أريستوبولوس قد خدعنا، واغتصب بحد السيف مملكة أخيه هيركانوس، الذي هو أحق بها منه، لأنه الأخ الأكبر، وأسلوب حياة أفضل وأصح

\par 29 ولم يكتفِ بظلم أخيه، بل ظلم كل الأمم التي حولنا، سفك دماءهم ونهب أموالهم ظلماً، وأبقى العداوة بيننا وبينهم، وهو أمر نكرهه

\par 30 ثم وقف ألف رجل مسن، يشهدون على صحة أقواله

\par 31 فقال أرسطوبولس: «حقًا، هذا أخي أفضل مني؛ لكنني لم أطلب العرش حتى رأيت أن جميع الذين كانوا خاضعين لأبينا الإسكندر كانوا يعاملوننا بالكذب بعد وفاته، لعلمهم بعجز أخي

\par 32 وعندما نظرت في الأمر، أدركت أنه كان من واجبي أن أتولى السيادة، لأنني كنت أفضل منه في أمور الحرب، وبالتالي كنت أكثر ملاءمة للحفاظ على النظام الملكي.

\par 33 ودخلت في حرب مع كل من ظلمنا، وأخضعتهم للطاعة. وهذه كانت وصية أبينا قبل وفاته

\par 34 وقدم شهودًا يشهدون على صدق أقواله

\par 35 بعد هذه الأمور، غادر بومبي مدينة دمشق، مسافرًا إلى البيت المقدس

\par 36 لكن أنتيباتر أرسل سرًا إلى سكان المدن التي استولى عليها أريستوبولوس، يحرضهم على الشكوى إلى غنوس، موضحًا الطغيان الذي مارسه عليهم؛ وهو ما فعلوه

\par 37 وأمره غنوس أن يكتب لهم شهادةً بحريتهم، وأن يقول إنه لن يُزعجهم بعد الآن بأي حال من الأحوال؛ وهو ما فعله بالفعل، وتحررت الأمم من طاعتها لليهود

\par 38 ولكن عندما رأى أريستوبولوس ما فعله به غنوس، انصرف هو ورجاله ليلًا من جيش غنوس دون أن يطلعوه عليه، وذهبوا إلى مدينة البيت المقدس:

\par 39 وتبعه غنوس حتى وصل إلى مدينة البيت المقدس، فخيم حولها

\par 40 ولكن عندما رأى ارتفاع أسوارها، وقوة مبانيها، وكثرة الرجال فيها، والجبال المحيطة بها، أدرك أن التملق والمكر سيكونان أكثر فائدة ضد أرسطوبولس من أعمال الاستفزاز:

\par 41 لذلك أرسل إليه سفراء ليخرج إليه، واعدًا إياه بالسلامة. فخرج إليه أريستوبولس، الذي استقبله غنوس بلطف، ولم ينطق بكلمة واحدة عن أفعاله السابقة. بعد ذلك، قال أريستوبولس لغنوس،

\par 42 «أتمنى أن تساعدني ضد أخي، فلا تمنح أعدائي أي سلطة عليّ؛ ولهذا سيكون لك ما تشاء.»

\par 43 أجاب غنوس: "إذا كنت ترغب في ذلك، فأحضر لي أي أموال وأحجار كريمة موجودة في الهيكل، وسأجعلك مالكًا لما ترغب فيه." فقال له أرسطوبولس:

\par 44 «لا شك أنني سأفعل هذا.» وأرسل غنوس قائدًا يُدعى غابينيوس مع عدد كبير من الرجال، لاستلام ما كان في الهيكل من ذهب ومجوهرات

\par 45 لكن المواطنين والكهنة رفضوا السماح بذلك، ولذلك قاوموا غابينيوس، وقتلوا العديد من رجاله وأصدقائه، وطردوه من المدينة

\par 46 وعندها، غضب جنيوس على أريستوبولوس، فألقاه في السجن

\par 47 ثم سار مع جيشه، ليشق طريقه إلى المدينة ويدخلها. لكن عددًا كبيرًا من المواطنين الخارجين، منعوه من القيام بذلك، بقتل أعداد كبيرة من رجاله

\par 48 "والحقيقة أن الأعداد والروح والشجاعة التي رآها في الأمة قد أخافته، حتى أنه قرر الانسحاب منها خوفًا من أن تنشأ خلافات مؤذية في المدينة بين أصدقاء أريستوبولس وأصدقاء هيركانوس."

\par 49 أراد بعضهم فتح أبواب بومبي، لكن آخرين عارضوا ذلك. ولذلك، تبادلوا الضربات بسبب ذلك؛ ومع ازدياد هذه الحالة بدلًا من أن تتراجع، استمرت الحرب

\par 50 لاحظ بومبي ذلك، فحاصر بجيشه بوابة المدينة، وعندما فتح له بعض الناس بوابة صغيرة، دخل واستولى على قصر الملك، لكنه لم يستطع الوصول إلى المعبد، لأن الكهنة أغلقوا الأبواب، وأمنوا المداخل برجال مسلحين

\par 51 فأرسل إليهم رجالاً لمهاجمتهم من كل جانب، فهزموهم. فجاء أصدقاؤه إلى الهيكل، وصعدوا على السور ونزلوا إليه، وفتحوا أبوابه، بعد أن قتلوا جمعًا من الكهنة

\par 52 ثم جاء جنيوس ودخلها، وأعجب بجمالها وعظمتها التي رآها، واندهش عندما رأى ثرواتها والأحجار الكريمة التي كانت فيها

\par 53 وامتنع عن أخذ أي شيء منه، وأمر الكهنة بتطهير البيت من القتلى، وتقديم الذبائح حسب طقوس بلادهم

\chapter{37}

\par \textit{رواية تعيين هيركانوس ابن الإسكندر ملكًا على اليهود، وعودة قائد الجيش الروماني إلى روما.}

\par 1 بعد أن رتب بومبي هذه الأمور، عيّن هيركانوس ملكًا؛ وأسر أخاه أريستوبولوس مقيدًا بالسلاسل:

\par 2 كما أمر بألا يكون لليهود أي سلطان على تلك الأمم التي خضعت لملوكها قبل وصوله؛

\par 3 وطلب الجزية من مدينة البيت المقدس، وعقد عهدًا مع هيركانوس بأن ينال التنصيب من الرومان كل عام

\par 4 ورحل، آخذًا معه أريستوبولس، واثنين من أبنائه، وبناته. وبقي له ابن اسمه الإسكندر، لم يستطع بومبي القبض عليه لأنه هرب

\par 5 لذلك وضع بومبي في غرفته بمدينة البيت المقدس، هيركانوس، وأنتيباتر، ومعهما زميله سكوروس

\chapter{38}

\par \textit{تاريخ الإسكندر بن أريستوبولوس}

\par 1 عندما انطلق بومبي إلى روما، سار هيركانوس وأنتيباتر ضد العرب، لإخضاعهم لسيطرة الرومان

\par 2 وقد استسلم العرب لذلك، واثقين بعلاقتهم الحميمة مع أنتيباتر، وأولوا اهتمامًا كبيرًا لنصيحته؛ وبهذه الأفعال كان أنتيباتر يهدف إلى مصالحة الرومان معه.

\par 3 "ولذلك عندما علم الإسكندر بن أرسطو بحملة هيركانوس وأنتيباتر وسكوروس ضد العرب، وأنهم قد غادروا مسافة كبيرة من المدينة المقدسة؛

\par 4 فسافر حتى وصل إلى هناك، ودخل القصر وأخرج منه المال اللازم لإصلاح سور المدينة الذي هدمه بومبي.

\par 5 وجمع لنفسه جيشًا، ورتب كل ما أراده، قبل أن يعود هيركانوس وحزبه إلى مدينة البيت المقدس، وعندما عادوا،

\par 6 خرج للقائهم، واشتبك معهم، ودفعهم إلى الفرار.


\chapter{39}

\par \textit{تاريخ جابينيوس والإسكندر بن أريستوبولوس.}

\par 1 وكان غابينيوس قد خرج من روما ليسكن في أرض سورية ليعتني بها؛

\par 2 وأُخبر بما فعله الإسكندر بن أريستوبولوس، ببناء ما هدمه بومبي، ومعارضة خليفته، وقتل أصدقائه

\par 3 لذلك ذهب مباشرة حتى وصل إلى أورشليم، وانضم إليه هيركانوس وحزبه

\par 4 الذين خرج ضدهم الإسكندر بعشرة آلاف راجل وألف وخمسمائة فارس، وواجههم:

\par 5 فهزموه وقتلوا نفراً من أصحابه، فهرب إلى مدينة في أرض يهوذا يقال لها الإسكندرية، وهناك تحصن هو ورفاقه.

\par 6 فزحف عليه هيركانوس وجابينيوس وقواتهما وحاصروه.

\par 7 وخرج الإسكندر لملاقاتهم، واشتبك معهم، وقتل من رجالهم أعدادًا كبيرة

\par 8 وزحف عليه مرقس، الملقب أنطونيوس، وأجبره على الفرار مرة أخرى إلى الإسكندرية

\par 9 وخرجت والدة الإسكندر إلى غابينيوس، مستهجنة غضبه، وتوسلت إليه أن يمنح ابنها الإسكندر حياته:

\par 10 الذي وافقه غابينيوس في هذه النقطة، فخرج إليه الإسكندر، فقتله غابينيوس، ورأى أنه من المناسب تقسيم أراضي يهوذا إلى خمسة أقسام

\par 11 إحداها هي بلاد القدس والأجزاء المجاورة لها؛ وقد عُيّن هيركانوس حاكمًا على هذا الجزء. جزء آخر هو غاديرا، والأماكن المحيطة بها

\par 12 الثالثة هي أريحا والسهول. الرابعة هي حماة في أرض يهوذا. والخامسة هي صفوريس

\par 13 كان ينوي بهذه الوسائل إزالة الحروب والفتن من أرض يهوذا؛ لكنها لم تُزل بأي حال من الأحوال

\chapter{40}

\par \textit2{تاريخ قتال أريستوبولوس وابنه أنتيجونوس من روما، وعودتهما إلى أرض يهوذا: أيضًا، سرد لوفاة أريستوبولوس}

\par 1 ثم خطط أريستوبولس للأمور، حتى نجح في الهروب من روما مع ابنه أنتيجونوس، ووصل إلى مدينة يهوذا.

\par 2 ولما ظهر أريستوبولس أمام العامة، اجتمع حوله جمع غفير من الرجال، فاختار منهم ثمانية آلاف، وزحف على جابينيوس، واشتبك معه، فقتل من الجيش الروماني عدداً كبيراً جداً.

\par 3 وسقط من رجاله أيضًا سبعة آلاف، لكن ألفًا واحدًا نجا؛ وطارده جيش العدو؛ لكنه والذين بقوا له لم يتوقفوا عن المقاومة حتى دمر رجاله بالكامل؛

\par 4 ولم يبق أحد سواه وحده؛ فقاتل بشراسة حتى سقط مغمورًا بالجروح، فأُخذ واقتيد إلى غابينيوس؛ الذي أمر بالعناية به حتى يُشفى

\par 5 ثم أرسله مقيدًا إلى روما.

\par [وبقي محبوسًا في السجن حتى عهد قيصر، فأخرجه من السجن، وملأه بالهدايا والنعم؛

\par 6 وأعطوه قائدين واثني عشر ألف رجل، وأرسلوه إلى أرض يهوذا [49 ق.م.] لفصل اليهود عن حزب بومبي، وإحضارهم لطاعة قيصر: لأن بومبي كان في ذلك الوقت حاكمًا لأرض مصر

\par 7 ووصل خبر أريستوبولوس وحزبه إلى هيركانوس، الذي كان خائفًا جدًا، وكتب إلى أنتيباتر ليحول سلطته عنه بمكائده المعتادة

\par 8 فأرسل أنتيباتر بعضًا من كبار رجال أورشليم، وأعطى أحدهم سمًا، وأوصاه بإعطائه بدهاء لأريستوبولس

\par 9 فاستقبلوه في أرض سورية كأنهم سفراء له من المدينة المقدسة، فاستقبلهم بفرح، وأكلوا وشربوا معه

\par 10 ودبر أولئك الرجال المكائد حتى أعطوه السم، فمات ودُفن في أرض سورية

\par 11 وكانت مدة حكمه، حتى أُسر أول مرة، ثلاث سنوات ونصف؛ وكان رجلاً شجاعًا وذا وزن وحسن خلق

\par 12 وكان غابينيوس قد كتب إلى مجلس الشيوخ، ليُرسل ابنيه إلى والدتهما، لأنها طلبت ذلك؛ وهو ما فعلاه

\par 13 ولكن حدث أنه عندما رحل بومبي إلى مسافة بعيدة من القدس، نقضوا عهدهم بالطاعة للرومان:

\par 14 لذلك ذهب غابينيوس ضدهم، وواجههم، وانتصر عليهم، وأخضعهم مرة أخرى للرومان

\par 15 في هذه الأثناء، تمردت أرض مصر على بطليموس، وطردته من مدينته الملكية، رافضة دفع الجزية للرومان

\par 16 عند ذلك كتب بطليموس إلى غابينيوس أنه يجب أن يأتي ويساعده ضد المصريين، حتى يتمكن من إخضاعهم مرة أخرى للرومان

\par 17 وخرج غابينيوس من بلاد سوريا، وكتب إلى هيركانوس لملاقاته بجيش، حتى يتمكنوا من الذهاب إلى بطليموس

\par 18 وذهب أنتيباتر بجيش كبير إلى غابينيوس، والتقى به في دمشق، وهنأه معه بالنصر الذي حققه على الفرس:

\par 19 وأمره غابينيوس بالإسراع إلى بطليموس، ففعل، وحارب المصريين، وقتل منهم عددًا كبيرًا جدًا

\par 20 وبعد ذلك جاء جابينيوس، وتولى مكان بطليموس على عرشه، وعاد إلى المدينة المقدسة، وجدد سيادة هيركانوس، ثم عاد إلى روما.


\chapter{41}

\par \text{تاريخ كراسوس}

\par 1 عندما عاد غابينيوس إلى روما، خدع الفرس الرومان؛

\par 2 وسار كراسوس بجيش كبير إلى سوريا، وجاء إلى أورشليم، طالبًا من الكهنة أن يسلموه أي أموال موجودة في بيت الله

\par 3 فأجابوه: كيف يكون هذا جائزًا لك، وقد اعتبره بومبي وجابينيوس وآخرون غير جائز؟ لكنه أجاب: يجب أن أفعل ذلك على أي حال

\par 4 فقال له ألعازار الكاهن: احلف لي أنك لا تمد يدك إلى شيء مما له، وسأعطيك ثلاثمائة منجم من الذهب

\par 5 وحلف له أنه لا يأخذ من كنز بيت الله شيئًا إذا سلمه ما ذكره

\par 6 فأعطاه ألعازار قضيبًا من ذهب مشغول، كان الجزء العلوي منه قد أُدخل في جدار خزانة الهيكل، وكان يوضع عليه كل سنة ستائر البيت القديمة، وتُستبدل بها ستائر جديدة

\par 7 وكان وزن السبيكة ثلاثمائة منى من الذهب، وكانت مغطاة بالحجب التي تراكمت على مدى سنوات طويلة، ولم تكن معروفة لأحد سوى أليعازار

\par 8 بعد أن تلقى كراسوس هذا الوعد، نكث وعده، وتراجع عن الاتفاق المبرم مع أليعازار؛ واستولى على جميع كنوز الهيكل، ونهب ما كان فيه من أموال، حتى بلغ مقداره ألفي وزنة

\par 9 لأن هذه الأموال كانت تتراكم منذ بناء الهيكل حتى ذلك الوقت، من غنائم ملوك يهوذا وتقدماتهم، وكذلك من الهدايا التي أرسلها ملوك الأمم؛

\par 10 وكثروا وكثروا على مر السنين، كل ما أخذه

\par 11 ثم انطلق ذلك الحقير كراسوس بالمال وجيشه إلى بلاد الفرس؛ فهزموه هو وجيشه في المعركة، وقتلوهم في يوم واحد:

\par 12 وغنم الجيش الفارسي كل ما كان في معسكر كراسوس

\par 13 بعد هذا الإنجاز، زحفوا إلى بلاد سوريا، التي سيطروا عليها، وتحرروا من خضوعها للرومان

\par 14 ولما علم الرومان بذلك، أرسلوا قائدًا مشهورًا يُدعى كاسيوس على رأس جيش عظيم: والذي، عند وصوله إلى بلاد سوريا، طرد الفرس الذين كانوا فيها

\par 15 ثم توجه إلى المدينة المقدسة، وأنقذ هيركانوس من الحرب التي كان اليهود يخوضونها ضده، وأصلح بين الطرفين

\par 16 بعد ذلك، عبر نهر الفرات، وحارب الفرس، وأعادهم إلى خضوعهم للرومان:

\par 17 كما أخضع الملوك الاثنين والعشرين الذين أخضعهم بومبي؛ وأخضع كل شيء في بلدان الشرق للرومان

\chapter{42}

\par \textit{تاريخ قيصر ملك الرومان}

\par 1 يُقال إنه كانت في روما امرأة حامل، وعندما اقترب موعد ولادتها، كانت تعاني من آلام شديدة أثناء الولادة، فماتت.

\par 2 ولكن بينما كان الطفل يتحرك، انفتح بطن أمه، فولد من هناك وعاش ونما، وسمي يوليوس، لأنه وُلد في الشهر الخامس، وسمي قيصر،

\par 3 لأن بطن أمه، الذي أُخرج منه، كان ممزقًا. (اللاتينية: caesa.)

\par 4 ولكن عندما أرسل شيخ روما بومبي إلى الشرق، أرسل أيضًا قيصر إلى الغرب، لإخضاع بعض الأمم التي ثارت على الرومان

\par 5 فذهب قيصر، وقهرهم، وأخضعهم للرومان، وعاد إلى روما بمجد عظيم

\par 6 وازدادت شهرته، واشتهرت أعماله، واستولت عليه كبرياء مفرط، لذلك طلب من الرومان أن يسموه ملكًا

\par 7 فأجابه الشيخ والولاة: «حقًا، إن آباءنا أقسموا في أيام الملك تاركوين، الذي سلب امرأة رجل آخر، ووضعت يديها على نفسها حتى لا يستمتع بها،

\par 8 —أنهم لن يمنحوا لقب الملك لأي ممن يتولى رئاسة شؤونهم؛ ولهذا السبب (قالوا) لسنا قادرين على إرضائكم في هذا الخصوص

\par 9 لذلك أثار الفتن، وخاض معارك ضارية في روما، فقتل الكثير من الناس، حتى استولى على عرش الرومان، ونصب نفسه ملكًا، ووضع إكليلًا على رأسه

\par 10 ومنذ ذلك الحين، أُطلق عليهم اسم ملوك الرومان، نسبةً إلى مملكتهم: كما أُطلق عليهم أيضًا اسم القياصرة

\par 11 فلما سمع بومبي هذا الخبر عن قيصر، وأنه قتل الحكام الثلاثمائة والعشرين، جمع جيوشه وسار إلى كابادوكيا:

\par 12 وذهب قيصر للقائه، فقاتله، وانتصر عليه، وقتله، واستولى على كامل أراضي الرومان

\par 13 بعد ذلك، ذهب قيصر إلى مقاطعة سوريا؛ حيث التقى به ميثريداتس الأرمني مع جيشه، مؤكدًا له أنه جاء بخطط سلمية، وأنه مستعد لمهاجمة أي أعداء يقودهم

\par 14 أمره قيصر بالرحيل إلى مصر، وسار ميثريداتس حتى وصل إلى عسقلان

\par 15 كان هيركانوس يخشى قيصر بشدة، لأن خضوعه لبومبي، الذي قتله قيصر، كان معروفًا

\par 16 لذلك أرسل أنتيباتر على عجل مع جيش شجاع لمساعدة ميثريداتس، فسار أنتيباتر إليه وساعده ضد إحدى مدن مصر، واستولوا عليها

\par 17 ولكن بينما كانوا يغادرون من هناك، وجدوا جيشًا من اليهود المقيمين في مصر، يقفون عند المدخل، لمنع ميثريداتس من دخول مصر

\par 18 فأخرج لهم أنتيباتر رسالة من هيركانوس يأمرهم فيها بالكف عن معارضة ميثريداتس، صديق قيصر، فامتنعوا.

\par 19 أما الآخرون فقد ساروا حتى وصلوا إلى مدينة الملك الحاكم آنذاك، فخرج إليهم بكل جيوش المصريين، وعندما اشتبكوا معه، غلبهم وهزمهم؛

\par 20 وأدار ميثريداتس ظهره وهرب؛ والذي عندما "حاصرته القوات المصرية، أنقذه أنتيباتر من الموت:

\par 21 ولم يكف أنتيباتر ورجاله عن مقاومة المصريين في المعركة، الذين هزمهم وانتصر عليهم، وفاز بكامل بلاد مصر

\par 22 وكتب ميثريداتس إلى قيصر، يُريه ما فعله أنتيباتر، والمعارك التي خاضها، والجروح التي أصيب بها؛

\par 23 وأن انتصار البلاد لا يُنسب إليه بل إلى أنتيباتر، وأنه أخضع المصريين لطاعة قيصر

\par 24 وعندما قرأ قيصر رسالة ميثريداتس، أثنى على أنتيباتر لمآثره، وقرر ترقيته وتمجيده

\par 25 بعد هذه الأعمال، ذهب ميثريداتس وأنتيباتر إلى قيصر، الذي كان آنذاك في دمشق؛ وحصل من قيصر على ما يشاء، ووعده بكل ما تمنى

\chapter{43}

\par \textit{رواية مجيء أنتيجونوس ابن أريستوبولوس إلى قيصر، يشكو من أنتيباتر الذي تسبب في وفاة والده.}

\par 1 لكن أنتيجونوس ابن أريستوبولوس جاء إلى قيصر، وأخبره بحملة أريستوبولوس والده لمهاجمة بومبي، وكيف كان مطيعًا ومذلولًا له

\par 2 ثم أخبره أن هيركانوس وأنتيباتر أرسلا رجلاً سرًا إلى والده لتدميره بالسم، بهدف مساعدة بومبي ضد أصدقائك

\par 3 فأرسل قيصر إلى أنتيباتر وسأله عن هذا الأمر، فأجابه أنتيباتر:

\par 4 «بالتأكيد لقد أطعت بومبي، لأنه كان الحاكم آنذاك، وكان يمنحني امتيازات؛ لكنني الآن لم أقاتل المصريين من أجل بومبي، الذي مات بالفعل؛

\par 5 ولم أواجه صعوبات في هزيمتهم وإخضاعهم لبومبي؛ لكنني فعلت ذلك بدافع الواجب تجاه قيصر، ولكي أتمكن من إخضاع لينز له

\par 6 ثم كشف أنتيباتر عن رأسه ويديه، وقال: "هذه الجروح التي على رأسي وجسدي تشهد على أن محبتي وطاعتي لقيصر أعظم من محبتي وطاعتي لبومبي؛

\par 7 لأني لم أُعرِّض نفسي في أيام بومبي للأمور التي عرَّضت نفسي لها في أيام الملك قيصر

\par 8 فقال له قيصر: «السلام عليك وعلى جميع أصدقائك يا أشجع اليهود، فقد أظهرت لنا حقًا هذه الشجاعة والكرم والطاعة والمودة».

\par 9 ومنذ ذلك الوقت ازداد قيصر محبة لأنتيباتر، ورفعه فوق جميع أصدقائه، ورقاه إلى أن يكون قائداً لجيوشه، وأخذه معه إلى بلاد الفرس.

\par 10 ورأى من شجاعته ومآثره الناجحة أنه أثار فيه شوقًا ومودةً متزايدين تجاهه:

\par 11 أخيرًا أعاده إلى أرض يهوذا، مُغطىً بالأوسمة ومُتوَّجًا بمنصب ذي سلطة

\par 12 وسار قيصر إلى روما، بعد أن حسم شؤون هيركانوس؛ الذي بنى أسوار المدينة المقدسة، وتصرف مع الناس بأسلوب رائع للغاية:

\par 13 لأنه كان رجلاً صالحًا، موهوبًا بالفضائل، وحياة لا تشوبها شائبة، لكن عجزه في الحروب كان معروفًا لدى جميع الرجال


\chapter{44}

\par \textit{سرد سفارة هيركانوس إلى قيصر، طالبًا تجديد المعاهدة بينهما؛ ونسخة المعاهدة التي أرسلها هيركانوس إليه.}

\par 1 لذلك أرسل هيركانوس سفراء إلى قيصر، برسالة تتعلق بتجديد المعاهدة التي كانت بينه وبين الرومان

\par 2 ولما جاء سفراء هيركانوس إلى قيصر، أمرهم بالجلوس أمامه؛ وهو شرف لم يمنحه لأحد من سفراء الملوك الذين اعتادوا أن يأتوا إليه

\par 3 علاوة على ذلك، فقد تعامل معهم بلطف، بتسريع أعمالهم، وأمر بالرد على رسالة هيركانوس؛ الذي كتب له أيضًا المعاهدة، وفيما يلي نسخة منها

\par 4 «من قيصر ملك الملوك إلى أمراء الرومانيين الذين في صور وصيدا، السلام عليكم

\par 5 أعلمكم أنه قد أُحضرت إليّ رسالة من هيركانوس بن الإسكندر، وكلاهما ملكا اليهود؛

\par 6 وقد فرحتُ بقدومه، وذلك بفضل حسن النية المستمر الذي يُعلنه هو وشعبه تجاهي وتجاه الأمة الرومانية

\par 7 ولقد أثبتُ صحة كلامه بهذا؛ أنه أرسل سابقًا أنتيباتر، قائد اليهود، وفرسانهم، مع ميثريداتس صديقي، الذي هاجمته جيوش مصر؛

\par 8 وأنقذ ميثريداتس من الموت، بعد أن فاز لنا بمصر، وأخضع المصريين للرومان. كما سار معي إلى بلاد الفرس، متطوعًا

\par 9 ولذلك آمُر جميع سكان ساحل البحر، من غزة إلى صيدا، أن يدفعوا جميع الجزية التي يدينون بها لنا، كل عام، لبيت الإله العظيم الذي في أورشليم؛

\par 10 ما عدا أهل صيدا، فيدفعون لها حسب فرض الجزية عشرين ألفاً وخمسمائة وخمسين قنطاراً من القمح كل سنة.

\par 11 وأمرت أيضا أن تأخذ لاودكية وممتلكاتها وكل ما كان في أيدي ملوك يهوذا حتى شاطئ الفرات.

\par 12 مع كل تلك الأماكن التي غزاها الأسمونيون بعبور الأردن، تُعاد إلى هيركانوس بن الإسكندر ملك يهوذا.

\par 13 على كل هذه الأشياء، ظفر بها آباؤه بسيفهم، لكن بومبي انتزعها ظلماً في عهد أريستوبولوس:

\par 14 ومن الآن فصاعدًا، فلتكن هذه الأراضي تابعة لهيركانوس، ولملوك يهوذا الذين يخلفونه

\par 15 وهذه المعاهدة لي، ولكل ملك من ملوك روما خلفائي: من يخالفها أو ينقض أي جزء منها، فليُهلكه الله بالسيف، وليُخرب بيته وحكومته ويُقطع!

\par 16 ومتى قرأتم رسالتي هذه، فاكتبوها بأحرف منقوشة على ألواح من نحاس، بلغة الرومان وحروفهم، ولغة اليونانيين وحروفهم

\par 17 ووضع الألواح في أماكن ظاهرة من معابد صور وصيدا، حتى يتمكن كل شخص من رؤيتها، ويفهم ما عينته لـ "هيركانوس واليهود".

\chapter{45}

\par \textit{تاريخ وفاة قيصر}

\par 1 وكان مع قيصر اثنان من أصدقاء بومبي، أحدهما يُدعى كاسيوس، والآخر بروتوس، اللذان خططا لقتل قيصر.

\par 2 لأي غرض اختبأوا في الهيكل في روما الذي خصصه لنفسه ليصلي فيه

\par 3 لذلك، عندما جاء، غير مبالٍ، وآمن، وغير منتبه لنفسه، انقضوا عليه وقتلوه

\par 4 واستولى كاسيوس على العرش، وجمع جيشًا كبيرًا، ونقله إلى ما وراء البحر، خوفًا من حزب قيصر إذا استمر في الإقامة في روما

\par 5 وسار إلى أرض آسيا ودمرها، ومن هناك ذهب إلى بلاد يهوذا

\par 6 وأراد أنتيباتر مهاجمته؛ ولكن عندما رأى أن قوته لم تكن كافية للمهمة، تصالح معه

\par 7 ووضع كاسيوس جزية قدرها سبعمائة وزنة من الذهب على أرض يهوذا، وضمن أنتيباتر نفسه كضمان للمال؛

\par 8 وأمر ابنه هيرودس برفعها على بلاد يهوذا، وأن يحملها إلى كاسيوس. الذي لما استلمها سار إلى بلاد مقدونيا، ومكث هناك خوفًا من الرومان

\chapter{46}

\par \textit{قصة موت أنتيباتر}

\par 1 وكان أمراء يهوذا قد تشاوروا على قتل أنتيباتر، ولهذا الغرض أرسلوا عليه سرًا رجلاً يُدعى ملكيا

\par 2 فحاول ملكيا، ولكن تنفيذه تأخر طويلاً.

\par 3 فبلغ الخبر أنتيباتر، فطلب ملكيا ليقتله.

\par 4 ولكن ملكيا برأ نفسه أمام أنتيباتر من الأمور التي اتهم بها، وحلف له أن التقرير لا أساس له، فصدقه أنتيباتر، ونحى عنه كل شك.

\par 5 ولكن ملكياه، بعد أن أعطى مبلغًا كبيرًا من المال لساقي هيركانوس، وافق معه على إعطاء أنتيباتر السم، بينما كان على سرير الوليمة في حضور الملك.

\par 6 ففعل ساقي الخمر هذا، فمات الملك أنتيباتر في ذلك اليوم نفسه: ولم يكن الأمر بتدبير الملك ولا بعلمه. وعندما مات أنتيباتر، حلّ هيركانوس محله ملكيا

\chapter{47}

\par \textit{تاريخ وفاة ملكيا}

\par 1 ولما علم هيرودس بن أنتيباتر أن ملكيا هو الذي تسبب في موت أبيه، فكر في الهجوم على ملكيا علانية. ولكن أخاه منعه من ذلك، مشيراً إلى أنه ينبغي أن يُقتل بحيلة.

\par 2 فذهب هيرودس إلى كاسيوس، وأخبره بما فعله ملكيا. فأجابه الآخر: عندما أذهب إلى صور، ويكون هيركانوس معي، ومعه ملكيا، فانقضوا عليه واقتلوه

\par 3 فلما ذهب كاسيوس إلى صور، وذهب هيركانوس للانضمام إليه، آخذًا معه ملكيا؛ وكانا واقفين معًا في حضور كاسيوس، في وليمة دعاهما إليها كاسيوس مع جميع أصدقائه:

\par 4 (الآن، كان كاسيوس قد أصدر أوامره لعبيده أن يفعلوا كل ما يأمرهم به هيرودس)

\par 5 وكان هيرودس أيضًا واقفًا مع أخيه بين رفاق هيركانوس، واتفق هيرودس مع بعض الخدم على قتل ملكيا، عندما كانت الإشارة تُعطى بطرفة عين

\par 6 وبعد أن أكل هيركانوس وشرب مع أصدقائه ذهبوا إلى النوم في فترة ما بعد الظهر.

\par 7 وعندما استيقظوا من النوم، أمر هيركانوس بإعداد سرير له في الهواء الطلق، أمام مدخل قاعة المأدبة التي ناموا فيها:

\par 8 فجلس هو وأمر ملكيا أن يجلس معه، وأمر هيرودس وأخاه أن يجلسا

\par 9 ووقف عبيد كاسيوس بالقرب من هيركانوس، الذي غمز له هيرودس بملكيا، فاندفعوا إليه في الحال وقتلوه:

\par 10 فخاف هيركانوس بشدة، وسقط في نوبة إغماء

\par 11 ولكن عندما انصرف خدم كاسيوس، ونُقل ملكيا المقتول، عاد هيركانوس إلى رشده مرة أخرى، وسأل هيرودس عن سبب موت ملكيا

\par 12 فأجاب هيرودس: "أنا جاهل تمامًا، ولا أعرف سبب الأمر". والتزم هيركانوس الصمت، ولم يسأل مرة أخرى عن الأمر

\par 13 وسار كاسيوس إلى مقدونيا لملاقاة أوكتافيان ابن شقيق قيصر، وأنطوني قائد جيشه: لأنهما انطلقا من روما بجيش عظيم بحثًا عن كاسيوس

\chapter{48}

\par \textit{تاريخ أوكتافيان (وهو نفسه أغسطس ابن شقيق قيصر)، وأنطوني، قائد جيشه، ووفاة كاسيوس.}

\par 1 عندما سار أوكتافيان إلى مقدونيا، خرج كاسيوس للقائه، واشتبك معه؛ فهرب كاسيوس؛

\par 2 الذي طارده أوكتافيان، وهزمه تمامًا وقتله: وفاز أوكتافيان بالمملكة بدلاً من عمه قيصر؛ ولقب أيضًا قيصر، على اسم عمه.

\par 3 عندما علم هيركانوس بوفاة كاسيوس، أرسل سفراءً يحملون الهدايا والمال والمجوهرات إلى أغسطس وأنطوني:

\par 4 وكتب إليه يطلب تجديد المعاهدة التي أبرمت مع قيصر؛

\par 5 وأنه سيأمر بإطلاق سراح جميع أسرى يهوذا الذين كانوا في مملكته، وأولئك الذين سُبوا في أيام كاسيوس؛

\par 6 وأنه سيسمح لجميع اليهود الذين كانوا في بلاد اليونانيين، وفي أرض آسيا، بالعودة إلى بلاد يهوذا،

\par 7 دون الحاجة إلى أي فدية، أو فداء، أو وضع أي عقبة في الطريق من قبل أي شخص

\par 8 لذلك عندما جاء سفراء هيركانوس إلى أغسطس، برسائلهم وهداياهم، أكرم السفراء،

\par 9 وقبل الهدايا، ووافق على جميع الأشياء التي طلبها هيركانوس، وكتب إليه رسالة، وهذه هي نسختها

\par 10 «من أغسطس، ملك الملوك، وأنطوني زميله، إلى هيركانوس ملك يهوذا؛ لك الصحة

\par 11 لقد وصلتنا رسالتكم الآن، وقد فرحنا بها؛ وأرسلنا ما رغبتم فيه، بخصوص تجديد المعاهدة، والكتابة، إلى جميع مقاطعاتنا، الممتدة من بلاد الهند حتى المحيط الغربي

\par 12 لكن ما أخرنا عن الكتابة إليكم عاجلاً بشأن تجديد المعاهدة هو انشغالنا بإخضاع كاسيوس، ذلك الطاغية القذر؛

\par 13 الذي، إذ تصرف بسوء تجاه قيصر،

\par 14 لذلك قاومناه بكل قوتنا، حتى انتصر علينا الإله العظيم الصالح، وجعله يقع في أيدينا؛

\par 15 الذي قتلناه. وقتلنا أيضًا بروتوس زميله؛ وحررنا بلاد آسيا من يده، بعد أن دمرها وأباد سكانها

\par 16 ولم يلتزم بأي التزام؛ ولم يكرم أي معبد؛ ولم ينصف المظلوم؛ ولم يشفق على يهودي أو أي من رعايانا:

\par 17 لكنه مع أتباعه ارتكب شرورًا كثيرة لجميع الناس من خلال الظلم والاستبداد:

\par 18 لذلك حوّل الله شرهم على رؤوسهم، وسلمهم مع الذين كانوا حلفاءهم

\par 19 افرحوا الآن أيها الملك هيركانوس، وسائر اليهود، وسكان المنطقة المقدسة، والكهنة الذين في هيكل أورشليم:

\par 20 ودعهم يقبلون الهدية التي أرسلناها إلى أعظم معبد، ويصلون من أجل أغسطس إلى الأبد

\par 21 وكتبنا أيضًا إلى جميع مقاطعاتنا، أنه لا يبقى في أي منها أحد من اليهود، سواء كان عبدًا أو أمة، بل يُطلق سراح الجميع بلا ثمن ولا فدية

\par 22 "وأن لا يمنعهم أحد من العودة إلى أرض يهوذا، وذلك بأمر أغسطس، وكذلك أنطونيوس زميله."

\par 23 علاوة على ذلك، كتب إلى أصدقائه الذين في صور وصيدا، وفي أماكن أخرى، ليعيدوا ما أخذوه من أرض يهوذا في أيام ذلك الكاسي القذر:

\par 24 وأن يعاملوا اليهود بسلام، ولا يعارضوهم في أي شيء، وأن يفعلوا لهم كل ما أمر به قيصر في معاهدته معهم

\par 25 وأقام أنطونيوس في بلاد سوريا، فجاءت إليه كليوباترا ملكة مصر، فاتخذها زوجة له

\par 26 كانت امرأة حكيمة، ماهرة في الفنون السحرية وخواص الأشياء: لدرجة أنها أغوته، واستحوذت على قلبه لدرجة أنه لم يستطع أن ينكر عليها شيئًا

\par 27 في هذا الوقت نفسه، ذهب مئة رجل من زعماء اليهود إلى أنطونيوس، واشتكوا من هيرودس وأخيه فاسيلوس ابني أنتيباتر، قائلين:

\par 28 لقد حصلوا الآن على كل ما يخص هيركانوس، ولم يبقَ له شيء من المملكة سوى الاسم؛ وإخفاء هذا الأمر دليل على أسر سيدهم

\par 29 ولكن عندما سأل أنطونيوس هيركانوس عن حقيقة الأمور التي ذكروها له، أعلن هيركانوس أنهم كذبوا، مبرئًا هيرودس وشقيقه مما اتهموهما به

\par 30 ففرح أنطونيوس بهذا، لأنه كان يميل إليهم كثيرًا، ويحبهم.

\par 31 علاوة على ذلك، اشتكى إليه أشخاص آخرون في وقت آخر من هيرودس وأخيه، عندما كان في صور:

\par 32 لكنه لم يرفض فقط الاستماع إلى كلماتهم، بل قتل بعضهم، وألقى الباقين في السجن؛

\par 33 ورفع شأن هيرودس وأخيه، وقدم لهما خدمات، وأعادهما إلى أورشليم بإكرام عظيم. أما أنطونيوس نفسه، فذهب إلى بلاد الفرس، وهزمهم، وأخضعهم، وعاد إلى روما

\chapter{49}

\par \textit{تاريخ أنتيجونوس ابن أريستوبولوس، وتحالفه ضد عمه هيركانوس: والمساعدات التي حصل عليها من ملك الفرس.}

\par 1 عندما وصل أغسطس وأنطوني إلى روما، ذهب أنتيجونوس إلى ملك الفرس، ووعده بألف وزنة من الذهب المسكوك، وثمانمائة عذراء من بنات يهوذا وأمرائها، جميلات وحكيمات؛

\par 2 إذا أرسل معه قائدًا يقود جيشًا عظيمًا ضد أورشليم، وأمره بجعله ملكًا على يهوذا، وأسر عمه هيركانوس، وقتل هيرودس وشقيقه

\par 3 فوافق الملك، وأرسل معه قائدًا على رأس جيش عظيم:

\par 4 وساروا حتى وصلوا إلى أرض سوريا، وقتلوا صديقًا لأنطوني وبعض الرومان الذين كانوا يقيمون هناك

\par 5 ومن هناك ساروا نحو المدينة المقدسة، مدّعين الأمن والسلام، وأن أنتيجونوس لم يأتِ إلا للصلاة في الحرم، ثم سيعود إلى أصدقائه

\par 6 ودخلوا المدينة، ولما دخلوها فسدوا، وبدأوا يقتلون الرجال، وينهبون المدينة، حسب أوامر ملك فارس لهم

\par 7 فركض هيرودس ورجاله للدفاع عن قصر هيركانوس، لكنه أرسل أخاه وأمره بحراسة الطريق المؤدي من الأسوار إلى القصر

\par 8 وبعد أن استولى على كل موقع، اختار بعض رجاله، وسار ضد الفرس الذين كانوا في المدينة؛

\par 9 وتبعه أخوه مع عدد من رجاله، فقتلوا معظم الفرس الذين كانوا في المدينة، أما الباقون فهربوا من المدينة

\par 10 ولما رأى قائد الفرس أن الأمور لم تسر على ما يرام، أرسل رسلاً إلى هيرودس وأخيه ليتفاوضا على الصلح؛

\par 11 وأبلغهم أنه الآن مقتنع بشجاعتهم وبسالتهم، وأنه ينبغي تفضيلهم على أنتيجونوس؛ وأنه لهذا السبب سيقنع قواته بمساعدة هيركانوس وهم بدلاً من أنتيجونوس:

\par 12 وقد أكد هذه رغبته بأشد الأيمان، حتى أن هيركانوس وفاسيلوس صدقاه، ولكن هيرودس لم يصدقه

\par 13 فخرج هيركانوس وفسيلوس إلى قائد الفرس وأظهرا له اعتمادهما عليه، فأشار عليهما أن يذهبا إلى زميله الذي في دمشق، فذهبا.

\par 14 ولما وصلوا إليه، استقبلهم بإكرام، وأظهر لهم تقديره، وعاملهم بلطف؛ مع أنه كان قد أمر سرًا بأسرهم

\par 15 فجاء إليهم بعض كبار رجال البلاد، وأخبروهم بهذا المخطط بالذات، ونصحوهم بالفرار، مع وعد بمساعدتهم على الفرار

\par 16 لكنهم لم يثقوا بهؤلاء الرجال، خوفًا من أن تكون هناك مؤامرة ضدهم؛ لذلك بقوا

\par 17 ولما حل الليل، أُلقي القبض عليهم: فقام فاسيلوس بانتحار نفسه؛ أما هيركانوس فقد قُيّد بالسلاسل، وبأمر من قائد الفرس، قُطعت أذنه حتى لا يعود رئيسًا للكهنة مرة أخرى؛

\par 18 وأرسله إلى هرق، إلى ملك الفرس، فلما وصل أمر الملك بقطع سلاسله، وأظهر له لطفًا؛

\par 19 وبقي في هرقلة مملوءًا بالأوسمة، حتى طلبه هيرودس من ملك الفرس. ولما أُعيد إلى هيرودس، أصابته ما أصابه

\par 20 بعد ذلك، صعد القائد مع أنتيجونوس إلى المدينة المقدسة، وأُخبر هيرودس بما حدث لهيركانوس وفاسيلوس:

\par 21 فأخذ أمه قبرس، وزوجته مريم ابنة أرسطوبولس، وأمها ألكسندرا، وأرسلهم مع الخيول والأمتعة الكثيرة إلى يوسف أخيه ليركبوا سارة.

\par 22 لكنه سار ببطء مع جيش قوامه ألف رجل، وانتظر أولئك الفرس الذين قد يحاولون مطاردته

\par 23 فطارده قائد الفرس مع جيشه، فهاجمهم هيرودس، فغلبهم، وهزمهم

\par 24 بعد ذلك، طاردته قوات أنتيجونوس أيضًا، وقاتلته بشراسة شديدة: فضربهم وقتل منهم أعدادًا كبيرة

\par 25 ثم سار إلى جبال سارة، فوجد أخاه يوسيفوس، فأمره بتأمين العائلات في مكان آمن، وتوفير كل ما يلزمهم:

\par 26 وأعطاهم مالًا كثيرًا، حتى إذا احتاجوا، اشتروا لأنفسهم مؤنًا

\par 27 وبعد أن ترك رجاله مع أخيه يوسيفوس، ذهب هو وبعض رفاقه إلى مصر، ليتمكن من ركوب السفينة والتوجه إلى بلاد الرومان

\par 28 استقبلته كليوباترا بلطف، وطلبت منه أن يتولى قيادة جيوشها وإدارة جميع شؤونها؛ فأبلغه أنه من الضروري جدًا أن يذهب إلى روما

\par 29 وأعطته مالًا وسفنًا، فذهب حتى وصل إلى روما، وأقام عند أنطونيوس، وأخبره بما فعله أنتيجونوس، وما ارتكبه ضد هيركانوس وأخيه، بمساعدة ملك الفرس

\par 30 وركب أنطونيوس معه إلى أغسطس وإلى مجلس الشيوخ، وأخبرهما بالشيء نفسه

\chapter{50}

\par \textit{تاريخ هيرودس عندما عيّنه الرومان ملكًا على اليهود، وخروجه من روما مع جيش لمحاربة البيت المقدس.}

\par 1 أُبلغ أوغسطس ومجلس الشيوخ بما فعله أنتيجونوس، فعيّنا هيرودس ملكًا على اليهود بموافقة واحدة؛

\par 2 وأمره أن يضع على رأسه إكليلًا من ذهب، وأن يمتطي جوادًا، وأن يُنادى بالأبواق التي تسبقه: «هيرودس ملك على اليهود والمدينة المقدسة أورشليم»، وقد تم ذلك

\par 3 وعاد إلى أغسطس، وركب هو وأغسطس وأنطوني. وذهبوا إلى منزل أنطونيوس، الذي دعا مجلس الشيوخ وجميع مواطني روما إلى وليمة أعدها

\par 4 وكانوا يأكلون ويشربون ويفرحون بهيرودس فرحًا عظيمًا، ويصنعون معه ميثاقًا منقوشًا على ألواح من نحاس، ووضعوه في الهياكل

\par 5 وكتبوا ذلك اليوم على أنه الأول من عهد هيرودس، ومن ذلك الوقت أصبح صفرًا تُحسب به الأوقات

\par 6 بعد هذه الأمور، انطلق أنطونيوس وهيرودس عن طريق البحر بجيش عظيم وفير. ولما وصلا إلى أنطاكية، قسما قواتهما:

\par 7 "فأخذ أنطونيوس قسمًا وقاده إلى بلاد الفرس التي هي "هرق" والأجزاء المجاورة لها، وأما هيرودس، فأخذ قسمًا آخر وذهب مباشرة حتى وصل إلى بطليموس."

\par 8 لذلك، عندما سمع أنتيجونوس أن أنطونيوس قد قام بحملة إلى بلاد الفرس، وأن هيرودس قد وصل إلى بطليموس، خرج من البيت المقدس إلى جبل سارة، ليأخذ يوسيفوس، شقيق هيرودس، والذين كانوا معه

\par 9 الذين هاجمهم وحاصرهم، وقطع قناة، واعترض الماء الذي كان يتدفق إليهم، حتى ساد العطش بينهم، وتدهورت أمورهم إلى ضائقة شديدة

\par 10 لذلك قرر يوسيفوس الفرار؛ وتداولت العائلات مسألة تسليم أنفسهم لأنتيغونوس إذا هرب يوسيفوس

\par 11 فأرسل الله عليهم مطرًا غزيرًا، فملأ جميع آبارهم وأوانيهم، فتشجعت قلوبهم، وتحسنت حالتهم

\par 12 واستمر يوسيفوس في صد أنطونيوس ورجاله عن الحصن الحصين، ولم يتمكن الأخير من اكتساب أي ميزة عليه

\par 13 أما هيرودس فسار مباشرة إلى الجبل مع سارة، ليرد أخاه وعائلاته والرجال الذين معه إلى أورشليم

\par 14 ووجد أنتيغونوس يحاصر أخاه، فشن عليهم هجومًا مفاجئًا، فخرج يوسيفوس ورجاله إليهم، فدُمر الجزء الأكبر من جيش أنتيغونوس، فهرب إلى أورشليم

\par 15 الذي طارده هيرودس مع جيش عظيم من اليهود، الذين جاؤوا إليه من كل حدب وصوب، عندما وجدوا أنه قد عاد، وكان مزودًا جيدًا بالمساعدات، حتى أنه كان أقل حاجة إلى جيش الرومان

\par 16 ولما وصل هيرودس إلى المدينة المقدسة، أغلق أنطيغونوس الأبواب في وجهه، وحاربه، وأرسل أموالاً كثيرة إلى رؤساء جيش الرومان، يطلب منهم عدم مساعدة هيرودس، ففعلوا ذلك من أجله.

\par 17 لذلك، استمرت الحرب لفترة طويلة بين أنتيجونوس وهيرودس، ولم ينتصر أي منهما على زميله [أي خصمه].

\chapter{51}

\par \textit{تاريخ كرم بعض رجال هيرودس وشجاعتهم.}

\par 1 تكاثر اللصوص، وأولئك الذين يتوقون إلى ممتلكات الآخرين، في زمن أنتيجونوس؛

\par 2 يتجهون إلى بعض الكهوف في الجبال، والتي لم يكن هناك سبيل للوصول إليها إلا لرجل واحد في كل مرة، عبر أماكن معينة أعدوها لهذا الغرض، ومعروفة لهم وحدهم:

\par 3 وحتى لو كان الآخرون يعرفونهم، لم يتمكنوا من الصعود إلى الكهف؛ لأنه كان هناك رجل على وشك الصعود، والذي، مع القليل من العناء، يمكنه بسهولة صد الشخص الذي كان يتسلق

\par 4 والآن، كان بعض هؤلاء الرجال قد حصلوا على وفرة من الأسلحة والمؤن والمشروبات في ذلك الكهف، وكل ما يحتاجون إليه؛

\par 5 مع كل الغنائم التي حصلوا عليها من خلال مهاجمة أولئك الذين التقوا بهم، وما أخذوه صوابًا أو خطأً.

\par 6 فلما علم هيرودس بأمرهم، ووجد أن أمورهم من المحتمل أن تسبب تأخيرًا؛ وأن الرجال لا يستطيعون في الوقت الحالي الصعود إليهم بالسلالم، ولا في الواقع الصعود بأي شكل من الأشكال:

\par 7 استخدم صناديق خشبية كبيرة مُركبة ومُتصلة ببعضها، وملأها برجال (وأضاف إليها الطعام والماء)، يحملون رماحًا طويلة جدًا معقوفة:

\par 8 وأمر بإنزال تلك الصناديق من قمم الجبال، التي تقع الكهوف في وسطها، حتى وُضعت مقابل أفواهها:

\par 9 وعندما كانوا في مواجهة هؤلاء، طلب من رجاله أن يهاجموهم في قتال متلاحم بالسيوف، وأن يسحبوهم من مسافة بعيدة بتلك الرماح

\par 10 وصُنعت الصناديق وامتلأت بالرجال.

\par 11 ولما نزل بعضهم، وكانوا مقابل أفواه تلك الكهوف، ولم يعطوا أي خبر لساكنيها، اندفع أحد الرجال الذين كانوا في الصناديق إلى الكهوف، وتبعه رفاقه؛

\par 12 فقتلوا اللصوص الذين كانوا فيها مع أتباعهم، وألقوهم في الوديان بالأسفل، جميع الرجال الذين أرسلهم هيرودس، وهم على شاكلتهم

\par 13 وفي هذا العمل المأجور، كانت شجاعتهم وبسالتهم وجرأتهم واضحة جدًا، لدرجة أنه لم يُرَ مثلها قط: واجتثوا اللصوص تمامًا من جميع تلك الأنحاء

\chapter{52}

\par \textit{سرد لعودة أنطونيوس من بلاد الفرس بعد قتل ملك الفرس، ولقائه مع هيرودس.}

\par 1 ثم بعد أن ترك أنطونيوس هيرودس، سار من أنطاكية إلى بلاد الفرس، وحارب ملك الفرس، وتغلب عليه، وقتله، واستعاد أرضه؛

\par 2 وبعد أن أخضع الفرس للرومان، اتجه إلى نهر الفرات

\par 3 ولما أُخبر هيرودس بخبره، خرج ليهنئه معه بانتصاره، ويطلب منه أن يأتي معه إلى البلاد المقدسة

\par 4 ووجد جمعًا كبيرًا جدًا متجمعًا، راغبين في الاقتراب من أنطونيوس؛ وقد اعترضت عليه جماعات كثيرة من العرب، ومنعته من القدوم إلى حضرة أنطونيوس

\par 5 وزحف هيرودس على العرب وقتلهم، فاتحًا ممرًا لكل من أراد الاقتراب من أنطونيوس

\par 6 فأخبر أنطونيوس بذلك قبل وصول هيرودس، فأرسل إليه إكليلاً من ذهب، وعدداً كبيراً من الخيول.

\par 7 "ولكن لما جاء هيرودس، استقبله أنطونيوس بكل لطف، وأثنى عليه على مغامراته ضد العرب، وألحق به سوسيوس قائد جيشه مع قوة كبيرة، وأمره أن يذهب معه إلى مدينة البيت المقدس.

\par 8 وأعطاه رسائل إلى كل كورة سورية التي من دمشق إلى الفرات، ومن الفرات إلى كورة أرمينيا.

\par 9 "وقال لهم: "إن أغسطس ملك الملوك، وأنطونيوس زميله، ومجلس الشيوخ الروماني، قد عيّنوا هيرودس ملكًا على اليهود، وهم يطلبون منكم أن تقودوا جميع رجال حربكم مع هيرودس لمساعدته: إذا كنتم تتصرفون خلافًا لهذا، فيجب عليكم أن تذهبوا إلى الحرب معنا".

\par 10 ثم سار أنطونيوس إلى ساحل البحر، ومن ثم إلى مصر، أما هيرودس وسوسيوس مع جيشه فقد قادوا قوات سوريا

\par 11 ولكن لما اقترب هيرودس من دمشق، وجد أن أخاه يوسيفوس قد خرج من البيت المقدس مع جيش من الرومان لمحاصرة أريحا وقطع قمحها:

\par 12 الذي خرج ضده بابوس قائد قوات أنتيجونوس، وقتل منهم ثلاثين ألفًا، بعد أن قتل أيضًا يوسيفوس شقيق هيرودس:

\par 13 وعندما قُدِّم رأسه إلى أنتيجونوس، اشتراه أخوه فيروراس بخمسمائة وزنة، ودفنه في قبر آبائه:

\par 14 وسمع أيضًا أن أنتيجونوس وبابوس يتقدمان ضده بجيش كبير

\par 15 وبعد أن تأكد هيرودس تمامًا، قرر الهجوم على أنتيجونوس وسحقه على نحو غير متوقع:

\par 16 واتفق مع سوسيوس على أن يأخذ اثني عشر ألف روماني وعشرين ألف يهودي، ويسير ضد أنتيجونوس، على أن يتبع الآخر خطواته ببطء مع بقية الجيش

\par 17 فسار هيرودس بجيشه في جماعة، والتقى بأنتيغونوس في جبال الجليل، وقاتلوه من الظهر حتى الليل

\par 18 ثم تفرق الجيش، وبات هيرودس مع بعض رجاله في بيت، فسقط البيت عليهم، فنجوا جميعًا من الخراب بحياتهم، ولم يُكسر عظم لأحد منهم

\par 19 بعد ذلك بوقت قصير، سارع هيرودس للقتال مع أنتيجونوس، ودارت بينهما معركة شديدة، فهرب أنتيجونوس إلى البيت المقدس؛ بينما قاوم بابوس بشجاعة، واستمر في القتال، لأنه كان يتمتع بروح معنوية عالية وشجاع للغاية

\par 20 وقُتل معظم جيش أنتيجونوس في ذلك اليوم، وقُتل بابوس أيضًا، وقطع فيروراس رأسه، وحملوه إلى هيرودس، الذي أمر بدفنه

\par 21 لذلك، عندما لم يبق أحد من جيش أنتيجونوس، باستثناء السجناء أو الهاربين، أمر هيرودس رجاله بالراحة وتناول الطعام والشراب

\par 22 لكنه ذهب بنفسه إلى حمام معين كان في المدينة المجاورة، ودخل الحمام دون سلاح

\par 23 كان هناك ثلاثة رجال أقوياء وشجعان مختبئين في الحمام، يحملون في أيديهم سيوفًا مسلولة. وعندما رأوه يدخل الحمام، وهو أعزل، أسرعوا للخروج واحدًا تلو الآخر، خائفين منه؛ وهكذا هرب

\par 24 بعد ذلك جاء سوسيوس، وساروا معًا إلى مدينة البيت المقدس، التي حاصروها بخندق، ودارت معارك ضارية بينهم وبين أنتيجونوس:

\par 25 وقُتل عدد كبير من رجال سوسيوس، وكان أنتيجونوس يتغلب عليهم مرارًا وتكرارًا؛ لكنه لم يستطع إجبارهم على الفرار، بسبب صلابتهم وقدرتهم على تحمل الهجمات

\par 26 ثم غلب هيرودس على أنتيجونوس، فهرب أنتيجونوس، ودخل المدينة وأغلق الأبواب في وجه هيرودس، وحاصره هيرودس زمانًا طويلاً

\par 27 وفي إحدى الليالي نام حراس الباب، فلما اكتشف ذلك بعض رجال هيرودس، ركض عشرون منهم، وأخذوا سلالم ووضعوها على السور، وصعدوا وقتلوا الحراس

\par 28 فأسرع هيرودس مع رجاله إلى باب المدينة الذي مقابلهم، فاندفع ودخل المدينة

\par 29 فأخذها الرومان، وبدأوا بذبح المواطنين، فانزعج هيرودس وقال لسوسيوس: "إذا كنت ستهلك كل شعبي، فعلى من ستجعلني ملكًا؟"

\par 30 وأمر سوسيوس بإصدار إعلان بإيقاف السيف؛ ولم يُقتل أي شخص بعد الإعلان

\par 31 لكن قواد سوسيوس، متلهفين للنهب، ركضوا لنهب بيت الله. لكن هيرودس كان واقفًا عند الباب، وفي يده سيف مسلول، فمنعهم، وأرسل إلى سوسيوس ليكبح جماح رجاله، ووعدهم بالمال

\par 32 وأمر سوسيوس رجاله بالامتناع عن النهب، فامتنعوا. وبحثوا عن أنتيجونوس ووجدوه، وأُسر أنتيجونوس

\par 33 بعد هذه الأمور، ذهب سوسيوس إلى مصر إلى زميله أنطونيوس، حاملاً معه أنتيجونوس مقيدًا بالسلاسل

\par 34 "ولكن هيرودس أرسل إلى أنطونيوس هدية عظيمة وجميلة يطلب منه أن يقتل أنتيجونوس، فقتله أنطونيوس، وكان ذلك في السنة الثالثة من حكم هيرودس، والتي كانت أيضًا السنة الثالثة من حكم أنتيجونوس."

\chapter{53}

\par \textit{تاريخ هيرودس بعد وفاة أنتيجونوس}

\par 1 عندما تأكد هيرودس من وفاة أنتيجونوس، اعتبر نفسه مطمئنًا إلى أن أحدًا من العائلة المالكة أسمونزان لن يتنافس معه:

\par 2 لذلك، عمل على تعزيز الكرامات، واللطف والترقيات، لأولئك الذين كانوا يميلون إليه جيدًا وأطاعوا إرادته

\par 3 كما بذل جهدًا كبيرًا في تدمير أولئك الأشخاص، مع عائلاتهم، ونهب ماشيتهم وممتلكاتهم، الذين عارضوه، وقدموا العون ضده

\par 4 واضطهد الناس، وسلب أموالهم، ونهب كل من نبذ طاعة اليهود، وقتل من قاوموه، ونهب أموالهم

\par 5 وعقد اتفاقًا مع جميع الذين كانوا يطيعونه أن يدفعوا له فضة

\par 6 ووضع أيضًا حراسًا على أبواب البيت المقدس، ليفتّشوا الخارجين، ويأخذوا ما يجدونه من ذهب أو فضة لدى أي شخص، ويأتيوا به إليه

\par 7 وأمر أيضًا بتفتيش توابيت الموتى، وأمر بمصادرة أي أموال يحاول أي شخص الاستيلاء عليها بحيلة.

\par 8 وجمع من المال ما لم يجمعه أي من ملوك البيت الثاني

\chapter{54}

\par \textit{تاريخ هيركانوس ابن الإسكندر، عم أنتيجونوس، وعودته إلى القدس بناءً على طلب هيرودس، والموت الذي قتله به.}

\par 1 بعد أن أطلقه ملك الفرس، بقي هيركانوس في هيراكين، في حالة محترمة للغاية وشرف عظيم:

\par 2 لذلك خاف هيرودس أن يدفع أي شيء ملك الفرس إلى تعيينه ملكًا وإرساله إلى أرض يهوذا

\par 3 لذلك، رغبةً منه في طمأنة نفسه، دبر مكائد لهذه المهمة؛ وأرسل إلى ملك الفرس هديةً عظيمةً جدًا، ورسالةً؛

\par 4 الذي ذكر فيه استحقاقات هيركانوس وأعماله الطيبة تجاهه؛ وكيف ذهب إلى روما بسبب ما فعله به أنتيجونوس ابن أخيه؛

\par 5 وأنه بعد أن أصبح الآن على العرش، وأصبحت أموره منظمة، فقد رغب في مكافأته بطريقة مناسبة على المنافع التي منحها له

\par 6 فأرسل ملك الفرس رسولاً إلى هيركانوس قائلاً: "إن كنت ترغب في العودة إلى أرض يهوذا، فارجع:

\par 7 لكنني أحذركم من "هيرودس"؛ وأعلمكم بوضوح أنه "لا يسعى إليكم ليقدم لكم أي خير، بل هدفه هو أن يجعل نفسه آمنًا، إذ لا يوجد من يخافه سواكم: لذلك احترسوا منه بشدة، ولا تقعوا في فخ".

\par 8 فجاء إليه يهود بابل أيضًا وقالوا له مثل هذا الكلام. فقالوا له أيضًا:

\par 9 أنت الآن رجل شيخ، ولست أهلاً لشغل منصب رئيس الكهنة، بسبب الوصمة التي سببها لك ابن أخيك

\par 10 لكن هيرودس رجل شرير، وسافك دماء. وهو لا يذكرك إلا لأنه يخافك. وأنت لا تعوزك أي شيء بيننا، وأنت معنا في المكانة التي ينبغي أن تكون فيها

\par 11 وأهلك هناك في خير حال، فابق معنا، ولا تُعين عدوك على نفسك

\par 12 لكن هيركانوس لم يستجيب لأقوالهم، ولم يستمع لنصيحة من نصحه جيدًا

\par 13 وانطلق وسافر حتى وصل إلى المدينة المقدسة، لشدة شوقه إلى بيت الله وعائلته ووطنه

\par 14 ولما اقترب من المدينة، استقبله هيرودس، وأظهر له شرفًا وعظمة، حتى انخدع هيركانوس ووثق به

\par 15 وكان هيرودس في المجمع العام، وأمام أصدقائه، يناديه "يا أبتاه"، ولكنه مع ذلك لم يكف عن تدبير المكائد في قلبه، فقط لكي لا تُنسب إليه

\par 16 ولذلك ذهبت ألكسندرا وابنتها مريم إلى هيركانوس، وأخافته من هيرودس، ونصحته بالعناية بنفسه؛

\par 17 لكنه لم يحضر إليهما أيضًا، على الرغم من أنهم كرروا ذلك عليه مرارًا وتكرارًا، ونصحوه بالفرار إلى أحد ملوك العرب:

\par 18 ومع ذلك، لم ينتبه إلى كل هذه الأمور، حتى دفعته إليها من خلال التحذيرات والإنذارات المتكررة

\par 19 لذلك، كتب إلى ملك العرب، وبعد أن أرسل في طلب رجل (كان هيرودس قد قتل أخاه، وصادر ممتلكاته، وألحق به شرورًا كثيرة)، أخبره أنه يرغب في إخباره بسر معين، وحلف عليه ألا يخبر به أحدًا؛

\par 20 وأعطاه المال والرسالة إلى ملك العرب، وأبلغه بما طلبه في الرسالة

\par 21 فلما تلقى الرسول الرسالة، فكر في أنه سيحصل على منصب رفيع لدى هيرودس، وسيزيل عن نفسه الشر الذي كان يخافه باستمرار من يديه، إذا أبلغ هيرودس الأمر؛

\par 22 وأن هذا سيكون أكثر ربحًا له من حفظ سر هيركانوس: لأنه في الحالة الأخرى لم يكن آمنًا، ومتأكدًا من أن الأمر لن يُخبر هيرودس في وقت أو آخر، وبالتالي سيكون سببًا في هلاكه

\par 23 فحمل الرسالة إلى هيرودس، وكشف له الأمر كله. فقال له: احمل الرسالة كما هي إلى ملك العرب، وأتني بجوابه لأعلمه

\par 24 أخبرني أيضًا عن المكان الذي سيكون فيه الرجال الذين سيرسلهم ملك العرب، حتى يعود هيركانوس معهم

\par 25 فذهب الرسول، وحمل رسالة هيركانوس إلى ملك العرب، الذي فرح وأرسل بعض رجاله؛

\par 26 أمرهم بالذهاب إلى مكان معين بالقرب من المدينة المقدسة، والانتظار هناك حتى يأتي هيركانوس إليهم؛ ثم رعاية هيركانوس حتى يحضروه إلى حضرته

\par 27 وكتب أيضًا إلى هيركانوس ردًا على رسالته، وأرسله مع الرسول

\par 28 فمضى الرجال مع الرسول إلى المكان المعين، وانتظروا هناك. أما الرسول فحمل الرسالة إلى هيرودس، فعرف محتواها، وأخبره أيضًا عن مكان الرجال الذين أرسل هيرودس إليهم أشخاصًا ليأخذوهم

\par 29 بعد ذلك، أرسل في طلب سبعين شيخًا من شيوخ اليهود، وأرسل أيضًا في طلب هيركانوس؛ فلما جاء، قال له: هل من تبادل رسائل بينك وبين ملك العرب؟

\par 30 فقال هيركانوس: لا. ثم قال له: هل أرسلت لتهرب إليه؟ فقال: لا

\par 31 فأمر هيرودس رسوله أن يتقدم، والعربان، والخيل، وأخرج أيضًا جواب رسالته، فقُرئ

\par 32 ثم أمر بقطع رأس هيركانوس، فقطع رأسه، ولم يجرؤ أحد على النطق بكلمة عنه

\par 33 وكان هيركانوس قد أنقذ هيرودس من الموت الذي كان من العدل أن يُحكم عليه به في مجمع القضاء، إذ أمر بتأجيل المجمع إلى الغد، وأرسل هيرودس في تلك الليلة نفسها

\par 34 ومن ثم كان مقدرًا له أن يصبح قاتله، بغض النظر عن خدماته له ولوالده

\par 35 أُعدم هيركانوس وهو في الثمانين من عمره، وحكم أربعين عامًا: ولم يكن هناك أي من ملوك سلالة الأسمونزان ذوي سلوك أكرم، أو أسلوب حياة أشرف

\chapter{55}

\par \textit{تاريخ أرستوبولوس بن هيركانوس}

\par 1 كان أريستوبولس بن هيركانوس يتمتع بجمال الشكل والشخصية الرائعة والفهم، لدرجة أن نظيرًا له لم يكن معروفًا.

\par 2 وكانت أخته مريم أيضًا، زوجة هيرودس، تشبهه في الجمال، وكان هيرودس متعلقًا بها تعلقًا عجيبًا

\par 3 لكن هيرودس كان يكره تعيين أريستوبولس رئيسًا للكهنة مكان أبيه، خشية أن يجعله اليهود، المرتبطين به بسبب محبتهم لأبيه، ملكًا في وقت ما في المستقبل

\par 4 لذلك عيّن أحد الكهنة العاديين، الذي لم يكن من عائلة الأسمونيين، رئيسًا للكهنة

\par 5 مما أثار استياء ألكسندرا، والدة أريستوبولس، فكتبت إلى كليوباترا تطلب رسالة من أنطونيوس إلى هيرودس، يطلب فيها منه عزل الكاهن الذي رقاه، وتعيين ابنها أريستوبولس رئيسًا للكهنة بدلاً منه

\par 6 ووافقت كليوباترا على ذلك؛ وطلبت من أنطونيوس أن يكتب رسالة إلى هيرودس بهذا الشأن، وأن يرسلها مع أحد رؤساء خدمه

\par 7 فكتب أنطونيوس رسالة وأرسلها مع خادمه غيليوس. ولما جاء غيليوس إلى هيرودس، سلمه رسالة أنطونيوس

\par 8 ولكن هيرودس امتنع عن فعل ما كتبه أنطونيوس في الأمر، مؤكداً أنه لم يكن من العادة بين اليهود عزل أي كاهن من منصبه.

\par 9 وحدث أن جيليوس رأى أريستوبولوس، فانبهر بشدة بجمال هيئته وكمال عربته اللذين رآهما

\par 10 لذلك رسم صورة تشبهه، وأرسلها إلى أنطونيوس، وكتب أسفل الصورة ما يلي: أنه لم ينجب أريستوبولوس رجل، بل أن ملاكًا كان يعيش مع ألكسندرا، أنجبه منها

\par 11 لذلك، عندما وصلت الصورة إلى أنطونيوس، انتابته رغبة شديدة في رؤية أريستوبولوس

\par 12 وكتب رسالة إلى هيرودس، يذكره فيها كيف نصبه ملكًا، وكيف عاونه ضد أعدائه، ويروي له لطفه به:

\par 13 مضيفًا طلبًا، أنه سيرسل إليه أريستوبولوس؛ وهدده في هذا الأمر بسبب الكلمات التي أرسلها

\par 14 ولكن عندما وصلت رسالة أنطونيوس إلى هيرودس، رفض إرسال أريستوبولس، لأنه كان يعلم ما ينوي أنطونيوس القيام به؛ ولهذا السبب ازدرى بالقيام بذلك: وعزل على عجل رئيس الكهنة الذي عيّنه، وأقام أريستوبولس مكانه

\par 15 ثم كتب إلى أنطونيوس، يُعلمه أنه قد نفّذ بالفعل ما كتبه إليه سابقًا، بشأن تعيين أريستوبولوس في منصب والده، قبل وصول رسالته الأخيرة:

\par 16 أي عمل كان لديه تأخر في ذلك الوقت، لأنه كان من الضروري مناقشة الأمر مع الكهنة واليهود، بعد فترة من الأيام، لأن الأمر كان غير عادي؛ ولكن بعد أن مر الأمر وفقًا لرغبته، عيّنه على الفور

\par 17 ولكن الآن وقد عُيّن، لم يكن يجوز له أن يخرج من أورشليم، لأنه لم يكن ملكًا، بل كاهنًا تابعًا لخدمة الهيكل

\par 18 وكلما أراد إجباره على الخروج، رفض اليهود، ولم يسمحوا له بذلك، حتى لو قتل معظمهم

\par 19 لذلك عندما وصلت رسالة هيرودس إلى أنطونيوس، كف عن طلب أريستوبولس، فعُيّن أريستوبولس رئيسًا للكهنة

\par 20 ثم جاء عيد المظال، فرأى الرجال المجتمعون أمام بيت الله أرسطوبولس واقفًا عند المذبح لابسًا ثياب الكهنوت، فسمعوه يباركهم:

\par 21 وقد أرضى الرجال كثيرًا، لدرجة أنهم أظهروا عاطفتهم تجاهه بطريقة ملحوظة للغاية

\par 22 ولما علم هيرودس بذلك تمامًا، حزن حزنًا شديدًا، وخشي أنه عندما تزداد قوة حزب أريستوبولوس، سيطالبه بالملك إذا أراد إطالة عمره، لذلك بدأ يخطط لقتله

\par 23 وكان من المعتاد أن يخرج الملوك بعد عيد المظال إلى بعض المساكن الآمنة في أريحا التي بناها الملوك السابقون

\par 24 وكانت هناك حدائق كثيرة متجاورة، فيها برك سمك واسعة وعميقة، أجروا إليها مجاري المياه، وأقاموا فيها مبانٍ جميلة. كما بنوا في أريحا قصوراً جميلة ومبانٍ جميلة.

\par 25 يروي مؤلف الكتاب أن أشجار البلسم كانت تنمو بكثرة في أريحا، ولم تكن موجودة في أي مكان آخر سوى هناك، وأن ملوكًا كثيرين حملوها من هناك إلى بلادهم، لكن لم ينمُ أي منها، إلا تلك التي حملت إلى مصر

\par 26 وأنهم لم يفشلوا في أريحا إلا بعد خراب البيت الثاني، ولكنهم ذبُلوا بعد ذلك، ولم ينبتوا مرة أخرى

\par 27 فخرج هيرودس إلى أريحا يطلب المتعة، وتبعه أريستوبولس

\par 28 ولما وصلوا إلى أريحا، أمر هيرودس بعض عبيده أن ينزلوا إلى برك السمك ويلعبوا كما جرت العادة، وإن نزل أرسطوبولس إليهم، فعليهم أن يلعبوا معه زمانًا ثم يغرقوه

\par 29 وأما هيرودس فكان جالساً في بيت للعشاء أعده لنفسه ليجلس فيه. فأرسل هيرودس واستدعى أرستوبولس وأجلسه بجانبه. وجلس أيضاً رؤساء حاشيته وأصحابه بين يديه.

\par 30 وأمر بإحضار أطعمة وأشربة، فأكلوا وشربوا، ونزل الخدام مسرعين إلى الماء كعادتهم، ولعبوا

\par 31 وأراد أريستوبولس بشدة أن ينزل معهم إلى الماء، وقد غلب عليهم الخمر، فطلب الإذن من هيرودس أن يفعل ذلك، فأجابه:

\par 32 هذا لا يليق بك ولا بأحد مثلك، وعندما استعجل، نبهه ونهاه، ولكن عندما كرر أرسطوبولس طلبه عليه، قال له: افعل ما تشاء

\par 33 ثم قام هيرودس ومضى إلى قصر لينام هناك

\par 34 ونزل أريستوبولس إلى المياه، ولعب طويلاً مع الخدم: الذين، عندما أدركوا أنه الآن متعب ومنهك ويريد الصعود، أمسكوه تحت الماء وقتلوه وحملوه ميتًا

\par 35 وكان هناك ضجيج عظيم من الشعب، وصراخ، ونُصبت مرثية

\par 36 فركض هيرودس وخرج ليرى ما حدث، فلما رأى أرسطوبولس ميتًا، ندبه وبكى عليه بكاءً شديدًا شديدًا

\par 37 ثم أمر بنقله إلى المدينة المقدسة، ورافقه حتى دخلها، وأجبر الناس على حضور جنازته، ولم يكن هناك أي جانب من أسمى درجات التكريم لم يقصر في تقديمه له

\par 38 ومات وهو شاب في السادسة عشرة من عمره، ولم تستمر رئاسته للكهنوت إلا لبضعة أيام

\par 39 ولهذا السبب نشأت عداوة بين والدته ألكسندرا وابنتها مريم زوجة هيرودس، وأم هيرودس وأخته

\par 40 وكانت اللعنات والشتائم التي وجهتها مريم العذراء إليهم معروفة، ومع أن هذه اللعنات والشتائم وصلت إلى هيرودس، إلا أنه لم يمنعها ولم يوبخها، من خلال محبته الكبيرة لها

\par 41 كان يخشى أيضًا أن تتخيل في ذهنها أنه يميل إلى الآخرين: ومن هنا استمرت هذه التصرفات طويلًا بين هؤلاء النساء

\par 42 وبدأت أخت هيرودس، التي كانت موهوبة بأعظم الخبث والمكر، بالتخطيط ضد مريم:

\par 43 لكن مريم كانت متدينة، ومستقيمة، ومتواضعة، وفاضلة: لكنها كانت ممزوجة بقليل من الغطرسة والكبرياء والكراهية تجاه زوجها

\chapter{56}

\par \textit{تاريخ أنطونيوس، وحملته ضد أغسطس، والمساعدة التي طلبها من هيرودس. وسرد للزلزال الذي حدث في أرض يهوذا، والمعركة التي دارت بينهم وبين العرب.}

\par 1 كانت كليوباترا، ملكة مصر، زوجة أنطونيوس: واكتشفت أساليب تزيين وتجميل بنفسها، تستخدمها النساء لإغراء الرجال، لم تكتشفها أي امرأة أخرى في العالم:

\par 2 حتى أنها، على الرغم من تقدمها في السن، بدت كفتاة صغيرة غير متزوجة، بل وأكثر رقة وجمالاً

\par 3 وجد أنطونيوس فيها أيضًا أساليب الجمال، ووسائل خلق المتعة، التي لم يجدها قط في العدد الهائل من النساء اللواتي استمتع بهن. ولذلك، استحوذت على قلب أنطونيوس تمامًا، بحيث لم يتبقَّ فيه مجال للمودة تجاه أي شخص آخر

\par 4 لذلك أقنعته بإزعاج بعض الملوك الخاضعين للرومان، لأسباب خاصة بها؛ وأطاعها في ذلك، فقتل بعض الملوك بناءً على طلبها؛ وترك بعضهم أحياءً بأمرها، جاعلاً إياهم خدماً وعبيداً لها

\par 5 وأُبلغ أغسطس بذلك، فكتب إليه، ينتقد مثل هذا السلوك، ويطلب منه ألا يرتكب مثله مرة أخرى

\par 6 وأخبر أنطونيوس كليوباترا بما كتبه إليه أغسطس؛ ونصحته بالثورة على أغسطس، وأظهرت له أن الأمر سهل للغاية

\par 7 وافق على رأيه، وخدع أغسطس علانيةً؛ وجمع جيشًا وإمدادات، حتى يتمكن من الذهاب بحرًا إلى أنطاكية، ومن ثم يسير برًا لمقابلة أغسطس أينما وجده

\par 8 ثم أرسل في طلب هيرودس ليرافقه. فذهب إليه هيرودس بجيش عظيم ومؤن كاملة

\par 9 وعندما جاء إليه، قال له أنطونيوس: «إن العقل السليم ينصحنا بالقيام بحملة ضد العرب، والاشتباك معهم: لأننا لسنا متأكدين بأي حال من الأحوال من أنهم لن يقوموا بغزو اليهود وأرض مصر، بمجرد أن ندير ظهورنا.»

\par 10 ورحل أنطونيوس عن طريق البحر، لكن هيرودس شن هجوماً على العرب، فأرسلت كليوباترا قائداً يدعى أثينيو على رأس جيش عظيم لمساعدة هيرودس في إخضاع العرب.

\par 11 وأمرته أن يضع هيرودس ورجاله في الصف الأول، وأن يعقد اتفاقًا مع ملك العرب على أن يحاصروا هيرودس ويقطعوا رجاله إربًا

\par 12 قادتها إلى ذلك رغبة في امتلاك كل ما يستحقه هيرودس:

\par 13 كانت ألكسندرا قد طلبت منها سابقًا أن تحث أنطونيوس على قتل هيرودس؛ وهو ما فعلته بالفعل، لكن أنطونيوس رفض ارتكاب هذا الفعل

\par 14 أُضيف إلى ذلك أن كليوباترا كانت تتوق سابقًا إلى هيرودس، وقد رغبت في وقت ما في ممارسة الجنس معه؛ لكنه كبح جماح نفسه، لأنه كان عفيفًا. وكانت هذه هي الأسباب التي دفعتها إلى هذا السلوك

\par 15 فجاء أثينا إلى هيرودس، بناءً على أمر كليوباترا، وأرسل ليعقد اتفاقًا مع ملك العرب ليحاصره

\par 16 ولما التقى هيرودس وعربه، هاجم أثينيو ورجاله هيرودس، الذي اعترضه الجيشان، واشتدت المعركة ضده من أمامه ومن خلفه

\par 17 فلما رأى هيرودس ما حدث، جمع رجاله، وقاتل بشدة حتى أصبحوا خارج متناول الجيشين، بعد جهد عظيم، ثم رجع إلى البيت المقدس

\par 18 وحدث زلزال عظيم في أرض يهوذا لم يحدث مثله منذ زمن الملك حربة، فأهلك فيه عدد كبير من الناس والبهائم

\par 19 فأفزَعَ هذا هيرودس كثيرًا، وسبَّبَ له خوفًا عظيمًا، وانهَكَ روحه. لذلك تشاور مع شيوخ يهوذا في عقد ميثاق مع جميع الأمم المحيطة بهم، بهدف السلام والهدوء، وإزالة الحروب وسفك الدماء

\par 20 كما أرسل سفراء في هذه الأمور إلى الدول المجاورة، فاعتنقت جميعها السلام الذي دعاهم إليه، باستثناء ملك العرب؛

\par 21 الذي أمر بقتل السفراء الذين أرسلهم هيرودس إليه؛ لأنه ظن أن هيرودس فعل هذا لأن رجاله هلكوا في الزلزال، ولذلك إذ ضعف، تحول إلى صنع السلام

\par 22 لذلك عزم على خوض حرب مع هيرودس، وجمع جيشًا كبيرًا وقويًا، وزحف ضده

\par 23 وأُخبر هيرودس بهذا، فانزعج بشدة لسببين: الأول، بسبب مذبحة سفرائه، وهو فعل لم يرتكبه أي من الملوك حتى الآن؛ والثاني، لأنه تجرأ على مهاجمته، متخيلًا في ذهنه ضعفه ونقص جنوده

\par 24 ولكنه أراد أن يظهر له أن الأمر كان على خلاف ذلك: حتى يعرف كل من أرسل إليهم سفراء للسلام أنه لم يفعل هذا من باب الخوف أو الضعف، بل من باب الرغبة في ما هو لطيف وصالح؛ حتى لا يجرؤ أحد على القيام بمحاولات ضد اليهود، أو يتخيل في ذهنه أنهم ضعفاء.

\par 25 علاوة على ذلك، أراد الانتقام من ملك العرب بسبب سفرائه: ولهذا السبب قرر بكل عجل الزحف ضده

\par 26 لذلك جمع جنودًا من أرض يهوذا، وقال لهم: "أنتم على علم بمذبحة سفرائنا التي ارتكبها ذلك العربي؛ وهو عمل لم يرتكبه أي ملك حتى الآن

\par 27 لأنه يعتقد أننا ضعفنا وأصبحنا عاجزين، وقد تجرأ على استفزازنا، ويعتقد أنه سيحقق كل رغباته علينا، ولن يكف عن محاربتنا باستمرار

\par 28 لذلك يجب عليك أن تكافح الصعوبات، لكي تُظهر شجاعتك، وتتمكن من "إخضاع أعداءك، وجني غنائمهم:

\par 29 على الرغم من أن الحظ قد يكون في وقت ما مواتيًا لنا، وفي وقت آخر معاكسًا لنا، وفقًا لعادات وتقلبات هذا العالم المعتادة

\par 30 في الحقيقة، يجب عليك القيام برحلة استكشافية فورًا، للانتقام من هؤلاء الظالمين، وكبح جرأة كل من لا يُقدّرك

\par 31 ولكن إذا قلت أن هذا الزلزال قد أثبط عزيمتنا ودمر أعدادًا كبيرة منا، فأنت تعلم جيدًا أنه لم يدمر أحدًا من المقاتلين، بل بعض الآخرين.

\par 32 ولا ينبغي لنا أن نعتبره أمرًا غير معقول على الإطلاق، أنه دمر أسوأ ما في أمتنا، لكنه ترك الأفضل ليبقى. ومما لا شك فيه أيضًا أن هذا قد حسّن معنوياتكم ومشاعركم الداخلية

\par 33 لكن واجب من أنقذه الله من الهلاك وحفظه من الهلاك، يقتضي منه طاعته وفعل الخير والصواب

\par 34 ولا طاعة أشرف ولا أمجد من إنصاف المظلوم بالظالم، وإخضاع أعداء الله ودينه وأمته، بمعاونة أهل طاعته والتوجه إليه

\par 35 ولا يخفى عليك ما حل بنا مؤخرًا مع هؤلاء العرب، عندما حاصرونا بأثينيو؛ وكيف ساعدنا الإله العظيم الصالح ضدهم، وأنقذنا منهم

\par 36 فاتقوا الله، على عادتكم القديمة، وسنة آبائكم الحميدة، وأعدّوا أنفسكم لهذا العدو قبل أن يستعد لكم، وكونوا معه مسبقًا قبل أن يسبقكم، وسيؤمّن لكم الله عونًا ونصرًا على عدوكم

\par 37 فلما سمع الرجال خطاب هيرودس، أجابوا أنهم مستعدون للقيام بالحملة ولن يتأخروا

\par 38 فشكر الله لهم على ذلك، وأمر بتقديم ذبائح كثيرة، وأمر بجمع جيش، فجمع جمع غفير من سبط يهوذا وبنيامين.

\par 39 وكان هيرودس سائرًا لمحاربة ملك العرب، فقابله؛ فاشتدت المعركة بينهما، فقتل من العرب خمسة آلاف

\par 40 وحدثت معركة أخرى، وقُتل أربعة آلاف من العرب. فعاد العرب إلى معسكرهم، وبقوا هناك. ولم يستطع هيرودس أن يفعل شيئًا ضدهم، لأن المكان كان محصّنًا. لكنه بقي مع جيشه، وحاصرهم في المكان نفسه، ولم يسمح لهم بالخروج

\par 41 ومكثوا على هذه الحال خمسة أيام، فجاءهم عطش شديد، فأرسلوا سفراء إلى هيرودس مع هدية ثمينة، يطلبون هدنة وحرية استسقاء ماء للشرب، لكنه لم يستمع إليهم، بل استمر في نفس العداء الشديد

\p ar{42}
فقال العرب: لنخرج لمحاربة هذه الأمة، لأنه خير لنا أن ننتصر أو نموت من أن نهلك عطشًا

\par 43 فخرجوا للقائهم، فغلبهم جيش هيرودس، وقتل منهم تسعة آلاف، وطارد هيرودس ورجاله العرب وهم هاربون، فقتل منهم أعدادًا كثيرة، وحاصر مدنهم وأخذها

\par 44 لذلك ضحوا بحياتهم، ووعدوا بالطاعة، فوافق، وانصرف عنهم، وعاد إلى البيت المقدس

\par 45 أما العرب المذكورون في هذا الكتاب فهم العرب الذين سكنوا من بلاد سارة إلى الحجاز وما حولها، وكانوا ذوي شهرة عظيمة وأعداد كبيرة


\chapter{57}

\par \textit{تاريخ معركة أنطونيوس مع أغسطس، وموت أنطونيوس، وذهاب هيرودس إلى أغسطس.}

\par 1 عندما خرج أنطونيوس من مصر إلى بلاد الرومان، وواجه أغسطس، دارت بينهما معارك ضارية، انتصر فيها أغسطس، وسقط أنطونيوس في المعركة؛

\par 2 واستولى أغسطس على معسكره وكل ما فيه. وبعد ذلك، توجه إلى رودس، ليأخذ سفينة إلى هناك ليعبر إلى مصر

\par 3 ووصل الخبر إلى هيرودس، وكان قلقًا للغاية بشأن وفاة أنطونيوس، وكان يخاف أغسطس خوفًا شديدًا، فقرر أن يذهب إليه ليُحييه ويهنئه معه

\par 4 لذلك أرسل أمه وأخته مع أخيه إلى حصن منيع كان له في جبل سارة. كما أرسل زوجته مريم وأمها ألكسندرا إلى الإسكندرية، تحت رعاية يوسيفوس الصوري، وحلف له أن يقتل زوجته وأمها بمجرد إبلاغه بوفاته

\par 5 بعد ذلك، ذهب إلى أغسطس ومعه هدية ثمينة للغاية. وكان أغسطس قد عزم بالفعل على قتل هيرودس؛

\par 6 لأنه كان صديقًا وداعمًا لأنطوني، ولأنه كان قد تشاور سابقًا بشأن السير مع أنطوني لمهاجمته

\par 7 فلما أُخبر أوغسطس بوصول هيرودس، أمره أن يحضر إليه مرتديًا ردائه الملكي الذي كان يرتديه، ما عدا الإكليل، لأنه كان قد أمر بخلعه عن رأسه

\par 8 الذي، عندما كان في حضرته، بعد أن وضع تاجه جانبًا كما أمر أغسطس، قال:

\par 9 «أيها الملك، ربما بسبب حبي لأنطوني، غضبت مني بشدة لدرجة أنك نزعت الإكليل عن رأسي؛

\par 10 أم كان ذلك لسبب آخر؟ بما أنه "إذا كنت غاضبًا مني بسبب التزامي بأنطوني، فأنا أقول حقًا إنني التزمت به لأنه استحق مني كل خير، ووضع على رأسي ذلك الإكليل الذي خلعته."

\par 11 وبالفعل، فقد طلب مساعدتي ضدك، وقد قدمتها له؛ "كما قدم لي مساعدته مرات عديدة أيضًا:

\par 12 ولكن لم يكن من نصيبي أن أكون حاضرًا في المعركة التي خاضها معكم، ولم أسحب سيفي ضدكم، ولم أقاتل؛ والسبب في ذلك هو انشغالي بإخضاع العرب

\par 13 لكنني لم أفشل قط في تزويده بالرجال والأسلحة والمؤن، كما تقتضي صداقته وأعماله الطيبة معي. وفي الحقيقة، أنا آسف لأنني تركته؛ خشية أن يظن الناس أنني تخليت عن صديقي عندما كان في حاجة إلى مساعدتي

\par 14 بالتأكيد، لو كنت معه لأعنته بكل قوتي، ولشجعته إذا خاف، ولقويته إذا ضعف، ولرفعته إذا سقط، حتى يحكم الله الأمور بما يشاء

\par 15 وكان هذا ليكون أقل إيلامًا بالنسبة لي من أن يتصور أحد أنني خذلت رجلاً التمس مساعدتي، وبالتالي تقل قيمة صداقتي

\par 16 في رأيي، لقد سقط بالفعل بسبب سياسته السيئة، بالاستسلام لتلك الساحرة كليوباترا؛ التي نصحته بقتلها، وبالتالي إزالة حقدها عنه؛ لكنه لم يوافق

\par 17 ولكن الآن، إذا نزعتَ الإكليل من رأسي، فلن تنزعَ مني فهمي وشجاعتي بالتأكيد؛ ومهما كنتُ، فسأكون صديقًا لأصدقائي وعدوًا لأعدائي

\par 18 أجابه أغسطس: "لقد تغلبنا على أنطونيوس بالفعل بقواتنا؛ لكننا سنسيطر عليك بإغرائك لنا؛ وسنحرص، من خلال خدماتنا الطيبة تجاهك، على مضاعفة محبتك لنا، لأنك جدير بهذا

\par 19 وكما خدع أنطونيوس بنصيحة كليوباترا، فقد تصرف بنفس السبب تجاهنا بجحود؛ ردًا على إحساناتنا شرورًا، وعلى معروفنا تمردًا

\par 20 ولكننا نفرح بالحرب التي خضتها مع العرب الذين هم أعداؤنا، لأن من كان عدوك فهو عدونا أيضًا، ومن أطاعك فهو أطاعنا أيضًا.

\par 21 ثم أمر أغسطس بوضع الإكليل الذهبي على رأس هيرودس، وإضافة عدد من المقاطعات إليه بقدر ما كان لديه بالفعل

\par 22 ورافق هيرودس أغسطس إلى مصر، وسُلِّم إليه كل ما كان أنطونيوس قد أعده لكليوباترا. وغادر أغسطس إلى روما، أما هيرودس فعاد إلى المدينة المقدسة

\chapter{58}

\par \textit{تاريخ جريمة القتل التي ارتكبها هيرودس بحق زوجته مريم.}

\par 1 كان يوسيفوس، زوج أخت هيرودس، قد كشف لمريمنة أن هيرودس أمره بقتلها هي وأمها، حالما يهلك هو نفسه في صعوده إلى أغسطس

\par 2 وكانت بالفعل تكره هيرودس، منذ الوقت الذي قتل فيه أباها وأخاها؛ وقد زاد على ذلك كرهها بشكل كبير عندما أُبلغت بالأوامر التي أصدرها ضدها

\par 3 لذلك، عندما خرج هيرودس من مصر، وجدها غارقة تمامًا في الكراهية تجاهه: فاضطرب بشدة، فحاول أن يصالحها معه بكل الطرق الممكنة

\par 4 ولكن في أحد الأيام، جاءت أخته بعد بعض المشاجرات التي حدثت بينها وبين مريم، وقالت له: بالتأكيد، يوسف زوجي قد انزوى مع مريم

\par 5 لكن هيرودس لم يُعر كلامها اهتمامًا، لأنه كان يعلم كم كانت مريم طاهرة وعفيفة

\par 6 وبعد ذلك ذهب هيرودس لرؤية مريم في الليلة التي تلت ذلك اليوم، وتصرف معها بلطف ومودة، وسرد لها حبه لها، وقال الكثير في هذا الصدد:

\par 7 قالت له: "هل رأيت رجلاً يحب آخر، ثم يأمر بقتله؟ وهل يكون كارهًا إلا إذا أظهر مثل هذه الأدلة؟"

\par 8 ثم أدرك هيرودس أن يوسيفوس قد كشف لمريمنة السر الذي عهد به إليه؛ واعتقد أنه لم يكن ليفعل ذلك، لولا أنها سلمت نفسها له:

\par 9 وصدق ما أخبرته به أخته في هذا الموضوع، وابتعد عن مريم على الفور، وكرهها وبغضها

\par 10 فلما علمت أخته بذلك، ذهبت إلى الساقي، وأعطته نقودًا، وأعطته سمًا، وقالت: احمل هذا إلى الملك، وقل له: أعطتني مريم زوجة الملك هذا السم، وهذه النقود، وأمرت بخلطه في شراب الملك

\par 11 فعل هذا ساقي الملك. ولما رأى الملك السم، لم يشك في صحة الأمر: فأمر بقطع رأس يوسيفوس صهره على الفور؛ وأمر أيضًا بتقييد مريم بالسلاسل، حتى يحضر الشيوخ السبعون، ويصدروا الحكم المناسب عليها

\par 12 فخافت أخت هيرودس أن يُكشف ما فعلته، وتهلك هي نفسها، إذا أُطلق سراح مريم. فقالت له: أيها الملك، إذا أجّلت موت مريم إلى الغد، فلن تتمكن من إتمامه على الإطلاق

\par 13 لأنه بمجرد أن يُعرف أنك تريد قتلها، سيأتي كل بيت أبيها، وجميع خدمهم وجيرانهم، وسيتدخلون؛ ولن تتمكن من الحصول على قتلها إلا بعد اضطرابات كبيرة

\par 14 فقال هيرودس: افعل ما يحسن في عينيك.

\par 15 فأرسلت أخت هيرودس رجلاً مسرعاً ليحضر مريم إلى موضع القتل، وأقامت عليها جواريها ونساءً أخريات ليهينوها ويوبخوها بكل أنواع الفحش.

\par 16 لكنها لم تجب أيًا منهم، ولم تحرك رأسها على الإطلاق: ولم يتغير لونها من كل هذا العلاج، ولم يظهر عليها أي خوف أو ارتباك، ولم تتغير مشيتها؛

\par 17 ولكن بطريقتها المعتادة، توجهت إلى المكان الذي اقتيدت إليه لتقتل؛ وانحنت ركبتيها، ومدت رقبتها طواعية:

\par 18 ورحلت عن هذه الحياة، مشهورة بالدين والعفة، بلا جريمة، بلا ذنب؛ ومع ذلك، لم تكن خالية تمامًا من الغطرسة، وفقًا لعادة عائلتها

\par 19 وكان من أهم أسباب ذلك الاهتمام المتذلّل والمودة التي يكنّها لها هيرودس، بسبب أناقة هيئتها؛ ولذلك لم تشكّ في حدوث أي تغيير فيه تجاهها

\par 20 وكان هيرودس قد أنجب منها ابنين، هما الإسكندر وأرسطوبولس، اللذين كانا يعيشان في روما بعد مقتل أمهما، لأنه أرسلهما إلى هناك ليتعلما أدب الرومان ولغتهم

\par 21 بعد ذلك، تاب هيرودس عن قتله زوجته، وتأثر بحزن شديد بسبب وفاتها، لدرجة أنه أصيب بمرض كاد أن يموت منه

\par 22 بعد وفاة مريم، وضعت والدتها ألكسندرا خططًا لقتل هيرودس؛ الأمر الذي علمه، فتخلص منها

\chapter{59}

\par \textit{تاريخ مجيء ابني هيرودس، الإسكندر وآري، بمجرد أن سمعا أن أمهما قد قُتلت على يد هيرودس.}

\par 1 عندما وصل خبر قتل هيرودس لأمهما إلى الإسكندر وأريستوبولس، غلب عليهما حزن شديد؛

\par 2 وغادروا روما وأتوا إلى المدينة المقدسة، ولم يُظهروا أي احترام لأبيهم هيرودس كما اعتادوا أن يفعلوا سابقًا، بسبب الكراهية التي شعروا بها تجاهه في أذهانهم بسبب وفاة والدتهم

\par 3 وكان الإسكندر قد تزوج ابنة الملك أرخيلاوس، وتزوج أريستوبولس ابنة أخت هيرودس

\par 4 فلما رأى هيرودس أنهم لا يحترمونه، رأى أنه مكروه منهم، فتجنبهم. ولم يخفِ هذا الأمر على الشبان وأهل بيته

\par 5 وكان الملك هيرودس قد تزوج امرأة قبل مريم العذراء اسمها دوسيثيا، وأنجب منها ابنًا اسمه أنتيباتر

\par 6 فلما اطمأن هيرودس بشأن ابنيه، كما ذكرنا آنفًا، أحضر زوجته دوسيثيا إلى قصره، وضم إلى نفسه ابنه أنتيباتر، وسلمه جميع أعماله؛ وعينه بوصية خليفته

\par 7 وأن أنتيباتر اضطهد شقيقيه ألكسندر وأريستوبولوس، مصممًا على أن ينعم بالسلام لنفسه أثناء حياة والده، حتى لا يكون له منافس بعد وفاته

\par 8 لذلك قال لأبيه: «في الحقيقة، إخوتي يطلبون ميراثًا بسبب عائلة أمهم، لأنها أشرف من عائلة أمي؛ ولذلك فإن لهم حقًا أوفر مني في الثروة التي اعتبرني الملك مستحقًا لها

\par 9 لهذا السبب يسعون لقتلك، وأنا أيضًا سيقتلونني بعد ذلك بقليل

\par 10 وكان يكرر هذا مرارًا على هيرودس، ويرسل إليه أيضًا أشخاصًا سرًا ليوحوا إليه بأشياء قد تزيد من بغضه لهم

\par 11 في هذه الأثناء، ذهب هيرودس إلى روما إلى أغسطس، آخذًا معه ابنه الإسكندر. "ولما جاء إلى أغسطس، شكا هيرودس إليه من ابنه، طالبًا منه أن يوبخه

\par 12 "ولكن قال الإسكندر: "إنني لا أنكر حزني بسبب مقتل أمي دون أي ذنب؛ لأن الحيوانات نفسها تُظهر عاطفة لأمهاتها أفضل بكثير من البشر، وتحبهن أكثر.

\par 13 لكنني أنكر تمامًا أي قصد لقتل والدي، وأبرئ نفسي منه أمام الله: لأن لدي نفس المشاعر تجاه والدي كما لدي تجاه والدتي:

\par 14 ولست من ذلك النوع من الرجال الذين يجلبون على أنفسهم الذنب لارتكابهم جريمة تجاه والدي، وخاصة العذاب الأبدي

\par 15 فبكى الإسكندر بكاءً مريرًا وشديدًا، وأشفق عليه أغسطس، وبكى أيضًا جميع زعماء الرومان الذين كانوا واقفين بالقرب منه

\par 16 ثم طلب أغسطس من هيرودس أن يعيد أبناءه إلى لطفه وحميميته السابقة: وطلب من الإسكندر أن يقبل قدمي أبيه، ففعل ذلك. كما أمر هيرودس أن يعانقه ويقبله، فأطاعه هيرودس

\par 17 بعد ذلك، أمر أغسطس بهدية رائعة لهيرودس، فحملت إليه. وبعد أن أمضى بضعة أيام معه، عاد هيرودس إلى البيت المقدس، ودعا إليه شيوخ يهوذا، وقال:

\par 18 «اعلموا أن أنتيباتر هو ابني الأكبر والبكر، لكن أمه من عائلة وضيعة. أما أم الإسكندر وأرسطوبولس ابنيّ فهي من عائلة رؤساء الكهنة والملوك.»

\par 19 علاوة على ذلك، فقد وسّع الله مملكتي، وبسط سلطاني؛ ولذلك: يبدو لي أنه من الجيد أن أعيّن هؤلاء أبنائي الثلاثة في سلطة متساوية؛ بحيث لا يكون لأنتيباتر سلطان على إخوته، ولا يكون لإخوته سلطان عليه

\par 20 فأطيعوا الثلاثة، يا جماعة الرجال، ولا تتدخلوا في أي أمر قد يتفق عليه أذهانهم؛ ولا تقترحوا أي أمر قد يؤدي إلى التضليل والخلاف بينهم.

\par 21 ولا تشربوا معهم، ولا تكثروا الكلام معهم. لأنه من ثم سيحدث أن ينطق أحدهم عليكم دون حذر بما لديه من نوايا ضد أخيه:

\par 22 الذي عليه، لكي تصالحوهم إليكم، ستتبعون اتفاقكم مع كل واحد منهم، حسب ما يراه مناسبًا؛ وستُهلكوهم، وستُهلك أنتم أيضًا

\par 23 من واجبكم يا أبنائي أن تطيعوا الله ولي، لكي تعمروا طويلاً، وتنجح أموركم. بعد ذلك بوقت قصير، عانقهم وقبلهم، وأمر الناس بالانصراف

\par 24 لكن ما فعله هيرودس لم يُفضِ إلى نتيجة سعيدة، ولم تتفق قلوب أبنائه. لأن أنتيباتر أراد أن يُوضع كل شيء بين يديه، كما عيّن والده سابقًا: ولم يبدُ لإخوته أنه من العدل على الإطلاق أن يُعتبر مساوٍ لهم

\par 25 وكان أنتيباتر موهوبًا بالصبر، وصداقة سيئة ومصطنعة، ولكن لم يكن الأمر كذلك مع أخويه: فأرسل أنتيباتر جواسيس على أخويه ليخبروه عنهما، وزرع آخرين ليحملوا أخبارًا كاذبة عنهما إلى بيلاطس

\par 26 ولكن عندما كان أنتيباتر في حضرة الملك، وسمع أحدًا يروي مثل هذه الأمور عن إخوته، صدّ التهمة الموجهة إليهم، معلنًا أن المؤلفين لا يستحقون الثقة، وتوسل إلى الملك ألا يصدق هذه التقارير

\par 27 وهو ما فعله أنتيباتر، حتى لا يثير لدى الملك أي شك أو ريبة في نفسه

\par 28 من هنا، لم يكن لدى الملك أدنى شك في أنه كان يميل إلى إخوته، ولم يتمنى لهم أي أذى

\par 29 وعندما علم أنتيباتر بذلك، انحنى لمشيئته، فعرض على فيروراس عمه وعمته (لأنهما كانا على عداوة مع إخوته بسبب والدتهم)، هدية ثمينة للغاية، طالبين منه إبلاغ الملك أن الإسكندر وأريستوبولوس قد وضعا خطة لقتل الملك

\par 30 (وكان هيرودس يميل إلى فيروراس أخيه، وكان يهتم بكل ما يقوله، حتى إنه كان يدفع له كل سنة مبلغًا كبيرًا من الأقاليم التي كان يحكمها على ضفة الفرات.)

\par 31 ففعل فيروراس هذا. وبعد ذلك ذهب أنتيباتر إلى هيرودس وقال له: "أيها الملك، في هذه الأيام دبر إخوتي مؤامرة لتدميري."

\par 32 علاوة على ذلك، أعطى أنتيباتر مالًا لخصيان الملك الثلاثة، ليقولوا: لقد أعطانا الإسكندر مالًا، ليستخدمنا بطريقة شريرة، ولكي نقتلك: وعندما تراجعنا عنه، هددنا بالموت

\par 33 فغضب الملك على الإسكندر وأمر بوضعه في السلاسل، وقبض على جميع خدم الإسكندر وعذبهم حتى اعترفوا بما عرفوه عن مؤامرة الإسكندر لقتله.

\par 34 وكثيرون منهم، على الرغم من أنهم ماتوا تحت التعذيب، لم يكذبوا قط بشأن الإسكندر: لكن بعضهم، لعدم قدرتهم على تحمل عنف العذاب، ابتكروا أكاذيب من خلال الرغبة في تحرير أنفسهم؛

\par 35 مؤكدين أن الإسكندر وأريستوبولوس خططا لمهاجمة الملك وقتله والفرار إلى روما؛ وبعد أن استقبلا جيشًا من أغسطس، للزحف على البيت المقدس، وقتل شقيقهما أنتيباتر، والاستيلاء على عرش يهوذا

\par 36 وأمر الملك بالقبض على أرسطوبولس وتقييده بالسلاسل، وتم ربطه ووضعه مع أخيه

\par 37 ولكن عندما وصل خبر الإسكندر إلى حميه أرخيلاوس، ذهب إلى هيرودس، متظاهرًا بأنه في غضب شديد على الإسكندر:


\par 38 كما لو أنه، عند سماعه تقريرًا عن جريمة قتل الأب المُخطط لها، قد جاء عمدًا ليرى ما إذا كانت ابنته، زوجة الإسكندر، على علم بالأمر، ولم تكشف له ذلك حتى يقتلها: ولكن إذا لم تكن على علم بأي شيء من هذا القبيل، فقد يفصلها عن الإسكندر، ويأخذها إلى منزله

\par 39 كان أرخيلاوس هذا رجلاً حكيماً فصيحاً. ولما سمع هيرودس كلامه، واقتنع بفطنته وصدقه، امتلك قلبه بطريقة عجيبة، ووثق به، واعتمد عليه دون أدنى تردد

\par 40 فلما وجد أرخيلاوس ميل هيرودس إليه، بعد علاقة حميمة طويلة، قال له في أحد الأيام عندما كانا معزولين معًا:

\par 41 «حقًا، أيها الملك، من خلال التفكير في شؤونك، وجدت أنك الآن، وأنت في سن متقدمة، تحتاج بشدة إلى راحة البال، وإلى العزاء في أبنائك؛ بينما على العكس من ذلك، فقد جلبت منهم الحزن والقلق

\par 42 علاوة على ذلك، فقد فكرت في هذين ابنيك، ولا أجد أنك كنت مقصرًا في استحقاقهما؛ لأنك رقيتهما، وجعلتهما ملكين، ولم تترك شيئًا لم تفعله، من شأنه أن يدفعهما إلى التدبير لقتلك، وليس لديهما أي سبب للدخول في هذا العمل

\par 43 ولكن ربما جاء هذا من شخص خبيث، يريد الشر لك ولهم، أو دفعك من خلال الحسد أو العداوة إلى بغضهم

\par 44 إذا كان قد نال نفوذًا عليك، وأنت رجل شيخ، موهوب بالمعرفة والمعلومات والخبرة، فغيّرك من الوداعة الأبوية إلى القسوة والغضب على أبنائك؛

\par 45 كم كان من الأسهل عليه أن يفعل بهم، وهم صغار السن، عديمو الخبرة، وغير حذرين، وليس لديهم معرفة بالرجال ومكائدهم، حتى حصل منهم على ما تمنى في هذا الأمر

\par 46 فانظر في أمورك أيها الملك، ولا تستمع إلى كلام المخبرين، ولا تفعل أي شيء بسرعة ضد أولادك، واسأل من الذي كان يدبر الشر ضدك وضدهم.

\par 47 فأجابه الملك: "في الواقع، الأمر كما ذكرت: ليتني أعرف من حرضهم على هذا." أجاب أرخيلاوس: "هذا أخوك فيروياس." أجاب الملك: "لعل الأمر كذلك."

\par 48 بعد ذلك، تغير سلوك الملك تجاه فيروراس بشكل كبير: فلما لاحظ فيروراس ذلك، خاف منه، فجاء إلى أرخيلاوس وقال له:

\par 49 «أرى كيف تغير الملك تجاهي؛ لذلك أتوسل إليكم أن توفقوا بيني وبين رأيه، وتزيلوا المشاعر التي يحملها في قلبه ضدي.»

\par 50 فأجابه أرخيلاوس: "سأفعل ذلك بالفعل، إذا وعدتَ بكشف الحقيقة للملك بشأن المؤامرات التي دبرتها ضد الإسكندر وأريستوبولوس." ووافق على ذلك

\par 51 وبعد أيام قليلة، قال أرخيلاوس للملك: "أيها الملك، إن أقارب الإنسان بالنسبة له كأعضائه، وكما أنه من الجيد للإنسان، إذا أصيب أي من أعضائه بمرض ما، أن يعالجه بالأدوية، حتى لو تسبب له ذلك في ألم؛

\par 52 وليس من الجيد قطعه، لئلا يزداد الألم، ويضعف الجسم، وتفشل الأطراف؛ وبالتالي من فقدان ذلك الطرف، يشعر بالنقص في العديد من وسائل الراحة:

\par 53 بل ليتحمل آلام العلاج الطبي، حتى يتحسن عضوه، ويُشفى، ويعود جسده إلى كماله وقوته السابقة

\par 54 فهل يليق بالرجل، كلما تغير أحد أقاربه تجاهه، لأي سبب بغيض مهما كان، أن يتصالح معه؛

\par 55 إغرائه بالكياسة والصداقة، والاعتراف بأعذاره، ورفض التهم الموجهة إليه: وألا يقتله على عجل، ولا يبعده عن حضرته لفترة طويلة

\par 56 فإن أقارب الرجل هم أعوانه ومساعدوه، وفيهم يكمن شرفه ومجده، ومن خلالهم ينال ما لا يستطيع الحصول عليه لولا ذلك

\par 57 فيروراس هو حقًا شقيق الملك، وابن أبيه وأمه: وهو يعترف بخطئه، ويتوسل إلى الملك أن يعفو عنه، وأن يطرد خطئه من ذهنه. فأجاب الملك: "هذا ما سأفعله."

\par 58 وأمر فيروراس بالمثول أمامه، الذي قال له عندما كان في حضرته: "لقد أخطأت الآن أمام الله العظيم الصالح، وأمام الملك، إذ دبرت أذىً وخططًا قد تضر بشؤون الملك وأبنائه، من خلال الكذب بالأكاذيب

\par 59 لكن الذي دفعني إلى التصرف هكذا هو أن الملك أخذ مني امرأة، سريتي، وفصلني عنها

\par 60 قال الملك لأرخيلاوس: "لقد عفوت الآن عن فيروراس كما طلبت مني: لأني أجد أنك قد شفيت المرض الذي كان في شؤوننا بطرقك المهدئة، كما يشفي الطبيب العبقري فساد الجسد المريض

\par 61 لذلك أتوسل إليك أن تسامح ألكسندر، وأن تصالح ابنتك مع زوجها؛ لأنك تعتبرها ابنتي، لأني أعلم أنها أحكم منه، وأنها تصده عن أشياء كثيرة بفطنتها وتحذيراتها

\par 62 لذلك أدعوك ألا تفرق بينهما وتدمره، لأنه يتفق معها، ويحصل على فوائد كثيرة من إرشادها

\par 63 فأجاب أرخيلاوس: "ابنتي هي أمة الملك، لكن نفسي كرهته مؤخرًا بسبب تدبيره الشرير. فليأذن لي الملك أن أفصله عن ابنتي، التي يجوز للملك أن يضمها إلى من يشاء من عبيده."

\par 64 فأجابه الملك: «لا تتجاوز طلبي، ولتبق ابنتك عنده ولا تعترضني». فقال أرخيلاوس: «بالتأكيد سأفعل ذلك، ولن أعترض الملك في أي شيء يأمرني به».

\par 65 بعد ذلك بوقت قصير، أمر هيرودس بفك قيود الإسكندر وأريستوبولوس، والمثول أمامه: اللذان عندما كانا في حضرته، سجدا أمامه، معترفين بأخطائهما، ومعتذرين، ومتوسلين إليه بالعفو والمغفرة

\par 66 فأمرهم بالقيام، ثم دعاهم إلى الاقتراب منه، وقبّلهم، وأمرهم بالذهاب إلى منازلهم، والعودة في الغد. فجاءوا ليأكلوا ويشربوا معه، فأعادهم إلى مكانة أعلى

\par 67 وأعطى أرخيلاوس سبعين وزنة وسريرًا من ذهب، وأمر أيضًا جميع كبار أصدقائه بتقديم هدايا ثمينة لأرخيلاوس، ففعلوا ذلك

\par 68 ولما تم ذلك، غادر أرخيلاوس مدينة البيت المقدس إلى بلده، فرافقه هيرودس، وبعد أن ودعه أخيرًا عاد إلى البيت المقدس

\par 69 ومع ذلك، لم يتوقف أنتيباتر عن مؤامراته ضد إخوته، ليجعلهم بغيضين

\par 70 وحدث أن رجلاً جاء إلى هيرودس ومعه أشياء ثمينة وجميلة، مما يُكسب به الملوك عادةً؛

\par 71 قدمها للملك، الذي أخذها منه، وردّها إليه؛ وحصل الرجل على مكانة عالية جدًا في عواطفه، وبعد أن تم ضمه إلى حاشيته، تمتع بثقته: كان اسم هذا الرجل يوريكليس

\par 72 فلما رأى أنتيباتر أن هذا الرجل قد نال رضا أبيه تمامًا، عرض عليه مالًا، طالبًا منه أن يلمح ببراعة إلى هيرودس، وأن يزعم أن ابنيه الإسكندر وأريستوبولوس يخططان لقتله؛ وهو ما وعده به الرجل

\par 73 بعد ذلك بوقت قصير، ذهب إلى الإسكندر، وأصبح على علاقة حميمة معه لدرجة أنه كان معروفًا بصداقته، وأُبلغ الملك أنه كان على علاقة حميمة معه

\par 74 بعد ذلك، انزوى مع الملك، وقال له: "بالتأكيد لديك هذا الحق عليّ، أيها الملك، أنه لا ينبغي أن يمنعني شيء من تقديم النصيحة الجيدة لك: وفي الحقيقة لدي أمر ينبغي أن يعرفه الملك، وينبغي أن أكشفه لك."

\par 75 قال له الملك: "ماذا لديك؟" أجابه الرجل: "سمعت الإسكندر يقول: حقًا إن الله قد أرجأ الانتقام من أبي لموت أمي وجدي وأقاربي بغير ذنب، ليقع ذلك بيدي، وأرجو أن أنتقم لهم منه."

\بار{76}
والآن اتفق مع بعض الزعماء على مهاجمتك، وأراد أن يورطني في الخطط التي وضعها: لكنني اعتبرتها جريمة، بسبب تصرفات الملك اللطيفة تجاهي، وكرمه

\par 77 لكن نيتي هي أن أنصحه جيدًا، وأن أبلغه بذلك، لأنه يمتلك كلتا العينين والفهم

\par 78 وعندما سمع الملك هذه الكلمات، لم ينكرها أبدًا، بل بدأ بسرعة في الاستفسار عن صحتها:

\par 79 لكنه لم يجد شيئًا يمكنه الاعتماد عليه، باستثناء رسالة مزورة باسم الإسكندر وأريستوبولوس إلى حاكم مدينة معينة

\par 80 وجاء في الرسالة: "نريد أن نقتل والدنا ونهرب إليكم، فأعدّوا لنا مكانًا نبقى فيه حتى يجتمع الناس حولنا، وتستقر أمورنا."

\par 81 وقد تأكد هذا بالفعل للملك، وبدا محتملاً: لذلك قبض على حاكم تلك المدينة وعذبه، حتى يعترف بما وُضع في تلك الرسالة

\par 82 وهو ما أنكره هذا الرجل، مُبرئًا نفسه من التهمة: ولم يُثبت ضدهم أي شيء في هذا الأمر، أو في أي شيء آخر اتهمهم به المُخبر

\par 83 فأمر هيرودس أن يُقبض عليهم ويُقيدوا بالسلاسل والقيود. ثم ذهب إلى صور، ومن صور إلى قيصرية، حاملاً إياهم معه مقيدين بالسلاسل

\par 84 وأشفق عليهم جميع القادة وجميع الجنود، لكن لم يشفع لهم أحد لدى الملك، خشية أن يعترف هو بنفسه بما أكده المخبر

\par 85 كان في الجيش محارب عجوز كان له ابن في خدمة الإسكندر. فلما رأى الرجل العجوز الحالة البائسة لابني هيرودس، شفق على حالهما شفقة عجيبة، وصرخ بأعلى صوت استطاعه: "لقد زالت الشفقة؛ واختفى الخير والتقوى؛ واختفى الحق من العالم."

\par 86 ثم قال للملك: "يا أيها القاسي على أولادك، عدو أصدقائك، وصديق أعدائك، الذي يتلقى كلام المخبرين والأشخاص الذين لا يريدون لك الخير!"

\par 87 فركض إليه أعداء الإسكندر وأرسطوبولس ووبخوه وقالوا للملك: أيها الملك، ليس الحب نحوك ونحو أبنائك هو الذي دفع هذا الرجل إلى أن يتكلم هكذا؛

\par 88 لكنه أراد أن يثرثِر بالكراهية التي يكنها لك في قلبه، وأن يتحدث بسوء عن مشورتك وإدارتك، باعتبارك مستشارًا أمينًا

\par 89 وبالفعل، أخبرنا بعض المراقبين عنه أنه كان قد عاهد حلاق الملك على قتله بالموس أثناء حلاقته

\par 90 وأمر الملك بالقبض على الشيخ وابنه والحلاق، وجلد الشيخ والحلاق بالعصي حتى يعترفوا. وضُربوا بالعصي بقسوة بالغة، وتعرضوا لأنواع مختلفة من التعذيب، لكنهم لم يعترفوا بشيء مما لم يفعلوه

\par 91 فلما رأى ابن الشيخ حالة أبيه الحزينة، والحالة التي وصل إليها، أشفق عليه، وظن أنه سيتحرر إذا اعترف هو نفسه بما وُضع على أبيه، بعد أن تلقى من الملك وعدًا بحياته

\par 92 لذلك قال للملك: "أيها الملك، أعطني ضمانًا لأبي ولنفسي، لأخبرك بما تطلبه." فقال الملك: "لك هذا."

\par 93 قال له: «كان الإسكندر قد اتفق مع والدي على قتلك، لكن والدي اتفق مع الحلاق كما قيل لك».

\par 94 ثم أمر الملك بقتل ذلك الرجل العجوز وابنه والحلاق. وأمر أيضًا بأخذ ابنيه الإسكندر وأرسطوبولس إلى سبسطية، وهناك يُقتلان ويُعلقان على المشنقة. فأخذوهما وقُتلوا وشُنقا على المشنقة

\par 95 ترك الإسكندر ولدين بقيا على قيد الحياة، وهما تيركانس والإسكندر، من ابنة الملك أرخيلاوس. وترك أريستوبولس ثلاثة أبناء، وهم أريستوبولس، وأغريبا، وهيرودس

\par 96 لكن تاريخ ابن هيرودس، أنتيباتر، قد وُصف بالفعل في رواياتنا السابقة

\end{document}