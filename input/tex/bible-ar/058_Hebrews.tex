\begin{document}

\title{عبرانيين}


\chapter{1}

\par 1 اَللهُ، بَعْدَ مَا كَلَّمَ الآبَاءَ بِالأَنْبِيَاءِ قَدِيماً، بِأَنْوَاعٍ وَطُرُقٍ كَثِيرَةٍ،
\par 2 كَلَّمَنَا فِي هَذِهِ الأَيَّامِ الأَخِيرَةِ فِي ابْنِهِ - الَّذِي جَعَلَهُ وَارِثاً لِكُلِّ شَيْءٍ، الَّذِي بِهِ أَيْضاً عَمِلَ الْعَالَمِينَ.
\par 3 الَّذِي، وَهُوَ بَهَاءُ مَجْدِهِ، وَرَسْمُ جَوْهَرِهِ، وَحَامِلٌ كُلَّ الأَشْيَاءِ بِكَلِمَةِ قُدْرَتِهِ، بَعْدَ مَا صَنَعَ بِنَفْسِهِ تَطْهِيراً لِخَطَايَانَا، جَلَسَ فِي يَمِينِ الْعَظَمَةِ فِي الأَعَالِي،
\par 4 صَائِراً أَعْظَمَ مِنَ الْمَلاَئِكَةِ بِمِقْدَارِ مَا وَرِثَ اسْماً أَفْضَلَ مِنْهُمْ.
\par 5 لأَنَّهُ لِمَنْ مِنَ الْمَلاَئِكَةِ قَالَ قَطُّ: «أَنْتَ ابْنِي أَنَا الْيَوْمَ وَلَدْتُكَ»؟ وَأَيْضاً: «أَنَا أَكُونُ لَهُ أَباً وَهُوَ يَكُونُ لِيَ ابْناً»؟
\par 6 وَأَيْضاً مَتَى أَدْخَلَ الْبِكْرَ إِلَى الْعَالَمِ يَقُولُ: «وَلْتَسْجُدْ لَهُ كُلُّ مَلاَئِكَةِ اللهِ».
\par 7 وَعَنِ الْمَلاَئِكَةِ يَقُولُ: «الصَّانِعُ مَلاَئِكَتَهُ رِيَاحاً وَخُدَّامَهُ لَهِيبَ نَارٍ».
\par 8 وَأَمَّا عَنْ الاِبْنِ: «كُرْسِيُّكَ يَا أَللهُ إِلَى دَهْرِ الدُّهُورِ. قَضِيبُ اسْتِقَامَةٍ قَضِيبُ مُلْكِكَ.
\par 9 أَحْبَبْتَ الْبِرَّ وَأَبْغَضْتَ الإِثْمَ. مِنْ أَجْلِ ذَلِكَ مَسَحَكَ اللهُ إِلَهُكَ بِزَيْتِ الاِبْتِهَاجِ أَكْثَرَ مِنْ شُرَكَائِكَ».
\par 10 وَ«أَنْتَ يَا رَبُّ فِي الْبَدْءِ أَسَّسْتَ الأَرْضَ، وَالسَّمَاوَاتُ هِيَ عَمَلُ يَدَيْكَ.
\par 11 هِيَ تَبِيدُ وَلَكِنْ أَنْتَ تَبْقَى، وَكُلُّهَا كَثَوْبٍ تَبْلَى،
\par 12 وَكَرِدَاءٍ تَطْوِيهَا فَتَتَغَيَّرُ. وَلَكِنْ أَنْتَ أَنْتَ، وَسِنُوكَ لَنْ تَفْنَى».
\par 13 ثُمَّ لِمَنْ مِنَ الْمَلاَئِكَةِ قَالَ قَطُّ: «اِجْلِسْ عَنْ يَمِينِي حَتَّى أَضَعَ أَعْدَاءَكَ مَوْطِئاً لِقَدَمَيْكَ؟»
\par 14 أَلَيْسَ جَمِيعُهُمْ أَرْوَاحاً خَادِمَةً مُرْسَلَةً لِلْخِدْمَةِ لأَجْلِ الْعَتِيدِينَ أَنْ يَرِثُوا الْخَلاَصَ!

\chapter{2}

\par 1 لِذَلِكَ يَجِبُ أَنْ نَتَنَبَّهَ أَكْثَرَ إِلَى مَا سَمِعْنَا لِئَلاَّ نَفُوتَهُ،
\par 2 لأَنَّهُ إِنْ كَانَتِ الْكَلِمَةُ الَّتِي تَكَلَّمَ بِهَا مَلاَئِكَةٌ قَدْ صَارَتْ ثَابِتَةً، وَكُلُّ تَعَدٍّ وَمَعْصِيَةٍ نَالَ مُجَازَاةً عَادِلَةً،
\par 3 فَكَيْفَ نَنْجُو نَحْنُ إِنْ أَهْمَلْنَا خَلاَصاً هَذَا مِقْدَارُهُ، قَدِ ابْتَدَأَ الرَّبُّ بِالتَّكَلُّمِ بِهِ، ثُمَّ تَثَبَّتَ لَنَا مِنَ الَّذِينَ سَمِعُوا،
\par 4 شَاهِداً اللهُ مَعَهُمْ بِآيَاتٍ وَعَجَائِبَ وَقُوَّاتٍ مُتَنَّوِعَةٍ وَمَوَاهِبِ الرُّوحِ الْقُدُسِ، حَسَبَ إِرَادَتِهِ؟
\par 5 فَإِنَّهُ لِمَلاَئِكَةٍ لَمْ يُخْضِعِ «الْعَالَمَ الْعَتِيدَ» الَّذِي نَتَكَلَّمُ عَنْهُ.
\par 6 لَكِنْ شَهِدَ وَاحِدٌ فِي مَوْضِعٍ قَائِلاً: «مَا هُوَ الإِنْسَانُ حَتَّى تَذْكُرَهُ، أَوِ ابْنُ الإِنْسَانِ حَتَّى تَفْتَقِدَهُ؟
\par 7 وَضَعْتَهُ قَلِيلاً عَنِ الْمَلاَئِكَةِ. بِمَجْدٍ وَكَرَامَةٍ كَلَّلْتَهُ، وَأَقَمْتَهُ عَلَى أَعْمَالِ يَدَيْكَ.
\par 8 أَخْضَعْتَ كُلَّ شَيْءٍ تَحْتَ قَدَمَيْهِ». لأَنَّهُ إِذْ أَخْضَعَ الْكُلَّ لَهُ لَمْ يَتْرُكْ شَيْئاً غَيْرَ خَاضِعٍ لَهُ - عَلَى أَنَّنَا الآنَ لَسْنَا نَرَى الْكُلَّ بَعْدُ مُخْضَعاً لَهُ -
\par 9 وَلَكِنَّ الَّذِي وُضِعَ قَلِيلاً عَنِ الْمَلاَئِكَةِ، يَسُوعَ، نَرَاهُ مُكَلَّلاً بِالْمَجْدِ وَالْكَرَامَةِ، مِنْ أَجْلِ أَلَمِ الْمَوْتِ، لِكَيْ يَذُوقَ بِنِعْمَةِ اللهِ الْمَوْتَ لأَجْلِ كُلِّ وَاحِدٍ.
\par 10 لأَنَّهُ لاَقَ بِذَاكَ الَّذِي مِنْ أَجْلِهِ الْكُلُّ وَبِهِ الْكُلُّ، وَهُوَ آتٍ بِأَبْنَاءٍ كَثِيرِينَ إِلَى الْمَجْدِ أَنْ يُكَمِّلَ رَئِيسَ خَلاَصِهِمْ بِالآلاَمِ.
\par 11 لأَنَّ الْمُقَدِّسَ وَالْمُقَدَّسِينَ جَمِيعَهُمْ مِنْ وَاحِدٍ، فَلِهَذَا السَّبَبِ لاَ يَسْتَحِي أَنْ يَدْعُوَهُمْ إِخْوَةً،
\par 12 قَائِلاً: «أُخَبِّرُ بِاسْمِكَ إِخْوَتِي، وَفِي وَسَطِ الْكَنِيسَةِ أُسَبِّحُكَ».
\par 13 وَأَيْضاً: «أَنَا أَكُونُ مُتَوَكِّلاً عَلَيْهِ». وَأَيْضاً: «هَا أَنَا وَالأَوْلاَدُ الَّذِينَ أَعْطَانِيهِمِ اللهُ».
\par 14 فَإِذْ قَدْ تَشَارَكَ الأَوْلاَدُ فِي اللَّحْمِ وَالدَّمِ اشْتَرَكَ هُوَ أَيْضاً كَذَلِكَ فِيهِمَا، لِكَيْ يُبِيدَ بِالْمَوْتِ ذَاكَ الَّذِي لَهُ سُلْطَانُ الْمَوْتِ، أَيْ إِبْلِيسَ،
\par 15 وَيُعْتِقَ أُولَئِكَ الَّذِينَ خَوْفاً مِنَ الْمَوْتِ كَانُوا جَمِيعاً كُلَّ حَيَاتِهِمْ تَحْتَ الْعُبُودِيَّةِ.
\par 16 لأَنَّهُ حَقّاً لَيْسَ يُمْسِكُ الْمَلاَئِكَةَ، بَلْ يُمْسِكُ نَسْلَ إِبْرَاهِيمَ.
\par 17 مِنْ ثَمَّ كَانَ يَنْبَغِي أَنْ يُشْبِهَ إِخْوَتَهُ فِي كُلِّ شَيْءٍ، لِكَيْ يَكُونَ رَحِيماً، وَرَئِيسَ كَهَنَةٍ أَمِيناً فِي مَا لِلَّهِ حَتَّى يُكَفِّرَ خَطَايَا الشَّعْبِ.
\par 18 لأَنَّهُ فِي مَا هُوَ قَدْ تَأَلَّمَ مُجَرَّباً يَقْدِرُ أَنْ يُعِينَ الْمُجَرَّبِينَ.

\chapter{3}

\par 1 مِنْ ثَمَّ أَيُّهَا الإِخْوَةُ الْقِدِّيسُونَ، شُرَكَاءُ الدَّعْوَةِ السَّمَاوِيَّةِ، لاَحِظُوا رَسُولَ اعْتِرَافِنَا وَرَئِيسَ كَهَنَتِهِ الْمَسِيحَ يَسُوعَ،
\par 2 حَالَ كَوْنِهِ أَمِيناً لِلَّذِي أَقَامَهُ، كَمَا كَانَ مُوسَى أَيْضاً فِي كُلِّ بَيْتِهِ.
\par 3 فَإِنَّ هَذَا قَدْ حُسِبَ أَهْلاً لِمَجْدٍ أَكْثَرَ مِنْ مُوسَى، بِمِقْدَارِ مَا لِبَانِي الْبَيْتِ مِنْ كَرَامَةٍ أَكْثَرَ مِنَ الْبَيْتِ.
\par 4 لأَنَّ كُلَّ بَيْتٍ يَبْنِيهِ إِنْسَانٌ مَا، وَلَكِنَّ بَانِيَ الْكُلِّ هُوَ اللهُ.
\par 5 وَمُوسَى كَانَ أَمِيناً فِي كُلِّ بَيْتِهِ كَخَادِمٍ، شَهَادَةً لِلْعَتِيدِ أَنْ يُتَكَلَّمَ بِهِ.
\par 6 وَأَمَّا الْمَسِيحُ فَكَابْنٍ عَلَى بَيْتِهِ. وَبَيْتُهُ نَحْنُ إِنْ تَمَسَّكْنَا بِثِقَةِ الرَّجَاءِ وَافْتِخَارِهِ ثَابِتَةً إِلَى النِّهَايَةِ.
\par 7 لِذَلِكَ كَمَا يَقُولُ الرُّوحُ الْقُدُسُ: «الْيَوْمَ إِنْ سَمِعْتُمْ صَوْتَهُ
\par 8 فَلاَ تُقَسُّوا قُلُوبَكُمْ، كَمَا فِي الإِسْخَاطِ، يَوْمَ التَّجْرِبَةِ فِي الْقَفْرِ
\par 9 حَيْثُ جَرَّبَنِي آبَاؤُكُمُ. اخْتَبَرُونِي وَأَبْصَرُوا أَعْمَالِي أَرْبَعِينَ سَنَةً.
\par 10 لِذَلِكَ مَقَتُّ ذَلِكَ الْجِيلَ، وَقُلْتُ إِنَّهُمْ دَائِماً يَضِلُّونَ فِي قُلُوبِهِمْ، وَلَكِنَّهُمْ لَمْ يَعْرِفُوا سُبُلِي.
\par 11 حَتَّى أَقْسَمْتُ فِي غَضَبِي لَنْ يَدْخُلُوا رَاحَتِي».
\par 12 اُنْظُرُوا أَيُّهَا الإِخْوَةُ أَنْ لاَ يَكُونَ فِي أَحَدِكُمْ قَلْبٌ شِرِّيرٌ بِعَدَمِ إِيمَانٍ فِي الاِرْتِدَادِ عَنِ اللهِ الْحَيِّ،
\par 13 بَلْ عِظُوا أَنْفُسَكُمْ كُلَّ يَوْمٍ، مَا دَامَ الْوَقْتُ يُدْعَى الْيَوْمَ، لِكَيْ لاَ يُقَسَّى أَحَدٌ مِنْكُمْ بِغُرُورِ الْخَطِيَّةِ.
\par 14 لأَنَّنَا قَدْ صِرْنَا شُرَكَاءَ الْمَسِيحِ، إِنْ تَمَسَّكْنَا بِبَدَاءَةِ الثِّقَةِ ثَابِتَةً إِلَى النِّهَايَةِ،
\par 15 إِذْ قِيلَ: «الْيَوْمَ إِنْ سَمِعْتُمْ صَوْتَهُ فَلاَ تُقَسُّوا قُلُوبَكُمْ، كَمَا فِي الإِسْخَاطِ».
\par 16 فَمَنْ هُمُ الَّذِينَ إِذْ سَمِعُوا أَسْخَطُوا؟ أَلَيْسَ جَمِيعُ الَّذِينَ خَرَجُوا مِنْ مِصْرَ بِوَاسِطَةِ مُوسَى؟
\par 17 وَمَنْ مَقَتَ أَرْبَعِينَ سَنَةً؟ أَلَيْسَ الَّذِينَ أَخْطَأُوا، الَّذِينَ جُثَثُهُمْ سَقَطَتْ فِي الْقَفْرِ؟
\par 18 وَلِمَنْ أَقْسَمَ لَنْ يَدْخُلُوا رَاحَتَهُ، إِلاَّ لِلَّذِينَ لَمْ يُطِيعُوا؟
\par 19 فَنَرَى أَنَّهُمْ لَمْ يَقْدِرُوا أَنْ يَدْخُلُوا لِعَدَمِ الإِيمَانِ.

\chapter{4}

\par 1 فَلْنَخَفْ، أَنَّهُ مَعَ بَقَاءِ وَعْدٍ بِالدُّخُولِ إِلَى رَاحَتِهِ، يُرَى أَحَدٌ مِنْكُمْ أَنَّهُ قَدْ خَابَ مِنْهُ!
\par 2 لأَنَّنَا نَحْنُ أَيْضاً قَدْ بُشِّرْنَا كَمَا أُولَئِكَ، لَكِنْ لَمْ تَنْفَعْ كَلِمَةُ الْخَبَرِ أُولَئِكَ. إِذْ لَمْ تَكُنْ مُمْتَزِجَةً بِالإِيمَانِ فِي الَّذِينَ سَمِعُوا.
\par 3 لأَنَّنَا نَحْنُ الْمُؤْمِنِينَ نَدْخُلُ الرَّاحَةَ، كَمَا قَالَ: «حَتَّى أَقْسَمْتُ فِي غَضَبِي لَنْ يَدْخُلُوا رَاحَتِي!» مَعَ كَوْنِ الأَعْمَالِ قَدْ أُكْمِلَتْ مُنْذُ تَأْسِيسِ الْعَالَمِ.
\par 4 لأَنَّهُ قَالَ فِي مَوْضِعٍ عَنِ السَّابِعِ: «وَاسْتَرَاحَ اللهُ فِي الْيَوْمِ السَّابِعِ مِنْ جَمِيعِ أَعْمَالِهِ».
\par 5 وَفِي هَذَا أَيْضاً: «لَنْ يَدْخُلُوا رَاحَتِي».
\par 6 فَإِذْ بَقِيَ أَنَّ قَوْماً يَدْخُلُونَهَا، وَالَّذِينَ بُشِّرُوا أَوَّلاً لَمْ يَدْخُلُوا لِسَبَبِ الْعِصْيَانِ،
\par 7 يُعَيِّنُ أَيْضاً يَوْماً قَائِلاً فِي دَاوُدَ: «الْيَوْمَ» بَعْدَ زَمَانٍ هَذَا مِقْدَارُهُ، كَمَا قِيلَ: «الْيَوْمَ إِنْ سَمِعْتُمْ صَوْتَهُ فَلاَ تُقَسُّوا قُلُوبَكُمْ».
\par 8 لأَنَّهُ لَوْ كَانَ يَشُوعُ قَدْ أَرَاحَهُمْ لَمَا تَكَلَّمَ بَعْدَ ذَلِكَ عَنْ يَوْمٍ آخَرَ.
\par 9 إِذاً بَقِيَتْ رَاحَةٌ لِشَعْبِ اللهِ!
\par 10 لأَنَّ الَّذِي دَخَلَ رَاحَتَهُ اسْتَرَاحَ هُوَ أَيْضاً مِنْ أَعْمَالِهِ، كَمَا اللهُ مِنْ أَعْمَالِهِ.
\par 11 فَلْنَجْتَهِدْ أَنْ نَدْخُلَ تِلْكَ الرَّاحَةَ، لِئَلاَّ يَسْقُطَ أَحَدٌ فِي عِبْرَةِ الْعِصْيَانِ هَذِهِ عَيْنِهَا.
\par 12 لأَنَّ كَلِمَةَ اللهِ حَيَّةٌ وَفَعَّالَةٌ وَأَمْضَى مِنْ كُلِّ سَيْفٍ ذِي حَدَّيْنِ، وَخَارِقَةٌ إِلَى مَفْرَقِ النَّفْسِ وَالرُّوحِ وَالْمَفَاصِلِ وَالْمِخَاخِ، وَمُمَيِّزَةٌ أَفْكَارَ الْقَلْبِ وَنِيَّاتِهِ.
\par 13 وَلَيْسَتْ خَلِيقَةٌ غَيْرَ ظَاهِرَةٍ قُدَّامَهُ، بَلْ كُلُّ شَيْءٍ عُرْيَانٌ وَمَكْشُوفٌ لِعَيْنَيْ ذَلِكَ الَّذِي مَعَهُ أَمْرُنَا.
\par 14 فَإِذْ لَنَا رَئِيسُ كَهَنَةٍ عَظِيمٌ قَدِ اجْتَازَ السَّمَاوَاتِ، يَسُوعُ ابْنُ اللهِ، فَلْنَتَمَسَّكْ بِالإِقْرَارِ.
\par 15 لأَنْ لَيْسَ لَنَا رَئِيسُ كَهَنَةٍ غَيْرُ قَادِرٍ أَنْ يَرْثِيَ لِضَعَفَاتِنَا، بَلْ مُجَرَّبٌ فِي كُلِّ شَيْءٍ مِثْلُنَا، بِلاَ خَطِيَّةٍ.
\par 16 فَلْنَتَقَدَّمْ بِثِقَةٍ إِلَى عَرْشِ النِّعْمَةِ لِكَيْ نَنَالَ رَحْمَةً وَنَجِدَ نِعْمَةً عَوْناً فِي حِينِهِ.

\chapter{5}

\par 1 لأَنَّ كُلَّ رَئِيسِ كَهَنَةٍ مَأْخُوذٍ مِنَ النَّاسِ يُقَامُ لأَجْلِ النَّاسِ فِي مَا لِلَّهِ، لِكَيْ يُقَدِّمَ قَرَابِينَ وَذَبَائِحَ عَنِ الْخَطَايَا،
\par 2 قَادِراً أَنْ يَتَرَفَّقَ بِالْجُهَّالِ وَالضَّالِّينَ، إِذْ هُوَ أَيْضاً مُحَاطٌ بِالضُّعْفِ.
\par 3 وَلِهَذَا الضُّعْفِ يَلْتَزِمُ أَنَّهُ كَمَا يُقَدِّمُ عَنِ الْخَطَايَا لأَجْلِ الشَّعْبِ هَكَذَا أَيْضاً لأَجْلِ نَفْسِهِ.
\par 4 وَلاَ يَأْخُذُ أَحَدٌ هَذِهِ الْوَظِيفَةَ بِنَفْسِهِ، بَلِ الْمَدْعُّوُ مِنَ اللهِ، كَمَا هَارُونُ أَيْضاً.
\par 5 كَذَلِكَ الْمَسِيحُ أَيْضاً لَمْ يُمَجِّدْ نَفْسَهُ لِيَصِيرَ رَئِيسَ كَهَنَةٍ، بَلِ الَّذِي قَالَ لَهُ: «أَنْتَ ابْنِي أَنَا الْيَوْمَ وَلَدْتُكَ».
\par 6 كَمَا يَقُولُ أَيْضاً فِي مَوْضِعٍ آخَرَ: «أَنْتَ كَاهِنٌ إِلَى الأَبَدِ عَلَى رُتْبَةِ مَلْكِي صَادِقَ».
\par 7 الَّذِي، فِي أَيَّامِ جَسَدِهِ، إِذْ قَدَّمَ بِصُرَاخٍ شَدِيدٍ وَدُمُوعٍ طِلْبَاتٍ وَتَضَرُّعَاتٍ لِلْقَادِرِ أَنْ يُخَلِّصَهُ مِنَ الْمَوْتِ، وَسُمِعَ لَهُ مِنْ أَجْلِ تَقْوَاهُ،
\par 8 مَعَ كَوْنِهِ ابْناً تَعَلَّمَ الطَّاعَةَ مِمَّا تَأَلَّمَ بِهِ.
\par 9 وَإِذْ كُمِّلَ صَارَ لِجَمِيعِ الَّذِينَ يُطِيعُونَهُ سَبَبَ خَلاَصٍ أَبَدِيٍّ،
\par 10 مَدْعُّواً مِنَ اللهِ رَئِيسَ كَهَنَةٍ عَلَى رُتْبَةِ مَلْكِي صَادِقَ.
\par 11 اَلَّذِي مِنْ جِهَتِهِ الْكَلاَمُ كَثِيرٌ عِنْدَنَا، وَعَسِرُ التَّفْسِيرِ لِنَنْطِقَ بِهِ، إِذْ قَدْ صِرْتُمْ مُتَبَاطِئِي الْمَسَامِعِ.
\par 12 لأَنَّكُمْ إِذْ كَانَ يَنْبَغِي أَنْ تَكُونُوا مُعَلِّمِينَ لِسَبَبِ طُولِ الزَّمَانِ، تَحْتَاجُونَ أَنْ يُعَلِّمَكُمْ أَحَدٌ مَا هِيَ أَرْكَانُ بَدَاءَةِ أَقْوَالِ اللهِ، وَصِرْتُمْ مُحْتَاجِينَ إِلَى اللَّبَنِ لاَ إِلَى طَعَامٍ قَوِيٍّ.
\par 13 لأَنَّ كُلَّ مَنْ يَتَنَاوَلُ اللَّبَنَ هُوَ عَدِيمُ الْخِبْرَةِ فِي كَلاَمِ الْبِرِّ لأَنَّهُ طِفْلٌ،
\par 14 وَأَمَّا الطَّعَامُ الْقَوِيُّ فَلِلْبَالِغِينَ، الَّذِينَ بِسَبَبِ التَّمَرُّنِ قَدْ صَارَتْ لَهُمُ الْحَوَاسُّ مُدَرَّبَةً عَلَى التَّمْيِيزِ بَيْنَ الْخَيْرِ وَالشَّرِّ.

\chapter{6}

\par 1 لِذَلِكَ وَنَحْنُ تَارِكُونَ كَلاَمَ بَدَاءَةِ الْمَسِيحِ لِنَتَقَدَّمْ إِلَى الْكَمَالِ، غَيْرَ وَاضِعِينَ أَيْضاً أَسَاسَ التَّوْبَةِ مِنَ الأَعْمَالِ الْمَيِّتَةِ، وَالإِيمَانِ بِاللهِ،
\par 2 تَعْلِيمَ الْمَعْمُودِيَّاتِ، وَوَضْعَ الأَيَادِي، قِيَامَةَ الأَمْوَاتِ، وَالدَّيْنُونَةَ الأَبَدِيَّةَ -
\par 3 وَهَذَا سَنَفْعَلُهُ إِنْ أَذِنَ اللهُ.
\par 4 لأَنَّ الَّذِينَ اسْتُنِيرُوا مَرَّةً، وَذَاقُوا الْمَوْهِبَةَ السَّمَاوِيَّةَ وَصَارُوا شُرَكَاءَ الرُّوحِ الْقُدُسِ،
\par 5 وَذَاقُوا كَلِمَةَ اللهِ الصَّالِحَةَ وَقُوَّاتِ الدَّهْرِ الآتِي،
\par 6 وَسَقَطُوا، لاَ يُمْكِنُ تَجْدِيدُهُمْ أَيْضاً لِلتَّوْبَةِ، إِذْ هُمْ يَصْلِبُونَ لأَنْفُسِهِمُِ ابْنَ اللهِ ثَانِيَةً وَيُشَهِّرُونَهُ.
\par 7 لأَنَّ أَرْضاً قَدْ شَرِبَتِ الْمَطَرَ الآتِيَ عَلَيْهَا مِرَاراً كَثِيرَةً، وَأَنْتَجَتْ عُشْباً صَالِحاً لِلَّذِينَ فُلِحَتْ مِنْ أَجْلِهِمْ، تَنَالُ بَرَكَةً مِنَ اللهِ.
\par 8 وَلَكِنْ إِنْ أَخْرَجَتْ شَوْكاً وَحَسَكاً، فَهِيَ مَرْفُوضَةٌ وَقَرِيبَةٌ مِنَ اللَّعْنَةِ، الَّتِي نِهَايَتُهَا لِلْحَرِيقِ.
\par 9 وَلَكِنَّنَا قَدْ تَيَقَّنَّا مِنْ جِهَتِكُمْ أَيُّهَا الأَحِبَّاءُ أُمُوراً أَفْضَلَ، وَمُخْتَصَّةً بِالْخَلاَصِ، وَإِنْ كُنَّا نَتَكَلَّمُ هَكَذَا.
\par 10 لأَنَّ اللهَ لَيْسَ بِظَالِمٍ حَتَّى يَنْسَى عَمَلَكُمْ وَتَعَبَ الْمَحَبَّةِ الَّتِي أَظْهَرْتُمُوهَا نَحْوَ اسْمِهِ، إِذْ قَدْ خَدَمْتُمُ الْقِدِّيسِينَ وَتَخْدِمُونَهُمْ.
\par 11 وَلَكِنَّنَا نَشْتَهِي أَنَّ كُلَّ وَاحِدٍ مِنْكُمْ يُظْهِرُ هَذَا الاِجْتِهَادَ عَيْنَهُ لِيَقِينِ الرَّجَاءِ إِلَى النِّهَايَةِ،
\par 12 لِكَيْ لاَ تَكُونُوا مُتَبَاطِئِينَ بَلْ مُتَمَثِّلِينَ بِالَّذِينَ بِالإِيمَانِ وَالأَنَاةِ يَرِثُونَ الْمَوَاعِيدَ.
\par 13 فَإِنَّهُ لَمَّا وَعَدَ اللهُ إِبْرَاهِيمَ، إِذْ لَمْ يَكُنْ لَهُ أَعْظَمُ يُقْسِمُ بِهِ، أَقْسَمَ بِنَفْسِهِ،
\par 14 قَائِلاً: «إِنِّي لَأُبَارِكَنَّكَ بَرَكَةً وَأُكَثِّرَنَّكَ تَكْثِيراً».
\par 15 وَهَكَذَا إِذْ تَأَنَّى نَالَ الْمَوْعِدَ.
\par 16 فَإِنَّ النَّاسَ يُقْسِمُونَ بِالأَعْظَمِ، وَنِهَايَةُ كُلِّ مُشَاجَرَةٍ عِنْدَهُمْ لأَجْلِ التَّثْبِيتِ هِيَ الْقَسَمُ.
\par 17 فَلِذَلِكَ إِذْ أَرَادَ اللهُ أَنْ يُظْهِرَ أَكْثَرَ كَثِيراً لِوَرَثَةِ الْمَوْعِدِ عَدَمَ تَغَيُّرِ قَضَائِهِ، تَوَسَّطَ بِقَسَمٍ،
\par 18 حَتَّى بِأَمْرَيْنِ عَدِيمَيِ التَّغَيُّرِ، لاَ يُمْكِنُ أَنَّ اللهَ يَكْذِبُ فِيهِمَا، تَكُونُ لَنَا تَعْزِيَةٌ قَوِيَّةٌ، نَحْنُ الَّذِينَ الْتَجَأْنَا لِنُمْسِكَ بِالرَّجَاءِ الْمَوْضُوعِ أَمَامَنَا،
\par 19 الَّذِي هُوَ لَنَا كَمِرْسَاةٍ لِلنَّفْسِ مُؤْتَمَنَةٍ وَثَابِتَةٍ، تَدْخُلُ إِلَى مَا دَاخِلَ الْحِجَابِ،
\par 20 حَيْثُ دَخَلَ يَسُوعُ كَسَابِقٍ لأَجْلِنَا، صَائِراً عَلَى رُتْبَةِ مَلْكِي صَادَقَ، رَئِيسَ كَهَنَةٍ إِلَى الأَبَدِ.

\chapter{7}

\par 1 لأَنَّ مَلْكِي صَادِقَ هَذَا، مَلِكَ سَالِيمَ، كَاهِنَ اللهِ الْعَلِيِّ، الَّذِي اسْتَقْبَلَ إِبْرَاهِيمَ رَاجِعاً مِنْ كَسْرَةِ الْمُلُوكِ وَبَارَكَهُ،
\par 2 الَّذِي قَسَمَ لَهُ إِبْرَاهِيمُ عُشْراً مِنْ كُلِّ شَيْءٍ. الْمُتَرْجَمَ أَوَّلاً «مَلِكَ الْبِرِّ» ثُمَّ أَيْضاً «مَلِكَ سَالِيمَ» أَيْ مَلِكَ السَّلاَمِ
\par 3 بِلاَ أَبٍ بِلاَ أُمٍّ بِلاَ نَسَبٍ. لاَ بَدَاءَةَ أَيَّامٍ لَهُ وَلاَ نِهَايَةَ حَيَاةٍ. بَلْ هُوَ مُشَبَّهٌ بِابْنِ اللهِ. هَذَا يَبْقَى كَاهِناً إِلَى الأَبَدِ.
\par 4 ثُمَّ انْظُرُوا مَا أَعْظَمَ هَذَا الَّذِي أَعْطَاهُ إِبْرَاهِيمُ رَئِيسُ الآبَاءِ عُشْراً أَيْضاً مِنْ رَأْسِ الْغَنَائِمِ.
\par 5 وَأَمَّا الَّذِينَ هُمْ مِنْ بَنِي لاَوِي، الَّذِينَ يَأْخُذُونَ الْكَهَنُوتَ، فَلَهُمْ وَصِيَّةٌ أَنْ يُعَشِّرُوا الشَّعْبَ بِمُقْتَضَى النَّامُوسِ - أَيْ إِخْوَتَهُمْ، مَعَ أَنَّهُمْ قَدْ خَرَجُوا مِنْ صُلْبِ إِبْرَاهِيمَ.
\par 6 وَلَكِنَّ الَّذِي لَيْسَ لَهُ نَسَبٌ مِنْهُمْ قَدْ عَشَّرَ إِبْرَاهِيمَ، وَبَارَكَ الَّذِي لَهُ الْمَوَاعِيدُ!
\par 7 وَبِدُونِ كُلِّ مُشَاجَرَةٍ: الأَكْبَرُ يُبَارِكُ الأَصْغَرَ.
\par 8 وَهُنَا أُنَاسٌ مَائِتُونَ يَأْخُذُونَ عُشْراً، وَأَمَّا هُنَاكَ فَالْمَشْهُودُ لَهُ بِأَنَّهُ حَيٌّ.
\par 9 حَتَّى أَقُولُ كَلِمَةً: إِنَّ لاَوِي أَيْضاً الآخِذَ الأَعْشَارَ قَدْ عُشِّرَ بِإِبْرَاهِيمَ!
\par 10 لأَنَّهُ كَانَ بَعْدُ فِي صُلْبِ أَبِيهِ حِينَ اسْتَقْبَلَهُ مَلْكِي صَادِقَ.
\par 11 فَلَوْ كَانَ بِالْكَهَنُوتِ اللاَّوِيِّ كَمَالٌ - إِذِ الشَّعْبُ أَخَذَ النَّامُوسَ عَلَيْهِ - مَاذَا كَانَتِ الْحَاجَةُ بَعْدُ إِلَى أَنْ يَقُومَ كَاهِنٌ آخَرُ عَلَى رُتْبَةِ مَلْكِي صَادِقَ، وَلاَ يُقَالُ «عَلَى رُتْبَةِ هَارُونَ»؟
\par 12 لأَنَّهُ إِنْ تَغَيَّرَ الْكَهَنُوتُ فَبِالضَّرُورَةِ يَصِيرُ تَغَيُّرٌ لِلنَّامُوسِ أَيْضاً.
\par 13 لأَنَّ الَّذِي يُقَالُ عَنْهُ هَذَا كَانَ شَرِيكاً فِي سِبْطٍ آخَرَ لَمْ يُلاَزِمْ أَحَدٌ مِنْهُ الْمَذْبَحَ.
\par 14 فَإِنَّهُ وَاضِحٌ أَنَّ رَبَّنَا قَدْ طَلَعَ مِنْ سِبْطِ يَهُوذَا، الَّذِي لَمْ يَتَكَلَّمْ عَنْهُ مُوسَى شَيْئاً مِنْ جِهَةِ الْكَهَنُوتِ.
\par 15 وَذَلِكَ أَكْثَرُ وُضُوحاً أَيْضاً إِنْ كَانَ عَلَى شِبْهِ مَلْكِي صَادِقَ يَقُومُ كَاهِنٌ آخَرُ،
\par 16 قَدْ صَارَ لَيْسَ بِحَسَبِ نَامُوسِ وَصِيَّةٍ جَسَدِيَّةٍ، بَلْ بِحَسَبِ قُوَّةِ حَيَاةٍ لاَ تَزُولُ.
\par 17 لأَنَّهُ يَشْهَدُ أَنَّكَ «كَاهِنٌ إِلَى الأَبَدِ عَلَى رُتْبَةِ مَلْكِي صَادِقَ».
\par 18 فَإِنَّهُ يَصِيرُ إِبْطَالُ الْوَصِيَّةِ السَّابِقَةِ مِنْ أَجْلِ ضُعْفِهَا وَعَدَمِ نَفْعِهَا،
\par 19 إِذِ النَّامُوسُ لَمْ يُكَمِّلْ شَيْئاً. وَلَكِنْ يَصِيرُ إِدْخَالُ رَجَاءٍ أَفْضَلَ بِهِ نَقْتَرِبُ إِلَى اللهِ.
\par 20 وَعَلَى قَدْرِ مَا إِنَّهُ لَيْسَ بِدُونِ قَسَمٍ -
\par 21 لأَنَّ أُولَئِكَ بِدُونِ قَسَمٍ قَدْ صَارُوا كَهَنَةً، وَأَمَّا هَذَا فَبِقَسَمٍ مِنَ الْقَائِلِ لَهُ: «أَقْسَمَ الرَّبُّ وَلَنْ يَنْدَمَ، أَنْتَ كَاهِنٌ إِلَى الأَبَدِ عَلَى رُتْبَةِ مَلْكِي صَادِقَ».
\par 22 عَلَى قَدْرِ ذَلِكَ قَدْ صَارَ يَسُوعُ ضَامِناً لِعَهْدٍ أَفْضَلَ.
\par 23 وَأُولَئِكَ قَدْ صَارُوا كَهَنَةً كَثِيرِينَ لأَنَّ الْمَوْتَ مَنَعَهُمْ مِنَ الْبَقَاءِ،
\par 24 وَأَمَّا هَذَا فَلأَنَّهُ يَبْقَى إِلَى الأَبَدِ، لَهُ كَهَنُوتٌ لاَ يَزُولُ.
\par 25 فَمِنْ ثَمَّ يَقْدِرُ أَنْ يُخَلِّصَ أَيْضاً إِلَى التَّمَامِ الَّذِينَ يَتَقَدَّمُونَ بِهِ إِلَى اللهِ، إِذْ هُوَ حَيٌّ فِي كُلِّ حِينٍ لِيَشْفَعَ فِيهِمْ.
\par 26 لأَنَّهُ كَانَ يَلِيقُ بِنَا رَئِيسُ كَهَنَةٍ مِثْلُ هَذَا، قُدُّوسٌ بِلاَ شَرٍّ وَلاَ دَنَسٍ، قَدِ انْفَصَلَ عَنِ الْخُطَاةِ وَصَارَ أَعْلَى مِنَ السَّمَاوَاتِ
\par 27 الَّذِي لَيْسَ لَهُ اضْطِرَارٌ كُلَّ يَوْمٍ مِثْلُ رُؤَسَاءِ الْكَهَنَةِ أَنْ يُقَدِّمَ ذَبَائِحَ أَوَّلاً عَنْ خَطَايَا نَفْسِهِ ثُمَّ عَنْ خَطَايَا الشَّعْبِ، لأَنَّهُ فَعَلَ هَذَا مَرَّةً وَاحِدَةً، إِذْ قَدَّمَ نَفْسَهُ.
\par 28 فَإِنَّ النَّامُوسَ يُقِيمُ أُنَاساً بِهِمْ ضُعْفٌ رُؤَسَاءَ كَهَنَةٍ. وَأَمَّا كَلِمَةُ الْقَسَمِ الَّتِي بَعْدَ النَّامُوسِ فَتُقِيمُ ابْناً مُكَمَّلاً إِلَى الأَبَدِ.

\chapter{8}

\par 1 وَأَمَّا رَأْسُ الْكَلاَمِ فَهُوَ أَنَّ لَنَا رَئِيسَ كَهَنَةٍ مِثْلَ هَذَا ،قَدْ جَلَسَ فِي يَمِينِ عَرْشِ الْعَظَمَةِ فِي السَّمَاوَاتِ
\par 2 خَادِماً لِلأَقْدَاسِ وَالْمَسْكَنِ الْحَقِيقِيِّ الَّذِي نَصَبَهُ الرَّبُّ لاَ إِنْسَانٌ.
\par 3 لأَنَّ كُلَّ رَئِيسِ كَهَنَةٍ يُقَامُ لِكَيْ يُقَدِّمَ قَرَابِينَ وَذَبَائِحَ. فَمِنْ ثَمَّ يَلْزَمُ أَنْ يَكُونَ لِهَذَا أَيْضاً شَيْءٌ يُقَدِّمُهُ.
\par 4 فَإِنَّهُ لَوْ كَانَ عَلَى الأَرْضِ لَمَا كَانَ كَاهِناً، إِذْ يُوجَدُ الْكَهَنَةُ الَّذِينَ يُقَدِّمُونَ قَرَابِينَ حَسَبَ النَّامُوسِ،
\par 5 الَّذِينَ يَخْدِمُونَ شِبْهَ السَّمَاوِيَّاتِ وَظِلَّهَا، كَمَا أُوحِيَ إِلَى مُوسَى وَهُوَ مُزْمِعٌ أَنْ يَصْنَعَ الْمَسْكَنَ. لأَنَّهُ قَالَ: «انْظُرْ أَنْ تَصْنَعَ كُلَّ شَيْءٍ حَسَبَ الْمِثَالِ الَّذِي أُظْهِرَ لَكَ فِي الْجَبَلِ».
\par 6 وَلَكِنَّهُ الآنَ قَدْ حَصَلَ عَلَى خِدْمَةٍ أَفْضَلَ بِمِقْدَارِ مَا هُوَ وَسِيطٌ أَيْضاً لِعَهْدٍ أَعْظَمَ، قَدْ تَثَبَّتَ عَلَى مَوَاعِيدَ أَفْضَلَ.
\par 7 فَإِنَّهُ لَوْ كَانَ ذَلِكَ الأَوَّلُ بِلاَ عَيْبٍ لَمَا طُلِبَ مَوْضِعٌ لِثَانٍ.
\par 8 لأَنَّهُ يَقُولُ لَهُمْ لاَئِماً: «هُوَذَا أَيَّامٌ تَأْتِي يَقُولُ الرَّبُّ، حِينَ أُكَمِّلُ مَعَ بَيْتِ إِسْرَائِيلَ وَمَعَ بَيْتِ يَهُوذَا عَهْداً جَدِيداً.
\par 9 لاَ كَالْعَهْدِ الَّذِي عَمِلْتُهُ مَعَ آبَائِهِمْ يَوْمَ أَمْسَكْتُ بِيَدِهِمْ لِأُخْرِجَهُمْ مِنْ أَرْضِ مِصْرَ، لأَنَّهُمْ لَمْ يَثْبُتُوا فِي عَهْدِي، وَأَنَا أَهْمَلْتُهُمْ يَقُولُ الرَّبُّ.
\par 10 لأَنَّ هَذَا هُوَ الْعَهْدُ الَّذِي أَعْهَدُهُ مَعَ بَيْتِ إِسْرَائِيلَ بَعْدَ تِلْكَ الأَيَّامِ يَقُولُ الرَّبُّ: أَجْعَلُ نَوَامِيسِي فِي أَذْهَانِهِمْ، وَأَكْتُبُهَا عَلَى قُلُوبِهِمْ، وَأَنَا أَكُونُ لَهُمْ إِلَهاً وَهُمْ يَكُونُونَ لِي شَعْباً.
\par 11 وَلاَ يُعَلِّمُونَ كُلُّ وَاحِدٍ قَرِيبَهُ وَكُلُّ وَاحِدٍ أَخَاهُ قَائِلاً: اعْرِفِ الرَّبَّ، لأَنَّ الْجَمِيعَ سَيَعْرِفُونَنِي مِنْ صَغِيرِهِمْ إِلَى كَبِيرِهِمْ.
\par 12 لأَنِّي أَكُونُ صَفُوحاً عَنْ آثَامِهِمْ، وَلاَ أَذْكُرُ خَطَايَاهُمْ وَتَعَدِّيَاتِهِمْ فِي مَا بَعْدُ».
\par 13 فَإِذْ قَالَ «جَدِيداً» عَتَّقَ الأَوَّلَ. وَأَمَّا مَا عَتَقَ وَشَاخَ فَهُوَ قَرِيبٌ مِنَ الاِضْمِحْلاَلِ.

\chapter{9}

\par 1 ثُمَّ الْعَهْدُ الأَوَّلُ كَانَ لَهُ أَيْضاً فَرَائِضُ خِدْمَةٍ وَالْقُدْسُ الْعَالَمِيُّ،
\par 2 لأَنَّهُ نُصِبَ الْمَسْكَنُ الأَوَّلُ الَّذِي يُقَالُ لَهُ «الْقُدْسُ» الَّذِي كَانَ فِيهِ الْمَنَارَةُ، وَالْمَائِدَةُ، وَخُبْزُ التَّقْدِمَةِ.
\par 3 وَوَرَاءَ الْحِجَابِ الثَّانِي الْمَسْكَنُ الَّذِي يُقَالُ لَهُ «قُدْسُ الأَقْدَاسِ»
\par 4 فِيهِ مِبْخَرَةٌ مِنْ ذَهَبٍ، وَتَابُوتُ الْعَهْدِ مُغَشًّى مِنْ كُلِّ جِهَةٍ بِالذَّهَبِ، الَّذِي فِيهِ قِسْطٌ مِنْ ذَهَبٍ فِيهِ الْمَنُّ، وَعَصَا هَارُونَ الَّتِي أَفْرَخَتْ، وَلَوْحَا الْعَهْدِ.
\par 5 وَفَوْقَهُ كَرُوبَا الْمَجْدِ مُظَلِّلَيْنِ الْغِطَاءَ. أَشْيَاءُ لَيْسَ لَنَا الآنَ أَنْ نَتَكَلَّمَ عَنْهَا بِالتَّفْصِيلِ.
\par 6 ثُمَّ إِذْ صَارَتْ هَذِهِ مُهَيَّأَةً هَكَذَا، يَدْخُلُ الْكَهَنَةُ إِلَى الْمَسْكَنِ الأَوَّلِ كُلَّ حِينٍ، صَانِعِينَ الْخِدْمَةَ.
\par 7 وَأَمَّا إِلَى الثَّانِي فَرَئِيسُ الْكَهَنَةِ فَقَطْ مَرَّةً فِي السَّنَةِ، لَيْسَ بِلاَ دَمٍ يُقَدِّمُهُ عَنْ نَفْسِهِ وَعَنْ جَهَالاَتِ الشَّعْبِ،
\par 8 مُعْلِناً الرُّوحُ الْقُدُسُ بِهَذَا أَنَّ طَرِيقَ الأَقْدَاسِ لَمْ يُظْهَرْ بَعْدُ، مَا دَامَ الْمَسْكَنُ الأَوَّلُ لَهُ إِقَامَةٌ،
\par 9 الَّذِي هُوَ رَمْزٌ لِلْوَقْتِ الْحَاضِرِ، الَّذِي فِيهِ تُقَدَّمُ قَرَابِينُ وَذَبَائِحُ لاَ يُمْكِنُ مِنْ جِهَةِ الضَّمِيرِ أَنْ تُكَمِّلَ الَّذِي يَخْدِمُ،
\par 10 وَهِيَ قَائِمَةٌ بِأَطْعِمَةٍ وَأَشْرِبَةٍ وَغَسَلاَتٍ مُخْتَلِفَةٍ وَفَرَائِضَ جَسَدِيَّةٍ فَقَطْ، مَوْضُوعَةٍ إِلَى وَقْتِ الإِصْلاَحِ.
\par 11 وَأَمَّا الْمَسِيحُ، وَهُوَ قَدْ جَاءَ رَئِيسَ كَهَنَةٍ لِلْخَيْرَاتِ الْعَتِيدَةِ، فَبِالْمَسْكَنِ الأَعْظَمِ وَالأَكْمَلِ، غَيْرِ الْمَصْنُوعِ بِيَدٍ، أَيِ الَّذِي لَيْسَ مِنْ هَذِهِ الْخَلِيقَةِ.
\par 12 وَلَيْسَ بِدَمِ تُيُوسٍ وَعُجُولٍ، بَلْ بِدَمِ نَفْسِهِ، دَخَلَ مَرَّةً وَاحِدَةً إِلَى الأَقْدَاسِ، فَوَجَدَ فِدَاءً أَبَدِيّاً.
\par 13 لأَنَّهُ إِنْ كَانَ دَمُ ثِيرَانٍ وَتُيُوسٍ وَرَمَادُ عِجْلَةٍ مَرْشُوشٌ عَلَى الْمُنَجَّسِينَ يُقَدِّسُ إِلَى طَهَارَةِ الْجَسَدِ،
\par 14 فَكَمْ بِالْحَرِيِّ يَكُونُ دَمُ الْمَسِيحِ، الَّذِي بِرُوحٍ أَزَلِيٍّ قَدَّمَ نَفْسَهُ لِلَّهِ بِلاَ عَيْبٍ، يُطَهِّرُ ضَمَائِرَكُمْ مِنْ أَعْمَالٍ مَيِّتَةٍ لِتَخْدِمُوا اللهَ الْحَيَّ!
\par 15 وَلأَجْلِ هَذَا هُوَ وَسِيطُ عَهْدٍ جَدِيدٍ، لِكَيْ يَكُونَ الْمَدْعُّوُونَ - إِذْ صَارَ مَوْتٌ لِفِدَاءِ التَّعَدِّيَاتِ الَّتِي فِي الْعَهْدِ الأَوَّلِ - يَنَالُونَ وَعْدَ الْمِيرَاثِ الأَبَدِيِّ.
\par 16 لأَنَّهُ حَيْثُ تُوجَدُ وَصِيَّةٌ يَلْزَمُ بَيَانُ مَوْتِ الْمُوصِي.
\par 17 لأَنَّ الْوَصِيَّةَ ثَابِتَةٌ عَلَى الْمَوْتَى، إِذْ لاَ قُوَّةَ لَهَا الْبَتَّةَ مَا دَامَ الْمُوصِي حَيّاً.
\par 18 فَمِنْ ثَمَّ الأَوَّلُ أَيْضاً لَمْ يُكَرَّسْ بِلاَ دَمٍ،
\par 19 لأَنَّ مُوسَى بَعْدَمَا كَلَّمَ جَمِيعَ الشَّعْبِ بِكُلِّ وَصِيَّةٍ بِحَسَبِ النَّامُوسِ، أَخَذَ دَمَ الْعُجُولِ وَالتُّيُوسِ، مَعَ مَاءٍ وَصُوفاً قِرْمِزِيّاً وَزُوفَا، وَرَشَّ الْكِتَابَ نَفْسَهُ وَجَمِيعَ الشَّعْبِ،
\par 20 قَائِلاً: «هَذَا هُوَ دَمُ الْعَهْدِ الَّذِي أَوْصَاكُمُ اللهُ بِهِ».
\par 21 وَالْمَسْكَنَ أَيْضاً وَجَمِيعَ آنِيَةِ الْخِدْمَةِ رَشَّهَا كَذَلِكَ بِالدَّمِ.
\par 22 وَكُلُّ شَيْءٍ تَقْرِيباً يَتَطَهَّرُ حَسَبَ النَّامُوسِ بِالدَّمِ، وَبِدُونِ سَفْكِ دَمٍ لاَ تَحْصُلُ مَغْفِرَةٌ!
\par 23 فَكَانَ يَلْزَمُ أَنَّ أَمْثِلَةَ الأَشْيَاءِ الَّتِي فِي السَّمَاوَاتِ تُطَهَّرُ بِهَذِهِ، وَأَمَّا السَّمَاوِيَّاتُ عَيْنُهَا فَبِذَبَائِحَ أَفْضَلَ مِنْ هَذِهِ.
\par 24 لأَنَّ الْمَسِيحَ لَمْ يَدْخُلْ إِلَى أَقْدَاسٍ مَصْنُوعَةٍ بِيَدٍ أَشْبَاهِ الْحَقِيقِيَّةِ، بَلْ إِلَى السَّمَاءِ عَيْنِهَا، لِيَظْهَرَ الآنَ أَمَامَ وَجْهِ اللهِ لأَجْلِنَا.
\par 25 وَلاَ لِيُقَدِّمَ نَفْسَهُ مِرَاراً كَثِيرَةً، كَمَا يَدْخُلُ رَئِيسُ الْكَهَنَةِ إِلَى الأَقْدَاسِ كُلَّ سَنَةٍ بِدَمِ آخَرَ.
\par 26 فَإِذْ ذَاكَ كَانَ يَجِبُ أَنْ يَتَأَلَّمَ مِرَاراً كَثِيرَةً مُنْذُ تَأْسِيسِ الْعَالَمِ، وَلَكِنَّهُ الآنَ قَدْ أُظْهِرَ مَرَّةً عِنْدَ انْقِضَاءِ الدُّهُورِ لِيُبْطِلَ الْخَطِيَّةَ بِذَبِيحَةِ نَفْسِهِ.
\par 27 وَكَمَا وُضِعَ لِلنَّاسِ أَنْ يَمُوتُوا مَرَّةً ثُمَّ بَعْدَ ذَلِكَ الدَّيْنُونَةُ،
\par 28 هَكَذَا الْمَسِيحُ أَيْضاً، بَعْدَمَا قُدِّمَ مَرَّةً لِكَيْ يَحْمِلَ خَطَايَا كَثِيرِينَ، سَيَظْهَرُ ثَانِيَةً بِلاَ خَطِيَّةٍ لِلْخَلاَصِ لِلَّذِينَ يَنْتَظِرُونَهُ.

\chapter{10}

\par 1 لأَنَّ النَّامُوسَ، إِذْ لَهُ ظِلُّ الْخَيْرَاتِ الْعَتِيدَةِ لاَ نَفْسُ صُورَةِ الأَشْيَاءِ، لاَ يَقْدِرُ أَبَداً بِنَفْسِ الذَّبَائِحِ كُلَّ سَنَةٍ، الَّتِي يُقَدِّمُونَهَا عَلَى الدَّوَامِ، أَنْ يُكَمِّلَ الَّذِينَ يَتَقَدَّمُونَ.
\par 2 وَإِلاَّ، أَفَمَا زَالَتْ تُقَدَّمُ؟ مِنْ أَجْلِ أَنَّ الْخَادِمِينَ، وَهُمْ مُطَهَّرُونَ مَرَّةً، لاَ يَكُونُ لَهُمْ أَيْضاً ضَمِيرُ خَطَايَا.
\par 3 لَكِنْ فِيهَا كُلَّ سَنَةٍ ذِكْرُ خَطَايَا.
\par 4 لأَنَّهُ لاَ يُمْكِنُ أَنَّ دَمَ ثِيرَانٍ وَتُيُوسٍ يَرْفَعُ خَطَايَا.
\par 5 لِذَلِكَ عِنْدَ دُخُولِهِ إِلَى الْعَالَمِ يَقُولُ: «ذَبِيحَةً وَقُرْبَاناً لَمْ تُرِدْ، وَلَكِنْ هَيَّأْتَ لِي جَسَداً.
\par 6 بِمُحْرَقَاتٍ وَذَبَائِحَ لِلْخَطِيَّةِ لَمْ تُسَرَّ.
\par 7 ثُمَّ قُلْتُ: هَئَنَذَا أَجِيءُ. فِي دَرْجِ الْكِتَابِ مَكْتُوبٌ عَنِّي، لأَفْعَلَ مَشِيئَتَكَ يَا أَللهُ».
\par 8 إِذْ يَقُولُ آنِفاً: «إِنَّكَ ذَبِيحَةً وَقُرْبَاناً وَمُحْرَقَاتٍ وَذَبَائِحَ لِلْخَطِيَّةِ لَمْ تُرِدْ وَلاَ سُرِرْتَ بِهَا». الَّتِي تُقَدَّمُ حَسَبَ النَّامُوسِ.
\par 9 ثُمَّ قَالَ: «هَئَنَذَا أَجِيءُ لأَفْعَلَ مَشِيئَتَكَ يَا أَللهُ». يَنْزِعُ الأَوَّلَ لِكَيْ يُثَبِّتَ الثَّانِيَ.
\par 10 فَبِهَذِهِ الْمَشِيئَةِ نَحْنُ مُقَدَّسُونَ بِتَقْدِيمِ جَسَدِ يَسُوعَ الْمَسِيحِ مَرَّةً وَاحِدَةً.
\par 11 وَكُلُّ كَاهِنٍ يَقُومُ كُلَّ يَوْمٍ يَخْدِمُ وَيُقَدِّمُ مِرَاراً كَثِيرَةً تِلْكَ الذَّبَائِحَ عَيْنَهَا، الَّتِي لاَ تَسْتَطِيعُ الْبَتَّةَ أَنْ تَنْزِعَ الْخَطِيَّةَ.
\par 12 وَأَمَّا هَذَا فَبَعْدَمَا قَدَّمَ عَنِ الْخَطَايَا ذَبِيحَةً وَاحِدَةً، جَلَسَ إِلَى الأَبَدِ عَنْ يَمِينِ اللهِ،
\par 13 مُنْتَظِراً بَعْدَ ذَلِكَ حَتَّى تُوضَعَ أَعْدَاؤُهُ مَوْطِئاً لِقَدَمَيْهِ.
\par 14 لأَنَّهُ بِقُرْبَانٍ وَاحِدٍ قَدْ أَكْمَلَ إِلَى الأَبَدِ الْمُقَدَّسِينَ.
\par 15 وَيَشْهَدُ لَنَا الرُّوحُ الْقُدُسُ أَيْضاً. لأَنَّهُ بَعْدَمَا قَالَ سَابِقاً:
\par 16 «هَذَا هُوَ الْعَهْدُ الَّذِي أَعْهَدُهُ مَعَهُمْ بَعْدَ تِلْكَ الأَيَّامِ، يَقُولُ الرَّبُّ، أَجْعَلُ نَوَامِيسِي فِي قُلُوبِهِمْ وَأَكْتُبُهَا فِي أَذْهَانِهِمْ»
\par 17 وَ: «لَنْ أَذْكُرَ خَطَايَاهُمْ وَتَعَدِّيَاتِهِمْ فِي مَا بَعْدُ».
\par 18 وَإِنَّمَا حَيْثُ تَكُونُ مَغْفِرَةٌ لِهَذِهِ لاَ يَكُونُ بَعْدُ قُرْبَانٌ عَنِ الْخَطِيَّةِ.
\par 19 فَإِذْ لَنَا أَيُّهَا الإِخْوَةُ ثِقَةٌ بِالدُّخُولِ إِلَى «الأَقْدَاسِ» بِدَمِ يَسُوعَ،
\par 20 طَرِيقاً كَرَّسَهُ لَنَا حَدِيثاً حَيّاً، بِالْحِجَابِ، أَيْ جَسَدِهِ،
\par 21 وَكَاهِنٌ عَظِيمٌ عَلَى بَيْتِ اللهِ،
\par 22 لِنَتَقَدَّمْ بِقَلْبٍ صَادِقٍ فِي يَقِينِ الإِيمَانِ، مَرْشُوشَةً قُلُوبُنَا مِنْ ضَمِيرٍ شِرِّيرٍ، وَمُغْتَسِلَةً أَجْسَادُنَا بِمَاءٍ نَقِيٍّ.
\par 23 لِنَتَمَسَّكْ بِإِقْرَارِ الرَّجَاءِ رَاسِخاً، لأَنَّ الَّذِي وَعَدَ هُوَ أَمِينٌ.
\par 24 وَلْنُلاَحِظْ بَعْضُنَا بَعْضاً لِلتَّحْرِيضِ عَلَى الْمَحَبَّةِ وَالأَعْمَالِ الْحَسَنَةِ،
\par 25 غَيْرَ تَارِكِينَ اجْتِمَاعَنَا كَمَا لِقَوْمٍ عَادَةٌ، بَلْ وَاعِظِينَ بَعْضُنَا بَعْضاً، وَبِالأَكْثَرِ عَلَى قَدْرِ مَا تَرَوْنَ الْيَوْمَ يَقْرُبُ،
\par 26 فَإِنَّهُ إِنْ أَخْطَأْنَا بِاخْتِيَارِنَا بَعْدَمَا أَخَذْنَا مَعْرِفَةَ الْحَقِّ، لاَ تَبْقَى بَعْدُ ذَبِيحَةٌ عَنِ الْخَطَايَا،
\par 27 بَلْ قُبُولُ دَيْنُونَةٍ مُخِيفٌ، وَغَيْرَةُ نَارٍ عَتِيدَةٍ أَنْ تَأْكُلَ الْمُضَادِّينَ.
\par 28 مَنْ خَالَفَ نَامُوسَ مُوسَى فَعَلَى شَاهِدَيْنِ أَوْ ثَلاَثَةِ شُهُودٍ يَمُوتُ بِدُونِ رَأْفَةٍ.
\par 29 فَكَمْ عِقَاباً أَشَرَّ تَظُنُّونَ أَنَّهُ يُحْسَبُ مُسْتَحِقّاً مَنْ دَاسَ ابْنَ اللهِ، وَحَسِبَ دَمَ الْعَهْدِ الَّذِي قُدِّسَ بِهِ دَنِساً، وَازْدَرَى بِرُوحِ النِّعْمَةِ؟
\par 30 فَإِنَّنَا نَعْرِفُ الَّذِي قَالَ: «لِيَ الاِنْتِقَامُ، أَنَا أُجَازِي، يَقُولُ الرَّبُّ». وَأَيْضاً: «الرَّبُّ يَدِينُ شَعْبَهُ».
\par 31 مُخِيفٌ هُوَ الْوُقُوعُ فِي يَدَيِ اللهِ الْحَيِّ!
\par 32 وَلَكِنْ تَذَكَّرُوا الأَيَّامَ السَّالِفَةَ الَّتِي فِيهَا بَعْدَمَا أُنِرْتُمْ صَبِرْتُمْ عَلَى مُجَاهَدَةِ آلاَمٍ كَثِيرَةٍ.
\par 33 مِنْ جِهَةٍ مَشْهُورِينَ بِتَعْيِيرَاتٍ وَضِيقَاتٍ، وَمِنْ جِهَةٍ صَائِرِينَ شُرَكَاءَ الَّذِينَ تُصُرِّفَ فِيهِمْ هَكَذَا.
\par 34 لأَنَّكُمْ رَثَيْتُمْ لِقُيُودِي أَيْضاً، وَقَبِلْتُمْ سَلْبَ أَمْوَالِكُمْ بِفَرَحٍ، عَالِمِينَ فِي أَنْفُسِكُمْ أَنَّ لَكُمْ مَالاً أَفْضَلَ فِي السَّمَاوَاتِ وَبَاقِياً.
\par 35 فَلاَ تَطْرَحُوا ثِقَتَكُمُ الَّتِي لَهَا مُجَازَاةٌ عَظِيمَةٌ.
\par 36 لأَنَّكُمْ تَحْتَاجُونَ إِلَى الصَّبْرِ، حَتَّى إِذَا صَنَعْتُمْ مَشِيئَةَ اللهِ تَنَالُونَ الْمَوْعِدَ.
\par 37 لأَنَّهُ بَعْدَ قَلِيلٍ جِدّاً «سَيَأْتِي الآتِي وَلاَ يُبْطِئُ.
\par 38 أَمَّا الْبَارُّ فَبِالإِيمَانِ يَحْيَا، وَإِنِ ارْتَدَّ لاَ تُسَرَُّ بِهِ نَفْسِي».
\par 39 وَأَمَّا نَحْنُ فَلَسْنَا مِنَ الاِرْتِدَادِ لِلْهَلاَكِ، بَلْ مِنَ الإِيمَانِ لاِقْتِنَاءِ النَّفْسِ.

\chapter{11}

\par 1 وَأَمَّا الإِيمَانُ فَهُوَ الثِّقَةُ بِمَا يُرْجَى وَالإِيقَانُ بِأُمُورٍ لاَ تُرَى.
\par 2 فَإِنَّهُ فِي هَذَا شُهِدَ لِلْقُدَمَاءِ.
\par 3 بِالإِيمَانِ نَفْهَمُ أَنَّ الْعَالَمِينَ أُتْقِنَتْ بِكَلِمَةِ اللهِ، حَتَّى لَمْ يَتَكَوَّنْ مَا يُرَى مِمَّا هُوَ ظَاهِرٌ.
\par 4 بِالإِيمَانِ قَدَّمَ هَابِيلُ لِلَّهِ ذَبِيحَةً أَفْضَلَ مِنْ قَايِينَ، فَبِهِ شُهِدَ لَهُ أَنَّهُ بَارٌّ، إِذْ شَهِدَ اللهُ لِقَرَابِينِهِ. وَبِهِ، وَإِنْ مَاتَ، يَتَكَلَّمْ بَعْدُ!
\par 5 بِالإِيمَانِ نُقِلَ أَخْنُوخُ لِكَيْ لاَ يَرَى الْمَوْتَ، وَلَمْ يُوجَدْ لأَنَّ اللهَ نَقَلَهُ - إِذْ قَبْلَ نَقْلِهِ شُهِدَ لَهُ بِأَنَّهُ قَدْ أَرْضَى اللهَ.
\par 6 وَلَكِنْ بِدُونِ إِيمَانٍ لاَ يُمْكِنُ إِرْضَاؤُهُ، لأَنَّهُ يَجِبُ أَنَّ الَّذِي يَأْتِي إِلَى اللهِ يُؤْمِنُ بِأَنَّهُ مَوْجُودٌ، وَأَنَّهُ يُجَازِي الَّذِينَ يَطْلُبُونَهُ.
\par 7 بِالإِيمَانِ نُوحٌ لَمَّا أُوحِيَ إِلَيْهِ عَنْ أُمُورٍ لَمْ تُرَ بَعْدُ خَافَ، فَبَنَى فُلْكاً لِخَلاَصِ بَيْتِهِ، فَبِهِ دَانَ الْعَالَمَ، وَصَارَ وَارِثاً لِلْبِرِّ الَّذِي حَسَبَ الإِيمَانِ.
\par 8 بِالإِيمَانِ إِبْرَاهِيمُ لَمَّا دُعِيَ أَطَاعَ أَنْ يَخْرُجَ إِلَى الْمَكَانِ الَّذِي كَانَ عَتِيداً أَنْ يَأْخُذَهُ مِيرَاثاً، فَخَرَجَ وَهُوَ لاَ يَعْلَمُ إِلَى أَيْنَ يَأْتِي.
\par 9 بِالإِيمَانِ تَغَرَّبَ فِي أَرْضِ الْمَوْعِدِ كَأَنَّهَا غَرِيبَةٌ، سَاكِناً فِي خِيَامٍ مَعَ إِسْحَاقَ وَيَعْقُوبَ الْوَارِثَيْنِ مَعَهُ لِهَذَا الْمَوْعِدِ عَيْنِهِ.
\par 10 لأَنَّهُ كَانَ يَنْتَظِرُ الْمَدِينَةَ الَّتِي لَهَا الأَسَاسَاتُ، الَّتِي صَانِعُهَا وَبَارِئُهَا اللهُ.
\par 11 بِالإِيمَانِ سَارَةُ نَفْسُهَا أَيْضاً أَخَذَتْ قُدْرَةً عَلَى إِنْشَاءِ نَسْلٍ، وَبَعْدَ وَقْتِ السِّنِّ وَلَدَتْ، إِذْ حَسِبَتِ الَّذِي وَعَدَ صَادِقاً.
\par 12 لِذَلِكَ وُلِدَ أَيْضاً مِنْ وَاحِدٍ، وَذَلِكَ مِنْ مُمَاتٍ، مِثْلُ نُجُومِ السَّمَاءِ فِي الْكَثْرَةِ، وَكَالرَّمْلِ الَّذِي عَلَى شَاطِئِ الْبَحْرِ الَّذِي لاَ يُعَدُّ.
\par 13 فِي الإِيمَانِ مَاتَ هَؤُلاَءِ أَجْمَعُونَ، وَهُمْ لَمْ يَنَالُوا الْمَوَاعِيدَ، بَلْ مِنْ بَعِيدٍ نَظَرُوهَا وَصَدَّقُوهَا وَحَيُّوهَا، وَأَقَرُّوا بِأَنَّهُمْ غُرَبَاءُ وَنُزَلاَءُ عَلَى الأَرْضِ.
\par 14 فَإِنَّ الَّذِينَ يَقُولُونَ مِثْلَ هَذَا يُظْهِرُونَ أَنَّهُمْ يَطْلُبُونَ وَطَناً.
\par 15 فَلَوْ ذَكَرُوا ذَلِكَ الَّذِي خَرَجُوا مِنْهُ، لَكَانَ لَهُمْ فُرْصَةٌ لِلرُّجُوعِ.
\par 16 وَلَكِنِ الآنَ يَبْتَغُونَ وَطَناً أَفْضَلَ، أَيْ سَمَاوِيّاً. لِذَلِكَ لاَ يَسْتَحِي بِهِمِ اللهُ أَنْ يُدْعَى إِلَهَهُمْ، لأَنَّهُ أَعَدَّ لَهُمْ مَدِينَةً.
\par 17 بِالإِيمَانِ قَدَّمَ إِبْرَاهِيمُ إِسْحَاقَ وَهُوَ مُجَرَّبٌ - قَدَّمَ الَّذِي قَبِلَ الْمَوَاعِيدَ، وَحِيدَهُ
\par 18 الَّذِي قِيلَ لَهُ: «إِنَّهُ بِإِسْحَاقَ يُدْعَى لَكَ نَسْلٌ».
\par 19 إِذْ حَسِبَ أَنَّ اللهَ قَادِرٌ عَلَى الإِقَامَةِ مِنَ الأَمْوَاتِ أَيْضاً، الَّذِينَ مِنْهُمْ أَخَذَهُ أَيْضاً فِي مِثَالٍ.
\par 20 بِالإِيمَانِ إِسْحَاقُ بَارَكَ يَعْقُوبَ وَعِيسُو مِنْ جِهَةِ أُمُورٍ عَتِيدَةٍ.
\par 21 بِالإِيمَانِ يَعْقُوبُ عِنْدَ مَوْتِهِ بَارَكَ كُلَّ وَاحِدٍ مِنِ ابْنَيْ يُوسُفَ، وَسَجَدَ عَلَى رَأْسِ عَصَاهُ.
\par 22 بِالإِيمَانِ يُوسُفُ عِنْدَ مَوْتِهِ ذَكَرَ خُرُوجَ بَنِي إِسْرَائِيلَ وَأَوْصَى مِنْ جِهَةِ عِظَامِهِ.
\par 23 بِالإِيمَانِ مُوسَى، بَعْدَمَا وُلِدَ، أَخْفَاهُ أَبَوَاهُ ثَلاَثَةَ أَشْهُرٍ، لأَنَّهُمَا رَأَيَا الصَّبِيَّ جَمِيلاً، وَلَمْ يَخْشَيَا أَمْرَ الْمَلِكِ.
\par 24 بِالإِيمَانِ مُوسَى لَمَّا كَبِرَ أَبَى أَنْ يُدْعَى ابْنَ ابْنَةِ فِرْعَوْنَ،
\par 25 مُفَضِّلاً بِالأَحْرَى أَنْ يُذَلَّ مَعَ شَعْبِ اللهِ عَلَى أَنْ يَكُونَ لَهُ تَمَتُّعٌ وَقْتِيٌّ بِالْخَطِيَّةِ،
\par 26 حَاسِباً عَارَ الْمَسِيحِ غِنًى أَعْظَمَ مِنْ خَزَائِنِ مِصْرَ، لأَنَّهُ كَانَ يَنْظُرُ إِلَى الْمُجَازَاةِ.
\par 27 بِالإِيمَانِ تَرَكَ مِصْرَ غَيْرَ خَائِفٍ مِنْ غَضَبِ الْمَلِكِ، لأَنَّهُ تَشَدَّدَ، كَأَنَّهُ يَرَى مَنْ لاَ يُرَى.
\par 28 بِالإِيمَانِ صَنَعَ الْفِصْحَ وَرَشَّ الدَّمَ لِئَلاَّ يَمَسَّهُمُ الَّذِي أَهْلَكَ الأَبْكَارَ.
\par 29 بِالإِيمَانِ اجْتَازُوا فِي الْبَحْرِ الأَحْمَرِ كَمَا فِي الْيَابِسَةِ، الأَمْرُ الَّذِي لَمَّا شَرَعَ فِيهِ الْمِصْرِيُّونَ غَرِقُوا.
\par 30 بِالإِيمَانِ سَقَطَتْ أَسْوَارُ أَرِيحَا بَعْدَمَا طِيفَ حَوْلَهَا سَبْعَةَ أَيَّامٍ.
\par 31 بِالإِيمَانِ رَاحَابُ الزَّانِيَةُ لَمْ تَهْلِكْ مَعَ الْعُصَاةِ، إِذْ قَبِلَتِ الْجَاسُوسَيْنِ بِسَلاَمٍ.
\par 32 وَمَاذَا أَقُولُ أَيْضاً؟ لأَنَّهُ يُعْوِزُنِي الْوَقْتُ إِنْ أَخْبَرْتُ عَنْ جِدْعُونَ، وَبَارَاقَ، وَشَمْشُونَ، وَيَفْتَاحَ، وَدَاوُدَ، وَصَمُوئِيلَ، وَالأَنْبِيَاءِ،
\par 33 الَّذِينَ بِالإِيمَانِ قَهَرُوا مَمَالِكَ، صَنَعُوا بِرّاً، نَالُوا مَوَاعِيدَ، سَدُّوا أَفْوَاهَ أُسُودٍ،
\par 34 أَطْفَأُوا قُوَّةَ النَّارِ، نَجَوْا مِنْ حَدِّ السَّيْفِ، تَقَّوُوا مِنْ ضُعْفٍ، صَارُوا أَشِدَّاءَ فِي الْحَرْبِ، هَزَمُوا جُيُوشَ غُرَبَاءَ،
\par 35 أَخَذَتْ نِسَاءٌ أَمْوَاتَهُنَّ بِقِيَامَةٍ. وَآخَرُونَ عُذِّبُوا وَلَمْ يَقْبَلُوا النَّجَاةَ لِكَيْ يَنَالُوا قِيَامَةً أَفْضَلَ.
\par 36 وَآخَرُونَ تَجَرَّبُوا فِي هُزُءٍ وَجَلْدٍ، ثُمَّ فِي قُيُودٍ أَيْضاً وَحَبْسٍ.
\par 37 رُجِمُوا، نُشِرُوا، جُرِّبُوا، مَاتُوا قَتْلاً بِالسَّيْفِ، طَافُوا فِي جُلُودِ غَنَمٍ وَجُلُودِ مِعْزَى، مُعْتَازِينَ مَكْرُوبِينَ مُذَلِّينَ،
\par 38 وَهُمْ لَمْ يَكُنِ الْعَالَمُ مُسْتَحِقّاً لَهُمْ. تَائِهِينَ فِي بَرَارِيَّ وَجِبَالٍ وَمَغَايِرَ وَشُقُوقِ الأَرْضِ.
\par 39 فَهَؤُلاَءِ كُلُّهُمْ، مَشْهُوداً لَهُمْ بِالإِيمَانِ، لَمْ يَنَالُوا الْمَوْعِدَ،
\par 40 إِذْ سَبَقَ اللهُ فَنَظَرَ لَنَا شَيْئاً أَفْضَلَ، لِكَيْ لاَ يُكْمَلُوا بِدُونِنَا.

\chapter{12}

\par 1 لِذَلِكَ نَحْنُ أَيْضاً إِذْ لَنَا سَحَابَةٌ مِنَ الشُّهُودِ مِقْدَارُ هَذِهِ مُحِيطَةٌ بِنَا، لِنَطْرَحْ كُلَّ ثِقْلٍ وَالْخَطِيَّةَ الْمُحِيطَةَ بِنَا بِسُهُولَةٍ، وَلْنُحَاضِرْ بِالصَّبْرِ فِي الْجِهَادِ الْمَوْضُوعِ أَمَامَنَا،
\par 2 نَاظِرِينَ إِلَى رَئِيسِ الإِيمَانِ وَمُكَمِّلِهِ يَسُوعَ، الَّذِي مِنْ أَجْلِ السُّرُورِ الْمَوْضُوعِ أَمَامَهُ احْتَمَلَ الصَّلِيبَ مُسْتَهِيناً بِالْخِزْيِ، فَجَلَسَ فِي يَمِينِ عَرْشِ اللهِ.
\par 3 فَتَفَكَّرُوا فِي الَّذِي احْتَمَلَ مِنَ الْخُطَاةِ مُقَاوَمَةً لِنَفْسِهِ مِثْلَ هَذِهِ لِئَلاَّ تَكِلُّوا وَتَخُورُوا فِي نُفُوسِكُمْ.
\par 4 لَمْ تُقَاوِمُوا بَعْدُ حَتَّى الدَّمِ مُجَاهِدِينَ ضِدَّ الْخَطِيَّةِ،
\par 5 وَقَدْ نَسِيتُمُ الْوَعْظَ الَّذِي يُخَاطِبُكُمْ كَبَنِينَ: «يَا ابْنِي لاَ تَحْتَقِرْ تَأْدِيبَ الرَّبِّ، وَلاَ تَخُرْ إِذَا وَبَّخَكَ.
\par 6 لأَنَّ الَّذِي يُحِبُّهُ الرَّبُّ يُؤَدِّبُهُ، وَيَجْلِدُ كُلَّ ابْنٍ يَقْبَلُهُ».
\par 7 إِنْ كُنْتُمْ تَحْتَمِلُونَ التَّأْدِيبَ يُعَامِلُكُمُ اللهُ كَالْبَنِينَ. فَأَيُّ ابْنٍ لاَ يُؤَدِّبُهُ أَبُوهُ؟
\par 8 وَلَكِنْ إِنْ كُنْتُمْ بِلاَ تَأْدِيبٍ، قَدْ صَارَ الْجَمِيعُ شُرَكَاءَ فِيهِ، فَأَنْتُمْ نُغُولٌ لاَ بَنُونَ.
\par 9 ثُمَّ قَدْ كَانَ لَنَا آبَاءُ أَجْسَادِنَا مُؤَدِّبِينَ، وَكُنَّا نَهَابُهُمْ. أَفَلاَ نَخْضَعُ بِالأَوْلَى جِدّاً لأَبِي الأَرْوَاحِ، فَنَحْيَا؟
\par 10 لأَنَّ أُولَئِكَ أَدَّبُونَا أَيَّاماً قَلِيلَةً حَسَبَ اسْتِحْسَانِهِمْ، وَأَمَّا هَذَا فَلأَجْلِ الْمَنْفَعَةِ، لِكَيْ نَشْتَرِكَ فِي قَدَاسَتِهِ.
\par 11 وَلَكِنَّ كُلَّ تَأْدِيبٍ فِي الْحَاضِرِ لاَ يُرَى أَنَّهُ لِلْفَرَحِ بَلْ لِلْحَزَنِ. وَأَمَّا أَخِيراً فَيُعْطِي الَّذِينَ يَتَدَرَّبُونَ بِهِ ثَمَرَ بِرٍّ لِلسَّلاَمِ.
\par 12 لِذَلِكَ قَّوِمُوا الأَيَادِيَ الْمُسْتَرْخِيَةَ وَالرُّكَبَ الْمُخَلَّعَةَ،
\par 13 وَاصْنَعُوا لأَرْجُلِكُمْ مَسَالِكَ مُسْتَقِيمَةً، لِكَيْ لاَ يَعْتَسِفَ الأَعْرَجُ، بَلْ بِالْحَرِيِّ يُشْفَى.
\par 14 اِتْبَعُوا السَّلاَمَ مَعَ الْجَمِيعِ، وَالْقَدَاسَةَ الَّتِي بِدُونِهَا لَنْ يَرَى أَحَدٌ الرَّبَّ.
\par 15 مُلاَحِظِينَ لِئَلاَّ يَخِيبَ أَحَدٌ مِنْ نِعْمَةِ اللهِ. لِئَلاَّ يَطْلُعَ أَصْلُ مَرَارَةٍ وَيَصْنَعَ انْزِعَاجاً، فَيَتَنَجَّسَ بِهِ كَثِيرُونَ.
\par 16 لِئَلاَّ يَكُونَ أَحَدٌ زَانِياً أَوْ مُسْتَبِيحاً كَعِيسُو، الَّذِي لأَجْلِ أَكْلَةٍ وَاحِدَةٍ بَاعَ بَكُورِيَّتَهُ.
\par 17 فَإِنَّكُمْ تَعْلَمُونَ أَنَّهُ أَيْضاً بَعْدَ ذَلِكَ، لَمَّا أَرَادَ أَنْ يَرِثَ الْبَرَكَةَ رُفِضَ، إِذْ لَمْ يَجِدْ لِلتَّوْبَةِ مَكَاناً، مَعَ أَنَّهُ طَلَبَهَا بِدُمُوعٍ.
\par 18 لأَنَّكُمْ لَمْ تَأْتُوا إِلَى جَبَلٍ مَلْمُوسٍ مُضْطَرِمٍ بِالنَّارِ، وَإِلَى ضَبَابٍ وَظَلاَمٍ وَزَوْبَعَةٍ،
\par 19 وَهُتَافِ بُوقٍ وَصَوْتِ كَلِمَاتٍ، اسْتَعْفَى الَّذِينَ سَمِعُوهُ مِنْ أَنْ تُزَادَ لَهُمْ كَلِمَةٌ،
\par 20 لأَنَّهُمْ لَمْ يَحْتَمِلُوا مَا أُمِرَ بِهِ، وَإِنْ مَسَّتِ الْجَبَلَ بَهِيمَةٌ تُرْجَمُ أَوْ تُرْمَى بِسَهْمٍ.
\par 21 وَكَانَ الْمَنْظَرُ هَكَذَا مُخِيفاً حَتَّى قَالَ مُوسَى: «أَنَا مُرْتَعِبٌ وَمُرْتَعِدٌ!».
\par 22 بَلْ قَدْ أَتَيْتُمْ إِلَى جَبَلِ صِهْيَوْنَ، وَإِلَى مَدِينَةِ اللهِ الْحَيِّ: أُورُشَلِيمَ السَّمَاوِيَّةِ، وَإِلَى رَبَوَاتٍ هُمْ مَحْفِلُ مَلاَئِكَةٍ،
\par 23 وَكَنِيسَةِ أَبْكَارٍ مَكْتُوبِينَ فِي السَّمَاوَاتِ، وَإِلَى اللهِ دَيَّانِ الْجَمِيعِ، وَإِلَى أَرْوَاحِ أَبْرَارٍ مُكَمَّلِينَ،
\par 24 وَإِلَى وَسِيطِ الْعَهْدِ الْجَدِيدِ: يَسُوعَ، وَإِلَى دَمِ رَشٍّ يَتَكَلَّمُ أَفْضَلَ مِنْ هَابِيلَ.
\par 25 اُنْظُرُوا أَنْ لاَ تَسْتَعْفُوا مِنَ الْمُتَكَلِّمِ. لأَنَّهُ إِنْ كَانَ أُولَئِكَ لَمْ يَنْجُوا إِذِ اسْتَعْفَوْا مِنَ الْمُتَكَلِّمِ عَلَى الأَرْضِ، فَبِالأَوْلَى جِدّاً لاَ نَنْجُو نَحْنُ الْمُرْتَدِّينَ عَنِ الَّذِي مِنَ السَّمَاءِ،
\par 26 الَّذِي صَوْتُهُ زَعْزَعَ الأَرْضَ حِينَئِذٍ، وَأَمَّا الآنَ فَقَدْ وَعَدَ قَائِلاً: «إِنِّي مَرَّةً أَيْضاً أُزَلْزِلُ لاَ الأَرْضَ فَقَطْ بَلِ السَّمَاءَ أَيْضاً».
\par 27 فَقَوْلُهُ «مَرَّةً أَيْضاً» يَدُلُّ عَلَى تَغْيِيرِ الأَشْيَاءِ الْمُتَزَعْزِعَةِ كَمَصْنُوعَةٍ، لِكَيْ تَبْقَى الَّتِي لاَ تَتَزَعْزَعُ.
\par 28 لِذَلِكَ وَنَحْنُ قَابِلُونَ مَلَكُوتاً لاَ يَتَزَعْزَعُ لِيَكُنْ عِنْدَنَا شُكْرٌ بِهِ نَخْدِمُ اللهَ خِدْمَةً مَرْضِيَّةً، بِخُشُوعٍ وَتَقْوَى.
\par 29 لأَنَّ إِلَهَنَا نَارٌ آكِلَةٌ.

\chapter{13}

\par 1 لِتَثْبُتِ الْمَحَبَّةُ الأَخَوِيَّةُ.
\par 2 لاَ تَنْسُوا إِضَافَةَ الْغُرَبَاءِ، لأَنْ بِهَا أَضَافَ أُنَاسٌ مَلاَئِكَةً وَهُمْ لاَ يَدْرُونَ.
\par 3 اُذْكُرُوا الْمُقَيَّدِينَ كَأَنَّكُمْ مُقَيَّدُونَ مَعَهُمْ، وَالْمُذَلِّينَ كَأَنَّكُمْ أَنْتُمْ أَيْضاً فِي الْجَسَدِ.
\par 4 لِيَكُنِ الزِّوَاجُ مُكَرَّماً عِنْدَ كُلِّ وَاحِدٍ، وَالْمَضْجَعُ غَيْرَ نَجِسٍ. وَأَمَّا الْعَاهِرُونَ وَالزُّنَاةُ فَسَيَدِينُهُمُ اللهُ.
\par 5 لِتَكُنْ سِيرَتُكُمْ خَالِيَةً مِنْ مَحَبَّةِ الْمَالِ. كُونُوا مُكْتَفِينَ بِمَا عِنْدَكُمْ، لأَنَّهُ قَالَ: «لاَ أُهْمِلُكَ وَلاَ أَتْرُكُكَ»
\par 6 حَتَّى إِنَّنَا نَقُولُ وَاثِقِينَ: «الرَّبُّ مُعِينٌ لِي فَلاَ أَخَافُ. مَاذَا يَصْنَعُ بِي إِنْسَانٌ؟»
\par 7 اُذْكُرُوا مُرْشِدِيكُمُ الَّذِينَ كَلَّمُوكُمْ بِكَلِمَةِ اللهِ. انْظُرُوا إِلَى نِهَايَةِ سِيرَتِهِمْ فَتَمَثَّلُوا بِإِيمَانِهِمْ.
\par 8 يَسُوعُ الْمَسِيحُ هُوَ هُوَ أَمْساً وَالْيَوْمَ وَإِلَى الأَبَدِ.
\par 9 لاَ تُسَاقُوا بِتَعَالِيمَ مُتَنَّوِعَةٍ وَغَرِيبَةٍ، لأَنَّهُ حَسَنٌ أَنْ يُثَبَّتَ الْقَلْبُ بِالنِّعْمَةِ، لاَ بِأَطْعِمَةٍ لَمْ يَنْتَفِعْ بِهَا الَّذِينَ تَعَاطَوْهَا.
\par 10 لَنَا «مَذْبَحٌ» لاَ سُلْطَانَ لِلَّذِينَ يَخْدِمُونَ الْمَسْكَنَ أَنْ يَأْكُلُوا مِنْهُ.
\par 11 فَإِنَّ الْحَيَوَانَاتِ الَّتِي يُدْخَلُ بِدَمِهَا عَنِ الْخَطِيَّةِ إِلَى «الأَقْدَاسِ» بِيَدِ رَئِيسِ الْكَهَنَةِ تُحْرَقُ أَجْسَامُهَا خَارِجَ الْمَحَلَّةِ.
\par 12 لِذَلِكَ يَسُوعُ أَيْضاً، لِكَيْ يُقَدِّسَ الشَّعْبَ بِدَمِ نَفْسِهِ، تَأَلَّمَ خَارِجَ الْبَابِ.
\par 13 فَلْنَخْرُجْ إِذاً إِلَيْهِ خَارِجَ الْمَحَلَّةِ حَامِلِينَ عَارَهُ.
\par 14 لأَنْ لَيْسَ لَنَا هُنَا مَدِينَةٌ بَاقِيَةٌ، لَكِنَّنَا نَطْلُبُ الْعَتِيدَةَ.
\par 15 فَلْنُقَدِّمْ بِهِ فِي كُلِّ حِينٍ لِلَّهِ ذَبِيحَةَ التَّسْبِيحِ، أَيْ ثَمَرَ شِفَاهٍ مُعْتَرِفَةٍ بِاسْمِهِ.
\par 16 وَلَكِنْ لاَ تَنْسُوا فِعْلَ الْخَيْرِ وَالتَّوْزِيعَ، لأَنَّهُ بِذَبَائِحَ مِثْلِ هَذِهِ يُسَرُّ اللهُ.
\par 17 أَطِيعُوا مُرْشِدِيكُمْ وَاخْضَعُوا، لأَنَّهُمْ يَسْهَرُونَ لأَجْلِ نُفُوسِكُمْ كَأَنَّهُمْ سَوْفَ يُعْطُونَ حِسَاباً، لِكَيْ يَفْعَلُوا ذَلِكَ بِفَرَحٍ، لاَ آنِّينَ، لأَنَّ هَذَا غَيْرُ نَافِعٍ لَكُمْ.
\par 18 صَلُّوا لأَجْلِنَا، لأَنَّنَا نَثِقُ أَنَّ لَنَا ضَمِيراً صَالِحاً، رَاغِبِينَ أَنْ نَتَصَرَّفَ حَسَناً فِي كُلِّ شَيْءٍ.
\par 19 وَلَكِنْ أَطْلُبُ أَكْثَرَ أَنْ تَفْعَلُوا هَذَا لِكَيْ أُرَدَّ إِلَيْكُمْ بِأَكْثَرِ سُرْعَةٍ.
\par 20 وَإِلَهُ السَّلاَمِ الَّذِي أَقَامَ مِنَ الأَمْوَاتِ رَاعِيَ الْخِرَافِ الْعَظِيمَ، رَبَّنَا يَسُوعَ، بِدَمِ الْعَهْدِ الأَبَدِيِّ،
\par 21 لِيُكَمِّلْكُمْ فِي كُلِّ عَمَلٍ صَالِحٍ لِتَصْنَعُوا مَشِيئَتَهُ، عَامِلاً فِيكُمْ مَا يُرْضِي أَمَامَهُ بِيَسُوعَ الْمَسِيحِ، الَّذِي لَهُ الْمَجْدُ إِلَى أَبَدِ الآبِدِينَ. آمِينَ.
\par 22 وَأَطْلُبُ إِلَيْكُمْ أَيُّهَا الإِخْوَةُ أَنْ تَحْتَمِلُوا كَلِمَةَ الْوَعْظِ، لأَنِّي بِكَلِمَاتٍ قَلِيلَةٍ كَتَبْتُ إِلَيْكُمْ.
\par 23 اِعْلَمُوا أَنَّهُ قَدْ أُطْلِقَ الأَخُ تِيمُوثَاوُسُ، الَّذِي مَعَهُ سَوْفَ أَرَاكُمْ، إِنْ أَتَى سَرِيعاً.
\par 24 سَلِّمُوا عَلَى جَمِيعِ مُرْشِدِيكُمْ وَجَمِيعِ الْقِدِّيسِينَ. يُسَلِّمُ عَلَيْكُمُ الَّذِينَ مِنْ إِيطَالِيَا.
\par 25 اَلنِّعْمَةُ مَعَ جَمِيعِكُمْ. آمِينَ.

\end{document}