\begin{document}

\title{وصية يساكر}

\chapter{1}

يساكر، الابن الخامس ليعقوب وليئة. الابن البريء من الخطيئة، الذي يُؤجر باللفّاح. يدعو إلى البساطة.

\chapter{1}
نسخة من كلام يساكر.

\par 2 لأنه دعا بنيه وقال لهم: اسمعوا يا بنيّ من يساكر أبيكم، وأصغوا إلى كلام حبيب الرب

\par 3 وُلِدتُ الابن الخامس ليعقوب، على سبيل الأجرة مقابل اللفاح.

\par 4 فأتى أخي رأوبين باللفاح من الحقل، فتلقته راحيل وأخذته.

\par 5 فبكى رأوبين، وعند صوته خرجت ليئة أمي.

\par 6 وكانت هذه اللفاحات تفاحات طيبة الرائحة، تُنتج في أرض حاران تحت وادي الماء.

\par 7 فقالت راحيل: لا أعطيك إياهم، بل يكونون لي عوضًا عن الأولاد

\par 8 لأن الرب قد احتقرني، ولم أنجب ليعقوب

\par 9 وكانت هناك تفاحتان، فقالت ليئة لراحيل: يكفيك أنك أخذت زوجي. أتأخذين هاتين أيضًا؟

\par 10 فقالت لها راحيل: يعقوب يكون لكِ الليلة لفّاح ابنكِ،

\par 11 فقالت لها ليئة: يعقوب لي، لأني امرأة صباه.

\par 12 فقالت راحيل لا تفتخر ولا تتباهى، فإنه خطبني أمامك، وخدم أبانا من أجلي أربع عشرة سنة.

\par 13 ولولا أن الكيد قد كثر على الأرض وازدهر شر البشر، لما رأيت الآن وجه يعقوب

\par 14 لأنكِ لستِ زوجته، ولكنكِ في المِهنةِ اتُّخذتِ لهُ بدلًا مني

\par 15 فخدعني أبي، ونقلني في تلك الليلة، ولم يدع يعقوب يراني، لأنه لو كنت هناك، لما أصابه هذا

\par 16 ومع ذلك، فإني أستأجر يعقوب لك ليلة واحدة مقابل اللفاح

\par 17 وعرف يعقوب ليئة، فحبلت وولدتني، ومن أجل الأجرة دُعيتُ يساكر

\par 18 فظهر ملاك الرب ليعقوب قائلًا: راحيل ستلد ولدين لأنها رفضت مصاحبة زوجها واختارت العفة

\par 19 ولولا أن ليا أمي دفعت ثمن التفاحتين من أجل شركته، لكانت قد أنجبت ثمانية أبناء. لذلك ولدت ستة، وولدت راحيل الاثنين: لأن الرب افتقدها بسبب اللفاح

\par 20 لأنه كان يعلم أنها أرادت أن تصاحب يعقوب من أجل الأطفال، وليس من أجل شهوة اللذة

\par 21 ففي الغد أيضًا أسلمت يعقوب مرة أخرى.

\par 22 ولذلك سمع الرب لراحيل بسبب اللفاح.

\par 23 لأنها وإن اشتهتها لم تذقها، بل قدمتها في بيت الرب، وقدمتها لكاهن العلي الذي كان في ذلك الوقت

\par 24 فلما كَبُرتُ يا أبنائي، سلكتُ باستقامة القلب، وصرتُ فلاحًا لأبي ولإخوتي، وأتيتُ بثمرٍ من الحقل في أوانه

\par 25 وباركني أبي، لأنه رأى أنني أسير باستقامة أمامه

\par 26 ولم أكن فضوليًا في أعمالي، ولا حاسدًا ولا حاقدًا على جاري

\par 27 لم أشتم أحدًا قط، ولم أنتقد حياة أي إنسان، كما كنت أسير بعين واحدة

\par 28 لذلك، عندما بلغت الخامسة والثلاثين من عمري، تزوجت، لأن عملي أنهك قوتي، ولم أفكر قط في اللذة مع النساء؛ ولكن بسبب تعبِي، غلبني النوم

\par 29 وكان أبي يفرح دائمًا باستقامتي، لأني كنت أقدم للرب كل باكورة عن طريق الكاهن، ثم لأبي أيضًا

\par 30 وزاد الرب بركاته في يدي عشرة آلاف ضعف، وعلم يعقوب أبي أيضًا أن الله أعانني على عزوبيتي

\par 31 لأني منحت كل الفقراء والمظلومين خيرات الأرض بتواضع قلبي

\par 32 والآن، اسمعوا لي يا أبنائي، وامشوا في بساطة قلوبكم، لأني رأيت فيها كل ما يرضي الرب

\par 33 الرجل ذو الرأي الواحد لا يشتهي الذهب، ولا يتجاوز حدود جاره، ولا يتوق إلى أنواع متعددة من الأطعمة الشهية، ولا يُسعده تنوع الملابس

\par 34 إنه لا يرغب في أن يعيش حياة طويلة، بل ينتظر فقط مشيئة الله

\par 35 وليس لأرواح الخداع سلطان عليه، لأنه لا ينظر إلى جمال النساء، لئلا يدنس عقله بالفساد

\par 36 لا يوجد حسد في أفكاره، ولا شخص خبيث يجعل روحه تذبل، ولا يقلق برغبة لا تشبع في عقله

\par 37 لأنه يسلك ببراءة نفس، وينظر إلى كل شيء باستقامة قلب، متجنبًا العيون الشريرة بسبب ضلال العالم، لئلا يرى انحراف أي من وصايا الرب

\par 38 لذلك يا أبنائي، احفظوا شريعة الله، واحصلوا على حياة العزوبية، وامشوا في براءة، ولا تتطفلوا على شؤون قريبكم، بل أحبوا الرب وقريبكم، وارحموا الفقراء والضعفاء

\par 39 انحنِ ظهرك للفلاحة، واتعب في كل أنواع الفلاحة، مقدمًا عطاياك للرب مع الشكر

\par 40 لأنه بباكورة الأرض يبارككم الرب، كما بارك جميع القديسين من هابيل إلى الآن

\par 41 فإنه لا يُعطى لكم نصيب إلا من دسم الأرض، التي تُرفع ثمارها بالتعب

\par 42 لأن أبانا يعقوب باركني ببركات الأرض والبواكير

\par 43 ومجد الرب لاوي ويهوذا بين بني يعقوب، لأن الرب أعطاهم ميراثًا، وأعطى لاوي الكهنوت، ويهوذا المملكة

\par 44 فأطيعواهما وامشوا في بساطة أبيكم، لأنه لجاد أُعطي أن يُهلك الجيوش الآتية على إسرائيل

\chapter{2}

\chapter{1}
فاعلموا يا أبنائي أنه في الأيام الأخيرة سيتخلى أبناؤكم عن العزوبية، وسيتمسكون بالشهوة التي لا تشبع.

\par 2 ويتركون البراءة، ويقتربون من الشر، ويتركون وصايا الرب، ويلتصقون ببليعار

\par 3 ويتركون الزراعة، ويتبعون مكائدهم الشريرة، ويتشتتون بين الأمم، ويستعبدون لأعدائهم

\par 4 فأوصوا أولادكم بهذه الوصايا، حتى إذا أخطأوا، يرجعوا إلى الرب أسرع؛ لأنه رحيم، وسينقذهم، حتى يردهم إلى أرضهم

\par 5 فها أنا ذا، كما ترون، أبلغ من العمر مئة وستة وعشرين عامًا، ولا أعلم أنني ارتكبت أي خطيئة

\par 6 لم أعرف امرأةً إلا زوجتي. ولم أزنِ قط برفع عيني

\par 7 لم أشرب الخمر لأُضلَّ به؛

\par 8 لم أشتهِ أي شيء مرغوب فيه مما كان لدى جاري.

\par 9 لم ينشأ مكر في قلبي.

\par 10 لم يمر كذب من بين شفتي

\par 11 إذا كان أي رجل في محنة، كنت أضم تنهداتي إلى تنهداته،

\par 12 وتقاسمت خبزي مع الفقراء.

\par 13 صنعتُ التقوى، وحفظتُ الحق كل أيامي.

\par 14 أحببت الرب، وكذلك كل إنسان من كل قلبي.

\par 15 هكذا افعلوا أنتم أيضًا هذه الأشياء يا أبنائي، فيهرب منكم كل روح بليعار، ولا يتسلط عليكم عمل الأشرار؛

\par 16 وتستعبدون كل وحش بري، لأن معكم إله السماء والأرض، وتسيرون مع الناس بقلب واحد

\par 17 وبعد أن قال هذا، أوصى بنيه أن يحملوه إلى حبرون، ويدفنوه هناك في المغارة مع آبائه

\par 18 ومد قدميه ومات، في شيخوخة صالحة؛ بكل عضو سليم، وبقوة لا تلين، نام نومًا أبديًا



\end{document}