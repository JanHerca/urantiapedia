\begin{document}

\title{مرقس}


\chapter{1}

\par 1 بَدْءُ إِنْجِيلِ يَسُوعَ الْمَسِيحِ ابْنِ اللَّهِ:
\par 2 كَمَا هُوَ مَكْتُوبٌ فِي الأَنْبِيَاءِ: «هَا أَنَا أُرْسِلُ أَمَامَ وَجْهِكَ مَلاَكِي الَّذِي يُهَيِّئُ طَرِيقَكَ قُدَّامَكَ.
\par 3 صَوْتُ صَارِخٍ فِي الْبَرِّيَّةِ: أَعِدُّوا طَرِيقَ الرَّبِّ اصْنَعُوا سُبُلَهُ مُسْتَقِيمَةً».
\par 4 كَانَ يُوحَنَّا يُعَمِّدُ فِي الْبَرِّيَّةِ وَيَكْرِزُ بِمَعْمُودِيَّةِ التَّوْبَةِ لِمَغْفِرَةِ الْخَطَايَا.
\par 5 وَخَرَجَ إِلَيْهِ جَمِيعُ كُورَةِ الْيَهُودِيَّةِ وَأَهْلُ أُورُشَلِيمَ وَاعْتَمَدُوا جَمِيعُهُمْ مِنْهُ فِي نَهْرِ الأُرْدُنِّ مُعْتَرِفِينَ بِخَطَايَاهُمْ.
\par 6 وَكَانَ يُوحَنَّا يَلْبَسُ وَبَرَ الإِبِلِ وَمِنْطَقَةً مِنْ جِلْدٍ عَلَى حَقَوَيْهِ وَيَأْكُلُ جَرَاداً وَعَسَلاً بَرِّيّاً.
\par 7 وَكَانَ يَكْرِزُ قَائِلاً: «يَأْتِي بَعْدِي مَنْ هُوَ أَقْوَى مِنِّي الَّذِي لَسْتُ أَهْلاً أَنْ أَنْحَنِيَ وَأَحُلَّ سُيُورَ حِذَائِهِ.
\par 8 أَنَا عَمَّدْتُكُمْ بِالْمَاءِ وَأَمَّا هُوَ فَسَيُعَمِّدُكُمْ بِالرُّوحِ الْقُدُسِ».
\par 9 وَفِي تِلْكَ الأَيَّامِ جَاءَ يَسُوعُ مِنْ نَاصِرَةِ الْجَلِيلِ وَاعْتَمَدَ مِنْ يُوحَنَّا فِي الأُرْدُنِّ.
\par 10 وَلِلْوَقْتِ وَهُوَ صَاعِدٌ مِنَ الْمَاءِ رَأَى السَّمَاوَاتِ قَدِ انْشَقَّتْ وَالرُّوحَ مِثْلَ حَمَامَةٍ نَازِلاً عَلَيْهِ.
\par 11 وَكَانَ صَوْتٌ مِنَ السَّمَاوَاتِ: «أَنْتَ ابْنِي الْحَبِيبُ الَّذِي بِهِ سُرِرْتُ!».
\par 12 وَلِلْوَقْتِ أَخْرَجَهُ الرُّوحُ إِلَى الْبَرِّيَّةِ
\par 13 وَكَانَ هُنَاكَ فِي الْبَرِّيَّةِ أَرْبَعِينَ يَوْماً يُجَرَّبُ مِنَ الشَّيْطَانِ. وَكَانَ مَعَ الْوُحُوشِ. وَصَارَتِ الْمَلاَئِكَةُ تَخْدِمُهُ.
\par 14 وَبَعْدَ مَا أُسْلِمَ يُوحَنَّا جَاءَ يَسُوعُ إِلَى الْجَلِيلِ يَكْرِزُ بِبِشَارَةِ مَلَكُوتِ اللَّهِ
\par 15 وَيَقُولُ: «قَدْ كَمَلَ الزَّمَانُ وَاقْتَرَبَ مَلَكُوتُ اللَّهِ فَتُوبُوا وَآمِنُوا بِالإِنْجِيلِ».
\par 16 وَفِيمَا هُوَ يَمْشِي عِنْدَ بَحْرِ الْجَلِيلِ أَبْصَرَ سِمْعَانَ وَأَنْدَرَاوُسَ أَخَاهُ يُلْقِيَانِ شَبَكَةً فِي الْبَحْرِ فَإِنَّهُمَا كَانَا صَيَّادَيْنِ.
\par 17 فَقَالَ لَهُمَا يَسُوعُ: «هَلُمَّ وَرَائِي فَأَجْعَلُكُمَا تَصِيرَانِ صَيَّادَيِ النَّاسِ».
\par 18 فَلِلْوَقْتِ تَرَكَا شِبَاكَهُمَا وَتَبِعَاهُ.
\par 19 ثُمَّ اجْتَازَ مِنْ هُنَاكَ قَلِيلاً فَرَأَى يَعْقُوبَ بْنَ زَبْدِي وَيُوحَنَّا أَخَاهُ وَهُمَا فِي السَّفِينَةِ يُصْلِحَانِ الشِّبَاكَ.
\par 20 فَدَعَاهُمَا لِلْوَقْتِ. فَتَرَكَا أَبَاهُمَا زَبْدِي فِي السَّفِينَةِ مَعَ الأَجْرَى وَذَهَبَا وَرَاءَهُ.
\par 21 ثُمَّ دَخَلُوا كَفْرَنَاحُومَ وَلِلْوَقْتِ دَخَلَ الْمَجْمَعَ فِي السَّبْتِ وَصَارَ يُعَلِّمُ.
\par 22 فَبُهِتُوا مِنْ تَعْلِيمِهِ لأَنَّهُ كَانَ يُعَلِّمُهُمْ كَمَنْ لَهُ سُلْطَانٌ وَلَيْسَ كَالْكَتَبَةِ.
\par 23 وَكَانَ فِي مَجْمَعِهِمْ رَجُلٌ بِهِ رُوحٌ نَجِسٌ فَصَرَخَ
\par 24 قَائِلاً: «آهِ! مَا لَنَا وَلَكَ يَا يَسُوعُ النَّاصِرِيُّ! أَتَيْتَ لِتُهْلِكَنَا! أَنَا أَعْرِفُكَ مَنْ أَنْتَ قُدُّوسُ اللَّهِ!»
\par 25 فَانْتَهَرَهُ يَسُوعُ قَائِلاً: «اخْرَسْ وَاخْرُجْ مِنْهُ!»
\par 26 فَصَرَعَهُ الرُّوحُ النَّجِسُ وَصَاحَ بِصَوْتٍ عَظِيمٍ وَخَرَجَ مِنْهُ.
\par 27 فَتَحَيَّرُوا كُلُّهُمْ حَتَّى سَأَلَ بَعْضُهُمْ بَعْضاً قَائِلِينَ: «مَا هَذَا؟ مَا هُوَ هَذَا التَّعْلِيمُ الْجَدِيدُ؟ لأَنَّهُ بِسُلْطَانٍ يَأْمُرُ حَتَّى الأَرْوَاحَ النَّجِسَةَ فَتُطِيعُهُ!»
\par 28 فَخَرَجَ خَبَرُهُ لِلْوَقْتِ فِي كُلِّ الْكُورَةِ الْمُحِيطَةِ بِالْجَلِيلِ.
\par 29 وَلَمَّا خَرَجُوا مِنَ الْمَجْمَعِ جَاءُوا لِلْوَقْتِ إِلَى بَيْتِ سِمْعَانَ وَأَنْدَرَاوُسَ مَعَ يَعْقُوبَ وَيُوحَنَّا
\par 30 وَكَانَتْ حَمَاةُ سِمْعَانَ مُضْطَجِعَةً مَحْمُومَةً فَلِلْوَقْتِ أَخْبَرُوهُ عَنْهَا.
\par 31 فَتَقَدَّمَ وَأَقَامَهَا مَاسِكاً بِيَدِهَا فَتَرَكَتْهَا الْحُمَّى حَالاً وَصَارَتْ تَخْدِمُهُمْ.
\par 32 وَلَمَّا صَارَ الْمَسَاءُ إِذْ غَرَبَتِ الشَّمْسُ قَدَّمُوا إِلَيْهِ جَمِيعَ السُّقَمَاءِ وَالْمَجَانِينَ.
\par 33 وَكَانَتِ الْمَدِينَةُ كُلُّهَا مُجْتَمِعَةً عَلَى الْبَابِ.
\par 34 فَشَفَى كَثِيرِينَ كَانُوا مَرْضَى بِأَمْرَاضٍ مُخْتَلِفَةٍ وَأَخْرَجَ شَيَاطِينَ كَثِيرَةً وَلَمْ يَدَعِ الشَّيَاطِينَ يَتَكَلَّمُونَ لأَنَّهُمْ عَرَفُوهُ.
\par 35 وَفِي الصُّبْحِ بَاكِراً جِدّاً قَامَ وَخَرَجَ وَمَضَى إِلَى مَوْضِعٍ خَلاَءٍ وَكَانَ يُصَلِّي هُنَاكَ
\par 36 فَتَبِعَهُ سِمْعَانُ وَالَّذِينَ مَعَهُ.
\par 37 وَلَمَّا وَجَدُوهُ قَالُوا لَهُ: «إِنَّ الْجَمِيعَ يَطْلُبُونَكَ».
\par 38 فَقَالَ لَهُمْ: «لِنَذْهَبْ إِلَى الْقُرَى الْمُجَاوِرَةِ لأَكْرِزَ هُنَاكَ أَيْضاً لأَنِّي لِهَذَا خَرَجْتُ».
\par 39 فَكَانَ يَكْرِزُ فِي مَجَامِعِهِمْ فِي كُلِّ الْجَلِيلِ وَيُخْرِجُ الشَّيَاطِينَ.
\par 40 فَأَتَى إِلَيْهِ أَبْرَصُ يَطْلُبُ إِلَيْهِ جَاثِياً وَقَائِلاً لَهُ: «إِنْ أَرَدْتَ تَقْدِرْ أَنْ تُطَهِّرَنِي!»
\par 41 فَتَحَنَّنَ يَسُوعُ وَمَدَّ يَدَهُ وَلَمَسَهُ وَقَالَ لَهُ: «أُرِيدُ فَاطْهُرْ».
\par 42 فَلِلْوَقْتِ وَهُوَ يَتَكَلَّمُ ذَهَبَ عَنْهُ الْبَرَصُ وَطَهَرَ.
\par 43 فَانْتَهَرَهُ وَأَرْسَلَهُ لِلْوَقْتِ
\par 44 وَقَالَ لَهُ: «انْظُرْ لاَ تَقُلْ لأَحَدٍ شَيْئاً بَلِ اذْهَبْ أَرِ نَفْسَكَ لِلْكَاهِنِ وَقَدِّمْ عَنْ تَطْهِيرِكَ مَا أَمَرَ بِهِ مُوسَى شَهَادَةً لَهُمْ».
\par 45 وَأَمَّا هُوَ فَخَرَجَ وَابْتَدَأَ يُنَادِي كَثِيراً وَيُذِيعُ الْخَبَرَ حَتَّى لَمْ يَعُدْ يَقْدِرُ أَنْ يَدْخُلَ مَدِينَةً ظَاهِراً بَلْ كَانَ خَارِجاً فِي مَوَاضِعَ خَالِيَةٍ وَكَانُوا يَأْتُونَ إِلَيْهِ مِنْ كُلِّ نَاحِيَةٍ.

\chapter{2}

\par 1 ثُمَّ دَخَلَ كَفْرَنَاحُومَ أَيْضاً بَعْدَ أَيَّامٍ فَسُمِعَ أَنَّهُ فِي بَيْتٍ.
\par 2 وَلِلْوَقْتِ اجْتَمَعَ كَثِيرُونَ حَتَّى لَمْ يَعُدْ يَسَعُ وَلاَ مَا حَوْلَ الْبَابِ. فَكَانَ يُخَاطِبُهُمْ بِالْكَلِمَةِ.
\par 3 وَجَاءُوا إِلَيْهِ مُقَدِّمِينَ مَفْلُوجاً يَحْمِلُهُ أَرْبَعَةٌ.
\par 4 وَإِذْ لَمْ يَقْدِرُوا أَنْ يَقْتَرِبُوا إِلَيْهِ مِنْ أَجْلِ الْجَمْعِ كَشَفُوا السَّقْفَ حَيْثُ كَانَ. وَبَعْدَ مَا نَقَبُوهُ دَلَّوُا السَّرِيرَ الَّذِي كَانَ الْمَفْلُوجُ مُضْطَجِعاً عَلَيْهِ.
\par 5 فَلَمَّا رَأَى يَسُوعُ إِيمَانَهُمْ قَالَ لِلْمَفْلُوجِ: «يَا بُنَيَّ مَغْفُورَةٌ لَكَ خَطَايَاكَ».
\par 6 وَكَانَ قَوْمٌ مِنَ الْكَتَبَةِ هُنَاكَ جَالِسِينَ يُفَكِّرُونَ فِي قُلُوبِهِمْ:
\par 7 «لِمَاذَا يَتَكَلَّمُ هَذَا هَكَذَا بِتَجَادِيفَ؟ مَنْ يَقْدِرُ أَنْ يَغْفِرَ خَطَايَا إلاَّ اللَّهُ وَحْدَهُ؟»
\par 8 فَلِلْوَقْتِ شَعَرَ يَسُوعُ بِرُوحِهِ أَنَّهُمْ يُفَكِّرُونَ هَكَذَا فِي أَنْفُسِهِمْ فَقَالَ لَهُمْ: «لِمَاذَا تُفَكِّرُونَ بِهَذَا فِي قُلُوبِكُمْ؟
\par 9 أَيُّمَا أَيْسَرُ: أَنْ يُقَالَ لِلْمَفْلُوجِ مَغْفُورَةٌ لَكَ خَطَايَاكَ أَمْ أَنْ يُقَالَ: قُمْ وَاحْمِلْ سَرِيرَكَ وَامْشِ؟
\par 10 وَلَكِنْ لِكَيْ تَعْلَمُوا أَنَّ لاِبْنِ الإِنْسَانِ سُلْطَاناً عَلَى الأَرْضِ أَنْ يَغْفِرَ الْخَطَايَا» - قَالَ لِلْمَفْلُوجِ:
\par 11 «لَكَ أَقُولُ قُمْ وَاحْمِلْ سَرِيرَكَ وَاذْهَبْ إِلَى بَيْتِكَ».
\par 12 فَقَامَ لِلْوَقْتِ وَحَمَلَ السَّرِيرَ وَخَرَجَ قُدَّامَ الْكُلِّ حَتَّى بُهِتَ الْجَمِيعُ وَمَجَّدُوا اللَّهَ قَائِلِينَ: «مَا رَأَيْنَا مِثْلَ هَذَا قَطُّ!».
\par 13 ثُمَّ خَرَجَ أَيْضاً إِلَى الْبَحْرِ وَأَتَى إِلَيْهِ كُلُّ الْجَمْعِ فَعَلَّمَهُمْ.
\par 14 وَفِيمَا هُوَ مُجْتَازٌ رَأَى لاَوِيَ بْنَ حَلْفَى جَالِساً عِنْدَ مَكَانِ الْجِبَايَةِ فَقَالَ لَهُ: «اتْبَعْنِي». فَقَامَ وَتَبِعَهُ.
\par 15 وَفِيمَا هُوَ مُتَّكِئٌ فِي بَيْتِهِ كَانَ كَثِيرُونَ مِنَ الْعَشَّارِينَ وَالْخُطَاةِ يَتَّكِئُونَ مَعَ يَسُوعَ وَتَلاَمِيذِهِ لأَنَّهُمْ كَانُوا كَثِيرِينَ وَتَبِعُوهُ.
\par 16 وَأَمَّا الْكَتَبَةُ وَالْفَرِّيسِيُّونَ فَلَمَّا رَأَوْهُ يَأْكُلُ مَعَ الْعَشَّارِينَ وَالْخُطَاةِ قَالُوا لِتَلاَمِيذِهِ: «مَا بَالُهُ يَأْكُلُ وَيَشْرَبُ مَعَ الْعَشَّارِينَ وَالْخُطَاةِ؟»
\par 17 فَلَمَّا سَمِعَ يَسُوعُ قَالَ لَهُمْ: «لاَ يَحْتَاجُ الأَصِحَّاءُ إِلَى طَبِيبٍ بَلِ الْمَرْضَى. لَمْ آتِ لأَدْعُوَ أَبْرَاراً بَلْ خُطَاةً إِلَى التَّوْبَةِ».
\par 18 وَكَانَ تَلاَمِيذُ يُوحَنَّا وَالْفَرِّيسِيِّينَ يَصُومُونَ فَجَاءُوا وَقَالُوا لَهُ: «لِمَاذَا يَصُومُ تَلاَمِيذُ يُوحَنَّا وَالْفَرِّيسِيِّينَ وَأَمَّا تَلاَمِيذُكَ فَلاَ يَصُومُونَ؟»
\par 19 فَقَالَ لَهُمْ يَسُوعُ: «هَلْ يَسْتَطِيعُ بَنُو الْعُرْسِ أَنْ يَصُومُوا وَالْعَرِيسُ مَعَهُمْ؟ مَا دَامَ الْعَرِيسُ مَعَهُمْ لاَ يَسْتَطِيعُونَ أَنْ يَصُومُوا.
\par 20 وَلَكِنْ سَتَأْتِي أَيَّامٌ حِينَ يُرْفَعُ الْعَرِيسُ عَنْهُمْ فَحِينَئِذٍ يَصُومُونَ فِي تِلْكَ الأَيَّامِ.
\par 21 لَيْسَ أَحَدٌ يَخِيطُ رُقْعَةً مِنْ قِطْعَةٍ جَدِيدَةٍ عَلَى ثَوْبٍ عَتِيقٍ وَإِلاَّ فَالْمِلْءُ الْجَدِيدُ يَأْخُذُ مِنَ الْعَتِيقِ فَيَصِيرُ الْخَرْقُ أَرْدَأَ.
\par 22 وَلَيْسَ أَحَدٌ يَجْعَلُ خَمْراً جَدِيدَةً فِي زِقَاقٍ عَتِيقَةٍ لِئَلاَّ تَشُقَّ الْخَمْرُ الْجَدِيدَةُ الزِّقَاقَ فَالْخَمْرُ تَنْصَبُّ وَالزِّقَاقُ تَتْلَفُ. بَلْ يَجْعَلُونَ خَمْراً جَدِيدَةً فِي زِقَاقٍ جَدِيدَةٍ».
\par 23 وَاجْتَازَ فِي السَّبْتِ بَيْنَ الزُّرُوعِ فَابْتَدَأَ تَلاَمِيذُهُ يَقْطِفُونَ السَّنَابِلَ وَهُمْ سَائِرُونَ.
\par 24 فَقَالَ لَهُ الْفَرِّيسِيُّونَ: «انْظُرْ. لِمَاذَا يَفْعَلُونَ فِي السَّبْتِ مَا لاَ يَحِلُّ؟»
\par 25 فَقَالَ لَهُمْ: «أَمَا قَرَأْتُمْ قَطُّ مَا فَعَلَهُ دَاوُدُ حِينَ احْتَاجَ وَجَاعَ هُوَ وَالَّذِينَ مَعَهُ
\par 26 كَيْفَ دَخَلَ بَيْتَ اللَّهِ فِي أَيَّامِ أَبِيَاثَارَ رَئِيسِ الْكَهَنَةِ وَأَكَلَ خُبْزَ التَّقْدِمَةِ الَّذِي لاَ يَحِلُّ أَكْلُهُ إلاَّ لِلْكَهَنَةِ وَأَعْطَى الَّذِينَ كَانُوا مَعَهُ أَيْضاً؟»
\par 27 ثُمَّ قَالَ لَهُمُ: «السَّبْتُ إِنَّمَا جُعِلَ لأَجْلِ الإِنْسَانِ لاَ الإِنْسَانُ لأَجْلِ السَّبْتِ.
\par 28 إِذاً ابْنُ الإِنْسَانِ هُوَ رَبُّ السَّبْتِ أَيْضاً».

\chapter{3}

\par 1 ثُمَّ دَخَلَ أَيْضاً إِلَى الْمَجْمَعِ وَكَانَ هُنَاكَ رَجُلٌ يَدُهُ يَابِسَةٌ.
\par 2 فَصَارُوا يُرَاقِبُونَهُ: هَلْ يَشْفِيهِ فِي السَّبْتِ؟ لِكَيْ يَشْتَكُوا عَلَيْهِ.
\par 3 فَقَالَ لِلرَّجُلِ الَّذِي لَهُ الْيَدُ الْيَابِسَةُ: «قُمْ فِي الْوَسَطِ!»
\par 4 ثُمَّ قَالَ لَهُمْ: «هَلْ يَحِلُّ فِي السَّبْتِ فِعْلُ الْخَيْرِ أَوْ فِعْلُ الشَّرِّ؟ تَخْلِيصُ نَفْسٍ أَوْ قَتْلٌ؟». فَسَكَتُوا.
\par 5 فَنَظَرَ حَوْلَهُ إِلَيْهِمْ بِغَضَبٍ حَزِيناً عَلَى غِلاَظَةِ قُلُوبِهِمْ وَقَالَ لِلرَّجُلِ: «مُدَّ يَدَكَ». فَمَدَّهَا فَعَادَتْ يَدُهُ صَحِيحَةً كَالأُخْرَى.
\par 6 فَخَرَجَ الْفَرِّيسِيُّونَ لِلْوَقْتِ مَعَ الْهِيرُودُسِيِّينَ وَتَشَاوَرُوا عَلَيْهِ لِكَيْ يُهْلِكُوهُ.
\par 7 فَانْصَرَفَ يَسُوعُ مَعَ تَلاَمِيذِهِ إِلَى الْبَحْرِ وَتَبِعَهُ جَمْعٌ كَثِيرٌ مِنَ الْجَلِيلِ وَمِنَ الْيَهُودِيَّةِ
\par 8 وَمِنْ أُورُشَلِيمَ وَمِنْ أَدُومِيَّةَ وَمِنْ عَبْرِ الأُرْدُنِّ. وَالَّذِينَ حَوْلَ صُورَ وَصَيْدَاءَ جَمْعٌ كَثِيرٌ إِذْ سَمِعُوا كَمْ صَنَعَ أَتَوْا إِلَيْهِ.
\par 9 فَقَالَ لِتَلاَمِيذِهِ أَنْ تُلاَزِمَهُ سَفِينَةٌ صَغِيرَةٌ لِسَبَبِ الْجَمْعِ كَيْ لاَ يَزْحَمُوهُ
\par 10 لأَنَّهُ كَانَ قَدْ شَفَى كَثِيرِينَ حَتَّى وَقَعَ عَلَيْهِ لِيَلْمِسَهُ كُلُّ مَنْ فِيهِ دَاءٌ.
\par 11 وَالأَرْوَاحُ النَّجِسَةُ حِينَمَا نَظَرَتْهُ خَرَّتْ لَهُ وَصَرَخَتْ قَائِلَةً: «إِنَّكَ أَنْتَ ابْنُ اللَّهِ!»
\par 12 وَأَوْصَاهُمْ كَثِيراً أَنْ لاَ يُظْهِرُوهُ.
\par 13 ثُمَّ صَعِدَ إِلَى الْجَبَلِ وَدَعَا الَّذِينَ أَرَادَهُمْ فَذَهَبُوا إِلَيْهِ.
\par 14 وَأَقَامَ اثْنَيْ عَشَرَ لِيَكُونُوا مَعَهُ وَلْيُرْسِلَهُمْ لِيَكْرِزُوا
\par 15 وَيَكُونَ لَهُمْ سُلْطَانٌ عَلَى شِفَاءِ الأَمْرَاضِ وَإِخْرَاجِ الشَّيَاطِينِ.
\par 16 وَجَعَلَ لِسِمْعَانَ اسْمَ بُطْرُسَ.
\par 17 وَيَعْقُوبَ بْنَ زَبْدِي وَيُوحَنَّا أَخَا يَعْقُوبَ وَجَعَلَ لَهُمَا اسْمَ بُوَانَرْجِسَ (أَيِ ابْنَيِ الرَّعْدِ).
\par 18 وَأَنْدَرَاوُسَ وَفِيلُبُّسَ وَبَرْثُولَمَاوُسَ وَمَتَّى وَتُومَا وَيَعْقُوبَ بْنَ حَلْفَى وَتَدَّاوُسَ وَسِمْعَانَ الْقَانَوِيَّ
\par 19 وَيَهُوذَا الإِسْخَرْيُوطِيَّ الَّذِي أَسْلَمَهُ. ثُمَّ أَتَوْا إِلَى بَيْتٍ.
\par 20 فَاجْتَمَعَ أَيْضاً جَمْعٌ حَتَّى لَمْ يَقْدِرُوا وَلاَ عَلَى أَكْلِ خُبْزٍ.
\par 21 وَلَمَّا سَمِعَ أَقْرِبَاؤُهُ خَرَجُوا لِيُمْسِكُوهُ لأَنَّهُمْ قَالُوا: «إِنَّهُ مُخْتَلٌّ!».
\par 22 وَأَمَّا الْكَتَبَةُ الَّذِينَ نَزَلُوا مِنْ أُورُشَلِيمَ فَقَالُوا: «إِنَّ مَعَهُ بَعْلَزَبُولَ وَإِنَّهُ بِرَئِيسِ الشَّيَاطِينِ يُخْرِجُ الشَّيَاطِينَ».
\par 23 فَدَعَاهُمْ وَقَالَ لَهُمْ بِأَمْثَالٍ: «كَيْفَ يَقْدِرُ شَيْطَانٌ أَنْ يُخْرِجَ شَيْطَاناً؟
\par 24 وَإِنِ انْقَسَمَتْ مَمْلَكَةٌ عَلَى ذَاتِهَا لاَ تَقْدِرُ تِلْكَ الْمَمْلَكَةُ أَنْ تَثْبُتَ.
\par 25 وَإِنِ انْقَسَمَ بَيْتٌ عَلَى ذَاتِهِ لاَ يَقْدِرُ ذَلِكَ الْبَيْتُ أَنْ يَثْبُتَ.
\par 26 وَإِنْ قَامَ الشَّيْطَانُ عَلَى ذَاتِهِ وَانْقَسَمَ لاَ يَقْدِرُ أَنْ يَثْبُتَ بَلْ يَكُونُ لَهُ انْقِضَاءٌ.
\par 27 لاَ يَسْتَطِيعُ أَحَدٌ أَنْ يَدْخُلَ بَيْتَ قَوِيٍّ وَيَنْهَبَ أَمْتِعَتَهُ إِنْ لَمْ يَرْبِطِ الْقَوِيَّ أَوَّلاً وَحِينَئِذٍ يَنْهَبُ بَيْتَهُ.
\par 28 اَلْحَقَّ أَقُولُ لَكُمْ: إِنَّ جَمِيعَ الْخَطَايَا تُغْفَرُ لِبَنِي الْبَشَرِ وَالتَّجَادِيفَ الَّتِي يُجَدِّفُونَهَا.
\par 29 وَلَكِنْ مَنْ جَدَّفَ عَلَى الرُّوحِ الْقُدُسِ فَلَيْسَ لَهُ مَغْفِرَةٌ إِلَى الأَبَدِ بَلْ هُوَ مُسْتَوْجِبٌ دَيْنُونَةً أَبَدِيَّةً».
\par 30 لأَنَّهُمْ قَالُوا: «إِنَّ مَعَهُ رُوحاً نَجِساً».
\par 31 فَجَاءَتْ حِينَئِذٍ إِخْوَتُهُ وَأُمُّهُ وَوَقَفُوا خَارِجاً وَأَرْسَلُوا إِلَيْهِ يَدْعُونَهُ.
\par 32 وَكَانَ الْجَمْعُ جَالِساً حَوْلَهُ فَقَالُوا لَهُ: «هُوَذَا أُمُّكَ وَإِخْوَتُكَ خَارِجاً يَطْلُبُونَكَ».
\par 33 فَأَجَابَهُمْ: «مَنْ أُمِّي وَإِخْوَتِي؟»
\par 34 ثُمَّ نَظَرَ حَوْلَهُ إِلَى الْجَالِسِينَ وَقَالَ: «هَا أُمِّي وَإِخْوَتِي
\par 35 لأَنَّ مَنْ يَصْنَعُ مَشِيئَةَ اللَّهِ هُوَ أَخِي وَأُخْتِي وَأُمِّي».

\chapter{4}

\par 1 وَابْتَدَأَ أَيْضاً يُعَلِّمُ عِنْدَ الْبَحْرِ فَاجْتَمَعَ إِلَيْهِ جَمْعٌ كَثِيرٌ حَتَّى إِنَّهُ دَخَلَ السَّفِينَةَ وَجَلَسَ عَلَى الْبَحْرِ وَالْجَمْعُ كُلُّهُ كَانَ عِنْدَ الْبَحْرِ عَلَى الأَرْضِ.
\par 2 فَكَانَ يُعَلِّمُهُمْ كَثِيراً بِأَمْثَالٍ. وَقَالَ لَهُمْ فِي تَعْلِيمِهِ:
\par 3 «اسْمَعُوا. هُوَذَا الزَّارِعُ قَدْ خَرَجَ لِيَزْرَعَ
\par 4 وَفِيمَا هُوَ يَزْرَعُ سَقَطَ بَعْضٌ عَلَى الطَّرِيقِ فَجَاءَتْ طُيُورُ السَّمَاءِ وَأَكَلَتْهُ.
\par 5 وَسَقَطَ آخَرُ عَلَى مَكَانٍ مُحْجِرٍ حَيْثُ لَمْ تَكُنْ لَهُ تُرْبَةٌ كَثِيرَةٌ فَنَبَتَ حَالاً إِذْ لَمْ يَكُنْ لَهُ عُمْقُ أَرْضٍ.
\par 6 وَلَكِنْ لَمَّا أَشْرَقَتِ الشَّمْسُ احْتَرَقَ وَإِذْ لَمْ يَكُنْ لَهُ أَصْلٌ جَفَّ.
\par 7 وَسَقَطَ آخَرُ فِي الشَّوْكِ فَطَلَعَ الشَّوْكُ وَخَنَقَهُ فَلَمْ يُعْطِ ثَمَراً.
\par 8 وَسَقَطَ آخَرُ فِي الأَرْضِ الْجَيِّدَةِ فَأَعْطَى ثَمَراً يَصْعَدُ وَيَنْمُو فَأَتَى وَاحِدٌ بِثَلاَثِينَ وَآخَرُ بِسِتِّينَ وَآخَرُ بِمِئَةٍ».
\par 9 ثُمَّ قَالَ لَهُمْ: «مَنْ لَهُ أُذُنَانِ لِلسَّمْعِ فَلْيَسْمَعْ!»
\par 10 وَلَمَّا كَانَ وَحْدَهُ سَأَلَهُ الَّذِينَ حَوْلَهُ مَعَ الاِثْنَيْ عَشَرَ عَنِ الْمَثَلِ
\par 11 فَقَالَ لَهُمْ: «قَدْ أُعْطِيَ لَكُمْ أَنْ تَعْرِفُوا سِرَّ مَلَكُوتِ اللَّهِ. وَأَمَّا الَّذِينَ هُمْ مِنْ خَارِجٍ فَبِالأَمْثَالِ يَكُونُ لَهُمْ كُلُّ شَيْءٍ
\par 12 لِكَيْ يُبْصِرُوا مُبْصِرِينَ وَلاَ يَنْظُرُوا وَيَسْمَعُوا سَامِعِينَ وَلاَ يَفْهَمُوا لِئَلاَّ يَرْجِعُوا فَتُغْفَرَ لَهُمْ خَطَايَاهُمْ».
\par 13 ثُمَّ قَالَ لَهُمْ: «أَمَا تَعْلَمُونَ هَذَا الْمَثَلَ؟ فَكَيْفَ تَعْرِفُونَ جَمِيعَ الأَمْثَالِ؟
\par 14 اَلزَّارِعُ يَزْرَعُ الْكَلِمَةَ.
\par 15 وَهَؤُلاَءِ هُمُ الَّذِينَ عَلَى الطَّرِيقِ: حَيْثُ تُزْرَعُ الْكَلِمَةُ َحِينَمَا يَسْمَعُونَ يَأْتِي الشَّيْطَانُ لِلْوَقْتِ وَيَنْزِعُ الْكَلِمَةَ الْمَزْرُوعَةَ فِي قُلُوبِهِمْ.
\par 16 وَهَؤُلاَءِ كَذَلِكَ هُمُ الَّذِينَ زُرِعُوا عَلَى الأَمَاكِنِ الْمُحْجِرَةِ: الَّذِينَ حِينَمَا يَسْمَعُونَ الْكَلِمَةَ يَقْبَلُونَهَا لِلْوَقْتِ بِفَرَحٍ
\par 17 وَلَكِنْ لَيْسَ لَهُمْ أَصْلٌ فِي ذَوَاتِهِمْ بَلْ هُمْ إِلَى حِينٍ. فَبَعْدَ ذَلِكَ إِذَا حَدَثَ ضِيقٌ أَوِ اضْطِهَادٌ مِنْ أَجْلِ الْكَلِمَةِ فَلِلْوَقْتِ يَعْثُرُونَ.
\par 18 وَهَؤُلاَءِ هُمُ الَّذِينَ زُرِعُوا بَيْنَ الشَّوْكِ: هَؤُلاَءِ هُمُ الَّذِينَ يَسْمَعُونَ الْكَلِمَةَ
\par 19 وَهُمُومُ هَذَا الْعَالَمِ وَغُرُورُ الْغِنَى وَشَهَوَاتُ سَائِرِ الأَشْيَاءِ تَدْخُلُ وَتَخْنُقُ الْكَلِمَةَ فَتَصِيرُ بِلاَ ثَمَرٍ.
\par 20 وَهَؤُلاَءِ هُمُ الَّذِينَ زُرِعُوا عَلَى الأَرْضِ الْجَيِّدَةِ: الَّذِينَ يَسْمَعُونَ الْكَلِمَةَ وَيَقْبَلُونَهَا وَيُثْمِرُونَ وَاحِدٌ ثَلاَثِينَ وَآخَرُ سِتِّينَ وَآخَرُ مِئَةً».
\par 21 ثُمَّ قَالَ لَهُمْ: «هَلْ يُؤْتَى بِسِرَاجٍ لِيُوضَعَ تَحْتَ الْمِكْيَالِ أَوْ تَحْتَ السَّرِيرِ؟ أَلَيْسَ لِيُوضَعَ عَلَى الْمَنَارَةِ؟
\par 22 لأَنَّهُ لَيْسَ شَيْءٌ خَفِيٌّ لاَ يُظْهَرُ وَلاَ صَارَ مَكْتُوماً إلاَّ لِيُعْلَنَ.
\par 23 إِنْ كَانَ لأَحَدٍ أُذُنَانِ لِلسَّمْعِ فَلْيَسْمَعْ!»
\par 24 وَقَالَ لَهُمُ: «انْظُرُوا مَا تَسْمَعُونَ! بِالْكَيْلِ الَّذِي بِهِ تَكِيلُونَ يُكَالُ لَكُمْ وَيُزَادُ لَكُمْ أَيُّهَا السَّامِعُونَ.
\par 25 لأَنَّ مَنْ لَهُ سَيُعْطَى وَأَمَّا مَنْ لَيْسَ لَهُ فَالَّذِي عِنْدَهُ سَيُؤْخَذُ مِنْهُ».
\par 26 وَقَالَ: «هَكَذَا مَلَكُوتُ اللَّهِ: كَأَنَّ إِنْسَاناً يُلْقِي الْبِذَارَ عَلَى الأَرْضِ
\par 27 وَيَنَامُ وَيَقُومُ لَيْلاً وَنَهَاراً وَالْبِذَارُ يَطْلُعُ وَيَنْمُو وَهُوَ لاَ يَعْلَمُ كَيْفَ
\par 28 لأَنَّ الأَرْضَ مِنْ ذَاتِهَا تَأْتِي بِثَمَرٍ. أَوَّلاً نَبَاتاً ثُمَّ سُنْبُلاً ثُمَّ قَمْحاً مَلآنَ فِي السُّنْبُلِ.
\par 29 وَأَمَّا مَتَى أَدْرَكَ الثَّمَرُ فَلِلْوَقْتِ يُرْسِلُ الْمِنْجَلَ لأَنَّ الْحَصَادَ قَدْ حَضَرَ».
\par 30 وَقَالَ: «بِمَاذَا نُشَبِّهُ مَلَكُوتَ اللَّهِ أَوْ بِأَيِّ مَثَلٍ نُمَثِّلُهُ؟
\par 31 مِثْلُ حَبَّةِ خَرْدَلٍ مَتَى زُرِعَتْ فِي الأَرْضِ فَهِيَ أَصْغَرُ جَمِيعِ الْبُزُورِ الَّتِي عَلَى الأَرْضِ.
\par 32 وَلَكِنْ مَتَى زُرِعَتْ تَطْلُعُ وَتَصِيرُ أَكْبَرَ جَمِيعِ الْبُقُولِ وَتَصْنَعُ أَغْصَاناً كَبِيرَةً حَتَّى تَسْتَطِيعَ طُيُورُ السَّمَاءِ أَنْ تَتَآوَى تَحْتَ ظِلِّهَا».
\par 33 وَبِأَمْثَالٍ كَثِيرَةٍ مِثْلِ هَذِهِ كَانَ يُكَلِّمُهُمْ حَسْبَمَا كَانُوا يَسْتَطِيعُونَ أَنْ يَسْمَعُوا
\par 34 وَبِدُونِ مَثَلٍ لَمْ يَكُنْ يُكَلِّمُهُمْ. وَأَمَّا عَلَى انْفِرَادٍ فَكَانَ يُفَسِّرُ لِتَلاَمِيذِهِ كُلَّ شَيْءٍ.
\par 35 وَقَالَ لَهُمْ فِي ذَلِكَ الْيَوْمِ لَمَّا كَانَ الْمَسَاءُ: «لِنَجْتَزْ إِلَى الْعَبْرِ».
\par 36 فَصَرَفُوا الْجَمْعَ وَأَخَذُوهُ كَمَا كَانَ فِي السَّفِينَةِ. وَكَانَتْ مَعَهُ أَيْضاً سُفُنٌ أُخْرَى صَغِيرَةٌ.
\par 37 فَحَدَثَ نَوْءُ رِيحٍ عَظِيمٌ فَكَانَتِ الأَمْوَاجُ تَضْرِبُ إِلَى السَّفِينَةِ حَتَّى صَارَتْ تَمْتَلِئُ.
\par 38 وَكَانَ هُوَ فِي الْمُؤَخَّرِ عَلَى وِسَادَةٍ نَائِماً. فَأَيْقَظُوهُ وَقَالُوا لَهُ: «يَا مُعَلِّمُ أَمَا يَهُمُّكَ أَنَّنَا نَهْلِكُ؟»
\par 39 فَقَامَ وَانْتَهَرَ الرِّيحَ وَقَالَ لِلْبَحْرِ: «اسْكُتْ. ابْكَمْ». فَسَكَنَتِ الرِّيحُ وَصَارَ هُدُوءٌ عَظِيمٌ.
\par 40 وَقَالَ لَهُمْ: «مَا بَالُكُمْ خَائِفِينَ هَكَذَا؟ كَيْفَ لاَ إِيمَانَ لَكُمْ؟»
\par 41 فَخَافُوا خَوْفاً عَظِيماً وَقَالُوا بَعْضُهُمْ لِبَعْضٍ: «مَنْ هُوَ هَذَا؟ فَإِنَّ الرِّيحَ أَيْضاً وَالْبَحْرَ يُطِيعَانِهِ!».

\chapter{5}

\par 1 وَجَاءُوا إِلَى عَبْرِ الْبَحْرِ إِلَى كُورَةِ الْجَدَرِيِّينَ.
\par 2 وَلَمَّا خَرَجَ مِنَ السَّفِينَةِ لِلْوَقْتِ اسْتَقْبَلَهُ مِنَ الْقُبُورِ إِنْسَانٌ بِهِ رُوحٌ نَجِسٌ
\par 3 كَانَ مَسْكَنُهُ فِي الْقُبُورِ وَلَمْ يَقْدِرْ أَحَدٌ أَنْ يَرْبِطَهُ وَلاَ بِسَلاَسِلَ
\par 4 لأَنَّهُ قَدْ رُبِطَ كَثِيراً بِقُيُودٍ وَسَلاَسِلَ فَقَطَّعَ السَّلاَسِلَ وَكَسَّرَ الْقُيُودَ فَلَمْ يَقْدِرْ أَحَدٌ أَنْ يُذَلِّلَهُ.
\par 5 وَكَانَ دَائِماً لَيْلاً وَنَهَاراً فِي الْجِبَالِ وَفِي الْقُبُورِ يَصِيحُ وَيُجَرِّحُ نَفْسَهُ بِالْحِجَارَةِ.
\par 6 فَلَمَّا رَأَى يَسُوعَ مِنْ بَعِيدٍ رَكَضَ وَسَجَدَ لَهُ
\par 7 وَصَرَخَ بِصَوْتٍ عَظِيمٍ: «مَا لِي وَلَكَ يَا يَسُوعُ ابْنَ اللَّهِ الْعَلِيِّ! أَسْتَحْلِفُكَ بِاللَّهِ أَنْ لاَ تُعَذِّبَنِي!»
\par 8 لأَنَّهُ قَالَ لَهُ: «اخْرُجْ مِنَ الإِنْسَانِ يَا أَيُّهَا الرُّوحُ النَّجِسُ».
\par 9 وَسَأَلَهُ: «مَا اسْمُكَ؟» فَأَجَابَ: «اسْمِي لَجِئُونُ لأَنَّنَا كَثِيرُونَ».
\par 10 وَطَلَبَ إِلَيْهِ كَثِيراً أَنْ لاَ يُرْسِلَهُمْ إِلَى خَارِجِ الْكُورَةِ.
\par 11 وَكَانَ هُنَاكَ عِنْدَ الْجِبَالِ قَطِيعٌ كَبِيرٌ مِنَ الْخَنَازِيرِ يَرْعَى
\par 12 فَطَلَبَ إِلَيْهِ كُلُّ الشَّيَاطِينِ قَائِلِينَ: «أَرْسِلْنَا إِلَى الْخَنَازِيرِ لِنَدْخُلَ فِيهَا».
\par 13 فَأَذِنَ لَهُمْ يَسُوعُ لِلْوَقْتِ. فَخَرَجَتِ الأَرْوَاحُ النَّجِسَةُ وَدَخَلَتْ فِي الْخَنَازِيرِ فَانْدَفَعَ الْقَطِيعُ مِنْ عَلَى الْجُرْفِ إِلَى الْبَحْرِ - وَكَانَ نَحْوَ أَلْفَيْنِ فَاخْتَنَقَ فِي الْبَحْرِ.
\par 14 وَأَمَّا رُعَاةُ الْخَنَازِيرِ فَهَرَبُوا وَأَخْبَرُوا فِي الْمَدِينَةِ وَفِي الضِّيَاعِ فَخَرَجُوا لِيَرَوْا مَا جَرَى.
\par 15 وَجَاءُوا إِلَى يَسُوعَ فَنَظَرُوا الْمَجْنُونَ الَّذِي كَانَ فِيهِ اللَّجِئُونُ جَالِساً وَلاَبِساً وَعَاقِلاً فَخَافُوا.
\par 16 فَحَدَّثَهُمُ الَّذِينَ رَأَوْا كَيْفَ جَرَى لِلْمَجْنُونِ وَعَنِ الْخَنَازِيرِ.
\par 17 فَابْتَدَأُوا يَطْلُبُونَ إِلَيْهِ أَنْ يَمْضِيَ مِنْ تُخُومِهِمْ.
\par 18 وَلَمَّا دَخَلَ السَّفِينَةَ طَلَبَ إِلَيْهِ الَّذِي كَانَ مَجْنُوناً أَنْ يَكُونَ مَعَهُ
\par 19 فَلَمْ يَدَعْهُ يَسُوعُ بَلْ قَالَ لَهُ: «اذْهَبْ إِلَى بَيْتِكَ وَإِلَى أَهْلِكَ وَأَخْبِرْهُمْ كَمْ صَنَعَ الرَّبُّ بِكَ وَرَحِمَكَ».
\par 20 فَمَضَى وَابْتَدَأَ يُنَادِي فِي الْعَشْرِ الْمُدُنِ كَمْ صَنَعَ بِهِ يَسُوعُ. فَتَعَجَّبَ الْجَمِيعُ.
\par 21 وَلَمَّا اجْتَازَ يَسُوعُ فِي السَّفِينَةِ أَيْضاً إِلَى الْعَبْرِ اجْتَمَعَ إِلَيْهِ جَمْعٌ كَثِيرٌ وَكَانَ عِنْدَ الْبَحْرِ.
\par 22 وَإِذَا وَاحِدٌ مِنْ رُؤَسَاءِ الْمَجْمَعِ اسْمُهُ يَايِرُسُ جَاءَ. وَلَمَّا رَآهُ خَرَّ عِنْدَ قَدَمَيْهِ
\par 23 وَطَلَبَ إِلَيْهِ كَثِيراً قَائِلاً: «ابْنَتِي الصَّغِيرَةُ عَلَى آخِرِ نَسَمَةٍ. لَيْتَكَ تَأْتِي وَتَضَعُ يَدَكَ عَلَيْهَا لِتُشْفَى فَتَحْيَا».
\par 24 فَمَضَى مَعَهُ وَتَبِعَهُ جَمْعٌ كَثِيرٌ وَكَانُوا يَزْحَمُونَهُ.
\par 25 وَامْرَأَةٌ بِنَزْفِ دَمٍ مُنْذُ اثْنَتَيْ عَشْرَةَ سَنَةً
\par 26 وَقَدْ تَأَلَّمَتْ كَثِيراً مِنْ أَطِبَّاءَ كَثِيرِينَ وَأَنْفَقَتْ كُلَّ مَا عِنْدَهَا وَلَمْ تَنْتَفِعْ شَيْئاً بَلْ صَارَتْ إِلَى حَالٍ أَرْدَأَ -
\par 27 لَمَّا سَمِعَتْ بِيَسُوعَ جَاءَتْ فِي الْجَمْعِ مِنْ وَرَاءٍ وَمَسَّتْ ثَوْبَهُ
\par 28 لأَنَّهَا قَالَتْ: «إِنْ مَسَسْتُ وَلَوْ ثِيَابَهُ شُفِيتُ».
\par 29 فَلِلْوَقْتِ جَفَّ يَنْبُوعُ دَمِهَا وَعَلِمَتْ فِي جِسْمِهَا أَنَّهَا قَدْ بَرِئَتْ مِنَ الدَّاءِ.
\par 30 فَلِلْوَقْتِ الْتَفَتَ يَسُوعُ بَيْنَ الْجَمْعِ شَاعِراً فِي نَفْسِهِ بِالْقُوَّةِ الَّتِي خَرَجَتْ مِنْهُ وَقَالَ: «مَنْ لَمَسَ ثِيَابِي؟»
\par 31 فَقَالَ لَهُ تَلاَمِيذُهُ: «أَنْتَ تَنْظُرُ الْجَمْعَ يَزْحَمُكَ وَتَقُولُ مَنْ لَمَسَنِي؟»
\par 32 وَكَانَ يَنْظُرُ حَوْلَهُ لِيَرَى الَّتِي فَعَلَتْ هَذَا.
\par 33 وَأَمَّا الْمَرْأَةُ فَجَاءَتْ وَهِيَ خَائِفَةٌ وَمُرْتَعِدَةٌ عَالِمَةً بِمَا حَصَلَ لَهَا فَخَرَّتْ وَقَالَتْ لَهُ الْحَقَّ كُلَّهُ.
\par 34 فَقَالَ لَهَا: «يَا ابْنَةُ إِيمَانُكِ قَدْ شَفَاكِ. اذْهَبِي بِسَلاَمٍ وَكُونِي صَحِيحَةً مِنْ دَائِكِ».
\par 35 وَبَيْنَمَا هُوَ يَتَكَلَّمُ جَاءُوا مِنْ دَارِ رَئِيسِ الْمَجْمَعِ قَائِلِينَ: «ابْنَتُكَ مَاتَتْ. لِمَاذَا تُتْعِبُ الْمُعَلِّمَ بَعْدُ؟»
\par 36 فَسَمِعَ يَسُوعُ لِوَقْتِهِ الْكَلِمَةَ الَّتِي قِيلَتْ فَقَالَ لِرَئِيسِ الْمَجْمَعِ: «لاَ تَخَفْ. آمِنْ فَقَطْ».
\par 37 وَلَمْ يَدَعْ أَحَداً يَتْبَعُهُ إلاَّ بُطْرُسَ وَيَعْقُوبَ وَيُوحَنَّا أَخَا يَعْقُوبَ.
\par 38 فَجَاءَ إِلَى بَيْتِ رَئِيسِ الْمَجْمَعِ وَرَأَى ضَجِيجاً. يَبْكُونَ وَيُوَلْوِلُونَ كَثِيراً.
\par 39 فَدَخَلَ وَقَالَ لَهُمْ: «لِمَاذَا تَضِجُّونَ وَتَبْكُونَ؟ لَمْ تَمُتِ الصَّبِيَّةُ لَكِنَّهَا نَائِمَةٌ».
\par 40 فَضَحِكُوا عَلَيْهِ. أَمَّا هُوَ فَأَخْرَجَ الْجَمِيعَ وَأَخَذَ أَبَا الصَّبِيَّةِ وَأُمَّهَا وَالَّذِينَ مَعَهُ وَدَخَلَ حَيْثُ كَانَتِ الصَّبِيَّةُ مُضْطَجِعَةً
\par 41 وَأَمْسَكَ بِيَدِ الصَّبِيَّةِ وَقَالَ لَهَا: «طَلِيثَا قُومِي». (الَّذِي تَفْسِيرُهُ: يَا صَبِيَّةُ لَكِ أَقُولُ قُومِي).
\par 42 وَلِلْوَقْتِ قَامَتِ الصَّبِيَّةُ وَمَشَتْ لأَنَّهَا كَانَتِ ابْنَةَ اثْنَتَيْ عَشْرَةَ سَنَةً. فَبُهِتُوا بَهَتاً عَظِيماً.
\par 43 فَأَوْصَاهُمْ كَثِيراً أَنْ لاَ يَعْلَمَ أَحَدٌ بِذَلِكَ. وَقَالَ أَنْ تُعْطَى لِتَأْكُلَ.

\chapter{6}

\par 1 وَخَرَجَ مِنْ هُنَاكَ وَجَاءَ إِلَى وَطَنِهِ وَتَبِعَهُ تَلاَمِيذُهُ.
\par 2 وَلَمَّا كَانَ السَّبْتُ ابْتَدَأَ يُعَلِّمُ فِي الْمَجْمَعِ. وَكَثِيرُونَ إِذْ سَمِعُوا بُهِتُوا قَائِلِينَ: «مِنْ أَيْنَ لِهَذَا هَذِهِ؟ وَمَا هَذِهِ الْحِكْمَةُ الَّتِي أُعْطِيَتْ لَهُ حَتَّى تَجْرِيَ عَلَى يَدَيْهِ قُوَّاتٌ مِثْلُ هَذِهِ؟
\par 3 أَلَيْسَ هَذَا هُوَ النَّجَّارَ ابْنَ مَرْيَمَ وَأَخَا يَعْقُوبَ وَيُوسِي وَيَهُوذَا وَسِمْعَانَ؟ أَوَلَيْسَتْ أَخَوَاتُهُ هَهُنَا عِنْدَنَا؟» فَكَانُوا يَعْثُرُونَ بِهِ.
\par 4 فَقَالَ لَهُمْ يَسُوعُ: «لَيْسَ نَبِيٌّ بِلاَ كَرَامَةٍ إلاَّ فِي وَطَنِهِ وَبَيْنَ أَقْرِبَائِهِ وَفِي بَيْتِهِ».
\par 5 وَلَمْ يَقْدِرْ أَنْ يَصْنَعَ هُنَاكَ وَلاَ قُوَّةً وَاحِدَةً غَيْرَ أَنَّهُ وَضَعَ يَدَيْهِ عَلَى مَرْضَى قَلِيلِينَ فَشَفَاهُمْ.
\par 6 وَتَعَجَّبَ مِنْ عَدَمِ إِيمَانِهِمْ. وَصَارَ يَطُوفُ الْقُرَى الْمُحِيطَةَ يُعَلِّمُ.
\par 7 وَدَعَا الاِثْنَيْ عَشَرَ وَابْتَدَأَ يُرْسِلُهُمُ اثْنَيْنِ اثْنَيْنِ وَأَعْطَاهُمْ سُلْطَاناً عَلَى الأَرْوَاحِ النَّجِسَةِ
\par 8 وَأَوْصَاهُمْ أَنْ لاَ يَحْمِلُوا شَيْئاً لِلطَّرِيقِ غَيْرَ عَصاً فَقَطْ لاَ مِزْوَداً وَلاَ خُبْزاً وَلاَ نُحَاساً فِي الْمِنْطَقَةِ.
\par 9 بَلْ يَكُونُوا مَشْدُودِينَ بِنِعَالٍ وَلاَ يَلْبَسُوا ثَوْبَيْنِ.
\par 10 وَقَالَ لَهُمْ: «حَيْثُمَا دَخَلْتُمْ بَيْتاً فَأَقِيمُوا فِيهِ حَتَّى تَخْرُجُوا مِنْ هُنَاكَ.
\par 11 وَكُلُّ مَنْ لاَ يَقْبَلُكُمْ وَلاَ يَسْمَعُ لَكُمْ فَاخْرُجُوا مِنْ هُنَاكَ وَانْفُضُوا التُّرَابَ الَّذِي تَحْتَ أَرْجُلِكُمْ شَهَادَةً عَلَيْهِمْ. اَلْحَقَّ أَقُولُ لَكُمْ: سَتَكُونُ لأَرْضِ سَدُومَ وَعَمُورَةَ يَوْمَ الدِّينِ حَالَةٌ أَكْثَرُ احْتِمَالاً مِمَّا لِتِلْكَ الْمَدِينَةِ».
\par 12 فَخَرَجُوا وَصَارُوا يَكْرِزُونَ أَنْ يَتُوبُوا.
\par 13 وَأَخْرَجُوا شَيَاطِينَ كَثِيرَةً وَدَهَنُوا بِزَيْتٍ مَرْضَى كَثِيرِينَ فَشَفَوْهُمْ.
\par 14 فَسَمِعَ هِيرُودُسُ الْمَلِكُ لأَنَّ اسْمَهُ صَارَ مَشْهُوراً. وَقَالَ: «إِنَّ يُوحَنَّا الْمَعْمَدَانَ قَامَ مِنَ الأَمْوَاتِ وَلِذَلِكَ تُعْمَلُ بِهِ الْقُوَّاتُ».
\par 15 قَالَ آخَرُونَ: «إِنَّهُ إِيلِيَّا». وَقَالَ آخَرُونَ: «إِنَّهُ نَبِيٌّ أَوْ كَأَحَدِ الأَنْبِيَاءِ».
\par 16 وَلَكِنْ لَمَّا سَمِعَ هِيرُودُسُ قَالَ: «هَذَا هُوَ يُوحَنَّا الَّذِي قَطَعْتُ أَنَا رَأْسَهُ. إِنَّهُ قَامَ مِنَ الأَمْوَاتِ!»
\par 17 لأَنَّ هِيرُودُسَ نَفْسَهُ كَانَ قَدْ أَرْسَلَ وَأَمْسَكَ يُوحَنَّا وَأَوْثَقَهُ فِي السِّجْنِ مِنْ أَجْلِ هِيرُودِيَّا امْرَأَةِ فِيلُبُّسَ أَخِيهِ إِذْ كَانَ قَدْ تَزَوَّجَ بِهَا.
\par 18 لأَنَّ يُوحَنَّا كَانَ يَقُولُ لِهِيرُودُسَ: «لاَ يَحِلُّ أَنْ تَكُونَ لَكَ امْرَأَةُ أَخِيكَ!»
\par 19 فَحَنِقَتْ هِيرُودِيَّا عَلَيْهِ وَأَرَادَتْ أَنْ تَقْتُلَهُ وَلَمْ تَقْدِرْ
\par 20 لأَنَّ هِيرُودُسَ كَانَ يَهَابُ يُوحَنَّا عَالِماً أَنَّهُ رَجُلٌ بَارٌّ وَقِدِّيسٌ وَكَانَ يَحْفَظُهُ. وَإِذْ سَمِعَهُ فَعَلَ كَثِيراً وَسَمِعَهُ بِسُرُورٍ.
\par 21 وَإِذْ كَانَ يَوْمٌ مُوافِقٌ لَمَّا صَنَعَ هِيرُودُسُ فِي مَوْلِدِهِ عَشَاءً لِعُظَمَائِهِ وَقُوَّادِ الأُلُوفِ وَوُجُوهِ الْجَلِيلِ
\par 22 دَخَلَتِ ابْنَةُ هِيرُودِيَّا وَرَقَصَتْ فَسَرَّتْ هِيرُودُسَ وَالْمُتَّكِئِينَ مَعَهُ. فَقَالَ الْمَلِكُ لِلصَّبِيَّةِ: «مَهْمَا أَرَدْتِ اطْلُبِي مِنِّي فَأُعْطِيَكِ».
\par 23 وَأَقْسَمَ لَهَا أَنْ «مَهْمَا طَلَبْتِ مِنِّي لَأُعْطِيَنَّكِ حَتَّى نِصْفَ مَمْلَكَتِي».
\par 24 فَخَرَجَتْ وَقَالَتْ لِأُمِّهَا: «مَاذَا أَطْلُبُ؟» فَقَالَتْ: «رَأْسَ يُوحَنَّا الْمَعْمَدَانِ».
\par 25 فَدَخَلَتْ لِلْوَقْتِ بِسُرْعَةٍ إِلَى الْمَلِكِ وَطَلَبَتْ قَائِلَةً: «أُرِيدُ أَنْ تُعْطِيَنِي حَالاً رَأْسَ يُوحَنَّا الْمَعْمَدَانِ عَلَى طَبَقٍ».
\par 26 فَحَزِنَ الْمَلِكُ جِدّاً. وَلأَجْلِ الأَقْسَامِ وَالْمُتَّكِئِينَ لَمْ يُرِدْ أَنْ يَرُدَّهَا.
\par 27 فَلِلْوَقْتِ أَرْسَلَ الْمَلِكُ سَيَّافاً وَأَمَرَ أَنْ يُؤْتَى بِرَأْسِهِ.
\par 28 فَمَضَى وَقَطَعَ رَأْسَهُ فِي السِّجْنِ. وَأَتَى بِرَأْسِهِ عَلَى طَبَقٍ وَأَعْطَاهُ لِلصَّبِيَّةِ وَالصَّبِيَّةُ أَعْطَتْهُ لِأُمِّهَا.
\par 29 وَلَمَّا سَمِعَ تَلاَمِيذُهُ جَاءُوا وَرَفَعُوا جُثَّتَهُ وَوَضَعُوهَا فِي قَبْرٍ.
\par 30 وَاجْتَمَعَ الرُّسُلُ إِلَى يَسُوعَ وَأَخْبَرُوهُ بِكُلِّ شَيْءٍ كُلِّ مَا فَعَلُوا وَكُلِّ مَا عَلَّمُوا.
\par 31 فَقَالَ لَهُمْ: «تَعَالَوْا أَنْتُمْ مُنْفَرِدِينَ إِلَى مَوْضِعٍ خَلاَءٍ وَاسْتَرِيحُوا قَلِيلاً». لأَنَّ الْقَادِمِينَ وَالذَّاهِبِينَ كَانُوا كَثِيرِينَ وَلَمْ تَتَيَسَّرْ لَهُمْ فُرْصَةٌ لِلأَكْلِ.
\par 32 فَمَضَوْا فِي السَّفِينَةِ إِلَى مَوْضِعٍ خَلاَءٍ مُنْفَرِدِينَ.
\par 33 فَرَآهُمُ الْجُمُوعُ مُنْطَلِقِينَ وَعَرَفَهُ كَثِيرُونَ. فَتَرَاكَضُوا إِلَى هُنَاكَ مِنْ جَمِيعِ الْمُدُنِ مُشَاةً وَسَبَقُوهُمْ وَاجْتَمَعُوا إِلَيْهِ.
\par 34 فَلَمَّا خَرَجَ يَسُوعُ رَأَى جَمْعاً كَثِيراً فَتَحَنَّنَ عَلَيْهِمْ إِذْ كَانُوا كَخِرَافٍ لاَ رَاعِيَ لَهَا فَابْتَدَأَ يُعَلِّمُهُمْ كَثِيراً.
\par 35 وَبَعْدَ سَاعَاتٍ كَثِيرَةٍ تَقَدَّمَ إِلَيْهِ تَلاَمِيذُهُ قَائِلِينَ: «الْمَوْضِعُ خَلاَءٌ وَالْوَقْتُ مَضَى.
\par 36 اصْرِفْهُمْ لِكَيْ يَمْضُوا إِلَى الضِّيَاعِ وَالْقُرَى حَوَالَيْنَا وَيَبْتَاعُوا لَهُمْ خُبْزاً لأَنْ لَيْسَ عِنْدَهُمْ مَا يَأْكُلُونَ».
\par 37 فَأَجَابَ: «أَعْطُوهُمْ أَنْتُمْ لِيَأْكُلُوا». فَقَالُوا لَهُ: «أَنَمْضِي وَنَبْتَاعُ خُبْزاً بِمِئَتَيْ دِينَارٍ وَنُعْطِيهُمْ لِيَأْكُلُوا؟»
\par 38 فَقَالَ لَهُمْ: «كَمْ رَغِيفاً عِنْدَكُمْ؟ اذْهَبُوا وَانْظُرُوا». وَلَمَّا عَلِمُوا قَالُوا: «خَمْسَةٌ وَسَمَكَتَانِ».
\par 39 فَأَمَرَهُمْ أَنْ يَجْعَلُوا الْجَمِيعَ يَتَّكِئُونَ رِفَاقاً رِفَاقاً عَلَى الْعُشْبِ الأَخْضَرِ.
\par 40 فَاتَّكَأُوا صُفُوفاً صُفُوفاً: مِئَةً مِئَةً وَخَمْسِينَ خَمْسِينَ.
\par 41 فَأَخَذَ الأَرْغِفَةَ الْخَمْسَةَ وَالسَّمَكَتَيْنِ وَرَفَعَ نَظَرَهُ نَحْوَ السَّمَاءِ وَبَارَكَ ثُمَّ كَسَّرَ الأَرْغِفَةَ وَأَعْطَى تَلاَمِيذَهُ لِيُقَدِّمُوا إِلَيْهِمْ وَقَسَّمَ السَّمَكَتَيْنِ لِلْجَمِيعِ
\par 42 فَأَكَلَ الْجَمِيعُ وَشَبِعُوا
\par 43 ثُمَّ رَفَعُوا مِنَ الْكِسَرِ اثْنَتَيْ عَشْرَةَ قُفَّةً مَمْلُوَّةً وَمِنَ السَّمَكِ.
\par 44 وَكَانَ الَّذِينَ أَكَلُوا مِنَ الأَرْغِفَةِ نَحْوَ خَمْسَةِ آلاَفِ رَجُلٍ.
\par 45 وَلِلْوَقْتِ أَلْزَمَ تَلاَمِيذَهُ أَنْ يَدْخُلُوا السَّفِينَةَ وَيَسْبِقُوا إِلَى الْعَبْرِ إِلَى بَيْتِ صَيْدَا حَتَّى يَكُونَ قَدْ صَرَفَ الْجَمْعَ.
\par 46 وَبَعْدَمَا وَدَّعَهُمْ مَضَى إِلَى الْجَبَلِ لِيُصَلِّيَ.
\par 47 وَلَمَّا صَارَ الْمَسَاءُ كَانَتِ السَّفِينَةُ فِي وَسَطِ الْبَحْرِ وَهُوَ عَلَى الْبَرِّ وَحْدَهُ.
\par 48 وَرَآهُمْ مُعَذَّبِينَ فِي الْجَذْفِ لأَنَّ الرِّيحَ كَانَتْ ضِدَّهُمْ. وَنَحْوَ الْهَزِيعِ الرَّابِعِ مِنَ اللَّيْلِ أَتَاهُمْ مَاشِياً عَلَى الْبَحْرِ وَأَرَادَ أَنْ يَتَجَاوَزَهُمْ.
\par 49 فَلَمَّا رَأَوْهُ مَاشِياً عَلَى الْبَحْرِ ظَنُّوهُ خَيَالاً فَصَرَخُوا
\par 50 لأَنَّ الْجَمِيعَ رَأَوْهُ وَاضْطَرَبُوا. فَلِلْوَقْتِ قَالَ لَهُمْ: «ثِقُوا. أَنَا هُوَ. لاَ تَخَافُوا».
\par 51 فَصَعِدَ إِلَيْهِمْ إِلَى السَّفِينَةِ فَسَكَنَتِ الرِّيحُ فَبُهِتُوا وَتَعَجَّبُوا فِي أَنْفُسِهِمْ جِدّاً إِلَى الْغَايَةِ
\par 52 لأَنَّهُمْ لَمْ يَفْهَمُوا بِالأَرْغِفَةِ إِذْ كَانَتْ قُلُوبُهُمْ غَلِيظَةً.
\par 53 فَلَمَّا عَبَرُوا جَاءُوا إِلَى أَرْضِ جَنِّيسَارَتَ وَأَرْسَوْا.
\par 54 وَلَمَّا خَرَجُوا مِنَ السَّفِينَةِ لِلْوَقْتِ عَرَفُوهُ
\par 55 فَطَافُوا جَمِيعَ تِلْكَ الْكُورَةِ الْمُحِيطَةِ وَابْتَدَأُوا يَحْمِلُونَ الْمَرْضَى عَلَى أَسِرَّةٍ إِلَى حَيْثُ سَمِعُوا أَنَّهُ هُنَاكَ.
\par 56 وَحَيْثُمَا دَخَلَ إِلَى قُرىً أَوْ مُدُنٍ أَوْ ضِيَاعٍ وَضَعُوا الْمَرْضَى فِي الأَسْوَاقِ وَطَلَبُوا إِلَيْهِ أَنْ يَلْمِسُوا وَلَوْ هُدْبَ ثَوْبِهِ. وَكُلُّ مَنْ لَمَسَهُ شُفِيَ!

\chapter{7}

\par 1 وَاجْتَمَعَ إِلَيْهِ الْفَرِّيسِيُّونَ وَقَوْمٌ مِنَ الْكَتَبَةِ قَادِمِينَ مِنْ أُورُشَلِيمَ.
\par 2 وَلَمَّا رَأَوْا بَعْضاً مِنْ تَلاَمِيذِهِ يَأْكُلُونَ خُبْزاً بِأَيْدٍ دَنِسَةٍ أَيْ غَيْرِ مَغْسُولَةٍ لاَمُوا -
\par 3 لأَنَّ الْفَرِّيسِيِّينَ وَكُلَّ الْيَهُودِ إِنْ لَمْ يَغْسِلُوا أَيْدِيَهُمْ بِاعْتِنَاءٍ لاَ يَأْكُلُونَ مُتَمَسِّكِينَ بِتَقْلِيدِ الشُّيُوخِ.
\par 4 وَمِنَ السُّوقِ إِنْ لَمْ يَغْتَسِلُوا لاَ يَأْكُلُونَ. وَأَشْيَاءُ أُخْرَى كَثِيرَةٌ تَسَلَّمُوهَا لِلتَّمَسُّكِ بِهَا مِنْ غَسْلِ كُؤُوسٍ وَأَبَارِيقَ وَآنِيَةِ نُحَاسٍ وَأَسِرَّةٍ.
\par 5 ثُمَّ سَأَلَهُ الْفَرِّيسِيُّونَ وَالْكَتَبَةُ: «لِمَاذَا لاَ يَسْلُكُ تَلاَمِيذُكَ حَسَبَ تَقْلِيدِ الشُّيُوخِ بَلْ يَأْكُلُونَ خُبْزاً بِأَيْدٍ غَيْرِ مَغْسُولَةٍ؟»
\par 6 فَأَجَابَ: «حَسَناً تَنَبَّأَ إِشَعْيَاءُ عَنْكُمْ أَنْتُمُ الْمُرَائِينَ كَمَا هُوَ مَكْتُوبٌ: هَذَا الشَّعْبُ يُكْرِمُنِي بِشَفَتَيْهِ وَأَمَّا قَلْبُهُ فَمُبْتَعِدٌ عَنِّي بَعِيداً
\par 7 وَبَاطِلاً يَعْبُدُونَنِي وَهُمْ يُعَلِّمُونَ تَعَالِيمَ هِيَ وَصَايَا النَّاسِ.
\par 8 لأَنَّكُمْ تَرَكْتُمْ وَصِيَّةَ اللَّهِ وَتَتَمَسَّكُونَ بِتَقْلِيدِ النَّاسِ: غَسْلَ الأَبَارِيقِ وَالْكُؤُوسِ وَأُمُوراً أُخَرَ كَثِيرَةً مِثْلَ هَذِهِ تَفْعَلُونَ».
\par 9 ثُمَّ قَالَ لَهُمْ: «حَسَناً! رَفَضْتُمْ وَصِيَّةَ اللَّهِ لِتَحْفَظُوا تَقْلِيدَكُمْ.
\par 10 لأَنَّ مُوسَى قَالَ: أَكْرِمْ أَبَاكَ وَأُمَّكَ وَمَنْ يَشْتِمُ أَباً أَوْ أُمّاً فَلْيَمُتْ مَوْتاً.
\par 11 وَأَمَّا أَنْتُمْ فَتَقُولُونَ: إِنْ قَالَ إِنْسَانٌ لأَبِيهِ أَوْ أُمِّهِ: قُرْبَانٌ أَيْ هَدِيَّةٌ هُوَ الَّذِي تَنْتَفِعُ بِهِ مِنِّي
\par 12 فَلاَ تَدَعُونَهُ فِي مَا بَعْدُ يَفْعَلُ شَيْئاً لأَبِيهِ أَوْ أُمِّهِ.
\par 13 مُبْطِلِينَ كَلاَمَ اللَّهِ بِتَقْلِيدِكُمُ الَّذِي سَلَّمْتُمُوهُ. وَأُمُوراً كَثِيرَةً مِثْلَ هَذِهِ تَفْعَلُونَ».
\par 14 ثُمَّ دَعَا كُلَّ الْجَمْعِ وَقَالَ لَهُمُ: «اسْمَعُوا مِنِّي كُلُّكُمْ وَافْهَمُوا.
\par 15 لَيْسَ شَيْءٌ مِنْ خَارِجِ الإِنْسَانِ إِذَا دَخَلَ فِيهِ يَقْدِرُ أَنْ يُنَجِّسَهُ لَكِنَّ الأَشْيَاءَ الَّتِي تَخْرُجُ مِنْهُ هِيَ الَّتِي تُنَجِّسُ الإِنْسَانَ.
\par 16 إِنْ كَانَ لأَحَدٍ أُذْنَانِ لِلسَّمْعِ فَلْيَسْمَعْ».
\par 17 وَلَمَّا دَخَلَ مِنْ عِنْدِ الْجَمْعِ إِلَى الْبَيْتِ سَأَلَهُ تَلاَمِيذُهُ عَنِ الْمَثَلِ.
\par 18 فَقَالَ لَهُمْ: «أَفَأَنْتُمْ أَيْضاً هَكَذَا غَيْرُ فَاهِمِينَ؟ أَمَا تَفْهَمُونَ أَنَّ كُلَّ مَا يَدْخُلُ الإِنْسَانَ مِنْ خَارِجٍ لاَ يَقْدِرُ أَنْ يُنَجِّسَهُ
\par 19 لأَنَّهُ لاَ يَدْخُلُ إِلَى قَلْبِهِ بَلْ إِلَى الْجَوْفِ ثُمَّ يَخْرُجُ إِلَى الْخَلاَءِ وَذَلِكَ يُطَهِّرُ كُلَّ الأَطْعِمَةِ».
\par 20 ثُمَّ قَالَ: «إِنَّ الَّذِي يَخْرُجُ مِنَ الإِنْسَانِ ذَلِكَ يُنَجِّسُ الإِنْسَانَ.
\par 21 لأَنَّهُ مِنَ الدَّاخِلِ مِنْ قُلُوبِ النَّاسِ تَخْرُجُ الأَفْكَارُ الشِّرِّيرَةُ: زِنىً فِسْقٌ قَتْلٌ
\par 22 سِرْقَةٌ طَمَعٌ خُبْثٌ مَكْرٌ عَهَارَةٌ عَيْنٌ شِرِّيرَةٌ تَجْدِيفٌ كِبْرِيَاءُ جَهْلٌ.
\par 23 جَمِيعُ هَذِهِ الشُّرُورِ تَخْرُجُ مِنَ الدَّاخِلِ وَتُنَجِّسُ الإِنْسَانَ».
\par 24 ثُمَّ قَامَ مِنْ هُنَاكَ وَمَضَى إِلَى تُخُومِ صُورَ وَصَيْدَاءَ وَدَخَلَ بَيْتاً وَهُوَ يُرِيدُ أَنْ لاَ يَعْلَمَ أَحَدٌ فَلَمْ يَقْدِرْ أَنْ يَخْتَفِيَ
\par 25 لأَنَّ امْرَأَةً كَانَ بِابْنَتِهَا رُوحٌ نَجِسٌ سَمِعَتْ بِهِ فَأَتَتْ وَخَرَّتْ عِنْدَ قَدَمَيْهِ.
\par 26 وَكَانَتْ الْمَرْأَةُ أُمَمِيَّةً وَفِي جِنْسِهَا فِينِيقِيَّةً سُورِيَّةً - فَسَأَلَتْهُ أَنْ يُخْرِجَ الشَّيْطَانَ مِنِ ابْنَتِهَا.
\par 27 وَأَمَّا يَسُوعُ فَقَالَ لَهَا: «دَعِي الْبَنِينَ أَوَّلاً يَشْبَعُونَ لأَنَّهُ لَيْسَ حَسَناً أَنْ يُؤْخَذَ خُبْزُ الْبَنِينَ وَيُطْرَحَ لِلْكِلاَبِ».
\par 28 فَأَجَابَتْ: «نَعَمْ يَا سَيِّدُ! وَالْكِلاَبُ أَيْضاً تَحْتَ الْمَائِدَةِ تَأْكُلُ مِنْ فُتَاتِ الْبَنِينَ».
\par 29 فَقَالَ لَهَا: «لأَجْلِ هَذِهِ الْكَلِمَةِ اذْهَبِي. قَدْ خَرَجَ الشَّيْطَانُ مِنِ ابْنَتِكِ».
\par 30 فَذَهَبَتْ إِلَى بَيْتِهَا وَوَجَدَتِ الشَّيْطَانَ قَدْ خَرَجَ وَالاِبْنَةَ مَطْرُوحَةً عَلَى الْفِرَاشِ.
\par 31 ثُمَّ خَرَجَ أَيْضاً مِنْ تُخُومِ صُورَ وَصَيْدَاءَ وَجَاءَ إِلَى بَحْرِ الْجَلِيلِ فِي وَسْطِ حُدُودِ الْمُدُنِ الْعَشْرِ.
\par 32 وَجَاءُوا إِلَيْهِ بِأَصَمَّ أَعْقَدَ وَطَلَبُوا إِلَيْهِ أَنْ يَضَعَ يَدَهُ عَلَيْهِ.
\par 33 فَأَخَذَهُ مِنْ بَيْنِ الْجَمْعِ عَلَى نَاحِيَةٍ وَوَضَعَ أَصَابِعَهُ فِي أُذُنَيْهِ وَتَفَلَ وَلَمَسَ لِسَانَهُ
\par 34 وَرَفَعَ نَظَرَهُ نَحْوَ السَّمَاءِ وَأَنَّ وَقَالَ لَهُ: «إِفَّثَا». أَيِ انْفَتِحْ.
\par 35 وَلِلْوَقْتِ انْفَتَحَتْ أُذْنَاهُ وَانْحَلَّ رِبَاطُ لِسَانِهِ وَتَكَلَّمَ مُسْتَقِيماً.
\par 36 فَأَوْصَاهُمْ أَنْ لاَ يَقُولُوا لأَحَدٍ. وَلَكِنْ عَلَى قَدْرِ مَا أَوْصَاهُمْ كَانُوا يُنَادُونَ أَكْثَرَ كَثِيراً.
\par 37 وَبُهِتُوا إِلَى الْغَايَةِ قَائِلِينَ: «إِنَّهُ عَمِلَ كُلَّ شَيْءٍ حَسَناً! جَعَلَ الصُّمَّ يَسْمَعُونَ وَالْخُرْسَ يَتَكَلَّمُونَ!».

\chapter{8}

\par 1 فِي تِلْكَ الأَيَّامِ إِذْ كَانَ الْجَمْعُ كَثِيراً جِدّاً وَلَمْ يَكُنْ لَهُمْ مَا يَأْكُلُونَ دَعَا يَسُوعُ تَلاَمِيذَهُ وَقَالَ لَهُمْ:
\par 2 «إِنِّي أُشْفِقُ عَلَى الْجَمْعِ لأَنَّ الآنَ لَهُمْ ثَلاَثَةَ أَيَّامٍ يَمْكُثُونَ مَعِي وَلَيْسَ لَهُمْ مَا يَأْكُلُونَ.
\par 3 وَإِنْ صَرَفْتُهُمْ إِلَى بُيُوتِهِمْ صَائِمِينَ يُخَوِّرُونَ فِي الطَّرِيقِ لأَنَّ قَوْماً مِنْهُمْ جَاءُوا مِنْ بَعِيدٍ».
\par 4 فَأَجَابَهُ تَلاَمِيذُهُ: «مِنْ أَيْنَ يَسْتَطِيعُ أَحَدٌ أَنْ يُشْبِعَ هَؤُلاَءِ خُبْزاً هُنَا فِي الْبَرِّيَّةِ؟»
\par 5 فَسَأَلَهُمْ: «كَمْ عِنْدَكُمْ مِنَ الْخُبْزِ؟» فَقَالُوا: «سَبْعَةٌ».
\par 6 فَأَمَرَ الْجَمْعَ أَنْ يَتَّكِئُوا عَلَى الأَرْضِ وَأَخَذَ السَّبْعَ خُبْزَاتٍ وَشَكَرَ وَكَسَرَ وَأَعْطَى تَلاَمِيذَهُ لِيُقَدِّمُوا فَقَدَّمُوا إِلَى الْجَمْعِ.
\par 7 وَكَانَ مَعَهُمْ قَلِيلٌ مِنْ صِغَارِ السَّمَكِ فَبَارَكَ وَقَالَ أَنْ يُقَدِّمُوا هَذِهِ أَيْضاً.
\par 8 فَأَكَلُوا وَشَبِعُوا ثُمَّ رَفَعُوا فَضَلاَتِ الْكِسَرِ: سَبْعَةَ سِلاَلٍ.
\par 9 وَكَانَ الآكِلُونَ نَحْوَ أَرْبَعَةِ آلاَفٍ. ثُمَّ صَرَفَهُمْ.
\par 10 وَلِلْوَقْتِ دَخَلَ السَّفِينَةَ مَعَ تَلاَمِيذِهِ وَجَاءَ إِلَى نَوَاحِي دَلْمَانُوثَةَ.
\par 11 فَخَرَجَ الْفَرِّيسِيُّونَ وَابْتَدَأُوا يُحَاوِرُونَهُ طَالِبِينَ مِنْهُ آيَةً مِنَ السَّمَاءِ لِكَيْ يُجَرِّبُوهُ.
\par 12 فَتَنَهَّدَ بِرُوحِهِ وَقَالَ: «لِمَاذَا يَطْلُبُ هَذَا الْجِيلُ آيَةً؟ اَلْحَقَّ أَقُولُ لَكُمْ: لَنْ يُعْطَى هَذَا الْجِيلُ آيَةً!»
\par 13 ثُمَّ تَرَكَهُمْ وَدَخَلَ أَيْضاً السَّفِينَةَ وَمَضَى إِلَى الْعَبْرِ.
\par 14 وَنَسُوا أَنْ يَأْخُذُوا خُبْزاً وَلَمْ يَكُنْ مَعَهُمْ فِي السَّفِينَةِ إلاَّ رَغِيفٌ وَاحِدٌ.
\par 15 وَأَوْصَاهُمْ قَائِلاً: «انْظُرُوا وَتَحَرَّزُوا مِنْ خَمِيرِ الْفَرِّيسِيِّينَ وَخَمِيرِ هِيرُودُسَ.
\par 16 فَفَكَّرُوا قَائِلِينَ بَعْضُهُمْ لِبَعْضٍ: «لَيْسَ عِنْدَنَا خُبْزٌ».
\par 17 فَعَلِمَ يَسُوعُ وَقَالَ لَهُمْ: «لِمَاذَا تُفَكِّرُونَ أَنْ لَيْسَ عِنْدَكُمْ خُبْزٌ؟ أَلاَ تَشْعُرُونَ بَعْدُ وَلاَ تَفْهَمُونَ؟ أَحَتَّى الآنَ قُلُوبُكُمْ غَلِيظَةٌ؟
\par 18 أَلَكُمْ أَعْيُنٌ وَلاَ تُبْصِرُونَ وَلَكُمْ آذَانٌ وَلاَ تَسْمَعُونَ وَلاَ تَذْكُرُونَ؟
\par 19 حِينَ كَسَّرْتُ الأَرْغِفَةَ الْخَمْسَةَ لِلْخَمْسَةِ الآلاَفِ كَمْ قُفَّةً مَمْلُوَّةً كِسَراً رَفَعْتُمْ؟» قَالُوا لَهُ: «اثْنَتَيْ عَشْرَةَ».
\par 20 «وَحِينَ السَّبْعَةِ لِلأَرْبَعَةِ الآلاَفِ كَمْ سَلَّ كِسَرٍ مَمْلُوّاً رَفَعْتُمْ؟» قَالُوا: «سَبْعَةً».
\par 21 فَقَالَ لَهُمْ: «كَيْفَ لاَ تَفْهَمُونَ؟»
\par 22 وَجَاءَ إِلَى بَيْتِ صَيْدَا فَقَدَّمُوا إِلَيْهِ أَعْمَى وَطَلَبُوا إِلَيْهِ أَنْ يَلْمِسَهُ
\par 23 فَأَخَذَ بِيَدِ الأَعْمَى وَأَخْرَجَهُ إِلَى خَارِجِ الْقَرْيَةِ وَتَفَلَ فِي عَيْنَيْهِ وَوَضَعَ يَدَيْهِ عَلَيْهِ وَسَأَلَهُ هَلْ أَبْصَرَ شَيْئاً؟
\par 24 فَتَطَلَّعَ وَقَالَ: «أُبْصِرُ النَّاسَ كَأَشْجَارٍ يَمْشُونَ».
\par 25 ثُمَّ وَضَعَ يَدَيْهِ أَيْضاً عَلَى عَيْنَيْهِ وَجَعَلَهُ يَتَطَلَّعُ. فَعَادَ صَحِيحاً وَأَبْصَرَ كُلَّ إِنْسَانٍ جَلِيّاً.
\par 26 فَأَرْسَلَهُ إِلَى بَيْتِهِ قَائِلاً: «لاَ تَدْخُلِ الْقَرْيَةَ وَلاَ تَقُلْ لأَحَدٍ فِي الْقَرْيَةِ».
\par 27 ثُمَّ خَرَجَ يَسُوعُ وَتَلاَمِيذُهُ إِلَى قُرَى قَيْصَرِيَّةِ فِيلُبُّسَ. وَفِي الطَّرِيقِ سَأَلَ تَلاَمِيذَهُ: «مَنْ يَقُولُ النَّاسُ إِنِّي أَنَا؟»
\par 28 فَأَجَابُوا: «يُوحَنَّا الْمَعْمَدَانُ وَآخَرُونَ إِيلِيَّا وَآخَرُونَ وَاحِدٌ مِنَ الأَنْبِيَاءِ».
\par 29 فَقَالَ لَهُمْ: «وَأَنْتُمْ مَنْ تَقُولُونَ إِنِّي أَنَا؟» فَأَجَابَ بُطْرُسُ: «أَنْتَ الْمَسِيحُ!»
\par 30 فَانْتَهَرَهُمْ كَيْ لاَ يَقُولُوا لأَحَدٍ عَنْهُ.
\par 31 وَابْتَدَأَ يُعَلِّمُهُمْ أَنَّ ابْنَ الإِنْسَانِ يَنْبَغِي أَنْ يَتَأَلَّمَ كَثِيراً وَيُرْفَضَ مِنَ الشُّيُوخِ وَرُؤَسَاءِ الْكَهَنَةِ وَالْكَتَبَةِ وَيُقْتَلَ وَبَعْدَ ثَلاَثَةِ أَيَّامٍ يَقُومُ.
\par 32 وَقَالَ الْقَوْلَ عَلاَنِيَةً فَأَخَذَهُ بُطْرُسُ إِلَيْهِ وَابْتَدَأَ يَنْتَهِرُهُ.
\par 33 فَالْتَفَتَ وَأَبْصَرَ تَلاَمِيذَهُ فَانْتَهَرَ بُطْرُسَ قَائِلاً: «اذْهَبْ عَنِّي يَا شَيْطَانُ لأَنَّكَ لاَ تَهْتَمُّ بِمَا لِلَّهِ لَكِنْ بِمَا لِلنَّاسِ».
\par 34 وَدَعَا الْجَمْعَ مَعَ تَلاَمِيذِهِ وَقَالَ لَهُمْ: «مَنْ أَرَادَ أَنْ يَأْتِيَ وَرَائِي فَلْيُنْكِرْ نَفْسَهُ وَيَحْمِلْ صَلِيبَهُ وَيَتْبَعْنِي.
\par 35 فَإِنَّ مَنْ أَرَادَ أَنْ يُخَلِّصَ نَفْسَهُ يُهْلِكُهَا وَمَنْ يُهْلِكُ نَفْسَهُ مِنْ أَجْلِي وَمِنْ أَجْلِ الإِنْجِيلِ فَهُوَ يُخَلِّصُهَا.
\par 36 لأَنَّهُ مَاذَا يَنْتَفِعُ الإِنْسَانُ لَوْ رَبِحَ الْعَالَمَ كُلَّهُ وَخَسِرَ نَفْسَهُ؟
\par 37 أَوْ مَاذَا يُعْطِي الإِنْسَانُ فِدَاءً عَنْ نَفْسِهِ؟
\par 38 لأَنَّ مَنِ اسْتَحَى بِي وَبِكَلاَمِي فِي هَذَا الْجِيلِ الْفَاسِقِ الْخَاطِئِ فَإِنَّ ابْنَ الإِنْسَانِ يَسْتَحِي بِهِ مَتَى جَاءَ بِمَجْدِ أَبِيهِ مَعَ الْمَلاَئِكَةِ الْقِدِّيسِينَ».

\chapter{9}

\par 1 وَقَالَ لَهُمُ: «الْحَقَّ أَقُولُ لَكُمْ: إِنَّ مِنَ الْقِيَامِ هَهُنَا قَوْماً لاَ يَذُوقُونَ الْمَوْتَ حَتَّى يَرَوْا مَلَكُوتَ اللَّهِ قَدْ أَتَى بِقُوَّةٍ».
\par 2 وَبَعْدَ سِتَّةِ أَيَّامٍ أَخَذَ يَسُوعُ بُطْرُسَ وَيَعْقُوبَ وَيُوحَنَّا وَصَعِدَ بِهِمْ إِلَى جَبَلٍ عَالٍ مُنْفَرِدِينَ وَحْدَهُمْ. وَتَغَيَّرَتْ هَيْئَتُهُ قُدَّامَهُمْ
\par 3 وَصَارَتْ ثِيَابُهُ تَلْمَعُ بَيْضَاءَ جِدّاً كَالثَّلْجِ لاَ يَقْدِرُ قَصَّارٌ عَلَى الأَرْضِ أَنْ يُبَيِّضَ مِثْلَ ذَلِكَ.
\par 4 وَظَهَرَ لَهُمْ إِيلِيَّا مَعَ مُوسَى وَكَانَا يَتَكَلَّمَانِ مَعَ يَسُوعَ.
\par 5 فَجَعَلَ بُطْرُسُ يَقولُ لِيَسُوعَ: «يَا سَيِّدِي جَيِّدٌ أَنْ نَكُونَ هَهُنَا. فَلْنَصْنَعْ ثَلاَثَ مَظَالَّ لَكَ وَاحِدَةً وَلِمُوسَى وَاحِدَةً وَلِإِيلِيَّا وَاحِدَةً».
\par 6 لأَنَّهُ لَمْ يَكُنْ يَعْلَمُ مَا يَتَكَلَّمُ بِهِ إِذْ كَانُوا مُرْتَعِبِينَ.
\par 7 وَكَانَتْ سَحَابَةٌ تُظَلِّلُهُمْ. فَجَاءَ صَوْتٌ مِنَ السَّحَابَةِ قَائِلاً: «هَذَا هُوَ ابْنِي الْحَبِيبُ. لَهُ اسْمَعُوا».
\par 8 فَنَظَرُوا حَوْلَهُمْ بَغْتَةً وَلَمْ يَرَوْا أَحَداً غَيْرَ يَسُوعَ وَحْدَهُ مَعَهُمْ.
\par 9 وَفِيمَا هُمْ نَازِلُونَ مِنَ الْجَبَلِ أَوْصَاهُمْ أَنْ لاَ يُحَدِّثُوا أَحَداً بِمَا أَبْصَرُوا إلاَّ مَتَى قَامَ ابْنُ الإِنْسَانِ مِنَ الأَمْوَاتِ.
\par 10 فَحَفِظُوا الْكَلِمَةَ لأَنْفُسِهِمْ يَتَسَاءَلُونَ: «مَا هُوَ الْقِيَامُ مِنَ الأَمْوَاتِ؟»
\par 11 فَسَأَلُوهُ: «لِمَاذَا يَقُولُ الْكَتَبَةُ إِنَّ إِيلِيَّا يَنْبَغِي أَنْ يَأْتِيَ أَوَّلاً؟»
\par 12 فَأَجَابَ: «إِنَّ إِيلِيَّا يَأْتِي أَوَّلاً وَيَرُدُّ كُلَّ شَيْءٍ. وَكَيْفَ هُوَ مَكْتُوبٌ عَنِ ابْنِ الإِنْسَانِ أَنْ يَتَأَلَّمَ كَثِيراً وَيُرْذَلَ.
\par 13 لَكِنْ أَقُولُ لَكُمْ: إِنَّ إِيلِيَّا أَيْضاً قَدْ أَتَى وَعَمِلُوا بِهِ كُلَّ مَا أَرَادُوا كَمَا هُوَ مَكْتُوبٌ عَنْهُ».
\par 14 وَلَمَّا جَاءَ إِلَى التَّلاَمِيذِ رَأَى جَمْعاً كَثِيراً حَوْلَهُمْ وَكَتَبَةً يُحَاوِرُونَهُمْ.
\par 15 وَلِلْوَقْتِ كُلُّ الْجَمْعِ لَمَّا رَأَوْهُ تَحَيَّرُوا وَرَكَضُوا وَسَلَّمُوا عَلَيْهِ.
\par 16 فَسَأَلَ الْكَتَبَةَ: «بِمَاذَا تُحَاوِرُونَهُمْ؟»
\par 17 فَأَجَابَ وَاحِدٌ مِنَ الْجَمْعِ: «يَا مُعَلِّمُ قَدْ قَدَّمْتُ إِلَيْكَ ابْنِي بِهِ رُوحٌ أَخْرَسُ
\par 18 وَحَيْثُمَا أَدْرَكَهُ يُمَزِّقْهُ فَيُزْبِدُ وَيَصِرُّ بِأَسْنَانِهِ وَيَيْبَسُ. فَقُلْتُ لِتَلاَمِيذِكَ أَنْ يُخْرِجُوهُ فَلَمْ يَقْدِرُوا».
\par 19 فَقَالَ لَهُمْ: «أَيُّهَا الْجِيلُ غَيْرُ الْمُؤْمِنِ إِلَى مَتَى أَكُونُ مَعَكُمْ؟ إِلَى مَتَى أَحْتَمِلُكُمْ؟ قَدِّمُوهُ إِلَيَّ!».
\par 20 فَقَدَّمُوهُ إِلَيْهِ. فَلَمَّا رَآهُ لِلْوَقْتِ صَرَعَهُ الرُّوحُ فَوَقَعَ عَلَى الأَرْضِ يَتَمَرَّغُ وَيُزْبِدُ.
\par 21 فَسَأَلَ أَبَاهُ: «كَمْ مِنَ الزَّمَانِ مُنْذُ أَصَابَهُ هَذَا؟» فَقَالَ: «مُنْذُ صِبَاهُ.
\par 22 وَكَثِيراً مَا أَلْقَاهُ فِي النَّارِ وَفِي الْمَاءِ لِيُهْلِكَهُ. لَكِنْ إِنْ كُنْتَ تَسْتَطِيعُ شَيْئاً فَتَحَنَّنْ عَلَيْنَا وَأَعِنَّا».
\par 23 فَقَالَ لَهُ يَسُوعُ: «إِنْ كُنْتَ تَسْتَطِيعُ أَنْ تُؤْمِنَ فَكُلُّ شَيْءٍ مُسْتَطَاعٌ لِلْمُؤْمِنِ».
\par 24 فَلِلْوَقْتِ صَرَخَ أَبُو الْوَلَدِ بِدُمُوعٍ وَقَالَ: «أُومِنُ يَا سَيِّدُ فَأَعِنْ عَدَمَ إِيمَانِي».
\par 25 فَلَمَّا رَأَى يَسُوعُ أَنَّ الْجَمْعَ يَتَرَاكَضُونَ انْتَهَرَ الرُّوحَ النَّجِسَ قَائِلاً لَهُ: «أَيُّهَا الرُّوحُ الأَخْرَسُ الأَصَمُّ أَنَا آمُرُكَ: اخْرُجْ مِنْهُ وَلاَ تَدْخُلْهُ أَيْضاً!»
\par 26 فَصَرَخَ وَصَرَعَهُ شَدِيداً وَخَرَجَ فَصَارَ كَمَيْتٍ حَتَّى قَالَ كَثِيرُونَ: إِنَّهُ مَاتَ.
\par 27 فَأَمْسَكَهُ يَسُوعُ بِيَدِهِ وَأَقَامَهُ فَقَامَ.
\par 28 وَلَمَّا دَخَلَ بَيْتاً سَأَلَهُ تَلاَمِيذُهُ عَلَى انْفِرَادٍ: «لِمَاذَا لَمْ نَقْدِرْ نَحْنُ أَنْ نُخْرِجَهُ؟»
\par 29 فَقَالَ لَهُمْ: «هَذَا الْجِنْسُ لاَ يُمْكِنُ أَنْ يَخْرُجَ بِشَيْءٍ إلاَّ بِالصَّلاَةِ وَالصَّوْمِ».
\par 30 وَخَرَجُوا مِنْ هُنَاكَ وَاجْتَازُوا الْجَلِيلَ وَلَمْ يُرِدْ أَنْ يَعْلَمَ أَحَدٌ
\par 31 لأَنَّهُ كَانَ يُعَلِّمُ تَلاَمِيذَهُ وَيَقُولُ لَهُمْ إِنَّ ابْنَ الإِنْسَانِ يُسَلَّمُ إِلَى أَيْدِي النَّاسِ فَيَقْتُلُونَهُ وَبَعْدَ أَنْ يُقْتَلَ يَقُومُ فِي الْيَوْمِ الثَّالِثِ.
\par 32 وَأَمَّا هُمْ فَلَمْ يَفْهَمُوا الْقَوْلَ وَخَافُوا أَنْ يَسْأَلُوهُ.
\par 33 وَجَاءَ إِلَى كَفْرِنَاحُومَ. وَإِذْ كَانَ فِي الْبَيْتِ سَأَلَهُمْ: «بِمَاذَا كُنْتُمْ تَتَكَالَمُونَ فِي مَا بَيْنَكُمْ فِي الطَّرِيقِ؟»
\par 34 فَسَكَتُوا لأَنَّهُمْ تَحَاجُّوا فِي الطَّرِيقِ بَعْضُهُمْ مَعَ بَعْضٍ فِي مَنْ هُوَ أَعْظَمُ.
\par 35 فَجَلَسَ وَنَادَى الاِثْنَيْ عَشَرَ وَقَالَ لَهُمْ: «إِذَا أَرَادَ أَحَدٌ أَنْ يَكُونَ أَوَّلاً فَيَكُونُ آخِرَ الْكُلِّ وَخَادِماً لِلْكُلِّ».
\par 36 فَأَخَذَ وَلَداً وَأَقَامَهُ فِي وَسَطِهِمْ ثُمَّ احْتَضَنَهُ وَقَالَ لَهُمْ:
\par 37 «مَنْ قَبِلَ وَاحِداً مِنْ أَوْلاَدٍ مِثْلَ هَذَا بِاسْمِي يَقْبَلُنِي وَمَنْ قَبِلَنِي فَلَيْسَ يَقْبَلُنِي أَنَا بَلِ الَّذِي أَرْسَلَنِي».
\par 38 وَقَالَ يُوحَنَّا: «يَا مُعَلِّمُ رَأَيْنَا وَاحِداً يُخْرِجُ شَيَاطِينَ بِاسْمِكَ وَهُوَ لَيْسَ يَتْبَعُنَا فَمَنَعْنَاهُ لأَنَّهُ لَيْسَ يَتْبَعُنَا».
\par 39 فَقَالَ يَسُوعُ: «لاَ تَمْنَعُوهُ لأَنَّهُ لَيْسَ أَحَدٌ يَصْنَعُ قُوَّةً بِاسْمِي وَيَسْتَطِيعُ سَرِيعاً أَنْ يَقُولَ عَلَيَّ شَرّاً.
\par 40 لأَنَّ مَنْ لَيْسَ عَلَيْنَا فَهُوَ مَعَنَا.
\par 41 لأَنَّ مَنْ سَقَاكُمْ كَأْسَ مَاءٍ بِاسْمِي لأَنَّكُمْ لِلْمَسِيحِ فَالْحَقَّ أَقُولُ لَكُمْ إِنَّهُ لاَ يُضِيعُ أَجْرَهُ.
\par 42 وَمَنْ أَعْثَرَ أَحَدَ الصِّغَارِ الْمُؤْمِنِينَ بِي فَخَيْرٌ لَهُ لَوْ طُوِّقَ عُنُقُهُ بِحَجَرِ رَحًى وَطُرِحَ فِي الْبَحْرِ.
\par 43 وَإِنْ أَعْثَرَتْكَ يَدُكَ فَاقْطَعْهَا. خَيْرٌ لَكَ أَنْ تَدْخُلَ الْحَيَاةَ أَقْطَعَ مِنْ أَنْ تَكُونَ لَكَ يَدَانِ وَتَمْضِيَ إِلَى جَهَنَّمَ إِلَى النَّارِ الَّتِي لاَ تُطْفَأُ
\par 44 حَيْثُ دُودُهُمْ لاَ يَمُوتُ وَالنَّارُ لاَ تُطْفَأُ.
\par 45 وَإِنْ أَعْثَرَتْكَ رِجْلُكَ فَاقْطَعْهَا. خَيْرٌ لَكَ أَنْ تَدْخُلَ الْحَيَاةَ أَعْرَجَ مِنْ أَنْ تَكُونَ لَكَ رِجْلاَنِ وَتُطْرَحَ فِي جَهَنَّمَ فِي النَّارِ الَّتِي لاَ تُطْفَأُ
\par 46 حَيْثُ دُودُهُمْ لاَ يَمُوتُ وَالنَّارُ لاَ تُطْفَأُ.
\par 47 وَإِنْ أَعْثَرَتْكَ عَيْنُكَ فَاقْلَعْهَا. خَيْرٌ لَكَ أَنْ تَدْخُلَ مَلَكُوتَ اللَّهِ أَعْوَرَ مِنْ أَنْ تَكُونَ لَكَ عَيْنَانِ وَتُطْرَحَ فِي جَهَنَّمَ النَّارِ
\par 48 حَيْثُ دُودُهُمْ لاَ يَمُوتُ وَالنَّارُ لاَ تُطْفَأُ.
\par 49 لأَنَّ كُلَّ وَاحِدٍ يُمَلَّحُ بِنَارٍ وَكُلَّ ذَبِيحَةٍ تُمَلَّحُ بِمِلْحٍ.
\par 50 اَلْمِلْحُ جَيِّدٌ. وَلَكِنْ إِذَا صَارَ الْمِلْحُ بِلاَ مُلُوحَةٍ فَبِمَاذَا تُصْلِحُونَهُ؟ لِيَكُنْ لَكُمْ فِي أَنْفُسِكُمْ مِلْحٌ وَسَالِمُوا بَعْضُكُمْ بَعْضاً».

\chapter{10}

\par 1 وَقَامَ مِنْ هُنَاكَ وَجَاءَ إِلَى تُخُومِ الْيَهُودِيَّةِ مِنْ عَبْرِ الأُرْدُنِّ فَاجْتَمَعَ إِلَيْهِ جُمُوعٌ أَيْضاً وَكَعَادَتِهِ كَانَ أَيْضاً يُعَلِّمُهُمْ.
\par 2 فَتَقَدَّمَ الْفَرِّيسِيُّونَ وَسَأَلُوهُ: «هَلْ يَحِلُّ لِلرَّجُلِ أَنْ يُطَلِّقَ امْرَأَتَهُ؟» لِيُجَرِّبُوهُ.
\par 3 فَأَجَابَ: «بِمَاذَا أَوْصَاكُمْ مُوسَى؟»
\par 4 فَقَالُوا: «مُوسَى أَذِنَ أَنْ يُكْتَبَ كِتَابُ طَلاَقٍ فَتُطَلَّقُ».
\par 5 فَأَجَابَ يَسُوعُ: «مِنْ أَجْلِ قَسَاوَةِ قُلُوبِكُمْ كَتَبَ لَكُمْ هَذِهِ الْوَصِيَّةَ
\par 6 وَلَكِنْ مِنْ بَدْءِ الْخَلِيقَةِ ذَكَراً وَأُنْثَى خَلَقَهُمَا اللَّهُ.
\par 7 مِنْ أَجْلِ هَذَا يَتْرُكُ الرَّجُلُ أَبَاهُ وَأُمَّهُ وَيَلْتَصِقُ بِامْرَأَتِهِ
\par 8 وَيَكُونُ الاِثْنَانِ جَسَداً وَاحِداً. إِذاً لَيْسَا بَعْدُ اثْنَيْنِ بَلْ جَسَدٌ وَاحِدٌ.
\par 9 فَالَّذِي جَمَعَهُ اللَّهُ لاَ يُفَرِّقْهُ إِنْسَانٌ».
\par 10 ثُمَّ فِي الْبَيْتِ سَأَلَهُ تَلاَمِيذُهُ أَيْضاً عَنْ ذَلِكَ
\par 11 فَقَالَ لَهُمْ: «مَنْ طَلَّقَ امْرَأَتَهُ وَتَزَوَّجَ بِأُخْرَى يَزْنِي عَلَيْهَا.
\par 12 وَإِنْ طَلَّقَتِ امْرَأَةٌ زَوْجَهَا وَتَزَوَّجَتْ بِآخَرَ تَزْنِي».
\par 13 وَقَدَّمُوا إِلَيْهِ أَوْلاَداً لِكَيْ يَلْمِسَهُمْ. وَأَمَّا التَّلاَمِيذُ فَانْتَهَرُوا الَّذِينَ قَدَّمُوهُمْ.
\par 14 فَلَمَّا رَأَى يَسُوعُ ذَلِكَ اغْتَاظَ وَقَالَ لَهُمْ: «دَعُوا الأَوْلاَدَ يَأْتُونَ إِلَيَّ وَلاَ تَمْنَعُوهُمْ لأَنَّ لِمِثْلِ هَؤُلاَءِ مَلَكُوتَ اللَّهِ.
\par 15 اَلْحَقَّ أَقُولُ لَكُمْ: مَنْ لاَ يَقْبَلُ مَلَكُوتَ اللَّهِ مِثْلَ وَلَدٍ فَلَنْ يَدْخُلَهُ».
\par 16 فَاحْتَضَنَهُمْ وَوَضَعَ يَدَيْهِ عَلَيْهِمْ وَبَارَكَهُمْ.
\par 17 وَفِيمَا هُوَ خَارِجٌ إِلَى الطَّرِيقِ رَكَضَ وَاحِدٌ وَجَثَا لَهُ وَسَأَلَهُ: «أَيُّهَا الْمُعَلِّمُ الصَّالِحُ مَاذَا أَعْمَلُ لأَرِثَ الْحَيَاةَ الأَبَدِيَّةَ؟»
\par 18 فَقَالَ لَهُ يَسُوعُ: «لِمَاذَا تَدْعُونِي صَالِحاً؟ لَيْسَ أَحَدٌ صَالِحاً إلاَّ وَاحِدٌ وَهُوَ اللَّهُ.
\par 19 أَنْتَ تَعْرِفُ الْوَصَايَا: لاَ تَزْنِ. لاَ تَقْتُلْ. لاَ تَسْرِقْ. لاَ تَشْهَدْ بِالزُّورِ. لاَ تَسْلِبْ. أَكْرِمْ أَبَاكَ وَأُمَّكَ».
\par 20 فَأَجَابَ: «يَا مُعَلِّمُ هَذِهِ كُلُّهَا حَفِظْتُهَا مُنْذُ حَدَاثَتِي».
\par 21 فَنَظَرَ إِلَيْهِ يَسُوعُ وَأَحَبَّهُ وَقَالَ لَهُ: «يُعْوِزُكَ شَيْءٌ وَاحِدٌ. اذْهَبْ بِعْ كُلَّ مَا لَكَ وَأَعْطِ الْفُقَرَاءَ فَيَكُونَ لَكَ كَنْزٌ فِي السَّمَاءِ وَتَعَالَ اتْبَعْنِي حَامِلاً الصَّلِيبَ».
\par 22 فَاغْتَمَّ عَلَى الْقَوْلِ وَمَضَى حَزِيناً لأَنَّهُ كَانَ ذَا أَمْوَالٍ كَثِيرَةٍ.
\par 23 فَنَظَرَ يَسُوعُ حَوْلَهُ وَقَالَ لِتَلاَمِيذِهِ: «مَا أَعْسَرَ دُخُولَ ذَوِي الأَمْوَالِ إِلَى مَلَكُوتِ اللَّهِ!»
\par 24 فَتَحَيَّرَ التَّلاَمِيذُ مِنْ كَلاَمِهِ. فَقَالَ يَسُوعُ أَيْضاً: «يَا بَنِيَّ مَا أَعْسَرَ دُخُولَ الْمُتَّكِلِينَ عَلَى الأَمْوَالِ إِلَى مَلَكُوتِ اللَّهِ!
\par 25 مُرُورُ جَمَلٍ مِنْ ثَقْبِ إِبْرَةٍ أَيْسَرُ مِنْ أَنْ يَدْخُلَ غَنِيٌّ إِلَى مَلَكُوتِ اللَّهِ!»
\par 26 فَبُهِتُوا إِلَى الْغَايَةِ قَائِلِينَ بَعْضُهُمْ لِبَعْضٍ: «فَمَنْ يَسْتَطِيعُ أَنْ يَخْلُصَ؟»
\par 27 فَنَظَرَ إِلَيْهِمْ يَسُوعُ وَقَالَ: «عِنْدَ النَّاسِ غَيْرُ مُسْتَطَاعٍ وَلَكِنْ لَيْسَ عِنْدَ اللَّهِ لأَنَّ كُلَّ شَيْءٍ مُسْتَطَاعٌ عِنْدَ اللَّهِ».
\par 28 وَابْتَدَأَ بُطْرُسُ يَقُولُ لَهُ: «هَا نَحْنُ قَدْ تَرَكْنَا كُلَّ شَيْءٍ وَتَبِعْنَاكَ».
\par 29 فَأَجَابَ يَسُوعُ: «الْحَقَّ أَقُولُ لَكُمْ لَيْسَ أَحَدٌ تَرَكَ بَيْتاً أَوْ إِخْوَةً أَوْ أَخَوَاتٍ أَوْ أَباً أَوْ أُمّاً أَوِ امْرَأَةً أَوْ أَوْلاَداً أَوْ حُقُولاً لأَجْلِي وَلأَجْلِ الإِنْجِيلِ
\par 30 إِلاَّ وَيَأْخُذُ مِئَةَ ضِعْفٍ الآنَ فِي هَذَا الزَّمَانِ بُيُوتاً وَإِخْوَةً وَأَخَوَاتٍ وَأُمَّهَاتٍ وَأَوْلاَداً وَحُقُولاً مَعَ اضْطِهَادَاتٍ وَفِي الدَّهْرِ الآتِي الْحَيَاةَ الأَبَدِيَّةَ.
\par 31 وَلَكِنْ كَثِيرُونَ أَوَّلُونَ يَكُونُونَ آخِرِينَ وَالآخِرُونَ أَوَّلِينَ».
\par 32 وَكَانُوا فِي الطَّرِيقِ صَاعِدِينَ إِلَى أُورُشَلِيمَ وَيَتَقَدَّمُهُمْ يَسُوعُ وَكَانُوا يَتَحَيَّرُونَ. وَفِيمَا هُمْ يَتْبَعُونَ كَانُوا يَخَافُونَ. فَأَخَذَ الاِثْنَيْ عَشَرَ أَيْضاً وَابْتَدَأَ يَقُولُ لَهُمْ عَمَّا سَيَحْدُثُ لَهُ:
\par 33 «هَا نَحْنُ صَاعِدُونَ إِلَى أُورُشَلِيمَ وَابْنُ الإِنْسَانِ يُسَلَّمُ إِلَى رُؤَسَاءِ الْكَهَنَةِ وَالْكَتَبَةِ فَيَحْكُمُونَ عَلَيْهِ بِالْمَوْتِ وَيُسَلِّمُونَهُ إِلَى الأُمَمِ
\par 34 فَيَهْزَأُونَ بِهِ وَيَجْلِدُونَهُ وَيَتْفُلُونَ عَلَيْهِ وَيَقْتُلُونَهُ وَفِي الْيَوْمِ الثَّالِثِ يَقُومُ».
\par 35 وَتَقَدَّمَ إِلَيْهِ يَعْقُوبُ وَيُوحَنَّا ابْنَا زَبْدِي قَائِلَيْنِ: «يَا مُعَلِّمُ نُرِيدُ أَنْ تَفْعَلَ لَنَا كُلَّ مَا طَلَبْنَا».
\par 36 فَسَأَلَهُمَا: «مَاذَا تُرِيدَانِ أَنْ أَفْعَلَ لَكُمَا؟»
\par 37 فَقَالاَ لَهُ: «أَعْطِنَا أَنْ نَجْلِسَ وَاحِدٌ عَنْ يَمِينِكَ وَالآخَرُ عَنْ يَسَارِكَ فِي مَجْدِكَ».
\par 38 فَقَالَ لَهُمَا يَسُوعُ: «لَسْتُمَا تَعْلَمَانِ مَا تَطْلُبَانِ. أَتَسْتَطِيعَانِ أَنْ تَشْرَبَا الْكَأْسَ الَّتِي أَشْرَبُهَا أَنَا وَأَنْ تَصْطَبِغَا بِالصِّبْغَةِ الَّتِي أَصْطَبِغُ بِهَا أَنَا؟»
\par 39 فَقَالاَ لَهُ: «نَسْتَطِيعُ». فَقَالَ لَهُمَا يَسُوعُ: «أَمَّا الْكَأْسُ الَّتِي أَشْرَبُهَا أَنَا فَتَشْرَبَانِهَا وَبَالصِّبْغَةِ الَّتِي أَصْطَبِغُ بِهَا أَنَا تَصْطَبِغَانِ.
\par 40 وَأَمَّا الْجُلُوسُ عَنْ يَمِينِي وَعَنْ يَسَارِي فَلَيْسَ لِي أَنْ أُعْطِيَهُ إلاَّ لِلَّذِينَ أُعِدَّ لَهُمْ».
\par 41 وَلَمَّا سَمِعَ الْعَشَرَةُ ابْتَدَأُوا يَغْتَاظُونَ مِنْ أَجْلِ يَعْقُوبَ وَيُوحَنَّا.
\par 42 فَدَعَاهُمْ يَسُوعُ وَقَالَ لَهُمْ: «أَنْتُمْ تَعْلَمُونَ أَنَّ الَّذِينَ يُحْسَبُونَ رُؤَسَاءَ الأُمَمِ يَسُودُونَهُمْ وَأَنَّ عُظَمَاءَهُمْ يَتَسَلَّطُونَ عَلَيْهِمْ.
\par 43 فَلاَ يَكُونُ هَكَذَا فِيكُمْ. بَلْ مَنْ أَرَادَ أَنْ يَصِيرَ فِيكُمْ عَظِيماً يَكُونُ لَكُمْ خَادِماً
\par 44 وَمَنْ أَرَادَ أَنْ يَصِيرَ فِيكُمْ أَوَّلاً يَكُونُ لِلْجَمِيعِ عَبْداً.
\par 45 لأَنَّ ابْنَ الإِنْسَانِ أَيْضاً لَمْ يَأْتِ لِيُخْدَمَ بَلْ لِيَخْدِمَ وَلِيَبْذِلَ نَفْسَهُ فِدْيَةً عَنْ كَثِيرِينَ».
\par 46 وَجَاءُوا إِلَى أَرِيحَا. وَفِيمَا هُوَ خَارِجٌ مِنْ أَرِيحَا مَعَ تَلاَمِيذِهِ وَجَمْعٍ غَفِيرٍ كَانَ بَارْتِيمَاوُسُ الأَعْمَى ابْنُ تِيمَاوُسَ جَالِساً عَلَى الطَّرِيقِ يَسْتَعْطِي.
\par 47 فَلَمَّا سَمِعَ أَنَّهُ يَسُوعُ النَّاصِرِيُّ ابْتَدَأَ يَصْرُخُ وَيَقُولُ: «يَا يَسُوعُ ابْنَ دَاوُدَ ارْحَمْنِي!»
\par 48 فَانْتَهَرَهُ كَثِيرُونَ لِيَسْكُتَ فَصَرَخَ أَكْثَرَ كَثِيراً: «يَا ابْنَ دَاوُدَ ارْحَمْنِي».
\par 49 فَوَقَفَ يَسُوعُ وَأَمَرَ أَنْ يُنَادَى. فَنَادَوُا الأَعْمَى قَائِلِينَ لَهُ: «ثِقْ. قُمْ. هُوَذَا يُنَادِيكَ».
\par 50 فَطَرَحَ رِدَاءَهُ وَقَامَ وَجَاءَ إِلَى يَسُوعَ.
\par 51 فَسَأَلَهُ يَسُوعُ: «مَاذَا تُرِيدُ أَنْ أَفْعَلَ بِكَ؟» فَقَالَ لَهُ الأَعْمَى: «يَا سَيِّدِي أَنْ أُبْصِرَ».
\par 52 فَقَالَ لَهُ يَسُوعُ: «اذْهَبْ. إِيمَانُكَ قَدْ شَفَاكَ». فَلِلْوَقْتِ أَبْصَرَ وَتَبِعَ يَسُوعَ فِي الطَّرِيقِ.

\chapter{11}

\par 1 وَلَمَّا قَرُبُوا مِنْ أُورُشَلِيمَ إِلَى بَيْتِ فَاجِي وَبَيْتِ عَنْيَا عِنْدَ جَبَلِ الزَّيْتُونِ أَرْسَلَ اثْنَيْنِ مِنْ تَلاَمِيذِهِ
\par 2 وَقَالَ لَهُمَا: «اذْهَبَا إِلَى الْقَرْيَةِ الَّتِي أَمَامَكُمَا فَلِلْوَقْتِ وَأَنْتُمَا دَاخِلاَنِ إِلَيْهَا تَجِدَانِ جَحْشاً مَرْبُوطاً لَمْ يَجْلِسْ عَلَيْهِ أَحَدٌ مِنَ النَّاسِ. فَحُلاَّهُ وَأْتِيَا بِهِ.
\par 3 وَإِنْ قَالَ لَكُمَا أَحَدٌ: لِمَاذَا تَفْعَلاَنِ هَذَا؟ فَقُولاَ: الرَّبُّ مُحْتَاجٌ إِلَيْهِ. فَلِلْوَقْتِ يُرْسِلُهُ إِلَى هُنَا».
\par 4 فَمَضَيَا وَوَجَدَا الْجَحْشَ مَرْبُوطاً عِنْدَ الْبَابِ خَارِجاً عَلَى الطَّرِيقِ فَحَلاَّهُ.
\par 5 فَقَالَ لَهُمَا قَوْمٌ مِنَ الْقِيَامِ هُنَاكَ: «مَاذَا تَفْعَلاَنِ تَحُلاَّنِ الْجَحْشَ؟»
\par 6 فَقَالاَ لَهُمْ كَمَا أَوْصَى يَسُوعُ. فَتَرَكُوهُمَا.
\par 7 فَأَتَيَا بِالْجَحْشِ إِلَى يَسُوعَ وَأَلْقَيَا عَلَيْهِ ثِيَابَهُمَا فَجَلَسَ عَلَيْهِ.
\par 8 وَكَثِيرُونَ فَرَشُوا ثِيَابَهُمْ فِي الطَّرِيقِ وَآخَرُونَ قَطَعُوا أَغْصَاناً مِنَ الشَّجَرِ وَفَرَشُوهَا فِي الطَّرِيقِ.
\par 9 وَالَّذِينَ تَقَدَّمُوا وَالَّذِينَ تَبِعُوا كَانُوا يَصْرُخُونَ قَائِلِينَ: «أُوصَنَّا! مُبَارَكٌ الآتِي بِاسْمِ الرَّبِّ!
\par 10 مُبَارَكَةٌ مَمْلَكَةُ أَبِينَا دَاوُدَ الآتِيَةُ بِاسْمِ الرَّبِّ! أُوصَنَّا فِي الأَعَالِي!».
\par 11 فَدَخَلَ يَسُوعُ أُورُشَلِيمَ وَالْهَيْكَلَ وَلَمَّا نَظَرَ حَوْلَهُ إِلَى كُلِّ شَيْءٍ إِذْ كَانَ الْوَقْتُ قَدْ أَمْسَى خَرَجَ إِلَى بَيْتِ عَنْيَا مَعَ الاِثْنَيْ عَشَرَ.
\par 12 وَفِي الْغَدِ لَمَّا خَرَجُوا مِنْ بَيْتِ عَنْيَا جَاعَ
\par 13 فَنَظَرَ شَجَرَةَ تِينٍ مِنْ بَعِيدٍ عَلَيْهَا وَرَقٌ وَجَاءَ لَعَلَّهُ يَجِدُ فِيهَا شَيْئاً. فَلَمَّا جَاءَ إِلَيْهَا لَمْ يَجِدْ شَيْئاً إلاَّ وَرَقاً لأَنَّهُ لَمْ يَكُنْ وَقْتَ التِّينِ.
\par 14 فَقَالَ يَسُوعُ لَهَا: «لاَ يَأْكُلْ أَحَدٌ مِنْكِ ثَمَراً بَعْدُ إِلَى الأَبَدِ». وَكَانَ تَلاَمِيذُهُ يَسْمَعُونَ.
\par 15 وَجَاءُوا إِلَى أُورُشَلِيمَ. وَلَمَّا دَخَلَ يَسُوعُ الْهَيْكَلَ ابْتَدَأَ يُخْرِجُ الَّذِينَ كَانُوا يَبِيعُونَ وَيَشْتَرُونَ فِي الْهَيْكَلِ وَقَلَّبَ مَوَائِدَ الصَّيَارِفَةِ وَكَرَاسِيَّ بَاعَةِ الْحَمَامِ.
\par 16 وَلَمْ يَدَعْ أَحَداً يَجْتَازُ الْهَيْكَلَ بِمَتَاعٍ.
\par 17 وَكَانَ يُعَلِّمُ قَائِلاً لَهُمْ: «أَلَيْسَ مَكْتُوباً: بَيْتِي بَيْتَ صَلاَةٍ يُدْعَى لِجَمِيعِ الأُمَمِ؟ وَأَنْتُمْ جَعَلْتُمُوهُ مَغَارَةَ لُصُوصٍ».
\par 18 وَسَمِعَ الْكَتَبَةُ وَرُؤَسَاءُ الْكَهَنَةِ فَطَلَبُوا كَيْفَ يُهْلِكُونَهُ لأَنَّهُمْ خَافُوهُ إِذْ بُهِتَ الْجَمْعُ كُلُّهُ مِنْ تَعْلِيمِهِ.
\par 19 وَلَمَّا صَارَ الْمَسَاءُ خَرَجَ إِلَى خَارِجِ الْمَدِينَةِ.
\par 20 وَفِي الصَّبَاحِ إِذْ كَانُوا مُجْتَازِينَ رَأَوُا التِّينَةَ قَدْ يَبِسَتْ مِنَ الأُصُولِ
\par 21 فَتَذَكَّرَ بُطْرُسُ وَقَالَ لَهُ: «يَا سَيِّدِي انْظُرْ التِّينَةُ الَّتِي لَعَنْتَهَا قَدْ يَبِسَتْ!»
\par 22 فَأَجَابَ يَسُوعُ: «لِيَكُنْ لَكُمْ إِيمَانٌ بِاللَّهِ.
\par 23 لأَنِّي الْحَقَّ أَقُولُ لَكُمْ: إِنَّ مَنْ قَالَ لِهَذَا الْجَبَلِ انْتَقِلْ وَانْطَرِحْ فِي لْبَحْرِ وَلاَ يَشُكُّ فِي قَلْبِهِ بَلْ يُؤْمِنُ أَنَّ مَا يَقُولُهُ يَكُونُ فَمَهْمَا قَالَ يَكُونُ لَهُ.
\par 24 لِذَلِكَ أَقُولُ لَكُمْ: كُلُّ مَا تَطْلُبُونَهُ حِينَمَا تُصَلُّونَ فَآمِنُوا أَنْ تَنَالُوهُ فَيَكُونَ لَكُمْ.
\par 25 وَمَتَى وَقَفْتُمْ تُصَلُّونَ فَاغْفِرُوا إِنْ كَانَ لَكُمْ عَلَى أَحَدٍ شَيْءٌ لِكَيْ يَغْفِرَ لَكُمْ أَيْضاً أَبُوكُمُ الَّذِي فِي السَّمَاوَاتِ زَلاَّتِكُمْ.
\par 26 وَإِنْ لَمْ تَغْفِرُوا أَنْتُمْ لاَ يَغْفِرْ أَبُوكُمُ الَّذِي فِي السَّمَاوَاتِ أَيْضاً زَلاَّتِكُمْ».
\par 27 وَجَاءُوا أَيْضاً إِلَى أُورُشَلِيمَ. وَفِيمَا هُوَ يَمْشِي فِي الْهَيْكَلِ أَقْبَلَ إِلَيْهِ رُؤَسَاءُ الْكَهَنَةِ وَالْكَتَبَةُ وَالشُّيُوخُ
\par 28 وَقَالُوا لَهُ: «بِأَيِّ سُلْطَانٍ تَفْعَلُ هَذَا وَمَنْ أَعْطَاكَ هَذَا السُّلْطَانَ حَتَّى تَفْعَلَ هَذَا؟»
\par 29 فَأَجَابَ يَسُوعُ: «وَأَنَا أَيْضاً أَسْأَلُكُمْ كَلِمَةً وَاحِدَةً. أَجِيبُونِي فَأَقُولَ لَكُمْ بِأَيِّ سُلْطَانٍ أَفْعَلُ هَذَا:
\par 30 مَعْمُودِيَّةُ يُوحَنَّا: مِنَ السَّمَاءِ كَانَتْ أَمْ مِنَ النَّاسِ؟ أَجِيبُونِي».
\par 31 فَفَكَّرُوا فِي أَنْفُسِهِمْ قَائِلِينَ: «إِنْ قُلْنَا مِنَ السَّمَاءِ يَقُولُ: فَلِمَاذَا لَمْ تُؤْمِنُوا بِهِ؟
\par 32 وَإِنْ قُلْنَا مِنَ النَّاسِ». فَخَافُوا الشَّعْبَ. لأَنَّ يُوحَنَّا كَانَ عِنْدَ الْجَمِيعِ أَنَّهُ بِالْحَقِيقَةِ نَبِيٌّ.
\par 33 فَأَجَابُوا: «لاَ نَعْلَمُ». فَقَالَ يَسُوعُ: «وَلاَ أَنَا أَقُولُ لَكُمْ بِأَيِّ سُلْطَانٍ أَفْعَلُ هَذَا».

\chapter{12}

\par 1 وَابْتَدَأَ يَقُولُ لَهُمْ بِأَمْثَالٍ: «إِنْسَانٌ غَرَسَ كَرْماً وَأَحَاطَهُ بِسِيَاجٍ وَحَفَرَ حَوْضَ مَعْصَرَةٍ وَبَنَى بُرْجاً وَسَلَّمَهُ إِلَى كَرَّامِينَ وَسَافَرَ.
\par 2 ثُمَّ أَرْسَلَ إِلَى الْكَرَّامِينَ فِي الْوَقْتِ عَبْداً لِيَأْخُذَ مِنَ الْكَرَّامِينَ مِنْ ثَمَرِ الْكَرْمِ
\par 3 فَأَخَذُوهُ وَجَلَدُوهُ وَأَرْسَلُوهُ فَارِغاً.
\par 4 ثُمَّ أَرْسَلَ إِلَيْهِمْ أَيْضاً عَبْداً آخَرَ فَرَجَمُوهُ وَشَجُّوهُ وَأَرْسَلُوهُ مُهَاناً.
\par 5 ثُمَّ أَرْسَلَ أَيْضاً آخَرَ فَقَتَلُوهُ. ثُمَّ آخَرِينَ كَثِيرِينَ فَجَلَدُوا مِنْهُمْ بَعْضاً وَقَتَلُوا بَعْضاً.
\par 6 فَإِذْ كَانَ لَهُ أَيْضاً ابْنٌ وَاحِدٌ حَبِيبٌ إِلَيْهِ أَرْسَلَهُ أَيْضاً إِلَيْهِمْ أَخِيراً قَائِلاً: إِنَّهُمْ يَهَابُونَ ابْنِي.
\par 7 وَلَكِنَّ أُولَئِكَ الْكَرَّامِينَ قَالُوا فِيمَا بَيْنَهُمْ: هَذَا هُوَ الْوَارِثُ! هَلُمُّوا نَقْتُلْهُ فَيَكُونَ لَنَا الْمِيرَاثُ!
\par 8 فَأَخَذُوهُ وَقَتَلُوهُ وَأَخْرَجُوهُ خَارِجَ الْكَرْمِ.
\par 9 فَمَاذَا يَفْعَلُ صَاحِبُ الْكَرْمِ؟ يَأْتِي وَيُهْلِكُ الْكَرَّامِينَ وَيُعْطِي الْكَرْمَ إِلَى آخَرِينَ.
\par 10 أَمَا قَرَأْتُمْ هَذَا الْمَكْتُوبَ: الْحَجَرُ الَّذِي رَفَضَهُ الْبَنَّاؤُونَ هُوَ قَدْ صَارَ رَأْسَ الزَّاوِيَةِ
\par 11 مِنْ قِبَلِ الرَّبِّ كَانَ هَذَا وَهُوَ عَجِيبٌ فِي أَعْيُنِنَا!»
\par 12 فَطَلَبُوا أَنْ يُمْسِكُوهُ وَلَكِنَّهُمْ خَافُوا مِنَ الْجَمْعِ لأَنَّهُمْ عَرَفُوا أَنَّهُ قَالَ الْمَثَلَ عَلَيْهِمْ. فَتَرَكُوهُ وَمَضَوْا.
\par 13 ثُمَّ أَرْسَلُوا إِلَيْهِ قَوْماً مِنَ الْفَرِّيسِيِّينَ وَالْهِيرُودُسِيِّينَ لِكَيْ يَصْطَادُوهُ بِكَلِمَةٍ.
\par 14 فَلَمَّا جَاءُوا قَالُوا لَهُ: «يَا مُعَلِّمُ نَعْلَمُ أَنَّكَ صَادِقٌ وَلاَ تُبَالِي بِأَحَدٍ لأَنَّكَ لاَ تَنْظُرُ إِلَى وُجُوهِ النَّاسِ بَلْ بِالْحَقِّ تُعَلِّمُ طَرِيقَ اللَّهِ. أَيَجُوزُ أَنْ تُعْطَى جِزْيَةٌ لِقَيْصَرَ أَمْ لاَ؟ نُعْطِي أَمْ لاَ نُعْطِي؟»
\par 15 فَعَلِمَ رِيَاءَهُمْ وَقَالَ لَهُمْ: «لِمَاذَا تُجَرِّبُونَنِي؟ ايتُونِي بِدِينَارٍ لأَنْظُرَهُ».
\par 16 فَأَتَوْا بِهِ. فَقَالَ لَهُمْ: «لِمَنْ هَذِهِ الصُّورَةُ وَالْكِتَابَةُ؟» فَقَالُوا لَهُ: «لِقَيْصَرَ».
\par 17 فَأَجَابَ يَسُوعُ: «أَعْطُوا مَا لِقَيْصَرَ لِقَيْصَرَ وَمَا لِلَّهِ لِلَّهِ». فَتَعَجَّبُوا مِنْهُ.
\par 18 وَجَاءَ إِلَيْهِ قَوْمٌ مِنَ الصَّدُّوقِيِّينَ الَّذِينَ يَقُولُونَ لَيْسَ قِيَامَةٌ وَسَأَلُوهُ:
\par 19 «يَا مُعَلِّمُ كَتَبَ لَنَا مُوسَى: إِنْ مَاتَ لأَحَدٍ أَخٌ وَتَرَكَ امْرَأَةً وَلَمْ يُخَلِّفْ أَوْلاَداً أَنْ يَأْخُذَ أَخُوهُ امْرَأَتَهُ وَيُقِيمَ نَسْلاً لأَخِيهِ.
\par 20 فَكَانَ سَبْعَةُ إِخْوَةٍ. أَخَذَ الأَوَّلُ امْرَأَةً وَمَاتَ وَلَمْ يَتْرُكْ نَسْلاً.
\par 21 فَأَخَذَهَا الثَّانِي وَمَاتَ وَلَمْ يَتْرُكْ هُوَ أَيْضاً نَسْلاً. وَهَكَذَا الثَّالِثُ.
\par 22 فَأَخَذَهَا السَّبْعَةُ وَلَمْ يَتْرُكُوا نَسْلاً. وَآخِرَ الْكُلِّ مَاتَتِ الْمَرْأَةُ أَيْضاً.
\par 23 فَفِي الْقِيَامَةِ مَتَى قَامُوا لِمَنْ مِنْهُمْ تَكُونُ زَوْجَةً؟ لأَنَّهَا كَانَتْ زَوْجَةً لِلسَّبْعَةِ».
\par 24 فَأَجَابَ يَسُوعُ: «أَلَيْسَ لِهَذَا تَضِلُّونَ إِذْ لاَ تَعْرِفُونَ الْكُتُبَ وَلاَ قُوَّةَ اللَّهِ؟
\par 25 لأَنَّهُمْ مَتَى قَامُوا مِنَ الأَمْوَاتِ لاَ يُزَوِّجُونَ وَلاَ يُزَوَّجُونَ بَلْ يَكُونُونَ كَمَلاَئِكَةٍ فِي السَّمَاوَاتِ.
\par 26 وَأَمَّا مِنْ جِهَةِ الأَمْوَاتِ إِنَّهُمْ يَقُومُونَ: أَفَمَا قَرَأْتُمْ فِي كِتَابِ مُوسَى فِي أَمْرِ الْعُلَّيْقَةِ كَيْفَ كَلَّمَهُ اللَّهُ قَائِلاً: أَنَا إِلَهُ إِبْرَاهِيمَ وَإِلَهُ إِسْحَاقَ وَإِلَهُ يَعْقُوبَ؟
\par 27 لَيْسَ هُوَ إِلَهَ أَمْوَاتٍ بَلْ إِلَهُ أَحْيَاءٍ. فَأَنْتُمْ إِذاً تَضِلُّونَ كَثِيراً».
\par 28 فَجَاءَ وَاحِدٌ مِنَ الْكَتَبَةِ وَسَمِعَهُمْ يَتَحَاوَرُونَ فَلَمَّا رَأَى أَنَّهُ أَجَابَهُمْ حَسَناً سَأَلَهُ: «أَيَّةُ وَصِيَّةٍ هِيَ أَوَّلُ الْكُلِّ؟»
\par 29 فَأَجَابَهُ يَسُوعُ: «إِنَّ أَوَّلَ كُلِّ الْوَصَايَا هِيَ: اسْمَعْ يَا إِسْرَائِيلُ. الرَّبُّ إِلَهُنَا رَبٌّ وَاحِدٌ.
\par 30 وَتُحِبُّ الرَّبَّ إِلَهَكَ مِنْ كُلِّ قَلْبِكَ وَمِنْ كُلِّ نَفْسِكَ وَمِنْ كُلِّ فِكْرِكَ وَمِنْ كُلِّ قُدْرَتِكَ. هَذِهِ هِيَ الْوَصِيَّةُ الأُولَى.
\par 31 وَثَانِيَةٌ مِثْلُهَا هِيَ: تُحِبُّ قَرِيبَكَ كَنَفْسِكَ. لَيْسَ وَصِيَّةٌ أُخْرَى أَعْظَمَ مِنْ هَاتَيْنِ».
\par 32 فَقَالَ لَهُ الْكَاتِبُ: «جَيِّداً يَا مُعَلِّمُ. بِالْحَقِّ قُلْتَ لأَنَّهُ اللَّهُ وَاحِدٌ وَلَيْسَ آخَرُ سِوَاهُ.
\par 33 وَمَحَبَّتُهُ مِنْ كُلِّ الْقَلْبِ وَمِنْ كُلِّ الْفَهْمِ وَمِنْ كُلِّ النَّفْسِ وَمِنْ كُلِّ الْقُدْرَةِ وَمَحَبَّةُ الْقَرِيبِ كَالنَّفْسِ هِيَ أَفْضَلُ مِنْ جَمِيعِ الْمُحْرَقَاتِ وَالذَّبَائِحِ».
\par 34 فَلَمَّا رَآهُ يَسُوعُ أَنَّهُ أَجَابَ بِعَقْلٍ قَالَ لَهُ: «لَسْتَ بَعِيداً عَنْ مَلَكُوتِ اللَّهِ». وَلَمْ يَجْسُرْ أَحَدٌ بَعْدَ ذَلِكَ أَنْ يَسْأَلَهُ!
\par 35 ثُمَّ سَأَلَ يَسُوعُ وَهُوَ يُعَلِّمُ فِي الْهَيْكَلِ: «كَيْفَ يَقُولُ الْكَتَبَةُ إِنَّ الْمَسِيحَ ابْنُ دَاوُدَ؟
\par 36 لأَنَّ دَاوُدَ نَفْسَهُ قَالَ بِالرُّوحِ الْقُدُسِ: قَالَ الرَّبُّ لِرَبِّي: اجْلِسْ عَنْ يَمِينِي حَتَّى أَضَعَ أَعْدَاءَكَ مَوْطِئاً لِقَدَمَيْكَ.
\par 37 فَدَاوُدُ نَفْسُهُ يَدْعُوهُ رَبّاً. فَمِنْ أَيْنَ هُوَ ابْنُهُ؟» وَكَانَ الْجَمْعُ الْكَثِيرُ يَسْمَعُهُ بِسُرُورٍ.
\par 38 وَقَالَ لَهُمْ فِي تَعْلِيمِهِ: «تَحَرَّزُوا مِنَ الْكَتَبَةِ الَّذِينَ يَرْغَبُونَ الْمَشْيَ بِالطَّيَالِسَةِ وَالتَّحِيَّاتِ فِي الأَسْوَاقِ
\par 39 وَالْمَجَالِسَ الأُولَى فِي الْمَجَامِعِ وَالْمُتَّكَآتِ الأُولَى فِي الْوَلاَئِمِ.
\par 40 الَّذِينَ يَأْكُلُونَ بُيُوتَ الأَرَامِلِ وَلِعِلَّةٍ يُطِيلُونَ الصَّلَوَاتِ. هَؤُلاَءِ يَأْخُذُونَ دَيْنُونَةً أَعْظَمَ».
\par 41 وَجَلَسَ يَسُوعُ تُجَاهَ الْخِزَانَةِ وَنَظَرَ كَيْفَ يُلْقِي الْجَمْعُ نُحَاساً فِي الْخِزَانَةِ. وَكَانَ أَغْنِيَاءُ كَثِيرُونَ يُلْقُونَ كَثِيراً.
\par 42 فَجَاءَتْ أَرْمَلَةٌ فَقِيرَةٌ وَأَلْقَتْ فَلْسَيْنِ قِيمَتُهُمَا رُبْعٌ.
\par 43 فَدَعَا تَلاَمِيذَهُ وَقَالَ لَهُمُ: «الْحَقَّ أَقُولُ لَكُمْ: إِنَّ هَذِهِ الأَرْمَلَةَ الْفَقِيرَةَ قَدْ أَلْقَتْ أَكْثَرَ مِنْ جَمِيعِ الَّذِينَ أَلْقَوْا فِي الْخِزَانَةِ
\par 44 لأَنَّ الْجَمِيعَ مِنْ فَضْلَتِهِمْ أَلْقَوْا. وَأَمَّا هَذِهِ فَمِنْ إِعْوَازِهَا أَلْقَتْ كُلَّ مَا عِنْدَهَا كُلَّ مَعِيشَتِهَا».

\chapter{13}

\par 1 وَفِيمَا هُوَ خَارِجٌ مِنَ الْهَيْكَلِ قَالَ لَهُ وَاحِدٌ مِنْ تَلاَمِيذِهِ: «يَا مُعَلِّمُ انْظُرْ مَا هَذِهِ الْحِجَارَةُ وَهَذِهِ الأَبْنِيَةُ؟»
\par 2 فَأَجَابَ يَسُوعُ: «أَتَنْظُرُ هَذِهِ الأَبْنِيَةَ الْعَظِيمَةَ؟ لاَ يُتْرَكُ حَجَرٌ عَلَى حَجَرٍ لاَ يُنْقَضُ».
\par 3 وَفِيمَا هُوَ جَالِسٌ عَلَى جَبَلِ الزَّيْتُونِ تُجَاهَ الْهَيْكَلِ سَأَلَهُ بُطْرُسُ وَيَعْقُوبُ وَيُوحَنَّا وَأَنْدَرَاوُسُ عَلَى انْفِرَادٍ:
\par 4 «قُلْ لَنَا مَتَى يَكُونُ هَذَا وَمَا هِيَ الْعَلاَمَةُ عِنْدَمَا يَتِمُّ جَمِيعُ هَذَا؟»
\par 5 فَأَجَابَهُمْ يَسُوعُ: «انْظُرُوا! لاَ يُضِلُّكُمْ أَحَدٌ.
\par 6 فَإِنَّ كَثِيرِينَ سَيَأْتُونَ بِاسْمِي قَائِلِينَ: إِنِّي أَنَا هُوَ. وَيُضِلُّونَ كَثِيرِينَ.
\par 7 فَإِذَا سَمِعْتُمْ بِحُرُوبٍ وَبِأَخْبَارِ حُرُوبٍ فَلاَ تَرْتَاعُوا لأَنَّهَا لاَ بُدَّ أَنْ تَكُونَ وَلَكِنْ لَيْسَ الْمُنْتَهَى بَعْدُ.
\par 8 لأَنَّهُ تَقُومُ أُمَّةٌ عَلَى أُمَّةٍ وَمَمْلَكَةٌ عَلَى مَمْلَكَةٍ وَتَكُونُ زَلاَزِلُ فِي أَمَاكِنَ وَتَكُونُ مَجَاعَاتٌ وَاضْطِرَابَاتٌ. هَذِهِ مُبْتَدَأُ الأَوْجَاعِ.
\par 9 فَانْظُرُوا إِلَى نُفُوسِكُمْ. لأَنَّهُمْ سَيُسَلِّمُونَكُمْ إِلَى مَجَالِسَ وَتُجْلَدُونَ فِي مَجَامِعَ وَتُوقَفُونَ أَمَامَ وُلاَةٍ وَمُلُوكٍ مِنْ أَجْلِي شَهَادَةً لَهُمْ.
\par 10 وَيَنْبَغِي أَنْ يُكْرَزَ أَوَّلاً بِالإِنْجِيلِ فِي جَمِيعِ الأُمَمِ.
\par 11 فَمَتَى سَاقُوكُمْ لِيُسَلِّمُوكُمْ فَلاَ تَعْتَنُوا مِنْ قَبْلُ بِمَا تَتَكَلَّمُونَ وَلاَ تَهْتَمُّوا بَلْ مَهْمَا أُعْطِيتُمْ فِي تِلْكَ السَّاعَةِ فَبِذَلِكَ تَكَلَّمُوا لأَنْ لَسْتُمْ أَنْتُمُ الْمُتَكَلِّمِينَ بَلِ الرُّوحُ الْقُدُسُ.
\par 12 وَسَيُسْلِمُ الأَخُ أَخَاهُ إِلَى الْمَوْتِ وَالأَبُ وَلَدَهُ وَيَقُومُ الأَوْلاَدُ عَلَى وَالِدِيهِمْ وَيَقْتُلُونَهُمْ.
\par 13 وَتَكُونُونَ مُبْغَضِينَ مِنَ الْجَمِيعِ مِنْ أَجْلِ اسْمِي. وَلَكِنَّ الَّذِي يَصْبِرُ إِلَى الْمُنْتَهَى فَهَذَا يَخْلُصُ.
\par 14 فَمَتَى نَظَرْتُمْ «رِجْسَةَ الْخَرَابِ» الَّتِي قَالَ عَنْهَا دَانِيآلُ النَّبِيُّ قَائِمَةً حَيْثُ لاَ يَنْبَغِي - لِيَفْهَمِ الْقَارِئُ - فَحِينَئِذٍ لِيَهْرُبِ الَّذِينَ فِي الْيَهُودِيَّةِ إِلَى الْجِبَالِ
\par 15 وَالَّذِي عَلَى السَّطْحِ فَلاَ يَنْزِلْ إِلَى الْبَيْتِ وَلاَ يَدْخُلْ لِيَأْخُذَ مِنْ بَيْتِهِ شَيْئاً
\par 16 وَالَّذِي فِي الْحَقْلِ فَلاَ يَرْجِعْ إِلَى الْوَرَاءِ لِيَأْخُذَ ثَوْبَهُ.
\par 17 وَوَيْلٌ لِلْحَبَالَى وَالْمُرْضِعَاتِ فِي تِلْكَ الأَيَّامِ.
\par 18 وَصَلُّوا لِكَيْ لاَ يَكُونَ هَرَبُكُمْ فِي شِتَاءٍ.
\par 19 لأَنَّهُ يَكُونُ فِي تِلْكَ الأَيَّامِ ضِيقٌ لَمْ يَكُنْ مِثْلُهُ مُنْذُ ابْتِدَاءِ الْخَلِيقَةِ الَّتِي خَلَقَهَا اللَّهُ إِلَى الآنَ وَلَنْ يَكُونَ.
\par 20 وَلَوْ لَمْ يُقَصِّرِ الرَّبُّ تِلْكَ الأَيَّامَ لَمْ يَخْلُصْ جَسَدٌ. وَلَكِنْ لأَجْلِ الْمُخْتَارِينَ الَّذِينَ اخْتَارَهُمْ قَصَّرَ الأَيَّامَ.
\par 21 حِينَئِذٍ إِنْ قَالَ لَكُمْ أَحَدٌ: هُوَذَا الْمَسِيحُ هُنَا أَوْ هُوَذَا هُنَاكَ فَلاَ تُصَدِّقُوا.
\par 22 لأَنَّهُ سَيَقُومُ مُسَحَاءُ كَذَبَةٌ وَأَنْبِيَاءُ كَذَبَةٌ وَيُعْطُونَ آيَاتٍ وَعَجَائِبَ لِكَيْ يُضِلُّوا - لَوْ أَمْكَنَ - الْمُخْتَارِينَ أَيْضاً.
\par 23 فَانْظُرُوا أَنْتُمْ. هَا أَنَا قَدْ سَبَقْتُ وَأَخْبَرْتُكُمْ بِكُلِّ شَيْءٍ.
\par 24 «وَأَمَّا فِي تِلْكَ الأَيَّامِ بَعْدَ ذَلِكَ الضِّيقِ فَالشَّمْسُ تُظْلِمُ وَالْقَمَرُ لاَ يُعْطِي ضَوْءَهُ
\par 25 وَنُجُومُ السَّمَاءِ تَتَسَاقَطُ وَالْقُوَّاتُ الَّتِي فِي السَّمَاوَاتِ تَتَزَعْزَعُ.
\par 26 وَحِينَئِذٍ يُبْصِرُونَ ابْنَ الإِنْسَانِ آتِياً فِي سَحَابٍ بِقُوَّةٍ كَثِيرَةٍ وَمَجْدٍ
\par 27 فَيُرْسِلُ حِينَئِذٍ مَلاَئِكَتَهُ وَيَجْمَعُ مُخْتَارِيهِ مِنَ الأَرْبَعِ الرِّيَاحِ مِنْ أَقْصَاءِ الأَرْضِ إِلَى أَقْصَاءِ السَّمَاءِ.
\par 28 فَمِنْ شَجَرَةِ التِّينِ تَعَلَّمُوا الْمَثَلَ: مَتَى صَارَ غُصْنُهَا رَخْصاً وَأَخْرَجَتْ أَوْرَاقاً تَعْلَمُونَ أَنَّ الصَّيْفَ قَرِيبٌ.
\par 29 هَكَذَا أَنْتُمْ أَيْضاً مَتَى رَأَيْتُمْ هَذِهِ الأَشْيَاءَ صَائِرَةً فَاعْلَمُوا أَنَّهُ قَرِيبٌ عَلَى الأَبْوَابِ.
\par 30 اَلْحَقَّ أَقُولُ لَكُمْ: لاَ يَمْضِي هَذَا الْجِيلُ حَتَّى يَكُونَ هَذَا كُلُّهُ.
\par 31 اَلسَّمَاءُ وَالأَرْضُ تَزُولاَنِ وَلَكِنَّ كَلاَمِي لاَ يَزُولُ.
\par 32 وَأَمَّا ذَلِكَ الْيَوْمُ وَتِلْكَ السَّاعَةُ فَلاَ يَعْلَمُ بِهِمَا أَحَدٌ وَلاَ الْمَلاَئِكَةُ الَّذِينَ فِي السَّمَاءِ وَلاَ الاِبْنُ إلاَّ الآبُ.
\par 33 انْظُرُوا! اسْهَرُوا وَصَلُّوا لأَنَّكُمْ لاَ تَعْلَمُونَ مَتَى يَكُونُ الْوَقْتُ.
\par 34 كَأَنَّمَا إِنْسَانٌ مُسَافِرٌ تَرَكَ بَيْتَهُ وَأَعْطَى عَبِيدَهُ السُّلْطَانَ وَلِكُلِّ وَاحِدٍ عَمَلَهُ وَأَوْصَى الْبَوَّابَ أَنْ يَسْهَرَ.
\par 35 اسْهَرُوا إِذاً لأَنَّكُمْ لاَ تَعْلَمُونَ مَتَى يَأْتِي رَبُّ الْبَيْتِ أَمَسَاءً أَمْ نِصْفَ اللَّيْلِ أَمْ صِيَاحَ الدِّيكِ أَمْ صَبَاحاً.
\par 36 لِئَلاَّ يَأْتِيَ بَغْتَةً فَيَجِدَكُمْ نِيَاماً!
\par 37 وَمَا أَقُولُهُ لَكُمْ أَقُولُهُ لِلْجَمِيعِ: اسْهَرُوا».

\chapter{14}

\par 1 وَكَانَ الْفِصْحُ وَأَيَّامُ الْفَطِيرِ بَعْدَ يَوْمَيْنِ. وَكَانَ رُؤَسَاءُ الْكَهَنَةِ وَالْكَتَبَةُ يَطْلُبُونَ كَيْفَ يُمْسِكُونَهُ بِمَكْرٍ وَيَقْتُلُونَهُ
\par 2 وَلَكِنَّهُمْ قَالُوا: «لَيْسَ فِي الْعِيدِ لِئَلاَّ يَكُونَ شَغَبٌ فِي الشَّعْبِ».
\par 3 وَفِيمَا هُوَ فِي بَيْتِ عَنْيَا فِي بَيْتِ سِمْعَانَ الأَبْرَصِ وَهُوَ مُتَّكِئٌ جَاءَتِ امْرَأَةٌ مَعَهَا قَارُورَةُ طِيبِ نَارِدِينٍ خَالِصٍ كَثِيرِ الثَّمَنِ. فَكَسَرَتِ الْقَارُورَةَ وَسَكَبَتْهُ عَلَى رَأْسِهِ.
\par 4 وَكَانَ قَوْمٌ مُغْتَاظِينَ فِي أَنْفُسِهِمْ فَقَالُوا: «لِمَاذَا كَانَ تَلَفُ الطِّيبِ هَذَا؟
\par 5 لأَنَّهُ كَانَ يُمْكِنُ أَنْ يُبَاعَ هَذَا بِأَكْثَرَ مِنْ ثَلاَثِمِئَةِ دِينَارٍ وَيُعْطَى لِلْفُقَرَاءِ». وَكَانُوا يُؤَنِّبُونَهَا.
\par 6 أَمَّا يَسُوعُ فَقَالَ: «اتْرُكُوهَا! لِمَاذَا تُزْعِجُونَهَا؟ قَدْ عَمِلَتْ بِي عَمَلاً حَسَناً.
\par 7 لأَنَّ الْفُقَرَاءَ مَعَكُمْ فِي كُلِّ حِينٍ وَمَتَى أَرَدْتُمْ تَقْدِرُونَ أَنْ تَعْمَلُوا بِهِمْ خَيْراً. وَأَمَّا أَنَا فَلَسْتُ مَعَكُمْ فِي كُلِّ حِينٍ.
\par 8 عَمِلَتْ مَا عِنْدَهَا. قَدْ سَبَقَتْ وَدَهَنَتْ بِالطِّيبِ جَسَدِي لِلتَّكْفِينِ.
\par 9 اَلْحَقَّ أَقُولُ لَكُمْ: حَيْثُمَا يُكْرَزْ بِهَذَا الإِنْجِيلِ فِي كُلِّ الْعَالَمِ يُخْبَرْ أَيْضاً بِمَا فَعَلَتْهُ هَذِهِ تَذْكَاراً لَهَا».
\par 10 ثُمَّ إِنَّ يَهُوذَا الإِسْخَرْيُوطِيَّ وَاحِداً مِنَ الاِثْنَيْ عَشَرَ مَضَى إِلَى رُؤَسَاءِ الْكَهَنَةِ لِيُسَلِّمَهُ إِلَيْهِمْ.
\par 11 وَلَمَّا سَمِعُوا فَرِحُوا وَوَعَدُوهُ أَنْ يُعْطُوهُ فِضَّةً. وَكَانَ يَطْلُبُ كَيْفَ يُسَلِّمُهُ فِي فُرْصَةٍ مُوافِقَةٍ.
\par 12 وَفِي الْيَوْمِ الأَوَّلِ مِنَ الْفَطِيرِ. حِينَ كَانُوا يَذْبَحُونَ الْفِصْحَ قَالَ لَهُ تَلاَمِيذُهُ: «أَيْنَ تُرِيدُ أَنْ نَمْضِيَ وَنُعِدَّ لِتَأْكُلَ الْفِصْحَ؟»
\par 13 فَأَرْسَلَ اثْنَيْنِ مِنْ تَلاَمِيذِهِ وَقَالَ لَهُمَا: «اذْهَبَا إِلَى الْمَدِينَةِ فَيُلاَقِيَكُمَا إِنْسَانٌ حَامِلٌ جَرَّةَ مَاءٍ. اتْبَعَاهُ.
\par 14 وَحَيْثُمَا يَدْخُلْ فَقُولاَ لِرَبِّ الْبَيْتِ: إِنَّ الْمُعَلِّمَ يَقُولُ: أَيْنَ الْمَنْزِلُ حَيْثُ آكُلُ الْفِصْحَ مَعَ تَلاَمِيذِي؟
\par 15 فَهُوَ يُرِيكُمَا عِلِّيَّةً كَبِيرَةً مَفْرُوشَةً مُعَدَّةً. هُنَاكَ أَعِدَّا لَنَا».
\par 16 فَخَرَجَ تِلْمِيذَاهُ وَأَتَيَا إِلَى الْمَدِينَةِ وَوَجَدَا كَمَا قَالَ لَهُمَا. فَأَعَدَّا الْفِصْحَ.
\par 17 وَلَمَّا كَانَ الْمَسَاءُ جَاءَ مَعَ الاِثْنَيْ عَشَرَ.
\par 18 وَفِيمَا هُمْ مُتَّكِئُونَ يَأْكُلُونَ قَالَ يَسُوعُ: «الْحَقَّ أَقُولُ لَكُمْ: إِنَّ وَاحِداً مِنْكُمْ يُسَلِّمُنِي. اَلآكِلُ مَعِي!»
\par 19 فَابْتَدَأُوا يَحْزَنُونَ وَيَقُولُونَ لَهُ وَاحِداً فَوَاحِداً: «هَلْ أَنَا؟» وَآخَرُ: «هَلْ أَنَا؟»
\par 20 فَأَجَابَ: «هُوَ وَاحِدٌ مِنَ الاِثْنَيْ عَشَرَ الَّذِي يَغْمِسُ مَعِي فِي الصَّحْفَةِ.
\par 21 إِنَّ ابْنَ الإِنْسَانِ مَاضٍ كَمَا هُوَ مَكْتُوبٌ عَنْهُ وَلَكِنْ وَيْلٌ لِذَلِكَ الرَّجُلِ الَّذِي بِهِ يُسَلَّمُ ابْنُ الإِنْسَانِ. كَانَ خَيْراً لِذَلِكَ الرَّجُلِ لَوْ لَمْ يُولَدْ!».
\par 22 وَفِيمَا هُمْ يَأْكُلُونَ أَخَذَ يَسُوعُ خُبْزاً وَبَارَكَ وَكَسَّرَ وَأَعْطَاهُمْ وَقَالَ: «خُذُوا كُلُوا هَذَا هُوَ جَسَدِي».
\par 23 ثُمَّ أَخَذَ الْكَأْسَ وَشَكَرَ وَأَعْطَاهُمْ فَشَرِبُوا مِنْهَا كُلُّهُمْ.
\par 24 وَقَالَ لَهُمْ: «هَذَا هُوَ دَمِي الَّذِي لِلْعَهْدِ الْجَدِيدِ الَّذِي يُسْفَكُ مِنْ أَجْلِ كَثِيرِينَ.
\par 25 اَلْحَقَّ أَقُولُ لَكُمْ: إِنِّي لاَ أَشْرَبُ بَعْدُ مِنْ نِتَاجِ الْكَرْمَةِ إِلَى ذَلِكَ الْيَوْمِ حِينَمَا أَشْرَبُهُ جَدِيداً فِي مَلَكُوتِ اللَّهِ».
\par 26 ثُمَّ سَبَّحُوا وَخَرَجُوا إِلَى جَبَلِ الزَّيْتُونِ.
\par 27 وَقَالَ لَهُمْ يَسُوعُ: «إِنَّ كُلَّكُمْ تَشُكُّونَ فِيَّ فِي هَذِهِ اللَّيْلَةِ لأَنَّهُ مَكْتُوبٌ: أَنِّي أَضْرِبُ الرَّاعِيَ فَتَتَبَدَّدُ الْخِرَافُ.
\par 28 وَلَكِنْ بَعْدَ قِيَامِي أَسْبِقُكُمْ إِلَى الْجَلِيلِ».
\par 29 فَقَالَ لَهُ بُطْرُسُ: «وَإِنْ شَكَّ الْجَمِيعُ فَأَنَا لاَ أَشُكُّ!»
\par 30 فَقَالَ لَهُ يَسُوعُ: «الْحَقَّ أَقُولُ لَكَ إِنَّكَ الْيَوْمَ فِي هَذِهِ اللَّيْلَةِ قَبْلَ أَنْ يَصِيحَ الدِّيكُ مَرَّتَيْنِ تُنْكِرُنِي ثَلاَثَ مَرَّاتٍ».
\par 31 فَقَالَ بِأَكْثَرِ تَشْدِيدٍ: «وَلَوِ اضْطُرِرْتُ أَنْ أَمُوتَ مَعَكَ لاَ أُنْكِرُكَ». وَهَكَذَا قَالَ أَيْضاً الْجَمِيعُ.
\par 32 وَجَاءُوا إِلَى ضَيْعَةٍ اسْمُهَا جَثْسَيْمَانِي فَقَالَ لِتَلاَمِيذِهِ: «اجْلِسُوا هَهُنَا حَتَّى أُصَلِّيَ».
\par 33 ثُمَّ أَخَذَ مَعَهُ بُطْرُسَ وَيَعْقُوبَ وَيُوحَنَّا وَابْتَدَأَ يَدْهَشُ وَيَكْتَئِبُ.
\par 34 فَقَالَ لَهُمْ: «نَفْسِي حَزِينَةٌ جِدّاً حَتَّى الْمَوْتِ! امْكُثُوا هُنَا وَاسْهَرُوا».
\par 35 ثُمَّ تَقَدَّمَ قَلِيلاً وَخَرَّ عَلَى الأَرْضِ وَكَانَ يُصَلِّي لِكَيْ تَعْبُرَ عَنْهُ السَّاعَةُ إِنْ أَمْكَنَ.
\par 36 وَقَالَ: «يَا أَبَا الآبُ كُلُّ شَيْءٍ مُسْتَطَاعٌ لَكَ فَأَجِزْ عَنِّي هَذِهِ الْكَأْسَ. وَلَكِنْ لِيَكُنْ لاَ مَا أُرِيدُ أَنَا بَلْ مَا تُرِيدُ أَنْتَ».
\par 37 ثُمَّ جَاءَ وَوَجَدَهُمْ نِيَاماً فَقَالَ لِبُطْرُسَ: «يَا سِمْعَانُ أَنْتَ نَائِمٌ! أَمَا قَدَرْتَ أَنْ تَسْهَرَ سَاعَةً وَاحِدَةً؟
\par 38 اِسْهَرُوا وَصَلُّوا لِئَلاَّ تَدْخُلُوا فِي تَجْرِبَةٍ. أَمَّا الرُّوحُ فَنَشِيطٌ وَأَمَّا الْجَسَدُ فَضَعِيفٌ».
\par 39 وَمَضَى أَيْضاً وَصَلَّى قَائِلاً ذَلِكَ الْكَلاَمَ بِعَيْنِهِ.
\par 40 ثُمَّ رَجَعَ وَوَجَدَهُمْ أَيْضاً نِيَاماً إِذْ كَانَتْ أَعْيُنُهُمْ ثَقِيلَةً فَلَمْ يَعْلَمُوا بِمَاذَا يُجِيبُونَهُ.
\par 41 ثُمَّ جَاءَ ثَالِثَةً وَقَالَ لَهُمْ: «نَامُوا الآنَ وَاسْتَرِيحُوا! يَكْفِي! قَدْ أَتَتِ السَّاعَةُ! هُوَذَا ابْنُ الإِنْسَانِ يُسَلَّمُ إِلَى أَيْدِي الْخُطَاةِ.
\par 42 قُومُوا لِنَذْهَبَ. هُوَذَا الَّذِي يُسَلِّمُنِي قَدِ اقْتَرَبَ».
\par 43 وَلِلْوَقْتِ فِيمَا هُوَ يَتَكَلَّمُ أَقْبَلَ يَهُوذَا وَاحِدٌ مِنَ الاِثْنَيْ عَشَرَ وَمَعَهُ جَمْعٌ كَثِيرٌ بِسُيُوفٍ وَعِصِيٍّ مِنْ عِنْدِ رُؤَسَاءِ الْكَهَنَةِ وَالْكَتَبَةِ وَالشُّيُوخِ.
\par 44 وَكَانَ مُسَلِّمُهُ قَدْ أَعْطَاهُمْ عَلاَمَةً قَائِلاً: «الَّذِي أُقَبِّلُهُ هُوَ هُوَ. أَمْسِكُوهُ وَامْضُوا بِهِ بِحِرْصٍ».
\par 45 فَجَاءَ لِلْوَقْتِ وَتَقَدَّمَ إِلَيْهِ قَائِلاً: «يَا سَيِّدِي يَا سَيِّدِي!» وَقَبَّلَهُ.
\par 46 فَأَلْقَوْا أَيْدِيَهُمْ عَلَيْهِ وَأَمْسَكُوهُ.
\par 47 فَاسْتَلَّ وَاحِدٌ مِنَ الْحَاضِرِينَ السَّيْفَ وَضَرَبَ عَبْدَ رَئِيسِ الْكَهَنَةِ فَقَطَعَ أُذْنَهُ.
\par 48 فَقَالَ يَسُوعُ: «كَأَنَّهُ عَلَى لِصٍّ خَرَجْتُمْ بِسُيُوفٍ وَعِصِيٍّ لِتَأْخُذُونِي!
\par 49 كُلَّ يَوْمٍ كُنْتُ مَعَكُمْ فِي الْهَيْكَلِ أُعَلِّمُ وَلَمْ تُمْسِكُونِي! وَلَكِنْ لِكَيْ تُكْمَلَ الْكُتُبُ».
\par 50 فَتَرَكَهُ الْجَمِيعُ وَهَرَبُوا.
\par 51 وَتَبِعَهُ شَابٌّ لاَبِساً إِزَاراً عَلَى عُرْيِهِ فَأَمْسَكَهُ الشُّبَّانُ
\par 52 فَتَرَكَ الإِزَارَ وَهَرَبَ مِنْهُمْ عُرْيَاناً.
\par 53 فَمَضَوْا بِيَسُوعَ إِلَى رَئِيسِ الْكَهَنَةِ فَاجْتَمَعَ مَعَهُ جَمِيعُ رُؤَسَاءِ الْكَهَنَةِ وَالشُّيُوخُ وَالْكَتَبَةُ.
\par 54 وَكَانَ بُطْرُسُ قَدْ تَبِعَهُ مِنْ بَعِيدٍ إِلَى دَاخِلِ دَارِ رَئِيسِ الْكَهَنَةِ وَكَانَ جَالِساً بَيْنَ الْخُدَّامِ يَسْتَدْفِئُ عِنْدَ النَّارِ.
\par 55 وَكَانَ رُؤَسَاءُ الْكَهَنَةِ وَالْمَجْمَعُ كُلُّهُ يَطْلُبُونَ شَهَادَةً عَلَى يَسُوعَ لِيَقْتُلُوهُ فَلَمْ يَجِدُوا
\par 56 لأَنَّ كَثِيرِينَ شَهِدُوا عَلَيْهِ زُوراً وَلَمْ تَتَّفِقْ شَهَادَاتُهُمْ.
\par 57 ثُمَّ قَامَ قَوْمٌ وَشَهِدُوا عَلَيْهِ زُوراً قَائِلِينَ:
\par 58 «نَحْنُ سَمِعْنَاهُ يَقُولُ: إِنِّي أَنْقُضُ هَذَا الْهَيْكَلَ الْمَصْنُوعَ بِالأَيَادِي وَفِي ثَلاَثَةِ أَيَّامٍ أَبْنِي آخَرَ غَيْرَ مَصْنُوعٍ بِأَيَادٍ».
\par 59 وَلاَ بِهَذَا كَانَتْ شَهَادَتُهُمْ تَتَّفِقُ.
\par 60 فَقَامَ رَئِيسُ الْكَهَنَةِ فِي الْوَسَطِ وَسَأَلَ يَسُوعَ: «أَمَا تُجِيبُ بِشَيْءٍ؟ مَاذَا يَشْهَدُ بِهِ هَؤُلاَءِ عَلَيْكَ؟»
\par 61 أَمَّا هُوَ فَكَانَ سَاكِتاً وَلَمْ يُجِبْ بِشَيْءٍ. فَسَأَلَهُ رَئِيسُ الْكَهَنَةِ أَيْضاً: «أَأَنْتَ الْمَسِيحُ ابْنُ الْمُبَارَكِ؟»
\par 62 فَقَالَ يَسُوعُ: «أَنَا هُوَ. وَسَوْفَ تُبْصِرُونَ ابْنَ الإِنْسَانِ جَالِساً عَنْ يَمِينِ الْقُوَّةِ وَآتِياً فِي سَحَابِ السَّمَاءِ».
\par 63 فَمَزَّقَ رَئِيسُ الْكَهَنَةِ ثِيَابَهُ وَقَالَ: «مَا حَاجَتُنَا بَعْدُ إِلَى شُهُودٍ؟
\par 64 قَدْ سَمِعْتُمُ التَّجَادِيفَ! مَا رَأْيُكُمْ؟» فَالْجَمِيعُ حَكَمُوا عَلَيْهِ أَنَّهُ مُسْتَوْجِبُ الْمَوْتِ.
\par 65 فَابْتَدَأَ قَوْمٌ يَبْصُقُونَ عَلَيْهِ وَيُغَطُّونَ وَجْهَهُ وَيَلْكُمُونَهُ وَيَقُولُونَ لَهُ: «تَنَبَّأْ». وَكَانَ الْخُدَّامُ يَلْطِمُونَهُ.
\par 66 وَبَيْنَمَا كَانَ بُطْرُسُ فِي الدَّارِ أَسْفَلَ جَاءَتْ إِحْدَى جَوَارِي رَئِيسِ الْكَهَنَةِ.
\par 67 فَلَمَّا رَأَتْ بُطْرُسَ يَسْتَدْفِئُ نَظَرَتْ إِلَيْهِ وَقَالَتْ: «وَأَنْتَ كُنْتَ مَعَ يَسُوعَ النَّاصِرِيِّ!»
\par 68 فَأَنْكَرَ قَائِلاً: «لَسْتُ أَدْرِي وَلاَ أَفْهَمُ مَا تَقُولِينَ!» وَخَرَجَ خَارِجاً إِلَى الدِّهْلِيزِ فَصَاحَ الدِّيكُ.
\par 69 فَرَأَتْهُ الْجَارِيَةُ أَيْضاً وَابْتَدَأَتْ تَقُولُ لِلْحَاضِرِينَ: «إِنَّ هَذَا مِنْهُمْ!»
\par 70 فَأَنْكَرَ أَيْضاً. وَبَعْدَ قَلِيلٍ أَيْضاً قَالَ الْحَاضِرُونَ لِبُطْرُسَ: «حَقّاً أَنْتَ مِنْهُمْ لأَنَّكَ جَلِيلِيٌّ أَيْضاً وَلُغَتُكَ تُشْبِهُ لُغَتَهُمْ».
\par 71 فَابْتَدَأَ يَلْعَنُ وَيَحْلِفُ: «إِنِّي لاَ أَعْرِفُ هَذَا الرَّجُلَ الَّذِي تَقُولُونَ عَنْهُ!»
\par 72 وَصَاحَ الدِّيكُ ثَانِيَةً فَتَذَكَّرَ بُطْرُسُ الْقَوْلَ الَّذِي قَالَهُ لَهُ يَسُوعُ: «إِنَّكَ قَبْلَ أَنْ يَصِيحَ الدِّيكُ مَرَّتَيْنِ تُنْكِرُنِي ثَلاَثَ مَرَّاتٍ». فَلَمَّا تَفَكَّرَ بِهِ بَكَى.

\chapter{15}

\par 1 وَلِلْوَقْتِ فِي الصَّبَاحِ تَشَاوَرَ رُؤَسَاءُ الْكَهَنَةِ وَاشُّيُوخُ وَالْكَتَبَةُ وَالْمَجْمَعُ كُلُّهُ فَأَوْثَقُوا يَسُوعَ وَمَضَوْا بِهِ وَأَسْلَمُوهُ إِلَى بِيلاَطُسَ.
\par 2 فَسَأَلَهُ بِيلاَطُسُ: «أَأَنْتَ مَلِكُ الْيَهُودِ؟» فَأَجَابَ: «أَنْتَ تَقُولُ».
\par 3 وَكَانَ رُؤَسَاءُ الْكَهَنَةِ يَشْتَكُونَ عَلَيْهِ كَثِيراً.
\par 4 فَسَأَلَهُ بِيلاَطُسُ أَيْضاً: «أَمَا تُجِيبُ بِشَيْءٍ؟ انْظُرْ كَمْ يَشْهَدُونَ عَلَيْكَ!»
\par 5 فَلَمْ يُجِبْ يَسُوعُ أَيْضاً بِشَيْءٍ حَتَّى تَعَجَّبَ بِيلاَطُسُ.
\par 6 وَكَانَ يُطْلِقُ لَهُمْ فِي كُلِّ عِيدٍ أَسِيراً وَاحِداً مَنْ طَلَبُوهُ.
\par 7 وَكَانَ الْمُسَمَّى بَارَابَاسَ مُوثَقاً مَعَ رُفَقَائِهِ فِي الْفِتْنَةِ الَّذِينَ فِي الْفِتْنَةِ فَعَلُوا قَتْلاً.
\par 8 فَصَرَخَ الْجَمْعُ وَابْتَدَأُوا يَطْلُبُونَ أَنْ يَفْعَلَ كَمَا كَانَ دَائِماً يَفْعَلُ لَهُمْ.
\par 9 فَأَجَابَهُمْ بِيلاَطُسُ: «أَتُرِيدُونَ أَنْ أُطْلِقَ لَكُمْ مَلِكَ الْيَهُودِ؟».
\par 10 لأَنَّهُ عَرَفَ أَنَّ رُؤَسَاءَ الْكَهَنَةِ كَانُوا قَدْ أَسْلَمُوهُ حَسَداً.
\par 11 فَهَيَّجَ رُؤَسَاءُ الْكَهَنَةِ الْجَمْعَ لِكَيْ يُطْلِقَ لَهُمْ بِالْحَرِيِّ بَارَابَاسَ.
\par 12 فَسَأَلَ بِيلاَطُسُ: «فَمَاذَا تُرِيدُونَ أَنْ أَفْعَلَ بِالَّذِي تَدْعُونَهُ مَلِكَ الْيَهُودِ؟»
\par 13 فَصَرَخُوا أَيْضاً: «اصْلِبْهُ!»
\par 14 فَسَأَلَهُمْ بِيلاَطُسُ: «وَأَيَّ شَرٍّ عَمِلَ؟» فَازْدَادُوا جِدّاً صُرَاخاً: «اصْلِبْهُ!»
\par 15 فَبِيلاَطُسُ إِذْ كَانَ يُرِيدُ أَنْ يَعْمَلَ لِلْجَمْعِ مَا يُرْضِيهِمْ أَطْلَقَ لَهُمْ بَارَابَاسَ وَأَسْلَمَ يَسُوعَ بَعْدَمَا جَلَدَهُ لِيُصْلَبَ.
\par 16 فَمَضَى بِهِ الْعَسْكَرُ إِلَى دَاخِلِ الدَّارِ الَّتِي هِيَ دَارُ الْوِلاَيَةِ وَجَمَعُوا كُلَّ الْكَتِيبَةِ.
\par 17 وَأَلْبَسُوهُ أُرْجُواناً وَضَفَرُوا إِكْلِيلاً مِنْ شَوْكٍ وَوَضَعُوهُ عَلَيْهِ
\par 18 وَابْتَدَأُوا يُسَلِّمُونَ عَلَيْهِ قَائِلِينَ: «السَّلاَمُ يَا مَلِكَ الْيَهُودِ!»
\par 19 وَكَانُوا يَضْرِبُونَهُ عَلَى رَأْسِهِ بِقَصَبَةٍ وَيَبْصُقُونَ عَلَيْهِ ثُمَّ يَسْجُدُونَ لَهُ جَاثِينَ عَلَى رُكَبِهِمْ.
\par 20 وَبَعْدَمَا اسْتَهْزَأُوا بِهِ نَزَعُوا عَنْهُ الأُرْجُوانَ وَأَلْبَسُوهُ ثِيَابَهُ ثُمَّ خَرَجُوا بِهِ لِيَصْلِبُوهُ.
\par 21 فَسَخَّرُوا رَجُلاً مُجْتَازاً كَانَ آتِياً مِنَ الْحَقْلِ وَهُوَ سِمْعَانُ الْقَيْرَوَانِيُّ أَبُو أَلَكْسَنْدَرُسَ وَرُوفُسَ لِيَحْمِلَ صَلِيبَهُ.
\par 22 وَجَاءُوا بِهِ إِلَى مَوْضِعِ «جُلْجُثَةَ» الَّذِي تَفْسِيرُهُ مَوْضِعُ «جُمْجُمَةٍ».
\par 23 وَأَعْطَوْهُ خَمْراً مَمْزُوجَةً بِمُرٍّ لِيَشْرَبَ فَلَمْ يَقْبَلْ.
\par 24 وَلَمَّا صَلَبُوهُ اقْتَسَمُوا ثِيَابَهُ مُقْتَرِعِينَ عَلَيْهَا: مَاذَا يَأْخُذُ كُلُّ وَاحِدٍ؟
\par 25 وَكَانَتِ السَّاعَةُ الثَّالِثَةُ فَصَلَبُوهُ.
\par 26 وَكَانَ عُنْوَانُ عِلَّتِهِ مَكْتُوباً «مَلِكُ الْيَهُودِ».
\par 27 وَصَلَبُوا مَعَهُ لِصَّيْنِ وَاحِداً عَنْ يَمِينِهِ وَآخَرَ عَنْ يَسَارِهِ.
\par 28 فَتَمَّ الْكِتَابُ الْقَائِلُ: «وَأُحْصِيَ مَعَ أَثَمَةٍ».
\par 29 وَكَانَ الْمُجْتَازُونَ يُجَدِّفُونَ عَلَيْهِ وَهُمْ يَهُزُّونَ رُؤُوسَهُمْ قَائِلِينَ: «آهِ يَا نَاقِضَ الْهَيْكَلِ وَبَانِيَهُ فِي ثَلاَثَةِ أَيَّامٍ!
\par 30 خَلِّصْ نَفْسَكَ وَانْزِلْ عَنِ الصَّلِيبِ!»
\par 31 وَكَذَلِكَ رُؤَسَاءُ الْكَهَنَةِ وَهُمْ مُسْتَهْزِئُونَ فِيمَا بَيْنَهُمْ مَعَ الْكَتَبَةِ قَالُوا: «خَلَّصَ آخَرِينَ وَأَمَّا نَفْسُهُ فَمَا يَقْدِرُ أَنْ يُخَلِّصَهَا.
\par 32 لِيَنْزِلِ الآنَ الْمَسِيحُ مَلِكُ إِسْرَائِيلَ عَنِ الصَّلِيبِ لِنَرَى وَنُؤْمِنَ». وَاللَّذَانِ صُلِبَا مَعَهُ كَانَا يُعَيِّرَانِهِ.
\par 33 وَلَمَّا كَانَتِ السَّاعَةُ السَّادِسَةُ كَانَتْ ظُلْمَةٌ عَلَى الأَرْضِ كُلِّهَا إِلَى السَّاعَةِ التَّاسِعَةِ.
\par 34 وَفِي السَّاعَةِ التَّاسِعَةِ صَرَخَ يَسُوعُ بِصَوْتٍ عَظِيمٍ قَائِلاً: «إِلُوِي إِلُوِي لَمَا شَبَقْتَنِي؟» (اَلَّذِي تَفْسِيرُهُ: إِلَهِي إِلَهِي لِمَاذَا تَرَكْتَنِي؟)
\par 35 فَقَالَ قَوْمٌ مِنَ الْحَاضِرِينَ لَمَّا سَمِعُوا: «هُوَذَا يُنَادِي إِيلِيَّا».
\par 36 فَرَكَضَ وَاحِدٌ وَمَلَأَ إِسْفِنْجَةً خَلاًّ وَجَعَلَهَا عَلَى قَصَبَةٍ وَسَقَاهُ قَائِلاً: «اتْرُكُوا. لِنَرَ هَلْ يَأْتِي إِيلِيَّا لِيُنْزِلَهُ!»
\par 37 فَصَرَخَ يَسُوعُ بِصَوْتٍ عَظِيمٍ وَأَسْلَمَ الرُّوحَ.
\par 38 وَانْشَقَّ حِجَابُ الْهَيْكَلِ إِلَى اثْنَيْنِ مِنْ فَوْقُ إِلَى أَسْفَلُ.
\par 39 وَلَمَّا رَأَى قَائِدُ الْمِئَةِ الْوَاقِفُ مُقَابِلَهُ أَنَّهُ صَرَخَ هَكَذَا وَأَسْلَمَ الرُّوحَ قَالَ: «حَقّاً كَانَ هَذَا الإِنْسَانُ ابْنَ اللَّهِ!»
\par 40 وَكَانَتْ أَيْضاً نِسَاءٌ يَنْظُرْنَ مِنْ بَعِيدٍ بَيْنَهُنَّ مَرْيَمُ الْمَجْدَلِيَّةُ وَمَرْيَمُ أُمُّ يَعْقُوبَ الصَّغِيرِ وَيُوسِي وَسَالُومَةُ
\par 41 اللَّوَاتِي أَيْضاً تَبِعْنَهُ وَخَدَمْنَهُ حِينَ كَانَ فِي الْجَلِيلِ. وَأُخَرُ كَثِيرَاتٌ اللَّوَاتِي صَعِدْنَ مَعَهُ إِلَى أُورُشَلِيمَ.
\par 42 وَلَمَّا كَانَ الْمَسَاءُ إِذْ كَانَ الاِسْتِعْدَادُ - أَيْ مَا قَبْلَ السَّبْتِ -
\par 43 جَاءَ يُوسُفُ الَّذِي مِنَ الرَّامَةِ مُشِيرٌ شَرِيفٌ وَكَانَ هُوَ أَيْضاً مُنْتَظِراً مَلَكُوتَ اللَّهِ فَتَجَاسَرَ وَدَخَلَ إِلَى بِيلاَطُسَ وَطَلَبَ جَسَدَ يَسُوعَ.
\par 44 فَتَعَجَّبَ بِيلاَطُسُ أَنَّهُ مَاتَ كَذَا سَرِيعاً. فَدَعَا قَائِدَ الْمِئَةِ وَسَأَلَهُ: «هَلْ لَهُ زَمَانٌ قَدْ مَاتَ؟»
\par 45 وَلَمَّا عَرَفَ مِنْ قَائِدِ الْمِئَةِ وَهَبَ الْجَسَدَ لِيُوسُفَ.
\par 46 فَاشْتَرَى كَتَّاناً فَأَنْزَلَهُ وَكَفَّنَهُ بِالْكَتَّانِ وَوَضَعَهُ فِي قَبْرٍ كَانَ مَنْحُوتاً فِي صَخْرَةٍ وَدَحْرَجَ حَجَراً عَلَى بَابِ الْقَبْرِ.
\par 47 وَكَانَتْ مَرْيَمُ الْمَجْدَلِيَّةُ وَمَرْيَمُ أُمُّ يُوسِي تَنْظُرَانِ أَيْنَ وُضِعَ.

\chapter{16}

\par 1 وَبَعْدَمَا مَضَى السَّبْتُ اشْتَرَتْ مَرْيَمُ الْمَجْدَلِيَّةُ وَمَرْيَمُ أُمُّ يَعْقُوبَ وَسَالُومَةُ حَنُوطاً لِيَأْتِينَ وَيَدْهَنَّهُ.
\par 2 وَبَاكِراً جِدّاً فِي أَوَّلِ الأُسْبُوعِ أَتَيْنَ إِلَى الْقَبْرِ إِذْ طَلَعَتِ الشَّمْسُ.
\par 3 وَكُنَّ يَقُلْنَ فِيمَا بَيْنَهُنَّ: «مَنْ يُدَحْرِجُ لَنَا الْحَجَرَ عَنْ بَابِ الْقَبْرِ؟»
\par 4 فَتَطَلَّعْنَ وَرَأَيْنَ أَنَّ الْحَجَرَ قَدْ دُحْرِجَ! لأَنَّهُ كَانَ عَظِيماً جِدّاً.
\par 5 وَلَمَّا دَخَلْنَ الْقَبْرَ رَأَيْنَ شَابّاً جَالِساً عَنِ الْيَمِينِ لاَبِساً حُلَّةً بَيْضَاءَ فَانْدَهَشْنَ.
\par 6 فَقَالَ لَهُنَّ: «لاَ تَنْدَهِشْنَ! أَنْتُنَّ تَطْلُبْنَ يَسُوعَ النَّاصِرِيَّ الْمَصْلُوبَ. قَدْ قَامَ! لَيْسَ هُوَ هَهُنَا. هُوَذَا الْمَوْضِعُ الَّذِي وَضَعُوهُ فِيهِ.
\par 7 لَكِنِ اذْهَبْنَ وَقُلْنَ لِتَلاَمِيذِهِ وَلِبُطْرُسَ إِنَّهُ يَسْبِقُكُمْ إِلَى الْجَلِيلِ. هُنَاكَ تَرَوْنَهُ كَمَا قَالَ لَكُمْ».
\par 8 فَخَرَجْنَ سَرِيعاً وَهَرَبْنَ مِنَ الْقَبْرِ لأَنَّ الرِّعْدَةَ وَالْحَيْرَةَ أَخَذَتَاهُنَّ. وَلَمْ يَقُلْنَ لأَحَدٍ شَيْئاً لأَنَّهُنَّ كُنَّ خَائِفَاتٍ.
\par 9 وَبَعْدَمَا قَامَ بَاكِراً فِي أَوَّلِ الأُسْبُوعِ ظَهَرَ أَوَّلاً لِمَرْيَمَ الْمَجْدَلِيَّةِ الَّتِي كَانَ قَدْ أَخْرَجَ مِنْهَا سَبْعَةَ شَيَاطِينَ.
\par 10 فَذَهَبَتْ هَذِهِ وَأَخْبَرَتِ الَّذِينَ كَانُوا مَعَهُ وَهُمْ يَنُوحُونَ وَيَبْكُونَ.
\par 11 فَلَمَّا سَمِعَ أُولَئِكَ أَنَّهُ حَيٌّ وَقَدْ نَظَرَتْهُ لَمْ يُصَدِّقُوا.
\par 12 وَبَعْدَ ذَلِكَ ظَهَرَ بِهَيْئَةٍ أُخْرَى لاِثْنَيْنِ مِنْهُمْ وَهُمَا يَمْشِيَانِ مُنْطَلِقَيْنِ إِلَى الْبَرِّيَّةِ.
\par 13 وَذَهَبَ هَذَانِ وَأَخْبَرَا الْبَاقِينَ فَلَمْ يُصَدِّقُوا وَلاَ هَذَيْنِ.
\par 14 أَخِيراً ظَهَرَ لِلأَحَدَ عَشَرَ وَهُمْ مُتَّكِئُونَ وَوَبَّخَ عَدَمَ إِيمَانِهِمْ وَقَسَاوَةَ قُلُوبِهِمْ لأَنَّهُمْ لَمْ يُصَدِّقُوا الَّذِينَ نَظَرُوهُ قَدْ قَامَ.
\par 15 وَقَالَ لَهُمُ: «اذْهَبُوا إِلَى الْعَالَمِ أَجْمَعَ وَاكْرِزُوا بِالإِنْجِيلِ لِلْخَلِيقَةِ كُلِّهَا.
\par 16 مَنْ آمَنَ وَاعْتَمَدَ خَلَصَ وَمَنْ لَمْ يُؤْمِنْ يُدَنْ.
\par 17 وَهَذِهِ الآيَاتُ تَتْبَعُ الْمُؤْمِنِينَ: يُخْرِجُونَ الشَّيَاطِينَ بِاسْمِي وَيَتَكَلَّمُونَ بِأَلْسِنَةٍ جَدِيدَةٍ.
\par 18 يَحْمِلُونَ حَيَّاتٍ وَإِنْ شَرِبُوا شَيْئاً مُمِيتاً لاَ يَضُرُّهُمْ وَيَضَعُونَ أَيْدِيَهُمْ عَلَى الْمَرْضَى فَيَبْرَأُونَ».
\par 19 ثُمَّ إِنَّ الرَّبَّ بَعْدَمَا كَلَّمَهُمُ ارْتَفَعَ إِلَى السَّمَاءِ وَجَلَسَ عَنْ يَمِينِ اللَّهِ.
\par 20 وَأَمَّا هُمْ فَخَرَجُوا وَكَرَزُوا فِي كُلِّ مَكَانٍ وَالرَّبُّ يَعْمَلُ مَعَهُمْ وَيُثَبِّتُ الْكَلاَمَ بِالآيَاتِ التَّابِعَةِ. آمِينَ.


\end{document}