\begin{document}

\title{عهد إبراهيم}

\part{الإصدار 1}

\chapter{1}

\par 1 عاش إبراهيم مقدار حياته تسعمائة وخمسة وتسعين سنة، وبعد أن عاش كل سني حياته في هدوء ووداعة وبر، كان البار مضيافًا للغاية؛

\par 2 لأنه، عندما نصب خيمته على مفترق طرق بلوطة ممرا، استقبل الجميع، الأغنياء والفقراء، الملوك والحكام، المشوهين والعاجزين، الأصدقاء والغرباء، الجيران والمسافرين، الجميع على حد سواء استضاف إبراهيم التقي، القدوس، البار، والمضياف

\par 3 ومع ذلك، فقد جاء عليه القدر الشائع، الذي لا يرحم، والمرير، للموت، ونهاية الحياة غير المؤكدة

\par 4 لذلك، دعا الرب الإله رئيس ملائكته ميخائيل، وقال له: "انزل يا رئيس القواد ميخائيل إلى إبراهيم وكلمه بشأن موته، لكي يُرتب أموره،

\par 5 لأني باركته كنجوم السماء، وكالرمل على شاطئ البحر، وهو يتمتع بعمر طويل وممتلكات كثيرة، ويزداد ثراءً فاحشًا. علاوة على ذلك، فهو بار في كل صلاح يفوق كل الناس، مضياف ومحب حتى نهاية حياته؛

\par 6 بل اذهب يا رئيس الملائكة ميخائيل إلى إبراهيم صديقي الحبيب، وأعلن له موته وطمئنه هكذا:

\par 7 في هذا الوقت، ستغادر هذا العالم الباطل، وستترك الجسد، وتذهب إلى ربك بين الصالحين

\chapter{2}

\par 1 فخرج رئيس الجند من أمام وجه الله، ونزل إلى إبراهيم إلى بلوطة ممرا، فوجد إبراهيم البار في الحقل الذي بالقرب منه، جالسًا بجانب أنيار ثيران للحرث، ومعه بنو ماشق وعبيد آخرون، وكان عددهم اثني عشر

\par 2 وإذا رئيس الجند مقبل إليه، فلما رأى إبراهيم رئيس الجند ميخائيل قادماً من بعيد كرجل وسيم جداً، قام واستقبله كعادته، ملتقياً ومضيفاً كل الغرباء.

\par 3 فسلم عليه رئيس الضباط وقال: السلام عليك أيها الأب المحترم، أيها النفس البارة المختارة من الله، والابن الحقيقي للسماوي.

\par 4 قال إبراهيم للقائد الرئيس: مرحبًا بك أيها المحارب الأكثر تكريمًا، والمشرق كالشمس والأجمل فوق كل أبناء البشر؛ مرحبًا بك؛

\par 5 "لذلك أتوسل إلى حضرتك، أخبرني من أين جاء شبابك في هذا العمر؛ علمني، أنا متوسل إليك، من أين ومن أي جيش ومن أي رحلة جاء جمالك إلى هنا."

\par 6 قال رئيس القبطان: «أنا، يا إبراهيم البار، قادم من المدينة العظيمة. أرسلني الملك العظيم لأقوم مقام صديق عزيز له، فقد استدعاه الملك».

\par 7 فقال إبراهيم: تعال يا سيدي، اذهب معي إلى حقلي. فقال رئيس الجند: أنا آتي.

\par 8 ودخلوا حقل المحراث وجلسوا بجانب الشركة

\par 9 فقال إبراهيم لعبيده بني ماسك: «اذهبوا إلى قطيع الخيل، وأحضروا فرسين هادئين ووديعين وأليفين، لكي أجلس أنا وهذا الغريب عليهما».

\par 10 ولكن قال رئيس القبطان: لا يا سيدي إبراهيم، لا يحضروا خيولاً، لأني أمتنع عن الجلوس على أي حيوان ذي أربع قوائم.

\par 11 أليس ملكي غنيًا بتجارة كثيرة، وله سلطان على الناس وعلى جميع أنواع الماشية؟ لكنني أمتنع عن الجلوس على أي حيوان ذي أربع أقدام

\par 12 هيا بنا إذًا، أيتها النفس الصالحة، نسير بخفة حتى نصل إلى بيتك. فقال إبراهيم: آمين، ليكن


\chapter{3}

\par 1 وفيما هم سائرون من الحقل إلى بيته،

\par 2 بجانب ذلك الطريق كانت هناك شجرة سرو،

\par 3 وبأمر الرب صرخت الشجرة بصوت بشري قائلة: قدوس، قدوس، قدوس هو الرب الإله الذي يدعو نفسه للذين يحبونه.

\par 4 لكن إبراهيم أخفى السر، ظانًا أن رئيس القبطان لم يسمع صوت الشجرة

\par 5 ولما اقتربا من البيت وجلسا في الدار، فلما رأى إسحاق وجه الملاك، قال لسارة أمه: «يا سيدتي أمي، هوذا الرجل الجالس مع أبي إبراهيم ليس ابنًا من نسل الساكنين على الأرض».

\par 6 فركض إسحاق وسلم عليه، وسقط عند قدمي غير المتجسد، فباركه غير المتجسد وقال: "سيمنحك الرب الإله وعده الذي وعد به أبيك إبراهيم ونسله، وسيمنحك أيضًا صلاة أبيك وأمك الثمينة."

\par 7 قال إبراهيم لإسحاق ابنه: «يا ابني إسحاق، استقِ ماءً من البئر وأتِ به إليّ في الإناء، لنغسل قدمي هذا الغريب، لأنه مُتعب، إذ جاء إلينا من سفر طويل».

\par 8 فركض إسحاق إلى البئر واستقى ماءً من الإناء وأحضره لهم،

\par 9 فصعد إبراهيم وغسل قدمي رئيس القواد ميخائيل، فتأثر قلب إبراهيم وبكى على الغريب

\par 10 فلما رأى إسحاق أباه يبكي، بكى أيضًا، ورآهم رئيس الجند يبكون، بكى أيضًا معهم،

\par 11 فسقطت دموع رئيس الجند على السفينة في ماء الحوض فصارت أحجاراً كريمة.

\par 12 فلما رأى إبراهيم الأعجوبة تعجب، فأخذ الحجارة سراً وأخفى السر في قلبه.

\chapter{4}

\par 1 وقال إبراهيم لإسحاق ابنه: «اذهب يا ابني الحبيب إلى مخدع البيت الداخلي وزينه. وافرش لنا هناك سريرين، واحد لي وواحد لهذا الرجل الذي معنا اليوم

\par 2 هيئ لنا هناك مقعدًا ومنارة ومائدة من كل خير. زيّن المخدع يا بني، وافرش تحتنا كتانًا وأرجوانًا وبزًا ناعمًا.

\par 3 أحرقوا هناك كل بخور ثمين وكريم، وأتوا بنباتات عطرة من الحديقة واملأوا بيتنا منها. أشعلوا سبعة مصابيح مملوءة زيتًا لنبتهج، لأن هذا الرجل الذي لنا اليوم أعظم من الملوك والحكام، ومنظره يفوق كل بني البشر.

\par 4 فأعد إسحق كل شيء جيداً، وأخذ إبراهيم رئيس الملائكة ميخائيل ودخل إلى الحجرة، وجلسا كلاهما على الأريكة، ووضع بينهما مائدة عليها وفرة من كل خير.

\par 5 فقام رئيس الجند وخرج، كما لو كان يضطر بطنه ليخرج ماءً، وصعد إلى السماء في طرفة عين، ووقف أمام الرب، وقال له:

\par 6 «يا سيدي وربي، لتعلم قدرتك أنني لا أستطيع تذكير ذلك الرجل الصالح بموته، لأني لم أرَ على الأرض رجلاً مثله، رحيمًا، مضيافًا، بارًا، صادقًا، تقيًا، يمتنع عن كل عمل شرير. والآن اعلم يا سيدي أنني لا أستطيع تذكيره بموته.»

\par 7 فقال الرب: «انزل يا ميخائيل رئيس القواد إلى صديقي إبراهيم، وافعل كل ما يقوله لك، وكل معه مما يأكل

\par 8 وسأرسل روحي القدوس على ابنه إسحاق، وسأضع ذكر موته في قلب إسحاق، حتى يرى في المنام موت أبيه، وسيقص إسحاق الحلم، وستفسره، وسيعرف هو نهايته

\par 9 فقال رئيس القبطان: "يا رب، جميع الأرواح السماوية غير مادية، ولا تأكل ولا تشرب، وقد وضع هذا الرجل أمامي مائدة وفيرة من كل خير أرضي وفاسد. والآن يا رب، ماذا أفعل؟ كيف أهرب منه وأنا جالس معه على مائدة واحدة؟"

\par 10 قال الرب: "انزل إليه، ولا تهتم لهذا، لأنه عندما تجلس معه، سأرسل عليك روحًا آكلًا، فيأكل من يديك ومن خلال فمك كل ما على المائدة. افرحوا معه في كل شيء،

\par 11 أنت فقط تُفَسِّر أمور الرؤيا جيدًا، لكي يعرف إبراهيم منجل الموت ونهاية الحياة غير المؤكدة، ويتخلص من جميع ممتلكاته، لأني باركته فوق رمل البحر ومثل نجوم السماء

\chapter{5}

\par 1 ثم نزل رئيس الجند إلى بيت إبراهيم وجلس معه على المائدة، وكان إسحاق يخدمهم

\par 2 ولما انتهى العشاء، صلى إبراهيم كعادته، وصلى رئيس القبطان معه، واضطجع كل واحد على فراشه لينام

\par 3 فقال إسحاق لأبيه: «يا أبي، أنا أيضًا أُريد أن أنام معك في هذه الحجرة، لأسمع حديثك أيضًا، لأني أُحب أن أسمع فضل حديث هذا الرجل الفاضل».

\par 4 قال إبراهيم: «لا يا بني، ولكن اذهب إلى حجرتك ونم على فراشك لئلا نسبب إزعاجًا لهذا الرجل».

\par 5 ثم أخذ إسحق الصلاة منهم وباركهم وذهب إلى حجرته واضطجع على سريره.

\par 6 "ولكن الرب ألقى في قلب إسحق فكرة الموت كما في الحلم،

\par 7 وفي نحو الساعة الثالثة من الليل استيقظ إسحق وقام عن سريره وجاء راكضا إلى الحجرة التي كان أبوه نائما فيها مع رئيس الملائكة.

\par 8 فلما وصل إسحاق إلى الباب صرخ قائلًا: "يا أبي إبراهيم، قم وافتح لي سريعًا، لأدخل وأتعلق بعنقك وأعانقك قبل أن يأخذوك مني."

\par 9 فقام إبراهيم وفتح له، فدخل إسحاق وتعلق بعنقه، وبدأ يبكي بصوت عظيم

\par 10 فانزعج إبراهيم في قلبه، وبكى هو أيضًا بصوت عظيم، ولما رآهم رئيس الجند يبكون، بكى هو أيضًا

\par 11 وكانت سارة في حجرتها، فسمعت بكاءهم، فجاءت راكضة إليهم، فوجدتهم متعانقين ويبكون

\par 12 فقالت سارة وهي تبكي: «يا سيدي إبراهيم، ما هذا الذي تبكيه؟»

\par 13 أخبرني يا سيدي، هل جاءك هذا الأخ الذي استضفناه اليوم بخبر لوط ابن أخيك أنه قد مات؟ ألهذا تحزن هكذا؟

\par 14 أجاب رئيس الضباط وقال لها: "لا يا أختي سارة، ليس الأمر كما تقولين، ولكن ابنك إسحاق، على ما أظن، رأى حلمًا، وجاء إلينا يبكي، فلما رأيناه تحركت قلوبنا وبكينا."

\chapter{6}

\par 1 فلما سمعت سارة كلام رئيس القبطان الرائع، عرفت في الحال أن الذي تكلم هو ملاك الرب

\par 2 فأشارت سارة لإبراهيم أن يخرج نحو الباب، وقالت له: "سيدي إبراهيم، هل تعرف من هذا الرجل؟"

\par 3 قال إبراهيم: "لا أعلم."

\par 4 قالت سارة: "أنت تعلم يا سيدي الرجال الثلاثة من السماء الذين استضفناهم في خيمتنا عند بلوطة ممرا، حين ذبحت جدي المعز بلا عيب، وأعددت لهم مائدة

\par 5 بعد أن أُكل اللحم، قام الجدي ورضع أمه بفرح عظيم. أما تعلم يا سيدي إبراهيم أنهم بوعدهم أعطونا إسحاق ثمرة البطن؟ من هؤلاء الرجال القديسين الثلاثة هذا واحد

\par 6 قال إبراهيم: "يا سارة، في هذا تقولين الحق. المجد والتسبيح من إلهنا أبينا. ففي وقت متأخر من المساء عندما غسلت قدميه في المغسلة قلت في قلبي: هاتان قدما أحد الرجال الثلاثة الذين غسلتهم آنذاك؛

\par 7 فدموعه التي سقطت في الحوض صارت أحجارًا كريمة. فنفضها من حجره وأعطاها لسارة قائلًا: «إن كنتِ لا تصدقينني، فانظري الآن إلى هذه».

\par 8 فاستقبلتهم سارة وانحنت وسلمت وقالت: «المجد لله الذي يُرينا العجائب. والآن اعلم يا سيدي إبراهيم أن بيننا وحيًا، شرًا كان أم خيرًا!»


\chapter{7}

\par 1 فترك إبراهيم سارة ودخل الحجرة، وقال لإسحاق: «تعال إلى هنا يا ابني الحبيب، أخبرني بالحق، ماذا رأيت وماذا أصابك حتى أتيت إلينا مسرعًا».

\par 2 فأجاب إسحق قائلاً: رأيت يا سيدي في هذه الليلة الشمس والقمر فوق رأسي، يحيطان بي بأشعتهما ويعطيانني نوراً.

\par 3 بينما كنت أنظر إلى هذا وأبتهج، رأيت السماء مفتوحة، ورجلًا يحمل نورًا ينزل منها، يضيء أكثر من سبع شموس

\par 4 وجاء هذا الرجل كالشمس، فأخذ الشمس من رأسي، وصعد إلى السماء التي أتى منها، لكنني حزنت كثيرًا لأنه أخذ الشمس مني

\par 5 بعد قليل، وبينما كنت لا أزال حزينًا ومضطربًا للغاية، رأيت هذا الرجل يخرج من السماء للمرة الثانية، وأخذ مني القمر أيضًا من رأسي،

\par 6 فبكيت بكاءً شديدًا ودعوت ذلك الرجل النوراني، وقلت: يا سيدي، لا تنزع مني مجدي، بل ارحمني واستجب لي، وإذا نزعت الشمس مني، فاترك لي القمر

\par 7 قال: «دعوهم يُؤخذون إلى الملك في الأعلى، فهو يُريدهم هناك». وأخذهم مني، لكنه ترك الأشعة عليّ

\par 8 قال رئيس القبطان: "اسمع يا إبراهيم البار، الشمس التي رآها ابنك هي أنت أبوه، والقمر أيضًا هو سارة أمه. الرجل الحامل للنور الذي نزل من السماء، هذا هو المرسل من الله الذي سيأخذ روحك الصالحة منك

\par 9 واعلم الآن يا إبراهيم الكريم أنك في هذا الوقت ستترك هذه الحياة الدنيا وتنتقل إلى الله

\par 10 قال إبراهيم لرئيس القبطان: "يا لها من عجائب! والآن هل أنت الذي سيأخذ روحي مني؟"

\par 11 قال له رئيس القبطان: "أنا رئيس القبطان ميخائيل، الواقف أمام الرب، وقد أُرسلت إليك لأذكرك بموتك، وبعد ذلك سأذهب إليه كما أُمرت."

\par 12 قال إبراهيم: «الآن علمتُ أنك ملاك الرب، وأُرسلت لأخذ روحي، لكنني لن أذهب معك، بل افعل كل ما تُؤمر به».

\chapter{8}

\par 1 فلما سمع رئيس القبطان هذه الكلمات، اختفى على الفور، وصعد إلى السماء ووقف أمام الله، وأخبر بكل ما رآه في بيت إبراهيم

\par 2 فقال رئيس القبطان أيضًا لسيده: "هكذا يقول صديقك إبراهيم: لن أذهب معك، بل أفعل كل ما تؤمر به؛

\par 3 والآن يا رب القدير هل مجدك وملكوتك الخالد يأمران بشيء؟

\par 4 قال الله لرئيس القبطان ميخائيل: اذهب إلى صديقي إبراهيم مرة أخرى وقل له هكذا،

\par 5 هكذا قال الرب إلهكم الذي أدخلكم إلى أرض الموعد وبارككم فوق رمل البحر وفوق نجوم السماء.

\par 6 الذي فتح رحم عقم سارة، وأعطاك إسحق ثمرة البطن في شيخوخة،

\par 7 الحق أقول لكم إني أبارككم بركة وأكثر نسلكم تكثيرا وأعطيكم كل ما تطلبون مني لأني أنا الرب إلهكم وليس آخر غيري.

\par 8 أخبرني لماذا تمردت علي، ولماذا يوجد حزن فيك، ولماذا تمردت على رئيس ملائكتي ميخائيل؟

\par 9 ألا تعلمون أن جميع الذين جاءوا من آدم وحواء قد ماتوا، وأن أحدًا من الأنبياء لم ينجُ من الموت؟ لا أحد ممن يحكمون كملوك خالد؛ لم ينجُ أحد من أجدادكم من سر الموت. لقد ماتوا جميعًا، ورحلوا جميعًا إلى الجحيم، وجمعهم منجل الموت جميعًا

\par 10 لكنني لم أرسل عليك الموت، ولم أسمح لأي مرض مميت أن يصيبك، ولم أسمح لمنجل الموت أن يقابلك، ولم أسمح لشِباك الجحيم أن تحيط بك، ولم أرغب قط في أن تقابل أي شر

\par 11 ولكن من أجل التعزية، أرسلتُ إليكَ رئيسَ قوادتي ميخائيل، لكي تعلمَ رحيلَكَ من العالم، وتُرتِّبَ بيتَكَ وكلَّ ما يخصُّكَ، وتُبارِكْ إسحاقَ ابنَكَ الحبيب. والآن اعلمْ أنني فعلتُ هذا وأنا لا أُريدُ أن أُحزنَكَ

\par 12 فلماذا قلتَ لرئيس قوادتي: "لن أذهب معك؟" لماذا تكلمتَ هكذا؟ ألا تعلم أنني إذا أذنتُ للموت وجاء إليك، فسأرى إن كنت ستأتي أم لا؟

\chapter{9}

\par 1 فأخذ رئيس الجند تحذيرات الرب ونزل إلى إبراهيم، فلما رآه البار سقط على وجهه إلى الأرض كميت،

\par 2 وأخبره رئيس القبطان بكل ما سمعه من العلي. ثم قام إبراهيم القديس البار بدموع كثيرة، وسقط عند قدمي غير المتجسد، وتضرع إليه قائلًا:

\par 3 «أتوسل إليك، يا رئيس قادة الجيوش أعلاه، بما أنك قد تفضلت تمامًا بالمجيء إليّ وأنت خاطئ وخادمك غير المستحق في كل شيء، أتوسل إليك الآن، يا رئيس القادة، أن تحمل كلمتي مرة أخرى إلى العلي، وتقول له:

\par 4 هكذا قال إبراهيم عبدك، يا رب، يا رب، في كل عمل وكلام طلبته منك سمعتني، وأتممت كل مشورتي

\par 5 الآن يا رب، أنا لا أقاوم قدرتك، لأني أنا أيضًا أعلم أنني لست خالدًا بل فانٍ. وبما أن كل شيء يخضع لأوامرك، ويخاف ويرتعد من وجه قدرتك، فأنا أيضًا أخاف، لكنني أطلب منك طلبًا واحدًا،

\par 6 والآن، يا رب وسيدي، اسمع صلاتي، لأني وأنا لا أزال في هذا الجسد أرغب في رؤية كل الأرض المأهولة، وكل الخلائق التي أسستها بكلمة واحدة، وعندما أرى هذه، فعندئذ إذا رحلت عن الحياة سأكون بلا حزن

\par 7 فعاد رئيس القبطان مرة أخرى، ووقف أمام الله، وأخبره بكل شيء، قائلاً: "هكذا يقول صديقك إبراهيم، لقد رغبت في أن أرى كل الأرض في حياتي قبل أن أموت."

\par 8 "ولما سمع العلي هذا أمر أيضاً رئيس القواد ميخائيل وقال له: خذ سحابة من نور والملائكة الذين لهم سلطان على المركبات وانزلوا وخذوا إبراهيم البار على مركبة الكروبيم وارفعوه إلى هواء السماء حتى ينظر إلى كل الأرض."

\chapter{10}

\par 1 ونزل رئيس الملائكة ميخائيل وأخذ إبراهيم على مركبة الكروبيم، ورفعه إلى هواء السماء، وساقه على السحابة مع ستين ملائكة، وصعد إبراهيم على المركبة فوق كل الأرض

\par 2 ورأى إبراهيم العالم كما كان في ذلك اليوم، بعضهم يحرثون، وآخرون يقودون عربات، وفي مكان رجال يرعون قطعانهم، وفي مكان آخر يراقبونها ليلًا، ويرقصون ويلعبون ويعزفون على القيثارة، وفي مكان آخر رجال يتنازعون ويتخاصمون في المحاكم، وفي مكان آخر رجال يبكون ويذكرون الموتى

\par 3 رأى أيضًا العروسين يُستقبلان بتكريم، وباختصار، رأى كل ما يُفعل في العالم، سواء كان جيدًا أو سيئًا

\par 4 فمر إبراهيم فوقهم فرأى رجالاً يحملون سيوفًا، يحملون في أيديهم سيوفًا مسنونة، فسأل إبراهيم رئيس القواد: "من هؤلاء؟"

\par 5 قال القائد: "هؤلاء لصوص، ينوون ارتكاب جريمة قتل، وسرقة، وحرق، وتدمير."

\par 6 قال إبراهيم: «يا رب، يا رب، اسمع صوتي، وأمر أن تخرج الوحوش من الغابة وتأكلهم».

\par 7 وبينما هو يتكلم، خرجت وحوش برية من الغابة وأكلتهم

\par 8 ورأى في موضع آخر رجلاً مع امرأة يزنيان مع بعضهما البعض،

\par 9 وقال: «يا رب، يا رب، قل أن تنشق الأرض وتبتلعهم». ففي الحال انشقّت الأرض وابتلعتهم

\par 10 ورأى في مكان آخر رجالاً يحفرون منزلاً، ويحملون ممتلكات رجال آخرين،

\par 11 فقال: «يا رب، يا رب، قل أن تنزل نار من السماء وتأكلهم». وبينما هو يتكلم، نزلت نار من السماء وأكلتهم

\par 12 وفي الحال جاء صوت من السماء إلى رئيس القواد قائلاً: "يا رئيس القواد ميخائيل، مُر المركبة أن تتوقف، وأرجع إبراهيم بعيدًا حتى لا يرى كل الأرض،

\par 13 لأنه إن رأى كل من يعيش في الشر، فسوف يُهلك الخليقة كلها. لأنه هوذا إبراهيم لم يُخطئ، ولا يرحم الخطاة،

\par 14 ولكني خلقت العالم، ولا أريد أن أهلك أحدًا منه، بل أنتظر موت الخاطئ حتى يرجع ويحيا

\par 15 لكن اصعدوا إبراهيم إلى باب السماء الأول، لكي يرى هناك الدينونة والجزاء، ويتوب عن نفوس الخطاة الذين أهلكهم

\chapter{11}

\par 1 فأدار ميخائيل المركبة وأتى بإبراهيم شرقًا، إلى باب السماء الأول؛

\par 2 ورأى إبراهيم طريقين أحدهما ضيق وضيق والآخر واسع ورحب.

\par 3 فرأى هناك بابين أحدهما واسع على الطريق الواسع والآخر ضيق على الطريق الضيق.

\par 4 ورأى خارج البابين هناك رجلاً جالسًا على عرش مذهَّب، وكان منظر ذلك الرجل هائلًا كما لو كان من الرب

\par 5 ورأوا أرواحًا كثيرة يقودها الملائكة ويقتادونها عبر البوابة الواسعة، وأرواحًا أخرى، قليلة العدد، أخذها الملائكة عبر البوابة الضيقة

\par 6 ولما رأى العجيب الجالس على العرش الذهبي قليلين يدخلون من الباب الضيق وكثيرين يدخلون من الباب الواسع، نزع ذلك العجيب شعر رأسه وجانبي لحيته، وألقى بنفسه على الأرض من على عرشه، يبكي وينوح

\par 7 ولكن عندما رأى نفوسًا كثيرة تدخل من الباب الضيق، نهض عن الأرض وجلس على عرشه في فرح عظيم، فرحًا وابتهاجًا

\par 8 وسأل إبراهيم رئيس القواد: "سيدي رئيس القواد، من هذا الرجل العجيب، المزين بمثل هذا المجد، والذي يبكي وينوح أحيانًا، ويفرح أحيانًا أخرى؟"

\par 9 قال غير المتجسد: "هذا هو آدم المخلوق الأول الذي هو في مثل هذا المجد، وهو ينظر إلى العالم لأن الجميع ولدوا منه،

\par 10 وعندما يرى أرواحًا كثيرة تدخل من الباب الضيق، فإنه ينهض ويجلس على عرشه فرحًا ومبتهجًا، لأن هذا الباب الضيق هو باب الأبرار، الذي يؤدي إلى الحياة، والذين يدخلون منه يدخلون الفردوس. لهذا السبب، يفرح آدم المخلوق الأول، لأنه يرى النفوس تُخلَّص

\par 11 ولكن عندما يرى أرواحًا كثيرة تدخل من الباب الواسع، فإنه ينتزع شعر رأسه، ويلقي بنفسه على الأرض يبكي وينوح بمرارة، لأن الباب الواسع هو باب الخطاة، الذي يؤدي إلى الهلاك والعقاب الأبدي. ولهذا سقط آدم الأول من عرشه يبكي وينوح على هلاك الخطاة، لأن كثيرين هم الهالكون، وقليلون هم المخلصون،

\par 12 لأنه في سبعة آلاف بالكاد توجد نفس واحدة تخلص وهي بارة وبلا دنس

\chapter{12}

\par 1 بينما كان لا يزال يقول لي هذه الأشياء، إذا بملاكين، ناريين في مظهرهما، وقاسيين في عقلهما، وقاسيين في مظهرهما، وكانا يقودان آلاف الأرواح، ويجلدانها بلا رحمة بأسياج نارية

\par 2 أمسك الملاك بنفس واحدة، وطردوا جميع النفوس من الباب الواسع إلى الهلاك

\par 3 فذهبنا أيضًا مع الملائكة، ودخلنا إلى داخل ذلك الباب الواسع،

\par 4 وبين البوابتين وقف عرش رهيب المنظر، من بلورة رهيبة، يلمع كالنار،

\par 5 وجلس عليه رجل عجيب منير كالشمس، يشبه ابن الله.

\par 6 وكان أمامه مائدة مثل البلور، كلها من ذهب وكتان ناعم،

\par 7 وعلى المائدة كان كتاب موضوعا سمكه ستة أذرع وعرضه عشرة أذرع.

\par 8 وعلى يمينها ويسارها وقف ملاكان يحملان ورقة وحبرًا وقلمًا.

\par 9 أمام الطاولة جلس ملاك من النور، يحمل في يده ميزانًا،

\par 10 وعلى يساره جلس ملاكٌ ناريٌّ، لا يرحم، صارم، يحمل في يده بوقًا، وفي داخله نارٌ آكلةٌ ليختبر بها الخطاة

\par 11 الرجل العجيب الذي جلس على العرش بنفسه حكم على النفوس وأدانها،

\par 12 وكتب الملاكان عن اليمين واليسار، الذي عن اليمين البر والذي عن اليسار الشر

\par 13 الذي أمام المائدة، الذي يحمل الميزان، يزن الأرواح،

\par 14 والملاك الناري الذي كان يحمل النار، اختبر النفوس.

\par 15 فسأل إبراهيم رئيس القواد ميخائيل: "ما هذا الذي نراه؟" فقال رئيس القواد: "هذه الأشياء التي تراها، يا قديس إبراهيم، هي الدينونة والجزاء

\par 16 وها هو الملاك ممسكًا بالنفس في يده، وقد أحضرها أمام القاضي،

\par 17 فقال القاضي لأحد الملائكة الذين كانوا يخدمونه: افتح لي هذا الكتاب، واعثر لي على خطايا هذه النفس

\par 18 ففتح الكتاب فوجد خطاياه وبره متساويين، فلم يسلمه للمعذبين ولا للذين نالوا الخلاص، بل وضعه في الوسط


\chapter{13}

\par 1 فقال إبراهيم: «يا سيدي رئيس القبطان، من هذا القاضي العجيب؟ ومن هم الملائكة الذين يكتبون؟ ومن هو الملاك الذي مثل الشمس، يمسك الميزان؟ ومن هو الملاك الناري الذي يمسك النار؟»

\par 2 قال رئيس القبطان: "أترى، يا قدوس إبراهيم، الرجل الرهيب الجالس على العرش؟ هذا هو ابن آدم الأول، المسمى هابيل، الذي قتله قابيل الشرير،

\par 3 ويجلس هكذا ليدين الخليقة كلها، ويفحص الأبرار والخطاة. لأن الله قال: لا أدينكم، بل كل إنسان مولود من الإنسان سيدان

\par 4 لذلك أعطاه الله الدينونة، ليدين العالم حتى مجيئه العظيم والمجيد، وحينئذٍ، يا إبراهيم البار، يأتي الدينونة والجزاء الكاملان، الأبدي وغير المتغير، الذي لا يستطيع أحد تغييره

\par 5 لأن كل إنسان قد جاء من المخلوق الأول، ولذلك يُحاكم هنا أولاً من قبل ابنه،

\par 6 وفي المجيء الثاني، سيُدانون من قِبَل أسباط إسرائيل الاثني عشر، كل نفس وكل خليقة

\par 7 ولكن في المرة الثالثة سيُحاكمون من قِبَل الرب إله الجميع، وحينئذٍ تكون نهاية ذلك الدينونة قريبة، والحكم رهيب، وليس هناك من يُنقذ

\par 8 والآن، من خلال ثلاث محاكم، يُصدر حكم العالم ويُجازى، ولهذا السبب لا يُثبت الأمر نهائيًا بشاهد واحد أو شاهدين، بل بثلاثة شهود يُثبت كل شيء

\par 9 الملاكان عن اليمين واليسار، هما اللذان يكتبان الخطايا والبر، الملاك الذي عن اليمين يكتب البر، والملاك الذي عن اليسار يكتب الخطايا

\par 10 الملاك مثل الشمس، يحمل الميزان في يده، هو رئيس الملائكة، دوكيال الميزان العادل، وهو يزن البر والخطايا ببر الله.

\par 11 الملاك الناري عديم الرحمة، الذي يحمل النار في يده، هو رئيس الملائكة بورويل، الذي لديه سلطة على النار، ويختبر أعمال البشر من خلال النار،

\par 12 وإذا التهمت النار عمل أي إنسان، فإن ملاك الدينونة يقبض عليه على الفور، ويأخذه إلى مكان الخطاة، وهو مكان عقاب مرير للغاية

\par 13 ولكن إن أقرت النار عمل أحد ولم تقبض عليه، فذلك الإنسان يتبرر، ويأخذه ملاك البر ويرفعه ليخلص في نصيب الأبرار

\par 14 وهكذا، يا إبراهيم البار، تُمتحن كل الأشياء في كل البشر بالنار والميزان

\chapter{14}

\par 1 فقال إبراهيم لرئيس القواد: «يا سيدي رئيس القواد، النفس التي كان الملاك يمسكها في يده، لماذا حُكم عليها أن تُوضع في الوسط؟»

\par 2 قال رئيس القبطان: "اسمع يا إبراهيم البار. لأن القاضي وجد خطاياها وبرّها متساويين، لم يُسلمها للدينونة ولا للخلاص، حتى يأتي ديان الجميع."

\par 3 قال إبراهيم لرئيس القبطان: "وماذا ينقص النفس بعد لتخلص؟"

\par 4 قال رئيس القبطان: "إذا نال برًا واحدًا على خطاياه، فإنه يدخل الخلاص."

\par 5 قال إبراهيم لرئيس القواد: "تعال إلى هنا يا رئيس القواد ميخائيل، لنصلي من أجل هذه النفس، ونرى هل يسمعنا الله؟" قال رئيس القواد: "آمين، ليكن كذلك".

\par 6 فقاموا بالصلاة والتضرع من أجل النفس، فاستجاب الله لهم، وعندما قاموا من صلاتهم لم يروا النفس واقفة هناك

\par 7 فقال إبراهيم للملاك: «أين النفس التي كنت تحملها في الوسط؟»

\par 8 فأجاب الملاك: "لقد تم إنقاذه بفضل صلاتك الصالحة، وها هو ملاك نور قد أخذه وصعد به إلى الفردوس."

\par 9 قال إبراهيم: «أُمَجِّدُ اسْمَ اللَّهِ الْعَلِيَّ وَرَحْمَتَهُ الَّتِي لَا تُحْصَى».

\par 10 فقال إبراهيم لرئيس القبطان: «أتوسل إليك يا رئيس الملائكة، أنصت إلى صلاتي، ولندع الرب بعد،

\par 11 وأتوسل إليه أن يرحمني، وألتمس رحمته من أجل نفوس الخطاة الذين لعنتهم وأهلكتهم في غضبي، والذين التهمتهم الأرض، ومزقتهم الوحوش، وأحرقتهم النار بكلماتي.

\par 12 الآن أعلم أنني أخطأت أمام الرب إلهنا. هلمَّ يا ميخائيل، رئيس جيوش السماء، تعالَ نبتهل إلى الله بدموع ليغفر لي خطيئتي ويمنحني إياها.

\par 13 فسمعه رئيس الجند، فتضرعوا إلى الرب، وبعد أن طلبوا منه طويلاً، جاء صوت من السماء قائلاً:

\par 14 "إبراهيم، إبراهيم، قد سمعت صوتك وصلاتك، وغفرت لك خطيتك، والذين تظن أني أهلكتهم فقد دعوتهم وأحييتهم بإحساني الفائق، لأني جازيتهم إلى حين في الدينونة، والذين أهلكهم أحياء على الأرض فلا أجازيهم بالموت."

\chapter{15}

\par 1 وقال صوت الرب أيضًا لميخائيل رئيس القواد: يا ميخائيل عبدي، أرجع إبراهيم إلى بيته، لأنه هوذا نهايته قد اقتربت، ومقدار حياته قد كمل، حتى يُرتب كل شيء، ثم خذه وأحضره إليّ

\par 2 فأدار رئيس القبطان المركبة والسحابة، وأتى بإبراهيم إلى بيته،

\par 3 ودخل حجرته وجلس على أريكته.

\par 4 فجاءت سارة امرأته وقبلت قدمي غير المتجسد وتكلمت بتواضع قائلة: أشكرك يا سيدي لأنك أحضرت سيدي إبراهيم لأننا ظننا أنه ارتفع عنا.

\par 5 وجاء ابنه إسحاق أيضًا ووقع على عنقه، وكذلك أحاط به جميع عبيده وإماءه واحتضنوه ممجدين الله

\par 6 فقال لهم غير المتجسد: «اسمع أيها البار إبراهيم. هوذا امرأتك سارة، وهوذا أيضًا ابنك الحبيب إسحاق، وهوذا أيضًا جميع عبيدك وإماءك حولك».

\par 7 تخلص من كل ما لديك، فقد اقترب اليوم الذي ستغادر فيه الجسد وتذهب إلى الرب مرة واحدة وإلى الأبد

\par 8 قال إبراهيم: "هل قال الرب ذلك، أم أنت تقول هذا عن نفسك؟"

\par 9 أجاب رئيس القبطان: "اسمع أيها الصديق إبراهيم. لقد أمر الرب وأنا أخبرك به."

\par 10 قال إبراهيم: "لن أذهب معك."

\par 11 فلما سمع رئيس الجند هذه الكلمات خرج للوقت من عند إبراهيم وصعد إلى السماء ووقف أمام الله العلي،

\par 12 وقال: «أيها الرب القدير، ها قد سمعت لصديقك إبراهيم في كل ما قاله لك، وأتممت طلباته. لقد أريته قدرتك، وكل الأرض والبحر الذي تحت السماء. لقد أريته الدينونة والجزاء بواسطة السحاب والمركبات، وقال مرة أخرى: لن أذهب معك».

\par 13 فقال العلي للملاك: «هل يقول خليلي إبراهيم هذا أيضًا: لن أذهب معك؟»

\par 14 قال رئيس الملائكة: «يا رب القدير، إنه يقول هكذا، وأنا أمتنع عن وضع يدي عليه، لأنه صديقك منذ البداية، وقد فعل كل شيء مرضيًا في نظرك».

\par 15 لا يوجد رجل مثله على الأرض، ولا حتى أيوب الرجل العجيب، ولذلك أمتنع عن وضع يدي عليه. لذا، أيها الملك الخالد، فأمر بما يجب فعله

\chapter{16}

\par 1 ثم قال العلي: "ادعني إلى هنا أيها الموت المدعو الوجه القاسي والنظرة القاسية."

\par 2 فذهب ميخائيل غير المتجسد وقال للموت: تعال إلى هنا، الرب الخالق الملك الخالد يدعوك.

\par 3 ولما سمع الموت ذلك، ارتجف وارتجف، وقد تملكه رعب عظيم، وجاء بخوف عظيم، ووقف أمام الأب غير المرئي، يرتجف ويتأوه ويرتجف، منتظرًا أمر الرب

\par 4 لذلك قال الإله غير المنظور للموت: "تعال إلى هنا، أيها الاسم المرير والشرس للعالم، أخفِ شراستك، وستر فسادك، واطرح عنك مرارتك، والبس جمالك وكل مجدك،

\par 5 وانزل إلى إبراهيم صديقي، وخذْه وأحضره إليّ. والآن أقول لك أيضًا لا تُخيفه، بل أحضره بكلامٍ حسن، لأنه صديقي

\par 6 ولما سمع الموت هذا، خرج من حضرة العلي، ولبس ثوبًا عظيم السطوع، وجعل منظره كالشمس، وصار جميلًا وجميلًا فوق بني البشر، متخذًا شكل رئيس ملائكة، وخداه ملتهبتان بالنار، ومضى إلى إبراهيم

\par 7 فخرج إبراهيم البار من مخدعه، وجلس تحت أشجار ممرا، ممسكًا ذقنه بيده، منتظرًا مجيء رئيس الملائكة ميخائيل

\par 8 وإذا برائحة طيبة قد هبَّت إليه، ولمع نور، فالتفت إبراهيم فرأى الموت مقبلًا نحوه في مجد وجمال عظيمين. فقام إبراهيم وذهب للقائه، ظانًّا أنه رئيس جند الله،

\par 9 فنظر إليه الموت، فسلم عليه قائلًا: افرح، يا إبراهيم الثمين، أيها النفس الصالحة، الصديق الحقيقي لله العلي، ورفيق الملائكة القديسين

\par 10 قال إبراهيم للموت: "مرحباً بك يا من تشبه الشمس في مظهرك وشكلك، أيها المعين المجيد، وحامل النور، أيها الإنسان العجيب، من أين يأتي مجدك إلينا، ومن أنت، ومن أين أتيت؟"

\par 11 ثم قال الموت: "يا إبراهيم البار، ها أنا أقول لك الحق. أنا نصيب الموت المرير."

\par 12 قال له إبراهيم: "بل أنت جمال العالم، أنت مجد وجمال الملائكة والبشر، أنت أجمل شكلاً من أي شخص آخر، وتقول أنا نصيب الموت المرير، وليس بالأحرى أنا أجمل من كل شيء صالح."

\par 13 قال الموت: «الحق أقول لكم. ما سماني به الرب، فذلك أقول لكم أيضًا».

\par 14 قال إبراهيم: "لماذا أتيتم إلى هنا؟"

\par 15 قال الموت: "لقد أتيت من أجل روحك المقدسة".

\par 16 فقال إبراهيم: «أعلم ما تقصد، لكنني لن أذهب معك. وسكت الموت ولم يجبه بكلمة».

\chapter{17}

\par 1 ثم قام إبراهيم وذهب إلى بيته، وكان الموت أيضًا يرافقه إلى هناك. وصعد إبراهيم إلى حجيرة بيته، وصعد الموت معه. واضطجع إبراهيم على فراشه، فجاء الموت وجلس عند قدميه

\par 2 ثم قال إبراهيم: «اذهبوا، اذهبوا عني، فإني أريد أن أرتاح على فراشي».

\par 3 قال الموت: لن أرحل حتى آخذ روحك منك.

\par 4 قال له إبراهيم: "والله الذي لا يموت، أُوصيك أن تُخبرني بالحق. هل أنت الموت؟"

\par 5 قال له الموت: "أنا الموت. أنا مُدمِّر العالم."

\par 6 قال إبراهيم: "أتوسل إليك، بما أنك الموت، أن تخبرني هل أتيت إلى الجميع بهذا القدر من العدل والمجد والجمال؟"

\par 7 قال الموت: "لا يا سيدي إبراهيم، فقد أصبحت بركاتك، وبحر ضيافتك اللامتناهي، وعظمة محبتك لله، تاجًا على رأسي، وفي جمال وسلام عظيم ولطف أقترب من الصالحين،

\par 8 ولكني آتي إلى الخطاة بفساد عظيم وعنف ومرارة شديدة وبنظرة شرسة لا ترحم

\par 9 قال إبراهيم: "أتوسل إليك، اسمع لي، وأرني شراستك وكل فسادك ومرارتك."

\par 10 فقال الموت: "لا يمكنك أن تنظر إلى ضراوتي، يا إبراهيم البار."

\par 11 قال إبراهيم: "نعم، سأستطيع أن أرى كل عنفكم باسم الله الحي، لأن قوة إلهي الذي في السماء معي."

\par 12 ثم أزال الموت كل بهائه وجماله، وكل مجده، وصورته التي تشبه الشمس التي كان يلبسها،

\par 13 ولبس رداء الطاغية، وجعل منظره كئيبًا وأشرس من كل أنواع الوحوش البرية، وأكثر نجاسة من كل نجاسة

\par 14 وأرى لإبراهيم سبعة رؤوس أفاعي نارية وأربعة عشر وجهًا، وجهًا ملتهبًا شديد الشراسة، ووجهًا ظلامًا، ووجهًا كئيبًا جدًا لأفعى، ووجهًا كجرفٍ رهيب، ووجهًا أشرس من الأفعى، ووجهًا كأسدٍ رهيب، ووجهًا كأفعى الزينة والريحان

\par 15 أراه أيضًا وجه سيف ناري، ووجهًا يحمل سيفًا، ووجهًا من برق، برق رهيب، وصوت رعد مروع

\par 16 أراه أيضًا وجهًا آخر لبحر عاصف هائج، ونهر متدفق هائج، وثعبان رهيب ذو ثلاثة رؤوس، وكأس ممزوجة بالسموم،

\par 17 وباختصار، أظهر له شراسة عظيمة ومرارة لا تُطاق، وكل مرض مميت كرائحة الموت

\par 18 ومن شدة المرارة والشدة مات عبيد وإماء وكان عددهم نحو سبعة آلاف،

\par 19 ودخل إبراهيم البار في غفلة من الموت حتى خذلته روحه

\chapter{18}

\par 1 فلما رأى إبراهيم القدوس هذه الأمور، قال للموت: "أتوسل إليك، أيها الموت المدمر، أن تخفي شراستك، والبس جمالك والشكل الذي كان عليك من قبل."

\par 2 وفي الحال أخفى الموت شراسته، ولبس جماله الذي كان عليه من قبل

\par 3 فقال إبراهيم للموت: "لماذا فعلت هذا، حتى قتلت جميع عبيدي وإماءي؟ هل أرسلك الله إلى هنا لهذا اليوم؟"

\par 4 فقال الموت: كلا يا سيدي إبراهيم، ليس الأمر كما تقول، ولكن لأجلك أرسلت إلى هنا.

\par 5 قال إبراهيم للموت: "فكيف مات هؤلاء؟ ألم يتكلم الرب؟"

\par 6 قال الموت: «صدق يا إبراهيم البار، إن هذا أيضًا عجيب، أنك لم تُؤخذ معهم. ولكني أقول لك الحق،

\par 7 لأنه لو لم تكن يمين الله معك في ذلك الوقت، لكان ينبغي لك أيضًا أن تفارق هذه الحياة

\par 8 قال إبراهيم البار: «الآن أعلم أنني قد وصلت إلى حالة من عدم المبالاة بالموت، حتى أن روحي قد ضاعت،

\par 9 لكني أتوسل إليك، أيها الموت المدمر، بما أن خدامي قد ماتوا قبل أوانهم، تعالَ نصلي إلى الرب إلهنا لكي يسمعنا ويقيم أولئك الذين ماتوا بعنفك قبل أوانهم

\par 10 فقال الموت: "آمين، فليكن". فقام إبراهيم وسقط على وجه الأرض يصلي، والموت معه،

\par 11 وأرسل الرب روح الحياة على الأموات فأحياهم. حينئذٍ أعطى إبراهيم البار مجدًا لله

\chapter{19}

\par 1 وصعد إلى حجرته واضطجع، فجاء الموت ووقف أمامه

\par 2 فقال له إبراهيم: «اذهب عني، لأني أريد أن أرتاح، لأن روحي في حالة سكون».

\par 3 قال الموت: "لن أفارقك حتى أقبض روحك."

\par 4 فقال إبراهيم للموت بوجه صارم ونظرة غاضبة: من أمرك أن تقول هذا؟

\par 5 تقول هذا الكلام عن نفسك متباهيًا، ولن أذهب معك حتى يأتي إليّ رئيس القواد ميخائيل، فأذهب معه. ولكن أقول لك أيضًا: إن كنت ترغب في أن أرافقك، فأخبرني بكل تغيراتك: رؤوس الأفاعي السبعة النارية، وما وجه الهاوية، وما السيف الحاد، وما النهر الهادر، وما البحر الهائج الهائج.

\par 6 علّمني أيضًا الرعد الذي لا يُطاق، والبرق المُريع، والكأس النتنة الممزوجة بالسم. علّمني عن كل هذه.

\par 7  فأجاب الموت: "اسمع يا إبراهيم البار. سأدمر العالم لسبعة عصور، وأُنزل الجميع إلى الجحيم، الملوك والحكام، الأغنياء والفقراء، العبيد والأحرار، وأقودهم إلى قاع الجحيم، ولهذا أريتك رؤوس الأفاعي السبعة.

\par 8 لقد أريتك وجه النار لأن كثيرين يموتون بالنار، وينظرون الموت من خلال وجه النار.

\par 9 لقد أريتك وجه الهاوية، لأن كثيرًا من الرجال يموتون وهم ينزلون من قمم الأشجار أو المنحدرات الرهيبة ويفقدون حياتهم، ويرون الموت في شكل هاوية رهيبة.

\par 10 وأريتك وجه السيف لأن كثيرين يقتلون في الحروب بالسيف، ويرون الموت سيفاً.

\par 11 لقد أريتك وجه النهر العظيم المتدفق لأن كثيرين غرقوا وهلكوا عندما عبروا مياهًا كثيرة وحملتهم الأنهار العظيمة، ورأوا الموت قبل أوانه.

\par 12 وجه البحر الهائج الهائج أريتك إياه لأن كثيرين في البحر وقعوا في أمواج عظيمة وانكسرت بهم السفن وابتلعهم البحر ونظروا الموت كالبحر.

\par 13 لقد أريتكم الرعد الذي لا يُطاق والبرق المروع لأن كثيرين من الرجال في لحظة الغضب يواجهون رعدًا لا يُطاق وبرقًا مروعًا قادمًا ليُمسك بالرجال، ويرون الموت على هذا النحو

\par 14 أريتكم أيضًا الوحوش السامة، الأفاعي والثعابين البرية، والفهود والأسود وأشبال الأسود، والدببة والأفاعي، وباختصار وجه كل وحش بري أريتكم إياه، أيها الصالح، لأن كثيرين من الناس يهلكهم الوحوش البرية،

\par 15 وغيرهم من الثعابين السامة، والأفاعي، والآفات، والآفات الشراعية، والبازيليسق، والأفاعي، يتنفسون ويموتون

\par 16 أريتكم أيضًا الكؤوس المُهلكة الممزوجة بالسم، لأن كثيرًا من الرجال الذين يُسقون السم ليشربوه من رجال آخرين، يغادرون على الفور دون سابق إنذار

\chapter{20}

\par 1 قال إبراهيم: "أسألك، هل هناك موتٌ غير متوقع أيضًا؟ أخبرني."

\par 2 قال الموت: "الحق الحق أقول لكم بحق الله: هناك اثنان وسبعون موتة. واحدة هي الموت العادل، الذي يشتري وقته المحدد، وكثيرون من الناس يدخلون الموت في ساعة واحدة ويُسلمون إلى القبر

\par 3 ها قد أخبرتك بكل ما طلبت، والآن أقول لك، أيها البار إبراهيم، أن ترفض كل مشورة، وتتوقف عن طلب أي شيء مرة واحدة وإلى الأبد، وتعالي معي، كما أمرني إله الجميع ديانهم

\par 4 قال إبراهيم للموت: «اذهب عني قليلاً أيضًا لأستريح على فراشي، لأني ضعيف القلب جدًا،

\par 5 لأنه منذ أن رأيتك بعيني، خارت قوتي، وبدت لي جميع أطراف جسدي ثقيلة كالرصاص، وروحي مضطربة للغاية. انصرف قليلًا؛ لأني قلت إنني لا أطيق رؤية هيئتك

\par 6 فجاء إسحاق ابنه وسقط على صدره يبكي، وجاءت سارة زوجته واحتضنت قدميه وهي تبكي بمرارة

\par 7 وجاء أيضًا عبيده وإماءه، وأحاطوا بسريره وهم ينوحون بشدة. ودخل إبراهيم في حالة من عدم الاكتراث بالموت،

\par 8 فقال الموت لإبراهيم: "تعال، خذ بيدي اليمنى، وليأتِك البهجة والحياة والقوة."

\par 9 لأن الموت خدع إبراهيم، فأمسك بيده اليمنى، وللوقت التصقت نفسه بيد الموت

\par 10 وفي الحال جاء رئيس الملائكة ميخائيل مع حشد من الملائكة، وأخذ روحه الثمينة بين يديه في كفن منسوج إلهيًا،

\par 11 ورعوا جسد إبراهيم البار بالمراهم والعطور الإلهية إلى اليوم الثالث بعد وفاته، ودفنوه في أرض الموعد، بلوطة ممرا،

\par 12 لكن الملائكة استقبلوا روحه الثمينة، وصعدوا إلى السماء، وهم يغنون ترنيمة الثالوث الأقدس للرب إله الكل، ووضعوها هناك لتعبد الله الآب

\par 13 وبعد أن أعطي الرب تسبيحًا عظيمًا ومجدًا، وسجد إبراهيم ليسجد، جاء صوت الله الآب الطاهر قائلاً هكذا،

\par 14 خذ صديقي إبراهيم إلى الفردوس، حيث خيام أبيّ، ومساكن قديسيّ إسحاق ويعقوب في حضنه، حيث لا همّ ولا حزن ولا تنهد، بل سلام وفرح وحياة لا تنتهي

\par 15 (ولنقتدي نحن أيضًا، يا إخوتي الأحباء، بكرم ضيافة أبينا إبراهيم، ونبلغ إلى فضائل حياته، حتى نُحسب أهلًا للحياة الأبدية، ممجدين الآب والابن والروح القدس، الذي له المجد والقدرة إلى الأبد. آمين.)

\part{النسخة 2}

\chapter{21}

\par 1 وحدث لما اقتربت أيام موت إبراهيم أن الرب قال لميخائيل:

\par 2 قم واذهب إلى إبراهيم عبدي، وقل له: ستفارق الحياة، لأنه هوذا!

\par 3 لقد كَمُلَتْ أيامُ حَياتِكَ الدنيوية، لكي يُرتِّبَ بَيْتَه قبلَ أنْ يَمُوت

\chapter{22}

\par 1 فذهب ميخائيل وجاء إلى إبراهيم، فوجدَهُ جالسًا أمام ثيرانه للحرث، وكان قد شيخوخةً عظيمةً، وكان ابنه بين ذراعيه

\par 2 فلما رأى إبراهيم رئيس الملائكة ميخائيل قام من الأرض وسلم عليه وهو لا يعرف من هو.

\par 3 وقال له: «يحفظك الرب، ويوفقك في رحلتك».

\par 4 فأجابه ميخائيل: "أنت طيب، أيها الأب الصالح."

\par 5 فأجاب إبراهيم وقال له: «تقدم إلي يا أخي واجلس قليلاً حتى آمر بإحضار دابة لنذهب إلى بيتي وتستريح معي لأنه نحو المساء.

\par 6 وفي الصباح استيقظ واذهب إلى أي مكان تريد، لئلا يقابلك وحش شرير ويؤذيك

\par 7 فسأل ميخائيل إبراهيم قائلاً: «أخبرني باسمك قبل أن أدخل بيتك لئلا أثقل عليك».

\par 8 أجاب إبراهيم وقال: «دعاني أبواي أبرام، ودعاني الرب إبراهيم، قائلاً: قم واخرج من بيتك ومن عشيرتك، واذهب إلى الأرض التي أريك».

\par 9 ولما ذهبتُ إلى الأرض التي أراني الرب، قال لي: «لا يُدعى اسمك بعد أبرام، بل يكون اسمك إبراهيم».

\par 10 أجاب ميخائيل وقال له: "عفواً يا أبي، يا رجل الله الخبير، فأنا غريب، وقد سمعت عنك أنك ذهبت أربعين غلوة وأحضرت تيسًا وذبحته، واستضفت الملائكة في بيتك، ليسكنوا هناك."

\par 11 هكذا تحدثا معًا، ثم نهضا وذهبا نحو المنزل.

\par 12 فدعا إبراهيم واحداً من عبيده وقال له: اذهب فأتني بدابة ليركب عليها الغريب لأنه قد تعب من السفر.

\par 13 فقال ميخائيل: "لا تُزعجوا الشاب، بل دعونا نذهب بخفة حتى نصل إلى المنزل، فأنا أحب صحبتكم".

\chapter{23}

\par 1 فقاموا ومضوا، ولما اقتربوا من المدينة،

\par 2 على بُعد حوالي ثلاثة فيرلنغ منها، وجدوا شجرة كبيرة لها ثلاثمائة فرع، تشبه شجرة الطرفاء

\par 3 وسمعوا صوتًا من أغصانها يُرنم: قدوس أنت، لأنك حفظت الغاية التي أُرسلت من أجلها

\par 4 فسمع إبراهيم الصوت، فأخفى ​​السر في قلبه، قائلاً في نفسه: ما هو السر الذي سمعته؟

\par 5 ولما دخل البيت، قال إبراهيم لعبيده: قوموا اخرجوا إلى الغنم، وأتوا بثلاثة خراف، واذبحوها سريعًا، وجهّزوها لنأكل ونشرب، لأن اليوم عيد لنا

\par 6 فأحضر العبيد الغنم، ودعا إبراهيم ابنه إسحاق، وقال له: يا ابني إسحاق، قم وضع ماءً في الإناء لنغسل قدمي هذا الغريب. فأحضره كما أُمر،

\par 7 فقال إبراهيم: إني أرى، وهكذا يكون، أنني لن أغسل في هذا المغسل قدمي أي إنسان يأتي إلينا ضيفًا بعد الآن

\par 8 فلما سمع إسحاق كلام أبيه هذا بكى، وقال له: يا أبي، ما هذا الذي تقوله؟ أهذه آخر مرة أغسل فيها قدمي غريب؟ فلما رأى إبراهيم ابنه يبكي بكى بكاءً شديدًا،

\par 9 فلما رآهم ميخائيل يبكون، بكى أيضًا، فسقطت دموع ميخائيل على الإناء فصارت جوهرة كريمة

\chapter{24}

\par 1 ولما كانت سارة داخل بيتها، سمعت بكاءهم، فخرجت وقالت لإبراهيم: «يا سيد، لماذا تبكي هكذا؟»

\par 2 فأجاب إبراهيم وقال لها: «ليس في الأمر شر. ادخلي بيتك واعملي عملك لئلا نثقل على الرجل».

\par 3 فذهبت سارة لكي تُعد العشاء.

\par 4 "وإذ اقتربت الشمس إلى المغيب، خرج ميخائيل من البيت وأُصعد إلى السماء ليسجد أمام الله،

\par 5 لأنه عند غروب الشمس كل الملائكة يعبدون الله وميخائيل نفسه هو أول الملائكة.

\par 6 وسجدوا له جميعهم، ومضوا كل واحد إلى مكانه،

\par 7 ولكن ميخائيل تكلم أمام الرب وقال: يا رب، مرني أن أمثل أمام مجدك القدوس!

\par 8 فقال الرب لميخائيل: «أعلن ما شئت».

\par 9 فأجاب رئيس الملائكة وقال: يا رب، أرسلتني إلى إبراهيم لأقول له: اذهب من جسدك واترك هذا العالم. الرب يدعوك.

\par 10 ولا أجرؤ يا رب أن أكشف له عن نفسي، لأنه صديقك، ورجل صالح، ويستقبل الغرباء

\par 11 لكني أتوسل إليك يا رب، أن تأمر بذكرى موت إبراهيم أن تدخل قلبه، ولا تطلب مني أن أخبره بها،

\par 12 لأنه من المفاجئ جدًا أن نقول: اترك العالم، وخاصةً أن يترك المرء جسده،

\par 13 لأنك خلقته من البداية ليرحم أرواح جميع البشر

\par 14 ثم قال الرب لميخائيل: «قم واذهب إلى إبراهيم وأقم عنده،

\par 15 وكل ما ترونه يأكله، فكلوه أيضًا، وحيثما ينام، فهناك ينام أيضًا

\par 16 "لأني سألقي في قلب إسحق ابنه فكرة موت إبراهيم في الحلم."

\chapter{25}

\par 1 ثم دخل ميخائيل بيت إبراهيم في ذلك المساء، فوجدهم يُعِدّون العشاء، فأكلوا وشربوا وفرحوا

\par 2 وقال إبراهيم لابنه إسحاق: «قم يا ابني، وافرش سرير الرجل لينام، وضع السراج على المنارة».

\par 3 ففعل إسحاق كما أمره أبوه،

\par 4 فقال إسحق لأبيه: أنا أيضاً آتي لأنام بجانبك.

\par 5 أجابه إبراهيم: «لا يا بني، لعلنا نثقل على هذا الرجل، فاذهب إلى مخدعك ونم».

\par 6 ولما لم يشأ إسحاق أن يعصي أمر أبيه، مضى ونام في حجرته

\chapter{26}

\par 1 وفي نحو الساعة السابعة من الليل استيقظ إسحاق وجاء إلى باب حجرة أبيه وهو يصرخ قائلًا: «افتح يا أبي لأمسك قبل أن يأخذوك عني».

\par 2 فقام إبراهيم وفتح له، فدخل إسحق وتعلق بعنق أبيه وهو يبكي، وقبله بكاءً شديداً.

\par 3 فبكى إبراهيم وابنه، فرأى ميخائيل البكاء فبكى أيضًا.

\par 4 وسمعت سارة البكاء فنادتهم من حجرة نومها،

\par 5 قائلين: يا سيدي إبراهيم، ما هذا البكاء؟ هل أخبرك الغريب بموت لوط ابن أخيك؟ أم أصابنا شيء آخر؟

\par 6 فأجاب ميخائيل وقال لسارة: لا يا سارة لم أبشر بلوط ولكني عرفت كل لطف قلبك الذي في ذلك تفوقين كل الناس على الأرض وقد ذكرك الرب.

\par 7 فقالت سارة لإبراهيم: "كيف تجرؤ على البكاء وقد دخل إليك رجل الله،

\par 8 ولماذا ذرفت عيناكِ والفرح اليوم عظيم؟ قال لها إبراهيم:

\par 9 كيف تعرف أن هذا رجل الله؟

\par 10 فأجابت سارة وقالت: «لأني أقول وأصرح أن هذا هو أحد الرجال الثلاثة الذين استضفناهم عند بلوطة ممرا، حين ذهب أحد الغلمان وأحضر جديً فذبحتموه،

\par 11 وقال لي: قم، أعدّ لنأكل مع هؤلاء الرجال في بيتنا

\par 12 أجاب إبراهيم وقال: "لقد أحسنتِ الفهم أيتها المرأة،

\par 13 لأني أنا أيضاً لما غسلت رجليه عرفت في قلبي أن هاتين الرجلين هما اللتان غسلتهما عند بلوطة ممرا. وحين بدأت أسأل عن سفره قال لي أنا ماضٍ لأنقذ لوطا أخاك من رجال سدوم. وحينئذ عرفت السر.


\chapter{27}

\par 1 فقال إبراهيم لميخائيل: «أخبرني يا رجل الله،

\par 2 وأرني لماذا أتيت إلى هنا».

\par 3 فقال ميخائيل: «ابنك إسحق هو الذي سيريك».

\par 4 فقال إبراهيم لابنه: يا بني الحبيب، أخبرني بما رأيت في حلمك اليوم وخفته، وأخبرني به.

\par 5 فأجاب إسحق أباه: رأيت في حلمي الشمس والقمر، وعلى رأسي إكليل.

\par 6 وجاء من السماء إنسان عظيم القامة ويلمع كالنور الذي يدعى أبا النور.

\par 7 أخذ الشمس من رأسي، ومع ذلك ترك أشعتها معي.

\par 8 فبكيت وقلت أطلب إليك يا سيدي أن لا تنزع مني مجد رأسي ونور بيتي وكل مجدي.

\par 9 وناحت الشمس والقمر والنجوم قائلة: لا تنزعوا منا مجد قدرتنا

\par 10 فأجابني ذلك الرجل المُنير وقال لي: لا تبك على أني آخذ نور بيتك، فإنه قد ارتفع من الضيقات إلى الراحة، من وضع وضيع إلى وضع عالٍ

\par 11 يرفعونه من مكان ضيق إلى مكان واسع، يرفعونه من الظلمة إلى النور

\par 12 فقلت له: أتوسل إليك يا رب، أن تأخذ معها أيضًا الأشعة.

\par 13 قال لي: إن النهار اثنتي عشرة ساعة، ثم سآخذ كل الأشعة.

\par 14 بينما قال الرجل المتألق هذا، رأيت شمس بيتي تصعد إلى السماء، لكنني لم أرَ ذلك التاج بعد ذلك،

\par 15 وتلك الشمس كانت مثلك يا أبي.

\par 16 فقال ميخائيل لإبراهيم: «لقد صدق ابنك إسحاق، لأنك ستذهب وتُرفع إلى السماء،

\par 17 لكن جسدك سيبقى على الأرض، حتى تتم سبعة آلاف عصر، لأنه حينئذٍ سيقوم كل جسد

\par 18 فالآن يا إبراهيم، رتّب بيتك وأولادك، لأنك قد سمعت ما قُضي عليك

\par 19 أجاب إبراهيم وقال لميخائيل: "أتوسل إليك يا رب، إن كنت سأخرج من جسدي، فقد رغبت في أن أُؤخذ إلى جسدي لأرى الخلائق التي خلقها الرب إلهي في السماء وعلى الأرض."

\par 20 أجاب ميخائيل وقال: "ليس من شأني أن أفعل هذا، ولكن سأذهب وأخبر الرب بهذا، وإذا أُمرت سأريكم كل هذه الأشياء."

\chapter{28}

\par 1 وصعد ميخائيل إلى السماء، وتكلم أمام الرب عن إبراهيم،

\par 2 فأجاب الرب ميخائيل: «اذهب وخذ إبراهيم بالجسد، وأره كل شيء، وكل ما يقوله لك فافعل به كما لصديقي».

\par 3 فخرج ميخائيل وأخذ إبراهيم بالجسد على سحابة، وأتى به إلى نهر المحيط

\chapter{29}

\par \textit{نسخة الإصحاح 10 في النسخة 1}

\par 1 وبعد أن رأى إبراهيم موضع القضاء، نزلت به السحابة إلى الجلد من أسفل،

\par 2 فنظر إبراهيم إلى الأرض فرأى رجلاً يزني مع امرأة متزوجة.

\par 3 فالتفت إبراهيم وقال لميخائيل: «أترى هذا الشر؟ لكن يا رب، أرسل نارًا من السماء لتأكلهم».

\par 4 وللوقت نزلت نار وأكلتهم،

\par 5 لأن الرب قال لميخائيل: مهما طلب منك إبراهيم أن تفعل له فافعل.

\par 6 نظر إبراهيم مرة أخرى، فرأى رجالاً آخرين يشتمون أصحابهم،

\par 7 فقال: "لتنشق الأرض وتبتلعهم."

\par 8 وبينما كان يتكلم ابتلعتهم الأرض أحياء.

\par 9 "ثم أخذته السحابة أيضًا إلى مكان آخر، فرأى إبراهيم قومًا منطلقين إلى مكان قفر ليرتكبوا قتلًا،

\par 10 وقال لميخائيل: "أترى هذا الشر؟ فلتخرج الوحوش من البرية وتمزقهم إربًا."

\par 11 وفي تلك الساعة خرجت وحوش برية من البرية وأكلتهم

\par 12 ثم كلم الرب الإله ميخائيل قائلاً: «رد إبراهيم إلى بيته، ولا يطوف حول كل الخليقة التي عملتها، لأنه لا يرحم الخطاة،

\par 13 ولكني أشفق على الخطاة لكي يرجعوا ويحيوا ويتوبوا عن خطاياهم فيخلصوا

\chapter{30}

\par \textit{نسخة الفصل 11 في الإصدار 1}

\par 1 فنظر إبراهيم فإذا بابان أحدهما صغير والآخر كبير.

\par 2 وبين البابين كان رجل جالسًا على عرش مجد عظيم، وحوله حشد من الملائكة،

\par 3 وكان يبكي، ثم يضحك، ففاق بكاؤه ضحكه سبعة أضعاف

\par 4 فقال إبراهيم لميخائيل: "من هذا الجالس بين البابين في مجد عظيم؟ تارة يضحك، وتارة يبكي، ويفوق بكاؤه ضحكه سبعة أضعاف؟"

\par 5 فقال ميخائيل لإبراهيم: «ألا تعلم من هو؟»

\par 6 فقال: لا يا رب.

\par 7 فقال ميخائيل لإبراهيم: «أترى هذين البابين، الصغير والكبير؟»

\par 8 هذه هي التي تؤدي إلى الحياة وإلى الهلاك.

\par 9 وهذا الرجل الذي يجلس بينهما هو آدم، أول رجل خلقه الرب،

\par 10 وأجلسه في هذا المكان لينظر كل نفس تخرج من الجسد، لأن الجميع هم منه.

\par 11 فإذا رأيتموه يبكي فاعلموا أنه رأى أرواحًا كثيرة تُقاد إلى الهلاك،

\par 12 ولكن عندما تراه يضحك، فقد رأى العديد من النفوس تقاد إلى الحياة.

\par 13 أترون كيف يفوق بكاؤه ضحكه؟ لأنه يرى معظم العالم يُقاد عبر الباب الواسع إلى الهلاك، ولذلك يفوق بكاؤه ضحكه سبعة أضعاف.

\chapter{31}

\par 1 فقال إبراهيم: «ومن لا يستطيع أن يدخل من الباب الضيق، ألا يقدر أن يدخل الحياة؟»

\par 2 فبكى إبراهيم وقال: ويل لي ماذا أفعل؟

\par 3 لأني رجل عريض الجسد فكيف أقدر أن أدخل من الباب الضيق الذي لا يستطيع ابن خمس عشرة سنة أن يدخل منه؟

\par 4 فأجاب ميخائيل وقال لإبراهيم: لا تخف يا أبي ولا تحزن، لأنك ستدخل منها دون عائق، أنت وجميع من هم مثلك.

\par 5 وبينما إبراهيم واقفاً متعجباً، إذا ملاك الرب يسوق ستين ألف نفس من الخطاة إلى الهلاك.

\par 6 فقال إبراهيم لميخائيل: «أتذهب كل هؤلاء إلى الهلاك؟»

\par 7 فقال له ميخائيل: نعم، ولكن دعنا نذهب ونبحث بين هذه النفوس، هل يوجد بينهم ولو واحد صالح؟

\par 8 "ولما ذهبوا وجدوا ملاكا يحمل في يده نفسا واحدة من بين هؤلاء الستين ألفا، لأنه وجد خطاياها تزن بالتساوي مع جميع أعمالها، ولم تكن في حالة حركة ولا سكون، بل في حالة بينهما؛

\par 9 لكنه قاد النفوس الأخرى إلى الهلاك.

\par 10 فقال إبراهيم لميخائيل: «يا رب، أهذا هو الملاك الذي ينزع النفوس من الجسد أم لا؟» فأجاب ميخائيل وقال: «هذا هو الموت، وهو يسوقهم إلى دار القضاء ليحاكمهم القاضي».

\chapter{32}

\par 1 وقال إبراهيم: يا رب، أسألك أن تقودني إلى دار القضاء لأرى أنا أيضًا كيف يُحاكمون

\par 2 ثم أخذ ميخائيل إبراهيم على سحابة، وأدخله الفردوس،

\par 3 ولما وصل إلى المكان الذي كان فيه القاضي، جاء الملاك وأعطى تلك الروح للقاضي.

\par 4 فقالت الروح: "يا رب ارحمني."

\par 5 فقال القاضي: كيف أرحمك وأنت لم ترحم ابنتك التي ولدتها ثمرة بطنك؟ لماذا قتلتها؟

\par 6 أجابت: "لا يا سيدي، لم أقتل، ولكن ابنتي كذبت عليّ."

\par 7 لكن القاضي أمره أن يأتي ويدون السجلات،

\par 8 وإذا بالكروبيم يحملون سفرين، ومعهم رجل طويل القامة جدًا، على رأسه ثلاثة تيجان،

\par 9 وكان أحد الإكليلين أعلى من الآخر، وهذان هما إكليلان الشهادة.

\par 10 وكان الرجل في يده قلم من ذهب، فقال له القاضي: أظهر خطيئة هذه النفس.

\par 11 ففتح ذلك الرجل أحد أسفار الكروبيم، وبحث عن خطيئة نفس المرأة فوجدها

\par 12 فقال القاضي: "يا أيتها النفس البائسة، لماذا تقولين إنك لم ترتكبي جريمة قتل؟

\par 13 ألم تذهبي بعد وفاة زوجك وتزني مع زوج ابنتك وتقتليها؟

\par 14 ووبَّخها أيضًا على خطاياها الأخرى، مهما كانت قد فعلتها منذ صغرها

\par 15 فلما سمعت المرأة هذا صرخت قائلة: "ويل لي، لقد نسيتُ جميع خطاياي التي فعلتها في العالم، أما هنا فلم تُنسى."

\par 16 ثم أخذوها أيضًا وسلموها إلى المعذبين.

\chapter{33}

\par 1 فقال إبراهيم لميخائيل: يا سيد، من هو هذا القاضي، ومن هو الآخر الذي يوبخ على الخطايا؟

\par 2 فقال ميخائيل لإبراهيم: «أرأيت القاضي؟ هذا هو هابيل الذي شهد أولاً، فأتى به الله إلى هنا ليقضي،

\par 3 والذي يشهد هنا هو معلم السماء والأرض وكاتب البر حنوك.

\par 4 "فإن الرب أرسلهم إلى هنا ليكتبوا خطايا كل واحد وصلاحه."

\par 5 قال إبراهيم: «وكيف يستطيع أخنوخ أن يحمل ثقل النفوس وهو لم يرَ الموت؟ أو كيف يستطيع أن يُدين جميع النفوس؟»

\par 6 قال ميخائيل: "إذا كان يصدر حكمًا بشأن النفوس، فهذا غير مسموح به؛ ولكن أخنوخ نفسه لا يصدر حكمًا،

\par 7 لكن الرب هو الذي يفعل ذلك، وليس عليه أن يفعل أكثر من مجرد الكتابة.

\par 8 لأن أخنوخ صلى إلى الرب قائلاً: يا رب، لا أريد أن أُصدر حكمًا على النفوس، لئلا أكون عبئًا على أحد؛

\par 9 وقال الرب لأخنوخ: سأأمرك أن تكتب خطايا النفس التي تكفّر، فتدخل الحياة،

\par 10 وإن لم تكفّر النفس وتتوب، ستجد خطاياها مكتوبة وستُلقى في العقاب. ونحو الساعة التاسعة، أعاد ميخائيل إبراهيم إلى بيته. أما سارة امرأته، فلما لم ترَ ما حل بإبراهيم، استحوذ عليها الحزن وأسلمت الروح. وبعد عودة إبراهيم وجدها ميتة فدفنها

\chapter{34}

\par 1 ولما اقترب يوم وفاة إبراهيم، قال الرب الإله لميخائيل:

\par 2 لن يجرؤ الموت على الاقتراب ليأخذ روح خادمي، لأنه صديقي، لكن اذهب وزين الموت بجمال عظيم، وأرسله هكذا إلى إبراهيم، حتى يراه بعينيه

\par 3 وفي الحال، كما أُمر، زيّن ميخائيل الموت بجمال عظيم، وأرسله هكذا إلى إبراهيم ليراه

\par 4 وجلس بالقرب من إبراهيم، فلما رأى إبراهيم الموت جالسًا بالقرب منه، خاف خوفًا عظيمًا

\par 5 فقال الموت لإبراهيم: "السلام عليك أيتها الروح القدس! السلام عليك يا صديق الرب الإله! السلام عليك يا عزاء وتسلية المسافرين!"

\par 6 فقال إبراهيم: "مرحبًا بك يا عبد الله العلي. أتوسل إليك، أخبرني من أنت؛ وادخل بيتي وتناول الطعام والشراب، ثم اذهب عني، لأني منذ أن رأيتك جالسًا بقربي اضطربت نفسي."

\par 7 لأني لستُ أهلاً للاقتراب منك إطلاقًا، لأنك روحٌ سامية، وأنا لحمٌ ودم،

\par 8 "ولذلك لا أستطيع أن أتحمل مجدك، لأني أرى أن جمالك ليس من هذا العالم."

\par 9 فقال الموت لإبراهيم: أقول لك: في كل الخليقة التي عملها الله لم يوجد مثلك،

\par 10 لأن الرب نفسه لم يبحث ولم يجد مثل هذا على كل الأرض.

\par 11 فقال إبراهيم للموت: "كيف تجرؤ على الكذب؟ لأني أرى أن جمالك ليس من هذا العالم."

\par 12 فقال الموت لإبراهيم: "لا تظن يا إبراهيم أن هذا الجمال لي، أو أنني آتي هكذا إلى كل إنسان. كلا، بل إن كان أحد بارًا مثلك، فسآخذ التيجان وأأتي إليه، وإن كان خاطئًا، فإني آتي في فساد عظيم، ومن خطيئتهم أصنع إكليلًا لرأسي، وأهزهم بخوف عظيم، فيرتاعون."

\par 13 فقال له إبراهيم: «ومن أين جمالك؟»

\par 14 فقال الموت: ليس هناك أحد أكثر فسادًا مني.

\par 15 قال له إبراهيم: "وهل أنت حقًا هو الذي يُدعى الموت؟"

\par 16 فأجابه وقال: أنا الاسم المرّ، أبكي...

\chapter{35}

\par 1 فقال إبراهيم للموت: «أرنا فسادك».

\par 2 "وأظهر الموت فساده، وكان له رأسان،

\par 3 كان لأحدهم وجه ثعبان، وبسببه يموت البعض دفعة واحدة بسبب الأفاعي،

\par 4 وكان الرأس الآخر كالسيف، يموت به البعض بالسيف كما بالأقواس

\par 5 في ذلك اليوم مات عبيد إبراهيم خوفًا من الموت، فلما رآهم إبراهيم صلى إلى الرب فأقامهم

\par 6 ولكن الله رجع وأخذ نفس إبراهيم كما في حلم، ورفعها رئيس الملائكة ميخائيل إلى السماء

\par 7 ودفن إسحاق أباه بجانب أمه سارة، مُمجِّدًا الله ومُسبِّحًا إياه، لأنه يليق به المجد والإكرام والسجود، أيها الآب والابن والروح القدس، الآن وكل أوان وإلى دهر الدهور. آمين

\end{document}