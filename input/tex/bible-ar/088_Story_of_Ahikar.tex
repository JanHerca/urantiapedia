\begin{document}

\title{قصة أحيقار}

\chapter{1}

\par \textit{أحيقار، الصدر الأعظم لآشور، لديه 60 زوجة، لكن قُدِّر له ألا يكون له ابن. لذلك تبنى ابن أخيه. وملأه بالحكمة والمعرفة أكثر مما ملأه بالخبز والماء.}

\par 1 قصة حيقار الحكيم، وزير الملك سنحاريب، ونادان، ابن أخت حيقار الحكيم

\par 2 كان هناك وزير في أيام الملك سنحاريب، ابن سرهادوم، ملك آشور ونينوى، رجل حكيم اسمه حيقار، وكان وزير الملك سنحاريب

\par 3 كان لديه ثروة طائلة وممتلكات كثيرة، وكان ماهرًا وحكيمًا وفيلسوفًا في المعرفة والرأي والحكم، وقد تزوج من ستين امرأة، وبنى لكل واحدة منهن قلعة

\par 4 ولكن مع كل ذلك، لم ينجب أي طفل من أي من هؤلاء النساء، اللواتي قد يكنّ وريثته

\par 5 فحزن حزنًا شديدًا على ذلك، فجمع يومًا من الأيام المنجمين والعلماء والسحرة، وشرح لهم حالته وأمر عقمه

\par 6 فقالوا له: اذهب، ذبح للآلهة، واطلب منهم أن يرزقوك بولد

\par 7 ففعل كما قالوا له، وذبح ذبائح للأصنام، وتضرع إليها وتضرع إليها بالطلب والتضرع

\par 8 فلم يجيبوه بكلمة واحدة. فمضى حزينًا ومكتئبًا، وقلبه يخفق بشدة

\par 9 فعاد وتضرع إلى الله العلي، وآمن، متوسلاً إليه بحرارة في قلبه، قائلاً: يا الله العلي، يا خالق السماوات والأرض، يا خالق كل المخلوقات!

\par 10 «أتوسل إليك أن تمنحني ولدًا، حتى أحظى بتعزيته، حتى يكون حاضرًا في صحتي، حتى يغمض عيني، حتى يدفنني.»

\par 11 ثم جاءه صوت قائلًا: «بما أنك اعتمدت أولًا على التماثيل المنحوتة، وقدمت لها ذبائح، فمن أجل هذا ستبقى بلا أولاد طوال حياتك».

\par 12 «لكن خذ نادان ابن أختك، واجعله ابنك، وعلمه علمك وتربيتك الحسنة، وعند وفاتك يدفنك.»

\par 13 ثم أخذ نادان ابن أخته وهو رضيع صغير، ودفعه إلى ثماني مرضعات ليرضعنه ويربينه

\par 14 وربوه بطعام جيد وأدب رقيق وملابس حريرية وأرجوان وقرمز. وكان جالسًا على أسِرّة من حرير

\par 15 وعندما كبر نادان ومشى، منتصبًا كشجرة أرز طويلة، علمه الأخلاق الحميدة والكتابة والعلم والفلسفة

\par 16 وبعد أيام عديدة، نظر الملك سنحاريب إلى حيقار ورأى أنه قد كبر كثيرًا، وعلاوة على ذلك، قال له

\par 17 يا صديقي الكريم، الماهر، الأمين، الحكيم، الحاكم، أمين سرّي، وزيري، مستشاري ومديري؛ إنك قد كبرت في السنّ واثقلتك السنين؛ ولا بد أن رحيلك من هذه الدنيا قريب.

\par 18 «أخبرني من سيتولى خدمتي بعدك». فقال له حيقار: «يا سيدي، ليحي رأسك إلى الأبد! هذا نادان ابن أختي، جعلته ابني».

\par 19 «وربيته وعلمته حكمتي وعلمي».

\par 20 فقال له الملك: يا حيقار! أحضره إليّ لأراه، فإن وجدته مناسبًا، فأجلسه مكانك، واذهب في طريقك لتستريح وتعيش ما تبقى من حياتك في راحة وسكينة

\par 21 ثم ذهب حيقار وقدم نادان ابن أخته. فسجد له وتمنى له القوة والشرف

\par 22 فنظر إليه وأعجب به وفرح به وقال لحَيْقَار: «هل هذا ابنك يا حَيْقَار؟ أسأل الله أن يحفظه. وكما خدمتني وخدمت أبي سرهادم، فليخدمني هذا الغلام ويؤدي واجباتي واحتياجاتي وأعمالي، حتى أكرمه وأجعله قويًا من أجلك».

\par 23 فسجد حيقار للملك وقال له: "ليحي رأسك يا سيدي الملك إلى الأبد! أسألك أن تصبر على ابني نادان وتغفر له أخطائه حتى يخدمك كما يليق."

\par 24 ثم أقسم له الملك أنه سيجعله أعظم مفضليه، وأقوى أصدقائه، وأن يكون معه بكل شرف واحترام. وقبل يديه وودّعه

\par 25 وأخذ نادان ابن أخته معه وأجلسه في غرفة، وبدأ يعلمه ليلًا ونهارًا حتى ملأه بالحكمة والمعرفة أكثر من الخبز والماء



\chapter{2}

\par \textit{تقويم ريتشارد المسكين من العصور القديمة. مبادئ خالدة للسلوك البشري فيما يتعلق بالمال والنساء واللباس والأعمال والأصدقاء. توجد أمثال مثيرة للاهتمام بشكل خاص في الآيات 12، 17، 23، 37، 45، 47. قارن الآية 63 ببعض السخرية السائدة اليوم.}

\par 1 هكذا علمه قائلاً: يا بني! اسمع كلامي واتبع نصيحتي واحفظ ما أقول

\par 2 يا بني! إذا سمعت كلمة، فدعها تموت في قلبك، ولا تكشفها لغيرك، لئلا تصبح جمرة متقدة وتحرق لسانك وتسبب ألمًا في جسدك، وتنال العار، وتُخزى أمام الله والناس

\par 3 يا بني! إذا سمعت خبرًا فلا تنشره، وإذا رأيت شيئًا فلا تحدث به

\par 4 «يا بني! سهّل بلاغتك على المستمع، ولا تتعجل في الرد.»

\par 5 يا بني! إذا سمعت شيئًا فلا تخفه.

\par 6 يا بني لا تحل عقدة ولا تحل عقدة ولا تغلق عقدة ملتوية

\par 7 يا بني لا تشتهِ الجمال الخارجي، فإنه يزول ويزول، أما الذكرى الشريفة فهي باقية إلى الأبد.

\par 8 يا بني! لا تخدعك امرأة حمقاء بكلامها، لئلا تموت ميتة بائسة، وتعلقك في الشبكة حتى تقع في الفخ

\par 9 يا بني، لا تشتهِ امرأةً متبرجةً باللباس والأدهان، حقيرةً وحمقاءً في نفسها. ويلٌ لك إن أعطيتها شيئًا مما لك، أو سلمتها ما في يدك، فأغوتك بالمعصية، وغضب الله عليك

\par 10 يا بني! لا تكن مثل شجرة اللوز، فإنها تُنتج أوراقًا قبل كل الأشجار، وثمارًا صالحة للأكل بعد كل منها، بل كن مثل شجرة التوت، التي تُنتج ثمارًا صالحة للأكل قبل كل الأشجار، وأوراقًا بعد كل منها

\par 11 يا بني! اخفض رأسك، وأرخِ صوتك، وكن مهذبًا، وامشِ في الطريق المستقيم، ولا تكن أحمق. ولا ترفع صوتك عندما تضحك، فلو بُني منزل بصوت عالٍ، لبنى الحمار منازل كثيرة كل يوم؛ ولو سُوق المحراث بقوة، لما أُزيل المحراث من تحت أكتاف الجمال

\par 12 «يا بني! إزالة الحجارة مع عاقل خير من شرب الخمر مع تعيس».

\par 13 يا بني! اسكب خمرك على قبور الأبرار، ولا تشرب مع الجهلاء الحقراء

\par 14 يا بني، التزم بالحكماء الذين يخافون الله وكن مثلهم، ولا تقترب من الجاهلين لئلا تصبح مثله وتتعلم طرقه

\par 15 يا بني! إذا كان لك رفيق أو صديق، فاختبره، ثم اجعله رفيقًا وصديقًا؛ ولا تمدحه دون تجربة؛ ولا تفسد حديثك مع من يفتقر إلى الحكمة

\par 16 يا بني! ما دام الحذاء في قدمك، فامشِ به على الأشواك، واصنع طريقًا لابنك، ولأهل بيتك، ولأولادك، وشدّ سفينتك قبل أن تبحر في البحر وأمواجه وتغرق ولا سبيل لإنقاذها

\par 17 يا بني! إذا أكل الغني حية، قالوا: "بحكمته"، وإذا أكلها الفقير، قال الناس: "من جوعه".

\par 18 «يا بني! اقنع بخبزك اليومي وممتلكاتك، ولا تشتهِ ما هو للغير.»

\par 19 يا بني! لا تكن جارًا للأحمق، ولا تأكل معه خبزًا، ولا تفرح بمصائب جيرانك. 1 إذا ظلمك عدوك، فأظهر له معروفًا

\par 20 يا بني! من يخاف الله فاخشه وأكرمه.

\par 21 يا بني! الجاهل يتعثر ويتعثر، والحكيم لا يهتز حتى لو تعثر، وإن سقط نهض سريعًا، وإن مرض استطاع أن يتدبر أمره. أما الجاهل الغبي فلا دواء لمرضه.

\par 22 يا بني إذا جاءك من هو دونك فاقبل إليه وقم فإن لم يجزك فإن ربه يجزيك عنه.

\par 23 يا بني! لا تدخر جهدًا في ضرب ابنك، فإن ضربه كسماد الحديقة، ومثل ربط فم الكيس، ومثل ربط البهائم، ومثل إغلاق الباب

\par 24 يا بني! كف ابنك عن الشر، وعلمه الأدب قبل أن يتمرد عليك ويجلب لك الاحتقار بين الناس، وتحني رأسك في الشوارع والمجالس، وتعاقب على شر أفعاله الشريرة

\par 25 يا بني! احصل لنفسك على ثور سمين بقلفة، وحمار كبير ذو حوافر، ولا تختر ثورًا ذا قرون كبيرة، ولا تصادق رجلاً مخادعًا، ولا تختر عبدًا مشاكسًا، ولا أمة لصًا، فكل ما تُسلمه إليهم سيُفسدونه

\par 26 يا بني! لا يلعنك والداك، وليكن الرب راضيًا عنهما؛ فقد قيل: "من احتقر أباه أو أمه فليمت موتة الخطيئة؛ ومن أكرم والديه تطول أيامه وعمره ويرى كل خير".

\par 27 يا بني! لا تمشِ في الطريق بلا سلاح، لأنك لا تعرف متى قد يقابلك العدو، حتى تكون مستعدًا له

\par 28 يا بني! لا تكن كشجرة عارية لا أوراق لها لا تنمو، بل كن كشجرة مغطاة بأوراقها وأغصانها؛ لأن الرجل الذي ليس له زوجة ولا أولاد يُهان في العالم ويكرهه، كشجرة عارية لا ثمر لها

\par 29 يا بني، كن كشجرة مثمرة على جانب الطريق، يأكل ثمرها كل من يمر، وتستريح وحوش البرية تحت ظلها وتأكل من ورقها

\par 30 يا بني! كل شاة تضل عن طريقها وصحابتها تصبح طعامًا للذئب

\par 31 يا بني! لا تقل: سيدي أحمق وأنا حكيم، ولا تروي كلام الجهل والحماقة، لئلا يحتقرك

\par 32 يا بني! لا تكن من أولئك الخدم الذين يقول لهم أسيادهم: ابتعدوا عنا، ولكن كن ممن يقولون لهم: اقترب وادن منا

\par 33 يا بني! لا تداعب عبدك في حضرة صاحبه، لأنك لا تدري أيهما سيكون الأكثر قيمة لك في النهاية

\par 34 «يا بني، لا تخف من ربك الذي خلقك فيسكت عنك».

\par 35 يا بني! اجعل كلامك جميلًا وحلو لسانك؛ ولا تدع صاحبك يدوس على قدمك لئلا يدوس على صدرك مرة أخرى

\par 36 يا بني! إذا ضربت رجلاً حكيماً بكلمة حكمة، فإنها ستختبئ في صدره كشعور خفي بالعار؛ ولكن إذا ضربت الجاهل بعصا فلن يفهم ولن يسمع

\par 37 يا بني، إذا أرسلت حكيمًا لقضاء حاجتك، فلا تُكثر من الأوامر، فإنه سيُنفِّذ أمرك كما تُريد. وإذا أرسلت جاهلًا، فلا تأمره، بل اذهب أنت واقضِ أمرك، فإنه إن أمرته، فلن يفعل ما تُريد. وإذا أرسلوك في مهمة، فأسرع في إنجازها.

\par 38 يا بني! لا تعتد على من هو أقوى منك، فإنه سيأخذ قدرك وينتقم منك

\par 39 يا بني! جرّب ابنك وخادمك قبل أن تُسلم إليهما ممتلكاتك، لئلا يتلفاها؛ لأن من كانت يده ممتلئة يُدعى حكيمًا، حتى لو كان غبيًا وجاهلًا، ومن كانت يده فارغة يُدعى فقيرًا وجاهلًا، حتى لو كان أمير الحكماء

\par 40 يا بني! لقد أكلت الحنظل، وشربت الصبر، فلم أجد شيئًا أشد مرارة من الفقر والقلة

\par 41 «يا بني! علّم ابنك الاقتصاد والجوع، حتى يُحسن تدبير بيته.»

\par 42 يا بني! لا تُعلِّم الجاهل لغة الحكماء، فإنها ستكون ثقيلة عليه

\par 43 يا بني! لا تُظهر حالتك لصديقك لئلا يحتقرك

\par 44 يا بني! إن عمى القلب أشد من عمى البصر، فإن عمى البصر قد يهتدي شيئًا فشيئًا، وأما عمى القلب فلا يهتدي، فيترك الطريق المستقيم، ويسلك طريقًا أعوج

\par 45 «يا بني! عثرة الرجل برجله خير من عثرة الرجل بلسانه».

\par 46 يا بني! صديق قريب خير من أخ أفضل منه وهو بعيد

\par 47 يا بني! الجمال يذبل، لكن العلم يدوم، والعالم يذبل ويصبح باطلاً، لكن السمعة الطيبة لا تذبل ولا تذبل

\par 48 يا بني! من لا راحة له، موته خير من حياته؛ وصوت بكائه خير من صوت الغناء؛ لأن الحزن والبكاء، إذا كان فيهما خوف الله، خير من صوت الغناء والفرح

\par 49 يا بني! فخذ ضفدع في يدك خير من إوزة في قدر جارك؛ وخروف قريب منك خير من ثور بعيد؛ وعصفور في يدك خير من ألف عصفور يطير؛ والفقر الذي يجمع خير من تبديد رزق كثير؛ والثعلب الحي خير من أسد ميت؛ ورطل من الصوف خير من رطل من الثروة، أعني من الذهب والفضة؛ لأن الذهب والفضة مخفيان ومغطاة في الأرض، ولا يُرى؛ أما الصوف فيبقى في الأسواق ويُرى، وهو جمال لمن يرتديه

\par 50 «يا بني! ثروة قليلة خير من ثروة متناثرة.»

\par 51 يا بني، كلب حي خير من فقير ميت.

\par 52 «يا بني! فقير يعمل الصالحات خير من غني ميت بالذنوب.»

\par 53 "يا بني! احتفظ بكلمة في قلبك، وسوف يكون لها أثر كبير عليك، واحذر أن تكشف سر صديقك."

\par 54 يا بني! لا تخرج كلمة من فمك حتى تتشاور مع قلبك. ولا تقف بين الناس المتخاصمين، لأنه من الكلمة السيئة يأتي شجار، ومن الشجار تأتي حرب، ومن الحرب يأتي قتال، وستُجبر على الشهادة؛ لكن اهرب من هناك واسترح

\par 55 يا بني! لا تقاوم من هو أقوى منك، بل تحلَّ بروح الصبر والتحمل والسلوك القويم، فلا شيء أفضل من ذلك

\par 56 يا بني! لا تكره صديقك الأول، فالثاني قد لا يدوم

\par 57 يا بني! زُر الفقير في محنته، وتحدث عنه في حضرة السلطان، واجتهد في إنقاذه من فم الأسد

\par 58 يا بني! لا تفرح بموت عدوك، فبعد قليل ستكون جاره، ومن يسخر منك، فاحترمه وأكرمه، وكن معه في التحية مسبقًا

\par 59 يا بني! لو أن الماء توقف في السماء، والغراب الأسود أصبح أبيض، والمر أصبح حلوًا كالعسل، لكان الجهلاء والحمقى قد فهموا وأصبحوا حكماء

\par 60 يا بني! إن أردت أن تكون حكيمًا، فاكف لسانك عن الكذب، ويدك عن السرقة، وعينيك عن النظر إلى الشر؛ حينئذٍ تُدعى حكيمًا

\par 61 يا بني! ليضربك الحكيم بقضيب، ولكن لا يدهنك الأحمق بدهن حلو. كن متواضعًا في شبابك تُكرّم في شيخوختك

\par 62 يا بني! لا تصمد أمام رجل في أيام قوته، ولا نهر في أيام فيضانه

\par 63 يا بني! لا تتعجل في زواج المرأة، فإن كان خيرًا قالت: يا سيدي زودني، وإن كان شرًا عوقبت على من كان سببًا فيه

\par 64 يا بني! من كان أنيقًا في لباسه، فهو كذلك في كلامه؛ ومن كان قبيح المنظر في لباسه، فهو أيضًا كذلك في كلامه

\par 65 يا بني! إذا ارتكبت سرقة، فأخبر السلطان، وأعطه نصيبًا منها، حتى تنجو منه، وإلا ستعاني المرارة

\par 66 يا بني! اجعل صديقًا لمن شبعت يده وشبعت، ولا تجعل صديقًا لمن شبعت يده وجائعة

\par 67 «أربعة أشياء لا يأمن عليها الملك ولا جيشه: ظلم الوزير، وسوء الحكم، وتحريف الإرادة، والطغيان على الرعية؛ وأربعة أشياء لا تُخفى: العاقل، والساذج، والغني، والفقير.»

\par \textit{الحواشي}

\par \textit{201:1 قارن المزامير 141: 4.}

\par \textit{203:1 قارن "عصفور في اليد خير من عشرة على الشجرة."}

\par \textit{203:2 قارن 2 تيموثاوس، 4، 17.}

\chapter{3}

\par \textit{ينسحب أحيكار من المشاركة الفعلية في شؤون الدولة. ويُسلم ممتلكاته لابن أخيه الخائن. إليكم القصة المذهلة لكيفية تحوّل مسرف جاحد إلى مزور. تُحكم عليه بالإعدام نتيجة مؤامرة ذكية للإيقاع بأحيكار. ويبدو أنها نهاية أحيكار.}

\par 1 هكذا تكلم حيقار، وعندما انتهى من هذه الوصايا والأمثال لنادان، ابن أخته، تخيل أنه سيحفظها جميعًا، ولم يكن يعلم أنه بدلاً من ذلك كان يُظهر له التعب والازدراء والاستهزاء

\par 2 بعد ذلك، جلس حيقار ساكنًا في منزله، وسلم إلى نادان جميع ممتلكاته، والعبيد، والإماء، والخيول، والماشية، وكل ما كان يملكه ويكتسبه؛ وظلت سلطة الأمر والنهي في يد نادان

\par 3 وجلس حيقار مستريحًا في منزله، وكان بين الحين والآخر يذهب لتقديم احتراماته للملك، ثم يعود إلى منزله

\par 4 فلما أدرك نادان أن الأمر والنهي بيده، احتقر منصب حيقار وسخر منه، وأخذ يلومه كلما ظهر، قائلاً: عمي حيقار في شيخوخته، ولا يعرف شيئًا الآن

\par 5 وبدأ يضرب العبيد والإماء، ويبيع الخيول والإبل، ويبذر بكل ما يملك عمه حيقار

\par 6 ولما رأى حيقار أنه لا يرحم عبيده ولا أهل بيته، قام وطرده من بيته، وأرسل ليخبر الملك أنه قد بدد ممتلكاته وزاده

\par 7 فقام الملك ودعا نادان وقال له: ما دام حيقار بصحة جيدة، فلن يتسلط أحد على ممتلكاته، ولا على بيته، ولا على ممتلكاته

\par 8 رُفعت يد نادان عن عمه حيقار وعن جميع ممتلكاته، وفي هذه الأثناء لم يدخل ولم يخرج، ولم يُسلم عليه

\par 9 عندئذٍ تاب حيقار عن تعبه مع نادان ابن أخته، وظل حزينًا جدًا

\par 10 وكان لنادان أخ أصغر اسمه بُنُزَرْدان، فاتخذه حيقار لنفسه مكان نادان، ورباه وأكرمه إكرامًا عظيمًا. وسلم إليه كل ما كان يملك، وجعله واليًا على بيته

\par 11 فلما أدرك نادان ما حدث، استولى عليه الحسد والغيرة، وبدأ يشكو لكل من سأله، ويسخر من عمه حيقار، قائلاً: "لقد طردني عمي من منزله، وفضل أخي عليّ، ولكن إذا منحني الله العلي القدرة، فسأجلب عليه مصيبة القتل".

\par 12 وظل نادان يفكر في العقبة التي قد يدبرها له. وبعد برهة، قلب نادان الأمر في ذهنه، وكتب رسالة إلى أخيش، ابن شاه الحكيم، ملك فارس، يقول فيها:

\par 13 السلام والصحة والقوة والكرامة من سنحاريب ملك آشور ونينوى، ومن وزيره وكاتبه هيقار إليك أيها الملك العظيم! فليكن بيني وبينك بنسات.

\par 14 «وعندما تصل إليك هذه الرسالة، فإن قمتَ وذهبتَ سريعًا إلى سهل نسرين، وإلى آشور، ونينوى، فسأُسلم إليك المملكة بلا حرب ولا حشد قتالي.»

\par 15 وكتب أيضًا رسالة أخرى باسم حيقار إلى فرعون ملك مصر. «ليكن بيني وبينك سلام أيها الملك العظيم!»

\par 16 «إذا قمتَ وقتَ وصولِ هذه الرسالةِ إليكَ وذهبتَ إلى آشورَ ونينوى إلى سهلِ نسرين، فسأُسلِّمُ لكَ المملكةَ بلا حربٍ ولا قتالٍ.»

\par 17 وكانت كتابة نادان مثل كتابة عمه حيقار

\par 18 ثم طوى الرسالتين وختمهما بخاتم عمه حيقار، وظلتا مع ذلك في قصر الملك

\par 19 ثم ذهب وكتب رسالة مماثلة من الملك إلى عمه حيقار: «السلام والصحة لوزيري، وكاتبي، ومستشاري، حيقار».

\par 20 يا حيقار، إذا وصلت إليك هذه الرسالة، فاجمع جميع الجنود الذين معك، وليكنوا كاملين في الملابس والعدد، وأحضرهم إليّ في اليوم الخامس في سهل نسرين

\par 21 «وإذا رأيتني آتيًا إليك، فأسرع وأرسل الجيش ضدي كعدو يريد القتال معي، لأن معي سفراء فرعون ملك مصر، لكي يروا قوة جيشنا ويخافونا، لأنهم أعداؤنا ويكرهوننا.»

\par 22 ثم ختم الرسالة وأرسلها إلى حيقار مع أحد خدم الملك. وأخذ الرسالة الأخرى التي كتبها ونشرها أمام الملك وقرأها عليه وأراه الخاتم

\par 23 فلما سمع الملك ما في الرسالة، ارتبك حيرةً عظيمةً، وغضب غضبًا شديدًا، وقال: آه، لقد أظهرتُ حكمتي! ماذا فعلتُ بحيقار حتى كتب هذه الرسائل إلى أعدائي؟ هل هذا جزاءٌ لي منه على ما أسديتُه له؟

\par 24 فقال له نادان: لا تحزن أيها الملك ولا تغضب، بل دعنا نذهب إلى سهل نسرين ونرى إن كانت القصة صحيحة أم لا

\par 25 ثم قام نادان في اليوم الخامس وأخذ الملك والجنود والوزير، وذهبوا إلى الصحراء إلى سهل نسرين. ونظر الملك، وإذا حيقار والجيش مصطفون

\par 26 ولما رأى حيقار أن الملك موجود، اقترب وأشار إلى الجيش بالتحرك كما في حالة حرب، والقتال في صفوف ضد الملك كما وُجد في الرسالة، وهو لا يدري أي حفرة حفرها له نادان

\par 27 ولما رأى الملك فعل حيقار، استولى عليه القلق والرعب والحيرة، وغضب غضبًا شديدًا

\par 28 فقال له نادان: أرأيت يا سيدي الملك ما فعل هذا الوغد؟ ولكن لا تغضب ولا تحزن ولا تتألم، بل اذهب إلى بيتك واجلس على عرشك، وأنا آتي إليك بحيقار مقيدًا بالسلاسل، وأطرد عدوك عنك بلا عناء.

\par 29 فعاد الملك إلى عرشه، وقد استاء من حيقار، ولم يفعل شيئًا بشأنه. وذهب نادان إلى حيقار وقال له: "والله يا عم! إن الملك يفرح بك فرحًا عظيمًا ويشكرك على فعل ما أمرك به."

\par 30 «والآن أرسلني إليك حتى تأمر الجنود بمهامهم وتأتي إليه مقيد اليدين والقدمين، حتى يرى سفراء فرعون ذلك، وحتى يهاب الملك منهم ومن ملكهم.»

\par 31 فأجاب حيقار وقال: السمع والطاعة. فقام في الحال وربط يديه خلف ظهره وقيد رجليه بالسلاسل

\par 32 فأخذه نادان وذهب معه إلى الملك. وعندما دخل حيقار إلى حضرة الملك، سجد أمامه على الأرض، وتمنى للملك القوة والحياة الدائمة

\par 33 ثم قال الملك: يا حيقار، يا أمين سرّي، يا حاكم أموري، يا مستشاري، يا حاكم ولايتي، أخبرني ما هو الشر الذي فعلته لك حتى كافأني بهذا العمل القبيح

\par 34 ثم أروه الحروف بخطه وخاتمه. فلما رأى حيقار ذلك، ارتجفت أطرافه وانعقد لسانه في الحال، ولم يستطع النطق بكلمة من الخوف، بل أطرق رأسه نحو الأرض وصمت

\par 35 فلما رأى الملك ذلك، أيقن أن الأمر منه، فقام على الفور وأمرهم بقتل حيقار، وضرب عنقه بالسيف خارج المدينة

\par 36 ثم صرخ نادان وقال: يا حيقار، يا صاحب الوجه الأسود! ما الذي ينفعك تفكيرك أو قدرتك في فعل هذا الفعل للملك؟

\par 37 هكذا قال الراوي. وكان اسم السياف أبو سميك. فقال له الملك: يا سياف! قم، انطلق، اقطع عنق حيقار عند باب داره، وألق رأسه عن جسده مئة ذراع

\par 38 ثم ركع حيقار أمام الملك، وقال: "ليحي سيدي الملك إلى الأبد! وإن كنت ترغب في قتلي، فلتتحقق أمنيتك؛ وأنا أعلم أنني لست مذنبًا، لكن على الرجل الشرير أن يُعطي حسابًا عن شره؛ ومع ذلك، يا سيدي الملك! أتوسل إليك وإلى صداقتك، أن تسمح للسياف أن يُعطي جسدي لعبيدي، ليدفنوني، وليكن عبدك ذبيحتك."

\par 39 نهض الملك وأمر السياف أن يفعل به ما يشاء

\par 40 وأمر عبيده على الفور أن يأخذوا حيقار والسياف ويذهبوا معه عراة حتى يقتلوه

\par 41 ولما علم حيقار أنه سيقتل أرسل إلى امرأته وقال لها: اخرجي للقائي، وليكن معك ألف فتاة عذارى، وألبسيهن ثياباً من أرجواني وحرير حتى يبكين علي قبل موتي.

\par 42 «وأعدّوا مائدة للسياف ولخدمه. واخلطوا خمرًا كثيرة ليشربوا.»

\par 43 وفعلت كل ما أمرها به. وكانت حكيمة وذكية وحكيمة جدًا. وجمعت كل ما أمكنها من أدب وعلم

\par 44 ولما وصل جيش الملك والسياف، وجدوا المائدة مرتبة، والنبيذ والأطعمة الفاخرة، فبدأوا يأكلون ويشربون حتى شبعوا وسكروا

\par 45 ثم أخذ حيقار السياف جانبًا من بين الرفاق وقال: يا أبا سميك، ألا تعلم أنه عندما أراد سرهادوم الملك، والد سنحاريب، قتلك، أخذتك وأخفيتك في مكان ما حتى هدأ غضب الملك وطلبك؟

\par 46 «وعندما أحضرتك إلى حضرته فرح بك. والآن تذكر المعروف الذي صنعته إليك.»

\par 47 «وأنا أعلم أن الملك سيندم عليّ، وسيغضب غضبًا عظيمًا بسبب إعدامي.»

\par 48 «لأني لستُ مذنبًا، وعندما تُقدّمني أمامه في قصره، ستُلاقي حظًا سعيدًا كبيرًا، وتعلم أن نادان ابن أختي قد خدعني وفعل بي هذا الفعل السيئ، وسيندم الملك على قتلي؛ والآن لديّ قبو في حديقة منزلي، ولا أحد يعلم بذلك.»

\par 49 «أخفوني فيه بعلم زوجتي. ولدي عبد في السجن يستحق القتل.»

\par 50 «أخرجوه وألبسوه ملابسي، وأمروا الخدم أن يقتلوه عندما يسكرون. لن يعرفوا من يقتلونه.»

\par 51 «واطرح رأسه مئة ذراع عن جسده، وأعطِ جسده لعبيدي ليدفنوه. وستكون قد ادخرت لي كنزًا عظيمًا.»

\par 52 ثم فعل السياف ما أمره به حيقار، وذهب إلى الملك وقال له: "ليحي رأسك إلى الأبد!"

\par 53 «ثم كانت زوجة حيقار تُنزِل إليه في المخبأ كل أسبوع ما يكفيه، ولم يكن أحد يعلم بذلك إلا هي.»

\par 54 «وُرِدت القصة وتكررت وانتشرت في كل مكان عن مقتل حيقار الحكيم ووفاته، وحزن عليه جميع أهل تلك المدينة.»

\par 55 فبكوا وقالوا: ويل لك يا حيقار، وعلى علمك وأدبك! يا حسرة عليك وعلى علمك! أين يوجد مثلك؟ وأين يوجد رجل عاقل، عالم، ماهر في الحكم، يشبهك فيملأ مكانك؟

\par 56 «لكن الملك كان يتوب عن حيقار، ولم تنفعه توبته شيئًا».

\par 57 "ثم دعا نادان وقال له: اذهب وخذ معك أصحابك واعمل رثاءً وبكاءً على عمك حيقار، وأنوح عليه كما جرت العادة تكريماً لذكراه."

\par 58 «ولكن عندما ذهب نادان، الأحمق، الجاهل، قاسي القلب، إلى بيت عمه، لم يبك ولم يحزن ولم ينوح، بل جمع أناسًا قساة القلوب ومنحلين، وشرع يأكل ويشرب.» 1

\par 59 «وبدأ نادان في الاستيلاء على الجواري والعبيد التابعين لحيقار، وقيدهم وعذبهم وضربهم ضربًا مبرحًا.»

\par 60 «ولم يكن يحترم زوجة عمه التي ربته كابنها، بل أرادها أن تقع في الخطيئة معه.»

\par 61 «لكن حيقار كان قد حُشِر في المخبئ، فسمع بكاء عبيده وجيرانه، فحمد الله العلي الرحيم، وشكر، وكان دائمًا يصلي ويتضرع إلى الله العلي.»

\par 62 وكان السياف يأتي من حين لآخر إلى حيقار وهو في وسط المخبأ، فيأتي حيقار ويتوسل إليه، فيعزيه ويتمنى له النجاة

\par 63 «وعندما رُويت القصة في بلدان أخرى عن مقتل هيقار الحكيم، حزن جميع الملوك واحتقروا الملك سنحاريب، وندبوا على هيقار حل الألغاز.»

\par \textit{الحواشي}

\par \textit{207:1 قارن هذه الرواية عن احتفالات نادان وضربه للخدم مع متى 24: 48-51 ولوقا 12: 43-46. سترى أن لغة أخيقار قد أثرت على أحد أمثال ربنا.}

\chapter{4}

\par \textit{"ألغاز أبو الهول." ما حدث حقًا لأحيكار. عودته.}

\par 1 ولما تأكد ملك مصر من مقتل هيقار، قام على الفور وكتب رسالة إلى الملك سنحاريب، يذكره فيها "بالسلام والصحة والقوة والشرف الذي نتمناه لك بشكل خاص، يا أخي الحبيب الملك سنحاريب".

\par 2 «لقد كنت أرغب في بناء قلعة بين السماء والأرض، وأريد منك أن ترسل لي رجلاً حكيماً وذكياً من بين يديك ليبنيها لي، وليجيب على جميع أسئلتي، ولكي أحصل على ضرائب وجمارك آشور لمدة ثلاث سنوات.»

\par 3 ثم ختم الرسالة وأرسلها إلى سنحاريب.

\par 4 فأخذه وقرأه وأعطاه لوزرائه ولعظماء مملكته، فاضطربوا وخجلوا، وغضب غضباً عظيماً، واحتار كيف يتصرف.

\par 5 ثم جمع الشيوخ والعلماء والحكماء والفلاسفة والعرافين والمنجمين وكل من كان في بلده، وقرأ عليهم الرسالة وقال لهم: من منكم يذهب إلى فرعون ملك مصر ويجيبه على أسئلته؟

\par 6 فقالوا له: يا سيدنا الملك، اعلم أنه ليس في مملكتك من يعلم هذه الأمور إلا حيقار وزيرك وكاتبك.

\par 7 «أما نحن، فليس لدينا مهارة في هذا، إلا أن يكون نادان، ابن أخته، لأنه علمه كل حكمته وعلمه ومعرفته. ادعه إليك، لعله يفك هذه العقدة الصعبة.»

\par 8 ثم دعا الملك نادان وقال له: انظر إلى هذه الرسالة وافهم ما فيها. فلما قرأها نادان قال: يا سيدي! من يقدر على بناء قصر بين السماء والأرض؟

\par 9 ولما سمع الملك كلام نادان، حزن حزنًا شديدًا، ونزل عن عرشه وجلس على الرماد، وبدأ يبكي وينوح على حيقار

\par 10 قائلاً: «يا حزني! يا حيقار، يا من عرفت الأسرار والألغاز! ويل لي لك يا حيقار! يا معلم وطني وحاكم مملكتي، أين أجد مثلك؟ يا حيقار، يا معلم وطني، أين ألجأ إليك؟ ويل لي لك! كيف أهلكتك! واستمعت إلى حديث صبي غبي جاهل بلا علم، بلا دين، بلا رجولة».

\par 11 آه! وآه لي أيضًا! من يستطيع أن يعطيك لي ولو لمرة واحدة، أو يخبرني أن حيقار على قيد الحياة؟ وسأعطيه نصف مملكتي

\par 12 «من أين لي هذا؟ آه يا ​​حيقار! أن أراك ولو لمرة واحدة، وأن أشبع من النظر إليك، وأن أتلذذ بك.»

\par 13 «آه! يا حزني عليك إلى الأبد! يا حيقار، كيف قتلتك! ولم أتأخر في أمرك حتى رأيت نهاية الأمر.»

\par 14 وظل الملك يبكي ليلًا ونهارًا. فلما رأى السياف غضب الملك وحزنه على حيقار، رق قلبه تجاهه، واقترب منه وقال له:

\par 15 «يا سيدي! مُر عبيدك بقطع رأسي». ثم قال له الملك: «ويلك يا أبا سميك، ما ذنبك؟»

\par 16 فقال له السياف: يا سيدي! كل عبد يخالف قول سيده يُقتل، وأنا خالفت أمرك

\par 17 ثم قال له الملك: ويل لك يا أبا سميك، بماذا خالفت أمري؟

\par 18 فقال له السياف: يا سيدي! لقد أمرتني بقتل حيقار، وعلمت أنك ستتوب عنه، وأنه قد ظُلِم، فأخفيته في مكان ما، وقتلت أحد عبيده، وهو الآن آمن في البئر، وإن أمرتني لأحضرته إليك

\par 19 فقال له الملك: ويل لك يا أبا سميك! لقد سخرت مني وأنا سيدك

\par 20 فقال له السياف: لا، وحياة رأسك يا سيدي! حيقار حيّ وسالم

\par 21 فلما سمع الملك ذلك الكلام تأكد الأمر، ودار رأسه، وغشي عليه من الفرح، وأمر بإحضار حيقار.

\par 22 وقال للسياف: "أيها العبد الأمين! إذا كان كلامك صادقًا، فأنا أرغب في إثرائك، ورفع كرامتك فوق كرامة جميع أصدقائك."

\par 23 ومضى السياف فرحًا حتى وصل إلى بيت حيقار. ففتح باب المخبأ، ونزل فوجد حيقار جالسًا يحمد الله ويشكره

\par 24 ونادى عليه قائلًا: يا حيقار، إني أحمل إليك أعظم الفرح والسرور والبهجة!

\par 25 فقال له حيقار: ما الخبر يا أبا سميك؟ فأخبره عن فرعون من البداية إلى النهاية. ثم أخذه وذهب إلى الملك

\par 26 فلما نظر إليه الملك، رآه في حالة عوز، وأن شعره قد طال كشعر الوحوش، وأظافره كمخالب النسر، وأن جسده متسخ بالتراب، وأن لون وجهه قد تغير وباهت وأصبح الآن كالرماد

\par 27 فلما رآه الملك حزن عليه وقام من فوره واحتضنه وقبله وبكى عليه وقال: الحمد لله الذي أعادك إليّ

\par 28 ثم عزاه وواساه. وخلع رداءه، وألبسه السياف، وأكرمه كثيرًا، وأعطاه ثروةً جزيلةً، وأراح حيقار

\par 29 ثم قال حيقار للملك: "ليحي سيدي الملك إلى الأبد! هذه أعمال أبناء الدنيا. لقد رفعت لنفسي نخلة لأستند عليها، فانحنت جانبًا وطرحتني أرضًا."

\par 30 لكن يا سيدي! منذ أن ظهرت لك، لا يثقل عليك الهم! فقال له الملك: تبارك الله الذي رحمك، وعلم أنك مظلوم، وخلصك ونجاك من القتل

\par 31 «لكن اذهب إلى الحمام الدافئ، واحلق رأسك، وقلم أظافرك، وغير ملابسك، واستمتع بقضاء أربعين يومًا، حتى تتمكن من فعل الخير لنفسك وتحسن حالتك، ويعود إليك لون وجهك.»

\par 32 ثم خلع الملك رداءه الثمين، وألبسه حيقار، فشكر حيقار الله وسجد للملك، وانصرف إلى مسكنه سعيدًا وسعيدًا، يسبح الله العلي

\par 33 وفرح معه أهل بيته، وفرح أيضًا أصدقاؤه وكل من سمع أنه حي

\chapter{5}

\par \textit{يُعرض حرف "الألغاز" على أحيقار. الأولاد على النسور. أول رحلة "طائرة". في طريقهم إلى مصر. أحيقار، كونه رجلًا حكيمًا، لديه أيضًا حس فكاهة. (الآية 27).}

\par 1 ففعل كما أمره الملك، واستراح أربعين يومًا.

\par 2 ثم ارتدى أروع ثيابه، وذهب راكباً إلى الملك، وعبيده خلفه وأمامه، فرحين ومسرورين.

\par 3 ولكن عندما رأى نادان ابن أخته ما كان يحدث، أخذه الخوف والرعب، وارتبك، ولم يكن يعرف ماذا يفعل.

\par 4 فلما رأى حيقار ذلك دخل على الملك وسلم عليه، فرد عليه السلام، وأجلسه إلى جانبه، وقال له: يا حبيبي حيقار! انظر إلى هذه الرسائل التي أرسلها إلينا ملك مصر بعد أن سمع أنك قُتلت

\par 5 لقد استفزونا وتغلبوا علينا، وهرب كثير من أهل بلادنا إلى مصر خوفًا من الضرائب التي أرسلها ملك مصر ليطلبها منا

\par 6 ثم أخذ حيقار الرسالة وقرأها وفهم محتواها.

\par 7 ثم قال للملك: «لا تغضب يا سيدي! سأذهب إلى مصر، وسأرد الإجابات إلى فرعون، وسأريه هذه الرسالة، وسأجيبه بشأن الضرائب، وسأرد جميع الهاربين؛ وسأخزي أعداءك بمعونة الله العلي، ولسعادة مملكتك».

\par 8 ولما سمع الملك هذا الكلام من حيقار، فرح فرحًا عظيمًا، واتسع قلبه، وأظهر له عطفًا

\par 9 فقال حيقار للملك: أمهلني أربعين يومًا حتى أنظر في هذه المسألة وأتدبرها. فأذن الملك بذلك

\par 10 وذهب حيقار إلى مسكنه، وأمر الصيادين أن يصطادوا له فرخين من النسور، فأسروهما وأتوا بهما إليه، وأمر النساجين أن ينسجوا له حبلين من القطن، طول كل منهما ألفي ذراع، وأحضر النجارين وأمرهم أن يصنعوا صندوقين كبيرين، ففعلوا ذلك

\par 11 ثم أخذ غلامين صغيرين، وقضى كل يوم في ذبح الحملان وإطعام النسور والأولاد، وجعل الأولاد يركبون على ظهور النسور، وربطهما بعقدة محكمة، وربط الحبل بأقدام النسور، وتركهما يحلقان إلى الأعلى شيئًا فشيئًا كل يوم، لمسافة عشرة أذرع، حتى اعتادا ذلك وتعلماه؛ وارتفعا على طول الحبل حتى وصلا إلى السماء، والأولاد على ظهورهم. ثم جذبهما إليه

\par 12 ولما رأى حيقار أن أمنيته قد تحققت، أوصى الصبية أنه عندما يُرفعون إلى السماء، عليهم أن يهتفوا قائلين:

\par 13 «أتونا بالطين والحجارة لنبني قصرًا للملك فرعون، فنحن كسالى».

\par 14 ولم ينتهِ هيقار من تدريبهم وتمرينهم حتى وصلوا إلى أقصى درجة ممكنة (من المهارة).

\par 15 ثم تركهم وذهب إلى الملك وقال له: يا سيدي! لقد انتهى العمل حسب رغبتك. قم معي لأريك الأعجوبة

\par 16 فقام الملك وجلس مع حيقار، وذهب إلى مكان واسع، وأرسل لإحضار النسور والغلمان، فربطهم حيقار وأطلقهم في الهواء بطول الحبال، فجعلوا يصيحون كما علمهم، ثم جذبهم إليه ووضعهم في أماكنهم.

\par 17 فتعجب الملك ومن معه عجبًا عظيمًا، وقبل الملك حيقار بين عينيه وقال له: «اذهب بسلام يا حبيبي! يا فخر مملكتي! إلى مصر وأجب على أسئلة فرعون وانتصر عليه بقوة الله العلي».

\par 18 ثم ودعه، وأخذ جنوده وجيشه والفتيان والنسور، وانطلق نحو ديار مصر. ولما وصل، اتجه نحو بلاد الملك

\par 19 ولما علم أهل مصر أن سنحاريب قد أرسل رجلاً من مجلسه الخاص للتحدث مع فرعون والإجابة على أسئلته، نقلوا الخبر إلى الملك فرعون، فأرسل فريقاً من مستشاريه الخاصين لإحضاره أمامه

\par 20 فجاء ودخل أمام فرعون وسجد له كما يليق بالملوك

\par 21 فقال له: «يا سيدي الملك! يُحييك سنحاريب الملك بكثرة السلام والقوة والكرامة».

\par 22 «وقد أرسلني، أنا أحد عبيده، لأجيبك على أسئلتك، وأُحقق لك كل رغبتك: لأنك أرسلتَ لتطلب من سيدي الملك رجلاً يبني لك قصرًا بين السماء والأرض.»

\par 23 «وأنا، بمعونة الله العلي ورضاك الكريم وقوة سيدي الملك، سأبنيه لك كما تريد.»

\par 24 «لكن يا سيدي الملك! ما قلته فيه عن ضرائب مصر لمدة ثلاث سنوات - الآن استقرار المملكة هو العدل الصارم، وإذا فزت ولم تكن يدي بارعة في الرد عليك، فسيرسل لك سيدي الملك الضرائب التي ذكرتها.»

\par 25 «وإذا أجبتك على أسئلتك، فسيبقى عليك إرسال ما ذكرته إلى سيدي الملك.»

\par 26 فلما سمع فرعون ذلك الكلام تعجب وتحير من حرية لسانه ولطف كلامه

\par 27 فقال له الملك فرعون: يا رجل، ما اسمك؟ فقال: عبدك أبيقام، وأنا نملة صغيرة من نمل الملك سنحاريب

\par 28 فقال له فرعون: ألم يكن لدى ربك من هو أكرم منك حتى أرسل لي نملة صغيرة تجيبني وتكلمني؟

\par 29 فقال له حيقار: يا سيدي الملك، أرجو من الله العلي العظيم أن أُحقق لك ما في نفسك، فإن الله مع الضعيف ليُخزي القوي

\par 30 ثم أمر فرعون أن يُعدّوا مسكنًا لأبيقام، وأن يزوِّدوه بالعلف واللحم والشراب، وكل ما يحتاج إليه

\par 31 ولما كمل ذلك بعد ثلاثة أيام لبس فرعون الأرجوان والأحمر وجلس على كرسيه وجميع وزرائه وعظماء مملكته واقفون وأيديهم متصالبة وأقدامهم متقاربة ورؤوسهم منحنية.

\par 32 فأرسل فرعون ليحضر أبيقام، فلما قُدِّم إليه سجد له وقبل الأرض بين يديه

\par 33 فقال له الملك فرعون: يا أبيقام، من أشبه؟ وأشراف مملكتي، من يشبهون؟

\par 34 فقال له حيقار: يا سيدي، إنك مثل الصنم بعل، وأشراف مملكتك مثل عبيده

\par 35 قال له: اذهب وارجع إلى هنا غدًا. فذهب حيقار كما أمره الملك فرعون

\par 36 وفي الغد دخل حيقار إلى حضرة فرعون، وسجد، ووقف أمام الملك. وكان فرعون يرتدي ثوبًا أحمر، وكان النبلاء يرتدون ثيابًا بيضاء

\par 37 فقال له فرعون: يا أبيقام، من أشبه؟ وأشراف مملكتي، من يشبهون؟

\par 38 فقال له أبيقام: يا سيدي، أنت كالشمس، وعبيدك كأشعتها. فقال له فرعون: اذهب إلى مسكنك، وتعال إلى هنا غدًا

\par 39 ثم أمر فرعون حاشيته أن يلبسوا بياضًا نقيًا، ولبس فرعون مثلهم وجلس على عرشه، وأمرهم بإحضار حيقار. فدخل وجلس أمامه

\par 40 فقال له فرعون: يا أبيقام، من أشبه؟ وأشرافي، من يشبهون؟

\par 41 فقال له أبيقام: يا سيدي! أنت كالقمر، وعظماؤك كالكواكب والنجوم. فقال له فرعون: اذهب وكن هنا غدًا

\par 42 ثم أمر فرعون عبيده بارتداء ثياب مختلفة الألوان، فارتدى فرعون ثوبًا من القطيفة الحمراء، وجلس على عرشه، وأمرهم بإحضار أبيقام. فدخل وسجد أمامه

\par 43 وقال: يا أبيقام، من أشبه؟ وجنودي، من يشبهون؟ وقال: يا سيدي، أنت كشهر إبريل، وجنودك كأزهاره

\par 44 فلما سمع الملك ذلك فرح فرحًا عظيمًا وقال: يا أبيقام! أول مرة قارنتني بالصنم بعل، وشبهت نبلائي بعبيده

\par 45 «وفي المرة الثانية قارنتني بالشمس، ونبلائي بأشعة الشمس.»

\par 46 «وفي المرة الثالثة قارنتني بالقمر، ونبلائي بالكواكب والنجوم.»

\par 47 «والمرة الرابعة قارنتني بشهر أبريل، وشبهت نبلائي بأزهاره. والآن يا أبيقام، أخبرني يا سيدك الملك سنحاريب، من يشبه؟ ونبلاءه، بمن يشبهون؟»

\par 48 فصاح حيقار بصوت عظيم وقال: حاشا لي أن أذكر سيدي الملك وأنت جالس على عرشه، ولكن قم على قدميك لأخبرك من يشبه سيدي الملك ومن يشبه عظماؤه.

\par 49 فاستغرب فرعون من فصاحة لسانه وجرأته في الإجابة. فقام فرعون عن عرشه ووقف أمام حيقار، وقال له: أخبرني الآن لأرى من يشبه سيدك الملك، وعظمائه، ومن يشبهون

\par 50 فقال له حيقار: سيدي هو إله السماء، وعظماؤه البرق والرعد، وعندما يشاء تهب الرياح وينزل المطر

\par 51 «ويأمر الرعد فيبرق ويمطر، ويمسك الشمس فلا تعطي ضوءها، والقمر والنجوم فلا تدور.»

\par 52 «ويأمر العاصفة، فتهب ويسقط المطر ويدوس على أبريل ويدمر أزهاره وبيوته.»

\par 53 فلما سمع فرعون هذا الكلام، احتار في أمره وغضب غضبًا شديدًا، وقال له: يا رجل! قل لي الحقيقة، وأخبرني من أنت حقًا

\par 54 فأخبره الحقيقة. "أنا حيقار الكاتب، أعظم مستشاري الملك سنحاريب، وأنا وزيره وحاكم مملكته ومستشاره."

\par 55 فقال له: «لقد صدقت في هذا القول. لكننا سمعنا عن حيقار أن الملك سنحاريب قتله، ومع ذلك يبدو أنك حي وبصحة جيدة».

\par 56 فقال له حيقار: نعم، كان كذلك، ولكن الحمد لله الذي يعلم الغيب، فإن سيدي الملك أمر بقتلي، وصدق كلام الفاسقين، ولكن الرب أنقذني، وطوبى لمن توكل عليه

\par 57 فقال فرعون لحَيْقَر: اذهب وكن هنا غدًا، وقل لي كلمة لم أسمعها من عظمائي ولا من أهل مملكتي وبلادي

\chapter{6}

\par \textit{نجحت الحيلة. أجاب أحيقار على كل أسئلة فرعون. كان الصبيان على النسور هم ذروة اليوم. تم الكشف عن الذكاء، الذي نادرًا ما يوجد في الكتب المقدسة القديمة، في الآيات 34-45.}

\par 1 وذهب حيقار إلى مسكنه، وكتب رسالة، قال فيها:

\par 2 من سنحاريب ملك آشور ونينوى إلى فرعون ملك مصر

\par 3 السلام عليك يا أخي، وما نعلمك به هو أن الأخ يحتاج إلى أخيه، والملوك بعضهم إلى بعض، وأرجو منك أن تقرضني تسعمائة وزنة من الذهب، فأنا بحاجة إليها لإطعام بعض الجنود، حتى أنفقها عليهم. وبعد قليل سأرسلها إليك

\par 4 ثم طوى الرسالة، وقدمها في الغد إلى فرعون.

\par 5 فلما رأى ذلك، تحير وقال له: إني لم أسمع قط مثل هذا الكلام من أحد.

\par 6 فقال له حيقار: إن هذا دين عليك لسيدي الملك.

\par 7 فقبل فرعون ذلك، وقال: يا حيقار، مثلك من يكون أمينًا على خدمة الملوك

\par 8 «تبارك الله الذي أكملك بالحكمة وزينك بالفلسفة والمعرفة.»

\par 9 «والآن يا حيقار، بقي ما نريده منك، أن تبني قصرًا بين السماء والأرض.»

\par 10 ثم قال حيقار: «السمع طاعة. سأبني لك قصرًا حسب رغبتك واختيارك؛ ولكن يا سيدي، أنا أُعِدّ لنا الجير والحجر والطين والعمال، ولدي بناؤون ماهرون سيبنون لك كما تريد.»

\par 11 وهيأ الملك له كل ذلك، وذهبوا إلى مكان واسع، وجاء إليه حيقار وفتيانه، وأخذ معه النسور والفتيان، وذهب الملك وجميع عظمائه، واجتمعت المدينة كلها ليروا ما سيفعله حيقار

\par 12 ثم أطلق حيقار النسور من الصناديق، وربط الشباب على ظهورهم، وربط الحبال بأقدام النسور، وأطلقهم في الهواء. فحلقوا إلى أعلى حتى استقروا بين السماء والأرض

\par 13 وبدأ الصبية يصرخون قائلين: "أحضروا طوبًا، أحضروا طينًا، لنبني قلعة الملك، لأننا نقف مكتوفي الأيدي!"

\par 14 فدهش الجمع وتحيروا وتعجبوا. وتعجب الملك ونبلاؤه

\par 15 وبدأ حيقار وعبيده يضربون العمال، وصرخوا على جنود الملك، قائلين لهم: «أحضروا للعمال ما يريدون ولا تمنعوهم عن عملهم».

\par 16 فقال له الملك: أنت مجنون، من يستطيع أن يأتي بأي شيء إلى هذه المسافة؟

\par 17 فقال له حيقار: يا سيدي! كيف نبني قصرًا في الهواء؟ ولو كان سيدي الملك هنا لبنى عدة قلاع في يوم واحد

\par 18 فقال له فرعون: اذهب يا حيقار إلى مسكنك واسترح، فقد تركنا بناء القلعة، وتعال إليّ غدًا

\par 19 ثم ذهب حيقار إلى مسكنه، وفي الغد مثل أمام فرعون. فقال فرعون: يا حيقار، ما أخبار فرس سيدك؟ فإنه إذا صهل في بلاد آشور ونينوى، وسمعت أفراسنا صوته، ألقت صغارها

\par 20 ولما سمع حيقار هذا الكلام ذهب وأخذ قطة وربطها وبدأ يجلدها جلدًا شديدًا حتى سمع المصريون ذلك، فذهبوا وأخبروا الملك بذلك

\par 21 فأرسل فرعون ليحضر حيقار، وقال له: يا حيقار، لماذا تجلد هكذا وتضرب هذه البهيمة العجماء؟

\par 22 فقال له حيقار: يا سيدي الملك! حقًا لقد فعلت بي فعلًا قبيحًا، واستحقت هذا الضرب والجلد، لأن سيدي الملك سنحاريب أعطاني ديكًا جميلًا، وكان صوته قويًا وصادقًا، وكان يعرف ساعات النهار والليل

\par 23 فقامت القطة في تلك الليلة وقطعت رأسها وذهبت، وبسبب هذا الفعل ألحقتها بهذا الضرب.

\par 24 فقال له فرعون: يا حيقار، أرى من كل هذا أنك تشيخ في شيخوختك، فما بين مصر ونينوى ثمانية وستون فرسخًا، فكيف ذهبت هذه الليلة وقطعت رأس ديكك وعادت؟

\par 25 فقال له حيقار: يا سيدي! لو كانت المسافة بين مصر ونينوى كهذه، فكيف تسمع أفراسك عندما يصهل فرس سيدي الملك ويلقي صغاره؟ وكيف يصل صوت الفرس إلى مصر؟

\par 26 فلما سمع فرعون ذلك، علم أن حيقار قد أجاب عن أسئلته

\par 27 وقال فرعون يا حيقار أريد أن تصنع لي حبالاً من رمل البحر

\par 28 فقال له حيقار: يا سيدي الملك، مُرهم أن يُحضروا لي حبلًا من الخزانة لأصنع مثله

\par 29 ثم ذهب حيقار إلى خلف المنزل، وحفر ثقوبًا في شاطئ البحر الخشن، وأخذ حفنة من الرمل في يده، رمل البحر، وعندما أشرقت الشمس، واخترقت الثقوب، نشر الرمل في الشمس حتى أصبح كأنه منسوج كالحبال

\par 30 فقال حيقار: «مر عبيدك أن يأخذوا هذه الحبال، ومتى شئت نسجت لك مثلها».

\par 31 وقال فرعون يا حيقار إن عندنا هنا رحى وقد كسرت وأريد منك أن تخيطها

\par 32 ثم نظر إليه حيقار، فوجد حجرًا آخر.

\par 33 فقال لفرعون يا سيدي إني غريب وليس عندي آلة للخياطة.

\par 34 «لكنني أريدك أن تأمر صانعي الأحذية المخلصين لديك بقطع مثاقب من هذا الحجر، حتى أتمكن من خياطة حجر الرحى هذا.»

\par 35 فضحك فرعون وجميع عظمائه. وقال: تبارك الله العلي الذي أعطاك هذا العقل والمعرفة

\par 36 ولما رأى فرعون أن حيقار قد غلبه، ورد عليه جوابه، غضب من فوره، وأمرهم أن يجمعوا له جزية ثلاث سنين، وأن يأتوا بها إلى حيقار

\par 37 وخلع ثيابه وألبسها حيقار وجنوده وخدمه، وأعطاه نفقة رحلته

\par 38 فقال له: «اذهب بسلام، يا قوة سيده وفخر أطبائه! هل من السلاطين من يشبهك؟ بلّغ تحياتي لسيدك الملك سنحاريب، وقل له كيف أرسلنا إليه هدايا، لأن الملوك يرضون بالقليل».

\par 39 ثم قام حيقار وقبل يدي الملك فرعون وقبل الأرض أمامه، وتمنى له القوة والثبات والوفرة في خزينته، ​​وقال له: يا سيدي! أريد منك ألا يبقى أحد من أهل بلدنا في مصر

\par 40 فقام فرعون وأرسل رسلاً لينادوا في شوارع مصر أنه لا يبقى أحد من أهل آشور أو نينوى في أرض مصر، بل يذهبون مع حيقار

\par 41 ثم ذهب حيقار وودع الملك فرعون وسافر يطلب أرض أشور ونينوى، وكان له كنوز وأموال كثيرة.

\par 42 ولما وصل الخبر إلى الملك سنحاريب بقدوم حيقار، خرج للقائه وفرح به فرحًا عظيمًا واحتضنه وقبله وقال له: أهلًا بك في بيتك: يا قريبي! أخي حيقار، قوة مملكتي وفخر مملكتي

\par 43 «اطلب مني ما تريد، حتى لو كنت ترغب في نصف مملكتي ومن ممتلكاتي.»

\par 44 ثم قال له حيقار: يا سيدي الملك، عش إلى الأبد! يا سيدي الملك، أظهر معروفك لأبي سميك بدلًا مني، فإن حياتي كانت بين يدي الله وفي يده

\par 45 ثم قال سنحاريب الملك: "تشرفت بك يا حبيبي حيقار! سأجعل مقام أبي ساميك السياف أعلى من جميع مستشاري الخاصين ومفضلي."

\par 46 ثم بدأ الملك يسأله عن سيرته مع فرعون منذ وصوله الأول حتى خروجه من عنده، وكيف أجاب على جميع أسئلته، وكيف استلم منه الجزية، وبدلات الملابس والهدايا

\par 47 ففرح سنحاريب الملك فرحًا عظيمًا، وقال لحَيْقَار: «خذ ما شئت من هذه الجزية، فهي كلها في قبضة يدك».

\par 48 وقال حيقار: «ليعش الملك إلى الأبد! لا أريد سوى سلامة سيدي الملك واستمرار عظمته».

\par 49 يا سيدي! ماذا أفعل بالثروة وما شابهها؟ ولكن إن كنتَ راضيًا، فأعطني نادان، ابن أختي، لأكافئه على ما فعله بي، وأمنحني دمه وأبرئني منه

\par 50 فقال سنحاريب الملك: خذه، فقد أعطيتك إياه. فأخذ حيقار نادان، ابن أخته، وقيد يديه بسلاسل من حديد، واقتاده إلى مسكنه، ووضع قيدًا ثقيلًا على قدميه، وربطه بعقدة محكمة، وبعد أن ربطه هكذا ألقاه في غرفة مظلمة بجانب الملجأ، وجعل نبوحل حارسًا عليه ليعطيه رغيف خبز وقليلًا من الماء كل يوم

\chapter{7}

\par \textit{أمثال أحيقار التي يُكمل فيها تعليم أبناء أخيه. تشبيهات مذهلة. يُطلق أحيقار على الصبي أسماءً رائعة. وهنا تنتهي قصة أحيقار.}

\par 1 وكان كلما دخل حيقار أو خرج، يوبخ نادان، ابن أخته، قائلاً له بحكمة:

\par 2 يا نادان يا بني! لقد فعلت لك كل ما هو جيد ولطيف، وكافأني عليه بالقبيح والسيء وبالقتل

\par 3 يا بني! يُقال في الأمثال: من لا يسمع بأذنه، يُسمعه بقفا رقبته

\par 4 فقال نادان: «لماذا غضبت عليّ؟»

\par 5 فقال له حيقار: لأني ربيتك وعلمتك وأكرمتك وعظمتك وربيتك أحسن تربية وأجلستك مكاني لتكون وارثي في ​​الدنيا فعاملتني بالقتل ورددت علي بالهلاك.

\par 6 لكن الرب علم أني مظلوم، وأنقذني من الخداع الذي وضعته لي، لأن الرب يشفي المنكسري القلوب، ويمنع الحسود والمتكبرين

\par 7 يا بني! لقد كنتَ لي كالعقرب الذي إذا ضرب النحاس طعنه

\par 8 يا بني! أنت مثل الغزال الذي كان يأكل جذور الفوة، ويأكلني اليوم وغدًا سيختبئون في جذوري

\par 9 يا بني! لقد ذهبت إلى من رأى رفيقه عاريًا في برد الشتاء، فأخذ ماءً باردًا وسكبه عليه

\par 10 يا بني! لقد كنتَ لي كرجلٍ أخذ حجرًا، ورماه إلى السماء ليرجم به ربه. ولم يُصبه الحجر، ولم يصل إلى ارتفاعٍ كافٍ، بل أصبح سببًا للذنب والخطيئة

\par 11 يا بني! لو أنك كرمتني واحترمتني وأصغيت لكلامي لكنت وارثي وملكتَ على مملكتي

\par 12 يا بني! اعلم أنه لو كان ذيل الكلب أو الخنزير عشرة أذرع لما بلغ قيمة ذيل الحصان ولو كان كالحرير

\par 13 يا بني! ظننت أنك ستكون وارثي عند وفاتي؛ وأنت بحسدك ووقاحةك أردت قتلي. لكن الرب أنقذني من مكرِك

\par 14 يا بني! لقد كنت لي كفخ منصوب على المزبلة، فجاء عصفور فوجد الفخ منصوبًا. فقال العصفور للفخ: ماذا تفعل هنا؟ فقال الفخ: أنا أصلي هنا إلى الله

\par 15 وسألته القبرة أيضًا: "ما قطعة الخشب التي تحملها؟" فقالت الفخ: "إنها شجرة بلوط صغيرة أتكئ عليها وقت الصلاة."

\par 16 قالت القبرة: "وما هذا الشيء في فمك؟" قال الفخ: "هذا هو الخبز والطعام الذي أحمله لكل الجياع والفقراء الذين يقتربون مني."

\par 17 قالت القبرة: "والآن هل لي أن أتقدم وآكل، فأنا جائعة؟" فقالت له الفخ: "تقدم". واقتربت القبرة لتأكل

\par 18 لكن الفخ قفز وأمسك بالقبرة من رقبتها.

\par 19 فأجاب القبرة وقال للفخ: إذا كان هذا خبزك للجائع فإن الله لا يقبل صدقاتك وأعمالك الصالحة.

\par 20 وإن كان ذلك صيامك وصلاتك، فلا يقبل الله منك صيامك ولا صلاتك، ولا يتم الله عليك ما فيه خير

\par 21 يا بني! لقد كنتَ لي كأسدٍ صادق حمارًا، وظلَّ الحمار يمشي أمام الأسد لبعض الوقت؛ وفي يومٍ من الأيام انقضَّ الأسد على الحمار وأكله

\par 22 يا بني، لقد كنت بالنسبة لي مثل السوسة في القمح، فهي لا تفيد شيئًا، بل تفسد القمح و تقضمه.

\par 23 يا بني! لقد كنت كرجل زرع عشرة مكاييل قمح، وعندما حان وقت الحصاد، قام وحصدها، وجمعها، ودرسها، وتعب فيها حتى النهاية، فصار عشرة مكاييل، فقال له سيده: "يا لك من شيء كسول! لم تكبر ولم تتقلص."

\par 24 يا بني! لقد كنتَ لي كالحجل الذي أُلقي في الشبكة، ولم يستطع إنقاذ نفسه، لكنه نادت الحجل لتُلقيه معها في الشبكة

\par 25 يا بني! لقد كنتَ لي كالكلب الذي كان باردًا فدخل بيت الخزاف ليدفئ

\par 26 ولما صار دافئًا، بدأ ينبح عليهم، فطردوه وضربوه حتى لا يعضهم

\par 27 يا بني! لقد كنتَ لي كالخنزير الذي دخل الحمام الساخن مع أهل الفضل، وعندما خرج من الحمام الساخن، رأى حفرة قذرة فنزل وتمرغ فيها

\par 28 يا بني! لقد كنتَ لي كالتيس الذي انضم إلى رفاقه في طريقهم إلى الذبيحة، ولم يستطع إنقاذ نفسه

\par 29 يا بني! الكلب الذي لا يتغذى من صيده يصبح طعامًا للذباب

\par 30 يا بني! اليد التي لا تعمل ولا تحرث، والتي تكون جشعة وماكرة، ستُقطع من على كتفها

\par 31 يا بني! العين التي لا يُرى فيها النور، ستلتقطها الغربان وتقتلعها

\par 32 يا بني! لقد كنتَ لي كشجرةٍ يقطعون أغصانها، فقالت لهم: لو لم يكن شيءٌ مني بأيديكم لما قطعتموني

\par 33 يا بني! أنت مثل القط الذي قالوا له: دعك من السرقة حتى نصنع لك سلسلة من ذهب ونطعمك سكرًا ولوزًا

\par 34 فقالت: ما أنسى حرفة أبي وأمي

\par 35 يا بني! لقد كنت مثل الحية الراكبة على شجيرة شوك عندما كانت في وسط نهر، ورآها الذئب وقال: "شر على شر، ومن هو أكثر شرًا منهما فليُدبر كليهما."

\par 36 فقالت الحية للذئب: «الحملان والماعز والغنم التي أكلتها طوال حياتك، أتردها إلى آبائها ووالديها أم لا؟»

\par 37 قال الذئب: "لا". فقالت له الحية: "أعتقد أنك أسوأنا بعدي".

\par 38 يا بني! أطعمتك طعامًا طيبًا ولم تطعمني خبزًا يابسًا

\par 39 يا بني! لقد سقيتك ماءً مُحلىً وشرابًا جيدًا، ولم تسقني ماء البئر

\par 40 يا بني! لقد علمتك، وربيتك، وحفرت لي مخبأً وأخفيتني

\par 41 يا بني، لقد ربيتك أحسن تربية، وربيتك كشجرة أرز طويلة، ثم لويتني وثنيتني.

\par 42 يا بني! كان رجائي فيك أن تبني لي قصرًا حصينًا، لأختبئ فيه من أعدائي، فصرت لي كمن يدفن في عمق الأرض، لكن الرب رحمني وأنقذني من مكرِك

\par 43 يا بني! لقد تمنيتُ لك الخير، فكافأنيتَني بالشر والبغضاء، والآن أودُّ أن أقتلع عينيك، وأجعلك طعامًا للكلاب، وأقطع لسانك، وأقطع رأسك بحد السيف، وأجازيك على أفعالك البغيضة

\par 44 فلما سمع ندان هذا الكلام من عمه حيقار، قال: يا عم! عاملني بعلمك، واغفر لي ذنوبي، فمن ذا الذي أخطأ مثلي، أو من ذا الذي يغفر مثلك؟

\par 45 «اقبلني يا عمي! الآن سأخدم في منزلك، وأعتني بخيولك، وأكنس روث ماشيتك، وأرعى غنمك، لأني أنا الشرير وأنت البار: أنا المذنب وأنت الغفور.»

\par 46 فقال له حيقار: يا بني، أنت كالشجرة التي كانت بلا ثمر بجانب الماء، فأراد صاحبها أن يقطعها، فقالت له: انقلني إلى مكان آخر، فإن لم أثمر فاقطعني

\par 47 فقال لها سيدها: «أنتِ بجانب الماء ولم تُثمري، فكيف تُثمرين وأنتِ في مكان آخر؟»

\par 48 يا بني! شيخوخة النسر خير من شباب الغراب

\par 49 يا بني! قالوا للذئب: ابتعد عن الغنم لئلا يؤذيك غبارها. فقال الذئب: روث لبن الغنم جيد لعيني

\par 50 يا بني! أرسلوا الذئب إلى المدرسة ليتعلم القراءة، وقالوا له: "قل أ، ب". قال: "خروف وماعز في جرسي".

\par 51 يا بني! وضعوا الحمار على المائدة فسقط، وبدأ يتدحرج في التراب، وقال أحدهم: "دعه يتدحرج، فهذه طبيعته، لن يتغير."

\par 52 يا بني! لقد تأكد القول المأثور: "إذا أنجبت ولدًا، فادعه ابنك، وإذا ربّيت ولدًا، فادعه عبدك."

\par 53 يا بني! من يفعل الخير يُلاقي خيرًا، ومن يفعل الشر يُلاقي شرًا، لأن الرب يجازي الإنسان حسب مقدار عمله

\par 54 يا بني! ماذا أقول لك أكثر من هذه الأقوال؟ لأن الرب يعلم الخفيات، ويعلم الأسرار والخفايا

\par 55 «فيجازيك ويحكم بيني وبينك، ويجازيك حسب استحقاقك.»

\par 56 ولما سمع نادان ذلك الكلام من عمه حيقار، انتفخ على الفور وصار كمثانة منتفخة

\par 57 فانتفخت أعضاؤه وساقاه ورجلاه وجنبه وتمزق وانشق بطنه وتبعثرت أحشاؤه فمات وهلك.

\par 58 وكانت نهايته الهلاك، فذهب إلى الجحيم. لأن من يحفر حفرة لأخيه يسقط فيها، ومن ينصب فخاخًا يُؤخذ فيها

\par 59 هذا ما حدث وما وجدناه بشأن قصة حيقار، والحمد لله دائمًا وأبدًا. آمين، والسلام

\par 60 انتهى هذا السجل بعون الله تعالى! آمين، آمين، آمين

\par \textit{الحواشي}

\par \textit{218:1 قارن مثل الابن الضال في لوقا 15: 19.}


\end{document}