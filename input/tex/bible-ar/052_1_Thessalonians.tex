\begin{document}

\title{1تسالونيكي}


\chapter{1}

\par 1 بُولُسُ وَسِلْوَانُسُ وَتِيمُوثَاوُسُ، إِلَى كَنِيسَةِ التَّسَالُونِيكِيِّينَ، فِي اللهِ الآبِ وَالرَّبِّ يَسُوعَ الْمَسِيحِ. نِعْمَةٌ لَكُمْ وَسَلاَمٌ مِنَ اللهِ أَبِينَا وَالرَّبِّ يَسُوعَ الْمَسِيحِ.
\par 2 نَشْكُرُ اللهَ كُلَّ حِينٍ مِنْ جِهَةِ جَمِيعِكُمْ، ذَاكِرِينَ إِيَّاكُمْ فِي صَلَوَاتِنَا،
\par 3 مُتَذَكِّرِينَ بِلاَ انْقِطَاعٍ عَمَلَ إِيمَانِكُمْ، وَتَعَبَ مَحَبَّتِكُمْ، وَصَبْرَ رَجَائِكُمْ، رَبَّنَا يَسُوعَ الْمَسِيحَ، أَمَامَ اللهِ وَأَبِينَا.
\par 4 عَالِمِينَ أَيُّهَا الإِخْوَةُ الْمَحْبُوبُونَ مِنَ اللهِ اخْتِيَارَكُمْ،
\par 5 أَنَّ إِنْجِيلَنَا لَمْ يَصِرْ لَكُمْ بِالْكَلاَمِ فَقَطْ، بَلْ بِالْقُوَّةِ أَيْضاً، وَبِالرُّوحِ الْقُدُسِ، وَبِيَقِينٍ شَدِيدٍ، كَمَا تَعْرِفُونَ أَيَّ رِجَالٍ كُنَّا بَيْنَكُمْ مِنْ أَجْلِكُمْ.
\par 6 وَأَنْتُمْ صِرْتُمْ مُتَمَثِّلِينَ بِنَا وَبِالرَّبِّ، إِذْ قَبِلْتُمُ الْكَلِمَةَ فِي ضِيقٍ كَثِيرٍ، بِفَرَحِ الرُّوحِ الْقُدُسِ،
\par 7 حَتَّى صِرْتُمْ قُدْوَةً لِجَمِيعِ الَّذِينَ يُؤْمِنُونَ فِي مَكِدُونِيَّةَ وَفِي أَخَائِيَةَ.
\par 8 لأَنَّهُ مِنْ قِبَلِكُمْ قَدْ أُذِيعَتْ كَلِمَةُ الرَّبِّ، لَيْسَ فِي مَكِدُونِيَّةَ وَأَخَائِيَةَ فَقَطْ، بَلْ فِي كُلِّ مَكَانٍ أَيْضاً قَدْ ذَاعَ إِيمَانُكُمْ بِاللهِ، حَتَّى لَيْسَ لَنَا حَاجَةٌ أَنْ نَتَكَلَّمَ شَيْئاً.
\par 9 لأَنَّهُمْ هُمْ يُخْبِرُونَ عَنَّا أَيُّ دُخُولٍ كَانَ لَنَا إِلَيْكُمْ، وَكَيْفَ رَجَعْتُمْ إِلَى اللهِ مِنَ الأَوْثَانِ لِتَعْبُدُوا اللهَ الْحَيَّ الْحَقِيقِيَّ،
\par 10 وَتَنْتَظِرُوا ابْنَهُ مِنَ السَّمَاءِ، الَّذِي أَقَامَهُ مِنَ الأَمْوَاتِ، يَسُوعَ، الَّذِي يُنْقِذُنَا مِنَ الْغَضَبِ الآتِي.

\chapter{2}

\par 1 لأَنَّكُمْ أَنْتُمْ أَيُّهَا الإِخْوَةُ تَعْلَمُونَ دُخُولَنَا إِلَيْكُمْ أَنَّهُ لَمْ يَكُنْ بَاطِلاً،
\par 2 بَلْ بَعْدَ مَا تَأَلَّمْنَا قَبْلاً وَبُغِيَ عَلَيْنَا كَمَا تَعْلَمُونَ، فِي فِيلِبِّي، جَاهَرْنَا فِي إِلَهِنَا أَنْ نُكَلِّمَكُمْ بِإِنْجِيلِ اللهِ، فِي جِهَادٍ كَثِيرٍ.
\par 3 لأَنَّ وَعْظَنَا لَيْسَ عَنْ ضَلاَلٍ، وَلاَ عَنْ دَنَسٍ، وَلاَ بِمَكْرٍ،
\par 4 بَلْ كَمَا اسْتُحْسِنَّا مِنَ اللهِ أَنْ نُؤْتَمَنَ عَلَى الإِنْجِيلِ هَكَذَا نَتَكَلَّمُ، لاَ كَأَنَّنَا نُرْضِي النَّاسَ بَلِ اللهَ الَّذِي يَخْتَبِرُ قُلُوبَنَا.
\par 5 فَإِنَّنَا لَمْ نَكُنْ قَطُّ فِي كَلاَمِ تَمَلُّقٍ كَمَا تَعْلَمُونَ، وَلاَ فِي عِلَّةِ طَمَعٍ. اللهُ شَاهِدٌ.
\par 6 وَلاَ طَلَبْنَا مَجْداً مِنَ النَّاسِ، لاَ مِنْكُمْ وَلاَ مِنْ غَيْرِكُمْ مَعَ أَنَّنَا قَادِرُونَ أَنْ نَكُونَ فِي وَقَارٍ كَرُسُلِ الْمَسِيحِ.
\par 7 بَلْ كُنَّا مُتَرَفِّقِينَ فِي وَسَطِكُمْ كَمَا تُرَبِّي الْمُرْضِعَةُ أَوْلاَدَهَا،
\par 8 هَكَذَا إِذْ كُنَّا حَانِّينَ إِلَيْكُمْ كُنَّا نَرْضَى أَنْ نُعْطِيَكُمْ، لاَ إِنْجِيلَ اللهِ فَقَطْ بَلْ أَنْفُسَنَا أَيْضاً، لأَنَّكُمْ صِرْتُمْ مَحْبُوبِينَ إِلَيْنَا.
\par 9 فَإِنَّكُمْ تَذْكُرُونَ أَيُّهَا الإِخْوَةُ تَعَبَنَا وَكَدَّنَا، إِذْ كُنَّا نَكْرِزُ لَكُمْ بِإِنْجِيلِ اللهِ، وَنَحْنُ عَامِلُونَ لَيْلاً وَنَهَاراً كَيْ لاَ نُثَقِّلَ عَلَى أَحَدٍ مِنْكُمْ.
\par 10 أَنْتُمْ شُهُودٌ، وَاللهُ، كَيْفَ بِطَهَارَةٍ وَبِبِرٍّ وَبِلاَ لَوْمٍ كُنَّا بَيْنَكُمْ أَنْتُمُ الْمُؤْمِنِينَ.
\par 11 كَمَا تَعْلَمُونَ كَيْفَ كُنَّا نَعِظُ كُلَّ وَاحِدٍ مِنْكُمْ كَالأَبِ لأَوْلاَدِهِ، وَنُشَجِّعُكُمْ،
\par 12 وَنُشْهِدُكُمْ لِكَيْ تَسْلُكُوا كَمَا يَحِقُّ لِلَّهِ الَّذِي دَعَاكُمْ إِلَى مَلَكُوتِهِ وَمَجْدِهِ.
\par 13 مِنْ أَجْلِ ذَلِكَ نَحْنُ أَيْضاً نَشْكُرُ اللهَ بِلاَ انْقِطَاعٍ، لأَنَّكُمْ إِذْ تَسَلَّمْتُمْ مِنَّا كَلِمَةَ خَبَرٍ مِنَ اللهِ، قَبِلْتُمُوهَا لاَ كَكَلِمَةِ أُنَاسٍ، بَلْ كَمَا هِيَ بِالْحَقِيقَةِ كَكَلِمَةِ اللهِ، الَّتِي تَعْمَلُ أَيْضاً فِيكُمْ أَنْتُمُ الْمُؤْمِنِينَ.
\par 14 فَإِنَّكُمْ أَيُّهَا الإِخْوَةُ صِرْتُمْ مُتَمَثِّلِينَ بِكَنَائِسِ اللهِ الَّتِي هِيَ فِي الْيَهُودِيَّةِ فِي الْمَسِيحِ يَسُوعَ، لأَنَّكُمْ تَأَلَّمْتُمْ أَنْتُمْ أَيْضاً مِنْ أَهْلِ عَشِيرَتِكُمْ تِلْكَ الآلاَمَ عَيْنَهَا كَمَا هُمْ أَيْضاً مِنَ الْيَهُودِ،
\par 15 الَّذِينَ قَتَلُوا الرَّبَّ يَسُوعَ وَأَنْبِيَاءَهُمْ، وَاضْطَهَدُونَا نَحْنُ. وَهُمْ غَيْرُ مُرْضِينَ لِلَّهِ وَأَضْدَادٌ لِجَمِيعِ النَّاسِ
\par 16 يَمْنَعُونَنَا عَنْ أَنْ نُكَلِّمَ الأُمَمَ لِكَيْ يَخْلُصُوا حَتَّى يُتَمِّمُوا خَطَايَاهُمْ كُلَّ حِينٍ. وَلَكِنْ قَدْ أَدْرَكَهُمُ الْغَضَبُ إِلَى النِّهَايَةِ.
\par 17 وَأَمَّا نَحْنُ أَيُّهَا الإِخْوَةُ، فَإِذْ قَدْ فَقَدْنَاكُمْ زَمَانَ سَاعَةٍ، بِالْوَجْهِ لاَ بِالْقَلْبِ، اجْتَهَدْنَا أَكْثَرَ بِاشْتِهَاءٍ كَثِيرٍ أَنْ نَرَى وُجُوهَكُمْ.
\par 18 لِذَلِكَ أَرَدْنَا أَنْ نَأْتِيَ إِلَيْكُمْ أَنَا بُولُسَ مَرَّةً وَمَرَّتَيْنِ. وَإِنَّمَا عَاقَنَا الشَّيْطَانُ.
\par 19 لأَنْ مَنْ هُوَ رَجَاؤُنَا وَفَرَحُنَا وَإِكْلِيلُ افْتِخَارِنَا؟ أَمْ لَسْتُمْ أَنْتُمْ أَيْضاً أَمَامَ رَبِّنَا يَسُوعَ الْمَسِيحِ فِي مَجِيئِهِ؟
\par 20 لأَنَّكُمْ أَنْتُمْ مَجْدُنَا وَفَرَحُنَا.

\chapter{3}

\par 1 لِذَلِكَ إِذْ لَمْ نَحْتَمِلْ أَيْضاً اسْتَحْسَنَّا أَنْ نُتْرَكَ فِي أَثِينَا وَحْدَنَا.
\par 2 فَأَرْسَلْنَا تِيمُوثَاوُسَ أَخَانَا، وَخَادِمَ اللهِ، وَالْعَامِلَ مَعَنَا فِي إِنْجِيلِ الْمَسِيحِ، حَتَّى يُثَبِّتَكُمْ وَيَعِظَكُمْ لأَجْلِ إِيمَانِكُمْ،
\par 3 كَيْ لاَ يَتَزَعْزَعَ أَحَدٌ فِي هَذِهِ الضِّيقَاتِ. فَإِنَّكُمْ أَنْتُمْ تَعْلَمُونَ أَنَّنَا مَوْضُوعُونَ لِهَذَا.
\par 4 لأَنَّنَا لَمَّا كُنَّا عِنْدَكُمْ سَبَقْنَا فَقُلْنَا لَكُمْ: إِنَّنَا عَتِيدُونَ أَنْ نَتَضَايَقَ، كَمَا حَصَلَ أَيْضاً، وَأَنْتُمْ تَعْلَمُونَ.
\par 5 مِنْ أَجْلِ هَذَا إِذْ لَمْ أَحْتَمِلْ أَيْضاً، أَرْسَلْتُ لِكَيْ أَعْرِفَ إِيمَانَكُمْ، لَعَلَّ الْمُجَرِّبَ يَكُونُ قَدْ جَرَّبَكُمْ، فَيَصِيرَ تَعَبُنَا بَاطِلاً.
\par 6 وَأَمَّا الآنَ فَإِذْ جَاءَ إِلَيْنَا تِيمُوثَاوُسُ مِنْ عِنْدِكُمْ، وَبَشَّرَنَا بِإِيمَانِكُمْ وَمَحَبَّتِكُمْ، وَبِأَنَّ عِنْدَكُمْ ذِكْراً لَنَا حَسَناً كُلَّ حِينٍ، وَأَنْتُمْ مُشْتَاقُونَ أَنْ تَرَوْنَا، كَمَا نَحْنُ أَيْضاً أَنْ نَرَاكُمْ،
\par 7 فَمِنْ أَجْلِ هَذَا تَعَزَّيْنَا أَيُّهَا الإِخْوَةُ مِنْ جِهَتِكُمْ فِي ضِيقَتِنَا وَضَرُورَتِنَا بِإِيمَانِكُمْ.
\par 8 لأَنَّنَا الآنَ نَعِيشُ إِنْ ثَبَتُّمْ أَنْتُمْ فِي الرَّبِّ.
\par 9 لأَنَّهُ أَيَّ شُكْرٍ نَسْتَطِيعُ أَنْ نُعَوِّضَ إِلَى اللهِ مِنْ جِهَتِكُمْ عَنْ كُلِّ الْفَرَحِ الَّذِي نَفْرَحُ بِهِ مِنْ أَجْلِكُمْ قُدَّامَ إِلَهِنَا؟
\par 10 طَالِبِينَ لَيْلاً وَنَهَاراً أَوْفَرَ طَلَبٍ أَنْ نَرَى وُجُوهَكُمْ، وَنُكَمِّلَ نَقَائِصَ إِيمَانِكُمْ.
\par 11 وَاللهُ نَفْسُهُ أَبُونَا وَرَبُّنَا يَسُوعُ الْمَسِيحُ يَهْدِي طَرِيقَنَا إِلَيْكُمْ.
\par 12 وَالرَّبُّ يُنْمِيكُمْ وَيَزِيدُكُمْ فِي الْمَحَبَّةِ بَعْضَكُمْ لِبَعْضٍ وَلِلْجَمِيعِ، كَمَا نَحْنُ أَيْضاً لَكُمْ،
\par 13 لِكَيْ يُثَبِّتَ قُلُوبَكُمْ بِلاَ لَوْمٍ فِي الْقَدَاسَةِ، أَمَامَ اللهِ أَبِينَا فِي مَجِيءِ رَبِّنَا يَسُوعَ الْمَسِيحِ مَعَ جَمِيعِ قِدِّيسِيهِ.

\chapter{4}

\par 1 فَمِنْ ثَمَّ أَيُّهَا الإِخْوَةُ نَسْأَلُكُمْ وَنَطْلُبُ إِلَيْكُمْ فِي الرَّبِّ يَسُوعَ، أَنَّكُمْ كَمَا تَسَلَّمْتُمْ مِنَّا كَيْفَ يَجِبُ أَنْ تَسْلُكُوا وَتُرْضُوا اللهَ، تَزْدَادُونَ أَكْثَرَ.
\par 2 لأَنَّكُمْ تَعْلَمُونَ أَيَّةَ وَصَايَا أَعْطَيْنَاكُمْ بِالرَّبِّ يَسُوعَ.
\par 3 لأَنَّ هَذِهِ هِيَ إِرَادَةُ اللهِ: قَدَاسَتُكُمْ. أَنْ تَمْتَنِعُوا عَنِ الزِّنَا،
\par 4 أَنْ يَعْرِفَ كُلُّ وَاحِدٍ مِنْكُمْ أَنْ يَقْتَنِيَ إِنَاءَهُ بِقَدَاسَةٍ وَكَرَامَةٍ،
\par 5 لاَ فِي هَوَى شَهْوَةٍ كَالأُمَمِ الَّذِينَ لاَ يَعْرِفُونَ اللهَ.
\par 6 أَنْ لاَ يَتَطَاوَلَ أَحَدٌ وَيَطْمَعَ عَلَى أَخِيهِ فِي هَذَا الأَمْرِ، لأَنَّ الرَّبَّ مُنْتَقِمٌ لِهَذِهِ كُلِّهَا كَمَا قُلْنَا لَكُمْ قَبْلاً وَشَهِدْنَا.
\par 7 لأَنَّ اللهَ لَمْ يَدْعُنَا لِلنَّجَاسَةِ بَلْ فِي الْقَدَاسَةِ.
\par 8 إِذاً مَنْ يَرْذُلُ لاَ يَرْذُلُ إِنْسَاناً، بَلِ اللهَ الَّذِي أَعْطَانَا أَيْضاً رُوحَهُ الْقُدُّوسَ.
\par 9 وَأَمَّا الْمَحَبَّةُ الأَخَوِيَّةُ فَلاَ حَاجَةَ لَكُمْ أَنْ أَكْتُبَ إِلَيْكُمْ عَنْهَا، لأَنَّكُمْ أَنْفُسَكُمْ مُتَعَلِّمُونَ مِنَ اللهِ أَنْ يُحِبَّ بَعْضُكُمْ بَعْضاً.
\par 10 فَإِنَّكُمْ تَفْعَلُونَ ذَلِكَ أَيْضاً لِجَمِيعِ الإِخْوَةِ الَّذِينَ فِي مَكِدُونِيَّةَ كُلِّهَا. وَإِنَّمَا أَطْلُبُ إِلَيْكُمْ أَيُّهَا الإِخْوَةُ أَنْ تَزْدَادُوا أَكْثَرَ،
\par 11 وَأَنْ تَحْرِصُوا عَلَى أَنْ تَكُونُوا هَادِئِينَ، وَتُمَارِسُوا أُمُورَكُمُ الْخَاصَّةَ، وَتَشْتَغِلُوا بِأَيْدِيكُمْ أَنْتُمْ كَمَا أَوْصَيْنَاكُمْ،
\par 12 لِكَيْ تَسْلُكُوا بِلِيَاقَةٍ عِنْدَ الَّذِينَ هُمْ مِنْ خَارِجٍ، وَلاَ تَكُونَ لَكُمْ حَاجَةٌ إِلَى أَحَدٍ.
\par 13 ثُمَّ لاَ أُرِيدُ أَنْ تَجْهَلُوا أَيُّهَا الإِخْوَةُ مِنْ جِهَةِ الرَّاقِدِينَ، لِكَيْ لاَ تَحْزَنُوا كَالْبَاقِينَ الَّذِينَ لاَ رَجَاءَ لَهُمْ.
\par 14 لأَنَّهُ إِنْ كُنَّا نُؤْمِنُ أَنَّ يَسُوعَ مَاتَ وَقَامَ، فَكَذَلِكَ الرَّاقِدُونَ بِيَسُوعَ سَيُحْضِرُهُمُ اللهُ أَيْضاً مَعَهُ.
\par 15 فَإِنَّنَا نَقُولُ لَكُمْ هَذَا بِكَلِمَةِ الرَّبِّ: إِنَّنَا نَحْنُ الأَحْيَاءَ الْبَاقِينَ إِلَى مَجِيءِ الرَّبِّ لاَ نَسْبِقُ الرَّاقِدِينَ.
\par 16 لأَنَّ الرَّبَّ نَفْسَهُ سَوْفَ يَنْزِلُ مِنَ السَّمَاءِ بِهُتَافٍ، بِصَوْتِ رَئِيسِ مَلاَئِكَةٍ وَبُوقِ اللهِ، وَالأَمْوَاتُ فِي الْمَسِيحِ سَيَقُومُونَ أَوَّلاً.
\par 17 ثُمَّ نَحْنُ الأَحْيَاءَ الْبَاقِينَ سَنُخْطَفُ جَمِيعاً مَعَهُمْ فِي السُّحُبِ لِمُلاَقَاةِ الرَّبِّ فِي الْهَوَاءِ، وَهَكَذَا نَكُونُ كُلَّ حِينٍ مَعَ الرَّبِّ.
\par 18 لِذَلِكَ عَزُّوا بَعْضُكُمْ بَعْضاً بِهَذَا الْكَلاَمِ.

\chapter{5}

\par 1 وَأَمَّا الأَزْمِنَةُ وَالأَوْقَاتُ فَلاَ حَاجَةَ لَكُمْ أَيُّهَا الإِخْوَةُ أَنْ أَكْتُبَ إِلَيْكُمْ عَنْهَا،
\par 2 لأَنَّكُمْ أَنْتُمْ تَعْلَمُونَ بِالتَّحْقِيقِ أَنَّ يَوْمَ الرَّبِّ كَلِصٍّ فِي اللَّيْلِ هَكَذَا يَجِيءُ.
\par 3 لأَنَّهُ حِينَمَا يَقُولُونَ: «سَلاَمٌ وَأَمَانٌ» حِينَئِذٍ يُفَاجِئُهُمْ هَلاَكٌ بَغْتَةً، كَالْمَخَاضِ لِلْحُبْلَى، فَلاَ يَنْجُونَ.
\par 4 وَأَمَّا أَنْتُمْ أَيُّهَا الإِخْوَةُ فَلَسْتُمْ فِي ظُلْمَةٍ حَتَّى يُدْرِكَكُمْ ذَلِكَ الْيَوْمُ كَلَِصٍّ.
\par 5 جَمِيعُكُمْ أَبْنَاءُ نُورٍ وَأَبْنَاءُ نَهَارٍ. لَسْنَا مِنْ لَيْلٍ وَلاَ ظُلْمَةٍ.
\par 6 فَلاَ نَنَمْ إِذاً كَالْبَاقِينَ، بَلْ لِنَسْهَرْ وَنَصْحُ،
\par 7 لأَنَّ الَّذِينَ يَنَامُونَ فَبِاللَّيْلِ يَنَامُونَ، وَالَّذِينَ يَسْكَرُونَ فَبِاللَّيْلِ يَسْكَرُونَ.
\par 8 وَأَمَّا نَحْنُ الَّذِينَ مِنْ نَهَارٍ، فَلْنَصْحُ لاَبِسِينَ دِرْعَ الإِيمَانِ وَالْمَحَبَّةِ، وَخُوذَةً هِيَ رَجَاءُ الْخَلاَصِ.
\par 9 لأَنَّ اللهَ لَمْ يَجْعَلْنَا لِلْغَضَبِ، بَلْ لاِقْتِنَاءِ الْخَلاَصِ بِرَبِّنَا يَسُوعَ الْمَسِيحِ،
\par 10 الَّذِي مَاتَ لأَجْلِنَا، حَتَّى إِذَا سَهِرْنَا أَوْ نِمْنَا نَحْيَا جَمِيعاً مَعَهُ.
\par 11 لِذَلِكَ عَزُّوا بَعْضُكُمْ بَعْضاً وَابْنُوا أَحَدُكُمُ الآخَرَ، كَمَا تَفْعَلُونَ أَيْضاً.
\par 12 ثُمَّ نَسْأَلُكُمْ أَيُّهَا الإِخْوَةُ أَنْ تَعْرِفُوا الَّذِينَ يَتْعَبُونَ بَيْنَكُمْ وَيُدَبِّرُونَكُمْ فِي الرَّبِّ وَيُنْذِرُونَكُمْ،
\par 13 وَأَنْ تَعْتَبِرُوهُمْ كَثِيراً جِدّاً فِي الْمَحَبَّةِ مِنْ أَجْلِ عَمَلِهِمْ. سَالِمُوا بَعْضُكُمْ بَعْضاً.
\par 14 وَنَطْلُبُ إِلَيْكُمْ أَيُّهَا الإِخْوَةُ: أَنْذِرُوا الَّذِينَ بِلاَ تَرْتِيبٍ. شَجِّعُوا صِغَارَ النُّفُوسِ، أَسْنِدُوا الضُّعَفَاءَ. تَأَنَّوْا عَلَى الْجَمِيعِ.
\par 15 انْظُرُوا أَنْ لاَ يُجَازِيَ أَحَدٌ أَحَداً عَنْ شَرٍّ بِشَرٍّ، بَلْ كُلَّ حِينٍ اتَّبِعُوا الْخَيْرَ بَعْضُكُمْ لِبَعْضٍ وَلِلْجَمِيعِ.
\par 16 اِفْرَحُوا كُلَّ حِينٍ.
\par 17 صَلُّوا بِلاَ انْقِطَاعٍ.
\par 18 اشْكُرُوا فِي كُلِّ شَيْءٍ، لأَنَّ هَذِهِ هِيَ مَشِيئَةُ اللهِ فِي الْمَسِيحِ يَسُوعَ مِنْ جِهَتِكُمْ.
\par 19 لاَ تُطْفِئُوا الرُّوحَ.
\par 20 لاَ تَحْتَقِرُوا النُّبُوَّاتِ.
\par 21 امْتَحِنُوا كُلَّ شَيْءٍ. تَمَسَّكُوا بِالْحَسَنِ.
\par 22 اِمْتَنِعُوا عَنْ كُلِّ شِبْهِ شَرٍّ.
\par 23 وَإِلَهُ السَّلاَمِ نَفْسُهُ يُقَدِّسُكُمْ بِالتَّمَامِ. وَلْتُحْفَظْ رُوحُكُمْ وَنَفْسُكُمْ وَجَسَدُكُمْ كَامِلَةً بِلاَ لَوْمٍ عِنْدَ مَجِيءِ رَبِّنَا يَسُوعَ الْمَسِيحِ.
\par 24 أَمِينٌ هُوَ الَّذِي يَدْعُوكُمُ الَّذِي سَيَفْعَلُ أَيْضاً.
\par 25 أَيُّهَا الإِخْوَةُ صَلُّوا لأَجْلِنَا.
\par 26 سَلِّمُوا عَلَى الإِخْوَةِ جَمِيعاً بِقُبْلَةٍ مُقَدَّسَةٍ.
\par 27 أُنَاشِدُكُمْ بِالرَّبِّ أَنْ تُقْرَأَ هَذِهِ الرِّسَالَةُ عَلَى جَمِيعِ الإِخْوَةِ الْقِدِّيسِينَ.
\par 28 نِعْمَةُ رَبِّنَا يَسُوعَ الْمَسِيحِ مَعَكُمْ. آمِينَ.

\end{document}