\begin{document}

\title{متى}


\chapter{1}

\par 1 كِتَابُ مِيلاَدِ يَسُوعَ الْمَسِيحِ ابْنِ دَاوُدَ ابْنِ إِبْراهِيمَ.
\par 2 إِبْراهِيمُ وَلَدَ إِسْحاقَ. وَإِسْحاقُ وَلَدَ يَعْقُوبَ. وَيَعْقُوبُ وَلَدَ يَهُوذَا وَإِخْوَتَهُ.
\par 3 وَيَهُوذَا وَلَدَ فَارِصَ وَزَارَحَ مِنْ ثَامَارَ. وَفَارِصُ وَلَدَ حَصْرُونَ. وَحَصْرُونُ وَلَدَ أَرَامَ.
\par 4 وَأَرَامُ وَلَدَ عَمِّينَادَابَ. وَعَمِّينَادَابُ وَلَدَ نَحْشُونَ. وَنَحْشُونُ وَلَدَ سَلْمُونَ.
\par 5 وَسَلْمُونُ وَلَدَ بُوعَزَ مِنْ رَاحَابَ. وَبُوعَزُ وَلَدَ عُوبِيدَ مِنْ رَاعُوثَ. وَعُوبِيدُ وَلَدَ يَسَّى.
\par 6 وَيَسَّى وَلَدَ دَاوُدَ الْمَلِكَ. وَدَاوُدُ الْمَلِكُ وَلَدَ سُلَيْمَانَ مِنَ الَّتِي لأُورِيَّا.
\par 7 وَسُلَيْمَانُ وَلَدَ رَحَبْعَامَ. وَرَحَبْعَامُ وَلَدَ أَبِيَّا. وَأَبِيَّا وَلَدَ آسَا.
\par 8 وَآسَا وَلَدَ يَهُوشَافَاطَ. وَيَهُوشَافَاطُ وَلَدَ يُورَامَ. وَيُورَامُ وَلَدَ عُزِّيَّا.
\par 9 وَعُزِّيَّا وَلَدَ يُوثَامَ. وَيُوثَامُ وَلَدَ أَحَازَ. وَأَحَازُ وَلَدَ حَزَقِيَّا.
\par 10 وَحَزَقِيَّا وَلَدَ مَنَسَّى. وَمَنَسَّى وَلَدَ آمُونَ. وَآمُونُ وَلَدَ يُوشِيَّا.
\par 11 وَيُوشِيَّا وَلَدَ يَكُنْيَا وَإِخْوَتَهُ عِنْدَ سَبْيِ بَابِلَ.
\par 12 وَبَعْدَ سَبْيِ بَابِلَ يَكُنْيَا وَلَدَ شَأَلْتِئِيلَ. وَشَأَلْتِئِيلُ وَلَدَ زَرُبَّابِلَ.
\par 13 وَزَرُبَّابِلُ وَلَدَ أَبِيهُودَ. وَأَبِيهُودُ وَلَدَ أَلِيَاقِيمَ. وَأَلِيَاقِيمُ وَلَدَ عَازُورَ.
\par 14 وَعَازُورُ وَلَدَ صَادُوقَ. وَصَادُوقُ وَلَدَ أَخِيمَ. وَأَخِيمُ وَلَدَ أَلِيُودَ.
\par 15 وَأَلِيُودُ وَلَدَ أَلِيعَازَرَ. وَأَلِيعَازَرُ وَلَدَ مَتَّانَ. وَمَتَّانُ وَلَدَ يَعْقُوبَ.
\par 16 وَيَعْقُوبُ وَلَدَ يُوسُفَ رَجُلَ مَرْيَمَ الَّتِي وُلِدَ مِنْهَا يَسُوعُ الَّذِي يُدْعَى الْمَسِيحَ.
\par 17 فَجَمِيعُ الأَجْيَالِ مِنْ إِبْراهِيمَ إِلَى دَاوُدَ أَرْبَعَةَ عَشَرَ جِيلاً وَمِنْ دَاوُدَ إِلَى سَبْيِ بَابِلَ أَرْبَعَةَ عَشَرَ جِيلاً وَمِنْ سَبْيِ بَابِلَ إِلَى الْمَسِيحِ أَرْبَعَةَ عَشَرَ جِيلاً.
\par 18 أَمَّا وِلاَدَةُ يَسُوعَ الْمَسِيحِ فَكَانَتْ هَكَذَا: لَمَّا كَانَتْ مَرْيَمُ أُمُّهُ مَخْطُوبَةً لِيُوسُفَ قَبْلَ أَنْ يَجْتَمِعَا وُجِدَتْ حُبْلَى مِنَ الرُّوحِ الْقُدُسِ.
\par 19 فَيُوسُفُ رَجُلُهَا إِذْ كَانَ بَارّاً وَلَمْ يَشَأْ أَنْ يُشْهِرَهَا أَرَادَ تَخْلِيَتَهَا سِرّاً.
\par 20 وَلَكِنْ فِيمَا هُوَ مُتَفَكِّرٌ فِي هَذِهِ الأُمُورِ إِذَا مَلاَكُ الرَّبِّ قَدْ ظَهَرَ لَهُ فِي حُلْمٍ قَائِلاً: «يَا يُوسُفُ ابْنَ دَاوُدَ لاَ تَخَفْ أَنْ تَأْخُذَ مَرْيَمَ امْرَأَتَكَ لأَنَّ الَّذِي حُبِلَ بِهِ فِيهَا هُوَ مِنَ الرُّوحِ الْقُدُسِ.
\par 21 فَسَتَلِدُ ابْناً وَتَدْعُو اسْمَهُ يَسُوعَ لأَنَّهُ يُخَلِّصُ شَعْبَهُ مِنْ خَطَايَاهُمْ».
\par 22 وَهَذَا كُلُّهُ كَانَ لِكَيْ يَتِمَّ مَا قِيلَ مِنَ الرَّبِّ بِالنَّبِيِّ:
\par 23 «هُوَذَا الْعَذْرَاءُ تَحْبَلُ وَتَلِدُ ابْناً وَيَدْعُونَ اسْمَهُ عِمَّانُوئِيلَ» (الَّذِي تَفْسِيرُهُ: اَللَّهُ مَعَنَا).
\par 24 فَلَمَّا اسْتَيْقَظَ يُوسُفُ مِنَ النَّوْمِ فَعَلَ كَمَا أَمَرَهُ مَلاَكُ الرَّبِّ وَأَخَذَ امْرَأَتَهُ.
\par 25 وَلَمْ يَعْرِفْهَا حَتَّى وَلَدَتِ ابْنَهَا الْبِكْرَ. وَدَعَا اسْمَهُ يَسُوعَ.

\chapter{2}

\par 1 وَلَمَّا وُلِدَ يَسُوعُ فِي بَيْتِ لَحْمِ الْيَهُودِيَّةِ فِي أَيَّامِ هِيرُودُسَ الْمَلِكِ إِذَا مَجُوسٌ مِنَ الْمَشْرِقِ قَدْ جَاءُوا إِلَى أُورُشَلِيمَ
\par 2 قَائِلِينَ: «أَيْنَ هُوَ الْمَوْلُودُ مَلِكُ الْيَهُودِ؟ فَإِنَّنَا رَأَيْنَا نَجْمَهُ فِي الْمَشْرِقِ وَأَتَيْنَا لِنَسْجُدَ لَهُ».
\par 3 فَلَمَّا سَمِعَ هِيرُودُسُ الْمَلِكُ اضْطَرَبَ وَجَمِيعُ أُورُشَلِيمَ مَعَهُ.
\par 4 فَجَمَعَ كُلَّ رُؤَسَاءِ الْكَهَنَةِ وَكَتَبَةِ الشَّعْبِ وَسَأَلَهُمْ: «أَيْنَ يُولَدُ الْمَسِيحُ؟»
\par 5 فَقَالُوا لَهُ: «فِي بَيْتِ لَحْمِ الْيَهُودِيَّةِ لأَنَّهُ هَكَذَا مَكْتُوبٌ بِالنَّبِيِّ:
\par 6 وَأَنْتِ يَا بَيْتَ لَحْمٍ أَرْضَ يَهُوذَا لَسْتِ الصُّغْرَى بَيْنَ رُؤَسَاءِ يَهُوذَا لأَنْ مِنْكِ يَخْرُجُ مُدَبِّرٌ يَرْعَى شَعْبِي إِسْرَائِيلَ».
\par 7 حِينَئِذٍ دَعَا هِيرُودُسُ الْمَجُوسَ سِرّاً وَتَحَقَّقَ مِنْهُمْ زَمَانَ النَّجْمِ الَّذِي ظَهَرَ.
\par 8 ثُمَّ أَرْسَلَهُمْ إِلَى بَيْتِ لَحْمٍ وَقَالَ: «اذْهَبُوا وَافْحَصُوا بِالتَّدْقِيقِ عَنِ الصَّبِيِّ وَمَتَى وَجَدْتُمُوهُ فَأَخْبِرُونِي لِكَيْ آتِيَ أَنَا أَيْضاً وَأَسْجُدَ لَهُ».
\par 9 فَلَمَّا سَمِعُوا مِنَ الْمَلِكِ ذَهَبُوا. وَإِذَا النَّجْمُ الَّذِي رَأَوْهُ فِي الْمَشْرِقِ يَتَقَدَّمُهُمْ حَتَّى جَاءَ وَوَقَفَ فَوْقُ حَيْثُ كَانَ الصَّبِيُّ.
\par 10 فَلَمَّا رَأَوُا النَّجْمَ فَرِحُوا فَرَحاً عَظِيماً جِدّاً
\par 11 وَأَتَوْا إِلَى الْبَيْتِ وَرَأَوُا الصَّبِيَّ مَعَ مَرْيَمَ أُمِّهِ فَخَرُّوا وَسَجَدُوا لَهُ ثُمَّ فَتَحُوا كُنُوزَهُمْ وَقَدَّمُوا لَهُ هَدَايَا: ذَهَباً وَلُبَاناً وَمُرّاً.
\par 12 ثُمَّ إِذْ أُوحِيَ إِلَيْهِمْ فِي حُلْمٍ أَنْ لاَ يَرْجِعُوا إِلَى هِيرُودُسَ انْصَرَفُوا فِي طَرِيقٍ أُخْرَى إِلَى كُورَتِهِمْ.
\par 13 وَبَعْدَمَا انْصَرَفُوا إِذَا مَلاَكُ الرَّبِّ قَدْ ظَهَرَ لِيُوسُفَ فِي حُلْمٍ قَائِلاً: «قُمْ وَخُذِ الصَّبِيَّ وَأُمَّهُ وَاهْرُبْ إِلَى مِصْرَ وَكُنْ هُنَاكَ حَتَّى أَقُولَ لَكَ. لأَنَّ هِيرُودُسَ مُزْمِعٌ أَنْ يَطْلُبَ الصَّبِيَّ لِيُهْلِكَهُ».
\par 14 فَقَامَ وَأَخَذَ الصَّبِيَّ وَأُمَّهُ لَيْلاً وَانْصَرَفَ إِلَى مِصْرَ
\par 15 وَكَانَ هُنَاكَ إِلَى وَفَاةِ هِيرُودُسَ لِكَيْ يَتِمَّ مَا قِيلَ مِنَ الرَّبِّ بِالنَّبِيِّ: «مِنْ مِصْرَ دَعَوْتُ ابْنِي».
\par 16 حِينَئِذٍ لَمَّا رَأَى هِيرُودُسُ أَنَّ الْمَجُوسَ سَخِرُوا بِهِ غَضِبَ جِدّاً فَأَرْسَلَ وَقَتَلَ جَمِيعَ الصِّبْيَانِ الَّذِينَ فِي بَيْتِ لَحْمٍ وَفِي كُلِّ تُخُومِهَا مِنِ ابْنِ سَنَتَيْنِ فَمَا دُونُ بِحَسَبِ الزَّمَانِ الَّذِي تَحَقَّقَهُ مِنَ الْمَجُوسِ.
\par 17 حِينَئِذٍ تَمَّ مَا قِيلَ بِإِرْمِيَا النَّبِيِّ:
\par 18 «صَوْتٌ سُمِعَ فِي الرَّامَةِ نَوْحٌ وَبُكَاءٌ وَعَوِيلٌ كَثِيرٌ. رَاحِيلُ تَبْكِي عَلَى أَوْلاَدِهَا وَلاَ تُرِيدُ أَنْ تَتَعَزَّى لأَنَّهُمْ لَيْسُوا بِمَوْجُودِينَ».
\par 19 فَلَمَّا مَاتَ هِيرُودُسُ إِذَا مَلاَكُ الرَّبِّ قَدْ ظَهَرَ فِي حُلْمٍ لِيُوسُفَ فِي مِصْرَ
\par 20 قَائِلاً: «قُمْ وَخُذِ الصَّبِيَّ وَأُمَّهُ وَاذْهَبْ إِلَى أَرْضِ إِسْرَائِيلَ لأَنَّهُ قَدْ مَاتَ الَّذِينَ كَانُوا يَطْلُبُونَ نَفْسَ الصَّبِيِّ».
\par 21 فَقَامَ وَأَخَذَ الصَّبِيَّ وَأُمَّهُ وَجَاءَ إِلَى أَرْضِ إِسْرَائِيلَ.
\par 22 وَلَكِنْ لَمَّا سَمِعَ أَنَّ أَرْخِيلاَوُسَ يَمْلِكُ عَلَى الْيَهُودِيَّةِ عِوَضاً عَنْ هِيرُودُسَ أَبِيهِ خَافَ أَنْ يَذْهَبَ إِلَى هُنَاكَ. وَإِذْ أُوحِيَ إِلَيْهِ فِي حُلْمٍ انْصَرَفَ إِلَى نَوَاحِي الْجَلِيلِ.
\par 23 وَأَتَى وَسَكَنَ فِي مَدِينَةٍ يُقَالُ لَهَا نَاصِرَةُ لِكَيْ يَتِمَّ مَا قِيلَ بِالأَنْبِيَاءِ: «إِنَّهُ سَيُدْعَى نَاصِرِيّاً».

\chapter{3}

\par 1 وَفِي تِلْكَ الأَيَّامِ جَاءَ يُوحَنَّا الْمَعْمَدَانُ يَكْرِزُ فِي بَرِّيَّةِ الْيَهُودِيَّةِ
\par 2 قَائِلاً: «تُوبُوا لأَنَّهُ قَدِ اقْتَرَبَ مَلَكُوتُ السَّماوَاتِ.
\par 3 فَإِنَّ هَذَا هُوَ الَّذِي قِيلَ عَنْهُ بِإِشَعْيَاءَ النَّبِيِّ: صَوْتُ صَارِخٍ فِي الْبَرِّيَّةِ: أَعِدُّوا طَرِيقَ الرَّبِّ. اصْنَعُوا سُبُلَهُ مُسْتَقِيمَةً».
\par 4 وَيُوحَنَّا هَذَا كَانَ لِبَاسُهُ مِنْ وَبَرِ الإِبِلِ وَعَلَى حَقْوَيْهِ مِنْطَقَةٌ مِنْ جِلْدٍ. وَكَانَ طَعَامُهُ جَرَاداً وَعَسَلاً بَرِّيّاً.
\par 5 حِينَئِذٍ خَرَجَ إِلَيْهِ أُورُشَلِيمُ وَكُلُّ الْيَهُودِيَّةِ وَجَمِيعُ الْكُورَةِ الْمُحِيطَةِ بِالأُرْدُنّ
\par 6 وَاعْتَمَدُوا مِنْهُ فِي الأُرْدُنِّ مُعْتَرِفِينَ بِخَطَايَاهُمْ.
\par 7 فَلَمَّا رَأَى كَثِيرِينَ مِنَ الْفَرِّيسِيِّينَ وَالصَّدُّوقِيِّينَ يَأْتُونَ إِلَى مَعْمُودِيَّتِهِ قَالَ لَهُمْ: «يَا أَوْلاَدَ الأَفَاعِي مَنْ أَرَاكُمْ أَنْ تَهْرُبُوا مِنَ الْغَضَبِ الآتِي؟
\par 8 فَاصْنَعُوا أَثْمَاراً تَلِيقُ بِالتَّوْبَةِ.
\par 9 وَلاَ تَفْتَكِرُوا أَنْ تَقُولُوا فِي أَنْفُسِكُمْ: لَنَا إِبْراهِيمُ أَباً. لأَنِّي أَقُولُ لَكُمْ: إِنَّ اللَّهَ قَادِرٌ أَنْ يُقِيمَ مِنْ هَذِهِ الْحِجَارَةِ أَوْلاَداً لِإِبْراهِيمَ.
\par 10 وَالآنَ قَدْ وُضِعَتِ الْفَأْسُ عَلَى أَصْلِ الشَّجَرِ فَكُلُّ شَجَرَةٍ لاَ تَصْنَعُ ثَمَراً جَيِّداً تُقْطَعُ وَتُلْقَى فِي النَّارِ.
\par 11 أَنَا أُعَمِّدُكُمْ بِمَاءٍ لِلتَّوْبَةِ وَلَكِنِ الَّذِي يَأْتِي بَعْدِي هُوَ أَقْوَى مِنِّي الَّذِي لَسْتُ أَهْلاً أَنْ أَحْمِلَ حِذَاءَهُ. هُوَ سَيُعَمِّدُكُمْ بِالرُّوحِ الْقُدُسِ وَنَارٍ.
\par 12 الَّذِي رَفْشُهُ فِي يَدِهِ وَسَيُنَقِّي بَيْدَرَهُ وَيَجْمَعُ قَمْحَهُ إِلَى الْمَخْزَنِ وَأَمَّا التِّبْنُ فَيُحْرِقُهُ بِنَارٍ لاَ تُطْفَأُ».
\par 13 حِينَئِذٍ جَاءَ يَسُوعُ مِنَ الْجَلِيلِ إِلَى الأُرْدُنِّ إِلَى يُوحَنَّا لِيَعْتَمِدَ مِنْهُ.
\par 14 وَلَكِنْ يُوحَنَّا مَنَعَهُ قَائِلاً: «أَنَا مُحْتَاجٌ أَنْ أَعْتَمِدَ مِنْكَ وَأَنْتَ تَأْتِي إِلَيَّ!»
\par 15 فَقَالَ يَسُوعُ لَهُ: «اسْمَحِ الآنَ لأَنَّهُ هَكَذَا يَلِيقُ بِنَا أَنْ نُكَمِّلَ كُلَّ بِرٍّ». حِينَئِذٍ سَمَحَ لَهُ.
\par 16 فَلَمَّا اعْتَمَدَ يَسُوعُ صَعِدَ لِلْوَقْتِ مِنَ الْمَاءِ وَإِذَا السَّمَاوَاتُ قَدِ انْفَتَحَتْ لَهُ فَرَأَى رُوحَ اللَّهِ نَازِلاً مِثْلَ حَمَامَةٍ وَآتِياً عَلَيْهِ
\par 17 وَصَوْتٌ مِنَ السَّمَاوَاتِ قَائِلاً: «هَذَا هُوَ ابْنِي الْحَبِيبُ الَّذِي بِهِ سُرِرْتُ».

\chapter{4}

\par 1 ثُمَّ أُصْعِدَ يَسُوعُ إِلَى الْبَرِّيَّةِ مِنَ الرُّوحِ لِيُجَرَّبَ مِنْ إِبْلِيسَ.
\par 2 فَبَعْدَ مَا صَامَ أَرْبَعِينَ نَهَاراً وَأَرْبَعِينَ لَيْلَةً جَاعَ أَخِيراً.
\par 3 فَتَقَدَّمَ إِلَيْهِ الْمُجَرِّبُ وَقَالَ لَهُ: «إِنْ كُنْتَ ابْنَ اللَّهِ فَقُلْ أَنْ تَصِيرَ هَذِهِ الْحِجَارَةُ خُبْزاً».
\par 4 فَأَجَابَ: «مَكْتُوبٌ: لَيْسَ بِالْخُبْزِ وَحْدَهُ يَحْيَا الإِنْسَانُ بَلْ بِكُلِّ كَلِمَةٍ تَخْرُجُ مِنْ فَمِ اللَّهِ».
\par 5 ثُمَّ أَخَذَهُ إِبْلِيسُ إِلَى الْمَدِينَةِ الْمُقَدَّسَةِ وَأَوْقَفَهُ عَلَى جَنَاحِ الْهَيْكَلِ
\par 6 وَقَالَ لَهُ: «إِنْ كُنْتَ ابْنَ اللَّهِ فَاطْرَحْ نَفْسَكَ إِلَى أَسْفَلُ لأَنَّهُ مَكْتُوبٌ: أَنَّهُ يُوصِي مَلاَئِكَتَهُ بِكَ فَعَلَى أيَادِيهِمْ يَحْمِلُونَكَ لِكَيْ لاَ تَصْدِمَ بِحَجَرٍ رِجْلَكَ».
\par 7 قَالَ لَهُ يَسُوعُ: «مَكْتُوبٌ أَيْضاً: لاَ تُجَرِّبِ الرَّبَّ إِلَهَكَ».
\par 8 ثُمَّ أَخَذَهُ أَيْضاً إِبْلِيسُ إِلَى جَبَلٍ عَالٍ جِدّاً وَأَرَاهُ جَمِيعَ مَمَالِكِ الْعَالَمِ وَمَجْدَهَا
\par 9 وَقَالَ لَهُ: «أُعْطِيكَ هَذِهِ جَمِيعَهَا إِنْ خَرَرْتَ وَسَجَدْتَ لِي».
\par 10 حِينَئِذٍ قَالَ لَهُ يَسُوعُ: «اذْهَبْ يَا شَيْطَانُ! لأَنَّهُ مَكْتُوبٌ: لِلرَّبِّ إِلَهِكَ تَسْجُدُ وَإِيَّاهُ وَحْدَهُ تَعْبُدُ».
\par 11 ثُمَّ تَرَكَهُ إِبْلِيسُ وَإِذَا مَلاَئِكَةٌ قَدْ جَاءَتْ فَصَارَتْ تَخْدِمُهُ.
\par 12 وَلَمَّا سَمِعَ يَسُوعُ أَنَّ يُوحَنَّا أُسْلِمَ انْصَرَفَ إِلَى الْجَلِيلِ.
\par 13 وَتَرَكَ النَّاصِرَةَ وَأَتَى فَسَكَنَ فِي كَفْرِنَاحُومَ الَّتِي عِنْدَ الْبَحْرِ فِي تُخُومِ زَبُولُونَ وَنَفْتَالِيمَ
\par 14 لِكَيْ يَتِمَّ مَا قِيلَ بِإِشَعْيَاءَ النَّبِيِّ:
\par 15 «أَرْضُ زَبُولُونَ وَأَرْضُ نَفْتَالِيمَ طَرِيقُ الْبَحْرِ عَبْرُ الأُرْدُنِّ جَلِيلُ الأُمَمِ-
\par 16 الشَّعْبُ الْجَالِسُ فِي ظُلْمَةٍ أَبْصَرَ نُوراً عَظِيماً وَالْجَالِسُونَ فِي كُورَةِ الْمَوْتِ وَظِلاَلِهِ أَشْرَقَ عَلَيْهِمْ نُورٌ».
\par 17 مِنْ ذَلِكَ الزَّمَانِ ابْتَدَأَ يَسُوعُ يَكْرِزُ وَيَقُولُ : «تُوبُوا لأَنَّهُ قَدِ اقْتَرَبَ مَلَكُوتُ السَّمَاوَاتِ».
\par 18 وَإِذْ كَانَ يَسُوعُ مَاشِياً عِنْدَ بَحْرِ الْجَلِيلِ أَبْصَرَ أَخَوَيْنِ: سِمْعَانَ الَّذِي يُقَالُ لَهُ بُطْرُسُ وَأَنْدَرَاوُسَ أَخَاهُ يُلْقِيَانِ شَبَكَةً فِي الْبَحْرِ فَإِنَّهُمَا كَانَا صَيَّادَيْنِ.
\par 19 فَقَالَ لَهُمَا: «هَلُمَّ وَرَائِي فَأَجْعَلُكُمَا صَيَّادَيِ النَّاسِ».
\par 20 فَلِلْوَقْتِ تَرَكَا الشِّبَاكَ وَتَبِعَاهُ.
\par 21 ثُمَّ اجْتَازَ مِنْ هُنَاكَ فَرَأَى أَخَوَيْنِ آخَرَيْنِ: يَعْقُوبَ بْنَ زَبْدِي وَيُوحَنَّا أَخَاهُ فِي السَّفِينَةِ مَعَ زَبْدِي أَبِيهِمَا يُصْلِحَانِ شِبَاكَهُمَا فَدَعَاهُمَا.
\par 22 فَلِلْوَقْتِ تَرَكَا السَّفِينَةَ وَأَبَاهُمَا وَتَبِعَاهُ.
\par 23 وَكَانَ يَسُوعُ يَطُوفُ كُلَّ الْجَلِيلِ يُعَلِّمُ فِي مَجَامِعِهِمْ وَيَكْرِزُ بِبِشَارَةِ الْمَلَكُوتِ وَيَشْفِي كُلَّ مَرَضٍ وَكُلَّ ضَعْفٍ فِي الشَّعْبِ.
\par 24 فَذَاعَ خَبَرُهُ فِي جَمِيعِ سُورِيَّةَ. فَأَحْضَرُوا إِلَيْهِ جَمِيعَ السُّقَمَاءِ الْمُصَابِينَ بِأَمْرَاضٍ وَأَوْجَاعٍ مُخْتَلِفَةٍ وَالْمَجَانِينَ وَالْمَصْرُوعِينَ وَالْمَفْلُوجِينَ فَشَفَاهُمْ.
\par 25 فَتَبِعَتْهُ جُمُوعٌ كَثِيرَةٌ مِنَ الْجَلِيلِ وَالْعَشْرِ الْمُدُنِ وَأُورُشَلِيمَ وَالْيَهُودِيَّةِ وَمِنْ عَبْرِ الأُرْدُنِّ.

\chapter{5}

\par 1 وَلَمَّا رَأَى الْجُمُوعَ صَعِدَ إِلَى الْجَبَلِ فَلَمَّا جَلَسَ تَقَدَّمَ إِلَيْهِ تَلاَمِيذُهُ.
\par 2 فَعَلَّمَهُمْ قَائِلاً:
\par 3 «طُوبَى لِلْمَسَاكِينِ بِالرُّوحِ لأَنَّ لَهُمْ مَلَكُوتَ السَّمَاوَاتِ.
\par 4 طُوبَى لِلْحَزَانَى لأَنَّهُمْ يَتَعَزَّوْنَ.
\par 5 طُوبَى لِلْوُدَعَاءِ لأَنَّهُمْ يَرِثُونَ الأَرْضَ.
\par 6 طُوبَى لِلْجِيَاعِ وَالْعِطَاشِ إِلَى الْبِرِّ لأَنَّهُمْ يُشْبَعُونَ.
\par 7 طُوبَى لِلرُّحَمَاءِ لأَنَّهُمْ يُرْحَمُونَ.
\par 8 طُوبَى لِلأَنْقِيَاءِ الْقَلْبِ لأَنَّهُمْ يُعَايِنُونَ اللَّهَ.
\par 9 طُوبَى لِصَانِعِي السَّلاَمِ لأَنَّهُمْ أَبْنَاءَ اللَّهِ يُدْعَوْنَ.
\par 10 طُوبَى لِلْمَطْرُودِينَ مِنْ أَجْلِ الْبِرِّ لأَنَّ لَهُمْ مَلَكُوتَ السَّمَاوَاتِ.
\par 11 طُوبَى لَكُمْ إِذَا عَيَّرُوكُمْ وَطَرَدُوكُمْ وَقَالُوا عَلَيْكُمْ كُلَّ كَلِمَةٍ شِرِّيرَةٍ مِنْ أَجْلِي كَاذِبِينَ.
\par 12 افْرَحُوا وَتَهَلَّلُوا لأَنَّ أَجْرَكُمْ عَظِيمٌ فِي السَّمَاوَاتِ فَإِنَّهُمْ هَكَذَا طَرَدُوا الأَنْبِيَاءَ الَّذِينَ قَبْلَكُمْ.
\par 13 «أَنْتُمْ مِلْحُ الأَرْضِ وَلَكِنْ إِنْ فَسَدَ الْمِلْحُ فَبِمَاذَا يُمَلَّحُ؟ لاَ يَصْلُحُ بَعْدُ لِشَيْءٍ إِلاَّ لأَنْ يُطْرَحَ خَارِجاً وَيُدَاسَ مِنَ النَّاسِ.
\par 14 أَنْتُمْ نُورُ الْعَالَمِ. لاَ يُمْكِنُ أَنْ تُخْفَى مَدِينَةٌ مَوْضُوعَةٌ عَلَى جَبَلٍ
\par 15 وَلاَ يُوقِدُونَ سِرَاجاً وَيَضَعُونَهُ تَحْتَ الْمِكْيَالِ بَلْ عَلَى الْمَنَارَةِ فَيُضِيءُ لِجَمِيعِ الَّذِينَ فِي الْبَيْتِ.
\par 16 فَلْيُضِئْ نُورُكُمْ هَكَذَا قُدَّامَ النَّاسِ لِكَيْ يَرَوْا أَعْمَالَكُمُ الْحَسَنَةَ وَيُمَجِّدُوا أَبَاكُمُ الَّذِي فِي السَّمَاوَاتِ.
\par 17 «لاَ تَظُنُّوا أَنِّي جِئْتُ لأَنْقُضَ النَّامُوسَ أَوِ الأَنْبِيَاءَ. مَا جِئْتُ لأَنْقُضَ بَلْ لِأُكَمِّلَ.
\par 18 فَإِنِّي الْحَقَّ أَقُولُ لَكُمْ: إِلَى أَنْ تَزُولَ السَّمَاءُ وَالأَرْضُ لاَ يَزُولُ حَرْفٌ وَاحِدٌ أَوْ نُقْطَةٌ وَاحِدَةٌ مِنَ النَّامُوسِ حَتَّى يَكُونَ الْكُلُّ.
\par 19 فَمَنْ نَقَضَ إِحْدَى هَذِهِ الْوَصَايَا الصُّغْرَى وَعَلَّمَ النَّاسَ هَكَذَا يُدْعَى أَصْغَرَ فِي مَلَكُوتِ السَّمَاوَاتِ. وَأَمَّا مَنْ عَمِلَ وَعَلَّمَ فَهَذَا يُدْعَى عَظِيماً فِي مَلَكُوتِ السَّمَاوَاتِ.
\par 20 فَإِنِّي أَقُولُ لَكُمْ: إِنَّكُمْ إِنْ لَمْ يَزِدْ بِرُّكُمْ عَلَى الْكَتَبَةِ وَالْفَرِّيسِيِّينَ لَنْ تَدْخُلُوا مَلَكُوتَ السَّماوَاتِ.
\par 21 «قَدْ سَمِعْتُمْ أَنَّهُ قِيلَ لِلْقُدَمَاءِ: لاَ تَقْتُلْ وَمَنْ قَتَلَ يَكُونُ مُسْتَوْجِبَ الْحُكْمِ.
\par 22 وَأَمَّا أَنَا فَأَقُولُ لَكُمْ: إِنَّ كُلَّ مَنْ يَغْضَبُ عَلَى أَخِيهِ بَاطِلاً يَكُونُ مُسْتَوْجِبَ الْحُكْمِ وَمَنْ قَالَ لأَخِيهِ: رَقَا يَكُونُ مُسْتَوْجِبَ الْمَجْمَعِ وَمَنْ قَالَ: يَا أَحْمَقُ يَكُونُ مُسْتَوْجِبَ نَارِ جَهَنَّمَ.
\par 23 فَإِنْ قَدَّمْتَ قُرْبَانَكَ إِلَى الْمَذْبَحِ وَهُنَاكَ تَذَكَّرْتَ أَنَّ لأَخِيكَ شَيْئاً عَلَيْكَ
\par 24 فَاتْرُكْ هُنَاكَ قُرْبَانَكَ قُدَّامَ الْمَذْبَحِ وَاذْهَبْ أَوَّلاً اصْطَلِحْ مَعَ أَخِيكَ وَحِينَئِذٍ تَعَالَ وَقَدِّمْ قُرْبَانَكَ.
\par 25 كُنْ مُرَاضِياً لِخَصْمِكَ سَرِيعاً مَا دُمْتَ مَعَهُ فِي الطَّرِيقِ لِئَلَّا يُسَلِّمَكَ الْخَصْمُ إِلَى الْقَاضِي وَيُسَلِّمَكَ الْقَاضِي إِلَى الشُّرَطِيِّ فَتُلْقَى فِي السِّجْنِ.
\par 26 اَلْحَقَّ أَقُولُ لَكَ: لاَ تَخْرُجُ مِنْ هُنَاكَ حَتَّى تُوفِيَ الْفَلْسَ الأَخِيرَ!
\par 27 «قَدْ سَمِعْتُمْ أَنَّهُ قِيلَ لِلْقُدَمَاءِ: لاَ تَزْنِ.
\par 28 وَأَمَّا أَنَا فَأَقُولُ لَكُمْ: إِنَّ كُلَّ مَنْ يَنْظُرُ إِلَى امْرَأَةٍ لِيَشْتَهِيَهَا فَقَدْ زَنَى بِهَا فِي قَلْبِهِ.
\par 29 فَإِنْ كَانَتْ عَيْنُكَ الْيُمْنَى تُعْثِرُكَ فَاقْلَعْهَا وَأَلْقِهَا عَنْكَ لأَنَّهُ خَيْرٌ لَكَ أَنْ يَهْلِكَ أَحَدُ أَعْضَائِكَ وَلاَ يُلْقَى جَسَدُكَ كُلُّهُ فِي جَهَنَّمَ.
\par 30 وَإِنْ كَانَتْ يَدُكَ الْيُمْنَى تُعْثِرُكَ فَاقْطَعْهَا وَأَلْقِهَا عَنْكَ لأَنَّهُ خَيْرٌ لَكَ أَنْ يَهْلِكَ أَحَدُ أَعْضَائِكَ وَلاَ يُلْقَى جَسَدُكَ كُلُّهُ فِي جَهَنَّمَ.
\par 31 «وَقِيلَ: مَنْ طَلَّقَ امْرَأَتَهُ فَلْيُعْطِهَا كِتَابَ طَلاَقٍ
\par 32 وَأَمَّا أَنَا فَأَقُولُ لَكُمْ: إِنَّ مَنْ طَلَّقَ امْرَأَتَهُ إِلاَّ لِعِلَّةِ الزِّنَى يَجْعَلُهَا تَزْنِي وَمَنْ يَتَزَوَّجُ مُطَلَّقَةً فَإِنَّهُ يَزْنِي.
\par 33 «أَيْضاً سَمِعْتُمْ أَنَّهُ قِيلَ لِلْقُدَمَاءِ:لاَ تَحْنَثْ بَلْ أَوْفِ لِلرَّبِّ أَقْسَامَكَ.
\par 34 وَأَمَّا أَنَا فَأَقُولُ لَكُمْ: لاَ تَحْلِفُوا الْبَتَّةَ لاَ بِالسَّمَاءِ لأَنَّهَا كُرْسِيُّ اللَّهِ
\par 35 وَلاَ بِالأَرْضِ لأَنَّهَا مَوْطِئُ قَدَمَيْهِ وَلاَ بِأُورُشَلِيمَ لأَنَّهَا مَدِينَةُ الْمَلِكِ الْعَظِيمِ.
\par 36 وَلاَ تَحْلِفْ بِرَأْسِكَ لأَنَّكَ لاَ تَقْدِرُ أَنْ تَجْعَلَ شَعْرَةً وَاحِدَةً بَيْضَاءَ أَوْ سَوْدَاءَ.
\par 37 بَلْ لِيَكُنْ كَلاَمُكُمْ: نَعَمْ نَعَمْ لاَ لاَ. وَمَا زَادَ عَلَى ذَلِكَ فَهُوَ مِنَ الشِّرِّيرِ.
\par 38 «سَمِعْتُمْ أَنَّهُ قِيلَ: عَيْنٌ بِعَيْنٍ وَسِنٌّ بِسِنٍّ.
\par 39 وَأَمَّا أَنَا فَأَقُولُ لَكُمْ: لاَ تُقَاوِمُوا الشَّرَّ بَلْ مَنْ لَطَمَكَ عَلَى خَدِّكَ الأَيْمَنِ فَحَوِّلْ لَهُ الآخَرَ أَيْضاً.
\par 40 وَمَنْ أَرَادَ أَنْ يُخَاصِمَكَ وَيَأْخُذَ ثَوْبَكَ فَاتْرُكْ لَهُ الرِّدَاءَ أَيْضاً.
\par 41 وَمَنْ سَخَّرَكَ مِيلاً وَاحِداً فَاذْهَبْ مَعَهُ اثْنَيْنِ.
\par 42 مَنْ سَأَلَكَ فَأَعْطِهِ وَمَنْ أَرَادَ أَنْ يَقْتَرِضَ مِنْكَ فَلاَ تَرُدَّهُ.
\par 43 «سَمِعْتُمْ أَنَّهُ قِيلَ: تُحِبُّ قَرِيبَكَ وَتُبْغِضُ عَدُوَّكَ.
\par 44 وَأَمَّا أَنَا فَأَقُولُ لَكُمْ: أَحِبُّوا أَعْدَاءَكُمْ. بَارِكُوا لاَعِنِيكُمْ. أَحْسِنُوا إِلَى مُبْغِضِيكُمْ وَصَلُّوا لأَجْلِ الَّذِينَ يُسِيئُونَ إِلَيْكُمْ وَيَطْرُدُونَكُمْ
\par 45 لِكَيْ تَكُونُوا أَبْنَاءَ أَبِيكُمُ الَّذِي فِي السَّمَاوَاتِ فَإِنَّهُ يُشْرِقُ شَمْسَهُ عَلَى الأَشْرَارِ وَالصَّالِحِينَ وَيُمْطِرُ عَلَى الأَبْرَارِ وَالظَّالِمِينَ.
\par 46 لأَنَّهُ إِنْ أَحْبَبْتُمُ الَّذِينَ يُحِبُّونَكُمْ فَأَيُّ أَجْرٍ لَكُمْ؟ أَلَيْسَ الْعَشَّارُونَ أَيْضاً يَفْعَلُونَ ذَلِكَ؟
\par 47 وَإِنْ سَلَّمْتُمْ عَلَى إِخْوَتِكُمْ فَقَطْ فَأَيَّ فَضْلٍ تَصْنَعُونَ؟ أَلَيْسَ الْعَشَّارُونَ أَيْضاً يَفْعَلُونَ هَكَذَا؟
\par 48 فَكُونُوا أَنْتُمْ كَامِلِينَ كَمَا أَنَّ أَبَاكُمُ الَّذِي فِي السَّمَاوَاتِ هُوَ كَامِلٌ.

\chapter{6}

\par 1 «احْتَرِزُوا مِنْ أَنْ تَصْنَعُوا صَدَقَتَكُمْ قُدَّامَ النَّاسِ لِكَيْ يَنْظُرُوكُمْ وَإِلَّا فَلَيْسَ لَكُمْ أَجْرٌ عِنْدَ أَبِيكُمُ الَّذِي فِي السَّمَاوَاتِ.
\par 2 فَمَتَى صَنَعْتَ صَدَقَةً فَلاَ تُصَوِّتْ قُدَّامَكَ بِالْبُوقِ كَمَا يَفْعَلُ الْمُرَاؤُونَ فِي الْمَجَامِعِ وَفِي الأَزِقَّةِ لِكَيْ يُمَجَّدُوا مِنَ النَّاسِ. اَلْحَقَّ أَقُولُ لَكُمْ: إِنَّهُمْ قَدِ اسْتَوْفَوْا أَجْرَهُمْ!
\par 3 وَأَمَّا أَنْتَ فَمَتَى صَنَعْتَ صَدَقَةً فَلاَ تُعَرِّفْ شِمَالَكَ مَا تَفْعَلُ يَمِينُكَ
\par 4 لِكَيْ تَكُونَ صَدَقَتُكَ فِي الْخَفَاءِ. فَأَبُوكَ الَّذِي يَرَى فِي الْخَفَاءِ هُوَ يُجَازِيكَ عَلاَنِيَةً.
\par 5 «وَمَتَى صَلَّيْتَ فَلاَ تَكُنْ كَالْمُرَائِينَ فَإِنَّهُمْ يُحِبُّونَ أَنْ يُصَلُّوا قَائِمِينَ فِي الْمَجَامِعِ وَفِي زَوَايَا الشَّوَارِعِ لِكَيْ يَظْهَرُوا لِلنَّاسِ. اَلْحَقَّ أَقُولُ لَكُمْ: إِنَّهُمْ قَدِ اسْتَوْفَوْا أَجْرَهُمْ!
\par 6 وَأَمَّا أَنْتَ فَمَتَى صَلَّيْتَ فَادْخُلْ إِلَى مِخْدَعِكَ وَأَغْلِقْ بَابَكَ وَصَلِّ إِلَى أَبِيكَ الَّذِي فِي الْخَفَاءِ. فَأَبُوكَ الَّذِي يَرَى فِي الْخَفَاءِ يُجَازِيكَ عَلاَنِيَةً.
\par 7 وَحِينَمَا تُصَلُّونَ لاَ تُكَرِّرُوا الْكَلاَمَ بَاطِلاً كَالأُمَمِ فَإِنَّهُمْ يَظُنُّونَ أَنَّهُ بِكَثْرَةِ كَلاَمِهِمْ يُسْتَجَابُ لَهُمْ.
\par 8 فَلاَ تَتَشَبَّهُوا بِهِمْ. لأَنَّ أَبَاكُمْ يَعْلَمُ مَا تَحْتَاجُونَ إِلَيْهِ قَبْلَ أَنْ تَسْأَلُوهُ.
\par 9 «فَصَلُّوا أَنْتُمْ هَكَذَا: أَبَانَا الَّذِي فِي السَّمَاوَاتِ لِيَتَقَدَّسِ اسْمُكَ.
\par 10 لِيَأْتِ مَلَكُوتُكَ. لِتَكُنْ مَشِيئَتُكَ كَمَا فِي السَّمَاءِ كَذَلِكَ عَلَى الأَرْضِ.
\par 11 خُبْزَنَا كَفَافَنَا أَعْطِنَا الْيَوْمَ.
\par 12 وَاغْفِرْ لَنَا ذُنُوبَنَا كَمَا نَغْفِرُ نَحْنُ أَيْضاً لِلْمُذْنِبِينَ إِلَيْنَا.
\par 13 وَلاَ تُدْخِلْنَا فِي تَجْرِبَةٍ لَكِنْ نَجِّنَا مِنَ الشِّرِّيرِ. لأَنَّ لَكَ الْمُلْكَ وَالْقُوَّةَ وَالْمَجْدَ إِلَى الأَبَدِ. آمِينَ.
\par 14 فَإِنَّهُ إِنْ غَفَرْتُمْ لِلنَّاسِ زَلَّاتِهِمْ يَغْفِرْ لَكُمْ أَيْضاً أَبُوكُمُ السَّمَاوِيُّ.
\par 15 وَإِنْ لَمْ تَغْفِرُوا لِلنَّاسِ زَلَّاتِهِمْ لاَ يَغْفِرْ لَكُمْ أَبُوكُمْ أَيْضاً زَلَّاتِكُمْ.
\par 16 «وَمَتَى صُمْتُمْ فَلاَ تَكُونُوا عَابِسِينَ كَالْمُرَائِينَ فَإِنَّهُمْ يُغَيِّرُونَ وُجُوهَهُمْ لِكَيْ يَظْهَرُوا لِلنَّاسِ صَائِمِينَ. اَلْحَقَّ أَقُولُ لَكُمْ: إِنَّهُمْ قَدِ اسْتَوْفَوْا أَجْرَهُمْ.
\par 17 وَأَمَّا أَنْتَ فَمَتَى صُمْتَ فَادْهُنْ رَأْسَكَ وَاغْسِلْ وَجْهَكَ
\par 18 لِكَيْ لاَ تَظْهَرَ لِلنَّاسِ صَائِماً بَلْ لأَبِيكَ الَّذِي فِي الْخَفَاءِ. فَأَبُوكَ الَّذِي يَرَى فِي الْخَفَاءِ يُجَازِيكَ عَلاَنِيَةً.
\par 19 «لاَ تَكْنِزُوا لَكُمْ كُنُوزاً عَلَى الأَرْضِ حَيْثُ يُفْسِدُ السُّوسُ وَالصَّدَأُ وَحَيْثُ يَنْقُبُ السَّارِقُونَ وَيَسْرِقُونَ.
\par 20 بَلِ اكْنِزُوا لَكُمْ كُنُوزاً فِي السَّمَاءِ حَيْثُ لاَ يُفْسِدُ سُوسٌ وَلاَ صَدَأٌ وَحَيْثُ لاَ يَنْقُبُ سَارِقُونَ وَلاَ يَسْرِقُونَ
\par 21 لأَنَّهُ حَيْثُ يَكُونُ كَنْزُكَ هُنَاكَ يَكُونُ قَلْبُكَ أَيْضاً.
\par 22 سِرَاجُ الْجَسَدِ هُوَ الْعَيْنُ فَإِنْ كَانَتْ عَيْنُكَ بَسِيطَةً فَجَسَدُكَ كُلُّهُ يَكُونُ نَيِّراً
\par 23 وَإِنْ كَانَتْ عَيْنُكَ شِرِّيرَةً فَجَسَدُكَ كُلُّهُ يَكُونُ مُظْلِماً فَإِنْ كَانَ النُّورُ الَّذِي فِيكَ ظَلاَماً فَالظَّلاَمُ كَمْ يَكُونُ!
\par 24 «لاَ يَقْدِرُ أَحَدٌ أَنْ يَخْدِمَ سَيِّدَيْنِ لأَنَّهُ إِمَّا أَنْ يُبْغِضَ الْوَاحِدَ وَيُحِبَّ الآخَرَ أَوْ يُلاَزِمَ الْوَاحِدَ وَيَحْتَقِرَ الآخَرَ. لاَ تَقْدِرُونَ أَنْ تَخْدِمُوا اللَّهَ وَالْمَالَ.
\par 25 لِذَلِكَ أَقُولُ لَكُمْ: لاَ تَهْتَمُّوا لِحَيَاتِكُمْ بِمَا تَأْكُلُونَ وَبِمَا تَشْرَبُونَ وَلاَ لأَجْسَادِكُمْ بِمَا تَلْبَسُونَ. أَلَيْسَتِ الْحَيَاةُ أَفْضَلَ مِنَ الطَّعَامِ وَالْجَسَدُ أَفْضَلَ مِنَ اللِّبَاسِ؟
\par 26 اُنْظُرُوا إِلَى طُيُورِ السَّمَاءِ: إِنَّهَا لاَ تَزْرَعُ وَلاَ تَحْصُدُ وَلاَ تَجْمَعُ إِلَى مَخَازِنَ وَأَبُوكُمُ السَّمَاوِيُّ يَقُوتُهَا. أَلَسْتُمْ أَنْتُمْ بِالْحَرِيِّ أَفْضَلَ مِنْهَا؟
\par 27 وَمَنْ مِنْكُمْ إِذَا اهْتَمَّ يَقْدِرُ أَنْ يَزِيدَ عَلَى قَامَتِهِ ذِرَاعاً وَاحِدَةً؟
\par 28 وَلِمَاذَا تَهْتَمُّونَ بِاللِّبَاسِ؟ تَأَمَّلُوا زَنَابِقَ الْحَقْلِ كَيْفَ تَنْمُو! لاَ تَتْعَبُ وَلاَ تَغْزِلُ.
\par 29 وَلَكِنْ أَقُولُ لَكُمْ إِنَّهُ وَلاَ سُلَيْمَانُ فِي كُلِّ مَجْدِهِ كَانَ يَلْبَسُ كَوَاحِدَةٍ مِنْهَا.
\par 30 فَإِنْ كَانَ عُشْبُ الْحَقْلِ الَّذِي يُوجَدُ الْيَوْمَ وَيُطْرَحُ غَداً فِي التَّنُّورِ يُلْبِسُهُ اللَّهُ هَكَذَا أَفَلَيْسَ بِالْحَرِيِّ جِدّاً يُلْبِسُكُمْ أَنْتُمْ يَا قَلِيلِي الإِيمَانِ؟
\par 31 فَلاَ تَهْتَمُّوا قَائِلِينَ: مَاذَا نَأْكُلُ أَوْ مَاذَا نَشْرَبُ أَوْ مَاذَا نَلْبَسُ؟
\par 32 فَإِنَّ هَذِهِ كُلَّهَا تَطْلُبُهَا الأُمَمُ. لأَنَّ أَبَاكُمُ السَّمَاوِيَّ يَعْلَمُ أَنَّكُمْ تَحْتَاجُونَ إِلَى هَذِهِ كُلِّهَا.
\par 33 لَكِنِ اطْلُبُوا أَوَّلاً مَلَكُوتَ اللَّهِ وَبِرَّهُ وَهَذِهِ كُلُّهَا تُزَادُ لَكُمْ.
\par 34 فَلاَ تَهْتَمُّوا لِلْغَدِ لأَنَّ الْغَدَ يَهْتَمُّ بِمَا لِنَفْسِهِ. يَكْفِي ايَوْمَ شَرُّهُ.

\chapter{7}

\par 1 «لاَ تَدِينُوا لِكَيْ لاَ تُدَانُوا
\par 2 لأَنَّكُمْ بِالدَّيْنُونَةِ الَّتِي بِهَا تَدِينُونَ تُدَانُونَ وَبِالْكَيْلِ الَّذِي بِهِ تَكِيلُونَ يُكَالُ لَكُمْ.
\par 3 وَلِمَاذَا تَنْظُرُ الْقَذَى الَّذِي فِي عَيْنِ أَخِيكَ وَأَمَّا الْخَشَبَةُ الَّتِي فِي عَيْنِكَ فَلاَ تَفْطَنُ لَهَا؟
\par 4 أَمْ كَيْفَ تَقُولُ لأَخِيكَ: دَعْنِي أُخْرِجِ الْقَذَى مِنْ عَيْنِكَ وَهَا الْخَشَبَةُ فِي عَيْنِكَ.
\par 5 يَا مُرَائِي أَخْرِجْ أَوَّلاً الْخَشَبَةَ مِنْ عَيْنِكَ وَحِينَئِذٍ تُبْصِرُ جَيِّداً أَنْ تُخْرِجَ الْقَذَى مِنْ عَيْنِ أَخِيكَ!
\par 6 لاَ تُعْطُوا الْمُقَدَّسَ لِلْكِلاَبِ وَلاَ تَطْرَحُوا دُرَرَكُمْ قُدَّامَ الْخَنَازِيرِ لِئَلَّا تَدُوسَهَا بِأَرْجُلِهَا وَتَلْتَفِتَ فَتُمَزِّقَكُمْ.
\par 7 «اسْأَلُوا تُعْطَوْا. اطْلُبُوا تَجِدُوا. اقْرَعُوا يُفْتَحْ لَكُمْ.
\par 8 لأَنَّ كُلَّ مَنْ يَسْأَلُ يَأْخُذُ وَمَنْ يَطْلُبُ يَجِدُ وَمَنْ يَقْرَعُ يُفْتَحُ لَهُ.
\par 9 أَمْ أَيُّ إِنْسَانٍ مِنْكُمْ إِذَا سَأَلَهُ ابْنُهُ خُبْزاً يُعْطِيهِ حَجَراً؟
\par 10 وَإِنْ سَأَلَهُ سَمَكَةً يُعْطِيهِ حَيَّةً؟
\par 11 فَإِنْ كُنْتُمْ وَأَنْتُمْ أَشْرَارٌ تَعْرِفُونَ أَنْ تُعْطُوا أَوْلاَدَكُمْ عَطَايَا جَيِّدَةً فَكَمْ بِالْحَرِيِّ أَبُوكُمُ الَّذِي فِي السَّمَاوَاتِ يَهَبُ خَيْرَاتٍ لِلَّذِينَ يَسْأَلُونَهُ.
\par 12 فَكُلُّ مَا تُرِيدُونَ أَنْ يَفْعَلَ النَّاسُ بِكُمُ افْعَلُوا هَكَذَا أَنْتُمْ أَيْضاً بِهِمْ لأَنَّ هَذَا هُوَ النَّامُوسُ وَالأَنْبِيَاءُ.
\par 13 «ادْخُلُوا مِنَ الْبَابِ الضَّيِّقِ لأَنَّهُ وَاسِعٌ الْبَابُ وَرَحْبٌ الطَّرِيقُ الَّذِي يُؤَدِّي إِلَى الْهَلاَكِ وَكَثِيرُونَ هُمُ الَّذِينَ يَدْخُلُونَ مِنْهُ!
\par 14 مَا أَضْيَقَ الْبَابَ وَأَكْرَبَ الطَّرِيقَ الَّذِي يُؤَدِّي إِلَى الْحَيَاةِ وَقَلِيلُونَ هُمُ الَّذِينَ يَجِدُونَهُ!
\par 15 «احْتَرِزُوا مِنَ الأَنْبِيَاءِ الْكَذَبَةِ الَّذِينَ يَأْتُونَكُمْ بِثِيَابِ الْحُمْلاَنِ وَلَكِنَّهُمْ مِنْ دَاخِلٍ ذِئَابٌ خَاطِفَةٌ!
\par 16 مِنْ ثِمَارِهِمْ تَعْرِفُونَهُمْ. هَلْ يَجْتَنُونَ مِنَ الشَّوْكِ عِنَباً أَوْ مِنَ الْحَسَكِ تِيناً؟
\par 17 هَكَذَا كُلُّ شَجَرَةٍ جَيِّدَةٍ تَصْنَعُ أَثْمَاراً جَيِّدَةً وَأَمَّا الشَّجَرَةُ الرَّدِيَّةُ فَتَصْنَعُ أَثْمَاراً رَدِيَّةً
\par 18 لاَ تَقْدِرُ شَجَرَةٌ جَيِّدَةٌ أَنْ تَصْنَعَ أَثْمَاراً رَدِيَّةً وَلاَ شَجَرَةٌ رَدِيَّةٌ أَنْ تَصْنَعَ أَثْمَاراً جَيِّدَةً.
\par 19 كُلُّ شَجَرَةٍ لاَ تَصْنَعُ ثَمَراً جَيِّداً تُقْطَعُ وَتُلْقَى فِي النَّارِ.
\par 20 فَإِذاً مِنْ ثِمَارِهِمْ تَعْرِفُونَهُمْ.
\par 21 «لَيْسَ كُلُّ مَنْ يَقُولُ لِي: يَا رَبُّ يَا رَبُّ يَدْخُلُ مَلَكُوتَ السَّمَاوَاتِ. بَلِ الَّذِي يَفْعَلُ إِرَادَةَ أَبِي الَّذِي فِي السَّمَاوَاتِ.
\par 22 كَثِيرُونَ سَيَقُولُونَ لِي فِي ذَلِكَ الْيَوْمِ: يَا رَبُّ يَا رَبُّ أَلَيْسَ بِاسْمِكَ تَنَبَّأْنَا وَبِاسْمِكَ أَخْرَجْنَا شَيَاطِينَ وَبِاسْمِكَ صَنَعْنَا قُوَّاتٍ كَثِيرَةً؟
\par 23 فَحِينَئِذٍ أُصَرِّحُ لَهُمْ: إِنِّي لَمْ أَعْرِفْكُمْ قَطُّ! اذْهَبُوا عَنِّي يَا فَاعِلِي الإِثْمِ!
\par 24 «فَكُلُّ مَنْ يَسْمَعُ أَقْوَالِي هَذِهِ وَيَعْمَلُ بِهَا أُشَبِّهُهُ بِرَجُلٍ عَاقِلٍ بَنَى بَيْتَهُ عَلَى الصَّخْرِ.
\par 25 فَنَزَلَ الْمَطَرُ وَجَاءَتِ الأَنْهَارُ وَهَبَّتِ الرِّيَاحُ وَوَقَعَتْ عَلَى ذَلِكَ الْبَيْتِ فَلَمْ يَسْقُطْ لأَنَّهُ كَانَ مُؤَسَّساً عَلَى الصَّخْرِ.
\par 26 وَكُلُّ مَنْ يَسْمَعُ أَقْوَالِي هَذِهِ وَلاَ يَعْمَلُ بِهَا يُشَبَّهُ بِرَجُلٍ جَاهِلٍ بَنَى بَيْتَهُ عَلَى الرَّمْلِ.
\par 27 فَنَزَلَ الْمَطَرُ وَجَاءَتِ الأَنْهَارُ وَهَبَّتِ الرِّيَاحُ وَصَدَمَتْ ذَلِكَ الْبَيْتَ فَسَقَطَ وَكَانَ سُقُوطُهُ عَظِيماً!».
\par 28 فَلَمَّا أَكْمَلَ يَسُوعُ هَذِهِ الأَقْوَالَ بُهِتَتِ الْجُمُوعُ مِنْ تَعْلِيمِهِ
\par 29 لأَنَّهُ كَانَ يُعَلِّمُهُمْ كَمَنْ لَهُ سُلْطَانٌ وَلَيْسَ كَالْكَتَبَةِ.

\chapter{8}

\par 1 وَلَمَّا نَزَلَ مِنَ الْجَبَلِ تَبِعَتْهُ جُمُوعٌ كَثِيرَةٌ.
\par 2 وَإِذَا أَبْرَصُ قَدْ جَاءَ وَسَجَدَ لَهُ قَائِلاً: «يَا سَيِّدُ إِنْ أَرَدْتَ تَقْدِرْ أَنْ تُطَهِّرَنِي».
\par 3 فَمَدَّ يَسُوعُ يَدَهُ وَلَمَسَهُ قَائِلاً: «أُرِيدُ فَاطْهُرْ». وَلِلْوَقْتِ طَهُرَ بَرَصُهُ.
\par 4 فَقَالَ لَهُ يَسُوعُ: «انْظُرْ أَنْ لاَ تَقُولَ لأَحَدٍ. بَلِ اذْهَبْ أَرِ نَفْسَكَ لِلْكَاهِنِ وَقَدَّمِ الْقُرْبَانَ الَّذِي أَمَرَ بِهِ مُوسَى شَهَادَةً لَهُمْ».
\par 5 وَلَمَّا دَخَلَ يَسُوعُ كَفْرَنَاحُومَ جَاءَ إِلَيْهِ قَائِدُ مِئَةٍ يَطْلُبُ إِلَيْهِ
\par 6 وَيَقُولُ: «يَا سَيِّدُ غُلاَمِي مَطْرُوحٌ فِي الْبَيْتِ مَفْلُوجاً مُتَعَذِّباً جِدَّاً».
\par 7 فَقَالَ لَهُ يَسُوعُ: «أَنَا آتِي وَأَشْفِيهِ».
\par 8 فَأَجَابَ قَائِدُ الْمِئَةِ: «يَا سَيِّدُ لَسْتُ مُسْتَحِقّاً أَنْ تَدْخُلَ تَحْتَ سَقْفِي لَكِنْ قُلْ كَلِمَةً فَقَطْ فَيَبْرَأَ غُلاَمِي.
\par 9 لأَنِّي أَنَا أَيْضاً إِنْسَانٌ تَحْتَ سُلْطَانٍ. لِي جُنْدٌ تَحْتَ يَدِي. أَقُولُ لِهَذَا: اذْهَبْ فَيَذْهَبُ وَلِآخَرَ: ايتِ فَيَأْتِي وَلِعَبْدِيَ: افْعَلْ هَذَا فَيَفْعَلُ».
\par 10 فَلَمَّا سَمِعَ يَسُوعُ تَعَجَّبَ وَقَالَ لِلَّذِينَ يَتْبَعُونَ: «اَلْحَقَّ أَقُولُ لَكُمْ لَمْ أَجِدْ وَلاَ فِي إِسْرَائِيلَ إِيمَاناً بِمِقْدَارِ هَذَا.
\par 11 وَأَقُولُ لَكُمْ: إِنَّ كَثِيرِينَ سَيَأْتُونَ مِنَ الْمَشَارِقِ وَالْمَغَارِبِ وَيَتَّكِئُونَ مَعَ إِبْراهِيمَ وَإِسْحاقَ وَيَعْقُوبَ فِي مَلَكُوتِ السَّمَاوَاتِ
\par 12 وَأَمَّا بَنُو الْمَلَكُوتِ فَيُطْرَحُونَ إِلَى الظُّلْمَةِ الْخَارِجِيَّةِ. هُنَاكَ يَكُونُ الْبُكَاءُ وَصَرِيرُ الأَسْنَانِ».
\par 13 ثُمَّ قَالَ يَسُوعُ لِقَائِدِ الْمِئَةِ: «اذْهَبْ وَكَمَا آمَنْتَ لِيَكُنْ لَكَ». فَبَرَأَ غُلاَمُهُ فِي تِلْكَ السَّاعَةِ.
\par 14 وَلَمَّا جَاءَ يَسُوعُ إِلَى بَيْتِ بُطْرُسَ رَأَى حَمَاتَهُ مَطْرُوحَةً وَمَحْمُومَةً
\par 15 فَلَمَسَ يَدَهَا فَتَرَكَتْهَا الْحُمَّى فَقَامَتْ وَخَدَمَتْهُمْ.
\par 16 وَلَمَّا صَارَ الْمَسَاءُ قَدَّمُوا إِلَيْهِ مَجَانِينَ كَثِيرِينَ فَأَخْرَجَ الأَرْوَاحَ بِكَلِمَةٍ وَجَمِيعَ الْمَرْضَى شَفَاهُمْ
\par 17 لِكَيْ يَتِمَّ مَا قِيلَ بِإِشَعْيَاءَ النَّبِيِّ: «هُوَ أَخَذَ أَسْقَامَنَا وَحَمَلَ أَمْرَاضَنَا».
\par 18 وَلَمَّا رَأَى يَسُوعُ جُمُوعاً كَثِيرَةً حَوْلَهُ أَمَرَ بِالذَّهَابِ إِلَى الْعَبْرِ.
\par 19 فَتَقَدَّمَ كَاتِبٌ وَقَالَ لَهُ: «يَا مُعَلِّمُ أَتْبَعُكَ أَيْنَمَا تَمْضِي».
\par 20 فَقَالَ لَهُ يَسُوعُ: «لِلثَّعَالِبِ أَوْجِرَةٌ وَلِطُيُورِ السَّمَاءِ أَوْكَارٌ وَأَمَّا ابْنُ الإِنْسَانِ فَلَيْسَ لَهُ أَيْنَ يُسْنِدُ رَأْسَهُ».
\par 21 وَقَالَ لَهُ آخَرُ مِنْ تَلاَمِيذِهِ: «يَا سَيِّدُ ائْذَنْ لِي أَنْ أَمْضِيَ أَوَّلاً وَأَدْفِنَ أَبِي».
\par 22 فَقَالَ لَهُ يَسُوعُ: «اتْبَعْنِي وَدَعِ الْمَوْتَى يَدْفِنُونَ مَوْتَاهُمْ».
\par 23 وَلَمَّا دَخَلَ السَّفِينَةَ تَبِعَهُ تَلاَمِيذُهُ.
\par 24 وَإِذَا اضْطِرَابٌ عَظِيمٌ قَدْ حَدَثَ فِي الْبَحْرِ حَتَّى غَطَّتِ الأَمْوَاجُ السَّفِينَةَ وَكَانَ هُوَ نَائِماً.
\par 25 فَتَقَدَّمَ تَلاَمِيذُهُ وَأَيْقَظُوهُ قَائِلِينَ: «يَا سَيِّدُ نَجِّنَا فَإِنَّنَا نَهْلِكُ!»
\par 26 فَقَالَ لَهُمْ: «مَا بَالُكُمْ خَائِفِينَ يَا قَلِيلِي الإِيمَانِ؟» ثُمَّ قَامَ وَانْتَهَرَ الرِّيَاحَ وَالْبَحْرَ فَصَارَ هُدُوءٌ عَظِيمٌ.
\par 27 فَتَعَجَّبَ النَّاسُ قَائِلِينَ: «أَيُّ إِنْسَانٍ هَذَا! فَإِنَّ الرِّيَاحَ وَالْبَحْرَ جَمِيعاً تُطِيعُهُ».
\par 28 وَلَمَّا جَاءَ إِلَى الْعَبْرِ إِلَى كُورَةِ الْجِرْجَسِيِّينَ اسْتَقْبَلَهُ مَجْنُونَانِ خَارِجَانِ مِنَ الْقُبُورِ هَائِجَانِ جِدَّاً حَتَّى لَمْ يَكُنْ أَحَدٌ يَقْدِرُ أَنْ يَجْتَازَ مِنْ تِلْكَ الطَّرِيقِ.
\par 29 وَإِذَا هُمَا قَدْ صَرَخَا قَائِلَيْنِ: «مَا لَنَا وَلَكَ يَا يَسُوعُ ابْنَ اللَّهِ؟ أَجِئْتَ إِلَى هُنَا قَبْلَ الْوَقْتِ لِتُعَذِّبَنَا؟»
\par 30 وَكَانَ بَعِيداً مِنْهُمْ قَطِيعُ خَنَازِيرَ كَثِيرَةٍ تَرْعَى.
\par 31 فَالشَّيَاطِينُ طَلَبُوا إِلَيْهِ قَائِلِينَ: «إِنْ كُنْتَ تُخْرِجُنَا فَأْذَنْ لَنَا أَنْ نَذْهَبَ إِلَى قَطِيعِ الْخَنَازِيرِ».
\par 32 فَقَالَ لَهُمُ: «امْضُوا». فَخَرَجُوا وَمَضَوْا إِلَى قَطِيعِ الْخَنَازِيرِ وَإِذَا قَطِيعُ الْخَنَازِيرِ كُلُّهُ قَدِ انْدَفَعَ مِنْ عَلَى الْجُرْفِ إِلَى الْبَحْرِ وَمَاتَ فِي الْمِيَاهِ.
\par 33 أَمَّا الرُّعَاةُ فَهَرَبُوا وَمَضَوْا إِلَى الْمَدِينَةِ وَأَخْبَرُوا عَنْ كُلِّ شَيْءٍ وَعَنْ أَمْرِ الْمَجْنُونَيْنِ.
\par 34 فَإِذَا كُلُّ الْمَدِينَةِ قَدْ خَرَجَتْ لِمُلاَقَاةِ يَسُوعَ. وَلَمَّا أَبْصَرُوهُ طَلَبُوا أَنْ يَنْصَرِفَ عَنْ تُخُومِهِمْ.

\chapter{9}

\par 1 فَدَخَلَ السَّفِينَةَ وَاجْتَازَ وَجَاءَ إِلَى مَدِينَتِهِ.
\par 2 وَإِذَا مَفْلُوجٌ يُقَدِّمُونَهُ إِلَيْهِ مَطْرُوحاً عَلَى فِرَاشٍ. فَلَمَّا رَأَى يَسُوعُ إِيمَانَهُمْ قَالَ لِلْمَفْلُوجِ: «ثِقْ يَا بُنَيَّ. مَغْفُورَةٌ لَكَ خَطَايَاكَ».
\par 3 وَإِذَا قَوْمٌ مِنَ الْكَتَبَةِ قَدْ قَالُوا فِي أَنْفُسِهِمْ: «هَذَا يُجَدِّفُ!»
\par 4 فَعَلِمَ يَسُوعُ أَفْكَارَهُمْ فَقَالَ: «لِمَاذَا تُفَكِّرُونَ بِالشَّرِّ فِي قُلُوبِكُمْ؟
\par 5 أَيُّمَا أَيْسَرُ أَنْ يُقَالَ: مَغْفُورَةٌ لَكَ خَطَايَاكَ أَمْ أَنْ يُقَالَ: قُمْ وَامْشِ؟
\par 6 وَلَكِنْ لِكَيْ تَعْلَمُوا أَنَّ لاِبْنِ الإِنْسَانِ سُلْطَاناً عَلَى الأَرْضِ أَنْ يَغْفِرَ الْخَطَايَا» - حِينَئِذٍ قَالَ لِلْمَفْلُوجِ: «قُمِ احْمِلْ فِرَاشَكَ وَاذْهَبْ إِلَى بَيْتِكَ!»
\par 7 فَقَامَ وَمَضَى إِلَى بَيْتِهِ.
\par 8 فَلَمَّا رَأَى الْجُمُوعُ تَعَجَّبُوا وَمَجَّدُوا اللَّهَ الَّذِي أَعْطَى النَّاسَ سُلْطَاناً مِثْلَ هَذَا.
\par 9 وَفِيمَا يَسُوعُ مُجْتَازٌ مِنْ هُنَاكَ رَأَى إِنْسَاناً جَالِساً عِنْدَ مَكَانِ الْجِبَايَةِ اسْمُهُ مَتَّى. فَقَالَ لَهُ: «اتْبَعْنِي». فَقَامَ وَتَبِعَهُ.
\par 10 وَبَيْنَمَا هُوَ مُتَّكِئٌ فِي الْبَيْتِ إِذَا عَشَّارُونَ وَخُطَاةٌ كَثِيرُونَ قَدْ جَاءُوا وَاتَّكَأُوا مَعَ يَسُوعَ وَتَلاَمِيذِهِ.
\par 11 فَلَمَّا نَظَرَ الْفَرِّيسِيُّونَ قَالُوا لِتَلاَمِيذِهِ: «لِمَاذَا يَأْكُلُ مُعَلِّمُكُمْ مَعَ الْعَشَّارِينَ وَالْخُطَاةِ؟»
\par 12 فَلَمَّا سَمِعَ يَسُوعُ قَالَ لَهُمْ: «لاَ يَحْتَاجُ الأَصِحَّاءُ إِلَى طَبِيبٍ بَلِ الْمَرْضَى.
\par 13 فَاذْهَبُوا وَتَعَلَّمُوا مَا هُوَ: إِنِّي أُرِيدُ رَحْمَةً لاَ ذَبِيحَةً لأَنِّي لَمْ آتِ لأَدْعُوَ أَبْرَاراً بَلْ خُطَاةً إِلَى التَّوْبَةِ».
\par 14 حِينَئِذٍ أَتَى إِلَيْهِ تَلاَمِيذُ يُوحَنَّا قَائِلِينَ: «لِمَاذَا نَصُومُ نَحْنُ وَالْفَرِّيسِيُّونَ كَثِيراً وَأَمَّا تَلاَمِيذُكَ فَلاَ يَصُومُونَ؟»
\par 15 فَقَالَ لَهُمْ يَسُوعُ: «هَلْ يَسْتَطِيعُ بَنُو الْعُرْسِ أَنْ يَنُوحُوا مَا دَامَ الْعَرِيسُ مَعَهُمْ؟ وَلَكِنْ سَتَأْتِي أَيَّامٌ حِينَ يُرْفَعُ الْعَرِيسُ عَنْهُمْ فَحِينَئِذٍ يَصُومُونَ.
\par 16 لَيْسَ أَحَدٌ يَجْعَلُ رُقْعَةً مِنْ قِطْعَةٍ جَدِيدَةٍ عَلَى ثَوْبٍ عَتِيقٍ لأَنَّ الْمِلْءَ يَأْخُذُ مِنَ الثَّوْبِ فَيَصِيرُ الْخَرْقُ أَرْدَأَ.
\par 17 وَلاَ يَجْعَلُونَ خَمْراً جَدِيدَةً فِي زِقَاقٍ عَتِيقَةٍ لِئَلَّا تَنْشَقَّ الزِّقَاقُ فَالْخَمْرُ تَنْصَبُّ وَالزِّقَاقُ تَتْلَفُ. بَلْ يَجْعَلُونَ خَمْراً جَدِيدَةً فِي زِقَاقٍ جَدِيدَةٍ فَتُحْفَظُ جَمِيعاً».
\par 18 وَفِيمَا هُوَ يُكَلِّمُهُمْ بِهَذَا إِذَا رَئِيسٌ قَدْ جَاءَ فَسَجَدَ لَهُ قَائِلاً: «إِنَّ ابْنَتِي الآنَ مَاتَتْ لَكِنْ تَعَالَ وَضَعْ يَدَكَ عَلَيْهَا فَتَحْيَا».
\par 19 فَقَامَ يَسُوعُ وَتَبِعَهُ هُوَ وَتَلاَمِيذُهُ.
\par 20 وَإِذَا امْرَأَةٌ نَازِفَةُ دَمٍ مُنْذُ اثْنَتَيْ عَشْرَةَ سَنَةً قَدْ جَاءَتْ مِنْ وَرَائِهِ وَمَسَّتْ هُدْبَ ثَوْبِهِ
\par 21 لأَنَّهَا قَالَتْ فِي نَفْسِهَا: «إِنْ مَسَسْتُ ثَوْبَهُ فَقَطْ شُفِيتُ».
\par 22 فَالْتَفَتَ يَسُوعُ وَأَبْصَرَهَا فَقَالَ: «ثِقِي يَا ابْنَةُ. إِيمَانُكِ قَدْ شَفَاكِ». فَشُفِيَتِ الْمَرْأَةُ مِنْ تِلْكَ السَّاعَةِ.
\par 23 وَلَمَّا جَاءَ يَسُوعُ إِلَى بَيْتِ الرَّئِيسِ وَنَظَرَ الْمُزَمِّرِينَ وَالْجَمْعَ يَضِجُّونَ
\par 24 قَالَ لَهُمْ: «تَنَحَّوْا فَإِنَّ الصَّبِيَّةَ لَمْ تَمُتْ لَكِنَّهَا نَائِمَةٌ». فَضَحِكُوا عَلَيْهِ.
\par 25 فَلَمَّا أُخْرِجَ الْجَمْعُ دَخَلَ وَأَمْسَكَ بِيَدِهَا فَقَامَتِ الصَّبِيَّةُ.
\par 26 فَخَرَجَ ذَلِكَ الْخَبَرُ إِلَى تِلْكَ الأَرْضِ كُلِّهَا.
\par 27 وَفِيمَا يَسُوعُ مُجْتَازٌ مِنْ هُنَاكَ تَبِعَهُ أَعْمَيَانِ يَصْرَخَانِ وَيَقُولاَنِ: «ارْحَمْنَا يَا ابْنَ دَاوُدَ».
\par 28 وَلَمَّا جَاءَ إِلَى الْبَيْتِ تَقَدَّمَ إِلَيْهِ الأَعْمَيَانِ فَقَالَ لَهُمَا يَسُوعُ: «أَتُؤْمِنَانِ أَنِّي أَقْدِرُ أَنْ أَفْعَلَ هَذَا؟» قَالاَ لَهُ: «نَعَمْ يَا سَيِّدُ».
\par 29 حِينَئِذٍ لَمَسَ أَعْيُنَهُمَا قَائِلاً: «بِحَسَبِ إِيمَانِكُمَا لِيَكُنْ لَكُمَا».
\par 30 فَانْفَتَحَتْ أَعْيُنُهُمَا. فَانْتَهَرَهُمَا يَسُوعُ قَائِلاً: «انْظُرَا لاَ يَعْلَمْ أَحَدٌ!»
\par 31 وَلَكِنَّهُمَا خَرَجَا وَأَشَاعَاهُ فِي تِلْكَ الأَرْضِ كُلِّهَا.
\par 32 وَفِيمَا هُمَا خَارِجَانِ إِذَا إِنْسَانٌ أَخْرَسُ مَجْنُونٌ قَدَّمُوهُ إِلَيْهِ.
\par 33 فَلَمَّا أُخْرِجَ الشَّيْطَانُ تَكَلَّمَ الأَخْرَسُ فَتَعَجَّبَ الْجُمُوعُ قَائِلِينَ: «لَمْ يَظْهَرْ قَطُّ مِثْلُ هَذَا فِي إِسْرَائِيلَ!»
\par 34 أَمَّا الْفَرِّيسِيُّونَ فَقَالُوا: «بِرَئِيسِ الشَّيَاطِينِ يُخْرِجُ الشَّيَاطِينَ».
\par 35 وَكَانَ يَسُوعُ يَطُوفُ الْمُدُنَ كُلَّهَا وَالْقُرَى يُعَلِّمُ فِي مَجَامِعِهَا وَيَكْرِزُ بِبِشَارَةِ الْمَلَكُوتِ وَيَشْفِي كُلَّ مَرَضٍ وَكُلَّ ضُعْفٍ فِي الشَّعْبِ.
\par 36 وَلَمَّا رَأَى الْجُمُوعَ تَحَنَّنَ عَلَيْهِمْ إِذْ كَانُوا مُنْزَعِجِينَ وَمُنْطَرِحِينَ كَغَنَمٍ لاَ رَاعِيَ لَهَا.
\par 37 حِينَئِذٍ قَالَ لِتَلاَمِيذِهِ: «الْحَصَادُ كَثِيرٌ وَلَكِنَّ الْفَعَلَةَ قَلِيلُونَ.
\par 38 فَاطْلُبُوا مِنْ رَبِّ الْحَصَادِ أَنْ يُرْسِلَ فَعَلَةً إِلَى حَصَادِهِ».

\chapter{10}

\par 1 ثُمَّ دَعَا تَلاَمِيذَهُ الاِثْنَيْ عَشَرَ وَأَعْطَاهُمْ سُلْطَاناً عَلَى أَرْوَاحٍ نَجِسَةٍ حَتَّى يُخْرِجُوهَا وَيَشْفُوا كُلَّ مَرَضٍ وَكُلَّ ضُعْفٍ.
\par 2 وَأَمَّا أَسْمَاءُ الاِثْنَيْ عَشَرَ رَسُولاً فَهِيَ هَذِهِ: الأَََوَّلُ سِمْعَانُ الَّذِي يُقَالُ لَهُ بُطْرُسُ وَأَنْدَرَاوُسُ أَخُوهُ. يَعْقُوبُ بْنُ زَبْدِي وَيُوحَنَّا أَخُوهُ.
\par 3 فِيلُبُّسُ وَبَرْثُولَمَاوُسُ. تُومَا وَمَتَّى الْعَشَّارُ. يَعْقُوبُ بْنُ حَلْفَى وَلَبَّاوُسُ الْمُلَقَّبُ تَدَّاوُسَ.
\par 4 سِمْعَانُ الْقَانَوِيُّ وَيَهُوذَا الإِسْخَرْيُوطِيُّ الَّذِي أَسْلَمَهُ.
\par 5 هَؤُلاَءِ الاِثْنَا عَشَرَ أَرْسَلَهُمْ يَسُوعُ وَأَوْصَاهُمْ قَائِلاً: «إِلَى طَرِيقِ أُمَمٍ لاَ تَمْضُوا وَإِلَى مَدِينَةٍ لِلسَّامِرِيِّينَ لاَ تَدْخُلُوا.
\par 6 بَلِ اذْهَبُوا بِالْحَرِيِّ إِلَى خِرَافِ بَيْتِ إِسْرَائِيلَ الضَّالَّةِ.
\par 7 وَفِيمَا أَنْتُمْ ذَاهِبُونَ اكْرِزُوا قَائِلِينَ: إِنَّهُ قَدِ اقْتَرَبَ مَلَكُوتُ السَّمَاوَاتِ.
\par 8 اشْفُوا مَرْضَى. طَهِّرُوا بُرْصاً. أَقِيمُوا مَوْتَى. أَخْرِجُوا شَيَاطِينَ. مَجَّاناً أَخَذْتُمْ مَجَّاناً أَعْطُوا.
\par 9 لاَ تَقْتَنُوا ذَهَباً وَلاَ فِضَّةً وَلاَ نُحَاساً فِي مَنَاطِقِكُمْ
\par 10 وَلاَ مِزْوَداً لِلطَّرِيقِ وَلاَ ثَوْبَيْنِ وَلاَ أَحْذِيَةً وَلاَ عَصاً لأَنَّ الْفَاعِلَ مُسْتَحِقٌّ طَعَامَهُ.
\par 11 «وَأَيَّةُ مَدِينَةٍ أَوْ قَرْيَةٍ دَخَلْتُمُوهَا فَافْحَصُوا مَنْ فِيهَا مُسْتَحِقٌّ وَأَقِيمُوا هُنَاكَ حَتَّى تَخْرُجُوا.
\par 12 وَحِينَ تَدْخُلُونَ الْبَيْتَ سَلِّمُوا عَلَيْهِ
\par 13 فَإِنْ كَانَ الْبَيْتُ مُسْتَحِقّاً فَلْيَأْتِ سَلاَمُكُمْ عَلَيْهِ وَلَكِنْ إِنْ لَمْ يَكُنْ مُسْتَحِقّاً فَلْيَرْجِعْ سَلاَمُكُمْ إِلَيْكُمْ.
\par 14 وَمَنْ لاَ يَقْبَلُكُمْ وَلاَ يَسْمَعُ كَلاَمَكُمْ فَاخْرُجُوا خَارِجاً مِنْ ذَلِكَ الْبَيْتِ أَوْ مِنْ تِلْكَ الْمَدِينَةِ وَانْفُضُوا غُبَارَ أَرْجُلِكُمْ.
\par 15 اَلْحَقَّ أَقُولُ لَكُمْ: سَتَكُونُ لأَرْضِ سَدُومَ وَعَمُورَةَ يَوْمَ الدِّينِ حَالَةٌ أَكْثَرُ احْتِمَالاً مِمَّا لِتِلْكَ الْمَدِينَةِ.
\par 16 «هَا أَنَا أُرْسِلُكُمْ كَغَنَمٍ فِي وَسَطِ ذِئَابٍ فَكُونُوا حُكَمَاءَ كَالْحَيَّاتِ وَبُسَطَاءَ كَالْحَمَامِ.
\par 17 وَلَكِنِ احْذَرُوا مِنَ النَّاسِ لأَنَّهُمْ سَيُسْلِمُونَكُمْ إِلَى مَجَالِسَ وَفِي مَجَامِعِهِمْ يَجْلِدُونَكُمْ.
\par 18 وَتُسَاقُونَ أَمَامَ وُلاَةٍ وَمُلُوكٍ مِنْ أَجْلِي شَهَادَةً لَهُمْ وَلِلأُمَمِ.
\par 19 فَمَتَى أَسْلَمُوكُمْ فَلاَ تَهْتَمُّوا كَيْفَ أَوْ بِمَا تَتَكَلَّمُونَ لأَنَّكُمْ تُعْطَوْنَ فِي تِلْكَ السَّاعَةِ مَا تَتَكَلَّمُونَ بِهِ
\par 20 لأَنْ لَسْتُمْ أَنْتُمُ الْمُتَكَلِّمِينَ بَلْ رُوحُ أَبِيكُمُ الَّذِي يَتَكَلَّمُ فِيكُمْ.
\par 21 وَسَيُسْلِمُ الأَخُ أَخَاهُ إِلَى الْمَوْتِ وَالأَبُ وَلَدَهُ وَيَقُومُ الأَوْلاَدُ عَلَى وَالِدِيهِمْ وَيَقْتُلُونَهُمْ
\par 22 وَتَكُونُونَ مُبْغَضِينَ مِنَ الْجَمِيعِ مِنْ أَجْلِ اسْمِي. وَلَكِنِ الَّذِي يَصْبِرُ إِلَى الْمُنْتَهَى فَهَذَا يَخْلُصُ.
\par 23 وَمَتَى طَرَدُوكُمْ فِي هَذِهِ الْمَدِينَةِ فَاهْرُبُوا إِلَى الأُخْرَى. فَإِنِّي الْحَقَّ أَقُولُ لَكُمْ لاَ تُكَمِّلُونَ مُدُنَ إِسْرَائِيلَ حَتَّى يَأْتِيَ ابْنُ الإِنْسَانِ.
\par 24 «لَيْسَ التِّلْمِيذُ أَفْضَلَ مِنَ الْمُعَلِّمِ وَلاَ الْعَبْدُ أَفْضَلَ مِنْ سَيِّدِهِ.
\par 25 يَكْفِي التِّلْمِيذَ أَنْ يَكُونَ كَمُعَلِّمِهِ وَالْعَبْدَ كَسَيِّدِهِ. إِنْ كَانُوا قَدْ لَقَّبُوا رَبَّ الْبَيْتِ بَعْلَزَبُولَ فَكَمْ بِالْحَرِيِّ أَهْلَ بَيْتِهِ!
\par 26 فَلاَ تَخَافُوهُمْ. لأَنْ لَيْسَ مَكْتُومٌ لَنْ يُسْتَعْلَنَ وَلاَ خَفِيٌّ لَنْ يُعْرَفَ.
\par 27 اَلَّذِي أَقُولُهُ لَكُمْ فِي الظُّلْمَةِ قُولُوهُ فِي النُّورِ وَالَّذِي تَسْمَعُونَهُ فِي الأُذُنِ نَادُوا بِهِ عَلَى السُّطُوحِ
\par 28 وَلاَ تَخَافُوا مِنَ الَّذِينَ يَقْتُلُونَ الْجَسَدَ وَلَكِنَّ النَّفْسَ لاَ يَقْدِرُونَ أَنْ يَقْتُلُوهَا بَلْ خَافُوا بِالْحَرِيِّ مِنَ الَّذِي يَقْدِرُ أَنْ يُهْلِكَ النَّفْسَ وَالْجَسَدَ كِلَيْهِمَا فِي جَهَنَّمَ.
\par 29 أَلَيْسَ عُصْفُورَانِ يُبَاعَانِ بِفَلْسٍ؟ وَوَاحِدٌ مِنْهُمَا لاَ يَسْقُطُ عَلَى الأَرْضِ بِدُونِ أَبِيكُمْ.
\par 30 وَأَمَّا أَنْتُمْ فَحَتَّى شُعُورُ رُؤُوسِكُمْ جَمِيعُهَا مُحْصَاةٌ.
\par 31 فَلاَ تَخَافُوا. أَنْتُمْ أَفْضَلُ مِنْ عَصَافِيرَ كَثِيرَةٍ.
\par 32 فَكُلُّ مَنْ يَعْتَرِفُ بِي قُدَّامَ النَّاسِ أَعْتَرِفُ أَنَا أَيْضاً بِهِ قُدَّامَ أَبِي الَّذِي فِي السَّمَاوَاتِ
\par 33 وَلَكِنْ مَنْ يُنْكِرُنِي قُدَّامَ النَّاسِ أُنْكِرُهُ أَنَا أَيْضاً قُدَّامَ أَبِي الَّذِي فِي السَّمَاوَاتِ.
\par 34 «لاَ تَظُنُّوا أَنِّي جِئْتُ لِأُلْقِيَ سَلاَماً عَلَى الأَرْضِ. مَا جِئْتُ لِأُلْقِيَ سَلاَماً بَلْ سَيْفاً.
\par 35 فَإِنِّي جِئْتُ لِأُفَرِّقَ الإِنْسَانَ ضِدَّ أَبِيهِ وَالاِبْنَةَ ضِدَّ أُمِّهَا وَالْكَنَّةَ ضِدَّ حَمَاتِهَا.
\par 36 وَأَعْدَاءُ الإِنْسَانِ أَهْلُ بَيْتِهِ.
\par 37 مَنْ أَحَبَّ أَباً أَوْ أُمّاً أَكْثَرَ مِنِّي فَلاَ يَسْتَحِقُّنِي وَمَنْ أَحَبَّ ابْناً أَوِ ابْنَةً أَكْثَرَ مِنِّي فَلاَ يَسْتَحِقُّنِي
\par 38 وَمَنْ لاَ يَأْخُذُ صَلِيبَهُ وَيَتْبَعُنِي فَلاَ يَسْتَحِقُّنِي.
\par 39 مَنْ وَجَدَ حَيَاتَهُ يُضِيعُهَا وَمَنْ أَضَاعَ حَيَاتَهُ مِنْ أَجْلِي يَجِدُهَا.
\par 40 مَنْ يَقْبَلُكُمْ يَقْبَلُنِي وَمَنْ يَقْبَلُنِي يَقْبَلُ الَّذِي أَرْسَلَنِي.
\par 41 مَنْ يَقْبَلُ نَبِيّاً بِاسْمِ نَبِيٍّ فَأَجْرَ نَبِيٍّ يَأْخُذُ وَمَنْ يَقْبَلُ بَارّاً بِاسْمِ بَارٍّ فَأَجْرَ بَارٍّ يَأْخُذُ
\par 42 وَمَنْ سَقَى أَحَدَ هَؤُلاَءِ الصِّغَارِ كَأْسَ مَاءٍ بَارِدٍ فَقَطْ بِاسْمِ تِلْمِيذٍ فَالْحَقَّ أَقُولُ لَكُمْ إِنَّهُ لاَ يُضِيعُ أَجْرَهُ».

\chapter{11}

\par 1 وَلَمَّا أَكْمَلَ يَسُوعُ أَمْرَهُ لِتَلاَمِيذِهِ الاِثْنَيْ عَشَرَ انْصَرَفَ مِنْ هُنَاكَ لِيُعَلِّمَ وَيَكْرِزَ فِي مُدُنِهِمْ.
\par 2 أَمَّا يُوحَنَّا فَلَمَّا سَمِعَ فِي السِّجْنِ بِأَعْمَالِ الْمَسِيحِ أَرْسَلَ اثْنَيْنِ مِنْ تَلاَمِيذِهِ
\par 3 وَقَالَ لَهُ: «أَنْتَ هُوَ الآتِي أَمْ نَنْتَظِرُ آخَرَ؟»
\par 4 فَأَجَابَهُمَا يَسُوعُ: «اذْهَبَا وَأَخْبِرَا يُوحَنَّا بِمَا تَسْمَعَانِ وَتَنْظُرَانِ:
\par 5 اَلْعُمْيُ يُبْصِرُونَ وَالْعُرْجُ يَمْشُونَ وَالْبُرْصُ يُطَهَّرُونَ وَالصُّمُّ يَسْمَعُونَ وَالْمَوْتَى يَقُومُونَ وَالْمَسَاكِينُ يُبَشَّرُونَ.
\par 6 وَطُوبَى لِمَنْ لاَ يَعْثُرُ فِيَّ».
\par 7 وَبَيْنَمَا ذَهَبَ هَذَانِ ابْتَدَأَ يَسُوعُ يَقُولُ لِلْجُمُوعِ عَنْ يُوحَنَّا: «مَاذَا خَرَجْتُمْ إِلَى الْبَرِّيَّةِ لِتَنْظُرُوا؟ أَقَصَبَةً تُحَرِّكُهَا الرِّيحُ؟
\par 8 لَكِنْ مَاذَا خَرَجْتُمْ لِتَنْظُرُوا؟ أَإِنْسَاناً لاَبِساً ثِيَاباً نَاعِمَةً؟ هُوَذَا الَّذِينَ يَلْبَسُونَ الثِّيَابَ النَّاعِمَةَ هُمْ فِي بُيُوتِ الْمُلُوكِ.
\par 9 لَكِنْ مَاذَا خَرَجْتُمْ لِتَنْظُرُوا؟ أَنَبِيّاً؟ نَعَمْ أَقُولُ لَكُمْ وَأَفْضَلَ مِنْ نَبِيٍّ.
\par 10 فَإِنَّ هَذَا هُوَ الَّذِي كُتِبَ عَنْهُ: هَا أَنَا أُرْسِلُ أَمَامَ وَجْهِكَ مَلاَكِي الَّذِي يُهَيِّئُ طَرِيقَكَ قُدَّامَكَ.
\par 11 اَلْحَقَّ أَقُولُ لَكُمْ: لَمْ يَقُمْ بَيْنَ الْمَوْلُودِينَ مِنَ النِّسَاءِ أَعْظَمُ مِنْ يُوحَنَّا الْمَعْمَدَانِ وَلَكِنَّ الأَصْغَرَ فِي مَلَكُوتِ السَّمَاوَاتِ أَعْظَمُ مِنْهُ.
\par 12 وَمِنْ أَيَّامِ يُوحَنَّا الْمَعْمَدَانِ إِلَى الآنَ مَلَكُوتُ السَّمَاوَاتِ يُغْصَبُ وَالْغَاصِبُونَ يَخْتَطِفُونَهُ.
\par 13 لأَنَّ جَمِيعَ الأَنْبِيَاءِ وَالنَّامُوسَ إِلَى يُوحَنَّا تَنَبَّأُوا.
\par 14 وَإِنْ أَرَدْتُمْ أَنْ تَقْبَلُوا فَهَذَا هُوَ إِيلِيَّا الْمُزْمِعُ أَنْ يَأْتِيَ.
\par 15 مَنْ لَهُ أُذُنَانِ لِلسَّمْعِ فَلْيَسْمَعْ.
\par 16 «وَبِمَنْ أُشَبِّهُ هَذَا الْجِيلَ؟ يُشْبِهُ أَوْلاَداً جَالِسِينَ فِي الأَسْوَاقِ يُنَادُونَ إِلَى أَصْحَابِهِمْ
\par 17 وَيَقُولُونَ: زَمَّرْنَا لَكُمْ فَلَمْ تَرْقُصُوا! نُحْنَا لَكُمْ فَلَمْ تَلْطِمُوا!
\par 18 لأَنَّهُ جَاءَ يُوحَنَّا لاَ يَأْكُلُ وَلاَ يَشْرَبُ فَيَقُولُونَ: فِيهِ شَيْطَانٌ.
\par 19 جَاءَ ابْنُ الإِنْسَانِ يَأْكُلُ وَيَشْرَبُ فَيَقُولُونَ: هُوَذَا إِنْسَانٌ أَكُولٌ وَشِرِّيبُ خَمْرٍ مُحِبٌّ لِلْعَشَّارِينَ وَالْخُطَاةِ. وَالْحِكْمَةُ تَبَرَّرَتْ مِنْ بَنِيهَا».
\par 20 حِينَئِذٍ ابْتَدَأَ يُوَبِّخُ الْمُدُنَ الَّتِي صُنِعَتْ فِيهَا أَكْثَرُ قُوَّاتِهِ لأَنَّهَا لَمْ تَتُبْ:
\par 21 «وَيْلٌ لَكِ يَا كُورَزِينُ! وَيْلٌ لَكِ يَا بَيْتَ صَيْدَا! لأَنَّهُ لَوْ صُنِعَتْ فِي صُورَ وَصَيْدَاءَ الْقُوَّاتُ الْمَصْنُوعَةُ فِيكُمَا لَتَابَتَا قَدِيماً فِي الْمُسُوحِ وَالرَّمَادِ.
\par 22 وَلَكِنْ أَقُولُ لَكُمْ: إِنَّ صُورَ وَصَيْدَاءَ تَكُونُ لَهُمَا حَالَةٌ أَكْثَرُ احْتِمَالاً يَوْمَ الدِّينِ مِمَّا لَكُمَا.
\par 23 وَأَنْتِ يَا كَفْرَنَاحُومَ الْمُرْتَفِعَةَ إِلَى السَّمَاءِ سَتُهْبَطِينَ إِلَى الْهَاوِيَةِ. لأَنَّهُ لَوْ صُنِعَتْ فِي سَدُومَ الْقُوَّاتُ الْمَصْنُوعَةُ فِيكِ لَبَقِيَتْ إِلَى الْيَوْمِ.
\par 24 وَلَكِنْ أَقُولُ لَكُمْ: إِنَّ أَرْضَ سَدُومَ تَكُونُ لَهَا حَالَةٌ أَكْثَرُ احْتِمَالاً يَوْمَ الدِّينِ مِمَّا لَكِ».
\par 25 فِي ذَلِكَ الْوَقْتِ قَالَ يَسُوعُ: «أَحْمَدُكَ أَيُّهَا الآبُ رَبُّ السَّمَاءِ وَالأَرْضِ لأَنَّكَ أَخْفَيْتَ هَذِهِ عَنِ الْحُكَمَاءِ وَالْفُهَمَاءِ وَأَعْلَنْتَهَا لِلأَطْفَالِ.
\par 26 نَعَمْ أَيُّهَا الآبُ لأَنْ هَكَذَا صَارَتِ الْمَسَرَّةُ أَمَامَكَ.
\par 27 كُلُّ شَيْءٍ قَدْ دُفِعَ إِلَيَّ مِنْ أَبِي وَلَيْسَ أَحَدٌ يَعْرِفُ الاِبْنَ إِلاَّ الآبُ وَلاَ أَحَدٌ يَعْرِفُ الآبَ إِلاَّ الاِبْنُ وَمَنْ أَرَادَ الاِبْنُ أَنْ يُعْلِنَ لَهُ.
\par 28 تَعَالَوْا إِلَيَّ يَا جَمِيعَ الْمُتْعَبِينَ وَالثَّقِيلِي الأَحْمَالِ وَأَنَا أُرِيحُكُمْ.
\par 29 اِحْمِلُوا نِيرِي عَلَيْكُمْ وَتَعَلَّمُوا مِنِّي لأَنِّي وَدِيعٌ وَمُتَوَاضِعُ الْقَلْبِ فَتَجِدُوا رَاحَةً لِنُفُوسِكُمْ.
\par 30 لأَنَّ نِيرِي هَيِّنٌ وَحِمْلِي خَفِيفٌ».

\chapter{12}

\par 1 فِي ذَلِكَ الْوَقْتِ ذَهَبَ يَسُوعُ فِي السَّبْتِ بَيْنَ الزُّرُوعِ فَجَاعَ تَلاَمِيذُهُ وَابْتَدَأُوا يَقْطِفُونَ سَنَابِلَ وَيَأْكُلُونَ.
\par 2 فَالْفَرِّيسِيُّونَ لَمَّا نَظَرُوا قَالُوا لَهُ: «هُوَذَا تَلاَمِيذُكَ يَفْعَلُونَ مَا لاَ يَحِلُّ فِعْلُهُ فِي السَّبْتِ!»
\par 3 فَقَالَ لَهُمْ: «أَمَا قَرَأْتُمْ مَا فَعَلَهُ دَاوُدُ حِينَ جَاعَ هُوَ وَالَّذِينَ مَعَهُ
\par 4 كَيْفَ دَخَلَ بَيْتَ اللَّهِ وَأَكَلَ خُبْزَ التَّقْدِمَةِ الَّذِي لَمْ يَحِلَّ أَكْلُهُ لَهُ وَلاَ لِلَّذِينَ مَعَهُ بَلْ لِلْكَهَنَةِ فَقَطْ؟
\par 5 أَوَ مَا قَرَأْتُمْ فِي التَّوْرَاةِ أَنَّ الْكَهَنَةَ فِي السَّبْتِ فِي الْهَيْكَلِ يُدَنِّسُونَ السَّبْتَ وَهُمْ أَبْرِيَاءُ؟
\par 6 وَلَكِنْ أَقُولُ لَكُمْ: إِنَّ هَهُنَا أَعْظَمَ مِنَ الْهَيْكَلِ!
\par 7 فَلَوْ عَلِمْتُمْ مَا هُوَ: إِنِّي أُرِيدُ رَحْمَةً لاَ ذَبِيحَةً لَمَا حَكَمْتُمْ عَلَى الأَبْرِيَاءِ!
\par 8 فَإِنَّ ابْنَ الإِنْسَانِ هُوَ رَبُّ السَّبْتِ أَيْضاً».
\par 9 ثُمَّ انْصَرَفَ مِنْ هُنَاكَ وَجَاءَ إِلَى مَجْمَعِهِمْ
\par 10 وَإِذَا إِنْسَانٌ يَدُهُ يَابِسَةٌ فَسَأَلُوهُ: «هَلْ يَحِلُّ الإِبْرَاءُ فِي السُّبُوتِ؟» لِكَيْ يَشْتَكُوا عَلَيْهِ.
\par 11 فَقَالَ لَهُمْ: «أَيُّ إِنْسَانٍ مِنْكُمْ يَكُونُ لَهُ خَرُوفٌ وَاحِدٌ فَإِنْ سَقَطَ هَذَا فِي السَّبْتِ فِي حُفْرَةٍ أَفَمَا يُمْسِكُهُ وَيُقِيمُهُ؟
\par 12 فَالإِنْسَانُ كَمْ هُوَ أَفْضَلُ مِنَ الْخَرُوفِ! إِذاً يَحِلُّ فِعْلُ الْخَيْرِ فِي السُّبُوتِ!»
\par 13 ثُمَّ قَالَ لِلإِنْسَانِ: «مُدَّ يَدَكَ». فَمَدَّهَا. فَعَادَتْ صَحِيحَةً كَالأُخْرَى.
\par 14 فَلَمَّا خَرَجَ الْفَرِّيسِيُّونَ تَشَاوَرُوا عَلَيْهِ لِكَيْ يُهْلِكُوهُ
\par 15 فَعَلِمَ يَسُوعُ وَانْصَرَفَ مِنْ هُنَاكَ. وَتَبِعَتْهُ جُمُوعٌ كَثِيرَةٌ فَشَفَاهُمْ جَمِيعاً.
\par 16 وَأَوْصَاهُمْ أَنْ لاَ يُظْهِرُوهُ
\par 17 لِكَيْ يَتِمَّ مَا قِيلَ بِإِشَعْيَاءَ النَّبِيِّ:
\par 18 «هُوَذَا فَتَايَ الَّذِي اخْتَرْتُهُ حَبِيبِي الَّذِي سُرَّتْ بِهِ نَفْسِي. أَضَعُ رُوحِي عَلَيْهِ فَيُخْبِرُ الأُمَمَ بِالْحَقِّ.
\par 19 لاَ يُخَاصِمُ وَلاَ يَصِيحُ وَلاَ يَسْمَعُ أَحَدٌ فِي الشَّوَارِعِ صَوْتَهُ.
\par 20 قَصَبَةً مَرْضُوضَةً لاَ يَقْصِفُ وَفَتِيلَةً مُدَخِّنَةً لاَ يُطْفِئُ حَتَّى يُخْرِجَ الْحَقَّ إِلَى النُّصْرَةِ.
\par 21 وَعَلَى اسْمِهِ يَكُونُ رَجَاءُ الأُمَمِ».
\par 22 حِينَئِذٍ أُحْضِرَ إِلَيْهِ مَجْنُونٌ أَعْمَى وَأَخْرَسُ فَشَفَاهُ حَتَّى إِنَّ الأَعْمَى الأَخْرَسَ تَكَلَّمَ وَأَبْصَرَ.
\par 23 فَبُهِتَ كُلُّ الْجُمُوعِ وَقَالُوا: «أَلَعَلَّ هَذَا هُوَ ابْنُ دَاوُدَ؟»
\par 24 أَمَّا الْفَرِّيسِيُّونَ فَلَمَّا سَمِعُوا قَالُوا: «هَذَا لاَ يُخْرِجُ الشَّيَاطِينَ إِلاَّ بِبَعْلَزَبُولَ رَئِيسِ الشَّيَاطِينِ».
\par 25 فَعَلِمَ يَسُوعُ أَفْكَارَهُمْ وَقَالَ لَهُمْ: «كُلُّ مَمْلَكَةٍ مُنْقَسِمَةٍ عَلَى ذَاتِهَا تُخْرَبُ وَكُلُّ مَدِينَةٍ أَوْ بَيْتٍ مُنْقَسِمٍ عَلَى ذَاتِهِ لاَ يَثْبُتُ.
\par 26 فَإِنْ كَانَ الشَّيْطَانُ يُخْرِجُ الشَّيْطَانَ فَقَدِ انْقَسَمَ عَلَى ذَاتِهِ. فَكَيْفَ تَثْبُتُ مَمْلَكَتُهُ؟
\par 27 وَإِنْ كُنْتُ أَنَا بِبَعْلَزَبُولَ أُخْرِجُ الشَّيَاطِينَ فَأَبْنَاؤُكُمْ بِمَنْ يُخْرِجُونَ؟ لِذَلِكَ هُمْ يَكُونُونَ قُضَاتَكُمْ!
\par 28 وَلَكِنْ إِنْ كُنْتُ أَنَا بِرُوحِ اللَّهِ أُخْرِجُ الشَّيَاطِينَ فَقَدْ أَقْبَلَ عَلَيْكُمْ مَلَكُوتُ اللَّهِ!
\par 29 أَمْ كَيْفَ يَسْتَطِيعُ أَحَدٌ أَنْ يَدْخُلَ بَيْتَ الْقَوِيِّ وَيَنْهَبَ أَمْتِعَتَهُ إِنْ لَمْ يَرْبِطِ الْقَوِيَّ أَوَّلاً وَحِينَئِذٍ يَنْهَبُ بَيْتَهُ؟
\par 30 مَنْ لَيْسَ مَعِي فَهُوَ عَلَيَّ وَمَنْ لاَ يَجْمَعُ مَعِي فَهُوَ يُفَرِّقُ.
\par 31 لِذَلِكَ أَقُولُ لَكُمْ: كُلُّ خَطِيَّةٍ وَتَجْدِيفٍ يُغْفَرُ لِلنَّاسِ وَأَمَّا التَّجْدِيفُ عَلَى الرُّوحِ فَلَنْ يُغْفَرَ لِلنَّاسِ.
\par 32 وَمَنْ قَالَ كَلِمَةً عَلَى ابْنِ الإِنْسَانِ يُغْفَرُ لَهُ وَأَمَّا مَنْ قَالَ عَلَى الرُّوحِ الْقُدُسِ فَلَنْ يُغْفَرَ لَهُ لاَ فِي هَذَا الْعَالَمِ وَلاَ فِي الآتِي.
\par 33 اِجْعَلُوا الشَّجَرَةَ جَيِّدَةً وَثَمَرَهَا جَيِّداً أَوِ اجْعَلُوا الشَّجَرَةَ رَدِيَّةً وَثَمَرَهَا رَدِيّاً لأَنْ مِنَ الثَّمَرِ تُعْرَفُ الشَّجَرَةُ.
\par 34 يَا أَوْلاَدَ الأَفَاعِي! كَيْفَ تَقْدِرُونَ أَنْ تَتَكَلَّمُوا بِالصَّالِحَاتِ وَأَنْتُمْ أَشْرَارٌ؟ فَإِنَّهُ مِنْ فَضْلَةِ الْقَلْبِ يَتَكَلَّمُ الْفَمُ.
\par 35 اَلإِنْسَانُ الصَّالِحُ مِنَ الْكَنْزِ الصَّالِحِ فِي الْقَلْبِ يُخْرِجُ الصَّالِحَاتِ وَالإِنْسَانُ الشِّرِّيرُ مِنَ الْكَنْزِ الشِّرِّيرِ يُخْرِجُ الشُّرُورَ.
\par 36 وَلَكِنْ أَقُولُ لَكُمْ: إِنَّ كُلَّ كَلِمَةٍ بَطَّالَةٍ يَتَكَلَّمُ بِهَا النَّاسُ سَوْفَ يُعْطُونَ عَنْهَا حِسَاباً يَوْمَ الدِّينِ.
\par 37 لأَنَّكَ بِكَلاَمِكَ تَتَبَرَّرُ وَبِكَلاَمِكَ تُدَانُ».
\par 38 حِينَئِذٍ قَالَ قَوْمٌ مِنَ الْكَتَبَةِ وَالْفَرِّيسِيِّينَ: «يَا مُعَلِّمُ نُرِيدُ أَنْ نَرَى مِنْكَ آيَةً».
\par 39 فَقَالَ لَهُمْ: «جِيلٌ شِرِّيرٌ وَفَاسِقٌ يَطْلُبُ آيَةً وَلاَ تُعْطَى لَهُ آيَةٌ إِلاَّ آيَةَ يُونَانَ النَّبِيِّ.
\par 40 لأَنَّهُ كَمَا كَانَ يُونَانُ فِي بَطْنِ الْحُوتِ ثَلاَثَةَ أَيَّامٍ وَثَلاَثَ لَيَالٍ هَكَذَا يَكُونُ ابْنُ الإِنْسَانِ فِي قَلْبِ الأَرْضِ ثَلاَثَةَ أَيَّامٍ وَثَلاَثَ لَيَالٍ.
\par 41 رِجَالُ نِينَوَى سَيَقُومُونَ فِي الدِّينِ مَعَ هَذَا الْجِيلِ وَيَدِينُونَهُ لأَنَّهُمْ تَابُوا بِمُنَادَاةِ يُونَانَ وَهُوَذَا أَعْظَمُ مِنْ يُونَانَ هَهُنَا!
\par 42 مَلِكَةُ التَّيْمَنِ سَتَقُومُ فِي الدِّينِ مَعَ هَذَا الْجِيلِ وَتَدِينُهُ لأَنَّهَا أَتَتْ مِنْ أَقَاصِي الأَرْضِ لِتَسْمَعَ حِكْمَةَ سُلَيْمَانَ وَهُوَذَا أَعْظَمُ مِنْ سُلَيْمَانَ هَهُنَا!
\par 43 إِذَا خَرَجَ الرُّوحُ النَّجِسُ مِنَ الإِنْسَانِ يَجْتَازُ فِي أَمَاكِنَ لَيْسَ فِيهَا مَاءٌ يَطْلُبُ رَاحَةً وَلاَ يَجِدُ.
\par 44 ثُمَّ يَقُولُ: أَرْجِعُ إِلَى بَيْتِي الَّذِي خَرَجْتُ مِنْهُ. فَيَأْتِي وَيَجِدُهُ فَارِغاً مَكْنُوساً مُزَيَّناً.
\par 45 ثُمَّ يَذْهَبُ وَيَأْخُذُ مَعَهُ سَبْعَةَ أَرْوَاحٍ أُخَرَ أَشَرَّ مِنْهُ فَتَدْخُلُ وَتَسْكُنُ هُنَاكَ فَتَصِيرُ أَوَاخِرُ ذَلِكَ الإِنْسَانِ أَشَرَّ مِنْ أَوَائِلِهِ. هَكَذَا يَكُونُ أَيْضاً لِهَذَا الْجِيلِ الشَّرِّيرِ».
\par 46 وَفِيمَا هُوَ يُكَلِّمُ الْجُمُوعَ إِذَا أُمُّهُ وَإِخْوَتُهُ قَدْ وَقَفُوا خَارِجاً طَالِبِينَ أَنْ يُكَلِّمُوهُ.
\par 47 فَقَالَ لَهُ وَاحِدٌ: «هُوَذَا أُمُّكَ وَإِخْوَتُكَ وَاقِفُونَ خَارِجاً طَالِبِينَ أَنْ يُكَلِّمُوكَ».
\par 48 فَأَجَابَهُ: «مَنْ هِيَ أُمِّي وَمَنْ هُمْ إِخْوَتِي؟»
\par 49 ثُمَّ مَدَّ يَدَهُ نَحْوَ تَلاَمِيذِهِ وَقَالَ: «هَا أُمِّي وَإِخْوَتِي.
\par 50 لأَنَّ مَنْ يَصْنَعُ مَشِيئَةَ أَبِي الَّذِي فِي السَّمَاوَاتِ هُوَ أَخِي وَأُخْتِي وَأُمِّي».

\chapter{13}

\par 1 فِي ذَلِكَ الْيَوْمِ خَرَجَ يَسُوعُ مِنَ الْبَيْتِ وَجَلَسَ عِنْدَ الْبَحْرِ
\par 2 فَاجْتَمَعَ إِلَيْهِ جُمُوعٌ كَثِيرَةٌ حَتَّى إِنَّهُ دَخَلَ السَّفِينَةَ وَجَلَسَ. وَالْجَمْعُ كُلُّهُ وَقَفَ عَلَى الشَّاطِئِ.
\par 3 فَكَلَّمَهُمْ كَثِيراً بِأَمْثَالٍ قَائِلاً: «هُوَذَا الزَّارِعُ قَدْ خَرَجَ لِيَزْرَعَ
\par 4 وَفِيمَا هُوَ يَزْرَعُ سَقَطَ بَعْضٌ عَلَى الطَّرِيقِ فَجَاءَتِ الطُّيُورُ وَأَكَلَتْهُ.
\par 5 وَسَقَطَ آخَرُ عَلَى الأَمَاكِنِ الْمُحْجِرَةِ حَيْثُ لَمْ تَكُنْ لَهُ تُرْبَةٌ كَثِيرَةٌ فَنَبَتَ حَالاً إِذْ لَمْ يَكُنْ لَهُ عُمْقُ أَرْضٍ.
\par 6 وَلَكِنْ لَمَّا أَشْرَقَتِ الشَّمْسُ احْتَرَقَ وَإِذْ لَمْ يَكُنْ لَهُ أَصْلٌ جَفَّ.
\par 7 وَسَقَطَ آخَرُ عَلَى الشَّوْكِ فَطَلَعَ الشَّوْكُ وَخَنَقَهُ.
\par 8 وَسَقَطَ آخَرُ عَلَى الأَرْضِ الْجَيِّدَةِ فَأَعْطَى ثَمَراً بَعْضٌ مِئَةً وَآخَرُ سِتِّينَ وَآخَرُ ثَلاَثِينَ.
\par 9 مَنْ لَهُ أُذُنَانِ لِلسَّمْعِ فَلْيَسْمَعْ»
\par 10 فَتَقَدَّمَ التَّلاَمِيذُ وَقَالُوا لَهُ: «لِمَاذَا تُكَلِّمُهُمْ بِأَمْثَالٍ؟»
\par 11 فَأَجَابَ: «لأَنَّهُ قَدْ أُعْطِيَ لَكُمْ أَنْ تَعْرِفُوا أَسْرَارَ مَلَكُوتِ السَّمَاوَاتِ وَأَمَّا لِأُولَئِكَ فَلَمْ يُعْطَ.
\par 12 فَإِنَّ مَنْ لَهُ سَيُعْطَى وَيُزَادُ وَأَمَّا مَنْ لَيْسَ لَهُ فَالَّذِي عِنْدَهُ سَيُؤْخَذُ مِنْهُ.
\par 13 مِنْ أَجْلِ هَذَا أُكَلِّمُهُمْ بِأَمْثَالٍ لأَنَّهُمْ مُبْصِرِينَ لاَ يُبْصِرُونَ وَسَامِعِينَ لاَ يَسْمَعُونَ وَلاَ يَفْهَمُونَ.
\par 14 فَقَدْ تَمَّتْ فِيهِمْ نُبُوَّةُ إِشَعْيَاءَ: تَسْمَعُونَ سَمْعاً وَلاَ تَفْهَمُونَ وَمُبْصِرِينَ تُبْصِرُونَ وَلاَ تَنْظُرُونَ.
\par 15 لأَنَّ قَلْبَ هَذَا الشَّعْبِ قَدْ غَلُظَ وَآذَانَهُمْ قَدْ ثَقُلَ سَمَاعُهَا. وَغَمَّضُوا عُيُونَهُمْ لِئَلَّا يُبْصِرُوا بِعُيُونِهِمْ وَيَسْمَعُوا بِآذَانِهِمْ وَيَفْهَمُوا بِقُلُوبِهِمْ وَيَرْجِعُوا فَأَشْفِيَهُمْ.
\par 16 وَلَكِنْ طُوبَى لِعُيُونِكُمْ لأَنَّهَا تُبْصِرُ وَلِآذَانِكُمْ لأَنَّهَا تَسْمَعُ.
\par 17 فَإِنِّي الْحَقَّ أَقُولُ لَكُمْ: إِنَّ أَنْبِيَاءَ وَأَبْرَاراً كَثِيرِينَ اشْتَهَوْا أَنْ يَرَوْا مَا أَنْتُمْ تَرَوْنَ وَلَمْ يَرَوْا وَأَنْ يَسْمَعُوا مَا أَنْتُمْ تَسْمَعُونَ وَلَمْ يَسْمَعُوا.
\par 18 «فَاسْمَعُوا أَنْتُمْ مَثَلَ الزَّارِعِ:
\par 19 كُلُّ مَنْ يَسْمَعُ كَلِمَةَ الْمَلَكُوتِ وَلاَ يَفْهَمُ فَيَأْتِي الشِّرِّيرُ وَيَخْطَفُ مَا قَدْ زُرِعَ فِي قَلْبِهِ. هَذَا هُوَ الْمَزْرُوعُ عَلَى الطَّرِيقِ.
\par 20 وَالْمَزْرُوعُ عَلَى الأَمَاكِنِ الْمُحْجِرَةِ هُوَ الَّذِي يَسْمَعُ الْكَلِمَةَ وَحَالاً يَقْبَلُهَا بِفَرَحٍ
\par 21 وَلَكِنْ لَيْسَ لَهُ أَصْلٌ فِي ذَاتِهِ بَلْ هُوَ إِلَى حِينٍ. فَإِذَا حَدَثَ ضِيقٌ أَوِ اضْطِهَادٌ مِنْ أَجْلِ الْكَلِمَةِ فَحَالاً يَعْثُرُ.
\par 22 وَالْمَزْرُوعُ بَيْنَ الشَّوْكِ هُوَ الَّذِي يَسْمَعُ الْكَلِمَةَ وَهَمُّ هَذَا الْعَالَمِ وَغُرُورُ الْغِنَى يَخْنُقَانِ الْكَلِمَةَ فَيَصِيرُ بِلاَ ثَمَرٍ.
\par 23 وَأَمَّا الْمَزْرُوعُ عَلَى لأَرْضِ الْجَيِّدَةِ فَهُوَ الَّذِي يَسْمَعُ الْكَلِمَةَ وَيَفْهَمُ. وَهُوَ الَّذِي يَأْتِي بِثَمَرٍ فَيَصْنَعُ بَعْضٌ مِئَةً وَآخَرُ سِتِّينَ وَآخَرُ ثَلاَثِينَ».
\par 24 قَالَ لَهُمْ مَثَلاً آخَرَ: «يُشْبِهُ مَلَكُوتُ السَّمَاوَاتِ إِنْسَاناً زَرَعَ زَرْعاً جَيِّداً فِي حَقْلِهِ.
\par 25 وَفِيمَا النَّاسُ نِيَامٌ جَاءَ عَدُّوُهُ وَزَرَعَ زَوَاناً فِي وَسَطِ الْحِنْطَةِ وَمَضَى.
\par 26 فَلَمَّا طَلَعَ النَّبَاتُ وَصَنَعَ ثَمَراً حِينَئِذٍ ظَهَرَ الزَّوَانُ أَيْضاً.
\par 27 فَجَاءَ عَبِيدُ رَبِّ الْبَيْتِ وَقَالُوا لَهُ: يَا سَيِّدُ أَلَيْسَ زَرْعاً جَيِّداً زَرَعْتَ فِي حَقْلِكَ؟ فَمِنْ أَيْنَ لَهُ زَوَانٌ؟.
\par 28 فَقَالَ لَهُمْ: إِنْسَانٌ عَدُوٌّ فَعَلَ هَذَا فَقَالَ لَهُ الْعَبِيدُ: أَتُرِيدُ أَنْ نَذْهَبَ وَنَجْمَعَهُ؟
\par 29 فَقَالَ: لاَ! لِئَلَّا تَقْلَعُوا الْحِنْطَةَ مَعَ الزَّوَانِ وَأَنْتُمْ تَجْمَعُونَهُ.
\par 30 دَعُوهُمَا يَنْمِيَانِ كِلاَهُمَا مَعاً إِلَى الْحَصَادِ وَفِي وَقْتِ الْحَصَادِ أَقُولُ لِلْحَصَّادِينَ: اجْمَعُوا أوَّلاً الزَّوَانَ وَاحْزِمُوهُ حُزَماً لِيُحْرَقَ وَأَمَّا الْحِنْطَةَ فَاجْمَعُوهَا إِلَى مَخْزَنِي».
\par 31 قَالَ لَهُمْ مَثَلاً آخَرَ: «يُشْبِهُ مَلَكُوتُ السَّمَاوَاتِ حَبَّةَ خَرْدَلٍ أَخَذَهَا إِنْسَانٌ وَزَرَعَهَا فِي حَقْلِهِ
\par 32 وَهِيَ أَصْغَرُ جَمِيعِ الْبُزُورِ. وَلَكِنْ مَتَى نَمَتْ فَهِيَ أَكْبَرُ الْبُقُولِ وَتَصِيرُ شَجَرَةً حَتَّى إِنَّ طُيُورَ السَّمَاءِ تَأْتِي وَتَتَآوَى فِي أَغْصَانِهَا».
\par 33 قَالَ لَهُمْ مَثَلاً آخَرَ: «يُشْبِهُ مَلَكُوتُ السَّمَاوَاتِ خَمِيرَةً أَخَذَتْهَا امْرَأَةٌ وَخَبَّأَتْهَا فِي ثَلاَثَةِ أَكْيَالِ دَقِيقٍ حَتَّى اخْتَمَرَ الْجَمِيعُ».
\par 34 هَذَا كُلُّهُ كَلَّمَ بِهِ يَسُوعُ الْجُمُوعَ بِأَمْثَالٍ وَبِدُونِ مَثَلٍ لَمْ يَكُنْ يُكَلِّمُهُمْ
\par 35 لِكَيْ يَتِمَّ مَا قِيلَ بِالنَّبِيِّ: «سَأَفْتَحُ بِأَمْثَالٍ فَمِي وَأَنْطِقُ بِمَكْتُومَاتٍ مُنْذُ تَأْسِيسِ الْعَالَمِ».
\par 36 حِينَئِذٍ صَرَفَ يَسُوعُ الْجُمُوعَ وَجَاءَ إِلَى الْبَيْتِ. فَتَقَدَّمَ إِلَيْهِ تَلاَمِيذُهُ قَائِلِينَ: «فَسِّرْ لَنَا مَثَلَ زَوَانِ الْحَقْلِ».
\par 37 فَأَجَابَ: «اَلزَّارِعُ الزَّرْعَ الْجَيِّدَ هُوَ ابْنُ الإِنْسَانِ.
\par 38 وَالْحَقْلُ هُوَ الْعَالَمُ. وَالزَّرْعُ الْجَيِّدُ هُوَ بَنُو الْمَلَكُوتِ. وَالزَّوَانُ هُوَ بَنُو الشِّرِّيرِ.
\par 39 وَالْعَدُّوُ الَّذِي زَرَعَهُ هُوَ إِبْلِيسُ. وَالْحَصَادُ هُوَ انْقِضَاءُ الْعَالَمِ. وَالْحَصَّادُونَ هُمُ الْمَلاَئِكَةُ.
\par 40 فَكَمَا يُجْمَعُ الزَّوَانُ وَيُحْرَقُ بِالنَّارِ هَكَذَا يَكُونُ فِي انْقِضَاءِ هَذَا الْعَالَمِ:
\par 41 يُرْسِلُ ابْنُ الإِنْسَانِ مَلاَئِكَتَهُ فَيَجْمَعُونَ مِنْ مَلَكُوتِهِ جَمِيعَ الْمَعَاثِرِ وَفَاعِلِي الإِثْمِ
\par 42 وَيَطْرَحُونَهُمْ فِي أَتُونِ النَّارِ. هُنَاكَ يَكُونُ الْبُكَاءُ وَصَرِيرُ الأَسْنَانِ.
\par 43 حِينَئِذٍ يُضِيءُ الأَبْرَارُ كَالشَّمْسِ فِي مَلَكُوتِ أَبِيهِمْ. مَنْ لَهُ أُذُنَانِ لِلسَّمْعِ فَلْيَسْمَعْ».
\par 44 «أَيْضاً يُشْبِهُ مَلَكُوتُ السَّمَاوَاتِ كَنْزاً مُخْفىً فِي حَقْلٍ وَجَدَهُ إِنْسَانٌ فَأَخْفَاهُ. وَمِنْ فَرَحِهِ مَضَى وَبَاعَ كُلَّ مَا كَانَ لَهُ وَاشْتَرَى ذَلِكَ الْحَقْلَ.
\par 45 أَيْضاً يُشْبِهُ مَلَكُوتُ السَّمَاوَاتِ إِنْسَاناً تَاجِراً يَطْلُبُ لَآلِئَ حَسَنَةً
\par 46 فَلَمَّا وَجَدَ لُؤْلُؤَةً وَاحِدَةً كَثِيرَةَ الثَّمَنِ مَضَى وَبَاعَ كُلَّ مَا كَانَ لَهُ وَاشْتَرَاهَا.
\par 47 أَيْضاً يُشْبِهُ مَلَكُوتُ السَّمَاوَاتِ شَبَكَةً مَطْرُوحَةً فِي الْبَحْرِ وَجَامِعَةً مِنْ كُلِّ نَوْعٍ.
\par 48 فَلَمَّا امْتَلَأَتْ أَصْعَدُوهَا عَلَى الشَّاطِئِ وَجَلَسُوا وَجَمَعُوا الْجِيَادَ إِلَى أَوْعِيَةٍ وَأَمَّا الأَرْدِيَاءُ فَطَرَحُوهَا خَارِجاً.
\par 49 هَكَذَا يَكُونُ فِي انْقِضَاءِ الْعَالَمِ: يَخْرُجُ الْمَلاَئِكَةُ وَيُفْرِزُونَ الأَشْرَارَ مِنْ بَيْنِ الأَبْرَارِ
\par 50 وَيَطْرَحُونَهُمْ فِي أَتُونِ النَّارِ. هُنَاكَ يَكُونُ الْبُكَاءُ وَصَرِيرُ الأَسْنَانِ».
\par 51 قَالَ لَهُمْ يَسُوعُ: «أَفَهِمْتُمْ هَذَا كُلَّهُ؟» فَقَالُوا: «نَعَمْ يَا سَيِّدُ».
\par 52 فَقَالَ لَهُمْ: «مِنْ أَجْلِ ذَلِكَ كُلُّ كَاتِبٍ مُتَعَلِّمٍ فِي مَلَكُوتِ السَّمَاوَاتِ يُشْبِهُ رَجُلاً رَبَّ بَيْتٍ يُخْرِجُ مِنْ كَنْزِهِ جُدُداً وَعُتَقَاءَ».
\par 53 وَلَمَّا أَكْمَلَ يَسُوعُ هَذِهِ الأَمْثَالَ انْتَقَلَ مِنْ هُنَاكَ.
\par 54 وَلَمَّا جَاءَ إِلَى وَطَنِهِ كَانَ يُعَلِّمُهُمْ فِي مَجْمَعِهِمْ حَتَّى بُهِتُوا وَقَالُوا: «مِنْ أَيْنَ لِهَذَا هَذِهِ الْحِكْمَةُ وَالْقُوَّاتُ؟
\par 55 أَلَيْسَ هَذَا ابْنَ النَّجَّارِ؟ أَلَيْسَتْ أُمُّهُ تُدْعَى مَرْيَمَ وَإِخْوَتُهُ يَعْقُوبَ وَيُوسِي وَسِمْعَانَ وَيَهُوذَا؟
\par 56 أَوَلَيْسَتْ أَخَوَاتُهُ جَمِيعُهُنَّ عِنْدَنَا؟ فَمِنْ أَيْنَ لِهَذَا هَذِهِ كُلُّهَا؟»
\par 57 فَكَانُوا يَعْثُرُونَ بِهِ. وَأَمَّا يَسُوعُ فَقَالَ لَهُمْ: «لَيْسَ نَبِيٌّ بِلاَ كَرَامَةٍ إِلاَّ فِي وَطَنِهِ وَفِي بَيْتِهِ».
\par 58 وَلَمْ يَصْنَعْ هُنَاكَ قُوَّاتٍ كَثِيرَةً لِعَدَمِ إِيمَانِهِمْ.

\chapter{14}

\par 1 فِي ذَلِكَ الْوَقْتِ سَمِعَ هِيرُودُسُ رَئِيسُ الرُّبْعِ خَبَرَ يَسُوعَ
\par 2 فَقَالَ لِغِلْمَانِهِ: «هَذَا هُوَ يُوحَنَّا الْمَعْمَدَانُ قَدْ قَامَ مِنَ الأَمْوَاتِ وَلِذَلِكَ تُعْمَلُ بِهِ الْقُوَّاتُ».
\par 3 فَإِنَّ هِيرُودُسَ كَانَ قَدْ أَمْسَكَ يُوحَنَّا وَأَوْثَقَهُ وَطَرَحَهُ فِي سِجْنٍ مِنْ أَجْلِ هِيرُودِيَّا امْرَأَةِ فِيلُبُّسَ أَخِيهِ
\par 4 لأَنَّ يُوحَنَّا كَانَ يَقُولُ لَهُ: «لاَ يَحِلُّ أَنْ تَكُونَ لَكَ».
\par 5 وَلَمَّا أَرَادَ أَنْ يَقْتُلَهُ خَافَ مِنَ الشَّعْبِ لأَنَّهُ كَانَ عِنْدَهُمْ مِثْلَ نَبِيٍّ.
\par 6 ثُمَّ لَمَّا صَارَ مَوْلِدُ هِيرُودُسَ رَقَصَتِ ابْنَةُ هِيرُودِيَّا فِي الْوَسَطِ فَسَرَّتْ هِيرُودُسَ.
\par 7 مِنْ ثَمَّ وَعَدَ بِقَسَمٍ أَنَّهُ مَهْمَا طَلَبَتْ يُعْطِيهَا.
\par 8 فَهِيَ إِذْ كَانَتْ قَدْ تَلَقَّنَتْ مِنْ أُمِّهَا قَالَتْ: «أَعْطِنِي هَهُنَا عَلَى طَبَقٍ رَأْسَ يُوحَنَّا الْمَعْمَدَانِ».
\par 9 فَاغْتَمَّ الْمَلِكُ. وَلَكِنْ مِنْ أَجْلِ الأَقْسَامِ وَالْمُتَّكِئِينَ مَعَهُ أَمَرَ أَنْ يُعْطَى.
\par 10 فَأَرْسَلَ وَقَطَعَ رَأْسَ يُوحَنَّا فِي السِّجْنِ.
\par 11 فَأُحْضِرَ رَأْسُهُ عَلَى طَبَقٍ وَدُفِعَ إِلَى الصَّبِيَّةِ فَجَاءَتْ بِهِ إِلَى أُمِّهَا.
\par 12 فَتَقَدَّمَ تَلاَمِيذُهُ وَرَفَعُوا الْجَسَدَ وَدَفَنُوهُ. ثُمَّ أَتَوْا وَأَخْبَرُوا يَسُوعَ.
\par 13 فَلَمَّا سَمِعَ يَسُوعُ انْصَرَفَ مِنْ هُنَاكَ فِي سَفِينَةٍ إِلَى مَوْضِعٍ خَلاَءٍ مُنْفَرِداً. فَسَمِعَ الْجُمُوعُ وَتَبِعُوهُ مُشَاةً مِنَ الْمُدُنِ.
\par 14 فَلَمَّا خَرَجَ يَسُوعُ أَبْصَرَ جَمْعاً كَثِيراً فَتَحَنَّنَ عَلَيْهِمْ وَشَفَى مَرْضَاهُمْ.
\par 15 وَلَمَّا صَارَ الْمَسَاءُ تَقَدَّمَ إِلَيْهِ تَلاَمِيذُهُ قَائِلِينَ: «الْمَوْضِعُ خَلاَءٌ وَالْوَقْتُ قَدْ مَضَى. اصْرِفِ الْجُمُوعَ لِكَيْ يَمْضُوا إِلَى الْقُرَى وَيَبْتَاعُوا لَهُمْ طَعَاماً».
\par 16 فَقَالَ لَهُمْ يَسُوعُ: «لاَ حَاجَةَ لَهُمْ أَنْ يَمْضُوا. أَعْطُوهُمْ أَنْتُمْ لِيَأْكُلُوا».
\par 17 فَقَالُوا لَهُ: «لَيْسَ عِنْدَنَا هَهُنَا إِلاَّ خَمْسَةُ أَرْغِفَةٍ وَسَمَكَتَانِ».
\par 18 فَقَالَ: «ائْتُونِي بِهَا إِلَى هُنَا».
\par 19 فَأَمَرَ الْجُمُوعَ أَنْ يَتَّكِئُوا عَلَى الْعُشْبِ ثُمَّ أَخَذَ الأَرْغِفَةَ لْخَمْسَةَ وَالسَّمَكَتَيْنِ وَرَفَعَ نَظَرَهُ نَحْوَ السَّمَاءِ وَبَارَكَ وَكَسَّرَ وَأَعْطَى الأَرْغِفَةَ لِلتَّلاَمِيذِ وَالتَّلاَمِيذُ لِلْجُمُوعِ.
\par 20 فَأَكَلَ الْجَمِيعُ وَشَبِعُوا. ثُمَّ رَفَعُوا مَا فَضَلَ مِنَ الْكِسَرِ: اثْنَتَيْ عَشْرَةَ قُفَّةً مَمْلُوءةً.
\par 21 وَالآكِلُونَ كَانُوا نَحْوَ خَمْسَةِ آلاَفِ رَجُلٍ مَا عَدَا النِّسَاءَ وَالأَوْلاَدَ.
\par 22 وَلِلْوَقْتِ أَلْزَمَ يَسُوعُ تَلاَمِيذَهُ أَنْ يَدْخُلُوا السَّفِينَةَ وَيَسْبِقُوهُ إِلَى الْعَبْرِ حَتَّى يَصْرِفَ الْجُمُوعَ.
\par 23 وَبَعْدَمَا صَرَفَ الْجُمُوعَ صَعِدَ إِلَى الْجَبَلِ مُنْفَرِداً لِيُصَلِّيَ. وَلَمَّا صَارَ الْمَسَاءُ كَانَ هُنَاكَ وَحْدَهُ.
\par 24 وَأَمَّا السَّفِينَةُ فَكَانَتْ قَدْ صَارَتْ فِي وَسَطِ الْبَحْرِ مُعَذَّبَةً مِنَ الأَمْوَاجِ. لأَنَّ الرِّيحَ كَانَتْ مُضَادَّةً.
\par 25 وَفِي الْهَزِيعِ الرَّابِعِ مِنَ اللَّيْلِ مَضَى إِلَيْهِمْ يَسُوعُ مَاشِياً عَلَى الْبَحْرِ.
\par 26 فَلَمَّا أَبْصَرَهُ التَّلاَمِيذُ مَاشِياً عَلَى الْبَحْرِ اضْطَرَبُوا قَائِلِينَ: «إِنَّهُ خَيَالٌ». وَمِنَ الْخَوْفِ صَرَخُوا!
\par 27 فَلِلْوَقْتِ قَالَ لَهُمْ يَسُوعُ: «تَشَجَّعُوا! أَنَا هُوَ. لاَ تَخَافُوا».
\par 28 فَأَجَابَهُ بُطْرُسُ: «يَا سَيِّدُ إِنْ كُنْتَ أَنْتَ هُوَ فَمُرْنِي أَنْ آتِيَ إِلَيْكَ عَلَى الْمَاءِ».
\par 29 فَقَالَ: «تَعَالَ». فَنَزَلَ بُطْرُسُ مِنَ السَّفِينَةِ وَمَشَى عَلَى الْمَاءِ لِيَأْتِيَ إِلَى يَسُوعَ.
\par 30 وَلَكِنْ لَمَّا رَأَى الرِّيحَ شَدِيدَةً خَافَ. وَإِذِ ابْتَدَأَ يَغْرَقُ صَرَخَ: «يَا رَبُّ نَجِّنِي».
\par 31 فَفِي الْحَالِ مَدَّ يَسُوعُ يَدَهُ وَأَمْسَكَ بِهِ وَقَالَ لَهُ: «يَا قَلِيلَ الإِيمَانِ لِمَاذَا شَكَكْتَ؟»
\par 32 وَلَمَّا دَخَلاَ السَّفِينَةَ سَكَنَتِ الرِّيحُ.
\par 33 وَالَّذِينَ فِي السَّفِينَةِ جَاءُوا وَسَجَدُوا لَهُ قَائِلِينَ: «بِالْحَقِيقَةِ أَنْتَ ابْنُ اللَّهِ!».
\par 34 فَلَمَّا عَبَرُوا جَاءُوا إِلَى أَرْضِ جَنِّيسَارَتَ
\par 35 فَعَرَفَهُ رِجَالُ ذَلِكَ الْمَكَانِ. فَأَرْسَلُوا إِلَى جَمِيعِ تِلْكَ الْكُورَةِ الْمُحِيطَةِ وَأَحْضَرُوا إِلَيْهِ جَمِيعَ الْمَرْضَى
\par 36 وَطَلَبُوا إِلَيْهِ أَنْ يَلْمِسُوا هُدْبَ ثَوْبِهِ فَقَطْ. فَجَمِيعُ الَّذِينَ لَمَسُوهُ نَالُوا الشِّفَاءَ.

\chapter{15}

\par 1 حِينَئِذٍ جَاءَ إِلَى يَسُوعَ كَتَبَةٌ وَفَرِّيسِيُّونَ الَّذِينَ مِنْ أُورُشَلِيمَ قَائِلِينَ:
\par 2 «لِمَاذَا يَتَعَدَّى تَلاَمِيذُكَ تَقْلِيدَ الشُّيُوخِ فَإِنَّهُمْ لاَ يَغْسِلُونَ أَيْدِيَهُمْ حِينَمَا يَأْكُلُونَ خُبْزاً؟»
\par 3 فَأَجَابَ: «وَأَنْتُمْ أَيْضاً لِمَاذَا تَتَعَدَّوْنَ وَصِيَّةَ اللَّهِ بِسَبَبِ تَقْلِيدِكُمْ؟
\par 4 فَإِنَّ اللَّهَ أَوْصَى قَائِلاً: أَكْرِمْ أَبَاكَ وَأُمَّكَ وَمَنْ يَشْتِمْ أَباً أَوْ أُمّاً فَلْيَمُتْ مَوْتاً.
\par 5 وَأَمَّا أَنْتُمْ فَتَقُولُونَ: مَنْ قَالَ لأَبِيهِ أَوْ أُمِّهِ: قُرْبَانٌ هُوَ الَّذِي تَنْتَفِعُ بِهِ مِنِّي. فَلاَ يُكْرِمُ أَبَاهُ أَوْ أُمَّهُ.
\par 6 فَقَدْ أَبْطَلْتُمْ وَصِيَّةَ اللَّهِ بِسَبَبِ تَقْلِيدِكُمْ!
\par 7 يَا مُرَاؤُونَ! حَسَناً تَنَبَّأَ عَنْكُمْ إِشَعْيَاءُ قَائِلاً:
\par 8 يَقْتَرِبُ إِلَيَّ هَذَا الشَّعْبُ بِفَمِهِ وَيُكْرِمُنِي بِشَفَتَيْهِ وَأَمَّا قَلْبُهُ فَمُبْتَعِدٌ عَنِّي بَعِيداً.
\par 9 وَبَاطِلاً يَعْبُدُونَنِي وَهُمْ يُعَلِّمُونَ تَعَالِيمَ هِيَ وَصَايَا النَّاسِ».
\par 10 ثُمَّ دَعَا الْجَمْعَ وَقَالَ لَهُمُ: «اسْمَعُوا وَافْهَمُوا.
\par 11 لَيْسَ مَا يَدْخُلُ الْفَمَ يُنَجِّسُ الإِنْسَانَ بَلْ مَا يَخْرُجُ مِنَ الْفَمِ هَذَا يُنَجِّسُ الإِنْسَانَ».
\par 12 حِينَئِذٍ تَقَدَّمَ تَلاَمِيذُهُ وَقَالُوا لَهُ: «أَتَعْلَمُ أَنَّ الْفَرِّيسِيِّينَ لَمَّا سَمِعُوا الْقَوْلَ نَفَرُوا؟»
\par 13 فَأَجَابَ: «كُلُّ غَرْسٍ لَمْ يَغْرِسْهُ أَبِي السَّمَاوِيُّ يُقْلَعُ.
\par 14 اُتْرُكُوهُمْ. هُمْ عُمْيَانٌ قَادَةُ عُمْيَانٍ. وَإِنْ كَانَ أَعْمَى يَقُودُ أَعْمَى يَسْقُطَانِ كِلاَهُمَا فِي حُفْرَةٍ».
\par 15 فَقَالَ بُطْرُسُ لَهُ: «فَسِّرْ لَنَا هَذَا الْمَثَلَ».
\par 16 فَقَالَ يَسُوعُ: «هَلْ أَنْتُمْ أَيْضاً حَتَّى الآنَ غَيْرُ فَاهِمِينَ؟
\par 17 أَلاَ تَفْهَمُونَ بَعْدُ أَنَّ كُلَّ مَا يَدْخُلُ الْفَمَ يَمْضِي إِلَى الْجَوْفِ وَيَنْدَفِعُ إِلَى الْمَخْرَجِ
\par 18 وَأَمَّا مَا يَخْرُجُ مِنَ الْفَمِ فَمِنَ الْقَلْبِ يَصْدُرُ وَذَاكَ يُنَجِّسُ الإِنْسَانَ
\par 19 لأَنْ مِنَ الْقَلْبِ تَخْرُجُ أَفْكَارٌ شِرِّيرَةٌ: قَتْلٌ زِنىً فِسْقٌ سِرْقَةٌ شَهَادَةُ زُورٍ تَجْدِيفٌ.
\par 20 هَذِهِ هِيَ الَّتِي تُنَجِّسُ الإِنْسَانَ. وَأَمَّا الأَكْلُ بِأَيْدٍ غَيْرِ مَغْسُولَةٍ فَلاَ يُنَجِّسُ الإِنْسَانَ».
\par 21 ثُمَّ خَرَجَ يَسُوعُ مِنْ هُنَاكَ وَانْصَرَفَ إِلَى نَوَاحِي صُورَ وَصَيْدَاءَ.
\par 22 وَإِذَا امْرَأَةٌ كَنْعَانِيَّةٌ خَارِجَةٌ مِنْ تِلْكَ التُّخُومِ صَرَخَتْ إِلَيْهِ: «ارْحَمْنِي يَا سَيِّدُ يَا ابْنَ دَاوُدَ. ابْنَتِي مَجْنُونَةٌ جِدّاً».
\par 23 فَلَمْ يُجِبْهَا بِكَلِمَةٍ. فَتَقَدَّمَ تَلاَمِيذُهُ وَطَلَبُوا إِلَيْهِ قَائِلِينَ: «اصْرِفْهَا لأَنَّهَا تَصِيحُ وَرَاءَنَا!»
\par 24 فَأَجَابَ: «لَمْ أُرْسَلْ إِلاَّ إِلَى خِرَافِ بَيْتِ إِسْرَائِيلَ الضَّالَّةِ».
\par 25 فَأَتَتْ وَسَجَدَتْ لَهُ قَائِلَةً: «يَا سَيِّدُ أَعِنِّي!»
\par 26 فَأَجَابَ: «لَيْسَ حَسَناً أَنْ يُؤْخَذَ خُبْزُ الْبَنِينَ وَيُطْرَحَ لِلْكِلاَبِ».
\par 27 فَقَالَتْ: «نَعَمْ يَا سَيِّدُ. وَالْكِلاَبُ أَيْضاً تَأْكُلُ مِنَ الْفُتَاتِ الَّذِي يَسْقُطُ مِنْ مَائِدَةِ أَرْبَابِهَا».
\par 28 حِينَئِذٍ قَالَ يَسُوعُ لَهَا: «يَا امْرَأَةُ عَظِيمٌ إِيمَانُكِ! لِيَكُنْ لَكِ كَمَا تُرِيدِينَ». فَشُفِيَتِ ابْنَتُهَا مِنْ تِلْكَ السَّاعَةِ.
\par 29 ثُمَّ انْتَقَلَ يَسُوعُ مِنْ هُنَاكَ وَجَاءَ إِلَى جَانِبِ بَحْرِ الْجَلِيلِ وَصَعِدَ إِلَى الْجَبَلِ وَجَلَسَ هُنَاكَ.
\par 30 فَجَاءَ إِلَيْهِ جُمُوعٌ كَثِيرَةٌ مَعَهُمْ عُرْجٌ وَعُمْيٌ وَخُرْسٌ وَشُلٌّ وَآخَرُونَ كَثِيرُونَ وَطَرَحُوهُمْ عِنْدَ قَدَمَيْ يَسُوعَ. فَشَفَاهُمْ
\par 31 حَتَّى تَعَجَّبَ الْجُمُوعُ إِذْ رَأَوُا الْخُرْسَ يَتَكَلَّمُونَ وَالشُّلَّ يَصِحُّونَ وَالْعُرْجَ يَمْشُونَ وَالْعُمْيَ يُبْصِرُونَ. وَمَجَّدُوا إِلَهَ إِسْرَائِيلَ.
\par 32 وَأَمَّا يَسُوعُ فَدَعَا تَلاَمِيذَهُ وَقَالَ: «إِنِّي أُشْفِقُ عَلَى الْجَمْعِ لأَنَّ الآنَ لَهُمْ ثَلاَثَةَ أَيَّامٍ يَمْكُثُونَ مَعِي وَلَيْسَ لَهُمْ مَا يَأْكُلُونَ. وَلَسْتُ أُرِيدُ أَنْ أَصْرِفَهُمْ صَائِمِينَ لِئَلَّا يُخَوِّرُوا فِي الطَّرِيقِ».
\par 33 فَقَالَ لَهُ تَلاَمِيذُهُ: «مِنْ أَيْنَ لَنَا فِي الْبَرِّيَّةِ خُبْزٌ بِهَذَا الْمِقْدَارِ حَتَّى يُشْبِعَ جَمْعاً هَذَا عَدَدُهُ؟»
\par 34 فَقَالَ لَهُمْ يَسُوعُ: «كَمْ عِنْدَكُمْ مِنَ الْخُبْزِ؟» فَقَالُوا: «سَبْعَةٌ وَقَلِيلٌ مِنْ صِغَارِ السَّمَكِ».
\par 35 فَأَمَرَ الْجُمُوعَ أَنْ يَتَّكِئُوا عَلَى الأَرْضِ
\par 36 وَأَخَذَ السَّبْعَ خُبْزَاتٍ وَالسَّمَكَ وَشَكَرَ وَكَسَّرَ وَأَعْطَى تَلاَمِيذَهُ وَالتَّلاَمِيذُ أَعْطَوُا الْجَمْعَ.
\par 37 فَأَكَلَ الْجَمِيعُ وَشَبِعُوا. ثُمَّ رَفَعُوا مَا فَضَلَ مِنَ الْكِسَرِ سَبْعَةَ سِلاَلٍ مَمْلُوءَةٍ
\par 38 وَالآكِلُونَ كَانُوا أَرْبَعَةَ آلاَفِ رَجُلٍ مَا عَدَا النِّسَاءَ وَالأَوْلاَدَ.
\par 39 ثُمَّ صَرَفَ الْجُمُوعَ وَصَعِدَ إِلَى السَّفِينَةِ وَجَاءَ إِلَى تُخُومِ مَجْدَلَ.

\chapter{16}

\par 1 وَجَاءَ إِلَيْهِ الْفَرِّيسِيُّونَ وَالصَّدُّوقِيُّونَ لِيُجَرِّبُوهُ فَسَأَلُوهُ أَنْ يُرِيَهُمْ آيَةً مِنَ السَّمَاءِ.
\par 2 فَأَجَابَ: «إِذَا كَانَ الْمَسَاءُ قُلْتُمْ: صَحْوٌ لأَنَّ السَّمَاءَ مُحْمَرَّةٌ.
\par 3 وَفِي الصَّبَاحِ: الْيَوْمَ شِتَاءٌ لأَنَّ السَّمَاءَ مُحْمَرَّةٌ بِعُبُوسَةٍ. يَا مُرَاؤُونَ! تَعْرِفُونَ أَنْ تُمَيِّزُوا وَجْهَ السَّمَاءِ وَأَمَّا عَلاَمَاتُ الأَزْمِنَةِ فَلاَ تَسْتَطِيعُونَ!
\par 4 جِيلٌ شِرِّيرٌ فَاسِقٌ يَلْتَمِسُ آيَةً وَلاَ تُعْطَى لَهُ آيَةٌ إِلاَّ آيَةَ يُونَانَ النَّبِيِّ». ثُمَّ تَرَكَهُمْ وَمَضَى.
\par 5 وَلَمَّا جَاءَ تَلاَمِيذُهُ إِلَى الْعَبْرِ نَسُوا أَنْ يَأْخُذُوا خُبْزاً.
\par 6 وَقَالَ لَهُمْ يَسُوعُ: «انْظُرُوا وَتَحَرَّزُوا مِنْ خَمِيرِ الْفَرِّيسِيِّينَ وَالصَّدُّوقِيِّينَ».
\par 7 فَفَكَّرُوا فِي أَنْفُسِهِمْ قَائِلِينَ: «إِنَّنَا لَمْ نَأْخُذْ خُبْزاً».
\par 8 فَعَلِمَ يَسُوعُ وَقَالَ لَهُمْ: «لِمَاذَا تُفَكِّرُونَ فِي أَنْفُسِكُمْ يَا قَلِيلِي الإِيمَانِ أَنَّكُمْ لَمْ تَأْخُذُوا خُبْزاً؟
\par 9 أَحَتَّى الآنَ لاَ تَفْهَمُونَ وَلاَ تَذْكُرُونَ خَمْسَ خُبْزَاتِ الْخَمْسَةِ الآلاَفِ وَكَمْ قُفَّةً أَخَذْتُمْ
\par 10 وَلاَ سَبْعَ خُبْزَاتِ الأَرْبَعَةِ الآلاَفِ وَكَمْ سَلاًّ أَخَذْتُمْ؟
\par 11 كَيْفَ لاَ تَفْهَمُونَ أَنِّي لَيْسَ عَنِ الْخُبْزِ قُلْتُ لَكُمْ أَنْ تَتَحَرَّزُوا مِنْ خَمِيرِ الْفَرِّيسِيِّينَ وَالصَّدُّوقِيِّينَ؟»
\par 12 حِينَئِذٍ فَهِمُوا أَنَّهُ لَمْ يَقُلْ أَنْ يَتَحَرَّزُوا مِنْ خَمِيرِ الْخُبْزِ بَلْ مِنْ تَعْلِيمِ الْفَرِّيسِيِّينَ وَالصَّدُّوقِيِّينَ.
\par 13 وَلَمَّا جَاءَ يَسُوعُ إِلَى نَوَاحِي قَيْصَرِيَّةِ فِيلُبُّسَ سَأَلَ تَلاَمِيذَهُ: «مَنْ يَقُولُ النَّاسُ إِنِّي أَنَا ابْنُ الإِنْسَانِ؟»
\par 14 فَقَالُوا: «قَوْمٌ يُوحَنَّا الْمَعْمَدَانُ وَآخَرُونَ إِيلِيَّا وَآخَرُونَ إِرْمِيَا أَوْ وَاحِدٌ مِنَ الأَنْبِيَاءِ».
\par 15 قَالَ لَهُمْ: «وَأَنْتُمْ مَنْ تَقُولُونَ إِنِّي أَنَا؟»
\par 16 فَأَجَابَ سِمْعَانُ بُطْرُسُ: «أَنْتَ هُوَ الْمَسِيحُ ابْنُ اللَّهِ الْحَيِّ».
\par 17 فَقَالَ لَهُ يَسُوعُ: «طُوبَى لَكَ يَا سِمْعَانُ بْنَ يُونَا إِنَّ لَحْماً وَدَماً لَمْ يُعْلِنْ لَكَ لَكِنَّ أَبِي الَّذِي فِي السَّمَاوَاتِ.
\par 18 وَأَنَا أَقُولُ لَكَ أَيْضاً: أَنْتَ بُطْرُسُ وَعَلَى هَذِهِ الصَّخْرَةِ أَبْنِي كَنِيسَتِي وَأَبْوَابُ الْجَحِيمِ لَنْ تَقْوَى عَلَيْهَا.
\par 19 وَأُعْطِيكَ مَفَاتِيحَ مَلَكُوتِ السَّمَاوَاتِ فَكُلُّ مَا تَرْبِطُهُ عَلَى الأَرْضِ يَكُونُ مَرْبُوطاً فِي السَّمَاوَاتِ. وَكُلُّ مَا تَحُلُّهُ عَلَى الأَرْضِ يَكُونُ مَحْلُولاً فِي السَّمَاوَاتِ».
\par 20 حِينَئِذٍ أَوْصَى تَلاَمِيذَهُ أَنْ لاَ يَقُولُوا لأَحَدٍ إِنَّهُ يَسُوعُ الْمَسِيحُ.
\par 21 مِنْ ذَلِكَ الْوَقْتِ ابْتَدَأَ يَسُوعُ يُظْهِرُ لِتَلاَمِيذِهِ أَنَّهُ يَنْبَغِي أَنْ يَذْهَبَ إِلَى أُورُشَلِيمَ وَيَتَأَلَّمَ كَثِيراً مِنَ الشُّيُوخِ وَرُؤَسَاءِ الْكَهَنَةِ وَالْكَتَبَةِ وَيُقْتَلَ وَفِي الْيَوْمِ الثَّالِثِ يَقُومَ.
\par 22 فَأَخَذَهُ بُطْرُسُ إِلَيْهِ وَابْتَدَأَ يَنْتَهِرُهُ قَائِلاً: «حَاشَاكَ يَا رَبُّ! لاَ يَكُونُ لَكَ هَذَا!»
\par 23 فَالْتَفَتَ وَقَالَ لِبُطْرُسَ: «اذْهَبْ عَنِّي يَا شَيْطَانُ. أَنْتَ مَعْثَرَةٌ لِي لأَنَّكَ لاَ تَهْتَمُّ بِمَا لِلَّهِ لَكِنْ بِمَا لِلنَّاسِ».
\par 24 حِينَئِذٍ قَالَ يَسُوعُ لِتَلاَمِيذِهِ: «إِنْ أَرَادَ أَحَدٌ أَنْ يَأْتِيَ وَرَائِي فَلْيُنْكِرْ نَفْسَهُ وَيَحْمِلْ صَلِيبَهُ وَيَتْبَعْنِي
\par 25 فَإِنَّ مَنْ أَرَادَ أَنْ يُخَلِّصَ نَفْسَهُ يُهْلِكُهَا وَمَنْ يُهْلِكُ نَفْسَهُ مِنْ أَجْلِي يَجِدُهَا.
\par 26 لأَنَّهُ مَاذَا يَنْتَفِعُ الإِنْسَانُ لَوْ رَبِحَ الْعَالَمَ كُلَّهُ وَخَسِرَ نَفْسَهُ؟ أَوْ مَاذَا يُعْطِي الإِنْسَانُ فِدَاءً عَنْ نَفْسِهِ؟
\par 27 فَإِنَّ ابْنَ الإِنْسَانِ سَوْفَ يَأْتِي فِي مَجْدِ أَبِيهِ مَعَ مَلاَئِكَتِهِ وَحِينَئِذٍ يُجَازِي كُلَّ وَاحِدٍ حَسَبَ عَمَلِهِ.
\par 28 اَلْحَقَّ أَقُولُ لَكُمْ إِنَّ مِنَ الْقِيَامِ هَهُنَا قَوْماً لاَ يَذُوقُونَ الْمَوْتَ حَتَّى يَرَوُا ابْنَ الإِنْسَانِ آتِياً فِي مَلَكُوتِهِ».

\chapter{17}

\par 1 وَبَعْدَ سِتَّةِ أَيَّامٍ أَخَذَ يَسُوعُ بُطْرُسَ وَيَعْقُوبَ وَيُوحَنَّا أَخَاهُ وَصَعِدَ بِهِمْ إِلَى جَبَلٍ عَالٍ مُنْفَرِدِينَ.
\par 2 وَتَغَيَّرَتْ هَيْئَتُهُ قُدَّامَهُمْ وَأَضَاءَ وَجْهُهُ كَالشَّمْسِ وَصَارَتْ ثِيَابُهُ بَيْضَاءَ كَالنُّورِ.
\par 3 وَإِذَا مُوسَى وَإِيلِيَّا قَدْ ظَهَرَا لَهُمْ يَتَكَلَّمَانِ مَعَهُ.
\par 4 فَجَعَلَ بُطْرُسُ يَقُولُ لِيَسُوعَ: «يَا رَبُّ جَيِّدٌ أَنْ نَكُونَ هَهُنَا! فَإِنْ شِئْتَ نَصْنَعْ هُنَا ثَلاَثَ مَظَالَّ. لَكَ وَاحِدَةٌ وَلِمُوسَى وَاحِدَةٌ وَلِإِيلِيَّا وَاحِدَةٌ».
\par 5 وَفِيمَا هُوَ يَتَكَلَّمُ إِذَا سَحَابَةٌ نَيِّرَةٌ ظَلَّلَتْهُمْ وَصَوْتٌ مِنَ السَّحَابَةِ قَائِلاً: «هَذَا هُوَ ابْنِي الْحَبِيبُ الَّذِي بِهِ سُرِرْتُ. لَهُ اسْمَعُوا».
\par 6 وَلَمَّا سَمِعَ التَّلاَمِيذُ سَقَطُوا عَلَى وُجُوهِهِمْ وَخَافُوا جِدّاً.
\par 7 فَجَاءَ يَسُوعُ وَلَمَسَهُمْ وَقَالَ: «قُومُوا وَلاَ تَخَافُوا».
\par 8 فَرَفَعُوا أَعْيُنَهُمْ وَلَمْ يَرَوْا أَحَداً إِلاَّ يَسُوعَ وَحْدَهُ.
\par 9 وَفِيمَا هُمْ نَازِلُونَ مِنَ الْجَبَلِ أَوْصَاهُمْ يَسُوعُ قَائِلاً: «لاَ تُعْلِمُوا أَحَداً بِمَا رَأَيْتُمْ حَتَّى يَقُومَ ابْنُ الإِنْسَانِ مِنَ الأَمْوَاتِ».
\par 10 وَسَأَلَهُ تَلاَمِيذُهُ: «فَلِمَاذَا يَقُولُ الْكَتَبَةُ إِنَّ إِيلِيَّا يَنْبَغِي أَنْ يَأْتِيَ أَوَّلاً؟»
\par 11 فَأَجَابَ يَسُوعُ: «إِنَّ إِيلِيَّا يَأْتِي أَوَّلاً وَيَرُدُّ كُلَّ شَيْءٍ.
\par 12 وَلَكِنِّي أَقُولُ لَكُمْ إِنَّ إِيلِيَّا قَدْ جَاءَ وَلَمْ يَعْرِفُوهُ بَلْ عَمِلُوا بِهِ كُلَّ مَا أَرَادُوا. كَذَلِكَ ابْنُ الإِنْسَانِ أَيْضاً سَوْفَ يَتَأَلَّمُ مِنْهُمْ».
\par 13 حِينَئِذٍ فَهِمَ التَّلاَمِيذُ أَنَّهُ قَالَ لَهُمْ عَنْ يُوحَنَّا الْمَعْمَدَانِ.
\par 14 وَلَمَّا جَاءُوا إِلَى الْجَمْعِ تَقَدَّمَ إِلَيْهِ رَجُلٌ جَاثِياً لَهُ
\par 15 وَقَائِلاً: «يَا سَيِّدُ ارْحَمِ ابْنِي فَإِنَّهُ يُصْرَعُ وَيَتَأَلَّمُ شَدِيداً وَيَقَعُ كَثِيراً فِي النَّارِ وَكَثِيراً فِي الْمَاءِ.
\par 16 وَأَحْضَرْتُهُ إِلَى تَلاَمِيذِكَ فَلَمْ يَقْدِرُوا أَنْ يَشْفُوهُ».
\par 17 فَأَجَابَ يَسُوعُ: «أَيُّهَا الْجِيلُ غَيْرُ الْمُؤْمِنِ الْمُلْتَوِي إِلَى مَتَى أَكُونُ مَعَكُمْ؟ إِلَى مَتَى أَحْتَمِلُكُمْ؟ قَدِّمُوهُ إِلَيَّ هَهُنَا!»
\par 18 فَانْتَهَرَهُ يَسُوعُ فَخَرَجَ مِنْهُ الشَّيْطَانُ. فَشُفِيَ الْغُلاَمُ مِنْ تِلْكَ السَّاعَةِ.
\par 19 ثُمَّ تَقَدَّمَ التَّلاَمِيذُ إِلَى يَسُوعَ عَلَى انْفِرَادٍ وَقَالُوا: «لِمَاذَا لَمْ نَقْدِرْ نَحْنُ أَنْ نُخْرِجَهُ؟»
\par 20 فَقَالَ لَهُمْ يَسُوعُ: «لِعَدَمِ إِيمَانِكُمْ. فَالْحَقَّ أَقُولُ لَكُمْ: لَوْ كَانَ لَكُمْ إِيمَانٌ مِثْلُ حَبَّةِ خَرْدَلٍ لَكُنْتُمْ تَقُولُونَ لِهَذَا الْجَبَلِ: انْتَقِلْ مِنْ هُنَا إِلَى هُنَاكَ فَيَنْتَقِلُ وَلاَ يَكُونُ شَيْءٌ غَيْرَ مُمْكِنٍ لَدَيْكُمْ.
\par 21 وَأَمَّا هَذَا الْجِنْسُ فَلاَ يَخْرُجُ إِلاَّ بِالصَّلاَةِ وَالصَّوْمِ».
\par 22 وَفِيمَا هُمْ يَتَرَدَّدُونَ فِي الْجَلِيلِ قَالَ لَهُمْ يَسُوعُ: «ابْنُ الإِنْسَانِ سَوْفَ يُسَلَّمُ إِلَى أَيْدِي النَّاسِ
\par 23 فَيَقْتُلُونَهُ وَفِي الْيَوْمِ الثَّالِثِ يَقُومُ». فَحَزِنُوا جِدّاً.
\par 24 وَلَمَّا جَاءُوا إِلَى كَفْرِنَاحُومَ تَقَدَّمَ الَّذِينَ يَأْخُذُونَ الدِّرْهَمَيْنِ إِلَى بُطْرُسَ وَقَالُوا: «أَمَا يُوفِي مُعَلِّمُكُمُ الدِّرْهَمَيْنِ؟»
\par 25 قَالَ: «بَلَى». فَلَمَّا دَخَلَ الْبَيْتَ سَبَقَهُ يَسُوعُ قَائِلاً: «مَاذَا تَظُنُّ يَا سِمْعَانُ؟ مِمَّنْ يَأْخُذُ مُلُوكُ الأَرْضِ الْجِبَايَةَ أَوِ الْجِزْيَةَ أَمِنْ بَنِيهِمْ أَمْ مِنَ الأَجَانِبِ؟»
\par 26 قَالَ لَهُ بُطْرُسُ: «مِنَ الأَجَانِبِ». قَالَ لَهُ يَسُوعُ: «فَإِذاً الْبَنُونَ أَحْرَارٌ.
\par 27 وَلَكِنْ لِئَلَّا نُعْثِرَهُمُ اذْهَبْ إِلَى الْبَحْرِ وَأَلْقِ صِنَّارَةً وَالسَّمَكَةُ الَّتِي تَطْلُعُ أَوَّلاً خُذْهَا وَمَتَى فَتَحْتَ فَاهَا تَجِدْ إِسْتَاراً فَخُذْهُ وَأَعْطِهِمْ عَنِّي وَعَنْكَ».

\chapter{18}

\par 1 فِي تِلْكَ السَّاعَةِ تَقَدَّمَ التَّلاَمِيذُ إِلَى يَسُوعَ قَائِلِينَ: «فَمَنْ هُوَ أَعْظَمُ فِي مَلَكُوتِ السَّمَاوَاتِ؟»
\par 2 فَدَعَا يَسُوعُ إِلَيْهِ وَلَداً وَأَقَامَهُ فِي وَسَطِهِمْ
\par 3 وَقَالَ: «اَلْحَقَّ أَقُولُ لَكُمْ: إِنْ لَمْ تَرْجِعُوا وَتَصِيرُوا مِثْلَ الأَوْلاَدِ فَلَنْ تَدْخُلُوا مَلَكُوتَ السَّمَاوَاتِ.
\par 4 فَمَنْ وَضَعَ نَفْسَهُ مِثْلَ هَذَا الْوَلَدِ فَهُوَ الأَعْظَمُ فِي مَلَكُوتِ السَّمَاوَاتِ.
\par 5 وَمَنْ قَبِلَ وَلَداً وَاحِداً مِثْلَ هَذَا بِاسْمِي فَقَدْ قَبِلَنِي.
\par 6 وَمَنْ أَعْثَرَ أَحَدَ هَؤُلاَءِ الصِّغَارِ الْمُؤْمِنِينَ بِي فَخَيْرٌ لَهُ أَنْ يُعَلَّقَ فِي عُنُقِهِ حَجَرُ الرَّحَى وَيُغْرَقَ فِي لُجَّةِ الْبَحْرِ.
\par 7 وَيْلٌ لِلْعَالَمِ مِنَ الْعَثَرَاتِ. فَلاَ بُدَّ أَنْ تَأْتِيَ الْعَثَرَاتُ وَلَكِنْ وَيْلٌ لِذَلِكَ الإِنْسَانِ الَّذِي بِهِ تَأْتِي الْعَثْرَةُ.
\par 8 فَإِنْ أَعْثَرَتْكَ يَدُكَ أَوْ رِجْلُكَ فَاقْطَعْهَا وَأَلْقِهَا عَنْكَ. خَيْرٌ لَكَ أَنْ تَدْخُلَ الْحَيَاةَ أَعْرَجَ أَوْ أَقْطَعَ مِنْ أَنْ تُلْقَى فِي النَّارِ الأَبَدِيَّةِ وَلَكَ يَدَانِ أَوْ رِجْلاَنِ.
\par 9 وَإِنْ أَعْثَرَتْكَ عَيْنُكَ فَاقْلَعْهَا وَأَلْقِهَا عَنْكَ. خَيْرٌ لَكَ أَنْ تَدْخُلَ الْحَيَاةَ أَعْوَرَ مِنْ أَنْ تُلْقَى فِي جَهَنَّمَ النَّارِ وَلَكَ عَيْنَانِ.
\par 10 انْظُرُوا لاَ تَحْتَقِرُوا أَحَدَ هَؤُلاَءِ الصِّغَارِ لأَنِّي أَقُولُ لَكُمْ إِنَّ مَلاَئِكَتَهُمْ فِي السَّمَاوَاتِ كُلَّ حِينٍ يَنْظُرُونَ وَجْهَ أَبِي الَّذِي فِي السَّمَاوَاتِ.
\par 11 لأَنَّ ابْنَ الإِنْسَانِ قَدْ جَاءَ لِكَيْ يُخَلِّصَ مَا قَدْ هَلَكَ.
\par 12 مَاذَا تَظُنُّونَ؟ إِنْ كَانَ لِإِنْسَانٍ مِئَةُ خَرُوفٍ وَضَلَّ وَاحِدٌ مِنْهَا أَفَلاَ يَتْرُكُ التِّسْعَةَ وَالتِّسْعِينَ عَلَى الْجِبَالِ وَيَذْهَبُ يَطْلُبُ الضَّالَّ؟
\par 13 وَإِنِ اتَّفَقَ أَنْ يَجِدَهُ فَالْحَقَّ أَقُولُ لَكُمْ إِنَّهُ يَفْرَحُ بِهِ أَكْثَرَ مِنَ التِّسْعَةِ وَالتِّسْعِينَ الَّتِي لَمْ تَضِلَّ.
\par 14 هَكَذَا لَيْسَتْ مَشِيئَةً أَمَامَ أَبِيكُمُ الَّذِي فِي السَّمَاوَاتِ أَنْ يَهْلِكَ أَحَدُ هَؤُلاَءِ الصِّغَارِ
\par 15 «وَإِنْ أَخْطَأَ إِلَيْكَ أَخُوكَ فَاذْهَبْ وَعَاتِبْهُ بَيْنَكَ وَبَيْنَهُ وَحْدَكُمَا. إِنْ سَمِعَ مِنْكَ فَقَدْ رَبِحْتَ أَخَاكَ.
\par 16 وَإِنْ لَمْ يَسْمَعْ فَخُذْ مَعَكَ أَيْضاً وَاحِداً أَوِ اثْنَيْنِ لِكَيْ تَقُومَ كُلُّ كَلِمَةٍ عَلَى فَمِ شَاهِدَيْنِ أَوْ ثَلاَثَةٍ.
\par 17 وَإِنْ لَمْ يَسْمَعْ مِنْهُمْ فَقُلْ لِلْكَنِيسَةِ. وَإِنْ لَمْ يَسْمَعْ مِنَ الْكَنِيسَةِ فَلْيَكُنْ عِنْدَكَ كَالْوَثَنِيِّ وَالْعَشَّارِ.
\par 18 اَلْحَقَّ أَقُولُ لَكُمْ: كُلُّ مَا تَرْبِطُونَهُ عَلَى الأَرْضِ يَكُونُ مَرْبُوطاً فِي السَّمَاءِ وَكُلُّ مَا تَحُلُّونَهُ عَلَى الأَرْضِ يَكُونُ مَحْلُولاً فِي السَّمَاءِ.
\par 19 وَأَقُولُ لَكُمْ أَيْضاً: إِنِ اتَّفَقَ اثْنَانِ مِنْكُمْ عَلَى الأَرْضِ فِي أَيِّ شَيْءٍ يَطْلُبَانِهِ فَإِنَّهُ يَكُونُ لَهُمَا مِنْ قِبَلِ أَبِي الَّذِي فِي السَّمَاوَاتِ
\par 20 لأَنَّهُ حَيْثُمَا اجْتَمَعَ اثْنَانِ أَوْ ثَلاَثَةٌ بِاسْمِي فَهُنَاكَ أَكُونُ فِي وَسَطِهِمْ».
\par 21 حِينَئِذٍ تَقَدَّمَ إِلَيْهِ بُطْرُسُ وَقَالَ: «يَا رَبُّ كَمْ مَرَّةً يُخْطِئُ إِلَيَّ أَخِي وَأَنَا أَغْفِرُ لَهُ؟ هَلْ إِلَى سَبْعِ مَرَّاتٍ؟»
\par 22 قَالَ لَهُ يَسُوعُ: «لاَ أَقُولُ لَكَ إِلَى سَبْعِ مَرَّاتٍ بَلْ إِلَى سَبْعِينَ مَرَّةً سَبْعَ مَرَّاتٍ.
\par 23 لِذَلِكَ يُشْبِهُ مَلَكُوتُ السَّمَاوَاتِ إِنْسَاناً مَلِكاً أَرَادَ أَنْ يُحَاسِبَ عَبِيدَهُ.
\par 24 فَلَمَّا ابْتَدَأَ فِي الْمُحَاسَبَةِ قُدِّمَ إِلَيْهِ وَاحِدٌ مَدْيُونٌ بِعَشْرَةِ آلاَفِ وَزْنَةٍ.
\par 25 وَإِذْ لَمْ يَكُنْ لَهُ مَا يُوفِي أَمَرَ سَيِّدُهُ أَنْ يُبَاعَ هُوَ وَامْرَأَتُهُ وَأَوْلاَدُهُ وَكُلُّ مَا لَهُ وَيُوفَى الدَّيْنُ.
\par 26 فَخَرَّ الْعَبْدُ وَسَجَدَ لَهُ قَائِلاً: يَا سَيِّدُ تَمَهَّلْ عَلَيَّ فَأُوفِيَكَ الْجَمِيعَ.
\par 27 فَتَحَنَّنَ سَيِّدُ ذَلِكَ الْعَبْدِ وَأَطْلَقَهُ وَتَرَكَ لَهُ الدَّيْنَ.
\par 28 وَلَمَّا خَرَجَ ذَلِكَ الْعَبْدُ وَجَدَ وَاحِداً مِنَ الْعَبِيدِ رُفَقَائِهِ كَانَ مَدْيُوناً لَهُ بِمِئَةِ دِينَارٍ فَأَمْسَكَهُ وَأَخَذَ بِعُنُقِهِ قَائِلاً: أَوْفِنِي مَا لِي عَلَيْكَ.
\par 29 فَخَرَّ الْعَبْدُ رَفِيقُهُ عَلَى قَدَمَيْهِ وَطَلَبَ إِلَيْهِ قَائِلاً: تَمَهَّلْ عَلَيَّ فَأُوفِيَكَ الْجَمِيعَ.
\par 30 فَلَمْ يُرِدْ بَلْ مَضَى وَأَلْقَاهُ فِي سِجْنٍ حَتَّى يُوفِيَ الدَّيْنَ.
\par 31 فَلَمَّا رَأَى الْعَبِيدُ رُفَقَاؤُهُ مَا كَانَ حَزِنُوا جِدّاً. وَأَتَوْا وَقَصُّوا عَلَى سَيِّدِهِمْ كُلَّ مَا جَرَى.
\par 32 فَدَعَاهُ حِينَئِذٍ سَيِّدُهُ وَقَالَ لَهُ: أَيُّهَا الْعَبْدُ الشِّرِّيرُ كُلُّ ذَلِكَ الدَّيْنِ تَرَكْتُهُ لَكَ لأَنَّكَ طَلَبْتَ إِلَيَّ.
\par 33 أَفَمَا كَانَ يَنْبَغِي أَنَّكَ أَنْتَ أَيْضاً تَرْحَمُ الْعَبْدَ رَفِيقَكَ كَمَا رَحِمْتُكَ أَنَا؟.
\par 34 وَغَضِبَ سَيِّدُهُ وَسَلَّمَهُ إِلَى الْمُعَذِّبِينَ حَتَّى يُوفِيَ كُلَّ مَا كَانَ لَهُ عَلَيْهِ.
\par 35 فَهَكَذَا أَبِي السَّمَاوِيُّ يَفْعَلُ بِكُمْ إِنْ لَمْ تَتْرُكُوا مِنْ قُلُوبِكُمْ كُلُّ وَاحِدٍ لأَخِيهِ زَلَّاتِهِ».

\chapter{19}

\par 1 وَلَمَّا أَكْمَلَ يَسُوعُ هَذَا الْكَلاَمَ انْتَقَلَ مِنَ الْجَلِيلِ وَجَاءَ إِلَى تُخُومِ الْيَهُودِيَّةِ مِنْ عَبْرِ الأُرْدُنِّ.
\par 2 وَتَبِعَتْهُ جُمُوعٌ كَثِيرَةٌ فَشَفَاهُمْ هُنَاكَ.
\par 3 وَجَاءَ إِلَيْهِ الْفَرِّيسِيُّونَ لِيُجَرِّبُوهُ قَائِلِينَ لَهُ: «هَلْ يَحِلُّ لِلرَّجُلِ أَنْ يُطَلِّقَ امْرَأَتَهُ لِكُلِّ سَبَبٍ؟»
\par 4 فَأَجَابَ: «أَمَا قَرَأْتُمْ أَنَّ الَّذِي خَلَقَ مِنَ الْبَدْءِ خَلَقَهُمَا ذَكَراً وَأُنْثَى؟»
\par 5 وَقَالَ: «مِنْ أَجْلِ هَذَا يَتْرُكُ الرَّجُلُ أَبَاهُ وَأُمَّهُ وَيَلْتَصِقُ بِامْرَأَتِهِ وَيَكُونُ الاِثْنَانِ جَسَداً وَاحِداً.
\par 6 إِذاً لَيْسَا بَعْدُ اثْنَيْنِ بَلْ جَسَدٌ وَاحِدٌ. فَالَّذِي جَمَعَهُ اللَّهُ لاَ يُفَرِّقُهُ إِنْسَانٌ».
\par 7 فَسَأَلُوهُ: «فَلِمَاذَا أَوْصَى مُوسَى أَنْ يُعْطَى كِتَابُ طَلاَقٍ فَتُطَلَّقُ؟»
\par 8 قَالَ لَهُمْ: «إِنَّ مُوسَى مِنْ أَجْلِ قَسَاوَةِ قُلُوبِكُمْ أَذِنَ لَكُمْ أَنْ تُطَلِّقُوا نِسَاءَكُمْ. وَلَكِنْ مِنَ الْبَدْءِ لَمْ يَكُنْ هَكَذَا.
\par 9 وَأَقُولُ لَكُمْ: إِنَّ مَنْ طَلَّقَ امْرَأَتَهُ إِلاَّ بِسَبَبِ الزِّنَا وَتَزَوَّجَ بِأُخْرَى يَزْنِي وَالَّذِي يَتَزَوَّجُ بِمُطَلَّقَةٍ يَزْنِي».
\par 10 قَالَ لَهُ تَلاَمِيذُهُ: «إِنْ كَانَ هَكَذَا أَمْرُ الرَّجُلِ مَعَ الْمَرْأَةِ فَلاَ يُوافِقُ أَنْ يَتَزَوَّجَ!»
\par 11 فَقَالَ لَهُمْ: «لَيْسَ الْجَمِيعُ يَقْبَلُونَ هَذَا الْكَلاَمَ بَلِ الَّذِينَ أُعْطِيَ لَهُم
\par 12 لأَنَّهُ يُوجَدُ خِصْيَانٌ وُلِدُوا هَكَذَا مِنْ بُطُونِ أُمَّهَاتِهِمْ وَيُوجَدُ خِصْيَانٌ خَصَاهُمُ النَّاسُ وَيُوجَدُ خِصْيَانٌ خَصَوْا أَنْفُسَهُمْ لأَجْلِ مَلَكُوتِ السَّمَاوَاتِ. مَنِ اسْتَطَاعَ أَنْ يَقْبَلَ فَلْيَقْبَلْ».
\par 13 حِينَئِذٍ قُدِّمَ إِلَيْهِ أَوْلاَدٌ لِكَيْ يَضَعَ يَدَيْهِ عَلَيْهِمْ وَيُصَلِّيَ فَانْتَهَرَهُمُ التَّلاَمِيذُ.
\par 14 أَمَّا يَسُوعُ فَقَالَ: «دَعُوا الأَوْلاَدَ يَأْتُونَ إِلَيَّ وَلاَ تَمْنَعُوهُمْ لأَنَّ لِمِثْلِ هَؤُلاَءِ مَلَكُوتَ السَّمَاوَاتِ».
\par 15 فَوَضَعَ يَدَيْهِ عَلَيْهِمْ. وَمَضَى مِنْ هُنَاكَ.
\par 16 وَإِذَا وَاحِدٌ تَقَدَّمَ وَقَالَ لَهُ: «أَيُّهَا الْمُعَلِّمُ الصَّالِحُ أَيَّ صَلاَحٍ أَعْمَلُ لِتَكُونَ لِيَ الْحَيَاةُ الأَبَدِيَّةُ؟»
\par 17 فَقَالَ لَهُ: «لِمَاذَا تَدْعُونِي صَالِحاً؟ لَيْسَ أَحَدٌ صَالِحاً إِلاَّ وَاحِدٌ وَهُوَ اللَّهُ. وَلَكِنْ إِنْ أَرَدْتَ أَنْ تَدْخُلَ الْحَيَاةَ فَاحْفَظِ الْوَصَايَا».
\par 18 قَالَ لَهُ: «أَيَّةَ الْوَصَايَا؟» فَقَالَ يَسُوعُ: «لاَ تَقْتُلْ. لاَ تَزْنِ. لاَ تَسْرِقْ. لاَ تَشْهَدْ بِالزُّورِ.
\par 19 أَكْرِمْ أَبَاكَ وَأُمَّكَ وَأَحِبَّ قَرِيبَكَ كَنَفْسِكَ».
\par 20 قَالَ لَهُ الشَّابُّ: «هَذِهِ كُلُّهَا حَفِظْتُهَا مُنْذُ حَدَاثَتِي. فَمَاذَا يُعْوِزُنِي بَعْدُ؟»
\par 21 قَالَ لَهُ يَسُوعُ: «إِنْ أَرَدْتَ أَنْ تَكُونَ كَامِلاً فَاذْهَبْ وَبِعْ أَمْلاَكَكَ وَأَعْطِ الْفُقَرَاءَ فَيَكُونَ لَكَ كَنْزٌ فِي السَّمَاءِ وَتَعَالَ اتْبَعْنِي».
\par 22 فَلَمَّا سَمِعَ الشَّابُّ الْكَلِمَةَ مَضَى حَزِيناً لأَنَّهُ كَانَ ذَا أَمْوَالٍ كَثِيرَةٍ.
\par 23 فَقَالَ يَسُوعُ لِتَلاَمِيذِهِ: «الْحَقَّ أَقُولُ لَكُمْ: إِنَّهُ يَعْسُرُ أَنْ يَدْخُلَ غَنِيٌّ إِلَى مَلَكُوتِ السَّمَاوَاتِ.
\par 24 وَأَقُولُ لَكُمْ أَيْضاً: إِنَّ مُرُورَ جَمَلٍ مِنْ ثَقْبِ إِبْرَةٍ أَيْسَرُ مِنْ أَنْ يَدْخُلَ غَنِيٌّ إِلَى مَلَكُوتِ اللَّهِ».
\par 25 فَلَمَّا سَمِعَ تَلاَمِيذُهُ بُهِتُوا جِدّاً قَائِلِينَ: «إِذاً مَنْ يَسْتَطِيعُ أَنْ يَخْلُصَ؟»
\par 26 فَنَظَرَ إِلَيْهِمْ يَسُوعُ وَقَالَ: «هَذَا عِنْدَ النَّاسِ غَيْرُ مُسْتَطَاعٍ وَلَكِنْ عِنْدَ اللَّهِ كُلُّ شَيْءٍ مُسْتَطَاعٌ».
\par 27 فَأَجَابَ بُطْرُسُ حِينَئِذٍ: «هَا نَحْنُ قَدْ تَرَكْنَا كُلَّ شَيْءٍ وَتَبِعْنَاكَ. فَمَاذَا يَكُونُ لَنَا؟»
\par 28 فَقَالَ لَهُمْ يَسُوعُ: «الْحَقَّ أَقُولُ لَكُمْ: إِنَّكُمْ أَنْتُمُ الَّذِينَ تَبِعْتُمُونِي فِي التَّجْدِيدِ مَتَى جَلَسَ ابْنُ الإِنْسَانِ عَلَى كُرْسِيِّ مَجْدِهِ تَجْلِسُونَ أَنْتُمْ أَيْضاً عَلَى اثْنَيْ عَشَرَ كُرْسِيّاً تَدِينُونَ أَسْبَاطَ إِسْرَائِيلَ الاِثْنَيْ عَشَرَ.
\par 29 وَكُلُّ مَنْ تَرَكَ بُيُوتاً أَوْ إِخْوَةً أَوْ أَخَوَاتٍ أَوْ أَباً أَوْ أُمّاً أَوِ امْرَأَةً أَوْ أَوْلاَداً أَوْ حُقُولاً مِنْ أَجْلِ اسْمِي يَأْخُذُ مِئَةَ ضِعْفٍ وَيَرِثُ الْحَيَاةَ الأَبَدِيَّةَ.
\par 30 وَلَكِنْ كَثِيرُونَ أَوَّلُونَ يَكُونُونَ آخِرِينَ وَآخِرُونَ أَوَّلِينَ».

\chapter{20}

\par 1 «فَإِنَّ مَلَكُوتَ السَّمَاوَاتِ يُشْبِهُ رَجُلاً رَبَّ بَيْتٍ خَرَجَ مَعَ الصُّبْحِ لِيَسْتَأْجِرَ فَعَلَةً لِكَرْمِهِ
\par 2 فَاتَّفَقَ مَعَ الْفَعَلَةِ عَلَى دِينَارٍ فِي الْيَوْمِ وَأَرْسَلَهُمْ إِلَى كَرْمِهِ.
\par 3 ثُمَّ خَرَجَ نَحْوَ السَّاعَةِ الثَّالِثَةِ وَرَأَى آخَرِينَ قِيَاماً فِي السُّوقِ بَطَّالِينَ
\par 4 فَقَالَ لَهُمُ: اذْهَبُوا أَنْتُمْ أَيْضاً إِلَى الْكَرْمِ فَأُعْطِيَكُمْ مَا يَحِقُّ لَكُمْ. فَمَضَوْا.
\par 5 وَخَرَجَ أَيْضاً نَحْوَ السَّاعَةِ السَّادِسَةِ وَالتَّاسِعَةِ وَفَعَلَ كَذَلِكَ.
\par 6 ثُمَّ نَحْوَ السَّاعَةِ الْحَادِيَةَ عَشْرَةَ خَرَجَ وَوَجَدَ آخَرِينَ قِيَاماً بَطَّالِينَ فَقَالَ لَهُمْ: لِمَاذَا وَقَفْتُمْ هَهُنَا كُلَّ النَّهَارِ بَطَّالِينَ؟
\par 7 قَالُوا لَهُ: لأَنَّهُ لَمْ يَسْتَأْجِرْنَا أَحَدٌ. قَالَ لَهُمُ: اذْهَبُوا أَنْتُمْ أَيْضاً إِلَى الْكَرْمِ فَتَأْخُذُوا مَا يَحِقُّ لَكُمْ.
\par 8 فَلَمَّا كَانَ الْمَسَاءُ قَالَ صَاحِبُ الْكَرْمِ لِوَكِيلِهِ: ادْعُ الْفَعَلَةَ وَأَعْطِهِمُِ الأُجْرَةَ مُبْتَدِئاً مِنَ الآخِرِينَ إِلَى الأَوَّلِينَ.
\par 9 فَجَاءَ أَصْحَابُ السَّاعَةِ الْحَادِيَةَ عَشْرَةَ وَأَخَذُوا دِينَاراً دِينَاراً.
\par 10 فَلَمَّا جَاءَ الأَوَّلُونَ ظَنُّوا أَنَّهُمْ يَأْخُذُونَ أَكْثَرَ. فَأَخَذُوا هُمْ أَيْضاً دِينَاراً دِينَاراً.
\par 11 وَفِيمَا هُمْ يَأْخُذُونَ تَذَمَّرُوا عَلَى رَبِّ الْبَيْتِ
\par 12 قَائِلِينَ: هَؤُلاَءِ الآخِرُونَ عَمِلُوا سَاعَةً وَاحِدَةً وَقَدْ سَاوَيْتَهُمْ بِنَا نَحْنُ الَّذِينَ احْتَمَلْنَا ثِقَلَ النَّهَارِ وَالْحَرَّ!
\par 13 فَقَالَ لِوَاحِدٍ مِنْهُمْ: يَا صَاحِبُ مَا ظَلَمْتُكَ! أَمَا اتَّفَقْتَ مَعِي عَلَى دِينَارٍ؟
\par 14 فَخُذِ الَّذِي لَكَ وَاذْهَبْ فَإِنِّي أُرِيدُ أَنْ أُعْطِيَ هَذَا الأَخِيرَ مِثْلَكَ.
\par 15 أَوَ مَا يَحِلُّ لِي أَنْ أَفْعَلَ مَا أُرِيدُ بِمَالِي؟ أَمْ عَيْنُكَ شِرِّيرَةٌ لأَنِّي أَنَا صَالِحٌ؟
\par 16 هَكَذَا يَكُونُ الآخِرُونَ أَوَّلِينَ وَالأَوَّلُونَ آخِرِينَ لأَنَّ كَثِيرِينَ يُدْعَوْنَ وَقَلِيلِينَ يُنْتَخَبُونَ».
\par 17 وَفِيمَا كَانَ يَسُوعُ صَاعِداً إِلَى أُورُشَلِيمَ أَخَذَ الاِثْنَيْ عَشَرَ تِلْمِيذاً عَلَى انْفِرَادٍ فِي الطَّرِيقِ وَقَالَ لَهُمْ:
\par 18 «هَا نَحْنُ صَاعِدُونَ إِلَى أُورُشَلِيمَ وَابْنُ الإِنْسَانِ يُسَلَّمُ إِلَى رُؤَسَاءِ الْكَهَنَةِ وَالْكَتَبَةِ فَيَحْكُمُونَ عَلَيْهِ بِالْمَوْتِ
\par 19 وَيُسَلِّمُونَهُ إِلَى الأُمَمِ لِكَيْ يَهْزَأُوا بِهِ وَيَجْلِدُوهُ وَيَصْلِبُوهُ وَفِي الْيَوْمِ الثَّالِثِ يَقُومُ».
\par 20 حِينَئِذٍ تَقَدَّمَتْ إِلَيْهِ أُمُّ ابْنَيْ زَبْدِي مَعَ ابْنَيْهَا وَسَجَدَتْ وَطَلَبَتْ مِنْهُ شَيْئاً.
\par 21 فَقَالَ لَهَا: «مَاذَا تُرِيدِينَ؟» قَالَتْ لَهُ: «قُلْ أَنْ يَجْلِسَ ابْنَايَ هَذَانِ وَاحِدٌ عَنْ يَمِينِكَ وَالآخَرُ عَنِ الْيَسَارِ فِي مَلَكُوتِكَ».
\par 22 فَأَجَابَ يَسُوعُ: «لَسْتُمَا تَعْلَمَانِ مَا تَطْلُبَانِ. أَتَسْتَطِيعَانِ أَنْ تَشْرَبَا الْكَأْسَ الَّتِي سَوْفَ أَشْرَبُهَا أَنَا وَأَنْ تَصْطَبِغَا بِالصِّبْغَةِ الَّتِي أَصْطَبِغُ بِهَا أَنَا؟» قَالاَ لَهُ: «نَسْتَطِيعُ».
\par 23 فَقَالَ لَهُمَا: «أَمَّا كَأْسِي فَتَشْرَبَانِهَا وَبِالصِّبْغَةِ الَّتِي أَصْطَبِغُ بِهَا أَنَا تَصْطَبِغَانِ. وَأَمَّا الْجُلُوسُ عَنْ يَمِينِي وَعَنْ يَسَارِي فَلَيْسَ لِي أَنْ أُعْطِيَهُ إِلاَّ لِلَّذِينَ أُعِدَّ لَهُمْ مِنْ أَبِي».
\par 24 فَلَمَّا سَمِعَ الْعَشَرَةُ اغْتَاظُوا مِنْ أَجْلِ الأَخَوَيْنِ.
\par 25 فَدَعَاهُمْ يَسُوعُ وَقَالَ: «أَنْتُمْ تَعْلَمُونَ أَنَّ رُؤَسَاءَ الأُمَمِ يَسُودُونَهُمْ وَالْعُظَمَاءَ يَتَسَلَّطُونَ عَلَيْهِمْ.
\par 26 فَلاَ يَكُونُ هَكَذَا فِيكُمْ. بَلْ مَنْ أَرَادَ أَنْ يَكُونَ فِيكُمْ عَظِيماً فَلْيَكُنْ لَكُمْ خَادِماً
\par 27 وَمَنْ أَرَادَ أَنْ يَكُونَ فِيكُمْ أَوَّلاً فَلْيَكُنْ لَكُمْ عَبْداً
\par 28 كَمَا أَنَّ ابْنَ الإِنْسَانِ لَمْ يَأْتِ لِيُخْدَمَ بَلْ لِيَخْدِمَ وَلِيَبْذِلَ نَفْسَهُ فِدْيَةً عَنْ كَثِيرِينَ».
\par 29 وَفِيمَا هُمْ خَارِجُونَ مِنْ أَرِيحَا تَبِعَهُ جَمْعٌ كَثِيرٌ
\par 30 وَإِذَا أَعْمَيَانِ جَالِسَانِ عَلَى الطَّرِيقِ. فَلَمَّا سَمِعَا أَنَّ يَسُوعَ مُجْتَازٌ صَرَخَا قَائِلَيْنِ: «ارْحَمْنَا يَا سَيِّدُ يَا ابْنَ دَاوُدَ».
\par 31 فَانْتَهَرَهُمَا الْجَمْعُ لِيَسْكُتَا فَكَانَا يَصْرَخَانِ أَكْثَرَ قَائِلَيْنِ: «ارْحَمْنَا يَا سَيِّدُ يَا ابْنَ دَاوُدَ»
\par 32 فَوَقَفَ يَسُوعُ وَنَادَاهُمَا وَقَالَ: «مَاذَا تُرِيدَانِ أَنْ أَفْعَلَ بِكُمَا؟»
\par 33 قَالاَ لَهُ: «يَا سَيِّدُ أَنْ تَنْفَتِحَ أَعْيُنُنَا!»
\par 34 فَتَحَنَّنَ يَسُوعُ وَلَمَسَ أَعْيُنَهُمَا فَلِلْوَقْتِ أَبْصَرَتْ أَعْيُنُهُمَا فَتَبِعَاهُ.

\chapter{21}

\par 1 وَلَمَّا قَرُبُوا مِنْ أُورُشَلِيمَ وَجَاءُوا إِلَى بَيْتِ فَاجِي عِنْدَ جَبَلِ الزَّيْتُونِ حِينَئِذٍ أَرْسَلَ يَسُوعُ تِلْمِيذَيْنِ
\par 2 قَائِلاً لَهُمَا: «اذْهَبَا إِلَى الْقَرْيَةِ الَّتِي أَمَامَكُمَا فَلِلْوَقْتِ تَجِدَانِ أَتَاناً مَرْبُوطَةً وَجَحْشاً مَعَهَا فَحُلَّاهُمَا وَأْتِيَانِي بِهِمَا.
\par 3 وَإِنْ قَالَ لَكُمَا أَحَدٌ شَيْئاً فَقُولاَ: الرَّبُّ مُحْتَاجٌ إِلَيْهِمَا. فَلِلْوَقْتِ يُرْسِلُهُمَا».
\par 4 فَكَانَ هَذَا كُلُّهُ لِكَيْ يَتِمَّ مَا قِيلَ بِالنَّبِيِّ:
\par 5 «قُولُوا لاِبْنَةِ صِهْيَوْنَ: هُوَذَا مَلِكُكِ يَأْتِيكِ وَدِيعاً رَاكِباً عَلَى أَتَانٍ وَجَحْشٍ ابْنِ أَتَانٍ».
\par 6 فَذَهَبَ التِّلْمِيذَانِ وَفَعَلاَ كَمَا أَمَرَهُمَا يَسُوعُ
\par 7 وَأَتَيَا بِالأَتَانِ وَالْجَحْشِ وَوَضَعَا عَلَيْهِمَا ثِيَابَهُمَا فَجَلَسَ عَلَيْهِمَا.
\par 8 وَالْجَمْعُ الأَكْثَرُ فَرَشُوا ثِيَابَهُمْ فِي الطَّرِيقِ. وَآخَرُونَ قَطَعُوا أَغْصَاناً مِنَ الشَّجَرِ وَفَرَشُوهَا فِي الطَّرِيقِ.
\par 9 وَالْجُمُوعُ الَّذِينَ تَقَدَّمُوا وَالَّذِينَ تَبِعُوا كَانُوا يَصْرَخُونَ: «أُوصَنَّا لاِبْنِ دَاوُدَ! مُبَارَكٌ الآتِي بِاسْمِ الرَّبِّ! أُوصَنَّا فِي الأَعَالِي!».
\par 10 وَلَمَّا دَخَلَ أُورُشَلِيمَ ارْتَجَّتِ الْمَدِينَةُ كُلُّهَا قَائِلَةً: «مَنْ هَذَا؟»
\par 11 فَقَالَتِ الْجُمُوعُ: «هَذَا يَسُوعُ النَّبِيُّ الَّذِي مِنْ نَاصِرَةِ الْجَلِيلِ».
\par 12 وَدَخَلَ يَسُوعُ إِلَى هَيْكَلِ اللَّهِ وَأَخْرَجَ جَمِيعَ الَّذِينَ كَانُوا يَبِيعُونَ وَيَشْتَرُونَ فِي الْهَيْكَلِ وَقَلَبَ مَوَائِدَ الصَّيَارِفَةِ وَكَرَاسِيَّ بَاعَةِ الْحَمَامِ
\par 13 وَقَالَ لَهُمْ: «مَكْتُوبٌ: بَيْتِي بَيْتَ الصَّلاَةِ يُدْعَى. وَأَنْتُمْ جَعَلْتُمُوهُ مَغَارَةَ لُصُوصٍ!»
\par 14 وَتَقَدَّمَ إِلَيْهِ عُمْيٌ وَعُرْجٌ فِي الْهَيْكَلِ فَشَفَاهُمْ.
\par 15 فَلَمَّا رَأَى رُؤَسَاءُ الْكَهَنَةِ وَالْكَتَبَةُ الْعَجَائِبَ الَّتِي صَنَعَ وَالأَوْلاَدَ يَصْرَخُونَ فِي الْهَيْكَلِ وَيَقُولُونَ: «أُوصَنَّا لاِبْنِ دَاوُدَ» غَضِبُوا
\par 16 وَقَالُوا لَهُ: «أَتَسْمَعُ مَا يَقُولُ هَؤُلاَءِ؟» فَقَالَ لَهُمْ يَسُوعُ: «نَعَمْ! أَمَا قَرَأْتُمْ قَطُّ: مِنْ أَفْوَاهِ الأَطْفَالِ وَالرُّضَّعِ هَيَّأْتَ تَسْبِيحاً؟».
\par 17 ثُمَّ تَرَكَهُمْ وَخَرَجَ خَارِجَ الْمَدِينَةِ إِلَى بَيْتِ عَنْيَا وَبَاتَ هُنَاكَ.
\par 18 وَفِي الصُّبْحِ إِذْ كَانَ رَاجِعاً إِلَى الْمَدِينَةِ جَاعَ
\par 19 فَنَظَرَ شَجَرَةَ تِينٍ عَلَى الطَّرِيقِ وَجَاءَ إِلَيْهَا فَلَمْ يَجِدْ فِيهَا شَيْئاً إِلاَّ وَرَقاً فَقَطْ. فَقَالَ لَهَا: «لاَ يَكُنْ مِنْكِ ثَمَرٌ بَعْدُ إِلَى الأَبَدِ». فَيَبِسَتِ التِّينَةُ فِي الْحَالِ.
\par 20 فَلَمَّا رَأَى التَّلاَمِيذُ ذَلِكَ تَعَجَّبُوا قَائِلِينَ: «كَيْفَ يَبِسَتِ التِّينَةُ فِي الْحَالِ؟»
\par 21 فَأَجَابَ يَسُوعُ: «اَلْحَقَّ أَقُولُ لَكُمْ: إِنْ كَانَ لَكُمْ إِيمَانٌ وَلاَ تَشُكُّونَ فَلاَ تَفْعَلُونَ أَمْرَ التِّينَةِ فَقَطْ بَلْ إِنْ قُلْتُمْ أَيْضاً لِهَذَا الْجَبَلِ: انْتَقِلْ وَانْطَرِحْ فِي الْبَحْرِ فَيَكُونُ.
\par 22 وَكُلُّ مَا تَطْلُبُونَهُ فِي الصَّلاَةِ مُؤْمِنِينَ تَنَالُونَهُ».
\par 23 وَلَمَّا جَاءَ إِلَى الْهَيْكَلِ تَقَدَّمَ إِلَيْهِ رُؤَسَاءُ الْكَهَنَةِ وَشُيُوخُ الشَّعْبِ وَهُوَ يُعَلِّمُ قَائِلِينَ: «بِأَيِّ سُلْطَانٍ تَفْعَلُ هَذَا وَمَنْ أَعْطَاكَ هَذَا السُّلْطَانَ؟»
\par 24 فَأَجَابَ يَسُوعُ: «وَأَنَا أَيْضاً أَسْأَلُكُمْ كَلِمَةً وَاحِدَةً فَإِنْ قُلْتُمْ لِي عَنْهَا أَقُولُ لَكُمْ أَنَا أَيْضاً بِأَيِّ سُلْطَانٍ أَفْعَلُ هَذَا:
\par 25 مَعْمُودِيَّةُ يُوحَنَّا مِنْ أَيْنَ كَانَتْ؟ مِنَ السَّمَاءِ أَمْ مِنَ النَّاسِ؟» فَفَكَّرُوا فِي أَنْفُسِهِمْ قَائِلِينَ: «إِنْ قُلْنَا مِنَ السَّمَاءِ يَقُولُ لَنَا: فَلِمَاذَا لَمْ تُؤْمِنُوا بِهِ؟
\par 26 وَإِنْ قُلْنَا: مِنَ النَّاسِ نَخَافُ مِنَ الشَّعْبِ لأَنَّ يُوحَنَّا عِنْدَ الْجَمِيعِ مِثْلُ نَبِيٍّ».
\par 27 فَأَجَابُوا يَسُوعَ: «لاَ نَعْلَمُ». فَقَالَ لَهُمْ هُوَ أَيْضاً: «وَلاَ أَنَا أَقُولُ لَكُمْ بِأَيِّ سُلْطَانٍ أَفْعَلُ هَذَا».
\par 28 «مَاذَا تَظُنُّونَ؟ كَانَ لِإِنْسَانٍ ابْنَانِ فَجَاءَ إِلَى الأَوَّلِ وَقَالَ: يَا ابْنِي اذْهَبِ الْيَوْمَ اعْمَلْ فِي كَرْمِي.
\par 29 فَأَجَابَ: مَا أُرِيدُ. وَلَكِنَّهُ نَدِمَ أَخِيراً وَمَضَى.
\par 30 وَجَاءَ إِلَى الثَّانِي وَقَالَ كَذَلِكَ. فَأَجَابَ: هَا أَنَا يَا سَيِّدُ. وَلَمْ يَمْضِ.
\par 31 فَأَيُّ الاِثْنَيْنِ عَمِلَ إِرَادَةَ الأَبِ؟» قَالُوا لَهُ: «الأَوَّلُ». قَالَ لَهُمْ يَسُوعُ: «الْحَقَّ أَقُولُ لَكُمْ إِنَّ الْعَشَّارِينَ وَالزَّوَانِيَ يَسْبِقُونَكُمْ إِلَى مَلَكُوتِ اللَّهِ
\par 32 لأَنَّ يُوحَنَّا جَاءَكُمْ فِي طَرِيقِ الْحَقِّ فَلَمْ تُؤْمِنُوا بِهِ وَأَمَّا الْعَشَّارُونَ وَالزَّوَانِي فَآمَنُوا بِهِ. وَأَنْتُمْ إِذْ رَأَيْتُمْ لَمْ تَنْدَمُوا أَخِيراً لِتُؤْمِنُوا بِهِ».
\par 33 «اسْمَعُوا مَثَلاً آخَرَ: كَانَ إِنْسَانٌ رَبُّ بَيْتٍ غَرَسَ كَرْماً وَأَحَاطَهُ بِسِيَاجٍ وَحَفَرَ فِيهِ مَعْصَرَةً وَبَنَى بُرْجاً وَسَلَّمَهُ إِلَى كَرَّامِينَ وَسَافَرَ.
\par 34 وَلَمَّا قَرُبَ وَقْتُ الأَثْمَارِ أَرْسَلَ عَبِيدَهُ إِلَى الْكَرَّامِينَ لِيَأْخُذَ أَثْمَارَهُ.
\par 35 فَأَخَذَ الْكَرَّامُونَ عَبِيدَهُ وَجَلَدُوا بَعْضاً وَقَتَلُوا بَعْضاً وَرَجَمُوا بَعْضاً.
\par 36 ثُمَّ أَرْسَلَ أَيْضاً عَبِيداً آخَرِينَ أَكْثَرَ مِنَ الأَوَّلِينَ فَفَعَلُوا بِهِمْ كَذَلِكَ.
\par 37 فَأَخِيراً أَرْسَلَ إِلَيْهِمُ ابْنَهُ قَائِلاً: يَهَابُونَ ابْنِي!
\par 38 وَأَمَّا الْكَرَّامُونَ فَلَمَّا رَأَوْا الاِبْنَ قَالُوا فِيمَا بَيْنَهُمْ: هَذَا هُوَ الْوَارِثُ. هَلُمُّوا نَقْتُلْهُ وَنَأْخُذْ مِيرَاثَهُ!
\par 39 فَأَخَذُوهُ وَأَخْرَجُوهُ خَارِجَ الْكَرْمِ وَقَتَلُوهُ.
\par 40 فَمَتَى جَاءَ صَاحِبُ الْكَرْمِ مَاذَا يَفْعَلُ بِأُولَئِكَ الْكَرَّامِينَ؟»
\par 41 قَالُوا لَهُ: «أُولَئِكَ الأَرْدِيَاءُ يُهْلِكُهُمْ هَلاَكاً رَدِيّاً وَيُسَلِّمُ الْكَرْمَ إِلَى كَرَّامِينَ آخَرِينَ يُعْطُونَهُ الأَثْمَارَ فِي أَوْقَاتِهَا».
\par 42 قَالَ لَهُمْ يَسُوعُ: «أَمَا قَرَأْتُمْ قَطُّ فِي الْكُتُبِ: الْحَجَرُ الَّذِي رَفَضَهُ الْبَنَّاؤُونَ هُوَ قَدْ صَارَ رَأْسَ الزَّاوِيَةِ. مِنْ قِبَلِ الرَّبِّ كَانَ هَذَا وَهُوَ عَجِيبٌ فِي أَعْيُنِنَا؟
\par 43 لِذَلِكَ أَقُولُ لَكُمْ: إِنَّ مَلَكُوتَ اللَّهِ يُنْزَعُ مِنْكُمْ وَيُعْطَى لِأُمَّةٍ تَعْمَلُ أَثْمَارَهُ.
\par 44 وَمَنْ سَقَطَ عَلَى هَذَا الْحَجَرِ يَتَرَضَّضُ وَمَنْ سَقَطَ هُوَ عَلَيْهِ يَسْحَقُهُ».
\par 45 وَلَمَّا سَمِعَ رُؤَسَاءُ الْكَهَنَةِ وَالْفَرِّيسِيُّونَ أَمْثَالَهُ عَرَفُوا أَنَّهُ تَكَلَّمَ عَلَيْهِمْ.
\par 46 وَإِذْ كَانُوا يَطْلُبُونَ أَنْ يُمْسِكُوهُ خَافُوا مِنَ الْجُمُوعِ لأَنَّهُ كَانَ عِنْدَهُمْ مِثْلَ نَبِيٍّ.

\chapter{22}

\par 1 وَجَعَلَ يَسُوعُ يُكَلِّمُهُمْ أَيْضاً بِأَمْثَالٍ قَائِلاً:
\par 2 «يُشْبِهُ مَلَكُوتُ السَّمَاوَاتِ إِنْسَاناً مَلِكاً صَنَعَ عُرْساً لاِبْنِهِ
\par 3 وَأَرْسَلَ عَبِيدَهُ لِيَدْعُوا الْمَدْعُوِّينَ إِلَى الْعُرْسِ فَلَمْ يُرِيدُوا أَنْ يَأْتُوا.
\par 4 فَأَرْسَلَ أَيْضاً عَبِيداً آخَرِينَ قَائِلاً: قُولُوا لِلْمَدْعُوِّينَ: هُوَذَا غَدَائِي أَعْدَدْتُهُ. ثِيرَانِي وَمُسَمَّنَاتِي قَدْ ذُبِحَتْ وَكُلُّ شَيْءٍ مُعَدٌّ. تَعَالَوْا إِلَى الْعُرْسِ!
\par 5 وَلَكِنَّهُمْ تَهَاوَنُوا وَمَضَوْا وَاحِدٌ إِلَى حَقْلِهِ وَآخَرُ إِلَى تِجَارَتِهِ
\par 6 وَالْبَاقُونَ أَمْسَكُوا عَبِيدَهُ وَشَتَمُوهُمْ وَقَتَلُوهُمْ.
\par 7 فَلَمَّا سَمِعَ الْمَلِكُ غَضِبَ وَأَرْسَلَ جُنُودَهُ وَأَهْلَكَ أُولَئِكَ الْقَاتِلِينَ وَأَحْرَقَ مَدِينَتَهُمْ.
\par 8 ثُمَّ قَالَ لِعَبِيدِهِ: أَمَّا الْعُرْسُ فَمُسْتَعَدٌّ وَأَمَّا الْمَدْعُوُّونَ فَلَمْ يَكُونُوا مُسْتَحِقِّينَ.
\par 9 فَاذْهَبُوا إِلَى مَفَارِقِ الطُّرُقِ وَكُلُّ مَنْ وَجَدْتُمُوهُ فَادْعُوهُ إِلَى الْعُرْسِ.
\par 10 فَخَرَجَ أُولَئِكَ الْعَبِيدُ إِلَى الطُّرُقِ وَجَمَعُوا كُلَّ الَّذِينَ وَجَدُوهُمْ أَشْرَاراً وَصَالِحِينَ. فَامْتَلَأَ الْعُرْسُ مِنَ الْمُتَّكِئِينَ.
\par 11 فَلَمَّا دَخَلَ الْمَلِكُ لِيَنْظُرَ الْمُتَّكِئِينَ رَأَى هُنَاكَ إِنْسَاناً لَمْ يَكُنْ لاَبِساً لِبَاسَ الْعُرْسِ.
\par 12 فَقَالَ لَهُ: يَا صَاحِبُ كَيْفَ دَخَلْتَ إِلَى هُنَا وَلَيْسَ عَلَيْكَ لِبَاسُ الْعُرْسِ؟ فَسَكَتَ.
\par 13 حِينَئِذٍ قَالَ الْمَلِكُ لِلْخُدَّامِ: ارْبُطُوا رِجْلَيْهِ وَيَدَيْهِ وَخُذُوهُ وَاطْرَحُوهُ فِي الظُّلْمَةِ الْخَارِجِيَّةِ. هُنَاكَ يَكُونُ الْبُكَاءُ وَصَرِيرُ الأَسْنَانِ.
\par 14 لأَنَّ كَثِيرِينَ يُدْعَوْنَ وَقَلِيلِينَ يُنْتَخَبُونَ».
\par 15 حِينَئِذٍ ذَهَبَ الْفَرِّيسِيُّونَ وَتَشَاوَرُوا لِكَيْ يَصْطَادُوهُ بِكَلِمَةٍ.
\par 16 فَأَرْسَلُوا إِلَيْهِ تَلاَمِيذَهُمْ مَعَ الْهِيرُودُسِيِّينَ قَائِلِينَ: «يَا مُعَلِّمُ نَعْلَمُ أَنَّكَ صَادِقٌ وَتُعَلِّمُ طَرِيقَ اللَّهِ بِالْحَقِّ وَلاَ تُبَالِي بِأَحَدٍ لأَنَّكَ لاَ تَنْظُرُ إِلَى وُجُوهِ النَّاسِ.
\par 17 فَقُلْ لَنَا مَاذَا تَظُنُّ؟ أَيَجُوزُ أَنْ تُعْطَى جِزْيَةٌ لِقَيْصَرَ أَمْ لاَ؟»
\par 18 فَعَلِمَ يَسُوعُ خُبْثَهُمْ وَقَالَ: «لِمَاذَا تُجَرِّبُونَنِي يَا مُرَاؤُونَ؟
\par 19 أَرُونِي مُعَامَلَةَ الْجِزْيَةِ». فَقَدَّمُوا لَهُ دِينَاراً.
\par 20 فَقَالَ لَهُمْ: «لِمَنْ هَذِهِ الصُّورَةُ وَالْكِتَابَةُ؟»
\par 21 قَالُوا لَهُ: «لِقَيْصَرَ». فَقَالَ لَهُمْ: «أَعْطُوا إِذاً مَا لِقَيْصَرَ لِقَيْصَرَ وَمَا لِلَّهِ لِلَّهِ».
\par 22 فَلَمَّا سَمِعُوا تَعَجَّبُوا وَتَرَكُوهُ وَمَضَوْا.
\par 23 فِي ذَلِكَ الْيَوْمِ جَاءَ إِلَيْهِ صَدُّوقِيُّونَ الَّذِينَ يَقُولُونَ لَيْسَ قِيَامَةٌ فَسَأَلُوهُ:
\par 24 «يَا مُعَلِّمُ قَالَ مُوسَى: إِنْ مَاتَ أَحَدٌ وَلَيْسَ لَهُ أَوْلاَدٌ يَتَزَوَّجْ أَخُوهُ بِامْرَأَتِهِ وَيُقِمْ نَسْلاً لأَخِيهِ.
\par 25 فَكَانَ عِنْدَنَا سَبْعَةُ إِخْوَةٍ وَتَزَوَّجَ الأَوَّلُ وَمَاتَ. وَإِذْ لَمْ يَكُنْ لَهُ نَسْلٌ تَرَكَ امْرَأَتَهُ لأَخِيهِ.
\par 26 وَكَذَلِكَ الثَّانِي وَالثَّالِثُ إِلَى السَّبْعَةِ.
\par 27 وَآخِرَ الْكُلِّ مَاتَتِ الْمَرْأَةُ أَيْضاً.
\par 28 فَفِي الْقِيَامَةِ لِمَنْ مِنَ السَّبْعَةِ تَكُونُ زَوْجَةً؟ فَإِنَّهَا كَانَتْ لِلْجَمِيعِ!»
\par 29 فَأَجَابَ يَسُوعُ: «تَضِلُّونَ إِذْ لاَ تَعْرِفُونَ الْكُتُبَ وَلاَ قُوَّةَ اللَّهِ.
\par 30 لأَنَّهُمْ فِي الْقِيَامَةِ لاَ يُزَوِّجُونَ وَلاَ يَتَزَوَّجُونَ بَلْ يَكُونُونَ كَمَلاَئِكَةِ اللَّهِ فِي السَّمَاءِ.
\par 31 وَأَمَّا مِنْ جِهَةِ قِيَامَةِ الأَمْوَاتِ أَفَمَا قَرَأْتُمْ مَا قِيلَ لَكُمْ مِنْ قِبَلِ اللَّهِ:
\par 32 أَنَا إِلَهُ إِبْراهِيمَ وَإِلَهُ إِسْحاقَ وَإِلَهُ يَعْقُوبَ. لَيْسَ اللَّهُ إِلَهَ أَمْوَاتٍ بَلْ إِلَهُ أَحْيَاءٍ».
\par 33 فَلَمَّا سَمِعَ الْجُمُوعُ بُهِتُوا مِنْ تَعْلِيمِهِ.
\par 34 أَمَّا الْفَرِّيسِيُّونَ فَلَمَّا سَمِعُوا أَنَّهُ أَبْكَمَ الصَّدُّوقِيِّينَ اجْتَمَعُوا مَعاً
\par 35 وَسَأَلَهُ وَاحِدٌ مِنْهُمْ وَهُوَ نَامُوسِيٌّ لِيُجَرِّبَهُ:
\par 36 «يَا مُعَلِّمُ أَيَّةُ وَصِيَّةٍ هِيَ الْعُظْمَى فِي النَّامُوسِ؟»
\par 37 فَقَالَ لَهُ يَسُوعُ: «تُحِبُّ الرَّبَّ إِلَهَكَ مِنْ كُلِّ قَلْبِكَ وَمِنْ كُلِّ نَفْسِكَ وَمِنْ كُلِّ فِكْرِكَ.
\par 38 هَذِهِ هِيَ الْوَصِيَّةُ الأُولَى وَالْعُظْمَى.
\par 39 وَالثَّانِيَةُ مِثْلُهَا: تُحِبُّ قَرِيبَكَ كَنَفْسِكَ.
\par 40 بِهَاتَيْنِ الْوَصِيَّتَيْنِ يَتَعَلَّقُ النَّامُوسُ كُلُّهُ وَالأَنْبِيَاءُ».
\par 41 وَفِيمَا كَانَ الْفَرِّيسِيُّونَ مُجْتَمِعِينَ سَأَلَهُمْ يَسُوعُ:
\par 42 «مَاذَا تَظُنُّونَ فِي الْمَسِيحِ؟ ابْنُ مَنْ هُوَ؟» قَالُوا لَهُ: «ابْنُ دَاوُدَ».
\par 43 قَالَ لَهُمْ: «فَكَيْفَ يَدْعُوهُ دَاوُدُ بِالرُّوحِ رَبّاً قَائِلاً:
\par 44 قَالَ الرَّبُّ لِرَبِّي اجْلِسْ عَنْ يَمِينِي حَتَّى أَضَعَ أَعْدَاءَكَ مَوْطِئاً لِقَدَمَيْكَ؟
\par 45 فَإِنْ كَانَ دَاوُدُ يَدْعُوهُ رَبّاً فَكَيْفَ يَكُونُ ابْنَهُ؟»
\par 46 فَلَمْ يَسْتَطِعْ أَحَدٌ أَنْ يُجِيبَهُ بِكَلِمَةٍ. وَمِنْ ذَلِكَ الْيَوْمِ لَمْ يَجْسُرْ أَحَدٌ أَنْ يَسْأَلَهُ بَتَّةً.

\chapter{23}

\par 1 حِينَئِذٍ خَاطَبَ يَسُوعُ الْجُمُوعَ وَتَلاَمِيذَهُ
\par 2 قَائِلاً: «عَلَى كُرْسِيِّ مُوسَى جَلَسَ الْكَتَبَةُ وَالْفَرِّيسِيُّونَ
\par 3 فَكُلُّ مَا قَالُوا لَكُمْ أَنْ تَحْفَظُوهُ فَاحْفَظُوهُ وَافْعَلُوهُ وَلَكِنْ حَسَبَ أَعْمَالِهِمْ لاَ تَعْمَلُوا لأَنَّهُمْ يَقُولُونَ وَلاَ يَفْعَلُونَ.
\par 4 فَإِنَّهُمْ يَحْزِمُونَ أَحْمَالاً ثَقِيلَةً عَسِرَةَ الْحَمْلِ وَيَضَعُونَهَا عَلَى أَكْتَافِ النَّاسِ وَهُمْ لاَ يُرِيدُونَ أَنْ يُحَرِّكُوهَا بِإِصْبِعِهِمْ
\par 5 وَكُلَّ أَعْمَالِهِمْ يَعْمَلُونَهَا لِكَيْ تَنْظُرَهُمُ النَّاسُ فَيُعَرِّضُونَ عَصَائِبَهُمْ وَيُعَظِّمُونَ أَهْدَابَ ثِيَابِهِمْ
\par 6 وَيُحِبُّونَ الْمُتَّكَأَ الأَوَّلَ فِي الْوَلاَئِمِ وَالْمَجَالِسَ الأُولَى فِي الْمَجَامِعِ
\par 7 وَالتَّحِيَّاتِ فِي الأَسْوَاقِ وَأَنْ يَدْعُوَهُمُ النَّاسُ: سَيِّدِي سَيِّدِي!
\par 8 وَأَمَّا أَنْتُمْ فَلاَ تُدْعَوْا سَيِّدِي لأَنَّ مُعَلِّمَكُمْ وَاحِدٌ الْمَسِيحُ وَأَنْتُمْ جَمِيعاً إِخْوَةٌ.
\par 9 وَلاَ تَدْعُوا لَكُمْ أَباً عَلَى الأَرْضِ لأَنَّ أَبَاكُمْ وَاحِدٌ الَّذِي فِي السَّمَاوَاتِ.
\par 10 وَلاَ تُدْعَوْا مُعَلِّمِينَ لأَنَّ مُعَلِّمَكُمْ وَاحِدٌ الْمَسِيحُ.
\par 11 وَأَكْبَرُكُمْ يَكُونُ خَادِماً لَكُمْ.
\par 12 فَمَنْ يَرْفَعْ نَفْسَهُ يَتَّضِعْ وَمَنْ يَضَعْ نَفْسَهُ يَرْتَفِعْ.
\par 13 «لَكِنْ وَيْلٌ لَكُمْ أَيُّهَا الْكَتَبَةُ وَالْفَرِّيسِيُّونَ الْمُرَاؤُونَ لأَنَّكُمْ تُغْلِقُونَ مَلَكُوتَ السَّمَاوَاتِ قُدَّامَ النَّاسِ فَلاَ تَدْخُلُونَ أَنْتُمْ وَلاَ تَدَعُونَ الدَّاخِلِينَ يَدْخُلُونَ!
\par 14 وَيْلٌ لَكُمْ أَيُّهَا الْكَتَبَةُ وَالْفَرِّيسِيُّونَ الْمُرَاؤُونَ لأَنَّكُمْ تَأْكُلُونَ بُيُوتَ الأَرَامِلِ ولِعِلَّةٍ تُطِيلُونَ صَلَوَاتِكُمْ. لِذَلِكَ تَأْخُذُونَ دَيْنُونَةً أَعْظَمَ.
\par 15 وَيْلٌ لَكُمْ أَيُّهَا الْكَتَبَةُ وَالْفَرِّيسِيُّونَ الْمُرَاؤُونَ لأَنَّكُمْ تَطُوفُونَ الْبَحْرَ وَالْبَرَّ لِتَكْسَبُوا دَخِيلاً وَاحِداً وَمَتَى حَصَلَ تَصْنَعُونَهُ ابْناً لِجَهَنَّمَ أَكْثَرَ مِنْكُمْ مُضَاعَفاً!
\par 16 وَيْلٌ لَكُمْ أَيُّهَا الْقَادَةُ الْعُمْيَانُ الْقَائِلُونَ: مَنْ حَلَفَ بِالْهَيْكَلِ فَلَيْسَ بِشَيْءٍ وَلَكِنْ مَنْ حَلَفَ بِذَهَبِ الْهَيْكَلِ يَلْتَزِمُ!
\par 17 أَيُّهَا الْجُهَّالُ وَالْعُمْيَانُ أَيُّمَا أَعْظَمُ: أَلذَّهَبُ أَمِ الْهَيْكَلُ الَّذِي يُقَدِّسُ الذَّهَبَ؟
\par 18 وَمَنْ حَلَفَ بِالْمَذْبَحِ فَلَيْسَ بِشَيْءٍ وَلَكِنْ مَنْ حَلَفَ بِالْقُرْبَانِ الَّذِي عَلَيْهِ يَلْتَزِمُ!
\par 19 أَيُّهَا الْجُهَّالُ وَالْعُمْيَانُ أَيُّمَا أَعْظَمُ: أَلْقُرْبَانُ أَمِ الْمَذْبَحُ الَّذِي يُقَدِّسُ الْقُرْبَانَ؟
\par 20 فَإِنَّ مَنْ حَلَفَ بِالْمَذْبَحِ فَقَدْ حَلَفَ بِهِ وَبِكُلِّ مَا عَلَيْهِ
\par 21 وَمَنْ حَلَفَ بِالْهَيْكَلِ فَقَدْ حَلَفَ بِهِ وَبِالسَّاكِنِ فِيهِ
\par 22 وَمَنْ حَلَفَ بِالسَّمَاءِ فَقَدْ حَلَفَ بِعَرْشِ اللَّهِ وَبِالْجَالِسِ عَلَيْهِ!
\par 23 وَيْلٌ لَكُمْ أَيُّهَا الْكَتَبَةُ وَالْفَرِّيسِيُّونَ الْمُرَاؤُونَ لأَنَّكُمْ تُعَشِّرُونَ النَّعْنَعَ وَالشِّبِثَّ وَالْكَمُّونَ وَتَرَكْتُمْ أَثْقَلَ النَّامُوسِ: الْحَقَّ وَالرَّحْمَةَ وَالإِيمَانَ. كَانَ يَنْبَغِي أَنْ تَعْمَلُوا هَذِهِ وَلاَ تَتْرُكُوا تِلْكَ.
\par 24 أَيُّهَا الْقَادَةُ الْعُمْيَانُ الَّذِينَ يُصَفُّونَ عَنِ الْبَعُوضَةِ وَيَبْلَعُونَ الْجَمَلَ!
\par 25 وَيْلٌ لَكُمْ أَيُّهَا الْكَتَبَةُ وَالْفَرِّيسِيُّونَ الْمُرَاؤُونَ لأَنَّكُمْ تُنَقُّونَ خَارِجَ الْكَأْسِ وَالصَّحْفَةِ وَهُمَا مِنْ دَاخِلٍ مَمْلُوآنِ اخْتِطَافاً وَدَعَارَةً!
\par 26 أَيُّهَا الْفَرِّيسِيُّ الأَعْمَى نَقِّ أَوَّلاً دَاخِلَ الْكَأْسِ وَالصَّحْفَةِ لِكَيْ يَكُونَ خَارِجُهُمَا أَيْضاً نَقِيّاً.
\par 27 وَيْلٌ لَكُمْ أَيُّهَا الْكَتَبَةُ وَالْفَرِّيسِيُّونَ الْمُرَاؤُونَ لأَنَّكُمْ تُشْبِهُونَ قُبُوراً مُبَيَّضَةً تَظْهَرُ مِنْ خَارِجٍ جَمِيلَةً وَهِيَ مِنْ دَاخِلٍ مَمْلُوءَةٌ عِظَامَ أَمْوَاتٍ وَكُلَّ نَجَاسَةٍ.
\par 28 هَكَذَا أَنْتُمْ أَيْضاً: مِنْ خَارِجٍ تَظْهَرُونَ لِلنَّاسِ أَبْرَاراً وَلَكِنَّكُمْ مِنْ دَاخِلٍ مَشْحُونُونَ رِيَاءً وَإِثْماً!
\par 29 وَيْلٌ لَكُمْ أَيُّهَا الْكَتَبَةُ وَالْفَرِّيسِيُّونَ الْمُرَاؤُونَ لأَنَّكُمْ تَبْنُونَ قُبُورَ الأَنْبِيَاءِ وَتُزَيِّنُونَ مَدَافِنَ الصِّدِّيقِينَ
\par 30 وَتَقُولُونَ: لَوْ كُنَّا فِي أَيَّامِ آبَائِنَا لَمَا شَارَكْنَاهُمْ فِي دَمِ الأَنْبِيَاءِ!
\par 31 فَأَنْتُمْ تَشْهَدُونَ عَلَى أَنْفُسِكُمْ أَنَّكُمْ أَبْنَاءُ قَتَلَةِ الأَنْبِيَاءِ.
\par 32 فَامْلَأُوا أَنْتُمْ مِكْيَالَ آبَائِكُمْ.
\par 33 أَيُّهَا الْحَيَّاتُ أَوْلاَدَ الأَفَاعِي كَيْفَ تَهْرُبُونَ مِنْ دَيْنُونَةِ جَهَنَّمَ؟
\par 34 لِذَلِكَ هَا أَنَا أُرْسِلُ إِلَيْكُمْ أَنْبِيَاءَ وَحُكَمَاءَ وَكَتَبَةً فَمِنْهُمْ تَقْتُلُونَ وَتَصْلِبُونَ وَمِنْهُمْ تَجْلِدُونَ فِي مَجَامِعِكُمْ وَتَطْرُدُونَ مِنْ مَدِينَةٍ إِلَى مَدِينَةٍ
\par 35 لِكَيْ يَأْتِيَ عَلَيْكُمْ كُلُّ دَمٍ زَكِيٍّ سُفِكَ عَلَى الأَرْضِ مِنْ دَمِ هَابِيلَ الصِّدِّيقِ إِلَى دَمِ زَكَرِيَّا بْنِ بَرَخِيَّا الَّذِي قَتَلْتُمُوهُ بَيْنَ الْهَيْكَلِ وَالْمَذْبَحِ.
\par 36 اَلْحَقَّ أَقُولُ لَكُمْ: إِنَّ هَذَا كُلَّهُ يَأْتِي عَلَى هَذَا الْجِيلِ!
\par 37 «يَا أُورُشَلِيمُ يَا أُورُشَلِيمُ يَا قَاتِلَةَ الأَنْبِيَاءِ وَرَاجِمَةَ الْمُرْسَلِينَ إِلَيْهَا كَمْ مَرَّةٍ أَرَدْتُ أَنْ أَجْمَعَ أَوْلاَدَكِ كَمَا تَجْمَعُ الدَّجَاجَةُ فِرَاخَهَا تَحْتَ جَنَاحَيْهَا وَلَمْ تُرِيدُوا.
\par 38 هُوَذَا بَيْتُكُمْ يُتْرَكُ لَكُمْ خَرَاباً!
\par 39 لأَنِّي أَقُولُ لَكُمْ: إِنَّكُمْ لاَ تَرَوْنَنِي مِنَ الآنَ حَتَّى تَقُولُوا: مُبَارَكٌ الآتِي بِاسْمِ الرَّبِّ!».

\chapter{24}

\par 1 ثُمَّ خَرَجَ يَسُوعُ وَمَضَى مِنَ الْهَيْكَلِ فَتَقَدَّمَ تَلاَمِيذُهُ لِكَيْ يُرُوهُ أَبْنِيَةَ الْهَيْكَلِ.
\par 2 فَقَالَ لَهُمْ يَسُوعُ: «أَمَا تَنْظُرُونَ جَمِيعَ هَذِهِ؟ اَلْحَقَّ أَقُولُ لَكُمْ إِنَّهُ لاَ يُتْرَكُ هَهُنَا حَجَرٌ عَلَى حَجَرٍ لاَ يُنْقَضُ!».
\par 3 وَفِيمَا هُوَ جَالِسٌ عَلَى جَبَلِ الزَّيْتُونِ تَقَدَّمَ إِلَيْهِ التَّلاَمِيذُ عَلَى انْفِرَادٍ قَائِلِينَ: «قُلْ لَنَا مَتَى يَكُونُ هَذَا وَمَا هِيَ عَلاَمَةُ مَجِيئِكَ وَانْقِضَاءِ الدَّهْرِ؟»
\par 4 فَأَجَابَ يَسُوعُ: «انْظُرُوا لاَ يُضِلَّكُمْ أَحَدٌ.
\par 5 فَإِنَّ كَثِيرِينَ سَيَأْتُونَ بِاسْمِي قَائِلِينَ: أَنَا هُوَ الْمَسِيحُ وَيُضِلُّونَ كَثِيرِينَ.
\par 6 وَسَوْفَ تَسْمَعُونَ بِحُرُوبٍ وَأَخْبَارِ حُرُوبٍ. اُنْظُرُوا لاَ تَرْتَاعُوا. لِأَنَّهُ لاَ بُدَّ أَنْ تَكُونَ هَذِهِ كُلُّهَا. وَلكِنْ لَيْسَ الْمُنْتَهَى بَعْدُ.
\par 7 لِأَنَّهُ تَقُومُ أُمَّةٌ عَلى أُمَّةٍ وَمَمْلَكَةٌ عَلى مَمْلَكَةٍ وَتَكُونُ مَجَاعَاتٌ وَأَوْبِئَةٌ وَزَلاَزِلُ فِي أَمَاكِنَ.
\par 8 وَلَكِنَّ هَذِهِ كُلَّهَا مُبْتَدَأُ الأَوْجَاعِ.
\par 9 حِينَئِذٍ يُسَلِّمُونَكُمْ إِلَى ضِيقٍ وَيَقْتُلُونَكُمْ وَتَكُونُونَ مُبْغَضِينَ مِنْ جَمِيعِ الأُمَمِ لأَجْلِ اسْمِي.
\par 10 وَحِينَئِذٍ يَعْثُرُ كَثِيرُونَ وَيُسَلِّمُونَ بَعْضُهُمْ بَعْضاً وَيُبْغِضُونَ بَعْضُهُمْ بَعْضاً.
\par 11 وَيَقُومُ أَنْبِيَاءُ كَذَبَةٌ كَثِيرُونَ وَيُضِلُّونَ كَثِيرِينَ.
\par 12 وَلِكَثْرَةِ الإِثْمِ تَبْرُدُ مَحَبَّةُ الْكَثِيرِينَ.
\par 13 وَلَكِنِ الَّذِي يَصْبِرُ إِلَى الْمُنْتَهَى فَهَذَا يَخْلُصُ.
\par 14 وَيُكْرَزُ بِبِشَارَةِ الْمَلَكُوتِ هَذِهِ فِي كُلِّ الْمَسْكُونَةِ شَهَادَةً لِجَمِيعِ الأُمَمِ. ثُمَّ يَأْتِي الْمُنْتَهَى.
\par 15 «فَمَتَى نَظَرْتُمْ «رِجْسَةَ الْخَرَابِ» الَّتِي قَالَ عَنْهَا دَانِيآلُ النَّبِيُّ قَائِمَةً فِي الْمَكَانِ الْمُقَدَّسِ - لِيَفْهَمِ الْقَارِئُ -
\par 16 فَحِينَئِذٍ لِيَهْرُبِ الَّذِينَ فِي الْيَهُودِيَّةِ إِلَى الْجِبَالِ
\par 17 وَالَّذِي عَلَى السَّطْحِ فَلاَ يَنْزِلْ لِيَأْخُذَ مِنْ بَيْتِهِ شَيْئاً
\par 18 وَالَّذِي فِي الْحَقْلِ فَلاَ يَرْجِعْ إِلَى وَرَائِهِ لِيَأْخُذَ ثِيَابَهُ.
\par 19 وَوَيْلٌ لِلْحَبَالَى وَالْمُرْضِعَاتِ فِي تِلْكَ الأَيَّامِ!
\par 20 وَصَلُّوا لِكَيْ لاَ يَكُونَ هَرَبُكُمْ فِي شِتَاءٍ وَلاَ فِي سَبْتٍ
\par 21 لأَنَّهُ يَكُونُ حِينَئِذٍ ضِيقٌ عَظِيمٌ لَمْ يَكُنْ مِثْلُهُ مُنْذُ ابْتِدَاءِ الْعَالَمِ إِلَى الآنَ وَلَنْ يَكُونَ.
\par 22 وَلَوْ لَمْ تُقَصَّرْ تِلْكَ الأَيَّامُ لَمْ يَخْلُصْ جَسَدٌ. وَلَكِنْ لأَجْلِ الْمُخْتَارِينَ تُقَصَّرُ تِلْكَ الأَيَّامُ.
\par 23 حِينَئِذٍ إِنْ قَالَ لَكُمْ أَحَدٌ: هُوَذَا الْمَسِيحُ هُنَا أَوْ هُنَاكَ فَلاَ تُصَدِّقُوا.
\par 24 لأَنَّهُ سَيَقُومُ مُسَحَاءُ كَذَبَةٌ وَأَنْبِيَاءُ كَذَبَةٌ وَيُعْطُونَ آيَاتٍ عَظِيمَةً وَعَجَائِبَ حَتَّى يُضِلُّوا لَوْ أَمْكَنَ الْمُخْتَارِينَ أَيْضاً.
\par 25 هَا أَنَا قَدْ سَبَقْتُ وَأَخْبَرْتُكُمْ.
\par 26 فَإِنْ قَالُوا لَكُمْ: هَا هُوَ فِي الْبَرِّيَّةِ فَلاَ تَخْرُجُوا! هَا هُوَ فِي الْمَخَادِعِ فَلاَ تُصَدِّقُوا!
\par 27 لأَنَّهُ كَمَا أَنَّ الْبَرْقَ يَخْرُجُ مِنَ الْمَشَارِقِ وَيَظْهَرُ إِلَى الْمَغَارِبِ هَكَذَا يَكُونُ أَيْضاً مَجِيءُ ابْنِ الإِنْسَانِ.
\par 28 لأَنَّهُ حَيْثُمَا تَكُنِ الْجُثَّةُ فَهُنَاكَ تَجْتَمِعُ النُّسُورُ.
\par 29 «وَلِلْوَقْتِ بَعْدَ ضِيقِ تِلْكَ الأَيَّامِ تُظْلِمُ الشَّمْسُ وَالْقَمَرُ لاَ يُعْطِي ضَوْءَهُ وَالنُّجُومُ تَسْقُطُ مِنَ السَّمَاءِ وَقُوَّاتُ السَّمَاوَاتِ تَتَزَعْزَعُ.
\par 30 وَحِينَئِذٍ تَظْهَرُ عَلاَمَةُ ابْنِ الإِنْسَانِ فِي السَّمَاءِ. وَحِينَئِذٍ تَنُوحُ جَمِيعُ قَبَائِلِ الأَرْضِ وَيُبْصِرُونَ ابْنَ الإِنْسَانِ آتِياً عَلَى سَحَابِ السَّمَاءِ بِقُوَّةٍ وَمَجْدٍ كَثِيرٍ.
\par 31 فَيُرْسِلُ مَلاَئِكَتَهُ بِبُوقٍ عَظِيمِ الصَّوْتِ فَيَجْمَعُونَ مُخْتَارِيهِ مِنَ الأَرْبَعِ الرِّيَاحِ مِنْ أَقْصَاءِ السَّمَاوَاتِ إِلَى أَقْصَائِهَا.
\par 32 فَمِنْ شَجَرَةِ التِّينِ تَعَلَّمُوا الْمَثَلَ: مَتَى صَارَ غُصْنُهَا رَخْصاً وَأَخْرَجَتْ أَوْرَاقَهَا تَعْلَمُونَ أَنَّ الصَّيْفَ قَرِيبٌ.
\par 33 هَكَذَا أَنْتُمْ أَيْضاً مَتَى رَأَيْتُمْ هَذَا كُلَّهُ فَاعْلَمُوا أَنَّهُ قَرِيبٌ عَلَى الأَبْوَابِ.
\par 34 اَلْحَقَّ أَقُولُ لَكُمْ: لاَ يَمْضِي هَذَا الْجِيلُ حَتَّى يَكُونَ هَذَا كُلُّهُ.
\par 35 اَلسَّمَاءُ وَالأَرْضُ تَزُولاَنِ وَلَكِنَّ كَلاَمِي لاَ يَزُولُ.
\par 36 وَأَمَّا ذَلِكَ الْيَوْمُ وَتِلْكَ السَّاعَةُ فَلاَ يَعْلَمُ بِهِمَا أَحَدٌ وَلاَ مَلاَئِكَةُ السَّمَاوَاتِ إِلاَّ أَبِي وَحْدَهُ.
\par 37 وَكَمَا كَانَتْ أَيَّامُ نُوحٍ كَذَلِكَ يَكُونُ أَيْضاً مَجِيءُ ابْنِ الإِنْسَانِ.
\par 38 لأَنَّهُ كَمَا كَانُوا فِي الأَيَّامِ الَّتِي قَبْلَ الطُّوفَانِ يَأْكُلُونَ وَيَشْرَبُونَ وَيَتَزَوَّجُونَ وَيُزَوِّجُونَ إِلَى الْيَوْمِ الَّذِي دَخَلَ فِيهِ نُوحٌ الْفُلْكَ
\par 39 وَلَمْ يَعْلَمُوا حَتَّى جَاءَ الطُّوفَانُ وَأَخَذَ الْجَمِيعَ كَذَلِكَ يَكُونُ أَيْضاً مَجِيءُ ابْنِ الإِنْسَانِ.
\par 40 حِينَئِذٍ يَكُونُ اثْنَانِ فِي الْحَقْلِ يُؤْخَذُ الْوَاحِدُ وَيُتْرَكُ الآخَرُ.
\par 41 اثْنَتَانِ تَطْحَنَانِ عَلَى الرَّحَى تُؤْخَذُ الْوَاحِدَةُ وَتُتْرَكُ الأُخْرَى.
\par 42 «اِسْهَرُوا إِذاً لأَنَّكُمْ لاَ تَعْلَمُونَ فِي أَيَّةِ سَاعَةٍ يَأْتِي رَبُّكُمْ.
\par 43 وَاعْلَمُوا هَذَا أَنَّهُ لَوْ عَرَفَ رَبُّ الْبَيْتِ فِي أَيِّ هَزِيعٍ يَأْتِي السَّارِقُ لَسَهِرَ وَلَمْ يَدَعْ بَيْتَهُ يُنْقَبُ.
\par 44 لِذَلِكَ كُونُوا أَنْتُمْ أَيْضاً مُسْتَعِدِّينَ لأَنَّهُ فِي سَاعَةٍ لاَ تَظُنُّونَ يَأْتِي ابْنُ الإِنْسَانِ.
\par 45 فَمَنْ هُوَ الْعَبْدُ الأَمِينُ الْحَكِيمُ الَّذِي أَقَامَهُ سَيِّدُهُ عَلَى خَدَمِهِ لِيُعْطِيَهُمُ الطَّعَامَ فِي حِينِهِ؟
\par 46 طُوبَى لِذَلِكَ الْعَبْدِ الَّذِي إِذَا جَاءَ سَيِّدُهُ يَجِدُهُ يَفْعَلُ هَكَذَا!
\par 47 اَلْحَقَّ أَقُولُ لَكُمْ إِنَّهُ يُقِيمُهُ عَلَى جَمِيعِ أَمْوَالِهِ.
\par 48 وَلَكِنْ إِنْ قَالَ ذَلِكَ الْعَبْدُ الرَّدِيُّ فِي قَلْبِهِ: سَيِّدِي يُبْطِئُ قُدُومَهُ.
\par 49 فَيَبْتَدِئُ يَضْرِبُ الْعَبِيدَ رُفَقَاءَهُ وَيَأْكُلُ وَيَشْرَبُ مَعَ السُّكَارَى.
\par 50 يَأْتِي سَيِّدُ ذَلِكَ الْعَبْدِ فِي يَوْمٍ لاَ يَنْتَظِرُهُ وَفِي سَاعَةٍ لاَ يَعْرِفُهَا
\par 51 فَيُقَطِّعُهُ وَيَجْعَلُ نَصِيبَهُ مَعَ الْمُرَائِينَ. هُنَاكَ يَكُونُ الْبُكَاءُ وَصَرِيرُ الأَسْنَانِ».

\chapter{25}

\par 1 «حِينَئِذٍ يُشْبِهُ مَلَكُوتُ السَّمَاوَاتِ عَشْرَ عَذَارَى أَخَذْنَ مَصَابِيحَهُنَّ وَخَرَجْنَ لِلِقَاءِ الْعَرِيسِ.
\par 2 وَكَانَ خَمْسٌ مِنْهُنَّ حَكِيمَاتٍ وَخَمْسٌ جَاهِلاَتٍ.
\par 3 أَمَّا الْجَاهِلاَتُ فَأَخَذْنَ مَصَابِيحَهُنَّ وَلَمْ يَأْخُذْنَ مَعَهُنَّ زَيْتاً
\par 4 وَأَمَّا الْحَكِيمَاتُ فَأَخَذْنَ زَيْتاً فِي آنِيَتِهِنَّ مَعَ مَصَابِيحِهِنَّ.
\par 5 وَفِيمَا أَبْطَأَ الْعَرِيسُ نَعَسْنَ جَمِيعُهُنَّ وَنِمْنَ.
\par 6 فَفِي نِصْفِ اللَّيْلِ صَارَ صُرَاخٌ: هُوَذَا الْعَرِيسُ مُقْبِلٌ فَاخْرُجْنَ لِلِقَائِهِ!
\par 7 فَقَامَتْ جَمِيعُ أُولَئِكَ الْعَذَارَى وَأَصْلَحْنَ مَصَابِيحَهُنَّ.
\par 8 فَقَالَتِ الْجَاهِلاَتُ لِلْحَكِيمَاتِ: أَعْطِينَنَا مِنْ زَيْتِكُنَّ فَإِنَّ مَصَابِيحَنَا تَنْطَفِئُ.
\par 9 فَأَجَابَتِ الْحَكِيمَاتُ: لَعَلَّهُ لاَ يَكْفِي لَنَا وَلَكُنَّ بَلِ اذْهَبْنَ إِلَى الْبَاعَةِ وَابْتَعْنَ لَكُنَّ.
\par 10 وَفِيمَا هُنَّ ذَاهِبَاتٌ لِيَبْتَعْنَ جَاءَ الْعَرِيسُ وَالْمُسْتَعِدَّاتُ دَخَلْنَ مَعَهُ إِلَى الْعُرْسِ وَأُغْلِقَ الْبَابُ.
\par 11 أَخِيراً جَاءَتْ بَقِيَّةُ الْعَذَارَى أَيْضاً قَائِلاَتٍ: يَا سَيِّدُ يَا سَيِّدُ افْتَحْ لَنَا.
\par 12 فَأَجَابَ: الْحَقَّ أَقُولُ لَكُنَّ: إِنِّي مَا أَعْرِفُكُنَّ.
\par 13 فَاسْهَرُوا إِذاً لأَنَّكُمْ لاَ تَعْرِفُونَ الْيَوْمَ وَلاَ السَّاعَةَ الَّتِي يَأْتِي فِيهَا ابْنُ الإِنْسَانِ.
\par 14 «وَكَأَنَّمَا إِنْسَانٌ مُسَافِرٌ دَعَا عَبِيدَهُ وَسَلَّمَهُمْ أَمْوَالَهُ
\par 15 فَأَعْطَى وَاحِداً خَمْسَ وَزَنَاتٍ وَآخَرَ وَزْنَتَيْنِ وَآخَرَ وَزْنَةً - كُلَّ وَاحِدٍ عَلَى قَدْرِ طَاقَتِهِ. وَسَافَرَ لِلْوَقْتِ.
\par 16 فَمَضَى الَّذِي أَخَذَ الْخَمْسَ* وَزَنَاتٍ وَتَاجَرَ بِهَا فَرَبِحَ خَمْسَ وَزَنَاتٍ أُخَرَ.
\par 17 وَهَكَذَا الَّذِي أَخَذَ الْوَزْنَتَيْنِ رَبِحَ أَيْضاً وَزْنَتَيْنِ أُخْرَيَيْنِ.
\par 18 وَأَمَّا الَّذِي أَخَذَ الْوَزْنَةَ فَمَضَى وَحَفَرَ فِي الأَرْضِ وَأَخْفَى فِضَّةَ سَيِّدِهِ.
\par 19 وَبَعْدَ زَمَانٍ طَوِيلٍ أَتَى سَيِّدُ أُولَئِكَ الْعَبِيدِ وَحَاسَبَهُمْ.
\par 20 فَجَاءَ الَّذِي أَخَذَ الْخَمْسَ وَزَنَاتٍ وَقَدَّمَ خَمْسَ وَزَنَاتٍ أُخَرَ قَائِلاً: يَا سَيِّدُ خَمْسَ وَزَنَاتٍ سَلَّمْتَنِي. هُوَذَا خَمْسُ وَزَنَاتٍ أُخَرُ رَبِحْتُهَا فَوْقَهَا.
\par 21 فَقَالَ لَهُ سَيِّدُهُ: نِعِمَّا أَيُّهَا الْعَبْدُ الصَّالِحُ وَالأَمِينُ. كُنْتَ أَمِيناً فِي الْقَلِيلِ فَأُقِيمُكَ عَلَى الْكَثِيرِ. ادْخُلْ إِلَى فَرَحِ سَيِّدِكَ.
\par 22 ثُمَّ جَاءَ الَّذِي أَخَذَ الْوَزْنَتَيْنِ وَقَالَ: يَا سَيِّدُ وَزْنَتَيْنِ سَلَّمْتَنِي. هُوَذَا وَزْنَتَانِ أُخْرَيَانِ رَبِحْتُهُمَا فَوْقَهُمَا.
\par 23 قَالَ لَهُ سَيِّدُهُ: نِعِمَّا أَيُّهَا الْعَبْدُ الصَّالِحُ الأَمِينُ. كُنْتَ أَمِيناً فِي الْقَلِيلِ فَأُقِيمُكَ عَلَى الْكَثِيرِ. ادْخُلْ إِلَى فَرَحِ سَيِّدِكَ.
\par 24 ثُمَّ جَاءَ أَيْضاً الَّذِي أَخَذَ الْوَزْنَةَ الْوَاحِدَةَ وَقَالَ: يَا سَيِّدُ عَرَفْتُ أَنَّكَ إِنْسَانٌ قَاسٍ تَحْصُدُ حَيْثُ لَمْ تَزْرَعْ وَتَجْمَعُ مِنْ حَيْثُ لَمْ تَبْذُرْ.
\par 25 فَخِفْتُ وَمَضَيْتُ وَأَخْفَيْتُ وَزْنَتَكَ فِي الأَرْضِ. هُوَذَا الَّذِي لَكَ.
\par 26 فَأَجَابَ سَيِّدُهُ: أَيُّهَا الْعَبْدُ الشِّرِّيرُ وَالْكَسْلاَنُ عَرَفْتَ أَنِّي أَحْصُدُ حَيْثُ لَمْ أَزْرَعْ وَأَجْمَعُ مِنْ حَيْثُ لَمْ أَبْذُرْ
\par 27 فَكَانَ يَنْبَغِي أَنْ تَضَعَ فِضَّتِي عِنْدَ الصَّيَارِفَةِ فَعِنْدَ مَجِيئِي كُنْتُ آخُذُ الَّذِي لِي مَعَ رِباً.
\par 28 فَخُذُوا مِنْهُ الْوَزْنَةَ وَأَعْطُوهَا لِلَّذِي لَهُ الْعَشْرُ وَزَنَاتٍ.
\par 29 لأَنَّ كُلَّ مَنْ لَهُ يُعْطَى فَيَزْدَادُ وَمَنْ لَيْسَ لَهُ فَالَّذِي عِنْدَهُ يُؤْخَذُ مِنْهُ.
\par 30 وَالْعَبْدُ الْبَطَّالُ اطْرَحُوهُ إِلَى الظُّلْمَةِ الْخَارِجِيَّةِ هُنَاكَ يَكُونُ الْبُكَاءُ وَصَرِيرُ الأَسْنَانِ.
\par 31 «وَمَتَى جَاءَ ابْنُ الإِنْسَانِ فِي مَجْدِهِ وَجَمِيعُ الْمَلاَئِكَةِ الْقِدِّيسِينَ مَعَهُ فَحِينَئِذٍ يَجْلِسُ عَلَى كُرْسِيِّ مَجْدِهِ.
\par 32 وَيَجْتَمِعُ أَمَامَهُ جَمِيعُ الشُّعُوبِ فَيُمَيِّزُ بَعْضَهُمْ مِنْ بَعْضٍ كَمَا يُمَيِّزُ الرَّاعِي الْخِرَافَ مِنَ الْجِدَاءِ
\par 33 فَيُقِيمُ الْخِرَافَ عَنْ يَمِينِهِ وَالْجِدَاءَ عَنِ الْيَسَارِ.
\par 34 ثُمَّ يَقُولُ الْمَلِكُ لِلَّذِينَ عَنْ يَمِينِهِ: تَعَالَوْا يَا مُبَارَكِي أَبِي رِثُوا الْمَلَكُوتَ الْمُعَدَّ لَكُمْ مُنْذُ تَأْسِيسِ الْعَالَمِ.
\par 35 لأَنِّي جُعْتُ فَأَطْعَمْتُمُونِي. عَطِشْتُ فَسَقَيْتُمُونِي. كُنْتُ غَرِيباً فَآوَيْتُمُونِي.
\par 36 عُرْيَاناً فَكَسَوْتُمُونِي. مَرِيضاً فَزُرْتُمُونِي. مَحْبُوساً فَأَتَيْتُمْ إِلَيَّ.
\par 37 فَيُجِيبُهُ الأَبْرَارُ حِينَئِذٍ: يَارَبُّ مَتَى رَأَيْنَاكَ جَائِعاً فَأَطْعَمْنَاكَ أَوْ عَطْشَاناً فَسَقَيْنَاكَ؟
\par 38 وَمَتَى رَأَيْنَاكَ غَرِيباً فَآوَيْنَاكَ أَوْ عُرْيَاناً فَكَسَوْنَاكَ؟
\par 39 وَمَتَى رَأَيْنَاكَ مَرِيضاً أَوْ مَحْبُوساً فَأَتَيْنَا إِلَيْكَ؟
\par 40 فَيُجِيبُ الْمَلِكُ: الْحَقَّ أَقُولُ لَكُمْ: بِمَا أَنَّكُمْ فَعَلْتُمُوهُ بِأَحَدِ إِخْوَتِي هَؤُلاَءِ الأَصَاغِرِ فَبِي فَعَلْتُمْ.
\par 41 «ثُمَّ يَقُولُ أَيْضاً لِلَّذِينَ عَنِ الْيَسَارِ: اذْهَبُوا عَنِّي يَا مَلاَعِينُ إِلَى النَّارِ الأَبَدِيَّةِ الْمُعَدَّةِ لِإِبْلِيسَ وَمَلاَئِكَتِهِ
\par 42 لأَنِّي جُعْتُ فَلَمْ تُطْعِمُونِي. عَطِشْتُ فَلَمْ تَسْقُونِي.
\par 43 كُنْتُ غَرِيباً فَلَمْ تَأْوُونِي. عُرْيَاناً فَلَمْ تَكْسُونِي. مَرِيضاً وَمَحْبُوساً فَلَمْ تَزُورُونِي.
\par 44 حِينَئِذٍ يُجِيبُونَهُ هُمْ أَيْضاً: يَارَبُّ مَتَى رَأَيْنَاكَ جَائِعاً أَوْ عَطْشَاناً أَوْ غَرِيباً أَوْ عُرْيَاناً أَوْ مَرِيضاً أَوْ مَحْبُوساً وَلَمْ نَخْدِمْكَ؟
\par 45 فَيُجِيبُهُمْ: الْحَقَّ أَقُولُ لَكُمْ: بِمَا أَنَّكُمْ لَمْ تَفْعَلُوهُ بِأَحَدِ هَؤُلاَءِ الأَصَاغِرِ فَبِي لَمْ تَفْعَلُوا.
\par 46 فَيَمْضِي هَؤُلاَءِ إِلَى عَذَابٍ أَبَدِيٍّ وَالأَبْرَارُ إِلَى حَيَاةٍ أَبَدِيَّةٍ».

\chapter{26}

\par 1 وَلَمَّا أَكْمَلَ يَسُوعُ هَذِهِ الأَقْوَالَ كُلَّهَا قَالَ لِتَلاَمِيذِهِ:
\par 2 «تَعْلَمُونَ أَنَّهُ بَعْدَ يَوْمَيْنِ يَكُونُ الْفِصْحُ وَابْنُ الإِنْسَانِ يُسَلَّمُ لِيُصْلَبَ».
\par 3 حِينَئِذٍ اجْتَمَعَ رُؤَسَاءُ الْكَهَنَةِ وَالْكَتَبَةُ وَشُيُوخُ الشَّعْبِ إِلَى دَارِ رَئِيسِ الْكَهَنَةِ الَّذِي يُدْعَى قَيَافَا
\par 4 وَتَشَاوَرُوا لِكَيْ يُمْسِكُوا يَسُوعَ بِمَكْرٍ وَيَقْتُلُوهُ.
\par 5 وَلَكِنَّهُمْ قَالُوا: «لَيْسَ فِي الْعِيدِ لِئَلَّا يَكُونَ شَغَبٌ فِي الشَّعْبِ».
\par 6 وَفِيمَا كَانَ يَسُوعُ فِي بَيْتِ عَنْيَا فِي بَيْتِ سِمْعَانَ الأَبْرَصِ
\par 7 تَقَدَّمَتْ إِلَيْهِ امْرَأَةٌ مَعَهَا قَارُورَةُ طِيبٍ كَثِيرِ الثَّمَنِ فَسَكَبَتْهُ عَلَى رَأْسِهِ وَهُوَ مُتَّكِئٌ.
\par 8 فَلَمَّا رَأَى تَلاَمِيذُهُ ذَلِكَ اغْتَاظُوا قَائِلِينَ: «لِمَاذَا هَذَا الإِتْلاَفُ؟
\par 9 لأَنَّهُ كَانَ يُمْكِنُ أَنْ يُبَاعَ هَذَا الطِّيبُ بِكَثِيرٍ وَيُعْطَى لِلْفُقَرَاءِ».
\par 10 فَعَلِمَ يَسُوعُ وَقَالَ لَهُمْ: «لِمَاذَا تُزْعِجُونَ الْمَرْأَةَ؟ فَإِنَّهَا قَدْ عَمِلَتْ بِي عَمَلاً حَسَناً!
\par 11 لأَنَّ الْفُقَرَاءَ مَعَكُمْ فِي كُلِّ حِينٍ وَأَمَّا أَنَا فَلَسْتُ مَعَكُمْ فِي كُلِّ حِينٍ.
\par 12 فَإِنَّهَا إِذْ سَكَبَتْ هَذَا الطِّيبَ عَلَى جَسَدِي إِنَّمَا فَعَلَتْ ذَلِكَ لأَجْلِ تَكْفِينِي.
\par 13 اَلْحَقَّ أَقُولُ لَكُمْ: حَيْثُمَا يُكْرَزْ بِهَذَا الإِنْجِيلِ فِي كُلِّ الْعَالَمِ يُخْبَرْ أَيْضاً بِمَا فَعَلَتْهُ هَذِهِ تَذْكَاراً لَهَا».
\par 14 حِينَئِذٍ ذَهَبَ وَاحِدٌ مِنَ الاِثْنَيْ عَشَرَ الَّذِي يُدْعَى يَهُوذَا الإِسْخَرْيُوطِيَّ إِلَى رُؤَسَاءِ الْكَهَنَةِ
\par 15 وَقَالَ: «مَاذَا تُرِيدُونَ أَنْ تُعْطُونِي وَأَنَا أُسَلِّمُهُ إِلَيْكُمْ؟» فَجَعَلُوا لَهُ ثَلاَثِينَ مِنَ الْفِضَّةِ.
\par 16 وَمِنْ ذَلِكَ الْوَقْتِ كَانَ يَطْلُبُ فُرْصَةً لِيُسَلِّمَهُ.
\par 17 وَفِي أَوَّلِ أَيَّامِ الْفَطِيرِ تَقَدَّمَ التَّلاَمِيذُ إِلَى يَسُوعَ قَائِلِينَ: «أَيْنَ تُرِيدُ أَنْ نُعِدَّ لَكَ لِتَأْكُلَ الْفِصْحَ؟»
\par 18 فَقَالَ: «اذْهَبُوا إِلَى الْمَدِينَةِ إِلَى فُلاَنٍ وَقُولُوا لَهُ: الْمُعَلِّمُ يَقُولُ إِنَّ وَقْتِي قَرِيبٌ. عِنْدَكَ أَصْنَعُ الْفِصْحَ مَعَ تَلاَمِيذِي».
\par 19 فَفَعَلَ التَّلاَمِيذُ كَمَا أَمَرَهُمْ يَسُوعُ وَأَعَدُّوا الْفِصْحَ.
\par 20 وَلَمَّا كَانَ الْمَسَاءُ اتَّكَأَ مَعَ الاِثْنَيْ عَشَرَ.
\par 21 وَفِيمَا هُمْ يَأْكُلُونَ قَالَ: «الْحَقَّ أَقُولُ لَكُمْ إِنَّ وَاحِداً مِنْكُمْ يُسَلِّمُنِي».
\par 22 فَحَزِنُوا جِدّاً وَابْتَدَأَ كُلُّ وَاحِدٍ مِنْهُمْ يَقُولُ لَهُ: «هَلْ أَنَا هُوَ يَا رَبُّ؟»
\par 23 فَأَجَابَ: «الَّذِي يَغْمِسُ يَدَهُ مَعِي فِي الصَّحْفَةِ هُوَ يُسَلِّمُنِي.
\par 24 إِنَّ ابْنَ الإِنْسَانِ مَاضٍ كَمَا هُوَ مَكْتُوبٌ عَنْهُ وَلَكِنْ وَيْلٌ لِذَلِكَ الرَّجُلِ الَّذِي بِهِ يُسَلَّمُ ابْنُ الإِنْسَانِ. كَانَ خَيْراً لِذَلِكَ الرَّجُلِ لَوْ لَمْ يُولَدْ».
\par 25 فَسَأَلَ يَهُوذَا مُسَلِّمُهُ: «هَلْ أَنَا هُوَ يَا سَيِّدِي؟» قَالَ لَهُ: «أَنْتَ قُلْتَ».
\par 26 وَفِيمَا هُمْ يَأْكُلُونَ أَخَذَ يَسُوعُ الْخُبْزَ وَبَارَكَ وَكَسَّرَ وَأَعْطَى التَّلاَمِيذَ وَقَالَ: «خُذُوا كُلُوا. هَذَا هُوَ جَسَدِي».
\par 27 وَأَخَذَ الْكَأْسَ وَشَكَرَ وَأَعْطَاهُمْ قَائِلاً: «اشْرَبُوا مِنْهَا كُلُّكُمْ
\par 28 لأَنَّ هَذَا هُوَ دَمِي الَّذِي لِلْعَهْدِ الْجَدِيدِ الَّذِي يُسْفَكُ مِنْ أَجْلِ كَثِيرِينَ لِمَغْفِرَةِ الْخَطَايَا.
\par 29 وَأَقُولُ لَكُمْ: إِنِّي مِنَ الآنَ لاَ أَشْرَبُ مِنْ نِتَاجِ الْكَرْمَةِ هَذَا إِلَى ذَلِكَ الْيَوْمِ حِينَمَا أَشْرَبُهُ مَعَكُمْ جَدِيداً فِي مَلَكُوتِ أَبِي».
\par 30 ثُمَّ سَبَّحُوا وَخَرَجُوا إِلَى جَبَلِ الزَّيْتُونِ.
\par 31 حِينَئِذٍ قَالَ لَهُمْ يَسُوعُ: «كُلُّكُمْ تَشُكُّونَ فِيَّ فِي هَذِهِ اللَّيْلَةِ لأَنَّهُ مَكْتُوبٌ: أَنِّي أَضْرِبُ الرَّاعِيَ فَتَتَبَدَّدُ خِرَافُ الرَّعِيَّةِ.
\par 32 وَلَكِنْ بَعْدَ قِيَامِي أَسْبِقُكُمْ إِلَى الْجَلِيلِ».
\par 33 فَقَالَ بُطْرُسُ لَهُ: «وَإِنْ شَكَّ فِيكَ الْجَمِيعُ فَأَنَا لاَ أَشُكُّ أَبَداً».
\par 34 قَالَ لَهُ يَسُوعُ: «الْحَقَّ أَقُولُ لَكَ: إِنَّكَ فِي هَذِهِ اللَّيْلَةِ قَبْلَ أَنْ يَصِيحَ دِيكٌ تُنْكِرُنِي ثَلاَثَ مَرَّاتٍ».
\par 35 قَالَ لَهُ بُطْرُسُ: «وَلَوِ اضْطُرِرْتُ أَنْ أَمُوتَ مَعَكَ لاَ أُنْكِرُكَ!» هَكَذَا قَالَ أَيْضاً جَمِيعُ التَّلاَمِيذِ.
\par 36 حِينَئِذٍ جَاءَ مَعَهُمْ يَسُوعُ إِلَى ضَيْعَةٍ يُقَالُ لَهَا جَثْسَيْمَانِي فَقَالَ لِلتَّلاَمِيذِ: «اجْلِسُوا هَهُنَا حَتَّى أَمْضِيَ وَأُصَلِّيَ هُنَاكَ».
\par 37 ثُمَّ أَخَذَ مَعَهُ بُطْرُسَ وَابْنَيْ زَبْدِي وَابْتَدَأَ يَحْزَنُ وَيَكْتَئِبُ.
\par 38 فَقَالَ لَهُمْ: «نَفْسِي حَزِينَةٌ جِدّاً حَتَّى الْمَوْتِ. امْكُثُوا هَهُنَا وَاسْهَرُوا مَعِي».
\par 39 ثُمَّ تَقَدَّمَ قَلِيلاً وَخَرَّ عَلَى وَجْهِهِ وَكَانَ يُصَلِّي قَائِلاً: «يَا أَبَتَاهُ إِنْ أَمْكَنَ فَلْتَعْبُرْ عَنِّي هَذِهِ الْكَأْسُ وَلَكِنْ لَيْسَ كَمَا أُرِيدُ أَنَا بَلْ كَمَا تُرِيدُ أَنْتَ».
\par 40 ثُمَّ جَاءَ إِلَى التَّلاَمِيذِ فَوَجَدَهُمْ نِيَاماً فَقَالَ لِبُطْرُسَ: «أَهَكَذَا مَا قَدَرْتُمْ أَنْ تَسْهَرُوا مَعِي سَاعَةً وَاحِدَةً؟
\par 41 اسْهَرُوا وَصَلُّوا لِئَلَّا تَدْخُلُوا فِي تَجْرِبَةٍ. أَمَّا الرُّوحُ فَنَشِيطٌ وَأَمَّا الْجَسَدُ فَضَعِيفٌ».
\par 42 فَمَضَى أَيْضاً ثَانِيَةً وَصَلَّى قَائِلاً: «يَا أَبَتَاهُ إِنْ لَمْ يُمْكِنْ أَنْ تَعْبُرَ عَنِّي هَذِهِ الْكَأْسُ إِلاَّ أَنْ أَشْرَبَهَا فَلْتَكُنْ مَشِيئَتُكَ».
\par 43 ثُمَّ جَاءَ فَوَجَدَهُمْ أَيْضاً نِيَاماً إِذْ كَانَتْ أَعْيُنُهُمْ ثَقِيلَةً.
\par 44 فَتَرَكَهُمْ وَمَضَى أَيْضاً وَصَلَّى ثَالِثَةً قَائِلاً ذَلِكَ الْكَلاَمَ بِعَيْنِهِ.
\par 45 ثُمَّ جَاءَ إِلَى تَلاَمِيذِهِ وَقَالَ لَهُمْ: «نَامُوا الآنَ وَاسْتَرِيحُوا. هُوَذَا السَّاعَةُ قَدِ اقْتَرَبَتْ وَابْنُ الإِنْسَانِ يُسَلَّمُ إِلَى أَيْدِي الْخُطَاةِ.
\par 46 قُومُوا نَنْطَلِقْ. هُوَذَا الَّذِي يُسَلِّمُنِي قَدِ اقْتَرَبَ».
\par 47 وَفِيمَا هُوَ يَتَكَلَّمُ إِذَا يَهُوذَا أَحَدُ الاِثْنَيْ عَشَرَ قَدْ جَاءَ وَمَعَهُ جَمْعٌ كَثِيرٌ بِسُيُوفٍ وَعِصِيٍّ مِنْ عِنْدِ رُؤَسَاءِ الْكَهَنَةِ وَشُيُوخِ الشَّعْبِ.
\par 48 وَالَّذِي أَسْلَمَهُ أَعْطَاهُمْ عَلاَمَةً قَائِلاً: «الَّذِي أُقَبِّلُهُ هُوَ هُوَ. أَمْسِكُوهُ».
\par 49 فَلِلْوَقْتِ تَقَدَّمَ إِلَى يَسُوعَ وَقَالَ: «السَّلاَمُ يَا سَيِّدِي!» وَقَبَّلَهُ.
\par 50 فَقَالَ لَهُ يَسُوعُ: «يَا صَاحِبُ لِمَاذَا جِئْتَ؟» حِينَئِذٍ تَقَدَّمُوا وَأَلْقَوُا الأَيَادِيَ عَلَى يَسُوعَ وَأَمْسَكُوهُ.
\par 51 وَإِذَا وَاحِدٌ مِنَ الَّذِينَ مَعَ يَسُوعَ مَدَّ يَدَهُ وَاسْتَلَّ سَيْفَهُ وَضَرَبَ عَبْدَ رَئِيسِ الْكَهَنَةِ فَقَطَعَ أُذْنَهُ.
\par 52 فَقَالَ لَهُ يَسُوعُ: «رُدَّ سَيْفَكَ إِلَى مَكَانِهِ. لأَنَّ كُلَّ الَّذِينَ يَأْخُذُونَ السَّيْفَ بِالسَّيْفِ يَهْلِكُونَ!
\par 53 أَتَظُنُّ أَنِّي لاَ أَسْتَطِيعُ الآنَ أَنْ أَطْلُبَ إِلَى أَبِي فَيُقَدِّمَ لِي أَكْثَرَ مِنِ اثْنَيْ عَشَرَ جَيْشاً مِنَ الْمَلاَئِكَةِ؟
\par 54 فَكَيْفَ تُكَمَّلُ الْكُتُبُ: أَنَّهُ هَكَذَا يَنْبَغِي أَنْ يَكُونَ؟».
\par 55 فِي تِلْكَ السَّاعَةِ قَالَ يَسُوعُ لِلْجُمُوعِ: «كَأَنَّهُ عَلَى لِصٍّ خَرَجْتُمْ بِسُيُوفٍ وَعِصِيٍّ لِتَأْخُذُونِي! كُلَّ يَوْمٍ كُنْتُ أَجْلِسُ مَعَكُمْ أُعَلِّمُ فِي الْهَيْكَلِ وَلَمْ تُمْسِكُونِي.
\par 56 وَأَمَّا هَذَا كُلُّهُ فَقَدْ كَانَ لِكَيْ تُكَمَّلَ كُتُبُ الأَنْبِيَاءِ». حِينَئِذٍ تَرَكَهُ التَّلاَمِيذُ كُلُّهُمْ وَهَرَبُوا.
\par 57 وَالَّذِينَ أَمْسَكُوا يَسُوعَ مَضَوْا بِهِ إِلَى قَيَافَا رَئِيسِ الْكَهَنَةِ حَيْثُ اجْتَمَعَ الْكَتَبَةُ وَالشُّيُوخُ.
\par 58 وَأَمَّا بُطْرُسُ فَتَبِعَهُ مِنْ بَعِيدٍ إِلَى دَارِ رَئِيسِ الْكَهَنَةِ فَدَخَلَ إِلَى دَاخِلٍ وَجَلَسَ بَيْنَ الْخُدَّامِ لِيَنْظُرَ النِّهَايَةَ.
\par 59 وَكَانَ رُؤَسَاءُ الْكَهَنَةِ وَالشُّيُوخُ وَالْمَجْمَعُ كُلُّهُ يَطْلُبُونَ شَهَادَةَ زُورٍ عَلَى يَسُوعَ لِكَيْ يَقْتُلُوهُ
\par 60 فَلَمْ يَجِدُوا. وَمَعَ أَنَّهُ جَاءَ شُهُودُ زُورٍ كَثِيرُونَ لَمْ يَجِدُوا. وَلَكِنْ أَخِيراً تَقَدَّمَ شَاهِدَا زُورٍ
\par 61 وَقَالاَ: «هَذَا قَالَ إِنِّي أَقْدِرُ أَنْ أَنْقُضَ هَيْكَلَ اللَّهِ وَفِي ثَلاَثَةِ أَيَّامٍ أَبْنِيهِ».
\par 62 فَقَامَ رَئِيسُ الْكَهَنَةِ وَقَالَ لَهُ: «أَمَا تُجِيبُ بِشَيْءٍ؟ مَاذَا يَشْهَدُ بِهِ هَذَانِ عَلَيْكَ؟»
\par 63 وَأَمَّا يَسُوعُ فَكَانَ سَاكِتاً. فَسَأَلَهُ رَئِيسُ الْكَهَنَةِ: «أَسْتَحْلِفُكَ بِاللَّهِ الْحَيِّ أَنْ تَقُولَ لَنَا: هَلْ أَنْتَ الْمَسِيحُ ابْنُ اللَّهِ؟»
\par 64 قَالَ لَهُ يَسُوعُ: «أَنْتَ قُلْتَ! وَأَيْضاً أَقُولُ لَكُمْ: مِنَ الآنَ تُبْصِرُونَ ابْنَ الإِنْسَانِ جَالِساً عَنْ يَمِينِ الْقُوَّةِ وَآتِياً عَلَى سَحَابِ السَّمَاءِ».
\par 65 فَمَزَّقَ رَئِيسُ الْكَهَنَةِ حِينَئِذٍ ثِيَابَهُ قَائِلاً: «قَدْ جَدَّفَ! مَا حَاجَتُنَا بَعْدُ إِلَى شُهُودٍ؟ هَا قَدْ سَمِعْتُمْ تَجْدِيفَهُ!
\par 66 مَاذَا تَرَوْنَ؟» فَأَجَابُوا: «إِنَّهُ مُسْتَوْجِبُ الْمَوْتِ».
\par 67 حِينَئِذٍ بَصَقُوا فِي وَجْهِهِ وَلَكَمُوهُ وَآخَرُونَ لَطَمُوهُ
\par 68 قَائِلِينَ: «تَنَبَّأْ لَنَا أَيُّهَا الْمَسِيحُ مَنْ ضَرَبَكَ؟».
\par 69 أَمَّا بُطْرُسُ فَكَانَ جَالِساً خَارِجاً فِي الدَّارِ فَجَاءَتْ إِلَيْهِ جَارِيَةٌ قَائِلَةً: «وَأَنْتَ كُنْتَ مَعَ يَسُوعَ الْجَلِيلِيِّ».
\par 70 فَأَنْكَرَ قُدَّامَ الْجَمِيعِ قَائِلاً: «لَسْتُ أَدْرِي مَا تَقُولِينَ!»
\par 71 ثُمَّ إِذْ خَرَجَ إِلَى الدِّهْلِيزِ رَأَتْهُ أُخْرَى فَقَالَتْ لِلَّذِينَ هُنَاكَ: «وَهَذَا كَانَ مَعَ يَسُوعَ النَّاصِرِيِّ!»
\par 72 فَأَنْكَرَ أَيْضاً بِقَسَمٍ: «إِنِّي لَسْتُ أَعْرِفُ الرَّجُلَ!»
\par 73 وَبَعْدَ قَلِيلٍ جَاءَ الْقِيَامُ وَقَالُوا لِبُطْرُسَ: «حَقّاً أَنْتَ أَيْضاً مِنْهُمْ فَإِنَّ لُغَتَكَ تُظْهِرُكَ!»
\par 74 فَابْتَدَأَ حِينَئِذٍ يَلْعَنُ وَيَحْلِفُ: «إِنِّي لاَ أَعْرِفُ الرَّجُلَ!» وَلِلْوَقْتِ صَاحَ الدِّيكُ.
\par 75 فَتَذَكَّرَ بُطْرُسُ كَلاَمَ يَسُوعَ الَّذِي قَالَ لَهُ: «إِنَّكَ قَبْلَ أَنْ يَصِيحَ الدِّيكُ تُنْكِرُنِي ثَلاَثَ مَرَّاتٍ». فَخَرَجَ إِلَى خَارِجٍ وَبَكَى بُكَاءً مُرّاً.

\chapter{27}

\par 1 وَلَمَّا كَانَ الصَّبَاحُ تَشَاوَرَ جَمِيعُ رُؤَسَاءِ الْكَهَنَةِ وَشُيُوخُ الشَّعْبِ عَلَى يَسُوعَ حَتَّى يَقْتُلُوهُ
\par 2 فَأَوْثَقُوهُ وَمَضَوْا بِهِ وَدَفَعُوهُ إِلَى بِيلاَطُسَ الْبُنْطِيِّ الْوَالِي.
\par 3 حِينَئِذٍ لَمَّا رَأَى يَهُوذَا الَّذِي أَسْلَمَهُ أَنَّهُ قَدْ دِينَ نَدِمَ وَرَدَّ الثَّلاَثِينَ مِنَ الْفِضَّةِ إِلَى رُؤَسَاءِ الْكَهَنَةِ وَالشُّيُوخِ
\par 4 قَائِلاً: «قَدْ أَخْطَأْتُ إِذْ سَلَّمْتُ دَماً بَرِيئاً». فَقَالُوا: «مَاذَا عَلَيْنَا؟ أَنْتَ أَبْصِرْ!»
\par 5 فَطَرَحَ الْفِضَّةَ فِي الْهَيْكَلِ وَانْصَرَفَ ثُمَّ مَضَى وَخَنَقَ نَفْسَهُ.
\par 6 فَأَخَذَ رُؤَسَاءُ الْكَهَنَةِ الْفِضَّةَ وَقَالُوا: «لاَ يَحِلُّ أَنْ نُلْقِيَهَا فِي الْخِزَانَةِ لأَنَّهَا ثَمَنُ دَمٍ».
\par 7 فَتَشَاوَرُوا وَاشْتَرَوْا بِهَا حَقْلَ الْفَخَّارِيِّ مَقْبَرَةً لِلْغُرَبَاءِ.
\par 8 لِهَذَا سُمِّيَ ذَلِكَ الْحَقْلُ «حَقْلَ الدَّمِ» إِلَى هَذَا الْيَوْمِ.
\par 9 حِينَئِذٍ تَمَّ مَا قِيلَ بِإِرْمِيَا النَّبِيِّ: «وَأَخَذُوا الثَّلاَثِينَ مِنَ الْفِضَّةِ ثَمَنَ الْمُثَمَّنِ الَّذِي ثَمَّنُوهُ مِنْ بَنِي إِسْرَائِيلَ
\par 10 وَأَعْطَوْهَا عَنْ حَقْلِ الْفَخَّارِيِّ كَمَا أَمَرَنِي الرَّبُّ».
\par 11 فَوَقَفَ يَسُوعُ أَمَامَ الْوَالِي. فَسَأَلَهُ الْوَالِي: «أَأَنْتَ مَلِكُ الْيَهُودِ؟» فَقَالَ لَهُ يَسُوعُ: «أَنْتَ تَقُولُ».
\par 12 وَبَيْنَمَا كَانَ رُؤَسَاءُ الْكَهَنَةِ وَالشُّيُوخُ يَشْتَكُونَ عَلَيْهِ لَمْ يُجِبْ بِشَيْءٍ.
\par 13 فَقَالَ لَهُ بِيلاَطُسُ: «أَمَا تَسْمَعُ كَمْ يَشْهَدُونَ عَلَيْكَ؟»
\par 14 فَلَمْ يُجِبْهُ وَلاَ عَنْ كَلِمَةٍ وَاحِدَةٍ حَتَّى تَعَجَّبَ الْوَالِي جِدّاً.
\par 15 وَكَانَ الْوَالِي مُعْتَاداً فِي الْعِيدِ أَنْ يُطْلِقَ لِلْجَمْعِ أَسِيراً وَاحِداً مَنْ أَرَادُوهُ.
\par 16 وَكَانَ لَهُمْ حِينَئِذٍ أَسِيرٌ مَشْهُورٌ يُسَمَّى بَارَابَاسَ.
\par 17 فَفِيمَا هُمْ مُجْتَمِعُونَ قَالَ لَهُمْ بِيلاَطُسُ: «مَنْ تُرِيدُونَ أَنْ أُطْلِقَ لَكُمْ؟ بَارَابَاسَ أَمْ يَسُوعَ الَّذِي يُدْعَى الْمَسِيحَ؟»
\par 18 لأَنَّهُ عَلِمَ أَنَّهُمْ أَسْلَمُوهُ حَسَداً.
\par 19 وَإِذْ كَانَ جَالِساً عَلَى كُرْسِيِّ الْوِلاَيَةِ أَرْسَلَتْ إِلَيْهِ امْرَأَتُهُ قَائِلَةً: «إِيَّاكَ وَذَلِكَ الْبَارَّ لأَنِّي تَأَلَّمْتُ الْيَوْمَ كَثِيراً فِي حُلْمٍ مِنْ أَجْلِهِ».
\par 20 وَلَكِنَّ رُؤَسَاءَ الْكَهَنَةِ وَالشُّيُوخَ حَرَّضُوا الْجُمُوعَ عَلَى أَنْ يَطْلُبُوا بَارَابَاسَ وَيُهْلِكُوا يَسُوعَ.
\par 21 فَسَأَلَ الْوَالِي: «مَنْ مِنَ الاِثْنَيْنِ تُرِيدُونَ أَنْ أُطْلِقَ لَكُمْ؟» فَقَالُوا: «بَارَابَاسَ».
\par 22 قَالَ لَهُمْ بِيلاَطُسُ: «فَمَاذَا أَفْعَلُ بِيَسُوعَ الَّذِي يُدْعَى الْمَسِيحَ؟» قَالَ لَهُ الْجَمِيعُ: «لِيُصْلَبْ!»
\par 23 فَقَالَ الْوَالِي: «وَأَيَّ شَرٍّ عَمِلَ؟» فَكَانُوا يَزْدَادُونَ صُرَاخاً قَائِلِينَ: «لِيُصْلَبْ!»
\par 24 فَلَمَّا رَأَى بِيلاَطُسُ أَنَّهُ لاَ يَنْفَعُ شَيْئاً بَلْ بِالْحَرِيِّ يَحْدُثُ شَغَبٌ أَخَذَ مَاءً وَغَسَلَ يَدَيْهِ قُدَّامَ الْجَمْعِ قَائِلاً: «إِنِّي بَرِيءٌ مِنْ دَمِ هَذَا الْبَارِّ. أَبْصِرُوا أَنْتُمْ».
\par 25 فَأَجَابَ جَمِيعُ الشَّعْبِ: «دَمُهُ عَلَيْنَا وَعَلَى أَوْلاَدِنَا».
\par 26 حِينَئِذٍ أَطْلَقَ لَهُمْ بَارَابَاسَ وَأَمَّا يَسُوعُ فَجَلَدَهُ وَأَسْلَمَهُ لِيُصْلَبَ.
\par 27 فَأَخَذَ عَسْكَرُ الْوَالِي يَسُوعَ إِلَى دَارِ الْوِلاَيَةِ وَجَمَعُوا عَلَيْهِ كُلَّ الْكَتِيبَةِ
\par 28 فَعَرَّوْهُ وَأَلْبَسُوهُ رِدَاءً قِرْمِزِيَّاً
\par 29 وَضَفَرُوا إِكْلِيلاً مِنْ شَوْكٍ وَوَضَعُوهُ عَلَى رَأْسِهِ وَقَصَبَةً فِي يَمِينِهِ. وَكَانُوا يَجْثُونَ قُدَّامَهُ وَيَسْتَهْزِئُونَ بِهِ قَائِلِينَ: «السَّلاَمُ يَا مَلِكَ الْيَهُودِ!»
\par 30 وَبَصَقُوا عَلَيْهِ وَأَخَذُوا الْقَصَبَةَ وَضَرَبُوهُ عَلَى رَأْسِهِ.
\par 31 وَبَعْدَ مَا اسْتَهْزَأُوا بِهِ نَزَعُوا عَنْهُ الرِّدَاءَ وَأَلْبَسُوهُ ثِيَابَهُ وَمَضَوْا بِهِ لِلصَّلْبِ.
\par 32 وَفِيمَا هُمْ خَارِجُونَ وَجَدُوا إِنْسَاناً قَيْرَوَانِيّاً اسْمُهُ سِمْعَانُ فَسَخَّرُوهُ لِيَحْمِلَ صَلِيبَهُ.
\par 33 وَلَمَّا أَتَوْا إِلَى مَوْضِعٍ يُقَالُ لَهُ جُلْجُثَةُ وَهُوَ الْمُسَمَّى «مَوْضِعَ الْجُمْجُمَةِ»
\par 34 أَعْطَوْهُ خَلاًّ مَمْزُوجاً بِمَرَارَةٍ لِيَشْرَبَ. وَلَمَّا ذَاقَ لَمْ يُرِدْ أَنْ يَشْرَبَ.
\par 35 وَلَمَّا صَلَبُوهُ اقْتَسَمُوا ثِيَابَهُ مُقْتَرِعِينَ عَلَيْهَا لِكَيْ يَتِمَّ مَا قِيلَ بِالنَّبِيِّ: «اقْتَسَمُوا ثِيَابِي بَيْنَهُمْ وَعَلَى لِبَاسِي أَلْقَوْا قُرْعَةً».
\par 36 ثُمَّ جَلَسُوا يَحْرُسُونَهُ هُنَاكَ.
\par 37 وَجَعَلُوا فَوْقَ رَأْسِهِ عِلَّتَهُ مَكْتُوبَةً: «هَذَا هُوَ يَسُوعُ مَلِكُ الْيَهُودِ».
\par 38 حِينَئِذٍ صُلِبَ مَعَهُ لِصَّانِ وَاحِدٌ عَنِ الْيَمِينِ وَوَاحِدٌ عَنِ الْيَسَارِ.
\par 39 وَكَانَ الْمُجْتَازُونَ يُجَدِّفُونَ عَلَيْهِ وَهُمْ يَهُزُّونَ رُؤُوسَهُمْ
\par 40 قَائِلِينَ: «يَا نَاقِضَ الْهَيْكَلِ وَبَانِيَهُ فِي ثَلاَثَةِ أَيَّامٍ خَلِّصْ نَفْسَكَ! إِنْ كُنْتَ ابْنَ اللَّهِ فَانْزِلْ عَنِ الصَّلِيبِ!».
\par 41 وَكَذَلِكَ رُؤَسَاءُ الْكَهَنَةِ أَيْضاً وَهُمْ يَسْتَهْزِئُونَ مَعَ الْكَتَبَةِ وَالشُّيُوخِ قَالُوا:
\par 42 «خَلَّصَ آخَرِينَ وَأَمَّا نَفْسُهُ فَمَا يَقْدِرُ أَنْ يُخَلِّصَهَا». إِنْ كَانَ هُوَ مَلِكَ إِسْرَائِيلَ فَلْيَنْزِلِ الآنَ عَنِ الصَّلِيبِ فَنُؤْمِنَ بِهِ!
\par 43 قَدِ اتَّكَلَ عَلَى اللَّهِ فَلْيُنْقِذْهُ الآنَ إِنْ أَرَادَهُ! لأَنَّهُ قَالَ: أَنَا ابْنُ اللَّهِ!».
\par 44 وَبِذَلِكَ أَيْضاً كَانَ اللِّصَّانِ اللَّذَانِ صُلِبَا مَعَهُ يُعَيِّرَانِهِ.
\par 45 وَمِنَ السَّاعَةِ السَّادِسَةِ كَانَتْ ظُلْمَةٌ عَلَى كُلِّ الأَرْضِ إِلَى السَّاعَةِ التَّاسِعَةِ.
\par 46 وَنَحْوَ السَّاعَةِ التَّاسِعَةِ صَرَخَ يَسُوعُ بِصَوْتٍ عَظِيمٍ قَائِلاً: «إِيلِي إِيلِي لَمَا شَبَقْتَنِي» (أَيْ: إِلَهِي إِلَهِي لِمَاذَا تَرَكْتَنِي؟)
\par 47 فَقَوْمٌ مِنَ الْوَاقِفِينَ هُنَاكَ لَمَّا سَمِعُوا قَالُوا: «إِنَّهُ يُنَادِي إِيلِيَّا».
\par 48 وَلِلْوَقْتِ رَكَضَ وَاحِدٌ مِنْهُمْ وَأَخَذَ إِسْفِنْجَةً وَمَلَأَهَا خَلاًّ وَجَعَلَهَا عَلَى قَصَبَةٍ وَسَقَاهُ.
\par 49 وَأَمَّا الْبَاقُونَ فَقَالُوا: «اتْرُكْ. لِنَرَى هَلْ يَأْتِي إِيلِيَّا يُخَلِّصُهُ».
\par 50 فَصَرَخَ يَسُوعُ أَيْضاً بِصَوْتٍ عَظِيمٍ وَأَسْلَمَ الرُّوحَ.
\par 51 وَإِذَا حِجَابُ الْهَيْكَلِ قَدِ انْشَقَّ إِلَى اثْنَيْنِ مِنْ فَوْقُ إِلَى أَسْفَلُ. وَالأَرْضُ تَزَلْزَلَتْ وَالصُّخُورُ تَشَقَّقَتْ
\par 52 وَالْقُبُورُ تَفَتَّحَتْ وَقَامَ كَثِيرٌ مِنْ أَجْسَادِ الْقِدِّيسِينَ الرَّاقِدِينَ
\par 53 وَخَرَجُوا مِنَ الْقُبُورِ بَعْدَ قِيَامَتِهِ وَدَخَلُوا الْمَدِينَةَ الْمُقَدَّسَةَ وَظَهَرُوا لِكَثِيرِينَ.
\par 54 وَأَمَّا قَائِدُ الْمِئَةِ وَالَّذِينَ مَعَهُ يَحْرُسُونَ يَسُوعَ فَلَمَّا رَأَوُا الزَّلْزَلَةَ وَمَا كَانَ خَافُوا جِدّاً وَقَالُوا: «حَقّاً كَانَ هَذَا ابْنَ اللَّهِ».
\par 55 وَكَانَتْ هُنَاكَ نِسَاءٌ كَثِيرَاتٌ يَنْظُرْنَ مِنْ بَعِيدٍ وَهُنَّ كُنَّ قَدْ تَبِعْنَ يَسُوعَ مِنَ الْجَلِيلِ يَخْدِمْنَهُ
\par 56 وَبَيْنَهُنَّ مَرْيَمُ الْمَجْدَلِيَّةُ وَمَرْيَمُ أُمُّ يَعْقُوبَ وَيُوسِي وَأُمُّ ابْنَيْ زَبْدِي.
\par 57 وَلَمَّا كَانَ الْمَسَاءُ جَاءَ رَجُلٌ غَنِيٌّ مِنَ الرَّامَةِ اسْمُهُ يُوسُفُ - وَكَانَ هُوَ أَيْضاً تِلْمِيذاً لِيَسُوعَ.
\par 58 فَهَذَا تَقَدَّمَ إِلَى بِيلاَطُسَ وَطَلَبَ جَسَدَ يَسُوعَ. فَأَمَرَ بِيلاَطُسُ حِينَئِذٍ أَنْ يُعْطَى الْجَسَدُ.
\par 59 فَأَخَذَ يُوسُفُ الْجَسَدَ وَلَفَّهُ بِكَتَّانٍ نَقِيٍّ
\par 60 وَوَضَعَهُ فِي قَبْرِهِ الْجَدِيدِ الَّذِي كَانَ قَدْ نَحَتَهُ فِي الصَّخْرَةِ ثُمَّ دَحْرَجَ حَجَراً كَبِيراً عَلَى بَابِ الْقَبْرِ وَمَضَى.
\par 61 وَكَانَتْ هُنَاكَ مَرْيَمُ الْمَجْدَلِيَّةُ وَمَرْيَمُ الأُخْرَى جَالِسَتَيْنِ تُجَاهَ الْقَبْرِ.
\par 62 وَفِي الْغَدِ الَّذِي بَعْدَ الاِسْتِعْدَادِ اجْتَمَعَ رُؤَسَاءُ الْكَهَنَةِ وَالْفَرِّيسِيُّونَ إِلَى بِيلاَطُسَ
\par 63 قَائِلِينَ: «يَا سَيِّدُ قَدْ تَذَكَّرْنَا أَنَّ ذَلِكَ الْمُضِلَّ قَالَ وَهُوَ حَيٌّ: إِنِّي بَعْدَ ثَلاَثَةِ أَيَّامٍ أَقُومُ.
\par 64 فَمُرْ بِضَبْطِ الْقَبْرِ إِلَى الْيَوْمِ الثَّالِثِ لِئَلَّا يَأْتِيَ تَلاَمِيذُهُ لَيْلاً وَيَسْرِقُوهُ وَيَقُولُوا لِلشَّعْبِ إِنَّهُ قَامَ مِنَ الأَمْوَاتِ فَتَكُونَ الضَّلاَلَةُ الأَخِيرَةُ أَشَرَّ مِنَ الأُولَى!»
\par 65 فَقَالَ لَهُمْ بِيلاَطُسُ: «عِنْدَكُمْ حُرَّاسٌ. اذْهَبُوا وَاضْبُطُوهُ كَمَا تَعْلَمُونَ».
\par 66 فَمَضَوْا وَضَبَطُوا الْقَبْرَ بِالْحُرَّاسِ وَخَتَمُوا الْحَجَرَ.

\chapter{28}

\par 1 وَبَعْدَ السَّبْتِ عِنْدَ فَجْرِ أَوَّلِ الأُسْبُوعِ جَاءَتْ مَرْيَمُ الْمَجْدَلِيَّةُ وَمَرْيَمُ الأُخْرَى لِتَنْظُرَا الْقَبْرَ.
\par 2 وَإِذَا زَلْزَلَةٌ عَظِيمَةٌ حَدَثَتْ لأَنَّ مَلاَكَ الرَّبِّ نَزَلَ مِنَ السَّمَاءِ وَجَاءَ وَدَحْرَجَ الْحَجَرَ عَنِ الْبَابِ وَجَلَسَ عَلَيْهِ.
\par 3 وَكَانَ مَنْظَرُهُ كَالْبَرْقِ وَلِبَاسُهُ أَبْيَضَ كَالثَّلْجِ.
\par 4 فَمِنْ خَوْفِهِ ارْتَعَدَ الْحُرَّاسُ وَصَارُوا كَأَمْوَاتٍ.
\par 5 فَقَالَ الْمَلاَكُ لِلْمَرْأَتَيْنِ: «لاَ تَخَافَا أَنْتُمَا فَإِنِّي أَعْلَمُ أَنَّكُمَا تَطْلُبَانِ يَسُوعَ الْمَصْلُوبَ.
\par 6 لَيْسَ هُوَ هَهُنَا لأَنَّهُ قَامَ كَمَا قَالَ. هَلُمَّا انْظُرَا الْمَوْضِعَ الَّذِي كَانَ الرَّبُّ مُضْطَجِعاً فِيهِ.
\par 7 وَاذْهَبَا سَرِيعاً قُولاَ لِتَلاَمِيذِهِ إِنَّهُ قَدْ قَامَ مِنَ الأَمْوَاتِ. هَا هُوَ يَسْبِقُكُمْ إِلَى الْجَلِيلِ. هُنَاكَ تَرَوْنَهُ. هَا أَنَا قَدْ قُلْتُ لَكُمَا».
\par 8 فَخَرَجَتَا سَرِيعاً مِنَ الْقَبْرِ بِخَوْفٍ وَفَرَحٍ عَظِيمٍ رَاكِضَتَيْنِ لِتُخْبِرَا تَلاَمِيذَهُ.
\par 9 وَفِيمَا هُمَا مُنْطَلِقَتَانِ لِتُخْبِرَا تَلاَمِيذَهُ إِذَا يَسُوعُ لاَقَاهُمَا وَقَالَ: «سَلاَمٌ لَكُمَا». فَتَقَدَّمَتَا وَأَمْسَكَتَا بِقَدَمَيْهِ وَسَجَدَتَا لَهُ.
\par 10 فَقَالَ لَهُمَا يَسُوعُ: «لاَ تَخَافَا. اذْهَبَا قُولاَ لِإِخْوَتِي أَنْ يَذْهَبُوا إِلَى الْجَلِيلِ وَهُنَاكَ يَرَوْنَنِي».
\par 11 وَفِيمَا هُمَا ذَاهِبَتَانِ إِذَا قَوْمٌ مِنَ الْحُرَّاسِ جَاءُوا إِلَى الْمَدِينَةِ وَأَخْبَرُوا رُؤَسَاءَ الْكَهَنَةِ بِكُلِّ مَا كَانَ.
\par 12 فَاجْتَمَعُوا مَعَ الشُّيُوخِ وَتَشَاوَرُوا وَأَعْطَوُا الْعَسْكَرَ فِضَّةً كَثِيرَةً
\par 13 قَائِلِينَ: «قُولُوا إِنَّ تَلاَمِيذَهُ أَتَوْا لَيْلاً وَسَرَقُوهُ وَنَحْنُ نِيَامٌ.
\par 14 وَإِذَا سُمِعَ ذَلِكَ عِنْدَ الْوَالِي فَنَحْنُ نَسْتَعْطِفُهُ وَنَجْعَلُكُمْ مُطْمَئِنِّينَ».
\par 15 فَأَخَذُوا الْفِضَّةَ وَفَعَلُوا كَمَا عَلَّمُوهُمْ فَشَاعَ هَذَا الْقَوْلُ عِنْدَ الْيَهُودِ إِلَى هَذَا الْيَوْمِ.
\par 16 وَأَمَّا الأَحَدَ عَشَرَ تِلْمِيذاً فَانْطَلَقُوا إِلَى الْجَلِيلِ إِلَى الْجَبَلِ حَيْثُ أَمَرَهُمْ يَسُوعُ.
\par 17 وَلَمَّا رَأَوْهُ سَجَدُوا لَهُ وَلَكِنَّ بَعْضَهُمْ شَكُّوا.
\par 18 فَتَقَدَّمَ يَسُوعُ وَكَلَّمَهُمْ قَائِلاً: «دُفِعَ إِلَيَّ كُلُّ سُلْطَانٍ فِي السَّمَاءِ وَعَلَى الأَرْضِ
\par 19 فَاذْهَبُوا وَتَلْمِذُوا جَمِيعَ الأُمَمِ وَعَمِّدُوهُمْ بِاسْمِ الآبِ وَالاِبْنِ وَالرُّوحِ الْقُدُسِ.
\par 20 وَعَلِّمُوهُمْ أَنْ يَحْفَظُوا جَمِيعَ مَا أَوْصَيْتُكُمْ بِهِ. وَهَا أَنَا مَعَكُمْ كُلَّ الأَيَّامِ إِلَى انْقِضَاءِ الدَّهْرِ». آمِينَ.



\end{document}