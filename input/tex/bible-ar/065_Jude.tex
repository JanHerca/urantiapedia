\begin{document}

\title{مراثي}


\chapter{1}

\par 1 يَهُوذَا، عَبْدُ يَسُوعَ الْمَسِيحِ، وَأَخُو يَعْقُوبَ، إِلَى الْمَدْعُوِّينَ الْمُقَدَّسِينَ فِي اللهِ الآبِ، وَالْمَحْفُوظِينَ لِيَسُوعَ الْمَسِيحِ.
\par 2 لِتَكْثُرْ لَكُمُ الرَّحْمَةُ وَالسَّلاَمُ وَالْمَحَبَّةُ.
\par 3 أَيُّهَا الأَحِبَّاءُ، إِذْ كُنْتُ أَصْنَعُ كُلَّ الْجَهْدِ لأَكْتُبَ إِلَيْكُمْ عَنِ الْخَلاَصِ الْمُشْتَرَكِ، اضْطُرِرْتُ أَنْ أَكْتُبَ إِلَيْكُمْ وَاعِظاً أَنْ تَجْتَهِدُوا لأَجْلِ الإِيمَانِ الْمُسَلَّمِ مَرَّةً لِلْقِدِّيسِينَ.
\par 4 لأَنَّهُ دَخَلَ خُلْسَةً أُنَاسٌ قَدْ كُتِبُوا مُنْذُ الْقَدِيمِ لِهَذِهِ الدَّيْنُونَةِ، فُجَّارٌ، يُحَوِّلُونَ نِعْمَةَ إِلَهِنَا إِلَى الدَّعَارَةِ، وَيُنْكِرُونَ السَّيِّدَ الْوَحِيدَ: اللهَ وَرَبَّنَا يَسُوعَ الْمَسِيحَ.
\par 5 فَأُرِيدُ أَنْ أُذَكِّرَكُمْ، وَلَوْ عَلِمْتُمْ هَذَا مَرَّةً، أَنَّ الرَّبَّ بَعْدَمَا خَلَّصَ الشَّعْبَ مِنْ أَرْضِ مِصْرَ، أَهْلَكَ أَيْضاً الَّذِينَ لَمْ يُؤْمِنُوا.
\par 6 وَالْمَلاَئِكَةُ الَّذِينَ لَمْ يَحْفَظُوا رِيَاسَتَهُمْ، بَلْ تَرَكُوا مَسْكَنَهُمْ حَفِظَهُمْ إِلَى دَيْنُونَةِ الْيَوْمِ الْعَظِيمِ بِقُيُودٍ أَبَدِيَّةٍ تَحْتَ الظَّلاَمِ.
\par 7 كَمَا أَنَّ سَدُومَ وَعَمُورَةَ وَالْمُدُنَ الَّتِي حَوْلَهُمَا، إِذْ زَنَتْ عَلَى طَرِيقٍ مِثْلِهِمَا وَمَضَتْ وَرَاءَ جَسَدٍ آخَرَ، جُعِلَتْ عِبْرَةً مُكَابِدَةً عِقَابَ نَارٍ أَبَدِيَّةٍ.
\par 8 وَلَكِنْ كَذَلِكَ هَؤُلاَءِ أَيْضاً، الْمُحْتَلِمُونَ، يُنَجِّسُونَ الْجَسَدَ، وَيَتَهَاوَنُونَ بِالسِّيَادَةِ، وَيَفْتَرُونَ عَلَى ذَوِي الأَمْجَادِ.
\par 9 وَأَمَّا مِيخَائِيلُ رَئِيسُ الْمَلاَئِكَةِ، فَلَمَّا خَاصَمَ إِبْلِيسَ مُحَاجّاً عَنْ جَسَدِ مُوسَى، لَمْ يَجْسُرْ أَنْ يُورِدَ حُكْمَ افْتِرَاءٍ، بَلْ قَالَ: «لِيَنْتَهِرْكَ الرَّبُّ».
\par 10 وَلَكِنَّ هَؤُلاَءِ يَفْتَرُونَ عَلَى مَا لاَ يَعْلَمُونَ. وَأَمَّا مَا يَفْهَمُونَهُ بِالطَّبِيعَةِ، كَالْحَيَوَانَاتِ غَيْرِ النَّاطِقَةِ، فَفِي ذَلِكَ يَفْسُدُونَ.
\par 11 وَيْلٌ لَهُمْ لأَنَّهُمْ سَلَكُوا طَرِيقَ قَايِينَ، وَانْصَبُّوا إِلَى ضَلاَلَةِ بَلْعَامَ لأَجْلِ أُجْرَةٍ، وَهَلَكُوا فِي مُشَاجَرَةِ قُورَحَ.
\par 12 هَؤُلاَءِ صُخُورٌ فِي وَلاَئِمِكُمُ الْمَحَبِّيَّةِ، صَانِعِينَ وَلاَئِمَ مَعاً بِلاَ خَوْفٍ، رَاعِينَ أَنْفُسَهُمْ. غُيُومٌ بِلاَ مَاءٍ تَحْمِلُهَا الرِّيَاحُ. أَشْجَارٌ خَرِيفِيَّةٌ بِلاَ ثَمَرٍ مَيِّتَةٌ مُضَاعَفاً، مُقْتَلَعَةٌ.
\par 13 أَمْوَاجُ بَحْرٍ هَائِجَةٌ مُزْبِدَةٌ بِخِزْيِهِمْ. نُجُومٌ تَائِهَةٌ مَحْفُوظٌ لَهَا قَتَامُ الظَّلاَمِ إِلَى الأَبَدِ.
\par 14 وَتَنَبَّأَ عَنْ هَؤُلاَءِ أَيْضاً أَخْنُوخُ السَّابِعُ مِنْ آدَمَ قَائِلاً: «هُوَذَا قَدْ جَاءَ الرَّبُّ فِي رَبَوَاتِ قِدِّيسِيهِ
\par 15 لِيَصْنَعَ دَيْنُونَةً عَلَى الْجَمِيعِ، وَيُعَاقِبَ جَمِيعَ فُجَّارِهِمْ عَلَى جَمِيعِ أَعْمَالِ فُجُورِهِمُِ الَّتِي فَجَرُوا بِهَا، وَعَلَى جَمِيعِ الْكَلِمَاتِ الصَّعْبَةِ الَّتِي تَكَلَّمَ بِهَا عَلَيْهِ خُطَاةٌ فُجَّارٌ».
\par 16 هَؤُلاَءِ هُمْ مُدَمْدِمُونَ مُتَشَكُّونَ، سَالِكُونَ بِحَسَبِ شَهَوَاتِهِمْ، وَفَمُهُمْ يَتَكَلَّمُ بِعَظَائِمَ، يُحَابُونَ بِالْوُجُوهِ مِنْ أَجْلِ الْمَنْفَعَةِ.
\par 17 وَأَمَّا أَنْتُمْ أَيُّهَا الأَحِبَّاءُ فَاذْكُرُوا الأَقْوَالَ الَّتِي قَالَهَا سَابِقاً رُسُلُ رَبِّنَا يَسُوعَ الْمَسِيحِ.
\par 18 فَإِنَّهُمْ قَالُوا لَكُمْ إِنَّهُ فِي الزَّمَانِ الأَخِيرِ سَيَكُونُ قَوْمٌ مُسْتَهْزِئُونَ، سَالِكِينَ بِحَسَبِ شَهَوَاتِ فُجُورِهِمْ.
\par 19 هَؤُلاَءِ هُمُ الْمُعْتَزِلُونَ بِأَنْفُسِهِمْ، نَفْسَانِيُّونَ لاَ رُوحَ لَهُمْ.
\par 20 وَأَمَّا أَنْتُمْ أَيُّهَا الأَحِبَّاءُ فَابْنُوا أَنْفُسَكُمْ عَلَى إِيمَانِكُمُ الأَقْدَسِ، مُصَلِّينَ فِي الرُّوحِ الْقُدُسِ،
\par 21 وَاحْفَظُوا أَنْفُسَكُمْ فِي مَحَبَّةِ اللهِ، مُنْتَظِرِينَ رَحْمَةَ رَبِّنَا يَسُوعَ الْمَسِيحِ لِلْحَيَاةِ الأَبَدِيَّةِ.
\par 22 وَارْحَمُوا الْبَعْضَ مُمَيِّزِينَ،
\par 23 وَخَلِّصُوا الْبَعْضَ بِالْخَوْفِ مُخْتَطِفِينَ مِنَ النَّارِ، مُبْغِضِينَ حَتَّى الثَّوْبَ الْمُدَنَّسَ مِنَ الْجَسَدِ.
\par 24 وَالْقَادِرُ أَنْ يَحْفَظَكُمْ غَيْرَ عَاثِرِينَ، وَيُوقِفَكُمْ أَمَامَ مَجْدِهِ بِلاَ عَيْبٍ فِي الاِبْتِهَاجِ،
\par 25 اَلإِلَهُ الْحَكِيمُ الْوَحِيدُ مُخَلِّصُنَا، لَهُ الْمَجْدُ وَالْعَظَمَةُ وَالْقُدْرَةُ وَالسُّلْطَانُ، الآنَ وَإِلَى كُلِّ الدُّهُورِ. آمِينَ.

\end{document}