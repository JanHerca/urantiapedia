\begin{document}

\title{عزرا}


\chapter{1}

\par 1 وَفِي السَّنَةِ الأُولَى لِكُورَشَ مَلِكِ فَارِسَ عِنْدَ تَمَامِ كَلاَمِ الرَّبِّ بِفَمِ إِرْمِيَا نَبَّهَ الرَّبُّ رُوحَ كُورَشَ مَلِكِ فَارِسَ فَأَطْلَقَ نِدَاءً فِي كُلِّ مَمْلَكَتِهِ وَبِالْكِتَابَةِ أَيْضاً قَائِلاً:
\par 2 [هَكَذَا قَالَ كُورَشُ مَلِكُ فَارِسَ: جَمِيعُ مَمَالِكِ الأَرْضِ دَفَعَهَا لِي الرَّبُّ إِلَهُ السَّمَاءِ وَهُوَ أَوْصَانِي أَنْ أَبْنِيَ لَهُ بَيْتاً فِي أُورُشَلِيمَ الَّتِي فِي يَهُوذَا.
\par 3 مَنْ مِنْكُمْ مِنْ كُلِّ شَعْبِهِ لِيَكُنْ إِلَهُهُ مَعَهُ وَيَصْعَدْ إِلَى أُورُشَلِيمَ الَّتِي فِي يَهُوذَا فَيَبْنِيَ بَيْتَ الرَّبِّ إِلَهِ إِسْرَائِيلَ. هُوَ الإِلَهُ الَّذِي فِي أُورُشَلِيمَ.
\par 4 وَكُلُّ مَنْ بَقِيَ فِي أَحَدِ الأَمَاكِنِ حَيْثُ هُوَ مُتَغَرِّبٌ فَلْيُنْجِدْهُ أَهْلُ مَكَانِهِ بِفِضَّةٍ وَبِذَهَبٍ وَبِأَمْتِعَةٍ وَبِبَهَائِمَ مَعَ التَّبَرُّعِ لِبَيْتِ الرَّبِّ الَّذِي فِي أُورُشَلِيمَ].
\par 5 فَقَامَ رُؤُوسُ آبَاءِ يَهُوذَا وَبِنْيَامِينَ وَالْكَهَنَةُ وَاللاَّوِيُّونَ مَعَ كُلِّ مَنْ نَبَّهَ اللَّهُ رُوحَهُ لِيَصْعَدُوا لِيَبْنُوا بَيْتَ الرَّبِّ الَّذِي فِي أُورُشَلِيمَ.
\par 6 وَكُلُّ الَّذِينَ حَوْلَهُمْ أَعَانُوهُمْ بِآنِيَةِ فِضَّةٍ وَبِذَهَبٍ وَبِأَمْتِعَةٍ وَبِبَهَائِمَ وَبِتُحَفٍ فَضْلاً عَنْ كُلِّ مَا تُبُرِّعَ بِهِ.
\par 7 وَالْمَلِكُ كُورَشُ أَخْرَجَ آنِيَةَ بَيْتِ الرَّبِّ الَّتِي أَخْرَجَهَا نَبُوخَذْنَصَّرُ مِنْ أُورُشَلِيمَ وَجَعَلَهَا فِي بَيْتِ آلِهَتِهِ.
\par 8 أَخْرَجَهَا كُورَشُ مَلِكُ فَارِسَ عَنْ يَدِ مِثْرَدَاثَ الْخَازِنِ وَعَدَّهَا لِشِيشْبَصَّرَ رَئِيسِ يَهُوذَا.
\par 9 وَهَذَا عَدَدُهَا: ثَلاَثُونَ طَسْتاً مِنْ ذَهَبٍ وَأَلْفُ طَسْتٍ مِنْ فِضَّةٍ وَتِسْعَةٌ وَعِشْرُونَ سِكِّيناً
\par 10 وَثَلاَثُونَ قَدَحاً مِنْ ذَهَبٍ وَأَقْدَاحُ فِضَّةٍ مِنَ الرُّتْبَةِ الثَّانِيَةِ أَرْبَعُ مِئَةٍ وَعَشَرَةٌ وَأَلْفٌ مِنْ آنِيَةٍ أُخْرَى.
\par 11 جَمِيعُ الآنِيَةِ مِنَ الذَّهَبِ وَالْفِضَّةِ خَمْسَةُ آلاَفٍ وَأَرْبَعُ مِئَةٍ. الْكُلُّ أَصْعَدَهُ شِيشْبَصَّرُ عِنْدَ إِصْعَادِ السَّبْيِ مِنْ بَابِلَ إِلَى أُورُشَلِيمَ.

\chapter{2}

\par 1 وَهَؤُلاَءِ هُمْ بَنُو الْكُورَةِ الصَّاعِدُونَ مِنْ سَبْيِ الْمَسْبِيِّينَ الَّذِينَ سَبَاهُمْ نَبُوخَذْنَصَّرُ مَلِكُ بَابِلَ إِلَى بَابِلَ وَرَجَعُوا إِلَى أُورُشَلِيمَ وَيَهُوذَا كُلُّ وَاحِدٍ إِلَى مَدِينَتِهِ.
\par 2 الَّذِينَ جَاءُوا مَعَ زَرُبَّابَِلَ: يَشُوعُ نَحَمْيَا سَرَايَا رَعْلاَيَا مُرْدَخَايُ بِلْشَانُ مِسْفَارُ بَغْوَايُ رَحُومُ بَعْنَةَ. عَدَدُ رِجَالِ شَعْبِ إِسْرَائِيلَ.
\par 3 بَنُو فَرْعُوشَ أَلْفَانِ وَمِئَةٌ وَاثْنَانِ وَسَبْعُونَ.
\par 4 بَنُو شَفَطْيَا ثَلاَثُ مِئَةٍ وَاثْنَانِ وَسَبْعُونَ.
\par 5 بَنُو آرَحَ سَبْعُ مِئَةٍ وَخَمْسَةٌ وَسَبْعُونَ.
\par 6 بَنُو فَحَثَ مُوآبَ مِنْ بَنِي يَشُوعَ وَيُوآبَ أَلْفَانِ وَثَمَانُ مِئَةٍ وَاثْنَا عَشَرَ.
\par 7 بَنُو عِيلاَمَ أَلْفٌ وَمِئَتَانِ وَأَرْبَعَةٌ وَخَمْسُونَ.
\par 8 بَنُو زَتُّو تِسْعُ مِئَةٍ وَخَمْسَةٌ وَأَرْبَعُونَ.
\par 9 بَنُو زَكَّايَ سَبْعُ مِئَةٍ وَسِتُّونَ.
\par 10 بَنُو بَانِي سِتُّ مِئَةٍ وَاثْنَانِ وَأَرْبَعُونَ.
\par 11 بَنُو بَابَايَ سِتُّ مِئَةٍ وَثَلاَثَةٌ وَعِشْرُونَ.
\par 12 بَنُو عَرْجَدَ أَلْفٌ وَمِئَتَانِ وَاثْنَانِ وَعِشْرُونَ.
\par 13 بَنُو أَدُونِيقَامَ سِتُّ مِئَةٍ وَسِتَّةٌ وَسِتُّونَ.
\par 14 بَنُو بَغْوَايَ أَلْفَانِ وَسِتَّةٌ وَخَمْسُونَ.
\par 15 بَنُو عَادِينَ أَرْبَعُ مِئَةٍ وَأَرْبَعَةٌ وَخَمْسُونَ.
\par 16 بَنُو آطِيرَ مِنْ يَحَزَقِيَّا ثَمَانِيَةٌ وَتِسْعُونَ.
\par 17 بَنُو بِيصَايَ ثَلاَثُ مِئَةٍ وَثَلاَثَةٌ وَعِشْرُونَ.
\par 18 بَنُو يُورَةَ مِئَةٌ وَاثْنَا عَشَرَ.
\par 19 بَنُو حَشُومَ مِئَتَانِ وَثَلاَثَةٌ وَعِشْرُونَ.
\par 20 بَنُو جِبَّارَ خَمْسَةٌ وَتِسْعُونَ.
\par 21 بَنُو بَيْتِ لَحْمٍ مِئَةٌ وَثَلاَثَةٌ وَعِشْرُونَ.
\par 22 رِجَالُ نَطُوفَةَ سِتَّةٌ وَخَمْسُونَ.
\par 23 رِجَالُ عَنَاثُوثَ مِئَةٌ وَثَمَانِيَةٌ وَعِشْرُونَ.
\par 24 بَنُو عَزْمُوتَ اثْنَانِ وَأَرْبَعُونَ.
\par 25 بَنُو قَرْيَةِ عَارِيمَ كَفِيرَةَ وَبَئِيرُوتَ سَبْعُ مِئَةٍ وَثَلاَثَةٌ وَأَرْبَعُونَ.
\par 26 بَنُو الرَّامَةِ وَجَبَعَ سِتُّ مِئَةٍ وَوَاحِدٌ وَعِشْرُونَ.
\par 27 رِجَالُ مِخْمَاسَ مِئَةٌ وَاثْنَانِ وَعِشْرُونَ.
\par 28 رِجَالُ بَيْتِ إِيلَ وَعَايَ مِئَتَانِ وَثَلاَثَةٌ وَعِشْرُونَ.
\par 29 بَنُو نَبُو اثْنَانِ وَخَمْسُونَ.
\par 30 بَنُو مَغْبِيشَ مِئَةٌ وَسِتَّةٌ وَخَمْسُونَ.
\par 31 بَنُو عِيلاَمَ الآخَرِ أَلْفٌ وَمِئَتَانِ وَأَرْبَعَةٌ وَخَمْسُونَ.
\par 32 بَنُو حَارِيمَ ثَلاَثُ مِئَةٍ وَعِشْرُونَ.
\par 33 بَنُو لُودَ بَنُو حَادِيدَ وَأُونُو سَبْعُ مِئَةٍ وَخَمْسَةٌ وَعِشْرُونَ.
\par 34 بَنُو أَرِيحَا ثَلاَثُ مِئَةٍ وَخَمْسَةٌ وَأَرْبَعُونَ.
\par 35 بَنُو سَنَاءَةَ ثَلاَثَةُ آلاَفٍ وَسِتُّ مِئَةٍ وَثَلاَثُونَ.
\par 36 أَمَّا الْكَهَنَةُ فَبَنُو يَدْعِيَا مِنْ بَيْتِ يَشُوعَ تِسْعُ مِئَةٍ وَثَلاَثَةٌ وَسَبْعُونَ.
\par 37 بَنُو إِمِّيرَ أَلْفٌ وَاثْنَانِ وَخَمْسُونَ.
\par 38 بَنُو فَشْحُورَ أَلْفٌ وَمِئَتَانِ وَسَبْعَةٌ وَأَرْبَعُونَ.
\par 39 بَنُو حَارِيمَ أَلْفٌ وَسَبْعَةَ عَشَرَ.
\par 40 أَمَّا اللاَّوِيُّونَ فَبَنُو يَشُوعَ وَقَدْمِيئِيلَ مِنْ بَنِي هُودُويَا أَرْبَعَةٌ وَسَبْعُونَ.
\par 41 الْمُغَنُّونَ بَنُو آسَافَ مِئَةٌ وَثَمَانِيَةٌ وَعِشْرُونَ.
\par 42 بَنُو الْبَوَّابِينَ بَنُو شَلُّومَ بَنُو آطِيرَ بَنُو طَلْمُونَ بَنُو عَقُّوبَ بَنُو حَطِيطَا بَنُو شُوبَايَ الْجَمِيعُ مِئَةٌ وَتِسْعَةٌ وَثَلاَثُونَ.
\par 43 النَّثِينِيمُ بَنُو صِيحَا بَنُو حَسُوفَا بَنُو طَبَاعُوتَ
\par 44 بَنُو قِيرُوسَ بَنُو سِيعَهَا بَنُو فَادُونَ
\par 45 بَنُو لَبَانَةَ بَنُو حَجَابَةَ بَنُو عَقُّوبَ
\par 46 بَنُو حَاجَابَ بَنُو شَمُلاَيَ بَنُو حَانَانَ.
\par 47 بَنُو جَدِّيلَ بَنُو حَجَرَ بَنُو رَآيَا
\par 48 بَنُو رَصِينَ بَنُو نَقُودَا بَنُو جَزَّامَ
\par 49 بَنُو عُزَّا بَنُو فَاسِيحَ بَنُو بِيسَايَ
\par 50 بَنُو أَسْنَةَ بَنُو مَعُونِيمَ بَنُو نَفُوسِيمَ
\par 51 بَنُو بَقْبُوقَ بَنُو حَقُوفَا بَنُو حَرْحُورَ
\par 52 بَنُو بَصْلُوتَ بَنُو مَحِيدَا بَنُو حَرْشَا
\par 53 بَنُو بَرْقُوسَ بَنُو سِيسَرَا بَنُو ثَامَحَ
\par 54 بَنُو نَصِيحَ بَنُو حَطِيفَا.
\par 55 بَنُو عَبِيدِ سُلَيْمَانَ بَنُو سَوْطَايَ بَنُو هَسُّوفَرَثَ بَنُو فَرُودَا
\par 56 بَنُو يَعْلَةَ بَنُو دَرْقُونَ بَنُو جَدِّيلَ
\par 57 بَنُو شَفَطْيَا بَنُو حَطِّيلَ بَنُو فُوخَرَةِ الظِّبَاءِ بَنُو آمِي.
\par 58 جَمِيعُ النَّثِينِيمِ وَبَنِي عَبِيدِ سُلَيْمَانَ ثَلاَثُ مِئَةٍ وَاثْنَانِ وَتِسْعُونَ.
\par 59 وَهَؤُلاَءِ هُمُ الَّذِينَ صَعِدُوا مِنْ تَلِّ مِلْحٍ وَتَلِّ حَرْشَا كَرُوبُ أَدَّانُ إِمِّيرُ. وَلَمْ يَسْتَطِيعُوا أَنْ يُبَيِّنُوا بُيُوتَ آبَائِهِمْ وَنَسْلَهُمْ هَلْ هُمْ مِنْ إِسْرَائِيلَ.
\par 60 بَنُو دَلاَيَا بَنُو طُوبِيَّا بَنُو نَقُودَا سِتُّ مِئَةٍ وَاثْنَانِ وَخَمْسُونَ.
\par 61 وَمِنْ بَنِي الْكَهَنَةِ بَنُو حَبَايَا بَنُو هَقُّوصَ بَنُو بَرْزِلاَّيَ الَّذِي أَخْذَ امْرَأَةً مِنْ بَنَاتِ بَرْزِلاَّيَ الْجِلْعَادِيِّ وَتَسَمَّى بِاسْمِهِمْ.
\par 62 هَؤُلاَءِ فَتَّشُوا عَلَى كِتَابَةِ أَنْسَابِهِمْ فَلَمْ تُوجَدْ فَرُذِلُوا مِنَ الْكَهَنُوتِ.
\par 63 وَقَالَ لَهُمُ التِّرْشَاثَا أَنْ لاَ يَأْكُلُوا مِنْ قُدْسِ الأَقْدَاسِ حَتَّى يَقُومَ كَاهِنٌ لِلأُورِيمِ وَالتُّمِّيمِ.
\par 64 كُلُّ الْجُمْهُورِ مَعاً اثْنَانِ وَأَرْبَعُونَ أَلْفاً وَثَلاَثُ مِئَةٍ وَسِتُّونَ
\par 65 فَضْلاً عَنْ عَبِيدِهِمْ وَإِمَائِهِمْ فَهَؤُلاَءِ كَانُوا سَبْعَةَ آلاَفٍ وَثَلاَثَ مِئَةٍ وَسَبْعَةً وَثَلاَثِينَ وَلَهُمْ مِنَ الْمُغَنِّينَ وَالْمُغَنِّيَاتِ مِئَتَانِ.
\par 66 خَيْلُهُمْ سَبْعُ مِئَةٍ وَسِتَّةٌ وَثَلاَثُونَ. بِغَالُهُمْ مِئَتَانِ وَخَمْسَةٌ وَأَرْبَعُونَ
\par 67 جِمَالُهُمْ أَرْبَعُ مِئَةٍ وَخَمْسَةٌ وَثَلاَثُونَ. حَمِيرُهُمْ سِتَّةُ آلاَفٍ وَسَبْعُ مِئَةٍ وَعِشْرُونَ.
\par 68 وَالْبَعْضُ مِنْ رُؤُوسِ الآبَاءِ عِنْدَ مَجِيئِهِمْ إِلَى بَيْتِ الرَّبِّ الَّذِي فِي أُورُشَلِيمَ تَبَرَّعُوا لِبَيْتِ الرَّبِّ لإِقَامَتِهِ فِي مَكَانِهِ.
\par 69 أَعْطُوا حَسَبَ طَاقَتِهِمْ لِخِزَانَةِ الْعَمَلِ وَاحِداً وَسِتِّينَ أَلْفَ دِرْهَمٍ مِنَ الذَّهَبِ وَخَمْسَةَ آلاَفِ مَناً مِنَ الْفِضَّةِ وَمِئَةَ قَمِيصٍ لِلْكَهَنَةِ.
\par 70 فَأَقَامَ الْكَهَنَةُ وَاللاَّوِيُّونَ وَبَعْضُ الشَّعْبِ وَالْمُغَنُّونَ وَالْبَوَّابُونَ وَالنَّثِينِيمُ فِي مُدُنِهِمْ وَكُلُّ إِسْرَائِيلَ فِي مُدُنِهِمْ.

\chapter{3}

\par 1 وَلَمَّا اسْتُهِلَّ الشَّهْرُ السَّابِعُ وَبَنُو إِسْرَائِيلَ فِي مُدُنِهِمُِ اجْتَمَعَ الشَّعْبُ كَرَجُلٍ وَاحِدٍ إِلَى أُورُشَلِيمَ.
\par 2 وَقَامَ يَشُوعُ بْنُ يُوصَادَاقَ وَإِخْوَتُهُ الْكَهَنَةُ وَزَرُبَّابَِلُ بْنُ شَأَلْتِئِيلَ وَإِخْوَتُهُ وَبَنُوا مَذْبَحَ إِلَهِ إِسْرَائِيلَ لِيُصْعِدُوا عَلَيْهِ مُحْرَقَاتٍ كَمَا هُوَ مَكْتُوبٌ فِي شَرِيعَةِ مُوسَى رَجُلِ اللَّهِ
\par 3 وَأَقَامُوا الْمَذْبَحَ فِي مَكَانِهِ. لأَنَّهُ كَانَ عَلَيْهِمْ رُعْبٌ مِنْ شُعُوبِ الأَرَاضِي وَأَصْعَدُوا عَلَيْهِ مُحْرَقَاتٍ لِلرَّبِّ مُحْرَقَاتِ الصَّبَاحِ وَالْمَسَاءِ.
\par 4 وَحَفِظُوا عِيدَ الْمَظَالِّ كَمَا هُوَ مَكْتُوبٌ وَمُحْرَقَةَ يَوْمٍ فَيَوْمٍ بِالْعَدَدِ كَالْمَرْسُومِ أَمْرَ الْيَوْمِ بِيَوْمِهِ.
\par 5 وَبَعْدَ ذَلِكَ الْمُحْرَقَةُ الدَّائِمَةُ وَلِلأَهِلَّةِ وَلِجَمِيعِ مَوَاسِمِ الرّبِّ الْمُقَدَّسَةِ وَلِكُلِّ مَن تبرَّعَ بِمُتَبَرَّعٍ لِلرَّبِّ.
\par 6 ابْتَدَأُوا مِنَ الْيَوْمِ الأَوَّلِ مِنَ الشَّهْرِ السَّابِعِ يُصْعِدُونَ مُحْرَقَاتٍ لِلرَّبِّ وَهَيْكَلُ الرَّبِّ لَمْ يَكُنْ قَدْ تَأَسَّسَ.
\par 7 وَأَعْطُوا فِضَّةً لِلنَّحَّاتِينَ وَالنَّجَّارِينَ وَمَأْكَلاً وَمَشْرَباً وَزَيْتاً لِلصَّيْدُونِيِّينَ وَالصُّورِيِّينَ لِيَأْتُوا بِخَشَبِ أَرْزٍ مِنْ لُبْنَانَ إِلَى بَحْرِ يَافَا حَسَبَ إِذْنِ كُورَشَ مَلِكِ فَارِسَ لَهُمْ.
\par 8 وَفِي السَّنَةِ الثَّانِيَةِ مِنْ مَجِيئِهِمْ إِلَى بَيْتِ اللَّهِ إِلَى أُورُشَلِيمَ فِي الشَّهْرِ الثَّانِي شَرَعَ زَرُبَّابَِلُ بْنُ شَأَلْتِئِيلَ وَيَشُوعُ بْنُ يُوصَادَاقَ وَبَقِيَّةُ إِخْوَتِهِمِ الْكَهَنَةِ وَاللاَّوِيِّينَ وَجَمِيعُ الْقَادِمِينَ مِنَ السَّبْيِ إِلَى أُورُشَلِيمَ وَأَقَامُوا اللاَّوِيِّينَ مِنِ ابْنِ عِشْرِينَ سَنَةً فَمَا فَوْقُ لِلإِشْرَافِ عَلَى عَمَلِ بَيْتِ الرَّبِّ.
\par 9 وَوَقَفَ يَشُوعُ مَعَ بَنِيهِ وَإِخْوَتِهِ قَدْمِيئِيلَ وَبَنِيهِ بَنِي يَهُوذَا مَعاً لِلإِشْرَافِ عَامِلِي الشُّغْلِ فِي بَيْتِ اللَّهِ وَبَنِي حِينَادَادَ مَعَ بَنِيهِمْ وَإِخْوَتِهِمِ اللاَّوِيِّينَ.
\par 10 وَلَمَّا أَسَّسَ الْبَانُونَ هَيْكَلَ الرَّبِّ أَقَامُوا الْكَهَنَةَ بِمَلاَبِسِهِمْ بِأَبْوَاقٍ وَاللاَّوِيِّينَ بَنِي آسَافَ بِالصُّنُوجِ لِتَسْبِيحِ الرَّبِّ عَلَى تَرْتِيبِ دَاوُدَ مَلِكِ إِسْرَائِيلَ.
\par 11 وَغَنُّوا بِالتَّسْبِيحِ وَالْحَمْدِ لِلرَّبِّ لأَنَّهُ صَالِحٌ لأَنَّ إِلَى الأَبَدِ رَحْمَتَهُ عَلَى إِسْرَائِيلَ. وَكُلُّ الشَّعْبِ هَتَفُوا هُتَافاً عَظِيماً بِالتَّسْبِيحِ لِلرَّبِّ لأَجْلِ تَأْسِيسِ بَيْتِ الرَّبِّ.
\par 12 وَكَثِيرُونَ مِنَ الْكَهَنَةِ وَاللاَّوِيِّينَ وَرُؤُوسِ الآبَاءِ الشُّيُوخِ الَّذِينَ رَأَوُا الْبَيْتَ الأَوَّلَ بَكُوا بِصَوْتٍ عَظِيمٍ عِنْدَ تَأْسِيسِ هَذَا الْبَيْتِ أَمَامَ أَعْيُنِهِمْ. وَكَثِيرُونَ كَانُوا يَرْفَعُونَ أَصْوَاتَهُمْ بِالْهُتَافِ بِفَرَحٍ.
\par 13 وَلَمْ يَكُنِ الشَّعْبُ يُمَيِّزُ هُتَافَ الْفَرَحِ مِنْ صَوْتِ بُكَاءِ الشَّعْبِ لأَنَّ الشَّعْبَ كَانَ يَهْتِفُ هُتَافاً عَظِيماً حَتَّى أَنَّ الصَّوْتَ سُمِعَ مِنْ بُعْدٍ.

\chapter{4}

\par 1 وَلَمَّا سَمِعَ أَعْدَاءُ يَهُوذَا وَبِنْيَامِينَ أَنَّ بَنِي السَّبْيِ يَبْنُونَ هَيْكَلاً لِلرَّبِّ إِلَهِ إِسْرَائِيلَ
\par 2 تَقَدَّمُوا إِلَى زَرُبَّابَِلَ وَرُؤُوسِ الآبَاءِ وَقَالُوا لَهُمْ: [نَبْنِي مَعَكُمْ لأَنَّنَا نَظِيرَكُمْ نَطْلُبُ إِلَهَكُمْ وَلَهُ قَدْ ذَبَحْنَا مِنْ أَيَّامِ أَسَرْحَدُّونَ مَلِكِ أَشُّورَ الَّذِي أَصْعَدَنَا إِلَى هُنَا].
\par 3 فَقَالَ لَهُمْ زَرُبَّابِلُ وَيَشُوعُ وَبَقِيَّةُ رُؤُوسِ آبَاءِ إِسْرَائِيلَ: [لَيْسَ لَكُمْ وَلَنَا أَنْ نَبْنِيَ بَيْتاً لإِلَهِنَا وَلَكِنَّنَا نَحْنُ وَحْدَنَا نَبْنِي لِلرَّبِّ إِلَهِ إِسْرَائِيلَ كَمَا أَمَرَنَا الْمَلِكُ كُورَشُ مَلِكُ فَارِسَ].
\par 4 وَكَانَ شَعْبُ الأَرْضِ يُرْخُونَ أَيْدِيَ شَعْبِ يَهُوذَا وَيُذْعِرُونَهُمْ عَنِ الْبِنَاءِ.
\par 5 وَاسْتَأْجَرُوا ضِدَّهُمْ مُشِيرِينَ لِيُبْطِلُوا مَشُورَتَهُمْ كُلَّ أَيَّامِ كُورَشَ مَلِكِ فَارِسَ وَحَتَّى مُلْكِ دَارِيُوسَ مَلِكِ فَارِسَ.
\par 6 وَفِي مُلْكِ أَحْشَوِيرُوشَ فِي ابْتِدَاءِ مُلْكِهِ كَتَبُوا شَكْوَى عَلَى سُكَّانِ يَهُوذَا وَأُورُشَلِيمَ.
\par 7 وَفِي أَيَّامِ أَرْتَحْشَسْتَا كَتَبَ بِشْلاَمُ وَمِثْرَدَاثُ وَطَبْئِيلُ وَسَائِرُ رُفَقَائِهِمْ إِلَى أَرْتَحْشَسْتَا مَلِكِ فَارِسَ. وَكِتَابَةُ الرِّسَالَةِ مَكْتُوبَةٌ بِالأَرَامِيَّةِ وَمُتَرْجَمَةٌ بِالأَرَامِيَّةِ.
\par 8 رَحُومُ صَاحِبُ الْقَضَاءِ وَشِمْشَايُ الْكَاتِبُ كَتَبَا رِسَالَةً ضِدَّ أُورُشَلِيمَ إِلَى أَرْتَحْشَسْتَا الْمَلِكِ هَكَذَا:
\par 9 [كَتَبَ حِينَئِذٍ رَحُومُ صَاحِبُ الْقَضَاءِ وَشِمْشَايُ الْكَاتِبُ وَسَائِرُ رُفَقَائِهِمَا الدِّينِيِّينَ وَالأَفَرَسْتِكِيِّينَ وَالطَّرْفِلِيِّينَ وَالأَفْرَسِيِّينَ وَالأَرَكْوِيِّينَ وَالْبَابَِلِيِّينَ وَالشُّوشَنِيِّينَ وَالدَّهْوِيِّينَ وَالْعِيلاَمِيِّينَ
\par 10 وَسَائِرِ الأُمَمِ الَّذِينَ سَبَاهُمْ أُسْنَفَّرُ الْعَظِيمُ الشَّرِيفُ وَأَسْكَنَهُمْ مُدُنَ السَّامِرَةِ وَسَائِرِ الَّذِينَ فِي عَبْرِ النَّهْرِ وَإِلَى آخِرِهِ].
\par 11 هَذِهِ صُورَةُ الرِّسَالَةِ الَّتِي أَرْسَلُوهَا إِلَيْهِ إِلَى أَرْتَحْشَسْتَا الْمَلِكِْ: [عَبِيدُكَ الْقَوْمُ الَّذِينَ فِي عَبْرِ النَّهْرِ إِلَى آخِرِهِ.
\par 12 لِيُعْلَمِ الْمَلِكُ أَنَّ الْيَهُودَ الَّذِينَ صَعِدُوا مِنْ عِنْدِكَ إِلَيْنَا قَدْ أَتُوا إِلَى أُورُشَلِيمَ وَيَبْنُونَ الْمَدِينَةَ الْعَاصِيَةَ الرَّدِيئَةَ وَقَدْ أَكْمَلُوا أَسْوَارَهَا وَرَمَّمُوا أُسُسَهَا.
\par 13 لِيَكُنِ الآنَ مَعْلُوماً لَدَى الْمَلِكِ أَنَّهُ إِذَا بُنِيَتْ هَذِهِ الْمَدِينَةُ وَأُكْمِلَتْ أَسْوَارُهَا لاَ يُؤَدُّونَ جِزْيَةً وَلاَ خَرَاجاً وَلاَ خِفَارَةً فَأَخِيراً تَضُرُّ الْمُلُوكَ.
\par 14 وَالآنَ بِمَا إِنَّنَا نَأْكُلُ مِلْحَ دَارِ الْمَلِكِ وَلاَ يَلِيقُ بِنَا أَنْ نَرَى ضَرَرَ الْمَلِكِ لِذَلِكَ أَرْسَلْنَا فَأَعْلَمْنَا الْمَلِكَ
\par 15 لِيُفَتَّشَ فِي سِفْرِ أَخْبَارِ آبَائِكَ فَتَجِدَ فِي سِفْرِ الأَخْبَارِ وَتَعْلَمَ أَنَّ هَذِهِ الْمَدِينَةَ مَدِينَةٌ عَاصِيَةٌ وَمُضِرَّةٌ لِلْمُلُوكِ وَالْبِلاَدِ وَقَدْ عَمِلُوا عِصْيَاناً فِي وَسَطِهَا مُنْذُ الأَيَّامِ الْقَدِيمَةِ لِذَلِكَ أُخْرِبَتْ هَذِهِ الْمَدِينَةُ.
\par 16 وَنَحْنُ نُعْلِمُ الْمَلِكَ أَنَّهُ إِذَا بُنِيَتْ هَذِهِ الْمَدِينَةُ وَأُكْمِلَتْ أَسْوَارُهَا لاَ يَكُونُ لَكَ عِنْدَ ذَلِكَ نَصِيبٌ فِي عَبْرِ النَّهْرِ].
\par 17 فَأَرْسَلَ الْمَلِكُ جَوَاباً: [إِلَى رَحُومَ صَاحِبِ الْقَضَاءِ وَشَمْشَايَ الْكَاتِبِ وَسَائِرِ رُفَقَائِهِمَا السَّاكِنِينَ فِي السَّامِرَةِ وَبَاقِي الَّذِينَ فِي عَبْرِ النَّهْرِ. سَلاَمٌ إِلَى آخِرِهِ.
\par 18 الرِّسَالَةُ الَّتِي أَرْسَلْتُمُوهَا إِلَيْنَا قَدْ قُرِئَتْ بِوُضُوحٍ أَمَامِي.
\par 19 وَقَدْ خَرَجَ مِنْ عِنْدِي أَمْرٌ فَفَتَّشُوا وَوُجِدَ أَنَّ هَذِهِ الْمَدِينَةَ مُنْذُ الأَيَّامِ الْقَدِيمَةِ تَقُومُ عَلَى الْمُلُوكِ وَقَدْ جَرَى فِيهَا تَمَرُّدٌ وَعِصْيَانٌ.
\par 20 وَقَدْ كَانَ مُلُوكٌ مُقْتَدِرُونَ عَلَى أُورُشَلِيمَ وَتَسَلَّطُوا عَلَى جَمِيعِ عَبْرِ النَّهْرِ وَقَدْ أُعْطُوا جِزْيَةً وَخَرَاجاً وَخِفَارَةً.
\par 21 فَالآنَ أَخْرِجُوا أَمْراً بِتَوْقِيفِ أُولَئِكَ الرِّجَالِ فَلاَ تُبْنَى هَذِهِ الْمَدِينَةُ حَتَّى يَصْدُرَ مِنِّي أَمْرٌ.
\par 22 فَاحْذَرُوا مِنْ أَنْ تَتَهَاوَنُوا عَنْ عَمَلِ ذَلِكَ. لِمَاذَا يَكْثُرُ الضَّرَرُ لِخَسَارَةِ الْمُلُوكِ؟].
\par 23 حِينَئِذٍ لَمَّا قُرِئَتْ رِسَالَةُ أَرْتَحْشَسْتَا الْمَلِكِ أَمَامَ رَحُومَ وَشِمْشَايَ الْكَاتِبِ وَرُفَقَائِهِمَا ذَهَبُوا بِسُرْعَةٍ إِلَى أُورُشَلِيمَ إِلَى الْيَهُودِ وَأَوْقَفُوهُمْ بِذِرَاعٍ وَقُوَّةٍ.
\par 24 حِينَئِذٍ تَوَقَّفَ عَمَلُ بَيْتِ اللَّهِ الَّذِي فِي أُورُشَلِيمَ وَكَانَ مُتَوَقِّفاً إِلَى السَّنَةِ الثَّانِيَةِ مِنْ مُلْكِ دَارِيُوسَ مَلِكِ فَارِسَ.

\chapter{5}

\par 1 فَتَنَبَّأَ النَّبِيَّانِ حَجَّيِ النَّبِيُّ وَزَكَرِيَّا بْنُ عِدُّوَ لِلْيَهُودِ الَّذِينَ فِي يَهُوذَا وَأُورُشَلِيمَ بِاسْمِ إِلَهِ إِسْرَائِيلَ عَلَيْهِمْ.
\par 2 حِينَئِذٍ قَامَ زَرُبَّابَِلُ بْنُ شَأَلْتِئِيلَ وَيَشُوعُ بْنُ يُوصَادَاقَ وَشَرَعَا بِبُنْيَانِ بَيْتِ اللَّهِ الَّذِي فِي أُورُشَلِيمَ وَمَعَهُمَا أَنْبِيَاءُ اللَّهِ يُسَاعِدُونَهُمَا.
\par 3 فِي ذَلِكَ الزَّمَانِ جَاءَ إِلَيْهِمْ تَتْنَايُ وَالِي عَبْرِ النَّهْرِ وَشَتَرْبُوزْنَايُ وَرُفَقَاؤُهُمَا وَقَالُوا لَهُمْ: [مَنْ أَمَرَكُمْ أَنْ تَبْنُوا هَذَا الْبَيْتَ وَتُكَمِّلُوا هَذَا السُّورَ؟].
\par 4 حِينَئِذٍ أَخْبَرْنَاهُمْ بِأَسْمَاءِ الرِّجَالِ الَّذِينَ يَبْنُونَ هَذَا الْبِنَاءَ.
\par 5 وَكَانَتْ عَلَى شُيُوخِ الْيَهُودِ عَيْنُ إِلَهِهِمْ فَلَمْ يُوقِفُوهُمْ حَتَّى وَصَلَ الأَمْرُ إِلَى دَارِيُوسَ وَحِينَئِذٍ جَاوَبُوا بِرِسَالَةٍ عَنْ هَذَا.
\par 6 صُورَةُ الرِّسَالَةِ الَّتِي أَرْسَلَهَا تَتْنَايُ وَالِي عَبْرِ النَّهْرِ وَشَتَرْبُوزْنَايُ وَرُفَقَاؤُهُمَا الأَفَرْسَكِيِّينَ الَّذِينَ فِي عَبْرِ النَّهْرِ إِلَى دَارِيُوسَ الْمَلِكِ:
\par 7 [لِدَارِيُوسَ الْمَلِكِ كُلُّ سَلاَمٍ.
\par 8 لِيَكُنْ مَعْلُوماً لَدَى الْمَلِكِ أَنَّنَا ذَهَبْنَا إِلَى بِلاَدِ يَهُوذَا إِلَى بَيْتِ الإِلَهِ الْعَظِيمِ وَإِذَا بِهِ يُبْنَى بِحِجَارَةٍ عَظِيمَةٍ وَيُوضَعُ خَشَبٌ فِي الْحِيطَانِ. وَهَذَا الْعَمَلُ يُعْمَلُ بِسُرْعَةٍ وَيَنْجَحُ فِي أَيْدِيهِمْ.
\par 9 حِينَئِذٍ سَأَلْنَا أُولَئِكَ الشُّيُوخَ: مَنْ أَمَرَكُمْ بِبِنَاءِ هَذَا الْبَيْتِ وَتَكْمِيلِ هَذِهِ الأَسْوَارِ؟
\par 10 وَسَأَلْنَاهُمْ أَيْضاً عَنْ أَسْمَائِهِمْ لِنُعْلِمَكَ وَكَتَبْنَا أَسْمَاءَ الرِّجَالِ رُؤُوسِهِمْ.
\par 11 وَبِمِثْلِ هَذَا الْجَوَابِ جَاوَبُوا: نَحْنُ عَبِيدُ إِلَهِ السَّمَاءِ وَالأَرْضِ وَنَبْنِي هَذَا الْبَيْتَ الَّذِي بُنِيَ قَبْلَ هَذِهِ السِّنِينَ الْكَثِيرَةِ وَقَدْ بَنَاهُ مَلِكٌ عَظِيمٌ لإِسْرَائِيلَ وَأَكْمَلَهُ.
\par 12 وَلَكِنْ بَعْدَ أَنْ أَسْخَطَ آبَاؤُنَا إِلَهَ السَّمَاءِ دَفَعَهُمْ لِيَدِ نَبُوخَذْنَصَّرَ مَلِكِ بَابِلَ الْكِلْدَانِيِّ الَّذِي هَدَمَ هَذَا الْبَيْتَ وَسَبَى الشَّعْبَ إِلَى بَابِلَ.
\par 13 عَلَى أَنَّهُ فِي السَّنَةِ الأُولَى لِكُورَشَ مَلِكَ بَابِلَ أَصْدَرَ كُورَشُ الْمَلِكُ أَمْراً بِبِنَاءِ بَيْتِ اللَّهِ هَذَا.
\par 14 حَتَّى إِنَّ آنِيَةَ بَيْتِ اللَّهِ هَذَا الَّتِي مِنْ ذَهَبٍ وَفِضَّةٍ الَّتِي أَخْرَجَهَا نَبُوخَذْنَصَّرُ مِنَ الْهَيْكَلِ الَّذِي فِي أُورُشَلِيمَ وَأَتَى بِهَا إِلَى الْهَيْكَلِ الَّذِي فِي بَابِلَ أَخْرَجَهَا كُورَشُ الْمَلِكُ مِنَ الْهَيْكَلِ الَّذِي فِي بَابِلَ وَأُعْطِيَتْ لِوَاحِدٍ اسْمُهُ شِيشْبَصَّرُ الَّذِي جَعَلَهُ وَالِياً.
\par 15 وَقَالَ لَهُ: خُذْ هَذِهِ الآنِيَةَ وَاذْهَبْ وَاحْمِلْهَا إِلَى الْهَيْكَلِ الَّذِي فِي أُورُشَلِيمَ وَلْيُبْنَ بَيْتُ اللَّهِ فِي مَكَانِهِ.
\par 16 حِينَئِذٍ جَاءَ شِيشْبَصَّرُ هَذَا وَوَضَعَ أَسَاسَ بَيْتِ اللَّهِ الَّذِي فِي أُورُشَلِيمَ. وَمِنْ ذَلِكَ الْوَقْتِ إِلَى الآنَ يُبْنَى وَلَمْ يُكْمَلْ.
\par 17 وَالآنَ إِذَا حَسُنَ عِنْدَ الْمَلِكِ فَلْيُفَتَّشْ فِي بَيْتِ خَزَائِنِ الْمَلِكِ الَّذِي هُوَ هُنَاكَ فِي بَابِلَ هَلْ كَانَ قَدْ صَدَرَ أَمْرٌ مِنْ كُورَشَ الْمَلِكِ بِبِنَاءِ بَيْتِ اللَّهِ هَذَا فِي أُورُشَلِيمَ وَلْيُرْسِلِ الْمَلِكُ إِلَيْنَا مُرَادَهُ فِي ذَلِكَ].

\chapter{6}

\par 1 حِينَئِذٍ أَمَرَ دَارِيُوسُ الْمَلِكُ فَفَتَّشُوا فِي بَيْتِ الأَسْفَارِ حَيْثُ كَانَتِ الْخَزَائِنُ مَوْضُوعَةً فِي بَابِلَ
\par 2 فَوُجِدَ فِي أَحْمَثَا فِي الْقَصْرِ الَّذِي فِي بِلاَدِ مَادِي دَرْجٌ مَكْتُوبٌ فِيهِ هَكَذَا: [تِذْكَارٌ.
\par 3 فِي السَّنَةِ الأُولَى لِكُورَشَ الْمَلِكِ أَمَرَ كُورَشُ الْمَلِكُ مِنْ جِهَةِ بَيْتِ اللَّهِ فِي أُورُشَلِيمَ: لِيُبْنَ الْبَيْتُ الْمَكَانُ الَّذِي يَذْبَحُونَ فِيهِ ذَبَائِحَ وَلْتُوضَعْ أُسُسُهُ ارْتِفَاعُهُ سِتُّونَ ذِرَاعاً وَعَرْضُهُ سِتُّونَ ذِرَاعاً.
\par 4 بِثَلاَثَةِ صُفُوفٍ مِنْ حِجَارَةٍ عَظِيمَةٍ وَصَفٍّ مِنْ خَشَبٍ جَدِيدٍ. وَلْتُعْطَ النَّفَقَةُ مِنْ بَيْتِ الْمَلِكِ.
\par 5 وَأَيْضاً آنِيَةُ بَيْتِ اللَّهِ الَّتِي مِنْ ذَهَبٍ وَفِضَّةٍ الَّتِي أَخْرَجَهَا نَبُوخَذْنَصَّرُ مِنَ الْهَيْكَلِ الَّذِي فِي أُورُشَلِيمَ وَأَتَى بِهَا إِلَى بَابَِلَ فَلْتُرَدَّ وَتُرْجَعْ إِلَى الْهَيْكَلِ الَّذِي فِي أُورُشَلِيمَ إِلَى مَكَانِهَا وَتُوضَعْ فِي بَيْتِ اللَّهِ].
\par 6 [وَالآنَ يَا تَتْنَايُ وَالِي عَبْرِ النَّهْرِ وَشَتَرْبُوزْنَايُ وَرُفَقَاءَكُمَا الأَفَرْسَكِيِّينَ الَّذِينَ فِي عَبْرِ النَّهْرِ ابْتَعِدُوا مِنْ هُنَاكَ.
\par 7 اتْرُكُوا عَمَلَ بَيْتِ اللَّهِ هَذَا. أَمَّا وَالِي الْيَهُودِ وَشُيُوخُ الْيَهُودِ فَلْيَبْنُوا بَيْتَ اللَّهِ هَذَا فِي مَكَانِهِ.
\par 8 وَقَدْ صَدَرَ مِنِّي أَمْرٌ بِمَا تَعْمَلُونَ مَعَ شُيُوخِ الْيَهُودِ هَؤُلاَءِ فِي بِنَاءِ بَيْتِ اللَّهِ هَذَا. فَمِنْ مَالِ الْمَلِكِ مِنْ جِزْيَةِ عَبْرِ النَّهْرِ تُعْطَ النَّفَقَةُ عَاجِلاً لِهَؤُلاَءِ الرِّجَالِ حَتَّى لاَ يَبْطُلُوا.
\par 9 وَمَا يَحْتَاجُونَ إِلَيْهِ مِنَ الثِّيرَانِ وَالْكِبَاشِ وَالْخِرَافِ مُحْرَقَةً لإِلَهِ السَّمَاءِ وَحِنْطَةٍ وَمِلْحٍ وَخَمْرٍ وَزَيْتٍ حَسَبَ قَوْلِ الْكَهَنَةِ الَّذِينَ فِي أُورُشَلِيمَ لِتُعْطَ لَهُمْ يَوْماً فَيَوْماً حَتَّى لاَ يَهْدَأُوا
\par 10 عَنْ تَقْرِيبِ رَوَائِحِ سُرُورٍ لإِلَهِ السَّمَاءِ وَالصَّلاَةِ لأَجْلِ حَيَاةِ الْمَلِكِ وَبَنِيهِ.
\par 11 وَقَدْ صَدَرَ مِنِّي أَمْرٌ أَنَّ كُلَّ إِنْسَانٍ يُغَيِّرُ هَذَا الْكَلاَمَ تُسْحَبُ خَشَبَةٌ مِنْ بَيْتِهِ وَيُعَلَّقُ مَصْلُوباً عَلَيْهَا وَيُجْعَلُ بَيْتُهُ مَزْبَلَةً مِنْ أَجْلِ هَذَا.
\par 12 وَاللَّهُ الَّذِي أَسْكَنَ اسْمَهُ هُنَاكَ يُهْلِكُ كُلَّ مَلِكٍ وَشَعْبٍ يَمُدُّ يَدَهُ لِتَغْيِيرِ أَوْ لِهَدْمِ بَيْتِ اللَّهِ هَذَا الَّذِي فِي أُورُشَلِيمَ. أَنَا دَارِيُوسُ قَدْ أَمَرْتُ فَلْيُفْعَلْ عَاجِلاً].
\par 13 حِينَئِذٍ تَتْنَايُ وَالِي عَبْرِ النَّهْرِ وَشَتَرْبُوزْنَايُ وَرُفَقَاؤُهُمَا عَمِلُوا عَاجِلاً حَسْبَمَا أَرْسَلَ دَارِيُوسُ الْمَلِكُ.
\par 14 وَكَانَ شُيُوخُ الْيَهُودِ يَبْنُونَ وَيَنْجَحُونَ حَسَبَ نُبُوَّةِ حَجَّيِ النَّبِيِّ وَزَكَرِيَّا بْنِ عِدُّو. فَبَنُوا وَأَكْمَلُوا حَسَبَ أَمْرِ إِلَهِ إِسْرَائِيلَ وَأَمْرِ كُورَشَ وَدَارِيُوسَ وَأَرْتَحْشَسْتَا مَلِكِ فَارِسَ.
\par 15 وَكَمُلَ هَذَا الْبَيْتُ فِي الْيَوْمِ الثَّالِثِ مِنْ شَهْرِ أَذَارَ فِي السَّنَةِ السَّادِسَةِ مِنْ مُلْكِ دَارِيُوسَ الْمَلِكِ.
\par 16 وَبَنُو إِسْرَائِيلَ الْكَهَنَةُ وَاللاَّوِيُّونَ وَبَاقِي بَنِي السَّبْيِ دَشَّنُوا بَيْتَ اللَّهِ هَذَا بِفَرَحٍ.
\par 17 وَقَرَّبُوا تَدْشِيناً لِبَيْتِ اللَّهِ هَذَا مِئَةَ ثَوْرٍ وَمِئَتَيْ كَبْشٍ وَأَرْبَعَ مِئَةِ خَرُوفٍ وَاثْنَيْ عَشَرَ تَيْسَ مِعْزًى ذَبِيحَةَ خَطِيَّةٍ عَنْ جَمِيعِ إِسْرَائِيلَ حَسَبَ عَدَدِ أَسْبَاطِ إِسْرَائِيلَ.
\par 18 وَأَقَامُوا الْكَهَنَةَ فِي فِرَقِهِمْ وَاللاَّوِيِّينَ فِي أَقْسَامِهِمْ عَلَى خِدْمَةِ اللَّهِ الَّتِي فِي أُورُشَلِيمَ كَمَا هُوَ مَكْتُوبٌ فِي سِفْرِ مُوسَى.
\par 19 وَعَمِلَ بَنُو السَّبْيِ الْفِصْحَ فِي الرَّابِعَ عَشَرَ مِنَ الشَّهْرِ الأَوَّلِ.
\par 20 لأَنَّ الْكَهَنَةَ وَاللاَّوِيِّينَ تَطَهَّرُوا جَمِيعاً كَانُوا كُلُّهُمْ طَاهِرِينَ وَذَبَحُوا الْفِصْحَ لِجَمِيعِ بَنِي السَّبْيِ وَلإِخْوَتِهِمِ الْكَهَنَةِ وَلأَنْفُسِهِمْ
\par 21 وَأَكَلَهُ بَنُو إِسْرَائِيلَ الرَّاجِعُونَ مِنَ السَّبْيِ مَعَ جَمِيعِ الَّذِينَ انْفَصَلُوا إِلَيْهِمْ مِنْ رَجَاسَةِ أُمَمِ الأَرْضِ. لِيَطْلُبُوا الرَّبَّ إِلَهَ إِسْرَائِيلَ
\par 22 وَعَمِلُوا عِيدَ الْفَطِيرِ سَبْعَةَ أَيَّامٍ بِفَرَحٍ. لأَنَّ الرَّبَّ فَرَّحَهُمْ وَحَوَّلَ قَلْبَ مَلِكِ أَشُّورَ نَحْوَهُمْ لِتَقْوِيَةِ أَيْدِيهِمْ فِي عَمَلِ بَيْتِ اللَّهِ إِلَهِ إِسْرَائِيلَ.

\chapter{7}

\par 1 وَبَعْدَ هَذِهِ الأُمُورِ فِي مُلْكِ أَرْتَحْشَسْتَا مَلِكِ فَارِسَ عَزْرَا بْنُ سَِرَايَا بْنِ عَزَرْيَا بْنِ حِلْقِيَّا
\par 2 بْنِ شَلُّومَ بْنِ صَادُوقَ بْنِ أَخِيطُوبَ
\par 3 بْنِ أَمَرْيَا بْنِ عَزَرْيَا بْنِ مَرَايُوثَ
\par 4 بْنِ زَرَحْيَا بْنِ عُزِّي بْنِ بُقِّي
\par 5 بَنِ أَبِيشُوعَ بْنِ فِينَحَاسَ بْنِ أَلِعَازَارَ بْنِ هَارُونَ الْكَاهِنِ الرَّأْسِ
\par 6 عَزْرَا هَذَا صَعِدَ مِنْ بَابِلَ وَهُوَ كَاتِبٌ مَاهِرٌ فِي شَرِيعَةِ مُوسَى الَّتِي أَعْطَاهَا الرَّبُّ إِلَهُ إِسْرَائِيلَ. وَأَعْطَاهُ الْمَلِكُ حَسَبَ يَدِ الرَّبِّ إِلَهِهِ عَلَيْهِ كُلَّ سُؤْلِهِ.
\par 7 وَصَعِدَ مَعَهُ مِنْ بَنِي إِسْرَائِيلَ وَالْكَهَنَةِ وَاللاَّوِيِّينَ وَالْمُغَنِّينَ وَالْبَوَّابِينَ وَالنَّثِينِيمِ إِلَى أُورُشَلِيمَ فِي السَّنَةِ السَّابِعَةِ لأَرْتَحْشَسْتَا الْمَلِكِ.
\par 8 وَجَاءَ إِلَى أُورُشَلِيمَ فِي الشَّهْرِ الْخَامِسِ فِي السَّنَةِ السَّابِعَةِ لِلْمَلِكِ.
\par 9 لأَنَّهُ فِي الشَّهْرِ الأَوَّلِ ابْتَدَأَ يَصْعَدُ مِنْ بَابِلَ وَفِي أَوَّلِ الشَّهْرِ الْخَامِسِ جَاءَ إِلَى أُورُشَلِيمَ حَسَبَ يَدِ اللَّهِ الصَّالِحَةِ عَلَيْهِ.
\par 10 لأَنَّ عَزْرَا هَيَّأَ قَلْبَهُ لِطَلَبِ شَرِيعَةِ الرَّبِّ وَالْعَمَلِ بِهَا وَلِيُعَلِّمَ إِسْرَائِيلَ فَرِيضَةً وَقَضَاءً.
\par 11 وَهَذِهِ صُورَةُ الرِّسَالَةِ الَّتِي أَعْطَاهَا الْمَلِكُ أَرْتَحْشَسْتَا لِعَزْرَا الْكَاهِنِ الْكَاتِبِ كَاتِبِ كَلاَمِ وَصَايَا الرَّبِّ وَفَرَائِضِهِ عَلَى إِسْرَائِيلَ:
\par 12 [مِنْ أَرْتَحْشَسْتَا مَلِكِ الْمُلُوكِ إِلَى عَزْرَا الْكَاهِنِ كَاتِبِ شَرِيعَةِ إِلَهِ السَّمَاءِ الْكَامِلِ إِلَى آخِرِهِ.
\par 13 قَدْ صَدَرَ مِنِّي أَمْرٌ أَنَّ كُلَّ مَنْ أَرَادَ فِي مُلْكِي مِنْ شَعْبِ إِسْرَائِيلَ وَكَهَنَتِهِ وَاللاَّوِيِّينَ أَنْ يَرْجِعَ إِلَى أُورُشَلِيمَ مَعَكَ فَلْيَرْجِعْ.
\par 14 مِنْ أَجْلِ أَنَّكَ مُرْسَلٌ مِنْ قِبَلِ الْمَلِكِ وَمُشِيرِيهِ السَّبْعَةِ لأَجْلِ السُّؤَالِ عَنْ يَهُوذَا وَأُورُشَلِيمَ حَسَبَ شَرِيعَةِ إِلَهِكَ الَّتِي بِيَدِكَ
\par 15 وَلِحَمْلِ فِضَّةٍ وَذَهَبٍ تَبَرَّعَ بِهِ الْمَلِكُ وَمُشِيرُوهُ لإِلَهِ إِسْرَائِيلَ الَّذِي فِي أُورُشَلِيمَ مَسْكَنِهُ.
\par 16 وَكُلُّ الْفِضَّةِ وَالذَّهَبِ الَّتِي تَجِدُ فِي كُلِّ بِلاَدِ بَابَِلَ مَعَ تَبَرُّعَاتِ الشَّعْبِ وَالْكَهَنَةِ الْمُتَبَرِّعِينَ لِبَيْتِ إِلَهِهِمِ الَّذِي فِي أُورُشَلِيمَ
\par 17 لِتَشْتَرِيَ عَاجِلاً بِهَذِهِ الْفِضَّةِ ثِيرَاناً وَكِبَاشاً وَخِرَافاً وَتَقْدِمَاتِهَا وَسَكَائِبَهَا وَتُقَرِّبَهَا عَلَى الْمَذْبَحِ الَّذِي فِي بَيْتِ إِلَهِكُمُ الَّذِي فِي أُورُشَلِيمَ.
\par 18 وَمَهْمَا حَسُنَ عِنْدَكَ وَعِنْدَ إِخْوَتِكَ أَنْ تَعْمَلُوهُ بِبَاقِي الْفِضَّةِ وَالذَّهَبِ فَحَسَبَ إِرَادَةِ إِلَهِكُمْ تَعْمَلُونَهُ.
\par 19 وَالآنِيَةُ الَّتِي تُعْطَى لَكَ لِخِدْمَةِ بَيْتِ إِلَهِكَ فَسَلِّمْهَا أَمَامَ إِلَهِ أُورُشَلِيمَ.
\par 20 وَبَاقِي احْتِيَاجِ بَيْتِ إِلَهِكَ الَّذِي يَتَّفِقُ لَكَ أَنْ تُعْطِيَهُ فَأَعْطِهِ مِنْ بَيْتِ خَزَائِنِ الْمَلِكِ.
\par 21 وَمِنِّي أَنَا أَرْتَحْشَسْتَا الْمَلِكِ صَدَرَ أَمْرٌ إِلَى كُلِّ الْخَزَنَةِ الَّذِينَ فِي عَبْرِ النَّهْرِ أَنَّ كُلَّ مَا يَطْلُبُهُ مِنْكُمْ عَزْرَا الْكَاهِنُ كَاتِبُ شَرِيعَةِ إِلَهِ السَّمَاءِ فَلْيُعْمَلْ بِسُرْعَةٍ
\par 22 إِلَى مِئَةِ وَزْنَةٍ مِنَ الْفِضَّةِ وَمِئَةِ كُرٍّ مِنَ الْحِنْطَةِ وَمِئَةِ بَثٍّ مِنَ الْخَمْرِ وَمِئَةِ بَثٍّ مِنَ الزَّيْتِ وَالْمِلْحِ مِنْ دُونِ تَقْيِيدٍ.
\par 23 كُلُّ مَا أَمَرَ بِهِ إِلَهُ السَّمَاءِ فَلْيُعْمَلْ بِاجْتِهَادٍ لِبَيْتِ إِلَهِ السَّمَاءِ لأَنَّهُ لِمَاذَا يَكُونُ غَضَبٌ عَلَى مُلْكِ الْمَلِكِ وَبَنِيهِ؟
\par 24 وَنُعْلِمُكُمْ أَنَّ جَمِيعَ الْكَهَنَةِ وَاللاَّوِيِّينَ وَالْمُغَنِّينَ وَالْبَوَّابِينَ وَالنَّثِينِيمِ وَخُدَّامِ بَيْتِ اللَّهِ هَذَا لاَ يُؤْذَنُ أَنْ يُلْقَى عَلَيْهِمْ جِزْيَةٌ أَوْ خَرَاجٌ أَوْ خِفَارَةٌ.
\par 25 أَمَّا أَنْتَ يَا عَزْرَا فَحَسَبَ حِكْمَةِ إِلَهِكَ الَّتِي بِيَدِكَ ضَعْ حُكَّاماً وَقُضَاةً يَقْضُونَ لِجَمِيعِ الشَّعْبِ الَّذِي فِي عَبْرِ النَّهْرِ مِنْ جَمِيعِ مَنْ يَعْرِفُ شَرَائِعَ إِلَهِكَ. وَالَّذِينَ لاَ يَعْرِفُونَ فَعَلِّمُوهُمْ.
\par 26 وَكُلُّ مَنْ لاَ يَعْمَلُ شَرِيعَةَ إِلَهِكَ وَشَرِيعَةَ الْمَلِكِ فَلْيُقْضَ عَلَيْهِ عَاجِلاً إِمَّا بِالْمَوْتِ أَوْ بِالنَّفْيِ أَوْ بِغَرَامَةِ الْمَالِ أَوْ بِالْحَبْسِ].
\par 27 مُبَارَكٌ الرَّبُّ إِلَهُ آبَائِنَا الَّذِي جَعَلَ مِثْلَ هَذَا فِي قَلْبِ الْمَلِكِ لأَجْلِ تَزْيِينِ بَيْتِ الرَّبِّ الَّذِي فِي أُورُشَلِيمَ.
\par 28 وَقَدْ بَسَطَ عَلَيَّ رَحْمَةً أَمَامَ الْمَلِكِ وَمُشِيرِيهِ وَأَمَامَ جَمِيعِ رُؤَسَاءِ الْمَلِكِ الْمُقْتَدِرِينَ. وَأَمَّا أَنَا فَقَدْ تَشَدَّدْتُ حَسَبَ يَدِ الرَّبِّ إِلَهِي عَلَيَّ وَجَمَعْتُ مِنْ إِسْرَائِيلَ رُؤَسَاءَ لِيَصْعَدُوا مَعِي.

\chapter{8}

\par 1 وَهَؤُلاَءِ هُمْ رُؤُوسُ آبَائِهِمْ وَنِسْبَةُ الَّذِينَ صَعِدُوا مَعِي فِي مُلْكِ أَرْتَحْشَسْتَا الْمَلِكِ مِنْ بَابَِلَ.
\par 2 مِنْ بَنِي فِينَحَاسَ جِرْشُومُ. مِنْ بَنِي إِيثَامَارَ دَانِيَآلُ. مِنْ بَنِي دَاوُدَ حَطُّوشُ.
\par 3 مِنْ بَنِي شَكَنْيَا مِنْ بَنِي فَرْعُوشَ زَكَرِيَّا وَانْتَسَبَ مَعَهُ مِنَ الذُّكُورِ مِئَةٌ وَخَمْسُونَ.
\par 4 مِنْ بَنِي فَحَثَ مُوآبَ أَلِيهُوعِينَايُ بْنُ زَرَحْيَا وَمَعَهُ مِئَتَانِ مِنَ الذُّكُورِ.
\par 5 مِنْ بَنِي شَكَنْيَا ابْنُ يَحْزِيئِيلَ وَمَعَهُ ثَلاَثُ مِئَةٍ مِنَ الذُّكُورِ.
\par 6 مِنْ بَنِي عَادِينَ عَابِدُ بْنُ يُونَاثَانَ وَمَعَهُ خَمْسُونَ مِنَ الذُّكُورِ.
\par 7 مِنْ بَنِي عِيلاَمَ يَشَعْيَا ابْنُ عَثَلْيَا وَمَعَهُ سَبْعُونَ مِنَ الذُّكُورِ.
\par 8 وَمِنْ بَنِي شَفَطْيَا زَبَدْيَا بْنُ مِيخَائِيلَ وَمَعَهُ ثَمَانُونَ مِنَ الذُّكُورِ.
\par 9 مِنْ بَنِي يُوآبَ عُوبَدْيَا ابْنُ يَحِيئِيلَ وَمَعَهُ مِئَتَانِ وَثَمَانِيَةَ عَشَرَ مِنَ الذُّكُورِ.
\par 10 وَمِنْ بَنِي شَلُومِيثَ ابْنُ يُوشَفْيَا وَمَعَهُ مِئَةٌ وَسِتُّونَ مِنَ الذُّكُورِ.
\par 11 وَمِنْ بَنِي بَابَايَ زَكَرِيَّا بْنُ بَابَايَ وَمَعَهُ ثَمَانِيَةٌ وَعِشْرُونَ مِنَ الذُّكُورِ.
\par 12 وَمِنْ بَنِي عَزْجَدَ يُوحَانَانُ بْنُ هِقَّاطَانَ وَمَعَهُ مِئَةٌ وَعَشَْرَةٌ مِنَ الذُّكُورِ.
\par 13 وَمِنْ بَنِي أَدُونِيقَامَ الآخَرِينَ وَهَذِهِ أَسْمَاؤُهُمْ أَلِيفَلَطُ وَيَعِيئِيلُ وَشَمَعْيَا وَمَعَهُمْ سِتُّونَ مِنَ الذُّكُورِ.
\par 14 وَمِنْ بَنِي بَغْوَايَ عُوتَايُ وَزَبُّودُ وَمَعَهُمَا سَبْعُونَ مِنَ الذُّكُورِ.
\par 15 فَجَمَعْتُهُمْ إِلَى النَّهْرِ الْجَارِي إِلَى أَهْوَا وَنَزَلْنَا هُنَاكَ ثَلاَثَةَ أَيَّامٍ. وَتَأَمَّلْتُ الشَّعْبَ وَالْكَهَنَةَ وَلَكِنَّنِي لَمْ أَجِدْ أَحَداً مِنَ اللاَّوِيِّينَ هُنَاكَ.
\par 16 فَأَرْسَلْتُ إِلَى أَلِيعَزَرَ وَأَرِيئِيلَ وَشَمَعْيَا وَأَلْنَاثَانَ وَيَارِيبَ وَأَلْنَاثَانَ وَنَاثَانَ وَزَكَرِيَّا وَمَشُلاَّمَ الرُّؤُوسِ وَإِلَى يُويَارِيبَ وَأَلْنَاثَانَ الْفَهِيمَيْنِ
\par 17 وَأَرْسَلْتُهُمْ إِلَى إِدُّو الرَّأْسِ فِي الْمَكَانِ الْمُسَمَّى كَسِفْيَا وَجَعَلْتُ فِي أَفْوَاهِهِمْ كَلاَماً يُكَلِّمُونَ بِهِ إِدُّوَ وَإِخْوَتَهُ النَّثِينِيمَ فِي الْمَكَانِ كَسِفْيَا لِيَأْتُوا إِلَيْنَا بِخُدَّامٍ لِبَيْتِ إِلَهِنَا.
\par 18 فَأَتُوا إِلَيْنَا حَسَبَ يَدِ اللَّهِ الصَّالِحَةِ عَلَيْنَا بِرَجُلٍ فَطِنٍ مِنْ بَنِي مَحْلِي بْنِ لاَوِي بْنِ إِسْرَائِيلَ وَشَرَبْيَا وَبَنِيهِ وَإِخْوَتِهِ ثَمَانِيَةَ عَشَرَ
\par 19 وَحَشَبْيَا وَمَعَْهُ يَشَعْيَا مِنْ بَنِي مَرَارِي وَإِخْوَتُهُ وَبَنُوهُمْ عِشْرُونَ.
\par 20 وَمِنَ النَّثِينِيمِ الَّذِينَ جَعَلَهُمْ دَاوُدُ مَعَ الرُّؤَسَاءِ لِخِدْمَةِ اللاَّوِيِّينَ مِنَ النَّثِينِيمِ مِئَتَيْنِ وَعِشْرِينَ. الْجَمِيعُ تَعَيَّنُوا بِأَسْمَائِهِمْ.
\par 21 وَنَادَيْتُ هُنَاكَ بِصَوْمٍ عَلَى نَهْرِ أَهْوَا لِنَتَذَلَّلَ أَمَامَ إِلَهِنَا لِنَطْلُبَ مِنْهُ طَرِيقاً مُسْتَقِيمَةً لَنَا وَلأَطْفَالِنَا وَلِكُلِّ مَالِنَا.
\par 22 لأَنِّي خَجِلْتُ مِنْ أَنْ أَطْلُبَ مِنَ الْمَلِكِ جَيْشاً وَفُرْسَاناً لِيُنْجِدُونَا عَلَى الْعَدُوِّ فِي الطَّرِيقِ لأَنَّنَا قُلْنَا لِلْمَلِكَ: [إِنَّ يَدَ إِلَهِنَا عَلَى كُلِّ طَالِبِيهِ لِلْخَيْرِ وَصَوْلَتَهُ وَغَضَبَهُ عَلَى كُلِّ مَنْ يَتْرُكُهُ].
\par 23 فَصُمْنَا وَطَلَبْنَا ذَلِكَ مِنْ إِلَهِنَا فَاسْتَجَابَ لَنَا.
\par 24 وَأَفْرَزْتُ مِنْ رُؤَسَاءِ الْكَهَنَةِ اثْنَيْ عَشَرَ: شَرَبْيَا وَحَشَبْيَا وَمَعَْهُمَا مِنْ إِخْوَتِهِمَا عَشَرَةٌ.
\par 25 وَوَزَنْتُ لَهُمُ الْفِضَّةَ وَالذَّهَبَ وَالآنِيَةَ تَقْدِمَةَ بَيْتِ إِلَهِنَا الَّتِي قَدَّمَهَا الْمَلِكُ وَمُشِيرُوهُ وَرُؤَسَاؤُهُ وَجَمِيعُ إِسْرَائِيلَ الْمَوْجُودِينَ
\par 26 وَزَنْتُ لِيَدِهِمْ سِتَّ مِئَةٍ وَخَمْسِينَ وَزْنَةً مِنَ الْفِضَّةِ وَمِئَةَ وَزْنَةٍ مِنْ آنِيَةِ الْفِضَّةِ وَمِئَةَ وَزْنَةٍ مِنَ الذَّهَبِ
\par 27 وَعِشْرِينَ قَدَحاً مِنَ الذَّهَبِ أَلْفَ دِرْهَمٍ وَآنِيَةً مِنْ نُحَاسٍ صَقِيلٍ جَيِّدٍ ثَمِينٍ كَالذَّهَبِ.
\par 28 وَقُلْتُ لَهُمْ: [أَنْتُمْ مُقَدَّسُونَ لِلرَّبِّ وَالآنِيَةُ مُقَدَّسَةٌ وَالْفِضَّةُ وَالذَّهَبُ تَبَرُّعٌ لِلرَّبِّ إِلَهِ آبَائِكُمْ.
\par 29 فَاسْهَرُوا وَاحْفَظُوهَا حَتَّى تَزِنُوهَا أَمَامَ رُؤَسَاءِ الْكَهَنَةِ وَاللاَّوِيِّينَ وَرُؤَسَاءِ آبَاءِ إِسْرَائِيلَ فِي أُورُشَلِيمَ فِي مَخَادِعِ بَيْتِ الرَّبِّ].
\par 30 فَأَخَذَ الْكَهَنَةُ وَاللاَّوِيُّونَ وَزْنَ الْفِضَّةِ وَالذَّهَبِ وَالآنِيَةِ لِيَأْتُوا بِهَا إِلَى أُورُشَلِيمَ إِلَى بَيْتِ إِلَهِنَا.
\par 31 ثُمَّ رَحَلْنَا مِنْ نَهْرِ أَهْوَا فِي الثَّانِي عَشَرَ مِنَ الشَّهْرِ الأَوَّلِ لِنَذْهَبَ إِلَى أُورُشَلِيمَ وَكَانَتْ يَدُ إِلَهِنَا عَلَيْنَا فَأَنْقَذَنَا مِنْ يَدِ الْعَدُوِّ وَالْكَامِنِ عَلَى الطَّرِيقِ.
\par 32 فَأَتَيْنَا إِلَى أُورُشَلِيمَ وَأَقَمْنَا هُنَاكَ ثَلاَثَةَ أَيَّامٍ.
\par 33 وَفِي الْيَوْمِ الرَّابِعِ وُزِنَتِ الْفِضَّةُ وَالذَّهَبُ وَالآنِيَةُ فِي بَيْتِ إِلَهِنَا عَلَى يَدِ مَرِيمُوثَ بْنِ أُورِيَّا الْكَاهِنِ وَمَعَهُ أَلِعَازَارُ بْنُ فِينَحَاسَ وَمَعَهُمَا يُوزَابَادُ بْنُ يَشُوعَ وَنُوعَدْيَا بْنُ بَنُّويَ اللاَّوِيَّانِ.
\par 34 بِالْعَدَدِ وَالْوَزْنِ لِلْكُلِّ وَكُتِبَ كُلُّ الْوَزْنِ فِي ذَلِكَ الْوَقْتِ.
\par 35 وَبَنُو السَّبْيِ الْقَادِمُونَ مِنَ السَّبْيِ قَرَّبُوا مُحْرَقَاتٍ لإِلَهِ إِسْرَائِيلَ اثْنَيْ عَشَرَ ثَوْراً عَنْ كُلِّ إِسْرَائِيلَ وَسِتَّةً وَتِسْعِينَ كَبْشاً وَسَبْعَةً وَسَبْعِينَ خَرُوفاً وَاثْنَيْ عَشَرَ تَيْساً ذَبِيحَةَ خَطِيَّةٍ. الْجَمِيعُ مُحْرَقَةٌ لِلرَّبِّ.
\par 36 وَأَعْطُوا أَوَامِرَ الْمَلِكِ لِمَرَازِبَةِ الْمَلِكِ وَوُلاَةِ عَبْرِ النَّهْرِ فَأَعَانُوا الشَّعْبَ وَبَيْتَ اللَّهِ.

\chapter{9}

\par 1 وَلَمَّا كَمُلَتْ هَذِهِ تَقَدَّمَ إِلَيَّ الرُّؤَسَاءُ قَائِلِينَ: [لَمْ يَنْفَصِلْ شَعْبُ إِسْرَائِيلَ وَالْكَهَنَةُ وَاللاَّوِيُّونَ مِنْ شُعُوبِ الأَرَاضِي حَسَبَ رَجَاسَاتِهِمْ مِنَ الْكَنْعَانِيِّينَ وَالْحِثِّيِّينَ وَالْفِرِزِّيِّينَ وَالْيَبُوسِيِّينَ وَالْعَمُّونِيِّينَ وَالْمُوآبِيِّينَ وَالْمِصْرِيِّينَ وَالأَمُورِيِّينَ.
\par 2 لأَنَّهُمُ اتَّخَذُوا مِنْ بَنَاتِهِمْ لأَنْفُسِهِمْ وَلِبَنِيهِمْ وَاخْتَلَطَ الزَّرْعُ الْمُقَدَّسُ بِشُعُوبِ الأَرَاضِي. وَكَانَتْ يَدُ الرُّؤَسَاءِ وَالْوُلاَةِ فِي هَذِهِ الْخِيَانَةِ أَوَّلاً].
\par 3 فَلَمَّا سَمِعْتُ بِهَذَا الأَمْرِ مَزَّقْتُ ثِيَابِي وَرِدَائِي وَنَتَّفْتُ شَعْرَ رَأْسِي وَذَقْنِي وَجَلَسْتُ مُتَحَيِّراً.
\par 4 فَاجْتَمَعَ إِلَيَّ كُلُّ مَنِ ارْتَعَدَ مِنْ كَلاَمِ إِلَهِ إِسْرَائِيلَ مِنْ أَجْلِ خِيَانَةِ الْمَسْبِيِّينَ وَأَنَا جَلَسْتُ مُتَحَيِّراً إِلَى تَقْدِمَةِ الْمَسَاءِ.
\par 5 وَعِنْدَ تَقْدِمَةِ الْمَسَاءِ قُمْتُ مِنْ تَذَلُّلِي وَفِي ثِيَابِي وَرِدَائِي الْمُمَزَّقَةِ جَثَوْتُ عَلَى رُكْبَتَيَّ وَبَسَطْتُ يَدَيَّ إِلَى الرَّبِّ إِلَهِي
\par 6 وَقُلْتُ: [اللَّهُمَّ إِنِّي أَخْجَلُ وَأَخْزَى مِنْ أَنْ أَرْفَعَ يَا إِلَهِي وَجْهِي نَحْوَكَ لأَنَّ ذُنُوبَنَا قَدْ كَثُرَتْ فَوْقَ رُؤُوسِنَا وَآثَامَنَا تَعَاظَمَتْ إِلَى السَّمَاءِ.
\par 7 مُنْذُ أَيَّامِ آبَائِنَا نَحْنُ فِي إِثْمٍ عَظِيمٍ إِلَى هَذَا الْيَوْمِ. وَلأَجْلِ ذُنُوبِنَا قَدْ دُفِعْنَا نَحْنُ وَمُلُوكُنَا وَكَهَنَتُنَا لِيَدِ مُلُوكِ الأَرَاضِي لِلسَّيْفِ وَالسَّبْيِ وَالنَّهْبِ وَخِزْيِ الْوُجُوهِ كَهَذَا الْيَوْمِ.
\par 8 وَالآنَ كَلُحَيْظَةٍ كَانَتْ رَأْفَةٌ مِنْ لَدُنِ الرَّبِّ إِلَهِنَا لِيُبْقِيَ لَنَا نَجَاةً وَيُعْطِيَنَا وَتَداً فِي مَكَانِ قُدْسِهِ لِيُنِيرَ إِلَهُنَا أَعْيُنَنَا وَيُعْطِيَنَا حَيَاةً قَلِيلَةً فِي عُبُودِيَّتِنَا.
\par 9 لأَنَّنَا عَبِيدٌ نَحْنُ وَفِي عُبُودِيَّتِنَا لَمْ يَتْرُكْنَا إِلَهُنَا بَلْ بَسَطَ عَلَيْنَا رَحْمَةً أَمَامَ مُلُوكِ فَارِسَ لِيُعْطِيَنَا حَيَاةً لِنَرْفَعَ بَيْتَ إِلَهِنَا وَنُقِيمَ خَرَائِبَهُ وَلْيُعْطِيَنَا حَائِطاً فِي يَهُوذَا وَفِي أُورُشَلِيمَ.
\par 10 وَالآنَ فَمَاذَا نَقُولُ يَا إِلَهَنَا بَعْدَ هَذَا لأَنَّنَا قَدْ تَرَكْنَا وَصَايَاكَ
\par 11 الَّتِي أَوْصَيْتَ بِهَا عَنْ يَدِ عَبِيدِكَ الأَنْبِيَاءِ قَائِلاً: إِنَّ الأَرْضَ الَّتِي تَدْخُلُونَ لِتَمْتَلِكُوهَا هِيَ أَرْضٌ مُتَنَجِّسَةٌ بِنَجَاسَةِ شُعُوبِ الأَرَاضِي بِرَجَاسَاتِهِمِ الَّتِي مَلَأُوهَا بِهَا مِنْ جِهَةٍ إِلَى جِهَةٍ بِنَجَاسَتِهِمْ.
\par 12 وَالآنَ فَلاَ تُعْطُوا بَنَاتِكُمْ لِبَنِيهِمْ وَلاَ تَأْخُذُوا بَنَاتِهِمْ لِبَنِيكُمْ وَلاَ تَطْلُبُوا سَلاَمَتَهُمْ وَخَيْرَهُمْ إِلَى الأَبَدِ لِتَتَشَدَّدُوا وَتَأْكُلُوا خَيْرَ الأَرْضِ وَتُورِثُوا بَنِيكُمْ إِيَّاهَا إِلَى الأَبَدِ.
\par 13 وَبَعْدَ كُلِّ مَا جَاءَ عَلَيْنَا لأَجْلِ أَعْمَالِنَا الرَّدِيئَةِ وَآثَامِنَا الْعَظِيمَةِ - لأَنَّكَ قَدْ جَازَيْتَنَا يَا إِلَهَنَا أَقَلَّ مِنْ آثَامِنَا وَأَعْطَيْتَنَا نَجَاةً كَهَذِهِ
\par 14 أَفَنَعُودُ وَنَتَعَدَّى وَصَايَاكَ وَنُصَاهِرُ شُعُوبَ هَذِهِ الرَّجَاسَاتِ؟ أَمَا تَسْخَطُ عَلَيْنَا حَتَّى تُفْنِيَنَا فَلاَ تَكُونُ بَقِيَّةٌ وَلاَ نَجَاةٌ؟
\par 15 أَيُّهَا الرَّبُّ إِلَهَ إِسْرَائِيلَ أَنْتَ بَارٌّ لأَنَّنَا بَقِينَا نَاجِينَ كَهَذَا الْيَوْمِ. هَا نَحْنُ أَمَامَكَ فِي آثَامِنَا لأَنَّهُ لَيْسَ لَنَا أَنْ نَقِفَ أَمَامَكَ مِنْ أَجْلِ هَذَا].

\chapter{10}

\par 1 فَلَمَّا صَلَّى عَزْرَا وَاعْتَرَفَ وَهُوَ بَاكٍ وَسَاقِطٌ أَمَامَ بَيْتِ اللَّهِ اجْتَمَعَ إِلَيْهِ مِنْ إِسْرَائِيلَ جَمَاعَةٌ كَثِيرَةٌ جِدّاً مِنَ الرِّجَالِ وَالنِّسَاءِ وَالأَوْلاَدِ لأَنَّ الشَّعْبَ بَكَى بُكَاءً عَظِيماً.
\par 2 وَقَالَ شَكَنْيَا بْنُ يَحِيئِيلَ مِنْ بَنِي عِيلاَمَ لِعَزْرَا: [إِنَّنَا قَدْ خُنَّا إِلَهَنَا وَاتَّخَذْنَا نِسَاءً غَرِيبَةً مِنْ شُعُوبِ الأَرْضِ. وَلَكِنِ الآنَ يُوجَدُ رَجَاءٌ لإِسْرَائِيلَ فِي هَذَا.
\par 3 فَلْنَقْطَعِ الآنَ عَهْداً مَعَ إِلَهِنَا أَنْ نُخْرِجَ كُلَّ النِّسَاءِ وَالَّذِينَ وُلِدُوا مِنْهُنَّ حَسَبَ مَشُورَةِ سَيِّدِي وَالَّذِينَ يَخْشُونَ وَصِيَّةَ إِلَهِنَا وَلْيُعْمَلْ حَسَبَ الشَّرِيعَةِ.
\par 4 قُمْ فَإِنَّ عَلَيْكَ الأَمْرَ وَنَحْنُ مَعَكَ. تَشَجَّعْ وَافْعَلْ].
\par 5 فَقَامَ عَزْرَا وَاسْتَحْلَفَ رُؤَسَاءَ الْكَهَنَةِ وَاللاَّوِيِّينَ وَكُلَّ إِسْرَائِيلَ أَنْ يَعْمَلُوا حَسَبَ هَذَا الأَمْرِ فَحَلَفُوا.
\par 6 ثُمَّ قَامَ عَزْرَا مِنْ أَمَامِ بَيْتِ اللَّهِ وَذَهَبَ إِلَى مِخْدَعِ يَهُوحَانَانَ بْنِ أَلْيَاشِيبَ. فَانْطَلَقَ إِلَى هُنَاكَ وَهُوَ لَمْ يَأْكُلْ خُبْزاً وَلَمْ يَشْرَبْ مَاءً لأَنَّهُ كَانَ يَنُوحُ بِسَبَبِ خِيَانَةِ أَهْلِ السَّبْيِ.
\par 7 وَأَطْلَقُوا نِدَاءً فِي يَهُوذَا وَأُورُشَلِيمَ إِلَى جَمِيعِ بَنِي السَّبْيِ لِكَيْ يَجْتَمِعُوا إِلَى أُورُشَلِيمَ.
\par 8 وَكُلُّ مَنْ لاَ يَأْتِي فِي ثَلاَثَةِ أَيَّامٍ حَسَبَ مَشُورَةِ الرُّؤَسَاءِ وَالشُّيُوخِ يُحَرَّمُ كُلُّ مَالِهِ وَهُوَ يُفْرَزُ مِنْ جَمَاعَةِ أَهْلِ السَّبْيِ.
\par 9 فَاجْتَمَعَ كُلُّ رِجَالِ يَهُوذَا وَبِنْيَامِينَ إِلَى أُورُشَلِيمَ فِي الثَّلاَثَةِ الأَيَّامِ أَيْ فِي الشَّهْرِ التَّاسِعِ فِي الْعِشْرِينَ مِنَ الشَّهْرِ وَجَلَسَ جَمِيعُ الشَّعْبِ فِي سَاحَةِ بَيْتِ اللَّهِ مُرْتَعِدِينَ مِنَ الأَمْرِ وَمِنَ الأَمْطَارِ.
\par 10 فَقَامَ عَزْرَا الْكَاهِنُ وَقَالَ لَهُمْ: [إِنَّكُمْ قَدْ خُنْتُمْ وَاتَّخَذْتُمْ نِسَاءً غَرِيبَةً لِتَزِيدُوا عَلَى إِثْمِ إِسْرَائِيلَ.
\par 11 فَاعْتَرِفُوا الآنَ لِلرَّبِّ إِلَهِ آبَائِكُمْ وَاعْمَلُوا مَرْضَاتَهُ وَانْفَصِلُوا عَنْ شُعُوبِ الأَرْضِ وَعَنِ النِّسَاءِ الْغَرِيبَةِ].
\par 12 فَأَجَابَ كُلُّ الْجَمَاعَةِ بِصَوْتٍ عَظِيمٍ: [كَمَا كَلَّمْتَنَا كَذَلِكَ نَعْمَلُ.
\par 13 إِلاَّ أَنَّ الشَّعْبَ كَثِيرٌ وَالْوَقْتَ وَقْتُ أَمْطَارٍ وَلاَ طَاقَةَ لَنَا عَلَى الْوُقُوفِ فِي الْخَارِجِ وَالْعَمَلُ لَيْسَ لِيَوْمٍ وَاحِدٍ أَوْ لاِثْنَيْنِ لأَنَّنَا قَدْ أَكْثَرْنَا الذَّنْبَ فِي هَذَا الأَمْرِ.
\par 14 فَلْيَقِفْ رُؤَسَاؤُنَا لِكُلِّ الْجَمَاعَةِ. وَكُلُّ الَّذِينَ فِي مُدُنِنَا قَدِ اتَّخَذُوا نِسَاءً غَرِيبَةً فَلْيَأْتُوا فِي أَوْقَاتٍ مُعَيَّنَةٍ وَمَعَهُمْ شُيُوخُ مَدِينَةٍ فَمَدِينَةٍ وَقُضَاتُهَا حَتَّى يَرْتَدَّ عَنَّا حُمُوُّ غَضَبِ إِلَهِنَا مِنْ أَجْلِ هَذَا الأَمْرِ].
\par 15 وَيُونَاثَانُ بْنُ عَسَائِيلَ وَيَحْزِيَا بْنُ تِقْوَةَ فَقَطْ قَامَا عَلَى هَذَا وَمَشُلاَّمُ وَشَبْتَايُ اللاَّوِيُّ سَاعَدَاهُمَا.
\par 16 وَفَعَلَ هَكَذَا بَنُو السَّبْيِ. وَانْفَصَلَ عَزْرَا الْكَاهِنُ وَرِجَالٌ رُؤُوسُ آبَاءٍ حَسَبَ بُيُوتِ آبَائِهِمْ وَجَمِيعُهُمْ بِأَسْمَائِهِمْ وَجَلَسُوا فِي الْيَوْمِ الأَوَّلِ مِنَ الشَّهْرِ الْعَاشِرِ لِلْفَحْصِ عَنِ الأَمْرِ.
\par 17 وَانْتَهُوا مِنْ كُلِّ الرِّجَالِ الَّذِينَ اتَّخَذُوا نِسَاءً غَرِيبَةً فِي الْيَوْمِ الأَوَّلِ مِنَ الشَّهْرِ الأَوَّلِ.
\par 18 فَوُجِدَ بَيْنَ بَنِي الْكَهَنَةِ مَنِ اتَّخَذَ نِسَاءً غَرِيبَةً. فَمِنْ بَنِي يَشُوعَ بْنِ يُوصَادَاقَ وَإِخْوَتِهِ مَعْشِيَّا وَأَلِيعَزَرُ وَيَارِيبُ وَجَدَلْيَا.
\par 19 وَأَعْطُوا أَيْدِيَهُمْ لإِخْرَاجِ نِسَائِهِمْ مُقَرِّبِينَ كَبْشَ غَنَمٍ لأَجْلِ إِثْمِهِمْ.
\par 20 وَمِنْ بَنِي إِمِّيرَ حَنَانِي وَزَبْدِيَا.
\par 21 وَمِنْ بَنِي حَارِيمَ مَعْسِيَّا وَإِيلِيَّا وَشَمَعْيَا وَيَحِيئِيلُ وَعُزِّيَّا.
\par 22 وَمِنْ بَنِي فَشْحُورَ أَلْيُوعِينَايُ وَمَعْسِيَّا وَإِسْمَعِيلُ وَنَثَنْئِيلُ وَيُوزَابَادُ وَأَلْعَاسَةُ.
\par 23 وَمِنَ اللاَّوِيِّينَ يُوزَابَادُ وَشَمْعِي وَقَلاَيَا (هُوَ قَلِيطَا). وَفَتَحْيَا وَيَهُوذَا وَأَلِيعَزَرُ.
\par 24 وَمِنَ الْمُغَنِّينَ أَلْيَاشِيبُ. وَمِنَ الْبَوَّابِينَ شَلُّومُ وَطَالَمُ وَأُورِي.
\par 25 وَمِنْ إِسْرَائِيلَ مِنْ بَنِي فَرْعُوشَ رَمْيَا وَيِزِّيَّا وَمَلْكِيَّا وَمِيَّامِينُ وَأَلِعَازَارُ وَمَلْكِيَّا وَبَنَايَا.
\par 26 وَمِنْ بَنِي عِيلاَمَ مَتَّنْيَا وَزَكَرِيَّا وَيَحِيئِيلُ وَعَبْدِي وَيَرِيمُوثُ وَإِيلِيَّا.
\par 27 وَمِنْ بَنِي زَتُّو أَلْيُوعِينَايُ وَأَلْيَاشِيبُ وَمَتَّنْيَا وَيَرِيمُوثُ وَزَابَادُ وَعَزِيزَا.
\par 28 وَمِنْ بَنِي بَابَايَ يَهُوحَانَانُ وَحَنَنْيَا وَزَبَايُ وَعَثْلاَيُ.
\par 29 وَمِنْ بَنِي بَانِي مَشُلاَّمُ وَمَلُّوخُ وَعَدَايَا وَيَاشُوبُ وَشَآلُ وَرَامُوثُ.
\par 30 وَمِنْ بَنِي فَحَثَ مُوآبُ عَدْنَا وَكَلاَلُ وَبَنَايَا وَمَعْسِيَّا وَمَتَّنْيَا وَبَصَلْئِيلُ وَبِنُّويُ وَمَنَسَّى.
\par 31 وَبَنُو حَارِيمَ أَلِيعَزَرُ وَيِشِّيَّا وَمَلْكِيَّا وَشَمَعْيَا وَشَمْعُونُ
\par 32 وَبِنْيَامِينُ وَمَلُّوخُ وَشَمَرْيَا.
\par 33 مِنْ بَنِي حَشُومَ مَتَّنَايُ ومَتَّاثَا وزَابَادُ وَأَلِيفَلَطُ ويَرِيمَايُ وَمَنَسَّى وَشَمْعِي.
\par 34 مِنْ بَنِي بَانِي مَعَدَايُ وَعَمْرَامُ وَأُوئِيلُ
\par 35 وَبَنَايَا وَبِيدْيَا وكَلُوهِي
\par 36 ووَنْيَا وَمَرِيمُوثُ وَأَلْيَاشِيبُ
\par 37 وَمَتَّنْيَا وَمَتَّنَايُ وَيَعْسُو
\par 38 وَبَانِي وَبِنُّويُ وَشَمْعِي
\par 39 وَشَلَمْيَا وَنَاثَانُ وَعَدَايَا
\par 40 وَمَكْنَدْبَايُ وَشَاشَايُ وَشَارَايُ
\par 41 وَعَزَرْئِيلُ وَشَلْمِيَا وَشَمَرْيَا
\par 42 وَشَلُّومُ وَأَمَرْيَا وَيُوسُفُ.
\par 43 مِنْ بَنِي نَبُو يَعِيئِيلُ وَمَتَّثْيَا وَزَابَادُ وَزَبِينَا وَيَدُّو وَيُوئِيلُ وَبَنَايَا.
\par 44 كُلُّ هَؤُلاَءِ اتَّخَذُوا نِسَاءً غَرِيبَةً وَمِنْهُنَّ نِسَاءٌ قَدْ وَضَعْنَ بَنِينَ.

\end{document}