\begin{document}

\title{قضاة }


\chapter{1}

\par 1 وَكَانَ بَعْدَ مَوْتِ يَشُوعَ أَنَّ بَنِي إِسْرَائِيلَ سَأَلُوا الرَّبَّ: «مَنْ مِنَّا يَصْعَدُ إِلَى الْكَنْعَانِيِّينَ أَّوَلاً لِمُحَارَبَتِهِمْ؟»
\par 2 فَقَالَ الرَّبُّ: «يَهُوذَا يَصْعَدُ. هُوَذَا قَدْ دَفَعْتُ الأَرْضَ لِيَدِهِ».
\par 3 فَقَالَ يَهُوذَا لِشَمْعُونَ أَخِيهِ: «اِصْعَدْ مَعِي فِي قُرْعَتِي لِنُحَارِبَ الْكَنْعَانِيِّينَ, فَأَصْعَدَ أَنَا أَيْضاً مَعَكَ فِي قُرْعَتِكَ». فَذَهَبَ شَمْعُونُ مَعَهُ.
\par 4 فَصَعِدَ يَهُوذَا. وَدَفَعَ الرَّبُّ الْكَنْعَانِيِّينَ وَالْفِرِّزِيِّينَ بِيَدِهِمْ, فَضَرَبُوا مِنْهُمْ فِي بَازَقَ عَشَرَةَ آلاَفِ رَجُلٍ.
\par 5 وَوَجَدُوا أَدُونِيَ بَازَقَ فِي بَازَقَ, فَحَارَبُوهُ وَضَرَبُوا الْكَنْعَانِيِّينَ وَالْفِرِّزِيِّينَ.
\par 6 فَهَرَبَ أَدُونِي بَازَقَ. فَتَبِعُوهُ وَأَمْسَكُوهُ وَقَطَعُوا أَبَاهِمَ يَدَيْهِ وَرِجْلَيْهِ.
\par 7 فَقَالَ أَدُونِي بَازَقَ: «سَبْعُونَ مَلِكاً مَقْطُوعَةٌ أَبَاهِمُ أَيْدِيهِمْ وَأَرْجُلِهِمْ كَانُوا يَلْتَقِطُونَ تَحْتَ مَائِدَتِي. كَمَا فَعَلْتُ كَذَلِكَ جَازَانِيَ اللَّهُ». وَأَتُوا بِهِ إِلَى أُورُشَلِيمَ فَمَاتَ هُنَاكَ.
\par 8 وَحَارَبَ بَنُو يَهُوذَا أُورُشَلِيمَ وَأَخَذُوهَا وَضَرَبُوهَا بِحَدِّ السَّيْفِ وَأَشْعَلُوا الْمَدِينَةَ بِالنَّارِ.
\par 9 وَبَعْدَ ذَلِكَ نَزَلَ بَنُو يَهُوذَا لِمُحَارَبَةِ الْكَنْعَانِيِّينَ سُكَّانِ الْجَبَلِ وَالْجَنُوبِ وَالسَّهْلِ.
\par 10 وَسَارَ يَهُوذَا عَلَى الْكَنْعَانِيِّينَ السَّاكِنِينَ فِي حَبْرُونَ (وَكَانَ اسْمُ حَبْرُونَ قَبْلاً قَرْيَةَ أَرْبَعَ) وَضَرَبُوا شِيشَايَ وَأَخِيمَانَ وَتَلْمَايَ.
\par 11 وَسَارَ مِنْ هُنَاكَ عَلَى سُكَّانِ دَبِيرَ (وَاسْمُ دَبِيرَ قَبْلاً قَرْيَةُ سَفَرٍ).
\par 12 فَقَالَ كَالِبُ: «الَّذِي يَضْرِبُ قَرْيَةَ سَفَرٍ وَيَأْخُذُهَا, أُعْطِيهِ عَكْسَةَ ابْنَتِي امْرَأَةً».
\par 13 فَأَخَذَهَا عُثْنِيئِيلُ بْنُ قَنَازَ أَخُو كَالِبَ الأَصْغَرِ مِنْهُ. فَأَعْطَاهُ عَكْسَةَ ابْنَتَهُ امْرَأَةً.
\par 14 وَكَانَ عِنْدَ دُخُولِهَا أَنَّهَا غَرَّتْهُ بِطَلَبِ حَقْلٍ مِنْ أَبِيهَا. فَنَزَلَتْ عَنِ الْحِمَارِ, فَقَالَ لَهَا كَالِبُ: «مَا لَكِ؟»
\par 15 فَقَالَتْ لَهُ: «أَعْطِنِي بَرَكَةً. لأَنَّكَ أَعْطَيْتَنِي أَرْضَ الْجَنُوبِ فَأَعْطِنِي يَنَابِيعَ مَاءٍ». فَأَعْطَاهَا كَالِبُ الْيَنَابِيعَ الْعُلْيَا وَالْيَنَابِيعَ السُّفْلَى.
\par 16 وَبَنُو الْقِينِيِّ حَمِي مُوسَى صَعِدُوا مِنْ مَدِينَةِ النَّخْلِ مَعَ بَنِي يَهُوذَا إِلَى بَرِّيَّةِ يَهُوذَا الَّتِي فِي جَنُوبِ عَرَادَ, وَذَهَبُوا وَسَكَنُوا مَعَ الشَّعْبِ.
\par 17 وَذَهَبَ يَهُوذَا مَعَ شَمْعُونَ أَخِيهِ وَضَرَبُوا الْكَنْعَانِيِّينَ سُكَّانَ صَفَاةَ وَحَرَّمُوهَا, وَدَعَوُا اسْمَ الْمَدِينَةِ «حُرْمَةَ».
\par 18 وَأَخَذَ يَهُوذَا غَّزَةَ وَتُخُومَهَا وَأَشْقَلُونَ وَتُخُومَهَا وَعَقْرُونَ وَتُخُومَهَا.
\par 19 وَكَانَ الرَّبُّ مَعَ يَهُوذَا فَمَلَكَ الْجَبَلَ, وَلَكِنْ لَمْ يُطْرَدْ سُكَّانُ الْوَادِي لأَنَّ لَهُمْ مَرْكَبَاتِ حَدِيدٍ.
\par 20 وَأَعْطُوا لِكَالِبَ حَبْرُونَ كَمَا تَكَلَّمَ مُوسَى. فَطَرَدَ مِنْ هُنَاكَ بَنِي عَنَاقَ الثَّلاَثَةَ.
\par 21 وَبَنُو بِنْيَامِينَ لَمْ يَطْرُدُوا الْيَبُوسِيِّينَ سُكَّانَ أُورُشَلِيمَ, فَسَكَنَ الْيَبُوسِيُّونَ مَعَ بَنِي بِنْيَامِينَ فِي أُورُشَلِيمَ إِلَى هَذَا الْيَوْمِ.
\par 22 وَصَعِدَ بَيْتُ يُوسُفَ أَيْضاً إِلَى بَيْتِ إِيلَ وَالرَّبُّ مَعَهُمْ.
\par 23 وَاسْتَكْشَفَ بَيْتُ يُوسُفَ عَنْ بَيْتِ إِيلَ (وَكَانَ اسْمُ الْمَدِينَةِ قَبْلاً لُوزَ).
\par 24 فَرَأَى الْمُرَاقِبُونَ رَجُلاً خَارِجاً مِنَ الْمَدِينَةِ, فَقَالُوا لَهُ: «أَرِنَا مَدْخَلَ الْمَدِينَةِ فَنَعْمَلَ مَعَكَ مَعْرُوفاً».
\par 25 فَأَرَاهُمْ مَدْخَلَ الْمَدِينَةِ, فَضَرَبُوا الْمَدِينَةَ بِحَدِّ السَّيْفِ, وَأَمَّا الرَّجُلُ وَكُلُّ عَشِيرَتِهِ فَأَطْلَقُوهُمْ.
\par 26 فَانْطَلَقَ الرَّجُلُ إِلَى أَرْضِ الْحِثِّيِّينَ وَبَنَى مَدِينَةً وَدَعَا اسْمَهَا »لُوزَ» وَهُوَ اسْمُهَا إِلَى هَذَا الْيَوْمِ.
\par 27 وَلَمْ يَطْرُدْ مَنَسَّى أَهْلَ بَيْتِ شَانَ وَقُرَاهَا, وَلاَ أَهْلَ تَعْنَكَ وَقُرَاهَا, وَلاَ سُكَّانَ دُورَ وَقُرَاهَا, وَلاَ سُكَّانَ يِبْلَعَامَ وَقُرَاهَا, وَلاَ سُكَّانَ مَجِدُّو وَقُرَاهَا. فَعَزَمَ الْكَنْعَانِيُّونَ عَلَى السَّكَنِ فِي تِلْكَ الأَرْضِ.
\par 28 وَكَانَ لَمَّا تَشَدَّدَ إِسْرَائِيلُ أَنَّهُ وَضَعَ الْكَنْعَانِيِّينَ تَحْتَ الْجِزْيَةِ وَلَمْ يَطْرُدْهُمْ طَرْداً.
\par 29 وَأَفْرَايِمُ لَمْ يَطْرُدِ الْكَنْعَانِيِّينَ السَّاكِنِينَ فِي جَازَرَ, فَسَكَنَ الْكَنْعَانِيُّونَ فِي وَسَطِهِ فِي جَازَرَ.
\par 30 زَبُولُونُ لَمْ يَطْرُدْ سُكَّانَ قِطْرُونَ وَلاَ سُكَّانَ نَهْلُولَ, فَسَكَنَ الْكَنْعَانِيُّونَ فِي وَسَطِهِ وَكَانُوا تَحْتَ الْجِزْيَةِ.
\par 31 وَلَمْ يَطْرُدْ أَشِيرُ سُكَّانَ عَكُّو وَلاَ سُكَّانَ صَيْدُونَ وَأَحْلَبَ وَأَكْزِيبَ وَحَلْبَةَ وَأَفِيقَ وَرَحُوبَ.
\par 32 فَسَكَنَ الأَشِيرِيُّونَ فِي وَسَطِ الْكَنْعَانِيِّينَ سُكَّانِ الأَرْضِ, لأَنَّهُمْ لَمْ يَطْرُدُوهُمْ.
\par 33 وَنَفْتَالِي لَمْ يَطْرُدْ سُكَّانَ بَيْتِ شَمْسٍ وَلاَ سُكَّانَ بَيْتِ عَنَاةَ, بَلْ سَكَنَ فِي وَسَطِ الْكَنْعَانِيِّينَ سُكَّانِ الأَرْضِ. فَكَانَ سُكَّانُ بَيْتِ شَمْسٍ وَبَيْتِ عَنَاةَ تَحْتَ الْجِزْيَةِ لَهُمْ.
\par 34 وَحَصَرَ الأَمُورِيُّونَ بَنِي دَانَ فِي الْجَبَلِ لأَنَّهُمْ لَمْ يَدَعُوهُمْ يَنْزِلُونَ إِلَى الْوَادِي.
\par 35 فَعَزَمَ الأَمُورِيُّونَ عَلَى السَّكَنِ فِي جَبَلِ حَارَسَ فِي أَيَّلُونَ وَفِي شَعَلُبِّيمَ. وَقَوِيَتْ يَدُ بَيْتِ يُوسُفَ فَكَانُوا تَحْتَ الْجِزْيَةِ.
\par 36 وَكَانَ تُخُمُ الأَمُورِيِّينَ مِنْ عَقَبَةِ عَقْرِبِّيمَ مِنْ سَالِعَ فَصَاعِداً.

\chapter{2}

\par 1 وَصَعِدَ مَلاَكُ الرَّبِّ مِنَ الْجِلْجَالِ إِلَى بُوكِيمَ وَقَالَ: «قَدْ أَصْعَدْتُكُمْ مِنْ مِصْرَ وَأَتَيْتُ بِكُمْ إِلَى الأَرْضِ الَّتِي أَقْسَمْتُ لآِبَائِكُمْ, وَقُلْتُ: لاَ أَنْكُثُ عَهْدِي مَعَكُمْ إِلَى الأَبَدِ.
\par 2 وَأَنْتُمْ فَلاَ تَقْطَعُوا عَهْداً مَعَ سُكَّانِ هَذِهِ الأَرْضِ. اهْدِمُوا مَذَابِحَهُمْ. وَلَمْ تَسْمَعُوا لِصَوْتِي. فَمَاذَا عَمِلْتُمْ؟
\par 3 فَقُلْتُ أَيْضاً: لاَ أَطْرُدُهُمْ مِنْ أَمَامِكُمْ بَلْ يَكُونُونَ لَكُمْ مُضَايِقِينَ, وَتَكُونُ آلِهَتُهُمْ لَكُمْ شَرَكاً».
\par 4 وَكَانَ لَمَّا تَكَلَّمَ مَلاَكُ الرَّبِّ بِهَذَا الْكَلاَمِ إِلَى جَمِيعِ بَنِي إِسْرَائِيلَ أَنَّ الشَّعْبَ رَفَعُوا صَوْتَهُمْ وَبَكُوا.
\par 5 فَدَعُوا اسْمَ ذَلِكَ الْمَكَانِ «بُوكِيمَ». وَذَبَحُوا هُنَاكَ لِلرَّبِّ.
\par 6 وَصَرَفَ يَشُوعُ الشَّعْبَ, فَذَهَبَ بَنُو إِسْرَائِيلَ كُلُّ وَاحِدٍ إِلَى مُلْكِهِ لأَجْلِ امْتِلاَكِ الأَرْضِ.
\par 7 وَعَبَدَ الشَّعْبُ الرَّبَّ كُلَّ أَيَّامِ يَشُوعَ, وَكُلَّ أَيَّامِ الشُّيُوخِ الَّذِينَ طَالَتْ أَيَّامُهُمْ بَعْدَ يَشُوعَ الَّذِينَ رَأُوا كُلَّ عَمَلِ الرَّبِّ الْعَظِيمِ الَّذِي عَمِلَ لإِسْرَائِيلَ.
\par 8 وَمَاتَ يَشُوعُ بْنُ نُونَ عَبْدُ الرَّبِّ ابْنَ مِئَةٍ وَعَشَرَ سِنِينَ.
\par 9 فَدَفَنُوهُ فِي تُخُمِ مُلْكِهِ فِي تِمْنَةَ حَارَسَ فِي جَبَلِ أَفْرَايِمَ, شِمَالِيَّ جَبَلِ جَاعَشَ.
\par 10 وَكُلُّ ذَلِكَ الْجِيلِ أَيْضاً انْضَمَّ إِلَى آبَائِهِ, وَقَامَ بَعْدَهُمْ جِيلٌ آخَرُ لَمْ يَعْرِفِ الرَّبَّ وَلاَ الْعَمَلَ الَّذِي عَمِلَ لإِسْرَائِيلَ.
\par 11 وَفَعَلَ بَنُو إِسْرَائِيلَ الشَّرَّ فِي عَيْنَيِ الرَّبِّ وَعَبَدُوا الْبَعْلِيمَ,
\par 12 وَتَرَكُوا الرَّبَّ إِلَهَ آبَائِهِمِ الَّذِي أَخْرَجَهُمْ مِنْ أَرْضِ مِصْرَ وَسَارُوا وَرَاءَ آلِهَةٍ أُخْرَى مِنْ آلِهَةِ الشُّعُوبِ الَّذِينَ حَوْلَهُمْ, وَسَجَدُوا لَهَا وَأَغَاظُوا الرَّبَّ.
\par 13 تَرَكُوا الرَّبَّ وَعَبَدُوا الْبَعْلَ وَعَشْتَارُوثَ.
\par 14 فَحَمِيَ غَضَبُ الرَّبِّ عَلَى إِسْرَائِيلَ, فَدَفَعَهُمْ بِأَيْدِي نَاهِبِينَ نَهَبُوهُمْ, وَبَاعَهُمْ بِيَدِ أَعْدَائِهِمْ حَوْلَهُمْ, وَلَمْ يَقْدِرُوا بَعْدُ عَلَى الْوُقُوفِ أَمَامَ أَعْدَائِهِمْ.
\par 15 حَيْثُمَا خَرَجُوا كَانَتْ يَدُ الرَّبِّ عَلَيْهِمْ لِلشَّرِّ كَمَا تَكَلَّمَ الرَّبُّ وَكَمَا أَقْسَمَ الرَّبُّ لَهُمْ. فَضَاقَ بِهِمُ الأَمْرُ جِدّاً.
\par 16 وَأَقَامَ الرَّبُّ قُضَاةً فَخَلَّصُوهُمْ مِنْ يَدِ نَاهِبِيهِمْ.
\par 17 وَلِقُضَاتِهِمْ أَيْضاً لَمْ يَسْمَعُوا, بَلْ زَنُوا وَرَاءَ آلِهَةٍ أُخْرَى وَسَجَدُوا لَهَا. حَادُوا سَرِيعاً عَنِ الطَّرِيقِ الَّتِي سَارَ بِهَا آبَاؤُهُمْ لِسَمْعِ وَصَايَا الرَّبِّ. لَمْ يَفْعَلُوا هَكَذَا.
\par 18 وَحِينَمَا أَقَامَ الرَّبُّ لَهُمْ قُضَاةً كَانَ الرَّبُّ مَعَ الْقَاضِي, وَخَلَّصَهُمْ مِنْ يَدِ أَعْدَائِهِمْ كُلَّ أَيَّامِ الْقَاضِي, لأَنَّ الرَّبَّ نَدِمَ مِنْ أَجْلِ أَنِينِهِمْ بِسَبَبِ مُضَايِقِيهِمْ وَزَاحِمِيهِمْ.
\par 19 وَعِنْدَ مَوْتِ الْقَاضِي كَانُوا يَرْجِعُونَ وَيَفْسُدُونَ أَكْثَرَ مِنْ آبَائِهِمْ بِالذَّهَابِ وَرَاءَ آلِهَةٍ أُخْرَى لِيَعْبُدُوهَا وَيَسْجُدُوا لَهَا. لَمْ يَكُفُّوا عَنْ أَفْعَالِهِمْ وَطَرِيقِهِمِ الْقَاسِيَةِ.
\par 20 فَحَمِيَ غَضَبُ الرَّبِّ عَلَى إِسْرَائِيلَ وَقَالَ: «مِنْ أَجْلِ أَنَّ هَذَا الشَّعْبَ قَدْ تَعَدُّوا عَهْدِيَ الَّذِي أَوْصَيْتُ بِهِ آبَاءَهُمْ وَلَمْ يَسْمَعُوا لِصَوْتِي
\par 21 فَأَنَا أَيْضاً لاَ أَعُودُ أَطْرُدُ إِنْسَاناً مِنْ أَمَامِهِمْ مِنَ الأُمَمِ الَّذِينَ تَرَكَهُمْ يَشُوعُ عِنْدَ مَوْتِهِ
\par 22 لأَمْتَحِنَ بِهِمْ إِسْرَائِيلَ: أَيَحْفَظُونَ طَرِيقَ الرَّبِّ لِيَسْلُكُوا بِهَا كَمَا حَفِظَهَا آبَاؤُهُمْ, أَمْ لاَ».
\par 23 فَتَرَكَ الرَّبُّ أُولَئِكَ الأُمَمَ وَلَمْ يَطْرُدْهُمْ سَرِيعاً وَلَمْ يَدْفَعْهُمْ بِيَدِ يَشُوعَ.

\chapter{3}

\par 1 فَهَؤُلاَءِ هُمُ الأُمَمُ الَّذِينَ تَرَكَهُمُ الرَّبُّ لِيَمْتَحِنَ بِهِمْ إِسْرَائِيلَ, كُلَّ الَّذِينَ لَمْ يَعْرِفُوا جَمِيعَ حُرُوبِ كَنْعَانَ
\par 2 (إِنَّمَا لِمَعْرِفَةِ أَجْيَالِ بَنِي إِسْرَائِيلَ لِتَعْلِيمِهِمِ الْحَرْبَ. الَّذِينَ لَمْ يَعْرِفُوهَا قَبْلُ فَقَطْ)
\par 3 أَقْطَابُ الْفِلِسْطِينِيِّينَ الْخَمْسَةُ وَجَمِيعُ الْكَنْعَانِيِّينَ وَالصَّيْدُونِيِّينَ وَالْحِّوِيِّينَ سُكَّانِ جَبَلِ لُبْنَانَ مِنْ جَبَلِ بَعْلِ حَرْمُونَ إِلَى مَدْخَلِ حَمَاةَ.
\par 4 كَانُوا لاِمْتِحَانِ إِسْرَائِيلَ بِهِمْ, لِيُعْلَمَ هَلْ يَسْمَعُونَ وَصَايَا الرَّبِّ الَّتِي أَوْصَى بِهَا آبَاءَهُمْ عَنْ يَدِ مُوسَى.
\par 5 فَسَكَنَ بَنُو إِسْرَائِيلَ فِي وَسَطِ الْكَنْعَانِيِّينَ وَالْحِثِّيِّينَ وَالأَمُورِيِّينَ وَالْفِرِّزِيِّينَ وَالْحِّوِيِّينَ وَالْيَبُوسِيِّينَ,
\par 6 وَاتَّخَذُوا بَنَاتِهِمْ لأَنْفُسِهِمْ نِسَاءً وَأَعْطُوا بَنَاتِهِمْ لِبَنِيهِمْ وَعَبَدُوا آلِهَتَهُمْ.
\par 7 فَعَمِلَ بَنُو إِسْرَائِيلَ الشَّرَّ فِي عَيْنَيِ الرَّبِّ, وَنَسُوا الرَّبَّ إِلَهَهُمْ وَعَبَدُوا الْبَعْلِيمَ وَالسَّوَارِيَ.
\par 8 فَحَمِيَ غَضَبُ الرَّبِّ عَلَى إِسْرَائِيلَ, فَبَاعَهُمْ بِيَدِ كُوشَانَ رِشَعْتَايِمَ مَلِكِ أَرَامِ النَّهْرَيْنِ. فَعَبَدَ بَنُو إِسْرَائِيلَ كُوشَانَ رِشَعْتَايِمَ ثَمَانِيَ سِنِينَ.
\par 9 وَصَرَخَ بَنُو إِسْرَائِيلَ إِلَى الرَّبِّ, فَأَقَامَ الرَّبُّ مُخَلِّصاً لِبَنِي إِسْرَائِيلَ فَخَلَّصَهُمْ. عُثْنِيئِيلَ بْنَ قَنَازَ أَخَا كَالِبَ الأَصْغَرَ.
\par 10 فَكَانَ عَلَيْهِ رُوحُ الرَّبِّ, وَقَضَى لإِسْرَائِيلَ. وَخَرَجَ لِلْحَرْبِ فَدَفَعَ الرَّبُّ لِيَدِهِ كُوشَانَ رِشَعْتَايِمَ مَلِكَ أَرَامَ, وَاعْتَّزَتْ يَدُهُ عَلَى كُوشَانِ رِشَعْتَايِمَ.
\par 11 وَاسْتَرَاحَتِ الأَرْضُ أَرْبَعِينَ سَنَةً. وَمَاتَ عُثْنِيئِيلُ بْنُ قَنَازَ.
\par 12 وَعَادَ بَنُو إِسْرَائِيلَ يَعْمَلُونَ الشَّرَّ فِي عَيْنَيِ الرَّبِّ, فَشَدَّدَ الرَّبُّ عِجْلُونَ مَلِكَ مُوآبَ عَلَى إِسْرَائِيلَ, لأَنَّهُمْ عَمِلُوا الشَّرَّ فِي عَيْنَيِ الرَّبِّ.
\par 13 فَجَمَعَ إِلَيْهِ بَنِي عَمُّونَ وَعَمَالِيقَ, وَسَارَ وَضَرَبَ إِسْرَائِيلَ وَامْتَلَكُوا مَدِينَةَ النَّخْلِ.
\par 14 فَعَبَدَ بَنُو إِسْرَائِيلَ عِجْلُونَ مَلِكَ مُوآبَ ثَمَانِيَ عَشَرَةَ سَنَةً.
\par 15 وَصَرَخَ بَنُو إِسْرَائِيلَ إِلَى الرَّبِّ, فَأَقَامَ لَهُمُ الرَّبُّ مُخَلِّصاً إِهُودَ بْنَ جِيرَا الْبِنْيَامِينِيَّ, رَجُلاً أَعْسَرَ. فَأَرْسَلَ بَنُو إِسْرَائِيلَ بِيَدِهِ هَدِيَّةً لِعِجْلُونَ مَلِكِ مُوآبَ.
\par 16 فَعَمِلَ إِهُودُ لِنَفْسِهِ سَيْفاً ذَا حَدَّيْنِ طُولُهُ ذِرَاعٌ, وَتَقَلَّدَهُ تَحْتَ ثِيَابِهِ عَلَى فَخِْذِهِ الْيُمْنَى.
\par 17 وَقَدَمَّ الْهَدِيَّةَ لِعِجْلُونَ مَلِكِ مُوآبَ. (وَكَانَ عِجْلُونُ رَجُلاً سَمِيناً جِدّاً).
\par 18 وَكَانَ لَمَّا انْتَهَى مِنْ تَقْدِيمِ الْهَدِيَّةِ صَرَفَ الْقَوْمَ حَامِلِي الْهَدِيَّةِ,
\par 19 وَأَمَّا هُوَ فَرَجَعَ مِنْ عِنْدِ الْمَنْحُوتَاتِ الَّتِي لَدَى الْجِلْجَالِ وَقَالَ: «لِي كَلاَمُ سِرٍّ إِلَيْكَ أَيُّهَا الْمَلِكُ». فَقَالَ: «اسْكُتْ». وَخَرَجَ مِنْ عِنْدِهِ جَمِيعُ الْوَاقِفِينَ لَدَيْهِ.
\par 20 فَدَخَلَ إِلَيْهِ إِهُودُ وَهُوَ جَالِسٌ فِي غُرْفَةٍ صَيْفِيَّةٍ كَانَتْ لَهُ وَحْدَهُ. وَقَالَ إِهُودُ: «عِنْدِي كَلاَمُ اللَّهِ إِلَيْكَ». فَقَامَ عَنِ الْكُرْسِيِّ.
\par 21 فَمَدَّ إِهُودُ يَدَهُ الْيُسْرَى وَأَخَذَ السَّيْفَ عَنْ فَخْذِهِ الْيُمْنَى وَضَرَبَهُ فِي بَطْنِهِ.
\par 22 فَدَخَلَ الْمِقْبَضُ أَيْضاً وَرَاءَ النَّصْلِ, وَطَبَقَ الشَّحْمُ وَرَاءَ النَّصْلِ لأَنَّهُ لَمْ يَجْذُبِ السَّيْفَ مِنْ بَطْنِهِ. وَخَرَجَ مِنَ الْحِتَارِ.
\par 23 فَخَرَجَ إِهُودُ مِنَ الرِّوَاقِ وَأَغْلَقَ أَبْوَابَ الْعِلِّيَّةِ وَرَاءَهُ وَأَقْفَلَهَا.
\par 24 وَلَمَّا خَرَجَ, جَاءَ عَبِيدُهُ وَنَظَرُوا وَإِذَا أَبْوَابُ الْعِلِّيَّةِ مُقْفَلَةٌ, فَقَالُوا: «إِنَّهُ مُغَطٍّ رِجْلَيْهِ فِي الْغُرْفَةِ الصَّيْفِيَّةِ».
\par 25 فَلَبِثُوا حَتَّى خَجِلُوا وَإِذَا هُوَ لاَ يَفْتَحُ أَبْوَابَ الْعِلِّيَّةِ. فَأَخَذُوا الْمِفْتَاحَ وَفَتَحُوا وَإِذَا سَيِّدُهُمْ سَاقِطٌ عَلَى الأَرْضِ مَيِّتاً.
\par 26 وَأَمَّا إِهُودُ فَنَجَا إِذْ هُمْ مَبْهُوتُونَ, وَعَبَرَ الْمَنْحُوتَاتِ وَنَجَا إِلَى سَعِيرَةَ.
\par 27 وَكَانَ عِنْدَ مَجِيئِهِ أَنَّهُ ضَرَبَ بِالْبُوقِ فِي جَبَلِ أَفْرَايِمَ, فَنَزَلَ مَعَهُ بَنُو إِسْرَائِيلَ عَنِ الْجَبَلِ وَهُوَ قُدَّامَهُمْ.
\par 28 وَقَالَ لَهُمُ: «اتْبَعُونِي لأَنَّ الرَّبَّ قَدْ دَفَعَ أَعْدَاءَكُمُ الْمُوآبِيِّينَ لِيَدِكُمْ». فَنَزَلُوا وَرَاءَهُ وَأَخَذُوا مَخَاوِضَ الأُرْدُنِّ إِلَى مُوآبَ, وَلَمْ يَدَعُوا أَحَداً يَعْبُرُ.
\par 29 فَضَرَبُوا مِنْ مُوآبَ فِي ذَلِكَ الْوَقْتِ نَحْوَ عَشَرَةِ آلاَفِ رَجُلٍ, كُلَّ نَشِيطٍ وَكُلَّ ذِي بَأْسٍ, وَلَمْ يَنْجُ أَحَدٌ.
\par 30 فَذَلَّ الْمُوآبِيُّونَ فِي ذَلِكَ الْيَوْمِ تَحْتَ يَدِ إِسْرَائِيلَ. وَاسْتَرَاحَتِ الأَرْضُ ثَمَانِينَ سَنَةً.
\par 31 وَكَانَ بَعْدَهُ شَمْجَرُ بْنُ عَنَاةَ, فَضَرَبَ مِنَ الْفِلِسْطِينِيِّينَ سِتَّ مِئَةِ رَجُلٍ بِمِنْخَسِ الْبَقَرِ. وَهُوَ أَيْضاً خَلَّصَ إِسْرَائِيلَ.

\chapter{4}

\par 1 وَعَادَ بَنُو إِسْرَائِيلَ يَعْمَلُونَ الشَّرَّ فِي عَيْنَيِ الرَّبِّ بَعْدَ مَوْتِ إِهُودَ,
\par 2 فَبَاعَهُمُ الرَّبُّ بِيَدِ يَابِينَ مَلِكِ كَنْعَانَ الَّذِي مَلَكَ فِي حَاصُورَ. وَرَئِيسُ جَيْشِهِ سِيسَرَا. وَهُوَ سَاكِنٌ فِي حَرُوشَةِ الأُمَمِ.
\par 3 فَصَرَخَ بَنُو إِسْرَائِيلَ إِلَى الرَّبِّ, لأَنَّهُ كَانَ لَهُ تِسْعُ مِئَةِ مَرْكَبَةٍ مِنْ حَدِيدٍ, وَهُوَ ضَايَقَ بَنِي إِسْرَائِيلَ بِشِدَّةٍ, عِشْرِينَ سَنَةً.
\par 4 وَدَبُورَةُ امْرَأَةٌ نَبِيَّةٌ زَوْجَةُ لَفِيدُوتَ, هِيَ قَاضِيَةُ إِسْرَائِيلَ فِي ذَلِكَ الْوَقْتِ.
\par 5 وَهِيَ جَالِسَةٌ تَحْتَ نَخْلَةِ دَبُورَةَ بَيْنَ الرَّامَةِ وَبَيْتِ إِيلَ فِي جَبَلِ أَفْرَايِمَ. وَكَانَ بَنُو إِسْرَائِيلَ يَصْعَدُونَ إِلَيْهَا لِلْقَضَاءِ.
\par 6 فَأَرْسَلَتْ وَدَعَتْ بَارَاقَ بْنَ أَبِينُوعَمَ مِنْ قَادِشِ نَفْتَالِي, وَقَالَتْ لَهُ: «أَلَمْ يَأْمُرِ الرَّبُّ إِلَهُ إِسْرَائِيلَ: اذْهَبْ وَازْحَفْ إِلَى جَبَلِ تَابُورَ, وَخُذْ مَعَكَ عَشَرَةَ آلاَفِ رَجُلٍ مِنْ بَنِي نَفْتَالِي وَمِنْ بَنِي زَبُولُونَ,
\par 7 فَأَجْذِبَ إِلَيْكَ إِلَى نَهْرِ قِيشُونَ سِيسَرَا رَئِيسَ جَيْشِ يَابِينَ بِمَرْكَبَاتِهِ وَجُمْهُورِهِ وَأَدْفَعَهُ لِيَدِكَ؟»
\par 8 فَقَالَ لَهَا بَارَاقُ: «إِنْ ذَهَبْتِ مَعِي أَذْهَبْ, وَإِنْ لَمْ تَذْهَبِي مَعِي فَلاَ أَذْهَبُ».
\par 9 فَقَالَتْ: «إِنِّي أَذْهَبُ مَعَكَ, غَيْرَ أَنَّهُ لاَ يَكُونُ لَكَ فَخْرٌ فِي الطَّرِيقِ الَّتِي أَنْتَ سَائِرٌ فِيهَا. لأَنَّ الرَّبَّ يَبِيعُ سِيسَرَا بِيَدِ امْرَأَةٍ». فَقَامَتْ دَبُورَةُ وَذَهَبَتْ مَعَ بَارَاقَ إِلَى قَادِشَ.
\par 10 وَدَعَا بَارَاقُ زَبُولُونَ وَنَفْتَالِيَ إِلَى قَادِشَ, وَصَعِدَ وَمَعَهُ عَشَرَةُ آلاَفِ رَجُلٍ. وَصَعِدَتْ دَبُورَةُ مَعَهُ.
\par 11 وَحَابِرُ الْقِينِيُّ انْفَرَدَ مِنْ قَايِنَ مِنْ بَنِي حُوبَابَ حَمِي مُوسَى وَخَيَّمَ حَتَّى إِلَى بَلُّوطَةٍ فِي صَعْنَايِمَ الَّتِي عِنْدَ قَادِشَ.
\par 12 وَأَخْبَرُوا سِيسَرَا بِأَنَّهُ قَدْ صَعِدَ بَارَاقُ بْنُ أَبِينُوعَمَ إِلَى جَبَلِ تَابُورَ.
\par 13 فَدَعَا سِيسَرَا جَمِيعَ مَرْكَبَاتِهِ, تِسْعَ مِئَةِ مَرْكَبَةٍ مِنْ حَدِيدٍ, وَجَمِيعَ الشَّعْبِ الَّذِي مَعَهُ مِنْ حَرُوشَةِ الأُمَمِ إِلَى نَهْرِ قِيشُونَ.
\par 14 فَقَالَتْ دَبُورَةُ لِبَارَاقَ: «قُمْ, لأَنَّ هَذَا هُوَ الْيَوْمُ الَّذِي دَفَعَ فِيهِ الرَّبُّ سِيسَرَا لِيَدِكَ. أَلَمْ يَخْرُجِ الرَّبُّ قُدَّامَكَ؟» فَنَزَلَ بَارَاقُ مِنْ جَبَلِ تَابُورَ وَوَرَاءَهُ عَشَرَةُ آلاَفِ رَجُلٍ.
\par 15 فَأَزْعَجَ الرَّبُّ سِيسَرَا وَكُلَّ الْمَرْكَبَاتِ وَكُلَّ الْجَيْشِ بِحَدِّ السَّيْفِ أَمَامَ بَارَاقَ. فَنَزَلَ سِيسَرَا عَنِ الْمَرْكَبَةِ وَهَرَبَ عَلَى رِجْلَيْهِ.
\par 16 وَتَبِعَ بَارَاقُ الْمَرْكَبَاتِ وَالْجَيْشَ إِلَى حَرُوشَةِ الأُمَمِ. وَسَقَطَ كُلُّ جَيْشِ سِيسَرَا بِحَدِّ السَّيْفِ. لَمْ يَبْقَ وَلاَ وَاحِدٌ.
\par 17 وَأَمَّا سِيسَرَا فَهَرَبَ عَلَى رِجْلَيْهِ إِلَى خَيْمَةِ يَاعِيلَ امْرَأَةِ حَابِرَ الْقِينِيِّ, لأَنَّهُ كَانَ صُلْحٌ بَيْنَ يَابِينَ مَلِكِ حَاصُورَ وَبَيْتِ حَابِرَ الْقِينِيِّ.
\par 18 فَخَرَجَتْ يَاعِيلُ لاِسْتِقْبَالِ سِيسَرَا وَقَالَتْ لَهُ: «مِلْ يَا سَيِّدِي, مِلْ إِلَيَّ. لاَ تَخَفْ». فَمَالَ إِلَيْهَا إِلَى الْخَيْمَةِ وَغَطَّتْهُ بِاللِّحَافِ.
\par 19 فَقَالَ لَهَا: «اسْقِينِي قَلِيلَ مَاءٍ لأَنِّي قَدْ عَطِشْتُ». فَفَتَحَتْ قِرْبَةَ اللَّبَنِ وَأَسْقَتْهُ ثُمَّ غَطَّتْهُ.
\par 20 فَقَالَ لَهَا: «قِفِي بِبَابِ الْخَيْمَةِ, وَيَكُونُ إِذَا جَاءَ أَحَدٌ وَسَأَلَكِ: أَهُنَا رَجُلٌ؟ أَنَّكِ تَقُولِينَ لاَ.
\par 21 فَأَخَذَتْ يَاعِيلُ امْرَأَةُ حَابِرَ وَتَدَ الْخَيْمَةِ وَالْمِطْرَقَةَ فِي يَدِهَا, وَسَارَتْ إِلَيْهِ بِهُدُوءٍ وَضَرَبَتِ الْوَتَدَ فِي صُدْغِهِ فَنَفَذَ إِلَى الأَرْضِ وَهُوَ مُتَثَقِّلٌ فِي النَّوْمِ وَمُتْعَبٌ فَمَاتَ.
\par 22 وَإِذَا بِبَارَاقَ يُطَارِدُ سِيسَرَا, فَخَرَجَتْ يَاعِيلُ لاِسْتِقْبَالِهِ وَقَالَتْ لَهُ: «تَعَالَ فَأُرِيَكَ الرَّجُلَ الَّذِي أَنْتَ طَالِبُهُ». فَجَاءَ إِلَيْهَا وَإِذَا سِيسَرَا سَاقِطٌ مَيِّتاً وَالْوَتَدُ فِي صُدْغِهِ.
\par 23 فَأَذَلَّ اللَّهُ فِي ذَلِكَ الْيَوْمِ يَابِينَ مَلِكَ كَنْعَانَ أَمَامَ بَنِي إِسْرَائِيلَ.
\par 24 وَأَخَذَتْ يَدُ بَنِي إِسْرَائِيلَ تَتَزَايَدُ وَتَقْسُو عَلَى يَابِينَ مَلِكِ كَنْعَانَ حَتَّى قَرَضُوا يَابِينَ مَلِكَ كَنْعَانَ.

\chapter{5}

\par 1 فَتَرَنَّمَتْ دَبُورَةُ وَبَارَاقُ بْنُ أَبِينُوعَمَ فِي ذَلِكَ الْيَوْمِ قَائِلَيْنِ:
\par 2 «لأَجْلِ قِيَادَةِ الْقُّوَادِ فِي إِسْرَائِيلَ, لأَجْلِ طَاعَةِ الشَّعْبِ, بَارِكُوا الرَّبَّ.
\par 3 اِسْمَعُوا أَيُّهَا الْمُلُوكُ وَاصْغُوا أَيُّهَا الْعُظَمَاءُ. أَنَا, أَنَا لِلرَّبِّ أَتَرَنَّمُ. أُزَمِّرُ لِلرَّبِّ إِلَهِ إِسْرَائِيلَ.
\par 4 يَا رَبُّ بِخُرُوجِكَ مِنْ سَعِيرَ, بِصُعُودِكَ مِنْ صَحْرَاءِ أَدُومَ, الأَرْضُ ارْتَعَدَتِ. السَّمَاوَاتُ أَيْضاً قَطَرَتْ. كَذَلِكَ السُّحُبُ قَطَرَتْ مَاءً.
\par 5 تَزَلْزَلَتِ الْجِبَالُ مِنْ وَجْهِ الرَّبِّ, وَسِينَاءُ هَذَا مِنْ وَجْهِ الرَّبِّ إِلَهِ إِسْرَائِيلَ.
\par 6 «فِي أَيَّامِ شَمْجَرَ بْنِ عَنَاةَ, فِي أَيَّامِ يَاعِيلَ, اسْتَرَاحَتِ الطُّرُقُ, وَعَابِرُو السُّبُلِ سَارُوا فِي مَسَالِكَ مُعْوَجَّةٍ.
\par 7 خُذِلَ الْحُكَّامُ فِي إِسْرَائِيلَ. خُذِلُوا حَتَّى قُمْتُ أَنَا دَبُورَةُ. قُمْتُ أُمّاً فِي إِسْرَائِيلَ.
\par 8 اِخْتَارَ آلِهَةً حَدِيثَةً. حِينَئِذٍ حَرْبُ الأَبْوَابِ. هَلْ كَانَ يُرَى مِجَنٌّ أَوْ رُمْحٌ فِي أَرْبَعِينَ أَلْفاً مِنْ إِسْرَائِيلَ؟
\par 9 قَلْبِي نَحْوَ قُضَاةِ إِسْرَائِيلَ الْمُتَطَّوِعِينَ فِي الشَّعْبِ. بَارِكُوا الرَّبَّ.
\par 10 أَيُّهَا الرَّاكِبُونَ الأُتُنَ الصُّحْرَ, الْجَالِسُونَ عَلَى طَنَافِسَ, وَالسَّالِكُونَ فِي الطَّرِيقِ, سَبِّحُوا!
\par 11 مِنْ صَوْتِ الْمُحَاصِّينَ بَيْنَ الأَحْوَاضِ هُنَاكَ يُثْنُونَ عَلَى حَقِّ الرَّبِّ حَقِّ حُكَّامِهِ فِي إِسْرَائِيلَ. حِينَئِذٍ نَزَلَ شَعْبُ الرَّبِّ إِلَى الأَبْوَابِ.
\par 12 «اِسْتَيْقِظِي اسْتَيْقِظِي يَا دَبُورَةُ! اسْتَيْقِظِي اسْتَيْقِظِي وَتَكَلَّمِي بِنَشِيدٍ! قُمْ يَا بَارَاقُ وَاسْبِ سَبْيَكَ, يَا ابْنَ أَبِينُوعَمَ!
\par 13 حِينَئِذٍ تَسَلَّطَ الشَّارِدُ عَلَى عُظَمَاءِ الشَّعْبِ. الرَّبُّ سَلَّطَنِي عَلَى الْجَبَابِرَةِ.
\par 14 جَاءَ مِنْ أَفْرَايِمَ الَّذِينَ مَقَرُّهُمْ بَيْنَ عَمَالِيقَ, وَبَعْدَكَ بِنْيَامِينُ مَعَ قَوْمِكَ. مِنْ مَاكِيرَ نَزَلَ قُضَاةٌ, وَمِنْ زَبُولُونَ مَاسِكُونَ بِقَضِيبِ الْقَائِدِ.
\par 15 وَالرُّؤَسَاءُ فِي يَسَّاكَرَ مَعَ دَبُورَةَ. وَكَمَا يَسَّاكَرُ هَكَذَا بَارَاقُ. انْدَفَعَ إِلَى الْوَادِي وَرَاءَهُ. عَلَى مَسَاقِي رَأُوبَيْنَ أَقْضِيَةُ قَلْبٍ عَظِيمَةٌ.
\par 16 لِمَاذَا أَقَمْتَ بَيْنَ الْحَظَائِرِ لِسَمْعِ الصَّفِيرِ لِلْقُطْعَانِ. لَدَى مَسَاقِي رَأُوبَيْنَ مَبَاحِثُ قَلْبٍ عَظِيمَةٌ.
\par 17 جِلْعَادُ فِي عَبْرِ الأُرْدُنِّ سَكَنَ. وَدَانُ, لِمَاذَا اسْتَوْطَنَ لَدَى السُّفُنِ؟ وَأَشِيرُ أَقَامَ عَلَى سَاحِلِ الْبَحْرِ, وَفِي شَاطِئِهِ سَكَنَ.
\par 18 زَبُولُونُ شَعْبٌ أَهَانَ نَفْسَهُ إِلَى الْمَوْتِ مَعَ نَفْتَالِي عَلَى رَوَابِي الْحَقْلِ.
\par 19 «جَاءَ مُلُوكٌ. حَارَبُوا. حِينَئِذٍ حَارَبَ مُلُوكُ كَنْعَانَ فِي تَعْنَكَ عَلَى مِيَاهِ مَجِدُّو. بِضْعَ فِضَّةٍ لَمْ يَأْخُذُوا.
\par 20 مِنَ السَّمَاوَاتِ حَارَبُوا. الْكَوَاكِبُ مِنْ أَفْلاَكِهَا حَارَبَتْ سِيسَرَا.
\par 21 نَهْرُ قِيشُونَ جَرَفَهُمْ. نَهْرُ وَقَائِعَ نَهْرُ قِيشُونَ. دُوسِي يَا نَفْسِي بِعِّزٍ.
\par 22 «حِينَئِذٍ ضَرَبَتْ أَعْقَابُ الْخَيْلِ مِنَ السَّوْقِ, سَوْقِ أَقْوِيَائِهِ.
\par 23 اِلْعَنُوا مِيرُوزَ قَالَ مَلاَكُ الرَّبِّ. الْعَنُوا سَاكِنِيهَا لَعْناً, لأَنَّهُمْ لَمْ يَأْتُوا لِمَعُونَةِ الرَّبِّ, مَعُونَةِ الرَّبِّ بَيْنَ الْجَبَابِرَةِ.
\par 24 تُبَارَكُ عَلَى النِّسَاءِ يَاعِيلُ امْرَأَةُ حَابِرَ الْقِينِيِّ. عَلَى النِّسَاءِ فِي الْخِيَامِ تُبَارَكُ.
\par 25 طَلَبَ مَاءً فَأَعْطَتْهُ لَبَناً. فِي قَصْعَةِ الْعُظَمَاءِ قَدَّمَتْ زُبْدَةً.
\par 26 مَدَّتْ يَدَهَا إِلَى الْوَتَدِ وَيَمِينَهَا إِلَى مِضْرَابِ الْعَمَلَةِ, وَضَرَبَتْ سِيسَرَا وَسَحَقَتْ رَأْسَهُ, شَدَّخَتْ وَخَرَّقَتْ صُدْغَهُ.
\par 27 بَيْنَ رِجْلَيْهَا انْطَرَحَ, سَقَطَ اضْطَجَعَ. بَيْنَ رِجْلَيْهَا انْطَرَحَ سَقَطَ. حَيْثُ انْطَرَحَ فَهُنَاكَ سَقَطَ مَقْتُولاً.
\par 28 مِنَ الْكُّوَةِ أَشْرَفَتْ وَوَلْوَلَتْ أُمُّ سِيسَرَا مِنَ الشُّبَّاكِ: لِمَاذَا أَبْطَأَتْ مَرْكَبَاتُهُ عَنِ الْمَجِيءِ؟ لِمَاذَا تَأَخَّرَتْ خَطَوَاتُ مَرَاكِبِهِ؟
\par 29 فَأَجَابَتْهَا أَحْكَمُ سَيِّدَاتِهَا, بَلْ هِيَ رَدَّتْ جَوَاباً لِنَفْسِهَا:
\par 30 أَلَمْ يَجِدُوا وَيَقْسِمُوا الْغَنِيمَةَ! فَتَاةً أَوْ فَتَاتَيْنِ لِكُلِّ رَجُلٍ! غَنِيمَةَ ثِيَابٍ مَصْبُوغَةٍ لِسِيسَرَا! غَنِيمَةَ ثِيَابٍ مَصْبُوغَةٍ مُطَرَّزَةٍ! ثِيَابٍ مَصْبُوغَةٍ مُطَرَّزَةِ الْوَجْهَيْنِ غَنِيمَةً لِعُنُقِي!
\par 31 هَكَذَا يَبِيدُ جَمِيعُ أَعْدَائِكَ يَا رَبُّ. وَأَحِبَّاؤُهُ كَخُرُوجِ الشَّمْسِ فِي جَبَرُوتِهَا». وَاسْتَرَاحَتِ الأَرْضُ أَرْبَعِينَ سَنَةً.

\chapter{6}

\par 1 وَعَمِلَ بَنُو إِسْرَائِيلَ الشَّرَّ فِي عَيْنَيِ الرَّبِّ, فَدَفَعَهُمُ الرَّبُّ لِيَدِ مِدْيَانَ سَبْعَ سِنِينَ.
\par 2 فَاعْتَّزَتْ يَدُ مِدْيَانَ عَلَى إِسْرَائِيلَ. بِسَبَبِ الْمِدْيَانِيِّينَ عَمِلَ بَنُو إِسْرَائِيلَ لأَنْفُسِهِمِ الْكُهُوفَ الَّتِي فِي الْجِبَالِ وَالْمَغَايِرَ وَالْحُصُونَ.
\par 3 وَإِذَا زَرَعَ إِسْرَائِيلُ كَانَ الْمِدْيَانِيُّونَ وَالْعَمَالِقَةُ وَبَنُو الْمَشْرِقِ يَصْعَدُونَ عَلَيْهِمْ
\par 4 وَيَنْزِلُونَ عَلَيْهِمْ وَيُتْلِفُونَ غَلَّةَ الأَرْضِ إِلَى مَجِيئِكَ إِلَى غَّزَةَ, وَلاَ يَتْرُكُونَ لإِسْرَائِيلَ قُوتَ الْحَيَاةِ, وَلاَ غَنَماً وَلاَ بَقَراً وَلاَ حَمِيراً.
\par 5 لأَنَّهُمْ كَانُوا يَصْعَدُونَ بِمَوَاشِيهِمْ وَخِيَامِهِمْ وَيَجِيئُونَ كَالْجَرَادِ فِي الْكَثْرَةِ وَلَيْسَ لَهُمْ وَلِجِمَالِهِمْ عَدَدٌ, وَدَخَلُوا الأَرْضَ لِيُخْرِبُوهَا.
\par 6 فَذَلَّ إِسْرَائِيلُ جِدّاً مِنْ قِبَلِ الْمِدْيَانِيِّينَ. وَصَرَخَ بَنُو إِسْرَائِيلَ إِلَى الرَّبِّ.
\par 7 وَكَانَ لَمَّا صَرَخَ بَنُو إِسْرَائِيلَ إِلَى الرَّبِّ بِسَبَبِ الْمِدْيَانِيِّينَ
\par 8 أَنَّ الرَّبَّ أَرْسَلَ نَبِيّاً إِلَى بَنِي إِسْرَائِيلَ, فَقَالَ لَهُمْ: «هَكَذَا قَالَ الرَّبُّ إِلَهُ إِسْرَائِيلَ: إِنِّي قَدْ أَصْعَدْتُكُمْ مِنْ مِصْرَ وَأَخْرَجْتُكُمْ مِنْ بَيْتِ الْعُبُودِيَّةِ
\par 9 وَأَنْقَذْتُكُمْ مِنْ يَدِ الْمِصْرِيِّينَ وَمِنْ يَدِ جَمِيعِ مُضَايِقِيكُمْ, وَطَرَدْتُهُمْ مِنْ أَمَامِكُمْ وَأَعْطَيْتُكُمْ أَرْضَهُمْ.
\par 10 وَقُلْتُ لَكُمْ: أَنَا الرَّبُّ إِلَهُكُمْ. لاَ تَخَافُوا آلِهَةَ الأَمُورِيِّينَ الَّذِينَ أَنْتُمْ سَاكِنُونَ أَرْضَهُمْ. وَلَمْ تَسْمَعُوا لِصَوْتِي».
\par 11 وَأَتَى مَلاَكُ الرَّبِّ وَجَلَسَ تَحْتَ الْبُطْمَةِ الَّتِي فِي عَفْرَةَ الَّتِي لِيُوآشَ الأَبِيعَزَرِيِّ. وَابْنُهُ جِدْعُونُ كَانَ يَخْبِطُ حِنْطَةً فِي الْمِعْصَرَةِ لِيُهَرِّبَهَا مِنَ الْمِدْيَانِيِّينَ.
\par 12 فَظَهَرَ لَهُ مَلاَكُ الرَّبِّ وَقَالَ لَهُ: «الرَّبُّ مَعَكَ يَا جَبَّارَ الْبَأْسِ!»
\par 13 فَقَالَ لَهُ جِدْعُونُ: «أَسْأَلُكَ يَا سَيِّدِي, إِذَا كَانَ الرَّبُّ مَعَنَا فَلِمَاذَا أَصَابَتْنَا كُلُّ هَذِهِ, وَأَيْنَ كُلُّ عَجَائِبِهِ الَّتِي أَخْبَرَنَا بِهَا آبَاؤُنَا قَائِلِينَ: أَلَمْ يُصْعِدْنَا الرَّبُّ مِنْ مِصْرَ؟ وَالآنَ قَدْ رَفَضَنَا الرَّبُّ وَجَعَلَنَا فِي كَفِّ مِدْيَانَ».
\par 14 فَالْتَفَتَ إِلَيْهِ الرَّبُّ وَقَالَ: «اذْهَبْ بِقُّوَتِكَ هَذِهِ وَخَلِّصْ إِسْرَائِيلَ مِنْ كَفِّ مِدْيَانَ. أَمَا أَرْسَلْتُكَ؟»
\par 15 فَقَالَ لَهُ: «أَسْأَلُكَ يَا سَيِّدِي, بِمَاذَا أُخَلِّصُ إِسْرَائِيلَ؟ هَا عَشِيرَتِي هِيَ الذُّلَّى فِي مَنَسَّى, وَأَنَا الأَصْغَرُ فِي بَيْتِ أَبِي».
\par 16 فَقَالَ لَهُ الرَّبُّ: «إِنِّي أَكُونُ مَعَكَ, وَسَتَضْرِبُ الْمِدْيَانِيِّينَ كَرَجُلٍ وَاحِدٍ».
\par 17 فَقَالَ لَهُ: «إِنْ كُنْتُ قَدْ وَجَدْتُ نِعْمَةً فِي عَيْنَيْكَ فَاصْنَعْ لِي عَلاَمَةً أَنَّكَ أَنْتَ تُكَلِّمُنِي.
\par 18 لاَ تَبْرَحْ مِنْ هَهُنَا حَتَّى آتِيَ إِلَيْكَ وَأُخْرِجَ تَقْدِمَتِي وَأَضَعَهَا أَمَامَكَ». فَقَالَ: «إِنِّي أَبْقَى حَتَّى تَرْجِعَ».
\par 19 فَدَخَلَ جِدْعُونُ وَعَمِلَ جَدْيَ مِعْزىً وَإِيفَةَ دَقِيقٍ فَطِيراً. أَمَّا اللَّحْمُ فَوَضَعَهُ فِي سَلٍّ, وَأَمَّا الْمَرَقُ فَوَضَعَهُ فِي قِدْرٍ وَخَرَجَ بِهَا إِلَيْهِ إِلَى تَحْتِ الْبُطْمَةِ وَقَدَّمَهَا.
\par 20 فَقَالَ لَهُ مَلاَكُ اللَّهِ: «خُذِ اللَّحْمَ وَالْفَطِيرَ وَضَعْهُمَا عَلَى تِلْكَ الصَّخْرَةِ وَاسْكُبِ الْمَرَقَ». فَفَعَلَ كَذَلِكَ.
\par 21 فَمَدَّ مَلاَكُ الرَّبِّ طَرَفَ الْعُكَّازِ الَّذِي بِيَدِهِ وَمَسَّ اللَّحْمَ وَالْفَطِيرَ, فَصَعِدَتْ نَارٌ مِنَ الصَّخْرَةِ وَأَكَلَتِ اللَّحْمَ وَالْفَطِيرَ. وَذَهَبَ مَلاَكُ الرَّبِّ عَنْ عَيْنَيْهِ.
\par 22 فَرَأَى جِدْعُونُ أَنَّهُ مَلاَكُ الرَّبِّ, فَقَالَ جِدْعُونُ: «آهِ يَا سَيِّدِي الرَّبَّ! لأَنِّي قَدْ رَأَيْتُ مَلاَكَ الرَّبِّ وَجْهاً لِوَجْهٍ!»
\par 23 فَقَالَ لَهُ الرَّبُّ: «السَّلاَمُ لَكَ. لاَ تَخَفْ. لاَ تَمُوتُ».
\par 24 فَبَنَى جِدْعُونُ هُنَاكَ مَذْبَحاً لِلرَّبِّ وَدَعَاهُ «يَهْوَهَ شَلُومَ». إِلَى هَذَا الْيَوْمِ لَمْ يَزَلْ فِي عَفْرَةِ الأَبِيعَزَرِيِّينَ.
\par 25 وَكَانَ فِي تِلْكَ اللَّيْلَةِ أَنَّ الرَّبَّ قَالَ لَهُ: «خُذْ ثَوْرَ الْبَقَرِ الَّذِي لأَبِيكَ, وَثَوْراً ثَانِياً ابْنَ سَبْعِ سِنِينَ, وَاهْدِمْ مَذْبَحَ الْبَعْلِ الَّذِي لأَبِيكَ وَاقْطَعِ السَّارِيَةَ الَّتِي عِنْدَهُ,
\par 26 وَابْنِ مَذْبَحاً لِلرَّبِّ إِلَهِكَ عَلَى رَأْسِ هَذَا الْحِصْنِ بِتَرْتِيبٍ, وَخُذِ الثَّوْرَ الثَّانِيَ وَأَصْعِدْ مُحْرَقَةً عَلَى حَطَبِ السَّارِيَةِ الَّتِي تَقْطَعُهَا.
\par 27 فَأَخَذَ جِدْعُونُ عَشَرَةَ رِجَالٍ مِنْ عَبِيدِهِ وَعَمِلَ كَمَا كَلَّمَهُ الرَّبُّ. وَإِذْ كَانَ يَخَافُ مِنْ بَيْتِ أَبِيهِ وَأَهْلِ الْمَدِينَةِ أَنْ يَعْمَلَ ذَلِكَ نَهَاراً فَعَمِلَهُ لَيْلاً.
\par 28 فَبَكَّرَ أَهْلُ الْمَدِينَةِ فِي الْغَدِ وَإِذَا بِمَذْبَحِ الْبَعْلِ قَدْ هُدِمَ وَالسَّارِيَةُ الَّتِي عِنْدَهُ قَدْ قُطِعَتْ, وَالثَّوْرُ الثَّانِي قَدْ أُصْعِدَ عَلَى الْمَذْبَحِ الَّذِي بُنِيَ.
\par 29 فَقَالُوا الْوَاحِدُ لِصَاحِبِهِ: «مَنْ عَمِلَ هَذَا الأَمْرَ؟» فَسَأَلُوا وَبَحَثُوا فَقَالُوا: «إِنَّ جِدْعُونَ بْنَ يُوآشَ قَدْ فَعَلَ هَذَا الأَمْرَ».
\par 30 فَقَالَ أَهْلُ الْمَدِينَةِ لِيُوآشَ: «أَخْرِجِ ابْنَكَ لِنَقْتُلَهُ, لأَنَّهُ هَدَمَ مَذْبَحَ الْبَعْلِ وَقَطَعَ السَّارِيَةَ الَّتِي عِنْدَهُ».
\par 31 فَقَالَ يُوآشُ لِجَمِيعِ الْقَائِمِينَ عَلَيْهِ: «أَنْتُمْ تُقَاتِلُونَ لِلْبَعْلِ, أَمْ أَنْتُمْ تُخَلِّصُونَهُ؟ مَنْ يُقَاتِلْ لَهُ يُقْتَلْ فِي هَذَا الصَّبَاحِ. إِنْ كَانَ إِلَهاً فَلْيُقَاتِلْ لِنَفْسِهِ لأَنَّ مَذْبَحَهُ قَدْ هُدِمَ».
\par 32 فَدَعَاهُ فِي ذَلِكَ الْيَوْمِ «يَرُبَّعْلَ» قَائِلاً: «لِيُقَاتِلْهُ الْبَعْلُ لأَنَّهُ قَدْ هَدَمَ مَذْبَحَهُ».
\par 33 وَاجْتَمَعَ جَمِيعُ الْمِدْيَانِيِّينَ وَالْعَمَالِقَةِ وَبَنِي الْمَشْرِقِ مَعاً وَعَبَرُوا وَنَزَلُوا فِي وَادِي يَزْرَعِيلَ.
\par 34 وَلَبِسَ رُوحُ الرَّبِّ جِدْعُونَ فَضَرَبَ بِالْبُوقِ, فَاجْتَمَعَ أَبِيعَزَرُ وَرَاءَهُ.
\par 35 وَأَرْسَلَ رُسُلاً إِلَى جَمِيعِ مَنَسَّى, فَاجْتَمَعَ هُوَ أَيْضاً وَرَاءَهُ, وَأَرْسَلَ رُسُلاً إِلَى أَشِيرَ وَزَبُولُونَ وَنَفْتَالِي فَصَعِدُوا لِلِقَائِهِمْ.
\par 36 وَقَالَ جِدْعُونُ لِلَّهِ: «إِنْ كُنْتَ تُخَلِّصُ بِيَدِي إِسْرَائِيلَ كَمَا تَكَلَّمْتَ,
\par 37 فَهَا إِنِّي وَاضِعٌ جَّزَةَ الصُّوفِ فِي الْبَيْدَرِ. فَإِنْ كَانَ طَلٌّ عَلَى الْجَّزَةِ وَحْدَهَا, وَجَفَافٌ عَلَى الأَرْضِ كُلِّهَا, عَلِمْتُ أَنَّكَ تُخَلِّصُ بِيَدِي إِسْرَائِيلَ كَمَا تَكَلَّمْتَ».
\par 38 وَكَانَ كَذَلِكَ. فَبَكَّرَ فِي الْغَدِ وَضَغَطَ الْجَّزَةَ وَعَصَرَ طَلاًّ مِنَ الْجَّزَةِ, مِلْءَ قَصْعَةٍ مَاءً.
\par 39 فَقَالَ جِدْعُونُ لِلَّهِ: «لاَ يَحْمَ غَضَبُكَ عَلَيَّ فَأَتَكَلَّمَ هَذِهِ الْمَرَّةَ فَقَطْ. أَمْتَحِنُ هَذِهِ الْمَرَّةَ فَقَطْ بِالْجَّزَةِ. فَلْيَكُنْ جَفَافٌ فِي الْجَّزَةِ وَحْدَهَا وَعَلَى كُلِّ الأَرْضِ لِيَكُنْ طَلٌّ».
\par 40 فَفَعَلَ اللَّهُ كَذَلِكَ فِي تِلْكَ اللَّيْلَةِ. فَكَانَ جَفَافٌ فِي الْجَّزَةِ وَحْدَهَا وَعَلَى الأَرْضِ كُلِّهَا كَانَ طَلٌّ.

\chapter{7}

\par 1 فَبَكَّرَ يَرُبَّعْلُ (أَيْ جِدْعُونُ) وَكُلُّ الشَّعْبِ الَّذِي مَعَهُ وَنَزَلُوا عَلَى عَيْنِ حَرُودَ. وَكَانَ جَيْشُ الْمِدْيَانِيِّينَ شِمَالِيَّهُمْ عِنْدَ تَلِّ مُورَةَ فِي الْوَادِي.
\par 2 وَقَالَ الرَّبُّ لِجِدْعُونَ: «إِنَّ الشَّعْبَ الَّذِي مَعَكَ كَثِيرٌ عَلَيَّ لأَدْفَعَ الْمِدْيَانِيِّينَ بِيَدِهِمْ, لِئَلاَّ يَفْتَخِرَ عَلَيَّ إِسْرَائِيلُ قَائِلاً: يَدِي خَلَّصَتْنِي.
\par 3 وَالآنَ نَادِ فِي آذَانِ الشَّعْبِ: مَنْ كَانَ خَائِفاً وَمُرْتَعِداً فَلْيَرْجِعْ وَيَنْصَرِفْ مِنْ جَبَلِ جِلْعَادَ». فَرَجَعَ مِنَ الشَّعْبِ اثْنَانِ وَعِشْرُونَ أَلْفاً. وَبَقِيَ عَشَرَةُ آلاَفٍ.
\par 4 وَقَالَ الرَّبُّ لِجِدْعُونَ: «لَمْ يَزَلِ الشَّعْبُ كَثِيراً. انْزِلْ بِهِمْ إِلَى الْمَاءِ فَأُنَقِّيَهُمْ لَكَ هُنَاكَ. وَيَكُونُ أَنَّ الَّذِي أَقُولُ لَكَ عَنْهُ: هَذَا يَذْهَبُ مَعَكَ فَهُوَ يَذْهَبُ مَعَكَ. وَكُلُّ مَنْ أَقُولُ لَكَ عَنْهُ: هَذَا لاَ يَذْهَبُ مَعَكَ فَهُوَ لاَ يَذْهَبُ».
\par 5 فَنَزَلَ بِالشَّعْبِ إِلَى الْمَاءِ. وَقَالَ الرَّبُّ لِجِدْعُونَ: «كُلُّ مَنْ يَلَغُ بِلِسَانِهِ مِنَ الْمَاءِ كَمَا يَلَغُ الْكَلْبُ فَأَوْقِفْهُ وَحْدَهُ. وَكَذَا كُلُّ مَنْ جَثَا عَلَى رُكْبَتَيْهِ لِلشُّرْبِ».
\par 6 وَكَانَ عَدَدُ الَّذِينَ وَلَغُوا بِيَدِهِمْ إِلَى فَمِهِمْ ثَلاَثَ مِئَةِ رَجُلٍ. وَأَمَّا بَاقِي الشَّعْبِ جَمِيعاً فَجَثُوا عَلَى رُكَبِهِمْ لِشُرْبِ الْمَاءِ.
\par 7 فَقَالَ الرَّبُّ لِجِدْعُونَ: «بِالثَّلاَثِ مِئَةِ الرَّجُلِ الَّذِينَ وَلَغُوا أُخَلِّصُكُمْ وَأَدْفَعُ الْمِدْيَانِيِّينَ لِيَدِكَ. وَأَمَّا سَائِرُ الشَّعْبِ فَلْيَذْهَبُوا كُلُّ وَاحِدٍ إِلَى مَكَانِهِ».
\par 8 فَأَخَذَ الشَّعْبُ زَاداً بِيَدِهِمْ مَعَ أَبْوَاقِهِمْ. وَأَرْسَلَ سَائِرَ رِجَالِ إِسْرَائِيلَ كُلَّ وَاحِدٍ إِلَى خَيْمَتِهِ, وَأَمْسَكَ الثَّلاَثَ مِئَةِ الرَّجُلِ. وَكَانَتْ مَحَلَّةُ الْمِدْيَانِيِّينَ تَحْتَهُ فِي الْوَادِي.
\par 9 وَكَانَ فِي تِلْكَ اللَّيْلَةِ أَنَّ الرَّبَّ قَالَ لَهُ: «قُمِ انْزِلْ إِلَى الْمَحَلَّةِ, لأَنِّي قَدْ دَفَعْتُهَا إِلَى يَدِكَ.
\par 10 وَإِنْ كُنْتَ خَائِفاً مِنَ النُّزُولِ, فَانْزِلْ أَنْتَ وَفُورَةُ غُلاَمُكَ إِلَى الْمَحَلَّةِ,
\par 11 وَتَسْمَعُ مَا يَتَكَلَّمُونَ بِهِ, وَبَعْدُ تَتَشَدَّدُ يَدَاكَ وَتَنْزِلُ إِلَى الْمَحَلَّةِ». فَنَزَلَ هُوَ وَفُورَةُ غُلاَمُهُ إِلَى آخِرِ الْمُتَجَهِّزِينَ الَّذِينَ فِي الْمَحَلَّةِ.
\par 12 وَكَانَ الْمِدْيَانِيُّونَ وَالْعَمَالِقَةُ وَكُلُّ بَنِي الْمَشْرِقِ حَالِّينَ فِي الْوَادِي كَالْجَرَادِ فِي الْكَثْرَةِ, وَجِمَالُهُمْ لاَ عَدَدَ لَهَا كَالرَّمْلِ الَّذِي عَلَى شَاطِئِ الْبَحْرِ فِي الْكَثْرَةِ.
\par 13 وَجَاءَ جِدْعُونُ فَإِذَا رَجُلٌ يُخَبِّرُ صَاحِبَهُ بِحُلْمٍ وَيَقُولُ: «هُوَذَا قَدْ حَلُمْتُ حُلْماً, وَإِذَا رَغِيفُ خُبْزِ شَعِيرٍ يَتَدَحْرَجُ فِي مَحَلَّةِ الْمِدْيَانِيِّينَ وَجَاءَ إِلَى الْخَيْمَةِ وَضَرَبَهَا فَسَقَطَتْ, وَقَلَبَهَا إِلَى فَوْقٍ فَسَقَطَتِ الْخَيْمَةُ».
\par 14 فَأَجَابَ صَاحِبُهُ: «لَيْسَ ذَلِكَ إِلاَّ سَيْفَ جِدْعُونَ بْنِ يُوآشَ رَجُلِ إِسْرَائِيلَ. قَدْ دَفَعَ اللَّهُ إِلَى يَدِهِ الْمِدْيَانِيِّينَ وَكُلَّ الْجَيْشِ».
\par 15 وَكَانَ لَمَّا سَمِعَ جِدْعُونُ خَبَرَ الْحُلْمِ وَتَفْسِيرَهُ أَنَّهُ سَجَدَ وَرَجَعَ إِلَى مَحَلَّةِ إِسْرَائِيلَ وَقَالَ: «قُومُوا لأَنَّ الرَّبَّ قَدْ دَفَعَ إِلَى يَدِكُمْ جَيْشَ الْمِدْيَانِيِّينَ».
\par 16 وَقَسَمَ الثَّلاَثَ مِئَةِ الرَّجُلِ إِلَى ثَلاَثِ فِرَقٍ, وَجَعَلَ أَبْوَاقاً فِي أَيْدِيهِمْ كُلِّهِمْ, وَجِرَاراً فَارِغَةً وَمَصَابِيحَ فِي وَسَطِ الْجِرَارِ.
\par 17 وَقَالَ لَهُمُ: «انْظُرُوا إِلَيَّ وَافْعَلُوا كَذَلِكَ. وَهَا أَنَا آتٍ إِلَى طَرَفِ الْمَحَلَّةِ, فَيَكُونُ كَمَا أَفْعَلُ أَنَّكُمْ هَكَذَا تَفْعَلُونَ.
\par 18 وَمَتَى ضَرَبْتُ بِالْبُوقِ أَنَا وَكُلُّ الَّذِينَ مَعِي فَاضْرِبُوا أَنْتُمْ أَيْضاً بِالأَبْوَاقِ حَوْلَ كُلِّ الْمَحَلَّةِ, وَقُولُوا: لِلرَّبِّ وَلِجِدْعُونَ».
\par 19 فَجَاءَ جِدْعُونُ وَالْمِئَةُ الرَّجُلِ الَّذِينَ مَعَهُ إِلَى طَرَفِ الْمَحَلَّةِ فِي أَّوَلِ الْهَزِيعِ الأَوْسَطِ, وَكَانُوا إِذْ ذَاكَ قَدْ أَقَامُوا الْحُرَّاسَ, فَضَرَبُوا بِالأَبْوَاقِ وَكَسَّرُوا الْجِرَارَ الَّتِي بِأَيْدِيهِمْ.
\par 20 فَضَرَبَتِ الْفِرَقُ الثَّلاَثُ بِالأَبْوَاقِ وَكَسَّرُوا الْجِرَارَ, وَأَمْسَكُوا الْمَصَابِيحَ بِأَيْدِيهِمِ الْيُسْرَى وَالأَبْوَاقَ بِأَيْدِيهِمِ الْيُمْنَى لِيَضْرِبُوا بِهَا, وَصَرَخُوا: «سَيْفٌ لِلرَّبِّ وَلِجِدْعُونَ».
\par 21 وَوَقَفُوا كُلُّ وَاحِدٍ فِي مَكَانِهِ حَوْلَ الْمَحَلَّةِ. فَرَكَضَ كُلُّ الْجَيْشِ وَصَرَخُوا وَهَرَبُوا.
\par 22 وَضَرَبَ الثَّلاَثُ الْمِئِينَ بِالأَبْوَاقِ, وَجَعَلَ الرَّبُّ سَيْفَ كُلِّ وَاحِدٍ بِصَاحِبِهِ وَبِكُلِّ الْجَيْشِ. فَهَرَبَ الْجَيْشُ إِلَى بَيْتِ شِطَّةَ إِلَى صَرَدَةَ حَتَّى إِلَى حَافَةِ آبَلِ مَحُولَةَ إِلَى طَبَّاةَ.
\par 23 فَاجْتَمَعَ رِجَالُ إِسْرَائِيلَ مِنْ نَفْتَالِي وَمِنْ أَشِيرَ وَمِنْ كُلِّ مَنَسَّى وَتَبِعُوا الْمِدْيَانِيِّينَ.
\par 24 فَأَرْسَلَ جِدْعُونُ رُسُلاً إِلَى كُلِّ جَبَلِ أَفْرَايِمَ قَائِلاً: «انْزِلُوا لِلِقَاءِ الْمِدْيَانِيِّينَ وَخُذُوا مِنْهُمُ الْمِيَاهَ إِلَى بَيْتِ بَارَةَ وَالأُرْدُنِّ». فَاجْتَمَعَ كُلُّ رِجَالِ أَفْرَايِمَ وَأَخَذُوا الْمِيَاهَ إِلَى بَيْتِ بَارَةَ وَالأُرْدُنِّ.
\par 25 وَأَمْسَكُوا أَمِيرَيِ الْمِدْيَانِيِّينَ غُرَاباً وَذِئْباً, وَقَتَلُوا غُرَاباً عَلَى صَخْرَةِ غُرَابٍ, وَأَمَّا ذِئْبٌ فَقَتَلُوهُ فِي مِعْصَرَةِ ذِئْبٍ. وَتَبِعُوا الْمِدْيَانِيِّينَ وَأَتُوا بِرَأْسَيْ غُرَابٍ وَذِئْبٍ إِلَى جِدْعُونَ مِنْ عَبْرِ الأُرْدُنِّ.

\chapter{8}

\par 1 وَقَالَ لَهُ رِجَالُ أَفْرَايِمَ: «مَا هَذَا الأَمْرُ الَّذِي فَعَلْتَ بِنَا, إِذْ لَمْ تَدْعُنَا عِنْدَ ذِهَابِكَ لِمُحَارَبَةِ الْمِدْيَانِيِّينَ؟» وَخَاصَمُوهُ بِشِدَّةٍ.
\par 2 فَقَالَ لَهُمْ: «مَاذَا فَعَلْتُ الآنَ نَظِيرَكُمْ؟ أَلَيْسَ خُصَاصَةُ أَفْرَايِمَ خَيْراً مِنْ قَِطَافِ أَبِيعَزَرَ؟
\par 3 لِيَدِكُمْ دَفَعَ اللَّهُ أَمِيرَيِ الْمِدْيَانِيِّينَ غُرَاباً وَذِئْباً. وَمَاذَا قَدِرْتُ أَنْ أَعْمَلَ نَظِيرَكُمْ؟». حِينَئِذٍ ارْتَخَتْ رُوحُهُمْ عَنْهُ عِنْدَمَا تَكَلَّمَ بِهَذَا الْكَلاَمِ.
\par 4 وَجَاءَ جِدْعُونُ إِلَى الأُرْدُنِّ وَعَبَرَ هُوَ وَالثَّلاَثُ مِئَةِ الرَّجُلِ الَّذِينَ مَعَهُ مُعْيِينَ وَمُطَارِدِينَ.
\par 5 فَقَالَ لأَهْلِ سُكُّوتَ: «أَعْطُوا أَرْغِفَةَ خُبْزٍ لِلْقَوْمِ الَّذِينَ مَعِي لأَنَّهُمْ مُعْيُونَ, وَأَنَا سَاعٍ وَرَاءَ زَبَحَ وَصَلْمُنَّاعَ مَلِكَيْ مِدْيَانَ».
\par 6 فَقَالَ رُؤَسَاءُ سُكُّوتَ: «هَلْ أَيْدِي زَبَحَ وَصَلْمُنَّاعَ بِيَدِكَ الآنَ حَتَّى نُعْطِيَ جُنْدَكَ خُبْزاً؟»
\par 7 فَقَالَ جِدْعُونُ: «لِذَلِكَ عِنْدَمَا يَدْفَعُ الرَّبُّ زَبَحَ وَصَلْمُنَّاعَ بِيَدِي أَدْرُسُ لَحْمَكُمْ مَعَ أَشْوَاكِ الْبَرِّيَّةِ بِالنَّوَارِجِ».
\par 8 وَصَعِدَ مِنْ هُنَاكَ إِلَى فَنُوئِيلَ وَكَلَّمَهُمْ هَكَذَا. فَأَجَابَهُ أَهْلُ فَنُوئِيلَ كَمَا أَجَابَ أَهْلُ سُكُّوتَ,
\par 9 فَقَالَ أَيْضاً لأَهْلِ فَنُوئِيلَ: «عِنْدَ رُجُوعِي بِسَلاَمٍ أَهْدِمُ هَذَا الْبُرْجَ».
\par 10 وَكَانَ زَبَحُ وَصَلْمُنَّاعُ فِي قَرْقَرَ وَجَيْشُهُمَا مَعَهُمَا نَحْوُ خَمْسَةَ عَشَرَ أَلْفاً, كُلُّ الْبَاقِينَ مِنْ جَمِيعِ جَيْشِ بَنِي الْمَشْرِقِ. وَالَّذِينَ سَقَطُوا مِئَةٌ وَعِشْرُونَ أَلْفَ رَجُلٍ مُخْتَرِطِي السَّيْفِ.
\par 11 وَصَعِدَ جِدْعُونُ فِي طَرِيقِ سَاكِنِي الْخِيَامِ شَرْقِيَّ نُوبَحَ وَيُجْبَهَةَ, وَضَرَبَ الْجَيْشَ وَكَانَ الْجَيْشُ مُطْمَئِنّاً.
\par 12 فَهَرَبَ زَبَحُ وَصَلْمُنَّاعُ, فَتَبِعَهُمَا وَأَمْسَكَ مَلِكَيْ مِدْيَانَ زَبَحَ وَصَلْمُنَّاعَ وَأَزْعَجَ كُلَّ الْجَيْشِ.
\par 13 وَرَجَعَ جِدْعُونُ بْنُ يُوآشَ مِنَ الْحَرْبِ مِنْ عِنْدِ عَقَبَةِ حَارَسَ.
\par 14 وَأَمْسَكَ غُلاَماً مِنْ أَهْلِ سُكُّوتَ وَسَأَلَهُ, فَكَتَبَ لَهُ رُؤَسَاءَ سُكُّوتَ وَشُيُوخَهَا, سَبْعَةً وَسَبْعِينَ رَجُلاً.
\par 15 وَدَخَلَ إِلَى أَهْلِ سُكُّوتَ وَقَالَ: «هُوَذَا زَبَحُ وَصَلْمُنَّاعُ اللَّذَانِ عَيَّرْتُمُونِي بِهِمَا قَائِلِينَ: هَلْ أَيْدِي زَبَحَ وَصَلْمُنَّاعَ بِيَدِكَ الآنَ حَتَّى نُعْطِي رِجَالَكَ الْمُعْيِينَ خُبْزاً؟»
\par 16 وَأَخَذَ شُيُوخَ الْمَدِينَةِ وَأَشْوَاكَ الْبَرِّيَّةِ وَالنَّوَارِجَ وَعَلَّمَ بِهَا أَهْلَ سُكُّوتَ.
\par 17 وَهَدَمَ بُرْجَ فَنُوئِيلَ وَقَتَلَ رِجَالَ الْمَدِينَةِ.
\par 18 وَقَالَ لِزَبَحَ وَصَلْمُنَّاعَ: «كَيْفَ الرِّجَالُ الَّذِينَ قَتَلْتُمَاهُمْ فِي تَابُورَ؟» فَقَالاَ: «مَثَلُهُمْ مَثَلُكَ. كُلُّ وَاحِدٍ كَصُورَةِ أَوْلاَدِ مَلِكٍ».
\par 19 فَقَالَ: «هُمْ إِخْوَتِي بَنُو أُمِّي. حَيٌّ هُوَ الرَّبُّ لَوِ اسْتَحْيَيْتُمَاهُمْ لَمَا قَتَلْتُكُمَا!».
\par 20 وَقَالَ لِيَثَرَ بِكْرِهِ: «قُمِ اقْتُلْهُمَا». فَلَمْ يَخْتَرِطِ الْغُلاَمُ سَيْفَهُ, لأَنَّهُ خَافَ, بِمَا أَنَّهُ فَتىً بَعْدُ.
\par 21 فَقَالَ زَبَحُ وَصَلْمُنَّاعُ: «قُمْ أَنْتَ وَقَعْ عَلَيْنَا, لأَنَّهُ مِثْلُ الرَّجُلِ بَطْشُهُ». فَقَامَ جِدْعُونُ وَقَتَلَ زَبَحَ وَصَلْمُنَّاعَ, وَأَخَذَ الأَهِلَّةَ الَّتِي فِي أَعْنَاقِ جِمَالِهِمَا.
\par 22 وَقَالَ رِجَالُ إِسْرَائِيلَ لِجِدْعُونَ: «تَسَلَّطْ عَلَيْنَا أَنْتَ وَابْنُكَ وَابْنُ ابْنِكَ, لأَنَّكَ قَدْ خَلَّصْتَنَا مِنْ يَدِ مِدْيَانَ».
\par 23 فَقَالَ لَهُمْ جِدْعُونُ: «لاَ أَتَسَلَّطُ أَنَا عَلَيْكُمْ وَلاَ يَتَسَلَّطُ ابْنِي عَلَيْكُمُ. الرَّبُّ يَتَسَلَّطُ عَلَيْكُمْ».
\par 24 ثُمَّ قَالَ لَهُمْ جِدْعُونُ: «أَطْلُبُ مِنْكُمْ طِلْبَةً: أَنْ تُعْطُونِي كُلُّ وَاحِدٍ أَقْرَاطَ غَنِيمَتِهِ». لأَنَّهُ كَانَ لَهُمْ أَقْرَاطُ ذَهَبٍ لأَنَّهُمْ إِسْمَاعِيلِيُّونَ.
\par 25 فَقَالُوا: «إِنَّنَا نُعْطِي». وَفَرَشُوا رِدَاءً وَطَرَحُوا عَلَيْهِ كُلُّ وَاحِدٍ أَقْرَاطَ غَنِيمَتِهِ.
\par 26 وَكَانَ وَزْنُ أَقْرَاطِ الذَّهَبِ الَّتِي طَلَبَ أَلْفاً وَسَبْعَ مِئَةِ شَاقِلٍ ذَهَباً, مَا عَدَا الأَهِلَّةَ وَالْحَلَقَ وَأَثْوَابَ الأُرْجُوانِ الَّتِي عَلَى مُلُوكِ مِدْيَانَ, وَمَا عَدَا الْقَلاَئِدَ الَّتِي فِي أَعْنَاقِ جِمَالِهِمْ.
\par 27 فَصَنَعَ جِدْعُونُ مِنْهَا أَفُوداً وَجَعَلَهُ فِي مَدِينَتِهِ فِي عَفْرَةَ. وَزَنَى كُلُّ إِسْرَائِيلَ وَرَاءَهُ هُنَاكَ, فَكَانَ ذَلِكَ لِجِدْعُونَ وَبَيْتِهِ فَخّاً.
\par 28 وَذَلَّ مِدْيَانُ أَمَامَ بَنِي إِسْرَائِيلَ وَلَمْ يَعُودُوا يَرْفَعُونَ رُؤُوسَهُمْ. وَاسْتَرَاحَتِ الأَرْضُ أَرْبَعِينَ سَنَةً فِي أَيَّامِ جِدْعُونَ.
\par 29 وَذَهَبَ يَرُبَّعْلُ بْنُ يُوآشَ وَأَقَامَ فِي بَيْتِهِ.
\par 30 وَكَانَ لِجِدْعُونَ سَبْعُونَ وَلَداً خَارِجُونَ مِنْ صُلْبِهِ, لأَنَّهُ كَانَتْ لَهُ نِسَاءٌ كَثِيرَاتٌ.
\par 31 وَسُرِّيَّتُهُ الَّتِي فِي شَكِيمَ وَلَدَتْ لَهُ هِيَ أَيْضاً ابْناً فَسَمَّاهُ أَبِيمَالِكَ.
\par 32 وَمَاتَ جِدْعُونُ بْنُ يُوآشَ بِشَيْبَةٍ صَالِحَةٍ, وَدُفِنَ فِي قَبْرِ يُوآشَ أَبِيهِ فِي عَفْرَةِ أَبِيعَزَرَ.
\par 33 وَكَانَ بَعْدَ مَوْتِ جِدْعُونَ أَنَّ بَنِي إِسْرَائِيلَ رَجَعُوا وَزَنُوا وَرَاءَ الْبَعْلِيمِ, وَجَعَلُوا لَهُمْ بَعَلَ بَرِيثَ إِلَهاً.
\par 34 وَلَمْ يَذْكُرْ بَنُو إِسْرَائِيلَ الرَّبَّ إِلَهَهُمُ الَّذِي أَنْقَذَهُمْ مِنْ يَدِ جَمِيعِ أَعْدَائِهِمْ مِنْ حَوْلِهِمْ.
\par 35 وَلَمْ يَعْمَلُوا مَعْرُوفاً مَعَ بَيْتِ يَرُبَّعْلَ (جِدْعُونَ) نَظِيرَ كُلِّ الْخَيْرِ الَّذِي عَمِلَ مَعَ إِسْرَائِيلَ.

\chapter{9}

\par 1 وَذَهَبَ أَبِيمَالِكُ بْنُ يَرُبَّعْلَ إِلَى شَكِيمَ إِلَى أَخْوَالِهِ, وَقَالَ لِجَمِيعِ عَشِيرَةِ بَيْتِ أَبِي أُمِّهِ:
\par 2 «تَكَلَّمُوا الآنَ فِي آذَانِ جَمِيعِ أَهْلِ شَكِيمَ. أَيُّمَا هُوَ خَيْرٌ لَكُمْ: أَأَنْ يَتَسَلَّطَ عَلَيْكُمْ سَبْعُونَ رَجُلاً, جَمِيعُ بَنِي يَرُبَّعْلَ, أَمْ أَنْ يَتَسَلَّطَ عَلَيْكُمْ رَجُلٌ وَاحِدٌ؟ وَاذْكُرُوا أَنِّي أَنَا عَظْمُكُمْ وَلَحْمُكُمْ».
\par 3 فَتَكَلَّمَ أَخْوَالِهِ عَنْهُ فِي آذَانِ كُلِّ أَهْلِ شَكِيمَ بِجَمِيعِ هَذَا الْكَلاَمِ. فَمَالَ قَلْبُهُمْ وَرَاءَ أَبِيمَالِكَ, لأَنَّهُمْ قَالُوا: «أَخُونَا هُوَ».
\par 4 وَأَعْطُوهُ سَبْعِينَ شَاقِلَ فِضَّةٍ مِنْ بَيْتِ بَعْلِ بَرِيثَ, فَاسْتَأْجَرَ بِهَا أَبِيمَالِكُ رِجَالاً بَطَّالِينَ طَائِشِينَ, فَسَعُوا وَرَاءَهُ.
\par 5 ثُمَّ جَاءَ إِلَى بَيْتِ أَبِيهِ فِي عَفْرَةَ وَقَتَلَ إِخْوَتَهُ بَنِي يَرُبَّعْلَ, سَبْعِينَ رَجُلاً, عَلَى حَجَرٍ وَاحِدٍ. وَبَقِيَ يُوثَامُ بْنُ يَرُبَّعْلَ الأَصْغَرُ لأَنَّهُ اخْتَبَأَ.
\par 6 فَاجْتَمَعَ جَمِيعُ أَهْلِ شَكِيمَ وَكُلُّ سُكَّانِ الْقَلْعَةِ وَذَهَبُوا وَجَعَلُوا أَبِيمَالِكَ مَلِكاً عِنْدَ بَلُّوطَةِ النَّصَبِ الَّذِي فِي شَكِيمَ.
\par 7 وَأَخْبَرُوا يُوثَامَ فَذَهَبَ وَوَقَفَ عَلَى رَأْسِ جَبَلِ جِرِّزِيمَ, وَنَادَى: «اِسْمَعُوا لِي يَا أَهْلَ شَكِيمَ يَسْمَعْ لَكُمُ اللَّهُ.
\par 8 مَرَّةً ذَهَبَتِ الأَشْجَارُ لِتَمْسَحَ عَلَيْهَا مَلِكاً. فَقَالَتْ لِلّزَيْتُونَةِ: امْلِكِي عَلَيْنَا.
\par 9 فَقَالَتْ لَهَا الّزَيْتُونَةُ: أَأَتْرُكُ دُهْنِي الَّذِي بِهِ يُكَرِّمُونَ بِيَ اللَّهَ وَالنَّاسَ, وَأَذْهَبُ لأَمْلِكَ عَلَى الأَشْجَارِ؟
\par 10 ثُمَّ قَالَتِ الأَشْجَارُ لِلتِّينَةِ: تَعَالَيْ أَنْتِ وَامْلِكِي عَلَيْنَا.
\par 11 فَقَالَتْ لَهَا التِّينَةُ: أَأَتْرُكُ حَلاَوَتِي وَثَمَرِي الطَّيِّبَ وَأَذْهَبُ لأَمْلِكَ عَلَى الأَشْجَارِ؟
\par 12 فَقَالَتِ الأَشْجَارُ لِلْكَرْمَةِ: تَعَالَيْ أَنْتِ وَامْلِكِي عَلَيْنَا.
\par 13 فَقَالَتْ لَهَا الْكَرْمَةُ: أَأَتْرُكُ مِسْطَارِي الَّذِي يُفَرِّحُ اللَّهَ وَالنَّاسَ وَأَذْهَبُ لأَمْلِكَ عَلَى الأَشْجَارِ؟
\par 14 ثُمَّ قَالَتْ جَمِيعُ الأَشْجَارِ لِلْعَوْسَجِ: تَعَالَ أَنْتَ وَامْلِكْ عَلَيْنَا.
\par 15 فَقَالَ الْعَوْسَجُ لِلأَشْجَارِ: إِنْ كُنْتُمْ بِالْحَقِّ تَمْسَحُونَنِي عَلَيْكُمْ مَلِكاً فَتَعَالُوا وَاحْتَمُوا تَحْتَ ظِلِّي. وَإِلاَّ فَتَخْرُجَ نَارٌ مِنَ الْعَوْسَجِ وَتَأْكُلَ أَرْزَ لُبْنَانَ!
\par 16 فَالآنَ إِنْ كُنْتُمْ قَدْ عَمِلْتُمْ بِالْحَقِّ وَالصِّحَّةِ إِذْ جَعَلْتُمْ أَبِيمَالِكَ مَلِكاً, وَإِنْ كُنْتُمْ قَدْ فَعَلْتُمْ خَيْراً مَعَ يَرُبَّعْلَ وَمَعَ بَيْتِهِ. وَإِنْ كُنْتُمْ قَدْ فَعَلْتُمْ لَهُ حَسَبَ عَمَلِ يَدَيْهِ -
\par 17 لأَنَّ أَبِي قَدْ حَارَبَ عَنْكُمْ وَخَاطَرَ بِنَفْسِهِ وَأَنْقَذَكُمْ مِنْ يَدِ مِدْيَانَ -
\par 18 وَأَنْتُمْ قَدْ قُمْتُمُ الْيَوْمَ عَلَى بَيْتِ أَبِي وَقَتَلْتُمْ بَنِيهِ, سَبْعِينَ رَجُلاً عَلَى حَجَرٍ وَاحِدٍ وَمَلَّكْتُمْ أَبِيمَالِكَ ابْنَ أَمَتِهِ عَلَى أَهْلِ شَكِيمَ لأَنَّهُ أَخُوكُمْ!
\par 19 فَإِنْ كُنْتُمْ قَدْ عَمِلْتُمْ بِالْحَقِّ وَالصِّحَّةِ مَعَ يَرُبَّعْلَ وَمَعَ بَيْتِهِ فِي هَذَا الْيَوْمِ, فَافْرَحُوا أَنْتُمْ بِأَبِيمَالِكَ, وَلِْيَفْرَحْ هُوَ أَيْضاً بِكُمْ.
\par 20 وَإِلاَّ فَتَخْرُجَ نَارٌ مِنْ أَبِيمَالِكَ وَتَأْكُلَ أَهْلَ شَكِيمَ وَسُكَّانَ الْقَلْعَةِ, وَتَخْرُجَ نَارٌ مِنْ أَهْلِ شَكِيمَ وَمِنْ سُكَّانِ الْقَلْعَةِ وَتَأْكُلَ أَبِيمَالِكَ».
\par 21 ثُمَّ هَرَبَ يُوثَامُ وَفَرَّ وَذَهَبَ إِلَى بِئْرَ, وَأَقَامَ هُنَاكَ مِنْ وَجْهِ أَبِيمَالِكَ أَخِيهِ.
\par 22 فَتَرَأَّسَ أَبِيمَالِكُ عَلَى إِسْرَائِيلَ ثَلاَثَ سِنِينَ.
\par 23 وَأَرْسَلَ الرَّبُّ رُوحاً رَدِيئَاً بَيْنَ أَبِيمَالِكَ وَأَهْلِ شَكِيمَ, فَغَدَرَ أَهْلُ شَكِيمَ بِأَبِيمَالِكَ.
\par 24 لِيَأْتِيَ ظُلْمُ بَنِي يَرُبَّعْلَ السَّبْعِينَ وَيُجْلَبَ دَمُهُمْ عَلَى أَبِيمَالِكَ أَخِيهِمِ الَّذِي قَتَلَهُمْ, وَعَلَى أَهْلِ شَكِيمَ الَّذِينَ شَدَّدُوا يَدَيْهِ لِقَتْلِ إِخْوَتِهِ.
\par 25 فَوَضَعَ لَهُ أَهْلُ شَكِيمَ كَمِيناً عَلَى رُؤُوسِ الْجِبَالِ, وَكَانُوا يَسْتَلِبُونَ كُلَّ مَنْ عَبَرَ بِهِمْ فِي الطَّرِيقِ. فَأُخْبِرَ أَبِيمَالِكُ.
\par 26 وَجَاءَ جَعَلُ بْنُ عَابِدٍ مَعَ إِخْوَتِهِ وَعَبَرُوا إِلَى شَكِيمَ فَوَثِقَ بِهِ أَهْلُ شَكِيمَ.
\par 27 وَخَرَجُوا إِلَى الْحَقْلِ وَقَطَفُوا كُرُومَهُمْ وَدَاسُوا وَصَنَعُوا تَمْجِيداً, وَدَخَلُوا بَيْتَ إِلَهِهِمْ وَأَكَلُوا وَشَرِبُوا وَلَعَنُوا أَبِيمَالِكَ.
\par 28 فَقَالَ جَعَلُ بْنُ عَابِدٍ: «مَنْ هُوَ أَبِيمَالِكُ وَمَنْ هُوَ شَكِيمُ حَتَّى نَخْدِمَهُ؟ أَمَا هُوَ ابْنُ يَرُبَّعْلَ, وَزَبُولُ وَكِيلُهُ؟ اخْدِمُوا رِجَالَ حَمُورَ أَبِي شَكِيمَ. فَلِمَاذَا نَخْدِمُهُ نَحْنُ؟
\par 29 مَنْ يَجْعَلُ هَذَا الشَّعْبَ بِيَدِي فَأَعْزِلَ أَبِيمَالِكَ». وَقَالَ لأَبِيمَالِكَ: «كَثِّرْ جُنْدَكَ وَاخْرُجْ!»
\par 30 وَلَمَّا سَمِعَ زَبُولُ رَئِيسُ الْمَدِينَةِ كَلاَمَ جَعَلَ بْنِ عَابِدٍ حَمِيَ غَضَبُهُ,
\par 31 وَأَرْسَلَ رُسُلاً إِلَى أَبِيمَالِكَ فِي تُرْمَةَ يَقُولُ: «هُوَذَا جَعَلُ بْنُ عَابِدٍ وَإِخْوَتُهُ قَدْ أَتُوا إِلَى شَكِيمَ, وَهَا هُمْ يُهَيِّجُونَ الْمَدِينَةَ ضِدَّكَ.
\par 32 فَالآنَ قُمْ لَيْلاً أَنْتَ وَالشَّعْبُ الَّذِي مَعَكَ وَاكْمِنْ فِي الْحَقْلِ.
\par 33 وَيَكُونُ فِي الصَّبَاحِ عِنْدَ شُرُوقِ الشَّمْسِ أَنَّكَ تُبَكِّرُ وَتَقْتَحِمُ الْمَدِينَةَ. وَهَا هُوَ وَالشَّعْبُ الَّذِي مَعَهُ يَخْرُجُونَ إِلَيْكَ فَتَفْعَلُ بِهِ حَسَبَمَا تَجِدُهُ يَدُكَ».
\par 34 فَقَامَ أَبِيمَالِكُ وَكُلُّ الشَّعْبِ الَّذِي مَعَهُ لَيْلاً وَكَمَنُوا لِشَكِيمَ أَرْبَعَ فِرَقٍ.
\par 35 فَخَرَجَ جَعَلُ بْنُ عَابِدٍ وَوَقَفَ فِي مَدْخَلِ بَابِ الْمَدِينَةِ. فَقَامَ أَبِيمَالِكُ وَالشَّعْبُ الَّذِي مَعَهُ مِنَ الْمَكْمَنِ.
\par 36 وَرَأَى جَعَلُ الشَّعْبَ فَقَالَ لِزَبُولَ: «هُوَذَا شَعْبٌ نَازِلٌ عَنْ رُؤُوسِ الْجِبَالِ». فَقَالَ لَهُ زَبُولُ: «إِنَّكَ تَرَى ظِلَّ الْجِبَالِ كَأَنَّهُ أُنَاسٌ».
\par 37 فَعَادَ جَعَلُ وَقَالَ أَيْضاً: «هُوَذَا شَعْبٌ نَازِلٌ مِنْ عِنْدِ أَعَالِي الأَرْضِ, وَفِرْقَةٌ وَاحِدَةٌ آتِيَةٌ عَنْ طَرِيقِ بَلُّوطَةِ الْعَائِفِينَ».
\par 38 فَقَالَ لَهُ زَبُولُ: «أَيْنَ الآنَ فَمُكَ الَّذِي قُلْتَ بِهِ: مَنْ هُوَ أَبِيمَالِكُ حَتَّى نَخْدِمَهُ؟ أَلَيْسَ هَذَا هُوَ الشَّعْبُ الَّذِي رَذَلْتَهُ. فَاخْرُجِ الآنَ وَحَارِبْهُ».
\par 39 فَخَرَجَ جَعَلُ أَمَامَ أَهْلِ شَكِيمَ وَحَارَبَ أَبِيمَالِكَ.
\par 40 فَهَزَمَهُ أَبِيمَالِكُ, فَهَرَبَ مِنْ قُدَّامِهِ وَسَقَطَ قَتْلَى كَثِيرُونَ حَتَّى عِنْدَ مَدْخَلِ الْبَابِ.
\par 41 فَأَقَامَ أَبِيمَالِكُ فِي أَرُومَةَ. وَطَرَدَ زَبُولُ جَعَلاً وَإِخْوَتَهُ عَنِ الْإِقَامَةِ فِي شَكِيمَ.
\par 42 وَكَانَ فِي الْغَدِ أَنَّ الشَّعْبَ خَرَجَ إِلَى الْحَقْلِ وَأَخْبَرُوا أَبِيمَالِكَ.
\par 43 فَأَخَذَ الْقَوْمَ وَقَسَمَهُمْ إِلَى ثَلاَثِ فِرَقٍ, وَكَمَنَ فِي الْحَقْلِ وَنَظَرَ وَإِذَا الشَّعْبُ يَخْرُجُ مِنَ الْمَدِينَةِ, فَقَامَ عَلَيْهِمْ وَضَرَبَهُمْ.
\par 44 وَأَبِيمَالِكُ وَالْفِرْقَةُ الَّتِي مَعَهُ اقْتَحَمُوا وَوَقَفُوا فِي مَدْخَلِ بَابِ الْمَدِينَةِ. وَأَمَّا الْفِرْقَتَانِ فَهَجَمَتَا عَلَى كُلِّ مَنْ فِي الْحَقْلِ وَضَرَبَتَاهُ.
\par 45 وَحَارَبَ أَبِيمَالِكُ الْمَدِينَةَ كُلَّ ذَلِكَ الْيَوْمِ, وَأَخَذَ الْمَدِينَةَ وَقَتَلَ الشَّعْبَ الَّذِي بِهَا, وَهَدَمَ الْمَدِينَةَ وَزَرَعَهَا مِلْحاً.
\par 46 وَسَمِعَ كُلُّ أَهْلِ بُرْجِ شَكِيمَ فَدَخَلُوا إِلَى صَرْحِ بَيْتِ إِيلِ بَرِيثَ.
\par 47 فَأُخْبِرَ أَبِيمَالِكُ أَنَّ كُلَّ أَهْلِ بُرْجِ شَكِيمَ قَدِ اجْتَمَعُوا.
\par 48 فَصَعِدَ أَبِيمَالِكُ إِلَى جَبَلِ صَلْمُونَ هُوَ وَكُلُّ الشَّعْبِ الَّذِي مَعَهُ. وَأَخَذَ أَبِيمَالِكُ الْفُؤُوسَ بِيَدِهِ, وَقَطَعَ غُصْنَ شَجَرٍ وَرَفَعَهُ وَوَضَعَهُ عَلَى كَتِفِهِ, وَقَالَ لِلشَّعْبِ الَّذِي مَعَهُ: «مَا رَأَيْتُمُونِي أَفْعَلُهُ فَأَسْرِعُوا افْعَلُوا مِثْلِي».
\par 49 فَقَطَعَ الشَّعْبُ أَيْضاً كُلُّ وَاحِدٍ غُصْناً وَسَارُوا وَرَاءَ أَبِيمَالِكَ, وَوَضَعُوهَا عَلَى الصَّرْحِ وَأَحْرَقُوا عَلَيْهِمِ الصَّرْحَ بِالنَّارِ. فَمَاتَ أَيْضاً جَمِيعُ أَهْلِ بُرْجِ شَكِيمَ, نَحْوُ أَلْفِ رَجُلٍ وَامْرَأَةٍ.
\par 50 ثُمَّ ذَهَبَ أَبِيمَالِكُ إِلَى تَابَاصَ وَنَزَلَ فِي تَابَاصَ وَأَخَذَهَا.
\par 51 وَكَانَ بُرْجٌ قَوِيٌّ فِي وَسَطِ الْمَدِينَةِ فَهَرَبَ إِلَيْهِ جَمِيعُ الرِّجَالِ وَالنِّسَاءِ وَكُلُّ أَهْلِ الْمَدِينَةِ وَأَغْلَقُوا وَرَاءَهُمْ وَصَعِدُوا إِلَى سَطْحِ الْبُرْجِ.
\par 52 فَجَاءَ أَبِيمَالِكُ إِلَى الْبُرْجِ وَحَارَبَهُ, وَاقْتَرَبَ إِلَى بَابِ الْبُرْجِ لِيُحْرِقَهُ بِالنَّارِ.
\par 53 فَطَرَحَتِ امْرَأَةٌ قِطْعَةَ رَحىً عَلَى رَأْسِ أَبِيمَالِكَ فَشَجَّتْ جُمْجُمَتَهُ.
\par 54 فَدَعَا حَالاً الْغُلاَمَ حَامِلَ عُدَّتِهِ وَقَالَ لَهُ: «اخْتَرِطْ سَيْفَكَ وَاقْتُلْنِي, لِئَلاَّ يَقُولُوا عَنِّي: قَتَلَتْهُ امْرَأَةٌ». فَطَعَنَهُ الْغُلاَمُ فَمَاتَ.
\par 55 وَلَمَّا رَأَى رِجَالُ إِسْرَائِيلَ أَنَّ أَبِيمَالِكَ قَدْ مَاتَ, ذَهَبَ كُلُّ وَاحِدٍ إِلَى مَكَانِهِ.
\par 56 فَرَدَّ اللَّهُ شَرَّ أَبِيمَالِكَ الَّذِي فَعَلَهُ بِأَبِيهِ لِقَتْلِهِ إِخْوَتَهُ السَّبْعِينَ,
\par 57 وَكُلَّ شَرِّ أَهْلِ شَكِيمَ رَدَّهُ اللَّهُ عَلَى رُؤُوسِهِمْ, وَأَتَتْ عَلَيْهِمْ لَعْنَةُ يُوثَامَ بْنِ يَرُبَّعْلَ.

\chapter{10}

\par 1 وَقَامَ بَعْدَ أَبِيمَالِكَ لِتَخْلِيصِ إِسْرَائِيلَ تُولَعُ بْنُ فُوَاةَ بْنِ دُودُو, رَجُلٌ مِنْ يَسَّاكَرَ, كَانَ سَاكِناً فِي شَامِيرَ فِي جَبَلِ أَفْرَايِمَ.
\par 2 فَقَضَى لإِسْرَائِيلَ ثَلاَثاً وَعِشْرِينَ سَنَةً وَمَاتَ وَدُفِنَ فِي شَامِيرَ.
\par 3 ثُمَّ قَامَ بَعْدَهُ يَائِيرُ الْجِلْعَادِيُّ, فَقَضَى لإِسْرَائِيلَ اثْنَتَيْنِ وَعِشْرِينَ سَنَةً.
\par 4 وَكَانَ لَهُ ثَلاَثُونَ وَلَداً يَرْكَبُونَ عَلَى ثَلاَثِينَ جَحْشاً, وَلَهُمْ ثَلاَثُونَ مَدِينَةً. يَدْعُونَهَا «حَّوُوثَ يَائِيرَ» إِلَى هَذَا الْيَوْمِ. هِيَ فِي أَرْضِ جِلْعَادَ.
\par 5 وَمَاتَ يَائِيرُ وَدُفِنَ فِي قَامُونَ.
\par 6 وَعَادَ بَنُو إِسْرَائِيلَ يَعْمَلُونَ الشَّرَّ فِي عَيْنَيِ الرَّبِّ, وَعَبَدُوا الْبَعْلِيمَ وَالْعَشْتَارُوثَ وَآلِهَةَ أَرَامَ وَآلِهَةَ صَيْدُونَ وَآلِهَةَ مُوآبَ وَآلِهَةَ بَنِي عَمُّونَ وَآلِهَةَ الْفِلِسْطِينِيِّينَ, وَتَرَكُوا الرَّبَّ وَلَمْ يَعْبُدُوهُ.
\par 7 فَحَمِيَ غَضَبُ الرَّبِّ عَلَى إِسْرَائِيلَ وَبَاعَهُمْ بِيَدِ الْفِلِسْطِينِيِّينَ وَبِيَدِ بَنِي عَمُّونَ.
\par 8 فَحَطَّمُوا وَرَضَّضُوا بَنِي إِسْرَائِيلَ فِي تِلْكَ السَّنَةِ. ثَمَانِي عَشَرَةَ سَنَةً. جَمِيعَ بَنِي إِسْرَائِيلَ الَّذِينَ فِي عَبْرِ الأُرْدُنِّ فِي أَرْضِ الأَمُورِيِّينَ الَّذِينَ فِي جِلْعَادَ.
\par 9 وَعَبَرَ بَنُو عَمُّونَ الأُرْدُنَّ لِيُحَارِبُوا أَيْضاً يَهُوذَا وَبِنْيَامِينَ وَبَيْتَ أَفْرَايِمَ. فَتَضَايَقَ إِسْرَائِيلُ جِدّاً.
\par 10 فَصَرَخَ بَنُو إِسْرَائِيلَ إِلَى الرَّبِّ: «أَخْطَأْنَا إِلَيْكَ لأَنَّنَا تَرَكْنَا إِلَهَنَا وَعَبَدْنَا الْبَعْلِيمَ».
\par 11 فَقَالَ الرَّبُّ لِبَنِي إِسْرَائِيلَ: «أَلَيْسَ مِنَ الْمِصْرِيِّينَ وَالأَمُورِيِّينَ وَبَنِي عَمُّونَ وَالْفِلِسْطِينِيِّينَ خَلَّصْتُكُمْ؟
\par 12 وَالصَّيْدُونِيُّونَ وَالْعَمَالِقَةُوَالْمَعُونِيُّونَ قَدْ ضَايَقُوكُمْ فَصَرَخْتُمْ إِلَيَّ فَخَلَّصْتُكُمْ مِنْ أَيْدِيهِمْ؟
\par 13 وَأَنْتُمْ قَدْ تَرَكْتُمُونِي وَعَبَدْتُمْ آلِهَةً أُخْرَى. لِذَلِكَ لاَ أَعُودُ أُخَلِّصُكُمْ.
\par 14 اِمْضُوا وَاصْرُخُوا إِلَى الآلِهَةِ الَّتِي اخْتَرْتُمُوهَا. لِتُخَلِّصْكُمْ هِيَ فِي زَمَانِ ضِيقِكُمْ».
\par 15 فَقَالَ بَنُو إِسْرَائِيلَ لِلرَّبِّ: «أَخْطَأْنَا فَافْعَلْ بِنَا كُلَّ مَا يَحْسُنُ فِي عَيْنَيْكَ. إِنَّمَا أَنْقِذْنَا هَذَا الْيَوْمَ».
\par 16 وَأَزَالُوا الآلِهَةَ الْغَرِيبَةَ مِنْ وَسَطِهِمْ وَعَبَدُوا الرَّبَّ, فَضَاقَتْ نَفْسُهُ بِسَبَبِ مَشَقَّةِ إِسْرَائِيلَ.
\par 17 فَاجْتَمَعَ بَنُو عَمُّونَ وَنَزَلُوا فِي جِلْعَادَ, وَاجْتَمَعَ بَنُو إِسْرَائِيلَ وَنَزَلُوا فِي الْمِصْفَاةِ.
\par 18 فَقَالَ الشَّعْبُ رُؤَسَاءُ جِلْعَادَ الْوَاحِدُ لِصَاحِبِهِ: «أَيٌّ هُوَ الرَّجُلُ الَّذِي يَبْتَدِئُ بِمُحَارَبَةِ بَنِي عَمُّونَ, فَإِنَّهُ يَكُونُ رَأْساً لِجَمِيعِ سُكَّانِ جِلْعَادَ».

\chapter{11}

\par 1 وَكَانَ يَفْتَاحُ الْجِلْعَادِيُّ جَبَّارَ بَأْسٍ, وَهُوَ ابْنُ امْرَأَةٍ زَانِيَةٍ. وَجِلْعَادُ وَلَدَ يَفْتَاحَ.
\par 2 ثُمَّ وَلَدَتِ امْرَأَةُ جِلْعَادَ لَهُ بَنِينَ. فَلَمَّا كَبِرَ بَنُو الْمَرْأَةِ طَرَدُوا يَفْتَاحَ, وَقَالُوا لَهُ: «لاَ تَرِثْ فِي بَيْتِ أَبِينَا لأَنَّكَ أَنْتَ ابْنُ امْرَأَةٍ أُخْرَى».
\par 3 فَهَرَبَ يَفْتَاحُ مِنْ وَجْهِ إِخْوَتِهِ وَأَقَامَ فِي أَرْضِ طُوبٍ. فَاجْتَمَعَ إِلَى يَفْتَاحَ رِجَالٌ بَطَّالُونَ وَكَانُوا يَخْرُجُونَ مَعَهُ.
\par 4 وَكَانَ بَعْدَ أَيَّامٍ أَنَّ بَنِي عَمُّونَ حَارَبُوا إِسْرَائِيلَ.
\par 5 وَلَمَّا حَارَبَ بَنُو عَمُّونَ إِسْرَائِيلَ ذَهَبَ شُيُوخُ جِلْعَادَ لِيَأْتُوا بِيَفْتَاحَ مِنْ أَرْضِ طُوبٍ.
\par 6 وَقَالُوا لِيَفْتَاحَ: «تَعَالَ وَكُنْ لَنَا قَائِداً فَنُحَارِبَ بَنِي عَمُّونَ».
\par 7 فَقَالَ يَفْتَاحُ لِشُيُوخِ جِلْعَادَ: «أَمَا أَبْغَضْتُمُونِي أَنْتُمْ وَطَرَدْتُمُونِي مِنْ بَيْتِ أَبِي؟ فَلِمَاذَا أَتَيْتُمْ إِلَيَّ الآنَ إِذْ تَضَايَقْتُمْ؟»
\par 8 فَقَالَ شُيُوخُ جِلْعَادَ لِيَفْتَاحَ: «لِذَلِكَ قَدْ رَجَعْنَا الآنَ إِلَيْكَ لِتَذْهَبَ مَعَنَا وَتُحَارِبَ بَنِي عَمُّونَ, وَتَكُونَ لَنَا رَأْساً لِكُلِّ سُكَّانِ جِلْعَادَ».
\par 9 فَقَالَ يَفْتَاحُ لِشُيُوخِ جِلْعَادَ: «إِذَا أَرْجَعْتُمُونِي لِمُحَارَبَةِ بَنِي عَمُّونَ وَدَفَعَهُمُ الرَّبُّ أَمَامِي فَأَنَا أَكُونُ لَكُمْ رَأْساً».
\par 10 فَقَالَ شُيُوخُ جِلْعَادَ لِيَفْتَاحَ: «الرَّبُّ يَكُونُ سَامِعاً بَيْنَنَا إِنْ كُنَّا لاَ نَفْعَلُ هَكَذَا حَسَبَ كَلاَمِكَ».
\par 11 فَذَهَبَ يَفْتَاحُ مَعَ شُيُوخِ جِلْعَادَ, وَجَعَلَهُ الشَّعْبُ عَلَيْهِمْ رَأْساً وَقَائِداً. فَتَكَلَّمَ يَفْتَاحُ بِجَمِيعِ كَلاَمِهِ أَمَامَ الرَّبِّ فِي الْمِصْفَاةِ.
\par 12 فَأَرْسَلَ يَفْتَاحُ رُسُلاً إِلَى مَلِكِ بَنِي عَمُّونَ يَقُولُ: «مَا لِي وَلَكَ أَنَّكَ أَتَيْتَ إِلَيَّ لِلْمُحَارَبَةِ فِي أَرْضِي؟»
\par 13 فَقَالَ مَلِكُ بَنِي عَمُّونَ لِرُسُلِ يَفْتَاحَ: «لأَنَّ إِسْرَائِيلَ قَدْ أَخَذَ أَرْضِي عِنْدَ صُعُودِهِ مِنْ مِصْرَ مِنْ أَرْنُونَ إِلَى الْيَبُّوقِ وَإِلَى الأُرْدُنِّ. فَالآنَ رُدَّهَا بِسَلاَمٍ».
\par 14 وَعَادَ أَيْضاً يَفْتَاحُ وَأَرْسَلَ رُسُلاً إِلَى مَلِكِ بَنِي عَمُّونَ
\par 15 وَقَالَ لَهُ: «هَكَذَا يَقُولُ يَفْتَاحُ: لَمْ يَأْخُذْ إِسْرَائِيلُ أَرْضَ مُوآبَ وَلاَ أَرْضَ بَنِي عَمُّونَ,
\par 16 لأَنَّهُ عِنْدَ صُعُودِ إِسْرَائِيلَ مِنْ مِصْرَ سَارَ فِي الْقَفْرِ إِلَى بَحْرِ سُوفٍ وَأَتَى إِلَى قَادِشَ.
\par 17 وَأَرْسَلَ إِسْرَائِيلُ رُسُلاً إِلَى مَلِكِ أَدُومَ قَائِلاً: دَعْنِي أَعْبُرْ فِي أَرْضِكَ. فَلَمْ يَسْمَعْ مَلِكُ أَدُومَ. فَأَرْسَلَ أَيْضاً إِلَى مَلِكِ مُوآبَ فَلَمْ يَرْضَ. فَأَقَامَ إِسْرَائِيلُ فِي قَادِشَ.
\par 18 وَسَارَ فِي الْقَفْرِ وَدَارَ بِأَرْضِ أَدُومَ وَأَرْضِ مُوآبَ وَأَتَى مِنْ مَشْرِقِ الشَّمْسِ إِلَى أَرْضِ مُوآبَ وَنَزَلَ فِي عَبْرِ أَرْنُونَ, وَلَمْ يَأْتُوا إِلَى تُخُمِ مُوآبَ لأَنَّ أَرْنُونَ تُخُمُ مُوآبَ.
\par 19 ثُمَّ أَرْسَلَ إِسْرَائِيلُ رُسُلاً إِلَى سِيحُونَ مَلِكِ الأَمُورِيِّينَ مَلِكِ حَشْبُونَ, وَقَالَ لَهُ إِسْرَائِيلُ: دَعْنِي أَعْبُرْ فِي أَرْضِكَ إِلَى مَكَانِي.
\par 20 وَلَمْ يَأْمَنْ سِيحُونُ لإِسْرَائِيلَ أَنْ يَعْبُرَ فِي تُخُمِهِ, بَلْ جَمَعَ سِيحُونُ كُلَّ شَعْبِهِ وَنَزَلُوا فِي يَاهَصَ وَحَارَبُوا إِسْرَائِيلَ.
\par 21 فَدَفَعَ الرَّبُّ إِلَهُ إِسْرَائِيلَ سِيحُونَ وَكُلَّ شَعْبِهِ لِيَدِ إِسْرَائِيلَ فَضَرَبُوهُمْ, وَامْتَلَكَ إِسْرَائِيلُ كُلَّ أَرْضِ الأَمُورِيِّينَ سُكَّانِ تِلْكَ الأَرْضِ.
\par 22 فَامْتَلَكُوا كُلَّ تُخُمِ الأَمُورِيِّينَ مِنْ أَرْنُونَ إِلَى الْيَبُّوقِ وَمِنَ الْقَفْرِ إِلَى الأُرْدُنِّ.
\par 23 وَالآنَ الرَّبُّ إِلَهُ إِسْرَائِيلَ قَدْ طَرَدَ الأَمُورِيِّينَ مِنْ أَمَامِ شَعْبِهِ إِسْرَائِيلَ. أَفَأَنْتَ تَمْتَلِكُهُ؟
\par 24 أَلَيْسَ مَا يُمَلِّكُكَ إِيَّاهُ كَمُوشُ إِلَهُكَ تَمْتَلِكُ؟ وَجَمِيعُ الَّذِينَ طَرَدَهُمُ الرَّبُّ إِلَهُنَا مِنْ أَمَامِنَا فَإِيَّاهُمْ نَمْتَلِكُ.
\par 25 وَالآنَ فَهَلْ أَنْتَ خَيْرٌ مِنْ بَالاَقَ بْنِ صِفُّورَ مَلِكِ مُوآبَ, فَهَلْ خَاصَمَ إِسْرَائِيلَ أَوْ حَارَبَهُمْ مُحَارَبَةً؟
\par 26 حِينَ أَقَامَ إِسْرَائِيلُ فِي حَشْبُونَ وَقُرَاهَا وَعَرُوعِيرَ وَقُرَاهَا وَكُلِّ الْمُدُنِ الَّتِي عَلَى جَانِبِ أَرْنُونَ ثَلاَثَ مِئَةِ سَنَةٍ, فَلِمَاذَا لَمْ تَسْتَرِدَّهَا فِي تِلْكَ الْمُدَّةِ؟
\par 27 فَأَنَا لَمْ أُخْطِئْ إِلَيْكَ. وَأَمَّا أَنْتَ فَإِنَّكَ تَفْعَلُ بِي شَرّاً بِمُحَارَبَتِي. لِيَقْضِ الرَّبُّ الْقَاضِي الْيَوْمَ بَيْنَ بَنِي إِسْرَائِيلَ وَبَنِي عَمُّونَ».
\par 28 فَلَمْ يَسْمَعْ مَلِكُ بَنِي عَمُّونَ لِكَلاَمِ يَفْتَاحَ الَّذِي أَرْسَلَ إِلَيْهِ.
\par 29 فَكَانَ رُوحُ الرَّبِّ عَلَى يَفْتَاحَ, فَعَبَرَ جِلْعَادَ وَمَنَسَّى وَعَبَرَ مِصْفَاةَ جِلْعَادَ, وَمِنْ مِصْفَاةِ جِلْعَادَ عَبَرَ إِلَى بَنِي عَمُّونَ.
\par 30 وَنَذَرَ يَفْتَاحُ نَذْراً لِلرَّبِّ قَائِلاً: «إِنْ دَفَعْتَ بَنِي عَمُّونَ لِيَدِي
\par 31 فَالْخَارِجُ الَّذِي يَخْرُجُ مِنْ أَبْوَابِ بَيْتِي لِلِقَائِي عِنْدَ رُجُوعِي بِالسَّلاَمَةِ مِنْ عِنْدِ بَنِي عَمُّونَ يَكُونُ لِلرَّبِّ, وَأُصْعِدُهُ مُحْرَقَةً».
\par 32 ثُمَّ عَبَرَ يَفْتَاحُ إِلَى بَنِي عَمُّونَ لِمُحَارَبَتِهِمْ. فَدَفَعَهُمُ الرَّبُّ لِيَدِهِ.
\par 33 فَضَرَبَهُمْ مِنْ عَرُوعِيرَ إِلَى مَجِيئِكَ إِلَى مِنِّيتَ (عِشْرِينَ مَدِينَةً) وَإِلَى آبَلِ الْكُرُومِ ضَرْبَةً عَظِيمَةً جِدّاً. فَذَلَّ بَنُو عَمُّونَ أَمَامَ بَنِي إِسْرَائِيلَ.
\par 34 ثُمَّ أَتَى يَفْتَاحُ إِلَى الْمِصْفَاةِ إِلَى بَيْتِهِ, وَإِذَا بِابْنَتِهِ خَارِجَةً لِلِقَائِهِ بِدُفُوفٍ وَرَقْصٍ. وَهِيَ وَحِيدَةٌ. لَمْ يَكُنْ لَهُ ابْنٌ وَلاَ ابْنَةٌ غَيْرَهَا.
\par 35 وَكَانَ لَمَّا رَآهَا أَنَّهُ مَّزَقَ ثِيَابَهُ وَقَالَ: «آهِ يَا ابْنَتِي! قَدْ أَحْزَنْتِنِي حُزْناً وَصِرْتِ بَيْنَ مُكَدِّرِيَّ, لأَنِّي قَدْ فَتَحْتُ فَمِي إِلَى الرَّبِّ وَلاَ يُمْكِنُنِي الرُّجُوعُ».
\par 36 فَقَالَتْ لَهُ: «يَا أَبِي, هَلْ فَتَحْتَ فَاكَ إِلَى الرَّبِّ؟ فَافْعَلْ بِي كَمَا خَرَجَ مِنْ فِيكَ, بِمَا أَنَّ الرَّبَّ قَدِ انْتَقَمَ لَكَ مِنْ أَعْدَائِكَ بَنِي عَمُّونَ».
\par 37 ثُمَّ قَالَتْ لأَبِيهَا: «فَلْيُفْعَلْ لِي هَذَا الأَمْرُ: اتْرُكْنِي شَهْرَيْنِ فَأَذْهَبَ وَأَنْزِلَ عَلَى الْجِبَالِ وَأَبْكِيَ عَذْرَاوِيَّتِي أَنَا وَصَاحِبَاتِي».
\par 38 فَقَالَ: «اذْهَبِي». وَأَرْسَلَهَا إِلَى شَهْرَيْنِ. فَذَهَبَتْ هِيَ وَصَاحِبَاتُهَا وَبَكَتْ عَذْرَاوِيَّتَهَا عَلَى الْجِبَالِ.
\par 39 وَكَانَ عِنْدَ نِهَايَةِ الشَّهْرَيْنِ أَنَّهَا رَجَعَتْ إِلَى أَبِيهَا, فَفَعَلَ بِهَا نَذْرَهُ الَّذِي نَذَرَ. وَهِيَ لَمْ تَعْرِفْ رَجُلاً. فَصَارَتْ عَادَةً فِي إِسْرَائِيلَ
\par 40 أَنَّ بَنَاتِ إِسْرَائِيلَ يَذْهَبْنَ مِنْ سَنَةٍ إِلَى سَنَةٍ لِيَنُحْنَ عَلَى بِنْتِ يَفْتَاحَ الْجِلْعَادِيِّ أَرْبَعَةَ أَيَّامٍ فِي السَّنَةِ.

\chapter{12}

\par 1 وَاجْتَمَعَ رِجَالُ أَفْرَايِمَ وَعَبَرُوا إِلَى جِهَةِ الشِّمَالِ, وَقَالُوا لِيَفْتَاحَ: «لِمَاذَا عَبَرْتَ لِمُحَارَبَةِ بَنِي عَمُّونَ وَلَمْ تَدْعُنَا لِلذَّهَابِ مَعَكَ؟ نُحْرِقُ بَيْتَكَ عَلَيْكَ بِنَارٍ!»
\par 2 فَقَالَ لَهُمْ يَفْتَاحُ: «صَاحِبَ خِصَامٍ شَدِيدٍ كُنْتُ أَنَا وَشَعْبِي مَعَ بَنِي عَمُّونَ, وَنَادَيْتُكُمْ فَلَمْ تُخَلِّصُونِي مِنْ يَدِهِمْ.
\par 3 وَلَمَّا رَأَيْتُ أَنَّكُمْ لاَ تُخَلِّصُونَ, وَضَعْتُ نَفْسِي فِي يَدِي وَعَبَرْتُ إِلَى بَنِي عَمُّونَ, فَدَفَعَهُمُ الرَّبُّ لِيَدِي. فَلِمَاذَا صَعِدْتُمْ عَلَيَّ الْيَوْمَ هَذَا لِمُحَارَبَتِي؟».
\par 4 وَجَمَعَ يَفْتَاحُ كُلَّ رِجَالِ جِلْعَادَ وَحَارَبَ أَفْرَايِمَ, فَضَرَبَ رِجَالُ جِلْعَادَ أَفْرَايِمَ لأَنَّهُمْ قَالُوا: «أَنْتُمْ مُنْفَلِتُو أَفْرَايِمَ. جِلْعَادُ بَيْنَ أَفْرَايِمَ وَمَنَسَّى».
\par 5 فَأَخَذَ الْجِلْعَادِيُّونَ مَخَاوِضَ الأُرْدُنِّ لأَفْرَايِمَ. وَكَانَ إِذْ قَالَ مُنْفَلِتُو أَفْرَايِمَ: «دَعُونِي أَعْبُرْ». كَانَ رِجَالُ جِلْعَادَ يَسْأَلُونَهُ: «أَأَنْتَ أَفْرَايِمِيٌّ؟» فَإِنْ قَالَ: «لاَ»
\par 6 كَانُوا يَقُولُونَ لَهُ: «قُلْ إِذاً: شِبُّولَتْ» فَيَقُولُ: «سِبُّولَتْ» وَلَمْ يَتَحَفَّظْ لِلَّفْظِ بِحَقٍّ. فَكَانُوا يَأْخُذُونَهُ وَيَذْبَحُونَهُ عَلَى مَخَاوِضِ الأُرْدُنِّ. فَسَقَطَ فِي ذَلِكَ الْوَقْتِ مِنْ أَفْرَايِمَ اثْنَانِ وَأَرْبَعُونَ أَلْفاً.
\par 7 وَقَضَى يَفْتَاحُ لإِسْرَائِيلَ سِتَّ سِنِينٍ. وَمَاتَ يَفْتَاحُ الْجِلْعَادِيُّ وَدُفِنَ فِي إِحْدَى مُدُنِ جِلْعَادَ.
\par 8 وَقَضَى بَعْدَهُ لإِسْرَائِيلَ إِبْصَانُ مِنْ بَيْتِ لَحْمٍ.
\par 9 وَكَانَ لَهُ ثَلاَثُونَ ابْناً وَثَلاَثُونَ ابْنَةً أَرْسَلَهُنَّ إِلَى الْخَارِجِ وَأَتَى مِنَ الْخَارِجِ بِثَلاَثِينَ ابْنَةً لِبَنِيهِ. وَقَضَى لإِسْرَائِيلَ سَبْعَ سِنِينٍَ.
\par 10 وَمَاتَ إِبْصَانُ وَدُفِنَ فِي بَيْتِ لَحْمٍ.
\par 11 وَقَضَى بَعْدَهُ لإِسْرَائِيلَ إِيلُونُ الّزَبُولُونِيُّ. قَضَى لإِسْرَائِيلَ عَشَرَ سِنِينٍ.
\par 12 وَمَاتَ إِيلُونُ الّزَبُولُونِيُّ وَدُفِنَ فِي أَيَّلُونَ فِي أَرْضِ زَبُولُونَ.
\par 13 وَقَضَى بَعْدَهُ لإِسْرَائِيلَ عَبْدُونُ بْنُ هِلِّيلَ الْفَرْعَتُونِيُّ.
\par 14 وَكَانَ لَهُ أَرْبَعُونَ ابْناً وَثَلاَثُونَ حَفِيداً يَرْكَبُونَ عَلَى سَبْعِينَ جَحْشاً. قَضَى لإِسْرَائِيلَ ثَمَانِيَ سِنِينٍ.
\par 15 وَمَاتَ عَبْدُونُ بْنُ هِلِّيلَ الْفَرْعَتُونِيُّ وَدُفِنَ فِي فَرْعَتُونَ فِي أَرْضِ أَفْرَايِمَ فِي جَبَلِ الْعَمَالِقَةِ.

\chapter{13}

\par 1 ثُمَّ عَادَ بَنُو إِسْرَائِيلَ يَعْمَلُونَ الشَّرَّ فِي عَيْنَيِ الرَّبِّ, فَدَفَعَهُمُ الرَّبُّ لِيَدِ الْفِلِسْطِينِيِّينَ أَرْبَعِينَ سَنَةً.
\par 2 وَكَانَ رَجُلٌ مِنْ صُرْعَةَ مِنْ عَشِيرَةِ الدَّانِيِّينَ اسْمُهُ مَنُوحُ, وَامْرَأَتُهُ عَاقِرٌ لَمْ تَلِدْ.
\par 3 فَتَرَاءَى مَلاَكُ الرَّبِّ لِلْمَرْأَةِ وَقَالَ لَهَا: «هَا أَنْتِ عَاقِرٌ لَمْ تَلِدِي, وَلَكِنَّكِ تَحْبَلِينَ وَتَلِدِينَ ابْناً
\par 4 وَالآنَ فَاحْذَرِي وَلاَ تَشْرَبِي خَمْراً وَلاَ مُسْكِراً وَلاَ تَأْكُلِي شَيْئاً نَجِساً.
\par 5 فَهَا إِنَّكِ تَحْبَلِينَ وَتَلِدِينَ ابْناً, وَلاَ يَعْلُ مُوسَى رَأْسَهُ, لأَنَّ الصَّبِيَّ يَكُونُ نَذِيراً لِلَّهِ مِنَ الْبَطْنِ, وَهُوَ يَبْدَأُ يُخَلِّصُ إِسْرَائِيلَ مِنْ يَدِ الْفِلِسْطِينِيِّينَ».
\par 6 فَدَخَلَتِ الْمَرْأَةُ وَقَالَتْ لِرَجُلِهَا: «جَاءَ إِلَيَّ رَجُلُ اللَّهِ, وَمَنْظَرُهُ كَمَنْظَرِ مَلاَكِ اللَّهِ, مُرْهِبٌ جِدّاً. وَلَمْ أَسْأَلْهُ مِنْ أَيْنَ هُوَ, وَلاَ هُوَ أَخْبَرَنِي عَنِ اسْمِهِ.
\par 7 وَقَالَ لِي: «هَا أَنْتِ تَحْبَلِينَ وَتَلِدِينَ ابْناً. وَالآنَ فَلاَ تَشْرَبِي خَمْراً وَلاَ مُسْكِراً وَلاَ تَأْكُلِي شَيْئاً نَجِساً, لأَنَّ الصَّبِيَّ يَكُونُ نَذِيراً لِلَّهِ مِنَ الْبَطْنِ إِلَى يَوْمِ مَوْتِهِ».
\par 8 فَصَلَّى مَنُوحُ إِلَى الرَّبِّ: «أَسْأَلُكَ يَا سَيِّدِي أَنْ يَأْتِيَ أَيْضاً إِلَيْنَا رَجُلُ اللَّهِ الَّذِي أَرْسَلْتَهُ وَيُعَلِّمَنَا مَاذَا نَعْمَلُ لِلصَّبِيِّ الَّذِي يُولَدُ».
\par 9 فَسَمِعَ اللَّهُ لِصَوْتِ مَنُوحَ, فَجَاءَ مَلاَكُ اللَّهِ أَيْضاً إِلَى الْمَرْأَةِ وَهِيَ جَالِسَةٌ فِي الْحَقْلِ, وَمَنُوحُ رَجُلُهَا لَيْسَ مَعَهَا.
\par 10 فَأَسْرَعَتِ الْمَرْأَةُ وَرَكَضَتْ وَأَخْبَرَتْ رَجُلَهَا: «هُوَذَا قَدْ تَرَاءَى لِيَ الرَّجُلُ الَّذِي جَاءَ إِلَيَّ ذَلِكَ الْيَوْمَ».
\par 11 فَقَامَ مَنُوحُ وَسَارَ وَرَاءَ امْرَأَتِهِ وَجَاءَ إِلَى الرَّجُلِ, وَقَالَ لَهُ: «أَأَنْتَ الرَّجُلُ الَّذِي تَكَلَّمَ مَعَ الْمَرْأَةِ؟» فَقَالَ: «أَنَا هُوَ».
\par 12 فَقَالَ مَنُوحُ: «عِنْدَ مَجِيءِ كَلاَمِكَ, مَاذَا يَكُونُ حُكْمُ الصَّبِيِّ وَمُعَامَلَتُهُ؟»
\par 13 فَقَالَ مَلاَكُ الرَّبِّ لِمَنُوحَ: «مِنْ كُلِّ مَا قُلْتُ لِلْمَرْأَةِ فَلْتَحْتَفِظْ.
\par 14 مِنْ كُلِّ مَا يَخْرُجُ مِنْ جَفْنَةِ الْخَمْرِ لاَ تَأْكُلْ, وَخَمْراً وَمُسْكِراً لاَ تَشْرَبْ, وَكُلَّ نَجِسٍ لاَ تَأْكُلْ. لِتَحْذَرْ مِنْ كُلِّ مَا أَوْصَيْتُهَا».
\par 15 فَقَالَ مَنُوحُ لِمَلاَكِ الرَّبِّ: «دَعْنَا نُعَّوِقْكَ وَنَعْمَلْ لَكَ جَدْيَ مِعْزىً».
\par 16 فَقَالَ مَلاَكُ الرَّبِّ لِمَنُوحَ: «وَلَوْ عَّوَقْتَنِي لاَ آكُلُ مِنْ خُبْزِكَ, وَإِنْ عَمِلْتَ مُحْرَقَةً فَلِلرَّبِّ أَصْعِدْهَا». (لأَنَّ مَنُوحَ لَمْ يَعْلَمْ أَنَّهُ مَلاَكُ الرَّبِّ).
\par 17 فَقَالَ مَنُوحُ لِمَلاَكِ الرَّبِّ: «مَا اسْمُكَ حَتَّى إِذَا جَاءَ كَلاَمُكَ نُكْرِمُكَ؟»
\par 18 فَقَالَ لَهُ مَلاَكُ الرَّبِّ: «لِمَاذَا تَسْأَلُ عَنِ اسْمِي وَهُوَ عَجِيبٌ؟»
\par 19 فَأَخَذَ مَنُوحُ جَدْيَ الْمِعْزَى وَالتَّقْدِمَةَ وَأَصْعَدَهُمَا عَلَى الصَّخْرَةِ لِلرَّبِّ. فَعَمِلَ عَمَلاً عَجِيباً وَمَنُوحُ وَامْرَأَتُهُ يَنْظُرَانِ.
\par 20 فَكَانَ عِنْدَ صُعُودِ اللَّهِيبِ عَنِ الْمَذْبَحِ نَحْوَ السَّمَاءِ أَنَّ مَلاَكَ الرَّبِّ صَعِدَ فِي لَهِيبِ الْمَذْبَحِ وَمَنُوحُ وَامْرَأَتُهُ يَنْظُرَانِ. فَسَقَطَا عَلَى وَجْهَيْهِمَا إِلَى الأَرْضِ.
\par 21 وَلَمْ يَعُدْ مَلاَكُ الرَّبِّ يَتَرَاءَى لِمَنُوحَ وَامْرَأَتِهِ. حِينَئِذٍ عَرَفَ مَنُوحُ أَنَّهُ مَلاَكُ الرَّبِّ.
\par 22 فَقَالَ مَنُوحُ لاِمْرَأَتِهِ: «نَمُوتُ مَوْتاً لأَنَّنَا قَدْ رَأَيْنَا اللَّهَ!»
\par 23 فَقَالَتْ لَهُ امْرَأَتُهُ: «لَوْ أَرَادَ الرَّبُّ أَنْ يُمِيتَنَا لَمَا أَخَذَ مِنْ يَدِنَا مُحْرَقَةً وَتَقْدِمَةً, وَلَمَا أَرَانَا كُلَّ هَذِهِ, وَلَمَا كَانَ فِي مِثْلِ هَذَا الْوَقْتِ أَسْمَعَنَا مِثْلَ هَذِهِ».
\par 24 فَوَلَدَتِ الْمَرْأَةُ ابْناً وَدَعَتِ اسْمَهُ شَمْشُونَ. فَكَبِرَ الصَّبِيُّ وَبَارَكَهُ الرَّبُّ.
\par 25 وَابْتَدَأَ رُوحُ الرَّبِّ يُحَرِّكُهُ فِي مَحَلَّةِ دَانٍَ بَيْنَ صُرْعَةَ وَأَشْتَأُولَ.

\chapter{14}

\par 1 وَنَزَلَ شَمْشُونُ إِلَى تِمْنَةَ وَرَأَى امْرَأَةً فِي تِمْنَةَ مِنْ بَنَاتِ الْفِلِسْطِينِيِّينَ,
\par 2 فَصَعِدَ وَأَخْبَرَ أَبَاهُ وَأُمَّهُ وَقَالَ: «قَدْ رَأَيْتُ امْرَأَةً فِي تِمْنَةَ مِنْ بَنَاتِ الْفِلِسْطِينِيِّينَ, فَالآنَ خُذَاهَا لِيَ امْرَأَةً».
\par 3 فَقَالَ لَهُ أَبُوهُ وَأُمُّهُ: «أَلَيْسَ فِي بَنَاتِ إِخْوَتِكَ وَفِي كُلِّ شَعْبِي امْرَأَةٌ حَتَّى أَنَّكَ ذَاهِبٌ لِتَأْخُذَ امْرَأَةً مِنَ الْفِلِسْطِينِيِّينَ الْغُلْفِ؟» فَقَالَ شَمْشُونُ لأَبِيهِ: «إِيَّاهَا خُذْ لِي لأَنَّهَا حَسُنَتْ فِي عَيْنَيَّ».
\par 4 وَلَمْ يَعْلَمْ أَبُوهُ وَأُمُّهُ أَنَّ ذَلِكَ مِنَ الرَّبِّ لأَنَّهُ كَانَ يَطْلُبُ عِلَّةً عَلَى الْفِلِسْطِينِيِّينَ. وَفِي ذَلِكَ الْوَقْتِ كَانَ الْفِلِسْطِينِيُّونَ مُتَسَلِّطِينَ عَلَى إِسْرَائِيلَ.
\par 5 فَنَزَلَ شَمْشُونُ وَأَبُوهُ وَأُمُّهُ إِلَى تِمْنَةَ وَأَتَوْا إِلَى كُرُومِ تِمْنَةَ. وَإِذَا بِشِبْلِ أَسَدٍ يُزَمْجِرُ لِلِقَائِهِ.
\par 6 فَحَلَّ عَلَيْهِ رُوحُ الرَّبِّ, فَشَقَّهُ كَشَقِّ الْجَدْيِ وَلَيْسَ فِي يَدِهِ شَيْءٌ. وَلَمْ يُخْبِرْ أَبَاهُ وَأُمَّهُ بِمَا فَعَلَ.
\par 7 فَنَزَلَ وَكَلَّمَ الْمَرْأَةَ فَحَسُنَتْ فِي عَيْنَيْ شَمْشُونَ.
\par 8 وَلَمَّا رَجَعَ بَعْدَ أَيَّامٍ لِيَأْخُذَهَا مَالَ لِيَرَى رِمَّةَ الأَسَدِ, وَإِذَا جَمَاعَةٌ النَّحْلِ فِي جَوْفِ الأَسَدِ مَعَ عَسَلٍ.
\par 9 فَأَخَذَ مِنْهُ عَلَى كَفَّيْهِ, وَكَانَ يَمْشِي وَيَأْكُلُ, وَذَهَبَ إِلَى أَبِيهِ وَأُمِّهِ وَأَعْطَاهُمَا فَأَكَلاَ, وَلَمْ يُخْبِرْهُمَا أَنَّهُ مِنْ جَوْفِ الأَسَدِ أَخَذَ الْعَسَلَ.
\par 10 وَنَزَلَ أَبُوهُ إِلَى الْمَرْأَةِ, فَعَمِلَ هُنَاكَ شَمْشُونُ وَلِيمَةً لأَنَّهُ هَكَذَا كَانَ يَفْعَلُ الْفِتْيَانُ.
\par 11 فَلَمَّا رَأُوهُ أَحْضَرُوا ثَلاَثِينَ مِنَ الأَصْحَابِ فَكَانُوا مَعَهُ.
\par 12 فَقَالَ لَهُمْ شَمْشُونُ: «لَأُحَاجِيَنَّكُمْ لُغَزاً, فَإِذَا حَلَلْتُمُوهُ لِي فِي سَبْعَةِ أَيَّامِ الْوَلِيمَةِ وَأَصَبْتُمُوهُ أُعْطِيكُمْ ثَلاَثِينَ قَمِيصاً وَثَلاَثِينَ حُلَّةَ ثِيَابٍ.
\par 13 وَإِنْ لَمْ تَقْدِرُوا أَنْ تَحُلُّوهُ لِي تُعْطُونِي أَنْتُمْ ثَلاَثِينَ قَمِيصاً وَثَلاَثِينَ حُلَّةَ ثِيَابٍ». فَقَالُوا لَهُ: «حَاجِ لُغْزَكَ فَنَسْمَعْهُ».
\par 14 فَقَالَ لَهُمْ: «مِنَ الآكِلِ خَرَجَ أَكْلٌ وَمِنَ الْجَافِي خَرَجَتْ حَلاَوَةٌ». فَلَمْ يَسْتَطِيعُوا أَنْ يَحُلُّوا الُّلغْزَ فِي ثَلاَثَةِ أَيَّامٍ.
\par 15 وَكَانَ فِي الْيَوْمِ السَّابِعِ أَنَّهُمْ قَالُوا لاِمْرَأَةِ شَمْشُونَ: «تَمَلَّقِي رَجُلَكِ لِكَيْ يُظْهِرَ لَنَا الُّلغْزَ لِئَلاَّ نُحْرِقَكِ وَبَيْتَ أَبِيكِ بِنَارٍ. أَلِتَسْلِبُونَا دَعَوْتُمُونَا أَمْ لاَ؟»
\par 16 فَبَكَتِ امْرَأَةُ شَمْشُونَ لَدَيْهِ وَقَالَتْ: «إِنَّمَا كَرِهْتَنِي وَلاَ تُحِبُّنِي. قَدْ حَاجَيْتَ بَنِي شَعْبِي لُغْزاً وَإِيَّايَ لَمْ تُخْبِرْ». فَقَالَ لَهَا: «هُوَذَا أَبِي وَأُمِّي لَمْ أُخْبِرْهُمَا, فَهَلْ إِيَّاكِ أُخْبِرُ؟»
\par 17 فَبَكَتْ لَدَيْهِ السَّبْعَةَ الأَيَّامِ الَّتِي فِيهَا كَانَتْ لَهُمُ الْوَلِيمَةُ. وَكَانَ فِي الْيَوْمِ السَّابِعِ أَنَّهُ أَخْبَرَهَا لأَنَّهَا ضَايَقَتْهُ, فَأَظْهَرَتِ الُّلغْزَ لِبَنِي شَعْبِهَا.
\par 18 فَقَالَ لَهُ رِجَالُ الْمَدِينَةِ فِي الْيَوْمِ السَّابِعِ قَبْلَ غُرُوبِ الشَّمْسِ: «أَيُّ شَيْءٍ أَحْلَى مِنَ الْعَسَلِ, وَمَا أَجْفَى مِنَ الأَسَدِ؟» فَقَالَ لَهُمْ: «لَوْ لَمْ تَحْرُثُوا مَعَ عِجْلَتِي لَمَا وَجَدْتُمْ لُغْزِي».
\par 19 وَحَلَّ عَلَيْهِ رُوحُ الرَّبِّ فَنَزَلَ إِلَى أَشْقَلُونَ وَقَتَلَ مِنْهُمْ ثَلاَثِينَ رَجُلاً, وَأَخَذَ سَلَبَهُمْ وَأَعْطَى الْحُلَلَ لِمُظْهِرِي الُّلغْزِ. وَحَمِيَ غَضَبُهُ وَصَعِدَ إِلَى بَيْتِ أَبِيهِ.
\par 20 فَصَارَتِ امْرَأَةُ شَمْشُونَ لِصَاحِبِهِ الَّذِي كَانَ يُصَاحِبُهُ.

\chapter{15}

\par 1 وَكَانَ بَعْدَ مُدَّةٍ فِي أَيَّامِ حَصَادِ الْحِنْطَةِ أَنَّ شَمْشُونَ افْتَقَدَ امْرَأَتَهُ بِجَدْيِ مِعْزىً.
\par 2 وَقَالَ: «أَدْخُلُ إِلَى امْرَأَتِي إِلَى حُجْرَتِهَا». وَلَكِنَّ أَبَاهَا لَمْ يَدَعْهُ أَنْ يَدْخُلَ. وَقَالَ أَبُوهَا: «إِنِّي قُلْتُ إِنَّكَ قَدْ كَرِهْتَهَا فَأَعْطَيْتُهَا لِصَاحِبِكَ. أَلَيْسَتْ أُخْتُهَا الصَّغِيرَةُ أَحْسَنَ مِنْهَا؟ فَلِْتَكُنْ لَكَ عِوَضاً عَنْهَا».
\par 3 فَقَالَ لَهُمْ شَمْشُونُ: «إِنِّي بَرِيءٌ الآنَ مِنَ الْفِلِسْطِينِيِّينَ إِذَا عَمِلْتُ بِهِمْ شَرّاً».
\par 4 وَذَهَبَ شَمْشُونُ وَأَمْسَكَ ثَلاَثَ مِئَةِ ابْنِ آوَى, وَأَخَذَ مَشَاعِلَ وَجَعَلَ ذَنَباً إِلَى ذَنَبٍ, وَوَضَعَ مَشْعَلاً بَيْنَ كُلِّ ذَنَبَيْنِ فِي الْوَسَطِ,
\par 5 ثُمَّ أَضْرَمَ الْمَشَاعِلَ نَاراً وَأَطْلَقَهَا بَيْنَ زُرُوعِ الْفِلِسْطِينِيِّينَ, فَأَحْرَقَ الأَكْدَاسَ وَالّزَرْعَ وَكُرُومَ الّزَيْتُونِ.
\par 6 فَقَالَ الْفِلِسْطِينِيُّونَ: «مَنْ فَعَلَ هَذَا؟» فَقَالُوا: «شَمْشُونُ صِهْرُ التِّمْنِيِّ, لأَنَّهُ أَخَذَ امْرَأَتَهُ وَأَعْطَاهَا لِصَاحِبِهِ». فَصَعِدَ الْفِلِسْطِينِيُّونَ وَأَحْرَقُوهَا وَأَبَاهَا بِالنَّارِ.
\par 7 فَقَالَ لَهُمْ شَمْشُونُ: «وَلَوْ فَعَلْتُمْ هَذَا فَإِنِّي أَنْتَقِمُ مِنْكُمْ, وَبَعْدُ أَكُفُّ».
\par 8 وَضَرَبَهُمْ سَاقاً عَلَى فَخِْذٍ ضَرْباً عَظِيماً. ثُمَّ نَزَلَ وَأَقَامَ فِي شَقِّ صَخْرَةِ عِيطَمَ.
\par 9 وَصَعِدَ الْفِلِسْطِينِيُّونَ وَنَزَلُوا فِي يَهُوذَا وَتَفَرَّقُوا فِي لَحْيٍ.
\par 10 فَسَأَلَهُمْ رِجَالُ يَهُوذَا: «لِمَاذَا صَعِدْتُمْ عَلَيْنَا؟» فَقَالُوا: «صَعِدْنَا لِنُوثِقَ شَمْشُونَ لِنَفْعَلَ بِهِ كَمَا فَعَلَ بِنَا».
\par 11 فَنَزَلَ ثَلاَثَةُ آلاَفِ رَجُلٍ مِنْ يَهُوذَا إِلَى شَقِّ صَخْرَةِ عِيطَمَ, وَقَالُوا لِشَمْشُونَ: «أَمَا عَلِمْتَ أَنَّ الْفِلِسْطِينِيِّينَ مُتَسَلِّطُونَ عَلَيْنَا؟ فَمَاذَا فَعَلْتَ بِنَا؟» فَقَالَ لَهُمْ: «كَمَا فَعَلُوا بِي هَكَذَا فَعَلْتُ بِهِمْ».
\par 12 فَقَالُوا لَهُ: «نَزَلْنَا لِنُوثِقَكَ وَنُسَلِّمَكَ إِلَى يَدِ الْفِلِسْطِينِيِّينَ». فَقَالَ لَهُمْ شَمْشُونُ: «احْلِفُوا لِي أَنَّكُمْ أَنْتُمْ لاَ تَقَعُونَ عَلَيَّ».
\par 13 فَأَجَابُوهُ: «كَلاَّ. وَلَكِنَّنَا نُوثِقُكَ وَنُسَلِّمُكَ إِلَى يَدِهِمْ, وَقَتْلاً لاَ نَقْتُلُكَ». فَأَوْثَقُوهُ بِحَبْلَيْنِ جَدِيدَيْنِ وَأَصْعَدُوهُ مِنَ الصَّخْرَةِ.
\par 14 وَلَمَّا جَاءَ إِلَى لَحْيٍ صَاحَ الْفِلِسْطِينِيُّونَ لِلِقَائِهِ. فَحَلَّ عَلَيْهِ رُوحُ الرَّبِّ, فَكَانَ الْحَبْلاَنِ اللَّذَانِ عَلَى ذِرَاعَيْهِ كَكَتَّانٍ أُحْرِقَ بِالنَّارِ, فَانْحَلَّ الْوِثَاقُ عَنْ يَدَيْهِ.
\par 15 وَوَجَدَ فَكَّ حِمَارٍ طَرِيّاً, فَأَخَذَهُ وَضَرَبَ بِهِ أَلْفَ رَجُلٍ.
\par 16 فَقَالَ شَمْشُونُ: «بفَكِّ حِمَارٍ كُومَةً كُومَتَيْنِ. بِفَكِّ حِمَارٍ قَتَلْتُ أَلْفَ رَجُلٍ».
\par 17 وَلَمَّا فَرَغَ مِنَ الْكَلاَمِ رَمَى الْفَكَّ مِنْ يَدِهِ, وَدَعَا ذَلِكَ الْمَكَانَ «رَمَتَ لَحْيٍ».
\par 18 ثُمَّ عَطِشَ جِدّاً فَدَعَا الرَّبَّ: «إِنَّكَ قَدْ جَعَلْتَ بِيَدِ عَبْدِكَ هَذَا الْخَلاَصَ الْعَظِيمَ, وَالآنَ أَمُوتُ مِنَ الْعَطَشِ وَأَسْقُطُ بِيَدِ الْغُلْفِ».
\par 19 فَشَقَّ اللَّهُ الجَوْفَ الَّذِي فِي لَحْيٍ, فَخَرَجَ مِنْهَا مَاءٌ, فَشَرِبَ وَرَجَعَتْ رُوحُهُ فَانْتَعَشَ. لِذَلِكَ دَعَا اسْمَهُ «عَيْنَ هَقُّورِي» الَّتِي فِي لَحْيٍ إِلَى هَذَا الْيَوْمِ.
\par 20 وَقَضَى لإِسْرَائِيلَ فِي أَيَّامِ الْفِلِسْطِينِيِّينَ عِشْرِينَ سَنَةً.

\chapter{16}

\par 1 ثُمَّ ذَهَبَ شَمْشُونُ إِلَى غَّزَةَ وَرَأَى هُنَاكَ امْرَأَةً زَانِيَةً فَدَخَلَ إِلَيْهَا.
\par 2 فَقِيلَ لِلْغَّزِيِّينَ: «قَدْ أَتَى شَمْشُونُ إِلَى هُنَا». فَأَحَاطُوا بِهِ وَكَمَنُوا لَهُ اللَّيْلَ كُلَّهُ عِنْدَ بَابِ الْمَدِينَةِ. فَهَدَأُوا اللَّيْلَ كُلَّهُ قَائِلِينَ: «عَُِنْدَ ضُوءِ الصَّبَاحِ نَقْتُلُهُ».
\par 3 فَاضْطَجَعَ شَمْشُونُ إِلَى نِصْفِ اللَّيْلِ, ثُمَّ قَامَ فِي نِصْفِ اللَّيْلِ وَأَخَذَ مِصْرَاعَيْ بَابِ الْمَدِينَةِ وَالْقَائِمَتَيْنِ وَقَلَعَهُمَا مَعَ الْعَارِضَةِ, وَوَضَعَهَا عَلَى كَتِفَيْهِ وَصَعِدَ بِهَا إِلَى رَأْسِ الْجَبَلِ الَّذِي مُقَابِلَ حَبْرُونَ.
\par 4 وَكَانَ بَعْدَ ذَلِكَ أَنَّهُ أَحَبَّ امْرَأَةً فِي وَادِي سَوْرَقَ اسْمُهَا دَلِيلَةُ.
\par 5 فَصَعِدَ إِلَيْهَا أَقْطَابُ الْفِلِسْطِينِيِّينَ وَقَالُوا لَهَا: «تَمَلَّقِيهِ وَانْظُرِي بِمَاذَا قُّوَتُهُ الْعَظِيمَةُ, وَبِمَاذَا نَتَمَكَّنُ مِنْهُ لِنُوثِقَهُ لإِذْلاَلِهِ, فَنُعْطِيَكِ كُلُّ وَاحِدٍ أَلْفاً وَمِئَةَ شَاقِلِ فِضَّةٍ».
\par 6 فَقَالَتْ دَلِيلَةُ لِشَمْشُونَ: «أَخْبِرْنِي بِمَاذَا قُّوَتُكَ الْعَظِيمَةُ وَبِمَاذَا تُوثَقُ لإِذْلاَلِكَ؟»
\par 7 فَقَالَ لَهَا شَمْشُونُ: «إِذَا أَوْثَقُونِي بِسَبْعَةِ أَوْتَارٍ طَرِيَّةٍ لَمْ تَجِفَّ أَضْعُفُ وَأَصِيرُ كَوَاحِدٍ مِنَ النَّاسِ».
\par 8 فَأَصْعَدَ لَهَا أَقْطَابُ الْفِلِسْطِينِيِّينَ سَبْعَةَ أَوْتَارٍ طَرِيَّةٍ لَمْ تَجِفَّ فَأَوْثَقَتْهُ بِهَا,
\par 9 وَالْكَمِينُ لاَبِثٌ عِنْدَهَا فِي الْحُجْرَةِ. فَقَالَتْ لَهُ: «الْفِلِسْطِينِيُّونَ عَلَيْكَ يَا شَمْشُونُ». فَقَطَعَ الأَوْتَارَ كَمَا يُقْطَعُ فَتِيلُ الْمَشَاقَةِ إِذَا شَمَّ النَّارَ وَلَمْ تُعْلَمْ قُّوَتُهُ.
\par 10 فَقَالَتْ دَلِيلَةُ لِشَمْشُونَ: «هَا قَدْ خَتَلْتَنِي وَكَلَّمْتَنِي بِالْكَذِبِ! فَأَخْبِرْنِيَ الآنَ بِمَاذَا تُوثَقُ».
\par 11 فَقَالَ لَهَا: «إِذَا أَوْثَقُونِي بِحِبَالٍ جَدِيدَةٍ لَمْ تُسْتَعْمَلْ أَضْعُفُ وَأَصِيرُ كَوَاحِدٍ مِنَ النَّاسِ».
\par 12 فَأَخَذَتْ دَلِيلَةُ حِبَالاً جَدِيدَةً وَأَوْثَقَتْهُ بِهَا, وَقَالَتْ لَهُ: «الْفِلِسْطِينِيُّونَ عَلَيْكَ يَا شَمْشُونُ, وَالْكَمِينُ لاَبِثٌ فِي الْحُجْرَةِ». فَقَطَعَهَا عَنْ ذِرَاعَيْهِ كَخَيْطٍ.
\par 13 فَقَالَتْ دَلِيلَةُ لِشَمْشُونَ: «حَتَّى الآنَ خَتَلْتَنِي وَكَلَّمْتَنِي بِالْكَذِبِ! فَأَخْبِرْنِي بِمَاذَا تُوثَقُ». فَقَالَ لَهَا: «إِذَا ضَفَرْتِ سَبْعَ خُصَلِ رَأْسِي مَعَ السَّدَى,
\par 14 فَمَكَّنَتْهَا بِالْوَتَدِ». وَقَالَتْ لَهُ: «الْفِلِسْطِينِيُّونَ عَلَيْكَ يَا شَمْشُونُ». فَانْتَبَهَ مِنْ نَوْمِهِ وَقَلَعَ وَتَدَ النَّسِيجِ وَالسَّدَى.
\par 15 فَقَالَتْ لَهُ: «كَيْفَ تَقُولُ أُحِبُّكِ, وَقَلْبُكَ لَيْسَ مَعِي؟ هُوَذَا ثَلاَثَ مَرَّاتٍ قَدْ خَتَلْتَنِي وَلَمْ تُخْبِرْنِي بِمَاذَا قُّوَتُكَ الْعَظِيمَةُ».
\par 16 وَلَمَّا كَانَتْ تُضَايِقُهُ بِكَلاَمِهَا كُلَّ يَوْمٍ وَأَلَحَّتْ عَلَيْهِ, ضَاقَتْ نَفْسُهُ إِلَى الْمَوْتِ,
\par 17 فَكَشَفَ لَهَا كُلَّ قَلْبِهِ, وَقَالَ لَهَا: «لَمْ يَعْلُ مُوسَى رَأْسِي لأَنِّي نَذِيرُ اللَّهِ مِنْ بَطْنِ أُمِّي, فَإِنْ حُلِقْتُ تُفَارِقُنِي قُّوَتِي وَأَضْعُفُ وَأَصِيرُ كَأَحَدِ النَّاسِ».
\par 18 وَلَمَّا رَأَتْ دَلِيلَةُ أَنَّهُ قَدْ أَخْبَرَهَا بِكُلِّ مَا بِقَلْبِهِ, أَرْسَلَتْ فَدَعَتْ أَقْطَابَ الْفِلِسْطِينِيِّينَ وَقَالَتِ: «اصْعَدُوا هَذِهِ الْمَرَّةَ فَإِنَّهُ قَدْ كَشَفَ لِي كُلَّ قَلْبِهِ». فَصَعِدَ إِلَيْهَا أَقْطَابُ الْفِلِسْطِينِيِّينَ وَأَصْعَدُوا الْفِضَّةَ بِيَدِهِمْ.
\par 19 وَأَنَامَتْهُ عَلَى رُكْبَتَيْهَا وَدَعَتْ رَجُلاً وَحَلَقَتْ سَبْعَ خُصَلِ رَأْسِهِ, وَابْتَدَأَتْ بِإِذْلاَلِهِ, وَفَارَقَتْهُ قُّوَتُهُ.
\par 20 وَقَالَتِ: «الْفِلِسْطِينِيُّونَ عَلَيْكَ يَا شَمْشُونُ». فَانْتَبَهَ مِنْ نَوْمِهِ وَقَالَ: «أَخْرُجُ حَسَبَ كُلِّ مَرَّةٍ وَأَنْتَفِضُ». وَلَمْ يَعْلَمْ أَنَّ الرَّبَّ قَدْ فَارَقَهُ!
\par 21 فَأَخَذَهُ الْفِلِسْطِينِيُّونَ وَقَلَعُوا عَيْنَيْهِ, وَنَزَلُوا بِهِ إِلَى غَّزَةَ وَأَوْثَقُوهُ بِسَلاَسِلِ نُحَاسٍ. وَكَانَ يَطْحَنُ فِي بَيْتِ السِّجْنِ.
\par 22 وَابْتَدَأَ شَعْرُ رَأْسِهِ يَنْبُتُ بَعْدَ أَنْ حُلِقَ.
\par 23 وَأَمَّا أَقْطَابُ الْفِلِسْطِينِيِّينَ فَاجْتَمَعُوا لِيَذْبَحُوا ذَبِيْحَةً عَظِيْمَةً لِدَاجُوْنَ إِلَهِهِمْ وَيَفْرَحُوْا وَقَالُوا قَدْ دَفَعَ إِلَهُنَا لِيَدِنَا شَمشُوْنَ عَدُوَّنَا.
\par 24 وَلَمَّا رَآهُ الْشَّعْبُ مَجَّدُوا إِلَهَهُمْ لأَنَّهُمْ قَالُوْا قَدْ دَفَعَ إِلَهُنَا لِيَدِنَا عَدُوَّنَا الَّذِي خَرَّبَ أَرْضَنَا وَكَثَّرَ قَتْلَانَا.
\par 25 وَكَانَ لَمَّا طَابَتْ قُلُوبُهُمْ أَنَّهُمْ قَالُوا ادْعُوا شَمْشُوْنَ لِيَلْعَبَ لَنَا. فَدَعَوْا شَمْشُوْنَ مِنْ بَيْتِ السَّجْنِ فَلَعِبَ أَمَامَهُمْ وَأَوْقَفُوْهُ بَيْنَ الْأَعْمِدَةِ.
\par 26 فَقَالَ شَمْشُوْنُ لِلْغُلَامِ الْمَاسِكِ بِيَدِهِ دَعْنِيْ أَلْمِسِ الْأَعْمِدَةَ الَّتِيْ الْبَيْتُ قَائِمٌ عَلَيْهَا لِأَسْتَنِدَ عَلَيْهَا.
\par 27 وَكَانَ الْبَيْتُ مُمْلُوْءَاً رِجَالَاً وَنِسَاءً وَكَانَ هُنَاكَ جَمِيْعُ أَقْطَاْبِ الْفِلِسْطِينِيِّينَ وَعَلى السَّطْحِ نَحْوَ ثَلاثَةِ آلافِ رَجُلٍ وَامْرأَةٍ يَنْظُرُونَ لَعْبَ شَمْشُوْنَ.
\par 28 فدعا شَمْشُوْنُ الرَّبَّ وَقالَ: يَا سَيِّدِي الرَّبَّ اذْكُرْنِِي وَشَدِّدْنِي يَا اللَّهُ هَذِهِ الْمَرَّةَ فَقَطْ فَأَنْتَقِمَ نَِقْمَةً وَاحِدَةً عَنْ عَيْنَيَّ مِنَ الْفِلِسْطِيْنِيِّيْنَ.
\par 29 وَقَبَضَ شَمْشُوْنُ عَلَى الْعَمُودَيْنِ الْمُتَوَسِّطَيْنِ اللّذَيْنِ كَانَ الْبَيْتُ قَائِمَاً عَلَيْهِمَا وَاسْتَنَدَ عَلَيْهِمَا الْوَاحِدِ بِيَمِيْنِهِ وَالْآخَرِ بِيَسَارِهِ.
\par 30 وَقَالَ شَمْشُوْنُ لِتَمُتْ نَفْسِي مَعَ الْفِلِسْطِينِيّينَ. وَانْحَنى بِقُوّةٍ فَسَقَطَ الْبَيْتُ عَلى الْأقْطَابِ وَعَلى كُلِّ الشَّعْبِ الَّذي فِيْهِ فَكانَ الْمَوْتى الَّذيْنَ أَمَاتَهُمْ فِي مَوْتِهِ أَكْثَرَ مِنَ الَّذينَ أَمَاتَهُمْ فِي حَياتِهِ.
\par 31 فَنَزَلَ إِخْوَتُهُ وَكُلُّ بَيْتِ أَبِيْهِ وَحَمَلُوْهُ وَصَعِدُوا بِهِ وَدَفَنُوهُ بَيْنَ صُرْعَةَ وَأَشْتَأُولَ فِي قَبْرِ مَنُوْحَ أَبِيْهِ. وَهُوَ قَضَى لِإِسْرَائِيْلَ عِشْرِيْنَ سَنَةً. الْأَصْحَاحُ السَّابِعَُ عَشَرَ

\chapter{17}

\par 1 وَكَانَ رَجُلٌ مِنْ جَبَلِ أَفْرَايِمَ اسْمُهُ مِيْخَا.
\par 2 فَقَالَ لِأُمِّهِ إِنَّ الْأَلْفَ وَالْمِئَةَ شَاقِلِ الْفِضَّةِ الَّتِي أُخِذَتْ مِنْكِ وَأَنْتِ لَعَنْتِ وَقُلْتِ أَيْضَاً فِي أُذُنَيَّ. هُوَذَا الْفِضَّةُ مَعِي أَنَا أَخَذْتُهَا. فَقالَتْ أُمُّهُ مُبَارَكٌ أَنْتَ مِنَ الرَّبِّ يا ابْنِي.
\par 3 فَرَدَّ الْأَلْفَ والْمِئَةَ شَاقِلِ الْفِضَّةِ لِأُمِّهِ فَقالَتْ أُمُّهُ تَقْديْسَاً قَدَّسْتُ الْفِضَّةَ لِلرَّبِّ مِنْ يَدِي لابْنِي لِعَمَلِ تِمْثَالٍ مّنْحُوتٍ وَتِمْثَالٍ مَسْبُوكٍ. فَالْآنَ أَرُدُّهَا لَكَ.
\par 4 فَرَدَّ الْفِضَّةَ لِأُمِّهِ فَأَخَذَتْ أُمُّهُ مِئَتَيْ شَاقِلِ فِضَّةٍ وَأَعْطَتْهَا لِلصَّائِغِ فَعَمِلَهَا تِمْثَالاً مَنْحُوتاً وَتِمْثَالاً مَسْبُوكاً وَكَانَا في بَيْتِ مِيخَا.
\par 5 وَكَانَ لِلرَّجُلِ مِيخَا بَيْتٌ لِلْآلِهَةِ فَعَمِلَ أَفُوداً وَتَرَافِيمَ وَمَلَأَ يَدَ وَاحِدٍ مِنْ بَنِيهِ فَصَارَ لَهُ كَاهِناً.
\par 6 وفي تِلْكَ اْلْأَيَّامِ لَمْ يَكُنْ مَلِكٌ فِي إِسْرَائيلَ. كَانَ كُلُّ وَاحِدٍ يَعْمَلُ مَا يَحْسُنُ فِي عَيْنَيْهِ.
\par 7 وَكَانَ غُلَامٌ مِنْ بَيْتِ لَحْمِ يَهُوذَا مِنْ عَشِيرَةِ يَهُوذَا وَهُوَ لَاوِيٌّ مُتَغَرِّبٌ هُنَاكَ.
\par 8 فَذَهَبَ اْلْرَّجُلُ مِنَ الْمَدِينَةِ مِنْ بَيْتِ لَحْمِ يَهُوذَا لِكَيْ يَتَغَرَّبَ حَيْثُمَا اتّفَقَ. فَأَتَى إِلَى جَبَلِ أَفْرَايِمَ إلى بَيْتِ مِيخَا وَهُوَ آخِذٌ فِي طَرِيقِهِ.
\par 9 فَقَالَ لَهُ مِيخَا مِنْ أَيْنَ أَتَيْتَ. فَقَالَ لَهُ أَنَا لَاوِيٌّ مِنْ بَيْتِ لَحْمِ يَهُوذَا وَأَنَا ذَاهِبٌ لِكَيْ أَتَغَرَّبَ حَيْثُمَا اتَّفَقَ.
\par 10 فَقَالَ لَهُ مِيخَا أَقِمْ عِنْدِي وَكُنْ لِي أَبَاً وكّاهِنَاً وَأَنَا أُعْطِيكَ عَشَرَةَ شَوَاقِلِ فِضَّةٍ فِي السَّنَةِ وَحُلَّةَ ثِيَابٍ وَقُوتَكَ. فَذَهَبَ مَعَْهُ الْلاَّوِيُّ.
\par 11 فَرَضِيَ الْلَّاوِيُّ بِاْلإِقامَةِ مَعَ الْرَّجُلِ وكَانَ الْغُلامُ لَهُ كَأَحَدِ بَنِيهِ.
\par 12 فَمَلأَ مِيخَا يَدَ الْلَّاوِيَّ وَكَانَ الْغُلَامُ لَهُ كَاهِنَاً وَكَانَ فِي بَيْتِ مِيخَا.
\par 13 فَقَالَ مِيخَا الْآنَ عَلِمْتُ أَنَّ الْرَّبَّ يُحْسِنُ إِلَيَّ لِأَنَّهُ صَارَ لِيَ الْلاَّويّ كَاهِنَاً. الأَصْحَاحُ الْثَّامنُ عَشَرَ

\chapter{18}

\par 1 وَفِي تِلْكَ الْأَيَّامِ لَمْ يَكُنْ مَلِكٌ فِي إِسْرَائِيلَ. وَفِي تِلْكَ الْأَيَّامِ كَانَ سِبْطُ الْدَّانِيِيّنَ يَطْلُبُ لَهُ مُلْكاً للِسُّكْنَى. لِأَنَّهُ إِلَى ذلِكَ الْيَوْمِ لَمْ يَقَعْ لَهُ نَصِيبٌ فِي وَسَطِ أَسْبَاطِ إِسْرَائِيلَ.
\par 2 فَأَرْسَلَ بَنُو دَانَ مِنْ عَشِيرَتِهِمْ خَمْسَةَ رِجَالٍ مِنْهُمْ رِجَالاً بَنِي بَأْسٍ مِنْ صُرْعَةَ وَمِنْ أَشْتَأُولَ لِتَجَسُّسِ الْأَرْضِ وَفَحْصِهَا. وَقَالُوا لَهُمُ اْذْهَبُوا اْفْحَصُوا الْأَرْضَ. فَجَاءُوا إِلَى جَبَلِ أَفْرَايِمَ إِلَى بَيْتِ مِيخَا وَبَاتُوا هُنَاكَ.
\par 3 وَبَيْنَمَا هُمْ عِنْدَ بَيْتِ مِيخَا عَرَفُوا صَوْتَ الْغُلاَمِ الْلاَّوِيّ فَمَالُوا إِلَى هُنَاكَ وَقَالُوا لَهُ. مَنْ جَاءَ بِكَ إِلَى هُنَا وَمَاذَا أَنْتَ عَامِلٌ فِي هَذَا الْمَكَانِ وَمَا لَكَ هُنَا.
\par 4 فَقَالَ لَهُمْ كَذَا وَكَذَا عَمِلَ لِي مِيخَا وَقَدِ اسْتَأْجَرَنِي فَصِرْتُ لَهُ كَاهِناً.
\par 5 فَقَالُوا لَهُ اسْأَلْ إِذَنْ مِنَ اللّهِ لِنَعْلَمَ هَلْ يَنْجَحُ طَرِيقُنَا الَّذِي نَحْنُ سَائِرُونَ فِيهِ.
\par 6 فَقَالَ لَهُمْ الْكَاهِنُ اذْهَبُوا بِسَلاَمٍ. أَمَامَ الرَّبِّ طَرِيقُكُمُ الَّذِي تَسِيرُونَ فيهِ.
\par 7 فَذَهَبَ الْخَمْسَةُ الْرِّجَالِ وَجَاءُوا إِلَى لاَيِشَ وَرَأَوُا الشَّعْبَ الَّذِينَ فِيهَا سَاكِنِينَ بِطَمَأْنِينَةٍ كَعَادَةِ الصَّيْدُونِيِيّنَ مُسْتَرِيحِينَ مُطْمَئِنّينَ وَلَيْسَ فِي الْأَرْضِ مُؤْذٍ بِأَمْرٍ وَارِثٍ رِيَاسةً وَهُمْ بَعِيدُونَ عَنِ الصِّيدُونِيِيّنَ وَلَيْسَ لَهُمْ أَمْرٌ مَعَ إِنْسَانٍ.
\par 8 وَجَاءُوا إِلَى إِخْوَتِهِمْ إِلَى صُرْعَةَ وَأَشْتَأُولَ فَقَالَ لَهُمْ إِخْوَتُهُمْ مَا أَنْتُمْ.
\par 9 فَقَالُوا قُومُوا نَصْعَدْ إِلَيْهِمْ لِأَنَّنَا رَأَيْنَا الْأَرْضَ وَهُوَذَا هِيَ جَيِّدَةٌ جِدّاً وَأَنْتُمْ سَاكِتُونَ. لاَ تَتَكَاسَلُوا عَنِ الذَّهَابِ لِتَدْخُلُوا وَتَمْلُكُوا الْأَرْضَ.
\par 10 عِنْدَ مَجِيئِكُمْ تَأْتُونَ إِلَى شَعْبٍ مُطْمَئِنٍّ وَالْأَرْضُ وَاسِعَةُ الطَّرَفَيْنِ. إِنَّ الْلّهَ قَدْ دَفَعَهَا لِيَدِكُمْ. مَكَانٌ لَيْسَ فِيهِ عَوَزٌ لِشَيْءٍ ممَّا فِي الْأَرْضِ.
\par 11 فَارْتَحَلَ مِنْ هُنَاكَ مِنْ عَشِيرَةِ الْدانِيِيِّنَ مِنْ صُرْعَةَ وَمِنْ أَشْتَأُولَ سِتُّ مِئَةِ رَجُلٍ مُتَسَلِّحِينَ بِعُدَّةِ الْحَرْبِ.
\par 12 وَصَعَدُوا وَحَلُّوا فِي قَرْيَةِ يَعَارِيمَ فِي يَهُوذَا. لِذلِكَ دَعَوْا ذلِكَ الْمَكَانَ مَحَلَّةَ دَانٍَ إِلَى هذَا الْيَوْمِ. هُوَذَا هِيَ وَرَاءَ قَرْيَةِ يَعَارِيمَ.
\par 13 وَعَبَرُوا مِنْ هُنَاكَ إِلَى جَبَلِ افْرَايِمَ وَجَاءُوا إِلَى بَيْتِ مِيخَا.
\par 14 فَأَجابَ الْخَمْسَةُ الرِّجَالِ الَّذِينَ ذَهَبُوا لِتَجَسُّسِ أَرْضِ لاَيِشَ وَقَالُوا لِإِخْوَتِهِمْ أَتَعْلَمُونَ أَنَّ فِي هذِهِ الْبُيُوتِ أَفُوداً وَتَرَافِيمَ وَتِمْثَالاً مَنْحُوتاً مَسْبُوكاً. فَالْآنَ اعْلَمُوا مَا تَفْعَلُونَ.
\par 15 فَمَالُوا إِلَى هُنَاكَ وَجَاءُوا إِلَى بَيْتِ الْغُلاَمِ الْلاَّوِيِّ بَيْتِ مِيخَا وَسَلَّمُوا عَلَيْهِ.
\par 16 وَالْسِتُّ مِئَةِ الْرَّجُلِ الْمُتَسَلِّحُونَ بِعُدَّتِهِمْ لِلْحَرْبِ وَاقِفُونَ عِنْدَ مَدْخَلِ الْبَابِ. هؤُلاَءِ مِنْ بَنِي دَانَ.
\par 17 فَصَعِدَ الْخَمْسَةُ الْرِّجَالِ الَّذِينَ ذَهَبُوا لِتَجَسَّسِ اْلأَرْضِ وَدَخَلُوا إِلَى هُنَاكَ وَأَخَذُوا الْتِّمْثَالَ الْمَنْحُوتَ وَاُلأَفُودَ وَاْلتَّرافِيمَ واْلتِّمْثَالَ اْلمَسْبُوكَ. واْلكَاهِنُ وَاقِفٌ عِنْدَ مَدْخَلِ اْلبَابِ مَعَ اْلسِّتِّ مِئَةِ اْلرَّجُلِ الْمُتَسَلِّحِينَ بِعُدَّةِ الْحَرْبِ.
\par 18 وَهؤُلاَءِ دّخَلُوا بَيْتَ مِيخَا وَأَخَذُوا الْتِّمْثَالَ الْمَنْحُوتَ وَاْلأَفودَ واْلتَّرَافِيمَ واْلتِّمْثَالَ اْلمَسْبُوكَ. فَقَالَ لَهُمُ اْلكَاهِنُ مَاذَا تَفْعَلُونَ.
\par 19 فَقَالُوا لَهُ اْخْرَسْ. ضَعْ يَدَكَ عَلَى فَمِكَ وَاْذْهَبْ مَعَنَا وَكُنْ لَنَا أَبَاً وَكَاهِناً. أَهُوَ خَيْرٌ لَكَ أَنْ تَكُونَ كَاهِناً لِبَيْتِ رَجُلٍ وَاحِدٍ أَمْ أَنْ تَكُونَ كَاهِناً لِسِبْطٍ وَلِعَشِيرَةٍ فِي إِسْرَائِيلَ.
\par 20 فَطَابَ قَلْبُ اْلكَاهِنِ وَأَخَذَ اْلأَفُودَ وَاْلتَّرافِيمَ وَاْلتِّمْثَالَ اْلمَنْحُوتَ وَدَخَلَ فِي وَسْطِ اْلشَّعْبِ.
\par 21 ثُمَّ انْصَرَفُوا وَذَهَبُوا وَوَضَعُوا اْلأَطْفَالَ واْلمَاشِيَةَ وَاْلثَّقَلَ قُدَّامَهُم.
\par 22 وَلَمَّا ابْتَعَدُوا عَنْ بَيْتِ مِيخَا اجْتَمَعَ اْلرِّجَالُ اْلَّذِينَ فِي الْبُيُوتِ اْلَتي عِنْدَ بَيْتِ مِيخَا وَأَدْرَكُوا بَنِي دَانَ
\par 23 وصَاحُوا إِلَى بَنِي داَنٍ فَالْتَفَتُوا وَقَالُوا لِمِيخَا مَا لَكَ صّرَخَتْ.
\par 24 فَقَالَ. آلِهَتِي اْلَّتِي عَمِلَتُ قَدْ أَخَذْتُمُوهَا مَعَ اْلكَاهِنِ وَذَهَبْتُمْ فَمَاذَا لِي بَعْدُ. وَمَا هذَا تَقُولُونَ لِي مَا لَكَ.
\par 25 فَقَالَ لَهُ بَنُو دَانَ لَا تُسْمِّعْ صَوْتَكَ بَيْنَنَا لِئَلاَّ يَقَعَ بِكُمْ رِجَالٌ أَنْفُسُهُمْ مُرَّةٌ فَتَنْزَعَ نَفْسَكَ وَأَنْفُسَ بَيْتِكَ.
\par 26 وَسَارَ بَنُو دَانٍ فِي طَرِيقِهِمْ. وَلَمَّا رَأَى مِيخَا أَنَّهُمْ أَشَدُّ مِنْهُ انْصَرَفَ وَرَجَعَ إِلَى بَيْتِهِ.
\par 27 وَأَمَّا هُمْ فَأَخَذُوا مَا صَنَعَ مِيخَا والْكَاهِنُ اْلَّذِي كَانَ لَهُ وَجَاءُوا إِلَى لاَيِشَ إِلَى شَعْبٍ مُسْتَرِيحٍ مُطْمَئِنٍّ وَضَرَبُوهُمْ بِحَدِّ السَّيْفِ وَأَحْرَقُوا اْلمَدِينَةَ بِالنَّارِ.
\par 28 وَلَمْ يَكُنْ مَنْ يُنْقِذُ لِأَنَّهَا بَعِيدَةٌ عَنْ صِيدُونَ وَلَمْ يَكُنْ لَهُمْ أَمْرٌ مَعَ إِنْسَانٍ وَهِيَ فِي الْوَادِي الَّذِي لِبَيْتِ رَحُوبَ. فَبَنَوُا الْمَدِينَةَ وَسَكَنُوا بِها.
\par 29 وَدَعُوُا اسْمَ الْمَدِيْنَةِ دَانَ بِاسْمِ دَانَ أَبِيْهِمِ الَّذي وُلِدَ لِإِسْرائِيْلَ. وَلَكٍِنَّ اسْمَ الْمَدِينَةِ أَوَّلاً لاَيِشُ.
\par 30 وَأَقَامَ بَنُوا دَانَ لِأَنْفُسِهِمْ التِّمْثَالَ المَنْحُوتَ وَكَانَ يَهُونَاثَانُ ابْنُ جَرْشُومَ بِنْ مَنَسَّى هُوَ وَبَنُوهُ كَهَنَةً لِسِبْطِ الدَّانيِّيِنَ إِلَى يَوْمِ سَبْيِ الأَرْضِ.
\par 31 وَوَضَعُوا لِأَنْفُسِهِمْ تِمْثَالَ مِيخَا الْمَنْحُوتَ الَّذِي عَمِلَهُ كُلَّ الأَيَّامِ الَّتِي كَانَ فِيهَا بَيْتُ الَّلهِ فِي شِيلُوهَ. الْأَصحَاحُ الْتَّاسِعُ عَشَرَ

\chapter{19}

\par 1 وَفِي تِلْكَ اْلأَيَّامِ حِينَ لَمْ يَكُنْ مَلِكٌ فِي إِسْرَائِيلَ كَانَ رَجُلٌ لاَوِيٌّ مُتَغَرِّباً فِي عِقَابِ جَبَلِ أَفْرَايِمَ. فَاْتَّخَذَ لَهُ امْرَأَةً سُرِّيَّةً مِنْ بَيْتِ لَحْمَ يَهُوذَا.
\par 2 فَزَنَتْ عَلَيْهِ سُرِّيَّتُهُ وّذَهَبَتْ مِنْ عِنْدِهِ إِلَى بَيْتِ أَبِيهَا فِي بَيْتِ لَحْمِ يَهُوذَا وَكَانَتْ هُنَاكَ أَيَّاماً أَرْبَعَةَ أَشْهُرٍ.
\par 3 فَقَامَ رَجُلُهَا وَسَارَ وَرَاءَهَا لِيُطَيِّبَ قَلْبَها وَيَرُدَّهَا وَمَعَهُ غُلاَمُهُ وَحِمَارَانِ. فَأَدْخَلَتْهُ بَيْتَ أَبِيهَا. فَلَمَّا رَآهُ أَبُو الْفَتَاةِ فَرِحَ بِلِقَائِهِ.
\par 4 وَأَمْسَكَهُ حَمُوهُ أَبُو الْفَتَاةِ فَمَكَثَ مَعَهُ ثَلاَثَةَ أَيَّامٍ فَأَكَلُوا وَشَرِبُوا وَبَاتُوا هُنَاكَ.
\par 5 وَكَانَ فِي الْيَوْمِ الْرَّابِعِ أَنَّهُمْ بَكَّرُوا صَبَاحاً وَقَامَ لِلذَّهَابِ. فَقَالَ أَبُو الْفَتَاةِ لِصِهْرِهِ أَسْنِدْ قَلْبَكَ بِكِسْرَةِ خُبْزٍ وبَعْدُ تّذْهَبُونَ.
\par 6 فَجَلَسَا وَأَكَلاَ كِلاَهُمَا مَعاً وَشَرِبَا. وَقَالَ أَبُو الْفَتَاةِ لِلرَّجُلِ ارْتَضِ وَبِتْ وَلْيَطِبْ قَلْبُكَ.
\par 7 وَلَمَّا قَامَ الْرَّجُلُ لِلذَّهَابِ أَلَحَّ عَلَيْهِ حَمُوهُ فَعَادَ وَبَاتَ هُنَاكَ.
\par 8 ثُمَّ بَكَّرَ فِي الْغَدِ فِي الْيَوْمِ الْخَامِسِ لِلذَّهَابِ فَقَالَ أَبُو الْفَتَاةِ أَسْنِدْ قَلْبَكَ. وَتَوَانَوْا حَتَّى يَمِيلَ الْنَّهَارُ. وَأَكَلاَ كِلاَهُمَا.
\par 9 ثُمَّ قَامَ الرَّجُلُ لِلذَّهَابِ هُوَ وَسُرِيَّتُهُ وَغُلاَمُهُ فَقَالَ لَهُ حَمُوهُ أَبُو الْفَتَاةِ إِنَّ النَّهَارَ قَدْ مَالَ إِلَى الْغُرُوبِ. بِيتُوا الْآنَ. هُوَذَا آخِرُ الْنَّهَارِ. بِتْ هُنَا وَلْيَطِبْ قَلْبُكَ وَغَداً تُبَكِّرُونَ فِي طَرِيقِكُمْ وَتَذْهَبُ إِلَى خَيْمَتِكَ.
\par 10 فَلَمْ يُرِدِ الرَّجُلُ أَنْ يَبِيتَ بَلْ قَامَ وَذَهَبَ وَجَاءَ إِلَى مُقَابِلِ يَبُوسَ. هِيَ أُورُشَلِيمُ. وَمَعَهُ حِمَارَانِ مَشْدُودَانِ وَسُرِّيَتَهُ مَعَهُ.
\par 11 وفِيما هُمْ عِنْدَ يَبُوسَ وَالنَّهَارُ قَدِ انْحَدَرَ جِدّاً قَالَ الْغُلاَمُ لِسَيِّدِهِ تَعَالَ نَمِيلُ إِلَى مَدِينَةِ الْيَبُوسِييِّنَ هذِهِ وَنَبِيتَ فِيهَا.
\par 12 فَقَالَ لَهُ سَيِّدُهُ لاَ نَمِيلُ إِلَى مَدِينَةٍ غَرِيبَةٍ حَيْثُ لَيْسَ أَحَدٌ مِنْ بَنِي إِسْرَائِيلَ هُنَا. نَعْبُرُ إِلَى جِبْعَةَ.
\par 13 وَقَالَ لِغُلاَمِهِ تَعَالَ نَتَقَدَّمُ إِلَى أَحَدِ الأَمَاكِنِ وَنَبِيتَ فِي جِبْعَةَ أَوْ فِي الرَّامَةِ.
\par 14 فَعَبَرُوا وَذَهَبُوا وَغَابَتِ لَهُمُ الْشَّمْسُ عِنْدَ جِبْعَةَ الَّتِي لِبِنيَامِينَ.
\par 15 فَمَالُوا إِلَى هُنَاكَ لِكَيْ يَدْخُلُوا وَيَبِيتُوا فِي جِبْعَةَ. فَدَخَلَ وَجَلَسَ فِي سَاحَةِ الْمَدِينَةِ وَلَمْ يَضُمَّهُمْ أَحَدٌ إِلَى بَيْتِهِ لِلْمَبِيتِ.
\par 16 وَإِذَا بِرَجُلٍ شَيْخٍ جَاءَ مِنْ شُغْلِهِ مِنَ الْحَقْلِ عِنْدَ الْمَسَاءِ. وَالرَّجُلُ مِنْ جَبَلِ أَفْرَايِمَ وَهُوَ غَرِيبٌ فِي جِبْعَةَ وَرِجَالُ الْمَكَانِ بِنْيَامِينِيُّونَ.
\par 17 فَرَفَعَ عَيْنَيْهِ وَرَأَى الرَّجُلَ الْمُسَافِرَ فِي سَاحَةِ الْمَدِينَةِ فَقَالَ الرَّجُلُ الشَّيْخُ إِلَى أَيْنَ نَذْهَبُ وَمِنْ أَيْنَ أَتَيْتَ.
\par 18 فَقَالَ لَهُ نَحْنُ عَابِرُونَ مِنْ بَيْتِ لَحْمِ يَهُوذَا إِلَى عِقَابِ جَبَلِ أَفْرَايِمَ. أَنَا مِنْ هُنَاكَ وَقَدْ ذَهَبْتُ إِلَى بَيْتِ لَحْمِ يَهُوذَا وَأَنَا ذَاهِبٌ إِلَى بَيْتِ الرَّبِّ وَلَيْسَ أَحَدٌ يَضُمُّنِي إِلَى البَيْتِ.
\par 19 وَأَيْضاً عِنْدَنَا تِبْنٌ وَعَلَفٌ لِحَمِيرِنَا وَأَيْضاً خُبْزٌ وَخَمْرٌ لِي وَلِأمَتِكَ وَلِلْغُلاَمِ الَّذِي مَعَ عَبِيدِكَ لَيْسَ احْتِيَاجٌ إِلَى شَيْءٍ.
\par 20 فَقَالَ الرَّجُلُ الشَّيْخُ السَّلاَمُ لَكَ. إِنَّمَا كُلُّ احْتِيَاجِكَ عَلَيَّ وَلَكِنْ لاَ تَبِتْ فِي السَّاحَةِ.
\par 21 وَجَاءَ بِهِ إِلَى بَيْتِهِ وَعَلَفَ حَمِيرَهُمْ فَغَسَلُوا أَرْجُلَهُمْ وَأَكَلُوا وَشَرِبُوا.
\par 22 وَفِيمَا هُمْ يُطَيِّبُونَ قُلُوبَهُمْ إِذَا بِرِجَالِ الْمَدِينَةِ رِجَالَ بَلِيَّعَال أَحَاطُوا بِالْبَيْتِ قَارِعِينَ الْبَابَ وَكَلَّمُوا الرَّجُلَ صَاحِبَ الْبَيْتِ الشَيْخَ قَائِلِينَ أَخْرِجِ الرَّجُلَ الَّذِي دَخَلَ بَيْتَكَ فًَنَعْرِفُهُ.
\par 23 فَخَرَجَ إِلَيْهِمِ الرَّجُلُ صَاحِبُ الْبِيْتِ وَقَالَ لَهُمْ لاَ يَا إِخْوَتِي لاَ تَفْعَلُوا شَرّاً. بَعْدَمَا دَخَلَ هذَا الرَّجُلُ بَيْتِي لاَ تَفْعَلُوا هذِهِ الْقَبَاحَةَ.
\par 24 هُوَذَا ابْنَتِي الْعَذْرَاءُ وَسُرِّيَّتُهُ دَعُونِي أُخْرِجْهُمَا فَأَذِلُّوهُمَا وَافْعَلُوا بِهِمَا مَا يَحْسُنُ فِي أَعْيُنِكُمْ وَأَمَّا هذَا الرَّجُلُ فَلاَ تَعْمَلُوا بِهِ هذَا الأَمْرَ الْقَبِيحَ.
\par 25 فَلَمْ يُرِدِ الرِّجَالُ أَنْ يَسْمَعُوا لَهُ. فَأَمْسَكَ الرَّجُلُ سُرِّيَّتَهُ وَأَخْرَجَهَا إِلَيْهِمْ خَارِجاً فَعَرَفُوهَا وَتَعَلَّلُوا بِهَا الْلَّيْلَ كُلَّهُ إِلَى الصَّبَاحِ وَعِنْدَ طُلُوعِ الْفَجْرِ أَطْلَقُوهَا.
\par 26 فَجَاءَتْ الْمَرْأَةُ عِنْدَ إِقْبَالِ الصَّبَاحِ وَسَقَطَتْ عِنْدَ بَابِ بَيْتِ الرَّجُلِ حَيْثُ سَيِّدُهَا هُنَاكَ إِلَى الْضَوْءِ.
\par 27 فَقَامَ سَيِّدُهَا فِي الصَّبَاحِ وَفَتَحَ أَبْوَابَ الْبَيْتِ وَخَرَجَ لِلذَّهَابِ فِي طَرِيقِهِ وَإِذَا بِالْمَرْأَةِ سُرِّيَّتَهُ سَاقِطَةٌ عَلَى بَابِ الْبَيْتِ وَيَدَاهَا عَلَى الْعَتَبَةِ.
\par 28 فَقَالَ لَهَا قُومِي نَذْهَبْ. فَلَمْ يَكُنْ مُجِيبٌ. فَأَخَذَهَا عَلَى الْحِمَارِ وَقَامَ الرَّجُلُ وَذَهَبَ إِلَى مَكَانِهِ.
\par 29 وَدَخَلَ بَيْتَهُ وَأَخَذَ الْسِكَِينَ وَأَمْسَكَ سُرِّيَّتَهُ وَقَطَّعَهَا مَعَ عِظَامِهَا إِلَى اثْنَتَيْ عَشَْرَةَ قِطْعَةً وَأَرْسَلَهَا إِلَى جَمِيعِ تُخُومِ إِسْرَائِيلَ.
\par 30 وَكُلُّ مَنْ رَأَى قَالَ لَمْ يُرَ مِثْلُ هذَا مِنْ يَوْمِ صُعُودِ بَنِي إِسْرَائِيلَ مِنْ أَرْضِ مِصْرَ إِلَى هذَا الْيَوْمِ. تَبَصَّرُوا فِيهِ وَتَشَاوَرُوا وَتَكَلَّمُوا. الأَصحَاحُ الْعِشْرُونَ

\chapter{20}

\par 1 فَخَرَجَ جَمِيعُ بَنِي إِسْرَائِيلَ وَاجْتَمَعَتِ الْجَمَاعَةُ كَرَجُلٍ وَاحِدٍ مِنْ دَانَ إِلَى بِئْرِ سَبْعٍ مَعْ أَرْضِ جِلْعَادَ إِلَى الرَّبِّ فِي الْمِصْفَاةِ.
\par 2 وَوَقَفَ وُجُوهُ جَمِيعِ الشَّعْبِ جَمِيعُ أَسْبَاطِ إِسْرَائِيلَ فِي مَجْمَعِ شَعْبِ اللهِ أَرْبَعُ مِئَةِ أَلْفِ رَاجِلٍ مُخْتَرِطي السَّيْفِ.
\par 3 فَسَمِعَ بَنُو بِنْيَامِينَ أَنَّ بَنِي إِسْرَائِيلَ قَدْ صَعِدُوا إِلَى الْمِصْفَاةِ. وَقَالَ بَنُو إِسْرَائِيلَ تَكَلَّمُوا. كَيْفَ كَانَتْ هذِهِ الْقُبَاحَةُ.
\par 4 فَأَجَابَ الرَّجُلُ اللاَّوِيُّ بَعْلُ الْمَرْأَةِ الْمَقْتُولَةِ وَقَالَ دَخَلْتُ أَنَا وَسُرِيَّتِي إِلَى جِبْعَةَ الَتِي لِبِنْيَامِينَ لِنَبِيتَ.
\par 5 فَقَامَ عَلَيَّ أَصْحَابُ جِبْعَةَ وَأَحَاطُوا عَلَيَّ بِالْبَيْتِ لَيْلاً وَهَمُّوا بِقَتْلِي وَأَذَلُّوا سُرِّيَّتِي حَتَّى مَاتَتْ.
\par 6 فَأَمْسَكْتُ سُرِّيَّتِي وَقَطَّعْتُها وَأَرْسَلْتُهَا إِلَى جَمِيعِ حُقُولِ مُلْكِ إِسْرَائِيلَ. لِأَنَّهُمْ فَعَلُوا رَذَالَةً وَقَبَاحَةً فِي إِسْرَائِيلَ.
\par 7 هُوَذَا كُلُّكُمْ بَنُو إِسْرَائِيلَ هَاتُوا حُكْمَكُمْ وَرَأْيَكُمْ ههُنَا.
\par 8 فَقَامَ جَمِيعُ الْشَّعْبِ كَرَجُلٍ وَاحِدٍ وَقَالُوا لاَ يَذْهَبُ أَحَدٌ مِنَّا إِلَى خَيْمَتِهِ وَلاَ يَمِيلُ أَحَدٌ إِلَى بِيْتِهِ.
\par 9 وَالْآنَ هذَا هُوَ الْأَمْرُ الَّذِي نَعْمَلُهُ بَجِبْعَةَ. عَلَيْهَا بِالْقُرْعَةِ.
\par 10 فَنَأْخُذُ عَشْرَةَ رِجَالٍ مِنَ الْمِئَةِ مِنْ جَمِيعِ أَسْبَاطِ إِسْرَائِيلَ وَمِئَةً مِنَ الْأَلْفِ وَأَلْفاً مِنَ الْرَّبْوَةِ لِأَجْلِ أَخْذِ زَادٍ لِلشَّعْبِ لِيَفْعَلُوا عِنْدَ دُخُولِهِمْ جِبْعَةَ بِبِنْيَامِينَ حَسَبَ كُلِّ الْقَبَاحَةِ الَّتِي فَعَلَتْ بِإِسْرَائِيلَ.
\par 11 فَاجْتَمَعَ جَمِيعُ رِجَالِ إِسْرَائِيلَ عَلَى الْمَدِينَةِ مُتَّحِدِينَ كَرَجُلٍ وَاحِدٍ.
\par 12 وَأَرْسَلَ أَسْبَاطُ إِسْرَائِيلَ رِجَالاً إِلَى جَمِيعِ أَسْبَاطِ بَنْيَامِينَ قَائِلِينَ مَا هذَا الشَّرُّ الَّذِي صَارَ فِيكُمْ.
\par 13 فَالْآنَ سَلِّمُوا الْقَوْمَ بَنِي بَلِيَّعَالَ الَّذِينَ فِي جِبْعَةَ لِكَيْ نَقْتُلَهُمْ وَنَنْزَعَ الشَّرَّ مِنْ إِسْرَائِيلَ. فَلَمْ يُرِدْ بَنُو بَنْيَامِينَ أَنْ يَسْمَعُوا لِصَوْتِ إِخْوَتِهِمْ بَنِي إِسْرَائِيلَ.
\par 14 فَاجْتَمَعَ بَنُو بَنْيَامِينَ مِنَ الْمُدُنِ إِلَى جِبْعَةَ لِكَيْ يَخْرُجُوا لِمُحَارَبَةِ بَنِي إِسْرَائِيلَ.
\par 15 وَعُدَّ بَنُو بَنْيَامِينَ فِي ذلِكَ الْيَوْمِ مِنَ الْمُدُنِ سِتَّةً وَعِشْرِينَ أَلْفَ رَجُلٍ مُخْتَرِطِي السَّيْفَ مَا عَدَا سُكَّانَ جِبْعَةَ عُدُّوا سَبْعَ مِئَةِ رَجُلٍ مُنْتَخَبِينَ.
\par 16 مِنْ جَمِيعِ هذَا الشَّعْبِ سَبْعَ مِئَةِ رَجُلٍ مُنْتَخَبُونَ عُسْرٌ. كُلُّ هؤُلاَءِ يَرْمُونَ الْحَجَرَ بِالْمِقْلاَعِ عَلَى الشَّعْرَةِ وَلا يُخْطِئُونَ.
\par 17 وَعُدَّ رِجَالُ إِسْرَائِيلَ مَا عَدَا بَنْيَامِينَ أَرْبَعَ مِئَةِ أَلْفِ رَجُلٍ مُخْتَرِطِي السَّيْفِ. كُلُّ هؤُلاَءِ رِجَالُ حَرْبٍ.
\par 18 فَقَامُوا وَصَعِدُوا إِلَى بَيْتِ إِيلَ وَسَأَلُوا اللهَ وَقَالَ بَنُو إِسْرَائِيلَ مَنْ يَصْعَدُ مِنَّا أَوَّلاً لِمُحَارَبَةِ بَنِي بَنْيَامِينَ. فَقَالَ الرَّبُّ يًهُوذَا أَوَّلاً.
\par 19 فَقَامَ بَنُو إِسْرَائِيلَ فِي الصَّبَاحِ وَنَزَلُوا عَلَى جِبْعَةَ.
\par 20 وَخَرَجَ رِجَالُ إِسْرَائِيلَ لِمُحَارَبَةِ بَنْيَامِينَ وَصَفَّ رِجَالُ إِسْرَائِيلَ أَنْفُسَهُمْ لِلْحَرْبِ عِنْدَ جِبْعَةَ.
\par 21 فَخَرَجَ بَنُو بَنيَامِينَ مِنْ جِبْعَةَ وَأَهْلَكُوا مِنْ إِسْرَائِيلَ فِي ذَلِكَ الْيَوْمِ اثنَيْنِ وَعِشْرِينَ أَلْفَ رَجُلٍ إِلَى الْأَرْضِ.
\par 22 وَتَشَدَّدَ الشَّعْبُ رِجَالُ إِسْرَائِيلَ وَعَادُوا فَاصْطَفُّوا لِلْحَرْبِ فِي الْمَكَانِ الَّذِي اصْطَفُّوا فِيهِ فِي اليَوْمِ الأَوَّلِ.
\par 23 ثُمَّ صَعِدَ بَنُو إِسْرَئِيلَ وَبَكَوْا أَمَامَ الرَّبِّ إِلَى الْسَّمَاءِ وَسَأَلُوا الرَّبَّ قَائِلِينَ هَلْ أَعُودُ أَتَقَدَّمُ لِمُحَارَبَةِ بَنِي بَنْيَامِينَ أَخِي. فَقَالَ الرَّبُّ اصْعَدُوا إِلَيْهِ.
\par 24 فَتَقَدَّمَ بَنُو إِسْرَائِيلَ إِلَى بَنِي بَنْيَامِينَ فِي الْيَوْمِ الثَّانِي.
\par 25 فَخَرَجَ بَنْيَامِينُ لِلِقَائِهِم مِنْ جِبْعَةَ فِي اليَوْمِ الثَّانِي وَأَهْلَكَ مِنْ بَنِي إِسْرَائِيلَ أَيْضاً ثَمَانِيَةَ عَشَرَ أَلْفَ رَجُلٍ إِلَى الْأَرْضِ. كُلُّ هؤُلاَءِ مُخْتَرِطُو السَّيْفِ.
\par 26 فَصَعِدَ جَمِيعُ بَنِي إِسْرَائِيلَ وَكُلُّ الشَّعْبِ وَجَاءُوا إِلَى بَيْتِ إِيلَ وَبَكَوْا وَجَلَسُوا هُنَاكَ أَمَامَ الرَّبِّ وَصَامُوا ذلِكَ الْيَوْمَ إِلَى الْمَسَاءِ وَأَصْعَدُوا مُحْرِقَاتٍ وَذَبَائِحَ سَلاَمَةً أَمَامَ الرَّبِّ.
\par 27 وَسَأَلَ بَنُو إِسْرَائِيلَ الرَّبَّ. وَهُنَاكَ تَابُوتُ عَهْدِ اللهِ فِي تِلْكَ الْأَيَّامِ.
\par 28 وَفِينْحَاسُ بْنُ أَلِيعَازَرَ بْنِ هرُونَ وَاقِفٌ أَمَامَهُ فِي تِلْكَ الْأَيَّامِ. قَائِلينَ أَأَعُودُ أَيْضاً لِلْخُرُوجِ لِمُحَارَبَةِ بَنِي بَنْيَامِينَ أَخِي أَمْ أَكُفُّ. فَقَالَ اصْعَدُوا لِأَنِّي غَدَاً أَدْفَعُهُمْ لِيَدِكَ.
\par 29 وَوَضَعَ إِسْرَائِيلُ كَمِيناً عَلَى جِبْعَةَ مُحِيطاً.
\par 30 وَصَعِدَ بَنُو إِسْرَائِيلَ عَلَى بَنِي بَنْيَامِينَ فِي الْيَوْمِ الثَّالِثِ وَاصْطَفُّوا عِنْدَ جِبْعَةَ كَالْمَرَّةِ الْأُولَى وَالثَّانِيَةِ.
\par 31 فَخَرَجَ بَنُو بَنْيَامِينَ لِلِقَاءِ الشَّعْبِ وَانْجَذَبُوا عَنِ الْمَدِينَةِ وَأَخَذُوا يَضْرِبُونَ مِنَ الشَّعْبِ قَتْلَى كَالْمَرَّةِ الأُولَى وَالثَّانِيَةِ فِي السِّكَكِ الَّتِي إِحْدَاهَا تَصْعَدُ إِلَى بَيْتِ إِيلَ وَالْأُخْرَى إِلَى جِبْعَةَ فِي الْحَقْلِ نَحْوَ ثَلاَثِينَ رَجُلاً مِنْ إِسْرَائِيلَ.
\par 32 وَقَالَ بَنُو بَنْيَامِينَ مُنْهَزِمُونَ أَمَامَنَا كَمَا فِي الْأَوَّلِ. وَأَمَّا بَنُو إِسْرَائِيلَ فَقَالُوا لِنَهْرُبَ وَنَجْذُبَهُمْ عَنِ الْمَدِينَةِ إِلَى السِكَكِ.
\par 33 وَقَامَ جَمِيعُ رِجَالِ إِسْرَائِيلَ مِنْ أَمَاكِنِهِم وَاصْطَفُّوا فِي بَعْلِ تَامَارَ وَثَارَ كَمِينُ إِسْرائِيْلَ مِنْ مَكَانِهِ مِنْ عَرَاءِ جِبْعَةَ.
\par 34 وَجَاءَ مِنْ مُقَابِلِ جِبْعَةَ عَشَرَةُ آلافِ رَجُلٍ مُنْتَخَبُونَ مِنْ كُلِ إِسْرَائِيلَ وَكَانَتْ الْحَرْبُ شَدِيدَةً وَهُمْ يَعْلَمُونَ أَنَّ الشَّرَّ قَدْ مَسَّهُمْ.
\par 35 فَضَرَبَ الرَّبُّ بَنْيَامِينَ أَمَامَ إِسْرَائِيلَ وَأَهْلَكَ بَنُو إِسْرَائِيلَ فِي ذلِكَ الْيَوْمِ خَمْسَةً وَعِشْرِينَ أَلْفَ رَجُلٍ ومِئَةَ رَجُلٍ. كُلُّ هؤُلاَءِ مُخْتَرِطُو السَّيْفِ.
\par 36 وَرَأَى بَنُو بَنْيَامِينَ أَنَّهُمْ قَدِ انْكَسَرُوا. وَأَعْطَى رِجَالُ إِسْرَائِيلَ مَكَاناً لِبَنْيَامِينَ لِأَنَّهُم اتَّكَلُوا عَلَى الْكَمِينِ الَّذِي وَضَعُوهُ عَلَى جِبْعَةَ.
\par 37 فَأَسْرَعَ الْكَمِينُ وَاقْتَحَمُوا جِبْعَةَ وَزَحَفَ الْكَمِينُ وَضَرَبَ الْمَدِينَةَ كُلَّهَا بِحَدِّ السَّيْفِ.
\par 38 وَكَانَ الْمِيعَادُ بَيْنَ رِجَالِ إِسْرَائِيلَ وَبَيْنَ الْكَمِينِ إِصْعَادَهُمْ بِكَثْرَةٍ عَلاَمَةَ الدُّخَانِ مِنَ الْمَدِينَةِ.
\par 39 وَلَمَّا انْقَلَبَ رِجَالُ إِسْرَائِيلَ فِي الْحَرْبِ ابْتَدَأَ بَنْيَامِينُ يَضْرِبُونَ قَتْلَى مِنْ رِجَالِ إِسْرَائِيلَ نَحْوَ ثَلاَثِينَ رَجُلاً لِأَنَّهُمْ قَالُوا إِنَّمَا هُمْ مُنْهَزِمُونَ مِنْ أَمَامِنَا كَالْحَرْبِ الْأُولَى.
\par 40 وَلَمَّا ابْتَدَأَتِ الْعَلاَمَةُ تَصْعَدُ مِنَ الْمَدِينَةِ عَمُودَ دُخَانٍ الْتَفَتَ بَنْيَامِينُ إِلَى وَرَائِهِ وَإِذَا بِالْمَدِينَةِ كُلِّهَا تَصْعَدُ نَحْوَ السَّمَاءِ.
\par 41 وَرَجَعَ رِجَالُ إِسْرَائِيلَ وَهَرَبَ رِجَالُ بَنْيَامِينَ بِرَعْدَةٍ لِأَنَّهُمْ رَأَوْا أَنَّ الشَّرَّ قَدْ مَسَّهُمْ.
\par 42 وَرَجَعُوا أَمَامَ بَنِي إِسْرَئِيلَ فِي طَرِيقِ الْبَرِّيَّةِ وَلكِنَّ الْقِتَالَ أَدْرَكَهُمْ وَالَّذِينَ مِنَ الْمُدُنِ أَهْلَكُوهُمْ فِي وَسْطِهِمْ.
\par 43 فَحَاوَطُوا بَنْيَامِينَ ثَمَانِيَةَ عَشَرَ أَلْفَ رَجُلٍ جَمِيعَ هؤُلاَءِ ذَوُو بَأْسٍ.
\par 44 فَدَارُوا وَهَرَبُوا إِلَى الْبَرِيَّةِ إِلَى صَخْرَةِ رِمُّونَ.
\par 45 فَالْتَقَطُوا مِنْهُم فِي السِّكَكِ خَمْسَةَ آلاَفِ رَجُلٍ وَشَدُّوا وَرَاءَهُمْ إِلَى جِدْعُومَ وَقَتَلُوا مِنْهُم أَلْفَيْ رَجُلٍ.
\par 46 وَكَانَ جَمِيعُ السَّاقِطِينَ مِنْ بَنْيَامِينَ خَمْسَةً وَعِشْرِينَ أَلْفَ رَجُلٍ مُخْتَرِطِي السَّيْفِ فِي ذلِكَ الْيَوْمِ. جَمِيعُ هؤُلاَءِ ذَوُو بَأْسٍ.
\par 47 وَدَارَ وَهَرَبَ إِلَى الْبَرِّيَّةِ إِلَى صَخْرَةِ رِمُّونَ سِتُّ مِئَةِ رَجُلٍ وَأَقَامُوا فِي صَخْرَةِ رِمُّونَ أَرْبَعَةَ أَشْهُرٍ.
\par 48 وَرَجَعَ رِجَالُ بَنِي إِسْرَائِيلَ إِلَى بَنِي بَنْيَامِينَ وَضَرَبُوهُمْ بِحَدِّ السَّيْفِ مِنَ الْمَدِينَةِ بِأَسْرِهَا حَتَّى الْبَهَائِمَ حَتَّى كُلَّ مَا وُجِدَ وَأَيْضاً جَمِيعُ الْمُدُنِ الَّتِي وُجِدَتْ أَحْرَقُوهَا بِالنَّارِ. الأَصْحَاحُ الْحَادِي وَالْعِشْرُونَ

\chapter{21}

\par 1 وَرَجَالُ إِسْرَائِيلَ حَلَفُوا فِي الْمِصْفَاةِ قَائِلِينَ لاَ يُسَلِّمْ أَحَدٌ مِنَّا ابْنَتَهُ لِبِنْيَامِينَ امْرَأَةً.
\par 2 وَجَاءَ الشَّعْبُ إِلَى بَيْتِ إِيلَ وَأَقَامُوا هُنَاكَ إِلَى الْمَسَاءِ أَمَامَ اللهِ وَرَفَعُوا صَوْتَهُمْ وَبَكَوْا بُكَاءً عَظِيماً.
\par 3 وَقَالُوا لِمَاذَا يَا رَبُّ إلهَ إِسْرَائِيلَ حَدَثَتْ هذِهِ فِي إِسْرَائِيلَ حَتَّى يُفْقَدُ الْيَوْمَ مِنْ إِسْرَائِيلَ سِبْطٌ.
\par 4 وَفِي الْغَدِ بَكَّرَ الشَّعْبُ وَبَنَوْا هُنَاكَ مَذْبَحاً وَأَصْعَدُوا مُحْرَقَاتٍ وَذَبَائِحَ سَلاَمَةٍ.
\par 5 وَقَالَ بَنُو إِسْرَائِيلَ مَنْ هُوَ الّذِي لَمْ يَصْعَدْ فِي الْمَجْمَعِ مِنْ جَمِيعِ أَسْبَاطِ إِسْرَائِيلَ إِلَى الرَّبِّ. لِأَنَّهُ صَارَ الْحَلَفُ الْعَظِيمُ عَلَى الَّذِي لَمْ يَصْعِدْ إِلَى الرَّبِّ إِلَى الْمِصْفَاةِ قَائِلاً يُمَاتُ مَوْتاً.
\par 6 وَنَدِمَ بَنُو إِسْرَائِيلَ عَلَى بَنْيَامِينَ أَخِيهِمْ وَقَالُوا قَدِ انْقَطَعَ الْيَوْمَ سِبْطٌ وَاحِدٌ مِنْ إِسْرَائِيلَ.
\par 7 مَاذَا نَعْمَلُ لِلْبَاقِينَ مِنْهُمْ فِي أَمْرِ النِّسَاءِ وَقَدْ حَلَفْنَا نَحْنُ بِالرَّبِّ أَنْ لاَ نُعْطِيَهُمْ مِنْ بَنَاتِنَا نِسَاءً.
\par 8 وَقَالُوا أَيُّ سِبْطٍ مِنْ أَسْبَاطِ إِسْرَائِيلَ لَمْ يَصْعَدْ إِلَى الرَّبِّ إِلَى الْمِصْفَاةِ.
\par 9 وَهُوَذَا لَمْ يَأْتِ إِلَى الْمَحَلَّةِ رَجُلٌ مِنْ جِلْعَادَ.
\par 10 فَأَرْسَلَتِ الْجَمَاعَةُ إِلَى هُنَاكَ اثْنَيْ عَشَرَ أَلْفَ رَجُلٍ مِنْ بَنِي الْبَأْسِ وَأَوْصُوهُمْ قَائِلِينَ اذْهَبُوا وَاضْرُبُوا سُكَّانَ جِلْعَادَ بِحَدِّ السَّيْفِ مَعَ النِّسَاءِ وَالأَطْفَالِ.
\par 11 وَهذَا مَا تَعْمَلُونَهُ. تُحَرِّمُونَ كُلَّ ذَكَرٍ وَكُلَّ امْرَأَةٍ عَرَفَت اضْطِجَاعَ ذَكّرٍ.
\par 12 فَوَجَدُوا يَابِيشِ جِلْعَادَ أَرْبَعَ مِئَةِ فَتَاةٍ عذَارَى لَمْ يَعْرِفْنَ رَجُلاً بِالاضْطِجَاعِ مَعَ ذَكَرٍ وَجَاءُوا بِهِنَّ إِلَى الْمَحَلَّةِ إِلَى شِيلُوهَ الَّتِي فِي أَرْضِ كَنْعَانَ.
\par 13 وَأَرْسَلَتِ الْجَمَاعَةُ كُلُّهَا وَكَلَّمَتْ بَنِي بَنْيَامِينَ الَّذِينَ فِي صَخْرَةِ رِمُّونَ وَاسْتَدْعَتْهُمْ إِلَى الصُّلْحِ.
\par 14 فَرَجَعَ بَنْيَامِينُ فِي ذلِكَ الْوَقْتِ فَأَعْطَوْهُم النِّسَاءَ اللَّوَاتِي اسْتَحْيَوْهُنَّ مِنْ نِسَاءِ يَابِيشِ جِلْعَادَ وَلَمْ يَكْفُوهُمْ هكَذَا.
\par 15 وَنَدِمَ الشَّعْبُ مِنْ أَجْلِ بَنْيَامِينَ لِأَنَّ الرَّبَّ جَعَلَ شَقّاً فِي إِسْرَائِيلَ.
\par 16 فَقَالَ شُيُوخُ الْجَمَاعَةِ مَاذَا نَصْنَعُ بِالْبَاقِينَ فِي أَمْرِ النِّسَاءِ لِأَنَّه قَدِ انْقَطَعَتِ النِّسَاءُ مِنْ بَنْيَامِينَ.
\par 17 وَقَالُوا مِيْرَاثُ نَجَاةٍ لِبَنْيَامِينَ وَلاَ يُمْحَى سِبْطٌ مِنْ إِسْرَائِيلَ.
\par 18 وَنَحْنُ لاَ نَقْدُرُ أَنْ نُعْطِيَهُم نِسَاءً مِنْ بَنَاتِنَا لِأَنَّ بَنِي إِسْرَائِيلَ حَلَفُوا قَائِلِينَ مَلْعُونٌ مَنْ أَعْطَى امْرَأَةً لِبَنْيَامِينَ.
\par 19 ثُمَّ قَالُوا هُوَذَا عِيدُ الرَّبِّ فِي شِيلُوهَ مِنْ سَنَةٍ إِلَى سَنَةٍ شِمَالِيَّ بَيْتِ إِيلَ شَرْقِيَّ الطَّرِيقِ الصَّاعِدَةِ مِنْ بَيْتِ إِيلَ إِلَى شَكِيمَ وَجَنُوبِيِّ لَبُونَةَ.
\par 20 وَأَوْصَوْا بَنِي بَنْيَامِينَ قَائِلِينَ امْضُوا واكْمُنُوا فِي الْكُرُومِ.
\par 21 وانْظُرُوا فَإِذَا خَرَجَتْ بَنَاتُ شِيلُوهَ لِيَدُرْنَ فِي الرَّقْصِ فَاخْرُجُوا أَنْتُمْ مِنَ الْكُرُومِ وَاخْطُفُوا لِأَنْفُسِكُمْ كُلُّ وَاحِدٍ امْرَأَتَهُ مِنْ بَنَاتِ شِيلُوهَ وَاذْهَبُوا إِلَى أَرْضِ بَنْيَامِينَ.
\par 22 فَإِذَا جَاءَ آبَاؤُهُنَّ أَوْ إِخْوَتُهُنَّ لِكَيْ يَشْكُوا إِلَيْنَا نَقُولُ لَهُمْ تَرَاءَفُوا عَلَيْهِمْ لِأَجْلِنَا لِأَنَّنَا لَمْ نَجِدْ لِكُلِّ وَاحِدٍ امْرَأَتَهُ فِي الْحَرْبِ لِأَنَّكُمْ أَنْتُمْ لَمْ تُعْطُوهُمْ فِي الْوَقْتِ حَتَّى تَكُونُوا قَدْ أَثِمْتُمْ.
\par 23 فَفَعَل هكَذَا بَنُو بَنْيَامِينَ وَاتَّخَذُوا نِسَاءً حَسَبَ عَدِدِهِمْ مِنَ الرَّاقِصَاتِ اللَّوَاتِي اخْتَطَفُوهُنَّ وَذَهَبُوا وَرَجَعُوا إِلَى مُلْكِهِمْ وَبَنَوُا الْمُدُنَ وَسَكَنُوا بِهَا.
\par 24 فَسَارَ مِنْ هُنَاكَ بَنُو إِسْرَائِيلَ فِي ذلِكَ الْوَقْتِ كُلُّ وَاحِدٍ إِلَى سِبْطِهِ وَعَشِرَتِهِ وَخَرَجُوا مِنْ هُنَاكَ كُلُّ وَاحِدٍ إِلَى مُلْكِهِ.
\par 25 فِي تِلْكَ الْأَيَّامِ لَمْ يَكُنْ مَلِكٌ فِي إِسْرَائِيلَ. كُلُّ وَاحِدٍ عَمِلَ مَا حَسُنَ فِي عَيْنَيْهِ.


\end{document}