\begin{document}

\title{عهد دان}

\chapter{1}

\par \textit{الابن السابع ليعقوب وبلهة. الغيور. ينصح بعدم الغضب قائلاً إنه "يعطي رؤية غريبة". هذه أطروحة بارزة عن الغضب.}

\par 1 نسخة كلام دان الذي قاله لأبنائه في أيامه الأخيرة، في السنة المائة والخامسة والعشرين من حياته

\par 2 لأنه جمع عائلته وقال: اسمعوا لكلامي يا بني دان، وأصغوا لكلام أبيكم

\par 3 لقد أثبتُ في قلبي، وطوال حياتي، أن الصدق مع التعامل العادل أمرٌ جيدٌ ومرضيٌّ لله، وأن الكذب والغضب شرّان، لأنهما يُعلّمان الإنسان كل شر

\par 4 لذلك، أعترف لكم اليوم يا أبنائي، أنني قررت في قلبي موت يوسف أخي، الرجل الحقيقي والصالح

\par 5 وفرحتُ لأنه بِيعَ، لأن أباه كان يُحِبُّه أكثر منّا

\par 6 لأن روح الغيرة والكبرياء قال لي: وأنت أيضًا ابنه

\par 7 وأيقظني أحد أرواح بليعار قائلاً: خذ هذا السيف، واقتل به يوسف. هكذا يحبك أبوك بعد موته

\par 8 هذه هي روح الغضب التي أقنعتني بسحق يوسف كما يسحق النمر جديًا

\par 9 لكن إله آبائي لم يدعه يقع في يدي، حتى أجده وحيدًا وأقتله، وأُهلك سبطًا ثانيًا في إسرائيل

\par 10 والآن يا أبنائي، ها أنا أموت، وأقول لكم الحقيقة: إن لم تحفظوا أنفسكم من روح الكذب والغضب، وتحبوا الحق وطول الأناة، فستهلكون

\par 11 لأن الغضب عمى، ولا يسمح برؤية وجه أي إنسان بالحق

\par 12 لأنه حتى لو كان أبًا أو أمًا، فإنه يتصرف معهم كأعداء؛ حتى لو كان أخًا، فإنه لا يعرفه؛ حتى لو كان نبيًا للرب، فإنه يعصيه؛ حتى لو كان رجلاً بارًا، فإنه لا يعتبره؛ حتى لو كان صديقًا، فإنه لا يعترف به

\par 13 لأن روح الغضب تُحيط به بشبكة الخداع، وتُعمي عينيه، وبالكذب تُظلم عقله، وتمنحه رؤياه الخاصة

\par 14 وبماذا يُحيط به؟ ببغضة القلب، حتى يحسد أخاه

\par 15 لأن الغضب أمر شرير يا أبنائي، لأنه يزعج حتى الروح نفسها

\par 16 ويسيطر على جسد الغاضب، ويسيطر على روحه، ويمنح الجسد قوة ليفعل كل إثم

\par 17 وعندما يفعل الجسد كل هذه الأشياء، فإن النفس تبرر ما تم فعله، لأنها لا ترى بشكل صحيح

\par 18 لذلك، فإن من يغضب، إذا كان جبارًا، لديه قوة ثلاثية في غضبه: واحدة بمساعدة خدامه؛ والثانية بغناه، الذي به يقنع ويتغلب ظلماً؛ وثالثًا، بما لديه من قوة طبيعية، فإنه يفعل الشر من خلاله

\par 19 ورغم أن الرجل الغاضب ضعيف، إلا أنه يمتلك قوة مضاعفة عن القوة الطبيعية؛ لأن الغضب يساعد هؤلاء دائمًا على الإثم.

\par 20 هذه الروح ترافق دائمًا الكذب عن يمين الشيطان، لكي تتم أعماله بالقسوة والكذب

\par 21 فافهموا إذًا سلطان الغضب أنه باطل.

\par 22 فإنه يثير الغضب بالكلام أولاً، ثم بالأفعال يقوي الغاضب، ويزعزع عقله بخسائر فادحة، وبالتالي يثير غضبه الشديد.

\par 23 لذلك، عندما يتكلم أحد ضدكم، فلا تغضبوا، وإذا مدحكم أحد كأشخاص مقدسين، فلا ترتفعوا. لا تتحركوا إما للابتهاج أو للاشمئزاز

\par 24 لأنه في البداية يُرضي السمع، وبالتالي يجعل العقل حريصًا على إدراك أسباب الاستفزاز؛ ثم عندما يغضب، يعتقد أنه غاضب بحق

\par 25 إذا وقعتم في أي خسارة أو دمار، يا أبنائي، فلا تتضايقوا؛ لأن هذه الروح ذاتها تجعل الإنسان يرغب في ما هو فانٍ، لكي يغضب من البلاء

\par 26 وإن تعرضتم لخسارة طوعًا أو كرهًا، فلا تغضبوا؛ لأن الغضب ينشأ عنه غضب بالكذب

\par 27 علاوة على ذلك، فإن الشر المزدوج هو الغضب مع الكذب؛ وهما يساعدان بعضهما البعض على إزعاج القلب؛ وعندما تكون النفس مضطربة باستمرار، يبتعد عنها الرب، ويسود عليها بليعار



\chapter{2}

\par \textit{نبوءة عن الخطايا، والأسر، والأوبئة، والعودة النهائية للأمة. ما زالوا يتحدثون عن عدن (انظر الآية 18). الآية 23 رائعة في ضوء النبوة.}

\par 1 فاحفظوا يا أبنائي وصايا الرب، واحفظوا شريعته. ابتعدوا عن الغضب، وأبغضوا الكذب، لكي يحل الرب في وسطكم، ويهرب منكم بليعار

\par 2 تكلموا بالحق كل واحد مع قريبه. لئلا تقعوا في الغضب والارتباك، بل تكونوا في سلام، مع إله السلام، فلا تقوى عليكم حرب

\par 3 أحبوا الرب طوال حياتكم، وأحبوا بعضكم البعض بقلب صادق

\par 4 أعلم أنكم في آخر الأيام ستبتعدون عن الرب، وستثيرون غضب لاوي، وتحاربون يهوذا. لكنكم لن تقوى عليهما، لأن ملاك الرب يهديهما كليهما، لأن إسرائيل بهما يقوم

\par 5 ومتى ابتعدتم عن الرب، تسلكون في كل شر، وتفعلون رجاسات الأمم، وتزنون وراء نساء الأشرار، بينما تعمل فيكم أرواح الشر بكل شر

\par 6 لأني قرأت في سفر أخنوخ البار أن رئيسكم هو الشيطان، وأن كل أرواح الشر والكبرياء ستتآمر لتلاحق أبناء لاوي باستمرار، لتجعلهم يخطئون أمام الرب

\par 7 وسيقترب أبنائي من لاوي، ويخطئون معهم في كل شيء، وسيكون بنو يهوذا طماعين، ينهبون أموال الآخرين كالأسود

\par 8 لذلك ستؤخذون معهم إلى السبي، وهناك تأخذون جميع ضربات مصر، وجميع شرور الأمم.

\par 9 وهكذا عندما ترجعون إلى الرب، تنالون رحمة، ويدخلكم إلى قدسه، ويمنحكم السلام

\par 10 ويقوم لكم من سبط يهوذا ومن سبط لاوي خلاص الرب، ويحارب بليعار

\par 11 وينفذ انتقامًا أبديًا على أعدائنا، ويأخذ من بليعار أسرى نفوس القديسين، ويرد القلوب المتمردة إلى الرب، ويمنح الذين يدعونه السلام الأبدي

\par 12 وسيستريح القديسون في عدن، وفي أورشليم الجديدة يبتهج الأبرار، وستكون لمجد الله إلى الأبد

\par 13 ولن تبقى أورشليم خرابًا بعد، ولا يُؤخذ إسرائيل سبيًا؛ لأن الرب سيكون في وسطها [يعيش بين الناس]، وقدوس إسرائيل سيملك عليها بتواضع وفقر؛ ومن يؤمن به سيملك بين الناس بالحق

\par 14 والآن، اتقوا الرب يا أبنائي، واحذروا الشيطان وأرواحه

\par 15 اقتربوا من الله ومن الملاك الذي يشفع لكم، لأنه وسيط بين الله والإنسان، ومن أجل سلام إسرائيل سيقف ضد مملكة العدو

\par 16 لذلك يسعى العدو إلى تدمير كل من يدعو الرب.

\par 17 لأنه يعلم أنه في اليوم الذي يتوب فيه إسرائيل، تنتهي مملكة العدو.

\par 18 لأن ملاك السلام نفسه يُقوي إسرائيل، فلا يسقط في أقصى الشر

\par 19 ويكون في زمن فساد إسرائيل أن الرب لا يفارقهم، بل يحولهم إلى أمة تعمل مشيئته، لأنه لا يوجد من الملائكة من هو مساوٍ له

\par 20 ويكون اسمه في كل مكان في إسرائيل وبين الأمم

\par 21 لذلك، يا أبنائي، احفظوا أنفسكم من كل عمل شرير، واطرحوا الغضب وكل كذب، وأحبوا الصدق وطول الأناة

\par 22 وما سمعتموه من أبيكم، أنقلوه أيضًا إلى أولادكم لكي يقبلكم مخلص الأمم، لأنه صادق وطويل الأناة، وديع ومتواضع، ويعلم بأعماله ناموس الله

\par 23 لذلك، انصرفوا عن كل إثم، والزموا ببر الله، فيخلص جنسكم إلى الأبد

\par 24 وادفنوني عند آبائي.

\par 25 ولما قال هذه الأشياء قبلهم، ونام في شيخوخته الصالحة

\par 26 فدفنه أبناؤه، وبعد ذلك حملوا عظامه ووضعوها بالقرب من إبراهيم وإسحاق ويعقوب

\par 27 ومع ذلك، تنبأ لهم دان أنهم سينسون إلههم، وسيُطردون من أرض ميراثهم، ومن نسل إسرائيل، ومن عشيرة نسلهم



\end{document}