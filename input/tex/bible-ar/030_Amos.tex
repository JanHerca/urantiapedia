\begin{document}

\title{عاموس}


\chapter{1}

\par 1 أَقْوَالُ عَامُوسَ الَّذِي كَانَ بَيْنَ الرُّعَاةِ مِنْ تَقُوعَ الَّتِي رَآهَا عَنْ إِسْرَائِيلَ فِي أَيَّامِ عُزِّيَّا مَلِكِ يَهُوذَا وَفِي أَيَّامِ يَرُبْعَامَ بْنِ يُوآشَ مَلِكِ إِسْرَائِيلَ قَبْلَ الزَّلْزَلَةِ بِسَنَتَيْنِ.
\par 2 فَقَالَ: «إِنَّ الرَّبَّ يُزَمْجِرُ مِنْ صِهْيَوْنَ وَيُعْطِي صَوْتَهُ مِنْ أُورُشَلِيمَ فَتَنُوحُ مَرَاعِي الرُّعَاةِ وَيَيْبَسُ رَأْسُ الْكَرْمَلِ».
\par 3 هَكَذَا قَالَ الرَّبُّ: «مِنْ أَجْلِ ذُنُوبِ دِمَشْقَ الثَّلاَثَةِ وَالأَرْبَعَةِ لاَ أَرْجِعُ عَنْهُ لأَنَّهُمْ دَاسُوا جِلْعَادَ بِنَوَارِجَ مِنْ حَدِيدٍ.
\par 4 فَأُرْسِلُ نَاراً عَلَى بَيْتِ حَزَائِيلَ فَتَأْكُلُ قُصُورَ بَنْهَدَدَ.
\par 5 وَأُكَسِّرُ مِغْلاَقَ دِمَشْقَ وَأَقْطَعُ السَّاكِنَ مِنْ بُقْعَةِ آوَنَ وَمَاسِكَ الْقَضِيبِ مِنْ بَيْتِ عَدْنٍ وَيُسْبَى شَعْبُ أَرَامَ إِلَى قِيرَ» قَالَ الرَّبُّ.
\par 6 هَكَذَا قَالَ الرَّبُّ: «مِنْ أَجْلِ ذُنُوبِ غَزَّةَ الثَّلاَثَةِ وَالأَرْبَعَةِ لاَ أَرْجِعُ عَنْهُ لأَنَّهُمْ سَبُوا سَبْياً كَامِلاً لِيُسَلِّمُوهُ إِلَى أَدُومَ.
\par 7 فَأُرْسِلُ نَاراً عَلَى سُورِ غَزَّةَ فَتَأْكُلُ قُصُورَهَا.
\par 8 وَأَقْطَعُ السَّاكِنَ مِنْ أَشْدُودَ وَمَاسِكَ الْقَضِيبِ مِنْ أَشْقَلُونَ وَأَرُدُّ يَدِي عَلَى عَقْرُونَ فَتَهْلَِكُ بَقِيَّةُ الْفِلِسْطِينِيِّينَ» قَالَ السَّيِّدُ الرَّبُّ.
\par 9 هَكَذَا قَالَ الرَّبُّ: «مِنْ أَجْلِ ذُنُوبِ صُورَ الثَّلاَثَةِ وَالأَرْبَعَةِ لاَ أَرْجِعُ عَنْهُ لأَنَّهُمْ سَلَّمُوا سَبْياً كَامِلاً إِلَى أَدُومَ وَلَمْ يَذْكُرُوا عَهْدَ الإِخْوَةِ.
\par 10 فَأُرْسِلُ نَاراً عَلَى سُورِ صُورَ فَتَأْكُلُ قُصُورَهَا».
\par 11 هَكَذَا قَالَ الرَّبُّ: «مِنْ أَجْلِ ذُنُوبِ أَدُومَ الثَّلاَثَةِ وَالأَرْبَعَةِ لاَ أَرْجِعُ لأَنَّهُ تَبِعَ بِالسَّيْفِ أَخَاهُ وَأَفْسَدَ مَرَاحِمَهُ وَغضَبُهُ إِلَى الدَّهْرِ يَفْتَرِسُ وَسَخَطُهُ يَحْفَظُهُ إِلَى الأَبَدِ.
\par 12 فَأُرْسِلُ نَاراً عَلَى تَيْمَانَ فَتَأْكُلُ قُصُورَ بُصْرَةَ».
\par 13 هَكَذَا قَالَ الرَّبُّ: «مِنْ أَجْلِ ذُنُوبِ بَنِي عَمُّونَ الثَّلاَثَةِ وَالأَرْبَعَةِ لاَ أَرْجِعُ عَنْهُ لأَنَّهُمْ شَقُّوا حَوَامِلَ جِلْعَادَ لِيُوَسِّعُوا تُخُومَهُمْ.
\par 14 فَأُضْرِمُ نَاراً عَلَى سُورِ رَبَّةَ فَتَأْكُلُ قُصُورَهَا. بِجَلَبَةٍ فِي يَوْمِ الْقِتَالِ بِنَوْءٍ فِي يَوْمِ الزَّوْبَعَةِ.
\par 15 وَيَمْضِي مَلِكُهُمْ إِلَى السَّبْيِ هُوَ وَرُؤَسَاؤُهُ جَمِيعاً» قَالَ الرَّبُّ.

\chapter{2}

\par 1 هَكَذَا قَالَ الرَّبُّ: «مِنْ أَجْلِ ذُنُوبِ مُوآبَ الثَّلاَثَةِ وَالأَرْبَعَةِ لاَ أَرْجِعُ عَنْهُ لأَنَّهُمْ أَحْرَقُوا عِظَامَ مَلِكِ أَدُومَ كِلْساً.
\par 2 فَأُرْسِلُ نَاراً عَلَى مُوآبَ فَتَأْكُلُ قُصُورَ قَرْيُوتَ وَيَمُوتُ مُوآبُ بِضَجِيجٍ بِجَلَبَةٍ بِصَوْتِ الْبُوقِ.
\par 3 وَأَقْطَعُ الْقَاضِيَ مِنْ وَسَطِهَا وَأَقْتُلُ جَمِيعَ رُؤَسَائِهَا مَعَهُ» قَالَ الرَّبُّ.
\par 4 هَكَذَا قَالَ الرَّبُّ: «مِنْ أَجْلِ ذُنُوبِ يَهُوذَا الثَّلاَثَةِ وَالأَرْبَعَةِ لاَ أَرْجِعُ عَنْهُ لأَنَّهُمْ رَفَضُوا نَامُوسَ اللَّهِ وَلَمْ يَحْفَظُوا فَرَائِضَهُ وَأَضَلَّتْهُمْ أَكَاذِيبُهُمُ الَّتِي سَارَ آبَاؤُهُمْ وَرَاءَهَا.
\par 5 فَأُرْسِلُ نَاراً عَلَى يَهُوذَا فَتَأْكُلُ قُصُورَ أُورُشَلِيمَ».
\par 6 هَكَذَا قَالَ الرَّبُّ: «مِنْ أَجْلِ ذُنُوبِ إِسْرَائِيلَ الثَّلاَثَةِ وَالأَرْبَعَةِ لاَ أَرْجِعُ عَنْهُ لأَنَّهُمْ بَاعُوا الْبَارَّ بِالْفِضَّةِ وَالْبَائِسَ لأَجْلِ نَعْلَيْنِ.
\par 7 الَّذِينَ يَتَهَمَّمُونَ تُرَابَ الأَرْضِ عَلَى رُؤُوسِ الْمَسَاكِينِ وَيَصُدُّونَ سَبِيلَ الْبَائِسِينَ وَيَذْهَبُ رَجُلٌ وَأَبُوهُ إِلَى صَبِيَّةٍ وَاحِدَةٍ حَتَّى يُدَنِّسُوا اسْمَ قُدْسِي.
\par 8 وَيَتَمَدَّدُونَ عَلَى ثِيَابٍ مَرْهُونَةٍ بِجَانِبِ كُلِّ مَذْبَحٍ وَيَشْرَبُونَ خَمْرَ الْمُغَرَّمِينَ فِي بَيْتِ آلِهَتِهِمْ.
\par 9 وَأَنَا قَدْ أَبَدْتُ مِنْ أَمَامِهِمِ الأَمُورِيَّ الَّذِي قَامَتُهُ مِثْلُ قَامَةِ الأَرْزِ وَهُوَ قَوِيٌّ كَالْبَلُّوطِ. أَبَدْتُ ثَمَرَهُ مِنْ فَوْقُ وَأُصُولَهُ مِنْ تَحْتُ.
\par 10 وَأَنَا أَصْعَدْتُكُمْ مِنْ أَرْضِ مِصْرَ وَسِرْتُ بِكُمْ فِي الْبَرِّيَّةِ أَرْبَعِينَ سَنَةً لِتَرِثُوا أَرْضَ الأَمُورِيِّ.
\par 11 وَأَقَمْتُ مِنْ بَنِيكُمْ أَنْبِيَاءَ وَمِنْ فِتْيَانِكُمْ نَذِيرِينَ. أَلَيْسَ هَكَذَا يَا بَنِي إِسْرَائِيلَ يَقُولُ الرَّبُّ؟
\par 12 لَكِنَّكُمْ سَقَيْتُمُ النَّذِيرِينَ خَمْراً وَأَوْصَيْتُمُ الأَنْبِيَاءَ قَائِلِينَ: لاَ تَتَنَبَّأُوا.
\par 13 هَئَنَذَا أَضْغَطُ مَا تَحْتَكُمْ كَمَا تَضْغَطُ الْعَجَلَةُ الْمَلْآنَةُ حُزَماً.
\par 14 وَيَبِيدُ الْمَنَاصُ عَنِ السَّرِيعِ وَالْقَوِيُّ لاَ يُشَدِّدُ قُوَّتَهُ وَالْبَطَلُ لاَ يُنَجِّي نَفْسَهُ.
\par 15 وَمَاسِكُ الْقَوْسِ لاَ يَثْبُتُ وَسَرِيعُ الرِّجْلَيْنِ لاَ يَنْجُو وَرَاكِبُ الْخَيْلِ لاَ يُنَجِّي نَفْسَهُ.
\par 16 وَالْقَوِيُّ الْقَلْبِ بَيْنَ الأَبْطَالِ يَهْرُبُ عُرْيَاناً فِي ذَلِكَ الْيَوْمِ» يَقُولُ الرَّبُّ.

\chapter{3}

\par 1 اِسْمَعُوا هَذَا الْقَوْلَ الَّذِي تَكَلَّمَ بِهِ الرَّبُّ عَلَيْكُمْ يَا بَنِي إِسْرَائِيلَ عَلَى كُلِّ الْقَبِيلَةِ الَّتِي أَصْعَدْتُهَا مِنْ أَرْضِ مِصْرَ قَائِلاً:
\par 2 «إِيَّاكُمْ فَقَطْ عَرَفْتُ مِنْ جَمِيعِ قَبَائِلِ الأَرْضِ لِذَلِكَ أُعَاقِبُكُمْ عَلَى جَمِيعِ ذُنُوبِكُمْ».
\par 3 هَلْ يَسِيرُ اثْنَانِ مَعاً إِنْ لَمْ يَتَوَاعَدَا؟
\par 4 هَلْ يُزَمْجِرُ الأَسَدُ فِي الْوَعْرِ وَلَيْسَ لَهُ فَرِيسَةٌ؟ هَلْ يُعْطِي شِبْلُ الأَسَدِ زَئِيرَهُ مِنْ خِدْرِهِ إِنْ لَمْ يَخْطُفْ؟
\par 5 هَلْ يَسْقُطُ عُصْفُورٌ فِي فَخِّ الأَرْضِ وَلَيْسَ لَهُ شَرَكٌ؟ هَلْ يُرْفَعُ فَخٌّ عَنِ الأَرْضِ وَهُوَ لَمْ يُمْسِكْ شَيْئاً؟
\par 6 أَمْ يُضْرَبُ بِالْبُوقِ فِي مَدِينَةٍ وَالشَّعْبُ لاَ يَرْتَعِدُ؟ هَلْ تَحْدُثُ بَلِيَّةٌ فِي مَدِينَةٍ وَالرَّبُّ لَمْ يَصْنَعْهَا؟
\par 7 إِنَّ السَّيِّدَ الرَّبَّ لاَ يَصْنَعُ أَمْراً إِلاَّ وَهُوَ يُعْلِنُ سِرَّهُ لِعَبِيدِهِ الأَنْبِيَاءِ.
\par 8 الأَسَدُ قَدْ زَمْجَرَ فَمَنْ لاَ يَخَافُ؟ السَّيِّدُ الرَّبُّ قَدْ تَكَلَّمَ فَمَنْ لاَ يَتَنَبَّأُ؟
\par 9 نَادُوا عَلَى الْقُصُورِ فِي أَشْدُودَ وَعَلَى الْقُصُورِ فِي أَرْضِ مِصْرَ وَقُولُوا: «اجْتَمِعُوا عَلَى جِبَالِ السَّامِرَةِ وَانْظُرُوا شَغَباً عَظِيماً فِي وَسَطِهَا وَمَظَالِمَ فِي دَاخِلِهَا.
\par 10 فَإِنَّهُمْ لاَ يَعْرِفُونَ أَنْ يَصْنَعُوا الاِسْتِقَامَةَ يَقُولُ الرَّبُّ. أُولَئِكَ الَّذِينَ يَخْزِنُونَ الظُّلْمَ وَالاِغْتِصَابَ فِي قُصُورِهِمْ».
\par 11 لِذَلِكَ هَكَذَا قَالَ السَّيِّدُ الرَّبُّ: «ضِيقٌ حَتَّى فِي كُلِّ نَاحِيَةٍ مِنَ الأَرْضِ فَيُنْزِلَ عَنْكِ عِزَّكِ وَتُنْهَبُ قُصُورُكِ».
\par 12 هَكَذَا قَالَ الرَّبُّ: «كَمَا يَنْزِعُ الرَّاعِي مِنْ فَمِ الأَسَدِ كُرَاعَيْنِ أَوْ قِطْعَةَ أُذُنٍ هَكَذَا يُنْتَزَعُ بَنُو إِسْرَائِيلَ الْجَالِسُونَ في السَّامِرَةِ في زَاوِيَةِ السَّرِيرِ وَعَلَى دِمَقْسِ الْفِرَاشِ!
\par 13 اِسْمَعُوا وَاشْهَدُوا علَى بَيتِ يَعْقُوبَ يَقُولُ السّيِّدُ الرَّبُّ إِلَهُ الْجُنُودِ
\par 14 إِنِّي يَوْمَ مُعَاقَبَتِي إِسْرَائِيلَ عَلَى ذُنُوبِهِ أُعَاقِبُ مَذَابِحَ بَيْتَِ إِيلَ فَتُقْطَعُ قُرُونُ الْمَذْبَحِ وَتَسْقُطُ إِلَى الأَرْضِ.
\par 15 وَأَضْرِبُ بَيْتَ الشِّتَاءِ مَعَ بَيْتِ الصَّيْفِ فَتَبِيدُ بُيُوتُ الْعَاجِ وَتَضْمَحِلُّ الْبُيُوتُ الْعَظِيمَةُ يَقُولُ الرَّبُّ».

\chapter{4}

\par 1 اِسْمَعِي هَذَا الْقَوْلَ يَا بَقَرَاتِ بَاشَانَ الَّتِي فِي جَبَلِ السَّامِرَةِ الظَّالِمَةَ الْمَسَاكِينَِ السَّاحِقَةَ الْبَائِسِينَ الْقَائِلَةَ لِسَادَتِهَا: «هَاتِ لِنَشْرَبَ».
\par 2 قَدْ أَقْسَمَ السَّيِّدُ الرَّبُّ بِقُدْسِهِ: «هُوَذَا أَيَّامٌ تَأْتِي عَلَيْكُنَّ يَأْخُذُونَكُنَّ بِخَزَائِمَ وَذُرِّيَّتَكُنَّ بِشُصُوصِ السَّمَكِ.
\par 3 وَمِنَ الشُّقُوقِ تَخْرُجْنَ كُلُّ وَاحِدَةٍ عَلَى وَجْهِهَا وَتَنْدَفِعْنَ إِلَى الْحِصْنِ» يَقُولُ الرَّبُّ.
\par 4 «هَلُمَّ إِلَى بَيْتَِ إِيلَ وَأَذْنِبُوا إِلَى الْجِلْجَالِ وَأَكْثِرُوا الذُّنُوبَ وَأَحْضِرُوا كُلَّ صَبَاحٍ ذَبَائِحَكُمْ وَكُلَّ ثَلاَثَةِ أَيَّامٍ عُشُورَكُمْ.
\par 5 وَأَوْقِدُوا مِنَ الْخَمِيرِ تَقْدِمَةَ شُكْرٍ وَنَادُوا بِنَوَافِلَ وَسَمِّعُوا. لأَنَّكُمْ هَكَذَا أَحْبَبْتُمْ يَا بَنِي إِسْرَائِيلَ» يَقُولُ السَّيِّدُ الرَّبُّ.
\par 6 «وَأَنَا أَيْضاً أَعْطَيْتُكُمْ نَظَافَةَ الأَسْنَانِ فِي جَمِيعِ مُدُنِكُمْ وَعَوَزَ الْخُبْزِ فِي جَمِيعِ أَمَاكِنِكُمْ فَلَمْ تَرْجِعُوا إِلَيَّ يَقُولُ الرَّبُّ.
\par 7 وَأَنَا أَيْضاً مَنَعْتُ عَنْكُمُ الْمَطَرَ إِذْ بَقِيَ ثَلاَثَةُ أَشْهُرٍ لِلْحَصَادِ وَأَمْطَرْتُ عَلَى مَدِينَةٍ وَاحِدَةٍ وَعَلَى مَدِينَةٍ أُخْرَى لَمْ أُمْطِرْ. أُمْطِرَ عَلَى ضَيْعَةٍ وَاحِدَةٍ وَالضَّيْعَةُ الَّتِي لَمْ يُمْطَرْ عَلَيْهَا جَفَّتْ.
\par 8 فَجَالَتْ مَدِينَتَانِ أَوْ ثَلاَثٌ إِلَى مَدِينَةٍ وَاحِدَةٍ لِتَشْرَبَ مَاءً وَلَمْ تَشْبَعْ فَلَمْ تَرْجِعُوا إِلَيَّ يَقُولُ الرَّبُّ.
\par 9 ضَرْبَتُكُمْ بِاللَّفْحِ وَالْيَرَقَانِ. كَثِيراً مَا أَكَلَ الْقَمَصُ جَنَّاتِكُمْ وَكُرُومَكُمْ وَتِينَكُمْ وَزَيْتُونَكُمْ فَلَمْ تَرْجِعُوا إِلَيَّ يَقُولُ الرَّبُّ.
\par 10 أَرْسَلْتُ بَيْنَكُمْ وَبَأً عَلَى طَرِيقَةِ مِصْرَ. قَتَلْتُ بِالسَّيْفِ فِتْيَانَكُمْ مَعَ سَبْيِ خَيْلِكُمْ وَأَصْعَدْتُ نَتَنَ مَحَالِّكُمْ حَتَّى إِلَى أُنُوفِكُمْ فَلَمْ تَرْجِعُوا إِلَيَّ يَقُولُ الرَّبُّ.
\par 11 قَلَبْتُ بَعْضَكُمْ كَمَا قَلَبَ اللَّهُ سَدُومَ وَعَمُورَةَ فَصِرْتُمْ كَشُعْلَةٍ مُنْتَشَلَةٍ مِنَ الْحَرِيقِ فَلَمْ تَرْجِعُوا إِلَيَّ يَقُولُ الرَّبُّ.
\par 12 «لِذَلِكَ هَكَذَا أَصْنَعُ بِكَ يَا إِسْرَائِيلُ. فَمِنْ أَجْلِ أَنِّي أَصْنَعُ بِكَ هَذَا فَاسْتَعِدَّ لِلِقَاءِ إِلَهِكَ يَا إِسْرَائِيلُ».
\par 13 فَإِنَّهُ هُوَذَا الَّذِي صَنَعَ الْجِبَالَ وَخَلَقَ الرِّيحَ وَأَخْبَرَ الإِنْسَانَ مَا هُوَ فِكْرُهُ الَّذِي يَجْعَلُ الْفَجْرَ ظَلاَماً وَيَمْشِي عَلَى مَشَارِفِ الأَرْضِ يَهْوَهُ إِلَهُ الْجُنُودِ اسْمُهُ.

\chapter{5}

\par 1 اِسْمَعُوا هَذَا الْقَوْلَ الَّذِي أَنَا أُنَادِي بِهِ عَلَيْكُمْ مَرْثَاةً يَا بَيْتَ إِسْرَائِيلَ.
\par 2 سَقَطَتْ عَذْرَاءُ إِسْرَائِيلَ. لاَ تَعُودُ تَقُومُ. انْطَرَحَتْ عَلَى أَرْضِهَا لَيْسَ مَنْ يُقِيمُهَا.
\par 3 لأَنَّهُ هَكَذَا قَالَ السَّيِّدُ الرَّبُّ: «الْمَدِينَةُ الْخَارِجَةُ بِأَلْفٍ يَبْقَى لَهَا مِئَةٌ وَالْخَارِجَةُ بِمِئَةٍ يَبْقَى لَهَا عَشَرَةٌ مِنْ بَيْتِ إِسْرَائِيلَ».
\par 4 لأَنَّهُ هَكَذَا قَالَ الرَّبُّ لِبَيْتِ إِسْرَائِيلَ: «اطْلُبُوا فَتَحْيُوا.
\par 5 وَلاَ تَطْلُبُوا بَيْتَ إِيلَ وَإِلَى الْجِلْجَالِ لاَ تَذْهَبُوا وَإِلَى بِئْرَِ سَبْعٍ لاَ تَعْبُرُوا. لأَنَّ الْجِلْجَالَ تُسْبَى سَبْياً وَبَيْتَ إِيلَ تَصِيرُ عَدَماً.
\par 6 اُطْلُبُوا الرَّبَّ فَتَحْيُوا لِئَلاَّ يَقْتَحِمَ بَيْتَ يُوسُفَ كَنَارٍ تُحْرِقُ وَلاَ يَكُونُ مَنْ يُطْفِئُهَا مِنْ بَيْتَِ إِيلَ
\par 7 يَا أَيُّهَا الَّذِينَ يُحَوِّلُونَ الْحَقَّ أَفْسَنْتِيناً وَيُلْقُونَ الْبِرَّ إِلَى الأَرْضِ».
\par 8 اَلَّذِي صَنَعَ الثُّرَيَّا وَالْجَبَّارَ وَيُحَوِّلُ ظِلَّ الْمَوْتِ صُبْحاً وَيُظْلِمُ النَّهَارَ كَاللَّيْلِ. الَّذِي يَدْعُو مِيَاهَ الْبَحْرِ وَيَصُبُّهَا عَلَى وَجْهِ الأَرْضِ يَهْوَهُ اسْمُهُ.
\par 9 الَّذِي يُفْلِحُ الْخَرِبَ عَلَى الْقَوِيِّ فَيَأْتِي الْخَرَابُ عَلَى الْحِصْنِ.
\par 10 إِنَّهُمْ فِي الْبَابِ يُبْغِضُونَ الْمُنْذِرَ وَيَكْرَهُونَ الْمُتَكَلِّمَ بِالصِّدْقِ.
\par 11 لِذَلِكَ مِنْ أَجْلِ أَنَّكُمْ تَدُوسُونَ الْمِسْكِينَ وَتَأْخُذُونَ مِنْهُ هَدِيَّةَ قَمْحٍ بَنَيْتُمْ بُيُوتاً مِنْ حِجَارَةٍ مَنْحُوتَةٍ وَلاَ تَسْكُنُونَ فِيهَا وَغَرَسْتُمْ كُرُوماً شَهِيَّةً وَلاَ تَشْرَبُونَ خَمْرَهَا.
\par 12 لأَنِّي عَلِمْتُ أَنَّ ذُنُوبَكُمْ كَثِيرَةٌ وَخَطَايَاكُمْ وَافِرَةٌ أَيُّهَا الْمُضَايِقُونَ الْبَارَّ الْآخِذُونَ الرَّشْوَةَ الصَّادُّونَ الْبَائِسِينَ فِي الْبَابِ.
\par 13 لِذَلِكَ يَصْمُتُ الْعَاقِلُ فِي ذَلِكَ الزَّمَانِ لأَنَّهُ زَمَانٌ رَدِيءٌ.
\par 14 اُطْلُبُوا الْخَيْرَ لاَ الشَّرَّ لِتَحْيُوا فَعَلَى هَذَا يَكُونُ الرَّبُّ إِلَهُ الْجُنُودِ مَعَكُمْ كَمَا قُلْتُمْ.
\par 15 أَبْغِضُوا الشَّرَّ وَأَحِبُّوا الْخَيْرَ وَثَبِّتُوا الْحَقَّ فِي الْبَابِ لَعَلَّ الرَّبَّ إِلَهَ الْجُنُودِ يَتَرَأَّفُ عَلَى بَقِيَّةِ يُوسُفَ.
\par 16 لِذَلِكَ هَكَذَا قَالَ السَّيِّدُ الرَّبُّ إِلَهُ الْجُنُودِ: «فِي جَمِيعِ الأَسْوَاقِ نَحِيبٌ وَفِي جَمِيعِ الأَزِقَّةِ يَقُولُونَ: آهِ! آهِ! وَيَدْعُونَ الْفَلاَّحَ إِلَى النَّوْحِ وَجَمِيعَ عَارِفِي الرِّثَاءِ لِلنَّدْبِ.
\par 17 وَفِي جَمِيعِ الْكُرُومِ نَدْبٌ لأَنِّي أَعْبُرُ فِي وَسَطِكَ قَالَ الرَّبُّ.
\par 18 «وَيْلٌ لِلَّذِينَ يَشْتَهُونَ يَوْمَ الرَّبِّ. لِمَاذَا لَكُمْ يَوْمُ الرَّبِّ هُوَ ظَلاَمٌ لاَ نُورٌ؟
\par 19 كَمَا إِذَا هَرَبَ إِنْسَانٌ مِنْ أَمَامِ الأَسَدِ فَصَادَفَهُ الدُّبُّ أَوْ دَخَلَ الْبَيْتَ وَوَضَعَ يَدَهُ عَلَى الْحَائِطِ فَلَدَغَتْهُ الْحَيَّةُ!
\par 20 أَلَيْسَ يَوْمُ الرَّبِّ ظَلاَماً لاَ نُوراً وَقَتَاماً وَلاَ نُورَ لَهُ؟
\par 21 «بَغَضْتُ كَرِهْتُ أَعْيَادَكُمْ وَلَسْتُ أَلْتَذُّ بِاعْتِكَافَاتِكُمْ.
\par 22 إِنِّي إِذَا قَدَّمْتُمْ لِي مُحْرَقَاتِكُمْ وَتَقْدِمَاتِكُمْ لاَ أَرْتَضِي وَذَبَائِحَ السَّلاَمَةِ مِنْ مُسَمَّنَاتِكُمْ لاَ أَلْتَفِتُ إِلَيْهَا.
\par 23 أَبْعِدْ عَنِّي ضَجَّةَ أَغَانِيكَ وَنَغْمَةَ رَبَابِكَ لاَ أَسْمَعُ.
\par 24 وَلْيَجْرِ الْحَقُّ كَالْمِيَاهِ وَالْبِرُّ كَنَهْرٍ دَائِمٍ.
\par 25 «هَلْ قَدَّمْتُمْ لِي ذَبَائِحَ وَتَقْدِمَاتٍ فِي الْبَرِّيَّةِ أَرْبَعِينَ سَنَةً يَا بَيْتَ إِسْرَائِيلَ؟
\par 26 بَلْ حَمَلْتُمْ خَيْمَةَ مَلْكُومِكُمْ وَتِمْثَالَ أَصْنَامِكُمْ نَجْمَ إِلَهِكُمُ الَّذِي صَنَعْتُمْ لِنُفُوسِكُمْ.
\par 27 فَأَسْبِيكُمْ إِلَى مَا وَرَاءَ دِمَشْقَ قَالَ الرَّبُّ إِلَهُ الْجُنُودِ اسْمُهُ».

\chapter{6}

\par 1 وَيْلٌ لِلْمُسْتَرِيحِينَ فِي صِهْيَوْنَ وَالْمُطْمَئِنِّينَ فِي جَبَلِ السَّامِرَةِ نُقَبَاءِ أَوَّلِ الأُمَمِ. يَأْتِي إِلَيْهِمْ بَيْتُ إِسْرَائِيلَ.
\par 2 اُعْبُرُوا إِلَى كَلْنَةَ وَانْظُرُوا وَاذْهَبُوا مِنْ هُنَاكَ إِلَى حَمَاةَ الْعَظِيمَةِ ثُمَّ انْزِلُوا إِلَى جَتِّ الْفِلِسْطِينِيِّينَ. أَهِيَ أَفْضَلُ مِنْ هَذِهِ الْمَمَالِكِ أَمْ تُخُمُهُمْ أَوْسَعُ مِنْ تُخُمِكُمْ.
\par 3 أَنْتُمُ الَّذِينَ تُبْعِدُونَ يَوْمَ الْبَلِيَّةِ وَتُقَرِّبُونَ مَقْعَدَ الظُّلْمِ
\par 4 الْمُضْطَجِعُونَ عَلَى أَسِرَّةٍ مِنَ الْعَاجِ وَالْمُتَمَدِّدُونَ عَلَى فُرُشِهِمْ وَالْآكِلُونَ خِرَافاً مِنَ الْغَنَمِ وَعُجُولاً مِنْ وَسَطِ الصِّيرَةِ
\par 5 الْهَاذِرُونَ مَعَ صَوْتِ الرَّبَابِ الْمُخْتَرِعُونَ لأَنْفُسِهِمْ آلاَتِ الْغِنَاءِ كَدَاوُدَ
\par 6 الشَّارِبُونَ مِنْ كُؤُوسِ الْخَمْرِ وَالَّذِينَ يَدَّهِنُونَ بِأَفْضَلِ الأَدْهَانِ وَلاَ يَغْتَمُّونَ عَلَى انْسِحَاقِ يُوسُفَ.
\par 7 لِذَلِكَ الْآنَ يُسْبَوْنَ فِي أَوَّلِ الْمَسْبِيِّينَ وَيَزُولُ صِيَاحُ الْمُتَمَدِّدِينَ.
\par 8 قَدْ أَقْسَمَ السَّيِّدُ الرَّبُّ بِنَفْسِهِ يَقُولُ الرَّبُّ إِلَهُ الْجُنُودِ: «إِنِّي أَكْرَهُ عَظَمَةَ يَعْقُوبَ وَأُبْغِضُ قُصُورَهُ فَأُسَلِّمُ الْمَدِينَةَ وَمِلأَهَا».
\par 9 فَيَكُونُ إِذَا بَقِيَ عَشَرَةُ رِجَالٍ فِي بَيْتٍ وَاحِدٍ أَنَّهُمْ يَمُوتُونَ.
\par 10 وَإِذَا حَمَلَ أَحَداً عَمُّهُ وَمُحْرِقُهُ لِيُخْرِجَ الْعِظَامَ مِنَ الْبَيْتِ وَقَالَ لِمَنْ هُوَ فِي جَوَانِبِ الْبَيْتِ: «أَعِنْدَكَ بَعْدُ؟» يَقُولُ: «لَيْسَ بَعْدُ». فَيَقُولُ: «اسْكُتْ فَإِنَّهُ لاَ يُذْكَرُ اسْمُ الرَّبِّ».
\par 11 لأَنَّهُ هُوَذَا الرَّبُّ يَأْمُرُ فَيَضْرِبُ الْبَيْتَ الْكَبِيرَ رَدْماً وَالْبَيْتَ الصَّغِيرَ شُقُوقاً.
\par 12 هَلْ تَرْكُضُ الْخَيْلُ عَلَى الصَّخْرِ أَوْ يُحْرَثُ عَلَيْهِ بِالْبَقَرِ حَتَّى حَوَّلْتُمُ الْحَقَّ سِمّاً وَثَمَرَ الْبِرِّ أَفْسَنْتِيناً؟
\par 13 أَنْتُمُ الْفَرِحُونَ بِالْبُطْلِ الْقَائِلُونَ: «أَلَيْسَ بِقُوَّتِنَا اتَّخَذْنَا لأَنْفُسِنَا قُرُوناً؟»
\par 14 «لأَنِّي هَئَنَذَا أُقِيمُ عَلَيْكُمْ أُمَّةً يَا بَيْتَ إِسْرَائِيلَ يَقُولُ الرَّبُّ إِلَهُ الْجُنُودِ فَيُضَايِقُونَكُمْ مِنْ مَدْخَلِ حَمَاةَ إِلَى وَادِي الْعَرَبَةِ».

\chapter{7}

\par 1 هَكَذَا أَرَانِي السَّيِّدُ الرَّبُّ وَإِذَا هُوَ يَصْنَعُ جَرَاداً فِي أَوَّلِ طُلُوعِ خِلْفِ الْعُشْبِ. وَإِذَا خِلْفُ عُشْبٍ بَعْدَ جِزَازِ الْمَلِكِ.
\par 2 وَحَدَثَ لَمَّا فَرَغَ مِنْ أَكْلِ عُشْبِ الأَرْضِ أَنِّي قُلْتُ: «أَيُّهَا السَّيِّدُ الرَّبُّ اصْفَحْ. كَيْفَ يَقُومُ يَعْقُوبُ فَإِنَّهُ صَغِيرٌ؟»
\par 3 فَنَدِمَ الرَّبُّ عَلَى هَذَا وَقَالَ: «لاَ يَكُونُ».
\par 4 هَكَذَا أَرَانِي السَّيِّدُ الرَّبُّ وَإِذَا السَّيِّدُ الرَّبُّ قَدْ دَعَا لِلْمُحَاكَمَةِ بِالنَّارِ فَأَكَلَتِ الْغَمْرَ الْعَظِيمَ وَأَكَلَتِ الْحَقْلَ.
\par 5 فَقُلْتُ: «أَيُّهَا السَّيِّدُ الرَّبُّ كُفَّ. كَيْفَ يَقُومُ يَعْقُوبُ فَإِنَّهُ صَغِيرٌ؟»
\par 6 فَنَدِمَ الرَّبُّ عَلَى هَذَا وَقَالَ: «فَهُوَ أَيْضاً لاَ يَكُونُ».
\par 7 هَكَذَا أَرَانِي وَإِذَا الرَّبُّ وَاقِفٌ عَلَى حَائِطٍ قَائِمٍ وَفِي يَدِهِ زِيجٌ.
\par 8 فَسَأَلَنِي الرَّبُّ: «مَا أَنْتَ رَاءٍ يَا عَامُوسُ؟» فَقُلْتُ: «زِيجاً». فَقَالَ السَّيِّدُ: «هَئَنَذَا وَاضِعٌ زِيجاً فِي وَسَطِ شَعْبِي إِسْرَائِيلَ. لاَ أَعُودُ أَصْفَحُ لَهُ بَعْدُ.
\par 9 فَتُقْفِرُ مُرْتَفَعَاتُ إِسْحَاقَ وَتَخْرَبُ مَقَادِسُ إِسْرَائِيلَ وَأَقُومُ عَلَى بَيْتِ يَرُبْعَامَ بِالسَّيْفِ».
\par 10 فَأَرْسَلَ أَمَصْيَا كَاهِنُ بَيْتَِ إِيلَ إِلَى يَرُبْعَامَ مَلِكِ إِسْرَائِيلَ قَائِلاً: «قَدْ فَتَنَ عَلَيْكَ عَامُوسُ فِي وَسَطِ بَيْتِ إِسْرَائِيلَ. لاَ تَقْدِرُ الأَرْضُ أَنْ تُطِيقَ كُلَّ أَقْوَالِهِ.
\par 11 لأَنَّهُ هَكَذَا قَالَ عَامُوسُ: يَمُوتُ يَرُبْعَامُ بِالسَّيْفِ وَيُسْبَى إِسْرَائِيلُ عَنْ أَرْضِهِ».
\par 12 فَقَالَ أَمَصْيَا لِعَامُوسَ: «أَيُّهَا الرَّائِي اذْهَبِ اهْرُبْ إِلَى أَرْضِ يَهُوذَا وَكُلْ هُنَاكَ خُبْزاً وَهُنَاكَ تَنَبَّأْ.
\par 13 وَأَمَّا بَيْتُ إِيلَ فَلاَ تَعُدْ تَتَنَبَّأُ فِيهَا بَعْدُ لأَنَّهَا مَقْدِسُ الْمَلِكِ وَبَيْتُ الْمُلْكِ».
\par 14 فَأَجَابَ عَامُوسُ: «لَسْتُ أَنَا نَبِيّاً وَلاَ أَنَا ابْنُ نَبِيٍّ بَلْ أَنَا رَاعٍ وَجَانِي جُمَّيْزٍ.
\par 15 فَأَخَذَنِي الرَّبُّ مِنْ وَرَاءِ الضَّأْنِ وَقَالَ لِي الرَّبُّ: اذْهَبْ تَنَبَّأْ لِشَعْبِي إِسْرَائِيلَ.
\par 16 «فَالْآنَ اسْمَعْ قَوْلَ الرَّبِّ: أَنْتَ تَقُولُ: لاَ تَتَنَبَّأْ عَلَى إِسْرَائِيلَ وَلاَ تَتَكَلَّمْ عَلَى بَيْتِ إِسْحَاقَ.
\par 17 لِذَلِكَ هَكَذَا قَالَ الرَّبُّ: امْرَأَتُكَ تَزْنِي فِي الْمَدِينَةِ وَبَنُوكَ وَبَنَاتُكَ يَسْقُطُونَ بِالسَّيْفِ وَأَرْضُكَ تُقْسَمُ بِالْحَبْلِ وَأَنْتَ تَمُوتُ فِي أَرْضٍ نَجِسَةٍ وَإِسْرَائِيلُ يُسْبَى سَبْياً عَنْ أَرْضِهِ».

\chapter{8}

\par 1 هَكَذَا أَرَانِي السَّيِّدُ الرَّبُّ وَإِذَا سَلَّةٌ لِلْقِطَافِ.
\par 2 فَسَأَلَ: «مَاذَا أَنْتَ رَاءٍ يَا عَامُوسُ؟» فَقُلْتُ: «سَلَّةً لِلْقِطَافِ». فَقَالَ لِي الرَّبُّ: «قَدْ أَتَتِ النِّهَايَةُ عَلَى شَعْبِي إِسْرَائِيلَ. لاَ أَعُودُ أَصْفَحُ لَهُ بَعْدُ.
\par 3 فَتَصِيرُ أَغَانِي الْقَصْرِ وَلاَوِلَ فِي ذَلِكَ الْيَوْمِ يَقُولُ السَّيِّدُ الرَّبُّ. الْجُثَثُ كَثِيرَةٌ يَطْرَحُونَهَا فِي كُلِّ مَوْضِعٍ بِالسُّكُوتِ».
\par 4 اِسْمَعُوا هَذَا أَيُّهَا الْمُتَهَمِّمُونَ الْمَسَاكِينَ لِتُبِيدُوا بَائِسِي الأَرْضِ
\par 5 قَائِلِينَ: «مَتَى يَمْضِي رَأْسُ الشَّهْرِ لِنَبِيعَ قَمْحاً وَالسَّبْتُ لِنَعْرِضَ حِنْطَةً؟ لِنُصَغِّرَ الإِيفَةَ وَنُكَبِّرَ الشَّاقِلَ وَنُعَوِّجَ مَوَازِينَ الْغِشِّ.
\par 6 لِنَشْتَرِيَ الضُّعَفَاءَ بِفِضَّةٍ وَالْبَائِسَ بِنَعْلَيْنِ وَنَبِيعَ نُفَايَةَ الْقَمْحِ».
\par 7 قَدْ أَقْسَمَ الرَّبُّ بِفَخْرِ يَعْقُوبَ: «إِنِّي لَنْ أَنْسَى إِلَى الأَبَدِ جَمِيعَ أَعْمَالِهِمْ.
\par 8 أَلَيْسَ مِنْ أَجْلِ هَذَا تَرْتَعِدُ الأَرْضُ وَيَنُوحُ كُلُّ سَاكِنٍ فِيهَا وَتَطْمُو كُلُّهَا كَنَهْرٍ وَتَفِيضُ وَتَنْضُبُ كَنِيلِ مِصْرَ؟
\par 9 وَيَكُونُ فِي ذَلِكَ الْيَوْمِ يَقُولُ السَّيِّدُ الرَّبُّ أَنِّي أُغَيِّبُ الشَّمْسَ فِي الظُّهْرِ وَأُقْتِمُ الأَرْضَ فِي يَوْمِ نُورٍ
\par 10 وَأُحَوِّلُ أَعْيَادَكُمْ نَوْحاً وَجَمِيعَ أَغَانِيكُمْ مَرَاثِيَ وَأُصْعِدُ عَلَى كُلِّ الأَحْقَاءِ مِسْحاً وَعَلَى كُلِّ رَأْسٍ قَرْعَةً وَأَجْعَلُهَا كَمَنَاحَةِ الْوَحِيدِ وَآخِرَهَا يَوْماً مُرّاً!
\par 11 «هُوَذَا أَيَّامٌ تَأْتِي يَقُولُ السَّيِّدُ الرَّبُّ أُرْسِلُ جُوعاً فِي الأَرْضِ لاَ جُوعاً لِلْخُبْزِ وَلاَ عَطَشاً لِلْمَاءِ بَلْ لاِسْتِمَاعِ كَلِمَاتِ الرَّبِّ.
\par 12 فَيَجُولُونَ مِنْ بَحْرٍ إِلَى بَحْرٍ وَمِنَ الشِّمَالِ إِلَى الْمَشْرِقِ يَتَطَوَّحُونَ لِيَطْلُبُوا كَلِمَةَ الرَّبِّ فَلاَ يَجِدُونَهَا.
\par 13 فِي ذَلِكَ الْيَوْمِ تَذْبُلُ بِالْعَطَشِ الْعَذَارَى الْجَمِيلاَتُ وَالْفِتْيَانُ
\par 14 الَّذِينَ يَحْلِفُونَ بِذَنْبِ السَّامِرَةِ وَيَقُولُونَ: حَيٌّ إِلَهُكَ يَا دَانُ وَحَيَّةٌ طَرِيقَةُ بِئْرِ سَبْعٍ. فَيَسْقُطُونَ وَلاَ يَقُومُونَ بَعْدُ».

\chapter{9}

\par 1 رَأَيْتُ السَّيِّدَ قَائِماً عَلَى الْمَذْبَحِ فَقَالَ: «اضْرِبْ تَاجَ الْعَمُودِ حَتَّى تَرْجُفَ الأَعْتَابُ وَكَسِّرْهَا عَلَى رُؤُوسِ جَمِيعِهِمْ فَأَقْتُلَ آخِرَهُمْ بِالسَّيْفِ. لاَ يَهْرُبُ مِنْهُمْ هَارِبٌ وَلاَ يُفْلِتُ مِنْهُمْ نَاجٍ.
\par 2 إِنْ نَقَبُوا إِلَى الْهَاوِيَةِ فَمِنْ هُنَاكَ تَأْخُذُهُمْ يَدِي وَإِنْ صَعِدُوا إِلَى السَّمَاءِ فَمِنْ هُنَاكَ أُنْزِلُهُمْ!
\par 3 وَإِنِ اخْتَبَأُوا فِي رَأْسِ الْكَرْمَلِ فَمِنْ هُنَاكَ أُفَتِّشُ وَآخُذُهُمْ وَإِنِ اخْتَفُوا مِنْ أَمَامِ عَيْنَيَّ فِي قَعْرِ الْبَحْرِ فَمِنْ هُنَاكَ آمُرُ الْحَيَّةَ فَتَلْدَغُهُمْ.
\par 4 وَإِنْ مَضُوا فِي السَّبْيِ أَمَامَ أَعْدَائِهِمْ فَمِنْ هُنَاكَ آمُرُ السَّيْفَ فَيَقْتُلُهُمْ وَأَجْعَلُ عَيْنَيَّ عَلَيْهِمْ لِلشَّرِّ لاَ لِلْخَيْرِ».
\par 5 وَالسَّيِّدُ رَبُّ الْجُنُودِ الَّذِي يَمَسُّ الأَرْضَ فَتَذُوبُ وَيَنُوحُ السَّاكِنُونَ فِيهَا وَتَطْمُو كُلُّهَا كَنَهْرٍ وَتَنْضُبُ كَنِيلِ مِصْرَ
\par 6 الَّذِي بَنَى فِي السَّمَاءِ عَلاَلِيَهُ وَأَسَّسَ عَلَى الأَرْضِ قُبَّتَهُ الَّذِي يَدْعُو مِيَاهَ الْبَحْرِ وَيَصُبُّهَا عَلَى وَجْهِ الأَرْضِ يَهْوَهُ اسْمُهُ.
\par 7 «أَلَسْتُمْ لِي كَبَنِي الْكُوشِيِّينَ يَا بَنِي إِسْرَائِيلَ يَقُولُ الرَّبُّ؟ أَلَمْ أُصْعِدْ إِسْرَائِيلَ مِنْ أَرْضِ مِصْرَ وَالْفِلِسْطِينِيِّينَ مِنْ كَفْتُورَ وَالأَرَامِيِّينَ مِنْ قِيرٍ؟
\par 8 هُوَذَا عَيْنَا السَّيِّدِ الرَّبِّ عَلَى الْمَمْلَكَةِ الْخَاطِئَةِ وَأُبِيدُهَا عَنْ وَجْهِ الأَرْضِ. غَيْرَ أَنِّي لاَ أُبِيدُ بَيْتَ يَعْقُوبَ تَمَاماً يَقُولُ الرَّبُّ.
\par 9 لأَنَّهُ هَئَنَذَا آمُرُ فَأُغَرْبِلُ بَيْتَ إِسْرَائِيلَ بَيْنَ جَمِيعِ الأُمَمِ كَمَا يُغَرْبَلُ فِي الْغُرْبَالِ وَحَبَّةٌ لاَ تَقَعُ إِلَى الأَرْضِ.
\par 10 بِالسَّيْفِ يَمُوتُ كُلُّ خَاطِئِي شَعْبِي الْقَائِلِينَ: لاَ يَقْتَرِبُ الشَّرُّ وَلاَ يَأْتِي بَيْنَنَا.
\par 11 «فِي ذَلِكَ الْيَوْمِ أُقِيمُ مَظَلَّةَ دَاوُدَ السَّاقِطَةَ وَأُحَصِّنُ شُقُوقَهَا وَأُقِيمُ رَدْمَهَا وَأَبْنِيهَا كَأَيَّامِ الدَّهْرِ.
\par 12 لِيَرِثُوا بَقِيَّةَ أَدُومَ وَجَمِيعَ الأُمَمِ الَّذِينَ دُعِيَ اسْمِي عَلَيْهِمْ يَقُولُ الرَّبُّ الصَّانِعُ هَذَا.
\par 13 «هَا أَيَّامٌ تَأْتِي يَقُولُ الرَّبُّ يُدْرِكُ الْحَارِثُ الْحَاصِدَ وَدَائِسُ الْعِنَبِ بَاذِرَ الزَّرْعِ وَتَقْطُرُ الْجِبَالُ عَصِيراً وَتَسِيلُ جَمِيعُ التِّلاَلِ.
\par 14 وَأَرُدُّ سَبْيَ شَعْبِي إِسْرَائِيلَ فَيَبْنُونَ مُدُناً خَرِبَةً وَيَسْكُنُونَ وَيَغْرِسُونَ كُرُوماً وَيَشْرَبُونَ خَمْرَهَا وَيَصْنَعُونَ جَنَّاتٍ وَيَأْكُلُونَ أَثْمَارَهَا.
\par 15 وَأَغْرِسُهُمْ فِي أَرْضِهِمْ وَلَنْ يُقْلَعُوا بَعْدُ مِنْ أَرْضِهِمِ الَّتِي أَعْطَيْتُهُمْ» قَالَ الرَّبُّ إِلَهُكَ.

\end{document}