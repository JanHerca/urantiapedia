\begin{document}

\title{لوقا}


\chapter{1}

\par 1 إِذْ كَانَ كَثِيرُونَ قَدْ أَخَذُوا بِتَأْلِيفِ قِصَّةٍ فِي الأُمُورِ الْمُتَيَقَّنَةِ عِنْدَنَا
\par 2 كَمَا سَلَّمَهَا إِلَيْنَا الَّذِينَ كَانُوا مُنْذُ الْبَدْءِ مُعَايِنِينَ وَخُدَّاماً لِلْكَلِمَةِ
\par 3 رَأَيْتُ أَنَا أَيْضاً إِذْ قَدْ تَتَبَّعْتُ كُلَّ شَيْءٍ مِنَ الأَوَّلِ بِتَدْقِيقٍ أَنْ أَكْتُبَ عَلَى التَّوَالِي إِلَيْكَ أَيُّهَا الْعَزِيزُ ثَاوُفِيلُسُ
\par 4 لِتَعْرِفَ صِحَّةَ الْكَلاَمِ الَّذِي عُلِّمْتَ بِهِ.
\par 5 كَانَ فِي أَيَّامِ هِيرُودُسَ مَلِكِ الْيَهُودِيَّةِ كَاهِنٌ اسْمُهُ زَكَرِيَّا مِنْ فِرْقَةِ أَبِيَّا وَامْرَأَتُهُ مِنْ بَنَاتِ هَارُونَ وَاسْمُهَا أَلِيصَابَاتُ.
\par 6 وَكَانَا كِلاَهُمَا بَارَّيْنِ أَمَامَ اللهِ سَالِكَيْنِ فِي جَمِيعِ وَصَايَا الرَّبِّ وَأَحْكَامِهِ بِلاَ لَوْمٍ.
\par 7 وَلَمْ يَكُنْ لَهُمَا وَلَدٌ إِذْ كَانَتْ أَلِيصَابَاتُ عَاقِراً. وَكَانَا كِلاَهُمَا مُتَقَدِّمَيْنِ فِي أَيَّامِهِمَا.
\par 8 فَبَيْنَمَا هُوَ يَكْهَنُ فِي نَوْبَةِ فِرْقَتِهِ أَمَامَ اللهِ
\par 9 حَسَبَ عَادَةِ الْكَهَنُوتِ أَصَابَتْهُ الْقُرْعَةُ أَنْ يَدْخُلَ إِلَى هَيْكَلِ الرَّبِّ وَيُبَخِّرَ.
\par 10 وَكَانَ كُلُّ جُمْهُورِ الشَّعْبِ يُصَلُّونَ خَارِجاً وَقْتَ الْبَخُورِ.
\par 11 فَظَهَرَ لَهُ مَلاَكُ الرَّبِّ وَاقِفاً عَنْ يَمِينِ مَذْبَحِ الْبَخُورِ.
\par 12 فَلَمَّا رَآهُ زَكَرِيَّا اضْطَرَبَ وَوَقَعَ عَلَيْهِ خَوْفٌ.
\par 13 فَقَالَ لَهُ الْمَلاَكُ: «لاَ تَخَفْ يَا زَكَرِيَّا لأَنَّ طِلْبَتَكَ قَدْ سُمِعَتْ وَامْرَأَتُكَ أَلِيصَابَاتُ سَتَلِدُ لَكَ ابْناً وَتُسَمِّيهِ يُوحَنَّا.
\par 14 وَيَكُونُ لَكَ فَرَحٌ وَابْتِهَاجٌ وَكَثِيرُونَ سَيَفْرَحُونَ بِوِلاَدَتِهِ
\par 15 لأَنَّهُ يَكُونُ عَظِيماً أَمَامَ الرَّبِّ وَخَمْراً وَمُسْكِراً لاَ يَشْرَبُ وَمِنْ بَطْنِ أُمِّهِ يَمْتَلِئُ مِنَ الرُّوحِ الْقُدُسِ.
\par 16 وَيَرُدُّ كَثِيرِينَ مِنْ بَنِي إِسْرَائِيلَ إِلَى الرَّبِّ إِلَهِهِمْ.
\par 17 وَيَتَقَدَّمُ أَمَامَهُ بِرُوحِ إِيلِيَّا وَقُوَّتِهِ لِيَرُدَّ قُلُوبَ الآبَاءِ إِلَى الأَبْنَاءِ وَالْعُصَاةَ إِلَى فِكْرِ الأَبْرَارِ لِكَيْ يُهَيِّئَ لِلرَّبِّ شَعْباً مُسْتَعِدّاً».
\par 18 فَقَالَ زَكَرِيَّا لِلْمَلاَكِ: «كَيْفَ أَعْلَمُ هَذَا لأَنِّي أَنَا شَيْخٌ وَامْرَأَتِي مُتَقَدِّمَةٌ فِي أَيَّامِهَا؟»
\par 19 فَأَجَابَ الْمَلاَكُ: «أَنَا جِبْرَائِيلُ الْوَاقِفُ قُدَّامَ اللهِ وَأُرْسِلْتُ لأُكَلِّمَكَ وَأُبَشِّرَكَ بِهَذَا.
\par 20 وَهَا أَنْتَ تَكُونُ صَامِتاً وَلاَ تَقْدِرُ أَنْ تَتَكَلَّمَ إِلَى الْيَوْمِ الَّذِي يَكُونُ فِيهِ هَذَا لأَنَّكَ لَمْ تُصَدِّقْ كَلاَمِي الَّذِي سَيَتِمُّ فِي وَقْتِهِ».
\par 21 وَكَانَ الشَّعْبُ مُنْتَظِرِينَ زَكَرِيَّا وَمُتَعّجِّبِينَ مِنْ إِبْطَائِهِ فِي الْهَيْكَلِ.
\par 22 فَلَمَّا خَرَجَ لَمْ يَسْتَطِعْ أَنْ يُكَلِّمَهُمْ فَفَهِمُوا أَنَّهُ قَدْ رَأَى رُؤْيَا فِي الْهَيْكَلِ. فَكَانَ يُومِئُ إِلَيْهِمْ وَبَقِيَ صَامِتاً.
\par 23 وَلَمَّا كَمِلَتْ أَيَّامُ خِدْمَتِهِ مَضَى إِلَى بَيْتِهِ.
\par 24 وَبَعْدَ تِلْكَ الأَيَّامِ حَبِلَتْ أَلِيصَابَاتُ امْرَأَتُهُ وَأَخْفَتْ نَفْسَهَا خَمْسَةَ أَشْهُرٍ قَائِلَةً:
\par 25 «هَكَذَا قَدْ فَعَلَ بِيَ الرَّبُّ فِي الأَيَّامِ الَّتِي فِيهَا نَظَرَ إِلَيَّ لِيَنْزِعَ عَارِي بَيْنَ النَّاسِ».
\par 26 وَفِي الشَّهْرِ السَّادِسِ أُرْسِلَ جِبْرَائِيلُ الْمَلاَكُ مِنَ اللهِ إِلَى مَدِينَةٍ مِنَ الْجَلِيلِ اسْمُهَا نَاصِرَةُ
\par 27 إِلَى عَذْرَاءَ مَخْطُوبَةٍ لِرَجُلٍ مِنْ بَيْتِ دَاوُدَ اسْمُهُ يُوسُفُ. وَاسْمُ الْعَذْرَاءِ مَرْيَمُ.
\par 28 فَدَخَلَ إِلَيْهَا الْمَلاَكُ وَقَالَ: «سَلاَمٌ لَكِ أَيَّتُهَا الْمُنْعَمُ عَلَيْهَا! اَلرَّبُّ مَعَكِ. مُبَارَكَةٌ أَنْتِ فِي النِّسَاءِ».
\par 29 فَلَمَّا رَأَتْهُ اضْطَرَبَتْ مِنْ كَلاَمِهِ وَفَكَّرَتْ مَا عَسَى أَنْ تَكُونَ هَذِهِ التَّحِيَّةُ!
\par 30 فَقَالَ لَهَا الْمَلاَكُ: «لاَ تَخَافِي يَا مَرْيَمُ لأَنَّكِ قَدْ وَجَدْتِ نِعْمَةً عِنْدَ اللهِ.
\par 31 وَهَا أَنْتِ سَتَحْبَلِينَ وَتَلِدِينَ ابْناً وَتُسَمِّينَهُ يَسُوعَ.
\par 32 هَذَا يَكُونُ عَظِيماً وَابْنَ الْعَلِيِّ يُدْعَى وَيُعْطِيهِ الرَّبُّ الإِلَهُ كُرْسِيَّ دَاوُدَ أَبِيهِ
\par 33 وَيَمْلِكُ عَلَى بَيْتِ يَعْقُوبَ إِلَى الأَبَدِ وَلاَ يَكُونُ لِمُلْكِهِ نِهَايَةٌ».
\par 34 فَقَالَتْ مَرْيَمُ لِلْمَلاَكِ: «كَيْفَ يَكُونُ هَذَا وَأَنَا لَسْتُ أَعْرِفُ رَجُلاً؟»
\par 35 فَأَجَابَ الْمَلاَكُ: «اَلرُّوحُ الْقُدُسُ يَحِلُّ عَلَيْكِ وَقُوَّةُ الْعَلِيِّ تُظَلِّلُكِ فَلِذَلِكَ أَيْضاً الْقُدُّوسُ الْمَوْلُودُ مِنْكِ يُدْعَى ابْنَ اللهِ.
\par 36 وَهُوَذَا أَلِيصَابَاتُ نَسِيبَتُكِ هِيَ أَيْضاً حُبْلَى بِابْنٍ فِي شَيْخُوخَتِهَا وَهَذَا هُوَ الشَّهْرُ السَّادِسُ لِتِلْكَ الْمَدْعُوَّةِ عَاقِراً
\par 37 لأَنَّهُ لَيْسَ شَيْءٌ غَيْرَ مُمْكِنٍ لَدَى اللهِ».
\par 38 فَقَالَتْ مَرْيَمُ: «هُوَذَا أَنَا أَمَةُ الرَّبِّ. لِيَكُنْ لِي كَقَوْلِكَ». فَمَضَى مِنْ عِنْدِهَا الْمَلاَكُ.
\par 39 فَقَامَتْ مَرْيَمُ فِي تِلْكَ الأَيَّامِ وَذَهَبَتْ بِسُرْعَةٍ إِلَى الْجِبَالِ إِلَى مَدِينَةِ يَهُوذَا
\par 40 وَدَخَلَتْ بَيْتَ زَكَرِيَّا وَسَلَّمَتْ عَلَى أَلِيصَابَاتَ.
\par 41 فَلَمَّا سَمِعَتْ أَلِيصَابَاتُ سَلاَمَ مَرْيَمَ ارْتَكَضَ الْجَنِينُ فِي بَطْنِهَا وَامْتَلَأَتْ أَلِيصَابَاتُ مِنَ الرُّوحِ الْقُدُسِ
\par 42 وَصَرَخَتْ بِصَوْتٍ عَظِيمٍ وَقَالَتْ: «مُبَارَكَةٌ أَنْتِ فِي النِّسَاءِ وَمُبَارَكَةٌ هِيَ ثَمَرَةُ بَطْنِكِ!
\par 43 فَمِنْ أَيْنَ لِي هَذَا أَنْ تَأْتِيَ أُمُّ رَبِّي إِلَيَّ؟
\par 44 فَهُوَذَا حِينَ صَارَ صَوْتُ سَلاَمِكِ فِي أُذُنَيَّ ارْتَكَضَ الْجَنِينُ بِابْتِهَاجٍ فِي بَطْنِي.
\par 45 فَطُوبَى لِلَّتِي آمَنَتْ أَنْ يَتِمَّ مَا قِيلَ لَهَا مِنْ قِبَلِ الرَّبِّ».
\par 46 فَقَالَتْ مَرْيَمُ: «تُعَظِّمُ نَفْسِي الرَّبَّ
\par 47 وَتَبْتَهِجُ رُوحِي بِاللَّهِ مُخَلِّصِي
\par 48 لأَنَّهُ نَظَرَ إِلَى اتِّضَاعِ أَمَتِهِ. فَهُوَذَا مُنْذُ الآنَ جَمِيعُ الأَجْيَالِ تُطَوِّبُنِي
\par 49 لأَنَّ الْقَدِيرَ صَنَعَ بِي عَظَائِمَ وَاسْمُهُ قُدُّوسٌ
\par 50 وَرَحْمَتُهُ إِلَى جِيلِ الأَجْيَالِ لِلَّذِينَ يَتَّقُونَهُ.
\par 51 صَنَعَ قُوَّةً بِذِرَاعِهِ. شَتَّتَ الْمُسْتَكْبِرِينَ بِفِكْرِ قُلُوبِهِمْ.
\par 52 أَنْزَلَ الأَعِزَّاءَ عَنِ الْكَرَاسِيِّ وَرَفَعَ الْمُتَّضِعِينَ.
\par 53 أَشْبَعَ الْجِيَاعَ خَيْرَاتٍ وَصَرَفَ الأَغْنِيَاءَ فَارِغِينَ.
\par 54 عَضَدَ إِسْرَائِيلَ فَتَاهُ لِيَذْكُرَ رَحْمَةً
\par 55 كَمَا كَلَّمَ آبَاءَنَا. لِإِبْراهِيمَ وَنَسْلِهِ إِلَى الأَبَدِ».
\par 56 فَمَكَثَتْ مَرْيَمُ عِنْدَهَا نَحْوَ ثَلاَثَةِ أَشْهُرٍ ثُمَّ رَجَعَتْ إِلَى بَيْتِهَا.
\par 57 وَأَمَّا أَلِيصَابَاتُ فَتَمَّ زَمَانُهَا لِتَلِدَ فَوَلَدَتِ ابْناً.
\par 58 وَسَمِعَ جِيرَانُهَا وَأَقْرِبَاؤُهَا أَنَّ الرَّبَّ عَظَّمَ رَحْمَتَهُ لَهَا فَفَرِحُوا مَعَهَا.
\par 59 وَفِي الْيَوْمِ الثَّامِنِ جَاءُوا لِيَخْتِنُوا الصَّبِيَّ وَسَمَّوْهُ بِاسْمِ أَبِيهِ زَكَرِيَّا.
\par 60 فَقَالَتْ أُمُّهُ: «لاَ بَلْ يُسَمَّى يُوحَنَّا».
\par 61 فَقَالُوا لَهَا: «لَيْسَ أَحَدٌ فِي عَشِيرَتِكِ تَسَمَّى بِهَذَا الاِسْمِ».
\par 62 ثُمَّ أَوْمَأُوا إِلَى أَبِيهِ مَاذَا يُرِيدُ أَنْ يُسَمَّى.
\par 63 فَطَلَبَ لَوْحاً وَكَتَبَ: «اسْمُهُ يُوحَنَّا». فَتَعَجَّبَ الْجَمِيعُ.
\par 64 وَفِي الْحَالِ انْفَتَحَ فَمُهُ وَلِسَانُهُ وَتَكَلَّمَ وَبَارَكَ اللهَ.
\par 65 فَوَقَعَ خَوْفٌ عَلَى كُلِّ جِيرَانِهِمْ. وَتُحُدِّثَ بِهَذِهِ الأُمُورِ جَمِيعِهَا فِي كُلِّ جِبَالِ الْيَهُودِيَّةِ
\par 66 فَأَوْدَعَهَا جَمِيعُ السَّامِعِينَ فِي قُلُوبِهِمْ قَائِلِينَ: «أَتَرَى مَاذَا يَكُونُ هَذَا الصَّبِيُّ؟» وَكَانَتْ يَدُ الرَّبِّ مَعَهُ.
\par 67 وَامْتَلأَ زَكَرِيَّا أَبُوهُ مِنَ الرُّوحِ الْقُدُسِ وَتَنَبَّأَ قَائِلاً:
\par 68 «مُبَارَكٌ الرَّبُّ إِلَهُ إِسْرَائِيلَ لأَنَّهُ افْتَقَدَ وَصَنَعَ فِدَاءً لِشَعْبِهِ
\par 69 وَأَقَامَ لَنَا قَرْنَ خَلاَصٍ فِي بَيْتِ دَاوُدَ فَتَاهُ.
\par 70 كَمَا تَكَلَّمَ بِفَمِ أَنْبِيَائِهِ الْقِدِّيسِينَ الَّذِينَ هُمْ مُنْذُ الدَّهْرِ.
\par 71 خَلاَصٍ مِنْ أَعْدَائِنَا وَمِنْ أَيْدِي جَمِيعِ مُبْغِضِينَا.
\par 72 لِيَصْنَعَ رَحْمَةً مَعَ آبَائِنَا وَيَذْكُرَ عَهْدَهُ الْمُقَدَّسَ.
\par 73 الْقَسَمَ الَّذِي حَلَفَ لإِبْرَاهِيمَ أَبِينَا:
\par 74 أَنْ يُعْطِيَنَا إِنَّنَا بِلاَ خَوْفٍ مُنْقَذِينَ مِنْ أَيْدِي أَعْدَائِنَا نَعْبُدُهُ
\par 75 بِقَدَاسَةٍ وَبِرٍّ قُدَّامَهُ جَمِيعَ أَيَّامِ حَيَاتِنَا.
\par 76 وَأَنْتَ أَيُّهَا الصَّبِيُّ نَبِيَّ الْعَلِيِّ تُدْعَى لأَنَّكَ تَتَقَدَّمُ أَمَامَ وَجْهِ الرَّبِّ لِتُعِدَّ طُرُقَهُ.
\par 77 لِتُعْطِيَ شَعْبَهُ مَعْرِفَةَ الْخَلاَصِ بِمَغْفِرَةِ خَطَايَاهُمْ
\par 78 بِأَحْشَاءِ رَحْمَةِ إِلَهِنَا الَّتِي بِهَا افْتَقَدَنَا الْمُشْرَقُ مِنَ الْعَلاَءِ.
\par 79 لِيُضِيءَ عَلَى الْجَالِسِينَ فِي الظُّلْمَةِ وَظِلاَلِ الْمَوْتِ لِكَيْ يَهْدِيَ أَقْدَامَنَا فِي طَرِيقِ السَّلاَمِ».
\par 80 أَمَّا الصَّبِيُّ فَكَانَ يَنْمُو وَيَتَقَوَّى بِالرُّوحِ وَكَانَ فِي الْبَرَارِي إِلَى يَوْمِ ظُهُورِهِ لِإِسْرَائِيلَ.

\chapter{2}

\par 1 وَفِي تِلْكَ الأَيَّامِ صَدَرَ أَمْرٌ مِنْ أُوغُسْطُسَ قَيْصَرَ بِأَنْ يُكْتَتَبَ كُلُّ الْمَسْكُونَةِ.
\par 2 وَهَذَا الاِكْتِتَابُ الأَوَّلُ جَرَى إِذْ كَانَ كِيرِينِيُوسُ وَالِيَ سُورِيَّةَ.
\par 3 فَذَهَبَ الْجَمِيعُ لِيُكْتَتَبُوا كُلُّ وَاحِدٍ إِلَى مَدِينَتِهِ.
\par 4 فَصَعِدَ يُوسُفُ أَيْضاً مِنَ الْجَلِيلِ مِنْ مَدِينَةِ النَّاصِرَةِ إِلَى الْيَهُودِيَّةِ إِلَى مَدِينَةِ دَاوُدَ الَّتِي تُدْعَى بَيْتَ لَحْمٍ لِكَوْنِهِ مِنْ بَيْتِ دَاوُدَ وَعَشِيرَتِهِ
\par 5 لِيُكْتَتَبَ مَعَ مَرْيَمَ امْرَأَتِهِ الْمَخْطُوبَةِ وَهِيَ حُبْلَى.
\par 6 وَبَيْنَمَا هُمَا هُنَاكَ تَمَّتْ أَيَّامُهَا لِتَلِدَ.
\par 7 فَوَلَدَتِ ابْنَهَا الْبِكْرَ وَقَمَّطَتْهُ وَأَضْجَعَتْهُ فِي الْمِذْوَدِ إِذْ لَمْ يَكُنْ لَهُمَا مَوْضِعٌ فِي الْمَنْزِلِ.
\par 8 وَكَانَ فِي تِلْكَ الْكُورَةِ رُعَاةٌ مُتَبَدِّينَ يَحْرُسُونَ حِرَاسَاتِ اللَّيْلِ عَلَى رَعِيَّتِهِمْ
\par 9 وَإِذَا مَلاَكُ الرَّبِّ وَقَفَ بِهِمْ وَمَجْدُ الرَّبِّ أَضَاءَ حَوْلَهُمْ فَخَافُوا خَوْفاً عَظِيماً.
\par 10 فَقَالَ لَهُمُ الْمَلاَكُ: «لاَ تَخَافُوا. فَهَا أَنَا أُبَشِّرُكُمْ بِفَرَحٍ عَظِيمٍ يَكُونُ لِجَمِيعِ الشَّعْبِ:
\par 11 أَنَّهُ وُلِدَ لَكُمُ الْيَوْمَ فِي مَدِينَةِ دَاوُدَ مُخَلِّصٌ هُوَ الْمَسِيحُ الرَّبُّ.
\par 12 وَهَذِهِ لَكُمُ الْعَلاَمَةُ: تَجِدُونَ طِفْلاً مُقَمَّطاً مُضْجَعاً فِي مِذْوَدٍ».
\par 13 وَظَهَرَ بَغْتَةً مَعَ الْمَلاَكِ جُمْهُورٌ مِنَ الْجُنْدِ السَّمَاوِيِّ مُسَبِّحِينَ اللهَ وَقَائِلِينَ:
\par 14 «الْمَجْدُ لِلَّهِ فِي الأَعَالِي وَعَلَى الأَرْضِ السَّلاَمُ وَبِالنَّاسِ الْمَسَرَّةُ».
\par 15 وَلَمَّا مَضَتْ عَنْهُمُ الْمَلاَئِكَةُ إِلَى السَّمَاءِ قَالَ الرُّعَاةُ بَعْضُهُمْ لِبَعْضٍ: «لِنَذْهَبِ الآنَ إِلَى بَيْتِ لَحْمٍ وَنَنْظُرْ هَذَا الأَمْرَ الْوَاقِعَ الَّذِي أَعْلَمَنَا بِهِ الرَّبُّ».
\par 16 فَجَاءُوا مُسْرِعِينَ وَوَجَدُوا مَرْيَمَ وَيُوسُفَ وَالطِّفْلَ مُضْجَعاً فِي الْمِذْوَدِ.
\par 17 فَلَمَّا رَأَوْهُ أَخْبَرُوا بِالْكَلاَمِ الَّذِي قِيلَ لَهُمْ عَنْ هَذَا الصَّبِيِّ.
\par 18 وَكُلُّ الَّذِينَ سَمِعُوا تَعَجَّبُوا مِمَّا قِيلَ لَهُمْ مِنَ الرُّعَاةِ.
\par 19 وَأَمَّا مَرْيَمُ فَكَانَتْ تَحْفَظُ جَمِيعَ هَذَا الْكَلاَمِ مُتَفَكِّرَةً بِهِ فِي قَلْبِهَا.
\par 20 ثُمَّ رَجَعَ الرُّعَاةُ وَهُمْ يُمَجِّدُونَ اللهَ وَيُسَبِّحُونَهُ عَلَى كُلِّ مَا سَمِعُوهُ وَرَأَوْهُ كَمَا قِيلَ لَهُمْ.
\par 21 وَلَمَّا تَمَّتْ ثَمَانِيَةُ أَيَّامٍ لِيَخْتِنُوا الصَّبِيَّ سُمِّيَ يَسُوعَ كَمَا تَسَمَّى مِنَ الْمَلاَكِ قَبْلَ أَنْ حُبِلَ بِهِ فِي الْبَطْنِ.
\par 22 وَلَمَّا تَمَّتْ أَيَّامُ تَطْهِيرِهَا حَسَبَ شَرِيعَةِ مُوسَى صَعِدُوا بِهِ إِلَى أُورُشَلِيمَ لِيُقَدِّمُوهُ لِلرَّبِّ
\par 23 كَمَا هُوَ مَكْتُوبٌ فِي نَامُوسِ الرَّبِّ: أَنَّ كُلَّ ذَكَرٍ فَاتِحَ رَحِمٍ يُدْعَى قُدُّوساً لِلرَّبِّ.
\par 24 وَلِكَيْ يُقَدِّمُوا ذَبِيحَةً كَمَا قِيلَ فِي نَامُوسِ الرَّبِّ زَوْجَ يَمَامٍ أَوْ فَرْخَيْ حَمَامٍ.
\par 25 وَكَانَ رَجُلٌ فِي أُورُشَلِيمَ اسْمُهُ سِمْعَانُ كَانَ بَارّاً تَقِيّاً يَنْتَظِرُ تَعْزِيَةَ إِسْرَائِيلَ وَالرُّوحُ الْقُدُسُ كَانَ عَلَيْهِ.
\par 26 وَكَانَ قَدْ أُوحِيَ إِلَيْهِ بِالرُّوحِ الْقُدُسِ أَنَّهُ لاَ يَرَى الْمَوْتَ قَبْلَ أَنْ يَرَى مَسِيحَ الرَّبِّ.
\par 27 فَأَتَى بِالرُّوحِ إِلَى الْهَيْكَلِ. وَعِنْدَمَا دَخَلَ بِالصَّبِيِّ يَسُوعَ أَبَوَاهُ لِيَصْنَعَا لَهُ حَسَبَ عَادَةِ النَّامُوسِ
\par 28 أَخَذَهُ عَلَى ذِرَاعَيْهِ وَبَارَكَ اللهَ وَقَالَ:
\par 29 «الآنَ تُطْلِقُ عَبْدَكَ يَا سَيِّدُ حَسَبَ قَوْلِكَ بِسَلاَمٍ
\par 30 لأَنَّ عَيْنَيَّ قَدْ أَبْصَرَتَا خَلاَصَكَ
\par 31 الَّذِي أَعْدَدْتَهُ قُدَّامَ وَجْهِ جَمِيعِ الشُّعُوبِ.
\par 32 نُورَ إِعْلاَنٍ لِلأُمَمِ وَمَجْداً لِشَعْبِكَ إِسْرَائِيلَ».
\par 33 وَكَانَ يُوسُفُ وَأُمُّهُ يَتَعَجَّبَانِ مِمَّا قِيلَ فِيهِ.
\par 34 وَبَارَكَهُمَا سِمْعَانُ وَقَالَ لِمَرْيَمَ أُمِّهِ: «هَا إِنَّ هَذَا قَدْ وُضِعَ لِسُقُوطِ وَقِيَامِ كَثِيرِينَ فِي إِسْرَائِيلَ وَلِعَلاَمَةٍ تُقَاوَمُ.
\par 35 وَأَنْتِ أَيْضاً يَجُوزُ فِي نَفْسِكِ سَيْفٌ لِتُعْلَنَ أَفْكَارٌ مِنْ قُلُوبٍ كَثِيرَةٍ».
\par 36 وَكَانَتْ نَبِيَّةٌ حَنَّةُ بِنْتُ فَنُوئِيلَ مِنْ سِبْطِ أَشِيرَ وَهِيَ مُتَقّدِّمَةٌ فِي أَيَّامٍ كَثِيرَةٍ قَدْ عَاشَتْ مَعَ زَوْجٍ سَبْعَ سِنِينَ بَعْدَ بُكُورِيَّتِهَا.
\par 37 وَهِيَ أَرْمَلَةٌ نَحْوَ أَرْبَعٍ وَثَمَانِينَ سَنَةً لاَ تُفَارِقُ الْهَيْكَلَ عَابِدَةً بِأَصْوَامٍ وَطِلْبَاتٍ لَيْلاً وَنَهَاراً.
\par 38 فَهِيَ فِي تِلْكَ السَّاعَةِ وَقَفَتْ تُسَبِّحُ الرَّبَّ وَتَكَلَّمَتْ عَنْهُ مَعَ جَمِيعِ الْمُنْتَظِرِينَ فِدَاءً فِي أُورُشَلِيمَ.
\par 39 وَلَمَّا أَكْمَلُوا كُلَّ شَيْءٍ حَسَبَ نَامُوسِ الرَّبِّ رَجَعُوا إِلَى الْجَلِيلِ إِلَى مَدِينَتِهِمُ النَّاصِرَةِ.
\par 40 وَكَانَ الصَّبِيُّ يَنْمُو وَيَتَقَوَّى بِالرُّوحِ مُمْتَلِئاً حِكْمَةً وَكَانَتْ نِعْمَةُ اللهِ عَلَيْهِ.
\par 41 وَكَانَ أَبَوَاهُ يَذْهَبَانِ كُلَّ سَنَةٍ إِلَى أُورُشَلِيمَ فِي عِيدِ الْفِصْحِ.
\par 42 وَلَمَّا كَانَتْ لَهُ اثْنَتَا عَشْرَةَ سَنَةً صَعِدُوا إِلَى أُورُشَلِيمَ كَعَادَةِ الْعِيدِ.
\par 43 وَبَعْدَمَا أَكْمَلُوا الأَيَّامَ بَقِيَ عِنْدَ رُجُوعِهِمَا الصَّبِيُّ يَسُوعُ فِي أُورُشَلِيمَ وَيُوسُفُ وَأُمُّهُ لَمْ يَعْلَمَا.
\par 44 وَإِذْ ظَنَّاهُ بَيْنَ الرُّفْقَةِ ذَهَبَا مَسِيرَةَ يَوْمٍ وَكَانَا يَطْلُبَانِهِ بَيْنَ الأَقْرِبَاءِ وَالْمَعَارِفِ.
\par 45 وَلَمَّا لَمْ يَجِدَاهُ رَجَعَا إِلَى أُورُشَلِيمَ يَطْلُبَانِهِ.
\par 46 وَبَعْدَ ثَلاَثَةِ أَيَّامٍ وَجَدَاهُ فِي الْهَيْكَلِ جَالِساً فِي وَسْطِ الْمُعَلِّمِينَ يَسْمَعُهُمْ وَيَسْأَلُهُمْ.
\par 47 وَكُلُّ الَّذِينَ سَمِعُوهُ بُهِتُوا مِنْ فَهْمِهِ وَأَجْوِبَتِهِ.
\par 48 فَلَمَّا أَبْصَرَاهُ انْدَهَشَا. وَقَالَتْ لَهُ أُمُّهُ: «يَا بُنَيَّ لِمَاذَا فَعَلْتَ بِنَا هَكَذَا؟ هُوَذَا أَبُوكَ وَأَنَا كُنَّا نَطْلُبُكَ مُعَذَّبَيْنِ!»
\par 49 فَقَالَ لَهُمَا: «لِمَاذَا كُنْتُمَا تَطْلُبَانِنِي؟ أَلَمْ تَعْلَمَا أَنَّهُ يَنْبَغِي أَنْ أَكُونَ فِي مَا لأَبِي؟».
\par 50 فَلَمْ يَفْهَمَا الْكَلاَمَ الَّذِي قَالَهُ لَهُمَا.
\par 51 ثُمَّ نَزَلَ مَعَهُمَا وَجَاءَ إِلَى النَّاصِرَةِ وَكَانَ خَاضِعاً لَهُمَا. وَكَانَتْ أُمُّهُ تَحْفَظُ جَمِيعَ هَذِهِ الأُمُورِ فِي قَلْبِهَا.
\par 52 وَأَمَّا يَسُوعُ فَكَانَ يَتَقَدَّمُ فِي الْحِكْمَةِ وَالْقَامَةِ وَالنِّعْمَةِ عِنْدَ اللهِ وَالنَّاسِ.

\chapter{3}

\par 1 وَفِي السَّنَةِ الْخَامِسَةِ عَشْرَةَ مِنْ سَلْطَنَةِ طِيبَارِيُوسَ قَيْصَرَ إِذْ كَانَ بِيلاَطُسُ الْبُنْطِيُّ وَالِياً عَلَى الْيَهُودِيَّةِ وَهِيرُودُسُ رَئِيسَ رُبْعٍ عَلَى الْجَلِيلِ وَفِيلُبُّسُ أَخُوهُ رَئِيسَ رُبْعٍ عَلَى إِيطُورِيَّةَ وَكُورَةِ تَرَاخُونِيتِسَ وَلِيسَانِيُوسُ رَئِيسَ رُبْعٍ عَلَى الأَبِلِيَّةِ
\par 2 فِي أَيَّامِ رَئِيسِ الْكَهَنَةِ حَنَّانَ وَقَيَافَا كَانَتْ كَلِمَةُ اللهِ عَلَى يُوحَنَّا بْنِ زَكَرِيَّا فِي الْبَرِّيَّةِ
\par 3 فَجَاءَ إِلَى جَمِيعِ الْكُورَةِ الْمُحِيطَةِ بِالأُرْدُنِّ يَكْرِزُ بِمَعْمُودِيَّةِ التَّوْبَةِ لِمَغْفِرَةِ الْخَطَايَا
\par 4 كَمَا هُوَ مَكْتُوبٌ فِي سِفْرِ إِشَعْيَاءَ النَّبِيِّ: «صَوْتُ صَارِخٍ فِي الْبَرِّيَّةِ أَعِدُّوا طَرِيقَ الرَّبِّ اصْنَعُوا سُبُلَهُ مُسْتَقِيمَةً.
\par 5 كُلُّ وَادٍ يَمْتَلِئُ وَكُلُّ جَبَلٍ وَأَكَمَةٍ يَنْخَفِضُ وَتَصِيرُ الْمُعْوَجَّاتُ مُسْتَقِيمَةً وَالشِّعَابُ طُرُقاً سَهْلَةً
\par 6 وَيُبْصِرُ كُلُّ بَشَرٍ خَلاَصَ اللهِ».
\par 7 وَكَانَ يَقُولُ لِلْجُمُوعِ الَّذِينَ خَرَجُوا لِيَعْتَمِدُوا مِنْهُ: «يَا أَوْلاَدَ الأَفَاعِي مَنْ أَرَاكُمْ أَنْ تَهْرُبُوا مِنَ الْغَضَبِ الآتِي؟
\par 8 فَاصْنَعُوا أَثْمَاراً تَلِيقُ بِالتَّوْبَةِ. ولاَ تَبْتَدِئُوا تَقُولُونَ فِي أَنْفُسِكُمْ: لَنَا إِبْرَاهِيمُ أَباً. لأَنِّي أَقُولُ لَكُمْ إِنَّ اللهَ قَادِرٌ أَنْ يُقِيمَ مِنْ هَذِهِ الْحِجَارَةِ أَوْلاَداً لِإِبْرَاهِيمَ.
\par 9 وَالآنَ قَدْ وُضِعَتِ الْفَأْسُ عَلَى أَصْلِ الشَّجَرِ فَكُلُّ شَجَرَةٍ لاَ تَصْنَعُ ثَمَراً جَيِّداً تُقْطَعُ وَتُلْقَى فِي النَّارِ».
\par 10 وَسَأَلَهُ الْجُمُوعُ: «فَمَاذَا نَفْعَلُ؟»
\par 11 فَأَجَابَ: «مَنْ لَهُ ثَوْبَانِ فَلْيُعْطِ مَنْ لَيْسَ لَهُ وَمَنْ لَهُ طَعَامٌ فَلْيَفْعَلْ هَكَذَا».
\par 12 وَجَاءَ عَشَّارُونَ أَيْضاً لِيَعْتَمِدُوا وَسَأَلُوهُ: «يَا مُعَلِّمُ مَاذَا نَفْعَلُ؟»
\par 13 فَأَجَابَ: «لاَ تَسْتَوْفُوا أَكْثَرَ مِمَّا فُرِضَ لَكُمْ».
\par 14 وَسَأَلَهُ جُنْدِيُّونَ أَيْضاً: «وَمَاذَا نَفْعَلُ نَحْنُ؟» فَأجَابَ: «لاَ تَظْلِمُوا أَحَداً وَلاَ تَشُوا بِأَحَدٍ وَاكْتَفُوا بِعَلاَئِفِكُمْ».
\par 15 وَإِذْ كَانَ الشَّعْبُ يَنْتَظِرُ وَالْجَمِيعُ يُفَكِّرُونَ فِي قُلُوبِهِمْ عَنْ يُوحَنَّا لَعَلَّهُ الْمَسِيحُ
\par 16 قَالَ يُوحَنَّا لِلْجَمِيعِ: «أَنَا أُعَمِّدُكُمْ بِمَاءٍ وَلَكِنْ يَأْتِي مَنْ هُوَ أَقْوَى مِنِّي الَّذِي لَسْتُ أَهْلاً أَنْ أَحُلَّ سُيُورَ حِذَائِهِ. هُوَ سَيُعَمِّدُكُمْ بِالرُّوحِ الْقُدُسِ وَنَارٍ.
\par 17 الَّذِي رَفْشُهُ فِي يَدِهِ وَسَيُنَقِّي بَيْدَرَهُ وَيَجْمَعُ الْقَمْحَ إِلَى مَخْزَنِهِ وَأَمَّا التِّبْنُ فَيُحْرِقُهُ بِنَارٍ لاَ تُطْفَأُ».
\par 18 وَبِأَشْيَاءَ أُخَرَ كَثِيرَةٍ كَانَ يَعِظُ الشَّعْبَ وَيُبَشِّرُهُمْ.
\par 19 أَمَّا هِيرُودُسُ رَئِيسُ الرُّبْعِ فَإِذْ تَوَبَّخَ مِنْهُ لِسَبَبِ هِيرُودِيَّا امْرَأَةِ فِيلُبُّسَ أَخِيهِ وَلِسَبَبِ جَمِيعِ الشُّرُورِ الَّتِي كَانَ هِيرُودُسُ يَفْعَلُهَا
\par 20 زَادَ هَذَا أَيْضاً عَلَى الْجَمِيعِ أَنَّهُ حَبَسَ يُوحَنَّا فِي السِّجْنِ.
\par 21 وَلَمَّا اعْتَمَدَ جَمِيعُ الشَّعْبِ اعْتَمَدَ يَسُوعُ أَيْضاً. وَإِذْ كَانَ يُصَلِّي انْفَتَحَتِ السَّمَاءُ
\par 22 وَنَزَلَ عَلَيْهِ الرُّوحُ الْقُدُسُ بِهَيْئَةٍ جِسْمِيَّةٍ مِثْلِ حَمَامَةٍ. وَكَانَ صَوْتٌ مِنَ السَّمَاءِ قَائِلاً: «أَنْتَ ابْنِي الْحَبِيبُ بِكَ سُرِرْتُ!».
\par 23 وَلَمَّا ابْتَدَأَ يَسُوعُ كَانَ لَهُ نَحْوُ ثَلاَثِينَ سَنَةً وَهُوَ عَلَى مَا كَانَ يُظَنُّ ابْنَ يُوسُفَ بْنِ هَالِي
\par 24 بْنِ مَتْثَاتَ بْنِ لاَوِي بْنِ مَلْكِي بْنِ يَنَّا بْنِ يُوسُفَ
\par 25 بْنِ مَتَّاثِيَا بْنِ عَامُوصَ بْنِ نَاحُومَ بْنِ حَسْلِي بْنِ نَجَّايِ
\par 26 بْنِ مَآثَ بْنِ مَتَّاثِيَا بْنِ شِمْعِي بْنِ يُوسُفَ بْنِ يَهُوذَا
\par 27 بْنِ يُوحَنَّا بْنِ رِيسَا بْنِ زَرُبَّابِلَ بْنِ شَأَلْتِئِيلَ بْنِ نِيرِي
\par 28 بْنِ مَلْكِي بْنِ أَدِّي بْنِ قُصَمَ بْنِ أَلْمُودَامَ بْنِ عِيرِ
\par 29 بْنِ يُوسِي بْنِ أَلِيعَازَرَ بْنِ يُورِيمَ بْنِ مَتْثَاتَ بْنِ لاَوِي
\par 30 بْنِ شِمْعُونَ بْنِ يَهُوذَا بْنِ يُوسُفَ بْنِ يُونَانَ بْنِ أَلِيَاقِيمَ
\par 31 بْنِ مَلَيَا بْنِ مَيْنَانَ بْنِ مَتَّاثَا بْنِ نَاثَانَ بْنِ دَاوُدَ
\par 32 بْنِ يَسَّى بْنِ عُوبِيدَ بْنِ بُوعَزَ بْنِ سَلْمُونَ بْنِ نَحْشُونَ
\par 33 بْنِ عَمِّينَادَابَ بْنِ آرَامَ بْنِ حَصْرُونَ بْنِ فَارِصَ بْنِ يَهُوذَا
\par 34 بْنِ يَعْقُوبَ بْنِ إِسْحَاقَ بْنِ إِبْرَاهِيمَ بْنِ تَارَحَ بْنِ نَاحُورَ
\par 35 بْنِ سَرُوجَ بْنِ رَعُو بْنِ فَالَجَ بْنِ عَابِرَ بْنِ شَالَحَ
\par 36 بْنِ قِينَانَ بْنِ أَرْفَكْشَادَ بْنِ سَامِ بْنِ نُوحِ بْنِ لاَمَكَ
\par 37 بْنِ مَتُوشَالَحَ بْنِ أَخْنُوخَ بْنِ يَارِدَ بْنِ مَهْلَلْئِيلَ بْنِ قِينَانَ
\par 38 بْنِ أَنُوشَ بْنِ شِيتِ بْنِ آدَمَ ابْنِ اللهِ.

\chapter{4}

\par 1 أَمَّا يَسُوعُ فَرَجَعَ مِنَ الأُرْدُنِّ مُمْتَلِئاً مِنَ الرُّوحِ الْقُدُسِ وَكَانَ يُقْتَادُ بِالرُّوحِ فِي الْبَرِّيَّةِ
\par 2 أَرْبَعِينَ يَوْماً يُجَرَّبُ مِنْ إِبْلِيسَ. وَلَمْ يَأْكُلْ شَيْئاً فِي تِلْكَ الأَيَّامِ. وَلَمَّا تَمَّتْ جَاعَ أَخِيراً.
\par 3 وَقَالَ لَهُ إِبْلِيسُ: «إِنْ كُنْتَ ابْنَ اللهِ فَقُلْ لِهَذَا الْحَجَرِ أَنْ يَصِيرَ خُبْزاً».
\par 4 فَأَجَابَهُ يَسُوعُ: «مَكْتُوبٌ أَنْ لَيْسَ بِالْخُبْزِ وَحْدَهُ يَحْيَا الإِنْسَانُ بَلْ بِكُلِّ كَلِمَةٍ مِنَ اللهِ».
\par 5 ثُمَّ أَصْعَدَهُ إِبْلِيسُ إِلَى جَبَلٍ عَالٍ وَأَرَاهُ جَمِيعَ مَمَالِكِ الْمَسْكُونَةِ فِي لَحْظَةٍ مِنَ الزَّمَانِ.
\par 6 وَقَالَ لَهُ إِبْلِيسُ: «لَكَ أُعْطِي هَذَا السُّلْطَانَ كُلَّهُ وَمَجْدَهُنَّ لأَنَّهُ إِلَيَّ قَدْ دُفِعَ وَأَنَا أُعْطِيهِ لِمَنْ أُرِيدُ.
\par 7 فَإِنْ سَجَدْتَ أَمَامِي يَكُونُ لَكَ الْجَمِيعُ».
\par 8 فَأَجَابَهُ يَسُوعُ: «اذْهَبْ يَا شَيْطَانُ! إِنَّهُ مَكْتُوبٌ: لِلرَّبِّ إِلَهِكَ تَسْجُدُ وَإِيَّاهُ وَحْدَهُ تَعْبُدُ».
\par 9 ثُمَّ جَاءَ بِهِ إِلَى أُورُشَلِيمَ وَأَقَامَهُ عَلَى جَنَاحِ الْهَيْكَلِ وَقَالَ لَهُ: «إِنْ كُنْتَ ابْنَ اللهِ فَاطْرَحْ نَفْسَكَ مِنْ هُنَا إِلَى أَسْفَلَ
\par 10 لأَنَّهُ مَكْتُوبٌ: أَنَّهُ يُوصِي مَلاَئِكَتَهُ بِكَ لِكَيْ يَحْفَظُوكَ
\par 11 وَأَنَّهُمْ عَلَى أَيَادِيهِمْ يَحْمِلُونَكَ لِكَيْ لاَ تَصْدِمَ بِحَجَرٍ رِجْلَكَ».
\par 12 فَأَجَابَ يَسُوعُ: «إِنَّهُ قِيلَ: لاَ تُجَرِّبِ الرَّبَّ إِلَهَكَ».
\par 13 وَلَمَّا أَكْمَلَ إِبْلِيسُ كُلَّ تَجْرِبَةٍ فَارَقَهُ إِلَى حِينٍ.
\par 14 وَرَجَعَ يَسُوعُ بِقُوَّةِ الرُّوحِ إِلَى الْجَلِيلِ وَخَرَجَ خَبَرٌ عَنْهُ فِي جَمِيعِ الْكُورَةِ الْمُحِيطَةِ.
\par 15 وَكَانَ يُعَلِّمُ فِي مَجَامِعِهِمْ مُمَجَّداً مِنَ الْجَمِيعِ.
\par 16 وَجَاءَ إِلَى النَّاصِرَةِ حَيْثُ كَانَ قَدْ تَرَبَّى. وَدَخَلَ الْمَجْمَعَ حَسَبَ عَادَتِهِ يَوْمَ السَّبْتِ وَقَامَ لِيَقْرَأَ
\par 17 فَدُفِعَ إِلَيْهِ سِفْرُ إِشَعْيَاءَ النَّبِيِّ. وَلَمَّا فَتَحَ السِّفْرَ وَجَدَ الْمَوْضِعَ الَّذِي كَانَ مَكْتُوباً فِيهِ:
\par 18 «رُوحُ الرَّبِّ عَلَيَّ لأَنَّهُ مَسَحَنِي لأُبَشِّرَ الْمَسَاكِينَ أَرْسَلَنِي لأَشْفِيَ الْمُنْكَسِرِي الْقُلُوبِ لأُنَادِيَ لِلْمَأْسُورِينَ بِالإِطْلاَقِ ولِلْعُمْيِ بِالْبَصَرِ وَأُرْسِلَ الْمُنْسَحِقِينَ فِي الْحُرِّيَّةِ
\par 19 وَأَكْرِزَ بِسَنَةِ الرَّبِّ الْمَقْبُولَةِ».
\par 20 ثُمَّ طَوَى السِّفْرَ وَسَلَّمَهُ إِلَى الْخَادِمِ وَجَلَسَ. وَجَمِيعُ الَّذِينَ فِي الْمَجْمَعِ كَانَتْ عُيُونُهُمْ شَاخِصَةً إِلَيْهِ.
\par 21 فَابْتَدَأَ يَقُولُ لَهُمْ: «إِنَّهُ الْيَوْمَ قَدْ تَمَّ هَذَا الْمَكْتُوبُ فِي مَسَامِعِكُمْ».
\par 22 وَكَانَ الْجَمِيعُ يَشْهَدُونَ لَهُ وَيَتَعَجَّبُونَ مِنْ كَلِمَاتِ النِّعْمَةِ الْخَارِجَةِ مِنْ فَمِهِ وَيَقُولُونَ: «أَلَيْسَ هَذَا ابْنَ يُوسُفَ؟»
\par 23 فَقَالَ لَهُمْ: «عَلَى كُلِّ حَالٍ تَقُولُونَ لِي هَذَا الْمَثَلَ: أَيُّهَا الطَّبِيبُ اشْفِ نَفْسَكَ. كَمْ سَمِعْنَا أَنَّهُ جَرَى فِي كَفْرِنَاحُومَ فَافْعَلْ ذَلِكَ هُنَا أَيْضاً فِي وَطَنِكَ
\par 24 وَقَالَ: «الْحَقَّ أَقُولُ لَكُمْ إِنَّهُ لَيْسَ نَبِيٌّ مَقْبُولاً فِي وَطَنِهِ.
\par 25 وَبِالْحَقِّ أَقُولُ لَكُمْ إِنَّ أَرَامِلَ كَثِيرَةً كُنَّ فِي إِسْرَائِيلَ فِي أَيَّامِ إِيلِيَّا حِينَ أُغْلِقَتِ السَّمَاءُ مُدَّةَ ثَلاَثِ سِنِينَ وَسِتَّةِ أَشْهُرٍ لَمَّا كَانَ جُوعٌ عَظِيمٌ فِي الأَرْضِ كُلِّهَا
\par 26 وَلَمْ يُرْسَلْ إِيلِيَّا إِلَى وَاحِدَةٍ مِنْهَا إِلاَّ إِلَى أَرْمَلَةٍ إِلَى صِرْفَةِ صَيْدَاءَ.
\par 27 وَبُرْصٌ كَثِيرُونَ كَانُوا فِي إِسْرَائِيلَ فِي زَمَانِ أَلِيشَعَ النَّبِيِّ وَلَمْ يُطَهَّرْ وَاحِدٌ مِنْهُمْ إِلاَّ نُعْمَانُ السُّرْيَانِيُّ».
\par 28 فَامْتَلَأَ غَضَباً جَمِيعُ الَّذِينَ فِي الْمَجْمَعِ حِينَ سَمِعُوا هَذَا
\par 29 فَقَامُوا وَأَخْرَجُوهُ خَارِجَ الْمَدِينَةِ وَجَاءُوا بِهِ إِلَى حَافَّةَِ الْجَبَلِ الَّذِي كَانَتْ مَدِينَتُهُمْ مَبْنِيَّةً عَلَيْهِ حَتَّى يَطْرَحُوهُ إِلَى أَسْفَلُ.
\par 30 أَمَّا هُوَ فَجَازَ فِي وَسْطِهِمْ وَمَضَى.
\par 31 وَانْحَدَرَ إِلَى كَفْرِنَاحُومَ مَدِينَةٍ مِنَ الْجَلِيلِ وَكَانَ يُعَلِّمُهُمْ فِي السُّبُوتِ.
\par 32 فَبُهِتُوا مِنْ تَعْلِيمِهِ لأَنَّ كَلاَمَهُ كَانَ بِسُلْطَانٍ.
\par 33 وَكَانَ فِي الْمَجْمَعِ رَجُلٌ بِهِ رُوحُ شَيْطَانٍ نَجِسٍ فَصَرَخَ بِصَوْتٍ عَظِيمٍ:
\par 34 «آهِ مَا لَنَا وَلَكَ يَا يَسُوعُ النَّاصِرِيُّ! أَتَيْتَ لِتُهْلِكَنَا! أَنَا أَعْرِفُكَ مَنْ أَنْتَ: قُدُّوسُ اللهِ».
\par 35 فَانْتَهَرَهُ يَسُوعُ قَائِلاً: «اخْرَسْ وَاخْرُجْ مِنْهُ». فَصَرَعَهُ الشَّيْطَانُ فِي الْوَسَطِ وَخَرَجَ مِنْهُ وَلَمْ يَضُرَّهُ شَيْئاً.
\par 36 فَوَقَعَتْ دَهْشَةٌ عَلَى الْجَمِيعِ وَكَانُوا يُخَاطِبُونَ بَعْضُهُمْ بَعْضاً قَائِلِينَ: «مَا هَذِهِ الْكَلِمَةُ! لأَنَّهُ بِسُلْطَانٍ وَقُوَّةٍ يَأْمُرُ الأَرْوَاحَ النَّجِسَةَ فَتَخْرُجُ».
\par 37 وَخَرَجَ صِيتٌ عَنْهُ إِلَى كُلِّ مَوْضِعٍ فِي الْكُورَةِ الْمُحِيطَةِ.
\par 38 وَلَمَّا قَامَ مِنَ الْمَجْمَعِ دَخَلَ بَيْتَ سِمْعَانَ. وَكَانَتْ حَمَاةُ سِمْعَانَ قَدْ أَخَذَتْهَا حُمَّى شَدِيدَةٌ. فَسَأَلُوهُ مِنْ أَجْلِهَا.
\par 39 فَوَقَفَ فَوْقَهَا وَانْتَهَرَ الْحُمَّى فَتَرَكَتْهَا! وَفِي الْحَالِ قَامَتْ وَصَارَتْ تَخْدِمُهُمْ.
\par 40 وَعِنْدَ غُرُوبِ الشَّمْسِ جَمِيعُ الَّذِينَ كَانَ عِنْدَهُمْ سُقَمَاءُ بِأَمْرَاضٍ مُخْتَلِفَةٍ قَدَّمُوهُمْ إِلَيْهِ فَوَضَعَ يَدَيْهِ عَلَى كُلِّ وَاحِدٍ مِنْهُمْ وَشَفَاهُمْ.
\par 41 وَكَانَتْ شَيَاطِينُ أَيْضاً تَخْرُجُ مِنْ كَثِيرِينَ وَهِيَ تَصْرُخُ وَتَقُولُ: «أَنْتَ الْمَسِيحُ ابْنُ اللهِ!» فَانْتَهَرَهُمْ وَلَمْ يَدَعْهُمْ يَتَكَلَّمُونَ لأَنَّهُمْ عَرَفُوهُ أَنَّهُ الْمَسِيحُ.
\par 42 وَلَمَّا صَارَ النَّهَارُ خَرَجَ وَذَهَبَ إِلَى مَوْضِعٍ خَلاَءٍ وَكَانَ الْجُمُوعُ يُفَتِّشُونَ عَلَيْهِ. فَجَاءُوا إِلَيْهِ وَأَمْسَكُوهُ لِئَلاَّ يَذْهَبَ عَنْهُمْ.
\par 43 فَقَالَ لَهُمْ: «إِنَّهُ يَنْبَغِي لِي أَنْ أُبَشِّرَ الْمُدُنَ الأُخَرَ أَيْضاً بِمَلَكُوتِ اللهِ لأَنِّي لِهَذَا قَدْ أُرْسِلْتُ».
\par 44 فَكَانَ يَكْرِزُ فِي مَجَامِعِ الْجَلِيلِ.

\chapter{5}

\par 1 وَإِذْ كَانَ الْجَمْعُ يَزْدَحِمُ عَلَيْهِ لِيَسْمَعَ كَلِمَةَ اللهِ كَانَ وَاقِفاً عِنْدَ بُحَيْرَةِ جَنِّيسَارَتَ.
\par 2 فَرَأَى سَفِينَتَيْنِ وَاقِفَتَيْنِ عِنْدَ الْبُحَيْرَةِ وَالصَّيَّادُونَ قَدْ خَرَجُوا مِنْهُمَا وَغَسَلُوا الشِّبَاكَ.
\par 3 فَدَخَلَ إِحْدَى السَّفِينَتَيْنِ الَّتِي كَانَتْ لِسِمْعَانَ وَسَأَلَهُ أَنْ يُبْعِدَ قَلِيلاً عَنِ الْبَرِّ. ثُمَّ جَلَسَ وَصَارَ يُعَلِّمُ الْجُمُوعَ مِنَ السَّفِينَةِ.
\par 4 وَلَمَّا فَرَغَ مِنَ الْكَلاَمِ قَالَ لِسِمْعَانَ: «ابْعُدْ إِلَى الْعُمْقِ وَأَلْقُوا شِبَاكَكُمْ لِلصَّيْدِ».
\par 5 فَأَجَابَ سِمْعَانُ: «يَا مُعَلِّمُ قَدْ تَعِبْنَا اللَّيْلَ كُلَّهُ وَلَمْ نَأْخُذْ شَيْئاً. وَلَكِنْ عَلَى كَلِمَتِكَ أُلْقِي الشَّبَكَةَ».
\par 6 وَلَمَّا فَعَلُوا ذَلِكَ أَمْسَكُوا سَمَكاً كَثِيراً جِدّاً فَصَارَتْ شَبَكَتُهُمْ تَتَخَرَّقُ.
\par 7 فَأَشَارُوا إِلَى شُرَكَائِهِمُِ الَّذِينَ فِي السَّفِينَةِ الأُخْرَى أَنْ يَأْتُوا وَيُسَاعِدُوهُمْ. فَأَتَوْا وَمَلَأُوا السَّفِينَتَيْنِ حَتَّى أَخَذَتَا فِي الْغَرَقِ.
\par 8 فَلَمَّا رَأَى سِمْعَانُ بُطْرُسُ ذَلِكَ خَرَّ عِنْدَ رُكْبَتَيْ يَسُوعَ قَائِلاً: «اخْرُجْ مِنْ سَفِينَتِي يَارَبُّ لأَنِّي رَجُلٌ خَاطِئٌ».
\par 9 إِذِ اعْتَرَتْهُ وَجمِيعَ الَّذِينَ مَعَهُ دَهْشَةٌ عَلَى صَيْدِ السَّمَكِ الَّذِي أَخَذُوهُ.
\par 10 وَكَذَلِكَ أَيْضاً يَعْقُوبُ وَيُوحَنَّا ابْنَا زَبْدِي اللَّذَانِ كَانَا شَرِيكَيْ سِمْعَانَ. فَقَالَ يَسُوعُ لِسِمْعَانَ: «لاَ تَخَفْ! مِنَ الآنَ تَكُونُ تَصْطَادُ النَّاسَ!»
\par 11 وَلَمَّا جَاءُوا بِالسَّفِينَتَيْنِ إِلَى الْبَرِّ تَرَكُوا كُلَّ شَيْءٍ وَتَبِعُوهُ.
\par 12 وَكَانَ فِي إِحْدَى الْمُدُنِ. فَإِذَا رَجُلٌ مَمْلُوءٌ بَرَصاً. فَلَمَّا رَأَى يَسُوعَ خَرَّ عَلَى وَجْهِهِ وَطَلَبَ إِلَيْهِ قَائِلاً: «يَا سَيِّدُ إِنْ أَرَدْتَ تَقْدِرْ أَنْ تُطَهِّرَنِي».
\par 13 فَمَدَّ يَدَهُ وَلَمَسَهُ قَائِلاً: «أُرِيدُ فَاطْهُرْ». وَلِلْوَقْتِ ذَهَبَ عَنْهُ الْبَرَصُ.
\par 14 فَأَوْصَاهُ أَنْ لاَ يَقُولَ لأَحَدٍ. بَلِ «امْضِ وَأَرِ نَفْسَكَ لِلْكَاهِنِ وَقَدِّمْ عَنْ تَطْهِيرِكَ كَمَا أَمَرَ مُوسَى شَهَادَةً لَهُمْ».
\par 15 فَذَاعَ الْخَبَرُ عَنْهُ أَكْثَرَ. فَاجْتَمَعَ جُمُوعٌ كَثِيرَةٌ لِكَيْ يَسْمَعُوا وَيُشْفَوْا بِهِ مِنْ أَمْرَاضِهِمْ.
\par 16 وَأَمَّا هُوَ فَكَانَ يَعْتَزِلُ فِي الْبَرَارِي وَيُصَلِّي.
\par 17 وَفِي أَحَدِ الأَيَّامِ كَانَ يُعَلِّمُ وَكَانَ فَرِّيسِيُّونَ وَمُعَلِّمُونَ لِلنَّامُوسِ جَالِسِينَ وَهُمْ قَدْ أَتَوْا مِنْ كُلِّ قَرْيَةٍ مِنَ الْجَلِيلِ وَالْيَهُودِيَّةِ وَأُورُشَلِيمَ. وَكَانَتْ قُوَّةُ الرَّبِّ لِشِفَائِهِمْ.
\par 18 وَإِذَا بِرِجَالٍ يَحْمِلُونَ عَلَى فِرَاشٍ إِنْسَاناً مَفْلُوجاً وَكَانُوا يَطْلُبُونَ أَنْ يَدْخُلُوا بِهِ وَيَضَعُوهُ أَمَامَهُ.
\par 19 وَلَمَّا لَمْ يَجِدُوا مِنْ أَيْنَ يَدْخُلُونَ بِهِ لِسَبَبِ الْجَمْعِ صَعِدُوا عَلَى السَّطْحِ وَدَلَّوْهُ مَعَ الْفِرَاشِ مِنْ بَيْنِ الأَجُرِّ إِلَى الْوَسَطِ قُدَّامَ يَسُوعَ.
\par 20 فَلَمَّا رَأَى إِيمَانَهُمْ قَالَ لَهُ: «أَيُّهَا الإِنْسَانُ مَغْفُورَةٌ لَكَ خَطَايَاكَ».
\par 21 فَابْتَدَأَ الْكَتَبَةُ وَالْفَرِّيسِيُّونَ يُفَكِّرُونَ قَائِلِينَ: «مَنْ هَذَا الَّذِي يَتَكَلَّمُ بِتَجَادِيفَ؟ مَنْ يَقْدِرُ أَنْ يَغْفِرَ خَطَايَا إِلاَّ اللهُ وَحْدَهُ؟»
\par 22 فَشَعَرَ يَسُوعُ بِأَفْكَارِهِمْ وَقَالَ لَهُمْ: «مَاذَا تُفَكِّرُونَ فِي قُلُوبِكُمْ؟
\par 23 أَيُّمَا أَيْسَرُ: أَنْ يُقَالَ مَغْفُورَةٌ لَكَ خَطَايَاكَ أَمْ أَنْ يُقَالَ قُمْ وَامْشِ.
\par 24 وَلَكِنْ لِكَيْ تَعْلَمُوا أَنَّ لاِبْنِ الإِنْسَانِ سُلْطَاناً عَلَى الأَرْضِ أَنْ يَغْفِرَ الْخَطَايَا» - قَالَ لِلْمَفْلُوجِ: «لَكَ أَقُولُ قُمْ وَاحْمِلْ فِرَاشَكَ وَاذْهَبْ إِلَى بَيْتِكَ».
\par 25 فَفِي الْحَالِ قَامَ أَمَامَهُمْ وَحَمَلَ مَا كَانَ مُضْطَجِعاً عَلَيْهِ وَمَضَى إِلَى بَيْتِهِ وَهُوَ يُمَجِّدُ اللهَ.
\par 26 فَأَخَذَتِ الْجَمِيعَ حَيْرَةٌ وَمَجَّدُوا اللهَ وَامْتَلَأُوا خَوْفاً قَائِلِينَ: «إِنَّنَا قَدْ رَأَيْنَا الْيَوْمَ عَجَائِبَ!».
\par 27 وَبَعْدَ هَذَا خَرَجَ فَنَظَرَ عَشَّاراً اسْمُهُ لاَوِي جَالِساً عِنْدَ مَكَانِ الْجِبَايَةِ فَقَالَ لَهُ: «اتْبَعْنِي».
\par 28 فَتَرَكَ كُلَّ شَيْءٍ وَقَامَ وَتَبِعَهُ.
\par 29 وَصَنَعَ لَهُ لاَوِي ضِيَافَةً كَبِيرَةً فِي بَيْتِهِ. وَالَّذِينَ كَانُوا مُتَّكِئِينَ مَعَهُمْ كَانُوا جَمْعاً كَثِيراً مِنْ عَشَّارِينَ وَآخَرِينَ.
\par 30 فَتَذَمَّرَ كَتَبَتُهُمْ وَالْفَرِّيسِيُّونَ عَلَى تَلاَمِيذِهِ قَائِلِينَ: «لِمَاذَا تَأْكُلُونَ وَتَشْرَبُونَ مَعَ عَشَّارِينَ وَخُطَاةٍ؟»
\par 31 فَأَجَابَ يَسُوعُ: «لاَ يَحْتَاجُ الأَصِحَّاءُ إِلَى طَبِيبٍ بَلِ الْمَرْضَى.
\par 32 لَمْ آتِ لأَدْعُوَ أَبْرَاراً بَلْ خُطَاةً إِلَى التَّوْبَةِ».
\par 33 وَقَالُوا لَهُ: «لِمَاذَا يَصُومُ تَلاَمِيذُ يُوحَنَّا كَثِيراً وَيُقَدِّمُونَ طِلْبَاتٍ وَكَذَلِكَ تَلاَمِيذُ الْفَرِّيسِيِّينَ أَيْضاً وَأَمَّا تَلاَمِيذُكَ فَيَأْكُلُونَ وَيَشْرَبُونَ؟»
\par 34 فَقَالَ لَهُمْ: «أَتَقْدِرُونَ أَنْ تَجْعَلُوا بَنِي الْعُرْسِ يَصُومُونَ مَا دَامَ الْعَرِيسُ مَعَهُمْ؟
\par 35 وَلَكِنْ سَتَأْتِي أَيَّامٌ حِينَ يُرْفَعُ الْعَرِيسُ عَنْهُمْ فَحِينَئِذٍ يَصُومُونَ فِي تِلْكَ الأَيَّامِ».
\par 36 وَقَالَ لَهُمْ أَيْضاً مَثَلاً: «لَيْسَ أَحَدٌ يَضَعُ رُقْعَةً مِنْ ثَوْبٍ جَدِيدٍ عَلَى ثَوْبٍ عَتِيقٍ وَإِلاَّ فَالْجَدِيدُ يَشُقُّهُ وَالْعَتِيقُ لاَ تُوافِقُهُ الرُّقْعَةُ الَّتِي مِنَ الْجَدِيدِ.
\par 37 وَلَيْسَ أَحَدٌ يَجْعَلُ خَمْراً جَدِيدَةً فِي زِقَاقٍ عَتِيقَةٍ لِئَلاَّ تَشُقَّ الْخَمْرُ الْجَدِيدَةُ الزِّقَاقَ فَهِيَ تُهْرَقُ وَالزِّقَاقُ تَتْلَفُ.
\par 38 بَلْ يَجْعَلُونَ خَمْراً جَدِيدَةً فِي زِقَاقٍ جَدِيدَةٍ فَتُحْفَظُ جَمِيعاً.
\par 39 وَلَيْسَ أَحَدٌ إِذَا شَرِبَ الْعَتِيقَ يُرِيدُ لِلْوَقْتِ الْجَدِيدَ لأَنَّهُ يَقُولُ: الْعَتِيقُ أَطْيَبُ».

\chapter{6}

\par 1 وَفِي السَّبْتِ الثَّانِي بَعْدَ الأَوَّلِ اجْتَازَ بَيْنَ الزُّرُوعِ. وَكَانَ تَلاَمِيذُهُ يَقْطِفُونَ السَّنَابِلَ وَيَأْكُلُونَ وَهُمْ يَفْرُكُونَهَا بِأَيْدِيهِمْ.
\par 2 فَقَالَ لَهُمْ قَوْمٌ مِنَ الْفَرِّيسِيِّينَ: «لِمَاذَا تَفْعَلُونَ مَا لاَ يَحِلُّ فِعْلُهُ فِي السُّبُوتِ؟»
\par 3 فَأَجَابَ يَسُوعُ: «أَمَا قَرَأْتُمْ وَلاَ هَذَا الَّذِي فَعَلَهُ دَاوُدُ حِينَ جَاعَ هُوَ وَالَّذِينَ كَانُوا مَعَهُ
\par 4 كَيْفَ دَخَلَ بَيْتَ اللهِ وَأَخَذَ خُبْزَ التَّقْدِمَةِ وَأَكَلَ وَأَعْطَى الَّذِينَ مَعَهُ أَيْضاً الَّذِي لاَ يَحِلُّ أَكْلُهُ إِلاَّ لِلْكَهَنَةِ فَقَطْ؟»
\par 5 وَقَالَ لَهُمْ: «إِنَّ ابْنَ الإِنْسَانِ هُوَ رَبُّ السَّبْتِ أَيْضاً».
\par 6 وَفِي سَبْتٍ آخَرَ دَخَلَ الْمَجْمَعَ وَصَارَ يُعَلِّمُ. وَكَانَ هُنَاكَ رَجُلٌ يَدُهُ الْيُمْنَى يَابِسَةٌ
\par 7 وَكَانَ الْكَتَبَةُ وَالْفَرِّيسِيُّونَ يُرَاقِبُونَهُ: هَلْ يَشْفِي فِي السَّبْتِ لِكَيْ يَجِدُوا عَلَيْهِ شِكَايَةً.
\par 8 أَمَّا هُوَ فَعَلِمَ أَفْكَارَهُمْ وَقَالَ لِلرَّجُلِ الَّذِي يَدُهُ يَابِسَةٌ: «قُمْ وَقِفْ فِي الْوَسَطِ». فَقَامَ وَوَقَفَ.
\par 9 ثُمَّ قَالَ لَهُمْ يَسُوعُ: «أَسْأَلُكُمْ شَيْئاً: هَلْ يَحِلُّ فِي السَّبْتِ فِعْلُ الْخَيْرِ أَوْ فِعْلُ الشَّرِّ؟ تَخْلِيصُ نَفْسٍ أَوْ إِهْلاَكُهَا؟».
\par 10 ثُمَّ نَظَرَ حَوْلَهُ إِلَى جَمِيعِهِمْ وَقَالَ لِلرَّجُلِ: «مُدَّ يَدَكَ». فَفَعَلَ هَكَذَا. فَعَادَتْ يَدُهُ صَحِيحَةً كَالأُخْرَى.
\par 11 فَامْتَلَأُوا حُمْقاً وَصَارُوا يَتَكَالَمُونَ فِيمَا بَيْنَهُمْ: مَاذَا يَفْعَلُونَ بِيَسُوعَ؟
\par 12 وَفِي تِلْكَ الأَيَّامِ خَرَجَ إِلَى الْجَبَلِ لِيُصَلِّيَ. وَقَضَى اللَّيْلَ كُلَّهُ فِي الصَّلاَةِ لِلَّهِ.
\par 13 وَلَمَّا كَانَ النَّهَارُ دَعَا تَلاَمِيذَهُ وَاخْتَارَ مِنْهُمُ اثْنَيْ عَشَرَ الَّذِينَ سَمَّاهُمْ أَيْضاً «رُسُلاً»:
\par 14 سِمْعَانَ الَّذِي سَمَّاهُ أَيْضاً بُطْرُسَ وَأَنْدَرَاوُسَ أَخَاهُ. يَعْقُوبَ وَيُوحَنَّا. فِيلُبُّسَ وَبَرْثُولَمَاوُسَ.
\par 15 مَتَّى وَتُومَا. يَعْقُوبَ بْنَ حَلْفَى وَسِمْعَانَ الَّذِي يُدْعَى الْغَيُورَ.
\par 16 يَهُوذَا بْنَ يَعْقُوبَ وَيَهُوذَا الإِسْخَرْيُوطِيَّ الَّذِي صَارَ مُسَلِّماً أَيْضاً.
\par 17 وَنَزَلَ مَعَهُمْ وَوَقَفَ فِي مَوْضِعٍ سَهْلٍ هُوَ وَجَمْعٌ مِنْ تَلاَمِيذِهِ وَجُمْهُورٌ كَثِيرٌ مِنَ الشَّعْبِ مِنْ جَمِيعِ الْيَهُودِيَّةِ وَأُورُشَلِيمَ وَسَاحِلِ صُورَ وَصَيْدَاءَ الَّذِينَ جَاءُوا لِيَسْمَعُوهُ وَيُشْفَوْا مِنْ أَمْرَاضِهِمْ
\par 18 وَالْمُعَذَّبُونَ مِنْ أَرْوَاحٍ نَجِسَةٍ. وَكَانُوا يَبْرَأُونَ.
\par 19 وَكُلُّ الْجَمْعِ طَلَبُوا أَنْ يَلْمِسُوهُ لأَنَّ قُوَّةً كَانَتْ تَخْرُجُ مِنْهُ وَتَشْفِي الْجَمِيعَ.
\par 20 وَرَفَعَ عَيْنَيْهِ إِلَى تَلاَمِيذِهِ وَقَالَ: «طُوبَاكُمْ أَيُّهَا الْمَسَاكِينُ لأَنَّ لَكُمْ مَلَكُوتَ اللهِ.
\par 21 طُوبَاكُمْ أَيُّهَا الْجِيَاعُ الآنَ لأَنَّكُمْ تُشْبَعُونَ. طُوبَاكُمْ أَيُّهَا الْبَاكُونَ الآنَ لأَنَّكُمْ سَتَضْحَكُونَ.
\par 22 طُوبَاكُمْ إِذَا أَبْغَضَكُمُ النَّاسُ وَإِذَا أَفْرَزُوكُمْ وَعَيَّرُوكُمْ وَأَخْرَجُوا اسْمَكُمْ كَشِرِّيرٍ مِنْ أَجْلِ ابْنِ الإِنْسَانِ.
\par 23 افْرَحُوا فِي ذَلِكَ الْيَوْمِ وَتَهَلَّلُوا فَهُوَذَا أَجْرُكُمْ عَظِيمٌ فِي السَّمَاءِ. لأَنَّ آبَاءَهُمْ هَكَذَا كَانُوا يَفْعَلُونَ بِالأَنْبِيَاءِ.
\par 24 وَلَكِنْ وَيْلٌ لَكُمْ أَيُّهَا الأَغْنِيَاءُ لأَنَّكُمْ قَدْ نِلْتُمْ عَزَاءَكُمْ.
\par 25 وَيْلٌ لَكُمْ أَيُّهَا الشَّبَاعَى لأَنَّكُمْ سَتَجُوعُونَ. وَيْلٌ لَكُمْ أَيُّهَا الضَّاحِكُونَ الآنَ لأَنَّكُمْ سَتَحْزَنُونَ وَتَبْكُونَ.
\par 26 وَيْلٌ لَكُمْ إِذَا قَالَ فِيكُمْ جَمِيعُ النَّاسِ حَسَناً. لأَنَّهُ هَكَذَا كَانَ آبَاؤُهُمْ يَفْعَلُونَ بِالأَنْبِيَاءِ الْكَذَبَةِ.
\par 27 «لَكِنِّي أَقُولُ لَكُمْ أَيُّهَا السَّامِعُونَ: أَحِبُّوا أَعْدَاءَكُمْ أَحْسِنُوا إِلَى مُبْغِضِيكُمْ
\par 28 بَارِكُوا لاَعِنِيكُمْ وَصَلُّوا لأَجْلِ الَّذِينَ يُسِيئُونَ إِلَيْكُمْ.
\par 29 مَنْ ضَرَبَكَ عَلَى خَدِّكَ فَاعْرِضْ لَهُ الآخَرَ أَيْضاً وَمَنْ أَخَذَ رِدَاءَكَ فَلاَ تَمْنَعْهُ ثَوْبَكَ أَيْضاً.
\par 30 وَكُلُّ مَنْ سَأَلَكَ فَأَعْطِهِ وَمَنْ أَخَذَ الَّذِي لَكَ فَلاَ تُطَالِبْهُ.
\par 31 وَكَمَا تُرِيدُونَ أَنْ يَفْعَلَ النَّاسُ بِكُمُ افْعَلُوا أَنْتُمْ أَيْضاً بِهِمْ هَكَذَا.
\par 32 وَإِنْ أَحْبَبْتُمُ الَّذِينَ يُحِبُّونَكُمْ فَأَيُّ فَضْلٍ لَكُمْ؟ فَإِنَّ الْخُطَاةَ أَيْضاً يُحِبُّونَ الَّذِينَ يُحِبُّونَهُمْ.
\par 33 وَإِذَا أَحْسَنْتُمْ إِلَى الَّذِينَ يُحْسِنُونَ إِلَيْكُمْ فَأَيُّ فَضْلٍ لَكُمْ؟ فَإِنَّ الْخُطَاةَ أَيْضاً يَفْعَلُونَ هَكَذَا.
\par 34 وَإِنْ أَقْرَضْتُمُ الَّذِينَ تَرْجُونَ أَنْ تَسْتَرِدُّوا مِنْهُمْ فَأَيُّ فَضْلٍ لَكُمْ؟ فَإِنَّ الْخُطَاةَ أَيْضاً يُقْرِضُونَ الْخُطَاةَ لِكَيْ يَسْتَرِدُّوا مِنْهُمُ الْمِثْلَ.
\par 35 بَلْ أَحِبُّوا أَعْدَاءَكُمْ وَأَحْسِنُوا وَأَقْرِضُوا وَأَنْتُمْ لاَ تَرْجُونَ شَيْئاً فَيَكُونَ أَجْرُكُمْ عَظِيماً وَتَكُونُوا بَنِي الْعَلِيِّ فَإِنَّهُ مُنْعِمٌ عَلَى غَيْرِ الشَّاكِرِينَ وَالأَشْرَارِ.
\par 36 فَكُونُوا رُحَمَاءَ كَمَا أَنَّ أَبَاكُمْ أَيْضاً رَحِيمٌ.
\par 37 وَلاَ تَدِينُوا فَلاَ تُدَانُوا. لاَ تَقْضُوا عَلَى أَحَدٍ فَلاَ يُقْضَى عَلَيْكُمْ. اِغْفِرُوا يُغْفَرْ لَكُمْ.
\par 38 أَعْطُوا تُعْطَوْا كَيْلاً جَيِّداً مُلَبَّداً مَهْزُوزاً فَائِضاً يُعْطُونَ فِي أَحْضَانِكُمْ. لأَنَّهُ بِنَفْسِ الْكَيْلِ الَّذِي بِهِ تَكِيلُونَ يُكَالُ لَكُمْ».
\par 39 وَضَرَبَ لَهُمْ مَثَلاً: «هَلْ يَقْدِرُ أَعْمَى أَنْ يَقُودَ أَعْمَى؟ أَمَا يَسْقُطُ الاِثْنَانِ فِي حُفْرَةٍ؟
\par 40 لَيْسَ التِّلْمِيذُ أَفْضَلَ مِنْ مُعَلِّمِهِ بَلْ كُلُّ مَنْ صَارَ كَامِلاً يَكُونُ مِثْلَ مُعَلِّمِهِ.
\par 41 لِمَاذَا تَنْظُرُ الْقَذَى الَّذِي فِي عَيْنِ أَخِيكَ وَأَمَّا الْخَشَبَةُ الَّتِي فِي عَيْنِكَ فَلاَ تَفْطَنُ لَهَا؟
\par 42 أَوْ كَيْفَ تَقْدِرُ أَنْ تَقُولَ لأَخِيكَ: يَا أَخِي دَعْنِي أُخْرِجِ الْقَذَى الَّذِي فِي عَيْنِكَ وَأَنْتَ لاَ تَنْظُرُ الْخَشَبَةَ الَّتِي فِي عَيْنِكَ. يَا مُرَائِي! أَخْرِجْ أَوَّلاً الْخَشَبَةَ مِنْ عَيْنِكَ وَحِينَئِذٍ تُبْصِرُ جَيِّداً أَنْ تُخْرِجَ الْقَذَى الَّذِي فِي عَيْنِ أَخِيكَ.
\par 43 لأَنَّهُ مَا مِنْ شَجَرَةٍ جَيِّدَةٍ تُثْمِرُ ثَمَراً رَدِيّاً وَلاَ شَجَرَةٍ رَدِيَّةٍ تُثْمِرُ ثَمَراً جَيِّداً.
\par 44 لأَنَّ كُلَّ شَجَرَةٍ تُعْرَفُ مِنْ ثَمَرِهَا. فَإِنَّهُمْ لاَ يَجْتَنُونَ مِنَ الشَّوْكِ تِيناً وَلاَ يَقْطِفُونَ مِنَ الْعُلَّيْقِ عِنَباً.
\par 45 اَلإِنْسَانُ الصَّالِحُ مِنْ كَنْزِ قَلْبِهِ الصَّالِحِ يُخْرِجُ الصَّلاَحَ وَالإِنْسَانُ الشِّرِّيرُ مِنْ كَنْزِ قَلْبِهِ الشِّرِّيرِ يُخْرِجُ الشَّرَّ. فَإِنَّهُ مِنْ فَضْلَةِ الْقَلْبِ يَتَكَلَّمُ فَمُهُ.
\par 46 وَلِمَاذَا تَدْعُونَنِي: يَا رَبُّ يَا رَبُّ وَأَنْتُمْ لاَ تَفْعَلُونَ مَا أَقُولُهُ؟
\par 47 كُلُّ مَنْ يَأْتِي إِلَيَّ وَيَسْمَعُ كَلاَمِي وَيَعْمَلُ بِهِ
\par 48 يُشْبِهُ إِنْسَاناً بَنَى بَيْتاً وَحَفَرَ وَعَمَّقَ وَوَضَعَ الأَسَاسَ عَلَى الصَّخْرِ. فَلَمَّا حَدَثَ سَيْلٌ صَدَمَ النَّهْرُ ذَلِكَ الْبَيْتَ فَلَمْ يَقْدِرْ أَنْ يُزَعْزِعَهُ لأَنَّهُ كَانَ مُؤَسَّساً عَلَى الصَّخْرِ.
\par 49 وَأَمَّا الَّذِي يَسْمَعُ وَلاَ يَعْمَلُ فَيُشْبِهُ إِنْسَاناً بَنَى بَيْتَهُ عَلَى الأَرْضِ مِنْ دُونِ أَسَاسٍ فَصَدَمَهُ النَّهْرُ فَسَقَطَ حَالاً وَكَانَ خَرَابُ ذَلِكَ الْبَيْتِ عَظِيماً».

\chapter{7}

\par 1 وَلَمَّا أَكْمَلَ أَقْوَالَهُ كُلَّهَا فِي مَسَامِعِ الشَّعْبِ دَخَلَ كَفْرَنَاحُومَ.
\par 2 وَكَانَ عَبْدٌ لِقَائِدِ مِئَةٍ مَرِيضاً مُشْرِفاً عَلَى الْمَوْتِ وَكَانَ عَزِيزاً عِنْدَهُ.
\par 3 فَلَمَّا سَمِعَ عَنْ يَسُوعَ أَرْسَلَ إِلَيْهِ شُيُوخَ الْيَهُودِ يَسْأَلُهُ أَنْ يَأْتِيَ وَيَشْفِيَ عَبْدَهُ.
\par 4 فَلَمَّا جَاءُوا إِلَى يَسُوعَ طَلَبُوا إِلَيْهِ بِاجْتِهَادٍ قَائِلِينَ: «إِنَّهُ مُسْتَحِقٌّ أَنْ يُفْعَلَ لَهُ هَذَا
\par 5 لأَنَّهُ يُحِبُّ أُمَّتَنَا وَهُوَ بَنَى لَنَا الْمَجْمَعَ».
\par 6 فَذَهَبَ يَسُوعُ مَعَهُمْ. وَإِذْ كَانَ غَيْرَ بَعِيدٍ عَنِ الْبَيْتِ أَرْسَلَ إِلَيْهِ قَائِدُ الْمِئَةِ أَصْدِقَاءَ يَقُولُ لَهُ: «يَا سَيِّدُ لاَ تَتْعَبْ. لأَنِّي لَسْتُ مُسْتَحِقّاً أَنْ تَدْخُلَ تَحْتَ سَقْفِي.
\par 7 لِذَلِكَ لَمْ أَحْسِبْ نَفْسِي أَهْلاً أَنْ آتِيَ إِلَيْكَ. لَكِنْ قُلْ كَلِمَةً فَيَبْرَأَ غُلاَمِي.
\par 8 لأَنِّي أَنَا أَيْضاً إِنْسَانٌ مُرَتَّبٌ تَحْتَ سُلْطَانٍ لِي جُنْدٌ تَحْتَ يَدِي. وَأَقُولُ لِهَذَا: اذْهَبْ فَيَذْهَبُ وَلِآخَرَ: ائْتِ فَيَأْتِي وَلِعَبْدِي: افْعَلْ هَذَا فَيَفْعَلُ».
\par 9 وَلَمَّا سَمِعَ يَسُوعُ هَذَا تَعَجَّبَ مِنْهُ وَالْتَفَتَ إِلَى الْجَمْعِ الَّذِي يَتْبَعُهُ وَقَالَ: «أَقُولُ لَكُمْ: لَمْ أَجِدْ وَلاَ فِي إِسْرَائِيلَ إِيمَاناً بِمِقْدَارِ هَذَا».
\par 10 وَرَجَعَ الْمُرْسَلُونَ إِلَى الْبَيْتِ فَوَجَدُوا الْعَبْدَ الْمَرِيضَ قَدْ صَحَّ.
\par 11 وَفِي الْيَوْمِ التَّالِي ذَهَبَ إِلَى مَدِينَةٍ تُدْعَى نَايِينَ وَذَهَبَ مَعَهُ كَثِيرُونَ مِنْ تَلاَمِيذِهِ وَجَمْعٌ كَثِيرٌ.
\par 12 فَلَمَّا اقْتَرَبَ إِلَى بَابِ الْمَدِينَةِ إِذَا مَيْتٌ مَحْمُولٌ ابْنٌ وَحِيدٌ لأُمِّهِ وَهِيَ أَرْمَلَةٌ وَمَعَهَا جَمْعٌ كَثِيرٌ مِنَ الْمَدِينَةِ.
\par 13 فَلَمَّا رَآهَا الرَّبُّ تَحَنَّنَ عَلَيْهَا وَقَالَ لَهَا: «لاَ تَبْكِي».
\par 14 ثُمَّ تَقَدَّمَ وَلَمَسَ النَّعْشَ فَوَقَفَ الْحَامِلُونَ. فَقَالَ: «أَيُّهَا الشَّابُّ لَكَ أَقُولُ قُمْ».
\par 15 فَجَلَسَ الْمَيْتُ وَابْتَدَأَ يَتَكَلَّمُ فَدَفَعَهُ إِلَى أُمِّهِ.
\par 16 فَأَخَذَ الْجَمِيعَ خَوْفٌ وَمَجَّدُوا اللهَ قَائِلِينَ: «قَدْ قَامَ فِينَا نَبِيٌّ عَظِيمٌ وَافْتَقَدَ اللهُ شَعْبَهُ».
\par 17 وَخَرَجَ هَذَا الْخَبَرُ عَنْهُ فِي كُلِّ الْيَهُودِيَّةِ وَفِي جَمِيعِ الْكُورَةِ الْمُحِيطَةِ.
\par 18 فَأَخْبَرَ يُوحَنَّا تَلاَمِيذُهُ بِهَذَا كُلِّهِ.
\par 19 فَدَعَا يُوحَنَّا اثْنَيْنِ مِنْ تَلاَمِيذِهِ وَأَرْسَلَ إِلَى يَسُوعَ قَائِلاً: «أَنْتَ هُوَ الآتِي أَمْ نَنْتَظِرُ آخَرَ؟»
\par 20 فَلَمَّا جَاءَ إِلَيْهِ الرَّجُلاَنِ قَالاَ: «يُوحَنَّا الْمَعْمَدَانُ قَدْ أَرْسَلَنَا إِلَيْكَ قَائِلاً: أَنْتَ هُوَ الآتِي أَمْ نَنْتَظِرُ آخَرَ؟»
\par 21 وَفِي تِلْكَ السَّاعَةِ شَفَى كَثِيرِينَ مِنْ أَمْرَاضٍ وَأَدْوَاءٍ وَأَرْوَاحٍ شِرِّيرَةٍ وَوَهَبَ الْبَصَرَ لِعُمْيَانٍ كَثِيرِينَ.
\par 22 فَأَجَابَ يَسُوعُ: «اذْهَبَا وَأَخْبِرَا يُوحَنَّا بِمَا رَأَيْتُمَا وَسَمِعْتُمَا: إِنَّ الْعُمْيَ يُبْصِرُونَ وَالْعُرْجَ يَمْشُونَ وَالْبُرْصَ يُطَهَّرُونَ وَالصُّمَّ يَسْمَعُونَ وَالْمَوْتَى يَقُومُونَ وَالْمَسَاكِينَ يُبَشَّرُونَ.
\par 23 وَطُوبَى لِمَنْ لاَ يَعْثُرُ فِيَّ».
\par 24 فَلَمَّا مَضَى رَسُولاَ يُوحَنَّا ابْتَدَأَ يَقُولُ لِلْجُمُوعِ عَنْ يُوحَنَّا: «مَاذَا خَرَجْتُمْ إِلَى الْبَرِّيَّةِ لِتَنْظُرُوا؟ أَقَصَبَةً تُحَرِّكُهَا الرِّيحُ؟
\par 25 بَلْ مَاذَا خَرَجْتُمْ لِتَنْظُرُوا؟ أَإِنْسَاناً لاَبِساً ثِيَاباً نَاعِمَةً؟ هُوَذَا الَّذِينَ فِي اللِّبَاسِ الْفَاخِرِ وَالتَّنَعُّمِ هُمْ فِي قُصُورِ الْمُلُوكِ.
\par 26 بَلْ مَاذَا خَرَجْتُمْ لِتَنْظُرُوا؟ أَنَبِيّاً؟ نَعَمْ أَقُولُ لَكُمْ وَأَفْضَلَ مِنْ نَبِيٍّ!
\par 27 هَذَا هُوَ الَّذِي كُتِبَ عَنْهُ: هَا أَنَا أُرْسِلُ أَمَامَ وَجْهِكَ مَلاَكِي الَّذِي يُهَيِّئُ طَرِيقَكَ قُدَّامَكَ!
\par 28 لأَنِّي أَقُولُ لَكُمْ: إِنَّهُ بَيْنَ الْمَوْلُودِينَ مِنَ النِّسَاءِ لَيْسَ نَبِيٌّ أَعْظَمَ مِنْ يُوحَنَّا الْمَعْمَدَانِ وَلَكِنَّ الأَصْغَرَ فِي مَلَكُوتِ اللهِ أَعْظَمُ مِنْهُ».
\par 29 وَجَمِيعُ الشَّعْبِ إِذْ سَمِعُوا وَالْعَشَّارُونَ بَرَّرُوا اللهَ مُعْتَمِدِينَ بِمَعْمُودِيَّةِ يُوحَنَّا.
\par 30 وَأَمَّا الْفَرِّيسِيُّونَ وَالنَّامُوسِيُّونَ فَرَفَضُوا مَشُورَةَ اللهِ مِنْ جِهَةِ أَنْفُسِهِمْ غَيْرَ مُعْتَمِدِينَ مِنْهُ.
\par 31 ثُمَّ قَالَ الرَّبُّ: «فَبِمَنْ أُشَبِّهُ أُنَاسَ هَذَا الْجِيلِ وَمَاذَا يُشْبِهُونَ؟
\par 32 يُشْبِهُونَ أَوْلاَداً جَالِسِينَ فِي السُّوقِ يُنَادُونَ بَعْضُهُمْ بَعْضاً وَيَقُولُونَ: زَمَّرْنَا لَكُمْ فَلَمْ تَرْقُصُوا. نُحْنَا لَكُمْ فَلَمْ تَبْكُوا.
\par 33 لأَنَّهُ جَاءَ يُوحَنَّا الْمَعْمَدَانُ لاَ يَأْكُلُ خُبْزاً وَلاَ يَشْرَبُ خَمْراً فَتَقُولُونَ: بِهِ شَيْطَانٌ.
\par 34 جَاءَ ابْنُ الإِنْسَانِ يَأْكُلُ وَيَشْرَبُ فَتَقُولُونَ: هُوَذَا إِنْسَانٌ أَكُولٌ وَشِرِّيبُ خَمْرٍ مُحِبٌّ لِلْعَشَّارِينَ وَالْخُطَاةِ.
\par 35 وَالْحِكْمَةُ تَبَرَّرَتْ مِنْ جَمِيعِ بَنِيهَا».
\par 36 وَسَأَلَهُ وَاحِدٌ مِنَ الْفَرِّيسِيِّينَ أَنْ يَأْكُلَ مَعَهُ فَدَخَلَ بَيْتَ الْفَرِّيسِيِّ وَاتَّكَأَ.
\par 37 وَإِذَا امْرَأَةٌ فِي الْمَدِينَةِ كَانَتْ خَاطِئَةً إِذْ عَلِمَتْ أَنَّهُ مُتَّكِئٌ فِي بَيْتِ الْفَرِّيسِيِّ جَاءَتْ بِقَارُورَةِ طِيبٍ
\par 38 وَوَقَفَتْ عِنْدَ قَدَمَيْهِ مِنْ وَرَائِهِ بَاكِيَةً وَابْتَدَأَتْ تَبُلُّ قَدَمَيْهِ بِالدُّمُوعِ وَكَانَتْ تَمْسَحُهُمَا بِشَعْرِ رَأْسِهَا وَتُقَبِّلُ قَدَمَيْهِ وَتَدْهَنُهُمَا بِالطِّيبِ.
\par 39 فَلَمَّا رَأَى الْفَرِّيسِيُّ الَّذِي دَعَاهُ ذَلِكَ قَالَ فِي نَفْسِهِ: «لَوْ كَانَ هَذَا نَبِيّاً لَعَلِمَ مَنْ هَذِهِ الْمَرْأَةُ الَّتِي تَلْمِسُهُ وَمَا هِيَ! إِنَّهَا خَاطِئِةٌ».
\par 40 فَقَالَ يَسُوعُ: «يَا سِمْعَانُ عِنْدِي شَيْءٌ أَقُولُهُ لَكَ». فَقَالَ: «قُلْ يَا مُعَلِّمُ».
\par 41 «كَانَ لِمُدَايِنٍ مَدْيُونَانِ. عَلَى الْوَاحِدِ خَمْسُ مِئَةِ دِينَارٍ وَعَلَى الآخَرِ خَمْسُونَ.
\par 42 وَإِذْ لَمْ يَكُنْ لَهُمَا مَا يُوفِيَانِ سَامَحَهُمَا جَمِيعاً. فَقُلْ: أَيُّهُمَا يَكُونُ أَكْثَرَ حُبّاً لَهُ؟»
\par 43 فَأَجَابَ سِمْعَانُ: «أَظُنُّ الَّذِي سَامَحَهُ بِالأَكْثَرِ». فَقَالَ لَهُ: «بِالصَّوَابِ حَكَمْتَ».
\par 44 ثُمَّ الْتَفَتَ إِلَى الْمَرْأَةِ وَقَالَ لِسِمْعَانَ: «أَتَنْظُرُ هَذِهِ الْمَرْأَةَ؟ إِنِّي دَخَلْتُ بَيْتَكَ وَمَاءً لأَجْلِ رِجْلَيَّ لَمْ تُعْطِ. وَأَمَّا هِيَ فَقَدْ غَسَلَتْ رِجْلَيَّ بِالدُّمُوعِ وَمَسَحَتْهُمَا بِشَعْرِ رَأْسِهَا.
\par 45 قُبْلَةً لَمْ تُقَبِّلْنِي وَأَمَّا هِيَ فَمُنْذُ دَخَلْتُ لَمْ تَكُفَّ عَنْ تَقْبِيلِ رِجْلَيَّ.
\par 46 بِزَيْتٍ لَمْ تَدْهُنْ رَأْسِي وَأَمَّا هِيَ فَقَدْ دَهَنَتْ بِالطِّيبِ رِجْلَيَّ.
\par 47 مِنْ أَجْلِ ذَلِكَ أَقُولُ لَكَ: قَدْ غُفِرَتْ خَطَايَاهَا الْكَثِيرَةُ لأَنَّهَا أَحَبَّتْ كَثِيراً. وَالَّذِي يُغْفَرُ لَهُ قَلِيلٌ يُحِبُّ قَلِيلاً».
\par 48 ثُمَّ قَالَ لَهَا: «مَغْفُورَةٌ لَكِ خَطَايَاكِ».
\par 49 فَابْتَدَأَ الْمُتَّكِئُونَ مَعَهُ يَقُولُونَ فِي أَنْفُسِهِمْ: «مَنْ هَذَا الَّذِي يَغْفِرُ خَطَايَا أَيْضاً؟».
\par 50 فَقَالَ لِلْمَرْأَةِ: «إِيمَانُكِ قَدْ خَلَّصَكِ! اِذْهَبِي بِسَلاَمٍ».

\chapter{8}

\par 1 وَعَلَى أَثَرِ ذَلِكَ كَانَ يَسِيرُ فِي مَدِينَةٍ وَقَرْيَةٍ يَكْرِزُ وَيُبَشِّرُ بِمَلَكُوتِ اللهِ وَمَعَهُ الاِثْنَا عَشَرَ.
\par 2 وَبَعْضُ النِّسَاءِ كُنَّ قَدْ شُفِينَ مِنْ أَرْوَاحٍ شِرِّيرَةٍ وَأَمْرَاضٍ: مَرْيَمُ الَّتِي تُدْعَى الْمَجْدَلِيَّةَ الَّتِي خَرَجَ مِنْهَا سَبْعَةُ شَيَاطِينَ
\par 3 وَيُوَنَّا امْرَأَةُ خُوزِي وَكِيلِ هِيرُودُسَ وَسُوسَنَّةُ وَأُخَرُ كَثِيرَاتٌ كُنَّ يَخْدِمْنَهُ مِنْ أَمْوَالِهِنَّ.
\par 4 فَلَمَّا اجْتَمَعَ جَمْعٌ كَثِيرٌ أَيْضاً مِنَ الَّذِينَ جَاءُوا إِلَيْهِ مِنْ كُلِّ مَدِينَةٍ قَالَ بِمَثَلٍ:
\par 5 «خَرَجَ الزَّارِعُ لِيَزْرَعَ زَرْعَهُ. وَفِيمَا هُوَ يَزْرَعُ سَقَطَ بَعْضٌ عَلَى الطَّرِيقِ فَانْدَاسَ وَأَكَلَتْهُ طُيُورُ السَّمَاءِ.
\par 6 وَسَقَطَ آخَرُ عَلَى الصَّخْرِ فَلَمَّا نَبَتَ جَفَّ لأَنَّهُ لَمْ تَكُنْ لَهُ رُطُوبَةٌ.
\par 7 وَسَقَطَ آخَرُ فِي وَسَطِ الشَّوْكِ فَنَبَتَ مَعَهُ الشَّوْكُ وَخَنَقَهُ.
\par 8 وَسَقَطَ آخَرُ فِي الأَرْضِ الصَّالِحَةِ فَلَمَّا نَبَتَ صَنَعَ ثَمَراً مِئَةَ ضِعْفٍ». قَالَ هَذَا وَنَادَى: «مَنْ لَهُ أُذْنَانِ لِلسَّمْعِ فَلْيَسْمَعْ!».
\par 9 فَسَأَلَهُ تَلاَمِيذُهُ: «مَا عَسَى أَنْ يَكُونَ هَذَا الْمَثَلُ؟».
\par 10 فَقَالَ: «لَكُمْ قَدْ أُعْطِيَ أَنْ تَعْرِفُوا أَسْرَارَ مَلَكُوتِ اللهِ وَأَمَّا لِلْبَاقِينَ فَبِأَمْثَالٍ حَتَّى إِنَّهُمْ مُبْصِرِينَ لاَ يُبْصِرُونَ وَسَامِعِينَ لاَ يَفْهَمُونَ.
\par 11 وَهَذَا هُوَ الْمَثَلُ: الزَّرْعُ هُوَ كَلاَمُ اللهِ
\par 12 وَالَّذِينَ عَلَى الطَّرِيقِ هُمُ الَّذِينَ يَسْمَعُونَ ثُمَّ يَأْتِي إِبْلِيسُ وَيَنْزِعُ الْكَلِمَةَ مِنْ قُلُوبِهِمْ لِئَلاَّ يُؤْمِنُوا فَيَخْلُصُوا.
\par 13 وَالَّذِينَ عَلَى الصَّخْرِ هُمُ الَّذِينَ مَتَى سَمِعُوا يَقْبَلُونَ الْكَلِمَةَ بِفَرَحٍ. وَهَؤُلاَءِ لَيْسَ لَهُمْ أَصْلٌ فَيُؤْمِنُونَ إِلَى حِينٍ وَفِي وَقْتِ التَّجْرِبَةِ يَرْتَدُّونَ.
\par 14 وَالَّذِي سَقَطَ بَيْنَ الشَّوْكِ هُمُ الَّذِينَ يَسْمَعُونَ ثُمَّ يَذْهَبُونَ فَيَخْتَنِقُونَ مِنْ هُمُومِ الْحَيَاةِ وَغِنَاهَا وَلَذَّاتِهَا وَلاَ يُنْضِجُونَ ثَمَراً.
\par 15 وَالَّذِي فِي الأَرْضِ الْجَيِّدَةِ هُوَ الَّذِينَ يَسْمَعُونَ الْكَلِمَةَ فَيَحْفَظُونَهَا فِي قَلْبٍ جَيِّدٍ صَالِحٍ وَيُثْمِرُونَ بِالصَّبْرِ.
\par 16 «وَلَيْسَ أَحَدٌ يُوقِدُ سِرَاجاً وَيُغَطِّيهِ بِإِنَاءٍ أَوْ يَضَعُهُ تَحْتَ سَرِيرٍ بَلْ يَضَعُهُ عَلَى مَنَارَةٍ لِيَنْظُرَ الدَّاخِلُونَ النُّورَ.
\par 17 لأَنَّهُ لَيْسَ خَفِيٌّ لاَ يُظْهَرُ وَلاَ مَكْتُومٌ لاَ يُعْلَمُ وَيُعْلَنُ.
\par 18 فَانْظُرُوا كَيْفَ تَسْمَعُونَ! لأَنَّ مَنْ لَهُ سَيُعْطَى وَمَنْ لَيْسَ لَهُ فَالَّذِي يَظُنُّهُ لَهُ يُؤْخَذُ مِنْهُ».
\par 19 وَجَاءَ إِلَيْهِ أُمُّهُ وَإِخْوَتُهُ وَلَمْ يَقْدِرُوا أَنْ يَصِلُوا إِلَيْهِ لِسَبَبِ الْجَمْعِ.
\par 20 فَأَخْبَرُوهُ: «أُمُّكَ وَإِخْوَتُكَ وَاقِفُونَ خَارِجاً يُرِيدُونَ أَنْ يَرَوْكَ».
\par 21 فَأَجَابَ: «أُمِّي وَإِخْوَتِي هُمُ الَّذِينَ يَسْمَعُونَ كَلِمَةَ اللهِ وَيَعْمَلُونَ بِهَا».
\par 22 وَفِي أَحَدِ الأَيَّامِ دَخَلَ سَفِينَةً هُوَ وَتَلاَمِيذُهُ فَقَالَ لَهُمْ: «لِنَعْبُرْ إِلَى عَبْرِ الْبُحَيْرَةِ». فَأَقْلَعُوا.
\par 23 وَفِيمَا هُمْ سَائِرُونَ نَامَ. فَنَزَلَ نَوْءُ رِيحٍ فِي الْبُحَيْرَةِ وَكَانُوا يَمْتَلِئُونَ مَاءً وَصَارُوا فِي خَطَرٍ.
\par 24 فَتَقَدَّمُوا وَأَيْقَظُوهُ قَائِلِينَ: «يَا مُعَلِّمُ يَا مُعَلِّمُ إِنَّنَا نَهْلِكُ!». فَقَامَ وَانْتَهَرَ الرِّيحَ وَتَمَوُّجَ الْمَاءِ فَانْتَهَيَا وَصَارَ هُدُوءٌ.
\par 25 ثُمَّ قَالَ لَهُمْ: «أَيْنَ إِيمَانُكُمْ؟» فَخَافُوا وَتَعَجَّبُوا قَائِلِينَ فِيمَا بَيْنَهُمْ: «مَنْ هُوَ هَذَا؟ فَإِنَّهُ يَأْمُرُ الرِّيَاحَ أَيْضاً وَالْمَاءَ فَتُطِيعُهُ!».
\par 26 وَسَارُوا إِلَى كُورَةِ الْجَدَرِيِّينَ الَّتِي هِيَ مُقَابِلَ الْجَلِيلِ.
\par 27 وَلَمَّا خَرَجَ إِلَى الأَرْضِ اسْتَقْبَلَهُ رَجُلٌ مِنَ الْمَدِينَةِ كَانَ فِيهِ شَيَاطِينُ مُنْذُ زَمَانٍ طَوِيلٍ وَكَانَ لاَ يَلْبَسُ ثَوْباً وَلاَ يُقِيمُ فِي بَيْتٍ بَلْ فِي الْقُبُورِ.
\par 28 فَلَمَّا رَأَى يَسُوعَ صَرَخَ وَخَرَّ لَهُ وَقَالَ بِصَوْتٍ عَظِيمٍ: «مَا لِي وَلَكَ يَا يَسُوعُ ابْنَ اللهِ الْعَلِيِّ! أَطْلُبُ مِنْكَ أَنْ لاَ تُعَذِّبَنِي».
\par 29 لأَنَّهُ أَمَرَ الرُّوحَ النَّجِسَ أَنْ يَخْرُجَ مِنَ الإِنْسَانِ. لأَنَّهُ مُنْذُ زَمَانٍ كَثِيرٍ كَانَ يَخْطَفُهُ وَقَدْ رُبِطَ بِسَلاَسِلٍ وَقُيُودٍ مَحْرُوساً وَكَانَ يَقْطَعُ الرُّبُطَ وَيُسَاقُ مِنَ الشَّيْطَانِ إِلَى الْبَرَارِي.
\par 30 فَسَأَلَهُ يَسُوعُ: «مَا اسْمُكَ؟» فَقَالَ: «لَجِئُونُ». لأَنَّ شَيَاطِينَ كَثِيرَةً دَخَلَتْ فِيهِ.
\par 31 وَطَلَبَ إِلَيْهِ أَنْ لاَ يَأْمُرَهُمْ بِالذَّهَابِ إِلَى الْهَاوِيَةِ.
\par 32 وَكَانَ هُنَاكَ قَطِيعُ خَنَازِيرَ كَثِيرَةٍ تَرْعَى فِي الْجَبَلِ فَطَلَبُوا إِلَيْهِ أَنْ يَأْذِنَ لَهُمْ بِالدُّخُولِ فِيهَا فَأَذِنَ لَهُمْ.
\par 33 فَخَرَجَتِ الشَّيَاطِينُ مِنَ الإِنْسَانِ وَدَخَلَتْ فِي الْخَنَازِيرِ فَانْدَفَعَ الْقَطِيعُ مِنْ عَلَى الْجُرْفِ إِلَى الْبُحَيْرَةِ وَاخْتَنَقَ.
\par 34 فَلَمَّا رَأَى الرُّعَاةُ مَا كَانَ هَرَبُوا وَذَهَبُوا وَأَخْبَرُوا فِي الْمَدِينَةِ وَفِي الضِّيَاعِ
\par 35 فَخَرَجُوا لِيَرَوْا مَا جَرَى. وَجَاءُوا إِلَى يَسُوعَ فَوَجَدُوا الإِنْسَانَ الَّذِي كَانَتِ الشَّيَاطِينُ قَدْ خَرَجَتْ مِنْهُ لاَبِساً وَعَاقِلاً جَالِساً عِنْدَ قَدَمَيْ يَسُوعَ فَخَافُوا.
\par 36 فَأَخْبَرَهُمْ أَيْضاً الَّذِينَ رَأَوْا كَيْفَ خَلَصَ الْمَجْنُونُ.
\par 37 فَطَلَبَ إِلَيْهِ كُلُّ جُمْهُورِ كُورَةِ الْجَدَرِيِّينَ أَنْ يَذْهَبَ عَنْهُمْ لأَنَّهُ اعْتَرَاهُمْ خَوْفٌ عَظِيمٌ. فَدَخَلَ السَّفِينَةَ وَرَجَعَ.
\par 38 أَمَّا الرَّجُلُ الَّذِي خَرَجَتْ مِنْهُ الشَّيَاطِينُ فَطَلَبَ إِلَيْهِ أَنْ يَكُونَ مَعَهُ وَلَكِنَّ يَسُوعَ صَرَفَهُ قَائِلاً:
\par 39 «ارْجِعْ إِلَى بَيْتِكَ وَحَدِّثْ بِكَمْ صَنَعَ اللهُ بِكَ». فَمَضَى وَهُوَ يُنَادِي فِي الْمَدِينَةِ كُلِّهَا بِكَمْ صَنَعَ بِهِ يَسُوعُ.
\par 40 وَلَمَّا رَجَعَ يَسُوعُ قَبِلَهُ الْجَمْعُ لأَنَّهُمْ كَانُوا جَمِيعُهُمْ يَنْتَظِرُونَهُ.
\par 41 وَإِذَا رَجُلٌ اسْمُهُ يَايِرُسُ قَدْ جَاءَ - وَكَانَ رَئِيسَ الْمَجْمَعِ - فَوَقَعَ عِنْدَ قَدَمَيْ يَسُوعَ وَطَلَبَ إِلَيْهِ أَنْ يَدْخُلَ بَيْتَهُ
\par 42 لأَنَّهُ كَانَ لَهُ بِنْتٌ وَحِيدَةٌ لَهَا نَحْوُ اثْنَتَيْ عَشْرَةَ سَنَةً وَكَانَتْ فِي حَالِ الْمَوْتِ. فَفِيمَا هُوَ مُنْطَلِقٌ زَحَمَتْهُ الْجُمُوعُ.
\par 43 وَامْرَأَةٌ بِنَزْفِ دَمٍ مُنْذُ اثْنَتَيْ عَشْرَةَ سَنَةً وَقَدْ أَنْفَقَتْ كُلَّ مَعِيشَتِهَا لِلأَطِبَّاءِ وَلَمْ تَقْدِرْ أَنْ تُشْفَى مِنْ أَحَدٍ
\par 44 جَاءَتْ مِنْ وَرَائِهِ وَلَمَسَتْ هُدْبَ ثَوْبِهِ. فَفِي الْحَالِ وَقَفَ نَزْفُ دَمِهَا.
\par 45 فَقَالَ يَسُوعُ: «مَنِ الَّذِي لَمَسَنِي!» وَإِذْ كَانَ الْجَمِيعُ يُنْكِرُونَ قَالَ بُطْرُسُ وَالَّذِينَ مَعَهُ: «يَا مُعَلِّمُ الْجُمُوعُ يُضَيِّقُونَ عَلَيْكَ وَيَزْحَمُونَكَ وَتَقُولُ مَنِ الَّذِي لَمَسَنِي!»
\par 46 فَقَالَ يَسُوعُ: «قَدْ لَمَسَنِي وَاحِدٌ لأَنِّي عَلِمْتُ أَنَّ قُوَّةً قَدْ خَرَجَتْ مِنِّي».
\par 47 فَلَمَّا رَأَتِ الْمَرْأَةُ أَنَّهَا لَمْ تَخْتَفِ جَاءَتْ مُرْتَعِدَةً وَخَرَّتْ لَهُ وَأَخْبَرَتْهُ قُدَّامَ جَمِيعِ الشَّعْبِ لأَيِّ سَبَبٍ لَمَسَتْهُ وَكَيْفَ بَرِئَتْ فِي الْحَالِ.
\par 48 فَقَالَ لَهَا: «ثِقِي يَا ابْنَةُ. إِيمَانُكِ قَدْ شَفَاكِ. اِذْهَبِي بِسَلاَمٍ».
\par 49 وَبَيْنَمَا هُوَ يَتَكَلَّمُ جَاءَ وَاحِدٌ مِنْ دَارِ رَئِيسِ الْمَجْمَعِ قَائِلاً لَهُ: «قَدْ مَاتَتِ ابْنَتُكَ. لاَ تُتْعِبِ الْمُعَلِّمَ».
\par 50 فَسَمِعَ يَسُوعُ وَأَجَابَهُ: «لاَ تَخَفْ. آمِنْ فَقَطْ فَهِيَ تُشْفَى».
\par 51 فَلَمَّا جَاءَ إِلَى الْبَيْتِ لَمْ يَدَعْ أَحَداً يَدْخُلُ إِلاَّ بُطْرُسَ وَيَعْقُوبَ وَيُوحَنَّا وَأَبَا الصَّبِيَّةِ وَأُمَّهَا.
\par 52 وَكَانَ الْجَمِيعُ يَبْكُونَ عَلَيْهَا وَيَلْطِمُونَ. فَقَالَ: «لاَ تَبْكُوا. لَمْ تَمُتْ لَكِنَّهَا نَائِمَةٌ».
\par 53 فَضَحِكُوا عَلَيْهِ عَارِفِينَ أَنَّهَا مَاتَتْ.
\par 54 فَأَخْرَجَ الْجَمِيعَ خَارِجاً وَأَمْسَكَ بِيَدِهَا وَنَادَى قَائِلاً: «يَا صَبِيَّةُ قُومِي».
\par 55 فَرَجَعَتْ رُوحُهَا وَقَامَتْ فِي الْحَالِ. فَأَمَرَ أَنْ تُعْطَى لِتَأْكُلَ.
\par 56 فَبُهِتَ وَالِدَاهَا. فَأَوْصَاهُمَا أَنْ لاَ يَقُولاَ لأَحَدٍ عَمَّا كَانَ.

\chapter{9}

\par 1 وَدَعَا تَلاَمِيذَهُ الاِثْنَيْ عَشَرَ وَأَعْطَاهُمْ قُوَّةً وَسُلْطَاناً عَلَى جَمِيعِ الشَّيَاطِينِ وَشِفَاءِ أَمْرَاضٍ
\par 2 وَأَرْسَلَهُمْ لِيَكْرِزُوا بِمَلَكُوتِ اللهِ وَيَشْفُوا الْمَرْضَى.
\par 3 وَقَالَ لَهُمْ: «لاَ تَحْمِلُوا شَيْئاً لِلطَّرِيقِ لاَ عَصاً وَلاَ مِزْوَداً وَلاَ خُبْزاً وَلاَ فِضَّةً وَلاَ يَكُونُ لِلْوَاحِدِ ثَوْبَانِ.
\par 4 وَأَيَُّ بَيْتٍ دَخَلْتُمُوهُ فَهُنَاكَ أَقِيمُوا وَمِنْ هُنَاكَ اخْرُجُوا.
\par 5 وَكُلُّ مَنْ لاَ يَقْبَلُكُمْ فَاخْرُجُوا مِنْ تِلْكَ الْمَدِينَةِ وَانْفُضُوا الْغُبَارَ أَيْضاً عَنْ أَرْجُلِكُمْ شَهَادَةً عَلَيْهِمْ».
\par 6 فَلَمَّا خَرَجُوا كَانُوا يَجْتَازُونَ فِي كُلِّ قَرْيَةٍ يُبَشِّرُونَ وَيَشْفُونَ فِي كُلِّ مَوْضِعٍ.
\par 7 فَسَمِعَ هِيرُودُسُ رَئِيسُ الرُّبْعِ بِجَمِيعِ مَا كَانَ مِنْهُ وَارْتَابَ لأَنَّ قَوْماً كَانُوا يَقُولُونَ: «إِنَّ يُوحَنَّا قَدْ قَامَ مِنَ الأَمْوَاتِ».
\par 8 وَقَوْماً: «إِنَّ إِيلِيَّا ظَهَرَ». وَآخَرِينَ: «إِنَّ نَبِيّاً مِنَ الْقُدَمَاءِ قَامَ».
\par 9 فَقَالَ هِيرُودُسُ: «يُوحَنَّا أَنَا قَطَعْتُ رَأْسَهُ. فَمَنْ هُوَ هَذَا الَّذِي أَسْمَعُ عَنْهُ مِثْلَ هَذَا!» وَكَانَ يَطْلُبُ أَنْ يَرَاهُ.
\par 10 وَلَمَّا رَجَعَ الرُّسُلُ أَخْبَرُوهُ بِجَمِيعِ مَا فَعَلُوا فَأَخَذَهُمْ وَانْصَرَفَ مُنْفَرِداً إِلَى مَوْضِعٍ خَلاَءٍ لِمَدِينَةٍ تُسَمَّى بَيْتَ صَيْدَا.
\par 11 فَالْجُمُوعُ إِذْ عَلِمُوا تَبِعُوهُ فَقَبِلَهُمْ وَكَلَّمَهُمْ عَنْ مَلَكُوتِ اللهِ وَالْمُحْتَاجُونَ إِلَى الشِّفَاءِ شَفَاهُمْ.
\par 12 فَابْتَدَأَ النَّهَارُ يَمِيلُ. فَتَقَدَّمَ الاِثْنَا عَشَرَ وَقَالُوا لَهُ: «اصْرِفِ الْجَمْعَ لِيَذْهَبُوا إِلَى الْقُرَى وَالضِّيَاعِ حَوَالَيْنَا فَيَبِيتُوا وَيَجِدُوا طَعَاماً لأَنَّنَا هَهُنَا فِي مَوْضِعٍ خَلاَءٍ».
\par 13 فَقَالَ لَهُمْ: «أَعْطُوهُمْ أَنْتُمْ لِيَأْكُلُوا». فَقَالُوا: «لَيْسَ عِنْدَنَا أَكْثَرُ مِنْ خَمْسَةِ أَرْغِفَةٍ وَسَمَكَتَيْنِ إِلاَّ أَنْ نَذْهَبَ وَنَبْتَاعَ طَعَاماً لِهَذَا الشَّعْبِ كُلِّهِ».
\par 14 لأَنَّهُمْ كَانُوا نَحْوَ خَمْسَةِ آلاَفِ رَجُلٍ. فَقَالَ لِتَلاَمِيذِهِ: «أَتْكِئُوهُمْ فِرَقاً خَمْسِينَ خَمْسِينَ».
\par 15 فَفَعَلُوا هَكَذَا وَأَتْكَأُوا الْجَمِيعَ.
\par 16 فَأَخَذَ الأَرْغِفَةَ الْخَمْسَةَ وَالسَّمَكَتَيْنِ وَرَفَعَ نَظَرَهُ نَحْوَ السَّمَاءِ وَبَارَكَهُنَّ ثُمَّ كَسَّرَ وَأَعْطَى التَّلاَمِيذَ لِيُقَدِّمُوا لِلْجَمْعِ.
\par 17 فَأَكَلُوا وَشَبِعُوا جَمِيعاً. ثُمَّ رُفِعَ مَا فَضَلَ عَنْهُمْ مِنَ الْكِسَرِ: اثْنَتَا عَشْرَةَ قُفَّةً.
\par 18 وَفِيمَا هُوَ يُصَلِّي عَلَى انْفِرَادٍ كَانَ التَّلاَمِيذُ مَعَهُ. فَسَأَلَهُمْ: «مَنْ تَقُولُ الْجُمُوعُ إِنِّي أَنَا؟»
\par 19 فَأَجَابُوا: «يُوحَنَّا الْمَعْمَدَانُ. وَآخَرُونَ إِيلِيَّا. وَآخَرُونَ إِنَّ نَبِيّاً مِنَ الْقُدَمَاءِ قَامَ».
\par 20 فَقَالَ لَهُمْ: «وَأَنْتُمْ مَنْ تَقُولُونَ إِنِّي أَنَا؟» فَأَجَابَ بُطْرُسُ: «مَسِيحُ اللهِ».
\par 21 فَانْتَهَرَهُمْ وَأَوْصَى أَنْ لاَ يَقُولُوا ذَلِكَ لأَحَدٍ
\par 22 قَائِلاً: «إِنَّهُ يَنْبَغِي أَنَّ ابْنَ الإِنْسَانِ يَتَأَلَّمُ كَثِيراً وَيُرْفَضُ مِنَ الشُّيُوخِ وَرُؤَسَاءِ الْكَهَنَةِ وَالْكَتَبَةِ وَيُقْتَلُ وَفِي الْيَوْمِ الثَّالِثِ يَقُومُ».
\par 23 وَقَالَ لِلْجَمِيعِ: «إِنْ أَرَادَ أَحَدٌ أَنْ يَأْتِيَ وَرَائِي فَلْيُنْكِرْ نَفْسَهُ وَيَحْمِلْ صَلِيبَهُ كُلَّ يَوْمٍ وَيَتْبَعْنِي.
\par 24 فَإِنَّ مَنْ أَرَادَ أَنْ يُخَلِّصَ نَفْسَهُ يُهْلِكُهَا وَمَنْ يُهْلِكُ نَفْسَهُ مِنْ أَجْلِي فَهَذَا يُخَلِّصُهَا.
\par 25 لأَنَّهُ مَاذَا يَنْتَفِعُ الإِنْسَانُ لَوْ رَبِحَ الْعَالَمَ كُلَّهُ وَأَهْلَكَ نَفْسَهُ أَوْ خَسِرَهَا؟
\par 26 لأَنَّ مَنِ اسْتَحَى بِي وَبِكَلاَمِي فَبِهَذَا يَسْتَحِي ابْنُ الإِنْسَانِ مَتَى جَاءَ بِمَجْدِهِ وَمَجْدِ الآبِ وَالْمَلاَئِكَةِ الْقِدِّيسِينَ.
\par 27 حَقّاً أَقُولُ لَكُمْ: إِنَّ مِنَ الْقِيَامِ هَهُنَا قَوْماً لاَ يَذُوقُونَ الْمَوْتَ حَتَّى يَرَوْا مَلَكُوتَ اللهِ».
\par 28 وَبَعْدَ هَذَا الْكَلاَمِ بِنَحْوِ ثَمَانِيَةِ أَيَّامٍ أَخَذَ بُطْرُسَ وَيُوحَنَّا وَيَعْقُوبَ وَصَعِدَ إِلَى جَبَلٍ لِيُصَلِّيَ.
\par 29 وَفِيمَا هُوَ يُصَلِّي صَارَتْ هَيْئَةُ وَجْهِهِ مُتَغَيِّرَةً وَلِبَاسُهُ مُبْيَضّاً لاَمِعاً.
\par 30 وَإِذَا رَجُلاَنِ يَتَكَلَّمَانِ مَعَهُ وَهُمَا مُوسَى وَإِيلِيَّا
\par 31 اَللَّذَانِ ظَهَرَا بِمَجْدٍ وَتَكَلَّمَا عَنْ خُرُوجِهِ الَّذِي كَانَ عَتِيداً أَنْ يُكَمِّلَهُ فِي أُورُشَلِيمَ.
\par 32 وَأَمَّا بُطْرُسُ وَاللَّذَانِ مَعَهُ فَكَانُوا قَدْ تَثَقَّلُوا بِالنَّوْمِ. فَلَمَّا اسْتَيْقَظُوا رَأَوْا مَجْدَهُ وَالرَّجُلَيْنِ الْوَاقِفَيْنِ مَعَهُ.
\par 33 وَفِيمَا هُمَا يُفَارِقَانِهِ قَالَ بُطْرُسُ لِيَسُوعَ: «يَا مُعَلِّمُ جَيِّدٌ أَنْ نَكُونَ هَهُنَا. فَلْنَصْنَعْ ثَلاَثَ مَظَالَّ: لَكَ وَاحِدَةً وَلِمُوسَى وَاحِدَةً وَلِإِيلِيَّا وَاحِدَةً». وَهُوَ لاَ يَعْلَمُ مَا يَقُولُ.
\par 34 وَفِيمَا هُوَ يَقُولُ ذَلِكَ كَانَتْ سَحَابَةٌ فَظَلَّلَتْهُمْ. فَخَافُوا عِنْدَمَا دَخَلُوا فِي السَّحَابَةِ.
\par 35 وَصَارَ صَوْتٌ مِنَ السَّحَابَةِ قَائِلاً: «هَذَا هُوَ ابْنِي الْحَبِيبُ. لَهُ اسْمَعُوا».
\par 36 وَلَمَّا كَانَ الصَّوْتُ وُجِدَ يَسُوعُ وَحْدَهُ وَأَمَّا هُمْ فَسَكَتُوا وَلَمْ يُخْبِرُوا أَحَداً فِي تِلْكَ الأَيَّامِ بِشَيْءٍ مِمَّا أَبْصَرُوهُ.
\par 37 وَفِي الْيَوْمِ التَّالِي إِذْ نَزَلُوا مِنَ الْجَبَلِ اسْتَقْبَلَهُ جَمْعٌ كَثِيرٌ.
\par 38 وَإِذَا رَجُلٌ مِنَ الْجَمْعِ صَرَخَ: «يَا مُعَلِّمُ أَطْلُبُ إِلَيْكَ. اُنْظُرْ إِلَى ابْنِي فَإِنَّهُ وَحِيدٌ لِي.
\par 39 وَهَا رُوحٌ يَأْخُذُهُ فَيَصْرُخُ بَغْتَةً فَيَصْرَعُهُ مُزْبِداً وَبِالْجَهْدِ يُفَارِقُهُ مُرَضِّضاً إِيَّاهُ.
\par 40 وَطَلَبْتُ مِنْ تَلاَمِيذِكَ أَنْ يُخْرِجُوهُ فَلَمْ يَقْدِرُوا».
\par 41 فَأَجَابَ يَسُوعُ: «أَيُّهَا الْجِيلُ غَيْرُ الْمُؤْمِنِ وَالْمُلْتَوِي إِلَى مَتَى أَكُونُ مَعَكُمْ وَأَحْتَمِلُكُمْ؟ قَدِّمِ ابْنَكَ إِلَى هُنَا».
\par 42 وَبَيْنَمَا هُوَ آتٍ مَزَّقَهُ الشَّيْطَانُ وَصَرَعَهُ فَانْتَهَرَ يَسُوعُ الرُّوحَ النَّجِسَ وَشَفَى الصَّبِيَّ وَسَلَّمَهُ إِلَى أَبِيهِ.
\par 43 فَبُهِتَ الْجَمِيعُ مِنْ عَظَمَةِ اللهِ. وَإِذْ كَانَ الْجَمِيعُ يَتَعَجَّبُونَ مِنْ كُلِّ مَا فَعَلَ يَسُوعُ قَالَ لِتَلاَمِيذِهِ:
\par 44 «ضَعُوا أَنْتُمْ هَذَا الْكَلاَمَ فِي آذَانِكُمْ: إِنَّ ابْنَ الإِنْسَانِ سَوْفَ يُسَلَّمُ إِلَى أَيْدِي النَّاسِ».
\par 45 وَأَمَّا هُمْ فَلَمْ يَفْهَمُوا هَذَا الْقَوْلَ وَكَانَ مُخْفىً عَنْهُمْ لِكَيْ لاَ يَفْهَمُوهُ وَخَافُوا أَنْ يَسْأَلُوهُ عَنْ هَذَا الْقَوْلِ.
\par 46 وَدَاخَلَهُمْ فِكْرٌ: مَنْ عَسَى أَنْ يَكُونَ أَعْظَمَ فِيهِمْ؟
\par 47 فَعَلِمَ يَسُوعُ فِكْرَ قَلْبِهِمْ وَأَخَذَ وَلَداً وَأَقَامَهُ عِنْدَهُ
\par 48 وَقَالَ لَهُمْ: «مَنْ قَبِلَ هَذَا الْوَلَدَ بِاسْمِي يَقْبَلُنِي وَمَنْ قَبِلَنِي يَقْبَلُ الَّذِي أَرْسَلَنِي لأَنَّ الأَصْغَرَ فِيكُمْ جَمِيعاً هُوَ يَكُونُ عَظِيماً»
\par 49 فَقَالَ يُوحَنَّا: «يَا مُعَلِّمُ رَأَيْنَا وَاحِداً يُخْرِجُ الشَّيَاطِينَ بِاسْمِكَ فَمَنَعْنَاهُ لأَنَّهُ لَيْسَ يَتْبَعُ مَعَنَا».
\par 50 فَقَالَ لَهُ يَسُوعُ: «لاَ تَمْنَعُوهُ لأَنَّ مَنْ لَيْسَ عَلَيْنَا فَهُوَ مَعَنَا».
\par 51 وَحِينَ تَمَّتِ الأَيَّامُ لاِرْتِفَاعِهِ ثَبَّتَ وَجْهَهُ لِيَنْطَلِقَ إِلَى أُورُشَلِيمَ
\par 52 وَأَرْسَلَ أَمَامَ وَجْهِهِ رُسُلاً فَذَهَبُوا وَدَخَلُوا قَرْيَةً لِلسَّامِرِيِّينَ حَتَّى يُعِدُّوا لَهُ.
\par 53 فَلَمْ يَقْبَلُوهُ لأَنَّ وَجْهَهُ كَانَ مُتَّجِهاً نَحْوَ أُورُشَلِيمَ.
\par 54 فَلَمَّا رَأَى ذَلِكَ تِلْمِيذَاهُ يَعْقُوبُ وَيُوحَنَّا قَالاَ: «يَا رَبُّ أَتُرِيدُ أَنْ نَقُولَ أَنْ تَنْزِلَ نَارٌ مِنَ السَّمَاءِ فَتُفْنِيَهُمْ كَمَا فَعَلَ إِيلِيَّا أَيْضاً؟»
\par 55 فَالْتَفَتَ وَانْتَهَرَهُمَا وَقَالَ: «لَسْتُمَا تَعْلَمَانِ مِنْ أَيِّ رُوحٍ أَنْتُمَا!
\par 56 لأَنَّ ابْنَ الإِنْسَانِ لَمْ يَأْتِ لِيُهْلِكَ أَنْفُسَ النَّاسِ بَلْ لِيُخَلِّصَ». فَمَضَوْا إِلَى قَرْيَةٍ أُخْرَى.
\par 57 وَفِيمَا هُمْ سَائِرُونَ فِي الطَّرِيقِ قَالَ لَهُ وَاحِدٌ: «يَا سَيِّدُ أَتْبَعُكَ أَيْنَمَا تَمْضِي».
\par 58 فَقَالَ لَهُ يَسُوعُ: «لِلثَّعَالِبِ أَوْجِرَةٌ وَلِطُيُورِ السَّمَاءِ أَوْكَارٌ وَأَمَّا ابْنُ الإِنْسَانِ فَلَيْسَ لَهُ أَيْنَ يُسْنِدُ رَأْسَهُ».
\par 59 وَقَالَ لِآخَرَ: «اتْبَعْنِي». فَقَالَ: «يَا سَيِّدُ ائْذَنْ لِي أَنْ أَمْضِيَ أَوَّلاً وَأَدْفِنَ أَبِي».
\par 60 فَقَالَ لَهُ يَسُوعُ: «دَعِ الْمَوْتَى يَدْفِنُونَ مَوْتَاهُمْ وَأَمَّا أَنْتَ فَاذْهَبْ وَنَادِ بِمَلَكُوتِ اللهِ».
\par 61 وَقَالَ آخَرُ أَيْضاً: «أَتْبَعُكَ يَا سَيِّدُ وَلَكِنِ ائْذِنْ لِي أَوَّلاً أَنْ أُوَدِّعَ الَّذِينَ فِي بَيْتِي».
\par 62 فَقَالَ لَهُ يَسُوعُ: «لَيْسَ أَحَدٌ يَضَعُ يَدَهُ عَلَى الْمِحْرَاثِ وَيَنْظُرُ إِلَى الْوَرَاءِ يَصْلُحُ لِمَلَكُوتِ اللهِ».

\chapter{10}

\par 1 وَبَعْدَ ذَلِكَ عَيَّنَ الرَّبُّ سَبْعِينَ آخَرِينَ أَيْضاً وَأَرْسَلَهُمُ اثْنَيْنِ اثْنَيْنِ أَمَامَ وَجْهِهِ إِلَى كُلِّ مَدِينَةٍ وَمَوْضِعٍ حَيْثُ كَانَ هُوَ مُزْمِعاً أَنْ يَأْتِيَ.
\par 2 فَقَالَ لَهُمْ: «إِنَّ الْحَصَادَ كَثِيرٌ وَلَكِنَّ الْفَعَلَةَ قَلِيلُونَ. فَاطْلُبُوا مِنْ رَبِّ الْحَصَادِ أَنْ يُرْسِلَ فَعَلَةً إِلَى حَصَادِهِ.
\par 3 اِذْهَبُوا. هَا أَنَا أُرْسِلُكُمْ مِثْلَ حُمْلاَنٍ بَيْنَ ذِئَابٍ.
\par 4 لاَ تَحْمِلُوا كِيساً وَلاَ مِزْوَداً وَلاَ أَحْذِيَةً وَلاَ تُسَلِّمُوا عَلَى أَحَدٍ فِي الطَّرِيقِ.
\par 5 وَأَيُّ بَيْتٍ دَخَلْتُمُوهُ فَقُولُوا أَوَّلاً: سَلاَمٌ لِهَذَا الْبَيْتِ.
\par 6 فَإِنْ كَانَ هُنَاكَ ابْنُ السَّلاَمِ يَحِلُّ سَلاَمُكُمْ عَلَيْهِ وَإِلاَّ فَيَرْجِعُ إِلَيْكُمْ.
\par 7 وَأَقِيمُوا فِي ذَلِكَ الْبَيْتِ آكِلِينَ وَشَارِبِينَ مِمَّا عِنْدَهُمْ لأَنَّ الْفَاعِلَ مُسْتَحِقٌّ أُجْرَتَهُ. لاَ تَنْتَقِلُوا مِنْ بَيْتٍ إِلَى بَيْتٍ.
\par 8 وَأَيَّةُ مَدِينَةٍ دَخَلْتُمُوهَا وَقَبِلُوكُمْ فَكُلُوا مِمَّا يُقَدَّمُ لَكُمْ
\par 9 وَاشْفُوا الْمَرْضَى الَّذِينَ فِيهَا وَقُولُوا لَهُمْ: قَدِ اقْتَرَبَ مِنْكُمْ مَلَكُوتُ اللهِ
\par 10 وَأَيَّةُ مَدِينَةٍ دَخَلْتُمُوهَا وَلَمْ يَقْبَلُوكُمْ فَاخْرُجُوا إِلَى شَوَارِعِهَا وَقُولُوا:
\par 11 حَتَّى الْغُبَارُ الَّذِي لَصِقَ بِنَا مِنْ مَدِينَتِكُمْ نَنْفُضُهُ لَكُمْ. وَلَكِنِ اعْلَمُوا هَذَا أَنَّهُ قَدِ اقْتَرَبَ مِنْكُمْ مَلَكُوتُ اللهِ.
\par 12 وَأَقُولُ لَكُمْ إِنَّهُ يَكُونُ لِسَدُومَ فِي ذَلِكَ الْيَوْمِ حَالَةٌ أَكْثَرُ احْتِمَالاً مِمَّا لِتِلْكَ الْمَدِينَةِ.
\par 13 «وَيْلٌ لَكِ يَا كُورَزِينُ! وَيْلٌ لَكِ يَا بَيْتَ صَيْدَا! لأَنَّهُ لَوْ صُنِعَتْ فِي صُورَ وَصَيْدَاءَ الْقُوَّاتُ الْمَصْنُوعَةُ فِيكُمَا لَتَابَتَا قَدِيماً جَالِسَتَيْنِ فِي الْمُسُوحِ وَالرَّمَادِ.
\par 14 وَلَكِنَّ صُورَ وَصَيْدَاءَ يَكُونُ لَهُمَا فِي الدِّينِ حَالَةٌ أَكْثَرُ احْتِمَالاً مِمَّا لَكُمَا.
\par 15 وَأَنْتِ يَا كَفْرَنَاحُومُ الْمُرْتَفِعَةُ إِلَى السَّمَاءِ سَتُهْبَطِينَ إِلَى الْهَاوِيَةِ.
\par 16 اَلَّذِي يَسْمَعُ مِنْكُمْ يَسْمَعُ مِنِّي وَالَّذِي يُرْذِلُكُمْ يُرْذِلُنِي وَالَّذِي يُرْذِلُنِي يُرْذِلُ الَّذِي أَرْسَلَنِي».
\par 17 فَرَجَعَ السَّبْعُونَ بِفَرَحٍ قَائِلِينَ: «يَا رَبُّ حَتَّى الشَّيَاطِينُ تَخْضَعُ لَنَا بِاسْمِكَ».
\par 18 فَقَالَ لَهُمْ: «رَأَيْتُ الشَّيْطَانَ سَاقِطاً مِثْلَ الْبَرْقِ مِنَ السَّمَاءِ.
\par 19 هَا أَنَا أُعْطِيكُمْ سُلْطَاناً لِتَدُوسُوا الْحَيَّاتِ وَالْعَقَارِبَ وَكُلَّ قُوَّةِ الْعَدُّوِ وَلاَ يَضُرُّكُمْ شَيْءٌ.
\par 20 وَلَكِنْ لاَ تَفْرَحُوا بِهَذَا أَنَّ الأَرْوَاحَ تَخْضَعُ لَكُمْ بَلِ افْرَحُوا بِالْحَرِيِّ أَنَّ أَسْمَاءَكُمْ كُتِبَتْ فِي السَّمَاوَاتِ».
\par 21 وَفِي تِلْكَ السَّاعَةِ تَهَلَّلَ يَسُوعُ بِالرُّوحِ وَقَالَ: «أَحْمَدُكَ أَيُّهَا الآبُ رَبُّ السَّمَاءِ وَالأَرْضِ لأَنَّكَ أَخْفَيْتَ هَذِهِ عَنِ الْحُكَمَاءِ وَالْفُهَمَاءِ وَأَعْلَنْتَهَا لِلأَطْفَالِ. نَعَمْ أَيُّهَا الآبُ لأَنْ هَكَذَا صَارَتِ الْمَسَرَّةُ أَمَامَكَ».
\par 22 وَالْتَفَتَ إِلَى تَلاَمِيذِهِ وَقَالَ: «كُلُّ شَيْءٍ قَدْ دُفِعَ إِلَيَّ مِنْ أَبِي. وَلَيْسَ أَحَدٌ يَعْرِفُ مَنْ هُوَ الاِبْنُ إِلاَّ الآبُ وَلاَ مَنْ هُوَ الآبُ إِلاَّ الاِبْنُ وَمَنْ أَرَادَ الاِبْنُ أَنْ يُعْلِنَ لَهُ».
\par 23 وَالْتَفَتَ إِلَى تَلاَمِيذِهِ عَلَى انْفِرَادٍ وَقَالَ: «طُوبَى لِلْعُيُونِ الَّتِي تَنْظُرُ مَا تَنْظُرُونَهُ
\par 24 لأَنِّي أَقُولُ لَكُمْ: إِنَّ أَنْبِيَاءَ كَثِيرِينَ وَمُلُوكاً أَرَادُوا أَنْ يَنْظُرُوا مَا أَنْتُمْ تَنْظُرُونَ وَلَمْ يَنْظُرُوا وَأَنْ يَسْمَعُوا مَا أَنْتُمْ تَسْمَعُونَ وَلَمْ يَسْمَعُوا».
\par 25 وَإِذَا نَامُوسِيٌّ قَامَ يُجَرِّبُهُ قَائِلاً: «يَا مُعَلِّمُ مَاذَا أَعْمَلُ لأَرِثَ الْحَيَاةَ الأَبَدِيَّةَ؟»
\par 26 فَقَالَ لَهُ: «مَا هُوَ مَكْتُوبٌ فِي النَّامُوسِ. كَيْفَ تَقْرَأُ؟»
\par 27 فَأَجَابَ: «تُحِبُّ الرَّبَّ إِلَهَكَ مِنْ كُلِّ قَلْبِكَ وَمِنْ كُلِّ نَفْسِكَ وَمِنْ كُلِّ قُدْرَتِكَ وَمِنْ كُلِّ فِكْرِكَ وَقَرِيبَكَ مِثْلَ نَفْسِكَ».
\par 28 فَقَالَ لَهُ: «بِالصَّوَابِ أَجَبْتَ. اِفْعَلْ هَذَا فَتَحْيَا».
\par 29 وَأَمَّا هُوَ فَإِذْ أَرَادَ أَنْ يُبَرِّرَ نَفْسَهُ سَأَلَ يَسُوعَ: «وَمَنْ هُوَ قَرِيبِي؟»
\par 30 فَأَجَابَ يَسُوعُ: «إِنْسَانٌ كَانَ نَازِلاً مِنْ أُورُشَلِيمَ إِلَى أَرِيحَا فَوَقَعَ بَيْنَ لُصُوصٍ فَعَرَّوْهُ وَجَرَّحُوهُ وَمَضَوْا وَتَرَكُوهُ بَيْنَ حَيٍّ وَمَيْتٍ.
\par 31 فَعَرَضَ أَنَّ كَاهِناً نَزَلَ فِي تِلْكَ الطَّرِيقِ فَرَآهُ وَجَازَ مُقَابِلَهُ.
\par 32 وَكَذَلِكَ لاَوِيٌّ أَيْضاً إِذْ صَارَ عِنْدَ الْمَكَانِ جَاءَ وَنَظَرَ وَجَازَ مُقَابِلَهُ.
\par 33 وَلَكِنَّ سَامِرِيّاً مُسَافِراً جَاءَ إِلَيْهِ وَلَمَّا رَآهُ تَحَنَّنَ
\par 34 فَتَقَدَّمَ وَضَمَدَ جِرَاحَاتِهِ وَصَبَّ عَلَيْهَا زَيْتاً وَخَمْراً وَأَرْكَبَهُ عَلَى دَابَّتِهِ وَأَتَى بِهِ إِلَى فُنْدُقٍ وَاعْتَنَى بِهِ.
\par 35 وَفِي الْغَدِ لَمَّا مَضَى أَخْرَجَ دِينَارَيْنِ وَأَعْطَاهُمَا لِصَاحِبِ الْفُنْدُقِ وَقَالَ لَهُ: اعْتَنِ بِهِ وَمَهْمَا أَنْفَقْتَ أَكْثَرَ فَعِنْدَ رُجُوعِي أُوفِيكَ.
\par 36 فَأَيُّ هَؤُلاَءِ الثَّلاَثَةِ تَرَى صَارَ قَرِيباً لِلَّذِي وَقَعَ بَيْنَ اللُّصُوصِ؟»
\par 37 فَقَالَ: «الَّذِي صَنَعَ مَعَهُ الرَّحْمَةَ». فَقَالَ لَهُ يَسُوعُ: «اذْهَبْ أَنْتَ أَيْضاً وَاصْنَعْ هَكَذَا».
\par 38 وَفِيمَا هُمْ سَائِرُونَ دَخَلَ قَرْيَةً فَقَبِلَتْهُ امْرَأَةٌ اسْمُهَا مَرْثَا فِي بَيْتِهَا.
\par 39 وَكَانَتْ لِهَذِهِ أُخْتٌ تُدْعَى مَرْيَمَ الَّتِي جَلَسَتْ عِنْدَ قَدَمَيْ يَسُوعَ وَكَانَتْ تَسْمَعُ كَلاَمَهُ.
\par 40 وَأَمَّا مَرْثَا فَكَانَتْ مُرْتَبِكَةً فِي خِدْمَةٍ كَثِيرَةٍ فَوَقَفَتْ وَقَالَتْ: «يَا رَبُّ أَمَا تُبَالِي بِأَنَّ أُخْتِي قَدْ تَرَكَتْنِي أَخْدِمُ وَحْدِي؟ فَقُلْ لَهَا أَنْ تُعِينَنِي!»
\par 41 فَأَجَابَ يَسُوعُ: «مَرْثَا مَرْثَا أَنْتِ تَهْتَمِّينَ وَتَضْطَرِبِينَ لأَجْلِ أُمُورٍ كَثِيرَةٍ
\par 42 وَلَكِنَّ الْحَاجَةَ إِلَى وَاحِدٍ. فَاخْتَارَتْ مَرْيَمُ النَّصِيبَ الصَّالِحَ الَّذِي لَنْ يُنْزَعَ مِنْهَا».

\chapter{11}

\par 1 وَإِذْ كَانَ يُصَلِّي فِي مَوْضِعٍ لَمَّا فَرَغَ قَالَ وَاحِدٌ مِنْ تَلاَمِيذِهِ: «يَا رَبُّ عَلِّمْنَا أَنْ نُصَلِّيَ كَمَا عَلَّمَ يُوحَنَّا أَيْضاً تَلاَمِيذَهُ».
\par 2 فَقَالَ لَهُمْ: «مَتَى صَلَّيْتُمْ فَقُولُوا: أَبَانَا الَّذِي فِي السَّمَاوَاتِ لِيَتَقَدَّسِ اسْمُكَ لِيَأْتِ مَلَكُوتُكَ لِتَكُنْ مَشِيئَتُكَ كَمَا فِي السَّمَاءِ كَذَلِكَ عَلَى الأَرْضِ.
\par 3 خُبْزَنَا كَفَافَنَا أَعْطِنَا كُلَّ يَوْمٍ
\par 4 وَاغْفِرْ لَنَا خَطَايَانَا لأَنَّنَا نَحْنُ أَيْضاً نَغْفِرُ لِكُلِّ مَنْ يُذْنِبُ إِلَيْنَا وَلاَ تُدْخِلْنَا فِي تَجْرِبَةٍ لَكِنْ نَجِّنَا مِنَ الشِّرِّيرِ».
\par 5 ثُمَّ قَالَ لَهُمْ: «مَنْ مِنْكُمْ يَكُونُ لَهُ صَدِيقٌ وَيَمْضِي إِلَيْهِ نِصْفَ اللَّيْلِ وَيَقُولُ لَهُ: يَا صَدِيقُ أَقْرِضْنِي ثَلاَثَةَ أَرْغِفَةٍ
\par 6 لأَنَّ صَدِيقاً لِي جَاءَنِي مِنْ سَفَرٍ وَلَيْسَ لِي مَا أُقَدِّمُ لَهُ.
\par 7 فَيُجِيبَ ذَلِكَ مِنْ دَاخِلٍ وَيَقُولَ: لاَ تُزْعِجْنِي! اَلْبَابُ مُغْلَقٌ الآنَ وَأَوْلاَدِي مَعِي فِي الْفِرَاشِ. لاَ أَقْدِرُ أَنْ أَقُومَ وَأُعْطِيَكَ.
\par 8 أَقُولُ لَكُمْ: وَإِنْ كَانَ لاَ يَقُومُ وَيُعْطِيهِ لِكَوْنِهِ صَدِيقَهُ فَإِنَّهُ مِنْ أَجْلِ لَجَاجَتِهِ يَقُومُ وَيُعْطِيهِ قَدْرَ مَا يَحْتَاجُ.
\par 9 وَأَنَا أَقُولُ لَكُمُ: اسْأَلُوا تُعْطَوْا. اطْلُبُوا تَجِدُوا. اِقْرَعُوا يُفْتَحْ لَكُمْ.
\par 10 لأَنَّ كُلَّ مَنْ يَسْأَلُ يَأْخُذُ وَمَنْ يَطْلُبُ يَجِدُ وَمَنْ يَقْرَعُ يُفْتَحُ لَهُ.
\par 11 فَمَنْ مِنْكُمْ وَهُوَ أَبٌ يَسْأَلُهُ ابْنُهُ خُبْزاً أَفَيُعْطِيهِ حَجَراً؟ أَوْ سَمَكَةً أَفَيُعْطِيهِ حَيَّةً بَدَلَ السَّمَكَةِ؟
\par 12 أَوْ إِذَا سَأَلَهُ بَيْضَةً أَفَيُعْطِيهِ عَقْرَباً؟
\par 13 فَإِنْ كُنْتُمْ وَأَنْتُمْ أَشْرَارٌ تَعْرِفُونَ أَنْ تُعْطُوا أَوْلاَدَكُمْ عَطَايَا جَيِّدَةً فَكَمْ بِالْحَرِيِّ الآبُ الَّذِي مِنَ السَّمَاءِ يُعْطِي الرُّوحَ الْقُدُسَ لِلَّذِينَ يَسْأَلُونَهُ».
\par 14 وَكَانَ يُخْرِجُ شَيْطَاناً وَكَانَ ذَلِكَ أَخْرَسَ. فَلَمَّا أُخْرِجَ الشَّيْطَانُ تَكَلَّمَ الأَخْرَسُ فَتَعَجَّبَ الْجُمُوعُ.
\par 15 وَأَمَّا قَوْمٌ مِنْهُمْ فَقَالُوا: «بِبَعْلَزَبُولَ رَئِيسِ الشَّيَاطِينِ يُخْرِجُ الشَّيَاطِينَ».
\par 16 وَآخَرُونَ طَلَبُوا مِنْهُ آيَةً مِنَ السَّمَاءِ يُجَرِّبُونَهُ.
\par 17 فَعَلِمَ أَفْكَارَهُمْ وَقَالَ لَهُمْ: «كُلُّ مَمْلَكَةٍ مُنْقَسِمَةٍ عَلَى ذَاتِهَا تَخْرَبُ وَبَيْتٍ مُنْقَسِمٍ عَلَى بَيْتٍ يَسْقُطُ.
\par 18 فَإِنْ كَانَ الشَّيْطَانُ أَيْضاً يَنْقَسِمُ عَلَى ذَاتِهِ فَكَيْفَ تَثْبُتُ مَمْلَكَتُهُ؟ لأَنَّكُمْ تَقُولُونَ: إِنِّي بِبَعْلَزَبُولَ أُخْرِجُ الشَّيَاطِينَ.
\par 19 فَإِنْ كُنْتُ أَنَا بِبَعْلَزَبُولَ أُخْرِجُ الشَّيَاطِينَ فَأَبْنَاؤُكُمْ بِمَنْ يُخْرِجُونَ؟ لِذَلِكَ هُمْ يَكُونُونَ قُضَاتَكُمْ.
\par 20 وَلَكِنْ إِنْ كُنْتُ بِإِصْبِعِ اللهِ أُخْرِجُ الشَّيَاطِينَ فَقَدْ أَقْبَلَ عَلَيْكُمْ مَلَكُوتُ اللهِ.
\par 21 حِينَمَا يَحْفَظُ الْقَوِيُّ دَارَهُ مُتَسَلِّحاً تَكُونُ أَمْوَالُهُ فِي أَمَانٍ.
\par 22 وَلَكِنْ مَتَى جَاءَ مَنْ هُوَ أَقْوَى مِنْهُ فَإِنَّهُ يَغْلِبُهُ وَيَنْزِعُ سِلاَحَهُ الْكَامِلَ الَّذِي اتَّكَلَ عَلَيْهِ وَيُوَزِّعُ غَنَائِمَهُ.
\par 23 مَنْ لَيْسَ مَعِي فَهُوَ عَلَيَّ وَمَنْ لاَ يَجْمَعُ مَعِي فَهُوَ يُفَرِّقُ.
\par 24 مَتَى خَرَجَ الرُّوحُ النَّجِسُ مِنَ الإِنْسَانِ يَجْتَازُ فِي أَمَاكِنَ لَيْسَ فِيهَا مَاءٌ يَطْلُبُ رَاحَةً وَإِذْ لاَ يَجِدُ يَقُولُ: أَرْجِعُ إِلَى بَيْتِي الَّذِي خَرَجْتُ مِنْهُ
\par 25 فَيَأْتِي وَيَجِدُهُ مَكْنُوساً مُزَيَّناً.
\par 26 ثُمَّ يَذْهَبُ وَيَأْخُذُ سَبْعَةَ أَرْوَاحٍ أُخَرَ أَشَرَّ مِنْهُ فَتَدْخُلُ وَتَسْكُنُ هُنَاكَ فَتَصِيرُ أَوَاخِرُ ذَلِكَ الإِنْسَانِ أَشَرَّ مِنْ أَوَائِلِهِ!»
\par 27 وَفِيمَا هُوَ يَتَكَلَّمُ بِهَذَا رَفَعَتِ امْرَأَةٌ صَوْتَهَا مِنَ الْجَمْعِ وَقَالَتْ لَهُ: «طُوبَى لِلْبَطْنِ الَّذِي حَمَلَكَ وَالثَّدْيَيْنِ اللَّذَيْنِ رَضَعْتَهُمَا».
\par 28 أَمَّا هُوَ فَقَالَ: «بَلْ طُوبَى لِلَّذِينَ يَسْمَعُونَ كَلاَمَ اللهِ وَيَحْفَظُونَهُ».
\par 29 وَفِيمَا كَانَ الْجُمُوعُ مُزْدَحِمِينَ ابْتَدَأَ يَقُولُ: «هَذَا الْجِيلُ شِرِّيرٌ. يَطْلُبُ آيَةً وَلاَ تُعْطَى لَهُ آيَةٌ إِلاَّ آيَةُ يُونَانَ النَّبِيِّ.
\par 30 لأَنَّهُ كَمَا كَانَ يُونَانُ آيَةً لأَهْلِ نِينَوَى كَذَلِكَ يَكُونُ ابْنُ الإِنْسَانِ أَيْضاً لِهَذَا الْجِيلِ.
\par 31 مَلِكَةُ التَّيْمَنِ سَتَقُومُ فِي الدِّينِ مَعَ رِجَالِ هَذَا الْجِيلِ وَتَدِينُهُمْ لأَنَّهَا أَتَتْ مِنْ أَقَاصِي الأَرْضِ لِتَسْمَعَ حِكْمَةَ سُلَيْمَانَ وَهُوَذَا أَعْظَمُ مِنْ سُلَيْمَانَ هَهُنَا.
\par 32 رِجَالُ نِينَوَى سَيَقُومُونَ فِي الدِّينِ مَعَ هَذَا الْجِيلِ وَيَدِينُونَهُ لأَنَّهُمْ تَابُوا بِمُنَادَاةِ يُونَانَ وَهُوَذَا أَعْظَمُ مِنْ يُونَانَ هَهُنَا!
\par 33 «لَيْسَ أَحَدٌ يُوقِدُ سِرَاجاً وَيَضَعُهُ فِي خُفْيَةٍ وَلاَ تَحْتَ الْمِكْيَالِ بَلْ عَلَى الْمَنَارَةِ لِكَيْ يَنْظُرَ الدَّاخِلُونَ النُّورَ.
\par 34 سِرَاجُ الْجَسَدِ هُوَ الْعَيْنُ فَمَتَى كَانَتْ عَيْنُكَ بَسِيطَةً فَجَسَدُكَ كُلُّهُ يَكُونُ نَيِّراً وَمَتَى كَانَتْ شِرِّيرَةً فَجَسَدُكَ يَكُونُ مُظْلِماً.
\par 35 اُنْظُرْ إِذاً لِئَلاَّ يَكُونَ النُّورُ الَّذِي فِيكَ ظُلْمَةً.
\par 36 فَإِنْ كَانَ جَسَدُكَ كُلُّهُ نَيِّراً لَيْسَ فِيهِ جُزْءٌ مُظْلِمٌ يَكُونُ نَيِّراً كُلُّهُ كَمَا حِينَمَا يُضِيءُ لَكَ السِّرَاجُ بِلَمَعَانِهِ».
\par 37 وَفِيمَا هُوَ يَتَكَلَّمُ سَأَلَهُ فَرِّيسِيٌّ أَنْ يَتَغَدَّى عِنْدَهُ فَدَخَلَ وَاتَّكَأَ.
\par 38 وَأَمَّا الْفَرِّيسِيُّ فَلَمَّا رَأَى ذَلِكَ تَعَجَّبَ أَنَّهُ لَمْ يَغْتَسِلْ أَوَّلاً قَبْلَ الْغَدَاءِ.
\par 39 فَقَالَ لَهُ الرَّبُّ: «أَنْتُمُ الآنَ أَيُّهَا الْفَرِّيسِيُّونَ تُنَقُّونَ خَارِجَ الْكَأْسِ وَالْقَصْعَةِ وَأَمَّا بَاطِنُكُمْ فَمَمْلُوءٌ اخْتِطَافاً وَخُبْثاً.
\par 40 يَا أَغْبِيَاءُ أَلَيْسَ الَّذِي صَنَعَ الْخَارِجَ صَنَعَ الدَّاخِلَ أَيْضاً؟
\par 41 بَلْ أَعْطُوا مَا عِنْدَكُمْ صَدَقَةً فَهُوَذَا كُلُّ شَيْءٍ يَكُونُ نَقِيّاً لَكُمْ.
\par 42 وَلَكِنْ وَيْلٌ لَكُمْ أَيُّهَا الْفَرِّيسِيُّونَ لأَنَّكُمْ تُعَشِّرُونَ النَّعْنَعَ وَالسَّذَابَ وَكُلَّ بَقْلٍ وَتَتَجَاوَزُونَ عَنِ الْحَقِّ وَمَحَبَّةِ اللهِ. كَانَ يَنْبَغِي أَنْ تَعْمَلُوا هَذِهِ وَلاَ تَتْرُكُوا تِلْكَ!
\par 43 وَيْلٌ لَكُمْ أَيُّهَا الْفَرِّيسِيُّونَ لأَنَّكُمْ تُحِبُّونَ الْمَجْلِسَ الأَوَّلَ فِي الْمَجَامِعِ وَالتَّحِيَّاتِ فِي الأَسْوَاقِ.
\par 44 وَيْلٌ لَكُمْ أَيُّهَا الْكَتَبَةُ وَالْفَرِّيسِيُّونَ الْمُرَاؤُونَ لأَنَّكُمْ مِثْلُ الْقُبُورِ الْمُخْتَفِيَةِ وَالَّذِينَ يَمْشُونَ عَلَيْهَا لاَ يَعْلَمُونَ!».
\par 45 فَقَالَ لَهُ وَاحِدٌ مِنَ النَّامُوسِيِّينَ: «يَا مُعَلِّمُ حِينَ تَقُولُ هَذَا تَشْتِمُنَا نَحْنُ أَيْضاً».
\par 46 فَقَالَ: «وَوَيْلٌ لَكُمْ أَنْتُمْ أَيُّهَا النَّامُوسِيُّونَ لأَنَّكُمْ تُحَمِّلُونَ النَّاسَ أَحْمَالاً عَسِرَةَ الْحَمْلِ وَأَنْتُمْ لاَ تَمَسُّونَ الأَحْمَالَ بِإِحْدَى أَصَابِعِكُمْ.
\par 47 وَيْلٌ لَكُمْ لأَنَّكُمْ تَبْنُونَ قُبُورَ الأَنْبِيَاءِ وَآبَاؤُكُمْ قَتَلُوهُمْ.
\par 48 إِذاً تَشْهَدُونَ وَتَرْضَوْنَ بِأَعْمَالِ آبَائِكُمْ لأَنَّهُمْ هُمْ قَتَلُوهُمْ وَأَنْتُمْ تَبْنُونَ قُبُورَهُمْ.
\par 49 لِذَلِكَ أَيْضاً قَالَتْ حِكْمَةُ اللهِ: إِنِّي أُرْسِلُ إِلَيْهِمْ أَنْبِيَاءَ وَرُسُلاً فَيَقْتُلُونَ مِنْهُمْ وَيَطْرُدُونَ -
\par 50 لِكَيْ يُطْلَبَ مِنْ هَذَا الْجِيلِ دَمُ جَمِيعِ الأَنْبِيَاءِ الْمُهْرَقُ مُنْذُ إِنْشَاءِ الْعَالَمِ
\par 51 مِنْ دَمِ هَابِيلَ إِلَى دَمِ زَكَرِيَّا الَّذِي أُهْلِكَ بَيْنَ الْمَذْبَحِ وَالْبَيْتِ. نَعَمْ أَقُولُ لَكُمْ: إِنَّهُ يُطْلَبُ مِنْ هَذَا الْجِيلِ!
\par 52 وَيْلٌ لَكُمْ أَيُّهَا النَّامُوسِيُّونَ لأَنَّكُمْ أَخَذْتُمْ مِفْتَاحَ الْمَعْرِفَةِ. مَا دَخَلْتُمْ أَنْتُمْ وَالدَّاخِلُونَ مَنَعْتُمُوهُمْ».
\par 53 وَفِيمَا هُوَ يُكَلِّمُهُمْ بِهَذَا ابْتَدَأَ الْكَتَبَةُ وَالْفَرِّيسِيُّونَ يَحْنَقُونَ جِدّاً وَيُصَادِرُونَهُ عَلَى أُمُورٍ كَثِيرَةٍ
\par 54 وَهُمْ يُرَاقِبُونَهُ طَالِبِينَ أَنْ يَصْطَادُوا شَيْئاً مِنْ فَمِهِ لِكَيْ يَشْتَكُوا عَلَيْهِ.

\chapter{12}

\par 1 وَفِي أَثْنَاءِ ذَلِكَ إِذِ اجْتَمَعَ رَبَوَاتُ الشَّعْبِ حَتَّى كَانَ بَعْضُهُمْ يَدُوسُ بَعْضاً ابْتَدَأَ يَقُولُ لِتَلاَمِيذِهِ: «أَوَّلاً تَحَرَّزُوا لأَنْفُسِكُمْ مِنْ خَمِيرِ الْفَرِّيسِيِّينَ الَّذِي هُوَ الرِّيَاءُ
\par 2 فَلَيْسَ مَكْتُومٌ لَنْ يُسْتَعْلَنَ وَلاَ خَفِيٌّ لَنْ يُعْرَفَ.
\par 3 لِذَلِكَ كُلُّ مَا قُلْتُمُوهُ فِي الظُّلْمَةِ يُسْمَعُ فِي النُّورِ وَمَا كَلَّمْتُمْ بِهِ الأُذُنَ فِي الْمَخَادِعِ يُنَادَى بِهِ عَلَى السُّطُوحِ.
\par 4 وَلَكِنْ أَقُولُ لَكُمْ يَا أَحِبَّائِي: لاَ تَخَافُوا مِنَ الَّذِينَ يَقْتُلُونَ الْجَسَدَ وَبَعْدَ ذَلِكَ لَيْسَ لَهُمْ مَا يَفْعَلُونَ أَكْثَرَ.
\par 5 بَلْ أُرِيكُمْ مِمَّنْ تَخَافُونَ: خَافُوا مِنَ الَّذِي بَعْدَمَا يَقْتُلُ لَهُ سُلْطَانٌ أَنْ يُلْقِيَ فِي جَهَنَّمَ. نَعَمْ أَقُولُ لَكُمْ: مِنْ هَذَا خَافُوا!
\par 6 أَلَيْسَتْ خَمْسَةُ عَصَافِيرَ تُبَاعُ بِفَلْسَيْنِ وَوَاحِدٌ مِنْهَا لَيْسَ مَنْسِيّاً أَمَامَ اللهِ؟
\par 7 بَلْ شُعُورُ رُؤُوسِكُمْ أَيْضاً جَمِيعُهَا مُحْصَاةٌ! فَلاَ تَخَافُوا. أَنْتُمْ أَفْضَلُ مِنْ عَصَافِيرَ كَثِيرَةٍ!
\par 8 وَأَقُولُ لَكُمْ: كُلُّ مَنِ اعْتَرَفَ بِي قُدَّامَ النَّاسِ يَعْتَرِفُ بِهِ ابْنُ الإِنْسَانِ قُدَّامَ مَلاَئِكَةِ اللهِ.
\par 9 وَمَنْ أَنْكَرَنِي قُدَّامَ النَّاسِ يُنْكَرُ قُدَّامَ مَلاَئِكَةِ اللهِ.
\par 10 وَكُلُّ مَنْ قَالَ كَلِمَةً عَلَى ابْنِ الإِنْسَانِ يُغْفَرُ لَهُ وَأَمَّا مَنْ جَدَّفَ عَلَى الرُّوحِ الْقُدُسِ فَلاَ يُغْفَرُ لَهُ.
\par 11 وَمَتَى قَدَّمُوكُمْ إِلَى الْمَجَامِعِ وَالرُّؤَسَاءِ وَالسَّلاَطِينِ فَلاَ تَهْتَمُّوا كَيْفَ أَوْ بِمَا تَحْتَجُّونَ أَوْ بِمَا تَقُولُونَ
\par 12 لأَنَّ الرُّوحَ الْقُدُسَ يُعَلِّمُكُمْ فِي تِلْكَ السَّاعَةِ مَا يَجِبُ أَنْ تَقُولُوهُ».
\par 13 وَقَالَ لَهُ وَاحِدٌ مِنَ الْجَمْعِ: «يَا مُعَلِّمُ قُلْ لأَخِي أَنْ يُقَاسِمَنِي الْمِيرَاثَ».
\par 14 فَقَالَ لَهُ: «يَا إِنْسَانُ مَنْ أَقَامَنِي عَلَيْكُمَا قَاضِياً أَوْ مُقَسِّماً؟»
\par 15 وَقَالَ لَهُمُ: «انْظُرُوا وَتَحَفَّظُوا مِنَ الطَّمَعِ فَإِنَّهُ مَتَى كَانَ لأَحَدٍ كَثِيرٌ فَلَيْسَتْ حَيَاتُهُ مِنْ أَمْوَالِهِ».
\par 16 وَضَرَبَ لَهُمْ مَثَلاً قَائِلاً: «إِنْسَانٌ غَنِيٌّ أَخْصَبَتْ كُورَتُهُ
\par 17 فَفَكَّرَ فِي نَفْسِهِ قَائِلاً: مَاذَا أَعْمَلُ لأَنْ لَيْسَ لِي مَوْضِعٌ أَجْمَعُ فِيهِ أَثْمَارِي؟
\par 18 وَقَالَ: أَعْمَلُ هَذَا: أَهْدِمُ مَخَازِنِي وَأَبْنِي أَعْظَمَ وَأَجْمَعُ هُنَاكَ جَمِيعَ غَلاَّتِي وَخَيْرَاتِي
\par 19 وَأَقُولُ لِنَفْسِي: يَا نَفْسُ لَكِ خَيْرَاتٌ كَثِيرَةٌ مَوْضُوعَةٌ لِسِنِينَ كَثِيرَةٍ. اِسْتَرِيحِي وَكُلِي وَاشْرَبِي وَافْرَحِي.
\par 20 فَقَالَ لَهُ اللهُ: يَا غَبِيُّ هَذِهِ اللَّيْلَةَ تُطْلَبُ نَفْسُكَ مِنْكَ فَهَذِهِ الَّتِي أَعْدَدْتَهَا لِمَنْ تَكُونُ؟
\par 21 هَكَذَا الَّذِي يَكْنِزُ لِنَفْسِهِ وَلَيْسَ هُوَ غَنِيّاً لِلَّهِ».
\par 22 وَقَالَ لِتَلاَمِيذِهِ: «مِنْ أَجْلِ هَذَا أَقُولُ لَكُمْ: لاَ تَهْتَمُّوا لِحَيَاتِكُمْ بِمَا تَأْكُلُونَ وَلاَ لِلْجَسَدِ بِمَا تَلْبَسُونَ.
\par 23 اَلْحَيَاةُ أَفْضَلُ مِنَ الطَّعَامِ وَالْجَسَدُ أَفْضَلُ مِنَ اللِّبَاسِ.
\par 24 تَأَمَّلُوا الْغِرْبَانَ: أَنَّهَا لاَ تَزْرَعُ وَلاَ تَحْصُدُ وَلَيْسَ لَهَا مَخْدَعٌ وَلاَ مَخْزَنٌ وَاللهُ يُقِيتُهَا. كَمْ أَنْتُمْ بِالْحَرِيِّ أَفْضَلُ مِنَ الطُّيُورِ!
\par 25 وَمَنْ مِنْكُمْ إِذَا اهْتَمَّ يَقْدِرُ أَنْ يَزِيدَ عَلَى قَامَتِهِ ذِرَاعاً وَاحِدَةً؟
\par 26 فَإِنْ كُنْتُمْ لاَ تَقْدِرُونَ وَلاَ عَلَى الأَصْغَرِ فَلِمَاذَا تَهْتَمُّونَ بِالْبَوَاقِي؟
\par 27 تَأَمَّلُوا الزَّنَابِقَ كَيْفَ تَنْمُو! لاَ تَتْعَبُ وَلاَ تَغْزِلُ وَلَكِنْ أَقُولُ لَكُمْ إِنَّهُ وَلاَ سُلَيْمَانُ فِي كُلِّ مَجْدِهِ كَانَ يَلْبَسُ كَوَاحِدَةٍ مِنْهَا.
\par 28 فَإِنْ كَانَ الْعُشْبُ الَّذِي يُوجَدُ الْيَوْمَ فِي الْحَقْلِ وَيُطْرَحُ غَداً فِي التَّنُّورِ يُلْبِسُهُ اللهُ هَكَذَا فَكَمْ بِالْحَرِيِّ يُلْبِسُكُمْ أَنْتُمْ يَا قَلِيلِي الإِيمَانِ؟
\par 29 فَلاَ تَطْلُبُوا أَنْتُمْ مَا تَأْكُلُونَ وَمَا تَشْرَبُونَ وَلاَ تَقْلَقُوا
\par 30 فَإِنَّ هَذِهِ كُلَّهَا تَطْلُبُهَا أُمَمُ الْعَالَمِ. وَأَمَّا أَنْتُمْ فَأَبُوكُمْ يَعْلَمُ أَنَّكُمْ تَحْتَاجُونَ إِلَى هَذِهِ.
\par 31 بَلِ اطْلُبُوا مَلَكُوتَ اللهِ وَهَذِهِ كُلُّهَا تُزَادُ لَكُمْ.
\par 32 «لاَ تَخَفْ أَيُّهَا الْقَطِيعُ الصَّغِيرُ لأَنَّ أَبَاكُمْ قَدْ سُرَّ أَنْ يُعْطِيَكُمُ الْمَلَكُوتَ.
\par 33 بِيعُوا مَا لَكُمْ وَأَعْطُوا صَدَقَةً. اِعْمَلُوا لَكُمْ أَكْيَاساً لاَ تَفْنَى وَكَنْزاً لاَ يَنْفَدُ فِي السَّمَاوَاتِ حَيْثُ لاَ يَقْرَبُ سَارِقٌ وَلاَ يُبْلِي سُوسٌ
\par 34 لأَنَّهُ حَيْثُ يَكُونُ كَنْزُكُمْ هُنَاكَ يَكُونُ قَلْبُكُمْ أَيْضاً.
\par 35 لِتَكُنْ أَحْقَاؤُكُمْ مُمَنْطَقَةً وَسُرُجُكُمْ مُوقَدَةً
\par 36 وَأَنْتُمْ مِثْلُ أُنَاسٍ يَنْتَظِرُونَ سَيِّدَهُمْ مَتَى يَرْجِعُ مِنَ الْعُرْسِ حَتَّى إِذَا جَاءَ وَقَرَعَ يَفْتَحُونَ لَهُ لِلْوَقْتِ.
\par 37 طُوبَى لأُولَئِكَ الْعَبِيدِ الَّذِينَ إِذَا جَاءَ سَيِّدُهُمْ يَجِدُهُمْ سَاهِرِينَ. اَلْحَقَّ أَقُولُ لَكُمْ إِنَّهُ يَتَمَنْطَقُ وَيُتْكِئُهُمْ وَيَتَقَدَّمُ وَيَخْدِمُهُمْ.
\par 38 وَإِنْ أَتَى فِي الْهَزِيعِ الثَّانِي أَوْ أَتَى فِي الْهَزِيعِ الثَّالِثِ وَوَجَدَهُمْ هَكَذَا فَطُوبَى لأُولَئِكَ الْعَبِيدِ.
\par 39 وَإِنَّمَا اعْلَمُوا هَذَا: أَنَّهُ لَوْ عَرَفَ رَبُّ الْبَيْتِ فِي أَيَّةِ سَاعَةٍ يَأْتِي السَّارِقُ لَسَهِرَ وَلَمْ يَدَعْ بَيْتَهُ يُنْقَبُ.
\par 40 فَكُونُوا أَنْتُمْ إِذاً مُسْتَعِدِّينَ لأَنَّهُ فِي سَاعَةٍ لاَ تَظُنُّونَ يَأْتِي ابْنُ الإِنْسَانِ».
\par 41 فَقَالَ لَهُ بُطْرُسُ: «يَا رَبُّ أَلَنَا تَقُولُ هَذَا الْمَثَلَ أَمْ لِلْجَمِيعِ أَيْضاً؟»
\par 42 فَقَالَ الرَّبُّ: «فَمَنْ هُوَ الْوَكِيلُ الأَمِينُ الْحَكِيمُ الَّذِي يُقِيمُهُ سَيِّدُهُ عَلَى خَدَمِهِ لِيُعْطِيَهُمُ الْعُلُوفَةَ فِي حِينِهَا؟
\par 43 طُوبَى لِذَلِكَ الْعَبْدِ الَّذِي إِذَا جَاءَ سَيِّدُهُ يَجِدُهُ يَفْعَلُ هَكَذَا!
\par 44 بِالْحَقِّ أَقُولُ لَكُمْ إِنَّهُ يُقِيمُهُ عَلَى جَمِيعِ أَمْوَالِهِ.
\par 45 وَلَكِنْ إِنْ قَالَ ذَلِكَ الْعَبْدُ فِي قَلْبِهِ: سَيِّدِي يُبْطِئُ قُدُومَهُ فَيَبْتَدِئُ يَضْرِبُ الْغِلْمَانَ وَالْجَوَارِيَ وَيَأْكُلُ وَيَشْرَبُ وَيَسْكَرُ.
\par 46 يَأْتِي سَيِّدُ ذَلِكَ الْعَبْدِ فِي يَوْمٍ لاَ يَنْتَظِرُهُ وَفِي سَاعَةٍ لاَ يَعْرِفُهَا فَيَقْطَعُهُ وَيَجْعَلُ نَصِيبَهُ مَعَ الْخَائِنِينَ.
\par 47 وَأَمَّا ذَلِكَ الْعَبْدُ الَّذِي يَعْلَمُ إِرَادَةَ سَيِّدِهِ وَلاَ يَسْتَعِدُّ وَلاَ يَفْعَلُ بِحَسَبِ إِرَادَتِهِ فَيُضْرَبُ كَثِيراً.
\par 48 وَلَكِنَّ الَّذِي لاَ يَعْلَمُ وَيَفْعَلُ مَا يَسْتَحِقُّ ضَرَبَاتٍ يُضْرَبُ قَلِيلاً. فَكُلُّ مَنْ أُعْطِيَ كَثِيراً يُطْلَبُ مِنْهُ كَثِيرٌ وَمَنْ يُودِعُونَهُ كَثِيراً يُطَالِبُونَهُ بِأَكْثَرَ.
\par 49 «جِئْتُ لأُلْقِيَ نَاراً عَلَى الأَرْضِ فَمَاذَا أُرِيدُ لَوِ اضْطَرَمَتْ؟
\par 50 وَلِي صِبْغَةٌ أَصْطَبِغُهَا وَكَيْفَ أَنْحَصِرُ حَتَّى تُكْمَلَ؟
\par 51 أَتَظُنُّونَ أَنِّي جِئْتُ لأُعْطِيَ سَلاَماً عَلَى الأَرْضِ؟ كَلاَّ أَقُولُ لَكُمْ! بَلِ انْقِسَاماً.
\par 52 لأَنَّهُ يَكُونُ مِنَ الآنَ خَمْسَةٌ فِي بَيْتٍ وَاحِدٍ مُنْقَسِمِينَ: ثَلاَثَةٌ عَلَى اثْنَيْنِ وَاثْنَانِ عَلَى ثَلاَثَةٍ.
\par 53 يَنْقَسِمُ الأَبُ عَلَى الاِبْنِ وَالاِبْنُ عَلَى الأَبِ وَالأُمُّ عَلَى الْبِنْتِ وَالْبِنْتُ عَلَى الأُمِّ وَالْحَمَاةُ عَلَى كَنَّتِهَا وَالْكَنَّةُ عَلَى حَمَاتِهَا».
\par 54 ثُمَّ قَالَ أَيْضاً لِلْجُمُوعِ: «إِذَا رَأَيْتُمُ السَّحَابَ تَطْلُعُ مِنَ الْمَغَارِبِ فَلِلْوَقْتِ تَقُولُونَ: إِنَّهُ يَأْتِي مَطَرٌ. فَيَكُونُ هَكَذَا.
\par 55 وَإِذَا رَأَيْتُمْ رِيحَ الْجَنُوبِ تَهُبُّ تَقُولُونَ: إِنَّهُ سَيَكُونُ حَرٌّ. فَيَكُونُ.
\par 56 يَا مُرَاؤُونَ تَعْرِفُونَ أَنْ تُمَيِّزُوا وَجْهَ الأَرْضِ وَالسَّمَاءِ وَأَمَّا هَذَا الزَّمَانُ فَكَيْفَ لاَ تُمَيِّزُونَهُ؟
\par 57 وَلِمَاذَا لاَ تَحْكُمُونَ بِالْحَقِّ مِنْ قِبَلِ نُفُوسِكُمْ؟
\par 58 حِينَمَا تَذْهَبُ مَعَ خَصْمِكَ إِلَى الْحَاكِمِ ابْذُلِ الْجَهْدَ وَأَنْتَ فِي الطَّرِيقِ لِتَتَخَلَّصَ مِنْهُ لِئَلاَّ يَجُرَّكَ إِلَى الْقَاضِي وَيُسَلِّمَكَ الْقَاضِي إِلَى الْحَاكِمِ فَيُلْقِيَكَ الْحَاكِمُ فِي السِّجْنِ.
\par 59 أَقُولُ لَكَ: لاَ تَخْرُجُ مِنْ هُنَاكَ حَتَّى تُوفِيَ الْفَلْسَ الأَخِيرَ».

\chapter{13}

\par 1 وَكَانَ حَاضِراً فِي ذَلِكَ الْوَقْتِ قَوْمٌ يُخْبِرُونَهُ عَنِ الْجَلِيلِيِّينَ الَّذِينَ خَلَطَ بِيلاَطُسُ دَمَهُمْ بِذَبَائِحِهِمْ.
\par 2 فَقَالَ يَسُوعُ لَهُمْ: «أَتَظُنُّونَ أَنَّ هَؤُلاَءِ الْجَلِيلِيِّينَ كَانُوا خُطَاةً أَكْثَرَ مِنْ كُلِّ الْجَلِيلِيِّينَ لأَنَّهُمْ كَابَدُوا مِثْلَ هَذَا؟
\par 3 كَلاَّ أَقُولُ لَكُمْ. بَلْ إِنْ لَمْ تَتُوبُوا فَجَمِيعُكُمْ كَذَلِكَ تَهْلِكُونَ.
\par 4 أَوْ أُولَئِكَ الثَّمَانِيَةَ عَشَرَ الَّذِينَ سَقَطَ عَلَيْهِمُ الْبُرْجُ فِي سِلْوَامَ وَقَتَلَهُمْ أَتَظُنُّونَ أَنَّ هَؤُلاَءِ كَانُوا مُذْنِبِينَ أَكْثَرَ مِنْ جَمِيعِ النَّاسِ السَّاكِنِينَ فِي أُورُشَلِيمَ؟
\par 5 كَلاَّ أَقُولُ لَكُمْ! بَلْ إِنْ لَمْ تَتُوبُوا فَجَمِيعُكُمْ كَذَلِكَ تَهْلِكُونَ».
\par 6 وَقَالَ هَذَا الْمَثَلَ: «كَانَتْ لِوَاحِدٍ شَجَرَةُ تِينٍ مَغْرُوسَةٌ فِي كَرْمِهِ فَأَتَى يَطْلُبُ فِيهَا ثَمَراً وَلَمْ يَجِدْ.
\par 7 فَقَالَ لِلْكَرَّامِ: هُوَذَا ثَلاَثُ سِنِينَ آتِي أَطْلُبُ ثَمَراً فِي هَذِهِ التِّينَةِ وَلَمْ أَجِدْ. اِقْطَعْهَا. لِمَاذَا تُبَطِّلُ الأَرْضَ أَيْضاً؟
\par 8 فَأَجَابَ: يَا سَيِّدُ اتْرُكْهَا هَذِهِ السَّنَةَ أَيْضاً حَتَّى أَنْقُبَ حَوْلَهَا وَأَضَعَ زِبْلاً.
\par 9 فَإِنْ صَنَعَتْ ثَمَراً وَإِلاَّ فَفِيمَا بَعْدُ تَقْطَعُهَا».
\par 10 وَكَانَ يُعَلِّمُ فِي أَحَدِ الْمَجَامِعِ فِي السَّبْتِ
\par 11 وَإِذَا امْرَأَةٌ كَانَ بِهَا رُوحُ ضُعْفٍ ثَمَانِيَ عَشْرَةَ سَنَةً وَكَانَتْ مُنْحَنِيَةً وَلَمْ تَقْدِرْ أَنْ تَنْتَصِبَ الْبَتَّةَ.
\par 12 فَلَمَّا رَآهَا يَسُوعُ دَعَاهَا وَقَالَ لَهَا: «يَا امْرَأَةُ إِنَّكِ مَحْلُولَةٌ مِنْ ضُعْفِكِ».
\par 13 وَوَضَعَ عَلَيْهَا يَدَيْهِ فَفِي الْحَالِ اسْتَقَامَتْ وَمَجَّدَتِ اللهَ.
\par 14 فَرَئِيسُ الْمَجْمَعِ وَهُوَ مُغْتَاظٌ لأَنَّ يَسُوعَ أَبْرَأَ فِي السَّبْتِ قَالَ لِلْجَمْعِ: «هِيَ سِتَّةُ أَيَّامٍ يَنْبَغِي فِيهَا الْعَمَلُ فَفِي هَذِهِ ائْتُوا وَاسْتَشْفُوا وَلَيْسَ فِي يَوْمِ السَّبْتِ»
\par 15 فَأَجَابَهُ الرَّبُّ: «يَا مُرَائِي أَلاَ يَحُلُّ كُلُّ وَاحِدٍ مِنْكُمْ فِي السَّبْتِ ثَوْرَهُ أَوْ حِمَارَهُ مِنَ الْمِذْوَدِ وَيَمْضِي بِهِ وَيَسْقِيهِ؟
\par 16 وَهَذِهِ وَهِيَ ابْنَةُ إِبْرَهِيمَ قَدْ رَبَطَهَا الشَّيْطَانُ ثَمَانِيَ عَشْرَةَ سَنَةً أَمَا كَانَ يَنْبَغِي أَنْ تُحَلَّ مِنْ هَذَا الرِّبَاطِ فِي يَوْمِ السَّبْتِ؟»
\par 17 وَإِذْ قَالَ هَذَا أُخْجِلَ جَمِيعُ الَّذِينَ كَانُوا يُعَانِدُونَهُ وَفَرِحَ كُلُّ الْجَمْعِ بِجَمِيعِ الأَعْمَالِ الْمَجِيدَةِ الْكَائِنَةِ مِنْهُ.
\par 18 فَقَالَ: «مَاذَا يُشْبِهُ مَلَكُوتُ اللهِ وَبِمَاذَا أُشَبِّهُهُ؟
\par 19 يُشْبِهُ حَبَّةَ خَرْدَلٍ أَخَذَهَا إِنْسَانٌ وَأَلْقَاهَا فِي بُسْتَانِهِ فَنَمَتْ وَصَارَتْ شَجَرَةً كَبِيرَةً وَتَآوَتْ طُيُورُ السَّمَاءِ فِي أَغْصَانِهَا».
\par 20 وَقَالَ أَيْضاً: «بِمَاذَا أُشَبِّهُ مَلَكُوتَ اللهِ؟
\par 21 يُشْبِهُ خَمِيرَةً أَخَذَتْهَا امْرَأَةٌ وَخَبَّأَتْهَا فِي ثَلاَثَةِ أَكْيَالِ دَقِيقٍ حَتَّى اخْتَمَرَ الْجَمِيعُ».
\par 22 وَاجْتَازَ فِي مُدُنٍ وَقُرًى يُعَلِّمُ وَيُسَافِرُ نَحْوَ أُورُشَلِيمَ
\par 23 فَقَالَ لَهُ وَاحِدٌ: «يَا سَيِّدُ أَقَلِيلٌ هُمُ الَّذِينَ يَخْلُصُونَ؟» فَقَالَ لَهُمُ:
\par 24 «اجْتَهِدُوا أَنْ تَدْخُلُوا مِنَ الْبَابِ الضَّيِّقِ فَإِنِّي أَقُولُ لَكُمْ: إِنَّ كَثِيرِينَ سَيَطْلُبُونَ أَنْ يَدْخُلُوا وَلاَ يَقْدِرُونَ
\par 25 مِنْ بَعْدِ مَا يَكُونُ رَبُّ الْبَيْتِ قَدْ قَامَ وَأَغْلَقَ الْبَابَ وَابْتَدَأْتُمْ تَقِفُونَ خَارِجاً وَتَقْرَعُونَ الْبَابَ قَائِلِينَ: يَا رَبُّ يَا رَبُّ افْتَحْ لَنَا يُجِيبُكُمْ: لاَ أَعْرِفُكُمْ مِنْ أَيْنَ أَنْتُمْ!
\par 26 حِينَئِذٍ تَبْتَدِئُونَ تَقُولُونَ: أَكَلْنَا قُدَّامَكَ وَشَرِبْنَا وَعَلَّمْتَ فِي شَوَارِعِنَا.
\par 27 فَيَقُولُ: أَقُولُ لَكُمْ لاَ أَعْرِفُكُمْ مِنْ أَيْنَ أَنْتُمْ! تَبَاعَدُوا عَنِّي يَا جَمِيعَ فَاعِلِي الظُّلْمِ.
\par 28 هُنَاكَ يَكُونُ الْبُكَاءُ وَصَرِيرُ الأَسْنَانِ مَتَى رَأَيْتُمْ إِبْرَاهِيمَ وَإِسْحَاقَ وَيَعْقُوبَ وَجَمِيعَ الأَنْبِيَاءِ فِي مَلَكُوتِ اللهِ وَأَنْتُمْ مَطْرُوحُونَ خَارِجاً.
\par 29 وَيَأْتُونَ مِنَ الْمَشَارِقِ وَمِنَ الْمَغَارِبِ وَمِنَ الشِّمَالِ وَالْجَنُوبِ وَيَتَّكِئُونَ فِي مَلَكُوتِ اللهِ.
\par 30 وَهُوَذَا آخِرُونَ يَكُونُونَ أَوَّلِينَ وَأَوَّلُونَ يَكُونُونَ آخِرِينَ».
\par 31 فِي ذَلِكَ الْيَوْمِ تَقَدَّمَ بَعْضُ الْفَرِّيسِيِّينَ قَائِلِينَ لَهُ: «اخْرُجْ وَاذْهَبْ مِنْ هَهُنَا لأَنَّ هِيرُودُسَ يُرِيدُ أَنْ يَقْتُلَكَ».
\par 32 فَقَالَ لَهُمُ: «امْضُوا وَقُولُوا لِهَذَا الثَّعْلَبِ: هَا أَنَا أُخْرِجُ شَيَاطِينَ وَأَشْفِي الْيَوْمَ وَغَداً وَفِي الْيَوْمِ الثَّالِثِ أُكَمَّلُ.
\par 33 بَلْ يَنْبَغِي أَنْ أَسِيرَ الْيَوْمَ وَغَداً وَمَا يَلِيهِ لأَنَّهُ لاَ يُمْكِنُ أَنْ يَهْلِكَ نَبِيٌّ خَارِجاً عَنْ أُورُشَلِيمَ.
\par 34 يَا أُورُشَلِيمُ يَا أُورُشَلِيمُ يَا قَاتِلَةَ الأَنْبِيَاءِ وَرَاجِمَةَ الْمُرْسَلِينَ إِلَيْهَا كَمْ مَرَّةٍ أَرَدْتُ أَنْ أَجْمَعَ أَوْلاَدَكِ كَمَا تَجْمَعُ الدَّجَاجَةُ فِرَاخَهَا تَحْتَ جَنَاحَيْهَا وَلَمْ تُرِيدُوا.
\par 35 هُوَذَا بَيْتُكُمْ يُتْرَكُ لَكُمْ خَرَاباً! وَالْحَقَّ أَقُولُ لَكُمْ: إِنَّكُمْ لاَ تَرَوْنَنِي حَتَّى يَأْتِيَ وَقْتٌ تَقُولُونَ فِيهِ: مُبَارَكٌ الآتِي بِاسْمِ الرَّبِّ».

\chapter{14}

\par 1 وَإِذْ جَاءَ إِلَى بَيْتِ أَحَدِ رُؤَسَاءِ الْفَرِّيسِيِّينَ فِي السَّبْتِ لِيَأْكُلَ خُبْزاً كَانُوا يُرَاقِبُونَهُ.
\par 2 وَإِذَا إِنْسَانٌ مُسْتَسْقٍ كَانَ قُدَّامَهُ.
\par 3 فَسَأَلَ يَسُوعُ النَّامُوسِيِّينَ وَالْفَرِّيسِيِّينَ: «هَلْ يَحِلُّ الإِبْرَاءُ فِي السَّبْتِ؟»
\par 4 فَسَكَتُوا. فَأَمْسَكَهُ وَأَبْرَأَهُ وَأَطْلَقَهُ.
\par 5 ثُمَّ سَأَلَ: «مَنْ مِنْكُمْ يَسْقُطُ حِمَارُهُ أَوْ ثَوْرُهُ فِي بِئْرٍ وَلاَ يَنْشِلُهُ حَالاً فِي يَوْمِ السَّبْتِ؟»
\par 6 فَلَمْ يَقْدِرُوا أَنْ يُجِيبُوهُ عَنْ ذَلِكَ.
\par 7 وَقَالَ لِلْمَدْعُوِّينَ مَثَلاً وَهُوَ يُلاَحِظُ كَيْفَ اخْتَارُوا الْمُتَّكَآتِ الأُولَى:
\par 8 «مَتَى دُعِيتَ مِنْ أَحَدٍ إِلَى عُرْسٍ فَلاَ تَتَّكِئْ فِي الْمُتَّكَإِ الأَوَّلِ لَعَلَّ أَكْرَمَ مِنْكَ يَكُونُ قَدْ دُعِيَ مِنْهُ.
\par 9 فَيَأْتِيَ الَّذِي دَعَاكَ وَإِيَّاهُ وَيَقُولَ لَكَ: أَعْطِ مَكَاناً لِهَذَا. فَحِينَئِذٍ تَبْتَدِئُ بِخَجَلٍ تَأْخُذُ الْمَوْضِعَ الأَخِيرَ.
\par 10 بَلْ مَتَى دُعِيتَ فَاذْهَبْ وَاتَّكِئْ فِي الْمَوْضِعِ الأَخِيرِ حَتَّى إِذَا جَاءَ الَّذِي دَعَاكَ يَقُولُ لَكَ: يَا صَدِيقُ ارْتَفِعْ إِلَى فَوْقُ. حِينَئِذٍ يَكُونُ لَكَ مَجْدٌ أَمَامَ الْمُتَّكِئِينَ مَعَكَ.
\par 11 لأَنَّ كُلَّ مَنْ يَرْفَعُ نَفْسَهُ يَتَّضِعُ وَمَنْ يَضَعُ نَفْسَهُ يَرْتَفِعُ».
\par 12 وَقَالَ أَيْضاً لِلَّذِي دَعَاهُ: «إِذَا صَنَعْتَ غَدَاءً أَوْ عَشَاءً فَلاَ تَدْعُ أَصْدِقَاءَكَ وَلاَ إِخْوَتَكَ وَلاَ أَقْرِبَاءَكَ وَلاَ الْجِيرَانَ الأَغْنِيَاءَ لِئَلاَّ يَدْعُوكَ هُمْ أَيْضاً فَتَكُونَ لَكَ مُكَافَاةٌ.
\par 13 بَلْ إِذَا صَنَعْتَ ضِيَافَةً فَادْعُ الْمَسَاكِينَ: الْجُدْعَ الْعُرْجَ الْعُمْيَ
\par 14 فَيَكُونَ لَكَ الطُّوبَى إِذْ لَيْسَ لَهُمْ حَتَّى يُكَافُوكَ لأَنَّكَ تُكَافَى فِي قِيَامَةِ الأَبْرَارِ».
\par 15 فَلَمَّا سَمِعَ ذَلِكَ وَاحِدٌ مِنَ الْمُتَّكِئِينَ قَالَ لَهُ: «طُوبَى لِمَنْ يَأْكُلُ خُبْزاً فِي مَلَكُوتِ اللهِ».
\par 16 فَقَالَ لَهُ: «إِنْسَانٌ صَنَعَ عَشَاءً عَظِيماً وَدَعَا كَثِيرِينَ
\par 17 وَأَرْسَلَ عَبْدَهُ فِي سَاعَةِ الْعَشَاءِ لِيَقُولَ لِلْمَدْعُوِّينَ: تَعَالَوْا لأَنَّ كُلَّ شَيْءٍ قَدْ أُعِدَّ.
\par 18 فَابْتَدَأَ الْجَمِيعُ بِرَأْيٍ وَاحِدٍ يَسْتَعْفُونَ. قَالَ لَهُ الأَوَّلُ: إِنِّي اشْتَرَيْتُ حَقْلاً وَأَنَا مُضْطَرٌّ أَنْ أَخْرُجَ وَأَنْظُرَهُ. أَسْأَلُكَ أَنْ تُعْفِيَنِي.
\par 19 وَقَالَ آخَرُ: إِنِّي اشْتَرَيْتُ خَمْسَةَ أَزْوَاجِ بَقَرٍ وَأَنَا مَاضٍ لأَمْتَحِنَهَا. أَسْأَلُكَ أَنْ تُعْفِيَنِي.
\par 20 وَقَالَ آخَرُ: إِنِّي تَزَوَّجْتُ بِامْرَأَةٍ فَلِذَلِكَ لاَ أَقْدِرُ أَنْ أَجِيءَ.
\par 21 فَأَتَى ذَلِكَ الْعَبْدُ وَأَخْبَرَ سَيِّدَهُ بِذَلِكَ. حِينَئِذٍ غَضِبَ رَبُّ الْبَيْتِ وَقَالَ لِعَبْدِهِ: اخْرُجْ عَاجِلاً إِلَى شَوَارِعِ الْمَدِينَةِ وَأَزِقَّتِهَا وَأَدْخِلْ إِلَى هُنَا الْمَسَاكِينَ وَالْجُدْعَ وَالْعُرْجَ وَالْعُمْيَ.
\par 22 فَقَالَ الْعَبْدُ: يَا سَيِّدُ قَدْ صَارَ كَمَا أَمَرْتَ وَيُوجَدُ أَيْضاً مَكَانٌ.
\par 23 فَقَالَ السَّيِّدُ لِلْعَبْدِ: اخْرُجْ إِلَى الطُّرُقِ وَالسِّيَاجَاتِ وَأَلْزِمْهُمْ بِالدُّخُولِ حَتَّى يَمْتَلِئَ بَيْتِي
\par 24 لأَنِّي أَقُولُ لَكُمْ إِنَّهُ لَيْسَ وَاحِدٌ مِنْ أُولَئِكَ الرِّجَالِ الْمَدْعُوِّينَ يَذُوقُ عَشَائِي».
\par 25 وَكَانَ جُمُوعٌ كَثِيرَةٌ سَائِرِينَ مَعَهُ فَالْتَفَتَ وَقَالَ لَهُمْ:
\par 26 «إِنْ كَانَ أَحَدٌ يَأْتِي إِلَيَّ وَلاَ يُبْغِضُ أَبَاهُ وَأُمَّهُ وَامْرَأَتَهُ وَأَوْلاَدَهُ وَإِخْوَتَهُ وَأَخَوَاتِهِ حَتَّى نَفْسَهُ أَيْضاً فَلاَ يَقْدِرُ أَنْ يَكُونَ لِي تِلْمِيذاً.
\par 27 وَمَنْ لاَ يَحْمِلُ صَلِيبَهُ وَيَأْتِي وَرَائِي فَلاَ يَقْدِرُ أَنْ يَكُونَ لِي تِلْمِيذاً.
\par 28 وَمَنْ مِنْكُمْ وَهُوَ يُرِيدُ أَنْ يَبْنِيَ بُرْجاً لاَ يَجْلِسُ أَوَّلاً وَيَحْسِبُ النَّفَقَةَ هَلْ عِنْدَهُ مَا يَلْزَمُ لِكَمَالِهِ؟
\par 29 لِئَلاَّ يَضَعَ الأَسَاسَ وَلاَ يَقْدِرَ أَنْ يُكَمِّلَ فَيَبْتَدِئَ جَمِيعُ النَّاظِرِينَ يَهْزَأُونَ بِهِ
\par 30 قَائِلِينَ: هَذَا الإِنْسَانُ ابْتَدَأَ يَبْنِي وَلَمْ يَقْدِرْ أَنْ يُكَمِّلَ.
\par 31 وَأَيُّ مَلِكٍ إِنْ ذَهَبَ لِمُقَاتَلَةِ مَلِكٍ آخَرَ فِي حَرْبٍ لاَ يَجْلِسُ أَوَّلاً وَيَتَشَاوَرُ: هَلْ يَسْتَطِيعُ أَنْ يُلاَقِيَ بِعَشَرَةِ آلاَفٍ الَّذِي يَأْتِي عَلَيْهِ بِعِشْرِينَ أَلْفاً؟
\par 32 وَإِلاَّ فَمَا دَامَ ذَلِكَ بَعِيداً يُرْسِلُ سَفَارَةً وَيَسْأَلُ مَا هُوَ لِلصُّلْحِ.
\par 33 فَكَذَلِكَ كُلُّ وَاحِدٍ مِنْكُمْ لاَ يَتْرُكُ جَمِيعَ أَمْوَالِهِ لاَ يَقْدِرُ أَنْ يَكُونَ لِي تِلْمِيذاً.
\par 34 اَلْمِلْحُ جَيِّدٌ. وَلَكِنْ إِذَا فَسَدَ الْمِلْحُ فَبِمَاذَا يُصْلَحُ؟
\par 35 لاَ يَصْلُحُ لأَرْضٍ وَلاَ لِمَزْبَلَةٍ فَيَطْرَحُونَهُ خَارِجاً. مَنْ لَهُ أُذُنَانِ لِلسَّمْعِ فَلْيَسْمَعْ!».

\chapter{15}

\par 1 وَكَانَ جَمِيعُ الْعَشَّارِينَ وَالْخُطَاةِ يَدْنُونَ مِنْهُ لِيَسْمَعُوهُ.
\par 2 فَتَذَمَّرَ الْفَرِّيسِيُّونَ وَالْكَتَبَةُ قَائِلِينَ: «هَذَا يَقْبَلُ خُطَاةً وَيَأْكُلُ مَعَهُمْ».
\par 3 فَكَلَّمَهُمْ بِهَذَا الْمَثَلِ:
\par 4 «أَيُّ إِنْسَانٍ مِنْكُمْ لَهُ مِئَةُ خَرُوفٍ وَأَضَاعَ وَاحِداً مِنْهَا أَلاَ يَتْرُكُ التِّسْعَةَ وَالتِّسْعِينَ فِي الْبَرِّيَّةِ وَيَذْهَبَ لأَجْلِ الضَّالِّ حَتَّى يَجِدَهُ؟
\par 5 وَإِذَا وَجَدَهُ يَضَعُهُ عَلَى مَنْكِبَيْهِ فَرِحاً
\par 6 وَيَأْتِي إِلَى بَيْتِهِ وَيَدْعُو الأَصْدِقَاءَ وَالْجِيرَانَ قَائِلاً لَهُمُ: افْرَحُوا مَعِي لأَنِّي وَجَدْتُ خَرُوفِي الضَّالَّ.
\par 7 أَقُولُ لَكُمْ إِنَّهُ هَكَذَا يَكُونُ فَرَحٌ فِي السَّمَاءِ بِخَاطِئٍ وَاحِدٍ يَتُوبُ أَكْثَرَ مِنْ تِسْعَةٍ وَتِسْعِينَ بَارّاً لاَ يَحْتَاجُونَ إِلَى تَوْبَةٍ».
\par 8 «أَوْ أَيَّةُ امْرَأَةٍ لَهَا عَشْرَةُ دَرَاهِمَ إِنْ أَضَاعَتْ دِرْهَماً وَاحِداً أَلاَ تُوقِدُ سِرَاجاً وَتَكْنِسُ الْبَيْتَ وَتُفَتِّشُ بِاجْتِهَادٍ حَتَّى تَجِدَهُ؟
\par 9 وَإِذَا وَجَدَتْهُ تَدْعُو الصَّدِيقَاتِ وَالْجَارَاتِ قَائِلَةً: افْرَحْنَ مَعِي لأَنِّي وَجَدْتُ الدِّرْهَمَ الَّذِي أَضَعْتُهُ.
\par 10 هَكَذَا أَقُولُ لَكُمْ يَكُونُ فَرَحٌ قُدَّامَ مَلاَئِكَةِ اللهِ بِخَاطِئٍ وَاحِدٍ يَتُوبُ».
\par 11 وَقَالَ: «إِنْسَانٌ كَانَ لَهُ ابْنَانِ.
\par 12 فَقَالَ أَصْغَرُهُمَا لأَبِيهِ: يَا أَبِي أَعْطِنِي الْقِسْمَ الَّذِي يُصِيبُنِي مِنَ الْمَالِ. فَقَسَمَ لَهُمَا مَعِيشَتَهُ.
\par 13 وَبَعْدَ أَيَّامٍ لَيْسَتْ بِكَثِيرَةٍ جَمَعَ الاِبْنُ الأَصْغَرُ كُلَّ شَيْءٍ وَسَافَرَ إِلَى كُورَةٍ بَعِيدَةٍ وَهُنَاكَ بَذَّرَ مَالَهُ بِعَيْشٍ مُسْرِفٍ.
\par 14 فَلَمَّا أَنْفَقَ كُلَّ شَيْءٍ حَدَثَ جُوعٌ شَدِيدٌ فِي تِلْكَ الْكُورَةِ فَابْتَدَأَ يَحْتَاجُ.
\par 15 فَمَضَى وَالْتَصَقَ بِوَاحِدٍ مِنْ أَهْلِ تِلْكَ الْكُورَةِ فَأَرْسَلَهُ إِلَى حُقُولِهِ لِيَرْعَى خَنَازِيرَ.
\par 16 وَكَانَ يَشْتَهِي أَنْ يَمْلَأَ بَطْنَهُ مِنَ الْخُرْنُوبِ الَّذِي كَانَتِ الْخَنَازِيرُ تَأْكُلُهُ فَلَمْ يُعْطِهِ أَحَدٌ.
\par 17 فَرَجَعَ إِلَى نَفْسِهِ وَقَالَ: كَمْ مِنْ أَجِيرٍ لأَبِي يَفْضُلُ عَنْهُ الْخُبْزُ وَأَنَا أَهْلِكُ جُوعاً!
\par 18 أَقُومُ وَأَذْهَبُ إِلَى أَبِي وَأَقُولُ لَهُ: يَا أَبِي أَخْطَأْتُ إِلَى السَّمَاءِ وَقُدَّامَكَ
\par 19 وَلَسْتُ مُسْتَحِقّاً بَعْدُ أَنْ أُدْعَى لَكَ ابْناً. اِجْعَلْنِي كَأَحَدِ أَجْرَاكَ.
\par 20 فَقَامَ وَجَاءَ إِلَى أَبِيهِ. وَإِذْ كَانَ لَمْ يَزَلْ بَعِيداً رَآهُ أَبُوهُ فَتَحَنَّنَ وَرَكَضَ وَوَقَعَ عَلَى عُنُقِهِ وَقَبَّلَهُ.
\par 21 فَقَالَ لَهُ الاِبْنُ: يَا أَبِي أَخْطَأْتُ إِلَى السَّمَاءِ وَقُدَّامَكَ وَلَسْتُ مُسْتَحِقّاً بَعْدُ أَنْ أُدْعَى لَكَ ابْناً.
\par 22 فَقَالَ الأَبُ لِعَبِيدِهِ: أَخْرِجُوا الْحُلَّةَ الأُولَى وَأَلْبِسُوهُ وَاجْعَلُوا خَاتَماً فِي يَدِهِ وَحِذَاءً فِي رِجْلَيْهِ
\par 23 وَقَدِّمُوا الْعِجْلَ الْمُسَمَّنَ وَاذْبَحُوهُ فَنَأْكُلَ وَنَفْرَحَ
\par 24 لأَنَّ ابْنِي هَذَا كَانَ مَيِّتاً فَعَاشَ وَكَانَ ضَالاًّ فَوُجِدَ. فَابْتَدَأُوا يَفْرَحُونَ.
\par 25 وَكَانَ ابْنُهُ الأَكْبَرُ فِي الْحَقْلِ. فَلَمَّا جَاءَ وَقَرُبَ مِنَ الْبَيْتِ سَمِعَ صَوْتَ آلاَتِ طَرَبٍ وَرَقْصاً
\par 26 فَدَعَا وَاحِداً مِنَ الْغِلْمَانِ وَسَأَلَهُ: مَا عَسَى أَنْ يَكُونَ هَذَا؟
\par 27 فَقَالَ لَهُ: أَخُوكَ جَاءَ فَذَبَحَ أَبُوكَ الْعِجْلَ الْمُسَمَّنَ لأَنَّهُ قَبِلَهُ سَالِماً.
\par 28 فَغَضِبَ وَلَمْ يُرِدْ أَنْ يَدْخُلَ. فَخَرَجَ أَبُوهُ يَطْلُبُ إِلَيْهِ.
\par 29 فَقَالَ لأَبِيهِ: هَا أَنَا أَخْدِمُكَ سِنِينَ هَذَا عَدَدُهَا وَقَطُّ لَمْ أَتَجَاوَزْ وَصِيَّتَكَ وَجَدْياً لَمْ تُعْطِنِي قَطُّ لأَفْرَحَ مَعَ أَصْدِقَائِي.
\par 30 وَلَكِنْ لَمَّا جَاءَ ابْنُكَ هَذَا الَّذِي أَكَلَ مَعِيشَتَكَ مَعَ الزَّوَانِي ذَبَحْتَ لَهُ الْعِجْلَ الْمُسَمَّنَ.
\par 31 فَقَالَ لَهُ: يَا بُنَيَّ أَنْتَ مَعِي فِي كُلِّ حِينٍ وَكُلُّ مَا لِي فَهُوَ لَكَ.
\par 32 وَلَكِنْ كَانَ يَنْبَغِي أَنْ نَفْرَحَ وَنُسَرَّ لأَنَّ أَخَاكَ هَذَا كَانَ مَيِّتاً فَعَاشَ وَكَانَ ضَالاًّ فَوُجِدَ».

\chapter{16}

\par 1 وَقَالَ أَيْضاً لِتَلاَمِيذِهِ: «كَانَ إِنْسَانٌ غَنِيٌّ لَهُ وَكِيلٌ فَوُشِيَ بِهِ إِلَيْهِ بِأَنَّهُ يُبَذِّرُ أَمْوَالَهُ.
\par 2 فَدَعَاهُ وَقَالَ لَهُ: مَا هَذَا الَّذِي أَسْمَعُ عَنْكَ؟ أَعْطِ حِسَابَ وَكَالَتِكَ لأَنَّكَ لاَ تَقْدِرُ أَنْ تَكُونَ وَكِيلاً بَعْدُ.
\par 3 فَقَالَ الْوَكِيلُ فِي نَفْسِهِ: مَاذَا أَفْعَلُ؟ لأَنَّ سَيِّدِي يَأْخُذُ مِنِّي الْوَكَالَةَ. لَسْتُ أَسْتَطِيعُ أَنْ أَنْقُبَ وَأَسْتَحِي أَنْ أَسْتَعْطِيَ.
\par 4 قَدْ عَلِمْتُ مَاذَا أَفْعَلُ حَتَّى إِذَا عُزِلْتُ عَنِ الْوَكَالَةِ يَقْبَلُونِي فِي بُيُوتِهِمْ.
\par 5 فَدَعَا كُلَّ وَاحِدٍ مِنْ مَدْيُونِي سَيِّدِهِ وَقَالَ لِلأَوَّلِ: كَمْ عَلَيْكَ لِسَيِّدِي؟
\par 6 فَقَالَ: مِئَةُ بَثِّ زَيْتٍ. فَقَالَ لَهُ: خُذْ صَكَّكَ وَاجْلِسْ عَاجِلاً وَاكْتُبْ خَمْسِينَ.
\par 7 ثُمَّ قَالَ لِآخَرَ: وَأَنْتَ كَمْ عَلَيْكَ؟ فَقَالَ: مِئَةُ كُرِّ قَمْحٍ. فَقَالَ لَهُ: خُذْ صَكَّكَ وَاكْتُبْ ثَمَانِينَ.
\par 8 فَمَدَحَ السَّيِّدُ وَكِيلَ الظُّلْمِ إِذْ بِحِكْمَةٍ فَعَلَ لأَنَّ أَبْنَاءَ هَذَا الدَّهْرِ أَحْكَمُ مِنْ أَبْنَاءِ النُّورِ فِي جِيلِهِمْ.
\par 9 وَأَنَا أَقُولُ لَكُمُ: اصْنَعُوا لَكُمْ أَصْدِقَاءَ بِمَالِ الظُّلْمِ حَتَّى إِذَا فَنِيتُمْ يَقْبَلُونَكُمْ فِي الْمَظَالِّ الأَبَدِيَّةِ.
\par 10 اَلأَمِينُ فِي الْقَلِيلِ أَمِينٌ أَيْضاً فِي الْكَثِيرِ وَالظَّالِمُ فِي الْقَلِيلِ ظَالِمٌ أَيْضاً فِي الْكَثِيرِ.
\par 11 فَإِنْ لَمْ تَكُونُوا أُمَنَاءَ فِي مَالِ الظُّلْمِ فَمَنْ يَأْتَمِنُكُمْ عَلَى الْحَقِّ؟
\par 12 وَإِنْ لَمْ تَكُونُوا أُمَنَاءَ فِي مَا هُوَ لِلْغَيْرِ فَمَنْ يُعْطِيكُمْ مَا هُوَ لَكُمْ؟
\par 13 لاَ يَقْدِرُ خَادِمٌ أَنْ يَخْدِمَ سَيِّدَيْنِ لأَنَّهُ إِمَّا أَنْ يُبْغِضَ الْوَاحِدَ وَيُحِبَّ الآخَرَ أَوْ يُلاَزِمَ الْوَاحِدَ وَيَحْتَقِرَ الآخَرَ. لاَ تَقْدِرُونَ أَنْ تَخْدِمُوا اللهَ وَالْمَالَ».
\par 14 وَكَانَ الْفَرِّيسِيُّونَ أَيْضاً يَسْمَعُونَ هَذَا كُلَّهُ وَهُمْ مُحِبُّونَ لِلْمَالِ فَاسْتَهْزَأُوا بِهِ.
\par 15 فَقَالَ لَهُمْ: «أَنْتُمُ الَّذِينَ تُبَرِّرُونَ أَنْفُسَكُمْ قُدَّامَ النَّاسِ! وَلَكِنَّ اللهَ يَعْرِفُ قُلُوبَكُمْ. إِنَّ الْمُسْتَعْلِيَ عِنْدَ النَّاسِ هُوَ رِجْسٌ قُدَّامَ اللهِ.
\par 16 «كَانَ النَّامُوسُ وَالأَنْبِيَاءُ إِلَى يُوحَنَّا. وَمِنْ ذَلِكَ الْوَقْتِ يُبَشَّرُ بِمَلَكُوتِ اللهِ وَكُلُّ وَاحِدٍ يَغْتَصِبُ نَفْسَهُ إِلَيْهِ.
\par 17 وَلَكِنَّ زَوَالَ السَّمَاءِ وَالأَرْضِ أَيْسَرُ مِنْ أَنْ تَسْقُطَ نُقْطَةٌ وَاحِدَةٌ مِنَ النَّامُوسِ.
\par 18 كُلُّ مَنْ يُطَلِّقُ امْرَأَتَهُ وَيَتَزَوَّجُ بِأُخْرَى يَزْنِي وَكُلُّ مَنْ يَتَزَوَّجُ بِمُطَلَّقَةٍ مِنْ رَجُلٍ يَزْنِي.
\par 19 «كَانَ إِنْسَانٌ غَنِيٌّ وَكَانَ يَلْبَسُ الأَُرْجُوانَ وَالْبَزَّ وَهُوَ يَتَنَعَّمُ كُلَّ يَوْمٍ مُتَرَفِّهاً.
\par 20 وَكَانَ مِسْكِينٌ اسْمُهُ لِعَازَرُ الَّذِي طُرِحَ عِنْدَ بَابِهِ مَضْرُوباً بِالْقُرُوحِ
\par 21 وَيَشْتَهِي أَنْ يَشْبَعَ مِنَ الْفُتَاتِ السَّاقِطِ مِنْ مَائِدَةِ الْغَنِيِّ بَلْ كَانَتِ الْكِلاَبُ تَأْتِي وَتَلْحَسُ قُرُوحَهُ.
\par 22 فَمَاتَ الْمِسْكِينُ وَحَمَلَتْهُ الْمَلاَئِكَةُ إِلَى حِضْنِ إِبْرَاهِيمَ. وَمَاتَ الْغَنِيُّ أَيْضاً وَدُفِنَ
\par 23 فَرَفَعَ عَيْنَيْهِ فِي الْهَاوِيَةِ وَهُوَ فِي الْعَذَابِ وَرَأَى إِبْرَاهِيمَ مِنْ بَعِيدٍ وَلِعَازَرَ فِي حِضْنِهِ
\par 24 فَنَادَى: يَا أَبِي إِبْرَاهِيمُ ارْحَمْنِي وَأَرْسِلْ لِعَازَرَ لِيَبُلَّ طَرَفَ إِصْبَِعِهِ بِمَاءٍ وَيُبَرِّدَ لِسَانِي لأَنِّي مُعَذَّبٌ فِي هَذَا اللهِيبِ.
\par 25 فَقَالَ إِبْرَاهِيمُ: يَا ابْنِي اذْكُرْ أَنَّكَ اسْتَوْفَيْتَ خَيْرَاتِكَ فِي حَيَاتِكَ وَكَذَلِكَ لِعَازَرُ الْبَلاَيَا. وَالآنَ هُوَ يَتَعَزَّى وَأَنْتَ تَتَعَذَّبُ.
\par 26 وَفَوْقَ هَذَا كُلِّهِ بَيْنَنَا وَبَيْنَكُمْ هُوَّةٌ عَظِيمَةٌ قَدْ أُثْبِتَتْ حَتَّى إِنَّ الَّذِينَ يُرِيدُونَ الْعُبُورَ مِنْ هَهُنَا إِلَيْكُمْ لاَ يَقْدِرُونَ وَلاَ الَّذِينَ مِنْ هُنَاكَ يَجْتَازُونَ إِلَيْنَا.
\par 27 فَقَالَ: أَسْأَلُكَ إِذاً يَا أَبَتِ أَنْ تُرْسِلَهُ إِلَى بَيْتِ أَبِي
\par 28 لأَنَّ لِي خَمْسَةَ إِخْوَةٍ حَتَّى يَشْهَدَ لَهُمْ لِكَيْلاَ يَأْتُوا هُمْ أَيْضاً إِلَى مَوْضِعِ الْعَذَابِ هَذَا.
\par 29 قَالَ لَهُ إِبْرَاهِيمُ: عِنْدَهُمْ مُوسَى وَالأَنْبِيَاءُ. لِيَسْمَعُوا مِنْهُمْ.
\par 30 فَقَالَ: لاَ يَا أَبِي إِبْرَاهِيمَ. بَلْ إِذَا مَضَى إِلَيْهِمْ وَاحِدٌ مِنَ الأَمْوَاتِ يَتُوبُونَ.
\par 31 فَقَالَ لَهُ: إِنْ كَانُوا لاَ يَسْمَعُونَ مِنْ مُوسَى وَالأَنْبِيَاءِ وَلاَ إِنْ قَامَ وَاحِدٌ مِنَ الأَمْوَاتِ يُصَدِّقُونَ».

\chapter{17}

\par 1 وَقَالَ لِتَلاَمِيذِهِ: «لاَ يُمْكِنُ إِلاَّ أَنْ تَأْتِيَ الْعَثَرَاتُ وَلَكِنْ وَيْلٌ لِلَّذِي تَأْتِي بِوَاسِطَتِهِ!
\par 2 خَيْرٌ لَهُ لَوْ طُوِّقَ عُنُقُهُ بِحَجَرِ رَحىً وَطُرِحَ فِي الْبَحْرِ مِنْ أَنْ يُعْثِرَ أَحَدَ هَؤُلاَءِ الصِّغَارِ.
\par 3 اِحْتَرِزُوا لأَنْفُسِكُمْ. وَإِنْ أَخْطَأَ إِلَيْكَ أَخُوكَ فَوَبِّخْهُ وَإِنْ تَابَ فَاغْفِرْ لَهُ.
\par 4 وَإِنْ أَخْطَأَ إِلَيْكَ سَبْعَ مَرَّاتٍ فِي الْيَوْمِ وَرَجَعَ إِلَيْكَ سَبْعَ مَرَّاتٍ فِي الْيَوْمِ قَائِلاً: أَنَا تَائِبٌ فَاغْفِرْ لَهُ».
\par 5 فَقَالَ الرُّسُلُ لِلرَّبِّ: «زِدْ إِيمَانَنَا».
\par 6 فَقَالَ الرَّبُّ: «لَوْ كَانَ لَكُمْ إِيمَانٌ مِثْلُ حَبَّةِ خَرْدَلٍ لَكُنْتُمْ تَقُولُونَ لِهَذِهِ الْجُمَّيْزَةِ انْقَلِعِي وَانْغَرِسِي فِي الْبَحْرِ فَتُطِيعُكُمْ.
\par 7 «وَمَنْ مِنْكُمْ لَهُ عَبْدٌ يَحْرُثُ أَوْ يَرْعَى يَقُولُ لَهُ إِذَا دَخَلَ مِنَ الْحَقْلِ: تَقَدَّمْ سَرِيعاً وَاتَّكِئْ.
\par 8 بَلْ أَلاَ يَقُولُ لَهُ: أَعْدِدْ مَا أَتَعَشَّى بِهِ وَتَمَنْطَقْ وَاخْدِمْنِي حَتَّى آكُلَ وَأَشْرَبَ وَبَعْدَ ذَلِكَ تَأْكُلُ وَتَشْرَبُ أَنْتَ.
\par 9 فَهَلْ لِذَلِكَ الْعَبْدِ فَضْلٌ لأَنَّهُ فَعَلَ مَا أُمِرَ بِهِ؟ لاَ أَظُنُّ.
\par 10 كَذَلِكَ أَنْتُمْ أَيْضاً مَتَى فَعَلْتُمْ كُلَّ مَا أُمِرْتُمْ بِهِ فَقُولُوا: إِنَّنَا عَبِيدٌ بَطَّالُونَ. لأَنَّنَا إِنَّمَا عَمِلْنَا مَا كَانَ يَجِبُ عَلَيْنَا».
\par 11 وَفِي ذَهَابِهِ إِلَى أُورُشَلِيمَ اجْتَازَ فِي وَسَطِ السَّامِرَةِ وَالْجَلِيلِ.
\par 12 وَفِيمَا هُوَ دَاخِلٌ إِلَى قَرْيَةٍ اسْتَقْبَلَهُ عَشَرَةُ رِجَالٍ بُرْصٍ فَوَقَفُوا مِنْ بَعِيدٍ
\par 13 وَصَرَخُوا: «يَا يَسُوعُ يَا مُعَلِّمُ ارْحَمْنَا».
\par 14 فَنَظَرَ وَقَالَ لَهُمُ: «اذْهَبُوا وَأَرُوا أَنْفُسَكُمْ لِلْكَهَنَةِ». وَفِيمَا هُمْ مُنْطَلِقُونَ طَهَرُوا.
\par 15 فَوَاحِدٌ مِنْهُمْ لَمَّا رَأَى أَنَّهُ شُفِيَ رَجَعَ يُمَجِّدُ اللهَ بِصَوْتٍ عَظِيمٍ
\par 16 وَخَرَّ عَلَى وَجْهِهِ عِنْدَ رِجْلَيْهِ شَاكِراً لَهُ. وَكَانَ سَامِرِيّاً.
\par 17 فَقَالَ يَسُوعُ: «أَلَيْسَ الْعَشَرَةُ قَدْ طَهَرُوا؟ فَأَيْنَ التِّسْعَةُ؟
\par 18 أَلَمْ يُوجَدْ مَنْ يَرْجِعُ لِيُعْطِيَ مَجْداً لِلَّهِ غَيْرُ هَذَا الْغَرِيبِ الْجِنْسِ؟»
\par 19 ثُمَّ قَالَ لَهُ: «قُمْ وَامْضِ. إِيمَانُكَ خَلَّصَكَ».
\par 20 وَلَمَّا سَأَلَهُ الْفَرِّيسِيُّونَ: «مَتَى يَأْتِي مَلَكُوتُ اللهِ؟» أَجَابَهُمْ: «لاَ يَأْتِي مَلَكُوتُ اللهِ بِمُرَاقَبَةٍ
\par 21 وَلاَ يَقُولُونَ: هُوَذَا هَهُنَا أَوْ: هُوَذَا هُنَاكَ لأَنْ هَا مَلَكُوتُ اللهِ دَاخِلَكُمْ».
\par 22 وَقَالَ لِلتَّلاَمِيذِ: «سَتَأْتِي أَيَّامٌ فِيهَا تَشْتَهُونَ أَنْ تَرَوْا يَوْماً وَاحِداً مِنْ أَيَّامِ ابْنِ الإِنْسَانِ وَلاَ تَرَوْنَ.
\par 23 وَيَقُولُونَ لَكُمْ:هُوَذَا هَهُنَا أَوْ: هُوَذَا هُنَاكَ. لاَ تَذْهَبُوا وَلاَ تَتْبَعُوا
\par 24 لأَنَّهُ كَمَا أَنَّ الْبَرْقَ الَّذِي يَبْرُقُ مِنْ نَاحِيَةٍ تَحْتَ السَّمَاءِ يُضِيءُ إِلَى نَاحِيَةٍ تَحْتَ السَّمَاءِ كَذَلِكَ يَكُونُ أَيْضاً ابْنُ الإِنْسَانِ فِي يَوْمِهِ.
\par 25 وَلَكِنْ يَنْبَغِي أَوَّلاً أَنْ يَتَأَلَّمَ كَثِيراً وَيُرْفَضَ مِنْ هَذَا الْجِيلِ.
\par 26 وَكَمَا كَانَ فِي أَيَّامِ نُوحٍ كَذَلِكَ يَكُونُ أَيْضاً فِي أَيَّامِ ابْنِ الإِنْسَانِ.
\par 27 كَانُوا يَأْكُلُونَ وَيَشْرَبُونَ وَيُزوِّجُونَ وَيَتَزَوَّجُونَ إِلَى الْيَوْمِ الَّذِي فِيهِ دَخَلَ نُوحٌ الْفُلْكَ وَجَاءَ الطُّوفَانُ وَأَهْلَكَ الْجَمِيعَ.
\par 28 كَذَلِكَ أَيْضاً كَمَا كَانَ فِي أَيَّامِ لُوطٍ كَانُوا يَأْكُلُونَ وَيَشْرَبُونَ وَيَشْتَرُونَ وَيَبِيعُونَ وَيَغْرِسُونَ وَيَبْنُونَ.
\par 29 وَلَكِنَّ الْيَوْمَ الَّذِي فِيهِ خَرَجَ لُوطٌ مِنْ سَدُومَ أَمْطَرَ نَاراً وَكِبْرِيتاً مِنَ السَّمَاءِ فَأَهْلَكَ الْجَمِيعَ.
\par 30 هَكَذَا يَكُونُ فِي الْيَوْمِ الَّذِي فِيهِ يُظْهَرُ ابْنُ الإِنْسَانِ.
\par 31 فِي ذَلِكَ الْيَوْمِ مَنْ كَانَ عَلَى السَّطْحِ وَأَمْتِعَتُهُ فِي الْبَيْتِ فَلاَ يَنْزِلْ لِيَأْخُذَهَا وَالَّذِي فِي الْحَقْلِ كَذَلِكَ لاَ يَرْجِعْ إِلَى الْوَرَاءِ.
\par 32 اُذْكُرُوا امْرَأَةَ لُوطٍ!
\par 33 مَنْ طَلَبَ أَنْ يُخَلِّصَ نَفْسَهُ يُهْلِكُهَا وَمَنْ أَهْلَكَهَا يُحْيِيهَا.
\par 34 أَقُولُ لَكُمْ: إِنَّهُ فِي تِلْكَ اللَّيْلَةِ يَكُونُ اثْنَانِ عَلَى فِرَاشٍ وَاحِدٍ فَيُؤْخَذُ الْوَاحِدُ وَيُتْرَكُ الآخَرُ.
\par 35 تَكُونُ اثْنَتَانِ تَطْحَنَانِ مَعاً فَتُؤْخَذُ الْوَاحِدَةُ وَتُتْرَكُ الأُخْرَى.
\par 36 يَكُونُ اثْنَانِ فِي الْحَقْلِ فَيُؤْخَذُ الْوَاحِدُ وَيُتْرَكُ الآخَرُ».
\par 37 فَقَالُوا لَهُ: «أَيْنَ يَا رَبُّ؟» فَقَالَ لَهُمْ: «حَيْثُ تَكُونُ الْجُثَّةُ هُنَاكَ تَجْتَمِعُ النُّسُورُ».

\chapter{18}

\par 1 وَقَالَ لَهُمْ أَيْضاً مَثَلاً فِي أَنَّهُ يَنْبَغِي أَنْ يُصَلَّى كُلَّ حِينٍ وَلاَ يُمَلَّ:
\par 2 «كَانَ فِي مَدِينَةٍ قَاضٍ لاَ يَخَافُ اللهَ وَلاَ يَهَابُ إِنْسَاناً.
\par 3 وَكَانَ فِي تِلْكَ الْمَدِينَةِ أَرْمَلَةٌ. وَكَانَتْ تَأْتِي إِلَيْهِ قَائِلَةً: أَنْصِفْنِي مِنْ خَصْمِي.
\par 4 وَكَانَ لاَ يَشَاءُ إِلَى زَمَانٍ. وَلَكِنْ بَعْدَ ذَلِكَ قَالَ فِي نَفْسِهِ: وَإِنْ كُنْتُ لاَ أَخَافُ اللهَ وَلاَ أَهَابُ إِنْسَاناً
\par 5 فَإِنِّي لأَجْلِ أَنَّ هَذِهِ الأَرْمَلَةَ تُزْعِجُنِي أُنْصِفُهَا لِئَلاَّ تَأْتِيَ دَائِماً فَتَقْمَعَنِي».
\par 6 وَقَالَ الرَّبُّ: «اسْمَعُوا مَا يَقُولُ قَاضِي الظُّلْمِ.
\par 7 أَفَلاَ يُنْصِفُ اللهُ مُخْتَارِيهِ الصَّارِخِينَ إِلَيْهِ نَهَاراً وَلَيْلاً وَهُوَ مُتَمَهِّلٌ عَلَيْهِمْ؟
\par 8 أَقُولُ لَكُمْ إِنَّهُ يُنْصِفُهُمْ سَرِيعاً! وَلَكِنْ مَتَى جَاءَ ابْنُ الإِنْسَانِ أَلَعَلَّهُ يَجِدُ الإِيمَانَ عَلَى الأَرْضِ؟».
\par 9 وَقَالَ لِقَوْمٍ وَاثِقِينَ بِأَنْفُسِهِمْ أَنَّهُمْ أَبْرَارٌ وَيَحْتَقِرُونَ الآخَرِينَ هَذَا الْمَثَلَ:
\par 10 «إِنْسَانَانِ صَعِدَا إِلَى الْهَيْكَلِ لِيُصَلِّيَا وَاحِدٌ فَرِّيسِيٌّ وَالآخَرُ عَشَّارٌ.
\par 11 أَمَّا الْفَرِّيسِيُّ فَوَقَفَ يُصَلِّي فِي نَفْسِهِ هَكَذَا: اَللَّهُمَّ أَنَا أَشْكُرُكَ أَنِّي لَسْتُ مِثْلَ بَاقِي النَّاسِ الْخَاطِفِينَ الظَّالِمِينَ الزُّنَاةِ وَلاَ مِثْلَ هَذَا الْعَشَّارِ.
\par 12 أَصُومُ مَرَّتَيْنِ فِي الأُسْبُوعِ وَأُعَشِّرُ كُلَّ مَا أَقْتَنِيهِ.
\par 13 وَأَمَّا الْعَشَّارُ فَوَقَفَ مِنْ بَعِيدٍ لاَ يَشَاءُ أَنْ يَرْفَعَ عَيْنَيْهِ نَحْوَ السَّمَاءِ بَلْ قَرَعَ عَلَى صَدْرِهِ قَائِلاً: اللهُمَّ ارْحَمْنِي أَنَا الْخَاطِئَ.
\par 14 أَقُولُ لَكُمْ إِنَّ هَذَا نَزَلَ إِلَى بَيْتِهِ مُبَرَّراً دُونَ ذَاكَ لأَنَّ كُلَّ مَنْ يَرْفَعُ نَفْسَهُ يَتَّضِعُ وَمَنْ يَضَعُ نَفْسَهُ يَرْتَفِعُ».
\par 15 فَقَدَّمُوا إِلَيْهِ الأَطْفَالَ أَيْضاً لِيَلْمِسَهُمْ فَلَمَّا رَآهُمُ التَّلاَمِيذُ انْتَهَرُوهُمْ.
\par 16 أَمَّا يَسُوعُ فَدَعَاهُمْ وَقَالَ: «دَعُوا الأَوْلاَدَ يَأْتُونَ إِلَيَّ وَلاَ تَمْنَعُوهُمْ لأَنَّ لِمِثْلِ هَؤُلاَءِ مَلَكُوتَ اللهِ.
\par 17 اَلْحَقَّ أَقُولُ لَكُمْ: مَنْ لاَ يَقْبَلُ مَلَكُوتَ اللهِ مِثْلَ وَلَدٍ فَلَنْ يَدْخُلَهُ».
\par 18 وَسَأَلَهُ رَئِيسٌ: «أَيُّهَا الْمُعَلِّمُ الصَّالِحُ مَاذَا أَعْمَلُ لأَرِثَ الْحَيَاةَ الأَبَدِيَّةَ؟»
\par 19 فَقَالَ لَهُ يَسُوعُ: «لِمَاذَا تَدْعُونِي صَالِحاً؟ لَيْسَ أَحَدٌ صَالِحاً إِلاَّ وَاحِدٌ وَهُوَ اللهُ.
\par 20 أَنْتَ تَعْرِفُ الْوَصَايَا: لاَ تَزْنِ. لاَ تَقْتُلْ. لاَ تَسْرِقْ. لاَ تَشْهَدْ بِالزُّورِ. أَكْرِمْ أَبَاكَ وَأُمَّكَ».
\par 21 فَقَالَ: «هَذِهِ كُلُّهَا حَفِظْتُهَا مُنْذُ حَدَاثَتِي».
\par 22 فَلَمَّا سَمِعَ يَسُوعُ ذَلِكَ قَالَ لَهُ: «يُعْوِزُكَ أَيْضاً شَيْءٌ. بِعْ كُلَّ مَا لَكَ وَوَزِّعْ عَلَى الْفُقَرَاءِ فَيَكُونَ لَكَ كَنْزٌ فِي السَّمَاءِ وَتَعَالَ اتْبَعْنِي».
\par 23 فَلَمَّا سَمِعَ ذَلِكَ حَزِنَ لأَنَّهُ كَانَ غَنِيّاً جِدّاً.
\par 24 فَلَمَّا رَآهُ يَسُوعُ قَدْ حَزِنَ قَالَ: «مَا أَعْسَرَ دُخُولَ ذَوِي الأَمْوَالِ إِلَى مَلَكُوتِ اللهِ!
\par 25 لأَنَّ دُخُولَ جَمَلٍ مِنْ ثَقْبِ إِبْرَةٍ أَيْسَرُ مِنْ أَنْ يَدْخُلَ غَنِيٌّ إِلَى مَلَكُوتِ اللهِ!».
\par 26 فَقَالَ الَّذِينَ سَمِعُوا: «فَمَنْ يَسْتَطِيعُ أَنْ يَخْلُصَ؟»
\par 27 فَقَالَ: «غَيْرُ الْمُسْتَطَاعِ عِنْدَ النَّاسِ مُسْتَطَاعٌ عِنْدَ اللهِ».
\par 28 فَقَالَ بُطْرُسُ: «هَا نَحْنُ قَدْ تَرَكْنَا كُلَّ شَيْءٍ وَتَبِعْنَاكَ».
\par 29 فَقَالَ لَهُمُ: «الْحَقَّ أَقُولُ لَكُمْ: إِنْ لَيْسَ أَحَدٌ تَرَكَ بَيْتاً أَوْ وَالِدَيْنِ أَوْ إِخْوَةً أَوِ امْرَأَةً أَوْ أَوْلاَداً مِنْ أَجْلِ مَلَكُوتِ اللهِ
\par 30 إِلاَّ وَيَأْخُذُ فِي هَذَا الزَّمَانِ أَضْعَافاً كَثِيرَةً وَفِي الدَّهْرِ الآتِي الْحَيَاةَ الأَبَدِيَّةَ».
\par 31 وَأَخَذَ الاِثْنَيْ عَشَرَ وَقَالَ لَهُمْ: «هَا نَحْنُ صَاعِدُونَ إِلَى أُورُشَلِيمَ وَسَيَتِمُّ كُلُّ مَا هُوَ مَكْتُوبٌ بِالأَنْبِيَاءِ عَنِ ابْنِ الإِنْسَانِ
\par 32 لأَنَّهُ يُسَلَّمُ إِلَى الأُمَمِ وَيُسْتَهْزَأُ بِهِ وَيُشْتَمُ وَيُتْفَلُ عَلَيْهِ
\par 33 وَيَجْلِدُونَهُ وَيَقْتُلُونَهُ وَفِي الْيَوْمِ الثَّالِثِ يَقُومُ».
\par 34 وَأَمَّا هُمْ فَلَمْ يَفْهَمُوا مِنْ ذَلِكَ شَيْئاً وَكَانَ هَذَا الأَمْرُ مُخْفىً عَنْهُمْ وَلَمْ يَعْلَمُوا مَا قِيلَ.
\par 35 وَلَمَّا اقْتَرَبَ مِنْ أَرِيحَا كَانَ أَعْمَى جَالِساً علَى الطَّرِيقِ يَسْتَعْطِي.
\par 36 فَلَمَّا سَمِعَ الْجَمْعَ مُجْتَازاً سَأَلَ: «مَا عَسَى أَنْ يَكُونَ هَذَا؟»
\par 37 فَأَخْبَرُوهُ أَنَّ يَسُوعَ النَّاصِرِيَّ مُجْتَازٌ.
\par 38 فَصَرَخَ: «يَا يَسُوعُ ابْنَ دَاوُدَ ارْحَمْنِي!».
\par 39 فَانْتَهَرَهُ الْمُتَقَدِّمُونَ لِيَسْكُتَ أَمَّا هُوَ فَصَرَخَ أَكْثَرَ كَثِيراً: «يَا ابْنَ دَاوُدَ ارْحَمْنِي».
\par 40 فَوَقَفَ يَسُوعُ وَأَمَرَ أَنْ يُقَدَّمَ إِلَيْهِ. وَلَمَّا اقْتَرَبَ سَأَلَهُ:
\par 41 «مَاذَا تُرِيدُ أَنْ أَفْعَلَ بِكَ؟» فَقَالَ: «يَا سَيِّدُ أَنْ أُبْصِرَ».
\par 42 فَقَالَ لَهُ يَسُوعُ: «أَبْصِرْ. إِيمَانُكَ قَدْ شَفَاكَ».
\par 43 وَفِي الْحَالِ أَبْصَرَ وَتَبِعَهُ وَهُوَ يُمَجِّدُ اللهَ. وَجَمِيعُ الشَّعْبِ إِذْ رَأَوْا سَبَّحُوا اللهَ.

\chapter{19}

\par 1 ثُمَّ دَخَلَ وَاجْتَازَ فِي أَرِيحَا.
\par 2 وَإِذَا رَجُلٌ اسْمُهُ زَكَّا وَهُوَ رَئِيسٌ لِلْعَشَّارِينَ وَكَانَ غَنِيّاً
\par 3 وَطَلَبَ أَنْ يَرَى يَسُوعَ مَنْ هُوَ وَلَمْ يَقْدِرْ مِنَ الْجَمْعِ لأَنَّهُ كَانَ قَصِيرَ الْقَامَةِ.
\par 4 فَرَكَضَ مُتَقَدِّماً وَصَعِدَ إِلَى جُمَّيْزَةٍ لِكَيْ يَرَاهُ لأَنَّهُ كَانَ مُزْمِعاً أَنْ يَمُرَّ مِنْ هُنَاكَ.
\par 5 فَلَمَّا جَاءَ يَسُوعُ إِلَى الْمَكَانِ نَظَرَ إِلَى فَوْقُ فَرَآهُ وَقَالَ لَهُ: «يَا زَكَّا أَسْرِعْ وَانْزِلْ لأَنَّهُ يَنْبَغِي أَنْ أَمْكُثَ الْيَوْمَ فِي بَيْتِكَ».
\par 6 فَأَسْرَعَ وَنَزَلَ وَقَبِلَهُ فَرِحاً.
\par 7 فَلَمَّا رَأَى الْجَمِيعُ ذَلِكَ تَذَمَّرُوا قَائِلِينَ: «إِنَّهُ دَخَلَ لِيَبِيتَ عِنْدَ رَجُلٍ خَاطِئٍ».
\par 8 فَوَقَفَ زَكَّا وَقَالَ لِلرَّبِّ: «هَا أَنَا يَا رَبُّ أُعْطِي نِصْفَ أَمْوَالِي لِلْمَسَاكِينِ وَإِنْ كُنْتُ قَدْ وَشَيْتُ بِأَحَدٍ أَرُدُّ أَرْبَعَةَ أَضْعَافٍ».
\par 9 فَقَالَ لَهُ يَسُوعُ: «الْيَوْمَ حَصَلَ خَلاَصٌ لِهَذَا الْبَيْتِ إِذْ هُوَ أَيْضاً ابْنُ إِبْرَاهِيمَ
\par 10 لأَنَّ ابْنَ الإِنْسَانِ قَدْ جَاءَ لِكَيْ يَطْلُبَ وَيُخَلِّصَ مَا قَدْ هَلَكَ».
\par 11 وَإِذْ كَانُوا يَسْمَعُونَ هَذَا عَادَ فَقَالَ مَثَلاً لأَنَّهُ كَانَ قَرِيباً مِنْ أُورُشَلِيمَ وَكَانُوا يَظُنُّونَ أَنَّ مَلَكُوتَ اللهِ عَتِيدٌ أَنْ يَظْهَرَ فِي الْحَالِ.
\par 12 فَقَالَ: «إِنْسَانٌ شَرِيفُ الْجِنْسِ ذَهَبَ إِلَى كُورَةٍ بَعِيدَةٍ لِيَأْخُذَ لِنَفْسِهِ مُلْكاً وَيَرْجِعَ.
\par 13 فَدَعَا عَشَرَةَ عَبِيدٍ لَهُ وَأَعْطَاهُمْ عَشَرَةَ أَمْنَاءٍ وَقَالَ لَهُمْ: تَاجِرُوا حَتَّى آتِيَ.
\par 14 وَأَمَّا أَهْلُ مَدِينَتِهِ فَكَانُوا يُبْغِضُونَهُ فَأَرْسَلُوا وَرَاءَهُ سَفَارَةً قَائِلِينَ: لاَ نُرِيدُ أَنَّ هَذَا يَمْلِكُ عَلَيْنَا.
\par 15 وَلَمَّا رَجَعَ بَعْدَمَا أَخَذَ الْمُلْكَ أَمَرَ أَنْ يُدْعَى إِلَيْهِ أُولَئِكَ الْعَبِيدُ الَّذِينَ أَعْطَاهُمُ الْفِضَّةَ لِيَعْرِفَ بِمَا تَاجَرَ كُلُّ وَاحِدٍ.
\par 16 فَجَاءَ الأَوَّلُ قَائِلاً: يَا سَيِّدُ مَنَاكَ رَبِحَ عَشَرَةَ أَمْنَاءٍ.
\par 17 فَقَالَ لَهُ: نِعِمَّا أَيُّهَا الْعَبْدُ الصَّالِحُ لأَنَّكَ كُنْتَ أَمِيناً فِي الْقَلِيلِ فَلْيَكُنْ لَكَ سُلْطَانٌ عَلَى عَشْرِ مُدُنٍ.
\par 18 ثُمَّ جَاءَ الثَّانِي قَائِلاً: يَا سَيِّدُ مَنَاكَ عَمِلَ خَمْسَةَ أَمْنَاءٍ.
\par 19 فَقَالَ لِهَذَا أَيْضاً: وَكُنْ أَنْتَ عَلَى خَمْسِ مُدُنٍ.
\par 20 ثُمَّ جَاءَ آخَرُ قَائِلاً: يَا سَيِّدُ هُوَذَا مَنَاكَ الَّذِي كَانَ عِنْدِي مَوْضُوعاً فِي مِنْدِيلٍ
\par 21 لأَنِّي كُنْتُ أَخَافُ مِنْكَ إِذْ أَنْتَ إِنْسَانٌ صَارِمٌ تَأْخُذُ مَا لَمْ تَضَعْ وَتَحْصُدُ مَا لَمْ تَزْرَعْ.
\par 22 فَقَالَ لَهُ: مِنْ فَمِكَ أَدِينُكَ أَيُّهَا الْعَبْدُ الشِّرِّيرُ. عَرَفْتَ أَنِّي إِنْسَانٌ صَارِمٌ آخُذُ مَا لَمْ أَضَعْ وَأَحْصُدُ مَا لَمْ أَزْرَعْ
\par 23 فَلِمَاذَا لَمْ تَضَعْ فِضَّتِي عَلَى مَائِدَةِ الصَّيَارِفَةِ فَكُنْتُ مَتَى جِئْتُ أَسْتَوْفِيهَا مَعَ رِباً؟
\par 24 ثُمَّ قَالَ لِلْحَاضِرِينَ: خُذُوا مِنْهُ الْمَنَا وَأَعْطُوهُ لِلَّذِي عِنْدَهُ الْعَشَرَةُ الأَمْنَاءُ.
\par 25 فَقَالُوا لَهُ: يَا سَيِّدُ عِنْدَهُ عَشَرَةُ أَمْنَاءٍ.
\par 26 لأَنِّي أَقُولُ لَكُمْ: إِنَّ كُلَّ مَنْ لَهُ يُعْطَى وَمَنْ لَيْسَ لَهُ فَالَّذِي عِنْدَهُ يُؤْخَذُ مِنْهُ.
\par 27 أَمَّا أَعْدَائِي أُولَئِكَ الَّذِينَ لَمْ يُرِيدُوا أَنْ أَمْلِكَ عَلَيْهِمْ فَأْتُوا بِهِمْ إِلَى هُنَا وَاذْبَحُوهُمْ قُدَّامِي».
\par 28 وَلَمَّا قَالَ هَذَا تَقَدَّمَ صَاعِداً إِلَى أُورُشَلِيمَ.
\par 29 وَإِذْ قَرُبَ مِنْ بَيْتِ فَاجِي وَبَيْتِ عَنْيَا عِنْدَ الْجَبَلِ الَّذِي يُدْعَى جَبَلَ الزَّيْتُونِ أَرْسَلَ اثْنَيْنِ مِنْ تَلاَمِيذِهِ
\par 30 قَائِلاً: «اِذْهَبَا إِلَى الْقَرْيَةِ الَّتِي أَمَامَكُمَا وَحِينَ تَدْخُلاَنِهَا تَجِدَانِ جَحْشاً مَرْبُوطاً لَمْ يَجْلِسْ عَلَيْهِ أَحَدٌ مِنَ النَّاسِ قَطُّ. فَحُلاَّهُ وَأْتِيَا بِهِ.
\par 31 وَإِنْ سَأَلَكُمَا أَحَدٌ: لِمَاذَا تَحُلاَّنِهِ؟ فَقُولاَ لَهُ: إِنَّ الرَّبَّ مُحْتَاجٌ إِلَيْهِ».
\par 32 فَمَضَى الْمُرْسَلاَنِ وَوَجَدَا كَمَا قَالَ لَهُمَا.
\par 33 وَفِيمَا هُمَا يَحُلاَّنِ الْجَحْشَ قَالَ لَهُمَا أَصْحَابُهُ: «لِمَاذَا تَحُلاَّنِ الْجَحْشَ؟»
\par 34 فَقَالاَ: «الرَّبُّ مُحْتَاجٌ إِلَيْهِ».
\par 35 وَأَتَيَا بِهِ إِلَى يَسُوعَ وَطَرَحَا ثِيَابَهُمَا عَلَى الْجَحْشِ وَأَرْكَبَا يَسُوعَ.
\par 36 وَفِيمَا هُوَ سَائِرٌ فَرَشُوا ثِيَابَهُمْ فِي الطَّرِيقِ.
\par 37 وَلَمَّا قَرُبَ عِنْدَ مُنْحَدَرِ جَبَلِ الزَّيْتُونِ ابْتَدَأَ كُلُّ جُمْهُورِ التَّلاَمِيذِ يَفْرَحُونَ وَيُسَبِّحُونَ اللهَ بِصَوْتٍ عَظِيمٍ لأَجْلِ جَمِيعِ الْقُوَّاتِ الَّتِي نَظَرُوا
\par 38 قَائِلِينَ: «مُبَارَكٌ الْمَلِكُ الآتِي بِاسْمِ الرَّبِّ! سَلاَمٌ فِي السَّمَاءِ وَمَجْدٌ فِي الأَعَالِي!».
\par 39 وَأَمَّا بَعْضُ الْفَرِّيسِيِّينَ مِنَ الْجَمْعِ فَقَالُوا لَهُ: «يَا مُعَلِّمُ انْتَهِرْ تَلاَمِيذَكَ».
\par 40 فَأَجَابَ: «أَقُولُ لَكُمْ: إِنَّهُ إِنْ سَكَتَ هَؤُلاَءِ فَالْحِجَارَةُ تَصْرُخُ!».
\par 41 وَفِيمَا هُوَ يَقْتَرِبُ نَظَرَ إِلَى الْمَدِينَةِ وَبَكَى عَلَيْهَا
\par 42 قَائِلاً: «إِنَّكِ لَوْ عَلِمْتِ أَنْتِ أَيْضاً حَتَّى فِي يَوْمِكِ هَذَا مَا هُوَ لِسَلاَمِكِ. وَلَكِنِ الآنَ قَدْ أُخْفِيَ عَنْ عَيْنَيْكِ.
\par 43 فَإِنَّهُ سَتَأْتِي أَيَّامٌ وَيُحِيطُ بِكِ أَعْدَاؤُكِ بِمِتْرَسَةٍ وَيُحْدِقُونَ بِكِ وَيُحَاصِرُونَكِ مِنْ كُلِّ جِهَةٍ
\par 44 وَيَهْدِمُونَكِ وَبَنِيكِ فِيكِ وَلاَ يَتْرُكُونَ فِيكِ حَجَراً عَلَى حَجَرٍ لأَنَّكِ لَمْ تَعْرِفِي زَمَانَ افْتِقَادِكِ».
\par 45 وَلَمَّا دَخَلَ الْهَيْكَلَ ابْتَدَأَ يُخْرِجُ الَّذِينَ كَانُوا يَبِيعُونَ وَيَشْتَرُونَ فِيهِ
\par 46 قَائِلاً لَهُمْ: «مَكْتُوبٌ أَنَّ بَيْتِي بَيْتُ الصَّلاَةِ. وَأَنْتُمْ جَعَلْتُمُوهُ مَغَارَةَ لُصُوصٍ».
\par 47 وَكَانَ يُعَلِّمُ كُلَّ يَوْمٍ فِي الْهَيْكَلِ وَكَانَ رُؤَسَاءُ الْكَهَنَةِ وَالْكَتَبَةُ مَعَ وُجُوهِ الشَّعْبِ يَطْلُبُونَ أَنْ يُهْلِكُوهُ
\par 48 وَلَمْ يَجِدُوا مَا يَفْعَلُونَ لأَنَّ الشَّعْبَ كُلَّهُ كَانَ مُتَعَلِّقاً بِهِ يَسْمَعُ مِنْهُ.

\chapter{20}

\par 1 وَفِي أَحَدِ تِلْكَ الأَيَّامِ إِذْ كَانَ يُعَلِّمُ الشَّعْبَ فِي الْهَيْكَلِ وَيُبَشِّرُ وَقَفَ رُؤَسَاءُ الْكَهَنَةِ وَالْكَتَبَةُ مَعَ الشُّيُوخِ
\par 2 وَقَالُوا لَهُ: «قُلْ لَنَا بِأَيِّ سُلْطَانٍ تَفْعَلُ هَذَا أَوْ مَنْ هُوَ الَّذِي أَعْطَاكَ هَذَا السُّلْطَانَ؟»
\par 3 فَأَجَابَ: «وَأَنَا أَيْضاً أَسْأَلُكُمْ كَلِمَةً وَاحِدَةً فَقُولُوا لِي:
\par 4 مَعْمُودِيَّةُ يُوحَنَّا مِنَ السَّمَاءِ كَانَتْ أَمْ مِنَ النَّاسِ؟»
\par 5 فَتَآمَرُوا فِيمَا بَيْنَهُمْ قَائِلِينَ: «إِنْ قُلْنَا مِنَ السَّمَاءِ يَقُولُ: فَلِمَاذَا لَمْ تُؤْمِنُوا بِهِ؟
\par 6 وَإِنْ قُلْنَا: مِنَ النَّاسِ فَجَمِيعُ الشَّعْبِ يَرْجُمُونَنَا لأَنَّهُمْ وَاثِقُونَ بِأَنَّ يُوحَنَّا نَبِيٌّ».
\par 7 فَأَجَابُوا أَنَّهُمْ لاَ يَعْلَمُونَ مِنْ أَيْنَ.
\par 8 فَقَالَ لَهُمْ يَسُوعُ: «وَلاَ أَنَا أَقُولُ لَكُمْ بِأَيِّ سُلْطَانٍ أَفْعَلُ هَذَا».
\par 9 وَابْتَدَأَ يَقُولُ لِلشَّعْبِ هَذَا الْمَثَلَ: «إِنْسَانٌ غَرَسَ كَرْماً وَسَلَّمَهُ إِلَى كَرَّامِينَ وَسَافَرَ زَمَاناً طَوِيلاً.
\par 10 وَفِي الْوَقْتِ أَرْسَلَ إِلَى الْكَرَّامِينَ عَبْداً لِكَيْ يُعْطُوهُ مِنْ ثَمَرِ الْكَرْمِ فَجَلَدَهُ الْكَرَّامُونَ وَأَرْسَلُوهُ فَارِغاً.
\par 11 فَعَادَ وَأَرْسَلَ عَبْداً آخَرَ. فَجَلَدُوا ذَلِكَ أَيْضاً وَأَهَانُوهُ وَأَرْسَلُوهُ فَارِغاً.
\par 12 ثُمَّ عَادَ فَأَرْسَلَ ثَالِثاً. فَجَرَّحُوا هَذَا أَيْضاً وَأَخْرَجُوهُ.
\par 13 فَقَالَ صَاحِبُ الْكَرْمِ: مَاذَا أَفْعَلُ؟ أُرْسِلُ ابْنِي الْحَبِيبَ. لَعَلَّهُمْ إِذَا رَأَوْهُ يَهَابُونَ!
\par 14 فَلَمَّا رَآهُ الْكَرَّامُونَ تَآمَرُوا فِيمَا بَيْنَهُمْ قَائِلِينَ: هَذَا هُوَ الْوَارِثُ. هَلُمُّوا نَقْتُلْهُ لِكَيْ يَصِيرَ لَنَا الْمِيرَاثُ.
\par 15 فَأَخْرَجُوهُ خَارِجَ الْكَرْمِ وَقَتَلُوهُ. فَمَاذَا يَفْعَلُ بِهِمْ صَاحِبُ الْكَرْمِ؟
\par 16 يَأْتِي وَيُهْلِكُ هَؤُلاَءِ الْكَرَّامِينَ وَيُعْطِي الْكَرْمَ لِآخَرِينَ». فَلَمَّا سَمِعُوا قَالُوا: «حَاشَا!»
\par 17 فَنَظَرَ إِلَيْهِمْ وَقَالَ: «إِذاً مَا هُوَ هَذَا الْمَكْتُوبُ: الْحَجَرُ الَّذِي رَفَضَهُ الْبَنَّاؤُونَ هُوَ قَدْ صَارَ رَأْسَ الزَّاوِيَةِ.
\par 18 كُلُّ مَنْ يَسْقُطُ عَلَى ذَلِكَ الْحَجَرِ يَتَرَضَّضُ وَمَنْ سَقَطَ هُوَ عَلَيْهِ يَسْحَقُهُ؟»
\par 19 فَطَلَبَ رُؤَسَاءُ الْكَهَنَةِ وَالْكَتَبَةُ أَنْ يُلْقُوا الأَيَادِيَ عَلَيْهِ فِي تِلْكَ السَّاعَةِ وَلَكِنَّهُمْ خَافُوا الشَّعْبَ لأَنَّهُمْ عَرَفُوا أَنَّهُ قَالَ هَذَا الْمَثَلَ عَلَيْهِمْ.
\par 20 فَرَاقَبُوهُ وَأَرْسَلُوا جَوَاسِيسَ يَتَرَاءَوْنَ أَنَّهُمْ أَبْرَارٌ لِكَيْ يُمْسِكُوهُ بِكَلِمَةٍ حَتَّى يُسَلِّمُوهُ إِلَى حُكْمِ الْوَالِي وَسُلْطَانِهِ.
\par 21 فَسَأَلُوهُ: «يَا مُعَلِّمُ نَعْلَمُ أَنَّكَ بِالاِسْتِقَامَةِ تَتَكَلَّمُ وَتُعَلِّمُ وَلاَ تَقْبَلُ الْوُجُوهَ بَلْ بِالْحَقِّ تُعَلِّمُ طَرِيقَ اللهِ.
\par 22 أَيَجُوزُ لَنَا أَنْ نُعْطِيَ جِزْيَةً لِقَيْصَرَ أَمْ لاَ؟»
\par 23 فَشَعَرَ بِمَكْرِهِمْ وَقَالَ لَهُمْ: «لِمَاذَا تُجَرِّبُونَنِي؟
\par 24 أَرُونِي دِينَاراً. لِمَنِ الصُّورَةُ وَالْكِتَابَةُ؟» فَأَجَابُوا: «لِقَيْصَرَ».
\par 25 فَقَالَ لَهُمْ: «أَعْطُوا إِذاً مَا لِقَيْصَرَ لِقَيْصَرَ وَمَا لِلَّهِ لِلَّهِ».
\par 26 فَلَمْ يَقْدِرُوا أَنْ يُمْسِكُوهُ بِكَلِمَةٍ قُدَّامَ الشَّعْبِ وَتَعَجَّبُوا مِنْ جَوَابِهِ وَسَكَتُوا.
\par 27 وَحَضَرَ قَوْمٌ مِنَ الصَّدُّوقِيِّينَ الَّذِينَ يُقَاوِمُونَ أَمْرَ الْقِيَامَةِ وَسَأَلُوهُ:
\par 28 «يَا مُعَلِّمُ كَتَبَ لَنَا مُوسَى: إِنْ مَاتَ لأَحَدٍ أَخٌ وَلَهُ امْرَأَةٌ وَمَاتَ بِغَيْرِ وَلَدٍ يَأْخُذُ أَخُوهُ الْمَرْأَةَ وَيُقِيمُ نَسْلاً لأَخِيهِ.
\par 29 فَكَانَ سَبْعَةُ إِخْوَةٍ. وَأَخَذَ الأَوَّلُ امْرَأَةً وَمَاتَ بِغَيْرِ وَلَدٍ
\par 30 فَأَخَذَ الثَّانِي الْمَرْأَةَ وَمَاتَ بِغَيْرِ وَلَدٍ
\par 31 ثُمَّ أَخَذَهَا الثَّالِثُ وَهَكَذَا السَّبْعَةُ. وَلَمْ يَتْرُكُوا وَلَداً وَمَاتُوا.
\par 32 وَآخِرَ الْكُلِّ مَاتَتِ الْمَرْأَةُ أَيْضاً.
\par 33 فَفِي الْقِيَامَةِ لِمَنْ مِنْهُمْ تَكُونُ زَوْجَةً؟ لأَنَّهَا كَانَتْ زَوْجَةً لِلسَّبْعَةِ!»
\par 34 فَأَجَابَ يَسُوعُ: «أَبْنَاءُ هَذَا الدَّهْرِ يُزَوِّجُونَ وَيُزَوَّجُونَ
\par 35 وَلَكِنَّ الَّذِينَ حُسِبُوا أَهْلاً لِلْحُصُولِ عَلَى ذَلِكَ الدَّهْرِ وَالْقِيَامَةِ مِنَ الأَمْوَاتِ لاَ يُزَوِّجُونَ وَلاَ يُزَوَّجُونَ
\par 36 إِذْ لاَ يَسْتَطِيعُونَ أَنْ يَمُوتُوا أَيْضاً لأَنَّهُمْ مِثْلُ الْمَلاَئِكَةِ وَهُمْ أَبْنَاءُ اللهِ إِذْ هُمْ أَبْنَاءُ الْقِيَامَةِ.
\par 37 وَأَمَّا أَنَّ الْمَوْتَى يَقُومُونَ فَقَدْ دَلَّ عَلَيْهِ مُوسَى أَيْضاً فِي أَمْرِ الْعُلَّيْقَةِ كَمَا يَقُولُ: اَلرَّبُّ إِلَهُ إِبْرَاهِيمَ وَإِلَهُ إِسْحَاقَ وَإِلَهُ يَعْقُوبَ.
\par 38 وَلَيْسَ هُوَ إِلَهَ أَمْوَاتٍ بَلْ إِلَهُ أَحْيَاءٍ لأَنَّ الْجَمِيعَ عِنْدَهُ أَحْيَاءٌ».
\par 39 فَقَالَ قَوْمٌ مِنَ الْكَتَبَةِ: «يَا مُعَلِّمُ حَسَناً قُلْتَ!».
\par 40 وَلَمْ يَتَجَاسَرُوا أَيْضاً أَنْ يَسْأَلُوهُ عَنْ شَيْءٍ.
\par 41 وَقَالَ لَهُمْ: «كَيْفَ يَقُولُونَ إِنَّ الْمَسِيحَ ابْنُ دَاوُدَ
\par 42 وَدَاوُدُ نَفْسُهُ يَقُولُ فِي كِتَابِ الْمَزَامِيرِ: قَالَ الرَّبُّ لِرَبِّي: اجْلِسْ عَنْ يَمِينِي
\par 43 حَتَّى أَضَعَ أَعْدَاءَكَ مَوْطِئاً لِقَدَمَيْكَ.
\par 44 فَإِذاً دَاوُدُ يَدْعُوهُ رَبّاً. فَكَيْفَ يَكُونُ ابْنَهُ؟».
\par 45 وَفِيمَا كَانَ جَمِيعُ الشَّعْبِ يَسْمَعُونَ قَالَ لِتَلاَمِيذِهِ:
\par 46 «احْذَرُوا مِنَ الْكَتَبَةِ الَّذِينَ يَرْغَبُونَ الْمَشْيَ بِالطَّيَالِسَةِ وَيُحِبُّونَ التَّحِيَّاتِ فِي الأَسْوَاقِ وَالْمَجَالِسَ الأُولَى فِي الْمَجَامِعِ وَالْمُتَّكَآتِ الأُولَى فِي الْوَلاَئِمِ.
\par 47 اَلَّذِينَ يَأْكُلُونَ بُيُوتَ الأَرَامِلِ وَلِعِلَّةٍ يُطِيلُونَ الصَّلَوَاتِ. هَؤُلاَءِ يَأْخُذُونَ دَيْنُونَةً أَعْظَمَ!».

\chapter{21}

\par 1 وَتَطَلَّعَ فَرَأَى الأَغْنِيَاءَ يُلْقُونَ قَرَابِينَهُمْ فِي الْخِزَانَةِ
\par 2 وَرَأَى أَيْضاً أَرْمَلَةً مِسْكِينَةً أَلْقَتْ هُنَاكَ فَلْسَيْنِ.
\par 3 فَقَالَ: «بِالْحَقِّ أَقُولُ لَكُمْ إِنَّ هَذِهِ الأَرْمَلَةَ الْفَقِيرَةَ أَلْقَتْ أَكْثَرَ مِنَ الْجَمِيعِ
\par 4 لأَنَّ هَؤُلاَءِ مِنْ فَضْلَتِهِمْ أَلْقَوْا فِي قَرَابِينِ اللهِ وَأَمَّا هَذِهِ فَمِنْ إِعْوَازِهَا أَلْقَتْ كُلَّ الْمَعِيشَةِ الَّتِي لَهَا».
\par 5 وَإِذْ كَانَ قَوْمٌ يَقُولُونَ عَنِ الْهَيْكَلِ إِنَّهُ مُزَيَّنٌ بِحِجَارَةٍ حَسَنَةٍ وَتُحَفٍ قَالَ:
\par 6 «هَذِهِ الَّتِي تَرَوْنَهَا سَتَأْتِي أَيَّامٌ لاَ يُتْرَكُ فِيهَا حَجَرٌ عَلَى حَجَرٍ لاَ يُنْقَضُ».
\par 7 فَسَأَلُوهُ: «يَا مُعَلِّمُ مَتَى يَكُونُ هَذَا ومَا هِيَ الْعَلاَمَةُ عِنْدَمَا يَصِيرُ هَذَا؟»
\par 8 فَقَالَ: «انْظُرُوا! لاَ تَضِلُّوا. فَإِنَّ كَثِيرِينَ سَيَأْتُونَ بِاسْمِي قَائِلِينَ: إِنِّي أَنَا هُوَ وَالزَّمَانُ قَدْ قَرُبَ. فَلاَ تَذْهَبُوا وَرَاءَهُمْ.
\par 9 فَإِذَا سَمِعْتُمْ بِحُرُوبٍ وَقَلاَقِلٍ فَلاَ تَجْزَعُوا لأَنَّهُ لاَ بُدَّ أَنْ يَكُونَ هَذَا أَوَّلاً وَلَكِنْ لاَ يَكُونُ الْمُنْتَهَى سَرِيعاً».
\par 10 ثُمَّ قَالَ لَهُمْ: «تَقُومُ أُمَّةٌ عَلَى أُمَّةٍ وَمَمْلَكَةٌ عَلَى مَمْلَكَةٍ
\par 11 وَتَكُونُ زَلاَزِلُ عَظِيمَةٌ فِي أَمَاكِنَ وَمَجَاعَاتٌ وَأَوْبِئَةٌ. وَتَكُونُ مَخَاوِفُ وَعَلاَمَاتٌ عَظِيمَةٌ مِنَ السَّمَاءِ.
\par 12 وَقَبْلَ هَذَا كُلِّهِ يُلْقُونَ أَيْدِيَهُمْ عَلَيْكُمْ وَيَطْرُدُونَكُمْ وَيُسَلِّمُونَكُمْ إِلَى مَجَامِعٍ وَسُجُونٍ وَتُسَاقُونَ أَمَامَ مُلُوكٍ وَوُلاَةٍ لأَجْلِ اسْمِي.
\par 13 فَيَؤُولُ ذَلِكَ لَكُمْ شَهَادَةً.
\par 14 فَضَعُوا فِي قُلُوبِكُمْ أَنْ لاَ تَهْتَمُّوا مِنْ قَبْلُ لِكَيْ تَحْتَجُّوا
\par 15 لأَنِّي أَنَا أُعْطِيكُمْ فَماً وَحِكْمَةً لاَ يَقْدِرُ جَمِيعُ مُعَانِدِيكُمْ أَنْ يُقَاوِمُوهَا أَوْ يُنَاقِضُوهَا.
\par 16 وَسَوْفَ تُسَلَّمُونَ مِنَ الْوَالِدِينَ وَالإِخْوَةِ وَالأَقْرِبَاءِ وَالأَصْدِقَاءِ وَيَقْتُلُونَ مِنْكُمْ.
\par 17 وَتَكُونُونَ مُبْغَضِينَ مِنَ الْجَمِيعِ مِنْ أَجْلِ اسْمِي.
\par 18 وَلَكِنَّ شَعْرَةً مِنْ رُؤُوسِكُمْ لاَ تَهْلِكُ.
\par 19 بِصَبْرِكُمُ اقْتَنُوا أَنْفُسَكُمْ.
\par 20 وَمَتَى رَأَيْتُمْ أُورُشَلِيمَ مُحَاطَةً بِجُيُوشٍ فَحِينَئِذٍ اعْلَمُوا أَنَّهُ قَدِ اقْتَرَبَ خَرَابُهَا.
\par 21 حِينَئِذٍ لِيَهْرُبِ الَّذِينَ فِي الْيَهُودِيَّةِ إِلَى الْجِبَالِ وَالَّذِينَ فِي وَسَطِهَا فَلْيَفِرُّوا خَارِجاً وَالَّذِينَ فِي الْكُوَرِ فَلاَ يَدْخُلُوهَا
\par 22 لأَنَّ هَذِهِ أَيَّامُ انْتِقَامٍ لِيَتِمَّ كُلُّ مَا هُوَ مَكْتُوبٌ.
\par 23 وَوَيْلٌ لِلْحَبَالَى وَالْمُرْضِعَاتِ فِي تِلْكَ الأَيَّامِ لأَنَّهُ يَكُونُ ضِيقٌ عَظِيمٌ عَلَى الأَرْضِ وَسُخْطٌ عَلَى هَذَا الشَّعْبِ.
\par 24 وَيَقَعُونَ بِالسَّيْفِ وَيُسْبَوْنَ إِلَى جَمِيعِ الأُمَمِ وَتَكُونُ أُورُشَلِيمُ مَدُوسَةً مِنَ الأُمَمِ حَتَّى تُكَمَّلَ أَزْمِنَةُ الأُمَمِ.
\par 25 «وَتَكُونُ عَلاَمَاتٌ فِي الشَّمْسِ وَالْقَمَرِ وَالنُّجُومِ وَعَلَى الأَرْضِ كَرْبُ أُمَمٍ بِحَيْرَةٍ. اَلْبَحْرُ وَالأَمْوَاجُ تَضِجُّ
\par 26 وَالنَّاسُ يُغْشَى عَلَيْهِمْ مِنْ خَوْفٍ وَانْتِظَارِ مَا يَأْتِي عَلَى الْمَسْكُونَةِ لأَنَّ قُوَّاتِ السَّمَاوَاتِ تَتَزَعْزَعُ.
\par 27 وَحِينَئِذٍ يُبْصِرُونَ ابْنَ الإِنْسَانِ آتِياً فِي سَحَابَةٍ بِقُوَّةٍ وَمَجْدٍ كَثِيرٍ.
\par 28 وَمَتَى ابْتَدَأَتْ هَذِهِ تَكُونُ فَانْتَصِبُوا وَارْفَعُوا رُؤُوسَكُمْ لأَنَّ نَجَاتَكُمْ تَقْتَرِبُ».
\par 29 وَقَالَ لَهُمْ مَثَلاً: «اُنْظُرُوا إِلَى شَجَرَةِ التِّينِ وَكُلِّ الأَشْجَارِ.
\par 30 مَتَى أَفْرَخَتْ تَنْظُرُونَ وَتَعْلَمُونَ مِنْ أَنْفُسِكُمْ أَنَّ الصَّيْفَ قَدْ قَرُبَ.
\par 31 هَكَذَا أَنْتُمْ أَيْضاً مَتَى رَأَيْتُمْ هَذِهِ الأَشْيَاءَ صَائِرَةً فَاعْلَمُوا أَنَّ مَلَكُوتَ اللهِ قَرِيبٌ.
\par 32 اَلْحَقَّ أَقُولُ لَكُمْ: إِنَّهُ لاَ يَمْضِي هَذَا الْجِيلُ حَتَّى يَكُونَ الْكُلُّ.
\par 33 اَلسَّمَاءُ وَالأَرْضُ تَزُولاَنِ وَلَكِنَّ كَلاَمِي لاَ يَزُولُ.
\par 34 فَاحْتَرِزُوا لأَنْفُسِكُمْ لِئَلاَّ تَثْقُلَ قُلُوبُكُمْ فِي خُمَارٍ وَسُكْرٍ وَهُمُومِ الْحَيَاةِ فَيُصَادِفَكُمْ ذَلِكَ الْيَوْمُ بَغْتَةً.
\par 35 لأَنَّهُ كَالْفَخِّ يَأْتِي عَلَى جَمِيعِ الْجَالِسِينَ عَلَى وَجْهِ كُلِّ الأَرْضِ.
\par 36 اِسْهَرُوا إِذاً وَتَضَرَّعُوا فِي كُلِّ حِينٍ لِكَيْ تُحْسَبُوا أَهْلاً لِلنَّجَاةِ مِنْ جَمِيعِ هَذَا الْمُزْمِعِ أَنْ يَكُونَ وَتَقِفُوا قُدَّامَ ابْنِ الإِنْسَانِ».
\par 37 وَكَانَ فِي النَّهَارِ يُعَلِّمُ فِي الْهَيْكَلِ وَفِي اللَّيْلِ يَخْرُجُ وَيَبِيتُ فِي الْجَبَلِ الَّذِي يُدْعَى جَبَلَ الزَّيْتُونِ.
\par 38 وَكَانَ كُلُّ الشَّعْبِ يُبَكِّرُونَ إِلَيْهِ فِي الْهَيْكَلِ لِيَسْمَعُوهُ.

\chapter{22}

\par 1 وَقَرُبَ عِيدُ الْفَطِيرِ الَّذِي يُقَالُ لَهُ الْفِصْحُ.
\par 2 وَكَانَ رُؤَسَاءُ الْكَهَنَةِ وَالْكَتَبَةُ يَطْلُبُونَ كَيْفَ يَقْتُلُونَهُ لأَنَّهُمْ خَافُوا الشَّعْبَ.
\par 3 فَدَخَلَ الشَّيْطَانُ فِي يَهُوذَا الَّذِي يُدْعَى الإِسْخَرْيُوطِيَّ وَهُوَ مِنْ جُمْلَةِ الاِثْنَيْ عَشَرَ.
\par 4 فَمَضَى وَتَكَلَّمَ مَعَ رُؤَسَاءِ الْكَهَنَةِ وَقُوَّادِ الْجُنْدِ كَيْفَ يُسَلِّمُهُ إِلَيْهِمْ.
\par 5 فَفَرِحُوا وَعَاهَدُوهُ أَنْ يُعْطُوهُ فِضَّةً.
\par 6 فَوَاعَدَهُمْ. وَكَانَ يَطْلُبُ فُرْصَةً لِيُسَلِّمَهُ إِلَيْهِمْ خِلْواً مِنْ جَمْعٍ.
\par 7 وَجَاءَ يَوْمُ الْفَطِيرِ الَّذِي كَانَ يَنْبَغِي أَنْ يُذْبَحَ فِيهِ الْفِصْحُ.
\par 8 فَأَرْسَلَ بُطْرُسَ وَيُوحَنَّا قَائِلاً: «اذْهَبَا وَأَعِدَّا لَنَا الْفِصْحَ لِنَأْكُلَ».
\par 9 فَقَالاَ لَهُ: «أَيْنَ تُرِيدُ أَنْ نُعِدَّ؟».
\par 10 فَقَالَ لَهُمَا: «إِذَا دَخَلْتُمَا الْمَدِينَةَ يَسْتَقْبِلُكُمَا إِنْسَانٌ حَامِلٌ جَرَّةَ مَاءٍ. اِتْبَعَاهُ إِلَى الْبَيْتِ حَيْثُ يَدْخُلُ
\par 11 وَقُولاَ لِرَبِّ الْبَيْتِ: يَقُولُ لَكَ الْمُعَلِّمُ: أَيْنَ الْمَنْزِلُ حَيْثُ آكُلُ الْفِصْحَ مَعَ تَلاَمِيذِي؟
\par 12 فَذَاكَ يُرِيكُمَا عِلِّيَّةً كَبِيرَةً مَفْرُوشَةً. هُنَاكَ أَعِدَّا».
\par 13 فَانْطَلَقَا وَوَجَدَا كَمَا قَالَ لَهُمَا فَأَعَدَّا الْفِصْحَ.
\par 14 وَلَمَّا كَانَتِ السَّاعَةُ اتَّكَأَ وَالاِثْنَا عَشَرَ رَسُولاً مَعَهُ
\par 15 وَقَالَ لَهُمْ: «شَهْوَةً اشْتَهَيْتُ أَنْ آكُلَ هَذَا الْفِصْحَ مَعَكُمْ قَبْلَ أَنْ أَتَأَلَّمَ
\par 16 لأَنِّي أَقُولُ لَكُمْ: إِنِّي لاَ آكُلُ مِنْهُ بَعْدُ حَتَّى يُكْمَلَ فِي مَلَكُوتِ اللهِ».
\par 17 ثُمَّ تَنَاوَلَ كَأْساً وَشَكَرَ وَقَالَ: «خُذُوا هَذِهِ وَاقْتَسِمُوهَا بَيْنَكُمْ
\par 18 لأَنِّي أَقُولُ لَكُمْ: إِنِّي لاَ أَشْرَبُ مِنْ نِتَاجِ الْكَرْمَةِ حَتَّى يَأْتِيَ مَلَكُوتُ اللهِ».
\par 19 وَأَخَذَ خُبْزاً وَشَكَرَ وَكَسَّرَ وَأَعْطَاهُمْ قَائِلاً: «هَذَا هُوَ جَسَدِي الَّذِي يُبْذَلُ عَنْكُمْ. اِصْنَعُوا هَذَا لِذِكْرِي».
\par 20 وَكَذَلِكَ الْكَأْسَ أَيْضاً بَعْدَ الْعَشَاءِ قَائِلاً: «هَذِهِ الْكَأْسُ هِيَ الْعَهْدُ الْجَدِيدُ بِدَمِي الَّذِي يُسْفَكُ عَنْكُمْ.
\par 21 وَلَكِنْ هُوَذَا يَدُ الَّذِي يُسَلِّمُنِي هِيَ مَعِي عَلَى الْمَائِدَةِ.
\par 22 وَابْنُ الإِنْسَانِ مَاضٍ كَمَا هُوَ مَحْتُومٌ وَلَكِنْ وَيْلٌ لِذَلِكَ الإِنْسَانِ الَّذِي يُسَلِّمُهُ».
\par 23 فَابْتَدَأُوا يَتَسَاءَلُونَ فِيمَا بَيْنَهُمْ: «مَنْ تَرَى مِنْهُمْ هُوَ الْمُزْمِعُ أَنْ يَفْعَلَ هَذَا؟».
\par 24 وَكَانَتْ بَيْنَهُمْ أَيْضاً مُشَاجَرَةٌ مَنْ مِنْهُمْ يُظَنُّ أَنَّهُ يَكُونُ أَكْبَرَ.
\par 25 فَقَالَ لَهُمْ: «مُلُوكُ الأُمَمِ يَسُودُونَهُمْ وَالْمُتَسَلِّطُونَ عَلَيْهِمْ يُدْعَوْنَ مُحْسِنِينَ.
\par 26 وَأَمَّا أَنْتُمْ فَلَيْسَ هَكَذَا بَلِ الْكَبِيرُ فِيكُمْ لِيَكُنْ كَالأَصْغَرِ وَالْمُتَقَدِّمُ كَالْخَادِمِ.
\par 27 لأَنْ مَنْ هُوَ أَكْبَرُ؟ أَلَّذِي يَتَّكِئُ أَمِ الَّذِي يَخْدِمُ؟ أَلَيْسَ الَّذِي يَتَّكِئُ؟ وَلَكِنِّي أَنَا بَيْنَكُمْ كَالَّذِي يَخْدِمُ.
\par 28 أَنْتُمُ الَّذِينَ ثَبَتُوا مَعِي فِي تَجَارِبِي
\par 29 وَأَنَا أَجْعَلُ لَكُمْ كَمَا جَعَلَ لِي أَبِي مَلَكُوتاً
\par 30 لِتَأْكُلُوا وَتَشْرَبُوا عَلَى مَائِدَتِي فِي مَلَكُوتِي وَتَجْلِسُوا عَلَى كَرَاسِيَّ تَدِينُونَ أَسْبَاطَ إِسْرَائِيلَ الاِثْنَيْ عَشَرَ».
\par 31 وَقَالَ الرَّبُّ: «سِمْعَانُ سِمْعَانُ هُوَذَا الشَّيْطَانُ طَلَبَكُمْ لِكَيْ يُغَرْبِلَكُمْ كَالْحِنْطَةِ!
\par 32 وَلَكِنِّي طَلَبْتُ مِنْ أَجْلِكَ لِكَيْ لاَ يَفْنَى إِيمَانُكَ. وَأَنْتَ مَتَى رَجَعْتَ ثَبِّتْ إِخْوَتَكَ».
\par 33 فَقَالَ لَهُ: «يَا رَبُّ إِنِّي مُسْتَعِدٌّ أَنْ أَمْضِيَ مَعَكَ حَتَّى إِلَى السِّجْنِ وَإِلَى الْمَوْتِ».
\par 34 فَقَالَ: «أَقُولُ لَكَ يَا بُطْرُسُ لاَ يَصِيحُ الدِّيكُ الْيَوْمَ قَبْلَ أَنْ تُنْكِرَ ثَلاَثَ مَرَّاتٍ أَنَّكَ تَعْرِفُنِي».
\par 35 ثُمَّ قَالَ لَهُمْ: «حِينَ أَرْسَلْتُكُمْ بِلاَ كِيسٍ وَلاَ مِزْوَدٍ وَلاَ أَحْذِيَةٍ هَلْ أَعْوَزَكُمْ شَيْءٌ؟» فَقَالُوا: «لاَ».
\par 36 فَقَالَ لَهُمْ: «لَكِنِ الآنَ مَنْ لَهُ كِيسٌ فَلْيَأْخُذْهُ وَمِزْوَدٌ كَذَلِكَ. وَمَنْ لَيْسَ لَهُ فَلْيَبِعْ ثَوْبَهُ وَيَشْتَرِ سَيْفاً.
\par 37 لأَنِّي أَقُولُ لَكُمْ إِنَّهُ يَنْبَغِي أَنْ يَتِمَّ فِيَّ أَيْضاً هَذَا الْمَكْتُوبُ: وَأُحْصِيَ مَعَ أَثَمَةٍ. لأَنَّ مَا هُوَ مِنْ جِهَتِي لَهُ انْقِضَاءٌ».
\par 38 فَقَالُوا: «يَا رَبُّ هُوَذَا هُنَا سَيْفَانِ». فَقَالَ لَهُمْ: «يَكْفِي!».
\par 39 وَخَرَجَ وَمَضَى كَالْعَادَةِ إِلَى جَبَلِ الزَّيْتُونِ وَتَبِعَهُ أَيْضاً تَلاَمِيذُهُ.
\par 40 وَلَمَّا صَارَ إِلَى الْمَكَانِ قَالَ لَهُمْ: «صَلُّوا لِكَيْ لاَ تَدْخُلُوا فِي تَجْرِبَةٍ».
\par 41 وَانْفَصَلَ عَنْهُمْ نَحْوَ رَمْيَةِ حَجَرٍ وَجَثَا عَلَى رُكْبَتَيْهِ وَصَلَّى
\par 42 قَائِلاً: «يَا أَبَتَاهُ إِنْ شِئْتَ أَنْ تُجِيزَ عَنِّي هَذِهِ الْكَأْسَ. وَلَكِنْ لِتَكُنْ لاَ إِرَادَتِي بَلْ إِرَادَتُكَ».
\par 43 وَظَهَرَ لَهُ مَلاَكٌ مِنَ السَّمَاءِ يُقَوِّيهِ.
\par 44 وَإِذْ كَانَ فِي جِهَادٍ كَانَ يُصَلِّي بِأَشَدِّ لَجَاجَةٍ وَصَارَ عَرَقُهُ كَقَطَرَاتِ دَمٍ نَازِلَةٍ عَلَى الأَرْضِ.
\par 45 ثُمَّ قَامَ مِنَ الصَّلاَةِ وَجَاءَ إِلَى تَلاَمِيذِهِ فَوَجَدَهُمْ نِيَاماً مِنَ الْحُزْنِ.
\par 46 فَقَالَ لَهُمْ: «لِمَاذَا أَنْتُمْ نِيَامٌ؟ قُومُوا وَصَلُّوا لِئَلاَّ تَدْخُلُوا فِي تَجْرِبَةٍ».
\par 47 وَبَيْنَمَا هُوَ يَتَكَلَّمُ إِذَا جَمْعٌ وَالَّذِي يُدْعَى يَهُوذَا - أَحَدُ الاِثْنَيْ عَشَرَ - يَتَقَدَّمُهُمْ فَدَنَا مِنْ يَسُوعَ لِيُقَبِّلَهُ.
\par 48 فَقَالَ لَهُ يَسُوعُ: «يَا يَهُوذَا أَبِقُبْلَةٍ تُسَلِّمُ ابْنَ الإِنْسَانِ؟»
\par 49 فَلَمَّا رَأَى الَّذِينَ حَوْلَهُ مَا يَكُونُ قَالُوا: «يَا رَبُّ أَنَضْرِبُ بِالسَّيْفِ؟»
\par 50 وَضَرَبَ وَاحِدٌ مِنْهُمْ عَبْدَ رَئِيسِ الْكَهَنَةِ فَقَطَعَ أُذْنَهُ الْيُمْنَى.
\par 51 فَقَالَ يَسُوعُ: «دَعُوا إِلَى هَذَا!» وَلَمَسَ أُذْنَهُ وَأَبْرَأَهَا.
\par 52 ثُمَّ قَالَ يَسُوعُ لِرُؤَسَاءِ الْكَهَنَةِ وَقُوَّادِ جُنْدِ الْهَيْكَلِ وَالشُّيُوخِ الْمُقْبِلِينَ عَلَيْهِ: «كَأَنَّهُ عَلَى لِصٍّ خَرَجْتُمْ بِسُيُوفٍ وَعِصِيٍّ!
\par 53 إِذْ كُنْتُ مَعَكُمْ كُلَّ يَوْمٍ فِي الْهَيْكَلِ لَمْ تَمُدُّوا عَلَيَّ الأَيَادِيَ. وَلَكِنَّ هَذِهِ سَاعَتُكُمْ وَسُلْطَانُ الظُّلْمَةِ».
\par 54 فَأَخَذُوهُ وَسَاقُوهُ وَأَدْخَلُوهُ إِلَى بَيْتِ رَئِيسِ الْكَهَنَةِ. وَأَمَّا بُطْرُسُ فَتَبِعَهُ مِنْ بَعِيدٍ.
\par 55 وَلَمَّا أَضْرَمُوا نَاراً فِي وَسَطِ الدَّارِ وَجَلَسُوا مَعاً جَلَسَ بُطْرُسُ بَيْنَهُمْ.
\par 56 فَرَأَتْهُ جَارِيَةٌ جَالِساً عِنْدَ النَّارِ فَتَفَرَّسَتْ فيهِ وَقَالَتْ: «وَهَذَا كَانَ مَعَهُ».
\par 57 فَأَنْكَرَهُ قَائِلاً: «لَسْتُ أَعْرِفُهُ يَا امْرَأَةُ!»
\par 58 وَبَعْدَ قَلِيلٍ رَآهُ آخَرُ وَقَالَ: «وَأَنْتَ مِنْهُمْ!» فَقَالَ بُطْرُسُ: «يَا إِنْسَانُ لَسْتُ أَنَا!»
\par 59 وَلَمَّا مَضَى نَحْوُ سَاعَةٍ وَاحِدَةٍ أَكَّدَ آخَرُ قَائِلاً: «بِالْحَقِّ إِنَّ هَذَا أَيْضاً كَانَ مَعَهُ لأَنَّهُ جَلِيلِيٌّ أَيْضاً».
\par 60 فَقَالَ بُطْرُسُ: «يَا إِنْسَانُ لَسْتُ أَعْرِفُ مَا تَقُولُ». وَفِي الْحَالِ بَيْنَمَا هُوَ يَتَكَلَّمُ صَاحَ الدِّيكُ.
\par 61 فَالْتَفَتَ الرَّبُّ وَنَظَرَ إِلَى بُطْرُسَ فَتَذَكَّرَ بُطْرُسُ كَلاَمَ الرَّبِّ كَيْفَ قَالَ لَهُ: «إِنَّكَ قَبْلَ أَنْ يَصِيحَ الدِّيكُ تُنْكِرُنِي ثَلاَثَ مَرَّاتٍ».
\par 62 فَخَرَجَ بُطْرُسُ إِلَى خَارِجٍ وَبَكَى بُكَاءً مُرّاً.
\par 63 وَالرِّجَالُ الَّذِينَ كَانُوا ضَابِطِينَ يَسُوعَ كَانُوا يَسْتَهْزِئُونَ بِهِ وَهُمْ يَجْلِدُونَهُ
\par 64 وَغَطَّوْهُ وَكَانُوا يَضْرِبُونَ وَجْهَهُ وَيَسْأَلُونَهُ: «تَنَبَّأْ! مَنْ هُوَ الَّذِي ضَرَبَكَ؟»
\par 65 وَأَشْيَاءَ أُخَرَ كَثِيرَةً كَانُوا يَقُولُونَ عَلَيْهِ مُجَدِّفِينَ.
\par 66 وَلَمَّا كَانَ النَّهَارُ اجْتَمَعَتْ مَشْيَخَةُ الشَّعْبِ: رُؤَسَاءُ الْكَهَنَةِ وَالْكَتَبَةُ وَأَصْعَدُوهُ إِلَى مَجْمَعِهِمْ
\par 67 قَائِلِينَ: «إِنْ كُنْتَ أَنْتَ الْمسِيحَ فَقُلْ لَنَا». فَقَالَ لَهُمْ: «إِنْ قُلْتُ لَكُمْ لاَ تُصَدِّقُونَ
\par 68 وَإِنْ سَأَلْتُ لاَ تُجِيبُونَنِي وَلاَ تُطْلِقُونَنِي.
\par 69 مُنْذُ الآنَ يَكُونُ ابْنُ الإِنْسَانِ جَالِساً عَنْ يَمِينِ قُوَّةِ اللهِ».
\par 70 فَقَالَ الْجَمِيعُ: «أَفَأَنْتَ ابْنُ اللهِ؟» فَقَالَ لَهُمْ: «أَنْتُمْ تَقُولُونَ إِنِّي أَنَا هُوَ».
\par 71 فَقَالُوا: «مَا حَاجَتُنَا بَعْدُ إِلَى شَهَادَةٍ؟ لأَنَّنَا نَحْنُ سَمِعْنَا مِنْ فَمِهِ».

\chapter{23}

\par 1 فَقَامَ كُلُّ جُمْهُورِهِمْ وَجَاءُوا بِهِ إِلَى بِيلاَطُسَ
\par 2 وَابْتَدَأُوا يَشْتَكُونَ عَلَيْهِ قَائِلِينَ: «إِنَّنَا وَجَدْنَا هَذَا يُفْسِدُ الأُمَّةَ وَيَمْنَعُ أَنْ تُعْطَى جِزْيَةٌ لِقَيْصَرَ قَائِلاً: إِنَّهُ هُوَ مَسِيحٌ مَلِكٌ».
\par 3 فَسَأَلَهُ بِيلاَطُسُ: «أَنْتَ مَلِكُ الْيَهُودِ؟» فَأَجَابَهُ: «أَنْتَ تَقُولُ».
\par 4 فَقَالَ بِيلاَطُسُ لِرُؤَسَاءِ الْكَهَنَةِ وَالْجُمُوعِ: «إِنِّي لاَ أَجِدُ عِلَّةً فِي هَذَا الإِنْسَانِ».
\par 5 فَكَانُوا يُشَدِّدُونَ قَائِلِينَ: «إِنَّهُ يُهَيِّجُ الشَّعْبَ وَهُوَ يُعَلِّمُ فِي كُلِّ الْيَهُودِيَّةِ مُبْتَدِئاً مِنَ الْجَلِيلِ إِلَى هُنَا».
\par 6 فَلَمَّا سَمِعَ بِيلاَطُسُ ذِكْرَ الْجَلِيلِ سَأَلَ: «هَلِ الرَّجُلُ جَلِيلِيٌّ؟»
\par 7 وَحِينَ عَلِمَ أَنَّهُ مِنْ سَلْطَنَةِ هِيرُودُسَ أَرْسَلَهُ إِلَى هِيرُودُسَ إِذْ كَانَ هُوَ أَيْضاً تِلْكَ الأَيَّامَ فِي أُورُشَلِيمَ.
\par 8 وَأَمَّا هِيرُودُسُ فَلَمَّا رَأَى يَسُوعَ فَرِحَ جِدّاً لأَنَّهُ كَانَ يُرِيدُ مِنْ زَمَانٍ طَوِيلٍ أَنْ يَرَاهُ لِسَمَاعِهِ عَنْهُ أَشْيَاءَ كَثِيرَةً وَتَرَجَّى أَنْ يَرَاهُ يَصْنَعُ آيَةً.
\par 9 وَسَأَلَهُ بِكَلاَمٍ كَثِيرٍ فَلَمْ يُجِبْهُ بِشَيْءٍ.
\par 10 وَوَقَفَ رُؤَسَاءُ الْكَهَنَةِ وَالْكَتَبَةُ يَشْتَكُونَ عَلَيْهِ بِاشْتِدَادٍ
\par 11 فَاحْتَقَرَهُ هِيرُودُسُ مَعَ عَسْكَرِهِ وَاسْتَهْزَأَ بِهِ وَأَلْبَسَهُ لِبَاساً لاَمِعاً وَرَدَّهُ إِلَى بِيلاَطُسَ.
\par 12 فَصَارَ بِيلاَطُسُ وَهِيرُودُسُ صَدِيقَيْنِ مَعَ بَعْضِهِمَا فِي ذَلِكَ الْيَوْمِ لأَنَّهُمَا كَانَا مِنْ قَبْلُ فِي عَدَاوَةٍ بَيْنَهُمَا.
\par 13 فَدَعَا بِيلاَطُسُ رُؤَسَاءَ الْكَهَنَةِ وَالْعُظَمَاءَ وَالشَّعْبَ
\par 14 وَقَالَ لَهُمْ: «قَدْ قَدَّمْتُمْ إِلَيَّ هَذَا الإِنْسَانَ كَمَنْ يُفْسِدُ الشَّعْبَ. وَهَا أَنَا قَدْ فَحَصْتُ قُدَّامَكُمْ وَلَمْ أَجِدْ فِي هَذَا الإِنْسَانِ عِلَّةً مِمَّا تَشْتَكُونَ بِهِ عَلَيْهِ.
\par 15 وَلاَ هِيرُودُسُ أَيْضاً لأَنِّي أَرْسَلْتُكُمْ إِلَيْهِ. وَهَا لاَ شَيْءَ يَسْتَحِقُّ الْمَوْتَ صُنِعَ مِنْهُ.
\par 16 فَأَنَا أُؤَدِّبُهُ وَأُطْلِقُهُ».
\par 17 وَكَانَ مُضْطَرّاً أَنْ يُطْلِقَ لَهُمْ كُلَّ عِيدٍ وَاحِداً
\par 18 فَصَرَخُوا بِجُمْلَتِهِمْ قَائِلِينَ: «خُذْ هَذَا وَأَطْلِقْ لَنَا بَارَابَاسَ!»
\par 19 وَذَاكَ كَانَ قَدْ طُرِحَ فِي السِّجْنِ لأَجْلِ فِتْنَةٍ حَدَثَتْ فِي الْمَدِينَةِ وَقَتْلٍ.
\par 20 فَنَادَاهُمْ أَيْضاً بِيلاَطُسُ وَهُوَ يُرِيدُ أَنْ يُطْلِقَ يَسُوعَ
\par 21 فَصَرَخُوا: «اصْلِبْهُ! اصْلِبْهُ!»
\par 22 فَقَالَ لَهُمْ ثَالِثَةً: «فَأَيَّ شَرٍّ عَمِلَ هَذَا؟ إِنِّي لَمْ أَجِدْ فِيهِ عِلَّةً لِلْمَوْتِ فَأَنَا أُؤَدِّبُهُ وَأُطْلِقُهُ».
\par 23 فَكَانُوا يَلِجُّونَ بِأَصْوَاتٍ عَظِيمَةٍ طَالِبِينَ أَنْ يُصْلَبَ. فَقَوِيَتْ أَصْوَاتُهُمْ وَأَصْوَاتُ رُؤَسَاءِ الْكَهَنَةِ.
\par 24 فَحَكَمَ بِيلاَطُسُ أَنْ تَكُونَ طِلْبَتُهُمْ.
\par 25 فَأَطْلَقَ لَهُمُ الَّذِي طُرِحَ فِي السِّجْنِ لأَجْلِ فِتْنَةٍ وَقَتْلٍ الَّذِي طَلَبُوهُ وَأَسْلَمَ يَسُوعَ لِمَشِيئَتِهِمْ.
\par 26 وَلَمَّا مَضَوْا بِهِ أَمْسَكُوا سِمْعَانَ رَجُلاً قَيْرَوَانِيّاً كَانَ آتِياً مِنَ الْحَقْلِ وَوَضَعُوا عَلَيْهِ الصَّلِيبَ لِيَحْمِلَهُ خَلْفَ يَسُوعَ.
\par 27 وَتَبِعَهُ جُمْهُورٌ كَثِيرٌ مِنَ الشَّعْبِ وَالنِّسَاءِ اللَّوَاتِي كُنَّ يَلْطِمْنَ أَيْضاً وَيَنُحْنَ عَلَيْهِ.
\par 28 فَالْتَفَتَ إِلَيْهِنَّ يَسُوعُ وَقَالَ: «يَا بَنَاتِ أُورُشَلِيمَ لاَ تَبْكِينَ عَلَيَّ بَلِ ابْكِينَ عَلَى أَنْفُسِكُنَّ وَعَلَى أَوْلاَدِكُنَّ
\par 29 لأَنَّهُ هُوَذَا أَيَّامٌ تَأْتِي يَقُولُونَ فِيهَا: طُوبَى لِلْعَوَاقِرِ وَالْبُطُونِ الَّتِي لَمْ تَلِدْ وَالثُّدِيِّ الَّتِي لَمْ تُرْضِعْ.
\par 30 حِينَئِذٍ يَبْتَدِئُونَ يَقُولُونَ لِلْجِبَالِ: اسْقُطِي عَلَيْنَا وَلِلآكَامِ: غَطِّينَا.
\par 31 لأَنَّهُ إِنْ كَانُوا بِالْعُودِ الرَّطْبِ يَفْعَلُونَ هَذَا فَمَاذَا يَكُونُ بِالْيَابِسِ؟».
\par 32 وَجَاءُوا أَيْضاً بِاثْنَيْنِ آخَرَيْنِ مُذْنِبَيْنِ لِيُقْتَلاَ مَعَهُ.
\par 33 وَلَمَّا مَضَوْا بِهِ إِلَى الْمَوْضِعِ الَّذِي يُدْعَى «جُمْجُمَةَ» صَلَبُوهُ هُنَاكَ مَعَ الْمُذْنِبَيْنِ وَاحِداً عَنْ يَمِينِهِ وَالآخَرَ عَنْ يَسَارِهِ.
\par 34 فَقَالَ يَسُوعُ: «يَا أَبَتَاهُ اغْفِرْ لَهُمْ لأَنَّهُمْ لاَ يَعْلَمُونَ مَاذَا يَفْعَلُونَ». وَإِذِ اقْتَسَمُوا ثِيَابَهُ اقْتَرَعُوا عَلَيْهَا.
\par 35 وَكَانَ الشَّعْبُ وَاقِفِينَ يَنْظُرُونَ وَالرُّؤَسَاءُ أَيْضاً مَعَهُمْ يَسْخَرُونَ بِهِ قَائِلِينَ: «خَلَّصَ آخَرِينَ فَلْيُخَلِّصْ نَفْسَهُ إِنْ كَانَ هُوَ الْمَسِيحَ مُخْتَارَ اللهِ».
\par 36 وَالْجُنْدُ أَيْضاً اسْتَهْزَأُوا بِهِ وَهُمْ يَأْتُونَ وَيُقَدِّمُونَ لَهُ خَلاًّ
\par 37 قَائِلِينَ: «إِنْ كُنْتَ أَنْتَ مَلِكَ الْيَهُودِ فَخَلِّصْ نَفْسَكَ».
\par 38 وَكَانَ عُنْوَانٌ مَكْتُوبٌ فَوْقَهُ بِأَحْرُفٍ يُونَانِيَّةٍ وَرُومَانِيَّةٍ وَعِبْرَانِيَّةٍ: «هَذَا هُوَ مَلِكُ الْيَهُودِ».
\par 39 وَكَانَ وَاحِدٌ مِنَ الْمُذْنِبَيْنِ الْمُعَلَّقَيْنِ يُجَدِّفُ عَلَيْهِ قَائِلاً: «إِنْ كُنْتَ أَنْتَ الْمَسِيحَ فَخَلِّصْ نَفْسَكَ وَإِيَّانَا!»
\par 40 فَانْتَهَرَهُ الآخَرُ قَائِلاً: «أَوَلاَ أَنْتَ تَخَافُ اللهَ إِذْ أَنْتَ تَحْتَ هَذَا الْحُكْمِ بِعَيْنِهِ؟
\par 41 أَمَّا نَحْنُ فَبِعَدْلٍ لأَنَّنَا نَنَالُ اسْتِحْقَاقَ مَا فَعَلْنَا وَأَمَّا هَذَا فَلَمْ يَفْعَلْ شَيْئاً لَيْسَ فِي مَحَلِّهِ».
\par 42 ثُمَّ قَالَ لِيَسُوعَ: «اذْكُرْنِي يَا رَبُّ مَتَى جِئْتَ فِي مَلَكُوتِكَ».
\par 43 فَقَالَ لَهُ يَسُوعُ: «الْحَقَّ أَقُولُ لَكَ: إِنَّكَ الْيَوْمَ تَكُونُ مَعِي فِي الْفِرْدَوْسِ».
\par 44 وَكَانَ نَحْوُ السَّاعَةِ السَّادِسَةِ فَكَانَتْ ظُلْمَةٌ عَلَى الأَرْضِ كُلِّهَا إِلَى السَّاعَةِ التَّاسِعَةِ.
\par 45 وَأَظْلَمَتِ الشَّمْسُ وَانْشَقَّ حِجَابُ الْهَيْكَلِ مِنْ وَسَطِهِ.
\par 46 وَنَادَى يَسُوعُ بِصَوْتٍ عَظِيمٍ: «يَا أَبَتَاهُ فِي يَدَيْكَ أَسْتَوْدِعُ رُوحِي». وَلَمَّا قَالَ هَذَا أَسْلَمَ الرُّوحَ.
\par 47 فَلَمَّا رَأَى قَائِدُ الْمِئَةِ مَا كَانَ مَجَّدَ اللهَ قَائِلاً: «بِالْحَقِيقَةِ كَانَ هَذَا الإِنْسَانُ بَارّاً!»
\par 48 وَكُلُّ الْجُمُوعِ الَّذِينَ كَانُوا مُجْتَمِعِينَ لِهَذَا الْمَنْظَرِ لَمَّا أَبْصَرُوا مَا كَانَ رَجَعُوا وَهُمْ يَقْرَعُونَ صُدُورَهُمْ.
\par 49 وَكَانَ جَمِيعُ مَعَارِفِهِ وَنِسَاءٌ كُنَّ قَدْ تَبِعْنَهُ مِنَ الْجَلِيلِ وَاقِفِينَ مِنْ بَعِيدٍ يَنْظُرُونَ ذَلِكَ.
\par 50 وَإِذَا رَجُلٌ اسْمُهُ يُوسُفُ وَكَانَ مُشِيراً وَرَجُلاً صَالِحاً بَارّاً -
\par 51 هَذَا لَمْ يَكُنْ مُوافِقاً لِرَأْيِهِمْ وَعَمَلِهِمْ وَهُوَ مِنَ الرَّامَةِ مَدِينَةٍ لِلْيَهُودِ. وَكَانَ هُوَ أَيْضاً يَنْتَظِرُ مَلَكُوتَ اللهِ.
\par 52 هَذَا تَقَدَّمَ إِلَى بِيلاَطُسَ وَطَلَبَ جَسَدَ يَسُوعَ
\par 53 وَأَنْزَلَهُ وَلَفَّهُ بِكَتَّانٍ وَوَضَعَهُ فِي قَبْرٍ مَنْحُوتٍ حَيْثُ لَمْ يَكُنْ أَحَدٌ وُضِعَ قَطُّ.
\par 54 وَكَانَ يَوْمُ الاِسْتِعْدَادِ وَالسَّبْتُ يَلُوحُ.
\par 55 وَتَبِعَتْهُ نِسَاءٌ كُنَّ قَدْ أَتَيْنَ مَعَهُ مِنَ الْجَلِيلِ وَنَظَرْنَ الْقَبْرَ وَكَيْفَ وُضِعَ جَسَدُهُ.
\par 56 فَرَجَعْنَ وَأَعْدَدْنَ حَنُوطاً وَأَطْيَاباً. وَفِي السَّبْتِ اسْتَرَحْنَ حَسَبَ الْوَصِيَّةِ.

\chapter{24}

\par 1 ثُمَّ فِي أَوَّلِ الأُسْبُوعِ أَوَّلَ الْفَجْرِ أَتَيْنَ إِلَى الْقَبْرِ حَامِلاَتٍ الْحَنُوطَ الَّذِي أَعْدَدْنَهُ وَمَعَهُنَّ أُنَاسٌ.
\par 2 فَوَجَدْنَ الْحَجَرَ مُدَحْرَجاً عَنِ الْقَبْرِ
\par 3 فَدَخَلْنَ وَلَمْ يَجِدْنَ جَسَدَ الرَّبِّ يَسُوعَ.
\par 4 وَفِيمَا هُنَّ مُحْتَارَاتٌ فِي ذَلِكَ إِذَا رَجُلاَنِ وَقَفَا بِهِنَّ بِثِيَابٍ بَرَّاقَةٍ.
\par 5 وَإِذْ كُنَّ خَائِفَاتٍ وَمُنَكِّسَاتٍ وُجُوهَهُنَّ إِلَى الأَرْضِ قَالاَ لَهُنَّ: «لِمَاذَا تَطْلُبْنَ الْحَيَّ بَيْنَ الأَمْوَاتِ؟
\par 6 لَيْسَ هُوَ هَهُنَا لَكِنَّهُ قَامَ! اُذْكُرْنَ كَيْفَ كَلَّمَكُنَّ وَهُوَ بَعْدُ فِي الْجَلِيلِ
\par 7 قَائِلاً: إِنَّهُ يَنْبَغِي أَنْ يُسَلَّمَ ابْنُ الإِنْسَانِ فِي أَيْدِي أُنَاسٍ خُطَاةٍ وَيُصْلَبَ وَفِي الْيَوْمِ الثَّالِثِ يَقُومُ».
\par 8 فَتَذَكَّرْنَ كَلاَمَهُ
\par 9 وَرَجَعْنَ مِنَ الْقَبْرِ وَأَخْبَرْنَ الأَحَدَ عَشَرَ وَجَمِيعَ الْبَاقِينَ بِهَذَا كُلِّهِ.
\par 10 وَكَانَتْ مَرْيَمُ الْمَجْدَلِيَّةُ وَيُوَنَّا وَمَرْيَمُ أُمُّ يَعْقُوبَ وَالْبَاقِيَاتُ مَعَهُنَّ اللَّوَاتِي قُلْنَ هَذَا لِلرُّسُلِ.
\par 11 فَتَرَاءَى كَلاَمُهُنَّ لَهُمْ كَالْهَذَيَانِ وَلَمْ يُصَدِّقُوهُنَّ.
\par 12 فَقَامَ بُطْرُسُ وَرَكَضَ إِلَى الْقَبْرِ فَانْحَنَى وَنَظَرَ الأَكْفَانَ مَوْضُوعَةً وَحْدَهَا فَمَضَى مُتَعَجِّباً فِي نَفْسِهِ مِمَّا كَانَ.
\par 13 وَإِذَا اثْنَانِ مِنْهُمْ كَانَا مُنْطَلِقَيْنِ فِي ذَلِكَ الْيَوْمِ إِلَى قَرْيَةٍ بَعِيدَةٍ عَنْ أُورُشَلِيمَ سِتِّينَ غَلْوَةً اسْمُهَا «عِمْوَاسُ».
\par 14 وَكَانَا يَتَكَلَّمَانِ بَعْضُهُمَا مَعَ بَعْضٍ عَنْ جَمِيعِ هَذِهِ الْحَوَادِثِ.
\par 15 وَفِيمَا هُمَا يَتَكَلَّمَانِ وَيَتَحَاوَرَانِ اقْتَرَبَ إِلَيْهِمَا يَسُوعُ نَفْسُهُ وَكَانَ يَمْشِي مَعَهُمَا.
\par 16 وَلَكِنْ أُمْسِكَتْ أَعْيُنُهُمَا عَنْ مَعْرِفَتِهِ.
\par 17 فَقَالَ لَهُمَا: «مَا هَذَا الْكَلاَمُ الَّذِي تَتَطَارَحَانِ بِهِ وَأَنْتُمَا مَاشِيَانِ عَابِسَيْنِ؟»
\par 18 فَأَجَابَ أَحَدُهُمَا الَّذِي اسْمُهُ كَِلْيُوبَاسُ: «هَلْ أَنْتَ مُتَغَرِّبٌ وَحْدَكَ فِي أُورُشَلِيمَ وَلَمْ تَعْلَمِ الأُمُورَ الَّتِي حَدَثَتْ فِيهَا فِي هَذِهِ الأَيَّامِ؟»
\par 19 فَقَالَ لَهُمَا: «وَمَا هِيَ؟» فَقَالاَ: «الْمُخْتَصَّةُ بِيَسُوعَ النَّاصِرِيِّ الَّذِي كَانَ إِنْسَاناً نَبِيّاً مُقْتَدِراً فِي الْفِعْلِ وَالْقَوْلِ أَمَامَ اللهِ وَجَمِيعِ الشَّعْبِ.
\par 20 كَيْفَ أَسْلَمَهُ رُؤَسَاءُ الْكَهَنَةِ وَحُكَّامُنَا لِقَضَاءِ الْمَوْتِ وَصَلَبُوهُ.
\par 21 وَنَحْنُ كُنَّا نَرْجُو أَنَّهُ هُوَ الْمُزْمِعُ أَنْ يَفْدِيَ إِسْرَائِيلَ. وَلَكِنْ مَعَ هَذَا كُلِّهِ الْيَوْمَ لَهُ ثَلاَثَةُ أَيَّامٍ مُنْذُ حَدَثَ ذَلِكَ.
\par 22 بَلْ بَعْضُ النِّسَاءِ مِنَّا حَيَّرْنَنَا إِذْ كُنَّ بَاكِراً عِنْدَ الْقَبْرِ
\par 23 وَلَمَّا لَمْ يَجِدْنَ جَسَدَهُ أَتَيْنَ قَائِلاَتٍ: إِنَّهُنَّ رَأَيْنَ مَنْظَرَ مَلاَئِكَةٍ قَالُوا إِنَّهُ حَيٌّ.
\par 24 وَمَضَى قَوْمٌ مِنَ الَّذِينَ مَعَنَا إِلَى الْقَبْرِ فَوَجَدُوا هَكَذَا كَمَا قَالَتْ أَيْضاً النِّسَاءُ وَأَمَّا هُوَ فَلَمْ يَرَوْهُ».
\par 25 فَقَالَ لَهُمَا: «أَيُّهَا الْغَبِيَّانِ وَالْبَطِيئَا الْقُلُوبِ فِي الإِيمَانِ بِجَمِيعِ مَا تَكَلَّمَ بِهِ الأَنْبِيَاءُ
\par 26 أَمَا كَانَ يَنْبَغِي أَنَّ الْمَسِيحَ يَتَأَلَّمُ بِهَذَا وَيَدْخُلُ إِلَى مَجْدِهِ؟»
\par 27 ثُمَّ ابْتَدَأَ مِنْ مُوسَى وَمِنْ جَمِيعِ الأَنْبِيَاءِ يُفَسِّرُ لَهُمَا الأُمُورَ الْمُخْتَصَّةَ بِهِ فِي جَمِيعِ الْكُتُبِ.
\par 28 ثُمَّ اقْتَرَبُوا إِلَى الْقَرْيَةِ الَّتِي كَانَا مُنْطَلِقَيْنِ إِلَيْهَا وَهُوَ تَظَاهَرَ كَأَنَّهُ مُنْطَلِقٌ إِلَى مَكَانٍ أَبْعَدَ.
\par 29 فَأَلْزَمَاهُ قَائِلَيْنِ: «امْكُثْ مَعَنَا لأَنَّهُ نَحْوُ الْمَسَاءِ وَقَدْ مَالَ النَّهَارُ». فَدَخَلَ لِيَمْكُثَ مَعَهُمَا.
\par 30 فَلَمَّا اتَّكَأَ مَعَهُمَا أَخَذَ خُبْزاً وَبَارَكَ وَكَسَّرَ وَنَاوَلَهُمَا
\par 31 فَانْفَتَحَتْ أَعْيُنُهُمَا وَعَرَفَاهُ ثُمَّ اخْتَفَى عَنْهُمَا
\par 32 فَقَالَ بَعْضُهُمَا لِبَعْضٍ: «أَلَمْ يَكُنْ قَلْبُنَا مُلْتَهِباً فِينَا إِذْ كَانَ يُكَلِّمُنَا فِي الطَّرِيقِ وَيُوضِحُ لَنَا الْكُتُبَ؟»
\par 33 فَقَامَا فِي تِلْكَ السَّاعَةِ وَرَجَعَا إِلَى أُورُشَلِيمَ وَوَجَدَا الأَحَدَ عَشَرَ مُجْتَمِعِينَ هُمْ وَالَّذِينَ مَعَهُمْ
\par 34 وَهُمْ يَقُولُونَ: «إِنَّ الرَّبَّ قَامَ بِالْحَقِيقَةِ وَظَهَرَ لِسِمْعَانَ!»
\par 35 وَأَمَّا هُمَا فَكَانَا يُخْبِرَانِ بِمَا حَدَثَ فِي الطَّرِيقِ وَكَيْفَ عَرَفَاهُ عِنْدَ كَسْرِ الْخُبْزِ.
\par 36 وَفِيمَا هُمْ يَتَكَلَّمُونَ بِهَذَا وَقَفَ يَسُوعُ نَفْسُهُ فِي وَسَطِهِمْ وَقَالَ لَهُمْ: «سَلاَمٌ لَكُمْ!»
\par 37 فَجَزِعُوا وَخَافُوا وَظَنُّوا أَنَّهُمْ نَظَرُوا رُوحاً.
\par 38 فَقَالَ لَهُمْ: «مَا بَالُكُمْ مُضْطَرِبِينَ وَلِمَاذَا تَخْطُرُ أَفْكَارٌ فِي قُلُوبِكُمْ؟
\par 39 اُنْظُرُوا يَدَيَّ وَرِجْلَيَّ: إِنِّي أَنَا هُوَ. جُسُّونِي وَانْظُرُوا فَإِنَّ الرُّوحَ لَيْسَ لَهُ لَحْمٌ وَعِظَامٌ كَمَا تَرَوْنَ لِي».
\par 40 وَحِينَ قَالَ هَذَا أَرَاهُمْ يَدَيْهِ وَرِجْلَيْهِ.
\par 41 وَبَيْنَمَا هُمْ غَيْرُ مُصَدِّقِين مِنَ الْفَرَحِ وَمُتَعَجِّبُونَ قَالَ لَهُمْ: «أَعِنْدَكُمْ هَهُنَا طَعَامٌ؟»
\par 42 فَنَاوَلُوهُ جُزْءاً مِنْ سَمَكٍ مَشْوِيٍّ وَشَيْئاً مِنْ شَهْدِ عَسَلٍ.
\par 43 فَأَخَذَ وَأَكَلَ قُدَّامَهُمْ.
\par 44 وَقَالَ لَهُمْ: «هَذَا هُوَ الْكَلاَمُ الَّذِي كَلَّمْتُكُمْ بِهِ وَأَنَا بَعْدُ مَعَكُمْ أَنَّهُ لاَ بُدَّ أَنْ يَتِمَّ جَمِيعُ مَا هُوَ مَكْتُوبٌ عَنِّي فِي نَامُوسِ مُوسَى وَالأَنْبِيَاءِ وَالْمَزَامِيرِ».
\par 45 حِينَئِذٍ فَتَحَ ذِهْنَهُمْ لِيَفْهَمُوا الْكُتُبَ.
\par 46 وَقَالَ لَهُمْ: «هَكَذَا هُوَ مَكْتُوبٌ وَهَكَذَا كَانَ يَنْبَغِي أَنَّ الْمَسِيحَ يَتَأَلَّمُ وَيَقُومُ مِنَ الأَمْوَاتِ فِي الْيَوْمِ الثَّالِثِ
\par 47 وَأَنْ يُكْرَزَ بِاسْمِهِ بِالتَّوْبَةِ وَمَغْفِرَةِ الْخَطَايَا لِجَمِيعِ الأُمَمِ مُبْتَدَأً مِنْ أُورُشَلِيمَ.
\par 48 وَأَنْتُمْ شُهُودٌ لِذَلِكَ.
\par 49 وَهَا أَنَا أُرْسِلُ إِلَيْكُمْ مَوْعِدَ أَبِي. فَأَقِيمُوا فِي مَدِينَةِ أُورُشَلِيمَ إِلَى أَنْ تُلْبَسُوا قُوَّةً مِنَ الأَعَالِي».
\par 50 وَأَخْرَجَهُمْ خَارِجاً إِلَى بَيْتِ عَنْيَا وَرَفَعَ يَدَيْهِ وَبَارَكَهُمْ.
\par 51 وَفِيمَا هُوَ يُبَارِكُهُمُ انْفَرَدَ عَنْهُمْ وَأُصْعِدَ إِلَى السَّمَاءِ.
\par 52 فَسَجَدُوا لَهُ وَرَجَعُوا إِلَى أُورُشَلِيمَ بِفَرَحٍ عَظِيمٍ
\par 53 وَكَانُوا كُلَّ حِينٍ فِي الْهَيْكَلِ يُسَبِّحُونَ وَيُبَارِكُونَ اللهَ. آمِينَ.


\end{document}