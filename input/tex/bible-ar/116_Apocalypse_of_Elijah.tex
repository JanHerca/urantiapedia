\begin{document}

\title{رؤيا إيليا}

\chapter{1}

\par \textit{دعوة النبي}

\par 1 وكان إليّ كلام الرب قائلاً: «يا ابن آدم، قل لهذا الشعب: لماذا تضيفون خطيئة إلى خطاياكم وتغضبون الرب الإله خالقكم؟»

\par 2 لا تحبوا العالم ولا الأشياء التي في العالم، لأن افتخار العالم وهلاكه من إبليس

\par \textit{الخلاص من الأسر يكون من خلال الابن المتجسد}

\par 3 تذكر أن رب المجد، الذي خلق كل شيء، رحم

\par 4 أنتم حتى ينقذنا من أسر هذا الدهر. فلطالما رغب الشيطان في ألا تشرق الشمس على الأرض وألا تُثمر الأرض، لأنه يرغب في أن يلتهم البشر كالنار التي تشتعل في القش، و

\par 5 إنه يرغب في ابتلاعهم كالماء. لذلك، بسبب هذا، رحمنا إله المجد، وأرسل ابنه إلى العالم لكي

\par 6 خلصنا من الأسر. لم يُخبر ملاكًا أو رئيس ملائكة أو أي رئاسة عندما كان على وشك المجيء إلينا، لكنه غيّر نفسه ليكون مثل كثيرين عندما كان على وشك المجيء إلينا حتى يخلصنا [من الجسد].

\par 7 فكونوا أبناءً له كما هو أبٌ لكم.

\par \textit{ما أُعِدَّ للمختومين والخطاة}

\par 8 تذكروا أنه أعد لكم عروشًا وأكاليل في السماء، قائلًا: "كل من يطيعني سينال عروشًا وأكاليل بين الذين

\par 9 هم لي. قال الرب: سأكتب اسمي على جباههم وأكون

\par 10 سيختمون يمينهم، ولن يجوعوا ولن يعطشوا. ولن يسود عليهم ابن الإثم، ولن تمنعهم العروش، بل سيُصلون

\par 11 سيسير مع الملائكة إلى مدينتي. أما الخطاة، فسوف يُذبحون ولن يمروا بالعروش، بل بعروش الموت

\par 12 استولى عليهم وتسلط عليهم لأن الملائكة لن توافقهم. لقد نفروا من مساكنه

\par \textit{المخادعون الذين يصومون}

\par 13 اسمعوا يا حكماء الأرض، بشأن المخادعين الذين سيتكاثرون في الأزمنة الأخيرة ليضعوا لأنفسهم تعاليم لا تنتمي إلى الله، ويضعوا شريعة الله جانبًا، أولئك الذين جعلوا بطونهم إلهًا لهم، قائلين: "الصوم غير موجود، ولم يخلقه الله"، جاعلاً أنفسهم غرباء عن عهد الله، وسارقين أنفسهم من

\par 14 وعود مجيدة. هذه الوعود لا تُثبت أبدًا بشكل صحيح في الإيمان الراسخ. لذلك، لا تدع هؤلاء الناس يُضلّونك

\par \textit{فوائد الصيام}

\par 15 تذكر أنه منذ أن خلق الرب السماوات، خلق الصوم من أجل منفعة البشر بسبب الأهواء والرغبات التي تحارب

\par 16 ضدك حتى لا يشعلك الشر. "بل هو صوم خالص"

\par 17 "لقد خلقت"، قال الرب. من صام باستمرار لن يخطئ.

\par 18 مع أن الغيرة والخصام في داخله. فليصم الطاهر، ولكن كلما

\par 19 من لم يصوم فهو قد أغضب الرب والملائكة أيضًا، وأحزن نفسه، وجمع غضبًا على نفسه ليوم الغضب

\par 20 لكن الصوم النقي هو ما خلقته،

\par بقلب نقي وأيدٍ نقية

\par 21 يُطهِّر الخطيئة.

\par يُشفي الأمراض.

\par يُطرد الشياطين

\par 22 إنه فعال حتى عرش الله كمرهم

\par وللتحرر من الخطيئة عن طريق صلاة طاهرة

\par \textit{الحاجة إلى عقلية فردية}

\par 23 من منكم إذا كان مجيدًا في حرفته يخرج إلى الميدان وليس بيده أداة؟ أو من يخرج إلى المعركة ليقاتل وليس بيده

\par 24 يرتدي درعًا؟ إذا تم العثور عليه، فهل سيُقتل لأنه احتقر

\par 25 خدمة الملك؟ كذلك لا يستطيع أحد أن يدخل القدس إذا كان

\par 26 ذو رأيين. من كان ذا رأيين في صلاته فهو ظلمة

\par 27 نفسه. وحتى الملائكة لا يثقون به. لذلك، كونوا مخلصين في الرب في كل الأوقات حتى تعرفوا كل لحظة

\chapter{2}

\par \textit{العنوان الفرعي للجزء المتبقي من النص}

\par 1 وأيضاً فيما يتعلق بملوك آشور وزوال السماء والأرض وما تحت الأرض:

\par \textit{ملك الظلم الآشوري}

\par 2 «لذلك الآن لن يُغلب (أولئك الذين هم لي)» يقول الرب،

\par 3 «ولن يخافوا في المعركة». عندما يرون [ملكًا] يقوم في الشمال، [سيُدعى] «ملك [آشور] و] «ملك الظلم»،

\par 4 [سيزيد] معاركه واضطراباته ضد مصر. ستدمر الأرض

\par 5 تأوهوا معًا لأن أولادكم سيؤخذون. كثيرون سيرغبون في الموت في تلك الأيام، لكن الموت يهرب منهم

\par \textit{ملك السلام الغربي}

\par 6 وسيقوم ملك يُدعى "ملك السلام" في الغرب

\par 7 سيجري على البحر كأسد زائر. سيقتل ملك الظلم، وسينتقم من مصر بمعارك وسفك دماء كثير

\par 8 [...]

\par 9 ويكون في تلك الأيام أنه يأمر بالسلام و

\par 10 هبة [باطلة] في مصر. [سيعطي] سلامًا لهؤلاء القديسين، [قائلين]،

\par 11 «اسم الله واحد». [سيُكرم] القديسين و

\par 12 مُرَفِّعًا إلى أماكن القديسين. سيُعطي عطايا باطلة لبيت الله

\par 13 سيتجول في مدن مصر بمكر، دون أن يعلموا

\par 14 سيحصي الأماكن المقدسة. سيزن أصنام الوثنيين. هو

\par 15 سيُحصي ثرواتهم، ويُقيم لهم كهنة، ويأمر بأخذ حكماء الشعب وأعظمائهم، ويُؤتى بهم إلى العاصمة التي على البحر، قائلين: «لا يوجد إلا واحد»

\par 16 اللغة." ولكن عندما تسمع، "السلام والفرح موجودان،" سأ...

\par \textit{الابن الشرير على اليمين}

\par 17 سأخبركم الآن بآياته لتعرفوه. لأن له ابنين:

\par 18 [...]

\par 19 واحد عن يمينه وواحد عن يساره. الذي عن يمينه سيتلقى ضربة شيطانية

\par 20 وجهه، (و) سيُقاتل ضد اسم الله. الآن سينزل أربعة ملوك

\par 21 من ذلك الملك. في عامه الثلاثين، سيصل إلى ممفيس، (و) سيفعل

\par 22 يبني معبدًا في ممفيس. في ذلك اليوم، سيقوم ابنه ضده و

\par 23 اقتله. ستضطرب الأرض كلها.

\par 24 في ذلك اليوم يصدر أمرا على كل الأرض فيقبض على كهنة الأرض وجميع القديسين، قائلا: "تعوضون كل من ظلمكم ضعفين".

\par 25 الهدية وكل الأشياء الجيدة التي أعطاك إياها والدي. سيصمت

\par 26 الأماكن المقدسة. سيأخذ بيوتهم. سيأسر أبناءهم. سيأمر بذبح الذبائح وارتكاب الرجاسات والشرور المريرة في

\par 27 الأرض. سيظهر أمام الشمس والقمر. في ذلك اليوم سيمزق كهنة الأرض ثيابهم

\par 28 [...]

\par \textit{رثاء لمصر في تلك الأيام}

\par 29 ويلٌ لكم يا حكام مصر في تلك الأيام، لأن أيامكم قد ولّت. سيُعاديكم ظلم الفقراء، وسيُقتل أبناؤكم.

\par 30 [...]

\par 31 غُصبت غنيمة. في تلك الأيام، تئن مدن مصر لصوت البائع، ولا يُسمع صوت المشتري. أسواق

\par 32 ستصبح مدن مصر مغبرة. سيبكي أهل مصر معًا. سيرغبون في الموت، (لكن) الموت سيهرب ويتركهم

\par 33 في تلك الأيام، سيركضون إلى الصخور ويقفزون عنها، قائلين: "اسقط على

\par 34 لنا." ومع ذلك لن يموتوا. بلاء مضاعف سيتضاعف على الأرض كلها

\par 35 في تلك الأيام، يأمر الملك، ويُقبض على جميع الديدان المرضعة ويُؤتى بها إليه مقيدة. سيرضعن الثعابين. ويُسفك دمهن

\par 36 يُسحب من صدورهم، ويُوضع كسم على السهام. وبسبب محنة المدن، سيأمر مرة أخرى، ويُقبض على جميع الصبية من سن اثنتي عشرة سنة فما دون، ويُقدمون لتعليمهم رمي السهام

\par 37 القابلة التي على الأرض ستحزن.

\par المرأة التي ولدت سترفع عينيها إلى السماء،

\par قائلا، "لماذا جلست على كرسي الولادة،

"%\par لإنجاب ابن إلى الأرض؟"

\par 38 ستفرح العاقر والعذراء،

\par قائلة: "حان وقت فرحنا،

\par لأنه ليس لدينا طفل على الأرض،

\par ولكن أطفالنا في الجنة."

\par عودة اليهود إلى القدس

\par 39 في تلك الأيام، سيقوم ثلاثة ملوك على الفرس، وسيأسرون اليهود الذين في مصر، وسيأتون بهم إلى أورشليم، وسيسكنونها ويقيمون هناك

\par \textit{جانبًا عظيًا}

\par 40 حينئذٍ، متى سمعتم أن هناك أمانًا في أورشليم، فمزقوا ثيابكم يا كهنة الأرض، لأن ابن الهلاك سيأتي قريبًا

\par \textit{نبوءة مختصرة تذكر الخارج عن القانون}

\par 41 في تلك الأيام، سيظهر الخارج عن القانون في الأماكن المقدسة—

\par \textit{الحروب الفارسية الآشورية}

\par 42 في (تلك) الأيام، سيُسرع ملوك الفرس، وسيقفون للقتال

\par 43 مع ملوك آشور. سيقاتل أربعة ملوك ثلاثة. سيقضون ثلاث سنوات في ذلك المكان حتى ينهبوا ثروة الهيكل التي في ذلك المكان

\par 44 في تلك الأيام، سيتدفق من كوسو إلى ممفيس. وسيتحول نهر مصر إلى دم، ولن يتمكنوا من الشرب منه لمدة ثلاثة أيام

\par 45 ويل لمصر ومن فيها.

\par 46 في تلك الأيام، سيقوم ملك في المدينة التي تُدعى "مدينة الشمس"، وستضطرب الأرض كلها. (سيهرب) إلى ممفيس (مع بلاد فارس).

\par \textit{الانتصار الفارسي}

\par 47 في السنة السادسة، سيُدبِّر ملوك الفرس كمينًا في ممفيس. سيفعلون

\par 48 اقتل الملك الآشوري. سينتقم الفرس من الأرض، وسيأمرون بقتل جميع الوثنيين والمجرمين. سيأمرون بـ

\par 49 يبنون هياكل القديسين. سيعطون ضعف عطايا بيت الله

\par 50 سيقولون: "ربُّ الله واحد". ستُهَلِّل الأرض كلها للفرس

\par \textit{حكم الملك الصالح من مدينة الشمس}

\par 51 حتى البقية التي لم تمت تحت الضيقات ستقول: "لقد خلصنا الرب"

\par 52 أرسل لنا ملكًا صالحًا حتى لا تصبح الأرض صحراء. سيأمر بعدم تقديم أي أمر ملكي لمدة ثلاث سنوات وستة أشهر

\par 53 ستمتلئ الأرض بالخير في رخاء وافر. سيذهب الأحياء إلى الأموات قائلين: "قوموا وكونوا معنا في هذه الراحة."

\chapter{3}

\par \textit{مجيء ابن الإثم}

\par 1 في السنة الرابعة من حكم ذلك الملك، سيظهر ابن الإثم، قائلًا: "أنا المسيح"، مع أنه ليس كذلك. لا تُصدّقوه!

\par \textit{استطراد بشأن مجيء المسيح الحقيقي}

\par 2 عندما يأتي المسيح، سيأتي على شكل سرب من الحمائم، محاطًا بإكليل الحمائم. سيمشي على قبب السماء وعلامة الصليب تقوده

\par 3 سينظر إليه العالم أجمع كالشمس التي تشرق من الأفق الشرقي إلى الأفق الغربي

\par 4 هكذا سيأتي، وجميع ملائكته يحيطون به.


\par \textit{أعمال المسيح الدجال}

\par 5 لكن ابن الإثم سيبدأ بالوقوف مرة أخرى في الأماكن المقدسة

\par 6 سيقول للشمس: "اسقِطي"، فتسقط.

\par سيقول: "اسقِطي"، فتفعل

\par سيقول "أظلم" فيفعل ذلك.

\par 7 سيقول للقمر: "اصبح دمويًا"، وسيفعل ذلك.

\par 8 ويخرج معهم من السماء.

\par ويمشي على البحر والأنهار كما على اليابسة.

\par 9 سيجعل العرج يمشي.

\par سيجعل الصم يسمعون

\par فإنه سيجعل الأخرس يتكلم.

\par فإنه سيجعل الأعمى يبصر.

\par 10 يُطهِّرُ الْجُذَمَاءَ.

\par يُشْفِي الْمَرْضَى

\par الشياطين سوف يطردها.

\par 11 ويكثر آياته وعجائبه أمام كل إنسان

\par 12 سيعمل الأعمال التي عملها المسيح، ما عدا إقامة الأموات فقط

\par 13 بهذا تعرفون أنه ابن الإثم، لأنه لا يقدر أن يحيي

\par \textit{علامات المسيح الدجال}

\par 14 لأني ها أنا أخبركم بآياته لتعرفوه.

\par 15 هو ... فتى نحيف الساقين، خصلة من الشعر الرمادي في مقدمة رأسه الأصلع. حاجباه يصلان إلى أذنيه. وفي مقدم يديه بقعة جرداء.

\par 16 سيتغير في حضور من يرونه. سيصبح طفلاً صغيراً. سيصبح عجوزاً

\par 17 سيتغير في كل علامة. لكن علامات رأسه لن تكون قادرة على التغيير

\par 18 هناك ستعرف أنه ابن الإثم.

\chapter{4}

\par \textit{استشهاد طابيثا}

\par 1 ستسمع العذراء، التي اسمها طابيثا، أن الرجل عديم الحياء قد ظهر في الأماكن المقدسة. وستلبس ثوبها الكتاني

\par 2 وستطارده إلى اليهودية، وتوبخه إلى أورشليم قائلة: «يا وقحًا، يا ابن الإثم، يا من كنتَ مُعاديًا لجميع القديسين».

\par 3 حينئذٍ يغضب الوقح على العذراء. ويطاردها إلى جهات الغروب. ويمتص دمها في المساء

\par 4 ويطرحها على الهيكل، فتصبح شفاءً للشعب

\par 5 ستستيقظ عند الفجر. وستعيش وتوبخه قائلة: "يا وقح، ليس لك سلطان على نفسي ولا على جسدي، لأني أحيا في الرب دائمًا

\par 6 "وأيضاً دمي الذي سكبته على الهيكل صار شفاءً للشعب."


\par \textit{استشهاد إيليا وحنوك}

\par 7 "ثم عندما يسمع إيليا وحنوك أن الرجل القاسي قد ظهر في المكان المقدس، ينزلان ويقاتلانه قائلين:

\par 8 ألا تخجلون حقًا؟ عندما تلتصقون بالقديسين، لأنكم دائمًا غرباء.

\par 9 لقد كنتم عدائيين تجاه أهل السماء، وعملتم ضد أهل الأرض.

\par 10 لقد كنت معاديا للعروش، وتصرفت ضد الملائكة، وأنت غريب دائما.

\par 11 لقد سقطت من السماء مثل نجوم الصبح. لقد تغيرت، وأصبحت قبيلتك مظلمة بالنسبة لك.

\par 12 ولكنك لا تخجل، عندما تقف بثبات ضد الله فأنت شيطان.

\par 13 يسمع الرجل عديم الحياء فيغضب، ويقاتلهم في سوق المدينة العظيمة. ويقضي سبعة أيام في قتالهم

\par 14 وسيقضون ثلاثة أيام ونصف في السوق أمواتًا، بينما يراهم جميع الناس

\par 15 ولكن في اليوم الرابع يقومون ويوبخونه قائلين: "يا وقح، يا ابن الإثم! ألا تخجل من نفسك لأنك تضل شعب الله الذي لم تتألم من أجله؟ ألا تعلم أننا نعيش في الرب؟"

\par 16 وبينما كان يُقال، غلبوه قائلين: "سنضع الجسد بدلاً من الروح، وسنقتلك لأنك لا تستطيع الكلام في ذلك اليوم لأننا أقوياء دائمًا في الرب. ولكنك دائمًا أعداء لله."

\par 17 سيسمع الوقح، وسيغضب ويقاتلهم.

\par 18 وستحيط بهم المدينة كلها.

\par 19 في ذلك اليوم، سيصرخون إلى السماء وهم يلمعون بينما يراهم كل الناس وكل العالم


\par \textit{اضطهاد القديسين}

\par 20 لن يسود عليهم ابن الإثم. سيغضب على الأرض، وسيسعى للخطيئة ضد الشعب

\par 21 سيطارد جميع القديسين. سيُعادون هم وكهنة الأرض مقيدين

\par 22 سيقتلهم ويهلكهم... ويقطع عيونهم بمسامير من حديد

\par 23 سينزع جلودهم عن رؤوسهم. وسيزيل أظافرهم واحدًا تلو الآخر. وسيأمر بوضع الخل والليمون في أنوفهم

\par 24 الآن أولئك الذين لا يستطيعون تحمل عذابات ذلك الملك سيأخذون الذهب ويهربون عبر المخاضات إلى الأماكن الصحراوية. سيستلقون كالنائم

\par 25 سيستقبل الرب أرواحهم ونفوسهم لنفسه.

\par 26 سيتحجر لحمهم، ولن يأكلهم أي حيوان بري حتى يوم الدينونة العظيمة.

\par 27 وسيقومون ويجدون مكان راحة. لكنهم لن يكونوا في ملكوت المسيح كأولئك الذين صبروا لأن الرب قال: "سأعطيهم أن يجلسوا عن يميني".

\par 28 سينالون حظوة على الآخرين، وسينتصرون على ابن الإثم. وسيشهدون انحلال السماء والأرض

\par 29 سينالون عروش المجد والتيجان.


\par \textit{استشهاد الستين بارًا}


\par 30 سيسمع الستون بارًا المستعدون لهذه الساعة.

\par 31 فيتشبثون بدرع الرب، ويركضون إلى أورشليم ويقاتلون ذلك الرجل القاسي، قائلين: «لقد فعلتَ كل ما صنعه الأنبياء منذ البدء، لكنك لم تستطع إحياء الموتى، لأنك لا تملك القدرة على إحياء الموتى. بهذا عرفنا أنك ابن الإثم».

\par 32 فيسمع فيغضب ويأمر بإشعال المذابح

\par 33 وسيُقيَّد الأبرار. سيُرفعون ويُحرقون

\chapter{5}

\par \textit{الرجال يفرون من المسيح الدجال}

\par 1 وفي ذلك اليوم، ستقسو قلوب كثيرين، فيهربون منه قائلين: «ليس هذا هو المسيح. المسيح لا يقتل الصديق، ولا يطارد الناس ليطلبهم، بل يُقنعهم بالآيات والعجائب».

\par \textit{إزالة الصالحين}

\par 2 في ذلك اليوم، سيشفق المسيح على خاصته. وسيرسل من السماء أربعة وستين ألفًا من ملائكته، لكل منهم ستة أجنحة

\par 3 سيُحرك الصوت السماء والأرض عندما يُسبّحون ويُمجّدون

\par 4 الآن، أولئك الذين كُتب اسم المسيح على جباههم وخُتم على أيديهم الصغير والكبير، سيُرفعون على أجنحتهم ويرتفعون أمام غضبه

\par 5 ثم سيصبح جبرائيل وأورييل عمود نور يقودهم إلى الأرض المقدسة

\par 6 سيُمنح لهم أن يأكلوا من شجرة الحياة. سيلبسون ثيابًا بيضاء... وستحرسهم الملائكة. لن يعطشوا، ولن يقوى عليهم ابن الإثم

\par \textit{الكوارث الطبيعية التي أعقبت إزالة الصالحين}

\par 7 وفي ذلك اليوم ستضطرب الأرض، وستُظلم الشمس، وسيُنزع السلام من الأرض

\par 8 ستسقط الطيور على الأرض ميتة.

\par 9 ستجف الأرض. وستجف مياه البحر

\par 10 سيتأوه الخطاة على الأرض قائلين: "ماذا فعلت بنا يا ابن الإثم، قائلًا إني المسيح، وأنت إبليس؟

\par 11 أنت غير قادر على خلاص نفسك حتى تخلصنا. لقد صنعت آيات أمامنا حتى أبعدتنا عن المسيح الذي خلقنا. ويل لنا لأننا استمعنا إليك

\par 12 ها نحن الآن نموت في مجاعة. أين أثر البار فنعبده، أو أين من يعلمنا فنلجأ إليه

\par 13 والآن سنُهلك بغضب لأننا عصينا الرب

\par 14 ذهبنا إلى أعماق البحر، ولم نجد ماءً. حفرنا في الأنهار وقصب البردي، ولم نجد ماءً


\par \textit{رثاء المسيح الدجال وملاحقة الصالحين}


\par 15 ثم في ذلك اليوم يتكلم الوقح قائلاً: ويل لي لأن أجلي قد مضى وأنا أقول إن أجلي لن يمر

\par 16 أصبحت سنيني شهورًا، ومرت أيامي كما يمر الغبار. والآن سأهلك معكم

\par 17 الآن اركض إلى البرية. أمسك اللصوص واقتلهم

\par 18 أقموا القديسين. لأنه بسببهم تُثمر الأرض، ولأن الشمس تُشرق على الأرض. ولأن الندى ينزل على الأرض بسببهم

\par 19 سيبكي الخطاة قائلين: «لقد جعلتنا أعداءً لله. إن استطعت، فقم واتبعهم».

\par 20 ثم يأخذ أجنحته النارية ويطير وراء القديسين. سيقاتلهم مرة أخرى

\par 21 سيسمع الملائكة وينزلون. سيخوضون معه معركة سيوف كثيرة

\par \textit{النار الكونية}

\par 22 ويكون في ذلك اليوم أن الرب يسمع، فيأمر السماء والأرض بغضب عظيم، فيرسلون إلى النار

\par 23 وستسود النار على الأرض اثنين وسبعين ذراعًا. ستأكل الخطاة والشياطين كالقش

\par 24 سيحدث دينونة حقيقية.

\par \textit{كلمة الدينونة القادمة}

\par 25 في ذلك اليوم، تنطق الجبال والأرض بكلام. وتتحدث الطرقات بعضها مع بعض قائلة: "هل سمعتم اليوم صوت إنسان ماشي لم يأتِ إلى دينونة ابن الرب؟"

\par 26 ستقف خطايا كل واحد عليه في المكان الذي ارتكبت فيه، سواء كانت خطايا النهار أو خطايا الليل

\par 27 أولئك الذين ينتمون إلى الصالحين و... سيرون الخطاة والذين اضطهدوهم والذين سلموهم إلى الموت في عذاباتهم

\par 28 حينئذٍ سيرى الخطاة [في العذاب] مكان الأبرار.

\par 29 وهكذا ستحدث النعمة. وفي تلك الأيام، سيُعطى لهم ما يطلبه الصالحون مرارًا.

\par \textit{حكم وإعدام المسيح الدجال}

\par 30 في ذلك اليوم، سيدين الرب السماء والأرض. سيدين الذين أخطأوا في السماء، والذين فعلوا ذلك على الأرض

\par 31 سيحكم على رعاة الشعب. سيسأل عن قطيع الغنم، فيُعطى له، دون أن يكون فيها أي غش قاتل

\par 32 بعد هذه الأمور، سينزل إيليا وحنوك. سيطرحان جسد العالم، وسيأخذان جسدهما الروحي. سيطاردان ابن الإثم ويقتلانه لأنه لا يستطيع الكلام

\par 33 في ذلك اليوم، سيذوب في حضرتهم كما يذوب الجليد بالنار. سيهلك كالحية التي لا نفس فيها

\par 34 سيقولون له: "لقد مضى وقتك. والآن، سيهلك من يؤمنون بك."

\par 35 سيتم إلقاؤهم في قاع الهاوية وسيتم إغلاقها عليهم

\par 36 في ذلك اليوم، سيخرج المسيح والملك وجميع قديسيه من السماء

\par \textit{العصر الألفي}

\par 37 سيحرق الأرض. سيقضي عليها ألف عام.

\par 38 لأن الخطاة غلبوا عليه، فسيخلق سماءً جديدةً وأرضًا جديدةً، ولن يوجد فيهما شيطانٌ مميت.

\par 39 سيحكم مع قديسيه، صاعدين ونازلين، وهم دائمًا مع الملائكة ومع المسيح لألف عام

\end{document}