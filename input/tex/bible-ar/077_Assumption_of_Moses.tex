\begin{document}

\title{صعود موسى}

\chapter{1}

\par 1 "وصية موسى حتى الأشياء التي أمر بها في السنة المائة والعشرين من حياته،
\par 2 هذه هي السنة ألفين وخمسمائة من خلق العالم:
\par 3 [ولكن بحسب الحسابات الشرقية فإن العام ألفين وسبعمائة والعام أربعمائة بعد الخروج من فينيقيا]،
\par 4 ولما خرج الشعب بعد خروج موسى إلى عمان عبر الأردن،
\par 5 في النبوة التي قالها موسى في سفر التثنية:
\par 6 فدعا إليه يشوع بن نون رجلاً مقبولاً عند الرب.
\par 7 لكي يكون خادماً للشعب ولمسكن الشهادة مع كل أقداسه.
\par 8 ولكي يُدخل الشعب إلى الأرض التي أُعطيت لآبائهم،
\par 9 لكي يُعطى لهم حسب العهد والقسم اللذين تكلم بهما في المسكن أن يُعطيهما بيد يشوع قائلاً ليشوع هذه الكلمات:
\par 10 "(كونوا أقوياء) وشجعان لكي تفعلوا بكل قوتكم كل ما أُمرتم به لكي تكونوا بلا لوم أمام الله."
\par 11 هكذا قال رب العالمين.
\par 12 لأنه خلق العالم من أجل شعبه.
\par 13 ولكنه لم يسر أن يظهر هذا الهدف من الخلق منذ تأسيس العالم، حتى يوبخ الأمم بذلك، بل حتى يوبخ بعضهم بعضاً بحججهم لإذلالهم.
\par 14 "لذلك صممني وصممني وأعدّني قبل تأسيس العالم لأكون وسيطاً لعهده."
\par 15 "والآن أعلن لكم أن زمن سني حياتي قد كمل وأني أنتقل إلى النوم مع آبائي أمام كل الشعب."
\par 16 "وخذوا هذه الكتابة لكي تعرفوا كيف تحفظون الكتب التي أسلمها إليكم."
\par 17 "وترتبها وتدهنها بزيت الأرز وتضعها في أوانٍ خزفية في المكان الذي عمله منذ بدء خلق العالم."
\par 18 لكي يُدعى اسمه إلى يوم التوبة في الافتقاد الذي سيفتقدهم به الرب في استيفاء نهاية الأيام.

\chapter{2}

\par 1 "والآن يمرون بكم إلى الأرض التي عينها ووعد أن يعطيها لآبائهم،
\par 2 "فيها تباركهم وتعطيهم فردا فردا وتثبت لهم ميراثهم فيّ وتثبت لهم المملكة وتقيم لهم حكاما محليين حسب مسرة ربهم في الحق والعدل."
\par 3 وبعد خمس سنوات من دخولهم الأرض، يكونون بعد ذلك تحت حكم رؤساء وملوك لمدة ثماني عشرة سنة، وخلال تسع عشرة سنة تنفصل الأسباط العشرة.
\par 4 وينزل الأسباط الاثنا عشر وينقلون خيمة الشهادة. فيبني إله السماء دار مسكنه وبرج مقدسه، ويقيم هناك السبطان المقدسان.
\par 5 وأما العشرة الأسباط فتقيم لأنفسها ممالك حسب أحكامها.
\par 6 ويقدمون ذبائح عشرين سنة.
\par 7 وسبعة يحصنون الأسوار، وأنا أحمي تسعة، ولكن أربعة يتعدون عهد الرب، وينتهكون القسم الذي قطعه الرب معهم.
\par 8 ويذبحون بنيهم لآلهة أخرى، ويضعون أصناماً في المقدس لعبادتها.
\par 9 وفي بيت الرب يعملون الفجور وينقشون كل شكل من أشكال البهائم، حتى الكثير من الرجاسات.

\chapter{3}

\par 1 وفي تلك الأيام يأتي عليهم ملك من المشرق، وتغطي فرسانه أرضهم.
\par 2 ويحرق مستعمرتهم بالنار مع هيكل الرب المقدس، ويأخذ جميع الآنية المقدسة.
\par 3 ويطرد كل الشعب ويأخذهم إلى أرض ميلاده ويأخذ السبطين معه.
\par 4 ثم يدعو السبطان الأسباط العشرة، ويسيران كاللبؤة على السهول المغبرة، جائعين وعطشى.
\par 5 فيصرخون بصوت عال: "بار وقدوس هو الرب، لأنه بقدر ما أخطأتم، فقد حملنا نحن أيضًا معكم مع أولادنا".
\par 6 ثم تنوح الأسباط العشرة عند سماعها إهانة السبطين،
\par 7 فيقولون ماذا فعلنا بكم أيها الإخوة ألم يأت هذا الضيق على كل بيت إسرائيل؟
\par 8 وتبكي جميع الأسباط صارخين إلى السماء قائلين:
\par 9 "يا إله إبراهيم وإله إسحق وإله يعقوب، اذكر عهدك الذي قطعته معهم والقسم الذي أقسمت لهم بنفسك أن لا يزول نسلهم من الأرض التي أعطيتهم إياها."
\par 10 "فيذكرونني في ذلك اليوم قائلين من سبط إلى سبط وكل واحد إلى صاحبه."
\par 11 أليس هذا ما أنبأنا به موسى حينئذ في النبوات وهو الذي عانى كثيرا في مصر وفي البحر الأحمر وفي البرية مدة أربعين سنة؟
\par 12 "وشهد علينا السماء والأرض أن لا نتعدى وصاياه التي كان وسيطا لنا فيها؟"
\par 13 "ها هي هذه الأمور قد حدثت لنا بعد وفاته حسب تصريحه، كما أعلن لنا في ذلك الوقت، نعم، ها هي هذه قد حدثت حتى تم أسرنا إلى بلاد المشرق."
\par 14 ويكون أيضاً في العبودية نحو سبعين سنة.

\chapter{4}

\par 1 ثم يدخل عليهم من هو عليهم فيبسط يديه ويركع على ركبتيه ويصلي لأجلهم قائلا:
\par 2 "أيها الرب الجميع، الملك على العرش العالي، الذي يحكم العالم، وقد أردت أن يكون هذا الشعب شعبك المختار، فأنت (حقًا) أردت أن تُدعى إلههم حسب العهد الذي قطعته مع آبائهم."
\par 3 "ولكنهم ذهبوا في سبى إلى أرض أخرى مع نسائهم وأولادهم، وحول أبواب شعوب غريبة وحيث الباطل عظيم."
\par 4 "انظر إليهم وارحمهم يا رب السماء."
\par 5 فيذكرهم الله بسبب العهد الذي قطعه مع آبائهم، ويظهر رحمته في تلك الأوقات أيضًا.
\par 6 "ويضع في قلب الملك أن يرحمهم، فيرسلهم إلى أرضهم وبلادهم."
\par 7 ثم يصعد بعض الأسباط ويأتون إلى مكانهم المعين، ويحيطون المكان من جديد بأسوار.
\par 8 ويستمر السبطان على إيمانهما، حزينين ونائحين، لأنهما لن يستطيعا تقديم ذبائح لرب آبائهما. وتنمو الأسباط العشرة وتكثر بين الأمم في زمن سبيهم.

\chapter{5}

\par 1 "وعندما تقترب أوقات العقاب وينشأ الانتقام من الملوك الذين يشاركونهم في ذنبهم ويعاقبونهم،
\par 2 وهم أيضًا سوف ينقسمون بشأن الحقيقة.
\par 3 ولذلك قيل: «ينحرفون عن البر ويقتربون من الإثم وينجسون بيت عبادتهم بالنجاسات»، و«يزنون أنفسهم بآلهة غريبة».
\par 4 لأنهم لن يتبعوا حقيقة الله، بل ينجس المذبح بالتقدمات التي يقدمونها للرب، وهم ليسوا كهنة بل عبيد وأبناء عبيد.
\par 5 وكثيرون في تلك الأوقات يحترمون الأشخاص المرغوب فيهم ويقبلون الهدايا، ويحرفون الحكم [عند تلقي الهدايا].
\par 6 "وعلى هذا الأساس تمتلئ المستعمرة وحدود مسكنهم بالأعمال غير القانونية والآثام: أولئك الذين يحيدون عن الرب سيكونون قضاة: سيكونون مستعدين للحكم على الأموال كما يرغب كل منهم."

\chapter{6}

\par 1 ويقوم لهم ملوك يتولون الحكم ويسمون أنفسهم كهنة الله العلي. يعملون إثماً في قدس الأقداس.
\par 2 ويخلفهم ملك متغطرس، لا يكون من جنس الكهنة، رجل جريء وقح، فيحكم عليهم كما يستحقون.
\par 3 ويقطع رؤوسهم بالسيف ويبيدهم في أماكن مخفية حتى لا يعلم أحد أين جثثهم.
\par 4 ويقتل الشيخ والشاب ولا يشفق.
\par 5 فيكون خوفه مراً عليهم في أرضهم.
\par 6 فينفذ عليهم أحكاما كما فعل بهم المصريون أربعا وثلاثين سنة ويعاقبهم.
\par 7 ويولد له أولاد، يحكمون بعده لمدة أقصر.
\par 8 وسوف تأتي إلى أجزائهم أفواج وملك قوي من الغرب، الذي سوف يقهرهم:
\par 9 فيأخذهم سبايا ويحرق جزءا من هيكلهم بالنار ويصلب من حول مستعمرتهم.

\chapter{7}

\par 1 وعندما يتم ذلك فإن الزمن سينتهي، وفي لحظة واحدة سوف ينتهي المسار (الثاني)، وسوف تأتي الساعات الأربع.
\par 2 وسوف يضطرون...
\par 3 وفي هذا الزمان سوف يحكم رجال مفسدون وفاسقون، يقولون إنهم عادلون.
\par 4 "وهؤلاء سيثيرون السم في نفوسهم، وهم رجال غادرون، مهتمون بإرضاء أنفسهم، مخادعون في كل شؤونهم الخاصة، ومحبون للولائم في كل ساعة من النهار، شرهون، ذوو ذوق رفيع."
\par 5 . . .
\par 6 آكلي أموال الفقراء قائلين إنهم يفعلون ذلك من أجل عدالتهم،
\par 7 بل في الحقيقة لتدميرهم، المشتكين، المخادعين، الذين يخفون أنفسهم حتى لا يتم التعرف عليهم، غير الأتقياء، المملوءين بالفوضى والإثم من شروق الشمس إلى غروبها.
\par 8 قائلين: «سنقيم حفلات ومناسبات فاخرة، ونأكل ونشرب، وسنعتبر أنفسنا أمراء».
\par 9 "وإن كانت أيديهم وأرواحهم تمس أشياء نجسة، فإن أفواههم تتكلم بعظائم، ويقولون أيضاً:
\par 10 "لا تلمسني لئلا تنجسني في المكان الذي أقف فيه..."

\chapter{8}

\par 1 "ويأتي عليهم غضب وضيق لم يحدثا مثلهما منذ البدء إلى ذلك الوقت الذي فيه يصعد عليهم ملك ملوك الأرض وذو سلطان عظيم الذي يصلب الذين يعترفون بختانهم.
\par 2 وأما الذين يكتمونه فيعذبهم ويسلمهم مقيدين ويساقون إلى السجن.
\par 3 وتُعطى زوجاتهم للآلهة بين الأمم، وتُجرى لأبنائهم الصغار عمليات جراحية على أيدي الأطباء لإخراج غلفتهم.
\par 4 وأما غيرهم فيعاقبون بالعذاب والنار والسيف، ويحملون أصنامهم على الملأ نجسة كحال أصحابها.
\par 5 وسوف يُجبرون أيضًا من قبل أولئك الذين يعذبونهم على دخول قدس أقداسهم، وسوف يُجبرون بالمخاوف على التجديف بوقاحة على الكلمة، وأخيرًا بعد هذه الأشياء على القوانين وما كان لديهم فوق مذبحهم.

\chapter{9}

\par 1 "وفي ذلك اليوم يكون رجل من سبط لاوي اسمه تاكسو، وله سبعة بنين، وهو يكلمهم ويوصيهم قائلا:
\par 2 "انظروا يا أبنائي، هوذا زيارة ثانية قاسية وغير طاهرة قد أتت على الشعب، وعقاب بلا رحمة ويفوق بكثير العقاب الأول."
\par 3 "فأية أمة أو منطقة أو شعب من الذين لا يفعلون الخير للرب، والذين فعلوا الكثير من الرجاسات، عانوا من كوارث عظيمة كما أصابتنا؟"
\par 4 "والآن يا أبنائي اسمعوا لي، وانظروا واعلموا أن الآباء ولا أجدادهم جربوا الله حتى يتعدوا على وصاياه."
\par 5 "وأنتم تعلمون أن هذه هي قوتنا، ولهذا سنفعل."
\par 6 فلنصم ثلاثة أيام، وفي اليوم الرابع نذهب إلى مغارة في الحقل، فنموت على أن نتعدى وصايا رب الأرباب إله آبائنا.
\par 7 "فإن فعلنا هذا ومتنا، فسوف يُنتقم لدمائنا أمام الرب."

\chapter{10}

\par 1 وحينئذٍ ستظهر مملكته في كل خليقته، وحينئذٍ لن يكون هناك شيطان، وسيذهب الحزن معه.
\par 2 حينئذ تمتلئ يدا الملاك الذي صار رئيساً، وينتقم لهم فوراً من أعدائهم.
\par 3 لأنه سيقوم السماوي من عرشه الملكي، ويخرج من مسكن قدسه بسخط وغضب من أجل أبنائه.
\par 4 وترتجف الأرض، وتهتز إلى حدودها، وتهبط الجبال العالية، وتهتز التلال وتسقط.
\par 5 وتنكسر قرون الشمس فتتحول إلى ظلام، والقمر لا يعطي ضوءه، فيتحول كله إلى دم، ودائرة النجوم تتزعزع.
\par 6 ويتراجع البحر إلى الهاوية، وتنضب ينابيع المياه، وتجف الأنهار.
\par 7 لأنه سوف يقوم العلي، الإله الأزلي وحده، وسوف يظهر لمعاقبة الأمم، وسوف يدمر كل أصنامهم.
\par 8 "حينئذٍ تفرح يا إسرائيل، وتصعد على أعناق وأجنحة النسر، وتفنى."
\par 9 ويرفعك الله، ويقرّبك إلى سماء النجوم، إلى مسكنها.
\par 10 وتنظرون من العلاء وتنظرون أعداءكم في جهنم فتعرفونهم وتفرحون وتحمدون خالقكم وتعترفون به.
\par 11 وأنت يا يشوع بن نون تحفظ هذه الكلمات وهذا الكتاب.
\par 12 فمن موتي إلى مجيئه يكون 250 مرة [= سنة - أسابيع = 1750 سنة].
\par 13 وهذا هو مجرى الأزمنة التي سوف يتبعونها حتى يتمموا.
\par 14 وسأذهب للنوم مع آبائي.
\par 15 لذلك يا يشوع بن نون، كن قوياً وتشجع، لأن الله اختارك لتكون خادماً في نفس العهد.

\chapter{11}

\par 1 ولما سمع يشوع كلام موسى المكتوب في كتابته كل ما قاله من قبل، مزق ثيابه وألقى نفسه عند رجلي موسى.
\par 2 فعزاه موسى وبكى معه.
\par 3 فأجابه يشوع وقال:
\par 4 لماذا تُعزيني يا سيدي موسى؟ وكيف أُعزى بالكلام المُرّ الذي تكلمت به، الذي خرج من فمك، المملوء دموعًا ونحيبًا، بانفصالك عن هذا الشعب؟
\par 5 "(ولكن الآن) أي مكان يستقبلكم؟"
\par 6 "أو ما هي العلامة التي تشير إلى قبرك؟"
\par 7 "أو من يجرؤ على نقل جسدك من هناك كجسد إنسان عادي من مكان إلى آخر؟"
\par 8 "لأن كل إنسان حين يموت يكون له حسب عمره قبور على الأرض، ولكن قبرك هو من مشرق الشمس إلى غروبها، ومن الجنوب إلى أقاصي الشمال: كل العالم هو قبرك."
\par 9 يا سيدي أنت ذاهب، فمن يرعى هذا الشعب؟
\par 10 "أو من هو الذي يرحمهم ويهديهم في الطريق؟"
\par 11 "أو من يصلي لأجلهم ولا يتأخر يوما واحدا حتى أدخلهم إلى أرض آبائهم؟"
\par 12 "فكيف إذن أرعى هذا الشعب كأب ابنه الوحيد، أو كعشيقة ابنتها العذراء التي تستعد للزواج الذي ستحترمه، بينما تحرس شخصها من الشمس وتحرص على أن لا تكون قدميها حذيفتين للجري على الأرض."
\par 13 "وكيف أطعمهم وأشربهم حسب شهواتهم؟"
\par 14 "فمنهم ستمائة ألف رجل، فقد تضاعفوا إلى هذا الحد بفضل صلواتك يا سيدي موسى."
\par 15 "وأي حكمة أو فهم لي حتى أقضي أو أجيب بالكلام في البيت (الرب)؟"
\par 16 "ويسمع ملوك الأموريين أيضًا أننا نهاجمهم، ويعتقدون أنه لم يعد بينهم الروح القدس الذي كان مستحقًا للرب، المتنوع وغير المدرك، رب الكلمة، الذي كان أمينًا في كل شيء، رئيس رسل الله في كل الأرض، المعلم الأكثر كمالًا في العالم، [أنه لم يعد بينهم]، سيقولون: "دعونا نذهب ضدهم".
\par 17 "إذا ارتكب العدو ولو مرة واحدة فعلًا شريرًا ضد ربه، فلن يكون لديه من يدافع عنه ليقدم الصلوات نيابة عنه إلى الرب، مثل موسى الرسول العظيم، الذي كان كل ساعة ليلًا ونهارًا يثبت ركبتيه على الأرض، يصلي ويبحث عن المساعدة من الذي يحكم العالم كله بالرحمة والبر، ويذكره بعهد الآباء ويسترضي الرب بالقسم".
\par 18 فيقولون: ليس معهم، فلنذهب ونبيدهم عن وجه الأرض. فماذا يكون مصير هذا الشعب يا سيدي موسى؟

\chapter{12}

\par 1 ولما فرغ يشوع من هذه الكلمات ألقى بنفسه أيضاً عند قدمي موسى.
\par 2 فأخذ موسى يده وأقامه على الكرسي أمامه وأجاب وقال له:
\par 3 "يا يشوع لا تحتقر نفسك بل اطمئن واسمع كلامي."
\par 4 "لقد خلق الله جميع الأمم التي على الأرض، وخلقنا نحن أيضًا، وتوقع حدوثها ورآها منذ بداية خلق الأرض إلى نهاية الدهر، ولم يهمل شيئًا حتى أقل شيء، بل تنبأ بكل شيء وجعله يحدث."
\par 5 "(نعم) كل الأشياء التي ستكون على هذه الأرض قد تنبأ بها الرب، وها هي تُعرض (إلى النور)..."
\par 6 "لقد عيّنني الرب نيابة عنهم للصلاة من أجل خطاياهم والشفاعة من أجلهم."
\par 7 "لأنه ليس من أجل فضيلة أو قوة مني، بل من أجل مسرته سقطت رحمته وطول أناته في نصيبي."
\par 8 "لأني أقول لك يا يشوع: ليس لأجل تقوى هذا الشعب تستأصل الأمم."
\par 9 "أضواء السماء وأساسات الأرض مصنوعة ومقبولة من الله وهي تحت خاتم يده اليمنى."
\par 10 "لذلك فإن الذين يعملون بوصايا الله وينفذونها سوف ينمون ويزدهرون"
\par 11 "ولكن أولئك الذين يخطئون ويتجاهلون الوصايا سوف يحرمون من البركات المذكورة أعلاه، وسوف يعاقبون بالعديد من العذابات من قبل الأمم."
\par 12 "ولكن اقتلاعها وتدميرها بالكامل أمر غير مسموح به."
\par 13 "لأن الله الذي سبق فرأى كل شيء إلى الأبد سيخرج، وقد تم تثبيت عهده والقسم الذي ..."

\end{document}