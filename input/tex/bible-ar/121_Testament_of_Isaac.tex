\begin{document}


\title{وصية إسحاق}

\chapter{1}

\par 1 بسم الآب والابن والروح القدس الإله الواحد.

\par 2 نبدأ بمعونة الله ومن خلال وساطته للاحتفال بوفاة البطريرك إسحاق، ابن البطريرك إبراهيم، وصعوده من

\par 3 في هذا اليوم نفسه، وهو الثامن والعشرون من شهر مصر. بركة شفاعته تكون معنا وتحفظنا من تجارب العدو! آمين!

\par 4 [وكتب البطريرك إسحاق وصيته ووجه كلماته الإرشادية إلى يعقوب ابنه وجميع المجتمعين معه.]

\par 5 قال: «اسمعوا يا إخوتي وأحبائي تعليم هذا الخطيب وهذا الدواء الشافي

\par 6 لأن طريق الله مستمر إلى الأبد، فاسمعوا ليس فقط بآذان جسدية عفيفة، بل أيضًا بعمق القلب وبإيمان حقيقي دون أي شك، كما هو مكتوب: "ها قد سمعتم كلمة حق عما ينبغي أن يصبح عليه الإنسان. إن سمعها بقلب طاهر، فسيرحمه الله عندما يطلب منه شيئًا."

\par 7 «ومكتوب أيضًا: «لا فائدة لأحد أن يسأل الله ما يطلبه الناس على الأرض». وإن كان الله قد أعطانا سلطانًا على الأرض، فكم ينتفع من كان راسخًا في الإيمان بكلمة الله، وتمسك بثبات بقلب مستقيم بمعرفة وصايا الله وقصص قديسيه، لأنه سيكون وارثًا لملكوت الله.»

\par 8 «لأنه هوذا الله رحيم ورؤوف، الذي قبل لنفسه لصوصًا وعشارين في الماضي من أجل صدق إيمانهم الذي من الله. والله أيضًا مع الدهور الآتية.»

\chapter{2}

\par 1 وحدث لما دنا وقت رحيل أبينا إسحاق، أبو الآباء، من هذا العالم وخروجه من جسده، أن الرحمن الرحيم أرسل إليه رئيس الملائكة ميخائيل، الذي أرسله إلى أبيه إبراهيم، في صباح اليوم الثامن والعشرين من شهر مصر

\par 2 فقال له الملاك: السلام عليك أيها الابن المختار أبونا إسحق.

\par 3 وكان من عادته أن يُكلِّمه الملائكة القديسون كل يوم، فسجد فرأى الملاك يُشبِّه أبيه إبراهيم.

\par 4 ثم فتح فمه وصرخ بصوت عظيم وقال بفرح وابتهاج: هوذا قد رأيت وجهك كما رأيت وجه الخالق الرحيم.

\par 5 ثم قال له الملاك: يا حبيبي إسحق، لقد أُرسلت إليك من وجه الله الحي لآخذك إلى السماء، لتكون مع أبيك إبراهيم وجميع القديسين.

\par 6 لأن أباكم إبراهيم ينتظركم، وهو مزمع أن يأتي إليكم، ولكنه الآن يستريح.

\par 7 قد أُعِدَّ لك العرش بجانب أبيك إبراهيم، وكذلك لابنك الحبيب يعقوب.

\par 8 "وأنتم جميعاً ستكونون فوق الجميع في ملكوت السماوات في مجد الآب والابن والروح القدس."

\par 9 سيؤتمن عليكم هذا الاسم لجميع الأجيال القادمة: الآباء. وهكذا ستكونون آباءً للعالم أجمع، أيها الشيخ الأمين، أبونا إسحاق

\par 10 أجاب إسحاق قائلًا للملاك: "إني مندهشٌ حقًا من أمرك. ألست أنت أبي إبراهيم؟"

\par 11 فقال له الملاك: «أنا لستُ أباك إبراهيم، بل أنا الذي أخدم أباك إبراهيم

\par 12 فالآن افرحوا وابتهجوا، لأنكم لن تُصابوا (بالمرض) ولن تُؤخذوا (بالموت) بألم بل بفرح

\par 13 ستحظى بالبركات والراحة الأبدية، وستخرج من الحبس إلى الرحابة

\par 14 ستذهب أيضًا إلى فرح لا نهاية له، وإلى نور ونعيم لا حدود لهما، وإلى هتاف وبهجة لا تنقطع

\par 15 «والآن، اكتب وصيتك ورتب بيتك؛ لأنك على وشك أن تذهب إلى الراحة (النهائية).»

\par 16 ولكن، طوبى للأب الذي ولدك، ولذريتك التي تأتي بعدك!

\par 17 فلما سمع أبونا يعقوب كلامهم هذا، بدأ يستمع إليهم، لكنه لم يتكلم

\par 18 فقال أبونا إسحاق للملاك بصبر وتواضع: «ماذا أفعل الآن بنور عيني يا حبيبي يعقوب؟»

\par 19 أنا خائف عليه بسبب عيسو. أنت، بالطبع، تعرف القصة كاملة

\par 20 ثم قال له الملاك: «يا حبيبي إسحاق، لو اجتمعت كل شعوب العالم في مكان واحد، لما استطاعوا أن يبطلوا بركتك على يعقوب؛ لأنه في ذلك الوقت الذي باركته فيه، كان مباركًا من الله الأعظم، ومن الابن والروح القدس، ومن أبيك إبراهيم؛ فأجابوا جميعًا قائلين: آمين».

\par 21 لن يخيفه (السيف؟) الحديد، لكنه سيكون قويًا للغاية وسيكتسب السيادة

\par 22 حينئذٍ يكون أبًا لأمم كثيرة، ويخرج منه اثنا عشر سبطًا

\par 23 فقال إسحاق للملاك: «لقد أخبرتني وبشّرتني

\par 24 ولكن لا يسمع يعقوب، لأنه سيحزن ويضطرب، لأني لم أحزن قلبه قط

\par 25 ثم قال ملاك الرب: "يا حبيبي إسحاق، كل الصالحين الذين يخرجون من أجسادهم يكونون مباركين، وهم في نعيم عندما يرون الله الرحيم الرؤوف

\par 26 لكن الويل، الويل ثلاث مرات، للخاطئ عندما يولد على الأرض، لأنه يعاني من آلام كثيرة

\par 27 وعلّم أبناءك طرقك ووصايا أبيك، جميع ما أوصاكم به

\par 28 ولا تخف هذه الأمور عن يعقوب لكي تكون تذكرة لأجيال نسله من بعده، لكي يعمل بها المؤمنون، وينالوا بها الحياة الأبدية التي إلى الأبد.

\par 29 لكنني سآخذ قلقك في الاعتبار.

\par 30 ها أنا أتيت إليكم سريعًا بفرح، والسلام الذي أعطاكم إياه الرب أعطيكم إياه.

\par 31 والآن أنا ماضٍ سريعًا إلى الذي أرسلني.

\chapter{3}

\par 1 ولما قال الملاك هذا، قام من فراش أبينا إسحاق وابتعد عنه

\par 2 ظل إسحاق ينظر إليه، وقد دهش مما سمع ورأى

\par 3 فتعهد قائلاً: "لن أرى النور حتى تستدعيني."

\par 4 بينما كان يعقوب يفكر في هذا، تقدم إلى باب حجرة أبيه

\par 5 كان الملاك قد ألقى عليه النوم ليمنعه من سماعهم

\par 6 فلما دخل على مدفن أبيه قال: يا أبت، مع من كنت تتحدث؟

\par 7 قال له إسحاق أبوه: "الآن يجب أن تسمعني يا ابني. لقد أُرسلت كلمة إلى أبيك الموقر أنه سيؤخذ منك يا ابني يعقوب."

\par 8 ثم احتضن يعقوب أباه وبكى قائلًا: «لقد ذهبت قوتي مني، أتجعلني يتيما يا أبي لأكون اليوم شقيًا؟»

\par 9 ثم عانق أبينا إسحاق مرة أخرى وقبله، وبكى كلاهما حتى تعبوا

\par 10 ثم قال يعقوب: «يا أبي، سأذهب معك ولن أتركك».

\par 11 فقال له إسحاق: «يا بني، ليس لي أن أفعل هذا يا ابني وحبيبي يعقوب، ولكني أشكر الله لأنك أنت أيضًا صرت أبًا، وأنك ستبقى حتى تُدعى...؟»

\par 12 كما أخبرني أبي إبراهيم، لا أستطيع أن أستبعد أي جزء من المرسوم، الذي يسري على الجميع؛ وهكذا سيحدث، لأن ما هو مكتوب لن يُبطَل

\par 13 لكن الله يعلم يا بني أن قلبي متعب بسببك. ومع ذلك فأنا سعيد بذهابي إلى الرب

\par 14 والآن بعد أن اختبرت النمو في الروح، اطرح عنك هذا البكاء والنحيب

\par 15 اسمع يا بني، لأُحدثك وأُعطيك فهمًا عن الإنسان الأول، أعني أبونا آدم، المخلوق، الذي جبله الله بيده؛ وكذلك أمنا حواء؛ وأيضًا هابيل وشيث وأبونا حنوك (أنوش؟) ومهللئيل، أبو متوشالح، ولامك، أبو يارد، وأنوش (أنوش؟)، أبو أبينا نوح وأبنائه سام وحام ويافث؛ وبعدهم فينحاس وقينان ونوح (؟) وعابر ورعو وتارح وناحور وأبي إبراهيم ولوط ابن أخيه

\par 16 كل هؤلاء أخذهم الموت ما عدا أبونا حنوك، الكامل الذي صعد إلى السماء

\par 17 «وبعد هذا سيخرج اثنا عشر عملاقًا.»

\par 18 ثم يأتي يسوع المسيح من نسلكم من العذراء مريم.

\par 19 "ويتجسّد الله فيه إلى مائة سنة."

\chapter{4}

\par 1 وكان إسحاق يصوم كل يوم، ولا يفطر إلا في المساء

\par 2 كان سيقدم ذبائح عن نفسه وعن جميع أهل بيته، من أجل خلاص نفوسهم

\par 3 كان يقوم للصلاة في منتصف الليل، وفي النهار يدعو الله. وظل يفعل ذلك لسنوات عديدة

\par 4 وكان يصوم أيضًا الأربعينيات الثلاث كلما حلت الأربعينية

\par 5 ولم يأكل لحمًا ولا يشرب خمرًا طوال حياته.

\par 6 كما أنه لم يكن يستمتع بطعم الفاكهة، ولم يكن ينام على فراش، لأنه كان مكرسًا للصلاة كل يوم والتضرع إلى الله طوال حياته.

\par 7 فلما سمعت الجموع أن رجل الله قد ظهر، توافدوا إليه من جميع الأنحاء والأماكن ليسمعوا تعليماته وتوصياته المحيية، وليتأكدوا من أن روح الله يتكلم فيه

\par 8 ثم قال العظماء الذين توافدوا إليه: "ما هذه القوة التي نزلت عليك بعد أن ذهب عنك بريق بصرك، وكيف حصلت على فرصة للرؤية الآن؟"

\par 9 ثم ابتسم الرجل العجوز المؤمن وقال لهم: "أما الذين قدموا أنفسهم، فسأخبرهم أن الله شفاني عندما رأى أنني اقتربت من بوابة الموت

\par 10 لقد منحني هذا الشرف في شيخوختي لأكون كاهنًا للرب

\par 11 ثم قال له أحدهم (يعقوب؟): «ابدأ لي كلامًا لعلّي أتعزّى به وأتمسك به».

\par 12 فقال له أبونا إسحاق: «إن تكلمت بغضب فاحفظ نفسك من القذف واحذر من التفاخر الباطل

\par 13 احذر أن تتحدث على انفراد (مع امرأة).

\par 14 احذر أن تخرج كلمة شريرة من فمك.

\par 15 احفظوا أجسادكم ليكونوا طاهرين، لأنه هيكل الروح القدس الذي يسكن فيه

\par 16 اعتنِ بالوظائف الأصغر في جسدك، ليكون طاهرًا ومقدسًا

\par 17 احذر أن تلعب بلسانك لئلا تخرج كلمة شريرة من فمك

\par 18 «احذر أن تمد يدك إلى ما لا تملك.»

\par 19 لا تقدم قربانًا عندما لا تكون طاهرًا طقسيًا؛ اغسل نفسك بالماء عندما تنوي الاقتراب من المذبح.

\par 20 لا تخلط أفكارك بأفكار العالم، وأنت تقف عند المذبح في حضرة الله

\par 21 قدّم قربانك لتكون صانع سلام بين الناس.

\par 22 وعندما تكون على وشك تقديم قربانك إلى الله، وعندما تتقدم نحو المذبح، يجب عليك أن تصلي إلى الله مائة مرة بلا انقطاع.

\par 23 "في البداية يجب أن تعرب عن هذا الشكر على النحو التالي، "يا الله، الذي لا يمكن إدراكه، والذي لا يمكن البحث عنه، مالك القوة، مصدر الطهارة، طهرني برحمتك، هدية مجانية منك لي.

\par 24 لأني مخلوق من لحم ودم، أهرب إليك.

\par 25 أنا أعلم نجاستي، وأنت تطهرني يا رب.

\par 26 «ها هي قضيتي بين أيديكم، وإليكم ملاذي.»

\par 27 أنا أعرف خطيئتي، لذا طهرني يا رب، لكي أدخل إلى حضرتك باحترام الذات.

\par 28 الآن قد عظمت معاصيّ، وقد اقتربتُ من النار المتقدة

\par 29 رحمتك على كل شيء، حتى تتمكن من إزالة جميع معاصي

\par 30 اغفر لي، حتى أنا الخاطئ.

\par 31 واغفر لجميع مخلوقاتك الذين خلقتهم، لكنهم لم يسمعوا أو يتعلموا منك

\par 32 «أنا مثل كل من هم على صورتك. لقد تحولت إلى فعل ما هو ممنوع عليّ.»

\par 33 لقد أتيتُ إليك، وأنا خادمك والابن الخاطئ لأمتك، لكنك أنت الغفور الرحيم

\par 34 اغفر لي بنعمتك التي تأتي من عندك، واستمع إلى توسلاتي لأكون مستحقًا للوقوف على مذبحك المقدس

\par 35 فلتكن هذه المحرقة مقبولة لديكم.

\par 36 لا تُعِدْني إلى جهلي بسبب خطاياي. اقبلني كالخروف الضال.

\par 37 ليكن الله الذي رعي أبينا آدم، وهابيل، ونوح، وأبينا إبراهيم، معك يا يعقوب، ومعي أيضًا

\par 38 اقبل تقدمتي مني.

\par 39 «فإن كنت قد تقدمت وفعلت هذا قبل صعودك إلى المذبح، فقدم ذبيحتك

\par 40 ولكن عليك أن تنتبه وتسهر لئلا تُحزن روح الرب

\par 41 لأن عمل الكهنوت ليس سهلاً، إذ يجب على كل كاهن، من اليوم وحتى اكتمال آخر الأجيال ونهاية العالم، ألا يشبع من شرب الخمر، ولا يشبع من أكل الخبز، وألا يتكلم عن

\par 42 لا ينبغي للكهنة أن يهتموا بأمور العالم، ولا يستمعوا إلى من يتحدث عنها. ولكن يجب على الكهنة أن يبذلوا كل جهودهم وحياتهم في الصلاة واليقظة والمثابرة في التقوى، حتى يتمكن كل واحد منهم من التضرع إلى الرب بنجاح

\par 43 علاوة على ذلك، فإن كل إنسان على وجه الأرض، سواء كان شقيًا أو محظوظًا، عليه واجب حفظ الوصايا الصحيحة

\par 44 فالرجال، بعد فترة قصيرة، سيُرفعون عن هذه الدنيا وقلقها الشديد

\par 45 ثم سينخرطون في خدمة مقدسة وملائكية بسبب الطهارة

\par 46 سيُقدَّمون أمام الرب وملائكته بسبب قرابينهم الطاهرة وخدمتهم الملائكية

\par 47 لأن سلوكهم الأرضي سينعكس في السماء، وسيكون الملائكة أصدقاء لهم بسبب إيمانهم الكامل ونقائهم

\par 48 عظيم هو تقديرهم أمام الرب، وليس صغير أو كبير إلا ويريد الرب أن يحسن إليه. لأن الرب يريد أن يكون كل واحد بلا عيب أو عثرة.

\par 49 «والآن، استمر في الدعاء إلى الله بالتوبة عن ذنوبك الماضية، ولا ترتكب المزيد من المعاصي

\par 50 وعليه، فلا تقتلوا بالسيف، ولا تقتلوا باللسان، ولا تزنوا بأجسادكم، ولا تغضبوا حتى تغرب الشمس

\par 51 لا تدع نفسك تتلقى مدحًا غير مبرر، ولا تفرح بسقوط أعدائك أو إخوتك

\par 52 لا تُجَدِّفْ، احذرْ القذف.

\par 53 لا تنظر إلى امرأة بعين شهوة

\par 54 احذروا من هذه الأشياء وما شابهها، لكي ينجو كل واحد منكم من الغضب الذي سيظهر من السماء

\chapter{5}

\par 1 فلما سمع الجموع المحيطون بهم ذلك، صرخوا بصوت واحد قائلين: «حقًا إن كل ما قاله هذا الرجل الجليل جدير بالانتباه».

\par 2 لكنه ظل صامتًا، ورفع عباءته، وغطى وجهه.

\par 3 فقال الجمع والكاهن الحاضر بعد صمت: «دعوه يستريح قليلاً».

\par 4 ثم جاء إليه ملاك الله وأخذه إلى السماء.

\par 5 هناك رأى أشياء معينة في خوف.

\par 6 كان العديد من الوحوش البرية في متناول اليد.

\par 7 الجانبين... (?) مثل الإخوة بحيث لا يستطيعون رؤية بعضهم البعض.

\par 8 كانت وجوههم كوجوه الإبل، وبعضهم كوجوه الكلاب

\par 9 كان البعض الآخر يشبه وجوه الأسود والضباع والنمور؛ وكان لدى البعض عين واحدة فقط

\par 10 قال إسحاق: «نظرت وإذا هم قد اتفقوا على رجل، وكانوا يعجلونه

\par 11 ولما أشاروا إلى الأسود، انصرف عنه السائرون معه

\par 12 ثم انقلبت عليه الأسود، فمزقته من المنتصف، وقطعته، ومضغته وابتلعته

\par 13 بعد ذلك أخرجوه من أفواههم فعاد إلى حالته الأصلية

\par 14 وبعد الأسود تقدم الآخرون وفعلوا الشيء نفسه به

\par 15 سيأخذونه واحدًا تلو الآخر، وكل واحد منهم سيمضغه ويبتلعه ويقذفه، فيعود إلى حالته الأصلية

\par 16 ثم قلت للملاك: يا سيدي، ما هي الخطيئة التي ارتكبها هذا الرجل حتى يتحمل عبئًا كهذا؟

\par 17 قال لي الملاك: "لأن هذا الرجل الذي تراه كان في عداوة مع جاره لمدة خمس ساعات، ومات دون أن يتصالح معه

\par 18 لذلك سُلِّم إلى خمسة من الجلادين ليُعذبوه لمدة عام كامل مقابل كل ساعة من الساعات الخمس التي قضاها عدوًا لصديقه

\par 19 ثم قال لي الملاك: يا حبيبي إسحق، انظر هنا إلى الستين ربوة الذين يوقعون العذاب في كل ساعة يظل فيها الرجل معاديا لجاره.

\par 20 يُؤتى به إلى هذه المخلوقات التي تُعذبه، كل واحد منهم لمدة ساعة حتى يكمل عامًا كاملًا، لو لم يكن قد تصالح وتب عن ذنبه قبل نقله وانفصاله عن جسده

\par 21 ثم أتى بي إلى نهر من نار. فرأيته ينبض، وأمواجه ترتفع نحو ثلاثين ذراعًا، وكان صوته كالرعد المدوّي

\par 22 نظرتُ إلى أرواح كثيرة مغمورة فيه على عمق حوالي تسعة أذرع

\par 23 وكان الذين في ذلك النهر يبكون ويصرخون بصوت عظيم وأنين عظيم

\par 24 وكان لهذا النهر حكمة في ناره: فهو لا يؤذي الصالحين، بل الخطاة فقط بإحراقهم

\par 25 سيحرق كل واحد منهم بسبب الرائحة الكريهة والنتنة المحيطة بالخطاة

\par 26 ثم لاحظت النهر العميق الذي تصاعد دخانه أمامي، ورأيت مجموعة من الناس في قاعه يصرخون ويبكون، وكل واحد منهم ينوح

\par 27 قال لي الملاك: "انظر إلى القاع لتلاحظ أولئك الذين تراهم في العمق الأدنى. إنهم الذين ارتكبوا خطيئة سدوم؛ حقًا، لقد استحقوا عقابًا شديدًا."

\par 28 ثم رأيتُ مُشرفَ العقابِ وكانَ كلُّهُ نارًا.

\par 29 وكان يضرب أعوانه في الجحيم ويقول لهم: اقتلوهم لكي يعلم أن الله موجود إلى الأبد.

\par 30 ثم قال لي الملاك: "ارفع عينيك وانظر إلى سلسلة العقوبات بأكملها."

\par 31 فقلت للملاك: «لا أستطيع أن أستوعبهم لكثرتهم، لكني أرغب في أن أفهم كم من الوقت سيقضيه هؤلاء الناس في هذا العذاب».

\par 32 قال لي: "حتى يرحمهم إله الرحمة ويرحمهم".

\chapter{6}

\par 1 بعد ذلك، أخذني الملاك إلى السماء ورأيت إبراهيم

\par 2 فسجدت له، فاستقبلني بحفاوة، هو وجميع الصالحين

\par 3 ثم اجتمعوا جميعًا وكرموني بفضل والدي.

\par 4 ثم أخذوني بيدي وقادوني إلى الحجاب أمام عرش الآب.

\par 5 فسجدت أمامه وسجدت له مع أبي وجميع القديسين، بينما كنا نسبح ونصرخ بصوت عالٍ قائلين: "قدوس، قدوس، قدوس هو رب الجنود! السماء والأرض مملوءتان من مجدك المقدس."

\par 6 ثم قال لي الرب من علوه المقدس: «كل من يسمي ابنه باسم حبيبي إسحاق، تحل عليه بركتي ​​وتكون في بيته إلى الأبد

\par 7 ممتاز هو مجيئك يا إبراهيم، أيها الأمين؛ ممتاز هو نسبك، وجيد هو حضور هذا النسب المبارك هنا

\par 8 "فالآن كل ما تطلبه باسم ابنك الحبيب إسحق يكون لك اليوم عهداً إلى الأبد."

\par 9 فأجاب أبي إبراهيم وقال: «لك السيادة يا رب، حاكم الكون».

\par 10 قال الرب من علوه المقدس لأبي إبراهيم: «كل من يدعو ابنه باسم حبيبي إسحاق، أو يكتب وصيته، تكون له بركة لا تزول، وبركتي ​​على بيته لا تزول

\par 11 أو إن أراد أحد أن يُطعم فقيرًا في يوم عيد حبيبي إسحاق، فإني أُعطيه لك في مملكتي

\par 12 ثم قال أبي إبراهيم: «يا أبتاه، الله، حاكم الكون، حتى وإن لم يكن قادرًا على كتابة وصيته أو عهده، فلتغمره بركتك ورحمتك، لأنك أنت الرحيم».

\par 13 قال الرب لإبراهيم: «ليُطعِم الجائع خبزًا، فأُعطيه مكانًا في ملكوتي، ويكون حاضرًا لديك من اللحظة الأولى للوليمة الألفية».

\par 14 قال الله المُخلِّص أيضًا لأبي إبراهيم: "وإن كان فقيرًا لدرجة أنه لا يجد خبزًا في بيته، فليقضِ ليلةً كاملةً في تذكار حبيبي إسحاق دون نوم، وسأمنحه ميراثًا في مملكتي."

\par 15 قال أبي إبراهيم: "وإن كان ضعيفًا لا يحتمل السهر، فلتشمله رحمتك وعطفك."

\par 16 فقال له الرب: «فليُقرِّب قليلًا من البخور باسمي في يوم ذكرى إسحاق ابنك الحبيب

\par 17 وإن كان لا يعرف القراءة، فليذهب ليسمع القراءة ممن يعرفها

\par 18 إن لم يستطع فعل أيٍّ من هذه الأشياء، فليدخل بيته، ويغلق الباب خلفه، ويصلي مئة صلاة توبة؛ ثم سأجعله لك ابنًا في مملكتي

\par 19 ولكن فوق كل هذا، فليُقدِّم قربانًا في يوم ذكرى حبيبي إسحاق

\par 20 وكل من يفعل كل ما قلته سينال ميراث الملكوت في سمائي

\par 21 وكل من بذل جهدًا في كتابة وصاياه وعهوده وقصص حياته، وأظهر الرحمة ولو (بإعطاء) كوب من الماء البارد، وآمن من كل قلبه - ستكون معهم قوتي وروحي القدوس من أجل ازدهار شؤونهم في العالم

\par 22 لن يكون هناك أي عائق في رحيلهم (من هذا العالم)، سأمنحهم عمرًا كاملاً في مملكتي، وسيكونون حاضرين منذ اللحظة الأولى للوليمة الألفية

\par 23 السلام عليكم يا أحبائي القديسين!

\par 24 ولما انتهى من هذا الكلام، صرخت الكائنات السماوية قائلة: «قدوس، قدوس، قدوس هو الرب يا صباؤوت! السماء والأرض مملوءتان من مجدك المقدس».

\par 25 فأجابه الآب الذي يتحكم في كل شيء من هذا المكان المقدس وقال: يا ميخائيل خادمي الأمين، نادِ جميع الملائكة وجميع القديسين.

\par 26 ثم ركب مركبة السرافيم، بينما كان الكروبيم يسيرون أمامه [مع الملائكة

\par 27 ولما وصلوا إلى سرير أبينا إسحاق، رأى أبونا إسحاق في الحال وجه ربنا ممتلئًا فرحًا نحوه

\par 28 صرخ قائلًا: "حسنًا، لقد أتيتَ يا سيدي، مع رئيس ملائكتك العظيم ميخائيل. حسنًا، لقد أتيتَ يا أبي، مع جميع القديسين."]؟

\par 29 عندما قال هذا، اضطرب يعقوب بشدة، وتمسك بأبيه وقبله باكيًا

\par 30 ثم أقامه أبونا إسحاق وأشار إليه بإشارة عينيه، أي: «اصمت يا بني».

\par 31 فقال إبراهيم للرب: يا رب، اذكر أيضًا ابني يعقوب

\par 32 ثم قال له الرب: «قوتي معه، ويمجد اسمي، ويسود على أرض الموعد، ولا يتسلط عليه العدو».

\par 33 فقال أبونا إسحاق: «يا يعقوب ابني الحبيب، احفظ وصيتي التي أضعها اليوم لتحفظ جسدي

\par 34 لا تُدنِّس صورة الله بكيفية معاملتك لها؛ لأن صورة الإنسان خُلقت على صورة الله؛ وسيعاملك الله وفقًا لذلك عندما تقابله وتراه وجهًا لوجه

\par 35 هو الأول والآخر، كما قال الأنبياء.

\chapter{7}

\par 1 ولما قال إسحاق هذا، أخذ الرب نفسه من جسده، فصارت بيضاء كالثلج، وامتلكها وحملها معه على مركبته المقدسة، وصعد بها إلى السماء، وكان الكروبيم يسبحون أمامها، وكذلك ملائكته القديسون

\par 2 وهب له الرب ملكوت السماوات، وكل ما رغب فيه أبونا من بركات الله الوفيرة التي كانت لديه، بما في ذلك إتمام عهده إلى الأبد

\chapter{8}

\par 1 هكذا كانت وفاة أبينا إبراهيم وأبينا إسحاق ابن إبراهيم، في اليوم الثامن والعشرين من شهر مصر، في مثل هذا اليوم. هذا اليوم قد كرسناه وعينّاه

\par 2 وفي اليوم الذي قدم فيه أبونا إبراهيم ذبيحته لله، في اليوم الثامن والعشرين من شهر أمشير، امتلأت السماء والأرض من رائحة سيرته الطيبة أمام الرب.

\par 3 وكان أبونا إسحاق كالفضة التي تُحرق وتُصهر وتُنقى وتُصفى في النار. وكذلك كل من يخرج من أبينا إسحاق، أبو الآباء

\par 4 في اليوم الذي قدم فيه إبراهيم، أبو الآباء، نفسه ذبيحة لله، صعد عطر ذبيحته إلى حجاب من يملك كل شيء

\par 5 طوبى لكل من يصنع رحمة في تذكار أبي الآباء إبراهيم وأبينا إسحق، لأن كل واحد منهما يكون له سكنى في ملكوت السماوات، لأن ربنا قطع معهما عهده الحق إلى الأبد.

\par 6 وسيحفظها لهم ولمن يأتون بعدهم، قائلاً لهم: "كل من أظهر رحمة باسم حبيبي إسحاق، ها أنا أعطيه لكم في ملكوت السماوات، وسيكون حاضرًا معهم في اللحظة الأولى من الوليمة الألفية ليحتفل معهم في النور الأبدي في ملكوت سيدنا وإلهنا وملكنا ومخلصنا يسوع المسيح

\par 7 هو الذي يستحق المجد والكرامة والجلال والسلطان والتبجيل والإكرام والتسبيح والسجود، مع الآب الرحيم والروح القدس الآن وإلى الأبد وإلى دهر الدهور، آمين!

\chapter{9}

\par 1 انتهت جنازة أبينا إسحاق. الحمد لله والتسبيح له، دائمًا وإلى الأبد



\end{document}