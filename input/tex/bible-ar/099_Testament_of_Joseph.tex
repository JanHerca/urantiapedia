\begin{document}

\title{وصية يوسف}

\chapter{1}

\par \textit{يوسف، الابن الحادي عشر ليعقوب وراحيل، الجميل والمحبوب. صراعه ضد الفاتنة المصرية.}

\par 1 نسخة وصية يوسف.

\par 2 عندما قارب على الموت، دعا أبناءه وإخوته، وقال لهم:

\par 3 يا إخوتي وأبنائي، اسمعوا ليوسف حبيب إسرائيل، وأصغوا يا أبنائي إلى أبيكم

\par 4 لقد رأيت في حياتي الحسد والموت، ومع ذلك لم أضل، بل ثابرت في حق الرب

\par 5 هؤلاء إخوتي أبغضوني، أما الرب فأحبني:

\par 6 أرادوا قتلي، لكن إله آبائي حفظني.

\par 7 أنزلوني في حُفرة، وأصعدني العليُّ أيضًا.

\par 8 لقد تم بيعي للعبودية، ورب الجميع حررني:

\par 9 أُخذتُ إلى الأسر، وأعانتني يده القوية.

\par 10 لقد كنت أعاني من الجوع، وكان الرب نفسه يغذيني.

\par 11 كنت وحدي، فعزاني الله:

\par 12 كنت مريضًا، وزارني الرب

\par 13 كنتُ في السجن، وأظهر إلهي نعمةً عليّ؛

\par 14 في القيود فأطلقني.

\par 15 شتموني فدافع عني.

\par 16 تكلم عليّ المصريون بمرارة فأنقذني

\par 17 حسدني رفاقي العبيد، فرفعني.

\par 18 وهذا رئيس قواد فرعون سلم إلي بيته.

\par 19 وجاهدتُ امرأةً بغيضةً تُلحّ عليّ على المعصية معها، لكن إله إسرائيل أبي أنقذني من لهيب النار

\par 20 أُلقيت في السجن، وتعرضت للضرب، وسخر مني؛ لكن الرب منحني أن أجد رحمة في نظر حارس السجن

\par 21 لأن الرب لا يترك خائفيه، لا في الظلمة، ولا في القيود، ولا في الضيقات، ولا في الضرورات

\par 22 لأن الله لا يخجل كإنسان، ولا يخاف كابن إنسان، ولا يضعف أو يخاف كإنسان من الأرض

\par 23 لكنه في كل تلك الأشياء يمنح الحماية، وبطرق متنوعة يُعزي، وإن كان يغادر لفترة قصيرة ليختبر ميول الروح

\par 24 في عشر تجارب أراني مقبولًا، وفي جميعها صبرت؛ لأن الصبر سحر عظيم، والصبر يعطي خيرات كثيرة

\par 25 كم مرة هددتني المرأة المصرية بالقتل!

\par 26 كم مرة سلمتني للعقاب ثم استدعتني وهددتني وعندما لم أرغب في صحبتها قالت لي:

\par 27 ستكون سيدًا لي، ولكل ما في بيتي، إذا سلمت نفسك لي، وستكون كسيدنا

\par 28 فتذكرت كلام أبي، ودخلت حجرتي وبكيت وصليت إلى الرب

\par 29 وصمتُ في تلك السنوات السبع، وظهرتُ للمصريين كشخصٍ مترف، لأن الذين يصومون من أجل الله ينالون جمال الوجه

\par 30 وإذا كان سيدي غائباً لم أكن أشرب خمراً ولا آكل طعاماً ثلاثة أيام، بل كنت أعطيه للفقراء والمرضى.

\par 31 فطلبت الرب باكرًا، وبكيت على المرأة المصرية من ممفيس، لأنها كانت تزعجني بلا انقطاع، لأنها كانت تأتي إليّ أيضًا في الليل بحجة زيارتي

\par 32 ولأنها لم تنجب طفلًا ذكرًا، فقد تظاهرت بأنها تعتبرني ابنًا

\par 33 ولفترة من الوقت احتضنتني كابن، ولم أكن أعلم ذلك؛ ولكن فيما بعد، سعت إلى جرّي إلى الزنا

\par 34 ولما أدركتُ ذلك حزنتُ حزنًا شديدًا، ولما خرجت، عدتُ إلى نفسي، ندبتُ عليها أيامًا كثيرة، لأني عرفتُ مكرها وخداعها

\par 35 وأخبرتها بكلام العلي، لعلها ترجع عن شهوتها الشريرة

\par 36 لذلك، غالبًا ما كانت تُجاملني بكلماتٍ كرجلٍ مُقدّس، وتمتدح عفتي بمكرٍ في حديثها أمام زوجها، بينما كانت ترغب في إيقاعي في الفخ عندما نكون وحدنا

\par 37 لأنها كانت تشيد بي علانيةً كعفيفة، وفي الخفاء قالت لي: لا تخافي زوجي؛ لأنه واثقٌ من عفتكِ؛ لأنه حتى لو أخبره أحدٌ عنا، فلن يُصدّق

\par 38 بسبب كل هذه الأشياء، استلقيت على الأرض، وتوسلت إلى الله أن ينقذني الرب من خداعها

\par 39 ولما لم تنجح في شيء، عادت إليّ بحجة التعليم، لتتعلم كلمة الله

\par 40 فقالت لي: إن كنت تريد أن أترك أصنامي، فاضطجع معي، وأنا أقنع زوجي بالتخلي عن أصنامه، ونسلك في الشريعة عند ربك

\par 41 فقلت لها: إن الرب لا يريد أن يكون المتقون في نجاسة، ولا يُسر بالزناة، بل بالذين يتقدمون إليه بقلب طاهر وشفاه غير نجسة

\par 42 لكنها التزمت بسلامها، متلهفة لتحقيق رغبتها الشريرة.

\par 43 فأكثرت من الصوم والصلاة، حتى ينقذني الرب منها.

\par 44 وفي وقت آخر قالت لي: إن لم تزنِ، فسأقتل زوجي بالسم، وأتخذك زوجًا لي

\par 45 فلما سمعتُ ذلك، مزّقتُ ثيابي وقلتُ لها:

\par 46 يا امرأة، اتقي الله، ولا تفعلي هذا الفعل الشرير لئلا تهلكي. واعلمي أني سأخبر جميع الناس بخطتك هذه.

\par 47 لذلك، ولأنها كانت خائفة، توسلت إليّ ألا أعلن عن هذه الحيلة

\par 48 ثم رحلت وهي تُهدئني بالهدايا، وتُرسل إليّ كل سرور بني البشر

\par 49 وبعد ذلك أرسلت لي طعامًا ممزوجًا بالسحر.

\par 50 ولما جاء الخصي الذي أحضرها، رفعت رأسي ونظرت فإذا رجل رهيب يعطيني مع الطبق سيفاً، فأدركت أن كيدها كان لإغرائي.

\par 51 ولما خرج بكيت، ولم أذق ذلك الطعام ولا غيره من طعامها

\par 52 وبعد يوم واحد جاءت إلي ولاحظت الطعام وقالت لي: لماذا لم تأكل من الطعام؟

\par 53 فقلت لها: لأنكِ ملأتِه بتعاويذ قاتلة، وكيف قلتِ: لا أقترب من الأصنام بل من الرب وحده

\par 54 فاعلم الآن أن إله أبي قد كشف لي عن طريق ملاكه شرّك، وحفظته لأوبّخك، لعلّك تبصر وتتوب

\par 55 ولكن لكي تتعلم أن شر الأشرار ليس له سلطان على الذين يعبدون الله بعفة، فها أنا آخذ منه وآكل أمامك

\par 56 وبعد أن قلت هذا، صليت هكذا: إله آبائي وملاك إبراهيم، يكون معي. وأكلت

57 فلما رأت ذلك سقطت على وجهها عند رجلي باكية فأقمتها ووبختها.

\par 58 ووعدت ألا تفعل هذا الإثم بعد الآن.

\par 59 لكن قلبها كان لا يزال متجهًا نحو الشر، وكانت تنظر حولها كيف تصطادني، وتنهدت بعمق وأصبحت مكتئبة، على الرغم من أنها لم تكن مريضة.

\par 60 فلما رآها زوجها قال لها: لماذا وجهك عابس؟

\par 61 فقالت له: إن قلبي يؤلمني، وأنين روحي يضايقني، فعزاها وهي ليست مريضة

\par 62 ثم انتهزت الفرصة، وهرعت إليّ بينما كان زوجها لا يزال بالخارج، وقالت لي: سأشنق نفسي، أو أرمي بنفسي من فوق جرف، إذا لم تضطجع معي

\par 63 ولما رأيت روح بليعار يزعجها، صليت إلى الرب وقلت لها:

\par 64 لماذا، أيتها المرأة البائسة، أنت مضطربة ومضطربة، وأعميتِ بالخطايا؟

\par 65 تذكري أنه إذا قتلتِ نفسكِ، فإن أستيهو، محظية زوجكِ، منافسكِ، ستضرب أطفالكِ، وستدمرين ذكراكِ من على وجه الأرض

\par 66 فقالت لي: انظر، فأنت تحبني إذًا؛ فليكفيني هذا: فقط اسعَ من أجل حياتي وأولادي، وأتوقع أن أتمتع برغبتي أيضًا

\par 67 لكنها لم تعلم أنني تكلمت هكذا من أجل سيدي، وليس من أجلها

\par 68 لأنه إذا وقع رجل تحت تأثير رغبة شريرة وأصبح عبدًا لها، مثلها، فإن كل ما يسمعه من خير فيما يتعلق بتلك الرغبة، فإنه يتلقاه في ضوء رغبته الشريرة

\par 69 لذلك، أعلن لكم يا أبنائي أنها كانت حوالي الساعة السادسة عندما غادرتني؛ وركعتُ أمام الرب طوال النهار والليل؛ وحوالي الفجر نهضتُ، أبكي طوال الوقت وأدعو من أجل تحررها

\par 70 أخيرًا، أمسكت بملابسي، وسحبتني بالقوة للتواصل معها

\par 71 فلما رأيت أنها في جنونها متمسكة بثوبى، تركته ورائى وهربت عاريًا

\par 72 وهي متمسكة بالثوب واتهمتني زوراً، فلما جاء رجلها ألقاني في سجن بيته، وفي الغد جلدني وأرسلني إلى سجن فرعون.

\par 73 ولما كنت مقيدًا، كانت المرأة المصرية مثقلةً بالحزن، فجاءت وسمعت كيف حمدت الرب ورنمت في مسكن الظلمة، ففرحت بصوتٍ مُبتهج، مُمجدةً إلهي لأني تحررت من شهوة المرأة المصرية

\par 74 وكثيرًا ما أرسلت إليّ قائلةً: وافق على تحقيق رغبتي، وسأحررك من قيودك، وسأحررك من الظلام

\par 75 ولم أُمل إليها حتى في الفكر.

\par 76 لأن الله يحب من يجمع الصوم مع العفة في جحر الشر، أكثر من من يجمع الترف مع الفجور في حجرات الملوك.

\par 77 وإن عاش الإنسان في عفة، ورغب أيضًا في المجد، وعلم العلي أنه مناسب له، فإنه يمنحني هذا أيضًا

\par 78 كم مرة، على الرغم من مرضها، جاءت إليّ في أوقات غير متوقعة، واستمعت إلى صوتي وأنا أصلي!

\par 79 وعندما سمعت أنينها، التزمت الصمت.

\par 80 "فإذا كنت في بيتها كانت تكشف ذراعيها وثدييها وساقيها لأضطجع معها. لأنها كانت جميلة جداً ومزينة بشكل باهظ لكي تغويني."

\par 81 وحفظني الرب من مكايدها.

\chapter{2}

يوسف ضحية مؤامرات عديدة من قِبَل امرأة ممفية. للاطلاع على مثل نبوي شيق، انظر الآيتين ٧٣-٧٤.

\par 1 فانظروا يا أبنائي كيف يعمل الصبر والصلاة مع الصوم

\par 2 وهكذا أنتم أيضًا، إن اتبعتم العفة والطهارة بالصبر والصلاة، والصوم بتواضع القلب، فسيحل الرب بينكم لأنه يحب العفة

\par 3 وحيثما يسكن العلي، حتى لو أصاب الإنسان حسد أو عبودية أو افتراء، فإن الرب الذي يسكن فيه، من أجل عفته، لا ينقذه من الشر فحسب، بل يرفعه أيضًا مثلي

\par 4 لأنه في كل شيء يرتفع الإنسان، سواء بالفعل، أو بالقول، أو بالفكر

\par 5 عرف إخوتي كيف أحبني أبي، ومع ذلك لم أتكبر على نفسي: فرغم صغر سني، إلا أن خوف الله كان يملأ قلبي؛ لأني كنت أعلم أن كل شيء سيزول

\par 6 ولم أقم عليهم بقصد شرير، بل كنت أكرم إخوتي. واحترامًا لهم، حتى عندما كنت أُباع، امتنعت عن إخبار الإسماعيليين بأني ابن يعقوب، رجل عظيم وذو سلطان

\par 7 وأنتم أيضًا يا أبنائي، فليكن خوف الله أمام أعينكم في جميع أعمالكم، وأكرموا إخوتكم

\par 8 لأن كل من يعمل بشريعة الرب يحبه.

\par 9 ولما أتيت إلى إندوكولبيتاي مع الإسماعيليين سألوني قائلين:

\par 10 أأنت عبد؟ فقلت إني عبد من أهل البيت، حتى لا أخزي إخوتي.

\par 11 فقال لي أكبرهم سنًا: أنت لست عبدًا، فحتى مظهرك يدل على ذلك

\par 12 فقلتُ إني عبدٌ لهم.

\par 13 ولما دخلنا مصر تخاصموا عليّ: من منهم يشتريني ويأخذني

\par 14 لذلك، رأى الجميع أن أبقى في مصر مع تاجر تجارتهم، حتى يعودوا حاملين بضائعهم

\par 15 وأعطاني الرب نعمة في عيني التاجر، فسلمني بيته

\par 16 وباركه الله من فضلي، وكثر له من الذهب والفضة ومن خدم البيوت

\par 17 وكنت معه ثلاثة أشهر وخمسة أيام.

\par 18 وفي ذلك الوقت نزلت امرأة ممفيس، امرأة بنتيفريس، في مركبة، في احتفال عظيم، لأنها سمعت من خصيانها عني.

\par 19 وأخبرت زوجها أن التاجر قد غنى بفضل شاب عبراني، ويقولون إنه قد سُرق بالتأكيد من أرض كنعان

\par 20 فالآن، أنصفه، وخذ الشاب إلى بيتك، فيباركك إله العبرانيين، لأن النعمة من السماء عليه

\par 21 فاقتنع بنتفريس بكلامها، وأمر بإحضار التاجر، وقال له:

\par 22 ما هذا الذي أسمعه عنك، أنك تسرق الناس من أرض كنعان وتبيعهم عبيدًا؟

\par 23 فخرّ التاجر عند قدميه، وتوسّل إليه قائلًا: أتوسل إليك يا سيدي، لا أعرف ما تقول

\par 24 فقال له بنتفريس: من أين إذن العبد العبراني؟

\par 25 فقال: إن الإسماعيليين استودعوه عندي حتى يرجعوا.

\par 26 لكنه لم يصدقه، بل أمر بتجريده من ملابسه وضربه

\par 27 وعندما أصر على هذا القول، قال بنتيفريس: ليؤتى بالشاب

\par 28 وعندما أُحضِرتُ، سجدتُ لبنتفريس لأنه كان الثالث في رتبة ضباط فرعون

\par 29 فانفصل بي عنه، وقال لي: أأنت عبد أم حر؟

\par 30 فقلت: عبد.

\par 31 فقال: لمن؟

\par 32 فقلت: الإسماعيليون.

\par 33 فقال: كيف أصبحت عبدًا لهم؟

\par 34 فقلت: اشتروني من أرض كنعان.

\par 35 فقال لي: حقا كذبت، وأمر في الحال أن أخلع ملابسي وأضرب.

\par 36 كانت المرأة الممفية تنظر إليّ من خلال النافذة بينما كنت أتعرض للضرب، لأن منزلها كان قريبًا، وأرسلت إليه قائلة:

\par 37 إن حكمك ظالم؛ لأنك تعاقب رجلاً حرًا سُرق، كما لو كان مذنبًا

\par 38 وعندما لم أغير في إفادتي، على الرغم من تعرضي للضرب، أمر بسجني، حتى يأتي أصحاب الصبي، كما قال

\par 39 فقالت المرأة لزوجها: لماذا تحبس الغلام الأسير والحسن في القيود، وهو الذي ينبغي بالحري أن يطلق حراً ويخدم؟

\par 40 لأنها أرادت رؤيتي رغبةً في الخطيئة، لكنني كنت جاهلاً بكل هذه الأمور

\par 41 فقال لها: ليس من عادة المصريين أن يأخذوا ما هو للآخرين قبل تقديم البينة

\par 42 لذلك، قال هذا عن التاجر؛ أما الصبي، فيجب سجنه

\par 43 وبعد أربعة وعشرين يومًا جاء الإسماعيليون، لأنهم سمعوا أن يعقوب أبي كان ينوح عليّ كثيرًا

\par 44 فجاءوا وقالوا لي: كيف قلت إنك عبد؟ وها قد علمنا أنك ابن رجل عظيم في أرض كنعان، وأبوك لا يزال ينوح عليك بالمسوح والرماد

\par 45 عندما سمعت هذا، ذابت أحشائي وذاب قلبي، ورغبت بشدة في البكاء، لكنني منعت نفسي حتى لا أخجل إخوتي

\par 46 فقلت لهم: لا أعلم، أنا عبد.

\par 47 ثم تشاوروا على بيعي حتى لا أوجد في أيديهم.

\par 48 لأنهم خافوا من أبي، خشية أن يأتي وينفذ عليهم انتقامًا شديدًا

\par 49 لأنهم سمعوا أنه قدير عند الله وعند الناس.

\par 50 فقال لهم التاجر: أنقذوني من حكم بنتيفري.

\par 51 فجاءوا وطلبوا مني قائلين: قل إننا اشتريناك بفضة، وسيُعتقنا

\par 52 فقالت المرأة الممفية لزوجها: اشترِ الشاب، لأني سمعت أنهم يبيعونه

\par 53 وفي الحال أرسلت خصيًا إلى الإسماعيليين وطلبت منهم أن يبيعوني

\par 54 ولكن بما أن الخصي لم يوافق على شرائي بثمنهم، فقد عاد بعد أن جرّبهم، وأخبر سيدته أنهم طلبوا ثمنًا كبيرًا لعبدهم

\par 55 فأرسلت خصيًا آخر قائلة: حتى لو طلبوا منين، فأعطهم، ولا تبخل بالذهب؛ فقط اشترِ الغلام وأحضره إليّ

\par 56 فذهب الخصي وأعطاهم ثمانين من الذهب، فأخذني. أما المرأة المصرية فقال لها: أعطيتك مئة

\par 57 ومع علمي بهذا، صمتُ، لئلا يُخزى الخصي

\par 58 ترون إذن يا أبنائي كم تحملت من أمور عظيمة لكي لا أخزى إخوتي

\par 59 فأحبوا أنتم أيضًا بعضكم بعضًا، وأخفوا عيوب بعضكم البعض بصبر

\par 60 لأن الله يُسرّ بوحدة الإخوة، وبقصد القلب الذي يُسرّ بالمحبة

\par 61 ولما جاء إخوتي إلى مصر علموا أنني قد رددت إليهم أموالهم، ولم أوبخهم، بل عزيتهم

\par 62 وبعد وفاة يعقوب أبي أحببتهم جدا، وكل ما أمر به فعلته لهم جدا.

\par 63 ولم أدعهم يتضايقون في أصغر أمر، وكل ما كان في يدي أعطيتهم إياه

\par 64 وكان أبناؤهم أبنائي، وكان أبناؤي خدمًا لهم؛ وكانت حياتهم حياتي، وكل معاناتهم معاناتي، وكل مرضهم ضعفي

\par 65 كانت أرضي أرضهم، ومشورتهم مشورتي.

\par 66 ولم أتكبر بينهم لأجل مجدي الدنيوي، بل كنت بينهم كأحد الأصاغر.

\par 67 إن سلكتم أنتم أيضًا في وصايا الرب يا أبنائي، فإنه سيرفعكم هناك، ويبارككم بالخير إلى الأبد

\par 68 وإن أراد أحدٌ أن يُسيء إليكم، فأحسنوا إليه وصلّوا لأجله، فيُخلّصكم الرب من كل شر

\par 69 فها أنتم ترون أنه من تواضعي وطول أناتي تزوجت ابنة كاهن هليوبوليس

\par 70 وأُعطيت لي معها مئة وزنة من الذهب، وجعلها الرب لخدمتي

\par 71 وأعطاني أيضًا جمالًا كزهرةٍ تفوق جمال بني إسرائيل، وحفظني إلى الشيخوخة في قوةٍ وجمال، لأني كنتُ أشبه يعقوب في كل شيء

\par 72 واسمعوا يا أبنائي أيضًا الرؤيا التي رأيتها.

\par 73 وكان اثنا عشر غزالاً ترعى، فانتشرت التسعة أولاً في كل الأرض، وكذلك الثلاثة أيضاً.

\par 74 ورأيت أنه من يهوذا وُلدت عذراء لابسة ثوب كتان، ومنه وُلد حمل بلا عيب، وعلى يده اليسرى كان مثل أسد، فانقضت عليه جميع الوحوش، فغلبها الحمل، وأهلكها وداسها

\par 75 وبسببه فرحت الملائكة والبشر وكل الأرض.

\par 76 وسوف يحدث هذا في حينه، في الأيام الأخيرة.

\par 77 فاحفظوا يا أبنائي وصايا الرب، وأكرموا لاوي ويهوذا، لأنه منهما سيقوم لكم حمل الله الذي يرفع خطيئة العالم، ويخلص جميع الأمم وإسرائيل

\par 78 لأن مملكته مملكة أبدية لا تزول. أما مملكتي بينكم فستزول كأرجوحة الحارس التي تزول بعد الصيف

\par 79 لأني أعلم أنه بعد موتي سيضايقكم المصريون، ولكن الله سينتقم لكم، ويدخلكم إلى ما وعد به آباءكم

\par 80 بل ستحملون عظامي معكم، لأنه حينما تُرفع عظامي إلى هناك، يكون الرب معكم في النور، ويكون بليعار في الظلمة مع المصريين

\par 81 واصعدوا أسنات أمكم إلى ميدان سباق الخيل، وادفنوها بالقرب من راحيل

\par 82 ولما قال هذا مدّ قدميه ومات عن شيخوخة صالحة

\بار{83}
فناح عليه كل إسرائيل وكل مصر ناحاً عظيماً.

\par 84 ولما خرج بنو إسرائيل من مصر، أخذوا معهم عظام يوسف، ودفنوه في حبرون مع آبائه، وكانت سنو حياته مئة وعشر سنين





\end{document}