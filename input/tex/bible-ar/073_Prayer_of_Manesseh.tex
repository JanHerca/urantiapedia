\begin{document}

\title{صلاة مانيسي}

\chapter{1}

\par 1 يا رب إله آبائنا إبراهيم وإسحق ويعقوب ونسلهم الصالحين
\par 2 أنت الذي صنعت السماء والأرض بكل نظامهما؛
\par 3 الذي قيد البحر بكلمتك الآمرة، وحاصر اللجج وختمه باسمك الرهيب المجيد.
\par 4 الذي ترتعد منه كل الأشياء، وترتجف أمام قدرتك،
\par 5 لأن مجدك العظيم لا يطاق، وغضب تهديدك للخطاة لا يقاوم؛
\par 6 ومع ذلك فإن رحمتك الموعودة لا تُقاس ولا تُفحص،
\par 7 لأنك أنت الرب العلي، كثير الرحمة، طويل الروح، ورحيم جدًا، وتتوب على شرور البشر. أنت يا رب بحسب رحمتك العظيمة، وعدت بالتوبة والمغفرة لمن أخطأوا إليك، وبكثرة رحمتك عينت التوبة للخطاة، لكي يخلصوا.
\par 8 لذلك، أيها الرب إله الأبرار، لم تجعل التوبة للأبرار، لإبراهيم وإسحاق ويعقوب الذين لم يخطئوا إليك، لكنك جعلت التوبة لي أنا الخاطئ
\par 9 لأن خطاياي التي ارتكبتها أكثر من رمل البحر، ومعاصيّ قد كثرت يا رب، كثرت! لا أستحق أن أنظر إلى علو السماء من كثرة آثامي
\par 10 لقد أُثقلت بي قيود من حديد كثيرة، حتى إنني رُفضت بسبب خطاياي، وليس لي أي راحة؛ لأني أسخطت غضبك، وفعلت الشر أمام عينيك، وأقمت رجاسات، وأكثرت من المعاصي
\par 11 والآن أحني ركبة قلبي، أتوسل إليك من أجل لطفك
\par 12 أخطأتُ يا رب، أخطأتُ، وأنا أعلم بمعاصيّ
\par 13 أتوسل إليك بإخلاص، | اغفر لي يا رب، اغفر لي! | لا تهلكني بمعاصي! | لا تغضب عليّ إلى الأبد ولا تخزن لي الشر؛ | لا تدينني في أعماق الأرض. | لأنك أنت يا رب إله التائبين،
\par 14 وستُظهِر صلاحك فيّ؛ لأني، على الرغم من عدم استحقاقي، ستخلصني برحمتك العظيمة،
\par 15 وسأسبحك دائمًا كل أيام حياتي. | لأن كل جند السماء يسبحونك، | ولك المجد إلى الأبد. آمين

\end{document}