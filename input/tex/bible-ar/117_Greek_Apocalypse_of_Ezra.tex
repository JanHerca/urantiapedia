\begin{document}

\title{سفر رؤيا عزرا اليوناني}

\chapter{1}

\par \textit{مقدمة وصلاة}

\par 1 حدث ذلك في السنة الثلاثين، في الثاني والعشرين من الشهر،

\par 2 كنت في بيتي وصرخت قائلًا للعلي: "يا رب، امنحني المجد لأرى أسرارك."


\par 3 ولما حل الليل، جاءني الملاك ميخائيل، رئيس الملائكة، وقال لي: "أيها النبي عزرا، خَزِّن خبزًا لسبعين أسبوعًا." وصمتُ كما قال لي

\par 4 وجاء رئيس القضاة رافائيل وأعطاني عصا ستوراكس،

\par 5 وصمت ستين أسبوعاً مرتين، ورأيت أسرار الله وملائكته.

\par 6 وقلت لهم: "أريد أن أتضرع إلى الله بشأن الشعب المسيحي. كان من الأفضل ألا يولد الإنسان من أن يدخل العالم."

\par \textit{عزرا رُفع إلى السماء: صلاته من أجل الرحمة}

\par 7 لذلك، رُفعتُ إلى السماء، ورأيتُ في السماء الأولى أمرًا عظيمًا من الملائكة، وقادوني إلى الدينونة

\par 8 وسمعت صوتًا يقول لي: "ارحمنا يا عزرا مختار الله."

\par 9 ثم بدأت أقول: "ويل للخطاة حين يرون الرجل الصالح (مرتفعًا) فوق الملائكة، وهم في نار جهنم."

\par 10 فقال عزرا: «ارحم أعمال يديك أيها الرحيم وعظيم الرحمة

\par 11 أدينوني بدلًا من أرواح الخطاة، لأنه من الأفضل معاقبة نفس واحدة بدلًا من هلاك العالم كله

\par 12 وقال الله: «سأُريح الصديقين في الفردوس وأنا رحيم».

\par 13 فقال عزرا: يا رب، لماذا تُحسن إلى البار؟

\par 14 فكما أن الأجير يكمل خدمته ويمضي، والعبد يخدم سيده لكي يأخذ أجرته، هكذا البار ينال أجره في السماوات.

\par 15 لكن ارحم الخطاة لأننا نعلم أنك رحيم.

\par 16 فقال الله: ليس عندي رحمة بهم.

\par 17 فقال عزرا: «(ارحمهم) لأنهم لا يطيقون غضبك».

\par 18 وقال الله: «(أنا غاضب) لأن مثل هؤلاء (رجال) مثل هؤلاء».

\par 19 وقال الله: "أريد أن أحفظكما كبولس ويوحنا."

\par 20 "لقد أعطيتني الكنز غير المفسد، كنز العذرية، وسور الرجال."

\par \textit{صلاة عزرا الثانية}

\par 21 فقال عزرا: «كان خيرًا لو لم يولد الإنسان، وكان خيرًا لو لم يعش

\par 22 الوحوش البكماء أفضل من الإنسان، لأنها لا تُعاقَب

\par 23 أخذتنا وسلمتنا إلى الدينونة.

\par 24 ويل للخطاة في العالم القادم، لأن دينونتهم لا نهاية لها ونارهم لا تنطفئ.

\chapter{2}

\par \textit{عزرا يحتج على الله: خطيئة آدم}

\par 1 وبينما قلت له هذا، جاء ميخائيل وجبرائيل وجميع الرسل وقالوا:

\par 2 "تحياتي!"

\par 3 [فقال عزرا: "يا رجل الله الأمين!"]

\par 4 قم تعال معي إلى هنا يا رب إلى القضاء. فقال الله: ها أنا أعطيك عهدي وعهدك لتقبله.

\par 5 فقال عزرا: «سنرفع قضيتنا أمامك».

\par 6 فقال الله: اسأل أباك إبراهيم أي ابن يخاصم أبيه، ثم تعال ودافع عنا.

\par 7 فقال عزرا: «حي هو الرب، إني لن أكف عن مرافعتك في القضية من أجل الشعب المسيحي

\par 8 أين رحمتك السابقة يا رب؟ أين طول أناتك؟

\par 9 وقال الله كما خلقت الليل والنهار خلقت البار والخاطئ. وكان من اللائق أن تسلك كالبار.

\par 10 وقال النبي: «من خلق آدم، البروتوبلاست، أول من خلق؟»

\par 11 وقال الله: «يديّ الطاهرتان، ووضعته في الفردوس ليحرس منطقة شجرة الحياة».

\par 12 «لأن الذي أسس المعصية جعل هذا (الإنسان) مذنبًا.»

\par 13 فقال النبي: أليس يحرسه ملك؟

\par 14 ألم تُحفظ الحياة بواسطة الكروبيم إلى الأبد؟

\par 15 وكيف انخدع من كان يحرسه الملائكة الذين أمرتهم بحضور ما حدث؟ انتبه أيضًا لما أقول!

\par 16 لو لم تُعطه حواء، لما خدعها الثعبان أبدًا

\par 17 إذا أنقذت من تشاء، فسوف تهلك أيضًا من تشاء.

\par \textit{عزرا يحتج على الله: خطايا البشر}

\par 18 فقال النبي: يا رب، دعنا ننتقل إلى حكم ثانٍ

\par 19 وقال الله: «ألقي نارًا على سدوم وعمورة».

\par 20 فقال النبي: يا رب، أنت تأتي علينا بما نستحق.

\par 21 فقال الله: «خطاياك تفوق لطفي».

\par 22 وقال النبي: اذكر الكتاب يا أبي الذي قاس أورشليم وبنىها.

\par 23 ارحم يا رب الخطاة، ارحم صنعتك، ارحم أعمالك

\par 24 ثم تذكر الله أعماله وقال للنبي: «كيف أرحمهم؟

\par 25 سَقَوْا خَلًّا وَمُرًّا و [...] تَابُوا.

\par \textit{يوم القيامة}

\par 26 فقال النبي: «أظهروا كروبيملكم ولنذهب معًا إلى الدينونة،

\par 27 وأرني ما هي شخصية يوم القيامة.

\par 28 فقال الله لقد انحرفت يا عزرا

\par 29 فإن ذلك يوم القيامة الذي لا مطر فيه على الأرض،

\par 30 لأن في ذلك اليوم حسابًا رحيمًا.

\par 31 وقال النبي صلى الله عليه وسلم: لا أزال أجادلكم حتى أرى يوم القيامة.

\par 32 (وقال الله): «أحصوا النجوم ورمال البحر، فإن استطعتم أن تحصوا هذا، فستتمكنون أيضًا من مجادلتي في القضية».

\chapter{3}

\par 1 فقال النبي: «يا رب، أنت تعلم أني أحمل جسدًا بشريًا

\par 2 وكيف أحصي نجوم السماء ورمال البحر؟

\par 3 وقال الله: يا نبي المختار، لن يعرف أحد ذلك اليوم العظيم والمظهر الذي سيأتي ليدين العالم.

\par 4 من أجلك يا نبي، أخبرتك باليوم، لكنني لم أخبرك بالساعة

\par 5 فقال النبي: «يا رب، أخبرني أيضًا بالسنين».

\par 6 وقال الله: «إن رأيتُ عدلَ العالم قد ازداد، فسأطولُ صبرًا عليهم. وإلا، فسأمدّ يدي وأمسكُ بالعالم المسكون من أطرافه الأربع، وأجمعهم جميعًا إلى وادي يهوشافاط، وأبيدُ الجنسَ البشري، ولا يبقى للعالم وجودٌ بعدُ».

\par 7 فقال النبي: «وكيف تُمَجَّد يمينك؟»

\par 8 وقال الله: «سأمجَّد من ملائكتي».

\par \textit{لماذا خُلِقَ الإنسان؟}

\par 9 فقال النبي: يا رب، إذا كان هذا هو حسابك، فلماذا خلقت الإنسان؟

\par 10 قلتَ لإبراهيم أبينا: «أكثِّرُ نسلكَ كنجومِ السماءِ وكالرملِ الذي على شاطئِ البحرِ». فأينَ وعدُكَ؟»

\par \textit{علامات النهاية}

\par 11 وقال الله: «أولاً سأُسبب بهزّ الدواب والبشر

\par 12 ومتى رأيتم الأخ يسلم أخاه إلى الموت، والأبناء يقومون على والديهم، والمرأة تترك زوجها،

\par 13 وحينما تقوم أمة على أمة في حرب، فحينئذ تعلمون أن النهاية قريبة

\par 14 وحينئذٍ لن يرحم الأخ أخاه، ولا الرجل زوجته، ولا الأبناء والديهم، ولا الأصدقاء أصدقاءهم، ولا العبد سيده

\par 15 لأن خصم البشر نفسه سيصعد من تارتاروس وسيُظهر للناس أشياءً كثيرة

\par 16 ماذا أفعل بك يا عزرا، وهل ستُجادلني في القضية؟

\chapter{4}

\par \textit{عزرا ينزل إلى تارتاروس}

\par 1 فقال النبي: يا رب، لا أزال أجادلك في القضية

\par 2 وقال الله: "أحصوا أزهار الأرض."

\par 3 إذا تمكنت من حسابهم، فسوف تتمكن أيضًا من مناقشة القضية معي.

\par 4 فقال النبي: يا رب، لا أستطيع أن أحصيهم، فأنا أحمل لحماً بشرياً، ولكني لا أكف عن مجادلتك في هذا الأمر.

\par 5 "أتمنى يا رب أن أرى الأجزاء السفلى من تارتاروس."

\par 6 وقال الله: انزل وانظر.

\par 7 وأعطاني ميكائيل وجبرائيل وأربعة وثلاثين ملاكا آخرين،

\par 8 ونزلت خمساً وثمانين درجة، ونزلوني خمسمائة درجة.

\par \textit{عقاب هيرودس}

\par 9 ورأيت عرشًا من نار ورجلًا شيخًا جالسًا عليه، وكان عقابه بلا رحمة.

\par 10 فقلت للملائكة: من هذا وما خطيئته؟

\par 11 فقالوا لي: هذا هيرودس، وكان ملكاً زماناً، وهو الذي أمر بقتل الأطفال من ابن سنتين فما دون.

\par 12 فقلت: ويلٌ لنفسه!

\par \textit{العصاة والهاوية}

\par 13 ثم أنزلوني مرة أخرى ثلاثين درجة. ورأيت هناك نيرانًا متقدة، وفيها جموع من الخطاة.

\par 14 وسمعت أصواتهم، لكنني لم أتمكن من رؤية أشكالهم.

\par 15 ثم أنزلوني إلى عمق أكبر بعدة درجات لم أستطع عدها.

\par 16، ورأيت هناك رجالاً مسنين، وكانت محاور نارية تدور على آذانهم.

\par 17 فقلت: ”من هؤلاء وما هي خطاياهم؟“

\par 18 فقالوا لي: ”هؤلاء هم المتنصتون“.

\par 19 ثم قادوني مرة أخرى إلى أسفل خمسمائة درجة أخرى. 

\par 20 وهناك رأيت الدودة التي لا تنام والنار التي تحرق الخطاة.

\par 21 ثم قادوني إلى أساس أبوليا (الدمار) وهناك رأيت الضربة الاثني عشر للهاوية.

\par 22 وأخذوني إلى الجنوب، وهناك رأيت رجلاً معلقاً من جفنيه والملائكة يضربونه.

\par 23 فسألت: ”من هذا وما هي خطيئته؟“

\par 24 فقال لي ميخائيل رئيس الملائكة: ”هذا الرجل زاني؛ وقد نفذ شهوة صغيرة، فأمر بشنقه“.

\par \textit{المسيح الدجال}

\par 25 وأخذوني إلى الشمال، فرأيت هناك رجلاً مقيداً بقضبان حديدية.

\par 26 فسألته: "من هذا؟" فقال لي:

\par 27 "هذا هو الذي يقول أنا ابن الله، وهو الذي صنع الحجارة خبزاً والماء خمراً."

\par 28 فقال النبي: «أخبروني ما هي هيئته، وسأخبر بني البشر لئلا يؤمنوا به».

\par 29 فقال لي: «منظر وجهه كوجه رجل وحشي. عينه اليمنى كنجم يطلع عند الفجر، والأخرى ساكنة

\par 30 فمه ذراع، وأسنانه شبر،

\par 31 أصابعه مثل المناجل، وباطن قدميه شبران، وعلى جبهته نقش "المسيح الدجال".

\par 32 رُفِعَ إلى السماء، وسينزل إلى أبعد مدى

\par 33 مثل هاديس. في بعض الأحيان سيكون طفلاً، وفي أخرى رجلاً عجوزًا.

\par 34 فقال النبي: يا رب، كيف تسمح لجنس البشر أن يضل؟

\par 35 وقال الله: "اسمع يا نبي! إنه يصبح طفلاً وشيخًا، ولا يصدقه أحد أنه ابني الحبيب

\par 36 وبعد هذا يُنفخ بوق، وتُفتح القبور، ويقوم الأموات بلا فساد

\par 37 ثم بعد أن يسمع الخصم التهديد الرهيب، يختبئ في الظلام الخارجي

\par 38 ثم تهلك السماء والأرض والبحر.

\par 39 "فأحرق السماء ثمانين ذراعاً والأرض ثمانمائة ذراع."

\par 40 فقال النبي: «وفي ماذا خطأت السماء؟»

\par 41 فقال الله: «لأن [...] هو الشر».

\par 42 فقال النبي: يا رب، بماذا خطئت الأرض؟

\par 43 وقال الله: «لأن الخصم الذي يسمع تهديدي الرهيب سيختبئ (فيه)، وبسبب ذلك سأذيب الأرض ومعها متمردي جنس البشر».

\chapter{5}

\par \textit{عقوبات أخرى}

\par 1 وقال النبي: "ارحم يا رب شعب النصارى".

\par 2 ورأيت امرأة معلقة وأربعة حيوانات برية تمص ثدييها

\par 3 فقال لي الملائكة: إنها كانت تكره إعطاء حليبها، بل كانت أيضًا ترمي الأطفال في الأنهار.

\par 4 ورأيت ظلامًا رهيبًا وليلًا بلا نجوم ولا قمر.

\par 5 لا يوجد شاب ولا كبير، ولا أخ مع أخيه، ولا أم مع ابنها، ولا امرأة مع زوجها.

\par 6 فبكيت وقلت: يا رب، يا رب، ارحم الخطاة.

\par {عزرا يُؤخذ إلى السماء}

\par 7 وبينما كنت أقول هذا، جاءت سحابة وأخذتني وأخذتني أيضًا إلى السماء

\par 8 ورأيت أحكامًا كثيرة فبكيت بكاءً مرًا وقلت:

\par 9 "كان من الأفضل لو لم يخرج الإنسان من بطن أمه"

\par 10 صاح الذين كانوا في العذاب قائلين: "منذ أن أتيت إلى هنا، يا قدوس الله، حصلنا على راحة طفيفة."

\par 11 وقال النبي: «طوبى للذين يبكون على خطاياهم».

\par \textit{الولادة وغايتها}

\par 12 وقال الله: «اسمع يا عزرا يا حبيبي! كما يلقي المزارع بذرة القمح في الأرض، كذلك يلقي الرجل بذره في مكان المرأة

\par 13 في الشهر الأول يكون كاملاً، وفي الثاني ينتفخ، وفي الثالث ينبت الشعر، وفي الرابع تنبت الأظافر، وفي الخامس يصبح حليبيًا، وفي السادس يكون جاهزًا ونشطًا، وفي السابع يكون مُهيأً، [في الثامن...]، وفي التاسع تُفتح قضبان أبواب المرأة وتولد سليمة على الأرض

\par 14 وقال النبي: «كان خيرًا للإنسان ألا يولد

\par 15 يا ويلكم أيها البشر، في ذلك الوقت الذي تأتي فيه الدينونة!

\par 16 فقلت للرب: يا رب لماذا خلقت الإنسان وأسلمته للدينونة؟

\par 17 وقال الله في بيانه العالي: ﴿لا أغفر لمن يخالف عهدي﴾

\par 18 فقال النبي: يا رب، أين صلاحك؟

\par 19 وقال الله أعددت كل شيء من أجل الإنسان، ولكن الإنسان لم يحفظ وصاياي.

\par \textit{العقوبات والمكافآت}

\par 20 وقال النبي: «ربِّ، أرني العذاب والجنة».

\par 21 وقادني الملائكة إلى المشرق، فرأيت شجرة الحياة.

\par 22 ورأيت هناك حنوك وإيليا وموسى وبطرس وبولس ولوقا ومتى وجميع الأبرار والآباء.

\par 23 ورأيت هناك عقاب الهواء وهبوب الرياح ومخازن الجليد والعقوبات الأبدية

\par 24 ورأيت هناك رجلاً معلقًا من جمجمته.

\par 25 فقالوا لي: هذا نقل الحدود.

\par 26 وهناك رأيت أحكامًا عظيمة، فقلت للرب: «يا رب، يا رب، أي من الناس وُلد لم يخطئ؟»

\par 27 وقادوني إلى أسفل في تارتاروس، ورأيت جميع الخطاة ينوحون ويبكون وينتحبون بشدة

\par 28 وأنا أيضًا بكيت عندما رأيتُ جنس البشر يُعاقَب هكذا.

\chapter{6}

\par 1 ثم قال لي الله: "عزرا، هل تعرف أسماء الملائكة الذين على إتمام النبوة:

\par 2 ميخائيل، جبرائيل، أوريل، رافائيل، جابوثلون، آكر، أرفوجيتونوس، بيبوروس، زبوليون؟»

\par \textit{عزرا يكافح من أجل روحه}

\par 3 ثم جاءني صوت: "تعال إلى هنا، مت يا عزرا يا حبيبي! أعد ما أُؤتمن عليه."

\par 4 فقال النبي: «ومن أين لك أن تخرج روحي؟»

\par 5 فقال الملائكة: نلقيه في فمك.

\par 6 فقال النبي: «لقد تكلمت مع الله فمًا إلى فم، ولن يخرج الأمر من هناك».

\par 7 وقالت الملائكة: سنخرجه من أنفك.

\par 8 "فقال النبي: «لقد شمت أنفي مجد الله»."

\par 9 فقالت الملائكة: «نخرجه بأعينكم».

\par 10 وقال النبي: «لقد رأت عيني ظهر الله».

\par 11 فقالت الملائكة: «نخرجه من رأسك».

\par 12 فقالت الملائكة: نخرجه من خلال رجليك. وقال النبي: لقد مشيت مع موسى على الجبل، ولن يخرج من هناك.

\par 13 فقالت الملائكة: نلقيه من خلال رؤوس أظفارك

\par 14 وقال النبي: «دخلت قدماي في المحراب».

\par 15 فانطلق الملائكة غير خاسرين قائلين: يا رب لا نقدر أن نأخذ روحه.

\par 16 ثم قال لابنه الوحيد: "انزل يا ابني الحبيب مع حشد غفير من الملائكة، خذ روح حبيبي عزرا."

\par 17 لأن الرب، إذ أخذ جيشًا غفيرًا من ملائكة كثيرين، قال للنبي: «أعطني العربون الذي استودعتك إياه. الإكليل مُعدّ لك».

\par 18 فقال النبي: "يا رب، إذا أخذت روحي مني، فمن سيبقى لك ليُرافع عن جنس البشر؟"

\par 19 فقال الله: "أيها البشر والأرضيون، لا ترافعوا لي في هذه القضية."

\par 20 وقال النبي: «لا أزال أدعو».

\par 21 وقال الله: «أعطوا في هذه الأثناء ما ائتمنتم عليه، فالإكليل مُعدّ لكم».

\par 22 تعال إلى هنا، مت، حتى تتمكن من تحقيقها.

\par 23 "فبدأ النبي يتكلم بدموع وقال: يا رب، ما الفائدة من أن أرافع إليك وأنا أسقط إلى الأرض؟

\par 24 ويل، ويل! لأني سأُستهلك بالديدان.

\par 25 ابكي عليّ يا جميع القديسين والأتقياء، أتوسل إليك كثيرًا وأرجوك أن ترحمني.

\par 26 أُسْلِمَتُ إِلَى الْمَوْتِ! اَبْكَوْا عَلَيَّ يَا جَمِيعَ الْقُدُّوسِ وَالأَبْرَارِ، لأَنِّي دَخَلْتُ جَزْءَ الْجَحِيمِ

\chapter{7}

\par \textit{النفس والجسد}

\par 1 فقال له الله: «اسمع يا عزرا يا حبيبي. أنا، وأنا خالد، تلقيت صليبًا، وذقت الخل والمرارة، ووُضعت في قبر

\par 2 وأقمتُ مختاريّ، واستدعيتُ آدم من الجحيم، حتى يكون جنس البشر

\par 3 [...]. فلا تخشَ الموت. فما هو مني، أي الروح، يرحل إلى السماء. وما هو من الأرض، أي الجسد، يرحل إلى الأرض التي أُخذ منها.

\par 4 فقال النبي: «ويل، ويل! ماذا أفعل؟ كيف أتصرف؟ لا أعلم».

\par \textit{صلاة ختامية}

\par 5 ثم بدأ عزرا المبارك يقول: "يا إلهي الأزلي، خالق الخليقة كلها، الذي قاس السماء بشبر، وضم الأرض بيده،

\par 6 الذي يقود الكروبيم، الذي أخذ النبي إيليا إلى السماء في مركبة نارية،

\par 7 الذي يُعطي الرعاية لكل ذي جسد، الذي يخافه كل شيء ويرتعد من وجه قدرتك،

\par 8 اسمعني يا من تضرعت كثيرًا

\par 9 وأعطِ كل من ينسخ هذا الكتاب ويحفظه ويتذكر اسمي ويحفظ ذكراي بالكامل، امنحهم بركة من السماء

\par 10 وبارك جميع ممتلكاته، كما بارك أطراف يوسف.

\par 11 ولا يذكر خطيئته السابقة يوم القيامة.

\par 12 أولئك الذين لا يؤمنون بهذا الكتاب سيُحرقون مثل سدوم وعمورة

\par 13 وسمع صوتًا يقول له: «عزرا، يا حبيبي، سأمنح كل واحد ما طلبته».

\par \textit{وفاة عزرا ودفنه}

\par 14 وعلى الفور سلم روحه الثمينة بشرف كبير في الثامن عشر من شهر أكتوبر

\par 15 ودفنوه بالبخور والمزامير. إن جسده الثمين والمقدس يُعطي تقويةً لا تنقطع للنفوس والأجساد لمن يتقدمون إليه طوعًا

\par 16 المجد والقوة والإكرام والسجود لمن يليق به، باسم الآب والابن والروح القدس، الآن وكل أوان وإلى دهر الدهور. آمين

\end{document}