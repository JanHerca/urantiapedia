\begin{document}

\title{حجي}


\chapter{1}

\par 1 فِي السَّنَةِ الثَّانِيَةِ لِدَارِيُوسَ الْمَلِكِ فِي الشَّهْرِ السَّادِسِ فِي أَوَّلِ يَوْمٍ مِنَ الشَّهْرِ كَانَتْ كَلِمَةُ الرَّبِّ عَنْ يَدِ حَجَّيِ النَّبِيِّ إِلَى زَرُبَّابِلَ بْنِ شَأَلْتِئِيلَ وَالِي يَهُوذَا وَإِلَى يَهُوشَعَ بْنِ يَهُوصَادَاقَ الْكَاهِنِ الْعَظِيمِ:
\par 2 هَكَذَا قَالَ رَبُّ الْجُنُودِ: [هَذَا الشَّعْبُ قَالَ إِنَّ الْوَقْتَ لَمْ يَبْلُغْ وَقْتَ بِنَاءِ بَيْتِ الرَّبِّ].
\par 3 فَكَانَتْ كَلِمَةُ الرَّبِّ عَنْ يَدِ حَجَّيِ النَّبِيِّ:
\par 4 [هَلِ الْوَقْتُ لَكُمْ أَنْتُمْ أَنْ تَسْكُنُوا فِي بُيُوتِكُمُ الْمُغَشَّاةِ وَهَذَا الْبَيْتُ خَرَابٌ؟
\par 5 وَالآنَ فَهَكَذَا قَالَ رَبُّ الْجُنُودِ: اجْعَلُوا قَلْبَكُمْ عَلَى طُرُقِكُمْ.
\par 6 زَرَعْتُمْ كَثِيراً وَدَخَّلْتُمْ قَلِيلاً. تَأْكُلُونَ وَلَيْسَ إِلَى الشَّبَعِ. تَشْرَبُونَ وَلاَ تَرْوُونَ. تَكْتَسُونَ وَلاَ تَدْفَأُونَ. وَالآخِذُ أُجْرَةً يَأْخُذُ أُجْرَةً لِكِيسٍ مَنْقُوبٍ].
\par 7 هَكَذَا قَالَ رَبُّ الْجُنُودِ: [اجْعَلُوا قَلْبَكُمْ عَلَى طُرُقِكُمْ.
\par 8 اِصْعَدُوا إِلَى الْجَبَلِ وَأْتُوا بِخَشَبٍ وَابْنُوا الْبَيْتَ فَأَرْضَى عَلَيْهِ وَأَتَمَجَّدَ قَالَ الرَّبُّ.
\par 9 انْتَظَرْتُمْ كَثِيراً وَإِذَا هُوَ قَلِيلٌ. وَلَمَّا أَدْخَلْتُمُوهُ الْبَيْتَ نَفَخْتُ عَلَيْهِ. لِمَاذَا؟ يَقُولُ رَبُّ الْجُنُودِ. لأَجْلِ بَيْتِي الَّذِي هُوَ خَرَابٌ وَأَنْتُمْ رَاكِضُونَ كُلُّ إِنْسَانٍ إِلَى بَيْتِهِ.
\par 10 لِذَلِكَ مَنَعَتِ السَّمَاوَاتُ مِنْ فَوْقِكُمُ النَّدَى وَمَنَعَتِ الأَرْضُ غَلَّتَهَا.
\par 11 وَدَعَوْتُ بِالْحَرِّ عَلَى الأَرْضِ وَعَلَى الْجِبَالِ وَعَلَى الْحِنْطَةِ وَعَلَى الْمِسْطَارِ وَعَلَى الزَّيْتِ وَعَلَى مَا تُنْبِتُهُ الأَرْضُ وَعَلَى النَّاسِ وَعَلَى الْبَهَائِمِ وَعَلَى كُلِّ أَتْعَابِ الْيَدَيْنِ].
\par 12 حِينَئِذٍ سَمِعَ زَرُبَّابِلُ بْنُ شَأَلْتِئِيلَ وَيَهُوشَعُ بْنُ يَهُوصَادَاقَ الْكَاهِنِ الْعَظِيمِ وَكُلُّ بَقِيَّةِ الشَّعْبِ صَوْتَ الرَّبِّ إِلَهِهِمْ وَكَلاَمَ حَجَّيِ النَّبِيِّ كَمَا أَرْسَلَهُ الرَّبُّ إِلَهُهُمْ. وَخَافَ الشَّعْبُ أَمَامَ وَجْهِ الرَّبِّ.
\par 13 فَقَالَ حَجَّي رَسُولُ الرَّبِّ بِرِسَالَةِ الرَّبِّ لِجَمِيعِ الشَّعْبِ: [أَنَا مَعَكُمْ يَقُولُ الرَّبُّ].
\par 14 وَنَبَّهَ الرَّبُّ رُوحَ زَرُبَّابِلَ بْنِ شَأَلْتِئِيلَ وَالِي يَهُوذَا وَرُوحَ يَهُوشَعَ بْنِ يَهُوصَادَاقَ الْكَاهِنِ الْعَظِيمِ وَرُوحَ كُلِّ بَقِيَّةِ الشَّعْبِ. فَجَاءُوا وَعَمِلُوا الشُّغْلَ فِي بَيْتِ رَبِّ الْجُنُودِ إِلَهِهِمْ
\par 15 فِي الْيَوْمِ الرَّابِعِ وَالْعِشْرِينَ مِنَ الشَّهْرِ السَّادِسِ فِي السَّنَةِ الثَّانِيَةِ لِدَارِيُوسَ الْمَلِكِ.

\chapter{2}

\par 1 فِي الشَّهْرِ السَّابِعِ فِي الْحَادِي وَالْعِشْرِينَ مِنَ الشَّهْرِ كَانَتْ كَلِمَةُ الرَّبِّ عَنْ يَدِ حَجَّيِ النَّبِيِّ:
\par 2 [قُلْ لِزَرُبَّابِلَ بْنَ شَأَلْتِئِيلَ وَالِي يَهُوذَا وَيَهُوشَعَ بْنِ يَهُوصَادَاقَ الْكَاهِنِ الْعَظِيمِ وَبَقِيَّةِ الشَّعْبِ:
\par 3 مَنِ الْبَاقِي فِيكُمُ الَّذِي رَأَى هَذَا الْبَيْتَ فِي مَجْدِهِ الأَوَّلِ؟ وَكَيْفَ تَنْظُرُونَهُ الآنَ؟ أَمَا هُوَ فِي أَعْيُنِكُمْ كَلاَ شَيْءٍ!
\par 4 فَالآنَ تَشَدَّدْ يَا زَرُبَّابِلُ يَقُولُ الرَّبُّ وَتَشَدَّدْ يَا يَهُوشَعُ بْنُ يَهُوصَادَاقَ الْكَاهِنُ الْعَظِيمُ وَتَشَدَّدُوا يَا جَمِيعَ شَعْبِ الأَرْضِ يَقُولُ الرَّبُّ وَاعْمَلُوا فَإِنِّي مَعَكُمْ يَقُولُ رَبُّ الْجُنُودِ.
\par 5 حَسَبَ الْكَلاَمِ الَّذِي عَاهَدْتُكُمْ بِهِ عِنْدَ خُرُوجِكُمْ مِنْ مِصْرَ وَرُوحِي قَائِمٌ فِي وَسَطِكُمْ. لاَ تَخَافُوا.
\par 6 لأَنَّهُ هَكَذَا قَالَ رَبُّ الْجُنُودِ: هِيَ مَرَّةٌ (بَعْدَ قَلِيلٍ) فَأُزَلْزِلُ السَّمَاوَاتِ وَالأَرْضَ وَالْبَحْرَ وَالْيَابِسَةَ
\par 7 وَأُزَلْزِلُ كُلَّ الأُمَمِ. وَيَأْتِي مُشْتَهَى كُلِّ الأُمَمِ فَأَمْلأُ هَذَا الْبَيْتَ مَجْداً قَالَ رَبُّ الْجُنُودِ.
\par 8 لِي الْفِضَّةُ وَلِي الذَّهَبُ يَقُولُ رَبُّ الْجُنُودِ.
\par 9 مَجْدُ هَذَا الْبَيْتِ الأَخِيرِ يَكُونُ أَعْظَمَ مِنْ مَجْدِ الأَوَّلِ قَالَ رَبُّ الْجُنُودِ. وَفِي هَذَا الْمَكَانِ أُعْطِي السَّلاَمَ يَقُولُ رَبُّ الْجُنُودِ].
\par 10 فِي الرَّابِعِ وَالْعِشْرِينَ مِنَ الشَّهْرِ التَّاسِعِ فِي السَّنَةِ الثَّانِيَةِ لِدَارِيُوسَ كَانَتْ كَلِمَةُ الرَّبِّ عَنْ يَدِ حَجَّيِ النَّبِيِّ:
\par 11 [هَكَذَا قَالَ رَبُّ الْجُنُودِ: اسْأَلِ الْكَهَنَةَ عَنِ الشَّرِيعَةِ:
\par 12 إِنْ حَمَلَ إِنْسَانٌ لَحْماً مُقَدَّساً فِي طَرَفِ ثَوْبِهِ وَمَسَّ بِطَرَفِهِ خُبْزاً أَوْ طَبِيخاً أَوْ خَمْراً أَوْ زَيْتاً أَوْ طَعَاماً مَا فَهَلْ يَتَقَدَّسُ؟] فَأَجَابَ الْكَهَنَةُ: [لاَ].
\par 13 فَقَالَ حَجَّي: [إِنْ كَانَ الْمُنَجَّسُ بِمَيِّتٍ يَمَسُّ شَيْئاً مِنْ هَذِهِ فَهَلْ يَتَنَجَّسُ؟] فَأَجَابَ الْكَهَنَةُ: [يَتَنَجَّسُ].
\par 14 فَقَالَ حَجَّي: [هَكَذَا هَذَا الشَّعْبُ وَهَكَذَا هَذِهِ الأُمَّةُ قُدَّامِي يَقُولُ الرَّبُّ وَهَكَذَا كُلُّ عَمَلِ أَيْدِيهِمْ وَمَا يُقَرِّبُونَهُ هُنَاكَ. هُوَ نَجِسٌ.
\par 15 وَالآنَ فَاجْعَلُوا قَلْبَكُمْ مِنْ هَذَا الْيَوْمِ فَرَاجِعاً قَبْلَ وَضْعِ حَجَرٍ عَلَى حَجَرٍ فِي هَيْكَلِ الرَّبِّ.
\par 16 مُذْ تِلْكَ الأَيَّامِ كَانَ أَحَدُكُمْ يَأْتِي إِلَى عَرَمَةِ عِشْرِينَ فَكَانَتْ عَشَرَةً. أَتَى إِلَى حَوْضِ الْمِعْصَرَةِ لِيَغْرُفَ خَمْسِينَ فُورَةً فَكَانَتْ عِشْرِينَ.
\par 17 قَدْ ضَرَبْتُكُمْ بِاللَّفْحِ وَبِالْيَرَقَانِ وَبِالْبَرَدِ فِي كُلِّ عَمَلِ أَيْدِيكُمْ وَمَا رَجَعْتُمْ إِلَيَّ يَقُولُ الرَّبُّ!
\par 18 فَاجْعَلُوا قَلْبَكُمْ مِنْ هَذَا الْيَوْمِ فَصَاعِداً مِنَ الْيَوْمِ الرَّابِعِ وَالْعِشْرِينَ مِنَ الشَّهْرِ التَّاسِعِ مِنَ الْيَوْمِ الَّذِي فِيهِ تَأَسَّسَ هَيْكَلُ الرَّبِّ اجْعَلُوا قَلْبَكُمْ.
\par 19 هَلِ الْبَذْرُ فِي الأَهْرَاءِ بَعْدُ؟ وَالْكَرْمُ وَالتِّينُ وَالرُّمَّانُ وَالزَّيْتُونُ لَمْ يَحْمِلْ بَعْدُ. فَمِنْ هَذَا الْيَوْمِ أُبَارِكُ].
\par 20 وَصَارَتْ كَلِمَةُ الرَّبِّ ثَانِيَةً إِلَى حَجَّي فِي الرَّابِعِ وَالْعِشْرِينَ مِنَ الشَّهْرِ:
\par 21 قُلْ لِزَرُبَّابِلَ وَالِي يَهُوذَا: [إِنِّي أُزَلْزِلُ السَّمَاوَاتِ وَالأَرْضَ
\par 22 وَأَقْلِبُ كُرْسِيَّ الْمَمَالِكِ وَأُبِيدُ قُوَّةَ مَمَالِكِ الأُمَمِ وَأَقْلِبُ الْمَرْكَبَاتِ وَالرَّاكِبِينَ فِيهَا وَيَنْحَطُّ الْخَيْلُ وَرَاكِبُوهَا كُلٌّ مِنْهَا بِسَيْفِ أَخِيهِ.
\par 23 فِي ذَلِكَ الْيَوْمِ يَقُولُ رَبُّ الْجُنُودِ آخُذُكَ يَا زَرُبَّابِلُ عَبْدِي ابْنُ شَأَلْتِئِيلَ يَقُولُ الرَّبُّ وَأَجْعَلُكَ كَخَاتِمٍ لأَنِّي قَدِ اخْتَرْتُكَ]. يَقُولُ رَبُّ الْجُنُودِ.


\end{document}