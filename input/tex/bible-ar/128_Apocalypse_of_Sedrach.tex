\begin{document}

\title{نهاية العالم لسدرخ}

\chapter{1}

عظة للقديس سدرخ المبارك عن المحبة والتوبة والمسيحيين الأرثوذكس والمجيء الثاني لربنا يسوع المسيح. يا معلم، بارك.

\par 1 أيها الأحباء، يجب ألا نفضل شيئًا أكثر من الحب غير المصطنع.

\par 2 نحن نرتكب العديد من الأخطاء في كل ساعة، ليلاً ونهاراً، ولهذا السبب علينا أن نكتسب المحبة، لأنها تغطي العديد من الخطايا.

\par 3 ماذا نكسب يا أبنائي إذا امتلكنا كل شيء ولكن لم يكن لدينا الحب المُخلِّص؟

\par 4 كيف يستفيد المرء يا أبنائي إذا أقام وليمة عظيمة ودعا إليها ملكًا ونبيلًا، وأعد كل أنواع الأطعمة الباهظة حتى لا يفوته شيء؛ ومع ذلك، إذا لم يكن هناك ملح، فلا يمكن تناول تلك الوليمة؛ ولا يتحمل المرء النفقات فحسب، بل يهدر جهوده أيضًا، ويخجل من الضيوف

\par 5 الأمر نفسه ينطبق على حالتنا يا إخوتي؛ فماذا نستفيد، فأي نعمة نملكها بدون محبة؟

\par 6 كل عمل نقوم به باطل، حتى لو كان لدينا عذراء وصامنا وسهرنا وصلينا وأقامنا وليمة للفقراء

\par 7 وإن قدم أحد عطايا لله، أو قدم باكورة جميع أمواله، أو بنى كنائس، أو فعل أي شيء آخر بدون محبة، فسيحسب ذلك عند الله لا شيء، لأن هذه الأشياء غير مقبولة

\par 8 هكذا يقول النبي: «ذبيحة الكافر رجسٌ عند الرب».

\par 9 لا يُنصح بفعل أي شيء بدون حب.

\par 10 إن قلت: "أنا أكره أخي ولكني أحب المسيح" فأنت كاذب، ويوبخك يوحنا اللاهوتي، لأنه كيف يستطيع أحد لا يحب أخاه الذي رآه أن يحب الله الذي لم يره؟

\par 11 من الواضح أن كل من يكره أخاه ولكنه يظن أنه يحب المسيح فهو كاذب ويخدع نفسه

\par 12 لأن يوحنا اللاهوتي يقول إن لنا هذه الوصية من الله: أن من يحب الله ينبغي أن يحب أخاه أيضًا

\par 13 ويقول الرب نفسه أيضًا: "بهاتين (الوصيتين) يتعلق الناموس كله والأنبياء."

\par 14 يا لها من معجزة عجيبة ومفارقة أن من يملك المحبة يُتمم الناموس كله؛ فالمحبة هي إتمام الناموس

\par 15 يا قوة الحب التي تفوق الخيال؛ يا قوة الحب التي لا تُقاس!

\par 16 لا يوجد شيء أشرف من الحب، ولا يوجد شيء أعظم منه سواء في السماء أو على الأرض

\par 17 هذا الحب الإلهي هو رأس المال (الفضيلة)؛ ومن بين جميع الفضائل، يُعد الحب أعلى كمال في العالم

\par 18 سكنت في قلب هابيل؛ وعملت مع الآباء؛ وحرست موسى؛ وجعلت داود مسكنًا للروح القدس؛ وقوّت يوسف

\par 19 لكن لماذا أقول هذه الأشياء؟

\par 20 الأهم هو أن هذا الحب أنزل ابن الله من السماء

\par 21 من خلال المحبة تم الكشف عن كل الأشياء الجيدة؛ تم سحق الموت، وتم أسر الجحيم، وتم استدعاء آدم (من الموت)، ومن خلال المحبة تم تكوين قطيع واحد بعد ذلك من الملائكة والبشر.

\par 22 من خلال المحبة، فُتح الفردوس؛ ووُعِد بملكوت السماوات؛ وحول الأماكن الخراب إلى مدن، وملأ الجبال والكهوف بالأغاني؛ وعلّم الرجال والنساء الذين كانوا يسيرون في الطريق الضيق والحزين

\par 23 ولكن إلى متى سنطيل هذه العظة عن إنجازات المحبة التي لا تستطيع حتى الملائكة تحقيقها؟

\par 24 يا له من حب مبارك يمنح كل الخير!

\par 25 طوبى للرجل الذي يملك الإيمان الحقيقي والمحبة غير المزيفة؛ لأنه، كما قال المعلم، ليس هناك شيء أعظم من المحبة التي يضع الإنسان حياته من أجلها من أجل أصدقائه.

\chapter{2}

\par 1 وسمع صوتًا خفيًا في أذنيه: "ها أنت يا سدرخ، أنت الذي ترغب وترغب في التحدث مع الله وتطلب منه أن يكشف لك الأشياء التي ترغب في طلبها."

\par 2 فقال سدرخ: «ما هو يا سيدي؟»

\par 3 فقال له الصوت: «لقد أُرسلت إليك لأرفعك إلى السماء».

\par 4 فقال: «أريد أن أكلم الله وجهًا لوجه، ولكنني يا رب لا أستطيع الصعود إلى السموات».

\par 5 لكن الملاك بسط جناحيه، وأخذه وصعد إلى السماء، ثم صعد به إلى السماء الثالثة، ووقفت هناك لهيب اللاهوت

\chapter{3}

\par 1 فقال له الرب: "أهلًا وسهلًا بك يا عزيزي سدرخ."

\par 2 ما نوع الشكوى التي لديك على الله الذي خلقك، لأنك قلت: "أريد أن أتحدث مع الله وجهًا لوجه"؟

\par 3 فقال له سدرخ: إن الابن يشكو من الآب: يا سيدي لماذا خلقت الأرض؟

\par 4 فقال له الرب: «للإنسان».

\par 5 قال سدرخ: لماذا خلقت البحر ولماذا نشرت كل الخير على الأرض؟

\par 6 قال الرب: "للإنسان."

\par 7 قال له سدرخ: "إن كنت قد فعلت هذه الأشياء، فلماذا أهلكت الإنسان؟"

\par 8 وقال الرب: «الإنسان عملي وخليقة يدي، فأؤدبه كما أرى ذلك صحيحًا».

\chapter{4}

\par 1 قال له سدرخ: «تأديبك هو عقاب ونار، وهما مُرّان جدًا يا سيدي

\par 2 من الأفضل للإنسان أن لا يولد.

\par 3 فماذا فعلت يا سيدي؟ لماذا تعبت بيديك الطاهرة وخلقت الإنسان، ولم ترد أن ترحمه؟

\par 4 قال له الله: لقد خلقت الإنسان الأول آدم ووضعته في الجنة في وسط شجرة الحياة، وقلت له: كل من كل الفاكهة، واحذر من شجرة الحياة فقط، لأنك إذا أكلت منها ستموت موتًا.

\par 5 ولكنه عصى وصيتي، فغوا له الشيطان فأكل من الشجرة.

\chapter{5}

\par 1 قال له سدرخ: «بإرادتك انخدع آدم يا سيدي

\par 2 لقد أمرت ملائكتك بالسجود لآدم، ولكن الذي كان أول الملائكة عصى أمرك ولم يسجد له، فنفيتّه لأنه خالف أمرك ولم يخرج من خلق يديك.

\par 3 إن كنت تحب الإنسان، فلماذا لم تقتل الشيطان، خالق كل إثم؟

\par 4 من يستطيع محاربة روح غير مرئية؟

\par 5 إنه يدخل إلى قلوب البشر كالدخان ويعلمهم كل أنواع الخطيئة.

\par 6 حتى أنه يحاربك، أيها الإله الخالد، فماذا يستطيع الإنسان البائس أن يفعل ضده؟

\par 7 لكن ارحم يا سيدي، وأبطل العقاب؛ وإلا فاقبلني أيضًا مع الخطاة، لأنه إن لم تكن رحيمًا مع الخطاة، فأين رحمتك وأين رأفتك يا رب؟


\chapter{6}

\par 1 وقال له الله: «ليكن معلومًا لك أن كل ما أمرت الإنسان أن يفعله كان في متناوله

\par 2 جعلته حكيمًا ووارثًا للسماء والأرض، وأخضعت كل شيء تحته، فهرب كل حي منه ومن وجهه

\par 3 ولكن بعد أن تلقى هداياي، أصبح أجنبيًا، زانيًا، وخاطئًا.

\par 4 قل لي، أي نوع من الأب يعطي ميراثًا لابنه، وبعد أن يأخذ المال (الابن) يذهب بعيدًا ويترك أباه، ويصبح غريبًا وفي خدمة الأجانب.

\par 5 ثم إن الأب إذ يرى أن ابنه قد تخلى عنه (وذهب بعيدًا)، يظلم قلبه ويذهب ويسترد ثروته وينفي ابنه من مجده لأنه تخلى عن أباه.

\par 6 كيف أعطيته أنا الإله العجيب الغيور كل شيء، وهو بعدما أخذه صار زانياً وخاطئاً؟

\chapter{7}

\par 1 قال له سدرخ: "أنت يا سيدي خلقت الإنسان، وأنت تعرف حالة إرادته ومعرفته المتدنية؟ وأنت ترسل الإنسان إلى العقاب بحجة كاذبة، لذا انزعه

\par 2 هل أنا وحدي من المفترض أن أملأ العوالم السماوية؟

\par 3 وإن لم يكن الأمر كذلك يا رب، فخلص الإنسان أيضًا.

\par 4 لقد تجاوز الإنسان البائس مشيئتك يا رب.

\par 5 "لماذا ترمي الكلمات من حولي وكأنها شبكة يا سدرخ؟

\par 6 لقد خلقت آدم وزوجته والشمس وقلت انظروا إلى بعضكم البعض من هو منور.

\par 7 وكانت الشمس وآدم من خلق واحد، ولكن امرأة آدم كانت أجمل من القمر، وأعطت نفسها الحياة.

\par 8 قال سدرخ: "ما فائدة الأشياء الجميلة إذا ذبلت وتحولت إلى تراب؟

\par 9 كيف قلت يا رب: لا تجازِ شرًا بشر؟

\par 10 كيف يا سيدي، وكلمة إلهك لا تكذب أبدًا؟

\par 11 ولماذا كافأت الإنسان هكذا، إن كنت لا تريد الشر بالشر؟

\par 12 أعلم أن البغل حيوان ماكر من بين الحيوانات ذات الأربع، فهو ليس سوى حيوان ماكر؛ ومع ذلك، فإننا نستخدم اللجام لتوجيهه حيث نريد

\par 13 لديك ملائكة؛ أرسلهم لمراقبة الإنسان، وعندما يخطو نحو الخطيئة، تمسك بقدمه، ولن يذهب إلى حيث يريد

\chapter{8}

\par 1 فقال له الله: إذا أمسكت بقدمه يقول: لم تعطني نعمة في العالم، ولذلك تركته لشهواته لأني أحببته، ولذلك أرسلت ملائكتي الصالحين لمراقبته ليلاً ونهاراً.

\par 2 قال سدرخ: "أعلم أنك يا سيد، أحببت الإنسان أولًا بين مخلوقاتك؛ وبين ذوات الأربع، الخراف؛ وبين الأشجار، الزيتونة؛ وبين النباتات المثمرة، الكرمة؛ وبين الأشياء الطائرة، النحلة؛ وبين الأنهار، الأردن؛ وبين المدن، أورشليم

\par 3 لكن الإنسان أيضًا يحب كل هؤلاء يا سيدي.

\par 4 قال الله لسدرخ: "سأسألك شيئًا واحدًا يا سدرخ؛ إذا استطعت أن تجيبني، فقد تحديتني بحق، مع أنك أغريت خالقك."

\par 5 قال سدرخ: "تكلم."

\par 6 قال الرب الإله: "منذ أن خلقت كل شيء، كم من الناس وُلدوا، وكم من ماتوا، وكم من سيموتون، وكم شعرة لديهم؟

\par 7 أخبرني يا سدرخ، منذ أن خُلقت السماء والأرض، كم شجرة خُلقت في العالم، وكم منها ستسقط، وكم منها ستُصنع، وكم ورقة فيها؟

\par 8 أخبرني يا سدرخ، منذ أن خلقت البحر، كم موجة ارتفعت، وكم موجة تدحرجت قليلاً، وكم موجة سترتفع، وكم ريح تهب بالقرب من شاطئ البحر؟

\par 9 أخبرني يا سدرخ، منذ خلق العالم منذ العصور عندما كان الهواء مليئًا بالمطر، كم قطرة سقطت على العالم وكم ستسقط؟

\par 10 فقال سدرخ: "أنت وحدك تعلم كل هذه الأمور يا رب؛ أنت وحدك على دراية بكل هذه الأمور؛ أتوسل إليك فقط أن تحرر الإنسان من العقاب، وإلا فسأذهب أنا نفسي إلى العقاب ولن أنفصل عن جنسنا."

\chapter{9}

\par 1 وقال الله لابنه الوحيد: «اذهب، خذ نفس سدرخ حبيبي، وضعها في الفردوس».

\par 2 وقال الابن الوحيد لسدرخ: أعطني ما أودعه أبونا في بطن أمك في مسكنك المقدس منذ ولادتك.

\par 3 قال سدرخ: لا أعطيك نفسي.

\par 4 فقال له الله: ولماذا أرسلت ولماذا أتيت إلى هنا وأنت تكذب علي؟

\par 5 لقد أمرني والدي ألا أتردد في أخذ روحك، لذا أعطني روحك التي ترغب بها بشدة.

\chapter{10}

\par 1 فقال سدرخ لله: «من أين تأخذ نفسي، ومن أي عضو؟»

\par 2 فقال له الله: أما تعلم أنها موضوعة في وسط رئتيك وقلبك وأنها موزعة على جميع الأعضاء؟

\par 3 "إنه يخرج عن طريق البلعوم والحنجرة والفم، وعندما يكون من الواجب خروجه فإنه يسحب بصعوبة في أول الأمر، وعندما يتجمع من الأظافر ومن جميع الأعضاء يكون هناك بالضرورة جهد كبير في انفصاله عن الجسم وانفصاله عن القلب".

\par 4 عندما سمع سدرخ كل هذه الأشياء، وتذكر ذكرى الموت، اضطرب للغاية وقال لله: "يا رب، أعطني بعض الوقت حتى أتمكن من البكاء، لأنني سمعت أن الدموع تفعل الكثير ويمكن أن تصبح شفاءً كافياً لجسد مخلوقاتك المتواضع".

\chapter{11}


\par 1 وبدأ يبكي وينوح قائلاً: "يا رأسًا عجيبًا، مُزينًا كالسماء؛ يا نور الشمس على السماء والأرض؛ شعرك معروف من تيماء، وعيناك من بصرى، وأذناك من الرعد، ولسانك من البوق، ودماغك مخلوق صغير؛ الرأس، حركة الجسم كله، جدير بالثقة وجميل جدًا، محبوب من الجميع، ولكن بمجرد أن يسقط في الأرض لا يُعرف."

\par 2 يا أيها الأيدي التي تمسك جيدًا، والتي يسهل تدريبها وتعمل بجد، والتي من خلالها يتغذى الجسد

\par 3 يا أيها الأيدي الماهرة، في جمع المواد، زينتم معًا المنازل

\par 4 أيتها الأصابع المزينة والمزينة بالذهب والفضة، حتى الهياكل العظيمة تُصنع بالأصابع، والمفاصل الثلاثة تمد الكفين وتجمع الطيبات، لكنكِ الآن أصبحتِ غرباء عن هذا العالم

\par 5 يا أقدامًا، التي تمشي جيدًا، تتحرك من تلقاء نفسها بسرعة كبيرة ولا تكل

\par 6 يا ركبتين متصلتين هكذا، بدونكِ لا يتحرك الجسد؛ تجري القدمان مع الشمس والقمر، الليل والنهار، جامعتين كل الأشياء معًا، الطعام والشراب اللذين يغذيان الجسد

\par 7 يا أقدامًا سريعة الحركة، تُحرك وجه الأرض وتُزين البيوت بكل خير

\par 8 أيتها الأقدام التي تحمل الجسد كله، التي تسير مباشرة إلى المعابد، تتوب وتتضرع إلى القديسين، والآن فجأة يجب أن تبقوا ثابتين

\par 9 يا رأسي، ويدي، وقدمي، حتى الآن أمسكتكم بقوة.

\par 10 أيتها النفس، ما الذي وضعك في هذا الجسد المتواضع البائس؟

\par 11 ولكن الآن، وأنت منفصل عنه، تصعد إلى حيث يدعوك الرب، ويذهب الجسد البائس للدينونة

\par 12 يا لجسدك الجميل، وشعرك الذي تساقطته النجوم، ورأسك الذي يزدان كالسماء.

\par 13 "وجهك زكي الرائحة، وعيناك كالنوافذ، وصوتك كصوت البوق، ولسانك يتكلم بسهولة، ولحيتك مشذبة جيداً، وشعرك كالنجوم، ورأسك مرتفع كالسماء، وجسدك مزين، والمنير أنيق ومشهور، ولكن الآن بعد سقوطك في الأرض، فإن جمالك تحت الأرض غير مرئي."

\chapter{12}

\par 1 قال له المسيح: «قف يا سدرخ، إلى متى تذرف الدموع وتتأوه؟   

\par 2 لقد فتحت لك الجنة وبعد موتك ستحيا.

\par 3 فقال له صدراخ: «يا رب، مرة أخرى أكلمك وأنا حي قبل أن أموت، فلا تتجاهل طلبي».

\par 4 فقال له الرب: تكلم يا سدرخ.

\par 5 (وقال سدرخ) «إذا عاش الإنسان ثمانين أو تسعين أو مائة سنة، وعاشها في الخطيئة، ثم تاب في النهاية وعاش في التوبة، فكم من أيام التوبة تغفر له خطاياه؟»

\par 6 فقال له الله: إن رجع بعد مائة سنة أو ثمانين سنة وتاب ثلاث سنين وأثمر براً فأصابه الموت، ألا أذكر كل خطاياه؟

\chapter{13}

\par 1 قال له سدرخ: ثلاث سنوات كثيرة يا سيدي

\par 2 قد يأتي موته ولن يُكمل توبته

\par 3 ارحم يا رب صورتك وكن رحيمًا، لأن ثلاث سنوات كثيرة جدًا

\par 4 قال له الله: "إذا عاش رجل بعد مئة عام وتذكر موته واعترف أمام الناس، ووجدته بعد عام واحد؟ سأغفر له جميع خطاياه."

\par 5 قال سدرخ مرة أخرى: "يا رب، أطلب رحمتك مرة أخرى على مخلوقك؛ سنة واحدة كثيرة، وربما يأتي موته ويخطفه فجأة."

\par 6 قال له المخلص: "يا سدرخ، يا حبيبي، سأسألك سؤالاً واحدًا، ثم يمكنك استئناف استفساراتك: إذا تاب الخاطئ أربعين يومًا، ألا أتذكر حقًا جميع خطاياه؟"

\chapter{14}

\par 1 وقال سدرخ لرئيس الملائكة ميخائيل: "اسمعني أيها الحامي القوي؛ ساعدني وتشفع ليرحم الله العالم."

\par 2 فسقطوا على وجوههم وتضرعوا إلى الله قائلين: «يا رب، علمنا كيف وبأي توبة يخلص الإنسان، وبأي تعب».

\par 3 قال الله: "بالتوبة والدعاء والطقوس، وبذرف الدموع والتأوه الحار".

\par 4 أما تعلمون أن نبيي داود نجا بسبب الدموع، وأن الباقين نجاوا في لحظة واحدة؟

\par 5 أنت تعلم يا صدراخ أنه توجد أمم ليس لها ناموس ولكنهم يكملون الناموس وهم لا يعتمدون ولكن روحي الإلهي يدخل فيهم فيتحولون إلى معموديتي وأستقبلهم مع الصديقين في حضن إبراهيم.

\par 6 وهناك من اعتمدوا بمعموديتي ومُسحوا بمرّي الإلهي، ولكنهم امتلأوا باليأس ولم يغيروا رأيهم.

\par 7 ومع ذلك، فإني أنتظرهم بشفقة كبيرة ورحمة كبيرة، حتى يتوبوا

\par 8 لكنهم يفعلون ما تكرهه ألوهيتي، ولم يسمعوا الرجل الحكيم الذي سأل وقال: "نحن لا نبرر الخاطئ بأي شكل من الأشكال".

\par 9 ألا تعلمون أنه مكتوب: ﴿وَمَنْ تَابَ فَلاَ يَرَى عَذَابًا﴾؟

\par 10 ولم يسمعوا الرسل ولا كلمتي في الأناجيل، وهم يسببون الحزن لملائكتي، ومن المؤكد أنهم في اجتماعاتي وفي طقوسي لا يصغون إلى ملاكي، ولا يقفون في كنائسي المقدسة؛ يقفون ولا يسجدون خوفًا ورعدة، بل ينطقون بكلمات طويلة لا أقبلها أنا ولا ملائكتي

\chapter{15}

\par 1 قال سدرخ لله: "يا رب، أنت وحدك بلا خطيئة ورحيم جدًا، تُظهر الشفقة والنعمة للخطاة، لكن لاهوتك قالت: لم آتِ لأدعو الأبرار بل الخطاة إلى التوبة."

\par 2 فقال الرب لسدرخ: أما تعلم يا سدرخ أن اللص نجا في لحظة واحدة بعد أن ندم؟

\par 3 ألا تعلمون أن رسولي ومبشري قد خلص في لحظة؟ [... أما الخطاة فلا يخلصون] لأن قلوبهم كالحجر المتحلل، وهم الذين يسيرون في طرق غير صالحة ويهلكون مع المسيح الدجال.

\par 4 قال سدرخ: "يا سيدي، لقد قلتَ أيضًا: 'دخل روحي الإلهي إلى الأمم الذين ليس لديهم ناموس، ومع ذلك فهم يفعلون ما هو منصوص عليه في الناموس'."

\par 5 ولكن كما أن اللص والرسول والمبشر وسائر الذين عثروا في ملكوتك يا رب، كذلك اغفر لمن أخطأوا إليك في الأيام الأخيرة يا رب، لأن الحياة مليئة بالتعب والقسوة

\chapter{16}

\par 1 قال الرب لسدرخ: "لقد خلقت الإنسان على ثلاث مراحل؛ عندما يكون شابًا، أتغاضى عن أخطائه بسبب شبابه؛ ومرة ​​أخرى، عندما يكون رجلاً، أراقب عقله؛ ومرة ​​أخرى، عندما يكبر، أحفظه حتى يتوب."

\par 2 قال سدرخ: "يا رب، أنت تعلم وتعرف كل هذا، ومع ذلك ترحم الخطاة."

\par 3 قال له الرب: «يا حبيبي سدرخ، أعدك أن أرحمك ولو لأقل من أربعين يومًا، بل لعشرين يومًا، ومن يذكر اسمك لن يرى مكان العقاب، بل سيكون مع الأبرار في مكان انتعاش وراحة، ولن تُحسب خطيئة من ينسخ هذه العظة الرائعة إلى الأبد».

\par 4 فقال سدرخ: «يا رب، من يُصلي لعبدك، فأنقذه من كل شر». فقال سدرخ، خادم الله: «الآن يا سيدي، خذ نفسي».

\par 5 وأخذه الله ووضعه في الفردوس مع جميع القديسين. له المجد والقدرة إلى دهر الدهور، آمين.




\en{المستند}