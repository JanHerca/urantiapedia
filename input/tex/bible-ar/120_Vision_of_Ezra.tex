\begin{document}

\title{رؤية عزرا}

\chapter{1}

\par 1 فصلى عزرا إلى الرب قائلاً: أعطني الشجاعة يا رب، لكي لا أخاف عندما أرى أحكام الخطاة.

\par 2 ومُنح سبعة ملائكة من الجحيم حملوه إلى ما وراء الدرجة السبعين في المناطق الجهنمية

\par 3 ورأى بوابات نارية، وعند هذه البوابات رأى أسدين مستلقين هناك، ومن فمهما ومنخريهما وعينيه تخرج ألسنة لهب شديدة

\par 4 كان أقوى الرجال يدخلون ويمرون عبر النار، فلم تمسهم

\par 5 فقال عزرا: "من هم الذين يتقدمون بسلام؟"

\par 6 فقال له الملائكة: هم العادلون الذين ارتفع ذكرهم إلى السماء،

\par 7 الذي تصدّق بسخاء، وكسى العريان، ورغب في الخير

\par 8 وكان آخرون يدخلون ليمروا من الأبواب، وكانت الكلاب تنهشهم والنار تأكلهم

\par 9 فقال عزرا: "من هم؟" فقال الملائكة:

\par 10 "أنكروا الرب وأخطأوا مع النساء في يوم الرب."

\par 11 فقال عزرا: «يا رب، ارحم الخطاة!»

\par 12 وقادوه إلى ما وراء الصف الخمسين، فرأى في ذلك المكان رجالاً واقفين في سيول.

\par 13 كان البعض يرشقون وجوههم بالنار، بينما كان آخرون يجلدونهم بسياط نارية

\par 14 فصرخت الأرض بصوت عالٍ قائلة: «اجلدوهم ولا ترحموهم، لأنهم عملوا بي كفرًا».

\par 15 فقال عزرا: "من هم الذين يعانون مثل هذه العذابات كل يوم؟"

\par 16 قالت الملائكة: كانوا يسكنون مع النساء المتزوجات.

\par 17 النساء المتزوجات هنّ اللواتي زَيَّنْنَ أنفسهنّ ليس لأزواجهنّ، بل لإرضاء الآخرين، راغبات في شهوة شريرة

\par 18 قال عزرا: "يا رب، ارحم الخطاة!"

\par 19 ثم أتوا به إلى الجنوب فرأى ناراً وفقراء ونساء معلقين والملائكة تجلدهم بهراوات من نار.

\par 20 فقال عزرا: «يا رب، ارحم الخطاة! من هم؟»

\par 21 فقالت الملائكة إنهم سكنوا عند أمهاتهم رغبوا في رغباة رديئة.

\par 22 فقال عزرا: «يا رب، ارحم الخطاة!»

\par 23 وقادوه إلى أسفل في المناطق الجهنمية، فرأى مرجلًا فيه كبريت وقار، وكان يغلي كأمواج البحر

\par 24 وكان الأبرار يدخلون، وفي وسطها كانوا يسيرون على الأمواج المشتعلة، يسبحون اسم الرب كثيرًا، تمامًا كما يسيرون على الندى أو الماء البارد

\par 25 فقال عزرا: "من هم؟" فقال الملائكة:

\par 26 «هم الذين كانوا يقدمون اعترافًا أفضل يوميًا أمام الله والكهنة القديسين، ويأتون بالصدقات بسخاء (و) يقاومون الخطايا.»

\par 27 وجاء الخطاة راغبين في العبور، فجاءت ملائكة الجحيم وغمرتهم في النهر الناري

\par 28 فصرخوا من النار قائلين: «يا رب، ارحمنا!» لكنه لم يرحم

\par 29 سُمع صوت، لكن لم يُرَ جسد بسبب النار والألم

\par 30 فقال عزرا: من هم؟

\par 31 قال الملائكة: «لقد أذلوا أنفسهم بالشهوة كل أيامهم؛ لم يقبلوا غرباء؛ لم يعطوا صدقات؛

\par 32 أخذوا لأنفسهم أشياء الآخرين ظلماً؛ كانت لديهم رغبة شريرة؛ لذلك، هم في عذاب

\par 33 فقال عزرا: «يا رب، ارحم الخطاة!»

\par 34 وسار كما كان من قبل فرأى في مكان مظلم دودة خالدة لم يستطع أن يحسب حجمها.

\par 35 وأمام فمه وقف العديد من الخطاة، وعندما أخذ نفسًا، دخلوا إلى فمه مثل الذباب؛ ثم عندما زفر، خرجوا جميعًا بلون مختلف

\par 36 فقال عزرا: "من هم؟" فقالوا: "كانوا ممتلئين بكل شر، وساروا بلا اعتراف ولا توبة."

\par 37 فرأى إنسانًا جالسًا على عرش من نار، ومشيروه واقفون حوله في النار، وكانوا يخدمونه من النار ومن كل جانب

\par 38 فقال عزرا: "من هذا؟" فقال الملائكة: "ذلك الرجل، الذي اسمه هيرودس، كان ملكًا زمانًا طويلًا، وهو الذي قتل الأطفال الذكور في بيت لحم اليهودية من أجل الرب."

\par 39 فقال عزرا: «يا رب، اقضِ عدلاً!»

\par 40 وكان يمشي فرأى رجالاً مقيدين وملائكة الجحيم يثقبون عيونهم بالشوك.

\par 41 فقال عزرا: "من هم؟" فقال الملائكة: "لقد أظهروا للتائهين دروبًا غريبة."

\par 42 قال عزرا: "يا رب، ارحم الخطاة!"

\par 43 ورأى عذارىً مُقيداتٍ بأغلالٍ وزنها خمسمائة رطل، كأنهن على وشك الموت، قادماتٍ إلى الغرب. فقال عزرا: "من هنّ؟"

\par 44 وقالت الملائكة: «انتهكوا بكارتهم قبل الزواج».

\par 45 وكان هناك جمع من الشيوخ مطروحين، وفوقهم يُسكب الحديد المصهور والرصاص. فقال: "من هم؟"

\par 46 فقال الملائكة: «هؤلاء هم علماء الناموس الذين خلطوا بين المعمودية وناموس الرب، لأنهم كانوا يعلمون بالكلام، لكنهم لم يحثوا على العمل. وفي هذا يُدانون».

\par 47 فقال عزرا: "يا رب، ارحم الخطاة!"

\par 48 ورأى رؤى أتونًا عند غروب الشمس يشتعل بنار عظيمة، أُرسل إليه كثيرون من الملوك والأمراء من هذا العالم.

\par 49 وكان آلاف الفقراء يتهمونهم ويقولون: "لقد جرحونا بقوتهم وجرُّوا الرجال الأحرار إلى العبودية".

\par 50 ورأى تنورًا آخر يحترق بالزفت والكبريت، يُلقى فيه أبناء يتصرفون ببشاعة على أيدي آبائهم ويتسببون في الأذى بأفواههم

\par 51 ورأى في مكانٍ مُظلمٍ جدًا حريقًا آخر، أُلقيت فيه نساءٌ كثيرات. فقال: "من هنّ؟"

\par 52 فقالت الملائكة: «كان لهم أبناء في الزنا فقتلوهم».

\par 53 وكان أولئك الأطفال أنفسهم يشتكون منهم قائلين: يا رب، إن هؤلاء النساء أخذن النفوس التي أعطيتها لنا.

\par 54 فقال: من هم؟ فقالت الملائكة: قتلوا أبناءهم.

\par 55 فقال عزرا: «يا رب، ارحم الخطاة!»

\par 56 ثم جاء ميخائيل وجبرائيل وقالا له: تعال إلى السماء.

\par 57 فقال عزرا: «حيٌّ هو ربي، إنه لا يجوز لي أن آتي حتى أرى دينونة كل الخطاة».

\par 58 وقادوه إلى أسفل إلى المناطق الداخلية بعد المستوى الرابع عشر. فرأى أسودًا وكلابًا صغيرة مستلقية حول ألسنة اللهب. ومر الأبرار من خلالها وعبروا إلى الفردوس

\par 59 ورأى آلافًا كثيرة من الأبرار، وكانت مساكنهم من أروع ما يكون على الإطلاق

\par 60 ولما رأى ذلك، رُفع إلى السماء، وجاء إلى جمع من الملائكة، فقالوا له: «صلِّ إلى الرب من أجل الخطاة». فأنزلوه أمام الرب

\par 61 وقال: «يا رب، ارحم الخطاة!» فقال الرب: «عزرا، فليأخذوا حسب أعمالهم».

\par 62 فقال عزرا: «يا رب، لقد أظهرت رأفةً بالبهائم التي تأكل العشب ولم ترد لك المديح أكثر مما أظهرته لنا؛ فهي تموت وليس عليها خطيئة؛ ومع ذلك تعذبنا نحن الأحياء والأموات».

\par 63 وقال الرب: «على صورتي خلقت الإنسان، وأوصيته ألا يخطئ، فأخطأ، لذلك هو في العذاب

\par 64 والمختارون هم الذين يذهبون إلى الراحة الأبدية بفضل الاعتراف والتوبة والسخاء في الصدقة

\par 65 فقال عزرا: «يا رب، ماذا يفعل الأبرار حتى لا يدخلوا في الدينونة؟»

\par 66 فقال له الرب: «كما أن العبد الذي أحسن إلى سيده ينال الحرية، كذلك البار في ملكوت السماوات.» آمين

\end{document}