\begin{document}

\title{مزامير سليمان}

\chapter{1}

\par \textit{"تَفَخَّوْا فِي غِنَايَتِهِمْ..."}

\par 1 صرختُ إلى الرب في ضيقي،
\par     إلى الله عندما هاجمني الخطاة
\par 2 فجأةً، سُمع أمامي صوتُ إنذارِ الحرب.
\par     قلتُ: سيسمعُ لي لأني ممتلئٌ برًا
\par 3 ظننت في قلبي أني ممتلئ برًا،
\par    لأني كنت ميسور الحال وقد رزقت بأولاد
\par 4 امتدت ثروتهم إلى كل الأرض،
\par     ومجدهم إلى أقاصي الأرض
\par 5 لقد رُفعوا إلى النجوم؛
\par     قالوا إنهم لن يُعجبوا أبدًا
\par 6 لكنهم تكبروا في غناهم،
\par     وكانوا بلا فهم،
\par 7 كانت خطاياهم في الخفاء،
\par     وحتى أنا لم يكن لدي علم بها
\par 8 تجاوزت ذنوبهم ذنوب الأمم التي سبقتهم.
\par     لقد نجّسوا مقدسات الرب تمامًا

\chapter{2}

\par \textit{تدنيس القدس؛ أسر، قتل، واغتصاب. مزمور يأس تام.}

\par 1 عندما أصبح الخاطئ متكبرًا، هدم الجدران المحصنة بكبش،
\par    ولم تمنعه.
\par 2 صعدت الأمم الغريبة إلى مذبحك،
\par    داسوها بفخر بأحذيتهم؛
\par 3 لأن بني أورشليم نجّسوا أقداس الرب،
\par    قد دنّس بالآثام قرابين الله.
\par 4 لذلك قال: اطرحهم عني.
\par     . . . . .
\par 5 لقد تم وضعه في شيء أمام الله،
\par    لقد كان الأمر مهانًا تمامًا؛
\par 6 وكان الأبناء والبنات في أسر شديد،
\par    كانت أعناقهم مختومة، وكانت موسومة بين الأمم.
\par 7 بحسب خطاياهم فعل بهم.
\par     لأنه تركهم في أيدي الغالبين.
\par 8 لقد صرف وجهه عن الشفقة عليهم.
\par    الصغار والكبار وأطفالهم معًا؛
\par 9 لأنهم أساءوا جميعًا بعدم استماعهم.
\par     وغضبت السماوات،
\par 10 وكرهتهم الأرض
\par     لأنه لم يفعل أحد عليها ما فعلوه،
\par 11 وأدركت الأرض كل
\par     أحكامك العادلة يا الله
\par 12 جعلوا أبناء أورشليم يُسخر بهم عوضًا عن الزواني فيها؛
\par    دخل كل مسافر في ضوء النهار الكامل.
\par 13 وكانوا يسخرون من معاصيهم كما كانوا يفعلون هم أنفسهم.
\par    وفي وضح النهار كشفوا آثامهم.
\par 14 وتنجست بنات أورشليم حسب حكمك،
\par    لأنهما نجسا بمعاشرة غير طبيعية.
\par 15 أشعر بألم في أحشائي وأحشائي بسبب هذه الأشياء.
\par   
\par 16 ولكني أبرركَ يا الله باستقامة قلب،
\par    لأنَّ عدلَكَ قد ظهرَ في أحكامِكَ يا الله
\par 17 لأنكَ كافأت الخطاةَ حسب أعمالهم،
\par    نعم، حسب خطاياهم التي كانت شريرةً جدًا
\par 18 لقد كشفت خطاياهم، لكي يكون حكمك جليًا؛
\par    لقد محوت ذكراهم من الأرض.
\par 19 الله قاضٍ عادل،
\par     ولا يُحابي الوجوه
\par 20 لأن الأمم عيَّروا أورشليم وداسوها.
\par     انحدر جمالها عن عرش المجد
\par 21 كانت تتنطق بمسح بدلًا من ثياب الزينة،
\par     وكان حبل حول رأسها بدلًا من التاج
\par 22 خلعت تاجها المجيد الذي وضعه الله عليها،
\par     وألقي جمالها على الأرض عارًا
\par 23 فرأيت وتضرعت إلى الرب وقلت:
\par     كفى يا رب ثقلًا على إسرائيل، إذ جلبت الأمم عليهم
\par 24 لأنهم لعبوا بلا هوادة في غضب وغضب شديد.
\par     وسيهلكون هلاكًا تامًا، ما لم تُوبِّخهم أنت يا رب بغضبك
\par 25 لأنهم لم يفعلوا ذلك بدافع الغيرة، بل بدافع شهوة النفس،
\par     يصبون غضبهم علينا بهدف السلب والنهب
\par 26 لا تؤخر يا الله مكافأتهم على رؤوسهم،
\par     لتحويل كبرياء التنين إلى عار
\par 27 ولم أنتظر طويلاً حتى أراني الله المتغطرس
\par     مقتولاً على جبال مصر،
\par    مُقَدَّرٌ بِأَقْلَى قَدْرِ مَنْ هُمْ أَقَلُّ قَدْرَةً، فِي الْبَرِّ وَالْبَحْرِ؛
\par 28 جسده أيضًا، محمولًا هنا وهناك على الأمواج بوقاحة شديدة،
\par     ولم يكن أحد ليدفنه لأنه رفضه بالعار.
\par 29 لم يفكر في أنه إنسان،
\par     ولم يفكر في النهاية الأخيرة؛
\par 30 قال: سأكون سيد البر والبحر.
\par     ولم يعلم أن الله هو العظيم،
\par    قوي في قوته العظيمة.
\par 31 فهو ملك على السماوات،
\par    ويحكم الملوك والممالك.
\par 32 هو الذي يرفعني في المجد،
\par     ويهبط المتكبرين إلى الهلاك الأبدي في العار،
\par    لأنهم لم يعرفوه.
\par 33 والآن هوذا يا أمراء الأرض دينونة الرب،
\par     لأنه ملك عظيم وبار، يدين كل ما تحت السماء
\par 34 باركوا الله يا خائفي الرب بالحكمة،
\par     لأن رحمة الرب تكون على خائفيه في الدينونة؛
\par 35 لكي يميز بين البار والخاطئ،
\par     ويجازي الخطاة إلى الأبد حسب أعمالهم؛
\par 36 وارحم البار، مُنجِّيه من بلاء الخاطئ،
\par    وجزاء المذنب بما فعل بالصالحين.
\par 37 لأن الرب صالح للذين يدعونه بصبر،
\par     صانعًا حسب رحمته لأتقيائه،
\par    يثبتهم في كل وقت أمامه في قوة.
\par 38 مبارك الرب إلى الأبد أمام عبيده.

\chapter{3}

\par \textit{البر مقابل الخطيئة.}

\par 1 لماذا تنامين يا نفسي؟
\par     ولا تباركين الرب؟
\par 2 غنوا ترنيمة جديدة،
\par     لله الذي يستحق الحمد
\par 3 غنوا واستيقظوا من يقظته،
\par     لأن مزمورًا صالحًا يُرنم لله من قلب فرح
\par   
\par 4 يذكر الصديقون الرب في كل وقت،
\par     بالشكر وإعلان عدل أحكام الرب
\par 5 لا يحتقر البار تأديب الرب.
\par     مشيئته دائمًا أمام الرب
\par 6 البار يعثر ويُبرر الرب.
\par     يسقط وينتظر ما سيفعله الله به؛
\par 7 يبحث عن من أين يأتي خلاصه.
\par     ثبات الصديق من الله منقذه
\par 8 لا تبيت في بيت البار خطيئة على خطيئة.
\par     البار يفتش بيته دائمًا،
\par 9لإزالة كل إثم فعله في السهو.
\par    يكفر عن خطايا الجهل بالصوم وتأنيب النفس،
\par 10 والرب يحسب كل رجل تقي وبيته بريئين.
\par     يتعثر الخاطئ ويلعن حياته
\par 11 يوم وُلد، وكد أمه
\par    يزيد خطايا إلى خطايا وهو حيّ؛
\par 12 يسقط - إن سقوطه مؤلم حقًا - ولا يقوم بعد ذلك.
\par     إن هلاك الخاطئ إلى الأبد،
\par 13 ولا يُذكر إذا افتقد الصديق.
\par     هذا نصيب الخطاة إلى الأبد
\par   
\par 14 أما الذين يخافون الرب فسيقومون إلى الحياة الأبدية،
\par     وستكون حياتهم في نور الرب، ولن تنتهي بعد

\chapter{4}

\par \textit{حوار سليمان مع الذين يرضون الناس.}

\par 1 لماذا تجلس أيها الإنسان الفاسد في مجلس الأتقياء؟
\par    إذ قد ابتعد قلبك عن الرب،
\par    استفزاز إله إسرائيل بالمعصية؟
\par 2 مُسرف في الكلام، مُسرف في المظهر الخارجي الذي يفوق كل الناس،
\par هو من يكون شديد الكلام في إدانة الخطاة في الدينونة
\par 3 ويده عليه أولاً كما لو كان يتصرف بغيرة،
\par     ومع ذلك فهو نفسه مذنب بارتكاب خطايا كثيرة وفجور
\par 4 عيناه على كل امرأة بلا تمييز؛
\par     لسانه يكذب عندما يعقد قَسَمًا
\par 5 في الليل وفي الخفاء يخطئ كأنه لا يُرى،
\par     بعينيه يُحدث كل امرأة عن مواثيق الشر
\par 6 إنه سريع الدخول إلى كل بيت بمرح كما لو كان بريئًا
\par   
\par 7 ليُزِل اللهُ من يعيشون في نفاقٍ مع المتقين،
\par    حتى حياة مثل هذا الشخص مع فساد جسده وعوزه.
\par 8 ليكشف الله أعمال الذين يرضون الناس،
\par     أعمال مثل هذا بالضحك والاستهزاء؛
\par 9 لكي يحسب الأتقياء دينونة إلههم عادلة،
\par     عندما يُزاح الخطاة من أمام الأبرار،
\par     حتى من يسعى لإرضاء الناس وينطق بالقانون بخبث.
\par 10 وأعينهم مثبتة على بيت أي رجل لا يزال آمنًا،
\par    لكي يتمكنوا، مثل الحية، من تدمير حكمة ... بكلمات المخالفين،
\par 11 كلماته خادعة ليُتمم شهوته الشريرة.
\par     لا يكف عن تشتيت العائلات كما لو كانوا أيتامًا،
\par     نعم، يهدم بيتًا بسبب رغباته غير القانونية.
\par 12 فهو يخدع بالكلام قائلاً: ليس من يرى ولا من يحكم.
\par 13 فيملأ بيتًا واحدًا إثمًا،
\par    ثم تركز عيناه على المنزل المجاور،
\par    لتدميرها بكلمات تعطي جناحًا للرغبة.
\par 14 ومع كل هذه، فإن نفسه كالهاوية، لم تشبع.
\par   
\par 15 يا رب، ليُهان نصيبه أمامك.
\par    ليخرج يئن، وليرجع إلى بيته ملعونًا
\par 16 ليقضِ حياته في ضيقٍ وعوزٍ وفقرٍ يا رب.
\par    ليكن نومه مُحاطًا بالآلام، ويقظته مُحاطة بالحيرة
\par 17 ليُسحب النوم من جفنيه ليلاً؛
\par     فليفشل فشلاً ذريعاً في كل عمل من أعمال يديه
\par 18 فليرجع إلى بيته خالي الوفاض،
\par     وليكن بيته خاليًا من كل ما يشبع به شهوته
\par 19 فليقضِ شيخوخته في عزلة بلا أطفال حتى يزول بالموت
\par   
\par 20 لتتمزق أجساد من يرضون الناس بالوحوش،
\par     ولتكن عظام الأشرار مهينة أمام أعين الشمس
\par 21 لتنقر الغربان عيون المنافقين.
\par 22 لأنهم دمروا بيوت كثير من الناس في عار،
\par    وشتتهم في شهوتهم؛
\par 23 ولم يذكروا الله،
\par     ولم يخافوا الله في كل هذه الأمور؛
\par     ولكنهم أثاروا غضب الله وأغاظوه.
\par 24 لينزعهم من على الأرض،
\par    لأنهم بالخداع خدعوا أرواح الأطهار
\par   
\par 25 طوبى للذين يخافون الرب في كمالهم.
\par     يُنجيهم الرب من الناس الماكرين والخطاة،
\par     وأنقذنا من كل معثرة الأثمة.
\par 26 ليهلك الله الذين يرتكبون كل إثم بوقاحة،
\par     لأن الرب إلهنا قاضٍ عظيم وقدير بالعدل
\par   
\par 27 لتكن رحمتك يا رب على كل الذين يحبونك.

\chapter{5}

عبارة عن فلسفة عدم قابلية المادة للفناء. أحد مبادئ الفيزياء الحديثة.

\par 1 يا رب الإله، سأسبح اسمك بفرح،
\par     في وسط عارفي أحكام عدلك
\par 2 لأنك أنت الصالح والرحيم، ملجأ الفقراء؛
\par     عندما أصرخ إليك، فلا تتجاهلني في صمت
\par 3 لأنه لا أحد يأخذ غنيمة من جبار.
\par    فمن يستطيع إذن أن يأخذ شيئًا مما صنعته، إلا إذا أعطيته أنت؟
\par 4 لأن الإنسان ونصيبه أمامك في الميزان؛
\par     لا يمكنه أن يزيد على ما شرعته، بحيث يوسعه
\par   
\par 5 يا الله، عندما نكون في محنة نلجأ إليك طلباً للمساعدة،
\par    ولا تردّ طلبنا لأنّك أنت إلهنا.
\par 6 لا تجعل يدك ثقيلة علينا،
\par     لئلا نخطئ بالضرورة
\par 7 حتى وإن لم تردنا، فلن نبتعد.
\par    ولكن إليك نأتي
\par 8 لأني إن جعت، فإليك أصرخ يا الله.
\par      وأنت تعطيني
\par   
\par 9 أنت تُغذي الطيور والأسماك،
\par     لأنك تُعطي المطر للسهوب حتى ينبت العشب الأخضر،
\par    لذا لإعداد العلف في السهوب لكل كائن حي؛
\par 10 وإذا جاعوا، فإليك يرفعون وجوههم.
\par 11 الملوك والحكام والشعوب أنت تغذيهم يا الله.
\par     ومن هو معين الفقير والمسكين إلا أنت يا رب؟
\par 12 وأنتَ تُصغي - فمن هو الصالح والوديع سواك؟ -
\par     مُسَرّاً نفوس المتواضعين بِفَتحِ يَدِكَ بِالرَّحْمَةِ
\par   
\par 13 يُمنح الإنسان صلاحه على مضض و...؛
\par    وإذا كرره دون تذمر، فذلك رائع
\par 14 لكن عطيتك عظيمة في الخير والغنى،
\par     ومن وضع رجاءه عليك فلن يفتقر إلى المواهب
\par 15 على كل الأرض رحمتك يا رب بالخير.
\par   
\par 16 طوبى لمن يذكره الله حين يمنحه الكفاية التي يستحقها.
\par    إذا زاد الإنسان عن حده فإنه يخطئ.
\par 17 تكفي الوسيلة المعتدلة مع البر،
\par    وبهذا تصير بركة الرب وفرة مع البر
\par 18 الذين يخافون الرب يفرحون بالعطايا الصالحة،
\par     وصلاحك على إسرائيل في ملكوتك
\par   
\par 19 مبارك مجد الرب، لأنه ملكنا.



\chapter{6}

\par \textit{أغنية الأمل والشجاعة والسلام.}

\par 1 طوبى للرجل الذي قلبه مثبّت على الدعاء باسم الرب
\par    عندما يذكر اسم الرب يخلص.
\par 2 طرقه مصنوعة من قبل الرب،
\par     وأعمال يديه محفوظة من قبل الرب إلهه
\par 3 لن تضطرب نفسه مما يراه في أحلامه السيئة.
\par     عندما يمر عبر الأنهار وأمواج البحار، لن ينزعج
\par 4 يستيقظ من نومه ويبارك اسم الرب.
\par     عندما يطمئن قلبه، يرنم لاسم إلهه،
\par    ويصلي إلى الرب لأجل كل بيته.
\par 5 "ويسمع الرب صلاة كل من يتقي الله،
\par     وكل طلب النفس التي ترجوه يفعله الرب.
\par   
\par 6 تبارك الرب، الذي يرحم الذين يحبونه بإخلاص

\chapter{7}

\par \textit{العقيدة القديمة الجميلة - "أنت درعنا!"}

\par 1 لا تبعد مسكنك عنا يا الله.
\par     لئلا يهاجمنا الذين يبغضوننا بلا سبب.
\par 2 لأنك رفضتهم يا الله.
\par     لا تدع أقدامهم تدوس ميراثك المقدس
\par 3 أدبنا أنت بمسرتك
\par    ولا تسلمنا إلى الأمم؛
\par 4 لأنه إذا أرسلت وباءً،
\par     فأنت بنفسك تُوصي به علينا؛
\par 5 فإنك أنت الرحيم،
\par     ولن تغضب حتى تلتهمنا
\par   
\par 6 ما دام اسمك قائماً بيننا، فإننا سنجد الرحمة.
\par    ولن تقوى علينا الأمم.
\par 7 لأنك أنت درعنا،
\par     وعندما ندعوك، فأنت تستجيب لنا؛
\par 8 لأنك سترحم نسل إسرائيل إلى الأبد
\par     ولن ترفضهم:
\par 9 لكننا سنكون تحت نيرك إلى الأبد،
\par     وتحت قضيب تأديبك
\par 10 تُثَبِّتُنَا فِي الْوَقْتِ الَّذِي تُعِينُنَا فيهِ،
\par     مُظْهِرًا رَحْمَةً لِبَيتِ يَعْقُوبَ فِي الْيَوْمِ الَّذِي وَعَدْتَ بِإِعَانَتِهِمْ

\chapter{8}

\par \textit{بعض التشبيهات اللافتة للحرب الزاحفة على القدس. لمحة عامة عن الخطايا التي جلبت كل هذه المشاكل.}

\par 1 "سمعت أذني ضائقة وصوت حرب،
\par    صوت البوق معلناً المذبحة والكارثة،
\par 2 صوت كثير من الناس كأنه صوت ريح شديدة شديدة،
\par     كعاصفة لها نار عظيمة تجتاح النقب.
\par 3 فقلت في قلبي: إن الله يحكم علينا.
\par    صوت أسمعه يتحرك نحو القدس، المدينة المقدسة
\par 4 لقد انكسرت أحشائي عندما سمعت ما حدث، وارتجفت ركبتي؛
\par    لقد خاف قلبي، وارتعشت عظامي مثل الكتان.
\par 5 قلت: يقيمون طريقهم على البر.
\par   
\par 6 فكرت في أحكام الله منذ خلق السماء والأرض؛
\par    لقد اعتبرت الله بارًا في أحكامه التي كانت منذ القديم.
\par 7 لقد كشف الله خطاياهم في ضوء النهار الكامل؛
\par     وعرفت كل الأرض أحكام الله العادلة.
\par 8 وفي أماكن سرية تحت الأرض ارتكبوا آثامهم لإغضاب الله.
\par 9 لقد أحدثوا الفوضى، الابن مع أمه والأب مع ابنته؛
\par    فزنوا كل رجل مع امرأة جاره.
\par 10 عقدوا عهودًا مع بعضهم البعض بقسم على هذه الأمور؛
\par    نهبوا قدس الأقداس كأنه لا منتقم.
\par 11 داسوا مذبح الرب، قادمين مباشرة من كل أنواع النجاسة؛
\par    و بدم الحيض نجّسوا الذبائح كأنها لحم عادي.
\par 12 لم يتركوا خطيئةً دون أن يرتكبوها، ولم يتجاوزوا فيها الأمم.
\par   
\par 13 لذلك مزج الله لهم روح التيه.
\par     وأعطاهم كأسًا من خمر غير مخفف ليسكروا
\par 14 أتى بالذي من أقاصي الأرض، الضارب بشدة
\par    فأمر بالحرب على أورشليم وعلى أرضها.
\par 15 ذهب أمراء الأرض للقائه بفرح. قالوا له:
\par     فلتكن طريقك مباركة! هلموا وادخلوا بسلام.
\par 16 سهّلوا الطرق الوعرة قبل دخوله
\par     فتحوا أبواب أورشليم، وتوجوا أسوارها
\par   
\par 17 كما يدخل الأب بيت أبنائه، كذلك يدخل أورشليم بسلام؛
\par     وأقام قدميه هناك آمنًا عظيمًا.
\par 18 استولى على حصونها وسور أورشليم؛
\par     لأن الله نفسه قاده بأمان، بينما هم تائهون
\par 19 أهلك رؤساءهم وكل حكيم في المشورة.
\par     سفك دماء سكان أورشليم كماء نجاسة
\par 20 وساق أبناءهم وبناتهم الذين ولدوهم في نجاسة
\par   
\par 21 ففعلوا حسب نجاستهم، كما فعل آباؤهم:
\par     ونجّسوا أورشليم والأشياء المقدسة لاسم الله.
\par 22 لكن الله أظهر نفسه بارًا في أحكامه على أمم الأرض
\par    وعباد الله الصالحون كالحملان البريئة في وسطهم.
\par 23 يستحق الحمد هو الرب الذي يدين كل الأرض ببره
\par   
\par 24 انظر الآن يا الله، لقد أريتنا حكمك في عدلك
\par    لقد رأت عيوننا أحكامك يا الله.
\par 25 بررنا اسمك المكرّم إلى الأبد.
\par    لأنك أنت إله البر، تدين إسرائيل بالتأديب
\par   
\par 26 يا الله، انقلب برحمتك علينا، وارحمنا.
\par    اجمع مُشتتي إسرائيل بالرحمة والخير
\par 27 لأن أمانتك معنا،
\par 28 وإن كنا قد شددنا أعناقنا، فأنت مؤدبنا.
\par     لا تتغافل عنا يا إلهنا لئلا تبتلعنا الأمم كأن ليس منقذ.
\par   
\par 29 لكنك أنت إلهنا من البدء،
\par     وعليك رجاؤنا يا رب
\par 30 ولن نحيد عنك،
\par     لأن أحكامك صالحة علينا
\par 31 لتكن مسرتنا ومسرات أولادنا إلى الأبد.
\par    يا رب مخلصنا، لن نتزعزع بعد الآن
\par 32 يستحق الرب أن يُمدح على أحكامه بفم أتقيائه؛
\par     ويكون إسرائيل الرب مباركًا إلى الأبد.

\chapter{9}

\par \textit{سبي أسباط إسرائيل. إشارة إلى العهد الذي قطعه الله مع آدم. (انظر الكتاب الأول لآدم وحواء، الفصل الثالث، الآية 7).}

\par 1 عندما أُسر بنو إسرائيل إلى أرض غريبة،
\par     عندما ابتعدوا عن الرب الذي فداهم،
\par    لقد طردوا من الميراث الذي أعطاهم الرب إياه.
\par 2 من بين كل الأمم كان مشتت إسرائيل حسب كلمة الله،
\par     لكي تتبرر يا الله في برك من أجل معاصينا.
\par     لأنك أنت قاضٍ عادل على جميع شعوب الأرض.
\par 3 لأنه لا يخفى عليك من علمك ظالم،
\par     وأعمال صالحيكَ أمامك يا رب؛
\par     أين يستطيع الإنسان أن يختبئ من معرفتك يا الله؟
\par 4 أعمالنا تخضع لاختيارنا وقدرتنا
\par     أن نفعل الصواب أو الخطأ في أعمال أيدينا؛
\par     وببرك تفتقد بني البشر.
\par 5 من يفعل البر يضع نفسه لنفسه عند الرب.
\par     ومن يفعل ظلما يفقد نفسه للهلاك
\par 6 لأن أحكام الرب تُعطى بالعدل لكل إنسان وبيته
\par 7 لمن أنت صالح يا الله إلا لمن يدعون الرب؟
\par    يطهر النفس من الخطايا حين تعترف، حين تقر؛
\par     لأن العار علينا وعلى وجوهنا بسبب كل هذه الأمور.
\par 8 ولمن يغفر الخطايا إلا للمذنبين؟
\par    تبارك الصديق ولا توبخه على خطاياه التي ارتكبها
\par     ورحمتك على الذين يخطئون إذا تابوا.
\par 9 والآن أنت إلهنا ونحن الشعب الذي أحببته.
\par     انظر وارحمنا يا إله إسرائيل لأننا لك.
\par     ولا تنزع رحمتك عنا فيطغوا علينا.
\par 10 لأنك اخترت نسل إبراهيم قبل كل الأمم،
\par    ووضعت اسمك علينا يا رب،
\par    ولن ترفضنا إلى الأبد.
\par 11 لقد قطعت عهدًا مع آبائنا بشأننا؛
\par     ونحن نتوكل عليك عندما تلجأ إليك نفوسنا
\par    لتكن رحمة الرب على بيت إسرائيل إلى الأبد.

\chapter{10}

\par \textit{ترنيمة مجيدة. إشارة إضافية إلى العهد الأبدي بين الله والإنسان.}

\par 1 طوبى للرجل الذي يذكره الرب بالتوبيخ،
\par     ويصرفه عن طريق الشر بضربات
\par     لكي يتطهر من الخطيئة فلا تكثر.
\par 2 من جهز ظهره للضربات يطهر،
\par     لأن الرب صالح للذين يحتملون التأديب
\par 3 لأنه يُقوِّم سبل الأبرار،
\par     ولا يُفسدهم بتأديبه
\par 4 ورحمة الرب على الذين يحبونه بالحق،
\par     ويذكر الرب عبيده بالرحمة
\par 5 لأن الشهادة هي في ناموس العهد الأبدي،
\par     شهادة الرب هي على طرق البشر عند افتقاده
\par 6 عادل ورحيم هو ربنا في أحكامه إلى الأبد،
\par     وسيُسبِّح إسرائيل اسم الرب بفرح
\par 7 ويشكر الأتقياء في مجمع الشعب.
\par     ويرحم الله المساكين في فرح إسرائيل
\par 8 لأن الله صالح ورحيم إلى الأبد،
\par     وتُمَجِّد جماعات إسرائيل اسم الرب
\par   
\par 9 خلاص الرب على بيت إسرائيل للفرح الأبدي!

\chapter{11}

\par \textit{تسمع القدس بوقًا وتقف على أطراف أصابعها لترى أطفالها عائدين من الشمال والشرق والغرب.}

\par 1 انفخوا في البوق في صهيون لاستدعاء القديسين،
\par     ليُسمَع في أورشليم صوتُ البشارة.
\par     لأن الله شفق على إسرائيل وافتقدهم.
\par 2 قفي على المرتفعات يا أورشليم وانظري إلى أولادك
\par    من الشرق والغرب، جمعهم الرب؛
\par 3 من الشمال يأتون بفرح إلههم،
\par     من الجزائر البعيدة جمعهم الله.
\par 4 جعل لهم جبالاً شاهقة سهلاً
\par     هربت التلال عند مدخلهم
\par 5 وفرت لهم الغابات مأوى أثناء مرورهم؛
\par     أنبت الله لهم كل شجرة طيبة الرائحة،
\par     لكي يمر إسرائيل في افتقاد مجد إلهه.
\par 6 البسي يا أورشليم ثياب مجدك
\par     جهزي ثوبك المقدس
\par     لأن الله تكلم بالخير عن إسرائيل إلى الأبد.
\par 7 فليفعل الرب ما تكلم به عن إسرائيل وأورشليم.
\par     فليُقِم الرب إسرائيل باسمه المجيد
\par   
\par 8 رحمة الرب على إسرائيل إلى الأبد.



\chapter{12}

\par \textit{نداء من أجل الهدوء والسكينة العائلية في المنزل.}

\par 1 يا رب، نجِّ نفسي من الإنسان الأثيم والشرير،
\par     من اللسان الآثم والشتم، الذي يتكلم بالكذب والغش
\par 2 كلمات لسان الشرير ملتوية بشكل متعدد،
\par     كما بين الناس نار تحرق جمالهم
\par 3 لذلك يُسرّ بملء البيوت بلسان كاذب،
\par     لقطع أشجار الفرح التي تُشعل النار في المخالفين،
\par    إقحام الأسر في الحرب من خلال الشفاه القبيحة.
\par   
\par 4 اللهم أبعد عن الأبرياء أفواه المخالفين بجلبهم إلى الفقر.
\par     ولتتبدد عظام النمامين بعيدا عن خائفي الرب.
\par    في نار ملتهبة يهلك اللسان الكاذب بعيدا عن الصالحين!
\par 5 ليحفظ الرب النفس الهادئة التي تبغض الأشرار.
\par    وليثبت الرب الرجل الذي يتبع السلام في بيته.
\par 6 خلاص الرب على إسرائيل عبده إلى الأبد.
\par     وليهلك الخطاة جميعا أمام الرب.
\par     ولكن ليرث أتقياء الرب مواعيد الرب.

\chapter{13}

\par \textit{لسليمان. مزمور. تعزية للصديق.}

\par 1 يمين الرب غطتني
\par    إن يمين الرب أشفقت علينا.
\par 2 "ذراع الرب أنقذتنا من السيف الذي عبر،
\par    من المجاعة وموت الخطاة.
\par 3 ركضت عليهم وحوش شريرة:
\par     بأسنانهم مزقوا لحمهم،
\par    وبأضراسهم سحقوا عظامهم.
\par 4 ولكن من كل هذه الأمور أنقذنا الرب.
\par 5 اضطرب البار بسبب أخطائه،
\par     لئلا يؤخذ مع الخطاة؛
\par 6 لأن سقوط الخاطئ رهيب.
\par     ولكن لا شيء من هذه الأمور يمس الصالحين
\par 7 لأنه ليس تأديب البار على الخطايا التي ترتكب عن جهل،
\par    وإسقاط الخطاة.
\par 8 يُؤدَّب البار سرًّا،
\par     لئلا يفرح الخاطئ بالبار
\par 9 لأنه يؤدب البار كابن محبوب.
\par     وتأديبه كعقاب البكر
\par 10 لأن الرب يحفظ أتقياءه،
\par     ويمحو خطاياهم بتأديبه
\par 11 لأن حياة الصديق تكون إلى الأبد.
\par     أما الخطاة فيؤخذون إلى الهلاك,,
\par     ولا يوجد ذكرهم بعد.
\par 12 وأما على الأتقياء رحمة الرب،
\par     وعلى الذين يخافونه رحمته



\chapter{14}

\par \textit{يحب الخطاة اليوم القصير الذي يقضونه في صحبة خطيئتهم. حكمة عميقة، معبر عنها بشكل جميل.}

\par 1 الرب أمين للذين يحبونه بالحق،
\par     للذين يصبرون على تأديبه،
\par 2 للسالكين في بر وصاياه،
\par     في الناموس الذي أوصانا به لكي نحيا
\par 3 سيحيا به أتباع الرب إلى الأبد.
\par     جنة الرب، أشجار الحياة، هم أتباعه
\par 4 غرسهم متأصل إلى الأبد؛
\par     لن يُقتلعوا كل أيام السماء:
\par 5 لأن نصيب الله وميراثه هو إسرائيل.
\par 6 ولكن ليس كذلك الخطاة والمتجاوزون،
\par    الذين يحبون اليوم القصير الذي يقضونه في صحبة خطيئتهم؛
\par 7 لذتهم في الفساد الزائل،
\par    ولا يذكرون الله.
\par 8 لأن طرق البشر معروفة أمامه في كل الأوقات،
\par     وهو يعلم خفايا القلوب قبل أن تكون.
\par 9 لذلك فإن ميراثهم هو الهاوية والظلمة والهلاك.
\par     ولا يوجدون في اليوم الذي يرحم فيه الصديق.
\par 10 وأما أتقياء الرب فيرثون الحياة بالفرح.

\chapter{15}

\par \textit{يعيد كاتب المزمور صياغة الفلسفة العظيمة للصواب والخطأ.}

\par 1 حينما كنت في ضيق دعوت اسم الرب،
\par    كنت أرجو معونة إله يعقوب فنجيت؛
\par     لأنك أنت رجاء الفقراء وملجأهم يا الله.
\par 2 لأنه من قوي يا الله إلا أن يشكرك بالحق؟
\par    وأي شيء يكون الإنسان قويًا إلا بشكر اسمك؟
\par 3 مزمور جديد بترنيمة في فرح القلب،
\par    ثمرة الشفتين مع آلة اللسان المضبوطة،
\par    باكورة الشفتين من قلب تقي وبار—
\par 4 من يقدم هذه الأشياء لن يزعزعه الشر أبدًا.
\par     لن يمسه لهيب النار والغضب على الظالمين،
\par 5 عندما يخرج من وجه الرب ضد الخطاة،
\par     لتدمير كل مال الخطاة،
\par 6 لأن علامة الله على الأبرار للخلاص.
\par   
\par 7 ستكون المجاعة والسيف والوباء بعيدة عن الصديقين،
\par     لأنهم سيهربون من الأتقياء كما يهرب الرجال في الحرب؛
\par 8 لكنهم سيطاردون الخطاة ويدركونهم،
\par     والذين يفعلون الإثم لن يفلتوا من دينونة الله
\par 9 كما يُهزمون من قِبل أعداء ذوي خبرة في الحرب،
\par     لأن علامة الهلاك على جباههم
\par 10 وميراث الخطاة هو الهلاك والظلمة،
\par     وستتبعهم آثامهم إلى الهاوية من أسفل
\par 11 لن يُوجد ميراثهم من أبنائهم،
\par     لأن الخطايا تُخرب بيوت الخطاة
\par 12 وسيهلك الخطاة إلى الأبد في يوم دينونة الرب،
\par    عندما يزور الله الأرض بحكمه.
\par 13 أما الذين يخافون الرب فيجدون فيه رحمة،
\par     ويحيون برحمة إلههم
\par 14 لكن الخطاة سيهلكون إلى الأبد.



\chapter{16}

\par \textit{يُعبّر كاتب المزمور مرة أخرى عن حقيقة عميقة - "لأنه إن لم تُعطِ قوة، فمن يستطيع أن يحتمل التأديب؟"}

\par 1 عندما نامت روحي بعيدًا عن الرب، كدتُ أنزلق إلى الحفرة،
\par    حين كنت بعيدًا عن الله، كادت روحي أن تُسكب حتى الموت،
\par 2 لقد كنت قريبًا من أبواب الهاوية مع الخاطئ،
\par    حين فارقت نفسي الرب إله إسرائيل—
\par 3 ولولا أن الرب ساعدني برحمته الأبدية.
\par   
\par 4 لقد وخزني كما يُوخز الحصان لأتمكن من خدمته،
\par     لقد أنقذني مخلصي ومعيني في كل الأوقات.
\par 5 أحمدك يا ​​الله لأنك أعنتني على خلاصي.
\par     ولم تحسبني مع الخطاة للهلاك.
\par 6 لا تنزع رحمتك عني يا الله،
\par     ولا تنزع ذكرك من قلبي حتى أموت
\par 7 احفظني يا الله من الخطايا الشريرة،
\par     ومن كل امرأة شريرة تُعثِّر الساذج
\par 8 ولا يغرني جمال المرأة الفاجرة،
\par     ولا من هو عرضة للخطيئة غير النافعة
\par   
\par 9 ثبّت أعمال يدي أمامك،
\par     واحفظ خطواتي في ذكرك
\par 10 احفظ لساني وشفتي بكلمات الحق.
\par     أبعد عني الغضب والغضب غير المبرر
\par 11 أبعد عني التذمر وعدم الصبر في الضيق
\par     عندما أخطأت، تؤدبني لأعود إليك
\par 12 ولكن بحسن النية والبهجة، ادعم روحي.
\par     عندما تقوي روحي، فإن ما يُعطى لي سيكون كافيًا لي
\par 13 لأنه إن لم تُعطِ القوة،
\par 14 من يستطيع أن يتحمل العقاب مع الفقر؟
\par 15 عندما يُوبَّخ الإنسان بسبب فساده،
\par      فإن اختبارك له في جسده وفي ضيق الفقر
\par 16 إن صمد البار في كل هذه التجارب، فسوف ينال رحمة من الرب

\chapter{17}

\par \textit{"أقاموا مملكة دنيوية... دمروا عرش داود!" سرد شعري عن التفكك التام لأمة عظيمة.}

\par 1 يا رب، أنت ملكنا إلى الأبد،
\par     لأنه فيك يا الله، تُمجَّد نفوسنا
\par 2 كم هي أيام حياة الإنسان على الأرض؟
\par     كما هي أيامه، كذلك يكون رجاؤه عليه
\par 3 لكننا نترجى الله مخلصنا.
\par 4 لأن قوة إلهنا إلى الأبد مع الرحمة،
\par     وتكون مملكة إلهنا إلى الأبد على الأمم في الدينونة.
\par   
\par 5 أنت يا رب اخترت داود ملكًا على إسرائيل،
\par     وأقسمت له على نسله أن مملكته لا تفشل أمامك أبدًا
\par 6 ولكن من أجل خطايانا قام علينا الخطاة.
\par    هاجمونا ودفعونا للخارج؛
\par     ما لم تعدهم به انتزعوه منا بالعنف.
\par 7 ولم يمجدوا اسمك الكريم بأي حال من الأحوال.
\par    لقد أقاموا مملكة دنيوية في مكان ما كان تفوقهم؛
\par    فحطموا عرش داود بغطرسة شديدة.
\par 8 "ولكن أنت يا الله طرحتهم وأزلت بذورهم من الأرض،
\par    فقام عليهم رجل غريب عن جنسنا.
\par 9 حسب خطاياهم كافأتهم يا الله
\par     فأصابهم حسب أعمالهم
\par 10 لم يُظهر الله لهم أي شفقة.
\par     لقد بحث عن نسلهم ولم يُطلق سراح أحد منهم
\par 11 الرب أمين في كل أحكامه
\par     الذي يصنعه على الأرض
\par   
\par 12 خرب الخارج عن القانون أرضنا حتى لم يسكنها أحد،
\par     أهلكوا الصغار والكبار وأطفالهم معًا
\par 13 في حمى غضبه، أرسلهم حتى إلى الغرب،
\par     وعرض حكام الأرض للسخرية بلا هوادة
\par 14 ولأنه كان أجنبيًا، تصرف العدو بفخر،
\par     وكان قلبه غريبًا عن إلهنا
\par 15 وكل ما صنع في أورشليم،
\par     وكذلك الأمم في المدن لآلهتهم
\par   
\par 16 وتفوق عليهم بنو العهد في الشر في وسط الشعوب المختلطة
\par     ولم يكن بينهم من يعمل رحمة وحقاً في وسط أورشليم.
\par 17 الذين أحبوا مجامع الأتقياء هربوا منها،
\par     كالعصافير التي تطير من أعشاشها
\par 18 تجولوا في الصحاري حتى تنجو أرواحهم من الأذى،
\par     وكان ثمينًا في عيون الذين يعيشون في الخارج كل من نجا منهم حيًا
\par    تشتتوا في كل أنحاء الأرض على يد رجال أشرار.
\par 19 لأن السماوات منعت المطر عن السقوط على الأرض،
\par     توقفت الينابيع التي كانت تنبع باستمرار من الأعماق، والتي كانت تنحدر من الجبال الشامخة
\par 20 لأنه لم يكن بينهم من يعمل برًا وعدلًا
\par    من كبيرهم إلى صغيرهم كلهم ​​كانوا خاطئين؛
\par     وكان الملك عاصيًا، والقاضي عاصيًا، والشعب خاطئًا.
\par 21 انظر يا رب، وأقم لهم ملكهم ابن داود،
\par     في الوقت الذي ترى فيه يا الله، ليملك على إسرائيل عبدك
\par 22 ويقوّيه بالقوة، لكي يحطم الحكام الظالمين،
\par     ولكي يُطهّر أورشليم من الأمم التي تدوسها حتى الهلاك
\par 23 بحكمةٍ وعدلٍ، سيطرد الخطاة من الميراث،
\par    يُهلك كبرياء الخاطئ كإناء خزاف.
\par 24 بقضيب من حديد يسحق كل ما لديهم،
\par    يُهلِكُ الأُمَمَ الْفَاسِرَةَ بِكَلِمَةِ فَمِهِ.
\par 25 عند توبيخه تهرب الأمم من أمامه،
\par     ويوبخ الخطاة على أفكار قلوبهم
\par   
\par 26 ويجمع شعبًا مقدسًا، ويقوده في البر،
\par    ويحكم على أسباط الشعب الذين قدسهم الرب إلهه.
\par 27 ولن يسمح للظلم أن يسكن في وسطهم بعد الآن،
\par     ولا يسكن معهم إنسان يعرف الشر،
\par     لأنه سيعرفهم أنهم جميعا أبناء إلههم.
\par 28 ويقسمهم حسب أسباطهم على الأرض،
\par     ولا ينزل معهم بعد غريب ولا نزيل
\par 29 فيحكم الشعوب والأمم بحكمة بره. سلاه
\par   
\par 30 ويكون له الأمم الوثنية لتخدمه تحت نيره.
\par     ويمجد الرب في مكان يراه كل الأرض
\par    ويُطهِّرُ أُورُشَلِيمَ ويُقَدِّسُهَا كَمَا كَانَتْ مِنَ الْقَدِيمِ.
\par 31 لكي تأتي الأمم من أقاصي الأرض لينظروا مجده،
\par     جلبت لها هدايا من أبنائها الذين أغمي عليهم.
\par     ولكي تبصر مجد الرب الذي مجدها الله به.
\par 32 ويكون عليهم ملكًا عادلًا، مُعَلَّمًا من الله،
\par 33 ولا يكون ظلم في أيامه في وسطهم.
\par     لأن الجميع يكونون قديسين وملكهم مسيح الرب.
\par 34 لأنه لا يتوكل على الخيل والفارس والقوس،
\par     ولا يكثر لنفسه الذهب والفضة للحرب،
\par   
\par 35 ولا يكتسب ثقة من جمع غفير ليوم المعركة
\par 36 الرب نفسه هو ملكه، ورجاء القدير من خلال رجائه في الله
\par   
\par 37 جميع الأمم تكون في خوف من أمامه،
\par     لأنه سيضرب الأرض بكلمة فمه إلى الأبد
\par 38 ويبارك شعب الرب بالحكمة والفرح،
\par     ويكون هو نفسه طاهرًا من الخطيئة، حتى يحكم شعبًا عظيمًا
\par 39 يُوبِّخ الحكام، ويُزيل الخطاة بقوة كلمته.
\par     ومتوكلاً على إلهه، فلن يعثر طوال أيامه
\par 40 لأن الله سيجعله قويًا بروحه القدوس،
\par     وحكيمًا بروح الفهم، بقوة وبر
\par 41 وتكون بركة الرب معه. فيتقوى ولا يعثر
\par     رجاءه في الرب فمن يغلبه؟
\par 42 سيكون قديرًا في أعماله، وقويًا في مخافة الله،
\par     سيرعى رعية الرب بأمانة واستقامة،
\par     ولا يدع أحداً منهم يعثر في مرعاهم.
\par 43 سيهديهم جميعًا إلى الصواب،
\par     ولن يكون بينهم فخر بأن يُظلم أحد منهم
\par 44 هذا يكون جلال ملك إسرائيل الذي يعلمه الله.
\par    يرفعه على بيت إسرائيل ليؤدبه
\par 45 كلماته أنقى من الذهب الثمين، وأجودها
\par     في المجالس يدين الشعوب، قبائل المقدسين
\par 46 تكون كلماته ككلمات القديسين في وسط الشعوب المقدسين
\par 47 طوبى للذين سيكونون في تلك الأيام،
\par     في أنهم سيرون نعيم إسرائيل الذي سيُحققه الله في جمع الأسباط
\par 48 ليُعجِّل الرب رحمته على إسرائيل!
\par    ليُنجِّنا من نجاسة الأعداء الأشرار!
\par 49 الرب نفسه هو ملكنا إلى الأبد.

\chapter{18}

مع هذا المزمور تنتهي أغاني سليمان الحربية.

\par 1 يا رب، رحمتك على أعمال يديك إلى الأبد.
\par     صلاحك على إسرائيل بعطية غنية
\par 2 تنظر إليهم عيناك، فلا يعاني أي منهم من العوز.
\par      تستمع أذناك إلى دعاء الفقراء المفعم بالأمل
\par 3 تُنفَّذ أحكامك على كل الأرض بالرحمة؛
\par     ومحبتك نحو نسل إبراهيم، بني إسرائيل
\par 4 تأديبك علينا كما على الابن البكر الوحيد،
\par     ليرد النفس المطيعة عن الحماقة التي صنعت في الجهل
\par 5 ليُطهِّر الله إسرائيل ليوم الرحمة والبركة،
\par     ليوم الاختيار عندما
\par 6 طوبى للذين يكونون في تلك الأيام،
\par     يُعيد مسيحه
\par 7 فيرون إحسان الرب الذي يصنعه للجيل الآتي،
\par 8 تحت قضيب تأديب مسيح الرب في مخافة إلهه،
\par    بروح الحكمة والبر والقوة؛
\par 9 لكي يُرشد كل إنسان إلى أعمال البر بمخافة الله،
\par    لكي يثبتهم جميعًا أمام الرب،
\par     جيل صالح يعيش في خوف الله في أيام الرحمة. سلاه.
\par   
\par 10 عظيم هو إلهنا ومجيد، ساكن في الأعالي.
\par 11 وهو الذي جعل في مساراتها أنوار السماء لتحديد الفصول من سنة إلى سنة.
\par    ولم يحيدوا عن الطريق الذي رسمه لهم.
\par 12 في خوف الله، يسلكون طريقهم كل يوم،
\par    منذ اليوم الذي خلقهم الله فيه وإلى الأبد
\par 13 ولم يضلوا منذ يوم خلقهم.
\par     منذ الأجيال القديمة لم يحيدوا عن طريقهم،
\par    إلا أن يأمرهم الله بذلك بأمر عباده.


\end{document}