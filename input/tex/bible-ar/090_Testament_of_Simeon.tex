\begin{document}

\title{وصية شمعون}

\chapter{1}

\par \textit{شمعون، الابن الثاني ليعقوب وليا. الرجل القوي. يغار من يوسف، وهو المحرض على المؤامرة ضده.}

\par 1 نسخة أقوال سمعان، الأشياء التي تكلم بها لأبنائه قبل وفاته، في السنة المئة والعشرين من حياته، حين مات يوسف أخوه

\par 2 لأنه لما مرض سمعان، جاء أبناؤه لزيارته. فتشدد وجلس وقبلهم وقال:

\par 3 اسمعوا يا أبنائي، لسمعان أبيكم، وسأخبركم بما في قلبي

\par 4 وُلِدتُ ليعقوبَ الابنَ الثانيَ لأبي، ودعتني أمي ليئةُ شمعونَ، لأنَّ الربَّ قد سمعَ صلواتها

\par 5 علاوة على ذلك، أصبحتُ قويًا للغاية؛ لم أتراجع عن أي إنجاز ولم أخشَ شيئًا. لأن قلبي كان قاسيًا، وكبدي كان جامدًا، وأحشائي بلا رحمة

\par 6 لأن الشجاعة قد أُعطيت أيضًا من العلي للناس في النفس والجسد

\par 7 لأني في شبابي كنت أغار من يوسف في أمور كثيرة، لأن أبي كان يحبه أكثر من كل شيء

\par 8 فجعلتُ قلبي عليه لأُبيده، لأن رئيس الضلال أرسل روح الغيرة فأعمى عقلي، فلم أعتبره أخًا، ولم أُشفق حتى على يعقوب أبي

\par 9 لكن إلهه وإله آبائه أرسل ملاكه وأنقذه من يدي

\par 10 لأنه لما ذهبتُ إلى شكيم لأحضر طيبًا للغنم، ورأوبين إلى دوثان، حيث كانت ضرورياتنا وجميع مؤننا، باعه أخي يهوذا للإسماعيليين

\par 11 فلما سمع رأوبين هذا الكلام حزن، لأنه أراد أن يرده إلى أبيه

\par 12 ولكن لما سمعتُ هذا غضبتُ غضبًا شديدًا على يهوذا لأنه تركه حيًا، وظللتُ غاضبًا عليه لمدة خمسة أشهر

\par 13 لكن الرب منعني، ومنع عني قوة يدي، لأن يدي اليمنى كانت يابسة نصفًا سبعة أيام

\par 14 وعلمتُ يا أبنائي أن هذا قد أصابني بسبب يوسف، فتبتُ وبكيت، وتوسلتُ إلى الرب الإله أن تُرد يدي، وأن أبتعد عن كل دنس وحسد وكل حماقة

\par 15 لأني علمت أني قد فكرت في أمر رديء أمام الرب وأبي يعقوب بسبب يوسف أخي إذ حسدته

\par 16 والآن يا أبنائي، استمعوا إليّ واحذروا روح الخداع والحسد

\par 17 لأن الحسد يسيطر على عقل الإنسان كله، فلا يدعه يأكل ولا يشرب ولا يفعل أي شيء صالح. بل يوحي إليه دائمًا بأنه يحسد، وما دام من يُحسد يزدهر، فإن من يحسد يذبل

\par 18 لذلك، لمدة سنتين، أذللت نفسي بالصوم في مخافة الرب، وتعلمت أن النجاة من الحسد تأتي بمخافة الله

\par 19 لأنه إذا هرب الإنسان إلى الرب، فإن الروح الشريرة تهرب منه، ويسترخي قلبه.

\par 20 ومن الآن فصاعدًا يتعاطف مع من حسده، ويغفر لمن يعاديه، وهكذا يكف عن حسده



\chapter{2}

\par \textit{ينصح رأوبين سامعيه ضد الحسد.}

\par 1 وسألني أبي عني لأنه رأى أني حزينة، فقلت له: أنا متوجع في كبدي.

\par 2 لأني حزنت أكثر منهم جميعًا، لأني كنت مذنبًا ببيع يوسف

\par 3 ولما نزلنا إلى مصر، وقيدني كجاسوس، علمت أنني أتألم بعدل، ولم أحزن

\par 4 وكان يوسف رجلاً صالحًا، وكان روح الله فيه. وكان حنونًا ورؤوفًا، ولم يحمل عليّ ضغينة، بل أحبني كباقي إخوته

\par 5 فاحذروا إذن يا أبنائي من كل غيرة وحسد، وامشوا بقلب صافي، لكي يعطيكم الله أيضًا نعمة ومجدًا وبركة على رؤوسكم، كما رأيتم في حالة يوسف

\par 6 لم يعاتبنا على هذا الأمر كل أيامه، بل أحبنا كنفسه، ومجدنا أكثر من أبنائه، وأعطانا غنىً ومواشي وثمارًا

\par 7 وأنتم أيضًا يا أبنائي، أحبوا كل واحد أخاه بقلب طيب، فتزول عنكم روح الحسد

\par 8 لأن هذا يُهلك الروح ويُدمر الجسد؛ ويُسبب الغضب والحرب في العقل، ويُثير أعمالًا دموية، ويقود العقل إلى الجنون، ويُسبب اضطرابًا في الروح ورعشة في الجسد

\par 9 لأنه حتى في النوم، تقضم الغيرة الخبيثة، ومع الأرواح الشريرة تُقلق النفس، وتُسبب اضطراب الجسد، وتوقظ العقل من النوم في ارتباك؛ وكروح شريرة سامة، هكذا تظهر للبشر

\par 10 لذلك كان يوسف جميل المنظر، وحسن المنظر، لأنه لم يكن فيه شر، لأن الوجه يظهر بعض ضيق الروح

\par 11 والآن يا أبنائي، اجعلوا قلوبكم صالحة أمام الرب، وطرقكم مستقيمة أمام الناس، فتجدوا نعمة أمام الرب والناس

\par 12 فاحذروا إذن من الزنا، فإنه أم كل الشرور، فهو انفصال عن الله، وتقريب إلى بليعر

\par 13 لأني رأيت مكتوبًا في كتابة أخنوخ أن أبناءك سيفسدون بالزنى، وسيؤذون أبناء لاوي بالسيف

\par 14 لكنهم لن يستطيعوا الصمود أمام لاوي، لأنه سيُحارب الرب، وسيُهزم جميع جيوشك

\par 15 ويكونون قليلين، منقسمين إلى لاوي ويهوذا، ولا يكون منكم أحد للسيادة، كما تنبأ أبونا أيضًا في بركاته

\chapter{3}

\par \textit{نبوءة عن مجيء المسيح.}

\par 1 ها أنا قد أخبرتكم بكل شيء، لكي أُبرَّر من خطيتكم

\par 2 الآن إن أزلتم من عنكم حسدكم وكل تصلب رقابكم، فستزهر وردة عظامي في إسرائيل، وكسوسنة لحمي في يعقوب، وتكون رائحتي كرائحة لبنان، ومثل الأرز يكثر القديسون عني إلى الأبد، وتمتد أغصانهم بعيدا.

\par 3 حينئذٍ تهلك نسل كنعان، ولا تبقى لعماليق بقية، ويهلك جميع الكبادوكيين، ويُباد جميع الحثيين تمامًا

\par 4 حينئذٍ تخرب أرض حام، ويهلك كل الشعب.

\par 5 حينئذ تستريح كل الأرض من الضيق، وكل المسكونة التي تحت السماء من الحرب.

\par 6 حينئذٍ يُمجِّدُ قَوِيُّ إِسْرَائِيلَ شَامًا.

\par 7 لأن الرب الإله سوف يظهر على الأرض ويخلص الناس بنفسه.

\par 8 حينئذٍ تُسلَّم جميع أرواح الخداع لتداس، ويتسلط البشر على الأرواح الشريرة

\par 9 حينئذٍ سأقوم فرحًا وأبارك العلي من أجل أعماله العجيبة، لأن الله اتخذ جسدًا وأكل مع البشر وخلص البشر

\par 10 والآن يا أبنائي، أطيعوا لاوي ويهوذا، ولا ترتفعوا على هذين السبطين، لأنه منهما يقوم لكم خلاص الله

\par 11 لأن الرب سيقيم من لاوي كرئيس كهنة، ومن يهوذا كملك، إلهًا وإنسانًا، ويخلص جميع الأمم وجنس إسرائيل

\par 12 لذلك أعطيكم هذه الوصايا لكي توصوا أنتم أيضًا أولادكم، لكي يحفظوها طوال أجيالهم

\par 13 ولما فرغ سمعان من وصية أبنائه، اضطجع مع آبائه وهو ابن مئة وعشرين سنة

\par 14 ووضعوه في تابوت من خشب، ليُنقلوا عظامه إلى الخليل. وقد نقلوها سرًا في حرب المصريين. لأن عظام يوسف حُفظت في قبور الملوك

\par 15 لأن السحرة أخبروهم أنه عند ذهاب عظام يوسف، سيكون هناك ظلام وكآبة في كل الأرض، ووباء عظيم جدًا للمصريين، حتى أنه حتى من خلال مصباح لا يمكن للإنسان أن يتعرف على أخيه

\par 16 وناح بنو شمعون على أبيهم.

\par 17 وكانوا في مصر إلى يوم خروجهم على يد موسى.


\end{document}