\begin{document}

\title{المكابيين الأول}

\chapter{1}

\par 1 وحدث بعد أن ضرب الإسكندر بن فيليب المقدوني الذي خرج من أرض كثييم داريوس ملك الفرس والماديين، أنه ملك عوضا عنه أولا على اليونان،
\par 2 وخاض حروبًا كثيرة، واستولوا على حصون كثيرة، وقتلوا ملوك الأرض،
\par 3 وبلغ أقاصي الأرض، وغنم غنائم أمم كثيرة، حتى هدأت الأرض أمامه، فارتقى وارتفع قلبه
\par 4 وجمع جيشًا عظيمًا قويًا، وملك على بلاد وأمم وملوك، فصاروا خاضعين له
\par 5 وبعد هذه الأمور مرض، وأدرك أنه سيموت
\par 6 لذلك دعا عبيده الكرام الذين نشأوا معه منذ صغره، وقسم مملكته بينهم وهو لا يزال حيًا
\par 7 فملك الإسكندر اثنتي عشرة سنة، ثم مات.
\par 8 وكان عبيده يحكمون كل واحد في مكانه.
\par 9 وبعد وفاته، وضعوا جميعًا أكاليل على أنفسهم، وكذلك فعل أبناؤهم من بعدهم سنين عديدة، وكثرت الشرور في الأرض
\par 10 وخرج منهم جذر شرير يُدعى أنطيوخس الملقب بإبيفانيس، ابن أنطيوخس الملك، الذي كان رهينة في روما، وملك في السنة المئة والسابعة والثلاثين لمملكة اليونانيين
\par 11 في تلك الأيام خرج من إسرائيل رجال أشرار، أغووا كثيرين قائلين: هلمَّ بنا ونقطع عهدًا مع الأمم الذين حولنا، لأنه منذ أن انفصلنا عنهم أصابنا حزنٌ كثير
\par 12 فسرّهم هذا التدبير.
\par 13 ثم استبشر بعض الناس بهذا الأمر، فذهبوا إلى الملك، فأعطاهم ترخيصًا بالعمل وفقًا لأحكام الوثنيين:
\par 14 وعند ذلك بنوا مكانًا للتريض في أورشليم حسب عادات الوثنيين:
\par 15 وجعلوا أنفسهم غرلة، وتركوا العهد المقدس، وانضموا إلى الأمم، وبيعوا لفعل الشر
\par 16 ولما توطدت المملكة أمام أنطيوخس، فكر في أن يملك على مصر ليكون له سلطان على مملكتين
\par 17 لذلك دخل مصر بحشد عظيم، بمركبات، وفيلة، وفرسان، وأسطول عظيم،
\par 18 وحارب بطليموس ملك مصر، فخاف منه بطليموس وهرب، فسقط كثيرون جرحى حتى الموت
\par 19 وهكذا حصلوا على المدن الحصينة في أرض مصر ونهب غنائمها
\par 20 وبعد أن ضرب أنطيوخس مصر، عاد مرة أخرى في السنة المائة والثالثة والأربعين، وصعد على إسرائيل وأورشليم بجيش عظيم،
\par 21 ودخل إلى المقدس بفخر، وأخذ مذبح الذهب ومنارة النور وكل آنيته،
\par 22 ومائدة خبز الوجوه، والأواني، والجامات، والمجامر من ذهب، والحجاب، والتاج، وأمتعة الذهب التي أمام الهيكل، كل ذلك نزعه.
\par 23 وأخذ أيضًا الفضة والذهب والأواني الثمينة، وأخذ أيضًا الكنوز المخبأة التي وجدها
\par 24 وبعد أن أخذ كل شيء، ذهب إلى أرضه، بعد أن ارتكب مذبحة عظيمة، وتكلم بكبرياء شديد
\par 25 فكان حزن عظيم في إسرائيل، في كل مكان كانوا فيه
\par 26 حتى أن الأمراء والشيوخ حزنوا، وضعف العذارى والشبان، وتغير جمال النساء
\par 27 أخذ كل عريس رثاءً، وكانت الجالسة في حجْل العرس في حزن،
\par 28 واهتزت الأرض أيضًا على سكانها، وغطى الفوضى كل بيت يعقوب
\par 29 وبعد انقضاء عامين كاملين، أرسل الملك رئيس جباة الجزية إلى مدن يهوذا، فجاء إلى أورشليم بحشد كبير،
\par 30 وكلمهم بكلام سلام، وكان كل ذلك بمكر. لأنه لما صدقوه، هجم بغتة على المدينة، وضربها ضربًا مبرحًا، وأهلك شعوبًا كثيرة من إسرائيل
\par 31 ولما غنم المدينة أحرقها بالنار، وهدم بيوتها وأسوارها من كل جانب
\par 32 أما النساء والأطفال فقد أسروا واستولوا على الماشية
\par 33 حينئذ بنوا مدينة داود سورًا عظيمًا وقويًا، وأبراجًا حصينة، وجعلوها حصنًا منيعًا لهم
\par 34 وجعلوا فيها أمة خاطئة، رجالاً أشرارًا، وتحصنوا فيها
\par 35 وخزنوها أيضًا مع الدروع والمؤن، ولما جمعوا غنائم أورشليم، وضعوها هناك، وهكذا أصبحت فخًا موجعًا
\par 36 لأنه كان مكمنًا للمقدس، وخصمًا شريرًا لإسرائيل
\par 37 فسفكوا دمًا بريئًا حول المقدس ونجّسوه
\par 38 حتى هرب سكان أورشليم بسببهم، فأصبحت المدينة مسكنا للغرباء، وغريبة على الذين ولدوا فيها، وتركها أبناؤها
\par 39 خُرِّبَ مَقْدِسُهَا كَبَرِيَّةٍ، وَتَحَوَّلَتْ أَعْدَامُهَا إِلَى حُزْنٍ، وَسُبُوتُهَا إِلَى تَعَييرِ، وَشَهْرَتُهَا إِلَى ازْدِرَاءٍ
\par 40 كما كان مجدها، كذلك ازداد عارها، وتحول شرفها إلى حداد
\par 41 علاوة على ذلك، كتب الملك أنطيوخس إلى كل مملكته، أن يكون الجميع شعبًا واحدًا،
\par 42 وكان على كل واحد أن يترك شرائعه، فوافق جميع الوثنيين على أمر الملك
\par 43 نعم، وافق كثيرون أيضًا من بني إسرائيل على دينه، وذبحوا للأصنام، ودنسوا السبت
\par 44 لأن الملك أرسل رسائل بواسطة رسل إلى أورشليم ومدن يهوذا لكي يتبعوا القوانين الغريبة في البلاد،
\par 45 ويمنعون المحرقات والذبائح والسكيب في الهيكل، ويمنعون تدنيس السبوت والأعياد
\par 46 ودنسوا الحرم والمقدسات:
\par 47 أقيموا مذابح وأسوارا ومزارات للأصنام، وأذبحوا لحم الخنزير والحيوانات النجسة.
\par 48 وأن يتركوا أولادهم أيضًا غير مختونين، ويجعلوا نفوسهم نجسة بكل أنواع النجاسة والدنس:
\par 49 لكي ينسوا الشريعة، ويغيروا جميع الأحكام.
\par 50 "ومن لا يعمل حسب أمر الملك فليموت."
\par 51 وكتب بنفس الطريقة إلى كل مملكته، وعيّن مراقبين على جميع الشعب، وأمر مدن يهوذا بالذبح مدينةً مدينةً
\par 52 فاجتمع إليهم كثير من الشعب، أي كل من ترك الشريعة، ففعلوا الشرور في الأرض
\par 53 ودفع بني إسرائيل إلى أماكن سرية، حتى إلى أي مكان يمكنهم الفرار إليه طلبًا للنجدة
\par 54 وفي اليوم الخامس عشر من شهر كسلو، في السنة المئة والخامسة والأربعين، أقاموا رجسة الخراب على المذبح، وبنوا مذابح للأصنام في جميع مدن يهوذا من كل جانب؛
\par 55 وأوقدوا البخور على أبواب بيوتهم وفي الشوارع.
\par 56 فمزقوا أسفار الشريعة التي وجدوها وأحرقوها بالنار.
\par 57 وكل من وُجد معه شيء من سفر العهد، أو من كان مُسلَّمًا عليه بالناموس، كانت وصية الملك أن يقتلوه
\par 58 وهكذا كانوا يفعلون بسلطانهم على بني إسرائيل كل شهر، لكل من وجد في المدن
\par 59 وفي اليوم الخامس والعشرين من الشهر ذبحوا على مذبح الأصنام الذي على مذبح الله
\par 60 في ذلك الوقت، حسب الوصية، قتلوا بعض النساء اللواتي ختن أطفالهن
\par 61 وعلقوا الأطفال على أعناقهم، ونهبوا بيوتهم، وقتلوا الذين ختنوهم
\par 62 ولكن كثيرين في إسرائيل عزموا وثبتوا في أنفسهم على عدم أكل أي شيء نجس
\par 63 لذلك كانوا يفضلون أن يموتوا، لئلا يتنجسوا بالأطعمة، لئلا ينجسوا العهد المقدس. لذلك ماتوا
\par 64 وكان غضب عظيم جدًا على إسرائيل.

\chapter{2}

\par 1 في تلك الأيام قام متثيا بن يوحنا بن سمعان، كاهن من بني يوياريب، من أورشليم، وسكن في مودين
\par 2 وكان لديه خمسة أبناء، جوانان، يُدعى كاديس:
\par 3 سمعان الملقب بـ تاسي:
\par 4 يهوذا الملقب بالمكابي:
\par 5 أليعازار، الملقب بأواران: ويوناثان، الملقب بأفوس.
\par 6 ولما رأى التجاديف التي صنعت في يهوذا وأورشليم،
\par 7 قال: ويل لي! لماذا وُلدتُ لأرى بؤس شعبي، وبؤس المدينة المقدسة، وأسكن هناك، وقد سُلِّمت في يد العدو، والحرم في يد الغرباء؟
\par 8 أصبح هيكلها كرجل بلا مجد.
\par 9 تُؤخذ أوانيها المجيدة إلى الأسر، ويُقتل أطفالها في الشوارع، ويُقتل شبابها بسيف العدو.
\par 10 أي أمة لم يكن لها نصيب في مملكتها ولم تنل من غنائمها؟
\par 11 أُخذت منها جميع زينتها؛ وأصبحت من امرأة حرة عبدة
\par 12 وهوذا مقدسنا، بهاءنا ومجدنا، قد خُرِّب، وقد دنسه الأمم
\par 13 لأي غاية نعيش بعد الآن؟
\par 14 فمزق متتيا وبنوه ثيابهم ولبسوا المسوح وناحوا حزناً عظيماً.
\par 15 في هذه الأثناء، جاء ضباط الملك، الذين أجبروا الناس على الثورة، إلى مدينة مودين، ليجبروهم على التضحية
\par 16 ولما جاء إليهم كثيرون من إسرائيل، اجتمع متثيا أيضًا وبنوه
\par 17 فأجاب ضباط الملك وقالوا لمتثيا: أنت رئيس ورجل شريف وعظيم في هذه المدينة، ومدعم بالأبناء والإخوة
\par 18 فالآن تعال أنت أولاً، وأنفذ أمر الملك، كما فعل جميع الوثنيين، بل ورجال يهوذا أيضًا، ومن بقي في أورشليم. هكذا تكون أنت وأهل بيتك في عداد أصدقاء الملك، وتُكرم أنت وأبناؤك بالفضة والذهب ومكافآت كثيرة
\par 19 فأجاب متثيا وتكلم بصوت عظيم: ولو أطاعته جميع الأمم التي تحت سلطان الملك، وارتد كل واحد عن ديانة آبائه، ورضي بأوامره،
\par 20 ومع ذلك، سأسلك أنا وأبنائي وإخوتي في عهد آبائنا
\par 21 حاشا لله أن نترك الشريعة والأحكام.
\par 22 لن نسمع لكلام الملك، ولن نحيد عن ديننا، لا يمنة ولا يسرة.
\par 23 وبعد أن انتهى من التكلم بهذه الكلمات، جاء واحد من اليهود أمام الجميع ليذبح على المذبح الذي في مودين حسب أمر الملك
\par 24 فلما رأى متثيا ذلك، اشتعلت غيرته وارتجفت لثامه، ولم يستطع أن يكبح غضبه حسب القضاء، فركض وقتله على المذبح
\par 25 كما قتل في ذلك الوقت مفوض الملك الذي كان يُجبر الرجال على التضحية، وهدم المذبح
\par 26 وهكذا عمل بغيرة من أجل شريعة الله كما فعل فينحاس بزمري بن شالوم
\par 27 فنادى متتيا في المدينة بصوت عظيم قائلاً: كل من غار للناموس وحافظ على العهد فليتبعني.
\par 28 فهرب هو وأبناؤه إلى الجبال، وتركوا كل ما كانوا يملكونه في المدينة
\par 29 حينئذٍ نزل كثيرون إلى البرية، الذين كانوا يطلبون العدل والحق، ليسكنوا هناك
\par 30 هم وأولادهم ونساؤهم ومواشيهم، لأن المحن اشتدت عليهم
\par 31 فأُخبر عبيد الملك والجيش الذي في أورشليم في مدينة داود أن رجالاً قد خالفوا أمر الملك ونزلوا إلى مخابئ في البرية،
\par 32 فطاردوهم عددًا كبيرًا، فلما أدركوهم خيموا عليهم وحاربوهم في يوم السبت
\par 33 فقالوا لهم: يكفيكم ما فعلتم حتى الآن. اخرجوا وافعلوا حسب أمر الملك فتحيوا
\par 34 فقالوا: لا نخرج ولا نعمل بأمر الملك لئلا ندنس يوم السبت
\par 35 ثم خاضوا معهم المعركة بكل سرعة.
\par 36 ولكنهم لم يجيبوهم، ولا رموا عليهم حجراً، ولا سدوا الأماكن التي كانوا مختبئين فيها.
\par 37 لكنهم قالوا: لنمت جميعاً في براءتنا. السماء والأرض تشهدان لنا بأنكم تقتلوننا ظلماً
\par 38 فقاموا عليهم في قتال يوم السبت، فقتلوهم مع نسائهم وأولادهم ومواشيهم، وكان عددهم ألف نفس
\par 39 فلما علم متاثياس وأصدقاؤه بذلك، حزنوا عليهم حزنًا شديدًا
\par 40 وقال أحدهم لآخر: إذا فعلنا جميعًا كما فعل إخوتنا، ولم نقاتل من أجل أرواحنا وقوانيننا ضد الوثنيين، فسوف يقتلعوننا الآن بسرعة من الأرض
\par 41 في ذلك الوقت، قضوا قائلين: كل من يأتي لمحاربتنا يوم السبت، نقاتله، ولا نموت جميعًا كما قُتل إخوتنا في المخابئ
\par 42 ثم جاءت إليه جماعة من الأسيديين، وهم من أشد رجال إسرائيل، كل من تطوع للشريعة
\par 43 وانضم إليهم أيضًا جميع الذين فروا من الاضطهاد، وكانوا لهم سندًا
\par 44 فجمعوا قواهم، وضربوا الخطاة بغضبهم، والأشرار بسخطهم، أما الباقون فهربوا إلى الأمم طلبًا للمساعدة
\par 45 ثم طاف متثيا وأصحابه وهدموا المذابح
\par 46 وكل من وجدوه من الأطفال في ساحل إسرائيل غير مختونين، فإنهم ختنوه بشجاعة
\par 47 وطاردوا أيضًا الرجال المتكبرين، ونجح العمل في أيديهم
\par 48 فاستعادوا الشريعة من أيدي الأمم، ومن أيدي الملوك، ولم يدعوا الخاطئ ينتصر
\par 49 ولما اقتربت أيام متتيا أن يموت، قال لبنيه: الآن اشتدت الكبرياء والتوبيخ، ووقت الهلاك، وغضب السخط.
\par 50 والآن يا أبنائي، كونوا غيورين للشريعة، وابذلوا أنفسكم من أجل عهد آبائكم
\par 51 تذكروا أعمال آبائنا في زمانهم، فتنالون شرفًا عظيمًا واسمًا أبديًا
\par 52 ألم يُوجَد إبراهيم مُؤمِنًا في التجربة، فَحُسِبَت له تلك التجربة برًّا؟
\par 53 حفظ يوسف الوصية في وقت ضيقه، فصار سيدًا على مصر
\par 54 فينياس أبونا، إذ كان غيورًا ومتقدًا، نال عهد كهنوت أبدي
\par 55 جُعِلَ يسوعُ قاضيًا في إسرائيلَ لإتمامِهِ الكلمةَ.
\par 56 وكالب لأنه كان يشهد أمام الجماعة نال ميراث الأرض.
\par 57 لقد امتلك داود، لكونه رحيمًا، عرش مملكة أبدية
\par 58 رُفِعَ إيليا إلى السماء لأنه كان غيورًا ومُحِمًّا للناموس
\par 59 حننيا وعزريا وميشائيل، بإيمانهم، نجوا من اللهيب
\par 60 لقد نجا دانيال من أفواه الأسود بفضل براءته.
\par 61 "وهكذا اعتبروا في كل العصور أنه لا أحد من الذين يتوكلون عليه يغلب."
\par 62 لا تخف إذن من كلام الرجل الخاطئ، لأن مجده سيكون روثًا ودودًا
\par 63 اليوم يُرفع، وغدًا لا يُوجد، لأنه يعود إلى ترابه، وقد ضاع فكره
\par 64 لذلك يا أبنائي، كونوا شجعانًا وأظهروا أنفسكم رجالًا في سبيل الشريعة؛ لأنكم بها تنالون المجد
\par 65 وهوذا أعلم أن أخاكم سمعان رجل مشورة، فاسمعوا له كل حين، فهو يكون لكم أباً
\par 66 أما يهوذا المكابي، فقد كان جبارًا وقويًا، منذ شبابه. فليكن قائدكم، ويقاتل في معركة الشعب
\par 67 خذوا إليكم أيضًا جميع الذين يلتزمون بالقانون، وانتقموا لظلم شعبكم
\par 68 كافئوا الأمم جزاءً كاملاً، واحفظوا وصايا الشريعة
\par 69 فباركهم وانضم إلى آبائه.
\par 70 ومات في السنة المائة والسادسة والأربعين، فدفنه بنوه في قبور آبائه في مودين، وناح عليه كل إسرائيل حزناً عظيماً.

\chapter{3}

\par 1 ثم قام مكانه ابنه يهوذا، الملقب بالمكابي
\par 2 وساعده جميع إخوته، وكذلك جميع الذين ساندوا أبيه، وحاربوا بفرح في معركة إسرائيل
\par 3 فمنح شعبه شرفًا عظيمًا، وارتدى درعًا كجبار، وشد حوله حزامه الحربي، وخاض المعارك، وحمى الجيش بسيفه
\par 4 كان في تصرفاته كالأسد، وكشبل أسد يزأر باحثًا عن فريسته
\par 5 لأنه طارد الأشرار، وبحث عنهم، وأحرق الذين أغاظوا شعبه
\par 6 لذلك انكمش الأشرار خوفًا منه، واضطرب كل فاعلي الإثم، لأن الخلاص نجح على يده
\par 7 وأحزن أيضًا ملوكا كثيرين، وأفرح يعقوب بأعماله، فبقي ذكره مباركًا إلى الأبد
\par 8 ثم طاف في مدن يهوذا، مهلكًا الأشرار منها، وصرف الغضب عن إسرائيل:
\par 9 حتى اشتهر إلى أقصى بقاع الأرض، واستقبل فيه من كانوا على وشك الهلاك
\par 10 ثم جمع أبولونيوس الأمم، وجيشًا عظيمًا من السامرة، لمحاربة إسرائيل
\par 11 فلما علم يهوذا بذلك خرج للقائه، فضربه وقتله. وسقط كثيرون أيضًا قتلى، أما الباقون فهربوا
\par 12 لذلك أخذ يهوذا غنائمهم، وسيف أبولونيوس أيضًا، وحارب به طوال حياته
\par 13 ولما سمع سارون، أمير جيش سورية، أن يهوذا قد جمع إليه جمهورًا وجماعة من المؤمنين للخروج معه إلى الحرب،
\par 14 قال: «سأكون لي اسم وكرامة في المملكة، لأني سأذهب لأحارب يهوذا والذين معه الذين يحتقرون أمر الملك».
\par 15 فأعدّ نفسه للصعود، وذهب معه جيش عظيم من الأشرار لمساعدته والانتقام من بني إسرائيل
\par 16 ولما اقترب من صعود بيت حورون، خرج يهوذا للقائه في جماعة صغيرة
\par 17 فلما رأوا الجيش مقبلاً للقائهم، قالوا ليهوذا: كيف نستطيع ونحن قليلون أن نقاتل ضد هذا الجمع الكثير والقوي، ونحن على وشك أن نكل من الصوم كل هذا اليوم؟
\par 18 فأجابه يهوذا: ليس من الصعب على الكثيرين أن يُحبسوا في أيدي القليلين، ومع إله السماء الأمر واحد، أن يُنقذ بجمع كثير أو جماعة قليلة
\par 19 لأن النصر في المعركة ليس بكثرة الجيش، بل القوة تأتي من السماء
\par 20 يأتون إلينا بكبرياء وإثم كبيرين ليهلكونا، ونساؤنا وأطفالنا، ويسلبونا:
\par 21 لكننا نناضل من أجل حياتنا وقوانيننا.
\par 22 لذلك فإن الرب نفسه سيهدمهم أمام وجوهنا، وأما أنتم فلا تخافوا منهم.
\par 23 وبمجرد أن انتهى من الكلام، قفز عليهم فجأة، وهكذا انهزم سيرون وجيشه أمامه
\par 24 فتبعوهم من منحدر بيت حورون إلى السهل، فقتلوا منهم نحو ثمانمائة رجل، وهرب الباقون إلى أرض الفلسطينيين
\par 25 فبدأ خوف يهوذا وإخوته، ورعب عظيم جدًا، يقع على الأمم المحيطة بهم
\par 26 حتى وصل صيته إلى الملك، وتحدثت جميع الأمم عن معارك يهوذا
\par 27 ولما سمع الملك أنطيوخس هذه الأمور امتلأ غضبا، فأرسل وجمع كل جيوش مملكته، حتى كان جيشا قويا جدا.
\par 28 كما فتح كنزه، وأعطى جنوده رواتبهم لمدة عام، وأمرهم بأن يكونوا على أهبة الاستعداد كلما احتاج إليهم
\par 29 ومع ذلك، عندما رأى أن أموال كنوزه قد نفدت وأن الجزية في البلاد كانت قليلة، بسبب الفتنة والطاعون الذي جلبه على البلاد بإلغاء القوانين التي كانت موجودة في الزمن القديم؛
\par 30 كان يخشى ألا يتمكن من تحمل التكاليف بعد الآن، ولا أن تكون لديه مثل هذه الهدايا ليقدمها بسخاء كما فعل من قبل: لأنه كان يفوق الملوك الذين سبقوه
\par 31 لذلك، ولأنه كان في حيرة شديدة، قرر الذهاب إلى بلاد فارس، ليأخذ جزية البلاد هناك، ويجمع الكثير من المال
\par 32 فترك ليسياس، وهو رجل شريف وأحد النسل الملكي، ليشرف على شؤون الملك من نهر الفرات إلى حدود مصر:
\par 33 وأن يُربي ابنه أنطيوخس حتى يعود.
\par 34 ثم سلم إليه نصف جيوشه والفيلة، وأعطاه وصية على كل ما كان يريد أن يفعله، وكذلك على سكان يهوذا وأورشليم.
\par 35 أي أنه سيرسل جيشًا ضدهم، ليدمر ويستأصل قوة إسرائيل وبقية أورشليم، ويزيل ذكرهم من ذلك المكان؛
\par 36 وأن يُسكن غرباء في جميع ديارهم، ويقسم أرضهم بالقرعة
\par 37 فأخذ الملك نصف القوات التي بقيت، وخرج من أنطاكية، مدينته الملكية، في السنة المئة والسابعة والأربعين، وبعد أن عبر نهر الفرات، سار في المرتفعات
\par 38 ثم اختار ليسياس بطليموس بن دوريمانس، ونيكانور، وجورجياس، رجالاً أقوياء من أصدقاء الملك:
\par 39 وأرسل معهم أربعين ألف راجل، وسبعة آلاف فارس، ليدخلوا أرض يهوذا ويدمروها كما أمر الملك
\par 40 فخرجوا بكل قوتهم، وأتوا ونزلوا بالقرب من عمواس في أرض السهل
\par 41 فسمع تجار البلاد بخبرهم، فأخذوا فضة وذهبًا كثيرًا جدًا، مع عبيدهم، وجاءوا إلى المحلة ليشتروا بني إسرائيل عبيدًا. وانضمت إليهم أيضًا قوة من سورية وأرض الفلسطينيين
\par 42 ولما رأى يهوذا وإخوته أن الشرور قد كثرت، وأن الجيوش قد حاصرت تخومهم، لأنهم علموا أن الملك قد أمر بإهلاك الشعب وإبادةهم تمامًا؛
\par 43 قالوا بعضهم لبعض: دعونا نعيد لشعبنا ما تهدم، ولنقاتل من أجل شعبنا والمقدس
\par 44 ثم اجتمعت الجماعة لكي يكونوا مستعدين للقتال ولكي يصلوا ويطلبوا الرحمة والرأفة.
\par 45 وكانت أورشليم خاوية كقفر، لم يكن أحد من بنيها يدخل أو يخرج. ودُوس المقدس، وحرس الأجانب الحصن. وكان للأمم مسكنهم في ذلك المكان. ونزع الفرح من يعقوب، وتوقف المزمار والقيثارة
\par 46 لذلك اجتمع بنو إسرائيل وجاءوا إلى المصفاة، مقابل أورشليم؛ لأن المصفاة كانت المكان الذي صلوا فيه سابقًا في إسرائيل
\par 47 فصاموا في ذلك اليوم، ولبسوا المسوح، وذروا الرماد على رؤوسهم، ومزقوا ثيابهم،
\par 48 وفتح سفر الشريعة الذي كان الوثنيون يسعون فيه إلى رسم صورة أصنامهم
\par 49 وأتوا أيضًا بثياب الكهنة، والبواكير، والعشور، وأثاروا النذير الذين أكملوا أيامهم
\par 50 فصرخوا بصوت عظيم نحو السماء قائلين: ماذا نفعل بهؤلاء وإلى أين نحملهم؟
\par 51 لأن مقدسك قد دُوسَ ودُنِّسَ، وكهنتك في حزنٍ وذل
\par 52 وها إن الأمم قد اجتمعت علينا ليهلكونا، وأنت تعلم ما يظنونه ضدنا
\par 53 كيف نستطيع أن نقف في وجههم إلا إذا كنت أنت يا الله معيننا؟
\par 54 ثم نفخوا في الأبواق وصرخوا بصوت عظيم.
\par 55 وبعد ذلك أقام يهوذا رؤساء على الشعب، رؤساء ألوف، ورؤساء مئات، ورؤساء خمسين، ورؤساء عشرات.
\par 56 وأما الذين كانوا يبنون بيوتًا، أو كانوا قد خطبوا نساءً، أو غرسوا كرومًا، أو كانوا خائفين، فأولئك أمرهم أن يرجعوا كل واحد إلى بيته حسب الناموس
\par 57 ثم انتقل المخيم ونزل على الجانب الجنوبي من عمواس.
\par 58 فقال يهوذا تسلحوا وكونوا رجالا ذوي بأس واستعدوا للصباح لكي تحاربوا هؤلاء الأمم المجتمعين علينا ليهلكونا ومقدسنا.
\par 59 لأنه من الأفضل لنا أن نموت في المعركة، من أن نرى مصائب شعبنا ومقدسنا
\par 60 ولكن كما أن مشيئة الله في السماء، فليفعل.

\chapter{4}

\par 1 ثم أخذ جورجياس خمسة آلاف راجل، وألفًا من أفضل الفرسان، وخرج من المخيم ليلًا؛
\par 2 لكي يتمكن من الاندفاع نحو معسكر اليهود وضربهم فجأة. وكان رجال الحصن مرشديه
\par 3 فلما سمع يهوذا بذلك، انطلق هو والرجال البواسل الذين معه، ليضرب جيش الملك الذي في عمواس،
\par 4 بينما كانت القوات قد تفرقت من المعسكر حتى الآن.
\par 5 وفي تلك الأثناء جاء جورجياس ليلا إلى محلة يهوذا، ولما لم يجد هناك أحدا، طلبهم في الجبال، لأنه قال: هؤلاء الرجال هربوا منا.
\par 6 ولكن بمجرد أن طلع النهار، ظهر يهوذا في السهل مع ثلاثة آلاف رجل، ومع ذلك لم تكن لديهم دروع ولا سيوف في أذهانهم
\par 7 ورأوا معسكر الأمم أنه قوي ومجهز جيدًا، ومحاط بفرسان، وكانوا خبراء في الحرب
\par 8 ثم قال يهوذا للرجال الذين معه: لا تخافوا من جمعهم، ولا تخشوا هجومهم
\par 9 اذكر كيف نجا آباؤنا في البحر الأحمر، حين طاردهم فرعون بجيش
\par 10 والآن فلنصرخ إلى السماء، عسى أن يرحمنا الرب، ويتذكر عهد آبائنا، ويهلك هذا الجيش أمام وجوهنا اليوم
\par 11 لكي يعلم جميع الأمم أن هناك من يخلص إسرائيل ويخلصها
\par 12 فرفع الغرباء أعينهم ورأوهم مقبلين نحوهم
\par 13 فخرجوا من المحلة للقتال، وأما الذين مع يهوذا فنفخوا في أبواقهم
\par 14 فانضموا إلى المعركة، فهرب الوثنيون إلى السهل بعد أن انهزموا
\par 15 فقتل جميع من تأخر منهم بالسيف، وطاردوهم إلى جازرة، وإلى سهول أدوم، وأشدود، ويمنيا، فقتل منهم ثلاثة آلاف رجل
\par 16 بعد أن فعل ذلك، عاد يهوذا مع جيشه من مطاردتهم،
\par 17 وقال للشعب لا تطمعوا في الغنيمة لأن أمامنا معركة.
\par 18 وجورجياس وجيشه هنا بجانبنا في الجبل. لكن قف الآن في وجه أعدائنا، واهزمهم، وبعد ذلك يمكنك أن تأخذ الغنائم بشجاعة
\par 19 وبينما كان يهوذا يتكلم بهذه الكلمات، ظهر جزء منهم ينظر من الجبل:
\par 20 الذين لما رأوا أن اليهود قد هزموا جيشهم وأحرقوا الخيام، لأن الدخان الذي شوهد أنبأ بما حدث:
\par 21 فلما أدركوا هذه الأمور، خافوا خوفًا شديدًا، ورأوا أيضًا جيش يهوذا في السهل مستعدًا للقتال،
\par 22 هربوا جميعًا إلى أرض الغرباء.
\par 23 ثم رجع يهوذا لينهب الخيام، فغنموا ذهباً كثيراً، وفضة، وحريراً أزرق، وأرجوان البحر، وثروة عظيمة.
\par 24 بعد هذا ذهبوا إلى منازلهم، وغنوا ترنيمة شكر، وسبحوا الرب في السماء لأنه صالح، ولأن إلى الأبد رحمته
\par 25 وهكذا نال إسرائيل خلاصًا عظيمًا في ذلك اليوم.
\par 26 فجاء جميع الغرباء الذين هربوا وأخبروا ليسياس بما حدث.
\par 27 فلما سمع بذلك، خاب أمله واضطرب، لأنه لم يُفعل بإسرائيل ما أراد، ولم يحدث ما أمره به الملك
\par 28 "وفي العام التالي جمع ليسياس ستين ألف رجل من المشاة المختارين وخمسة آلاف فارس لكي يقهرهم."
\par 29 فجاءوا إلى أدوم، ونزلوا في بيت صور، فقابلهم يهوذا بعشرة آلاف رجل
\par 30 ولما رأى ذلك الجيش الجبار، صلى وقال: مبارك أنت يا مخلص إسرائيل، الذي قمع عنف الجبار بيد عبدك داود، ودفع جيش الغرباء إلى يد يوناثان بن شاول وحامل سلاحه
\par 31 احبس هذا الجيش في يد شعبك إسرائيل، وليخزَ في قوتهم وفرسانهم
\par 32 أفقدهم الشجاعة، وأسقط جرأة قوتهم، ودعهم يرتعدون من هول دمارهم:
\par 33 اطرحهم بسيف محبيك، وليسبحك كل من يعرف اسمك بشكر
\par 34 فاشتبكوا في المعركة، فقُتل من جيش ليسياس نحو خمسة آلاف رجل، حتى أنهم قُتلوا أمامهم
\par 35 ولما رأى ليسياس هزيمة جيشه، وشجاعة جنود يهوذا، واستعدادهم إما للحياة أو الموت بشجاعة، ذهب إلى أنطاكية، وجمع جماعة من الغرباء، وبعد أن جعل جيشه أعظم مما كان، عزم على العودة إلى اليهودية
\par 36 فقال يهوذا وإخوته: هوذا أعداؤنا قد انهزموا. هلمَّ نصعد لنطهر ونُدَشِّن المقدس
\par 37 عند ذلك اجتمع كل الجيش وصعدوا إلى جبل صهيون
\par 38 ولما رأوا المقدس خرابًا، والمذبح مدنسًا، والأبواب محترقة، والشجيرات نابتة في الديار كما في غابة أو في أحد الجبال، نعم، وغرف الكهنة مهدومة؛
\par 39 مزقوا ثيابهم، وأقاموا نحيبًا عظيمًا، وذروا الرماد على رؤوسهم،
\par 40 وسقطوا على الأرض على وجوههم، ونفخوا في الأبواق، وصرخوا نحو السماء
\par 41 ثم عيّن يهوذا رجالاً لقتال الذين في الحصن، حتى يُطهّر المقدس
\par 42 فاختار كهنة ذوي سيرة بلا لوم، الذين ارتضوا بالناموس:
\par 43 الذي طهر المقدس، ورفع الحجارة النجسة إلى مكان نجس
\par 44 ولما تشاوروا ماذا يفعلون بمذبح المحرقة الذي كان مدنسًا؛
\par 45 رأوا أنه من الأفضل هدمه، لئلا يكون عارا عليهم، لأن الوثنيين قد دنسوه. لذلك هدموه،
\par 46 ووضعوا الحجارة في جبل الهيكل في مكان مناسب، إلى أن يأتي نبي ليُخبرهم ماذا يُفعل بها
\par 47 ثم أخذوا حجارة صحيحة حسب الناموس، وبنوا مذبحًا جديدًا حسب الناموس السابق؛
\par 48 وبنى المقدس وما داخل الهيكل وقدس الديار
\par 49 وصنعوا أيضًا أواني مقدسة جديدة، وأدخلوا إلى الهيكل المنارة ومذبح المحرقة والبخور والمائدة
\par 50 وأوقدوا البخور على المذبح، وأوقدوا السرج التي على المنارة، لتُضيء في الهيكل
\par 51 ثم وضعوا الأرغفة على المائدة، وبسطوا الستر، وأكملوا جميع الأعمال التي بدأوا في صنعها
\par 52 وفي اليوم الخامس والعشرين من الشهر التاسع، الذي يُدعى شهر كسلو، في السنة المائة والثامنة والأربعين، قاموا باكرًا في الصباح،
\par 53 وقدموا ذبيحة حسب الشريعة على مذبح المحرقة الجديد الذي صنعوه
\par 54 انظروا، في أي وقت وأي يوم دنسه الوثنيون، حتى في ذلك الوقت تم تكريسه بالأغاني والقيثارات والقيثارات والصنوج
\par 55 فخرّ جميع الشعب على وجوههم ساجدين ومسبّحين لإله السماء الذي وفقهم
\par 56 وهكذا احتفلوا بتدشين المذبح ثمانية أيام، وقدموا محرقات بفرح، وذبحوا ذبيحة الخلاص والتسبيح
\par 57 وزينوا أيضًا واجهة الهيكل بتيجان من ذهب وتروس، وجددوا الأبواب والغرف، وعلقوا عليها مصاريع
\par 58 فكان فرح عظيم بين الشعب، لأن عار الوثنيين قد زال
\par 59 ثم رسم يهوذا وإخوته مع كل جماعة إسرائيل أن تُحفظ أيام تدشين المذبح في وقتها من سنة إلى سنة بثمانية أيام، من اليوم الخامس والعشرين من شهر كسلو، بفرح وابتهاج
\par 60 في ذلك الوقت أيضًا بنوا جبل صهيون بأسوار عالية وأبراج حصينة حوله، لئلا تأتي الأمم وتدوسه كما فعلوا من قبل
\par 61 وأقاموا هناك حامية لحراستها، وحصنوا بيت صور للحفاظ عليها، حتى يكون للشعب دفاع ضد أدوم

\chapter{5}

\par 1 فلما سمع الأمم من حولهم أن المذبح قد بُني وأن المقدس قد تجدد كما كان من قبل، ساء ذلك في أعينهم جدًا
\par 2 لذلك فكروا في إهلاك جيل يعقوب الذي كان بينهم، ومن ثم بدأوا في قتل الشعب وإهلاكه
\par 3 ثم حارب يهوذا بني عيسو في أدوم عند أراباتين، لأنهم حاصروا غيل، وألحق بهم هزيمة نكراء، وضعف شجاعتهم، واستولى على غنائمهم
\par 4 وتذكر أيضًا إصابة أبناء بين، الذين كانوا فخًا وذنبًا للشعب، إذ كانوا يتربصون لهم في الطرق
\par 5 فحاصرهم في الأبراج، ونزل عليهم، وحرمهم تمامًا، وأحرق أبراج ذلك المكان بالنار مع كل ما فيها
\par 6 وبعد ذلك انتقل إلى بني عمون، حيث وجد جيشاً عظيماً وشعباً كثيراً، وكان قائدهم تيموثاوس.
\par 7 فقاتل معهم معارك كثيرة، حتى هُزموا في النهاية أمامه، فضربهم
\par 8 ثم استولى على يعزار والقرى التابعة لها، ورجع إلى اليهودية
\par 9 ثم اجتمع الوثنيون الذين في جلعاد على بني إسرائيل الذين في مساكنهم لإهلاكهم، لكنهم هربوا إلى حصن ديثما
\par 10 وأرسلوا رسائل إلى يهوذا وإخوته: إن الأمم التي حولنا قد اجتمعت علينا ليهلكونا
\par 11 وهم يستعدون للمجيء والاستيلاء على الحصن الذي هربنا إليه، وكان تيموثاوس قائد جيشهم
\par 12 فتعالوا الآن، وأنقذونا من أيديهم، فقد قُتل منا كثيرون
\par 13 نعم، جميع إخوتنا الذين كانوا في أماكن طوبيا قُتلوا. وأسروا نسائهم وأطفالهم أيضًا، وحملوا أمتعتهم، وأهلكوا هناك نحو ألف رجل
\par 14 بينما كانوا يقرأون هذه الرسائل، إذا برسل آخرين قد جاءوا من الجليل وثيابهم ممزقة، وأخبروا بذلك،
\par 15 وقال: قد اجتمع علينا أهل بطلمايس وصور وصيدا وكل جليل الأمم ليبيدونا
\par 16 فلما سمع يهوذا والشعب هذا الكلام، اجتمعوا جماعة عظيمة ليتشاوروا ماذا يفعلون لإخوتهم الذين كانوا في ضيق ومضطهدين منهم
\par 17 ثم قال يهوذا لسمعان أخيه: اختر لنفسك رجالاً واذهب وأنقذ إخوتك الذين في الجليل، لأني أنا ويوناثان أخي سنذهب إلى بلاد جلعاد
\par 18 فترك يوسف بن زكريا وعزريا قائدي الشعب مع بقية الجيش في اليهودية لحفظه
\par 19 الذي أوصى به قائلاً: تولوا أمر هذا الشعب، واحذروا أن تحاربوا الأمم حتى نعود
\par 20 فأُعطي لسمعان ثلاثة آلاف رجل للذهاب إلى الجليل، ولِيهوذا ثمانية آلاف رجل للذهاب إلى بلاد جلعاد
\par 21 حينئذٍ ذهب سمعان إلى الجليل، حيث خاض معارك كثيرة مع الوثنيين، حتى انهزم الوثنيون أمامه
\par 22 وطاردهم إلى باب بطليموس، فقتل من الوثنيين نحو ثلاثة آلاف رجل، وغنم غنائمهم
\par 23 وأما الذين في الجليل وفي العرباتيس، مع نسائهم وأولادهم وكل ما كان لهم، فأخذهم معه، وجاء بهم إلى اليهودية بفرح عظيم
\par 24 وعبر يهوذا المكابي وأخوه يوناثان الأردن أيضًا، وسافرا مسيرة ثلاثة أيام في البرية،
\par 25 حيث التقوا بالنابثيين، فجاءوا إليهم بسلام، وأخبروهم بكل ما حدث لإخوتهم في أرض جلعاد.
\par 26 وكيف أن كثيرين منهم كانوا محبوسين في بصرى، وباشور، وعليما، وكافور، ومكيد، وقرنايم؛ كل هذه المدن قوية وعظيمة:
\par 27 وأنهم كانوا محاصرين في بقية مدن بلاد جلعاد، وأنهم قد عينوا غدًا أن يحشدوا جيشهم ضد الحصون، ويستولوا عليها، ويدمروها جميعًا في يوم واحد
\par 28 عندئذٍ انعطف يهوذا وجيشه فجأةً في طريق البرية إلى بصرى، ولما استولى على المدينة، قتل كل ذكر بحد السيف، وأخذ كل غنائمهم، وأحرق المدينة بالنار،
\par 29 انطلق من هناك ليلًا، وذهب حتى وصل إلى الحصن
\par 30 وفي الصباح الباكر، نظروا إلى أعلى، وإذا بقوم لا يحصى عددهم يحملون سلالم وأدوات حرب أخرى، للاستيلاء على الحصن، لأنهم هاجموهم
\par 31 فلما رأى يهوذا أن القتال قد بدأ، وأن صراخ المدينة قد ارتفع إلى السماء بأبواق وصوت عظيم،
\par 32 قال لجنده: قاتلوا اليوم عن إخوتكم.
\par 33 فخرج خلفهم في ثلاث فرق، وهم ينفخون في الأبواق ويصلون.
\par 34 حينئذٍ، لما علم جيش تيموثاوس أنه المكابي، هربوا منه، فضربهم قتلاً عظيماً، حتى سقط منهم في ذلك اليوم نحو ثمانية آلاف رجل
\par 35 بعد أن فعل ذلك، توجه يهوذا إلى المصفاة؛ وبعد أن هاجمها، استولى عليها وقتل جميع الذكور فيها، وأخذ غنائمها وأحرقها بالنار
\par 36 وانطلق من هناك، واستولى على كسفون، وماجد، وباصور، وسائر مدن بلاد جلعاد
\par 37 بعد هذه الأمور، جمع تيموثاوس جيشًا آخر وعسكر ضد رافون وراء الوادي
\par 38 فأرسل يهوذا رجالاً ليتجسسوا الجيش، فأخبروه قائلين: «لقد اجتمع عليهم كل الأمم الذين حولنا، جيش عظيم جدًا».
\par 39 وقد استأجر أيضًا العرب لمساعدتهم، وقد نصبوا خيامهم وراء الوادي، مستعدين للقدوم ومحاربتك. عند ذلك ذهب يهوذا لملاقاتهم
\par 40 ثم قال تيموثاوس لقواد جيشه: «متى اقترب يهوذا وجيشه من الوادي، فإن عبر إلينا أولاً، فلن نقدر على الصمود أمامه، لأنه يغلبنا بشدة
\par 41 ولكن إن خاف ونزل في عبر النهر، فإننا نعبر إليه ونتغلب عليه
\par 42 ولما اقترب يهوذا من الوادي، أمر كتبة الشعب بالبقاء عند الوادي، فأوصاهم قائلاً: لا تدع أحدًا يمكث في المحلة، بل ليأت الجميع إلى القتال
\par 43 فعبر إليهم أولاً، وكل الشعب في إثره. حينئذ انهزمت كل الأمم أمامه، وألقوا أسلحتهم وهربوا إلى الهيكل الذي في قرنائيم.
\par 44 لكنهم استولوا على المدينة وأحرقوا الهيكل بكل ما فيه. وهكذا خضعت قرنايم، ولم يتمكنوا من الصمود أمام يهوذا
\par 45 فجمع يهوذا جميع بني إسرائيل الذين في بلاد جلعاد، من صغيرهم إلى كبيرهم، حتى نسائهم وأولادهم وأمتعتهم، جيشًا عظيمًا جدًا، لكي يدخلوا أرض اليهودية
\par 46 ولما وصلوا إلى عفرون، (كانت مدينة عظيمة في الطريق الذي كانوا يسيرون فيه، محصنة تحصينًا جيدًا)، لم يستطيعوا أن يحيدوا عنها لا يمينًا ولا يسارًا، بل كان لا بد أن يمروا في وسطها
\par 47 فأغلق أهل المدينة أبوابهم، وسدُّوا الأبواب بالحجارة
\par 48 فأرسل يهوذا إليهم بسلام قائلاً: دعونا نمر بأرضكم لنذهب إلى أرضنا، ولن يضركم أحد. سنمر فقط سيرًا على الأقدام. لكنهم لم يفتحوا له
\par 49 لذلك أمر يهوذا أن يُنادى في جميع أنحاء الجيش، أن ينصب كل رجل خيمته في المكان الذي يكون فيه
\par 50 فنزل الجنود وهاجموا المدينة طوال ذلك اليوم والليلة، حتى سُلمت المدينة في يديه:
\par 51 فقتل كل ذكر بحد السيف، وهدم المدينة، ونهب غنائمها، وعبر المدينة فوق القتلى
\par 52 بعد ذلك عبروا الأردن إلى السهل الكبير أمام بيت شان
\par 53 فجمع يهوذا الذين تقدموا، وحثّ الشعب طوال الطريق حتى وصلوا إلى أرض اليهودية
\par 54 فصعدوا إلى جبل صهيون بفرح وابتهاج، حيث قدموا محرقات، لأنه لم يُقتل منهم أحد حتى رجعوا بسلام
\par 55 وفي الوقت الذي كان فيه يهوذا ويوناثان في أرض جلعاد، وسمعان أخاه في الجليل أمام بطليموس،
\par 56 سمع يوسف بن زكريا وعزريا، قائدا الحاميات، بالأعمال الباسلة والأعمال الحربية التي صنعوها
\par 57 لذلك قالوا: لنأخذ لأنفسنا اسمًا أيضًا، ونذهب لمحاربة الأمم التي حولنا
\par 58 فلما سلموا أمرهم إلى الحامية التي معهم، ذهبوا نحو يمنيا
\par 59 ثم خرج جورجياس ورجاله من المدينة لمحاربتهم.
\par 60 فهزم يوسف وعزرا وتبعا إلى حدود اليهودية. فقتل في ذلك اليوم من شعب إسرائيل نحو ألفي رجل.
\par 61 وهكذا حدثت انقلابة عظيمة بين بني إسرائيل، لأنهم لم يطيعوا يهوذا وإخوته، بل ظنوا أنهم يقومون بعمل شجاع
\par 62 علاوة على ذلك، لم يكن هؤلاء الرجال من نسل أولئك الذين على أيديهم أعطيت الخلاص لإسرائيل.
\par 63 على أن الرجل يهوذا وإخوته كانوا مشهورين جدًا في عيون كل إسرائيل وجميع الأمم حيثما سمع اسمهم
\par 64 حتى اجتمع إليهم الناس بالهتاف
\par 65 بعد ذلك خرج يهوذا مع إخوته، وحارب بني عيسو في الأرض نحو الجنوب، حيث ضرب حبرون ومدنها، وهدم حصنها، وأحرق أبراجها المحيطة
\par 66 ومن هناك انتقل ليذهب إلى أرض الفلسطينيين، وعبر السامرة
\par 67 في ذلك الوقت، قُتل بعض الكهنة في المعركة، راغبين في إظهار شجاعتهم، لأنهم خرجوا للقتال دون مشورة
\par 68 فتوجه يهوذا إلى أشدود في أرض الفلسطينيين، وبعد أن هدم مذابحهم، وأحرق تماثيلهم المنحوتة بالنار، ونهب مدنهم، رجع إلى أرض اليهودية

\chapter{6}

\par 1 في ذلك الوقت، سمع الملك أنطيوخس أثناء سفره عبر البلاد المرتفعة، أن إليمايس في بلاد فارس كانت مدينة مشهورة جدًا بالثروة والفضة والذهب؛
\par 2 وكان فيها معبد غني جدًا، فيه أغطية من ذهب، ودروع، ودروع، تركها هناك الإسكندر، ابن فيليب، الملك المقدوني، الذي كان أول من حكم بين اليونانيين
\par 3 لذلك جاء وسعى إلى الاستيلاء على المدينة ونهبها، لكنه لم يستطع، لأن أهل المدينة، بعد أن أُنذروا بذلك،
\par 4 قام عليه في المعركة، فهرب وانصرف من هناك بثقل عظيم، ورجع إلى بابل
\par 5 ثم جاء من جاءه بخبر إلى بلاد فارس، أن الجيوش التي كانت متجهة ضد أرض يهودا قد هُزمت:
\par 6 وأن ليسياس، الذي خرج أولاً بجيش عظيم، طُرد من قبل اليهود، وأنهم تقووا بفضل الدروع والقوة والغنائم التي حصلوا عليها من الجيوش التي أهلكوها
\par 7 وأنهم هدموا أيضًا الرجس الذي أقامه على المذبح في أورشليم، وأحاطوا بالقدس بأسوار عالية كما في السابق، وبمدينته بيت صور
\par 8 فلما سمع الملك هذه الكلمات، دهش وتأثر بشدة، فوضعه على فراشه، ومرض حزنًا، لأنه لم يصبه ما كان يتوقع
\par 9 ومكث هناك أيامًا كثيرة، لأن حزنه كان يزداد أكثر فأكثر، وكان يحسب أنه سيموت
\par 10 لذلك دعا جميع أصدقائه، وقال لهم: لقد ذهب النوم من عيني، وفجأة سقط قلبي من شدة الهم
\par 11 وفكرت في نفسي: إلى أي ضيق أتيت، وما أعظم طوفان البؤس الذي أنا فيه الآن! لأني كنت كريمًا ومحبوبًا في قدرتي
\par 12 ولكن الآن تذكرت الشرور التي فعلتها في أورشليم، وأني أخذت كل آنية الذهب والفضة التي كانت فيها، وأرسلت لإبادة سكان اليهودية بلا سبب.
\par 13 فأرى أنه لهذا السبب حلت بي هذه البلاءات، وها أنا أهلك من حزن عظيم في أرض غريبة
\par 14 ثم دعا فيليب، أحد أصدقائه، الذي جعله حاكمًا على كل مملكته،
\par 15 وأعطاه التاج، وردائه، وخاتمه، لكي يربي ابنه أنطيوخس، ويربيه للملكوت
\par 16 فمات الملك أنطيوخس هناك في السنة المئة والتاسعة والأربعين.
\par 17 ولما علم ليسياس أن الملك قد مات، أقام أنطيوخس ابنه الذي رباه وهو صغير ملكاً عوضاً عنه، ودعا اسمه أوباتور.
\par 18 في ذلك الوقت، حاصر الذين كانوا في البرج بني إسرائيل حول المقدس، وكانوا يسعون دائمًا لإيذائهم وتقوية الأمم
\par 19 لذلك، أراد يهوذا إهلاكهم، فجمع كل الشعب لمحاصرتهم
\par 20 فاجتمعوا وحاصروهم في السنة المائة والخمسين، وصنع لهم دوابًا للرماية ومجانيق أخرى
\par 21 فخرج بعض الذين حوصروا، فانضم إليهم رجال فجار من إسرائيل
\par 22 فذهبوا إلى الملك وقالوا: إلى متى لا تُجري حكمًا وتنتقم لإخوتنا؟
\par 23 لقد كنا على استعداد لخدمة والدك، وفعل ما يريد منا، وطاعة وصاياه؛
\par 24 ولهذا السبب يحاصر أبناء أمتنا البرج، وينفصلون عنا: علاوة على ذلك، فقد قتلوا منا كل من استطاعوا مهاجمته، ونهبوا ميراثنا
\par 25 ولم يمدوا أيديهم علينا فقط، بل على تخومهم أيضًا
\par 26 وها هم اليوم يحاصرون برج أورشليم ليستولوا عليه، وقد حصنوا المقدس وبيت صور
\par 27 لذلك، إن لم تمنعهم بسرعة، فسيفعلون أشياءً أعظم من هذه، ولن تتمكن من السيطرة عليهم
\par 28 فلما سمع الملك ذلك، غضب، وجمع كل أصدقائه وقادة جيشه وحراس الخيل
\par 29 وجاءت إليه أيضًا من ممالك أخرى، ومن جزر البحر، جيوش من الجنود المأجورين
\par 30 فكان عدد جيشه مائة ألف راجل، وعشرين ألف فارس، واثنين وثلاثين فيلًا مدربًا على القتال
\par 31 مر هؤلاء عبر أدوم، ونزلوا على بيت صور، التي هاجموها أيامًا عديدة، وصنعوا مجانيق حربية؛ لكن أهل بيت صور خرجوا، وأحرقوها بالنار، وقاتلوا بشجاعة
\par 32 عند ذلك، نزل يهوذا من البرج، ونزل في بثزاكريا، مقابل معسكر الملك
\par 33 ثم قام الملك مبكرا جدا وسار بجيشه نحو بثزاكريا، حيث أعدت جيوشه للقتال، ونفخوا في الأبواق.
\par 34 وحتى يتمكنوا من استفزاز الفيلة للقتال، أروهم دم العنب والتوت
\par 35 علاوة على ذلك، قاموا بتقسيم الحيوانات بين الجيوش، وعينوا لكل فيل ألف رجل، مسلحين بدروع وخوذات من نحاس على رؤوسهم؛ بالإضافة إلى ذلك، تم تعيين خمسمائة فارس من أفضل الفرسان لكل حيوان
\par 36 كانوا مستعدين في كل مناسبة: حيثما كان الوحش، وحيثما ذهب الوحش، ذهبوا أيضًا، ولم يفارقوه
\par 37 وعلى الحيوانات أبراج خشبية قوية، تغطي كل واحدة منها، ومُثبتة عليها بأدوات. وكان على كل واحدة منها اثنان وثلاثون رجلاً أقوياء، يقاتلون عليها، إلى جانب الهندي الذي كان يحكمه
\par 38 وأما بقية الفرسان، فقد جعلوهم على هذا الجانب وعلى ذاك الجانب في طرفي الجيش، وأعطوهم العلامات على ما يجب عليهم فعله، وكانوا مُسَخَّرين في كل مكان وسط الصفوف
\par 39 عندما أشرقت الشمس على دروع الذهب والنحاس، تلألأت الجبال بها، وأشرقت كمصابيح من نار
\par 40 وهكذا، بعد أن انتشر جزء من جيش الملك على الجبال العالية، وجزء آخر على الوديان في الأسفل، ساروا بأمان وبنظام
\par 41 لذلك، تحرك كل من سمع ضجيج جمعهم، وزحف الفرقة، وخشخشة السرج، لأن الجيش كان عظيمًا وقويًا جدًا
\par 42 ثم تقدم يهوذا وجيشه، ودخلوا في المعركة، فقتل من جيش الملك ستمائة رجل
\par 43 وأدرك أليعازار أيضًا، الملقب بسواران، أن أحد الحيوانات، المسلح بزمام ملكي، كان أعلى من البقية، وافترض أن الملك كان فوقه،
\par 44 وضع نفسه في خطر، حتى يتمكن من إنقاذ شعبه، والحصول على اسم دائم:
\par 45 لذلك ركض نحوه بشجاعة في وسط المعركة، قاتلًا عن يمينه وعن يساره، حتى انفصلوا عنه من كلا الجانبين
\par 46 بعد أن فعل ذلك، زحف تحت الفيل، ودفعه تحته، وقتله، وعندها سقط الفيل عليه، ومات هناك
\par 47 وأما بقية اليهود فلما رأوا قوة الملك وعنف جيوشه، انصرفوا عنهم
\par 48 ثم صعد جيش الملك إلى أورشليم لملاقاتهم، ونصب الملك خيامه على اليهودية وعلى جبل صهيون
\par 49 وأما الذين في بيت صور، فصنع صلحًا، لأنهم خرجوا من المدينة، لأنه لم يكن لديهم طعام هناك ليتحملوا الحصار، إذ كانت سنة راحة للأرض
\par 50 فأخذ الملك بيت صور، وأقام هناك حامية لحمايتها.
\par 51 وأما الحرم فحاصره أياماً كثيرة، وأقام هناك مدافع ومدافع وأدوات لإلقاء النار والحجارة، وقطعاً لإلقاء السهام والمقاليع.
\par 52 وعندها صنعوا أيضًا محركات ضد محركاتهم، وخاضوا معهم معركة طويلة
\par 53 ولكن في النهاية، إذ كانت آنيتهم ​​خالية من الطعام، (لأنها كانت السنة السابعة، وكان الذين تحرروا من الأمم في اليهودية قد أكلوا ما فضل من المؤن)؛
\par 54 لم يبقَ في الحرم إلا عدد قليل، لأن المجاعة غلبت عليهم، فاضطروا إلى التفرق، كل واحد إلى مكانه
\par 55 في ذلك الوقت سمع ليسياس أن فيليب، الذي عيّنه الملك أنطيوخس، وهو حي، لتربية ابنه أنطيوخس، ليكون ملكًا،
\par 56 عاد من بلاد فارس ومادي، وكذلك جيش الملك الذي ذهب معه، وأنه طلب أن يتولى إدارة الأمور
\par 57 لذلك ذهب مسرعًا، وقال للملك وقادة الجيش والجيش: نحن ننحط كل يوم، ومؤونتنا قليلة، والمكان الذي نحاصره حصين، وشؤون المملكة تقع علينا
\par 58 والآن فلنكن أصدقاء لهؤلاء الرجال، ونصنع السلام معهم ومع كل أمتهم؛
\par 59 ونعاهدهم أن يعيشوا حسب شرائعهم كما كانوا يفعلون من قبل، لأنهم لذلك مستاؤون، وفعلوا كل هذه الأشياء، لأننا ألغينا شرائعهم
\par 60 فرضَ الملك والأمراء، فأرسل إليهم ليُصْلِح، فقبلوا
\par 61 وأقسم لهم الملك والرؤساء، فخرجوا من الحصن
\par 62 ثم دخل الملك جبل صهيون، ولكن لما رأى حصن المكان، نكث يمينه الذي قطعه، وأمر بهدم السور المحيط به
\par 63 بعد ذلك غادر مسرعًا وعاد إلى أنطاكية، حيث وجد فيليب حاكمًا للمدينة، فحاربه واستولى على المدينة بالقوة

\chapter{7}

\par 1 في السنة المائة والحادية والخمسين، غادر ديمتريوس بن سلوقس رومية، وصعد مع بضعة رجال إلى مدينة على ساحل البحر، وملك هناك
\par 2 ولما دخل قصر آبائه، كانت قواته قد أخذت أنطيوخس وليسياس لتأتي بهما إليه
\par 3 لذلك، عندما علم بذلك، قال: لا أرى وجوههم.
\par 4 فقتلهم جيشه. ولما اعتلى ديمتريوس عرش مملكته،
\par 5 فجاء إليه جميع رجال إسرائيل الأشرار والفجار، وكان ألكيمس، الذي كان يرغب في أن يكون رئيس كهنة، رئيسًا لهم
\par 6 واشتكوا على الشعب لدى الملك قائلين: إن يهوذا وإخوته قتلوا جميع أصدقائك وطردونا من أرضنا
\par 7 فالآن أرسل رجلاً تثق به، ليذهب ويرى ما أحدثه من خراب بيننا وفي أرض الملك، وليعاقبهم مع جميع مساعديهم.
\par 8 ثم اختار الملك بكيديس، صديق الملك، الذي حكم بعد الطوفان، وكان رجلاً عظيماً في المملكة، ومخلصاً للملك،
\par 9 فأرسله مع ذلك الشرير ألكيمس، الذي جعله رئيس كهنة، وأمره بالانتقام من بني إسرائيل
\par 10 فانطلقوا وجاءوا بجيش عظيم إلى أرض اليهودية، حيث أرسلوا رسلاً إلى يهوذا وإخوته بكلام سلامٍ مخادع
\par 11 لكنهم لم يُصغوا إلى كلامهم، لأنهم رأوا أنهم قد أتوا بقوة عظيمة
\par 12 ثم اجتمعت جماعة من الكتبة إلى ألكيموس وبكيديس للمطالبة بالعدل
\par 13 وكان الأسيديون أول من طلب السلام من بني إسرائيل:
\par 14 فإنهم قالوا: إن كاهنًا من نسل هارون قد جاء مع هذا الجيش، ولن يسيء إلينا
\par 15 فكلمهم سلمًا، وأقسم لهم قائلًا: لن نُلحق بكم ولا بأصدقائكم الأذى
\par 16 فآمنوا به، فأخذ منهم ستين رجلاً وقتلهم في يوم واحد، كما كتب،
\par 17 لقد طردوا أجساد قديسيك، وسفكوا دمائهم حول أورشليم، ولم يكن من يدفنهم
\par 18 فوقع خوفهم ورعبهم على جميع الشعب، الذين قالوا: ليس فيهم حق ولا بر، لأنهم نكثوا العهد والقسم الذي قطعوه
\par 19 بعد ذلك، أخرج بكيديس من أورشليم، ونصب خيامه في بازيت، حيث أرسل وأخذ كثيرين من الرجال الذين تركوه، وبعضًا من الشعب أيضًا، وبعد أن قتلهم، ألقاهم في الجب العظيم
\par 20 ثم سلم البلاد إلى ألكيموس، وترك معه قوة لمساعدته: فذهب بكيديس إلى الملك
\par 21 لكن ألكيموس ناضل من أجل رئاسة الكهنوت.
\par 22 وإليه لجأ كل من أزعج الشعب، والذين بعد أن استولوا على أرض يهوذا، أحدثوا شروراً كثيرة في إسرائيل.
\par 23 ولما رأى يهوذا كل الشر الذي فعله ألكيمس وجماعته بين بني إسرائيل، حتى فوق الأمم،
\par 24 وخرج إلى جميع حدود اليهودية المحيطة، وانتقم من الذين ثاروا عليه، حتى لم يجرؤوا بعد على الخروج إلى البلاد
\par 25 على الجانب الآخر، عندما رأى ألكيموس أن يهوذا وجماعته قد انتصروا، وعلم أنه لا يستطيع تحمل قوتهم، ذهب مرة أخرى إلى الملك، وقال أسوأ ما استطاع منهم
\par 26 ثم أرسل الملك نكانور، أحد أمرائه الكرام، وهو رجل يحمل بغضًا مميتًا لإسرائيل، وأمر بإهلاك الشعب
\par 27 فجاء نكانور إلى أورشليم بجيش عظيم وأرسل إلى يهوذا وإخوته كلاما ودودا مكراً قائلا:
\par 28 لا تكن معركة بيني وبينكم؛ سآتي مع بضعة رجال، لأتمكن من رؤيتكم بسلام
\par 29 فجاء إلى يهوذا، وسلما بعضهما على بعض بسلام. إلا أن الأعداء كانوا مستعدين لأخذ يهوذا بالعنف
\par 30 وهذا الأمر بعد أن علم يهوذا أنه جاء إليه بمكر، خاف منه خوفًا شديدًا، ولم يعد يرى وجهه بعد ذلك
\par 31 ولما رأى نكانور أيضًا أن مشورته قد انكشفت، خرج ليقاتل يهوذا بجانب كفر سلامة
\par 32 حيث قُتل من جانب نكانور نحو خمسة آلاف رجل، وهرب الباقون إلى مدينة داود
\par 33 بعد هذا صعد نكانور إلى جبل صهيون، فخرج من المقدس بعض الكهنة وبعض شيوخ الشعب، ليحيوه بسلام، وليُروه المحرقة التي قُدِّمت عن الملك
\par 34 لكنه سخر منهم، وضحك عليهم، وشتمهم بشتم، وتكلم بغطرسة،
\par 35 وأقسم في غضبه قائلاً: إن لم يُسلم يهوذا وجيشه الآن إلى يدي، فإني إن رجعت سالماً سأحرق هذا البيت. ومع ذلك خرج في غضب شديد
\par 36 فدخل الكهنة ووقفوا أمام المذبح والهيكل وهم يبكون قائلين:
\par 37 أنت يا رب، اخترت هذا البيت ليُدعى باسمك، وليكون بيت صلاة ودعاء لشعبك
\par 38 انتقم من هذا الرجل وجيشه، وليسقطوا بالسيف. تذكر تجديفاتهم، ولا تدعهم يستمرون بعد الآن
\par 39 فخرج نكانور من أورشليم ونصب خيامه في بيت حورون، حيث استقبله جيش من سورية
\par 40 فنزل يهوذا في أداسة بثلاثة آلاف رجل، وصلى هناك قائلاً:
\par 41 يا رب، عندما جدف المرسلون من ملك آشور، خرج ملاكك وضرب منهم مئة وخمسة وثمانين ألفًا
\par 42 هكذا دمر هذا الجيش أمامنا اليوم، لكي يعلم الباقون أنه تكلم بتجديف على مقدسك، وتحكم عليه حسب شره
\par 43 وفي اليوم الثالث عشر من شهر أذار، تقاتلت الجيوش، ولكن جيش نيكانور انهزم، وقُتل هو نفسه أولًا في المعركة
\par 44 فلما رأى جيش نيكانور أنه قد قُتل، ألقوا أسلحتهم وهربوا
\par 45 ثم طاردوهم مسيرة يوم واحد من أداسة إلى جزيرة، وهم ينفخون في أبواقهم في أعقابهم
\par 46 فخرجوا من جميع مدن اليهودية المحيطة، وحاصروهم، فانقلبوا على الذين طاردوهم، فقتلوا جميعًا بالسيف، ولم يبق منهم أحد
\par 47 وبعد ذلك أخذوا الغنائم والغنائم، وضربوا رأس نيكانور ويده اليمنى التي مدها بغطرسة، وجاءوا بهما وعلقوهما نحو أورشليم.
\par 48 لهذا السبب فرح الشعب فرحًا عظيمًا، وجعلوا ذلك اليوم يوم فرح عظيم
\par 49 علاوة على ذلك، فقد رسموا أن يُحتفل بهذا اليوم سنويًا، وهو الثالث عشر من شهر أدار
\par 50 وهكذا استراحت أرض يهوذا قليلًا.

\chapter{8}

\par 1 كان يهوذا قد سمع عن الرومان، أنهم رجال أقوياء وشجعان، وأنهم يقبلون بمحبة كل من ينضم إليهم، ويعقدون عهد صداقة مع كل من يأتي إليهم؛
\par 2 وأنهم كانوا رجالاً ذوي شجاعة عظيمة. وأُخبر أيضًا عن حروبهم وأعمالهم النبيلة التي قاموا بها بين الغلاطيين، وكيف قهروهم، وأخضعوهم للجزية؛
\par 3 وما فعلوه في بلاد إسبانيا، من أجل الاستيلاء على مناجم الفضة والذهب الموجودة هناك؛
\par 4 وأنهم بفضل سياستهم وصبرهم غزوا كل مكان، مع أنه كان بعيدًا عنهم جدًا؛ وكذلك الملوك الذين هاجموهم من أقصى الأرض، حتى هزموهم، وألحقوا بهم هزيمة ساحقة، حتى أن الباقين كانوا يدفعون لهم الجزية كل عام
\par 5 إلى جانب ذلك، كيف هزموا فيليب، وبرسيوس، ملك السيتيم، في المعركة، مع آخرين ثاروا عليهم، وتغلبوا عليهم:
\par 6 كيف هزمهم أيضًا أنطيوخس ملك آسيا العظيم، الذي جاء لمواجهتهم في معركة، وكان معه مئة وعشرون فيلاً وفرسان ومركبات وجيش عظيم جدًا؛
\par 7 وكيف أمسكوه حيًا، وعاهدوه على أن يدفع هو ومن ملك بعده جزية عظيمة، ويعطوا رهائن، وما اتفق عليه،
\par 8 وبلاد الهند ومادي وليديا ومن خير البلاد التي أخذوها منه وأعطوها للملك يومينس:
\par 9 علاوة على ذلك، كيف قرر اليونانيون المجيء وتدميرهم؛
\par 10 وأنهم لما علموا بذلك أرسلوا إليهم قائدا فقاتلهم فقتل منهم كثيرين وسبوا نسائهم وأولادهم ونهبهم واستولوا على أراضيهم وهدموا حصونهم وأتوا بهم عبيدا إلى هذا اليوم.
\par 11 قيل له أيضًا كيف دمروا وأخضعوا جميع الممالك والجزر الأخرى التي قاومتهم في أي وقت؛
\par 12 لكنهم حافظوا على صداقتهم مع أصدقائهم ومن اعتمد عليهم، وأنهم غزوا ممالك بعيدة وقريبة، حتى أن كل من سمع باسمهم خاف منهم
\par 13 وأيضًا، فإن من أرادوا مساعدته في إقامة مملكة، فإنهم يحكمون؛ ومن أرادوا مساعدته مرة أخرى، فإنهم يخلعونه؛ وأخيرًا، فإنهم قد رُفعوا إلى أعلى درجات التعظيم:
\par 14 ومع كل هذا لم يلبس أحد منهم تاجًا أو ثوبًا أرجوانيًا لكي يتم تكبيرهم بذلك:
\par 15 علاوة على ذلك، كيف أنشأوا لأنفسهم مجلس شيوخ، حيث كان ثلاثمائة وعشرون رجلاً يجلسون في المجلس يوميًا، ويتشاورون دائمًا من أجل الشعب، حتى يتمكنوا من التنظيم بشكل جيد:
\par 16 وأنهم ألزموا حكمهم لرجل واحد كل عام، يحكم كل بلادهم، وأن الجميع كانوا مطيعين لذلك الرجل، وأنه لم يكن بينهم حسد ولا منافسة
\par 17 في ضوء هذه الأمور، اختار يهوذا يوبوليموس بن يوحنا بن أكوس، وياسون بن ألعازار، وأرسلهما إلى روما، لعقد حلف صداقة وتحالف معهم،
\par 18 ويتوسلون إليهم أن يرفعوا عنهم النير، لأنهم رأوا أن مملكة اليونانيين تضطهد إسرائيل بالعبودية
\par 19 فذهبوا إلى روما، وكانت رحلة طويلة جدًا، ودخلوا مجلس الشيوخ، حيث تحدثوا وقالوا
\par 20 أرسلنا إليكم يهوذا المكابي مع إخوته وشعب اليهود، لنعقد تحالفًا وسلامًا معكم، ولنُسجل في عداد حلفاءكم وأصدقاءكم
\par 21 لذا فقد رضي الرومان عن هذا الأمر.
\par 22 وهذه هي نسخة الرسالة التي كتبها مجلس الشيوخ أيضًا على ألواح من نحاس، وأرسلها إلى أورشليم، حتى يكون لهم هناك تذكارًا للسلام والاتحاد:
\par 23 النجاح للرومان ولشعب اليهود، بحرًا وبرًا إلى الأبد. وليبتعد عنهم السيف والعدو،
\par 24 إذا اندلعت أولًا أي حرب على الرومان أو أي من حلفائهم في جميع أنحاء سيطرتهم،
\par 25 سيساعدهم شعب اليهود، كما يُعيَّن الوقت، بكل قلوبهم
\par 26 ولا يجوز لهم أن يعطوا شيئًا لمن يحاربهم، أو يساعدوهم بمؤن أو أسلحة أو أموال أو سفن، كما رأى الرومان، بل يجب عليهم أن يوفوا بعهودهم دون أن يأخذوا شيئًا من ذلك
\par 27 وعلى النحو نفسه أيضًا، إذا اندلعت حرب أولًا على أمة اليهود، فسيساعدهم الرومان بكل قلوبهم، وفقًا للوقت المحدد لهم:
\par 28 ولا يُعطى طعامٌ لمن يحاربهم، ولا سلاحٌ، ولا مالٌ، ولا سفنٌ، كما استحسن الرومان، بل يُوفون بعهودهم، وذلك بغير غش
\par 29 وفقًا لهذه المواد، أبرم الرومان عهدًا مع شعب اليهود
\par 30 ومع ذلك، إذا فكر أحد الطرفين في الاجتماع لاحقًا لإضافة أو تقليل أي شيء، فيجوز لهما القيام بذلك حسب رغبتهما، وسيتم التصديق على كل ما يضيفانه أو ينقصانه
\par 31 وأما فيما يتعلق بالشرور التي يفعلها ديمتريوس باليهود، فقد كتبنا إليه قائلين: لماذا جعلت نيرك ثقيلاً على أصدقائنا وحلفائنا اليهود؟
\par 32 فإن اشتكوا عليك بعد ذلك فإننا سننصفهم ونقاتلك بحرا وبراً.

\chapter{9}

\par 1 علاوة على ذلك، عندما سمع ديمتريوس بمقتل نيكانور وجيشه في المعركة، أرسل بكيديس وألكيموس إلى أرض يهودا للمرة الثانية، ومعهما القوة الرئيسية لجيشه:
\par 2 الذين خرجوا من الطريق المؤدية إلى جلجالا، ونصبوا خيامهم أمام ماسالوت التي في أربيل، وبعد أن استولوا عليها، قتلوا أناسًا كثيرين
\par 3 وفي الشهر الأول من السنة المئة والثانية والخمسين نزلوا أمام أورشليم:
\par 4 انطلقوا من هناك، وذهبوا إلى بيريا، مع عشرين ألف راجل وألفي فارس
\par 5 وكان يهوذا قد نصب خيامه في إلياسة، ومعه ثلاثة آلاف رجل مختار
\par 6 الذين لما رأوا حشد الجيش الآخر عظيمًا جدًا، خافوا خوفًا شديدًا، فغادر كثيرون الجيش، ولم يبقَ بينهم سوى ثمانمائة رجل
\par 7 فلما رأى يهوذا أن جيشه قد انسحب، وأن المعركة تضغط عليه، اضطرب قلبه بشدة، واغتمّ كثيرًا، لأنه لم يكن لديه وقت لجمعهم
\par 8 ومع ذلك قال للذين بقوا: لنقم ونصعد على أعدائنا، لعلنا نستطيع أن نحاربهم
\par 9 لكنهم رفضوه قائلين: لن نقدر أبدًا. فلننجِ الآن بالحري أنفسنا، وبعد ذلك نعود مع إخوتنا ونقاتلهم، لأننا قليلون
\par 10 فقال يهوذا: حاشا لي أن أفعل هذا الأمر وأهرب منهم. إن جاء أجلنا فلنمت بشجاعة من أجل إخوتنا ولا نلوث شرفنا
\par 11 عند ذلك، خرج جيش بكيديس من خيامه، ووقف في مواجهتهم، وانقسم فرسانهم إلى فرقتين، وكان رماة المقاليع والسهام يسيرون أمام الجيش، وكان السائرون في المقدمة جميعًا رجالًا أقوياء
\par 12 وأما بكيديس فكان في الجناح الأيمن، فاقترب الجيش من الطرفين ونفخوا في أبواقهم
\par 13 ونفخ رجال يهوذا أيضًا في أبواقهم، حتى اهتزت الأرض من صوت الجيوش، واستمرت المعركة من الصباح إلى الليل
\par 14 ولما رأى يهوذا أن بكيديس وقوة جيشه في الجانب الأيمن، أخذ معه جميع الرجال الأشداء،
\par 15 الذي هزم الجناح الأيمن، وطاردهم إلى جبل أشدود
\par 16 فلما رأى رجال اليسار أن رجال اليمين قد انهزموا، لحقوا بيهوذا والذين معه من خلفهم
\par 17 فقامت معركة شرسة، قُتل فيها كثيرون من كلا الجانبين
\par 18 قُتل يهوذا أيضًا، وهرب الباقون.
\par 19 ثم أخذ يوناتان وسمعان يهوذا أخاهما ودفناه في قبر آبائه في مودين.
\par 20 ثم ندبوه، وناح عليه كل إسرائيل حزنا عظيما، وندبوا أياما كثيرة قائلين:
\par 21 كيف سقط الرجل الشجاع الذي خلص إسرائيل!
\par 22 وأما سائر الأمور المتعلقة بيهوذا وحروبه والأعمال النبيلة التي صنعها وعظمته فلم تكتب لأنها كانت كثيرة جداً.
\par 23 وبعد موت يهوذا، بدأ الأشرار يُخرجون رؤوسهم في جميع تخوم إسرائيل، وقام كل من فعل الإثم
\par 24 في تلك الأيام أيضًا كانت هناك مجاعة شديدة جدًا، مما أدى إلى تمرد البلاد ورحيلها
\par 25 ثم اختار بكيديس الرجال الأشرار، وجعلهم أسيادًا على البلاد
\par 26 فبحثوا وبحثوا عن أصدقاء يهوذا، وأتوا بهم إلى بكيديس، فانتقم منهم واستغلهم شرًا
\par 27 فكانت بلاء عظيم في إسرائيل، لم يحدث مثله منذ أن لم يُرَ نبيٌّ بينهم
\par 28 لهذا السبب اجتمع جميع أصدقاء يهوذا وقالوا ليوناثان:
\par 29 منذ أن مات أخوك يهوذا، ليس لدينا رجل مثله ليخرج ضد أعدائنا، وضد بكيديس، وضد أعداء أمتنا
\par 30 لذلك اخترناك اليوم لتكون أميرنا وقائدنا بدلاً منه، حتى تتمكن من خوض معاركنا
\par 31 عند ذلك، تولى يوناثان الحكم في ذلك الوقت، وقام بدلاً من أخيه يهوذا
\par 32 ولكن عندما علم بكيديس بذلك، سعى لقتله
\par 33 فعرف يوناتان وسمعان أخوه وكل الذين معه ذلك، فهربوا إلى برية تيكوي ونزلوا عند ماء بركة أسفار.
\par 34 فلما علم بكيديس، تقدم إلى الأردن مع كل جيشه يوم السبت
\par 35 وكان يوناثان قد أرسل أخاه يوحنا، قائد الشعب، ليطلب من أصدقائه النبطيين أن يتركوا معهم مركبتهم، وكانت كثيرة
\par 36 فخرج بنو يمبري من ميدابا، وأخذوا يوحنا وكل ما كان له، ومضوا به
\par 37 بعد ذلك، وصل خبر إلى يوناثان وسمعان أخاه، أن أبناء يمري أقاموا زواجًا عظيمًا، وكانوا يحضرون العروس من ناداباتا بقافلة عظيمة، لأنها ابنة أحد أمراء كنعان العظام
\par 38 فتذكروا يوحنا أخاهم، فصعدوا واختبأوا تحت ستر الجبل
\par 39 حيث رفعوا أعينهم ونظروا، وإذا ضجيج كثير ومركبة عظيمة، وخرج العريس وأصدقاؤه وإخوته للقائهم بالطبول وآلات الموسيقى والأسلحة الكثيرة
\par 40 فقام يوناتان والذين معه عليهم من مكمنهم وضربوهم ضربا مبرحا فسقط كثيرون قتلى وهرب الباقون إلى الجبل وأخذوا كل غنائمهم.
\par 41 وهكذا تحول الزواج إلى حزن، وضجيج لحنهم إلى رثاء
\par 42 فلما انتقموا لدم أخيهم، رجعوا إلى غور الأردن
\par 43 فلما سمع بكيديس بذلك، جاء يوم السبت إلى ضفاف الأردن بجيش عظيم
\par 44 ثم قال يوناثان لجماعته: لنصعد الآن ونقاتل من أجل أنفسنا، لأنها ليست قائمة معنا اليوم كما في الأمس
\par 45 فها هي المعركة أمامنا وخلفنا، ومياه الأردن من هنا ومن هناك، والمستنقع والغابة كذلك، وليس لنا مكان ننحرف فيه
\par 46 لذلك اصرخوا الآن إلى السماء، لكي تنقذوا من أيدي أعدائكم
\par 47 عند ذلك انضموا إلى المعركة، ومد يوناثان يده ليضرب بكيديس، لكنه تراجع عنه
\par 48 فقفز يوناثان والذين معه إلى الأردن وسبحوا إلى الضفة الأخرى، إلا أن الضفة الأخرى لم تعبر الأردن إليهم
\par 49 فقُتل من صف بكيديس في ذلك اليوم نحو ألف رجل
\par 50 بعد ذلك، عاد بكيديس إلى أورشليم، وأصلح المدن الحصينة في اليهودية؛ حصن أريحا، وعماوس، وبيت حورون، وبيت إيل، وتمناتا، وفرثوني، وتافون، وقد حصنها بأسوار عالية وأبواب ومزاليج
\par 51 وجعل فيهم حامية لكي يعملوا الشر على إسرائيل
\par 52 وحصن أيضًا مدينة بيت صور وجازيرا والبرج، ووضع فيها قوات ومؤنًا
\par 53 إلى جانب ذلك، أخذ أبناء زعماء البلاد رهائن، ووضعهم في برج أورشليم للحراسة
\par 54 وفي السنة المائة والثالثة والخمسين، في الشهر الثاني، أمر ألكيمس بهدم جدار دار المقدس الداخلية، وهدم أيضًا أعمال الأنبياء
\par 55 وبينما بدأ في الهدم، كان ألكيموس في ذلك الوقت مصابًا، وتعطلت أعماله: إذ كان فمه مسدودًا، وأُصيب بالشلل، حتى إنه لم يعد قادرًا على التكلم بأي شيء، ولا على إعطاء أمر بشأن بيته
\par 56 فمات ألكيموس في ذلك الوقت بعذاب عظيم.
\par 57 ولما رأى بكيديس أن ألكيموس قد مات رجع إلى الملك، فاستحسنت أرض اليهودية سنتين.
\par 58 ثم عقد جميع الرجال الأشرار مجلسًا، قائلين: هوذا يوناثان وجماعته مرتاحون، ويقيمون في أمان. والآن سنحضر بكيديس إلى هنا، فيقبض عليهم جميعًا في ليلة واحدة
\par 59 فذهبوا واستشاروه.
\par 60 ثم انطلق وجاء في جيش عظيم وأرسل رسائل سرية إلى أتباعه في اليهودية أن يقبضوا على يوناثان والذين معه. ولكنهم لم يستطيعوا لأن مشورتهم كانت معروفة لديهم.
\par 61 لذلك أخذوا من رجال البلد، الذين كانوا سببًا في تلك الشرور، حوالي خمسين شخصًا، وقتلوهم
\par 62 بعد ذلك، ذهب يوناثان وسمعان والذين معه إلى بيت حشيش التي في البرية، فرمموا خرائبها وقويوها
\par 63 فلما علم بكيديس بذلك، جمع كل جيشه وأرسل إلى أهل اليهودية
\par 64 ثم ذهب وحاصر بيت حاشي، وحاربوها مدة طويلة وصنعوا المجانيق
\par 65 أما يوناثان فترك أخاه سمعان في المدينة، وخرج هو إلى الريف، وخرج مع عدد من الناس
\par 66 وضرب أودوناركس وإخوته وبني فاسيرون في خيمتهم
\par 67 ولما ابتدأ يضربهم، وصعد بجيوشه، خرج سمعان وجماعته من المدينة، وأحرقوا المجانيق،
\par 68 وحاربوا بكيديس، الذي هزمه، وأذلوه بشدة: لأن مشورته وجهده كانا باطلين
\par 69 لذلك كان غاضبًا جدًا من الأشرار الذين نصحوه بالقدوم إلى البلاد، حيث قتل الكثير منهم، وكان ينوي العودة إلى بلده
\par 70 ولما علم يوناثان بذلك، أرسل إليه سفراء لكي يُصالحه ويُسلم إليه الأسرى
\par 71 فقبل ذلك، وفعل حسب مطالبه، وأقسم له أنه لن يؤذيه كل أيام حياته
\par 72 وبعد أن رد إليه الأسرى الذين سبق أن أسرهم من أرض اليهودية، رجع ومضى إلى أرضه، ولم يعد يدخل تخومهم
\par 73 وهكذا كف السيف عن إسرائيل، وأقام يوناثان في مخماس، وبدأ يحكم الشعب، وأباد الرجال الأشرار من إسرائيل

\chapter{10}

\par 1 في السنة المائة والستين، صعد الإسكندر بن أنطيوخس الملقب بإبيفانيس، واستولى على بطليموس، لأن الشعب استقبله، وملك هناك،
\par 2 فلما سمع الملك ديمتريوس بذلك، جمع جيشًا عظيمًا جدًا، وخرج لمواجهته للقتال
\par 3 علاوة على ذلك، أرسل ديمتريوس رسائل إلى يوناثان بكلمات محبة، حتى أنه عظمه
\par 4 لأنه قال: فلنُصْلِحْهُ أولًا قبل أن ينضم إلى الإسكندر ضدنا
\par 5 وإلا فإنه سيتذكر كل الشرور التي فعلناها ضده وضد إخوته وشعبه
\par 6 لذلك أعطاه سلطة جمع جيش وتوفير الأسلحة حتى يتمكن من مساعدته في المعركة، وأمر أيضًا بتسليم الرهائن الذين كانوا في البرج.
\par 7 فجاء يوناثان إلى أورشليم، وقرأ الرسائل في مسامع كل الشعب، ومن في البرج
\par 8 الذين خافوا خوفًا شديدًا عندما سمعوا أن الملك قد أعطاه سلطة جمع جيش
\par 9 وعندئذٍ سلم أهل البرج رهائنهم إلى يوناثان، فسلمهم إلى والديهم
\par 10 بعد أن فعل ذلك، استقر يوناثان في أورشليم، وبدأ في بناء المدينة وإصلاحها
\par 11 وأمر العمال ببناء الأسوار وجبل صهيون وما حوله بحجارة مربعة للتحصين، ففعلوا ذلك
\par 12 ثم هرب الغرباء الذين كانوا في الحصون التي بناها بكيديس
\par 13 حتى ترك كل إنسان مكانه وذهب إلى وطنه.
\par 14 في بيت صور فقط بقي بعض أولئك الذين تركوا الشريعة والوصايا، لأنها كانت مكان ملجأهم.
\par 15 ولما سمع الملك الإسكندر بالوعود التي أرسلها ديمتريوس إلى يوناثان، وأُخبر أيضًا بالمعارك والأعمال النبيلة التي قام بها هو وإخوته، وبالآلام التي تحملوها،
\par 16 قال: هل نجد رجلاً آخر مثله؟ فالآن نجعله صديقًا لنا وشريكًا
\par 17 وبناءً على ذلك، كتب رسالة وأرسلها إليه، وفقًا لهذه الكلمات، قائلًا:
\par 18 الملك الإسكندر يرسل تحية إلى أخيه يوناثان:
\par 19 لقد سمعنا عنك أنك رجل ذو قوة عظيمة، وتستحق أن تكون صديقًا لنا.
\par 20 لذلك، في هذا اليوم، نرسمك رئيس كهنة أمتك، ونُدعى صديق الملك؛ (وأرسل إليه مع ذلك ثوبًا أرجوانيًا وتاجًا من ذهب): ونطلب منك أن تأخذ دورنا، وأن تحافظ على الصداقة معنا
\par 21 ففي الشهر السابع من السنة المئة والستين، في عيد المظال، لبس يوناثان الرداء المقدس، وجمع جيوشًا، وجهز سلاحًا كثيرًا
\par 22 فلما سمع ديمتريوس ذلك، حزن بشدة، وقال:
\par 23 ماذا فعلنا حتى منعنا الإسكندر من إقامة صداقة مع اليهود لتقوية نفسه؟
\par 24 سأكتب لهم أيضًا كلمات تشجيع، وأعدهم بكرامات وهدايا، حتى أتمكن من مساعدتهم
\par 25 فأرسل إليهم هكذا: الملك ديمتريوس يرسل سلامًا إلى شعب اليهود:
\par 26 لئن كنتم قد حفظتم عهودكم معنا، وواصلتم صداقتنا، ولم تنضموا إلى أعدائنا، فقد سمعنا بهذا ففرحنا
\par 27 لذلك، استمروا الآن في أن تكونوا مخلصين لنا، وسنكافئكم جيدًا على الأشياء التي تفعلونها من أجلنا،
\par 28 وسيمنحك حصانات كثيرة، ويمنحك مكافآت.
\par 29 والآن أحرركم، ومن أجلكم أعفي كل اليهود من الجزية، ومن عادات الملح، ومن ضرائب التاج،
\par 30 ومما يخصني أن آخذه كثلث البذر ونصف ثمر الأشجار، أُعفيه من هذا اليوم فصاعدًا، حتى لا يُؤخذ من أرض اليهودية، ولا من الأراضي الثلاث التي تُضاف إليها من بلاد السامرة والجليل، من هذا اليوم فصاعدًا إلى الأبد
\par 31 فلتكن أورشليم أيضًا مقدسة وحرة، مع حدودها، من الأعشار والجزية
\par 32 وأما البرج الذي في أورشليم، فسأسلمه إلى رئيس الكهنة، لكي يقيم فيه من يختاره من الرجال لحراسته
\par 33 علاوة على ذلك، أُطلق سراح جميع اليهود الذين سُبوا من أرض يهودا إلى أي جزء من مملكتي، وأُريد أن يُعفي جميع ضباطي من الجزية حتى من مواشيهم
\par 34 علاوة على ذلك، أريد أن تكون جميع الأعياد، والسبوت، والأهلة، والأيام الرسمية، والأيام الثلاثة التي تسبق العيد، والأيام الثلاثة التي تليها، كلها حصانة وحرية لجميع اليهود في مملكتي
\par 35 كما لا يجوز لأي رجل التدخل في أي أمر أو التحرش بأي منهم
\par 36 وأود أيضًا أن يُسجل في قوات الملك حوالي ثلاثين ألف رجل من اليهود، وتُدفع لهم رواتبهم، كما هو الحال مع جميع قوات الملك
\par 37 ويُوضع بعضهم في حصون الملك، ويُقام منهم أيضًا بعضٌ على شؤون المملكة التي هي أمانة. وأريد أن يكون رؤساؤهم وولاتهم من أنفسهم، وأن يعيشوا وفقًا لقوانينهم الخاصة، كما أمر الملك في أرض يهوذا
\par 38 وأما الحكومات الثلاث التي أُضيفت إلى اليهودية من بلاد السامرة، فلتنضم إلى اليهودية، لكي تُحسب تحت سلطة واحدة، ولا تُلزم بطاعة سلطة أخرى سوى سلطة رئيس الكهنة
\par 39 أما بطليموس والأرض التابعة لها، فأنا أعطيها هبة مجانية للمقدس في أورشليم لتغطية النفقات الضرورية للمقدس
\par 40 علاوة على ذلك، أعطي كل سنة خمسة عشر ألف شاقل من الفضة من حسابات الملك من الأماكن ذات الصلة
\par 41 وكل الفائض الذي لم يدفعه الضباط كما في السابق، يُدفع من الآن فصاعدًا لأعمال الهيكل
\par 42 وإلى جانب هذا، فإن الخمسة آلاف شاقل من الفضة التي أخذوها من نفقات الهيكل من الحسابات سنة بعد سنة، تُفرج عنها أيضًا، لأنها تخص الكهنة الذين يخدمون
\par 43 وكل من يفرّ إلى هيكل أورشليم، أو يتمتع بحرياته، سواء كان مدينًا للملك، أو لأي أمر آخر، فليكن حرًا، وكل ما يملك في مملكتي
\par 44 وأما بناء وترميم أعمال المقدس فتعطى نفقاته من حساب الملك.
\par 45 نعم، ولبناء أسوار أورشليم وتحصينها من حولها، تُدفع نفقات من حسابات الملك، وكذلك لبناء الأسوار في اليهودية
\par 46 فلما سمع يوناثان والشعب هذه الكلمات، لم يصدقوها ولم يقبلوها، لأنهم تذكروا الشر العظيم الذي صنعه بإسرائيل، لأنه أذلهم جدًا
\par 47 لكنهم كانوا راضين عن الإسكندر، لأنه كان أول من طلب السلام الحقيقي معهم، وكانوا متحالفين معه دائمًا
\par 48 ثم جمع الملك الإسكندر قوات عظيمة، وعسكر مقابل ديمتريوس
\par 49 وبعد أن اشتبك الملكان في المعركة، هرب جيش ديمتريوس، لكن الإسكندر تبعه وانتصر عليهم
\par 50 واستمر في القتال بشراسة حتى غربت الشمس، وفي ذلك اليوم قُتل ديمتريوس
\par 51 بعد ذلك، أرسل الإسكندر سفراء إلى بطليموس ملك مصر برسالة بهذا المعنى:
\par 52 بما أنني قد عدت إلى مملكتي، وتربعت على عرش أجدادي، واستوليت على السلطة، وهزمت ديمتريوس، واستعدت بلادنا؛
\par 53 لأنه بعد أن انضممت إليه في المعركة، هُزم هو وجيشه من قبلنا، حتى جلسنا على عرش مملكته
\par 54 والآن دعنا نعقد عهد صداقة معًا، ونعطيني ابنتك زوجة، وأكون صهرك، وأعطيك وإياها حسب كرامتك
\par 55 فأجاب بطليموس الملك قائلاً: ليكن يوم سعيد الذي رجعت فيه إلى أرض آبائك، وجلست على عرش مملكتهم
\par 56 والآن سأفعل بك كما كتبتَ: قابلني إذن في بطليموس لنرى بعضنا البعض، لأني سأزوجك ابنتي حسب رغبتك
\par 57 فخرج بطليموس من مصر مع ابنته كليوباترا، ووصلا إلى بطليموس في السنة الثانية والستين بعد المائة
\par 58 حيث التقى به الملك الإسكندر، أعطاه ابنته كليوباترا، واحتفل بزواجها في بطليموس بمجد عظيم، كما هي عادة الملوك
\par 59 وكان الملك الإسكندر قد كتب إلى يوناثان أن يأتي لمقابلته
\par 60 ثم ذهب بشرف إلى بطليموس، حيث التقى بالملكين، وأعطاهما وأصدقائهما فضة وذهبًا، وهدايا كثيرة، ونال حظوة في أعينهما
\par 61 في ذلك الوقت، اجتمع عليه رجالٌ مُفسدون من بني إسرائيل، رجالٌ ذوو حياةٍ شريرة، ليُتهموه، لكن الملك لم يسمع لهم
\par 62 بل أكثر من ذلك، أمر الملك بخلع ثيابه وإلباسه الأرجوان، ففعلوا ذلك
\par 63 وأجلسه وحده، وقال لرؤسائه: اذهبوا معه إلى وسط المدينة، ونادوا أن لا يتذمر عليه أحد في أمر ما، ولا يزعجه أحد في أي سبب كان.
\par 64 فلما رأى المشتكون عليه أنه قد أُكرم حسب الإعلان، وأنه لابس الأرجوان، هربوا جميعًا
\par 65 فأكرمه الملك، وكتبه من بين أصدقائه الرئيسيين، وجعله دوقًا وشريكًا في سلطته
\par 66 بعد ذلك، عاد يوناثان إلى أورشليم بسلام وفرح.
\par 67 وفي السنة المائة والخامسة والستين خرج ديمتريوس بن ديمتريوس من كريت إلى أرض آبائه.
\par 68 وعندما سمع الملك الإسكندر الخبر، حزن بشدة، وعاد إلى أنطاكية
\par 69 ثم جعل ديمتريوس أبولونيوس حاكمًا على كلسورية قائدًا له، فجمع أبولونيوس جيشًا عظيمًا ونزل في يمنيا، وأرسل إلى يوناثان رئيس الكهنة قائلًا:
\par 70 أنت وحدك من يرفع نفسه علينا، وأنا أضحك وأحتقر من أجلك، وأُوبَّخ. فلماذا تتباهى بقوتك علينا في الجبال؟
\par 71 والآن، إن كنت تثق بقوتك، فانزل إلينا إلى الحقل السهلي، وهناك دعنا نفحص الأمر معًا، لأن قوة المدن معي
\par 72 اسأل واعرف من أنا، ومن هم الباقون الذين يقفون في صفنا، وسيخبرونك أن قدمك لا تستطيع الهرب في أرضهم
\par 73 لذلك، لن تتمكن الآن من تحمل الفرسان وقوة عظيمة كهذه في السهل، حيث لا يوجد حجر ولا صوان، ولا مكان للهروب إليه
\par 74 فلما سمع يوناثان كلمات أبولونيوس هذه، تحرك قلبه، واختار عشرة آلاف رجل وخرج من أورشليم، حيث التقى به أخوه سمعان لمساعدته
\par 75 ونصب خيامه على يافا، ولكن أهل يافا أغلقوا عليه أبواب المدينة، لأن أبولونيوس كان له حامية هناك
\par 76 فحاصرها يوناثان، فتركه أهل المدينة خوفًا، وهكذا استولى يوناثان على يافا
\par 77 ولما سمع أبولونيوس بذلك، أخذ ثلاثة آلاف فارس، مع جيش عظيم من المشاة، وذهب إلى أزوط كمسافر، ومعه جره إلى السهل لأنه كان لديه عدد كبير من الفرسان، الذين وضع ثقته فيهم
\par 78 ثم تبعه يوناثان إلى أشدود، حيث التقت الجيوش في المعركة
\par 79 كان أبولونيوس قد ترك ألف فارس في كمين.
\par 80 فعرف يوناتان أن وراءه كميناً، لأنهم أحاطوا بجيشه وألقوا السهام على الشعب من الصباح إلى المساء.
\par 81 لكن الشعب وقف كما أمرهم يوناثان، فتعبت خيول الأعداء
\par 82 ثم أخرج سمعان جيشه، ووجههم ضد المشاة (لأن الفرسان كانوا قد أنهكوا) الذين هزمهم وهربوا
\par 83 وأما الفرسان فقد تشتتوا في الحقل، فهربوا إلى أشدود، ودخلوا بيت داجون معبد أصنامهم طلبا للنجاة.
\par 84 فأشعل يوناثان النار في أشدود والمدن التي حولها، ونهب غنائمها، وأحرق بالنار هيكل داجون مع الذين فروا إليه
\par 85 وهكذا أُحرق وقتل بالسيف ما يقرب من ثمانية آلاف رجل
\par 86 ومن هناك نقل يوناثان جيشه، ونزل أمام عسقلان، حيث خرج رجال المدينة، وقابلوه بفخر عظيم
\par 87 بعد ذلك، عاد يوناثان وجيشه إلى أورشليم، وقد غنموا بعض الغنائم
\par 88 فلما سمع الملك الإسكندر هذه الأمور، زاد من تكريم يوناثان
\par 89 وأرسل إليه مشبكًا من ذهب، كما يُمنح لمن هم من نسل الملك، وأعطاه أيضًا عكارون مع حدودها في حوزته

\chapter{11}

\par 1 وجمع ملك مصر جيشًا عظيمًا كالرمل الذي على شاطئ البحر، وسفنًا كثيرة، وسعى بالمكر للاستيلاء على مملكة الإسكندر وضمها إلى مملكته
\par 2 عندها انطلق في رحلته إلى إسبانيا بسلام، حتى فتح له أهل المدن أبوابهم واستقبلوه، لأن الملك الإسكندر أمرهم بذلك، لأنه صهره
\par 3 ولما دخل بطليموس المدن، جعل في كل واحدة منها حامية من الجنود لحراستها
\par 4 ولما اقترب من أشدود، أروه هيكل داجون المحترق، وأشدود وضواحيها المهدمة، والجثث التي ألقيت خارجًا، والتي أحرقها في المعركة؛ لأنهم جعلوا منها أكوامًا على الطريق الذي كان سيمر فيه
\par 5 وأخبروا الملك أيضًا بكل ما فعله يوناثان، ليلومه، لكن الملك سكت
\par 6 ثم استقبل يوناثان الملك في يافا بحفاوة بالغة، حيث سلموا على بعضهم البعض، وباتوا هناك
\par 7 بعد ذلك، ذهب يوناثان مع الملك إلى النهر المسمى إليوثيروس، ثم عاد إلى أورشليم
\par 8 لذلك، بعد أن استولى الملك بطليموس على المدن الواقعة على البحر حتى سلوقية على ساحل البحر، فكر في أفكار شريرة ضد الإسكندر
\par 9 عند ذلك أرسل سفراء إلى الملك ديمتريوس قائلًا: تعالَ نعقد عهدًا بيننا، فأعطيك ابنتي التي للإسكندر، فتملك في مملكة أبيك
\par 10 لأني نادم على أني أعطيته ابنتي، لأنه سعى لقتلي
\par 11 هكذا افترى عليه، لأنه كان راغبا في ملكوته.
\par 12 لذلك أخذ ابنته منه وأعطاها لديمتريوس، وترك الإسكندر، حتى أصبح بغضهما معروفًا للجميع.
\par 13 ثم دخل بطليموس أنطاكية، حيث وضع على رأسه تاجين، تاج آسيا وتاج مصر
\par 14 في تلك الأثناء كان الملك الإسكندر في كيليكيا، لأن سكان تلك الأنحاء ثاروا عليه
\par 15 ولكن عندما سمع الإسكندر بهذا، جاء لمحاربته، وعندها أخرج الملك بطليموس جيشه، وقابله بقوة عظيمة، وجعله يهزم.
\par 16 فهرب الإسكندر إلى شبه الجزيرة العربية للدفاع عنها، لكن الملك بطليموس رُفع شأنه:
\par 17 لأن زبديئيل العربي قطع رأس الإسكندر وأرسله إلى بطليموس
\par 18 وتوفي الملك بطليموس أيضًا في اليوم الثالث بعد ذلك، وقُتل الذين كانوا في الحصون بعضهم بعضًا
\par 19 وبهذه الطريقة، حكم ديمتريوس في السنة المئة والسابعة والستين
\par 20 وفي ذلك الوقت، جمع يوناثان الذين في اليهودية للاستيلاء على البرج الذي في أورشليم، وصنع عليه مجانيق كثيرة
\par 21 ثم جاء أناسٌ كفارٌ يكرهون قومهم، وذهبوا إلى الملك، وأخبروه أن يوناثان حاصر البرج،
\par 22 فلما سمع بذلك، غضب، وغادر على الفور، وجاء إلى بطليموس، وكتب إلى يوناثان، أنه لا ينبغي له أن يحاصر البرج، بل أن يأتي ويتحدث معه في بطليموس على عجل
\par 23 ومع ذلك، فلما سمع يوناثان ذلك، أمر بمحاصرتها أيضًا، واختار بعضًا من شيوخ إسرائيل والكهنة، وعرض نفسه للخطر؛
\par 24 وأخذ فضة وذهبًا وملابسًا وهدايا متنوعة أيضًا، وذهب إلى بطليموس إلى الملك، حيث وجد نعمة في عينيه
\par 25 وعلى الرغم من أن بعض الرجال الأشرار من الشعب قد تقدموا بشكاوى ضده،
\par 26 ومع ذلك، فقد توسل إليه الملك كما فعل أسلافه من قبل، ورفعه في نظر جميع أصدقائه،
\par 27 وأثبته في رئاسة الكهنوت، وفي جميع الأوسمة التي كان يتمتع بها سابقًا، وأعطاه مكانة مميزة بين أصدقائه الرئيسيين
\par 28 ثم طلب يوناثان من الملك أن يُحرر اليهودية، وكذلك الحكومات الثلاث، مع بلاد السامرة، ووعده بثلاثمائة وزنة
\par 29 فوافق الملك، وكتب رسائل إلى يوناثان بكل هذه الأمور على هذا النحو:
\par 30 يرسل الملك ديمتريوس إلى أخيه يوناثان وإلى أمة اليهود سلامًا:
\par 31 نرسل إليك هنا نسخة من الرسالة التي كتبناها إلى ابن عمنا لاستينيس بشأنك، حتى يتسنى لك الاطلاع عليها
\par 32 الملك ديمتريوس يرسل تحياته إلى أبيه لاستينيس:
\par 33 ونحن مصممون على عمل الخير لشعب اليهود، الذين هم أصدقاؤنا، والذين يحافظون على العهود معنا، بسبب حسن نواياهم نحونا.
\par 34 لذلك، أقرينا لهم حدود اليهودية، مع الحكومات الثلاث: أفيريما، ولِدّة، والرامة، المضافة إلى اليهودية من بلاد السامرة، وكل ما يتعلق بها، لكل من يقدم ذبائح في أورشليم، عوضًا عن المبالغ التي كان الملك يتقاضاها منهم سنويًا من ثمار الأرض والأشجار
\par 35 وأما الأشياء الأخرى التي تخصنا، من العشور والجمارك المتعلقة بنا، وكذلك مناجم الملح، وضرائب التاج المستحقة لنا، فإننا نتنازل عنها جميعًا من أجل تخفيفها.
\par 36 ولن يُلغى أي شيء من هذا من الآن فصاعدًا إلى الأبد.
\par 37 فالآن انظر أن تصنع نسخة من هذه الأشياء، وتسلّمها إلى يوناثان، وتضعها على الجبل المقدس في مكان بارز.
\par 38 بعد ذلك، عندما رأى الملك ديمتريوس أن الأرض هادئة أمامه، وأنه لا توجد مقاومة ضده، أرسل جميع قواته، كل واحد إلى مكانه، باستثناء بعض العصابات من الغرباء، الذين جمعهم من جزائر الوثنيين: لذلك كرهته جميع قوات آبائه
\par 39 علاوة على ذلك، كان هناك تريفون، الذي كان من أنصار الإسكندر سابقًا، والذي عندما رأى أن كل الجيش يتذمر ضد ديمتريوس، ذهب إلى سيمالكوي العربي الذي ربى أنطيوخس الابن الصغير للإسكندر،
\par 40 وألح عليه بشدة أن يسلم إليه هذا الشاب أنطيوخس، ليملك مكان أبيه. فأخبره بكل ما فعله ديمتريوس، وكيف كان رجال حربه في عداوة معه، ومكث هناك مدة طويلة
\par 41 في هذه الأثناء، أرسل يوناثان إلى الملك ديمتريوس ليطرد أهل البرج من أورشليم، وكذلك أهل الحصون، لأنهم حاربوا إسرائيل
\par 42 فأرسل ديمتريوس إلى يوناثان قائلًا: لن أفعل هذا لك ولشعبك فحسب، بل سأكرمك أنت وأمتك تكريمًا عظيمًا، إذا سنحت الفرصة
\par 43 والآن، ستفعل جيدًا إذا أرسلت لي رجالًا لمساعدتي؛ لأن جميع قواتي قد ذهبت مني
\par 44 عند ذلك، أرسل يوناثان ثلاثة آلاف رجل قوي إلى أنطاكية. وعندما وصلوا إلى الملك، فرح الملك جدًا بقدومهم
\par 45 فاجتمع أهل المدينة إلى وسط المدينة، وكان عددهم مئة وعشرين ألف رجل، وأرادوا أن يقتلوا الملك
\par 46 لذلك هرب الملك إلى الفناء، لكن أهل المدينة احتفظوا بممرات المدينة، وبدأوا القتال
\par 47 ثم دعا الملك اليهود للمساعدة، فجاءوا إليه دفعة واحدة، وتفرقوا في أنحاء المدينة وقتلوا في ذلك اليوم في المدينة مئة ألف
\par 48 وأشعلوا النار في المدينة، وغنموا غنائم كثيرة في ذلك اليوم، وأنقذوا الملك
\par 49 فلما رأى أهل المدينة أن اليهود قد استولوا على المدينة كما أرادوا، خفت شجاعتهم، فتوسلوا إلى الملك وصرخوا قائلين:
\par 50 امنحنا السلام، وليتوقف اليهود عن الاعتداء علينا وعلى المدينة
\par 51 عند ذلك ألقوا أسلحتهم، وعقدوا الصلح، وتم تكريم اليهود في نظر الملك، وفي نظر جميع من في مملكته، ورجعوا إلى أورشليم وقد غنموا غنائم كثيرة
\par 52 فجلس الملك ديمتريوس على عرش مملكته، وكانت الأرض هادئة أمامه.
\par 53 ومع ذلك، فقد تظاهر بالكذب في كل ما قاله، وانفصل عن يوناثان، ولم يكافئه حسب الفوائد التي نالها منه، بل أزعجه بشدة
\par 54 بعد ذلك عاد تريفون، ومعه الطفل الصغير أنطيوخس، الذي ملك وتوج
\par 55 ثم اجتمع إليه جميع رجال الحرب الذين عزلهم ديمتريوس، وقاتلوا ديمتريوس، الذي أدار ظهره وهرب
\par 56 علاوة على ذلك، استولى تريفون على الفيلة، وفاز بأنطاكية.
\par 57 في ذلك الوقت كتب الشاب أنطيوخس إلى يوناثان قائلا: أقرك في الكهنوت الأعظم، وأجعلك رئيسا على الحكومات الأربع، وتكون أحد أصدقاء الملك.
\par 58 عند ذلك أرسل إليه أواني ذهبية ليُخدم فيها، وأذن له أن يشرب بالذهب، وأن يلبس الأرجوان، وأن يلبس عروة من ذهب
\par 59 وجعل أيضًا أخاه سمعان قائدًا من المكان الذي يُدعى سلم صور إلى حدود مصر
\par 60 ثم خرج يوناثان وعبر مدن ما وراء الماء، فاجتمعت إليه كل جيوش سورية لمساعدته، ولما وصل إلى عسقلان استقبله أهل المدينة بإكرام
\par 61 ومن هناك ذهب إلى غزة، لكن أهل غزة أغلقوا عليه أبوابها، فحاصرها وأحرق ضواحيها بالنار ونهبها
\par 62 بعد ذلك، عندما توسل أهل غزة إلى يوناثان، عقد معهم الصلح، وأخذ أبناء رؤساءهم رهائن، وأرسلهم إلى أورشليم، وعبر البلاد إلى دمشق
\par 63 ولما سمع يوناثان أن أمراء ديمتريوس قد قدموا إلى قادس التي في الجليل بجيش عظيم، عازمون على إخراجه من البلاد،
\par 64 فذهب للقائهم، وترك سمعان أخاه في الريف.
\par 65 ثم نزل سمعان على بيت صور وحاربها أياما كثيرة وحاصرها.
\par 66 لكنهم أرادوا أن يُسالموه، فأعطاهم إياه، ثم أخرجهم من هناك، واستولى على المدينة، وأقام فيها حامية
\par 67 وأما يوناثان وجيشه، فقد نزلوا عند مياه جنيسار، ومن هناك في الصباح الباكر وصلوا إلى سهل ناصور
\par 68 وإذا بجيش الغرباء قد لاقوهم في السهل، فنصبوا له كمينًا في الجبال، ثم جابوه
\par 69 فلما قام الكمينون من أماكنهم والتحقوا بالقتال، هرب جميع من كان من صف يوناثان؛
\par 70 ولم يبق منهم أحد إلا متثيا بن أبشالوم ويهوذا بن كالفي قائدي الجيش
\par 71 فمزق يوناثان ثيابه، وطرح التراب على رأسه، وصلى
\par 72 بعد ذلك عاد مرة أخرى إلى المعركة، فهزمهم، فهربوا
\par 73 ولما رأى ذلك رجاله الذين هربوا رجعوا إليه وتبعوهم معه إلى قادس حتى إلى خيامهم وهناك نزلوا.
\par 74 فقتل من الأمم في ذلك اليوم نحو ثلاثة آلاف رجل. وأما يوناثان فرجع إلى أورشليم

\chapter{12}

\par 1 ولما رأى يوناثان أن الوقت قد نفعه، اختار رجالاً وأرسلهم إلى روما لتثبيت وتجديد الصداقة التي كانت بينهم وبينهم
\par 2 كما أرسل رسائل إلى اللاكديمونيين، وإلى أماكن أخرى، لنفس الغرض
\par 3 فذهبوا إلى روما، ودخلوا مجلس الشيوخ، وقالوا: أرسلنا إليكم يوناثان رئيس الكهنة وشعب اليهود، لكي تجددوا الصداقة التي كانت بينكم وبينهم، والعهد الذي كان بينكم في السابق
\par 4 بناءً على ذلك، أعطاهم الرومان رسائل إلى حكام كل مكان لكي يحضروهم إلى أرض اليهودية بسلام
\par 5 وهذه نسخة الرسائل التي كتبها يوناثان إلى الإسكندرانيين:
\par 6 يوناثان رئيس الكهنة، وشيوخ الأمة، والكهنة، وسائر اليهود، إلى الإسكندريين إخوتهم يُرسلون سلامًا
\par 7 كانت هناك رسائل أُرسلت قديمًا إلى أونياس رئيس الكهنة من داريوس، الذي كان يملك آنذاك بينكم، ليعلمكم أنكم إخوتنا، كما هو موضح في النسخة المكتوبة هنا
\par 8 في ذلك الوقت، توسّل أونياس إلى السفير المُرسَل بشرف، وتسلم الرسائل التي تضمنت إعلانًا عن العصبة والصداقة
\par 9 لذلك نحن أيضًا، وإن لم نكن بحاجة إلى شيء من هذه الأشياء، فإن لدينا كتب الكتاب المقدس في أيدينا لتعزيتنا،
\par 10 ومع ذلك، فقد حاولنا أن نرسل إليكم لتجديد الأخوة والصداقة، لئلا نصبح غرباء عنكم تمامًا: فقد مضى وقت طويل منذ أن أرسلتم إلينا
\par 11 لذلك، فنحن في كل وقت بلا انقطاع، سواء في أعيادنا أو في الأيام المناسبة الأخرى، نتذكركم في الذبائح التي نقدمها، وفي صلواتنا، كما هو منطقي، وكما يليق بنا أن نفكر في إخوتنا
\par 12 ونحن سعداء جدًا بشرفك.
\par 13 أما نحن، فقد واجهنا مشاكل وحروبًا كبيرة من كل جانب، لأن الملوك الذين حولنا حاربوا ضدنا
\par 14 ومع ذلك، لن نكون مصدر إزعاج لكم، ولا لغيركم من حلفائنا وأصدقائنا، في هذه الحروب:
\par 15 لأن لنا عونًا من السماء يعيننا، إذ نُنجى من أعدائنا، ويُسحق أعداؤنا
\par 16 لهذا السبب اخترنا نومينيوس بن أنطيوخس، وأنتيباتر بن ياسون، وأرسلناهما إلى الرومان، لتجديد الصداقة التي كانت بيننا وبينهم، والعهد السابق
\par 17 وأمرناهم أيضًا بالذهاب إليكم، وإلقاء التحية عليكم، وتسليمكم رسائلنا المتعلقة بتجديد أخوتنا
\par 18 لذا، من الجيد أن تعطونا إجابة على ذلك.
\par 19 وهذه هي نسخة الرسائل التي أرسلها أونياريس.
\par 20 أريوس ملك اللاكديمونيين إلى أونياس رئيس الكهنة، تحية:
\par 21 وقد وجد في الكتابة أن الإسكندرانيين واليهود هم إخوة، وأنهم من نسل إبراهيم:
\par 22 والآن، بما أننا علمنا بذلك، فمن الجيد أن تكتبوا إلينا عن نجاحكم
\par 23 نكتب إليكم مرة أخرى، أن ماشيتكم وبضائعكم لنا، ومواشينا وبضائعكم لكم. لذلك، نأمر سفراءنا بتقديم تقرير إليكم بهذا الشأن
\par 24 ولما سمع يوناثان أن أمراء ديمبيوس قد جاؤوا لمحاربته بجيش أعظم من ذي قبل،
\par 25 ثم انطلق من أورشليم، والتقى بهم في أرض أماثيس، لأنه لم يمهلهم لدخول بلاده
\par 26 أرسل جواسيس أيضًا إلى خيامهم، فعادوا وأخبروه أنهم مُعينون للقدوم إليهم في موسم الليل
\par 27 لذلك، بمجرد غروب الشمس، أمر يوناثان رجاله بالسهر والتسلح، حتى يكونوا مستعدين للقتال طوال الليل. كما أرسل حراسًا حول الجيش
\par 28 فلما سمع الأعداء أن يوناثان ورجاله مستعدون للقتال، خافوا وارتجفت قلوبهم، وأشعلوا النيران في معسكرهم
\par 29 ولكن يوناثان ورفاقه لم يعلموا بذلك إلا في الصباح، لأنهم رأوا الأنوار مشتعلة
\par 30 ثم طاردهم يوناثان، لكنه لم يدركهم، لأنهم كانوا قد عبروا نهر إليوثيروس
\par 31 لذلك التفت يوناثان إلى العرب، الذين كانوا يُدعون الزبديين، وضربهم وأخذ غنائمهم
\par 32 ثم انتقل من هناك وجاء إلى دمشق، وهكذا طاف في جميع أنحاء البلاد،
\par 33 وخرج سمعان أيضًا واجتاز في البلاد إلى عسقلان والحصون المجاورة لها، ومن هناك اتجه إلى يافا وانتصر عليها
\par 34 لأنه سمع أنهم سيسلمون الحصن إلى الذين اتخذوا موقف ديمتريوس، لذلك وضع حامية هناك لحفظه
\par 35 بعد هذا، عاد يوناثان إلى منزله، وجمع شيوخ الشعب، وتشاور معهم بشأن بناء حصون في اليهودية،
\par 36 ورفع أسوار أورشليم، ورفع جبلًا عظيمًا بين البرج والمدينة، ليفصلها عن المدينة، لتكون وحدها، فلا يستطيع الناس البيع أو الشراء فيها
\par 37 عند ذلك اجتمعوا لبناء المدينة، إذ كان جزء من السور الذي في جهة الوادي من جهة الشرق قد سقط، فرمموا ما كان يسمى كفرناثا
\par 38 كما أسس سمعان أديدا في سيفيلا، وحصنها بالبوابات والقضبان
\par 39 وكان تريفون يسعى للاستيلاء على مملكة آسيا، وقتل أنطيوخس الملك، ليضع التاج على رأسه
\par 40 ولكنه خاف ألا يدعه يوناثان فيُقاتله، فبحث عن طريقة ليقبض على يوناثان ويقتله. فسار ووصل إلى بيت شان.
\par 41 فخرج يوناثان للقائه بأربعين ألف رجل مختارين للقتال، وجاء إلى بيت شان
\par 42 فلما رأى تريفون أن يوناثان قادم بهذه القوة العظيمة، لم يجرؤ على مد يده إليه؛
\par 43 بل استقبلوه بإكرام، وأوصى به لجميع أصدقائه، وأعطوه هدايا، وأمر رجال حربه أن يكونوا مطيعين له كما يكونون مطيعين لنفسه
\par 44 وقال ليوناثان أيضًا: لماذا سببت كل هذا الشعب كل هذا الضيق، وليس بيننا حرب؟
\par 45 لذلك أرسلهم الآن إلى ديارهم، واختر بضعة رجال لخدمتك، وتعال معي إلى بطليموس، لأني سأسلمها لك، هي وبقية الحصون والجيوش، وكل من له مهمة. أما أنا، فسأعود وأنطلق، لأن هذا هو سبب مجيئي
\par 46 فآمن يوناثان به، ففعل كما أمره، وأرسل جيشه، فذهب إلى أرض يهوذا
\par 47 ولم يبق معه إلا ثلاثة آلاف رجل، أرسل منهم ألفين إلى الجليل، وذهب معه ألف
\par 48 ولما دخل يوناثان بطليماس، أغلق أهل بطليماس الأبواب وأخذوه، وقتلوا جميع الذين جاءوا معه بالسيف
\par 49 ثم أرسل تريفون جيشًا من المشاة والفرسان إلى الجليل وإلى السهل الكبير، ليبيد كل جماعة يوناثان
\par 50 فلما علموا أن يوناثان والذين معه قد أُسروا ومُقتَلوا، شجّع بعضهم بعضًا، واقتربوا من بعضهم البعض، مُستعدّين للقتال
\par 51 فأما الذين تبعوهم، فلما أدركوا أنهم مستعدون للقتال من أجل حياتهم، رجعوا
\par 52 وعند ذلك وصلوا جميعًا إلى أرض يهوذا بسلام، وهناك بكوا على يوناثان والذين معه، وخافوا خوفًا شديدًا، ولذلك حزن جميع إسرائيل كثيرًا
\par 53 ثم سعى جميع الوثنيين الذين كانوا حولهم إلى إهلاكهم، لأنهم قالوا: ليس لهم قائد، ولا من يساعدهم. والآن فلنشن حربًا عليهم، وننزع ذكرهم من بين الناس

\chapter{13}

\par 1 ولما سمع سمعان أن تريفون قد جمع جيشًا عظيمًا ليغزو أرض اليهودية ويدمرها،
\par 2 ورأى أن الشعب في رعدة وخوف عظيمين، فصعد إلى أورشليم وجمع الشعب،
\par 3 وحثهم قائلاً: أنتم تعلمون ما فعلت أنا وإخوتي وبيت أبي من أجل الشرائع والأقداس، وكذلك المعارك والاضطرابات التي رأيناها
\par 4 بسبب ذلك قُتل جميع إخوتي من أجل إسرائيل، وبقيت وحدي
\par 5 والآن فليكن حاشا لي أن أحافظ على نفسي في كل وقت من أوقات الضيق، لأني لست أفضل من إخوتي.
\par 6 لا شك أنني سأنتقم لأمتي، وللمقدس، ولنسائنا، وأطفالنا: لأن كل الأمم قد اجتمعوا ليدمرونا بخبث شديد
\par 7 فلما سمع الناس هذه الكلمات، انتعشت نفوسهم.
\par 8 فأجابوا بصوت عظيم قائلين: أنت تكون لنا قائداً مكان يهوذا ويوناثان أخيك.
\par 9 قاتل في معاركنا، ومهما أمرتنا به، فسنفعله
\par 10 فجمع حينئذٍ كل رجال الحرب، وأسرع في إكمال أسوار أورشليم، وحصنها من حولها
\par 11 وأرسل يوناثان بن أبشالوم ومعه جيش عظيم إلى يافا، فطرد الذين فيها وبقي هناك فيها
\par 12 فسار تريفون من بطليموس بجيش عظيم لغزو أرض يهودا، وكان يوناثان معه في الحراسة
\par 13 لكن سمعان نصب خيامه في عديدا، قبالة السهل.
\par 14 ولما علم تريفون أن سمعان قد قام مكان أخيه يوناثان، وأنه يريد أن ينضم إليه في الحرب، أرسل إليه رسلاً قائلاً:
\par 15 بما أن يوناثان أخاك موقوف لدينا، فهو مدين لخزانة الملك بمال يتعلق بالعمل الذي عُهد إليه
\par 16 لذلك أرسل الآن مئة وزنة من الفضة، واثنين من أبنائه كرهائن، حتى عندما يكون حراً لا يثور علينا، وسنطلق سراحه
\par 17 عندئذٍ، مع أن سمعان شعر أنهم يكلمونه بمكر، إلا أنه أرسل المال والأولاد، لئلا يجلب على نفسه بغضاء الناس الشديد
\par 18 من كان ليقول: لأني لم أرسل له المال والأولاد، فقد مات يوناثان؟
\par 19 فأرسل إليهم الأولاد ومئة وزنة. إلا أن تريفون تنكر ولم يطلق يوناثان
\par 20 وبعد ذلك جاء تريفون ليغزو الأرض ويدمرها، ودار حول الطريق المؤدية إلى أدورا. أما سمعان وجيشه فقد زحفوا ضده في كل مكان، أينما ذهب
\par 21 فأرسل الذين كانوا في البرج رسلًا إلى تريفون لكي يُعجِّل مجيئه إليهم من البرية ويُرسل لهم طعامًا
\par 22 لذلك أعد تريفون جميع فرسانه للمجيء في تلك الليلة، ولكن تساقط ثلوج كثيفة جدًا، فلم يأتِ بسببها. فانطلق وجاء إلى بلاد جلعاد
\par 23 ولما اقترب من بسكمة قتل يوناثان، ودُفن هناك
\par 24 بعد ذلك عاد تريفون وذهب إلى أرضه.
\par 25 ثم أرسل سمعان فأخذ عظام يوناثان أخيه ودفنها في مودين مدينة آبائه.
\par 26 وناح عليه جميع إسرائيل ندبًا عظيمًا، وندبوه أيامًا كثيرة
\par 27 وبنى سمعان أيضًا صرحًا على قبر أبيه وإخوته ورفعه عاليًا للنظر، بحجارة منحوتة من خلف ومن أمام.
\par 28 علاوة على ذلك، بنى سبعة أهرامات، واحدًا مقابل الآخر، لأبيه وأمه وإخوته الأربعة
\par 29 وفي هذه صنع مخترعات بارعة، وأقام حولها أعمدة عظيمة، وعلى الأعمدة صنع جميع أسلحة هذه الأعمدة لذكرى أبدية، وبجانب الأسلحة نحت سفنًا، لكي تكون مرئية لجميع الذين يبحرون في البحر
\par 30 هذا هو القبر الذي صنعه في مودين، وهو قائم إلى هذا اليوم
\par 31 ثم خدع تريفون الملك الشاب أنطيوخس وقتله
\par 32 وملك مكانه، وتوج نفسه ملكًا على آسيا، وجلب كارثة عظيمة على الأرض
\par 33 حينئذٍ بنى سمعان حصون اليهودية، وحاصرها بأبراج عالية وأسوار عظيمة وأبواب ومزاليج، وخزن فيها مؤنًا
\par 34 علاوة على ذلك، اختار سمعان رجالاً، وأرسل إلى الملك ديمتريوس، لكي يمنح الأرض حصانة، لأن كل ما فعله تريفون هو النهب
\par 35 فأجابه الملك ديمتريوس وكتب إليه هكذا:
\par 36 من الملك ديمتريوس إلى سمعان الكاهن الأعظم وصديق الملوك وكذلك إلى الشيوخ وأمة اليهود سلام.
\par 37 لقد استلمنا التاج الذهبي والرداء القرمزي اللذين أرسلتموهما إلينا، ونحن مستعدون لإبرام سلام دائم معكم، نعم، وللكتابة إلى ضباطنا، لتأكيد الحصانات التي منحناها
\par 38 وكل العهود التي قطعناها معكم تبقى، والحصون التي بنيتموها تكون لكم
\par 39 أما عن أي سهو أو تقصير ارتكبناه حتى هذا اليوم، فنحن نغفره، وكذلك ضريبة الإكليل التي تدينون بها لنا. وإذا كانت هناك جزية أخرى دُفعت في أورشليم، فلن تُدفع بعد الآن
\par 40 وانظروا من منكم أهلٌ لأن يكون في مجلسنا، فليُكتَب، وليكن سلامٌ بيننا
\par 41 وهكذا أُزيل نير الأمم عن إسرائيل في السنة المئة والسبعين
\par 42 حينئذٍ بدأ بنو إسرائيل يكتبون في صكوكهم وعقودهم، في السنة الأولى لسمعان الكاهن الأعظم والي اليهود ورئيسهم
\par 43 في تلك الأيام نزل سمعان على غزة وحاصرها من كل جانب، وصنع أيضًا مدفعية وأقامها بجانب المدينة، وضرب برجًا واستولى عليه
\par 44 وقفز الذين كانوا في المركبة إلى المدينة، فحدث اضطراب عظيم في المدينة
\par 45 فمزق أهل المدينة ثيابهم، وصعدوا على الأسوار مع زوجاتهم وأطفالهم، وصرخوا بصوت عظيم، متوسلين إلى سمعان أن يمنحهم السلام
\par 46 فقالوا: لا تعاملنا حسب شرنا، بل حسب رحمتك
\par 47 فهدأ سمعان منهم، ولم يعد يقاتلهم، بل أخرجهم من المدينة، وطهر البيوت التي كانت فيها الأصنام، ودخل إليها بالأغاني والشكر.
\par 48 نعم، لقد أزال كل نجاسة منها، ووضع فيها رجالاً يحفظون الشريعة، وجعلها أقوى مما كانت عليه من قبل، وبنى لنفسه فيها مسكنًا
\par 49 كما أن سكان البرج في أورشليم كانوا في ضائقة شديدة، لدرجة أنهم لم يتمكنوا من الخروج، أو دخول الريف، أو الشراء، أو البيع: ولذلك كانوا في ضائقة شديدة بسبب نقص الطعام، وهلك عدد كبير منهم بسبب المجاعة
\par 50 ثم صرخوا إلى سمعان، يتوسلون إليه أن يكون معهم، فأعطاهم ذلك، وبعد أن أخرجهم من هناك، طهر البرج من النجاسات
\par 51 ودخلوها في اليوم الثالث والعشرين من الشهر الثاني في السنة المئة والحادية والسبعين، بحمد، وأغصان النخيل، والقيثارات، والصنوج، والعيدان، والتسابيح، والأغاني، لأنه هُدم عدوٌّ عظيم من إسرائيل
\par 52 وأمر أيضًا أن يُحتفل بذلك اليوم كل عام بفرح. علاوة على ذلك، عزز تل الهيكل الذي بجانب البرج أكثر مما كان عليه، وأقام هناك مع رفاقه
\par 53 ولما رأى سمعان أن يوحنا ابنه رجل ذو بأس، جعله رئيسًا على جميع الجيوش، وأقام في جازرة

\chapter{14}

\par 1 وفي السنة المائة والثانية والستين، جمع الملك ديمتريوس قواته، وذهب إلى ميديا ​​ليطلب مساعدته في محاربة تريفون
\par 2 ولكن عندما سمع أرساكيس، ملك فارس وميديا، أن ديمتريوس قد دخل حدوده، أرسل أحد أمرائه ليأخذه حيًا:
\par 3 فذهب وضرب جيش ديمتريوس، وأخذه، وأتى به إلى أرساكيس، الذي وضعه في الحراسة
\par 4 أما أرض اليهودية، فقد كانت هادئة كل أيام سمعان، لأنه كان يسعى لخير أمته، حتى أن سلطته وكرامته كانتا ترضيانهم دائمًا
\par 5 وكما كان مُكرّمًا في جميع أعماله، كذلك في أنه اتخذ يافا مرفأً، وفتح مدخلًا إلى جزائر البحر،
\par 6 ووسّع حدود أمته، واستعاد البلاد،
\par 7 فجمع سبيا كثيرا واستولى على جازرة وبيت صور والبرج وأخرج منه كل نجاسة ولم يكن أحد يقاومه.
\par 8 ثم عملوا أرضهم بسلام، وأعطت الأرض غلتها، وأشجار الحقل ثمرها
\par 9 جلس الرجال القدماء جميعًا في الشوارع، يتبادلون أطراف الحديث حول الأشياء الطيبة، وارتدى الشباب ملابس مجيدة وحربية
\par 10 وفّر المؤن للمدن، ووضع فيها كل أنواع الذخيرة، حتى اشتهر اسمه الشريف إلى نهاية العالم
\par 11 وصنع السلام في الأرض، ففرح إسرائيل فرحاً عظيماً.
\par 12 لأن كل إنسان جلس تحت كرمته وتينته، ​​ولم يكن من يضايقهم:
\par 13 ولم يبق في الأرض من يحاربهم: نعم، حتى الملوك أنفسهم هُزموا في تلك الأيام
\par 14 علاوة على ذلك، شدد جميع الذين كانوا منحطين من شعبه: فحص الشريعة، وأزال كل من احتقر الشريعة وكل منكر
\par 15 جَمَّلَ المَقْدِسَ، وَكَثَّرَ أَوْانِيَ الْبَيْتِ.
\par 16 ولما سمع في روما، وحتى في إسبرطة، أن يوناثان قد مات، حزنوا جداً.
\par 17 فلما سمعوا أن أخاه سمعان قد رُسِّمَ كهنةً عوضًا عنه، وملك البلاد والمدن التي فيها،
\par 18 وكتبوا إليه على ألواح من نحاس، لتجديد الصداقة والحلف الذي عقدوه مع يهوذا ويوناثان إخوته:
\par 19 ما هي الكتابات التي قُرئت أمام الجماعة في أورشليم.
\par 20 وهذه نسخة الرسائل التي أرسلها الإسكندريون: من رؤساء الإسكندريون والمدينة إلى سمعان رئيس الكهنة والشيوخ والكهنة وباقي شعب اليهود إخوتنا يهدون سلاماً.
\par 21 لقد شهد لنا السفراء الذين أُرسلوا إلى شعبنا بمجدكم وشرفكم، ولذلك سررنا بقدومهم،
\par 22 وسجلوا ما تكلموا به في مجلس الشعب على هذا النحو: جاء إلينا نومينيوس بن أنطيوخس، وأنتيباتر بن ياسون، سفيرا اليهود، ليجددا صداقتهما معنا
\par 23 فسرّ الشعب أن يستقبل الرجلين باحترام، وأن يسجّلوا نسخة سفارتهما في السجلات العامة، حتى يكون لشعب الإسكندريون ذكرى بذلك. علاوة على ذلك، كتبنا نسخة منها إلى سمعان رئيس الكهنة
\par 24 بعد ذلك، أرسل سمعان نومينيوس إلى روما ومعه درع عظيم من الذهب وزنه ألف رطل لتأكيد العهد معهم
\par 25 فلما سمع الشعب قالوا: بماذا نشكر سمعان وبنيه؟
\par 26 لأنه هو وإخوته وبيت أبيه ثبّتوا إسرائيل، وطردوا أعداءهم عنهم بالحرب، وأثبتوا حريتهم
\par 27 فكتبوها على ألواح من نحاس وضعها على أعمدة في جبل صهيون. وهذه نسخة الكتابة: في اليوم الثامن عشر من شهر إيلول، في السنة المئة والثانية والستين، وهي السنة الثالثة لسمعان رئيس الكهنة،
\par 28 في ساراميل، في الجماعة الكبيرة من الكهنة والشعب وحكام الأمة وشيوخ البلاد، أُبلغنا بهذه الأمور
\par 29 "ولما كانت هناك حروب كثيرة في البلاد، حيث من أجل الحفاظ على مقدساتهم والشريعة، عرض سمعان بن متتيا من نسل ياريب، مع إخوته، أنفسهم للخطر، ومقاومة أعداء أمتهم جلبت لأمتهم شرفًا عظيمًا.
\par 30 (لأنه بعد أن جمع يوناثان أمته، وكان رئيس كهنتهم، انضم إلى شعبه،
\par 31 استعد أعداؤهم لغزو بلادهم، لتدميرها، ووضع أيديهم على المقدس:
\par 32 في ذلك الوقت، نهض سمعان، وقاتل من أجل أمته، وأنفق الكثير من ماله، وسلح رجال أمته البواسل وأعطاهم أجورًا،
\par 33 وحصن مدن اليهودية، مع بيت صور، التي تقع على حدود اليهودية، حيث كانت دروع الأعداء من قبل؛ وأقام هناك حامية من اليهود
\par 34 علاوة على ذلك، حصن يافا الواقعة على البحر، وجازرة المتاخمة لأشدود، حيث كان الأعداء يسكنون من قبل: لكنه أسكن يهودًا هناك، وزودهم بكل ما يلزم لإصلاحها.)
\par 35 لذلك أنشد الشعب أعمال سمعان، ولما ظن أنه سيجلبه لأمته من مجد، جعلوه حاكمًا عليهم ورئيسًا للكهنة، لأنه فعل كل هذه الأشياء، ومن أجل البر والإيمان اللذين حافظ عليهما لأمته، ومن أجل ذلك سعى بكل الوسائل إلى تمجيد شعبه
\par 36 لأنه في أيامه، ازدهرت الأمور على يديه، حتى أُخرج الوثنيون من بلادهم، وكذلك الذين كانوا في مدينة داود في أورشليم، الذين بنوا لأنفسهم برجًا، وكانوا يخرجون منه، ودنسوا كل ما حول المقدس، وألحقوا أضرارًا كثيرة بالقدس
\par 37 لكنه وضع يهودًا فيها وحصنها من أجل سلامة البلاد والمدينة، ورفع أسوار القدس
\par 38 وأثبته الملك ديمتريوس أيضًا في رئاسة الكهنوت حسب تلك الأمور،
\par 39 واتخذه من أصدقائه، وأكرمه إكرامًا عظيمًا.
\par 40 لأنه سمع أن الرومان كانوا يدعون اليهود أصدقاءهم وحلفائهم وإخوتهم، وأنهم استقبلوا سفراء سمعان بشرف.
\par 41 وأن اليهود والكهنة سُرّوا بأن يكون سمعان حاكمًا عليهم ورئيس كهنة إلى الأبد، إلى أن يقوم نبي أمين؛
\par 42 علاوة على ذلك، يجب أن يكون قائدهم، وأن يتولى مسؤولية المقدس، وأن يجعلهم على أعمالهم، وعلى البلاد، وعلى الأسلحة، وعلى الحصون، أي أنه، كما أقول، يجب أن يتولى مسؤولية المقدس؛
\par 43 بالإضافة إلى ذلك، أن يُطاع من كل إنسان، وأن تُكتب جميع الكتابات في البلاد باسمه، وأن يلبس الأرجوان والذهب:
\par 44 وأيضاً فإنه لا يجوز لأحد من الشعب أو الكهنة أن ينقض شيئاً من هذه الأشياء، أو أن يعارض كلامه، أو أن يجمع جماعة في البلاد بدونه، أو أن يلبس الأرجوان، أو يلبس عروة من ذهب.
\par 45 ومن يفعل خلاف ذلك، أو يخالف أيًا من هذه الأشياء، فيجب معاقبته
\par 46 وهكذا أحب جميع الناس أن يتعاملوا مع سمعان، وأن يفعلوا ما قيل لهم
\par 47 فقبل سمعان هذا، وارتضى أن يكون رئيس كهنة وقائدًا وحاكمًا لليهود والكهنة، ويدافع عنهم جميعًا
\par 48 فأمروا أن توضع هذه الكتابة في لوحين من نحاس، وأن توضعا داخل دائرة القدس في مكان ظاهر
\par 49 وأن تُحفظ صورها في الخزانة، حتى تكون في متناول سمعان وأبنائه

\chapter{15}

\par 1 ثم أرسل الملك أنطيوخس بن ديمتريوس رسائل من جزائر البحر إلى سمعان الكاهن رئيس اليهود وإلى كل الشعب
\par 2 وكان محتواها: من الملك أنطيوخس إلى سمعان رئيس الكهنة ورئيس أمته، وإلى شعب اليهود، سلام:
\par 3 بما أن بعض الرجال الأشرار قد اغتصبوا مملكة آبائنا، وهدفي هو تحديها مرة أخرى، حتى أتمكن من إعادتها إلى حالتها القديمة، ولهذا الغرض جمعت حشدًا من الجنود الأجانب، وجهزت سفنًا حربية؛
\par 4 أقصد أيضًا أن أجوب البلاد، لأنتقم ممن دمروا البلاد، وجعلوا مدنًا كثيرة في المملكة خربت:
\par 5 والآن أُقرّ لك جميع القرابين التي قدمها لك الملوك قبلي، وكل الهدايا الأخرى التي قدموها لك
\par 6 أعطيك الإذن أيضًا بسك النقود لبلدك بطابعك الخاص
\par 7 وأما أورشليم والقدس، فليكن حرين، وجميع الأسلحة التي صنعتها، والحصون التي بنيتها، وحفظتها في يدك، فلتبق لك
\par 8 وإذا كان هناك أي شيء، أو سيكون هناك أي شيء، مدينًا به للملك، فليُغفر لك من الآن فصاعدًا إلى الأبد
\par 9 علاوة على ذلك، عندما نحصل على مملكتنا، سنكرمك أنت وأمتك ومعبدك بإكرام عظيم، حتى يُعرف شرفك في جميع أنحاء العالم
\par 10 في السنة المائة والرابعة عشرة، ذهب أنطيوخس إلى أرض آبائه، وفي ذلك الوقت اجتمعت كل القوات إليه، حتى لم يبق مع تريفون إلا عدد قليل
\par 11 لذلك، بعد أن طارده الملك أنطيوخس، هرب إلى دورا الواقعة على شاطئ البحر
\par 12 لأنه رأى أن المتاعب حلت به دفعة واحدة، وأن قواته قد تخلت عنه
\par 13 ثم نزل أنطيوخس على دورا، ومعه مئة وعشرون ألف رجل حرب، وثمانية آلاف فارس
\par 14 ولما أحاط بالمدينة، وجمع السفن القريبة من المدينة من جهة البحر، ضايق المدينة براً وبحراً، ولم يدع أحداً يخرج أو يدخل.
\par 15 في تلك الأثناء، جاء نومينيوس ورفاقه من روما، ومعهم رسائل إلى الملوك والبلاد؛ كُتبت فيها هذه الأمور:
\par 16 لوسيوس، قنصل الرومان لدى الملك بطليموس، تحية:
\par 17 "فجاء إلينا سفراء اليهود، أصدقاؤنا وحلفائنا، لتجديد الصداقة والتحالف القديم، مرسلين من سمعان رئيس الكهنة ومن شعب اليهود.
\par 18 وأتوا بدرع من ذهب وزنه ألف منا.
\par 19 لذلك رأينا أنه من الجيد أن نكتب إلى الملوك والبلاد، أن لا يضروهم، ولا يحاربوهم، ولا يحاربوا مدنهم أو بلادهم، ولا يساعدوا أعداءهم ضدهم.
\par 20 بدا لنا أيضًا أنه من الجيد أن نتلقى درعهم.
\par 21 فإن كان أحد من الناس مفسداً قد هرب إليكم من بلادهم، فسلموهم إلى سمعان رئيس الكهنة لكي يعاقبهم حسب ناموسهم.
\par 22 وكتب نفس الأمور كذلك إلى ديمتريوس الملك، وأتالوس، وأرياراثيس، وأرساكيس،
\par 23 وإلى جميع البلدان وإلى سمبسامس، واللاسديمونيين، وديلوس، وميندوس، وسكيون، وكاريا، وساموس، وبمفيلية، وليكيا، وهاليكارناسوس، ورودس، وأرادوس، وكوس، وسيع، وأرادوس، وغورتينا، وكنيدوس، وقبرص، وقيروان
\par 24 وكتبوا نسخة هذا إلى سمعان رئيس الكهنة.
\par 25 فنزل الملك أنطيوخس على دورا في اليوم الثاني، وهاجمها باستمرار، وصنع آلات، فبها حاصر تريفون حتى لم يستطع أن يخرج ولا يدخل.
\par 26 في ذلك الوقت، أرسل إليه سمعان ألفي رجل مختار لمساعدته، وفضةً وذهبًا ودروعًا كثيرة
\par 27 ومع ذلك لم يقبلها، بل نقض جميع العهود التي قطعها معه من قبل، وأصبح غريبًا عنه
\par 28 ثم أرسل إليه أثينوبيوس، أحد أصدقائه، ليتحدث معه ويقول له: أنت تمنع يافا وجازرة والبرج الذي في أورشليم، وهما مدينتان من مدن مملكتي
\par 29 لقد خربت حدودها، وألحقت ضررًا كبيرًا بالأرض، واستوليتم على أماكن كثيرة داخل مملكتي
\par 30 فالآن، سلموا المدن التي استوليتم عليها، وجزية الأماكن التي تسلطتم عليها خارج حدود اليهودية
\par 31 أو أعطوني بدلهم خمسمائة وزنة من الفضة، وبدل الضرر الذي سببتموه، وبدل جزية المدن، خمسمائة وزنة أخرى. وإلا، فسنأتي ونقاتلكم
\par 32 فجاء أثينوبيوس صديق الملك إلى أورشليم، ولما رأى مجد سمعان، وخزانة الذهب والفضة، وحضوره العظيم، اندهش، وأخبره برسالة الملك
\par 33 فأجاب سمعان وقال له: نحن لم نأخذ أرضاً لغيرنا، ولا نملك ما هو للآخرين، بل ميراث آبائنا الذي استولى عليه أعداؤنا ظلماً في زمن ما.
\par 34 لذلك، إذ لدينا فرصة، نحمل ميراث آبائنا
\par 35 وبينما تطالب يافا وجازيرة، فمع أنهما ألحقتا ضررًا كبيرًا بشعب بلدنا، فإننا سنعطيك مئة وزنة مقابلهما. لم يُجبه أثينوبيوس على ذلك بكلمة واحدة؛
\par 36 لكنه رجع إلى الملك غاضبًا، وأخبره بهذه الكلمات، وبمجد سمعان، وبكل ما رآه، فغضب الملك غضبًا شديدًا
\par 37 في هذه الأثناء، هرب تريفون بالسفينة إلى أورثوسياس.
\par 38 ثم جعل الملك كندبيوس قائدا على ساحل البحر، وأعطاه جيشا من المشاة والفرسان،
\par 39 وأمره بنقل جيشه نحو اليهودية، وأمره أيضًا ببناء قدرون، وتحصين الأبواب، ومحاربة الشعب، وأما الملك نفسه، فقد طارد تريفون
\par 40 فجاء كندباوس إلى يمنيا وبدأ يُثير الشعب ويغزو اليهودية، ويأخذ الناس أسرى ويقتلهم
\par 41 ولما بنى سيدرو، وضع هناك فرسانًا وجيشًا من المشاة، لكي يخرجوا ويشقوا طرقًا في اليهودية، كما أمره الملك

\chapter{16}

\par 1 فصعد يوحنا من جزيرة وأخبر سمعان أباه بما فعل كندباوس
\par 2 لذلك دعا سمعان ابنيه الأكبرين، يهوذا ويوحنا، وقال لهما: أنا وإخوتي وبيت أبي، منذ صباي إلى هذا اليوم، حاربنا أعداء إسرائيل. وقد نجحت الأمور في أيدينا حتى خلصنا إسرائيل مرات عديدة
\par 3 لكنني الآن عجوز، وأنتم، برحمة الله، بلغتم سنًا كافية: كونوا مكاني أنا وأخي، واذهبوا وقاتلوا من أجل أمتنا، وليكن عون السماء معكم
\par 4 فاختار من البلاد عشرين ألف رجل حرب مع فرسان، وخرجوا لمحاربة كندبيوس، واستراحوا تلك الليلة في مودين
\par 5 ولما استيقظوا في الصباح ودخلوا السهل، إذا بجيش عظيم عظيم من المشاة والفرسان قد أقبل عليهم، وكان بينهم جدول ماء
\par 6 فنزل هو وقومه مقابلهم. ولما رأى أن الشعب خائف من عبور النهر، عبر هو أولاً، ثم رآه الرجال وعبروا خلفه
\par 7 بعد أن فعل ذلك، قسم رجاله، وجعل الفرسان في وسط المشاة، لأن فرسان الأعداء كانوا كثيرين جدًا
\par 8 ثم نفخوا في الأبواق المقدسة، فهزم كندبيوس وجيشه، فقُتل كثيرون منهم، ووصل الباقون إلى الحصن
\par 9 وفي ذلك الوقت كان يهوذا أخو يوحنا مصاباً، ولكن يوحنا كان لا يزال يتبعهم حتى جاء إلى قدرون التي بناها كندباوس.
\par 10 فهربوا حتى إلى الأبراج في حقول أشدود، فأحرقها بالنار، فقتل منهم نحو ألفي رجل. ثم عاد بعد ذلك إلى أرض يهوذا بسلام
\par 11 علاوة على ذلك، في سهل أريحا، عُيّن بطليموس بن أبوبس قائدًا، وكان لديه وفرة من الفضة والذهب:
\par 12 لأنه كان صهر رئيس الكهنة.
\par 13 ولذلك ارتفع قلبه، وفكر في الاستيلاء على البلاد لنفسه، ومن ثم استشار بمكر ضد سمعان وبنيه لتدميرهم.
\par 14 وكان سمعان يتجول في المدن التي في الريف، ويهتم بحسن تنظيمها. وفي ذلك الوقت نزل هو إلى أريحا مع ابنيه متثيا ويهوذا، في السنة المئة والسابعة والستين، في الشهر الحادي عشر الذي يدعى سبت
\par 15 حيث استقبلهم ابن أبو بوسة بخداع في حصن صغير، يُدعى دوكوس، كان قد بناه، وأقام لهم وليمة عظيمة: مع أنه كان قد أخفى رجالاً هناك
\par 16 فلما شرب سمعان وأبناؤه كثيرًا، قام بطليموس ورجاله، وأخذوا أسلحتهم، ودخلوا سمعان إلى مكان الوليمة، فقتلوه هو وابنيه وبعض عبيده
\par 17 وبهذا الفعل ارتكب خيانة عظمى، وجزا الخير عن الشر
\par 18 ثم كتب بطليموس هذه الأمور، وأرسل إلى الملك أن يرسل له جيشًا لمساعدته، وأن يسلمه البلاد والمدن
\par 19 وأرسل آخرين أيضًا إلى جازيرة ليقتلوا يوحنا، وأرسل إلى الولاة رسائل ليأتوا إليه، ليعطيهم فضة وذهبًا ومكافآت
\par 20 وأرسل آخرين ليستولوا على أورشليم وجبل الهيكل
\par 21 سبق أن ركض أحدهم إلى جازيرا وأخبر يوحنا أن أباه وإخوته قد قُتلوا، فقال: "لقد أرسل بطليموس ليقتلك أنت أيضًا".
\par 22 فلما سمع ذلك دهش دهشة شديدة، فوضع يديه على الذين جاءوا ليهلكوه، فقتلهم، لأنه علم أنهم كانوا يطلبون قتله
\par 23 وأما بقية أعمال يوحنا، وحروبه، وأعماله الصالحة التي صنعها، وبناء الأسوار التي بناها، وأعماله،
\par 24 ها هي هذه مكتوبة في أخبار كهنوته، منذ أن رُسِمَ رئيسًا للكهنة بعد أبيه

\end{document}