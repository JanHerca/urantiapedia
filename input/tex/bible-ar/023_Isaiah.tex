\begin{document}

\title{اشعياء}


\chapter{1}

\par 1 رُؤْيَا إِشَعْيَاءَ بْنِ آمُوصَ الَّتِي رَآهَا عَلَى يَهُوذَا وَأُورُشَلِيمَ فِي أَيَّامِ عُزِّيَّا وَيُوثَامَ وَآحَازَ وَحَزَقِيَّا مُلُوكِ يَهُوذَا:
\par 2 اِسْمَعِي أَيَّتُهَا السَّمَاوَاتُ وَأَصْغِي أَيَّتُهَا الأَرْضُ لأَنَّ الرَّبَّ يَتَكَلَّمُ: «رَبَّيْتُ بَنِينَ وَنَشَّأْتُهُمْ أَمَّا هُمْ فَعَصُوا عَلَيَّ.
\par 3 اَلثَّوْرُ يَعْرِفُ قَانِيهِ وَالْحِمَارُ مِعْلَفَ صَاحِبِهِ أَمَّا إِسْرَائِيلُ فَلاَ يَعْرِفُ. شَعْبِي لاَ يَفْهَمُ».
\par 4 وَيْلٌ لِلأُمَّةِ الْخَاطِئَةِ الشَّعْبِ الثَّقِيلِ الإِثْمِ نَسْلِ فَاعِلِي الشَّرِّ أَوْلاَدِ مُفْسِدِينَ! تَرَكُوا الرَّبَّ اسْتَهَانُوا بِقُدُّوسِ إِسْرَائِيلَ ارْتَدُّوا إِلَى وَرَاءٍ.
\par 5 عَلَى مَ تُضْرَبُونَ بَعْدُ؟ تَزْدَادُونَ زَيَغَاناً! كُلُّ الرَّأْسِ مَرِيضٌ وَكُلُّ الْقَلْبِ سَقِيمٌ.
\par 6 مِنْ أَسْفَلِ الْقَدَمِ إِلَى الرَّأْسِ لَيْسَ فِيهِ صِحَّةٌ بَلْ جُرْحٌ وَأَحْبَاطٌ وَضَرْبَةٌ طَرِيَّةٌ لَمْ تُعْصَرْ وَلَمْ تُعْصَبْ وَلَمْ تُلَيَّنْ بِالزَّيْتِ.
\par 7 بِلاَدُكُمْ خَرِبَةٌ. مُدُنُكُمْ مُحْرَقَةٌ بِالنَّارِ. أَرْضُكُمْ تَأْكُلُهَا غُرَبَاءُ قُدَّامَكُمْ وَهِيَ خَرِبَةٌ كَانْقِلاَبِ الْغُرَبَاءِ.
\par 8 فَبَقِيَتِ ابْنَةُ صِهْيَوْنَ كَمِظَلَّةٍ فِي كَرْمٍ كَخَيْمَةٍ فِي مَقْثَأَةٍ كَمَدِينَةٍ مُحَاصَرَةٍ.
\par 9 لَوْلاَ أَنَّ رَبَّ الْجُنُودِ أَبْقَى لَنَا بَقِيَّةً صَغِيرَةً لَصِرْنَا مِثْلَ سَدُومَ وَشَابَهْنَا عَمُورَةَ.
\par 10 اِسْمَعُوا كَلاَمَ الرَّبِّ يَا قُضَاةَ سَدُومَ! أَصْغُوا إِلَى شَرِيعَةِ إِلَهِنَا يَا شَعْبَ عَمُورَةَ:
\par 11 «لِمَاذَا لِي كَثْرَةُ ذَبَائِحِكُمْ؟» يَقُولُ الرَّبُّ «اتَّخَمْتُ مِنْ مُحْرَقَاتِ كِبَاشٍ وَشَحْمِ مُسَمَّنَاتٍ وَبِدَمِ عُجُولٍ وَخِرْفَانٍ وَتُيُوسٍ مَا أُسَرُّ.
\par 12 حِينَمَا تَأْتُونَ لِتَظْهَرُوا أَمَامِي مَنْ طَلَبَ هَذَا مِنْ أَيْدِيكُمْ أَنْ تَدُوسُوا دِيَارِي؟
\par 13 لاَ تَعُودُوا تَأْتُونَ بِتَقْدِمَةٍ بَاطِلَةٍ. الْبَخُورُ هُوَ مَكْرُهَةٌ لِي. رَأْسُ الشَّهْرِ وَالسَّبْتُ وَنِدَاءُ الْمَحْفَلِ. لَسْتُ أُطِيقُ الإِثْمَ وَالاِعْتِكَافَ.
\par 14 رُؤُوسُ شُهُورِكُمْ وَأَعْيَادُكُمْ بَغَضَتْهَا نَفْسِي. صَارَتْ عَلَيَّ ثِقْلاً. مَلِلْتُ حِمْلَهَا.
\par 15 فَحِينَ تَبْسُطُونَ أَيْدِيكُمْ أَسْتُرُ عَيْنَيَّ عَنْكُمْ وَإِنْ كَثَّرْتُمُ الصَّلاَةَ لاَ أَسْمَعُ. أَيْدِيكُمْ مَلآنَةٌ دَماً.
\par 16 اِغْتَسِلُوا. تَنَقُّوا. اعْزِلُوا شَرَّ أَفْعَالِكُمْ مِنْ أَمَامِ عَيْنَيَّ. كُفُّوا عَنْ فِعْلِ الشَّرِّ.
\par 17 تَعَلَّمُوا فِعْلَ الْخَيْرِ. اطْلُبُوا الْحَقَّ. انْصِفُوا الْمَظْلُومَ. اقْضُوا لِلْيَتِيمِ. حَامُوا عَنِ الأَرْمَلَةِ.
\par 18 هَلُمَّ نَتَحَاجَجْ يَقُولُ الرَّبُّ. إِنْ كَانَتْ خَطَايَاكُمْ كَالْقِرْمِزِ تَبْيَضُّ كَالثَّلْجِ. إِنْ كَانَتْ حَمْرَاءَ كَالدُّودِيِّ تَصِيرُ كَالصُّوفِ.
\par 19 إِنْ شِئْتُمْ وَسَمِعْتُمْ تَأْكُلُونَ خَيْرَ الأَرْضِ.
\par 20 وَإِنْ أَبَيْتُمْ وَتَمَرَّدْتُمْ تُؤْكَلُونَ بِالسَّيْفِ». لأَنَّ فَمَ الرَّبِّ تَكَلَّمَ.
\par 21 كَيْفَ صَارَتِ الْقَرْيَةُ الأَمِينَةُ زَانِيَةً! مَلآنَةً حَقّاً. كَانَ الْعَدْلُ يَبِيتُ فِيهَا. وَأَمَّا الآنَ فَالْقَاتِلُونَ.
\par 22 صَارَتْ فِضَّتُكِ زَغَلاً وَخَمْرُكِ مَغْشُوشَةً بِمَاءٍ.
\par 23 رُؤَسَاؤُكِ مُتَمَرِّدُونَ وَلُغَفَاءُ اللُّصُوصِ. كُلُّ وَاحِدٍ مِنْهُمْ يُحِبُّ الرَّشْوَةَ وَيَتْبَعُ الْعَطَايَا. لاَ يَقْضُونَ لِلْيَتِيمِ وَدَعْوَى الأَرْمَلَةِ لاَ تَصِلُ إِلَيْهِمْ.
\par 24 لِذََلِكَ يَقُولُ السَّيِّدُ رَبُّ الْجُنُودِ عَزِيزُ إِسْرَائِيلَ: «آهِ! إِنِّي أَسْتَرِيحُ مِنْ خُصَمَائِي وَأَنْتَقِمُ مِنْ أَعْدَائِي
\par 25 وَأَرُدُّ يَدِي عَلَيْكِ وَأُنَقِّي زَغَلَكِ كَأَنَّهُ بِالْبَوْرَقِ وَأَنْزِعُ كُلَّ قَصْدِيرِكِ
\par 26 وَأُعِيدُ قُضَاتَكِ كَمَا فِي الأَوَّلِ وَمُشِيرِيكِ كَمَا فِي الْبَدَاءَةِ. بَعْدَ ذَلِكَ تُدْعَيْنَ مَدِينَةَ الْعَدْلِ الْقَرْيَةَ الأَمِينَةَ».
\par 27 صِهْيَوْنُ تُفْدَى بِالْحَقِّ وَتَائِبُوهَا بِالْبِرِّ.
\par 28 وَهَلاَكُ الْمُذْنِبِينَ وَالْخُطَاةِ يَكُونُ سَوَاءً وَتَارِكُو الرَّبِّ يَفْنُونَ.
\par 29 لأَنَّهُمْ يَخْجَلُونَ مِنْ أَشْجَارِ الْبُطْمِ الَّتِي اشْتَهَيْتُمُوهَا وَتُخْزَوْنَ مِنَ الْجَنَّاتِ الَّتِي اخْتَرْتُمُوهَا.
\par 30 لأَنَّكُمْ تَصِيرُونَ كَبُطْمَةٍ قَدْ ذَبُلَ وَرَقُهَا وَكَجَنَّةٍ لَيْسَ لَهَا مَاءٌ.
\par 31 وَيَصِيرُ الْقَوِيُّ مَشَاقَةً وَعَمَلُهُ شَرَاراً فَيَحْتَرِقَانِ كِلاَهُمَا مَعاً وَلَيْسَ مَنْ يُطْفِئُ.

\chapter{2}

\par 1 اَلأُمُورُ الَّتِي رَآهَا إِشَعْيَاءُ بْنُ آمُوصَ مِنْ جِهَةِ يَهُوذَا وَأُورُشَلِيمَ:
\par 2 وَيَكُونُ فِي آخِرِ الأَيَّامِ أَنَّ جَبَلَ بَيْتِ الرَّبِّ يَكُونُ ثَابِتاً فِي رَأْسِ الْجِبَالِ وَيَرْتَفِعُ فَوْقَ التِّلاَلِ وَتَجْرِي إِلَيْهِ كُلُّ الأُمَمِ.
\par 3 وَتَسِيرُ شُعُوبٌ كَثِيرَةٌ وَيَقُولُونَ: «هَلُمَّ نَصْعَدْ إِلَى جَبَلِ الرَّبِّ إِلَى بَيْتِ إِلَهِ يَعْقُوبَ فَيُعَلِّمَنَا مِنْ طُرُقِهِ وَنَسْلُكَ فِي سُبُلِهِ». لأَنَّهُ مِنْ صِهْيَوْنَ تَخْرُجُ الشَّرِيعَةُ وَمِنْ أُورُشَلِيمَ كَلِمَةُ الرَّبِّ.
\par 4 فَيَقْضِي بَيْنَ الأُمَمِ وَيُنْصِفُ لِشُعُوبٍ كَثِيرِينَ فَيَطْبَعُونَ سُيُوفَهُمْ سِكَكاً وَرِمَاحَهُمْ مَنَاجِلَ. لاَ تَرْفَعُ أُمَّةٌ عَلَى أُمَّةٍ سَيْفاً وَلاَ يَتَعَلَّمُونَ الْحَرْبَ فِي مَا بَعْدُ.
\par 5 يَا بَيْتَ يَعْقُوبَ هَلُمَّ فَنَسْلُكُ فِي نُورِ الرَّبِّ.
\par 6 فَإِنَّكَ رَفَضْتَ شَعْبَكَ بَيْتَ يَعْقُوبَ لأَنَّهُمُ امْتَلَأُوا مِنَ الْمَشْرِقِ وَهُمْ عَائِفُونَ كَالْفِلِسْطِينِيِّينَ وَيُصَافِحُونَ أَوْلاَدَ الأَجَانِبِ.
\par 7 وَامْتَلَأَتْ أَرْضُهُمْ فِضَّةً وَذَهَباً وَلاَ نِهَايَةَ لِكُنُوزِهِمْ وَامْتَلَأَتْ أَرْضُهُمْ خَيْلاً وَلاَ نِهَايَةَ لِمَرْكَبَاتِهِمْ.
\par 8 وَامْتَلَأَتْ أَرْضُهُمْ أَوْثَاناً. يَسْجُدُونَ لِعَمَلِ أَيْدِيهِمْ لِمَا صَنَعَتْهُ أَصَابِعُهُمْ.
\par 9 وَيَنْخَفِضُ الإِنْسَانُ وَيَنْطَرِحُ الرَّجُلُ فَلاَ تَغْفِرْ لَهُمْ.
\par 10 اُدْخُلْ إِلَى الصَّخْرَةِ وَاخْتَبِئْ فِي التُّرَابِ مِنْ أَمَامِ هَيْبَةِ الرَّبِّ وَمِنْ بَهَاءِ عَظَمَتِهِ.
\par 11 تُوضَعُ عَيْنَا تَشَامُخِ الإِنْسَانِ وَتُخْفَضُ رِفْعَةُ النَّاسِ وَيَسْمُو الرَّبُّ وَحْدَهُ فِي ذَلِكَ الْيَوْمِ.
\par 12 فَإِنَّ لِرَبِّ الْجُنُودِ يَوْماً عَلَى كُلِّ مُتَعَظِّمٍ وَعَالٍ وَعَلَى كُلِّ مُرْتَفِعٍ فَيُوضَعُ
\par 13 وَعَلَى كُلِّ أَرْزِ لُبْنَانَ الْعَالِي الْمُرْتَفِعِ وَعَلَى كُلِّ بَلُّوطِ بَاشَانَ
\par 14 وَعَلَى كُلِّ الْجِبَالِ الْعَالِيَةِ وَعَلَى كُلِّ التِّلاَلِ الْمُرْتَفِعَةِ
\par 15 وَعَلَى كُلِّ بُرْجٍ عَالٍ وَعَلَى كُلِّ سُورٍ مَنِيعٍ
\par 16 وَعَلَى كُلِّ سُفُنِ تَرْشِيشَ وَعَلَى كُلِّ الأَعْلاَمِ الْبَهِجَةِ.
\par 17 فَيُخْفَضُ تَشَامُخُ الإِنْسَانِ وَتُوضَعُ رِفْعَةُ النَّاسِ وَيَسْمُو الرَّبُّ وَحْدَهُ فِي ذَلِكَ الْيَوْمِ.
\par 18 وَتَزُولُ الأَوْثَانُ بِتَمَامِهَا.
\par 19 وَيَدْخُلُونَ فِي مَغَايِرِ الصُّخُورِ وَفِي حَفَائِرِ التُّرَابِ مِنْ أَمَامِ هَيْبَةِ الرَّبِّ وَمِنْ بَهَاءِ عَظَمَتِهِ عِنْدَ قِيَامِهِ لِيُرْعِبَ الأَرْضَ.
\par 20 فِي ذَلِكَ الْيَوْمِ يَطْرَحُ الإِنْسَانُ أَوْثَانَهُ الْفَضِّيَّةَ وَأَوْثَانَهُ الذَّهَبِيَّةَ الَّتِي عَمِلُوهَا لَهُ لِلسُّجُودِ لِلْجُرْذَانِ وَالْخَفَافِيشِ
\par 21 لِيَدْخُلَ فِي نُقَرِ الصُّخُورِ وَفِي شُقُوقِ الْمَعَاقِلِ مِنْ أَمَامِ هَيْبَةِ الرَّبِّ وَمِنْ بَهَاءِ عَظَمَتِهِ عِنْدَ قِيَامِهِ لِيُرْعِبَ الأَرْضَ.
\par 22 كُفُّوا عَنِ الإِنْسَانِ الَّذِي فِي أَنْفِهِ نَسَمَةٌ لأَنَّهُ مَاذَا يُحْسَبُ؟

\chapter{3}

\par 1 فَإِنَّهُ هُوَذَا السَّيِّدُ رَبُّ الْجُنُودِ يَنْزِعُ مِنْ أُورُشَلِيمَ وَمِنْ يَهُوذَا السَّنَدَ وَالرُّكْنَ كُلَّ سَنَدِ خُبْزٍ وَكُلَّ سَنَدِ مَاءٍ.
\par 2 الْجَبَّارَ وَرَجُلَ الْحَرْبِ. الْقَاضِي وَالنَّبِيَّ وَالْعَرَّافَ وَالشَّيْخَ.
\par 3 رَئِيسَ الْخَمْسِينَ وَالْمُعْتَبَرَ وَالْمُشِيرَ وَالْمَاهِرَ بَيْنَ الصُّنَّاعِ وَالْحَاذِقَ بِالرُّقْيَةِ.
\par 4 وَأَجْعَلُ صِبْيَاناً رُؤَسَاءَ لَهُمْ وَأَطْفَالاً تَتَسَلَّطُ عَلَيْهِمْ.
\par 5 وَيَظْلِمُ الشَّعْبُ بَعْضُهُمْ بَعْضاً وَالرَّجُلُ صَاحِبَهُ. يَتَمَرَّدُ الصَّبِيُّ عَلَى الشَّيْخِ وَالدَّنِيءُ عَلَى الشَّرِيفِ.
\par 6 إِذَا أَمْسَكَ إِنْسَانٌ بِأَخِيهِ فِي بَيْتِ أَبِيهِ قَائِلاً: «لَكَ ثَوْبٌ فَتَكُونُ لَنَا رَئِيساً وَهَذَا الْخَرَابُ تَحْتَ يَدِكَ»
\par 7 يَرْفَعُ صَوْتَهُ فِي ذَلِكَ الْيَوْمِ قَائِلاً: «لاَ أَكُونُ عَاصِباً وَفِي بَيْتِي لاَ خُبْزَ وَلاَ ثَوْبَ. لاَ تَجْعَلُونِي رَئِيسَ الشَّعْبِ».
\par 8 لأَنَّ أُورُشَلِيمَ عَثَرَتْ وَيَهُوذَا سَقَطَتْ لأَنَّ لِسَانَهُمَا وَأَفْعَالَهُمَا ضِدَّ الرَّبِّ لإِغَاظَةِ عَيْنَيْ مَجْدِهِ.
\par 9 نَظَرُ وُجُوهِهِمْ يَشْهَدُ عَلَيْهِمْ وَهُمْ يُخْبِرُونَ بِخَطِيَّتِهِمْ كَسَدُومَ. لاَ يُخْفُونَهَا. وَيْلٌ لِنُفُوسِهِمْ لأَنَّهُمْ يَصْنَعُونَ لأَنْفُسِهِمْ شَرّاً.
\par 10 «قُولُوا لِلصِّدِّيقِ خَيْرٌ! لأَنَّهُمْ يَأْكُلُونَ ثَمَرَ أَفْعَالِهِمْ.
\par 11 وَيْلٌ لِلشِّرِّيرِ. شَرٌّ! لأَنَّ مُجَازَاةَ يَدَيْهِ تُعْمَلُ بِهِ.
\par 12 شَعْبِي ظَالِمُوهُ أَوْلاَدٌ وَنِسَاءٌ يَتَسَلَّطْنَ عَلَيْهِ. يَا شَعْبِي مُرْشِدُوكَ مُضِلُّونَ وَيَبْلَعُونَ طَرِيقَ مَسَالِكِكَ».
\par 13 قَدِ انْتَصَبَ الرَّبُّ لِلْمُخَاصَمَةِ وَهُوَ قَائِمٌ لِدَيْنُونَةِ الشُّعُوبِ.
\par 14 اَلرَّبُّ يَدْخُلُ فِي الْمُحَاكَمَةِ مَعَ شُيُوخِ شَعْبِهِ وَرُؤَسَائِهِمْ: «وَأَنْتُمْ قَدْ أَكَلْتُمُ الْكَرْمَ. سَلْبُ الْبَائِسِ فِي بُيُوتِكُمْ.
\par 15 مَا لَكُمْ تَسْحَقُونَ شَعْبِي وَتَطْحَنُونَ وُجُوهَ الْبَائِسِينَ؟» يَقُولُ السَّيِّدُ رَبُّ الْجُنُودِ.
\par 16 وَقَالَ الرَّبُّ: «مِنْ أَجْلِ أَنَّ بَنَاتِ صِهْيَوْنَ يَتَشَامَخْنَ وَيَمْشِينَ مَمْدُودَاتِ الأَعْنَاقِ وَغَامِزَاتٍ بِعُيُونِهِنَّ وَخَاطِرَاتٍ فِي مَشْيِهِنَّ وَيُخَشْخِشْنَ بِأَرْجُلِهِنَّ
\par 17 يُصْلِعُ السَّيِّدُ هَامَةَ بَنَاتِ صِهْيَوْنَ وَيُعَرِّي الرَّبُّ عَوْرَتَهُنَّ.
\par 18 يَنْزِعُ السَّيِّدُ فِي ذَلِكَ الْيَوْمِ زِينَةَ الْخَلاَخِيلِ وَالضَّفَائِرِ وَالأَهِلَّةِ
\par 19 وَالْحَلَقِ وَالأَسَاوِرِ وَالْبَرَاقِعِ
\par 20 وَالْعَصَائِبِ وَالسَّلاَسِلِ وَالْمَنَاطِقِ وَحَنَاجِرِ الشَّمَّامَاتِ وَالأَحْرَازِ
\par 21 وَالْخَوَاتِمِ وَخَزَائِمِ الأَنْفِ
\par 22 وَالثِّيَابِ الْمُزَخْرَفَةِ وَالْعُطْفِ وَالأَرْدِيَةِ وَالأَكْيَاسِ
\par 23 وَالْمَرَائِي وَالْقُمْصَانِ وَالْعَمَائِمِ وَالأُزُرِ.
\par 24 فَيَكُونُ عِوَضَ الطِّيبِ عُفُونَةٌ وَعِوَضَ الْمِنْطَقَةِ حَبْلٌ وَعِوَضَ الْجَدَائِلِ قَرْعَةٌ وَعِوَضَ الدِّيبَاجِ زُنَّارُ مِسْحٍ وَعِوَضَ الْجَمَالِ كَيٌّ!
\par 25 رِجَالُكِ يَسْقُطُونَ بِالسَّيْفِ وَأَبْطَالُكِ فِي الْحَرْبِ.
\par 26 فَتَئِنُّ وَتَنُوحُ أَبْوَابُهَا وَهِيَ فَارِغَةً تَجْلِسُ عَلَى الأَرْضِ.

\chapter{4}

\par 1 فَتُمْسِكُ سَبْعُ نِسَاءٍ بِرَجُلٍ وَاحِدٍ فِي ذَلِكَ الْيَوْمِ قَائِلاَتٍ: «نَأْكُلُ خُبْزَنَا وَنَلْبِسُ ثِيَابَنَا. لِيُدْعَ فَقَطِ اسْمُكَ عَلَيْنَا. انْزِعْ عَارَنَا».
\par 2 فِي ذَلِكَ الْيَوْمِ يَكُونُ غُصْنُ الرَّبِّ بَهَاءً وَمَجْداً وَثَمَرُ الأَرْضِ فَخْراً وَزِينَةً لِلنَّاجِينَ مِنْ إِسْرَائِيلَ.
\par 3 وَيَكُونُ أَنَّ الَّذِي يَبْقَى فِي صِهْيَوْنَ وَالَّذِي يُتْرَكُ فِي أُورُشَلِيمَ يُسَمَّى قُدُّوساً. كُلُّ مَنْ كُتِبَ لِلْحَيَاةِ فِي أُورُشَلِيمَ.
\par 4 إِذَا غَسَلَ السَّيِّدُ قَذَرَ بَنَاتِ صِهْيَوْنَ وَنَقَّى دَمَ أُورُشَلِيمَ مِنْ وَسَطِهَا بِرُوحِ الْقَضَاءِ وَبِرُوحِ الإِحْرَاقِ
\par 5 يَخْلُقُ الرَّبُّ عَلَى كُلِّ مَكَانٍ مِنْ جَبَلِ صِهْيَوْنَ وَعَلَى مَحْفَلِهَا سَحَابَةً نَهَاراً وَدُخَاناً وَلَمَعَانَ نَارٍ مُلْتَهِبَةٍ لَيْلاً. لأَنَّ عَلَى كُلِّ مَجْدٍ غِطَاءً.
\par 6 وَتَكُونُ مَظَلَّةٌ لِلْفَيْءِ نَهَاراً مِنَ الْحَرِّ وَلِمَلْجَأٍ وَمَخْبَأٍ مِنَ السَّيْلِ وَمِنَ الْمَطَرِ.

\chapter{5}

\par 1 لأُنْشِدَنَّ عَنْ حَبِيبِي نَشِيدَ مُحِبِّي لِكَرْمِهِ. كَانَ لِحَبِيبِي كَرْمٌ عَلَى أَكَمَةٍ خَصِبَةٍ
\par 2 فَنَقَبَهُ وَنَقَّى حِجَارَتَهُ وَغَرَسَهُ كَرْمَ سَوْرَقَ وَبَنَى بُرْجاً فِي وَسَطِهِ وَنَقَرَ فِيهِ أَيْضاً مِعْصَرَةً فَانْتَظَرَ أَنْ يَصْنَعَ عِنَباً فَصَنَعَ عِنَباً رَدِيئاً.
\par 3 «وَالآنَ يَا سُكَّانَ أُورُشَلِيمَ وَرِجَالَ يَهُوذَا احْكُمُوا بَيْنِي وَبَيْنَ كَرْمِي.
\par 4 مَاذَا يُصْنَعُ أَيْضاً لِكَرْمِي وَأَنَا لَمْ أَصْنَعْهُ لَهُ؟ لِمَاذَا إِذِ انْتَظَرْتُ أَنْ يَصْنَعَ عِنَباً صَنَعَ عِنَباً رَدِيئاً؟
\par 5 فَالآنَ أُعَرِّفُكُمْ مَاذَا أَصْنَعُ بِكَرْمِي. أَنْزِعُ سِيَاجَهُ فَيَصِيرُ لِلرَّعْيِ. أَهْدِمُ جُدْرَانَهُ فَيَصِيرُ لِلدَّوْسِ.
\par 6 وَأَجْعَلُهُ خَرَاباً لاَ يُقْضَبُ وَلاَ يُنْقَبُ فَيَطْلَعُ شَوْكٌ وَحَسَكٌ. وَأُوصِي الْغَيْمَ أَنْ لاَ يُمْطِرَ عَلَيْهِ مَطَراً».
\par 7 إِنَّ كَرْمَ رَبِّ الْجُنُودِ هُوَ بَيْتُ إِسْرَائِيلَ وَغَرْسَ لَذَّتِهِ رِجَالُ يَهُوذَا. فَانْتَظَرَ حَقّاً فَإِذَا سَفْكُ دَمٍ وَعَدْلاً فَإِذَا صُرَاخٌ.
\par 8 وَيْلٌ لِلَّذِينَ يَصِلُونَ بَيْتاً بِبَيْتٍ وَيَقْرِنُونَ حَقْلاً بِحَقْلٍ حَتَّى لَمْ يَبْقَ مَوْضِعٌ. فَصِرْتُمْ تَسْكُنُونَ وَحْدَكُمْ فِي وَسَطِ الأَرْضِ.
\par 9 فِي أُذُنَيَّ قَالَ رَبُّ الْجُنُودِ: «أَلاَ إِنَّ بُيُوتاً كَثِيرَةً تَصِيرُ خَرَاباً. بُيُوتاً كَبِيرَةً وَحَسَنَةً بِلاَ سَاكِنٍ.
\par 10 لأَنَّ عَشَرَةَ فَدَادِينِ كَرْمٍ تَصْنَعُ بَثّاً وَاحِداً وَحُومَرَ بِذَارٍ يَصْنَعُ إِيفَةً».
\par 11 وَيْلٌ لِلْمُبَكِّرِينَ صَبَاحاً يَتْبَعُونَ الْمُسْكِرَ لِلْمُتَأَخِّرِينَ فِي الْعَتَمَةِ تُلْهِبُهُمُ الْخَمْرُ.
\par 12 وَصَارَ الْعُودُ وَالرَّبَابُ وَالدُّفُّ وَالنَّايُ وَالْخَمْرُ وَلاَئِمَهُمْ وَإِلَى فَعْلِ الرَّبِّ لاَ يَنْظُرُونَ وَعَمَلَ يَدَيْهِ لاَ يَرُونَ.
\par 13 لِذَلِكَ سُبِيَ شَعْبِي لِعَدَمِ الْمَعْرِفَةِ وَتَصِيرُ شُرَفَاؤُهُ رِجَالَ جُوعٍ وَعَامَّتُهُ يَابِسِينَ مِنَ الْعَطَشِ.
\par 14 لِذَلِكَ وَسَّعَتِ الْهَاوِيَةُ نَفْسَهَا وَفَغَرَتْ فَمَهَا بِلاَ حَدٍّ فَيَنْزِلُ بَهَاؤُهَا وَجُمْهُورُهَا وَضَجِيجُهَا وَالْمُبْتَهِجُ فِيهَا!
\par 15 وَيُذَلُّ الإِنْسَانُ وَيُحَطُّ الرَّجُلُ وَعُيُونُ الْمُسْتَعْلِينَ تُوضَعُ.
\par 16 وَيَتَعَالَى رَبُّ الْجُنُودِ بِالْعَدْلِ وَيَتَقَدَّسُ الإِلَهُ الْقُدُّوسُ بِالْبِرِّ.
\par 17 وَتَرْعَى الْخِرْفَانُ حَيْثُمَا تُسَاقُ وَخِرَبُ السِّمَانِ تَأْكُلُهَا الْغُرَبَاءُ.
\par 18 وَيْلٌ لِلْجَاذِبِينَ الإِثْمَ بِحِبَالِ الْبُطْلِ وَالْخَطِيَّةَ كَأَنَّهُ بِرُبُطِ الْعَجَلَةِ
\par 19 الْقَائِلِينَ: «لِيُسْرِعْ. لِيُعَجِّلْ عَمَلَهُ لِكَيْ نَرَى وَلْيَقْرُبْ وَيَأْتِ مَقْصَدُ قُدُّوسِ إِسْرَائِيلَ لِنَعْلَمَ».
\par 20 وَيْلٌ لِلْقَائِلِينَ لِلشَّرِّ خَيْراً وَلِلْخَيْرِ شَرّاً الْجَاعِلِينَ الظَّلاَمَ نُوراً وَالنُّورَ ظَلاَماً الْجَاعِلِينَ الْمُرَّ حُلْواً وَالْحُلْوَ مُرّاً.
\par 21 وَيْلٌ لِلْحُكَمَاءِ فِي أَعْيُنِ أَنْفُسِهِمْ وَالْفُهَمَاءِ عِنْدَ ذَوَاتِهِمْ.
\par 22 وَيْلٌ لِلأَبْطَالِ عَلَى شُرْبِ الْخَمْرِ وَلِذَوِي الْقُدْرَةِ عَلَى مَزْجِ الْمُسْكِرِ.
\par 23 الَّذِينَ يُبَرِّرُونَ الشِّرِّيرَ مِنْ أَجْلِ الرَّشْوَةِ. وَأَمَّا حَقُّ الصِّدِّيقِينَ فَيَنْزِعُونَهُ مِنْهُمْ.
\par 24 لِذَلِكَ كَمَا يَأْكُلُ لَهِيبُ النَّارِ الْقَشَّ وَيَهْبِطُ الْحَشِيشُ الْمُلْتَهِبُ يَكُونُ أَصْلُهُمْ كَالْعُفُونَةِ وَيَصْعَدُ زَهْرُهُمْ كَالْغُبَارِ لأَنَّهُمْ رَذَلُوا شَرِيعَةَ رَبِّ الْجُنُودِ وَاسْتَهَانُوا بِكَلاَمِ قُدُّوسِ إِسْرَائِيلَ.
\par 25 مِنْ أَجْلِ ذَلِكَ حَمِيَ غَضَبُ الرَّبِّ عَلَى شَعْبِهِ وَمَدَّ يَدَهُ عَلَيْهِ وَضَرَبَهُ حَتَّى ارْتَعَدَتِ الْجِبَالُ وَصَارَتْ جُثَثُهُمْ كَالزِّبْلِ فِي الأَزِقَّةِ. مَعَ كُلِّ هَذَا لَمْ يَرْتَدَّ غَضَبُهُ بَلْ يَدُهُ مَمْدُودَةٌ بَعْدُ.
\par 26 فَيَرْفَعُ رَايَةً لِلأُمَمِ مِنْ بَعِيدٍ وَيَصْفِرُ لَهُمْ مِنْ أَقْصَى الأَرْضِ فَإِذَا هُمْ بِالْعَجَلَةِ يَأْتُونَ سَرِيعاً.
\par 27 لَيْسَ فِيهِمْ رَازِحٌ وَلاَ عَاثِرٌ. لاَ يَنْعَسُونَ وَلاَ يَنَامُونَ وَلاَ تَنْحَلُّ حُزُمُ أَحْقَائِهِمْ وَلاَ تَنْقَطِعُ سُيُورُ أَحْذِيَتِهِمُِ.
\par 28 الَّذِينَ سِهَامُهُمْ مَسْنُونَةٌ وَجَمِيعُ قِسِيِّهِمْ مَمْدُودَةٌ. حَوَافِرُ خَيْلِهِمْ تُحْسَبُ كَالصَّوَّانِ وَبَكَرَاتُهُمْ كَالزَّوْبَعَةِ.
\par 29 لَهُمْ زَمْجَرَةٌ كَاللَّبْوَةِ وَيُزَمْجِرُونَ كَالشِّبْلِ وَيَهِرُّونَ وَيُمْسِكُونَ الْفَرِيسَةَ وَيَسْتَخْلِصُونَهَا وَلاَ مُنْقِذَ.
\par 30 يَهِرُّونَ عَلَيْهِمْ فِي ذَلِكَ الْيَوْمِ كَهَدِيرِ الْبَحْرِ. فَإِنْ نُظِرَ إِلَى الأَرْضِ فَهُوَذَا ظَلاَمُ الضِّيقِ وَالنُّورُ قَدْ أَظْلَمَ بِسُحُبِهَا.

\chapter{6}

\par 1 فِي سَنَةِ وَفَاةِ عُزِّيَّا الْمَلِكِ رَأَيْتُ السَّيِّدَ جَالِساً عَلَى كُرْسِيٍّ عَالٍ وَمُرْتَفِعٍ وَأَذْيَالُهُ تَمْلَأُ الْهَيْكَلَ.
\par 2 السَّرَافِيمُ وَاقِفُونَ فَوْقَهُ لِكُلِّ وَاحِدٍ سِتَّةُ أَجْنِحَةٍ. بِاثْنَيْنِ يُغَطِّي وَجْهَهُ وَبِاثْنَيْنِ يُغَطِّي رِجْلَيْهِ وَبَاثْنَيْنِ يَطِيرُ.
\par 3 وَهَذَا نَادَى ذَاكَ: «قُدُّوسٌ قُدُّوسٌ قُدُّوسٌ رَبُّ الْجُنُودِ. مَجْدُهُ مِلْءُ كُلِّ الأَرْضِ».
\par 4 فَاهْتَزَّتْ أَسَاسَاتُ الْعَتَبِ مِنْ صَوْتِ الصَّارِخِ وَامْتَلَأَ الْبَيْتُ دُخَاناً.
\par 5 فَقُلْتُ: «وَيْلٌ لِي! إِنِّي هَلَكْتُ لأَنِّي إِنْسَانٌ نَجِسُ الشَّفَتَيْنِ وَأَنَا سَاكِنٌ بَيْنَ شَعْبٍ نَجِسِ الشَّفَتَيْنِ لأَنَّ عَيْنَيَّ قَدْ رَأَتَا الْمَلِكَ رَبَّ الْجُنُودِ».
\par 6 فَطَارَ إِلَيَّ وَاحِدٌ مِنَ السَّرَافِيمِ وَبِيَدِهِ جَمْرَةٌ قَدْ أَخَذَهَا بِمِلْقَطٍ مِنْ عَلَى الْمَذْبَحِ
\par 7 وَمَسَّ بِهَا فَمِي وَقَالَ: «إِنَّ هَذِهِ قَدْ مَسَّتْ شَفَتَيْكَ فَانْتُزِعَ إِثْمُكَ وَكُفِّرَ عَنْ خَطِيَّتِكَ».
\par 8 ثُمَّ سَمِعْتُ صَوْتَ السَّيِّدِ: «مَنْ أُرْسِلُ وَمَنْ يَذْهَبُ مِنْ أَجْلِنَا؟» فَأَجَبْتُ: «هَئَنَذَا أَرْسِلْنِي».
\par 9 فَقَالَ: «اذْهَبْ وَقُلْ لِهَذَا الشَّعْبِ: اسْمَعُوا سَمْعاً وَلاَ تَفْهَمُوا وَأَبْصِرُوا إِبْصَاراً وَلاَ تَعْرِفُوا.
\par 10 غَلِّظْ قَلْبَ هَذَا الشَّعْبِ وَثَقِّلْ أُذُنَيْهِ وَاطْمُسْ عَيْنَيْهِ لِئَلاَّ يُبْصِرَ بِعَيْنَيْهِ وَيَسْمَعَ بِأُذُنَيْهِ وَيَفْهَمْ بِقَلْبِهِ وَيَرْجِعَ فَيُشْفَى».
\par 11 فَسَأَلْتُ: «إِلَى مَتَى أَيُّهَا السَّيِّدُ؟» فَقَالَ: «إِلَى أَنْ تَصِيرَ الْمُدُنُ خَرِبَةً بِلاَ سَاكِنٍ وَالْبُيُوتُ بِلاَ إِنْسَانٍ وَتَخْرَبَ الأَرْضُ وَتُقْفِرَ
\par 12 وَيُبْعِدَ الرَّبُّ الإِنْسَانَ وَيَكْثُرُ الْخَرَابُ فِي وَسَطِ الأَرْضِ.
\par 13 وَإِنْ بَقِيَ فِيهَا عُشْرٌ بَعْدُ فَيَعُودُ وَيَصِيرُ لِلْخَرَابِ وَلَكِنْ كَالْبُطْمَةِ وَالْبَلُّوطَةِ الَّتِي وَإِنْ قُطِعَتْ فَلَهَا سَاقٌ يَكُونُ سَاقُهُ زَرْعاً مُقَدَّساً».

\chapter{7}

\par 1 وَحَدَثَ فِي أَيَّامِ آحَازَ بْنِ يُوثَامَ بْنِ عُزِّيَّا مَلِكِ يَهُوذَا أَنَّ رَصِينَ مَلِكَ أَرَامَ صَعِدَ مَعَ فَقْحَ بْنِ رَمَلْيَا مَلِكِ إِسْرَائِيلَ إِلَى أُورُشَلِيمَ لِمُحَارَبَتِهَا فَلَمْ يَقْدِرْ أَنْ يُحَارِبَهَا.
\par 2 وَأُخْبِرَ بَيْتُ دَاوُدَ: «قَدْ حَلَّتْ أَرَامُ فِي أَفْرَايِمَ». فَرَجَفَ قَلْبُهُ وَقُلُوبُ شَعْبِهِ كَرَجَفَانِ شَجَرِ الْوَعْرِ قُدَّامَ الرِّيحِ.
\par 3 فَقَالَ الرَّبُّ لإِشَعْيَاءَ: «اخْرُجْ لِمُلاَقَاةِ آحَازَ أَنْتَ وَشَآرَ يَاشُوبَ ابْنُكَ إِلَى طَرَفِ قَنَاةِ الْبِرْكَةِ الْعُلْيَا إِلَى سِكَّةِ حَقْلِ الْقَصَّارِ
\par 4 وَقُلْ لَهُ: احْتَرِزْ وَاهْدَأْ. لاَ تَخَفْ وَلاَ يَضْعُفْ قَلْبُكَ مِنْ أَجْلِ ذَنَبَيْ هَاتَيْنِ الشُّعْلَتَيْنِ الْمُدَخِّنَتَيْنِ بِحُمُوِّ غَضَبِ رَصِينَ وَأَرَامَ وَابْنِ رَمَلْيَا.
\par 5 لأَنَّ أَرَامَ تَآمَرَتْ عَلَيْكَ بِشَرٍّ مَعَ أَفْرَايِمَ وَابْنِ رَمَلْيَا قَائِلَةً:
\par 6 نَصْعَدُ عَلَى يَهُوذَا وَنُقَوِّضُهَا وَنَسْتَفْتِحُهَا لأَنْفُسِنَا وَنُمَلِّكُ فِي وَسَطِهَا مَلِكاً ابْنَ طَبْئِيلَ.
\par 7 هَكَذَا يَقُولُ السَّيِّدُ الرَّبُّ: لاَ تَقُومُ! لاَ تَكُونُ!
\par 8 لأَنَّ رَأْسَ أَرَامَ دِمَشْقَ وَرَأْسَ دِمَشْقَ رَصِينُ. وَفِي مُدَّةِ خَمْسٍ وَسِتِّينَ سَنَةً يَنْكَسِرُ أَفْرَايِمُ حَتَّى لاَ يَكُونَ شَعْباً.
\par 9 وَرَأْسُ أَفْرَايِمَ السَّامِرَةُ وَرَأْسُ السَّامِرَةِ ابْنُ رَمَلْيَا. إِنْ لَمْ تُؤْمِنُوا فَلاَ تَأْمَنُوا».
\par 10 ثُمَّ عَادَ الرَّبُّ فَقَالَ لِآحَازَ:
\par 11 «اُطْلُبْ لِنَفْسِكَ آيَةً مِنَ الرَّبِّ إِلَهِكَ. عَمِّقْ طَلَبَكَ أَوْ رَفِّعْهُ إِلَى فَوْقٍ».
\par 12 فَقَالَ آحَازُ: «لاَ أَطْلُبُ وَلاَ أُجَرِّبُ الرَّبَّ».
\par 13 فَقَالَ: «اسْمَعُوا يَا بَيْتَ دَاوُدَ. هَلْ هُوَ قَلِيلٌ عَلَيْكُمْ أَنْ تُضْجِرُوا النَّاسَ حَتَّى تُضْجِرُوا إِلَهِي أَيْضاً؟
\par 14 وَلَكِنْ يُعْطِيكُمُ السَّيِّدُ نَفْسُهُ آيَةً: هَا الْعَذْرَاءُ تَحْبَلُ وَتَلِدُ ابْناً وَتَدْعُو اسْمَهُ «عِمَّانُوئِيلَ».
\par 15 زُبْداً وَعَسَلاً يَأْكُلُ مَتَى عَرَفَ أَنْ يَرْفُضَ الشَّرَّ وَيَخْتَارَ الْخَيْرَ.
\par 16 لأَنَّهُ قَبْلَ أَنْ يَعْرِفَ الصَّبِيُّ أَنْ يَرْفُضَ الشَّرَّ وَيَخْتَارَ الْخَيْرَ تُخْلَى الأَرْضُ الَّتِي أَنْتَ خَاشٍ مِنْ مَلِكَيْهَا».
\par 17 يَجْلِبُ الرَّبُّ مَلِكَ أَشُّورَ عَلَيْكَ وَعَلَى شَعْبِكَ وَعَلَى بَيْتِ أَبِيكَ أَيَّاماً لَمْ تَأْتِ مُنْذُ يَوْمِ اعْتِزَالِ أَفْرَايِمَ عَنْ يَهُوذَا.
\par 18 وَيَكُونُ فِي ذَلِكَ الْيَوْمِ أَنَّ الرَّبَّ يَصْفِرُ لِلذُّبَابِ الَّذِي فِي أَقْصَى تُرَعِ مِصْرَ وَلِلنَّحْلِ الَّذِي فِي أَرْضِ أَشُّورَ
\par 19 فَتَأْتِي وَتَحِلُّ جَمِيعُهَا فِي الأَوْدِيَةِ الْخَرِبَةِ وَفِي شُقُوقِ الصُّخُورِ وَفِي كُلِّ غَابِ الشَّوْكِ وَفِي كُلِّ الْمَرَاعِي.
\par 20 فِي ذَلِكَ الْيَوْمِ يَحْلِقُ السَّيِّدُ بِمُوسَى مُسْتَأْجَرَةٍ فِي عَبْرِ النَّهْرِ بِمَلِكِ أَشُّورَ الرَّأْسَ وَشَعْرَ الرِّجْلَيْنِ وَتَنْزِعُ اللِّحْيَةَ أَيْضاً.
\par 21 وَيَكُونُ فِي ذَلِكَ الْيَوْمِ أَنَّ الإِنْسَانَ يُرَبِّي عِجْلَةَ بَقَرٍ وَشَاتَيْنِ.
\par 22 وَيَكُونُ أَنَّهُ مِنْ كَثْرَةِ صُنْعِهَا اللَّبَنَ يَأْكُلُ زُبْداً فَإِنَّ كُلَّ مَنْ أُبْقِيَ فِي الأَرْضِ يَأْكُلُ زُبْداً وَعَسَلاً.
\par 23 وَيَكُونُ فِي ذَلِكَ الْيَوْمِ أَنَّ كُلَّ مَوْضِعٍ كَانَ فِيهِ أَلْفُ جَفْنَةٍ بِأَلْفٍ مِنَ الْفِضَّةِ يَكُونُ لِلشَّوْكِ وَالْحَسَكِ.
\par 24 بِالسِّهَامِ وَالْقَوْسِ يُؤْتَى إِلَى هُنَاكَ لأَنَّ كُلَّ الأَرْضِ تَكُونُ شَوْكاً وَحَسَكاً.
\par 25 وَجَمِيعُ الْجِبَالِ الَّتِي تُنْقَبُ بِالْمِعْوَلِ لاَ يُؤْتَى إِلَيْهَا خَوْفاً مِنَ الشَّوْكِ وَالْحَسَكِ فَتَكُونُ لِسَرْحِ الْبَقَرِ وَلِدَوْسِ الْغَنَمِ.

\chapter{8}

\par 1 وَقَالَ لِي الرَّبُّ: «خُذْ لِنَفْسِكَ لَوْحاً كَبِيراً وَاكْتُبْ عَلَيْهِ بِقَلَمِ إِنْسَانٍ: لِمَهَيْرَ شَلاَلَ حَاشَ بَزَ.
\par 2 وَأَنْ أُشْهِدَ لِنَفْسِي شَاهِدَيْنِ أَمِينَيْنِ: أُورِيَّا الْكَاهِنَ وَزَكَرِيَّا بْنَ يَبْرَخْيَا».
\par 3 فَاقْتَرَبْتُ إِلَى النَّبِيَّةِ فَحَبِلَتْ وَوَلَدَتِ ابْناً. فَقَالَ لِي الرَّبُّ: «ادْعُ اسْمَهُ مَهَيْرَ شَلاَلَ حَاشَ بَزَ.
\par 4 لأَنَّهُ قَبْلَ أَنْ يَعْرِفَ الصَّبِيُّ أَنْ يَدْعُوَ: يَا أَبِي وَيَا أُمِّي تُحْمَلُ ثَرْوَةُ دِمَشْقَ وَغَنِيمَةُ السَّامِرَةِ قُدَّامَ مَلِكِ أَشُّورَ».
\par 5 ثُمَّ عَادَ الرَّبُّ أَيْضاً يَقولُ لِي:
\par 6 «لأَنَّ هَذَا الشَّعْبَ رَذَلَ مِيَاهَ شِيلُوهَ الْجَارِيَةَ بِسُكُوتٍ وَسُرَّ بِرَصِينَ وَابْنِ رَمَلْيَا.
\par 7 لِذَلِكَ هُوَذَا السَّيِّدُ يُصْعِدُ عَلَيْهِمْ مِيَاهَ النَّهْرِ الْقَوِيَّةَ وَالْكَثِيرَةَ مَلِكَ أَشُّورَ وَكُلَّ مَجْدِهِ فَيَصْعَدُ فَوْقَ جَمِيعِ مَجَارِيهِ وَيَجْرِي فَوْقَ جَمِيعِ شُطُوطِهِ
\par 8 وَيَنْدَفِقُ إِلَى يَهُوذَا. يَفِيضُ وَيَعْبُرُ. يَبْلُغُ الْعُنُقَ. وَيَكُونُ بَسْطُ جَنَاحَيْهِ مِلْءَ عَرْضِ بِلاَدِكَ يَا عِمَّانُوئِيلُ».
\par 9 هِيجُوا أَيُّهَا الشُّعُوبُ وَانْكَسِرُوا وَأَصْغِي يَا جَمِيعَ أَقَاصِي الأَرْضِ. احْتَزِمُوا وَانْكَسِرُوا! احْتَزِمُوا وَانْكَسِرُوا!
\par 10 تَشَاوَرُوا مَشُورَةً فَتَبْطُلَ. تَكَلَّمُوا كَلِمَةً فَلاَ تَقُومُ. لأَنَّ اللَّهَ مَعَنَا.
\par 11 فَإِنَّهُ هَكَذَا قَالَ لِي الرَّبُّ بِشِدَّةِ الْيَدِ وَأَنْذَرَنِي أَنْ لاَ أَسْلُكَ فِي طَرِيقِ هَذَا الشَّعْبِ قَائِلاً:
\par 12 «لاَ تَقُولُوا: فِتْنَةً لِكُلِّ مَا يَقُولُ لَهُ هَذَا الشَّعْبُ فِتْنَةً وَلاَ تَخَافُوا خَوْفَهُ وَلاَ تَرْهَبُوا.
\par 13 قَدِّسُوا رَبَّ الْجُنُودِ فَهُوَ خَوْفُكُمْ وَهُوَ رَهْبَتُكُمْ.
\par 14 وَيَكُونُ مَقْدِساً وَحَجَرَ صَدْمَةٍ وَصَخْرَةَ عَثْرَةٍ لِبَيْتَيْ إِسْرَائِيلَ وَفَخّاً وَشَرَكاً لِسُكَّانِ أُورُشَلِيمَ.
\par 15 فَيَعْثُرُ بِهَا كَثِيرُونَ وَيَسْقُطُونَ فَيَنْكَسِرُونَ وَيَعْلَقُونَ فَيُلْقَطُونَ.
\par 16 صُرَّ الشَّهَادَةَ. اخْتِمِ الشَّرِيعَةَ بِتَلاَمِيذِي».
\par 17 فَأَصْطَبِرُ لِلرَّبِّ السَّاتِرِ وَجْهَهُ عَنْ بَيْتِ يَعْقُوبَ وَأَنْتَظِرُهُ.
\par 18 هَئَنَذَا وَالأَوْلاَدُ الَّذِينَ أَعْطَانِيهِمُ الرَّبُّ آيَاتٍ وَعَجَائِبَ فِي إِسْرَائِيلَ مِنْ عِنْدِ رَبِّ الْجُنُودِ السَّاكِنِ فِي جَبَلِ صِهْيَوْنَ.
\par 19 وَإِذَا قَالُوا لَكُمُ: «اطْلُبُوا إِلَى أَصْحَابِ التَّوَابِعِ وَالْعَرَّافِينَ الْمُشَقْشِقِينَ وَالْهَامِسِينَ». أَلاَ يَسْأَلُ شَعْبٌ إِلَهَهُ؟ أَيُسْأَلُ الْمَوْتَى لأَجْلِ الأَحْيَاءِ؟
\par 20 إِلَى الشَّرِيعَةِ وَإِلَى الشَّهَادَةِ. إِنْ لَمْ يَقُولُوا مِثْلَ هَذَا الْقَوْلِ فَلَيْسَ لَهُمْ فَجْرٌ!
\par 21 فَيَعْبُرُونَ فِيهَا مُضَايَقِينَ وَجَائِعِينَ. وَيَكُونُ حِينَمَا يَجُوعُونَ أَنَّهُمْ يَحْنَقُونَ وَيَسُبُّونَ مَلِكَهُمْ وَإِلَهَهُمْ وَيَلْتَفِتُونَ إِلَى فَوْقُ.
\par 22 وَيَنْظُرُونَ إِلَى الأَرْضِ وَإِذَا شِدَّةٌ وَظُلْمَةٌ قَتَامُ الضِّيقِ وَإِلَى الظَّلاَمِ هُمْ مَطْرُودُونَ.

\chapter{9}

\par 1 وَلَكِنْ لاَ يَكُونُ ظَلاَمٌ لِلَّتِي عَلَيْهَا ضِيقٌ. كَمَا أَهَانَ الزَّمَانُ الأَوَّلُ أَرْضَ زَبُولُونَ وَأَرْضَ نَفْتَالِي يُكْرِمُ الأَخِيرُ طَرِيقَ الْبَحْرِ عَبْرَ الأُرْدُنِّ جَلِيلَ الأُمَمِ.
\par 2 اَلشَّعْبُ السَّالِكُ فِي الظُّلْمَةِ أَبْصَرَ نُوراً عَظِيماً. الْجَالِسُونَ فِي أَرْضِ ظَِلاَلِ الْمَوْتِ أَشْرَقَ عَلَيْهِمْ نُورٌ.
\par 3 أَكْثَرْتَ الأُمَّةَ. عَظَّمْتَ لَهَا الْفَرَحَ. يَفْرَحُونَ أَمَامَكَ كَالْفَرَحِ فِي الْحَصَادِ. كَالَّذِينَ يَبْتَهِجُونَ عِنْدَمَا يَقْتَسِمُونَ غَنِيمَةً.
\par 4 لأَنَّ نِيرَ ثِقْلِهِ وَعَصَا كَتِفِهِ وَقَضِيبَ مُسَخِّرِهِ كَسَّرْتَهُنَّ كَمَا فِي يَوْمِ مِدْيَانَ.
\par 5 لأَنَّ كُلَّ سِلاَحِ الْمُتَسَلِّحِ فِي الْوَغَى وَكُلَّ رِدَاءٍ مُدَحْرَجٍ فِي الدِّمَاءِ يَكُونُ لِلْحَرِيقِ مَأْكَلاً لِلنَّارِ.
\par 6 لأَنَّهُ يُولَدُ لَنَا وَلَدٌ وَنُعْطَى ابْناً وَتَكُونُ الرِّيَاسَةُ عَلَى كَتِفِهِ وَيُدْعَى اسْمُهُ عَجِيباً مُشِيراً إِلَهاً قَدِيراً أَباً أَبَدِيّاً رَئِيسَ السَّلاَمِ.
\par 7 لِنُمُوِّ رِيَاسَتِهِ وَلِلسَّلاَمِ لاَ نِهَايَةَ عَلَى كُرْسِيِّ دَاوُدَ وَعَلَى مَمْلَكَتِهِ لِيُثَبِّتَهَا وَيَعْضُدَهَا بِالْحَقِّ وَالْبِرِّ مِنَ الآنَ إِلَى الأَبَدِ. غَيْرَةُ رَبِّ الْجُنُودِ تَصْنَعُ هَذَا.
\par 8 أَرْسَلَ الرَّبُّ قَوْلاً فِي يَعْقُوبَ فَوَقَعَ فِي إِسْرَائِيلَ.
\par 9 فَيَعْرِفُ الشَّعْبُ كُلُّهُ أَفْرَايِمُ وَسُكَّانُ السَّامِرَةِ الْقَائِلُونَ بِكِبْرِيَاءٍ وَبِعَظَمَةِ قَلْبٍ:
\par 10 «قَدْ هَبَطَ اللِّبْنُ فَنَبْنِي بِحِجَارَةٍ مَنْحُوتَةٍ. قُطِعَ الْجُمَّيْزُ فَنَسْتَخْلِفُهُ بِأَرْزٍ».
\par 11 فَيَرْفَعُ الرَّبُّ أَخْصَامَ رَصِينَ عَلَيْهِ وَيُهَيِّجُ أَعْدَاءَهُ:
\par 12 الأَرَامِيِّينَ مِنْ قُدَّامُ وَالْفِلِسْطِينِيِّينَ مِنْ وَرَاءُ فَيَأْكُلُونَ إِسْرَائِيلَ بِكُلِّ الْفَمِ. مَعَ كُلِّ هَذَا لَمْ يَرْتَدَّ غَضَبُهُ بَلْ يَدُهُ مَمْدُودَةٌ بَعْدُ!
\par 13 وَالشَّعْبُ لَمْ يَرْجِعْ إِلَى ضَارِبِهِ وَلَمْ يَطْلُبْ رَبَّ الْجُنُودِ.
\par 14 فَيَقْطَعُ الرَّبُّ مِنْ إِسْرَائِيلَ الرَّأْسَ وَالذَّنَبَ النَّخْلَ وَالأَسَلَ فِي يَوْمٍ وَاحِدٍ.
\par 15 اَلشَّيْخُ وَالْمُعْتَبَرُ هُوَ الرَّأْسُ وَالنَّبِيُّ الَّذِي يُعَلِّمُ بِالْكَذِبِ هُوَ الذَّنَبُ.
\par 16 وَصَارَ مُرْشِدُو هَذَا الشَّعْبِ مُضِلِّينَ وَمُرْشَدُوهُ مُبْتَلَعِينَ.
\par 17 لأَجْلِ ذَلِكَ لاَ يَفْرَحُ السَّيِّدُ بِفِتْيَانِهِ وَلاَ يَرْحَمُ يَتَامَاهُ وَأَرَامِلَهُ لأَنَّ كُلَّ وَاحِدٍ مِنْهُمْ مُنَافِقٌ وَفَاعِلُ شَرٍّ. وَكُلُّ فَمٍ مُتَكَلِّمٌ بِالْحَمَاقَةِ. مَعَ كُلِّ هَذَا لَمْ يَرْتَدَّ غَضَبُهُ بَلْ يَدُهُ مَمْدُودَةٌ بَعْدُ!
\par 18 لأَنَّ الْفُجُورَ يُحْرِقُ كَالنَّارِ. تَأْكُلُ الشَّوْكَ وَالْحَسَكَ وَتُشْعِلُ غَابَ الْوَعْرِ فَتَلْتَفُّ عَمُودَ دُخَانٍ.
\par 19 بِسَخَطِ رَبِّ الْجُنُودِ تُحْرَقُ الأَرْضُ وَيَكُونُ الشَّعْبُ كَمَأْكَلٍ لِلنَّارِ. لاَ يُشْفِقُ الإِنْسَانُ عَلَى أَخِيهِ.
\par 20 يَلْتَهِمُ عَلَى الْيَمِينِ فَيَجُوعُ وَيَأْكُلُ عَلَى الشِّمَالِ فَلاَ يَشْبَعُ. يَأْكُلُونَ كُلُّ وَاحِدٍ لَحْمَ ذِرَاعِهِ:
\par 21 مَنَسَّى أَفْرَايِمَ وَأَفْرَايِمُ مَنَسَّى وَهُمَا مَعاً عَلَى يَهُوذَا. مَعَ كُلِّ هَذَا لَمْ يَرْتَدَّ غَضَبُهُ بَلْ يَدُهُ مَمْدُودَةٌ بَعْدُ!

\chapter{10}

\par 1 وَيْلٌ لِلَّذِينَ يَقْضُونَ أَقْضِيَةَ الْبُطْلِ وَلِلْكَتَبَةِ الَّذِينَ يُسَجِّلُونَ جَوْراً
\par 2 لِيَصُدُّوا الضُّعَفَاءَ عَنِ الْحُكْمِ وَيَسْلِبُوا حَقَّ بَائِسِي شَعْبِي لِتَكُونَ الأَرَامِلُ غَنِيمَتَهُمْ وَيَنْهَبُوا الأَيْتَامَ.
\par 3 وَمَاذَا تَفْعَلُونَ فِي يَوْمِ الْعِقَابِ حِينَ تَأْتِي التَّهْلُكَةُ مِنْ بَعِيدٍ؟ إِلَى مَنْ تَهْرُبُونَ لِلْمَعُونَةِ وَأَيْنَ تَتْرُكُونَ مَجْدَكُمْ؟
\par 4 إِمَّا يَجْثُونَ بَيْنَ الأَسْرَى وَإِمَّا يَسْقُطُونَ تَحْتَ الْقَتْلَى. مَعَ كُلِّ هَذَا لَمْ يَرْتَدَّ غَضَبُهُ بَلْ يَدُهُ مَمْدُودَةٌ بَعْدُ!
\par 5 وَيْلٌ لأَشُّورَ قَضِيبِ غَضَبِي. وَالْعَصَا فِي يَدِهِمْ هِيَ سَخَطِي.
\par 6 عَلَى أُمَّةٍ مُنَافِقَةٍ أُرْسِلُهُ وَعَلَى شَعْبِ سَخَطِي أُوصِيهِ لِيَغْتَنِمَ غَنِيمَةً وَيَنْهَبَ نَهْباً وَيَجْعَلَهُمْ مَدُوسِينَ كَطِينِ الأَزِقَّةِ.
\par 7 أَمَّا هُوَ فَلاَ يَفْتَكِرُ هَكَذَا وَلاَ يَحْسِبُ قَلْبُهُ هَكَذَا. بَلْ فِي قَلْبِهِ أَنْ يُبِيدَ وَيَقْرِضَ أُمَماً لَيْسَتْ بِقَلِيلَةٍ.
\par 8 فَإِنَّهُ يَقُولُ: «أَلَيْسَتْ رُؤَسَائِي جَمِيعاً مُلُوكاً؟
\par 9 أَلَيْسَتْ كَلْنُو مِثْلَ كَرْكَمِيشَ؟ أَلَيْسَتْ حَمَاةُ مِثْلَ أَرْفَادَ؟ أَلَيْسَتِ السَّامِرَةُ مِثْلَ دِمَشْقَ؟
\par 10 كَمَا أَصَابَتْ يَدِي مَمَالِكَ الأَوْثَانِ وَأَصْنَامُهَا الْمَنْحُوتَةُ هِيَ أَكْثَرُ مِنَ الَّتِي لأُورُشَلِيمَ وَلِلسَّامِرَةِ
\par 11 أَفَلَيْسَ كَمَا صَنَعْتُ بِالسَّامِرَةِ وَبِأَوْثَانِهَا أَصْنَعُ بِأُورُشَلِيمَ وَأَصْنَامِهَا؟
\par 12 فَيَكُونُ مَتَى أَكْمَلَ السَّيِّدُ كُلَّ عَمَلِهِ بِجَبَلِ صِهْيَوْنَ وَبِأُورُشَلِيمَ أَنِّي أُعَاقِبُ ثَمَرَ عَظَمَةِ قَلْبِ مَلِكِ أَشُّورَ وَفَخْرَ رِفْعَةِ عَيْنَيْهِ.
\par 13 لأَنَّهُ قَالَ: «بِقُدْرَةِ يَدِي صَنَعْتُ وَبِحِكْمَتِي. لأَنِّي فَهِيمٌ. وَنَقَلْتُ تُخُومَ شُعُوبٍ وَنَهَبْتُ ذَخَائِرَهُمْ وَحَطَطْتُ الْمُلُوكَ كَبَطَلٍ.
\par 14 فَأَصَابَتْ يَدِي ثَرْوَةَ الشُّعُوبِ كَعُشٍّ وَكَمَا يُجْمَعُ بَيْضٌ مَهْجُورٌ جَمَعْتُ أَنَا كُلَّ الأَرْضِ وَلَمْ يَكُنْ مُرَفْرِفُ جَنَاحٍ وَلاَ فَاتِحُ فَمٍ وَلاَ مُصَفْصِفٌ».
\par 15 هَلْ تَفْتَخِرُ الْفَأْسُ عَلَى الْقَاطِعِ بِهَا أَوْ يَتَكَبَّرُ الْمِنْشَارُ عَلَى مُرَدِّدِهِ؟ كَأَنَّ الْقَضِيبَ يُحَرِّكُ رَافِعَهُ! كَأَنَّ الْعَصَا تَرْفَعُ مَنْ لَيْسَ هُوَ عُوداً!
\par 16 لِذَلِكَ يُرْسِلُ سَيِّدُ الْجُنُودِ عَلَى سِمَانِهِ هُزَالاً وَيُوقِدُ تَحْتَ مَجْدِهِ وَقِيداً كَوَقِيدِ النَّارِ.
\par 17 وَيَصِيرُ نُورُ إِسْرَائِيلَ نَاراً وَقُدُّوسُهُ لَهِيباً فَيُحْرِقُ وَيَأْكُلُ حَسَكَهُ وَشَوْكَهُ فِي يَوْمٍ وَاحِدٍ
\par 18 وَيُفْنِي مَجْدَ وَعْرِهِ وَبُسْتَانِهِ النَّفْسَ وَالْجَسَدَ جَمِيعاً. فَيَكُونُ كَذَوَبَانِ الْمَرِيضِ.
\par 19 وَبَقِيَّةُ أَشْجَارِ وَعْرِهِ تَكُونُ قَلِيلَةً حَتَّى يَكْتُبَهَا صَبِيٌّ.
\par 20 وَيَكُونُ فِي ذَلِكَ الْيَوْمِ أَنَّ بَقِيَّةَ إِسْرَائِيلَ وَالنَّاجِينَ مِنْ بَيْتِ يَعْقُوبَ لاَ يَعُودُونَ يَتَوَكَّلُونَ أَيْضاً عَلَى ضَارِبِهِمْ بَلْ يَتَوَكَّلُونَ عَلَى الرَّبِّ قُدُّوسِ إِسْرَائِيلَ بِالْحَقِّ.
\par 21 تَرْجِعُ بَقِيَّةُ يَعْقُوبَ إِلَى اللَّهِ الْقَدِيرِ.
\par 22 لأَنَّهُ وَإِنْ كَانَ شَعْبُكَ يَا إِسْرَائِيلُ كَرَمْلِ الْبَحْرِ تَرْجِعُ بَقِيَّةٌ مِنْهُ. قَدْ قُضِيَ بِفَنَاءٍ فَائِضٍ بِالْعَدْلِ.
\par 23 لأَنَّ السَّيِّدَ رَبَّ الْجُنُودِ يَصْنَعُ فَنَاءً وَقَضَاءً فِي كُلِّ الأَرْضِ.
\par 24 وَلَكِنْ هَكَذَا يَقُولُ السَّيِّدُ رَبُّ الْجُنُودِ: «لاَ تَخَفْ مِنْ أَشُّورَ يَا شَعْبِي السَّاكِنُ فِي صِهْيَوْنَ. يَضْرِبُكَ بِالْقَضِيبِ وَيَرْفَعُ عَصَاهُ عَلَيْكَ عَلَى أُسْلُوبِ مِصْرَ.
\par 25 لأَنَّهُ بَعْدَ قَلِيلٍ جِدّاً يَتِمُّ السَّخَطُ وَغَضَبِي فِي إِبَادَتِهِمْ».
\par 26 وَيُقِيمُ عَلَيْهِ رَبُّ الْجُنُودِ سَوْطاً كَضَرْبَةِ مِدْيَانَ عِنْدَ صَخْرَةِ غُرَابَ وَعَصَاهُ عَلَى الْبَحْرِ وَيَرْفَعُهَا عَلَى أُسْلُوبِ مِصْرَ.
\par 27 وَيَكُونُ فِي ذَلِكَ الْيَوْمِ أَنَّ حِمْلَهُ يَزُولُ عَنْ كَتِفِكَ وَنِيرَهُ عَنْ عُنُقِكَ وَيَتْلَفُ النِّيرُ بِسَبَبِ السَّمَانَةِ.
\par 28 قَدْ جَاءَ إِلَى عَيَّاثَ. عَبَرَ بِمِجْرُونَ. وَضَعَ فِي مِخْمَاشَ أَمْتِعَتَهُ.
\par 29 عَبَرُوا الْمَعْبَرَ. بَاتُوا فِي جَبْعَ. ارْتَعَدَتِ الرَّامَةُ. هَرَبَتْ جِبْعَةُ شَاوُلَ.
\par 30 اِصْهِلِي بِصَوْتِكِ يَا بِنْتَ جَلِّيمَ. اسْمَعِي يَا لَيْشَةُ. مِسْكِينَةٌ هِيَ عَنَاثُوثُ.
\par 31 هَرَبَتْ مَدْمِينَةُ. احْتَمَى سُكَّانُ جِيبِيمَ.
\par 32 الْيَوْمَ يَقِفُ فِي نُوبَ. يَهُزُّ يَدَهُ عَلَى جَبَلِ بِنْتِ صِهْيَوْنَ أَكَمَةِ أُورُشَلِيمَ.
\par 33 هُوَذَا السَّيِّدُ رَبُّ الْجُنُودِ يَقْضِبُ الأَغْصَانَ بِرُعْبٍ وَالْمُرْتَفِعُو الْقَامَةِ يُقْطَعُونَ وَالْمُتَشَامِخُونَ يَنْخَفِضُونَ.
\par 34 وَيُقْطَعُ غَابُ الْوَعْرِ بِالْحَدِيدِ وَيَسْقُطُ لُبْنَانُ بِقَدِيرٍ.

\chapter{11}

\par 1 وَيَخْرُجُ قَضِيبٌ مِنْ جِذْعِ يَسَّى وَيَنْبُتُ غُصْنٌ مِنْ أُصُولِهِ
\par 2 وَيَحِلُّ عَلَيْهِ رُوحُ الرَّبِّ رُوحُ الْحِكْمَةِ وَالْفَهْمِ رُوحُ الْمَشُورَةِ وَالْقُوَّةِ رُوحُ الْمَعْرِفَةِ وَمَخَافَةِ الرَّبِّ.
\par 3 وَلَذَّتُهُ تَكُونُ فِي مَخَافَةِ الرَّبِّ فَلاَ يَقْضِي بِحَسَبِ نَظَرِ عَيْنَيْهِ وَلاَ يَحْكُمُ بِحَسَبِ سَمْعِ أُذُنَيْهِ
\par 4 بَلْ يَقْضِي بِالْعَدْلِ لِلْمَسَاكِينِ وَيَحْكُمُ بِالإِنْصَافِ لِبَائِسِي الأَرْضِ وَيَضْرِبُ الأَرْضَ بِقَضِيبِ فَمِهِ وَيُمِيتُ الْمُنَافِقَ بِنَفْخَةِ شَفَتَيْهِ.
\par 5 وَيَكُونُ الْبِرُّ مِنْطَقَةَ مَتْنَيْهِ وَالأَمَانَةُ مِنْطَقَةَ حَقَوَيْهِ.
\par 6 فَيَسْكُنُ الذِّئْبُ مَعَ الْخَرُوفِ وَيَرْبُضُ النَّمِرُ مَعَ الْجَدْيِ وَالْعِجْلُ وَالشِّبْلُ وَالْمُسَمَّنُ مَعاً وَصَبِيٌّ صَغِيرٌ يَسُوقُهَا.
\par 7 وَالْبَقَرَةُ وَالدُّبَّةُ تَرْعَيَانِ. تَرْبُضُ أَوْلاَدُهُمَا مَعاً وَالأَسَدُ كَالْبَقَرِ يَأْكُلُ تِبْناً.
\par 8 وَيَلْعَبُ الرَّضِيعُ عَلَى سَرَبِ الصِّلِّ وَيَمُدُّ الْفَطِيمُ يَدَهُ عَلَى جُحْرِ الأُفْعُوانِ.
\par 9 لاَ يَسُوؤُونَ وَلاَ يُفْسِدُونَ فِي كُلِّ جَبَلِ قُدْسِي لأَنَّ الأَرْضَ تَمْتَلِئُ مِنْ مَعْرِفَةِ الرَّبِّ كَمَا تُغَطِّي الْمِيَاهُ الْبَحْرَ.
\par 10 وَيَكُونُ فِي ذَلِكَ الْيَوْمِ أَنَّ أَصْلَ يَسَّى الْقَائِمَ رَايَةً لِلشُّعُوبِ إِيَّاهُ تَطْلُبُ الأُمَمُ وَيَكُونُ مَحَلُّهُ مَجْداً.
\par 11 وَيَكُونُ فِي ذَلِكَ الْيَوْمِ أَنَّ السَّيِّدَ يُعِيدُ يَدَهُ ثَانِيَةً لِيَقْتَنِي بَقِيَّةَ شَعْبِهِ الَّتِي بَقِيَتْ مِنْ أَشُّورَ وَمِنْ مِصْرَ وَمِنْ فَتْرُوسَ وَمِنْ كُوشَ وَمِنْ عِيلاَمَ وَمِنْ شِنْعَارَ وَمِنْ حَمَاةَ وَمِنْ جَزَائِرِ الْبَحْرِ.
\par 12 وَيَرْفَعُ رَايَةً لِلأُمَمِ وَيَجْمَعُ مَنْفِيِّي إِسْرَائِيلَ وَيَضُمُّ مُشَتَّتِي يَهُوذَا مِنْ أَرْبَعَةِ أَطْرَافِ الأَرْضِ.
\par 13 فَيَزُولُ حَسَدُ أَفْرَايِمَ وَيَنْقَرِضُ الْمُضَايِقُونَ مِنْ يَهُوذَا. أَفْرَايِمُ لاَ يَحْسِدُ يَهُوذَا وَيَهُوذَا لاَ يُضَايِقُ أَفْرَايِمَ.
\par 14 وَيَنْقَضَّانِ عَلَى أَكْتَافِ الْفِلِسْطِينِيِّينَ غَرْباً وَيَنْهَبُونَ بَنِي الْمَشْرِقِ مَعاً. يَكُونُ عَلَى أَدُومَ وَمُوآبَ امْتِدَادُ يَدِهِمَا وَبَنُو عَمُّونَ فِي طَاعَتِهِمَا.
\par 15 وَيُبِيدُ الرَّبُّ لِسَانَ بَحْرِ مِصْرَ وَيَهُزُّ يَدَهُ عَلَى النَّهْرِ بِقُوَّةِ رِيحِهِ وَيَضْرِبُهُ إِلَى سَبْعِ سَوَاقٍ وَيُجِيزُ فِيهَا بِالأَحْذِيَةِ.
\par 16 وَتَكُونُ سِكَّةٌ لِبَقِيَّةِ شَعْبِهِ الَّتِي بَقِيَتْ مِنْ أَشُّورَ كَمَا كَانَ لإِسْرَائِيلَ يَوْمَ صُعُودِهِ مِنْ أَرْضِ مِصْرَ.

\chapter{12}

\par 1 وَتَقُولُ فِي ذَلِكَ الْيَوْمِ: «أَحْمَدُكَ يَا رَبُّ لأَنَّهُ إِذْ غَضِبْتَ عَلَيَّ ارْتَدَّ غَضَبُكَ فَتُعَزِّينِي.
\par 2 هُوَذَا اللَّهُ خَلاَصِي فَأَطْمَئِنُّ وَلاَ أَرْتَعِبُ لأَنَّ يَاهَ يَهْوَهَ قُوَّتِي وَتَرْنِيمَتِي وَقَدْ صَارَ لِي خَلاَصاً».
\par 3 فَتَسْتَقُونَ مِيَاهاً بِفَرَحٍ مِنْ يَنَابِيعِ الْخَلاَصِ.
\par 4 وَتَقُولُونَ فِي ذَلِكَ الْيَوْمِ: «احْمَدُوا الرَّبَّ. ادْعُوا بِاسْمِهِ. عَرِّفُوا بَيْنَ الشُّعُوبِ بِأَفْعَالِهِ. ذَكِّرُوا بِأَنَّ اسْمَهُ قَدْ تَعَالَى.
\par 5 رَنِّمُوا لِلرَّبِّ لأَنَّهُ قَدْ صَنَعَ مُفْتَخَراً. لِيَكُنْ هَذَا مَعْرُوفاً فِي كُلِّ الأَرْضِ.
\par 6 صَوِّتِي وَاهْتِفِي يَا سَاكِنَةَ صِهْيَوْنَ لأَنَّ قُدُّوسَ إِسْرَائِيلَ عَظِيمٌ فِي وَسَطِكِ».

\chapter{13}

\par 1 وَحْيٌ مِنْ جِهَةِ بَابِلَ رَآهُ إِشَعْيَاءُ بْنُ آمُوصَ:
\par 2 «أَقِيمُوا رَايَةً عَلَى جَبَلٍ أَقْرَعَ. ارْفَعُوا صَوْتاً إِلَيْهِمْ. أَشِيرُوا بِالْيَدِ لِيَدْخُلُوا أَبْوَابَ الْعُتَاةِ.
\par 3 أَنَا أَوْصَيْتُ مُقَدَّسِيَّ وَدَعَوْتُ أَبْطَالِي لأَجْلِ غَضَبِي مُفْتَخِرِي عَظَمَتِي».
\par 4 صَوْتُ جُمْهُورٍ عَلَى الْجِبَالِ شِبْهَ قَوْمٍ كَثِيرِينَ. صَوْتُ ضَجِيجِ مَمَالِكِ أُمَمٍ مُجْتَمِعَةٍ. رَبُّ الْجُنُودِ يَعْرِضُ جَيْشَ الْحَرْبِ.
\par 5 يَأْتُونَ مِنْ أَرْضٍ بَعِيدَةٍ مِنْ أَقْصَى السَّمَاوَاتِ. الرَّبُّ وَأَدَوَاتُ سَخَطِهِ لِيُخْرِبَ كُلَّ الأَرْضِ.
\par 6 وَلْوِلُوا لأَنَّ يَوْمَ الرَّبِّ قَرِيبٌ قَادِمٌ كَخَرَابٍ مِنَ الْقَادِرِ عَلَى كُلِّ شَيْءٍ.
\par 7 لِذَلِكَ تَرْتَخِي كُلُّ الأَيَادِي وَيَذُوبُ كُلُّ قَلْبِ إِنْسَانٍ
\par 8 فَيَرْتَاعُونَ. تَأْخُذُهُمْ أَوْجَاعٌ وَمَخَاضٌ. يَتَلَوُّونَ كَوَالِدَةٍ. يَبْهَتُونَ بَعْضُهُمْ إِلَى بَعْضٍ. وُجُوهُهُمْ وُجُوهُ لَهِيبٍ.
\par 9 هُوَذَا يَوْمُ الرَّبِّ قَادِمٌ قَاسِياً بِسَخَطٍ وَحُمُوِّ غَضَبٍ لِيَجْعَلَ الأَرْضَ خَرَاباً وَيُبِيدَ مِنْهَا خُطَاتَهَا.
\par 10 فَإِنَّ نُجُومَ السَّمَاوَاتِ وَجَبَابِرَتَهَا لاَ تُبْرِزُ نُورَهَا. تُظْلِمُ الشَّمْسُ عِنْدَ طُلُوعِهَا وَالْقَمَرُ لاَ يَلْمَعُ بِضُوئِهِ.
\par 11 وَأُعَاقِبُ الْمَسْكُونَةَ عَلَى شَرِّهَا وَالْمُنَافِقِينَ عَلَى إِثْمِهِمْ وَأُبَطِّلُ تَعَظُّمَ الْمُسْتَكْبِرِينَ وَأَضَعُ تَجَبُّرَ الْعُتَاةِ.
\par 12 وَأَجْعَلُ الرَّجُلَ أَعَزَّ مِنَ الذَّهَبِ الإِبْرِيزِ وَالإِنْسَانَ أَعَزَّ مِنْ ذَهَبِ أُوفِيرَ.
\par 13 لِذَلِكَ أُزَلْزِلُ السَّمَاوَاتِ وَتَتَزَعْزَعُ الأَرْضُ مِنْ مَكَانِهَا فِي سَخَطِ رَبِّ الْجُنُودِ وَفِي يَوْمِ حُمُوِّ غَضَبِهِ.
\par 14 وَيَكُونُونَ كَظَبْيٍ طَرِيدٍ وَكَغَنَمٍ بِلاَ مَنْ يَجْمَعُهَا. يَلْتَفِتُونَ كُلُّ وَاحِدٍ إِلَى شَعْبِهِ وَيَهْرُبُونَ كُلُّ وَاحِدٍ إِلَى أَرْضِهِ.
\par 15 كُلُّ مَنْ وُجِدَ يُطْعَنُ وَكُلُّ مَنِ انْحَاشَ يَسْقُطُ بِالسَّيْفِ.
\par 16 وَتُحَطَّمُ أَطْفَالُهُمْ أَمَامَ عُيُونِهِمْ وَتُنْهَبُ بُيُوتُهُمْ وَتُفْضَحُ نِسَاؤُهُمْ.
\par 17 هَئَنَذَا أُهَيِّجُ عَلَيْهِمِ الْمَادِيِّينَ الَّذِينَ لاَ يَعْتَدُّونَ بِالْفِضَّةِ وَلاَ يُسَرُّونَ بِالذَّهَبِ
\par 18 فَتُحَطِّمُ الْقِسِيُّ الْفِتْيَانَ ولاَ يَرْحَمُونَ ثَمَرَةَ الْبَطْنِ. لاَ تُشْفِقُ عُيُونُهُمْ عَلَى الأَوْلاَدِ.
\par 19 وَتَصِيرُ بَابِلُ بَهَاءُ الْمَمَالِكِ وَزِينَةُ فَخْرِ الْكِلْدَانِيِّينَ كَتَقْلِيبِ اللَّهِ سَدُومَ وَعَمُورَةَ.
\par 20 لاَ تُعْمَرُ إِلَى الأَبَدِ وَلاَ تُسْكَنُ إِلَى دَوْرٍ فَدَوْرٍ وَلاَ يُخَيِّمُ هُنَاكَ أَعْرَابِيٌّ وَلاَ يُرْبِضُ هُنَاكَ رُعَاةٌ.
\par 21 بَلْ تَرْبُضُ هُنَاكَ وُحُوشُ الْقَفْرِ وَيَمْلَأُ الْبُومُ بُيُوتَهُمْ وَتَسْكُنُ هُنَاكَ بَنَاتُ النَّعَامِ وَتَرْقُصُ هُنَاكَ مَعْزُ الْوَحْشِ
\par 22 وَتَصِيحُ بَنَاتُ آوَى فِي قُصُورِهِمْ وَالذِّئَابُ فِي هَيَاكِلِ التَّنَعُّمِ وَوَقْتُهَا قَرِيبُ الْمَجِيءِ وَأَيَّامُهَا لاَ تَطُولُ.

\chapter{14}

\par 1 لأَنَّ الرَّبَّ سَيَرْحَمُ يَعْقُوبَ وَيَخْتَارُ أَيْضاً إِسْرَائِيلَ وَيُرِيحُهُمْ فِي أَرْضِهِمْ فَتَقْتَرِنُ بِهِمِ الْغُرَبَاءُ وَيَنْضَمُّونَ إِلَى بَيْتِ يَعْقُوبَ.
\par 2 وَيَأْخُذُهُمْ شُعُوبٌ وَيَأْتُونَ بِهِمْ إِلَى مَوْضِعِهِمْ وَيَمْتَلِكُهُمْ بَيْتُ إِسْرَائِيلَ فِي أَرْضِ الرَّبِّ عَبِيداً وَإِمَاءً وَيَسْبُونَ الَّذِينَ سَبُوهُمْ وَيَتَسَلَّطُونَ عَلَى ظَالِمِيهِمْ.
\par 3 وَيَكُونُ فِي يَوْمٍ يُرِيحُكَ الرَّبُّ مِنْ تَعَبِكَ وَمِنِ انْزِعَاجِكَ وَمِنَ الْعُبُودِيَّةِ الْقَاسِيَةِ الَّتِي اسْتُعْبِدْتَ بِهَا
\par 4 أَنَّكَ تَنْطِقُ بِهَذَا الْهَجْوِ عَلَى مَلِكِ بَابِلَ وَتَقُولُ: «كَيْفَ بَادَ الظَّالِمُ بَادَتِ الْمُغَطْرِسَةُ؟
\par 5 قَدْ كَسَّرَ الرَّبُّ عَصَا الأَشْرَارِ قَضِيبَ الْمُتَسَلِّطِينَ.
\par 6 الضَّارِبُ الشُّعُوبَ بِسَخَطٍ ضَرْبَةً بِلاَ فُتُورٍ. الْمُتَسَلِّطُ بِغَضَبٍ عَلَى الأُمَمِ بِاضْطِهَادٍ بِلاَ إِمْسَاكٍ.
\par 7 اِسْتَرَاحَتِ اطْمَأَنَّتْ كُلُّ الأَرْضِ. هَتَفُوا تَرَنُّماً.
\par 8 حَتَّى السَّرْوُ يَفْرَحُ عَلَيْكَ وَأَرْزُ لُبْنَانَ قَائِلاً: مُنْذُ اضْطَجَعْتَ لَمْ يَصْعَدْ عَلَيْنَا قَاطِعٌ.
\par 9 اَلْهَاوِيَةُ مِنْ أَسْفَلُ مُهْتَزَّةٌ لَكَ لاِسْتِقْبَالِ قُدُومِكَ مُنْهِضَةٌ لَكَ الأَخِيلَةَ جَمِيعَ عُظَمَاءِ الأَرْضِ. أَقَامَتْ كُلَّ مُلُوكِ الأُمَمِ عَنْ كَرَاسِيِّهِمْ.
\par 10 كُلُّهُمْ يُجِيبُونَ وَيَقُولُونَ لَكَ: أَأَنْتَ أَيْضاً قَدْ ضَعُفْتَ نَظِيرَنَا وَصِرْتَ مِثْلَنَا؟
\par 11 أُهْبِطَ إِلَى الْهَاوِيَةِ فَخْرُكَ رَنَّةُ أَعْوَادِكَ. تَحْتَكَ تُفْرَشُ الرِّمَّةُ وَغِطَاؤُكَ الدُّودُ.
\par 12 كَيْفَ سَقَطْتِ مِنَ السَّمَاءِ يَا زُهَرَةُ بِنْتَ الصُّبْحِ؟ كَيْفَ قُطِعْتَ إِلَى الأَرْضِ يَا قَاهِرَ الأُمَمِ؟
\par 13 وَأَنْتَ قُلْتَ فِي قَلْبِكَ: أَصْعَدُ إِلَى السَّمَاوَاتِ. أَرْفَعُ كُرْسِيِّي فَوْقَ كَوَاكِبِ اللَّهِ وَأَجْلِسُ عَلَى جَبَلِ الاِجْتِمَاعِ فِي أَقَاصِي الشِّمَالِ.
\par 14 أَصْعَدُ فَوْقَ مُرْتَفَعَاتِ السَّحَابِ. أَصِيرُ مِثْلَ الْعَلِيِّ.
\par 15 لَكِنَّكَ انْحَدَرْتَ إِلَى الْهَاوِيَةِ إِلَى أَسَافِلِ الْجُبِّ.
\par 16 اَلَّذِينَ يَرُونَكَ يَتَطَلَّعُونَ إِلَيْكَ. يَتَأَمَّلُونَ فِيكَ. أَهَذَا هُوَ الرَّجُلُ الَّذِي زَلْزَلَ الأَرْضَ وَزَعْزَعَ الْمَمَالِكَ
\par 17 الَّذِي جَعَلَ الْعَالَمَ كَقَفْرٍ وَهَدَمَ مُدُنَهُ الَّذِي لَمْ يُطْلِقْ أَسْرَاهُ إِلَى بُيُوتِهِمْ؟
\par 18 كُلُّ مُلُوكِ الأُمَمِ بِأَجْمَعِهِمِ اضْطَجَعُوا بِالْكَرَامَةِ كُلُّ وَاحِدٍ فِي بَيْتِهِ.
\par 19 وَأَمَّا أَنْتَ فَقَدْ طُرِحْتَ مِنْ قَبْرِكَ كَغُصْنٍ أَشْنَعَ. كَلِبَاسِ الْقَتْلَى الْمَضْرُوبِينَ بِالسَّيْفِ الْهَابِطِينَ إِلَى حِجَارَةِ الْجُبِّ. كَجُثَّةٍ مَدُوسَةٍ.
\par 20 لاَ تَتَّحِدُ بِهِمْ فِي الْقَبْرِ لأَنَّكَ أَخْرَبْتَ أَرْضَكَ قَتَلْتَ شَعْبَكَ. لاَ يُسَمَّى إِلَى الأَبَدِ نَسْلُ فَاعِلِي الشَّرِّ.
\par 21 هَيِّئُوا لِبَنِيهِ قَتْلاً بِإِثْمِ آبَائِهِمْ فَلاَ يَقُومُوا وَلاَ يَرِثُوا الأَرْضَ وَلاَ يَمْلَأُوا وَجْهَ الْعَالَمِ مُدُناً.
\par 22 فَأَقُومُ عَلَيْهِمْ يَقُولُ رَبُّ الْجُنُودِ وَأَقْطَعُ مِنْ بَابِلَ اسْماً وَبَقِيَّةً وَنَسْلاً وَذُرِّيَّةً يَقُولُ الرَّبُّ.
\par 23 وَأَجْعَلُهَا مِيرَاثاً لِلْقُنْفُذِ وَآجَامَ مِيَاهٍ وَأُكَنِّسُهَا بِمِكْنَسَةِ الْهَلاَكِ يَقُولُ رَبُّ الْجُنُودِ».
\par 24 قَدْ حَلَفَ رَبُّ الْجُنُودِ قَائِلاً: «إِنَّهُ كَمَا قَصَدْتُ يَصِيرُ وَكَمَا نَوَيْتُ يَثْبُتُ:
\par 25 أَنْ أُحَطِّمَ أَشُّورَ فِي أَرْضِي وَأَدُوسَهُ عَلَى جِبَالِي فَيَزُولَ عَنْهُمْ نِيرُهُ وَيَزُولَ عَنْ كَتِفِهِمْ حِمْلُهُ».
\par 26 هَذَا هُوَ القَضَاءُ الْمَقْضِيُّ بِهِ عَلَى كُلِّ الأَرْضِ وَهَذِهِ هِيَ الْيَدُ الْمَمْدُودَةُ عَلَى كُلِّ الأُمَمِ.
\par 27 فَإِنَّ رَبَّ الْجُنُودِ قَدْ قَضَى فَمَنْ يُبَطِّلُ؟ وَيَدُهُ هِيَ الْمَمْدُودَةُ فَمَنْ يَرُدُّهَا؟
\par 28 فِي سَنَةِ وَفَاةِ الْمَلِكِ آحَازَ كَانَ هَذَا الْوَحْيُ:
\par 29 «لاَ تَفْرَحِي يَا جَمِيعَ فِلِسْطِينَ لأَنَّ الْقَضِيبَ الضَّارِبَكِ انْكَسَرَ. فَإِنَّهُ مِنْ أَصْلِ الْحَيَّةِ يَخْرُجُ أُفْعُوانٌ وَثَمَرَتُهُ تَكُونُ ثُعْبَاناً مُسِمّاً طَيَّاراً.
\par 30 وَتَرْعَى أَبْكَارُ الْمَسَاكِينِ وَيَرْبُضُ الْبَائِسُونَ بِالأَمَانِ وَأُمِيتُ أَصْلَكِ بِالْجُوعِ فَيَقْتُلُ بَقِيَّتَكِ.
\par 31 وَلْوِلْ أَيُّهَا الْبَابُ. اصْرُخِي أَيَّتُهَا الْمَدِينَةُ. قَدْ ذَابَ جَمِيعُكِ يَا فِلِسْطِينُ. لأَنَّهُ مِنَ الشِّمَالِ يَأْتِي دُخَانٌ وَلَيْسَ شَاذٌّ فِي جُيُوشِهِ.
\par 32 فَبِمَاذَا يُجَابُ رُسُلُ الأُمَمِ؟ إِنَّ الرَّبَّ أَسَّسَ صِهْيَوْنَ وَبِهَا يَحْتَمِي بَائِسُو شَعْبِهِ».

\chapter{15}

\par 1 وَحْيٌ مِنْ جِهَةِ مُوآبَ: «إِنَّهُ فِي لَيْلَةٍ خَرِبَتْ «عَارُ» مُوآبَ وَهَلَكَتْ. إِنَّهُ فِي لَيْلَةٍ خَرِبَتْ «قِيرُ» مُوآبَ وَهَلَكَتْ.
\par 2 إِلَى الْبَيْتِ وَدِيبُونَ يَصْعَدُونَ إِلَى الْمُرْتَفَعَاتِ لِلْبُكَاءِ. تُوَلْوِلُ مُوآبُ عَلَى نَبُو وَعَلَى مَيْدِبَا. فِي كُلِّ رَأْسٍ مِنْهَا قَرْعَةٌ. كُلُّ لِحْيَةٍ مَجْزُوزَةٌ.
\par 3 فِي أَزِقَّتِهَا يَأْتَزِرُونَ بِمِسْحٍ. عَلَى سُطُوحِهَا وَفِي سَاحَاتِهَا يُوَلْوِلُ كُلُّ وَاحِدٍ مِنْهَا سَيَّالاً بِالْبُكَاءِ.
\par 4 وَتَصْرُخُ حَشْبُونُ وَأَلْعَالَةُ. يُسْمَعُ صَوْتُهُمَا إِلَى يَاهَصَ. لِذَلِكَ يَصْرُخُ مُتَسَلِّحُو مُوآبَ. نَفْسُهَا تَرْتَعِدُ فِيهَا.
\par 5 يَصْرُخُ قَلْبِي مِنْ أَجْلِ مُوآبَ. الْهَارِبِينَ مِنْهَا إِلَى صُوغَرَ كَعِجْلَةٍ ثُلاَثِيَّةٍ لأَنَّهُمْ يَصْعَدُونَ فِي «عَقَبَةِ اللُّوحِيثِ» بِالْبُكَاءِ لأَنَّهُمْ فِي طَرِيقِ حُورُونَايِمَ يَرْفَعُونَ صُرَاخَ الاِنْكِسَارِ.
\par 6 لأَنَّ مِيَاهَ نِمْرِيمَ تَصِيرُ خَرِبَةً. لأَنَّ الْعُشْبَ يَبِسَ. الْكَلَأُ فَنِيَ. الْخُضْرَةُ لاَ تُوجَدُ.
\par 7 لِذَلِكَ الثَّرْوَةُ الَّتِي اكْتَسَبُوهَا وَذَخَائِرُهُمْ يَحْمِلُونَهَا إِلَى عَبْرِ وَادِي الصَّفْصَافِ.
\par 8 لأَنَّ الصُّرَاخَ قَدْ أَحَاطَ بِتُخُومِ مُوآبَ. إِلَى أَجْلاَيِمَ وَلْوَلَتُهَا. وَإِلَى بِئْرِ إِيلِيمَ وَلْوَلَتُهَا.
\par 9 لأَنَّ مِيَاهَ دِيمُونَ تَمْتَلِئُ دَماً لأَنِّي أَجْعَلُ عَلَى دِيمُونَ زَوَائِدَ. عَلَى النَّاجِينَ مِنْ مُوآبَ أَسَداً وَعَلَى بَقِيَّةِ الأَرْضِ».

\chapter{16}

\par 1 أَرْسِلُوا خِرْفَانَ حَاكِمِ الأَرْضِ مِنْ سَالَعَ نَحْوَ الْبَرِّيَّةِ إِلَى جَبَلِ ابْنَةِ صِهْيَوْنَ.
\par 2 وَيَحْدُثُ أَنَّهُ كَطَائِرٍ تَائِهٍ كَفِرَاخٍ مُنَفَّرَةٍ تَكُونُ بَنَاتُ مُوآبَ فِي مَعَابِرِ أَرْنُونَ.
\par 3 هَاتِي مَشُورَةً. اصْنَعِي إِنْصَافاً. اجْعَلِي ظِلَّكِ كَاللَّيْلِ فِي وَسَطِ الظَّهِيرَةِ. اسْتُرِي الْمَطْرُودِينَ. لاَ تُظْهِرِي الْهَارِبِينَ.
\par 4 لِيَتَغَرَّبْ عِنْدَكِ مَطْرُودُو مُوآبَ. كُونِي سِتْراً لَهُمْ مِنْ وَجْهِ الْمُخَرِّبِ لأَنَّ الظَّالِمَ يَبِيدُ وَيَنْتَهِي الْخَرَابُ وَيَفْنَى عَنِ الأَرْضِ الدَّائِسُونَ.
\par 5 فَيُثَبَّتُ الْكُرْسِيُّ بِالرَّحْمَةِ وَيَجْلِسُ عَلَيْهِ بِالأَمَانَةِ فِي خَيْمَةِ دَاوُدَ قَاضٍ وَيَطْلُبُ الْحَقَّ وَيُبَادِرُ بِالْعَدْلِ.
\par 6 قَدْ سَمِعْنَا بِكِبْرِيَاءِ مُوآبَ الْمُتَكَبِّرَةِ جِدّاً عَظَمَتِهَا وَكِبْرِيَائِهَا وَصَلَفِهَا بُطْلِ افْتِخَارِهَا.
\par 7 لِذَلِكَ تُوَلْوِلُ مُوآبُ. عَلَى مُوآبَ كُلُّهَا يُوَلْوِلُ. تَئِنُّونَ عَلَى أُسُسِ قِيرَ حَارِسَةَ. إِنَّمَا هِيَ مَضْرُوبَةٌ.
\par 8 لأَنَّ حُقُولَ حَشْبُونَ ذَبُلَتْ. كَرْمَةُ سَبْمَةَ كَسَّرَ أُمَرَاءُ الأُمَمِ أَفْضَلَهَا. وَصَلَتْ إِلَى يَعْزِيرَ. تَاهَتْ فِي الْبَرِّيَّةِ. امْتَدَّتْ أَغْصَانُهَا. عَبَرَتِ الْبَحْرَ.
\par 9 لِذَلِكَ أَبْكِي بُكَاءَ يَعْزِيرَ عَلَى كَرْمَةِ سَبْمَةَ. أُرْوِيكُمَا بِدُمُوعِي يَا حَشْبُونُ وَأَلْعَالَةُ. لأَنَّهُ عَلَى قِطَافِكِ وَعَلَى حَصَادِكِ قَدْ وَقَعَتْ جَلَبَةٌ
\par 10 وَانْتُزِعَ الْفَرَحُ وَالاِبْتِهَاجُ مِنَ الْبُسْتَانِ وَلاَ يُغَنَّى فِي الْكُرُومِ وَلاَ يُتَرَنَّمُ وَلاَ يَدُوسُ دَائِسٌ خَمْراً فِي الْمَعَاصِرِ. أَبْطَلْتُ الْهُتَافَ.
\par 11 لِذَلِكَ تَرِنُّ أَحْشَائِي كَعُودٍ مِنْ أَجْلِ مُوآبَ وَبَطْنِي مِنْ أَجْلِ قِيرَ حَارِسَ.
\par 12 وَيَكُونُ إِذَا ظَهَرَتْ إِذَا تَعِبَتْ مُوآبُ عَلَى الْمُرْتَفَعَةِ وَدَخَلَتْ إِلَى مَقْدِسِهَا تُصَلِّي أَنَّهَا لاَ تَفُوزُ.
\par 13 هَذَا هُوَ الْكَلاَمُ الَّذِي كَلَّمَ بِهِ الرَّبُّ مُوآبَ مُنْذُ زَمَانٍ.
\par 14 وَالآنَ تَكَلَّمَ الرَّبُّ قَائِلاً: «فِي ثَلاَثِ سِنِينٍَ كَسِنِي الأَجِيرِ يُهَانُ مَجْدُ مُوآبَ بِكُلِّ الْجُمْهُورِ الْعَظِيمِ وَتَكُونُ الْبَقِيَّةُ قَلِيلَةً صَغِيرَةً لاَ كَبِيرَةً».

\chapter{17}

\par 1 وَحْيٌ مِنْ جِهَةِ دِمَِشْقَ: «هُوَذَا دِمَشْقُ تُزَالُ مِنْ بَيْنِ الْمُدُنِ وَتَكُونُ رُجْمَةَ رَدْمٍ.
\par 2 مُدُنُ عَرُوعِيرَ مَتْرُوكَةٌ. تَكُونُ لِلْقُطْعَانِ فَتَرْبِضُ وَلَيْسَ مَنْ يُخِيفُ.
\par 3 وَيَزُولُ الْحِصْنُ مِنْ أَفْرَايِمَ وَالْمُلْكُ مِنْ دِمَشْقَ وَبَقِيَّةِ أَرَامَ. فَتَصِيرُ كَمَجْدِ بَنِي إِسْرَائِيلَ يَقُولُ رَبُّ الْجُنُودِ.
\par 4 «وَيَكُونُ فِي ذَلِكَ الْيَوْمِ أَنَّ مَجْدَ يَعْقُوبَ يُذَلُّ وَسَمَانَةَ لَحْمِهِ تَهْزُلُ
\par 5 وَيَكُونُ كَجَمْعِ الْحَصَّادِينَ الزَّرْعَ وَذِرَاعُهُ تَحْصِدُ السَّنَابِلَ وَيَكُونُ كَمَنْ يَلْقُطُ سَنَابِلَ فِي وَادِي رَفَايِمَ.
\par 6 وَتَبْقَى فِيهِ خُصَاصَةٌ كَنَفْضِ زَيْتُونَةٍ حَبَّتَانِ أَوْ ثَلاَثٌ فِي رَأْسِ الْفَرْعِ وَأَرْبَعٌ أَوْ خَمْسٌ فِي أَفْنَانِ الْمُثْمِرَةِ يَقُولُ الرَّبُّ إِلَهُ إِسْرَائِيلَ».
\par 7 فِي ذَلِكَ الْيَوْمِ يَلْتَفِتُ الإِنْسَانُ إِلَى صَانِعِهِ وَتَنْظُرُ عَيْنَاهُ إِلَى قُدُّوسِ إِسْرَائِيلَ.
\par 8 وَلاَ يَلْتَفِتُ إِلَى الْمَذَابِحِ صَنْعَةِ يَدَيْهِ وَلاَ يَنْظُرُ إِلَى مَا صَنَعَتْهُ أَصَابِعُهُ: السَّوَارِيَ وَالشَّمْسَاتِ.
\par 9 فِي ذَلِكَ الْيَوْمِ تَصِيرُ مُدُنُهُ الْحَصِينَةُ كَالرَّدْمِ فِي الْغَابِ وَالشَّوَامِخُ الَّتِي تَرَكُوهَا مِنْ وَجْهِ بَنِي إِسْرَائِيلَ فَصَارَتْ خَرَاباً.
\par 10 لأَنَّكِ نَسِيتِ إِلَهَ خَلاَصِكِ وَلَمْ تَذْكُرِي صَخْرَةَ حِصْنِكِ لِذَلِكَ تَغْرِسِينَ أَغْرَاساً نَزِهَةً وَتَنْصِبِينَ نُصْبَةً غَرِيبَةً.
\par 11 يَوْمَ غَرْسِكِ تُسَيِّجِينَهَا وَفِي الصَّبَاحِ تَجْعَلِينَ زَرْعَكِ يُزْهِرُ. وَلَكِنْ يَهْرُبُ الْحَصِيدُ فِي يَوْمِ الضَّرْبَةِ الْمُهْلِكَةِ وَالْكآبَةِ الْعَدِيمَةِ الرَّجَاءِ.
\par 12 آهِ! ضَجِيجُ شُعُوبٍ كَثِيرَةٍ تَضِجُّ كَضَجِيجِ الْبَحْرِ وَهَدِيرِ قَبَائِلَ تَهْدِرُ كَهَدِيرِ مِيَاهٍ غَزِيرَةٍ.
\par 13 قَبَائِلُ تَهْدِرُ كَهَدِيرِ مِيَاهٍ كَثِيرَةٍ. وَلَكِنَّهُ يَنْتَهِرُهَا فَتَهْرُبُ بَعِيداً وَتُطْرَدُ كَعُصَافَةِ الْجِبَالِ أَمَامَ الرِّيحِ وَكَالْجُلِّ أَمَامَ الزَّوْبَعَةِ.
\par 14 فِي وَقْتِ الْمَسَاءِ إِذَا رُعْبٌ. قَبْلَ الصُّبْحِ لَيْسُوا هُمْ. هَذَا نَصِيبُ نَاهِبِينَا وَحَظُّ سَالِبِينَا.

\chapter{18}

\par 1 يَا أَرْضَ حَفِيفِ الأَجْنِحَةِ الَّتِي فِي عَبْرِ أَنْهَارِ كُوشَ
\par 2 الْمُرْسِلَةَ رُسُلاً فِي الْبَحْرِ وَفِي قَوَارِبَ مِنَ الْبَرْدِيِّ عَلَى وَجْهِ الْمِيَاهِ. اذْهَبُوا أَيُّهَا الرُّسُلُ السَّرِيعُونَ إِلَى أُمَّةٍ طَوِيلَةٍ وَجَرْدَاءَ إِلَى شَعْبٍ مَخُوفٍ مُنْذُ كَانَ فَصَاعِداً أُمَّةِ قُوَّةٍ وَشِدَّةٍ وَدَوْسٍ قَدْ خَرَقَتِ الأَنْهَارُ أَرْضَهَا.
\par 3 يَا جَمِيعَ سُكَّانِ الْمَسْكُونَةِ وَقَاطِنِي الأَرْضِ عَُِنْدَمَا تَرْتَفِعُ الرَّايَةُ عَلَى الْجِبَالِ تَنْظُرُونَ وَعِنْدَمَا يُضْرَبُ بِالْبُوقِ تَسْمَعُونَ.
\par 4 لأَنَّهُ هَكَذَا قَالَ لِي الرَّبُّ: «إِنِّي أَهْدَأُ وَأَنْظُرُ فِي مَسْكَنِي كَالْحَرِّ الصَّافِي عَلَى الْبَقْلِ كَغَيْمِ النَّدَى فِي حَرِّ الْحَصَادِ».
\par 5 فَإِنَّهُ قَبْلَ الْحَصَادِ عِنْدَ تَمَامِ الزَّهْرِ وَعِنْدَمَا يَصِيرُ الزَّهْرُ حِصْرِماً نَضِيجاً يَقْطَعُ الْقُضْبَانَ بِالْمَنَاجِلِ وَيَنْزِعُ الأَفْنَانَ وَيَطْرَحُهَا.
\par 6 تُتْرَكُ مَعاً لِجَوَارِحِ الْجِبَالِ وَلِوُحُوشِ الأَرْضِ فَتُصَيِّفُ عَلَيْهَا الْجَوَارِحُ وَتُشَتِّي عَلَيْهَا جَمِيعُ وُحُوشِ الأَرْضِ.
\par 7 فِي ذَلِكَ الْيَوْمِ تُقَدَّمُ هَدِيَّةٌ لِرَبِّ الْجُنُودِ مِنْ شَعْبٍ طَوِيلٍ وَأَجْرَدَ وَمِنْ شَعْبٍ مَخُوفٍ مُنْذُ كَانَ فَصَاعِداً مِنْ أُمَّةٍ ذَاتِ قُوَّةٍ وَشِدَّةٍ وَدَوْسٍ قَدْ خَرَقَتِ الأَنْهَارُ أَرْضَهَا إِلَى مَوْضِعِ اسْمِ رَبِّ الْجُنُودِ جَبَلِ صِهْيَوْنَ.

\chapter{19}

\par 1 وَحْيٌ مِنْ جِهَةِ مِصْرَ: «هُوَذَا الرَّبُّ رَاكِبٌ عَلَى سَحَابَةٍ سَرِيعَةٍ وَقَادِمٌ إِلَى مِصْرَ فَتَرْتَجِفُ أَوْثَانُ مِصْرَ مِنْ وَجْهِهِ وَيَذُوبُ قَلْبُ مِصْرَ دَاخِلَهَا.
\par 2 وَأُهَيِّجُ مِصْرِيِّينَ عَلَى مِصْرِيِّينَ فَيُحَارِبُونَ كُلُّ وَاحِدٍ أَخَاهُ وَكُلُّ وَاحِدٍ صَاحِبَهُ: مَدِينَةٌ مَدِينَةً وَمَمْلَكَةٌ مَمْلَكَةً.
\par 3 وَتُهْرَاقُ رُوحُ مِصْرَ دَاخِلَهَا. وَأُفْنِي مَشُورَتَهَا فَيَسْأَلُونَ الأَوْثَانَ وَالْعَازِفِينَ وَأَصْحَابَ التَّوَابِعِ وَالْعَرَّافِينَ.
\par 4 وَأُغْلِقُ عَلَى الْمِصْرِيِّينَ فِي يَدِ مَوْلىً قَاسٍ فَيَتَسَلَّطُ عَلَيْهِمْ مَلِكٌ عَزِيزٌ يَقُولُ السَّيِّدُ رَبُّ الْجُنُودِ.
\par 5 «وَتُنَشَّفُ الْمِيَاهُ مِنَ الْبَحْرِ وَيَجِفُّ النَّهْرُ وَيَيْبَسُ.
\par 6 وَتُنْتِنُ الأَنْهَارُ وَتَضْعُفُ وَتَجِفُّ سَوَاقِي مِصْرَ وَيَتْلَفُ الْقَصَبُ وَالأَسَلُ.
\par 7 وَالرِّيَاضُ عَلَى حَافَةِ النِّيلِ وَكُلُّ مَزْرَعَةٍ عَلَى النِّيلِ تَيْبَسُ وَتَتَبَدَّدُ وَلاَ تَكُونُ.
\par 8 وَالصَّيَّادُونَ يَئِنُّونَ وَكُلُّ الَّذِينَ يُلْقُونَ شِصّاً فِي النِّيلِ يَنُوحُونَ. وَالَّذِينَ يَبْسُطُونَ شَبَكَةً عَلَى وَجْهِ الْمِيَاهِ يَحْزَنُونَ
\par 9 وَيَخْزَى الَّذِينَ يَعْمَلُونَ الْكَتَّانَ الْمُمَشَّطَ وَالَّذِينَ يَحِيكُونَ الأَنْسِجَةَ الْبَيْضَاءَ.
\par 10 وَتَكُونُ عُمُدُهَا مَسْحُوقَةً وَكُلُّ الْعَامِلِينَ بِالأُجْرَةِ مُكْتَئِبِي النَّفْسِ.
\par 11 «إِنَّ رُؤَسَاءَ صُوعَنَ أَغْبِيَاءَ! حُكَمَاءُ مُشِيرِي فِرْعَوْنَ مَشُورَتُهُمْ بَهِيمِيَّةٌ. كَيْفَ تَقُولُونَ لِفِرْعَوْنَ: أَنَا ابْنُ حُكَمَاءَ ابْنُ مُلُوكٍ قُدَمَاءَ.
\par 12 فَأَيْنَ هُمْ حُكَمَاؤُكَ؟ فَلْيُخْبِرُوكَ. لِيَعْرِفُوا مَاذَا قَضَى بِهِ رَبُّ الْجُنُودِ عَلَى مِصْرَ.
\par 13 رُؤَسَاءُ صُوعَنَ صَارُوا أَغْبِيَاءَ. رُؤَسَاءُ نُوفَ انْخَدَعُوا. وَأَضَلَّ مِصْرَ وُجُوهُ أَسْبَاطِهَا.
\par 14 مَزَجَ الرَّبُّ فِي وَسَطِهَا رُوحَ غَيٍّ فَأَضَلُّوا مِصْرَ فِي كُلِّ عَمَلِهَا كَتَرَنُّحِ السَّكْرَانِ فِي قَيْئِهِ.
\par 15 فَلاَ يَكُونُ لِمِصْرَ عَمَلٌ يَعْمَلُهُ رَأْسٌ أَوْ ذَنَبٌ نَخْلَةٌ أَوْ أَسَلَةٌ.
\par 16 فِي ذَلِكَ الْيَوْمِ تَكُونُ مِصْرُ كَالنِّسَاءِ فَتَرْتَعِدُ وَتَرْجُفُ مِنْ هَزَّةِ يَدِ رَبِّ الْجُنُودِ الَّتِي يَهُزُّهَا عَلَيْهَا.
\par 17 «وَتَكُونُ أَرْضُ يَهُوذَا رُعْباً لِمِصْرَ. كُلُّ مَنْ تَذَكَّرَهَا يَرْتَعِبُ مِنْ أَمَامِ قَضَاءِ رَبِّ الْجُنُودِ الَّذِي يَقْضِي بِهِ عَلَيْهَا.
\par 18 «فِي ذَلِكَ الْيَوْمِ يَكُونُ فِي أَرْضِ مِصْرَ خَمْسُ مُدُنٍ تَتَكَلَّمُ بِلُغَةِ كَنْعَانَ وَتَحْلِفُ لِرَبِّ الْجُنُودِ يُقَالُ لإِحْدَاهَا «مَدِينَةُ الشَّمْسِ».
\par 19 فِي ذَلِكَ الْيَوْمِ يَكُونُ مَذْبَحٌ لِلرَّبِّ فِي وَسَطِ أَرْضِ مِصْرَ وَعَمُودٌ لِلرَّبِّ عِنْدَ تُخُمِهَا.
\par 20 فَيَكُونُ عَلاَمَةً وَشَهَادَةً لِرَبِّ الْجُنُودِ فِي أَرْضِ مِصْرَ. لأَنَّهُمْ يَصْرُخُونَ إِلَى الرَّبِّ بِسَبَبِ الْمُضَايِقِينَ فَيُرْسِلُ لَهُمْ مُخَلِّصاً وَمُحَامِياً وَيُنْقِذُهُمْ.
\par 21 فَيُعْرَفُ الرَّبُّ فِي مِصْرَ وَيَعْرِفُ الْمِصْريُّونَ الرَّبَّ فِي ذَلِكَ الْيَوْمِ وَيُقَدِّمُونَ ذَبِيحَةً وَتَقْدِمَةً وَيَنْذُرُونَ لِلرَّبِّ نَذْراً وَيُوفُونَ بِهِ.
\par 22 وَيَضْرِبُ الرَّبُّ مِصْرَ ضَارِباً فَشَافِياً فَيَرْجِعُونَ إِلَى الرَّبِّ فَيَسْتَجِيبُ لَهُمْ وَيَشْفِيهِمْ.
\par 23 «فِي ذَلِكَ الْيَوْمِ تَكُونُ سِكَّةٌ مِنْ مِصْرَ إِلَى أَشُّورَ فَيَجِيءُ الأَشُّورِيُّونَ إِلَى مِصْرَ وَالْمِصْرِيُّونَ إِلَى أَشُّورَ وَيَعْبُدُ الْمِصْرِيُّونَ مَعَ الأَشُّورِيِّينَ.
\par 24 فِي ذَلِكَ الْيَوْمِ يَكُونُ إِسْرَائِيلُ ثُلْثاً لِمِصْرَ وَلأَشُّورَ بَرَكَةً فِي الأَرْضِ
\par 25 بِهَا يُبَارِكُ رَبُّ الْجُنُودِ قَائِلاً: مُبَارَكٌ شَعْبِي مِصْرُ وَعَمَلُ يَدَيَّ أَشُّورُ وَمِيرَاثِي إِسْرَائِيلُ».

\chapter{20}

\par 1 فِي سَنَةِ مَجِيءِ تَرْتَانَ إِلَى أَشْدُودَ حِينَ أَرْسَلَهُ سَرْجُونُ مَلِكُ أَشُّورَ فَحَارَبَ أَشْدُودَ وَأَخَذَهَا
\par 2 فِي ذَلِكَ الْوَقْتِ قَالَ الرَّبُّ عَنْ يَدِ إِشَعْيَاءَ بْنِ آمُوصَ: «اذْهَبْ وَحُلَّ الْمِسْحَ عَنْ حَقَوَيْكَ وَاخْلَعْ حِذَاءَكَ عَنْ رِجْلَيْكَ». فَفَعَلَ هَكَذَا وَمَشَى مُعَرًّى وَحَافِياً.
\par 3 فَقَالَ الرَّبُّ: «كَمَا مَشَى عَبْدِي إِشَعْيَاءُ مُعَرًّى وَحَافِياً ثَلاَثَ سِنِينٍ آيَةً وَأُعْجُوبَةً عَلَى مِصْرَ وَعَلَى كُوشَ
\par 4 هَكَذَا يَسُوقُ مَلِكُ أَشُّورَ سَبْيَ مِصْرَ وَجَلاَءَ كُوشَ الْفِتْيَانَ وَالشُّيُوخَ عُرَاةً وَحُفَاةً وَمَكْشُوفِي الأَسْتَاهِ خِزْياً لِمِصْرَ.
\par 5 فَيَرْتَاعُونَ وَيَخْجَلُونَ مِنْ أَجْلِ كُوشَ رَجَائِهِمْ وَمِنْ أَجْلِ مِصْرَ فَخْرِهِمْ.
\par 6 وَيَقُولُ سَاكِنُ هَذَا السَّاحِلِ فِي ذَلِكَ الْيَوْمِ: هُوَذَا هَكَذَا مَلْجَأُنَا الَّذِي هَرَبْنَا إِلَيْهِ لِلْمَعُونَةِ لِنَنْجُو مِنْ مَلِكِ أَشُّورَ فَكَيْفَ نَسْلَمُ نَحْنُ؟».

\chapter{21}

\par 1 وَحْيٌ مِنْ جِهَةِ بَرِّيَّةِ الْبَحْرِ: كَزَوَابِعَ فِي الْجَنُوبِ عَاصِفَةٍ يَأْتِي مِنَ الْبَرِّيَّةِ مِنْ أَرْضٍ مَخُوفَةٍ.
\par 2 قَدْ أُعْلِنَتْ لِي رُؤْيَا قَاسِيَةٌ. النَّاهِبُ نَاهِباً وَالْمُخْرِبُ مُخْرِباً. اصْعَدِي يَا عِيلاَمُ. حَاصِرِي يَا مَادِي. قَدْ أَبْطَلْتُ كُلَّ أَنِينِهَا.
\par 3 لِذَلِكَ امْتَلَأَتْ حَقَوَايَ وَجَعاً وَأَخَذَنِي مَخَاضٌ كَمَخَاضِ الْوَالِدَةِ. تَلَوَّيْتُ حَتَّى لاَ أَسْمَعُ. انْدَهَشْتُ حَتَّى لاَ أَنْظُرُ.
\par 4 تَاهَ قَلْبِي. بَغَتَنِي رُعْبٌ. لَيْلَةُ لَذَّتِي جَعَلَهَا لِي رِعْدَةً.
\par 5 يُرَتِّبُونَ الْمَائِدَةَ يَحْرُسُونَ الْحِرَاسَةَ يَأْكُلُونَ. يَشْرَبُونَ - قُومُوا أَيُّهَا الرُّؤَسَاءُ امْسَحُوا الْمِجَنَّ!
\par 6 لأَنَّهُ هَكَذَا قَالَ لِي السَّيِّدُ: «اذْهَبْ أَقِمِ الْحَارِسَ لِيُخْبِرْ بِمَا يَرَى».
\par 7 فَرَأَى رُكَّاباً أَزْوَاجَ فُرْسَانٍ. رُكَّابَ حَمِيرٍ. رُكَّابَ جِمَالٍ. فَأَصْغَى إِصْغَاءً شَدِيداً
\par 8 ثُمَّ صَرَخَ كَأَسَدٍ: «أَيُّهَا السَّيِّدُ أَنَا قَائِمٌ عَلَى الْمَرْصَدِ دَائِماً فِي النَّهَارِ وَأَنَا وَاقِفٌ عَلَى الْمَحْرَسِ كُلَّ اللَّيَالِي.
\par 9 وَهُوَذَا رُكَّابٌ مِنَ الرِّجَالِ. أَزْوَاجٌ مِنَ الْفُرْسَانِ». فَأَجَابَ: «سَقَطَتْ سَقَطَتْ بَابِلُ وَجَمِيعُ تَمَاثِيلِ آلِهَتِهَا الْمَنْحُوتَةِ كَسَّرَهَا إِلَى الأَرْضِ».
\par 10 يَا دِيَاسَتِي وَبَنِي بَيْدَرِي. مَا سَمِعْتُهُ مِنْ رَبِّ الْجُنُودِ إِلَهِ إِسْرَائِيلَ أَخْبَرْتُكُمْ بِهِ.
\par 11 وَحْيٌ مِنْ جِهَةِ دُومَةَ: صَرَخَ إِلَيَّ صَارِخٌ مِنْ سَعِيرَ: «يَا حَارِسُ مَا مِنَ اللَّيْلِ؟ يَا حَارِسُ مَا مِنَ اللَّيْلِ؟»
\par 12 قَالَ الْحَارِسُ: «أَتَى صَبَاحٌ وَأَيْضاً لَيْلٌ. إِنْ كُنْتُمْ تَطْلُبُونَ فَاطْلُبُوا. ارْجِعُوا تَعَالُوا».
\par 13 وَحْيٌ مِنْ جِهَةِ بِلاَدِ الْعَرَبِ: فِي الْوَعْرِ فِي بِلاَدِ الْعَرَبِ تَبِيتِينَ يَا قَوَافِلَ الدَّدَانِيِّينَ.
\par 14 هَاتُوا مَاءً لِمُلاَقَاةِ الْعَطْشَانِ يَا سُكَّانَ أَرْضِ تَيْمَاءَ. وَافُوا الْهَارِبَ بِخُبْزِهِ.
\par 15 فَإِنَّهُمْ مِنْ أَمَامِ السُّيُوفِ قَدْ هَرَبُوا. مِنْ أَمَامِ السَّيْفِ الْمَسْلُولِ وَمِنْ أَمَامِ الْقَوْسِ الْمَشْدُودَةِ وَمِنْ أَمَامِ شِدَّةِ الْحَرْبِ.
\par 16 فَإِنَّهُ هَكَذَا قَالَ لِي السَّيِّدُ: «فِي مُدَّةِ سَنَةٍ كَسَنَةِ الأَجِيرِ يَفْنَى كُلُّ مَجْدِ قِيدَارَ
\par 17 وَبَقِيَّةُ عَدَدِ قِسِيِّ أَبْطَالِ بَنِي قِيدَارَ تَقِلُّ لأَنَّ الرَّبَّ إِلَهَ إِسْرَائِيلَ قَدْ تَكَلَّمَ».

\chapter{22}

\par 1 وَحْيٌ مِنْ جِهَةِ وَادِي الرُّؤْيَا: فَمَا لَكِ أَنَّكِ صَعِدْتِ جَمِيعاً عَلَى السُّطُوحِ
\par 2 يَا مَلآنَةُ مِنَ الْجَلَبَةِ الْمَدِينَةُ الْعَجَّاجَةُ الْقَرْيَةُ الْمُفْتَخِرَةُ؟ قَتْلاَكِ لَيْسَ هُمْ قَتْلَى السَّيْفِ وَلاَ مَوْتَى الْحَرْبِ.
\par 3 جَمِيعُ رُؤَسَائِكِ هَرَبُوا مَعاً. أُسِرُوا بِالْقِسِيِّ. كُلُّ الْمَوْجُودِينَ بِكِ أُسِرُوا مَعاً. مِنْ بَعِيدٍ فَرُّوا.
\par 4 لِذَلِكَ قُلْتُ: «اقْتَصِرُوا عَنِّي فَأَبْكِي بِمَرَارَةٍ. لاَ تُلِحُّوا بِتَعْزِيَتِي عَنْ خَرَابِ بِنْتِ شَعْبِي».
\par 5 إِنَّ لِلسَّيِّدِ رَبِّ الْجُنُودِ فِي وَادِي الرُّؤْيَا يَوْمَ شَغَبٍ وَدَوْسٍ وَارْتِبَاكٍ. نَقْبُ سُورٍ وَصُرَاخٌ إِلَى الْجَبَلِ
\par 6 فَعِيلاَمُ قَدْ حَمَلَتِ الْجُعْبَةَ بِمَرْكَبَاتِ رِجَالٍ فُرْسَانٍ. وَقِيرُ قَدْ كَشَفَتِ الْمِجَنَّ.
\par 7 فَتَكُونُ أَفْضَلُ أَوْدِيَتِكِ مَلآنَةً مَرْكَبَاتٍ وَالْفُرْسَانُ تَصْطَفُّ اصْطِفَافاً نَحْوَ الْبَابِ.
\par 8 وَيَكْشِفُ سِتْرَ يَهُوذَا فَتَنْظُرُ فِي ذَلِكَ الْيَوْمِ إِلَى أَسْلِحَةِ بَيْتِ الْوَعْرِ.
\par 9 وَرَأَيْتُمْ شُقُوقَ مَدِينَةِ دَاوُدَ أَنَّهَا صَارَتْ كَثِيرَةً وَجَمَعْتُمْ مِيَاهَ الْبِرْكَةِ السُّفْلَى.
\par 10 وَعَدَدْتُمْ بُيُوتَ أُورُشَلِيمَ وَهَدَمْتُمُ الْبُيُوتَ لِتَحْصِينِ السُّورِ.
\par 11 وَصَنَعْتُمْ خَنْدَقاً بَيْنَ السُّورَيْنِ لِمِيَاهِ الْبِرْكَةِ الْعَتِيقَةِ. لَكِنْ لَمْ تَنْظُرُوا إِلَى صَانِعِهِ وَلَمْ تَرُوا مُصَوِّرَهُ مِنْ قَدِيمٍ.
\par 12 وَدَعَا السَّيِّدُ رَبُّ الْجُنُودِ فِي ذَلِكَ الْيَوْمِ إِلَى الْبُكَاءِ وَالنَّوْحِ وَالْقَرْعَةِ وَالتَّنَطُّقِ بِالْمِسْحِ
\par 13 فَهُوَذَا بَهْجَةٌ وَفَرَحٌ ذَبْحُ بَقَرٍ وَنَحْرُ غَنَمٍ أَكْلُ لَحْمٍ وَشُرْبُ خَمْرٍ! «لِنَأْكُلْ وَنَشْرَبْ لأَنَّنَا غَداً نَمُوتُ».
\par 14 فَأَعْلَنَ فِي أُذُنَيَّ رَبُّ الْجُنُودِ: «لاَ يُغْفَرَنَّ لَكُمْ هَذَا الإِثْمُ حَتَّى تَمُوتُوا» يَقُولُ السَّيِّدُ رَبُّ الْجُنُودِ.
\par 15 هَكَذَا قَالَ السَّيِّدُ رَبُّ الْجُنُودِ: «اذْهَبِ ادْخُلْ إِلَى هَذَا جَلِيسِ الْمَلِكِ إِلَى شِبْنَا الَّذِي عَلَى الْبَيْتِ.
\par 16 مَا لَكَ هَهُنَا وَمَنْ لَكَ هَهُنَا حَتَّى نَقَرْتَ لِنَفْسِكَ هَهُنَا قَبْراً أَيُّهَا النَّاقِرُ فِي الْعُلُوِّ قَبْرَهُ النَّاحِتُ لِنَفْسِهِ فِي الصَّخْرِ مَسْكَناً؟
\par 17 هُوَذَا الرَّبُّ يَطْرَحُكَ طَرْحاً يَا رَجُلُ وَيُغَطِّيكَ تَغْطِيَةً
\par 18 يَلُفُّكَ لَفَّ لَفِيفَةٍ كَالْكُرَةِ إِلَى أَرْضٍ وَاسِعَةِ الطَّرَفَيْنِ. هُنَاكَ تَمُوتُ وَهُنَاكَ تَكُونُ مَرْكَبَاتُ مَجْدِكَ يَا خِزْيَ بَيْتِ سَيِّدِكَ.
\par 19 وَأَطْرُدُكَ مِنْ مَنْصِبِكَ وَمِنْ مَقَامِكَ يَحُطُّكَ.
\par 20 «وَيَكُونُ فِي ذَلِكَ الْيَوْمِ أَنِّي أَدْعُو عَبْدِي أَلِيَاقِيمَ بْنَ حَلْقِيَّا
\par 21 وَأُلْبِسُهُ ثَوْبَكَ وَأَشُدُّهُ بِمِنْطَقَتِكَ وَأَجْعَلُ سُلْطَانَكَ فِي يَدِهِ فَيَكُونُ أَباً لِسُكَّانِ أُورُشَلِيمَ وَلِبَيْتِ يَهُوذَا.
\par 22 وَأَجْعَلُ مِفْتَاحَ بَيْتِ دَاوُدَ عَلَى كَتِفِهِ فَيَفْتَحُ وَلَيْسَ مَنْ يُغْلِقُ وَيُغْلِقُ وَلَيْسَ مَنْ يَفْتَحُ.
\par 23 وَأُثَبِّتُهُ وَتَداً فِي مَوْضِعٍ أَمِينٍ وَيَكُونُ كُرْسِيَّ مَجْدٍ لِبَيْتِ أَبِيهِ.
\par 24 وَيُعَلِّقُونَ عَلَيْهِ كُلَّ مَجْدِ بَيْتِ أَبِيهِ الْفُرُوعَ وَالْقُضْبَانَ كُلَّ آنِيَةٍ صَغِيرَةٍ مِنْ آنِيَةِ الطُّسُوسِ إِلَى آنِيَةِ الْقَنَّانِيِّ جَمِيعاً.
\par 25 فِي ذَلِكَ الْيَوْمِ يَقُولُ رَبُّ الْجُنُودِ يَزُولُ الْوَتَدُ الْمُثَبَّتُ فِي مَوْضِعٍ أَمِينٍ وَيُقْطَعُ وَيَسْقُطُ. وَيُبَادُ الثِّقْلُ الَّذِي عَلَيْهِ لأَنَّ الرَّبَّ قَدْ تَكَلَّمَ».

\chapter{23}

\par 1 وَحْيٌ مِنْ جِهَةِ صُورَ: وَلْوِلِي يَا سُفُنَ تَرْشِيشَ لأَنَّهَا خَرِبَتْ حَتَّى لَيْسَ بَيْتٌ حَتَّى لَيْسَ مَدْخَلٌ. مِنْ أَرْضِ كِتِّيمَ أُعْلِنَ لَهُمْ.
\par 2 اِنْدَهِشُوا يَا سُكَّانَ السَّاحِلِ. تُجَّارُ صَيْدُونَ الْعَابِرُونَ الْبَحْرَ مَلأُوكِ.
\par 3 وَغَلَّتُهَا زَرْعُ شِيحُورَ حَصَادُ النِّيلِ عَلَى مِيَاهٍ كَثِيرَةٍ فَصَارَتْ مَتْجَرَةً لأُمَمٍ.
\par 4 اِخْجَلِي يَا صَيْدُونُ لأَنَّ حِصْنَ الْبَحْرِ نَطَقَ قَائِلاً: «لَمْ أَتَمَخَّضْ وَلاَ وَلَدْتُ وَلاَ رَبَّيْتُ شَبَاباً وَلاَ نَشَّأْتُ عَذَارَى».
\par 5 عِنْدَ وُصُولِ الْخَبَرِ إِلَى مِصْرَ يَتَوَجَّعُونَ عِنْدَ وُصُولِ خَبَرِ صُورَ.
\par 6 اُعْبُرُوا إِلَى تَرْشِيشَ. وَلْوِلُوا يَا سُكَّانَ السَّاحِلِ.
\par 7 أَهَذِهِ لَكُمُ الْمُفْتَخِرَةُ الَّتِي مُنْذُ الأَيَّامِ الْقَدِيمَةِ قِدَمُهَا؟ تَنْقُلُهَا رِجْلاَهَا بَعِيداً لِلتَّغَرُّبِ.
\par 8 مَنْ قَضَى بِهَذَا عَلَى صُورَ الْمُتَوَّجَةِ الَّتِي تُجَّارُهَا رُؤَسَاءُ؟ مُتَسَبِّبُوهَا مُوَقَّرُو الأَرْضِ.
\par 9 رَبُّ الْجُنُودِ قَضَى بِهِ لِيُدَنِّسَ كِبْرِيَاءَ كُلِّ مَجْدٍ وَيُهِينَ كُلَّ مُوَقَّرِي الأَرْضِ.
\par 10 اِجْتَازِي أَرْضَكِ كَالنِّيلِ يَا بِنْتَ تَرْشِيشَ. لَيْسَ حَصْرٌ فِي مَا بَعْدُ.
\par 11 مَدَّ يَدَهُ عَلَى الْبَحْرِ. أَرْعَدَ مَمَالِكَ. أَمَرَ الرَّبُّ مِنْ جِهَةِ كَنْعَانَ أَنْ تُخْرَبَ حُصُونُهَا.
\par 12 وَقَالَ: «لاَ تَعُودِينَ تَفْتَخِرِينَ أَيْضاً أَيَّتُهَا الْمُنْهَتِكَةُ الْعَذْرَاءُ بِنْتُ صَيْدُونَ. قُومِي إِلَى كِتِّيمَ. اعْبُرِي. هُنَاكَ أَيْضاً لاَ رَاحَةَ لَكِ».
\par 13 هُوَذَا أَرْضُ الْكِلْدَانِيِّينَ. هَذَا الشَّعْبُ لَمْ يَكُنْ. أَسَّسَهَا أَشُّورُ لأَهْلِ الْبَرِّيَّةِ. قَدْ أَقَامُوا أَبْرَاجَهُمْ. دَمَّرُوا قُصُورَهَا. جَعَلَهَا رَدْماً.
\par 14 وَلْوِلِي يَا سُفُنَ تَرْشِيشَ لأَنَّ حِصْنَكِ قَدْ أُخْرِبَ.
\par 15 وَيَكُونُ فِي ذَلِكَ الْيَوْمِ أَنَّ صُورَ تُنْسَى سَبْعِينَ سَنَةً كَأَيَّامِ مَلِكٍ وَاحِدٍ. مِنْ بَعْدِ سَبْعِينَ سَنَةً يَكُونُ لِصُورَ كَأُغْنِيَةِ الزَّانِيَةِ.
\par 16 خُذِي عُوداً. طُوفِي فِي الْمَدِينَةِ أَيَّتُهَا الزَّانِيَةُ الْمَنْسِيَّةُ. أَحْسِنِي الْعَزْفَ أَكْثِرِي الْغِنَاءَ لِكَيْ تُذْكَرِي.
\par 17 وَيَكُونُ مِنْ بَعْدِ سَبْعِينَ سَنَةً أَنَّ الرَّبَّ يَتَعَهَّدُ صُورَ فَتَعُودُ إِلَى أُجْرَتِهَا وَتَزْنِي مَعَ كُلِّ مَمَالِكِ الْبِلاَدِ عَلَى وَجْهِ الأَرْضِ.
\par 18 وَتَكُونُ تِجَارَتُهَا وَأُجْرَتُهَا قُدْساً لِلرَّبِّ. لاَ تُخْزَنُ وَلاَ تُكْنَزُ بَلْ تَكُونُ تِجَارَتُهَا لِلْمُقِيمِينَ أَمَامَ الرَّبِّ لأَكْلٍ إِلَى الشَّبَعِ وَلِلِبَاسٍ فَاخِرٍ.

\chapter{24}

\par 1 هُوَذَا الرَّبُّ يُخْلِي الأَرْضَ وَيُفْرِغُهَا وَيَقْلِبُ وَجْهَهَا وَيُبَدِّدُ سُكَّانَهَا.
\par 2 وَكَمَا يَكُونُ الشَّعْبُ هَكَذَا الْكَاهِنُ. كَمَا الْعَبْدُ هَكَذَا سَيِّدُهُ. كَمَا الأَمَةُ هَكَذَا سَيِّدَتُهَا. كَمَا الشَّارِي هَكَذَا الْبَائِعُ. كَمَا الْمُقْرِضُ هَكَذَا الْمُقْتَرِضُ. وَكَمَا الدَّائِنُ هَكَذَا الْمَدْيُونُ.
\par 3 تُفْرَغُ الأَرْضُ إِفْرَاغاً وَتُنْهَبُ نَهْباً لأَنَّ الرَّبَّ قَدْ تَكَلَّمَ بِهَذَا الْقَوْلِ.
\par 4 نَاحَتْ ذَبُلَتِ الأَرْضُ. حَزِنَتْ ذَبُلَتِ الْمَسْكُونَةُ. حَزِنَ مُرْتَفِعُو شَعْبِ الأَرْضِ.
\par 5 وَالأَرْضُ تَدَنَّسَتْ تَحْتَ سُكَّانِهَا لأَنَّهُمْ تَعَدُّوا الشَّرَائِعَ غَيَّرُوا الْفَرِيضَةَ نَكَثُوا الْعَهْدَ الأَبَدِيَّ.
\par 6 لِذَلِكَ لَعْنَةٌ أَكَلَتِ الأَرْضَ وَعُوقِبَ السَّاكِنُونَ فِيهَا. لِذَلِكَ احْتَرَقَ سُكَّانُ الأَرْضِ وَبَقِيَ أُنَاسٌ قَلاَئِلُ.
\par 7 نَاحَ الْمِسْطَارُ. ذَبُلَتِ الْكَرْمَةُ. أَنَّ كُلُّ مَسْرُورِي الْقُلُوبِ.
\par 8 بَطَلَ فَرَحُ الدُّفُوفِ. انْقَطَعَ ضَجِيجُ الْمُبْتَهِجِينَ. بَطَلَ فَرَحُ الْعُودِ.
\par 9 لاَ يَشْرَبُونَ خَمْراً بِالْغِنَاءِ. يَكُونُ الْمُسْكِرُ مُرّاً لِشَارِبِيهِ.
\par 10 دُمِّرَتْ قَرْيَةُ الْخَرَابِ. أُغْلِقَ كُلُّ بَيْتٍ عَنِ الدُّخُولِ.
\par 11 صُرَاخٌ عَلَى الْخَمْرِ فِي الأَزِقَّةِ. غَرَبَ كُلُّ فَرَحٍ. انْتَفَى سُرُورُ الأَرْضِ.
\par 12 اَلْبَاقِي فِي الْمَدِينَةِ خَرَابٌ وَضُرِبَ الْبَابُ رَدْماً.
\par 13 إِنَّهُ هَكَذَا يَكُونُ فِي وَسَطِ الأَرْضِ بَيْنَ الشُّعُوبِ كَنُفَاضَةِ زَيْتُونَةٍ كَالْخُصَاصَةِ إِذِ انْتَهَى الْقِطَافُ.
\par 14 هُمْ يَرْفَعُونَ أَصْوَاتَهُمْ وَيَتَرَنَّمُونَ. لأَجْلِ عَظَمَةِ الرَّبِّ يُصَوِّتُونَ مِنَ الْبَحْرِ.
\par 15 لِذَلِكَ فِي الْمَشَارِقِ مَجِّدُوا الرَّبَّ. فِي جَزَائِرِ الْبَحْرِ مَجِّدُوا اسْمَ الرَّبِّ إِلَهِ إِسْرَائِيلَ.
\par 16 مِنْ أَطْرَافِ الأَرْضِ سَمِعْنَا تَرْنِيمَةً: «مَجْداً لِلْبَارِّ». فَقُلْتُ: «يَا تَلَفِي! يَا تَلَفِي! وَيْلٌ لِي! النَّاهِبُونَ نَهَبُوا. النَّاهِبُونَ نَهَبُوا نَهْباً».
\par 17 عَلَيْكَ رُعْبٌ وَحُفْرَةٌ وَفَخٌّ يَا سَاكِنَ الأَرْضِ.
\par 18 وَيَكُونُ أَنَّ الْهَارِبَ مِنْ صَوْتِ الرُّعْبِ يَسْقُطُ فِي الْحُفْرَةِ وَالصَّاعِدَ مِنْ وَسَطِ الْحُفْرَةِ يُؤْخَذُ بِالْفَخِّ. لأَنَّ مَيَازِيبَ مِنَ الْعَلاَءِ انْفَتَحَتْ وَأُسُسَ الأَرْضِ تَزَلْزَلَتْ.
\par 19 اِنْسَحَقَتِ الأَرْضُ انْسِحَاقاً. تَشَقَّقَتِ الأَرْضُ تَشَقُّقاً. تَزَعْزَعَتِ الأَرْضُ تَزَعْزُعاً.
\par 20 تَرَنَّحَتِ الأَرْضُ تَرَنُّحاً كَالسَّكْرَانِ وَتَدَلْدَلَتْ كَالْعِرْزَالِ وَثَقُلَ عَلَيْهَا ذَنْبُهَا فَسَقَطَتْ وَلاَ تَعُودُ تَقُومُ.
\par 21 وَيَكُونُ فِي ذَلِكَ الْيَوْمِ أَنَّ الرَّبَّ يُطَالِبُ جُنْدَ الْعَلاَءِ فِي الْعَلاَءِ وَمُلُوكَ الأَرْضِ عَلَى الأَرْضِ.
\par 22 وَيُجْمَعُونَ جَمْعاً كَأَسَارَى فِي سِجْنٍ وَيُغْلَقُ عَلَيْهِمْ فِي حَبْسٍ. ثُمَّ بَعْدَ أَيَّامٍ كَثِيرَةٍ يَتَعَهَّدُونَ.
\par 23 وَيَخْجَلُ الْقَمَرُ وَتُخْزَى الشَّمْسُ لأَنَّ رَبَّ الْجُنُودِ قَدْ مَلَكَ فِي جَبَلِ صِهْيَوْنَ وَفِي أُورُشَلِيمَ. وَقُدَّامَ شُيُوخِهِ مَجْدٌ.

\chapter{25}

\par 1 يَا رَبُّ أَنْتَ إِلَهِي أُعَظِّمُكَ. أَحْمَدُ اسْمَكَ لأَنَّكَ صَنَعْتَ عَجَباً. مَقَاصِدُكَ مُنْذُ الْقَدِيمِ أَمَانَةٌ وَصِدْقٌ.
\par 2 لأَنَّكَ جَعَلْتَ مَدِينَةً رُجْمَةً. قَرْيَةً حَصِينَةً رَدْماً. قَصْرَ أَعَاجِمَ أَنْ لاَ تَكُونَ مَدِينَةً. لاَ يُبْنَى إِلَى الأَبَدِ.
\par 3 لِذَلِكَ يُكْرِمُكَ شَعْبٌ قَوِيٌّ وَتَخَافُ مِنْكَ قَرْيَةُ أُمَمٍ عُتَاةٍ.
\par 4 لأَنَّكَ كُنْتَ حِصْناً لِلْمِسْكِينِ حِصْناً لِلْبَائِسِ فِي ضِيقِهِ مَلْجَأً مِنَ السَّيْلِ ظِلاًّ مِنَ الْحَرِّ إِذْ كَانَتْ نَفْخَةُ الْعُتَاةِ كَسَيْلٍ عَلَى حَائِطٍ.
\par 5 كَحَرٍّ فِي يَبَسٍ تَخْفِضُ ضَجِيجَ الأَعَاجِمِ. كَحَرٍّ بِظِلِّ غَيْمٍ يُذَلُّ غِنَاءُ الْعُتَاةِ.
\par 6 وَيَصْنَعُ رَبُّ الْجُنُودِ لِجَمِيعِ الشُّعُوبِ فِي هَذَا الْجَبَلِ وَلِيمَةَ سَمَائِنَ وَلِيمَةَ خَمْرٍ عَلَى دُرْدِيٍّ سَمَائِنَ مُمِخَّةٍ دُرْدِيٍّ مُصَفّىً.
\par 7 وَيُفْنِي فِي هَذَا الْجَبَلِ وَجْهَ النِّقَابِ الَّذِي عَلَى كُلِّ الشُّعُوبِ وَالْغِطَاءَ الْمُغَطَّى بِهِ عَلَى كُلِّ الأُمَمِ.
\par 8 يَبْلَعُ الْمَوْتَ إِلَى الأَبَدِ وَيَمْسَحُ السَّيِّدُ الرَّبُّ الدُّمُوعَ عَنْ كُلِّ الْوُجُوهِ وَيَنْزِعُ عَارَ شَعْبِهِ عَنْ كُلِّ الأَرْضِ لأَنَّ الرَّبَّ قَدْ تَكَلَّمَ.
\par 9 وَيُقَالُ فِي ذَلِكَ الْيَوْمِ: «هُوَذَا هَذَا إِلَهُنَا. انْتَظَرْنَاهُ فَخَلَّصَنَا. هَذَا هُوَ الرَّبُّ انْتَظَرْنَاهُ. نَبْتَهِجُ وَنَفْرَحُ بِخَلاَصِهِ».
\par 10 لأَنَّ يَدَ الرَّبِّ تَسْتَقِرُّ عَلَى هَذَا الْجَبَلِ وَيُدَاسُ مُوآبُ فِي مَكَانِهِ كَمَا يُدَاسُ التِّبْنُ فِي مَاءِ الْمَزْبَلَةِ.
\par 11 فَيَبْسُطُ يَدَيْهِ فِيهِ كَمَا يَبْسُطُ السَّابِحُ لِيَسْبَحَ فَيَضَعُ كِبْرِيَاءَهُ مَعَ مَكَايِدِ يَدَيْهِ.
\par 12 وَصَرْحَ ارْتِفَاعِ أَسْوَارِكِ يَخْفِضُهُ. يَضَعُهُ يُلْصِقُهُ بِالأَرْضِ إِلَى التُّرَابِ.

\chapter{26}

\par 1 فِي ذَلِكَ الْيَوْمِ يُغَنَّى بِهَذِهِ الأُغْنِيَةِ فِي أَرْضِ يَهُوذَا: «لَنَا مَدِينَةٌ قَوِيَّةٌ. يَجْعَلُ الْخَلاَصَ أَسْوَاراً وَمِتْرَسَةً.
\par 2 اِفْتَحُوا الأَبْوَابَ لِتَدْخُلَ الأُمَّةُ الْبَارَّةُ الْحَافِظَةُ الأَمَانَةَ.
\par 3 ذُو الرَّأْيِ الْمُمَكَّنِ تَحْفَظُهُ سَالِماً سَالِماً لأَنَّهُ عَلَيْكَ مُتَوَكِّلٌ.
\par 4 تَوَكَّلُوا عَلَى الرَّبِّ إِلَى الأَبَدِ لأَنَّ فِي يَاهَ الرَّبِّ صَخْرَ الدُّهُورِ.
\par 5 لأَنَّهُ يَخْفِضُ سُكَّانَ الْعَلاَءِ يَضَعُ الْقَرْيَةَ الْمُرْتَفِعَةَ. يَضَعُهَا إِلَى الأَرْضِ. يُلْصِقُهَا بِالتُّرَابِ.
\par 6 تَدُوسُهَا الرِّجْلُ رِجْلاَ الْبَائِسِ أَقْدَامُ الْمَسَاكِينِ».
\par 7 طَرِيقُ الصِّدِّيقِ اسْتِقَامَةٌ. تُمَهِّدُ أَيُّهَا الْمُسْتَقِيمُ سَبِيلَ الصِّدِّيقِ.
\par 8 فَفِي طَرِيقِ أَحْكَامِكَ يَا رَبُّ انْتَظَرْنَاكَ. إِلَى اسْمِكَ وَإِلَى ذِكْرِكَ شَهْوَةُ النَّفْسِ.
\par 9 بِنَفْسِي اشْتَهَيْتُكَ فِي اللَّيْلِ. أَيْضاً بِرُوحِي فِي دَاخِلِي إِلَيْكَ أَبْتَكِرُ. لأَنَّهُ حِينَمَا تَكُونُ أَحْكَامُكَ فِي الأَرْضِ يَتَعَلَّمُ سُكَّانُ الْمَسْكُونَةِ الْعَدْلَ.
\par 10 يُرْحَمُ الْمُنَافِقُ وَلاَ يَتَعَلَّمُ الْعَدْلَ. فِي أَرْضِ الاِسْتِقَامَةِ يَصْنَعُ شَرّاً وَلاَ يَرَى جَلاَلَ الرَّبِّ.
\par 11 يَا رَبُّ ارْتَفَعَتْ يَدُكَ وَلاَ يَرُونَ. يَرُونَ وَيَخْزُونَ مِنَ الْغَيْرَةِ عَلَى الشَّعْبِ وَتَأْكُلُهُمْ نَارُ أَعْدَائِكَ.
\par 12 يَا رَبُّ تَجْعَلُ لَنَا سَلاَماً لأَنَّكَ كُلَّ أَعْمَالِنَا صَنَعْتَهَا لَنَا.
\par 13 أَيُّهَا الرَّبُّ إِلَهُنَا قَدِ اسْتَوْلَى عَلَيْنَا سَادَةٌ سِوَاكَ. بِكَ وَحْدَكَ نَذْكُرُ اسْمَكَ.
\par 14 هُمْ أَمْوَاتٌ لاَ يَحْيُونَ. أَخْيِلَةٌ لاَ تَقُومُ. لِذَلِكَ عَاقَبْتَ وَأَهْلَكْتَهُمْ وَأَبَدْتَ كُلَّ ذِكْرِهِمْ.
\par 15 زِدْتَ الأُمَّةَ يَا رَبُّ زِدْتَ الأُمَّةَ. تَمَجَّدْتَ. وَسَّعْتَ كُلَّ أَطْرَافِ الأَرْضِ.
\par 16 يَا رَبُّ فِي الضِّيقِ طَلَبُوكَ. سَكَبُوا مُخَافَتَةً عِنْدَ تَأْدِيبِكَ إِيَّاهُمْ.
\par 17 كَمَا أَنَّ الْحُبْلَى الَّتِي تُقَارِبُ الْوِلاَدَةَ تَتَلَوَّى وَتَصْرُخُ فِي مَخَاضِهَا هَكَذَا كُنَّا قُدَّامَكَ يَا رَبُّ.
\par 18 حَبِلْنَا تَلَوَّيْنَا كَأَنَّنَا وَلَدْنَا رِيحاً. لَمْ نَصْنَعْ خَلاَصاً فِي الأَرْضِ وَلَمْ يَسْقُطْ سُكَّانُ الْمَسْكُونَةِ.
\par 19 تَحْيَا أَمْوَاتُكَ. تَقُومُ الْجُثَثُ. اسْتَيْقِظُوا. تَرَنَّمُوا يَا سُكَّانَ التُّرَابِ. لأَنَّ طَلَّكَ طَلُّ أَعْشَابٍ وَالأَرْضُ تُسْقِطُ الأَخْيِلَةَ.
\par 20 هَلُمَّ يَا شَعْبِي ادْخُلْ مَخَادِعَكَ وَأَغْلِقْ أَبْوَابَكَ خَلْفَكَ. اخْتَبِئْ نَحْوَ لُحَيْظَةٍ حَتَّى يَعْبُرَ الْغَضَبُ.
\par 21 لأَنَّهُ هُوَذَا الرَّبُّ يَخْرُجُ مِنْ مَكَانِهِ لِيُعَاقِبَ إِثْمَ سُكَّانِ الأَرْضِ فِيهِمْ فَتَكْشِفُ الأَرْضُ دِمَاءَهَا وَلاَ تُغَطِّي قَتْلاَهَا فِي مَا بَعْدُ.

\chapter{27}

\par 1 فِي ذَلِكَ الْيَوْمِ يُعَاقِبُ الرَّبُّ بِسَيْفِهِ الْقَاسِي الْعَظِيمِ الشَّدِيدِ لَوِيَاثَانَ الْحَيَّةَ الْهَارِبَةَ. لَوِيَاثَانَ الْحَيَّةَ الْمُتَحَوِّيَةَ وَيَقْتُلُ التِّنِّينَ الَّذِي فِي الْبَحْرِ.
\par 2 فِي ذَلِكَ الْيَوْمِ غَنُّوا لِلْكَرْمَةِ الْمُشْتَهَاةِ:
\par 3 «أَنَا الرَّبُّ حَارِسُهَا. أَسْقِيهَا كُلَّ لَحْظَةٍ. لِئَلاَّ يُوقَعَ بِهَا أَحْرُسُهَا لَيْلاً وَنَهَاراً.
\par 4 لَيْسَ لِي غَيْظٌ. لَيْتَ عَلَيَّ الشَّوْكَ وَالْحَسَكَ فِي الْقِتَالِ فَأَهْجِمَ عَلَيْهَا وَأَحْرِقَهَا مَعاً.
\par 5 أَوْ يَتَمَسَّكُ بِحِصْنِي فَيَصْنَعُ صُلْحاً مَعِي. صُلْحاً يَصْنَعُ مَعِي».
\par 6 فِي الْمُسْتَقْبَلِ يَتَأَصَّلُ يَعْقُوبُ. يُزْهِرُ وَيُفْرِعُ إِسْرَائِيلُ وَيَمْلَأُونَ وَجْهَ الْمَسْكُونَةِ ثِمَاراً.
\par 7 هَلْ ضَرَبَهُ كَضَرْبَةِ ضَارِبِيهِ أَوْ قُتِلَ كَقَتْلِ قَتْلاَهُ؟
\par 8 بِزَجْرٍ إِذْ طَلَّقْتَهَا خَاصَمْتَهَا. أَزَالَهَا بِرِيحِهِ الْعَاصِفَةِ فِي يَوْمِ الشَّرْقِيَّةِ.
\par 9 لِذَلِكَ بِهَذَا يُكَفَّرُ إِثْمُ يَعْقُوبَ. وَهَذَا كُلُّ الثَّمَرِ نَزْعُ خَطِيَّتِهِ: فِي جَعْلِهِ كُلَّ حِجَارَةِ الْمَذْبَحِ كَحِجَارَةِ كِلْسٍ مُكَسَّرَةٍ. لاَ تَقُومُ السَّوَارِي وَلاَ الشَّمْسَاتُ.
\par 10 لأَنَّ الْمَدِينَةَ الْحَصِينَةَ مُتَوَحِّدَةٌ. الْمَسْكَنُ مَهْجُورٌ وَمَتْرُوكٌ كَالْقَفْرِ. هُنَاكَ يَرْعَى الْعِجْلُ وَهُنَاكَ يَرْبُضُ وَيُتْلِفُ أَغْصَانَهَا.
\par 11 حِينَمَا تَيْبَسُ أَغْصَانُهَا تَتَكَسَّرُ فَتَأْتِي نِسَاءٌ وَتُوقِدُهَا. لأَنَّهُ لَيْسَ شَعْباً ذَا فَهْمٍ لِذَلِكَ لاَ يَرْحَمُهُ صَانِعُهُ وَلاَ يَتَرَأَّفُ عَلَيْهِ جَابِلُهُ.
\par 12 وَيَكُونُ فِي ذَلِكَ الْيَوْمِ أَنَّ الرَّبَّ يَجْنِي مِنْ مَجْرَى النَّهْرِ إِلَى وَادِي مِصْرَ. وَأَنْتُمْ تُلْقَطُونَ وَاحِداً وَاحِداً يَا بَنِي إِسْرَائِيلَ.
\par 13 وَيَكُونُ فِي ذَلِكَ الْيَوْمِ أَنَّهُ يُضْرَبُ بِبُوقٍ عَظِيمٍ فَيَأْتِي التَّائِهُونَ فِي أَرْضِ أَشُّورَ وَالْمَنْفِيُّونَ فِي أَرْضِ مِصْرَ وَيَسْجُدُونَ لِلرَّبِّ فِي الْجَبَلِ الْمُقَدَّسِ فِي أُورُشَلِيمَ.

\chapter{28}

\par 1 وَيْلٌ لإِكْلِيلِ فَخْرِ سُكَارَى أَفْرَايِمَ وَلِلزَّهْرِ الذَّابِلِ جَمَالِ بَهَائِهِ الَّذِي عَلَى رَأْسِ وَادِي سَمَائِنَ الْمَضْرُوبِينَ بِالْخَمْرِ.
\par 2 هُوَذَا شَدِيدٌ وَقَوِيٌّ لِلسَّيِّدِ كَانْهِيَالِ الْبَرَدِ كَنَوْءٍ مُهْلِكٍ كَسَيْلِ مِيَاهٍ غَزِيرَةٍ جَارِفَةٍ قَدْ أَلْقَاهُ إِلَى الأَرْضِ بِشِدَّةٍ.
\par 3 بِالأَرْجُلِ يُدَاسُ إِكْلِيلُ فَخْرِ سُكَارَى أَفْرَايِمَ.
\par 4 وَيَكُونُ الزَّهْرُ الذَّابِلُ جَمَالُ بَهَائِهِ الَّذِي عَلَى رَأْسِ وَادِي السَّمَائِنِ كَبَاكُورَةِ التِّينِ قَبْلَ الصَّيْفِ الَّتِي يَرَاهَا النَّاظِرُ فَيَبْلَعُهَا وَهِيَ فِي يَدِهِ.
\par 5 فِي ذَلِكَ الْيَوْمِ يَكُونُ رَبُّ الْجُنُودِ إِكْلِيلَ جَمَالٍ وَتَاجَ بَهَاءٍ لِبَقِيَّةِ شَعْبِهِ
\par 6 وَرُوحَ الْقَضَاءِ لِلْجَالِسِ لِلْقَضَاءِ وَبَأْساً لِلَّذِينَ يَرُدُّونَ الْحَرْبَ إِلَى الْبَابِ.
\par 7 وَلَكِنَّ هَؤُلاَءِ أَيْضاً ضَلُّوا بِالْخَمْرِ وَتَاهُوا بِالْمُسْكِرِ. الْكَاهِنُ وَالنَّبِيُّ تَرَنَّحَا بِالْمُسْكِرِ. ابْتَلَعَتْهُمَا الْخَمْرُ. تَاهَا مِنَ الْمُسْكِرِ. ضَلاَّ فِي الرُّؤْيَا. قَلِقَا فِي الْقَضَاءِ.
\par 8 فَإِنَّ جَمِيعَ الْمَوَائِدِ امْتَلَأَتْ قَيْئاً وَقَذَراً. لَيْسَ مَكَانٌ.
\par 9 لِمَنْ يُعَلِّمُ مَعْرِفَةً وَلِمَنْ يُفْهِمُ تَعْلِيماً؟ أَلِلْمَفْطُومِينَ عَنِ اللَّبَنِ لِلْمَفْصُولِينَ عَنِ الثُّدِيِّ؟
\par 10 لأَنَّهُ أَمْرٌ عَلَى أَمْرٍ. أَمْرٌ عَلَى أَمْرٍ. فَرْضٌ عَلَى فَرْضٍ. فَرْضٌ عَلَى فَرْضٍ. هُنَا قَلِيلٌ هُنَاكَ قَلِيلٌ.
\par 11 إِنَّهُ بِشَفَةٍ لَكْنَاءَ وَبِلِسَانٍ آخَرَ يُكَلِّمُ هَذَا الشَّعْبَ
\par 12 الَّذِينَ قَالَ لَهُمْ: «هَذِهِ هِيَ الرَّاحَةُ. أَرِيحُوا الرَّازِحَ وَهَذَا هُوَ السُّكُونُ». وَلَكِنْ لَمْ يَشَاؤُوا أَنْ يَسْمَعُوا.
\par 13 فَكَانَ لَهُمْ قَوْلُ الرَّبِّ: «أَمْراً عَلَى أَمْرٍ. أَمْراً عَلَى أَمْرٍ. فَرْضاً عَلَى فَرْضٍ. فَرْضاً عَلَى فَرْضٍ. هُنَا قَلِيلاً هُنَاكَ قَلِيلاً» لِيَذْهَبُوا وَيَسْقُطُوا إِلَى الْوَرَاءِ وَيَنْكَسِرُوا وَيُصَادُوا فَيُؤْخَذُوا.
\par 14 لِذَلِكَ اسْمَعُوا كَلاَمَ الرَّبِّ يَا رِجَالَ الْهُزْءِ وُلاَةَ هَذَا الشَّعْبِ الَّذِي فِي أُورُشَلِيمَ.
\par 15 لأَنَّكُمْ قُلْتُمْ: «قَدْ عَقَدْنَا عَهْداً مَعَ الْمَوْتِ وَصَنَعْنَا مِيثَاقاً مَعَ الْهَاوِيَةِ. السَّوْطُ الْجَارِفُ إِذَا عَبَرَ لاَ يَأْتِينَا لأَنَّنَا جَعَلْنَا الْكَذِبَ مَلْجَأَنَا وَبِالْغِشِّ اسْتَتَرْنَا».
\par 16 لِذَلِكَ هَكَذَا يَقُولُ السَّيِّدُ الرَّبُّ: «هَئَنَذَا أُؤَسِّسُ فِي صِهْيَوْنَ حَجَرَ امْتِحَانٍ حَجَرَ زَاوِيَةٍ كَرِيماً أَسَاساً مُؤَسَّساً. مَنْ آمَنَ لاَ يَهْرُبُ.
\par 17 وَأَجْعَلُ الْحَقَّ خَيْطاً وَالْعَدْلَ مِطْمَاراً فَيَخْطُفُ الْبَرَدُ مَلْجَأَ الْكَذِبِ وَيَجْرُفُ الْمَاءُ السِّتَارَةَ.
\par 18 وَيُمْحَى عَهْدُكُمْ مَعَ الْمَوْتِ وَلاَ يَثْبُتُ مِيثَاقُكُمْ مَعَ الْهَاوِيَةِ. السَّوْطُ الْجَارِفُ إِذَا عَبَرَ تَكُونُونَ لَهُ لِلدَّوْسِ.
\par 19 كُلَّمَا عَبَرَ يَأْخُذُكُمْ فَإِنَّهُ كُلَّ صَبَاحٍ يَعْبُرُ فِي النَّهَارِ وَفِي اللَّيْلِ وَيَكُونُ فَهْمُ الْخَبَرِ فَقَطِ انْزِعَاجاً».
\par 20 لأَنَّ الْفِرَاشَ قَدْ قَصَرَ عَنِ التَّمَدُّدِ وَالْغِطَاءَ ضَاقَ عَنِ الاِلْتِحَافِ.
\par 21 لأَنَّهُ كَمَا فِي جَبَلِ فَرَاصِيمَ يَقُومُ الرَّبُّ وَكَمَا فِي الْوَطَاءِ عِنْدَ جِبْعُونَ يَسْخَطُ لِيَفْعَلَ فِعْلَهُ الْغَرِيبَ وَلِيَعْمَلَ عَمَلَهُ الْغَرِيبَ.
\par 22 فَالآنَ لاَ تَكُونُوا مُتَهَكِّمِينَ لِئَلاَّ تُشَدَّدَ رُبُطُكُمْ لأَنِّي سَمِعْتُ فَنَاءً قَضَى بِهِ السَّيِّدُ رَبُّ الْجُنُودِ عَلَى كُلِّ الأَرْضِ.
\par 23 اُصْغُوا وَاسْمَعُوا صَوْتِي. انْصُتُوا وَاسْمَعُوا قَوْلِي.
\par 24 هَلْ يَحْرُثُ الْحَارِثُ كُلَّ يَوْمٍ لِيَزْرَعَ وَيَشُقَّ أَرْضَهُ وَيُمَهِّدَهَا؟
\par 25 أَلَيْسَ أَنَّهُ إِذَا سَوَّى وَجْهَهَا يَبْذُرُ الشُّونِيزَ وَيُذَرِّي الْكَمُّونَ وَيَضَعُ الْحِنْطَةَ فِي أَتْلاَمٍ وَالشَّعِيرَ فِي مَكَانٍ مُعَيَّنٍ وَالْقَطَانِيَّ فِي حُدُودِهَا؟
\par 26 فَيُرْشِدُهُ. بِالْحَقِّ يُعَلِّمُهُ إِلَهُهُ.
\par 27 إِنَّ الشُّونِيزَ لاَ يُدْرَسُ بِالنَّوْرَجِ وَلاَ تُدَارُ بَكَرَةُ الْعَجَلَةِ عَلَى الْكَمُّونِ بَلْ بِالْقَضِيبِ يُخْبَطُ الشُّونِيزُ وَالْكَمُّونُ بِالْعَصَا.
\par 28 يُدَقُّ الْقَمْحُ لأَنَّهُ لاَ يَدْرُسُهُ إِلَى الأَبَدِ فَيَسُوقُ بَكَرَةَ عَجَلَتِهِ وَخَيْلَهُ. لاَ يَسْحَقُهُ.
\par 29 هَذَا أَيْضاً خَرَجَ مِنْ قِبَلِ رَبِّ الْجُنُودِ. عَجِيبُِ الرَّأْيِ عَظِيمُِ الْفَهْمِ.

\chapter{29}

\par 1 وَيْلٌ لأَرِيئِيلَ لأَرِيئِيلَ قَرْيَةٍ نَزَلَ عَلَيْهَا دَاوُدُ. زِيدُوا سَنَةً عَلَى سَنَةٍ. لِتَدُرِ الأَعْيَادُ.
\par 2 وَأَنَا أُضَايِقُ أَرِيئِيلَ فَيَكُونُ نَوْحٌ وَحُزْنٌ وَتَكُونُ لِي كَأَرِيئِيلَ.
\par 3 وَأُحِيطُ بِكِ كَالدَّائِرَةِ وَأُضَايِقُ عَلَيْكِ بِحِصْنٍ وَأُقِيمُ عَلَيْكِ مَتَارِسَ.
\par 4 فَتَتَّضِعِينَ وَتَتَكَلَّمِينَ مِنَ الأَرْضِ وَيَنْخَفِضُ قَوْلُكِ مِنَ التُّرَابِ وَيَكُونُ صَوْتُكِ كَخِيَالٍ مِنَ الأَرْضِ وَيُشَقْشَقُ قَوْلُكِ مِنَ التُّرَابِ.
\par 5 وَيَصِيرُ جُمْهُورُ أَعْدَائِكِ كَالْغُبَارِ الدَّقِيقِ وَجُمْهُورُ الْعُتَاةِ كَالْعُصَافَةِ الْمَارَّةِ. وَيَكُونُ ذَلِكَ فِي لَحْظَةٍ بَغْتَةً.
\par 6 مِنْ قِبَلِ رَبِّ الْجُنُودِ تُفْتَقَدُ بِرَعْدٍ وَزَلْزَلَةٍ وَصَوْتٍ عَظِيمٍ بِزَوْبَعَةٍ وَعَاصِفٍ وَلَهِيبِ نَارٍ آكِلَةٍ.
\par 7 وَيَكُونُ كَحُلْمٍ كَرُؤْيَا اللَّيْلِ جُمْهُورُ كُلِّ الأُمَمِ الْمُتَجَنِّدِينَ عَلَى أَرِيئِيلَ كُلُّ الْمُتَجَنِّدِينَ عَلَيْهَا وَعَلَى قِلاَعِهَا وَالَّذِينَ يُضَايِقُونَهَا.
\par 8 وَيَكُونُ كَمَا يَحْلُمُ الْجَائِعُ أَنَّهُ يَأْكُلُ ثُمَّ يَسْتَيْقِظُ وَإِذَا نَفْسُهُ فَارِغَةٌ. وَكَمَا يَحْلُمُ الْعَطْشَانُ أَنَّهُ يَشْرَبُ ثُمَّ يَسْتَيْقِظُ وَإِذَا هُوَ رَازِحٌ وَنَفْسُهُ مُشْتَهِيَةٌ. هَكَذَا يَكُونُ جُمْهُورُ كُلِّ الأُمَمِ الْمُتَجَنِّدِينَ عَلَى جَبَلِ صِهْيَوْنَ.
\par 9 تَوَانُوا وَابْهَتُوا. تَلَذَّذُوا وَاعْمُوا. قَدْ سَكِرُوا وَلَيْسَ مِنَ الْخَمْرِ. تَرَنَّحُوا وَلَيْسَ مِنَ الْمُسْكِرِ.
\par 10 لأَنَّ الرَّبَّ قَدْ سَكَبَ عَلَيْكُمْ رُوحَ سُبَاتٍ وَأَغْمَضَ عُيُونَكُمُ. الأَنْبِيَاءُ وَرُؤَسَاؤُكُمُ النَّاظِرُونَ غَطَّاهُمْ.
\par 11 وَصَارَتْ لَكُمْ رُؤْيَا الْكُلِّ مِثْلَ كَلاَمِ السِّفْرِ الْمَخْتُومِ الَّذِي يَدْفَعُونَهُ لِعَارِفِ الْكِتَابَةِ قَائِلِينَ: «اقْرَأْ هَذَا» فَيَقُولُ: «لاَ أَسْتَطِيعُ لأَنَّهُ مَخْتُومٌ».
\par 12 أَوْ يُدْفَعُ الْكِتَابُ لِمَنْ لاَ يَعْرِفُ الْكِتَابَةَ وَيُقَالُ لَهُ: «اقْرَأْ هَذَا» فَيَقُولُ: « لاَ أَعْرِفُ الْكِتَابَةَ».
\par 13 فَقَالَ السَّيِّدُ: «لأَنَّ هَذَا الشَّعْبَ قَدِ اقْتَرَبَ إِلَيَّ بِفَمِهِ وَأَكْرَمَنِي بِشَفَتَيْهِ وَأَمَّا قَلْبُهُ فَأَبْعَدَهُ عَنِّي وَصَارَتْ مَخَافَتُهُمْ مِنِّي وَصِيَّةَ النَّاسِ مُعَلَّمَةً
\par 14 لِذَلِكَ هَئَنَذَا أَعُودُ أَصْنَعُ بِهَذَا الشَّعْبِ عَجَباً وَعَجِيباً فَتَبِيدُ حِكْمَةُ حُكَمَائِهِ وَيَخْتَفِي فَهْمُ فُهَمَائِهِ».
\par 15 وَيْلٌ لِلَّذِينَ يَتَعَمَّقُونَ لِيَكْتُمُوا رَأْيَهُمْ عَنِ الرَّبِّ فَتَصِيرُ أَعْمَالُهُمْ فِي الظُّلْمَةِ وَيَقُولُونَ: «مَنْ يُبْصِرُنَا وَمَنْ يَعْرِفُنَا؟».
\par 16 يَا لَتَحْرِيفِكُمْ! هَلْ يُحْسَبُ الْجَابِلُ كَالطِّينِ حَتَّى يَقُولَ الْمَصْنُوعُ عَنْ صَانِعِهِ: «لَمْ يَصْنَعْنِي». أَوْ تَقُولُ الْجُبْلَةُ عَنْ جَابِلِهَا: «لَمْ يَفْهَمْ»؟
\par 17 أَلَيْسَ فِي مُدَّةٍ يَسِيرَةٍ جِدّاً يَتَحَوَّلُ لُبْنَانُ بُسْتَاناً وَالْبُسْتَانُ يُحْسَبُ وَعْراً؟
\par 18 وَيَسْمَعُ فِي ذَلِكَ الْيَوْمِ الصُّمُّ أَقْوَالَ السِّفْرِ وَتَنْظُرُ مِنَ الْقَتَامِ وَالظُّلْمَةِ عُيُونُ الْعُمْيِ
\par 19 وَيَزْدَادُ الْبَائِسُونَ فَرَحاً بِالرَّبِّ وَيَهْتِفُ مَسَاكِينُ النَّاسِ بِقُدُّوسِ إِسْرَائِيلَ.
\par 20 لأَنَّ الْعَاتِيَ قَدْ بَادَ وَفَنِيَ الْمُسْتَهْزِئُ وَانْقَطَعَ كُلُّ السَّاهِرِينَ عَلَى الإِثْمِ
\par 21 الَّذِينَ جَعَلُوا الإِنْسَانَ يُخْطِئُ بِكَلِمَةٍ وَنَصَبُوا فَخّاً لِلْمُنْصِفِ فِي الْبَابِ وَصَدُّوا الْبَارَّ بِالْبُطْلِ.
\par 22 لِذَلِكَ هَكَذَا يَقُولُ الرَّبُّ الَّذِي فَدَى إِبْرَاهِيمَ لِبَيْتِ يَعْقُوبَ: «لَيْسَ الآنَ يَخْجَلُ يَعْقُوبُ وَلَيْسَ الآنَ يَصْفَرُّ وَجْهُهُ.
\par 23 بَلْ عِنْدَ رُؤْيَةِ أَوْلاَدِهِ عَمَلِ يَدَيَّ فِي وَسَطِهِ يُقَدِّسُونَ اسْمِي وَيُقَدِّسُونَ قُدُّوسَ يَعْقُوبَ وَيَرْهَبُونَ إِلَهَ إِسْرَائِيلَ.
\par 24 وَيَعْرِفُ الضَّالُّو الأَرْوَاحِ فَهْماً وَيَتَعَلَّمُ الْمُتَمَرِّدُونَ تَعْلِيماً.

\chapter{30}

\par 1 وَيْلٌ لِلْبَنِينَ الْمُتَمَرِّدِينَ يَقُولُ الرَّبُّ حَتَّى أَنَّهُمْ يُجْرُونَ رَأْياً وَلَيْسَ مِنِّي وَيَسْكُبُونَ سَكِيباً وَلَيْسَ بِرُوحِي لِيَزِيدُوا خَطِيئَةً عَلَى خَطِيئَةٍ.
\par 2 الَّذِينَ يَذْهَبُونَ لِيَنْزِلُوا إِلَى مِصْرَ وَلَمْ يَسْأَلُوا فَمِي لِيَلْتَجِئُوا إِلَى حِصْنِ فِرْعَوْنَ وَيَحْتَمُوا بِظِلِّ مِصْرَ.
\par 3 فَيَصِيرُ لَكُمْ حِصْنُ فِرْعَوْنَ خَجَلاً وَالاِحْتِمَاءُ بِظِلِّ مِصْرَ خِزْياً.
\par 4 لأَنَّ رُؤَسَاءَهُ صَارُوا فِي صُوعَنَ وَبَلَغَ رُسُلُهُ إِلَى حَانِيسَ.
\par 5 قَدْ خَجِلَ الْجَمِيعُ مِنْ شَعْبٍ لاَ يَنْفَعُهُمْ. لَيْسَ لِلْمَعُونَةِ وَلاَ لِلْمَنْفَعَةِ بَلْ لِلْخَجَلِ وَلِلْخِزْيِ.
\par 6 وَحْيٌ مِنْ جِهَةِ بَهَائِمِ الْجَنُوبِ: فِي أَرْضِ شِدَّةٍ وَضِيقَةٍ مِنْهَا اللَّبْوَةُ وَالأَسَدُ الأَفْعَى وَالثُّعْبَانُ السَّامُّ الطَّيَّارُ يَحْمِلُونَ عَلَى أَكْتَافِ الْحَمِيرِ ثَرْوَتَهُمْ وَعَلَى أَسْنِمَةِ الْجِمَالِ كُنُوزَهُمْ إِلَى شَعْبٍ لاَ يَنْفَعُ.
\par 7 فَإِنَّ مِصْرَ تُعِينُ بَاطِلاً وَعَبَثاً لِذَلِكَ دَعَوْتُهَا «رَهَبَ الْجُلُوسِ».
\par 8 تَعَالَ الآنَ اكْتُبْ هَذَا عَُِنْدَهُمْ عَلَى لَوْحٍ وَارْسِمْهُ فِي سِفْرٍ لِيَكُونَ لِزَمَنٍ آتٍ لِلأَبَدِ إِلَى الدُّهُورِ.
\par 9 لأَنَّهُ شَعْبٌ مُتَمَرِّدٌ أَوْلاَدٌ كَذَبَةٌ أَوْلاَدٌ لَمْ يَشَاءُوا أَنْ يَسْمَعُوا شَرِيعَةَ الرَّبِّ.
\par 10 الَّذِينَ يَقُولُونَ لِلرَّائِينَ: «لاَ تَرُوا» وَلِلنَّاظِرِينَ: «لاَ تَنْظُرُوا لَنَا مُسْتَقِيمَاتٍ. كَلِّمُونَا بِالنَّاعِمَاتِ. انْظُرُوا مُخَادِعَاتٍ.
\par 11 حِيدُوا عَنِ الطَّرِيقِ. مِيلُوا عَنِ السَّبِيلِ. اعْزِلُوا مِنْ أَمَامِنَا قُدُّوسَ إِسْرَائِيلَ».
\par 12 لِذَلِكَ هَكَذَا يَقُولُ قُدُّوسُ إِسْرَائِيلَ: «لأَنَّكُمْ رَفَضْتُمْ هَذَا الْقَوْلَ وَتَوَكَّلْتُمْ عَلَى الظُّلْمِ وَالاِعْوِجَاجِ وَاسْتَنَدْتُمْ عَلَيْهِمَا
\par 13 لِذَلِكَ يَكُونُ لَكُمْ هَذَا الإِثْمُ كَصَدْعٍ مُنْقَضٍّ نَاتِئٍ فِي جِدَارٍ مُرْتَفِعٍ يَأْتِي هَدُّهُ بَغْتَةً فِي لَحْظَةٍ.
\par 14 وَيُكْسَرُ كَكَسْرِ إِنَاءِ الْخَزَّافِينَ مَسْحُوقاً بِلاَ شَفَقَةٍ حَتَّى لاَ يُوجَدُ فِي مَسْحُوقِهِ شَقْفَةٌ لأَخْذِ نَارٍ مِنَ الْمَوْقَدَةِ أَوْ لِغَرْفِ مَاءٍ مِنَ الْجُبِّ».
\par 15 لأَنَّهُ هَكَذَا قَالَ السَّيِّدُ الرَّبُّ قُدُّوسُ إِسْرَائِيلَ: «بِالرُّجُوعِ وَالسُّكُونِ تَخْلُصُونَ. بِالْهُدُوءِ وَالطُّمَأْنِينَةِ تَكُونُ قُوَّتُكُمْ». فَلَمْ تَشَاءُوا.
\par 16 وَقُلْتُمْ: «لاَ بَلْ عَلَى خَيْلٍ نَهْرُبُ». لِذَلِكَ تَهْرُبُونَ. وَ«عَلَى خَيْلٍ سَرِيعَةٍ نَرْكَبُ». لِذَلِكَ يُسْرِعُ طَارِدُوكُمْ.
\par 17 يَهْرُبُ أَلْفٌ مِنْ زَجْرَةِ وَاحِدٍ. مِنْ زَجْرَةِ خَمْسَةٍ تَهْرُبُونَ حَتَّى أَنَّكُمْ تَبْقُونَ كَسَارِيَةٍ عَلَى رَأْسِ جَبَلٍ وَكَرَايَةٍ عَلَى أَكَمَةٍ.
\par 18 وَلِذَلِكَ يَنْتَظِرُ الرَّبُّ لِيَتَرَأَّفَ عَلَيْكُمْ. وَلِذَلِكَ يَقُومُ لِيَرْحَمَكُمْ لأَنَّ الرَّبَّ إِلَهُ حَقٍّ. طُوبَى لِجَمِيعِ مُنْتَظِرِيهِ.
\par 19 لأَنَّ الشَّعْبَ فِي صِهْيَوْنَ يَسْكُنُ فِي أُورُشَلِيمَ. لاَ تَبْكِي بُكَاءً. يَتَرَأَّفُ عَلَيْكَ عِنْدَ صَوْتِ صُرَاخِكَ. حِينَمَا يَسْمَعُ يَسْتَجِيبُ لَكَ.
\par 20 وَيُعْطِيكُمُ السَّيِّدُ خُبْزاً فِي الضِّيقِ وَمَاءً فِي الشِّدَّةِ. لاَ يَخْتَبِئُ مُعَلِّمُوكَ بَعْدُ بَلْ تَرَى عَيْنَاكَ مُعَلِّمِيكَ
\par 21 وَأُذُنَاكَ تَسْمَعَانِ كَلِمَةً خَلْفَكَ قَائِلَةً: «هَذِهِ هِيَ الطَّرِيقُ. اسْلُكُوا فِيهَا». حِينَمَا تَمِيلُونَ إِلَى الْيَمِينِ وَحِينَمَا تَمِيلُونَ إِلَى الْيَسَارِ.
\par 22 وَتُنَجِّسُونَ صَفَائِحَ تَمَاثِيلِ فِضَّتِكُمُ الْمَنْحُوتَةِ وَغِشَاءَ تِمْثَالِ ذَهَبِكُمُ الْمَسْبُوكِ. تَطْرَحُهَا مِثْلَ فِرْصَةِ حَائِضٍ. تَقُولُ لَهَا: «اخْرُجِي».
\par 23 ثُمَّ يُعْطِي مَطَرَ زَرْعِكَ الَّذِي تَزْرَعُ الأَرْضَ بِهِ وَخُبْزَ غَلَّةِ الأَرْضِ فَيَكُونُ دَسَماً وَسَمِيناً. وَتَرْعَى مَاشِيَتُكَ فِي ذَلِكَ الْيَوْمِ فِي مَرْعًى وَاسِعٍ.
\par 24 وَالأَبْقَارُ وَالْحَمِيرُ الَّتِي تَعْمَلُ الأَرْضَ تَأْكُلُ عَلَفاً مُمَلَّحاً مُذَرَّى بِالْمِنْسَفِ وَالْمِذْرَاةِ.
\par 25 وَيَكُونُ عَلَى كُلِّ جَبَلٍ عَالٍ وَعَلَى كُلِّ أَكَمَةٍ مُرْتَفِعَةٍ سَوَاقٍ وَمَجَارِي مِيَاهٍ فِي يَوْمِ الْمَقْتَلَةِ الْعَظِيمَةِ حِينَمَا تَسْقُطُ الأَبْرَاجُ.
\par 26 وَيَكُونُ نُورُ الْقَمَرِ كَنُورِ الشَّمْسِ وَنُورُ الشَّمْسِ يَكُونُ سَبْعَةَ أَضْعَافٍ كَنُورِ سَبْعَةِ أَيَّامٍ فِي يَوْمٍ يَجْبُرُ الرَّبُّ كَسْرَ شَعْبِهِ وَيَشْفِي رَضَّ ضَرْبِهِ.
\par 27 هُوَذَا اسْمُ الرَّبِّ يَأْتِي مِنْ بَعِيدٍ. غَضَبُهُ مُشْتَعِلٌ وَالْحَرِيقُ عَظِيمٌ. شَفَتَاهُ مُمْتَلِئَتَانِ سَخَطاً وَلِسَانُهُ كَنَارٍ آكِلَةٍ
\par 28 وَنَفْخَتُهُ كَنَهْرٍ غَامِرٍ يَبْلُغُ إِلَى الرَّقَبَةِ. لِغَرْبَلَةِ الأُمَمِ بِغُرْبَالِ السُّوءِ وَعَلَى فُكُوكِ الشُّعُوبِ رَسَنٌ مُضِلٌّ.
\par 29 تَكُونُ لَكُمْ أُغْنِيَةٌ كَلَيْلَةِ تَقْدِيسِ عِيدٍ وَفَرَحُ قَلْبٍ كَالسَّائِرِ بِالنَّايِ لِيَأْتِيَ إِلَى جَبَلِ الرَّبِّ إِلَى صَخْرِ إِسْرَائِيلَ.
\par 30 وَيُسَمِّعُ الرَّبُّ جَلاَلَ صَوْتِهِ وَيُرِي نُزُولَ ذِرَاعِهِ بِهَيَجَانِ غَضَبٍ وَلَهِيبِ نَارٍ آكِلَةٍ نَوْءٍ وَسَيْلٍ وَحِجَارَةِ بَرَدٍ.
\par 31 لأَنَّهُ مِنْ صَوْتِ الرَّبِّ يَرْتَاعُ أَشُّورُ. بِالْقَضِيبِ يَضْرِبُ.
\par 32 وَيَكُونُ كُلُّ مُرُورِ عَصَا الْقَضَاءِ الَّتِي يُنْزِلُهَا الرَّبُّ عَلَيْهِ بِالدُّفُوفِ وَالْعِيدَانِ. وَبِحُرُوبٍ ثَائِرَةٍ يُحَارِبُهُ.
\par 33 لأَنَّ «تُفْتَةَ» مُرَتَّبَةٌ مُنْذُ الأَمْسِ مُهَيَّأَةٌ هِيَ أَيْضاً لِلْمَلِكِ عَمِيقَةٌ وَاسِعَةٌ كُومَتُهَا نَارٌ وَحَطَبٌ بِكَثْرَةٍ. نَفْخَةُ الرَّبِّ كَنَهْرِ كِبْرِيتٍ تُوقِدُهَا.

\chapter{31}

\par 1 وَيْلٌ لِلَّذِينَ يَنْزِلُونَ إِلَى مِصْرَ لِلْمَعُونَةِ وَيَسْتَنِدُونَ عَلَى الْخَيْلِ وَيَتَوَكَّلُونَ عَلَى الْمَرْكَبَاتِ لأَنَّهَا كَثِيرَةٌ وَعَلَى الْفُرْسَانِ لأَنَّهُمْ أَقْوِيَاءُ جِدّاً وَلاَ يَنْظُرُونَ إِلَى قُدُّوسِ إِسْرَائِيلَ وَلاَ يَطْلُبُونَ الرَّبَّ.
\par 2 وَهُوَ أَيْضاً حَكِيمٌ وَيَأْتِي بِالشَّرِّ وَلاَ يَرْجِعُ بِكَلاَمِهِ وَيَقُومُ عَلَى بَيْتِ فَاعِلِي الشَّرِّ وَعَلَى مَعُونَةِ فَاعِلِي الإِثْمِ.
\par 3 وَأَمَّا الْمِصْرِيُّونَ فَهُمْ أُنَاسٌ لاَ آلِهَةٌ وَخَيْلُهُمْ جَسَدٌ لاَ رُوحٌ. وَالرَّبُّ يَمُدُّ يَدَهُ فَيَعْثُرُ الْمُعِينُ وَيَسْقُطُ الْمُعَانُ وَيَفْنَيَانِ كِلاَهُمَا مَعاً.
\par 4 لأَنَّهُ هَكَذَا قَالَ لِي الرَّبُّ: «كَمَا يَهِرُّ فَوْقَ فَرِيسَتِهِ الأَسَدُ وَالشِّبْلُ (الَّذِي يُدْعَى عَلَيْهِ جَمَاعَةٌ مِنَ الرُّعَاةِ وَهُوَ لاَ يَرْتَاعُ مِنْ صَوْتِهِمْ وَلاَ يَتَذَلَّلُ لِجُمْهُورِهِمْ) هَكَذَا يَنْزِلُ رَبُّ الْجُنُودِ لِلْمُحَارَبَةِ عَنْ جَبَلِ صِهْيَوْنَ وَعَنْ أَكَمَتِهَا.
\par 5 كَطُيُورٍ مُرِفَّةٍ هَكَذَا يُحَامِي رَبُّ الْجُنُودِ عَنْ أُورُشَلِيمَ. يُحَامِي فَيُنْقِذُ. يَعْفُو فَيُنَجِّي».
\par 6 اِرْجِعُوا إِلَى الَّذِي ارْتَدَّ بَنُو إِسْرَائِيلَ عَنْهُ مُتَعَمِّقِينَ.
\par 7 لأَنْ فِي ذَلِكَ الْيَوْمِ يَرْفُضُونَ كُلُّ وَاحِدٍ أَوْثَانَ فِضَّتِهِ وَأَوْثَانَ ذَهَبِهِ الَّتِي صَنَعَتْهَا لَكُمْ أَيْدِيكُمْ خَطِيئَةً.
\par 8 وَيَسْقُطُ أَشُّورُ بِسَيْفِ غَيْرِ رَجُلٍ وَسَيْفُ غَيْرِ إِنْسَانٍ يَأْكُلُهُ فَيَهْرُبُ مِنْ أَمَامِ السَّيْفِ وَيَكُونُ مُخْتَارُوهُ تَحْتَ الْجِزْيَةِ.
\par 9 وَصَخْرُهُ مِنَ الْخَوْفِ يَزُولُ وَمِنَ الرَّايَةِ يَرْتَعِبُ رُؤَسَاؤُهُ يَقُولُ الرَّبُّ الَّذِي لَهُ نَارٌ فِي صِهْيَوْنَ وَلَهُ تَنُّورٌ فِي أُورُشَلِيمَ.

\chapter{32}

\par 1 هُوَذَا بِالْعَدْلِ يَمْلِكُ مَلِكٌ وَرُؤَسَاءُ بِالْحَقِّ يَتَرَأَّسُونَ.
\par 2 وَيَكُونُ إِنْسَانٌ كَمَخْبَأٍ مِنَ الرِّيحِ وَسِتَارَةٍ مِنَ السَّيْلِ كَسَوَاقِي مَاءٍ فِي مَكَانٍ يَابِسٍ كَظِلِّ صَخْرَةٍ عَظِيمَةٍ فِي أَرْضٍ مُعْيِيَةٍ.
\par 3 وَلاَ تَحْسِرُ عُيُونُ النَّاظِرِينَ وَآذَانُ السَّامِعِينَ تَصْغَى
\par 4 وَقُلُوبُ الْمُتَسَرِّعِينَ تَفْهَمُ عِلْماً وَأَلْسِنَةُ الْعَيِيِّينَ تُبَادِرُ إِلَى التَّكَلُّمِ فَصِيحاً.
\par 5 وَلاَ يُدْعَى اللَّئِيمُ بَعْدُ كَرِيماً وَلاَ الْمَاكِرُ يُقَالُ لَهُ نَبِيلٌ.
\par 6 لأَنَّ اللَّئِيمَ يَتَكَلَّمُ بِاللُّؤْمِ وَقَلْبُهُ يَعْمَلُ إِثْماً لِيَصْنَعَ نِفَاقاً وَيَتَكَلَّمَ عَلَى الرَّبِّ بِافْتِرَاءٍ وَيُفْرِغَ نَفْسَ الْجَائِعِ وَيَقْطَعَ شُرْبَ الْعَطْشَانِ.
\par 7 وَالْمَاكِرُ آلاَتُهُ رَدِيئَةٌ. هُوَ يَتَآمَرُ بِالْخَبَائِثِ لِيُهْلِكَ الْبَائِسِينَ بِأَقْوَالِ الْكَذِبِ حَتَّى فِي تَكَلُّمِ الْمِسْكِينِ بِالْحَقِّ.
\par 8 وَأَمَّا الْكَرِيمُ فَبِالْكَرَائِمِ يَتَآمَرُ وَهُوَ بِالْكَرَائِمِ يَقُومُ.
\par 9 أَيَّتُهَا النِّسَاءُ الْمُطْمَئِنَّاتُ قُمْنَ اسْمَعْنَ صَوْتِي. أَيَّتُهَا الْبَنَاتُ الْوَاثِقَاتُ اصْغِينَ لِقَوْلِي.
\par 10 أَيَّاماً عَلَى سَنَةٍ تَرْتَعِدْنَ أَيَّتُهَا الْوَاثِقَاتُ لأَنَّهُ قَدْ مَضَى الْقِطَافُ. الاِجْتِنَاءُ لاَ يَأْتِي.
\par 11 اِرْتَجِفْنَ أَيَّتُهَا الْمُطْمَئِنَّاتُ. ارْتَعِدْنَ أَيَّتُهَا الْوَاثِقَاتُ. تَجَرَّدْنَ وَتَعَرَّيْنَ وَتَنَطَّقْنَ عَلَى الأَحْقَاءِ
\par 12 لاَطِمَاتٍ عَلَى الثُّدِيِّ مِنْ أَجْلِ الْحُقُولِ الْمُشْتَهَاةِ وَمِنْ أَجْلِ الْكَرْمَةِ الْمُثْمِرَةِ.
\par 13 عَلَى أَرْضِ شَعْبِي يَطْلَعُ شَوْكٌ وَحَسَكٌ حَتَّى فِي كُلِّ بُيُوتِ الْفَرَحِ مِنَ الْمَدِينَةِ الْمُبْتَهِجَةِ.
\par 14 لأَنَّ الْقَصْرَ قَدْ هُدِمَ. جُمْهُورُ الْمَدِينَةِ قَدْ تُرِكَ. الأَكَمَةُ وَالْبُرْجُ صَارَا مَغَايِرَ إِلَى الأَبَدِ مَرَحاً لِحَمِيرِ الْوَحْشِ مَرْعىً لِلْقُطْعَانِ.
\par 15 إِلَى أَنْ يُسْكَبَ عَلَيْنَا رُوحٌ مِنَ الْعَلاَءِ فَتَصِيرَ الْبَرِّيَّةُ بُسْتَاناً وَيُحْسَبَ الْبُسْتَانُ وَعْراً.
\par 16 فَيَسْكُنُ فِي الْبَرِّيَّةِ الْحَقُّ وَالْعَدْلُ فِي الْبُسْتَانِ يُقِيمُ.
\par 17 وَيَكُونُ صُنْعُ الْعَدْلِ سَلاَماً وَعَمَلُ الْعَدْلِ سُكُوناً وَطُمَأْنِينَةً إِلَى الأَبَدِ.
\par 18 وَيَسْكُنُ شَعْبِي فِي مَسْكَنِ السَّلاَمِ وَفِي مَسَاكِنَ مُطْمَئِنَّةٍ وَفِي مَحَلاَّتٍ أَمِينَةٍ.
\par 19 وَيَنْزِلُ بَرَدٌ بِهُبُوطِ الْوَعْرِ وَإِلَى الْحَضِيضِ تُوضَعُ الْمَدِينَةُ.
\par 20 طُوبَاكُمْ أَيُّهَا الزَّارِعُونَ عَلَى كُلِّ الْمِيَاهِ الْمُسَرِّحُونَ أَرْجُلَ الثَّوْرِ وَالْحِمَارِ.

\chapter{33}

\par 1 وَيْلٌ لَكَ أَيُّهَا الْمُخْرِبُ وَأَنْتَ لَمْ تُخْرَبْ وَأَيُّهَا النَّاهِبُ وَلَمْ يَنْهَبُوكَ. حِينَ تَنْتَهِي مِنَ التَّخْرِيبِ تُخْرَبُ وَحِينَ تَفْرَغُ مِنَ النَّهْبِ يَنْهَبُونَكَ.
\par 2 يَا رَبُّ تَرَأَّفْ عَلَيْنَا. إِيَّاكَ انْتَظَرْنَا. كُنْ عَضُدَهُمْ فِي الْغَدَوَاتِ. خَلاَصَنَا أَيْضاً فِي وَقْتِ الشِّدَّةِ.
\par 3 مِنْ صَوْتِ الضَّجِيجِ هَرَبَتِ الشُّعُوبُ. مِنِ ارْتِفَاعِكَ تَبَدَّدَتِ الأُمَمُ.
\par 4 وَيُجْنَى سَلْبُكُمْ جَنْي الْجَرَادِ. كَتَرَاكُضِ الْجُنْدُبِ يُتَرَاكَضُ عَلَيْهِ.
\par 5 تَعَالَى الرَّبُّ لأَنَّهُ سَاكِنٌ فِي الْعَلاَءِ. مَلَأَ صِهْيَوْنَ حَقّاً وَعَدْلاً.
\par 6 فَيَكُونُ أَمَانُ أَوْقَاتِكَ وَفْرَةَ خَلاَصٍ وَحِكْمَةٍ وَمَعْرِفَةٍ. مَخَافَةُ الرَّبِّ هِيَ كَنْزُهُ.
\par 7 هُوَذَا أَبْطَالُهُمْ قَدْ صَرَخُوا خَارِجاً. رُسُلُ السَّلاَمِ يَبْكُونَ بِمَرَارَةٍ.
\par 8 خَلَتِ السِّكَكُ. بَادَ عَابِرُ السَّبِيلِ. نَكَثَ الْعَهْدَ. رَذَلَ الْمُدُنَ. لَمْ يَعْتَدَّ بِإِنْسَانٍ.
\par 9 نَاحَتْ ذَبُلَتِ الأَرْضُ. خَجِلَ لُبْنَانُ وَتَلِفَ. صَارَ شَارُونُ كَالْبَادِيَةِ. نُثِرَ بَاشَانُ وَكَرْمَلُ.
\par 10 اَلآنَ أَقُومُ يَقُولُ الرَّبُّ. الآنَ أَصْعَدُ. الآنَ أَرْتَفِعُ.
\par 11 تَحْبَلُونَ بِحَشِيشٍ تَلِدُونَ قَشِيشاً. نَفَسُكُمْ نَارٌ تَأْكُلُكُمْ.
\par 12 وَتَصِيرُ الشُّعُوبُ وَقُودَ كِلْسٍ أَشْوَاكاً مَقْطُوعَةً تُحْرَقُ بِالنَّارِ.
\par 13 اِسْمَعُوا أَيُّهَا الْبَعِيدُون مَا صَنَعْتُ وَاعْرِفُوا أَيُّهَا الْقَرِيبُونَ بَطْشِي.
\par 14 ارْتَعَبَ فِي صِهْيَوْنَ الْخُطَاةُ. أَخَذَتِ الرِّعْدَةُ الْمُنَافِقِينَ. مَنْ مِنَّا يَسْكُنُ فِي نَارٍ آكِلَةٍ؟ مَنْ مِنَّا يَسْكُنُ فِي وَقَائِدَ أَبَدِيَّةٍ؟
\par 15 السَّالِكُ بِالْحَقِّ وَالْمُتَكَلِّمُ بِالاِسْتِقَامَةِ الرَّاذِلُ مَكْسَبَ الْمَظَالِمِ النَّافِضُ يَدَيْهِ مِنْ قَبْضِ الرَّشْوَةِ الَّذِي يَسُدُّ أُذُنَيْهِ عَنْ سَمْعِ الدِّمَاءِ وَيُغَمِّضُ عَيْنَيْهِ عَنِ النَّظَرِ إِلَى الشَّرِّ
\par 16 هُوَ فِي الأَعَالِي يَسْكُنُ. حُصُونُ الصُّخُورِ مَلْجَأُهُ. يُعْطَى خُبْزَهُ وَمِيَاهُهُ مَأْمُونَةٌ.
\par 17 اَلْمَلِكَ بِبَهَائِهِ تَنْظُرُ عَيْنَاكَ. تَرَيَانِ أَرْضاً بَعِيدَةً.
\par 18 قَلْبُكَ يَتَذَكَّرُ الرُّعْبَ. أَيْنَ الْكَاتِبُ أَيْنَ الْجَابِي أَيْنَ الَّذِي عَدَّ الأَبْرَاجَ؟
\par 19 الشَّعْبَ الشَّرِسَ لاَ تَرَى: الشَّعْبَ الْغَامِضَ اللُّغَةِ عَنِ الإِدْرَاكِ الْعَيِيَّ بِلِسَانٍ لاَ يُفْهَمُ.
\par 20 اُنْظُرْ صِهْيَوْنَ مَدِينَةَ أَعْيَادِنَا. عَيْنَاكَ تَرَيَانِ أُورُشَلِيمَ مَسْكَناً مُطْمَئِنّاً خَيْمَةً لاَ تَنْتَقِلُ. لاَ تُقْلَعُ أَوْتَادُهَا إِلَى الأَبَدِ وَشَيْءٌ مِنْ أَطْنَابِهَا لاَ يَنْقَطِعُ.
\par 21 بَلْ هُنَاكَ الرَّبُّ الْعَزِيزُ لَنَا مَكَانُ أَنْهَارٍ وَتُرَعٍ وَاسِعَةِ الشَّوَاطِئِ. لاَ يَسِيرُ فِيهَا قَارِبٌ بِمِقْذَافٍ وَسَفِينَةٌ عَظِيمَةٌ لاَ تَجْتَازُ فِيهَا.
\par 22 (فَإِنَّ الرَّبَّ قَاضِينَا. الرَّبُّ شَارِعُنَا. الرَّبُّ مَلِكُنَا هُوَ يُخَلِّصُنَا).
\par 23 ارْتَخَتْ حِبَالُكِ. لاَ يُشَدِّدُونَ قَاعِدَةَ سَارِيَتِهِمْ. لاَ يَنْشُرُونَ قِلْعاً. حِينَئِذٍ قُسِمَ سَلْبُ غَنِيمَةٍ كَثِيرَةٍ. الْعُرْجُ نَهَبُوا نَهْباً.
\par 24 وَلاَ يَقُولُ سَاكِنٌ: «أَنَا مَرِضْتُ». الشَّعْبُ السَّاكِنُ فِيهَا مَغْفُورُ الإِثْمِ.

\chapter{34}

\par 1 اِقْتَرِبُوا أَيُّهَا الأُمَمُ لِتَسْمَعُوا وَأَيُّهَا الشُّعُوبُ اصْغُوا. لِتَسْمَعِ الأَرْضُ وَمِلْؤُهَا. الْمَسْكُونَةُ وَكُلُّ نَتَائِجِهَا.
\par 2 لأَنَّ لِلرَّبِّ سَخَطاً عَلَى كُلِّ الأُمَمِ وَحُمُوّاً عَلَى كُلِّ جَيْشِهِمْ. قَدْ حَرَّمَهُمْ دَفَعَهُمْ إِلَى الذَّبْحِ.
\par 3 فَقَتْلاَهُمْ تُطْرَحُ وَجِيَفُهُمْ تَصْعَدُ نَتَانَتُهَا وَتَسِيلُ الْجِبَالُ بِدِمَائِهِمْ.
\par 4 وَيَفْنَى كُلُّ جُنْدِ السَّمَاوَاتِ وَتَلْتَفُّ السَّمَاوَاتُ كَدَرْجٍ وَكُلُّ جُنْدِهَا يَنْتَثِرُ كَانْتِثَارِ الْوَرَقِ مِنَ الْكَرْمَةِ وَالسُّقَاطِ مِنَ التِّينَةِ.
\par 5 لأَنَّهُ قَدْ رَوِيَ فِي السَّمَاوَاتِ سَيْفِي. هُوَذَا عَلَى أَدُومَ يَنْزِلُ وَعَلَى شَعْبٍ حَرَّمْتُهُ لِلدَّيْنُونَةِ.
\par 6 لِلرَّبِّ سَيْفٌ قَدِ امْتَلَأَ دَماً اطَّلَى بِشَحْمٍ بِدَمِ خِرَافٍ وَتُيُوسٍ بِشَحْمِ كُلَى كِبَاشٍ. لأَنَّ لِلرَّبِّ ذَبِيحَةً فِي بُصْرَةَ وَذَبْحاً عَظِيماً فِي أَرْضِ أَدُومَ.
\par 7 وَيَسْقُطُ الْبَقَرُ الْوَحْشِيُّ مَعَهَا وَالْعُجُولُ مَعَ الثِّيرَانِ وَتُرْوَى أَرْضُهُمْ مِنَ الدَّمِ وَتُرَابُهُمْ مِنَ الشَّحْمِ يُسَمَّنُ.
\par 8 لأَنَّ لِلرَّبِّ يَوْمَ انْتِقَامٍ سَنَةَ جَزَاءٍ مِنْ أَجْلِ دَعْوَى صِهْيَوْنَ.
\par 9 وَتَتَحَوَّلُ أَنْهَارُهَا زِفْتاً وَتُرَابُهَا كِبْرِيتاً وَتَصِيرُ أَرْضُهَا زِفْتاً مُشْتَعِلاً.
\par 10 لَيْلاً وَنَهَاراً لاَ تَنْطَفِئُ. إِلَى الأَبَدِ يَصْعَدُ دُخَانُهَا. مِنْ دَوْرٍ إِلَى دَوْرٍ تُخْرَبُ. إِلَى أَبَدِ الآبِدِينَ لاَ يَكُونُ مَنْ يَجْتَازُ فِيهَا.
\par 11 وَيَرِثُهَا الْقُوقُ وَالْقُنْفُذُ وَالْكَرْكِيُّ وَالْغُرَابُ يَسْكُنَانِ فِيهَا وَيُمَدُّ عَلَيْهَا خَيْطُ الْخَرَابِ وَمِطْمَارُ الْخَلاَءِ.
\par 12 أَشْرَافُهَا لَيْسَ هُنَاكَ مَنْ يَدْعُونَهُ لِلْمُلْكِ وَكُلُّ رُؤَسَائِهَا يَكُونُونَ عَدَماً.
\par 13 وَيَطْلَعُ فِي قُصُورِهَا الشَّوْكُ. الْقَرِيصُ وَالْعَوْسَجُ فِي حُصُونِهَا فَتَكُونُ مَسْكَناً لِلذِّئَابِ وَدَاراً لِبَنَاتِ النَّعَامِ.
\par 14 وَتُلاَقِي وُحُوشُ الْقَفْرِ بَنَاتِ آوَى وَمَعْزُ الْوَحْشِ يَدْعُو صَاحِبَهُ. هُنَاكَ يَسْتَقِرُّ اللَّيْلُ وَيَجِدُ لِنَفْسِهِ مَحَلاًّ.
\par 15 هُنَاكَ تُحْجِرُ النَّكَّازَةُ وَتَبِيضُ وَتُفْرِخُ وَتُرَبِّي تَحْتَ ظِلِّهَا. وَهُنَاكَ تَجْتَمِعُ الشَّوَاهِينُ بَعْضُهَا بِبَعْضٍ.
\par 16 فَتِّشُوا فِي سِفْرِ الرَّبِّ وَاقْرَأُوا. وَاحِدَةٌ مِنْ هَذِهِ لاَ تُفْقَدُ. لاَ يُغَادِرُ شَيْءٌ صَاحِبَهُ لأَنَّ فَمَهُ هُوَ قَدْ أَمَرَ وَرُوحَهُ هُوَ جَمَعَهَا.
\par 17 وَهُوَ قَدْ أَلْقَى لَهَا قُرْعَةً وَيَدُهُ قَسَمَتْهَا لَهَا بِالْخَيْطِ. إِلَى الأَبَدِ تَرِثُهَا. إِلَى دَوْرٍ فَدَوْرٍ تَسْكُنُ فِيهَا.

\chapter{35}

\par 1 تَفْرَحُ الْبَرِّيَّةُ وَالأَرْضُ الْيَابِسَةُ وَيَبْتَهِجُ الْقَفْرُ وَيُزْهِرُ كَالنَّرْجِسِ.
\par 2 يُزْهِرُ إِزْهَاراً وَيَبْتَهِجُ ابْتِهَاجاً وَيُرَنِّمُ. يُدْفَعُ إِلَيْهِ مَجْدُ لُبْنَانَ. بَهَاءُ كَرْمَلَ وَشَارُونَ. هُمْ يَرُونَ مَجْدَ الرَّبِّ بَهَاءَ إِلَهِنَا.
\par 3 شَدِّدُوا الأَيَادِيَ الْمُسْتَرْخِيَةَ وَالرُّكَبَ الْمُرْتَعِشَةَ ثَبِّتُوهَا.
\par 4 قُولُوا لِخَائِفِي الْقُلُوبِ: «تَشَدَّدُوا لاَ تَخَافُوا. هُوَذَا إِلَهُكُمُ. الاِنْتِقَامُ يَأْتِي. جِزَاءُ اللَّهِ. هُوَ يَأْتِي وَيُخَلِّصُكُمْ».
\par 5 حِينَئِذٍ تَتَفَتَّحُ عُيُونُ الْعُمْيِ وَآذَانُ الصُّمِّ تَتَفَتَّحُ.
\par 6 حِينَئِذٍ يَقْفِزُ الأَعْرَجُ كَالإِيَّلِ وَيَتَرَنَّمُ لِسَانُ الأَخْرَسِ لأَنَّهُ قَدِ انْفَجَرَتْ فِي الْبَرِّيَّةِ مِيَاهٌ وَأَنْهَارٌ فِي الْقَفْرِ.
\par 7 وَيَصِيرُ السَّرَابُ أَجَماً وَالْمَعْطَشَةُ يَنَابِيعَ مَاءٍ. فِي مَسْكَنِ الذِّئَابِ فِي مَرْبِضِهَا دَارٌ لِلْقَصَبِ وَالْبَرْدِيِّ.
\par 8 وَتَكُونُ هُنَاكَ سِكَّةٌ وَطَرِيقٌ يُقَالُ لَهَا «الطَّرِيقُ الْمُقَدَّسَةُ». لاَ يَعْبُرُ فِيهَا نَجِسٌ بَلْ هِيَ لَهُمْ. مَنْ سَلَكَ فِي الطَّرِيقِ حَتَّى الْجُهَّالُ لاَ يَضِلُّ.
\par 9 لاَ يَكُونُ هُنَاكَ أَسَدٌ. وَحْشٌ مُفْتَرِسٌ لاَ يَصْعَدُ إِلَيْهَا. لاَ يُوجَدُ هُنَاكَ. بَلْ يَسْلُكُ الْمَفْدِيُّونَ فِيهَا.
\par 10 وَمَفْدِيُّو الرَّبِّ يَرْجِعُونَ وَيَأْتُونَ إِلَى صِهْيَوْنَ بِتَرَنُّمٍ وَفَرَحٌ أَبَدِيٌّ عَلَى رُؤُوسِهِمِ. ابْتِهَاجٌ وَفَرَحٌ يُدْرِكَانِهِمْ. وَيَهْرُبُ الْحُزْنُ وَالتَّنَهُّدُ.

\chapter{36}

\par 1 وَكَانَ فِي السَّنَةِ الرَّابِعَةِ عَشَرَةَ لِلْمَلِكِ حَزَقِيَّا أَنَّ سَنْحَارِيبَ مَلِكَ أَشُّورَ صَعِدَ عَلَى كُلِّ مُدُنِ يَهُوذَا الْحَصِينَةِ وَأَخَذَهَا.
\par 2 وَأَرْسَلَ مَلِكُ أَشُّورَ رَبْشَاقَى مِنْ لَخِيشَ إِلَى أُورُشَلِيمَ إِلَى الْمَلِكِ حَزَقِيَّا بِجَيْشٍ عَظِيمٍ فَوَقَفَ عِنْدَ قَنَاةِ الْبِرْكَةِ الْعُلْيَا فِي طَرِيقِ حَقْلِ الْقَصَّارِ.
\par 3 فَخَرَجَ إِلَيْهِ أَلِيَاقِيمُ بْنُ حِلْقِيَّا الَّذِي عَلَى الْبَيْتِ وَشِبْنَةُ الْكَاتِبُ وَيُوآخُ بْنُ آسَافَ الْمُسَجِّلُ.
\par 4 فَقَالَ لَهُمْ رَبْشَاقَى: «قُولُوا لِحَزَقِيَّا: هَكَذَا يَقُولُ الْمَلِكُ الْعَظِيمُ مَلِكُ أَشُّورَ: مَا هُوَ هَذَا الاِتِّكَالُ الَّذِي اتَّكَلْتَهُ؟
\par 5 أَقُولُ إِنَّمَا كَلاَمُ الشَّفَتَيْنِ هُوَ مَشُورَةٌ وَبَأْسٌ لِلْحَرْبِ. وَالآنَ عَلَى مَنِ اتَّكَلْتَ حَتَّى عَصِيتَ عَلَيَّ؟
\par 6 إِنَّكَ قَدِ اتَّكَلْتَ عَلَى عُكَّازِ هَذِهِ الْقَصَبَةِ الْمَرْضُوضَةِ عَلَى مِصْرَ الَّتِي إِذَا تَوَكَّأَ أَحَدٌ عَلَيْهَا دَخَلَتْ فِي كَفِّهِ وَثَقَبَتْهَا. هَكَذَا فِرْعَوْنُ مَلِكُ مِصْرَ لِجَمِيعِ الْمُتَوَكِّلِينَ عَلَيْهِ.
\par 7 وَإِذَا قُلْتَ لِي: عَلَى الرَّبِّ إِلَهِنَا اتَّكَلْنَا أَفَلَيْسَ هُوَ الَّذِي أَزَالَ حَزَقِيَّا مُرْتَفَعَاتِهِ وَمَذَابِحَهُ وَقَالَ لِيَهُوذَا وَلأُورُشَلِيمَ: أَمَامَ هَذَا الْمَذْبَحِ تَسْجُدُونَ.
\par 8 فَالآنَ رَاهِنْ سَيِّدِي مَلِكَ أَشُّورَ فَأُعْطِيكَ أَلْفَيْ فَرَسٍ إِنِ اسْتَطَعْتَ أَنْ تَجْعَلَ عَلَيْهَا رَاكِبِينَ!
\par 9 فَكَيْفَ تَرُدُّ وَجْهَ وَالٍ وَاحِدٍ مِنْ عَبِيدِ سَيِّدِي الصِّغَارِ وَتَتَّكِلُ عَلَى مِصْرَ لأَجْلِ مَرْكَبَاتٍ وَفُرْسَانٍ؟
\par 10 وَالآنَ هَلْ بِدُونِ الرَّبِّ صَعِدْتُ عَلَى هَذِهِ الأَرْضِ لأُخْرِبَهَا؟ الرَّبُّ قَالَ لِي اصْعَدْ: إِلَى هَذِهِ الأَرْضِ وَاخْرِبْهَا».
\par 11 فَقَالَ أَلِيَاقِيمُ وَشِبْنَةُ وَيُوآخُ لِرَبْشَاقَى: «كَلِّمْ عَبِيدَكَ بِالأَرَامِيِّ لأَنَّنَا نَفْهَمُهُ وَلاَ تُكَلِّمْنَا بِالْيَهُودِيِّ فِي مَسَامِعِ الشَّعْبِ الَّذِينَ عَلَى السُّورِ».
\par 12 فَقَالَ رَبْشَاقَى: «هَلْ إِلَى سَيِّدِكَ وَإِلَيْكَ أَرْسَلَنِي سَيِّدِي لأَتَكَلَّمَ بِهَذَا الْكَلاَم؟ أَلَيْسَ إِلَى الرِّجَالِ الْجَالِسِينَ عَلَى السُّورِ لِيَأْكُلُوا عَذِرَتَهُمْ وَيَشْرَبُوا بَوْلَهُمْ مَعَكُمْ؟».
\par 13 ثُمَّ وَقَفَ رَبْشَاقَى وَنَادَى بِصَوْتٍ عَظِيمٍ بِالْيَهُودِيِّ: «اسْمَعُوا كَلاَمَ الْمَلِكِ الْعَظِيمِ مَلِكِ أَشُّورَ.
\par 14 هَكَذَا يَقُولُ الْمَلِكُ: لاَ يَخْدَعْكُمْ حَزَقِيَّا لأَنَّهُ لاَ يَقْدِرُ أَنْ يُنْقِذَكُمْ
\par 15 وَلاَ يَجْعَلْكُمْ حَزَقِيَّا تَتَّكِلُونَ عَلَى الرَّبِّ قَائِلاً: إِنْقَاذاً يُنْقِذُنَا الرَّبُّ. لاَ تُدْفَعُ هَذِهِ الْمَدِينَةُ إِلَى يَدِ مَلِكِ أَشُّورَ.
\par 16 لاَ تَسْمَعُوا لِحَزَقِيَّا. لأَنَّهُ هَكَذَا يَقُولُ مَلِكُ أَشُّورَ: اعْقِدُوا مَعِي صُلْحاً وَاخْرُجُوا إِلَيَّ وَكُلُوا كُلُّ وَاحِدٍ مِنْ جَفْنَتِهِ وَكُلُّ وَاحِدٍ مِنْ تِينَتِهِ وَاشْرَبُوا كُلُّ وَاحِدٍ مَاءَ بِئْرِهِ
\par 17 حَتَّى آتِيَ وَآخُذَكُمْ إِلَى أَرْضٍ مِثْلِ أَرْضِكُمْ أَرْضِ حِنْطَةٍ وَخَمْرٍ أَرْضِ خُبْزٍ وَكُرُومٍ.
\par 18 لاَ يَغُرَّكُمْ حَزَقِيَّا قَائِلاً: الرَّبُّ يُنْقِذُنَا. هَلْ أَنْقَذَ آلِهَةُ الأُمَمِ كُلُّ وَاحِدٍ أَرْضَهُ مِنْ يَدِ مَلِكِ أَشُّورَ؟
\par 19 أَيْنَ آلِهَةُ حَمَاةَ وَأَرْفَادَ؟ أَيْنَ آلِهَةُ سَفَرْوَايِمَ؟ هَلْ أَنْقَذُوا السَّامِرَةَ مِنْ يَدِي؟
\par 20 مَنْ مِنْ كُلِّ آلِهَةِ هَذِهِ الأَرَاضِي أَنْقَذَ أَرْضَهُمْ مِنْ يَدِي حَتَّى يُنْقِذَ الرَّبُّ أُورُشَلِيمَ مِنْ يَدِي؟»
\par 21 فَسَكَتُوا وَلَمْ يُجِيبُوا بِكَلِمَةٍ لأَنَّ أَمْرَ الْمَلِكِ كَانَ: «لاَ تُجِيبُوهُ».
\par 22 فَجَاءَ أَلِيَاقِيمُ بْنُ حِلْقِيَّا الَّذِي عَلَى الْبَيْتِ وَشِبْنَةُ الْكَاتِبُ وَيُوآخُ بْنُ آسَافَ الْمُسَجِّلُ إِلَى حَزَقِيَّا وَثِيَابُهُمْ مُمَزَّقَةٌ فَأَخْبَرُوهُ بِكَلاَمِ رَبْشَاقَى.

\chapter{37}

\par 1 فَلَمَّا سَمِعَ الْمَلِكُ حَزَقِيَّا ذَلِكَ مَزَّقَ ثِيَابَهُ وَتَغَطَّى بِمِسْحٍ وَدَخَلَ بَيْتَ الرَّبِّ.
\par 2 وَأَرْسَلَ أَلِيَاقِيمَ الَّذِي عَلَى الْبَيْتِ وَشِبْنَةَ الْكَاتِبَ وَشُيُوخَ الْكَهَنَةِ مُتَغَطِّينَ بِمُسُوحٍ إِلَى إِشَعْيَاءَ بْنِ آمُوصَ النَّبِيِّ.
\par 3 فَقَالُوا لَهُ: «هَكَذَا يَقُولُ حَزَقِيَّا: هَذَا الْيَوْمُ يَوْمُ شِدَّةٍ وَتَأْدِيبٍ وَإِهَانَةٍ لأَنَّ الأَجِنَّةَ دَنَتْ إِلَى الْمَوْلِدِ وَلاَ قُوَّةَ عَلَى الْوِلاَدَةِ.
\par 4 لَعَلَّ الرَّبَّ إِلَهَكَ يَسْمَعُ كَلاَمَ رَبْشَاقَى الَّذِي أَرْسَلَهُ مَلِكُ أَشُّورَ سَيِّدُهُ لِيُعَيِّرَ الإِلَهَ الْحَيَّ فَيُوَبِّخَ عَلَى الْكَلاَمِ الَّذِي سَمِعَهُ الرَّبُّ إِلَهُكَ. فَارْفَعْ صَلاَةً لأَجْلِ الْبَقِيَّةِ الْمَوْجُودَةِ».
\par 5 فَجَاءَ عَبِيدُ الْمَلِكِ حَزَقِيَّا إِلَى إِشَعْيَاءَ.
\par 6 فَقَالَ لَهُمْ إِشَعْيَاءُ: «هَكَذَا تَقُولُونَ لِسَيِّدِكُمْ: هَكَذَا يَقُولُ الرَّبُّ: لاَ تَخَفْ بِسَبَبِ الْكَلاَمِ الَّذِي سَمِعْتَهُ الَّذِي جَدَّفَ عَلَيَّ بِهِ غِلْمَانُ مَلِكِ أَشُّورَ.
\par 7 هَئَنَذَا أَجْعَلُ فِيهِ رُوحاً فَيَسْمَعُ خَبَراً وَيَرْجِعُ إِلَى أَرْضِهِ وَأُسْقِطُهُ بِالسَّيْفِ فِي أَرْضِهِ».
\par 8 فَرَجَعَ رَبْشَاقَى وَوَجَدَ مَلِكَ أَشُّورَ يُحَارِبُ لِبْنَةَ لأَنَّهُ سَمِعَ أَنَّهُ ارْتَحَلَ عَنْ لَخِيشَ.
\par 9 وَسَمِعَ عَنْ تِرْهَاقَةَ مَلِكِ كُوشَ قَوْلاً: «قَدْ خَرَجَ لِيُحَارِبَكَ». فَلَمَّا سَمِعَ أَرْسَلَ رُسُلاً إِلَى حَزَقِيَّا قَائِلاً:
\par 10 «هَكَذَا تُكَلِّمُونَ حَزَقِيَّا مَلِكَ يَهُوذَا: لاَ يَخْدَعْكَ إِلَهُكَ الَّذِي أَنْتَ مُتَوَكِّلٌ عَلَيْهِ قَائِلاً: لاَ تُدْفَعُ أُورُشَلِيمُ إِلَى يَدِ مَلِكِ أَشُّورَ.
\par 11 إِنَّكَ قَدْ سَمِعْتَ مَا فَعَلَ مُلُوكُ أَشُّورَ بِجَمِيعِ الأَرَاضِي لِتَحْرِيمِهَا. وَهَلْ تَنْجُو أَنْتَ؟
\par 12 هَلْ أَنْقَذَ آلِهَةُ الأُمَمِ هَؤُلاَءِ الَّذِينَ أَهْلَكَهُمْ آبَائِي جُوزَانَ وَحَارَانَ وَرَصَفَ وَبَنِي عَدَنَ الَّذِينَ فِي تَلَسَّارَ؟
\par 13 أَيْنَ مَلِكُ حَمَاةَ وَمَلِكُ أَرْفَادَ وَمَلِكُ مَدِينَةِ سَفَرْوَايِمَ وَهَيْنَعَ وَعِوَّا؟».
\par 14 فَأَخَذَ حَزَقِيَّا الرَّسَائِلَ مِنْ يَدِ الرُّسُلِ وَقَرَأَهَا ثُمَّ صَعِدَ إِلَى بَيْتِ الرَّبِّ وَنَشَرَهَا أَمَامَ الرَّبِّ
\par 15 وَصَلَّى:
\par 16 «يَا رَبَّ الْجُنُودِ إِلَهَ إِسْرَائِيلَ الْجَالِسَ فَوْقَ الْكَرُوبِيمِ أَنْتَ هُوَ الإِلَهُ وَحْدَكَ لِكُلِّ مَمَالِكِ الأَرْضِ. أَنْتَ صَنَعْتَ السَّمَاوَاتِ وَالأَرْضَ.
\par 17 أَمِلْ يَا رَبُّ أُذُنَكَ وَاسْمَعِ. افْتَحْ يَا رَبُّ عَيْنَيْكَ وَانْظُرْ وَاسْمَعْ كُلَّ كَلاَمِ سَنْحَارِيبَ الَّذِي أَرْسَلَهُ لِيُعَيِّرَ اللَّهَ الْحَيَّ.
\par 18 حَقّاً يَا رَبُّ إِنَّ مُلُوكَ أَشُّورَ قَدْ خَرَّبُوا كُلَّ الأُمَمِ وَأَرْضَهُمْ
\par 19 وَدَفَعُوا آلِهَتَهُمْ إِلَى النَّارِ لأَنَّهُمْ لَيْسُوا آلِهَةً بَلْ صَنْعَةُ أَيْدِي النَّاسِ خَشَبٌ وَحَجَرٌ. فَأَبَادُوهُمْ.
\par 20 وَالآنَ أَيُّهَا الرَّبُّ إِلَهُنَا خَلِّصْنَا مِنْ يَدِهِ فَتَعْلَمَ مَمَالِكُ الأَرْضِ كُلِّهَا أَنَّكَ أَنْتَ الرَّبُّ وَحْدَكَ».
\par 21 فَأَرْسَلَ إِشَعْيَاءُ بْنُ آمُوصَ إِلَى حَزَقِيَّا قَائِلاً: «هَكَذَا يَقُولُ الرَّبُّ إِلَهُ إِسْرَائِيلَ الَّذِي صَلَّيْتَ إِلَيْهِ مِنْ جِهَةِ سَنْحَارِيبَ مَلِكِ أَشُّورَ:
\par 22 هَذَا هُوَ الْكَلاَمُ الَّذِي تَكَلَّمَ بِهِ الرَّبُّ عَلَيْهِ. احْتَقَرَتْكَ. اسْتَهْزَأَتْ بِكَ الْعَذْرَاءُ ابْنَةُ صِهْيَوْنَ. نَحْوَكَ أَنْغَضَتِ ابْنَةُ أُورُشَلِيمَ رَأْسَهَا.
\par 23 مَنْ عَيَّرْتَ وَجَدَّفْتَ وَعَلَى مَنْ عَلَّيْتَ صَوْتاً وَقَدْ رَفَعْتَ إِلَى الْعَلاَءِ عَيْنَيْكَ؟ عَلَى قُدُّوسِ إِسْرَائِيلَ!
\par 24 عَنْ يَدِ عَبِيدِكَ عَيَّرْتَ السَّيِّدَ وَقُلْتَ: بِكَثْرَةِ مَرْكَبَاتِي قَدْ صَعِدْتُ إِلَى عُلُوِّ الْجِبَالِ عِقَابِ لُبْنَانَ فَأَقْطَعُ أَرْزَهُ الطَّوِيلَ وَأَفْضَلَ سَرْوِهِ وَأَدْخُلُ أَقْصَى عُلُوِّهِ وَعْرَ كَرْمَلِهِ.
\par 25 أَنَا قَدْ حَفَرْتُ وَشَرِبْتُ مِيَاهاً وَأُنَشِّفُ بِبَطْنِ قَدَمِي جَمِيعَ خُلْجَانِ مِصْرَ.
\par 26 أَلَمْ تَسْمَعْ؟ مُنْذُ الْبَعِيدِ صَنَعْتُهُ. مُنْذُ الأَيَّامِ الْقَدِيمَةِ صَوَّرْتُهُ. الآنَ أَتَيْتُ بِهِ. فَتَكُونُ لِتَخْرِيبِ مُدُنٍ مُحَصَّنَةٍ حَتَّى تَصِيرَ رَوَابِيَ خَرِبَةً.
\par 27 فَسُكَّانُهَا قِصَارُ الأَيْدِي قَدِ ارْتَاعُوا وَخَجِلُوا. صَارُوا كَعُشْبِ الْحَقْلِ وَكَالنَّبَاتِ الأَخْضَرِ كَحَشِيشِ السُّطُوحِ وَكَالْمَلْفُوحِ قَبْلَ نُمُوِّهِ.
\par 28 وَلَكِنَّنِي عَالِمٌ بِجُلُوسِكَ وَخُرُوجِكَ وَدُخُولِكَ وَهَيَجَانِكَ عَلَيَّ.
\par 29 لأَنَّ هَيَجَانَكَ عَلَيَّ وَعَجْرَفَتَكَ قَدْ صَعِدَا إِلَى أُذُنَيَّ أَضَعُ خِزَامَتِي فِي أَنْفِكَ وَشَكِيمَتِي فِي شَفَتَيْكَ وَأَرُدُّكَ فِي الطَّرِيقِ الَّذِي جِئْتَ فِيهِ.
\par 30 «وَهَذِهِ لَكَ الْعَلاَمَةُ: تَأْكُلُونَ هَذِهِ السَّنَةَ زِرِّيعاً وَفِي السَّنَةِ الثَّانِيَةِ خِلْفَةً وَأَمَّا السَّنَةُ الثَّالِثَةُ فَفِيهَا تَزْرَعُونَ وَتَحْصِدُونَ وَتَغْرِسُونَ كُرُوماً وَتَأْكُلُونَ أَثْمَارَهَا.
\par 31 وَيَعُودُ النَّاجُونَ مِنْ بَيْتِ يَهُوذَا الْبَاقُونَ يَتَأَصَّلُونَ إِلَى أَسْفَلَ وَيَصْنَعُونَ ثَمَراً إِلَى مَا فَوْقُ.
\par 32 لأَنَّهُ مِنْ أُورُشَلِيمَ تَخْرُجُ بَقِيَّةٌ وَنَاجُونَ مِنْ جَبَلِ صِهْيَوْنَ. غَيْرَةُ رَبِّ الْجُنُودِ تَصْنَعُ هَذَا».
\par 33 لِذَلِكَ هَكَذَا يَقُولُ الرَّبُّ عَنْ مَلِكِ أَشُّورَ: «لاَ يَدْخُلُ هَذِهِ الْمَدِينَةَ وَلاَ يَرْمِي هُنَاكَ سَهْماً وَلاَ يَتَقَدَّمُ عَلَيْهَا بِتُرْسٍ وَلاَ يُقِيمُ عَلَيْهَا مِتْرَسَةً.
\par 34 فِي الطَّرِيقِ الَّذِي جَاءَ فِيهِ يَرْجِعُ وَإِلَى هَذِهِ الْمَدِينَةِ لاَ يَدْخُلُ يَقُولُ الرَّبُّ.
\par 35 وَأُحَامِي عَنْ هَذِهِ الْمَدِينَةِ لأُخَلِّصَهَا مِنْ أَجْلِ نَفْسِي وَمِنْ أَجْلِ دَاوُدَ عَبْدِي».
\par 36 فَخَرَجَ مَلاَكُ الرَّبِّ وَضَرَبَ مِنْ جَيْشِ أَشُّورَ مِئَةً وَخَمْسَةً وَثَمَانِينَ أَلْفاً. فَلَمَّا بَكَّرُوا صَبَاحاً إِذَا هُمْ جَمِيعاً جُثَثٌ مَيِّتَةٌ.
\par 37 فَانْصَرَفَ سَنْحَارِيبُ مَلِكُ أَشُّورَ وَذَهَبَ رَاجِعاً وَأَقَامَ فِي نِينَوَى.
\par 38 وَفِيمَا هُوَ سَاجِدٌ فِي بَيْتِ نِسْرُوخَ إِلَهِهِ ضَرَبَهُ أَدْرَمَّلَكُ وَشَرْآصَرُ ابْنَاهُ بِالسَّيْفِ وَنَجَوَا إِلَى أَرْضِ أَرَارَاطَ. وَمَلَكَ أَسَرْحَدُّونَ ابْنُهُ عِوَضاً عَنْهُ.

\chapter{38}

\par 1 فِي تِلْكَ الأَيَّامِ مَرِضَ حَزَقِيَّا لِلْمَوْتِ فَجَاءَ إِلَيْهِ إِشَعْيَاءُ بْنُ آمُوصَ النَّبِيُّ وَقَالَ لَهُ: «هَكَذَا يَقُولُ الرَّبُّ: أَوْصِ بَيْتَكَ لأَنَّكَ تَمُوتُ وَلاَ تَعِيشُ».
\par 2 فَوَجَّهَ حَزَقِيَّا وَجْهَهُ إِلَى الْحَائِطِ وَصَلَّى:
\par 3 «آهِ يَا رَبُّ اذْكُرْ كَيْفَ سِرْتُ أَمَامَكَ بِالأَمَانَةِ وَبِقَلْبٍ سَلِيمٍ وَفَعَلْتُ الْحَسَنَ فِي عَيْنَيْكَ». وَبَكَى حَزَقِيَّا بُكَاءً عَظِيماً.
\par 4 فَصَارَ قَوْلُ الرَّبِّ إِلَى إِشَعْيَاءَ:
\par 5 «اذْهَبْ وَقُلْ لِحَزَقِيَّا: هَكَذَا يَقُولُ الرَّبُّ إِلَهُ دَاوُدَ أَبِيكَ: قَدْ سَمِعْتُ صَلاَتَكَ. قَدْ رَأَيْتُ دُمُوعَكَ. هَئَنَذَا أُضِيفُ إِلَى أَيَّامِكَ خَمْسَ عَشَرَةَ سَنَةً.
\par 6 وَمِنْ يَدِ مَلِكِ أَشُّورَ أُنْقِذُكَ وَهَذِهِ الْمَدِينَةَ. وَأُحَامِي عَنْ هَذِهِ الْمَدِينَةِ.
\par 7 وَهَذِهِ لَكَ الْعَلاَمَةُ مِنْ قِبَلِ الرَّبِّ عَلَى أَنَّ الرَّبَّ يَفْعَلُ هَذَا الأَمْرَ الَّذِي تَكَلَّمَ بِهِ:
\par 8 هَئَنَذَا أُرَجِّعُ ظِلَّ الدَّرَجَاتِ الَّذِي نَزَلَ فِي دَرَجَاتِ آحَازَ بِالشَّمْسِ عَشَرَ دَرَجَاتٍ إِلَى الْوَرَاءِ». فَرَجَعَتِ الشَّمْسُ عَشَرَ دَرَجَاتٍ فِي الدَّرَجَاتِ الَّتِي نَزَلَتْهَا.
\par 9 كِتَابَةٌ لِحَزَقِيَّا مَلِكِ يَهُوذَا إِذْ مَرِضَ وَشُفِيَ مِنْ مَرَضِهِ.
\par 10 أَنَا قُلْتُ: «فِي عِزِّ أَيَّامِي أَذْهَبُ إِلَى أَبْوَابِ الْهَاوِيَةِ. قَدْ أُعْدِمْتُ بَقِيَّةَ سِنِيَّ.
\par 11 قُلْتُ لاَ أَرَى الرَّبَّ فِي أَرْضِ الأَحْيَاءِ. لاَ أَنْظُرُ إِنْسَاناً بَعْدُ مَعَ سُكَّانِ الْفَانِيَةِ.
\par 12 مَسْكَنِي قَدِ انْقَلَعَ وَانْتَقَلَ عَنِّي كَخَيْمَةِ الرَّاعِي. لَفَفْتُ كَالْحَائِكِ حَيَاتِي. مِنَ النَّوْلِ يَقْطَعُنِي. النَّهَارَ وَاللَّيْلَ تُفْنِينِي.
\par 13 صَرَخْتُ إِلَى الصَّبَاحِ. كَالأَسَدِ هَكَذَا يُهَشِّمُ جَمِيعَ عِظَامِي. النَّهَارَ وَالَّلَيْلَ تُفْنِينِي.
\par 14 كَسُنُونةٍ مُزَقْزِقةٍ هَكَذَا أَصِيحُ. أَهْدِرُ كَحَمَامَةٍ. قَدْ ضَعُفَتْ عَيْنَايَ نَاظِرَةً إِلَى العَلاَءِ. يَا رَبُّ قَدْ تَضَايَقْتُ. كُنْ لِي ضَامِناً.
\par 15 بِمَاذَا أَتَكَلَّمُ فَإِنَّهُ قَالَ لِي وَهُوَ قَدْ فَعَلَ. أَتَمَشَّى مُتَمَهِّلاً كُلَّ سِنِيَّ مِن أَجْلِ مَرَارَةِ نَفْسِي.
\par 16 أَيُّهَا السَّيِّدُ بِهَذِهِ يَحْيُونَ وَبِهَا كُلُّ حَيَاةِ رُوحِي فَتَشْفِينِي وَتُحْيِينِي.
\par 17 هُوَذَا لِلسَّلاَمَةِ قَدْ تَحَوَّلَتْ لِيَ الْمَرَارَةُ وَأَنْتَ تَعَلَّقْتَ بِنَفْسِي مِنْ وَهْدَةِ الْهَلاَكِ فَإِنَّكَ طَرَحْتَ وَرَاءَ ظَهْرِكَ كُلَّ خَطَايَايَ.
\par 18 لأَنَّ الْهَاوِيَةَ لاَ تَحْمَدُكَ. الْمَوْتُ لاَ يُسَبِّحُكَ. لاَ يَرْجُو الْهَابِطُونَ إِلَى الْجُبِّ أَمَانَتَكَ.
\par 19 الْحَيُّ الْحَيُّ هُوَ يَحْمَدُكَ كَمَا أَنَا الْيَوْمَ. الأَبُ يُعَرِّفُ الْبَنِينَ حَقَّكَ.
\par 20 الرَّبُّ لِخَلاَصِي. فَنَعْزِفُ بِأَوْتَارِنَا كُلَّ أَيَّامِ حَيَاتِنَا فِي بَيْتِ الرَّبِّ».
\par 21 وَكَانَ إِشَعْيَاءُ قَدْ قَالَ: «لِيَأْخُذُوا قُرْصَ تِينٍ وَيُضَمِّدُوهُ عَلَى الدَّبْلِ فَيَبْرَأَ».
\par 22 وَحَزَقِيَّا سَأَلَ: «مَا هِيَ الْعَلاَمَةُ أَنِّي أَصْعَدُ إِلَى بَيْتِ الرَّبِّ؟».

\chapter{39}

\par 1 فِي ذَلِكَ الزَّمَانِ أَرْسَلَ مَرُودَخُ بَلاَدَانَ بْنُ بَلاَدَانَ مَلِكُ بَابِلَ رَسَائِلَ وَهَدِيَّةً إِلَى حَزَقِيَّا لأَنَّهُ سَمِعَ أَنَّهُ مَرِضَ ثُمَّ صَحَّ.
\par 2 فَفَرِحَ بِهِمْ حَزَقِيَّا وَأَرَاهُمْ بَيْتَ ذَخَائِرِهِ: الْفِضَّةَ وَالذَّهَبَ وَالأَطْيَابَ وَالزَّيْتَ الطَّيِّبَ وَكُلَّ بَيْتِ أَسْلِحَتِهِ وَكُلَّ مَا وُجِدَ فِي خَزَائِنِهِ. لَمْ يَكُنْ شَيْءٌ لَمْ يُرِهِمْ إِيَّاهُ حَزَقِيَّا فِي بَيْتِهِ وَفِي كُلِّ مُلْكِهِ.
\par 3 فَجَاءَ إِشَعْيَاءُ النَّبِيُّ إِلَى الْمَلِكِ حَزَقِيَّا وَسَأَلَهُ: «مَاذَا قَالَ هَؤُلاَءِ الرِّجَالُ وَمِنْ أَيْنَ جَاءُوا إِلَيْكَ؟» فَقَالَ حَزَقِيَّا: «جَاءُوا إِلَيَّ مِنْ أَرْضٍ بَعِيدَةٍ مِنْ بَابِلَ».
\par 4 فَسَأَلَ: «مَاذَا رَأُوا فِي بَيْتِكَ؟» فَقَالَ حَزَقِيَّا: «رَأُوا كُلَّ مَا فِي بَيْتِي. لَيْسَ فِي خَزَائِنِي شَيْءٌ لَمْ أُرِهِمْ إِيَّاهُ».
\par 5 فَقَالَ إِشَعْيَاءُ لِحَزَقِيَّا: «اسْمَعْ قَوْلَ رَبِّ الْجُنُودِ:
\par 6 هُوَذَا تَأْتِي أَيَّامٌ يُحْمَلُ فِيهَا كُلُّ مَا فِي بَيْتِكَ وَمَا خَزَنَهُ آبَاؤُكَ إِلَى هَذَا الْيَوْمِ إِلَى بَابِلَ. لاَ يُتْرَكُ شَيْءٌ يَقُولُ الرَّبُّ.
\par 7 وَمِنْ بَنِيكَ الَّذِينَ يَخْرُجُونَ مِنْكَ الَّذِينَ تَلِدُهُمْ يَأْخُذُونَ فَيَكُونُونَ خِصْيَاناً فِي قَصْرِ مَلِكِ بَابِلَ».
\par 8 فَقَالَ حَزَقِيَّا لإِشَعْيَاءَ: «جَيِّدٌ هُوَ قَوْلُ الرَّبِّ الَّذِي تَكَلَّمْتَ بِهِ». وَقَالَ: «فَإِنَّهُ يَكُونُ سَلاَمٌ وَأَمَانٌ فِي أَيَّامِي».

\chapter{40}

\par 1 عَزُّوا عَزُّوا شَعْبِي يَقُولُ إِلَهُكُمْ.
\par 2 طَيِّبُوا قَلْبَ أُورُشَلِيمَ وَنَادُوهَا بِأَنَّ جِهَادَهَا قَدْ كَمِلَ أَنَّ إِثْمَهَا قَدْ عُفِيَ عَنْهُ أَنَّهَا قَدْ قَبِلَتْ مِنْ يَدِ الرَّبِّ ضِعْفَيْنِ عَنْ كُلِّ خَطَايَاهَا.
\par 3 صَوْتُ صَارِخٍ فِي الْبَرِّيَّةِ: أَعِدُّوا طَرِيقَ الرَّبِّ. قَوِّمُوا فِي الْقَفْرِ سَبِيلاً لإِلَهِنَا.
\par 4 كُلُّ وَطَاءٍ يَرْتَفِعُ وَكُلُّ جَبَلٍ وَأَكَمَةٍ يَنْخَفِضُ وَيَصِيرُ الْمُعَوَّجُ مُسْتَقِيماً وَالْعَرَاقِيبُ سَهْلاً.
\par 5 فَيُعْلَنُ مَجْدُ الرَّبِّ وَيَرَاهُ كُلُّ بَشَرٍ جَمِيعاً لأَنَّ فَمَ الرَّبِّ تَكَلَّمَ.
\par 6 صَوْتُ قَائِلٍ: «نَادِ». فَقَالَ: «بِمَاذَا أُنَادِي؟» «كُلُّ جَسَدٍ عُشْبٌ وَكُلُّ جَمَالِهِ كَزَهْرِ الْحَقْلِ.
\par 7 يَبِسَ الْعُشْبُ ذَبُلَ الزَّهْرُ لأَنَّ نَفْخَةَ الرَّبِّ هَبَّتْ عَلَيْهِ. حَقّاً الشَّعْبُ عُشْبٌ!
\par 8 يَبِسَ الْعُشْبُ ذَبُلَ الزَّهْرُ. وَأَمَّا كَلِمَةُ إِلَهِنَا فَتَثْبُتُ إِلَى الأَبَدِ».
\par 9 عَلَى جَبَلٍ عَالٍ اصْعَدِي يَا مُبَشِّرَةَ صِهْيَوْنَ. ارْفَعِي صَوْتَكِ بِقُوَّةٍ يَا مُبَشِّرَةَ أُورُشَلِيمَ. ارْفَعِي لاَ تَخَافِي. قُولِي لِمُدُنِ يَهُوذَا: «هُوَذَا إِلَهُكِ.
\par 10 هُوَذَا السَّيِّدُ الرَّبُّ بِقُوَّةٍ يَأْتِي وَذِرَاعُهُ تَحْكُمُ لَهُ. هُوَذَا أُجْرَتُهُ مَعَهُ وَعُمْلَتُهُ قُدَّامَهُ.
\par 11 كَرَاعٍ يَرْعَى قَطِيعَهُ. بِذِرَاعِهِ يَجْمَعُ الْحُمْلاَنَ وَفِي حِضْنِهِ يَحْمِلُهَا وَيَقُودُ الْمُرْضِعَاتِ».
\par 12 مَنْ كَالَ بِكَفِّهِ الْمِيَاهَ وَقَاسَ السَّمَاوَاتِ بِالشِّبْرِ وَكَالَ بِالْكَيْلِ تُرَابَ الأَرْضِ وَوَزَنَ الْجِبَالَ بِالْقَبَّانِ وَالآكَامَ بِالْمِيزَانِ؟
\par 13 مَنْ قَاسَ رُوحَ الرَّبِّ وَمَنْ مُشِيرُهُ يُعَلِّمُهُ؟
\par 14 مَنِ اسْتَشَارَهُ فَأَفْهَمَهُ وَعَلَّمَهُ فِي طَرِيقِ الْحَقِّ وَعَلَّمَهُ مَعْرِفَةً وَعَرَّفَهُ سَبِيلَ الْفَهْمِ.؟
\par 15 هُوَذَا الأُمَمُ كَنُقْطَةٍ مِنْ دَلْوٍ وَكَغُبَارِ الْمِيزَانِ تُحْسَبُ. هُوَذَا الْجَزَائِرُ يَرْفَعُهَا كَدُقَّةٍ!
\par 16 وَلُبْنَانُ لَيْسَ كَافِياً لِلإِيقَادِ وَحَيَوَانُهُ لَيْسَ كَافِياً لِمُحْرَقَةٍ.
\par 17 كُلُّ الأُمَمِ كَلاَ شَيْءٍ قُدَّامَهُ. مِنَ الْعَدَمِ وَالْبَاطِلِ تُحْسَبُ عَُِنْدَهُ.
\par 18 فَبِمَنْ تُشَبِّهُونَ اللَّهَ وَأَيَّ شَبَهٍ تُعَادِلُونَ بِهِ؟
\par 19 اَلصَّنَمُ يَسْبِكُهُ الصَّانِعُ وَالصَّائِغُ يُغَشِّيهِ بِذَهَبٍ وَيَصُوغُ سَلاَسِلَ فِضَّةٍ.
\par 20 الْفَقِيرُ عَنِ التَّقْدِمَةِ يَنْتَخِبُ خَشَباً لاَ يُسَوِّسُ يَطْلُبُ لَهُ صَانِعاً مَاهِراً لِيَنْصُبَ صَنَماً لاَ يَتَزَعْزَعُ!
\par 21 أَلاَ تَعْلَمُونَ؟ أَلاَ تَسْمَعُونَ؟ أَلَمْ تُخْبَرُوا مِنَ الْبَدَاءَةِ؟ أَلَمْ تَفْهَمُوا مِنْ أَسَاسَاتِ الأَرْضِ؟
\par 22 الْجَالِسُ عَلَى كُرَةِ الأَرْضِ وَسُكَّانُهَا كَالْجُنْدُبِ. الَّذِي يَنْشُرُ السَّمَاوَاتِ كَسَرَادِقَ وَيَبْسُطُهَا كَخَيْمَةٍ لِلسَّكَنِ.
\par 23 الَّذِي يَجْعَلُ الْعُظَمَاءَ لاَ شَيْئاً وَيُصَيِّرُ قُضَاةَ الأَرْضِ كَالْبَاطِلِ.
\par 24 لَمْ يُغْرَسُوا بَلْ لَمْ يُزْرَعُوا وَلَمْ يَتَأَصَّلْ فِي الأَرْضِ سَاقُهُمْ. فَنَفَخَ أَيْضاً عَلَيْهِمْ فَجَفُّوا وَالْعَاصِفُ كَالْعَصْفِ يَحْمِلُهُمْ.
\par 25 فَبِمَنْ تُشَبِّهُونَنِي فَأُسَاوِيهِ؟ يَقُولُ الْقُدُّوسُ.
\par 26 ارْفَعُوا إِلَى الْعَلاَءِ عُيُونَكُمْ وَانْظُرُوا مَنْ خَلَقَ هَذِهِ؟ مَنِ الَّذِي يُخْرِجُ بِعَدَدٍ جُنْدَهَا يَدْعُو كُلَّهَا بِأَسْمَاءٍ؟ لِكَثْرَةِ الْقُوَّةِ وَكَوْنِهِ شَدِيدَ الْقُدْرَةِ لاَ يُفْقَدُ أَحَدٌ.
\par 27 لِمَاذَا تَقُولُ يَا يَعْقُوبُ وَتَتَكَلَّمُ يَا إِسْرَائِيلُ: «قَدِ اخْتَفَتْ طَرِيقِي عَنِ الرَّبِّ وَفَاتَ حَقِّي إِلَهِي»؟
\par 28 أَمَا عَرَفْتَ أَمْ لَمْ تَسْمَعْ؟ إِلَهُ الدَّهْرِ الرَّبُّ خَالِقُ أَطْرَافِ الأَرْضِ لاَ يَكِلُّ وَلاَ يَعْيَا. لَيْسَ عَنْ فَهْمِهِ فَحْصٌ.
\par 29 يُعْطِي الْمُعْيِيَ قُدْرَةً وَلِعَدِيمِ الْقُوَّةِ يُكَثِّرُ شِدَّةً.
\par 30 اَلْغِلْمَانُ يُعْيُونَ وَيَتْعَبُونَ وَالْفِتْيَانُ يَتَعَثَّرُونَ تَعَثُّراً.
\par 31 وَأَمَّا مُنْتَظِرُو الرَّبِّ فَيُجَدِّدُونَ قُوَّةً. يَرْفَعُونَ أَجْنِحَةً كَالنُّسُورِ. يَرْكُضُونَ وَلاَ يَتْعَبُونَ يَمْشُونَ وَلاَ يُعْيُونَ.

\chapter{41}

\par 1 اُنْصُتِي إِلَيَّ أَيَّتُهَا الْجَزَائِرُ وَلْتُجَدِّدِ الْقَبَائِلُ قُوَّةً. لِيَقْتَرِبُوا ثُمَّ يَتَكَلَّمُوا. لِنَتَقَدَّمْ مَعاً إِلَى الْمُحَاكَمَةِ.
\par 2 مَنْ أَنْهَضَ مِنَ الْمَشْرِقِ الَّذِي يُلاَقِيهِ النَّصْرُ عِنْدَ رِجْلَيْهِ؟ دَفَعَ أَمَامَهُ أُمَماً وَعَلَى مُلُوكٍ سَلَّطَهُ. جَعَلَهُمْ كَالتُّرَابِ بِسَيْفِهِ وَكَالْقَشِّ الْمُنْذَرِي بِقَوْسِهِ.
\par 3 طَرَدَهُمْ. مَرَّ سَالِماً فِي طَرِيقٍ لَمْ يَسْلُكْهُ بِرِجْلَيْهِ.
\par 4 مَنْ فَعَلَ وَصَنَعَ دَاعِياً الأَجْيَالَ مِنَ الْبَدْءِ؟ أَنَا الرَّبُّ الأَوَّلُ وَمَعَ الآخِرِينَ أَنَا هُوَ.
\par 5 نَظَرَتِ الْجَزَائِرُ فَخَافَتْ. أَطْرَافُ الأَرْضِ ارْتَعَدَتِ. اقْتَرَبَتْ وَجَاءَتْ.
\par 6 كُلُّ وَاحِدٍ يُسَاعِدُ صَاحِبَهُ وَيَقُولُ لأَخِيهِ: «تَشَدَّدْ».
\par 7 فَشَدَّدَ النَّجَّارُ الصَّائِغَ. الصَّاقِلُ بِالْمِطْرَقَةِ الضَّارِبَ عَلَى السَّنْدَانِ قَائِلاً عَنِ الإِلْحَامِ: «هُوَ جَيِّدٌ». فَمَكَّنَهُ بِمَسَامِيرَ حَتَّى لاَ يَتَقَلْقَلَ!
\par 8 وَأَمَّا أَنْتَ يَا إِسْرَائِيلُ عَبْدِي يَا يَعْقُوبُ الَّذِي اخْتَرْتُهُ نَسْلَ إِبْرَاهِيمَ خَلِيلِي
\par 9 الَّذِي أَمْسَكْتُهُ مِنْ أَطْرَافِ الأَرْضِ وَمِنْ أَقْطَارِهَا دَعَوْتُهُ وَقُلْتُ لَكَ: «أَنْتَ عَبْدِي. اخْتَرْتُكَ وَلَمْ أَرْفُضْكَ
\par 10 لاَ تَخَفْ لأَنِّي مَعَكَ. لاَ تَتَلَفَّتْ لأَنِّي إِلَهُكَ. قَدْ أَيَّدْتُكَ وَأَعَنْتُكَ وَعَضَدْتُكَ بِيَمِينِ بِرِّي.
\par 11 إِنَّهُ سَيَخْزَى وَيَخْجَلُ جَمِيعُ الْمُغْتَاظِينَ عَلَيْكَ. يَكُونُ كَلاَ شَيْءٍ مُخَاصِمُوكَ وَيَبِيدُونَ.
\par 12 تُفَتِّشُ عَلَى مُنَازِعِيكَ وَلاَ تَجِدُهُمْ. يَكُونُ مُحَارِبُوكَ كَلاَ شَيْءٍ وَكَالْعَدَمِ.
\par 13 لأَنِّي أَنَا الرَّبُّ إِلَهُكَ الْمُمْسِكُ بِيَمِينِكَ الْقَائِلُ لَكَ: لاَ تَخَفْ. أَنَا أُعِينُكَ».
\par 14 لاَ تَخَفْ يَا دُودَةَ يَعْقُوبَ يَا شِرْذِمَةَ إِسْرَائِيلَ. أَنَا أُعِينُكَ يَقُولُ الرَّبُّ وَفَادِيكَ قُدُّوسُ إِسْرَائِيلَ.
\par 15 هَئَنَذَا قَدْ جَعَلْتُكَ نَوْرَجاً مُحَدَّداً جَدِيداً ذَا أَسْنَانٍ. تَدْرُسُ الْجِبَالَ وَتَسْحَقُهَا وَتَجْعَلُ الآكَامَ كَالْعُصَافَةِ.
\par 16 تُذَرِّيهَا فَالرِّيحُ تَحْمِلُهَا وَالْعَاصِفُ تُبَدِّدُهَا وَأَنْتَ تَبْتَهِجُ بِالرَّبِّ. بِقُدُّوسِ إِسْرَائِيلَ تَفْتَخِرُ.
\par 17 اَلْبَائِسُونَ وَالْمَسَاكِينُ طَالِبُونَ مَاءً وَلاَ يُوجَدُ. لِسَانُهُمْ مِنَ الْعَطَشِ قَدْ يَبِسَ. أَنَا الرَّبُّ أَسْتَجِيبُ لَهُمْ. أَنَا إِلَهَ إِسْرَائِيلَ لاَ أَتْرُكُهُمْ.
\par 18 أَفْتَحُ عَلَى الْهِضَابِ أَنْهَاراً وَفِي وَسَطِ الْبِقَاعِ يَنَابِيعَ. أَجْعَلُ الْقَفْرَ أَجَمَةَ مَاءٍ وَالأَرْضَ الْيَابِسَةَ مَفَاجِرَ مِيَاهٍ.
\par 19 أَجْعَلُ فِي الْبَرِّيَّةِ الأَرْزَ وَالسَّنْطَ وَالآسَ وَشَجَرَةَ الزَّيْتِ. أَضَعُ فِي الْبَادِيَةِ السَّرْوَ وَالسِّنْدِيَانَ وَالشَّرْبِينَ مَعاً.
\par 20 لِيَنْظُرُوا وَيَعْرِفُوا وَيَتَنَبَّهُوا وَيَتَأَمَّلُوا مَعاً أَنَّ يَدَ الرَّبِّ فَعَلَتْ هَذَا وَقُدُّوسَ إِسْرَائِيلَ أَبْدَعَهُ.
\par 21 قَدِّمُوا دَعْوَاكُمْ يَقُولُ الرَّبُّ. أَحْضِرُوا حُجَجَكُمْ يَقُولُ مَلِكُ يَعْقُوبَ.
\par 22 لِيُقَدِّمُوهَا وَيُخْبِرُونَا بِمَا سَيَعْرِضُ. مَا هِيَ الأَوَّلِيَّاتُ؟ أَخْبِرُوا فَنَجْعَلَ عَلَيْهَا قُلُوبَنَا وَنَعْرِفَ آخِرَتَهَا أَوْ أَعْلِمُونَا الْمُسْتَقْبِلاَتِ.
\par 23 أَخْبِرُوا بِالآتِيَاتِ فِيمَا بَعْدُ فَنَعْرِفَ أَنَّكُمْ آلِهَةٌ وَافْعَلُوا خَيْراً أَوْ شَرّاً فَنَلْتَفِتَ وَنَنْظُرَ مَعاً.
\par 24 هَا أَنْتُمْ مِنْ لاَ شَيْءٍ وَعَمَلُكُمْ مِنَ الْعَدَمِ. رِجْسٌ هُوَ الَّذِي يَخْتَارُكُمْ.
\par 25 قَدْ أَنْهَضْتُهُ مِنَ الشِّمَالِ فَأَتَى. مِنْ مَشْرِقِ الشَّمْسِ يَدْعُو بِاسْمِي. يَأْتِي عَلَى الْوُلاَةِ كَمَا عَلَى الْمِلاَطِ وَكَخَزَّافٍ يَدُوسُ الطِّينَ.
\par 26 مَنْ أَخْبَرَ مِنَ الْبَدْءِ حَتَّى نَعْرِفَ وَمِنْ قَبْلٍ حَتَّى نَقُولَ: «هُوَ صَادِقٌ»؟ لاَ مُخْبِرٌ وَلاَ مُسْمِعٌ وَلاَ سَامِعٌ أَقْوَالَكُمْ.
\par 27 أَنَا أَوَّلاً قُلْتُ لِصِهْيَوْنَ: «هَا! هَا هُمْ» وَلأُورُشَلِيمَ جَعَلْتُ مُبَشِّراً.
\par 28 وَنَظَرْتُ فَلَيْسَ إِنْسَانٌ وَمِنْ هَؤُلاَءِ فَلَيْسَ مُشِيرٌ حَتَّى أَسْأَلَهُمْ فَيَرُدُّونَ كَلِمَةً.
\par 29 هَا كُلُّهُمْ بَاطِلٌ وَأَعْمَالُهُمْ عَدَمٌ وَمَسْبُوكَاتُهُمْ رِيحٌ وَخَلاَءٌ.

\chapter{42}

\par 1 هُوَذَا عَبْدِي الَّذِي أَعْضُدُهُ مُخْتَارِي الَّذِي سُرَّتْ بِهِ نَفْسِي. وَضَعْتُ رُوحِي عَلَيْهِ فَيُخْرِجُ الْحَقَّ لِلأُمَمِ.
\par 2 لاَ يَصِيحُ وَلاَ يَرْفَعُ وَلاَ يُسْمِعُ فِي الشَّارِعِ صَوْتَهُ.
\par 3 قَصَبَةً مَرْضُوضَةً لاَ يَقْصِفُ وَفَتِيلَةً خَامِدَةً لاَ يُطْفِئُ. إِلَى الأَمَانِ يُخْرِجُ الْحَقَّ.
\par 4 لاَ يَكِلُّ وَلاَ يَنْكَسِرُ حَتَّى يَضَعَ الْحَقَّ فِي الأَرْضِ وَتَنْتَظِرُ الْجَزَائِرُ شَرِيعَتَهُ.
\par 5 هَكَذَا يَقُولُ اللَّهُ الرَّبُّ خَالِقُ السَّمَاوَاتِ وَنَاشِرُهَا بَاسِطُ الأَرْضِ وَنَتَائِجِهَا مُعْطِي الشَّعْبِ عَلَيْهَا نَسَمَةً وَالسَّاكِنِينَ فِيهَا رُوحاً.
\par 6 أَنَا الرَّبَّ قَدْ دَعَوْتُكَ بِالْبِرِّ فَأُمْسِكُ بِيَدِكَ وَأَحْفَظُكَ وَأَجْعَلُكَ عَهْداً لِلشَّعْبِ وَنُوراً لِلأُمَمِ
\par 7 لِتَفْتَحَ عُيُونَ الْعُمْيِ لِتُخْرِجَ مِنَ الْحَبْسِ الْمَأْسُورِينَ مِنْ بَيْتِ السِّجْنِ الْجَالِسِينَ فِي الظُّلْمَةِ.
\par 8 أَنَا الرَّبُّ هَذَا اسْمِي وَمَجْدِي لاَ أُعْطِيهِ لِآخَرَ وَلاَ تَسْبِيحِي لِلْمَنْحُوتَاتِ.
\par 9 هُوَذَا الأَوَّلِيَّاتُ قَدْ أَتَتْ وَالْحَدِيثَاتُ أَنَا مُخْبِرٌ بِهَا. قَبْلَ أَنْ تَنْبُتَ أُعْلِمُكُمْ بِهَا.
\par 10 غَنُّوا لِلرَّبِّ أُغْنِيَةً جَدِيدَةً تَسْبِيحَهُ مِنْ أَقْصَى الأَرْضِ. أَيُّهَا الْمُنْحَدِرُونَ فِي الْبَحْرِ وَمِلْؤُهُ وَالْجَزَائِرُ وَسُكَّانُهَا
\par 11 لِتَرْفَعِ الْبَرِّيَّةُ وَمُدُنُهَا صَوْتَهَا الدِّيَارُ الَّتِي سَكَنَهَا قِيدَارُ. لِتَتَرَنَّمْ سُكَّانُ سَالِعَ. مِنْ رُؤُوسِ الْجِبَالِ لِيَهْتِفُوا.
\par 12 لِيُعْطُوا الرَّبَّ مَجْداً وَيُخْبِرُوا بِتَسْبِيحِهِ فِي الْجَزَائِرِ.
\par 13 الرَّبُّ كَالْجَبَّارِ يَخْرُجُ. كَرَجُلِ حُرُوبٍ يُنْهِضُ غَيْرَتَهُ. يَهْتِفُ وَيَصْرُخُ وَيَقْوَى عَلَى أَعْدَائِهِ.
\par 14 قَدْ صَمَتُّ مُنْذُ الدَّهْرِ. سَكَتُّ. تَجَلَّدْتُ. كَالْوَالِدَةِ أَصِيحُ. أَنْفُخُ وَأَنْخِرُ مَعاً.
\par 15 أَخْرِبُ الْجِبَالَ وَالآكَامَ وَأُجَفِّفُ كُلَّ عُشْبِهَا وَأَجْعَلُ الأَنْهَارَ يَبَساً وَأُنَشِّفُ الآجَامَ
\par 16 وَأُسَيِّرُ الْعُمْيَ فِي طَرِيقٍ لَمْ يَعْرِفُوهَا. فِي مَسَالِكَ لَمْ يَدْرُوهَا أُمَشِّيهِمْ. أَجْعَلُ الظُّلْمَةَ أَمَامَهُمْ نُوراً وَالْمُعْوَجَّاتِ مُسْتَقِيمَةً. هَذِهِ الأُمُورُ أَفْعَلُهَا وَلاَ أَتْرُكُهُمْ.
\par 17 قَدِ ارْتَدُّوا إِلَى الْوَرَاءِ. يَخْزَى خِزْياً الْمُتَّكِلُونَ عَلَى الْمَنْحُوتَاتِ الْقَائِلُونَ لِلْمَسْبُوكَاتِ: «أَنْتُنَّ آلِهَتُنَا!»
\par 18 أَيُّهَا الصُّمُّ اسْمَعُوا. أَيُّهَا الْعُمْيُ انْظُرُوا لِتُبْصِرُوا.
\par 19 مَنْ هُوَ أَعْمَى إِلاَّ عَبْدِي وَأَصَمُّ كَرَسُولِي الَّذِي أُرْسِلُهُ؟ مَنْ هُوَ أَعْمَى كَالْكَامِلِ وَأَعْمَى كَعَبْدِ الرَّبِّ؟
\par 20 نَاظِرٌ كَثِيراً وَلاَ تُلاَحِظُ. مَفْتُوحُ الأُذُنَيْنِ وَلاَ يَسْمَعُ.
\par 21 الرَّبُّ قَدْ سُرَّ مِنْ أَجْلِ بِرِّهِ. يُعَظِّمُ الشَّرِيعَةَ وَيُكْرِمُهَا.
\par 22 وَلَكِنَّهُ شَعْبٌ مَنْهُوبٌ وَمَسْلُوبٌ. قَدِ اصْطِيدَ فِي الْحُفَرِ كُلُّهُ وَفِي بُيُوتِ الْحُبُوسِ اخْتَبَأُوا. صَارُوا نَهْباً وَلاَ مُنْقِذَ وَسَلْباً وَلَيْسَ مَنْ يَقُولُ: «رُدَّ!»
\par 23 مَنْ مِنْكُمْ يَسْمَعُ هَذَا؟ يَصْغَى وَيَسْمَعُ لِمَا بَعْدُ؟
\par 24 مَنْ دَفَعَ يَعْقُوبَ إِلَى السَّلْبِ وَإِسْرَائِيلَ إِلَى النَّاهِبِينَ؟ أَلَيْسَ الرَّبُّ الَّذِي أَخْطَأْنَا إِلَيْهِ وَلَمْ يَشَاءُوا أَنْ يَسْلُكُوا فِي طُرُقِهِ وَلَمْ يَسْمَعُوا لِشَرِيعَتِهِ.
\par 25 فَسَكَبَ عَلَيْهِ حُمُوَّ غَضَبِهِ وَشِدَّةَ الْحَرْبِ فَأَوْقَدَتْهُ مِنْ كُلِّ نَاحِيَةٍ وَلَمْ يَعْرِفْ وَأَحْرَقَتْهُ وَلَمْ يَضَعْ فِي قَلْبِهِ.

\chapter{43}

\par 1 وَالآنَ هَكَذَا يَقُولُ الرَّبُّ خَالِقُكَ يَا يَعْقُوبُ وَجَابِلُكَ يَا إِسْرَائِيلُ: «لاَ تَخَفْ لأَنِّي فَدَيْتُكَ. دَعَوْتُكَ بِاسْمِكَ. أَنْتَ لِي.
\par 2 إِذَا اجْتَزْتَ فِي الْمِيَاهِ فَأَنَا مَعَكَ وَفِي الأَنْهَارِ فَلاَ تَغْمُرُكَ. إِذَا مَشَيْتَ فِي النَّارِ فَلاَ تُلْذَعُ وَاللَّهِيبُ لاَ يُحْرِقُكَ.
\par 3 لأَنِّي أَنَا الرَّبُّ إِلَهُكَ قُدُّوسُ إِسْرَائِيلَ مُخَلِّصُكَ. جَعَلْتُ مِصْرَ فِدْيَتَكَ كُوشَ وَسَبَا عِوَضَكَ.
\par 4 إِذْ صِرْتَ عَزِيزاً فِي عَيْنَيَّ مُكَرَّماً وَأَنَا قَدْ أَحْبَبْتُكَ. أُعْطِي أُنَاساً عِوَضَكَ وَشُعُوباً عِوَضَ نَفْسِكَ.
\par 5 لاَ تَخَفْ فَإِنِّي مَعَكَ. مِنَ الْمَشْرِقِ آتِي بِنَسْلِكَ وَمِنَ الْمَغْرِبِ أَجْمَعُكَ.
\par 6 أَقُولُ لِلشِّمَالِ: أَعْطِ وَلِلْجَنُوبِ: لاَ تَمْنَعْ. ايتِ بِبَنِيَّ مِنْ بَعِيدٍ وَبِبَنَاتِي مِنْ أَقْصَى الأَرْضِ.
\par 7 بِكُلِّ مَنْ دُعِيَ بِاسْمِي وَلِمَجْدِي خَلَقْتُهُ وَجَبَلْتُهُ وَصَنَعْتُهُ.
\par 8 أَخْرِجِ الشَّعْبَ الأَعْمَى وَلَهُ عُيُونٌ وَالأَصَمَّ وَلَهُ آذَانٌ.
\par 9 «اِجْتَمِعُوا يَا كُلَّ الأُمَمِ مَعاً وَلْتَلْتَئِمِ الْقَبَائِلُ. مَنْ مِنْهُمْ يُخْبِرُ بِهَذَا وَيُعْلِمُنَا بِالأَوَّلِيَّاتِ؟ لِيُقَدِّمُوا شُهُودَهُمْ وَيَتَبَرَّرُوا. أَوْ لِيَسْمَعُوا فَيَقُولُوا: صِدْقٌ.
\par 10 أَنْتُمْ شُهُودِي يَقُولُ الرَّبُّ وَعَبْدِي الَّذِي اخْتَرْتُهُ لِكَيْ تَعْرِفُوا وَتُؤْمِنُوا بِي وَتَفْهَمُوا أَنِّي أَنَا هُوَ. قَبْلِي لَمْ يُصَوَّرْ إِلَهٌ وَبَعْدِي لاَ يَكُونُ.
\par 11 أَنَا أَنَا الرَّبُّ وَلَيْسَ غَيْرِي مُخَلِّصٌ.
\par 12 أَنَا أَخْبَرْتُ وَخَلَّصْتُ وَأَعْلَمْتُ وَلَيْسَ بَيْنَكُمْ غَرِيبٌ. وَأَنْتُمْ شُهُودِي يَقُولُ الرَّبُّ وَأَنَا اللَّهُ.
\par 13 أَيْضاً مِنَ الْيَوْمِ أَنَا هُوَ وَلاَ مُنْقِذَ مِنْ يَدِي. أَفْعَلُ وَمَنْ يَرُدُّ؟».
\par 14 هَكَذَا يَقُولُ الرَّبُّ فَادِيكُمْ قُدُّوسُ إِسْرَائِيلَ: «لأَجْلِكُمْ أَرْسَلْتُ إِلَى بَابِلَ وَأَلْقَيْتُ الْمَغَالِيقَ كُلَّهَا وَالْكِلْدَانِيِّينَ فِي سُفُنِ تَرَنُّمِهِمْ.
\par 15 أَنَا الرَّبُّ قُدُّوسُكُمْ خَالِقُ إِسْرَائِيلَ مَلِكُكُمْ.
\par 16 هَكَذَا يَقُولُ الرَّبُّ الْجَاعِلُ فِي الْبَحْرِ طَرِيقاً وَفِي الْمِيَاهِ الْقَوِيَّةِ مَسْلَكاً.
\par 17 الْمُخْرِجُ الْمَرْكَبَةَ وَالْفَرَسَ الْجَيْشَ وَالْعِزَّ. يَضْطَجِعُونَ مَعاً لاَ يَقُومُونَ. قَدْ خَمِدُوا. كَفَتِيلَةٍ انْطَفَأُوا.
\par 18 «لاَ تَذْكُرُوا الأَوَّلِيَّاتِ وَالْقَدِيمَاتُ لاَ تَتَأَمَّلُوا بِهَا.
\par 19 هَئَنَذَا صَانِعٌ أَمْراً جَدِيداً. الآنَ يَنْبُتُ. أَلاَ تَعْرِفُونَهُ؟ أَجْعَلُ فِي الْبَرِّيَّةِ طَرِيقاً فِي الْقَفْرِ أَنْهَاراً.
\par 20 يُمَجِّدُنِي حَيَوَانُ الصَّحْرَاءِ الذِّئَابُ وَبَنَاتُ النَّعَامِ لأَنِّي جَعَلْتُ فِي الْبَرِّيَّةِ مَاءً أَنْهَاراً فِي الْقَفْرِ لأَسْقِيَ شَعْبِي مُخْتَارِي.
\par 21 هَذَا الشَّعْبُ جَبَلْتُهُ لِنَفْسِي. يُحَدِّثُ بِتَسْبِيحِي.
\par 22 «وَأَنْتَ لَمْ تَدْعُنِي يَا يَعْقُوبُ حَتَّى تَتْعَبَ مِنْ أَجْلِي يَا إِسْرَائِيلُ.
\par 23 لَمْ تُحْضِرْ لِي شَاةَ مُحْرَقَتِكَ وَبِذَبَائِحِكَ لَمْ تُكْرِمْنِي. لَمْ أَسْتَخْدِمْكَ بِتَقْدِمَةٍ وَلاَ أَتْعَبْتُكَ بِلُبَانٍ.
\par 24 لَمْ تَشْتَرِ لِي بِفِضَّةٍ قَصَباً وَبِشَحْمِ ذَبَائِحِكَ لَمْ تُرْوِنِي. لَكِنِ اسْتَخْدَمْتَنِي بِخَطَايَاكَ وَأَتْعَبْتَنِي بِآثَامِكَ.
\par 25 أَنَا أَنَا هُوَ الْمَاحِي ذُنُوبَكَ لأَجْلِ نَفْسِي وَخَطَايَاكَ لاَ أَذْكُرُهَا.
\par 26 «ذَكِّرْنِي فَنَتَحَاكَمَ مَعاً. حَدِّثْ لِكَيْ تَتَبَرَّرَ.
\par 27 أَبُوكَ الأَوَّلُ أَخْطَأَ وَوُسَطَاؤُكَ عَصُوا عَلَيَّ.
\par 28 فَدَنَّسْتُ رُؤَسَاءَ الْقُدْسِ وَدَفَعْتُ يَعْقُوبَ إِلَى اللَّعْنِ وَإِسْرَائِيلَ إِلَى الشَّتَائِمِ.

\chapter{44}

\par 1 «وَالآنَ اسْمَعْ يَا يَعْقُوبُ عَبْدِي وَإِسْرَائِيلُ الَّذِي اخْتَرْتُهُ.
\par 2 هَكَذَا يَقُولُ الرَّبُّ صَانِعُكَ وَجَابِلُكَ مِنَ الرَّحِمِ مُعِينُكَ: لاَ تَخَفْ يَا عَبْدِي يَعْقُوبُ وَيَا يَشُورُونُ الَّذِي اخْتَرْتُهُ.
\par 3 لأَنِّي أَسْكُبُ مَاءً عَلَى الْعَطْشَانِ وَسُيُولاً عَلَى الْيَابِسَةِ. أَسْكُبُ رُوحِي عَلَى نَسْلِكَ وَبَرَكَتِي عَلَى ذُرِّيَّتِكَ.
\par 4 فَيَنْبُتُونَ بَيْنَ الْعُشْبِ مِثْلَ الصَّفْصَافِ عَلَى مَجَارِي الْمِيَاهِ.
\par 5 هَذَا يَقُولُ: أَنَا لِلرَّبِّ وَهَذَا يُكَنِّي بِاسْمِ يَعْقُوبَ وَهَذَا يَكْتُبُ بِيَدِهِ: لِلرَّبِّ وَبِاسْمِ إِسْرَائِيلَ يُلَقِّبُ».
\par 6 هَكَذَا يَقُولُ الرَّبُّ مَلِكُ إِسْرَائِيلَ وَفَادِيهِ رَبُّ الْجُنُودِ: «أَنَا الأَوَّلُ وَأَنَا الآخِرُ وَلاَ إِلَهَ غَيْرِي.
\par 7 وَمَنْ مِثْلِي يُنَادِي فَلْيُخْبِرْ بِهِ وَيَعْرِضْهُ لِي مُنْذُ وَضَعْتُ الشَّعْبَ الْقَدِيمَ. وَالْمُسْتَقْبَلاَتُ وَمَا سَيَأْتِي لِيُخْبِرُوهُمْ بِهَا.
\par 8 لاَ تَرْتَعِبُوا وَلاَ تَرْتَاعُوا. أَمَا أَعْلَمْتُكَ مُنْذُ الْقَدِيمِ وَأَخْبَرْتُكَ؟ فَأَنْتُمْ شُهُودِي. هَلْ يُوجَدُ إِلَهٌ غَيْرِي؟ وَلاَ صَخْرَةَ لاَ أَعْلَمُ بِهَا.
\par 9 الَّذِينَ يُصَوِّرُونَ صَنَماً كُلُّهُمْ بَاطِلٌ وَمُشْتَهَيَاتُهُمْ لاَ تَنْفَعُ وَشُهُودُهُمْ هِيَ. لاَ تُبْصِرُ وَلاَ تَعْرِفُ حَتَّى تَخْزَى.
\par 10 مَنْ صَوَّرَ إِلَهاً وَسَبَكَ صَنَماً لِغَيْرِ نَفْعٍ؟
\par 11 هَا كُلُّ أَصْحَابِهِ يَخْزُونَ وَالصُّنَّاعُ هُمْ مِنَ النَّاسِ. يَجْتَمِعُونَ كُلُّهُمْ يَقِفُونَ يَرْتَعِبُونَ وَيَخْزُونَ مَعاً.
\par 12 «طَبَعَ الْحَدِيدَ قَدُوماً وَعَمِلَ فِي الْفَحْمِ وَبِالْمَطَارِقِ يُصَوِّرُهُ فَيَصْنَعُهُ بِذِرَاعِ قُوَّتِهِ. يَجُوعُ أَيْضاً فَلَيْسَ لَهُ قُوَّةٌ. لَمْ يَشْرَبْ مَاءً وَقَدْ تَعِبَ.
\par 13 نَجَّرَ خَشَباً. مَدَّ الْخَيْطَ. بِالْمِخْرَزِ يُعَلِّمُهُ يَصْنَعُهُ بِالأَزَامِيلِ وَبِالدَّوَّارَةِ يَرْسِمُهُ. فَيَصْنَعُهُ كَشَبَهِ رَجُلٍ كَجَمَالِ إِنْسَانٍ لِيَسْكُنَ فِي الْبَيْتِ!
\par 14 قَطَعَ لِنَفْسِهِ أَرْزاً وَأَخَذَ سِنْدِيَاناً وَبَلُّوطاً وَاخْتَارَ لِنَفْسِهِ مِنْ أَشْجَارِ الْوَعْرِ. غَرَسَ سَنُوبَراً وَالْمَطَرُ يُنْمِيهِ.
\par 15 فَيَصِيرُ لِلنَّاسِ لِلإِيقَادِ. وَيَأْخُذُ مِنْهُ وَيَتَدَفَّأُ. يُشْعِلُ أَيْضاً وَيَخْبِزُ خُبْزاً ثُمَّ يَصْنَعُ إِلَهاً فَيَسْجُدُ! قَدْ صَنَعَهُ صَنَماً وَخَرَّ لَهُ.
\par 16 نِصْفُهُ أَحْرَقَهُ بِالنَّارِ. عَلَى نِصْفِهِ يَأْكُلُ لَحْماً. يَشْوِي مَشْوِيّاً وَيَشْبَعُ! يَتَدَفَّأُ أَيْضاً وَيَقُولُ: بَخْ! قَدْ تَدَفَّأْتُ. رَأَيْتُ نَاراً.
\par 17 وَبَقِيَّتُهُ قَدْ صَنَعَهَا إِلَهاً صَنَماً لِنَفْسِهِ! يَخُرُّ لَهُ وَيَسْجُدُ وَيُصَلِّي إِلَيْهِ وَيَقُولُ: نَجِّنِي لأَنَّكَ أَنْتَ إِلَهِي.
\par 18 «لاَ يَعْرِفُونَ وَلاَ يَفْهَمُونَ لأَنَّهُ قَدْ طُمِسَتْ عُيُونُهُمْ عَنِ الإِبْصَارِ وَقُلُوبُهُمْ عَنِ التَّعَقُّلِ.
\par 19 وَلاَ يُرَدِّدُ فِي قَلْبِهِ وَلَيْسَ لَهُ مَعْرِفَةٌ وَلاَ فَهْمٌ حَتَّى يَقُولَ: نِصْفَهُ قَدْ أَحْرَقْتُ بِالنَّارِ وَخَبَزْتُ أَيْضاً عَلَى جَمْرِهِ خُبْزاً شَوَيْتُ لَحْماً وَأَكَلْتُ. أَفَأَصْنَعُ بَقِيَّتَهُ رِجْساً وَلِسَاقِ شَجَرَةٍ أَخُرُّ؟
\par 20 يَرْعَى رَمَاداً. قَلْبٌ مَخْدُوعٌ قَدْ أَضَلَّهُ فَلاَ يُنَجِّي نَفْسَهُ وَلاَ يَقُولُ: أَلَيْسَ كَذِبٌ فِي يَمِينِي؟
\par 21 «اُذْكُرْ هَذِهِ يَا يَعْقُوبُ يَا إِسْرَائِيلُ فَإِنَّكَ أَنْتَ عَبْدِي. قَدْ جَبَلْتُكَ. عَبْدٌ لِي أَنْتَ. يَا إِسْرَائِيلُ لاَ تُنْسَى مِنِّي.
\par 22 قَدْ مَحَوْتُ كَغَيْمٍ ذُنُوبَكَ وَكَسَحَابَةٍ خَطَايَاكَ. ارْجِعْ إِلَيَّ لأَنِّي فَدَيْتُكَ.
\par 23 تَرَنَّمِي أَيَّتُهَا السَّمَاوَاتُ لأَنَّ الرَّبَّ قَدْ فَعَلَ. اهْتِفِي يَا أَسَافِلَ الأَرْضِ. أَشِيدِي أَيَّتُهَا الْجِبَالُ تَرَنُّماً الْوَعْرُ وَكُلُّ شَجَرَةٍ فِيهِ لأَنَّ الرَّبَّ قَدْ فَدَى يَعْقُوبَ وَفِي إِسْرَائِيلَ تَمَجَّدَ».
\par 24 هَكَذَا يَقُولُ الرَّبُّ فَادِيكَ وَجَابِلُكَ مِنَ الْبَطْنِ: «أَنَا الرَّبُّ صَانِعٌ كُلَّ شَيْءٍ نَاشِرٌ السَّمَاوَاتِ وَحْدِي. بَاسِطٌ الأَرْضَ. مَنْ مَعِي؟
\par 25 مُبَطِّلٌ آيَاتِ الْمُخَادِعِينَ وَمُحَمِّقٌ الْعَرَّافِينَ. مُرَجِّعٌ الْحُكَمَاءَ إِلَى الْوَرَاءِ وَمُجَهِّلٌ مَعْرِفَتَهُمْ.
\par 26 مُقِيمٌ كَلِمَةَ عَبْدِهِ وَمُتَمِّمٌ رَأْيَ رُسُلِهِ. الْقَائِلُ عَنْ أُورُشَلِيمَ: سَتُعْمَرُ وَلِمُدُنِ يَهُوذَا: سَتُبْنَيْنَ وَخِرَبَهَا أُقِيمُ.
\par 27 الْقَائِلُ لِلُّجَّةِ: انْشَفِي وَأَنْهَارَكِ أُجَفِّفُ.
\par 28 الْقَائِلُ عَنْ كُورَشَ: رَاعِيَّ فَكُلَّ مَسَرَّتِي يُتَمِّمُ. وَيَقُولُ عَنْ أُورُشَلِيمَ: سَتُبْنَى وَلِلْهَيْكَلِ: سَتُؤَسَّسُ».

\chapter{45}

\par 1 هَكَذَا يَقُولُ الرَّبُّ لِمَسِيحِهِ لِكُورَشَ الَّذِي أَمْسَكْتُ بِيَمِينِهِ لأَدُوسَ أَمَامَهُ أُمَماً وَأَحْقَاءَ مُلُوكٍ أَحُلُّ. لأَفْتَحَ أَمَامَهُ الْمِصْرَاعَيْنِ وَالأَبْوَابُ لاَ تُغْلَقُ:
\par 2 «أَنَا أَسِيرُ قُدَّامَكَ وَالْهِضَابَ أُمَهِّدُ. أُكَسِّرُ مِصْرَاعَيِ النُّحَاسِ وَمَغَالِيقَ الْحَدِيدِ أَقْصِفُ.
\par 3 وَأُعْطِيكَ ذَخَائِرَ الظُّلْمَةِ وَكُنُوزَ الْمَخَابِئِ لِتَعْرِفَ أَنِّي أَنَا الرَّبُّ الَّذِي يَدْعُوكَ بِاسْمِكَ إِلَهُ إِسْرَائِيلَ.
\par 4 لأَجْلِ عَبْدِي يَعْقُوبَ وَإِسْرَائِيلَ مُخْتَارِي دَعَوْتُكَ بِاسْمِكَ. لَقَّبْتُكَ وَأَنْتَ لَسْتَ تَعْرِفُنِي.
\par 5 أَنَا الرَّبُّ وَلَيْسَ آخَرُ. لاَ إِلَهَ سِوَايَ. نَطَّقْتُكَ وَأَنْتَ لَمْ تَعْرِفْنِي.
\par 6 لِيَعْلَمُوا مِنْ مَشْرِقِ الشَّمْسِ وَمِنْ مَغْرِبِهَا أَنْ لَيْسَ غَيْرِي. أَنَا الرَّبُّ وَلَيْسَ آخَرُ.
\par 7 مُصَوِّرُ النُّورِ وَخَالِقُ الظُّلْمَةِ صَانِعُ السَّلاَمِ وَخَالِقُ الشَّرِّ. أَنَا الرَّبُّ صَانِعُ كُلِّ هَذِهِ.
\par 8 اُقْطُرِي أَيَّتُهَا السَّمَاوَاتُ مِنْ فَوْقُ وَلْيُنْزِلِ الْجَوُّ بِرّاً. لِتَنْفَتِحِ الأَرْضُ فَيُثْمِرَ الْخَلاَصُ وَلْتُنْبِتْ بِرّاً مَعاً. أَنَا الرَّبَّ قَدْ خَلَقْتُهُ.
\par 9 «وَيْلٌ لِمَنْ يُخَاصِمُ جَابِلَهُ. خَزَفٌ بَيْنَ أَخْزَافِ الأَرْضِ. هَلْ يَقُولُ الطِّينُ لِجَابِلِهِ: مَاذَا تَصْنَعُ؟ أَوْ يَقُولُ: عَمَلُكَ لَيْسَ لَهُ يَدَانِ؟
\par 10 وَيْلٌ لِلَّذِي يَقُولُ لأَبِيهِ: مَاذَا تَلِدُ؟ وَلِلْمَرْأَةِ: مَاذَا تَلِدِينَ؟».
\par 11 هَكَذَا يَقُولُ الرَّبُّ قُدُّوسُ إِسْرَائِيلَ وَجَابِلُهُ: «اسْأَلُونِي عَنِ الآتِيَاتِ. مِنْ جِهَةِ بَنِيَّ وَمِنْ جِهَةِ عَمَلِ يَدِي أَوْصُونِي.
\par 12 أَنَا صَنَعْتُ الأَرْضَ وَخَلَقْتُ الإِنْسَانَ عَلَيْهَا. يَدَايَ أَنَا نَشَرَتَا السَّمَاوَاتِ وَكُلَّ جُنْدِهَا أَنَا أَمَرْتُ.
\par 13 أَنَا قَدْ أَنْهَضْتُهُ بِالنَّصْرِ وَكُلَّ طُرُقِهِ أُسَهِّلُ. هُوَ يَبْنِي مَدِينَتِي وَيُطْلِقُ سَبْيِي لاَ بِثَمَنٍ وَلاَ بِهَدِيَّةٍ قَالَ رَبُّ الْجُنُودِ».
\par 14 هَكَذَا قَالَ الرَّبُّ: «تَعَبُ مِصْرَ وَتِجَارَةُ كُوشٍ وَالسَّبَئِيُّونَ ذَوُو الْقَامَةِ إِلَيْكِ يَعْبُرُونَ وَلَكِ يَكُونُونَ. خَلْفَكِ يَمْشُونَ. بِالْقُيُودِ يَمُرُّونَ وَلَكِ يَسْجُدُونَ. إِلَيْكِ يَتَضَرَّعُونَ قَائِلِينَ: فِيكِ وَحْدَكِ اللَّهُ وَلَيْسَ آخَرُ. لَيْسَ إِلَهٌ».
\par 15 حَقّاً أَنْتَ إِلَهٌ مُحْتَجِبٌ يَا إِلَهَ إِسْرَائِيلَ الْمُخَلِّصَ.
\par 16 قَدْ خَزُوا وَخَجِلُوا كُلُّهُمْ. مَضُوا بِالْخَجَلِ جَمِيعاً الصَّانِعُونَ التَّمَاثِيلَ.
\par 17 أَمَّا إِسْرَائِيلُ فَيَخْلُصُ بِالرَّبِّ خَلاَصاً أَبَدِيّاً. لاَ تَخْزُونَ وَلاَ تَخْجَلُونَ إِلَى دُهُورِ الأَبَدِ.
\par 18 لأَنَّهُ هَكَذَا قَالَ الرَّبُّ: «خَالِقُ السَّمَاوَاتِ هُوَ اللَّهُ. مُصَوِّرُ الأَرْضِ وَصَانِعُهَا. هُوَ قَرَّرَهَا. لَمْ يَخْلُقْهَا بَاطِلاً. لِلسَّكَنِ صَوَّرَهَا. أَنَا الرَّبُّ وَلَيْسَ آخَرُ.
\par 19 لَمْ أَتَكَلَّمْ بِالْخِفَاءِ فِي مَكَانٍ مِنَ الأَرْضِ مُظْلِمٍ. لَمْ أَقُلْ لِنَسْلِ يَعْقُوبَ: بَاطِلاً اطْلُبُونِي. أَنَا الرَّبُّ مُتَكَلِّمٌ بِالصِّدْقِ مُخْبِرٌ بِالاِسْتِقَامَةِ.
\par 20 «اِجْتَمِعُوا وَهَلُمُّوا تَقَدَّمُوا مَعاً أَيُّهَا النَّاجُونَ مِنَ الأُمَمِ. لاَ يَعْلَمُ الْحَامِلُونَ خَشَبَ صَنَمِهِمْ وَالْمُصَلُّونَ إِلَى إِلَهٍ لاَ يُخَلِّصُ.
\par 21 أَخْبِرُوا. قَدِّمُوا. وَلْيَتَشَاوَرُوا مَعاً. مَنْ أَعْلَمَ بِهَذِهِ مُنْذُ الْقَدِيمِ أَخْبَرَ بِهَا مُنْذُ زَمَانٍ؟ أَلَيْسَ أَنَا الرَّبُّ وَلاَ إِلَهَ آخَرَ غَيْرِي؟ إِلَهٌ بَارٌّ وَمُخَلِّصٌ. لَيْسَ سِوَايَ.
\par 22 اِلْتَفِتُوا إِلَيَّ وَاخْلُصُوا يَا جَمِيعَ أَقَاصِي الأَرْضِ لأَنِّي أَنَا اللَّهُ وَلَيْسَ آخَرَ.
\par 23 بِذَاتِي أَقْسَمْتُ. خَرَجَ مِنْ فَمِي الصِّدْقُ كَلِمَةٌ لاَ تَرْجِعُ: إِنَّهُ لِي تَجْثُو كُلُّ رُكْبَةٍ. يَحْلِفُ كُلُّ لِسَانٍ.
\par 24 قَالَ لِي: إِنَّمَا بِالرَّبِّ الْبِرُّ وَالْقُوَّةُ. إِلَيْهِ يَأْتِي. وَيَخْزَى جَمِيعُ الْمُغْتَاظِينَ عَلَيْهِ.
\par 25 بِالرَّبِّ يَتَبَرَّرُ وَيَفْتَخِرُ كُلُّ نَسْلِ إِسْرَائِيلَ».

\chapter{46}

\par 1 قَدْ جَثَا بِيلُ انْحَنَى نَبُو. صَارَتْ تَمَاثِيلُهُمَا عَلَى الْحَيَوَانَاتِ وَالْبَهَائِمِ. مَحْمُولاَتُكُمْ مُحَمَّلَةٌ حِمْلاً لِلْمُعْيِي.
\par 2 قَدِ انْحَنَتْ. جَثَتْ مَعاً. لَمْ تَقْدِرْ أَنْ تُنَجِّيَ الْحِمْلَ وَهِيَ نَفْسُهَا قَدْ مَضَتْ فِي السَّبْيِ.
\par 3 «اِسْمَعُوا لِي يَا بَيْتَ يَعْقُوبَ وَكُلَّ بَقِيَّةِ بَيْتِ إِسْرَائِيلَ الْمُحَمَّلِينَ عَلَيَّ مِنَ الْبَطْنِ الْمَحْمُولِينَ مِنَ الرَّحِمِ.
\par 4 وَإِلَى الشَّيْخُوخَةِ أَنَا هُوَ وَإِلَى الشَّيْبَةِ أَنَا أَحْمِلُ. قَدْ فَعَلْتُ وَأَنَا أَرْفَعُ وَأَنَا أَحْمِلُ وَأُنَجِّي.
\par 5 بِمَنْ تُشَبِّهُونَنِي وَتُسَوُّونَنِي وَتُمَثِّلُونَنِي لِنَتَشَابَهَ؟.
\par 6 «اَلَّذِينَ يُفْرِغُونَ الذَّهَبَ مِنَ الْكِيسِ وَالْفِضَّةَ بِالْمِيزَانِ يَزِنُونَ. يَسْتَأْجِرُونَ صَائِغاً لِيَصْنَعَهَا إِلَهاً. يَخُرُّونَ وَيَسْجُدُونَ!
\par 7 يَرْفَعُونَهُ عَلَى الْكَتِفِ. يَحْمِلُونَهُ وَيَضَعُونَهُ فِي مَكَانِهِ لِيَقِفَ. مِنْ مَوْضِعِهِ لاَ يَبْرَحُ. يَزْعَقُ أَحَدٌ إِلَيْهِ فَلاَ يُجِيبُ. مِنْ شِدَّتِهِ لاَ يُخَلِّصُهُ.
\par 8 «اُذْكُرُوا هَذَا وَكُونُوا رِجَالاً. رَدِّدُوهُ فِي قُلُوبِكُمْ أَيُّهَا الْعُصَاةُ.
\par 9 اُذْكُرُوا الأَوَّلِيَّاتِ مُنْذُ الْقَدِيمِ لأَنِّي أَنَا اللَّهُ وَلَيْسَ آخَرُ. الإِلَهُ وَلَيْسَ مِثْلِي.
\par 10 مُخْبِرٌ مُنْذُ الْبَدْءِ بِالأَخِيرِ وَمُنْذُ الْقَدِيمِ بِمَا لَمْ يُفْعَلْ قَائِلاً: رَأْيِي يَقُومُ وَأَفْعَلُ كُلَّ مَسَرَّتِي.
\par 11 دَاعٍ مِنَ الْمَشْرِقِ الْكَاسِرَ. مِنْ أَرْضٍ بَعِيدَةٍ رَجُلَ مَشُورَتِي. قَدْ تَكَلَّمْتُ فَأُجْرِيهِ. قَضَيْتُ فَأَفْعَلُهُ.
\par 12 «اِسْمَعُوا لِي يَا أَشِدَّاءَ الْقُلُوبِ الْبَعِيدِينَ عَنِ الْبِرِّ.
\par 13 قَدْ قَرَّبْتُ بِرِّي. لاَ يَبْعُدُ وَخَلاَصِي لاَ يَتَأَخَّرُ. وَأَجْعَلُ فِي صِهْيَوْنَ خَلاَصاً. لإِسْرَائِيلَ جَلاَلِي».

\chapter{47}

\par 1 «اِنْزِلِي وَاجْلِسِي عَلَى التُّرَابِ أَيَّتُهَا الْعَذْرَاءُ ابْنَةَ بَابِلَ. اجْلِسِي عَلَى الأَرْضِ بِلاَ كُرْسِيٍّ يَا ابْنَةَ الْكِلْدَانِيِّينَ لأَنَّكِ لاَ تَعُودِينَ تُدْعَيْنَ نَاعِمَةً وَمُتَرَفِّهَةً.
\par 2 خُذِي الرَّحَى وَاطْحَنِي دَقِيقاً. اكْشِفِي نُقَابَكِ. شَمِّرِي الذَّيْلَ. اكْشِفِي السَّاقَ. اعْبُرِي الأَنْهَارَ.
\par 3 تَنْكَشِفُ عَوْرَتُكِ وَتُرَى مَعَارِيكِ. آخُذُ نَقْمَةً وَلاَ أُصَالِحُ أَحَداً».
\par 4 فَادِينَا رَبُّ الْجُنُودِ اسْمُهُ. قُدُّوسُ إِسْرَائِيلَ.
\par 5 «اجْلِسِي صَامِتَةً وَادْخُلِي فِي الظَّلاَمِ يَا ابْنَةَ الْكِلْدَانِيِّينَ لأَنَّكِ لاَ تَعُودِينَ تُدْعَيْنَ سَيِّدَةَ الْمَمَالِكِ.
\par 6 «غَضِبْتُ عَلَى شَعْبِي. دَنَّسْتُ مِيرَاثِي وَدَفَعْتُهُمْ إِلَى يَدِكِ. لَمْ تَصْنَعِي لَهُمْ رَحْمَةً. عَلَى الشَّيْخِ ثَقَّلْتِ نِيرَكِ جِدّاً.
\par 7 وَقُلْتِ: إِلَى الأَبَدِ أَكُونُ سَيِّدَةً حَتَّى لَمْ تَضَعِي هَذِهِ فِي قَلْبِكِ. لَمْ تَذْكُرِي آخِرَتَهَا.
\par 8 فَالآنَ اسْمَعِي هَذَا أَيَّتُهَا الْمُتَنَعِّمَةُ الْجَالِسَةُ بِالطُّمَأْنِينَةِ الْقَائِلَةُ فِي قَلْبِهَا: أَنَا وَلَيْسَ غَيْرِي. لاَ أَقْعُدُ أَرْمَلَةً وَلاَ أَعْرِفُ الثَّكَلَ.
\par 9 فَيَأْتِي عَلَيْكِ هَذَانِ الاِثْنَانِ بَغْتَةً فِي يَوْمٍ وَاحِدٍ: الثَّكَلُ وَالتَّرَمُّلُ. بِالتَّمَامِ قَدْ أَتَيَا عَلَيْكِ مَعَ كَثْرَةِ سُحُورِكِ مَعَ وُفُورِ رُقَاكِ جِدّاً.
\par 10 وَأَنْتِ اطْمَأْنَنْتِ فِي شَرِّكِ. قُلْتِ: لَيْسَ مَنْ يَرَانِي. حِكْمَتُكِ وَمَعْرِفَتُكِ هُمَا أَفْتَنَاكِ فَقُلْتِ فِي قَلْبِكِ: أَنَا وَلَيْسَ غَيْرِي.
\par 11 فَيَأْتِي عَلَيْكِ شَرٌّ لاَ تَعْرِفِينَ فَجْرَهُ وَتَقَعُ عَلَيْكِ مُصِيبَةٌ لاَ تَقْدِرِينَ أَنْ تَصُدِّيهَا وَتَأْتِي عَلَيْكِ بَغْتَةً تَهْلُكَةٌ لاَ تَعْرِفِينَ بِهَا.
\par 12 «قِفِي فِي رُقَاكِ وَفِي كَثْرَةِ سُحُورِكِ الَّتِي فِيهَا تَعِبْتِ مُنْذُ صِبَاكِ. رُبَّمَا يُمْكِنُكِ أَنْ تَنْفَعِي. رُبَّمَا تُرْعِبِينَ.
\par 13 قَدْ ضَعُفْتِ مِنْ كَثْرَةِ مَشُورَاتِكِ. لِيَقِفْ قَاسِمُو السَّمَاءِ الرَّاصِدُونَ النُّجُومَ الْمُعَرِّفُونَ عِنْدَ رُؤُوسِ الشُّهُورِ وَيُخَلِّصُوكِ مِمَّا يَأْتِي عَلَيْكِ.
\par 14 هَا إِنَّهُمْ قَدْ صَارُوا كَالْقَشِّ. أَحْرَقَتْهُمُ النَّارُ. لاَ يُنَجُّونَ أَنْفُسَهُمْ مِنْ يَدِ اللَّهِيبِ. لَيْسَ هُوَ جَمْراً لِلاِسْتِدْفَاءِ وَلاَ نَاراً لِلْجُلُوسِ تُجَاهَهَا.
\par 15 هَكَذَا صَارَ لَكِ الَّذِينَ تَعِبْتِ فِيهِمْ. تُجَّارُكِ مُنْذُ صَبَاكِ قَدْ شَرَدُوا كُلُّ وَاحِدٍ عَلَى وَجْهِهِ وَلَيْسَ مَنْ يُخَلِّصُكِ».

\chapter{48}

\par 1 «اِسْمَعُوا هَذَا يَا بَيْتَ يَعْقُوبَ الْمَدْعُوِّينَ بِاسْمِ إِسْرَائِيلَ الَّذِينَ خَرَجُوا مِنْ مِيَاهِ يَهُوذَا الْحَالِفِينَ بِاسْمِ الرَّبِّ وَالَّذِينَ يَذْكُرُونَ إِلَهَ إِسْرَائِيلَ لَيْسَ بِالصِّدْقِ وَلاَ بِالْحَقِّ!
\par 2 فَإِنَّهُمْ يُسَمَّوْنَ مِنْ مَدِينَةِ الْقُدْسِ وَيُسْنَدُونَ إِلَى إِلَهِ إِسْرَائِيلَ. رَبُّ الْجُنُودِ اسْمُهُ.
\par 3 بِالأَوَّلِيَّاتِ مُنْذُ زَمَانٍ أَخْبَرْتُ وَمِنْ فَمِي خَرَجَتْ وَأَنْبَأْتُ بِهَا. بَغْتَةً صَنَعْتُهَا فَأَتَتْ.
\par 4 لِمَعْرِفَتِي أَنَّكَ قَاسٍ وَعَضَلٌ مِنْ حَدِيدٍ عُنُقُكَ وَجِبْهَتُكَ نُحَاسٌ
\par 5 أَخْبَرْتُكَ مُنْذُ زَمَانٍ. قَبْلَمَا أَتَتْ أَنْبَأْتُكَ لِئَلاَّ تَقُولَ: صَنَمِي قَدْ صَنَعَهَا وَمَنْحُوتِي وَمَسْبُوكِي أَمَرَ بِهَا.
\par 6 قَدْ سَمِعْتَ فَانْظُرْ كُلَّهَا. وَأَنْتُمْ أَلاَ تُخْبِرُونَ؟ قَدْ أَنْبَأْتُكَ بِحَدِيثَاتٍ مُنْذُ الآنَ وَبِمَخْفِيَّاتٍ لَمْ تَعْرِفْهَا.
\par 7 الآنَ خُلِقَتْ وَلَيْسَ مُنْذُ زَمَانٍ وَقَبْلَ الْيَوْمِ لَمْ تَسْمَعْ بِهَا لِئَلاَّ تَقُولَ: هَئَنَذَا قَدْ عَرَفْتُهَا.
\par 8 لَمْ تَسْمَعْ وَلَمْ تَعْرِفْ وَمُنْذُ زَمَانٍ لَمْ تَنْفَتِحْ أُذُنُكَ فَإِنِّي عَلِمْتُ أَنَّكَ تَغْدُرُ غَدْراً وَمِنَ الْبَطْنِ سُمِّيتَ عَاصِياً.
\par 9 مِنْ أَجْلِ اسْمِي أُبَطِّئُ غَضَبِي وَمِنْ أَجْلِ فَخْرِي أُمْسِكُ عَنْكَ حَتَّى لاَ أَقْطَعَكَ.
\par 10 هَئَنَذَا قَدْ نَقَّيْتُكَ وَلَيْسَ بِفِضَّةٍ. اخْتَرْتُكَ فِي كُورِ الْمَشَقَّةِ.
\par 11 مِنْ أَجْلِ نَفْسِي مِنْ أَجْلِ نَفْسِي أَفْعَلُ. لأَنَّهُ كَيْفَ يُدَنَّسُ اسْمِي؟ وَكَرَامَتِي لاَ أُعْطِيهَا لِآخَرَ.
\par 12 «اِسْمَعْ لِي يَا يَعْقُوبُ. وَإِسْرَائِيلُ الَّذِي دَعَوْتُهُ. أَنَا هُوَ. أَنَا الأَوَّلُ وَأَنَا الآخِرُ
\par 13 وَيَدِي أَسَّسَتِ الأَرْضَ وَيَمِينِي نَشَرَتِ السَّمَاوَاتِ. أَنَا أَدْعُوهُنَّ فَيَقِفْنَ مَعاً.
\par 14 اِجْتَمِعُوا كُلُّكُمْ وَاسْمَعُوا. مَنْ مِنْهُمْ أَخْبَرَ بِهَذِهِ؟ قَدْ أَحَبَّهُ الرَّبُّ. يَصْنَعُ مَسَرَّتَهُ بِبَابِلَ وَيَكُونُ ذِرَاعُهُ عَلَى الْكِلْدَانِيِّينَ.
\par 15 أَنَا أَنَا تَكَلَّمْتُ وَدَعَوْتُهُ. أَتَيْتُ بِهِ فَيَنْجَحُ طَرِيقُهُ.
\par 16 تَقَدَّمُوا إِلَيَّ. اسْمَعُوا هَذَا. لَمْ أَتَكَلَّمْ مِنَ الْبَدْءِ فِي الْخَفَاءِ. مُنْذُ وُجُودِهِ أَنَا هُنَاكَ وَالآنَ السَّيِّدُ الرَّبُّ أَرْسَلَنِي وَرُوحُهُ.
\par 17 «هَكَذَا يَقُولُ الرَّبُّ فَادِيكَ قُدُّوسُ إِسْرَائِيلَ: أَنَا الرَّبُّ إِلَهُكَ مُعَلِّمُكَ لِتَنْتَفِعَ وَأُمَشِّيكَ فِي طَرِيقٍ تَسْلُكُ فِيهِ.
\par 18 لَيْتَكَ أَصْغَيْتَ لِوَصَايَايَ فَكَانَ كَنَهْرٍ سَلاَمُكَ وَبِرُّكَ كَلُجَجِ الْبَحْرِ.
\par 19 وَكَانَ كَالرَّمْلِ نَسْلُكَ وَذُرِّيَّةُ أَحْشَائِكَ كَأَحْشَائِهِ. لاَ يَنْقَطِعُ وَلاَ يُبَادُ اسْمُهُ مِنْ أَمَامِي.
\par 20 «اُخْرُجُوا مِنْ بَابِلَ اهْرُبُوا مِنْ أَرْضِ الْكِلْدَانِيِّينَ. بِصَوْتِ التَّرَنُّمِ أَخْبِرُوا. نَادُوا بِهَذَا. شَيِّعُوهُ إِلَى أَقْصَى الأَرْضِ. قُولُوا: قَدْ فَدَى الرَّبُّ عَبْدَهُ يَعْقُوبَ.
\par 21 وَلَمْ يَعْطَشُوا فِي الْقِفَارِ الَّتِي سَيَّرَهُمْ فِيهَا. أَجْرَى لَهُمْ مِنَ الصَّخْرِ مَاءً وَشَقَّ الصَّخْرَ فَفَاضَتِ الْمِيَاهُ.
\par 22 لاَ سَلاَمَ قَالَ الرَّبُّ لِلأَشْرَارِ».

\chapter{49}

\par 1 اِسْمَعِي لِي أَيَّتُهَا الْجَزَائِرُ وَاصْغُوا أَيُّهَا الأُمَمُ مِنْ بَعِيدٍ: الرَّبُّ مِنَ الْبَطْنِ دَعَانِي. مِنْ أَحْشَاءِ أُمِّي ذَكَرَ اسْمِي
\par 2 وَجَعَلَ فَمِي كَسَيْفٍ حَادٍّ. فِي ظِلِّ يَدِهِ خَبَّأَنِي وَجَعَلَنِي سَهْماً مَبْرِيّاً. فِي كِنَانَتِهِ أَخْفَانِي.
\par 3 وَقَالَ لِي: «أَنْتَ عَبْدِي إِسْرَائِيلُ الَّذِي بِهِ أَتَمَجَّدُ».
\par 4 أَمَّا أَنَا فَقُلْتُ عَبَثاً تَعِبْتُ. بَاطِلاً وَفَارِغاً أَفْنَيْتُ قُدْرَتِي. لَكِنَّ حَقِّي عِنْدَ الرَّبِّ وَعَمَلِي عِنْدَ إِلَهِي.
\par 5 وَالآنَ قَالَ الرَّبُّ جَابِلِي مِنَ الْبَطْنِ عَبْداً لَهُ لإِرْجَاعِ يَعْقُوبَ إِلَيْهِ فَيَنْضَمُّ إِلَيْهِ إِسْرَائِيلُ (فَأَتَمَجَّدُ فِي عَيْنَيِ الرَّبِّ وَإِلَهِي يَصِيرُ قُوَّتِي).
\par 6 فَقَالَ: «قَلِيلٌ أَنْ تَكُونَ لِي عَبْداً لإِقَامَةِ أَسْبَاطِ يَعْقُوبَ وَرَدِّ مَحْفُوظِي إِسْرَائِيلَ. فَقَدْ جَعَلْتُكَ نُوراً لِلأُمَمِ لِتَكُونَ خَلاَصِي إِلَى أَقْصَى الأَرْضِ».
\par 7 هَكَذَا قَالَ الرَّبُّ فَادِي إِسْرَائِيلَ قُدُّوسُهُ لِلْمُهَانِ النَّفْسِ لِمَكْرُوهِ الأُمَّةِ لِعَبْدِ الْمُتَسَلِّطِينَ: «يَنْظُرُ مُلُوكٌ فَيَقُومُونَ. رُؤَسَاءُ فَيَسْجُدُونَ. لأَجْلِ الرَّبِّ الَّذِي هُوَ أَمِينٌ وَقُدُّوسِ إِسْرَائِيلَ الَّذِي قَدِ اخْتَارَكَ».
\par 8 هَكَذَا قَالَ الرَّبُّ: «فِي وَقْتِ الْقُبُولِ اسْتَجَبْتُكَ وَفِي يَوْمِ الْخَلاَصِ أَعَنْتُكَ. فَأَحْفَظُكَ وَأَجْعَلُكَ عَهْداً لِلشَّعْبِ لإِقَامَةِ الأَرْضِ لِتَمْلِيكِ أَمْلاَكِ الْبَرَارِيِّ
\par 9 قَائِلاً لِلأَسْرَى: اخْرُجُوا. لِلَّذِينَ فِي الظَّلاَمِ: اظْهَرُوا. عَلَى الطُّرُقِ يَرْعُونَ وَفِي كُلِّ الْهِضَابِ مَرْعَاهُمْ.
\par 10 لاَ يَجُوعُونَ وَلاَ يَعْطَشُونَ وَلاَ يَضْرِبُهُمْ حَرٌّ وَلاَ شَمْسٌ لأَنَّ الَّذِي يَرْحَمُهُمْ يَهْدِيهِمْ وَإِلَى يَنَابِيعِ الْمِيَاهِ يُورِدُهُمْ.
\par 11 وَأَجْعَلُ كُلَّ جِبَالِي طَرِيقاً وَمَنَاهِجِي تَرْتَفِعُ.
\par 12 هَؤُلاَءِ مِنْ بَعِيدٍ يَأْتُونَ وَهَؤُلاَءِ مِنَ الشِّمَالِ وَمِنَ الْمَغْرِبِ وَهَؤُلاَءِ مِنْ أَرْضِ سِينِيمَ».
\par 13 تَرَنَّمِي أَيَّتُهَا السَّمَاوَاتُ وَابْتَهِجِي أَيَّتُهَا الأَرْضُ. لِتُشِدِ الْجِبَالُ بِالتَّرَنُّمِ لأَنَّ الرَّبَّ قَدْ عَزَّى شَعْبَهُ وَعَلَى بَائِسِيهِ يَتَرَحَّمُ.
\par 14 وَقَالَتْ صِهْيَوْنُ: «قَدْ تَرَكَنِي الرَّبُّ وَسَيِّدِي نَسِينِي».
\par 15 هَلْ تَنْسَى الْمَرْأَةُ رَضِيعَهَا فَلاَ تَرْحَمَ ابْنَ بَطْنِهَا؟ حَتَّى هَؤُلاَءِ يَنْسِينَ وَأَنَا لاَ أَنْسَاكِ.
\par 16 هُوَذَا عَلَى كَفَّيَّ نَقَشْتُكِ. أَسْوَارُكِ أَمَامِي دَائِماً.
\par 17 قَدْ أَسْرَعَ بَنُوكِ. هَادِمُوكِ وَمُخْرِبُوكِ مِنْكِ يَخْرُجُونَ.
\par 18 اِرْفَعِي عَيْنَيْكِ حَوَالَيْكِ وَانْظُرِي. كُلُّهُمْ قَدِ اجْتَمَعُوا أَتُوا إِلَيْكِ. حَيٌّ أَنَا يَقُولُ الرَّبُّ: إِنَّكِ تَلْبِسِينَ كُلَّهُمْ كَحُلِيٍّ وَتَتَنَطَّقِينَ بِهِمْ كَعَرُوسٍ.
\par 19 إِنَّ خِرَبَكِ وَبَرَارِيَّكِ وَأَرْضَ خَرَابِكِ إِنَّكِ تَكُونِينَ الآنَ ضَيِّقَةً عَلَى السُّكَّانِ وَيَتَبَاعَدُ مُبْتَلِعُوكِ.
\par 20 يَقُولُ أَيْضاً فِي أُذُنَيْكِ بَنُو ثُكْلِكِ: «ضَيِّقٌ عَلَيَّ الْمَكَانُ. وَسِّعِي لِي لأَسْكُنَ».
\par 21 فَتَقُولِينَ فِي قَلْبِكِ: «مَنْ وَلَدَ لِي هَؤُلاَءِ وَأَنَا ثَكْلَى وَعَاقِرٌ مَنْفِيَّةٌ وَمَطْرُودَةٌ؟ وَهَؤُلاَءِ مَنْ رَبَّاهُمْ؟ هَئَنَذَا كُنْتُ مَتْرُوكَةً وَحْدِي. هَؤُلاَءِ أَيْنَ كَانُوا؟».
\par 22 هَكَذَا قَالَ السَّيِّدُ الرَّبُّ: «هَا إِنِّي أَرْفَعُ إِلَى الأُمَمِ يَدِي وَإِلَى الشُّعُوبِ أُقِيمُ رَايَتِي فَيَأْتُونَ بِأَوْلاَدِكِ فِي الأَحْضَانِ وَبَنَاتُكِ عَلَى الأَكْتَافِ يُحْمَلْنَ.
\par 23 وَيَكُونُ الْمُلُوكُ حَاضِنِيكِ وَسَيِّدَاتُهُمْ مُرْضِعَاتِكِ. بِالْوُجُوهِ إِلَى الأَرْضِ يَسْجُدُونَ لَكِ وَيَلْحَسُونَ غُبَارَ رِجْلَيْكِ فَتَعْلَمِينَ أَنِّي أَنَا الرَّبُّ الَّذِي لاَ يَخْزَى مُنْتَظِرُوهُ».
\par 24 هَلْ تُسْلَبُ مِنَ الْجَبَّارِ غَنِيمَةٌ وَهَلْ يُفْلِتُ سَبْيُ الْمَنْصُورِ؟
\par 25 فَإِنَّهُ هَكَذَا قَالَ الرَّبُّ: «حَتَّى سَبْيُ الْجَبَّارِ يُسْلَبُ وَغَنِيمَةُ الْعَاتِي تُفْلِتُ. وَأَنَا أُخَاصِمُ مُخَاصِمَكِ وَأُخَلِّصُ أَوْلاَدَكِ
\par 26 وَأَطْعِمُ ظَالِمِيكِ لَحْمَ أَنْفُسِهِمْ وَيَسْكَرُونَ بِدَمِهِمْ كَمَا مِنْ سُلاَفٍ فَيَعْلَمُ كُلُّ بَشَرٍ أَنِّي أَنَا الرَّبُّ مُخَلِّصُكِ وَفَادِيكِ عَزِيزُ يَعْقُوبَ».

\chapter{50}

\par 1 هَكَذَا قَالَ الرَّبُّ: «أَيْنَ كِتَابُ طَلاَقِ أُمِّكُمُ الَّتِي طَلَّقْتُهَا أَوْ مَنْ هُوَ مِنْ غُرَمَائِي الَّذِي بِعْتُهُ إِيَّاكُمْ؟ هُوَذَا مِنْ أَجْلِ آثَامِكُمْ قَدْ بُعْتُمْ وَمِنْ أَجْلِ ذُنُوبِكُمْ طُلِّقَتْ أُمُّكُمْ.
\par 2 لِمَاذَا جِئْتُ وَلَيْسَ إِنْسَانٌ نَادَيْتُ وَلَيْسَ مُجِيبٌ؟ هَلْ قَصَرَتْ يَدِي عَنِ الْفِدَاءِ وَهَلْ لَيْسَ فِيَّ قُدْرَةٌ لِلإِنْقَاذِ؟ هُوَذَا بِزَجْرَتِي أُنَشِّفُ الْبَحْرَ. أَجْعَلُ الأَنْهَارَ قَفْراً. يُنْتِنُ سَمَكُهَا مِنْ عَدَمِ الْمَاءِ وَيَمُوتُ بِالْعَطَشِ.
\par 3 أُلْبِسُ السَّمَاوَاتِ ظَلاَماً وَأَجْعَلُ الْمِسْحَ غِطَاءَهَا».
\par 4 أَعْطَانِي السَّيِّدُ الرَّبُّ لِسَانَ الْمُتَعَلِّمِينَ لأَعْرِفَ أَنْ أُغِيثَ الْمُعْيِيَ بِكَلِمَةٍ. يُوقِظُ كُلَّ صَبَاحٍ يُوقِظُ لِي أُذُناً لأَسْمَعَ كَالْمُتَعَلِّمِينَ.
\par 5 السَّيِّدُ الرَّبُّ فَتَحَ لِي أُذُناً وَأَنَا لَمْ أُعَانِدْ. إِلَى الْوَرَاءِ لَمْ أَرْتَدَّ.
\par 6 بَذَلْتُ ظَهْرِي لِلضَّارِبِينَ وَخَدَّيَّ لِلنَّاتِفِينَ. وَجْهِي لَمْ أَسْتُرْ عَنِ الْعَارِ وَالْبَصْقِ.
\par 7 وَالسَّيِّدُ الرَّبُّ يُعِينُنِي لِذَلِكَ لاَ أَخْجَلُ. لِذَلِكَ جَعَلْتُ وَجْهِي كَالصَّوَّانِ وَعَرَفْتُ أَنِّي لاَ أَخْزَى.
\par 8 قَرِيبٌ هُوَ الَّذِي يُبَرِّرُنِي. مَنْ يُخَاصِمُنِي؟ لِنَتَوَاقَفْ! مَنْ هُوَ صَاحِبُ دَعْوَى مَعِي؟ لِيَتَقَدَّمْ إِلَيّ!
\par 9 هُوَذَا السَّيِّدُ الرَّبُّ يُعِينُنِي. مَنْ هُوَ الَّذِي يَحْكُمُ عَلَيَّ؟ هُوَذَا كُلُّهُمْ كَالثَّوْبِ يَبْلُونَ. يَأْكُلُهُمُ الْعُثُّ.
\par 10 مَنْ مِنْكُمْ خَائِفُ الرَّبِّ سَامِعٌ لِصَوْتِ عَبْدِهِ؟ مَنِ الَّذِي يَسْلُكُ فِي الظُّلُمَاتِ وَلاَ نُورَ لَهُ؟ فَلْيَتَّكِلْ عَلَى اسْمِ الرَّبِّ وَيَسْتَنِدْ إِلَى إِلَهِهِ.
\par 11 يَا هَؤُلاَءِ جَمِيعُكُمُ الْقَادِحِينَ نَاراً الْمُتَنَطِّقِينَ بِشَرَارٍ اسْلُكُوا بِنُورِ نَارِكُمْ وَبِالشَّرَارِ الَّذِي أَوْقَدْتُمُوهُ. مِنْ يَدِي صَارَ لَكُمْ هَذَا. فِي الْوَجَعِ تَضْطَجِعُونَ.

\chapter{51}

\par 1 اِسْمَعُوا لِي أَيُّهَا التَّابِعُونَ الْبِرَّ الطَّالِبُونَ الرَّبَّ. انْظُرُوا إِلَى الصَّخْرِ الَّذِي مِنْهُ قُطِعْتُمْ وَإِلَى نُقْرَةِ الْجُبِّ الَّتِي مِنْهَا حُفِرْتُمُ.
\par 2 انْظُرُوا إِلَى إِبْرَاهِيمَ أَبِيكُمْ وَإِلَى سَارَةَ الَّتِي وَلَدَتْكُمْ. لأَنِّي دَعَوْتُهُ وَهُوَ وَاحِدٌ وَبَارَكْتُهُ وَأَكْثَرْتُهُ.
\par 3 فَإِنَّ الرَّبَّ قَدْ عَزَّى صِهْيَوْنَ. عَزَّى كُلَّ خِرَبِهَا وَيَجْعَلُ بَرِّيَّتَهَا كَعَدْنٍ وَبَادِيَتَهَا كَجَنَّةِ الرَّبِّ. الْفَرَحُ وَالاِبْتِهَاجُ يُوجَدَانِ فِيهَا. الْحَمْدُ وَصَوْتُ التَّرَنُّمِ.
\par 4 اُنْصُتُوا إِلَيَّ يَا شَعْبِي وَيَا أُمَّتِي اصْغِي إِلَيَّ. لأَنَّ شَرِيعَةً مِنْ عِنْدِي تَخْرُجُ وَحَقِّي أُثَبِّتُهُ نُوراً لِلشُّعُوبِ.
\par 5 قَرِيبٌ بِرِّي. قَدْ بَرَزَ خَلاَصِي وَذِرَاعَايَ يَقْضِيَانِ لِلشُّعُوبِ. إِيَّايَ تَرْجُو الْجَزَائِرُ وَتَنْتَظِرُ ذِرَاعِي.
\par 6 اِرْفَعُوا إِلَى السَّمَاوَاتِ عُيُونَكُمْ وَانْظُرُوا إِلَى الأَرْضِ مِنْ تَحْتٍ. فَإِنَّ السَّمَاوَاتِ كَالدُّخَانِ تَضْمَحِلُّ وَالأَرْضَ كَالثَّوْبِ تَبْلَى وَسُكَّانَهَا كَالْبَعُوضِ يَمُوتُونَ. أَمَّا خَلاَصِي فَإِلَى الأَبَدِ يَكُونُ وَبِرِّي لاَ يُنْقَضُ.
\par 7 اِسْمَعُوا لِي يَا عَارِفِي الْبِرِّ الشَّعْبَ الَّذِي شَرِيعَتِي فِي قَلْبِهِ. لاَ تَخَافُوا مِنْ تَعْيِيرِ النَّاسِ وَمِنْ شَتَائِمِهِمْ لاَ تَرْتَاعُوا
\par 8 لأَنَّهُ كَالثَّوْبِ يَأْكُلُهُمُ الْعُثُّ وَكَالصُّوفِ يَأْكُلُهُمُ السُّوسُ. أَمَّا بِرِّي فَإِلَى الأَبَدِ يَكُونُ وَخَلاَصِي إِلَى دَوْرِ الأَدْوَارِ.
\par 9 اِسْتَيْقِظِي اسْتَيْقِظِي! الْبِسِي قُوَّةً يَا ذِرَاعَ الرَّبِّ! اسْتَيْقِظِي كَمَا فِي أَيَّامِ الْقِدَمِ كَمَا فِي الأَدْوَارِ الْقَدِيمَةِ. أَلَسْتِ أَنْتِ الْقَاطِعَةَ رَهَبَ الطَّاعِنَةَ التِّنِّينَ؟
\par 10 أَلَسْتِ أَنْتِ هِيَ الْمُنَشِّفَةَ الْبَحْرَ مِيَاهَ الْغَمْرِ الْعَظِيمِ الْجَاعِلَةَ أَعْمَاقَ الْبَحْرِ طَرِيقاً لِعُبُورِ الْمَفْدِيِّينَ؟
\par 11 وَمَفْدِيُّو الرَّبِّ يَرْجِعُونَ وَيَأْتُونَ إِلَى صِهْيَوْنَ بِالتَّرَنُّمِ وَعَلَى رُؤُوسِهِمْ فَرَحٌ أَبَدِيٌّ. ابْتِهَاجٌ وَفَرَحٌ يُدْرِكَانِهِمْ. يَهْرُبُ الْحُزْنُ وَالتَّنَهُّدُ.
\par 12 أَنَا أَنَا هُوَ مُعَزِّيكُمْ. مَنْ أَنْتِ حَتَّى تَخَافِي مِنْ إِنْسَانٍ يَمُوتُ وَمِنِ ابْنِ الإِنْسَانِ الَّذِي يُجْعَلُ كَالْعُشْبِ؟
\par 13 وَتَنْسَى الرَّبَّ صَانِعَكَ بَاسِطَ السَّمَاوَاتِ وَمُؤَسِّسَ الأَرْضِ وَتَفْزَعُ دَائِماً كُلَّ يَوْمٍ مِنْ غَضَبِ الْمُضَايِقِ عِنْدَمَا هَيَّأَ لِلإِهْلاَكِ. وَأَيْنَ غَضَبُ الْمُضَايِقِ؟
\par 14 سَرِيعاً يُطْلَقُ الْمُنْحَنِي وَلاَ يَمُوتُ فِي الْجُبِّ وَلاَ يُعْدَمُ خُبْزُهُ.
\par 15 وَأَنَا الرَّبُّ إِلَهُكَ مُزْعِجُ الْبَحْرِ فَتَعِجُّ لُجَجُهُ. رَبُّ الْجُنُودِ اسْمُهُ.
\par 16 وَقَدْ جَعَلْتُ أَقْوَالِي فِي فَمِكَ وَبِظِلِّ يَدِي سَتَرْتُكَ لِغَرْسِ السَّمَاوَاتِ وَتَأْسِيسِ الأَرْضِ وَلِتَقُولَ لِصِهْيَوْنَ: «أَنْتِ شَعْبِي».
\par 17 اِنْهَضِي انْهَضِي! قُومِي يَا أُورُشَلِيمُ الَّتِي شَرِبْتِ مِنْ يَدِ الرَّبِّ كَأْسَ غَضَبِهِ. ثُفْلَ كَأْسِ التَّرَنُّحِ شَرِبْتِ. مَصَصْتِ.
\par 18 لَيْسَ لَهَا مَنْ يَقُودُهَا مِنْ جَمِيعِ الْبَنِينَ الَّذِينَ وَلَدَتْهُمْ وَلَيْسَ مَنْ يُمْسِكُ بِيَدِهَا مِنْ جَمِيعِ الْبَنِينَ الَّذِينَ رَبَّتْهُمْ.
\par 19 اِثْنَانِ هُمَا مُلاَقِيَاكِ. مَنْ يَرْثِي لَكِ؟ الْخَرَابُ وَالاِنْسِحَاقُ وَالْجُوعُ وَالسَّيْفُ. بِمَنْ أُعَزِّيكِ؟
\par 20 بَنُوكِ قَدْ أَعْيُوا. اضْطَجَعُوا فِي رَأْسِ كُلِّ زُقَاقٍ كَالْوَعْلِ فِي شَبَكَةٍ. الْمَلآنُونَ مِنْ غَضَبِ الرَّبِّ مِنْ زَجْرَةِ إِلَهِكِ.
\par 21 لِذَلِكَ اسْمَعِي هَذَا أَيَّتُهَا الْبَائِسَةُ وَالسَّكْرَى وَلَيْسَ بِالْخَمْرِ.
\par 22 هَكَذَا قَالَ سَيِّدُكِ الرَّبُّ وَإِلَهُكِ الَّذِي يُحَاكِمُ لِشَعْبِهِ: «هَئَنَذَا قَدْ أَخَذْتُ مِنْ يَدِكِ كَأْسَ التَّرَنُّحِ ثُفْلَ كَأْسِ غَضَبِي. لاَ تَعُودِينَ تَشْرَبِينَهَا فِي مَا بَعْدُ.
\par 23 وَأَضَعُهَا فِي يَدِ مُعَذِّبِيكِ الَّذِينَ قَالُوا لِنَفْسِكِ: انْحَنِي لِنَعْبُرَ فَوَضَعْتِ كَالأَرْضِ ظَهْرَكِ وَكَالزُّقَاقِ لِلْعَابِرِينَ».

\chapter{52}

\par 1 اِسْتَيْقِظِي اسْتَيْقِظِي! الْبِسِي عِزَّكِ يَا صِهْيَوْنُ! الْبِسِي ثِيَابَ جَمَالِكِ يَا أُورُشَلِيمُ الْمَدِينَةُ الْمُقَدَّسَةُ لأَنَّهُ لاَ يَعُودُ يَدْخُلُكِ فِي مَا بَعْدُ أَغْلَفُ وَلاَ نَجِسٌ.
\par 2 اِنْتَفِضِي مِنَ التُّرَابِ. قُومِي اجْلِسِي يَا أُورُشَلِيمُ. انْحَلِّي مِنْ رُبُطِ عُنُقِكِ أَيَّتُهَا الْمَسْبِيَّةُ ابْنَةُ صِهْيَوْنَ.
\par 3 فَإِنَّهُ هَكَذَا قَالَ الرَّبُّ: «مَجَّاناً بُعْتُمْ وَبِلاَ فِضَّةٍ تُفَكُّونَ».
\par 4 لأَنَّهُ هَكَذَا قَالَ السَّيِّدُ الرَّبُّ: «إِلَى مِصْرَ نَزَلَ شَعْبِي أَوَّلاً لِيَتَغَرَّبَ هُنَاكَ. ثُمَّ ظَلَمَهُ أَشُّورُ بِلاَ سَبَبٍ.
\par 5 فَالآنَ مَاذَا لِي هُنَا يَقُولُ الرَّبُّ حَتَّى أُخِذَ شَعْبِي مَجَّاناً؟ الْمُتَسَلِّطُونَ عَلَيْهِ يَصِيحُونَ يَقُولُ الرَّبُّ وَدَائِماً كُلَّ يَوْمٍ اسْمِي يُهَانُ.
\par 6 لِذَلِكَ يَعْرِفُ شَعْبِيَ اسْمِي. لِذَلِكَ فِي ذَلِكَ الْيَوْمِ يَعْرِفُونَ أَنِّي أَنَا هُوَ الْمُتَكَلِّمُ. هَئَنَذَا».
\par 7 مَا أَجْمَلَ عَلَى الْجِبَالِ قَدَمَيِ الْمُبَشِّرِ الْمُخْبِرِ بِالسَّلاَمِ الْمُبَشِّرِ بِالْخَيْرِ الْمُخْبِرِ بِالْخَلاَصِ الْقَائِلِ لِصِهْيَوْنَ: «قَدْ مَلَكَ إِلَهُكِ!»
\par 8 صَوْتُ مُرَاقِبِيكِ. يَرْفَعُونَ صَوْتَهُمْ. يَتَرَنَّمُونَ مَعاً لأَنَّهُمْ يُبْصِرُونَ عَيْناً لِعَيْنٍ عَُِنْدَ رُجُوعِ الرَّبِّ إِلَى صِهْيَوْنَ.
\par 9 أَشِيدِي تَرَنَّمِي مَعاً يَا خِرَبَ أُورُشَلِيمَ لأَنَّ الرَّبَّ قَدْ عَزَّى شَعْبَهُ. فَدَى أُورُشَلِيمَ.
\par 10 قَدْ شَمَّرَ الرَّبُّ عَنْ ذِرَاعِ قُدْسِهِ أَمَامَ عُيُونِ كُلِّ الأُمَمِ فَتَرَى كُلُّ أَطْرَافِ الأَرْضِ خَلاَصَ إِلَهِنَا.
\par 11 اِعْتَزِلُوا. اعْتَزِلُوا. اخْرُجُوا مِنْ هُنَاكَ. لاَ تَمَسُّوا نَجِساً. اخْرُجُوا مِنْ وَسَطِهَا. تَطَهَّرُوا يَا حَامِلِي آنِيَةِ الرَّبِّ.
\par 12 لأَنَّكُمْ لاَ تَخْرُجُونَ بِالْعَجَلَةِ وَلاَ تَذْهَبُونَ هَارِبِينَ. لأَنَّ الرَّبَّ سَائِرٌ أَمَامَكُمْ وَإِلَهَ إِسْرَائِيلَ يَجْمَعُ سَاقَتَكُمْ.
\par 13 هُوَذَا عَبْدِي يَعْقِلُ يَتَعَالَى وَيَرْتَقِي وَيَتَسَامَى جِدّاً.
\par 14 كَمَا انْدَهَشَ مِنْكَ كَثِيرُونَ. كَانَ مَنْظَرُهُ كَذَا مُفْسَداً أَكْثَرَ مِنَ الرَّجُلِ وَصُورَتُهُ أَكْثَرَ مِنْ بَنِي آدَمَ.
\par 15 هَكَذَا يَنْضِحُ أُمَماً كَثِيرِينَ. مِنْ أَجْلِهِ يَسُدُّ مُلُوكٌ أَفْوَاهَهُمْ لأَنَّهُمْ قَدْ أَبْصَرُوا مَا لَمْ يُخْبَرُوا بِهِ وَمَا لَمْ يَسْمَعُوهُ فَهِمُوهُ.

\chapter{53}

\par 1 مَنْ صَدَّقَ خَبَرَنَا وَلِمَنِ اسْتُعْلِنَتْ ذِرَاعُ الرَّبِّ؟
\par 2 نَبَتَ قُدَّامَهُ كَفَرْخٍ وَكَعِرْقٍ مِنْ أَرْضٍ يَابِسَةٍ لاَ صُورَةَ لَهُ وَلاَ جَمَالَ فَنَنْظُرَ إِلَيْهِ وَلاَ مَنْظَرَ فَنَشْتَهِيهِ.
\par 3 مُحْتَقَرٌ وَمَخْذُولٌ مِنَ النَّاسِ رَجُلُ أَوْجَاعٍ وَمُخْتَبِرُ الْحُزْنِ وَكَمُسَتَّرٍ عَنْهُ وُجُوهُنَا مُحْتَقَرٌ فَلَمْ نَعْتَدَّ بِهِ.
\par 4 لَكِنَّ أَحْزَانَنَا حَمَلَهَا وَأَوْجَاعَنَا تَحَمَّلَهَا. وَنَحْنُ حَسِبْنَاهُ مُصَاباً مَضْرُوباً مِنَ اللَّهِ وَمَذْلُولاً.
\par 5 وَهُوَ مَجْرُوحٌ لأَجْلِ مَعَاصِينَا مَسْحُوقٌ لأَجْلِ آثَامِنَا. تَأْدِيبُ سَلاَمِنَا عَلَيْهِ وَبِحُبُرِهِ شُفِينَا.
\par 6 كُلُّنَا كَغَنَمٍ ضَلَلْنَا. مِلْنَا كُلُّ وَاحِدٍ إِلَى طَرِيقِهِ وَالرَّبُّ وَضَعَ عَلَيْهِ إِثْمَ جَمِيعِنَا.
\par 7 ظُلِمَ أَمَّا هُوَ فَتَذَلَّلَ وَلَمْ يَفْتَحْ فَاهُ كَشَاةٍ تُسَاقُ إِلَى الذَّبْحِ وَكَنَعْجَةٍ صَامِتَةٍ أَمَامَ جَازِّيهَا فَلَمْ يَفْتَحْ فَاهُ.
\par 8 مِنَ الضُّغْطَةِ وَمِنَ الدَّيْنُونَةِ أُخِذَ. وَفِي جِيلِهِ مَنْ كَانَ يَظُنُّ أَنَّهُ قُطِعَ مِنْ أَرْضِ الأَحْيَاءِ أَنَّهُ ضُرِبَ مِنْ أَجْلِ ذَنْبِ شَعْبِي؟
\par 9 وَجُعِلَ مَعَ الأَشْرَارِ قَبْرُهُ وَمَعَ غَنِيٍّ عِنْدَ مَوْتِهِ. عَلَى أَنَّهُ لَمْ يَعْمَلْ ظُلْماً وَلَمْ يَكُنْ فِي فَمِهِ غِشٌّ.
\par 10 أَمَّا الرَّبُّ فَسُرَّ بِأَنْ يَسْحَقَهُ بِالْحُزْنِ. إِنْ جَعَلَ نَفْسَهُ ذَبِيحَةَ إِثْمٍ يَرَى نَسْلاً تَطُولُ أَيَّامُهُ وَمَسَرَّةُ الرَّبِّ بِيَدِهِ تَنْجَحُ.
\par 11 مِنْ تَعَبِ نَفْسِهِ يَرَى وَيَشْبَعُ وَعَبْدِي الْبَارُّ بِمَعْرِفَتِهِ يُبَرِّرُ كَثِيرِينَ وَآثَامُهُمْ هُوَ يَحْمِلُهَا.
\par 12 لِذَلِكَ أَقْسِمُ لَهُ بَيْنَ الأَعِزَّاءِ وَمَعَ الْعُظَمَاءِ يَقْسِمُ غَنِيمَةً مِنْ أَجْلِ أَنَّهُ سَكَبَ لِلْمَوْتِ نَفْسَهُ وَأُحْصِيَ مَعَ أَثَمَةٍ وَهُوَ حَمَلَ خَطِيَّةَ كَثِيرِينَ وَشَفَعَ فِي الْمُذْنِبِينَ.

\chapter{54}

\par 1 تَرَنَّمِي أَيَّتُهَا الْعَاقِرُ الَّتِي لَمْ تَلِدْ. أَشِيدِي بِالتَّرَنُّمِ أَيَّتُهَا الَّتِي لَمْ تَمْخَضْ لأَنَّ بَنِي الْمُسْتَوْحِشَةِ أَكْثَرُ مِنْ بَنِي ذَاتِ الْبَعْلِ قَالَ الرَّبُّ.
\par 2 أَوْسِعِي مَكَانَ خَيْمَتِكِ وَلْتُبْسَطْ شُقَقُ مَسَاكِنِكِ. لاَ تُمْسِكِي. أَطِيلِي أَطْنَابَكِ وَشَدِّدِي أَوْتَادَكِ
\par 3 لأَنَّكِ تَمْتَدِّينَ إِلَى الْيَمِينِ وَإِلَى الْيَسَارِ وَيَرِثُ نَسْلُكِ أُمَماً وَيُعَمِّرُ مُدُناً خَرِبَةً.
\par 4 لاَ تَخَافِي لأَنَّكِ لاَ تَخْزِينَ وَلاَ تَخْجَلِي لأَنَّكِ لاَ تَسْتَحِينَ. فَإِنَّكِ تَنْسِينَ خِزْيَ صَبَاكِ وَعَارُ تَرَمُّلِكِ لاَ تَذْكُرِينَهُ بَعْدُ.
\par 5 لأَنَّ بَعْلَكِ هُوَ صَانِعُكِ رَبُّ الْجُنُودِ اسْمُهُ وَوَلِيُّكِ قُدُّوسُ إِسْرَائِيلَ. إِلَهَ كُلِّ الأَرْضِ يُدْعَى.
\par 6 لأَنَّهُ كَامْرَأَةٍ مَهْجُورَةٍ وَمَحْزُونَةِ الرُّوحِ دَعَاكِ الرَّبُّ وَكَزَوْجَةِ الصِّبَا إِذَا رُذِلَتْ قَالَ إِلَهُكِ.
\par 7 لُحَيْظَةً تَرَكْتُكِ وَبِمَرَاحِمَ عَظِيمَةٍ سَأَجْمَعُكِ.
\par 8 بِفَيَضَانِ الْغَضَبِ حَجَبْتُ وَجْهِي عَنْكِ لَحْظَةً وَبِإِحْسَانٍ أَبَدِيٍّ أَرْحَمُكِ قَالَ وَلِيُّكِ الرَّبُّ.
\par 9 لأَنَّهُ كَمِيَاهِ نُوحٍ هَذِهِ لِي. كَمَا حَلَفْتُ أَنْ لاَ تَعْبُرَ بَعْدُ مِيَاهُ نُوحٍ عَلَى الأَرْضِ هَكَذَا حَلَفْتُ أَنْ لاَ أَغْضَبَ عَلَيْكِ وَلاَ أَزْجُرَكِ.
\par 10 فَإِنَّ الْجِبَالَ تَزُولُ وَالآكَامَ تَتَزَعْزَعُ أَمَّا إِحْسَانِي فَلاَ يَزُولُ عَنْكِ وَعَهْدُ سَلاَمِي لاَ يَتَزَعْزَعُ قَالَ رَاحِمُكِ الرَّبُّ.
\par 11 أَيَّتُهَا الذَّلِيلَةُ الْمُضْطَرِبَةُ غَيْرُ الْمُتَعَزِّيَةِ هَئَنَذَا أَبْنِي بِالأُثْمُدِ حِجَارَتَكِ وَبِالْيَاقُوتِ الأَزْرَقِ أُؤَسِّسُكِ
\par 12 وَأَجْعَلُ شُرَفَكِ يَاقُوتاً وَأَبْوَابَكِ حِجَارَةً بَهْرَمَانِيَّةً وَكُلَّ تُخُومِكِ حِجَارَةً كَرِيمَةً
\par 13 وَكُلَّ بَنِيكِ تَلاَمِيذَ الرَّبِّ وَسَلاَمَ بَنِيكِ كَثِيراً.
\par 14 بِالْبِرِّ تُثَبَّتِينَ بَعِيدَةً عَنِ الظُّلْمِ فَلاَ تَخَافِينَ وَعَنِ الاِرْتِعَابِ فَلاَ يَدْنُو مِنْكِ.
\par 15 هَا إِنَّهُمْ يَجْتَمِعُونَ اجْتِمَاعاً لَيْسَ مِنْ عِنْدِي. مَنِ اجْتَمَعَ عَلَيْكِ فَإِلَيْكِ يَسْقُطُ.
\par 16 هَئَنَذَا قَدْ خَلَقْتُ الْحَدَّادَ الَّذِي يَنْفُخُ الْفَحْمَ فِي النَّارِ وَيُخْرِجُ آلَةً لِعَمَلِهِ وَأَنَا خَلَقْتُ الْمُهْلِكَ لِيَخْرِبَ.
\par 17 كُلُّ آلَةٍ صُوِّرَتْ ضِدَّكِ لاَ تَنْجَحُ وَكُلُّ لِسَانٍ يَقُومُ عَلَيْكِ فِي الْقَضَاءِ تَحْكُمِينَ عَلَيْهِ. هَذَا هُوَ مِيرَاثُ عَبِيدِ الرَّبِّ وَبِرُّهُمْ مِنْ عِنْدِي يَقُولُ الرَّبُّ.

\chapter{55}

\par 1 أَيُّهَا الْعِطَاشُ جَمِيعاً هَلُمُّوا إِلَى الْمِيَاهِ وَالَّذِي لَيْسَ لَهُ فِضَّةٌ تَعَالُوا اشْتَرُوا وَكُلُوا. هَلُمُّوا اشْتَرُوا بِلاَ فِضَّةٍ وَبِلاَ ثَمَنٍ خَمْراً وَلَبَناً.
\par 2 لِمَاذَا تَزِنُونَ فِضَّةً لِغَيْرِ خُبْزٍ وَتَعَبَكُمْ لِغَيْرِ شَبَعٍ؟ اسْتَمِعُوا لِي اسْتِمَاعاً وَكُلُوا الطَّيِّبَ وَلْتَتَلَذَّذْ بِالدَّسَمِ أَنْفُسُكُمْ.
\par 3 أَمِيلُوا آذَانَكُمْ وَهَلُمُّوا إِلَيَّ. اسْمَعُوا فَتَحْيَا أَنْفُسُكُمْ. وَأَقْطَعَ لَكُمْ عَهْداً أَبَدِيّاً مَرَاحِمَ دَاوُدَ الصَّادِقَةَ.
\par 4 هُوَذَا قَدْ جَعَلْتُهُ شَارِعاً لِلشُّعُوبِ رَئِيساً وَمُوصِياً لِلشُّعُوبِ.
\par 5 هَا أُمَّةٌ لاَ تَعْرِفُهَا تَدْعُوهَا وَأُمَّةٌ لَمْ تَعْرِفْكَ تَرْكُضُ إِلَيْكَ مِنْ أَجْلِ الرَّبِّ إِلَهِكَ وَقُدُّوسِ إِسْرَائِيلَ لأَنَّهُ قَدْ مَجَّدَكَ.
\par 6 اُطْلُبُوا الرَّبَّ مَا دَامَ يُوجَدُ. ادْعُوهُ وَهُوَ قَرِيبٌ.
\par 7 لِيَتْرُكِ الشِّرِّيرُ طَرِيقَهُ وَرَجُلُ الإِثْمِ أَفْكَارَهُ وَلْيَتُبْ إِلَى الرَّبِّ فَيَرْحَمَهُ وَإِلَى إِلَهِنَا لأَنَّهُ يُكْثِرُ الْغُفْرَانَ.
\par 8 لأَنَّ أَفْكَارِي لَيْسَتْ أَفْكَارَكُمْ وَلاَ طُرُقُكُمْ طُرُقِي يَقُولُ الرَّبُّ.
\par 9 لأَنَّهُ كَمَا عَلَتِ السَّمَاوَاتُ عَنِ الأَرْضِ هَكَذَا عَلَتْ طُرُقِي عَنْ طُرُقِكُمْ وَأَفْكَارِي عَنْ أَفْكَارِكُمْ.
\par 10 لأَنَّهُ كَمَا يَنْزِلُ الْمَطَرُ وَالثَّلْجُ مِنَ السَّمَاءِ وَلاَ يَرْجِعَانِ إِلَى هُنَاكَ بَلْ يُرْوِيَانِ الأَرْضَ وَيَجْعَلاَنِهَا تَلِدُ وَتُنْبِتُ وَتُعْطِي زَرْعاً لِلزَّارِعِ وَخُبْزاً لِلآكِلِ
\par 11 هَكَذَا تَكُونُ كَلِمَتِي الَّتِي تَخْرُجُ مِنْ فَمِي. لاَ تَرْجِعُ إِلَيَّ فَارِغَةً بَلْ تَعْمَلُ مَا سُرِرْتُ بِهِ وَتَنْجَحُ فِي مَا أَرْسَلْتُهَا لَهُ.
\par 12 لأَنَّكُمْ بِفَرَحٍ تَخْرُجُونَ وَبِسَلاَمٍ تُحْضَرُونَ. الْجِبَالُ وَالآكَامُ تُشِيدُ أَمَامَكُمْ تَرَنُّماً وَكُلُّ شَجَرِ الْحَقْلِ تُصَفِّقُ بِالأَيَادِي.
\par 13 عِوَضاً عَنِ الشَّوْكِ يَنْبُتُ سَرْوٌ وَعِوَضاً عَنِ الْقَرِيسِ يَطْلَعُ آسٌ. وَيَكُونُ لِلرَّبِّ اسْماً عَلاَمَةً أَبَدِيَّةً لاَ تَنْقَطِعُ.

\chapter{56}

\par 1 هَكَذَا قَالَ الرَّبُّ: «احْفَظُوا الْحَقَّ وَأَجْرُوا الْعَدْلَ. لأَنَّهُ قَرِيبٌ مَجِيءُ خَلاَصِي وَاسْتِعْلاَنُ بِرِّي.
\par 2 طُوبَى لِلإِنْسَانِ الَّذِي يَعْمَلُ هَذَا وَلاِبْنِ الإِنْسَانِ الَّذِي يَتَمَسَّكُ بِهِ الْحَافِظِ السَّبْتَ لِئَلاَّ يُنَجِّسَهُ وَالْحَافِظِ يَدَهُ مِنْ كُلِّ عَمَلِ شَرٍّ.
\par 3 «فَلاَ يَتَكَلَّمِ ابْنُ الْغَرِيبِ الَّذِي اقْتَرَنَ بِالرَّبِّ قَائِلاً: إِفْرَازاً أَفْرَزَنِي الرَّبُّ مِنْ شَعْبِهِ. وَلاَ يَقُلِ الْخَصِيُّ: هَا أَنَا شَجَرَةٌ يَابِسَةٌ.
\par 4 لأَنَّهُ هَكَذَا قَالَ الرَّبُّ لِلْخِصْيَانِ الَّذِينَ يَحْفَظُونَ سُبُوتِي وَيَخْتَارُونَ مَا يَسُرُّنِي وَيَتَمَسَّكُونَ بِعَهْدِي:
\par 5 إِنِّي أُعْطِيهِمْ فِي بَيْتِي وَفِي أَسْوَارِي نُصُباً وَاسْماً أَفْضَلَ مِنَ الْبَنِينَ وَالْبَنَاتِ. أُعْطِيهِمُِ اسْماً أَبَدِيّاً لاَ يَنْقَطِعُ.
\par 6 وَأَبْنَاءُ الْغَرِيبِ الَّذِينَ يَقْتَرِنُونَ بِالرَّبِّ لِيَخْدِمُوهُ وَلِيُحِبُّوا اسْمَ الرَّبِّ لِيَكُونُوا لَهُ عَبِيداً كُلُّ الَّذِينَ يَحْفَظُونَ السَّبْتَ لِئَلاَّ يُنَجِّسُوهُ وَيَتَمَسَّكُونَ بِعَهْدِي
\par 7 آتِي بِهِمْ إِلَى جَبَلِ قُدْسِي وَأُفَرِّحُهُمْ فِي بَيْتِ صَلاَتِي وَتَكُونُ مُحْرَقَاتُهُمْ وَذَبَائِحُهُمْ مَقْبُولَةً عَلَى مَذْبَحِي لأَنَّ بَيْتِي بَيْتَ الصَّلاَةِ يُدْعَى لِكُلِّ الشُّعُوبِ».
\par 8 يَقُولُ السَّيِّدُ الرَّبُّ جَامِعُ مَنْفِيِّي إِسْرَائِيلَ: «أَجْمَعُ بَعْدُ إِلَيْهِ إِلَى مَجْمُوعِيهِ».
\par 9 يَا جَمِيعَ وُحُوشِ الْبَرِّ تَعَالِي لِلأَكْلِ. يَا جَمِيعَ الْوُحُوشِ الَّتِي فِي الْوَعْرِ.
\par 10 مُرَاقِبُوهُ عُمْيٌ كُلُّهُمْ. لاَ يَعْرِفُونَ. كُلُّهُمْ كِلاَبٌ بُكْمٌ لاَ تَقْدُِرُ أَنْ تَنْبَحَ. حَالِمُونَ مُضْطَجِعُونَ مُحِبُّو النَّوْمِ.
\par 11 وَالْكِلاَبُ شَرِهَةٌ لاَ تَعْرِفُ الشَّبَعَ. وَهُمْ رُعَاةٌ لاَ يَعْرِفُونَ الْفَهْمَ. الْتَفَتُوا جَمِيعاً إِلَى طُرُقِهِمْ كُلُّ وَاحِدٍ إِلَى الرِّبْحِ عَنْ أَقْصَى.
\par 12 هَلُمُّوا آخُذُ خَمْراً وَلْنَشْتَفَّ مُسْكِراً وَيَكُونُ الْغَدُ كَهَذَا الْيَوْمِ عَظِيماً بَلْ أَزْيَدَ جِدّاً.

\chapter{57}

\par 1 بَادَ الصِّدِّيقُ وَلَيْسَ أَحَدٌ يَضَعُ ذَلِكَ فِي قَلْبِهِ. وَرِجَالُ الإِحْسَانِ يُضَمُّونَ وَلَيْسَ مَنْ يَفْطِنُ بِأَنَّهُ مِنْ وَجْهِ الشَّرِّ يُضَمُّ الصِّدِّيقُ.
\par 2 يَدْخُلُ السَّلاَمَ. يَسْتَرِيحُونَ فِي مَضَاجِعِهِمِ. السَّالِكُ بِالاِسْتِقَامَةِ.
\par 3 أَمَّا أَنْتُمْ فَتَقَدَّمُوا إِلَى هُنَا يَا بَنِي السَّاحِرَةِ نَسْلَ الْفَاسِقِ وَالزَّانِيَةِ.
\par 4 بِمَنْ تَسْخَرُونَ وَعَلَى مَنْ تَفْغَرُونَ الْفَمَ وَتَدْلَعُونَ اللِّسَانَ؟ أَمَا أَنْتُمْ أَوْلاَدُ الْمَعْصِيَةِ نَسْلُ الْكَذِبِ؟
\par 5 الْمُتَوَقِّدُونَ إِلَى الأَصْنَامِ تَحْتَ كُلِّ شَجَرَةٍ خَضْرَاءَ الْقَاتِلُونَ الأَوْلاَدَ فِي الأَوْدِيَةِ تَحْتَ شُقُوقِ الْمَعَاقِلِ.
\par 6 فِي حِجَارَةِ الْوَادِي الْمُلْسِ نَصِيبُكِ. تِلْكَ هِيَ قُرْعَتُكِ. لِتِلْكَ سَكَبْتِ سَكِيباً وَأَصْعَدْتِ تَقْدِمَةً. أَعَنْ هَذِهِ أَتَعَزَّى؟
\par 7 عَلَى جَبَلٍ عَالٍ وَمُرْتَفِعٍ وَضَعْتِ مَضْجَعَكِ وَإِلَى هُنَاكَ صَعِدْتِ لِتَذْبَحِي ذَبِيحَةً.
\par 8 وَرَاءَ الْبَابِ وَالْقَائِمَةِ وَضَعْتِ تِذْكَارَكِ لأَنَّكِ لِغَيْرِي كَشَفْتِ وَصَعِدْتِ. أَوْسَعْتِ مَضْجَعَكِ وَقَطَعْتِ لِنَفْسِكِ عَهْداً مَعَهُمْ. أَحْبَبْتِ مَضْجَعَهُمْ. نَظَرْتِ فُرْصَةً.
\par 9 وَسِرْتِ إِلَى الْمَلِكِ بِالدُّهْنِ وَأَكْثَرْتِ أَطْيَابَكِ وَأَرْسَلْتِ رُسُلَكِ إِلَى بُعْدٍ وَنَزَلْتِ حَتَّى إِلَى الْهَاوِيَةِ.
\par 10 بِطُولِ أَسْفَارِكِ أَعْيَيْتِ وَلَمْ تَقُولِي: «يَئِسْتُ». شَهْوَتَكِ وَجَدْتِ لِذَلِكَ لَمْ تَضْعُفِي.
\par 11 وَمِمَّنْ خَشِيتِ وَخِفْتِ حَتَّى خُنْتِ وَإِيَّايَ لَمْ تَذْكُرِي وَلاَ وَضَعْتِ فِي قَلْبِكِ؟ أَمَّا أَنَا سَاكِتٌ وَذَلِكَ مُنْذُ الْقَدِيمِ فَإِيَّايَ لَمْ تَخَافِي.
\par 12 أَنَا أُخْبِرُ بِبِرِّكِ وَبِأَعْمَالِكِ فَلاَ تُفِيدُكِ.
\par 13 إِذْ تَصْرُخِينَ فَلْيُنْقِذْكِ جُمُوعُكِ. وَلَكِنِ الرِّيحُ تَحْمِلُهُمْ كُلَّهُمْ. تَأْخُذُهُمْ نَفَخَةٌ. أَمَّا الْمُتَوَكِّلُ عَلَيَّ فَيَمْلِكُ الأَرْضَ وَيَرِثُ جَبَلَ قُدْسِي
\par 14 وَيَقُولُ: «أَعِدُّوا. أَعِدُّوا. هَيِّئُوا الطَّرِيقَ. ارْفَعُوا الْمَعْثَرَةَ مِنْ طَرِيقِ شَعْبِي».
\par 15 لأَنَّهُ هَكَذَا قَالَ الْعَلِيُّ الْمُرْتَفِعُ سَاكِنُ الأَبَدِ الْقُدُّوسُ اسْمُهُ: «فِي الْمَوْضِعِ الْمُرْتَفِعِ الْمُقَدَّسِ أَسْكُنُ وَمَعَ الْمُنْسَحِقِ وَالْمُتَوَاضِعِ الرُّوحِ لأُحْيِيَ رُوحَ الْمُتَوَاضِعِينَ وَلأُحْيِيَ قَلْبَ الْمُنْسَحِقِينَ.
\par 16 لأَنِّي لاَ أُخَاصِمُ إِلَى الأَبَدِ وَلاَ أَغْضَبُ إِلَى الدَّهْرِ. لأَنَّ الرُّوحَ يُغْشَى عَلَيْهَا أَمَامِي وَالنَّسَمَاتُ الَّتِي صَنَعْتُهَا.
\par 17 مِنْ أَجْلِ إِثْمِ مَكْسَبِهِ غَضِبْتُ وَضَرَبْتُهُ. اسْتَتَرْتُ وَغَضِبْتُ فَذَهَبَ عَاصِياً فِي طَرِيقِ قَلْبِهِ.
\par 18 رَأَيْتُ طُرُقَهُ وَسَأَشْفِيهِ وَأَقُودُهُ وَأَرُدُّ تَعْزِيَاتٍ لَهُ وَلِنَائِحِيهِ
\par 19 خَالِقاً ثَمَرَ الشَّفَتَيْنِ. «سَلاَمٌ سَلاَمٌ لِلْبَعِيدِ وَلِلْقَرِيبِ» قَالَ الرَّبُّ «وَسَأَشْفِيهِ».
\par 20 أَمَّا الأَشْرَارُ فَكَالْبَحْرِ الْمُضْطَرِبِ لأَنَّهُ لاَ يَسْتَطِيعُ أَنْ يَهْدَأَ وَتَقْذِفُ مِيَاهُهُ حَمْأَةً وَطِيناً.
\par 21 لَيْسَ سَلاَمٌ قَالَ إِلَهِي لِلأَشْرَارِ.

\chapter{58}

\par 1 نَادِ بِصَوْتٍ عَالٍ. لاَ تُمْسِكْ. ارْفَعْ صَوْتَكَ كَبُوقٍ وَأَخْبِرْ شَعْبِي بِتَعَدِّيهِمْ وَبَيْتَ يَعْقُوبَ بِخَطَايَاهُمْ.
\par 2 وَإِيَّايَ يَطْلُبُونَ يَوْماً فَيَوْماً وَيُسَرُّونَ بِمَعْرِفَةِ طُرُقِي كَأُمَّةٍ عَمِلَتْ بِرّاً وَلَمْ تَتْرُكْ قَضَاءَ إِلَهِهَا. يَسْأَلُونَنِي عَنْ أَحْكَامِ الْبِرِّ. يُسَرُّونَ بِالتَّقَرُّبِ إِلَى اللَّهِ.
\par 3 يَقُولُونَ: «لِمَاذَا صُمْنَا وَلَمْ تَنْظُرْ ذَلَّلْنَا أَنْفُسَنَا وَلَمْ تُلاَحِظْ؟» هَا إِنَّكُمْ فِي يَوْمِ صَوْمِكُمْ تُوجِدُونَ مَسَرَّةً وَبِكُلِّ أَشْغَالِكُمْ تُسَخِّرُونَ.
\par 4 هَا إِنَّكُمْ لِلْخُصُومَةِ وَالنِّزَاعِ تَصُومُونَ وَلِتَضْرِبُوا بِلَكْمَةِ الشَّرِّ. لَسْتُمْ تَصُومُونَ كَمَا الْيَوْمَ لِتَسْمِيعِ صَوْتِكُمْ فِي الْعَلاَءِ.
\par 5 أَمِثْلُ هَذَا يَكُونُ صَوْمٌ أَخْتَارُهُ؟ يَوْماً يُذَلِّلُ الإِنْسَانُ فِيهِ نَفْسَهُ يُحْنِي كَالأَسَلَةِ رَأْسَهُ وَيَفْرِشُ تَحْتَهُ مِسْحاً وَرَمَاداً. هَلْ تُسَمِّي هَذَا صَوْماً وَيَوْماً مَقْبُولاً لِلرَّبِّ؟
\par 6 أَلَيْسَ هَذَا صَوْماً أَخْتَارُهُ: حَلَّ قُيُودِ الشَّرِّ. فَكَّ عُقَدِ النِّيرِ وَإِطْلاَقَ الْمَسْحُوقِينَ أَحْرَاراً وَقَطْعَ كُلِّ نِيرٍ.
\par 7 أَلَيْسَ أَنْ تَكْسِرَ لِلْجَائِعِ خُبْزَكَ وَأَنْ تُدْخِلَ الْمَسَاكِينَ التَّائِهِينَ إِلَى بَيْتِكَ؟ إِذَا رَأَيْتَ عُرْيَاناً أَنْ تَكْسُوهُ وَأَنْ لاَ تَتَغَاضَى عَنْ لَحْمِكَ.
\par 8 حِينَئِذٍ يَنْفَجِرُ مِثْلَ الصُّبْحِ نُورُكَ وَتَنْبُتُ صِحَّتُكَ سَرِيعاً وَيَسِيرُ بِرُّكَ أَمَامَكَ وَمَجْدُ الرَّبِّ يَجْمَعُ سَاقَتَكَ.
\par 9 حِينَئِذٍ تَدْعُو فَيُجِيبُ الرَّبُّ. تَسْتَغِيثُ فَيَقُولُ: «هَئَنَذَا». إِنْ نَزَعْتَ مِنْ وَسَطِكَ النِّيرَ وَالإِيمَاءَ بِالإِصْبِعِ وَكَلاَمَ الإِثْمِ
\par 10 وَأَنْفَقْتَ نَفْسَكَ لِلْجَائِعِ وَأَشْبَعْتَ النَّفْسَ الذَّلِيلَةَ يُشْرِقُ فِي الظُّلْمَةِ نُورُكَ وَيَكُونُ ظَلاَمُكَ الدَّامِسُ مِثْلَ الظُّهْرِ
\par 11 وَيَقُودُكَ الرَّبُّ عَلَى الدَّوَامِ وَيُشْبِعُ فِي الْجَدُوبِ نَفْسَكَ وَيُنَشِّطُ عِظَامَكَ فَتَصِيرُ كَجَنَّةٍ رَيَّا وَكَنَبْعِ مِيَاهٍ لاَ تَنْقَطِعُ مِيَاهُهُ.
\par 12 وَمِنْكَ تُبْنَى الْخِرَبُ الْقَدِيمَةُ. تُقِيمُ أَسَاسَاتِ دَوْرٍ فَدَوْرٍ فَيُسَمُّونَكَ «مُرَمِّمَ الثُّغْرَةِ مُرْجِعَ الْمَسَالِكِ لِلسُّكْنَى».
\par 13 إِنْ رَدَدْتَ عَنِ السَّبْتِ رِجْلَكَ عَنْ عَمَلِ مَسَرَّتِكَ يَوْمَ قُدْسِي وَدَعَوْتَ السَّبْتَ لَذَّةً وَمُقَدَّسَ الرَّبِّ مُكَرَّماً وَأَكْرَمْتَهُ عَنْ عَمَلِ طُرُقِكَ وَعَنْ إِيجَادِ مَسَرَّتِكَ وَالتَّكَلُّمِ بِكَلاَمِكَ
\par 14 فَإِنَّكَ حِينَئِذٍ تَتَلَذَّذُ بِالرَّبِّ وَأُرَكِّبُكَ عَلَى مُرْتَفَعَاتِ الأَرْضِ وَأُطْعِمُكَ مِيرَاثَ يَعْقُوبَ أَبِيكَ لأَنَّ فَمَ الرَّبِّ تَكَلَّمَ.

\chapter{59}

\par 1 هَا إِنَّ يَدَ الرَّبِّ لَمْ تَقْصُرْ عَنْ أَنْ تُخَلِّصَ وَلَمْ تَثْقَلْ أُذُنُهُ عَنْ أَنْ تَسْمَعَ.
\par 2 بَلْ آثَامُكُمْ صَارَتْ فَاصِلَةً بَيْنَكُمْ وَبَيْنَ إِلَهِكُمْ وَخَطَايَاكُمْ سَتَرَتْ وَجْهَهُ عَنْكُمْ حَتَّى لاَ يَسْمَعَ.
\par 3 لأَنَّ أَيْدِيكُمْ قَدْ تَنَجَّسَتْ بِالدَّمِ وَأَصَابِعَكُمْ بِالإِثْمِ. شِفَاهُكُمْ تَكَلَّمَتْ بِالْكَذِبِ وَلِسَانُكُمْ يَلْهَجُ بِالشَّرِّ.
\par 4 لَيْسَ مَنْ يَدْعُو بِالْعَدْلِ وَلَيْسَ مَنْ يُحَاكِمُ بِالْحَقِّ. يَتَّكِلُونَ عَلَى الْبَاطِلِ وَيَتَكَلَّمُونَ بِالْكَذِبِ. قَدْ حَبِلُوا بِتَعَبٍ وَوَلَدُوا إِثْماً.
\par 5 فَقَسُوا بَيْضَ أَفْعَى وَنَسَجُوا خُيُوطَ الْعَنْكَبُوتِ. الآكِلُ مِنْ بَيْضِهِمْ يَمُوتُ وَالَّتِي تُكْسَرُ تُخْرِجُ أَفْعَى.
\par 6 خُيُوطُهُمْ لاَ تَصِيرُ ثَوْباً وَلاَ يَكْتَسُونَ بِأَعْمَالِهِمْ. أَعْمَالُهُمْ أَعْمَالُ إِثْمٍ وَفِعْلُ الظُّلْمِ فِي أَيْدِيهِمْ.
\par 7 أَرْجُلُهُمْ إِلَى الشَّرِّ تَجْرِي وَتُسْرِعُ إِلَى سَفْكِ الدَّمِ الزَّكِيِّ. أَفْكَارُهُمْ أَفْكَارُ إِثْمٍ. فِي طُرُقِهِمِ اغْتِصَابٌ وَسَحْقٌ.
\par 8 طَرِيقُ السَّلاَمِ لَمْ يَعْرِفُوهُ وَلَيْسَ فِي مَسَالِكِهِمْ عَدْلٌ. جَعَلُوا لأَنْفُسِهِمْ سُبُلاً مُعَوَّجَةً. كُلُّ مَنْ يَسِيرُ فِيهَا لاَ يَعْرِفُ سَلاَماً.
\par 9 مِنْ أَجْلِ ذَلِكَ ابْتَعَدَ الْحَقُّ عَنَّا وَلَمْ يُدْرِكْنَا الْعَدْلُ. نَنْتَظِرُ نُوراً فَإِذَا ظَلاَمٌ. ضِيَاءً فَنَسِيرُ فِي ظَلاَمٍ دَامِسٍ.
\par 10 نَتَلَمَّسُ الْحَائِطَ كَعُمْيٍ وَكَالَّذِي بِلاَ أَعْيُنٍ نَتَجَسَّسُ. قَدْ عَثَرْنَا فِي الظُّهْرِ كَمَا فِي الْعَتَمَةِ فِي الضَّبَابِ كَمَوْتَى.
\par 11 نَزْأَرُ كُلُّنَا كَدُبَّةٍ وَكَحَمَامٍ هَدْراً نَهْدِرُ. نَنْتَظِرُ عَدْلاً وَلَيْسَ هُوَ وَخَلاَصاً فَيَبْتَعِدُ عَنَّا.
\par 12 لأَنَّ مَعَاصِيَنَا كَثُرَتْ أَمَامَكَ وَخَطَايَانَا تَشْهَدُ عَلَيْنَا لأَنَّ مَعَاصِيَنَا مَعَنَا وَآثَامَنَا نَعْرِفُهَا.
\par 13 تَعَدَّيْنَا وَكَذِبْنَا عَلَى الرَّبِّ وَحِدْنَا مِنْ وَرَاءِ إِلَهِنَا. تَكَلَّمْنَا بِالظُّلْمِ وَالْمَعْصِيَةِ. حَبِلْنَا وَلَهَجْنَا مِنَ الْقَلْبِ بِكَلاَمِ الْكَذِبِ.
\par 14 وَقَدِ ارْتَدَّ الْحَقُّ إِلَى الْوَرَاءِ وَالْعَدْلُ يَقِفُ بَعِيداً. لأَنَّ الصِّدْقَ سَقَطَ فِي الشَّارِعِ وَالاِسْتِقَامَةَ لاَ تَسْتَطِيعُ الدُّخُولَ.
\par 15 وَصَارَ الصِّدْقُ مَعْدُوماً وَالْحَائِدُ عَنِ الشَّرِّ يُسْلَبُ. فَرَأَى الرَّبُّ وَسَاءَ فِي عَيْنَيْهِ أَنَّهُ لَيْسَ عَدْلٌ.
\par 16 فَرَأَى أَنَّهُ لَيْسَ إِنْسَانٌ وَتَحَيَّرَ مِنْ أَنَّهُ لَيْسَ شَفِيعٌ. فَخَلَّصَتْ ذِرَاعُهُ لِنَفْسِهِ وَبِرُّهُ هُوَ عَضَدَهُ.
\par 17 فَلَبِسَ الْبِرَّ كَدِرْعٍ وَخُوذَةَ الْخَلاَصِ عَلَى رَأْسِهِ. وَلَبِسَ ثِيَابَ الاِنْتِقَامِ كَلِبَاسٍ وَاكْتَسَى بِالْغَيْرَةِ كَرِدَاءٍ.
\par 18 حَسَبَ الأَعْمَالِ هَكَذَا يُجَازِي مُبْغِضِيهِ سَخَطاً وَأَعْدَاءَهُ عِقَاباً. جَزَاءً يُجَازِي الْجَزَائِرَ.
\par 19 فَيَخَافُونَ مِنَ الْمَغْرِبِ اسْمَ الرَّبِّ وَمِنْ مَشْرِقِ الشَّمْسِ مَجْدَهُ. عِنْدَمَا يَأْتِي الْعَدُوُّ كَنَهْرٍ فَنَفْخَةُ الرَّبِّ تَدْفَعُهُ!
\par 20 وَيَأْتِي الْفَادِي إِلَى صِهْيَوْنَ وَإِلَى التَّائِبِينَ عَنِ الْمَعْصِيَةِ فِي يَعْقُوبَ يَقُولُ الرَّبُّ.
\par 21 أَمَّا أَنَا فَهَذَا عَهْدِي مَعَهُمْ قَالَ الرَّبُّ: «رُوحِي الَّذِي عَلَيْكَ وَكَلاَمِي الَّذِي وَضَعْتُهُ فِي فَمِكَ لاَ يَزُولُ مِنْ فَمِكَ وَلاَ مِنْ فَمِ نَسْلِكَ وَلاَ مِنْ فَمِ نَسْلِ نَسْلِكَ» قَالَ الرَّبُّ «مِنَ الآنَ وَإِلَى الأَبَدِ».

\chapter{60}

\par 1 قُومِي اسْتَنِيرِي لأَنَّهُ قَدْ جَاءَ نُورُكِ وَمَجْدُ الرَّبِّ أَشْرَقَ عَلَيْكِ.
\par 2 لأَنَّهُ هَا هِيَ الظُّلْمَةُ تُغَطِّي الأَرْضَ وَالظَّلاَمُ الدَّامِسُ الأُمَمَ. أَمَّا عَلَيْكِ فَيُشْرِقُ الرَّبُّ وَمَجْدُهُ عَلَيْكِ يُرَى.
\par 3 فَتَسِيرُ الأُمَمُ فِي نُورِكِ وَالْمُلُوكُ فِي ضِيَاءِ إِشْرَاقِكِ.
\par 4 اِرْفَعِي عَيْنَيْكِ حَوَالَيْكِ وَانْظُرِي. قَدِ اجْتَمَعُوا كُلُّهُمْ. جَاءُوا إِلَيْكِ. يَأْتِي بَنُوكِ مِنْ بَعِيدٍ وَتُحْمَلُ بَنَاتُكِ عَلَى الأَيْدِي.
\par 5 حِينَئِذٍ تَنْظُرِينَ وَتُنِيرِينَ وَيَخْفُقُ قَلْبُكِ وَيَتَّسِعُ لأَنَّهُ تَتَحَوَّلُ إِلَيْكِ ثَرْوَةُ الْبَحْرِ وَيَأْتِي إِلَيْكِ غِنَى الأُمَمِ.
\par 6 تُغَطِّيكِ كَثْرَةُ الْجِمَالِ بُكْرَانُ مِدْيَانَ وَعِيفَةَ كُلُّهَا تَأْتِي مِنْ شَبَا. تَحْمِلُ ذَهَباً وَلُبَاناً وَتُبَشِّرُ بِتَسَابِيحِ الرَّبِّ.
\par 7 كُلُّ غَنَمِ قِيدَارَ تَجْتَمِعُ إِلَيْكِ. كِبَاشُ نَبَايُوتَ تَخْدِمُكِ. تَصْعَدُ مَقْبُولَةً عَلَى مَذْبَحِي وَأُزَيِّنُ بَيْتَ جَمَالِي.
\par 8 مَنْ هَؤُلاَءِ الطَّائِرُونَ كَسَحَابٍ وَكَالْحَمَامِ إِلَى بُيُوتِهَا؟
\par 9 إِنَّ الْجَزَائِرَ تَنْتَظِرُنِي وَسُفُنَ تَرْشِيشَ فِي الأَوَّلِ لِتَأْتِيَ بِبَنِيكِ مِنْ بَعِيدٍ وَفِضَّتُهُمْ وَذَهَبُهُمْ مَعَهُمْ لاِسْمِ الرَّبِّ إِلَهِكِ وَقُدُّوسِ إِسْرَائِيلَ لأَنَّهُ قَدْ مَجَّدَكِ.
\par 10 وَبَنُو الْغَرِيبِ يَبْنُونَ أَسْوَارَكِ وَمُلُوكُهُمْ يَخْدِمُونَكِ. لأَنِّي بِغَضَبِي ضَرَبْتُكِ وَبِرِضْوَانِي رَحِمْتُكِ.
\par 11 وَتَنْفَتِحُ أَبْوَابُكِ دَائِماً. نَهَاراً وَلَيْلاً لاَ تُغْلَقُ. لِيُؤْتَى إِلَيْكِ بِغِنَى الأُمَمِ وَتُقَادَ مُلُوكُهُمْ.
\par 12 لأَنَّ الأُمَّةَ وَالْمَمْلَكَةَ الَّتِي لاَ تَخْدِمُكِ تَبِيدُ وَخَرَاباً تُخْرَبُ الأُمَمُ.
\par 13 مَجْدُ لُبْنَانَ إِلَيْكِ يَأْتِي. السَّرْوُ وَالسِّنْدِيَانُ وَالشَّرْبِينُ مَعاً لِزِينَةِ مَكَانِ مَقْدِسِي وَأُمَجِّدُ مَوْضِعَ رِجْلَيَّ.
\par 14 وَبَنُو الَّذِينَ قَهَرُوكِ يَسِيرُونَ إِلَيْكِ خَاضِعِينَ وَكُلُّ الَّذِينَ أَهَانُوكِ يَسْجُدُونَ لَدَى بَاطِنِ قَدَمَيْكِ وَيَدْعُونَكِ «مَدِينَةَ الرَّبِّ» «صِهْيَوْنَ قُدُّوسِ إِسْرَائِيلَ».
\par 15 عِوَضاً عَنْ كَوْنِكِ مَهْجُورَةً وَمُبْغَضَةً بِلاَ عَابِرٍ بِكِ أَجْعَلُكِ فَخْراً أَبَدِيّاً فَرَحَ دَوْرٍ فَدَوْرٍ.
\par 16 وَتَرْضَعِينَ لَبَنَ الأُمَمِ وَتَرْضَعِينَ ثُدِيَّ مُلُوكٍ وَتَعْرِفِينَ أَنِّي أَنَا الرَّبُّ مُخَلِّصُكِ وَوَلِيُّكِ عَزِيزُ يَعْقُوبَ.
\par 17 عِوَضاً عَنِ النُّحَاسِ آتِي بِالذَّهَبِ وَعِوَضاً عَنِ الْحَدِيدِ آتِي بِالْفِضَّةِ وَعِوَضاً عَنِ الْخَشَبِ بِالنُّحَاسِ وَعِوَضاً عَنِ الْحِجَارَةِ بِالْحَدِيدِ وَأَجْعَلُ وُكَلاَءَكِ سَلاَماً وَوُلاَتَكِ بِرّاً.
\par 18 لاَ يُسْمَعُ بَعْدُ ظُلْمٌ فِي أَرْضِكِ وَلاَ خَرَابٌ أَوْ سَحْقٌ فِي تُخُومِكِ بَلْ تُسَمِّينَ أَسْوَارَكِ «خَلاَصاً» وَأَبْوَابَكِ «تَسْبِيحاً».
\par 19 لاَ تَكُونُ لَكِ بَعْدُ الشَّمْسُ نُوراً فِي النَّهَارِ وَلاَ الْقَمَرُ يُنِيرُ لَكِ مُضِيئاً بَلِ الرَّبُّ يَكُونُ لَكِ نُوراً أَبَدِيّاً وَإِلَهُكِ زِينَتَكِ.
\par 20 لاَ تَغِيبُ بَعْدُ شَمْسُكِ وَقَمَرُكِ لاَ يَنْقُصُ لأَنَّ الرَّبَّ يَكُونُ لَكِ نُوراً أَبَدِيّاً وَتُكْمَلُ أَيَّامُ نَوْحِكِ.
\par 21 وَشَعْبُكِ كُلُّهُمْ أَبْرَارٌ. إِلَى الأَبَدِ يَرِثُونَ الأَرْضَ غُصْنُ غَرْسِي عَمَلُ يَدَيَّ لأَتَمَجَّدَ.
\par 22 اَلصَّغِيرُ يَصِيرُ أَلْفاً وَالْحَقِيرُ أُمَّةً قَوِيَّةً. أَنَا الرَّبُّ فِي وَقْتِهِ أُسْرِعُ بِهِ.

\chapter{61}

\par 1 رُوحُ السَّيِّدِ الرَّبِّ عَلَيَّ لأَنَّ الرَّبَّ مَسَحَنِي لأُبَشِّرَ الْمَسَاكِينَ أَرْسَلَنِي لأَعْصِبَ مُنْكَسِرِي الْقَلْبِ لأُنَادِيَ لِلْمَسْبِيِّينَ بِالْعِتْقِ وَلِلْمَأْسُورِينَ بِالإِطْلاَقِ.
\par 2 لأُنَادِيَ بِسَنَةٍ مَقْبُولَةٍ لِلرَّبِّ وَبِيَوْمِ انْتِقَامٍ لإِلَهِنَا. لأُعَزِّيَ كُلَّ النَّائِحِينَ.
\par 3 لأَجْعَلَ لِنَائِحِي صِهْيَوْنَ لأُعْطِيَهُمْ جَمَالاً عِوَضاً عَنِ الرَّمَادِ وَدُهْنَ فَرَحٍ عِوَضاً عَنِ النَّوْحِ وَرِدَاءَ تَسْبِيحٍ عِوَضاً عَنِ الرُّوحِ الْيَائِسَةِ فَيُدْعَوْنَ أَشْجَارَ الْبِرِّ غَرْسَ الرَّبِّ لِلتَّمْجِيدِ.
\par 4 وَيَبْنُونَ الْخِرَبَ الْقَدِيمَةَ. يُقِيمُونَ الْمُوحِشَاتِ الأُوَلَ. وَيُجَدِّدُونَ الْمُدُنَ الْخَرِبَةَ مُوحِشَاتِ دَوْرٍ فَدَوْرٍ.
\par 5 وَيَقِفُ الأَجَانِبُ وَيَرْعُونَ غَنَمَكُمْ وَيَكُونُ بَنُو الْغَرِيبِ حَرَّاثِيكُمْ وَكَرَّامِيكُمْ.
\par 6 أَمَّا أَنْتُمْ فَتُدْعَوْنَ كَهَنَةَ الرَّبِّ تُسَمُّونَ خُدَّامَ إِلَهِنَا. تَأْكُلُونَ ثَرْوَةَ الأُمَمِ وَعَلَى مَجْدِهِمْ تَتَأَمَّرُونَ.
\par 7 عِوَضاً عَنْ خِزْيِكُمْ ضِعْفَانِ وَعِوَضاً عَنِ الْخَجَلِ يَبْتَهِجُونَ بِنَصِيبِهِمْ. لِذَلِكَ يَرِثُونَ فِي أَرْضِهِمْ ضِعْفَيْنِ. بَهْجَةٌ أَبَدِيَّةٌ تَكُونُ لَهُمْ.
\par 8 لأَنِّي أَنَا الرَّبُّ مُحِبُّ الْعَدْلِ مُبْغِضُ الْمُخْتَلِسِ بِالظُّلْمِ. وَأَجْعَلُ أُجْرَتَهُمْ أَمِينَةً وَأَقْطَعُ لَهُمْ عَهْداً أَبَدِيّاً.
\par 9 وَيُعْرَفُ بَيْنَ الأُمَمِ نَسْلُهُمْ وَذُرِّيَّتُهُمْ فِي وَسَطِ الشُّعُوبِ. كُلُّ الَّذِينَ يَرُونَهُمْ يَعْرِفُونَهُمْ أَنَّهُمْ نَسْلٌ بَارَكَهُ الرَّبُّ.
\par 10 فَرَحاً أَفْرَحُ بِالرَّبِّ. تَبْتَهِجُ نَفْسِي بِإِلَهِي لأَنَّهُ قَدْ أَلْبَسَنِي ثِيَابَ الْخَلاَصِ. كَسَانِي رِدَاءَ الْبِرِّ مِثْلَ عَرِيسٍ يَتَزَيَّنُ بِعِمَامَةٍ وَمِثْلَ عَرُوسٍ تَتَزَيَّنُ بِحُلِيِّهَا.
\par 11 لأَنَّهُ كَمَا أَنَّ الأَرْضَ تُخْرِجُ نَبَاتَهَا وَكَمَا أَنَّ الْجَنَّةَ تُنْبِتُ مَزْرُوعَاتِهَا هَكَذَا السَّيِّدُ الرَّبُّ يُنْبِتُ بِرّاً وَتَسْبِيحاً أَمَامَ كُلِّ الأُمَمِ.

\chapter{62}

\par 1 مِنْ أَجْلِ صِهْيَوْنَ لاَ أَسْكُتُ وَمِنْ أَجْلِ أُورُشَلِيمَ لاَ أَهْدَأُ حَتَّى يَخْرُجَ بِرُّهَا كَضِيَاءٍ وَخَلاَصُهَا كَمِصْبَاحٍ يَتَّقِدُ.
\par 2 فَتَرَى الأُمَمُ بِرَّكِ وَكُلُّ الْمُلُوكِ مَجْدَكِ وَتُسَمَّيْنَ بِاسْمٍ جَدِيدٍ يُعَيِّنُهُ فَمُ الرَّبِّ.
\par 3 وَتَكُونِينَ إِكْلِيلَ جَمَالٍ بِيَدِ الرَّبِّ وَتَاجاً مَلَكِيّاً بِكَفِّ إِلَهِكِ.
\par 4 لاَ يُقَالُ بَعْدُ لَكِ «مَهْجُورَةٌ» وَلاَ يُقَالُ بَعْدُ لأَرْضِكِ «مُوحَشَةٌ» بَلْ تُدْعَيْنَ «حَفْصِيبَةَ» وَأَرْضُكِ تُدْعَى «بَعُولَةَ». لأَنَّ الرَّبَّ يُسَرُّ بِكِ وَأَرْضُكِ تَصِيرُ ذَاتِ بَعْلٍ.
\par 5 لأَنَّهُ كَمَا يَتَزَوَّجُ الشَّابُّ عَذْرَاءَ يَتَزَوَّجُكِ بَنُوكِ. وَكَفَرَحِ الْعَرِيسِ بِالْعَرُوسِ يَفْرَحُ بِكِ إِلَهُكِ.
\par 6 عَلَى أَسْوَارِكِ يَا أُورُشَلِيمُ أَقَمْتُ حُرَّاساً لاَ يَسْكُتُونَ كُلَّ النَّهَارِ وَكُلَّ اللَّيْلِ عَلَى الدَّوَامِ. يَا ذَاكِرِي الرَّبِّ لاَ تَسْكُتُوا
\par 7 وَلاَ تَدَعُوهُ يَسْكُتُ حَتَّى يُثَبِّتَ وَيَجْعَلَ أُورُشَلِيمَ تَسْبِيحَةً فِي الأَرْضِ.
\par 8 حَلَفَ الرَّبُّ بِيَمِينِهِ وَبِذِرَاعِ عِزَّتِهِ قَائِلاً: «إِنِّي لاَ أَدْفَعُ بَعْدُ قَمْحَكِ مَأْكَلاً لأَعْدَائِكِ وَلاَ يَشْرَبُ بَنُو الْغُرَبَاءِ خَمْرَكِ الَّتِي تَعِبْتِ فِيهَا.
\par 9 بَلْ يَأْكُلُهُ الَّذِينَ جَنُوهُ وَيُسَبِّحُونَ الرَّبَّ وَيَشْرَبُهُ جَامِعُوهُ فِي دِيَارِ قُدْسِي».
\par 10 اُعْبُرُوا اعْبُرُوا بِالأَبْوَابِ. هَيِّئُوا طَرِيقَ الشَّعْبِ. أَعِدُّوا أَعِدُّوا السَّبِيلَ. نَقُّوهُ مِنَ الْحِجَارَةِ. ارْفَعُوا الرَّايَةَ لِلشَّعْبِ.
\par 11 هُوَذَا الرَّبُّ قَدْ أَخْبَرَ إِلَى أَقْصَى الأَرْضِ قُولُوا لاِبْنَةِ صِهْيَوْنَ: «هُوَذَا مُخَلِّصُكِ آتٍ. هَا أُجْرَتُهُ مَعَهُ وَجِزَاؤُهُ أَمَامَهُ».
\par 12 وَيُسَمُّونَهُمْ «شَعْباً مُقَدَّساً» «مَفْدِيِّي الرَّبِّ». وَأَنْتِ تُسَمَّيْنَ «الْمَطْلُوبَةَ» «الْمَدِينَةَ غَيْرَ الْمَهْجُورَةِ».

\chapter{63}

\par 1 مَنْ ذَا الآتِي مِنْ أَدُومَ بِثِيَابٍ حُمْرٍ مِنْ بُصْرَةَ؟ هَذَا الْبَهِيُّ بِمَلاَبِسِهِ. الْمُتَعَظِّمُ بِكَثْرَةِ قُوَّتِهِ. «أَنَا الْمُتَكَلِّمُ بِالْبِرِّ الْعَظِيمُ لِلْخَلاَصِ».
\par 2 مَا بَالُ لِبَاسِكَ مُحَمَّرٌ وَثِيَابُكَ كَدَائِسِ الْمِعْصَرَةِ؟
\par 3 «قَدْ دُسْتُ الْمِعْصَرَةَ وَحْدِي وَمِنَ الشُّعُوبِ لَمْ يَكُنْ مَعِي أَحَدٌ. فَدُسْتُهُمْ بِغَضَبِي وَوَطِئْتُهُمْ بِغَيْظِي. فَرُشَّ عَصِيرُهُمْ عَلَى ثِيَابِي فَلَطَخْتُ كُلَّ مَلاَبِسِي.
\par 4 لأَنَّ يَوْمَ النَّقْمَةِ فِي قَلْبِي وَسَنَةَ مَفْدِيِّيَّ قَدْ أَتَتْ.
\par 5 فَنَظَرْتُ وَلَمْ يَكُنْ مُعِينٌ وَتَحَيَّرْتُ إِذْ لَمْ يَكُنْ عَاضِدٌ فَخَلَّصَتْ لِي ذِرَاعِي وَغَيْظِي عَضَدَنِي.
\par 6 فَدُسْتُ شُعُوباً بِغَضَبِي وَأَسْكَرْتُهُمْ بِغَيْظِي وَأَجْرَيْتُ عَلَى الأَرْضِ عَصِيرَهُمْ».
\par 7 إِحْسَانَاتِ الرَّبِّ أَذْكُرُ. تَسَابِيحَ الرَّبِّ. حَسَبَ كُلِّ مَا كَافَأَنَا بِهِ الرَّبُّ وَالْخَيْرَ الْعَظِيمَ لِبَيْتِ إِسْرَائِيلَ الَّذِي كَافَأَهُمْ بِهِ حَسَبَ مَرَاحِمِهِ وَحَسَبَ كَثْرَةِ إِحْسَانَاتِهِ.
\par 8 وَقَدْ قَالَ حَقّاً: «إِنَّهُمْ شَعْبِي بَنُونَ لاَ يَخُونُونَ». فَصَارَ لَهُمْ مُخَلِّصاً.
\par 9 فِي كُلِّ ضِيقِهِمْ تَضَايَقَ وَمَلاَكُ حَضْرَتِهِ خَلَّصَهُمْ. بِمَحَبَّتِهِ وَرَأْفَتِهِ هُوَ فَكَّهُمْ وَرَفَعَهُمْ وَحَمَلَهُمْ كُلَّ الأَيَّامِ الْقَدِيمَةِ.
\par 10 وَلَكِنَّهُمْ تَمَرَّدُوا وَأَحْزَنُوا رُوحَ قُدْسِهِ فَتَحَوَّلَ لَهُمْ عَدُوّاً وَهُوَ حَارَبَهُمْ.
\par 11 ثُمَّ ذَكَرَ الأَيَّامَ الْقَدِيمَةَ: مُوسَى وَشَعْبَهُ. «أَيْنَ الَّذِي أَصْعَدَهُمْ مِنَ الْبَحْرِ مَعَ رَاعِي غَنَمِهِ؟ أَيْنَ الَّذِي جَعَلَ فِي وَسَطِهِمْ رُوحَ قُدْسِهِ
\par 12 الَّذِي سَيَّرَ لِيَمِينِ مُوسَى ذِرَاعَ مَجْدِهِ الَّذِي شَقَّ الْمِيَاهَ قُدَّامَهُمْ لِيَصْنَعَ لِنَفْسِهِ اسْماً أَبَدِيّاً
\par 13 الَّذِي سَيَّرَهُمْ فِي اللُّجَجِ كَفَرَسٍ فِي الْبَرِّيَّةِ فَلَمْ يَعْثُرُوا؟»
\par 14 كَبَهَائِمَ تَنْزِلُ إِلَى وَطَاءٍ رُوحُ الرَّبِّ أَرَاحَهُمْ. هَكَذَا قُدْتَ شَعْبَكَ لِتَصْنَعَ لِنَفْسِكَ اسْمَ مَجْدٍ.
\par 15 تَطَلَّعْ مِنَ السَّمَاوَاتِ وَانْظُرْ مِنْ مَسْكَنِ قُدْسِكَ وَمَجْدِكَ. أَيْنَ غَيْرَتُكَ وَجَبَرُوتُكَ؟ زَفِيرُ أَحْشَائِكَ وَمَرَاحِمُكَ نَحْوِي امْتَنَعَتْ.
\par 16 فَإِنَّكَ أَنْتَ أَبُونَا وَإِنْ لَمْ يَعْرِفْنَا إِبْرَاهِيمُ وَإِنْ لَمْ يَدْرِنَا إِسْرَائِيلُ. أَنْتَ يَا رَبُّ أَبُونَا وَلِيُّنَا مُنْذُ الأَبَدِ اسْمُكَ.
\par 17 لِمَاذَا أَضْلَلْتَنَا يَا رَبُّ عَنْ طُرُقِكَ قَسَّيْتَ قُلُوبَنَا عَنْ مَخَافَتِكَ؟ ارْجِعْ مِنْ أَجْلِ عَبِيدِكَ أَسْبَاطِ مِيرَاثِكَ.
\par 18 إِلَى قَلِيلٍ امْتَلَكَ شَعْبُ قُدْسِكَ. مُضَايِقُونَا دَاسُوا مَقْدِسَكَ.
\par 19 قَدْ كُنَّا مُنْذُ زَمَانٍ كَالَّذِينَ لَمْ تَحْكُمْ عَلَيْهِمْ وَلَمْ يُدْعَ عَلَيْهِمْ بِاسْمِكَ.

\chapter{64}

\par 1 لَيْتَكَ تَشُقُّ السَّمَاوَاتِ وَتَنْزِلُ! مِنْ حَضْرَتِكَ تَتَزَلْزَلُ الْجِبَالُ.
\par 2 كَمَا تُشْعِلُ النَّارُ الْهَشِيمَ وَتَجْعَلُ النَّارُ الْمِيَاهَ تَغْلِي لِتُعَرِّفَ أَعْدَاءَكَ اسْمَكَ لِتَرْتَعِدَ الأُمَمُ مِنْ حَضْرَتِكَ.
\par 3 حِينَ صَنَعْتَ مَخَاوِفَ لَمْ نَنْتَظِرْهَا نَزَلْتَ. تَزَلْزَلَتِ الْجِبَالُ مِنْ حَضْرَتِكَ.
\par 4 وَمُنْذُ الأَزَلِ لَمْ يَسْمَعُوا وَلَمْ يُصْغُوا. لَمْ تَرَ عَيْنٌ إِلَهاً غَيْرَكَ يَصْنَعُ لِمَنْ يَنْتَظِرُهُ.
\par 5 تُلاَقِي الْفَرِحَ الصَّانِعَ الْبِرَّ. الَّذِينَ يَذْكُرُونَكَ فِي طُرُقِكَ. هَا أَنْتَ سَخَطْتَ إِذْ أَخْطَأْنَا. هِيَ إِلَى الأَبَدِ فَنَخْلُصُ.
\par 6 وَقَدْ صِرْنَا كُلُّنَا كَنَجِسٍ وَكَثَوْبِ عِدَّةٍ كُلُّ أَعْمَالِ بِرِّنَا وَقَدْ ذَبُلْنَا كَوَرَقَةٍ وَآثَامُنَا كَرِيحٍ تَحْمِلُنَا.
\par 7 وَلَيْسَ مَنْ يَدْعُو بِاسْمِكَ أَوْ يَنْتَبِهُ لِيَتَمَسَّكَ بِكَ لأَنَّكَ حَجَبْتَ وَجْهَكَ عَنَّا وَأَذَبْتَنَا بِسَبَبِ آثَامِنَا.
\par 8 وَالآنَ يَا رَبُّ أَنْتَ أَبُونَا. نَحْنُ الطِّينُ وَأَنْتَ جَابِلُنَا وَكُلُّنَا عَمَلُ يَدَيْكَ.
\par 9 لاَ تَسْخَطْ كُلَّ السَّخَطِ يَا رَبُّ وَلاَ تَذْكُرِ الإِثْمَ إِلَى الأَبَدِ. هَا انْظُرْ. شَعْبُكَ كُلُّنَا.
\par 10 مُدُنُ قُدْسِكَ صَارَتْ بَرِّيَّةً. صِهْيَوْنُ صَارَتْ بَرِّيَّةً وَأُورُشَلِيمُ مُوحَشَةً.
\par 11 بَيْتُ قُدْسِنَا وَجَمَالِنَا حَيْثُ سَبَّحَكَ آبَاؤُنَا قَدْ صَارَ حَرِيقَ نَارٍ وَكُلُّ مُشْتَهَيَاتِنَا صَارَتْ خَرَاباً.
\par 12 أَلأَجْلِ هَذِهِ تَتَجَلَّدُ يَا رَبُّ؟ أَتَسْكُتُ وَتُذِلُّنَا كُلَّ الذِّلِّ؟

\chapter{65}

\par 1 أَصْغَيْتُ إِلَى الَّذِينَ لَمْ يَسْأَلُوا. وُجِدْتُ مِنَ الَّذِينَ لَمْ يَطْلُبُونِي. قُلْتُ: «هَئَنَذَا هَئَنَذَا» لأُمَّةٍ لَمْ تُسَمَّ بِاسْمِي.
\par 2 بَسَطْتُ يَدَيَّ طُولَ النَّهَارِ إِلَى شَعْبٍ مُتَمَرِّدٍ سَائِرٍ فِي طَرِيقٍ غَيْرِ صَالِحٍ وَرَاءَ أَفْكَارِهِ.
\par 3 شَعْبٍ يُغِيظُنِي بِوَجْهِي. دَائِماً يَذْبَحُ فِي الْجَنَّاتِ وَيُبَخِّرُ عَلَى الآجُرِّ.
\par 4 يَجْلِسُ فِي الْقُبُورِ وَيَبِيتُ فِي الْمَدَافِنِ. يَأْكُلُ لَحْمَ الْخِنْزِيرِ وَفِي آنِيَتِهِ مَرَقُ لُحُومٍ نَجِسَةٍ.
\par 5 يَقُولُ: «قِفْ عِنْدَكَ. لاَ تَدْنُ مِنِّي لأَنِّي أَقْدَسُ مِنْكَ». هَؤُلاَءِ دُخَانٌ فِي أَنْفِي. نَارٌ مُتَّقِدَةٌ كُلَّ النَّهَارِ.
\par 6 هَا قَدْ كُتِبَ أَمَامِي. لاَ أَسْكُتُ بَلْ أُجَازِي. أُجَازِي فِي حِضْنِهِمْ
\par 7 آثَامَكُمْ وَآثَامَ آبَائِكُمْ مَعاً قَالَ الرَّبُّ الَّذِينَ بَخَّرُوا عَلَى الْجِبَالِ وَعَيَّرُونِي عَلَى الآكَامِ فَأَكِيلُ عَمَلَهُمُ الأَوَّلَ فِي حِضْنِهِمْ.
\par 8 هَكَذَا قَالَ الرَّبُّ: «كَمَا أَنَّ السُّلاَفَ يُوجَدُ فِي الْعُنْقُودِ فَيَقُولُ قَائِلٌ: لاَ تُهْلِكْهُ لأَنَّ فِيهِ بَرَكَةً. هَكَذَا أَعْمَلُ لأَجْلِ عَبِيدِي حَتَّى لاَ أُهْلِكَ الْكُلَّ.
\par 9 بَلْ أُخْرِجُ مِنْ يَعْقُوبَ نَسْلاً وَمِنْ يَهُوذَا وَارِثاً لِجِبَالِي فَيَرِثُهَا مُخْتَارِيَّ وَتَسْكُنُ عَبِيدِي هُنَاكَ.
\par 10 فَيَكُونُ شَارُونُ مَرْعَى غَنَمٍ وَوَادِي عَخُورَ مَرْبِضَ بَقَرٍ لِشَعْبِي الَّذِينَ طَلَبُونِي.
\par 11 «أَمَّا أَنْتُمُ الَّذِينَ تَرَكُوا الرَّبَّ وَنَسُوا جَبَلَ قُدْسِي وَرَتَّبُوا لِلسَّعْدِ الأَكْبَرِ مَائِدَةً وَمَلَأُوا لِلسَّعْدِ الأَصْغَرِ خَمْراً مَمْزُوجَةً
\par 12 فَإِنِّي أُعَيِّنُكُمْ لِلسَّيْفِ وَتَجْثُونَ كُلُّكُمْ لِلذَّبْحِ لأَنِّي دَعَوْتُ فَلَمْ تُجِيبُوا تَكَلَّمْتُ فَلَمْ تَسْمَعُوا بَلْ عَمِلْتُمُ الشَّرَّ فِي عَيْنَيَّ وَاخْتَرْتُمْ مَا لَمْ أُسَرَّ بِهِ».
\par 13 لِذَلِكَ هَكَذَا قَالَ السَّيِّدُ الرَّبُّ: «هُوَذَا عَبِيدِي يَأْكُلُونَ وَأَنْتُمْ تَجُوعُونَ. هُوَذَا عَبِيدِي يَشْرَبُونَ وَأَنْتُمْ تَعْطَشُونَ. هُوَذَا عَبِيدِي يَفْرَحُونَ وَأَنْتُمْ تَخْزُونَ.
\par 14 هُوَذَا عَبِيدِي يَتَرَنَّمُونَ مِنْ طِيبَةِ الْقَلْبِ وَأَنْتُمْ تَصْرُخُونَ مِنْ كآبَةِ الْقَلْبِ وَمِنِ انْكِسَارِ الرُّوحِ تُوَلْوِلُونَ.
\par 15 وَتُخْلِفُونَ اسْمَكُمْ لَعْنَةً لِمُخْتَارِيَّ فَيُمِيتُكَ السَّيِّدُ الرَّبُّ وَيُسَمِّي عَبِيدَهُ اسْماً آخَرَ.
\par 16 فَالَّذِي يَتَبَرَّكُ فِي الأَرْضِ يَتَبَرَّكُ بِإِلَهِ الْحَقِّ وَالَّذِي يَحْلِفُ فِي الأَرْضِ يَحْلِفُ بِإِلَهِ الْحَقِّ لأَنَّ الضِّيقَاتِ الأُولَى قَدْ نُسِيَتْ وَلأَنَّهَا اسْتَتَرَتْ عَنْ عَيْنَيَّ.
\par 17 «لأَنِّي هَئَنَذَا خَالِقٌ سَمَاوَاتٍ جَدِيدَةً وَأَرْضاً جَدِيدَةً فَلاَ تُذْكَرُ الأُولَى وَلاَ تَخْطُرُ عَلَى بَالٍ.
\par 18 بَلِ افْرَحُوا وَابْتَهِجُوا إِلَى الأَبَدِ فِي مَا أَنَا خَالِقٌ لأَنِّي هَئَنَذَا خَالِقٌ أُورُشَلِيمَ بَهْجَةً وَشَعْبَهَا فَرَحاً.
\par 19 فَأَبْتَهِجُ بِأُورُشَلِيمَ وَأَفْرَحُ بِشَعْبِي وَلاَ يُسْمَعُ بَعْدُ فِيهَا صَوْتُ بُكَاءٍ وَلاَ صَوْتُ صُرَاخٍ.
\par 20 لاَ يَكُونُ بَعْدُ هُنَاكَ طِفْلُ أَيَّامٍ وَلاَ شَيْخٌ لَمْ يُكْمِلْ أَيَّامَهُ. لأَنَّ الصَّبِيَّ يَمُوتُ ابْنَ مِئَةِ سَنَةٍ وَالْخَاطِئَ يُلْعَنُ ابْنَ مِئَةِ سَنَةٍ.
\par 21 وَيَبْنُونَ بُيُوتاً وَيَسْكُنُونَ فِيهَا وَيَغْرِسُونَ كُرُوماً وَيَأْكُلُونَ أَثْمَارَهَا.
\par 22 لاَ يَبْنُونَ وَآخَرُ يَسْكُنُ وَلاَ يَغْرِسُونَ وَآخَرُ يَأْكُلُ. لأَنَّهُ كَأَيَّامِ شَجَرَةٍ أَيَّامُ شَعْبِي وَيَسْتَعْمِلُ مُخْتَارِيَّ عَمَلَ أَيْدِيهِمْ.
\par 23 لاَ يَتْعَبُونَ بَاطِلاً وَلاَ يَلِدُونَ لِلرُّعْبِ لأَنَّهُمْ نَسْلُ مُبَارَكِي الرَّبِّ وَذُرِّيَّتُهُمْ مَعَهُمْ.
\par 24 وَيَكُونُ أَنِّي قَبْلَمَا يَدْعُونَ أَنَا أُجِيبُ وَفِيمَا هُمْ يَتَكَلَّمُونَ بَعْدُ أَنَا أَسْمَعُ.
\par 25 الذِّئْبُ وَالْحَمَلُ يَرْعَيَانِ مَعاً وَالأَسَدُ يَأْكُلُ التِّبْنَ كَالْبَقَرِ. أَمَّا الْحَيَّةُ فَالتُّرَابُ طَعَامُهَا. لاَ يُؤْذُونَ وَلاَ يُهْلِكُونَ فِي كُلِّ جَبَلِ قُدْسِي» قَالَ الرَّبُّ.

\chapter{66}

\par 1 هَكَذَا قَالَ الرَّبُّ: «السَّمَاوَاتُ كُرْسِيِّي وَالأَرْضُ مَوْطِئُ قَدَمَيَّ. أَيْنَ الْبَيْتُ الَّذِي تَبْنُونَ لِي وَأَيْنَ مَكَانُ رَاحَتِي؟
\par 2 وَكُلُّ هَذِهِ صَنَعَتْهَا يَدِي فَكَانَتْ كُلُّ هَذِهِ يَقُولُ الرَّبُّ. وَإِلَى هَذَا أَنْظُرُ: إِلَى الْمِسْكِينِ وَالْمُنْسَحِقِ الرُّوحِ وَالْمُرْتَعِدِ مِنْ كَلاَمِي.
\par 3 مَنْ يَذْبَحُ ثَوْراً فَهُوَ قَاتِلُ إِنْسَانٍ. مَنْ يَذْبَحُ شَاةً فَهُوَ نَاحِرُ كَلْبٍ. مَنْ يُصْعِدُ تَقْدِمَةً يُصْعِدُ دَمَ خِنْزِيرٍ. مَنْ أَحْرَقَ لُبَاناً فَهُوَ مُبَارِكٌ وَثَناً. بَلْ هُمُ اخْتَارُوا طُرُقَهُمْ وَبِمَكْرُهَاتِهِمْ سُرَّتْ أَنْفُسُهُمْ.
\par 4 فَأَنَا أَيْضاً أَخْتَارُ مَصَائِبَهُمْ وَمَخَاوِفَهُمْ أَجْلِبُهَا عَلَيْهِمْ. مِنْ أَجْلِ أَنِّي دَعَوْتُ فَلَمْ يَكُنْ مُجِيبٌ. تَكَلَّمْتُ فَلَمْ يَسْمَعُوا. بَلْ عَمِلُوا الْقَبِيحَ فِي عَيْنَيَّ وَاخْتَارُوا مَا لَمْ أُسَرَّ بِهِ».
\par 5 اِسْمَعُوا كَلاَمَ الرَّبِّ أَيُّهَا الْمُرْتَعِدُونَ مِنْ كَلاَمِهِ. قَالَ إِخْوَتُكُمُ الَّذِينَ أَبْغَضُوكُمْ وَطَرَدُوكُمْ مِنْ أَجْلِ اسْمِي: «لِيَتَمَجَّدِ الرَّبُّ». فَيَظْهَرُ لِفَرَحِكُمْ وَأَمَّا هُمْ فَيَخْزُونَ.
\par 6 صَوْتُ ضَجِيجٍ مِنَ الْمَدِينَةِ. صَوْتٌ مِنَ الْهَيْكَلِ. صَوْتُ الرَّبِّ مُجَازِياً أَعْدَاءَهُ.
\par 7 قَبْلَ أَنْ يَأْخُذَهَا الطَّلْقُ وَلَدَتْ. قَبْلَ أَنْ يَأْتِيَ عَلَيْهَا الْمَخَاضُ وَلَدَتْ ذَكَراً.
\par 8 مَنْ سَمِعَ مِثْلَ هَذَا؟ مَنْ رَأَى مِثْلَ هَذِهِ؟ هَلْ تَمْخَضُ بِلاَدٌ فِي يَوْمٍ وَاحِدٍ أَوْ تُولَدُ أُمَّةٌ دَفْعَةً وَاحِدَةً؟ فَقَدْ مَخَضَتْ صِهْيَوْنُ بَلْ وَلَدَتْ بَنِيهَا!
\par 9 هَلْ أَنَا أُمْخَضُ وَلاَ أُوَلِّدُ يَقُولُ الرَّبُّ أَوْ أَنَا الْمُوَلِّدُ هَلْ أُغْلِقُ الرَّحِمَ قَالَ إِلَهُكِ؟
\par 10 افْرَحُوا مَعَ أُورُشَلِيمَ وَابْتَهِجُوا مَعَهَا يَا جَمِيعَ مُحِبِّيهَا. افْرَحُوا مَعَهَا فَرَحاً يَا جَمِيعَ النَّائِحِينَ عَلَيْهَا
\par 11 لِتَرْضَعُوا وَتَشْبَعُوا مِنْ ثَدْيِ تَعْزِيَاتِهَا. لِتَعْصِرُوا وَتَتَلَذَّذُوا مِنْ دِرَّةِ مَجْدِهَا.
\par 12 لأَنَّهُ هَكَذَا قَالَ الرَّبُّ: «هَئَنَذَا أُدِيرُ عَلَيْهَا سَلاَماً كَنَهْرٍ وَمَجْدَ الأُمَمِ كَسَيْلٍ جَارِفٍ فَتَرْضَعُونَ وَعَلَى الأَيْدِي تُحْمَلُونَ وَعَلَى الرُّكْبَتَيْنِ تُدَلَّلُونَ.
\par 13 كَإِنْسَانٍ تُعَزِّيهِ أُمُّهُ هَكَذَا أُعَزِّيكُمْ أَنَا وَفِي أُورُشَلِيمَ تُعَزَّوْنَ.
\par 14 فَتَرُونَ وَتَفْرَحُ قُلُوبُكُمْ وَتَزْهُو عِظَامُكُمْ كَالْعُشْبِ وَتُعْرَفُ يَدُ الرَّبِّ عِنْدَ عَبِيدِهِ وَيَحْنَقُ عَلَى أَعْدَائِهِ.
\par 15 لأَنَّهُ هُوَذَا الرَّبُّ بِالنَّارِ يَأْتِي وَمَرْكَبَاتُهُ كَزَوْبَعَةٍ لِيَرُدَّ بِحُمُوٍّ غَضَبَهُ وَزَجْرَهُ بِلَهِيبِ نَارٍ.
\par 16 لأَنَّ الرَّبَّ بِالنَّارِ يُعَاقِبُ وَبِسَيْفِهِ عَلَى كُلِّ بَشَرٍ وَيَكْثُرُ قَتْلَى الرَّبِّ.
\par 17 الَّذِينَ يُقَدِّسُونَ وَيُطَهِّرُونَ أَنْفُسَهُمْ فِي الْجَنَّاتِ وَرَاءَ وَاحِدٍ فِي الْوَسَطِ آكِلِينَ لَحْمَ الْخِنْزِيرِ وَالرِّجْسَ وَالْجُرَذَ يَفْنُونَ مَعاً يَقُولُ الرَّبُّ.
\par 18 وَأَنَا أُجَازِي أَعْمَالَهُمْ وَأَفْكَارَهُمْ. حَدَثَ لِجَمْعِ كُلِّ الأُمَمِ وَالأَلْسِنَةِ فَيَأْتُونَ وَيَرُونَ مَجْدِي.
\par 19 وَأَجْعَلُ فِيهِمْ آيَةً وَأُرْسِلُ مِنْهُمْ نَاجِينَ إِلَى الأُمَمِ إِلَى تَرْشِيشَ وَفُولَ وَلُودَ النَّازِعِينَ فِي الْقَوْسِ. إِلَى تُوبَالَ وَيَاوَانَ إِلَى الْجَزَائِرِ الْبَعِيدَةِ الَّتِي لَمْ تَسْمَعْ خَبَرِي وَلاَ رَأَتْ مَجْدِي فَيُخْبِرُونَ بِمَجْدِي بَيْنَ الأُمَمِ.
\par 20 وَيُحْضِرُونَ كُلَّ إِخْوَتِكُمْ مِنْ كُلِّ الأُمَمِ تَقْدِمَةً لِلرَّبِّ عَلَى خَيْلٍ وَبِمَرْكَبَاتٍ وَبِهَوَادِجَ وَبِغَالٍ وَهُجُنٍ إِلَى جَبَلِ قُدْسِي أُورُشَلِيمَ قَالَ الرَّبُّ كَمَا يُحْضِرُ بَنُو إِسْرَائِيلَ تَقْدِمَةً فِي إِنَاءٍ طَاهِرٍ إِلَى بَيْتِ الرَّبِّ.
\par 21 وَأَتَّخِذُ أَيْضاً مِنْهُمْ كَهَنَةً وَلاَوِيِّينَ قَالَ الرَّبُّ.
\par 22 لأَنَّهُ كَمَا أَنَّ السَّمَاوَاتِ الْجَدِيدَةَ وَالأَرْضَ الْجَدِيدَةَ الَّتِي أَنَا صَانِعٌ تَثْبُتُ أَمَامِي يَقُولُ الرَّبُّ هَكَذَا يَثْبُتُ نَسْلُكُمْ وَاسْمُكُمْ.
\par 23 وَيَكُونُ مِنْ هِلاَلٍ إِلَى هِلاَلٍ وَمِنْ سَبْتٍ إِلَى سَبْتٍ أَنَّ كُلَّ ذِي جَسَدٍ يَأْتِي لِيَسْجُدَ أَمَامِي قَالَ الرَّبُّ.
\par 24 وَيَخْرُجُونَ وَيَرُونَ جُثَثَ النَّاسِ الَّذِينَ عَصُوا عَلَيَّ لأَنَّ دُودَهُمْ لاَ يَمُوتُ وَنَارَهُمْ لاَ تُطْفَأُ وَيَكُونُونَ رَذَالَةً لِكُلِّ ذِي جَسَدٍ.

\end{document}