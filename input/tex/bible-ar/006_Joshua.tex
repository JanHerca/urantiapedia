\begin{document}

\title{يشوع}


\chapter{1}

\par 1 وَكَانَ بَعْدَ مَوْتِ مُوسَى عَبْدِ الرَّبِّ أَنَّ الرَّبَّ قَالَ لِيَشُوعَ بْنِ نُونٍ خَادِمِ مُوسَى:
\par 2 «مُوسَى عَبْدِي قَدْ مَاتَ. فَالآنَ قُمُِ اعْبُرْ هَذَا الأُرْدُنَّ أَنْتَ وَكُلُّ هَذَا الشَّعْبِ إِلَى الأَرْضِ الَّتِي أَنَا مُعْطِيهَا لِبَنِي إِسْرَائِيلَ.
\par 3 كُلَّ مَوْضِعٍ تَدُوسُهُ بُطُونُ أَقْدَامِكُمْ لَكُمْ أَعْطَيْتُهُ كَمَا كَلَّمْتُ مُوسَى.
\par 4 مِنَ الْبَرِّيَّةِ وَلُبْنَانَ هَذَا إِلَى النَّهْرِ الْكَبِيرِ نَهْرِ الْفُرَاتِ, جَمِيعِ أَرْضِ الْحِثِّيِّينَ, وَإِلَى الْبَحْرِ الْكَبِيرِ نَحْوَ مَغْرِبِ الشَّمْسِ يَكُونُ تُخُمُكُمْ.
\par 5 لاَ يَقِفُ إِنْسَانٌ فِي وَجْهِكَ كُلَّ أَيَّامِ حَيَاتِكَ. كَمَا كُنْتُ مَعَ مُوسَى أَكُونُ مَعَكَ. لاَ أُهْمِلُكَ وَلاَ أَتْرُكُكَ.
\par 6 تَشَدَّدْ وَتَشَجَّعْ, لأَنَّكَ أَنْتَ تَقْسِمُ لِهَذَا الشَّعْبِ الأَرْضَ الَّتِي حَلَفْتُ لِآبَائِهِمْ أَنْ أُعْطِيَهُمْ.
\par 7 إِنَّمَا كُنْ مُتَشَدِّداً, وَتَشَجَّعْ جِدّاً لِتَتَحَفَّظَ لِلْعَمَلِ حَسَبَ كُلِّ الشَّرِيعَةِ الَّتِي أَمَرَكَ بِهَا مُوسَى عَبْدِي. لاَ تَمِلْ عَنْهَا يَمِيناً وَلاَ شِمَالاً لِتُفْلِحَ حَيْثُمَا تَذْهَبُ.
\par 8 لاَ يَبْرَحْ سِفْرُ هَذِهِ الشَّرِيعَةِ مِنْ فَمِكَ, بَلْ تَلْهَجُ فِيهِ نَهَاراً وَلَيْلاً, لِتَتَحَفَّظَ لِلْعَمَلِ حَسَبَ كُلِّ مَا هُوَ مَكْتُوبٌ فِيهِ. لأَنَّكَ حِينَئِذٍ تُصْلِحُ طَرِيقَكَ وَحِينَئِذٍ تُفْلِحُ.
\par 9 أَمَا أَمَرْتُكَ؟ تَشَدَّدْ وَتَشَجَّعْ! لاَ تَرْهَبْ وَلاَ تَرْتَعِبْ لأَنَّ الرَّبَّ إِلَهَكَ مَعَكَ حَيْثُمَا تَذْهَبُ».
\par 10 فَأَمَرَ يَشُوعُ عُرَفَاءَ الشَّعْبِ:
\par 11 «جُوزُوا فِي وَسَطِ الْمَحَلَّةِ وَأْمُرُوا الشَّعْبَ قَائِلِينَ: هَيِّئُوا لأَنْفُسِكُمْ زَاداً, لأَنَّكُمْ بَعْدَ ثَلاَثَةِ أَيَّامٍ تَعْبُرُونَ الأُرْدُنَّ هَذَا لِتَدْخُلُوا فَتَمْتَلِكُوا الأَرْضَ الَّتِي يُعْطِيكُمُ الرَّبُّ إِلَهُكُمْ لِتَمْتَلِكُوهَا».
\par 12 ثُمَّ قَالَ يَشُوعُ لِلرَّأُوبَيْنِيِّينَ وَالْجَادِيِّينَ وَنِصْفَ سِبْطِ مَنَسَّى:
\par 13 «اذْكُرُوا الْكَلاَمَ الَّذِي أَمَرَكُمْ بِهِ مُوسَى عَبْدُ الرَّبِّ قَائِلاً: الرَّبُّ إِلَهُكُمْ قَدْ أَرَاحَكُمْ وَأَعْطَاكُمْ هَذِهِ الأَرْضَ.
\par 14 نِسَاؤُكُمْ وَأَطْفَالُكُمْ وَمَوَاشِيكُمْ تَلْبَثُ فِي الأَرْضِ الَّتِي أَعْطَاكُمْ مُوسَى فِي عَبْرِ الأُرْدُنِّ, وَأَنْتُمْ تَعْبُرُونَ مُتَجَهِّزِينَ أَمَامَ إِخْوَتِكُمْ, كُلُّ الأَبْطَالِ ذَوِي الْبَأْسِ, وَتُعِينُونَهُمْ
\par 15 حَتَّى يُرِيحَ الرَّبُّ إِخْوَتَكُمْ مِثْلَكُمْ, وَيَمْتَلِكُوا هُمْ أَيْضاً الأَرْضَ الَّتِي يُعْطِيهِمُ الرَّبُّ إِلَهُكُمْ. ثُمَّ تَرْجِعُونَ إِلَى أَرْضِ مِيرَاثِكُمْ وَتَمْتَلِكُونَهَا, الَّتِي أَعْطَاكُمْ مُوسَى عَبْدُ الرَّبِّ فِي عَبْرِ الأُرْدُنِّ نَحْوَ شُرُوقِ الشَّمْسِ».
\par 16 فَأَجَابُوا يَشُوعَ: «كُلَّ مَا أَمَرْتَنَا بِهِ نَعْمَلُهُ, وَحَيْثُمَا تُرْسِلْنَا نَذْهَبْ.
\par 17 حَسَبَ كُلِّ مَا سَمِعْنَا لِمُوسَى نَسْمَعُ لَكَ. إِنَّمَا الرَّبُّ إِلَهُكَ يَكُونُ مَعَكَ كَمَا كَانَ مَعَ مُوسَى.
\par 18 كُلُّ إِنْسَانٍ يَعْصَى قَوْلَكَ وَلاَ يَسْمَعُ كَلاَمَكَ فِي كُلِّ مَا تَأْمُرُهُ بِهِ يُقْتَلُ. إِنَّمَا كُنْ مُتَشَدِّداً وَتَشَجَّعْ».

\chapter{2}

\par 1 فَأَرْسَلَ يَشُوعُ بْنُ نُونٍ مِنْ شِطِّيمَ رَجُلَيْنِ جَاسُوسَيْنِ سِرّاً, قَائِلاً: «اذْهَبَا انْظُرَا الأَرْضَ وَأَرِيحَا». فَذَهَبَا وَدَخَلاَ بَيْتَ امْرَأَةٍ زَانِيَةٍ اسْمُهَا رَاحَابُ وَاضْطَجَعَا هُنَاكَ.
\par 2 فَقِيلَ لِمَلِكِ أَرِيحَا: «هُوَذَا قَدْ دَخَلَ إِلَى هُنَا اللَّيْلَةَ رَجُلاَنِ مِنْ بَنِي إِسْرَائِيلَ لِيَتَجَسَّسَا الأَرْضَ».
\par 3 فَأَرْسَلَ مَلِكُ أَرِيحَا إِلَى رَاحَابَ يَقُولُ: «أَخْرِجِي الرَّجُلَيْنِ اللَّذَيْنِ أَتَيَا إِلَيْكِ وَدَخَلاَ بَيْتَكِ, لأَنَّهُمَا قَدْ أَتَيَا لِيَتَجَسَّسَا الأَرْضَ كُلَّهَا».
\par 4 فَأَخَذَتِ الْمَرْأَةُ الرَّجُلَيْنِ وَخَبَّأَتْهُمَا وَقَالَتْ: «نَعَمْ جَاءَ إِلَيَّ الرَّجُلاَنِ وَلَمْ أَعْلَمْ مِنْ أَيْنَ هُمَا.
\par 5 وَكَانَ نَحْوَ انْغِلاَقِ الْبَابِ فِي الظَّلاَمِ أَنَّهُمَا خَرَجَا. لَسْتُ أَعْلَمُ أَيْنَ ذَهَبَا. اسْعُوا سَرِيعاً وَرَاءَهُمَا حَتَّى تُدْرِكُوهُمَا».
\par 6 وَأَمَّا هِيَ فَأَطْلَعَتْهُمَا عَلَى السَّطْحِ وَوَارَتْهُمَا بَيْنَ عِيدَانِ كَتَّانٍ لَهَا مُنَضَّدَةً عَلَى السَّطْحِ.
\par 7 فَسَعَى الْقَوْمُ وَرَاءَهُمَا فِي طَرِيقِ الأُرْدُنِّ إِلَى الْمَخَاوِضِ. وَحَالَمَا خَرَجَ الَّذِينَ سَعُوا وَرَاءَهُمَا أَغْلَقُوا الْبَابَ.
\par 8 وَأَمَّا هُمَا فَقَبْلَ أَنْ يَضْطَجِعَا صَعِدَتْ إِلَيْهِمَا إِلَى السَّطْحِ
\par 9 وَقَالَتْ: «عَلِمْتُ أَنَّ الرَّبَّ قَدْ أَعْطَاكُمُ الأَرْضَ, وَأَنَّ رُعْبَكُمْ قَدْ وَقَعَ عَلَيْنَا, وَأَنَّ جَمِيعَ سُكَّانِ الأَرْضِ ذَابُوا مِنْ أَجْلِكُمْ,
\par 10 لأَنَّنَا قَدْ سَمِعْنَا كَيْفَ يَبَّسَ الرَّبُّ مِيَاهَ بَحْرِ سُوفَ قُدَّامَكُمْ عِنْدَ خُرُوجِكُمْ مِنْ مِصْرَ, وَمَا عَمِلْتُمُوهُ بِمَلِكَيِ الأَمُورِيِّينَ اللَّذَيْنِ فِي عَبْرِ الأُرْدُنِّ: سِيحُونَ وَعُوجَ, اللَّذَيْنِ حَرَّمْتُمُوهُمَا.
\par 11 سَمِعْنَا فَذَابَتْ قُلُوبُنَا وَلَمْ تَبْقَ بَعْدُ رُوحٌ فِي إِنْسَانٍ بِسَبَبِكُمْ, لأَنَّ الرَّبَّ إِلَهَكُمْ هُوَ اللَّهُ فِي السَّمَاءِ مِنْ فَوْقُ وَعَلَى الأَرْضِ مِنْ تَحْتُ.
\par 12 فَالآنَ احْلِفَا لِي بِالرَّبِّ وَأَعْطِيَانِي عَلاَمَةَ أَمَانَةٍ. لأَنِّي قَدْ عَمِلْتُ مَعَكُمَا مَعْرُوفاً. بِأَنْ تَعْمَلاَ أَنْتُمَا أَيْضاً مَعَ بَيْتِ أَبِي مَعْرُوفاً.
\par 13 وَتَسْتَحْيِيَا أَبِي وَأُمِّي وَإِخْوَتِي وَأَخَوَاتِي وَكُلَّ مَا لَهُمْ وَتُخَلِّصَا أَنْفُسَنَا مِنَ الْمَوْتِ».
\par 14 فَقَالَ لَهَا الرَّجُلاَنِ: «نَفْسُنَا عِوَضُكُمْ لِلْمَوْتِ إِنْ لَمْ تُفْشُوا أَمْرَنَا هَذَا. وَيَكُونُ إِذَا أَعْطَانَا الرَّبُّ الأَرْضَ أَنَّنَا نَعْمَلُ مَعَكِ مَعْرُوفاً وَأَمَانَةً».
\par 15 فَأَنْزَلَتْهُمَا بِحَبْلٍ مِنَ الْكُوَّةِ, لأَنَّ بَيْتَهَا بِحَائِطِ السُّورِ, وَهِيَ سَكَنَتْ بِالسُّورِ.
\par 16 وَقَالَتْ لَهُمَا: «اذْهَبَا إِلَى الْجَبَلِ لِئَلاَّ يُصَادِفَكُمَا السُّعَاةُ, وَاخْتَبِئَا هُنَاكَ ثَلاَثَةَ أَيَّامٍ حَتَّى يَرْجِعَ السُّعَاةُ, ثُمَّ اذْهَبَا فِي طَرِيقِكُمَا».
\par 17 فَقَالَ لَهَا الرَّجُلاَنِ: «نَحْنُ بَرِيئَانِ مِنْ يَمِينِكِ هَذَا الَّذِي حَلَّفْتِنَا بِهِ.
\par 18 هُوَذَا نَحْنُ نَأْتِي إِلَى الأَرْضِ, فَارْبِطِي هَذَا الْحَبْلَ مِنْ خُيُوطِ الْقِرْمِزِ فِي الْكُوَّةِ الَّتِي أَنْزَلْتِنَا مِنْهَا, وَاجْمَعِي إِلَيْكِ فِي الْبَيْتِ أَبَاكِ وَأُمَّكِ وَإِخْوَتَكِ وَسَائِرَ بَيْتِ أَبِيكِ.
\par 19 فَيَكُونُ أَنَّ كُلَّ مَنْ يَخْرُجُ مِنْ أَبْوَابِ بَيْتِكِ إِلَى خَارِجٍ, فَدَمُهُ عَلَى رَأْسِهِ, وَنَحْنُ نَكُونُ بَرِيئَيْنِ. وَأَمَّا كُلُّ مَنْ يَكُونُ مَعَكِ فِي الْبَيْتِ فَدَمُهُ عَلَى رَأْسِنَا إِذَا وَقَعَتْ عَلَيْهِ يَدٌ.
\par 20 وَإِنْ أَفْشَيْتِ أَمْرَنَا هَذَا نَكُونُ بَرِيئَيْنِ مِنْ حَلْفِكِ الَّذِي حَلَّفْتِنَا».
\par 21 فَقَالَتْ: «هُوَ هَكَذَا حَسَبَ كَلاَمِكُمَا». وَصَرَفَتْهُمَا فَذَهَبَا. وَرَبَطَتْ حَبْلَ الْقِرْمِزِ فِي الْكُوَّةِ.
\par 22 فَانْطَلَقَا وَجَاءَا إِلَى الْجَبَلِ وَلَبِثَا هُنَاكَ ثَلاَثَةَ أَيَّامٍ حَتَّى رَجَعَ السُّعَاةُ. وَفَتَّشَ السُّعَاةُ فِي كُلِّ الطَّرِيقِ فَلَمْ يَجِدُوهُمَا.
\par 23 ثُمَّ رَجَعَ الرَّجُلاَنِ وَنَزَلاَ عَنِ الْجَبَلِ وَعَبَرَا وَأَتَيَا إِلَى يَشُوعَ بْنِ نُونٍ وَقَصَّا عَلَيْهِ كُلَّ مَا أَصَابَهُمَا.
\par 24 وَقَالاَ لِيَشُوعَ: «إِنَّ الرَّبَّ قَدْ دَفَعَ بِيَدِنَا الأَرْضَ كُلَّهَا, وَقَدْ ذَابَ كُلُّ سُكَّانِ الأَرْضِ بِسَبَبِنَا».

\chapter{3}

\par 1 فَبَكَّرَ يَشُوعُ فِي الْغَدِ وَارْتَحَلُوا مِنْ شِطِّيمَ وَأَتَوْا إِلَى الأُرْدُنِّ, هُوَ وَكُلُّ بَنِي إِسْرَائِيلَ, وَبَاتُوا هُنَاكَ قَبْلَ أَنْ عَبَرُوا.
\par 2 وَكَانَ بَعْدَ ثَلاَثَةِ أَيَّامٍ أَنَّ الْعُرَفَاءَ جَازُوا فِي وَسَطِ الْمَحَلَّةِ
\par 3 وَأَمَرُوا الشَّعْبَ: «عِنْدَمَا تَرُونَ تَابُوتَ عَهْدِ الرَّبِّ إِلَهِكُمْ وَالْكَهَنَةَ اللاَّوِيِّينَ حَامِلِينَ إِيَّاهُ, فَارْتَحِلُوا مِنْ أَمَاكِنِكُمْ وَسِيرُوا وَرَاءَهُ.
\par 4 وَلَكِنْ يَكُونُ بَيْنَكُمْ وَبَيْنَهُ مَسَافَةٌ نَحْوُ أَلْفَيْ ذِرَاعٍ بِالْقِيَاسِ. لاَ تَقْرُبُوا مِنْهُ لِكَيْ تَعْرِفُوا الطَّرِيقَ الَّذِي تَسِيرُونَ فِيهِ. لأَنَّكُمْ لَمْ تَعْبُرُوا هَذَا الطَّرِيقَ مِنْ قَبْلُ».
\par 5 وَقَالَ يَشُوعُ لِلشَّعْبِ: «تَقَدَّسُوا لأَنَّ الرَّبَّ يَعْمَلُ غَداً فِي وَسَطِكُمْ عَجَائِبَ».
\par 6 وَقَالَ يَشُوعُ لِلْكَهَنَةِ: «احْمِلُوا تَابُوتَ الْعَهْدِ وَاعْبُرُوا أَمَامَ الشَّعْبِ». فَحَمَلُوا تَابُوتَ الْعَهْدِ وَسَارُوا أَمَامَ الشَّعْبِ.
\par 7 فَقَالَ الرَّبُّ لِيَشُوعَ: «الْيَوْمَ أَبْتَدِئُ أُعَظِّمُكَ فِي أَعْيُنِ جَمِيعِ إِسْرَائِيلَ لِيَعْلَمُوا أَنِّي كَمَا كُنْتُ مَعَ مُوسَى أَكُونُ مَعَكَ.
\par 8 وَأَمَّا أَنْتَ فَأْمُرِ الْكَهَنَةَ حَامِلِي تَابُوتِ الْعَهْدِ قَائِلاً: عِنْدَمَا تَأْتُونَ إِلَى ضَفَّةِ مِيَاهِ الأُرْدُنِّ تَقِفُونَ فِي الأُرْدُنِّ».
\par 9 فَقَالَ يَشُوعُ لِبَنِي إِسْرَائِيلَ: «تَقَدَّمُوا إِلَى هُنَا وَاسْمَعُوا كَلاَمَ الرَّبِّ إِلَهِكُمْ».
\par 10 ثُمَّ قَالَ يَشُوعُ: «بِهَذَا تَعْلَمُونَ أَنَّ اللَّهَ الْحَيَّ فِي وَسَطِكُمْ, وَطَرْداً يَطْرُدُ مِنْ أَمَامِكُمُ الْكَنْعَانِيِّينَ وَالْحِثِّيِّينَ وَالْحِوِّيِّينَ وَالْفِرِزِّيِّينَ وَالْجِرْجَاشِيِّينَ وَالأَمُورِيِّينَ وَالْيَبُوسِيِّينَ.
\par 11 هُوَذَا تَابُوتُ عَهْدِ سَيِّدِ كُلِّ الأَرْضِ عَابِرٌ أَمَامَكُمْ فِي الأُرْدُنِّ.
\par 12 فَالآنَ انْتَخِبُوا اثْنَيْ عَشَرَ رَجُلاً مِنْ أَسْبَاطِ إِسْرَائِيلَ, رَجُلاً وَاحِداً مِنْ كُلِّ سِبْطٍ.
\par 13 وَيَكُونُ حِينَمَا تَسْتَقِرُّ بُطُونُ أَقْدَامِ الْكَهَنَةِ حَامِلِي تَابُوتِ الرَّبِّ سَيِّدِ الأَرْضِ كُلِّهَا فِي مِيَاهِ الأُرْدُنِّ, أَنَّ مِيَاهَ الأُرْدُنِّ الْمُنْحَدِرَةَ مِنْ فَوْقُ تَنْفَلِقُ وَتَقِفُ نَدّاً وَاحِداً».
\par 14 وَلَمَّا ارْتَحَلَ الشَّعْبُ مِنْ خِيَامِهِمْ لِيَعْبُرُوا الأُرْدُنَّ, وَالْكَهَنَةُ حَامِلُو تَابُوتِ الْعَهْدِ أَمَامَ الشَّعْبِ,
\par 15 فَعِنْدَ إِتْيَانِ حَامِلِي التَّابُوتِ إِلَى الأُرْدُنِّ وَانْغِمَاسِ أَرْجُلِ الْكَهَنَةِ حَامِلِي التَّابُوتِ فِي ضَفَّةِ الْمِيَاهِ - وَالأُرْدُنُّ مُمْتَلِئٌ إِلَى جَمِيعِ شُطُوطِهِ كُلَّ أَيَّامِ الْحَصَادِ -
\par 16 وَقَفَتِ الْمِيَاهُ الْمُنْحَدِرَةُ مِنْ فَوْقُ وَقَامَتْ نَدّاً وَاحِداً بَعِيداً جِدّاً عَنْ «أَدَامَ» الْمَدِينَةِ الَّتِي إِلَى جَانِبِ صَرْتَانَ, وَالْمُنْحَدِرَةُ إِلَى بَحْرِ الْعَرَبَةِ «بَحْرِ الْمِلْحِ» انْقَطَعَتْ تَمَاماً, وَعَبَرَ الشَّعْبُ مُقَابِلَ أَرِيحَا.
\par 17 فَوَقَفَ الْكَهَنَةُ حَامِلُو تَابُوتِ عَهْدِ الرَّبِّ عَلَى الْيَابِسَةِ فِي وَسَطِ الأُرْدُنِّ رَاسِخِينَ, وَجَمِيعُ إِسْرَائِيلَ عَابِرُونَ عَلَى الْيَابِسَةِ حَتَّى انْتَهَى جَمِيعُ الشَّعْبِ مِنْ عُبُورِ الأُرْدُنِّ.

\chapter{4}

\par 1 وَكَانَ لَمَّا انْتَهَى جَمِيعُ الشَّعْبِ مِنْ عُبُورِ الأُرْدُنِّ أَنَّ الرَّبَّ أَمَرَ يَشُوعَ:
\par 2 «انْتَخِبُوا مِنَ الشَّعْبِ اثْنَيْ عَشَرَ رَجُلاً. رَجُلاً وَاحِداً مِنْ كُلِّ سِبْطٍ
\par 3 وَأْمُرُوهُمْ قَائِلِينَ: احْمِلُوا مِنْ هُنَا مِنْ وَسَطِ الأُرْدُنِّ مِنْ مَوْقِفِ أَرْجُلِ الْكَهَنَةِ رَاسِخَةً اثْنَيْ عَشَرَ حَجَراً, وَعَبِّرُوهَا مَعَكُمْ وَضَعُوهَا فِي الْمَبِيتِ الَّذِي تَبِيتُونَ فِيهِ اللَّيْلَةَ».
\par 4 فَدَعَا يَشُوعُ الاِثْنَيْ عَشَرَ رَجُلاً الَّذِينَ عَيَّنَهُمْ مِنْ بَنِي إِسْرَائِيلَ, رَجُلاً وَاحِداً مِنْ كُلِّ سِبْطٍ.
\par 5 وَقَالَ لَهُمْ يَشُوعُ: «اعْبُرُوا أَمَامَ تَابُوتِ الرَّبِّ إِلَهِكُمْ إِلَى وَسَطِ الأُرْدُنِّ, وَارْفَعُوا كُلُّ رَجُلٍ حَجَراً وَاحِداً عَلَى كَتِفِهِ حَسَبَ عَدَدِ أَسْبَاطِ بَنِي إِسْرَائِيلَ,
\par 6 لِكَيْ تَكُونَ هَذِهِ عَلاَمَةً فِي وَسَطِكُمْ. إِذَا سَأَلَ غَداً بَنُوكُمْ: مَا لَكُمْ وَهَذِهِ الْحِجَارَةَ؟
\par 7 تَقُولُونَ لَهُمْ: إِنَّ مِيَاهَ الأُرْدُنِّ قَدِ انْفَلَقَتْ أَمَامَ تَابُوتِ عَهْدِ الرَّبِّ. عِنْدَ عُبُورِهِ الأُرْدُنَّ انْفَلَقَتْ مِيَاهُ الأُرْدُنِّ. فَتَكُونُ هَذِهِ الْحِجَارَةُ تِذْكَاراً لِبَنِي إِسْرَائِيلَ إِلَى الدَّهْرِ».
\par 8 فَفَعَلَ بَنُو إِسْرَائِيلَ كَمَا أَمَرَ يَشُوعُ, وَحَمَلُوا اثْنَيْ عَشَرَ حَجَراً مِنْ وَسَطِ الأُرْدُنِّ كَمَا قَالَ الرَّبُّ لِيَشُوعَ, حَسَبَ عَدَدِ أَسْبَاطِ بَنِي إِسْرَائِيلَ, وَعَبَّرُوهَا مَعَهُمْ إِلَى الْمَبِيتِ وَوَضَعُوهَا هُنَاكَ.
\par 9 وَنَصَبَ يَشُوعُ اثْنَيْ عَشَرَ حَجَراً فِي وَسَطِ الأُرْدُنِّ تَحْتَ مَوْقِفِ أَرْجُلِ الْكَهَنَةِ حَامِلِي تَابُوتِ الْعَهْدِ. وَهِيَ هُنَاكَ إِلَى هَذَا الْيَوْمِ.
\par 10 وَالْكَهَنَةُ حَامِلُو التَّابُوتِ وَقَفُوا فِي وَسَطِ الأُرْدُنِّ حَتَّى انْتَهَى كُلُّ شَيْءٍ أَمَرَ الرَّبُّ يَشُوعَ أَنْ يُكَلِّمَ بِهِ الشَّعْبَ, حَسَبَ كُلِّ مَا أَمَرَ بِهِ مُوسَى يَشُوعَ. وَأَسْرَعَ الشَّعْبُ فَعَبَرُوا.
\par 11 وَكَانَ لَمَّا انْتَهَى كُلُّ الشَّعْبِ مِنَ الْعُبُورِ أَنَّهُ عَبَرَ تَابُوتُ الرَّبِّ وَالْكَهَنَةُ فِي حَضْرَةِ الشَّعْبِ.
\par 12 وَعَبَرَ بَنُو رَأُوبَيْنَ وَبَنُو جَادٍ وَنِصْفُ سِبْطِ مَنَسَّى مُتَجَهِّزِينَ أَمَامَ بَنِي إِسْرَائِيلَ كَمَا كَلَّمَهُمْ مُوسَى.
\par 13 نَحْوَ أَرْبَعِينَ أَلْفاً مُتَجَرِّدِينَ لِلْجُنْدِ عَبَرُوا أَمَامَ الرَّبِّ لِلْحَرْبِ إِلَى عَرَبَاتِ أَرِيحَا.
\par 14 فِي ذَلِكَ الْيَوْمِ عَظَّمَ الرَّبُّ يَشُوعَ فِي أَعْيُنِ جَمِيعِ إِسْرَائِيلَ, فَهَابُوهُ كَمَا هَابُوا مُوسَى كُلَّ أَيَّامِ حَيَاتِهِ.
\par 15 وَقَالَ الرَّبُّ لِيَشُوعَ:
\par 16 «مُرِ الْكَهَنَةَ حَامِلِي تَابُوتِ الشَّهَادَةِ أَنْ يَصْعَدُوا مِنَ الأُرْدُنِّ».
\par 17 فَأَمَرَ يَشُوعُ الْكَهَنَةَ: «اصْعَدُوا مِنَ الأُرْدُنِّ».
\par 18 فَكَانَ لَمَّا صَعِدَ الْكَهَنَةُ حَامِلُو تَابُوتِ عَهْدِ الرَّبِّ مِنْ وَسَطِ الأُرْدُنِّ, وَاجْتُذِبَتْ بُطُونُ أَقْدَامِ الْكَهَنَةِ إِلَى الْيَابِسَةِ, أَنَّ مِيَاهَ الأُرْدُنِّ رَجَعَتْ إِلَى مَكَانِهَا وَجَرَتْ كَمَا مِنْ قَبْلُ إِلَى كُلِّ شُطُوطِهِ.
\par 19 وَصَعِدَ الشَّعْبُ مِنَ الأُرْدُنِّ فِي الْيَوْمِ الْعَاشِرِ مِنَ الشَّهْرِ الأَوَّلِ, وَحَلُّوا فِي الْجِلْجَالِ فِي تُخُمِ أَرِيحَا الشَّرْقِيِّ.
\par 20 وَالاِثْنَا عَشَرَ حَجَراً الَّتِي أَخَذُوهَا مِنَ الأُرْدُنِّ نَصَبَهَا يَشُوعُ فِي الْجِلْجَالِ.
\par 21 وَقَالَ لِبَنِي إِسْرَائِيلَ: «إِذَا سَأَلَ بَنُوكُمْ غَداً آبَاءَهُمْ قَائِلِينَ: مَا هَذِهِ الْحِجَارَةُ؟
\par 22 تُعْلِمُونَ بَنِيكُمْ قَائِلِينَ: عَلَى الْيَابِسَةِ عَبَرَ إِسْرَائِيلُ هَذَا الأُرْدُنَّ.
\par 23 لأَنَّ الرَّبَّ إِلَهَكُمْ قَدْ يَبَّسَ مِيَاهَ الأُرْدُنِّ مِنْ أَمَامِكُمْ حَتَّى عَبَرْتُمْ, كَمَا فَعَلَ الرَّبُّ إِلَهُكُمْ بِبَحْرِ سُوفٍ الَّذِي يَبَّسَهُ مِنْ أَمَامِنَا حَتَّى عَبَرْنَا.
\par 24 لِتَعْلَمَ جَمِيعُ شُعُوبِ الأَرْضِ يَدَ الرَّبِّ أَنَّهَا قَوِيَّةٌ, لِكَيْ تَخَافُوا الرَّبَّ إِلَهَكُمْ كُلَّ الأَيَّامِ».

\chapter{5}

\par 1 وَعِنْدَمَا سَمِعَ جَمِيعُ مُلُوكِ الأَمُورِيِّينَ الَّذِينَ فِي عَبْرِ الأُرْدُنِّ غَرْباً, وَجَمِيعُ مُلُوكِ الْكَنْعَانِيِّينَ الَّذِينَ عَلَى الْبَحْرِ, أَنَّ الرَّبَّ قَدْ يَبَّسَ مِيَاهَ الأُرْدُنِّ مِنْ أَمَامِ بَنِي إِسْرَائِيلَ حَتَّى عَبَرْنَا, ذَابَتْ قُلُوبُهُمْ وَلَمْ تَبْقَ فِيهِمْ رُوحٌ بَعْدُ مِنْ جَرَّاءِ بَنِي إِسْرَائِيلَ.
\par 2 فِي ذَلِكَ الْوَقْتِ قَالَ الرَّبُّ لِيَشُوعَ: «اصْنَعْ لِنَفْسِكَ سَكَاكِينَ مِنْ صَوَّانٍ, وَعُدْ فَاخْتُنْ بَنِي إِسْرَائِيلَ ثَانِيَةً».
\par 3 فَصَنَعَ يَشُوعُ سَكَاكِينَ مِنْ صَوَّانٍ وَخَتَنَ بَنِي إِسْرَائِيلَ فِي تَلِّ الْقُلَفِ.
\par 4 وَهَذَا هُوَ سَبَبُ خَتْنِ يَشُوعَ إِيَّاهُمْ: أَنَّ جَمِيعِ الشَّعْبِ الْخَارِجِينَ مِنْ مِصْرَ, الذُّكُورَِ, جَمِيعَِ رِجَالِ الْحَرْبِ, مَاتُوا فِي الْبَرِّيَّةِ عَلَى الطَّرِيقِ بِخُرُوجِهِمْ مِنْ مِصْرَ.
\par 5 لأَنَّ جَمِيعَ الشَّعْبِ الَّذِينَ خَرَجُوا كَانُوا مَخْتُونِينَ. وَأَمَّا جَمِيعُ الشَّعْبِ الَّذِينَ وُلِدُوا فِي الْقَفْرِ عَلَى الطَّرِيقِ بِخُرُوجِهِمْ مِنْ مِصْرَ فَلَمْ يُخْتَنُوا.
\par 6 لأَنَّ بَنِي إِسْرَائِيلَ سَارُوا أَرْبَعِينَ سَنَةً فِي الْقَفْرِ حَتَّى فَنِيَ جَمِيعُ الشَّعْبِ رِجَالُ الْحَرْبِ الْخَارِجِينَ مِنْ مِصْرَ, الَّذِينَ لَمْ يَسْمَعُوا لِقَوْلِ الرَّبِّ, الَّذِينَ حَلَفَ الرَّبُّ لَهُمْ أَنَّهُ لاَ يُرِيهِمِ الأَرْضَ الَّتِي حَلَفَ الرَّبُّ لِآبَائِهِمْ أَنْ يُعْطِيَنَا إِيَّاهَا, الأَرْضَ الَّتِي تَفِيضُ لَبَناً وَعَسَلاً.
\par 7 وَأَمَّا بَنُوهُمْ فَأَقَامَهُمْ مَكَانَهُمْ. فَإِيَّاهُمْ خَتَنَ يَشُوعُ لأَنَّهُمْ كَانُوا قُلْفاً, إِذْ لَمْ يَخْتِنُوهُمْ فِي الطَّرِيقِ.
\par 8 وَكَانَ بَعْدَمَا انْتَهَى جَمِيعُ الشَّعْبِ مِنَ الاِخْتِتَانِ أَنَّهُمْ أَقَامُوا فِي أَمَاكِنِهِمْ فِي الْمَحَلَّةِ حَتَّى بَرِئُوا.
\par 9 وَقَالَ الرَّبُّ لِيَشُوعَ: «الْيَوْمَ قَدْ دَحْرَجْتُ عَنْكُمْ عَارَ مِصْرَ». فَدُعِيَ اسْمُ ذَلِكَ الْمَكَانِ «الْجِلْجَالَ» إِلَى هَذَا الْيَوْمِ.
\par 10 فَحَلَّ بَنُو إِسْرَائِيلَ فِي الْجِلْجَالِ, وَعَمِلُوا الْفِصْحَ فِي الْيَوْمِ الرَّابِعَ عَشَرَ مِنَ الشَّهْرِ مَسَاءً فِي عَرَبَاتِ أَرِيحَا.
\par 11 وَأَكَلُوا مِنْ غَلَّةِ الأَرْضِ فِي الْغَدِ بَعْدَ الْفِصْحِ فَطِيراً وَفَرِيكاً فِي نَفْسِ ذَلِكَ الْيَوْمِ.
\par 12 وَانْقَطَعَ الْمَنُّ فِي الْغَدِ عِنْدَ أَكْلِهِمْ مِنْ غَلَّةِ الأَرْضِ, وَلَمْ يَكُنْ بَعْدُ لِبَنِي إِسْرَائِيلَ مَنٌّ. فَأَكَلُوا مِنْ مَحْصُولِ أَرْضِ كَنْعَانَ فِي تِلْكَ السَّنَةِ.
\par 13 وَحَدَثَ لَمَّا كَانَ يَشُوعُ عِنْدَ أَرِيحَا أَنَّهُ رَفَعَ عَيْنَيْهِ وَنَظَرَ, وَإِذَا بِرَجُلٍ وَاقِفٍ قُبَالَتَهُ, وَسَيْفُهُ مَسْلُولٌ بِيَدِهِ. فَسَارَ يَشُوعُ إِلَيْهِ وَسَأَلَهُ: «هَلْ لَنَا أَنْتَ أَوْ لأَعْدَائِنَا؟»
\par 14 فَقَالَ: «كَلاَّ, بَلْ أَنَا رَئِيسُ جُنْدِ الرَّبِّ. الآنَ أَتَيْتُ». فَسَقَطَ يَشُوعُ عَلَى وَجْهِهِ إِلَى الأَرْضِ وَسَجَدَ, وَقَالَ لَهُ: «بِمَاذَا يُكَلِّمُ سَيِّدِي عَبْدَهُ؟»
\par 15 فَقَالَ رَئِيسُ جُنْدِ الرَّبِّ لِيَشُوعَ: «اخْلَعْ نَعْلَكَ مِنْ رِجْلِكَ, لأَنَّ الْمَكَانَ الَّذِي أَنْتَ وَاقِفٌ عَلَيْهِ هُوَ مُقَدَّسٌ». فَفَعَلَ يَشُوعُ كَذَلِكَ.

\chapter{6}

\par 1 وَكَانَتْ أَرِيحَا مُغَلَّقَةً مُقَفَّلَةً بِسَبَبِ بَنِي إِسْرَائِيلَ. لاَ أَحَدٌ يَخْرُجُ وَلاَ أَحَدٌ يَدْخُلُ.
\par 2 فَقَالَ الرَّبُّ لِيَشُوعَ: «انْظُرْ. قَدْ دَفَعْتُ بِيَدِكَ أَرِيحَا وَمَلِكَهَا جَبَابِرَةَ الْبَأْسِ.
\par 3 تَدُورُونَ دَائِرَةَ الْمَدِينَةِ, جَمِيعُ رِجَالِ الْحَرْبِ. حَوْلَ الْمَدِينَةِ مَرَّةً وَاحِدَةً. هَكَذَا تَفْعَلُونَ سِتَّةَ أَيَّامٍ.
\par 4 وَسَبْعَةُ كَهَنَةٍ يَحْمِلُونَ أَبْوَاقَ الْهُتَافِ السَّبْعَةَ أَمَامَ التَّابُوتِ. وَفِي الْيَوْمِ السَّابِعِ تَدُورُونَ دَائِرَةَ الْمَدِينَةِ سَبْعَ مَرَّاتٍ, وَالْكَهَنَةُ يَضْرِبُونَ بِالأَبْوَاقِ.
\par 5 وَيَكُونُ عِنْدَ امْتِدَادِ صَوْتِ قَرْنِ الْهُتَافِ عِنْدَ اسْتِمَاعِكُمْ صَوْتَ الْبُوقِ, أَنَّ جَمِيعَ الشَّعْبِ يَهْتِفُ هُتَافاً عَظِيماً, فَيَسْقُطُ سُورُ الْمَدِينَةِ فِي مَكَانِهِ, وَيَصْعَدُ الشَّعْبُ كُلُّ رَجُلٍ مَعَ وَجْهِهِ».
\par 6 فَدَعَا يَشُوعُ بْنُ نُونٍ الْكَهَنَةَ وَقَالَ لَهُمُ: «احْمِلُوا تَابُوتَ الْعَهْدِ. وَلْيَحْمِلْ سَبْعَةُ كَهَنَةٍ سَبْعَةَ أَبْوَاقِ هُتَافٍ أَمَامَ تَابُوتِ الرَّبِّ».
\par 7 وَقَالُوا لِلشَّعْبِ اجْتَازُوا وَدُورُوا دَائِرَةَ الْمَدِينَةِ وَلْيَجْتَزِ الْمُتَجَرِّدُ أَمَامَ تَابُوتِ الرَّبِّ.
\par 8 وَكَانَ كَمَا قَالَ يَشُوعُ لِلشَّعْبِ. اجْتَازَ السَّبْعَةُ الْكَهَنَةُ حَامِلِينَ أَبْوَاقَ الْهُتَافِ السَّبْعَةَ أَمَامَ الرَّبِّ, وَضَرَبُوا بِالأَبْوَاقِ. وَتَابُوتُ عَهْدِ الرَّبِّ سَائِرٌ وَرَاءَهُمْ,
\par 9 وَكُلُّ مُتَجَرِّدٍ سَائِرٌ أَمَامَ الْكَهَنَةِ الضَّارِبِينَ بِالأَبْوَاقِ. وَالْبَقِيَّةُ سَائِرَةٌ وَرَاءَ التَّابُوتِ. كَانُوا يَسِيرُونَ وَيَضْرِبُونَ بِالأَبْوَاقِ.
\par 10 وَأَمَرَ يَشُوعُ الشَّعْبَ: لاَ تَهْتِفُوا وَلاَ تُسَمِّعُوا صَوْتَكُمْ, وَلاَ تَخْرُجْ مِنْ أَفْوَاهِكُمْ كَلِمَةٌ حَتَّى يَوْمَ أَقُولُ لَكُمُ: اهْتِفُوا. فَتَهْتِفُونَ».
\par 11 فَدَارَ تَابُوتُ الرَّبِّ حَوْلَ الْمَدِينَةِ مَرَّةً وَاحِدَةً. ثُمَّ دَخَلُوا الْمَحَلَّةَ وَبَاتُوا فِي الْمَحَلَّةِ.
\par 12 فَبَكَّرَ يَشُوعُ فِي الْغَدِ, وَحَمَلَ الْكَهَنَةُ تَابُوتَ الرَّبِّ,
\par 13 وَالسَّبْعَةُ الْكَهَنَةُ الْحَامِلُونَ أَبْوَاقَ الْهُتَافِ السَّبْعَةَ أَمَامَ تَابُوتِ الرَّبِّ سَائِرُونَ سَيْراً وَضَارِبُونَ بِالأَبْوَاقِ, وَالْمُتَجَرِّدُونَ سَائِرُونَ أَمَامَهُمْ, وَالْبَقِيَّةُ سَائِرَةٌ وَرَاءَ تَابُوتِ الرَّبِّ. كَانُوا يَسِيرُونَ وَيَضْرِبُونَ بِالأَبْوَاقِ.
\par 14 وَدَارُوا بِالْمَدِينَةِ فِي الْيَوْمِ الثَّانِي مَرَّةً وَاحِدَةً ثُمَّ رَجَعُوا إِلَى الْمَحَلَّةِ. هَكَذَا فَعَلُوا سِتَّةَ أَيَّامٍ.
\par 15 وَكَانَ فِي الْيَوْمِ السَّابِعِ أَنَّهُمْ بَكَّرُوا عِنْدَ طُلُوعِ الْفَجْرِ وَدَارُوا دَائِرَةَ الْمَدِينَةِ عَلَى هَذَا الْمِنْوَالِ سَبْعَ مَرَّاتٍ. فِي ذَلِكَ الْيَوْمِ فَقَطْ دَارُوا دَائِرَةَ الْمَدِينَةِ سَبْعَ مَرَّاتٍ.
\par 16 وَكَانَ فِي الْمَرَّةِ السَّابِعَةِ عِنْدَمَا ضَرَبَ الْكَهَنَةُ بِالأَبْوَاقِ أَنَّ يَشُوعَ قَالَ لِلشَّعْبِ: «اهْتِفُوا, لأَنَّ الرَّبَّ قَدْ أَعْطَاكُمُ الْمَدِينَةَ.
\par 17 فَتَكُونُ الْمَدِينَةُ وَكُلُّ مَا فِيهَا مُحَرَّماً لِلرَّبِّ. رَاحَابُ الزَّانِيَةُ فَقَطْ تَحْيَا هِيَ وَكُلُّ مَنْ مَعَهَا فِي الْبَيْتِ, لأَنَّهَا قَدْ خَبَّأَتِ الْمُرْسَلَيْنِ اللَّذَيْنِ أَرْسَلْنَاهُمَا.
\par 18 وَأَمَّا أَنْتُمْ فَاحْتَرِزُوا مِنَ الْحَرَامِ لِئَلاَّ تُحَرَّمُوا وَتَأْخُذُوا مِنَ الْحَرَامِ وَتَجْعَلُوا مَحَلَّةَ إِسْرَائِيلَ مُحَرَّمَةً وَتُكَدِّرُوهَا.
\par 19 وَكُلُّ الْفِضَّةِ وَالذَّهَبِ وَآنِيَةِ النُّحَاسِ وَالْحَدِيدِ تَكُونُ قُدْساً لِلرَّبِّ وَتَدْخُلُ فِي خِزَانَةِ الرَّبِّ».
\par 20 فَهَتَفَ الشَّعْبُ وَضَرَبُوا بِالأَبْوَاقِ. وَكَانَ حِينَ سَمِعَ الشَّعْبُ صَوْتَ الْبُوقِ أَنَّ الشَّعْبَ هَتَفَ هُتَافاً عَظِيماً, فَسَقَطَ السُّورُ فِي مَكَانِهِ, وَصَعِدَ الشَّعْبُ إِلَى الْمَدِينَةِ كُلُّ رَجُلٍ مَعَ وَجْهِهِ, وَأَخَذُوا الْمَدِينَةَ.
\par 21 وَحَرَّمُوا كُلَّ مَا فِي الْمَدِينَةِ مِنْ رَجُلٍ وَامْرَأَةٍ, مِنْ طِفْلٍ وَشَيْخٍ - حَتَّى الْبَقَرَ وَالْغَنَمَ وَالْحَمِيرَ بِحَدِّ السَّيْفِ.
\par 22 وَقَالَ يَشُوعُ لِلرَّجُلَيْنِ اللَّذَيْنِ تَجَسَّسَا الأَرْضَ: «ادْخُلاَ بَيْتَ الْمَرْأَةِ الزَّانِيَةِ وَأَخْرِجَا مِنْ هُنَاكَ الْمَرْأَةَ وَكُلَّ مَا لَهَا كَمَا حَلَفْتُمَا لَهَا».
\par 23 فَدَخَلَ الْجَاسُوسَانِ وَأَخْرَجَا رَاحَابَ وَأَبَاهَا وَأُمَّهَا وَإِخْوَتَهَا وَكُلَّ مَا لَهَا, وَكُلَّ عَشَائِرِهَا وَتَرَكَاهُمْ خَارِجَ مَحَلَّةِ إِسْرَائِيلَ.
\par 24 وَأَحْرَقُوا الْمَدِينَةَ بِالنَّارِ مَعَ كُلِّ مَا بِهَا. إِنَّمَا الْفِضَّةُ وَالذَّهَبُ وَآنِيَةُ النُّحَاسِ وَالْحَدِيدِ جَعَلُوهَا فِي خِزَانَةِ بَيْتِ الرَّبِّ.
\par 25 وَاسْتَحْيَا يَشُوعُ رَاحَابَ الزَّانِيَةَ وَبَيْتَ أَبِيهَا وَكُلَّ مَا لَهَا. وَسَكَنَتْ فِي وَسَطِ إِسْرَائِيلَ إِلَى هَذَا الْيَوْمِ, لأَنَّهَا خَبَّأَتِ الْمُرْسَلَيْنِ اللَّذَيْنِ أَرْسَلَهُمَا يَشُوعُ لِيَتَجَسَّسَا أَرِيحَا.
\par 26 وَحَلَفَ يَشُوعُ فِي ذَلِكَ الْوَقْتِ قَائِلاً: «مَلْعُونٌ قُدَّامَ الرَّبِّ الرَّجُلُ الَّذِي يَقُومُ وَيَبْنِي هَذِهِ الْمَدِينَةَ أَرِيحَا. بِبِكْرِهِ يُؤَسِّسُهَا وَبِصَغِيرِهِ يَنْصِبُ أَبْوَابَهَا».
\par 27 وَكَانَ الرَّبُّ مَعَ يَشُوعَ, وَكَانَ خَبَرُهُ فِي جَمِيعِ الأَرْضِ.

\chapter{7}

\par 1 وَخَانَ بَنُو إِسْرَائِيلَ خِيَانَةً فِي الْحَرَامِ, فَأَخَذَ عَخَانُ بْنُ كَرْمِي بْنُ زَبْدِي بْنُ زَارَحَ مِنْ سِبْطِ يَهُوذَا مِنَ الْحَرَامِ, فَحَمِيَ غَضَبُ الرَّبِّ عَلَى بَنِي إِسْرَائِيلَ.
\par 2 وَأَرْسَلَ يَشُوعُ رِجَالاً مِنْ أَرِيحَا إِلَى عَايَ الَّتِي عِنْدَ بَيْتِ آوِنَ شَرْقِيَّ بَيْتِ إِيلَ, وَقَال لَهُمْ: «اصْعَدُوا تَجَسَّسُوا الأَرْضَ». فَصَعِدَ الرِّجَالُ وَتَجَسَّسُوا عَايَ.
\par 3 ثُمَّ رَجَعُوا إِلَى يَشُوعَ وَقَالُوا لَهُ: «لاَ يَصْعَدْ كُلُّ الشَّعْبِ, بَلْ يَصْعَدْ نَحْوُ أَلْفَيْ رَجُلٍ أَوْ ثَلاَثَةُ آلاَفِ رَجُلٍ وَيَضْرِبُوا عَايَ. لاَ تُكَلِّفْ كُلَّ الشَّعْبِ إِلَى هُنَاكَ لأَنَّهُمْ قَلِيلُونَ».
\par 4 فَصَعِدَ مِنَ الشَّعْبِ إِلَى هُنَاكَ نَحْوُ ثَلاَثَةِ آلاَفِ رَجُلٍ. وَهَرَبُوا أَمَامَ أَهْلِ عَايَ.
\par 5 فَضَرَبَ مِنْهُمْ أَهْلُ عَايَ نَحْوَ سِتَّةٍ وَثَلاَثِينَ رَجُلاً, وَلَحِقُوهُمْ مِنْ أَمَامِ الْبَابِ إِلَى شَبَارِيمَ وَضَرَبُوهُمْ فِي الْمُنْحَدَرِ. فَذَابَ قَلْبُ الشَّعْبِ وَصَارَ مِثْلَ الْمَاءِ.
\par 6 فَمَزَّقَ يَشُوعُ ثِيَابَهُ وَسَقَطَ عَلَى وَجْهِهِ إِلَى الأَرْضِ أَمَامَ تَابُوتِ الرَّبِّ إِلَى الْمَسَاءِ, هُوَ وَشُيُوخُ إِسْرَائِيلَ, وَوَضَعُوا تُرَاباً عَلَى رُؤُوسِهِمْ.
\par 7 وَقَالَ يَشُوعُ: «آهِ يَا سَيِّدُ الرَّبُّ! لِمَاذَا عَبَّرْتَ هَذَا الشَّعْبَ الأُرْدُنَّ تَعْبِيراً لِكَيْ تَدْفَعَنَا إِلَى يَدِ الأَمُورِيِّينَ لِيُبِيدُونَا؟ لَيْتَنَا ارْتَضَيْنَا وَسَكَنَّا فِي عَبْرِ الأُرْدُنِّ.
\par 8 أَسْأَلُكَ يَا سَيِّدُ: مَاذَا أَقُولُ بَعْدَمَا حَوَّلَ إِسْرَائِيلُ قَفَاهُ أَمَامَ أَعْدَائِهِ؟
\par 9 فَيَسْمَعُ الْكَنْعَانِيُّونَ وَجَمِيعُ سُكَّانِ الأَرْضِ وَيُحِيطُونَ بِنَا وَيَقْرِضُونَ اسْمَنَا مِنَ الأَرْضِ. وَمَاذَا تَصْنَعُ لاِسْمِكَ الْعَظِيمِ؟».
\par 10 فَقَالَ الرَّبُّ لِيَشُوعَ: «قُمْ! لِمَاذَا أَنْتَ سَاقِطٌ عَلَى وَجْهِكَ؟
\par 11 قَدْ أَخْطَأَ إِسْرَائِيلُ, بَلْ تَعَدُّوا عَهْدِي الَّذِي أَمَرْتُهُمْ بِهِ, بَلْ أَخَذُوا مِنَ الْحَرَامِ, بَلْ سَرِقُوا, بَلْ أَنْكَرُوا, بَلْ وَضَعُوا فِي أَمْتِعَتِهِمْ.
\par 12 فَلَمْ يَتَمَكَّنْ بَنُو إِسْرَائِيلَ لِلثُّبُوتِ أَمَامَ أَعْدَائِهِمْ. يُدِيرُونَ قَفَاهُمْ أَمَامَ أَعْدَائِهِمْ لأَنَّهُمْ مَحْرُومُونَ, وَلاَ أَعُودُ أَكُونُ مَعَكُمْ إِنْ لَمْ تُبِيدُوا الْحَرَامَ مِنْ وَسَطِكُمْ.
\par 13 قُمْ قَدِّسِ الشَّعْبَ وَقُلْ: تَقَدَّسُوا لِلْغَدِ. لأَنَّهُ هَكَذَا قَالَ الرَّبُّ إِلَهُ إِسْرَائِيلَ: فِي وَسَطِكَ حَرَامٌ يَا إِسْرَائِيلُ, فَلاَ تَتَمَكَّنُ لِلثُّبُوتِ أَمَامَ أَعْدَائِكَ حَتَّى تَنْزِعُوا الْحَرَامَ مِنْ وَسَطِكُمْ.
\par 14 فَتَتَقَدَّمُونَ فِي الْغَدِ بِأَسْبَاطِكُمْ, وَيَكُونُ أَنَّ السِّبْطَ الَّذِي يَأْخُذُهُ الرَّبُّ يَتَقَدَّمُ بِعَشَائِرِهِ, وَالْعَشِيرَةُ الَّتِي يَأْخُذُهَا الرَّبُّ تَتَقَدَّمُ بِبُيُوتِهَا, وَالْبَيْتُ الَّذِي يَأْخُذُهُ الرَّبُّ يَتَقَدَّمُ بِرِجَالِهِ.
\par 15 وَيَكُونُ الْمَأْخُوذُ بِالْحَرَامِ يُحْرَقُ بِالنَّارِ هُوَ وَكُلُّ مَا لَهُ, لأَنَّهُ تَعَدَّى عَهْدَ الرَّبِّ, وَلأَنَّهُ عَمِلَ قَبَاحَةً فِي إِسْرَائِيلَ».
\par 16 فَبَكَّرَ يَشُوعُ فِي الْغَدِ وَقَدَّمَ إِسْرَائِيلَ بِأَسْبَاطِهِ, فَأُخِذَ سِبْطُ يَهُوذَا.
\par 17 ثُمَّ قَدَّمَ قَبِيلَةَ يَهُوذَا فَأُخِذَتْ عَشِيرَةُ الزَّارَحِيِّينَ. ثُمَّ قَدَّمَ عَشِيرَةَ الزَّارَحِيِّينَ بِرِجَالِهِمْ فَأُخِذَ زَبْدِي.
\par 18 فَقَدَّمَ بَيْتَهُ بِرِجَالِهِ فَأُخِذَ عَخَانُ بْنُ كَرْمِي بْنِ زَبْدِي بْنِ زَارَحَ مِنْ سِبْطِ يَهُوذَا.
\par 19 فَقَالَ يَشُوعُ لِعَخَانَ: «يَا ابْنِي, أَعْطِ الآنَ مَجْداً لِلرَّبِّ إِلَهِ إِسْرَائِيلَ, وَاعْتَرِفْ لَهُ وَأَخْبِرْنِي الآنَ مَاذَا عَمِلْتَ. لاَ تُخْفِ عَنِّي».
\par 20 فَأَجَابَ عَاخَانُ يَشُوعَ: «حَقّاً إِنِّي قَدْ أَخْطَأْتُ إِلَى الرَّبِّ إِلَهِ إِسْرَائِيلَ وَصَنَعْتُ كَذَا وَكَذَا.
\par 21 رَأَيْتُ فِي الْغَنِيمَةِ رِدَاءً شِنْعَارِيّاً نَفِيساً, وَمِئَتَيْ شَاقِلِ فِضَّةٍ, وَلِسَانَ ذَهَبٍ وَزْنُهُ خَمْسُونَ شَاقِلاً, فَاشْتَهَيْتُهَا وَأَخَذْتُهَا. وَهَا هِيَ مَطْمُورَةٌ فِي الأَرْضِ فِي وَسَطِ خَيْمَتِي, وَالْفِضَّةُ تَحْتَهَا».
\par 22 فَأَرْسَلَ يَشُوعُ رُسُلاً فَرَكَضُوا إِلَى الْخَيْمَةِ وَإِذَا هِيَ مَطْمُورَةٌ فِي خَيْمَتِهِ وَالْفِضَّةُ تَحْتَهَا.
\par 23 فَأَخَذُوهَا مِنْ وَسَطِ الْخَيْمَةِ وَأَتُوا بِهَا إِلَى يَشُوعَ وَإِلَى جَمِيعِ بَنِي إِسْرَائِيلَ, وَبَسَطُوهَا أَمَامَ الرَّبِّ.
\par 24 فَأَخَذَ يَشُوعُ عَخَانَ بْنَ زَارَحَ وَالْفِضَّةَ وَالرِّدَاءَ وَلِسَانَ الذَّهَبِ وَبَنِيهِ وَبَنَاتِهِ وَبَقَرَهُ وَحَمِيرَهُ وَغَنَمَهُ وَخَيْمَتَهُ وَكُلَّ مَا لَهُ, وَجَمِيعُ إِسْرَائِيلَ مَعَهُ, وَصَعِدُوا بِهِمْ إِلَى وَادِي عَخُورَ.
\par 25 فَقَالَ يَشُوعُ: «كَيْفَ كَدَّرْتَنَا؟ يُكَدِّرُكَ الرَّبُّ فِي هَذَا الْيَوْمِ!» فَرَجَمَهُ جَمِيعُ إِسْرَائِيلَ بِالْحِجَارَةِ وَأَحْرَقُوهُمْ بِالنَّارِ وَرَمُوهُمْ بِالْحِجَارَةِ
\par 26 وَأَقَامُوا فَوْقَهُ رُجْمَةَ حِجَارَةٍ عَظِيمَةً إِلَى هَذَا الْيَوْمِ. فَرَجَعَ الرَّبُّ عَنْ حُمُوِّ غَضَبِهِ. وَلِذَلِكَ دُعِيَ اسْمُ ذَلِكَ الْمَكَانِ «وَادِيَ عَخُورَ» إِلَى هَذَا الْيَوْمِ.

\chapter{8}

\par 1 فَقَالَ الرَّبُّ لِيَشُوعَ: «لاَ تَخَفْ وَلاَ تَرْتَعِبْ. خُذْ مَعَكَ جَمِيعَ رِجَالِ الْحَرْبِ, وَقُمِ اصْعَدْ إِلَى عَايَ. انْظُرْ. قَدْ دَفَعْتُ بِيَدِكَ مَلِكَ عَايٍ وَشَعْبَهُ وَمَدِينَتَهُ وَأَرْضَهُ,
\par 2 فَتَفْعَلُ بِعَايٍ وَمَلِكِهَا كَمَا فَعَلْتَ بِأَرِيحَا وَمَلِكِهَا. غَيْرَ أَنَّ غَنِيمَتَهَا وَبَهَائِمَهَا تَنْهَبُونَهَا لِنُفُوسِكُمُ. اجْعَلْ كَمِيناً لِلْمَدِينَةِ مِنْ وَرَائِهَا».
\par 3 فَقَامَ يَشُوعُ وَجَمِيعُ رِجَالِ الْحَرْبِ لِلصُّعُودِ إِلَى عَايٍ. وَانْتَخَبَ يَشُوعُ ثَلاَثِينَ أَلْفَ رَجُلٍ جَبَابِرَةَ الْبَأْسِ وَأَرْسَلَهُمْ لَيْلاً,
\par 4 وَأَوْصَاهُمْ: «انْظُرُوا! أَنْتُمْ تَكْمُنُونَ لِلْمَدِينَةِ مِنْ وَرَاءِ الْمَدِينَةِ. لاَ تَبْتَعِدُوا مِنَ الْمَدِينَةِ كَثِيراً, وَكُونُوا كُلُّكُمْ مُسْتَعِدِّينَ.
\par 5 وَأَمَّا أَنَا وَجَمِيعُ الشَّعْبِ الَّذِي مَعِي فَنَقْتَرِبُ إِلَى الْمَدِينَةِ. وَيَكُونُ حِينَمَا يَخْرُجُونَ لِلِقَائِنَا كَمَا فِي الأَوَّلِ أَنَّنَا نَهْرُبُ قُدَّامَهُمْ,
\par 6 فَيَخْرُجُونَ وَرَاءَنَا حَتَّى نَجْذِبَهُمْ عَنِ الْمَدِينَةِ. لأَنَّهُمْ يَقُولُونَ إِنَّهُمْ هَارِبُونَ أَمَامَنَا كَمَا فِي الأَوَّلِ. فَنَهْرُبُ قُدَّامَهُمْ.
\par 7 وَأَنْتُمْ تَقُومُونَ مِنَ الْمَكْمَنِ وَتَمْلِكُونَ الْمَدِينَةَ, وَيَدْفَعُهَا الرَّبُّ إِلَهُكُمْ بِيَدِكُمْ.
\par 8 وَيَكُونُ عِنْدَ أَخْذِكُمُ الْمَدِينَةَ أَنَّكُمْ تُضْرِمُونَ الْمَدِينَةَ بِالنَّارِ. كَقَوْلِ الرَّبِّ تَفْعَلُونَ. انْظُرُوا. قَدْ أَوْصَيْتُكُمْ».
\par 9 فَأَرْسَلَهُمْ يَشُوعُ, فَسَارُوا إِلَى الْمَكْمَنِ, وَلَبِثُوا بَيْنَ بَيْتِ إِيلٍَ وَعَايٍ غَرْبِيَّ عَايٍ. وَبَاتَ يَشُوعُ تِلْكَ اللَّيْلَةَ فِي وَسَطِ الشَّعْبِ.
\par 10 فَبَكَّرَ يَشُوعُ فِي الْغَدِ وَعَدَّ الشَّعْبَ, وَصَعِدَ هُوَ وَشُيُوخُ إِسْرَائِيلَ قُدَّامَ الشَّعْبِ إِلَى عَايٍ.
\par 11 وَجَمِيعُ رِجَالِ الْحَرْبِ الَّذِينَ مَعَهُ صَعِدُوا وَتَقَدَّمُوا وَأَتُوا إِلَى مُقَابِلِ الْمَدِينَةِ. وَنَزَلُوا شِمَالِيَّ عَايٍ, وَالْوَادِي بَيْنَهُمْ وَبَيْنَ عَايٍ.
\par 12 فَأَخَذَ نَحْوَ خَمْسَةِ آلاَفِ رَجُلٍ وَجَعَلَهُمْ كَمِيناً بَيْنَ بَيْتِ إِيلٍَ وَعَايٍ غَرْبِيَّ الْمَدِينَةِ.
\par 13 وَأَقَامُوا الشَّعْبَ, أَيْ كُلَّ الْجَيْشِ الَّذِي شِمَالِيَّ الْمَدِينَةِ, وَكَمِينَهُ غَرْبِيَّ الْمَدِينَةِ. وَسَارَ يَشُوعُ تِلْكَ اللَّيْلَةَ إِلَى وَسَطِ الْوَادِي.
\par 14 وَكَانَ لَمَّا رَأَى مَلِكُ عَايٍ ذَلِكَ أَنَّهُمْ أَسْرَعُوا وَبَكَّرُوا, وَخَرَجَ رِجَالُ الْمَدِينَةِ لِلِقَاءِ إِسْرَائِيلَ لِلْحَرْبِ هُوَ وَجَمِيعُ شَعْبِهِ فِي الْمِيعَادِ إِلَى قُدَّامِ السَّهْلِ, وَهُوَ لاَ يَعْلَمُ أَنَّ عَلَيْهِ كَمِيناً وَرَاءَ الْمَدِينَةِ.
\par 15 فَأَعْطَى يَشُوعُ وَجَمِيعُ إِسْرَائِيلَ انْكِسَاراً أَمَامَهُمْ وَهَرَبُوا فِي طَرِيقِ الْبَرِّيَّةِ.
\par 16 فَأُلْقِيَ الصَّوْتُ عَلَى جَمِيعِ الشَّعْبِ الَّذِينَ فِي الْمَدِينَةِ لِلسَّعْيِ وَرَاءَهُمْ, فَسَعُوا وَرَاءَ يَشُوعَ وَانْجَذَبُوا عَنِ الْمَدِينَةِ.
\par 17 وَلَمْ يَبْقَ فِي عَايٍ أَوْ فِي بَيْتِ إِيلٍ رَجُلٌ لَمْ يَخْرُجْ وَرَاءَ إِسْرَائِيلَ. فَتَرَكُوا الْمَدِينَةَ مَفْتُوحَةً وَسَعُوا وَرَاءَ إِسْرَائِيلَ.
\par 18 فَقَالَ الرَّبُّ لِيَشُوعَ: «مُدَّ الْمِزْرَاقَ الَّذِي بِيَدِكَ نَحْوَ عَايٍ لأَنِّي بِيَدِكَ أَدْفَعُهَا». فَمَدَّ يَشُوعُ الْمِزْرَاقَ الَّذِي بِيَدِهِ نَحْوَ الْمَدِينَةِ.
\par 19 فَقَامَ الْكَمِينُ بِسُرْعَةٍ مِنْ مَكَانِهِ وَرَكَضُوا عِنْدَمَا مَدَّ يَدَهُ, وَدَخَلُوا الْمَدِينَةَ وَأَخَذُوهَا, وَأَسْرَعُوا وَأَحْرَقُوا الْمَدِينَةَ بِالنَّارِ.
\par 20 فَالْتَفَتَ رِجَالُ عَايٍ إِلَى وَرَائِهِمْ وَنَظَرُوا وَإِذَا دُخَانُ الْمَدِينَةِ قَدْ صَعِدَ إِلَى السَّمَاءِ. فَلَمْ يَكُنْ لَهُمْ مَكَانٌ لِلْهَرَبِ هُنَا أَوْ هُنَاكَ. وَالشَّعْبُ الْهَارِبُ إِلَى الْبَرِّيَّةِ انْقَلَبَ عَلَى الطَّارِدِ.
\par 21 وَلَمَّا رَأَى يَشُوعُ وَجَمِيعُ إِسْرَائِيلَ أَنَّ الْكَمِينَ قَدْ أَخَذَ الْمَدِينَةَ, وَأَنَّ دُخَانَ الْمَدِينَةِ قَدْ صَعِدَ, انْثَنَوْا وَضَرَبُوا رِجَالَ عَايٍ.
\par 22 وَهَؤُلاَءِ خَرَجُوا مِنَ الْمَدِينَةِ لِلِقَائِهِمْ, فَكَانُوا فِي وَسَطِ إِسْرَائِيلَ, هَؤُلاَءِ مِنْ هُنَا وَأُولَئِكَ مِنْ هُنَاكَ. وَضَرَبُوهُمْ حَتَّى لَمْ يَبْقَ مِنْهُمْ شَارِدٌ وَلاَ مُنْفَلِتٌ.
\par 23 وَأَمَّا مَلِكُ عَايٍ فَأَمْسَكُوهُ حَيّاً وَتَقَدَّمُوا بِهِ إِلَى يَشُوعَ.
\par 24 وَكَانَ لَمَّا انْتَهَى إِسْرَائِيلُ مِنْ قَتْلِ جَمِيعِ سُكَّانِ عَايٍ فِي الْحَقْلِ فِي الْبَرِّيَّةِ حَيْثُ لَحِقُوهُمْ, وَسَقَطُوا جَمِيعاً بِحَدِّ السَّيْفِ حَتَّى فَنُوا أَنَّ جَمِيعَ إِسْرَائِيلَ رَجَعَ إِلَى عَايٍ وَضَرَبُوهَا بِحَدِّ السَّيْفِ.
\par 25 فَكَانَ جَمِيعُ الَّذِينَ سَقَطُوا فِي ذَلِكَ الْيَوْمِ مِنْ رِجَالٍ وَنِسَاءٍ اثْنَيْ عَشَرَ أَلْفاً, جَمِيعُ أَهْلِ عَايٍ.
\par 26 وَيَشُوعُ لَمْ يَرُدَّ يَدَهُ الَّتِي مَدَّهَا بِالْحَرْبَةِ حَتَّى حَرَّمَ جَمِيعَ سُكَّانِ عَايٍ.
\par 27 لَكِنِ الْبَهَائِمُ وَغَنِيمَةُ تِلْكَ الْمَدِينَةِ نَهَبَهَا إِسْرَائِيلُ لأَنْفُسِهِمْ حَسَبَ قَوْلِ الرَّبِّ الَّذِي أَمَرَ بِهِ يَشُوعَ.
\par 28 وَأَحْرَقَ يَشُوعُ عَايَ وَجَعَلَهَا تَلاًّ أَبَدِيّاً خَرَاباً إِلَى هَذَا الْيَوْمِ.
\par 29 وَمَلِكُ عَايٍ عَلَّقَهُ عَلَى الْخَشَبَةِ إِلَى وَقْتِ الْمَسَاءِ. وَعِنْدَ غُرُوبِ الشَّمْسِ أَمَرَ يَشُوعُ فَأَنْزَلُوا جُثَّتَهُ عَنِ الْخَشَبَةِ وَطَرَحُوهَا عِنْدَ مَدْخَلِ بَابِ الْمَدِينَةِ, وَأَقَامُوا عَلَيْهَا رُجْمَةَ حِجَارَةٍ عَظِيمَةً إِلَى هَذَا الْيَوْمِ.
\par 30 حِينَئِذٍ بَنَى يَشُوعُ مَذْبَحاً لِلرَّبِّ إِلَهِ إِسْرَائِيلَ فِي جَبَلِ عِيبَالَ,
\par 31 كَمَا أَمَرَ مُوسَى عَبْدُ الرَّبِّ بَنِي إِسْرَائِيلَ, كَمَا هُوَ مَكْتُوبٌ فِي سِفْرِ تَوْرَاةِ مُوسَى. مَذْبَحَ حِجَارَةٍ صَحِيحَةٍ لَمْ يَرْفَعْ أَحَدٌ عَلَيْهَا حَدِيداً, وَأَصْعَدُوا عَلَيْهِ مُحْرَقَاتٍ لِلرَّبِّ, وَذَبَحُوا ذَبَائِحَ سَلاَمَةٍ.
\par 32 وَكَتَبَ هُنَاكَ عَلَى الْحِجَارَةِ نُسْخَةَ تَوْرَاةِ مُوسَى الَّتِي كَتَبَهَا أَمَامَ بَنِي إِسْرَائِيلَ.
\par 33 وَجَمِيعُ إِسْرَائِيلَ وَشُيُوخُهُمْ, وَالْعُرَفَاءُ وَقُضَاتُهُمْ, وَقَفُوا جَانِبَ التَّابُوتِ مِنْ هُنَا وَمِنْ هُنَاكَ مُقَابِلَ الْكَهَنَةِ اللاَّوِيِّينَ حَامِلِي تَابُوتِ عَهْدِ الرَّبِّ. الْغَرِيبُ كَمَا الْوَطَنِيُّ. نِصْفُهُمْ إِلَى جِهَةِ جَبَلِ جِرِزِّيمَ, وَنِصْفُهُمْ إِلَى جِهَةِ جَبَلِ عِيبَالَ, كَمَا أَمَرَ مُوسَى عَبْدُ الرَّبِّ أَوَّلاً لِبَرَكَةِ شَعْبِ إِسْرَائِيلَ.
\par 34 وَبَعْدَ ذَلِكَ قَرَأَ جَمِيعَ كَلاَمِ التَّوْرَاةِ: الْبَرَكَةَ وَاللَّعْنَةَ, حَسَبَ كُلِّ مَا كُتِبَ فِي سِفْرِ التَّوْرَاةِ.
\par 35 لَمْ تَكُنْ كَلِمَةٌ مِنْ كُلِّ مَا أَمَرَ بِهِ مُوسَى لَمْ يَقْرَأْهَا يَشُوعُ قُدَّامَ كُلِّ جَمَاعَةِ إِسْرَائِيلَ وَالنِّسَاءِ وَالأَطْفَالِ وَالْغَرِيبِ السَّائِرِ فِي وَسَطِهِمْ.

\chapter{9}

\par 1 وَلَمَّا سَمِعَ جَمِيعُ الْمُلُوكِ الَّذِينَ فِي عَبْرِ الأُرْدُنِّ فِي الْجَبَلِ وَفِي السَّهْلِ وَفِي كُلِّ سَاحِلِ الْبَحْرِ الْكَبِيرِ إِلَى جِهَةِ لُبْنَانَ, الْحِثِّيُّونَ وَالأَمُورِيُّونَ وَالْكَنْعَانِيُّونَ وَالْفِرِزِّيُّونَ وَالْحِوِّيُّونَ وَالْيَبُوسِيُّونَ,
\par 2 اجْتَمَعُوا مَعاً لِمُحَارَبَةِ يَشُوعَ وَإِسْرَائِيلَ بِصَوْتٍ وَاحِدٍ.
\par 3 وَأَمَّا سُكَّانُ جِبْعُونَ لَمَّا سَمِعُوا بِمَا عَمِلَهُ يَشُوعُ بِأَرِيحَا وَعَايٍ
\par 4 عَمِلُوا بِغَدْرٍ, وَمَضُوا وَدَارُوا وَأَخَذُوا جَوَالِقَ بَالِيَةً لِحَمِيرِهِمْ, وَزِقَاقَ خَمْرٍ بَالِيَةً مُشَقَّقَةً وَمَرْبُوطَةً,
\par 5 وَنِعَالاً بَالِيَةً وَمُرَقَّعَةً فِي أَرْجُلِهِمْ, وَثِيَاباً رَثَّةً عَلَيْهِمْ, وَكُلُّ خُبْزِ زَادِهِمْ يَابِسٌ قَدْ صَارَ فُتَاتَاً.
\par 6 وَسَارُوا إِلَى يَشُوعَ إِلَى الْمَحَلَّةِ فِي الْجِلْجَالِ, وَقَالُوا لَهُ وَلِرِجَالِ إِسْرَائِيلَ: «مِنْ أَرْضٍ بَعِيدَةٍ جِئْنَا. وَالآنَ اقْطَعُوا لَنَا عَهْداً».
\par 7 فَقَالَ رِجَالُ إِسْرَائِيلَ لِلْحِوِّيِّينَ: «لَعَلَّكَ سَاكِنٌ فِي وَسَطِي, فَكَيْفَ أَقْطَعُ لَكَ عَهْداً؟»
\par 8 فَقَالُوا لِيَشُوعَ: «عَبِيدُكَ نَحْنُ». فَقَالَ لَهُمْ يَشُوعُ: «مَنْ أَنْتُمْ, وَمِنْ أَيْنَ جِئْتُمْ؟»
\par 9 فَقَالُوا لَهُ: «مِنْ أَرْضٍ بَعِيدَةٍ جِدّاً جَاءَ عَبِيدُكَ عَلَى اسْمِ الرَّبِّ إِلَهِكَ, لأَنَّنَا سَمِعْنَا خَبَرَهُ وَكُلَّ مَا عَمِلَ بِمِصْرَ
\par 10 وَكُلَّ مَا عَمِلَ بِمَلِكَيِ الأَمُورِيِّينَ اللَّذَيْنِ فِي عَبْرِ الأُرْدُنِّ, سِيحُونَ مَلِكِ حَشْبُونَ وَعُوجَ مَلِكِ بَاشَانَ الَّذِي فِي عَشْتَارُوثَ.
\par 11 فَكَلَّمَنَا شُيُوخُنَا وَجَمِيعُ سُكَّانِ أَرْضِنَا قَائِلِينَ: خُذُوا بِأَيْدِيكُمْ زَاداً لِلطَّرِيقِ, وَاذْهَبُوا لِلِقَائِهِمْ وَقُولُوا لَهُمْ: عَبِيدُكُمْ نَحْنُ. وَالآنَ اقْطَعُوا لَنَا عَهْداً.
\par 12 هَذَا خُبْزُنَا سُخْناً تَزَوَّدْنَاهُ مِنْ بُيُوتِنَا يَوْمَ خُرُوجِنَا لِنَسِيرَ إِلَيْكُمْ, وَهَا هُوَ الآنَ يَابِسٌ قَدْ صَارَ فُتَاتَاً.
\par 13 وَهَذِهِ زِقَاقُ الْخَمْرِ الَّتِي مَلَأْنَاهَا جَدِيدَةً, هُوَذَا قَدْ تَشَقَّقَتْ. وَهَذِهِ ثِيَابُنَا وَنِعَالُنَا قَدْ بَلِيَتْ مِنْ طُولِ الطَّرِيقِ جِدّاً.
\par 14 فَأَكَلَ الرِّجَالُ مِنْ زَادِهِمْ, وَمِنْ فَمِ الرَّبِّ لَمْ يَسْأَلُوا.
\par 15 فَعَمِلَ يَشُوعُ لَهُمْ صُلْحاً وَقَطَعَ لَهُمْ عَهْداً لاِسْتِحْيَائِهِمْ, وَحَلَفَ لَهُمْ رُؤَسَاءُ الْجَمَاعَةِ.
\par 16 وَفِي نِهَايَةِ ثَلاَثَةِ أَيَّامٍ بَعْدَمَا قَطَعُوا لَهُمْ عَهْداً سَمِعُوا أَنَّهُمْ قَرِيبُونَ إِلَيْهِمْ وَأَنَّهُمْ سَاكِنُونَ فِي وَسَطِهِمْ.
\par 17 فَارْتَحَلَ بَنُو إِسْرَائِيلَ وَجَاءُوا إِلَى مُدُنِهِمْ فِي الْيَوْمِ الثَّالِثِ. وَمُدُنُهُمْ هِيَ جِبْعُونُ وَالْكَفِيرَةُ وَبَئِيرُوتُ وَقَرْيَةُ يَعَارِيمَ.
\par 18 وَلَمْ يَضْرِبْهُمْ بَنُو إِسْرَائِيلَ لأَنَّ رُؤَسَاءَ الْجَمَاعَةِ حَلَفُوا لَهُمْ بِالرَّبِّ إِلَهِ إِسْرَائِيلَ. فَتَذَمَّرَ كُلُّ الْجَمَاعَةِ عَلَى الرُّؤَسَاءِ.
\par 19 فَقَالَ جَمِيعُ الرُّؤَسَاءِ لِكُلِّ الْجَمَاعَةِ: «إِنَّنَا قَدْ حَلَفْنَا لَهُمْ بِالرَّبِّ إِلَهِ إِسْرَائِيلَ. وَالآنَ لاَ نَتَمَكَّنُ مِنْ مَسِّهِمْ.
\par 20 هَذَا نَصْنَعُهُ لَهُمْ وَنَسْتَحْيِيهِمْ فَلاَ يَكُونُ عَلَيْنَا سَخَطٌ مِنْ أَجْلِ الْحَلْفِ الَّذِي حَلَفْنَا لَهُمْ».
\par 21 وَقَالَ لَهُمُ الرُّؤَسَاءُ: «يَحْيُونَ وَيَكُونُونَ مُحْتَطِبِي حَطَبٍ وَمُسْتَقِي مَاءٍ لِكُلِّ الْجَمَاعَةِ كَمَا كَلَّمَهُمُ الرُّؤَسَاءُ».
\par 22 فَدَعَاهُمْ يَشُوعُ وَقَالَ لَهُمْ: «لِمَاذَا خَدَعْتُمُونَا قَائِلِينَ: نَحْنُ بَعِيدُونَ عَنْكُمْ جِدّاً, وَأَنْتُمْ سَاكِنُونَ فِي وَسَطِنَا؟
\par 23 فَالآنَ مَلْعُونُونَ أَنْتُمْ. فَلاَ يَنْقَطِعُ مِنْكُمُ الْعَبِيدُ وَمُحْتَطِبُو الْحَطَبِ وَمُسْتَقُو الْمَاءِ لِبَيْتِ إِلَهِي».
\par 24 فَأَجَابُوا يَشُوعَ: «أُخْبِرَ عَبِيدُكَ بِمَا أَمَرَ بِهِ الرَّبُّ إِلَهُكَ مُوسَى عَبْدَهُ أَنْ يُعْطِيَكُمْ كُلَّ الأَرْضِ, وَيُبِيدَ جَمِيعَ سُكَّانِ الأَرْضِ مِنْ أَمَامِكُمْ. فَخِفْنَا جِدّاً عَلَى أَنْفُسِنَا مِنْ قِبَلِكُمْ, فَفَعَلْنَا هَذَا الأَمْرَ.
\par 25 وَالآنَ نَحْنُ بِيَدِكَ, فَافْعَلْ بِنَا مَا هُوَ صَالِحٌ وَحَقٌّ فِي عَيْنَيْكَ أَنْ تَعْمَلَ».
\par 26 فَفَعَلَ بِهِمْ هَكَذَا, وَأَنْقَذَهُمْ مِنْ يَدِ بَنِي إِسْرَائِيلَ فَلَمْ يَقْتُلُوهُمْ.
\par 27 وَجَعَلَهُمْ يَشُوعُ فِي ذَلِكَ الْيَوْمِ مُحْتَطِبِي حَطَبٍ وَمُسْتَقِي مَاءٍ لِلْجَمَاعَةِ وَلِمَذْبَحِ الرَّبِّ إِلَى هَذَا الْيَوْمِ, فِي الْمَكَانِ الَّذِي يَخْتَارُهُ.

\chapter{10}

\par 1 فَلَمَّا سَمِعَ أَدُونِي صَادَقَ مَلِكُ أُورُشَلِيمَ أَنَّ يَشُوعَ قَدْ أَخَذَ عَايَ وَحَرَّمَهَا. كَمَا فَعَلَ بِأَرِيحَا وَمَلِكِهَا فَعَلَ بِعَايٍ وَمَلِكِهَا, وَأَنَّ سُكَّانَ جِبْعُونَ قَدْ صَالَحُوا إِسْرَائِيلَ وَكَانُوا فِي وَسَطِهِمْ,
\par 2 خَافَ جِدّاً, لأَنَّ جِبْعُونَ مَدِينَةٌ عَظِيمَةٌ كَإِحْدَى الْمُدُنِ الْمَلَكِيَّةِ, وَهِيَ أَعْظَمُ مِنْ عَايٍ, وَكُلُّ رِجَالِهَا جَبَابِرَةٌ.
\par 3 فَأَرْسَلَ أَدُونِي صَادِقَ مَلِكُ أُورُشَلِيمَ إِلَى هُوهَامَ مَلِكِ حَبْرُونَ, وَفِرْآمَ مَلِكِ يَرْمُوتَ, وَيَافِيعَ مَلِكِ لَخِيشَ, وَدَبِيرَ مَلِكِ عَجْلُونَ يَقُولُ:
\par 4 «اصْعَدُوا إِلَيَّ وَأَعِينُونِي, فَنَضْرِبَ جِبْعُونَ لأَنَّهَا صَالَحَتْ يَشُوعَ وَبَنِي إِسْرَائِيلَ».
\par 5 فَاجْتَمَعَ مُلُوكُ الأَمُورِيِّينَ الْخَمْسَةُ: مَلِكُ أُورُشَلِيمَ وَمَلِكُ حَبْرُونَ وَمَلِكُ يَرْمُوتَ وَمَلِكُ لَخِيشَ وَمَلِكُ عَجْلُونَ, وَصَعِدُوا هُمْ وَكُلُّ جُيُوشِهِمْ وَنَزَلُوا عَلَى جِبْعُونَ وَحَارَبُوهَا.
\par 6 فَأَرْسَلَ أَهْلُ جِبْعُونَ إِلَى يَشُوعَ إِلَى الْمَحَلَّةِ فِي الْجِلْجَالِ يَقُولُونَ: «لاَ تُرْخِ يَدَيْكَ عَنْ عَبِيدِكَ. اصْعَدْ إِلَيْنَا عَاجِلاً وَخَلِّصْنَا وَأَعِنَّا, لأَنَّهُ قَدِ اجْتَمَعَ عَلَيْنَا جَمِيعُ مُلُوكِ الأَمُورِيِّينَ السَّاكِنِينَ فِي الْجَبَلِ».
\par 7 فَصَعِدَ يَشُوعُ مِنَ الْجِلْجَالِ هُوَ وَجَمِيعُ رِجَالِ الْحَرْبِ مَعَهُ وَكُلُّ جَبَابِرَةِ الْبَأْسِ.
\par 8 فَقَالَ الرَّبُّ لِيَشُوعَ: «لاَ تَخَفْهُمْ, لأَنِّي بِيَدِكَ قَدْ أَسْلَمْتُهُمْ. لاَ يَقِفُ رَجُلٌ مِنْهُمْ بِوَجْهِكَ».
\par 9 فَأَتَى إِلَيْهِمْ يَشُوعُ بَغْتَةً. صَعِدَ اللَّيْلَ كُلَّهُ مِنَ الْجِلْجَالِ.
\par 10 فَأَزْعَجَهُمُ الرَّبُّ أَمَامَ إِسْرَائِيلَ, وَضَرَبَهُمْ ضَرْبَةً عَظِيمَةً فِي جِبْعُونَ, وَطَرَدَهُمْ فِي طَرِيقِ عَقَبَةِ بَيْتِ حُورُونَ, وَضَرَبَهُمْ إِلَى عَزِيقَةَ وَإِلَى مَقِّيدَةَ.
\par 11 وَبَيْنَمَا هُمْ هَارِبُونَ مِنْ أَمَامِ إِسْرَائِيلَ وَهُمْ فِي مُنْحَدَرِ بَيْتِ حُورُونَ, رَمَاهُمُ الرَّبُّ بِحِجَارَةٍ عَظِيمَةٍ مِنَ السَّمَاءِ إِلَى عَزِيقَةَ فَمَاتُوا. وَالَّذِينَ مَاتُوا بِحِجَارَةِ الْبَرَدِ هُمْ أَكْثَرُ مِنَ الَّذِينَ قَتَلَهُمْ بَنُو إِسْرَائِيلَ بِالسَّيْفِ.
\par 12 حِينَئِذٍ قَالَ يَشُوعُ لِلرَّبَّ, يَوْمَ أَسْلَمَ الرَّبُّ الأَمُورِيِّينَ أَمَامَ بَنِي إِسْرَائِيلَ, أَمَامَ عُيُونِ إِسْرَائِيلَ: «يَا شَمْسُ دُومِي عَلَى جِبْعُونَ, وَيَا قَمَرُ عَلَى وَادِي أَيَّلُونَ».
\par 13 فَدَامَتِ الشَّمْسُ وَوَقَفَ الْقَمَرُ حَتَّى انْتَقَمَ الشَّعْبُ مِنْ أَعْدَائِهِ. أَلَيْسَ هَذَا مَكْتُوباً فِي سِفْرِ يَاشَرَ؟ فَوَقَفَتِ الشَّمْسُ فِي كَبِدِ السَّمَاءِ وَلَمْ تَعْجَلْ لِلْغُرُوبِ نَحْوَ يَوْمٍ كَامِلٍ.
\par 14 وَلَمْ يَكُنْ مِثْلُ ذَلِكَ الْيَوْمِ قَبْلَهُ وَلاَ بَعْدَهُ سَمِعَ فِيهِ الرَّبُّ صَوْتَ إِنْسَانٍ. لأَنَّ الرَّبَّ حَارَبَ عَنْ إِسْرَائِيلَ.
\par 15 ثُمَّ رَجَعَ يَشُوعُ وَجَمِيعُ إِسْرَائِيلَ مَعَهُ إِلَى الْمَحَلَّةِ فِي الْجِلْجَالِ.
\par 16 فَهَرَبَ أُولَئِكَ الْخَمْسَةُ الْمُلُوكِ وَاخْتَبَأُوا فِي مَغَارَةٍ فِي مَقِّيدَةَ.
\par 17 فَأُخْبِرَ يَشُوعُ: «قَدْ وُجِدَ الْمُلُوكُ الْخَمْسَةُ مُخْتَبِئِينَ فِي مَغَارَةٍ فِي مَقِّيدَةَ».
\par 18 فَقَالَ يَشُوعُ: «دَحْرِجُوا حِجَارَةً عَظِيمَةً عَلَى فَمِ الْمَغَارَةِ, وَأَقِيمُوا عَلَيْهَا رِجَالاً لأَجْلِ حِفْظِهِمْ.
\par 19 وَأَمَّا أَنْتُمْ فَلاَ تَقِفُوا, بَلِ اسْعُوا وَرَاءَ أَعْدَائِكُمْ وَاضْرِبُوا مُؤَخَّرَهُمْ. لاَ تَدَعُوهُمْ يَدْخُلُونَ مُدُنَهُمْ, لأَنَّ الرَّبَّ إِلَهَكُمْ قَدْ أَسْلَمَهُمْ بِيَدِكُمْ».
\par 20 وَلَمَّا انْتَهَى يَشُوعُ وَبَنُو إِسْرَائِيلَ مِنْ ضَرْبِهِمْ ضَرْبَةً عَظِيمَةً جِدّاً حَتَّى فَنُوا, وَالشَّرَدُ الَّذِينَ شَرَدُوا مِنْهُمْ دَخَلُوا الْمُدُنَ الْمُحَصَّنَةَ,
\par 21 رَجَعَ جَمِيعُ الشَّعْبِ إِلَى الْمَحَلَّةِ إِلَى يَشُوعَ فِي مَقِّيدَةَ بِسَلاَمٍ. لَمْ يَسُنَّ أَحَدٌ لِسَانَهُ عَلَى بَنِي إِسْرَائِيلَ.
\par 22 فَقَالَ يَشُوعُ: «افْتَحُوا فَمَ الْمَغَارَةِ وَأَخْرِجُوا إِلَيَّ هَؤُلاَءِ الْخَمْسَةَ الْمُلُوكِ مِنَ الْمَغَارَةِ».
\par 23 فَفَعَلُوا كَذَلِكَ, وَأَخْرَجُوا إِلَيْهِ أُولَئِكَ الْمُلُوكَ الْخَمْسَةَ مِنَ الْمَغَارَةِ: مَلِكَ أُورُشَلِيمَ وَمَلِكَ حَبْرُونَ وَمَلِكَ يَرْمُوتَ وَمَلِكَ لَخِيشَ وَمَلِكَ عَجْلُونَ.
\par 24 وَكَانَ لَمَّا أَخْرَجُوا أُولَئِكَ الْمُلُوكَ إِلَى يَشُوعَ أَنَّ يَشُوعَ دَعَا كُلَّ رِجَالِ إِسْرَائِيلَ وَقَالَ لِقُوَّادِ رِجَالِ الْحَرْبِ الَّذِينَ سَارُوا مَعَهُ: «تَقَدَّمُوا وَضَعُوا أَرْجُلَكُمْ عَلَى أَعْنَاقِ هَؤُلاَءِ الْمُلُوكِ». فَتَقَدَّمُوا وَوَضَعُوا أَرْجُلَهُمْ عَلَى أَعْنَاقِهِمْ.
\par 25 فَقَالَ لَهُمْ يَشُوعُ: «لاَ تَخَافُوا وَلاَ تَرْتَعِبُوا. تَشَدَّدُوا وَتَشَجَّعُوا. لأَنَّهُ هَكَذَا يَفْعَلُ الرَّبُّ بِجَمِيعِ أَعْدَائِكُمُ الَّذِينَ تُحَارِبُونَهُمْ».
\par 26 وَضَرَبَهُمْ يَشُوعُ بَعْدَ ذَلِكَ وَقَتَلَهُمْ وَعَلَّقَهُمْ عَلَى خَمْسِ خَشَبٍ, وَبَقُوا مُعَلَّقِينَ عَلَى الْخَشَبِ حَتَّى الْمَسَاءِ.
\par 27 وَكَانَ عِنْدَ غُرُوبِ الشَّمْسِ أَنَّ يَشُوعَ أَمَرَ فَأَنْزَلُوهُمْ عَنِ الْخَشَبِ وَطَرَحُوهُمْ فِي الْمَغَارَةِ الَّتِي اخْتَبَأُوا فِيهَا, وَوَضَعُوا حِجَارَةً كَبِيرَةً عَلَى فَمِ الْمَغَارةِ حَتَّى إِلَى هَذَا الْيَوْمِ عَيْنِهِ.
\par 28 وَأَخَذَ يَشُوعُ مَقِّيدَةَ فِي ذَلِكَ الْيَوْمِ وَضَرَبَهَا بِحَدِّ السَّيْفِ, وَحَرَّمَ مَلِكَهَا هُوَ وَكُلَّ نَفْسٍ بِهَا. لَمْ يُبْقِ شَارِداً. وَفَعَلَ بِمَلِكِ مَقِّيدَةَ كَمَا فَعَلَ بِمَلِكِ أَرِيحَا.
\par 29 ثُمَّ اجْتَازَ يَشُوعُ مِنْ مَقِّيدَةَ وَكُلُّ إِسْرَائِيلَ مَعَهُ إِلَى لِبْنَةَ, وَحَارَبَ لِبْنَةَ.
\par 30 فَدَفَعَهَا الرَّبُّ هِيَ أَيْضاً بِيَدِ إِسْرَائِيلَ مَعَ مَلِكِهَا, فَضَرَبَهَا بِحَدِّ السَّيْفِ وَكُلَّ نَفْسٍ بِهَا. لَمْ يُبْقِ بِهَا شَارِداً, وَفَعَلَ بِمَلِكِهَا كَمَا فَعَلَ بِمَلِكِ أَرِيحَا.
\par 31 ثُمَّ اجْتَازَ يَشُوعُ وَكُلُّ إِسْرَائِيلَ مَعَهُ مِنْ لِبْنَةَ إِلَى لَخِيشَ وَنَزَلَ عَلَيْهَا وَحَارَبَهَا.
\par 32 فَدَفَعَ الرَّبُّ لَخِيشَ بِيَدِ إِسْرَائِيلَ, فَأَخَذَهَا فِي الْيَوْمِ الثَّانِي وَضَرَبَهَا بِحَدِّ السَّيْفِ وَكُلَّ نَفْسٍ بِهَا حَسَبَ كُلِّ مَا فَعَلَ بِلِبْنَةَ.
\par 33 حِينَئِذٍ صَعِدَ هُورَامُ مَلِكُ جَازَرَ لإِعَانَةِ لَخِيشَ, وَضَرَبَهُ يَشُوعُ مَعَ شَعْبِهِ حَتَّى لَمْ يُبْقِ لَهُ شَارِداً.
\par 34 ثُمَّ اجْتَازَ يَشُوعُ وَكُلُّ إِسْرَائِيلَ مَعَهُ مِنْ لَخِيشَ إِلَى عَجْلُونَ فَنَزَلُوا عَلَيْهَا وَحَارَبُوهَا,
\par 35 وَأَخَذُوهَا فِي ذَلِكَ الْيَوْمِ وَضَرَبُوهَا بِحَدِّ السَّيْفِ, وَحَرَّمَ كُلَّ نَفْسٍ بِهَا فِي ذَلِكَ الْيَوْمِ حَسَبَ كُلِّ مَا فَعَلَ بِلَخِيشَ.
\par 36 ثُمَّ صَعِدَ يَشُوعُ وَجَمِيعُ إِسْرَائِيلَ مَعَهُ مِنْ عَجْلُونَ إِلَى حَبْرُونَ وَحَارَبُوهَا,
\par 37 وَأَخَذُوهَا وَضَرَبُوهَا بِحَدِّ السَّيْفِ مَعَ مَلِكِهَا وَكُلِّ مُدُنِهَا وَكُلِّ نَفْسٍ بِهَا. لَمْ يُبْقِ شَارِداً حَسَبَ كُلِّ مَا فَعَلَ بِعَجْلُونَ, فَحَرَّمَهَا وَكُلَّ نَفْسٍ بِهَا.
\par 38 ثُمَّ رَجَعَ يَشُوعُ وَكُلُّ إِسْرَائِيلَ مَعَهُ إِلَى دَبِيرَ وَحَارَبَهَا,
\par 39 وَأَخَذَهَا مَعَ مَلِكِهَا وَكُلِّ مُدُنِهَا, وَضَرَبُوهَا بِحَدِّ السَّيْفِ وَحَرَّمُوا كُلَّ نَفْسٍ بِهَا. لَمْ يُبْقِ شَارِداً. كَمَا فَعَلَ بِحَبْرُونَ كَذَلِكَ فَعَلَ بِدَبِيرَ وَمَلِكِهَا, وَكَمَا فَعَلَ بِلِبْنَةَ وَمَلِكِهَا.
\par 40 فَضَرَبَ يَشُوعُ كُلَّ أَرْضِ الْجَبَلِ وَالْجَنُوبِ وَالسَّهْلِ وَالسُّفُوحِ وَكُلَّ مُلُوكِهَا. لَمْ يُبْقِ شَارِداً, بَلْ حَرَّمَ كُلَّ نَسَمَةٍ كَمَا أَمَرَ الرَّبُّ إِلَهُ إِسْرَائِيلَ.
\par 41 فَضَرَبَهُمْ يَشُوعُ مِنْ قَادِشِ بَرْنِيعَ إِلَى غَزَّةَ وَجَمِيعَ أَرْضِ جُوشِنَ إِلَى جِبْعُونَ.
\par 42 وَأَخَذَ يَشُوعُ جَمِيعَ أُولَئِكَ الْمُلُوكِ وَأَرْضِهِمْ دُفْعَةً وَاحِدَةً, لأَنَّ الرَّبَّ إِلَهَ إِسْرَائِيلَ حَارَبَ عَنْ إِسْرَائِيلَ.
\par 43 ثُمَّ رَجَعَ يَشُوعُ وَجَمِيعُ إِسْرَائِيلَ مَعَهُ إِلَى الْمَحَلَّةِ إِلَى الْجِلْجَالِ.

\chapter{11}

\par 1 فَلَمَّا سَمِعَ يَابِينُ مَلِكُ حَاصُورَ, أَرْسَلَ إِلَى يُوبَابَ مَلِكِ مَادُونَ وَإِلَى مَلِكِ شِمْرُونَ وَإِلَى مَلِكِ أَكْشَافَ,
\par 2 وَإِلَى الْمُلُوكِ الَّذِينَ إِلَى الشِّمَالِ فِي الْجَبَلِ, وَفِي الْعَرَبَةِ جَنُوبِيَّ كِنَّرُوتَ وَفِي السَّهْلِ وَفِي مُرْتَفَعَاتِ دُورَ غَرْباً,
\par 3 الْكَنْعَانِيِّينَ فِي الشَّرْقِ وَالْغَرْبِ, وَالأَمُورِيِّينَ وَالْحِثِّيِّينَ وَالْفِرِزِّيِّينَ وَالْيَبُوسِيِّينَ فِي الْجَبَلِ, وَالْحِوِّيِّينَ تَحْتَ حَرْمُونَ فِي أَرْضِ الْمِصْفَاةِ.
\par 4 فَخَرَجُوا هُمْ وَكُلُّ جُيُوشِهِمْ مَعَهُمْ, شَعْباً غَفِيراً كَالرَّمْلِ الَّذِي عَلَى شَاطِئِ الْبَحْرِ فِي الْكَثْرَةِ, بِخَيْلٍ وَمَرْكَبَاتٍ كَثِيرَةٍ جِدّاً.
\par 5 فَاجْتَمَعَ جَمِيعُ هَؤُلاَءِ الْمُلُوكِ بِمِيعَادٍ وَجَاءُوا وَنَزَلُوا مَعاً عَلَى مِيَاهِ مَيْرُومَ لِيُحَارِبُوا إِسْرَائِيلَ.
\par 6 فَقَالَ الرَّبُّ لِيَشُوعَ: «لاَ تَخَفْهُمْ لأَنِّي غَداً فِي مِثْلِ هَذَا الْوَقْتِ أَدْفَعُهُمْ جَمِيعاً قَتْلَى أَمَامَ إِسْرَائِيلَ, فَتُعَرْقِبُ خَيْلَهُمْ وَتُحْرِقُ مَرْكَبَاتِهِمْ بِالنَّارِ».
\par 7 فَجَاءَ يَشُوعُ وَجَمِيعُ رِجَالِ الْحَرْبِ مَعَهُ عَلَيْهِمْ عِنْدَ مِيَاهِ مَيْرُومَ بَغْتَةً وَسَقَطُوا عَلَيْهِمْ.
\par 8 فَدَفَعَهُمُ الرَّبُّ بِيَدِ إِسْرَائِيلَ, فَضَرَبُوهُمْ وَطَرَدُوهُمْ إِلَى صَيْدُونَ الْعَظِيمَةِ وَإِلَى مِسْرَفُوتَ مَايِمَ وَإِلَى بُقْعَةِ مِصْفَاةَ شَرْقاً. فَضَرَبُوهُمْ حَتَّى لَمْ يَبْقَ لَهُمْ شَارِدٌ.
\par 9 فَفَعَلَ يَشُوعُ بِهِمْ كَمَا قَالَ لَهُ الرَّبُّ. عَرْقَبَ خَيْلَهُمْ وَأَحْرَقَ مَرْكَبَاتِهِمْ بِالنَّارِ.
\par 10 ثُمَّ رَجَعَ يَشُوعُ فِي ذَلِكَ الْوَقْتِ وَأَخَذَ حَاصُورَ وَضَرَبَ مَلِكَهَا بِالسَّيْفِ, لأَنَّ حَاصُورَ كَانَتْ قَبْلاً رَأْسَ جَمِيعِ تِلْكَ الْمَمَالِكِ.
\par 11 وَضَرَبُوا كُلَّ نَفْسٍ بِهَا بِحَدِّ السَّيْفِ. حَرَّمُوهُمْ. وَلَمْ تَبْقَ نَسَمَةٌ. وَأَحْرَقَ حَاصُورَ بِالنَّارِ.
\par 12 فَأَخَذَ يَشُوعُ كُلَّ مُدُنِ أُولَئِكَ الْمُلُوكِ وَجَمِيعَ مُلُوكِهَا وَضَرَبَهُمْ بِحَدِّ السَّيْفِ. حَرَّمَهُمْ كَمَا أَمَرَ مُوسَى عَبْدُ الرَّبِّ.
\par 13 غَيْرَ أَنَّ الْمُدُنَ الْقَائِمَةَ عَلَى تِلاَلِهَا لَمْ يُحْرِقْهَا إِسْرَائِيلُ, مَا عَدَا حَاصُورَ وَحْدَهَا أَحْرَقَهَا يَشُوعُ.
\par 14 وَكُلُّ غَنِيمَةِ تِلْكَ الْمُدُنِ وَالْبَهَائِمَ نَهَبَهَا بَنُو إِسْرَائِيلَ لأَنْفُسِهِمْ. وَأَمَّا الرِّجَالُ فَضَرَبُوهُمْ جَمِيعاً بِحَدِّ السَّيْفِ حَتَّى أَبَادُوهُمْ. لَمْ يُبْقُوا نَسَمَةً.
\par 15 كَمَا أَمَرَ الرَّبُّ مُوسَى عَبْدَهُ هَكَذَا أَمَرَ مُوسَى يَشُوعَ, وَهَكَذَا فَعَلَ يَشُوعُ. لَمْ يُهْمِلْ شَيْئاً مِنْ كُلِّ مَا أَمَرَ بِهِ الرَّبُّ مُوسَى.
\par 16 فَأَخَذَ يَشُوعُ كُلَّ تِلْكَ الأَرْضِ: الْجَبَلَ وَكُلَّ الْجَنُوبِ وَكُلَّ أَرْضِ جُوشِنَ وَالسَّهْلَ وَالْعَرَبَةَ وَجَبَلَ إِسْرَائِيلَ وَسَهْلَهُ,
\par 17 مِنَ الْجَبَلِ الأَقْرَعِ الصَّاعِدِ إِلَى سَعِيرَ إِلَى بَعْلِ جَادَ فِي بُقْعَةِ لُبْنَانَ تَحْتَ جَبَلِ حَرْمُونَ. وَأَخَذَ جَمِيعَ مُلُوكِهَا وَضَرَبَهُمْ وَقَتَلَهُمْ.
\par 18 فَعَمِلَ يَشُوعُ حَرْباً مَعَ أُولَئِكَ الْمُلُوكِ أَيَّاماً كَثِيرَةً.
\par 19 لَمْ تَكُنْ مَدِينَةٌ صَالَحَتْ بَنِي إِسْرَائِيلَ إِلاَّ الْحِوِّيِّينَ سُكَّانَ جِبْعُونَ, بَلْ أَخَذُوا الْجَمِيعَ بِالْحَرْبِ.
\par 20 لأَنَّهُ كَانَ مِنْ قِبَلِ الرَّبِّ أَنْ يُشَدِّدَ قُلُوبَهُمْ حَتَّى يُلاَقُوا إِسْرَائِيلَ لِلْمُحَارَبَةِ فَيُحَرَّمُوا, فَلاَ تَكُونُ عَلَيْهِمْ رَأْفَةٌ, بَلْ يُبَادُونَ كَمَا أَمَرَ الرَّبُّ مُوسَى.
\par 21 وَجَاءَ يَشُوعُ فِي ذَلِكَ الْوَقْتِ وَقَرَضَ الْعَنَاقِيِّينَ مِنَ الْجَبَلِ, مِنْ حَبْرُونَ وَمِنْ دَبِيرَ وَمِنْ عَنَابَ وَمِنْ جَمِيعِ جَبَلِ يَهُوذَا وَمِنْ كُلِّ جَبَلِ إِسْرَائِيلَ. حَرَّمَهُمْ يَشُوعُ مَعَ مُدُنِهِمْ.
\par 22 فَلَمْ يَتَبَقَّ عَنَاقِيُّونَ فِي أَرْضِ بَنِي إِسْرَائِيلَ, لَكِنْ بَقُوا فِي غَزَّةَ وَجَتَّ وَأَشْدُودَ.
\par 23 فَأَخَذَ يَشُوعُ كُلَّ الأَرْضِ حَسَبَ كُلِّ مَا كَلَّمَ بِهِ الرَّبُّ مُوسَى, وَأَعْطَاهَا يَشُوعُ مُلْكاً لإِسْرَائِيلَ حَسَبَ فِرَقِهِمْ وَأَسْبَاطِهِمْ. وَاسْتَرَاحَتِ الأَرْضُ مِنَ الْحَرْبِ.

\chapter{12}

\par 1 وَهَؤُلاَءِ هُمْ مُلُوكُ الأَرْضِ الَّذِينَ ضَرَبَهُمْ بَنُو إِسْرَائِيلَ وَامْتَلَكُوا أَرْضَهُمْ فِي عَبْرِ الأُرْدُنِّ نَحْوَ شُرُوقِ الشَّمْسِ, مِنْ وَادِي أَرْنُونَ إِلَى جَبَلِ حَرْمُونَ وَكُلِّ الْعَرَبَةِ نَحْوَ الشُّرُوقِ:
\par 2 سِيحُونُ مَلِكُ الأَمُورِيِّينَ السَّاكِنُ فِي حَشْبُونَ, الْمُتَسَلِّطُ مِنْ عَرُوعِيرَ الَّتِي عَلَى حَافَةِ وَادِي أَرْنُونَ وَوَسَطِ الْوَادِي وَنِصْفِ جِلْعَادَ إِلَى وَادِي يَبُّوقَ تُخُومِ بَنِي عَمُّونَ
\par 3 وَالْعَرَبَةِ إِلَى بَحْرِ كِنَّرُوتَ نَحْوَ الشُّرُوقِ, وَإِلَى بَحْرِ الْعَرَبَةِ (بَحْرِ الْمِلْحِ) نَحْوَ الشُّرُوقِ طَرِيقِ بَيْتِ يَشِيمُوتَ, وَمِنَ التَّيْمَنِ تَحْتَ سُفُوحِ الْفِسْجَةِ.
\par 4 وَتُخُومُ عُوجٍ مَلِكِ بَاشَانَ مِنْ بَقِيَّةِ الرَّفَائِيِّينَ السَّاكِنِ فِي عَشْتَارُوثَ وَفِي إِذْرَعِي,
\par 5 وَالْمُتَسَلِّطِ عَلَى جَبَلِ حَرْمُونَ وَسَلْخَةَ وَعَلَى كُلِّ بَاشَانَ إِلَى تُخُمِ الْجَشُورِيِّينَ وَالْمَعْكِيِّينَ وَنِصْفِ جِلْعَادَ, تُخُومِ سِيحُونَ مَلِكِ حَشْبُونَ.
\par 6 مُوسَى عَبْدُ الرَّبِّ وَبَنُو إِسْرَائِيلَ ضَرَبُوهَا. وَأَعْطَاهَا مُوسَى عَبْدُ الرَّبِّ مِيرَاثاً لِلرَّأُوبَيْنِيِّينَ وَالْجَادِيِّينَ وَلِنِصْفِ سِبْطِ مَنَسَّى.
\par 7 وَهَؤُلاَءِ هُمْ مُلُوكُ الأَرْضِ الَّذِينَ ضَرَبَهُمْ يَشُوعُ وَبَنُو إِسْرَائِيلَ فِي عَبْرِ الأُرْدُنِّ غَرْباً, مِنْ بَعْلِ جَادَ فِي بُقْعَةِ لُبْنَانَ إِلَى الْجَبَلِ الأَقْرَعِ الصَّاعِدِ إِلَى سَعِيرَ. وَأَعْطَاهَا يَشُوعُ لأَسْبَاطِ إِسْرَائِيلَ مِيرَاثاً حَسَبَ فِرَقِهِمْ.
\par 8 فِي الْجَبَلِ وَالسَّهْلِ وَالْعَرَبَةِ وَالسُّفُوحِ وَالْبَرِّيَّةِ وَالْجَنُوبِ: الْحِثِّيُّونَ وَالأَمُورِيُّونَ وَالْكَنْعَانِيُّونَ وَالْفَرِزِّيُّونَ وَالْحِوِّيُّونَ وَالْيَبُوسِيُّونَ.
\par 9 مَلِكُ أَرِيحَا وَاحِدٌ. مَلِكُ عَايَ الَّتِي بِجَانِبِ بَيْتِ إِيلَ وَاحِدٌ.
\par 10 مَلِكُ أُورُشَلِيمَ وَاحِدٌ. مَلِكُ حَبْرُونَ وَاحِدٌ.
\par 11 مَلِكُ يَرْمُوتَ وَاحِدٌ. مَلِكُ لَخِيشَ وَاحِدٌ.
\par 12 مَلِكُ عَجْلُونَ وَاحِدٌ. مَلِكُ جَازَرَ وَاحِدٌ.
\par 13 مَلِكُ دَبِيرَ وَاحِدٌ. مَلِكُ جَادَرَ وَاحِدٌ.
\par 14 مَلِكُ حُرْمَةَ وَاحِدٌ. مَلِكُ عِرَادَ وَاحِدٌ.
\par 15 مَلِكُ لِبْنَةَ وَاحِدٌ. مَلِكُ عَدُلاَّمَ وَاحِدٌ.
\par 16 مَلِكُ مَقِّيدَةَ وَاحِدٌ. مَلِكُ بَيْتِ إِيلَ وَاحِدٌ.
\par 17 مَلِكُ تَفُّوحَ وَاحِدٌ. مَلِكُ حَافَرَ وَاحِدٌ.
\par 18 مَلِكُ أَفِيقَ وَاحِدٌ. مَلِكُ لَشَّارُونَ وَاحِدٌ.
\par 19 مَلِكُ مَادُونَ وَاحِدٌ. مَلِكُ حَاصُورَ وَاحِدٌ.
\par 20 مَلِكُ شِمْرُونَ مَرَأُونَ وَاحِدٌ. مَلِكُ أَكْشَافَ وَاحِدٌ.
\par 21 مَلِكُ تَعْنَكَ وَاحِدٌ. مَلِكُ مَجِدُّو وَاحِدٌ.
\par 22 مَلِكُ قَادِشَ وَاحِدٌ. مَلِكُ يَقْنَعَامَ فِي كَرْمَلَ وَاحِدٌ.
\par 23 مَلِكُ دُوَرٍ فِي مُرْتَفَعَاتِ دُوَرٍ وَاحِدٌ. مَلِكُ جُويِيمَ فِي الْجِلْجَالِ وَاحِدٌ.
\par 24 مَلِكُ تِرْصَةَ وَاحِدٌ. جَمِيعُ الْمُلُوكِ وَاحِدٌ وَثَلاَثُونَ.

\chapter{13}

\par 1 وَشَاخَ يَشُوعُ. تَقَدَّمَ فِي الأَيَّامِ. فَقَالَ لَهُ الرَّبُّ: «أَنْتَ قَدْ شِخْتَ. تَقَدَّمْتَ فِي الأَيَّامِ. وَقَدْ بَقِيَتْ أَرْضٌ كَثِيرَةٌ جِدّاً لِلاِمْتِلاَكِ.
\par 2 هَذِهِ هِيَ الأَرْضُ الْبَاقِيَةُ: كُلُّ دَائِرَةِ الْفِلِسْطِينِيِّينَ, وَكُلُّ الْجَشُورِيِّينَ
\par 3 مِنَ الشِّيحُورِ الَّذِي هُوَ أَمَامَ مِصْرَ إِلَى تُخُمِ عَقْرُونَ شِمَالاً تُحْسَبُ (لِلْكَنْعَانِيِّينَ) أَقْطَابِ الْفِلِسْطِينِيِّينَ الْخَمْسَةِ: الْغَزِّيِّ وَالأَشْدُودِيِّ وَالأَشْقَلُونِيِّ وَالْجَتِّيِّ وَالْعَقْرُونِيِّ وَالْعَوِيِّينَ.
\par 4 مِنَ التَّيْمَنِ كُلُّ أَرْضِ الْكَنْعَانِيِّينَ وَمَُغَارَةُ الَّتِي لِلصَّيدُونِيِّينَ إِلَى أَفِيقَ إِلَى تُخُمِ الأَمُورِيِّينَ.
\par 5 وَأَرْضُ الْجِبْلِيِّينَ وَكُلُّ لُبْنَانَِ نَحْوَ شُرُوقِ الشَّمْسِ مِنْ بَعْلَِ جَادَ تَحْتَ جَبَلِ حَرْمُونَ إِلَى مَدْخَلِ حَمَاةَ.
\par 6 جَمِيعُ سُكَّانِ الْجَبَلِ مِنْ لُبْنَانَ إِلَى مِسْرَفُوتَِ مَايِمَ جَمِيعُ الصَّيدُونِيِّينَ. أَنَا أَطْرُدُهُمْ مِنْ أَمَامِ بَنِي إِسْرَائِيلَ. إِنَّمَا اقْسِمْهَا بِالْقُرْعَةِ لإِسْرَائِيلَ مُلْكاً كَمَا أَمَرْتُكَ.
\par 7 وَالآنَ اقْسِمْ هَذِهِ الأَرْضَ مُلْكاً لِلتِّسْعَةِ الأَسْبَاطِ وَنِصْفِ سِبْطِ مَنَسَّى».
\par 8 مَعَهُمْ أَخَذَ الرَّأُوبَيْنِيُّونَ وَالْجَادِيُّونَ مُلْكَهُمْ الَّذِي أَعْطَاهُمْ مُوسَى فِي عَبْرِ الأُرْدُنِّ نَحْوَ الشُّرُوقِ, كَمَا أَعْطَاهُمْ مُوسَى عَبْدُ الرَّبِّ.
\par 9 مِنْ عَرُوعِيرَ الَّتِي عَلَى حَافَةِ وَادِي أَرْنُونَ وَالْمَدِينَةِ الَّتِي فِي وَسَطِ الْوَادِي وَكُلُّ سَهْلِ مَيْدَبَا إِلَى دِيبُونَ,
\par 10 وَجَمِيعَ مُدُنِ سِيحُونَ مَلِكِ الأَمُورِيِّينَ الَّذِي مَلَكَ فِي حَشْبُونَ إِلَى تُخُمِ بَنِي عَمُّونَ
\par 11 وَجِلْعَادَ وَتُخُومَ الْجَشُورِيِّينَ وَالْمَعْكِيِّينَ وَكُلَّ جَبَلِ حَرْمُونَ وَكُلَّ بَاشَانَ إِلَى سَلْخَةَ,
\par 12 كُلَّ مَمْلَكَةِ عُوجٍَ فِي بَاشَانَ الَّذِي مَلَكَ فِي عَشْتَارُوثَ وَفِي إِذْرَعِي. هُوَ بَقِيَ مِنْ بَقِيَّةِ الرَّفَائِيِّينَ, وَضَرَبَهُمْ مُوسَى وَطَرَدَهُمْ.
\par 13 وَلَمْ يَطْرُدْ بَنُو إِسْرَائِيلَ الْجَشُورِيِّينَ وَالْمَعْكِيِّينَ فَسَكَنَ الْجَشُورِيُّ وَالْمَعْكِيُّ فِي وَسَطِ إِسْرَائِيلَ إِلَى هَذَا الْيَوْمِ.
\par 14 لَكِنْ لِسِبْطِ لاَوِي لَمْ يُعْطِ نَصِيباً. وَقَائِدُ الرَّبِّ إِلَهِ إِسْرَائِيلَ هِيَ نَصِيبُهُ كَمَا كَلَّمَهُ.
\par 15 وَأَعْطَى مُوسَى سِبْطَ بَنِي رَأُوبَيْنَ حَسَبَ عَشَائِرِهِمْ.
\par 16 فَكَانَ تُخُمُهُمْ مِنْ عَرُوعِيرَ الَّتِي عَلَى حَافَةِ وَادِي أَرْنُونَ وَالْمَدِينَةِ الَّتِي فِي وَسَطِ الْوَادِي وَكُلَّ السَّهْلِ عِنْدَ مَيْدَبَا.
\par 17 حَشْبُونَ وَجَمِيعَ مُدُنِهَا الَّتِي فِي السَّهْلِ وَدِيبُونَ وَبَامُوتَ بَعْلٍ وَبَيْتَ بَعْلِ مَعُونَ,
\par 18 وَيَهْصَةَ وَقَدِيمُوتَ وَمَيْفَعَةَ,
\par 19 وَقَرْيَتَايِمَ وَسَبْمَةَ وَصَارَثَ الشَّحْرِ فِي جَبَلِ الْوَادِي
\par 20 وَبَيْتَ فَغُورَ وَسُفُوحَ الْفِسْجَةِ وَبَيْتَ يَشِيمُوتَ
\par 21 وَكُلَّ مُدُنِ السَّهْلِ وَكُلَّ مَمْلَكَةِ سِيحُونَ مَلِكِ الأَمُورِيِّينَ الَّذِي مَلَكَ فِي حَشْبُونَ, الَّذِي ضَرَبَهُ مُوسَى مَعَ رُؤَسَاءِ مِدْيَانَ: أَوِي وَرَاقَمَ وَصُورَ وَحُورَ وَرَابَعَ, أُمَرَاءِ سِيحُونَ سَاكِنِي الأَرْضِ.
\par 22 وَبَلْعَامُ بْنُ بَعُورَ الْعَرَّافُ قَتَلَهُ بَنُو إِسْرَائِيلَ بِالسَّيْفِ مَعَ قَتْلاَهُمْ.
\par 23 وَكَانَ تُخُمُ بَنِي رَأُوبَيْنَ الأُرْدُنَّ وَتُخُومَهُ. هَذَا نَصِيبُ بَنِي رَأُوبَيْنَ حَسَبَ عَشَائِرِهِمْ, الْمُدُنُ وَضِيَاعُهَا.
\par 24 وَأَعْطَى مُوسَى لِسِبْطِ جَادَ, بَنِي جَادَ حَسَبَ عَشَائِرِهِمْ.
\par 25 فَكَانَ تُخُمُهُمْ يَعْزِيرَ وَكُلَّ مُدُنِ جِلْعَادَ وَنِصْفَ أَرْضِ بَنِي عَمُّونَ إِلَى عَرُوعِيرَ الَّتِي هِيَ أَمَامَ رَبَّةَ,
\par 26 وَمِنْ حَشْبُونَ إِلَى رَامَةِ الْمِصْفَاةِ وَبُطُونِيمَ, وَمِنْ مَحَنَايِمَ إِلَى تُخُمِ دَبِيرَ.
\par 27 وَفِي الْوَادِي بَيْتَ هَارَامَ وَبَيْتَ نِمْرَةَ وَسُكُّوتَ وَصَافُونَ بَقِيَّةَ مَمْلَكَةِ سِيحُونَ مَلِكِ حَشْبُونَ, الأُرْدُنَّ وَتُخُومَهُ إِلَى طَرَفِ بَحْرِ كِنَّرُوتَ فِي عَبْرِ الأُرْدُنِّ نَحْوَ الشُّرُوقِ.
\par 28 هَذَا نَصِيبُ بَنِي جَادَ حَسَبَ عَشَائِرِهِمِ, الْمُدُنُ وَضِيَاعُهَا.
\par 29 وَأَعْطَى مُوسَى لِنِصْفِ سِبْطِ مَنَسَّى, وَكَانَ لِنِصْفِ سِبْطِ بَنِي مَنَسَّى حَسَبَ عَشَائِرِهِمْ.
\par 30 وَكَانَ تُخُمُهُمْ مِنْ مَحَنَايِمَ كُلَّ بَاشَانَ, كُلَّ مَمْلَكَةِ عُوجٍَ مَلِكِ بَاشَانَ, وَكُلَّ حَوُّوثِ يَائِيرَ الَّتِي فِي بَاشَانَ, سِتِّينَ مَدِينَةً.
\par 31 وَنِصْفُ جِلْعَادَ وَعَشْتَارُوثَ وَإِذْرَعِي مُدُنُ مَمْلَكَةِ عُوجٍَ فِي بَاشَانَ لِبَنِي مَاكِيرَ بْنِ مَنَسَّى, لِنِصْفِ بَنِي مَاكِيرَ حَسَبَ عَشَائِرِهِمْ.
\par 32 فَهَذِهِ هِيَ الَّتِي قَسَمَهَا مُوسَى فِي عَرَبَاتِ مُوآبَ فِي عَبْرِ أُرْدُنِّ أَرِيحَا نَحْوَ الشُّرُوقِ.
\par 33 وَأَمَّا سِبْطُ لاَوِي فَلَمْ يُعْطِهِ مُوسَى نَصِيباً. الرَّبُّ إِلَهُ إِسْرَائِيلَ هُوَ نَصِيبُهُمْ كَمَا كَلَّمَهُمْ.

\chapter{14}

\par 1 فَهَذِهِ هِيَ الَّتِي امْتَلَكَهَا بَنُو إِسْرَائِيلَ فِي أَرْضِ كَنْعَانَ, الَّتِي مَلَّكَهُمْ إِيَّاهَا أَلِعَازَارُ الْكَاهِنُ وَيَشُوعُ بْنُ نُونَ وَرُؤَسَاءُ آبَاءِ أَسْبَاطِ بَنِي إِسْرَائِيلَ.
\par 2 نَصِيبُهُمْ بِالْقُرْعَةِ كَمَا أَمَرَ الرَّبُّ عَنْ يَدِ مُوسَى لِلتِّسْعَةِ الأَسْبَاطِ وَنِصْفِ السِّبْطِ.
\par 3 لأَنَّ مُوسَى أَعْطَى نَصِيبَ السِّبْطَيْنِ وَنِصْفِ السِّبْطِ فِي عَبْرِ الأُرْدُنِّ. وَأَمَّا اللاَّوِيُّونَ فَلَمْ يُعْطِهِمْ نَصِيباً فِي وَسَطِهِمْ.
\par 4 لأَنَّ بَنِي يُوسُفَ كَانُوا سِبْطَيْنِ, مَنَسَّى وَأَفْرَايِمَ. وَلَمْ يُعْطُوا اللاَّوِيِّينَ قِسْماً فِي الأَرْضِ إِلاَّ مُدُناً لِلسَّكَنِ, وَمَرَاعِيَهَا لِمَوَاشِيهِمْ وَمُقْتَنَاهُمْ.
\par 5 كَمَا أَمَرَ الرَّبُّ مُوسَى هَكَذَا فَعَلَ بَنُو إِسْرَائِيلَ وَقَسَمُوا الأَرْضَ.
\par 6 فَتَقَدَّمَ بَنُو يَهُوذَا إِلَى يَشُوعَ فِي الْجِلْجَالِ. وَقَالَ لَهُ كَالِبُ بْنُ يَفُنَّةَ الْقَنِزِّيُّ: «أَنْتَ تَعْلَمُ الْكَلاَمَ الَّذِي كَلَّمَ بِهِ الرَّبُّ مُوسَى رَجُلَ اللَّهِ مِنْ جِهَتِي وَمِنْ جِهَتِكَ فِي قَادِشِ بَرْنِيعَ.
\par 7 كُنْتُ ابْنَ أَرْبَعِينَ سَنَةً حِينَ أَرْسَلَنِي مُوسَى عَبْدُ الرَّبِّ مِنْ قَادِشِ بَرْنِيعَ لأَتَجَسَّسَ الأَرْضَ. فَرَجَعْتُ إِلَيْهِ بِكَلاَمٍ عَمَّا فِي قَلْبِي.
\par 8 وَأَمَّا إِخْوَتِيَ الَّذِينَ صَعِدُوا مَعِي فَأَذَابُوا قَلْبَ الشَّعْبِ. وَأَمَّا أَنَا فَاتَّبَعْتُ تَمَاماً الرَّبَّ إِلَهِي.
\par 9 فَحَلَفَ مُوسَى فِي ذَلِكَ الْيَوْمِ قَائِلاً: إِنَّ الأَرْضَ الَّتِي وَطِئَتْهَا رِجْلُكَ لَكَ تَكُونُ نَصِيباً وَلأَوْلاَدِكَ إِلَى الأَبَدِ, لأَنَّكَ اتَّبَعْتَ الرَّبَّ إِلَهِي تَمَاماً.
\par 10 وَالآنَ فَهَا قَدِ اسْتَحْيَانِيَ الرَّبُّ كَمَا تَكَلَّمَ هَذِهِ الْخَمْسَ وَالأَرْبَعِينَ سَنَةً, مِنْ حِينَ كَلَّمَ الرَّبُّ مُوسَى بِهَذَا الْكَلاَمِ حِينَ سَارَ إِسْرَائِيلُ فِي الْقَفْرِ. وَالآنَ فَهَا أَنَا الْيَوْمَ ابْنُ خَمْسٍ وَثَمَانِينَ سَنَةً.
\par 11 فَلَمْ أَزَلِ الْيَوْمَ مُتَشَدِّداً كَمَا فِي يَوْمَ أَرْسَلَنِي مُوسَى. كَمَا كَانَتْ قُوَّتِي حِينَئِذٍ هَكَذَا قُوَّتِي الآنَ لِلْحَرْبِ وَلِلْخُرُوجِ وَلِلدُّخُولِ.
\par 12 فَالآنَ أَعْطِنِي هَذَا الْجَبَلَ الَّذِي تَكَلَّمَ عَنْهُ الرَّبُّ فِي ذَلِكَ الْيَوْمِ. لأَنَّكَ أَنْتَ سَمِعْتَ فِي ذَلِكَ الْيَوْمِ أَنَّ الْعَنَاقِيِّينَ هُنَاكَ, وَالْمُدُنُ عَظِيمَةٌ مُحَصَّنَةٌ. لَعَلَّ الرَّبَّ مَعِي فَأَطْرُدَهُمْ كَمَا تَكَلَّمَ الرَّبُّ».
\par 13 فَبَارَكَهُ يَشُوعُ, وَأَعْطَى حَبْرُونَ لِكَالِبَ بْنِ يَفُنَّةَ مُلْكاً.
\par 14 لِذَلِكَ صَارَتْ حَبْرُونُ لِكَالِبَ بْنِ يَفُنَّةَ الْقَنِزِّيِّ مُلْكاً إِلَى هَذَا الْيَوْمِ, لأَنَّهُ اتَّبَعَ تَمَاماً الرَّبَّ إِلَهَ إِسْرَائِيلَ.
\par 15 وَاسْمُ حَبْرُونَ قَبْلاً قَرْيَةُ أَرْبَعَ, الرَّجُلِ الأَعْظَمِ فِي الْعَنَاقِيِّينَ. وَاسْتَرَاحَتِ الأَرْضُ مِنَ الْحَرْبِ.

\chapter{15}

\par 1 وَكَانَتِ الْقُرْعَةُ لِسِبْطِ بَنِي يَهُوذَا حَسَبَ عَشَائِرِهِمْ إِلَى تُخُمِ أَدُومَ بَرِّيَّةَ صِينَ نَحْوَ الْجَنُوبِ أَقْصَى التَّيْمَنِ.
\par 2 وَكَانَ تُخُمُهُمُ الْجَنُوبِيُّ أَقْصَى بَحْرِ الْمِلْحِ مِنَ اللِّسَانِ الْمُتَوَجِّهِ نَحْوَ الْجَنُوبِ.
\par 3 وَخَرَجَ إِلَى جَنُوبِ عَقَبَةِ عَقْرِبِّيمَ وَعَبَرَ إِلَى صِينَ, وَصَعِدَ مِنْ جَنُوبِ قَاِشِ بَرْنِيعَ وَعَبَرَ إِلَى حَصْرُونَ, وَصَعِدَ إِلَى أَدَّارَ إِلَى الْقَرْقَعِ,
\par 4 وَعَبَرَ إِلَى عَصْمُونَ وَخَرَجَ إِلَى وَادِي مِصْرَ. وَكَانَتْ مَخَارِجُ التُّخُمِ عَُِنْدَ الْبَحْرِ. هَذَا يَكُونُ تُخُمُكُمُ الْجَنُوبِيُّ.
\par 5 وَتُخُمُ الشَّرْقِ بَحْرُ الْمِلْحِ إِلَى طَرَفِ الأُرْدُنِّ. وَتُخُمُ جَانِبِ الشِّمَالِ مِنْ لِسَانِ الْبَحْرِ أَقْصَى الأُرْدُنِّ.
\par 6 وَصَعِدَ التُّخُمُ إِلَى بَيْتِ حُجْلَةَ وَعَبَرَ مِنْ شِمَالِ بَيْتِ الْعَرَبَةِ, وَصَعِدَ التُّخُمُ إِلَى حَجَرِ بُوهَنَ بْنِ رَأُوبَيْنَ
\par 7 وَصَعِدَ التُّخُمُ إِلَى دَبِيرَ مِنْ وَادِي عَخُورَ وَتَوَجَّهَ نَحْوَ الشِّمَالِ إِلَى الْجِلْجَالِ الَّتِي مُقَابَِلَ عَقَبَةِ أَدُمِّيمَ الَّتِي مِنْ جَنُوبِيِّ الْوَادِي. وَعَبَرَ التُّخُمُ إِلَى مِيَاهِ عَيْنِ شَمْسٍ, وَكَانَتْ مَخَارِجُهُ إِلَى عَيْنِ رُوجَلَ.
\par 8 وَصَعِدَ التُّخُمُ فِي وَادِي ابْنِ هِنُّومَ إِلَى جَانِبِ الْيَبُوسِيِّ مِنَ الْجَنُوبِ. (هِيَ أُورُشَلِيمُ) وَصَعِدَ التُّخُمُ إِلَى رَأْسِ الْجَبَلِ الَّذِي قُبَالَةَ وَادِي هِنُّومَ غَرْباً الَّذِي هُوَ فِي طَرَفِ وَادِي الرَّفَائِيِّينَ شِمَالاً.
\par 9 وَامْتَدَّ التُّخُمُ مِنْ رَأْسِ الْجَبَلِ إِلَى مَنْبَعِ مِيَاهِ نَفْتُوحَ, وَخَرَجَ إِلَى مُدُنِ جَبَلِ عِفْرُونَ وَامْتَدَّ التُّخُمُ إِلَى بَعَلَةَ. (هِيَ قَرْيَةُ يَعَارِيمَ)
\par 10 وَامْتَدَّ التُّخُمُ مِنْ بَعَلَةَ غَرْباً إِلَى جَبَلِ سَعِيرَ, وَعَبَرَ إِلَى جَانِبِ جَبَلِ يَعَارِيمَ مِنَ الشِّمَالِ. (هِيَ كَسَالُونُ) وَنَزَلَ إِلَى بَيْتِ شَمْسٍ وَعَبَرَ إِلَى تِمْنَةَ.
\par 11 وَخَرَجَ التُّخُمُ إِلَى جَانِبِ عَقِْرُونَ نَحْوَ الشِّمَالِ وَامْتَدَّ التُّخُمُ إِلَى شَكْرُونَ وَعَبَرَ جَبَلَ الْبَعَلَةِ وَخَرَجَ إِلَى يَبْنِئِيلَ. وَكَانَ مَخَارِجُ التُّخُمِ عِنْدَ الْبَحْرِ.
\par 12 وَالتُّخُمُ الْغَرْبِيُّ الْبَحْرُ الْكَبِيرُ وَتُخُومُهُ. هَذَا تُخُمُ بَنِي يَهُوذَا مُسْتَدِيراً حَسَبَ عَشَائِرِهِمْ.
\par 13 وَأَعْطَى كَالِبَ بْنَ يَفُنَّةَ قِسْماً فِي وَسَطِ بَنِي يَهُوذَا حَسَبَ قَوْلِ الرَّبِّ لِيَشُوعَ: قَرْيَةَ أَرْبَعَ (أَبِي عَنَاقَ) هِيَ حَبْرُونُ.
\par 14 وَطَرَدَ كَالِبُ مِنْ هُنَاكَ بَنِي عَنَاقَ الثَّلاَثَةَ: شِيشَايَ وَأَخِيمَانَ وَتَلْمَايَ, أَوْلاَدَ عَنَاقَ.
\par 15 وَصَعِدَ مِنْ هُنَاكَ إِلَى سُكَّانِ دَبِيرَ. (وَكَانَ اسْمُ دَبِيرَ قَبْلاً قَرْيَةَ سِفْرٍ)
\par 16 وَقَالَ كَالِبُ: «مَنْ يَضْرِبُ قَرْيَةَ سِفْرٍ وَيَأْخُذُهَا أُعْطِيهِ عَكْسَةَ ابْنَتِي امْرَأَةً».
\par 17 فَأَخَذَهَا عُثْنِيئِيلُ بْنُ قَنَازَ أَخُو كَالِبَ. فَأَعْطَاهُ عَكْسَةَ ابْنَتَهُ امْرَأَةً.
\par 18 وَكَانَ عِنْدَ دُخُولِهَا أَنَّهَا غَرَّتْهُ بِطَلَبِ حَقْلٍ مِنْ أَبِيهَا. فَنَزَلَتْ عَنِ الْحِمَارِ فَقَالَ لَهَا كَالِبُ: «مَا لَكِ؟»
\par 19 فَقَالَتْ: «أَعْطِنِي بَرَكَةً. لأَنَّكَ أَعْطَيْتَنِي أَرْضَ الْجَنُوبِ فَأَعْطِنِي يَنَابِيعَ مَاءٍ». فَأَعْطَاهَا الْيَنَابِيعَ الْعُلْيَا وَالْيَنَابِيعَ السُّفْلَى.
\par 20 هَذَا نَصِيبُ سِبْطِ بَنِي يَهُوذَا حَسَبَ عَشَائِرِهِمْ.
\par 21 وَكَانَتِ الْمُدُنُ الْقُصْوَى الَّتِي لِسِبْطِ بَنِي يَهُوذَا إِلَى تُخُمِ أَدُومَ جَنُوباً: قَبْصِئِيلَ وَعِيدَرَ وَيَاجُورَ
\par 22 وَقَيْنَةَ وَدِيمُونَةَ وَعَدْعَدَةَ
\par 23 وَقَادِشَ وَحَاصُورَ وَيِثْنَانَ
\par 24 وَزِيفَ وَطَالَمَ وَبَعَلُوتَ
\par 25 وَحَاصُورَ وَحَدَتَّةَ وَقَرْيُوتَ وَحَصْرُونَ. (هِيَ حَاصُورُ)
\par 26 وَأَمَامَ وَشَمَاعَ وَمُولاَدَةَ
\par 27 وَحَصَرَ جَدَّةَ وَحَشْمُونَ وَبَيْتَ فَالَطَ
\par 28 وَحَصَرَ شُوعَالَ وَبِئْرَ سَبْعٍ وَبِزْيُوتِيَةَ
\par 29 وَبَعَلَةَ وَعَيِّيمَ وَعَاصَمَ
\par 30 وَأَلْتُولَدَ وَكِسِيلَ وَحُرْمَةَ
\par 31 وَصِقْلَغَ وَمَدْمَنَّةَ وَسَنْسَنَّةَ
\par 32 وَلَبَاوُتَ وَشِلْحِيمَ وَعَيْنَ وَرِمُّونَ. كُلُّ الْمُدُنِ تِسْعٌ وَعِشْرُونَ مَعَ ضِيَاعِهَا.
\par 33 فِي السَّهْلِ أَشْتَأُولُ وَصَرْعَةُ وَأَشْنَةُ
\par 34 وَزَانُوحُ وَعَيْنُ جَنِّيمَ وَتَفُّوحُ وَعَيْنَامُ
\par 35 وَيَرْمُوتُ وَعَدُلاَّمُ وَسُوكُوهُ وَعَزِيقَةُ
\par 36 وَشَعَرَايِمُ وَعَدِيتَايِمُ وَالْجُدَيْرَةُ وَجُدَيْرُوتَايِمُ. أَرْبَعَ عَشَرَةَ مَدِينَةً مَعَ ضِيَاعِهَا.
\par 37 صَنَانُ وَحَدَاشَةُ وَمَجْدَلُ جَادٍَ
\par 38 وَدِلْعَانُ وَالْمِصْفَاةُ وَيَقْتِئِيلُ
\par 39 وَلَخِيشُ وَبَصْقَةُ وَعَجْلُونُ
\par 40 وَكَبُّونُ وَلَحْمَامُ وَكِتْلِيشُ
\par 41 وَجُدَيْرُوتُ بَيْتُِ دَاجُونَ وَنَعَمَةُ وَمَقِّيدَةُ. سِتَّ عَشَرَةَ مَدِينَةً مَعَ ضِيَاعِهَا.
\par 42 لِبْنَةُ وَعَاتَرُ وَعَاشَانُ
\par 43 وَيَفْتَاحُ وَأَشْنَةُ وَنَصِيبُ
\par 44 وَقَعِيلَةُ وَأَكْزِيبُ وَمَرِيشَةُ. تِسْعُ مُدُنٍ مَعَ ضِيَاعِهَا.
\par 45 عَقْرُونُ وَقُرَاهَا وَضِيَاعُهَا.
\par 46 مِنْ عَقْرُونَ غَرْباً كُلُّ مَا بِقُرْبِ أَشْدُودَ وَضِيَاعِهَا.
\par 47 أَشْدُودُ وَقُرَاهَا وَضِيَاعُهَا وَغَزَّةُ وَقُرَاهَا وَضِيَاعُهَا إِلَى وَادِي مِصْرَ وَالْبَحْرِ الْكَبِيرِ وَتُخُومِهِ.
\par 48 وَفِي الْجَبَلِ شَامِيرُ وَيَتِّيرُ وَسُوكُوهُ
\par 49 وَدَنَّةُ وَقَرْيَةُ سَنَّةَ. (هِيَ دَبِيرُ)
\par 50 وَعَنَابُ وَأَشْتِمُوهُ وَعَانِيمُ
\par 51 وَجُوشَنُ وَحُولُونُ وَجِيلُوهُ. إِحْدَى عَشَرَةَ مَدِينَةً مَعَ ضِيَاعِهَا.
\par 52 أَرَابُ وَدُومَةُ وَأَشْعَانُ
\par 53 وَيَنُومُ وَبَيْتُ تَفُّوحَ وَأَفِيقَةُ
\par 54 وَحُمْطَةُ وَقَرْيَةُ أَرْبَعَ. (هِيَ حَبْرُونُ) وَصِيعُورُ. تِسْعُ مُدُنٍ مَعَ ضِيَاعِهَا.
\par 55 مَعُونُ وَكَرْمَلُ وَزِيفُ وَيُوطَةُ
\par 56 وَيَزْرَعِيلُ وَيَقْدَعَامُ وَزَانُوحُ
\par 57 وَالْقَايِنُ وَجِبْعَةُ وَتِمْنَةُ. عَشَرُ مُدُنٍ مَعَ ضِيَاعِهَا.
\par 58 حَلْحُولُ وَبَيْتُ صُورٍ وَجَدُورُ
\par 59 وَمَعَارَةُ وَبَيْتُ عَنُوتَ وَأَلْتَقُونُ. سِتَُّ مُدُنٍ مَعَ ضِيَاعِهَا.
\par 60 قَرْيَةُ بَعْلٍ. (هِيَ قَرْيَةُ يَعَارِيمَ) وَالرَّبَّةُ. مَدِينَتَانِ مَعَ ضِيَاعِهِمَا.
\par 61 فِي الْبَرِّيَّةِ بَيْتُ الْعَرَبَةِ وَمِدِّينُ وَسَكَاكَةُ
\par 62 وَالنِّبْشَانُ وَمَدِينَةُ الْمِلْحِ وَعَيْنُ جَدْيٍ. سِتُّ مُدُنٍ مَعَ ضِيَاعِهَا.
\par 63 وَأَمَّا الْيَبُوسِيُّونَ السَّاكِنُونَ فِي أُورُشَلِيمَ فَلَمْ يَقْدِرْ بَنُو يَهُوذَا عَلَى طَرْدِهِمْ, فَسَكَنَ الْيَبُوسِيُّونَ مَعَ بَنِي يَهُوذَا فِي أُورُشَلِيمَ إِلَى هَذَا الْيَوْمِ.

\chapter{16}

\par 1 وَخَرَجَتِ الْقُرْعَةُ لِبَنِي يُوسُفَ مِنْ أُرْدُنِّ أَرِيحَا إِلَى مَاءِ أَرِيحَا نَحْوَ الشُّرُوقِ إِلَى الْبَرِّيَّةِ الصَّاعِدَةِ مِنْ أَرِيحَا فِي جَبَلِ بَيْتِ إِيلَ.
\par 2 وَخَرَجَتْ مِنْ بَيْتِ إِيلَ إِلَى لُوزَ وَعَبَرَتْ إِلَى تُخُمِ الأَرَكِيِّينَ إِلَى عَطَارُوتَ,
\par 3 وَنَزَلَتْ غَرْباً إِلَى تُخُمِ الْيَفْلَطِيِّينَ إِلَى تُخُمِ بَيْتِ حُورُونَ السُّفْلَى وَإِلَى جَازَرَ, وَكَانَتْ مَخَارِجُهَا عَُِنْدَ الْبَحْرِ.
\par 4 فَمَلَكَ ابْنَا يُوسُفَ مَنَسَّى وَأَفْرَايِمُ.
\par 5 وَكَانَ تُخُمُ بَنِي أَفْرَايِمَ حَسَبَ عَشَائِرِهِمْ. وَكَانَ تُخُمُ نَصِيبِهِمْ شَرْقاً: عَطَارُوتَ أَدَّارَ إِلَى بَيْتِ حُورُونَ الْعُلْيَا.
\par 6 وَخَرَجَ التُّخُمُ نَحْوَ الْبَحْرِ إِلَى الْمَكْمَتَةِ شِمَالاً, وَدَارَ التُّخُمُ شَرْقاً إِلَى تَآنَةِ شِيلُوهَ وَعَبَرَهَا شَرْقِيَّ يَنُوحَةَ.
\par 7 وَنَزَلَ مِنْ يَنُوحَةَ إِلَى عَطَارُوتَ وَنَعَرَاتَِ وَوَصَلَ إِلَى أَرِيحَا وَخَرَجَ إِلَى الأُرْدُنِّ.
\par 8 وَجَازَ التُّخُمُ مِنْ تَفُّوحَ غَرْباً إِلَى وَادِي قَانَةَ, وَكَانَتْ مَخَارِجُهُ عَُِنْدَ الْبَحْرِ.
\par 9 هَذَا هُوَ نَصِيبُ سِبْطِ بَنِي أَفْرَايِمَ حَسَبَ عَشَائِرِهِمْ مَعَ الْمُدُنِ الْمُفْرَزَةِ لِبَنِي أَفْرَايِمَ فِي وَسَطِ نَصِيبِ بَنِي مَنَسَّى. جَمِيعُ الْمُدُنِ وَضِيَاعِهَا.
\par 10 فَلَمْ يَطْرُدُوا الْكَنْعَانِيِّينَ السَّاكِنِينَ فِي جَازَرَ. فَسَكَنَ الْكَنْعَانِيُّونَ فِي وَسَطِ أَفْرَايِمَ إِلَى هَذَا الْيَوْمِ, وَكَانُوا عَبِيداً تَحْتَ الْجِزْيَةِ.

\chapter{17}

\par 1 وَكَانَتِ الْقُرْعَةُ لِسِبْطِ مَنَسَّى, لأَنَّهُ هُوَ بِكْرُ يُوسُفَ. لِمَاكِيرَ بِكْرِ مَنَسَّى أَبِي جِلْعَادَ, لأَنَّهُ كَانَ رَجُلَ حَرْبٍ, وَكَانَتْ جِلْعَادُ وَبَاشَانُ لَهُ.
\par 2 وَكَانَتْ لِبَنِي مَنَسَّى الْبَاقِينَ حَسَبَ عَشَائِرِهِمْ. لِبَنِي أَبِيعَزَرَ وَلِبَنِي حَالَقَ وَلِبَنِي أَسْرِيئِيلَ وَلِبَنِي شَكَمَ وَلِبَنِي حَافَرَ وَلِبَنِي شَمِيدَاعَ, هَؤُلاَءِ هُمْ بَنُو مَنَسَّى بْنِ يُوسُفَ, الذُّكُورُ حَسَبَ عَشَائِرِهِمْ.
\par 3 وَأَمَّا صَلُفْحَادُ بْنُ حَافَرَ بْنِ جِلْعَادَ بْنِ مَاكِيرَ بْنِ مَنَسَّى فَلَمْ يَكُنْ لَهُ بَنُونَ بَلْ بَنَاتٌ. وَهَذِهِ أَسْمَاءُ بَنَاتِهِ: مَحْلَةُ وَنُوعَةُ وَحُجْلَةُ وَمِلْكَةُ وَتِرْصَةُ.
\par 4 فَتَقَدَّمْنَ أَمَامَ أَلِعَازَارَ الْكَاهِنِ وَأَمَامَ يَشُوعَ بْنِ نُونَ وَأَمَامَ الرُّؤَسَاءِ وَقُلْنَ: «الرَّبُّ أَمَرَ مُوسَى أَنْ يُعْطِيَنَا نَصِيباً بَيْنَ إِخْوَتِنَا». فَأَعْطَاهُنَّ حَسَبَ قَوْلِ الرَّبِّ نَصِيباً بَيْنَ إِخْوَةِ أَبِيهِنَّ.
\par 5 فَأَصَابَ مَنَسَّى عَشَرُ حِصَصٍ, مَا عَدَا أَرْضَ جِلْعَادَ وَبَاشَانَ الَّتِي فِي عَبْرِ الأُرْدُنِّ.
\par 6 لأَنَّ بَنَاتِ مَنَسَّى أَخَذْنَ نَصِيباً بَيْنَ بَنِيهِ, وَكَانَتْ أَرْضُ جِلْعَادَ لِبَنِي مَنَسَّى الْبَاقِينَ.
\par 7 وَكَانَ تُخُمُ مَنَسَّى مِنْ أَشِيرَ إِلَى الْمَكْمَتَةِ الَّتِي مُقَابِلَ شَكِيمَ, وَامْتَدَّ التُّخُمُ نَحْوَ الْيَمِينِ إِلَى سُكَّانِ عَيْنِ تَفُّوحَ.
\par 8 كَانَ لِمَنَسَّى أَرْضُ تَفُّوحَ. وَأَمَّا تَفُّوحُ إِلَى تُخُمِ مَنَسَّى هِيَ لِبَنِي أَفْرَايِمَ.
\par 9 وَنَزَلَ التُّخُمُ إِلَى وَادِي قَانَةَ جَنُوبِيَّ الْوَادِي. هَذِهِ مُدُنُ أَفْرَايِمَ بَيْنَ مُدُنِ مَنَسَّى. وَتُخُمُ مَنَسَّى شِمَالِيُّ الْوَادِي, وَكَانَتْ مَخَارِجُهُ عَُِنْدَ الْبَحْرِ.
\par 10 مِنَ الْجَنُوبِ لأَفْرَايِمَ, وَمِنَ الشِّمَالِ لِمَنَسَّى. وَكَانَ الْبَحْرُ تُخُمَهُ. وَوَصَلَ إِلَى أَشِيرَ شِمَالاً وَإِلَى يَسَّاكَرَ نَحْوَ الشُّرُوقِ.
\par 11 وَكَانَ لِمَنَسَّى فِي يَسَّاكَرَ وَفِي أَشِيرَ بَيْتُ شَانَ وَقُرَاهَا, وَيَبْلَعَامُ وَقُرَاهَا, وَسُكَّانُ دُوَرٍ وَقُرَاهَا, وَسُكَّانُ عَيْنِ دُوَرٍ وَقُرَاهَا, وَسُكَّانُ تَعْنَكَ وَقُرَاهَا, وَسُكَّانُ مَجِدُّو وَقُرَاهَا الْمُرْتَفَعَاتُ الثَّلاَثُ.
\par 12 وَلَمْ يَقْدِرْ بَنُو مَنَسَّى أَنْ يَمْلِكُوا هَذِهِ الْمُدُنَ, فَعَزَمَ الْكَنْعَانِيُّونَ عَلَى السَّكَنِ فِي تِلْكَ الأَرْضِ.
\par 13 وَكَانَ لَمَّا تَشَدَّدَ بَنُو إِسْرَائِيلَ أَنَّهُمْ جَعَلُوا الْكَنْعَانِيِّينَ تَحْتَ الْجِزْيَةِ, وَلَمْ يَطْرُدُوهُمْ طَرْداً.
\par 14 وَقَالَ بَنُو يُوسُفَ لِيَشُوعَ: «لِمَاذَا أَعْطَيْتَنِي قُرْعَةً وَاحِدَةً وَحِصَّةً وَاحِدَةً نَصِيباً وَأَنَا شَعْبٌ عَظِيمٌ, لأَنَّهُ إِلَى الآنَ قَدْ بَارَكَنِيَ الرَّبُّ؟»
\par 15 فَقَالَ لَهُمْ يَشُوعُ: «إِنْ كُنْتَ شَعْباً عَظِيماً, فَاصْعَدْ إِلَى الْوَعْرِ وَاقْطَعْ لِنَفْسِكَ هُنَاكَ فِي أَرْضِ الْفِرِزِّيِّينَ وَالرَّفَائِيِّينَ, إِذَا ضَاقَ عَلَيْكَ جَبَلُ أَفْرَايِمَ».
\par 16 فَقَالَ بَنُو يُوسُفَ: «لاَ يَكْفِينَا الْجَبَلُ. وَلِجَمِيعِ الْكَنْعَانِيِّينَ السَّاكِنِينَ فِي أَرْضِ الْوَادِي مَرْكَبَاتُ حَدِيدٍ. لِلَّذِينَ فِي بَيْتِ شَانٍ وَقُرَاهَا وَلِلَّذِينَ فِي وَادِي يَزْرَعِيلَ».
\par 17 فَقَالمَ يَشُوعُ لِبَيْتِ يُوسُفَ, أَفْرَايِمَ وَمَنَسَّى: «أَنْتَ شَعْبٌ عَظِيمٌ وَلَكَ قُوَّةٌ عَظِيمَةٌ. لاَ تَكُونُ لَكَ قُرْعَةٌ وَاحِدَةٌ.
\par 18 بَلْ يَكُونُ لَكَ الْجَبَلُ لأَنَّهُ وَعْرٌ, فَتَقْطَعُهُ وَتَكُونُ لَكَ مَخَارِجُهُ. فَتَطْرُدُ الْكَنْعَانِيِّينَ لأَنَّ لَهُمْ مَرْكَبَاتِ حَدِيدٍ لأَنَّهُمْ أَشِدَّاءُ».

\chapter{18}

\par 1 وَاجْتَمَعَ كُلُّ جَمَاعَةِ بَنِي إِسْرَائِيلَ فِي شِيلُوهَ وَنَصَبُوا هُنَاكَ خَيْمَةَ الاِجْتِمَاعِ. وَأُخْضِعَتِ الأَرْضُ قُدَّامَهُمْ.
\par 2 وَبَقِيَ مِنْ بَنِي إِسْرَائِيلَ مِمَّنْ لَمْ يَقْسِمُوا نَصِيبَهُمْ, سَبْعَةُ أَسْبَاطٍ.
\par 3 فَقَالَ يَشُوعُ لِبَنِي إِسْرَائِيلَ: «حَتَّى مَتَى أَنْتُمْ مُتَرَاخُونَ عَنِ الدُّخُولِ لاِمْتِلاَكِ الأَرْضِ الَّتِي أَعْطَاكُمْ إِيَّاهَا الرَّبُّ إِلَهُ آبَائِكُمْ؟
\par 4 هَاتُوا ثَلاَثَةَ رِجَالٍ مِنْ كُلِّ سِبْطٍ فَأُرْسِلَهُمْ فَيَقُومُوا وَيَسِيرُوا فِي الأَرْضِ وَيَكْتُبُوهَا بِحَسَبِ أَنْصِبَتِهِمْ, ثُمَّ يَأْتُوا إِلَيَّ.
\par 5 وَلْيَقْسِمُوهَا إِلَى سَبْعَةِ أَقْسَامٍ, فَيُقِيمُ يَهُوذَا عَلَى تُخُمِهِ مِنَ الْجَنُوبِ, وَيُقِيمُ بَيْتُ يُوسُفَ عَلَى تُخُمِهِمْ مِنَ الشِّمَالِ.
\par 6 وَأَنْتُمْ تَكْتُبُونَ الأَرْضَ سَبْعَةَ أَقْسَامٍ, ثُمَّ تَأْتُونَ إِلَيَّ هُنَا فَأُلْقِي لَكُمْ قُرْعَةً هَهُنَا أَمَامَ الرَّبِّ إِلَهِنَا.
\par 7 لأَنَّهُ لَيْسَ لِلاَّوِيِّينَ قِسْمٌ فِي وَسَطِكُمْ, لأَنَّ كَهَنُوتَ الرَّبِّ هُوَ نَصِيبُهُمْ. وَجَادُ وَرَأُوبَيْنُ وَنِصْفُ سِبْطِ مَنَسَّى قَدْ أَخَذُوا نَصِيبَهُمْ فِي عَبْرِ الأُرْدُنِّ نَحْوَ الشُّرُوقِ, الَّذِي أَعْطَاهُمْ إِيَّاهُ مُوسَى عَبْدُ الرَّبِّ».
\par 8 فَقَامَ الرِّجَالُ وَذَهَبُوا. وَأَوْصَى يَشُوعُ الذَّاهِبِينَ لِكِتَابَةِ الأَرْضِ: «اِذْهَبُوا وَسِيرُوا فِي الأَرْضِ وَاكْتُبُوهَا, ثُمَّ ارْجِعُوا إِلَيَّ فَأُلْقِي لَكُمْ هُنَا قُرْعَةً أَمَامَ الرَّبِّ فِي شِيلُوهَ».
\par 9 فَسَارَ الرِّجَالُ وَعَبَرُوا فِي الأَرْضِ وَكَتَبُوهَا حَسَبَ الْمُدُنِ سَبْعَةَ أَقْسَامٍ فِي سِفْرٍ, ثُمَّ جَاءُوا إِلَى يَشُوعَ إِلَى الْمَحَلَّةِ فِي شِيلُوهَ.
\par 10 فَأَلْقَى لَهُمْ يَشُوعُ قُرْعَةً فِي شِيلُوهَ أَمَامَ الرَّبِّ, وَهُنَاكَ قَسَمَ يَشُوعُ الأَرْضَ لِبَنِي إِسْرَائِيلَ حَسَبَ فِرَقِهِمْ.
\par 11 وَطَلَعَتْ قُرْعَةُ سِبْطِ بَنِي بِنْيَامِينَ حَسَبَ عَشَائِرِهِمْ. وَخَرَجَ تُخُمُ قُرْعَتِهِمْ بَيْنَ بَنِي يَهُوذَا وَبَنِي يُوسُفَ.
\par 12 وَكَانَ تُخُمُهُمْ مِنْ جِهَةِ الشِّمَالِ مِنَ الأُرْدُنِّ. وَصَعِدَ التُّخُمُ إِلَى جَانِبِ أَرِيحَا مِنَ الشِّمَالِ وَصَعِدَ فِي الْجَبَلِ غَرْباً, وَكَانَتْ مَخَارِجُهُ عَُِنْدَ بَرِّيَّةِ بَيْتِ آوِنَ.
\par 13 وَعَبَرَ التُّخُمُ مِنْ هُنَاكَ إِلَى لُوزَ إِلَى جَانِبِ لُوزَ الْجَنُوبِيِّ. (هِيَ بَيْتُ إِيلَ) وَنَزَلَ التُّخُمُ إِلَى عَطَارُوتِ إِدَّارَ عَلَى الْجَبَلِ الَّذِي إِلَى جَنُوبِ بَيْتِ حُورُونَ السُّفْلَى.
\par 14 وَامْتَدَّ التُّخُمُ وَدَارَ إِلَى جِهَةِ الْغَرْبِ جَنُوباً مِنَ الْجَبَلِ الَّذِي مُقَابَِلَ بَيْتِ حُورُونَ جَنُوباً. وَكَانَتْ مَخَارِجُهُ عِنْدَ قَرْيَةِ بَعْلٍ. (هِيَ قَرْيَةُ يَعَارِيمَ) مَدِينَةٌ لِبَنِي يَهُوذَا. هَذِهِ هِيَ جِهَةُ الْغَرْبِ.
\par 15 وَجِهَةُ الْجَنُوبِ هِيَ أَقْصَى قَرْيَةِ يَعَارِيمَ. وَخَرَجَ التُّخُمُ غَرْباً وَخَرَجَ إِلَى مَنْبَعِ مِيَاهِ نَفْتُوحَ.
\par 16 وَنَزَلَ التُّخُمُ إِلَى طَرَفِ الْجَبَلِ الَّذِي مُقَابَِلَ وَادِي ابْنِ هِنُّومَ الَّذِي فِي وَادِي الرَّفَائِيِّينَ شِمَالاً, وَنَزَلَ إِلَى وَادِي هِنُّومَ إِلَى جَانِبِ الْيَبُوسِيِّينَ مِنَ الْجَنُوبِ, وَنَزَلَ إِلَى عَيْنِ رُوجَلَ.
\par 17 وَامْتَدَّ مِنَ الشِّمَالِ وَخَرَجَ إِلَى عَيْنِ شَمْسٍ, وَخَرَجَ إِلَى جَلِيلُوتَ الَّتِي مُقَابِلَ عَقَبَةِ أَدُمِّيمَ, وَنَزَلَ إِلَى حَجَرِ بُوهَنَ بْنِ رَأُوبَيْنَ,
\par 18 وَعَبَرَ إِلَى الْكَتِفِ مُقَابِلَ الْعَرَبَةِ شِمَالاً, وَنَزَلَ إِلَى الْعَرَبَةِ.
\par 19 وَعَبَرَ التُّخُمُ إِلَى جَانِبِ بَيْتِ حُجْلَةَ شِمَالاً. وَكَانَتْ مَخَارِجُ التُّخُمِ عِنْدَ لِسَانِ بَحْرِ الْمِلْحِ شِمَالاً إِلَى طَرَفِ الأُرْدُنِّ جَنُوباً. هَذَا هُوَ تُخُمُ الْجَنُوبِ.
\par 20 وَالأُرْدُنُّ يَتْخُمُهُ مِنْ جِهَةِ الشَّرْقِ. فَهَذَا هُوَ نَصِيبُ بَنِي بِنْيَامِينَ مَعَ تُخُومِهِ مُسْتَدِيراً حَسَبَ عَشَائِرِهِمْ.
\par 21 وَكَانَتْ مُدُنُ سِبْطِ بَنِي بَنْيَامِينَ حَسَبَ عَشَائِرِهِمْ: أَرِيحَا وَبَيْتَ حُجْلَةَ وَوَادِي قَصِيصَ
\par 22 وَبَيْتَ الْعَرَبَةِ وَصَمَارَايِمَ وَبَيْتَ إِيلَ
\par 23 وَالْعَوِّيمَ وَالْفَارَةَ وَعَفْرَةَ
\par 24 وَكَفْرَ الْعَمُّونِيِّ وَالْعُفْنِي وَجَبَعَ, سِتَّ عَشَرَةَ مَدِينَةً مَعَ ضِيَاعِهَا.
\par 25 جِبْعُونَ وَالرَّامَةَ وَبَئِيرُوتَ
\par 26 وَالْمِصْفَاةَ وَالْكَفِيرَةَ وَالْمُوصَةَ
\par 27 وَرَاقَمَ وَيَرَفْئِيلَ وَتَرَالَةَ
\par 28 وَصَيْلَعَ وَآلَفَ وَالْيَبُوسِيَّ. (هِيَ أُورُشَلِيمُ) وَجِبْعَةَ وَقِرْيَةَ. أَرْبَعَ عَشَرَةَ مَدِينَةً مَعَ ضِيَاعِهَا. هَذَا هُوَ نَصِيبُ بَنِي بِنْيَامِينَ حَسَبَ عَشَائِرِهِمْ.

\chapter{19}

\par 1 وَخَرَجَتِ الْقُرْعَةُ الثَّانِيَةُ لِشَمْعُونَ, لِسِبْطِ بَنِي شِمْعُونَ حَسَبَ عَشَائِرِهِمْ. وَكَانَ نَصِيبُهُمْ دَاخِلَ نَصِيبِ بَنِي يَهُوذَا.
\par 2 فَكَانَ لَهُمْ فِي نَصِيبِهِمْ بِئْرُ سَبْعٍ (وَشَبَعُ) وَمُولاَدَةُ.
\par 3 وَحَصَرُ شُوعَالَ وَبَالَةُ وَعَاصَمُ
\par 4 وَأَلْتُولَدُ وَبَتُولُ وَحُرْمَةُ
\par 5 وَصِقْلَغُ وَبَيْتُ الْمَرْكَبُوتِ وَحَصَرُ سُوسَةَ
\par 6 وَبَيْتُ لَبَاوُتَ وَشَارُوحَيْنِ. ثَلاَثَ عَشَرَةَ مَدِينَةً مَعَ ضِيَاعِهَا.
\par 7 عَيْنُ وَرِمُّونُ وَعَاتَرُ وَعَاشَانُ. أَرْبَعُ مُدُنٍ مَعَ ضِيَاعِهَا.
\par 8 وَجَمِيعُ الضِّيَاعِ الَّتِي حَوَالَيْ هَذِهِ الْمُدُنِ إِلَى بَعْلَةِ بَئْرِ رَامَةِ الْجَنُوبِ. هَذَا هُوَ نَصِيبُ سِبْطِ بَنِي شَمْعُونَ حَسَبَ عَشَائِرِهِمْ.
\par 9 وَمِنْ قِسْمِ بَنِي يَهُوذَا كَانَ نَصِيبُ بَنِي شَمْعُونَ. لأَنَّ قِسْمَ بَنِي يَهُوذَا كَانَ كَثِيراً عَلَيْهِمْ, فَمَلَكَ بَنُو شَمْعُونَ دَاخِلَ نَصِيبِهِمْ.
\par 10 وَطَلَعَتِ الْقُرْعَةُ الثَّالِثَةُ لِبَنِي زَبُولُونَ حَسَبَ عَشَائِرِهِمْ. وَكَانَ تُخُمُ نَصِيبِهِمْ إِلَى سَارِيدَ.
\par 11 وَصَعِدَ تُخُمُهُمْ نَحْوَ الْغَرْبِ وَمَرْعَلَةَ وَوَصَلَ إِلَى دَبَّاشَةَ وَوَصَلَ إِلَى الْوَادِي الَّذِي مُقَابَِلَ يَقْنَعَامَ,
\par 12 وَدَارَ مِنْ سَارِيدَ شَرْقاً نَحْوَ شُرُوقِ الشَّمْسِ عَلَى تُخُمِ كِسْلُوتِ تَابُورَ, وَخَرَجَ إِلَى الدَّبْرَةِ وَصَعِدَ إِلَى يَافِيعَ,
\par 13 وَمِنْ هُنَاكَ عَبَرَ شَرْقاً نَحْوَ الشُّرُوقِ إِلَى جَتَّ حَافَرَ إِلَى عِتَّ قَاصِينَ وَخَرَجَ إِلَى رِمُّونَ وَامْتَدَّ إِلَى نَيْعَةَ.
\par 14 وَدَارَ بِهَا التُّخُمُ شِمَالاً إِلَى حَنَّاتُونَ, وَكَانَتْ مَخَارِجُهُ عِنْدَ وَادِي يَفْتَحْئِيلَ
\par 15 وَقَطَّةَ وَنَهْلاَلَ وَشِمْرُونَ وَيَدَالَةَ وَبَيْتِ لَحْمٍ. اثْنَتَا عَشَرَةَ مَدِينَةً مَعَ ضِيَاعِهَا.
\par 16 هَذَا هُوَ نَصِيبُ بَنِي زَبُولُونَ حَسَبَ عَشَائِرِهِمْ. هَذِهِ الْمُدُنُ مَعَ ضِيَاعِهَا.
\par 17 وَخَرَجَتِ الْقُرْعَةُ الرَّابِعَةُ لِيَسَّاكَرَ. لِبَنِي يَسَّاكَرَ حَسَبَ عَشَائِرِهِمْ.
\par 18 وَكَانَ تُخُمُهُمْ إِلَى يَزْرَعِيلَ وَالْكِسْلُوتِ وَشُونَمَ
\par 19 وَحَفَارَايِمَ وَشِيئُونَ وَأَنَاحَرَةَ
\par 20 وَرَبِّيتَ وَقِشْيُونَ وَآبَصَ
\par 21 وَرَمَةَ وَعَيْنِ جَنِّيمَ وَعَيْنِ حَدَّةَ وَبَيْتِ فَصَّيْصَ.
\par 22 وَوَصَلَ التُّخُمُ إِلَى تَابُورَ وَشَحْصِيمَةَ وَبَيْتِ شَمْسٍ وَكَانَتْ مَخَارِجُ تُخُمِهِمْ عِنْدَ الأُرْدُنِّ. سِتَّ عَشَرَةَ مَدِينَةً مَعَ ضِيَاعِهَا.
\par 23 هَذَا هُوَ نَصِيبُ بَنِي يَسَّاكَرَ حَسَبَ عَشَائِرِهِمِ. الْمُدُنُ مَعَ ضِيَاعِهَا.
\par 24 وَخَرَجَتِ الْقُرْعَةُ الْخَامِسَةُ لِسِبْطِ بَنِي أَشِيرَ حَسَبَ عَشَائِرِهِمْ.
\par 25 وَكَانَ تُخُمُهُمْ حَلْقَةَ وَحَلِي وَبَاطَنَ وَأَكْشَافَ
\par 26 وَأَلَّمَّلَكَ وَعَمْعَادَ وَمِشْآلَ وَوَصَلَ إِلَى كَرْمَلَ غَرْباً وَإِلَى شِيحُورِ لِبْنَةَ.
\par 27 وَرَجَعَ نَحْوَ مَشْرِقِ الشَّمْسِ إِلَى بَيْتِ دَاجُونَ, وَوَصَلَ إِلَى زَبُولُونَ وَإِلَى وَادِي يَفْتَحْئِيلَ شِمَالِيَّ بَيْتِ الْعَامِقِ وَنَعِيئِيلَ وَخَرَجَ إِلَى كَابُولَ عَنِ الْيَسَارِ
\par 28 وَعَبْرُونَ وَرَحُوبَ وَحَمُّونَ وَقَانَةَ إِلَى صَيْدُونَ الْعَظِيمَةِ.
\par 29 وَرَجَعَ التُّخُمُ إِلَى الرَّامَةِ وَإِلَى الْمَدِينَةِ الْمُحَصَّنَةِ صُورٍ, ثُمَّ رَجَعَ التُّخُمُ إِلَى حُوصَةَ وَكَانَتْ مَخَارِجُهُ عِنْدَ الْبَحْرِ فِي كُورَةِ أَكْزِيبَ.
\par 30 وَعُمَّةَ وَأَفِيقَ وَرَحُوبَ. اثْنَتَانِ وَعِشْرُونَ مَدِينَةً مَعَ ضِيَاعِهَا.
\par 31 هَذَا هُوَ نَصِيبُ سِبْطِ بَنِي أَشِيرَ حَسَبَ عَشَائِرِهِمْ. هَذِهِ الْمُدُنُ مَعَ ضِيَاعِهَا.
\par 32 لِبَنِي نَفْتَالِي خَرَجَتِ الْقُرْعَةُ السَّادِسَةُ. لِبَنِي نَفْتَالِي حَسَبَ عَشَائِرِهِمْ.
\par 33 وَكَانَ تُخُمُهُمْ مِنْ حَالَفَ مِنَ الْبَلُّوطَةِ عِنْدَ صَعَنَنِّيمَ وَأَدَامِي النَّاقِبِ وَيَبْنِئِيلَ إِلَى لَقُّومَ. وَكَانَتْ مَخَارِجُهُ عِنْدَ الأُرْدُنِّ.
\par 34 وَرَجَعَ التُّخُمُ غَرْباً إِلَى أَزْنُوتِ تَابُورَ, وَخَرَجَ مِنْ هُنَاكَ إِلَى حُقُّوقَ وَوَصَلَ إِلَى زَبُولُونَ جَنُوباً, وَوَصَلَ إِلَى أَشِيرَ غَرْباً, وَإِلَى يَهُوذَا الأُرْدُنِّ نَحْوَ شُرُوقِ الشَّمْسِ.
\par 35 وَمُدُنٌ مُحَصَّنَةٌ الصِّدِّيمُ وَصَيْرُ وَحَمَّةُ وَرَقَّةُ وَكِنَّارَةُ
\par 36 وَأَدَامَةُ وَالرَّامَةُ وَحَاصُورُ
\par 37 وَقَادِشُ وَإِذْرَعِي وَعَيْنُ حَاصُورَ
\par 38 وَيِرْأُونُ وَمَجْدَلُ إِيلٍَ وَحُورِيمُ وَبَيْتُ عَنَاةَ وَبَيْتُ شَمْسٍ تِسْعَ عَشْرَةَ مَدِينَةً مَعَ ضِيَاعِهَا.
\par 39 هَذَا هُوَ نَصِيبُ سِبْطِ بَنِي نَفْتَالِي حَسَبَ عَشَائِرِهِمِ. الْمُدُنُ مَعَ ضِيَاعِهَا.
\par 40 لِسِبْطِ بَنِي دَانَ حَسَبَ عَشَائِرِهِمْ خَرَجَتِ الْقُرْعَةُ السَّابِعَةُ.
\par 41 وَكَانَ تُخُمُ نَصِيبِهِمْ صَرْعَةَ وَأَشْتَأُولَ وَعِيرَشَمْسٍ
\par 42 وَشَعَلَبَّيْنِ وَأَيَّلُونَ وَيِتْلَةَ
\par 43 وَإِيلُونَ وَتِمْنَةَ وَعَقْرُونَ
\par 44 وَإِلْتَقَيْهَ وَجِبَّثُونَ وَبَعْلَةَ
\par 45 وَيَهُودَ وَبَنِي بَرَقَ وَجَتَّ رِمُّونَ
\par 46 وَمِيَاهَ الْيَرْقُونَ وَالرَّقُّونَ مَعَ التُّخُومِ الَّتِي مُقَابَِلَ يَافَا.
\par 47 وَخَرَجَ تُخُمُ بَنِي دَانَ مِنْهُمْ وَصَعِدَ بَنُو دَانَ وَحَارَبُوا لَشَمَ وَأَخَذُوهَا وَضَرَبُوهَا بِحَدِّ السَّيْفِ وَمَلَكُوهَا وَسَكَنُوهَا, وَدَعُوا لَشَمَ دَانَ كَاسْمِ دَانَ أَبِيهِمْ.
\par 48 هَذَا هُوَ نَصِيبُ سِبْطِ بَنِي دَانَ حَسَبَ عَشَائِرِهِمْ. هَذِهِ الْمُدُنُ مَعَ ضِيَاعِهَا.
\par 49 وَلَمَّا انْتَهَوْا مِنْ قِسْمَةِ الأَرْضِ حَسَبَ تُخُومِهَا أَعْطَى بَنُو إِسْرَائِيلَ يَشُوعَ بْنَ نُونَ نَصِيباً فِي وَسَطِهِمْ.
\par 50 حَسَبَ قَوْلِ الرَّبِّ أَعْطَوْهُ الْمَدِينَةَ الَّتِي طَلَبَ: تِمْنَةَ سَارَحَ فِي جَبَلِ أَفْرَايِمَ, فَبَنَى الْمَدِينَةَ وَسَكَنَ بِهَا.
\par 51 هَذِهِ هِيَ الأَنْصِبَةُ الَّتِي قَسَمَهَا أَلِعَازَارُ الْكَاهِنُ وَيَشُوعُ بْنُ نُونَ وَرُؤَسَاءُ آبَاءِ أَسْبَاطِ بَنِي إِسْرَائِيلَ بِالْقُرْعَةِ فِي شِيلُوهَ أَمَامَ الرَّبِّ لَدَى بَابِ خَيْمَةِ الاِجْتِمَاعِ, وَانْتَهَوْا مِنْ قِسْمَةِ الأَرْضِ.

\chapter{20}

\par 1 وَقَالَ الرَّبُّ لِيَشُوعَ:
\par 2 «قُلْ لِبَنِي إِسْرَائِيلَ: اجْعَلُوا لأَنْفُسِكُمْ مُدُنَ الْمَلْجَإِ كَمَا كَلَّمْتُكُمْ عَلَى يَدِ مُوسَى
\par 3 لِيَهْرُبَ إِلَيْهَا الْقَاتِلُ ضَارِبُ نَفْسٍ سَهْواً بِغَيْرِ عِلْمٍ. فَتَكُونَ لَكُمْ مَلْجَأً مِنْ وَلِيِّ الدَّمِ.
\par 4 فَيَهْرُبُ إِلَى وَاحِدَةٍ مِنْ هَذِهِ الْمُدُنِ, وَيَقِفُ فِي مَدْخَلِ بَابِ الْمَدِينَةِ وَيَتَكَلَّمُ بِدَعْوَاهُ فِي آذَانِ شُيُوخِ تِلْكَ الْمَدِينَةِ, فَيَضُمُّونَهُ إِلَيْهِمْ إِلَى الْمَدِينَةِ وَيُعْطُونَهُ مَكَاناً فَيَسْكُنُ مَعَهُمْ.
\par 5 وَإِذَا تَبِعَهُ وَلِيُّ الدَّمِ فَلاَ يُسَلِّمُوا الْقَاتِلَ بِيَدِهِ لأَنَّهُ بِغَيْرِ عِلْمٍ ضَرَبَ قَرِيبَهُ, وَهُوَ غَيْرُ مُبْغِضٍ لَهُ مِنْ قَبْلُ.
\par 6 وَيَسْكُنُ فِي تِلْكَ الْمَدِينَةِ حَتَّى يَقِفُ أَمَامَ الْجَمَاعَةِ لِلْقَضَاءِ, إِلَى أَنْ يَمُوتَ الْكَاهِنُ الْعَظِيمُ الَّذِي يَكُونُ فِي تِلْكَ الأَيَّامِ. حِينَئِذٍ يَرْجِعُ الْقَاتِلُ وَيَأْتِي إِلَى مَدِينَتِهِ وَبَيْتِهِ إِلَى الْمَدِينَةِ الَّتِي هَرَبَ مِنْهَا».
\par 7 فَقَدَّسُوا قَادِشَ فِي الْجَلِيلِ فِي جَبَلِ نَفْتَالِي, وَشَكِيمَ فِي جَبَلِ أَفْرَايِمَ, وَقَرْيَةَ أَرْبَعَ (هِيَ حَبْرُونُ) فِي جَبَلِ يَهُوذَا.
\par 8 وَفِي عَبْرِ أُرْدُنِّ أَرِيحَا نَحْوَ الشُّرُوقِ جَعَلُوا بَاصَرَ فِي الْبَرِّيَّةِ فِي السَّهْلِ مِنْ سِبْطِ رَأُوبَيْنَ, وَرَامُوتَ فِي جِلْعَادَ مِنْ سِبْطِ جَادَ, وَجُولاَنَ فِي بَاشَانَ مِنْ سِبْطِ مَنَسَّى.
\par 9 هَذِهِ هِيَ مُدُنُ الْمَلْجَإِ لِكُلِّ بَنِي إِسْرَائِيلَ وَلِلْغَرِيبِ النَّازِلِ فِي وَسَطِهِمْ لِيَهْرُبَ إِلَيْهَا كُلُّ ضَارِبِ نَفْسٍ سَهْواً, فَلاَ يَمُوتَ بِيَدِ وَلِيِّ الدَّمِ حَتَّى يَقِفَ أَمَامَ الْجَمَاعَةِ.

\chapter{21}

\par 1 ثُمَّ تَقَدَّمَ رُؤَسَاءُ آبَاءِ اللاَّوِيِّينَ إِلَى أَلِعَازَارَ الْكَاهِنِ وَإِلَى يَشُوعَ بْنَ نُونَ وَإِلَى رُؤَسَاءِ آبَاءِ أَسْبَاطِ بَنِي إِسْرَائِيلَ.
\par 2 وَقَالُوا لَهُمْ فِي شِيلُوهَ فِي أَرْضِ كَنْعَانَ: «قَدْ أَمَرَ الرَّبُّ عَلَى يَدِ مُوسَى أَنْ نُعْطَى مُدُناً لِلسَّكَنِ مَعَ مَرَاعِيَهَا لِبَهَائِمِنَا».
\par 3 فَأَعْطَى بَنُو إِسْرَائِيلَ اللاَّوِيِّينَ مِنْ نَصِيبِهِمْ, حَسَبَ قَوْلِ الرَّبِّ, هَذِهِ الْمُدُنَ مَعَ مَرَاعِيَهَا:
\par 4 فَخَرَجَتِ الْقُرْعَةُ لِعَشَائِرِ الْقَهَاتِيِّينَ. فَكَانَ لِبَنِي هَارُونَ الْكَاهِنِ مِنَ اللاَّوِيِّينَ بِالْقُرْعَةِ ثَلاَثَ عَشَرَةَ مَدِينَةً مِنْ سِبْطِ يَهُوذَا وَمِنْ سِبْطِ شَمْعُونَ وَمِنْ سِبْطِ بَنْيَامِينَ.
\par 5 وَلِبَنِي قَهَاتَ الْبَاقِينَ عَشَرُ مُدُنٍ بِالْقُرْعَةِ مِنْ عَشَائِرِ سِبْطِ أَفْرَايِمَ وَمِنْ سِبْطِ دَانَ وَمِنْ نِصْفِ سِبْطِ مَنَسَّى.
\par 6 وَلِبَنِي جَرْشُونَ ثَلاَثَ عَشْرَةَ مَدِينَةً بِالْقُرْعَةِ مِنْ عَشَائِرِ سِبْطِ يَسَّاكَرَ وَمِنْ سِبْطِ أَشِيرَ وَمِنْ سِبْطِ نَفْتَالِي وَمِنْ نِصْفِ سِبْطِ مَنَسَّى فِي بَاشَانَ.
\par 7 وَلِبَنِي مَرَارِي حَسَبَ عَشَائِرِهِمِ اثْنَتَا عَشَرَةَ مَدِينَةً مِنْ سِبْطِ رَأُوبَيْنَ وَمِنْ سِبْطِ جَادَ وَمِنْ سِبْطِ زَبُولُونَ.
\par 8 فَأَعْطَى بَنُو إِسْرَائِيلَ اللاَّوِيِّينَ هَذِهِ الْمُدُنَ وَمَرَاعِيَهَا بِالْقُرْعَةِ, كَمَا أَمَرَ الرَّبُّ عَلَى يَدِ مُوسَى.
\par 9 وَأَعْطُوا مِنْ سِبْطِ بَنِي يَهُوذَا وَمِنْ سِبْطِ بَنِي شَمْعُونَ هَذِهِ الْمُدُنَ الْمُسَمَّاةَ بِأَسْمَائِهَا,
\par 10 فَكَانَتْ لِبَنِي هَارُونَ مِنْ عَشَائِرِ الْقَهَاتِيِّينَ مِنْ بَنِي لاَوِي, لأَنَّ الْقُرْعَةَ الأُولَى كَانَتْ لَهُمْ.
\par 11 وَأَعْطُوهُمْ قَرْيَةَ أَرْبَعَ (أَبِي عَنَاقٍ) هِيَ حَبْرُونَ. فِي جَبَلِ يَهُوذَا مَعَ مَسْرَحِهَا حَوَالَيْهَا.
\par 12 وَأَمَّا حَقْلُ الْمَدِينَةِ وَضِيَاعُهَا فَأَعْطَوْهَا لِكَالِبَ بْنِ يَفُنَّةَ مُلْكاً لَهُ.
\par 13 وَأَعْطُوا لِبَنِي هَارُونَ الْكَاهِنِ (مَدِينَةَ مَلْجَإِ الْقَاتِلِ) حَبْرُونَ مَعَ مَرَاعِيَهَا, وَلِبْنَةَ وَمَرَاعِيَهَا,
\par 14 وَيَتِّيرَ وَمَرَاعِيَهَا, وَأَشْتَمُوعَ وَمَرَاعِيَهَا
\par 15 وَحُولُونَ وَمَرَاعِيَهَا, وَدَبِيرَ وَمَرَاعِيَهَا,
\par 16 وَعَيْنَ وَمَرَاعِيَهَا, وَيُطَّةَ وَمَرَاعِيَهَا, وَبَيْتَ شَمْسٍ وَمَرَاعِيَهَا. تِسْعَ مُدُنٍ مِنْ هَذَيْنِ السِّبْطَيْنِ
\par 17 وَمِنْ سِبْطِ بِنْيَامِينَ جِبْعُونَ وَمَرَاعِيَهَا وَجِبْعَ وَمَرَاعِيَهَا
\par 18 عَنَاثُوثَ وَمَرَاعِيَهَا وَعَلْمُونَ وَمَرَاعِيَهَا. أَرْبَعَ مُدُنٍ.
\par 19 جَمِيعُ مُدُنِ بَنِي هَارُونَ الْكَهَنَةِ ثَلاَثَ عَشْرَةَ مَدِينَةً مَعَ مَرَاعِيَهَا.
\par 20 وَأَمَّا عَشَائِرُ بَنِي قَهَاتَ, اللاَّوِيِّينَ الْبَاقِينَ مِنْ بَنِي قَهَاتَ, فَكَانَتْ مُدُنُ قُرْعَتِهِمْ مِنْ سِبْطِ أَفْرَايِمَ.
\par 21 وَأَعْطُوهُمْ شَكِيمَ وَمَرَاعِيَهَا فِي جَبَلِ أَفْرَايِمَ (مَدِينَةَ مَلْجَإِ الْقَاتِلِ) وَجَازَرَ وَمَرَاعِيَهَا
\par 22 وَقِبْصَايِمَ وَمَرَاعِيَهَا وَبَيْتَ حُورُونَ وَمَرَاعِيَهَا. أَرْبَعَ مُدُنٍ.
\par 23 وَمِنْ سِبْطِ دَانَ إِلْتَقَى وَمَرَاعِيَهَا وَجِبَّثُونَ وَمَرَاعِيَهَا
\par 24 وَأَيَّلُونَ وَمَرَاعِيَهَا وَجَتَّ رِمُّونَ وَمَرَاعِيَهَا. أَرْبَعَ مُدُنٍ.
\par 25 وَمِنْ نِصْفِ سِبْطِ مَنَسَّى تَعْنَكَ وَمَرَاعِيَهَا وَجَتَّ رِمُّونَ وَمَرَاعِيَهَا. مَدِينَتَيْنِ اثْنَتَيْنِ.
\par 26 كُلُّ الْمُدُنِ عَشَرٌ مَعَ مَرَاعِيَهَا لِعَشَائِرِ بَنِي قَهَاتَ الْبَاقِينَ.
\par 27 وَلِبَنِي جَرْشُونَ مِنْ عَشَائِرِ اللاَّوِيِّينَ (مَدِينَةُ مَلْجَإِ الْقَاتِلِ) مِنْ نِصْفِ سِبْطِ مَنَسَّى جُولاَنُ فِي بَاشَانَ وَمَرَاعِيَهَا وَبَعَشْتَرَةُ وَمَرَاعِيَهَا مَدِينَتَانِ اثْنَتَانِ.
\par 28 وَمِنْ سِبْطِ يَسَّاكَرَ قِشْيُونُ وَمَرَاعِيَهَا وَدَبْرَةُ وَمَرَاعِيَهَا
\par 29 وَيَرْمُوتُ وَمَرَاعِيَهَا وَعَيْنُ جَنِّيمَ وَمَرَاعِيَهَا. أَرْبَعُ مُدُنٍ.
\par 30 وَمِنْ سِبْطِ أَشِيرَ مِشْآلُ وَمَرَاعِيَهَا وَعَبْدُونُ وَمَرَاعِيَهَا
\par 31 وَحَلْقَةُ وَمَرَاعِيَهَا وَرَحُوبُ وَمَرَاعِيَهَا. أَرْبَعُ مُدُنٍ.
\par 32 وَمِنْ سِبْطِ نَفْتَالِي (مَدِينَةُ مَلْجَإِ الْقَاتِلِ) قَادِشُ فِي الْجَلِيلِ وَمَرَاعِيَهَا وَحَمُّوتُ دُورٍ وَمَرَاعِيَهَا وَقَرْتَانُ وَمَرَاعِيَهَا. ثَلاَثُ مُدُنٍ.
\par 33 جَمِيعُ مُدُنِ الْجَرْشُونِيِّينَ حَسَبَ عَشَائِرِهِمْ ثَلاَثَ عَشَرَةَ مَدِينَةً مَعَ مَرَاعِيَهَا.
\par 34 وَلِعَشَائِرِ بَنِي مَرَارِي اللاَّوِيِّينَ الْبَاقِينَ مِنْ سِبْطِ زَبُولُونَ يَقْنَعَامُ وَمَرَاعِيَهَا وَقَرْتَةُ وَمَرَاعِيَهَا
\par 35 وَدِمْنَةُ وَمَرَاعِيَهَا وَنَحْلاَلُ وَمَرَاعِيَهَا. أَرْبَعُ مُدُنٍ.
\par 36 وَمِنْ سِبْطِ رَأُوبَيْنَ بَاصَرُ وَمَرَاعِيَهَا وَيَهْصَةُ وَمَرَاعِيَهَا
\par 37 وَقَدِيمُوتُ وَمَرَاعِيَهَا وَمَيْفَعَةُ وَمَرَاعِيَهَا. أَرْبَعُ مُدُنٍ.
\par 38 وَمِنْ سِبْطِ جَادَ (مَدِينَةُ مَلْجَإِ الْقَاتِلِ) رَامُوتُ فِي جِلْعَادَ وَمَرَاعِيَهَا وَمَحَنَايِمُ وَمَرَاعِيَهَا
\par 39 حَشْبُونُ وَمَرَاعِيَهَا وَيَعْزِيرُ وَمَرَاعِيَهَا. كُلُّ الْمُدُنِ أَرْبَعٌ.
\par 40 فَجَمِيعُ الْمُدُنِ الَّتِي لِبَنِي مَرَارِي حَسَبَ عَشَائِرِهِمِ الْبَاقِينَ مِنْ عَشَائِرِ اللاَّوِيِّينَ, وَكَانَتْ قُرْعَتُهُمُ اثْنَتَا عَشْرَةَ مَدِينَةً.
\par 41 جَمِيعُ مُدُنِ اللاَّوِيِّينَ فِي وَسَطِ مُلْكِ بَنِي إِسْرَائِيلَ ثَمَانٍ وَأَرْبَعُونَ مَدِينَةً مَعَ مَرَاعِيَهَا.
\par 42 كَانَتْ هَذِهِ الْمُدُنُ مَدِينَةً مَدِينَةً مَعَ مَرَاعِيَهَا حَوَالَيْهَا. هَكَذَا لِكُلِّ هَذِهِ الْمُدُنِ.
\par 43 فَأَعْطَى الرَّبُّ إِسْرَائِيلَ جَمِيعَ الأَرْضِ الَّتِي أَقْسَمَ أَنْ يُعْطِيَهَا لِآبَائِهِمْ فَامْتَلَكُوهَا وَسَكَنُوا بِهَا.
\par 44 فَأَرَاحَهُمُ الرَّبُّ حَوَالَيْهِمْ حَسَبَ كُلِّ مَا أَقْسَمَ لِآبَائِهِمْ, وَلَمْ يَقِفْ قُدَّامَهُمْ رَجُلٌ مِنْ جَمِيعِ أَعْدَائِهِمْ, بَلْ دَفَعَ الرَّبُّ جَمِيعَ أَعْدَائِهِمْ بِأَيْدِيهِمْ.
\par 45 لَمْ تَسْقُطْ كَلِمَةٌ مِنْ جَمِيعِ الْكَلاَمِ الصَّالِحِ الَّذِي كَلَّمَ بِهِ الرَّبُّ بَيْتَ إِسْرَائِيلَ, بَلِ الْكُلُّ صَارَ.

\chapter{22}

\par 1 حِينَئِذٍ دَعَا يَشُوعُ الرَّأُوبَيْنِيِّينَ وَالْجَادِيِّينَ وَنِصْفَ سِبْطِ مَنَسَّى,
\par 2 وَقَالَ لَهُمْ: «إِنَّكُمْ قَدْ حَفِظْتُمْ كُلَّ مَا أَمَرَكُمْ بِهِ مُوسَى عَبْدُ الرَّبِّ, وَسَمِعْتُمْ صَوْتِي فِي كُلِّ مَا أَمَرْتُكُمْ بِهِ,
\par 3 وَلَمْ تَتْرُكُوا إِخْوَتَكُمْ هَذِهِ الأَيَّامَ, الْكَثِيرَةَ إِلَى هَذَا الْيَوْمِ, وَحَفِظْتُمْ مَا يُحْفَظُ وَصِيَّةُ الرَّبِّ إِلَهِكُمْ.
\par 4 وَالآنَ قَدْ أَرَاحَ الرَّبُّ إِلَهُكُمْ إِخْوَتَكُمْ كَمَا قَالَ لَهُمْ. فَانْصَرِفُوا الآنَ وَاذْهَبُوا إِلَى خِيَامِكُمْ فِي أَرْضِ مُلْكِكُمُ الَّتِي أَعْطَاكُمْ مُوسَى عَبْدُ الرَّبِّ, فِي عَبْرِ الأُرْدُنِّ.
\par 5 وَإِنَّمَا احْرِصُوا جِدّاً أَنْ تَعْمَلُوا الْوَصِيَّةَ وَالشَّرِيعَةَ الَّتِي أَمَرَكُمْ بِهَا مُوسَى عَبْدُ الرَّبِّ أَنْ تُحِبُّوا الرَّبَّ إِلَهَكُمْ وَتَسِيرُوا فِي كُلِّ طُرُقِهِ وَتَحْفَظُوا وَصَايَاهُ وَتَلْصَقُوا بِهِ وَتَعْبُدُوهُ بِكُلِّ قَلْبِكُمْ وَبِكُلِّ نَفْسِكُمْ».
\par 6 ثُمَّ بَارَكَهُمْ يَشُوعُ وَصَرَفَهُمْ, فَذَهَبُوا إِلَى خِيَامِهِمْ.
\par 7 وَلِنِصْفِ سِبْطِ مَنَسَّى أَعْطَى مُوسَى فِي بَاشَانَ, وَأَمَّا نِصْفُهُ الآخَرُ فَأَعْطَاهُمْ يَشُوعُ مَعَ إِخْوَتِهِمْ فِي عَبْرِ الأُرْدُنِّ غَرْباً. وَعِنْدَمَا صَرَفَهُمْ يَشُوعُ أَيْضاً إِلَى خِيَامِهِمْ بَارَكَهُمْ
\par 8 وَقَالَ لَهُمْ: «بِمَالٍ كَثِيرٍ ارْجِعُوا إِلَى خِيَامِكُمْ, وَبِمَوَاشٍ كَثِيرَةٍ جِدّاً بِفِضَّةٍ وَذَهَبٍ وَنُحَاسٍ وَحَدِيدٍ وَمَلاَبِسَ كَثِيرَةٍ جِدّاً. اقْسِمُوا غَنِيمَةَ أَعْدَائِكُمْ مَعَ إِخْوَتِكُمْ».
\par 9 فَرَجَعَ بَنُو رَأُوبَيْنَ وَبَنُو جَادَ وَنِصْفُ سِبْطِ مَنَسَّى, وَذَهَبُوا مِنْ عِنْدِ بَنِي إِسْرَائِيلَ مِنْ شِيلُوهَ الَّتِي فِي أَرْضِ كَنْعَانَ لِيَسِيرُوا إِلَى أَرْضِ جِلْعَادَ, أَرْضِ مُلْكِهِمِ الَّتِي تَمَلَّكُوا بِهَا حَسَبَ قَوْلِ الرَّبِّ عَلَى يَدِ مُوسَى.
\par 10 وَجَاءُوا إِلَى دَائِرَةِ الأُرْدُنِّ الَّتِي فِي أَرْضِ كَنْعَانَ. وَبَنَى بَنُو رَأُوبَيْنَ وَبَنُو جَادَ وَنِصْفُ سِبْطِ مَنَسَّى هُنَاكَ مَذْبَحاً عَلَى الأُرْدُنِّ, مَذْبَحاً عَظِيمَ الْمَنْظَرِ.
\par 11 فَسَمِعَ بَنُو إِسْرَائِيلَ قَوْلاً: «هُوَذَا قَدْ بَنَى بَنُو رَأُوبَيْنَ وَبَنُو جَادَ وَنِصْفُ سِبْطِ مَنَسَّى مَذْبَحاً فِي وَجْهِ أَرْضِ كَنْعَانَ فِي دَائِرَةِ الأُرْدُنِّ مُقَابَِلَ بَنِي إِسْرَائِيلَ».
\par 12 وَلَمَّا سَمِعَ بَنُو إِسْرَائِيلَ اجْتَمَعَتْ كُلُّ جَمَاعَةِ بَنِي إِسْرَائِيلَ فِي شِيلُوهَ لِيَصْعَدُوا إِلَيْهِمْ لِلْحَرْبِ.
\par 13 فَأَرْسَلَ بَنُو إِسْرَائِيلَ إِلَى بَنِي رَأُوبَيْنَ وَبَنِي جَادَ وَنِصْفِ سِبْطِ مَنَسَّى إِلَى أَرْضِ جِلْعَادَ, فِينَحَاسَ بْنَ أَلِعَازَارَ الْكَاهِنَ
\par 14 وَعَشَرَةَ رُؤَسَاءَ مَعَهُ, رَئِيساً وَاحِداً مِنْ كُلِّ بَيْتِ أَبٍ مِنْ جَمِيعِ أَسْبَاطِ إِسْرَائِيلَ, كُلُّ وَاحِدٍ رَئِيسُ بَيْتِ آبَائِهِمْ فِي أُلُوفِ إِسْرَائِيلَ.
\par 15 فَجَاءُوا إِلَى بَنِي رَأُوبَيْنَ وَبَنِي جَادَ وَنِصْفِ سِبْطِ مَنَسَّى إِلَى أَرْضِ جِلْعَادَ, وَقَالَ لَهُمْ:
\par 16 «هَكَذَا قَالَتْ كُلُّ جَمَاعَةِ الرَّبِّ: مَا هَذِهِ الْخِيَانَةُ الَّتِي خُنْتُمْ بِهَا إِلَهَ إِسْرَائِيلَ, بِالرُّجُوعِ الْيَوْمَ عَنِ الرَّبِّ, بِبُنْيَانِكُمْ لأَنْفُسِكُمْ مَذْبَحاً لِتَتَمَرَّدُوا الْيَوْمَ عَلَى الرَّبِّ؟
\par 17 أَقَلِيلٌ لَنَا إِثْمُ فَغُورَ الَّذِي لَمْ نَتَطَهَّرْ مِنْهُ إِلَى هَذَا الْيَوْمِ, وَكَانَ الْوَبَأُ فِي جَمَاعَةِ الرَّبِّ,
\par 18 حَتَّى تَرْجِعُوا أَنْتُمُ الْيَوْمَ عَنِ الرَّبِّ؟ فَيَكُونُ أَنَّكُمُ الْيَوْمَ تَتَمَرَّدُونَ عَلَى الرَّبِّ, وَهُوَ غَداً يَسْخَطُ عَلَى كُلِّ جَمَاعَةِ إِسْرَائِيلَ.
\par 19 وَلَكِنْ إِذَا كَانَتْ نَجِسَةً أَرْضُ مُلْكِكُمْ فَاعْبُرُوا إِلَى أَرْضِ مُلْكِ الرَّبِّ الَّتِي يَسْكُنُ فِيهَا مَسْكَنُ الرَّبِّ وَتَمَلَّكُوا بَيْنَنَا, وَعَلَى الرَّبِّ لاَ تَتَمَرَّدُوا, وَعَلَيْنَا لاَ تَتَمَرَّدُوا بِبِنَائِكُمْ لأَنْفُسِكُمْ مَذْبَحاً غَيْرَ مَذْبَحِ الرَّبِّ إِلَهِنَا.
\par 20 أَمَا خَانَ عَخَانُ بْنُ زَارَحَ خِيَانَةً فِي الْحَرَامِ, فَكَانَ السَّخَطُ عَلَى كُلِّ جَمَاعَةِ إِسْرَائِيلَ, وَهُوَ رَجُلٌ لَمْ يَهْلِكْ وَحْدَهُ بِإِثْمِهِ؟»
\par 21 فَأَجَابَ بَنُو رَأُوبَيْنَ وَبَنُو جَادَ وَنِصْفُ سِبْطِ مَنَسَّى رُؤَسَاءَ أُلُوفِ إِسْرَائِيلَ:
\par 22 «إِلَهُ الآلِهَةِ الرَّبُّ, إِلَهُ الآلِهَةِ الرَّبُّ هُوَ يَعْلَمُ, وَإِسْرَائِيلُ سَيَعْلَمُ. إِنْ كَانَ بِتَمَرُّدٍ وَإِنْ كَانَ بِخِيَانَةٍ عَلَى الرَّبِّ, لاَ تُخَلِّصْنَا هَذَا الْيَوْمَ.
\par 23 بُنْيَانُنَا لأَنْفُسِنَا مَذْبَحاً لِلرُّجُوعِ عَنِ الرَّبِّ, أَوْ لإِصْعَادِ مُحْرَقَةٍ عَلَيْهِ أَوْ تَقْدِمَةٍ أَوْ لِعَمَلِ ذَبَائِحِ سَلاَمَةٍ عَلَيْهِ, فَالرَّبُّ هُوَ يُطَالِبُ.
\par 24 وَإِنْ كُنَّا لَمْ نَفْعَلْ ذَلِكَ خَوْفاً وَعَنْ سَبَبٍ قَائِلِينَ: غَداً يَقُولُ بَنُوكُمْ لِبَنِينَا: مَا لَكُمْ وَلِلرَّبِّ إِلَهِ إِسْرَائِيلَ!
\par 25 قَدْ جَعَلَ الرَّبُّ تُخُماً بَيْنَنَا وَبَيْنَكُمْ يَا بَنِي رَأُوبَيْنَ وَبَنِي جَادَ. الأُرْدُنُّ. لَيْسَ لَكُمْ قِسْمٌ فِي الرَّبِّ. فَيَرُدُّ بَنُوكُمْ بَنِينَا حَتَّى لاَ يَخَافُوا الرَّبَّ.
\par 26 فَقُلْنَا نَصْنَعُ نَحْنُ لأَنْفُسِنَا. نَبْنِي مَذْبَحاً, لاَ لِلْمُحْرَقَةِ وَلاَ لِلذَّبِيحَةِ,
\par 27 بَلْ لِيَكُونَ هُوَ شَاهِداً بَيْنَنَا وَبَيْنَكُمْ وَبَيْنَ أَجْيَالِنَا بَعْدَنَا, لِنَخْدِمَ خِدْمَةَ الرَّبِّ أَمَامَهُ بِمُحْرَقَاتِنَا وَذَبَائِحِنَا وَذَبَائِحِ سَلاَمَتِنَا, وَلاَ يَقُولُ بَنُوكُمْ غَداً لِبَنِينَا: لَيْسَ لَكُمْ قِسْمٌ فِي الرَّبِّ.
\par 28 وَقُلْنَا: يَكُونُ مَتَى قَالُوا كَذَا لَنَا وَلأَجْيَالِنَا غَداً, أَنَّنَا نَقُولُ: انْظُرُوا شِبْهَ مَذْبَحِ الرَّبِّ الَّذِي عَمِلَ آبَاؤُنَا, لاَ لِلْمُحْرَقَةِ وَلاَ لِلذَّبِيحَةِ, بَلْ هُوَ شَاهِدٌ بَيْنَنَا وَبَيْنَكُمْ.
\par 29 حَاشَا لَنَا مِنْهُ أَنْ نَتَمَرَّدَ عَلَى الرَّبِّ وَنَرْجِعَ الْيَوْمَ عَنِ الرَّبِّ لِبِنَاءِ مَذْبَحٍ لِلْمُحْرَقَةِ أَوِ التَّقْدِمَةِ أَوِ الذَّبِيحَةِ, عَدَا مَذْبَحِ الرَّبِّ إِلَهِنَا الَّذِي هُوَ قُدَّامَ مَسْكَنِهِ».
\par 30 فَسَمِعَ فِينَحَاسُ الْكَاهِنُ وَرُؤَسَاءُ الْجَمَاعَةِ وَرُؤُوسُ أُلُوفِ إِسْرَائِيلَ الَّذِينَ مَعَهُ الْكَلاَمَ الَّذِي تَكَلَّمَ بِهِ بَنُو رَأُوبَيْنَ وَبَنُو جَادَ وَبَنُو مَنَسَّى, فَحَسُنَ فِي أَعْيُنِهِمْ.
\par 31 فَقَالَ فِينَحَاسُ بْنُ أَلِعَازَارَ الْكَاهِنِ لِبَنِي رَأُوبَيْنَ وَبَنِي جَادَ وَبَنِي مَنَسَّى: «الْيَوْمَ عَلِمْنَا أَنَّ الرَّبَّ بَيْنَنَا لأَنَّكُمْ لَمْ تَخُونُوا الرَّبَّ بِهَذِهِ الْخِيَانَةِ. فَالآنَ قَدْ أَنْقَذْتُمْ بَنِي إِسْرَائِيلَ مِنْ يَدِ الرَّبِّ».
\par 32 ثُمَّ رَجَعَ فِينَحَاسُ بْنُ أَلِعَازَارَ الكَاهِنِ وَالرُّؤَسَاءُ مِنْ عِنْدِ بَنِي رَأُوبَيْنَ وَبَنِي جَادَ مِنْ أَرْضِ جِلْعَادَ إِلَى أَرْضِ كَنْعَانَ إِلَى بَنِي إِسْرَائِيلَ, وَرَدُّوا عَلَيْهِمْ خَبَراً.
\par 33 فَحَسُنَ الأَمْرُ فِي أَعْيُنِ بَنِي إِسْرَائِيلَ, وَبَارَكَ بَنُو إِسْرَائِيلَ اللّهَ, وَلَمْ يَفْتَكِرُوا بِالصُّعُودِ إِلَيْهِمْ لِلْحَرْبِ وَتَخْرِيبِ الأَرْضِ الَّتِي كَانَ بَنُو رَأُوبَيْنَ وَبَنُو جَادَ سَاكِنِينَ بِهَا.
\par 34 وَسَمَّى بَنُو رَأُوبَيْنَ وَبَنُو جَادَ الْمَذْبَحَ «عِيداً» لأَنَّهُ «شَاهِدٌ بَيْنَنَا أَنَّ الرَّبَّ هُوَ اللَّهُ».

\chapter{23}

\par 1 وَكَانَ غِبَّ أَيَّامٍ كَثِيرَةٍ, بَعْدَمَا أَرَاحَ الرَّبُّ إِسْرَائِيلَ مِنْ أَعْدَائِهِمْ حَوَالَيْهِمْ, أَنَّ يَشُوعَ شَاخَ. تَقَدَّمَ فِي الأَيَّامِ.
\par 2 فَدَعَا يَشُوعُ جَمِيعَ إِسْرَائِيلَ وَشُيُوخَهُ وَرُؤَسَاءَهُ وَقُضَاتَهُ وَعُرَفَاءَهُ وَقَالَ لَهُمْ: «أَنَا قَدْ شِخْتُ. تَقَدَّمْتُ فِي الأَيَّامِ.
\par 3 وَأَنْتُمْ قَدْ رَأَيْتُمْ كُلَّ مَا عَمِلَ الرَّبُّ إِلَهُكُمْ بِجَمِيعِ أُولَئِكَ الشُّعُوبِ مِنْ أَجْلِكُمْ, لأَنَّ الرَّبَّ إِلَهَكُمْ هُوَ الْمُحَارِبُ عَنْكُمْ.
\par 4 اُنْظُرُوا. قَدْ قَسَمْتُ لَكُمْ بِالْقُرْعَةِ هَؤُلاَءِ الشُّعُوبَ الْبَاقِينَ مُلْكاً حَسَبَ أَسْبَاطِكُمْ, مِنَ الأُرْدُنِّ وَجَمِيعِ الشُّعُوبِ الَّتِي قَرَضْتُهَا, وَالْبَحْرِ الْعَظِيمِ نَحْوَ غُرُوبِ الشَّمْسِ.
\par 5 وَالرَّبُّ إِلَهُكُمْ هُوَ يَنْفِيهِمْ مِنْ أَمَامِكُمْ وَيَطْرُدُهُمْ مِنْ قُدَّامِكُمْ, فَتَمْلِكُونَ أَرْضَهُمْ كَمَا كَلَّمَكُمُ الرَّبُّ إِلَهُكُمْ.
\par 6 فَتَشَدَّدُوا جِدّاً لِتَحْفَظُوا وَتَعْمَلُوا كُلَّ الْمَكْتُوبِ فِي سِفْرِ شَرِيعَةِ مُوسَى حَتَّى لاَ تَحِيدُوا عَنْهَا يَمِيناً أَوْ شِمَالاً.
\par 7 حَتَّى لاَ تَدْخُلُوا إِلَى هَؤُلاَءِ الشُّعُوبِ أُولَئِكَ الْبَاقِينَ مَعَكُمْ, وَلاَ تَذْكُرُوا اسْمَ آلِهَتِهِمْ وَلاَ تَحْلِفُوا بِهَا وَلاَ تَعْبُدُوهَا وَلاَ تَسْجُدُوا لَهَا.
\par 8 وَلَكِنِ الْصَقُوا بِالرَّبِّ إِلَهِكُمْ كَمَا فَعَلْتُمْ إِلَى هَذَا الْيَوْمِ.
\par 9 قَدْ طَرَدَ الرَّبُّ مِنْ أَمَامِكُمْ شُعُوباً عَظِيمَةً وَقَوِيَّةً. وَأَمَّا أَنْتُمْ فَلَمْ يَقِفْ أَحَدٌ قُدَّامَكُمْ إِلَى هَذَا الْيَوْمِ.
\par 10 رَجُلٌ وَاحِدٌ مِنْكُمْ يَطْرُدُ أَلْفاً, لأَنَّ الرَّبَّ إِلَهَكُمْ هُوَ الْمُحَارِبُ عَنْكُمْ كَمَا كَلَّمَكُمْ.
\par 11 فَاحْتَفِظُوا جِدّاً لأَنْفُسِكُمْ أَنْ تُحِبُّوا الرَّبَّ إِلَهَكُمْ.
\par 12 «وَلَكِنْ إِذَا رَجَعْتُمْ وَلَصِقْتُمْ بِبَقِيَّةِ هَؤُلاَءِ الشُّعُوبِ, أُولَئِكَ الْبَاقِينَ مَعَكُمْ, وَصَاهَرْتُمُوهُمْ وَدَخَلْتُمْ إِلَيْهِمْ وَهُمْ إِلَيْكُمْ,
\par 13 فَاعْلَمُوا يَقِيناً أَنَّ الرَّبَّ إِلَهَكُمْ لاَ يَعُودُ يَطْرُدُ أُولَئِكَ الشُّعُوبَ مِنْ أَمَامِكُمْ, فَيَكُونُوا لَكُمْ فَخّاً وَشَرَكاً وَسَوْطاً عَلَى جَوَانِبِكُمْ وَشَوْكاً فِي أَعْيُنِكُمْ, حَتَّى تَبِيدُوا عَنْ تِلْكَ الأَرْضِ الصَّالِحَةِ الَّتِي أَعْطَاكُمْ إِيَّاهَا الرَّبُّ إِلَهُكُمْ.
\par 14 وَهَا أَنَا الْيَوْمَ ذَاهِبٌ فِي طَرِيقِ الأَرْضِ كُلِّهَا. وَتَعْلَمُونَ بِكُلِّ قُلُوبِكُمْ وَكُلِّ أَنْفُسِكُمْ أَنَّهُ لَمْ تَسْقُطْ كَلِمَةٌ وَاحِدَةٌ مِنْ جَمِيعِ الْكَلاَمِ الصَّالِحِ الَّذِي تَكَلَّمَ بِهِ الرَّبُّ عَنْكُمُ. الْكُلُّ صَارَ لَكُمْ. لَمْ تَسْقُطْ مِنْهُ كَلِمَةٌ وَاحِدَةٌ.
\par 15 وَيَكُونُ كَمَا أَنَّهُ أَتَى عَلَيْكُمْ كُلُّ الْكَلاَمِ الصَّالِحِ الَّذِي تَكَلَّمَ بِهِ الرَّبُّ إِلَهُكُمْ عَنْكُمْ, كَذَلِكَ يَجْلِبُ عَلَيْكُمُ الرَّبُّ كُلَّ الْكَلاَمِ الرَّدِيءِ حَتَّى يُبِيدَكُمْ عَنْ هَذِهِ الأَرْضِ الصَّالِحَةِ الَّتِي أَعْطَاكُمُ الرَّبُّ إِلَهُكُمْ.
\par 16 حِينَمَا تَتَعَدُّونَ عَهْدَ الرَّبِّ إِلَهِكُمُ الَّذِي أَمَرَكُمْ بِهِ وَتَسِيرُونَ وَتَعْبُدُونَ آلِهَةً أُخْرَى وَتَسْجُدُونَ لَهَا, يَحْمَى غَضَبُ الرَّبِّ عَلَيْكُمْ فَتَبِيدُونَ سَرِيعاً عَنِ الأَرْضِ الصَّالِحَةِ الَّتِي أَعْطَاكُمْ».

\chapter{24}

\par 1 وَجَمَعَ يَشُوعُ جَمِيعَ أَسْبَاطِ إِسْرَائِيلَ إِلَى شَكِيمَ. وَدَعَا شُيُوخَ إِسْرَائِيلَ وَرُؤَسَاءَهُمْ وَقُضَاتَهُمْ وَعُرَفَاءَهُمْ فَمَثَلُوا أَمَامَ الرَّبِّ.
\par 2 وَقَالَ يَشُوعُ لِجَمِيعِ الشَّعْبِ: «هَكَذَا قَالَ الرَّبُّ إِلَهُ إِسْرَائِيلَ: آبَاؤُكُمْ سَكَنُوا فِي عَبْرِ النَّهْرِ مُنْذُ الدَّهْرِ. تَارَحُ أَبُو إِبْرَاهِيمَ وَأَبُو نَاحُورَ, وَعَبَدُوا آلِهَةً أُخْرَى.
\par 3 فَأَخَذْتُ إِبْرَاهِيمَ أَبَاكُمْ مِنْ عَبْرِ النَّهْرِ وَسِرْتُ بِهِ فِي كُلِّ أَرْضِ كَنْعَانَ, وَأَكْثَرْتُ نَسْلَهُ وَأَعْطَيْتُهُ إِسْحَقَ.
\par 4 وَأَعْطَيْتُ إِسْحَاقَ يَعْقُوبَ وَعِيسُوَ, وَأَعْطَيْتُ عِيسُوَ جَبَلَ سَعِيرَ لِيَمْلِكَهُ. وَأَمَّا يَعْقُوبُ وَبَنُوهُ فَنَزَلُوا إِلَى مِصْرَ.
\par 5 وَأَرْسَلْتُ مُوسَى وَهَارُونَ وَضَرَبْتُ مِصْرَ حَسَبَ مَا فَعَلْتُ فِي وَسَطِهَا, ثُمَّ أَخْرَجْتُكُمْ.
\par 6 فَأَخْرَجْتُ آبَاءَكُمْ مِنْ مِصْرَ, وَدَخَلْتُمُ الْبَحْرَ وَتَبِعَ الْمِصْرِيُّونَ آبَاءَكُمْ بِمَرْكَبَاتٍ وَفُرْسَانٍ إِلَى بَحْرِ سُوفٍ.
\par 7 فَصَرَخُوا إِلَى الرَّبِّ, فَجَعَلَ ظَلاَماً بَيْنَكُمْ وَبَيْنَ الْمِصْرِيِّينَ, وَجَلَبَ عَلَيْهِمِ الْبَحْرَ فَغَطَّاهُمْ. وَرَأَتْ أَعْيُنُكُمْ مَا فَعَلْتُ فِي مِصْرَ, وَأَقَمْتُمْ فِي الْقَفْرِ أَيَّاماً كَثِيرَةً.
\par 8 ثُمَّ أَتَيْتُ بِكُمْ إِلَى أَرْضِ الأَمُورِيِّينَ السَّاكِنِينَ فِي عَبْرِ الأُرْدُنِّ فَحَارَبُوكُمْ, وَدَفَعْتُهُمْ بِيَدِكُمْ فَمَلَكْتُمْ أَرْضَهُمْ وَأَهْلَكْتُهُمْ مِنْ أَمَامِكُمْ.
\par 9 وَقَامَ بَالاَقُ بْنُ صِفُّورَ مَلِكُ مُوآبَ وَحَارَبَ إِسْرَائِيلَ, وَأَرْسَلَ وَدَعَا بَلْعَامَ بْنَ بَعُورَ لِيَلْعَنَكُمْ.
\par 10 وَلَمْ أَشَأْ أَنْ أَسْمَعَ لِبَلْعَامَ, فَبَارَكَكُمْ بَرَكَةً وَأَنْقَذْتُكُمْ مِنْ يَدِهِ.
\par 11 ثُمَّ عَبَرْتُمُ الأُرْدُنَّ وَأَتَيْتُمْ إِلَى أَرِيحَا. فَحَارَبَكُمْ أَصْحَابُ أَرِيحَا: الأَمُورِيُّونَ وَالْفِرِزِّيُّونَ وَالْكَنْعَانِيُّونَ وَالْحِثِّيُّونَ وَالْجِرْجَاشِيُّونَ وَالْحِوِّيُّونَ وَالْيَبُوسِيُّونَ, فَدَفَعْتُهُمْ بِيَدِكُمْ.
\par 12 وَأَرْسَلْتُ قُدَّامَكُمُ الزَّنَابِيرَ وَطَرَدْتُ مِنْ أَمَامِكُمْ مَلِكَيِ الأَمُورِيِّينَ, لاَ بِسَيْفِكَ وَلاَ بِقَوْسِكَ.
\par 13 وَأَعْطَيْتُكُمْ أَرْضاً لَمْ تَتْعَبُوا عَلَيْهَا وَمُدُناً لَمْ تَبْنُوهَا وَتَسْكُنُونَ بِهَا, وَمِنْ كُرُومٍ وَزَيْتُونٍ لَمْ تَغْرِسُوهَا تَأْكُلُونَ.
\par 14 فَالآنَ اخْشُوا الرَّبَّ وَاعْبُدُوهُ بِكَمَالٍ وَأَمَانَةٍ, وَانْزِعُوا الآلِهَةَ الَّذِينَ عَبَدَهُمْ آبَاؤُكُمْ فِي عَِبْرِ النَّهْرِ وَفِي مِصْرَ, وَاعْبُدُوا الرَّبَّ.
\par 15 وَإِنْ سَاءَ فِي أَعْيُنِكُمْ أَنْ تَعْبُدُوا الرَّبَّ, فَاخْتَارُوا لأَنْفُسِكُمُ الْيَوْمَ مَنْ تَعْبُدُونَ: إِنْ كَانَ الآلِهَةَ الَّذِينَ عَبَدَهُمْ آبَاؤُكُمُ الَّذِينَ فِي عَبْرِ النَّهْرِ, وَإِنْ كَانَ آلِهَةَ الأَمُورِيِّينَ الَّذِينَ أَنْتُمْ سَاكِنُونَ فِي أَرْضِهِمْ. وَأَمَّا أَنَا وَبَيْتِي فَنَعْبُدُ الرَّبَّ».
\par 16 فَأَجَابَ الشَّعْبُ: «حَاشَا لَنَا أَنْ نَتْرُكَ الرَّبَّ لِنَعْبُدَ آلِهَةً أُخْرَى,
\par 17 لأَنَّ الرَّبَّ إِلَهَنَا هُوَ الَّذِي أَصْعَدَنَا وَآبَاءَنَا مِنْ أَرْضِ مِصْرَ مِنْ بَيْتِ الْعُبُودِيَّةِ, وَالَّذِي عَمِلَ أَمَامَ أَعْيُنِنَا تِلْكَ الآيَاتِ الْعَظِيمَةَ, وَحَفِظَنَا فِي كُلِّ الطَّرِيقِ الَّتِي سِرْنَا فِيهَا وَفِي جَمِيعِ الشُّعُوبِ الَّذِينَ عَبَرْنَا فِي وَسَطِهِمْ.
\par 18 وَطَرَدَ الرَّبُّ مِنْ أَمَامِنَا جَمِيعَ الشُّعُوبِ, وَالأَمُورِيِّينَ السَّاكِنِينَ الأَرْضَ. فَنَحْنُ أَيْضاً نَعْبُدُ الرَّبَّ لأَنَّهُ هُوَ إِلَهُنَا».
\par 19 فَقَالَ يَشُوعُ لِلشَّعْبِ: «لاَ تَقْدِرُونَ أَنْ تَعْبُدُوا الرَّبَّ لأَنَّهُ إِلَهٌ قُدُّوسٌ وَإِلَهٌ غَيُورٌ هُوَ. لاَ يَغْفِرُ ذُنُوبَكُمْ وَخَطَايَاكُمْ.
\par 20 وَإِذَا تَرَكْتُمُ الرَّبَّ وَعَبَدْتُمْ آلِهَةً غَرِيبَةً يَرْجِعُ فَيُسِيءُ إِلَيْكُمْ وَيُفْنِيكُمْ بَعْدَ أَنْ أَحْسَنَ إِلَيْكُمْ».
\par 21 فَقَالَ الشَّعْبُ لِيَشُوعَ: «لاَ. بَلِ الرَّبَّ نَعْبُدُ».
\par 22 فَقَالَ يَشُوعُ لِلشَّعْبِ: «أَنْتُمْ شُهُودٌ عَلَى أَنْفُسِكُمْ أَنَّكُمْ قَدِ اخْتَرْتُمْ لأَنْفُسِكُمُ الرَّبَّ لِتَعْبُدُوهُ». فَقَالُوا: «نَحْنُ شُهُودٌ».
\par 23 فَالآنَ انْزِعُوا الآلِهَةَ الْغَرِيبَةَ الَّتِي فِي وَسَطِكُمْ وَأَمِيلُوا قُلُوبَكُمْ إِلَى الرَّبِّ إِلَهِ إِسْرَائِيلَ».
\par 24 فَقَالَ الشَّعْبُ لِيَشُوعَ: «الرَّبَّ إِلَهَنَا نَعْبُدُ وَلِصَوْتِهِ نَسْمَعُ».
\par 25 وَقَطَعَ يَشُوعُ عَهْداً لِلشَّعْبِ فِي ذَلِكَ الْيَوْمِ وَجَعَلَ لَهُمْ فَرِيضَةً وَحُكْماً فِي شَكِيمَ.
\par 26 وَكَتَبَ يَشُوعُ هَذَا الْكَلاَمَ فِي سِفْرِ شَرِيعَةِ اللَّهِ. وَأَخَذَ حَجَراً كَبِيراً وَنَصَبَهُ هُنَاكَ تَحْتَ الْبَلُّوطَةِ الَّتِي عَُِنْدَ مَقْدِسِ الرَّبِّ,
\par 27 ثُمَّ قَالَ يَشُوعُ لِجَمِيعِ الشَّعْبِ: «إِنَّ هَذَا الْحَجَرَ يَكُونُ شَاهِداً عَلَيْنَا, لأَنَّهُ قَدْ سَمِعَ كُلَّ كَلاَمِ الرَّبِّ الَّذِي كَلَّمَنَا بِهِ, فَيَكُونُ شَاهِداً عَلَيْكُمْ لِئَلاَّ تَجْحَدُوا إِلَهَكُمْ».
\par 28 ثُمَّ صَرَفَ يَشُوعُ الشَّعْبَ كُلَّ وَاحِدٍ إِلَى مُلْكِهِ.
\par 29 وَكَانَ بَعْدَ هَذَا الْكَلاَمِ أَنَّهُ مَاتَ يَشُوعُ بْنُ نُونٍ عَبْدُ الرَّبِّ ابْنَ مِئَةٍ وَعَشَرِ سِنِينَ.
\par 30 فَدَفَنُوهُ فِي تُخُمِ مُلْكِهِ فِي تِمْنَةَ سَارَحَ الَّتِي فِي جَبَلِ أَفْرَايِمَ شِمَالِيَّ جَبَلِ جَاعَشَ.
\par 31 وَعَبَدَ إِسْرَائِيلُ الرَّبَّ كُلَّ أَيَّامِ يَشُوعَ, وَكُلَّ أَيَّامِ الشُّيُوخِ الَّذِينَ طَالَتْ أَيَّامُهُمْ بَعْدَ يَشُوعَ وَالَّذِينَ عَرَفُوا كُلَّ عَمَلِ الرَّبِّ الَّذِي عَمِلَهُ لإِسْرَائِيلَ.
\par 32 وَعِظَامُ يُوسُفَ الَّتِي أَصْعَدَهَا بَنُو إِسْرَائِيلَ مِنْ مِصْرَ دَفَنُوهَا فِي شَكِيمَ فِي قِطْعَةِ الْحَقْلِ الَّتِي اشْتَرَاهَا يَعْقُوبُ مِنْ بَنِي حَمُورَ أَبِي شَكِيمَ بِمِئَةِ قَسِيطَةٍ, فَصَارَتْ لِبَنِي يُوسُفَ مُلْكاً.
\par 33 وَمَاتَ أَلِعَازَارُ بْنُ هَارُونَ فَدَفَنُوهُ فِي جِبْعَةِ فِينَحَاسَ ابْنِهِ الَّتِي أُعْطِيَتْ لَهُ فِي جَبَلِ أَفْرَايِمَ.


\end{document}