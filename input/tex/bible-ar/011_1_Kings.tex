\begin{document}

\title{1 ملوك}


\chapter{1}

\par 1 وَشَاخَ الْمَلِكُ دَاوُدُ. تَقَدَّمَ فِي الأَيَّامِ. وَكَانُوا يُغَطُّونَهُ بِالثِّيَابِ فَلَمْ يَدْفَأْ.
\par 2 فَقَالَ لَهُ عَبِيدُهُ: [لِيُفَتِّشُوا لِسَيِّدِنَا الْمَلِكِ عَلَى فَتَاةٍ عَذْرَاءَ، فَلْتَقِفْ أَمَامَ الْمَلِكِ وَلْتَكُنْ لَهُ حَاضِنَةً وَلْتَضْطَجِعْ فِي حِضْنِكَ فَيَدْفَأَ سَيِّدُنَا الْمَلِكُ].
\par 3 فَفَتَّشُوا عَلَى فَتَاةٍ جَمِيلَةٍ فِي جَمِيعِ تُخُومِ إِسْرَائِيلَ، فَوَجَدُوا أَبِيشَجَ الشُّونَمِيَّةَ فَجَاءُوا بِهَا إِلَى الْمَلِكِ.
\par 4 وَكَانَتِ الْفَتَاةُ جَمِيلَةً جِدّاً، فَكَانَتْ حَاضِنَةَ الْمَلِكِ. وَكَانَتْ تَخْدِمُهُ وَلَكِنَّ الْمَلِكَ لَمْ يَعْرِفْهَا.
\par 5 ثُمَّ إِنَّ أَدُونِيَّا ابْنَ حَجِّيثَ تَرَفَّعَ قَائِلاً: [أَنَا أَمْلِكُ]. وَعَدَّ لِنَفْسِهِ عَجَلاَتٍ وَفُرْسَاناً وَخَمْسِينَ رَجُلاً يَجْرُونَ أَمَامَهُ.
\par 6 وَلَمْ يُغْضِبْهُ أَبُوهُ قَطُّ قَائِلاً: [لِمَاذَا فَعَلْتَ هَكَذَا؟] وَهُوَ أَيْضاً جَمِيلُ الصُّورَةِ جِدّاً، وَقَدْ وَلَدَتْهُ أُمُّهُ بَعْدَ أَبْشَالُومَ.
\par 7 وَكَانَ كَلاَمُهُ مَعَ يُوآبَ ابْنِ صَرُويَةَ وَمَعَ أَبِيَاثَارَ الْكَاهِنِ، فَأَعَانَا أَدُونِيَّا.
\par 8 وَأَمَّا صَادُوقُ الْكَاهِنُ وَبَنَايَاهُو بْنُ يَهُويَادَاعَ وَنَاثَانُ النَّبِيُّ وَشَمْعِي وَرِيعِي وَالْجَبَابِرَةُ الَّذِينَ لِدَاوُدَ فَلَمْ يَكُونُوا مَعَ أَدُونِيَّا.
\par 9 فَذَبَحَ أَدُونِيَّا غَنَماً وَبَقَراً وَمَعْلُوفَاتٍ عِنْدَ حَجَرِ الزَّاحِفَةِ الَّذِي بِجَانِبِ عَيْنِ رُوجَلَ، وَدَعَا جَمِيعَ إِخْوَتِهِ بَنِي الْمَلِكِ وَجَمِيعَ رِجَالِ يَهُوذَا عَبِيدِ الْمَلِكِ.
\par 10 وَأَمَّا نَاثَانُ النَّبِيُّ وَبَنَايَاهُو وَالْجَبَابِرَةُ وَسُلَيْمَانُ أَخُوهُ فَلَمْ يَدْعُهُمْ.
\par 11 فَقَالَ نَاثَانُ لِبَثْشَبَعَ أُمِّ سُلَيْمَانَ: [أَمَا سَمِعْتِ أَنَّ أَدُونِيَّا ابْنَ حَجِّيثَ قَدْ مَلَكَ، وَسَيِّدُنَا دَاوُدُ لاَ يَعْلَمُ؟
\par 12 فَالآنَ تَعَالَيْ أُشِيرُ عَلَيْكِ مَشُورَةً فَتُنَجِّي نَفْسَكِ وَنَفْسَ ابْنِكِ سُلَيْمَانَ.
\par 13 اِذْهَبِي وَادْخُلِي إِلَى الْمَلِكِ دَاوُدَ وَقُولِي لَهُ: أَمَا حَلَفْتَ أَنْتَ يَا سَيِّدِي الْمَلِكُ لأَمَتِكَ أَنَّ سُلَيْمَانَ ابْنَكِ يَمْلِكُ بَعْدِي، وَهُوَ يَجْلِسُ عَلَى كُرْسِيِّي. فَلِمَاذَا مَلَكَ أَدُونِيَّا؟
\par 14 وَفِيمَا أَنْتِ مُتَكَلِّمَةٌ هُنَاكَ مَعَ الْمَلِكِ أَدْخُلُ أَنَا وَرَاءَكِ وَأُكَمِّلُ كَلاَمَكِ].
\par 15 فَدَخَلَتْ بَثْشَبَعُ إِلَى الْمَلِكِ إِلَى الْمَخْدَعِ. وَكَانَ الْمَلِكُ قَدْ شَاخَ جِدّاً وَكَانَتْ أَبِيشَجُ الشُّونَمِيَّةُ تَخْدِمُ الْمَلِكَ.
\par 16 فَخَرَّتْ بَثْشَبَعُ وَسَجَدَتْ لِلْمَلِكِ. فَقَالَ الْمَلِكُ: [مَا لَكِ؟]
\par 17 فَقَالَتْ لَهُ: [أَنْتَ يَا سَيِّدِي حَلَفْتَ بِالرَّبِّ إِلَهِكَ لأَمَتِكَ أَنَّ سُلَيْمَانَ ابْنَكِ يَمْلِكُ بَعْدِي وَهُوَ يَجْلِسُ عَلَى كُرْسِيِّي.
\par 18 وَالآنَ هُوَذَا أَدُونِيَّا قَدْ مَلَكَ. وَالآنَ أَنْتَ يَا سَيِّدِي الْمَلِكُ لاَ تَعْلَمُ ذَلِكَ.
\par 19 وَقَدْ ذَبَحَ ثِيرَاناً وَمَعْلُوفَاتٍ وَغَنَماً بِكَثْرَةٍ، وَدَعَا جَمِيعَ بَنِي الْمَلِكِ، وَأَبِيَاثَارَ الْكَاهِنَ وَيُوآبَ رَئِيسَ الْجَيْشِ، وَلَمْ يَدْعُ سُلَيْمَانَ عَبْدَكَ.
\par 20 وَأَنْتَ يَا سَيِّدِي الْمَلِكُ أَعْيُنُ جَمِيعِ إِسْرَائِيلَ نَحْوَكَ لِتُخْبِرَهُمْ مَنْ يَجْلِسُ عَلَى كُرْسِيِّ سَيِّدِي الْمَلِكِ بَعْدَهُ.
\par 21 فَيَكُونُ إِذَا اضْطَجَعَ سَيِّدِي الْمَلِكُ مَعَ آبَائِهِ أَنِّي أَنَا وَابْنِي سُلَيْمَانَ نُحْسَبُ مُذْنِبَيْنِ].
\par 22 وَبَيْنَمَا هِيَ مُتَكَلِّمَةٌ مَعَ الْمَلِكِ إِذَا نَاثَانُ النَّبِيُّ دَاخِلٌ.
\par 23 فَأَخْبَرُوا الْمَلِكَ: [هُوَذَا نَاثَانُ النَّبِيُّ]. فَدَخَلَ إِلَى أَمَامِ الْمَلِكِ وَسَجَدَ لِلْمَلِكِ عَلَى وَجْهِهِ إِلَى الأَرْضِ.
\par 24 وَقَالَ نَاثَانُ: [يَا سَيِّدِي الْمَلِكَ، أَأَنْتَ قُلْتَ إِنَّ أَدُونِيَّا يَمْلِكُ بَعْدِي وَهُوَ يَجْلِسُ عَلَى كُرْسِيِّي؟
\par 25 لأَنَّهُ نَزَلَ الْيَوْمَ وَذَبَحَ ثِيرَاناً وَمَعْلُوفَاتٍ وَغَنَماً بِكَثْرَةٍ، وَدَعَا جَمِيعَ بَنِي الْمَلِكِ وَرُؤَسَاءَ الْجَيْشِ وَأَبِيَاثَارَ الْكَاهِنَ، وَهَا هُمْ يَأْكُلُونَ وَيَشْرَبُونَ أَمَامَهُ وَيَقُولُونَ: لِيَحْيَ الْمَلِكُ أَدُونِيَّا.
\par 26 وَأَمَّا أَنَا عَبْدُكَ وَصَادُوقُ الْكَاهِنُ وَبَنَايَاهُو بْنُ يَهُويَادَاعَ وَسُلَيْمَانُ عَبْدُكَ فَلَمْ يَدْعُنَا.
\par 27 هَلْ مِنْ قِبَلِ سَيِّدِي الْمَلِكِ كَانَ هَذَا الأَمْرُ وَلَمْ تُعْلِمْ عَبْدَكَ مَنْ يَجْلِسُ عَلَى كُرْسِيِّ سَيِّدِي الْمَلِكِ بَعْدَهُ؟]
\par 28 فَأَجَابَ الْمَلِكُ دَاوُدُ: [ادْعُ لِي بَثْشَبَعَ]. فَدَخَلَتْ إِلَى أَمَامِ الْمَلِكِ وَوَقَفَتْ بَيْنَ يَدَيِ الْمَلِكِ.
\par 29 فَحَلَفَ الْمَلِكُ: [حَيٌّ هُوَ الرَّبُّ الَّذِي فَدَى نَفْسِي مِنْ كُلِّ ضِيقَةٍ
\par 30 إِنَّهُ كَمَا حَلَفْتُ لَكِ بِالرَّبِّ إِلَهِ إِسْرَائِيلَ أَنَّ سُلَيْمَانَ ابْنَكِ يَمْلِكُ بَعْدِي وَهُوَ يَجْلِسُ عَلَى كُرْسِيِّي عِوَضاً عَنِّي، كَذَلِكَ أَفْعَلُ هَذَا الْيَوْمَ].
\par 31 فَخَرَّتْ بَثْشَبَعُ عَلَى وَجْهِهَا إِلَى الأَرْضِ وَسَجَدَتْ لِلْمَلِكِ وَقَالَتْ: [لِيَحْيَ سَيِّدِي الْمَلِكُ دَاوُدُ إِلَى الأَبَدِ].
\par 32 وَقَالَ الْمَلِكُ دَاوُدُ: [ادْعُ لِي صَادُوقَ الْكَاهِنَ وَنَاثَانَ النَّبِيَّ وَبَنَايَاهُوَ بْنَ يَهُويَادَاعَ]. فَدَخَلُوا إِلَى أَمَامِ الْمَلِكِ.
\par 33 فَقَالَ الْمَلِكُ لَهُمْ: [خُذُوا مَعَكُمْ عَبِيدَ سَيِّدِكُمْ، وَأَرْكِبُوا سُلَيْمَانَ ابْنِي عَلَى الْبَغْلَةِ الَّتِي لِي وَانْزِلُوا بِهِ إِلَى جِيحُونَ،
\par 34 وَلْيَمْسَحْهُ هُنَاكَ صَادُوقُ الْكَاهِنُ وَنَاثَانُ النَّبِيُّ مَلِكاً عَلَى إِسْرَائِيلَ، وَاضْرِبُوا بِالْبُوقِ وَقُولُوا: لِيَحْيَ الْمَلِكُ سُلَيْمَانُ.
\par 35 وَتَصْعَدُونَ وَرَاءَهُ فَيَأْتِي وَيَجْلِسُ عَلَى كُرْسِيِّي وَهُوَ يَمْلِكُ عِوَضاً عَنِّي، وَإِيَّاهُ قَدْ أَوْصَيْتُ أَنْ يَكُونَ رَئِيساً عَلَى إِسْرَائِيلَ وَيَهُوذَا].
\par 36 فَأَجَابَ بَنَايَاهُو بْنُ يَهُويَادَاعَ الْمَلِكَ: [آمِينَ. هَكَذَا يَقُولُ الرَّبُّ إِلَهُ سَيِّدِي الْمَلِكِ.
\par 37 كَمَا كَانَ الرَّبُّ مَعَ سَيِّدِي الْمَلِكِ كَذَلِكَ لِيَكُنْ مَعَ سُلَيْمَانَ، وَيَجْعَلْ كُرْسِيَّهُ أَعْظَمَ مِنْ كُرْسِيِّ سَيِّدِي الْمَلِكِ دَاوُدَ].
\par 38 فَنَزَلَ صَادُوقُ الْكَاهِنُ وَنَاثَانُ النَّبِيُّ وَبَنَايَاهُو بْنُ يَهُويَادَاعَ وَالْجَلاَّدُونَ وَالسُّعَاةُ وَأَرْكَبُوا سُلَيْمَانَ عَلَى بَغْلَةِ الْمَلِكِ دَاوُدَ، وَذَهَبُوا بِهِ إِلَى جِيحُونَ.
\par 39 فَأَخَذَ صَادُوقُ الْكَاهِنُ قَرْنَ الدُّهْنِ مِنَ الْخَيْمَةِ وَمَسَحَ سُلَيْمَانَ. وَضَرَبُوا بِالْبُوقِ، وَقَالَ جَمِيعُ الشَّعْبِ: [لِيَحْيَ الْمَلِكُ سُلَيْمَانُ].
\par 40 وَصَعِدَ جَمِيعُ الشَّعْبِ وَرَاءَهُ. وَكَانَ الشَّعْبُ يَضْرِبُونَ بِالنَّايِ وَيَفْرَحُونَ فَرَحاً عَظِيماً حَتَّى انْشَقَّتِ الأَرْضُ مِنْ أَصْوَاتِهِمْ.
\par 41 فَسَمِعَ أَدُونِيَّا وَجَمِيعُ الْمَدْعُوِّينَ الّذِينَ عِنْدهُ بَعْدَمَا انْتَهُوا مِنَ الأَكْلِ. وَسَمِعَ يُوآبُ صَوْتَ الْبُوقِ فَقَالَ: [لِمَاذَا صَوْتُ الْقَرْيَةِ مُضْطَرِبٌ؟]
\par 42 وَفِيمَا هُوَ يَتَكَلَّمُ إِذَا بِيُونَاثَانَ بْنِ أَبِيَاثَارَ الْكَاهِنِ قَدْ جَاءَ فَقَالَ أَدُونِيَّا: [تَعَالَ لأَنَّكَ ذُو بَأْسٍ وَتُبَشِّرُ بِالْخَيْرِ].
\par 43 فَأَجَابَ يُونَاثَانُ: [بَلْ سَيِّدُنَا الْمَلِكُ دَاوُدُ قَدْ مَلَّكَ سُلَيْمَانَ.
\par 44 وَأَرْسَلَ الْمَلِكُ مَعَهُ صَادُوقَ الْكَاهِنَ وَنَاثَانَ النَّبِيَّ وَبَنَايَاهُوَ بْنَ يَهُويَادَاعَ وَالْجَلاَّدِينَ وَالسُّعَاةَ، وَقَدْ أَرْكَبُوهُ عَلَى بَغْلَةِ الْمَلِكِ،
\par 45 وَمَسَحَهُ صَادُوقُ الْكَاهِنُ وَنَاثَانُ النَّبِيُّ مَلِكاً فِي جِيحُونَ، وَصَعِدُوا مِنْ هُنَاكَ فَرِحِينَ حَتَّى اضْطَرَبَتِ الْقَرْيَةُ. هَذَا هُوَ الصَّوْتُ الَّذِي سَمِعْتُمُوهُ.
\par 46 وَأَيْضاً قَدْ جَلَسَ سُلَيْمَانُ عَلَى كُرْسِيِّ الْمَمْلَكَةِ.
\par 47 وَأَيْضاً جَاءَ عَبِيدُ الْمَلِكِ لِيُبَارِكُوا سَيِّدَنَا الْمَلِكَ دَاوُدَ قَائِلِينَ: يَجْعَلُ إِلَهُكَ اسْمَ سُلَيْمَانَ أَحْسَنَ مِنِ اسْمِكَ، وَكُرْسِيَّهُ أَعْظَمَ مِنْ كُرْسِيِّكَ. فَسَجَدَ الْمَلِكُ عَلَى سَرِيرِهِ.
\par 48 وَأَيْضاً هَكَذَا قَالَ الْمَلِكُ: مُبَارَكٌ الرَّبُّ إِلَهُ إِسْرَائِيلَ الَّذِي أَعْطَانِيَ الْيَوْمَ مَنْ يَجْلِسُ عَلَى كُرْسِيِّي وَعَيْنَايَ تُبْصِرَانِ].
\par 49 فَارْتَعَدَ وَقَامَ جَمِيعُ مَدْعُوِّي أَدُونِيَّا وَذَهَبُوا كُلُّ وَاحِدٍ فِي طَرِيقِهِ.
\par 50 وَخَافَ أَدُونِيَّا مِنْ سُلَيْمَانَ، وَقَامَ وَانْطَلَقَ وَتَمَسَّكَ بِقُرُونِ الْمَذْبَحِ.
\par 51 فَأُخْبِرَ سُلَيْمَانُ: هُوَذَا أَدُونِيَّا خَائِفٌ مِنَ الْمَلِكِ سُلَيْمَانَ، وَهُوَذَا قَدْ تَمَسَّكَ بِقُرُونِ الْمَذْبَحِ قَائِلاً: [لِيَحْلِفْ لِي الْيَوْمَ الْمَلِكُ سُلَيْمَانُ إِنَّهُ لاَ يَقْتُلُ عَبْدَهُ بِالسَّيْفِ].
\par 52 فَقَالَ سُلَيْمَانُ: [إِنْ كَانَ ذَا فَضِيلَةٍ لاَ يَسْقُطُ مِنْ شَعْرِهِ إِلَى الأَرْضِ. وَلَكِنْ إِنْ وُجِدَ بِهِ شَرٌّ فَإِنَّهُ يَمُوتُ].
\par 53 فَأَرْسَلَ الْمَلِكُ سُلَيْمَانُ فَأَنْزَلُوهُ عَنِ الْمَذْبَحِ، فَأَتَى وَسَجَدَ لِلْمَلِكِ سُلَيْمَانَ. فَقَالَ لَهُ سُلَيْمَانُ: [اذْهَبْ إِلَى بَيْتِكَ].

\chapter{2}

\par 1 وَلَمَّا قَرُبَتْ أَيَّامُ وَفَاةِ دَاوُدَ أَوْصَى سُلَيْمَانَ ابْنَهُ:
\par 2 [أَنَا ذَاهِبٌ فِي طَرِيقِ الأَرْضِ كُلِّهَا. فَتَشَدَّدْ وَكُنْ رَجُلاً.
\par 3 اِحْفَظْ شَعَائِرَ الرَّبِّ إِلَهِكَ إِذْ تَسِيرُ فِي طُرُقِهِ وَتَحْفَظُ فَرَائِضَهُ وَصَايَاهُ وَأَحْكَامَهُ وَشَهَادَاتِهِ كَمَا هُوَ مَكْتُوبٌ فِي شَرِيعَةِ مُوسَى، لِتُفْلِحَ فِي كُلِّ مَا تَفْعَلُ وَحَيْثُمَا تَوَجَّهْتَ.
\par 4 لِيُقِيمَ الرَّبُّ كَلاَمَهُ الَّذِي تَكَلَّمَ بِهِ عَنِّي قَائِلاً: إِذَا حَفِظَ بَنُوكَ طَرِيقَهُمْ وَسَلَكُوا أَمَامِي بِالأَمَانَةِ مِنْ كُلِّ قُلُوبِهِمْ وَكُلِّ أَنْفُسِهِمْ لاَ يُعْدَمُ لَكَ رَجُلٌ عَنْ كُرْسِيِّ إِسْرَائِيلَ.
\par 5 وَأَنْتَ أَيْضاً تَعْلَمُ مَا فَعَلَ بِي يُوآبُ ابْنُ صَرُويَةَ، مَا فَعَلَ لِرَئِيسَيْ جُيُوشِ إِسْرَائِيلَ: ابْنَيْرَ بْنِ نَيْرَ وَعَمَاسَا بْنِ يَثْرٍ إِذْ قَتَلَهُمَا وَسَفَكَ دَمَ الْحَرْبِ فِي الصُّلْحِ، وَجَعَلَ دَمَ الْحَرْبِ فِي مِنْطَقَتِهِ الَّتِي عَلَى حَقَوَيْهِ وَفِي نَعْلَيْهِ اللَّتَيْنِ بِرِجْلَيْهِ.
\par 6 فَافْعَلْ حَسَبَ حِكْمَتِكَ وَلاَ تَدَعْ شَيْبَتَهُ تَنْحَدِرُ بِسَلاَمٍ إِلَى الْهَاوِيَةِ.
\par 7 وَافْعَلْ مَعْرُوفاً لِبَنِي بَرْزِلاَّيَ الْجِلْعَادِيِّ فَيَكُونُوا بَيْنَ الآكِلِينَ عَلَى مَائِدَتِكَ، لأَنَّهُمْ تَقَدَّمُوا إِلَيَّ عِنْدَ هَرَبِي مِنْ وَجْهِ أَبْشَالُومَ أَخِيكَ.
\par 8 وَهُوَذَا مَعَكَ شَمْعِي بْنُ جِيرَا الْبِنْيَامِينِيُّ مِنْ بَحُورِيمَ. وَهُوَ لَعَنَنِي لَعْنَةً شَدِيدَةً يَوْمَ انْطَلَقْتُ إِلَى مَحَنَايِمَ وَقَدْ نَزَلَ لِلِقَائِي إِلَى الأُرْدُنِّ، فَحَلَفْتُ لَهُ بِالرَّبِّ إِنِّي لاَ أُمِيتُكَ بِالسَّيْفِ.
\par 9 وَالآنَ فَلاَ تُبَرِّرْهُ لأَنَّكَ أَنْتَ رَجُلٌ حَكِيمٌ، فَاعْلَمْ مَا تَفْعَلُ بِهِ وَأَحْدِرْ شَيْبَتَهُ بِالدَّمِ إِلَى الْهَاوِيَةِ].
\par 10 وَاضْطَجَعَ دَاوُدُ مَعَ آبَائِهِ وَدُفِنَ فِي مَدِينَةِ دَاوُدَ.
\par 11 وَكَانَ الزَّمَانُ الَّذِي مَلَكَ فِيهِ دَاوُدُ عَلَى إِسْرَائِيلَ أَرْبَعِينَ سَنَةً. فِي حَبْرُونَ مَلَكَ سَبْعَ سِنِينٍ، وَفِي أُورُشَلِيمَ مَلَكَ ثَلاَثاً وَثَلاَثِينَ سَنَةً.
\par 12 وَجَلَسَ سُلَيْمَانُ عَلَى كُرْسِيِّ دَاوُدَ أَبِيهِ وَتَثَبَّتَ مُلْكُهُ جِدّاً.
\par 13 ثُمَّ جَاءَ أَدُونِيَّا ابْنُ حَجِّيثَ إِلَى بَثْشَبَعَ أُمِّ سُلَيْمَانَ. فَقَالَتْ: [أَلِلسَّلاَمِ جِئْتَ؟] فَقَالَ: [لِلسَّلاَمِ].
\par 14 ثُمَّ قَالَ: [لِي مَعَكِ كَلِمَةٌ]. فَقَالَتْ: [تَكَلَّمْ].
\par 15 فَقَالَ: [أَنْتِ تَعْلَمِينَ أَنَّ الْمُلْكَ كَانَ لِي، وَقَدْ جَعَلَ جَمِيعُ إِسْرَائِيلَ وُجُوهَهُمْ نَحْوِي لأَمْلِكَ، فَدَارَ الْمُلْكُ وَصَارَ لأَخِي لأَنَّهُ مِنْ قِبَلِ الرَّبِّ صَارَ لَهُ.
\par 16 وَالآنَ أَسْأَلُكِ سُؤَالاً وَاحِداً فَلاَ تَرُدِّينِي فِيهِ]. فَقَالَتْ لَهُ: [تَكَلَّمْ].
\par 17 فَقَالَ: [قُولِي لِسُلَيْمَانَ الْمَلِكِ، لأَنَّهُ لاَ يَرُدُّكِ، أَنْ يُعْطِيَنِي أَبِيشَجَ الشُّونَمِيَّةَ امْرَأَةً].
\par 18 فَقَالَتْ بَثْشَبَعُ: [حَسَناً. أَنَا أَتَكَلَّمُ عَنْكَ إِلَى الْمَلِكِ].
\par 19 فَدَخَلَتْ بَثْشَبَعُ إِلَى الْمَلِكِ سُلَيْمَانَ لِتُكَلِّمَهُ عَنْ أَدُونِيَّا. فَقَامَ الْمَلِكُ لِلِقَائِهَا وَسَجَدَ لَهَا وَجَلَسَ عَلَى كُرْسِيِّهِ، وَوَضَعَ كُرْسِيّاً لِأُمِّ الْمَلِكِ فَجَلَسَتْ عَنْ يَمِينِهِ.
\par 20 وَقَالَتْ: [إِنَّمَا أَسْأَلُكَ سُؤَالاً وَاحِداً صَغِيراً. لاَ تَرُدَّنِي]. فَقَالَ لَهَا الْمَلِكُ: [اسْأَلِي يَا أُمِّي لأَنِّي لاَ أَرُدُّكِ].
\par 21 فَقَالَتْ: [لِتُعْطَ أَبِيشَجُ الشُّونَمِيَّةُ لأَدُونِيَّا أَخِيكَ امْرَأَةً].
\par 22 فَأَجَابَ الْمَلِكُ سُلَيْمَانُ: [وَلِمَاذَا أَنْتِ تَسْأَلِينَ أَبِيشَجَ الشُّونَمِيَّةَ لأَدُونِيَّا؟ فَاسْأَلِي لَهُ الْمُلْكَ لأَنَّهُ أَخِي الأَكْبَرُ مِنِّي! لَهُ وَلأَبِيَاثَارَ الْكَاهِنِ وَلِيُوآبَ ابْنِ صَرُويَةَ].
\par 23 وَحَلَفَ سُلَيْمَانُ الْمَلِكُ بِالرَّبِّ: [هَكَذَا يَفْعَلُ لِيَ اللَّهُ وَهَكَذَا يَزِيدُ إِنَّهُ قَدْ تَكَلَّمَ أَدُونِيَّا بِهَذَا الْكَلاَمِ ضِدَّ نَفْسِهِ.
\par 24 وَالآنَ حَيٌّ هُوَ الرَّبُّ الَّذِي ثَبَّتَنِي وَأَجْلَسَنِي عَلَى كُرْسِيِّ دَاوُدَ أَبِي، وَالَّذِي صَنَعَ لِي بَيْتاً كَمَا تَكَلَّمَ، إِنَّهُ الْيَوْمَ يُقْتَلُ أَدُونِيَّا].
\par 25 فَأَرْسَلَ الْمَلِكُ سُلَيْمَانُ بِيَدِ بَنَايَاهُو بْنِ يَهُويَادَاعَ فَبَطَشَ بِهِ فَمَاتَ.
\par 26 وَقَالَ الْمَلِكُ لأَبِيَاثَارَ الْكَاهِنِ: [اذْهَبْ إِلَى عَنَاثُوثَ إِلَى حُقُولِكَ لأَنَّكَ مُسْتَوْجِبُ الْمَوْتِ، وَلَسْتُ أَقْتُلُكَ فِي هَذَا الْيَوْمِ لأَنَّكَ حَمَلْتَ تَابُوتَ سَيِّدِي الرَّبِّ أَمَامَ دَاوُدَ أَبِي، وَلأَنَّكَ تَذَلَّلْتَ بِكُلِّ مَا تَذَلَّلَ بِهِ أَبِي].
\par 27 وَطَرَدَ سُلَيْمَانُ أَبِيَاثَارَ عَنْ أَنْ يَكُونَ كَاهِناً لِلرَّبِّ لإِتْمَامِ كَلاَمِ الرَّبِّ الَّذِي تَكَلَّمَ بِهِ عَلَى بَيْتِ عَالِي فِي شِيلُوهَ.
\par 28 فَأَتَى الْخَبَرُ إِلَى يُوآبَ، لأَنَّ يُوآبَ مَالَ وَرَاءَ أَدُونِيَّا وَلَمْ يَمِلْ وَرَاءَ أَبْشَالُومَ. فَهَرَبَ يُوآبُ إِلَى خَيْمَةِ الرَّبِّ وَتَمَسَّكَ بِقُرُونِ الْمَذْبَحِ.
\par 29 فَأُخْبِرَ الْمَلِكُ سُلَيْمَانُ بِأَنَّ يُوآبَ قَدْ هَرَبَ إِلَى خَيْمَةِ الرَّبِّ وَهَا هُوَ بِجَانِبِ الْمَذْبَحِ. فَأَرْسَلَ سُلَيْمَانُ بَنَايَاهُوَ بْنَ يَهُويَادَاعَ قَائِلاً: [اذْهَبِ ابْطُِشْ بِهِ].
\par 30 فَدَخَلَ بَنَايَاهُو إِلَى خَيْمَةِ الرَّبِّ وَقَالَ لَهُ: [هَكَذَا يَقُولُ الْمَلِكُ: اخْرُجْ]. فَقَالَ: [كَلاَّ وَلَكِنَّنِي هُنَا أَمُوتُ]. فَرَدَّ بَنَايَاهُو الْجَوَابَ عَلَى الْمَلِكِ قَائِلاً: [هَكَذَا تَكَلَّمَ يُوآبُ وَهَكَذَا جَاوَبَنِي].
\par 31 فَقَالَ لَهُ الْمَلِكُ: [افْعَلْ كَمَا تَكَلَّمَ، وَابْطِشْ بِهِ وَادْفِنْهُ، وَأَزِلْ عَنِّي وَعَنْ بَيْتِ أَبِي الدَّمَ الزَّكِيَّ الَّذِي سَفَكَهُ يُوآبُ،
\par 32 فَيَرُدُّ الرَّبُّ دَمَهُ عَلَى رَأْسِهِ لأَنَّهُ بَطَشَ بِرَجُلَيْنِ بَرِيئَيْنِ وَخَيْرٍ مِنْهُ وَقَتَلَهُمَا بِالسَّيْفِ وَأَبِي دَاوُدُ لاَ يَعْلَمُ، وَهُمَا أَبْنَيْرُ بْنُ نَيْرٍ رَئِيسُ جَيْشِ إِسْرَائِيلَ وَعَمَاسَا بْنُ يَثَرٍ رَئِيسُ جَيْشِ يَهُوذَا.
\par 33 فَيَرْتَدُّ دَمُهُمَا عَلَى رَأْسِ يُوآبَ وَرَأْسِ نَسْلِهِ إِلَى الأَبَدِ، وَيَكُونُ لِدَاوُدَ وَنَسْلِهِ وَبَيْتِهِ وَكُرْسِيِّهِ سَلاَمٌ إِلَى الأَبَدِ مِنْ عِنْدِ الرَّبِّ].
\par 34 فَصَعِدَ بَنَايَاهُو بْنُ يَهُويَادَاعَ وَبَطَشَ بِهِ وَقَتَلَهُ، فَدُفِنَ فِي بَيْتِهِ فِي الْبَرِّيَّةِ.
\par 35 وَجَعَلَ الْمَلِكُ بَنَايَاهُوَ بْنَ يَهُويَادَاعَ مَكَانَهُ عَلَى الْجَيْشِ، وَجَعَلَ الْمَلِكُ صَادُوقَ الْكَاهِنَ مَكَانَ أَبِيَاثَارَ.
\par 36 ثُمَّ أَرْسَلَ الْمَلِكُ وَدَعَا شَمْعِيَ وَقَالَ لَهُ: [ابْنِ لِنَفْسِكَ بَيْتاً فِي أُورُشَلِيمَ وَأَقِمْ هُنَاكَ وَلاَ تَخْرُجْ مِنْ هُنَاكَ إِلَى هُنَا أَوْ هُنَالِكَ.
\par 37 فَيَوْمَ تَخْرُجُ وَتَعْبُرُ وَادِيَ قَدْرُونَ اعْلَمَنَّ بِأَنَّكَ مَوْتاً تَمُوتُ، وَيَكُونُ دَمُكَ عَلَى رَأْسِكَ].
\par 38 فَقَالَ شَمْعِي لِلْمَلِكِ: [حَسَنٌ الأَمْرُ. كَمَا تَكَلَّمَ سَيِّدِي الْمَلِكُ كَذَلِكَ يَصْنَعُ عَبْدُكَ]. فَأَقَامَ شَمْعِي فِي أُورُشَلِيمَ أَيَّاماً كَثِيرَةً.
\par 39 وَفِي نِهَايَةِ ثَلاَثِ سِنِينَ هَرَبَ عَبْدَانِ لِشَمْعِي إِلَى أَخِيشَ بْنِ مَعْكَةَ مَلِكِ جَتَّ، فَأَخْبَرُوا شَمْعِي: [هُوَذَا عَبْدَاكَ فِي جَتَّ].
\par 40 فَقَامَ شَمْعِي وَشَدَّ عَلَى حِمَارِهِ وَذَهَبَ إِلَى جَتَّ إِلَى أَخِيشَ لِيُفَتِّشَ عَلَى عَبْدَيْهِ، فَانْطَلَقَ شَمْعِي وَأَتَى بِعَبْدَيْهِ مِنْ جَتَّ.
\par 41 فَأُخْبِرَ سُلَيْمَانُ بِأَنَّ شَمْعِي قَدِ انْطَلَقَ مِنْ أُورُشَلِيمَ إِلَى جَتَّ وَرَجَعَ.
\par 42 فَأَرْسَلَ الْمَلِكُ وَدَعَا شَمْعِيَ وَقَالَ لَهُ: [أَمَا اسْتَحْلَفْتُكَ بِالرَّبِّ وَأَشْهَدْتُ عَلَيْكَ إِنَّكَ يَوْمَ تَخْرُجُ وَتَذْهَبُ إِلَى هُنَا وَهُنَالِكَ اعْلَمَنَّ بِأَنَّكَ مَوْتاً تَمُوتُ، فَقُلْتَ لِي: حَسَنٌ الأَمْرُ. قَدْ سَمِعْتُ.
\par 43 فَلِمَاذَا لَمْ تَحْفَظْ يَمِينَ الرَّبِّ وَالْوَصِيَّةَ الَّتِي أَوْصَيْتُكَ بِهَا؟]
\par 44 ثُمَّ قَالَ الْمَلِكُ لِشَمْعِي: [أَنْتَ عَرَفْتَ كُلَّ الشَّرِّ الَّذِي عَلِمَهُ قَلْبُكَ الَّذِي فَعَلْتَهُ لِدَاوُدَ أَبِي، فَلْيَرُدَّ الرَّبُّ شَرَّكَ عَلَى رَأْسِكَ.
\par 45 وَالْمَلِكُ سُلَيْمَانُ يُبَارَكُ وَكُرْسِيُّ دَاوُدَ يَكُونُ ثَابِتاً أَمَامَ الرَّبِّ إِلَى الأَبَدِ].
\par 46 وَأَمَرَ الْمَلِكُ بَنَايَاهُوَ بْنَ يَهُويَادَاعَ فَخَرَجَ وَبَطَشَ بِهِ فَمَاتَ. وَتَثَبَّتَ الْمُلْكُ بِيَدِ سُلَيْمَانَ.

\chapter{3}

\par 1 وَصَاهَرَ سُلَيْمَانُ فِرْعَوْنَ مَلِكَ مِصْرَ وَأَخَذَ بِنْتَ فِرْعَوْنَ وَأَتَى بِهَا إِلَى مَدِينَةِ دَاوُدَ إِلَى أَنْ أَكْمَلَ بِنَاءَ بَيْتِهِ وَبَيْتِ الرَّبِّ وَسُورِ أُورُشَلِيمَ حَوَالَيْهَا.
\par 2 إِلاَّ أَنَّ الشَّعْبَ كَانُوا يَذْبَحُونَ فِي الْمُرْتَفَعَاتِ، لأَنَّهُ لَمْ يُبْنَ بَيْتٌ لاِسْمِ الرَّبِّ إِلَى تِلْكَ الأَيَّامِ.
\par 3 وَأَحَبَّ سُلَيْمَانُ الرَّبَّ سَائِراً فِي فَرَائِضِ دَاوُدَ أَبِيهِ، إِلاَّ أَنَّهُ كَانَ يَذْبَحُ وَيُوقِدُ فِي الْمُرْتَفَعَاتِ.
\par 4 وَذَهَبَ الْمَلِكُ إِلَى جِبْعُونَ لِيَذْبَحَ هُنَاكَ، لأَنَّهَا هِيَ الْمُرْتَفَعَةُ الْعُظْمَى. وَأَصْعَدَ سُلَيْمَانُ أَلْفَ مُحْرَقَةٍ عَلَى ذَلِكَ الْمَذْبَحِ.
\par 5 فِي جِبْعُونَ تَرَاءَى الرَّبُّ لِسُلَيْمَانَ فِي حُلْمٍ لَيْلاً. وَقَالَ اللَّهُ: [اسْأَلْ مَاذَا أُعْطِيكَ].
\par 6 فَقَالَ سُلَيْمَانُ: [إِنَّكَ قَدْ فَعَلْتَ مَعَ عَبْدِكَ دَاوُدَ أَبِي رَحْمَةً عَظِيمَةً حَسْبَمَا سَارَ أَمَامَكَ بِأَمَانَةٍ وَبِرٍّ وَاسْتِقَامَةِ قَلْبٍ مَعَكَ، فَحَفِظْتَ لَهُ هَذِهِ الرَّحْمَةَ الْعَظِيمَةَ وَأَعْطَيْتَهُ ابْناً يَجْلِسُ عَلَى كُرْسِيِّهِ كَهَذَا الْيَوْمِ.
\par 7 وَالآنَ أَيُّهَا الرَّبُّ إِلَهِي، أَنْتَ مَلَّكْتَ عَبْدَكَ مَكَانَ دَاوُدَ أَبِي، وَأَنَا فَتىً صَغِيرٌ لاَ أَعْلَمُ الْخُرُوجَ وَالدُّخُولَ.
\par 8 وَعَبْدُكَ فِي وَسَطِ شَعْبِكَ الَّذِي اخْتَرْتَهُ شَعْبٌ كَثِيرٌ لاَ يُحْصَى وَلاَ يُعَدُّ مِنَ الْكَثْرَةِ.
\par 9 فَأَعْطِ عَبْدَكَ قَلْباً فَهِيماً لأَحْكُمَ عَلَى شَعْبِكَ وَأُمَيِّزَ بَيْنَ الْخَيْرِ وَالشَّرِّ، لأَنَّهُ مَنْ يَقْدُِرُ أَنْ يَحْكُمَ عَلَى شَعْبِكَ الْعَظِيمِ هَذَا؟]
\par 10 فَحَسُنَ الْكَلاَمُ فِي عَيْنَيِ الرَّبِّ، لأَنَّ سُلَيْمَانَ سَأَلَ هَذَا الأَمْرَ.
\par 11 فَقَالَ لَهُ اللَّهُ: [مِنْ أَجْلِ أَنَّكَ قَدْ سَأَلْتَ هَذَا الأَمْرَ وَلَمْ تَسْأَلْ لِنَفْسِكَ أَيَّاماً كَثِيرَةً وَلاَ سَأَلْتَ لِنَفْسِكَ غِنًى وَلاَ سَأَلْتَ أَنْفُسَ أَعْدَائِكَ، بَلْ سَأَلْتَ لِنَفْسِكَ تَمْيِيزاً لِتَفْهَمَ الْحُكْمَ،
\par 12 هُوَذَا قَدْ فَعَلْتُ حَسَبَ كَلاَمِكَ. هُوَذَا أَعْطَيْتُكَ قَلْباً حَكِيماً وَمُمَيِّزاً حَتَّى إِنَّهُ لَمْ يَكُنْ مِثْلُكَ قَبْلَكَ وَلاَ يَقُومُ بَعْدَكَ نَظِيرُكَ.
\par 13 وَقَدْ أَعْطَيْتُكَ أَيْضاً مَا لَمْ تَسْأَلْهُ، غِنًى وَكَرَامَةً حَتَّى إِنَّهُ لاَ يَكُونُ رَجُلٌ مِثْلَكَ فِي الْمُلُوكِ كُلَّ أَيَّامِكَ.
\par 14 فَإِنْ سَلَكْتَ فِي طَرِيقِي وَحَفِظْتَ فَرَائِضِي وَوَصَايَايَ كَمَا سَلَكَ دَاوُدُ أَبُوكَ فَإِنِّي أُطِيلُ أَيَّامَكَ].
\par 15 فَاسْتَيْقَظَ سُلَيْمَانُ وَإِذَا هُوَ حُلْمٌ. وَجَاءَ إِلَى أُورُشَلِيمَ وَوَقَفَ أَمَامَ تَابُوتِ عَهْدِ الرَّبِّ وَأَصْعَدَ مُحْرَقَاتٍ وَقَرَّبَ ذَبَائِحَ سَلاَمَةٍ وَعَمِلَ وَلِيمَةً لِكُلِّ عَبِيدِهِ.
\par 16 حِينَئِذٍ أَتَتْ زَانِيَتَانِ إِلَى الْمَلِكِ وَوَقَفَتَا بَيْنَ يَدَيْهِ.
\par 17 فَقَالَتِ الْوَاحِدَةُ: [اسْتَمِعْ يَا سَيِّدِي. إِنِّي أَنَا وَهَذِهِ الْمَرْأَةُ سَاكِنَتَانِ فِي بَيْتٍ وَاحِدٍ، وَقَدْ وَلَدْتُ مَعَهَا فِي الْبَيْتِ.
\par 18 وَفِي الْيَوْمِ الثَّالِثِ بَعْدَ وِلاَدَتِي وَلَدَتْ هَذِهِ الْمَرْأَةُ أَيْضاً، وَكُنَّا مَعاً وَلَمْ يَكُنْ مَعَنَا غَرِيبٌ فِي الْبَيْتِ.
\par 19 فَمَاتَ ابْنُ هَذِهِ فِي اللَّيْلِ لأَنَّهَا اضْطَجَعَتْ عَلَيْهِ.
\par 20 فَقَامَتْ فِي وَسَطِ اللَّيْلِ وَأَخَذَتِ ابْنِي مِنْ جَانِبِي وَأَمَتُكَ نَائِمَةٌ، وَأَضْجَعَتْهُ فِي حِضْنِهَا، وَأَضْجَعَتِ ابْنَهَا الْمَيِّتَ فِي حِضْنِي.
\par 21 فَلَمَّا قُمْتُ صَبَاحاً لِأُرَضِّعَ ابْنِي إِذَا هُوَ مَيِّتٌ. وَلَمَّا تَأَمَّلْتُ فِيهِ فِي الصَّبَاحِ إِذَا هُوَ لَيْسَ ابْنِيَ الَّذِي وَلَدْتُهُ].
\par 22 وَكَانَتِ الْمَرْأَةُ الأُخْرَى تَقُولُ: [كَلاَّ بَلِ ابْنِيَ الْحَيُّ وَابْنُكِ الْمَيِّتُ]. وَهَذِهِ تَقُولُ: [لاَ بَلِ ابْنُكِ الْمَيِّتُ وَابْنِيَ الْحَيُّ]. وَتَكَلَّمَتَا أَمَامَ الْمَلِكِ.
\par 23 فَقَالَ الْمَلِكُ: [هَذِهِ تَقُولُ: هَذَا ابْنِيَ الْحَيُّ وَابْنُكِ الْمَيِّتُ، وَتِلْكَ تَقُولُ: لاَ بَلِ ابْنُكِ الْمَيِّتُ وَابْنِيَ الْحَيُّ.
\par 24 اِيتُونِي بِسَيْفٍ]. فَأَتُوا بِسَيْفٍ بَيْنَ يَدَيِ الْمَلِكِ.
\par 25 فَقَالَ الْمَلِكُ: [اشْطُرُوا الْوَلَدَ الْحَيَّ اثْنَيْنِ، وَأَعْطُوا نِصْفاً لِلْوَاحِدَةِ وَنِصْفاً لِلأُخْرَى].
\par 26 فَقَالَتِ الْمَرْأَةُ الَّتِي ابْنُهَا الْحَيُّ لِلْمَلِكِ (لأَنَّ أَحْشَاءَهَا اضْطَرَمَتْ عَلَى ابْنِهَا): [اسْتَمِعْ يَا سَيِّدِي. أَعْطُوهَا الْوَلَدَ الْحَيَّ وَلاَ تُمِيتُوهُ]. وَأَمَّا تِلْكَ فَقَالَتْ: [لاَ يَكُونُ لِي وَلاَ لَكِ. اشْطُرُوهُ].
\par 27 فَأَمَرَ الْمَلِكُ: [أَعْطُوهَا الْوَلَدَ الْحَيَّ وَلاَ تُمِيتُوهُ فَإِنَّهَا أُمُّهُ].
\par 28 وَلَمَّا سَمِعَ جَمِيعُ إِسْرَائِيلَ بِالْحُكْمِ الَّذِي حَكَمَ بِهِ الْمَلِكُ هَابُوا الْمَلِكَ، لأَنَّهُمْ رَأُوا حِكْمَةَ اللَّهِ فِيهِ لإِجْرَاءِ الْحُكْمِ.

\chapter{4}

\par 1 وَكَانَ الْمَلِكُ سُلَيْمَانُ مَلِكاً عَلَى جَمِيعِ إِسْرَائِيلَ.
\par 2 وَهَؤُلاَءِ هُمُ الرُّؤَسَاءُ الَّذِينَ لَهُ: عَزَرْيَاهُو بْنُ صَادُوقَ الْكَاهِنِ،
\par 3 وَأَلِيحُورَفُ وَأَخِيَّا ابْنَا شِيشَا كَاتِبَانِ. وَيَهُوشَافَاطُ بْنُ أَخِيلُودَ الْمُسَجِّلُ،
\par 4 وَبَنَايَاهُو بْنُ يَهُويَادَاعَ عَلَى الْجَيْشِ، وَصَادُوقُ وَأَبِيَاثَارُ كَاهِنَانِ.
\par 5 وَعَزَرْيَاهُو بْنُ نَاثَانَ عَلَى الْوُكَلاَءِ، وَزَابُودُ بْنُ نَاثَانَ كَاهِنٌ وَصَاحِبُ الْمَلِكِ.
\par 6 وَأَخِيشَارُ عَلَى الْبَيْتِ، وَأَدُونِيرَامُ بْنُ عَبْدَا عَلَى التَّسْخِيرِ.
\par 7 وَكَانَ لِسُلَيْمَانَ اثْنَا عَشَرَ وَكِيلاً عَلَى جَمِيعِ إِسْرَائِيلَ يُمَوِّنُونَ الْمَلِكَ وَبَيْتَهُ. كَانَ عَلَى الْوَاحِدِ أَنْ يُمَوِّنَ شَهْراً فِي السَّنَةِ.
\par 8 وَهَذِهِ أَسْمَاؤُهُمُ: ابْنُ حُورَ فِي جَبَلِ أَفْرَايِمَ.
\par 9 ابْنُ دَقَرَ فِي مَاقَصَ وَشَعَلُبِّيمَ وَبَيْتِ شَمْسٍ وَأَيْلُونِ بَيْتِ حَانَانَ.
\par 10 ابْنُ حَسَدَ فِي أَرُبُوتَ. كَانَتْ لَهُ سُوكُوهُ وَكُلُّ أَرْضِ حَافَرَ.
\par 11 ابْنُ أَبِينَادَابَ فِي كُلِّ مُرْتَفَعَاتِ دُورٍ (كَانَتْ طَافَةُ بِنْتُ سُلَيْمَانَ لَهُ امْرَأَةً).
\par 12 بَعْنَا بْنُ أَخِيلُودَ فِي تَعْنَكَ وَمَجِدُّو وَكُلِّ بَيْتِ شَانٍ الَّتِي بِجَانِبِ صَرْتَانَ تَحْتَ يَزْرَعِيلَ، مِنْ بَيْتَِ شَانَ إِلَى آبَلَ مَحُولَةَ إِلَى مَعْبَرِ يَقْمَعَامَ.
\par 13 ابْنُ جَابِرَ فِي رَامُوتِ جِلْعَادَ. لَهُ حَوُّوثُ يَائِيرَ ابْنِ مَنَسَّى الَّتِي فِي جِلْعَادَ. وَلَهُ كُورَةُ أَرْجُوبَ الَّتِي فِي بَاشَانَ. سِتُّونَ مَدِينَةً عَظِيمَةً بِأَسْوَارٍ وَعَوَارِضَ مِنْ نُحَاسٍ.
\par 14 أَخِينَادَابُ بْنُ عِدُّو فِي مَحَنَايِمَ.
\par 15 أَخِيمَعَصُ فِي نَفْتَالِي (وَهُوَ أَيْضاً أَخَذَ بَاسِمَةَ بِنْتَ سُلَيْمَانَ امْرَأَةً).
\par 16 بَعْنَا بْنُ حُوشَايَ فِي أَشِيرَ وَبَعَلُوتَ.
\par 17 يَهُوشَافَاطُ بْنُ فَارُوحَ فِي يَسَّاكَرَ.
\par 18 شَمْعِي بْنُ أَيْلَةَ فِي بِنْيَامِينَ.
\par 19 جَابِرُ بْنُ أُورِي فِي أَرْضِ جِلْعَادَ، أَرْضِ سِيحُونَ مَلِكِ الأَمُورِيِّينَ وَعُوجَ مَلِكِ بَاشَانَ. وَوَكِيلٌ وَاحِدٌ الَّذِي فِي الأَرْضِ.
\par 20 وَكَانَ يَهُوذَا وَإِسْرَائِيلُ كَثِيرِينَ كَالرَّمْلِ الَّذِي عَلَى الْبَحْرِ فِي الْكَثْرَةِ. يَأْكُلُونَ وَيَشْرَبُونَ وَيَفْرَحُونَ.
\par 21 وَكَانَ سُلَيْمَانُ مُتَسَلِّطاً عَلَى جَمِيعِ الْمَمَالِكِ مِنَ النَّهْرِ إِلَى أَرْضِ فِلِسْطِينَ وَإِلَى تُخُومِ مِصْرَ. كَانُوا يُقَدِّمُونَ الْهَدَايَا وَيَخْدِمُونَ سُلَيْمَانَ كُلَّ أَيَّامِ حَيَاتِهِ.
\par 22 وَكَانَ طَعَامُ سُلَيْمَانَ لِلْيَوْمِ الْوَاحِدِ: ثَلاَثِينَ كُرَّ سَمِيذٍ وَسِتِّينَ كُرَّ دَقِيقٍ
\par 23 وَعَشَرَةَ ثِيرَانٍ مُسَمَّنَةٍ وَعِشْرِينَ ثَوْراً مِنَ الْمَرَاعِي وَمِئَةَ خَرُوفٍ، مَا عَدَا الأَيَائِلَ وَالظِّبَاءَ وَالْيَحَامِيرَ وَالأَوِزَّ الْمُسَمَّنَ.
\par 24 لأَنَّهُ كَانَ مُتَسَلِّطاً عَلَى كُلِّ مَا عَبْرَ النَّهْرِ مِنْ تَفْسَحَ إِلَى غَزَّةَ عَلَى كُلِّ مُلُوكِ عَبْرِ النَّهْرِ، وَكَانَ لَهُ صُلْحٌ مِنْ جَمِيعِ جَوَانِبِهِ حَوَالَيْهِ.
\par 25 وَسَكَنَ يَهُوذَا وَإِسْرَائِيلُ آمِنِينَ كُلُّ وَاحِدٍ تَحْتَ كَرْمَتِهِ وَتَحْتَ تِينَتِهِ مِنْ دَانَ إِلَى بِئْرِ سَبْعٍ كُلَّ أَيَّامِ سُلَيْمَانَ.
\par 26 وَكَانَ لِسُلَيْمَانَ أَرْبَعُونَ أَلْفَ مِذْوَدٍ لِخَيْلِ مَرْكَبَاتِهِ، وَاثْنَا عَشَرَ أَلْفَ فَارِسٍ.
\par 27 وَهَؤُلاَءِ الْوُكَلاَءُ كَانُوا يُمَوِّنُونَ الْمَلِكِ سُلَيْمَانَ وَلِكُلِّ مَنْ تَقَدَّمَ إِلَى مَائِدَةِ الْمَلِكِ سُلَيْمَانَ كُلُّ وَاحِدٍ فِي شَهْرِهِ. لَمْ يَكُونُوا يَحْتَاجُونَ إِلَى شَيْءٍ.
\par 28 وَكَانُوا يَأْتُونَ بِشَعِيرٍ وَتِبْنٍ لِلْخَيْلِ وَالْجِيَادِ إِلَى الْمَوْضِعِ الَّذِي يَكُونُ فِيهِ كُلُّ وَاحِدٍ حَسَبَ قَضَائِهِ.
\par 29 وَأَعْطَى اللَّهُ سُلَيْمَانَ حِكْمَةً وَفَهْماً كَثِيراً جِدّاً وَرَحْبَةَ قَلْبٍ كَالرَّمْلِ الَّذِي عَلَى شَاطِئِ الْبَحْرِ.
\par 30 وَفَاقَتْ حِكْمَةُ سُلَيْمَانَ حِكْمَةَ جَمِيعِ بَنِي الْمَشْرِقِ وَكُلَّ حِكْمَةِ مِصْرَ.
\par 31 وَكَانَ أَحْكَمَ مِنْ جَمِيعِ النَّاسِ مِنْ أَيْثَانَ الأَزْرَاحِيِّ وَهَيْمَانَ وَكَلْكُولَ وَدَرْدَعَ بَنِي مَاحُولَ. وَكَانَ صِيتُهُ فِي جَمِيعِ الأُمَمِ حَوَالَيْهِ.
\par 32 وَتَكَلَّمَ بِثَلاَثَةِ آلاَفِ مَثَلٍ، وَكَانَتْ نَشَائِدُهُ أَلْفاً وَخَمْساً.
\par 33 وَتَكَلَّمَ عَنِ الأَشْجَارِ، مِنَ الأَرْزِ الَّذِي فِي لُبْنَانَ إِلَى الزُّوفَا النَّابِتِ فِي الْحَائِطِ. وَتَكَلَّمَ عَنِ الْبَهَائِمِ وَعَنِ الطَّيْرِ وَعَنِ الدَّبِيبِ وَعَنِ السَّمَكِ.
\par 34 وَكَانُوا يَأْتُونَ مِنْ جَمِيعِ الشُّعُوبِ لِيَسْمَعُوا حِكْمَةَ سُلَيْمَانَ مِنْ جَمِيعِ مُلُوكِ الأَرْضِ الَّذِينَ سَمِعُوا بِحِكْمَتِهِ.

\chapter{5}

\par 1 وَأَرْسَلَ حِيرَامُ مَلِكُ صُورَ عَبِيدَهُ إِلَى سُلَيْمَانَ، لأَنَّهُ سَمِعَ أَنَّهُمْ مَسَحُوهُ مَلِكاً مَكَانَ أَبِيهِ، لأَنَّ حِيرَامَ كَانَ مُحِبّاً لِدَاوُدَ كُلَّ الأَيَّامِ.
\par 2 فَأَرْسَلَ سُلَيْمَانُ إِلَى حِيرَامَ يَقُولُ:
\par 3 [أَنْتَ تَعْلَمُ دَاوُدَ أَبِي أَنَّهُ لَمْ يَسْتَطِعْ أَنْ يَبْنِيَ بَيْتاً لاِسْمِ الرَّبِّ إِلَهِهِ بِسَبَبِ الْحُرُوبِ الَّتِي أَحَاطَتْ بِهِ، حَتَّى جَعَلَهُمُ الرَّبُّ تَحْتَ بَطْنِ قَدَمَيْهِ.
\par 4 وَالآنَ فَقَدْ أَرَاحَنِيَ الرَّبُّ إِلَهِي مِنْ كُلِّ الْجِهَاتِ فَلاَ يُوجَدُ خَصْمٌ وَلاَ حَادِثَةُ شَرٍّ.
\par 5 وَهَئَنَذَا قَائِلٌ عَلَى بِنَاءِ بَيْتٍ لاِسْمِ الرَّبِّ إِلَهِي كَمَا قَالَ الرَّبُّ لِدَاوُدَ أَبِي: إِنَّ ابْنَكَ الَّذِي أَجْعَلُهُ مَكَانَكَ عَلَى كُرْسِيِّكَ هُوَ يَبْنِي الْبَيْتَ لاِسْمِي.
\par 6 وَالآنَ فَأْمُرْ أَنْ يَقْطَعُوا لِي أَرْزاً مِنْ لُبْنَانَ وَيَكُونُ عَبِيدِي مَعَ عَبِيدِكَ، وَأُجْرَةُ عَبِيدِكَ أُعْطِيكَ إِيَّاهَا حَسَبَ كُلِّ مَا تَقُولُ، لأَنَّكَ تَعْلَمُ أَنَّهُ لَيْسَ بَيْنَنَا أَحَدٌ يَعْرِفُ قَطْعَ الْخَشَبِ مِثْلَ الصَّيْدُونِيِّينَ].
\par 7 فَلَمَّا سَمِعَ حِيرَامُ كَلاَمَ سُلَيْمَانَ فَرِحَ جِدّاً وَقَالَ: [مُبَارَكٌ الْيَوْمَ الرَّبُّ الَّذِي أَعْطَى دَاوُدَ ابْناً حَكِيماً عَلَى هَذَا الشَّعْبِ الْكَثِيرِ].
\par 8 وَأَرْسَلَ حِيرَامُ إِلَى سُلَيْمَانَ قَائِلاً: [قَدْ سَمِعْتُ مَا أَرْسَلْتَ بِهِ إِلَيَّ. أَنَا أَفْعَلُ كُلَّ مَسَرَّتِكَ فِي خَشَبِ الأَرْزِ وَخَشَبِ السَّرْوِ.
\par 9 عَبِيدِي يُنْزِلُونَ ذَلِكَ مِنْ لُبْنَانَ إِلَى الْبَحْرِ، وَأَنَا أَجْعَلُهُ أَرْمَاثاً فِي الْبَحْرِ إِلَى الْمَوْضِعِ الَّذِي تُعَرِّفُنِي عَنْهُ وَأَفُكُّهُ هُنَاكَ، وَأَنْتَ تَحْمِلُهُ وَتَعْمَلُ مَرْضَاتِي بِإِعْطَائِكَ طَعَاماً لِبَيْتِي].
\par 10 فَكَانَ حِيرَامُ يُعْطِي سُلَيْمَانَ خَشَبَ أَرْزٍ وَخَشَبَ سَرْوٍ حَسَبَ كُلِّ مَسَرَّتِهِ.
\par 11 وَأَعْطَى سُلَيْمَانُ حِيرَامَ عِشْرِينَ أَلْفَ كُرِّ حِنْطَةٍ طَعَاماً لِبَيْتِهِ، وَعِشْرِينَ كُرَّ زَيْتِ رَضٍّ. هَكَذَا كَانَ سُلَيْمَانُ يُعْطِي حِيرَامَ سَنَةً فَسَنَةً.
\par 12 وَالرَّبُّ أَعْطَى سُلَيْمَانَ حِكْمَةً كَمَا كَلَّمَهُ. وَكَانَ صُلْحٌ بَيْنَ حِيرَامَ وَسُلَيْمَانَ، وَقَطَعَا كِلاَهُمَا عَهْداً.
\par 13 وَسَخَّرَ الْمَلِكُ سُلَيْمَانُ مِنْ جَمِيعِ إِسْرَائِيلَ، وَكَانَتِ السُّخَرُ ثَلاَثِينَ أَلْفَ رَجُلٍ.
\par 14 فَأَرْسَلَهُمْ إِلَى لُبْنَانَ عَشْرَةَ آلاَفٍ فِي الشَّهْرِ بِالنَّوْبَةِ. يَكُونُونَ شَهْراً فِي لُبْنَانَ وَشَهْرَيْنِ فِي بُيُوتِهِمْ. وَكَانَ أَدُونِيرَامُ عَلَى التَّسْخِيرِ.
\par 15 وَكَانَ لِسُلَيْمَانَ سَبْعُونَ أَلْفاً يَحْمِلُونَ أَحْمَالاً، وَثَمَانُونَ أَلْفاً يَقْطَعُونَ فِي الْجَبَلِ،
\par 16 مَا عَدَا رُؤَسَاءَ الْوُكَلاَءِ لِسُلَيْمَانَ الَّذِينَ عَلَى الْعَمَلِ ثَلاَثَةَ آلاَفٍ وَثَلاَثَ مِئَةٍ الْمُتَسَلِّطِينَ عَلَى الشَّعْبِ الْعَامِلِينَ الْعَمَلَ.
\par 17 وَأَمَرَ الْمَلِكُ أَنْ يَقْلَعُوا حِجَارَةً كَبِيرَةً كَرِيمَةً مُرَبَّعَةً لِتَأْسِيسِ الْبَيْتِ.
\par 18 فَنَحَتَهَا بَنَّاؤُو سُلَيْمَانَ وَبَنَّاؤُو حِيرَامَ وَالْجِبْلِيُّونَ، وَهَيَّأُوا الأَخْشَابَ وَالْحِجَارَةَ لِبِنَاءِ الْبَيْتِ.

\chapter{6}

\par 1 وَكَانَ فِي سَنَةِ الأَرْبَعِ مِئَةٍ وَالثَّمَانِينَ لِخُرُوجِ بَنِي إِسْرَائِيلَ مِنْ أَرْضِ مِصْرَ، فِي السَّنَةِ الرَّابِعَةِ لِمُلْكِ سُلَيْمَانَ عَلَى إِسْرَائِيلَ، فِي شَهْرِ زِيُو وَهُوَ الشَّهْرُ الثَّانِي، أَنَّهُ بَنَى الْبَيْتَ لِلرَّبِّ.
\par 2 وَالْبَيْتُ الَّذِي بَنَاهُ الْمَلِكُ سُلَيْمَانُ لِلرَّبِّ طُولُهُ سِتُّونَ ذِرَاعاً، وَعَرْضُهُ عِشْرُونَ ذِرَاعاً، وَارْتِفَاعُهُ ثَلاَثُونَ ذِرَاعاً.
\par 3 وَالرِّوَاقُ قُدَّامَ هَيْكَلِ الْبَيْتِ طُولُهُ عِشْرُونَ ذِرَاعاً حَسَبَ عَرْضِ الْبَيْتِ، وَعَرْضُهُ عَشَرُ أَذْرُعٍ قُدَّامَ الْبَيْتِ.
\par 4 وَعَمِلَ لِلْبَيْتِ كُوًى مَسْقُوفَةً مُشَبَّكَةً.
\par 5 وَبَنَى مَعَ حَائِطِ الْبَيْتِ طِبَاقاً حَوَالَيْهِ مَعَ حِيطَانِ الْبَيْتِ حَوْلَ الْهَيْكَلِ وَالْمِحْرَابِ، وَعَمِلَ غُرُفَاتٍ فِي مُسْتَدِيرِهَا.
\par 6 فَالطَّبَقَةُ السُّفْلَى عَرْضُهَا خَمْسُ أَذْرُعٍ، وَالْوُسْطَى عَرْضُهَا سِتُّ أَذْرُعٍ، وَالثَّالِثَةُ عَرْضُهَا سَبْعُ أَذْرُعٍ، لأَنَّهُ جَعَلَ لِلْبَيْتِ حَوَالَيْهِ مِنْ خَارِجٍ زَوَايَا لِئَلاَّ تَتَمَكَّنَ الْجَوَائِزُ فِي حِيطَانِ الْبَيْتِ.
\par 7 وَالْبَيْتُ فِي بِنَائِهِ بُنِيَ بِحِجَارَةٍ صَحِيحَةٍ مُقْتَلَعَةٍ، وَلَمْ يُسْمَعْ فِي الْبَيْتِ عِنْدَ بِنَائِهِ مِنْحَتٌ وَلاَ مِعْوَلٌ وَلاَ أَدَاةٌ مِنْ حَدِيدٍ.
\par 8 وَكَانَ بَابُ الْغُرْفَةِ الْوُسْطَى فِي جَانِبِ الْبَيْتِ الأَيْمَنِ، وَكَانُوا يَصْعَدُونَ بِدَرَجٍ مُعَطَّفٍ إِلَى الْوُسْطَى، وَمِنَ الْوُسْطَى إِلَى الثَّالِثَةِ.
\par 9 فَبَنَى الْبَيْتَ وَأَكْمَلَهُ وَسَقَفَ الْبَيْتَ بِأَلْوَاحٍ وَجَوَائِزَ مِنَ الأَرْزِ.
\par 10 وَبَنَى الْغُرُفَاتِ عَلَى الْبَيْتِ كُلِّهِ ارْتِفَاعُهَا خَمْسُ أَذْرُعٍ، وَتَمَكَّنَتْ فِي الْبَيْتِ بِخَشَبِ أَرْزٍ.
\par 11 وَكَانَ كَلاَمُ الرَّبِّ إِلَى سُلَيْمَانَ:
\par 12 [هَذَا الْبَيْتُ الَّذِي أَنْتَ بَانِيهِ، إِنْ سَلَكْتَ فِي فَرَائِضِي وَعَمِلْتَ أَحْكَامِي وَحَفِظْتَ كُلَّ وَصَايَايَ لِلسُّلُوكِ بِهَا، فَإِنِّي أُقِيمُ مَعَكَ كَلاَمِي الَّذِي تَكَلَّمْتُ بِهِ إِلَى دَاوُدَ أَبِيكَ،
\par 13 وَأَسْكُنُ فِي وَسَطِ بَنِي إِسْرَائِيلَ وَلاَ أَتْرُكُ شَعْبِي إِسْرَائِيلَ].
\par 14 فَبَنَى سُلَيْمَانُ الْبَيْتَ وَأَكْمَلَهُ.
\par 15 وَبَنَى حِيطَانَ الْبَيْتِ مِنْ دَاخِلٍ بِأَضْلاَعِ أَرْزٍ مِنْ أَرْضِ الْبَيْتِ إِلَى حِيطَانِ السَّقْفِ، وَغَشَّاهُ مِنْ دَاخِلٍ بِخَشَبٍ، وَفَرَشَ أَرْضَ الْبَيْتِ بِأَخْشَابِ سَرْوٍ.
\par 16 وَبَنَى عِشْرِينَ ذِرَاعاً مِنْ مُؤَخَّرِ الْبَيْتِ بِأَضْلاَعِ أَرْزٍ مِنَ الأَرْضِ إِلَى الْحِيطَانِ. وَبَنَى دَاخِلَهُ لأَجْلِ الْمِحْرَابِ (أَيْ قُدْسِ الأَقْدَاسِ).
\par 17 وَأَرْبَعُونَ ذِرَاعاً كَانَتِ الْبَيْتَ (أَيِ الْهَيْكَلَ الَّذِي أَمَامَهُ).
\par 18 وَأَرْزُ الْبَيْتِ مِنْ دَاخِلٍ كَانَ مَنْقُوراً عَلَى شِكْلِ قُثَّاءٍ وَبَرَاعِمِ زُهُورٍ. الْجَمِيعُ أَرْزٌ. لَمْ يَكُنْ يُرَى حَجَرٌ.
\par 19 وَهَيَّأَ مِحْرَاباً فِي وَسَطِ الْبَيْتِ مِنْ دَاخِلٍ لِيَضَعَ هُنَاكَ تَابُوتَ عَهْدِ الرَّبِّ.
\par 20 وَلأَجْلِ الْمِحْرَابِ عِشْرُونَ ذِرَاعاً طُولاً وَعِشْرُونَ ذِرَاعاً عَرْضاً وَعِشْرُونَ ذِرَاعاً ارْتِفَاعاً. وَغَشَّاهُ بِذَهَبٍ خَالِصٍ، وَغَشَّى الْمَذْبَحَ بِأَرْزٍ.
\par 21 وَغَشَّى سُلَيْمَانُ الْبَيْتَ مِنْ دَاخِلٍ بِذَهَبٍ خَالِصٍ. وَسَدَّ بِسَلاَسِلِ ذَهَبٍ قُدَّامَ الْمِحْرَابِ. وَغَشَّاهُ بِذَهَبٍ.
\par 22 وَجَمِيعُ الْبَيْتِ غَشَّاهُ بِذَهَبٍ إِلَى تَمَامِ كُلِّ الْبَيْتِ، وَكُلُّ الْمَذْبَحِ الَّذِي لِلْمِحْرَابِ غَشَّاهُ بِذَهَبٍ.
\par 23 وَعَمِلَ فِي الْمِحْرَابِ كَرُوبَيْنِ مِنْ خَشَبِ الزَّيْتُونِ، عُلُوُّ الْوَاحِدِ عَشَرُ أَذْرُعٍ.
\par 24 وَخَمْسُ أَذْرُعٍ جَنَاحُ الْكَرُوبِ الْوَاحِدُ، وَخَمْسُ أَذْرُعٍ جَنَاحُ الْكَرُوبِ الآخَرُ. عَشَرُ أَذْرُعٍ مِنْ طَرَفِ جَنَاحِهِ إِلَى طَرَفِ جَنَاحِهِ.
\par 25 وَعَشَرُ أَذْرُعٍ الْكَرُوبُ الآخَرُ. قِيَاسٌ وَاحِدٌ وَشَكْلٌ وَاحِدٌ لِلْكَرُوبَيْنِ.
\par 26 عُلُوُّ الْكَرُوبِ الْوَاحِدِ عَشَرُ أَذْرُعٍ وَكَذَا الْكَرُوبُ الآخَرُ.
\par 27 وَجَعَلَ الْكَرُوبَيْنِ فِي وَسَطِ الْبَيْتِ الدَّاخِلِيِّ، وَبَسَطُوا أَجْنِحَةَ الْكَرُوبَيْنِ فَمَسَّ جَنَاحُ الْوَاحِدِ الْحَائِطَ وَجَنَاحُ الْكَرُوبِ الآخَرِ مَسَّ الْحَائِطَ الآخَرَ. وَكَانَتْ أَجْنِحَتُهُمَا فِي وَسَطِ الْبَيْتِ يَمَسُّ أَحَدُهُمَا الآخَرَ.
\par 28 وَغَشَّى الْكَرُوبَيْنِ بِذَهَبٍ.
\par 29 وَجَمِيعُ حِيطَانِ الْبَيْتِ فِي مُسْتَدِيرِهَا رَسَمَهَا نَقْشاً بِنَقْرِ كَرُوبِيمَ وَنَخِيلٍ وَبَرَاعِمِ زُهُورٍ مِنْ دَاخِلٍ وَمِنْ خَارِجٍ.
\par 30 وَغَشَّى أَرْضَ الْبَيْتِ بِذَهَبٍ مِنْ دَاخِلٍ وَمِنْ خَارِجٍ.
\par 31 وَعَمِلَ لِبَابِ الْمِحْرَابِ مِصْرَاعَيْنِ مِنْ خَشَبِ الزَّيْتُونِ. الْعَتَبَةِ الْعُلْيَا وَالْقَائِمَتَانِ مُخَمَّسَةٌ.
\par 32 وَالْمِصْرَاعَانِ مِنْ خَشَبِ الزَّيْتُونِ. وَرَسَمَ عَلَيْهِمَا نَقْشَ كَرُوبِيمَ وَنَخِيلٍ وَبَرَاعِمِ زُهُورٍ وَغَشَّاهُمَا بِذَهَبٍ، وَرَصَّعَ الْكَرُوبِيمَ وَالنَّخِيلَ بِذَهَبٍ.
\par 33 وَكَذَلِكَ عَمِلَ لِمَدْخَلِ الْهَيْكَلِ قَوَائِمَ مِنْ خَشَبِ الزَّيْتُونِ مُرَبَّعَةً،
\par 34 وَمِصْرَاعَيْنِ مِنْ خَشَبِ السَّرْوِ. الْمِصْرَاعُ الْوَاحِدُ دَفَّتَانِ تَنْطَوِيَانِ، وَالْمِصْرَاعُ الآخَرُ دَفَّتَانِ تَنْطَوِيَانِ.
\par 35 وَنَحَتَ كَرُوبِيمَ وَنَخِيلاً وَبَرَاعِمَ زُهُورٍ وَغَشَّاهَا بِذَهَبٍ مُطَرَّقٍ عَلَى الْمَنْقُوشِ.
\par 36 وَبَنَى الدَّارَ الدَّاخِلِيَّةَ ثَلاَثَةَ صُفُوفٍ مَنْحُوتَةٍ وَصَفّاً مِنْ جَوَائِزِ الأَرْزِ.
\par 37 فِي السَّنَةِ الرَّابِعَةِ أُسِّسَ بَيْتُ الرَّبِّ فِي شَهْرِ زِيُو.
\par 38 وَفِي السَّنَةِ الْحَادِيَةَِ عَشَرَةَ فِي شَهْرِ بُولَ، وَهُوَ الشَّهْرُ الثَّامِنُ، أُكْمِلَ الْبَيْتُ فِي جَمِيعِ أُمُورِهِ وَأَحْكَامِهِ. فَبَنَاهُ فِي سَبْعِ سِنِينٍ.

\chapter{7}

\par 1 وَأَمَّا بَيْتُهُ فَبَنَاهُ سُلَيْمَانُ فِي ثَلاَثَ عَشَرَةَ سَنَةً وَأَكْمَلَ كُلَّ بَيْتِهِ.
\par 2 وَبَنَى بَيْتَ وَعْرِ لُبْنَانَ طُولُهُ مِئَةُ ذِرَاعٍ وَعَرْضُهُ خَمْسُونَ ذِرَاعاً وَارْتِفَاعُهُ ثَلاَثُونَ ذِرَاعاً، عَلَى أَرْبَعَةِ صُفُوفٍ مِنْ أَعْمِدَةِ أَرْزٍ وَجَوَائِزُ أَرْزٍ عَلَى الأَعْمِدَةِ.
\par 3 وَسُقِفَ بِأَرْزٍ مِنْ فَوْقٍ عَلَى الْغُرُفَاتِ الْخَمْسِ وَالأَرْبَعِينَ الَّتِي عَلَى الأَعْمِدَةِ. كُلُّ صَفٍّ خَمْسَ عَشَرَةَ.
\par 4 وَالسُّقُوفُ ثَلاَثُ طِبَاقٍ وَكُوَّةٌ مُقَابَِلَ كُوَّةٍ ثَلاَثَ مَرَّاتٍ.
\par 5 وَجَمِيعُ الأَبْوَابِ وَالْقَوَائِمِ مُرَبَّعَةٌ مَسْقُوفَةٌ، وَوَجْهُ كُوَّةٍ مُقَابَِلَ كُوَّةٍ ثَلاَثَ مَرَّاتٍ.
\par 6 وَعَمِلَ رِوَاقَ الأَعْمِدَةِ طُولُهُ خَمْسُونَ ذِرَاعاً وَعَرْضُهُ ثَلاَثُونَ ذِرَاعاً. وَرِوَاقاً آخَرَ قُدَّامَهَا وَأَعْمِدَةً وَأَفَارِيزَ قُدَّامَهَا.
\par 7 وَعَمِلَ رِوَاقَ الْكُرْسِيِّ حَيْثُ يَقْضِي (أَيْ رِوَاقَ الْقَضَاءِ) وَغُشِّيَ بِأَرْزٍ مِنْ أَرْضٍ إِلَى سَقْفٍ.
\par 8 وَبَيْتُهُ الَّذِي كَانَ يَسْكُنُهُ فِي دَارٍ أُخْرَى دَاخِلَ الرِّوَاقِ كَانَ كَهَذَا الْعَمَلِ. وَعَمِلَ بَيْتاً لاِبْنَةِ فِرْعَوْنَ الَّتِي أَخَذَهَا سُلَيْمَانُ كَهَذَا الرِّوَاقِ.
\par 9 كُلُّ هَذِهِ مِنْ حِجَارَةٍ كَرِيمَةٍ كَقِيَاسِ الْحِجَارَةِ الْمَنْحُوتَةِ مَنْشُورَةٍ بِمِنْشَارٍ مِنْ دَاخِلٍ وَمِنْ خَارِجٍ مِنَ الأَسَاسِ إِلَى الإِفْرِيزِ وَمِنْ خَارِجٍ إِلَى الدَّارِ الْكَبِيرَةِ.
\par 10 وَكَانَ مُؤَسَّساً عَلَى حِجَارَةٍ كَرِيمَةٍ عَظِيمَةٍ، حِجَارَةِ عَشَرِ أَذْرُعٍ، وَحِجَارَةِ ثَمَانِ أَذْرُعٍ.
\par 11 وَمِنْ فَوْقٍ حِجَارَةٌ كَرِيمَةٌ كَقِيَاسِ الْمَنْحُوتَةِ، وَأَرْزٌ.
\par 12 وَلِلدَّارِ الْكَبِيرَةِ فِي مُسْتَدِيرِهَا ثَلاَثَةُ صُفُوفٍ مَنْحُوتَةٍ وَصَفٌّ مِنْ جَوَائِزِ الأَرْزِ. كَذَلِكَ دَارُ بَيْتِ الرَّبِّ الدَّاخِلِيَّةُ وَرِوَاقُ الْبَيْتِ.
\par 13 وَأَرْسَلَ الْمَلِكُ سُلَيْمَانُ وَأَخَذَ حِيرَامَ مِنْ صُورَ.
\par 14 وَهُوَ ابْنُ أَرْمَلَةٍ مِنْ سِبْطِ نَفْتَالِي، وَأَبُوهُ صُورِيٌّ نَحَّاسٌ، وَكَانَ مُمْتَلِئاً حِكْمَةً وَفَهْماً وَمَعْرِفَةً لِعَمَلِ كُلِّ عَمَلٍ فِي النُّحَاسِ. فَأَتَى إِلَى الْمَلِكِ سُلَيْمَانَ وَعَمِلَ كُلَّ عَمَلِهِ.
\par 15 وَصَوَّرَ الْعَمُودَيْنِ مِنْ نُحَاسٍ، طُولُ الْعَمُودِ الْوَاحِدِ ثَمَانِيَةَ عَشَرَ ذِرَاعاً. وَخَيْطٌ اثْنَتَا عَشَرَةَ ذِرَاعاً يُحِيطُ بِالْعَمُودِ الآخَرِ.
\par 16 وَعَمِلَ تَاجَيْنِ لِيَضَعَهُمَا عَلَى رَأْسَيِ الْعَمُودَيْنِ مِنْ نُحَاسٍ مَسْبُوكٍ. طُولُ التَّاجِ الْوَاحِدِ خَمْسُ أَذْرُعٍ، وَطُولُ التَّاجِ الآخَرِ خَمْسُ أَذْرُعٍ.
\par 17 وَشُبَّاكاً عَمَلاً مُشَبَّكاً وَضَفَائِرَ كَعَمَلِ السَّلاَسِلِ لِلتَّاجَيْنِ اللَّذَيْنِ عَلَى رَأْسَيِ الْعَمُودَيْنِ، سَبْعاً لِلتَّاجِ الْوَاحِدِ وَسَبْعاً لِلتَّاجِ الآخَرِ.
\par 18 وَعَمِلَ لِلْعَمُودَيْنِ صَفَّيْنِ مِنَ الرُّمَّانِ فِي مُسْتَدِيرِهِمَا عَلَى الشَّبَكَةِ الْوَاحِدَةِ لِتَغْطِيَةِ التَّاجِ الَّذِي عَلَى رَأْسِ الْعَمُودِ، وَهَكَذَا عَمِلَ لِلتَّاجِ الآخَرِ.
\par 19 وَالتَّاجَانِ اللَّذَانِ عَلَى رَأْسَيِ الْعَمُودَيْنِ مِنْ صِيغَةِ السَّوْسَنِّ كَمَا فِي الرِّوَاقِ هُمَا أَرْبَعُ أَذْرُعٍ
\par 20 وَكَذَلِكَ التَّاجَانِ اللَّذَانِ عَلَى الْعَمُودَيْنِ مِنْ عِنْدِ الْبَطْنِ الَّذِي مِنْ جِهَةِ الشَّبَكَةِ صَاعِداً. وَالرُّمَّانَاتُ مِئَتَانِ عَلَى صُفُوفٍ مُسْتَدِيرَةٍ عَلَى التَّاجِ الثَّانِي.
\par 21 وَأَوْقَفَ الْعَمُودَيْنِ فِي رِوَاقِ الْهَيْكَلِ. فَأَوْقَفَ الْعَمُودَ الأَيْمَنَ وَدَعَا اسْمَهُ [يَاكِينَ]. ثُمَّ أَوْقَفَ الْعَمُودَ الأَيْسَرَ وَدَعَا اسْمَهُ [بُوعَزَ].
\par 22 وَعَلَى رَأْسِ الْعَمُودَيْنِ صِيغَةُ السَّوْسَنِّ. فَكَمُلَ عَمَلُ الْعَمُودَيْنِ.
\par 23 وَعَمِلَ الْبَحْرَ مَسْبُوكاً. عَشَرَ أَذْرُعٍ مِنْ شَفَتِهِ إِلَى شَفَتِهِ وَكَانَ مُدَوَّراً مُسْتَدِيراً. ارْتِفَاعُهُ خَمْسُ أَذْرُعٍ، وَخَيْطٌ ثَلاَثُونَ ذِرَاعاً يُحِيطُ بِهِ بِدَائِرِهِ.
\par 24 وَتَحْتَ شَفَتِهِ قُثَّاءٌ مُسْتَدِيراً تُحِيطُ بِهِ. عَشَرٌ لِلذِّرَاعِ. مُحِيطَةٌ بِالْبَحْرِ بِمُسْتَدِيرِهِ صَفَّيْنِ. الْقِثَّاءُ قَدْ سُبِكَتْ بِسَبْكِهِ.
\par 25 وَكَانَ قَائِماً عَلَى اثْنَيْ عَشَرَ ثَوْراً ثَلاَثَةٌ مُتَوَجِّهَةٌ إِلَى الشِّمَالِ وَثَلاَثَةٌ مُتَوَجِّهَةٌ إِلَى الْغَرْبِ وَثَلاَثَةٌ مُتَوَجِّهَةٌ إِلَى الْجَنُوبِ وَثَلاَثَةٌ مُتَوَجِّهَةٌ إِلَى الشَّرْقِ. وَالْبَحْرُ عَلَيْهَا مِنْ فَوْقُ، وَجَمِيعُ أَعْجَازِهَا إِلَى دَاخِلٍ.
\par 26 وَسُمْكُهُ شِبْرٌ وَشَفَتُهُ كَعَمَلِ شَفَةِ كَأْسٍ بِزَهْرِ سَوْسَنٍّ. يَسَعُ أَلْفَيْ بَثٍّ.
\par 27 وَعَمِلَ الْقَوَاعِدَ الْعَشَرَ مِنْ نُحَاسٍ، طُولُ الْقَاعِدَةِ الْوَاحِدَةِ أَرْبَعُ أَذْرُعٍ وَعَرْضُهَا أَرْبَعُ أَذْرُعٍ وَارْتِفَاعُهَا ثَلاَثُ أَذْرُعٍ.
\par 28 وَهَذَا عَمَلُ الْقَوَاعِدِ. لَهَا أَتْرَاسٌ، وَالأَتْرَاسُ بَيْنَ الْحَوَاجِبِ.
\par 29 وَعَلَى الأَتْرَاسِ الَّتِي بَيْنَ الْحَوَاجِبِ أُسُودٌ وَثِيرَانٌ وَكَرُوبِيمُ، وَكَذَلِكَ عَلَى الْحَوَاجِبِ مِنْ فَوْقٍ. وَمِنْ تَحْتِ الأُسُودِ وَالثِّيرَانِ قَلاَئِدُ زُهُورٍ عَمَلٌ مُدَلَّى.
\par 30 وَلِكُلِّ قَاعِدَةٍ أَرْبَعُ بَكَرٍ مِنْ نُحَاسٍ وَقِطَابٌ مِنْ نُحَاسٍ، وَلِقَوَائِمِهَا الأَرْبَعِ أَكْتَافٌ، وَالأَكْتَافُ مَسْبُوكَةٌ تَحْتَ الْمِرْحَضَةِ بِجَانِبِ كُلِّ قِلاَدَةٍ.
\par 31 وَفَمُهَا دَاخِلَ الإِكْلِيلِ وَمِنْ فَوْقُ ذِرَاعٌ. وَفَمُهَا مُدَوَّرٌ كَعَمَلِ قَاعِدَةٍ ذِرَاعٌ وَنِصْفُ ذِرَاعٍ. وَأَيْضاً عَلَى فَمِهَا نَقْشٌ. وَأَتْرَاسُهَا مُرَبَّعَةٌ لاَ مُدَوَّرَةٌ.
\par 32 وَالْبَكَرُ الأَرْبَعُ تَحْتَ الأَتْرَاسِ، وَخَطَاطِيفُ الْبَكَرِ فِي الْقَاعِدَةِ، وَارْتِفَاعُ الْبَكَرَةِ الْوَاحِدَةِ ذِرَاعٌ وَنِصْفُ ذِرَاعٍ.
\par 33 وَعَمَلُ الْبَكَرِ كَعَمَلِ بَكَرَةِ مَرْكَبَةٍ. خَطَاطِيفُهَا وَأُطُرُهَا وَأَصَابِعُهَا وَقُبُوبُهَا كُلُّهَا مَسْبُوكَةٌ.
\par 34 وَأَرْبَعُ أَكْتَافٍ عَلَى أَرْبَعِ زَوَايَا الْقَاعِدَةِ الْوَاحِدَةِ، وَأَكْتَافُ الْقَاعِدَةِ مِنْهَا.
\par 35 وَأَعْلَى الْقَاعِدَةِ مُقَبَّبٌ مُسْتَدِيرٌ عَلَى ارْتِفَاعِ نِصْفِ ذِرَاعٍ مِنْ أَعْلَى الْقَاعِدَةِ. أَيَادِيهَا وَأَتْرَاسُهَا مِنْهَا.
\par 36 وَنَقَشَ عَلَى أَلْوَاحِ أَيَادِيهَا وَعَلَى أَتْرَاسِهَا كَرُوبِيمَ وَأُسُوداً وَنَخِيلاً كَسِعَةِ كُلِّ وَاحِدَةٍ، وَقَلاَئِدَ زُهُورٍ مُسْتَدِيرَةً.
\par 37 هَكَذَا عَمِلَ الْقَوَاعِدَ الْعَشَرَ. لِجَمِيعِهَا سَبْكٌ وَاحِدٌ وَقِيَاسٌ وَاحِدٌ وَشَكْلٌ وَاحِدٌ.
\par 38 وَعَمِلَ عَشَرَ مَرَاحِضَ مِنْ نُحَاسٍ تَسَعُ كُلُّ مِرْحَضَةٍ أَرْبَعِينَ بَثّاً. الْمِرْحَضَةُ الْوَاحِدَةُ أَرْبَعُ أَذْرُعٍ. مِرْحَضَةٌ وَاحِدَةٌ عَلَى الْقَاعِدَةِ الْوَاحِدَةِ لِلْعَشَرِ الْقَوَاعِدِ.
\par 39 وَجَعَلَ الْقَوَاعِدَ خَمْساً عَلَى جَانِبِ الْبَيْتِ الأَيْمَنِ وَخَمْساً عَلَى جَانِبِ الْبَيْتِ الأَيْسَرِ، وَجَعَلَ الْبَحْرَ عَلَى جَانِبِ الْبَيْتِ الأَيْمَنِ إِلَى الشَّرْقِ مِنْ جِهَةِ الْجَنُوبِ.
\par 40 وَعَمِلَ حِيرَامُ الْمَرَاحِضَ وَالرُّفُوشَ وَالْمَنَاضِحَ. وَانْتَهَى حِيرَامُ مِنْ جَمِيعِ الْعَمَلِ الَّذِي عَمِلَهُ لِلْمَلِكِ سُلَيْمَانَ لِبَيْتِ الرَّبِّ.
\par 41 الْعَمُودَيْنِ وَكُرَتَيِ التَّاجَيْنِ اللَّذَيْنِ عَلَى رَأْسَيِ الْعَمُودَيْنِ، وَالشَّبَكَتَيْنِ لِتَغْطِيَةِ كُرَتَيِ التَّاجَيْنِ اللَّذَيْنِ عَلَى رَأْسَيِ الْعَمُودَيْنِ.
\par 42 وَأَرْبَعَ مِئَةِ الرُّمَّانَةِ الَّتِي لِلشَّبَكَتَيْنِ (صَفَّا رُمَّانٍ لِلشَّبَكَةِ الْوَاحِدَةِ لأَجْلِ تَغْطِيَةِ كُرَتَيِ التَّاجَيْنِ اللَّذَيْنِ عَلَى الْعَمُودَيْنِ).
\par 43 وَالْقَوَاعِدَ الْعَشَرَ وَالْمَرَاحِضَ الْعَشَرَ عَلَى الْقَوَاعِدِ.
\par 44 وَالْبَحْرَ الْوَاحِدَ وَالاِثْنَيْ عَشَرَ ثَوْراً تَحْتَ الْبَحْرِ.
\par 45 وَالْقُدُورَ وَالرُّفُوشَ وَالْمَنَاضِحَ. وَجَمِيعُ هَذِهِ الآنِيَةِ الَّتِي عَمِلَهَا حِيرَامُ لِلْمَلِكِ سُلَيْمَانَ لِبَيْتِ الرَّبِّ هِيَ مِنْ نُحَاسٍ مَصْقُولٍ.
\par 46 فِي غَوْرِ الأُرْدُنِّ سَبَكَهَا الْمَلِكُ فِي أَرْضِ الْخَزَفِ بَيْنَ سُكُّوتَ وَصَرَتَانَ.
\par 47 وَتَرَكَ سُلَيْمَانُ وَزْنَ جَمِيعِ الآنِيَةِ لأَنَّهَا كَثِيرَةٌ جِدّاً جِدّاً. لَمْ يَتَحَقَّقْ وَزْنُ النُّحَاسِ.
\par 48 وَعَمِلَ سُلَيْمَانُ جَمِيعَ آنِيَةِ بَيْتِ الرَّبِّ: الْمَذْبَحَ مِنْ ذَهَبٍ، وَالْمَائِدَةَ الَّتِي عَلَيْهَا خُبْزُ الْوُجُوهِ، مِنْ ذَهَبٍ.
\par 49 وَالْمَنَائِرَ خَمْساً عَنِ الْيَمِينِ وَخَمْساً عَنِ الْيَسَارِ أَمَامَ الْمِحْرَابِ مِنْ ذَهَبٍ خَالِصٍ، وَالأَزْهَارَ وَالسُّرُجَ وَالْمَلاَقِطَ مِنْ ذَهَبٍ
\par 50 وَالطُّسُوسَ وَالْمَقَاصَّ وَالْمَنَاضِحَ وَالصُّحُونَ وَالْمَجَامِرَ مِنْ ذَهَبٍ خَالِصٍ. وَالْوُصَلَ لِمَصَارِيعِ الْبَيْتِ الدَّاخِلِيِّ (أَيْ لِقُدْسِ الأَقْدَاسِ) وَلأَبْوَابِ الْبَيْتِ (أَيِ الْهَيْكَلِ) مِنْ ذَهَبٍ.
\par 51 وَأُكْمِلَ جَمِيعُ الْعَمَلِ الَّذِي عَمِلَهُ الْمَلِكُ سُلَيْمَانُ لِبَيْتِ الرَّبِّ. وَأَدْخَلَ سُلَيْمَانُ أَقْدَاسَ دَاوُدَ أَبِيهِ: الْفِضَّةَ وَالذَّهَبَ وَالآنِيَةَ، وَجَعَلَهَا فِي خَزَائِنِ بَيْتِ الرَّبِّ.

\chapter{8}

\par 1 حِينَئِذٍ جَمَعَ سُلَيْمَانُ شُيُوخَ إِسْرَائِيلَ وَكُلَّ رُؤُوسِ الأَسْبَاطِ رُؤَسَاءَ الآبَاءِ مِنْ بَنِي إِسْرَائِيلَ إِلَى الْمَلِكِ سُلَيْمَانَ فِي أُورُشَلِيمَ لإِصْعَادِ تَابُوتِ عَهْدِ الرَّبِّ مِنْ مَدِينَةِ دَاوُدَ (هِيَ صِهْيَوْنُ).
\par 2 فَاجْتَمَعَ إِلَى الْمَلِكِ سُلَيْمَانَ جَمِيعُ رِجَالِ إِسْرَائِيلَ فِي الْعِيدِ فِي شَهْرِ أَيْثَانِيمَ. هُوَ الشَّهْرُ السَّابِعُ.
\par 3 وَجَاءَ جَمِيعُ شُيُوخِ إِسْرَائِيلَ، وَحَمَلَ الْكَهَنَةُ التَّابُوتَ
\par 4 وَأَصْعَدُوا تَابُوتَ الرَّبِّ وَخَيْمَةَ الاِجْتِمَاعِ مَعَ جَمِيعِ آنِيَةِ الْقُدْسِ الَّتِي فِي الْخَيْمَةِ، فَأَصْعَدَهَا الْكَهَنَةُ وَاللاَّوِيُّونَ.
\par 5 وَالْمَلِكُ سُلَيْمَانُ وَكُلُّ جَمَاعَةِ إِسْرَائِيلَ الْمُجْتَمِعِينَ إِلَيْهِ مَعَهُ أَمَامَ التَّابُوتِ كَانُوا يَذْبَحُونَ مِنَ الْغَنَمِ وَالْبَقَرِ مَا لاَ يُحْصَى وَلاَ يُعَدُّ مِنَ الْكَثْرَةِ.
\par 6 وَأَدْخَلَ الْكَهَنَةُ تَابُوتَ عَهْدِ الرَّبِّ إِلَى مَكَانِهِ فِي مِحْرَابِ الْبَيْتِ (فِي قُدْسِ الأَقْدَاسِ) إِلَى تَحْتِ جَنَاحَيِ الْكَرُوبَيْنِ،
\par 7 لأَنَّ الْكَرُوبَيْنِ بَسَطَا أَجْنِحَتَهُمَا عَلَى مَوْضِعِ التَّابُوتِ، وَظَلَّلَ الْكَرُوبَانِ التَّابُوتَ وَعِصِيَّهُ مِنْ فَوْقُ.
\par 8 وَجَذَبُوا الْعِصِيَّ فَتَرَاءَتْ رُؤُوسُ الْعِصِيِّ مِنَ الْقُدْسِ أَمَامَ الْمِحْرَابِ وَلَمْ تُرَ خَارِجاً، وَهِيَ هُنَاكَ إِلَى هَذَا الْيَوْمِ.
\par 9 لَمْ يَكُنْ فِي التَّابُوتِ إِلاَّ لَوْحَا الْحَجَرِ اللَّذَانِ وَضَعَهُمَا مُوسَى هُنَاكَ فِي حُورِيبَ حِينَ عَاهَدَ الرَّبُّ بَنِي إِسْرَائِيلَ عِنْدَ خُرُوجِهِمْ مِنْ أَرْضِ مِصْرَ.
\par 10 وَكَانَ لَمَّا خَرَجَ الْكَهَنَةُ مِنَ الْقُدْسِ أَنَّ السَّحَابَ مَلَأَ بَيْتَ الرَّبِّ،
\par 11 وَلَمْ يَسْتَطِعِ الْكَهَنَةُ أَنْ يَقِفُوا لِلْخِدْمَةِ بِسَبَبِ السَّحَابِ، لأَنَّ مَجْدَ الرَّبِّ مَلَأَ بَيْتَ الرَّبِّ.
\par 12 حِينَئِذٍ تَكَلَّمَ سُلَيْمَانُ: [قَالَ الرَّبُّ إِنَّهُ يَسْكُنُ فِي الضَّبَابِ.
\par 13 إِنِّي قَدْ بَنَيْتُ لَكَ بَيْتَ سُكْنَى مَكَاناً لِسُكْنَاكَ إِلَى الأَبَدِ].
\par 14 وَحَوَّلَ الْمَلِكُ وَجْهَهُ وَبَارَكَ كُلَّ جُمْهُورِ إِسْرَائِيلَ، وَكُلُّ جُمْهُورِ إِسْرَائِيلَ وَاقِفٌ.
\par 15 وَقَالَ: [مُبَارَكٌ الرَّبُّ إِلَهُ إِسْرَائِيلَ الَّذِي تَكَلَّمَ بِفَمِهِ إِلَى دَاوُدَ أَبِي وَأَكْمَلَ بِيَدِهِ قَائِلاً:
\par 16 مُنْذُ يَوْمَ أَخْرَجْتُ شَعْبِي إِسْرَائِيلَ مِنْ مِصْرَ لَمْ أَخْتَرْ مَدِينَةً مِنْ جَمِيعِ أَسْبَاطِ إِسْرَائِيلَ لِبِنَاءِ بَيْتٍ لِيَكُونَ اسْمِي هُنَاكَ، بَلِ اخْتَرْتُ دَاوُدَ لِيَكُونَ عَلَى شَعْبِي إِسْرَائِيلَ.
\par 17 وَكَانَ فِي قَلْبِ دَاوُدَ أَبِي أَنْ يَبْنِيَ بَيْتاً لاِسْمِ الرَّبِّ إِلَهِ إِسْرَائِيلَ.
\par 18 فَقَالَ الرَّبُّ لِدَاوُدَ أَبِي: مِنْ أَجْلِ أَنَّهُ كَانَ فِي قَلْبِكَ أَنْ تَبْنِيَ بَيْتاً لاِسْمِي قَدْ أَحْسَنْتَ بِكَوْنِهِ فِي قَلْبِكَ.
\par 19 إِلاَّ إِنَّكَ أَنْتَ لاَ تَبْنِي الْبَيْتَ، بَلِ ابْنُكَ الْخَارِجُ مِنْ صُلْبِكَ هُوَ يَبْنِي الْبَيْتَ لاِسْمِي.
\par 20 وَأَقَامَ الرَّبُّ كَلاَمَهُ الَّذِي تَكَلَّمَ بِهِ، وَقَدْ قُمْتُ أَنَا مَكَانَ دَاوُدَ أَبِي وَجَلَسْتُ عَلَى كُرْسِيِّ إِسْرَائِيلَ كَمَا تَكَلَّمَ الرَّبُّ، وَبَنَيْتُ الْبَيْتَ لاِسْمِ الرَّبِّ إِلَهِ إِسْرَائِيلَ
\par 21 وَجَعَلْتُ هُنَاكَ مَكَاناً لِلتَّابُوتِ الَّذِي فِيهِ عَهْدُ الرَّبِّ الَّذِي قَطَعَهُ مَعَ آبَائِنَا عِنْدَ إِخْرَاجِهِ إِيَّاهُمْ مِنْ أَرْضِ مِصْرَ].
\par 22 وَوَقَفَ سُلَيْمَانُ أَمَامَ مَذْبَحِ الرَّبِّ تُجَاهَ كُلِّ جَمَاعَةِ إِسْرَائِيلَ، وَبَسَطَ يَدَيْهِ إِلَى السَّمَاءِ
\par 23 وَقَالَ: [أَيُّهَا الرَّبُّ إِلَهُ إِسْرَائِيلَ، لَيْسَ إِلَهٌ مِثْلَكَ فِي السَّمَاءِ مِنْ فَوْقُ وَلاَ عَلَى الأَرْضِ مِنْ أَسْفَلُ، حَافِظُ الْعَهْدِ وَالرَّحْمَةِ لِعَبِيدِكَ السَّائِرِينَ أَمَامَكَ بِكُلِّ قُلُوبِهِمْ.
\par 24 الَّذِي قَدْ حَفِظْتَ لِعَبْدِكَ دَاوُدَ أَبِي مَا كَلَّمْتَهُ بِهِ، فَتَكَلَّمْتَ بِفَمِكَ وَأَكْمَلْتَ بِيَدِكَ كَهَذَا الْيَوْمِ.
\par 25 وَالآنَ أَيُّهَا الرَّبُّ إِلَهُ إِسْرَائِيلَ احْفَظْ لِعَبْدِكَ دَاوُدَ أَبِي مَا كَلَّمْتَهُ بِهِ قَائِلاً: لاَ يُعْدَمُ لَكَ أَمَامِي رَجُلٌ يَجْلِسُ عَلَى كُرْسِيِّ إِسْرَائِيلَ، إِنْ كَانَ بَنُوكَ يَحْفَظُونَ طُرُقَهُمْ حَتَّى يَسِيرُوا أَمَامِي كَمَا سِرْتَ أَنْتَ أَمَامِي.
\par 26 وَالآنَ يَا إِلَهَ إِسْرَائِيلَ فَلْيَتَحَقَّقْ كَلاَمُكَ الَّذِي كَلَّمْتَ بِهِ عَبْدَكَ دَاوُدَ أَبِي.
\par 27 لأَنَّهُ هَلْ يَسْكُنُ اللَّهُ حَقّاً عَلَى الأَرْضِ؟ هُوَذَا السَّمَاوَاتُ وَسَمَاءُ السَّمَاوَاتِ لاَ تَسَعُكَ، فَكَمْ بِالأَقَلِّ هَذَا الْبَيْتُ الَّذِي بَنَيْتُ؟
\par 28 فَالْتَفِتْ إِلَى صَلاَةِ عَبْدِكَ وَإِلَى تَضَرُّعِهِ أَيُّهَا الرَّبُّ إِلَهِي، وَاسْمَعِ الصُّرَاخَ وَالصَّلاَةَ الَّتِي يُصَلِّيهَا عَبْدُكَ أَمَامَكَ الْيَوْمَ.
\par 29 لِتَكُونَ عَيْنَاكَ مَفْتُوحَتَيْنِ عَلَى هَذَا الْبَيْتِ لَيْلاً وَنَهَاراً، عَلَى الْمَوْضِعِ الَّذِي قُلْتَ: إِنَّ اسْمِي يَكُونُ فِيهِ لِتَسْمَعَ الصَّلاَةَ الَّتِي يُصَلِّيهَا عَبْدُكَ فِي هَذَا الْمَوْضِعِ.
\par 30 وَاسْمَعْ تَضَرُّعَ عَبْدِكَ وَشَعْبِكَ إِسْرَائِيلَ الَّذِينَ يُصَلُّونَ فِي هَذَا الْمَوْضِعِ، وَاسْمَعْ أَنْتَ فِي مَوْضِعِ سُكْنَاكَ فِي السَّمَاءِ، وَإِذَا سَمِعْتَ فَاغْفِرْ.
\par 31 إِذَا أَخْطَأَ أَحَدٌ إِلَى صَاحِبِهِ وَوَضَعَ عَلَيْهِ حَلْفاً لِيُحَلِّفَهُ، وَجَاءَ الْحَلْفُ أَمَامَ مَذْبَحِكَ فِي هَذَا الْبَيْتِ،
\par 32 فَاسْمَعْ أَنْتَ فِي السَّمَاءِ وَاعْمَلْ وَاقْضِ بَيْنَ عَبِيدِكَ، إِذْ تَحْكُمُ عَلَى الْمُذْنِبِ فَتَجْعَلُ طَرِيقَهُ عَلَى رَأْسِهِ، وَتُبَرِّرُ الْبَارَّ إِذْ تُعْطِيهِ حَسَبَ بِرِّهِ.
\par 33 إِذَا انْكَسَرَ شَعْبُكَ إِسْرَائِيلُ أَمَامَ الْعَدُوِّ لأَنَّهُمْ أَخْطَأُوا إِلَيْكَ، ثُمَّ رَجَعُوا إِلَيْكَ وَاعْتَرَفُوا بِاسْمِكَ وَصَلُّوا وَتَضَرَّعُوا إِلَيْكَ نَحْوَ هَذَا الْبَيْتِ،
\par 34 فَاسْمَعْ أَنْتَ مِنَ السَّمَاءِ وَاغْفِرْ خَطِيَّةَ شَعْبِكَ إِسْرَائِيلَ، وَأَرْجِعْهُمْ إِلَى الأَرْضِ الَّتِي أَعْطَيْتَهَا لِآبَائِهِمْ.
\par 35 [إِذَا أُغْلِقَتِ السَّمَاءُ وَلَمْ يَكُنْ مَطَرٌ لأَنَّهُمْ أَخْطَأُوا إِلَيْكَ، ثُمَّ صَلُّوا فِي هَذَا الْمَوْضِعِ وَاعْتَرَفُوا بِاسْمِكَ، وَرَجَعُوا عَنْ خَطِيَّتِهِمْ لأَنَّكَ ضَايَقْتَهُمْ
\par 36 فَاسْمَعْ أَنْتَ مِنَ السَّمَاءِ وَاغْفِرْ خَطِيَّةَ عَبِيدِكَ وَشَعْبِكَ إِسْرَائِيلَ، فَتُعَلِّمَهُمُ الطَّرِيقَ الصَّالِحَ الَّذِي يَسْلُكُونَ فِيهِ، وَأَعْطِ مَطَراً عَلَى أَرْضِكَ الَّتِي أَعْطَيْتَهَا لِشَعْبِكَ مِيرَاثاً.
\par 37 إِذَا صَارَ فِي الأَرْضِ جُوعٌ، إِذَا صَارَ وَبَأٌ، إِذَا صَارَ لَفْحٌ أَوْ يَرَقَانٌ أَوْ جَرَادٌ جَرْدَمٌ، أَوْ إِذَا حَاصَرَهُ عَدُوُّهُ فِي أَرْضِ مُدُنِهِ فِي كُلِّ ضَرْبَةٍ وَكُلِّ مَرَضٍ،
\par 38 فَكُلُّ صَلاَةٍ وَكُلُّ تَضَرُّعٍ تَكُونُ مِنْ أَيِّ إِنْسَانٍ كَانَ مِنْ كُلِّ شَعْبِكَ إِسْرَائِيلَ، الَّذِينَ يَعْرِفُونَ كُلُّ وَاحِدٍ ضَرْبَةَ قَلْبِهِ فَيَبْسُطُ يَدَيْهِ نَحْوَ هَذَا الْبَيْتِ،
\par 39 فَاسْمَعْ أَنْتَ مِنَ السَّمَاءِ مَكَانِ سُكْنَاكَ وَاغْفِرْ، وَاعْمَلْ وَأَعْطِ كُلَّ إِنْسَانٍ حَسَبَ كُلِّ طُرُقِهِ كَمَا تَعْرِفُ قَلْبَهُ. لأَنَّكَ أَنْتَ وَحْدَكَ قَدْ عَرَفْتَ قُلُوبَ كُلِّ بَنِي الْبَشَرِ.
\par 40 لِيَخَافُوكَ كُلَّ الأَيَّامِ الَّتِي يَحْيُونَ فِيهَا عَلَى وَجْهِ الأَرْضِ الَّتِي أَعْطَيْتَ لِآبَائِنَا.
\par 41 وَكَذَلِكَ الأَجْنَبِيُّ الَّذِي لَيْسَ مِنْ شَعْبِكَ إِسْرَائِيلَ، وَجَاءَ مِنْ أَرْضٍ بَعِيدَةٍ مِنْ أَجْلِ اسْمِكَ -
\par 42 لأَنَّهُمْ يَسْمَعُونَ بِاسْمِكَ الْعَظِيمِ وَبِيَدِكَ الْقَوِيَّةِ وَذِرَاعِكَ الْمَمْدُودَةِ - فَمَتَى جَاءَ وَصَلَّى فِي هَذَا الْبَيْتِ،
\par 43 فَاسْمَعْ أَنْتَ مِنَ السَّمَاءِ مَكَانِ سُكْنَاكَ، وَافْعَلْ حَسَبَ كُلِّ مَا يَدْعُو بِهِ إِلَيْكَ الأَجْنَبِيُّ، لِيَعْلَمَ كُلُّ شُعُوبِ الأَرْضِ اسْمَكَ فَيَخَافُوكَ كَشَعْبِكَ إِسْرَائِيلَ، وَلِيَعْلَمُوا أَنَّهُ قَدْ دُعِيَ اسْمُكَ عَلَى هَذَا الْبَيْتِ الَّذِي بَنَيْتُ.
\par 44 [إِذَا خَرَجَ شَعْبُكَ لِمُحَارَبَةِ عَدُوِّهِ فِي الطَّرِيقِ الَّذِي تُرْسِلُهُمْ فِيهِ، وَصَلُّوا إِلَى الرَّبِّ نَحْوَ الْمَدِينَةِ الَّتِي اخْتَرْتَهَا وَالْبَيْتِ الَّذِي بَنَيْتُهُ لاِسْمِكَ،
\par 45 فَاسْمَعْ مِنَ السَّمَاءِ صَلاَتَهُمْ وَتَضَرُّعَهُمْ وَاقْضِ قَضَاءَهُمْ.
\par 46 إِذَا أَخْطَأُوا إِلَيْكَ - لأَنَّهُ لَيْسَ إِنْسَانٌ لاَ يُخْطِئُ - وَغَضِبْتَ عَلَيْهِمْ وَدَفَعْتَهُمْ أَمَامَ الْعَدُوِّ وَسَبَاهُمْ سَابُوهُمْ إِلَى أَرْضِ الْعَدُوِّ بَعِيدَةً أَوْ قَرِيبَةً،
\par 47 فَإِذَا رَدُّوا إِلَى قُلُوبِهِمْ فِي الأَرْضِ الَّتِي يُسْبَوْنَ إِلَيْهَا وَرَجَعُوا وَتَضَرَّعُوا إِلَيْكَ فِي أَرْضِ سَبْيِهِمْ قَائِلِينَ: قَدْ أَخْطَأْنَا وَعَوَّجْنَا وَأَذْنَبْنَا
\par 48 وَرَجَعُوا إِلَيْكَ مِنْ كُلِّ قُلُوبِهِمْ وَمِنْ كُلِّ أَنْفُسِهِمْ فِي أَرْضِ أَعْدَائِهِمُِ الَّذِينَ سَبُوهُمْ، وَصَلُّوا إِلَيْكَ نَحْوَ أَرْضِهِمُِ الَّتِي أَعْطَيْتَ لِآبَائِهِمْ، نَحْوَ الْمَدِينَةِ الَّتِي اخْتَرْتَ وَالْبَيْتِ الَّذِي بَنَيْتُ لاِسْمِكَ،
\par 49 فَاسْمَعْ فِي السَّمَاءِ مَكَانِ سُكْنَاكَ صَلاَتَهُمْ وَتَضَرُّعَهُمْ وَاقْضِ قَضَاءَهُمْ،
\par 50 وَاغْفِرْ لِشَعْبِكَ مَا أَخْطَأُوا بِهِ إِلَيْكَ، وَجَمِيعَ ذُنُوبِهِمِ الَّتِي أَذْنَبُوا بِهَا إِلَيْكَ، وَأَعْطِهِمْ رَحْمَةً أَمَامَ الَّذِينَ سَبُوهُمْ فَيَرْحَمُوهُمْ،
\par 51 لأَنَّهُمْ شَعْبُكَ وَمِيرَاثُكَ الَّذِينَ أَخْرَجْتَ مِنْ مِصْرَ مِنْ وَسَطِ كُورِ الْحَدِيدِ.
\par 52 لِتَكُونَ عَيْنَاكَ مَفْتُوحَتَيْنِ نَحْوَ تَضَرُّعِ عَبْدِكَ وَتَضَرُّعِ شَعْبِكَ إِسْرَائِيلَ، فَتُصْغِيَ إِلَيْهِمْ فِي كُلِّ مَا يَدْعُونَكَ.
\par 53 لأَنَّكَ أَنْتَ أَفْرَزْتَهُمْ لَكَ مِيرَاثاً مِنْ جَمِيعِ شُعُوبِ الأَرْضِ، كَمَا تَكَلَّمْتَ عَنْ يَدِ مُوسَى عَبْدِكَ عِنْدَ إِخْرَاجِكَ آبَاءَنَا مِنْ مِصْرَ يَا سَيِّدِي الرَّبَّ].
\par 54 وَكَانَ لَمَّا انْتَهَى سُلَيْمَانُ مِنَ الصَّلاَةِ إِلَى الرَّبِّ بِكُلِّ هَذِهِ الصَّلاَةِ وَالتَّضَرُّعِ، أَنَّهُ نَهَضَ مِنْ أَمَامِ مَذْبَحِ الرَّبِّ مِنَ الْجُثُوِّ عَلَى رُكْبَتَيْهِ وَيَدَاهُ مَبْسُوطَتَانِ نَحْوَ السَّمَاءِ،
\par 55 وَوَقَفَ وَبَارَكَ كُلَّ جَمَاعَةِ إِسْرَائِيلَ بِصَوْتٍ عَالٍ قَائِلاً:
\par 56 [مُبَارَكٌ الرَّبُّ الَّذِي أَعْطَى رَاحَةً لِشَعْبِهِ إِسْرَائِيلَ حَسَبَ كُلِّ مَا تَكَلَّمَ بِهِ، وَلَمْ تَسْقُطْ كَلِمَةٌ وَاحِدَةٌ مِنْ كُلِّ كَلاَمِهِ الصَّالِحِ الَّذِي تَكَلَّمَ بِهِ عَنْ يَدِ مُوسَى عَبْدِهِ.
\par 57 لِيَكُنِ الرَّبُّ إِلَهُنَا مَعَنَا كَمَا كَانَ مَعَ آبَائِنَا فَلاَ يَتْرُكَنَا وَلاَ يَرْفُضَنَا.
\par 58 لِيَمِيلَ بِقُلُوبِنَا إِلَيْهِ لِنَسِيرَ فِي جَمِيعِ طُرُقِهِ وَنَحْفَظَ وَصَايَاهُ وَفَرَائِضَهُ وَأَحْكَامَهُ الَّتِي أَوْصَى بِهَا آبَاءَنَا.
\par 59 وَلِيَكُنْ كَلاَمِي هَذَا الَّذِي تَضَرَّعْتُ بِهِ أَمَامَ الرَّبِّ قَرِيباً مِنَ الرَّبِّ إِلَهِنَا نَهَاراً وَلَيْلاً، لِيَقْضِيَ قَضَاءَ عَبْدِهِ وَقَضَاءَ شَعْبِهِ إِسْرَائِيلَ، أَمْرَ كُلِّ يَوْمٍ فِي يَوْمِهِ.
\par 60 لِيَعْلَمَ كُلُّ شُعُوبِ الأَرْضِ أَنَّ الرَّبَّ هُوَ اللَّهُ وَلَيْسَ آخَرُ.
\par 61 فَلِْيَكُنْ قَلْبُكُمْ كَامِلاً لَدَى الرَّبِّ إِلَهِنَا إِذْ تَسِيرُونَ فِي فَرَائِضِهِ وَتَحْفَظُونَ وَصَايَاهُ كَهَذَا الْيَوْمِ].
\par 62 ثُمَّ إِنَّ الْمَلِكَ وَجَمِيعَ إِسْرَائِيلَ مَعَهُ ذَبَحُوا ذَبَائِحَ أَمَامَ الرَّبِّ،
\par 63 وَذَبَحَ سُلَيْمَانُ ذَبَائِحَ السَّلاَمَةِ الَّتِي ذَبَحَهَا لِلرَّبِّ مِنَ الْبَقَرِ اثْنَيْنِ وَعِشْرِينَ أَلْفاً، وَمِنَ الْغَنَمِ مِئَةَ أَلْفٍ وَعِشْرِينَ أَلْفاً، فَدَشَّنَ الْمَلِكُ وَجَمِيعُ بَنِي إِسْرَائِيلَ بَيْتَ الرَّبِّ.
\par 64 فِي ذَلِكَ الْيَوْمِ قَدَّسَ الْمَلِكُ وَسَطَ الدَّارِ الَّتِي أَمَامَ بَيْتِ الرَّبِّ لأَنَّهُ قَرَّبَ هُنَاكَ الْمُحْرَقَاتِ وَالتَّقْدِمَاتِ وَشَحْمَ ذَبَائِحِ السَّلاَمَةِ، لأَنَّ مَذْبَحَ النُّحَاسِ الَّذِي أَمَامَ الرَّبِّ كَانَ صَغِيراً عَنْ أَنْ يَسَعَ الْمُحْرَقَاتِ وَالتَّقْدِمَاتِ وَشَحْمَ ذَبَائِحِ السَّلاَمَةِ.
\par 65 وَعَيَّدَ سُلَيْمَانُ الْعِيدَ فِي ذَلِكَ الْوَقْتِ وَجَمِيعُ إِسْرَائِيلَ مَعَهُ، جُمْهُورٌ كَبِيرٌ مِنْ مَدْخَلِ حَمَاةَ إِلَى وَادِي مِصْرَ أَمَامَ الرَّبِّ إِلَهِنَا سَبْعَةَ أَيَّامٍ وَسَبْعَةَ أَيَّامٍ، أَرْبَعَةَ عَشَرَ يَوْماً.
\par 66 وَفِي الْيَوْمِ الثَّامِنِ صَرَفَ الشَّعْبَ فَبَارَكُوا الْمَلِكَ وَذَهَبُوا إِلَى خِيَمِهِمْ فَرِحِينَ وَطَيِّبِي الْقُلُوبِ لأَجْلِ كُلِّ الْخَيْرِ الَّذِي عَمِلَ الرَّبُّ لِدَاوُدَ عَبْدِهِ وَلإِسْرَائِيلَ شَعْبِهِ.

\chapter{9}

\par 1 وَكَانَ لَمَّا أَكْمَلَ سُلَيْمَانُ بِنَاءَ بَيْتِ الرَّبِّ وَبَيْتِ الْمَلِكِ وَكُلَّ مَرْغُوبِ سُلَيْمَانَ الَّذِي سُرَّ أَنْ يَعْمَلَ،
\par 2 أَنَّ الرَّبَّ تَرَاءَى لِسُلَيْمَانَ ثَانِيَةً كَمَا تَرَاءَى لَهُ فِي جِبْعُونَ.
\par 3 وَقَالَ لَهُ الرَّبُّ: [قَدْ سَمِعْتُ صَلاَتَكَ وَتَضَرُّعَكَ الَّذِي تَضَرَّعْتَ بِهِ أَمَامِي. قَدَّسْتُ هَذَا الْبَيْتَ الَّذِي بَنَيْتَهُ لأَجْلِ وَضْعِ اسْمِي فِيهِ إِلَى الأَبَدِ، وَتَكُونُ عَيْنَايَ وَقَلْبِي هُنَاكَ كُلَّ الأَيَّامِ.
\par 4 وَأَنْتَ إِنْ سَلَكْتَ أَمَامِي كَمَا سَلَكَ دَاوُدُ أَبُوكَ بِسَلاَمَةِ قَلْبٍ وَاسْتِقَامَةٍ، وَعَمِلْتَ حَسَبَ كُلِّ مَا أَوْصَيْتُكَ وَحَفِظْتَ فَرَائِضِي وَأَحْكَامِي،
\par 5 فَإِنِّي أُقِيمُ كُرْسِيَّ مُلْكِكَ عَلَى إِسْرَائِيلَ إِلَى الأَبَدِ كَمَا قُلْتُ لِدَاوُدَ أَبِيكَ: لاَ يُعْدَمُ لَكَ رَجُلٌ عَنْ كُرْسِيِّ إِسْرَائِيلَ.
\par 6 إِنْ كُنْتُمْ تَنْقَلِبُونَ أَنْتُمْ أَوْ أَبْنَاؤُكُمْ مِنْ وَرَائِي، وَلاَ تَحْفَظُونَ وَصَايَايَ فَرَائِضِيَ الَّتِي جَعَلْتُهَا أَمَامَكُمْ، بَلْ تَذْهَبُونَ وَتَعْبُدُونَ آلِهَةً أُخْرَى وَتَسْجُدُونَ لَهَا،
\par 7 فَإِنِّي أَقْطَعُ إِسْرَائِيلَ عَنْ وَجْهِ الأَرْضِ الَّتِي أَعْطَيْتُهُمْ إِيَّاهَا، وَالْبَيْتُ الَّذِي قَدَّسْتُهُ لاِسْمِي أَنْفِيهِ مِنْ أَمَامِي، وَيَكُونُ إِسْرَائِيلُ مَثَلاً وَهُزْأَةً فِي جَمِيعِ الشُّعُوبِ،
\par 8 وَهَذَا الْبَيْتُ يَكُونُ عِبْرَةً. كُلُّ مَنْ يَمُرُّ عَلَيْهِ يَتَعَجَّبُ وَيَصْفُرُ، وَيَقُولُونَ: لِمَاذَا عَمِلَ الرَّبُّ هَكَذَا لِهَذِهِ الأَرْضِ وَلِهَذَا الْبَيْتِ؟
\par 9 فَيَقُولُونَ: مِنْ أَجْلِ أَنَّهُمْ تَرَكُوا الرَّبَّ إِلَهَهُمُ الَّذِي أَخْرَجَ آبَاءَهُمْ مِنْ أَرْضِ مِصْرَ، وَتَمَسَّكُوا بِآلِهَةٍ أُخْرَى وَسَجَدُوا لَهَا وَعَبَدُوهَا. لِذَلِكَ جَلَبَ الرَّبُّ عَلَيْهِمْ كُلَّ هَذَا الشَّرِّ].
\par 10 وَبَعْدَ نِهَايَةِ عِشْرِينَ سَنَةً بَعْدَمَا بَنَى سُلَيْمَانُ الْبَيْتَيْنِ، بَيْتَ الرَّبِّ وَبَيْتَ الْمَلِكِ.
\par 11 وَكَانَ حِيرَامُ مَلِكُ صُورَ قَدْ سَاعَدَ سُلَيْمَانَ بِخَشَبِ أَرْزٍ وَخَشَبِ سَرْوٍ، وَذَهَبٍ، حَسَبَ كُلِّ مَسَرَّتِهِ. أَعْطَى حِينَئِذٍ الْمَلِكُ سُلَيْمَانُ حِيرَامَ عِشْرِينَ مَدِينَةً فِي أَرْضِ الْجَلِيلِ.
\par 12 فَخَرَجَ حِيرَامُ مِنْ صُورَ لِيَرَى الْمُدُنَ الَّتِي أَعْطَاهُ إِيَّاهَا سُلَيْمَانُ، فَلَمْ تَحْسُنْ فِي عَيْنَيْهِ.
\par 13 فَقَالَ: [مَا هَذِهِ الْمُدُنُ الَّتِي أَعْطَيْتَنِي يَا أَخِي؟] وَدَعَاهَا [أَرْضَ كَابُولَ] إِلَى هَذَا الْيَوْمِ.
\par 14 وَأَرْسَلَ حِيرَامُ لِلْمَلِكِ مِئَةً وَعِشْرِينَ وَزْنَةَ ذَهَبٍ.
\par 15 وَهَذَا هُوَ سَبَبُ التَّسْخِيرِ الَّذِي جَعَلَهُ الْمَلِكُ سُلَيْمَانُ لِبِنَاءِ بَيْتِ الرَّبِّ وَبَيْتِهِ وَالْقَلْعَةِ وَسُورِ أُورُشَلِيمَ وَحَاصُورَ وَمَجِدُّو وَجَازَرَ.
\par 16 (صَعِدَ فِرْعَوْنُ مَلِكُ مِصْرَ وَأَخَذَ جَازَرَ وَأَحْرَقَهَا بِالنَّارِ، وَقَتَلَ الْكَنْعَانِيِّينَ السَّاكِنِينَ فِي الْمَدِينَةِ، وَأَعْطَاهَا مَهْراً لاِبْنَتِهِ امْرَأَةِ سُلَيْمَانَ).
\par 17 وَبَنَى سُلَيْمَانُ جَازَرَ وَبَيْتَ حُورُونَ السُّفْلَى
\par 18 وَبَعْلَةَ وَتَدْمُرَ فِي الْبَرِّيَّةِ فِي الأَرْضِ،
\par 19 وَجَمِيعَ مُدُنِ الْمَخَازِنِ الَّتِي كَانَتْ لِسُلَيْمَانَ، وَمُدُنَ الْمَرْكَبَاتِ وَمُدُنَ الْفُرْسَانِ، وَمَرْغُوبَ سُلَيْمَانَ الَّذِي رَغِبَ أَنْ يَبْنِيَهُ فِي أُورُشَلِيمَ وَفِي لُبْنَانَ وَفِي كُلِّ أَرْضِ سَلْطَنَتِهِ.
\par 20 جَمِيعُ الشَّعْبِ الْبَاقِينَ مِنَ الأَمُورِيِّينَ وَالْحِثِّيِّينَ وَالْفِرِزِّيِّينَ وَالْحِوِّيِّينَ وَالْيَبُوسِيِّينَ الَّذِينَ لَيْسُوا مِنْ بَنِي إِسْرَائِيلَ،
\par 21 أَبْنَاؤُهُمُ الَّذِينَ بَقُوا بَعْدَهُمْ فِي الأَرْضِ، الَّذِينَ لَمْ يَقْدِرْ بَنُو إِسْرَائِيلَ أَنْ يُحَرِّمُوهُمْ، جَعَلَ عَلَيْهِمْ سُلَيْمَانُ تَسْخِيرَ عَبِيدٍ إِلَى هَذَا الْيَوْمِ.
\par 22 وَأَمَّا بَنُو إِسْرَائِيلَ فَلَمْ يَجْعَلْ سُلَيْمَانُ مِنْهُمْ عَبِيداً لأَنَّهُمْ رِجَالُ الْقِتَالِ وَخُدَّامُهُ وَأُمَرَاؤُهُ وَثَوَالِثُهُ وَرُؤَسَاءُ مَرْكَبَاتِهِ وَفُرْسَانُهُ.
\par 23 هَؤُلاَءِ رُؤَسَاءُ الْمُوَكَّلِينَ عَلَى أَعْمَالِ سُلَيْمَانَ خَمْسُ مِئَةٍ وَخَمْسُونَ، الَّذِينَ كَانُوا يَتَسَلَّطُونَ عَلَى الشَّعْبِ الْعَامِلِينَ الْعَمَلَ.
\par 24 وَلَكِنَّ بِنْتَ فِرْعَوْنَ صَعِدَتْ مِنْ مَدِينَةِ دَاوُدَ إِلَى بَيْتِهَا الَّذِي بَنَاهُ لَهَا. حِينَئِذٍ بَنَى الْقَلْعَةَ.
\par 25 وَكَانَ سُلَيْمَانُ يُصْعِدُ ثَلاَثَ مَرَّاتٍ فِي السَّنَةِ مُحْرَقَاتٍ وَذَبَائِحَ سَلاَمَةٍ عَلَى الْمَذْبَحِ الَّذِي بَنَاهُ لِلرَّبِّ، وَكَانَ يُوقِدُ عَلَى الَّذِي أَمَامَ الرَّبِّ. وَأَكْمَلَ الْبَيْتَ.
\par 26 وَعَمِلَ الْمَلِكُ سُلَيْمَانُ سُفُناً فِي عِصْيُونَ جَابِرَ الَّتِي بِجَانِبِ أَيْلَةَ عَلَى شَاطِئِ بَحْرِ سُوفٍ فِي أَرْضِ أَدُومَ.
\par 27 فَأَرْسَلَ حِيرَامُ فِي السُّفُنِ عَبِيدَهُ النَّوَاتِيَّ الْعَارِفِينَ بِالْبَحْرِ مَعَ عَبِيدِ سُلَيْمَانَ،
\par 28 فَأَتُوا إِلَى أُوفِيرَ، وَأَخَذُوا مِنْ هُنَاكَ ذَهَباً أَرْبَعَ مِئَةِ وَزْنَةٍ وَعِشْرِينَ وَزْنَةً، وَأَتُوا بِهَا إِلَى الْمَلِكِ سُلَيْمَانَ.

\chapter{10}

\par 1 وَسَمِعَتْ مَلِكَةُ سَبَا بِخَبَرِ سُلَيْمَانَ لِمَجْدِ الرَّبِّ، فَأَتَتْ لِتَمْتَحِنَهُ بِمَسَائِلَ.
\par 2 فَأَتَتْ إِلَى أُورُشَلِيمَ بِمَوْكِبٍ عَظِيمٍ جِدّاً، بِجِمَالٍ حَامِلَةٍ أَطْيَاباً وَذَهَباً كَثِيراً جِدّاً وَحِجَارَةً كَرِيمَةً. وَأَتَتْ إِلَى سُلَيْمَانَ وَكَلَّمَتْهُ بِكُلِّ مَا كَانَ بِقَلْبِهَا.
\par 3 فَأَخْبَرَهَا سُلَيْمَانُ بِكُلِّ كَلاَمِهَا. لَمْ يَكُنْ أَمْرٌ مَخْفِيّاً عَنِ الْمَلِكِ لَمْ يُخْبِرْهَا بِهِ.
\par 4 فَلَمَّا رَأَتْ مَلِكَةُ سَبَا كُلَّ حِكْمَةِ سُلَيْمَانَ وَالْبَيْتَ الَّذِي بَنَاهُ
\par 5 وَطَعَامَ مَائِدَتِهِ وَمَجْلِسَ عَبِيدِهِ وَمَوْقِفَ خُدَّامِهِ وَمَلاَبِسَهُمْ وَسُقَاتَهُ وَمُحْرَقَاتِهِ الَّتِي كَانَ يُصْعِدُهَا فِي بَيْتِ الرَّبِّ، لَمْ يَبْقَ فِيهَا رُوحٌ بَعْدُ.
\par 6 فَقَالَتْ لِلْمَلِكِ: [صَحِيحاً كَانَ الْخَبَرُ الَّذِي سَمِعْتُهُ فِي أَرْضِي عَنْ أُمُورِكَ وَعَنْ حِكْمَتِكَ.
\par 7 وَلَمْ أُصَدِّقِ الأَخْبَارَ حَتَّى جِئْتُ وَأَبْصَرَتْ عَيْنَايَ، فَهُوَذَا النِّصْفُ لَمْ أُخْبَرْ بِهِ. زِدْتَ حِكْمَةً وَصَلاَحاً عَلَى الْخَبَرِ الَّذِي سَمِعْتُهُ.
\par 8 طُوبَى لِرِجَالِكَ وَطُوبَى لِعَبِيدِكَ هَؤُلاَءِ الْوَاقِفِينَ أَمَامَكَ دَائِماً السَّامِعِينَ حِكْمَتَكَ.
\par 9 لِيَكُنْ مُبَارَكاً الرَّبُّ إِلَهُكَ الَّذِي سُرَّ بِكَ وَجَعَلَكَ عَلَى كُرْسِيِّ إِسْرَائِيلَ. لأَنَّ الرَّبَّ أَحَبَّ إِسْرَائِيلَ إِلَى الأَبَدِ جَعَلَكَ مَلِكاً، لِتُجْرِيَ حُكْماً وَبِرّاً].
\par 10 وَأَعْطَتِ الْمَلِكَ مِئَةً وَعِشْرِينَ وَزْنَةَ ذَهَبٍ وَأَطْيَاباً كَثِيرَةً جِدّاً وَحِجَارَةً كَرِيمَةً. لَمْ يَأْتِ بَعْدُ مِثْلُ ذَلِكَ الطِّيبِ فِي الْكَثْرَةِ الَّتِي أَعْطَتْهُ مَلِكَةُ سَبَا لِلْمَلِكِ سُلَيْمَانَ.
\par 11 وَكَذَا سُفُنُ حِيرَامَ الَّتِي حَمَلَتْ ذَهَباً مِنْ أُوفِيرَ أَتَتْ مِنْ أُوفِيرَ بِخَشَبِ الصَّنْدَلِ كَثِيراً جِدّاً وَبِحِجَارَةٍ كَرِيمَةٍ.
\par 12 فَعَمِلَ سُلَيْمَانُ خَشَبَ الصَّنْدَلِ دَرَابَزِيناً لِبَيْتِ الرَّبِّ وَبَيْتِ الْمَلِكِ، وَأَعْوَاداً وَرَبَاباً لِلْمُغَنِّينَ. لَمْ يَأْتِ وَلَمْ يُرَ مِثْلُ خَشَبِ الصَّنْدَلِ ذَلِكَ إِلَى هَذَا الْيَوْمِ.
\par 13 وَأَعْطَى الْمَلِكُ سُلَيْمَانُ لِمَلِكَةِ سَبَا كُلَّ مُشْتَهَاهَا الَّذِي طَلَبَتْ، عَدَا مَا أَعْطَاهَا إِيَّاهُ حَسَبَ كَرَمِ الْمَلِكِ سُلَيْمَانَ. فَانْصَرَفَتْ وَذَهَبَتْ إِلَى أَرْضِهَا هِيَ وَعَبِيدُهَا.
\par 14 وَكَانَ وَزْنُ الذَّهَبِ الَّذِي أَتَى سُلَيْمَانَ فِي سَنَةٍ وَاحِدَةٍ سِتَّ مِئَةٍ وَسِتّاً وَسِتِّينَ وَزْنَةَ ذَهَبٍ.
\par 15 مَا عَدَا الَّذِي مِنْ عِنْدِ التُّجَّارِ وَتَجَارَةِ التُّجَّارِ وَجَمِيعِ مُلُوكِ الْعَرَبِ وَوُلاَةِ الأَرْضِ.
\par 16 وَعَمِلَ الْمَلِكُ سُلَيْمَانُ مِئَتَيْ تُرْسٍ مِنْ ذَهَبٍ مُطَرَّقٍ، خَصَّ التُّرْسَ الْوَاحِدَ سِتُّ مِئَةِ شَاقِلٍ مِنَ الذَّهَبِ.
\par 17 وَثَلاَثَ مِئَةِ مِجَنٍّ مِنْ ذَهَبٍ مُطَرَّقٍ. خَصَّ الْمِجَنَّ ثَلاَثَةُ أَمْنَاءٍ مِنَ الذَّهَبِ. وَجَعَلَهَا سُلَيْمَانُ فِي بَيْتِ وَعْرِ لُبْنَانَ.
\par 18 وَعَمِلَ الْمَلِكُ كُرْسِيّاً عَظِيماً مِنْ عَاجٍ وَغَشَّاهُ بِذَهَبٍ إِبْرِيزٍ.
\par 19 وَلِلْكُرْسِيِّ سِتُّ دَرَجَاتٍ. وَلِلْكُرْسِيِّ رَأْسٌ مُسْتَدِيرٌ مِنْ وَرَائِهِ، وَيَدَانِ مِنْ هُنَا وَمِنْ هُنَاكَ عَلَى مَكَانِ الْجُلُوسِ، وَأَسَدَانِ وَاقِفَانِ بِجَانِبِ الْيَدَيْنِ.
\par 20 وَاثْنَا عَشَرَ أَسَداً وَاقِفَةً هُنَاكَ عَلَى الدَّرَجَاتِ السِّتِّ مِنْ هُنَا وَمِنْ هُنَاكَ. لَمْ يُعْمَلْ مِثْلُهُ فِي جَمِيعِ الْمَمَالِكِ.
\par 21 وَجَمِيعُ آنِيَةِ شُرْبِ الْمَلِكِ سُلَيْمَانَ مِنْ ذَهَبٍ، وَجَمِيعُ آنِيَةِ بَيْتِ وَعْرِ لُبْنَانَ مِنْ ذَهَبٍ خَالِصٍ. لاَ فِضَّةَ. هِيَ لَمْ تُحْسَبْ شَيْئاً فِي أَيَّامِ سُلَيْمَانَ.
\par 22 لأَنَّهُ كَانَ لِلْمَلِكِ فِي الْبَحْرِ سُفُنُ تَرْشِيشَ مَعَ سُفُنِ حِيرَامَ. فَكَانَتْ سُفُنُ تَرْشِيشَ تَأْتِي مَرَّةً فِي كُلِّ ثَلاَثِ سَنَوَاتٍ. أَتَتْ سُفُنُ تَرْشِيشَ حَامِلَةً ذَهَباً وَفِضَّةً وَعَاجاً وَقُرُوداً وَطَوَاوِيسَ.
\par 23 فَتَعَاظَمَ الْمَلِكُ سُلَيْمَانُ عَلَى كُلِّ مُلُوكِ الأَرْضِ فِي الْغِنَى وَالْحِكْمَةِ.
\par 24 وَكَانَتْ كُلُّ الأَرْضِ مُلْتَمِسَةً وَجْهَ سُلَيْمَانَ لِتَسْمَعَ حِكْمَتَهُ الَّتِي جَعَلَهَا اللَّهُ فِي قَلْبِهِ.
\par 25 وَكَانُوا يَأْتُونَ كُلُّ وَاحِدٍ بِهَدِيَّتِهِ، بِآنِيَةِ فِضَّةٍ وَآنِيَةِ ذَهَبٍ وَحُلَلٍ وَسِلاَحٍ وَأَطْيَابٍ وَخَيْلٍ وَبِغَالٍ سَنَةً فَسَنَةً.
\par 26 وَجَمَعَ سُلَيْمَانُ مَرَاكِبَ وَفُرْسَاناً. فَكَانَ لَهُ أَلْفٌ وَأَرْبَعُ مِئَةِ مَرْكَبَةٍ وَاثْنَا عَشَرَ أَلْفَ فَارِسٍ، فَأَقَامَهُمْ فِي مُدُنِ الْمَرَاكِبِ وَمَعَ الْمَلِكِ فِي أُورُشَلِيمَ.
\par 27 وَجَعَلَ الْمَلِكُ الْفِضَّةَ فِي أُورُشَلِيمَ مِثْلَ الْحِجَارَةِ، وَجَعَلَ الأَرْزَ مِثْلَ الْجُمَّيْزِ الَّذِي فِي السَّهْلِ فِي الْكَثْرَةِ.
\par 28 وَكَانَ مَخْرَجُ الْخَيْلِ الَّتِي لِسُلَيْمَانَ مِنْ مِصْرَ. وَجَمَاعَةُ تُجَّارِ الْمَلِكِ أَخَذُوا جَلِيبَةً بِثَمَنٍ.
\par 29 وَكَانَتِ الْمَرْكَبَةُ تَصْعَدُ وَتَخْرُجُ مِنْ مِصْرَ بِسِتِّ مِئَةِ شَاقِلٍ مِنَ الْفِضَّةِ وَالْفَرَسُ بِمِئَةٍ وَخَمْسِينَ. وَهَكَذَا لِجَمِيعِ مُلُوكِ الْحِثِّيِّينَ وَمُلُوكِ أَرَامَ كَانُوا يُخْرِجُونَ عَنْ يَدِهِمْ.

\chapter{11}

\par 1 وَأَحَبَّ الْمَلِكُ سُلَيْمَانُ نِسَاءً غَرِيبَةً كَثِيرَةً مَعَ بِنْتِ فِرْعَوْنَ: مُوآبِيَّاتٍ وَعَمُّونِيَّاتٍ وَأَدُومِيَّاتٍ وَصَيْدُونِيَّاتٍ وَحِثِّيَّاتٍ
\par 2 مِنَ الأُمَمِ الَّذِينَ قَالَ عَنْهُمُ الرَّبُّ لِبَنِي إِسْرَائِيلَ: [لاَ تَدْخُلُونَ إِلَيْهِمْ وَهُمْ لاَ يَدْخُلُونَ إِلَيْكُمْ، لأَنَّهُمْ يُمِيلُونَ قُلُوبَكُمْ وَرَاءَ آلِهَتِهِمْ]. فَالْتَصَقَ سُلَيْمَانُ بِهَؤُلاَءِ بِالْمَحَبَّةِ.
\par 3 وَكَانَتْ لَهُ سَبْعُ مِئَةٍ مِنَ النِّسَاءِ السَّيِّدَاتِ، وَثَلاَثُ مِئَةٍ مِنَ السَّرَارِيِّ. فَأَمَالَتْ نِسَاؤُهُ قَلْبَهُ.
\par 4 وَكَانَ فِي زَمَانِ شَيْخُوخَةِ سُلَيْمَانَ أَنَّ نِسَاءَهُ أَمَلْنَ قَلْبَهُ وَرَاءَ آلِهَةٍ أُخْرَى، وَلَمْ يَكُنْ قَلْبُهُ كَامِلاً مَعَ الرَّبِّ إِلَهِهِ كَقَلْبِ دَاوُدَ أَبِيهِ.
\par 5 فَذَهَبَ سُلَيْمَانُ وَرَاءَ عَشْتُورَثَ إِلَهَةِ الصَّيْدُونِيِّينَ وَمَلْكُومَ رِجْسِ الْعَمُّونِيِّينَ.
\par 6 وَعَمِلَ سُلَيْمَانُ الشَّرَّ فِي عَيْنَيِ الرَّبِّ، وَلَمْ يَتْبَعِ الرَّبَّ تَمَاماً كَدَاوُدَ أَبِيهِ.
\par 7 حِينَئِذٍ بَنَى سُلَيْمَانُ مُرْتَفَعَةً لِكَمُوشَ رِجْسِ الْمُوآبِيِّينَ عَلَى الْجَبَلِ الَّذِي تُجَاهَ أُورُشَلِيمَ، وَلِمُولَكَ رِجْسِ بَنِي عَمُّونَ.
\par 8 وَهَكَذَا فَعَلَ لِجَمِيعِ نِسَائِهِ الْغَرِيبَاتِ اللَّوَاتِي كُنَّ يُوقِدْنَ وَيَذْبَحْنَ لِآلِهَتِهِنَّ.
\par 9 فَغَضِبَ الرَّبُّ عَلَى سُلَيْمَانَ لأَنَّ قَلْبَهُ مَالَ عَنِ الرَّبِّ إِلَهِ إِسْرَائِيلَ الَّذِي تَرَاءَى لَهُ مَرَّتَيْنِ،
\par 10 وَأَوْصَاهُ فِي هَذَا الأَمْرِ أَنْ لاَ يَتَّبِعَ آلِهَةً أُخْرَى. فَلَمْ يَحْفَظْ مَا أَوْصَى بِهِ الرَّبُّ.
\par 11 فَقَالَ الرَّبُّ لِسُلَيْمَانَ: [مِنْ أَجْلِ أَنَّ ذَلِكَ عِنْدَكَ، وَلَمْ تَحْفَظْ عَهْدِي وَفَرَائِضِيَ الَّتِي أَوْصَيْتُكَ بِهَا، فَإِنِّي أُمَزِّقُ الْمَمْلَكَةَ عَنْكَ تَمْزِيقاً وَأُعْطِيهَا لِعَبْدِكَ.
\par 12 إِلاَّ إِنِّي لاَ أَفْعَلُ ذَلِكَ فِي أَيَّامِكَ، مِنْ أَجْلِ دَاوُدَ أَبِيكَ، بَلْ مِنْ يَدِ ابْنِكَ أُمَزِّقُهَا.
\par 13 عَلَى أَنِّي لاَ أُمَزِّقُ مِنْكَ الْمَمْلَكَةَ كُلَّهَا، بَلْ أُعْطِي سِبْطاً وَاحِداً لاِبْنِكَ، لأَجْلِ دَاوُدَ عَبْدِي، وَلأَجْلِ أُورُشَلِيمَ الَّتِي اخْتَرْتُهَا].
\par 14 وَأَقَامَ الرَّبُّ خَصْماً لِسُلَيْمَانَ: هَدَدَ الأَدُومِيَّ كَانَ مِنْ نَسْلِ الْمَلِكِ فِي أَدُومَ.
\par 15 وَحَدَثَ لَمَّا كَانَ دَاوُدُ فِي أَدُومَ، عِنْدَ صُعُودِ يُوآبَ رَئِيسِ الْجَيْشِ لِدَفْنِ الْقَتْلَى، وَضَرَبَ كُلَّ ذَكَرٍ فِي أَدُومَ.
\par 16 (لأَنَّ يُوآبَ وَكُلَّ إِسْرَائِيلَ أَقَامُوا هُنَاكَ سِتَّةَ أَشْهُرٍ حَتَّى أَفْنُوا كُلَّ ذَكَرٍ فِي أَدُومَ).
\par 17 أَنَّ هَدَدَ هَرَبَ هُوَ وَرِجَالٌ أَدُومِيُّونَ مِنْ عَبِيدِ أَبِيهِ مَعَهُ لِيَأْتُوا مِصْرَ. وَكَانَ هَدَدُ غُلاَماً صَغِيراً.
\par 18 وَقَامُوا مِنْ مِدْيَانَ وَأَتُوا إِلَى فَارَانَ وَأَخَذُوا مَعَهُمْ رِجَالاً مِنْ فَارَانَ وَأَتُوا إِلَى مِصْرَ إِلَى فِرْعَوْنَ مَلِكِ مِصْرَ، فَأَعْطَاهُ بَيْتاً وَعَيَّنَ لَهُ طَعَاماً وَأَعْطَاهُ أَرْضاً.
\par 19 فَوَجَدَ هَدَدُ نِعْمَةً فِي عَيْنَيْ فِرْعَوْنَ جِدّاً، وَزَوَّجَهُ أُخْتَ امْرَأَتِهِ (أُخْتَ تَحْفَنِيسَ الْمَلِكَةِ).
\par 20 فَوَلَدَتْ لَهُ أُخْتُ تَحْفَنِيسَ جَنُوبَثَ ابْنَهُ، وَفَطَمَتْهُ تَحْفَنِيسُ فِي وَسَطِ بَيْتِ فِرْعَوْنَ. وَكَانَ جَنُوبَثُ فِي بَيْتِ فِرْعَوْنَ بَيْنَ بَنِي فِرْعَوْنَ.
\par 21 فَسَمِعَ هَدَدُ فِي مِصْرَ بِأَنَّ دَاوُدَ قَدِ اضْطَجَعَ مَعَ آبَائِهِ، وَبِأَنَّ يُوآبَ رَئِيسَ الْجَيْشِ قَدْ مَاتَ. فَقَالَ هَدَدُ لِفِرْعَوْنَ: [أَطْلِقْنِي إِلَى أَرْضِي].
\par 22 فَقَالَ لَهُ فِرْعَوْنُ: [مَاذَا أَعْوَزَكَ عِنْدِي حَتَّى إِنَّكَ تَطْلُبُ الذَّهَابَ إِلَى أَرْضِكَ؟] فَقَالَ: [لاَ شَيْءَ، وَإِنَّمَا أَطْلِقْنِي].
\par 23 وَأَقَامَ اللَّهُ لَهُ خَصْماً آخَرَ رَزُونَ بْنَ أَلِيدَاعَ الَّذِي هَرَبَ مِنْ عِنْدِ سَيِّدِهِ هَدَدَ عَزَرَ مَلِكِ صُوبَةَ،
\par 24 فَجَمَعَ إِلَيْهِ رِجَالاً فَصَارَ رَئِيسَ غُزَاةٍ عِنْدَ قَتْلِ دَاوُدَ إِيَّاهُمْ. فَانْطَلَقُوا إِلَى دِمَشْقَ وَأَقَامُوا بِهَا وَمَلَكُوا فِي دِمَشْقَ.
\par 25 وَكَانَ خَصْماً لإِسْرَائِيلَ كُلَّ أَيَّامِ سُلَيْمَانَ (مَعَ شَرِّ هَدَدَ). فَكَرِهَ إِسْرَائِيلَ، وَمَلَكَ عَلَى أَرَامَ.
\par 26 وَيَرُبْعَامُ بْنُ نَابَاطَ، أَفْرَايِمِيٌّ مِنْ صَرَدَةَ، عَبْدٌ لِسُلَيْمَانَ. وَاسْمُ أُمِّهِ صَرُوعَةُ، وَهِيَ أَرْمَلَةٌ، رَفَعَ يَدَهُ عَلَى الْمَلِكِ.
\par 27 وَهَذَا هُوَ سَبَبُ رَفْعِهِ يَدَهُ عَلَى الْمَلِكِ: أَنَّ سُلَيْمَانَ بَنَى الْقَلْعَةَ وَسَدَّ شُقُوقَ مَدِينَةِ دَاوُدَ أَبِيهِ.
\par 28 وَكَانَ يَرُبْعَامُ جَبَّارَ بَأْسٍ. فَلَمَّا رَأَى سُلَيْمَانُ الْغُلاَمَ أَنَّهُ عَامِلٌ شُغْلاً أَقَامَهُ عَلَى كُلِّ أَعْمَالِ بَيْتِ يُوسُفَ.
\par 29 وَكَانَ فِي ذَلِكَ الزَّمَانِ لَمَّا خَرَجَ يَرُبْعَامُ مِنْ أُورُشَلِيمَ أَنَّهُ لاَقَاهُ أَخِيَّا الشِّيلُونِيُّ النَّبِيُّ فِي الطَّرِيقِ وَهُوَ لاَبِسٌ رِدَاءً جَدِيداً، وَهُمَا وَحْدَهُمَا فِي الْحَقْلِ.
\par 30 فَقَبَضَ أَخِيَّا عَلَى الرِّدَاءِ الْجَدِيدِ الَّذِي عَلَيْهِ وَمَزَّقَهُ اثْنَتَيْ عَشَرَةَ قِطْعَةً
\par 31 وَقَالَ لِيَرُبْعَامَ: [خُذْ لِنَفْسِكَ عَشَرَ قِطَعٍ، لأَنَّهُ هَكَذَا قَالَ الرَّبُّ إِلَهُ إِسْرَائِيلَ: هَئَنَذَا أُمَزِّقُ الْمَمْلَكَةَ مِنْ يَدِ سُلَيْمَانَ وَأُعْطِيكَ عَشَرَةَ أَسْبَاطٍ.
\par 32 وَيَكُونُ لَهُ سِبْطٌ وَاحِدٌ مِنْ أَجْلِ عَبْدِي دَاوُدَ وَمِنْ أَجْلِ أُورُشَلِيمَ الْمَدِينَةِ الَّتِي اخْتَرْتُهَا مِنْ كُلِّ أَسْبَاطِ إِسْرَائِيلَ،
\par 33 لأَنَّهُمْ تَرَكُونِي وَسَجَدُوا لِعَشْتُورَثَ إِلَهَةُ الصَّيْدُونِيِّينَ وَلِكَمُوشَ إِلَهِ الْمُوآبِيِّينَ وَلِمَلْكُومَ إِلَهِ بَنِي عَمُّونَ، وَلَمْ يَسْلُكُوا فِي طُرُقِي لِيَعْمَلُوا الْمُسْتَقِيمَ فِي عَيْنَيَّ وَفَرَائِضِي وَأَحْكَامِي كَدَاوُدَ أَبِيهِ.
\par 34 وَلاَ آخُذُ كُلَّ الْمَمْلَكَةِ مِنْ يَدِهِ، بَلْ أُصَيِّرُهُ رَئِيساً كُلَّ أَيَّامِ حَيَاتِهِ لأَجْلِ دَاوُدَ عَبْدِي الَّذِي اخْتَرْتُهُ الَّذِي حَفِظَ وَصَايَايَ وَفَرَائِضِي.
\par 35 وَآخُذُ الْمَمْلَكَةَ مِنْ يَدِ ابْنِهِ وَأُعْطِيكَ إِيَّاهَا (أَيِ الأَسْبَاطَ الْعَشَرَةَ).
\par 36 وَأُعْطِي ابْنَهُ سِبْطاً وَاحِداً لِيَكُونَ سِرَاجٌ لِدَاوُدَ عَبْدِي كُلَّ الأَيَّامِ أَمَامِي فِي أُورُشَلِيمَ الْمَدِينَةِ الَّتِي اخْتَرْتُهَا لِنَفْسِي لأَضَعَ اسْمِي فِيهَا.
\par 37 وَآخُذُكَ فَتَمْلِكُ حَسَبَ كُلِّ مَا تَشْتَهِي نَفْسُكَ، وَتَكُونُ مَلِكاً عَلَى إِسْرَائِيلَ.
\par 38 فَإِذَا سَمِعْتَ لِكُلِّ مَا أُوصِيكَ بِهِ وَسَلَكْتَ فِي طُرُقِي وَفَعَلْتَ مَا هُوَ مُسْتَقِيمٌ فِي عَيْنَيَّ وَحَفِظْتَ فَرَائِضِي وَوَصَايَايَ كَمَا فَعَلَ دَاوُدُ عَبْدِي، أَكُونُ مَعَكَ وَأَبْنِي لَكَ بَيْتاً آمِناً كَمَا بَنَيْتُ لِدَاوُدَ، وَأُعْطِيكَ إِسْرَائِيلَ.
\par 39 وَأُذِلُّ نَسْلَ دَاوُدَ مِنْ أَجْلِ هَذَا، وَلَكِنْ لاَ كُلَّ الأَيَّامِ].
\par 40 وَطَلَبَ سُلَيْمَانُ قَتْلَ يَرُبْعَامَ، فَقَامَ يَرُبْعَامُ وَهَرَبَ إِلَى مِصْرَ إِلَى شِيشَقَ مَلِكِ مِصْرَ. وَكَانَ فِي مِصْرَ إِلَى وَفَاةِ سُلَيْمَانَ.
\par 41 وَبَقِيَّةُ أُمُورِ سُلَيْمَانَ وَكُلُّ مَا صَنَعَ وَحِكْمَتُهُ هِيَ مَكْتُوبَةٌ فِي سِفْرِ أُمُورِ سُلَيْمَانَ.
\par 42 وَكَانَتِ الأَيَّامُ الَّتِي مَلَكَ فِيهَا سُلَيْمَانُ فِي أُورُشَلِيمَ عَلَى كُلِّ إِسْرَائِيلَ أَرْبَعِينَ سَنَةً.
\par 43 ثُمَّ اضْطَجَعَ سُلَيْمَانُ مَعَ آبَائِهِ وَدُفِنَ فِي مَدِينَةِ دَاوُدَ أَبِيهِ، وَمَلَكَ رَحُبْعَامُ ابْنُهُ عِوَضاً عَنْهُ.

\chapter{12}

\par 1 وَذَهَبَ رَحُبْعَامُ إِلَى شَكِيمَ، لأَنَّهُ جَاءَ إِلَى شَكِيمَ جَمِيعُ إِسْرَائِيلَ لِيُمَلِّكُوهُ.
\par 2 وَلَمَّا سَمِعَ يَرُبْعَامُ بْنُ نَبَاطَ وَهُوَ بَعْدُ فِي مِصْرَ. (لأَنَّهُ هَرَبَ مِنْ وَجْهِ سُلَيْمَانَ الْمَلِكِ، وَأَقَامَ يَرُبْعَامُ فِي مِصْرَ)
\par 3 وَأَرْسَلُوا فَدَعُوهُ. أَتَى يَرُبْعَامُ وَكُلُّ جَمَاعَةِ إِسْرَائِيلَ وَقَالُوا لِرَحُبْعَامَ:
\par 4 [إِنَّ أَبَاكَ قَسَّى نِيرَنَا وَأَمَّا أَنْتَ فَخَفِّفِ الآنَ مِنْ عُبُودِيَّةِ أَبِيكَ الْقَاسِيَةِ وَمِنْ نِيرِهِ الثَّقِيلِ الَّذِي جَعَلَهُ عَلَيْنَا فَنَخْدِمَكَ].
\par 5 فَقَالَ لَهُمُ: [اذْهَبُوا إِلَى ثَلاَثَةِ أَيَّامٍ أَيْضاً ثُمَّ ارْجِعُوا إِلَيَّ]. فَذَهَبَ الشَّعْبُ.
\par 6 فَاسْتَشَارَ الْمَلِكُ رَحُبْعَامُ الشُّيُوخَ الَّذِينَ كَانُوا يَقِفُونَ أَمَامَ سُلَيْمَانَ أَبِيهِ وَهُوَ حَيٌّ قَائِلاً: [كَيْفَ تُشِيرُونَ أَنْ أَرُدَّ جَوَاباً إِلَى هَذَا الشَّعْبِ؟]
\par 7 فَقَالُوا: [إِنْ صِرْتَ الْيَوْمَ عَبْداً لِهَذَا الشَّعْبِ وَخَدَمْتَهُمْ وَأَجَبْتَهُمْ وَكَلَّمْتَهُمْ كَلاَماً حَسَناً، يَكُونُونَ لَكَ عَبِيداً كُلَّ الأَيَّامِ].
\par 8 فَتَرَكَ مَشُورَةَ الشُّيُوخِ الَّتِي أَشَارُوا بِهَا عَلَيْهِ وَاسْتَشَارَ الأَحْدَاثَ الَّذِينَ نَشَأُوا مَعَهُ وَوَقَفُوا أَمَامَهُ،
\par 9 وَقَالَ لَهُمْ: [بِمَاذَا تُشِيرُونَ أَنْتُمْ فَنَرُدَّ جَوَاباً عَلَى هَذَا الشَّعْبِ الَّذِينَ قَالُوا لِي: خَفِّفْ مِنَ النِّيرِ الَّذِي جَعَلَهُ عَلَيْنَا أَبُوكَ].
\par 10 فَقَالَ الأَحْدَاثُ الَّذِينَ نَشَأُوا مَعَهُ: [هَكَذَا تَقُولُ لِهَذَا الشَّعْبِ الَّذِينَ قَالُوا لَكَ إِنَّ أَبَاكَ ثَقَّلَ نِيرَنَا وَأَمَّا أَنْتَ فَخَفِّفْ مِنْ نِيرِنَا: إِنَّ خِنْصَرِي أَغْلَظُ مِنْ وَسْطِ أَبِي.
\par 11 وَالآنَ أَبِي حَمَّلَكُمْ نِيراً ثَقِيلاً وَأَنَا أَزِيدُ عَلَى نِيرِكُمْ. أَبِي أَدَّبَكُمْ بِالسِّيَاطِ وَأَنَا أُؤَدِّبُكُمْ بِالْعَقَارِبِ].
\par 12 فَجَاءَ يَرُبْعَامُ وَجَمِيعُ الشَّعْبِ إِلَى رَحُبْعَامَ فِي الْيَوْمِ الثَّالِثِ كَمَا قَالَ الْمَلِكُ: [ارْجِعُوا إِلَيَّ فِي الْيَوْمِ الثَّالِثِ].
\par 13 فَأَجَابَ الْمَلِكُ الشَّعْبَ بِقَسَاوَةٍ، وَتَرَكَ مَشُورَةَ الشُّيُوخِ الَّتِي أَشَارُوا بِهَا عَلَيْهِ،
\par 14 وَقَالَ حَسَبَ مَشُورَةِ الأَحْدَاثِ: [أَبِي ثَقَّلَ نِيرَكُمْ وَأَنَا أَزِيدُ عَلَى نِيرِكُمْ. أَبِي أَدَّبَكُمْ بِالسِّيَاطِ وَأَنَا أُؤَدِّبُكُمْ بِالْعَقَارِبِ].
\par 15 وَلَمْ يَسْمَعِ الْمَلِكُ لِلشَّعْبِ، لأَنَّ السَّبَبَ كَانَ مِنْ قِبَلِ الرَّبِّ لِيُقِيمَ كَلاَمَهُ الَّذِي تَكَلَّمَ بِهِ الرَّبُّ عَنْ يَدِ أَخِيَّا الشِّيلُونِيِّ إِلَى يَرُبْعَامَ بْنِ نَبَاطَ.
\par 16 فَلَمَّا رَأَى كُلُّ إِسْرَائِيلَ أَنَّ الْمَلِكَ لَمْ يَسْمَعْ لَهُمْ، أَجَابَ الشَّعْبُ الْمَلِكَ: [أَيُّ قِسْمٍ لَنَا فِي دَاوُدَ، وَلاَ نَصِيبَ لَنَا فِي ابْنِ يَسَّى! إِلَى خِيَامِكَ يَا إِسْرَائِيلُ. الآنَ انْظُرْ إِلَى بَيْتِكَ يَا دَاوُدُ]. وَذَهَبَ إِسْرَائِيلُ إِلَى خِيَامِهِمْ.
\par 17 وَأَمَّا بَنُو إِسْرَائِيلَ السَّاكِنُونَ فِي مُدُنِ يَهُوذَا فَمَلَكَ عَلَيْهِمْ رَحُبْعَامُ.
\par 18 ثُمَّ أَرْسَلَ الْمَلِكُ رَحُبْعَامُ أَدُورَامَ الَّذِي عَلَى التَّسْخِيرِ فَرَجَمَهُ جَمِيعُ إِسْرَائِيلَ بِالْحِجَارَةِ فَمَاتَ. فَبَادَرَ الْمَلِكُ رَحُبْعَامُ وَصَعِدَ إِلَى الْمَرْكَبَةِ لِيَهْرُبَ إِلَى أُورُشَلِيمَ.
\par 19 فَعَصَى إِسْرَائِيلُ عَلَى بَيْتِ دَاوُدَ إِلَى هَذَا الْيَوْمِ.
\par 20 وَلَمَّا سَمِعَ جَمِيعُ إِسْرَائِيلَ بِأَنَّ يَرُبْعَامَ قَدْ رَجَعَ، أَرْسَلُوا فَدَعُوهُ إِلَى الْجَمَاعَةِ وَمَلَّكُوهُ عَلَى جَمِيعِ إِسْرَائِيلَ. لَمْ يَتْبَعْ بَيْتَ دَاوُدَ إِلاَّ سِبْطُ يَهُوذَا وَحْدَهُ.
\par 21 وَلَمَّا جَاءَ رَحُبْعَامُ إِلَى أُورُشَلِيمَ جَمَعَ كُلَّ بَيْتِ يَهُوذَا وَسِبْطَ بِنْيَامِينَ، مِئَةً وَثَمَانِينَ أَلْفَ مُخْتَارٍ مُحَارِبٍ لِيُحَارِبُوا بَيْتَ إِسْرَائِيلَ وَيَرُدُّوا الْمَمْلَكَةَ لَرَحُبْعَامَ بْنِ سُلَيْمَانَ.
\par 22 وَكَانَ كَلاَمُ اللَّهِ إِلَى شَمَعْيَا رَجُلِ اللَّهِ:
\par 23 [قُلْ لِرَحُبْعَامَ بْنِ سُلَيْمَانَ مَلِكِ يَهُوذَا وَكُلِّ بَيْتِ يَهُوذَا وَبِنْيَامِينَ وَبَقِيَّةِ الشَّعْبِ:
\par 24 هَكَذَا قَالَ الرَّبُّ: لاَ تَصْعَدُوا وَلاَ تُحَارِبُوا إِخْوَتَكُمْ بَنِي إِسْرَائِيلَ. ارْجِعُوا كُلُّ وَاحِدٍ إِلَى بَيْتِهِ، لأَنَّ مِنْ عِنْدِي هَذَا الأَمْرَ]. فَسَمِعُوا لِكَلاَمِ الرَّبِّ وَرَجَعُوا لِيَنْطَلِقُوا حَسَبَ قَوْلِ الرَّبِّ.
\par 25 وَبَنَى يَرُبْعَامُ شَكِيمَ فِي جَبَلِ أَفْرَايِمَ وَسَكَنَ بِهَا. ثُمَّ خَرَجَ مِنْ هُنَاكَ وَبَنَى فَنُوئِيلَ.
\par 26 وَقَالَ يَرُبْعَامُ فِي قَلْبِهِ: [الآنَ تَرْجِعُ الْمَمْلَكَةُ إِلَى بَيْتِ دَاوُدَ.
\par 27 إِنْ صَعِدَ هَذَا الشَّعْبُ لِيُقَرِّبُوا ذَبَائِحَ فِي بَيْتِ الرَّبِّ فِي أُورُشَلِيمَ يَرْجِعْ قَلْبُ هَذَا الشَّعْبِ إِلَى سَيِّدِهِمْ إِلَى رَحُبْعَامَ مَلِكِ يَهُوذَا وَيَقْتُلُونِي وَيَرْجِعُوا إِلَى رَحُبْعَامَ مَلِكِ يَهُوذَا].
\par 28 فَاسْتَشَارَ الْمَلِكُ وَعَمِلَ عِجْلَيْ ذَهَبٍ، وَقَالَ لَهُمْ: [كَثِيرٌ عَلَيْكُمْ أَنْ تَصْعَدُوا إِلَى أُورُشَلِيمَ. هُوَذَا آلِهَتُكَ يَا إِسْرَائِيلُ الَّذِينَ أَصْعَدُوكَ مِنْ أَرْضِ مِصْرَ].
\par 29 وَوَضَعَ وَاحِداً فِي بَيْتِ إِيلَ وَجَعَلَ الآخَرَ فِي دَانَ.
\par 30 وَكَانَ هَذَا الأَمْرُ خَطِيَّةً. وَكَانَ الشَّعْبُ يَذْهَبُونَ إِلَى أَمَامِ أَحَدِهِمَا حَتَّى إِلَى دَانَ.
\par 31 وَبَنَى بَيْتَ الْمُرْتَفَعَاتِ، وَصَيَّرَ كَهَنَةً مِنْ أَطْرَافِ الشَّعْبِ لَمْ يَكُونُوا مِنْ بَنِي لاَوِي.
\par 32 وَعَمِلَ يَرُبْعَامُ عِيداً فِي الشَّهْرِ الثَّامِنِ فِي الْيَوْمِ الْخَامِسَ عَشَرَ مِنَ الشَّهْرِ كَالْعِيدِ الَّذِي فِي يَهُوذَا، وَأَصْعَدَ عَلَى الْمَذْبَحِ. هَكَذَا فَعَلَ فِي بَيْتِ إِيلَ بِذَبْحِهِ لِلْعِجْلَيْنِ اللَّذَيْنِ عَمِلَهُمَا. وَأَوْقَفَ فِي بَيْتِ إِيلَ كَهَنَةَ الْمُرْتَفَعَاتِ الَّتِي عَمِلَهَا.
\par 33 وَأَصْعَدَ عَلَى الْمَذْبَحِ الَّذِي عَمِلَ فِي بَيْتِ إِيلَ فِي الْيَوْمِ الْخَامِسَ عَشَرَ مِنَ الشَّهْرِ الثَّامِنِ، فِي الشَّهْرِ الَّذِي ابْتَدَعَهُ مِنْ قَلْبِهِ، فَعَمِلَ عِيداً لِبَنِي إِسْرَائِيلَ وَصَعِدَ عَلَى الْمَذْبَحِ لِيُوقِدَ.

\chapter{13}

\par 1 وَإِذَا بِرَجُلِ اللَّهِ قَدْ أَتَى مِنْ يَهُوذَا بِكَلاَمِ الرَّبِّ إِلَى بَيْتِ إِيلَ، وَيَرُبْعَامُ وَاقِفٌ لَدَى الْمَذْبَحِ لِيُوقِدَ.
\par 2 فَنَادَى نَحْوَ الْمَذْبَحِ بِكَلاَمِ الرَّبِّ: [يَا مَذْبَحُ يَا مَذْبَحُ، هَكَذَا قَالَ الرَّبُّ: هُوَذَا سَيُولَدُ لِبَيْتِ دَاوُدَ ابْنٌ اسْمُهُ يُوشِيَّا، وَيَذْبَحُ عَلَيْكَ كَهَنَةَ الْمُرْتَفَعَاتِ الَّذِينَ يُوقِدُونَ عَلَيْكَ، وَتُحْرَقُ عَلَيْكَ عِظَامُ النَّاسِ].
\par 3 وَأَعْطَى فِي ذَلِكَ الْيَوْمِ عَلاَمَةً قَائِلاً: [هَذِهِ هِيَ الْعَلاَمَةُ الَّتِي تَكَلَّمَ بِهَا الرَّبُّ: هُوَذَا الْمَذْبَحُ يَنْشَقُّ وَيُذْرَى الرَّمَادُ الَّذِي عَلَيْهِ].
\par 4 فَلَمَّا سَمِعَ الْمَلِكُ كَلاَمَ رَجُلِ اللَّهِ الَّذِي نَادَى نَحْوَ الْمَذْبَحِ فِي بَيْتِ إِيلَ، مَدَّ يَرُبْعَامُ يَدَهُ عَنِ الْمَذْبَحِ قَائِلاً: [أَمْسِكُوهُ]. فَيَبِسَتْ يَدُهُ الَّتِي مَدَّهَا نَحْوَهُ وَلَمْ يَسْتَطِعْ أَنْ يَرُدَّهَا إِلَيْهِ!
\par 5 وَانْشَقَّ الْمَذْبَحُ وَذُرِيَ الرَّمَادُ مِنْ عَلَيْهِ حَسَبَ الْعَلاَمَةِ الَّتِي أَعْطَاهَا رَجُلُ اللَّهِ بِكَلاَمِ الرَّبِّ.
\par 6 فَقَالَ الْمَلِكُ لِرَجُلِ اللَّهِ: [تَضَرَّعْ إِلَى وَجْهِ الرَّبِّ إِلَهِكَ وَصَلِّ مِنْ أَجْلِي فَتَرْجِعَ يَدِي إِلَيَّ]. فَتَضَرَّعَ رَجُلُ اللَّهِ إِلَى وَجْهِ الرَّبِّ فَرَجَعَتْ يَدُ الْمَلِكِ إِلَيْهِ وَكَانَتْ كَمَا فِي الأَوَّلِ.
\par 7 ثُمَّ قَالَ الْمَلِكُ لِرَجُلِ اللَّهِ: [ادْخُلْ مَعِي إِلَى الْبَيْتِ وَتَقَوَّتْ فَأُعْطِيَكَ أُجْرَةً].
\par 8 فَقَالَ رَجُلُ اللَّهِ لِلْمَلِكِ: [لَوْ أَعْطَيْتَنِي نِصْفَ بَيْتِكَ لاَ أَدْخُلُ مَعَكَ وَلاَ آكُلُ خُبْزاً وَلاَ أَشْرَبُ مَاءً فِي هَذَا الْمَوْضِعِ.
\par 9 لأَنِّي هَكَذَا أُوصِيتُ بِكَلاَمِ الرَّبِّ: لاَ تَأْكُلْ خُبْزاً وَلاَ تَشْرَبْ مَاءً وَلاَ تَرْجِعْ فِي الطَّرِيقِ الَّذِي ذَهَبْتَ فِيهِ].
\par 10 فَذَهَبَ فِي طَرِيقٍ آخَرَ، وَلَمْ يَرْجِعْ فِي الطَّرِيقِ الَّذِي جَاءَ فِيهِ إِلَى بَيْتِ إِيلَ.
\par 11 وَكَانَ نَبِيٌّ شَيْخٌ سَاكِناً فِي بَيْتِ إِيلَ. فَأَتَى بَنُوهُ وَقَصُّوا عَلَيْهِ كُلَّ الْعَمَلِ الَّذِي عَمِلَهُ رَجُلُ اللَّهِ ذَلِكَ الْيَوْمَ فِي بَيْتِ إِيلَ، وَقَصُّوا عَلَى أَبِيهِمِ الْكَلاَمَ الَّذِي تَكَلَّمَ بِهِ إِلَى الْمَلِكِ.
\par 12 فَقَالَ لَهُمْ أَبُوهُمْ: [مِنْ أَيِّ طَرِيقٍ ذَهَبَ؟] وَكَانَ بَنُوهُ قَدْ رَأُوا الطَّرِيقَ الَّذِي سَارَ فِيهِ رَجُلُ اللَّهِ.
\par 13 فَقَالَ لِبَنِيهِ: [شُدُّوا لِي عَلَى الْحِمَارِ]. فَشَدُّوا لَهُ عَلَى الْحِمَارِ فَرَكِبَ عَلَيْهِ
\par 14 وَسَارَ وَرَاءَ رَجُلِ اللَّهِ، فَوَجَدَهُ جَالِساً تَحْتَ الْبَلُّوطَةِ، فَقَالَ لَهُ: [أَأَنْتَ رَجُلُ اللَّهِ الَّذِي جَاءَ مِنْ يَهُوذَا؟] فَقَالَ: [أَنَا هُوَ].
\par 15 فَقَالَ لَهُ: [سِرْ مَعِي إِلَى الْبَيْتِ وَكُلْ خُبْزاً].
\par 16 فَقَالَ: [لاَ أَقْدِرُ أَنْ أَرْجِعَ مَعَكَ وَلاَ أَدْخُلُ مَعَكَ وَلاَ آكُلُ خُبْزاً وَلاَ أَشْرَبُ مَعَكَ مَاءً فِي هَذَا الْمَوْضِعِ.
\par 17 لأَنَّهُ قِيلَ لِي بِكَلاَمِ الرَّبِّ: لاَ تَأْكُلْ خُبْزاً وَلاَ تَشْرَبْ هُنَاكَ مَاءً وَلاَ تَرْجِعْ سَائِراً فِي الطَّرِيقِ الَّذِي ذَهَبْتَ فِيهِ].
\par 18 فَقَالَ لَهُ: [أَنَا أَيْضاً نَبِيٌّ مِثْلُكَ، وَقَدْ كَلَّمَنِي مَلاَكٌ بِكَلاَمِ الرَّبِّ قَائِلاً: ارْجِعْ بِهِ مَعَكَ إِلَى بَيْتِكَ فَيَأْكُلَ خُبْزاً وَيَشْرَبَ مَاءً]- كَذَبَ عَلَيْهِ.
\par 19 فَرَجَعَ مَعَهُ وَأَكَلَ خُبْزاً فِي بَيْتِهِ وَشَرِبَ مَاءً.
\par 20 وَبَيْنَمَا هُمَا جَالِسَانِ عَلَى الْمَائِدَةِ كَانَ كَلاَمُ الرَّبِّ إِلَى النَّبِيِّ الَّذِي أَرْجَعَهُ،
\par 21 فَصَاحَ إِلَى رَجُلِ اللَّهِ الَّذِي جَاءَ مِنْ يَهُوذَا: [هَكَذَا قَالَ الرَّبُّ: مِنْ أَجْلِ أَنَّكَ خَالَفْتَ قَوْلَ الرَّبِّ وَلَمْ تَحْفَظِ الْوَصِيَّةَ الَّتِي أَوْصَاكَ بِهَا الرَّبُّ إِلَهُكَ،
\par 22 فَرَجَعْتَ وَأَكَلْتَ خُبْزاً وَشَرِبْتَ مَاءً فِي الْمَوْضِعِ الَّذِي قَالَ لَكَ: لاَ تَأْكُلْ فِيهِ خُبْزاً وَلاَ تَشْرَبْ مَاءً، لاَ تَدْخُلُ جُثَّتُكَ قَبْرَ آبَائِكَ].
\par 23 ثُمَّ بَعْدَمَا أَكَلَ خُبْزاً وَبَعْدَ أَنْ شَرِبَ شَدَّ لَهُ عَلَى الْحِمَارِ (أَيْ لِلنَّبِيِّ الَّذِي أَرْجَعَهُ)
\par 24 وَانْطَلَقَ. فَصَادَفَهُ أَسَدٌ فِي الطَّرِيقِ وَقَتَلَهُ. وَكَانَتْ جُثَّتُهُ مَطْرُوحَةً فِي الطَّرِيقِ وَالْحِمَارُ وَاقِفٌ بِجَانِبِهَا وَالأَسَدُ وَاقِفٌ بِجَانِبِ الْجُثَّةِ.
\par 25 وَإِذَا بِقَوْمٍ يَعْبُرُونَ فَرَأُوا الْجُثَّةَ مَطْرُوحَةً فِي الطَّرِيقِ وَالأَسَدُ وَاقِفٌ بِجَانِبِ الْجُثَّةِ. فَأَتُوا وَأَخْبَرُوا فِي الْمَدِينَةِ الَّتِي كَانَ النَّبِيُّ الشَّيْخُ سَاكِناً بِهَا.
\par 26 وَلَمَّا سَمِعَ النَّبِيُّ الَّذِي أَرْجَعَهُ عَنِ الطَّرِيقِ قَالَ: [هُوَ رَجُلُ اللَّهِ الَّذِي خَالَفَ قَوْلَ الرَّبِّ، فَدَفَعَهُ الرَّبُّ لِلأَسَدِ فَافْتَرَسَهُ وَقَتَلَهُ حَسَبَ كَلاَمِ الرَّبِّ الَّذِي كَلَّمَهُ بِهِ].
\par 27 وَقَالَ لِبَنِيهِ: [شُدُّوا لِي عَلَى الْحِمَارِ]. فَشَدُّوا.
\par 28 فَذَهَبَ وَوَجَدَ جُثَّتَهُ مَطْرُوحَةً فِي الطَّرِيقِ، وَالْحِمَارَ وَالأَسَدَ وَاقِفَيْنِ بِجَانِبِ الْجُثَّةِ، وَلَمْ يَأْكُلِ الأَسَدُ الْجُثَّةَ وَلاَ افْتَرَسَ الْحِمَارَ.
\par 29 فَرَفَعَ النَّبِيُّ جُثَّةَ رَجُلِ اللَّهِ وَوَضَعَهَا عَلَى الْحِمَارِ وَرَجَعَ بِهَا، وَدَخَلَ النَّبِيُّ الشَّيْخُ الْمَدِينَةَ لِيَنْدُبَهُ وَيَدْفِنَهُ
\par 30 فَوَضَعَ جُثَّتَهُ فِي قَبْرِهِ وَنَاحُوا عَلَيْهِ قَائِلِينَ: [آهُ يَا أَخِي!]
\par 31 وَبَعْدَ دَفْنِهِ إِيَّاهُ قَالَ لِبَنِيهِ: [عِنْدَ وَفَاتِي ادْفِنُونِي فِي الْقَبْرِ الَّذِي دُفِنَ فِيهِ رَجُلُ اللَّهِ. بِجَانِبِ عِظَامِهِ ضَعُوا عِظَامِي.
\par 32 لأَنَّهُ تَمَاماً سَيَتِمُّ الْكَلاَمُ الَّذِي نَادَى بِهِ بِكَلاَمِ الرَّبِّ نَحْوَ الْمَذْبَحِ الَّذِي فِي بَيْتِ إِيلَ، وَنَحْوَ جَمِيعِ بُيُوتِ الْمُرْتَفَعَاتِ الَّتِي فِي مُدُنِ السَّامِرَةِ].
\par 33 بَعْدَ هَذَا الأَمْرِ لَمْ يَرْجِعْ يَرُبْعَامُ عَنْ طَرِيقِهِ الرَّدِيِئَةِ، بَلْ عَادَ فَعَمِلَ مِنْ أَطْرَافِ الشَّعْبِ كَهَنَةَ مُرْتَفَعَاتٍ. مَنْ شَاءَ مَلَأَ يَدَهُ فَصَارَ مِنْ كَهَنَةِ الْمُرْتَفَعَاتِ.
\par 34 وَكَانَ مِنْ هَذَا الأَمْرِ خَطِيَّةٌ لِبَيْتِ يَرُبْعَامَ، وَكَانَ لإِبَادَتِهِ وَخَرَابِهِ عَنْ وَجْهِ الأَرْضِ.

\chapter{14}

\par 1 فِي ذَلِكَ الزَّمَانِ مَرِضَ أَبِيَّا بْنُ يَرُبْعَامَ.
\par 2 فَقَالَ يَرُبْعَامُ لاِمْرَأَتِهِ: [قُومِي غَيِّرِي شَكْلَكِ حَتَّى لاَ يَعْلَمُوا أَنَّكِ امْرَأَةُ يَرُبْعَامَ وَاذْهَبِي إِلَى شِيلُوهَ. هُوَذَا هُنَاكَ أَخِيَّا النَّبِيُّ الَّذِي قَالَ عَنِّي إِنِّي أَمْلِكُ عَلَى هَذَا الشَّعْبِ.
\par 3 وَخُذِي بِيَدِكِ عَشَرَةَ أَرْغِفَةٍ وَكَعْكاً وَجَرَّةَ عَسَلٍ، وَسِيرِي إِلَيْهِ وَهُوَ يُخْبِرُكِ مَاذَا يَكُونُ لِلْغُلاَمِ].
\par 4 فَفَعَلَتِ امْرَأَةُ يَرُبْعَامَ هَكَذَا، وَقَامَتْ وَذَهَبَتْ إِلَى شِيلُوهَ وَدَخَلَتْ بَيْتَ أَخِيَّا. وَكَانَ أَخِيَّا لاَ يَقْدِرُ أَنْ يُبْصِرَ لأَنَّهُ قَدْ ضَعُفَتْ عَيْنَاهُ بِسَبَبِ شَيْخُوخَتِهِ.
\par 5 وَقَالَ الرَّبُّ لأَخِيَّا: [هُوَذَا امْرَأَةُ يَرُبْعَامَ آتِيَةٌ لِتَسْأَلَ مِنْكَ شَيْئاً مِنْ جِهَةِ ابْنِهَا لأَنَّهُ مَرِيضٌ. فَقُلْ لَهَا: كَذَا وَكَذَا، فَإِنَّهَا عِنْدَ دُخُولِهَا تَتَنَكَّرُ].
\par 6 فَلَمَّا سَمِعَ أَخِيَّا حِسَّ رِجْلَيْهَا وَهِيَ دَاخِلَةٌ فِي الْبَابِ قَالَ: [ادْخُلِي يَا امْرَأَةَ يَرُبْعَامَ. لِمَاذَا تَتَنَكَّرِينَ وَأَنَا مُرْسَلٌ إِلَيْكِ بِقَوْلٍ قَاسٍ؟
\par 7 اِذْهَبِي قُولِي لِيَرُبْعَامَ: هَكَذَا قَالَ الرَّبُّ إِلَهُ إِسْرَائِيلَ: مِنْ أَجْلِ أَنِّي قَدْ رَفَعْتُكَ مِنْ وَسَطِ الشَّعْبِ وَجَعَلْتُكَ رَئِيساً عَلَى شَعْبِي إِسْرَائِيلَ،
\par 8 وَشَقَقْتُ الْمَمْلَكَةَ مِنْ بَيْتِ دَاوُدَ وَأَعْطَيْتُكَ إِيَّاهَا، وَلَمْ تَكُنْ كَعَبْدِي دَاوُدَ الَّذِي حَفِظَ وَصَايَايَ وَالَّذِي سَارَ وَرَائِي بِكُلِّ قَلْبِهِ لِيَفْعَلَ مَا هُوَ مُسْتَقِيمٌ فَقَطْ فِي عَيْنَيَّ،
\par 9 وَقَدْ سَاءَ عَمَلُكَ أَكْثَرَ مِنْ جَمِيعِ الَّذِينَ كَانُوا قَبْلَكَ فَسِرْتَ وَعَمِلْتَ لِنَفْسِكَ آلِهَةً أُخْرَى وَمَسْبُوكَاتٍ لِتُغِيظَنِي وَقَدْ طَرَحْتَنِي وَرَاءَ ظَهْرِكَ،
\par 10 لِذَلِكَ هَئَنَذَا جَالِبٌ شَرّاً عَلَى بَيْتِ يَرُبْعَامَ، وَأَقْطَعُ لِيَرُبْعَامَ كُلَّ ذَكَرٍ مَحْجُوزاً وَمُطْلَقاً فِي إِسْرَائِيلَ. وَأَنْزِعُ آخِرَ بَيْتِ يَرُبْعَامَ كَمَا يُنْزَعُ الْبَعْرُ حَتَّى يَفْنَى.
\par 11 مَنْ مَاتَ لِيَرُبْعَامَ فِي الْمَدِينَةِ تَأْكُلُهُ الْكِلاَبُ، وَمَنْ مَاتَ فِي الْحَقْلِ تَأْكُلُهُ طُيُورُ السَّمَاءِ، لأَنَّ الرَّبَّ تَكَلَّمَ.
\par 12 وَأَنْتِ فَقُومِي وَانْطَلِقِي إِلَى بَيْتِكِ، وَعِنْدَ دُخُولِ رِجْلَيْكِ الْمَدِينَةَ يَمُوتُ الْوَلَدُ.
\par 13 وَيَنْدُبُهُ جَمِيعُ إِسْرَائِيلَ وَيَدْفِنُونَهُ، لأَنَّ هَذَا وَحْدَهُ مِنْ يَرُبْعَامَ يَدْخُلُ الْقَبْرَ لأَنَّهُ وُجِدَ فِيهِ أَمْرٌ صَالِحٌ نَحْوَ الرَّبِّ إِلَهِ إِسْرَائِيلَ فِي بَيْتِ يَرُبْعَامَ.
\par 14 وَيُقِيمُ الرَّبُّ لِنَفْسِهِ مَلِكاً عَلَى إِسْرَائِيلَ يَقْرِضُ بَيْتَ يَرُبْعَامَ هَذَا الْيَوْمَ. وَمَاذَا؟ الآنَ أَيْضاً!
\par 15 وَيَضْرِبُ الرَّبُّ إِسْرَائِيلَ كَاهْتِزَازِ الْقَصَبِ فِي الْمَاءِ، وَيَسْتَأْصِلُ إِسْرَائِيلَ عَنْ هَذِهِ الأَرْضِ الصَّالِحَةِ الَّتِي أَعْطَاهَا لِآبَائِهِمْ، وَيُبَدِّدُهُمْ إِلَى عَبْرِ النَّهْرِ لأَنَّهُمْ عَمِلُوا سَوَارِيَهُمْ وَأَغَاظُوا الرَّبَّ.
\par 16 وَيَدْفَعُ إِسْرَائِيلَ مِنْ أَجْلِ خَطَايَا يَرُبْعَامَ الَّذِي أَخْطَأَ وَجَعَلَ إِسْرَائِيلَ يُخْطِئُ].
\par 17 فَقَامَتِ امْرَأَةُ يَرُبْعَامَ وَذَهَبَتْ وَجَاءَتْ إِلَى تِرْصَةَ. وَلَمَّا وَصَلَتْ إِلَى عَتَبَةِ الْبَابِ مَاتَ الْغُلاَمُ،
\par 18 فَدَفَنَهُ وَنَدَبَهُ جَمِيعُ إِسْرَائِيلَ حَسَبَ كَلاَمِ الرَّبِّ الَّذِي تَكَلَّمَ بِهِ عَنْ يَدِ عَبْدِهِ أَخِيَّا النَّبِيِّ.
\par 19 وَأَمَّا بَقِيَّةُ أُمُورِ يَرُبْعَامَ، كَيْفَ حَارَبَ وَكَيْفَ مَلَكَ، فَإِنَّهَا مَكْتُوبَةٌ فِي سِفْرِ أَخْبَارِ الأَيَّامِ لِمُلُوكِ إِسْرَائِيلَ.
\par 20 وَالزَّمَانُ الَّذِي مَلَكَ فِيهِ يَرُبْعَامُ هُوَ اثْنَتَانِ وَعِشْرُونَ سَنَةً، ثُمَّ اضْطَجَعَ مَعَ آبَائِهِ وَمَلَكَ نَادَابُ ابْنُهُ عِوَضاً عَنْهُ.
\par 21 وَأَمَّا رَحُبْعَامُ بْنُ سُلَيْمَانَ فَمَلَكَ فِي يَهُوذَا. وَكَانَ رَحُبْعَامُ ابْنَ إِحْدَى وَأَرْبَعِينَ سَنَةً حِينَ مَلَكَ، وَمَلَكَ سَبْعَ عَشَرَةَ سَنَةً فِي أُورُشَلِيمَ الْمَدِينَةِ الَّتِي اخْتَارَهَا الرَّبُّ لِوَضْعِ اسْمِهِ فِيهَا مِنْ جَمِيعِ أَسْبَاطِ إِسْرَائِيلَ. وَاسْمُ أُمِّهِ نِعْمَةُ الْعَمُّونِيَّةُ.
\par 22 وَعَمِلَ يَهُوذَا الشَّرَّ فِي عَيْنَيِ الرَّبِّ وَأَغَارُوهُ أَكْثَرَ مِنْ جَمِيعِ مَا عَمِلَ آبَاؤُهُمْ بِخَطَايَاهُمُ الَّتِي أَخْطَأُوا بِهَا.
\par 23 وَبَنُوا هُمْ أَيْضاً لأَنْفُسِهِمْ مُرْتَفَعَاتٍ وَأَنْصَاباً وَسَوَارِيَ عَلَى كُلِّ تَلٍّ مُرْتَفِعٍ وَتَحْتَ كُلِّ شَجَرَةٍ خَضْرَاءَ.
\par 24 وَكَانَ أَيْضاً مَأْبُونُونَ فِي الأَرْضِ. فَعَلُوا حَسَبَ كُلِّ أَرْجَاسِ الأُمَمِ الَّذِينَ طَرَدَهُمُ الرَّبُّ مِنْ أَمَامِ بَنِي إِسْرَائِيلَ.
\par 25 وَفِي السَّنَةِ الْخَامِسَةِ لِلْمَلِكِ رَحُبْعَامَ صَعِدَ شِيشَقُ مَلِكُ مِصْرَ إِلَى أُورُشَلِيمَ
\par 26 وَأَخَذَ خَزَائِنَ بَيْتِ الرَّبِّ وَخَزَائِنَ بَيْتِ الْمَلِكِ، وَأَخَذَ كُلَّ شَيْءٍ. وَأَخَذَ جَمِيعَ أَتْرَاسِ الذَّهَبِ الَّتِي عَمِلَهَا سُلَيْمَانُ.
\par 27 فَعَمِلَ الْمَلِكُ رَحُبْعَامُ عِوَضاً عَنْهَا أَتْرَاسَ نُحَاسٍ وَسَلَّمَهَا لِيَدِ رُؤَسَاءِ السُّعَاةِ الْحَافِظِينَ بَابَ بَيْتِ الْمَلِكِ.
\par 28 وَكَانَ إِذَا دَخَلَ الْمَلِكُ بَيْتَ الرَّبِّ يَحْمِلُهَا السُّعَاةُ، ثُمَّ يُرْجِعُونَهَا إِلَى غُرْفَةِ السُّعَاةِ.
\par 29 وَبَقِيَّةُ أُمُورِ رَحُبْعَامَ وَكُلُّ مَا فَعَلَ مَكْتُوبَةٌ فِي سِفْرِ أَخْبَارِ الأَيَّامِ لِمُلُوكِ يَهُوذَا.
\par 30 وَكَانَتْ حَرْبٌ بَيْنَ رَحُبْعَامَ وَيَرُبْعَامَ كُلَّ الأَيَّامِ.
\par 31 ثُمَّ اضْطَجَعَ رَحُبْعَامُ مَعَ آبَائِهِ، وَدُفِنَ مَعَ آبَائِهِ فِي مَدِينَةِ دَاوُدَ. وَاسْمُ أُمِّهِ نِعْمَةُ الْعَمُّونِيَّةُ. وَمَلَكَ أَبِيَامُ ابْنُهُ عِوَضاً عَنْهُ.

\chapter{15}

\par 1 وَفِي السَّنَةِ الثَّامِنَةِ عَشَرَةَ لِلْمَلِكِ يَرُبْعَامَ بْنِ نَبَاطَ، مَلَكَ أَبِيَامُ عَلَى يَهُوذَا.
\par 2 مَلَكَ ثَلاَثَ سِنِينٍ فِي أُورُشَلِيمَ. وَاسْمُ أُمِّهِ مَعْكَةُ ابْنَةُ أَبْشَالُومَ.
\par 3 وَسَارَ فِي جَمِيعِ خَطَايَا أَبِيهِ الَّتِي عَمِلَهَا قَبْلَهُ، وَلَمْ يَكُنْ قَلْبُهُ كَامِلاً مَعَ الرَّبِّ إِلَهِهِ كَقَلْبِ دَاوُدَ أَبِيهِ.
\par 4 وَلَكِنْ لأَجْلِ دَاوُدَ أَعْطَاهُ الرَّبُّ إِلَهُهُ سِرَاجاً فِي أُورُشَلِيمَ، إِذْ أَقَامَ ابْنَهُ بَعْدَهُ وَثَبَّتَ أُورُشَلِيمَ.
\par 5 لأَنَّ دَاوُدَ عَمِلَ مَا هُوَ مُسْتَقِيمٌ فِي عَيْنَيِ الرَّبِّ وَلَمْ يَحِدْ عَنْ شَيْءٍ مِمَّا أَوْصَاهُ بِهِ كُلَّ أَيَّامِ حَيَاتِهِ، إِلاَّ فِي قَضِيَّةِ أُورِيَّا الْحِثِّيِّ.
\par 6 وَكَانَتْ حَرْبٌ بَيْنَ رَحُبْعَامَ وَيَرُبْعَامَ كُلَّ أَيَّامِ حَيَاتِهِ.
\par 7 وَبَقِيَّةُ أُمُورِ أَبِيَامَ وَكُلُّ مَا عَمِلَ مَكْتُوبَةٌ فِي سِفْرِ أَخْبَارِ الأَيَّامِ لِمُلُوكِ يَهُوذَا. وَكَانَتْ حَرْبٌ بَيْنَ أَبِيَامَ وَيَرُبْعَامَ.
\par 8 ثُمَّ اضْطَجَعَ أَبِيَامُ مَعَ آبَائِهِ، فَدَفَنُوهُ فِي مَدِينَةِ دَاوُدَ، وَمَلَكَ آسَا ابْنُهُ عِوَضاً عَنْهُ.
\par 9 وَفِي السَّنَةِ الْعِشْرِينَ لِيَرُبْعَامَ مَلِكِ إِسْرَائِيلَ مَلَكَ آسَا عَلَى يَهُوذَا.
\par 10 مَلَكَ إِحْدَى وَأَرْبَعِينَ سَنَةً فِي أُورُشَلِيمَ. وَاسْمُ أُمِّهِ مَعْكَةُ ابْنَةُ أَبْشَالُومَ.
\par 11 وَعَمِلَ آسَا مَا هُوَ مُسْتَقِيمٌ فِي عَيْنَيِ الرَّبِّ كَدَاوُدَ أَبِيهِ،
\par 12 وَأَزَالَ الْمَأْبُونِينَ مِنَ الأَرْضِ، وَنَزَعَ جَمِيعَ الأَصْنَامِ الَّتِي عَمِلَهَا آبَاؤُهُ،
\par 13 حَتَّى إِنَّ مَعْكَةَ أُمَّهُ خَلَعَهَا مِنْ أَنْ تَكُونَ مَلِكَةً لأَنَّهَا عَمِلَتْ تِمْثَالاً لِسَارِيَةٍ، وَقَطَعَ آسَا تِمْثَالَهَا وَأَحْرَقَهُ فِي وَادِي قَدْرُونَ.
\par 14 وَأَمَّا الْمُرْتَفَعَاتُ فَلَمْ تُنْزَعْ. إِلاَّ إِنَّ قَلْبَ آسَا كَانَ كَامِلاً مَعَ الرَّبِّ كُلَّ أَيَّامِهِ.
\par 15 وَأَدْخَلَ أَقْدَاسَ أَبِيهِ وَأَقْدَاسَهُ إِلَى بَيْتِ الرَّبِّ مِنَ الْفِضَّةِ وَالذَّهَبِ وَالآنِيَةِ.
\par 16 وَكَانَتْ حَرْبٌ بَيْنَ آسَا وَبَعْشَا مَلِكِ إِسْرَائِيلَ كُلَّ أَيَّامِهِمَا.
\par 17 وَصَعِدَ بَعْشَا مَلِكُ إِسْرَائِيلَ عَلَى يَهُوذَا وَبَنَى الرَّامَةَ لِكَيْ لاَ يَدَعَ أَحَداً يَخْرُجُ أَوْ يَدْخُلُ إِلَى آسَا مَلِكِ يَهُوذَا.
\par 18 وَأَخَذَ آسَا جَمِيعَ الْفِضَّةِ وَالذَّهَبِ الْبَاقِيَةِ فِي خَزَائِنِ بَيْتِ الرَّبِّ وَخَزَائِنِ بَيْتِ الْمَلِكِ وَدَفَعَهَا لِيَدِ عَبِيدِهِ، وَأَرْسَلَهُمُ الْمَلِكُ آسَا إِلَى بَنْهَدَدَ بْنِ طَبْرِيمُونَ بْنِ حَزْيُونَ مَلِكِ أَرَامَ السَّاكِنِ فِي دِمَشْقَ قَائِلاً:
\par 19 [إِنَّ بَيْنِي وَبَيْنَكَ وَبَيْنَ أَبِي وَأَبِيكَ عَهْداً. هُوَذَا قَدْ أَرْسَلْتُ لَكَ هَدِيَّةً مِنْ فِضَّةٍ وَذَهَبٍ، فَتَعَالَ انْقُضْ عَهْدَكَ مَعَ بَعْشَا مَلِكِ إِسْرَائِيلَ فَيَصْعَدَ عَنِّي].
\par 20 فَسَمِعَ بَنْهَدَدُ لِلْمَلِكِ آسَا وَأَرْسَلَ رُؤَسَاءَ الْجُيُوشِ الَّتِي لَهُ عَلَى مُدُنِ إِسْرَائِيلَ، وَضَرَبَ عُيُونَ وَدَانَ وَآبَلَ بَيْتِ مَعْكَةَ وَكُلَّ كِنَّرُوتَ مَعَ كُلِّ أَرْضِ نَفْتَالِي.
\par 21 وَلَمَّا سَمِعَ بَعْشَا كَفَّ عَنْ بِنَاءِ الرَّامَةِ وَأَقَامَ فِي تِرْصَةَ.
\par 22 فَاسْتَدْعَى الْمَلِكُ آسَا كُلَّ يَهُوذَا. لَمْ يَكُنْ بَرِيءٌ. فَحَمَلُوا كُلَّ حِجَارَةِ الرَّامَةِ وَأَخْشَابِهَا الَّتِي بَنَاهَا بَعْشَا، وَبَنَى بِهَا الْمَلِكُ آسَا جَبْعَ بِنْيَامِينَ وَالْمِصْفَاةَ.
\par 23 وَبَقِيَّةُ كُلِّ أُمُورِ آسَا وَكُلُّ جَبَرُوتِهِ وَكُلُّ مَا فَعَلَ وَالْمُدُنِ الَّتِي بَنَاهَا مَكْتُوبَةٌ فِي سِفْرِ أَخْبَارِ الأَيَّامِ لِمُلُوكِ يَهُوذَا. غَيْرَ أَنَّهُ فِي زَمَانِ شَيْخُوخَتِهِ مَرِضَ فِي رِجْلَيْهِ.
\par 24 ثُمَّ اضْطَجَعَ آسَا مَعَ آبَائِهِ، وَدُفِنَ مَعَ آبَائِهِ فِي مَدِينَةِ دَاوُدَ أَبِيهِ، وَمَلَكَ يَهُوشَافَاطُ ابْنُهُ عِوَضاً عَنْهُ.
\par 25 وَمَلَكَ نَادَابُ بْنُ يَرُبْعَامَ عَلَى إِسْرَائِيلَ فِي السَّنَةِ الثَّانِيَةِ لِآسَا مَلِكِ يَهُوذَا، فَمَلَكَ عَلَى إِسْرَائِيلَ سَنَتَيْنِ.
\par 26 وَعَمِلَ الشَّرَّ فِي عَيْنَيِ الرَّبِّ وَسَارَ فِي طَرِيقِ أَبِيهِ وَفِي خَطِيَّتِهِ الَّتِي جَعَلَ بِهَا إِسْرَائِيلَ يُخْطِئُ.
\par 27 وَفَتَنَ عَلَيْهِ بَعْشَا بْنُ أَخِيَّا مِنْ بَيْتِ يَسَّاكَرَ، وَضَرَبَهُ بَعْشَا فِي جِبَّثُونَ الَّتِي لِلْفِلِسْطِينِيِّينَ. وَكَانَ نَادَابُ وَكُلُّ إِسْرَائِيلَ مُحَاصِرِينَ جِبَّثُونَ.
\par 28 وَأَمَاتَهُ بَعْشَا فِي السَّنَةِ الثَّالِثَةِ لِآسَا مَلِكِ يَهُوذَا وَمَلَكَ عِوَضاً عَنْهُ.
\par 29 وَلَمَّا مَلَكَ ضَرَبَ كُلَّ بَيْتِ يَرُبْعَامَ. لَمْ يُبْقِ نَسَمَةً لِيَرُبْعَامَ حَتَّى أَفْنَاهُمْ حَسَبَ كَلاَمِ الرَّبِّ الَّذِي تَكَلَّمَ بِهِ عَنْ يَدِ عَبْدِهِ أَخِيَّا الشِّيلُونِيِّ -
\par 30 لأَجْلِ خَطَايَا يَرُبْعَامَ الَّتِي أَخْطَأَهَا وَالَّتِي جَعَلَ بِهَا إِسْرَائِيلَ يُخْطِئُ بِإِغَاظَتِهِ الَّتِي أَغَاظَ بِهَا الرَّبَّ إِلَهَ إِسْرَائِيلَ.
\par 31 وَبَقِيَّةُ أُمُورِ نَادَابَ وَكُلُّ مَا عَمِلَ مَكْتُوبَةٌ فِي سِفْرِ أَخْبَارِ الأَيَّامِ لِمُلُوكِ إِسْرَائِيلَ.
\par 32 وَكَانَتْ حَرْبٌ بَيْنَ آسَا وَبَعْشَا مَلِكِ إِسْرَائِيلَ كُلَّ أَيَّامِهِمَا.
\par 33 فِي السَّنَةِ الثَّالِثَةِ لِآسَا مَلِكِ يَهُوذَا مَلَكَ بَعْشَا بْنُ أَخِيَّا عَلَى جَمِيعِ إِسْرَائِيلَ فِي تِرْصَةَ أَرْبَعاً وَعِشْرِينَ سَنَةً.
\par 34 وَعَمِلَ الشَّرَّ فِي عَيْنَيِ الرَّبِّ، وَسَارَ فِي طَرِيقِ يَرُبْعَامَ وَفِي خَطِيَّتِهِ الَّتِي جَعَلَ بِهَا إِسْرَائِيلَ يُخْطِئُ.

\chapter{16}

\par 1 وَكَانَ كَلاَمُ الرَّبِّ إِلَى يَاهُو بْنِ حَنَانِي عَلَى بَعْشَا:
\par 2 [مِنْ أَجْلِ أَنِّي قَدْ رَفَعْتُكَ مِنَ التُّرَابِ وَجَعَلْتُكَ رَئِيساً عَلَى شَعْبِي إِسْرَائِيلَ، فَسِرْتَ فِي طَرِيقِ يَرُبْعَامَ وَجَعَلْتَ شَعْبِي إِسْرَائِيلَ يُخْطِئُونَ وَيُغِيظُونَنِي بِخَطَايَاهُمْ،
\par 3 هَئَنَذَا أَنْزِعُ نَسْلَ بَعْشَا وَنَسْلَ بَيْتِهِ، وَأَجْعَلُ بَيْتَكَ كَبَيْتِ يَرُبْعَامَ بْنِ نَبَاطَ.
\par 4 فَمَنْ مَاتَ لِبَعْشَا فِي الْمَدِينَةِ تَأْكُلُهُ الْكِلاَبُ، وَمَنْ مَاتَ لَهُ فِي الْحَقْلِ تَأْكُلُهُ طُيُورُ السَّمَاءِ].
\par 5 وَبَقِيَّةُ أُمُورِ بَعْشَا وَمَا عَمِلَ وَجَبَرُوتُهُ مَكْتُوبَةٌ فِي سِفْرِ أَخْبَارِ الأَيَّامِ لِمُلُوكِ إِسْرَائِيلَ.
\par 6 وَاضْطَجَعَ بَعْشَا مَعَ آبَائِهِ وَدُفِنَ فِي تِرْصَةَ، وَمَلَكَ أَيْلَةُ ابْنُهُ عِوَضاً عَنْهُ.
\par 7 وَأَيْضاً عَنْ يَدِ يَاهُو بْنِ حَنَانِي النَّبِيِّ كَانَ كَلاَمُ الرَّبِّ عَلَى بَعْشَا وَعَلَى بَيْتِهِ، وَعَلَى كُلِّ الشَّرِّ الَّذِي عَمِلَهُ فِي عَيْنَيِ الرَّبِّ بِإِغَاظَتِهِ إِيَّاهُ بِعَمَلِ يَدَيْهِ، وَكَوْنِهِ كَبَيْتِ يَرُبْعَامَ، وَلأَجْلِ قَتْلِهِ إِيَّاهُ.
\par 8 وَفِي السَّنَةِ السَّادِسَةِ وَالْعِشْرِينَ لِآسَا مَلِكِ يَهُوذَا مَلَكَ أَيْلَةُ بْنُ بَعْشَا عَلَى إِسْرَائِيلَ فِي تِرْصَةَ سَنَتَيْنِ.
\par 9 فَفَتَنَ عَلَيْهِ عَبْدُهُ زِمْرِي رَئِيسُ نِصْفِ الْمَرْكَبَاتِ، وَهُوَ فِي تِرْصَةَ يَشْرَبُ وَيَسْكَرُ فِي بَيْتِ أَرْصَا الَّذِي عَلَى الْبَيْتِ فِي تِرْصَةَ.
\par 10 فَدَخَلَ زِمْرِي وَضَرَبَهُ، فَقَتَلَهُ فِي السَّنَةِ السَّابِعَةِ وَالْعِشْرِينَ لِآسَا مَلِكِ يَهُوذَا، وَمَلَكَ عِوَضاً عَنْهُ.
\par 11 وَعِنْدَ تَمَلُّكِهِ وَجُلُوسِهِ عَلَى كُرْسِيِّهِ ضَرَبَ كُلَّ بَيْتِ بَعْشَا. لَمْ يُبْقِ لَهُ ذَكَراً، مَعَ أَوْلِيَائِهِ وَأَصْحَابِهِ.
\par 12 فَأَفْنَى زِمْرِي كُلَّ بَيْتِ بَعْشَا حَسَبَ كَلاَمِ الرَّبِّ الَّذِي تَكَلَّمَ بِهِ عَلَى بَعْشَا عَنْ يَدِ يَاهُو النَّبِيِّ،
\par 13 لأَجْلِ كُلِّ خَطَايَا بَعْشَا، وَخَطَايَا أَيْلَةَ ابْنِهِ الَّتِي أَخْطَئَا بِهَا وَجَعَلاَ إِسْرَائِيلَ يُخْطِئُ لإِغَاظَةِ الرَّبِّ إِلَهِ إِسْرَائِيلَ بِأَبَاطِيلِهِمْ.
\par 14 وَبَقِيَّةُ أُمُورِ أَيْلَةَ وَكُلُّ مَا فَعَلَ، مَكْتُوبَةٌ فِي سِفْرِ أَخْبَارِ الأَيَّامِ لِمُلُوكِ إِسْرَائِيلَ.
\par 15 فِي السَّنَةِ السَّابِعَةِ وَالْعِشْرِينَ لآسَا مَلِكِ يَهُوذَا مَلَكَ زِمْرِي سَبْعَةَ أَيَّامٍ فِي تِرْصَةَ. وَكَانَ الشَّعْبُ نَازِلاً عَلَى جِبَّثُونَ الَّتِي لِلْفِلِسْطِينِيِّينَ.
\par 16 فَسَمِعَ الشَّعْبُ النَّازِلُونَ مَنْ يَقُولُ: [قَدْ فَتَنَ زِمْرِي وَقَتَلَ أَيْضاً الْمَلِكَ]. فَمَلَّكَ كُلُّ إِسْرَائِيلَ عُمْرِيَ رَئِيسَ الْجَيْشِ عَلَى إِسْرَائِيلَ فِي ذَلِكَ الْيَوْمِ فِي الْمَحَلَّةِ.
\par 17 وَصَعِدَ عُمْرِي وَكُلُّ إِسْرَائِيلَ مَعَهُ مِنْ جِبَّثُونَ وَحَاصَرُوا تِرْصَةَ.
\par 18 وَلَمَّا رَأَى زِمْرِي أَنَّ الْمَدِينَةَ قَدْ أُخِذَتْ دَخَلَ إِلَى قَصْرِ بَيْتِ الْمَلِكِ وَأَحْرَقَ عَلَى نَفْسِهِ بَيْتَ الْمَلِكِ بِالنَّارِ، فَمَاتَ
\par 19 مِنْ أَجْلِ خَطَايَاهُ الَّتِي أَخْطَأَ بِهَا بِعَمَلِهِ الشَّرَّ فِي عَيْنَيِ الرَّبِّ وَسَيْرِهِ فِي طَرِيقِ يَرُبْعَامَ، وَمِنْ أَجْلِ خَطِيَّتِهِ الَّتِي عَمِلَ بِجَعْلِهِ إِسْرَائِيلَ يُخْطِئُ.
\par 20 وَبَقِيَّةُ أُمُورِ زِمْرِي وَفِتْنَتُهُ الَّتِي فَتَنَهَا مَكْتُوبَةٌ فِي سِفْرِ أَخْبَارِ الأَيَّامِ لِمُلُوكِ إِسْرَائِيلَ.
\par 21 حِينَئِذٍ انْقَسَمَ شَعْبُ إِسْرَائِيلَ نِصْفَيْنِ، فَنِصْفُ الشَّعْبِ كَانَ وَرَاءَ تِبْنِي بْنِ جِينَةَ لِتَمْلِيكِهِ، وَنِصْفُهُ وَرَاءَ عُمْرِي.
\par 22 وَقَوِيَ الشَّعْبُ الَّذِي وَرَاءَ عُمْرِي عَلَى الشَّعْبِ الَّذِي وَرَاءَ تِبْنِي بْنِ جِينَةَ، فَمَاتَ تِبْنِي وَمَلَكَ عُمْرِي.
\par 23 فِي السَّنَةِ الْوَاحِدَةِ وَالثَّلاَثِينَ لِآسَا مَلِكِ يَهُوذَا مَلَكَ عُمْرِي عَلَى إِسْرَائِيلَ اثْنَتَيْ عَشَرَةَ سَنَةً. مَلَكَ فِي تِرْصَةَ سِتَّ سِنِينَ.
\par 24 وَاشْتَرَى جَبَلَ السَّامِرَةِ مِنْ شَامِرَ بِوَزْنَتَيْنِ مِنَ الْفِضَّةِ، وَبَنَى عَلَى الْجَبَلِ. وَدَعَا اسْمَ الْمَدِينَةِ الَّتِي بَنَاهَا بِاسْمِ شَامِرَ صَاحِبِ الْجَبَلِ [السَّامِرَةَ].
\par 25 وَعَمِلَ عُمْرِي الشَّرَّ فِي عَيْنَيِ الرَّبِّ، وَأَسَاءَ أَكْثَرَ مِنْ جَمِيعِ الَّذِينَ قَبْلَهُ.
\par 26 وَسَارَ فِي جَمِيعِ طَرِيقِ يَرُبْعَامَ بْنِ نَبَاطَ وَفِي خَطِيَّتِهِ الَّتِي جَعَلَ بِهَا إِسْرَائِيلَ يُخْطِئُ، لإِغَاظَةِ الرَّبِّ إِلَهِ إِسْرَائِيلَ بِأَبَاطِيلِهِمْ.
\par 27 وَبَقِيَّةُ أُمُورِ عُمْرِي الَّتِي عَمِلَ وَجَبَرُوتُهُ الَّذِي أَبْدَى مَكْتُوبَةٌ فِي سِفْرِ أَخْبَارِ الأَيَّامِ لِمُلُوكِ إِسْرَائِيلَ.
\par 28 وَاضْطَجَعَ عُمْرِي مَعَ آبَائِهِ وَدُفِنَ فِي السَّامِرَةِ، وَمَلَكَ أَخْآبُ ابْنُهُ عِوَضاً عَنْهُ.
\par 29 وَأَخْآبُ بْنُ عُمْرِي مَلَكَ عَلَى إِسْرَائِيلَ فِي السَّنَةِ الثَّامِنَةِ وَالثَّلاَثِينَ لِآسَا مَلِكِ يَهُوذَا، وَمَلَكَ أَخْآبُ بْنُ عُمْرِي عَلَى إِسْرَائِيلَ فِي السَّامِرَةِ اثْنَتَيْنِ وَعِشْرِينَ سَنَةً.
\par 30 وَعَمِلَ الشَّرَّ فِي عَيْنَيِ الرَّبِّ أَكْثَرَ مِنْ جَمِيعِ الَّذِينَ قَبْلَهُ -
\par 31 وَكَأَنَّهُ كَانَ أَمْراً زَهِيداً سُلُوكُهُ فِي خَطَايَا يَرُبْعَامَ بْنِ نَبَاطَ حَتَّى اتَّخَذَ إِيزَابَلَ ابْنَةَ أَثْبَعَلَ مَلِكِ الصَّيْدُونِيِّينَ امْرَأَةً، وَعَبَدَ الْبَعْلَ وَسَجَدَ لَهُ.
\par 32 وَأَقَامَ مَذْبَحاً لِلْبَعْلِ فِي بَيْتِ الْبَعْلِ الَّذِي بَنَاهُ فِي السَّامِرَةِ.
\par 33 وَعَمِلَ أَخْآبُ سَوَارِيَ، وَزَادَ فِي الْعَمَلِ لإِغَاظَةِ الرَّبِّ إِلَهِ إِسْرَائِيلَ أَكْثَرَ مِنْ جَمِيعِ مُلُوكِ إِسْرَائِيلَ الَّذِينَ كَانُوا قَبْلَهُ.
\par 34 فِي أَيَّامِهِ بَنَى حِيئِيلُ الْبَيْتَئِيلِيُّ أَرِيحَا. بِأَبِيرَامَ بِكْرِهِ وَضَعَ أَسَاسَهَا وَبِسَجُوبَ صَغِيرِهِ نَصَبَ أَبْوَابَهَا، حَسَبَ كَلاَمِ الرَّبِّ الَّذِي تَكَلَّمَ بِهِ عَنْ يَدِ يَشُوعَ بْنِ نُونٍ.

\chapter{17}

\par 1 وَقَالَ إِيلِيَّا التِّشْبِيُّ مِنْ مُسْتَوْطِنِي جِلْعَادَ لأَخْآبَ: [حَيٌّ هُوَ الرَّبُّ إِلَهُ إِسْرَائِيلَ الَّذِي وَقَفْتُ أَمَامَهُ، إِنَّهُ لاَ يَكُونُ طَلٌّ وَلاَ مَطَرٌ فِي هَذِهِ السِّنِينَ إِلاَّ عِنْدَ قَوْلِي].
\par 2 وَكَانَ كَلاَمُ الرَّبِّ لَهُ:
\par 3 [انْطَلِقْ مِنْ هُنَا وَاتَّجِهْ نَحْوَ الْمَشْرِقِ، وَاخْتَبِئْ عِنْدَ نَهْرِ كَرِيثَ الَّذِي هُوَ مُقَابِلُ الأُرْدُنِّ،
\par 4 فَتَشْرَبَ مِنَ النَّهْرِ. وَقَدْ أَمَرْتُ الْغِرْبَانَ أَنْ تَعُولَكَ هُنَاكَ].
\par 5 فَانْطَلَقَ وَعَمِلَ حَسَبَ كَلاَمِ الرَّبِّ وَذَهَبَ فَأَقَامَ عِنْدَ نَهْرِ كَرِيثَ الَّذِي هُوَ مُقَابِلُ الأُرْدُنِّ.
\par 6 وَكَانَتِ الْغِرْبَانُ تَأْتِي إِلَيْهِ بِخُبْزٍ وَلَحْمٍ صَبَاحاً وَبِخُبْزٍ وَلَحْمٍ مَسَاءً، وَكَانَ يَشْرَبُ مِنَ النَّهْرِ.
\par 7 وَكَانَ بَعْدَ مُدَّةٍ مِنَ الزَّمَانِ أَنَّ النَّهْرَ يَبِسَ لأَنَّهُ لَمْ يَكُنْ مَطَرٌ فِي الأَرْضِ.
\par 8 وَكَانَ لَهُ كَلاَمُ الرَّبِّ:
\par 9 [قُمِ اذْهَبْ إِلَى صِرْفَةَ الَّتِي لِصَيْدُونَ وَأَقِمْ هُنَاكَ. هُوَذَا قَدْ أَمَرْتُ هُنَاكَ أَرْمَلَةً أَنْ تَعُولَكَ].
\par 10 فَقَامَ وَذَهَبَ إِلَى صِرْفَةَ. وَجَاءَ إِلَى بَابِ الْمَدِينَةِ، وَإِذَا بِامْرَأَةٍ أَرْمَلَةٍ هُنَاكَ تَقُشُّ عِيدَاناً، فَنَادَاهَا وَقَالَ: [هَاتِي لِي قَلِيلَ مَاءٍ فِي إِنَاءٍ فَأَشْرَبَ].
\par 11 وَفِيمَا هِيَ ذَاهِبَةٌ لِتَأْتِيَ بِهِ نَادَاهَا وَقَالَ: [هَاتِي لِي كِسْرَةَ خُبْزٍ فِي يَدِكِ].
\par 12 فَقَالَتْ: [حَيٌّ هُوَ الرَّبُّ إِلَهُكَ إِنَّهُ لَيْسَتْ عِنْدِي كَعْكَةٌ، وَلَكِنْ مِلْءُ كَفٍّ مِنَ الدَّقِيقِ فِي الْكُوَّارِ، وَقَلِيلٌ مِنَ الزَّيْتِ فِي الْكُوزِ، وَهَئَنَذَا أَقُشُّ عُودَيْنِ لِآتِيَ وَأَعْمَلَهُ لِي وَلاِبْنِي لِنَأْكُلَهُ ثُمَّ نَمُوتُ].
\par 13 فَقَالَ لَهَا إِيلِيَّا: [لاَ تَخَافِي. ادْخُلِي وَاعْمَلِي كَقَوْلِكِ، وَلَكِنِ اعْمَلِي لِي مِنْهَا كَعْكَةً صَغِيرَةً أَوَّلاً وَاخْرُجِي بِهَا إِلَيَّ، ثُمَّ اعْمَلِي لَكِ وَلاِبْنِكِ أَخِيراً.
\par 14 لأَنَّهُ هَكَذَا قَالَ الرَّبُّ إِلَهُ إِسْرَائِيلَ: إِنَّ كُوَّارَ الدَّقِيقِ لاَ يَفْرُغُ، وَكُوزَ الزَّيْتِ لاَ يَنْقُصُ، إِلَى الْيَوْمِ الَّذِي فِيهِ يُعْطِي الرَّبُّ مَطَراً عَلَى وَجْهِ الأَرْضِ].
\par 15 فَذَهَبَتْ وَفَعَلَتْ حَسَبَ قَوْلِ إِيلِيَّا، وَأَكَلَتْ هِيَ وَهُوَ وَبَيْتُهَا أَيَّاماً.
\par 16 كُوَّارُ الدَّقِيقِ لَمْ يَفْرُغْ، وَكُوزُ الزَّيْتِ لَمْ يَنْقُصْ، حَسَبَ قَوْلِ الرَّبِّ الَّذِي تَكَلَّمَ بِهِ عَنْ يَدِ إِيلِيَّا.
\par 17 وَبَعْدَ هَذِهِ الأُمُورِ مَرِضَ ابْنُ الْمَرْأَةِ صَاحِبَةِ الْبَيْتِ وَاشْتَدَّ مَرَضُهُ جِدّاً حَتَّى لَمْ تَبْقَ فِيهِ نَسَمَةٌ.
\par 18 فَقَالَتْ لإِيلِيَّا: [مَا لِي وَلَكَ يَا رَجُلَ اللَّهِ! هَلْ جِئْتَ إِلَيَّ لِتَذْكِيرِ إِثْمِي وَإِمَاتَةِ ابْنِي؟]
\par 19 فَقَالَ لَهَا: [أَعْطِينِي ابْنَكِ]. وَأَخَذَهُ مِنْ حِضْنِهَا وَصَعِدَ بِهِ إِلَى الْعُلِّيَّةِ الَّتِي كَانَ مُقِيماً بِهَا، وَأَضْجَعَهُ عَلَى سَرِيرِهِ
\par 20 وَصَرَخَ إِلَى الرَّبِّ: [أَيُّهَا الرَّبُّ إِلَهِي، أَأَيْضاً إِلَى الأَرْمَلَةِ الَّتِي أَنَا نَازِلٌ عِنْدَهَا قَدْ أَسَأْتَ بِإِمَاتَتِكَ ابْنَهَا؟]
\par 21 فَتَمَدَّدَ عَلَى الْوَلَدِ ثَلاَثَ مَرَّاتٍ، وَصَرَخَ إِلَى الرَّبِّ: [يَا رَبُّ إِلَهِي، لِتَرْجِعْ نَفْسُ هَذَا الْوَلَدِ إِلَى جَوْفِهِ].
\par 22 فَسَمِعَ الرَّبُّ لِصَوْتِ إِيلِيَّا، فَرَجَعَتْ نَفْسُ الْوَلَدِ إِلَى جَوْفِهِ فَعَاشَ.
\par 23 فَأَخَذَ إِيلِيَّا الْوَلَدَ وَنَزَلَ بِهِ مِنَ الْعُلِّيَّةِ إِلَى الْبَيْتِ وَدَفَعَهُ لِأُمِّهِ. وَقَالَ إِيلِيَّا: [انْظُرِي. ابْنُكِ حَيٌّ!]
\par 24 فَقَالَتِ الْمَرْأَةُ لإِيلِيَّا: [هَذَا الْوَقْتَ عَلِمْتُ أَنَّكَ رَجُلُ اللَّهِ، وَأَنَّ كَلاَمَ الرَّبِّ فِي فَمِكَ حَقٌّ].

\chapter{18}

\par 1 وَبَعْدَ أَيَّامٍ كَثِيرَةٍ كَانَ كَلاَمُ الرَّبِّ إِلَى إِيلِيَّا فِي السَّنَةِ الثَّالِثَةِ: [اذْهَبْ وَتَرَاءَ لأَخْآبَ فَأُعْطِيَ مَطَراً عَلَى وَجْهِ الأَرْضِ].
\par 2 فَذَهَبَ إِيلِيَّا لِيَتَرَاءَى لأَخْآبَ. وَكَانَ الْجُوعُ شَدِيداً فِي السَّامِرَةِ،
\par 3 فَدَعَا أَخْآبُ عُوبَدْيَا الَّذِي عَلَى الْبَيْتِ - وَكَانَ عُوبَدْيَا يَخْشَى الرَّبَّ جِدّاً.
\par 4 وَكَانَ حِينَمَا قَطَعَتْ إِيزَابَلُ أَنْبِيَاءَ الرَّبِّ أَنَّ عُوبَدْيَا أَخَذَ مِئَةَ نَبِيٍّ وَخَبَّأَهُمْ خَمْسِينَ رَجُلاً فِي مَغَارَةٍ وَعَالَهُمْ بِخُبْزٍ وَمَاءٍ -
\par 5 وَقَالَ أَخْآبُ لِعُوبَدْيَا: [اذْهَبْ فِي الأَرْضِ إِلَى جَمِيعِ عُيُونِ الْمَاءِ وَإِلَى جَمِيعِ الأَوْدِيَةِ، لَعَلَّنَا نَجِدُ عُشْباً فَنُحْيِيَ الْخَيْلَ وَالْبِغَالَ وَلاَ نُعْدَمَ الْبَهَائِمَ كُلَّهَا].
\par 6 فَقَسَمَا بَيْنَهُمَا الأَرْضَ لِيَعْبُرَا بِهَا. فَذَهَبَ أَخْآبُ فِي طَرِيقٍ وَاحِدٍ وَحْدَهُ، وَذَهَبَ عُوبَدْيَا فِي طَرِيقٍ آخَرَ وَحْدَهُ.
\par 7 وَفِيمَا كَانَ عُوبَدْيَا فِي الطَّرِيقِ إِذَا بِإِيلِيَّا قَدْ لَقِيَهُ. فَعَرَفَهُ وَخَرَّ عَلَى وَجْهِهِ وَقَالَ: [أَأَنْتَ هُوَ سَيِّدِي إِيلِيَّا؟]
\par 8 فَقَالَ لَهُ: [أَنَا هُوَ. اذْهَبْ وَقُلْ لِسَيِّدِكَ: هُوَذَا إِيلِيَّا].
\par 9 فَقَالَ: [مَا هِيَ خَطِيَّتِي حَتَّى إِنَّكَ تَدْفَعُ عَبْدَكَ لِيَدِ أَخْآبَ لِيُمِيتَنِي؟
\par 10 حَيٌّ هُوَ الرَّبُّ إِلَهُكَ إِنَّهُ لاَ تُوجَدُ أُمَّةٌ وَلاَ مَمْلَكَةٌ لَمْ يُرْسِلْ سَيِّدِي إِلَيْهَا لِيُفَتِّشَ عَلَيْكَ، وَكَانُوا يَقُولُونَ: إِنَّهُ لاَ يُوجَدُ. وَكَانَ يَسْتَحْلِفُ الْمَمْلَكَةَ وَالأُمَّةَ أَنَّهُمْ لَمْ يَجِدُوكَ.
\par 11 وَالآنَ أَنْتَ تَقُولُ: اذْهَبْ قُلْ لِسَيِّدِكَ هُوَذَا إِيلِيَّا.
\par 12 وَيَكُونُ إِذَا انْطَلَقْتُ مِنْ عَِنْدِكَ أَنَّ رُوحَ الرَّبِّ يَحْمِلُكَ إِلَى حَيْثُ لاَ أَعْلَمُ. فَإِذَا أَتَيْتُ وَأَخْبَرْتُ أَخْآبَ وَلَمْ يَجِدْكَ فَإِنَّهُ يَقْتُلُنِي. وَأَنَا عَبْدُكَ أَخْشَى الرَّبَّ مُنْذُ صَبَايَ.
\par 13 أَلَمْ يُخْبَرْ سَيِّدِي بِمَا فَعَلْتُ حِينَ قَتَلَتْ إِيزَابَلُ أَنْبِيَاءَ الرَّبِّ، إِذْ خَبَّأْتُ مِنْ أَنْبِيَاءِ الرَّبِّ مِئَةَ رَجُلٍ، خَمْسِينَ خَمْسِينَ رَجُلاً فِي مَغَارَةٍ وَعُلْتُهُمْ بِخُبْزٍ وَمَاءٍ؟
\par 14 وَأَنْتَ الآنَ تَقُولُ: اذْهَبْ قُلْ لِسَيِّدِكَ: هُوَذَا إِيلِيَّا. فَيَقْتُلُنِي].
\par 15 فَقَالَ إِيلِيَّا: [حَيٌّ هُوَ رَبُّ الْجُنُودِ الَّذِي أَنَا وَاقِفٌ أَمَامَهُ، إِنِّي الْيَوْمَ أَتَرَاءَى لَهُ].
\par 16 فَذَهَبَ عُوبَدْيَا لِلِقَاءِ أَخْآبَ وَأَخْبَرَهُ، فَسَارَ أَخْآبُ لِلِقَاءِ إِيلِيَّا.
\par 17 وَلَمَّا رَأَى أَخْآبُ إِيلِيَّا قَالَ لَهُ أَخْآبُ: [أَأَنْتَ هُوَ مُكَدِّرُ إِسْرَائِيلَ؟]
\par 18 فَقَالَ: [لَمْ أُكَدِّرْ إِسْرَائِيلَ، بَلْ أَنْتَ وَبَيْتُ أَبِيكَ بِتَرْكِكُمْ وَصَايَا الرَّبِّ وَبِسَيْرِكَ وَرَاءَ الْبَعْلِيمِ.
\par 19 فَالآنَ أَرْسِلْ وَاجْمَعْ إِلَيَّ كُلَّ إِسْرَائِيلَ إِلَى جَبَلِ الْكَرْمَلِ وَأَنْبِيَاءَ الْبَعْلِ أَرْبَعَ الْمِئَةِ وَالْخَمْسِينَ، وَأَنْبِيَاءَ السَّوَارِي أَرْبَعَ الْمِئَةِ الَّذِينَ يَأْكُلُونَ عَلَى مَائِدَةِ إِيزَابَلَ].
\par 20 فَأَرْسَلَ أَخْآبُ إِلَى جَمِيعِ بَنِي إِسْرَائِيلَ وَجَمَعَ الأَنْبِيَاءَ إِلَى جَبَلِ الْكَرْمَلِ.
\par 21 فَتَقَدَّمَ إِيلِيَّا إِلَى جَمِيعِ الشَّعْبِ وَقَالَ: [حَتَّى مَتَى تَعْرُجُونَ بَيْنَ الْفِرْقَتَيْنِ؟ إِنْ كَانَ الرَّبُّ هُوَ اللَّهَ فَاتَّبِعُوهُ، وَإِنْ كَانَ الْبَعْلُ فَاتَّبِعُوهُ]. فَلَمْ يُجِبْهُ الشَّعْبُ بِكَلِمَةٍ.
\par 22 ثُمَّ قَالَ إِيلِيَّا لِلشَّعْبِ: [أَنَا بَقِيتُ نَبِيّاً لِلرَّبِّ وَحْدِي، وَأَنْبِيَاءُ الْبَعْلِ أَرْبَعُ مِئَةٍ وَخَمْسُونَ رَجُلاً.
\par 23 فَلْيُعْطُونَا ثَوْرَيْنِ، فَيَخْتَارُوا لأَنْفُسِهِمْ ثَوْراً وَاحِداً وَيُقَطِّعُوهُ وَيَضَعُوهُ عَلَى الْحَطَبِ، وَلَكِنْ لاَ يَضَعُوا نَاراً. وَأَنَا أُقَرِّبُ الثَّوْرَ الآخَرَ وَأَجْعَلُهُ عَلَى الْحَطَبِ، وَلَكِنْ لاَ أَضَعُ نَاراً.
\par 24 ثُمَّ تَدْعُونَ بِاسْمِ آلِهَتِكُمْ وَأَنَا أَدْعُو بِاسْمِ الرَّبِّ. وَالإِلَهُ الَّذِي يُجِيبُ بِنَارٍ فَهُوَ اللَّهُ]. فَأَجَابَ جَمِيعُ الشَّعْبِ: [الْكَلاَمُ حَسَنٌ].
\par 25 فَقَالَ إِيلِيَّا لأَنْبِيَاءِ الْبَعْلِ: [اخْتَارُوا لأَنْفُسِكُمْ ثَوْراً وَاحِداً وَقَرِّبُوا أَوَّلاً، لأَنَّكُمْ أَنْتُمُ الأَكْثَرُ، وَادْعُوا بِاسْمِ آلِهَتِكُمْ، وَلَكِنْ لاَ تَضَعُوا نَاراً].
\par 26 فَأَخَذُوا الثَّوْرَ الَّذِي أُعْطِيَ لَهُمْ وَقَرَّبُوهُ، وَدَعُوا بِاسْمِ الْبَعْلِ مِنَ الصَّبَاحِ إِلَى الظُّهْرِ: [يَا بَعْلُ أَجِبْنَا]. فَلَمْ يَكُنْ صَوْتٌ وَلاَ مُجِيبٌ. وَكَانُوا يَرْقُصُونَ حَوْلَ الْمَذْبَحِ الَّذِي عُمِلَ.
\par 27 وَعِنْدَ الظُّهْرِ سَخِرَ بِهِمْ إِيلِيَّا وَقَالَ: [ادْعُوا بِصَوْتٍ عَالٍ لأَنَّهُ إِلَهٌ! لَعَلَّهُ مُسْتَغْرِقٌ أَوْ فِي خَلْوَةٍ أَوْ فِي سَفَرٍ، أَوْ لَعَلَّهُ نَائِمٌ فَيَتَنَبَّهَ!]
\par 28 فَصَرَخُوا بِصَوْتٍ عَالٍ، وَتَقَطَّعُوا حَسَبَ عَادَتِهِمْ بِالسُّيُوفِ وَالرِّمَاحِ حَتَّى سَالَ مِنْهُمُ الدَّمُ.
\par 29 وَلَمَّا جَازَ الظُّهْرُ وَتَنَبَّأُوا إِلَى حِينِ إِصْعَادِ التَّقْدِمَةِ، وَلَمْ يَكُنْ صَوْتٌ وَلاَ مُجِيبٌ وَلاَ مُصْغٍ،
\par 30 قَالَ إِيلِيَّا لِجَمِيعِ الشَّعْبِ: [تَقَدَّمُوا إِلَيَّ]. فَتَقَدَّمَ جَمِيعُ الشَّعْبِ إِلَيْهِ. فَرَمَّمَ مَذْبَحَ الرَّبِّ الْمُنْهَدِمَ.
\par 31 ثُمَّ أَخَذَ إِيلِيَّا اثْنَيْ عَشَرَ حَجَراً، بِعَدَدِ أَسْبَاطِ بَنِي يَعْقُوبَ (الَّذِي كَانَ كَلاَمُ الرَّبِّ إِلَيْهِ: [إِسْرَائِيلَ يَكُونُ اسْمُكَ])
\par 32 وَبَنَى الْحِجَارَةَ مَذْبَحاً بِاسْمِ الرَّبِّ، وَعَمِلَ قَنَاةً حَوْلَ الْمَذْبَحِ تَسَعُ كَيْلَتَيْنِ مِنَ الْبِزْرِ.
\par 33 ثُمَّ رَتَّبَ الْحَطَبَ وَقَطَّعَ الثَّوْرَ وَوَضَعَهُ عَلَى الْحَطَبِ وَقَالَ: [امْلَأُوا أَرْبَعَ جَرَّاتٍ مَاءً وَصُبُّوا عَلَى الْمُحْرَقَةِ وَعَلَى الْحَطَبِ].
\par 34 ثُمَّ قَالَ: [ثَنُّوا] فَثَنَّوْا. وَقَالَ: [ثَلِّثُوا فَثَلَّثُوا.
\par 35 فَجَرَى الْمَاءُ حَوْلَ الْمَذْبَحِ وَامْتَلَأَتِ الْقَنَاةُ أَيْضاً مَاءً.
\par 36 وَكَانَ عِنْدَ إِصْعَادِ التَّقْدِمَةِ أَنَّ إِيلِيَّا النَّبِيَّ تَقَدَّمَ وَقَالَ: [أَيُّهَا الرَّبُّ إِلَهُ إِبْرَاهِيمَ وَإِسْحَاقَ وَإِسْرَائِيلَ، لِيُعْلَمِ الْيَوْمَ أَنَّكَ أَنْتَ اللَّهُ فِي إِسْرَائِيلَ، وَأَنِّي أَنَا عَبْدُكَ، وَبِأَمْرِكَ قَدْ فَعَلْتُ كُلَّ هَذِهِ الأُمُورِ.
\par 37 اسْتَجِبْنِي يَا رَبُّ اسْتَجِبْنِي، لِيَعْلَمَ هَذَا الشَّعْبُ أَنَّكَ أَنْتَ الرَّبُّ الإِلَهُ، وَأَنَّكَ أَنْتَ حَوَّلْتَ قُلُوبَهُمْ رُجُوعاً].
\par 38 فَسَقَطَتْ نَارُ الرَّبِّ وَأَكَلَتِ الْمُحْرَقَةَ وَالْحَطَبَ وَالْحِجَارَةَ وَالتُّرَابَ، وَلَحَسَتِ الْمِيَاهَ الَّتِي فِي الْقَنَاةِ.
\par 39 فَلَمَّا رَأَى جَمِيعُ الشَّعْبِ ذَلِكَ سَقَطُوا عَلَى وُجُوهِهِمْ وَقَالُوا: [الرَّبُّ هُوَ اللَّهُ! الرَّبُّ هُوَ اللَّهُ!].
\par 40 فَقَالَ لَهُمْ إِيلِيَّا: [أَمْسِكُوا أَنْبِيَاءَ الْبَعْلِ وَلاَ يُفْلِتْ مِنْهُمْ رَجُلٌ]. فَأَمْسَكُوهُمْ، فَنَزَلَ بِهِمْ إِيلِيَّا إِلَى نَهْرِ قِيشُونَ وَذَبَحَهُمْ هُنَاكَ.
\par 41 وَقَالَ إِيلِيَّا لأَخْآبَ: [اصْعَدْ كُلْ وَاشْرَبْ، لأَنَّهُ حِسُّ دَوِيِّ مَطَرٍ].
\par 42 فَصَعِدَ أَخْآبُ لِيَأْكُلَ وَيَشْرَبَ، وَأَمَّا إِيلِيَّا فَصَعِدَ إِلَى رَأْسِ الْكَرْمَلِ وَخَرَّ إِلَى الأَرْضِ، وَجَعَلَ وَجْهَهُ بَيْنَ رُكْبَتَيْهِ.
\par 43 وَقَالَ لِغُلاَمِهِ: [اصْعَدْ تَطَلَّعْ نَحْوَ الْبَحْرِ]. فَصَعِدَ وَتَطَلَّعَ وَقَالَ: [لَيْسَ شَيْءٌ]. فَقَالَ: [ارْجِعْ] سَبْعَ مَرَّاتٍ.
\par 44 وَفِي الْمَرَّةِ السَّابِعَةِ قَالَ: [هُوَذَا غَيْمَةٌ صَغِيرَةٌ قَدْرُ كَفِّ إِنْسَانٍ صَاعِدَةٌ مِنَ الْبَحْرِ]. فَقَالَ: [اصْعَدْ قُلْ لأَخْآبَ: اشْدُدْ وَانْزِلْ لِئَلاَّ يَمْنَعَكَ الْمَطَرُ].
\par 45 وَكَانَ مِنْ هُنَا إِلَى هُنَا أَنَّ السَّمَاءَ اسْوَدَّتْ مِنَ الْغَيْمِ وَالرِّيحِ، وَكَانَ مَطَرٌ عَظِيمٌ. فَرَكِبَ أَخْآبُ وَمَضَى إِلَى يَزْرَعِيلَ.
\par 46 وَكَانَتْ يَدُ الرَّبِّ عَلَى إِيلِيَّا، فَشَدَّ حَقْوَيْهِ وَرَكَضَ أَمَامَ أَخْآبَ حَتَّى تَجِيءَ إِلَى يَزْرَعِيلَ.

\chapter{19}

\par 1 وَأَخْبَرَ أَخْآبُ إِيزَابَلَ بِكُلِّ مَا عَمِلَ إِيلِيَّا، وَكَيْفَ أَنَّهُ قَتَلَ جَمِيعَ الأَنْبِيَاءِ بِالسَّيْفِ.
\par 2 فَأَرْسَلَتْ إِيزَابَلُ رَسُولاً إِلَى إِيلِيَّا تَقُولُ: [هَكَذَا تَفْعَلُ الآلِهَةُ وَهَكَذَا تَزِيدُ إِنْ لَمْ أَجْعَلْ نَفْسَكَ كَنَفْسِ وَاحِدٍ مِنْهُمْ فِي نَحْوِ هَذَا الْوَقْتِ غَداً].
\par 3 فَلَمَّا رَأَى ذَلِكَ قَامَ وَمَضَى لأَجْلِ نَفْسِهِ، وَأَتَى إِلَى بِئْرِ سَبْعٍ الَّتِي لِيَهُوذَا وَتَرَكَ غُلاَمَهُ هُنَاكَ.
\par 4 ثُمَّ سَارَ فِي الْبَرِّيَّةِ مَسِيرَةَ يَوْمٍ، حَتَّى أَتَى وَجَلَسَ تَحْتَ رَتَمَةٍ وَطَلَبَ الْمَوْتَ لِنَفْسِهِ، وَقَالَ: [قَدْ كَفَى الآنَ يَا رَبُّ! خُذْ نَفْسِي لأَنِّي لَسْتُ خَيْراً مِنْ آبَائِي!]
\par 5 وَاضْطَجَعَ وَنَامَ تَحْتَ الرَّتَمَةِ. وَإِذَا بِمَلاَكٍ قَدْ مَسَّهُ وَقَالَ: [قُمْ وَكُلْ].
\par 6 فَتَطَلَّعَ وَإِذَا كَعْكَةُ رَضْفٍ وَكُوزُ مَاءٍ عِنْدَ رَأْسِهِ، فَأَكَلَ وَشَرِبَ ثُمَّ رَجَعَ فَاضْطَجَعَ.
\par 7 ثُمَّ عَادَ مَلاَكُ الرَّبِّ ثَانِيَةً فَمَسَّهُ وَقَالَ: [قُمْ وَكُلْ لأَنَّ الْمَسَافَةَ كَثِيرَةٌ عَلَيْكَ].
\par 8 فَقَامَ وَأَكَلَ وَشَرِبَ، وَسَارَ بِقُوَّةِ تِلْكَ الأَكْلَةِ أَرْبَعِينَ نَهَاراً وَأَرْبَعِينَ لَيْلَةً إِلَى جَبَلِ اللَّهِ حُورِيبَ،
\par 9 وَدَخَلَ هُنَاكَ الْمَغَارَةَ وَبَاتَ فِيهَا. وَكَانَ كَلاَمُ الرَّبِّ إِلَيْهِ: [مَا لَكَ هَهُنَا يَا إِيلِيَّا؟]
\par 10 فَقَالَ: [قَدْ غِرْتُ غَيْرَةً لِلرَّبِّ إِلَهِ الْجُنُودِ، لأَنَّ بَنِي إِسْرَائِيلَ قَدْ تَرَكُوا عَهْدَكَ وَنَقَضُوا مَذَابِحَكَ وَقَتَلُوا أَنْبِيَاءَكَ بِالسَّيْفِ، فَبَقِيتُ أَنَا وَحْدِي. وَهُمْ يَطْلُبُونَ نَفْسِي لِيَأْخُذُوهَا].
\par 11 فَقَالَ: [اخْرُجْ وَقِفْ عَلَى الْجَبَلِ أَمَامَ الرَّبِّ]. وَإِذَا بِالرَّبِّ عَابِرٌ وَرِيحٌ عَظِيمَةٌ وَشَدِيدَةٌ قَدْ شَقَّتِ الْجِبَالَ وَكَسَّرَتِ الصُّخُورَ أَمَامَ الرَّبِّ، وَلَمْ يَكُنِ الرَّبُّ فِي الرِّيحِ. وَبَعْدَ الرِّيحِ زَلْزَلَةٌ، وَلَمْ يَكُنِ الرَّبُّ فِي الزَّلْزَلَةِ.
\par 12 وَبَعْدَ الزَّلْزَلَةِ نَارٌ، وَلَمْ يَكُنِ الرَّبُّ فِي النَّارِ. وَبَعْدَ النَّارِ صَوْتٌ مُنْخَفِضٌ خَفِيفٌ.
\par 13 فَلَمَّا سَمِعَ إِيلِيَّا لَفَّ وَجْهَهُ بِرِدَائِهِ وَخَرَجَ وَوَقَفَ فِي بَابِ الْمَغَارَةِ، وَإِذَا بِصَوْتٍ إِلَيْهِ يَقُولُ: [مَا لَكَ هَهُنَا يَا إِيلِيَّا؟]
\par 14 فَقَالَ: [غِرْتُ غَيْرَةً لِلرَّبِّ إِلَهِ الْجُنُودِ لأَنَّ بَنِي إِسْرَائِيلَ قَدْ تَرَكُوا عَهْدَكَ وَنَقَضُوا مَذَابِحَكَ وَقَتَلُوا أَنْبِيَاءَكَ بِالسَّيْفِ، فَبَقِيتُ أَنَا وَحْدِي، وَهُمْ يَطْلُبُونَ نَفْسِي لِيَأْخُذُوهَا].
\par 15 فَقَالَ لَهُ الرَّبُّ: [اذْهَبْ رَاجِعاً فِي طَرِيقِكَ إِلَى بَرِّيَّةِ دِمِشْقَ، وَادْخُلْ وَامْسَحْ حَزَائِيلَ مَلِكاً عَلَى أَرَامَ،
\par 16 وَامْسَحْ يَاهُوَ بْنَ نِمْشِي مَلِكاً عَلَى إِسْرَائِيلَ، وَامْسَحْ أَلِيشَعَ بْنَ شَافَاطَ مِنْ آبَلَ مَحُولَةَ نَبِيّاً عِوَضاً عَنْكَ.
\par 17 فَالَّذِي يَنْجُو مِنْ سَيْفِ حَزَائِيلَ يَقْتُلُهُ يَاهُو، وَالَّذِي يَنْجُو مِنْ سَيْفِ يَاهُو يَقْتُلُهُ أَلِيشَعُ.
\par 18 وَقَدْ أَبْقَيْتُ فِي إِسْرَائِيلَ سَبْعَةَ آلاَفٍ، كُلَّ الرُّكَبِ الَّتِي لَمْ تَجْثُ لِلْبَعْلِ وَكُلَّ فَمٍ لَمْ يُقَبِّلْهُ].
\par 19 فَذَهَبَ مِنْ هُنَاكَ وَوَجَدَ أَلِيشَعَ بْنَ شَافَاطَ يَحْرُثُ، وَاثْنَا عَشَرَ زَوْجَ بَقَرٍ قُدَّامَهُ وَهُوَ مَعَ الثَّانِي عَشَرَ. فَمَرَّ إِيلِيَّا بِهِ وَطَرَحَ رِدَاءَهُ عَلَيْهِ.
\par 20 فَتَرَكَ الْبَقَرَ وَرَكَضَ وَرَاءَ إِيلِيَّا وَقَالَ: [دَعْنِي أُقَبِّلْ أَبِي وَأُمِّي وَأَسِيرَ وَرَاءَكَ]. فَقَالَ لَهُ: [اذْهَبْ رَاجِعاً، لأَنِّي مَاذَا فَعَلْتُ لَكَ؟]
\par 21 فَرَجَعَ مِنْ وَرَائِهِ وَأَخَذَ زَوْجَ بَقَرٍ وَذَبَحَهُمَا، وَسَلَقَ اللَّحْمَ بِأَدَوَاتِ الْبَقَرِ وَأَعْطَى الشَّعْبَ فَأَكَلُوا. ثُمَّ قَامَ وَمَضَى وَرَاءَ إِيلِيَّا وَكَانَ يَخْدِمُهُ.

\chapter{20}

\par 1 وَجَمَعَ بَنْهَدَدُ مَلِكُ أَرَامَ كُلَّ جَيْشِهِ، وَاثْنَيْنِ وَثَلاَثِينَ مَلِكاً مَعَهُ، وَخَيْلاً وَمَرْكَبَاتٍ وَصَعِدَ وَحَاصَرَ السَّامِرَةَ وَحَارَبَهَا.
\par 2 وَأَرْسَلَ رُسُلاً إِلَى أَخْآبَ مَلِكِ إِسْرَائِيلَ إِلَى الْمَدِينَةِ وَقَالَ لَهُ: [هَكَذَا يَقُولُ بَنْهَدَدُ:
\par 3 لِي فِضَّتُكَ وَذَهَبُكَ، وَلِي نِسَاؤُكَ وَبَنُوكَ الْحِسَانُ].
\par 4 فَأَجَابَ مَلِكُ إِسْرَائِيلَ: [حَسَبَ قَوْلِكَ يَا سَيِّدِي الْمَلِكَ، أَنَا وَجَمِيعُ مَا لِي لَكَ].
\par 5 فَرَجَعَ الرُّسُلُ وَقَالُوا: [هَكَذَا تَكَلَّمَ بَنْهَدَدُ: إِنِّي قَدْ أَرْسَلْتُ إِلَيْكَ أَنَّ فِضَّتَكَ وَذَهَبَكَ وَنِسَاءَكَ وَبَنِيكَ تُعْطِينِي إِيَّاهُمْ.
\par 6 فَإِنِّي فِي نَحْوِ هَذَا الْوَقْتِ غَداً أُرْسِلُ عَبِيدِي إِلَيْكَ فَيُفَتِّشُونَ بَيْتَكَ وَبُيُوتَ عَبِيدِكَ، وَكُلَّ مَا هُوَ شَهِيٌّ فِي عَيْنَيْكَ يَضَعُونَهُ فِي أَيْدِيهِمْ وَيَأْخُذُونَهُ].
\par 7 فَدَعَا مَلِكُ إِسْرَائِيلَ جَمِيعَ شُيُوخِ الأَرْضِ وَقَالَ: [اعْلَمُوا وَانْظُرُوا أَنَّ هَذَا يَطْلُبُ الشَّرَّ، لأَنَّهُ أَرْسَلَ إِلَيَّ بِطَلَبِ نِسَائِي وَبَنِيَّ وَفِضَّتِي وَذَهَبِي وَلَمْ أَمْنَعْهَا عَنْهُ].
\par 8 فَقَالَ لَهُ كُلُّ الشُّيُوخِ وَكُلُّ الشَّعْبِ: [لاَ تَسْمَعْ لَهُ وَلاَ تَقْبَلْ].
\par 9 فَقَالَ لِرُسُلِ بَنْهَدَدَ: [قُولُوا لِسَيِّدِي الْمَلِكِ إِنَّ كُلَّ مَا أَرْسَلْتَ فِيهِ إِلَى عَبْدِكَ أَوَّلاً أَفْعَلُهُ. وَأَمَّا هَذَا الأَمْرُ فَلاَ أَسْتَطِيعُ أَنْ أَفْعَلَهُ]. فَرَجَعَ الرُّسُلُ وَرَدُّوا عَلَيْهِ الْجَوَابَ.
\par 10 فَأَرْسَلَ إِلَيْهِ بَنْهَدَدُ وَقَالَ: [هَكَذَا تَفْعَلُ بِي الآلِهَةُ وَهَكَذَا تَزِيدُنِي إِنْ كَانَ تُرَابُ السَّامِرَةِ يَكْفِي قَبَضَاتٍ لِكُلِّ الشَّعْبِ الَّذِي يَتْبَعُنِي].
\par 11 فَأَجَابَ مَلِكُ إِسْرَائِيلَ: [قُولُوا: لاَ يَفْتَخِرَنَّ مَنْ يَشُدُّ كَمَنْ يَحِلُّ].
\par 12 فَلَمَّا سَمِعَ هَذَا الْكَلاَمَ وَهُوَ يَشْرَبُ مَعَ الْمُلُوكِ فِي الْخِيَامِ قَالَ لِعَبِيدِهِ: [اصْطَفُّوا] فَاصْطَفُّوا عَلَى الْمَدِينَةِ.
\par 13 وَإِذَا بِنَبِيٍّ تَقَدَّمَ إِلَى أَخْآبَ مَلِكِ إِسْرَائِيلَ وَقَالَ: [هَكَذَا قَالَ الرَّبُّ: هَلْ رَأَيْتَ كُلَّ هَذَا الْجُمْهُورِ الْعَظِيمِ؟ هَئَنَذَا أَدْفَعُهُ لِيَدِكَ الْيَوْمَ فَتَعْلَمُ أَنِّي أَنَا الرَّبُّ].
\par 14 فَقَالَ أَخْآبُ: [بِمَنْ؟] فَقَالَ: [هَكَذَا قَالَ الرَّبُّ: بِغِلْمَانِ رُؤَسَاءِ الْمُقَاطَعَاتِ]. فَقَالَ: [مَنْ يَبْتَدِئُ بِالْحَرْبِ؟] فَقَالَ: [أَنْتَ].
\par 15 فَعَدَّ غِلْمَانَ رُؤَسَاءِ الْمُقَاطَعَاتِ فَبَلَغُوا مِئَتَيْنِ وَاثْنَيْنِ وَثَلاَثِينَ. وَعَدَّ بَعْدَهُمْ كُلَّ الشَّعْبِ، كُلَّ بَنِي إِسْرَائِيلَ، سَبْعَةَ آلاَفٍ.
\par 16 وَخَرَجُوا عِنْدَ الظُّهْرِ وَبَنْهَدَدُ يَشْرَبُ وَيَسْكَرُ فِي الْخِيَامِ هُوَ وَالْمُلُوكُ الاِثْنَانِ وَالثَّلاَثُونَ الَّذِينَ سَاعَدُوهُ.
\par 17 فَخَرَجَ غِلْمَانُ رُؤَسَاءِ الْمُقَاطَعَاتِ أَوَّلاً. وَأَرْسَلَ بَنْهَدَدُ فَأَخْبَرُوهُ: [قَدْ خَرَجَ رِجَالٌ مِنَ السَّامِرَةِ].
\par 18 فَقَالَ: [إِنْ كَانُوا قَدْ خَرَجُوا لِلسَّلاَمِ فَأَمْسِكُوهُمْ أَحْيَاءً، وَإِنْ كَانُوا قَدْ خَرَجُوا لِلْقِتَالِ فَأَمْسِكُوهُمْ أَحْيَاءً].
\par 19 فَخَرَجَ غِلْمَانُ رُؤَسَاءِ الْمُقَاطَعَاتِ هَؤُلاَءِ مِنَ الْمَدِينَةِ هُمْ وَالْجَيْشُ الَّذِي وَرَاءَهُمْ
\par 20 وَضَرَبَ كُلُّ رَجُلٍ رَجُلَهُ، فَهَرَبَ الأَرَامِيُّونَ وَطَارَدَهُمْ إِسْرَائِيلُ، وَنَجَا بَنْهَدَدُ مَلِكُ أَرَامَ عَلَى فَرَسٍ مَعَ الْفُرْسَانِ.
\par 21 وَخَرَجَ مَلِكُ إِسْرَائِيلَ فَضَرَبَ الْخَيْلَ وَالْمَرْكَبَاتِ، وَضَرَبَ أَرَامَ ضَرْبَةً عَظِيمَةً.
\par 22 فَتَقَدَّمَ النَّبِيُّ إِلَى مَلِكِ إِسْرَائِيلَ وَقَالَ لَهُ: [اذْهَبْ تَشَدَّدْ، وَاعْلَمْ وَانْظُرْ مَا تَفْعَلُ. لأَنَّهُ عِنْدَ تَمَامِ السَّنَةِ يَصْعَدُ عَلَيْكَ مَلِكُ أَرَامَ].
\par 23 وَأَمَّا عَبِيدُ مَلِكِ أَرَامَ فَقَالُوا لَهُ: [إِنَّ آلِهَتَهُمْ آلِهَةُ جِبَالٍ، لِذَلِكَ قَوُوا عَلَيْنَا. وَلَكِنْ إِذَا حَارَبْنَاهُمْ فِي السَّهْلِ فَإِنَّنَا نَقْوَى عَلَيْهِمْ.
\par 24 وَافْعَلْ هَذَا الأَمْرَ: اعْزِلِ الْمُلُوكَ كُلَّ وَاحِدٍ مِنْ مَكَانِهِ، وَضَعْ قُوَّاداً مَكَانَهُمْ.
\par 25 وَأَحْصِ لِنَفْسِكَ جَيْشاً كَالْجَيْشِ الَّذِي سَقَطَ مِنْكَ فَرَساً بِفَرَسٍ وَمَرْكَبَةً بِمَرْكَبَةٍ، فَنُحَارِبَهُمْ فِي السَّهْلِ وَنَقْوَى عَلَيْهِمْ]. فَسَمِعَ لِقَوْلِهِمْ وَفَعَلَ كَذَلِكَ.
\par 26 وَعِنْدَ تَمَامِ السَّنَةِ عَدَّ بَنْهَدَدُ الأَرَامِيِّينَ وَصَعِدَ إِلَى أَفِيقَ لِيُحَارِبَ إِسْرَائِيلَ.
\par 27 وَأُحْصِيَ بَنُو إِسْرَائِيلَ وَتَزَوَّدُوا وَسَارُوا لِلِقَائِهِمْ. فَنَزَلَ بَنُو إِسْرَائِيلَ مُقَابِلَهُمْ نَظِيرَ قَطِيعَيْنِ صَغِيرَيْنِ مِنَ الْمِعْزَى. وَأَمَّا الأَرَامِيُّونَ فَمَلَأُوا الأَرْضَ.
\par 28 فَتَقَدَّمَ رَجُلُ اللَّهِ وَقَالَ لِمَلِكِ إِسْرَائِيلَ: [هَكَذَا قَالَ الرَّبُّ: مِنْ أَجْلِ أَنَّ الأَرَامِيِّينَ قَالُوا إِنَّ الرَّبَّ إِلَهُ جِبَالٍ وَلَيْسَ إِلَهَ أَوْدِيَةٍ، أَدْفَعُ كُلَّ هَذَا الْجُمْهُورِ الْعَظِيمِ لِيَدِكَ، فَتَعْلَمُونَ أَنِّي أَنَا الرَّبُّ].
\par 29 فَنَزَلَ هَؤُلاَءِ مُقَابَِلَ أُولَئِكَ سَبْعَةَ أَيَّامٍ. وَفِي الْيَوْمِ السَّابِعِ اشْتَبَكَتِ الْحَرْبُ، فَضَرَبَ بَنُو إِسْرَائِيلَ مِنَ الأَرَامِيِّينَ مِئَةَ أَلْفِ رَاجِلٍ فِي يَوْمٍ وَاحِدٍ.
\par 30 وَهَرَبَ الْبَاقُونَ إِلَى أَفِيقَ إِلَى الْمَدِينَةِ، وَسَقَطَ السُّورُ عَلَى السَّبْعَةِ وَالْعِشْرِينَ أَلْفَ رَجُلٍ الْبَاقِينَ. وَهَرَبَ بَنْهَدَدُ وَدَخَلَ الْمَدِينَةَ مِنْ مِخْدَعٍ إِلَى مِخْدَعٍ.
\par 31 فَقَالَ لَهُ عَبِيدُهُ: [إِنَّنَا قَدْ سَمِعْنَا أَنَّ مُلُوكَ بَيْتِ إِسْرَائِيلَ هُمْ مُلُوكٌ حَلِيمُونَ، فَلْنَضَعْ مُسُوحاً عَلَى أَحْقَائِنَا وَحِبَالاً عَلَى رُؤُوسِنَا وَنَخْرُجُ إِلَى مَلِكِ إِسْرَائِيلَ لَعَلَّهُ يُحْيِي نَفْسَكَ].
\par 32 فَشَدُّوا مُسُوحاً عَلَى أَحْقَائِهِمْ وَحِبَالاً عَلَى رُؤُوسِهِمْ وَأَتُوا إِلَى مَلِكِ إِسْرَائِيلَ وَقَالُوا: [يَقُولُ عَبْدُكَ بَنْهَدَدُ: لِتَحْيَ نَفْسِي]. فَقَالَ: [أَهُوَ حَيٌّ بَعْدُ؟ هُوَ أَخِي].
\par 33 فَتَفَاءَلَ الرِّجَالُ وَأَسْرَعُوا وَلَجُّوا هَلْ هُوَ مِنْهُ. وَقَالُوا: [أَخُوكَ بَنْهَدَدُ]. فَقَالَ: [ادْخُلُوا خُذُوهُ] فَخَرَجَ إِلَيْهِ بَنْهَدَدُ فَأَصْعَدَهُ إِلَى الْمَرْكَبَةِ.
\par 34 وَقَالَ لَهُ: [إِنِّي أَرُدُّ الْمُدُنَ الَّتِي أَخَذَهَا أَبِي مِنْ أَبِيكَ، وَتَجْعَلُ لِنَفْسِكَ أَسْوَاقاً فِي دِمَشْقَ كَمَا جَعَلَ أَبِي فِي السَّامِرَةِ]. فَقَالَ: [وَأَنَا أُطْلِقُكَ بِهَذَا الْعَهْدِ]. فَقَطَعَ لَهُ عَهْداً وَأَطْلَقَهُ.
\par 35 وَإِنَّ رَجُلاً مِنْ بَنِي الأَنْبِيَاءِ قَالَ لِصَاحِبِهِ: [عَنْ أَمْرِ الرَّبِّ اضْرِبْنِي]. فَأَبَى الرَّجُلُ أَنْ يَضْرِبَهُ.
\par 36 فَقَالَ لَهُ: [مِنْ أَجْلِ أَنَّكَ لَمْ تَسْمَعْ لِقَوْلِ الرَّبِّ فَحِينَمَا تَذْهَبُ مِنْ عِنْدِي يَقْتُلُكَ أَسَدٌ]. وَلَمَّا ذَهَبَ مِنْ عِنْدِهِ لَقِيَهُ أَسَدٌ وَقَتَلَهُ.
\par 37 ثُمَّ صَادَفَ رَجُلاً آخَرَ فَقَالَ: [اضْرِبْنِي]. فَضَرَبَهُ الرَّجُلُ ضَرْبَةً فَجَرَحَهُ.
\par 38 فَذَهَبَ النَّبِيُّ وَانْتَظَرَ الْمَلِكَ عَلَى الطَّرِيقِ، وَتَنَكَّرَ بِعِصَابَةٍ عَلَى عَيْنَيْهِ.
\par 39 وَلَمَّا عَبَرَ الْمَلِكُ نَادَى الْمَلِكَ: [خَرَجَ عَبْدُكَ إِلَى وَسَطِ الْقِتَالِ، وَإِذَا بِرَجُلٍ مَالَ وَأَتَى إِلَيَّ بِرَجُلٍ وَقَالَ: احْفَظْ هَذَا الرَّجُلَ. وَإِنْ فُقِدَ تَكُونُ نَفْسُكَ بَدَلَ نَفْسِهِ، أَوْ تَدْفَعُ وَزْنَةً مِنَ الْفِضَّةِ.
\par 40 وَفِيمَا عَبْدُكَ مُشْتَغِلٌ هُنَا وَهُنَاكَ إِذَا هُوَ مَفْقُودٌ]. فَقَالَ لَهُ مَلِكُ إِسْرَائِيلَ: [هَكَذَا حُكْمُكَ. أَنْتَ قَضَيْتَ].
\par 41 فَبَادَرَ وَرَفَعَ الْعِصَابَةَ عَنْ عَيْنَيْهِ فَعَرَفَهُ مَلِكُ إِسْرَائِيلَ أَنَّهُ مِنَ الأَنْبِيَاءِ.
\par 42 فَقَالَ لَهُ: [هَكَذَا قَالَ الرَّبُّ: لأَنَّكَ أَفْلَتَّ مِنْ يَدِكَ رَجُلاً قَدْ حَرَّمْتُهُ، تَكُونُ نَفْسُكَ بَدَلَ نَفْسِهِ، وَشَعْبُكَ بَدَلَ شَعْبِهِ].
\par 43 فَمَضَى مَلِكُ إِسْرَائِيلَ إِلَى بَيْتِهِ مُكْتَئِباً مَغْمُوماً وَجَاءَ إِلَى السَّامِرَةِ.

\chapter{21}

\par 1 وَحَدَثَ بَعْدَ هَذِهِ الأُمُورِ أَنَّهُ كَانَ لِنَابُوتَ الْيَزْرَعِيلِيِّ كَرْمٌ فِي يَزْرَعِيلَ بِجَانِبِ قَصْرِ أَخْآبَ مَلِكِ السَّامِرَةِ.
\par 2 فَقَالَ أَخْآبُ لِنَابُوتَ: [أَعْطِنِي كَرْمَكَ فَيَكُونَ لِي بُسْتَانَ بُقُولٍ لأَنَّهُ قَرِيبٌ بِجَانِبِ بَيْتِي، فَأُعْطِيَكَ عِوَضَهُ كَرْماً أَحْسَنَ مِنْهُ. أَوْ إِذَا حَسُنَ فِي عَيْنَيْكَ أَعْطَيْتُكَ ثَمَنَهُ فِضَّةً].
\par 3 فَقَالَ نَابُوتُ لأَخْآبَ: [حَاشَا لِي مِنْ قِبَلِ الرَّبِّ أَنْ أُعْطِيَكَ مِيرَاثَ آبَائِي].
\par 4 فَدَخَلَ أَخْآبُ بَيْتَهُ مُكْتَئِباً مَغْمُوماً مِنْ أَجْلِ قَوْلِ نَابُوتَ الْيَزْرَعِيلِيِّ: [لاَ أُعْطِيكَ مِيرَاثَ آبَائِي]. وَاضْطَجَعَ عَلَى سَرِيرِهِ وَحَوَّلَ وَجْهَهُ وَلَمْ يَأْكُلْ خُبْزاً.
\par 5 فَدَخَلَتْ إِلَيْهِ إِيزَابَلُ امْرَأَتُهُ وَقَالَتْ لَهُ: [لِمَاذَا رُوحُكَ مُكْتَئِبَةٌ وَلاَ تَأْكُلُ خُبْزاً؟]
\par 6 فَقَالَ لَهَا: [لأَنِّي قُلْتُ لِنَابُوتَ الْيَزْرَعِيلِيِّ: أَعْطِنِي كَرْمَكَ بِفِضَّةٍ وَإِذَا شِئْتَ أَعْطَيْتُكَ كَرْماً عِوَضَهُ فَقَالَ: لاَ أُعْطِيكَ كَرْمِي].
\par 7 فَقَالَتْ لَهُ إِيزَابَلُ: [أَأَنْتَ الآنَ تَحْكُمُ عَلَى إِسْرَائِيلَ! قُمْ كُلْ خُبْزاً وَلْيَطِبْ قَلْبُكَ. أَنَا أُعْطِيكَ كَرْمَ نَابُوتَ الْيَزْرَعِيلِيِّ].
\par 8 ثُمَّ كَتَبَتْ رَسَائِلَ بِاسْمِ أَخْآبَ وَخَتَمَتْهَا بِخَاتِمِهِ، وَأَرْسَلَتِ الرَّسَائِلَ إِلَى الشُّيُوخِ وَالأَشْرَافِ الَّذِينَ فِي مَدِينَتِهِ السَّاكِنِينَ مَعَ نَابُوتَ.
\par 9 وَكَتَبَتْ فِي الرَّسَائِلِ تَقُولُ: [نَادُوا بِصَوْمٍ وَأَجْلِسُوا نَابُوتَ فِي رَأْسِ الشَّعْبِ.
\par 10 وَأَجْلِسُوا رَجُلَيْنِ مِنْ بَنِي بَلِيَّعَالَ تُجَاهَهُ لِيَشْهَدَا قَائِلَيْنِ: قَدْ جَدَّفْتَ عَلَى اللَّهِ وَعَلَى الْمَلِكِ. ثُمَّ أَخْرِجُوهُ وَارْجُمُوهُ فَيَمُوتَ].
\par 11 فَفَعَلَ رِجَالُ مَدِينَتِهِ الشُّيُوخُ وَالأَشْرَافُ السَّاكِنُونَ فِي مَدِينَتِهِ كَمَا أَرْسَلَتْ إِلَيْهِمْ إِيزَابَلُ، كَمَا هُوَ مَكْتُوبٌ فِي الرَّسَائِلِ الَّتِي أَرْسَلَتْهَا إِلَيْهِمْ.
\par 12 فَنَادُوا بِصَوْمٍ وَأَجْلَسُوا نَابُوتَ فِي رَأْسِ الشَّعْبِ.
\par 13 وَأَتَى رَجُلاَنِ مِنْ بَنِي بَلِيَّعَالَ وَجَلَسَا تُجَاهَهُ، وَشَهِدَا عَلَى نَابُوتَ أَمَامَ الشَّعْبِ: [قَدْ جَدَّفَ نَابُوتُ عَلَى اللَّهِ وَعَلَى الْمَلِكِ]. فَأَخْرَجُوهُ خَارِجَ الْمَدِينَةِ وَرَجَمُوهُ بِحِجَارَةٍ فَمَاتَ.
\par 14 وَأَرْسَلُوا إِلَى إِيزَابَلَ يَقُولُونَ: [قَدْ رُجِمَ نَابُوتُ وَمَاتَ].
\par 15 وَلَمَّا سَمِعَتْ إِيزَابَلُ أَنَّ نَابُوتَ قَدْ رُجِمَ وَمَاتَ، قَالَتْ لأَخْآبَ: [قُمْ رِثْ كَرْمَ نَابُوتَ الْيَزْرَعِيلِيِّ الَّذِي أَبَى أَنْ يُعْطِيَكَ إِيَّاهُ بِفِضَّةٍ، لأَنَّ نَابُوتَ لَيْسَ حَيّاً بَلْ هُوَ مَيْتٌ].
\par 16 وَلَمَّا سَمِعَ أَخْآبُ أَنَّ نَابُوتَ قَدْ مَاتَ قَامَ لِيَنْزِلَ إِلَى كَرْمِ نَابُوتَ الْيَزْرَعِيلِيِّ لِيَرِثَهُ.
\par 17 فَكَانَ كَلاَمُ الرَّبِّ إِلَى إِيلِيَّا التِّشْبِيِّ:
\par 18 [قُمِ انْزِلْ لِلِقَاءِ أَخْآبَ مَلِكِ إِسْرَائِيلَ الَّذِي فِي السَّامِرَةِ. هُوَذَا هُوَ فِي كَرْمِ نَابُوتَ الَّذِي نَزَلَ إِلَيْهِ لِيَرِثَهُ.
\par 19 وَقُلْ لَهُ: هَكَذَا قَالَ الرَّبُّ: هَلْ قَتَلْتَ وَوَرِثْتَ أَيْضاً؟ فِي الْمَكَانِ الَّذِي لَحَسَتْ فِيهِ الْكِلاَبُ دَمَ نَابُوتَ تَلْحَسُ الْكِلاَبُ دَمَكَ أَنْتَ أَيْضاً].
\par 20 فَقَالَ أَخْآبُ لإِيلِيَّا: [هَلْ وَجَدْتَنِي يَا عَدُوِّي؟] فَقَالَ: [قَدْ وَجَدْتُكَ لأَنَّكَ قَدْ بِعْتَ نَفْسَكَ لِعَمَلِ الشَّرِّ فِي عَيْنَيِ الرَّبِّ.
\par 21 هَئَنَذَا أَجْلِبُ عَلَيْكَ شَرّاً، وَأُبِيدُ نَسْلَكَ، وَأَقْطَعُ لأَخْآبَ كُلَّ ذَكَرٍ وَمَحْجُوزٍ وَمُطْلَقٍ فِي إِسْرَائِيلَ.
\par 22 وَأَجْعَلُ بَيْتَكَ كَبَيْتِ يَرُبْعَامَ بْنِ نَبَاطَ، وَكَبَيْتِ بَعْشَا بْنِ أَخِيَّا، لأَجْلِ الإِغَاظَةِ الَّتِي أَغَظْتَنِي، وَلِجَعْلِكَ إِسْرَائِيلَ يُخْطِئُ].
\par 23 وَقَالَ الرَّبُّ عَنْ إِيزَابَلَ أَيْضاً: [إِنَّ الْكِلاَبَ تَأْكُلُ إِيزَابَلَ عِنْدَ مِتْرَسَةِ يَزْرَعِيلَ.
\par 24 مَنْ مَاتَ لأَخْآبَ فِي الْمَدِينَةِ تَأْكُلُهُ الْكِلاَبُ، وَمَنْ مَاتَ فِي الْحَقْلِ تَأْكُلُهُ طُيُورُ السَّمَاءِ].
\par 25 وَلَمْ يَكُنْ كَأَخْآبَ الَّذِي بَاعَ نَفْسَهُ لِعَمَلِ الشَّرِّ فِي عَيْنَيِ الرَّبِّ، الَّذِي أَغْوَتْهُ إِيزَابَلُ امْرَأَتُهُ.
\par 26 وَرَجِسَ جِدّاً بِذِهَابِهِ وَرَاءَ الأَصْنَامِ حَسَبَ كُلِّ مَا فَعَلَ الأَمُورِيُّونَ الَّذِينَ طَرَدَهُمُ الرَّبُّ مِنْ أَمَامِ بَنِي إِسْرَائِيلَ.
\par 27 وَلَمَّا سَمِعَ أَخْآبُ هَذَا الْكَلاَمَ شَقَّ ثِيَابَهُ وَجَعَلَ مِسْحاً عَلَى جَسَدِهِ وَصَامَ وَاضْطَجَعَ بِالْمِسْحِ وَمَشَى بِسُكُوتٍ.
\par 28 فَكَانَ كَلاَمُ الرَّبِّ إِلَى إِيلِيَّا التِّشْبِيِّ:
\par 29 [هَلْ رَأَيْتَ كَيْفَ اتَّضَعَ أَخْآبُ أَمَامِي؟ فَمِنْ أَجْلِ أَنَّهُ قَدِ اتَّضَعَ أَمَامِي لاَ أَجْلِبُ الشَّرَّ فِي أَيَّامِهِ، بَلْ فِي أَيَّامِ ابْنِهِ أَجْلِبُ الشَّرَّ عَلَى بَيْتِهِ].

\chapter{22}

\par 1 وَأَقَامُوا ثَلاَثَ سِنِينٍَ بِدُونِ حَرْبٍ بَيْنَ أَرَامَ وَإِسْرَائِيلَ.
\par 2 وَفِي السَّنَةِ الثَّالِثَةِ نَزَلَ يَهُوشَافَاطُ مَلِكُ يَهُوذَا إِلَى مَلِكِ إِسْرَائِيلَ.
\par 3 فَقَالَ مَلِكُ إِسْرَائِيلَ لِعَبِيدِهِ: [أَتَعْلَمُونَ أَنَّ رَامُوتَ جِلْعَادَ لَنَا وَنَحْنُ سَاكِتُونَ عَنْ أَخْذِهَا مِنْ يَدِ مَلِكِ أَرَامَ؟]
\par 4 وَقَالَ لِيَهُوشَافَاطَ: [أَتَذْهَبُ مَعِي لِلْحَرْبِ إِلَى رَامُوتَ جِلْعَادَ؟] فَقَالَ يَهُوشَافَاطُ لِمَلِكِ إِسْرَائِيلَ: [مَثَلِي مَثَلُكَ. شَعْبِي كَشَعْبِكَ وَخَيْلِي كَخَيْلِكَ].
\par 5 ثُمَّ قَالَ يَهُوشَافَاطُ لِمَلِكِ إِسْرَائِيلَ: [اسْأَلِ الْيَوْمَ عَنْ كَلاَمِ الرَّبِّ].
\par 6 فَجَمَعَ مَلِكُ إِسْرَائِيلَ الأَنْبِيَاءَ، نَحْوَ أَرْبَعِ مِئَةِ رَجُلٍ وَسَأَلَهُمْ: [أَأَذْهَبُ إِلَى رَامُوتَ جِلْعَادَ لِلْقِتَالِ أَمْ أَمْتَنِعُ؟] فَقَالُوا: [اصْعَدْ فَيَدْفَعَهَا السَّيِّدُ لِيَدِ الْمَلِكِ].
\par 7 فَسَأَلَ يَهُوشَافَاطُ: [أَمَا يُوجَدُ هُنَا بَعْدُ نَبِيٌّ لِلرَّبِّ فَنَسْأَلَ مِنْهُ؟]
\par 8 فَقَالَ مَلِكُ إِسْرَائِيلَ لِيَهُوشَافَاطَ: [يُوجَدُ بَعْدُ رَجُلٌ وَاحِدٌ لِسُؤَالِ الرَّبِّ بِهِ، وَلَكِنِّي أُبْغِضُهُ لأَنَّهُ لاَ يَتَنَبَّأُ عَلَيَّ خَيْراً بَلْ شَرّاً، وَهُوَ مِيخَا بْنُ يَمْلَةَ]. فَقَالَ يَهُوشَافَاطُ: [لاَ يَقُلِ الْمَلِكُ هَكَذَا].
\par 9 فَدَعَا مَلِكُ إِسْرَائِيلَ خَصِيّاً وَقَالَ: [أَسْرِعْ إِلَيَّ بِمِيخَا بْنِ يَمْلَةَ].
\par 10 وَكَانَ مَلِكُ إِسْرَائِيلَ وَيَهُوشَافَاطُ مَلِكُ يَهُوذَا جَالِسَيْنِ كُلُّ وَاحِدٍ عَلَى كُرْسِيِّهِ، لاَبِسَيْنِ ثِيَابَهُمَا فِي سَاحَةٍ عِنْدَ مَدْخَلِ بَابِ السَّامِرَةِ، وَجَمِيعُ الأَنْبِيَاءِ يَتَنَبَّأُونَ أَمَامَهُمَا.
\par 11 وَعَمِلَ صِدْقِيَّا بْنُ كَنْعَنَةَ لِنَفْسِهِ قَرْنَيْ حَدِيدٍ وَقَالَ: [هَكَذَا قَالَ الرَّبُّ: بِهَذِهِ تَنْطَحُ الأَرَامِيِّينَ حَتَّى يَفْنُوا].
\par 12 وَتَنَبَّأَ جَمِيعُ الأَنْبِيَاءِ قَائِلِينَ: [اصْعَدْ إِلَى رَامُوتَ جِلْعَادَ وَأَفْلِحْ، فَيَدْفَعَهَا الرَّبُّ لِيَدِ الْمَلِكِ].
\par 13 وَأَمَّا الرَّسُولُ الَّذِي ذَهَبَ لِيَدْعُوَ مِيخَا فَقَالَ لَهُ: [هُوَذَا كَلاَمُ جَمِيعِ الأَنْبِيَاءِ بِفَمٍ وَاحِدٍ خَيْرٌ لِلْمَلِكِ، فَلْيَكُنْ كَلاَمُكَ مِثْلَ كَلاَمِ وَاحِدٍ مِنْهُمْ، وَتَكَلَّمْ بِخَيْرٍ].
\par 14 فَقَالَ مِيخَا: [حَيٌّ هُوَ الرَّبُّ إِنَّ مَا يَقُولُهُ لِيَ الرَّبُّ بِهِ أَتَكَلَّمُ].
\par 15 وَلَمَّا أَتَى إِلَى الْمَلِكِ سَأَلَهُ الْمَلِكُ: [يَا مِيخَا، أَنَصْعَدُ إِلَى رَامُوتَ جِلْعَادَ لِلْقِتَالِ أَمْ نَمْتَنِعُ؟] فَقَالَ لَهُ: [اصْعَدْ وَأَفْلِحْ فَيَدْفَعَهَا الرَّبُّ لِيَدِ الْمَلِكِ].
\par 16 فَقَالَ لَهُ الْمَلِكُ: [كَمْ مَرَّةٍ اسْتَحْلَفْتُكَ أَنْ لاَ تَقُولَ لِي إِلاَّ الْحَقَّ بِاسْمِ الرَّبِّ].
\par 17 فَقَالَ: [رَأَيْتُ كُلَّ إِسْرَائِيلَ مُشَتَّتِينَ عَلَى الْجِبَالِ كَخِرَافٍ لاَ رَاعِيَ لَهَا. فَقَالَ الرَّبُّ: [لَيْسَ لِهَؤُلاَءِ أَصْحَابٌ، فَلْيَرْجِعُوا كُلُّ وَاحِدٍ إِلَى بَيْتِهِ بِسَلاَمٍ].
\par 18 فَقَالَ مَلِكُ إِسْرَائِيلَ لِيَهُوشَافَاطَ: [أَمَا قُلْتُ لَكَ إِنَّهُ لاَ يَتَنَبَّأُ عَلَيَّ خَيْراً بَلْ شَرّاً؟]
\par 19 وَقَالَ: [فَاسْمَعْ إِذاً كَلاَمَ الرَّبِّ: قَدْ رَأَيْتُ الرَّبَّ جَالِساً عَلَى كُرْسِيِّهِ، وَكُلُّ جُنْدِ السَّمَاءِ وُقُوفٌ لَدَيْهِ عَنْ يَمِينِهِ وَعَنْ يَسَارِهِ.
\par 20 فَقَالَ الرَّبُّ: مَنْ يُغْوِي أَخْآبَ فَيَصْعَدَ وَيَسْقُطَ فِي رَامُوتَ جِلْعَادَ؟ فَقَالَ هَذَا هَكَذَا وَقَالَ ذَاكَ هَكَذَا.
\par 21 ثُمَّ خَرَجَ الرُّوحُ وَوَقَفَ أَمَامَ الرَّبِّ وَقَالَ: أَنَا أُغْوِيهِ. وَسَأَلَهُ الرَّبُّ: بِمَاذَا؟
\par 22 فَقَالَ: أَخْرُجُ وَأَكُونُ رُوحَ كَذِبٍ فِي أَفْوَاهِ جَمِيعِ أَنْبِيَائِهِ. فَقَالَ: إِنَّكَ تُغْوِيهِ وَتَقْتَدِرُ. فَاخْرُجْ وَافْعَلْ هَكَذَا.
\par 23 وَالآنَ هُوَذَا قَدْ جَعَلَ الرَّبُّ رُوحَ كَذِبٍ فِي أَفْوَاهِ جَمِيعِ أَنْبِيَائِكَ هَؤُلاَءِ، وَالرَّبُّ تَكَلَّمَ عَلَيْكَ بِشَرٍّ].
\par 24 فَتَقَدَّمَ صِدْقِيَّا بْنُ كَنْعَنَةَ وَضَرَبَ مِيخَا عَلَى الْفَكِّ وَقَالَ: [مِنْ أَيْنَ عَبَرَ رُوحُ الرَّبِّ مِنِّي لِيُكَلِّمَكَ؟]
\par 25 فَقَالَ مِيخَا: [إِنَّكَ سَتَرَى فِي ذَلِكَ الْيَوْمِ الَّذِي تَدْخُلُ فِيهِ مِنْ مِخْدَعٍ إِلَى مِخْدَعٍ لِتَخْتَبِئَ].
\par 26 فَقَالَ مَلِكُ إِسْرَائِيلَ: [خُذْ مِيخَا وَرُدَّهُ إِلَى آمُونَ رَئِيسِ الْمَدِينَةِ وَإِلَى يُوآشَ ابْنِ الْمَلِكِ،
\par 27 وَقُلْ هَكَذَا قَالَ الْمَلِكُ: ضَعُوا هَذَا فِي السِّجْنِ، وَأَطْعِمُوهُ خُبْزَ الضِّيقِ وَمَاءَ الضِّيقِ حَتَّى آتِيَ بِسَلاَمٍ].
\par 28 فَقَالَ مِيخَا: [إِنْ رَجَعْتَ بِسَلاَمٍ فَلَمْ يَتَكَلَّمِ الرَّبُّ بِي]. وَقَالَ: [اسْمَعُوا أَيُّهَا الشَّعْبُ أَجْمَعُونَ].
\par 29 فَصَعِدَ مَلِكُ إِسْرَائِيلَ وَيَهُوشَافَاطُ مَلِكُ يَهُوذَا إِلَى رَامُوتَ جِلْعَادَ.
\par 30 فَقَالَ مَلِكُ إِسْرَائِيلَ لِيَهُوشَافَاطَ: [إِنِّي أَتَنَكَّرُ وَأَدْخُلُ الْحَرْبَ، وَأَمَّا أَنْتَ فَالْبَسْ ثِيَابَكَ]. فَتَنَكَّرَ مَلِكُ إِسْرَائِيلَ وَدَخَلَ الْحَرْبَ.
\par 31 وَأَمَرَ مَلِكُ أَرَامَ رُؤَسَاءَ الْمَرْكَبَاتِ الَّتِي لَهُ، الاِثْنَيْنِ وَالثَّلاَثِينَ، وَقَالَ: [لاَ تُحَارِبُوا صَغِيراً وَلاَ كَبِيراً إِلاَّ مَلِكَ إِسْرَائِيلَ وَحْدَهُ].
\par 32 فَلَمَّا رَأَى رُؤَسَاءُ الْمَرْكَبَاتِ يَهُوشَافَاطَ: [قَالُوا إِنَّهُ مَلِكُ إِسْرَائِيلَ] فَمَالُوا عَلَيْهِ لِيُقَاتِلُوهُ، فَصَرَخَ يَهُوشَافَاطُ.
\par 33 فَلَمَّا رَأَى رُؤَسَاءُ الْمَرْكَبَاتِ أَنَّهُ لَيْسَ مَلِكَ إِسْرَائِيلَ رَجَعُوا عَنْهُ.
\par 34 وَإِنَّ رَجُلاً نَزَعَ فِي قَوْسِهِ غَيْرَ مُتَعَمِّدٍ وَضَرَبَ مَلِكَ إِسْرَائِيلَ بَيْنَ أَوْصَالِ الدِّرْعِ. فَقَالَ لِمُدِيرِ مَرْكَبَتِهِ: [رُدَّ يَدَكَ وَأَخْرِجْنِي مِنَ الْجَيْشِ لأَنِّي قَدْ جُرِحْتُ].
\par 35 وَاشْتَدَّ الْقِتَالُ فِي ذَلِكَ الْيَوْمِ، وَأُوقِفَ الْمَلِكُ فِي مَرْكَبَتِهِ مُقَابِلَ أَرَامَ وَمَاتَ عِنْدَ الْمَسَاءِ، وَجَرَى دَمُ الْجُرْحِ إِلَى حِضْنِ الْمَرْكَبَةِ.
\par 36 وَعَبَرَ النِّدَاءُ فِي الْجُنْدِ عِنْدَ غُرُوبِ الشَّمْسِ قَائِلاً: [كُلُّ رَجُلٍ إِلَى مَدِينَتِهِ، وَكُلُّ رَجُلٍ إِلَى أَرْضِهِ].
\par 37 فَمَاتَ الْمَلِكُ وَأُدْخِلَ السَّامِرَةَ فَدَفَنُوا الْمَلِكَ فِي السَّامِرَةِ.
\par 38 وَغُسِلَتِ الْمَرْكَبَةُ فِي بِرْكَةِ السَّامِرَةِ فَلَحَسَتِ الْكِلاَبُ دَمَهُ. وَغَسَلُوا سِلاَحَهُ. حَسَبَ كَلاَمِ الرَّبِّ الَّذِي تَكَلَّمَ بِهِ.
\par 39 وَبَقِيَّةُ أُمُورِ أَخْآبَ وَكُلُّ مَا فَعَلَ، وَبَيْتُ الْعَاجِ الَّذِي بَنَاهُ وَكُلُّ الْمُدُنِ الَّتِي بَنَاهَا مَكْتُوبَةٌ فِي سِفْرِ أَخْبَارِ الأَيَّامِ لِمُلُوكِ إِسْرَائِيلَ.
\par 40 فَاضْطَجَعَ أَخْآبُ مَعَ آبَائِهِ، وَمَلَكَ أَخَزْيَا ابْنُهُ عِوَضاً عَنْهُ.
\par 41 وَمَلَكَ يَهُوشَافَاطُ بْنُ آسَا عَلَى يَهُوذَا فِي السَّنَةِ الرَّابِعَةِ لأَخْآبَ مَلِكِ إِسْرَائِيلَ.
\par 42 وَكَانَ يَهُوشَافَاطُ ابْنَ خَمْسٍ وَثَلاَثِينَ سَنَةً حِينَ مَلَكَ، وَمَلَكَ خَمْساً وَعِشْرِينَ سَنَةً فِي أُورُشَلِيمَ، وَاسْمُ أُمِّهِ عَزُوبَةُ بِنْتُ شَلْحِي.
\par 43 وَسَارَ فِي كُلِّ طَرِيقِ آسَا أَبِيهِ. لَمْ يَحِدْ عَنْهَا، إِذْ عَمِلَ الْمُسْتَقِيمَ فِي عَيْنَيِ الرَّبِّ. إِلاَّ أَنَّ الْمُرْتَفَعَاتِ لَمْ تَنْتَزِعْ، بَلْ كَانَ الشَّعْبُ لاَ يَزَالُ يَذْبَحُ وَيُوقِدُ عَلَى الْمُرْتَفَعَاتِ.
\par 44 وَصَالَحَ يَهُوشَافَاطُ مَلِكَ إِسْرَائِيلَ.
\par 45 وَبَقِيَّةُ أُمُورِ يَهُوشَافَاطَ وَجَبَرُوتُهُ الَّذِي أَظْهَرَهُ، وَكَيْفَ حَارَبَ مَكْتُوبَةٌ فِي سِفْرِ أَخْبَارِ الأَيَّامِ لِمُلُوكِ يَهُوذَا.
\par 46 وَبَقِيَّةُ الْمَأْبُونِينَ الَّذِينَ بَقُوا فِي أَيَّامِ آسَا أَبِيهِ أَبَادَهُمْ مِنَ الأَرْضِ.
\par 47 وَلَمْ يَكُنْ فِي أَدُومَ مَلِكٌ. مَلَكَ وَكِيلٌ.
\par 48 وَعَمِلَ يَهُوشَافَاطُ سُفُنَ تَرْشِيشَ لِتَذْهَبَ إِلَى أُوفِيرَ لأَجْلِ الذَّهَبِ فَلَمْ تَذْهَبْ، لأَنَّ السُّفُنَ تَكَسَّرَتْ فِي عِصْيُونَ جَابِرَ.
\par 49 حِينَئِذٍ قَالَ أَخَزْيَا بْنُ أَخْآبَ لِيَهُوشَافَاطَ: [لِيَذهَبْ عَبِيدِي مَعَ عَبِيدِكَ فِي السُّفُنِ]. فَلَمْ يَشَأْ يَهُوشَافَاطُ.
\par 50 وَاضْطَجَعَ يَهُوشَافَاطُ مَعَ آبَائِهِ وَدُفِنَ مَعَ آبَائِهِ فِي مَدِينَةِ دَاوُدَ أَبِيهِ، فَمَلَكَ يَهُورَامُ ابْنُهُ عِوَضاً عَنْهُ.
\par 51 وَمَلَكَ أَخَزْيَا بْنُ أَخْآبَ عَلَى إِسْرَائِيلَ فِي السَّامِرَةِ فِي السَّنَةِ السَّابِعَةَ عَشَرَةَ لِيَهُوشَافَاطَ مَلِكِ يَهُوذَا. مَلَكَ عَلَى إِسْرَائِيلَ سَنَتَيْنِ.
\par 52 وَعَمِلَ الشَّرَّ فِي عَيْنَيِ الرَّبِّ، وَسَارَ فِي طَرِيقِ أَبِيهِ وَطَرِيقِ أُمِّهِ، وَطَرِيقِ يَرُبْعَامَ بْنِ نَبَاطَ الَّذِي جَعَلَ إِسْرَائِيلَ يُخْطِئُ
\par 53 وَعَبَدَ الْبَعْلَ وَسَجَدَ لَهُ وَأَغَاظَ الرَّبَّ إِلَهَ إِسْرَائِيلَ حَسَبَ كُلِّ مَا فَعَلَ أَبُوهُ.


\end{document}