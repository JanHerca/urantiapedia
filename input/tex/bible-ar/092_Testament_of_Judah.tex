\begin{document}

\title{عهد يهوذا}

\chapter{1}

\par \textit{يهوذا، الابن الرابع ليعقوب وليا. إنه العملاق، الرياضي، المحارب؛ يروي أعمالًا بطولية. يركض بسرعة كبيرة لدرجة أنه يستطيع أن يتفوق على الظبية.}

\par 1 نسخة أقوال يهوذا، ما قاله لبنيه قبل وفاته

\par 2 فاجتمعوا وجاءوا إليه، فقال لهم: اسمعوا يا أبنائي يهوذا أبيكم

\par 3 كنتُ الابن الرابع لأبي يعقوب، وسمّتني ليئة أمي يهوذا، قائلةً: أحمد الرب لأنه أعطاني ابنًا رابعًا أيضًا

\par 4 كنتُ سريع البديهة في شبابي، مطيعًا لأبي في كل شيء.

\par 5 وأكرمت أمي وأخت أمي.

\par 6 وحدث لما صرت رجلاً أن أبي باركني قائلاً: ستكون ملكًا، ناجحًا في كل شيء

\par 7 وأظهر لي الرب نعمة في جميع أعمالي، في الحقل وفي البيت

\par 8 أعلم أنني سبقت غزالًا، وأمسكته، وأعددت اللحم لأبي، وقد أكله

\par 9 وكنتُ أتقنُ الظباء في المطاردة، وأُدركُ كلَّ ما كان في السهول

\par 10 فرس برية تفوقت عليها، وأمسكت بها وروّضتها.

\par 11 لقد قتلت أسدًا وانتزعت جديًا من فمه.

\par 12 أمسكت دبًا من مخلبه وألقيته أسفل الجرف، فسُحِق

\par 13 سبقتُ الخنزيرَ البريَّ في الجري، وأمسكت به وأنا أركض، فمزقته إربًا إربًا

\par 14 قفز نمر في الخليل على كلبي، فأمسكته من ذيله، وقذفته على الصخور، فانكسر إلى نصفين

\par 15 وجدت ثورًا بريًا يرعى في الحقول، فأمسكته من قرنيه، وحركته حوله وصعقته، ثم ألقيته عني وقتلته

\par 16 ولما جاء ملكا الكنعانيين مُغْمِدين، مُتَسَلِّحينَ بِسُلُحٍ لِمَواجهةِ قُطْعِنَا، وَجَمْعٍ غَزِيرٍ مَعَهُمَا، انْدَهَمْتُ وَحْدِي مَلِكَ حَاصُورَ، وَضَرَبْتُهُ عَلَى الْجُنُودِ وَجَرْثَتَهُ، فَقَتَلْتُهُ

\par 17 والآخر، ملك تفوح، وهو جالس على فرسه، قتلته، وهكذا شتتت كل شعبه

\par 18 وجدتُ عخور الملك، رجلاً ضخم القامة، يرمي الرماح أمامه وخلفه وهو جالس على ظهر حصانه، فأخذتُ حجرًا وزنه ستون رطلاً، ورميته فضربتُ حصانه فقتلته

\par 19 وقاتلتُ هذا الآخرَ ساعتين، وشقفتُ درعه إلى نصفين، وقطعتُ قدميه، وقتلته

\par 20 وبينما كنت أخلع درعه، إذا بتسعة رجال من رفاقه بدأوا يتقاتلون معي،

\par 21 ولففت ثوبي على يدي، وقذفتهم بالحجارة، فقتلت منهم أربعة، وهرب الباقون

\par 22 وقتل يعقوب أبي بعلشث، ملك جميع الملوك، جبارًا في القوة، طوله اثنا عشر ذراعًا

\par 23 فسقط عليهم الرعب، فكفوا عن حربنا.

\par 24 ولذلك كان والدي خالياً من القلق في الحروب عندما كنت مع إخوتي.

\par 25 فإنه رأى في رؤيا عني أن ملاك القوة يتبعني في كل مكان، لكي لا أُهزم.

\par 26 وفي الجنوب جاءت علينا حرب أعظم من تلك التي في شكيم، فانضممت إلى إخوتي في صف القتال، وطاردت ألف رجل، فقتلت منهم مئتي رجل وأربعة ملوك

\par 27 وصعدت على السور، وقتلت أربعة رجال أقوياء.

\par 28 فأخذنا حاصور، وأخذنا كل الغنيمة.

\par 29 وفي اليوم التالي انطلقنا إلى أريتان، وهي مدينة قوية ومحاطة بأسوار ومنيعة، تهددنا بالموت

\par 30 ولكنني أنا وجاد تقدمنا ​​من شرقي المدينة، ورأوبين ولاوي من غربها

\par 31 والذين كانوا على السور ظنوا أننا وحدنا، فانجذبوا إلينا

\par 32 وهكذا تسلق إخوتي السور من كلا الجانبين سرًا بالأوتاد، ودخلوا المدينة، بينما لم يعلم الرجال بذلك

\par 33 وأخذناه بحد السيف.

\par 34 وأما الذين لجأوا إلى البرج فأضرمنا النار في البرج فأخذناه معهم.

\par 35 ولما كنا منطلقين، استولى رجال تفوح على غنائمنا، ولما رأينا ذلك حاربناهم

\par 36 فقتلناهم جميعًا واستردينا غنائمنا.

\par 37 ولما كنت عند مياه كزيبا، جاء رجال يوبل للقتال.

\par 38 فحاربناهم وهزمناهم، وضربنا حلفاءهم من شيلوه، ولم نترك لهم سلطانًا ليأتوا إلينا

\par 39 وجاء رجال ماكير إلينا في اليوم الخامس ليستولوا على غنائمنا، فهاجمناهم وتغلبنا عليهم في قتال شرس، إذ كان بينهم جيش من الجبابرة، فقتلناهم قبل أن يصعدوا الصعود

\par 40 ولما وصلنا إلى مدينتهم، دحرجت نساؤهم علينا حجارة من حافة التل الذي كانت المدينة قائمة عليه

\par 41 وكنت أنا وشمعون وراء المدينة، واستولينا على المرتفعات، ودمرنا هذه المدينة أيضًا

\par 42 وفي الغد أُخبِرنا أن ملك مدينة جاعش قادم لملاقاتنا بجيش عظيم

\par 43 لذلك، تظاهرتُ أنا ودان بأننا أموريون، ودخلنا مدينتهم كحلفاء

\par 44 وفي ظلمة الليل جاء إخوتنا وفتحنا لهم الأبواب، ودمرنا جميع الرجال وممتلكاتهم، وغنمنا كل ما كان لهم، وهدمنا أسوارهم الثلاثة

\par 45 واقتربنا من ثمنة، حيث كانت كل ممتلكات الملوك المعادين

\par 46 ثم لما أهانوني، غضبت، واندفعت نحوهم إلى القمة؛ وظلوا يقذفونني بالحجارة والسهام

\par 47 ولولا أن أخي دان ساعدني، لقتلوني.

\par 48 فأتيناهم بغضب فهربوا أجمعين، ومروا في طريق آخر وقاتلوا أبي، فصالحهم.

\par 49 ولم نؤذِهم، وأصبحوا لنا خاضعين، ورددنا إليهم غنائمهم

\par 50 وبنيت ثامنة، وبنى أبي بابائيل.

\par 51 كنت في العشرين من عمري عندما اندلعت هذه الحرب. وكان الكنعانيون يخافون مني ومن إخوتي

\par 52 وكان لي ماشية كثيرة، وكان لي رئيس رعاة إرام العدلامي

\par 53 ولما ذهبت إليه رأيت بارسابا ملك عدلام، فكلمنا وأقام لنا وليمة، ولما سخنت أعطاني ابنته بثشوع زوجة

\par 54 فولدت لي عيرًا وأونان وشيلة، وضرب الرب اثنين منهم، وعاش شيلة، وأنتم بنوه

\chapter{2}

\par \textit{يصف يهوذا بعض الاكتشافات الأثرية، وهي مدينة ذات أسوار من حديد وبوابات من نحاس. ويقابل مغامرة.}

\par 1 وأقام أبي بسلام مع أخيه عيسو، وأبناؤه عندنا ثماني عشرة سنة، بعد أن قدمنا ​​من بلاد ما بين النهرين، من بلاد لابان

\par 2 ولما انقضت ثماني عشرة سنة، في السنة الأربعين من حياتي، جاء إلينا عيسو، أخو أبي، بشعب عظيم وقوي

\par 3 وضرب يعقوب عيسو بسهم، فرفع جريحًا إلى جبل سعير، وفيما هو ذاهب مات في أنونيرم

\par 4 وسعينا وراء بني عيسو.

\par 5 وكانت لهم مدينة بأسوار من حديد وأبواب من نحاس، ولم نستطع أن ندخلها، فخيمنا حولها وحاصرناها

\par 6 ولما لم يفتحوا لنا لمدة عشرين يومًا، نصبت سلمًا على مرأى من الجميع، وصعدت بترسي على رأسي، وصمدتُ أمام هجوم الحجارة، التي يزيد وزنها عن ثلاث مواهب، وقتلت أربعة من رجالهم الأبطال

\par 7 وقتل رأوبين وجاد ستة آخرين.

\par 8 ثم طلبوا منا شروط الصلح، وبعد أن تشاوروا مع أبينا، أخذناهم علينا جزية

\par 9 وأعطونا خمسمائة كر من القمح، وخمسمائة بث من الزيت، وخمسمائة مكيال من الخمر، حتى المجاعة حين نزلنا إلى مصر

\par 10 وبعد هذه الأمور، اتخذ ابني عير زوجةً ثامار، من بلاد ما بين النهرين، ابنة آرام

\par 11 وكان عير شريرًا، فاحتاج إلى ثامار، لأنها لم تكن من أرض كنعان

\par 12 وفي الليلة الثالثة ضربه ملاك الرب.

\par 13 ولم يكن يعرفها حسب مكر أمه الشرير، لأنه لم يرد أن يكون له منها أولاد.

\par 14 في أيام العرس أعطيتها أونان زوجة، وهو أيضًا في الشر لم يعرفها، مع أنه قضى معها سنة

\par 15 وعندما هددته دخل عليها، لكنه أراق البذرة على الأرض حسب أمر أمه، ومات هو أيضًا بسبب الشر

\par 16 وأردتُ أن أُعطيها شيلة أيضًا، ولكن أمه لم تسمح بذلك، لأنها عملت الشر على ثامار، لأنها لم تكن من بنات كنعان كما هي أيضًا

\par 17 وكنت أعلم أن جنس الكنعانيين شرير، لكن دافع الشباب أعمى عقلي

\par 18 ولما رأيتها تسكب الخمر، غررت من سكر الخمر، فأخذتها مع أن أبي لم يشير عليّ بذلك.

\par 19 وبينما كنتُ غائبًا، ذهبتْ وأخذتْ لشيلة زوجةً من كنعان

\par 20 ولما علمتُ ما فعلت، لعنتها في عذاب روحي

\par 21 وماتت أيضًا بسبب شرها مع أبنائها.

\par 22 وبعد هذه الأمور سمعت ثامار وهي أرملة بعد سنتين أني صاعد لأجز غنمي، فتزينت بزي العروس وجلست في مدينة عينايم عند الباب.

\par 23 لأنه كان من شريعة الأموريين أن من أرادت الزواج أن تجلس في زنا سبعة أيام عند الباب

\par 24 لذلك، كنت ثملًا من الخمر، ولم أتعرف عليها، وقد خدعني جمالها من خلال طريقة زينتها

\par 25 فالتفت إليها وقلت: دعيني أدخل إليكِ.

\par 26 فقالت ماذا تعطيني فأعطيتها عصاي ومنطقتي وتاج مملكتي رهنا.

\par 27 فدخلت عليها فحبلت.

\par 28 ولما لم أعلم بما فعلت أردت أن أقتلها، ولكنها أرسلت لي سراً تعهداتي، فأخجلتني.

\par 29 ولما دعوتها، سمعت أيضًا الكلمات السرية التي تكلمت بها عندما كنت أضطجع معها في سكري، ولم أستطع قتلها، لأنه كان من الرب

\par 30 لأني قلتُ لعلها فعلت ذلك بمكر، وقد أخذت الرهن من امرأة أخرى

\par 31 ولكنني لم أعد أقترب منها في حياتي، لأني كنت قد فعلت هذا الرجس في كل إسرائيل

\par 32 علاوة على ذلك، قال الذين كانوا في المدينة إنه لم تكن هناك زانية في الباب، لأنها جاءت من مكان آخر، وجلست عند الباب لفترة

\par 33 واعتقدتُ أن لا أحد يعلم أنني دخلتُ إليها.

\par 34 وبعد ذلك أتينا إلى مصر إلى يوسف بسبب الجوع.

\par 35 وكان عمري ستة وأربعين عامًا، وعشت في مصر ثلاثة وسبعين عامًا

\chapter{3}

\par \textit{ينصح بتجنب الخمر والشهوة باعتبارهما شرين توأمين. "لأن السكران لا يهاب أحدًا." (الآية 13).}

\par 1 والآن أوصيكم يا أبنائي أن تسمعوا ليهوذا أبيكم، وتحفظوا كلامي لتعملوا جميع أحكام الرب، وتطيعوا وصايا الله

\par 2 ولا تسلُكْ وراء شهواتك، ولا في تصورات أفكارك بكبرياء قلبك، ولا تفتخر بأعمال شبابك وقوته، لأن هذا أيضًا شر في عيني الرب

\par 3 ولأنني كنت أفتخر أيضًا بأنه لم يغويني وجه امرأة جميلة في الحروب، وكنت أوبّخ رأوبين أخي بشأن بلهة، امرأة أبي، اصطفت ضدي أرواح الغيرة والزنى، حتى اضطجعت مع بثشوع الكنعانية، وثامار التي كانت مخطوبة لأبنائي

\par 4 لأني قلت لحمي: سأتشاور مع أبي، فآخذ ابنتك أيضًا

\par 5 ولكنه لم يكن راغبًا في أن أراني مخزونًا لا حدود له من الذهب لصالح ابنته؛ لأنه كان ملكًا.

\par 6 وزيّنها بالذهب واللؤلؤ، وجعلها تسكب لنا خمرًا في الوليمة بجمال النساء

\par 7 وأبعد الخمر عيني، وأعمى اللذة قلبي.

\par 8 فأعجبت بها واضطجعت معها وعصيت وصية الرب ووصية آبائي فأخذتها زوجة.

\par 9 فأجازاني الرب حسب تصور قلبي، إذ لم أفرح بأولادها

\par 10 والآن يا أبنائي، أقول لكم: لا تسكروا بالخمر؛ لأن الخمر يصرف العقل عن الحق، ويثير شهوة الشهوة، ويقود العيون إلى الضلال

\par 11 لأن روح الزنا يستخدم الخمر كخادم لإسعاد العقل، لأن هذين الاثنين أيضًا يسلبان عقل الإنسان

\par 12 لأنه إن شرب أحد الخمر حتى السكر، فإنه يُقلق العقل بأفكار دنيئة تؤدي إلى الزنا، ويُسخن الجسد إلى الاتحاد الجسدي. وإن وُجدت مناسبة الشهوة، فإنه يعمل الخطية ولا يخجل

\par 13 هذا هو الرجل السكران يا أبنائي، لأن السكران لا يحترم أحدًا

\par 14 فها أنا أيضًا أضلني، فلم أخجل من جمهور المدينة، إذ ملتُ إلى ثامار أمام أعين الجميع، وفعلتُ خطيئة عظيمة، وكشفتُ غطاء عار أبنائي

\par 15 بعد أن شربت الخمر، لم أُراعِ وصية الله، واتخذت امرأة كنعانية زوجةً لي

\par 16 يا أبنائي، يحتاج شارب الخمر إلى الكثير من التعقل؛ وهنا يكمن التعقل في شرب الخمر، إذ يجوز للرجل أن يشربه ما دام محافظًا على الحياء

\par 17 ولكن إذا تجاوز هذا الحد، فإن روح الخداع تهاجم عقله، وتجعل السكير يتكلم بفظاظة، ويتعدى على الآخرين، ولا يخجل، بل يفتخر بعاره، ويعتبر نفسه محترمًا

\par 18 من يزني لا يشعر عندما يعاني من الخسارة، ولا يخجل عندما يُهان

\par 19 فإنه حتى لو كان إنسان ملكًا وزنى، فإنه يُجرد من ملكه عندما يصير عبدًا للزنى، كما عانيت أنا أيضًا

\par 20 لأني أعطيت عصاي، أي سند سبطي؛ ومنطقتي، أي قوتي؛ وإكليلي، أي مجد مملكتي

\par 21 ولقد ندمت على هذه الأمور؛ فلم آكل الخمر واللحم حتى شيخوختي، ولم أرَ أي فرح

\par 22 وأراني ملاك الله أن النساء تحكمن الملك والمتسول على حد سواء إلى الأبد

\par 23 ويسلبون من الملك مجده، ومن الشجاع قوته، ومن الفقير حتى ذلك القليل الذي هو سبب فقره

\par 24 لاحظوا، إذن، يا أبنائي، الحد الصحيح للخمر؛ لأنه يحتوي على أربعة أرواح شريرة: الشهوة، والرغبة الشديدة، والفجور، والربح القذر

\par 25 إذا شربتم الخمر في الفرح، فكونوا متواضعين في خوف الله.

\par 26 لأنه إن فارقكم خوف الله فرحًا، فعندئذٍ ينشأ السُّكر ويتسلل الوقاحة

\par 27 ولكن إن أردتم أن تعيشوا بعقلانية، فلا تلمسوا الخمر إطلاقًا، لئلا تخطئوا بكلام الفاحشة، والخصام، والقذف، ومخالفة وصايا الله، فتهلكوا قبل أوانكم

\par 28 علاوة على ذلك، يكشف الخمر أسرار الله والناس، كما كشفتُ أيضًا وصايا الله وأسرار يعقوب أبي للمرأة الكنعانية بثشوع، التي أمرني الله ألا أكشفها

\par 29 والخمر سبب للحرب والاضطراب.

\par 30 والآن أوصيكم يا أبنائي أن لا تحبوا المال، ولا تنظروا إلى جمال النساء، لأني من أجل المال والجمال أضللت إلى بثشوع الكنعانية.

\par 31 لأني أعلم أنه بسبب هذين الأمرين سيقع قومي في الشر

\par 32 لأنه حتى الحكماء من أبنائي سيفسدون، وسيُنقصون مملكة يهوذا التي أعطاني إياها الرب بسبب طاعتي لأبي

\par 33 لأني لم أُحزن يعقوب أبي قط، لأني فعلت كل ما أمر به

\par 34 وباركني إسحاق أبو أبي ملكًا على إسرائيل، وباركني يعقوب أيضًا مثله

\par 35 وأنا أعلم أنه مني سيُقام الملكوت.

\par 36 وأنا أعلم ما هي الشرور التي ستفعلونها في الأيام الأخيرة.

\par 37 فاحذروا إذن يا أبنائي من الزنا ومحبة المال، وأصغوا إلى يهوذا أبيكم

\par 38 لأن هذه الأشياء تنحرف عن شريعة الله، وتعمي ميل النفس، وتعلم الغطرسة، ولا تدع الإنسان يرحم جاره

\par 39 إنهم يسلبون روحه كل خير، ويرهقونه بالمتاعب والمتاعب، ويطردون عنه النوم، ويلتهمون لحمه

\par 40 ويمنع ذبائح الله، ولا يذكر بركة الله، ولا يسمع لنبي إذا تكلم، ويبغض كلام التقوى

\par 41 لأنه عبد لعاطفتين متعارضتين، ولا يستطيع أن يطيع الله، لأنهما أعميا نفسه، فيسلك في النهار كما في الليل

\par 42 يا أبنائي، إن حب المال يقود إلى عبادة الأصنام؛ لأنه عندما ينخدع الناس بالمال، فإنهم يسمون آلهةً أولئك الذين ليسوا بآلهة، وهذا يتسبب في وقوع من يملكه في الجنون

\par 43 من أجل المال فقدت أطفالي، ولولا توبتي، وإذلالي، ودعاء والدي، لكنت متُّ بلا أطفال

\par 44 لكن إله آبائي رحمني، لأني فعلت ذلك بجهل

\par 45 وأعماني أمير الخداع، فأخطأت كإنسان وكجسد، فاسدًا بالخطايا؛ وتعلمت ضعفي وأنا أظن نفسي لا أُقهر

\par 46 اعلموا إذن يا أبنائي أن هناك روحين تنتظران الإنسان - روح الحق وروح الخداع

\par 47 وفي وسطها روح الفهم، أي العقل الذي ينتمي إليه ليتجه إلى حيث يشاء.

\par 48 وأعمال الحق وأعمال الخداع مكتوبة على قلوب البشر، وكل واحدة منها يعلمها الرب

\par 49 وليس هناك وقت يمكن فيه إخفاء أعمال البشر؛ لأنها مكتوبة على القلب نفسه أمام الرب

\par 50 وروح الحق يشهد على كل شيء، ويتهم كل شيء. والخاطئ يحترق قلبه، ولا يستطيع أن يرفع وجهه إلى القاضي

\chapter{4}

\par \textit{يقدم يهوذا تشبيهًا حيًا بشأن الطغيان ونبوءة رهيبة بشأن أخلاقيات مستمعيه.}

\par 1 والآن يا أبنائي، أوصيكم: أحبوا لاوي، لكي تثبتوا، ولا تتكبروا عليه لئلا تهلكوا تمامًا

\par 2 لأن الرب أعطاني الملكوت، وله الكهنوت، وجعل الملكوت دون الكهنوت

\par 3 أعطاني ما على الأرض، وأعطاه ما في السماوات

\par 4 كما أن السماء أعلى من الأرض، فكذلك كهنوت الله أعلى من الملكوت الأرضي، ما لم يسقط عن الرب بسبب الخطيئة ويسيطر عليه الملكوت الأرضي

\par 5 لأن ملاك الرب قال لي: اختاره الرب دونك، لتقترب إليه، وتأكل من مائدته، وتقدم له باكورة خيرات بني إسرائيل. أما أنت فتكون ملكًا على يعقوب

\par 6 وتكون بينهم كالبحر.

\par 7 فكما يتقاذف البحر الأبرار والأشرار، فيؤخذ البعض إلى الأسر بينما يغني البعض الآخر، هكذا أيضاً كل جنس من البشر سيكون فيك: بعضهم سيفقرون عندما يؤخذون إلى الأسر، والبعض الآخر يصبحون أغنياء من خلال نهب ممتلكات الآخرين.

\par 8 لأن الملوك سيكونون كوحوش البحر.

\par 9 سيبتلعون البشر كالأسماك: يستعبدون أبناء وبنات الأحرار؛ وينهبون البيوت والأراضي والقطعان والأموال

\par 10 وبلحم كثيرين سيطعمون الغربان والكركي ظلماً، وسيتقدمون في الشر في الجشع المرتفع، وسيكون هناك أنبياء كذبة كالعاصفة، ويضطهدون كل الصالحين

\par 11 ويُجلب الرب عليهم انقسامات بعضهم على بعض.

\par 12 وتكون حروب مستمرة في إسرائيل، وتنتهي مملكتي بين الناس من جنس آخر، حتى يأتي خلاص إسرائيل.

\par 13 إلى ظهور إله البر، فيستريح يعقوب وجميع الأمم بسلام

\par 14 ويحفظ قوة مملكتي إلى الأبد، لأن الرب يعلم لي بقسم أنه لن يدمر المملكة من نسلي إلى الأبد

\par 15 الآن لدي حزن شديد يا أبنائي، بسبب فجوركم وسحركم وعبادة الأصنام التي ستمارسونها ضد المملكة، متبعين أصحاب الأرواح والعرافين وشياطين الضلال

\par 16 وتجعلون بناتكم مغنيات وزانيات وتختلطون برجاسات الأمم.

\par 17 من أجل هذه الأمور يجلب الرب عليكم الجوع والوباء، والموت والسيف، ومحاصرة الأعداء، وشتائم الأصدقاء، وقتل الأطفال، واغتصاب النساء، ونهب الممتلكات، وحرق هيكل الله، وخراب الأرض، واستعباد أنفسكم بين الأمم

\par 18 ويجعلون منكم خصيانًا لنسائهم.

\par 19 حتى يفتقدكم الرب، وتتوبون بقلب كامل، وتسلكون في جميع وصاياه، ويصعدكم من سبي الأمم.

\par 20 وبعد هذه الأمور يطلع لكم نجم من يعقوب بسلام،

\par 21 وسيقوم إنسان من نسلي، كشمس البر،

\par 22 السير مع بني البشر في الوداعة والبر.

\par 23 ولن توجد فيه خطية.

\par 24 وتنفتح له السماوات، ليسكب الروح، بركة الآب القدوس، فيسكب عليكم روح النعمة

\par 25 وتكونون له أبناءً بالحق، وتسلكون في وصاياه أولًا وآخرًا

\par 26 فيُشرق صولجان ملكوتي، ويرتفع من أصلك جذع، وينبت منه قضيب بر للأمم، ليحكم ويخلص جميع الذين يدعون الرب

\par 27 وبعد هذا يقوم إبراهيم وإسحاق ويعقوب إلى الحياة، وأنا وإخوتي نكون رؤساء أسباط إسرائيل

\par 28 لاوي أولًا، أنا ثانيًا، يوسف ثالثًا، بنيامين رابعًا، شمعون خامسًا، يساكر سادسًا، وهكذا بالترتيب

\par 29 وبارك الرب لاوي، وملاك الحضور أنا، وقوات المجد شمعون، والسماء رأوبين، والأرض يساكر، والبحر زبولون، والجبال يوسف، والمسكن بنيامين، والنجوم دان، وعدن نفتالي، والشمس جاد، والقمر أشير

\par 30 وتكونون شعبًا للرب، ولسانًا واحدًا، ولا يكون هناك روح خداع بليعار، لأنه يُلقى في النار إلى الأبد

\par 31 والذين ماتوا حزنًا سيقومون بفرح، والذين كانوا فقراء من أجل الرب سيُغنون، والذين قُتلوا من أجل الرب سيستيقظون

\par 32 فتركض أيائل يعقوب بفرح، وتطير نسور إسرائيل بفرح، ويمجد كل الشعوب الرب إلى الأبد

\par 33 فاحفظوا إذن يا أبنائي كل شريعة الرب، لأن هناك رجاءً لكل من يتمسك بطرقه

\par 34 فقال لهم: ها أنا أموت أمام أعينكم اليوم ابن مئة وتسع عشرة سنة

\par 35 لا يدفنني أحد بملابس فاخرة، ولا يشق أحشائي، لأن الملوك يفعلون هذا، واصعدوني معكم إلى الخليل

\par 36 ولما قال يهوذا هذا نام، ففعل بنوه حسب كل ما أوصاهم به، ودفنوه في حبرون مع آبائه.


\end{document}