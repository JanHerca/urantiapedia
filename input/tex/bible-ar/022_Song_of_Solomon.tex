\begin{document}

\title{نشيد الانشاد}


\chapter{1}

\par 1 نَشِيدُ الأَنَاشِيدِ الَّذِي لِسُلَيْمَانَ:
\par 2 لِيُقَبِّلْنِي بِقُبْلاَتِ فَمِهِ لأَنَّ حُبَّكَ أَطْيَبُ مِنَ الْخَمْرِ.
\par 3 لِرَائِحَةِ أَدْهَانِكَ الطَّيِّبَةِ. اسْمُكَ دُهْنٌ مُهْرَاقٌ لِذَلِكَ أَحَبَّتْكَ الْعَذَارَى.
\par 4 اُجْذُبْنِي وَرَاءَكَ فَنَجْرِيَ. أَدْخَلَنِي الْمَلِكُ إِلَى حِجَالِهِ. نَبْتَهِجُ وَنَفْرَحُ بِكَ. نَذْكُرُ حُبَّكَ أَكْثَرَ مِنَ الْخَمْرِ. بِالْحَقِّ يُحِبُّونَكَ.
\par 5 أَنَا سَوْدَاءُ وَجَمِيلَةٌ يَا بَنَاتِ أُورُشَلِيمَ كَخِيَامِ قِيدَارَ كَشُقَقِ سُلَيْمَانَ.
\par 6 لاَ تَنْظُرْنَ إِلَيَّ لِكَوْنِي سَوْدَاءَ لأَنَّ الشَّمْسَ قَدْ لَوَّحَتْنِي. بَنُو أُمِّي غَضِبُوا عَلَيَّ. جَعَلُونِي نَاطُورَةَ الْكُرُومِ. أَمَّا كَرْمِي فَلَمْ أَنْطُرْهُ.
\par 7 أَخْبِرْنِي يَا مَنْ تُحِبُّهُ نَفْسِي أَيْنَ تَرْعَى أَيْنَ تُرْبِضُ عِنْدَ الظَّهِيرَةِ. لِمَاذَا أَنَا أَكُونُ كَمُقَنَّعَةٍ عِنْدَ قُطْعَانِ أَصْحَابِكَ؟
\par 8 إِنْ لَمْ تَعْرِفِي أَيَّتُهَا الْجَمِيلَةُ بَيْنَ النِّسَاءِ فَاخْرُجِي عَلَى آثَارِ الْغَنَمِ وَارْعَيْ جِدَاءَكِ عِنْدَ مَسَاكِنِ الرُّعَاةِ.
\par 9 لَقَدْ شَبَّهْتُكِ يَا حَبِيبَتِي بِفَرَسٍ فِي مَرْكَبَاتِ فِرْعَوْنَ.
\par 10 مَا أَجْمَلَ خَدَّيْكِ بِسُمُوطٍ وَعُنُقَكِ بِقَلاَئِدَ!
\par 11 نَصْنَعُ لَكِ سَلاَسِلَ مِنْ ذَهَبٍ مَعَ جُمَانٍ مِنْ فِضَّةٍ.
\par 12 مَا دَامَ الْمَلِكُ فِي مَجْلِسِهِ أَفَاحَ نَارِدِينِي رَائِحَتَهُ.
\par 13 صُرَّةُ الْمُرِّ حَبِيبِي لِي. بَيْنَ ثَدْيَيَّ يَبِيتُ.
\par 14 طَاقَةُ فَاغِيَةٍ حَبِيبِي لِي فِي كُرُومِ عَيْنِ جَدْيٍ.
\par 15 هَا أَنْتِ جَمِيلَةٌ يَا حَبِيبَتِي هَا أَنْتِ جَمِيلَةٌ. عَيْنَاكِ حَمَامَتَانِ.
\par 16 هَا أَنْتَ جَمِيلٌ يَا حَبِيبِي وَحُلْوٌ وَسَرِيرُنَا أَخْضَرُ.
\par 17 جَوَائِزُ بَيْتِنَا أَرْزٌ وَرَوَافِدُنَا سَرْوٌ.

\chapter{2}

\par 1 أَنَا نَرْجِسُ شَارُونَ سَوْسَنَةُ الأَوْدِيَةِ.
\par 2 كَالسَّوْسَنَةِ بَيْنَ الشَّوْكِ كَذَلِكَ حَبِيبَتِي بَيْنَ الْبَنَاتِ.
\par 3 كَالتُّفَّاحِ بَيْنَ شَجَرِ الْوَعْرِ كَذَلِكَ حَبِيبِي بَيْنَ الْبَنِينَ. تَحْتَ ظِلِّهِ اشْتَهَيْتُ أَنْ أَجْلِسَ وَثَمَرَتُهُ حُلْوَةٌ لِحَلْقِي.
\par 4 أَدْخَلَنِي إِلَى بَيْتِ الْخَمْرِ وَعَلَمُهُ فَوْقِي مَحَبَّةٌ.
\par 5 أَسْنِدُونِي بِأَقْرَاصِ الزَّبِيبِ. أَنْعِشُونِي بِالتُّفَّاحِ فَإِنِّي مَرِيضَةٌ حُبّاً.
\par 6 شِمَالُهُ تَحْتَ رَأْسِي وَيَمِينُهُ تُعَانِقُنِي.
\par 7 أُحَلِّفُكُنَّ يَا بَنَاتِ أُورُشَلِيمَ بِالظِّبَاءِ وَبِأَيَائِلِ الْحُقُولِ أَلاَّ تُيَقِّظْنَ وَلاَ تُنَبِّهْنَ الْحَبِيبَ حَتَّى يَشَاءَ!
\par 8 صَوْتُ حَبِيبِي. هُوَذَا آتٍ طَافِراً عَلَى الْجِبَالِ قَافِزاً عَلَى التِّلاَلِ.
\par 9 حَبِيبِي هُوَ شَبِيهٌ بِالظَّبْيِ أَوْ بِغُفْرِ الأَيَائِلِ. هُوَذَا وَاقِفٌ وَرَاءَ حَائِطِنَا يَتَطَلَّعُ مِنَ الْكُوى يُوَصْوِصُ مِنَ الشَّبَابِيكِ.
\par 10 أَجَابَ حَبِيبِي وَقَالَ لِي: «قُومِي يَا حَبِيبَتِي يَا جَمِيلَتِي وَتَعَالَي.
\par 11 لأَنَّ الشِّتَاءَ قَدْ مَضَى وَالْمَطَرَ مَرَّ وَزَالَ.
\par 12 الزُّهُورُ ظَهَرَتْ فِي الأَرْضِ. بَلَغَ أَوَانُ الْقَضْبِ وَصَوْتُ الْيَمَامَةِ سُمِعَ فِي أَرْضِنَا.
\par 13 التِّينَةُ أَخْرَجَتْ فِجَّهَا وَقُعَالُ الْكُرُومِ تُفِيحُ رَائِحَتَهَا. قُومِي يَا حَبِيبَتِي يَا جَمِيلَتِي وَتَعَالَيْ.
\par 14 يَا حَمَامَتِي فِي مَحَاجِئِ الصَّخْرِ فِي سِتْرِ الْمَعَاقِلِ. أَرِينِي وَجْهَكِ. أَسْمِعِينِي صَوْتَكِ لأَنَّ صَوْتَكِ لَطِيفٌ وَوَجْهَكِ جَمِيلٌ».
\par 15 خُذُوا لَنَا الثَّعَالِبَ الثَّعَالِبَ الصِّغَارَ الْمُفْسِدَةَ الْكُرُومِ لأَنَّ كُرُومَنَا قَدْ أَقْعَلَتْ.
\par 16 حَبِيبِي لِي وَأَنَا لَهُ الرَّاعِي بَيْنَ السَّوْسَنِ.
\par 17 إِلَى أَنْ يَفِيحَ النَّهَارُ وَتَنْهَزِمَ الظِّلاَلُ ارْجِعْ وَأَشْبِهْ يَا حَبِيبِي الظَّبْيَ أَوْ غُفْرَ الأَيَائِلِ عَلَى الْجِبَالِ الْمُشَعَّبَةِ.

\chapter{3}

\par 1 فِي اللَّيْلِ عَلَى فِرَاشِي طَلَبْتُ مَنْ تُحِبُّهُ نَفْسِي طَلَبْتُهُ فَمَا وَجَدْتُهُ.
\par 2 إِنِّي أَقُومُ وَأَطُوفُ فِي الْمَدِينَةِ فِي الأَسْوَاقِ وَفِي الشَّوَارِعِ أَطْلُبُ مَنْ تُحِبُّهُ نَفْسِي. طَلَبْتُهُ فَمَا وَجَدْتُهُ.
\par 3 وَجَدَنِي الْحَرَسُ الطَّائِفُ فِي الْمَدِينَةِ فَقُلْتُ: «أَرَأَيْتُمْ مَنْ تُحِبُّهُ نَفْسِي؟»
\par 4 فَمَا جَاوَزْتُهُمْ إِلاَّ قَلِيلاً حَتَّى وَجَدْتُ مَنْ تُحِبُّهُ نَفْسِي فَأَمْسَكْتُهُ وَلَمْ أَرْخِهِ حَتَّى أَدْخَلْتُهُ بَيْتَ أُمِّي وَحُجْرَةَ مَنْ حَبِلَتْ بِي.
\par 5 أُحَلِّفُكُنَّ يَا بَنَاتِ أُورُشَلِيمَ بِالظِّبَاءِ وَبِأَيَائِلِ الْحَقْلِ أَلاَّ تُيَقِّظْنَ وَلاَ تُنَبِّهْنَ الْحَبِيبَ حَتَّى يَشَاءَ.
\par 6 مَنْ هَذِهِ الطَّالِعَةُ مِنَ الْبَرِّيَّةِ كَأَعْمِدَةٍ مِنْ دُخَانٍ مُعَطَّرَةً بِالْمُرِّ وَاللُّبَانِ وَبِكُلِّ أَذِرَّةِ التَّاجِرِ؟
\par 7 هُوَذَا تَخْتُ سُلَيْمَانَ حَوْلَهُ سِتُّونَ جَبَّاراً مِنْ جَبَابِرَةِ إِسْرَائِيلَ.
\par 8 كُلُّهُمْ قَابِضُونَ سُيُوفاً وَمُتَعَلِّمُونَ الْحَرْبَ. كُلُّ رَجُلٍ سَيْفُهُ عَلَى فَخْذِهِ مِنْ هَوْلِ اللَّيْلِ.
\par 9 اَلْمَلِكُ سُلَيْمَانُ عَمِلَ لِنَفْسِهِ تَخْتاً مِنْ خَشَبِ لُبْنَانَ.
\par 10 عَمِلَ أَعْمِدَتَهُ فِضَّةً وَرَوَافِدَهُ ذَهَباً وَمَقْعَدَهُ أُرْجُواناً وَوَسَطَهُ مَرْصُوفاً مَحَبَّةً مِنْ بَنَاتِ أُورُشَلِيمَ.
\par 11 اُخْرُجْنَ يَا بَنَاتِ صِهْيَوْنَ وَانْظُرْنَ الْمَلِكَ سُلَيْمَانَ بِالتَّاجِ الَّذِي تَوَّجَتْهُ بِهِ أُمُّهُ فِي يَوْمِ عُرْسِهِ وَفِي يَوْمِ فَرَحِ قَلْبِهِ.

\chapter{4}

\par 1 هَا أَنْتِ جَمِيلَةٌ يَا حَبِيبَتِي هَا أَنْتِ جَمِيلَةٌ! عَيْنَاكِ حَمَامَتَانِ مِنْ تَحْتِ نَقَابِكِ. شَعْرُكِ كَقَطِيعِ مِعْزٍ رَابِضٍ عَلَى جَبَلِ جِلْعَادَ.
\par 2 أَسْنَانُكِ كَقَطِيعِ الْجَزَائِزِ الصَّادِرَةِ مِنَ الْغَسْلِ اللَّوَاتِي كُلُّ وَاحِدَةٍ مُتْئِمٌ وَلَيْسَ فِيهِنَّ عَقِيمٌ.
\par 3 شَفَتَاكِ كَسِلْكَةٍ مِنَ الْقِرْمِزِ. وَفَمُكِ حُلْوٌ. خَدُّكِ كَفِلْقَةِ رُمَّانَةٍ تَحْتَ نَقَابِكِ.
\par 4 عُنُقُكِ كَبُرْجِ دَاوُدَ الْمَبْنِيِّ لِلأَسْلِحَةِ. أَلْفُ مِجَنٍّ عُلِّقَ عَلَيْهِ كُلُّهَا أَتْرَاسُ الْجَبَابِرَةِ.
\par 5 ثَدْيَاكِ كَخِشْفَتَيْ ظَبْيَةٍ تَوْأَمَيْنِ يَرْعَيَانِ بَيْنَ السَّوْسَنِ.
\par 6 إِلَى أَنْ يَفِيحَ النَّهَارُ وَتَنْهَزِمَ الظِّلاَلُ أَذْهَبُ إِلَى جَبَلِ الْمُرِّ وَإِلَى تَلِّ اللُّبَانِ.
\par 7 كُلُّكِ جَمِيلٌ يَا حَبِيبَتِي لَيْسَ فِيكِ عَيْبَةٌ.
\par 8 هَلُمِّي مَعِي مِنْ لُبْنَانَ يَا عَرُوسُ مَعِي مِنْ لُبْنَانَ! انْظُرِي مِنْ رَأْسِ أَمَانَةَ مِنْ رَأْسِ شَنِيرَ وَحَرْمُونَ مِنْ خُدُورِ الأُسُودِ مِنْ جِبَالِ النُّمُورِ.
\par 9 قَدْ سَبَيْتِ قَلْبِي يَا أُخْتِي الْعَرُوسُ. قَدْ سَبَيْتِ قَلْبِي بِإِحْدَى عَيْنَيْكِ بِقَلاَدَةٍ وَاحِدَةٍ مِنْ عُنُقِكِ.
\par 10 مَا أَحْسَنَ حُبَّكِ يَا أُخْتِي الْعَرُوسُ! كَمْ مَحَبَّتُكِ أَطْيَبُ مِنَ الْخَمْرِ وَكَمْ رَائِحَةُ أَدْهَانِكِ أَطْيَبُ مِنْ كُلِّ الأَطْيَابِ!
\par 11 شَفَتَاكِ يَا عَرُوسُ تَقْطُرَانِ شَهْداً. تَحْتَ لِسَانِكِ عَسَلٌ وَلَبَنٌ وَرَائِحَةُ ثِيَابِكِ كَرَائِحَةِ لُبْنَانَ.
\par 12 أُخْتِي الْعَرُوسُ جَنَّةٌ مُغْلَقَةٌ عَيْنٌ مُقْفَلَةٌ يَنْبُوعٌ مَخْتُومٌ.
\par 13 أَغْرَاسُكِ فِرْدَوْسُ رُمَّانٍ مَعَ أَثْمَارٍ نَفِيسَةٍ فَاغِيَةٍ وَنَارِدِينٍ.
\par 14 نَارِدِينٍ وَكُرْكُمٍ. قَصَبِ الذَّرِيرَةِ وَقِرْفَةٍ مَعَ كُلِّ عُودِ اللُّبَانِ. مُرٌّ وَعُودٌ مَعَ كُلِّ أَنْفَسِ الأَطْيَابِ.
\par 15 يَنْبُوعُ جَنَّاتٍ بِئْرُ مِيَاهٍ حَيَّةٍ وَسُيُولٌ مِنْ لُبْنَانَ.
\par 16 اِسْتَيْقِظِي يَا رِيحَ الشَّمَالِ وَتَعَالَيْ يَا رِيحَ الْجَنُوبِ! هَبِّي عَلَى جَنَّتِي فَتَقْطُرَ أَطْيَابُهَا. لِيَأْتِ حَبِيبِي إِلَى جَنَّتِهِ وَيَأْكُلْ ثَمَرَهُ النَّفِيسَ.

\chapter{5}

\par 1 قَدْ دَخَلْتُ جَنَّتِي يَا أُخْتِي الْعَرُوسُ. قَطَفْتُ مُرِّي مَعَ طِيبِي. أَكَلْتُ شَهْدِي مَعَ عَسَلِي. شَرِبْتُ خَمْرِي مَعَ لَبَنِي. كُلُوا أَيُّهَا الأَصْحَابُ. اشْرَبُوا وَاسْكَرُوا أَيُّهَا الأَحِبَّاءُ.
\par 2 أَنَا نَائِمَةٌ وَقَلْبِي مُسْتَيْقِظٌ. صَوْتُ حَبِيبِي قَارِعاً: «اِفْتَحِي لِي يَا أُخْتِي يَا حَبِيبَتِي يَا حَمَامَتِي يَا كَامِلَتِي لأَنَّ رَأْسِي امْتَلَأَ مِنَ الطَّلِّ وَقُصَصِي مِنْ نَدَى اللَّيْلِ».
\par 3 قَدْ خَلَعْتُ ثَوْبِي فَكَيْفَ أَلْبِسُهُ؟ قَدْ غَسَلْتُ رِجْلَيَّ فَكَيْفَ أُوَسِّخُهُمَا؟
\par 4 حَبِيبِي مَدَّ يَدَهُ مِنَ الْكُوَّةِ فَأَنَّتْ عَلَيْهِ أَحْشَائِي.
\par 5 قُمْتُ لأَفْتَحَ لِحَبِيبِي وَيَدَايَ تَقْطُرَانِ مُرّاً وَأَصَابِعِي مُرٌّ قَاطِرٌ عَلَى مَقْبَضِ الْقُفْلِ.
\par 6 فَتَحْتُ لِحَبِيبِي لَكِنَّ حَبِيبِي تَحَوَّلَ وَعَبَرَ. نَفْسِي خَرَجَتْ عِنْدَمَا أَدْبَرَ. طَلَبْتُهُ فَمَا وَجَدْتُهُ. دَعَوْتُهُ فَمَا أَجَابَنِي.
\par 7 وَجَدَنِي الْحَرَسُ الطَّائِفُ فِي الْمَدِينَةِ. ضَرَبُونِي. جَرَحُونِي. حَفَظَةُ الأَسْوَارِ رَفَعُوا إِزَارِي عَنِّي.
\par 8 أُحَلِّفُكُنَّ يَا بَنَاتِ أُورُشَلِيمَ إِنْ وَجَدْتُنَّ حَبِيبِي أَنْ تُخْبِرْنَهُ بِأَنِّي مَرِيضَةٌ حُبّاً.
\par 9 مَا حَبِيبُكِ مِنْ حَبِيبٍ أَيَّتُهَا الْجَمِيلَةُ بَيْنَ النِّسَاءِ! مَا حَبِيبُكِ مِنْ حَبِيبٍ حَتَّى تُحَلِّفِينَا هَكَذَا!
\par 10 حَبِيبِي أَبْيَضُ وَأَحْمَرُ. مُعْلَمٌ بَيْنَ رَبْوَةٍ.
\par 11 رَأْسُهُ ذَهَبٌ إِبْرِيزٌ. قُصَصُهُ مُسْتَرْسِلَةٌ حَالِكَةٌ كَالْغُرَابِ.
\par 12 عَيْنَاهُ كَالْحَمَامِ عَلَى مَجَارِي الْمِيَاهِ مَغْسُولَتَانِ بِاللَّبَنِ جَالِسَتَانِ فِي وَقْبَيْهِمَا.
\par 13 خَدَّاهُ كَخَمِيلَةِ الطِّيبِ وَأَتْلاَمِ رَيَاحِينَ ذَكِيَّةٍ. شَفَتَاهُ سَوْسَنٌ تَقْطُرَانِ مُرّاً مَائِعاً.
\par 14 يَدَاهُ حَلْقَتَانِ مِنْ ذَهَبٍ مُرَصَّعَتَانِ بِالزَّبَرْجَدِ. بَطْنُهُ عَاجٌ أَبْيَضُ مُغَلَّفٌ بِالْيَاقُوتِ الأَزْرَقِ.
\par 15 سَاقَاهُ عَمُودَا رُخَامٍ مُؤَسَّسَتَانِ عَلَى قَاعِدَتَيْنِ مِنْ إِبْرِيزٍ. طَلْعَتُهُ كَلُبْنَانَ. فَتًى كَالأَرْزِ.
\par 16 حَلْقُهُ حَلاَوَةٌ وَكُلُّهُ مُشْتَهَيَاتٌ. هَذَا حَبِيبِي وَهَذَا خَلِيلِي يَا بَنَاتِ أُورُشَلِيمَ.

\chapter{6}

\par 1 أَيْنَ ذَهَبَ حَبِيبُكِ أَيَّتُهَا الْجَمِيلَةُ بَيْنَ النِّسَاءِ؟ أَيْنَ تَوَجَّهَ حَبِيبُكِ فَنَطْلُبَهُ مَعَكِ؟
\par 2 حَبِيبِي نَزَلَ إِلَى جَنَّتِهِ إِلَى خَمَائِلِ الطِّيبِ لِيَرْعَى فِي الْجَنَّاتِ وَيَجْمَعَ السَّوْسَنَ.
\par 3 أَنَا لِحَبِيبِي وَحَبِيبِي لِي. الرَّاعِي بَيْنَ السَّوْسَنِ.
\par 4 أَنْتِ جَمِيلَةٌ يَا حَبِيبَتِي كَتِرْصَةَ حَسَنَةٌ كَأُورُشَلِيمَ مُرْهِبَةٌ كَجَيْشٍ بِأَلْوِيَةٍ.
\par 5 حَوِّلِي عَنِّي عَيْنَيْكِ فَإِنَّهُمَا قَدْ غَلَبَتَانِي. شَعْرُكِ كَقَطِيعِ الْمَعْزِ الرَّابِضِ فِي جِلْعَادَ.
\par 6 أَسْنَانُكِ كَقَطِيعِ نِعَاجٍ صَادِرَةٍ مِنَ الْغَسْلِ اللَّوَاتِي كُلُّ وَاحِدَةٍ مُتْئِمٌ وَلَيْسَ فِيهَا عَقِيمٌ.
\par 7 كَفِلْقَةِ رُمَّانَةٍ خَدُّكِ تَحْتَ نَقَابِكِ.
\par 8 هُنَّ سِتُّونَ مَلِكَةً وَثَمَانُونَ سُرِّيَّةً وَعَذَارَى بِلاَ عَدَدٍ.
\par 9 وَاحِدَةٌ هِيَ حَمَامَتِي كَامِلَتِي. الْوَحِيدَةُ لِأُمِّهَا هِيَ. عَقِيلَةُ وَالِدَتِهَا هِيَ. رَأَتْهَا الْبَنَاتُ فَطَوَّبْنَهَا. الْمَلِكَاتُ وَالسَّرَارِيُّ فَمَدَحْنَهَا.
\par 10 مَنْ هِيَ الْمُشْرِفَةُ مِثْلَ الصَّبَاحِ جَمِيلَةٌ كَالْقَمَرِ طَاهِرَةٌ كَالشَّمْسِ مُرْهِبَةٌ كَجَيْشٍ بِأَلْوِيَةٍ؟
\par 11 نَزَلْتُ إِلَى جَنَّةِ الْجَوْزِ لأَنْظُرَ إِلَى خُضَرِ الْوَادِي وَلأَنْظُرَ: هَلْ أَقْعَلَ الْكَرْمُ؟ هَلْ نَوَّرَ الرُّمَّانُ؟
\par 12 فَلَمْ أَشْعُرْ إِلاَّ وَقَدْ جَعَلَتْنِي نَفْسِي بَيْنَ مَرْكَبَاتِ قَوْمِ شَرِيفٍ.
\par 13 اِرْجِعِي ارْجِعِي يَا شُولَمِّيثُ. ارْجِعِي ارْجِعِي فَنَنْظُرَ إِلَيْكِ. مَاذَا تَرَوْنَ فِي شُولَمِّيثَ مِثْلَ رَقْصِ صَفَّيْنِ؟

\chapter{7}

\par 1 مَا أَجْمَلَ رِجْلَيْكِ بِالنَّعْلَيْنِ يَا بِنْتَ الْكَرِيمِ! دَوَائِرُ فَخْذَيْكِ مِثْلُ الْحَلِيِّ صَنْعَةِ يَدَيْ صَنَّاعٍ.
\par 2 سُرَّتُكِ كَأْسٌ مُدَوَّرَةٌ لاَ يُعْوِزُهَا شَرَابٌ مَمْزُوجٌ. بَطْنُكِ صُبْرَةُ حِنْطَةٍ مُسَيَّجَةٌ بِالسَّوْسَنِ.
\par 3 ثَدْيَاكِ كَخِشْفَتَيْنِ تَوْأَمَيْ ظَبْيَةٍ.
\par 4 عُنُقُكِ كَبُرْجٍ مِنْ عَاجٍ. عَيْنَاكِ كَالْبِرَكِ فِي حَشْبُونَ عِنْدَ بَابِ بَثِّ رَبِّيمَ. أَنْفُكِ كَبُرْجِ لُبْنَانَ النَّاظِرِ تُجَاهَ دِمَشْقَ.
\par 5 رَأْسُكِ عَلَيْكِ مِثْلُ الْكَرْمَلِ وَشَعْرُ رَأْسِكِ كَأُرْجُوَانٍ. مَلِكٌ قَدْ أُسِرَ بِالْخُصَلِ.
\par 6 مَا أَجْمَلَكِ وَمَا أَحْلاَكِ أَيَّتُهَا الْحَبِيبَةُ بِاللَّذَّاتِ!
\par 7 قَامَتُكِ هَذِهِ شَبِيهَةٌ بِالنَّخْلَةِ وَثَدْيَاكِ بِالْعَنَاقِيدِ.
\par 8 قُلْتُ: «إِنِّي أَصْعَدُ إِلَى النَّخْلَةِ وَأُمْسِكُ بِعُذُوقِهَا». وَتَكُونُ ثَدْيَاكِ كَعَنَاقِيدِ الْكَرْمِ وَرَائِحَةُ أَنْفِكِ كَالتُّفَّاحِ
\par 9 وَحَنَكُكِ كَأَجْوَدِ الْخَمْرِ. لِحَبِيبِي السَّائِغَةُ الْمُرَقْرِقَةُ السَّائِحَةُ عَلَى شِفَاهِ النَّائِمِينَ.
\par 10 أَنَا لِحَبِيبِي وَإِلَيَّ اشْتِيَاقُهُ.
\par 11 تَعَالَ يَا حَبِيبِي لِنَخْرُجْ إِلَى الْحَقْلِ وَلْنَبِتْ فِي الْقُرَى.
\par 12 لِنُبَكِّرَنَّ إِلَى الْكُرُومِ لِنَنْظُرَ هَلْ أَزْهَرَ الْكَرْمُ؟ هَلْ تَفَتَّحَ الْقُعَالُ؟ هَلْ نَوَّرَ الرُّمَّانُ؟ هُنَالِكَ أُعْطِيكَ حُبِّي.
\par 13 اَللُّفَّاحُ يَفُوحُ رَائِحَةً وَعِنْدَ أَبْوَابِنَا كُلُّ النَّفَائِسِ مِنْ جَدِيدَةٍ وَقَدِيمَةٍ ذَخَرْتُهَا لَكَ يَا حَبِيبِي.

\chapter{8}

\par 1 لَيْتَكَ كَأَخٍ لِي الرَّاضِعِ ثَدْيَيْ أُمِّي فَأَجِدَكَ فِي الْخَارِجِ وَأُقَبِّلَكَ وَلاَ يُخْزُونَنِي.
\par 2 وَأَقُودُكَ وَأَدْخُلُ بِكَ بَيْتَ أُمِّي وَهِيَ تُعَلِّمُنِي فَأَسْقِيكَ مِنَ الْخَمْرِ الْمَمْزُوجَةِ مِنْ سُلاَفِ رُمَّانِي.
\par 3 شِمَالُهُ تَحْتَ رَأْسِي وَيَمِينُهُ تُعَانِقُنِي.
\par 4 أُحَلِّفُكُنَّ يَا بَنَاتِ أُورُشَلِيمَ أَلاَّ تُيَقِّظْنَ وَلاَ تُنَبِّهْنَ الْحَبِيبَ حَتَّى يَشَاءَ.
\par 5 مَنْ هَذِهِ الطَّالِعَةُ مِنَ الْبَرِّيَّةِ مُسْتَنِدَةً عَلَى حَبِيبِهَا؟ تَحْتَ شَجَرَةِ التُّفَّاحِ شَوَّقْتُكَ هُنَاكَ خَطَبَتْ لَكَ أُمُّكَ هُنَاكَ خَطَبَتْ لَكَ وَالِدَتُكَ.
\par 6 اِجْعَلْنِي كَخَاتِمٍ عَلَى قَلْبِكَ كَخَاتِمٍ عَلَى سَاعِدِكَ. لأَنَّ الْمَحَبَّةَ قَوِيَّةٌ كَالْمَوْتِ. الْغَيْرَةُ قَاسِيَةٌ كَالْهَاوِيَةِ. لَهِيبُهَا لَهِيبُ نَارِ لَظَى الرَّبِّ.
\par 7 مِيَاهٌ كَثِيرَةٌ لاَ تَسْتَطِيعُ أَنْ تُطْفِئَ الْمَحَبَّةَ وَالسُّيُولُ لاَ تَغْمُرُهَا. إِنْ أَعْطَى الإِنْسَانُ كُلَّ ثَرْوَةِ بَيْتِهِ بَدَلَ الْمَحَبَّةِ تُحْتَقَرُ احْتِقَاراً.
\par 8 لَنَا أُخْتٌ صَغِيرَةٌ لَيْسَ لَهَا ثَدْيَانِ. فَمَاذَا نَصْنَعُ لِأُخْتِنَا فِي يَوْمٍ تُخْطَبُ؟
\par 9 إِنْ تَكُنْ سُوراً فَنَبْنِي عَلَيْهَا بُرْجَ فِضَّةٍ. وَإِنْ تَكُنْ بَاباً فَنَحْصُرُهَا بِأَلْوَاحِ أَرْزٍ.
\par 10 أَنَا سُورٌ وَثَدْيَايَ كَبُرْجَيْنِ. حِينَئِذٍ كُنْتُ فِي عَيْنَيْهِ كَوَاجِدَةٍ سَلاَمَةً.
\par 11 كَانَ لِسُلَيْمَانَ كَرْمٌ فِي بَعْلَ هَامُونَ. دَفَعَ الْكَرْمَ إِلَى نَوَاطِيرَ كُلُّ وَاحِدٍ يُؤَدِّي عَنْ ثَمَرِهِ أَلْفاً مِنَ الْفِضَّةِ.
\par 12 كَرْمِي الَّذِي لِي هُوَ أَمَامِي. الأَلْفُ لَكَ يَا سُلَيْمَانُ وَمِئَتَانِ لِنَوَاطِيرِ الثَّمَرِ.
\par 13 أَيَّتُهَا الْجَالِسَةُ فِي الْجَنَّاتِ الأَصْحَابُ يَسْمَعُونَ صَوْتَكِ فَأَسْمِعِينِي.
\par 14 اُهْرُبْ يَا حَبِيبِي وَكُنْ كَالظَّبْيِ أَوْ كَغُفْرِ الأَيَائِلِ عَلَى جِبَالِ الأَطْيَابِ.


\end{document}