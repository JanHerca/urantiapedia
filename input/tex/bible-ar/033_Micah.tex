\begin{document}

\title{ميخا}


\chapter{1}

\par 1 قَوْلُ الرَّبِّ الَّذِي صَارَ إِلَى مِيخَا الْمُورَشْتِيِّ فِي أَيَّامِ يُوثَامَ وَآحَازَ وَحَزَقِيَّا مُلُوكِ يَهُوذَا الَّذِي رَآهُ عَلَى السَّامِرَةِ وَأُورُشَلِيمَ:
\par 2 اِسْمَعُوا أَيُّهَا الشُّعُوبُ جَمِيعُكُمْ. أَصْغِي أَيَّتُهَا الأَرْضُ وَمِلْؤُهَا. وَلْيَكُنِ السَّيِّدُ الرَّبُّ شَاهِداً عَلَيْكُمُ السَّيِّدُ مِنْ هَيْكَلِ قُدْسِهِ.
\par 3 فَإِنَّهُ هُوَذَا الرَّبُّ يَخْرُجُ مِنْ مَكَانِهِ وَيَنْزِلُ وَيَمْشِي عَلَى شَوَامِخِ الأَرْضِ
\par 4 فَتَذُوبُ الْجِبَالُ تَحْتَهُ وَتَنْشَقُّ الْوِدْيَانُ كَالشَّمْعِ قُدَّامَ النَّارِ. كَالْمَاءِ الْمُنْصَبِّ فِي مُنْحَدَرٍ.
\par 5 كُلُّ هَذَا مِنْ أَجْلِ إِثْمِ يَعْقُوبَ وَمِنْ أَجْلِ خَطِيَّةِ بَيْتِ إِسْرَائِيلَ. مَا هُوَ ذَنْبُ يَعْقُوبَ؟ أَلَيْسَ هُوَ السَّامِرَةَ! وَمَا هِيَ مُرْتَفَعَاتُ يَهُوذَا؟ أَلَيْسَتْ هِيَ أُورُشَلِيمَ!
\par 6 «فَأَجْعَلُ السَّامِرَةَ خَرِبَةً فِي الْبَرِّيَّةِ مَغَارِسَ لِلْكُرُومِ وَأُلْقِي حِجَارَتَهَا إِلَى الْوَادِي وَأَكْشِفُ أُسُسَهَا.
\par 7 وَجَمِيعُ تَمَاثِيلِهَا الْمَنْحُوتَةِ تُحَطَّمُ وَكُلُّ أَعْقَارِهَا تُحْرَقُ بِالنَّارِ وَجَمِيعُ أَصْنَامِهَا أَجْعَلُهَا خَرَاباً لأَنَّهَا مِنْ عُقْرِ الزَّانِيَةِ جَمَعَتْهَا وَإِلَى عُقْرِ الزَّانِيَةِ تَعُودُ!».
\par 8 مِنْ أَجْلِ ذَلِكَ أَنُوحُ وَأُوَلْوِلُ. أَمْشِي حَافِياً وَعُرْيَاناً. أَصْنَعُ نَحِيباً كَبَنَاتِ آوَى وَنَوْحاً كَرِعَالِ النَّعَامِ.
\par 9 لأَنَّ جِرَاحَاتِهَا عَدِيمَةُ الشِّفَاءِ لأَنَّهَا قَدْ أَتَتْ إِلَى يَهُوذَا وَصَلَتْ إِلَى بَابِ شَعْبِي إِلَى أُورُشَلِيمَ.
\par 10 لاَ تُخْبِرُوا فِي جَتَّ لاَ تَبْكُوا فِي عَكَّاءَ. تَمَرَّغِي فِي التُّرَابِ فِي بَيْتِ عَفْرَةَ.
\par 11 اُعْبُرِي يَا سَاكِنَةَ شَافِيرَ عُرْيَانَةً وَخَجِلَةً. السَّاكِنَةُ فِي صَانَانَ لاَ تَخْرُجُ. نَوْحُ بَيْتِ هَأَيْصِلَ يَأْخُذُ عِنْدَكُمْ مَقَامَهُ
\par 12 لأَنَّ السَّاكِنَةَ فِي مَارُوثَ اغْتَمَّتْ لأَجْلِ خَيْرَاتِهَا لأَنَّ شَرّاً قَدْ نَزَلَ مِنْ عِنْدِ الرَّبِّ إِلَى بَابِ أُورُشَلِيمَ.
\par 13 شُدِّي الْمَرْكَبَةَ بِالْجَوَادِ يَا سَاكِنَةَ لاَخِيشَ. (هِيَ أَوَّلُ خَطِيَّةٍ لاِبْنَةِ صِهْيَوْنَ) لأَنَّهُ فِيكِ وُجِدَتْ ذُنُوبُ إِسْرَائِيلَ.
\par 14 لِذَلِكَ تُعْطِينَ إِطْلاَقاً لِمُورَشَةِ جَتَّ. تَصِيرُ بُيُوتُ أَكْزِيبَ كَاذِبَةً لِمُلُوكِ إِسْرَائِيلَ.
\par 15 آتِي إِلَيْكِ أَيْضاً بِالْوَارِثِ يَا سَاكِنَةَ مَرِيشَةَ. يَأْتِي إِلَى عَدُلاَّمَ مَجْدُ إِسْرَائِيلَ.
\par 16 كُونِي قَرْعَاءَ وَجُزِّي مِنْ أَجْلِ بَنِي تَنَعُّمِكِ. وَسِّعِي قَرْعَتَكِ كَالنَّسْرِ لأَنَّهُمْ قَدِ انْتَفُوا عَنْكِ.

\chapter{2}

\par 1 وَيْلٌ لِلْمُفْتَكِرِينَ بِالْبُطْلِ وَالصَّانِعِينَ الشَّرَّ عَلَى مَضَاجِعِهِمْ. فِي نُورِ الصَّبَاحِ يَفْعَلُونَهُ لأَنَّهُ فِي قُدْرَةِ يَدِهِمْ.
\par 2 فَإِنَّهُمْ يَشْتَهُونَ الْحُقُولَ وَيَغْتَصِبُونَهَا وَالْبُيُوتَ وَيَأْخُذُونَهَا وَيَظْلِمُونَ الرَّجُلَ وَبَيْتَهُ وَالإِنْسَانَ وَمِيرَاثَهُ.
\par 3 لِذَلِكَ هَكَذَا قَالَ الرَّبُّ: «هَئَنَذَا أَفْتَكِرُ عَلَى هَذِهِ الْعَشِيرَةِ بِشَرٍّ لاَ تُزِيلُونَ مِنْهُ أَعْنَاقَكُمْ وَلاَ تَسْلُكُونَ بِالتَّشَامُخِ لأَنَّهُ زَمَانٌ رَدِيءٌ.
\par 4 «فِي ذَلِكَ الْيَوْمِ يُنْطَقُ عَلَيْكُمْ بِهَجْوٍ وَيُرْثَى بِمَرْثَاةٍ وَيُقَالُ: خَرِبْنَا خَرَاباً. بَدَلَ نَصِيبِ شَعْبِي. كَيْفَ يَنْزِعُهُ عَنِّي؟ يَقْسِمُ لِلْمُرْتَدِّ حُقُولَنَا».
\par 5 لِذَلِكَ لاَ يَكُونُ لَكَ مَنْ يُلْقِي حَبْلاً فِي نَصِيبٍ بَيْنَ جَمَاعَةِ الرَّبِّ.
\par 6 يَتَنَبَّأُونَ قَائِلِينَ: «لاَ تَتَنَبَّأُوا». لاَ يَتَنَبَّأُونَ عَنْ هَذِهِ الأُمُورِ. لاَ يَزُولُ الْعَارُ.
\par 7 أَيُّهَا الْمُسَمَّى بَيْتَ يَعْقُوبَ هَلْ قَصُرَتْ رُوحُ الرَّبِّ؟ أَهَذِهِ أَفْعَالُهُ؟ «أَلَيْسَتْ أَقْوَالِي صَالِحَةً نَحْوَ مَنْ يَسْلُكُ بِالاِسْتِقَامَةِ؟
\par 8 وَلَكِنْ بِالأَمْسِ قَامَ شَعْبِي كَعَدُوٍّ. تَنْزِعُونَ الرِّدَاءَ عَنِ الثَّوْبِ مِنَ الْمُجْتَازِينَ بِالطُّمَأْنِينَةِ وَمِنَ الرَّاجِعِينَ مِنَ الْقِتَالِ.
\par 9 تَطْرُدُونَ نِسَاءَ شَعْبِي مِنْ بَيْتِ تَنَعُّمِهِنَّ. تَأْخُذُونَ عَنْ أَطْفَالِهِنَّ زِينَتِي إِلَى الأَبَدِ.
\par 10 «قُومُوا وَاذْهَبُوا لأَنَّهُ لَيْسَتْ هَذِهِ هِيَ الرَّاحَةَ. مِنْ أَجْلِ نَجَاسَةٍ تُهْلِكُ وَالْهَلاَكُ شَدِيدٌ.
\par 11 لَوْ كَانَ أَحَدٌ وَهُوَ سَالِكٌ بِالرِّيحِ وَالْكَذِبِ يَكْذِبُ قَائِلاً: أَتَنَبَّأُ لَكَ عَنِ الْخَمْرِ وَالْمُسْكِرِ لَكَانَ هُوَ نَبِيَّ هَذَا الشَّعْبِ!
\par 12 «إِنِّي أَجْمَعُ جَمِيعَكَ يَا يَعْقُوبُ. أَضُمُّ بَقِيَّةَ إِسْرَائِيلَ. أَضَعُهُمْ مَعاً كَغَنَمِ الْحَظِيرَةِ كَقَطِيعٍ فِي وَسَطِ مَرْعَاهُ يَضِجُّ مِنَ النَّاسِ.
\par 13 قَدْ صَعِدَ الْفَاتِكُ أَمَامَهُ. يَقْتَحِمُونَ وَيَعْبُرُونَ مِنَ الْبَابِ وَيَخْرُجُونَ مِنْهُ وَيَجْتَازُ مَلِكُهُمْ أَمَامَهُمْ وَالرَّبُّ فِي رَأْسِهِمْ».

\chapter{3}

\par 1 وَقُلْتُ: «اسْمَعُوا يَا رُؤَسَاءَ يَعْقُوبَ وَقُضَاةَ بَيْتِ إِسْرَائِيلَ. أَلَيْسَ لَكُمْ أَنْ تَعْرِفُوا الْحَقَّ؟
\par 2 الْمُبْغِضِينَ الْخَيْرَ وَالْمُحِبِّينَ الشَّرَّ النَّازِعِينَ جُلُودَهُمْ عَنْهُمْ وَلَحْمَهُمْ عَنْ عِظَامِهِمْ.
\par 3 وَالَّذِينَ يَأْكُلُونَ لَحْمَ شَعْبِي وَيَكْشُطُونَ جِلْدَهُمْ عَنْهُمْ وَيُهَشِّمُونَ عِظَامَهُمْ وَيُشَقِّقُونَ كَمَا فِي الْقِدْرِ وَكَاللَّحْمِ فِي وَسَطِ الْمِقْلَى».
\par 4 حِينَئِذٍ يَصْرُخُونَ إِلَى الرَّبِّ فَلاَ يُجِيبُهُمْ بَلْ يَسْتُرُ وَجْهَهُ عَنْهُمْ فِي ذَلِكَ الْوَقْتِ كَمَا أَسَاءُوا أَعْمَالَهُمْ.
\par 5 هَكَذَا قَالَ الرَّبُّ عَلَى الأَنْبِيَاءِ الَّذِينَ يُضِلُّونَ شَعْبِي الَّذِينَ يَنْهَشُونَ بِأَسْنَانِهِمْ وَيُنَادُونَ: سَلاَمٌ! وَالَّذِي لاَ يَجْعَلُ فِي أَفْوَاهِهِمْ شَيْئاً يَفْتَحُونَ عَلَيْهِ حَرْباً:
\par 6 «لِذَلِكَ تَكُونُ لَكُمْ لَيْلَةٌ بِلاَ رُؤْيَا. ظَلاَمٌ لَكُمْ بِدُونِ عِرَافَةٍ. وَتَغِيبُ الشَّمْسُ عَنِ الأَنْبِيَاءِ وَيُظْلِمُ عَلَيْهِمِ النَّهَارُ.
\par 7 فَيَخْزَى الرَّاؤُونَ وَيَخْجَلُ الْعَرَّافُونَ وَيُغَطُّونَ كُلُّهُمْ شَوَارِبَهُمْ لأَنَّهُ لَيْسَ جَوَابٌ مِنَ اللَّهِ».
\par 8 لَكِنَّنِي أَنَا مَلآنٌ قُوَّةَ رُوحِ الرَّبِّ وَحَقّاً وَبَأْساً لِأُخَبِّرَ يَعْقُوبَ بِذَنْبِهِ وَإِسْرَائِيلَ بِخَطِيَّتِهِ.
\par 9 اِسْمَعُوا هَذَا يَا رُؤَسَاءَ بَيْتِ يَعْقُوبَ وَقُضَاةَ بَيْتِ إِسْرَائِيلَ الَّذِينَ يَكْرَهُونَ الْحَقَّ وَيُعَوِّجُونَ كُلَّ مُسْتَقِيمٍ.
\par 10 الَّذِينَ يَبْنُونَ صِهْيَوْنَ بِالدِّمَاءِ وَأُورُشَلِيمَ بِالظُّلْمِ.
\par 11 رُؤَسَاؤُهَا يَقْضُونَ بِالرَّشْوَةِ وَكَهَنَتُهَا يُعَلِّمُونَ بِالأُجْرَةِ وَأَنْبِيَاؤُهَا يَعْرِفُونَ بِالْفِضَّةِ وَهُمْ يَتَوَكَّلُونَ عَلَى الرَّبِّ قَائِلِينَ: «أَلَيْسَ الرَّبُّ فِي وَسَطِنَا؟ لاَ يَأْتِي عَلَيْنَا شَرٌّ!»
\par 12 لِذَلِكَ بِسَبَبِكُمْ تُفْلَحُ صِهْيَوْنُ كَحَقْلٍ وَتَصِيرُ أُورُشَلِيمُ خِرَباً وَجَبَلُ الْبَيْتِ شَوَامِخَ وَعْرٍ.

\chapter{4}

\par 1 وَيَكُونُ فِي آخِرِ الأَيَّامِ أَنَّ جَبَلَ بَيْتِ الرَّبِّ يَكُونُ ثَابِتاً فِي رَأْسِ الْجِبَالِ وَيَرْتَفِعُ فَوْقَ التِّلاَلِ وَتَجْرِي إِلَيْهِ شُعُوبٌ.
\par 2 وَتَسِيرُ أُمَمٌ كَثِيرَةٌ وَيَقُولُونَ: «هَلُمَّ نَصْعَدْ إِلَى جَبَلِ الرَّبِّ وَإِلَى بَيْتِ إِلَهِ يَعْقُوبَ فَيُعَلِّمَنَا مِنْ طُرُقِهِ وَنَسْلُكَ فِي سُبُلِهِ». لأَنَّهُ مِنْ صِهْيَوْنَ تَخْرُجُ الشَّرِيعَةُ وَمِنْ أُورُشَلِيمَ كَلِمَةُ الرَّبِّ.
\par 3 فَيَقْضِي بَيْنَ شُعُوبٍ كَثِيرِينَ. يُنْصِفُ لِأُمَمٍ قَوِيَّةٍ بَعِيدَةٍ فَيَطْبَعُونَ سُيُوفَهُمْ سِكَكاً وَرِمَاحَهُمْ مَنَاجِلَ. لاَ تَرْفَعُ أُمَّةٌ عَلَى أُمَّةٍ سَيْفاً وَلاَ يَتَعَلَّمُونَ الْحَرْبَ فِي مَا بَعْدُ.
\par 4 بَلْ يَجْلِسُونَ كُلُّ وَاحِدٍ تَحْتَ كَرْمَتِهِ وَتَحْتَ تِينَتِهِ وَلاَ يَكُونُ مَنْ يُرْعِبُ لأَنَّ فَمَ رَبِّ الْجُنُودِ تَكَلَّمَ.
\par 5 لأَنَّ جَمِيعَ الشُّعُوبِ يَسْلُكُونَ كُلُّ وَاحِدٍ بِاسْمِ إِلَهِهِ وَنَحْنُ نَسْلُكُ بِاسْمِ الرَّبِّ إِلَهِنَا إِلَى الدَّهْرِ وَالأَبَدِ.
\par 6 «فِي ذلِكَ الْيَوْمِ يَقُولُ الرَّبُّ أَجْمَعُ الظَّالِعَةَ وَأَضُمُّ الْمَطْرُودَةَ وَالَّتِي أَضْرَرْتُ بِهَا
\par 7 وَأَجْعَلُ الظَّالِعَةَ بَقِيَّةً وَالْمُقْصَاةَ أُمَّةً قَوِيَّةً وَيَمْلِكُ الرَّبُّ عَلَيْهِمْ فِي جَبَلِ صِهْيَوْنَ مِنَ الآنَ إِلَى الأَبَدِ.
\par 8 وَأَنْتَ يَا بُرْجَ الْقَطِيعِ أَكَمَةَ بِنْتِ صِهْيَوْنَ إِلَيْكِ يَأْتِي. وَيَجِيءُ الْحُكْمُ الأَوَّلُ مُلْكُ بِنْتِ أُورُشَلِيمَ».
\par 9 اَلآنَ لِمَاذَا تَصْرُخِينَ صُرَاخاً؟ أَلَيْسَ فِيكِ مَلِكٌ أَمْ هَلَكَ مُشِيرُكِ حَتَّى أَخَذَكِ وَجَعٌ كَالْوَالِدَةِ؟
\par 10 تَلَوَّيِ ادْفَعِي يَا بِنْتَ صِهْيَوْنَ كَالْوَالِدَةِ لأَنَّكِ الآنَ تَخْرُجِينَ مِنَ الْمَدِينَةِ وَتَسْكُنِينَ فِي الْبَرِّيَّةِ وَتَأْتِينَ إِلَى بَابِلَ. هُنَاكَ تُنْقَذِينَ. هُنَاكَ يَفْدِيكِ الرَّبُّ مِنْ يَدِ أَعْدَائِكِ.
\par 11 وَالآنَ قَدِ اجْتَمَعَتْ عَلَيْكِ أُمَمٌ كَثِيرَةٌ الَّذِينَ يَقُولُونَ: «لِتَتَدَنَّسْ وَلْتَتَفَرَّسْ عُيُونُنَا فِي صِهْيَوْنَ».
\par 12 وَهُمْ لاَ يَعْرِفُونَ أَفْكَارَ الرَّبِّ وَلاَ يَفْهَمُونَ قَصْدَهُ إِنَّهُ قَدْ جَمَعَهُمْ كَحُزَمٍ إِلَى الْبَيْدَرِ.
\par 13 «قُومِي وَدُوسِي يَا بِنْتَ صِهْيَوْنَ لأَنِّي أَجْعَلُ قَرْنَكِ حَدِيداً وَأَظْلاَفَكِ أَجْعَلُهَا نُحَاساً فَتَسْحَقِينَ شُعُوباً كَثِيرِينَ وَأُحَرِّمُ غَنِيمَتَهُمْ لِلرَّبِّ وَثَرْوَتَهُمْ لِسَيِّدِ كُلِّ الأَرْضِ»

\chapter{5}

\par 1 اَلآنَ تَتَجَيَّشِينَ يَا بِنْتَ الْجُيُوشِ! قَدْ أَقَامَ عَلَيْنَا مِتْرَسَةً. يَضْرِبُونَ قَاضِيَ إِسْرَائِيلَ بِقَضِيبٍ عَلَى خَدِّهِ.
\par 2 «أَمَّا أَنْتِ يَا بَيْتَ لَحْمَِ أَفْرَاتَةَ وَأَنْتِ صَغِيرَةٌ أَنْ تَكُونِي بَيْنَ أُلُوفِ يَهُوذَا فَمِنْكِ يَخْرُجُ لِي الَّذِي يَكُونُ مُتَسَلِّطاً عَلَى إِسْرَائِيلَ وَمَخَارِجُهُ مُنْذُ الْقَدِيمِ مُنْذُ أَيَّامِ الأَزَلِ».
\par 3 لِذَلِكَ يُسَلِّمُهُمْ إِلَى حِينَمَا تَكُونُ قَدْ وَلَدَتْ وَالِدَةٌ ثُمَّ تَرْجِعُ بَقِيَّةُ إِخْوَتِهِ إِلَى بَنِي إِسْرَائِيلَ.
\par 4 وَيَقِفُ وَيَرْعَى بِقُدْرَةِ الرَّبِّ بِعَظَمَةِ اسْمِ الرَّبِّ إِلَهِهِ وَيَثْبُتُونَ. لأَنَّهُ الآنَ يَتَعَظَّمُ إِلَى أَقَاصِي الأَرْضِ.
\par 5 وَيَكُونُ هَذَا سَلاَماً. إِذَا دَخَلَ أَشُّورُ فِي أَرْضِنَا وَإِذَا دَاسَ فِي قُصُورِنَا نُقِيمُ عَلَيْهِ سَبْعَةَ رُعَاةٍ وَثَمَانِيَةً مِنْ أُمَرَاءِ النَّاسِ
\par 6 فَيَرْعُونَ أَرْضَ أَشُّورَ بِالسَّيْفِ وَأَرْضَ نِمْرُودَ فِي أَبْوَابِهَا فَيَنْفُذُ مِنْ أَشُّورَ إِذَا دَخَلَ أَرْضَنَا وَإِذَا دَاسَ تُخُومَنَا.
\par 7 وَتَكُونُ بَقِيَّةُ يَعْقُوبَ فِي وَسَطِ شُعُوبٍ كَثِيرِينَ كَالنَّدَى مِنْ عِنْدِ الرَّبِّ كَالْوَابِلِ عَلَى الْعُشْبِ الَّذِي لاَ يَنْتَظِرُ إِنْسَاناً وَلاَ يَصْبِرُ لِبَنِي الْبَشَرِ.
\par 8 وَتَكُونُ بَقِيَّةُ يَعْقُوبَ بَيْنَ الأُمَمِ فِي وَسَطِ شُعُوبٍ كَثِيرِينَ كَالأَسَدِ بَيْنَ وُحُوشِ الْوَعْرِ كَشِبْلِ الأَسَدِ بَيْنَ قُطْعَانِ الْغَنَمِ الَّذِي إِذَا عَبَرَ يَدُوسُ وَيَفْتَرِسُ وَلَيْسَ مَنْ يُنْقِذُ.
\par 9 لِتَرْتَفِعْ يَدُكَ عَلَى مُبْغِضِيكَ وَيَنْقَرِضْ كُلُّ أَعْدَائِكَ!
\par 10 «وَيَكُونُ فِي ذَلِكَ الْيَوْمِ يَقُولُ الرَّبُّ أَنِّي أَقْطَعُ خَيْلَكَ مِنْ وَسَطِكَ وَأُبِيدُ مَرْكَبَاتِكَ.
\par 11 وَأَقْطَعُ مُدُنَ أَرْضِكَ وَأَهْدِمُ كُلَّ حُصُونِكَ.
\par 12 وَأَقْطَعُ السِّحْرَ مِنْ يَدِكَ وَلاَ يَكُونُ لَكَ عَائِفُونَ.
\par 13 وَأَقْطَعُ تَمَاثِيلَكَ الْمَنْحُوتَةَ وَأَنْصَابَكَ مِنْ وَسَطِكَ فَلاَ تَسْجُدُ لِعَمَلِ يَدَيْكَ فِي مَا بَعْدُ.
\par 14 وَأَقْلَعُ سَوَارِيَكَ مِنْ وَسَطِكَ وَأُبِيدُ مُدُنَكَ.
\par 15 وَبِغَضَبٍ وَغَيْظٍ أَنْتَقِمُ مِنَ الأُمَمِ الَّذِينَ لَمْ يَسْمَعُوا».

\chapter{6}

\par 1 اِسْمَعُوا مَا قَالَهُ الرَّبُّ: «قُمْ خَاصِمْ لَدَى الْجِبَالِ وَلْتَسْمَعِ التِّلاَلُ صَوْتَكَ.
\par 2 اِسْمَعِي خُصُومَةَ الرَّبِّ أَيَّتُهَا الْجِبَالُ وَيَا أُسُسَ الأَرْضِ الدَّائِمَةَ. فَإِنَّ لِلرَّبِّ خُصُومَةً مَعَ شَعْبِهِ وَهُوَ يُحَاكِمُ إِسْرَائِيلَ.
\par 3 «يَا شَعْبِي مَاذَا صَنَعْتُ بِكَ وَبِمَاذَا أَضْجَرْتُكَ؟ اشْهَدْ عَلَيَّ!
\par 4 إِنِّي أَصْعَدْتُكَ مِنْ أَرْضِ مِصْرَ وَفَكَكْتُكَ مِنْ بَيْتِ الْعُبُودِيَّةِ وَأَرْسَلْتُ أَمَامَكَ مُوسَى وَهَارُونَ وَمَرْيَمَ.
\par 5 يَا شَعْبِي اذْكُرْ بِمَاذَا تَآمَرَ بَالاَقُ مَلِكُ مُوآبَ وَبِمَاذَا أَجَابَهُ بَلْعَامُ بْنُ بَعُورَ - مِنْ شِطِّيمَ إِلَى الْجِلْجَالِ - لِتَعْرِفَ إِجَادَةَ الرَّبِّ».
\par 6 بِمَ أَتَقَدَّمُ إِلَى الرَّبِّ وَأَنْحَنِي لِلإِلَهِ الْعَلِيِّ؟ هَلْ أَتَقَدَّمُ بِمُحْرَقَاتٍ بِعُجُولٍ أَبْنَاءِ سَنَةٍ؟
\par 7 هَلْ يُسَرُّ الرَّبُّ بِأُلُوفِ الْكِبَاشِ بِرَبَوَاتِ أَنْهَارِ زَيْتٍ؟ هَلْ أُعْطِي بِكْرِي عَنْ مَعْصِيَتِي ثَمَرَةَ جَسَدِي عَنْ خَطِيَّةِ نَفْسِي؟
\par 8 قَدْ أَخْبَرَكَ أَيُّهَا الإِنْسَانُ مَا هُوَ صَالِحٌ وَمَاذَا يَطْلُبُهُ مِنْكَ الرَّبُّ إِلاَّ أَنْ تَصْنَعَ الْحَقَّ وَتُحِبَّ الرَّحْمَةَ وَتَسْلُكَ مُتَوَاضِعاً مَعَ إِلَهِكَ.
\par 9 صَوْتُ الرَّبِّ يُنَادِي لِلْمَدِينَةِ وَالْحِكْمَةُ تَرَى اسْمَكَ: «اِسْمَعُوا لِلْقَضِيبِ وَمَنْ رَسَمَهُ.
\par 10 أَفِي بَيْتِ الشِّرِّيرِ بَعْدُ كُنُوزُ شَرٍّ وَإِيفَةٌ نَاقِصَةٌ مَلْعُونَةٌ؟
\par 11 هَلْ أَتَزَكَّى مَعَ مَوَازِينِ الشَّرِّ وَمَعَ كِيسِ مَعَايِيرِ الْغِشِّ؟
\par 12 فَإِنَّ أَغْنِيَاءَهَا مَلآنُونَ ظُلْماً وَسُكَّانَهَا يَتَكَلَّمُونَ بِالْكَذِبِ وَلِسَانَهُمْ فِي فَمِهِمْ غَاشٌّ.
\par 13 فَأَنَا قَدْ جَعَلْتُ جُرُوحَكَ عَدِيمَةَ الشِّفَاءِ مُخْرِباً مِنْ أَجْلِ خَطَايَاكَ.
\par 14 أَنْتَ تَأْكُلُ وَلاَ تَشْبَعُ وَجُوعُكَ فِي جَوْفِكَ. وَتُعَزِّلُ وَلاَ تُنَجِّي وَالَّذِي تُنَجِّيهِ أَدْفَعُهُ إِلَى السَّيْفِ.
\par 15 أَنْتَ تَزْرَعُ وَلاَ تَحْصُدُ. أَنْتَ تَدُوسُ زَيْتُوناً وَلاَ تَدَّهِنُ بِزَيْتٍ وَسُلاَفَةً وَلاَ تَشْرَبُ خَمْراً.
\par 16 وَتُحْفَظُ فَرَائِضُ «عُمْرِي» وَجَمِيعُ أَعْمَالِ بَيْتِ «أَخْآبَ» وَتَسْلُكُونَ بِمَشُورَاتِهِمْ لِكَيْ أُسَلِّمَكَ لِلْخَرَابِ وَسُكَّانَهَا لِلصَّفِيرِ فَتَحْمِلُونَ عَارَ شَعْبِي».

\chapter{7}

\par 1 وَيْلٌ لِي لأَنِّي صِرْتُ كَجَنَى الصَّيْفِ كَخُصَاصَةِ الْقِطَافِ. لاَ عُنْقُودَ لِلأَكْلِ وَلاَ بَاكُورَةَ تِينَةٍ اشْتَهَتْهَا نَفْسِي.
\par 2 قَدْ بَادَ التَّقِيُّ مِنَ الأَرْضِ وَلَيْسَ مُسْتَقِيمٌ بَيْنَ النَّاسِ. جَمِيعُهُمْ يَكْمُنُونَ لِلدِّمَاءِ يَصْطَادُونَ بَعْضُهُمْ بَعْضاً بِشَبَكَةٍ.
\par 3 اَلْيَدَانِ إِلَى الشَّرِّ مُجْتَهِدَتَانِ. الرَّئِيسُ وَالْقَاضِي طَالِبٌ بِالْهَدِيَّةِ وَالْكَبِيرُ مُتَكَلِّمٌ بِهَوَى نَفْسِهِ فَيُعَكِّشُونَهَا.
\par 4 أَحْسَنُهُمْ مِثْلُ الْعَوْسَجِ وَأَعْدَلُهُمْ مِنْ سِيَاجِ الشَّوْكِ! يَوْمَ مُرَاقِبِيكَ عِقَابُكَ قَدْ جَاءَ. الآنَ يَكُونُ ارْتِبَاكُهُمْ.
\par 5 لاَ تَأْتَمِنُوا صَاحِباً. لاَ تَثِقُوا بِصَدِيقٍ. احْفَظْ أَبْوَابَ فَمِكَ عَنِ الْمُضْطَجِعَةِ فِي حِضْنِكَ.
\par 6 لأَنَّ الاِبْنَ مُسْتَهِينٌ بِالأَبِ وَالْبِنْتَ قَائِمَةٌ عَلَى أُمِّهَا وَالْكَنَّةَ عَلَى حَمَاتِهَا وَأَعْدَاءُ الإِنْسَانِ أَهْلُ بَيْتِهِ.
\par 7 وَلَكِنَّنِي أُرَاقِبُ الرَّبَّ أَصْبِرُ لإِلَهِ خَلاَصِي. يَسْمَعُنِي إِلَهِي.
\par 8 لاَ تَشْمَتِي بِي يَا عَدُوَّتِي. إِذَا سَقَطْتُ أَقُومُ. إِذَا جَلَسْتُ فِي الظُّلْمَةِ فَالرَّبُّ نُورٌ لِي.
\par 9 أَحْتَمِلُ غَضَبَ الرَّبِّ لأَنِّي أَخْطَأْتُ إِلَيْهِ حَتَّى يُقِيمَ دَعْوَايَ وَيُجْرِيَ حَقِّي. سَيُخْرِجُنِي إِلَى النُّورِ. سَأَنْظُرُ بِرَّهُ.
\par 10 وَتَرَى عَدُوَّتِي فَيُغَطِّيهَا الْخِزْيُ الْقَائِلَةُ لِي: «أَيْنَ هُوَ الرَّبُّ إِلَهُكِ؟» عَيْنَايَ سَتَنْظُرَانِ إِلَيْهَا. الآنَ تَصِيرُ لِلدَّوْسِ كَطِينِ الأَزِقَّةِ.
\par 11 يَوْمَ بِنَاءِ حِيطَانِكِ ذَلِكَ الْيَوْمَ يَبْعُدُ الْمِيعَادُ.
\par 12 هُوَ يَوْمٌ يَأْتُونَ إِلَيْكِ مِنْ أَشُّورَ وَمُدُنِ مِصْرَ وَمِنْ مِصْرَ إِلَى النَّهْرِ. وَمِنَ الْبَحْرِ إِلَى الْبَحْرِ. وَمِنَ الْجَبَلِ إِلَى الْجَبَلِ.
\par 13 وَلَكِنْ تَصِيرُ الأَرْضُ خَرِبَةً بِسَبَبِ سُكَّانِهَا مِنْ أَجْلِ ثَمَرِ أَفْعَالِهِمْ.
\par 14 اِرْعَ بِعَصَاكَ شَعْبَكَ غَنَمَ مِيرَاثِكَ سَاكِنَةً وَحْدَهَا فِي وَعْرٍ فِي وَسَطِ الْكَرْمَلِ. لِتَرْعَ فِي بَاشَانَ وَجِلْعَادَ كَأَيَّامِ الْقِدَمِ.
\par 15 كَأَيَّامِ خُرُوجِكَ مِنْ أَرْضِ مِصْرَ أُرِيهِ عَجَائِبَ.
\par 16 يَنْظُرُ الأُمَمُ وَيَخْجَلُونَ مِنْ كُلِّ بَطْشِهِمْ. يَضَعُونَ أَيْدِيَهُمْ عَلَى أَفْوَاهِهِمْ وَتَصُمُّ آذَانُهُمْ.
\par 17 يَلْحَسُونَ التُّرَابَ كَالْحَيَّةِ كَزَوَاحِفِ الأَرْضِ. يَخْرُجُونَ بِالرِّعْدَةِ مِنْ حُصُونِهِمْ يَأْتُونَ بِالرُّعْبِ إِلَى الرَّبِّ إِلَهِنَا وَيَخَافُونَ مِنْكَ.
\par 18 مَنْ هُوَ إِلَهٌ مِثْلُكَ غَافِرٌ الإِثْمَ وَصَافِحٌ عَنِ الذَّنْبِ لِبَقِيَّةِ مِيرَاثِهِ! لاَ يَحْفَظُ إِلَى الأَبَدِ غَضَبَهُ فَإِنَّهُ يُسَرُّ بِالرَّأْفَةِ.
\par 19 يَعُودُ يَرْحَمُنَا يَدُوسُ آثَامَنَا وَتُطْرَحُ فِي أَعْمَاقِ الْبَحْرِ جَمِيعُ خَطَايَاهُمْ.
\par 20 تَصْنَعُ الأَمَانَةَ لِيَعْقُوبَ وَالرَّأْفَةَ لإِبْرَاهِيمَ اللَّتَيْنِ حَلَفْتَ لِآبَائِنَا مُنْذُ أَيَّامِ الْقِدَمِ.


\end{document}