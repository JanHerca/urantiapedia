\begin{document}

\title{وصية آشير}

\chapter{1}

آشر، الابن العاشر ليعقوب وزلفة. شرحٌ للشخصية المزدوجة. أول قصة عن جيكل وهايد. للاطلاع على بيان قانون التعويض الذي كان سيستمتع به إيمرسون، انظر الآية ٢٧.

\par 1 نسخة العهد إلى آشير، ما تكلم به لبنيه في السنة المائة والخامسة والعشرين من حياته.

\par 2 "فبينما هو بعد في صحته قال لهم: اسمعوا يا بني آشير لأبيكم فأخبركم بكل ما هو مستقيم في عيني الرب."

\par 3 لقد أعطى الله لأبناء البشر طريقين، وميولين، ونوعين من العمل، ووسيلتين للعمل، ونتيجتين.

\par 4 لذلك فإن كل الأشياء تكون أثنيناً، واحداً مقابل الآخر.

\par 5 لأن هناك طريقين للخير والشر، ومعهما يوجد الميلان في صدورنا للتمييز بينهما.

\par 6 لذلك، إذا كانت النفس تستمتع بالميل الصالح، فإن جميع أفعالها تكون في البر؛ وإذا أخطأت فإنها تتوب على الفور.

\par 7 فإنه عندما يركز أفكاره على البر، ويرفض الشر، فإنه يهدم الشر على الفور، ويستأصل الخطيئة.

\par 8 "ولكن إن مالت إلى الميول الشريرة فكل أفعالها تكون في الشر وتطرد الخير وتلتصق بالشر وتخضع لبليعار. وإن عملت الخير فهو يحرفها إلى الشر."

\par 9 لأنه كلما بدأ بفعل الخير، فإنه يفرض نتيجة الفعل عليه بالشر، لأن كنز الميل مملوء بروح شريرة.

\par 10 قد يساعد الإنسان بالكلام على الخير من أجل الشر، إلا أن نتيجة الفعل تؤدي إلى الشر.

\par 11 هناك رجل لا يرحم من يخدم دوره في الشر، وهذا الشيء له جانبان، ولكن كله شرير.

\par 12 ويوجد إنسان يحب من يعمل الشر، لأنه يفضل أن يموت في الشر من أجله؛ وفيما يتعلق بهذا الأمر من الواضح أن له جانبين، ولكن كله عمل شرير.

\par 13 وإن كان فيه محبة فهو شرير من يخفي الشر من أجل السمعة الطيبة، ولكن نهاية العمل تميل إلى الشر.

\par 14 وآخر يسرق ويفعل الظلم وينهب ويحتال ويشفق على الفقراء: وهذا أيضاً له جانبان، لكن المجموع شرير.

\par 15 من يخدع قريبه يغيظ الله ويحلف على العلي كذباً ومع ذلك يشفق على المسكين. الرب الذي أمر بالشريعة يحتقره ويغيظه ومع ذلك ينعش المسكين.

\par 16 إنه ينجس النفس ويجعل الجسد مرحًا، يقتل كثيرين ويرحم قليلين: هذا أيضًا له جانب مزدوج، ولكن الكل شرير.

\par 17 "وآخر يزني ويزني ويمتنع عن الأطعمة، وعندما يصوم يفعل الشر، وبقوة غناه يغلب كثيرين، ومع ذلك فهو شرير للغاية يفعل الوصايا. هذا أيضا له وجهان، ولكن الكل شرير."

\par 18 هؤلاء الرجال هم أرانب، طاهرون، مثل أولئك الذين يقسمون الحافر، ولكنهم في الحقيقة نجسون.

\par 19 لأن الله أعلن هكذا في ألواح الوصايا.

\par 20 ولكن لا تلبسوا يا أبنائي وجهين مثلهما، وجه الخير ووجه الشر، بل تمسكوا بالخير فقط، لأن الله له مسكنه فيه، والناس يرغبون فيه.

\par 21 "ولكن اهربوا من الشر، وهدموا الميل الشرير بأعمالكم الصالحة. لأن الذين لهم وجهان لا يخدمون الله بل شهواتهم الخاصة لكي يرضوا الكافرين والناس الذين يشبهونهم."

\par 22 لأن الرجال الصالحين، حتى أولئك الذين لديهم وجه واحد، على الرغم من أنهم يظنون من قبل أولئك الذين لديهم وجهان للخطيئة، هم أبرار أمام الله.

\par 23 فإن كثيرين عندما يقتلون الأشرار يفعلون عملين: الخير والشر، ولكن الكل صالح، لأنه اقتلع ودمر ما هو شرير.

\par 24 "إن رجلاً يكره الرجل الرحيم والظالم، والرجل الذي يزني ويصوم: وهذا أيضاً له جانبان، ولكن العمل كله صالح، لأنه يتبع مثال الرب، في أنه لا يقبل الخير الظاهر على أنه خير حقيقي.

\par 25 وآخر لا يرغب في رؤية يوم جيد مع من لا يرغب، لئلا ينجس جسده ويدنس نفسه. هذا أيضا ذو وجهين، ولكن الكل جيد.

\par 26 فإن مثل هؤلاء الرجال يشبهون الغزلان والظباء، لأنهم في ظاهرهم نجسون، ولكنهم طاهرون كل الطهارة، لأنهم يسيرون في غيرة للرب، ويمتنعون عما يبغضه الله ويحرمه بوصاياه، فيبعدون الشر عن الخير.

\par 27 "انظروا يا أبنائي، كيف أن هناك اثنين في كل شيء، أحدهما ضد الآخر، وواحد مخفي بواسطة الآخر: في الغنى الطمع مخفي، وفي الملذات سكر، وفي الضحك حزن، وفي الزواج فجور.

\par 28 "الموت يلي الحياة، والعار يلي المجد، والليل نهاراً، والظلمة نهاراً. وكل الأشياء تحت النهار، الأشياء الصالحة تحت الحياة، والأشياء غير الصالحة تحت الموت. لذلك تنتظر الحياة الأبدية الموت أيضاً."

\par 29 ولا يجوز أن يقال إن الحقيقة كذب، ولا إن الصواب خطأ؛ لأن كل الحقيقة هي تحت النور، كما أن كل الأشياء هي تحت الله.

\par 30 "فلقد اختبرتُ كل هذه الأشياء في حياتي، ولم أبتعد عن حقيقة الرب، بل بحثتُ عن وصايا العلي، وسرت بكل قوتي وبكل بساطة نحو ما هو صالح."

\par 31 فاحذروا أنتم أيضًا، يا أبنائي، من وصايا الرب، واتبعوا الحق بتواضع.

\par 32 "فإن الذين لهم وجهان هم مذنبون بخطيئة مضاعفة، إذ يفعلون الشر ويسعدون بمن يفعلونه، متبعين مثال أرواح الضلال، ومجاهدين ضد البشر.

\par 33 "فاحفظوا أنتم يا أبنائي ناموس الرب، ولا تنظروا إلى الشر كما تنظرون إلى الخير، بل انظروا إلى ما هو صالح حقاً، واحفظوه في كل وصايا الرب، وسلوكوا فيه، واستريحوا فيه."

\par 34 فإن نهايات البشر تظهر برهم أو إثمهم عندما يقابلون ملائكة الرب وملائكة الشيطان.

\par 35 لأنه عندما تغادر الروح مضطربة، فإنها تتعذب بالروح الشريرة التي خدمتها أيضًا في الشهوات والأعمال الشريرة.

\par 36 ولكن إذا كان مسالماً فرحاً فإنه يلتقي بملاك السلام، فيقوده إلى الحياة الأبدية.

\par 37 لا تعودوا يا أبنائي مثل سدوم التي أخطأت ضد ملائكة الرب وهلكت إلى الأبد.

\par 38 لأني أعلم أنكم ستخطئون وتسلمون إلى أيدي أعدائكم، وتصير أرضكم خرابا، ومقدساتكم خربت، وتتبددون إلى أربعة أطراف الأرض.

\par 39 وتكونون مهجورين في التبدد الفاني كالماء.

\par 40 حتى يزور العلي الأرض، ويأتي بنفسه كإنسان، فيأكل ويشرب الناس، ويكسر رأس التنين في الماء.

\par 41 "فإنه يخلص إسرائيل وجميع الأمم، الله يتكلم في شخص الإنسان."

\par 42 لذلك أنتم أيضًا، يا أبنائي، أخبروا أولادكم بهذه الأمور حتى لا يعصوه.

\par 43 لأني أعلم أنكم ستعصيون لا محالة، وستفعلون الشر لا تبعون ناموس الله، بل وصايا الناس، فاسدين في الشر.

\par 44 "ولذلك تتشتتون مثل جاد ودان إخوتي، ولا تعرفون أراضيكم ولا سبطكم ولا لغتكم."

\par 45 ولكن الرب يجمعكم بالإيمان برحمته، ومن أجل إبراهيم وإسحق ويعقوب.

\par 46 ولما قال لهم هذا أوصاهم قائلا: ادفنوني في حبرون.

\par 47 ونام ومات عن شيخوخة صالحة.

\par 48 ففعل بنوه كما أمرهم، فحملوه إلى حبرون ودفنوه مع آبائه.

\end{document}