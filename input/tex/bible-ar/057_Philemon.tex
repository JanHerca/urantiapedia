\begin{document}

\title{فليمون}


\chapter{1}

\par 1 بُولُسُ، أَسِيرُ يَسُوعَ الْمَسِيحِ، وَتِيمُوثَاوُسُ الأَخُ: إِلَى فِلِيمُونَ الْمَحْبُوبِ وَالْعَامِلِ مَعَنَا،
\par 2 وَإِلَى أَبْفِيَّةَ الْمَحْبُوبَةِ، وَأَرْخِبُّسَ الْمُتَجَنِّدِ مَعَنَا، وَإِلَى الْكَنِيسَةِ الَّتِي فِي بَيْتِكَ.
\par 3 نِعْمَةٌ لَكُمْ وَسَلاَمٌ مِنَ اللهِ أَبِينَا وَالرَّبِّ يَسُوعَ الْمَسِيحِ.
\par 4 أَشْكُرُ إِلَهِي كُلَّ حِينٍ ذَاكِراً إِيَّاكَ فِي صَلَوَاتِي،
\par 5 سَامِعاً بِمَحَبَّتِكَ، وَالإِيمَانِ الَّذِي لَكَ نَحْوَ الرَّبِّ يَسُوعَ وَلِجَمِيعِ الْقِدِّيسِينَ،
\par 6 لِكَيْ تَكُونَ شَرِكَةُ إِيمَانِكَ فَعَّالَةً فِي مَعْرِفَةِ كُلِّ الصَّلاَحِ الَّذِي فِيكُمْ لأَجْلِ الْمَسِيحِ يَسُوعَ.
\par 7 لأَنَّ لَنَا فَرَحاً كَثِيراً وَتَعْزِيَةً بِسَبَبِ مَحَبَّتِكَ، لأَنَّ أَحْشَاءَ الْقِدِّيسِينَ قَدِ اسْتَرَاحَتْ بِكَ أَيُّهَا الأَخُ.
\par 8 لِذَلِكَ، وَإِنْ كَانَ لِي بِالْمَسِيحِ ثِقَةٌ كَثِيرَةٌ أَنْ آمُرَكَ بِمَا يَلِيقُ،
\par 9 مِنْ أَجْلِ الْمَحَبَّةِ، أَطْلُبُ بِالْحَرِيِّ إِذْ أَنَا إِنْسَانٌ هَكَذَا نَظِيرُ بُولُسَ الشَّيْخِ، وَالآنَ أَسِيرُ يَسُوعَ الْمَسِيحِ أَيْضاً -
\par 10 أَطْلُبُ إِلَيْكَ لأَجْلِ ابْنِي أُنِسِيمُسَ، الَّذِي وَلَدْتُهُ فِي قُيُودِي،
\par 11 الَّذِي كَانَ قَبْلاً غَيْرَ نَافِعٍ لَكَ، وَلَكِنَّهُ الآنَ نَافِعٌ لَكَ وَلِي،
\par 12 الَّذِي رَدَدْتُهُ. فَاقْبَلْهُ، الَّذِي هُوَ أَحْشَائِي.
\par 13 الَّذِي كُنْتُ أَشَاءُ أَنْ أُمْسِكَهُ عِنْدِي لِكَيْ يَخْدِمَنِي عِوَضاً عَنْكَ فِي قُيُودِ الإِنْجِيلِ -
\par 14 وَلَكِنْ بِدُونِ رَأْيِكَ لَمْ أُرِدْ أَنْ أَفْعَلَ شَيْئاً، لِكَيْ لاَ يَكُونَ خَيْرُكَ كَأَنَّهُ عَلَى سَبِيلِ الِاضْطِرَارِ بَلْ عَلَى سَبِيلِ الِاخْتِيَارِ.
\par 15 لأَنَّهُ رُبَّمَا لأَجْلِ هَذَا افْتَرَقَ عَنْكَ إِلَى سَاعَةٍ، لِكَيْ يَكُونَ لَكَ إِلَى الأَبَدِ،
\par 16 لاَ كَعَبْدٍ فِي مَا بَعْدُ، بَلْ أَفْضَلَ مِنْ عَبْدٍ: أَخاً مَحْبُوباً، وَلاَ سِيَّمَا إِلَيَّ. فَكَمْ بِالْحَرِيِّ إِلَيْكَ فِي الْجَسَدِ وَالرَّبِّ جَمِيعاً!
\par 17 فَإِنْ كُنْتَ تَحْسِبُنِي شَرِيكاً فَاقْبَلْهُ نَظِيرِي.
\par 18 ثُمَّ إِنْ كَانَ قَدْ ظَلَمَكَ بِشَيْءٍ، أَوْ لَكَ عَلَيْهِ دَيْنٌ، فَاحْسِبْ ذَلِكَ عَلَيَّ.
\par 19 أَنَا بُولُسَ كَتَبْتُ بِيَدِي. أَنَا أُوفِي. حَتَّى لاَ أَقُولُ لَكَ إِنَّكَ مَدْيُونٌ لِي بِنَفْسِكَ أَيْضاً.
\par 20 نَعَمْ أَيُّهَا الأَخُ، لِيَكُنْ لِي فَرَحٌ بِكَ فِي الرَّبِّ. أَرِحْ أَحْشَائِي فِي الرَّبِّ.
\par 21 إِذْ أَنَا وَاثِقٌ بِإِطَاعَتِكَ كَتَبْتُ إِلَيْكَ، عَالِماً أَنَّكَ تَفْعَلُ أَيْضاً أَكْثَرَ مِمَّا أَقُولُ.
\par 22 وَمَعَ هَذَا أَعْدِدْ لِي أَيْضاً مَنْزِلاً، لأَنِّي أَرْجُو أَنَّنِي بِصَلَوَاتِكُمْ سَأُوهَبُ لَكُمْ.
\par 23 يُسَلِّمُ عَلَيْكَ أَبَفْرَاسُ الْمَأْسُورُ مَعِي فِي الْمَسِيحِ يَسُوعَ،
\par 24 وَمَرْقُسُ، وَأَرِسْتَرْخُسُ، وَدِيمَاسُ، وَلُوقَا الْعَامِلُونَ مَعِي.
\par 25 نِعْمَةُ رَبِّنَا يَسُوعَ الْمَسِيحِ مَعَ رُوحِكُمْ. آمِينَ.

\end{document}