\begin{document}

\title{يوئيل}


\chapter{1}

\par 1 قَوْلُ الرَّبِّ الَّذِي صَارَ إِلَى يُوئِيلَ بْنِ فَثُوئِيلَ:
\par 2 اِسْمَعُوا هَذَا أَيُّهَا الشُّيُوخُ وَأَصْغُوا يَا جَمِيعَ سُكَّانِ الأَرْضِ. هَلْ حَدَثَ هَذَا فِي أَيَّامِكُمْ أَوْ فِي أَيَّامِ آبَائِكُمْ؟
\par 3 أَخْبِرُوا بَنِيكُمْ عَنْهُ وَبَنُوكُمْ بَنِيهِمْ وَبَنُوهُمْ دَوْراً آخَرَ.
\par 4 فَضْلَةُ الْقَمَصِ أَكَلَهَا الزَّحَّافُ وَفَضْلَةُ الزَّحَّافِ أَكَلَهَا الْغَوْغَاءُ وَفَضْلَةُ الْغَوْغَاءِ أَكَلَهَا الطَّيَّارُ.
\par 5 اِصْحُوا أَيُّهَا السَّكَارَى وَابْكُوا وَوَلْوِلُوا يَا جَمِيعَ شَارِبِي الْخَمْرِ عَلَى الْعَصِيرِ لأَنَّهُ انْقَطَعَ عَنْ أَفْوَاهِكُمْ.
\par 6 إِذْ قَدْ صَعِدَتْ عَلَى أَرْضِي أُمَّةٌ قَوِيَّةٌ بِلاَ عَدَدٍ أَسْنَانُهَا أَسْنَانُ الأَسَدِ وَلَهَا أَضْرَاسُ اللَّبْوَةِ.
\par 7 جَعَلَتْ كَرْمَتِي خَرِبَةً وَتِينَتِي مُتَهَشِّمَةً. قَدْ قَشَرَتْهَا وَطَرَحَتْهَا فَابْيَضَّتْ قُضْبَانُهَا.
\par 8 نُوحِي يَا أَرْضِي كَعَرُوسٍ مُؤْتَزِرَةٍ بِمِسْحٍ مِنْ أَجْلِ بَعْلِ صِبَاهَا.
\par 9 انْقَطَعَتِ التَّقْدِمَةُ وَالسَّكِيبُ عَنْ بَيْتِ الرَّبِّ. نَاحَتِ الْكَهَنَةُ خُدَّامُ الرَّبِّ.
\par 10 تَلِفَ الْحَقْلُ نَاحَتِ الأَرْضُ لأَنَّهُ قَدْ تَلِفَ الْقَمْحُ جَفَّ الْمِسْطَارُ ذَبُلَ الزَّيْتُ.
\par 11 خَجِلَ الْفَلاَّحُونَ. وَلْوَلَ الْكَرَّامُونَ عَلَى الْحِنْطَةِ وَعَلَى الشَّعِيرِ لأَنَّهُ قَدْ تَلِفَ حَصِيدُ الْحَقْلِ.
\par 12 اَلْجَفْنَةُ يَبِسَتْ وَالتِّينَةُ ذَبُلَتْ. الرُّمَّانَةُ وَالنَّخْلَةُ وَالتُّفَّاحَةُ كُلُّ أَشْجَارِ الْحَقْلِ يَبِسَتْ. إِنَّهُ قَدْ يَبِسَتِ الْبَهْجَةُ مِنْ بَنِي الْبَشَرِ.
\par 13 تَنَطَّقُوا وَنُوحُوا أَيُّهَا الْكَهَنَةُ. وَلْوِلُوا يَا خُدَّامَ الْمَذْبَحِ. ادْخُلُوا بِيتُوا بِالْمُسُوحِ يَا خُدَّامَ إِلَهِي لأَنَّهُ قَدِ امْتَنَعَ عَنْ بَيْتِ إِلَهِكُمُ التَّقْدِمَةُ وَالسَّكِيبُ.
\par 14 قَدِّسُوا صَوْماً. نَادُوا بِاعْتِكَافٍ. اجْمَعُوا الشُّيُوخَ جَمِيعَ سُكَّانِ الأَرْضِ إِلَى بَيْتِ الرَّبِّ إِلَهِكُمْ وَاصْرُخُوا إِلَى الرَّبِّ.
\par 15 آهِ عَلَى الْيَوْمِ لأَنَّ يَوْمَ الرَّبِّ قَرِيبٌ. يَأْتِي كَخَرَابٍ مِنَ الْقَادِرِ عَلَى كُلِّ شَيْءٍ.
\par 16 أَمَا انْقَطَعَ الطَّعَامُ تُجَاهَ عُيُونِنَا؟ الْفَرَحُ وَالاِبْتِهَاجُ عَنْ بَيْتِ إِلَهِنَا؟
\par 17 عَفَّنَتِ الْحُبُوبُ تَحْتَ مَدَرِهَا. خَلَتِ الأَهْرَاءُ. انْهَدَمَتِ الْمَخَازِنُ لأَنَّهُ قَدْ يَبِسَ الْقَمْحُ.
\par 18 كَمْ تَئِنُّ الْبَهَائِمُ! هَامَتْ قُطْعَانُ الْبَقَرِ لأَنْ لَيْسَ لَهَا مَرْعًى. حَتَّى قُطْعَانُ الْغَنَمِ تَفْنَى.
\par 19 إِلَيْكَ يَا رَبُّ أَصْرُخُ لأَنَّ نَاراً قَدْ أَكَلَتْ مَرَاعِيَ الْبَرِّيَّةِ وَلَهِيباً أَحْرَقَ جَمِيعَ أَشْجَارِ الْحَقْلِ.
\par 20 حَتَّى بَهَائِمُ الصَّحْرَاءِ تَنْظُرُ إِلَيْكَ لأَنَّ جَدَاوِلَ الْمِيَاهِ قَدْ جَفَّتْ وَالنَّارَ أَكَلَتْ مَرَاعِيَ الْبَرِّيَّةِ.

\chapter{2}

\par 1 اِضْرِبُوا بِالْبُوقِ فِي صِهْيَوْنَ. صَوِّتُوا فِي جَبَلِ قُدْسِي. لِيَرْتَعِدْ جَمِيعُ سُكَّانِ الأَرْضِ لأَنَّ يَوْمَ الرَّبِّ قَادِمٌ لأَنَّهُ قَرِيبٌ.
\par 2 يَوْمُ ظَلاَمٍ وَقَتَامٍ. يَوْمُ غَيْمٍ وَضَبَابٍ مِثْلَ الْفَجْرِ مُمْتَدّاً عَلَى الْجِبَالِ. شَعْبٌ كَثِيرٌ وَقَوِيٌّ لَمْ يَكُنْ نَظِيرُهُ مُنْذُ الأَزَلِ وَلاَ يَكُونُ أَيْضاً بَعْدَهُ إِلَى سِنِي دَوْرٍ فَدَوْرٍ.
\par 3 قُدَّامَهُ نَارٌ تَأْكُلُ وَخَلْفَهُ لَهِيبٌ يُحْرِقُ. الأَرْضُ قُدَّامَهُ كَجَنَّةِ عَدْنٍ وَخَلْفَهُ قَفْرٌ خَرِبٌ وَلاَ تَكُونُ مِنْهُ نَجَاةٌ.
\par 4 كَمَنْظَرِ الْخَيْلِ مَنْظَرُهُ وَمِثْلَ الأَفْرَاسِ يَرْكُضُونَ.
\par 5 كَصَرِيفِ الْمَرْكَبَاتِ عَلَى رُؤُوسِ الْجِبَالِ يَثِبُونَ. كَزَفِيرِ لَهِيبِ نَارٍ تَأْكُلُ قَشّاً. كَقَوْمٍ أَقْوِيَاءَ مُصْطَفِّينَ لِلْقِتَالِ.
\par 6 مِنْهُ تَرْتَعِدُ الشُّعُوبُ. كُلُّ الْوُجُوهِ تَجْمَعُ حُمْرَةً.
\par 7 يَجْرُونَ كَأَبْطَالٍ. يَصْعَدُونَ السُّورَ كَرِجَالِ الْحَرْبِ وَيَمْشُونَ كُلُّ وَاحِدٍ فِي طَرِيقِهِ وَلاَ يُغَيِّرُونَ سُبُلَهُمْ.
\par 8 وَلاَ يُزَاحِمُ بَعْضُهُمْ بَعْضاً. يَمْشُونَ كُلُّ وَاحِدٍ فِي سَبِيلِهِ وَبَيْنَ الأَسْلِحَةِ يَقَعُونَ وَلاَ يَنْكَسِرُونَ.
\par 9 يَتَرَاكَضُونَ فِي الْمَدِينَةِ. يَجْرُونَ عَلَى السُّورِ. يَصْعَدُونَ إِلَى الْبُيُوتِ. يَدْخُلُونَ مِنَ الْكُوى كَاللِّصِّ.
\par 10 قُدَّامَهُ تَرْتَعِدُ الأَرْضُ وَتَرْجُفُ السَّمَاءُ. الشَّمْسُ وَالْقَمَرُ يُظْلِمَانِ وَالنُّجُومُ تَحْجِزُ لَمَعَانَهَا.
\par 11 وَالرَّبُّ يُعْطِي صَوْتَهُ أَمَامَ جَيْشِهِ. إِنَّ عَسْكَرَهُ كَثِيرٌ جِدّاً. فَإِنَّ صَانِعَ قَوْلِهِ قَوِيٌّ لأَنَّ يَوْمَ الرَّبِّ عَظِيمٌ وَمَخُوفٌ جِدّاً فَمَنْ يُطِيقُهُ؟
\par 12 وَلَكِنِ الآنَ يَقُولُ الرَّبُّ: «ارْجِعُوا إِلَيَّ بِكُلِّ قُلُوبِكُمْ وَبِالصَّوْمِ وَالْبُكَاءِ وَالنَّوْحِ».
\par 13 وَمَزِّقُوا قُلُوبَكُمْ لاَ ثِيَابَكُمْ وَارْجِعُوا إِلَى الرَّبِّ إِلَهِكُمْ لأَنَّهُ رَأُوفٌ رَحِيمٌ بَطِيءُ الْغَضَبِ وَكَثِيرُ الرَّأْفَةِ وَيَنْدَمُ عَلَى الشَّرِّ.
\par 14 لَعَلَّهُ يَرْجِعُ وَيَنْدَمُ فَيُبْقِيَ وَرَاءَهُ بَرَكَةَ تَقْدِمَةٍ وَسَكِيباً لِلرَّبِّ إِلَهِكُمْ.
\par 15 اِضْرِبُوا بِالْبُوقِ فِي صِهْيَوْنَ. قَدِّسُوا صَوْماً. نَادُوا بِاعْتِكَافٍ.
\par 16 اِجْمَعُوا الشَّعْبَ. قَدِّسُوا الْجَمَاعَةَ. احْشُِدُوا الشُّيُوخَ. اجْمَعُوا الأَطْفَالَ وَرَاضِعِي الثُّدِيِّ. لِيَخْرُجِ الْعَرِيسُ مِنْ مِخْدَعِهِ وَالْعَرُوسُ مِنْ حَجَلَتِهَا.
\par 17 لِيَبْكِ الْكَهَنَةُ خُدَّامُ الرَّبِّ بَيْنَ الرِّواقِ وَالْمَذْبَحِ وَيَقُولُوا: «اشْفِقْ يَا رَبُّ عَلَى شَعْبِكَ وَلاَ تُسَلِّمْ مِيرَاثَكَ لِلْعَارِ حَتَّى تَجْعَلَهُمُ الأُمَمُ مَثَلاً. لِمَاذَا يَقُولُونَ بَيْنَ الشُّعُوبِ: أَيْنَ إِلَهُهُمْ؟».
\par 18 فَيَغَارُ الرَّبُّ لأَرْضِهِ وَيَرِقُّ لِشَعْبِهِ.
\par 19 وَيُجِيبُ الرَّبُّ وَيَقُولُ لِشَعْبِهِ: «هَئَنَذَا مُرْسِلٌ لَكُمْ قَمْحاً وَمِسْطَاراً وَزَيْتاً لِتَشْبَعُوا مِنْهَا وَلاَ أَجْعَلُكُمْ أَيْضاً عَاراً بَيْنَ الأُمَمِ.
\par 20 وَالشِّمَالِيُّ أُبْعِدُهُ عَنْكُمْ وَأَطْرُدُهُ إِلَى أَرْضٍ نَاشِفَةٍ وَمُقْفِرَةٍ. مُقَدَّمَتُهُ إِلَى الْبَحْرِ الشَّرْقِيِّ وَسَاقَتُهُ إِلَى الْبَحْرِ الْغَرْبِيِّ فَيَصْعَدُ نَتَنُهُ وَتَطْلُعُ زُهْمَتُهُ لأَنَّهُ قَدْ تَصَلَّفَ فِي عَمَلِهِ».
\par 21 لاَ تَخَافِي أَيَّتُهَا الأَرْضُ. ابْتَهِجِي وَافْرَحِي لأَنَّ الرَّبَّ يُعَظِّمُ عَمَلَهُ.
\par 22 لاَ تَخَافِي يَا بَهَائِمَ الصَّحْرَاءِ فَإِنَّ مَرَاعِيَ الْبَرِّيَّةِ تَنْبُتُ لأَنَّ الأَشْجَارَ تَحْمِلُ ثَمَرَهَا التِّينَةُ وَالْكَرْمَةُ تُعْطِيَانِ قُوَّتَهُمَا.
\par 23 وَيَا بَنِي صِهْيَوْنَ ابْتَهِجُوا وَافْرَحُوا بِالرَّبِّ إِلَهِكُمْ لأَنَّهُ يُعْطِيكُمُ الْمَطَرَ الْمُبَكِّرَ عَلَى حَقِّهِ وَيُنْزِلُ عَلَيْكُمْ مَطَراً مُبَكِّراً وَمُتَأَخِّراً فِي أَوَّلِ الْوَقْتِ
\par 24 فَتُمْلَأُ الْبَيَادِرُ حِنْطَةً وَتَفِيضُ حِيَاضُ الْمَعَاصِرِ خَمْراً وَزَيْتاً.
\par 25 «وَأُعَوِّضُ لَكُمْ عَنِ السِّنِينَ الَّتِي أَكَلَهَا الْجَرَادُ الْغَوْغَاءُ وَالطَّيَّارُ وَالْقَمَصُ جَيْشِي الْعَظِيمُ الَّذِي أَرْسَلْتُهُ عَلَيْكُمْ.
\par 26 فَتَأْكُلُونَ أَكْلاً وَتَشْبَعُونَ وَتُسَبِّحُونَ اسْمَ الرَّبِّ إِلَهِكُمُ الَّذِي صَنَعَ مَعَكُمْ عَجَباً وَلاَ يَخْزَى شَعْبِي إِلَى الأَبَدِ.
\par 27 وَتَعْلَمُونَ أَنِّي أَنَا فِي وَسَطِ إِسْرَائِيلَ وَأَنِّي أَنَا الرَّبُّ إِلَهُكُمْ وَلَيْسَ غَيْرِي. وَلاَ يَخْزَى شَعْبِي إِلَى الأَبَدِ.
\par 28 «وَيَكُونُ بَعْدَ ذَلِكَ أَنِّي أَسْكُبُ رُوحِي عَلَى كُلِّ بَشَرٍ فَيَتَنَبَّأُ بَنُوكُمْ وَبَنَاتُكُمْ وَيَحْلَمُ شُيُوخُكُمْ أَحْلاَماً وَيَرَى شَبَابُكُمْ رُؤًى.
\par 29 وَعَلَى الْعَبِيدِ أَيْضاً وَعَلَى الإِمَاءِ أَسْكُبُ رُوحِي فِي تِلْكَ الأَيَّامِ
\par 30 وَأُعْطِي عَجَائِبَ فِي السَّمَاءِ وَالأَرْضِ دَماً وَنَاراً وَأَعْمِدَةَ دُخَانٍ.
\par 31 تَتَحَوَّلُ الشَّمْسُ إِلَى ظُلْمَةٍ وَالْقَمَرُ إِلَى دَمٍ قَبْلَ أَنْ يَجِيءَ يَوْمُ الرَّبِّ الْعَظِيمُ الْمَخُوفُ.
\par 32 وَيَكُونُ أَنَّ كُلَّ مَنْ يَدْعُو بِاسْمِ الرَّبِّ يَنْجُو». لأَنَّهُ فِي جَبَلِ صِهْيَوْنَ وَفِي أُورُشَلِيمَ تَكُونُ نَجَاةٌ. كَمَا قَالَ الرَّبُّ. وَبَيْنَ الْبَاقِينَ مَنْ يَدْعُوهُ الرَّبُّ.

\chapter{3}

\par 1 «لأَنَّهُ هُوَذَا فِي تِلْكَ الأَيَّامِ وَفِي ذَلِكَ الْوَقْتِ عِنْدَمَا أَرُدُّ سَبْيَ يَهُوذَا وَأُورُشَلِيمَ
\par 2 أَجْمَعُ كُلَّ الأُمَمِ وَأُنَزِّلُهُمْ إِلَى وَادِي يَهُوشَافَاطَ وَأُحَاكِمُهُمْ هُنَاكَ عَلَى شَعْبِي وَمِيرَاثِي إِسْرَائِيلَ الَّذِينَ بَدَّدُوهُمْ بَيْنَ الأُمَمِ وَقَسَمُوا أَرْضِي
\par 3 وَأَلْقُوا قُرْعَةً عَلَى شَعْبِي وَأَعْطَوُا الصَّبِيَّ لِزَانِيَةٍ وَبَاعُوا الْبِنْتَ بِخَمْرٍ لِيَشْرَبُوا.
\par 4 «وَمَاذَا أَنْتُنَّ لِي يَا صُورُ وَصَيْدُونُ وَجَمِيعَ دَائِرَةِ فِلِسْطِينَ؟ هَلْ تُكَافِئُونَنِي عَنِ الْعَمَلِ أَمْ هَلْ تَصْنَعُونَ بِي شَيْئاً؟ سَرِيعاً بِالْعَجَلِ أَرُدُّ عَمَلَكُمْ عَلَى رُؤُوسِكُمْ.
\par 5 لأَنَّكُمْ أَخَذْتُمْ فِضَّتِي وَذَهَبِي وَأَدْخَلْتُمْ نَفَائِسِي الْجَيِّدَةَ إِلَى هَيَاكِلِكُمْ.
\par 6 وَبِعْتُمْ بَنِي يَهُوذَا وَبَنِي أُورُشَلِيمَ لِبَنِي الْيَاوَانِيِّينَ لِتُبْعِدُوهُمْ عَنْ تُخُومِهِمْ.
\par 7 هَئَنَذَا أُنْهِضُهُمْ مِنَ الْمَوْضِعِ الَّذِي بِعْتُمُوهُمْ إِلَيْهِ وَأَرُدُّ عَمَلَكُمْ عَلَى رُؤُوسِكُمْ.
\par 8 وَأَبِيعُ بَنِيكُمْ وَبَنَاتِكُمْ بِيَدِ بَنِي يَهُوذَا لِيَبِيعُوهُمْ لِلسَّبَائِيِّينَ لِأُمَّةٍ بَعِيدَةٍ لأَنَّ الرَّبَّ قَدْ تَكَلَّمَ».
\par 9 نَادُوا بِهَذَا بَيْنَ الأُمَمِ. قَدِّسُوا حَرْباً. أَنْهِضُوا الأَبْطَالَ. لِيَتَقَدَّمْ وَيَصْعَدْ كُلُّ رِجَالِ الْحَرْبِ.
\par 10 اِطْبَعُوا سِكَّاتِكُمْ سُيُوفاً وَمَنَاجِلَكُمْ رِمَاحاً. لِيَقُلِ الضَّعِيفُ: بَطَلٌ أَنَا!
\par 11 أَسْرِعُوا وَهَلُمُّوا يَا جَمِيعَ الأُمَمِ مِنْ كُلِّ نَاحِيَةٍ وَاجْتَمِعُوا. إِلَى هُنَاكَ أَنْزِلْ يَا رَبُّ أَبْطَالَكَ.
\par 12 تَنْهَضُ وَتَصْعَدُ الأُمَمُ إِلَى وَادِي يَهُوشَافَاطَ لأَنِّي هُنَاكَ أَجْلِسُ لِأُحَاكِمَ جَمِيعَ الأُمَمِ مِنْ كُلِّ نَاحِيَةٍ.
\par 13 أَرْسِلُوا الْمِنْجَلَ لأَنَّ الْحَصِيدَ قَدْ نَضَجَ. هَلُمُّوا دُوسُوا لأَنَّهُ قَدِ امْتَلَأَتِ الْمِعْصَرَةُ. فَاضَتِ الْحِيَاضُ لأَنَّ شَرَّهُمْ كَثِيرٌ».
\par 14 جَمَاهِيرُ جَمَاهِيرُ فِي وَادِي الْقَضَاءِ لأَنَّ يَوْمَ الرَّبِّ قَرِيبٌ فِي وَادِي الْقَضَاءِ.
\par 15 اَلشَّمْسُ وَالْقَمَرُ يَظْلُمَانِ وَالنُّجُومُ تَحْجِزُ لَمَعَانَهَا.
\par 16 وَالرَّبُّ مِنْ صِهْيَوْنَ يُزَمْجِرُ. وَمِنْ أُورُشَلِيمَ يُعْطِي صَوْتَهُ فَتَرْجُفُ السَّمَاءُ وَالأَرْضُ. وَلَكِنَّ الرَّبَّ مَلْجَأٌ لِشَعْبِهِ وَحِصْنٌ لِبَنِي إِسْرَائِيلَ.
\par 17 «فَتَعْرِفُونَ أَنِّي أَنَا الرَّبُّ إِلَهُكُمْ سَاكِناً فِي صِهْيَوْنَ جَبَلِ قُدْسِي. وَتَكُونُ أُورُشَلِيمُ مُقَدَّسَةً وَلاَ يَجْتَازُ فِيهَا الأَعَاجِمُ فِي مَا بَعْدُ».
\par 18 وَيَكُونُ فِي ذَلِكَ الْيَوْمِ أَنَّ الْجِبَالَ تَقْطُرُ عَصِيراً وَالتِّلاَلَ تَفِيضُ لَبَناً وَجَمِيعَ يَنَابِيعِ يَهُوذَا تَفِيضُ مَاءً وَمِنْ بَيْتِ الرَّبِّ يَخْرُجُ يَنْبُوعٌ وَيَسْقِي وَادِي السَّنْطِ.
\par 19 «مِصْرُ تَصِيرُ خَرَاباً وَأَدُومُ تَصِيرُ قَفْراً خَرِباً مِنْ أَجْلِ ظُلْمِهِمْ لِبَنِي يَهُوذَا الَّذِينَ سَفَكُوا دَماً بَرِيئاً فِي أَرْضِهِمْ.
\par 20 وَلَكِنَّ يَهُوذَا تُسْكَنُ إِلَى الأَبَدِ وَأُورُشَلِيمَ إِلَى دَوْرٍ فَدَوْرٍ.
\par 21 وَأُبَرِّئُ دَمَهُمُ الَّذِي لَمْ أُبَرِّئْهُ وَالرَّبُّ يَسْكُنُ فِي صِهْيَوْنَ».


\end{document}