\begin{document}

\title{رسالة إرميا}

\chapter{1}


\par 1 نسخة من رسالة أرسلها إرميا إلى الذين كان ملك البابليين سيُسبونهم إلى بابل، ليُثبت لهم ما أمره به الله: بسبب خطاياكم التي ارتكبتموها أمام الله، سيُسبى بكم نبوخذنصر ملك البابليين إلى بابل.
\par 2 فمتى دخلتم بابل، ستبقون هناك سنينًا كثيرة، وزمانًا طويلًا، أي سبعة أجيال. وبعد ذلك أخرجكم من هناك بسلام
\par 3 الآن سترون في بابل آلهة من فضة وذهب وخشب محمولة على الأكتاف، تُرعب الأمم
\par 4 فاحذروا من أن تتشبهوا بالغرباء، ولا تكونوا منهم، عندما ترون الجموع أمامهم وخلفهم يسجدون لهم
\par 5 لكن قولوا في قلوبكم: يا رب، يجب أن نعبدك.
\par 6 لأن ملاكي معكم، وأنا أهتم بأرواحكم.
\par 7 أما ألسنتهم، فقد صقلت من قبل الصانع، وهم أنفسهم مطليون بالذهب والفضة؛ ومع ذلك فهم كاذبون، ولا يستطيعون التكلم
\par 8 ويأخذون الذهب، كما لو كان لعذراء تحب المرح، ويصنعون تيجانًا لرؤوس آلهتهم
\par 9 في بعض الأحيان، ينقل الكهنة أيضًا من آلهتهم الذهب والفضة، ويمنحونها لأنفسهم
\par 10 نعم، سيعطون منه للزانيات، ويزينونهن كالرجال بالثياب، [كونهم] آلهة من فضة، وآلهة من ذهب، وخشب
\par 11 ومع ذلك، لا تستطيع هذه الآلهة إنقاذ نفسها من الصدأ والعثة، حتى لو غطتها ثياب أرجوانية
\par 12 يمسحون وجوههم من غبار الهيكل، عندما يكون عليهم الكثير
\par 13 ومن لا يستطيع أن يقتل من يسيء إليه، فإنه يحمل صولجانًا، كما لو كان قاضيًا للبلاد
\par 14 وفي يده اليمنى خنجر وفأس، لكنه لا يستطيع أن ينقذ نفسه من الحرب واللصوص
\par 15 حيثُ يُعرف أنهم ليسوا آلهةً، فلا تخافوهم.
\par 16 فكما أن الإناء الذي يستعمله الإنسان لا قيمة له عندما ينكسر، كذلك آلهتهم أيضاً. متى أقاموا في الهيكل، تمتلئ عيونهم غباراً من تحت أقدام الذين يدخلون.
\par 17 وكما تُحكم الأبواب من كل جانب على من يُسيء إلى الملك، باعتباره محكومًا عليه بالموت، كذلك يُحكم الكهنة على معابدهم بالأبواب والأقفال والقضبان، لئلا تُنهب آلهتهم على يد اللصوص
\par 18 يُضيئون لهم شموعًا، بل أكثر من الشموع لأنفسهم، حيث لا يستطيعون رؤية واحدة منها
\par 19 إنهم كإحدى عوارض الهيكل، ومع ذلك يقولون إن قلوبهم تُقرضها أشياء زاحفة من الأرض؛ وعندما يأكلونها وملابسهم، لا يشعرون بذلك
\par 20 وجوههم سوداء بسبب الدخان المنبعث من الهيكل
\par 21 على أجسادهم ورؤوسهم تجلس الخفافيش والسنونو والطيور، وكذلك القطط
\par 22 وبهذا تعرفون أنهم ليسوا آلهة فلا تخافوهم.
\par 23 على الرغم من الذهب الذي يحيط بها ليجعلها جميلة، إلا أنها لن تلمع ما لم تمسح الصدأ: لأنها لم تشعر به أيضًا عندما سُكبت
\par 24 الأشياء التي لا روح فيها تُشترى بأغلى ثمن
\par 25 يُحملون على الأكتاف، بلا أقدام، فيُعلنون للناس أنهم لا قيمة لهم
\par 26 ويخزى أيضًا الذين يخدمونها، لأنها إن سقطت على الأرض لا تستطيع النهوض من تلقاء نفسها، ولا إن أقامها أحد لا تستطيع أن تتحرك من تلقاء نفسها، ولا إن انحنت لا تستطيع أن تقيم نفسها، بل تقدم أمامها الهدايا كما للأموات
\par 27 أما الأشياء التي تُذبح لهم، فيبيعها كهنتهم ويُسيئون استخدامها؛ وبالمثل، تدّخر زوجاتهم جزءًا منها في الملح؛ أما الفقراء والعاجزون فلا يُعطون منها شيئًا
\par 28 تأكل الحائض والنفساء ذبائحهن، فبهذا تعرفون أنها ليست آلهة، فلا تخافوها
\par 29 فكيف يُسمَّون آلهة؟ لأن النساء يُقدِّمن الطعام أمام آلهة الفضة والذهب والخشب
\par 30 ويجلس الكهنة في معابدهم، ثيابهم ممزقة، ورؤوسهم ولحاهم محلوقة، وليس على رؤوسهم شيء
\par 31 إنهم يزأرون ويصرخون أمام آلهتهم، كما يفعل الرجال في العيد عندما يموت أحدهم
\par 32 ويخلع الكهنة أيضًا ثيابهم، ويلبسون زوجاتهم وأولادهم
\par 33 سواء أكان ما يفعله أحد بهم شرًا أم خيرًا، فإنهم لا يستطيعون مجازاته: لا يستطيعون تنصيب ملك ولا إنزاله
\par 34 وبالمثل، لا يمكنهم إعطاء ثروة ولا مال: حتى لو نذر لهم رجل نذرًا ولم يفِ به، فلن يطلبوه
\par 35 لا يستطيعون إنقاذ إنسان من الموت، ولا إنقاذ الضعيف من القوي
\par 36 لا يستطيعون إعادة بصر أعمى، ولا مساعدة إنسان في محنته
\par 37 لا يرحمون الأرملة، ولا يحسنون إلى اليتيم.
\par 38 آلهتهم من خشب مغشّاة بالذهب والفضة تشبه الحجارة المنحوتة من الجبل. الذين يسجدون لها يخزى.
\par 39 كيف ينبغي للإنسان إذن أن يفكر ويقول إنهم آلهة، في حين أن الكلدانيين أنفسهم يهينونهم؟
\par 40 الذين إذا رأوا أبكمًا لا يستطيع التكلم، أحضروه، وطلبوا من بيل أن يتكلم، كما لو كان قادرًا على الفهم
\par 41 لكنهم لا يستطيعون أن يفهموا هذا بأنفسهم، فيتركونها: لأنهم لا يملكون معرفة
\par 42 والنساء أيضًا، جالسات على الطرق، متوشحات بالحبال، يحرقن النخالة للتطيب. ولكن إن ضاجعته إحداهن، يجذبها أحد المارة، فإنها تعاتب رفيقتها بأنها لم تُحسب مستحقة مثلها، ولا حبلها مقطوع
\par 43 كل ما يُفعل بينهم باطل: فكيف يُعقل إذن أن يُظن أو يُقال عنهم أنهم آلهة؟
\par 44 إنهم مصنوعون من النجارين والصاغة: لا يمكنهم أن يكونوا شيئًا آخر غير ما يريده العمال منهم.
\par 45 والذين صنعوها أنفسهم لا يمكنهم الاستمرار طويلاً؛ فكيف إذن تكون الأشياء المصنوعة منها آلهة؟
\par 46 لأنهم تركوا الأكاذيب والتوبيخات لمن يأتي بعدهم.
\par 47 لأنه عندما تأتي عليهم حرب أو وباء، يتشاور الكهنة مع أنفسهم حول المكان الذي يمكنهم الاختباء فيه معهم.
\par 48 فكيف إذًا لا يدرك البشر أنهم ليسوا آلهة، ولا يستطيعون إنقاذ أنفسهم من الحرب ولا من الطاعون؟
\par 49 لأنها مصنوعة من الخشب فقط، ومطلية بالفضة والذهب، فسيُعرف فيما بعد أنها مزيفة:
\par 50 وسيظهر جليًا لجميع الأمم والملوك أنهم ليسوا آلهة، بل عمل أيدي الناس، وليس فيهم عمل الله
\par 51 فمن ذا الذي قد لا يعلم أنهم ليسوا آلهة؟
\par 52 لأنهم لا يستطيعون أن يقيموا ملكاً على الأرض، ولا يعطوا الناس مطراً.
\par 53 لا يستطيعون الحكم على قضيتهم، ولا تصحيح الظلم، لعدم قدرتهم على ذلك: لأنهم كالغربان بين السماء والأرض
\par 54 وعندها، عندما تسقط نار على بيت الآلهة المصنوع من الخشب، أو المغطى بالذهب أو الفضة، فإن كهنةهم يهربون وينجون؛ أما هم فسيحترقون مثل العوارض
\par 55 علاوة على ذلك، فهم لا يستطيعون الصمود أمام أي ملك أو أعداء: فكيف يُمكن إذًا الاعتقاد أو القول بأنهم آلهة؟
\par 56 ولا تستطيع تلك الآلهة المصنوعة من الخشب، والمطلية بالفضة أو الذهب، الهروب من اللصوص أو قطاع الطرق
\par 57 يأخذ الأقوياء ذهبهم وفضتهم والثياب التي يلبسونها ويذهبون بها: ولا يقدرون على مساعدة أنفسهم
\par 58 لذلك، من الأفضل أن تكون ملكًا يُظهر سلطته، أو إناءً نافعًا في منزل، يستخدمه المالك، من أن تكون آلهةً زائفة كهذه؛ أو أن تكون بابًا في منزل، لحفظ مثل هذه الأشياء فيه، من أن تكون آلهةً زائفة كهذه أو عمودًا خشبيًا في قصر، من أن تكون آلهةً زائفة كهذه
\par 59 لأن الشمس والقمر والنجوم، كونها ساطعة ومُرسلة للقيام بمهامها، فهي مطيعة
\par 60 وبالمثل، يسهل رؤية البرق عند بزوغه؛ وعلى نفس المنوال تهب الرياح في كل بلد
\par 61 وعندما يأمر الله السحاب بالمرور فوق العالم أجمع، فإنه يفعل ما يُؤمر به
\par 62 والنار المرسلة من فوق لتلتهم التلال والغابات تفعل ما أُمرت به: لكن هؤلاء لا يشبهونها لا في المظهر ولا في القوة
\par 63 لذلك لا ينبغي افتراض أو قول أنهم آلهة، لأنهم غير قادرين على الحكم على الأسباب، ولا على فعل الخير للبشر
\par 64 فاعلموا أنهم ليسوا آلهة، فلا تخافوهم،
\par 65 لأنهم لا يستطيعون أن يلعنوا الملوك ولا يباركوهم.
\par 66 ولا يستطيعون أن يظهروا آيات في السماء بين الأمم، ولا أن يشرقوا كالشمس، ولا أن ينيروا كالقمر
\par 67 إن الحيوانات أفضل منهم، لأنها تستطيع أن تدخل تحت غطاء وتساعد نفسها.
\par 68 إذن، ليس واضحًا لنا بأي حال من الأحوال أنهم آلهة: فلا تخافوهم
\par 69 فكما أن الفزاعة في حديقة الخيار لا تدخر شيئًا، كذلك آلهتهم من الخشب، مطلية بالفضة والذهب
\par 70 وكذلك آلهتهم المصنوعة من الخشب، والمطلية بالفضة والذهب، تشبه شوكة بيضاء في بستان يجلس عليها كل طائر؛ كما تشبه جثة ميتة، تتجه شرقًا نحو الظلام
\par 71 ومن الأرجوان الفاقع الذي يتعفن عليه تعرفون أنهم ليسوا آلهة، وهم بعد ذلك يؤكلون ويكونون عارا في البلاد
\par 72 فالأفضل إذن أن يكون الصديق الذي ليس له أصنام، لأنه يكون بعيدًا عن العار



\end{document}