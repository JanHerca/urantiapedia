\begin{document}

\title{عهد آدم}

\chapter{1}

\par \textit{[نشر بيزولد النص الإثيوبي ونسخة عربية منه في كتاب نولدكه التذكاري، جيزن، 1906.]}

\par ساعات اليوم.

\par 1 علاوة على ذلك، افهم أنت أيضًا ساعات النهار والليل، وكيف ينبغي أن تتوسل إلى الله، وتصلي إليه في كل فصل من فصوله. لأن خالقي علمني كل هذا، وأخبرني بأسماء جميع الحيوانات والوحوش البرية، وطيور السماء، ثم جعلني الله أفهم عدد ساعات النهار والليل، وأخبرني كيف تسبح الملائكة الله. افهم إذن يا بني أنه في الساعة الأولى من النهار تصعد صلاة أبنائي إلى الله. وفي الساعة الثانية تتم صلاة الملائكة وتضرعهم. وفي الساعة الثالثة تسبحه طيور السماء. وفي الساعة الرابعة تسجد له الكائنات الروحية. وفي الساعة الخامسة تحييه جميع الوحوش والحيوانات البرية. وفي الساعة السادسة تتم صلاة الكروبيم وفي الساعة السابعة يدخل جميع الملائكة إلى حضرة الله، ويخرجون منها، لأنه في هذه الساعة تصعد صلاة كل كائن حي إلى الله. وفي الساعة الثامنة يسبحه سكان السماء المتألقون. وفي الساعة التاسعة يسجد له ملائكة الله الواقفون أمام عرش العلي. وفي الساعة العاشرة يُظلل الروح القدس المياه، فتهرب الشياطين وتخرج من المياه. ولو لم يُظلل الروح القدس المياه في هذه الساعة كل يوم، لما استطاع أحد أن يشرب من المياه، لأنه لو فعل ذلك لكان جسده قد هلك على يد الشياطين الأشرار. وإذا أخذ الكاهن ماءً في هذه الساعة وخلطه بزيت مقدس، ومسح به المرضى والممسوسين بالأرواح النجسة، يُشفون من أمراضهم. وفي الساعة الحادية عشرة تُمجد الأبرار. وفي الساعة الثانية عشرة يتقبل الله العلي صلوات وطلبات بني البشر.

\chapter{2}

\par ساعات الليل

\par 1 وفي الساعة الأولى من الليل يحمد الشياطين الله العلي ويحمدونه، ولا يكون فيهم شر ولا ضرر لأحد حتى ينتهوا من عبادتهم. وفي الساعة الثانية من الليل يسبح الله السمك وكل مخلوق في المياه، والوحوش والحيتان. وفي الساعة الثالثة تسبحه النار - وهي الآن في العمق الأدنى، وفي تلك الساعة لا يستطيع أحد أن يخاطبه (؟). وفي الساعة الرابعة يقدسه السيرافيم. وفي الساعة الخامسة تسبحه المياه التي فوق السماء. منذ زمن بعيد جلست واستمعت إلى الملائكة في هذه الساعة، و[تعجبت] كيف صرخوا؛ كان [صراخهم] مثل صوت عجلة عظيمة، وصرخوا مثل أمواج البحر بصوت تسبيح لله. وفي الساعة السادسة سبح الله السحاب في خوف ورعدة. وفي الساعة السابعة، ساد الصمت على الأرض، ونامت كل المخلوقات التي عليها، ونامت المياه. وإذا أخذ الكاهن في هذه الساعة ماءً وخلطه بزيت مقدس، ومسح به المرضى ومن لا يستطيعون النوم ليلاً من شدة الألم، شُفي المرضى، ونام المستيقظون. وفي الساعة الثامنة، تنبت الأرض عشبًا وأعشابًا خضراء، وتُخرج الأشجار أوراقها وثمارها. وفي الساعة التاسعة، يؤدي الملائكة عبادتهم لله، وتصل صلاة بني البشر إلى حضرة الله العلي. وفي الساعة العاشرة، تُفتح أبواب السماء، ويسمع الله صلاة أبناء المؤمنين، ويُستجاب لهم ما يطلبونه من الله. وعند صوت أجنحة السيرافيم في ذلك الوقت، تصيح الديوك وتُسبّح الله. وفي الساعة الحادية عشرة، يعم الفرح والسرور كل الأرض، لأن الشمس تدخل الجنة، ويشرق نورها في أقاصي الأرض، وينير كل مخلوق. وفي الساعة الثانية عشرة، يليق بأبنائي أن يقفوا بين يدي الله، ويعبدوه، لأن في هذه الساعة يسود صمتٌ عظيم على جميع الكائنات السماوية.

\chapter{3}

\par آدم يتنبأ بمجيء المسيح.

\par 1 الآن فاعلم كل هذا، واستمع لكلامي، وافهم أن كلمة الله العلي ستنزل على الأرض، كما أخبرني في اللحظة التي طردني فيها من الجنة (الفردوس). لأنه أخبرني أن كلمته في الأيام الأخيرة يجب أن تصبح إنسانًا من امرأة كانت عذراء اسمها مريم، ويجب أن تختبئ فيها، وتلبس جسدًا، وتولد كرجل ذو قوة عظيمة ومهارة ومعرفة عملية. لن يعرفه أحد إلا هو ومن أظهر له [نفسه]. وقال الله أنه يجب أن يتجول مع الناس على الأرض، وينمو في الأيام والسنين، ويجب أن يصنع الآيات والعجائب علانية، ويجب أن يمشي على البحر كما على اليابسة، ويجب أن ينتهر البحر والرياح علانية، ويجب أن تخضع له، وأنه يجب أن يصرخ إلى أمواج البحر ويجب أن تجيبه بسرعة. وأن يجعل العمي يبصرون، والبرص يطهرون، والصم يسمعون، والخرس يتكلمون، ويقيم المشلولين، ويجعل العرج يمشون، ويرد كثيرين من الضلال إلى معرفة الله، ويطرد الشياطين من الناس.

\par 2 وإلى جانب [هذه الأشياء]، كلّمني الله قائلاً: "لا تحزن يا آدم، لأنك أردت أن تصبح إلهًا وخالفت أمري. ها أنا أُثبّتك، ليس في الوقت الحاضر، بل بعد أيام قليلة." ثم كلمني مرة أخرى قائلاً: "أنا الله الذي أخرجك من جنة الفرح إلى الأرض التي ستنبت أشواكًا وعوسجًا، وستسكن فيها. انحنِ، واجعل ركبتيك ترتعشان في شيخوختك، وسأجعل لحمك طعامًا للديدان. وبعد خمسة أيام ونصف سأرحمك، وأظهر لك الرحمة بوفرة رحمتي وعطفي. وسأنزل إلى بيتك، وسأسكن في جسدك، ومن أجلك سأكون سعيدًا بأن أولد كطفل عادي. ومن أجلك سأكون سعيدًا بالسير في السوق. ومن أجلك سأكون سعيدًا بالصوم أربعين يومًا. ومن أجلك سأكون سعيدًا بقبول المعمودية. ومن أجلك سأكون سعيدًا بتحمل المعاناة. ومن أجلك سأكون سعيدًا بالتعليق على خشبة الصليب. كل هذه الأشياء [سأفعلها] من أجل "من أجلك يا آدم."

\par 3 له التسبيح والجلال والسلطان والمجد والسجود والتسبيح مع أبيه والروح القدس من الآن فصاعدًا وإلى دهر الدهور. آمين

\par 4 علاوة على ذلك، يجب أن تعلم يا بني شيث، هوذا طوفان سيأتي ويغسل الأرض كلها بسبب أبناء قابيل (قابيل)، القاتل، الذي قتل أخاه بدافع الغيرة، بسبب أخته لود. وبعد الطوفان وأسابيع عديدة، ستأتي الأيام الأخيرة، وسيكتمل كل شيء، وسيأتي وقته، وستأكل النار كل ما يوجد أمام الله، وستُقدس الأرض، وسيسير عليها رب الأرباب

\par 5 فكتب شيث هذه الوصية، وختمها بخاتمه، وخاتم أبيه آدم الذي أخذه معه من الجنة، وخاتم أمه حواء

\end{document}