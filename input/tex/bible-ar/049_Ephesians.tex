\begin{document}

\title{افسس}


\chapter{1}

\par 1 بُولُسُ، رَسُولُ يَسُوعَ الْمَسِيحِ بِمَشِيئَةِ اللهِ، إِلَى الْقِدِّيسِينَ الَّذِينَ فِي أَفَسُسَ، وَالْمُؤْمِنِينَ فِي الْمَسِيحِ يَسُوعَ.
\par 2 نِعْمَةٌ لَكُمْ وَسَلاَمٌ مِنَ اللهِ أَبِينَا وَالرَّبِّ يَسُوعَ الْمَسِيحِ.
\par 3 مُبَارَكٌ اللهُ أَبُو رَبِّنَا يَسُوعَ الْمَسِيحِ، الَّذِي بَارَكَنَا بِكُلِّ بَرَكَةٍ رُوحِيَّةٍ فِي السَّمَاوِيَّاتِ فِي الْمَسِيحِ،
\par 4 كَمَا اخْتَارَنَا فِيهِ قَبْلَ تَأْسِيسِ الْعَالَمِ، لِنَكُونَ قِدِّيسِينَ وَبِلاَ لَوْمٍ قُدَّامَهُ فِي الْمَحَبَّةِ،
\par 5 إِذْ سَبَقَ فَعَيَّنَنَا لِلتَّبَنِّي بِيَسُوعَ الْمَسِيحِ لِنَفْسِهِ، حَسَبَ مَسَرَّةِ مَشِيئَتِهِ،
\par 6 لِمَدْحِ مَجْدِ نِعْمَتِهِ الَّتِي أَنْعَمَ بِهَا عَلَيْنَا فِي الْمَحْبُوبِ،
\par 7 الَّذِي فِيهِ لَنَا الْفِدَاءُ، بِدَمِهِ غُفْرَانُ الْخَطَايَا، حَسَبَ غِنَى نِعْمَتِهِ،
\par 8 الَّتِي أَجْزَلَهَا لَنَا بِكُلِّ حِكْمَةٍ وَفِطْنَةٍ،
\par 9 إِذْ عَرَّفَنَا بِسِرِّ مَشِيئَتِهِ، حَسَبَ مَسَرَّتِهِ الَّتِي قَصَدَهَا فِي نَفْسِهِ،
\par 10 لِتَدْبِيرِ مِلْءِ الأَزْمِنَةِ، لِيَجْمَعَ كُلَّ شَيْءٍ فِي الْمَسِيحِ، مَا فِي السَّمَاوَاتِ وَمَا عَلَى الأَرْضِ، فِي ذَاكَ
\par 11 الَّذِي فِيهِ أَيْضاً نِلْنَا نَصِيباً، مُعَيَّنِينَ سَابِقاً حَسَبَ قَصْدِ الَّذِي يَعْمَلُ كُلَّ شَيْءٍ حَسَبَ رَأْيِ مَشِيئَتِهِ،
\par 12 لِنَكُونَ لِمَدْحِ مَجْدِهِ، نَحْنُ الَّذِينَ قَدْ سَبَقَ رَجَاؤُنَا فِي الْمَسِيحِ.
\par 13 الَّذِي فِيهِ أَيْضاً أَنْتُمْ، إِذْ سَمِعْتُمْ كَلِمَةَ الْحَقِّ، إِنْجِيلَ خَلاَصِكُمُ، الَّذِي فِيهِ أَيْضاً إِذْ آمَنْتُمْ خُتِمْتُمْ بِرُوحِ الْمَوْعِدِ الْقُدُّوسِ،
\par 14 الَّذِي هُوَ عَرْبُونُ مِيرَاثِنَا، لِفِدَاءِ الْمُقْتَنَى، لِمَدْحِ مَجْدِهِ.
\par 15 لِذَلِكَ أَنَا أَيْضاً إِذْ قَدْ سَمِعْتُ بِإِيمَانِكُمْ بِالرَّبِّ يَسُوعَ، وَمَحَبَّتِكُمْ نَحْوَ جَمِيعِ الْقِدِّيسِينَ،
\par 16 لاَ أَزَالُ شَاكِراً لأَجْلِكُمْ، ذَاكِراً إِيَّاكُمْ فِي صَلَوَاتِي،
\par 17 كَيْ يُعْطِيَكُمْ إِلَهُ رَبِّنَا يَسُوعَ الْمَسِيحِ، أَبُو الْمَجْدِ، رُوحَ الْحِكْمَةِ وَالإِعْلاَنِ فِي مَعْرِفَتِهِ،
\par 18 مُسْتَنِيرَةً عُيُونُ أَذْهَانِكُمْ، لِتَعْلَمُوا مَا هُوَ رَجَاءُ دَعْوَتِهِ، وَمَا هُوَ غِنَى مَجْدِ مِيرَاثِهِ فِي الْقِدِّيسِينَ،
\par 19 وَمَا هِيَ عَظَمَةُ قُدْرَتِهِ الْفَائِقَةُ نَحْوَنَا نَحْنُ الْمُؤْمِنِينَ، حَسَبَ عَمَلِ شِدَّةِ قُوَّتِهِ
\par 20 الَّذِي عَمِلَهُ فِي الْمَسِيحِ، إِذْ أَقَامَهُ مِنَ الأَمْوَاتِ، وَأَجْلَسَهُ عَنْ يَمِينِهِ فِي السَّمَاوِيَّاتِ،
\par 21 فَوْقَ كُلِّ رِيَاسَةٍ وَسُلْطَانٍ وَقُوَّةٍ وَسِيَادَةٍ، وَكُلِّ اسْمٍ يُسَمَّى لَيْسَ فِي هَذَا الدَّهْرِ فَقَطْ بَلْ فِي الْمُسْتَقْبَلِ أَيْضاً،
\par 22 وَأَخْضَعَ كُلَّ شَيْءٍ تَحْتَ قَدَمَيْهِ، وَإِيَّاهُ جَعَلَ رَأْساً فَوْقَ كُلِّ شَيْءٍ لِلْكَنِيسَةِ،
\par 23 الَّتِي هِيَ جَسَدُهُ، مِلْءُ الَّذِي يَمْلأُ الْكُلَّ فِي الْكُلِّ.

\chapter{2}

\par 1 وَأَنْتُمْ إِذْ كُنْتُمْ أَمْوَاتاً بِالذُّنُوبِ وَالْخَطَايَا،
\par 2 الَّتِي سَلَكْتُمْ فِيهَا قَبْلاً حَسَبَ دَهْرِ هَذَا الْعَالَمِ، حَسَبَ رَئِيسِ سُلْطَانِ الْهَوَاءِ، الرُّوحِ الَّذِي يَعْمَلُ الآنَ فِي أَبْنَاءِ الْمَعْصِيَةِ،
\par 3 الَّذِينَ نَحْنُ أَيْضاً جَمِيعاً تَصَرَّفْنَا قَبْلاً بَيْنَهُمْ فِي شَهَوَاتِ جَسَدِنَا، عَامِلِينَ مَشِيئَاتِ الْجَسَدِ وَالأَفْكَارِ، وَكُنَّا بِالطَّبِيعَةِ أَبْنَاءَ الْغَضَبِ كَالْبَاقِينَ أَيْضاً،
\par 4 اَللهُ الَّذِي هُوَ غَنِيٌّ فِي الرَّحْمَةِ، مِنْ أَجْلِ مَحَبَّتِهِ الْكَثِيرَةِ الَّتِي أَحَبَّنَا بِهَا،
\par 5 وَنَحْنُ أَمْوَاتٌ بِالْخَطَايَا أَحْيَانَا مَعَ الْمَسِيحِ - بِالنِّعْمَةِ أَنْتُمْ مُخَلَّصُونَ -
\par 6 وَأَقَامَنَا مَعَهُ، وَأَجْلَسَنَا مَعَهُ فِي السَّمَاوِيَّاتِ فِي الْمَسِيحِ يَسُوعَ،
\par 7 لِيُظْهِرَ فِي الدُّهُورِ الآتِيَةِ غِنَى نِعْمَتِهِ الْفَائِقَ بِاللُّطْفِ عَلَيْنَا فِي الْمَسِيحِ يَسُوعَ.
\par 8 لأَنَّكُمْ بِالنِّعْمَةِ مُخَلَّصُونَ، بِالإِيمَانِ، وَذَلِكَ لَيْسَ مِنْكُمْ. هُوَ عَطِيَّةُ اللهِ.
\par 9 لَيْسَ مِنْ أَعْمَالٍ كَيْلاَ يَفْتَخِرَ أَحَدٌ.
\par 10 لأَنَّنَا نَحْنُ عَمَلُهُ، مَخْلُوقِينَ فِي الْمَسِيحِ يَسُوعَ لأَعْمَالٍ صَالِحَةٍ، قَدْ سَبَقَ اللهُ فَأَعَدَّهَا لِكَيْ نَسْلُكَ فِيهَا.
\par 11 لِذَلِكَ اذْكُرُوا أَنَّكُمْ أَنْتُمُ الأُمَمُ قَبْلاً فِي الْجَسَدِ، الْمَدْعُوِّينَ غُرْلَةً مِنَ الْمَدْعُوِّ خِتَاناً مَصْنُوعاً بِالْيَدِ فِي الْجَسَدِ،
\par 12 أَنَّكُمْ كُنْتُمْ فِي ذَلِكَ الْوَقْتِ بِدُونِ مَسِيحٍ، أَجْنَبِيِّينَ عَنْ رَعَوِيَّةِ إِسْرَائِيلَ، وَغُرَبَاءَ عَنْ عُهُودِ الْمَوْعِدِ، لاَ رَجَاءَ لَكُمْ وَبِلاَ إِلَهٍ فِي الْعَالَمِ.
\par 13 وَلَكِنِ الآنَ فِي الْمَسِيحِ يَسُوعَ، أَنْتُمُ الَّذِينَ كُنْتُمْ قَبْلاً بَعِيدِينَ صِرْتُمْ قَرِيبِينَ بِدَمِ الْمَسِيحِ.
\par 14 لأَنَّهُ هُوَ سَلاَمُنَا، الَّذِي جَعَلَ الِاثْنَيْنِ وَاحِداً، وَنَقَضَ حَائِطَ السِّيَاجِ الْمُتَوَسِّطَ
\par 15 أَيِ الْعَدَاوَةَ. مُبْطِلاً بِجَسَدِهِ نَامُوسَ الْوَصَايَا فِي فَرَائِضَ، لِكَيْ يَخْلُقَ الِاثْنَيْنِ فِي نَفْسِهِ إِنْسَاناً وَاحِداً جَدِيداً، صَانِعاً سَلاَماً،
\par 16 وَيُصَالِحَ الِاثْنَيْنِ فِي جَسَدٍ وَاحِدٍ مَعَ اللهِ بِالصَّلِيبِ، قَاتِلاً الْعَدَاوَةَ بِهِ.
\par 17 فَجَاءَ وَبَشَّرَكُمْ بِسَلاَمٍ، أَنْتُمُ الْبَعِيدِينَ وَالْقَرِيبِينَ.
\par 18 لأَنَّ بِهِ لَنَا كِلَيْنَا قُدُوماً فِي رُوحٍ وَاحِدٍ إِلَى الآبِ.
\par 19 فَلَسْتُمْ إِذاً بَعْدُ غُرَبَاءَ وَنُزُلاً، بَلْ رَعِيَّةٌ مَعَ الْقِدِّيسِينَ وَأَهْلِ بَيْتِ اللهِ،
\par 20 مَبْنِيِّينَ عَلَى أَسَاسِ الرُّسُلِ وَالأَنْبِيَاءِ، وَيَسُوعُ الْمَسِيحُ نَفْسُهُ حَجَرُ الزَّاوِيَةِ،
\par 21 الَّذِي فِيهِ كُلُّ الْبِنَاءِ مُرَكَّباً مَعاً يَنْمُو هَيْكَلاً مُقَدَّساً فِي الرَّبِّ.
\par 22 الَّذِي فِيهِ أَنْتُمْ أَيْضاً مَبْنِيُّونَ مَعاً، مَسْكَناً لِلَّهِ فِي الرُّوحِ.

\chapter{3}

\par 1 بِسَبَبِ هَذَا أَنَا بُولُسُ، أَسِيرُ الْمَسِيحِ يَسُوعَ لأَجْلِكُمْ أَيُّهَا الأُمَمُ،
\par 2 إِنْ كُنْتُمْ قَدْ سَمِعْتُمْ بِتَدْبِيرِ نِعْمَةِ اللهِ الْمُعْطَاةِ لِي لأَجْلِكُمْ.
\par 3 أَنَّهُ بِإِعْلاَنٍ عَرَّفَنِي بِالسِّرِّ. كَمَا سَبَقْتُ فَكَتَبْتُ بِالإِيجَازِ.
\par 4 الَّذِي بِحَسَبِهِ حِينَمَا تَقْرَأُونَهُ تَقْدِرُونَ أَنْ تَفْهَمُوا دِرَايَتِي بِسِرِّ الْمَسِيحِ.
\par 5 الَّذِي فِي أَجْيَالٍ أُخَرَ لَمْ يُعَرَّفْ بِهِ بَنُو الْبَشَرِ، كَمَا قَدْ أُعْلِنَ الآنَ لِرُسُلِهِ الْقِدِّيسِينَ وَأَنْبِيَائِهِ بِالرُّوحِ:
\par 6 أَنَّ الأُمَمَ شُرَكَاءُ فِي الْمِيرَاثِ وَالْجَسَدِ وَنَوَالِ مَوْعِدِهِ فِي الْمَسِيحِ بِالإِنْجِيلِ.
\par 7 الَّذِي صِرْتُ أَنَا خَادِماً لَهُ حَسَبَ مَوْهِبَةِ نِعْمَةِ اللهِ الْمُعْطَاةِ لِي حَسَبَ فِعْلِ قُوَّتِهِ.
\par 8 لِي أَنَا أَصْغَرَ جَمِيعِ الْقِدِّيسِينَ أُعْطِيَتْ هَذِهِ النِّعْمَةُ، أَنْ أُبَشِّرَ بَيْنَ الأُمَمِ بِغِنَى الْمَسِيحِ الَّذِي لاَ يُسْتَقْصَى،
\par 9 وَأُنِيرَ الْجَمِيعَ فِي مَا هُوَ شَرِكَةُ السِّرِّ الْمَكْتُومِ مُنْذُ الدُّهُورِ فِي اللهِ خَالِقِ الْجَمِيعِ بِيَسُوعَ الْمَسِيحِ.
\par 10 لِكَيْ يُعَرَّفَ الآنَ عِنْدَ الرُّؤَسَاءِ وَالسَّلاَطِينِ فِي السَّمَاوِيَّاتِ بِوَاسِطَةِ الْكَنِيسَةِ بِحِكْمَةِ اللهِ الْمُتَنَوِّعَةِ،
\par 11 حَسَبَ قَصْدِ الدُّهُورِ الَّذِي صَنَعَهُ فِي الْمَسِيحِ يَسُوعَ رَبِّنَا.
\par 12 الَّذِي بِهِ لَنَا جَرَاءَةٌ وَقُدُومٌ بِإِيمَانِهِ عَنْ ثِقَةٍ.
\par 13 لِذَلِكَ أَطْلُبُ أَنْ لاَ تَكِلُّوا فِي شَدَائِدِي لأَجْلِكُمُ الَّتِي هِيَ مَجْدُكُمْ.
\par 14 بِسَبَبِ هَذَا أَحْنِي رُكْبَتَيَّ لَدَى أَبِي رَبِّنَا يَسُوعَ الْمَسِيحِ،
\par 15 الَّذِي مِنْهُ تُسَمَّى كُلُّ عَشِيرَةٍ فِي السَّمَاوَاتِ وَعَلَى الأَرْضِ.
\par 16 لِكَيْ يُعْطِيَكُمْ بِحَسَبِ غِنَى مَجْدِهِ أَنْ تَتَأَيَّدُوا بِالْقُوَّةِ بِرُوحِهِ فِي الإِنْسَانِ الْبَاطِنِ،
\par 17 لِيَحِلَّ الْمَسِيحُ بِالإِيمَانِ فِي قُلُوبِكُمْ،
\par 18 وَأَنْتُمْ مُتَأَصِّلُونَ وَمُتَأَسِّسُونَ فِي الْمَحَبَّةِ، حَتَّى تَسْتَطِيعُوا أَنْ تُدْرِكُوا مَعَ جَمِيعِ الْقِدِّيسِينَ مَا هُوَ الْعَرْضُ وَالطُّولُ وَالْعُمْقُ وَالْعُلْوُ،
\par 19 وَتَعْرِفُوا مَحَبَّةَ الْمَسِيحِ الْفَائِقَةَ الْمَعْرِفَةِ، لِكَيْ تَمْتَلِئُوا إِلَى كُلِّ مِلْءِ اللهِ.
\par 20 وَالْقَادِرُ أَنْ يَفْعَلَ فَوْقَ كُلِّ شَيْءٍ أَكْثَرَ جِدّاً مِمَّا نَطْلُبُ أَوْ نَفْتَكِرُ، بِحَسَبِ الْقُوَّةِ الَّتِي تَعْمَلُ فِينَا،
\par 21 لَهُ الْمَجْدُ فِي الْكَنِيسَةِ فِي الْمَسِيحِ يَسُوعَ إِلَى جَمِيعِ أَجْيَالِ دَهْرِ الدُّهُورِ. آمِينَ.

\chapter{4}

\par 1 فَأَطْلُبُ إِلَيْكُمْ، أَنَا الأَسِيرَ فِي الرَّبِّ، أَنْ تَسْلُكُوا كَمَا يَحِقُّ لِلدَّعْوَةِ الَّتِي دُعِيتُمْ بِهَا.
\par 2 بِكُلِّ تَوَاضُعٍ، وَوَدَاعَةٍ، وَبِطُولِ أَنَاةٍ، مُحْتَمِلِينَ بَعْضُكُمْ بَعْضاً فِي الْمَحَبَّةِ.
\par 3 مُجْتَهِدِينَ أَنْ تَحْفَظُوا وَحْدَانِيَّةَ الرُّوحِ بِرِبَاطِ السَّلاَمِ.
\par 4 جَسَدٌ وَاحِدٌ، وَرُوحٌ وَاحِدٌ، كَمَا دُعِيتُمْ أَيْضاً فِي رَجَاءِ دَعْوَتِكُمُ الْوَاحِدِ.
\par 5 رَبٌّ وَاحِدٌ، إِيمَانٌ وَاحِدٌ، مَعْمُودِيَّةٌ وَاحِدَةٌ،
\par 6 إِلَهٌ وَآبٌ وَاحِدٌ لِلْكُلِّ، الَّذِي عَلَى الْكُلِّ وَبِالْكُلِّ وَفِي كُلِّكُمْ.
\par 7 وَلَكِنْ لِكُلِّ وَاحِدٍ مِنَّا أُعْطِيَتِ النِّعْمَةُ حَسَبَ قِيَاسِ هِبَةِ الْمَسِيحِ.
\par 8 لِذَلِكَ يَقُولُ: «إِذْ صَعِدَ إِلَى الْعَلاَءِ سَبَى سَبْياً وَأَعْطَى النَّاسَ عَطَايَا».
\par 9 وَأَمَّا أَنَّهُ صَعِدَ، فَمَا هُوَ إِلاَّ إِنَّهُ نَزَلَ أَيْضاً أَوَّلاً إِلَى أَقْسَامِ الأَرْضِ السُّفْلَى.
\par 10 اَلَّذِي نَزَلَ هُوَ الَّذِي صَعِدَ أَيْضاً فَوْقَ جَمِيعِ السَّمَاوَاتِ، لِكَيْ يَمْلَأَ الْكُلَّ.
\par 11 وَهُوَ أَعْطَى الْبَعْضَ أَنْ يَكُونُوا رُسُلاً، وَالْبَعْضَ أَنْبِيَاءَ، وَالْبَعْضَ مُبَشِّرِينَ، وَالْبَعْضَ رُعَاةً وَمُعَلِّمِينَ،
\par 12 لأَجْلِ تَكْمِيلِ الْقِدِّيسِينَ، لِعَمَلِ الْخِدْمَةِ، لِبُنْيَانِ جَسَدِ الْمَسِيحِ،
\par 13 إِلَى أَنْ نَنْتَهِيَ جَمِيعُنَا إِلَى وَحْدَانِيَّةِ الإِيمَانِ وَمَعْرِفَةِ ابْنِ اللهِ. إِلَى إِنْسَانٍ كَامِلٍ. إِلَى قِيَاسِ قَامَةِ مِلْءِ الْمَسِيحِ.
\par 14 كَيْ لاَ نَكُونَ فِي مَا بَعْدُ أَطْفَالاً مُضْطَرِبِينَ وَمَحْمُولِينَ بِكُلِّ رِيحِ تَعْلِيمٍ، بِحِيلَةِ النَّاسِ، بِمَكْرٍ إِلَى مَكِيدَةِ الضَّلاَلِ.
\par 15 بَلْ صَادِقِينَ فِي الْمَحَبَّةِ، نَنْمُو فِي كُلِّ شَيْءٍ إِلَى ذَاكَ الَّذِي هُوَ الرَّأْسُ: الْمَسِيحُ،
\par 16 الَّذِي مِنْهُ كُلُّ الْجَسَدِ مُرَكَّباً مَعاً، وَمُقْتَرِناً بِمُؤَازَرَةِ كُلِّ مَفْصِلٍ، حَسَبَ عَمَلٍ، عَلَى قِيَاسِ كُلِّ جُزْءٍ، يُحَصِّلُ نُمُوَّ الْجَسَدِ لِبُنْيَانِهِ فِي الْمَحَبَّةِ.
\par 17 فَأَقُولُ هَذَا وَأَشْهَدُ فِي الرَّبِّ، أَنْ لاَ تَسْلُكُوا فِي مَا بَعْدُ كَمَا يَسْلُكُ سَائِرُ الأُمَمِ أَيْضاً بِبُطْلِ ذِهْنِهِمْ،
\par 18 إِذْ هُمْ مُظْلِمُو الْفِكْرِ، وَمُتَجَنِّبُونَ عَنْ حَيَاةِ اللهِ لِسَبَبِ الْجَهْلِ الَّذِي فِيهِمْ بِسَبَبِ غِلاَظَةِ قُلُوبِهِمْ.
\par 19 اَلَّذِينَ إِذْ هُمْ قَدْ فَقَدُوا الْحِسَّ، أَسْلَمُوا نُفُوسَهُمْ لِلدَّعَارَةِ لِيَعْمَلُوا كُلَّ نَجَاسَةٍ فِي الطَّمَعِ.
\par 20 وَأَمَّا أَنْتُمْ فَلَمْ تَتَعَلَّمُوا الْمَسِيحَ هَكَذَا -
\par 21 إِنْ كُنْتُمْ قَدْ سَمِعْتُمُوهُ وَعُلِّمْتُمْ فِيهِ كَمَا هُوَ حَقٌّ فِي يَسُوعَ،
\par 22 أَنْ تَخْلَعُوا مِنْ جِهَةِ التَّصَرُّفِ السَّابِقِ الإِنْسَانَ الْعَتِيقَ الْفَاسِدَ بِحَسَبِ شَهَوَاتِ الْغُرُورِ،
\par 23 وَتَتَجَدَّدُوا بِرُوحِ ذِهْنِكُمْ،
\par 24 وَتَلْبَسُوا الإِنْسَانَ الْجَدِيدَ الْمَخْلُوقَ بِحَسَبِ اللهِ فِي الْبِرِّ وَقَدَاسَةِ الْحَقِّ.
\par 25 لِذَلِكَ اطْرَحُوا عَنْكُمُ الْكَذِبَ وَتَكَلَّمُوا بِالصِّدْقِ كُلُّ وَاحِدٍ مَعَ قَرِيبِهِ، لأَنَّنَا بَعْضَنَا أَعْضَاءُ الْبَعْضِ.
\par 26 اِغْضَبُوا وَلاَ تُخْطِئُوا. لاَ تَغْرُبِ الشَّمْسُ عَلَى غَيْظِكُمْ
\par 27 وَلاَ تُعْطُوا إِبْلِيسَ مَكَاناً.
\par 28 لاَ يَسْرِقِ السَّارِقُ فِي مَا بَعْدُ، بَلْ بِالْحَرِيِّ يَتْعَبُ عَامِلاً الصَّالِحَ بِيَدَيْهِ، لِيَكُونَ لَهُ أَنْ يُعْطِيَ مَنْ لَهُ احْتِيَاجٌ.
\par 29 لاَ تَخْرُجْ كَلِمَةٌ رَدِيَّةٌ مِنْ أَفْوَاهِكُمْ، بَلْ كُلُّ مَا كَانَ صَالِحاً لِلْبُنْيَانِ، حَسَبَ الْحَاجَةِ، كَيْ يُعْطِيَ نِعْمَةً لِلسَّامِعِينَ.
\par 30 وَلاَ تُحْزِنُوا رُوحَ اللهِ الْقُدُّوسَ الَّذِي بِهِ خُتِمْتُمْ لِيَوْمِ الْفِدَاءِ.
\par 31 لِيُرْفَعْ مِنْ بَيْنِكُمْ كُلُّ مَرَارَةٍ وَسَخَطٍ وَغَضَبٍ وَصِيَاحٍ وَتَجْدِيفٍ مَعَ كُلِّ خُبْثٍ.
\par 32 وَكُونُوا لُطَفَاءَ بَعْضُكُمْ نَحْوَ بَعْضٍ، شَفُوقِينَ مُتَسَامِحِينَ كَمَا سَامَحَكُمُ اللهُ أَيْضاً فِي الْمَسِيحِ.

\chapter{5}

\par 1 فَكُونُوا مُتَمَثِّلِينَ بِاَللهِ كَأَوْلاَدٍ أَحِبَّاءَ،
\par 2 وَاسْلُكُوا فِي الْمَحَبَّةِ كَمَا أَحَبَّنَا الْمَسِيحُ أَيْضاً وَأَسْلَمَ نَفْسَهُ لأَجْلِنَا، قُرْبَاناً وَذَبِيحَةً لِلَّهِ رَائِحَةً طَيِّبَةً.
\par 3 وَأَمَّا الزِّنَا وَكُلُّ نَجَاسَةٍ أَوْ طَمَعٍ فَلاَ يُسَمَّ بَيْنَكُمْ كَمَا يَلِيقُ بِقِدِّيسِينَ،
\par 4 وَلاَ الْقَبَاحَةُ، وَلاَ كَلاَمُ السَّفَاهَةِ وَالْهَزْلُ الَّتِي لاَ تَلِيقُ، بَلْ بِالْحَرِيِّ الشُّكْرُ.
\par 5 فَإِنَّكُمْ تَعْلَمُونَ هَذَا أَنَّ كُلَّ زَانٍ أَوْ نَجِسٍ أَوْ طَمَّاعٍ، الَّذِي هُوَ عَابِدٌ لِلأَوْثَانِ لَيْسَ لَهُ مِيرَاثٌ فِي مَلَكُوتِ الْمَسِيحِ وَاللهِ.
\par 6 لاَ يَغُرَّكُمْ أَحَدٌ بِكَلاَمٍ بَاطِلٍ، لأَنَّهُ بِسَبَبِ هَذِهِ الأُمُورِ يَأْتِي غَضَبُ اللهِ عَلَى أَبْنَاءِ الْمَعْصِيَةِ.
\par 7 فَلاَ تَكُونُوا شُرَكَاءَهُمْ.
\par 8 لأَنَّكُمْ كُنْتُمْ قَبْلاً ظُلْمَةً وَأَمَّا الآنَ فَنُورٌ فِي الرَّبِّ. اسْلُكُوا كَأَوْلاَدِ نُورٍ.
\par 9 لأَنَّ ثَمَرَ الرُّوحِ هُوَ فِي كُلِّ صَلاَحٍ وَبِرٍّ وَحَقٍّ.
\par 10 مُخْتَبِرِينَ مَا هُوَ مَرْضِيٌّ عِنْدَ الرَّبِّ.
\par 11 وَلاَ تَشْتَرِكُوا فِي أَعْمَالِ الظُّلْمَةِ غَيْرِ الْمُثْمِرَةِ بَلْ بِالْحَرِيِّ وَبِّخُوهَا.
\par 12 لأَنَّ الأُمُورَ الْحَادِثَةَ مِنْهُمْ سِرّاً ذِكْرُهَا أَيْضاً قَبِيحٌ.
\par 13 وَلَكِنَّ الْكُلَّ إِذَا تَوَبَّخَ يُظْهَرُ بِالنُّورِ. لأَنَّ كُلَّ مَا أُظْهِرَ فَهُوَ نُورٌ.
\par 14 لِذَلِكَ يَقُولُ: «اسْتَيْقِظْ أَيُّهَا النَّائِمُ وَقُمْ مِنَ الأَمْوَاتِ فَيُضِيءَ لَكَ الْمَسِيحُ».
\par 15 فَانْظُرُوا كَيْفَ تَسْلُكُونَ بِالتَّدْقِيقِ، لاَ كَجُهَلاَءَ بَلْ كَحُكَمَاءَ،
\par 16 مُفْتَدِينَ الْوَقْتَ لأَنَّ الأَيَّامَ شِرِّيرَةٌ.
\par 17 مِنْ أَجْلِ ذَلِكَ لاَ تَكُونُوا أَغْبِيَاءَ بَلْ فَاهِمِينَ مَا هِيَ مَشِيئَةُ الرَّبِّ.
\par 18 وَلاَ تَسْكَرُوا بِالْخَمْرِ الَّذِي فِيهِ الْخَلاَعَةُ، بَلِ امْتَلِئُوا بِالرُّوحِ،
\par 19 مُكَلِّمِينَ بَعْضُكُمْ بَعْضاً بِمَزَامِيرَ وَتَسَابِيحَ وَأَغَانِيَّ رُوحِيَّةٍ، مُتَرَنِّمِينَ وَمُرَتِّلِينَ فِي قُلُوبِكُمْ لِلرَّبِّ.
\par 20 شَاكِرِينَ كُلَّ حِينٍ عَلَى كُلِّ شَيْءٍ فِي اسْمِ رَبِّنَا يَسُوعَ الْمَسِيحِ، لِلَّهِ وَالآبِ.
\par 21 خَاضِعِينَ بَعْضُكُمْ لِبَعْضٍ فِي خَوْفِ اللهِ.
\par 22 أَيُّهَا النِّسَاءُ اخْضَعْنَ لِرِجَالِكُنَّ كَمَا لِلرَّبِّ،
\par 23 لأَنَّ الرَّجُلَ هُوَ رَأْسُ الْمَرْأَةِ كَمَا أَنَّ الْمَسِيحَ أَيْضاً رَأْسُ الْكَنِيسَةِ، وَهُوَ مُخَلِّصُ الْجَسَدِ.
\par 24 وَلَكِنْ كَمَا تَخْضَعُ الْكَنِيسَةُ لِلْمَسِيحِ، كَذَلِكَ النِّسَاءُ لِرِجَالِهِنَّ فِي كُلِّ شَيْءٍ.
\par 25 أَيُّهَا الرِّجَالُ، أَحِبُّوا نِسَاءَكُمْ كَمَا أَحَبَّ الْمَسِيحُ أَيْضاً الْكَنِيسَةَ وَأَسْلَمَ نَفْسَهُ لأَجْلِهَا،
\par 26 لِكَيْ يُقَدِّسَهَا، مُطَهِّراً إِيَّاهَا بِغَسْلِ الْمَاءِ بِالْكَلِمَةِ،
\par 27 لِكَيْ يُحْضِرَهَا لِنَفْسِهِ كَنِيسَةً مَجِيدَةً، لاَ دَنَسَ فِيهَا وَلاَ غَضْنَ أَوْ شَيْءٌ مِنْ مِثْلِ ذَلِكَ، بَلْ تَكُونُ مُقَدَّسَةً وَبِلاَ عَيْبٍ.
\par 28 كَذَلِكَ يَجِبُ عَلَى الرِّجَالِ أَنْ يُحِبُّوا نِسَاءَهُمْ كَأَجْسَادِهِمْ. مَنْ يُحِبُّ امْرَأَتَهُ يُحِبُّ نَفْسَهُ.
\par 29 فَإِنَّهُ لَمْ يُبْغِضْ أَحَدٌ جَسَدَهُ قَطُّ بَلْ يَقُوتُهُ وَيُرَبِّيهِ، كَمَا الرَّبُّ أَيْضاً لِلْكَنِيسَةِ.
\par 30 لأَنَّنَا أَعْضَاءُ جِسْمِهِ، مِنْ لَحْمِهِ وَمِنْ عِظَامِهِ.
\par 31 مِنْ أَجْلِ هَذَا يَتْرُكُ الرَّجُلُ أَبَاهُ وَأُمَّهُ وَيَلْتَصِقُ بِامْرَأَتِهِ، وَيَكُونُ الِاثْنَانِ جَسَداً وَاحِداً.
\par 32 هَذَا السِّرُّ عَظِيمٌ، وَلَكِنَّنِي أَنَا أَقُولُ مِنْ نَحْوِ الْمَسِيحِ وَالْكَنِيسَةِ.
\par 33 وَأَمَّا أَنْتُمُ الأَفْرَادُ، فَلْيُحِبَّ كُلُّ وَاحِدٍ امْرَأَتَهُ هَكَذَا كَنَفْسِهِ، وَأَمَّا الْمَرْأَةُ فَلْتَهَبْ رَجُلَهَا.

\chapter{6}

\par 1 أَيُّهَا الأَوْلاَدُ، أَطِيعُوا وَالِدِيكُمْ فِي الرَّبِّ لأَنَّ هَذَا حَقٌّ.
\par 2 أَكْرِمْ أَبَاكَ وَأُمَّكَ، الَّتِي هِيَ أَوَّلُ وَصِيَّةٍ بِوَعْدٍ،
\par 3 لِكَيْ يَكُونَ لَكُمْ خَيْرٌ، وَتَكُونُوا طِوَالَ الأَعْمَارِ عَلَى الأَرْضِ.
\par 4 وَأَنْتُمْ أَيُّهَا الآبَاءُ، لاَ تُغِيظُوا أَوْلاَدَكُمْ، بَلْ رَبُّوهُمْ بِتَأْدِيبِ الرَّبِّ وَإِنْذَارِهِ.
\par 5 أَيُّهَا الْعَبِيدُ، أَطِيعُوا سَادَتَكُمْ حَسَبَ الْجَسَدِ بِخَوْفٍ وَرِعْدَةٍ، فِي بَسَاطَةِ قُلُوبِكُمْ كَمَا لِلْمَسِيحِ -
\par 6 لاَ بِخِدْمَةِ الْعَيْنِ كَمَنْ يُرْضِي النَّاسَ، بَلْ كَعَبِيدِ الْمَسِيحِ، عَامِلِينَ مَشِيئَةَ اللهِ مِنَ الْقَلْبِ،
\par 7 خَادِمِينَ بِنِيَّةٍ صَالِحَةٍ كَمَا لِلرَّبِّ، لَيْسَ لِلنَّاسِ.
\par 8 عَالِمِينَ أَنْ مَهْمَا عَمِلَ كُلُّ وَاحِدٍ مِنَ الْخَيْرِ فَذَلِكَ يَنَالُهُ مِنَ الرَّبِّ، عَبْداً كَانَ أَمْ حُرّاً.
\par 9 وَأَنْتُمْ أَيُّهَا السَّادَةُ، افْعَلُوا لَهُمْ هَذِهِ الأُمُورَ، تَارِكِينَ التَّهْدِيدَ، عَالِمِينَ أَنَّ سَيِّدَكُمْ أَنْتُمْ أَيْضاً فِي السَّمَاوَاتِ، وَلَيْسَ عِنْدَهُ مُحَابَاةٌ.
\par 10 أَخِيراً يَا إِخْوَتِي تَقَوُّوا فِي الرَّبِّ وَفِي شِدَّةِ قُوَّتِهِ.
\par 11 الْبَسُوا سِلاَحَ اللهِ الْكَامِلَ لِكَيْ تَقْدِرُوا أَنْ تَثْبُتُوا ضِدَّ مَكَايِدِ إِبْلِيسَ.
\par 12 فَإِنَّ مُصَارَعَتَنَا لَيْسَتْ مَعَ دَمٍ وَلَحْمٍ، بَلْ مَعَ الرُّؤَسَاءِ، مَعَ السَّلاَطِينِ، مَعَ وُلاَةِ الْعَالَمِ، عَلَى ظُلْمَةِ هَذَا الدَّهْرِ، مَعَ أَجْنَادِ الشَّرِّ الرُّوحِيَّةِ فِي السَّمَاوِيَّاتِ.
\par 13 مِنْ أَجْلِ ذَلِكَ احْمِلُوا سِلاَحَ اللهِ الْكَامِلَ لِكَيْ تَقْدِرُوا أَنْ تُقَاوِمُوا فِي الْيَوْمِ الشِّرِّيرِ، وَبَعْدَ أَنْ تُتَمِّمُوا كُلَّ شَيْءٍ أَنْ تَثْبُتُوا.
\par 14 فَاثْبُتُوا مُمَنْطِقِينَ أَحْقَاءَكُمْ بِالْحَقِّ، وَلاَبِسِينَ دِرْعَ الْبِرِّ،
\par 15 وَحَاذِينَ أَرْجُلَكُمْ بِاسْتِعْدَادِ إِنْجِيلِ السَّلاَمِ.
\par 16 حَامِلِينَ فَوْقَ الْكُلِّ تُرْسَ الإِيمَانِ، الَّذِي بِهِ تَقْدِرُونَ أَنْ تُطْفِئُوا جَمِيعَ سِهَامِ الشِّرِّيرِ الْمُلْتَهِبَةِ.
\par 17 وَخُذُوا خُوذَةَ الْخَلاَصِ، وَسَيْفَ الرُّوحِ الَّذِي هُوَ كَلِمَةُ اللهِ.
\par 18 مُصَلِّينَ بِكُلِّ صَلاَةٍ وَطِلْبَةٍ كُلَّ وَقْتٍ فِي الرُّوحِ، وَسَاهِرِينَ لِهَذَا بِعَيْنِهِ بِكُلِّ مُواظَبَةٍ وَطِلْبَةٍ، لأَجْلِ جَمِيعِ الْقِدِّيسِينَ،
\par 19 وَلأَجْلِي، لِكَيْ يُعْطَى لِي كَلاَمٌ عِنْدَ افْتِتَاحِ فَمِي، لِأُعْلِمَ جِهَاراً بِسِرِّ الإِنْجِيلِ،
\par 20 الَّذِي لأَجْلِهِ أَنَا سَفِيرٌ فِي سَلاَسِلَ، لِكَيْ أُجَاهِرَ فِيهِ كَمَا يَجِبُ أَنْ أَتَكَلَّمَ.
\par 21 وَلَكِنْ لِكَيْ تَعْلَمُوا أَنْتُمْ أَيْضاً أَحْوَالِي، مَاذَا أَفْعَلُ، يُعَرِّفُكُمْ بِكُلِّ شَيْءٍ تِيخِيكُسُ الأَخُ الْحَبِيبُ وَالْخَادِمُ الأَمِينُ فِي الرَّبِّ،
\par 22 الَّذِي أَرْسَلْتُهُ إِلَيْكُمْ لِهَذَا بِعَيْنِهِ لِكَيْ تَعْلَمُوا أَحْوَالَنَا، وَلِكَيْ يُعَزِّيَ قُلُوبَكُمْ.
\par 23 سَلاَمٌ عَلَى الإِخْوَةِ، وَمَحَبَّةٌ بِإِيمَانٍ مِنَ اللهِ الآبِ وَالرَّبِّ يَسُوعَ الْمَسِيحِ.
\par 24 اَلنِّعْمَةُ مَعَ جَمِيعِ الَّذِينَ يُحِبُّونَ رَبَّنَا يَسُوعَ الْمَسِيحَ فِي عَدَمِ فَسَادٍ. آمِينَ.

\end{document}