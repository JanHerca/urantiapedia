\begin{document}

\title{وصية بنيامين}

\chapter{1}

بنيامين، الابن الثاني عشر ليعقوب وراحيل، طفل العائلة، يتحول إلى فيلسوف وإنسان خير.

\par 1 نسخة أقوال بنيامين التي أوصى بنيه بالعمل بها بعد أن عاش مئة وخمسة وعشرين عامًا

\par 2 فقبلهما وقال: كما وُلِد إسحاق لإبراهيم في شيخوخته، هكذا وُلِدتُ أنا أيضًا ليعقوب

\par 3 ولما ماتت راحيل أمي عند ولادتي، لم يكن لي حليب، فأرضعتني بلهة أمتها

\par 4 لأن راحيل بقيت عاقرًا اثنتي عشرة سنة بعدما ولدت يوسف، وصلّت إلى الرب بالصوم اثني عشر يومًا، فحملت وولدتني

\par 5 لأن أبي أحب راحيل كثيرًا، وصلى أن يرى منها ولدين

\par 6 لذلك دُعيتُ بنيامين، أي ابن أيام.

\par 7 ولما دخلت مصر إلى يوسف، وعرفني أخي، قال لي: ماذا قالوا لأبي حين باعوني؟

\par 8 فقلت له: لقد لطخوا قميصك بدم وأرسلوه، وقالوا: اعلم هل هذا هو قميص ابنك؟

\par 9 فقال لي: هكذا يا أخي، لما جردوني من قميصي أعطوني للإسماعيليين، وأعطوني مئزرًا وجلدوني وأمروني بالركض

\par 10 وأما الذي ضربني بقضيب، فقد لقيه أسد فقتله

\par 11 فارتعب رفاقه.

\par 12 أنتم أيضًا، يا أبنائي، أحبوا الرب إله السماء والأرض، واحفظوا وصاياه، مقتدين بمثال الرجل الصالح والقديس يوسف

\par 13 وليكن فكركم صالحًا كما تعرفونني، لأن من يغسل عقله بالصواب يرى كل شيء على صواب

\par 14 اتقوا الرب، وأحبوا قريبكم. وحتى لو زعمت أرواح بليعار أن تصيبكم بكل شر، فلن يكون لها سلطان عليكم، كما لم يكن لها سلطان على يوسف أخي

\par 15 كم من رجل أراد قتله، فحفظه الله!

\par 16 لأن من يخاف الله ويحب قريبه لا يمكن أن تضربه روح بليعار، إذ هو محمي بمخافة الله.

\par 17 ولا يمكن السيطرة عليه بمكائد البشر أو الوحوش، لأنه يُعينه الرب من خلال محبته لقريبه

\par 18 لأن يوسف أيضًا طلب من أبينا أن يصلي من أجل إخوته، لكي لا يحسب لهم الرب أي شر فعلوه به كخطية

\par 19 وهكذا صرخ يعقوب: يا ابني الصالح، لقد غلبت أحشاء أبيك يعقوب

\par 20 واحتضنه وقبله ساعتين قائلًا:

\par 21 فيك تتم نبوة السماء عن حمل الله ومخلص العالم، وأنه سيسلم واحد بلا لوم إلى الناس الأثمة، ويموت واحد بلا خطيئة من أجل الناس الأشرار في دم العهد، لخلاص الأمم وإسرائيل، ويهلك بليعار وعبيده.

\par 22 أترون إذن يا أبنائي نهاية الرجل الصالح؟

\par 23 فكونوا متمثلين برحمته بعقل صالح، لكي تلبسوا أنتم أيضاً أكاليل المجد.

\par 24 لأن الرجل الصالح ليس له عينٌ سوداء، لأنه يُظهر الرحمة لجميع الناس، حتى وإن كانوا خطاة

\par 25 وإن دبروا له سوء النية، فإنه بفعل الخير يغلب الشر، إذ يحميه الله، ويحب البار كنفسه

\par 26 إن مُجِّد أحدٌ فلا يحسده، إن غَنِيَ أحدٌ فلا يحسده، إن كان شجاعًا فيمدحه، الفاضل يمدحه، الفقير يرحمه، الضعيف يرحمه، الله يسبحه

\par 27 ومن له نعمة الروح الصالح فإنه يحبه كنفسه

\par 28 فإن كان لديكم أيضًا رأي صالح، فسيكون الأشرار في سلام معكم، وسيحترمكم المسرفون ويتحولون إلى الخير؛ ولن يتوقف الطماعون عن شهواتهم المفرطة فحسب، بل سيعطون أيضًا أشياء طمعهم للمبتلين

\par 29 إذا أحسنتم، حتى الأرواح النجسة ستهرب منكم، وستخاف منكم الوحوش

\par 30 لأنه حيثما يوجد تبجيل للأعمال الصالحة والنور في العقل، حتى الظلام يهرب منه

\par 31 لأنه إن اعتدى أحد على رجل قديس فإنه يتوب، لأن الرجل القدوس يرحم من يسيئ إليه ويسكت

\par 32 وإن خان أحد بارًا، فالبار يصلي. وإن تواضع قليلًا، فإنه بعد قليل يظهر أكثر مجدًا، كما كان يوسف أخي

\par 33 إن ميول الرجل الصالح ليست في قوة خداع روح بليعار، لأن ملاك السلام يهدي نفسه

\par 34 ولا ينظر بشغف إلى الأشياء الفانيَة، ولا يجمع الثروات من خلال شهوة اللذة

\par 35 لا يُسرّ باللذة، ولا يُحزن قريبه، ولا يشبع بالنعيم، ولا يضل في رفع العينين، لأن الرب نصيبه

\par 36 لا يقبل الميول الصالحة مجدًا ولا إهانة من الناس، ولا تعرف مكرًا ولا كذبًا ولا قتالًا ولا شتمًا؛ لأن الرب يسكن فيه وينير نفسه، ويفرح بجميع الناس دائمًا

\par 37 ليس للعقل الصالح لسانان، لسان للبركة واللعنة، لسان للازدراء والشرف، لسان للحزن والفرح، لسان للهدوء والارتباك، لسان للنفاق ولساني للصدق، لسان للفقر والغنى؛ بل له طبع واحد، نقي ونقي، تجاه جميع الناس

\par 38 ليس له نظر مزدوج، ولا سمع مزدوج. لأنه في كل ما يفعله، أو يتكلم به، أو يراه، يعلم أن الرب ينظر إلى نفسه.

\par 39 ويُطهِّر عقله لئلا يُدان من الناس كما من الله

\par 40 وبالمثل، فإن أعمال بليار مزدوجة، وليس فيها أي وحدة

\par 41 لذلك، يا أبنائي، أقول لكم: اهربوا من شر بليعار، فإنه يعطي سيفًا لمن يطيعونه

\par 42 والسيف أم الشرور السبعة. أولًا، يتصور العقل من خلال بليار، وأولًا، سفك الدماء؛ ثانيًا، الخراب؛ ثالثًا، الضيق؛ رابعًا، النفي؛ خامسًا، المجاعة؛ سادسًا، الذعر؛ سابعًا، الدمار

\par 43 لذلك أُسلم قايين أيضًا إلى سبع انتقامات من الله، لأنه في كل مئة سنة كان الرب يُنزل عليه ضربة واحدة

\par 44 ولما بلغ من العمر مئتي سنة بدأ يتألم، وفي السنة التسعمائة هلك

\par 45 لأنه بسبب هابيل أخيه، حوكم بكل الشرور، أما لامك فقد حوكم سبعين مرة سبع مرات

\par 46 لأنه إلى الأبد، أولئك الذين يشبهون قابيل في الحسد وكراهية الإخوة، سيُعاقبون بنفس الحكم

\chapter{2}

\par \textit{تحتوي الآية 3 على مثال صارخ على بساطة وحيوية أسلوب هؤلاء الآباء القدماء في استخدام المجازات.}

\par 1 وأنتم يا أبنائي، اهربوا من فعل الشر والحسد وكراهية الإخوة، وتمسكوا بالخير والمحبة

\par 2 من كان له فكر طاهر في المحبة، فلا ينظر إلى امرأة بقصد الزنا؛ لأنه ليس في قلبه دنس، لأن روح الله يحل عليه

\par 3 فكما أن الشمس لا تتنجس بإشراقها على الروث والطين، بل تجففهما وتطرد الرائحة الكريهة، كذلك فإن العقل الطاهر، على الرغم من أنه محاط بنجاسات الأرض، فإنه يطهرها ولا يتنجس هو نفسه

\par 4 وأعتقد أنه ستكون بينكم أيضًا أعمال شريرة، من أقوال أخنوخ البار: ستزنون زنا سدوم، وستهلكون جميعًا إلا قليلًا، وتجددون الفجور مع النساء. ولن تكون مملكة الرب بينكم، لأنه سينزعها في الحال

\par 5 ومع ذلك، سيكون هيكل الله من نصيبكم، وسيكون الهيكل الأخير أكثر مجدًا من الأول

\par 6 ويُجمع هناك الأسباط الاثنا عشر، وجميع الأمم، حتى يرسل العلي خلاصه في زيارة نبي وحيد

\par 7 ويدخل الهيكل الأول، وهناك يُعامل الرب بغضب، ويُرفع على خشبة

\par 8 وينشق حجاب الهيكل، وينتقل روح الله إلى الأمم كنار مسكوبة

\par 9 ويصعد من الجحيم وينتقل من الأرض إلى السماء.

\par 10 وأنا أعلم مدى تواضعه على الأرض، ومدى مجده في السماء.

\par 11 "وعندما كان يوسف في مصر، اشتقت إلى أن أنظر صورته وشكل وجهه، فبسبب صلاة يعقوب أبي، رأيته مستيقظاً في النهار، حتى كل صورته كما هو."

\par 12 ولما قال هذا، قال لهم: اعلموا إذن يا أولادي أنني أموت

\par 13 فأصدقوا كل واحد قريبه، واحفظوا شريعة الرب ووصاياه

\par 14 لهذه الأشياء أترك لكم ميراثًا.

\par 15 فأعطوها أنتم أيضاً لأولادكم ملكاً أبدياً، كما فعل إبراهيم وإسحق ويعقوب أيضاً.

\par 16 لأن هذه جميعها أعطونا ميراثًا قائلين: احفظوا وصايا الله، حتى يُظهر الرب خلاصه لجميع الأمم

\par 17 وحينئذٍ ترون حنوك ونوحًا وسامًا وإبراهيم وإسحاق ويعقوب يقومون عن اليمين بفرح،

\par 18 حينئذٍ نقوم نحن أيضًا، كل واحد فوق سبطه، عابدين لملك السماء، الذي ظهر على الأرض في صورة إنسان متواضعًا

\par 19 وكل من يؤمن به على الأرض يفرح معه.

\par 20 حينئذٍ سوف يرتفع جميع الناس، بعضهم إلى المجد وبعضهم إلى العار.

\par 21 وسيُدين الرب إسرائيل أولاً على إثمهم؛ لأنه عندما ظهر كإله في الجسد ليُنقذهم لم يُؤمنوا به

\par 22 وحينئذٍ يدين جميع الأمم، كل الذين لم يؤمنوا به حين ظهر على الأرض

\par 23 ويُوبِّخ إسرائيل على يد مختاري الأمم، كما وبَّخ عيسو على يد المديانيين الذين خدعوا إخوتهم حتى وقعوا في الزنا وعبادة الأصنام، وانفصلوا عن الله، فصاروا أبناءً في نصيب الذين يتقيون الرب

\par 24 فإن كنتم يا أبنائي تسلكون في القداسة حسب وصايا الرب، فستسكنون معي آمنين مرة أخرى، ويجتمع كل إسرائيل إلى الرب

\par 25 ولن أُدعى بعد ذئبًا هائجًا بسبب تخريبكم، بل عاملًا في خدمة الرب، يُوزع الطعام على صانعي الخير

\par 26 ويقوم في آخر الأيام رجل محبوب للرب من سبط يهوذا ولاوي، عامل مسرته بفمه، بمعرفة جديدة تُنير الأمم

\par 27 إلى انقضاء الدهر يكون في مجامع الأمم، وبين رؤسائهم، كنغمة موسيقى في أفواه الجميع

\par 28 ويُكتب في الكتب المقدسة عمله وكلمته، ويكون مختارًا من الله إلى الأبد

\par 29 ويسير بهم ذهابًا وإيابًا كما فعل يعقوب أبي، قائلًا: يُكمل ما نقص من سبطك

\par 30 ولما قال هذا مدّ رجليه.

\par 31 ومات في نوم جميل وهادئ.

\par 32 ففعل بنوه كما أوصاهم، فحملوا جثته ودفنوها في حبرون مع آبائه.

\par 33 وكان عدد أيام حياته مئة وخمسًا وعشرين سنة


\end{document}