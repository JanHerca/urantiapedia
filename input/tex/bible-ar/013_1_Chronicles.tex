\begin{document}

\title{1 اخبار}


\chapter{1}

\par 1 آدَمُ, شِيتُ, أَنُوشُ,
\par 2 قِينَانُ, مَهْلَلْئِيلُ, يَارِدُ,
\par 3 أَخْنُوخُ, مَتُوشَالَحُ, لاَمَكُ,
\par 4 نُوحُ, سَامُ, حَامُ, يَافَثُ.
\par 5 بَنُو يَافَثَ: جُومَرُ وَمَاجُوجُ وَمَادَايُ وَيَاوَانُ وَتُوبَالُ وَمَاشِكُ وَتِيرَاسُ.
\par 6 وَبَنُو جُومَرَ: أَشْكَنَازُ وَرِيفَاثُ وَتُوجَرْمَةُ.
\par 7 وَبَنُو يَاوَانَ: أَلِيشَةُ وَتَرْشِيشَةُ وَكِتِّيمُ وَدُودَانِيمُ.
\par 8 بَنُو حَامَ: كُوشُ وَمِصْرَايِمُ وَفُوطُ وَكَنْعَانُ.
\par 9 وَبَنُو كُوشَ: سَبَا وَحَوِيلَةُ وَسَبْتَا وَرَعَمَا وَسَبْتَكَ. وَبَنُو رَعَمَا: شَبَا وَدَدَانُ.
\par 10 وَكُوشُ وَلَدَ نِمْرُودَ الَّذِي ابْتَدَأَ يَكُونُ جَبَّاراً فِي الأَرْضِ.
\par 11 وَمِصْرَايِمُ وَلَدَ: لُودِيمَ وَعَنَامِيمَ وَلَهَابِيمَ وَنَفْتُوحِيمَ
\par 12 وَفَتْرُوسِيمَ وَكَسْلُوحِيمَ (اَلَّذِينَ خَرَجَ مِنْهُمْ فِلِشْتِيمُ وَكَفْتُورِيمُ).
\par 13 وَكَنْعَانُ وَلَدَ: صَيْدُونَ بِكْرَهُ, وَحِثّاً
\par 14 وَالْيَبُوسِيَّ وَالأَمُورِيَّ وَالْجِرْجَاشِيَّ
\par 15 وَالْحِوِّيَّ وَالْعَرْقِيَّ وَالسِّينِيَّ
\par 16 وَالأَرْوَادِيَّ وَالصَّمَّارِيَّ وَالْحَمَاثِيَّ.
\par 17 بَنُو سَامَ: عِيلاَمُ وَأَشُّورُ وَأَرْفَكْشَادُ وَلُودُ وَأَرَامُ وَعُوصُ وَحُولُ وَجَاثَرُ وَمَاشِكُ.
\par 18 وَأَرْفَكْشَادُ وَلَدَ شَالَحَ وَشَالَحُ وَلَدَ عَابِرَ.
\par 19 وَلِعَابِرَ وُلِدَ ابْنَانِ اسْمُ الْوَاحِدِ فَالَجُ, لأَنَّ فِي أَيَّامِهِ قُسِمَتِ الأَرْضُ. وَاسْمُ أَخِيهِ يَقْطَانُ.
\par 20 وَيَقْطَانُ وَلَدَ: أَلْمُودَادَ وَشَالَفَ وَحَضَرْمَوْتَ وَيَارَحَ
\par 21 وَهَدُورَامَ وَأُوزَالَ وَدِقْلَةَ
\par 22 وَعِيبَالَ وَأَبِيمَايِلَ وَشَبَا
\par 23 وَأُوفِيرَ وَحَوِيلَةَ وَيُوبَابَ. كُلُّ هَؤُلاَءِ بَنُو يَقْطَانَ.
\par 24 سَامُ, أَرْفَكْشَادُ, شَالَحُ,
\par 25 عَابِرُ, فَالَجُ, رَعُو,
\par 26 سَرُوجُ, نَاحُورُ, تَارَحُ,
\par 27 أَبْرَامُ (وَهُوَ إِبْرَاهِيمُ).
\par 28 اِبْنَا إِبْرَاهِيمَ إِسْحَاقُ وَإِسْمَاعِيلُ.
\par 29 هَذِهِ مَوَالِيدُهُمْ. بِكْرُ إِسْمَاعِيلَ: نَبَايُوتُ, وَقِيدَارُ وَأَدَبْئِيلُ وَمِبْسَامُ
\par 30 وَمِشْمَاعُ وَدُومَةُ وَمَسَّا وَحَدَدُ وَتَيْمَاءُ
\par 31 وَيَطُورُ وَنَافِيشُ وَقِدْمَةُ. هَؤُلاَءِ هُمْ بَنُو إِسْمَاعِيلَ.
\par 32 وَأَمَّا بَنُو قَطُورَةَ سُرِّيَّةِ إِبْرَهِيمَ فَإِنَّهَا وَلَدَتْ: زِمْرَانَ وَيَقْشَانَ وَمَدَانَ وَمِدْيَانَ وَيِشْبَاقَ وَشُوحاً. وَابْنَا يَقْشَانَ شَبَا وَدَدَانُ.
\par 33 وَبَنُو مِدْيَانَ: عَيْفَةُ وَعِفْرُو وَحَنُوكُ وَأَبِيدَاعُ وَأَلْدَعَةُ. فَكُلُّ هَؤُلاَءِ بَنُو قَطُورَةَ.
\par 34 وَوَلَدَ إِبْرَاهِيمُ إِسْحَاقَ. وَابْنَا إِسْحَاقَ: عِيسُو وَإِسْرَائِيلُ.
\par 35 بَنُو عِيسُو: أَلِيفَازُ وَرَعُوئِيلُ وَيَعُوشُ وَيَعْلاَمُ وَقُورَحُ.
\par 36 بَنُو أَلِيفَازَ: تَيْمَانُ وَأُومَارُ وَصَفِي وَجَعْثَامُ وَقِنَازُ وَتِمْنَاعُ وَعَمَالِيقُ.
\par 37 بَنُو رَعُوئِيلَ: نَحَثُ وَزَارَحُ وَشَمَّةُ وَمِزَّةُ.
\par 38 وَبَنُو سَعِيرَ: لُوطَانُ وَشُوبَالُ وَصِبْعُونُ وَعَنَى وَدِيشُونُ وَإِيصَرُ وَدِيشَانُ.
\par 39 وَابْنَا لُوطَانَ: حُورِي وَهُومَامُ. وَأُخْتُ لُوطَانَ تِمْنَاعُ.
\par 40 بَنُو شُوبَالَ: عَلْيَانُ وَمَنَاحَةُ وَعِيبَالُ وَشَفِي وَأُونَامُ. وَابْنَا صِبْعُونَ: أَيَّةُ وَعَنَى.
\par 41 اِبْنُ عَنَى دِيشُونُ, وَبَنُو دِيشُونَ: حَمْرَانُ وَأَشْبَانُ وَيِثْرَانُ وَكَرَانُ.
\par 42 بَنُو إِيصَرَ: بِلْهَانُ وَزَعْوَانُ وَيَعْقَانُ. وَابْنَا دِيشَانَ: عُوصُ وَأَرَانُ.
\par 43 هَؤُلاَءِ هُمُ الْمُلُوكُ الَّذِينَ مَلَكُوا فِي أَرْضِ أَدُومَ قَبْلَمَا مَلَكَ مَلِكٌ لِبَنِي إِسْرَائِيلَ: بَالَعُ بْنُ بَعُورَ. وَاسْمُ مَدِينَتِهِ دِنْهَابَةُ.
\par 44 وَمَاتَ بَالِعُ فَمَلَكَ مَكَانَهُ يُوبَابُ بْنُ زَارَحَ مِنْ بُصْرَةَ.
\par 45 وَمَاتَ يُوبَابُ فَمَلَكَ مَكَانَهُ حُوشَامُ مِنْ أَرْضِ التَّيْمَانِيِّ.
\par 46 وَمَاتَ حُوشَامُ فَمَلَكَ مَكَانَهُ هَدَدُ بْنُ بَدَدَ الَّذِي كَسَّرَ مِدْيَانَ فِي بِلاَدِ مُوآبَ, وَاسْمُ مَدِينَتِهِ عَوِيتُ
\par 47 وَمَاتَ هَدَدُ فَمَلَكَ مَكَانَهُ سِمْلَةُ مِنْ مَسْرِيقَةَ.
\par 48 وَمَاتَ سِمْلَةُ فَمَلَكَ مَكَانَهُ شَاوُلُ مِنْ رَحُوبُوتِ النَّهْرِ.
\par 49 وَمَاتَ شَاوُلُ فَمَلَكَ مَكَانَهُ بَعْلُ حَانَانَ بْنُ عَكْبُورَ.
\par 50 وَمَاتَ بَعْلُ حَانَانَ فَمَلَكَ مَكَانَهُ هَدَدُ, وَاسْمُ مَدِينَتِهِ فَاعِي, وَاسْمُ امْرَأَتِهِ مَهِيطَبْئِيلُ بِنْتُ مَطْرِدَ بِنْتِ مَاءِ ذَهَبٍ.
\par 51 وَمَاتَ هَدَدُ. فَكَانَتْ أُمَرَاءُ أَدُومَ: أَمِيرُ تِمْنَاعَ, أَمِيرُ عَلْوَةَ, أَمِيرُ يَتِيتَ,
\par 52 أَمِيرُ أُهُولِيبَامَةَ, أَمِيرُ أَيْلَةَ, أَمِيرُ فِينُونَ,
\par 53 أَمِيرُ قَِنَازَ, أَمِيرُ تَيْمَانَ, أَمِيرُ مِبْصَارَ,
\par 54 أَمِيرُ مَجْدِيئِيلَ, أَمِيرُ عِيرَامَ. هَؤُلاَءِ أُمَرَاءُ أَدُومَ.

\chapter{2}

\par 1 هَؤُلاَءِ بَنُو إِسْرَائِيلَ: رَأُوبَيْنُ, شَمْعُونُ, لاَوِي وَيَهُوذَا, يَسَّاكَرُ وَزَبُولُونُ,
\par 2 دَانُ, يُوسُفُ وَبِنْيَامِينُ, نَفْتَالِي, جَادُ وَأَشِيرُ.
\par 3 بَنُو يَهُوذَا: عَيْرُ وَأُونَانُ وَشَيْلَةُ. وُلِدَ الثَّلاَثَةُ مِنْ بِنْتِ شُوعَ الْكَنْعَانِيَّةِ. وَكَانَ عَيْرُ بِكْرُ يَهُوذَا شِرِّيراً فِي عَيْنَيِ الرَّبِّ فَأَمَاتَهُ.
\par 4 وَثَامَارُ كَنَّتُهُ وَلَدَتْ لَهُ فَارَصَ وَزَارَحَ. كُلُّ بَنِي يَهُوذَا خَمْسَةٌ.
\par 5 اِبْنَا فَارَصَ حَصْرُونُ وَحَامُولُ.
\par 6 وَبَنُو زَارَحَ: زِمْرِي وَأَيْثَانُ وَهَيْمَانُ وَكَلْكُولُ وَدَارَعُ. الْجَمِيعُ خَمْسَةٌ.
\par 7 وَابْنُ كَرْمِي عَخَانُ مُكَدِّرُ إِسْرَائِيلَ الَّذِي خَانَ فِي الْحَرَامِ.
\par 8 وَابْنُ أَيْثَانَ عَزَرْيَا.
\par 9 وَبَنُو حَصْرُونَ الَّذِينَ وُلِدُوا لَهُ: يَرْحَمْئِيلُ وَرَامُ وَكَلُوبَايُ.
\par 10 وَرَامُ وَلَدَ عَمِّينَادَابَ, وَعَمِّينَادَابُ وَلَدَ نَحْشُونَ رَئِيسَ بَنِي يَهُوذَا,
\par 11 وَنَحْشُونُ وَلَدَ سَلْمُونَ وَسَلْمُونُ وَلَدَ بُوعَزَ,
\par 12 وَبُوعَزُ وَلَدَ عُوبِيدَ, وَعُوبِيدُ وَلَدَ يَسَّى,
\par 13 وَيَسَّى وَلَدَ: بِكْرَهُ أَلِيآبَ وَأَبِينَادَابَ الثَّانِي وَشَمْعَى الثَّالِثَ
\par 14 وَنَثْنِئِيلَ الرَّابِعَ وَرَدَّايَ الْخَامِسَ
\par 15 وَأُوصَمَ السَّادِسَ وَدَاوُدَ السَّابِعَ.
\par 16 وَأُخْتَاهُمْ صَرُويَةُ وَأَبِيجَايِلُ. وَبَنُو صَرُويَةَ أَبْشَايُ وَيُوآبُ وَعَسَائِيلُ ثَلاَثَةٌ.
\par 17 وَأَبِيجَايِلُ وَلَدَتْ عَمَاسَا, وَأَبُو عَمَاسَا يَثْرُ الإِسْمَاعِيلِيُّ.
\par 18 وَكَالِبُ بْنُ حَصْرُونَ وَلَدَ مِنْ عَزُوبَةَ امْرَأَتِهِ وَمِنْ يَرِيعُوثَ. وَهَؤُلاَءِ بَنُوهَا: يَاشَرُ وَشُوبَابُ وَأَرْدُونُ.
\par 19 وَمَاتَتْ عَزُوبَةُ فَاتَّخَذَ كَالِبُ لِنَفْسِهِ أَفْرَاتَ فَوَلَدَتْ لَهُ حُورَ.
\par 20 وَحُورُ وَلَدَ أُورِيَ وَأُورِي وَلَدَ بَصَلْئِيلَ.
\par 21 وَبَعْدُ دَخَلَ حَصْرُونُ عَلَى بِنْتِ مَاكِيرَ أَبِي جِلْعَادَ وَاتَّخَذَهَا وَهُوَ ابْنُ سِتِّينَ سَنَةً فَوَلَدَتْ لَهُ سَجُوبَ.
\par 22 وَسَجُوبُ وَلَدَ يَائِيرَ, وَكَانَ لَهُ ثَلاَثٌ وَعِشْرُونَ مَدِينَةً فِي أَرْضِ جِلْعَادَ.
\par 23 وَأَخَذَ جَشُورَ وَأَرَامَ حَوُّوثَ يَائِيرَ مِنْهُمْ مَعَ قَنَاةَ وَقُرَاهَا, سِتِّينَ مَدِينَةً. كُلُّ هَؤُلاَءِ بَنُو مَاكِيرَ أَبِي جِلْعَادَ.
\par 24 وَبَعْدَ وَفَاةِ حَصْرُونَ فِي كَالِبِ أَفْرَاتَةَ وَلَدَتْ لَهُ أَبِيَّاهُ امْرَأَةُ حَصْرُونَ أَشْحُورَ أَبَا تَقُوعَ.
\par 25 وَكَانَ بَنُو يَرْحَمْئِيلَ بِكْرِ حَصْرُونَ: الْبِكْرُ رَامَ, ثُمَّ بُونَةَ وَأَوْرَنَا وَأَوْصَمَ وَأَخِيَّا.
\par 26 وَكَانَتِ امْرَأَةٌ أُخْرَى لِيَرْحَمْئِيلَ اسْمُهَا عَطَارَةُ. هِيَ أُمُّ أُونَامَ.
\par 27 وَكَانَ بَنُو رَامَ بِكْرِ يَرْحَمْئِيلَ: مَعَصٌ وَيَمِينُ وَعَاقَرُ.
\par 28 وَكَانَ ابْنَا أُونَامَ: شَمَّايَ وَيَادَاعَ. وَابْنَا شَمَّايَ: نَادَابَ وَأَبِيشُورَ.
\par 29 وَاسْمُ امْرَأَةِ أَبِيشُورَ أَبِيجَايِلُ, وَوَلَدَتْ لَهُ أَحْبَانَ وَمُولِيدَ.
\par 30 وَابْنَا نَادَابَ: سَلَدُ وَأَفَّايِمُ. وَمَاتَ سَلَدُ بِلاَ بَنِينَ.
\par 31 وَابْنُ أَفَّايِمَ يَشْعِي وَابْنُ يَشْعِي شِيشَانُ وَابْنُ شِيشَانَ أَحْلاَيُ.
\par 32 وَابْنَا يَادَاعَ أَخِي شَمَّايَ يَثَرُ وَيُونَاثَانُ. وَمَاتَ يَثَرُ بِلاَ بَنِينَ.
\par 33 وَابْنَا يُونَاثَانَ فَالَتُ وَزَازَا. هَؤُلاَءِ هُمْ بَنُو يَرْحَمْئِيلَ.
\par 34 وَلَمْ يَكُنْ لِشِيشَانَ بَنُونَ بَلْ بَنَاتٌ. وَكَانَ لِشِيشَانَ عَبْدٌ مِصْرِيٌّ اسْمُهُ يَرْحَعُ,
\par 35 فَأَعْطَى شِيشَانُ ابْنَتَهُ لِيَرْحَعَ عَبْدِهِ امْرَأَةً فَوَلَدَتْ لَهُ عَتَّايَ.
\par 36 وَعَتَّايُ وَلَدَ نَاثَانَ وَنَاثَانُ وَلَدَ زَابَادَ
\par 37 وَزَابَادُ وَلَدَ أَفْلاَلَ وَأَفْلاَلُ وَلَدَ عُوبِيدَ
\par 38 وَعُوبِيدُ وَلَدَ يَاهُوَ وَيَاهُو وَلَدَ عَزَرْيَا
\par 39 وَعَزَرْيَا وَلَدَ حَالَصَ وَحَالَصُ وَلَدَ أَلْعَاسَةَ
\par 40 وَأَلْعَاسَةُ وَلَدَ سِسَمَايَ وَسِسَمَايُ وَلَدَ شَلُّومَ
\par 41 وَشَلُّومُ وَلَدَ يَقَمْيَةَ وَيَقَمْيَةُ وَلَدَ أَلِيشَمَعَ.
\par 42 وَبَنُو كَالِبَ أَخِي يَرْحَمْئِيلَ: مِيشَاعُ بِكْرُهُ. هُوَ أَبُو زِيفَ. وَبَنُو مَرِيشَةَ أَبِي حَبْرُونَ.
\par 43 وَبَنُو حَبْرُونَ: قُورَحُ وَتَفُّوحُ وَرَاقَمُ وَشَامَعُ.
\par 44 وَشَامَعُ وَلَدَ رَاقَمَ أَبَا يَرُقْعَامَ. وَرَاقَمُ وَلَدَ شَمَّايَ.
\par 45 وَابْنُ شَمَّايَ مَعُونُ وَمَعُونُ أَبُو بَيْتِ صُورَ.
\par 46 وَعِيفَةُ سُرِّيَّةُ كَالِبَ وَلَدَتْ: حَارَانَ وَمُوصَا وَجَازِيزَ. وَحَارَانُ وَلَدَ جَازِيزَ.
\par 47 وَبَنُو يَهْدَايَ: رَجَمُ وَيُوثَامُ وَجِيشَانُ وَفَلَطُ وَعِيفَةُ وَشَاعَفُ.
\par 48 وَأَمَّا مَعْكَةُ سُرِّيَّةُ كَالِبَ فَوَلَدَتْ شَبَرَ وَتَرْحَنَةَ.
\par 49 وَوَلَدَتْ شَاعَفُ أَبَا مَدْمَنَّةَ وَشَوَا أَبَا مَكْبِينَا وَأَبَا جَبَعَا. وَبِنْتُ كَالِبَ عَكْسَةُ.
\par 50 هَؤُلاَءِ هُمْ بَنُو كَالِبَ بْنِ حُورَ بِكْرِ أَفْرَاتَةَ. شُوبَالُ أَبُو قَرْيَةِ يَعَارِيمَ
\par 51 وَسَلْمَا أَبُو بَيْتِ لَحْمٍ, وَحَارِيفُ أَبُو بَيْتِ جَادِيرَ.
\par 52 وَكَانَ لِشُوبَالَ أَبِي قَرْيَةِ يَعَارِيمَ بَنُونَ: هَرُواهُ وَحَصِي هَمَّنُوحُوتَ.
\par 53 وَعَشَائِرُ قَرْيَةِ يَعَارِيمَ: الْيِثْرِيُّ وَالْفُوتِيُّ وَالشَّمَاتِيُّ وَالْمَشْرَاعِيُّ. مِنْ هَؤُلاَءِ خَرَجَ الصَّرْعِيُّ وَالأَشْتَأُولِيُّ.
\par 54 بَنُو سَلْمَا: بَيْتُ لَحْمٍ وَالنَّطُوفَاتِيُّ وَعَطْرُوتُ بَيْتِ يُوآبَ وَحَصِي الْمَنُوحِيِّ الصَّرْعِيِّ.
\par 55 وَعَشَائِرُ الْكَتَبَةِ سُكَّانِ يَعْبِيصَ: تَرْعَاتِيمُ وَشَمْعَاتِيمُ وَسُوكَاتِيمُ. هُمُ الْقِينِيُّونَ الْخَارِجُونَ مِنْ حَمَّةَ أَبِي بَيْتِ رَكَابَ.

\chapter{3}

\par 1 وَهَؤُلاَءِ هُمْ بَنُو دَاوُدَ الَّذِينَ وُلِدُوا لَهُ فِي حَبْرُونَ: الْبِكْرُ أَمْنُونُ مِنْ أَخِينُوعَمَ الْيَزْرَعِيلِيَّةِ. الثَّانِي دَانِيئِيلُ مِنْ أَبِيجَايِلَ الْكَرْمَلِيَّةِ
\par 2 الثَّالِثُ أَبْشَالُومُ ابْنُ مَعْكَةَ بِنْتِ تَلْمَايَ مَلِكِ جَشُورَ. الرَّابِعُ أَدُونِيَّا ابْنُ حَجِّيثَ
\par 3 الْخَامِسُ شَفَطْيَا مِنْ أَبِيطَالَ. السَّادِسُ يَثَرْعَامُ مِنْ عَجْلَةَ امْرَأَتِهِ.
\par 4 وُلِدَ لَهُ سِتَّةٌ فِي حَبْرُونَ. وَمَلَكَ هُنَاكَ سَبْعَ سِنِينٍ وَسِتَّةَ أَشْهُرٍ, ثُمَّ مَلَكَ ثَلاَثاً وَثَلاَثِينَ سَنَةً فِي أُورُشَلِيمَ:
\par 5 وَهَؤُلاَءِ وُلِدُوا لَهُ فِي أُورُشَلِيمَ. شَمْعَى وَشُوبَابُ وَنَاثَانُ وَسُلَيْمَانُ. أَرْبَعَةٌ مِنْ بَثْشُوعَ بِنْتِ عَمِّيئِيلَ.
\par 6 وَيِبْحَارُ وَأَلِيشَامَعُ وَأَلِيفَالَطُ
\par 7 وَنُوجَهُ وَنَافَجُ وَيَافِيعُ
\par 8 وَأَلِيشَمَعُ وَأَلِيَادَاعُ وَأَلِيفَلَطُ. تِسْعَةٌ.
\par 9 الْكُلُّ بَنُو دَاوُدَ مَا عَدَا بَنِي السَّرَارِيِّ. وَثَامَارُ هِيَ أُخْتُهُمْ.
\par 10 وَابْنُ سُلَيْمَانَ رَحُبْعَامُ وَابْنُهُ أَبِيَّا وَابْنُهُ آسَا وَابْنُهُ يَهُوشَافَاطُ
\par 11 وَابْنُهُ يَهُورَامُ وَابْنُهُ أَخَزْيَا وَابْنُهُ يَهُوآشُ
\par 12 وَابْنُهُ أَمَصْيَا وَابْنُهُ عَزَرْيَا وَابْنُهُ يُوثَامُ
\par 13 وَابْنُهُ آحَازُ وَابْنُهُ حَزَقِيَّا وَابْنُهُ مَنَسَّى
\par 14 وَابْنُهُ آمُونُ وَابْنُهُ يُوشِيَّا.
\par 15 وَبَنُو يُوشِيَّا: الْبِكْرُ يُوحَانَانُ, الثَّانِي يَهُويَاقِيمُ, الثَّالِثُ صِدْقِيَّا, الرَّابِعُ شَلُّومُ.
\par 16 وَابْنَا يَهُويَاقِيمَ: يَكُنْيَا وَصِدْقِيَّا.
\par 17 وَابْنَا يَكُنْيَا: أَسِّيرُ وَشَأَلْتِئِيلُ ابْنُهُ
\par 18 وَمَلْكِيرَامُ وَفَدَايَا وَشِنْأَصَّرُ وَيَقَمْيَا وَهُوشَامَاعُ وَنَدَبْيَا.
\par 19 وَابْنَا فَدَايَا: زَرُبَّابِلُ وَشَمْعِي. وَبَنُو زَرُبَّابِلَ: مَشُلاَّمُ وَحَنَنْيَا وَشَلُومِيَةُ أُخْتُهُمْ
\par 20 وَحَشُوبَةُ وَأُوهَلُ وَبَرَخْيَا وَحَسَدْيَا وَيُوشَبُ حَسَدَ. خَمْسَةٌ.
\par 21 وَبَنُو حَنَنْيَا: فَلَطْيَا وَيِشْعِيَا وَبَنُو رَفَايَا وَبَنُو أُرْنَانَ وَبَنُو عُوبَدْيَا وَبَنُو شَكَنْيَا.
\par 22 وَبَنُو شَكَنْيَا: شَمَعْيَا وَبَنُو شَمَعْيَا حَطُّوشُ وَيَجْآلُ وَبَارِيحُ وَنَعَرْيَا وَشَافَاطُ. سِتَّةٌ.
\par 23 وَبَنُو نَعَرْيَا: الْيُوعِينِيُّ وَحَزَقِيَّا وَعَزْرِيقَامُ. ثَلاَثَةٌ.
\par 24 وَبَنُو الْيُوعِينِيِّ: هُودَايَاهُو وَأَلْيَاشِيبُ وَفَلاَيَا وَعَقُّوبُ وَيُوحَانَانُ وَدَلاَيَا وَعَنَانِي. سَبْعَةٌ.

\chapter{4}

\par 1 بَنُو يَهُوذَا: فَارَصُ وَحَصْرُونُ وَكَرْمِي وَحُورُ وَشُوبَالُ.
\par 2 وَرَآيَا بْنُ شُوبَالَ وَلَدَ يَحَثَ, وَيَحَثُ وَلَدَ أَخُومَايَ وَلاَهَدَ. هَذِهِ عَشَائِرُ الصَّرْعِيِّينَ.
\par 3 وَهَؤُلاَءِ لأَبِي عِيطَمَ: يَزْرَعِيلُ وَيَشْمَا وَيَدْبَاشُ, وَاسْمُ أُخْتِهِمْ هَصَّلَلْفُونِي.
\par 4 وَفَنُوئِيلُ أَبُو جَدُورَ وَعَازَرُ أَبُو حُوشَةَ. هَؤُلاَءِ بَنُو حُورَ بِكْرِ أَفْرَاتَةَ أَبِي بَيْتِ لَحْمٍ.
\par 5 وَكَانَ لأَشْحُورَ أَبِي تَقُوعَ امْرَأَتَانِ: حَلاَةُ وَنَعْرَةُ.
\par 6 وَوَلَدَتْ لَهُ نَعْرَةُ: أَخُزَّامَ وَحَافَرَ وَالتَّيْمَانِيَّ وَالأَخَشْتَارِيَّ. هَؤُلاَءِ بَنُو نَعْرَةَ.
\par 7 وَبَنُو حَلاَةَ: صَرَثُ وَصُوحَرُ وَأَثْنَانُ.
\par 8 وَقُوصُ وَلَدَ عَانُوبَ وَهَصُوبِيبَةَ وَعَشَائِرَ أَخَرْحِيلَ بْنِ هَارُمَ.
\par 9 وَكَانَ يَعْبِيصُ أَشْرَفَ مِنْ إِخْوَتِهِ. وَسَمَّتْهُ أُمُّهُ يَعْبِيصَ قَائِلَةً: «لأَنِّي وَلَدْتُهُ بِحُزْنٍ».
\par 10 وَدَعَا يَعْبِيصُ إِلَهَ إِسْرَائِيلَ: «لَيْتَكَ تُبَارِكُنِي, وَتُوَسِّعُ تُخُومِي, وَتَكُونُ يَدُكَ مَعِي, وَتَحْفَظُنِي مِنَ الشَّرِّ حَتَّى لاَ يُتْعِبُنِي». فَآتَاهُ اللَّهُ بِمَا سَأَلَ.
\par 11 وَكَلُوبُ أَخُو شُوحَةَ وَلَدَ مَحِيرَ. هُوَ أَبُو أَشْتُونَ.
\par 12 وَأَشْتُونُ وَلَدَ بَيْتَ رَافَا وَفَاسِحَ وَتَحِنَّةَ أَبَا مَدِينَةِ نَاحَاشَ. هَؤُلاَءِ أَهْلُ رَيْكَةَ.
\par 13 وَابْنَا قَنَازَ: عُثْنِيئِيلُ وَسَرَايَا, وَابْنُ عُثْنِيئِيلَ حَثَاثُ.
\par 14 وَمَعُونُوثَايُ وَلَدَ عَفْرَةَ, وَسَرَايَا وَلَدَ يُوآبَ أَبَا وَادِي الصُّنَّاعِ (لأَنَّهُمْ كَانُوا صُنَّاعاً).
\par 15 وَبَنُو كَالِبَ بْنِ يَفُنَّةَ عِيرُو وَأَيْلَةُ وَنَاعِمُ. وَابْنُ أَيْلَةَ قَنَازُ.
\par 16 وَبَنُو يَهْلَلْئِيلَ: زِيفُ وَزِيفَةُ وَتِيرِيَّا وَأَسَرْئِيلُ.
\par 17 وَبَنُو عَزْرَةَ: يَثَرُ وَمَرَدُ وَعَافِرُ وَيَالُونُ. وَحَبِلَتْ بِمَرْيَمَ وَشَمَّايَ وَيِشْبَحَ أَبِي أَشْتَمُوعَ.
\par 18 (وَامْرَأَتُهُ الْيَهُودِيَّةُ وَلَدَتْ يَارِدَ أَبَا جَدُورَ, وَحَابِرَ أَبَا سُوكُوَ, وَيَقُوثِيئِيلَ أَبَا زَانُوحَ). وَهَؤُلاَءِ بَنُو بِثْيَةَ بِنْتِ فِرْعَوْنَ الَّتِي أَخَذَهَا مَرَدُ.
\par 19 وَبَنُو امْرَأَتِهِ الْيَهُودِيَّةِ أُخْتِ نَحَمَ أَبِي قَعِيلَةَ الْجَرْمِيِّ وَأَشْتَمُوعَ الْمَعْكِيِّ.
\par 20 وَبَنُو شِيمُونَ: أَمْنُونُ وَرِنَّةُ بْنُ حَانَانَ وَتِيلُونُ. وَابْنَا يِشْعِي: زُوحَيْتُ وَبَنْزُوحَيْتُ.
\par 21 بَنُو شِيلَةَ بْنِ يَهُوذَا: عِيرُ أَبُو لَيْكَةَ, وَلَعْدَةُ أَبُو مَرِيشَةَ, وَعَشَائِرِ بَيْتِ عَامِلِي الْبَزِّ مِنْ بَيْتِ أَشْبَيْعَ,
\par 22 وَيُوقِيمُ, وَأَهْلُ كَزِيبَا, وَيُوآشُ وَسَارَافُ, الَّذِينَ هُمْ أَصْحَابُ مُوآبَ وَيَشُوبِي لَحْمٍ. وَهَذِهِ الأُمُورُ قَدِيمَةٌ.
\par 23 هَؤُلاَءِ هُمُ الْخَزَّافُونَ وَسُكَّانُ نَتَاعِيمَ وَجَدِيرَةَ. أَقَامُوا هُنَاكَ مَعَ الْمَلِكِ لِشُغْلِهِ.
\par 24 بَنُو شَمْعُونَ: نَمُوئِيلُ وَيَامِينُ وَيَرِيبُ وَزَارَحُ وَشَاوُلُ,
\par 25 وَابْنُهُ شَلُّومُ وَابْنُهُ مِبْسَامُ وَابْنُهُ مِشْمَاعُ.
\par 26 وَبَنُو مِشْمَاعَ: حَمُوئِيلُ ابْنُهُ زَكُّورُ ابْنُهُ شَمْعِي ابْنُهُ.
\par 27 وَكَانَ لِشَمْعِي سِتَّةَ عَشَرَ ابْناً وَسِتَّ بَنَاتٍ. وَأَمَّا إِخْوَتُهُ فَلَمْ يَكُنْ لَهُمْ بَنُونَ كَثِيرُونَ, وَكُلُّ عَشَائِرِهِمْ لَمْ يَكْثُرُوا مِثْلَ بَنِي يَهُوذَا.
\par 28 وَأَقَامُوا فِي بِئْرِ سَبْعٍ وَمُولاَدَةَ وَحَصَرِ شُوعَالَ
\par 29 وَفِي بِلْهَةَ وَعَاصِمَ وَتُولاَدَ
\par 30 وَفِي بَتُوئِيلَ وَحُرْمَةَ وَصِقْلَغَ
\par 31 وَفِي بَيْتِ مَرْكَبُوتَ وَحَصَرِ سُوسِيمَ وَبَيْتِ بِرْئِي وَشَعَرَايِمَ. هَذِهِ مُدُنُهُمْ إِلَى حِينَمَا مَلَكَ دَاوُدُ.
\par 32 وَقُرَاهُمْ عِيطَمُ وَعَيْنٌ وَرِمُّونُ وَتُوكَنُ وَعَاشَانُ, خَمْسُ مُدُنٍ.
\par 33 وَجَمِيعُ قُرَاهُمُ الَّتِي حَوْلَ هَذِهِ الْمُدُنِ إِلَى بَعْلٍ. هَذِهِ مَسَاكِنُهُمْ وَأَنْسَابُهُمْ.
\par 34 وَمَشُوبَابُ وَيَمْلِيكُ وَيُوشَا بْنُ أَمَصْيَا,
\par 35 وَيُوئِيلُ وَيَاهُو بْنُ يُوشِبْيَا بْنِ سَرَايَا بْنِ عَسِيئيلَ,
\par 36 وَأَلِيُوعِينَايُ وَيَعْقُوبَا وَيَشُوحَايَا وَعَسَايَا وَعَدِيئِيلُ وَيَسِيمِيئِيلُ وَبَنَايَا
\par 37 وَزِيزَا بْنُ شِفْعِي بْنِ أَلُّونَ بْنِ يَدَايَا بْنِ شِمْرِي بْنِ شَمَعْيَا.
\par 38 هَؤُلاَءِ الْوَارِدُونَ بِأَسْمَائِهِمْ رُؤَسَاءُ فِي عَشَائِرِهِمْ وَبُيُوتِ آبَائِهِمِ امْتَدُّوا كَثِيراً,
\par 39 وَسَارُوا إِلَى مَدْخَلِ جَدُورَ إِلَى شَرْقِيِّ الْوَادِي لِيُفَتِّشُوا عَلَى مَرْعًى لِمَاشِيَتِهِمْ.
\par 40 فَوَجَدُوا مَرْعًى خَصِباً وَجَيِّداً, وَكَانَتِ الأَرْضُ وَاسِعَةَ الأَطْرَافِ مُسْتَرِيحَةً وَمُطْمَئِنَّةً, لأَنَّ آلَ حَامَ سَكَنُوا هُنَاكَ فِي الْقَدِيمِ.
\par 41 وَجَاءَ هَؤُلاَءِ الْمَكْتُوبَةُ أَسْمَاؤُهُمْ فِي أَيَّامِ حَزَقِيَّا مَلِكِ يَهُوذَا. وَضَرَبُوا خِيَمَهُمْ وَالْمَعُونِيِّينَ الَّذِينَ وُجِدُوا هُنَاكَ وَحَرَّمُوهُمْ إِلَى هَذَا الْيَوْمِ, وَسَكَنُوا مَكَانَهُمْ لأَنَّ هُنَاكَ مَرْعىً لِمَاشِيَتِهِمْ.
\par 42 وَمِنْهُمْ مِنْ بَنِي شَمْعُونَ ذَهَبَ إِلَى جَبَلِ سَعِيرَ خَمْسُ مِئَةِ رَجُلٍ, وَقُدَّامَهُمْ فَلَطْيَا وَنَعَرْيَا وَرَفَايَا وَعُزِّيئِيلُ بَنُو يِشْعِي.
\par 43 وَضَرَبُوا بَقِيَّةَ الْمُنْفَلِتِينَ مِنْ عَمَالِيقَ, وَسَكَنُوا هُنَاكَ إِلَى هَذَا الْيَوْمِ.

\chapter{5}

\par 1 وَبَنُو رَأُوبَيْنَ بِكْرِ إِسْرَائِيلَ. لأَنَّهُ هُوَ الْبِكْرُ وَلأَجْلِ تَدْنِيسِهِ فِرَاشَ أَبِيهِ, أُعْطِيَتْ بَكُورِيَّتُهُ لِبَنِي يُوسُفَ بْنِ إِسْرَائِيلَ, فَلَمْ يُنْسَبْ بِكْراً.
\par 2 لأَنَّ يَهُوذَا اعْتَزَّ عَلَى إِخْوَتِهِ وَمِنْهُ الرَّئِيسُ, وَأَمَّا الْبَكُورِيَّةُ فَلِيُوسُفَ.
\par 3 بَنُو رَأُوبَيْنَ بِكْرِ إِسْرَائِيلَ: حَنُوكُ وَفَلُّو وَحَصْرُونُ وَكَرْمِي.
\par 4 بَنُو يُوئِيلَ ابْنُهُ شَمَعْيَا وَابْنُهُ جُوجُ وَابْنُهُ شَمْعِي.
\par 5 وَابْنُهُ مِيخَا وَابْنُهُ رَآيَا وَابْنُهُ بَعْلٌ
\par 6 وَابْنُهُ بَئِيرَةُ الَّذِي سَبَاهُ تَغْلَثُ فَلاَسَرَ مَلِكُ أَشُّورَ. هُوَ رَئِيسُ الرَّأُوبَيْنِيِّينَ.
\par 7 وَإِخْوَتُهُ حَسَبَ عَشَائِرِهِمْ فِي الاِنْتِسَابِ حَسَبَ مَوَالِيدِهِمِ: الرَّئِيسُ يَعِيئِيلُ وَزَكَرِيَّا
\par 8 وَبَالِعُ بْنُ عَزَازَ بْنِ شَامِعَ بْنِ يُوئِيلَ الَّذِي سَكَنَ فِي عَرُوعِيرَ حَتَّى إِلَى نَبُوَ وَبَعْلِ مَعُونَ.
\par 9 وَسَكَنَ شَرْقاً إِلَى مَدْخَلِ الْبَرِّيَّةِ مِنْ نَهْرِ الْفُرَاتِ لأَنَّ مَاشِيَتَهُمْ كَثُرَتْ فِي أَرْضِ جِلْعَادَ.
\par 10 وَفِي أَيَّامِ شَاوُلَ عَمِلُوا حَرْباً مَعَ الْهَاجَرِيِّينَ فَسَقَطُوا بِأَيْدِيهِمْ وَسَكَنُوا فِي خِيَامِهِمْ فِي جَمِيعِ جِهَاتِ شَرْقِ جِلْعَادَ.
\par 11 وَبَنُو جَادَ سَكَنُوا مُقَابَِلَهُمْ فِي أَرْضِ بَاشَانَ حَتَّى إِلَى سَلْخَةَ.
\par 12 يُوئِيلُ الرَّأْسُ وَشَافَاطُ ثَانِيهِ وَيَعْنَايُ وَشَافَاطُ فِي بَاشَانَ.
\par 13 وَإِخْوَتُهُمْ حَسَبَ بُيُوتِ آبَائِهِمْ مِيخَائِيلُ وَمَشُلاَّمُ وَشَبَعُ وَيُورَايُ وَيَعْكَانُ وَزِيعُ وَعَابِرُ. سَبْعَةٌ.
\par 14 هَؤُلاَءِ بَنُو أَبِيجَايِلَ بْنِ حُورِيَ بْنِ يَارُوحَ بْنِ جِلْعَادَ بْنِ مِيخَائِيلَ بْنِ يَشِيشَايَ بْنِ يَحْدُوَ بْنِ بُوزٍ
\par 15 وَأَخِي بْنُ عَبْدِئِيلَ بْنِ جُونِي رَئِيسُ بَيْتِ آبَائِهِمْ.
\par 16 وَسَكَنُوا فِي جِلْعَادَ فِي بَاشَانَ وَقُرَاهَا وَفِي جَمِيعِ مَسَارِحِ شَارُونَ عِنْدَ مَخَارِجِهَا.
\par 17 جَمِيعُهُمُ انْتَسَبُوا فِي أَيَّامِ يُوثَامَ مَلِكِ يَهُوذَا, وَفِي أَيَّامِ يَرُبْعَامَ مَلِكِ إِسْرَائِيلَ.
\par 18 بَنُو رَأُوبَيْنَ وَالْجَادِيُّونَ وَنِصْفُ سِبْطِ مَنَسَّى مِنْ بَنِي الْبَأْسِ, رِجَالٌ يَحْمِلُونَ التُّرْسَ وَالسَّيْفَ وَيَشُدُّونَ الْقَوْسَ وَمُتَعَلِّمُونَ الْقِتَالَ, أَرْبَعَةٌ وَأَرْبَعُونَ أَلْفاً وَسَبْعُ مِئَةٍ وَسِتُّونَ مِنَ الْخَارِجِينَ فِي الْجَيْشِ.
\par 19 وَعَمِلُوا حَرْباً مَعَ الْهَاجَرِيِّينَ وَيَطُورَ وَنَافِيشَ وَنُودَابَ,
\par 20 فَانْتَصَرُوا عَلَيْهِمْ. فَدُفِعَ لِيَدِهِمِ الْهَاجَرِيُّونَ وَكُلُّ مَنْ مَعَهُمْ لأَنَّهُمْ صَرَخُوا إِلَى اللَّهِ فِي الْقِتَالِ, فَاسْتَجَابَ لَهُمْ لأَنَّهُمُ اتَّكَلُوا عَلَيْهِ.
\par 21 وَنَهَبُوا مَاشِيَتَهُمْ: جِمَالَهُمْ خَمْسِينَ أَلْفاً, وَغَنَماً مِئَتَيْنِ وَخَمْسِينَ أَلْفاً, وَحَمِيراً أَلْفَيْنِ. وَسَبُوا أُنَاساً مِئَةَ أَلْفٍ.
\par 22 لأَنَّهُ سَقَطَ قَتْلَى كَثِيرُونَ, لأَنَّ الْقِتَالَ إِنَّمَا كَانَ مِنَ اللَّهِ. وَسَكَنُوا مَكَانَهُمْ إِلَى السَّبْيِ.
\par 23 وَبَنُو نِصْفِ سِبْطِ مَنَسَّى سَكَنُوا فِي الأَرْضِ وَامْتَدُّوا مِنْ بَاشَانَ إِلَى بَعْلِ حَرْمُونَ وَسَنِيرَ وَجَبَلِ حَرْمُونَ.
\par 24 وَهَؤُلاَءِ رُؤُوسُ بُيُوتِ آبَائِهِمْ: عَافَرُ وَيَشْعِي وَأَلِيئِيلُ وَعَزْرِيئِيلُ وَيَرْمِيَا وَهُودَوْيَا وَيَحْدِيئِيلُ رِجَالٌ جَبَابِرَةُ بَأْسٍ وَذَوُو اسْمٍ وَرُؤُوسٌ لِبُيُوتِ آبَائِهِمْ.
\par 25 وَخَانُوا إِلَهَ آبَائِهِمْ وَزَنُوا وَرَاءَ آلِهَةِ شُعُوبِ الأَرْضِ الَّذِينَ طَرَدَهُمُ الرَّبُّ مِنْ أَمَامِهِمْ.
\par 26 فَنَبَّهَ إِلَهُ إِسْرَائِيلَ رُوحَ فُولَ مَلِكِ أَشُّورَ وَرُوحَ تَغْلَث فَلاَسَرَ مَلِكِ أَشُّورَ, فَسَبَاهُمُ الرَّأُوبَيْنِيِّينَ وَالْجَادِيِّينَ وَنِصْفَ سِبْطِ مَنَسَّى وَأَتَى بِهِمْ إِلَى حَلَحَ وَخَابُورَ وَهَارَا وَنَهْرِ جُوزَانَ إِلَى هَذَا الْيَوْمِ.

\chapter{6}

\par 1 بَنُو لاَوِي: جَرْشُونُ وَقَهَاتُ وَمَرَارِي.
\par 2 وَبَنُو قَهَاتَ: عَمْرَامُ وَيِصْهَارُ وَحَبْرُونُ وَعُزِّيئِيلُ.
\par 3 وَبَنُو عَمْرَامَ: هَارُونُ وَمُوسَى وَمَرْيَمُ. وَبَنُو هَارُونَ: نَادَابُ وَأَبِيهُو وَأَلِيعَازَارُ وَإِيثَامَارُ.
\par 4 أَلِعَازَارُ وَلَدَ فِينَحَاسَ, وَفِينَحَاسُ وَلَدَ أَبِيشُوعَ,
\par 5 وَأَبِيشُوعُ وَلَدَ بُقِّيَ, وَبُقِّي وَلَدَ عُزِّيَ,
\par 6 وَعُزِّي وَلَدَ زَرَحْيَا وَزَرَحْيَا, وَلَدَ مَرَايُوثَ,
\par 7 وَمَرَايُوثُ وَلَدَ أَمَرْيَا, وَأَمَرْيَا وَلَدَ أَخِيطُوبَ,
\par 8 وَأَخِيطُوبُ وَلَدَ صَادُوقَ, وَصَادُوقُ وَلَدَ أَخِيمَعَصَ,
\par 9 وَأَخِيمَعَصُ وَلَدَ عَزَرْيَا, وَعَزَرْيَا وَلَدَ يُوحَانَانَ,
\par 10 وَيُوحَانَانُ وَلَدَ عَزَرْيَا (وَهُوَ الَّذِي كَهَنَ فِي الْبَيْتِ الَّذِي بَنَاهُ سُلَيْمَانُ فِي أُورُشَلِيمَ)
\par 11 وَعَزَرْيَا وَلَدَ أَمَرْيَا, وَأَمَرْيَا وَلَدَ أَخِيطُوبَ,
\par 12 وَأَخِيطُوبُ وَلَدَ صَادُوقَ وَصَادُوقُ, وَلَدَ شَلُّومَ,
\par 13 وَشَلُّومُ وَلَدَ حِلْقِيَّا, وَحِلْقِيَّا وَلَدَ عَزَرْيَا,
\par 14 وَعَزَرْيَا وَلَدَ سَرَايَا, وَسَرَايَا وَلَدَ يَهُوصَادَاقَ,
\par 15 وَيَهُوصَادَاقُ سَارَ فِي سَبْيِ الرَّبِّ يَهُوذَا وَأُورُشَلِيمَ بِيَدِ نَبُوخَذْنَصَّرَ.
\par 16 بَنُو لاَوِي: جَرْشُومُ وَقَهَاتُ وَمَرَارِي.
\par 17 وَهَذَانِ اسْمَا ابْنَيْ جَرْشُومَ: لِبْنِي وَشَمْعِي.
\par 18 وَبَنُو قَهَاتَ: عَمْرَامُ وَيِصْهَارُ وَحَبْرُونُ وَعُزِّيئِيلُ.
\par 19 وَابْنَا مَرَارِي: مَحْلِي وَمُوشِي. فَهَذِهِ عَشَائِرُ اللاَّوِيِّينَ حَسَبَ آبَائِهِمْ.
\par 20 لِجَرْشُومَ: لِبْنِي ابْنُهُ, وَيَحَثُ ابْنُهُ, وَزِمَّةُ ابْنُهُ,
\par 21 وَيُوآخُ ابْنُهُ, وَعِدُّو ابْنُهُ, وَزَارَحُ ابْنُهُ, وَيَأَثْرَايُ ابْنُهُ.
\par 22 بَنُو قَهَاتَ: عَمِّينَادَابُ ابْنُهُ, وَقُورَحُ ابْنُهُ, وَأَسِّيرُ ابْنُهُ,
\par 23 وَأَلْقَانَةُ ابْنُهُ, وَأَبِيَاسَافُ ابْنُهُ, وَأَسِّيرُ ابْنُهُ,
\par 24 وَتَحَثُ ابْنُهُ, وَأُورِيئِيلُ ابْنُهُ, وَعُزِّيَّا ابْنُهُ, وَشَاوُلُ ابْنُهُ.
\par 25 وَابْنَا أَلْقَانَةَ: عَمَاسَايُ وَأَخِيمُوتُ
\par 26 وَأَلْقَانَةُ. بَنُو أَلْقَانَةَ صُوفَايُ ابْنُهُ, وَنَحَثُ ابْنُهُ,
\par 27 وَأَلِيآبُ ابْنُهُ, وَيَرُوحَامُ ابْنُهُ, وَأَلْقَانَةُ ابْنُهُ.
\par 28 وَابْنَا صَمُوئِيلَ: الْبِكْرُ وَشْنِي ثُمَّ أَبِيَّا.
\par 29 بَنُو مَرَارِي: مَحْلِي وَلِبْنِي ابْنُهُ وَشَمْعِي ابْنُهُ وَعُزَّةُ ابْنُهُ
\par 30 وَشِمْعَى ابْنُهُ وَحَجِيَّا ابْنُهُ وَعَسَايَا ابْنُهُ.
\par 31 وَهَؤُلاَءِ هُمُ الَّذِينَ أَقَامَهُمْ دَاوُدُ عَلَى الْغِنَاءِ فِي بَيْتِ الرَّبِّ بَعْدَمَا اسْتَقَرَّ التَّابُوتُ.
\par 32 وَكَانُوا يَخْدِمُونَ أَمَامَ مَسْكَنِ خَيْمَةِ الاِجْتِمَاعِ بِالْغِنَاءِ إِلَى أَنْ بَنَى سُلَيْمَانُ بَيْتَ الرَّبِّ فِي أُورُشَلِيمَ, فَقَامُوا عَلَى خِدْمَتِهِمْ حَسَبَ تَرْتِيبِهِمْ.
\par 33 وَهَؤُلاَءِ هُمُ الْقَائِمُونَ مَعَ بَنِيهِمْ. مِنْ بَنِي الْقَهَاتِيِّينَ: هَيْمَانُ الْمُغَنِّي ابْنُ يُوئِيلَ بْنِ صَمُوئِيلَ
\par 34 بْنِ أَلْقَانَةَ بْنِ يَرُوحَامَ بْنِ إِيلِيئِيلَ بْنِ تُوحَ
\par 35 بْنِ صُوفَ بْنِ أَلْقَانَةَ بْنِ مَحَثَ بْنِ عَمَاسَايَ
\par 36 بْنِ أَلْقَانَةَ بْنِ يُوئِيلَ بْنِ عَزَرْيَا بْنِ صَفَنْيَا
\par 37 بْنِ تَحَثَ بْنِ أَسِّيرَ بْنِ أَبِيَاسَافَ بْنِ قُورَحَ
\par 38 بْنِ يِصْهَارَ بْنِ قَهَاتَ بْنِ لاَوِي بْنِ إِسْرَائِيلَ.
\par 39 وَأَخُوهُ آسَافُ الْوَاقِفُ عَنْ يَمِينِهِ. آسَافُ بْنُ بَرَخْيَا بْنِ شَمْعِي
\par 40 بْنِ مِيخَائِيلَ بْنِ بَعَسِيَا بْنِ مَلْكِيَا
\par 41 بْنِ أَثْنَايَ بْنِ زَارَحَ بْنِ عَدَايَا
\par 42 بْنِ أَيْثَانَ بْنِ زِمَّةَ بْنِ شَمْعِي
\par 43 بْنِ يَحَثَ بْنِ جَرْشُومَ بْنِ لاَوِي.
\par 44 وَبَنُو مَرَارِي إِخْوَتُهُمْ. عَنِ الْيَسَارِ أَيْثَانُ بْنُ قِيشِي بْنِ عَبْدِي بْنِ مَلُّوخَ
\par 45 بْنِ حَشَبْيَا بْنِ أَمَصْيَا بْنِ حِلْقِيَّا
\par 46 بْنِ أَمْصِي بْنِ بَانِي بْنِ شَامَِرَ
\par 47 بْنِ مَحْلِي بْنِ مُوشِي بْنِ مَرَارِي بْنِ لاَوِي.
\par 48 وَإِخْوَتُهُمُ اللاَّوِيُّونَ مُقَامُونَ لِكُلِّ خِدْمَةِ مَسْكَنِ بَيْتِ اللَّهِ.
\par 49 وَأَمَّا هَارُونُ وَبَنُوهُ فَكَانُوا يُوقِدُونَ عَلَى مَذْبَحِ الْمُحْرَقَةِ وَعَلَى مَذْبَحِ الْبَخُورِ مَعَ كُلِّ عَمَلِ قُدْسِ الأَقْدَاسِ, وَلِلتَّكْفِيرِ عَنْ إِسْرَائِيلَ حَسَبَ كُلِّ مَا أَمَرَ بِهِ مُوسَى عَبْدُ اللَّهِ.
\par 50 وَهَؤُلاَءِ بَنُو هَارُونَ: أَلِعَازَارُ ابْنُهُ وَفِينَحَاسُ ابْنُهُ وَأَبِيشُوعُ ابْنُهُ
\par 51 وَبُقِّي ابْنُهُ وَعُزِّي ابْنُهُ وَزَرَحْيَا ابْنُهُ
\par 52 وَمَرَايُوثُ ابْنُهُ وَأَمَرْيَا ابْنُهُ وَأَخِيطُوبُ ابْنُهُ
\par 53 وَصَادُوقُ ابْنُهُ وَأَخِيمَعَصُ ابْنُهُ.
\par 54 وَهَذِهِ مَسَاكِنُهُمْ مَعَ ضِيَاعِهِمْ وَتُخُومِهِمْ: لِبَنِي هَارُونَ, لِعَشِيرَةِ الْقَهَاتِيِّينَ لأَنَّهُ لَهُمْ كَانَتِ الْقُرْعَةُ.
\par 55 وَأَعْطُوهُمْ حَبْرُونَ فِي أَرْضِ يَهُوذَا وَمَرَاعِيهَا حَوَالَيْهَا.
\par 56 وَأَمَّا حَقْلُ الْمَدِينَةِ وَدِيَارُهَا فَأَعْطُوهَا لِكَالَبَ بْنِ يَفُنَّةَ.
\par 57 وَأَعْطُوْا لِبَنِي هَارُونَ مُدُنَ الْمَلْجَإِ حَبْرُونَ وَلِبْنَةَ وَمَرَاعِيهَا وَيَتِّيرَ وَأَشْتَمُوعَ وَمَرَاعِيهَا
\par 58 وَحِيلَيْنَ وَمَرَاعِيهَا وَدَبِيرَ وَمَرَاعِيهَا
\par 59 وَعَاشَانَ وَمَرَاعِيهَا وَبَيْتَشَمْسَ وَمَرَاعِيهَا.
\par 60 وَمِنْ سِبْطِ بِنْيَامِينَ جَبْعَ وَمَرَاعِيهَا وَعَلْمَثَ وَمَرَاعِيهَا وَعَنَاثُوثَ وَمَرَاعِيهَا. جَمِيعُ مُدُنِهِمْ ثَلاَثَ عَشَرَةَ مَدِينَةً حَسَبَ عَشَائِرِهِمْ.
\par 61 وَلِبَنِي قَهَاتَ الْبَاقِينَ مِنْ عَشِيرَةِ السِّبْطِ مِنْ نِصْفِ السِّبْطِ, نِصْفِ مَنَسَّى, بِالْقُرْعَةِ عَشَرُ مُدُنٍ.
\par 62 وَلِبَنِي جَرْشُومَ حَسَبَ عَشَائِرِهِمْ. مِنْ سِبْطِ يَسَّاكَرَ وَمِنْ سِبْطِ أَشِيرَ وَمِنْ سِبْطِ نَفْتَالِي وَمِنْ سِبْطِ مَنَسَّى فِي بَاشَانَ ثَلاَثَ عَشَرَةَ مَدِينَةً.
\par 63 لِبَنِي مَرَارِي حَسَبَ عَشَائِرِهِمْ مِنْ سِبْطِ رَأُوبَيْنَ وَمِنْ سِبْطِ جَادَ وَمِنْ سِبْطِ زَبُولُونَ بِالْقُرْعَةِ اثْنَتَا عَشَرَةَ مَدِينَةً.
\par 64 فَأَعْطَى بَنُو إِسْرَائِيلَ اللاَّوِيِّينَ الْمُدُنَ وَمَرَاعِيهَا.
\par 65 وَأَعْطَوْا بِالْقُرْعَةِ مِنْ سِبْطِ يَهُوذَا وَمِنْ سِبْطِ بَنِي شَمْعُونَ وَمِنْ سِبْطِ بَنِي بِنْيَامِينَ هَذِهِ الْمُدُنَ الَّتِي سَمُّوهَا بِأَسْمَاءٍ.
\par 66 وَبَعْضُ عَشَائِرِ بَنِي قَهَاتَ كَانَتْ مُدُنُ تُخُمِهِمْ مِنْ سِبْطِ أَفْرَايِمَ.
\par 67 وَأَعْطُوهُمْ مُدُنَ الْمَلْجَإِ: شَكِيمَ وَمَرَاعِيهَا فِي جَبَلِ أَفْرَايِمَ وَجَازَرَ وَمَرَاعِيهَا
\par 68 وَيَقْمَعَامَ وَمَرَاعِيهَا وَبَيْتَ حُورُونَ وَمَرَاعِيهَا
\par 69 وَأَيَّلُونَ وَمَرَاعِيهَا وَجَتَّ رِمُّونَ وَمَرَاعِيهَا.
\par 70 وَمِنْ نِصْفِ سِبْطِ مَنَسَّى: عَانِيرَ وَمَرَاعِيهَا وَبِلْعَامَ وَمَرَاعِيهَا لِعَشِيرَةِ بَنِي قَهَاتَ الْبَاقِينَ.
\par 71 لِبَنِي جَرْشُومَ مِنْ نِصْفِ سِبْطِ مَنَسَّى جُولاَنُ فِي بَاشَانَ وَمَرَاعِيهَا وَعَشْتَارُوتُ وَمَرَاعِيهَا,
\par 72 وَمِنْ سِبْطِ يَسَّاكَرَ: قَادِشُ وَمَرَاعِيهَا وَدَبَرَةُ وَمَرَاعِيهَا
\par 73 وَرَامُوتُ وَمَرَاعِيهَا وَعَانِيمُ وَمَرَاعِيهَا.
\par 74 وَمِنْ سِبْطِ أَشِيرَ: مَشْآلُ وَمَرَاعِيهَا وَعَبْدُونُ وَمَرَاعِيهَا
\par 75 وَحُقُوقُ وَمَرَاعِيهَا وَرَحُوبُ وَمَرَاعِيهَا.
\par 76 وَمِنْ سِبْطِ نَفْتَالِي: قَادِشُ فِي الْجَلِيلِ وَمَرَاعِيهَا وَحَمُّونُ وَمَرَاعِيهَا وَقَرْيَتَايِمُ وَمَرَاعِيهَا.
\par 77 لِبَنِي مَرَارِي الْبَاقِينَ مِنْ سِبْطِ زَبُولُونَ: رِمُّونُو وَمَرَاعِيهَا وَتَابُورُ وَمَرَاعِيهَا.
\par 78 وَفِي عَبْرِ أُرْدُنِّ أَرِيحَا شَرْقِيَّ الأُرْدُنِّ مِنْ سِبْطِ رَأُوبَيْنَ: بَاصَرُ فِي الْبَرِّيَّةِ وَمَرَاعِيهَا وَيَهْصَةُ وَمَرَاعِيهَا
\par 79 وَقَدِيمُوتُ وَمَرَاعِيهَا وَمَيْفَعَةُ وَمَرَاعِيهَا.
\par 80 وَمِنْ سِبْطِ جَادَ: رَامُوتُ فِي جِلْعَادَ وَمَرَاعِيهَا وَمَحَنَايِمُ وَمَرَاعِيهَا
\par 81 وَحَشْبُونُ وَمَرَاعِيهَا وَيَعْزِيرُ وَمَرَاعِيهَا.

\chapter{7}

\par 1 وَبَنُو يَسَّاكَرَ: تُولاَعُ وَفُوَّةُ وَيَاشُوبُ وَشِمْرُونُ أَرْبَعَةٌ.
\par 2 وَبَنُو تُولاَعَ: عُزِّي وَرَفَايَا وَيَرِيئِيلُ وَيَحَمَايُ وَيِبْسَامُ وَشَمُوئِيلُ رُؤُوسُ بَيْتِ أَبِيهِمْ تُولاَعَ جَبَابِرَةُ بَأْسٍ حَسَبَ مَوَالِيدِهِمْ. كَانَ عَدَدُهُمْ فِي أَيَّامِ دَاوُدَ اثْنَيْنِ وَعِشْرِينَ أَلْفاً وَسِتَّ مِئَةٍ.
\par 3 وَابْنُ عُزِّي يَزْرَحْيَا. وَبَنُو يَزْرَحْيَا مِيخَائِيلُ وَعُوبَدْيَا وَيُوئِيلُ وَيِشِّيَّا خَمْسَةٌ. كُلُّهُمْ رُؤُوسٌ.
\par 4 وَمَعَهُمْ حَسَبَ مَوَالِيدِهِمْ وَبُيُوتِ آبَائِهِمْ جُيُوشُ أَجْنَادِ الْحَرْبِ سِتَّةٌ وَثَلاَثُونَ أَلْفاً, لأَنَّهُمْ كَثَّرُوا النِّسَاءَ وَالْبَنِينَ.
\par 5 وَإِخْوَتُهُمْ حَسَبَ كُلِّ عَشَائِرِ يَسَّاكَرَ جَبَابِرَةُ بَأْسٍ, سَبْعَةٌ وَثَمَانُونَ أَلْفاً مُجْمَلُ انْتِسَابِهِمْ.
\par 6 لِبِنْيَامِينَ بَالِعُ وَبَاكَرُ وَيَدِيعَئِيلُ. ثَلاَثَةٌ.
\par 7 وَبَنُو بَالَعَ: أَصْبُونُ وَعُزِّي وَعَزِّيئِيلُ وَيَرِيمُوثُ وَعَيْرِي. خَمْسَةٌ. رُؤُوسُ بُيُوتِ آبَاءٍ جَبَابِرَةُ بَأْسٍ, وَقَدِ انْتَسَبُوا اثْنَيْنِ وَعِشْرِينَ أَلْفاً وَأَرْبَعَةً وَثَلاَثِينَ.
\par 8 وَبَنُو بَاكَرَ: زَمِيرَةُ وَيُوعَاشُ وَأَلِيعَزَرُ وَأَلْيُوعِينَايُ وَعُمْرِي وَيَرِيمُوثُ وَأَبِيَّا وَعَنَاثُوثُ وَعَلاَمَثُ. كُلُّ هَؤُلاَءِ بَنُو بَاكَرَ.
\par 9 وَانْتِسَابُهُمْ حَسَبَ مَوَالِيدِهِمْ رُؤُوسُ بُيُوتِ آبَائِهِمْ جَبَابِرَةُ بَأْسٍ عِشْرُونَ أَلْفاً وَمِئَتَانِ.
\par 10 وَابْنُ يَدِيعِئِيلُ بَلْهَانُ, وَبَنُو بَلْهَانَ: يَعِيشُ وَبِنْيَامِينُ وَأَهُودُ وَكَنْعَنَةُ وَزَيْتَانُ وَتَرْشِيشُ وَأَخِيشَاحَرُ.
\par 11 كُلُّ هَؤُلاَءِ بَنُو يَدِيعِئِيلَ حَسَبَ رُؤُوسِ الآبَاءِ جَبَابِرَةُ الْبَأْسِ سَبْعَةَ عَشَرَ أَلْفاً وَمِئَتَانِ مِنَ الْخَارِجِينَ فِي الْجَيْشِ لِلْحَرْبِ.
\par 12 وَشُفِّيمُ وَحُفِّيمُ ابْنَا عَيْرَ وَحُوشِيمُ بْنُ أَحِيرَ.
\par 13 بَنُو نَفْتَالِي: يَحْصِيئِيلُ وَجُونِي وَيَصَرُ وَشَلُّومُ, بَنُو بِلْهَةَ.
\par 14 بَنُو مَنَسَّى: إِشْرِيئِيلُ الَّذِي وَلَدَتْهُ سُرِّيَّتُهُ الأَرَامِيَّةُ. وَلَدَتْ مَاكِيرَ أَبَا جِلْعَادَ.
\par 15 وَمَاكِيرُ اتَّخَذَ امْرَأَةً أُخْتَ حُفِّيمَ وَشُفِّيمَ وَاسْمُهَا مَعْكَةُ. وَاسْمُ ابْنِهِ الثَّانِي صَلُفْحَادُ. وَكَانَ لِصَلُفْحَادَ بَنَاتٌ.
\par 16 (وَوَلَدَتْ مَعْكَةُ امْرَأَةُ مَاكِيرَ ابْناً وَدَعَتِ اسْمَهُ فَرَشَ, وَاسْمُ أَخِيهِ شَارَشُ, وَابْنَاهُ أُولاَمُ وَرَاقَمُ.
\par 17 وَابْنُ أُولاَمَ بَدَانُ). هَؤُلاَءِ بَنُو جِلْعَادَ بْنِ مَاكِيرَ بْنِ مَنَسَّى.
\par 18 وَأُخْتُهُ هَمُّولَكَةُ وَلَدَتْ إِيشْهُودَ: وَأَبِيعَزَرَ وَمَحْلَةَ.
\par 19 وَكَانَ بَنُو شَمِيدَاعَ: أَخِيَانَ وَشَكِيمَ وَلِقْحِي وَأَنِيعَامَ.
\par 20 وَبَنُو أَفْرَايِمَ: شُوتَالَحُ وَبَرَدُ ابْنُهُ وَتَحَثُ ابْنُهُ وَأَلِعَادَا ابْنُهُ وَتَحَثُ ابْنُهُ
\par 21 وَزَابَادُ ابْنُهُ وَشُوتَالَحُ ابْنُهُ وَعَزَرُ وَأَلِعَادُ, وَقَتَلَهُمْ رِجَالُ جَتَّ الْمَوْلُودُونَ فِي الأَرْضِ لأَنَّهُمْ نَزَلُوا لِيَسُوقُوا مَاشِيَتَهُمْ.
\par 22 وَنَاحَ أَفْرَايِمُ أَبُوهُمْ أَيَّاماً كَثِيرَةً وَأَتَى إِخْوَتُهُ لِيُعَزُّوهُ.
\par 23 وَدَخَلَ عَلَى امْرَأَتِهِ فَحَبِلَتْ وَوَلَدَتِ ابْناً, فَدَعَا اسْمَهُ بَرِيعَةَ, لأَنَّ بَلِيَّةً كَانَتْ فِي بَيْتِهِ.
\par 24 وَبِنْتُهُ شِيرَةُ. وَقَدْ بَنَتْ بَيْتَ حُورُونَ السُّفْلَى وَالْعُلْيَا وَأُزَّيْنَ شِيرَةَ.
\par 25 وَرَفَحُ ابْنُهُ وَرَشَفُ وَتَلَحُ ابْنُهُ وَتَاحَنُ ابْنُهُ
\par 26 وَلَعْدَانُ ابْنُهُ وَعَمِّيهُودُ ابْنُهُ وَأَلِيشَمَعُ ابْنُهُ
\par 27 وَنُونُ ابْنُهُ وَيَشُوعُ ابْنُهُ.
\par 28 وَأَمْلاَكُهُمْ وَمَسَاكِنُهُمْ: بَيْتُ إِيلَ وَقُرَاهَا وَشَرْقاً نَعَرَانُ وَغَرْباً جَازَرُ وَقُرَاهَا وَشَكِيمُ وَقُرَاهَا إِلَى غَزَّةَ وَقُرَاهَا.
\par 29 وَلِجِهَةِ بَنِي مَنَسَّى بَيْتُ شَانَ وَقُرَاهَا وَتَعْنَكُ وَقُرَاهَا وَمَجِدُّو وَقُرَاهَا وَدُورُ وَقُرَاهَا. فِي هَذِهِ سَكَنَ بَنُو يُوسُفَ بْنِ إِسْرَائِيلَ.
\par 30 بَنُو أَشِيرَ يَمْنَةُ وَيِشْوَةُ وَيِشْوِي وَبَرِيعَةُ وَسَارَحُ أُخْتُهُمْ.
\par 31 وَابْنَا بَرِيعَةَ حَابِرُ وَمَلْكِيئِيلُ. هُوَ أَبُو بِرْزَاوَثَ.
\par 32 وَحَابِرُ وَلَدَ يَفْلِيطَ وَشُومَيْرَ وَحُوثَامَ وَشُوعَا أُخْتَهُمْ.
\par 33 وَبَنُو يَفْلِيطَ فَاسَكُ وَبِمْهَالُ وَعَشْوَةُ. هَؤُلاَءِ بَنُو يَفْلِيطَ.
\par 34 وَبَنُو شَامِرَ: آخِي وَرُهْجَةُ وَيَحُبَّةُ وَأَرَامُ.
\par 35 وَبَنُو هِيلاَمَ أَخِيهِ صُوفَحُ وَيَمْنَاعُ وَشَالَشُ وَعَامَالُ.
\par 36 وَبَنُو صُوفَحَ سُوحُ وَحَرَنْفَرُ وَشُوعَالُ وَبِيرِي وَيَمْرَةُ
\par 37 وَبَاصِرُ وَهُودُ وَشَمَّا وَشِلْشَةُ وَيِثْرَانُ وَبَئِيرَا.
\par 38 وَبَنُو يَثَرَ يَفُنَّةُ وَفِسْفَةُ وَأَرَا.
\par 39 وَبَنُو عُلاَّ آرَحُ وَحَنِيئِيلُ وَرَصِيَا.
\par 40 كُلُّ هَؤُلاَءِ بَنُو أَشِيرَ رُؤُوسُ بُيُوتِ آبَاءٍ مُنْتَخَبُونَ جَبَابِرَةُ بَأْسٍ رُؤُوسُ الرُّؤَسَاءِ وَانْتِسَابُهُمْ فِي الْجَيْشِ فِي الْحَرْبِ, عَدَدُهُمْ مِنَ الرِّجَالِ سِتَّةٌ وَعِشْرُونَ أَلْفاً.

\chapter{8}

\par 1 وَبِنْيَامِينُ وَلَدَ: بَالَعَ بِكْرَهُ وَأَشْبِيلَ الثَّانِيَ وَأَخْرَخَ الثَّالِثَ
\par 2 وَنُوحَةَ الرَّابِعَ وَرَافَا الْخَامِسَ.
\par 3 وَكَانَ بَنُو بَالَعَ أَدَّارَ وَجَيْرَا وَأَبِيهُودَ
\par 4 وَأَبِيشُوعَ وَنُعْمَانَ وَأَخُوخَ
\par 5 وَحَيْرَا وَشَفُوفَانَ وَحُورَامَ.
\par 6 وَهَؤُلاَءِ بَنُو آحُودَ. هَؤُلاَءِ رُؤُوسُ آبَاءِ سُكَّانِ جَبْعَ وَنَقَلُوهُمْ إِلَى مَنَاحَةَ:
\par 7 أَيْ نُعْمَانُ وَأَخِيَا. وَجَيْرَا هُوَ نَقَلَهُمْ وَوَلَدَ عُزَّا وَأَخِيحُودَ.
\par 8 وَشَجْرَايِمُ وَلَدَ فِي بِلاَدِ مُوآبَ بَعْدَ إِطْلاَقِهِ امْرَأَتَيْهِ حُوشِيمَ وَبَعْرَا.
\par 9 وَوَلَدَ مِنْ خُودَشَ امْرَأَتِهِ يُوبَابَ وَظِبْيَا وَمَيْشَا وَمَلْكَامَ.
\par 10 وَيَعُوصَ وَشَبْيَا وَمِرْمَةَ. هَؤُلاَءِ بَنُو رُؤُوسِ آبَاءٍ.
\par 11 وَمِنْ حُوشِيمَ وَلَدَ أَبِيطُوبَ وَأَلْفَعَلَ.
\par 12 وَبَنُو أَلْفَعَلَ: عَابِرُ وَمِشْعَامُ وَشَامِرُ, وَهُوَ بَنَى أُونُوَ وَلُودَ وَقُرَاهَا.
\par 13 وَبَرِيعَةُ وَشَمَعُ. هُمَا رَأْسَا آبَاءٍ لِسُكَّانِ أَيَّلُونَ, وَهُمَا طَرَدَا سُكَّانَ جَتَّ.
\par 14 وَأَخِيُو وَشَاشَقُ وَيَرِيمُوتُ
\par 15 وَزَبَدْيَا وَعَرَادُ وَعَادَرُ
\par 16 وَمِيخَائِيلُ وَيِشْفَةُ وَيُوخَا أَبْنَاءُ بَرِيعَةَ.
\par 17 وَزَبَدْيَا وَمَشُلاَّمُ وَحَزْقِي وَحَابِرُ
\par 18 وَيِشْمَرَايُ وَيَزَلْيَاهُ وَيُوبَابُ أَبْنَاءُ أَلْفَعَلَ.
\par 19 وَيَاقِيمُ وَزِكْرِي وَزَبْدِي
\par 20 وَأَلِيعِينَايُ وَصِلَّتَايُ وَإِيلِيئِيلُ
\par 21 وَعَدَايَا وَبَرَايَا وَشِمْرَةُ أَبْنَاءُ شَمْعِي.
\par 22 وَيِشْفَانُ وَعَابِرُ وَإِيلِيئِيلُ
\par 23 وَعَبْدُونُ وَزِكْرِي وَحَانَانُ
\par 24 وَحَنَنْيَا وَعِيلاَمُ وَعَنَثُوثِيَا
\par 25 وَيَفَدْيَا وَفَنُوئِيلُ أَبْنَاءُ شَاشَقَ.
\par 26 وَشِمْشَرَايُ وَشَحَرْيَا وَعَثَلْيَا
\par 27 وَيَعْرَشْيَا وَإِيلِيَّا وَزِكْرِي أَبْنَاءُ يَرُوحَامَ.
\par 28 هَؤُلاَءِ رُؤُوسُ آبَاءٍ. حَسَبَ مَوَالِيدِهِمْ رُؤُوسٌ. هَؤُلاَءِ سَكَنُوا فِي أُورُشَلِيمَ.
\par 29 وَفِي جِبْعُونَ سَكَنَ أَبُو جِبْعُونَ, وَاسْمُ امْرَأَتِهِ مَعْكَةُ.
\par 30 وَابْنُهُ الْبِكْرُ عَبْدُونُ, ثُمَّ صُورُ وَقَيْسُ وَبَعَلُ وَنَادَابُ
\par 31 وَجَدُورُ وَأَخِيُو وَزَاكِرُ.
\par 32 وَمِقْلُوثُ وَلَدَ شَمَاةَ. وَهُمْ أَيْضاً مَعَ إِخْوَتِهِمْ سَكَنُوا فِي أُورُشَلِيمَ مُقَابِلَ إِخْوَتِهِمْ.
\par 33 وَنِيرُ وَلَدَ قَيْسَ, وَقَيْسُ وَلَدَ شَاوُلَ وَشَاوُلُ وَلَدَ يُونَاثَانَ وَمَلْكِيشُوعَ وَأَبِينَادَابَ وَإِشْبَعَلَ.
\par 34 وَابْنُ يُونَاثَانَ مَرِيبْبَعَلُ, وَمَرِيبْبَعَلُ وَلَدَ مِيخَا.
\par 35 وَبَنُو مِيخَا: فِيثُونُ وَمَالِكُ وَتَارِيعُ وَآحَازُ.
\par 36 وَآحَازُ وَلَدَ يَهُوعَدَّةَ وَيَهُوعَدَّةُ وَلَدَ عَلْمَثَ وَعَزْمُوتَ وَزِمْرِي. وَزِمْرِي وَلَدَ مُوصَا
\par 37 وَمُوصَا وَلَدَ بِنْعَةَ وَرَافَةَ ابْنَهُ وَأَلِعَاسَةَ ابْنَهُ وَآصِيلَ ابْنَهُ
\par 38 وَلِآصِيلَ سِتَّةُ بَنِينَ وَهَذِهِ أَسْمَاؤُهُمْ: عَزْرِيقَامُ وَبُكْرُو وَإِسْمَاعِيلُ وَشَعَرْيَا وَعُوبَدْيَا وَحَانَانُ. كُلُّ هَؤُلاَءِ بَنُو آصِيلَ.
\par 39 وَبَنُو عَاشِقَ أَخِيهِ أُولاَمُ بِكْرُهُ وَيَعُوشُ الثَّانِي وَأَلِيفَلَطُ الثَّالِثُ.
\par 40 وَكَانَ بَنُو أُولاَمَ رِجَالاً جَبَابِرَةَ بَأْسٍ يُغْرِقُونَ فِي الْقِسِيِّ كَثِيرِي الْبَنِينَ وَبَنِي الْبَنِينَ مِئَةً وَخَمْسِينَ. كُلُّ هَؤُلاَءِ مِنْ بَنِي بِنْيَامِينَ.

\chapter{9}

\par 1 وَانْتَسَبَ كُلُّ إِسْرَائِيلَ, وَهَا هُمْ مَكْتُوبُونَ فِي سِفْرِ مُلُوكِ إِسْرَائِيلَ. وَسُبِيَ يَهُوذَا إِلَى بَابِلَ لأَجْلِ خِيَانَتِهِمْ.
\par 2 وَالسُّكَّانُ الأَوَّلُونَ فِي مُلْكِهِمْ وَمُدُنِهِمْ هُمْ إِسْرَائِيلُ الْكَهَنَةُ وَاللاَّوِيُّونَ وَالنَّثِينِيمُ.
\par 3 وَسَكَنَ فِي أُورُشَلِيمَ مِنْ بَنِي يَهُوذَا وَبَنِي بِنْيَامِينَ وَبَنِي أَفْرَايِمَ وَمَنَسَّى:
\par 4 عُوتَايُ بْنُ عَمِّيهُودَ بْنِ عُمْرِي بْنِ إِمْرِي بْنِ بَانِي مِنْ بَنِي فَارَصَ بْنِ يَهُوذَا.
\par 5 وَمِنَ الشِّيلُونِيِّينَ: عَسَايَا الْبِكْرُ وَبَنُوهُ.
\par 6 وَمِنْ بَنِي زَارَحَ: يَعُوئِيلُ وَإِخْوَتُهُمْ سِتُّ مِئَةٍ وَتِسْعُونَ.
\par 7 وَمِنْ بَنِي بِنْيَامِينَ: سَلُّو بْنُ مَشُلاَّمَ بْنِ هُودُويَا بْنِ هَسْنُوأَةَ
\par 8 وَيِبْنِيَا بْنُ يَرُوحَامَ وَأَيْلَةُ بْنُ عُزِّي بْنِ مِكْرِي وَمَشُلاَّمُ بْنُ شَفَطْيَا بْنِ رَعُوئِيلَ بْنِ يِبْنِيَا.
\par 9 وَإِخْوَتُهُمْ حَسَبَ مَوَالِيدِهِمْ تِسْعُ مِئَةٍ وَسِتَّةٌ وَخَمْسُونَ. كُلُّ هَؤُلاَءِ الرِّجَالِ رُؤُوسُ آبَاءٍ لِبُيُوتِ آبَائِهِمْ.
\par 10 وَمِنَ الْكَهَنَةِ يَدْعِيَا وَيَهُويَارِيبُ وَيَاكِينُ
\par 11 وَعَزَرْيَا بْنُ حِلْقِيَّا بْنِ مَشُلاَّمَ بْنِ صَادُوقَ بْنِ مَرَايُوثَ بْنِ أَخِيطُوبَ رَئِيسِ بَيْتِ اللَّهِ,
\par 12 وَعَدَايَا بْنُ يَرُوحَامَ بْنِ فَشْحُورَ بْنِ مَلْكِيَّا وَمَعْسَايُ بْنُ عَدِيئِيلَ بْنِ يَحْزِيرَةَ بْنِ مَشُلاَّمَ بْنِ مَشِلِّيمِيتَ بْنِ إِمِّيرَ
\par 13 وَإِخْوَتُهُمْ رُؤُوسُ بُيُوتِ آبَائِهِمْ أَلْفٌ وَسَبْعُ مِئَةٍ وَسِتُّونَ جَبَابِرَةُ بَأْسٍ لِعَمَلِ خِدْمَةِ بَيْتِ اللَّهِ.
\par 14 وَمِنَ اللاَّوِيِّينَ شَمَعْيَا بْنُ حَشُّوبَ بْنِ عَزْرِيقَامَ بْنِ حَشَبْيَا مِنْ بَنِي مَرَارِي.
\par 15 وَبَقْبَقَّرُ وَحَرَشُ وَجَلاَلُ وَمَتَنْيَا بْنُ مِيخَا بْنِ زِكْرِي بْنِ آسَافَ,
\par 16 وَعُوبَدْيَا بْنُ شَمَعْيَا بْنِ جَلاَلَ بْنِ يَدُوثُونَ وَبَرَخْيَا بْنُ آسَا بْنِ أَلْقَانَةَ السَّاكِنُ فِي قُرَى النَّطُوفَاتِيِّينَ.
\par 17 وَالْبَوَّابُونَ: شَلُّومُ وَعَقُّوبُ وَطَلْمُونُ وَأَخِيمَانُ وَإِخْوَتُهُمْ. شَلُّومُ الرَّأْسُ.
\par 18 وَحَتَّى الآنَ هُمْ فِي بَابِ الْمَلِكِ إِلَى الشَّرْقِ. هُمُ الْبَوَّابُونَ لِفِرَقِ بَنِي لاَوِي.
\par 19 وَشَلُّومُ بْنُ قُورِي بْنِ أَبِيَاسَافَ بْنِ قُورَحَ وَإِخْوَتُهُ لِبُيُوتِ آبَائِهِ. الْقُورَحِيُّونَ عَلَى عَمَلِ الْخِدْمَةِ حُرَّاسُ أَبْوَابِ الْخَيْمَةِ وَآبَاؤُهُمْ عَلَى مَحَلَّةِ الرَّبِّ حُرَّاسُ الْمَدْخَلِ.
\par 20 وَفِينَحَاسُ بْنُ أَلِعَازَارَ كَانَ رَئِيساً عَلَيْهِمْ سَابِقاً وَالرَّبُّ مَعَهُ.
\par 21 وَزَكَرِيَّا بْنَ مَشَلَمْيَا كَانَ بَوَّابَ بَابِ خَيْمَةِ الاِجْتِمَاعِ.
\par 22 جَمِيعُ هَؤُلاَءِ الْمُنْتَخَبِينَ بَوَّابِينَ لِلأَبْوَابِ مِئَتَانِ وَاثْنَا عَشَرَ, وَقَدِ انْتَسَبُوا حَسَبَ قُرَاهُمْ. أَقَامَهُمْ دَاوُدُ وَصَمُوئِيلُ الرَّائِي عَلَى وَظَائِفِهِمْ.
\par 23 وَكَانُوا هُمْ وَبَنُوهُمْ عَلَى أَبْوَابِ بَيْتِ الرَّبِّ بَيْتِ الْخَيْمَةِ لِلْحِرَاسَةِ.
\par 24 فِي الْجِهَاتِ الأَرْبَعِ كَانَ الْبَوَّابُونَ فِي الشَّرْقِ وَالْغَرْبِ وَالشِّمَالِ وَالْجَنُوبِ.
\par 25 وَكَانَ إِخْوَتُهُمْ فِي قُرَاهُمْ لِلْمَجِيءِ مَعَهُمْ فِي السَّبْعَةِ الأَيَّامِ حِيناً بَعْدَ حِينٍ.
\par 26 لأَنَّهُ بِالْوَظِيفَةِ رُؤَسَاءُ الْبَوَّابِينَ هَؤُلاَءِ الأَرْبَعَةُ هُمْ لاَوِيُّونَ وَكَانُوا عَلَى الْمَخَادِعِ وَعَلَى خَزَائِنِ بَيْتِ اللَّهِ.
\par 27 وَنَزَلُوا حَوْلَ بَيْتِ اللَّهِ لأَنَّ عَلَيْهِمِ الْحِرَاسَةَ وَعَلَيْهِمِ الْفَتْحَ كُلَّ صَبَاحٍ.
\par 28 وَبَعْضُهُمْ عَلَى آنِيَةِ الْخِدْمَةِ, لأَنَّهُمْ كَانُوا يُدْخِلُونَهَا بِعَدَدٍ وَيُخْرِجُونَهَا بِعَدَدٍ.
\par 29 وَبَعْضُهُمُ اؤْتُمِنُوا عَلَى الآنِيَةِ وَعَلَى كُلِّ أَمْتِعَةِ الْقُدْسِ وَعَلَى الدَّقِيقِ وَالْخَمْرِ وَاللُّبَانِ وَالأَطْيَابِ.
\par 30 وَالْبَعْضُ مِنْ بَنِي الْكَهَنَةِ كَانُوا يُرَكِّبُونَ دَهُونَ الأَطْيَابِ.
\par 31 وَمَتَّثْيَا وَاحِدٌ مِنَ اللاَّوِيِّينَ, وَهُوَ بِكْرُ شَلُّومَ الْقُورَحِيِّ, بِالْوَظِيفَةِ عَلَى عَمَلِ الْمَطْبُوخَاتِ.
\par 32 وَالْبَعْضُ مِنْ بَنِي الْقَهَاتِيِّينَ مِنْ إِخْوَتِهِمْ عَلَى خُبْزِ الْوُجُوهِ لِيُهَيِّئُوهُ فِي كُلِّ سَبْتٍ.
\par 33 فَهَؤُلاَءِ هُمُ الْمُغَنُّونَ رُؤُوسُ آبَاءِ اللاَّوِيِّينَ فِي الْمَخَادِعِ, وَهُمْ مُعْفَوْنَ, لأَنَّهُ نَهَاراً وَلَيْلاً عَلَيْهِمِ الْعَمَلُ.
\par 34 هَؤُلاَءِ رُؤُوسُ آبَاءِ اللاَّوِيِّينَ. حَسَبَ مَوَالِيدِهِمْ رُؤُوسٌ. هَؤُلاَءِ سَكَنُوا فِي أُورُشَلِيمَ.
\par 35 وَفِي جِبْعُونَ سَكَنَ أَبُو جِبْعُونَ يَعُوئِيلُ, وَاسْمُ امْرَأَتِهِ مَعْكَةُ.
\par 36 وَابْنُهُ الْبِكْرُ عَبْدُونُ ثُمَّ صُورُ وَقَيْسُ وَبَعْلُ وَنَيْرُ وَنَادَابُ
\par 37 وَجَدُورُ وَأَخِيُو وَزَكَرِيَّا وَمِقْلُوثُ.
\par 38 وَمِقْلُوثُ وَلَدَ شَمْآمَ. وَهُمْ أَيْضاً سَكَنُوا مُقَابَِلَ إِخْوَتِهِمْ فِي أُورُشَلِيمَ مَعَ إِخْوَتِهِمْ.
\par 39 وَنَيْرُ وَلَدَ قَيْسَ وَقَيْسُ وَلَدَ شَاوُلَ وَشَاوُلُ وَلَدَ: يُونَاثَانَ وَمَلْكِيشُوعَ وَأَبِينَادَابَ وَإِشْبَعَلَ.
\par 40 وَابْنُ يُونَاثَانَ مَرِيبْبَعَلُ وَمَرِيبْبَعَلُ, وَلَدَ مِيخَا.
\par 41 وَبَنُو مِيخَا: فِيثُونُ وَمَالِكُ وَتَحْرِيعُ وَآحَازُ.
\par 42 وَآحَازُ وَلَدَ: يَعْرَةَ, وَيَعْرَةُ وَلَدَ عَلْمَثَ وَعَزْمُوتَ وَزِمْرِي. وَزِمْرِي وَلَدَ مُوصَا,
\par 43 وَمُوصَا وَلَدَ يِنْعَا, وَرَفَايَا ابْنَهُ وَأَلْعَسَةَ ابْنَهُ وَآصِيلَ ابْنَهُ.
\par 44 وَكَانَ لِآصِيلَ سِتَّةُ بَنِينَ وَهَذِهِ أَسْمَاؤُهُمْ: عَزْرِيقَامُ وَبُكْرُو ثُمَّ إِسْمَاعِيلُ وَشَعَرْيَا وَعُوبَدْيَا وَحَانَانُ. هَؤُلاَءِ بَنُو آصِيلَ.

\chapter{10}

\par 1 وَحَارَبَ الْفِلِسْطِينِيُّونَ إِسْرَائِيلَ, فَهَرَبَ رِجَالُ إِسْرَائِيلَ مِنْ أَمَامِ الْفِلِسْطِينِيِّينَ وَسَقَطُوا قَتْلَى فِي جَبَلِ جِلْبُوعَ.
\par 2 وَشَدَّ الْفِلِسْطِينِيُّونَ وَرَاءَ شَاوُلَ وَوَرَاءَ بَنِيهِ, وَضَرَبَ الْفِلِسْطِينِيُّونَ يُونَاثَانَ وَأَبِينَادَابَ وَمَلْكِيشُوعَ أَبْنَاءَ شَاوُلَ.
\par 3 وَاشْتَدَّتِ الْحَرْبُ عَلَى شَاوُلَ فَأَصَابَتْهُ رُمَاةُ الْقِسِيِّ, فَـ/نْجَرَحَ مِنَ الرُّمَاةِ.
\par 4 فَقَالَ شَاوُلُ لِحَامِلِ سِلاَحِهِ: «اسْتَلَّ سَيْفَكَ وَاطْعَنِّي بِهِ لِئَلاَّ يَأْتِيَ هَؤُلاَءِ الْغُلْفُ وَيُقَبِّحُونِي». فَلَمْ يَشَأْ حَامِلُ سِلاَحِهِ لأَنَّهُ خَافَ جِدّاً. فَأَخَذَ شَاوُلُ السَّيْفَ وَسَقَطَ عَلَيْهِ.
\par 5 فَلَمَّا رَأَى حَامِلُ سِلاَحِهِ أَنَّهُ قَدْ مَاتَ شَاوُلُ, سَقَطَ هُوَ أَيْضاً عَلَى السَّيْفِ وَمَاتَ.
\par 6 فَمَاتَ شَاوُلُ وَبَنُوهُ الثَّلاَثَةُ وَكُلُّ بَيْتِهِ, مَاتُوا مَعاً.
\par 7 وَلَمَّا رَأَى جَمِيعُ رِجَالِ إِسْرَائِيلَ الَّذِينَ فِي الْوَادِي أَنَّهُمْ قَدْ هَرَبُوا, وَأَنَّ شَاوُلَ وَبَنِيهِ قَدْ مَاتُوا, تَرَكُوا مُدُنَهُمْ وَهَرَبُوا, فَأَتَى الْفِلِسْطِينِيُّونَ وَسَكَنُوا بِهَا.
\par 8 وَفِي الْغَدِ لَمَّا جَاءَ الْفِلِسْطِينِيُّونَ لِيُعَرُّوا الْقَتْلَى وَجَدُوا شَاوُلَ وَبَنِيهِ سَاقِطِينَ فِي جَبَلِ جِلْبُوعَ,
\par 9 فَعَرُّوهُ وَأَخَذُوا رَأْسَهُ وَسِلاَحَهُ, وَأَرْسَلُوا إِلَى أَرْضِ الْفِلِسْطِينِيِّينَ فِي كُلِّ نَاحِيَةٍ لأَجْلِ تَبْشِيرِ أَصْنَامِهِمْ وَالشَّعْبِ.
\par 10 وَوَضَعُوا سِلاَحَهُ فِي بَيْتِ آلِهَتِهِم,ْ وَسَمَّرُوا رَأْسَهُ فِي بَيْتِ دَاجُونَ.
\par 11 وَلَمَّا سَمِعَ كُلُّ يَابِيشِ جِلْعَادَ بِكُلِّ مَا فَعَلَ الْفِلِسْطِينِيُّونَ بِشَاوُلَ,
\par 12 قَامَ كُلُّ ذِي بَأْسٍ وَأَخَذُوا جُثَّةَ شَاوُلَ وَجُثَثَ بَنِيهِ وَجَاءُوا بِهَا إِلَى يَابِيشَ, وَدَفَنُوا عِظَامَهُمْ تَحْتَ الْبُطْمَةِ فِي يَابِيشَ, وَصَامُوا سَبْعَةَ أَيَّامٍ.
\par 13 فَمَاتَ شَاوُلُ بِخِيَانَتِهِ الَّتِي بِهَا خَانَ الرَّبَّ مِنْ أَجْلِ كَلاَمِ الرَّبِّ الَّذِي لَمْ يَحْفَظْهُ. وَأَيْضاً لأَجْلِ طَلَبِهِ إِلَى الْجَانِّ لِلسُّؤَالِ
\par 14 وَلَمْ يَسْأَلْ مِنَ الرَّبِّ, فَأَمَاتَهُ وَحَوَّلَ الْمَمْلَكَةَ إِلَى دَاوُدَ بْنِ يَسَّى.

\chapter{11}

\par 1 وَاجْتَمَعَ كُلُّ رِجَالِ إِسْرَائِيلَ إِلَى دَاوُدَ فِي حَبْرُونَ قَائِلِينَ: «هُوَذَا عَظْمُكَ وَلَحْمُكَ نَحْنُ.
\par 2 وَمُنْذُ أَمْسِ وَمَا قَبْلَهُ حِينَ كَانَ شَاوُلُ مَلِكاً كُنْتَ أَنْتَ تُخْرِجُ وَتُدْخِلُ إِسْرَائِيلَ, وَقَدْ قَالَ لَكَ الرَّبُّ إِلَهُكَ: أَنْتَ تَرْعَى شَعْبِي إِسْرَائِيلَ وَأَنْتَ تَكُونُ رَئِيساً لِشَعْبِي إِسْرَائِيلَ».
\par 3 وَجَاءَ جَمِيعُ شُيُوخِ إِسْرَائِيلَ إِلَى الْمَلِكِ إِلَى حَبْرُونَ, فَقَطَعَ دَاوُدُ مَعَهُمْ عَهْداً فِي حَبْرُونَ أَمَامَ الرَّبِّ, وَمَسَحُوا دَاوُدَ مَلِكاً عَلَى إِسْرَائِيلَ حَسَبَ كَلاَمِ الرَّبِّ عَنْ يَدِ صَمُوئِيلَ.
\par 4 وَذَهَبَ دَاوُدُ وَكُلُّ إِسْرَائِيلَ إِلَى أُورُشَلِيمَ (أَيْ يَبُوسَ). وَهُنَاكَ الْيَبُوسِيُّونَ سُكَّانُ الأَرْضِ.
\par 5 وَقَالَ سُكَّانُ يَبُوسَ لِدَاوُدَ: «لاَ تَدْخُلْ إِلَى هُنَا». فَأَخَذَ دَاوُدُ حِصْنَ صِهْيَوْنَ (هِيَ مَدِينَةُ دَاوُدَ).
\par 6 وَقَالَ دَاوُدُ: «إِنَّ الَّذِي يَضْرِبُ الْيَبُوسِيِّينَ أَوَّلاً يَكُونُ رَأْساً وَقَائِداً». فَصَعِدَ أَوَّلاً يُوآبُ ابْنُ صَرُويَةَ, فَصَارَ رَأْساً.
\par 7 وَأَقَامَ دَاوُدُ فِي الْحِصْنِ, لِذَلِكَ دَعُوهُ «مَدِينَةَ دَاوُدَ».
\par 8 وَبَنَى الْمَدِينَةَ حَوَالَيْهَا مِنَ الْقَلْعَةِ إِلَى مَا حَوْلِهَا. وَيُوآبُ جَدَّدَ سَائِرَ الْمَدِينَةِ.
\par 9 وَكَانَ دَاوُدُ يَتَزَايَدُ مُتَعَظِّماً وَرَبُّ الْجُنُودِ مَعَهُ
\par 10 وَهَؤُلاَءِ رُؤَسَاءُ الأَبْطَالِ الَّذِينَ لِدَاوُدَ, الَّذِينَ تَشَدَّدُوا مَعَهُ فِي مُلْكِهِ مَعَ كُلِّ إِسْرَائِيلَ لِتَمْلِيكِهِ حَسَبَ كَلاَمِ الرَّبِّ مِنْ جِهَةِ إِسْرَائِيلَ.
\par 11 وَهَذَا هُوَ عَدَدُ الأَبْطَالِ الَّذِينَ لِدَاوُدَ: يَشُبْعَامُ بْنُ حَكْمُونِي رَئِيسُ الثَّوَالِثِ. هُوَ هَزَّ رُمْحَهُ عَلَى ثَلاَثِ مِئَةٍ قَتَلَهُمْ دُفْعَةً وَاحِدَةً.
\par 12 وَبَعْدَهُ أَلِعَازَارُ بْنُ دُودُو الأَخُوخِيُّ. هُوَ مِنَ الأَبْطَالِ الثَّلاَثَةِ.
\par 13 هُوَ كَانَ مَعَ دَاوُدَ فِي فَسَّ دَمِّيمَ وَقَدِ اجْتَمَعَ هُنَاكَ الْفِلِسْطِينِيُّونَ لِلْحَرْبِ. وَكَانَتْ قِطْعَةُ الْحَقْلِ مَمْلُوءَةً شَعِيراً, فَهَرَبَ الشَّعْبُ مِنْ أَمَامِ الْفِلِسْطِينِيِّينَ.
\par 14 وَوَقَفُوا فِي وَسَطِ الْقِطْعَةِ وَأَنْقَذُوهَا, وَضَرَبُوا الْفِلِسْطِينِيِّينَ. وَخَلَّصَ الرَّبُّ خَلاَصاً عَظِيماً.
\par 15 وَنَزَلَ ثَلاَثَةٌ مِنَ الثَّلاَثِينَ رَئِيساً إِلَى الصَّخْرِ إِلَى دَاوُدَ إِلَى مَغَارَةِ عَدُلاَّمَ وَجَيْشُ الْفِلِسْطِينِيِّينَ نَازِلٌ فِي وَادِي الرَّفَائِيِّينَ.
\par 16 وَكَانَ دَاوُدُ حِينَئِذٍ فِي الْحِصْنِ, وَحَفَظَةُ الْفِلِسْطِينِيِّينَ حِينَئِذٍ فِي بَيْتِ لَحْمٍ.
\par 17 فَتَأَوَّهَ دَاوُدُ وَقَالَ: «مَنْ يَسْقِينِي مَاءً مِنْ بِئْرِ بَيْتِ لَحْمٍ الَّتِي عِنْدَ الْبَابِ؟»
\par 18 فَشَقَّ الثَّلاَثَةُ مَحَلَّةَ الْفِلِسْطِينِيِّينَ وَاسْتَقُوا مَاءً مِنْ بِئْرِ بَيْتِ لَحْمٍ الَّتِي عِنْدَ الْبَابِ وَحَمَلُوهُ وَأَتُوا بِهِ إِلَى دَاوُدَ, فَلَمْ يَشَأْ دَاوُدُ أَنْ يَشْرَبَهُ بَلْ سَكَبَهُ لِلرَّبِّ
\par 19 وَقَالَ: «حَاشَا لِي مِنْ قِبَلِ إِلَهِي أَنْ أَفْعَلَ ذَلِكَ! أَأَشْرَبُ دَمَ هَؤُلاَءِ الرِّجَالِ بِأَنْفُسِهِمْ؟ لأَنَّهُمْ إِنَّمَا أَتُوا بِهِ بِأَنْفُسِهِمْ». وَلَمْ يَشَأْ أَنْ يَشْرَبَهُ. هَذَا مَا فَعَلَهُ الأَبْطَالُ الثَّلاَثَةُ.
\par 20 وَأَبْشَايُ أَخُو يُوآبَ كَانَ رَئِيسَ ثَلاَثَةٍ. وَهُوَ قَدْ هَزَّ رُمْحَهُ عَلَى ثَلاَثِ مِئَةٍ فَقَتَلَهُمْ, فَكَانَ لَهُ اسْمٌ بَيْنَ الثَّلاَثَةِ.
\par 21 مِنَ الثَّلاَثَةِ أُكْرِمَ عَلَى الاِثْنَيْنِ وَكَانَ لَهُمَا رَئِيساً, إِلاَّ أَنَّهُ لَمْ يَصِلْ إِلَى الثَّلاَثَةِ الأُوَلِ.
\par 22 بَنَايَا بْنُ يَهُويَادَاعَ ابْنِ ذِي بَأْسٍ كَثِيرِ الأَفْعَالِ مِنْ قَبْصِيئِيلَ. هُوَ الَّذِي ضَرَبَ أَسَدَيْ مُوآبَ, وَهُوَ الَّذِي نَزَلَ وَضَرَبَ أَسَداً فِي وَسَطِ جُبٍّ يَوْمَ الثَّلْجِ.
\par 23 وَهُوَ ضَرَبَ الرَّجُلَ الْمِصْرِيَّ الَّذِي قَامَتُهُ خَمْسُ أَذْرُعٍ, وَفِي يَدِ الْمِصْرِيِّ رُمْحٌ كَنَوْلِ النَّسَّاجِينَ. فَنَزَلَ إِلَيْهِ بِعَصاً وَخَطَفَ الرُّمْحَ مِنْ يَدِ الْمِصْرِيِّ وَقَتَلَهُ بِرُمْحِهِ.
\par 24 هَذَا مَا فَعَلَهُ بَنَايَا بْنُ يَهُويَادَاعَ, فَكَانَ لَهُ اسْمٌ بَيْنَ الثَّلاَثَةِ الأَبْطَالِ.
\par 25 هُوَذَا أُكْرِمَ عَلَى الثَّلاَثِينَ إِلاَّ أَنَّهُ لَمْ يَصِلْ إِلَى الثَّلاَثَةِ. فَجَعَلَهُ دَاوُدُ مِنْ أَصْحَابِ سِرِّهِ.
\par 26 وَأَبْطَالُ الْجَيْشِ هُمْ: عَسَائِيلُ أَخُو يُوآبَ, وَأَلْحَانَانُ بْنُ دُودُوَ مِنْ بَيْتِ لَحْمٍ,
\par 27 شَمُّوتُ الْهَرُورِيُّ, حَالَصُ الْفَلُونِيُّ,
\par 28 عِيرَا بْنُ عِقِّيشَ التَّقُوعِيُّ, أَبِيعَزَرُ الْعَنَاثُوثِيُّ,
\par 29 سِبْكَايُ الْحُوشَاتِيُّ, عِيلاَيُ الأَخُوخِيُّ,
\par 30 مَهْرَايُ النَّطُوفَاتِيُّ, خَالِدُ بْنُ بَعْنَةَ النَّطُوفَاتِيُّ,
\par 31 إِتَّايُ بْنُ رِيبَايَ مِنْ جِبْعَةِ بَنِي بِنْيَامِينَ, بَنَايَا الْفَرْعَتُونِيُّ,
\par 32 حُورَايُ مِنْ أَوْدِيَةِ جَاعَشَ, أَبِيئِيلُ الْعَرَبَاتِيُّ,
\par 33 عَزْمُوتُ الْبَحْرُومِيُّ, إِلْيَحْبَا الشَّعْلُبُونِيُّ.
\par 34 بَنُو هَاشِمَ الْجَزُونِيُّ, يُونَاثَانُ بْنُ شَاجَايَ الْهَرَارِيُِّ,
\par 35 أَخِيآمُ بْنُ سَاكَارَ الْهَرَارِيُِّ, أَلِيفَالُ بْنُ أُورَ,
\par 36 حَافَرُ الْمَكِيرَاتِيُّ, وَأَخِيَا الْفَلُونِيُّ,
\par 37 حَصْرُو الْكَرْمَلِيُّ, نَعْرَايُ بْنُ أَزْبَايَ,
\par 38 يُوئِيلُ أَخُو نَاثَانَ, مَبْحَارُ بْنُ هَجْرِي,
\par 39 صَالِقُ الْعَمُّونِيُّ, نَحْرَايُ الْبَئِيرُوتِيُّ, (حَامِلُ سِلاَحِ يُوآبَ ابْنِ صَرُويَةَ),
\par 40 عِيرَا الْيِثْرِيُّ, جَارِبُ الْيِثْرِيُّ,
\par 41 أُورِيَّا الْحِثِّيُّ, زَابَادُ بْنُ أَحْلاَيَ,
\par 42 عَدِينَا بْنُ شِيزَا الرَّأُوبَيْنِيُّ (رَأْسُ الرَّأُوبَيْنِيِّينَ) وَمَعَهُ ثَلاَثُونَ,
\par 43 حَانَانُ ابْنُ مَعْكَةَ, يُوشَافَاطُ الْمَثْنِيُّ,
\par 44 عُزِّيَّا الْعَشْتَرُوتِيُّ, شَامَاعُ وَيَعُوئِيلُ ابْنَا حُوثَامَ الْعَرُوعِيرِيِ,
\par 45 يَدِيعَئِيلُ بْنُ شِمْرِي وَيُوحَا أَخُوهُ التِّيصِيُّ,
\par 46 إِيلِيئِيلُ مِنْ مَحْوِيمَ, وَيَرِيبَايُ وَيُوشُويَا ابْنَا أَلْنَعَمَ وَيِثْمَةُ الْمُوآبِيُّ,
\par 47 إِيلِيئِيلُ وَعُوبِيدُ وَيَعِيسِيئِيلُ مِنْ مَصُوبَايَا.

\chapter{12}

\par 1 وَهَؤُلاَءِ هُمُ الَّذِينَ جَاءُوا إِلَى دَاوُدَ إِلَى صِقْلَغَ وَهُوَ بَعْدُ مَحْجُوزٌ عَنْ وَجْهِ شَاوُلَ بْنِ قَيْسَ, وَهُمْ مِنَ الأَبْطَالِ مُسَاعِدُونَ فِي الْحَرْبِ,
\par 2 نَازِعُونَ فِي الْقِسِيِّ, يَرْمُونَ الْحِجَارَةَ وَالسِّهَامَ مِنَ الْقِسِيِّ بِالْيَمِينِ وَالْيَسَارِ, مِنْ إِخْوَةِ شَاوُلَ مِنْ بِنْيَامِينَ.
\par 3 الرَّأْسُ أَخِيعَزَرُ ثُمَّ يُوآشُ ابْنَا شَمَاعَةَ الْجِبْعِيُّ, وَيَزُوئِيلُ وَفَالَطُ ابْنَا عَزْمُوتَ, وَبَرَاخَةُ وَيَاهُو الْعَنَاثُوثِيُّ,
\par 4 وَيَشْمَعْيَا الْجِبْعُونِيُّ الْبَطَلُ بَيْنَ الثَّلاَثِينَ وَعَلَى الثَّلاَثِينَ, وَيَرْمِيَا وَيَحْزِيئِيلُ وَيُوحَانَانُ وَيُوزَابَادُ الْجَدِيرِيُّ
\par 5 وَإِلْعُوزَايُ وَيَرِيمُوثُ وَبَعْلِيَا وَشَمَرْيَا وَشَفَطْيَا الْحَرُوفِيُّ
\par 6 وَأَلْقَانَةُ وَيَشِيَّا وَعَزْرِيئِيلُ وَيُوعَزَرُ وَيَشُبْعَامُ الْقُورَحِيُّونَ
\par 7 وَيُوعِيلَةُ وَزَبَدْيَا ابْنَا يَرُوحَامَ مِنْ جَدُورَ.
\par 8 وَمِنَ الْجَادِيِّينَ انْفَصَلَ إِلَى دَاوُدَ إِلَى الْحِصْنِ فِي الْبَرِّيَّةِ جَبَابِرَةُ الْبَأْسِ رِجَالُ جَيْشٍ لِلْحَرْبِ, صَافُّو أَتْرَاسٍ وَرِمَاحٍ, وَوُجُوهُهُمْ كَوُجُوهِ الأُسُودِ, وَهُمْ كَالظَّبْيِ عَلَى الْجِبَالِ فِي السُّرْعَةِ:
\par 9 عَازَرُ الرَّأْسُ وَعُوبَدْيَا الثَّانِي وَأَلِيآبُ الثَّالِثُ
\par 10 وَمِشْمِنَّةُ الرَّابِعُ وَيَرْمِيَا الْخَامِسُ
\par 11 وَعَتَّايُ السَّادِسُ وَإِيلِيئِيلُ السَّابِعُ
\par 12 وَيُوحَانَانُ الثَّامِنُ وَأَلْزَابَادُ التَّاسِعُ
\par 13 وَيَرْمِيَا الْعَاشِرُ وَمَخْبَنَّايُ الْحَادِي عَشَرَ.
\par 14 هَؤُلاَءِ مِنْ بَنِي جَادَ رُؤُوسُ الْجَيْشِ. صَغِيرُهُمْ لِمِئَةٍ وَالْكَبِيرُ لأَلْفٍ.
\par 15 هَؤُلاَءِ هُمُ الَّذِينَ عَبَرُوا الأُرْدُنَّ فِي الشَّهْرِ الأَوَّلِ وَهُوَ مُمْتَلِئٌ إِلَى جَمِيعِ شُطُوطِهِ وَهَزَمُوا كُلَّ أَهْلِ الأَوْدِيَةِ شَرْقاً وَغَرْباً.
\par 16 وَجَاءَ قَوْمٌ مِنْ بَنِي بِنْيَامِينَ وَيَهُوذَا إِلَى الْحِصْنِ إِلَى دَاوُدَ.
\par 17 فَخَرَجَ دَاوُدُ لاِسْتِقْبَالِهِمْ وَقَالَ لَهُمْ: «إِنْ كُنْتُمْ قَدْ جِئْتُمْ بِسَلاَمٍ إِلَيَّ لِتُسَاعِدُونِي, يَكُونُ لِي مَعَكُمْ قَلْبٌ وَاحِدٌ. وَإِنْ كَانَ لِتَدْفَعُونِي لِعَدُوِّي وَلاَ ظُلْمَ فِي يَدَيَّ, فَلْيَنْظُرْ إِلَهُ آبَائِنَا وَيُنْصِفْ».
\par 18 فَحَلَّ الرُّوحُ عَلَى عَمَاسَايَ رَأْسِ الثَّوَالِثِ فَقَالَ : «لَكَ نَحْنُ يَا دَاوُدُ, وَمَعَكَ نَحْنُ يَا ابْنَ يَسَّى. سَلاَمٌ سَلاَمٌ لَكَ, وَسَلاَمٌ لِمُسَاعِدِيكَ. لأَنَّ إِلَهَكَ مُعِينُكَ». فَقَبِلَهُمْ دَاوُدُ وَجَعَلَهُمْ رُؤُوسَ الْجُيُوشِ.
\par 19 وَسَقَطَ إِلَى دَاوُدَ بَعْضٌ مِنْ مَنَسَّى حِينَ جَاءَ مَعَ الْفِلِسْطِينِيِّينَ ضِدَّ شَاوُلَ لِلْقِتَالِ وَلَمْ يُسَاعِدُوهُمْ, لأَنَّ أَقْطَابَ الْفِلِسْطِينِيِّينَ أَرْسَلُوهُ بِمَشُورَةٍ قَائِلِينَ: «إِنَّمَا بِرُؤُوسِنَا يَسْقُطُ إِلَى سَيِّدِهِ شَاوُلَ».
\par 20 حِينَ انْطَلَقَ إِلَى صِقْلَغَ سَقَطَ إِلَيْهِ مِنْ مَنَسَّى عَدْنَاحُ وَيُوزَابَادُ وَيَدِيعَئِيلُ وَمِيخَائِيلُ وَيُوزَابَادُ وَأَلِيهُو وَصِلْتَايُ رُؤُوسُ أُلُوفِ مَنَسَّى.
\par 21 وَهُمْ سَاعَدُوا دَاوُدَ عَلَى الْغُزَاةِ لأَنَّهُمْ جَمِيعاً جَبَابِرَةُ بَأْسٍ, وَكَانُوا رُؤَسَاءَ فِي الْجَيْشِ.
\par 22 لأَنَّهُ وَقْتَئِذٍ أَتَى أُنَاسٌ إِلَى دَاوُدَ يَوْماً فَيَوْماً لِمُسَاعَدَتِهِ حَتَّى صَارُوا جَيْشاً عَظِيماً كَجَيْشِ اللَّهِ.
\par 23 وَهَذَا عَدَدُ رُؤُوسِ الْمُتَجَرِّدِينَ لِلْقِتَالِ الَّذِينَ جَاءُوا إِلَى دَاوُدَ إِلَى حَبْرُونَ لِيُحَوِّلُوا مَمْلَكَةَ شَاوُلَ إِلَيْهِ حَسَبَ قَوْلِ الرَّبِّ.
\par 24 بَنُو يَهُوذَا حَامِلُو الأَتْرَاسِ وَالرِّمَاحِ سِتَّةُ آلاَفٍ وَثَمَانِ مِئَةِ مُتَجَرِّدٍ لِلْقِتَالِ.
\par 25 مِنْ بَنِي شَمْعُونَ جَبَابِرَةُ بَأْسٍ فِي الْحَرْبِ سَبْعَةُ آلاَفٍ وَمِئَةٌ.
\par 26 مِنْ بَنِي لاَوِي أَرْبَعَةُ آلاَفٍ وَسِتُّ مِئَةٍ.
\par 27 وَيَهُويَادَاعُ رَئِيسُ الْهَارُونِيِّينَ وَمَعَهُ ثَلاَثَةُ آلاَفٍ وَسَبْعُ مِئَةٍ.
\par 28 وَصَادُوقُ غُلاَمٌ جَبَّارُ بَأْسٍ وَبَيْتُ أَبِيهِ اثْنَانِ وَعِشْرُونَ قَائِداً.
\par 29 وَمِنْ بَنِي بِنْيَامِينَ إِخْوَةُ شَاوُلَ ثَلاَثَةُ آلاَفٍ (وَإِلَى هُنَا كَانَ أَكْثَرُهُمْ يَحْرُسُونَ حِرَاسَةَ بَيْتِ شَاوُلَ).
\par 30 وَمِنْ بَنِي أَفْرَايِمَ عِشْرُونَ أَلْفاً وَثَمَانُ مِئَةٍ, جَبَابِرَةُ بَأْسٍ وَذَوُو اسْمٍ فِي بُيُوتِ آبَائِهِمْ.
\par 31 وَمِنْ نِصْفِ سِبْطِ مَنَسَّى ثَمَانِيَةُ عَشَرَ أَلْفاً قَدْ تَعَيَّنُوا بِأَسْمَائِهِمْ لِيَأْتُوا وَيُمَلِّكُوا دَاوُدَ.
\par 32 وَمِنْ بَنِي يَسَّاكَرَ الْخَبِيرِينَ بِالأَوْقَاتِ لِمَعْرِفَةِ مَا يَعْمَلُ إِسْرَائِيلُ, رُؤُوسُهُمْ مِئَتَانِ وَكُلُّ إِخْوَتِهِمْ تَحْتَ أَمْرِهِمْ.
\par 33 مِنْ زَبُولُونَ الْخَارِجُونَ لِلْقِتَالِ الْمُصْطَفُّونَ لِلْحَرْبِ بِجَمِيعِ أَدَوَاتِ الْحَرْبِ خَمْسُونَ أَلْفاً, وَلِلاِصْطِفَافِ مِنْ دُونِ خِلاَفٍ.
\par 34 وَمِنْ نَفْتَالِي أَلْفُ رَئِيسٍ وَمَعَهُمْ سَبْعَةٌ وَثَلاَثُونَ أَلْفاً بِالأَتْرَاسِ وَالرِّمَاحِ.
\par 35 وَمِنَ الدَّانِيِّينَ مُصْطَفُّونَ لِلْحَرْبِ ثَمَانِيَةٌ وَعِشْرُونَ أَلْفاً وَسِتُّ مِئَةٍ.
\par 36 وَمِنْ أَشِيرَ الْخَارِجُونَ لِلْجَيْشِ لأَجْلِ الاِصْطِفَافِ لِلْحَرْبِ أَرْبَعُونَ أَلْفاً.
\par 37 وَمِنْ عَبْرِ الأُرْدُنِّ مِنَ الرَّأُوبَيْنِيِّينَ وَالْجَادِيِّينَ وَنِصْفِ سِبْطِ مَنَسَّى بِجَمِيعِ أَدَوَاتِ جَيْشِ الْحَرْبِ مِئَةٌ وَعِشْرُونَ أَلْفاً.
\par 38 كُلُّ هَؤُلاَءِ رِجَالُ حَرْبٍ يَصْطَفُّونَ صُفُوفاً, أَتُوا بِقَلْبٍ تَامٍّ إِلَى حَبْرُونَ لِيُمَلِّكُوا دَاوُدَ عَلَى كُلِّ إِسْرَائِيلَ. وَكَذَلِكَ كُلُّ بَقِيَّةِ إِسْرَائِيلَ بِقَلْبٍ وَاحِدٍ لِتَمْلِيكِ دَاوُدَ.
\par 39 وَكَانُوا هُنَاكَ مَعَ دَاوُدَ ثَلاَثَةَ أَيَّامٍ يَأْكُلُونَ وَيَشْرَبُونَ لأَنَّ إِخْوَتَهُمْ أَعَدُّوا لَهُمْ.
\par 40 وَكَذَلِكَ الْقَرِيبُونَ مِنْهُمْ حَتَّى يَسَّاكَرَ وَزَبُولُونَ وَنَفْتَالِي كَانُوا يَأْتُونَ بِخُبْزٍ عَلَى الْحَمِيرِ وَالْجِمَالِ وَالْبِغَالِ وَالْبَقَرِ, وَبِطَعَامٍ مِنْ دَقِيقٍ وَتِينٍ وَزَبِيبٍ وَخَمْرٍ وَزَيْتٍ وَبَقَرٍ وَغَنَمٍ بِكَثْرَةٍ, لأَنَّهُ كَانَ فَرَحٌ فِي إِسْرَائِيلَ.

\chapter{13}

\par 1 وَشَاوَرَ دَاوُدُ قُوَّادَ الأُلُوفِ وَالْمِئَاتِ وَكُلَّ رَئِيسٍ
\par 2 وَقَالَ دَاوُدُ لِكُلِّ جَمَاعَةِ إِسْرَائِيلَ: «إِنْ حَسُنَ عِنْدَكُمْ وَكَانَ ذَلِكَ مِنَ الرَّبِّ إِلَهِنَا, فَلْنُرْسِلْ إِلَى كُلِّ جِهَةٍ إِلَى إِخْوَتِنَا الْبَاقِينَ فِي كُلِّ أَرَاضِي إِسْرَائِيلَ وَمَعَهُمُ الْكَهَنَةُ وَاللاَّوِيُّونَ فِي مُدُنِ مَرَاعِيهِمْ لِيَجْتَمِعُوا إِلَيْنَا,
\par 3 فَنُرْجِعَ تَابُوتَ إِلَهِنَا إِلَيْنَا لأَنَّنَا لَمْ نَسْأَلْ بِهِ فِي أَيَّامِ شَاوُلَ».
\par 4 فَقَالَ كُلُّ الْجَمَاعَةِ بِأَنْ يَفْعَلُوا ذَلِكَ, لأَنَّ الأَمْرَ حَسُنَ فِي أَعْيُنِ جَمِيعِ الشَّعْبِ.
\par 5 وَجَمَعَ دَاوُدُ كُلَّ إِسْرَائِيلَ مِنْ شِيحُورِ مِصْرَ إِلَى مَدْخَلِ حَمَاةَ لِيَأْتُوا بِتَابُوتِ اللَّهِ مِنْ قَرْيَةِ يَعَارِيمَ.
\par 6 وَصَعِدَ دَاوُدُ وَكُلُّ إِسْرَائِيلَ إِلَى بَعْلَةَ, إِلَى قَرْيَةِ يَعَارِيمَ الَّتِي لِيَهُوذَا, لِيُصْعِدُوا مِنْ هُنَاكَ تَابُوتَ اللَّهِ الرَّبِّ الْجَالِسِ عَلَى الْكَرُوبِيمَ الَّذِي دُعِيَ بِالاِسْمِ.
\par 7 وَأَرْكَبُوا تَابُوتَ اللَّهِ عَلَى عَجَلَةٍ جَدِيدَةٍ مِنْ بَيْتِ أَبِينَادَابَ, وَكَانَ عُزَّةُ وَأَخِيُو يَسُوقَانِ الْعَجَلَةَ,
\par 8 وَدَاوُدُ وَكُلُّ إِسْرَائِيلَ يَلْعَبُونَ أَمَامَ اللَّهِ بِكُلِّ عِزٍّ وَبِأَغَانِيَّ وَعِيدَانٍ وَرَبَابٍ وَدُفُوفٍ وَصُنُوجٍ وَأَبْوَاقٍ.
\par 9 وَلَمَّا انْتَهُوا إِلَى بَيْدَرِ كِيدُونَ مَدَّ عُزَّةَ يَدَهُ لِيُمْسِكَ التَّابُوتَ, لأَنَّ الثِّيرَانَ انْشَمَصَتْ.
\par 10 فَحَمِيَ غَضَبُ الرَّبِّ عَلَى عُزَّةَ وَضَرَبَهُ مِنْ أَجْلِ أَنَّهُ مَدَّ يَدَهُ إِلَى التَّابُوتِ, فَمَاتَ هُنَاكَ أَمَامَ اللَّهِ.
\par 11 فَاغْتَاظَ دَاوُدُ لأَنَّ الرَّبَّ اقْتَحَمَ عُزَّةَ اقْتِحَاماً, وَسَمَّى ذَلِكَ الْمَوْضِعَ «فَارِصَ عُزَّةَ» إِلَى هَذَا الْيَوْمِ.
\par 12 وَخَافَ دَاوُدُ اللَّهَ فِي ذَلِكَ الْيَوْمِ قَائِلاً: «كَيْفَ آتِي بِتَابُوتِ اللَّهِ إِلَيَّ؟»
\par 13 وَلَمْ يَنْقُلْ دَاوُدُ التَّابُوتَ إِلَيْهِ إِلَى مَدِينَةِ دَاوُدَ, بَلْ مَالَ بِهِ إِلَى بَيْتِ عُوبِيدَ أَدُومَ الْجَتِّيِّ.
\par 14 وَبَقِيَ تَابُوتُ اللَّهِ عِنْدَ بَيْتِ عُوبِيدَ أَدُومَ فِي بَيْتِهِ ثَلاَثَةَ أَشْهُرٍ. وَبَارَكَ الرَّبُّ بَيْتَ عُوبِيدَ أَدُومَ وَكُلَّ مَا لَهُ.

\chapter{14}

\par 1 وَأَرْسَلَ حِيرَامُ مَلِكُ صُورَ رُسُلاً إِلَى دَاوُدَ وَخَشَبَ أَرْزٍ وَبَنَّائِينَ وَنَجَّارِينَ لِيَبْنُوا لَهُ بَيْتاً.
\par 2 وَعَلِمَ دَاوُدُ أَنَّ الرَّبَّ قَدْ أَثْبَتَهُ مَلِكاً عَلَى إِسْرَائِيلَ, لأَنَّ مَمْلَكَتَهُ ارْتَفَعَتْ مُتَصَاعِدَةً مِنْ أَجْلِ شَعْبِهِ إِسْرَائِيلَ.
\par 3 وَأَخَذَ دَاوُدُ نِسَاءً أَيْضاً فِي أُورُشَلِيمَ وَوَلَدَ أَيْضاً دَاوُدُ بَنِينَ وَبَنَاتٍ.
\par 4 وَهَذِهِ أَسْمَاءُ الأَوْلاَدِ الَّذِينَ كَانُوا لَهُ فِي أُورُشَلِيمَ: شَمُّوعُ وَشُوبَابُ وَنَاثَانُ وَسُلَيْمَانُ
\par 5 وَيِبْحَارُ وَأَلِيشُوعُ وَأَلِفَالَطُ
\par 6 وَنُوجَهُ وَنَافَجُ وَيَافِيعُ
\par 7 وَأَلِيشَمَعُ وَبَعَلْيَادَاعُ وَأَلِيفَلَطُ.
\par 8 وَسَمِعَ الْفِلِسْطِينِيُّونَ أَنَّ دَاوُدَ قَدْ مُسِحَ مَلِكاً عَلَى كُلِّ إِسْرَائِيلَ, فَصَعِدَ كُلُّ الْفِلِسْطِينِيِّينَ لِيُفَتِّشُوا عَلَى دَاوُدَ. وَلَمَّا سَمِعَ دَاوُدُ خَرَجَ لاِسْتِقْبَالِهِمْ.
\par 9 فَجَاءَ الْفِلِسْطِينِيُّونَ وَانْتَشَرُوا فِي وَادِي الرَّفَائِيِّينَ.
\par 10 فَسَأَلَ دَاوُدُ مِنَ اللَّهِ: «أَأَصْعَدُ عَلَى الْفِلِسْطِينِيِّينَ فَتَدْفَعُهُمْ لِيَدِي؟» فَقَالَ لَهُ الرَّبُّ: «اصْعَدْ فَأَدْفَعَهُمْ لِيَدِكَ».
\par 11 فَصَعِدُوا إِلَى بَعْلِ فَرَاصِيمَ وَضَرَبَهُمْ دَاوُدُ هُنَاكَ. وَقَالَ دَاوُدُ: «قَدِ اقْتَحَمَ اللَّهُ أَعْدَائِي بِيَدِي كَاقْتِحَامِ الْمِيَاهِ». لِذَلِكَ دَعُوا اسْمَ ذَلِكَ الْمَوْضِعِ «بَعْلَ فَرَاصِيمَ».
\par 12 وَتَرَكُوا هُنَاكَ آلِهَتَهُمْ, فَأَمَرَ دَاوُدُ فَأُحْرِقَتْ بِالنَّارِ.
\par 13 ثُمَّ عَادَ الْفِلِسْطِينِيُّونَ أَيْضاً وَانْتَشَرُوا فِي الْوَادِي.
\par 14 فَسَأَلَ أَيْضاً دَاوُدُ مِنَ اللَّهِ, فَقَالَ لَهُ اللَّهُ: «لاَ تَصْعَدْ وَرَاءَهُمْ, تَحَوَّلْ عَنْهُمْ وَهَلُمَّ عَلَيْهِمْ مُقَابِلَ أَشْجَارِ الْبُكَا.
\par 15 وَعِنْدَمَا تَسْمَعُ صَوْتَ خَطَوَاتٍ فِي رُؤُوسِ أَشْجَارِ الْبُكَا فَاخْرُجْ حِينَئِذٍ لِلْحَرْبِ, لأَنَّ اللَّهَ يَخْرُجُ أَمَامَكَ لِضَرْبِ مَحَلَّةِ الْفِلِسْطِينِيِّينَ».
\par 16 فَفَعَلَ دَاوُدُ كَمَا أَمَرَهُ اللَّهُ, وَضَرَبُوا مَحَلَّةَ الْفِلِسْطِينِيِّينَ مِنْ جِبْعُونَ إِلَى جَازَرَ.
\par 17 وَخَرَجَ اسْمُ دَاوُدَ إِلَى جَمِيعِ الأَرَاضِي, وَجَعَلَ الرَّبُّ هَيْبَتَهُ عَلَى جَمِيعِ الأُمَمِ.

\chapter{15}

\par 1 وَعَمِلَ دَاوُدُ لِنَفْسِهِ بُيُوتاً فِي مَدِينَةِ دَاوُدَ, وَأَعَدَّ مَكَاناً لِتَابُوتِ اللَّهِ وَنَصَبَ لَهُ خَيْمَةً.
\par 2 حِينَئِذٍ قَالَ دَاوُدُ: «لَيْسَ لأَحَدٍ أَنْ يَحْمِلَ تَابُوتَ اللَّهِ إِلاَّ لِلاَّوِيِّينَ, لأَنَّ الرَّبَّ إِنَّمَا اخْتَارَهُمْ لِحَمْلِ تَابُوتِ اللَّهِ وَلِخِدْمَتِهِ إِلَى الأَبَدِ».
\par 3 وَجَمَعَ دَاوُدُ كُلَّ إِسْرَائِيلَ إِلَى أُورُشَلِيمَ لأَجْلِ إِصْعَادِ تَابُوتِ الرَّبِّ إِلَى مَكَانِهِ الَّذِي أَعَدَّهُ لَهُ.
\par 4 فَجَمَعَ دَاوُدُ بَنِي هَارُونَ وَاللاَّوِيِّينَ.
\par 5 مِنْ بَنِي قَهَاتَ أُورِيئِيلَ الرَّئِيسَ, وَإِخْوَتَهُ مِئَةً وَعِشْرِينَ.
\par 6 مِنْ بَنِي مَرَارِي عَسَايَا الرَّئِيسَ, وَإِخْوَتَهُ مِئَتَيْنِ وَعِشْرِينَ.
\par 7 مِنْ بَنِي جَرْشُومَ يُوئِيلَ الرَّئِيسَ, وَإِخْوَتَهُ مِئَةً وَثَلاَثِينَ.
\par 8 مِنْ بَنِي أَلِيصَافَانَ شَمَعْيَا الرَّئِيسَ, وَإِخْوَتَهُ مِئَتَيْنِ.
\par 9 مِنْ بَنِي حَبْرُونَ إِيلِيئِيلَ الرَّئِيسَ, وَإِخْوَتَهُ ثَمَانِينَ.
\par 10 مِنْ بَنِي عُزِّيئِيلَ عَمِّينَادَابَ الرَّئِيسَ, وَإِخْوَتَهُ مِئَةً وَاثْنَيْ عَشَرَ.
\par 11 وَدَعَا دَاوُدُ صَادُوقَ وَأَبِيَاثَارَ الْكَاهِنَيْنِ وَاللاَّوِيِّينَ أُورِيئِيلَ وَعَسَايَا وَيُوئِيلَ وَشَمَعْيَا وَإِيلِيئِيلَ وَعَمِّينَادَابَ
\par 12 وَقَالَ لَهُمْ: «أَنْتُمْ رُؤُوسُ آبَاءِ اللاَّوِيِّينَ, فَتَقَدَّسُوا أَنْتُمْ وَإِخْوَتُكُمْ وَأَصْعِدُوا تَابُوتَ الرَّبِّ إِلَهِ إِسْرَائِيلَ إِلَى حَيْثُ أَعْدَدْتُ لَهُ.
\par 13 لأَنَّهُ إِذْ لَمْ تَكُونُوا فِي الْمَرَّةِ الأُولَى, اقْتَحَمَنَا الرَّبُّ إِلَهُنَا, لأَنَّنَا لَمْ نَسْأَلْهُ حَسَبَ الْمَرْسُومِ».
\par 14 فَتَقَدَّسَ الْكَهَنَةُ وَاللاَّوِيُّونَ لِيُصْعِدُوا تَابُوتَ الرَّبِّ إِلَهِ إِسْرَائِيلَ.
\par 15 وَحَمَلَ بَنُو اللاَّوِيِّينَ تَابُوتَ اللَّهِ كَمَا أَمَرَ مُوسَى حَسَبَ كَلاَمِ الرَّبِّ بِالْعِصِيِّ عَلَى أَكْتَافِهِمْ.
\par 16 وَأَمَرَ دَاوُدُ رُؤَسَاءَ اللاَّوِيِّينَ أَنْ يُوقِفُوا إِخْوَتَهُمُ الْمُغَنِّينَ بِآلاَتِ غِنَاءٍ, بِعِيدَانٍ وَرَبَابٍ وَصُنُوجٍ, مُسَمِّعِينَ بِرَفْعِ الصَّوْتِ بِفَرَحٍ.
\par 17 فَأَوْقَفَ اللاَّوِيُّونَ هَيْمَانَ بْنَ يُوئِيلَ, وَمِنْ إِخْوَتِهِ آسَافَ بْنَ بَرَخْيَا, وَمِنْ بَنِي مَرَارِي إِخْوَتِهِمْ إِيثَانَ بْنَ قُوشِيَّا,
\par 18 وَمَعَهُمْ إِخْوَتَهُمْ الثَّوَانِيَ: زَكَرِيَّا وَبَيْنَ وَيَعْزِئِيلَ وَشَمِيرَامُوثَ وَيَحِيئِيلَ وَعُنِّيَ وَأَلِيآبَ وَبَنَايَا وَمَعْسِيَّا وَمَتَّثْيَا وَأَلِيفَلْيَا وَمَقَنْيَا وَعُوبِيدَ أَدُومَ وَيَعِيئِيلَ الْبَوَّابِينَ.
\par 19 وَالْمُغَنُّونَ هَيْمَانُ وَآسَافُ وَإِيثَانُ بِصُنُوجِ نُحَاسٍ لِلتَّسْمِيعِ.
\par 20 وَزَكَرِيَّا وَعُزِّيئِيلُ وَشَمِيرَامُوثُ وَيَحِيئِيلُ وَعُنِّي وَأَلِيَابُو وَمَعْسِيَّا وَبَنَايَا بِالرَّبَابِ عَلَى الْجَوَابِ.
\par 21 وَمَتَّثْيَا وَأَلِيفَلْيَا وَمَقَنْيَا وَعُوبِيدُ أَدُومَ وَيَعِيئِيلُ وَعَزَزْيَا بِالْعِيدَانِ عَلَى الْقَرَارِ لِلإِمَامَةِ.
\par 22 وَكَنَنْيَا رَئِيسُ اللاَّوِيِّينَ عَلَى الْحَمْلِ مُرْشِداً فِي الْحَمْلِ لأَنَّهُ كَانَ خَبِيراً.
\par 23 وَبَرَخْيَا وَأَلْقَانَةُ بَوَّابَانِ لِلتَّابُوتِ.
\par 24 وَشَبَنْيَا وَيُوشَافَاطُ وَنَثْنَئِيلُ وَعَمَاسَايُ وَزَكَرِيَّا وَبَنَايَا وَأَلِيعَزَرُ الْكَهَنَةُ يَنْفُخُونَ بِالأَبْوَاقِ أَمَامَ تَابُوتِ اللَّهِ, وَعُوبِيدُ أَدُومَ وَيَحِيَّى بَوَّابَانِ لِلتَّابُوتِ.
\par 25 وَكَانَ دَاوُدُ وَشُيُوخُ إِسْرَائِيلَ وَرُؤَسَاءُ الأُلُوفِ هُمُ الَّذِينَ ذَهَبُوا لإِصْعَادِ تَابُوتِ عَهْدِ الرَّبِّ, مِنْ بَيْتِ عُوبِيدَ أَدُومَ بِفَرَحٍ.
\par 26 وَلَمَّا أَعْلَنَ اللَّهُ اللاَّوِيِّينَ حَامِلِي تَابُوتِ عَهْدِ الرَّبِّ ذَبَحُوا سَبْعَةَ عُجُولٍ وَسَبْعَةَ كِبَاشٍ.
\par 27 وَكَانَ دَاوُدُ لاَبِساً جُبَّةً مِنْ كَتَّانٍ, وَجَمِيعُ اللاَّوِيِّينَ حَامِلِينَ التَّابُوتَ, وَالْمُغَنُّونَ وَكَنَنْيَا رَئِيسُ الْحَمْلِ مَعَ الْمُغَنِّينَ. وَكَانَ عَلَى دَاوُدَ أَفُودٌ مِنْ كَتَّانٍ.
\par 28 فَكَانَ جَمِيعُ إِسْرَائِيلَ يُصْعِدُونَ تَابُوتَ عَهْدِ الرَّبِّ بِهُتَافٍ, وَبِصَوْتِ الأَصْوَارِ وَالأَبْوَاقِ وَالصُّنُوجِ يُصَوِّتُونَ بِالرَّبَابِ وَالْعِيدَانِ.
\par 29 وَلَمَّا دَخَلَ تَابُوتُ عَهْدِ الرَّبِّ مَدِينَةَ دَاوُدَ أَشْرَفَتْ مِيكَالُ بِنْتُ شَاوُلَ مِنَ الْكُوَّةِ فَرَأَتِ الْمَلِكَ دَاوُدَ يَرْقُصُ وَيَلْعَبُ, فَاحْتَقَرَتْهُ فِي قَلْبِهَا.

\chapter{16}

\par 1 وَأَدْخَلُوا تَابُوتَ اللَّهِ وَأَثْبَتُوهُ فِي وَسَطِ الْخَيْمَةِ الَّتِي نَصَبَهَا لَهُ دَاوُدُ, وَقَرَّبُوا مُحْرَقَاتٍ وَذَبَائِحَ سَلاَمَةٍ أَمَامَ اللَّهِ.
\par 2 وَلَمَّا انْتَهَى دَاوُدُ مِنْ إِصْعَادِ الْمُحْرَقَاتِ وَذَبَائِحِ السَّلاَمَةِ بَارَكَ الشَّعْبَ بِاسْمِ الرَّبِّ.
\par 3 وَقَسَمَ عَلَى كُلِّ آلِ إِسْرَائِيلَ مِنَ الرِّجَالِ وَالنِّسَاءِ, عَلَى كُلِّ إِنْسَانٍ, رَغِيفَ خُبْزٍ وَكَأْسَ خَمْرٍ وَقُرْصَ زَبِيبٍ.
\par 4 وَجَعَلَ أَمَامَ تَابُوتِ الرَّبِّ مِنَ اللاَّوِيِّينَ خُدَّاماً وَلأَجْلِ التَّذْكِيرِ وَالشُّكْرِ وَتَسْبِيحِ الرَّبِّ إِلَهِ إِسْرَائِيلَ:
\par 5 آسَافَ الرَّأْسَ وَزَكَرِيَّا ثَانِيَهُ وَيَعِيئِيلَ وَشَمِيرَامُوثَ وَيَحِيئِيلَ وَمَتَّثْيَا وَأَلِيآبَ وَبَنَايَا وَعُوبِيدَ أَدُومَ وَيَعِيئِيلَ بِآلاَتٍ رَبَابٍ وَعِيدَانٍ. وَكَانَ آسَافُ يُصَوِّتُ بِالصُّنُوجِ.
\par 6 وَبَنَايَا وَيَحْزِيئِيلُ الْكَاهِنَانِ بِالأَبْوَاقِ دَائِماً أَمَامَ تَابُوتِ عَهْدِ اللَّهِ.
\par 7 حِينَئِذٍ فِي ذَلِكَ الْيَوْمِ أَوَّلاً جَعَلَ دَاوُدُ يَحْمَدُ الرَّبَّ بِيَدِ آسَافَ وَإِخْوَتِهِ:
\par 8 «اِحْمَدُوا الرَّبَّ. ادْعُوا بِاسْمِهِ. أَخْبِرُوا فِي الشُّعُوبِ بِأَعْمَالِهِ.
\par 9 غَنُّوا لَهُ. تَرَنَّمُوا لَهُ. تَحَادَثُوا بِكُلِّ عَجَائِبِهِ.
\par 10 افْتَخِرُوا بِاسْمِ قُدْسِهِ. تَفْرَحُ قُلُوبُ الَّذِينَ يَلْتَمِسُونَ الرَّبَّ.
\par 11 اطْلُبُوا الرَّبَّ وَعِزَّهُ. الْتَمِسُوا وَجْهَهُ دَائِماً.
\par 12 اذْكُرُوا عَجَائِبَهُ الَّتِي صَنَعَ. آيَاتِهِ وَأَحْكَامَ فَمِهِ.
\par 13 يَا ذُرِّيَّةَ إِسْرَائِيلَ عَبْدِهِ وَبَنِي يَعْقُوبَ مُخْتَارِيهِ.
\par 14 هُوَ الرَّبُّ إِلَهُنَا. فِي كُلِّ الأَرْضِ أَحْكَامُهُ.
\par 15 اذْكُرُوا إِلَى الأَبَدِ عَهْدَهُ, الْكَلِمَةَ الَّتِي أَوْصَى بِهَا إِلَى أَلْفِ جِيلٍ.
\par 16 الَّذِي قَطَعَهُ مَعَ إِبْرَاهِيمَ. وَقَسَمَهُ لإِسْحَاقَ.
\par 17 وَقَدْ أَقَامَهُ لِيَعْقُوبَ فَرِيضَةً وَلإِسْرَائِيلَ عَهْداً أَبَدِيّاً.
\par 18 قَائِلاً: لَكَ أُعْطِي أَرْضَ كَنْعَانَ حَبْلَ مِيرَاثِكُمْ.
\par 19 حِينَ كُنْتُمْ عَدَداً قَلِيلاً, قَلِيلِينَ جِدّاً وَغُرَبَاءَ فِيهَا.
\par 20 وَذَهَبُوا مِنْ أُمَّةٍ إِلَى أُمَّةٍ وَمِنْ مَمْلَكَةٍ إِلَى شَعْبٍ آخَرَ.
\par 21 لَمْ يَدَعْ أَحَداً يَظْلِمُهُمْ بَلْ وَبَّخَ مِنْ أَجْلِهِمْ مُلُوكاً.
\par 22 لاَ تَمَسُّوا مُسَحَائِي وَلاَ تُؤْذُوا أَنْبِيَائِي.
\par 23 «غَنُّوا لِلرَّبِّ يَا كُلَّ الأَرْضِ. بَشِّرُوا مِنْ يَوْمٍ إِلَى يَوْمٍ بِخَلاَصِهِ.
\par 24 حَدِّثُوا فِي الأُمَمِ بِمَجْدِهِ وَفِي كُلِّ الشُّعُوبِ بِعَجَائِبِهِ.
\par 25 لأَنَّ الرَّبَّ عَظِيمٌ وَمُفْتَخَرٌ جِدّاً. وَهُوَ مَرْهُوبٌ فَوْقَ جَمِيعِ الآلِهَةِ.
\par 26 لأَنَّ كُلَّ آلِهَةِ الأُمَمِ أَصْنَامٌ, وَأَمَّا الرَّبُّ فَقَدْ صَنَعَ السَّمَاوَاتِ.
\par 27 الْجَلاَلُ وَالْبَهَاءُ أَمَامَهُ. الْعِزَّةُ وَالْبَهْجَةُ فِي مَكَانِهِ.
\par 28 هَبُوا الرَّبَّ يَا عَشَائِرَ الشُّعُوبِ هَبُوا الرَّبَّ مَجْداً وَعِزَّةً.
\par 29 هَبُوا الرَّبَّ مَجْدَ اسْمِهِ. احْمِلُوا هَدَايَا وَتَعَالُوا إِلَى أَمَامِهِ. اسْجُدُوا لِلرَّبِّ فِي زِينَةٍ مُقَدَّسَةٍ.
\par 30 ارْتَعِدُوا أَمَامَهُ يَا جَمِيعَ الأَرْضِ. تَثَبَّتَتِ الْمَسْكُونَةُ أَيْضاً. لاَ تَتَزَعْزَعُ.
\par 31 لِتَفْرَحِ السَّمَاوَاتُ وَتَبْتَهِجِ الأَرْضُ وَيَقُولُوا فِي الأُمَمِ الرَّبُّ قَدْ مَلَكَ.
\par 32 لِيَعِجَّ الْبَحْرُ وَمِلْؤُهُ, وَلْتَبْتَهِجِ الْبَرِّيَّةُ وَكُلُّ مَا فِيهَا.
\par 33 حِينَئِذٍ تَتَرَنَّمُ أَشْجَارُ الْوَعْرِ أَمَامَ الرَّبِّ لأَنَّهُ جَاءَ لِيَدِينَ الأَرْضَ.
\par 34 احْمَدُوا الرَّبَّ لأَنَّهُ صَالِحٌ لأَنَّ إِلَى الأَبَدِ رَحْمَتَهُ.
\par 35 وَقُولُوا: خَلِّصْنَا يَا إِلَهَ خَلاَصِنَا, وَاجْمَعْنَا وَأَنْقِذْنَا مِنَ الأُمَمِ لِنَحْمَدَ اسْمَ قُدْسِكَ, وَنَتَفَاخَرَ بِتَسْبِيحَتِكَ.
\par 36 مُبَارَكٌ الرَّبُّ إِلَهُ إِسْرَائِيلَ مِنَ الأَزَلِ وَإِلَى الأَبَدِ». فَقَالَ كُلُّ الشَّعْبِ: «آمِينَ» وَسَبَّحُوا الرَّبَّ.
\par 37 وَتَرَكَ هُنَاكَ أَمَامَ تَابُوتِ عَهْدِ الرَّبِّ آسَافَ وَإِخْوَتَهُ لِيَخْدِمُوا أَمَامَ التَّابُوتِ دَائِماً خِدْمَةَ كُلِّ يَوْمٍ بِيَوْمِهَا
\par 38 وَعُوبِيدَ أَدُومَ وَإِخْوَتَهُمْ ثمَانِيَةً وَسِتِّينَ, وَعُوبِيدَ أَدُومَ بْنَ يَدِيثُونَ وَحُوسَةَ بَوَّابِينَ.
\par 39 وَصَادُوقَ الْكَاهِنَ وَإِخْوَتَهُ الْكَهَنَةَ أَمَامَ مَسْكَنِ الرَّبِّ فِي الْمُرْتَفَعَةِ الَّتِي فِي جِبْعُونَ
\par 40 لِيُصْعِدُوا مُحْرَقَاتٍ لِلرَّبِّ عَلَى مَذْبَحِ الْمُحْرَقَةِ دَائِماً صَبَاحاً وَمَسَاءً, وَحَسَبَ كُلِّ مَا هُوَ مَكْتُوبٌ فِي شَرِيعَةِ الرَّبِّ الَّتِي أَمَرَ بِهَا إِسْرَائِيلَ,
\par 41 وَمَعَهُمْ هَيْمَانَ وَيَدُوثُونَ وَبَاقِيَ الْمُنْتَخَبِينَ الَّذِينَ ذُكِرَتْ أَسْمَاؤُهُمْ لِيَحْمَدُوا الرَّبَّ, لأَنَّ إِلَى الأَبَدِ رَحْمَتَهُ.
\par 42 وَمَعَهُمْ هَيْمَانُ وَيَدُوثُونُ بِأَبْوَاقٍ وَصُنُوجٍ لِلْمُصَوِّتِينَ وَآلاَتِ غِنَاءٍ لِلَّهِ, وَبَنُو يَدُوثُونَ بَوَّابُونَ.
\par 43 ثُمَّ انْطَلَقَ كُلُّ الشَّعْبِ كُلُّ وَاحِدٍ إِلَى بَيْتِهِ, وَرَجَعَ دَاوُدُ لِيُبَارِكَ بَيْتَهُ.

\chapter{17}

\par 1 وَكَانَ لَمَّا سَكَنَ دَاوُدُ فِي بَيْتِهِ قَالَ دَاوُدُ لِنَاثَانَ النَّبِيِّ: «هَئَنَذَا سَاكِنٌ فِي بَيْتٍ مِنْ أَرْزٍ, وَتَابُوتُ عَهْدِ الرَّبِّ تَحْتَ شُقَقٍ!»
\par 2 فَقَالَ نَاثَانُ لِدَاوُدَ: «افْعَلْ كُلَّ مَا فِي قَلْبِكَ لأَنَّ اللَّهَ مَعَكَ».
\par 3 وَفِي تِلْكَ اللَّيْلَةِ كَانَ كَلاَمُ اللَّهِ إِلَى نَاثَانَ:
\par 4 «اذْهَبْ وَقُلْ لِدَاوُدَ عَبْدِي: هَكَذَا قَالَ الرَّبُّ: أَنْتَ لاَ تَبْنِي لِي بَيْتاً لِلسُّكْنَى,
\par 5 لأَنِّي لَمْ أَسْكُنْ فِي بَيْتٍ مُنْذُ يَوْمَ أَصْعَدْتُ إِسْرَائِيلَ إِلَى هَذَا الْيَوْمِ, بَلْ سِرْتُ مِنْ خَيْمَةٍ إِلَى خَيْمَةٍ وَمِنْ مَسْكَنٍ إِلَى مَسْكَنٍ.
\par 6 فِي كُلِّ مَا سِرْتُ مَعَ جَمِيعِ إِسْرَائِيلَ, هَلْ تَكَلَّمْتُ بِكَلِمَةٍ مَعَ أَحَدِ قُضَاةِ إِسْرَائِيلَ الَّذِينَ أَمَرْتُهُمْ أَنْ يَرْعُوا شَعْبِي إِسْرَائِيلَ قَائِلاً: لِمَاذَا لَمْ تَبْنُوا لِي بَيْتاً مِنْ أَرْزٍ؟
\par 7 وَالآنَ فَهَكَذَا تَقُولُ لِعَبْدِي دَاوُدَ: هَكَذَا قَالَ رَبُّ الْجُنُودِ: أَنَا أَخَذْتُكَ مِنَ الْمَرْبَضِ مِنْ وَرَاءِ الْغَنَمِ لِتَكُونَ رَئِيساً عَلَى شَعْبِي إِسْرَائِيلَ,
\par 8 وَكُنْتُ مَعَكَ حَيْثُمَا تَوَجَّهْتَ, وَقَرَضْتُ جَمِيعَ أَعْدَائِكَ مِنْ أَمَامِكَ, وَعَمِلْتُ لَكَ اسْماً كَاسْمِ الْعُظَمَاءِ الَّذِينَ فِي الأَرْضِ.
\par 9 وَعَيَّنْتُ مَكَاناً لِشَعْبِي إِسْرَائِيلَ وَغَرَسْتُهُ فَسَكَنَ فِي مَكَانِهِ, وَلاَ يَضْطَرِبُ بَعْدُ وَلاَ يَعُودُ بَنُو الإِثْمِ يَبْلُونَهُ كَمَا فِي الأَوَّلِ
\par 10 وَمُنْذُ الأَيَّامِ الَّتِي فِيهَا أَقَمْتُ قُضَاةً عَلَى شَعْبِي إِسْرَائِيلَ, وَأَذْلَلْتُ جَمِيعَ أَعْدَائِكَ, وَأُخْبِرُكَ أَنَّ الرَّبَّ يَبْنِي لَكَ بَيْتاً
\par 11 وَيَكُونُ مَتَى كَمُلَتْ أَيَّامُكَ لِتَذْهَبَ مَعَ آبَائِكَ أَنِّي أُقِيمُ بَعْدَكَ نَسْلَكَ الَّذِي يَكُونُ مِنْ بَنِيكَ وَأُثَبِّتُ مَمْلَكَتَهُ.
\par 12 هُوَ يَبْنِي لِي بَيْتاً وَأَنَا أُثَبِّتُ كُرْسِيَّهُ إِلَى الأَبَدِ.
\par 13 أَنَا أَكُونُ لَهُ أَباً وَهُوَ يَكُونُ لِيَ ابْناً, وَلاَ أَنْزِعُ رَحْمَتِي عَنْهُ كَمَا نَزَعْتُهَا عَنِ الَّذِي كَانَ قَبْلَكَ.
\par 14 وَأُقِيمُهُ فِي بَيْتِي وَمَلَكُوتِي إِلَى الأَبَدِ, وَيَكُونُ كُرْسِيُّهُ ثَابِتاً إِلَى الأَبَدِ».
\par 15 فَحَسَبَ جَمِيعِ هَذَا الْكَلاَمِ وَحَسَبَ كُلِّ هَذِهِ الرُّؤْيَا كَذَلِكَ كَلَّمَ نَاثَانُ دَاوُدَ.
\par 16 فَدَخَلَ الْمَلِكُ دَاوُدُ وَجَلَسَ أَمَامَ الرَّبِّ وَقَالَ: «مَنْ أَنَا أَيُّهَا الرَّبُّ الإِلَهُ, وَمَاذَا بَيْتِي حَتَّى أَوْصَلْتَنِي إِلَى هُنَا؟
\par 17 وَقَلَّ هَذَا فِي عَيْنَيْكَ يَا اللَّهُ فَتَكَلَّمْتَ عَنْ بَيْتِ عَبْدِكَ إِلَى زَمَانٍ طَوِيلٍ, وَنَظَرْتَ إِلَيَّ مِنَ الْعَلاَءِ كَعَادَةِ الإِنْسَانِ أَيُّهَا الرَّبُّ الإِلَهُ.
\par 18 فَمَاذَا يَزِيدُ دَاوُدُ بَعْدُ لَكَ لأَجْلِ إِكْرَامِ عَبْدِكَ وَأَنْتَ قَدْ عَرَفْتَ عَبْدَكَ؟
\par 19 يَا رَبُّ مِنْ أَجْلِ عَبْدِكَ وَحَسَبَ قَلْبِكَ قَدْ فَعَلْتَ كُلَّ هَذِهِ الْعَظَائِمِ, لِتَظْهَرَ جَمِيعُ الْعَظَائِمِ
\par 20 يَا رَبُّ لَيْسَ مِثْلُكَ وَلاَ إِلَهَ غَيْرُكَ حَسَبَ كُلِّ مَا سَمِعْنَاهُ بِآذَانِنَا!
\par 21 وَأَيَّةُ أُمَّةٍ عَلَى الأَرْضِ مِثْلُ شَعْبِكَ إِسْرَائِيلَ الَّذِي سَارَ اللَّهُ لِيَفْتَدِيَهُ لِنَفْسِهِ شَعْباً, لِتَجْعَلَ لَكَ اسْمَ عَظَائِمَ وَمَخَاوِفَ بِطَرْدِكَ أُمَماً مِنْ أَمَامِ شَعْبِكَ الَّذِي افْتَدَيْتَهُ مِنْ مِصْرَ.
\par 22 وَقَدْ جَعَلْتَ شَعْبَكَ إِسْرَائِيلَ لِنَفْسِكَ شَعْباً إِلَى الأَبَدِ, وَأَنْتَ أَيُّهَا الرَّبُّ صِرْتَ لَهُمْ إِلَهاً.
\par 23 وَالآنَ أَيُّهَا الرَّبُّ, لِيَثْبُتْ إِلَى الأَبَدِ الْكَلاَمُ الَّذِي تَكَلَّمْتَ بِهِ عَنْ عَبْدِكَ وَعَنْ بَيْتِهِ وَافْعَلْ كَمَا نَطَقْتَ.
\par 24 وَلْيَثْبُتْ وَيَتَعَظَّمِ اسْمُكَ إِلَى الأَبَدِ, فَيُقَالَ: رَبُّ الْجُنُودِ إِلَهُ إِسْرَائِيلَ. هُوَ اللَّهُ لإِسْرَائِيلَ وَلْيَثْبُتْ بَيْتُ دَاوُدَ عَبْدِكَ أَمَامَكَ.
\par 25 لأَنَّكَ يَا إِلَهِي قَدْ أَعْلَنْتَ لِعَبْدِكَ أَنَّكَ تَبْنِي لَهُ بَيْتاً, لِذَلِكَ وَجَدَ عَبْدُكَ أَنْ يُصَلِّيَ أَمَامَكَ.
\par 26 وَالآنَ أَيُّهَا الرَّبُّ, أَنْتَ هُوَ اللَّهُ, وَقَدْ وَعَدْتَ عَبْدَكَ بِهَذَا الْخَيْرِ.
\par 27 وَالآنَ قَدِ ارْتَضَيْتَ بِأَنْ تُبَارِكَ بَيْتَ عَبْدِكَ لِيَكُونَ إِلَى الأَبَدِ أَمَامَكَ, لأَنَّكَ أَنْتَ يَا رَبُّ قَدْ بَارَكْتَ وَهُوَ مُبَارَكٌ إِلَى الأَبَدِ».

\chapter{18}

\par 1 وَبَعْدَ ذَلِكَ ضَرَبَ دَاوُدُ الْفِلِسْطِينِيِّينَ وَذَلَّلَهُمْ, وَأَخَذَ جَتَّ وَقُرَاهَا مِنْ يَدِ الْفِلِسْطِينِيِّينَ.
\par 2 وَضَرَبَ مُوآبَ, فَصَارَ الْمُوآبِيُّونَ عَبِيداً لِدَاوُدَ يُقَدِّمُونَ هَدَايَا.
\par 3 وَضَرَبَ دَاوُدُ هَدَدَ عَزَرَ مَلِكَ صُوبَةَ فِي حَمَاةَ حِينَ ذَهَبَ لِيُقِيمَ سُلْطَتَهُ عِنْدَ نَهْرِ الْفُرَاتِ,
\par 4 وَأَخَذَ دَاوُدُ مِنْهُ أَلْفَ مَرْكَبَةٍ وَسَبْعَةَ آلاَفِ فَارِسٍ وَعِشْرِينَ أَلْفَ رَاجِلٍ, وَعَرْقَبَ دَاوُدُ كُلَّ خَيْلِ الْمَرْكَبَاتِ وَأَبْقَى مِنْهَا مِئَةَ مَرْكَبَةٍ.
\par 5 فَجَاءَ أَرَامُ دِمَشْقَ لِنَجْدَةِ هَدَدَ عَزَرَ مَلِكِ صُوبَةَ, فَضَرَبَ دَاوُدُ مِنْ أَرَامَ اثْنَيْنِ وَعِشْرِينَ أَلْفَ رَجُلٍ.
\par 6 وَجَعَلَ دَاوُدُ مُحَافِظِينَ فِي أَرَامَ دِمَشْقَ, وَصَارَ الأَرَامِيُّونَ لِدَاوُدَ عَبِيداً يُقَدِّمُونَ هَدَايَا. وَكَانَ الرَّبُّ يُخَلِّصُ دَاوُدَ حَيْثُمَا تَوَجَّهَ.
\par 7 وَأَخَذَ دَاوُدُ أَتْرَاسَ الذَّهَبِ الَّتِي كَانَتْ عَلَى عَبِيدِ هَدَدَ عَزَرَ وَأَتَى بِهَا إِلَى أُورُشَلِيمَ.
\par 8 وَمِنْ طَبْحَةَ وَخُونَ مَدِينَتَيْ هَدَدَ عَزَرَ أَخَذَ دَاوُدُ نُحَاساً كَثِيراً جِدّاً صَنَعَ مِنْهُ سُلَيْمَانُ بَحْرَ النُّحَاسِ وَالأَعْمِدَةَ وَآنِيَةَ النُّحَاسِ.
\par 9 وَسَمِعَ تُوعُو مَلِكُ حَمَاةَ أَنَّ دَاوُدَ قَدْ ضَرَبَ كُلَّ جَيْشِ هَدَدَ عَزَرَ مَلِكِ صُوبَةَ,
\par 10 فَأَرْسَلَ هَدُورَامَ ابْنَهُ إِلَى الْمَلِكِ دَاوُدَ لِيَسْأَلَ عَنْ سَلاَمَتِهِ وَيُبَارِكَهُ, لأَنَّهُ حَارَبَ هَدَدَ عَزَرَ وَضَرَبَهُ. (لأَنَّ هَدَدَ عَزَرَ كَانَتْ لَهُ حُرُوبٌ مَعَ تُوعُوَ). وَبِيَدِهِ جَمِيعُ آنِيَةِ الذَّهَبِ وَالْفِضَّةِ وَالنُّحَاسِ.
\par 11 هَذِهِ أَيْضاً قَدَّسَهَا الْمَلِكُ دَاوُدُ لِلرَّبِّ مَعَ الْفِضَّةِ وَالذَّهَبِ الَّذِي أَخَذَهُ مِنْ كُلِّ الأُمَمِ مِنْ: أَدُومَ وَمِنْ مُوآبَ وَمِنْ بَنِي عَمُّونَ وَمِنَ الْفِلِسْطِينِيِّينَ وَمِنْ عَمَالِيقَ.
\par 12 وَأَبْشَايُ ابْنُ صَرُويَةَ ضَرَبَ مِنْ أَدُومَ فِي وَادِي الْمِلْحِ ثَمَانِيَةَ عَشَرَ أَلْفاً.
\par 13 وَجَعَلَ فِي أَدُومَ مُحَافِظِينَ, فَصَارَ جَمِيعُ الأَدُومِيِّينَ عَبِيداً لِدَاوُدَ. وَكَانَ الرَّبُّ يُخَلِّصُ دَاوُدَ حَيْثُمَا تَوَجَّهَ.
\par 14 وَمَلَكَ دَاوُدُ عَلَى جَمِيعِ إِسْرَائِيلَ, وَكَانَ يُجْرِي قَضَاءً وَعَدْلاً لِكُلِّ شَعْبِهِ.
\par 15 وَكَانَ يُوآبُ ابْنُ صَرُويَةَ عَلَى الْجَيْشِ, وَيَهُوشَافَاطُ بْنُ أَخِيلُودَ مُسَجِّلاً,
\par 16 وَصَادُوقُ بْنُ أَخِيطُوبَ وَأَبِيمَالِكُ بْنُ أَبِيَاثَارَ كَاهِنَيْنِ, وَشَوْشَا كَاتِباً
\par 17 وَبَنَايَا بْنُ يَهُويَادَاعَ عَلَى الْجَلاَّدِينَ وَالسُّعَاةِ, وَبَنُو دَاوُدَ الأَوَّلِينَ بَيْنَ يَدَيِ الْمَلِكِ.

\chapter{19}

\par 1 وَكَانَ بَعْدَ ذَلِكَ أَنَّ نَاحَاشَ مَلِكَ بَنِي عَمُّونَ مَاتَ, فَمَلَكَ ابْنُهُ عِوَضاً عَنْهُ.
\par 2 فَقَالَ دَاوُدُ: «أَصْنَعُ مَعْرُوفاً مَعَ حَانُونَ بْنِ نَاحَاشَ, لأَنَّ أَبَاهُ صَنَعَ مَعِي مَعْرُوفاً». فَأَرْسَلَ دَاوُدُ رُسُلاً لِيُعَزِّيَهُ بِأَبِيهِ. فَجَاءَ عَبِيدُ دَاوُدَ إِلَى أَرْضِ بَنِي عَمُّونَ إِلَى حَانُونَ لِيُعَزُّوهُ.
\par 3 فَقَالَ رُؤَسَاءُ بَنِي عَمُّونَ لِحَانُونَ: «هَلْ يُكْرِمُ دَاوُدُ أَبَاكَ فِي عَيْنَيْكَ حَتَّى أَرْسَلَ إِلَيْكَ مُعَزِّينَ؟ أَلَيْسَ لأَجْلِ الْفَحْصِ وَالْقَلْبِ وَتَجَسُّسِ الأَرْضِ جَاءَ عَبِيدُهُ إِلَيْكَ؟»
\par 4 فَأَخَذَ حَانُونُ عَبِيدَ دَاوُدَ وَحَلَقَ لِحَاهُمْ وَقَصَّ ثِيَابَهُمْ مِنَ الْوَسَطِ عِنْدَ السَّوْءَةِ ثُمَّ أَطْلَقَهُمْ.
\par 5 فَذَهَبَ أُنَاسٌ وَأَخْبَرُوا دَاوُدَ عَنِ الرِّجَالِ. فَأَرْسَلَ لِلِقَائِهِمْ لأَنَّ الرِّجَالَ كَانُوا خَجِلِينَ جِدّاً. وَقَالَ الْمَلِكُ: «أَقِيمُوا فِي أَرِيحَا حَتَّى تَنْبُتَ لِحَاكُمْ ثُمَّ ارْجِعُوا».
\par 6 وَلَمَّا رَأَى بَنُو عَمُّونَ أَنَّهُمْ قَدْ أَنْتَنُوا عِنْدَ دَاوُدَ, أَرْسَلَ حَانُونُ وَبَنُو عَمُّونَ أَلْفَ وَزْنَةٍ مِنَ الْفِضَّةِ لِيَسْتَأْجِرُوا لأَنْفُسِهِمْ مِنْ أَرَامِ النَّهْرَيْنِ وَمِنْ أَرَامِ مَعْكَةَ وَمِنْ صُوبَةَ مَرْكَبَاتٍ وَفُرْسَاناً.
\par 7 فَاسْتَأْجَرُوا لأَنْفُسِهِمِ اثْنَيْنِ وَثَلاَثِينَ أَلْفَ مَرْكَبَةٍ, وَمَلِكَ مَعْكَةَ وَشَعْبَهُ. فَجَاءُوا وَنَزَلُوا مُقَابَِلَ مَيْدَبَا. وَاجْتَمَعَ بَنُو عَمُّونَ مِنْ مُدُنِهِمْ وَأَتُوا لِلْحَرْبِ.
\par 8 وَلَمَّا سَمِعَ دَاوُدُ أَرْسَلَ يُوآبَ وَكُلَّ جَيْشِ الْجَبَابِرَةِ.
\par 9 فَخَرَجَ بَنُو عَمُّونَ وَاصْطَفُّوا لِلْحَرْبِ عِنْدَ بَابِ الْمَدِينَةِ, وَالْمُلُوكُ الَّذِينَ جَاءُوا كَانُوا وَحْدَهُمْ فِي الْحَقْلِ.
\par 10 وَلَمَّا رَأَى يُوآبُ أَنَّ مُقَدِّمَةَ الْحَرْبِ كَانَتْ نَحْوَهُ مِنْ قُدَّامٍ وَمِنْ وَرَاءٍ, اخْتَارَ مِنْ جَمِيعِ مُنْتَخَبِي إِسْرَائِيلَ وَصَفَّهُمْ لِلِقَاءِ أَرَامَ.
\par 11 وَسَلَّمَ بَقِيَّةَ الشَّعْبِ لِيَدِ أَبْشَايَ أَخِيهِ, فَاصْطَفُّوا لِلِقَاءِ بَنِي عَمُّونَ.
\par 12 وَقَالَ: «إِنْ قَوِيَ أَرَامُ عَلَيَّ تَكُونُ لِي نَجْدَةً, وَإِنْ قَوِيَ بَنُو عَمُّونَ عَلَيْكَ أَنْجَدْتُكَ.
\par 13 تَجَلَّدْ, وَلْنَتَشَدَّدْ لأَجْلِ شَعْبِنَا وَلأَجْلِ مُدُنِ إِلَهِنَا, وَمَا يَحْسُنُ فِي عَيْنَيِ الرَّبِّ يَفْعَلُ».
\par 14 وَتَقَدَّمَ يُوآبُ وَالشَّعْبُ الَّذِينَ مَعَهُ نَحْوَ أَرَامَ لِلْمُحَارَبَةِ, فَهَرَبُوا مِنْ أَمَامِهِ.
\par 15 وَلَمَّا رَأَى بَنُو عَمُّونَ أَنَّهُ قَدْ هَرَبَ أَرَامُ هَرَبُوا هُمْ أَيْضاً مِنْ أَمَامِ أَبْشَايَ أَخِيهِ وَدَخَلُوا إِلَى الْمَدِينَةِ. وَجَاءَ يُوآبُ إِلَى أُورُشَلِيمَ.
\par 16 وَلَمَّا رَأَى أَرَامُ أَنَّهُمْ قَدِ انْكَسَرُوا أَمَامَ إِسْرَائِيلَ أَرْسَلُوا رُسُلاً, وَأَبْرَزُوا أَرَامَ الَّذِينَ فِي عَبْرِ النَّهْرِ, وَأَمَامَهُمْ شُوبَكُ رَئِيسُ جَيْشِ هَدَدَ عَزَرَ.
\par 17 وَلَمَّا أُخْبِرَ دَاوُدُ جَمَعَ كُلَّ إِسْرَائِيلَ وَعَبَرَ الأُرْدُنَّ وَجَاءَ إِلَيْهِمْ وَاصْطَفَّ ضِدَّهُمْ. اصْطَفَّ دَاوُدُ لِلِقَاءِ أَرَامَ فِي الْحَرْبِ فَحَارَبُوهُ.
\par 18 وَهَرَبَ أَرَامُ مِنْ أَمَامِ إِسْرَائِيلَ, وَقَتَلَ دَاوُدُ مِنْ أَرَامَ سَبْعَةَ آلاَفِ مَرْكَبَةٍ وَأَرْبَعِينَ أَلْفَ رَاجِلٍ, وَقَتَلَ شُوبَكَ رَئِيسَ الْجَيْشِ.
\par 19 وَلَمَّا رَأَى عَبِيدُ هَدَدَ عَزَرَ أَنَّهُمْ قَدِ انْكَسَرُوا أَمَامَ إِسْرَائِيلَ صَالَحُوا دَاوُدَ وَخَدَمُوهُ. وَلَمْ يَشَأْ أَرَامُ أَنْ يُنْجِدُوا بَنِي عَمُّونَ بَعْدُ.

\chapter{20}

\par 1 وَكَانَ عِنْدَ تَمَامِ السَّنَةِ فِي وَقْتِ خُرُوجِ الْمُلُوكِ اقْتَادَ يُوآبُ قُوَّةَ الْجَيْشِ وَأَخْرَبَ أَرْضَ بَنِي عَمُّونَ وَأَتَى وَحَاصَرَ رَبَّةَ. وَكَانَ دَاوُدُ مُقِيماً فِي أُورُشَلِيمَ. فَضَرَبَ يُوآبُ رَبَّةَ وَهَدَمَهَا.
\par 2 وَأَخَذَ دَاوُدُ تَاجَ مَلِكِهِمْ عَنْ رَأْسِهِ, فَوُجِدَ وَزْنُهُ وَزْنَةً مِنَ الذَّهَبِ, وَفِيهِ حَجَرٌ كَرِيمٌ. فَكَانَ عَلَى رَأْسِ دَاوُدَ. وَأَخْرَجَ غَنِيمَةَ الْمَدِينَةِ وَكَانَتْ كَثِيرَةً جِدّاً.
\par 3 وَأَخْرَجَ الشَّعْبَ الَّذِينَ بِهَا وَنَشَرَهُمْ بِمَنَاشِيرَِ وَنَوَارِجِ حَدِيدٍ وَفُؤُوسٍ. وَهَكَذَا صَنَعَ دَاوُدُ لِكُلِّ مُدُنِ بَنِي عَمُّونَ. ثُمَّ رَجَعَ دَاوُدُ وَكُلُّ الشَّعْبِ إِلَى أُورُشَلِيمَ.
\par 4 ثُمَّ بَعْدَ ذَلِكَ قَامَتْ حَرْبٌ فِي جَازِرَ مَعَ الْفِلِسْطِينِيِّينَ. حِينَئِذٍ سَبْكَايُ الْحُوشِيُّ قَتَلَ سَفَّايَ مِنْ أَوْلاَدِ رَافَا فَذَلُّوا.
\par 5 وَكَانَتْ أَيْضاً حَرْبٌ مَعَ الْفِلِسْطِينِيِّينَ, فَقَتَلَ أَلْحَانَانُ بْنُ يَاعُورَ لَحْمِيَ أَخَا جُلْيَاتَ الْجَتِّيِّ. وَكَانَتْ قَنَاةُ رُمْحِهِ كَنَوْلِ النَّسَّاجِينَ.
\par 6 ثُمَّ كَانَتْ أَيْضاً حَرْبٌ فِي جَتَّ, وَكَانَ رَجُلٌ طَوِيلُ الْقَامَةِ أَعْنَشُ, أَصَابِعُهُ أَرْبَعٌ وَعِشْرُونَ, وَهُوَ أَيْضاً وُلِدَ لِرَافَا.
\par 7 وَلَمَّا عَيَّرَ إِسْرَائِيلَ ضَرَبَهُ يَهُونَاثَانُ بْنُ شَمْعِي أَخِي دَاوُدَ.
\par 8 هَؤُلاَءِ وُلِدُوا لِرَافَا فِي جَتَّ وَسَقَطُوا بِيَدِ دَاوُدَ وَبِيَدِ عَبِيدِهِ.

\chapter{21}

\par 1 وَوَقَفَ الشَّيْطَانُ ضِدَّ إِسْرَائِيلَ وَأَغْوَى دَاوُدَ لِيُحْصِيَ إِسْرَائِيلَ.
\par 2 فَقَالَ دَاوُدُ لِيُوآبَ وَلِرُؤَسَاءِ الشَّعْبِ: «اذْهَبُوا عُدُّوا إِسْرَائِيلَ مِنْ بِئْرَِ سَبْعٍَ إِلَى دَانَ, وَأْتُوا إِلَيَّ فَأَعْلَمَ عَدَدَهُمْ».
\par 3 فَقَالَ يُوآبُ: «لِيَزِدِ الرَّبُّ عَلَى شَعْبِهِ أَمْثَالَهُمْ مِئَةَ ضِعْفٍ. أَلَيْسُوا جَمِيعاً يَا سَيِّدِي الْمَلِكَ عَبِيداً لِسَيِّدِي؟ لِمَاذَا يَطْلُبُ هَذَا سَيِّدِي؟ لِمَاذَا يَكُونُ سَبَبَ إِثْمٍ لإِسْرَائِيلَ؟»
\par 4 فَاشْتَدَّ كَلاَمُ الْمَلِكِ عَلَى يُوآبَ. فَخَرَجَ يُوآبُ وَطَافَ فِي كُلِّ إِسْرَائِيلَ ثُمَّ جَاءَ إِلَى أُورُشَلِيمَ.
\par 5 فَدَفَعَ يُوآبُ جُمْلَةَ عَدَدِ الشَّعْبِ إِلَى دَاوُدَ, فَكَانَ كُلُّ إِسْرَائِيلَ مِلْيُوناً وَمِئَةَ أَلْفِ رَجُلٍ مُسْتَلِّي السَّيْفِ وَيَهُوذَا أَرْبَعَ مِئَةٍ وَسَبْعِينَ أَلْفَ رَجُلٍ مُسْتَلِّي السَّيْفِ,
\par 6 وَأَمَّا لاَوِي وَبِنْيَامِينُ فَلَمْ يَعُدَّهُمْ مَعَهُمْ لأَنَّ كَلاَمَ الْمَلِكِ كَانَ مَكْرُوهاً لَدَى يُوآبَ.
\par 7 وَقَبُحَ فِي عَيْنَيِ اللَّهِ هَذَا الأَمْرُ فَضَرَبَ إِسْرَائِيلَ.
\par 8 فَقَالَ دَاوُدُ لِلَّهِ: «لَقَدْ أَخْطَأْتُ جِدّاً حَيْثُ عَمِلْتُ هَذَا الأَمْرَ. وَالآنَ أَزِلْ إِثْمَ عَبْدِكَ لأَنِّي سَفِهْتُ جِدّاً».
\par 9 فَقالَ الرَّبُّ لِجَادَ رَائِي دَاوُدَ:
\par 10 «اذْهَبْ وَقُلْ لِدَاوُدَ: هَكَذَا قَالَ الرَّبُّ: ثَلاَثَةً أَنَا عَارِضٌ عَلَيْكَ فَاخْتَرْ لِنَفْسِكَ وَاحِداً مِنْهَا فَأَفْعَلَهُ بِكَ».
\par 11 فَجَاءَ جَادُ إِلَى دَاوُدَ وَقَالَ لَهُ: «هَكَذَا قَالَ الرَّبُّ: اقْبَلْ لِنَفْسِكَ
\par 12 إِمَّا ثَلاَثَ سِنِينَ جُوعٌ, أَوْ ثَلاَثَةَ أَشْهُرٍ هَلاَكٌ أَمَامَ مُضَايِقِيكَ وَسَيْفُ أَعْدَائِكَ يُدْرِكُكَ, أَوْ ثَلاَثَةَ أَيَّامٍ يَكُونُ فِيهَا سَيْفُ الرَّبِّ وَوَبَأٌ فِي الأَرْضِ, وَمَلاَكُ الرَّبِّ يَعْثُو فِي كُلِّ تُخُومِ إِسْرَائِيلَ. فَانْظُرِ الآنَ مَاذَا أَرُدُّ جَوَاباً لِمُرْسِلِي».
\par 13 فَقَالَ دَاوُدُ لِجَادٍ: «قَدْ ضَاقَ بِيَ الأَمْرُ جِدّاً. دَعْنِي أَسْقُطْ فِي يَدِ الرَّبِّ لأَنَّ مَرَاحِمَهُ كَثِيرَةٌ, وَلاَ أَسْقُطُ فِي يَدِ إِنْسَانٍ».
\par 14 فَجَعَلَ الرَّبُّ وَبَأً فِي إِسْرَائِيلَ, فَسَقَطَ مِنْ إِسْرَائِيلَ سَبْعُونَ أَلْفَ رَجُلٍ.
\par 15 وَأَرْسَلَ اللَّهُ مَلاَكاً عَلَى أُورُشَلِيمَ لإِهْلاَكِهَا, وَفِيمَا هُوَ يُهْلِكُ رَأَى الرَّبُّ فَنَدِمَ عَلَى الشَّرِّ, وَقَالَ لِلْمَلاَكِ الْمُهْلِكِ: «كَفَى الآنَ, رُدَّ يَدَكَ!» وَكَانَ مَلاَكُ الرَّبِّ وَاقِفاً عَُِنْدَ بَيْدَرِ أُرْنَانَ الْيَبُوسِيِّ.
\par 16 وَرَفَعَ دَاوُدُ عَيْنَيْهِ فَرَأَى مَلاَكَ الرَّبِّ وَاقِفاً بَيْنَ الأَرْضِ وَالسَّمَاءِ, وَسَيْفُهُ مَسْلُولٌ بِيَدِهِ وَمَمْدُودٌ عَلَى أُورُشَلِيمَ. فَسَقَطَ دَاوُدُ وَالشُّيُوخُ عَلَى وُجُوهِهِمْ مُكْتَسِينَ بِالْمُسُوحِ.
\par 17 وَقَالَ دَاوُدُ لِلَّهِ: «أَلَسْتُ أَنَا هُوَ الَّذِي أَمَرَ بِإِحْصَاءِ الشَّعْبِ؟ وَأَنَا هُوَ الَّذِي أَخْطَأَ وَأَسَاءَ, وَأَمَّا هَؤُلاَءِ الْخِرَافُ فَمَاذَا عَمِلُوا؟ فَأَيُّهَا الرَّبُّ إِلَهِي لِتَكُنْ يَدُكَ عَلَيَّ وَعَلَى بَيْتِ أَبِي لاَ عَلَى شَعْبِكَ لِضَرْبِهِمْ».
\par 18 فَكَلَّمَ مَلاَكُ الرَّبِّ جَادَ أَنْ يَقُولَ لِدَاوُدَ أَنْ يَصْعَدَ دَاوُدُ لِيُقِيمَ مَذْبَحاً لِلرَّبِّ فِي بَيْدَرِ أُرْنَانَ الْيَبُوسِيِّ.
\par 19 فَصَعِدَ دَاوُدُ حَسَبَ كَلاَمِ جَادَ الَّذِي تَكَلَّمَ بِهِ بِاسْمِ الرَّبِّ.
\par 20 فَالْتَفَتَ أُرْنَانُ فَرَأَى الْمَلاَكَ. وَبَنُوهُ الأَرْبَعَةُ مَعَهُ اخْتَبَأُوا, وَكَانَ أُرْنَانُ يَدْرُسُ حِنْطَةً.
\par 21 وَجَاءَ دَاوُدُ إِلَى أُرْنَانَ. وَتَطَلَّعَ أُرْنَانُ فَرَأَى دَاوُدَ وَخَرَجَ مِنَ الْبَيْدَرِ وَسَجَدَ لِدَاوُدَ عَلَى وَجْهِهِ إِلَى الأَرْضِ.
\par 22 فَقَالَ دَاوُدُ لِأُرْنَانَ: «أَعْطِنِي مَكَانَ الْبَيْدَرِ فَأَبْنِيَ فِيهِ مَذْبَحاً لِلرَّبِّ. بِفِضَّةٍ كَامِلَةٍ أَعْطِنِي إِيَّاهُ, فَتَكُفَّ الضَّرْبَةُ عَنِ الشَّعْبِ».
\par 23 فَقَالَ أُرْنَانُ لِدَاوُدَ: «خُذْهُ لِنَفْسِكَ, وَلْيَفْعَلْ سَيِّدِي الْمَلِكُ مَا يَحْسُنُ فِي عَيْنَيْهِ. انْظُرْ. قَدْ أَعْطَيْتُ الْبَقَرَ لِلْمُحْرَقَةِ, وَالنَّوَارِجَ لِلْوَقُودِ, وَالْحِنْطَةَ لِلتَّقْدِمَةِ. الْجَمِيعَ أَعْطَيْتُ».
\par 24 فَقَالَ الْمَلِكُ دَاوُدُ لِأُرْنَانَ: «لاَ! بَلْ شِرَاءً أَشْتَرِيهِ بِفِضَّةٍ كَامِلَةٍ, لأَنِّي لاَ آخُذُ مَا لَكَ لِلرَّبِّ فَأُصْعِدَ مُحْرَقَةً مَجَّانِيَّةً».
\par 25 وَدَفَعَ دَاوُدُ لِأُرْنَانَ عَنِ الْمَكَانِ ذَهَباً وَزْنُهُ سِتُّ مِئَةِ شَاقِلٍ.
\par 26 وَبَنَى دَاوُدُ هُنَاكَ مَذْبَحاً لِلرَّبِّ, وَأَصْعَدَ مُحْرَقَاتٍ وَذَبَائِحَ سَلاَمَةٍ, وَدَعَا الرَّبَّ فَأَجَابَهُ بِنَارٍ مِنَ السَّمَاءِ عَلَى مَذْبَحِ الْمُحْرَقَةِ.
\par 27 وَأَمَرَ الرَّبُّ الْمَلاَكَ فَرَدَّ سَيْفَهُ إِلَى غِمْدِهِ.
\par 28 فِي ذَلِكَ الْوَقْتِ لَمَّا رَأَى دَاوُدُ أَنَّ الرَّبَّ قَدْ أَجَابَهُ فِي بَيْدَرِ أُرْنَانَ الْيَبُوسِيِّ ذَبَحَ هُنَاكَ.
\par 29 وَمَسْكَنُ الرَّبِّ الَّذِي عَمِلَهُ مُوسَى فِي الْبَرِّيَّةِ وَمَذْبَحُ الْمُحْرَقَةِ كَانَا فِي ذَلِكَ الْوَقْتِ فِي الْمُرْتَفَعَةِ فِي جِبْعُونَ.
\par 30 وَلَمْ يَسْتَطِعْ دَاوُدُ أَنْ يَذْهَبَ إِلَى أَمَامِهِ لِيَسْأَلَ اللَّهَ لأَنَّهُ خَافَ مِنْ جِهَةِ سَيْفِ مَلاَكِ الرَّبِّ

\chapter{22}

\par 1 فَقَالَ دَاوُدُ: «هَذَا هُوَ بَيْتُ الرَّبِّ الإِلَهِ, وَهَذَا هُوَ مَذْبَحُ الْمُحْرَقَةِ لإِسْرَائِيلَ».
\par 2 وَأَمَرَ دَاوُدُ بِجَمْعِ الأَجْنَبِيِّينَ الَّذِينَ فِي أَرْضِ إِسْرَائِيلَ وَأَقَامَ نَحَّاتِينَ لِنَحْتِ حِجَارَةٍ مُرَبَّعَةٍ لِبِنَاءِ بَيْتِ اللَّهِ.
\par 3 وَهَيَّأَ دَاوُدُ حَدِيداً كَثِيراً لِلْمَسَامِيرِ لِمَصَارِيعِ الأَبْوَابِ وَلِلْوُصَلِ, وَنُحَاساً كَثِيراً بِلاَ وَزْنٍ,
\par 4 وَخَشَبَ أَرْزٍ لَمْ يَكُنْ لَهُ عَدَدٌ (لأَنَّ الصَّيْدُونِيِّينَ وَالصُّورِيِّينَ أَتُوا بِخَشَبِ أَرْزٍ كَثِيرٍ إِلَى دَاوُدَ).
\par 5 وَقَالَ دَاوُدُ: «إِنَّ سُلَيْمَانَ ابْنِي صَغِيرٌ وَغَضٌّ, وَالْبَيْتُ الَّذِي يُبْنَى لِلرَّبِّ يَكُونُ عَظِيماً جِدّاً فِي الاِسْمِ وَالْمَجْدِ فِي جَمِيعِ الأَرَاضِي, فَأَنَا أُهَيِّئُ لَهُ». فَهَيَّأَ دَاوُدُ كَثِيراً قَبْلَ وَفَاتِهِ.
\par 6 وَدَعَا سُلَيْمَانَ ابْنَهُ وَأَوْصَاهُ أَنْ يَبْنِيَ بَيْتاً لِلرَّبِّ إِلَهِ إِسْرَائِيلَ.
\par 7 وَقَالَ دَاوُدُ لِسُلَيْمَانَ: «يَا ابْنِي, قَدْ كَانَ فِي قَلْبِي أَنْ أَبْنِيَ بَيْتاً لاِسْمِ الرَّبِّ إِلَهِي.
\par 8 فَكَانَ إِلَيَّ كَلاَمُ الرَّبِّ: قَدْ سَفَكْتَ دَماً كَثِيراً وَعَمِلْتَ حُرُوباً عَظِيمَةً, فَلاَ تَبْنِي بَيْتاً لاِسْمِي لأَنَّكَ سَفَكْتَ دِمَاءً كَثِيرَةً عَلَى الأَرْضِ أَمَامِي.
\par 9 هُوَذَا يُولَدُ لَكَ ابْنٌ يَكُونُ صَاحِبَ رَاحَةٍ, وَأُرِيحُهُ مِنْ جَمِيعِ أَعْدَائِهِ حَوَالَيْهِ, لأَنَّ اسْمَهُ يَكُونُ سُلَيْمَانَ. فَأَجْعَلُ سَلاَماً وَسَكِينَةً فِي إِسْرَائِيلَ فِي أَيَّامِهِ.
\par 10 هُوَ يَبْنِي بَيْتاً لاِسْمِي, وَهُوَ يَكُونُ لِيَ ابْناً, وَأَنَا لَهُ أَباً وَأُثَبِّتُ كُرْسِيَّ مُلْكِهِ عَلَى إِسْرَائِيلَ إِلَى الأَبَدِ.
\par 11 الآنَ يَا ابْنِي لِيَكُنِ الرَّبُّ مَعَكَ فَتُفْلِحَ وَتَبْنِيَ بَيْتَ الرَّبِّ إِلَهِكَ كَمَا تَكَلَّمَ عَنْكَ.
\par 12 إِنَّمَا يُعْطِيكَ الرَّبُّ فِطْنَةً وَفَهْماً وَيُوصِيكَ بِإِسْرَائِيلَ لِحِفْظِ شَرِيعَةِ الرَّبِّ إِلَهِكَ.
\par 13 حِينَئِذٍ تُفْلِحُ إِذَا تَحَفَّظْتَ لِعَمَلِ الْفَرَائِضِ وَالأَحْكَامِ الَّتِي أَمَرَ بِهَا الرَّبُّ مُوسَى لأَجْلِ إِسْرَائِيلَ. تَشَدَّدْ وَتَشَجَّعْ لاَ تَخَفْ وَلاَ تَرْتَعِبْ.
\par 14 هَئَنَذَا فِي مَذَلَّتِي هَيَّأْتُ لِبَيْتِ الرَّبِّ ذَهَباً مِئَةَ أَلْفِ وَزْنَةٍ, وَفِضَّةً مِلْيُونَ وَزْنَةٍ, وَنُحَاساً وَحَدِيداً بِلاَ وَزْنٍ لأَنَّهُ كَثِيرٌ. وَقَدْ هَيَّأْتُ خَشَباً وَحِجَارَةً فَتَزِيدُ عَلَيْهَا.
\par 15 وَعِنْدَكَ كَثِيرُونَ مِنْ عَامِلِي الشُّغْلِ: نَحَّاتِينَ وَبَنَّائِينَ وَنَجَّارِينَ وَكُلُّ حَكِيمٍ فِي كُلِّ عَمَلٍ.
\par 16 الذَّهَبُ وَالْفِضَّةُ وَالنُّحَاسُ وَالْحَدِيدُ لَيْسَ لَهَا عَدَدٌ. قُمْ وَاعْمَلْ, وَلْيَكُنِ الرَّبُّ مَعَكَ».
\par 17 وَأَمَرَ دَاوُدُ جَمِيعَ رُؤَسَاءِ إِسْرَائِيلَ أَنْ يُسَاعِدُوا سُلَيْمَانَ ابْنَهُ:
\par 18 «أَلَيْسَ الرَّبُّ إِلَهُكُمْ مَعَكُمْ, وَقَدْ أَرَاحَكُمْ مِنْ كُلِّ نَاحِيَةٍ, لأَنَّهُ دَفَعَ لِيَدِي سُكَّانَ الأَرْضِ فَخَضَعَتِ الأَرْضُ أَمَامَ الرَّبِّ وَأَمَامَ شَعْبِهِ؟
\par 19 فَالآنَ اجْعَلُوا قُلُوبَكُمْ وَأَنْفُسَكُمْ لِطَلَبِ الرَّبِّ إِلَهِكُمْ, وَقُومُوا وَابْنُوا مَقْدِسَ الرَّبِّ الإِلَهِ, لِيُؤْتَى بِتَابُوتِ عَهْدِ الرَّبِّ وَبِآنِيَةِ قُدْسِ اللَّهِ إِلَى الْبَيْتِ الَّذِي يُبْنَى لاِسْمِ الرَّبِّ».

\chapter{23}

\par 1 وَلَمَّا شَاخَ دَاوُدُ وَشَبِعَ أَيَّاماً مَلَّكَ سُلَيْمَانَ ابْنَهُ عَلَى إِسْرَائِيلَ.
\par 2 وَجَمَعَ كُلَّ رُؤَسَاءِ إِسْرَائِيلَ وَالْكَهَنَةِ وَاللاَّوِيِّينَ,
\par 3 فَعُدَّ اللاَّوِيُّونَ مِنِ ابْنِ ثَلاَثِينَ سَنَةً فَمَا فَوْقُ, فَكَانَ عَدَدُهُمْ حَسَبَ رُؤُوسِهِمْ مِنَ الرِّجَالِ ثَمَانِيَةً وَثَلاَثِينَ أَلْفاً.
\par 4 مِنْ هَؤُلاَءِ لِلْمُنَاظَرَةِ عَلَى عَمَلِ بَيْتِ الرَّبِّ أَرْبَعَةٌ وَعِشْرُونَ أَلْفاً. وَسِتَّةُ آلاَفٍ عُرَفَاءُ وَقُضَاةٌ.
\par 5 وَأَرْبَعَةُ آلاَفٍ بَوَّابُونَ, وَأَرْبَعَةُ آلاَفٍ مُسَبِّحُونَ لِلرَّبِّ بِالآلاَتِ الَّتِي عُمِلَتْ لِلتَّسْبِيحِ.
\par 6 وَقَسَمَهُمْ دَاوُدُ فِرَقاً لِبَنِي لاَوِي لِجَرْشُونَ وَقَهَاتَ وَمَرَارِي.
\par 7 مِنَ الْجَرْشُونِيِّينَ لَعْدَانُ وَشَمْعِي.
\par 8 بَنُو لَعْدَانَ: الرَّأْسُ يَحِيئِيلُ ثُمَّ زِيثَامُ وَيُوئِيلُ, ثَلاَثَةٌ.
\par 9 بَنُو شَمْعِي: شَلُومِيثُ وَحَزِيئِيلُ وَهَارَانُ, ثَلاَثَةٌ. هَؤُلاَءِ رُؤُوسُ آبَاءٍ لِلَعْدَانَ.
\par 10 وَبَنُو شَمْعِي: يَحَثُ وَزِينَا وَيَعُوشُ وَبَرِيعَةُ. هَؤُلاَءِ بَنُو شَمْعِي أَرْبَعَةٌ.
\par 11 وَكَانَ يَحَثُ الرَّأْسَ وَزِيزَةُ الثَّانِيَ. أَمَّا يَعُوشُ وَبَرِيعَةُ فَلَمْ يُكَثِّرَا الأَوْلاَدَ, فَكَانُوا فِي الإِحْصَاءِ لِبَيْتِ أَبٍ وَاحِدٍ.
\par 12 بَنُو قَهَاتَ: عَمْرَامُ وَيِصْهَارُ وَحَبْرُونُ وَعُزِّيئِيلُ, أَرْبَعَةٌ.
\par 13 اِبْنَا عَمْرَامَ هَارُونُ وَمُوسَى, وَأُفْرِزَ هَارُونُ لِتَقْدِيسِهِ قُدْسَ أَقْدَاسٍ هُوَ وَبَنُوهُ إِلَى الأَبَدِ, لِيُوقِدَ أَمَامَ الرَّبِّ وَيَخْدِمَهُ وَيُبَارِكَ بِاسْمِهِ إِلَى الأَبَدِ.
\par 14 وَأَمَّا مُوسَى رَجُلُ اللَّهِ فَدُعِيَ بَنُوهُ مَعَ سِبْطِ لاَوِي.
\par 15 اِبْنَا مُوسَى جَرْشُومُ وَأَلِيعَزَرُ.
\par 16 بَنُو جَرْشُومَ شَبُوئِيلُ الرَّأْسُ.
\par 17 وَكَانَ ابْنُ أَلِيعَزَرَ رَحَبْيَا الرَّأْسَ وَلَمْ يَكُنْ لأَلِيعَزَرَ بَنُونَ آخَرُونَ. وَأَمَّا بَنُو رَحَبْيَا فَكَانُوا كَثِيرِينَ جِدّاً.
\par 18 بَنُو يِصْهَارَ شَلُومِيثُ الرَّأْسُ.
\par 19 بَنُو حَبْرُونَ: يَرِيَّا الرَّأْسُ وَأَمَرْيَا الثَّانِي وَيَحْزِيئِيلُ الثَّالِثُ وَيَقْمَعَامُ الرَّابِعُ.
\par 20 اِبْنَا عُزِّيئِيلَ مِيخَا الرَّأْسُ وَيَشِّيَّا الثَّانِي.
\par 21 اِبْنَا مَرَارِي مَحْلِي وَمُوشِي. ابْنَا مَحْلِي أَلِعَازَارُ وَقَيْسُ.
\par 22 وَمَاتَ أَلِعَازَارُ وَلَمْ يَكُنْ لَهُ بَنُونَ بَلْ بَنَاتٌ, فَأَخَذَهُنَّ بَنُو قَيْسَ إِخْوَتُهُنَّ.
\par 23 بَنُو مُوشِي مَحْلِي وَعَادِرُ وَيَرِيمُوثُ, ثَلاَثَةٌ.
\par 24 هَؤُلاَءِ بَنُو لاَوِي حَسَبَ بُيُوتِ آبَائِهِمْ رُؤُوسُ الآبَاءِ حَسَبَ إِحْصَائِهِمْ فِي عَدَدِ الأَسْمَاءِ حَسَبَ رُؤُوسِهِمْ عَامِلُو الْعَمَلِ لِخِدْمَةِ بَيْتِ الرَّبِّ مِنِ ابْنِ عِشْرِينَ سَنَةً فَمَا فَوْقُ.
\par 25 لأَنَّ دَاوُدَ قَالَ: «قَدْ أَرَاحَ الرَّبُّ إِلَهُ إِسْرَائِيلَ شَعْبَهُ فَسَكَنَ فِي أُورُشَلِيمَ إِلَى الأَبَدِ.
\par 26 وَلَيْسَ لِلاَّوِيِّينَ بَعْدُ أَنْ يَحْمِلُوا الْمَسْكَنَ وَكُلَّ آنِيَتِهِ لِخِدْمَتِهِ».
\par 27 لأَنَّهُ حَسَبَ كَلاَمِ دَاوُدَ الأَخِيرِ عُدَّ بَنُو لاَوِي مِنِ ابْنِ عِشْرِينَ سَنَةً فَمَا فَوْقُ.
\par 28 لأَنَّهُمْ كَانُوا يَقِفُونَ بَيْنَ يَدَيْ بَنِي هَارُونَ عَلَى خِدْمَةِ بَيْتِ الرَّبِّ فِي الدُّورِ وَالْمَخَادِعِ وَعَلَى تَطْهِيرِ كُلِّ قُدْسٍ وَعَمَلِ خِدْمَةِ بَيْتِ اللَّهِ
\par 29 وَعَلَى خُبْزِ الْوُجُوهِ وَدَقِيقِ التَّقْدِمَةِ وَرِقَاقِ الْفَطِيرِ وَمَا يُعْمَلُ عَلَى الصَّاجِ وَالْمَرْبُوكَاتِ وَعَلَى كُلِّ كَيْلٍ وَقِيَاسٍ
\par 30 وَلأَجْلِ الْوُقُوفِ كُلَّ صَبَاحٍ لِحَمْدِ الرَّبِّ وَتَسْبِيحِهِ وَكَذَلِكَ فِي الْمَسَاءِ,
\par 31 وَلِكُلِّ إِصْعَادِ مُحْرَقَاتٍ لِلرَّبِّ فِي السُّبُوتِ وَالأَهِلَّةِ وَالْمَوَاسِمِ بِالْعَدَدِ حَسَبَ الْمَرْسُومِ عَلَيْهِمْ دَائِماً أَمَامَ الرَّبِّ,
\par 32 وَلِيَحْرُسُوا حِرَاسَةَ خَيْمَةِ الاِجْتِمَاعِ وَحِرَاسَةَ الْقُدْسِ وَحِرَاسَةَ بَنِي هَرُونَ إِخْوَتِهِمْ فِي خِدْمَةِ بَيْتِ الرَّبِّ.

\chapter{24}

\par 1 وَهَذِهِ فِرَقُ بَنِي هَارُونَ: بَنُو هَارُونَ: نَادَابُ وَأَبِيهُو أَلِعَازَارُ وَإِيثَامَارُ.
\par 2 وَمَاتَ نَادَابُ وَأَبِيهُو قَبْلَ أَبِيهِمَا وَلَمْ يَكُنْ لَهُمَا بَنُونَ, فَكَهَنَ أَلِعَازَارُ وَإِيثَامَارُ.
\par 3 وَقَسَمَهُمْ دَاوُدُ وَصَادُوقُ مِنْ بَنِي أَلِعَازَارَ وَأَخِيمَالِكَ مِنْ بَنِي إِيثَامَارَ حَسَبَ وَكَالَتِهِمْ فِي خِدْمَتِهِمْ.
\par 4 وَوُجِدَ لِبَنِي أَلِعَازَارَ رُؤُوسُ رِجَالٍ أَكْثَرَ مِنْ بَنِي إِيثَامَارَ, فَانْقَسَمُوا لِبَنِي أَلِعَازَارَ رُؤُوساً لِبَيْتِ آبَائِهِمْ سِتَّةَ عَشَرَ, وَلِبَنِي إِيثَامَارَ لِبَيْتِ آبَائِهِمْ ثَمَانِيَةٌ.
\par 5 وَانْقَسَمُوا بِالْقُرْعَةِ, هَؤُلاَءِ مَعَ هَؤُلاَءِ, لأَنَّ رُؤَسَاءَ الْقُدْسِ وَرُؤَسَاءَ بَيْتِ اللَّهِ كَانُوا مِنْ بَنِي أَلِعَازَارَ وَمِنْ بَنِي إِيثَامَارَ.
\par 6 وَكَتَبَهُمْ شَمَعْيَا بْنُ نَثَنْئِيلَ الْكَاتِبُ مِنَ اللاَّوِيِّينَ أَمَامَ الْمَلِكِ وَالرُّؤَسَاءِ وَصَادُوقَ الْكَاهِنُ وَأَخِيمَالِكَ بْنِ أَبِيَاثَارَ وَرُؤُوسِ الآبَاءِ لِلْكَهَنَةِ وَاللاَّوِيِّينَ, فَأُخِذَ بَيْتُ أَبٍ وَاحِدٍ لأَلِعَازَارَ, وَأُخِذَ وَاحِدٌ لإِيثَامَارَ.
\par 7 فَخَرَجَتِ الْقُرْعَةُ الأُولَى لِيَهُويَارِيبَ. الثَّانِيَةُ لِيَدْعِيَا.
\par 8 الثَّالِثَةُ لِحَارِيمَ. الرَّابِعَةُ لِسَعُورِيمَ.
\par 9 الْخَامِسَةُ لِمَلْكِيَّا. السَّادِسَةُ لِمَيَّامِينَ.
\par 10 السَّابِعَةُ لِهُقُّوصَ. الثَّامِنَةُ لأَبِيَّا.
\par 11 التَّاسِعَةُ لِيَشُوعَ. الْعَاشِرَةُ لِشَكُنْيَا.
\par 12 الْحَادِيَةَ عَشَرَةَ لأَلْيَاشِيبَ. الثَّانِيَةَ عَشَرَةَ لِيَاقِيمَ.
\par 13 الثَّالِثَةَ عَشَرَةَ لِحُفَّةَ. الرَّابِعَةَ عَشَرَةَ لِيَشَبْآبَ.
\par 14 الْخَامِسَةَ عَشَرَةَ لِبَلْجَةَ. السَّادِسَةَ عَشَرَةَ لإِيمِيرَ.
\par 15 السَّابِعَةَ عَشَرَةَ لِحِيزِيرَ. الثَّامِنَةَ عَشَرَةَ لِهَفْصِيصَ.
\par 16 التَّاسِعَةَ عَشَرَةَ لِفَقَحْيَا. الْعِشْرُونَ لِيَحَزْقِيئِيلَ.
\par 17 الْحَادِيَةُ وَالْعِشْرُونَ لِيَاكِينَ. الثَّانِيَةُ وَالْعِشْرُونَ لِجَامُولَ.
\par 18 الثَّالِثَةُ وَالْعِشْرُونَ لِدَلاَيَا. الرَّابِعَةُ وَالْعِشْرُونَ لِمَعَزْيَا.
\par 19 فَهَذِهِ وَكَالَتُهُمْ وَخِدْمَتُهُمْ لِلدُّخُولِ إِلَى بَيْتِ الرَّبِّ حَسَبَ حُكْمِهِمْ عَنْ يَدِ هَارُونَ أَبِيهِمْ كَمَا أَمَرَهُ الرَّبُّ إِلَهُ إِسْرَائِيلَ.
\par 20 وَأَمَّا بَنُو لاَوِي فَمِنْ بَنِي عَمْرَامَ شُوبَائِيلُ, وَمِنْ بَنِي شُوبَائِيلَ يَحَدْيَا.
\par 21 وَأَمَّا رَحَبْيَا فَمِنْ بَنِي رَحَبْيَا الرَّأْسُ يَشِّيَّا.
\par 22 وَمِنَ الْيِصْهَارِيِّينَ شَلُومُوثُ, وَمِنْ بَنِي شَلُومُوثَ يَحَثُ.
\par 23 وَمِنْ بَنِي حَبْرُونَ يَرِيَّا وَأَمَرْيَا الثَّانِي وَيَحْزِيئِيلُ الثَّالِثُ وَيَقْمَعَامُ الرَّابِعُ.
\par 24 مِنْ بَنِي عُزِّيئِيلَ مِيخَا. مِنْ بَنِي مِيخَا شَامُورُ.
\par 25 أَخُو مِيخَا يَِشِّيَّا وَمِنْ بَنِي يَِشِّيَّا زَكَرِيَّا.
\par 26 اِبْنَا مَرَارِي مَحْلِي وَمُوشِي. ابْنُ يَعَزْيَا بَنُو.
\par 27 مِنْ بَنِي مَرَارِي لِيَعَزْيَا: بَنُو وَشُوهَمُ وَزَكُّورُ وَعِبْرِي.
\par 28 مِنْ مَحْلِي أَلِعَازَارُ وَلَمْ يَكُنْ لَهُ بَنُونَ.
\par 29 وَأَمَّا قَيْسُ فَابْنُ قَيْسَ يَرْحَمْئِيلُ.
\par 30 وَبَنُو مُوشِي: مَحْلِي وَعَادِرُ وَيَرِيمُوثُ. هَؤُلاَءِ بَنُو اللاَّوِيِّينَ حَسَبَ بُيُوتِ آبَائِهِمْ.
\par 31 وَأَلْقُوا هُمْ أَيْضاً قُرَعاً مُقَابَِلَ إِخْوَتِهِمْ بَنِي هَارُونَ أَمَامَ دَاوُدَ الْمَلِكِ وَصَادُوقَ وَأَخِيمَالِكَ وَرُؤُوسِ آبَاءِ الْكَهَنَةِ وَاللاَّوِيِّينَ. الآبَاءُ الرُّؤُوسُ كَمَا إِخْوَتِهِمِ الأَصَاغِرِ.

\chapter{25}

\par 1 وَأَفْرَزَ دَاوُدُ وَرُؤَسَاءُ الْجَيْشِ لِلْخِدْمَةِ بَنِي آسَافَ وَهَيْمَانَ وَيَدُوثُونَ الْمُتَنَبِّئِينَ بِالْعِيدَانِ وَالرَّبَابِ وَالصُّنُوجِ. وَكَانَ عَدَدُهُمْ مِنْ رِجَالِ الْعَمَلِ حَسَبَ خِدْمَتِهِمْ.
\par 2 مِنْ بَنِي آسَافَ: زَكُّورُ وَيُوسُفُ وَنَثَنْيَا وَأَشَرْئِيلَةُ. بَنُو آسَافَ تَحْتَ يَدِ آسَافَ الْمُتَنَبِّئِ بَيْنَ يَدَيِ الْمَلِكِ.
\par 3 مِنْ يَدُوثُونَ بَنُو يَدُوثُونَ: جَدَلْيَا وَصَرِي وَيِشْعِيَا وَحَشَبْيَا وَمَتَّثْيَا, سِتَّةٌ. تَحْتَ يَدِ أَبِيهِمْ يَدُوثُونَ الْمُتَنَبِّئِ بِالْعُودِ لأَجْلِ الْحَمْدِ وَالتَّسْبِيحِ لِلرَّبِّ.
\par 4 مِنْ هَيْمَانَ: بُقِّيَّا وَمَتَّنْيَا وَعُزِّيئِيلُ وَشَبُوئِيلُ وَيَرِيمُوثُ وَحَنَنْيَا وَحَنَانِي وَإِيلِيآثَةُ وَجِدَّلْتِي وَرُومَمْتِي عَزَرُ وَيُشْبَقَاشَةُ وَمَلُوثِي وَهُوثِيرُ وَمَحْزِيُوثُ.
\par 5 جَمِيعُ هَؤُلاَءِ بَنُو هَيْمَانَ رَائِي الْمَلِكِ بِكَلاَمِ اللَّهِ لِرَفْعِ الْقَرْنِ. وَرَزَقَ الرَّبُّ هَيْمَانَ أَرْبَعَةَ عَشَرَ ابْناً وَثَلاَثَ بَنَاتٍ.
\par 6 كُلُّ هَؤُلاَءِ تَحْتَ يَدِ أَبِيهِمْ لأَجْلِ غِنَاءِ بَيْتِ الرَّبِّ بِالصُّنُوجِ وَالرَّبَابِ وَالْعِيدَانِ لِخِدْمَةِ بَيْتِ اللَّهِ. تَحْتَ يَدِ الْمَلِكِ وَآسَافَ وَيَدُوثُونَ وَهَيْمَانَ.
\par 7 وَكَانَ عَدَدُهُمْ مَعَ إِخْوَتِهِمِ الْمُتَعَلِّمِينَ الْغِنَاءَ لِلرَّبِّ, كُلِّ الْخَبِيرِينَ مِئَتَيْنِ وَثَمَانِيَةً وَثَمَانِينَ.
\par 8 وَأَلْقُوا قُرَعَ الْحِرَاسَةِ الصَّغِيرُ كَمَا الْكَبِيرِ, الْمُعَلِّمُ مَعَ التِّلْمِيذِ.
\par 9 فَخَرَجَتِ الْقُرْعَةُ الأُولَى الَّتِي هِيَ لِآسَافَ لِيُوسُفَ. الثَّانِيَةُ لِجَدَلْيَا. هُوَ وَإِخْوَتُهُ وَبَنُوهُ اثْنَا عَشَرَ.
\par 10 الثَّالِثَةُ لِزَكُّورَ. بَنُوهُ وَإِخْوَتُهُ اثْنَا عَشَرَ.
\par 11 الرَّابِعَةُ لِيَصْرِي. بَنُوهُ وَإِخْوَتُهُ اثْنَا عَشَرَ.
\par 12 الْخَامِسَةُ لِنَثَنْيَا. بَنُوهُ وَإِخْوَتُهُ اثْنَا عَشَرَ.
\par 13 السَّادِسَةُ لِبُقِّيَّا. بَنُوهُ وَإِخْوَتُهُ اثْنَا عَشَرَ.
\par 14 السَّابِعَةُ لِيَشَرْئِيلَةَ. بَنُوهُ وَإِخْوَتُهُ اثْنَا عَشَرَ.
\par 15 الثَّامِنَةُ لِيَشْعِيَا. بَنُوهُ وَإِخْوَتُهُ اثْنَا عَشَرَ.
\par 16 التَّاسِعَةُ لِمَتَّنْيَا. بَنُوهُ وَإِخْوَتُهُ اثْنَا عَشَرَ.
\par 17 الْعَاشِرَةُ لِشَمْعِي. بَنُوهُ وَإِخْوَتُهُ اثْنَا عَشَرَ.
\par 18 الْحَادِيَةَ عَشَرَةَ لِعَزَرْئِيلَ. بَنُوهُ وَإِخْوَتُهُ اثْنَا عَشَرَ.
\par 19 الثَّانِيَةَ عَشَرَةَ لِحَشَبْيَا. بَنُوهُ وَإِخْوَتُهُ اثْنَا عَشَرَ.
\par 20 الثَّالِثَةَ عَشْرَةَ لِشُوبَائِيلَ. بَنُوهُ وَإِخْوَتُهُ اثْنَا عَشَرَ.
\par 21 الرَّابِعَةَ عَشْرَةَ لِمَتَّثْيَا. بَنُوهُ وَإِخْوَتُهُ اثْنَا عَشَرَ.
\par 22 الْخَامِسَةَ عَشْرَةَ لِيَرِيمُوثَ. بَنُوهُ وَإِخْوَتُهُ اثْنَا عَشَرَ.
\par 23 السَّادِسَةَ عَشْرَةَ لِحَنَنْيَا. بَنُوهُ وَإِخْوَتُهُ اثْنَا عَشَرَ.
\par 24 السَّابِعَةَ عَشْرَةَ لِيَشْبَقَاشَةَ. بَنُوهُ وَإِخْوَتُهُ اثْنَا عَشَرَ.
\par 25 الثَّامِنَةَ عَشْرَةَ لِحَنَانِي. بَنُوهُ وَإِخْوَتُهُ اثْنَا عَشَرَ.
\par 26 التَّاسِعَةَ عَشْرَةَ لِمَلُّوثِي. بَنُوهُ وَإِخْوَتُهُ اثْنَا عَشَرَ.
\par 27 الْعِشْرُونَ لإِيلِيآثَةَ. بَنُوهُ وَإِخْوَتُهُ اثْنَا عَشَرَ.
\par 28 الْحَادِيَةُ وَالْعِشْرُونَ لِهُوثِيرَ. بَنُوهُ وَإِخْوَتُهُ اثْنَا عَشَرَ.
\par 29 الثَّانِيَةُ وَالْعِشْرُونَ لِجِدَّلْتِي. بَنُوهُ وَإِخْوَتُهُ اثْنَا عَشَرَ.
\par 30 الثَّالِثَةُ وَالْعِشْرُونَ لِمَحْزِيُوثَ. بَنُوهُ وَإِخْوَتُهُ اثْنَا عَشَرَ.
\par 31 الرَّابِعَةُ وَالْعِشْرُونَ لِرُومَمْتِي عَزَرَ. بَنُوهُ وَإِخْوَتُهُ اثْنَا عَشَرَ.

\chapter{26}

\par 1 وَأَمَّا أَقْسَامُ الْبَوَّابِينَ فَمِنَ الْقُورَحِيِّينَ: مَشَلَمْيَا بْنُ قُورِي مِنْ بَنِي آسَافَ.
\par 2 وَكَانَ لِمَشَلَمْيَا بَنُونَ: زَكَرِيَّا الْبِكْرُ وَيَدِيعَئِيلُ الثَّانِي وَزَبَدْيَا الثَّالِثُ وَيَثَنْئِيلُ الرَّابِعُ
\par 3 وَعِيلاَمُ الْخَامِسُ وَيَهُوحَانَانُ السَّادِسُ وَأَلِيهُو عِينَايُ السَّابِعُ.
\par 4 وَكَانَ لِعُوبِيدَ أَدُومَ بَنُونَ: شَمَعْيَا الْبِكْرُ وَيَهُوزَابَادُ الثَّانِي وَيُوآخُ الثَّالِثُ وَسَاكَارُ الرَّابِعُ وَنَثَنْئِيلُ الْخَامِسُ
\par 5 وَعَمِّيئِيلُ السَّادِسُ وَيَسَّاكَرُ السَّابِعُ وَفَعَلْتَايُ الثَّامِنُ. لأَنَّ اللَّهَ بَارَكَهُ.
\par 6 وَلِشَمَعْيَا ابْنِهِ وُلِدَ بَنُونَ تَسَلَّطُوا فِي بَيْتِ آبَائِهِمْ لأَنَّهُمْ جَبَابِرَةُ بَأْسٍ.
\par 7 بَنُو شَمَعْيَا: عَثْنِي وَرَفَائِيلُ وَعُوبِيدُ وَأَلْزَابَادُ إِخْوَتُهُ أَصْحَابُ بَأْسٍ. أَلِيهُو وَسَمَكْيَا.
\par 8 كُلُّ هَؤُلاَءِ مِنْ بَنِي عُوبِيدَ أَدُومَ هُمْ وَبَنُوهُمْ وَإِخْوَتُهُمْ أَصْحَابُ بَأْسٍ بِقُوَّةٍ فِي الْخِدْمَةِ, اثْنَانِ وَسِتُّونَ لِعُوبِيدَ أَدُومَ.
\par 9 وَكَانَ لِمَشَلَمْيَا بَنُونَ وَإِخْوَةٌ أَصْحَابُ بَأْسٍ ثَمَانِيَةَ عَشَرَ.
\par 10 وَكَانَ لِحُوسَةَ مِنْ بَنِي مَرَارِي بَنُونَ: شِمْرِي الرَّأْسُ (مَعَ أَنَّهُ لَمْ يَكُنْ بِكْراً جَعَلَهُ أَبُوهُ رَأْساً)
\par 11 حِلْقِيَّا الثَّانِي وَطَبَلْيَا الثَّالِثُ وَزَكَرِيَّا الرَّابِعُ. كُلُّ بَنِي حُوسَةَ وَإِخْوَتُهُ ثَلاَثَةَ عَشَرَ.
\par 12 لِفِرَقِ الْبَوَّابِينَ هَؤُلاَءِ حَسَبَ رُؤُوسِ الْجَبَابِرَةِ حِرَاسَةٌ كَمَا لإِخْوَتِهِمْ لِلْخِدْمَةِ فِي بَيْتِ الرَّبِّ.
\par 13 وَأَلْقُوا قُرَعاً الصَّغِيرُ كَالْكَبِيرِ حَسَبَ بُيُوتِ آبَائِهِمْ لِكُلِّ بَابٍ.
\par 14 فَأَصَابَتِ الْقُرْعَةُ مِنْ جِهَةِ الشَّرْقِ شَلَمْيَا. وَلِزَكَرِيَّا ابْنِهِ الْمُشِيرِ بِفِطْنَةٍ أَلْقُوا قُرَعاً فَخَرَجَتِ الْقُرْعَةُ لَهُ إِلَى الشِّمَالِ.
\par 15 لِعُوبِيدَ أَدُومَ إِلَى الْجَنُوبِ وَلِبَنِيهِ الْمَخَازِنُ.
\par 16 لِشُفِّيمَ وَحُوسَةَ إِلَى الْغَرْبِ مَعَ بَابِ شَلَّكَةَ فِي مَصْعَدِ الدَّرَجِ مَحْرَسٌ مُقَابَِلَ مَحْرَسٍ.
\par 17 مِنْ جِهَةِ الشَّرْقِ كَانَ اللاَّوِيُّونَ سِتَّةً. مِنْ جِهَةِ الشِّمَالِ أَرْبَعَةً لِلْيَوْمِ. مِنْ جِهَةِ الْجَنُوبِ أَرْبَعَةً لِلْيَوْمِ وَمِنْ جِهَةِ الْمَخَازِنِ اثْنَيْنِ اثْنَيْنِ.
\par 18 مِنْ جِهَةِ الرِّوَاقِ إِلَى الْغَرْبِ أَرْبَعَةً فِي الْمَصْعَدِ وَاثْنَيْنِ فِي الرِّوَاقِ.
\par 19 هَذِهِ أَقْسَامُ الْبَوَّابِينَ مِنْ بَنِي الْقُورَحِيِّينَ وَمِنْ بَنِي مَرَارِي.
\par 20 وَأَمَّا اللاَّوِيُّونَ فَأَخِيَّا عَلَى خَزَائِنِ بَيْتِ اللَّهِ وَعَلَى خَزَائِنِ الأَقْدَاسِ.
\par 21 وَأَمَّا بَنُو لَعْدَانَ فَبَنُو لَعْدَانَ الْجَرْشُونِيِّ رُؤُوسُ بَيْتِ الآبَاءِ لِلَعْدَانَ, الْجَرْشُونِيِّ يَحِيئِيلِي.
\par 22 بَنُو يَحِيئِيلِي: زِيثَامُ وَيُوئِيلُ أَخُوهُ عَلَى خَزَائِنِ بَيْتِ الرَّبِّ.
\par 23 مِنَ الْعَمْرَامِيِّينَ وَالْيِصْهَارِيِّينَ وَالْحَبْرُونِيِّينَ وَالْعُزِّيئِيلِيِّينَ,
\par 24 كَانَ شَبُوئِيلُ بْنُ جَرْشُومَ بْنِ مُوسَى وَكَانَ رَئِيساً عَلَى الْخَزَائِنِ.
\par 25 وَإِخْوَتُهُ مِنْ أَلِيعَزَرَ رَحَبْيَا ابْنُهُ وَيَشْعِيَا ابْنُهُ وَيُورَامُ ابْنُهُ وَزِكْرِي ابْنُهُ وَشَلُومِيثُ ابْنُهُ.
\par 26 شَلُومِيثُ هَذَا وَإِخْوَتُهُ كَانُوا عَلَى جَمِيعِ خَزَائِنِ الأَقْدَاسِ الَّتِي قَدَّسَهَا دَاوُدُ الْمَلِكُ وَرُؤُوسُ الآبَاءِ وَرُؤَسَاءُ الأُلُوفِ وَالْمِئَاتِ وَرُؤَسَاءُ الْجَيْشِ.
\par 27 مِنَ الْحُرُوبِ وَمِنَ الْغَنَائِمِ قَدَّسُوا لِتَشْدِيدِ بَيْتِ الرَّبِّ.
\par 28 وَكُلُّ مَا قَدَّسَهُ صَمُوئِيلُ الرَّائِي وَشَاوُلُ بْنُ قَيْسَ وَأَبْنَيْرُ بْنُ نَيْرَ وَيُوآبُ ابْنُ صَرُويَةَ, كُلُّ مُقَدَّسٍ كَانَ تَحْتَ يَدِ شَلُومِيثَ وَإِخْوَتِهِ.
\par 29 وَمِنَ الْيِصْهَارِيِّينَ كَنَنْيَا وَبَنُوهُ لِلْعَمَلِ الْخَارِجِيِّ عَلَى إِسْرَائِيلَ عُرَفَاءَ وَقُضَاةً.
\par 30 مِنَ الْحَبْرُونِيِّينَ حَشَبْيَا وَإِخْوَتُهُ ذَوُو بَأْسٍ أَلْفٌ وَسَبْعُ مِئَةٍ مُوَكَّلِينَ عَلَى إِسْرَائِيلَ فِي عَبْرِ الأُرْدُنِّ غَرْباً فِي كُلِّ عَمَلِ الرَّبِّ وَفِي خِدْمَةِ الْمَلِكِ.
\par 31 مِنَ الْحَبْرُونِيِّينَ يَرِيَّا رَأْسُ الْحَبْرُونِيِّينَ حَسَبَ مَوَالِيدِ آبَائِهِ. فِي السَّنَةِ الرَّابِعَةِ لِمُلْكِ دَاوُدَ طُلِبُوا فَوُجِدَ فِيهِمْ جَبَابِرَةُ بَأْسٍ فِي يَعْزِيرَ جِلْعَادَ.
\par 32 وَإِخْوَتُهُ ذَوُو بَأْسٍ أَلْفَانِ وَسَبْعُ مِئَةٍ رُؤُوسُ آبَاءٍ. وَوَكَّلَهُمْ دَاوُدُ الْمَلِكُ عَلَى الرَّأُوبَيْنِيِّينَ وَالْجَادِيِّينَ وَنِصْفِ سِبْطِ مَنَسَّى فِي كُلِّ أُمُورِ اللَّهِ وَأُمُورِ الْمَلِكِ.

\chapter{27}

\par 1 وَبَنُو إِسْرَائِيلَ حَسَبَ عَدَدِهِمْ مِنْ رُؤُوسِ الآبَاءِ وَرُؤَسَاءِ الأُلُوفِ وَالْمِئَاتِ وَعُرَفَاؤُهُمُ الَّذِينَ يَخْدِمُونَ الْمَلِكَ فِي كُلِّ أُمُورِ الْفِرَقِ الدَّاخِلِينَ وَالْخَارِجِينَ شَهْراً فَشَهْراً لِكُلِّ شُهُورِ السَّنَةِ, كُلُّ فِرْقَةٍ كَانَتْ أَرْبَعَةً وَعِشْرِينَ أَلْفاً.
\par 2 عَلَى الْفِرْقَةِ الأُولَى لِلشَّهْرِ الأَوَّلِ يَشُبْعَامُ بْنُ زَبْدِيئِيلَ, وَفِي فِرْقَتِهِ أَرْبَعَةٌ وَعِشْرُونَ أَلْفاً.
\par 3 مِنْ بَنِي فَارَصَ كَانَ رَأْسُ جَمِيعِ رُؤَسَاءِ الْجُيُوشِ لِلشَّهْرِ الأَوَّلِ.
\par 4 وَعَلَى فِرْقَةِ الشَّهْرِ الثَّانِي دُودَايُ الأَخُوخِيُّ, وَمِنْ فِرْقَتِهِ مَقْلُوثُ الرَّئِيسُ. وَفِي فِرْقَتِهِ أَرْبَعَةٌ وَعِشْرُونَ أَلْفاً.
\par 5 رَئِيسُ الْجَيْشِ الثَّالِثُ لِلشَّهْرِ الثَّالِثِ بَنَايَا بْنُ يَهُويَادَاعَ الْكَاهِنُ الرَّأْسُ, وَفِي فِرْقَتِهِ أَرْبَعَةٌ وَعِشْرُونَ أَلْفاً.
\par 6 هُوَ بَنَايَا جَبَّارُ الثَّلاَثِينَ, وَعَلَى الثَّلاَثِينَ وَمِنْ فِرْقَتِهِ عَمِّيزَابَادُ ابْنُهُ.
\par 7 الرَّابِعُ لِلشَّهْرِ الرَّابِعِ عَسَائِيلُ أَخُو يُوآبَ وَزَبَدْيَا ابْنُهُ بَعْدَهُ, وَفِي فِرْقَتِهِ أَرْبَعَةٌ وَعِشْرُونَ أَلْفاً.
\par 8 الْخَامِسُ لِلشَّهْرِ الْخَامِسِ الرَّئِيسُ شَمْحُوثُ الْيَزْرَاحِيُّ, وَفِي فِرْقَتِهِ أَرْبَعَةٌ وَعِشْرُونَ أَلْفاً.
\par 9 السَّادِسُ لِلشَّهْرِ السَّادِسِ عِيرَا بْنُ عِقِّيشَ التَّقُوعِيُّ, وَفِي فِرْقَتِهِ أَرْبَعَةٌ وَعِشْرُونَ أَلْفاً.
\par 10 السَّابِعُ لِلشَّهْرِ السَّابِعِ حَالِصُ الْفَلُونِيُّ مِنْ بَنِي أَفْرَايِمَ, وَفِي فِرْقَتِهِ أَرْبَعَةٌ وَعِشْرُونَ أَلْفاً.
\par 11 الثَّامِنُ لِلشَّهْرِ الثَّامِنِ سِبْكَايُ الْحُوشَاتِيُّ مِنَ الزَّارَحِيِّينَ, وَفِي فِرْقَتِهِ أَرْبَعَةٌ وَعِشْرُونَ أَلْفاً.
\par 12 التَّاسِعُ لِلشَّهْرِ التَّاسِعِ أَبِيعَزَرُ الْعَنَاثُوثِيُّ مِنْ بِنْيَامِينَ, وَفِي فِرْقَتِهِ أَرْبَعَةٌ وَعِشْرُونَ أَلْفاً.
\par 13 الْعَاشِرُ لِلشَّهْرِ الْعَاشِرِ مَهْرَايُ النَّطُوفَاتِيُّ مِنَ الزَّارَحِيِّينَ, وَفِي فِرْقَتِهِ أَرْبَعَةٌ وَعِشْرُونَ أَلْفاً.
\par 14 الْحَادِي عَشَرَ لِلشَّهْرِ الْحَادِي عَشَرَ بَنَايَا الْفَرْعَتُونِيُّ مِنْ بَنِي أَفْرَايِمَ, وَفِي فِرْقَتِهِ أَرْبَعَةٌ وَعِشْرُونَ أَلْفاً.
\par 15 الثَّانِي عَشَرَ لِلشَّهْرِ الثَّانِي عَشَرَ خَلْدَايُ النَّطُوفَاتِيُّ مِنْ عُثْنِيئِيلَ, وَفِي فِرْقَتِهِ أَرْبَعَةٌ وَعِشْرُونَ أَلْفاً.
\par 16 وَعَلَى أَسْبَاطِ إِسْرَائِيلَ. لِلرَّأُوبَيْنِيِّينَ الرَّئِيسُ أَلِيعَزَرُ بْنُ زِكْرِي. لِلشَّمْعُونِيِّينَ شَفَطْيَا بْنُ مَعْكَةَ.
\par 17 لِلاَّوِيِّينَ حَشَبْيَا بْنُ قَمُوئِيلَ. لِهَرُونَ صَادُوقُ.
\par 18 لِيَهُوذَا أَلِيهُو مِنْ إِخْوَةِ دَاوُدَ. لِيَسَّاكَرَ عَمْرِي بْنُ مِيخَائِيلَ.
\par 19 لِزَبُولُونَ يَشْمَعِيَا بْنُ عُوبَدْيَا. لِنَفْتَالِي يَرِيمُوثُ بْنُ عَزَرْئِيلَ.
\par 20 لِبَنِي أَفْرَايِمَ هُوشَعُ بْنُ عَزَزْيَا. لِنِصْفِ سِبْطِ مَنَسَّى يُوئِيلُ بْنُ فَدَايَا.
\par 21 لِنِصْفِ سِبْطِ مَنَسَّى فِي جِلْعَادَ يَدُّو بْنُ زَكَرِيَّا. لِبِنْيَامِينَ يَعْسِيئِيلُ بْنُ أَبْنَيْرَ.
\par 22 لِدَانَ عَزَرْئِيلُ بْنُ يَرُوحَامَ. هَؤُلاَءِ رُؤَسَاءُ أَسْبَاطِ إِسْرَائِيلَ.
\par 23 وَلَمْ يَأْخُذْ دَاوُدُ عَدَدَهُمْ مِنِ ابْنِ عِشْرِينَ سَنَةً فَمَا دُونَ, لأَنَّ الرَّبَّ قَالَ إِنَّهُ يُكَثِّرُ إِسْرَائِيلَ كَنُجُومِ السَّمَاءِ.
\par 24 يُوآبُ ابْنُ صَرُويَةَ ابْتَدَأَ يُحْصِي وَلَمْ يُكْمِلْ لأَنَّهُ كَانَ بِسَبَبِ ذَلِكَ سَخَطٌ عَلَى إِسْرَائِيلَ, وَلَمْ يُدَوَّنِ الْعَدَدُ فِي سِفْرِ أَخْبَارِ الأَيَّامِ لِلْمَلِكِ دَاوُدَ.
\par 25 وَعَلَى خَزَائِنِ الْمَلِكِ عَزْمُوتُ بْنُ عَدِيئِيلَ. وَعَلَى الْخَزَائِنِ فِي الْحَقْلِ فِي الْمُدُنِ وَالْقُرَى وَالْحُصُونِ يَهُونَاثَانُ بْنُ عُزِّيَّا.
\par 26 وَعَلَى الْفَعَلَةِ فِي الْحَقْلِ لِشُغْلِ الأَرْضِ عَزْرِي بْنُ كَلُوبَ.
\par 27 وَعَلَى الْكُرُومِ شَمْعِي الرَّامِيُّ. وَعَلَى مَا فِي الْكُرُومِ مِنْ خَزَائِنِ الْخَمْرِ زَبْدِي الشَّفْمِيُّ.
\par 28 وَعَلَى الزَّيْتُونِ وَالْجُمَّيْزِ اللَّذَيْنِ فِي السَّهْلِ بَعْلُ حَانَانَ الْجَدِيرِيُّ. وَعَلَى خَزَائِنِ الزَّيْتِ يُوعَاشُ.
\par 29 وَعَلَى الْبَقَرِ السَّائِمِ فِي شَارُونَ شَطْرَايُ الشَّارُونِيُّ. وَعَلَى الْبَقَرِ الَّذِي فِي الأَوْدِيَةِ شَافَاطُ بْنُ عَدْلاَيَ.
\par 30 وَعَلَى الْجِمَالِ أُوبِيلُ الإِسْمَاعِيلِيُّ. وَعَلَى الْحَمِيرِ يَحَدْيَا الْمِيرُونُوثِيُّ.
\par 31 وَعَلَى الْغَنَمِ يَازِيزُ الْهَاجِرِيُّ. كُلُّ هَؤُلاَءِ رُؤَسَاءُ الأَمْلاَكِ الَّتِي لِلْمَلِكِ دَاوُدَ.
\par 32 وَيَهُونَاثَانُ عَمُّ دَاوُدَ كَانَ مُشِيراً وَرَجُلاً مُخْتَبِراً وَفَقِيهاً. وَيَحِيئِيلُ بْنُ حَكْمُونِي كَانَ مَعَ بَنِي الْمَلِكِ.
\par 33 وَكَانَ أَخِيتُوفَلُ مُشِيراً لِلْمَلِكِ, وَحُوشَايُ الأَرْكِيُّ صَاحِبَ الْمَلِكِ.
\par 34 وَبَعْدَ أَخِيتُوفَلَ يَهُويَادَاعُ بْنُ بَنَايَا وَأَبِيَاثَارُ. وَكَانَ يُوآبُ رَئِيسَ جَيْشِ الْمَلِكِ.

\chapter{28}

\par 1 وَجَمَعَ دَاوُدُ كُلَّ رُؤَسَاءِ إِسْرَائِيلَ, رُؤَسَاءَ الأَسْبَاطِ وَرُؤَسَاءَ الْفِرَقِ الْخَادِمِينَ الْمَلِكَ, وَرُؤَسَاءَ الأُلُوفِ وَرُؤَسَاءَ الْمِئَاتِ, وَرُؤَسَاءَ كُلِّ الأَمْوَالِ وَالأَمْلاَكِ الَّتِي لِلْمَلِكِ وَلِبَنِيهِ, مَعَ الْخِصْيَانِ وَالأَبْطَالِ وَكُلِّ جَبَابِرَةِ الْبَأْسِ, إِلَى أُورُشَلِيمَ.
\par 2 وَوَقَفَ دَاوُدُ الْمَلِكُ وَقَالَ: «اِسْمَعُونِي يَا إِخْوَتِي وَشَعْبِي. كَانَ فِي قَلْبِي أَنْ أَبْنِيَ بَيْتَ قَرَارٍ لِتَابُوتِ عَهْدِ الرَّبِّ وَلِمَوْطِئِ قَدَمَيْ إِلَهِنَا, وَقَدْ هَيَّأْتُ لِلْبِنَاءِ.
\par 3 وَلَكِنَّ اللَّهَ قَالَ لِي: لاَ تَبْنِي بَيْتاً لاِسْمِي لأَنَّكَ أَنْتَ رَجُلُ حُرُوبٍ وَقَدْ سَفَكْتَ دَماً.
\par 4 وَقَدِ اخْتَارَنِي الرَّبُّ إِلَهُ إِسْرَائِيلَ مِنْ كُلِّ بَيْتِ أَبِي لأَكُونَ مَلِكاً عَلَى إِسْرَائِيلَ إِلَى الأَبَدِ, لأَنَّهُ إِنَّمَا اخْتَارَ يَهُوذَا رَئِيساً, وَمِنْ بَيْتِ يَهُوذَا بَيْتَ أَبِي, وَمِنْ بَنِي أَبِي سُرَّ بِي لِيُمَلِّكَنِي عَلَى كُلِّ إِسْرَائِيلَ.
\par 5 وَمِنْ كُلِّ بَنِيَّ (لأَنَّ الرَّبَّ أَعْطَانِي بَنِينَ كَثِيرِينَ) اخْتَارَ سُلَيْمَانَ ابْنِي لِيَجْلِسَ عَلَى كُرْسِيِّ مَمْلَكَةِ الرَّبِّ عَلَى إِسْرَائِيلَ.
\par 6 وَقَالَ لِي: إِنَّ سُلَيْمَانَ ابْنَكَ هُوَ يَبْنِي بَيْتِي وَدِيَارِي, لأَنِّي اخْتَرْتُهُ لِي ابْناً, وَأَنَا أَكُونُ لَهُ أَباً,
\par 7 وَأُثَبِّتُ مَمْلَكَتَهُ إِلَى الأَبَدِ إِذَا تَشَدَّدَ لِلْعَمَلِ حَسَبَ وَصَايَايَ وَأَحْكَامِي كَهَذَا الْيَوْمِ.
\par 8 وَالآنَ فِي أَعْيُنِ كُلِّ إِسْرَائِيلَ مَحْفَلِ الرَّبِّ, وَفِي سَمَاعِ إِلَهِنَا, احْفَظُوا وَاطْلُبُوا جَمِيعَ وَصَايَا الرَّبِّ إِلَهِكُمْ لِتَرِثُوا الأَرْضَ الْجَيِّدَةَ وَتُوَرِّثُوهَا لأَوْلاَدِكُمْ بَعْدَكُمْ إِلَى الأَبَدِ.
\par 9 وَأَنْتَ يَا سُلَيْمَانُ ابْنِي اعْرِفْ إِلَهَ أَبِيكَ وَاعْبُدْهُ بِقَلْبٍ كَامِلٍ وَنَفْسٍ رَاغِبَةٍ, لأَنَّ الرَّبَّ يَفْحَصُ جَمِيعَ الْقُلُوبِ وَيَفْهَمُ كُلَّ تَصَوُّرَاتِ الأَفْكَارِ. فَإِذَا طَلَبْتَهُ يُوجَدُ مِنْكَ, وَإِذَا تَرَكْتَهُ يَرْفُضُكَ إِلَى الأَبَدِ.
\par 10 اُنْظُرِ الآنَ لأَنَّ الرَّبَّ قَدِ اخْتَارَكَ لِتَبْنِيَ بَيْتاً لِلْمَقْدِسِ, فَتَشَدَّدْ وَاعْمَلْ».
\par 11 وَأَعْطَى دَاوُدُ سُلَيْمَانَ ابْنَهُ مِثَالَ الرِّوَاقِ وَبُيُوتِهِ وَخَزَائِنِهِ وَعَلاَلِيِّهِ وَمَخَادِعِهِ الدَّاخِلِيَّةِ وَبَيْتِ الْغِطَاءِ,
\par 12 وَمِثَالَ كُلِّ مَا كَانَ عِنْدَهُ بِالرُّوحِ لِدِيَارِ بَيْتِ الرَّبِّ وَلِجَمِيعِ الْمَخَادِعِ حَوَالَيْهِ, وَلِخَزَائِنِ بَيْتِ اللَّهِ وَخَزَائِنِ الأَقْدَاسِ,
\par 13 وَلِفِرَقِ الْكَهَنَةِ وَاللاَّوِيِّينَ, وَلِكُلِّ عَمَلِ خِدْمَةِ بَيْتِ الرَّبِّ, وَلِكُلِّ آنِيَةِ خِدْمَةِ بَيْتِ الرَّبِّ.
\par 14 فَمِنَ الذَّهَبِ بِالْوَزْنِ لِمَا هُوَ مِنْ ذَهَبٍ لِكُلِّ آنِيَةِ خِدْمَةٍ فَخِدْمَةٍ, وَلِجَمِيعِ آنِيَةِ الْفِضَّةِ فِضَّةً بِالْوَزْنِ لِكُلِّ آنِيَةِ خِدْمَةٍ فَخِدْمَةٍ.
\par 15 وَبِالْوَزْنِ لِمَنَائِرِ الذَّهَبِ وَسُرُجِهَا مِنْ ذَهَبٍ بِالْوَزْنِ لِكُلِّ مَنَارَةٍ فَمَنَارَةٍ وَسُرُجِهَا, وَلِمَنَائِرِ الْفِضَّةِ بِالْوَزْنِ لِكُلِّ مَنَارَةٍ وَسُرُجِهَا حَسَبَ خِدْمَةِ مَنَارَةٍ فَمَنَارَةٍ.
\par 16 وَذَهَباً بِالْوَزْنِ لِمَوَائِدِ خُبْزِ الْوُجُوهِ لِكُلِّ مَائِدَةٍ فَمَائِدَةٍ, وَفِضَّةً لِمَوَائِدِ الْفِضَّةِ.
\par 17 وَذَهَباً خَالِصاً لِلْمَنَاشِلِ وَالْمَنَاضِحِ وَالْكُؤُوسِ. وَلأَقْدَاحِ الذَّهَبِ بِالْوَزْنِ لِقَدَحٍ فَقَدَحٍ, وَلأَقْدَاحِ الْفِضَّةِ بِالْوَزْنِ لِقَدَحٍ فَقَدَحٍ.
\par 18 وَلِمَذْبَحِ الْبَخُورِ ذَهَباً مُصَفًّى بِالْوَزْنِ, وَذَهَباً لِمِثَالِ مَرْكَبَةِ الْكَرُوبِيمِ الْبَاسِطَةِ أَجْنِحَتَهَا الْمُظَلِّلَةِ تَابُوتَ عَهْدِ الرَّبِّ.
\par 19 وَقَالَ: «قَدْ أَفْهَمَنِي الرَّبُّ كُلَّ ذَلِكَ بِالْكِتَابَةِ بِيَدِهِ عَلَيَّ, أَيْ كُلَّ أَشْغَالِ الْمِثَالِ».
\par 20 وَقَالَ دَاوُدُ لِسُلَيْمَانَ ابْنِهِ: «تَشَدَّدْ وَتَشَجَّعْ وَاعْمَلْ. لاَ تَخَفْ وَلاَ تَرْتَعِبْ, لأَنَّ الرَّبَّ الإِلَهَ إِلَهِي مَعَكَ. لاَ يَخْذُلُكَ وَلاَ يَتْرُكُكَ حَتَّى تُكَمِّلَ كُلَّ عَمَلِ خِدْمَةِ بَيْتِ الرَّبِّ.
\par 21 وَهُوَذَا فِرَقُ الْكَهَنَةِ وَاللاَّوِيِّينَ لِكُلِّ خِدْمَةِ بَيْتِ اللَّهِ. وَمَعَكَ فِي كُلِّ عَمَلٍ كُلُّ نَبِيهٍ بِحِكْمَةٍ لِكُلِّ خِدْمَةٍ وَالرُّؤَسَاءُ, وَكُلُّ الشَّعْبِ تَحْتَ كُلِّ أَوَامِرِكَ».

\chapter{29}

\par 1 وَقَالَ دَاوُدُ الْمَلِكُ لِكُلِّ الْمَجْمَعِ: «إِنَّ سُلَيْمَانَ ابْنِي الَّذِي وَحْدَهُ اخْتَارَهُ اللَّهُ إِنَّمَا هُوَ صَغِيرٌ وَغَضٌّ, وَالْعَمَلُ عَظِيمٌ لأَنَّ الْهَيْكَلَ لَيْسَ لإِنْسَانٍ بَلْ لِلرَّبِّ الإِلَهِ.
\par 2 وَأَنَا بِكُلِّ قُوَّتِي هَيَّأْتُ لِبَيْتِ إِلَهِيَ الذَّهَبَ لِمَا هُوَ مِنْ ذَهَبٍ, وَالْفِضَّةَ لِمَا هُوَ مِنْ فِضَّةٍ, وَالنُّحَاسَ لِمَا هُوَ مِنْ نُحَاسٍ, وَالْحَدِيدَ لِمَا هُوَ مِنْ حَدِيدٍ, وَالْخَشَبَ لِمَا هُوَ مِنْ خَشَبٍ, وَحِجَارَةَ الْجَزَعِ وَحِجَارَةً لِلتَّرْصِيعِ وَحِجَارَةً كَحْلاَءَ وَرَقْمَاءَ, وَكُلَّ حِجَارَةٍ كَرِيمَةٍ وَحِجَارَةَ الرُّخَامِ بِكَثْرَةٍ.
\par 3 وَأَيْضاً لأَنِّي قَدْ سُرِرْتُ بِبَيْتِ إِلَهِي, لِي خَاصَّةٌ مِنْ ذَهَبٍ وَفِضَّةٍ قَدْ دَفَعْتُهَا لِبَيْتِ إِلَهِي فَوْقَ جَمِيعِ مَا هَيَّأْتُهُ لِبَيْتِ الْقُدْسِ:
\par 4 ثَلاَثَةَ آلاَفِ وَزْنَةِ ذَهَبٍ مِنْ ذَهَبِ أُوفِيرَ, وَسَبْعَةَ آلاَفِ وَزْنَةِ فِضَّةٍ مُصَفَّاةٍ, لأَجْلِ تَغْشِيَةِ حِيطَانِ الْبُيُوتِ.
\par 5 الذَّهَبُ لِلذَّهَبِ وَالْفِضَّةُ لِلْفِضَّةِ وَلِكُلِّ عَمَلٍ بِيَدِ أَرْبَابِ الصَّنَائِعِ. فَمَنْ يَنْتَدِبُ الْيَوْمَ لِمِلْءِ يَدِهِ لِلرَّبِّ؟»
\par 6 فَانْتَدَبَ رُؤَسَاءُ الآبَاءِ وَرُؤَسَاءُ أَسْبَاطِ إِسْرَائِيلَ وَرُؤَسَاءُ الأُلُوفِ وَالْمِئَاتِ وَرُؤَسَاءُ أَشْغَالِ الْمَلِكِ,
\par 7 وَأَعْطُوا لِخِدْمَةِ بَيْتِ اللَّهِ خَمْسَةَ آلاَفِ وَزْنَةٍ وَعَشَرَةَ آلاَفِ دِرْهَمٍ مِنَ الذَّهَبِ, وَعَشَرَةَ آلاَفِ وَزْنَةٍ مِنَ الْفِضَّةِ وَثَمَانِيَةَ عَشَرَ أَلْفَ وَزْنَةٍ مِنَ النُّحَاسِ, وَمِئَةَ أَلْفِ وَزْنَةٍ مِنَ الْحَدِيدِ.
\par 8 وَمَنْ وُجِدَ عِنْدَهُ حِجَارَةٌ أَعْطَاهَا لِخَزِينَةِ بَيْتِ الرَّبِّ عَنْ يَدِ يَحِيئِيلَ الْجَرْشُونِيِّ.
\par 9 وَفَرِحَ الشَّعْبُ بِانْتِدَابِهِمْ, لأَنَّهُمْ بِقَلْبٍ كَامِلٍ انْتَدَبُوا لِلرَّبِّ. وَدَاوُدُ الْمَلِكُ أَيْضاً فَرِحَ فَرَحاً عَظِيماً.
\par 10 وَبَارَكَ دَاوُدُ الرَّبَّ أَمَامَ كُلِّ الْجَمَاعَةِ, وَقَالَ: «مُبَارَكٌ أَنْتَ أَيُّهَا الرَّبُّ إِلَهُ إِسْرَائِيلَ أَبِينَا مِنَ الأَزَلِ وَإِلَى الأَبَدِ.
\par 11 لَكَ يَا رَبُّ الْعَظَمَةُ وَالْجَبَرُوتُ وَالْجَلاَلُ وَالْبَهَاءُ وَالْمَجْدُ, لأَنَّ لَكَ كُلَّ مَا فِي السَّمَاءِ وَالأَرْضِ. لَكَ يَا رَبُّ الْمُلْكُ, وَقَدِ ارْتَفَعْتَ رَأْساً عَلَى الْجَمِيعِ.
\par 12 وَالْغِنَى وَالْكَرَامَةُ مِنْ لَدُنْكَ, وَأَنْتَ تَتَسَلَّطُ عَلَى الْجَمِيعِ, وَبِيَدِكَ الْقُوَّةُ وَالْجَبَرُوتُ, وَبِيَدِكَ تَعْظِيمُ وَتَشْدِيدُ الْجَمِيعِ.
\par 13 وَالآنَ يَا إِلَهَنَا نَحْمَدُكَ وَنُسَبِّحُ اسْمَكَ الْجَلِيلَ.
\par 14 وَلَكِنْ مَنْ أَنَا وَمَنْ هُوَ شَعْبِي حَتَّى نَسْتَطِيعُ أَنْ نَتَبَرَّعَ هَكَذَا, لأَنَّ مِنْكَ الْجَمِيعَ وَمِنْ يَدِكَ أَعْطَيْنَاكَ!
\par 15 لأَنَّنَا نَحْنُ غُرَبَاءُ أَمَامَكَ, وَنُزَلاَءُ مِثْلُ كُلِّ آبَائِنَا. أَيَّامُنَا كَالظِّلِّ عَلَى الأَرْضِ وَلَيْسَ رَجَاءٌ.
\par 16 أَيُّهَا الرَّبُّ إِلَهُنَا, كُلُّ هَذِهِ الثَّرْوَةِ الَّتِي هَيَّأْنَاهَا لِنَبْنِيَ لَكَ بَيْتاً لاِسْمِ قُدْسِكَ إِنَّمَا هِيَ مِنْ يَدِكَ, وَلَكَ الْكُلُّ.
\par 17 وَقَدْ عَلِمْتُ يَا إِلَهِي أَنَّكَ أَنْتَ تَمْتَحِنُ الْقُلُوبَ وَتُسَرُّ بِالاِسْتِقَامَةِ. أَنَا بِاسْتِقَامَةِ قَلْبِي تَبَرَّعْتُ بِكُلِّ هَذِهِ, وَالآنَ شَعْبُكَ الْمَوْجُودُ هُنَا رَأَيْتُهُ بِفَرَحٍ يَتَبَرَّعُ لَكَ.
\par 18 يَا رَبُّ إِلَهَ إِبْرَاهِيمَ وَإِسْحَاقَ وَإِسْرَائِيلَ آبَائِنَا, احْفَظْ هَذِهِ إِلَى الأَبَدِ فِي تَصَوُّرِ أَفْكَارِ قُلُوبِ شَعْبِكَ, وَأَعِدَّ قُلُوبَهُمْ نَحْوَكَ.
\par 19 وَأَمَّا سُلَيْمَانُ ابْنِي فَأَعْطِهِ قَلْباً كَامِلاً لِيَحْفَظَ وَصَايَاكَ, شَهَادَاتِكَ وَفَرَائِضَكَ, وَلِيَعْمَلَ الْجَمِيعَ, وَلِيَبْنِيَ الْهَيْكَلَ الَّذِي هَيَّأْتُ لَهُ».
\par 20 ثُمَّ قَالَ دَاوُدُ لِكُلِّ الْجَمَاعَةِ: «بَارِكُوا الرَّبَّ إِلَهَكُمْ». فَبَارَكَ كُلُّ الْجَمَاعَةِ الرَّبَّ إِلَهَ آبَائِهِمْ, وَخَرُّوا وَسَجَدُوا لِلرَّبِّ وَلِلْمَلِكِ.
\par 21 وَذَبَحُوا لِلرَّبِّ ذَبَائِحَ وَأَصْعَدُوا مُحْرَقَاتٍ لِلرَّبِّ فِي غَدِ ذَلِكَ الْيَوْمِ: أَلْفَ ثَوْرٍ وَأَلْفَ كَبْشٍ وَأَلْفَ خَرُوفٍ مَعَ سَكَائِبِهَا, وَذَبَائِحَ كَثِيرَةً لِكُلِّ إِسْرَائِيلَ.
\par 22 وَأَكَلُوا وَشَرِبُوا أَمَامَ الرَّبِّ فِي ذَلِكَ الْيَوْمِ بِفَرَحٍ عَظِيمٍ. وَمَلَّكُوا ثَانِيَةً سُلَيْمَانَ بْنَ دَاوُدَ, وَمَسَحُوهُ لِلرَّبِّ رَئِيساً, وَصَادُوقَ كَاهِناً.
\par 23 وَجَلَسَ سُلَيْمَانُ عَلَى كُرْسِيِّ الرَّبِّ مَلِكاً مَكَانَ دَاوُدَ أَبِيهِ, وَنَجَحَ وَأَطَاعَهُ كُلُّ إِسْرَائِيلَ.
\par 24 وَجَمِيعُ الرُّؤَسَاءِ وَالأَبْطَالِ وَجَمِيعُ أَوْلاَدِ الْمَلِكِ دَاوُدَ أَيْضاً خَضَعُوا لِسُلَيْمَانَ الْمَلِكِ.
\par 25 وَعَظَّمَ الرَّبُّ سُلَيْمَانَ جِدّاً فِي أَعْيُنِ جَمِيعِ إِسْرَائِيلَ, وَجَعَلَ عَلَيْهِ جَلاَلاً مَلِكِيّاً لَمْ يَكُنْ عَلَى مَلِكٍ قَبْلَهُ فِي إِسْرَائِيلَ.
\par 26 وَدَاوُدُ بْنُ يَسَّى مَلَكَ عَلَى كُلِّ إِسْرَائِيلَ.
\par 27 وَالزَّمَانُ الَّذِي مَلَكَ فِيهِ عَلَى إِسْرَائِيلَ أَرْبَعُونَ سَنَةً. مَلَكَ سَبْعَ سِنِينَ فِي حَبْرُونَ, وَمَلَكَ ثَلاَثاً وَثَلاَثِينَ سَنَةً فِي أُورُشَلِيمَ.
\par 28 وَمَاتَ بِشَيْبَةٍ صَالِحَةٍ وَقَدْ شَبِعَ أَيَّاماً وَغِنىً وَكَرَامَةً. وَمَلَكَ سُلَيْمَانُ ابْنُهُ مَكَانَهُ.
\par 29 وَأُمُورُ دَاوُدَ الْمَلِكِ الأُولَى وَالأَخِيرَةُ مَكْتُوبَةٌ فِي سِفْرِ أَخْبَارِ صَمُوئِيلَ الرَّائِي, وَأَخْبَارِ نَاثَانَ النَّبِيِّ, وَأَخْبَارِ جَادَ الرَّائِي,
\par 30 مَعَ كُلِّ مُلْكِهِ وَجَبَرُوتِهِ وَالأَوْقَاتِ الَّتِي عَبَرَتْ عَلَيْهِ وَعَلَى إِسْرَائِيلَ وَعَلَى كُلِّ مَمَالِكِ الْبِلاَدِ.


\end{document}