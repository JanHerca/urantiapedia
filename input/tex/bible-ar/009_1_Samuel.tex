\begin{document}

\title{صموئيل1}


\chapter{1}

\par 1 كَانَ رَجُلٌ مِنْ رَامَتَايِمِ صُوفِيمَ مِنْ جَبَلِ أَفْرَايِمَ اسْمُهُ أَلْقَانَةُ بْنُ يَرُوحَامَ بْنِ أَلِيهُوَ بْنِ تُوحُوَ بْنِ صُوفٍ. هُوَ أَفْرَايِمِيٌّ.
\par 2 وَلَهُ امْرَأَتَانِ, اسْمُ الْوَاحِدَةِ حَنَّةُ وَاسْمُ الأُخْرَى فَنِنَّةُ. وَكَانَ لِفَنِنَّةَ أَوْلاَدٌ, وَأَمَّا حَنَّةُ فَلَمْ يَكُنْ لَهَا أَوْلاَدٌ.
\par 3 وَكَانَ هَذَا الرَّجُلُ يَصْعَدُ مِنْ مَدِينَتِهِ مِنْ سَنَةٍ إِلَى سَنَةٍ لِيَسْجُدَ وَيَذْبَحَ لِرَبِّ الْجُنُودِ فِي شِيلُوهَ. وَكَانَ هُنَاكَ ابْنَا عَالِي: حُفْنِي وَفِينَحَاسُ, كَاهِنَا الرَّبِّ.
\par 4 وَلَمَّا كَانَ الْوَقْتُ وَذَبَحَ أَلْقَانَةُ, أَعْطَى فَنِنَّةَ امْرَأَتَهُ وَجَمِيعَ بَنِيهَا وَبَنَاتِهَا أَنْصِبَةً.
\par 5 وَأَمَّا حَنَّةُ فَأَعْطَاهَا نَصِيبَ اثْنَيْنِ, لأَنَّهُ كَانَ يُحِبُّ حَنَّةَ. وَلَكِنَّ الرَّبَّ كَانَ قَدْ أَغْلَقَ رَحِمَهَا.
\par 6 وَكَانَتْ ضَرَّتُهَا تُغِيظُهَا أَيْضاً غَيْظاً لأَجْلِ الْمُرَغَمَةِ, لأَنَّ الرَّبَّ أَغْلَقَ رَحِمَهَا.
\par 7 وَهَكَذَا صَارَ سَنَةً بَعْدَ سَنَةٍ, كُلَّمَا صَعِدَتْ إِلَى بَيْتِ الرَّبِّ, هَكَذَا كَانَتْ تُغِيظُهَا. فَبَكَتْ وَلَمْ تَأْكُلْ.
\par 8 فَقَالَ لَهَا أَلْقَانَةُ رَجُلُهَا: «يَا حَنَّةُ, لِمَاذَا تَبْكِينَ وَلِمَاذَا لاَ تَأْكُلِينَ وَلِمَاذَا يَكْتَئِبُ قَلْبُكِ؟ أَمَا أَنَا خَيْرٌ لَكِ مِنْ عَشَرَةِ بَنِينَ؟».
\par 9 فَقَامَتْ حَنَّةُ بَعْدَمَا أَكَلُوا فِي شِيلُوهَ وَبَعْدَمَا شَرِبُوا, وَعَالِي الْكَاهِنُ جَالِسٌ عَلَى الْكُرْسِيِّ عِنْدَ قَائِمَةِ هَيْكَلِ الرَّبِّ,
\par 10 وَهِيَ مُرَّةُ النَّفْسِ. فَصَلَّتْ إِلَى الرَّبِّ, وَبَكَتْ بُكَاءً
\par 11 وَنَذَرَتْ نَذْراً وَقَالَتْ: «يَا رَبَّ الْجُنُودِ, إِنْ نَظَرْتَ نَظَراً إِلَى مَذَلَّةِ أَمَتِكَ, وَذَكَرْتَنِي وَلَمْ تَنْسَ أَمَتَكَ بَلْ أَعْطَيْتَ أَمَتَكَ زَرْعَ بَشَرٍ, فَإِنِّي أُعْطِيهِ لِلرَّبِّ كُلَّ أَيَّامِ حَيَاتِهِ, وَلاَ يَعْلُو رَأْسَهُ مُوسَى».
\par 12 وَكَانَ إِذْ أَكْثَرَتِ الصَّلاَةَ أَمَامَ الرَّبِّ وَعَالِي يُلاَحِظُ فَاهَا -
\par 13 فَإِنَّ حَنَّةَ كَانَتْ تَتَكَلَّمُ فِي قَلْبِهَا, وَشَفَتَاهَا فَقَطْ تَتَحَرَّكَانِ, وَصَوْتُهَا لَمْ يُسْمَعْ - أَنَّ عَالِيَ ظَنَّهَا سَكْرَى.
\par 14 فَقَالَ لَهَا: «حَتَّى مَتَى تَسْكَرِينَ؟ انْزِعِي خَمْرَكِ عَنْكِ».
\par 15 فَأَجَابَتْ حَنَّةُ: «لاَ يَا سَيِّدِي. إِنِّي امْرَأَةٌ حَزِينَةُ الرُّوحِ وَلَمْ أَشْرَبْ خَمْراً وَلاَ مُسْكِراً, بَلْ أَسْكُبُ نَفْسِي أَمَامَ الرَّبِّ.
\par 16 لاَ تَحْسِبْ أَمَتَكَ ابْنَةَ بَلِيَّعَالَ. لأَنِّي مِنْ كَثْرَةِ كُرْبَتِي وَغَيْظِي قَدْ تَكَلَّمْتُ إِلَى الآنَ».
\par 17 فَقَالَ لَهَا عَالِي: «اذْهَبِي بِسَلاَمٍ, وَإِلَهُ إِسْرَائِيلَ يُعْطِيكِ سُؤْلَكِ الَّذِي سَأَلْتِهِ مِنْ لَدُنْهُ».
\par 18 فَقَالَتْ: «لِتَجِدْ جَارِيَتُكَ نِعْمَةً فِي عَيْنَيْكَ». ثُمَّ مَضَتِ الْمَرْأَةُ فِي طَرِيقِهَا وَأَكَلَتْ, وَلَمْ يَكُنْ وَجْهُهَا بَعْدُ مُغَيَّراً.
\par 19 وَبَكَّرُوا فِي الصَّبَاحِ وَسَجَدُوا أَمَامَ الرَّبِّ, وَرَجَعُوا وَجَاءُوا إِلَى بَيْتِهِمْ فِي الرَّامَةِ. وَعَرَفَ أَلْقَانَةُ امْرَأَتَهُ حَنَّةَ, وَالرَّبُّ ذَكَرَهَا.
\par 20 وَكَانَ فِي مَدَارِ السَّنَةِ أَنَّ حَنَّةَ حَبِلَتْ وَوَلَدَتِ ابْناً وَدَعَتِ اسْمَهُ صَمُوئِيلَ قَائِلَةً: «لأَنِّي مِنَ الرَّبِّ سَأَلْتُهُ».
\par 21 وَصَعِدَ أَلْقَانَةُ وَجَمِيعُ بَيْتِهِ لِيَذْبَحَ لِلرَّبِّ الذَّبِيحَةَ السَّنَوِيَّةَ, وَنَذْرَهُ.
\par 22 وَلَكِنَّ حَنَّةَ لَمْ تَصْعَدْ لأَنَّهَا قَالَتْ لِرَجُلِهَا: «مَتَى فُطِمَ الصَّبِيُّ آتِي بِهِ لِيَتَرَاءَى أَمَامَ الرَّبِّ وَيُقِيمَ هُنَاكَ إِلَى الأَبَدِ».
\par 23 فَقَالَ لَهَا أَلْقَانَةُ رَجُلُهَا: «اعْمَلِي مَا يَحْسُنُ فِي عَيْنَيْكِ. امْكُثِي حَتَّى تَفْطِمِيهِ. إِنَّمَا الرَّبُّ يُقِيمُ كَلاَمَهُ». فَمَكَثَتِ الْمَرْأَةُ وَأَرْضَعَتِ ابْنَهَا حَتَّى فَطَمَتْهُ.
\par 24 ثُمَّ حِينَ فَطَمَتْهُ أَصْعَدَتْهُ مَعَهَا بِثَلاَثَةِ ثِيرَانٍ وَإِيفَةِ دَقِيقٍ وَزِقِّ خَمْرٍ, وَأَتَتْ بِهِ إِلَى الرَّبِّ فِي شِيلُوهَ وَالصَّبِيُّ صَغِيرٌ.
\par 25 فَذَبَحُوا الثَّوْرَ وَجَاءُوا بِالصَّبِيِّ إِلَى عَالِي.
\par 26 وَقَالَتْ: «أَسْأَلُكَ يَا سَيِّدِي. حَيَّةٌ هِيَ نَفْسُكَ يَا سَيِّدِي, أَنَا الْمَرْأَةُ الَّتِي وَقَفَتْ لَدَيْكَ هُنَا تُصَلِّي إِلَى الرَّبِّ.
\par 27 لأَجْلِ هَذَا الصَّبِيِّ صَلَّيْتُ فَأَعْطَانِيَ الرَّبُّ سُؤْلِيَ الَّذِي سَأَلْتُهُ مِنْ لَدُنْهُ.
\par 28 وَأَنَا أَيْضاً قَدْ أَعَرْتُهُ لِلرَّبِّ. جَمِيعَ أَيَّامِ حَيَاتِهِ هُوَ مُعَارٌ لِلرَّبِّ». وَسَجَدُوا هُنَاكَ لِلرَّبِّ.

\chapter{2}

\par 1 فَصَلَّتْ حَنَّةُ: «فَرِحَ قَلْبِي بِالرَّبِّ. ارْتَفَعَ قَرْنِي بِالرَّبِّ. اتَّسَعَ فَمِي عَلَى أَعْدَائِي, لأَنِّي قَدِ ابْتَهَجْتُ بِخَلاَصِكَ.
\par 2 لَيْسَ قُدُّوسٌ مِثْلَ الرَّبِّ, لأَنَّهُ لَيْسَ غَيْرَكَ, وَلَيْسَ صَخْرَةٌ مِثْلَ إِلَهِنَا.
\par 3 لاَ تُكَثِّرُوا الْكَلاَمَ الْعَالِيَ الْمُسْتَعْلِيَ, وَلْتَبْرَحْ وَقَاحَةٌ مِنْ أَفْوَاهِكُمْ. لأَنَّ الرَّبَّ إِلَهٌ عَلِيمٌ, وَبِهِ تُوزَنُ الأَعْمَالُ.
\par 4 قِسِيُّ الْجَبَابِرَةِ انْحَطَمَتْ وَالضُّعَفَاءُ تَمَنْطَقُوا بِالْبَأْسِ.
\par 5 الشَّبَاعَى آجَرُوا أَنْفُسَهُمْ بِالْخُبْزِ, وَالْجِيَاعُ كَفُّوا. حَتَّى أَنَّ الْعَاقِرَ وَلَدَتْ سَبْعَةً, وَكَثِيرَةَ الْبَنِينَ ذَبُلَتْ.
\par 6 الرَّبُّ يُمِيتُ وَيُحْيِي. يُهْبِطُ إِلَى الْهَاوِيَةِ وَيُصْعِدُ.
\par 7 الرَّبُّ يُفْقِرُ وَيُغْنِي. يَضَعُ وَيَرْفَعُ.
\par 8 يُقِيمُ الْمِسْكِينَ مِنَ التُّرَابِ. يَرْفَعُ الْفَقِيرَ مِنَ الْمَزْبَلَةِ لِلْجُلُوسِ مَعَ الشُّرَفَاءِ وَيُمَلِّكُهُمْ كُرْسِيَّ الْمَجْدِ. لأَنَّ لِلرَّبِّ أَعْمِدَةَ الأَرْضِ, وَقَدْ وَضَعَ عَلَيْهَا الْمَسْكُونَةَ.
\par 9 أَرْجُلَ أَتْقِيَائِهِ يَحْرُسُ, وَالأَشْرَارُ فِي الظَّلاَمِ يَصْمُتُونَ. لأَنَّهُ لَيْسَ بِالْقُوَّةِ يَغْلِبُ إِنْسَانٌ.
\par 10 مُخَاصِمُو الرَّبِّ يَنْكَسِرُونَ. مِنَ السَّمَاءِ يُرْعِدُ عَلَيْهِمْ. الرَّبُّ يَدِينُ أَقَاصِيَ الأَرْضِ, وَيُعْطِي عِزّاً لِمَلِكِهِ, وَيَرْفَعُ قَرْنَ مَسِيحِهِ».
\par 11 وَذَهَبَ أَلْقَانَةُ إِلَى الرَّامَةِ إِلَى بَيْتِهِ. وَكَانَ الصَّبِيُّ يَخْدِمُ الرَّبَّ أَمَامَ عَالِي الْكَاهِنِ.
\par 12 وَكَانَ بَنُو عَالِي بَنِي بَلِيَّعَالَ, لَمْ يَعْرِفُوا الرَّبَّ
\par 13 وَلاَ حَقَّ الْكَهَنَةِ مِنَ الشَّعْبِ. كُلَّمَا ذَبَحَ رَجُلٌ ذَبِيحَةً يَجِيءُ غُلاَمُ الْكَاهِنِ عِنْدَ طَبْخِ اللَّحْمِ, وَمِنْشَالٌ ذُو ثَلاَثَةِ أَسْنَانٍ بِيَدِهِ,
\par 14 فَيَضْرِبُ فِي الْمِرْحَضَةِ أَوِ الْمِرْجَلِ أَوِ الْمِقْلَى أَوِ الْقِدْرِ - كُلُّ مَا يَصْعَدُ بِهِ الْمِنْشَلُ يَأْخُذُهُ الْكَاهِنُ لِنَفْسِهِ. هَكَذَا كَانُوا يَفْعَلُونَ بِجَمِيعِ إِسْرَائِيلَ الآتِينَ إِلَى هُنَاكَ فِي شِيلُوهَ.
\par 15 كَذَلِكَ قَبْلَ مَا يُحْرِقُونَ الشَّحْمَ يَأْتِي غُلاَمُ الْكَاهِنِ وَيَقُولُ لِلرَّجُلِ الذَّابِحِ: «أَعْطِ لَحْماً لِيُشْوَى لِلْكَاهِنِ, فَإِنَّهُ لاَ يَأْخُذُ مِنْكَ لَحْماً مَطْبُوخاً بَلْ نَيْئاً».
\par 16 فَيَقُولُ لَهُ الرَّجُلُ: «لِيُحْرِقُوا أَوَّلاً الشَّحْمَ, ثُمَّ خُذْ مَا تَشْتَهِيهِ نَفْسُكَ». فَيَقُولُ لَهُ: «لاَ, بَلِ الآنَ تُعْطِي وَإِلَّا فَآخُذُ غَصْباً».
\par 17 فَكَانَتْ خَطِيَّةُ الْغِلْمَانِ عَظِيمَةً جِدّاً أَمَامَ الرَّبِّ, لأَنَّ النَّاسَ اسْتَهَانُوا بِتَقْدِمَةِ الرَّبِّ.
\par 18 وَكَانَ صَمُوئِيلُ يَخْدِمُ أَمَامَ الرَّبِّ وَهُوَ صَبِيٌّ مُتَمَنْطِقٌ بِأَفُودٍ مِنْ كَتَّانٍ.
\par 19 وَعَمِلَتْ لَهُ أُمُّهُ جُبَّةً صَغِيرَةً وَأَصْعَدَتْهَا لَهُ مِنْ سَنَةٍ إِلَى سَنَةٍ عِنْدَ صُعُودِهَا مَعَ رَجُلِهَا لِذَبْحِ الذَّبِيحَةِ السَّنَوِيَّةِ.
\par 20 وَبَارَكَ عَالِي أَلْقَانَةَ وَامْرَأَتَهُ وَقَالَ: «يَجْعَلْ لَكَ الرَّبُّ نَسْلاً مِنْ هَذِهِ الْمَرْأَةِ بَدَلَ الْعَارِيَّةِ الَّتِي أَعَارَتْ لِلرَّبِّ». وَذَهَبَا إِلَى مَكَانِهِمَا.
\par 21 وَلَمَّا افْتَقَدَ الرَّبُّ حَنَّةَ حَبِلَتْ وَوَلَدَتْ ثَلاَثَةَ بَنِينَ وَبِنْتَيْنِ. وَكَبِرَ الصَّبِيُّ صَمُوئِيلُ عِنْدَ الرَّبِّ.
\par 22 وَشَاخَ عَالِي جِدّاً, وَسَمِعَ بِكُلِّ مَا عَمِلَهُ بَنُوهُ بِجَمِيعِ إِسْرَائِيلَ وَبِأَنَّهُمْ كَانُوا يُضَاجِعُونَ النِّسَاءَ الْمُجْتَمِعَاتِ فِي بَابِ خَيْمَةِ الاِجْتِمَاعِ.
\par 23 فَقَالَ لَهُمْ: «لِمَاذَا تَعْمَلُونَ مِثْلَ هَذِهِ الأُمُورِ؟ لأَنِّي أَسْمَعُ بِأُمُورِكُمُ الْخَبِيثَةِ مِنْ جَمِيعِ هَذَا الشَّعْبِ.
\par 24 لاَ يَا بَنِيَّ, لأَنَّهُ لَيْسَ حَسَناً الْخَبَرُ الَّذِي أَسْمَعُ. تَجْعَلُونَ شَعْبَ الرَّبِّ يَتَعَدُّونَ.
\par 25 إِذَا أَخْطَأَ إِنْسَانٌ إِلَى إِنْسَانٍ يَدِينُهُ اللَّهُ. فَإِنْ أَخْطَأَ إِنْسَانٌ إِلَى الرَّبِّ فَمَنْ يُصَلِّي مِنْ أَجْلِهِ؟» وَلَمْ يَسْمَعُوا لِصَوْتِ أَبِيهِمْ لأَنَّ الرَّبَّ شَاءَ أَنْ يُمِيتَهُمْ.
\par 26 وَأَمَّا الصَّبِيُّ صَمُوئِيلُ فَتَزَايَدَ نُمُوّاً وَصَلاَحاً لَدَى الرَّبِّ وَالنَّاسِ أَيْضاً.
\par 27 وَجَاءَ رَجُلُ اللَّهِ إِلَى عَالِي وَقَالَ لَهُ: «هَكَذَا يَقُولُ الرَّبُّ: هَلْ تَجَلَّيْتُ لِبَيْتِ أَبِيكَ وَهُمْ فِي مِصْرَ فِي بَيْتِ فِرْعَوْنَ,
\par 28 وَانْتَخَبْتُهُ مِنْ جَمِيعِ أَسْبَاطِ إِسْرَائِيلَ لِي كَاهِناً لِيَصْعَدَ عَلَى مَذْبَحِي وَيُوقِدَ بَخُوراً وَيَلْبَسَ أَفُوداً أَمَامِي, وَدَفَعْتُ لِبَيْتِ أَبِيكَ جَمِيعَ وَقَائِدِ بَنِي إِسْرَائِيلَ!
\par 29 فَلِمَاذَا تَدُوسُونَ ذَبِيحَتِي وَتَقْدِمَتِي الَّتِي أَمَرْتُ بِهَا فِي الْمَسْكَنِ, وَتُكْرِمُ بَنِيكَ عَلَيَّ لِتُسَمِّنُوا أَنْفُسَكُمْ بِأَوَائِلِ كُلِّ تَقْدِمَاتِ إِسْرَائِيلَ شَعْبِي؟
\par 30 لِذَلِكَ يَقُولُ الرَّبُّ إِلَهُ إِسْرَائِيلَ: إِنِّي قُلْتُ إِنَّ بَيْتَكَ وَبَيْتَ أَبِيكَ يَسِيرُونَ أَمَامِي إِلَى الأَبَدِ. وَالآنَ يَقُولُ الرَّبُّ: حَاشَا لِي! فَإِنِّي أُكْرِمُ الَّذِينَ يُكْرِمُونَنِي, وَالَّذِينَ يَحْتَقِرُونَنِي يَصْغُرُونَ.
\par 31 هُوَذَا تَأْتِي أَيَّامٌ أَقْطَعُ فِيهَا ذِرَاعَكَ وَذِرَاعَ بَيْتِ أَبِيكَ حَتَّى لاَ يَكُونَ شَيْخٌ فِي بَيْتِكَ.
\par 32 وَتَرَى ضِيقَ الْمَسْكَنِ فِي كُلِّ مَا يُحْسَنُ بِهِ إِلَى إِسْرَائِيلَ, وَلاَ يَكُونُ شَيْخٌ فِي بَيْتِكَ كُلَّ الأَيَّامِ.
\par 33 وَرَجُلٌ لَكَ لاَ أَقْطَعُهُ مِنْ أَمَامِ مَذْبَحِي يَكُونُ لإِكْلاَلِ عَيْنَيْكَ وَتَذْوِيبِ نَفْسِكَ. وَجَمِيعُ ذُرِّيَّةِ بَيْتِكَ يَمُوتُونَ شُبَّاناً.
\par 34 وَهَذِهِ لَكَ عَلاَمَةٌ تَأْتِي عَلَى ابْنَيْكَ حُفْنِي وَفِينَحَاسَ: فِي يَوْمٍ وَاحِدٍ يَمُوتَانِ كِلاَهُمَا.
\par 35 وَأُقِيمُ لِنَفْسِي كَاهِناً أَمِيناً يَعْمَلُ حَسَبَ مَا بِقَلْبِي وَنَفْسِي, وَأَبْنِي لَهُ بَيْتاً أَمِيناً فَيَسِيرُ أَمَامَ مَسِيحِي كُلَّ الأَيَّامِ.
\par 36 وَيَكُونُ أَنَّ كُلَّ مَنْ يَبْقَى فِي بَيْتِكَ يَأْتِي لِيَسْجُدَ لَهُ لأَجْلِ قِطْعَةِ فِضَّةٍ وَرَغِيفِ خُبْزٍ, وَيَقُولُ: ضُمَّنِي إِلَى إِحْدَى وَظَائِفِ الْكَهَنُوتِ لِآكُلَ كِسْرَةَ خُبْزٍ».

\chapter{3}

\par 1 وَكَانَ الصَّبِيُّ صَمُوئِيلُ يَخْدِمُ الرَّبَّ أَمَامَ عَالِي. وَكَانَتْ كَلِمَةُ الرَّبِّ عَزِيزَةً فِي تِلْكَ الأَيَّامِ. لَمْ تَكُنْ رُؤْيَا كَثِيراً.
\par 2 وَكَانَ فِي ذَلِكَ الزَّمَانِ إِذْ كَانَ عَالِي مُضْطَجِعاً فِي مَكَانِهِ وَعَيْنَاهُ ابْتَدَأَتَا تَضْعُفَانِ - لَمْ يَقْدِرْ أَنْ يُبْصِرَ.
\par 3 وَقَبْلَ أَنْ يَنْطَفِئَ سِرَاجُ اللَّهِ, وَصَمُوئِيلُ مُضْطَجِعٌ فِي هَيْكَلِ الرَّبِّ الَّذِي فِيهِ تَابُوتُ اللَّهِ,
\par 4 أَنَّ الرَّبَّ دَعَا صَمُوئِيلَ, فَقَالَ: «هَئَنَذَا».
\par 5 وَرَكَضَ إِلَى عَالِي وَقَالَ: «هَئَنَذَا لأَنَّكَ دَعَوْتَنِي». فَقَالَ: «لَمْ أَدْعُ. ارْجِعِ اضْطَجِعْ». فَذَهَبَ وَاضْطَجَعَ.
\par 6 ثُمَّ عَادَ الرَّبُّ وَدَعَا أَيْضاً صَمُوئِيلَ. فَقَامَ صَمُوئِيلُ وَذَهَبَ إِلَى عَالِي وَقَالَ: «هَئَنَذَا لأَنَّكَ دَعَوْتَنِي». فَقَالَ: «لَمْ أَدْعُ يَا ابْنِي. ارْجِعِ اضْطَجِعْ».
\par 7 (وَلَمْ يَعْرِفْ صَمُوئِيلُ الرَّبَّ بَعْدُ, وَلاَ أُعْلِنَ لَهُ كَلاَمُ الرَّبِّ بَعْدُ).
\par 8 وَعَادَ الرَّبُّ فَدَعَا صَمُوئِيلَ ثَالِثَةً. فَقَامَ وَذَهَبَ إِلَى عَالِي وَقَالَ: «هَئَنَذَا لأَنَّكَ دَعَوْتَنِي». فَفَهِمَ عَالِي أَنَّ الرَّبَّ يَدْعُو الصَّبِيَّ.
\par 9 فَقَالَ عَالِي لِصَمُوئِيلَ: «اذْهَبِ اضْطَجِعْ, وَيَكُونُ إِذَا دَعَاكَ تَقُولُ: تَكَلَّمْ يَا رَبُّ لأَنَّ عَبْدَكَ سَامِعٌ». فَذَهَبَ صَمُوئِيلُ وَاضْطَجَعَ فِي مَكَانِهِ.
\par 10 فَجَاءَ الرَّبُّ وَوَقَفَ وَدَعَا كَالْمَرَّاتِ الأُوَلِ: «صَمُوئِيلُ صَمُوئِيلُ». فَقَالَ صَمُوئِيلُ: «تَكَلَّمْ لأَنَّ عَبْدَكَ سَامِعٌ».
\par 11 فَقَالَ الرَّبُّ لِصَمُوئِيلَ: «هُوَذَا أَنَا فَاعِلٌ أَمْراً فِي إِسْرَائِيلَ كُلُّ مَنْ سَمِعَ بِهِ تَطِنُّ أُذُنَاهُ.
\par 12 فِي ذَلِكَ الْيَوْمِ أُقِيمُ عَلَى عَالِي كُلَّ مَا تَكَلَّمْتُ بِهِ عَلَى بَيْتِهِ. أَبْتَدِئُ وَأُكَمِّلُ.
\par 13 وَقَدْ أَخْبَرْتُهُ بِأَنِّي أَقْضِي عَلَى بَيْتِهِ إِلَى الأَبَدِ مِنْ أَجْلِ الشَّرِّ الَّذِي يَعْلَمُ أَنَّ بَنِيهِ قَدْ أَوْجَبُوا بِهِ اللَّعْنَةَ عَلَى أَنْفُسِهِمْ, وَلَمْ يَرْدَعْهُمْ.
\par 14 وَلِذَلِكَ أَقْسَمْتُ لِبَيْتِ عَالِي أَنَّهُ لاَ يُكَفَّرُ عَنْ شَرِّ بَيْتِ عَالِي بِذَبِيحَةٍ أَوْ بِتَقْدِمَةٍ إِلَى الأَبَدِ».
\par 15 وَاضْطَجَعَ صَمُوئِيلُ إِلَى الصَّبَاحِ, وَفَتَحَ أَبْوَابَ بَيْتِ الرَّبِّ. وَخَافَ صَمُوئِيلُ أَنْ يُخْبِرَ عَالِيَ بِالرُّؤْيَا.
\par 16 فَدَعَا عَالِي صَمُوئِيلَ وَقَالَ: «يَا صَمُوئِيلُ ابْنِي» فَقَالَ: «هَئَنَذَا».
\par 17 فَقَالَ: «مَا الْكَلاَمُ الَّذِي كَلَّمَكَ بِهِ؟ لاَ تُخْفِ عَنِّي. هَكَذَا يَعْمَلُ لَكَ اللَّهُ وَهَكَذَا يَزِيدُ إِنْ أَخْفَيْتَ عَنِّي كَلِمَةً مِنْ كُلِّ الْكَلاَمِ الَّذِي كَلَّمَكَ بِهِ».
\par 18 فَأَخْبَرَهُ صَمُوئِيلُ بِجَمِيعِ الْكَلاَمِ وَلَمْ يُخْفِ عَنْهُ. فَقَالَ: «هُوَ الرَّبُّ. مَا يَحْسُنُ فِي عَيْنَيْهِ يَعْمَلُ».
\par 19 وَكَبِرَ صَمُوئِيلُ وَكَانَ الرَّبُّ مَعَهُ, وَلَمْ يَدَعْ شَيْئاً مِنْ جَمِيعِ كَلاَمِهِ يَسْقُطُ إِلَى الأَرْضِ.
\par 20 وَعَرَفَ جَمِيعُ إِسْرَائِيلَ مِنْ دَانَ إِلَى بِئْرِ سَبْعٍ أَنَّهُ قَدِ اؤْتُمِنَ صَمُوئِيلُ نَبِيّاً لِلرَّبِّ.
\par 21 وَعَادَ الرَّبُّ يَتَرَاءَى فِي شِيلُوهَ, لأَنَّ الرَّبَّ اسْتَعْلَنَ لِصَمُوئِيلَ فِي شِيلُوهَ بِكَلِمَةِ الرَّبِّ.

\chapter{4}

\par 1 وَكَانَ كَلاَمُ صَمُوئِيلَ إِلَى جَمِيعِ إِسْرَائِيلَ. وَخَرَجَ إِسْرَائِيلُ لِلِقَاءِ الْفِلِسْطِينِيِّينَ لِلْحَرْبِ وَنَزَلُوا عِنْدَ حَجَرِ الْمَعُونَةِ, وَأَمَّا الْفِلِسْطِينِيُّونَ فَنَزَلُوا فِي أَفِيقَ.
\par 2 وَاصْطَفَّ الْفِلِسْطِينِيُّونَ لِلِقَاءِ إِسْرَائِيلَ, وَاشْتَبَكَتِ الْحَرْبُ فَانْكَسَرَ إِسْرَائِيلُ أَمَامَ الْفِلِسْطِينِيِّينَ, وَضَرَبُوا مِنَ الصَّفِّ فِي الْحَقْلِ نَحْوَ أَرْبَعَةِ آلاَفِ رَجُلٍ.
\par 3 فَجَاءَ الشَّعْبُ إِلَى الْمَحَلَّةِ. وَقَالَ شُيُوخُ إِسْرَائِيلَ: «لِمَاذَا كَسَّرَنَا الْيَوْمَ الرَّبُّ أَمَامَ الْفِلِسْطِينِيِّينَ؟ لِنَأْخُذْ لأَنْفُسِنَا مِنْ شِيلُوهَ تَابُوتَ عَهْدِ الرَّبِّ فَيَدْخُلَ فِي وَسَطِنَا وَيُخَلِّصَنَا مِنْ يَدِ أَعْدَائِنَا».
\par 4 فَأَرْسَلَ الشَّعْبُ إِلَى شِيلُوهَ وَحَمَلُوا مِنْ هُنَاكَ تَابُوتَ عَهْدِ رَبِّ الْجُنُودِ الْجَالِسِ عَلَى الْكَرُوبِيمِ. وَكَانَ هُنَاكَ ابْنَا عَالِي حُفْنِي وَفِينَحَاسُ مَعَ تَابُوتِ عَهْدِ اللَّهِ.
\par 5 وَكَانَ عِنْدَ دُخُولِ تَابُوتِ عَهْدِ الرَّبِّ إِلَى الْمَحَلَّةِ أَنَّ جَمِيعَ إِسْرَائِيلَ هَتَفُوا هُتَافاً عَظِيماً حَتَّى ارْتَجَّتِ الأَرْضُ.
\par 6 فَسَمِعَ الْفِلِسْطِينِيُّونَ صَوْتَ الْهُتَافِ فَقَالُوا: «مَا هُوَ صَوْتُ هَذَا الْهُتَافِ الْعَظِيمِ فِي مَحَلَّةِ الْعِبْرَانِيِّينَ؟» وَعَلِمُوا أَنَّ تَابُوتَ الرَّبِّ جَاءَ إِلَى الْمَحَلَّةِ.
\par 7 فَخَافَ الْفِلِسْطِينِيُّونَ لأَنَّهُمْ قَالُوا: «قَدْ جَاءَ اللَّهُ إِلَى الْمَحَلَّةِ». وَقَالُوا: «وَيْلٌ لَنَا لأَنَّهُ لَمْ يَكُنْ مِثْلُ هَذَا مُنْذُ أَمْسِ وَلاَ مَا قَبْلَهُ!
\par 8 وَيْلٌ لَنَا! مَنْ يُنْقِذُنَا مِنْ يَدِ هَؤُلاَءِ الآلِهَةِ الْقَادِرِينَ؟ هَؤُلاَءِ هُمُ الآلِهَةُ الَّذِينَ ضَرَبُوا مِصْرَ بِجَمِيعِ الضَّرَبَاتِ فِي الْبَرِّيَّةِ.
\par 9 تَشَدَّدُوا وَكُونُوا رِجَالاً أَيُّهَا الْفِلِسْطِينِيُّونَ لِئَلَّا تُسْتَعْبَدُوا لِلْعِبْرَانِيِّينَ كَمَا اسْتُعْبِدُوا هُمْ لَكُمْ. فَكُونُوا رِجَالاً وَحَارِبُوا».
\par 10 فَحَارَبَ الْفِلِسْطِينِيُّونَ, وَانْكَسَرَ إِسْرَائِيلُ وَهَرَبُوا كُلُّ وَاحِدٍ إِلَى خَيْمَتِهِ. وَكَانَتِ الضَّرْبَةُ عَظِيمَةً جِدّاً. وَسَقَطَ مِنْ إِسْرَائِيلَ ثَلاَثُونَ أَلْفَ رَاجِلٍ.
\par 11 وَأُخِذَ تَابُوتُ اللَّهِ. وَمَاتَ ابْنَا عَالِي حُفْنِي وَفِينَحَاسُ.
\par 12 فَرَكَضَ رَجُلٌ مِنْ بِنْيَامِينَ مِنَ الصَّفِّ وَجَاءَ إِلَى شِيلُوهَ فِي ذَلِكَ الْيَوْمِ وَثِيَابُهُ مُمَزَّقَةٌ وَتُرَابٌ عَلَى رَأْسِهِ.
\par 13 وَلَمَّا جَاءَ فَإِذَا عَالِي جَالِسٌ عَلَى كُرْسِيٍّ بِجَانِبِ الطَّرِيقِ يُرَاقِبُ, لأَنَّ قَلْبَهُ كَانَ مُضْطَرِباً لأَجْلِ تَابُوتِ اللَّهِ. وَلَمَّا جَاءَ الرَّجُلُ لِيُخْبِرَ فِي الْمَدِينَةِ صَرَخَتِ الْمَدِينَةُ كُلُّهَا.
\par 14 فَسَمِعَ عَالِي صَوْتَ الصُّرَاخِ فَقَالَ: «مَا هُوَ صَوْتُ الضَّجِيجِ هَذَا؟» فَأَسْرَعَ الرَّجُلُ وَأَخْبَرَ عَالِيَ.
\par 15 وَكَانَ عَالِي ابْنَ ثَمَانٍ وَتِسْعِينَ سَنَةً, وَضَعُفَتْ عَيْنَاهُ وَلَمْ يَقْدِرْ أَنْ يُبْصِرَ.
\par 16 فَقَالَ الرَّجُلُ لِعَالِي: «أَنَا جِئْتُ مِنَ الصَّفِّ, وَأَنَا هَرَبْتُ الْيَوْمَ مِنَ الصَّفِّ». فَقَالَ: «كَيْفَ كَانَ الأَمْرُ يَا ابْنِي؟»
\par 17 فَأَجَابَ الْمُخَبِّرُ: «هَرَبَ إِسْرَائِيلُ أَمَامَ الْفِلِسْطِينِيِّينَ وَكَانَتْ أَيْضاً كَسْرَةٌ عَظِيمَةٌ فِي الشَّعْبِ, وَمَاتَ أَيْضاً ابْنَاكَ حُفْنِي وَفِينَحَاسُ, وَأُخِذَ تَابُوتُ اللَّهِ».
\par 18 وَكَانَ لَمَّا ذَكَرَ تَابُوتَ اللَّهِ أَنَّهُ سَقَطَ عَنِ الْكُرْسِيِّ إِلَى الْوَرَاءِ إِلَى جَانِبِ الْبَابِ, فَانْكَسَرَتْ رَقَبَتُهُ وَمَاتَ - لأَنَّهُ كَانَ رَجُلاً شَيْخاً وَثَقِيلاً. وَقَدْ قَضَى لإِسْرَائِيلَ أَرْبَعِينَ سَنَةً.
\par 19 وَكَنَّتُهُ امْرَأَةُ فِينَحَاسَ كَانَتْ حُبْلَى تَكَادُ تَلِدُ. فَلَمَّا سَمِعَتْ خَبَرَ أَخْذِ تَابُوتِ اللَّهِ وَمَوْتَ حَمِيهَا وَرَجُلِهَا, رَكَعَتْ وَوَلَدَتْ, لأَنَّ مَخَاضَهَا انْقَلَبَ عَلَيْهَا.
\par 20 وَعِنْدَ احْتِضَارِهَا قَالَتْ لَهَا الْوَاقِفَاتُ عِنْدَهَا: «لاَ تَخَافِي لأَنَّكِ قَدْ وَلَدْتِ ابْناً». فَلَمْ تُجِبْ وَلَمْ يُبَالِ قَلْبُهَا.
\par 21 فَدَعَتِ الصَّبِيَّ «إِيخَابُودَ» قَائِلَةً: «قَدْ زَالَ الْمَجْدُ مِنْ إِسْرَائِيلَ!» لأَنَّ تَابُوتَ اللَّهِ قَدْ أُخِذَ وَلأَجْلِ حَمِيهَا وَرَجُلِهَا.
\par 22 فَقَالَتْ: «زَالَ الْمَجْدُ مِنْ إِسْرَائِيلَ لأَنَّ تَابُوتَ اللَّهِ قَدْ أُخِذَ».

\chapter{5}

\par 1 فَأَخَذَ الْفِلِسْطِينِيُّونَ تَابُوتَ اللَّهِ وَأَتُوا بِهِ مِنْ حَجَرِ الْمَعُونَةِ إِلَى أَشْدُودَ.
\par 2 وَأَخَذَ الْفِلِسْطِينِيُّونَ تَابُوتَ اللَّهِ وَأَدْخَلُوهُ إِلَى بَيْتِ دَاجُونَ وَأَقَامُوهُ بِقُرْبِ دَاجُونَ.
\par 3 وَبَكَّرَ الأَشْدُودِيُّونَ فِي الْغَدِ وَإِذَا بِدَاجُونَ سَاقِطٌ عَلَى وَجْهِهِ إِلَى الأَرْضِ أَمَامَ تَابُوتِ الرَّبِّ, فَأَخَذُوا دَاجُونَ وَأَقَامُوهُ فِي مَكَانِهِ.
\par 4 وَبَكَّرُوا صَبَاحاً فِي الْغَدِ وَإِذَا بِدَاجُونَ سَاقِطٌ عَلَى وَجْهِهِ عَلَى الأَرْضِ أَمَامَ تَابُوتِ الرَّبِّ وَرَأْسُ دَاجُونَ وَيَدَاهُ مَقْطُوعَةٌ عَلَى الْعَتَبَةِ. بَقِيَ بَدَنُ السَّمَكَةِ فَقَطْ.
\par 5 لِذَلِكَ لاَ يَدُوسُ كَهَنَةُ دَاجُونَ وَجَمِيعُ الدَّاخِلِينَ إِلَى بَيْتِ دَاجُونَ عَلَى عَتَبَةِ دَاجُونَ فِي أَشْدُودَ إِلَى هَذَا الْيَوْمِ.
\par 6 فَثَقُلَتْ يَدُ الرَّبِّ عَلَى الأَشْدُودِيِّينَ, وَأَخْرَبَهُمْ وَضَرَبَهُمْ بِالْبَوَاسِيرِ فِي أَشْدُودَ وَتُخُومِهَا.
\par 7 وَلَمَّا رَأَى أَهْلُ أَشْدُودَ الأَمْرَ كَذَلِكَ قَالُوا: «لاَ يَمْكُثُ تَابُوتُ إِلَهِ إِسْرَائِيلَ عِنْدَنَا لأَنَّ يَدَهُ قَدْ قَسَتْ عَلَيْنَا وَعَلَى دَاجُونَ إِلَهِنَا».
\par 8 فَأَرْسَلُوا وَجَمَعُوا جَمِيعَ أَقْطَابِ الْفِلِسْطِينِيِّينَ إِلَيْهِمْ وَقَالُوا: «مَاذَا نَصْنَعُ بِتَابُوتِ إِلَهِ إِسْرَائِيلَ؟» فَقَالُوا: «لِيُنْقَلْ تَابُوتُ إِلَهِ إِسْرَائِيلَ إِلَى جَتَّ». فَنَقَلُوا تَابُوتَ إِلَهِ إِسْرَائِيلَ.
\par 9 وَكَانَ بَعْدَمَا نَقَلُوهُ أَنَّ يَدَ الرَّبِّ كَانَتْ عَلَى الْمَدِينَةِ بِاضْطِرَابٍ عَظِيمٍ جِدّاً, وَضَرَبَ أَهْلَ الْمَدِينَةِ مِنَ الصَّغِيرِ إِلَى الْكَبِيرِ وَنَفَرَتْ لَهُمُ الْبَوَاسِيرُ.
\par 10 فَأَرْسَلُوا تَابُوتَ اللَّهِ إِلَى عَقْرُونَ. وَكَانَ لَمَّا دَخَلَ تَابُوتُ اللَّهِ إِلَى عَقْرُونَ أَنَّهُ صَرَخَ الْعَقْرُونِيُّونَ: «قَدْ نَقَلُوا إِلَيْنَا تَابُوتَ إِلَهِ إِسْرَائِيلَ لِيُمِيتُونَا نَحْنُ وَشَعْبَنَا!».
\par 11 وَأَرْسَلُوا وَجَمَعُوا كُلَّ أَقْطَابِ الْفِلِسْطِينِيِّينَ وَقَالُوا: «أَرْسِلُوا تَابُوتَ إِلَهِ إِسْرَائِيلَ فَيَرْجِعَ إِلَى مَكَانِهِ وَلاَ يُمِيتَنَا نَحْنُ وَشَعْبَنَا». لأَنَّ اضْطِرَابَ الْمَوْتِ كَانَ فِي كُلِّ الْمَدِينَةِ. يَدُ اللَّهِ كَانَتْ ثَقِيلَةً جِدّاً هُنَاكَ.
\par 12 وَالنَّاسُ الَّذِينَ لَمْ يَمُوتُوا ضُرِبُوا بِالْبَوَاسِيرِ, فَصَعِدَ صُرَاخُ الْمَدِينَةِ إِلَى السَّمَاءِ.

\chapter{6}

\par 1 وَكَانَ تَابُوتُ اللَّهِ فِي بِلاَدِ الْفِلِسْطِينِيِّينَ سَبْعَةَ أَشْهُرٍ.
\par 2 فَسَأَلَ الْفِلِسْطِينِيُّونَ الْكَهَنَةَ وَالْعَرَّافِينَ: «مَاذَا نَعْمَلُ بِتَابُوتِ الرَّبِّ. أَخْبِرُونَا بِمَاذَا نُرْسِلُهُ إِلَى مَكَانِهِ».
\par 3 فَقَالُوا: «إِذَا أَرْسَلْتُمْ تَابُوتَ إِلَهِ إِسْرَائِيلَ فَلاَ تُرْسِلُوهُ فَارِغاً, بَلْ رُدُّوا لَهُ قُرْبَانَ إِثْمٍ. حِينَئِذٍ تَشْفُونَ وَيُعْلَمُ عِنْدَكُمْ لِمَاذَا لاَ تَرْتَفِعُ يَدُهُ عَنْكُمْ».
\par 4 فَقَالُوا: «وَمَا هُوَ قُرْبَانُ الْإِثْمِ الَّذِي نَرُدُّهُ لَهُ؟» فَقَالُوا: «حَسَبَ عَدَدِ أَقْطَابِ الْفِلِسْطِينِيِّينَ: خَمْسَةَ بَوَاسِيرَ مِنْ ذَهَبٍ وَخَمْسَةَ فِيرَانٍ مِنْ ذَهَبٍ. لأَنَّ الضَّرْبَةَ وَاحِدَةٌ عَلَيْكُمْ جَمِيعاً وَعَلَى أَقْطَابِكُمْ.
\par 5 وَاصْنَعُوا تَمَاثِيلَ بَوَاسِيرِكُمْ وَتَمَاثِيلَ فِيرَانِكُمُ الَّتِي تُفْسِدُ الأَرْضَ, وَأَعْطُوا إِلَهَ إِسْرَائِيلَ مَجْداً لَعَلَّهُ يُخَفِّفُ يَدَهُ عَنْكُمْ وَعَنْ آلِهَتِكُمْ وَعَنْ أَرْضِكُمْ.
\par 6 وَلِمَاذَا تُغْلِظُونَ قُلُوبَكُمْ كَمَا أَغْلَظَ الْمِصْرِيُّونَ وَفِرْعَوْنُ قُلُوبَهُمْ؟ أَلَيْسَ عَلَى مَا فَعَلَ بِهِمْ أَطْلَقُوهُمْ فَذَهَبُوا؟
\par 7 فَالآنَ خُذُوا وَاعْمَلُوا عَجَلَةً وَاحِدَةً جَدِيدَةً وَبَقَرَتَيْنِ مُرْضِعَتَيْنِ لَمْ يَعْلُهُمَا نِيرٌ, وَارْبِطُوا الْبَقَرَتَيْنِ إِلَى الْعَجَلَةِ, وَأَرْجِعُوا وَلَدَيْهِمَا عَنْهُمَا إِلَى الْبَيْتِ.
\par 8 وَخُذُوا تَابُوتَ الرَّبِّ وَاجْعَلُوهُ عَلَى الْعَجَلَةِ, وَضَعُوا أَمْتِعَةَ الذَّهَبِ الَّتِي تَرُدُّونَهَا لَهُ قُرْبَانَ إِثْمٍ فِي صُنْدُوقٍ بِجَانِبِهِ وَأَطْلِقُوهُ فَيَذْهَبَ.
\par 9 وَانْظُرُوا, فَإِنْ صَعِدَ فِي طَرِيقِ تُخُمِهِ إِلَى بَيْتَشَمْسَ فَإِنَّهُ هُوَ الَّذِي فَعَلَ بِنَا هَذَا الشَّرَّ الْعَظِيمَ. وَإِلَّا فَنَعْلَمُ أَنْ يَدَهُ لَمْ تَضْرِبْنَا. كَانَ ذَلِكَ عَلَيْنَا عَرَضاً».
\par 10 فَفَعَلَ الرِّجَالُ كَذَلِكَ, وَأَخَذُوا بَقَرَتَيْنِ مُرْضِعَتَيْنِ وَرَبَطُوهُمَا إِلَى الْعَجَلَةِ, وَحَبَسُوا وَلَدَيْهِمَا فِي الْبَيْتِ,
\par 11 وَوَضَعُوا تَابُوتَ الرَّبِّ عَلَى الْعَجَلَةِ مَعَ الصُّنْدُوقِ وَفِيرَانِ الذَّهَبِ وَتَمَاثِيلِ بَوَاسِيرِهِمْ.
\par 12 فَاسْتَقَامَتِ الْبَقَرَتَانِ فِي الطَّرِيقِ إِلَى طَرِيقِ بَيْتَشَمْسَ, وَكَانَتَا تَسِيرَانِ فِي سِكَّةٍ وَاحِدَةٍ وَتَجْأَرَانِ وَلَمْ تَمِيلاَ يَمِيناً وَلاَ شِمَالاً, وَأَقْطَابُ الْفِلِسْطِينِيِّينَ يَسِيرُونَ وَرَاءَهُمَا إِلَى تُخُمِ بَيْتَشَمْسَ.
\par 13 وَكَانَ أَهْلُ بَيْتَشَمْسَ يَحْصُدُونَ حَصَادَ الْحِنْطَةِ فِي الْوَادِي. فَرَفَعُوا أَعْيُنَهُمْ وَرَأَوُا التَّابُوتَ وَفَرِحُوا بِرُؤْيَتِهِ.
\par 14 فَأَتَتِ الْعَجَلَةُ إِلَى حَقْلِ يَهُوشَعَ الْبَيْتَشَمْسِيِّ وَوَقَفَتْ هُنَاكَ. وَهُنَاكَ حَجَرٌ كَبِيرٌ. فَشَقَّقُوا خَشَبَ الْعَجَلَةِ وَأَصْعَدُوا الْبَقَرَتَيْنِ مُحْرَقَةً لِلرَّبِّ.
\par 15 فَأَنْزَلَ اللَّاوِيُّونَ تَابُوتَ الرَّبِّ وَالصُّنْدُوقَ الَّذِي مَعَهُ الَّذِي فِيهِ أَمْتِعَةُ الذَّهَبِ وَوَضَعُوهُمَا عَلَى الْحَجَرِ الْكَبِيرِ. وَأَصْعَدَ أَهْلُ بَيْتَشَمْسَ مُحْرَقَاتٍ وَذَبَحُوا ذَبَائِحَ فِي ذَلِكَ الْيَوْمِ لِلرَّبِّ.
\par 16 فَرَأَى أَقْطَابُ الْفِلِسْطِينِيِّينَ الْخَمْسَةُ وَرَجَعُوا إِلَى عَقْرُونَ فِي ذَلِكَ الْيَوْمِ.
\par 17 وَهَذِهِ هِيَ بَوَاسِيرُ الذَّهَبِ الَّتِي رَدَّهَا الْفِلِسْطِينِيُّونَ قُرْبَانَ إِثْمٍ لِلرَّبِّ: وَاحِدٌ لأَشْدُودَ, وَوَاحِدٌ لِغَزَّةَ, وَوَاحِدٌ لأَشْقَلُونَ, وَوَاحِدٌ لِجَتَّ, وَوَاحِدٌ لِعَقْرُونَ.
\par 18 وَفِيرَانُ الذَّهَبِ بِعَدَدِ جَمِيعِ مُدُنِ الْفِلِسْطِينِيِّينَ لِلْخَمْسَةِ الأَقْطَابِ مِنَ الْمَدِينَةِ الْمُحَصَّنَةِ إِلَى قَرْيَةِ الصَّحْرَاءِ. وَشَاهِدٌ هُوَ الْحَجَرُ الْكَبِيرُ الَّذِي وَضَعُوا عَلَيْهِ تَابُوتَ الرَّبِّ. هُوَ إِلَى هَذَا الْيَوْمِ فِي حَقْلِ يَهُوشَعَ الْبَيْتِشَمْسِيِّ.
\par 19 وَضَرَبَ أَهْلَ بَيْتَشَمْسَ لأَنَّهُمْ نَظَرُوا إِلَى تَابُوتِ الرَّبِّ. وَضَرَبَ مِنَ الشَّعْبِ خَمْسِينَ أَلْفَ رَجُلٍ وَسَبْعِينَ رَجُلاً. فَنَاحَ الشَّعْبُ لأَنَّ الرَّبَّ ضَرَبَ الشَّعْبَ ضَرْبَةً عَظِيمَةً.
\par 20 وَقَالَ أَهْلُ بَيْتَشَمْسَ: «مَنْ يَقْدِرُ أَنْ يَقِفَ أَمَامَ الرَّبِّ الْإِلَهِ الْقُدُّوسِ هَذَا, وَإِلَى مَنْ يَصْعَدُ عَنَّا؟»
\par 21 وَأَرْسَلُوا رُسُلاً إِلَى سُكَّانِ قَرْيَةِ يَعَارِيمَ قَائِلِينَ: «قَدْ رَدَّ الْفِلِسْطِينِيُّونَ تَابُوتَ الرَّبِّ, فَانْزِلُوا وَأَصْعِدُوهُ إِلَيْكُمْ».

\chapter{7}

\par 1 فَجَاءَ أَهْلُ قَرْيَةِ يَعَارِيمَ وَأَصْعَدُوا تَابُوتَ الرَّبِّ وَأَدْخَلُوهُ إِلَى بَيْتِ أَبِينَادَابَ فِي الأَكَمَةِ, وَقَدَّسُوا أَلِعَازَارَ ابْنَهُ لأَجْلِ حِرَاسَةِ تَابُوتِ الرَّبِّ.
\par 2 وَكَانَ مِنْ يَوْمِ جُلُوسِ التَّابُوتِ فِي قَرْيَةِ يَعَارِيمَ أَنَّ الْمُدَّةَ طَالَتْ وَكَانَتْ عِشْرِينَ سَنَةً. وَنَاحَ كُلُّ بَيْتِ إِسْرَائِيلَ وَرَاءَ الرَّبِّ.
\par 3 وَقَالَ صَمُوئِيلُ لِكُلِّ بَيْتِ إِسْرَائِيلَ: «إِنْ كُنْتُمْ بِكُلِّ قُلُوبِكُمْ رَاجِعِينَ إِلَى الرَّبِّ فَانْزِعُوا الآلِهَةَ الْغَرِيبَةَ وَالْعَشْتَارُوثَ مِنْ وَسْطِكُمْ, وَأَعِدُّوا قُلُوبَكُمْ لِلرَّبِّ وَاعْبُدُوهُ وَحْدَهُ, فَيُنْقِذَكُمْ مِنْ يَدِ الْفِلِسْطِينِيِّينَ».
\par 4 فَنَزَعَ بَنُو إِسْرَائِيلَ الْبَعْلِيمَ وَالْعَشْتَارُوثَ وَعَبَدُوا الرَّبَّ وَحْدَهُ.
\par 5 فَقَالَ صَمُوئِيلُ: «اجْمَعُوا كُلَّ إِسْرَائِيلَ إِلَى الْمِصْفَاةِ فَأُصَلِّيَ لأَجْلِكُمْ إِلَى الرَّبِّ.
\par 6 فَاجْتَمَعُوا إِلَى الْمِصْفَاةِ وَاسْتَقُوا مَاءً وَسَكَبُوهُ أَمَامَ الرَّبِّ, وَصَامُوا فِي ذَلِكَ الْيَوْمِ وَقَالُوا: «هُنَاكَ قَدْ أَخْطَأْنَا إِلَى الرَّبِّ». وَقَضَى صَمُوئِيلُ لِبَنِي إِسْرَائِيلَ فِي الْمِصْفَاةِ.
\par 7 وَسَمِعَ الْفِلِسْطِينِيُّونَ أَنَّ بَنِي إِسْرَائِيلَ قَدِ اجْتَمَعُوا فِي الْمِصْفَاةِ, فَصَعِدَ أَقْطَابُ الْفِلِسْطِينِيِّينَ إِلَى إِسْرَائِيلَ. فَلَمَّا سَمِعَ بَنُو إِسْرَائِيلَ خَافُوا مِنَ الْفِلِسْطِينِيِّينَ.
\par 8 وَقَالَ بَنُو إِسْرَائِيلَ لِصَمُوئِيلَ: «لاَ تَكُفَّ عَنِ الصُّرَاخِ مِنْ أَجْلِنَا إِلَى الرَّبِّ إِلَهِنَا فَيُخَلِّصَنَا مِنْ يَدِ الْفِلِسْطِينِيِّينَ».
\par 9 فَأَخَذَ صَمُوئِيلُ حَمَلاً رَضِيعاً وَأَصْعَدَهُ مُحْرَقَةً بِتَمَامِهِ لِلرَّبِّ. وَصَرَخَ صَمُوئِيلُ إِلَى الرَّبِّ مِنْ أَجْلِ إِسْرَائِيلَ فَاسْتَجَابَ لَهُ الرَّبُّ.
\par 10 وَبَيْنَمَا كَانَ صَمُوئِيلُ يُصْعِدُ الْمُحْرَقَةَ تَقَدَّمَ الْفِلِسْطِينِيُّونَ لِمُحَارَبَةِ إِسْرَائِيلَ, فَأَرْعَدَ الرَّبُّ بِصَوْتٍ عَظِيمٍ فِي ذَلِكَ الْيَوْمِ عَلَى الْفِلِسْطِينِيِّينَ وَأَزْعَجَهُمْ, فَانْكَسَرُوا أَمَامَ إِسْرَائِيلَ.
\par 11 وَخَرَجَ رِجَالُ إِسْرَائِيلَ مِنَ الْمِصْفَاةِ وَتَبِعُوا الْفِلِسْطِينِيِّينَ وَضَرَبُوهُمْ إِلَى مَا تَحْتَ بَيْتِ كَارٍ.
\par 12 فَأَخَذَ صَمُوئِيلُ حَجَراً وَنَصَبَهُ بَيْنَ الْمِصْفَاةِ وَالسِّنِّ, وَدَعَا اسْمَهُ «حَجَرَ الْمَعُونَةِ» وَقَالَ: «إِلَى هُنَا أَعَانَنَا الرَّبُّ».
\par 13 فَذَلَّ الْفِلِسْطِينِيُّونَ وَلَمْ يَعُودُوا بَعْدُ لِلدُّخُولِ فِي تُخُمِ إِسْرَائِيلَ. وَكَانَتْ يَدُ الرَّبِّ عَلَى الْفِلِسْطِينِيِّينَ كُلَّ أَيَّامِ صَمُوئِيلَ.
\par 14 وَالْمُدُنُ الَّتِي أَخَذَهَا الْفِلِسْطِينِيُّونَ مِنْ إِسْرَائِيلَ رَجَعَتْ إِلَى إِسْرَائِيلَ مِنْ عَقْرُونَ إِلَى جَتَّ. وَاسْتَخْلَصَ إِسْرَائِيلُ تُخُومَهَا مِنْ يَدِ الْفِلِسْطِينِيِّينَ. وَكَانَ صُلْحٌ بَيْنَ إِسْرَائِيلَ وَالأَمُورِيِّينَ.
\par 15 وَقَضَى صَمُوئِيلُ لإِسْرَائِيلَ كُلَّ أَيَّامِ حَيَاتِهِ.
\par 16 وَكَانَ يَذْهَبُ مِنْ سَنَةٍ إِلَى سَنَةٍ وَيَدُورُ فِي بَيْتِ إِيلَ وَالْجِلْجَالِ وَالْمِصْفَاةِ وَيَقْضِي لإِسْرَائِيلَ فِي جَمِيعِ هَذِهِ الْمَوَاضِعِ.
\par 17 وَكَانَ رُجُوعُهُ إِلَى الرَّامَةِ لأَنَّ بَيْتَهُ هُنَاكَ. وَهُنَاكَ قَضَى لإِسْرَائِيلَ, وَبَنَى هُنَاكَ مَذْبَحاً لِلرَّبِّ.

\chapter{8}

\par 1 وَكَانَ لَمَّا شَاخَ صَمُوئِيلُ أَنَّهُ جَعَلَ بَنِيهِ قُضَاةً لإِسْرَائِيلَ.
\par 2 وَكَانَ اسْمُ ابْنِهِ الْبِكْرِ يُوئِيلَ, وَاسْمُ ثَانِيهِ أَبِيَّا. كَانَا قَاضِيَيْنِ فِي بِئْرِ سَبْعٍ.
\par 3 وَلَمْ يَسْلُكِ ابْنَاهُ فِي طَرِيقِهِ بَلْ مَالاَ وَرَاءَ الْمَكْسَبِ, وَأَخَذَا رَشْوَةً وَعَوَّجَا الْقَضَاءَ.
\par 4 فَاجْتَمَعَ كُلُّ شُيُوخِ إِسْرَائِيلَ وَجَاءُوا إِلَى صَمُوئِيلَ إِلَى الرَّامَةِ
\par 5 وَقَالُوا لَهُ: «هُوَذَا أَنْتَ قَدْ شِخْتَ, وَابْنَاكَ لَمْ يَسِيرَا فِي طَرِيقِكَ. فَالآنَ اجْعَلْ لَنَا مَلِكاً يَقْضِي لَنَا كَسَائِرِ الشُّعُوبِ».
\par 6 فَسَاءَ الأَمْرُ فِي عَيْنَيْ صَمُوئِيلَ إِذْ قَالُوا: «أَعْطِنَا مَلِكاً يَقْضِي لَنَا». وَصَلَّى صَمُوئِيلُ إِلَى الرَّبِّ.
\par 7 فَقَالَ الرَّبُّ لِصَمُوئِيلَ: «اسْمَعْ لِصَوْتِ الشَّعْبِ فِي كُلِّ مَا يَقُولُونَ لَكَ. لأَنَّهُمْ لَمْ يَرْفُضُوكَ أَنْتَ بَلْ إِيَّايَ رَفَضُوا حَتَّى لاَ أَمْلِكَ عَلَيْهِمْ.
\par 8 حَسَبَ كُلِّ أَعْمَالِهِمِ الَّتِي عَمِلُوا مِنْ يَوْمِ أَصْعَدْتُهُمْ مِنْ مِصْرَ إِلَى هَذَا الْيَوْمِ وَتَرَكُونِي وَعَبَدُوا آلِهَةً, أُخْرَى هَكَذَا هُمْ عَامِلُونَ بِكَ أَيْضاً.
\par 9 فَالآنَ اسْمَعْ لِصَوْتِهِمْ. وَلَكِنْ أَشْهِدَنَّ عَلَيْهِمْ وَأَخْبِرْهُمْ بِقَضَاءِ الْمَلِكِ الَّذِي يَمْلِكُ عَلَيْهِمْ».
\par 10 فَكَلَّمَ صَمُوئِيلُ الشَّعْبَ الَّذِينَ طَلَبُوا مِنْهُ مَلِكاً بِجَمِيعِ كَلاَمِ الرَّبِّ
\par 11 وَقَالَ: «هَذَا يَكُونُ قَضَاءُ الْمَلِكِ الَّذِي يَمْلِكُ عَلَيْكُمْ: يَأْخُذُ بَنِيكُمْ وَيَجْعَلُهُمْ لِنَفْسِهِ, لِمَرَاكِبِهِ وَفُرْسَانِهِ, فَيَرْكُضُونَ أَمَامَ مَرَاكِبِهِ.
\par 12 وَيَجْعَلُ لِنَفْسِهِ رُؤَسَاءَ أُلُوفٍ وَرُؤَسَاءَ خَمَاسِينَ فَيَحْرُثُونَ حِرَاثَتَهُ وَيَحْصُدُونَ حَصَادَهُ وَيَعْمَلُونَ عُدَّةَ حَرْبِهِ وَأَدَوَاتِ مَرَاكِبِهِ.
\par 13 وَيَأْخُذُ بَنَاتِكُمْ عَطَّارَاتٍ وَطَبَّاخَاتٍ وَخَبَّازَاتٍ,
\par 14 وَيَأْخُذُ حُقُولَكُمْ وَكُرُومَكُمْ وَزَيْتُونَكُمْ أَجْوَدَهَا وَيُعْطِيهَا لِعَبِيدِهِ.
\par 15 وَيُعَشِّرُ زُرُوعَكُمْ وَكُرُومَكُمْ وَيُعْطِي لِخِصْيَانِهِ وَعَبِيدِهِ.
\par 16 وَيَأْخُذُ عَبِيدَكُمْ وَجَوَارِيَكُمْ وَشُبَّانَكُمُ الْحِسَانَ وَحَمِيرَكُمْ وَيَسْتَعْمِلُهُمْ لِشُغْلِهِ.
\par 17 وَيُعَشِّرُ غَنَمَكُمْ وَأَنْتُمْ تَكُونُونَ لَهُ عَبِيداً.
\par 18 فَتَصْرُخُونَ فِي ذَلِكَ الْيَوْمِ مِنْ وَجْهِ مَلِكِكُمُ الَّذِي اخْتَرْتُمُوهُ لأَنْفُسِكُمْ فَلاَ يَسْتَجِيبُ لَكُمُ الرَّبُّ فِي ذَلِكَ الْيَوْمِ».
\par 19 فَأَبَى الشَّعْبُ أَنْ يَسْمَعُوا لِصَوْتِ صَمُوئِيلَ وَقَالُوا: «لاَ بَلْ يَكُونُ عَلَيْنَا مَلِكٌ,
\par 20 فَنَكُونُ نَحْنُ أَيْضاً مِثْلَ سَائِرِ الشُّعُوبِ, وَيَقْضِي لَنَا مَلِكُنَا وَيَخْرُجُ أَمَامَنَا وَيُحَارِبُ حُرُوبَنَا».
\par 21 فَسَمِعَ صَمُوئِيلُ كُلَّ كَلاَمِ الشَّعْبِ وَتَكَلَّمَ بِهِ فِي أُذُنَيِ الرَّبِّ.
\par 22 فَقَالَ الرَّبُّ لِصَمُوئِيلَ: «اسْمَعْ لِصَوْتِهِمْ وَمَلِّكْ عَلَيْهِمْ مَلِكاً». فَقَالَ صَمُوئِيلُ لِرِجَالِ إِسْرَائِيلَ: «اذْهَبُوا كُلُّ وَاحِدٍ إِلَى مَدِينَتِهِ».

\chapter{9}

\par 1 وَكَانَ رَجُلٌ مِنْ بِنْيَامِينَ اسْمُهُ قَيْسُ بْنُ أَبِيئِيلَ بْنِ صَرُورَ بْنِ بَكُورَةَ بْنِ أَفِيحَ, ابْنُ رَجُلٍ بِنْيَامِينِيٍّ جَبَّارَ بَأْسٍ.
\par 2 وَكَانَ لَهُ ابْنٌ اسْمُهُ شَاوُلُ, شَابٌّ وَحَسَنٌ, وَلَمْ يَكُنْ رَجُلٌ فِي بَنِي إِسْرَائِيلَ أَحْسَنَ مِنْهُ. مِنْ كَتِفِهِ فَمَا فَوْقُ كَانَ أَطْوَلَ مِنْ كُلِّ الشَّعْبِ.
\par 3 فَضَلَّتْ أُتُنُ قَيْسَ أَبِي شَاوُلَ. فَقَالَ قَِيْسُ لِشَاوُلَ ابْنِهِ: «خُذْ مَعَكَ وَاحِداً مِنَ الْغِلْمَانِ وَقُمِ اذْهَبْ فَتِّشْ عَلَى الأُتُنِ».
\par 4 فَعَبَرَ فِي جَبَلِ أَفْرَايِمَ, ثُمَّ عَبَرَ فِي أَرْضِ شَلِيشَةَ فَلَمْ يَجِدْهَا. ثُمَّ عَبَرَا فِي أَرْضِ شَعَلِيمَ فَلَمْ تُوجَدْ. ثُمَّ عَبَرَا فِي أَرْضِ بِنْيَامِينَ فَلَمْ يَجِدَاهَا.
\par 5 وَلَمَّا دَخَلاَ أَرْضَ صُوفٍ قَالَ شَاوُلُ لِغُلاَمِهِ الَّذِي مَعَهُ: «تَعَالَ نَرْجِعْ لِئَلَّا يَتْرُكَ أَبِي الأُتُنَ وَيَهْتَمَّ بِنَا».
\par 6 فَقَالَ لَهُ: «هُوَذَا رَجُلُ اللَّهِ فِي هَذِهِ الْمَدِينَةِ وَالرَّجُلُ مُكَرَّمٌ, كُلُّ مَا يَقُولُهُ يَصِيرُ. لِنَذْهَبِ الآنَ إِلَى هُنَاكَ لَعَلَّهُ يُخْبِرُنَا عَنْ طَرِيقِنَا الَّتِي نَسْلُكُ فِيهَا».
\par 7 فَقَالَ شَاوُلُ لِلْغُلاَمِ: «هُوَذَا نَذْهَبُ, فَمَاذَا نُقَدِّمُ لِلرَّجُلِ؟ لأَنَّ الْخُبْزَ قَدْ نَفَدَ مِنْ أَوْعِيَتِنَا وَلَيْسَ مِنْ هَدِيَّةٍ نُقَدِّمُهَا لِرَجُلِ اللَّهِ. مَاذَا مَعَنَا؟»
\par 8 فَعَادَ الْغُلاَمُ وَأَجَابَ شَاوُلَ: «هُوَذَا يُوجَدُ بِيَدِي رُبْعُ شَاقِلِ فِضَّةٍ فَأُعْطِيهِ لِرَجُلِ اللَّهِ فَيُخْبِرُنَا عَنْ طَرِيقِنَا».
\par 9 (سَابِقاً فِي إِسْرَائِيلَ هَكَذَا كَانَ يَقُولُ الرَّجُلُ عِنْدَ ذِهَابِهِ لِيَسْأَلَ اللَّهَ: «هَلُمَّ نَذْهَبْ إِلَى الرَّائِي». لأَنَّ النَّبِيَّ الْيَوْمَ كَانَ يُدْعَى سَابِقاً الرَّائِيَ).
\par 10 فَقَالَ شَاوُلُ لِغُلاَمِهِ: «كَلاَمُكَ حَسَنٌ. هَلُمَّ نَذْهَبْ». فَذَهَبَا إِلَى الْمَدِينَةِ الَّتِي فِيهَا رَجُلُ اللَّهِ.
\par 11 وَفِيمَا هُمَا صَاعِدَانِ فِي مَطْلَعِ الْمَدِينَةِ صَادَفَا فَتَيَاتٍ خَارِجَاتٍ لاِسْتِقَاءِ الْمَاءِ. فَقَالاَ لَهُنَّ: «أَهُنَا الرَّائِي؟»
\par 12 فَأَجَبْنَهُمَا: «نَعَمْ. هُوَذَا هُوَ أَمَامَكُمَا. أَسْرِعَا الآنَ, لأَنَّهُ جَاءَ الْيَوْمَ إِلَى الْمَدِينَةِ لأَنَّهُ الْيَوْمَ ذَبِيحَةٌ لِلشَّعْبِ عَلَى الْمُرْتَفَعَةِ.
\par 13 عِنْدَ دُخُولِكُمَا الْمَدِينَةَ لِلْوَقْتِ تَجِدَانِهِ قَبْلَ صُعُودِهِ إِلَى الْمُرْتَفَعَةِ لِيَأْكُلَ - لأَنَّ الشَّعْبَ لاَ يَأْكُلُ حَتَّى يَأْتِيَ لأَنَّهُ يُبَارِكُ الذَّبِيحَةَ. بَعْدَ ذَلِكَ يَأْكُلُ الْمَدْعُوُّونَ. فَالآنَ اصْعَدَا لأَنَّكُمَا فِي مِثْلِ الْيَوْمِ تَجِدَانِهِ».
\par 14 فَصَعِدَا إِلَى الْمَدِينَةِ. وَفِيمَا هُمَا آتِيَانِ فِي وَسَطِ الْمَدِينَةِ إِذَا بِصَمُوئِيلَ خَارِجٌ لِلِقَائِهِمَا لِيَصْعَدَ إِلَى الْمُرْتَفَعَةِ.
\par 15 وَالرَّبُّ كَشَفَ أُذُنَ صَمُوئِيلَ قَبْلَ مَجِيءِ شَاوُلَ بِيَوْمٍ قَائِلاً:
\par 16 «غَداً فِي مِثْلِ الآنَ أُرْسِلُ إِلَيْكَ رَجُلاً مِنْ أَرْضِ بِنْيَامِينَ, فَامْسَحْهُ رَئِيساً لِشَعْبِي إِسْرَائِيلَ, فَيُخَلِّصَ شَعْبِي مِنْ يَدِ الْفِلِسْطِينِيِّينَ, لأَنِّي نَظَرْتُ إِلَى شَعْبِي لأَنَّ صُرَاخَهُمْ قَدْ جَاءَ إِلَيَّ».
\par 17 فَلَمَّا رَأَى صَمُوئِيلُ شَاوُلَ قَالَ الرَّبُّ: «هُوَذَا الرَّجُلُ الَّذِي كَلَّمْتُكَ عَنْهُ. هَذَا يَضْبِطُ شَعْبِي».
\par 18 فَتَقَدَّمَ شَاوُلُ إِلَى صَمُوئِيلَ فِي وَسَطِ الْبَابِ وَقَالَ: «أَطْلُبُ إِلَيْكَ: أَخْبِرْنِي أَيْنَ بَيْتُ الرَّائِي؟»
\par 19 فَأَجَابَ صَمُوئِيلُ شَاوُلَ: «أَنَا الرَّائِي. إِصْعَدَا أَمَامِي إِلَى الْمُرْتَفَعَةِ فَتَأْكُلاَ مَعِيَ الْيَوْمَ ثُمَّ أُطْلِقَكَ صَبَاحاً وَأُخْبِرَكَ بِكُلِّ مَا فِي قَلْبِكَ.
\par 20 وَأَمَّا الأُتُنُ الضَّالَّةُ لَكَ مُنْذُ ثَلاَثَةِ أَيَّامٍ فَلاَ تَضَعْ قَلْبَكَ عَلَيْهَا لأَنَّهَا قَدْ وُجِدَتْ. وَلِمَنْ كُلُّ شَهِيِّ إِسْرَائِيلَ؟ أَلَيْسَ لَكَ وَلِكُلِّ بَيْتِ أَبِيكَ؟»
\par 21 فَقَالَ شَاوُلُ: «أَمَا أَنَا بِنْيَامِينِيٌّ مِنْ أَصْغَرِ أَسْبَاطِ إِسْرَائِيلَ, وَعَشِيرَتِي أَصْغَرُ كُلِّ عَشَائِرِ أَسْبَاطِ بِنْيَامِينَ؟ فَلِمَاذَا تُكَلِّمُنِي بِمِثْلِ هَذَا الْكَلاَمِ؟»
\par 22 فَأَخَذَ صَمُوئِيلُ شَاوُلَ وَغُلاَمَهُ وَأَدْخَلَهُمَا إِلَى الْمَنْسَكِ وَأَعْطَاهُمَا مَكَاناً فِي رَأْسِ الْمَدْعُوِّينَ, وَهُمْ نَحْوُ ثَلاَثِينَ رَجُلاً.
\par 23 وَقَالَ صَمُوئِيلُ لِلطَّبَّاخِ: «هَاتِ النَّصِيبَ الَّذِي أَعْطَيْتُكَ إِيَّاهُ, الَّذِي قُلْتُ لَكَ عَنْهُ ضَعْهُ عِنْدَكَ».
\par 24 فَرَفَعَ الطَّبَّاخُ السَّاقَ مَعَ مَا عَلَيْهَا وَجَعَلَهَا أَمَامَ شَاوُلَ. فَقَالَ: «هُوَذَا مَا أُبْقِيَ. ضَعْهُ أَمَامَكَ وَكُلْ. لأَنَّهُ إِلَى هَذَا الْمِيعَادِ مَحْفُوظٌ لَكَ مُنْذُ دَعَوْتُ الشَّعْبَ». فَأَكَلَ شَاوُلُ مَعَ صَمُوئِيلَ فِي ذَلِكَ الْيَوْمِ.
\par 25 وَلَمَّا نَزَلُوا مِنَ الْمُرْتَفَعَةِ إِلَى الْمَدِينَةِ تَكَلَّمَ مَعَ شَاوُلَ عَلَى السَّطْحِ.
\par 26 وَبَكَّرُوا. وَكَانَ عِنْدَ طُلُوعِ الْفَجْرِ أَنَّ صَمُوئِيلَ دَعَا شَاوُلَ عَنِ السَّطْحِ قَائِلاً: «قُمْ فَأَصْرِفَكَ». فَقَامَ شَاوُلُ وَخَرَجَا كِلاَهُمَا, هُوَ وَصَمُوئِيلُ إِلَى خَارِجٍ.
\par 27 وَفِيمَا هُمَا نَازِلاَنِ بِطَرَفِ الْمَدِينَةِ قَالَ صَمُوئِيلُ لِشَاوُلَ: «قُلْ لِلْغُلاَمِ أَنْ يَعْبُرَ قُدَّامَنَا». فَعَبَرَ. «وَأَمَّا أَنْتَ فَقِفِ الآنَ فَأُسْمِعَكَ كَلاَمَ اللَّهِ».

\chapter{10}

\par 1 فَأَخَذَ صَمُوئِيلُ قِنِّينَةَ الدُّهْنِ وَصَبَّ عَلَى رَأْسِهِ وَقَبَّلَهُ وَقَالَ: «أَلَيْسَ لأَنَّ الرَّبَّ قَدْ مَسَحَكَ عَلَى مِيرَاثِهِ رَئِيساً؟
\par 2 فِي ذَهَابِكَ الْيَوْمَ مِنْ عِنْدِي تُصَادِفُ رَجُلَيْنِ عِنْدَ قَبْرِ رَاحِيلَ فِي تُخُمِ بِنْيَامِينَ فِي صَلْصَحَ, فَيَقُولاَنِ لَكَ: قَدْ وُجِدَتِ الأُتُنُ الَّتِي ذَهَبْتَ تُفَتِّشُ عَلَيْهَا, وَهُوَذَا أَبُوكَ قَدْ تَرَكَ أَمْرَ الأُتُنِ وَاهْتَمَّ بِكُمَا قَائِلاً: مَاذَا أَصْنَعُ لاِبْنِي؟
\par 3 وَتَعْدُو مِنْ هُنَاكَ ذَاهِباً حَتَّى تَأْتِيَ إِلَى بَلُّوطَةِ تَابُورَ, فَيُصَادِفُكَ هُنَاكَ ثَلاَثَةُ رِجَالٍ صَاعِدُونَ إِلَى اللَّهِ إِلَى بَيْتِ إِيلٍ, وَاحِدٌ حَامِلٌ ثَلاَثَةَ جِدَاءٍ, وَوَاحِدٌ حَامِلٌ ثَلاَثَةَ أَرْغِفَةِ خُبْزٍ, وَوَاحِدٌ حَامِلٌ زِقَّ خَمْرٍ.
\par 4 فَيُسَلِّمُونَ عَلَيْكَ وَيُعْطُونَكَ رَغِيفَيْ خُبْزٍ, فَتَأْخُذُ مِنْ يَدِهِمْ.
\par 5 بَعْدَ ذَلِكَ تَأْتِي إِلَى جِبْعَةِ اللَّهِ حَيْثُ أَنْصَابُ الْفِلِسْطِينِيِّينَ. وَيَكُونُ عِنْدَ مَجِيئِكَ إِلَى هُنَاكَ إِلَى الْمَدِينَةِ أَنَّكَ تُصَادِفُ زُمْرَةً مِنَ الأَنْبِيَاءِ نَازِلِينَ مِنَ الْمُرْتَفَعَةِ وَأَمَامَهُمْ رَبَابٌ وَدُفٌّ وَنَايٌ وَعُودٌ وَهُمْ يَتَنَبَّأُونَ.
\par 6 فَيَحِلُّ عَلَيْكَ رُوحُ الرَّبِّ فَتَتَنَبَّأُ مَعَهُمْ وَتَتَحَوَّلُ إِلَى رَجُلٍ آخَرَ.
\par 7 وَإِذَا أَتَتْ هَذِهِ الآيَاتُ عَلَيْكَ فَافْعَلْ مَا وَجَدَتْهُ يَدُكَ لأَنَّ اللَّهَ مَعَكَ.
\par 8 وَتَنْزِلُ قُدَّامِي إِلَى الْجِلْجَالِ, وَهُوَذَا أَنَا أَنْزِلُ إِلَيْكَ لِأُصْعِدَ مُحْرَقَاتٍ وَأَذْبَحَ ذَبَائِحَ سَلاَمَةٍ. سَبْعَةَ أَيَّامٍ تَلْبَثُ حَتَّى آتِيَ إِلَيْكَ وَأُعَلِّمَكَ مَاذَا تَفْعَلُ».
\par 9 وَكَانَ عِنْدَمَا أَدَارَ كَتِفَهُ لِيَذْهَبَ مِنْ عِنْدِ صَمُوئِيلَ أَنَّ اللَّهَ أَعْطَاهُ قَلْباً آخَرَ. وَأَتَتْ جَمِيعُ هَذِهِ الآيَاتِ فِي ذَلِكَ الْيَوْمِ.
\par 10 وَلَمَّا جَاءُوا إِلَى هُنَاكَ إِلَى جِبْعَةَ, إِذَا بِزُمْرَةٍ مِنَ الأَنْبِيَاءِ لَقِيَتْهُ, فَحَلَّ عَلَيْهِ رُوحُ اللَّهِ فَتَنَبَّأَ فِي وَسَطِهِمْ.
\par 11 وَلَمَّا رَآهُ جَمِيعُ الَّذِينَ عَرَفُوهُ مُنْذُ أَمْسِ وَمَا قَبْلَهُ أَنَّهُ يَتَنَبَّأُ مَعَ الأَنْبِيَاءِ, قَالَ الشَّعْبُ الْوَاحِدُ لِصَاحِبِهِ: «مَاذَا صَارَ لاِبْنِ قَيْسٍ؟ أَشَاوُلُ أَيْضاً بَيْنَ الأَنْبِيَاءِ؟»
\par 12 فَقَالَ رَجُلٌ مِنْ هُنَاكَ: «وَمَنْ هُوَ أَبُوهُمْ؟» وَلِذَلِكَ ذَهَبَ مَثَلاً: «أَشَاوُلُ أَيْضاً بَيْنَ الأَنْبِيَاءِ؟»
\par 13 وَلَمَّا انْتَهَى مِنَ التَّنَبِّي جَاءَ إِلَى الْمُرْتَفَعَةِ.
\par 14 فَقَالَ عَمُّ شَاوُلَ لَهُ وَلِغُلاَمِهِ: «إِلَى أَيْنَ ذَهَبْتُمَا؟» فَقَالَ: «لِكَيْ نُفَتِّشَ عَلَى الأُتُنِ. وَلَمَّا رَأَيْنَا أَنَّهَا لَمْ تُوجَدْ جِئْنَا إِلَى صَمُوئِيلَ».
\par 15 فَقَالَ عَمُّ شَاوُلَ: «أَخْبِرْنِي مَاذَا قَالَ لَكُمَا صَمُوئِيلُ».
\par 16 فَقَالَ شَاوُلُ لِعَمِّهِ: «أَخْبَرَنَا بِأَنَّ الأُتُنَ قَدْ وُجِدَتْ». وَلَكِنَّهُ لَمْ يُخْبِرْهُ بِأَمْرِ الْمَمْلَكَةِ الَّذِي تَكَلَّمَ بِهِ صَمُوئِيلُ.
\par 17 وَاسْتَدْعَى صَمُوئِيلُ الشَّعْبَ إِلَى الرَّبِّ إِلَى الْمِصْفَاةِ,
\par 18 وَقَالَ لِبَنِي إِسْرَائِيلَ: «هَكَذَا يَقُولُ الرَّبُّ إِلَهُ إِسْرَائِيلَ: إِنِّي أَصْعَدْتُ إِسْرَائِيلَ مِنْ مِصْرَ وَأَنْقَذْتُكُمْ مِنْ يَدِ الْمِصْرِيِّينَ وَمِنْ يَدِ جَمِيعِ الْمَمَالِكِ الَّتِي ضَايَقَتْكُمْ.
\par 19 وَأَنْتُمْ قَدْ رَفَضْتُمُ الْيَوْمَ إِلَهَكُمُ الَّذِي هُوَ مُخَلِّصُكُمْ مِنْ جَمِيعِ الَّذِينَ يُسِيئُونَ إِلَيْكُمْ وَيُضَايِقُونَكُمْ, وَقُلْتُمْ لَهُ: بَلْ تَجْعَلُ عَلَيْنَا مَلِكاً. فَالآنَ امْثُلُوا أَمَامَ الرَّبِّ حَسَبَ أَسْبَاطِكُمْ وَأُلُوفِكُمْ».
\par 20 فَقَدَّمَ صَمُوئِيلُ جَمِيعَ أَسْبَاطِ إِسْرَائِيلَ فَأُخِذَ سِبْطُ بِنْيَامِينَ.
\par 21 ثُمَّ قَدَّمَ سِبْطَ بِنْيَامِينَ حَسَبَ عَشَائِرِهِ فَأُخِذَتْ عَشِيرَةُ مَطْرِي, وَأُخِذَ شَاوُلُ بْنُ قَيْسَ. فَفَتَّشُوا عَلَيْهِ فَلَمْ يُوجَدْ.
\par 22 فَسَأَلُوا أَيْضاً مِنَ الرَّبِّ: «هَلْ يَأْتِي الرَّجُلُ إِلَى هُنَا؟» فَقَالَ الرَّبُّ: «هُوَذَا قَدِ اخْتَبَأَ بَيْنَ الأَمْتِعَةِ».
\par 23 فَرَكَضُوا وَأَخَذُوهُ مِنْ هُنَاكَ, فَوَقَفَ بَيْنَ الشَّعْبِ, فَكَانَ أَطْوَلَ مِنْ كُلِّ الشَّعْبِ مِنْ كَتِفِهِ فَمَا فَوْقُ.
\par 24 فَقَالَ صَمُوئِيلُ لِجَمِيعِ الشَّعْبِ: «أَرَأَيْتُمُ الَّذِي اخْتَارَهُ الرَّبُّ أَنَّهُ لَيْسَ مِثْلُهُ فِي جَمِيعِ الشَّعْبِ؟» فَهَتَفَ كُلُّ الشَّعْبِ وَقَالُوا: «لِيَحْيَ الْمَلِكُ!».
\par 25 فَكَلَّمَ صَمُوئِيلُ الشَّعْبَ بِقَضَاءِ الْمَمْلَكَةِ وَكَتَبَهُ فِي السِّفْرِ وَوَضَعَهُ أَمَامَ الرَّبِّ. ثُمَّ أَطْلَقَ صَمُوئِيلُ جَمِيعَ الشَّعْبِ كُلَّ وَاحِدٍ إِلَى بَيْتِهِ.
\par 26 وَشَاوُلُ أَيْضاً ذَهَبَ إِلَى بَيْتِهِ إِلَى جِبْعَةَ, وَذَهَبَ مَعَهُ الْجَمَاعَةُ الَّتِي مَسَّ اللَّهُ قَلْبَهَا.
\par 27 وَأَمَّا بَنُو بَلِيَّعَالَ فَقَالُوا: «كَيْفَ يُخَلِّصُنَا هَذَا؟» فَاحْتَقَرُوهُ وَلَمْ يُقَدِّمُوا لَهُ هَدِيَّةً. فَكَانَ كَأَصَمَّ.

\chapter{11}

\par 1 وَصَعِدَ نَاحَاشُ الْعَمُّونِيُّ وَنَزَلَ عَلَى يَابِيشِ جِلْعَادَ. فَقَالَ جَمِيعُ أَهْلِ يَابِيشَ لِنَاحَاشَ: «اقْطَعْ لَنَا عَهْداً فَنُسْتَعْبَدَ لَكَ».
\par 2 فَقَالَ لَهُمْ نَاحَاشُ الْعَمُّونِيُّ: «بِهَذَا أَقْطَعُ لَكُمْ. بِتَقْوِيرِ كُلِّ عَيْنٍ يُمْنَى لَكُمْ وَجَعْلِ ذَلِكَ عَاراً عَلَى جَمِيعِ إِسْرَائِيلَ».
\par 3 فَقَالَ لَهُ شُيُوخُ يَابِيشَ: «اتْرُكْنَا سَبْعَةَ أَيَّامٍ فَنُرْسِلَ رُسُلاً إِلَى جَمِيعِ تُخُومِ إِسْرَائِيلَ. فَإِنْ لَمْ يُوجَدْ مَنْ يُخَلِّصُنَا نَخْرُجْ إِلَيْكَ».
\par 4 فَجَاءَ الرُّسُلُ إِلَى جِبْعَةِ شَاوُلَ وَتَكَلَّمُوا بِهَذَا الْكَلاَمِ فِي آذَانِ الشَّعْبِ, فَرَفَعَ كُلُّ الشَّعْبِ أَصْوَاتَهُمْ وَبَكُوا.
\par 5 وَإِذَا بِشَاوُلَ آتٍ وَرَاءَ الْبَقَرِ مِنَ الْحَقْلِ, فَقَالَ: «مَا بَالُ الشَّعْبِ يَبْكُونَ؟» فَقَصُّوا عَلَيْهِ كَلاَمَ أَهْلِ يَابِيشَ.
\par 6 فَحَلَّ رُوحُ اللَّهِ عَلَى شَاوُلَ عِنْدَمَا سَمِعَ هَذَا الْكَلاَمَ وَحَمِيَ غَضَبُهُ جِدّاً.
\par 7 فَأَخَذَ زَوْجَ بَقَرٍ وَقَطَّعَهُ, وَأَرْسَلَ إِلَى كُلِّ تُخُومِ إِسْرَائِيلَ بِيَدِ الرُّسُلِ قَائِلاً: «مَنْ لاَ يَخْرُجُ وَرَاءَ شَاوُلَ وَوَرَاءَ صَمُوئِيلَ, فَهَكَذَا يُفْعَلُ بِبَقَرِهِ». فَوَقَعَ رُعْبُ الرَّبِّ عَلَى الشَّعْبِ, فَخَرَجُوا كَرَجُلٍ وَاحِدٍ.
\par 8 وَعَدَّهُمْ فِي بَازَقَ فَكَانَ بَنُو إِسْرَائِيلَ ثَلاَثَ مِئَةِ أَلْفٍ, وَرِجَالُ يَهُوذَا ثَلاَثِينَ أَلْفاً.
\par 9 وَقَالُوا لِلرُّسُلِ الَّذِينَ جَاءُوا: «هَكَذَا تَقُولُونَ لأَهْلِ يَابِيشَ جِلْعَادَ: غَداً عِنْدَمَا تَحْمَى الشَّمْسُ يَكُونُ لَكُمْ خَلاَصٌ». فَأَتَى الرُّسُلُ وَأَخْبَرُوا أَهْلَ يَابِيشَ فَفَرِحُوا.
\par 10 وَقَالَ أَهْلُ يَابِيشَ: «غَداً نَخْرُجُ إِلَيْكُمْ فَتَفْعَلُونَ بِنَا حَسَبَ كُلِّ مَا يَحْسُنُ فِي أَعْيُنِكُمْ».
\par 11 وَكَانَ فِي الْغَدِ أَنَّ شَاوُلَ جَعَلَ الشَّعْبَ ثَلاَثَ فِرَقٍ, وَدَخَلُوا فِي وَسَطِ الْمَحَلَّةِ عِنْدَ سَحَرِ الصُّبْحِ وَضَرَبُوا الْعَمُّونِيِّينَ حَتَّى حَمِيَ النَّهَارُ. وَالَّذِينَ بَقُوا تَشَتَّتُوا حَتَّى لَمْ يَبْقَ مِنْهُمُ اثْنَانِ مَعاً.
\par 12 وَقَالَ الشَّعْبُ لِصَمُوئِيلَ: «مَنْ هُمُ الَّذِينَ يَقُولُونَ: هَلْ شَاوُلُ يَمْلِكُ عَلَيْنَا؟ ايتُوا بِالرِّجَالِ فَنَقْتُلَهُمْ».
\par 13 فَقَالَ شَاوُلُ: «لاَ يُقْتَلْ أَحَدٌ فِي هَذَا الْيَوْمِ, لأَنَّهُ فِي هَذَا الْيَوْمِ صَنَعَ الرَّبُّ خَلاَصاً فِي إِسْرَائِيلَ».
\par 14 وَقَالَ صَمُوئِيلُ لِلشَّعْبِ: «هَلُمُّوا نَذْهَبْ إِلَى الْجِلْجَالِ وَنُجَدِّدْ هُنَاكَ الْمَمْلَكَةَ».
\par 15 فَذَهَبَ كُلُّ الشَّعْبِ إِلَى الْجِلْجَالِ وَمَلَّكُوا هُنَاكَ شَاوُلَ أَمَامَ الرَّبِّ فِي الْجِلْجَالِ, وَذَبَحُوا هُنَاكَ ذَبَائِحَ سَلاَمَةٍ أَمَامَ الرَّبِّ. وَفَرِحَ هُنَاكَ شَاوُلُ وَجَمِيعُ رِجَالِ إِسْرَائِيلَ جِدّاً.

\chapter{12}

\par 1 وَقَالَ صَمُوئِيلُ لِكُلِّ إِسْرَائِيلَ: «هَئَنَذَا قَدْ سَمِعْتُ لِصَوْتِكُمْ فِي كُلِّ مَا قُلْتُمْ لِي وَمَلَّكْتُ عَلَيْكُمْ مَلِكاً.
\par 2 وَالآنَ هُوَذَا الْمَلِكُ يَمْشِي أَمَامَكُمْ. وَأَمَّا أَنَا فَقَدْ شِخْتُ وَشِبْتُ, وَهُوَذَا أَبْنَائِي مَعَكُمْ. وَأَنَا قَدْ سِرْتُ أَمَامَكُمْ مُنْذُ صِبَايَ إِلَى هَذَا الْيَوْمِ.
\par 3 هَئَنَذَا فَاشْهَدُوا عَلَيَّ قُدَّامَ الرَّبِّ وَقُدَّامَ مَسِيحِهِ: ثَوْرَ مَنْ أَخَذْتُ, وَحِمَارَ مَنْ أَخَذْتُ, وَمَنْ ظَلَمْتُ, وَمَنْ سَحَقْتُ, وَمِنْ يَدِ مَنْ أَخَذْتُ فِدْيَةً لِأُغْضِيَ عَيْنَيَّ عَنْهُ, فَأَرُدَّ لَكُمْ؟»
\par 4 فَقَالُوا: «لَمْ تَظْلِمْنَا وَلاَ سَحَقْتَنَا وَلاَ أَخَذْتَ مِنْ يَدِ أَحَدٍ شَيْئاً».
\par 5 فَقَالَ لَهُمْ: «شَاهِدٌ الرَّبُّ عَلَيْكُمْ وَشَاهِدٌ مَسِيحُهُ الْيَوْمَ هَذَا, أَنَّكُمْ لَمْ تَجِدُوا بِيَدِي شَيْئاً». فَقَالُوا: «شَاهِدٌ».
\par 6 وَقَالَ صَمُوئِيلُ لِلشَّعْبِ: «الرَّبُّ الَّذِي أَقَامَ مُوسَى وَهَارُونَ, وَأَصْعَدَ آبَاءَكُمْ مِنْ أَرْضِ مِصْرَ.
\par 7 فَالآنَ امْثُلُوا فَأُحَاكِمَكُمْ أَمَامَ الرَّبِّ بِجَمِيعِ حُقُوقِ الرَّبِّ الَّتِي صَنَعَهَا مَعَكُمْ وَمَعَ آبَائِكُمْ.
\par 8 لَمَّا جَاءَ يَعْقُوبُ إِلَى مِصْرَ وَصَرَخَ آبَاؤُكُمْ إِلَى الرَّبِّ, أَرْسَلَ الرَّبُّ مُوسَى وَهَارُونَ فَأَخْرَجَا آبَاءَكُمْ مِنْ مِصْرَ وَأَسْكَنَاهُمْ فِي هَذَا الْمَكَانِ.
\par 9 فَلَمَّا نَسُوا الرَّبَّ إِلَهَهُمْ بَاعَهُمْ لِيَدِ سِيسَرَا رَئِيسِ جَيْشِ حَاصُورَ, وَلِيَدِ الْفِلِسْطِينِيِّينَ, وَلِيَدِ مَلِكِ مُوآبَ فَحَارَبُوهُمْ.
\par 10 فَصَرَخُوا إِلَى الرَّبِّ وَقَالُوا: أَخْطَأْنَا لأَنَّنَا تَرَكْنَا الرَّبَّ وَعَبَدْنَا الْبَعْلِيمَ وَالْعَشْتَارُوثَ. فَالآنَ أَنْقِذْنَا مِنْ يَدِ أَعْدَائِنَا فَنَعْبُدَكَ.
\par 11 فَأَرْسَلَ الرَّبُّ يَرُبَّعَلَ وَبَدَانَ وَيَفْتَاحَ وَصَمُوئِيلَ, وَأَنْقَذَكُمْ مِنْ يَدِ أَعْدَائِكُمُ الَّذِينَ حَوْلَكُمْ فَسَكَنْتُمْ آمِنِينَ.
\par 12 وَلَمَّا رَأَيْتُمْ نَاحَاشَ مَلِكَ بَنِي عَمُّونَ آتِياً عَلَيْكُمْ قُلْتُمْ لِي: لاَ بَلْ يَمْلِكُ عَلَيْنَا مَلِكٌ. وَالرَّبُّ إِلَهُكُمْ مَلِكُكُمْ.
\par 13 فَالآنَ هُوَذَا الْمَلِكُ الَّذِي اخْتَرْتُمُوهُ, الَّذِي طَلَبْتُمُوهُ, وَهُوَذَا قَدْ جَعَلَ الرَّبُّ عَلَيْكُمْ مَلِكاً.
\par 14 إِنِ اتَّقَيْتُمُ الرَّبَّ وَعَبَدْتُمُوهُ وَسَمِعْتُمْ صَوْتَهُ وَلَمْ تَعْصُوا قَوْلَ الرَّبِّ, وَكُنْتُمْ أَنْتُمْ وَالْمَلِكُ أَيْضاً الَّذِي يَمْلِكُ عَلَيْكُمْ وَرَاءَ الرَّبِّ إِلَهِكُمْ.
\par 15 وَإِنْ لَمْ تَسْمَعُوا صَوْتَ الرَّبِّ بَلْ عَصَيْتُمْ قَوْلَ الرَّبِّ تَكُنْ يَدُ الرَّبِّ عَلَيْكُمْ كَمَا عَلَى آبَائِكُمْ.
\par 16 فَالآنَ امْثُلُوا أَيْضاً وَانْظُرُوا هَذَا الأَمْرَ الْعَظِيمَ الَّذِي يَفْعَلُهُ الرَّبُّ أَمَامَ أَعْيُنِكُمْ.
\par 17 أَمَا هُوَ حَصَادُ الْحِنْطَةِ الْيَوْمَ؟ فَإِنِّي أَدْعُو الرَّبَّ فَيُعْطِي رُعُوداً وَمَطَراً فَتَعْلَمُونَ وَتَرُونَ أَنَّهُ عَظِيمٌ شَرُّكُمُ الَّذِي عَمِلْتُمُوهُ فِي عَيْنَيِ الرَّبِّ بِطَلَبِكُمْ لأَنْفُسِكُمْ مَلِكاً».
\par 18 فَدَعَا صَمُوئِيلُ الرَّبَّ فَأَعْطَى رُعُوداً وَمَطَراً فِي ذَلِكَ الْيَوْمِ. وَخَافَ جَمِيعُ الشَّعْبِ الرَّبَّ وَصَمُوئِيلَ جِدّاً.
\par 19 وَقَالَ جَمِيعُ الشَّعْبِ لِصَمُوئِيلَ: «صَلِّ عَنْ عَبِيدِكَ إِلَى الرَّبِّ إِلَهِكَ حَتَّى لاَ نَمُوتَ, لأَنَّنَا قَدْ أَضَفْنَا إِلَى جَمِيعِ خَطَايَانَا شَرّاً بِطَلَبِنَا لأَنْفُسِنَا مَلِكاً».
\par 20 فَقَالَ صَمُوئِيلُ لِلشَّعْبِ: «لاَ تَخَافُوا. إِنَّكُمْ قَدْ فَعَلْتُمْ كُلَّ هَذَا الشَّرِّ, وَلَكِنْ لاَ تَحِيدُوا عَنِ الرَّبِّ, بَلِ اعْبُدُوا الرَّبَّ بِكُلِّ قُلُوبِكُمْ
\par 21 وَلاَ تَحِيدُوا. لأَنَّ ذَلِكَ وَرَاءَ الأَبَاطِيلِ الَّتِي لاَ تُفِيدُ وَلاَ تُنْقِذُ, لأَنَّهَا بَاطِلَةٌ.
\par 22 لأَنَّهُ لاَ يَتْرُكُ الرَّبُّ شَعْبَهُ مِنْ أَجْلِ اسْمِهِ الْعَظِيمِ. لأَنَّهُ قَدْ شَاءَ الرَّبُّ أَنْ يَجْعَلَكُمْ لَهُ شَعْباً.
\par 23 وَأَمَّا أَنَا فَحَاشَا لِي أَنْ أُخْطِئَ إِلَى الرَّبِّ فَأَكُفَّ عَنِ الصَّلاَةِ مِنْ أَجْلِكُمْ, بَلْ أُعَلِّمُكُمُ الطَّرِيقَ الصَّالِحَ الْمُسْتَقِيمَ.
\par 24 إِنَّمَا اتَّقُوا الرَّبَّ وَاعْبُدُوهُ بِالأَمَانَةِ مِنْ كُلِّ قُلُوبِكُمْ, بَلِ انْظُرُوا فِعْلَهُ الَّذِي عَظَّمَهُ مَعَكُمْ.
\par 25 وَإِنْ فَعَلْتُمْ شَرّاً فَإِنَّكُمْ تَهْلِكُونَ أَنْتُمْ وَمَلِكُكُمْ جَمِيعاً».

\chapter{13}

\par 1 كَانَ شَاوُلُ ابْنَ سَنَةٍ فِي مُلْكِهِ, وَمَلَكَ سَنَتَيْنِ عَلَى إِسْرَائِيلَ.
\par 2 وَاخْتَارَ شَاوُلُ لِنَفْسِهِ ثَلاَثَةَ آلاَفٍ مِنْ إِسْرَائِيلَ, فَكَانَ أَلْفَانِ مَعَ شَاوُلَ فِي مِخْمَاسَ وَفِي جَبَلِ بَيْتِ إِيلَ, وَأَلْفٌ كَانَ مَعَ يُونَاثَانَ فِي جِبْعَةِ بِنْيَامِينَ. وَأَمَّا بَقِيَّةُ الشَّعْبِ فَأَرْسَلَهُمْ كُلَّ وَاحِدٍ إِلَى خَيْمَتِهِ.
\par 3 وَضَرَبَ يُونَاثَانُ نَصَبَ الْفِلِسْطِينِيِّينَ الَّذِي فِي جَبْعَ. فَسَمِعَ الْفِلِسْطِينِيُّونَ. وَضَرَبَ شَاوُلُ بِالْبُوقِ فِي جَمِيعِ الأَرْضِ قَائِلاً: «لِيَسْمَعِ الْعِبْرَانِيُّونَ».
\par 4 فَسَمِعَ جَمِيعُ إِسْرَائِيلَ قَوْلاً: «قَدْ ضَرَبَ شَاوُلُ نَصَبَ الْفِلِسْطِينِيِّينَ, وَأَيْضاً قَدْ أَنْتَنَ إِسْرَائِيلُ لَدَى الْفِلِسْطِينِيِّينَ». فَاجْتَمَعَ الشَّعْبُ وَرَاءَ شَاوُلَ إِلَى الْجِلْجَالِ.
\par 5 وَتَجَمَّعَ الْفِلِسْطِينِيُّونَ لِمُحَارَبَةِ إِسْرَائِيلَ. ثَلاَثُونَ أَلْفَ مَرْكَبَةٍ, وَسِتَّةُ آلاَفِ فَارِسٍ, وَشَعْبٌ كَالرَّمْلِ الَّذِي عَلَى شَاطِئِ الْبَحْرِ فِي الْكَثْرَةِ. وَصَعِدُوا وَنَزَلُوا فِي مِخْمَاسَ شَرْقِيَّ بَيْتِ آوِنَ.
\par 6 وَلَمَّا رَأَى رِجَالُ إِسْرَائِيلَ أَنَّهُمْ فِي ضَنْكٍ (لأَنَّ الشَّعْبَ تَضَايَقَ) اخْتَبَأَ الشَّعْبُ فِي الْمَغَايِرِ وَالْغِيَاضِ وَالصُّخُورِ وَالصُّرُوحِ وَالآبَارِ.
\par 7 وَبَعْضُ الْعِبْرَانِيِّينَ عَبَرُوا الأُرْدُنَّ إِلَى أَرْضِ جَادَ وَجِلْعَادَ. وَكَانَ شَاوُلُ بَعْدُ فِي الْجِلْجَالِ وَكُلُّ الشَّعْبِ ارْتَعَدَ وَرَاءَهُ.
\par 8 فَمَكَثَ سَبْعَةَ أَيَّامٍ حَسَبَ مِيعَادِ صَمُوئِيلَ, وَلَمْ يَأْتِ صَمُوئِيلُ إِلَى الْجِلْجَالِ, وَالشَّعْبُ تَفَرَّقَ عَنْهُ.
\par 9 فَقَالَ شَاوُلُ: «قَدِّمُوا إِلَيَّ الْمُحْرَقَةَ وَذَبَائِحَ السَّلاَمَةِ». فَأَصْعَدَ الْمُحْرَقَةَ.
\par 10 وَكَانَ لَمَّا انْتَهَى مِنْ إِصْعَادِ الْمُحْرَقَةِ إِذَا صَمُوئِيلُ مُقْبِلٌ, فَخَرَجَ شَاوُلُ لِلِقَائِهِ لِيُبَارِكَهُ.
\par 11 فَقَالَ صَمُوئِيلُ: «مَاذَا فَعَلْتَ؟» فَقَالَ شَاوُلُ: «لأَنِّي رَأَيْتُ أَنَّ الشَّعْبَ قَدْ تَفَرَّقَ عَنِّي, وَأَنْتَ لَمْ تَأْتِ فِي أَيَّامِ الْمِيعَادِ, وَالْفِلِسْطِينِيُّونَ مُتَجَمِّعُونَ فِي مِخْمَاسَ
\par 12 فَقُلْتُ: الآنَ يَنْزِلُ الْفِلِسْطِينِيُّونَ إِلَيَّ إِلَى الْجِلْجَالِ وَلَمْ أَتَضَرَّعْ إِلَى وَجْهِ الرَّبِّ, فَتَجَلَّدْتُ وَأَصْعَدْتُ الْمُحْرَقَةَ».
\par 13 فَقَالَ صَمُوئِيلُ لِشَاوُلَ: «قَدِ انْحَمَقْتَ! لَمْ تَحْفَظْ وَصِيَّةَ الرَّبِّ إِلَهِكَ الَّتِي أَمَرَكَ بِهَا, لأَنَّهُ الآنَ كَانَ الرَّبُّ قَدْ ثَبَّتَ مَمْلَكَتَكَ عَلَى إِسْرَائِيلَ إِلَى الأَبَدِ.
\par 14 وَأَمَّا الآنَ فَمَمْلَكَتُكَ لاَ تَقُومُ. قَدِ انْتَخَبَ الرَّبُّ لِنَفْسِهِ رَجُلاً حَسَبَ قَلْبِهِ, وَأَمَرَهُ الرَّبُّ أَنْ يَتَرَأَّسَ عَلَى شَعْبِهِ. لأَنَّكَ لَمْ تَحْفَظْ مَا أَمَرَكَ بِهِ الرَّبُّ».
\par 15 وَقَامَ صَمُوئِيلُ وَصَعِدَ مِنَ الْجِلْجَالِ إِلَى جِبْعَةِ بِنْيَامِينَ. وَعَدَّ شَاوُلُ الشَّعْبَ الْمَوْجُودَ مَعَهُ نَحْوَ سِتِّ مِئَةِ رَجُلٍ.
\par 16 وَكَانَ شَاوُلُ وَيُونَاثَانُ ابْنُهُ وَالشَّعْبُ الْمَوْجُودُ مَعَهُمَا مُقِيمِينَ فِي جَبْعِ بِنْيَامِينَ, وَالْفِلِسْطِينِيُّونَ نَزَلُوا فِي مِخْمَاسَ.
\par 17 فَخَرَجَ الْمُخَرِّبُونَ مِنْ مَحَلَّةِ الْفِلِسْطِينِيِّينَ فِي ثَلاَثِ فِرَقٍ. الْفِرْقَةُ الْوَاحِدَةُ تَوَجَّهَتْ فِي طَرِيقِ عَفْرَةَ إِلَى أَرْضِ شُوعَالَ,
\par 18 وَالْفِرْقَةُ الأُخْرَى تَوَجَّهَتْ فِي طَرِيقِ بَيْتِ حُورُونَ, وَالْفِرْقَةُ الأُخْرَى تَوَجَّهَتْ فِي طَرِيقِ التُّخْمِ الْمُشْرِفِ عَلَى وَادِي صَبُوعِيمَ نَحْوَ الْبَرِّيَّةِ.
\par 19 وَلَمْ يُوجَدْ صَانِعٌ فِي كُلِّ أَرْضِ إِسْرَائِيلَ, لأَنَّ الْفِلِسْطِينِيِّينَ قَالُوا: لِئَلَّا يَعْمَلَ الْعِبْرَانِيُّونَ سَيْفاً أَوْ رُمْحاً.
\par 20 بَلْ كَانَ يَنْزِلُ كُلُّ إِسْرَائِيلَ إِلَى الْفِلِسْطِينِيِّينَ لِيُحَدِّدَ كُلُّ وَاحِدٍ سِكَّتَهُ وَمِنْجَلَهُ وَفَأْسَهُ وَمِعْوَلَهُ
\par 21 عِنْدَمَا كَلَّتْ حُدُودُ السِّكَكِ وَالْمَنَاجِلِ وَالْمُثَلَّثَاتِ الأَسْنَانِ وَالْفُؤُوسِ وَلِتَرْوِيسِ الْمَنَاسِيسِ.
\par 22 وَكَانَ فِي يَوْمِ الْحَرْبِ أَنَّهُ لَمْ يُوجَدْ سَيْفٌ وَلاَ رُمْحٌ بِيَدِ جَمِيعِ الشَّعْبِ الَّذِي مَعَ شَاوُلَ وَمَعَ يُونَاثَانَ. عَلَى أَنَّهُ وُجِدَ مَعَ شَاوُلَ وَيُونَاثَانَ ابْنِهِ.
\par 23 وَخَرَجَ حَفَظَةُ الْفِلِسْطِينِيِّينَ إِلَى مَعْبَرِ مِخْمَاسَ.

\chapter{14}

\par 1 وَفِي ذَاتِ يَوْمٍ قَالَ يُونَاثَانُ بْنُ شَاوُلَ لِلْغُلاَمِ حَامِلِ سِلاَحِهِ: «تَعَالَ نَعْبُرْ إِلَى حَفَظَةِ الْفِلِسْطِينِيِّينَ الَّذِينَ فِي ذَلِكَ الْعَبْرِ». وَلَمْ يُخْبِرْ أَبَاهُ.
\par 2 وَكَانَ شَاوُلُ مُقِيماً فِي طَرَفِ جِبْعَةَ تَحْتَ الرُّمَّانَةِ الَّتِي فِي مِغْرُونَ, وَالشَّعْبُ الَّذِي مَعَهُ نَحْوُ سِتِّ مِئَةِ رَجُلٍ.
\par 3 وَأَخِيَّا بْنُ أَخِيطُوبَ أَخِي إِيخَابُودَ بْنِ فِينَحَاسَ بْنِ عَالِي كَاهِنُ الرَّبِّ فِي شِيلُوهَ كَانَ لاَبِساً أَفُوداً. وَلَمْ يَعْلَمِ الشَّعْبُ أَنَّ يُونَاثَانَ قَدْ ذَهَبَ.
\par 4 وَبَيْنَ الْمَعَابِرِ الَّتِي الْتَمَسَ يُونَاثَانُ أَنْ يَعْبُرَهَا إِلَى حَفَظَةِ الْفِلِسْطِينِيِّينَ سِنُّ صَخْرَةٍ مِنْ هَذِهِ الْجِهَةِ وَسِنُّ صَخْرَةٍ مِنْ تِلْكَ الْجِهَةِ, وَاسْمُ الْوَاحِدَةِ «بُوصَيْصُ» وَاسْمُ الأُخْرَى «سَنَهُ».
\par 5 وَالسِّنُّ الْوَاحِدُ عَمُودٌ إِلَى الشِّمَالِ مُقَابَِلَ مِخْمَاسَ, وَالآخَرُ إِلَى الْجَنُوبِ مُقَابَِلَ جِبْعَ.
\par 6 فَقَالَ يُونَاثَانُ لِلْغُلاَمِ حَامِلِ سِلاَحِهِ: «تَعَالَ نَعْبُرْ إِلَى صَفِّ هَؤُلاَءِ الْغُلْفِ, لَعَلَّ اللَّهَ يَعْمَلُ مَعَنَا, لأَنَّهُ لَيْسَ لِلرَّبِّ مَانِعٌ عَنْ أَنْ يُخَلِّصَ بِالْكَثِيرِ أَوْ بِالْقَلِيلِ».
\par 7 فَقَالَ لَهُ حَامِلُ سِلاَحِهِ: «اعْمَلْ كُلَّ مَا بِقَلْبِكَ. تَقَدَّمْ. هَئَنَذَا مَعَكَ حَسَبَ قَلْبِكَ».
\par 8 فَقَالَ يُونَاثَانُ: «هُوَذَا نَحْنُ نَعْبُرُ إِلَى الْقَوْمِ وَنُظْهِرُ أَنْفُسَنَا لَهُمْ.
\par 9 فَإِنْ قَالُوا لَنَا: دُومُوا حَتَّى نَصِلَ إِلَيْكُمْ. نَقِفُ فِي مَكَانِنَا وَلاَ نَصْعَدُ إِلَيْهِمْ.
\par 10 وَلَكِنْ إِنْ قَالُوا: «اصْعَدُوا إِلَيْنَا. نَصْعَدُ, لأَنَّ الرَّبَّ قَدْ دَفَعَهُمْ لِيَدِنَا, وَهَذِهِ هِيَ الْعَلاَمَةُ لَنَا.
\par 11 فَأَظْهَرَا أَنْفُسَهُمَا لِصَفِّ الْفِلِسْطِينِيِّينَ. فَقَالَ الْفِلِسْطِينِيُّونَ: «هُوَذَا الْعِبْرَانِيُّونَ خَارِجُونَ مِنَ الثُّقُوبِ الَّتِي اخْتَبَأُوا فِيهَا».
\par 12 فَأَجَابَ رِجَالُ الصَّفِّ يُونَاثَانَ وَحَامِلَ سِلاَحِهِ: «اِصْعَدَا إِلَيْنَا فَنُعَلِّمَكُمَا شَيْئاً». فَقَالَ يُونَاثَانُ لِحَامِلِ سِلاَحِهِ: «اصْعَدْ وَرَائِي لأَنَّ الرَّبَّ قَدْ دَفَعَهُمْ لِيَدِ إِسْرَائِيلَ».
\par 13 فَصَعِدَ يُونَاثَانُ عَلَى يَدَيْهِ وَرِجْلَيْهِ وَحَامِلُ سِلاَحِهِ وَرَاءَهُ. فَسَقَطُوا أَمَامَ يُونَاثَانَ, وَكَانَ حَامِلُ سِلاَحِهِ يُقَتِّلُ وَرَاءَهُ.
\par 14 وَكَانَتِ الضَّرْبَةُ الأُولَى الَّتِي ضَرَبَهَا يُونَاثَانُ وَحَامِلُ سِلاَحِهِ نَحْوَ عِشْرِينَ رَجُلاً فِي نَحْوِ نِصْفِ فَدَّانِ أَرْضٍ.
\par 15 وَكَانَ ارْتِعَادٌ فِي الْمَحَلَّةِ فِي الْحَقْلِ وَفِي جَمِيعِ الشَّعْبِ. الصَّفُّ وَالْمُخَرِّبُونَ ارْتَعَدُوا هُمْ أَيْضاً, وَرَجَفَتِ الأَرْضُ فَكَانَ ارْتِعَادٌ عَظِيمٌ.
\par 16 فَنَظَرَ الْمُرَاقِبُونَ لِشَاوُلَ فِي جِبْعَةِ بِنْيَامِينَ, وَإِذَا بِالْجُمْهُورِ قَدْ ذَابَ وَذَهَبُوا مُتَبَدِّدِينَ.
\par 17 فَقَالَ شَاوُلُ لِلشَّعْبِ الَّذِي مَعَهُ: «عُدُّوا الآنَ وَانْظُرُوا مَنْ ذَهَبَ مِنْ عِنْدِنَا». فَعَدُّوا, وَهُوَذَا يُونَاثَانُ وَحَامِلُ سِلاَحِهِ لَيْسَا مَوْجُودَيْنِ.
\par 18 فَقَالَ شَاوُلُ لأَخِيَّا: «قَدِّمْ تَابُوتَ اللَّهِ». (لأَنَّ تَابُوتَ اللَّهِ كَانَ فِي ذَلِكَ الْيَوْمِ مَعَ بَنِي إِسْرَائِيلَ).
\par 19 وَفِيمَا كَانَ شَاوُلُ يَتَكَلَّمُ بَعْدُ مَعَ الْكَاهِنِ, تَزَايَدَ الضَّجِيجُ الَّذِي فِي مَحَلَّةِ الْفِلِسْطِينِيِّينَ وَكَثُرَ. فَقَالَ شَاوُلُ لِلْكَاهِنِ: «كُفَّ يَدَكَ».
\par 20 وَصَاحَ شَاوُلُ وَجَمِيعُ الشَّعْبِ الَّذِي مَعَهُ وَجَاءُوا إِلَى الْحَرْبِ, وَإِذَا بِسَيْفِ كُلِّ وَاحِدٍ عَلَى صَاحِبِهِ. اضْطِرَابٌ عَظِيمٌ جِدّاً.
\par 21 وَالْعِبْرَانِيُّونَ الَّذِينَ كَانُوا مَعَ الْفِلِسْطِينِيِّينَ مُنْذُ أَمْسِ وَمَا قَبْلَهُ, الَّذِينَ صَعِدُوا مَعَهُمْ إِلَى الْمَحَلَّةِ مِنْ حَوَالَيْهِمْ, صَارُوا هُمْ أَيْضاً مَعَ إِسْرَائِيلَ الَّذِينَ مَعَ شَاوُلَ وَيُونَاثَانَ.
\par 22 وَسَمِعَ جَمِيعُ رِجَالِ إِسْرَائِيلَ الَّذِينَ اخْتَبَأُوا فِي جَبَلِ أَفْرَايِمَ أَنَّ الْفِلِسْطِينِيِّينَ هَرَبُوا, فَشَدُّوا هُمْ أَيْضاً وَرَاءَهُمْ فِي الْحَرْبِ.
\par 23 فَخَلَّصَ الرَّبُّ إِسْرَائِيلَ فِي ذَلِكَ الْيَوْمِ. وَعَبَرَتِ الْحَرْبُ إِلَى بَيْتِ آوِنَ.
\par 24 وَضَنُكَ رِجَالُ إِسْرَائِيلَ فِي ذَلِكَ الْيَوْمِ لأَنَّ شَاوُلَ حَلَّفَ الشَّعْبَ قَائِلاً: «مَلْعُونٌ الرَّجُلُ الَّذِي يَأْكُلُ خُبْزاً إِلَى الْمَسَاءِ حَتَّى أَنْتَقِمَ مِنْ أَعْدَائِي». فَلَمْ يَذُقْ جَمِيعُ الشَّعْبِ خُبْزاً.
\par 25 وَجَاءَ كُلُّ الشَّعْبِ إِلَى الْوَعْرِ وَكَانَ عَسَلٌ عَلَى وَجْهِ الْحَقْلِ.
\par 26 وَلَمَّا دَخَلَ الشَّعْبُ الْوَعْرَ إِذَا بِالْعَسَلِ يَقْطُرُ وَلَمْ يَمُدَّ أَحَدٌ يَدَهُ إِلَى فَمِهِ, لأَنَّ الشَّعْبَ خَافَ مِنَ الْقَسَمِ.
\par 27 وَأَمَّا يُونَاثَانُ فَلَمْ يَسْمَعْ عِنْدَمَا اسْتَحْلَفَ أَبُوهُ الشَّعْبَ, فَمَدَّ طَرَفَ النُّشَّابَةِ الَّتِي بِيَدِهِ وَغَمَسَهُ فِي قَطْرِ الْعَسَلِ وَرَدَّ يَدَهُ إِلَى فَمِهِ فَاسْتَنَارَتْ عَيْنَاهُ.
\par 28 فَقَالَ وَاحِدٌ مِنَ الشَّعْبِ: «قَدْ حَلَّفَ أَبُوكَ الشَّعْبَ قَائِلاً: مَلْعُونٌ الرَّجُلُ الَّذِي يَأْكُلُ خُبْزاً الْيَوْمَ. فَأَعْيَا الشَّعْبُ».
\par 29 فَقَالَ يُونَاثَانُ: «قَدْ كَدَّرَ أَبِي الأَرْضَ. انْظُرُوا كَيْفَ اسْتَنَارَتْ عَيْنَايَ لأَنِّي ذُقْتُ قَلِيلاً مِنْ هَذَا الْعَسَلِ.
\par 30 فَكَمْ بِالْحَرِيِّ لَوْ أَكَلَ الْيَوْمَ الشَّعْبُ مِنْ غَنِيمَةِ أَعْدَائِهِمِ الَّتِي وَجَدُوا! أَمَا كَانَتِ الآنَ ضَرْبَةٌ أَعْظَمُ عَلَى الْفِلِسْطِينِيِّينَ؟»
\par 31 فَضَرَبُوا فِي ذَلِكَ الْيَوْمِ الْفِلِسْطِينِيِّينَ مِنْ مِخْمَاسَ إِلَى أَيَّلُونَ. وَأَعْيَا الشَّعْبُ جِدّاً.
\par 32 وَثَارَ الشَّعْبُ عَلَى الْغَنِيمَةِ, فَأَخَذُوا غَنَماً وَبَقَراً وَعُجُولاً, وَذَبَحُوا عَلَى الأَرْضِ وَأَكَلَ الشَّعْبُ عَلَى الدَّمِ.
\par 33 فَأَخْبَرُوا شَاوُلَ: «هُوَذَا الشَّعْبُ يُخْطِئُ إِلَى الرَّبِّ بِأَكْلِهِ عَلَى الدَّمِ». فَقَالَ: «قَدْ غَدَرْتُمْ. دَحْرِجُوا إِلَيَّ الآنَ حَجَراً كَبِيراً».
\par 34 وَقَالَ شَاوُلُ: «تَفَرَّقُوا بَيْنَ الشَّعْبِ وَقُولُوا لَهُمْ أَنْ يُقَدِّمُوا إِلَيَّ كُلُّ وَاحِدٍ ثَوْرَهُ وَكُلُّ وَاحِدٍ شَاتَهُ, وَاذْبَحُوا هَهُنَا وَكُلُوا وَلاَ تُخْطِئُوا إِلَى الرَّبِّ بِأَكْلِكُمْ مَعَ الدَّمِ». فَقَدَّمَ جَمِيعُ الشَّعْبِ كُلُّ وَاحِدٍ ثَوْرَهُ بِيَدِهِ فِي تِلْكَ اللَّيْلَةِ وَذَبَحُوا هُنَاكَ.
\par 35 وَبَنَى شَاوُلُ مَذْبَحاً لِلرَّبِّ. الَّذِي شَرَعَ بِبُنْيَانِهِ مَذْبَحاً لِلرَّبِّ.
\par 36 وَقَالَ شَاوُلُ: «لِنَنْزِلْ وَرَاءَ الْفِلِسْطِينِيِّينَ لَيْلاً وَنَنْهَبْهُمْ إِلَى ضُوءِ الصَّبَاحِ وَلاَ نُبْقِ مِنْهُمْ أَحَداً». فَقَالُوا: «افْعَلْ كُلَّ مَا يَحْسُنُ فِي عَيْنَيْكَ». وَقَالَ الْكَاهِنُ: «لِنَتَقَدَّمْ هُنَا إِلَى اللَّهِ».
\par 37 فَسَأَلَ شَاوُلُ اللَّهَ: «أَأَنْحَدِرُ وَرَاءَ الْفِلِسْطِينِيِّينَ؟ أَتَدْفَعُهُمْ لِيَدِ إِسْرَائِيلَ؟» فَلَمْ يُجِبْهُ فِي ذَلِكَ الْيَوْمِ.
\par 38 فَقَالَ شَاوُلُ: «تَقَدَّمُوا إِلَى هُنَا يَا جَمِيعَ وُجُوهِ الشَّعْبِ, وَاعْلَمُوا وَانْظُرُوا بِمَاذَا كَانَتْ هَذِهِ الْخَطِيَّةُ الْيَوْمَ.
\par 39 لأَنَّهُ حَيٌّ هُوَ الرَّبُّ مُخَلِّصُ إِسْرَائِيلَ, وَلَوْ كَانَتْ فِي يُونَاثَانَ ابْنِي فَإِنَّهُ يَمُوتُ مَوْتاً». وَلَمْ يَكُنْ مَنْ يُجِيبُهُ مِنْ كُلِّ الشَّعْبِ.
\par 40 فَقَالَ لِجَمِيعِ إِسْرَائِيلَ: «أَنْتُمْ تَكُونُونَ فِي جَانِبٍ وَأَنَا وَيُونَاثَانُ ابْنِي فِي جَانِبٍ». فَقَالَ الشَّعْبُ لِشَاوُلَ: «اصْنَعْ مَا يَحْسُنُ فِي عَيْنَيْكَ».
\par 41 وَقَالَ شَاوُلُ لِلرَّبِّ إِلَهِ إِسْرَائِيلَ: «هَبْ صِدْقاً». فَأُخِذَ يُونَاثَانُ وَشَاوُلُ. أَمَّا الشَّعْبُ فَخَرَجُوا.
\par 42 فَقَالَ شَاوُلُ: «أَلْقُوا بَيْنِي وَبَيْنَ يُونَاثَانَ ابْنِي. فَأُخِذَ يُونَاثَانُ».
\par 43 فَقَالَ شَاوُلُ لِيُونَاثَانَ: «أَخْبِرْنِي مَاذَا فَعَلْتَ!» فَأَخْبَرَهُ يُونَاثَانُ: «ذُقْتُ ذَوْقاً بِطَرَفِ النُّشَّابَةِ الَّتِي بِيَدِي قَلِيلَ عَسَلٍ. فَهَئَنَذَا أَمُوتُ».
\par 44 فَقَالَ شَاوُلُ: «هَكَذَا يَفْعَلُ اللَّهُ وَهَكَذَا يَزِيدُ إِنَّكَ مَوْتاً تَمُوتُ يَا يُونَاثَانُ».
\par 45 فَقَالَ الشَّعْبُ لِشَاوُلَ: «أَيَمُوتُ يُونَاثَانُ الَّذِي صَنَعَ هَذَا الْخَلاَصَ الْعَظِيمَ فِي إِسْرَائِيلَ؟ حَاشَا! حَيٌّ هُوَ الرَّبُّ لاَ تَسْقُطُ شَعْرَةٌ مِنْ رَأْسِهِ إِلَى الأَرْضِ لأَنَّهُ مَعَ اللَّهِ عَمِلَ هَذَا الْيَوْمَ». فَافْتَدَى الشَّعْبُ يُونَاثَانَ فَلَمْ يَمُتْ.
\par 46 فَصَعِدَ شَاوُلُ مِنْ وَرَاءِ الْفِلِسْطِينِيِّينَ, وَذَهَبَ الْفِلِسْطِينِيُّونَ إِلَى مَكَانِهِمْ.
\par 47 وَأَخَذَ شَاوُلُ الْمُلْكَ عَلَى إِسْرَائِيلَ, وَحَارَبَ جَمِيعَ أَعْدَائِهِ حَوَالَيْهِ: مُوآبَ وَبَنِي عَمُّونَ وَأَدُومَ, وَمُلُوكَ صُوبَةَ وَالْفِلِسْطِينِيِّينَ. وَحَيْثُمَا تَوَجَّهَ غَلَبَ.
\par 48 وَفَعَلَ بِبَأْسٍ وَضَرَبَ عَمَالِيقَ, وَأَنْقَذَ إِسْرَائِيلَ مِنْ يَدِ نَاهِبِيهِ.
\par 49 وَكَانَ بَنُو شَاوُلَ يُونَاثَانَ وَيِشْوِيَ وَمَلْكِيشُوعَ, وَاسْمَا ابْنَتَيْهِ: اسْمُ الْبِكْرِ مَيْرَبُ وَاسْمُ الصَّغِيرَةِ مِيكَالُ.
\par 50 وَاسْمُ امْرَأَةِ شَاوُلَ أَخِينُوعَمُ بِنْتُ أَخِيمَعَصَ. وَاسْمُ رَئِيسِ جَيْشِهِ أَبْنَيْرُ بْنُ نَيْرَ عَمِّ شَاوُلَ.
\par 51 وَقَيْسُ أَبُو شَاوُلَ وَنَيْرُ أَبُو أَبْنَيْرَ ابْنَا أَبِيئِيلَ.
\par 52 وَكَانَتْ حَرْبٌ شَدِيدَةٌ عَلَى الْفِلِسْطِينِيِّينَ كُلَّ أَيَّامِ شَاوُلَ. وَإِذَا رَأَى شَاوُلُ رَجُلاً جَبَّاراً أَوْ ذَا بَأْسٍ ضَمَّهُ إِلَى نَفْسِهِ.

\chapter{15}

\par 1 وَقَالَ صَمُوئِيلُ لِشَاوُلَ: «إِيَّايَ أَرْسَلَ الرَّبُّ لِمَسْحِكَ مَلِكاً عَلَى شَعْبِهِ إِسْرَائِيلَ. وَالآنَ فَاسْمَعْ صَوْتَ كَلاَمِ الرَّبِّ.
\par 2 هَكَذَا يَقُولُ رَبُّ الْجُنُودِ: إِنِّي قَدِ افْتَقَدْتُ مَا عَمِلَ عَمَالِيقُ بِإِسْرَائِيلَ حِينَ وَقَفَ لَهُ فِي الطَّرِيقِ عِنْدَ صُعُودِهِ مِنْ مِصْرَ.
\par 3 فَالآنَ اذْهَبْ وَاضْرِبْ عَمَالِيقَ وَحَرِّمُوا كُلَّ مَا لَهُ وَلاَ تَعْفُ عَنْهُمْ بَلِ اقْتُلْ رَجُلاً وَامْرَأَةً, طِفْلاً وَرَضِيعاً, بَقَراً وَغَنَماً, جَمَلاً وَحِمَاراً».
\par 4 فَاسْتَحْضَرَ شَاوُلُ الشَّعْبَ وَعَدَّهُ فِي طَلاَيِمَ, مِئَتَيْ أَلْفِ رَاجِلٍ وَعَشَرَةَ آلاَفِ رَجُلٍ مِنْ يَهُوذَا.
\par 5 ثُمَّ جَاءَ شَاوُلُ إِلَى مَدِينَةِ عَمَالِيقَ وَكَمَنَ فِي الْوَادِي.
\par 6 وَقَالَ شَاوُلُ لِلْقِيْنِيِّينَ: «اذْهَبُوا حِيدُوا انْزِلُوا مِنْ وَسَطِ الْعَمَالِقَةِ لِئَلَّا أُهْلِكَكُمْ مَعَهُمْ, وَأَنْتُمْ قَدْ فَعَلْتُمْ مَعْرُوفاً مَعَ جَمِيعِ بَنِي إِسْرَائِيلَ عِنْدَ صُعُودِهِمْ مِنْ مِصْرَ». فَحَادَ الْقِيْنِيُّ مِنْ وَسَطِ عَمَالِيقَ.
\par 7 وَضَرَبَ شَاوُلُ عَمَالِيقَ مِنْ حَوِيلَةَ حَتَّى مَجِيئِكَ إِلَى شُورَ الَّتِي مُقَابَِلَ مِصْرَ.
\par 8 وَأَمْسَكَ أَجَاجَ مَلِكَ عَمَالِيقَ حَيّاً, وَحَرَّمَ جَمِيعَ الشَّعْبِ بِحَدِّ السَّيْفِ.
\par 9 وَعَفَا شَاوُلُ وَالشَّعْبُ عَنْ أَجَاجَ وَعَنْ خِيَارِ الْغَنَمِ وَالْبَقَرِ وَالْحُمْلاَنِ وَالْخِرَافِ وَعَنْ كُلِّ الْجَيِّدِ, وَلَمْ يَرْضُوا أَنْ يُحَرِّمُوهَا. وَكُلُّ الأَمْلاَكِ الْمُحْتَقَرَةِ وَالْمَهْزُولَةِ حَرَّمُوهَا.
\par 10 وَكَانَ كَلاَمُ الرَّبِّ إِلَى صَمُوئِيلَ:
\par 11 «نَدِمْتُ عَلَى أَنِّي قَدْ جَعَلْتُ شَاوُلَ مَلِكاً, لأَنَّهُ رَجَعَ مِنْ وَرَائِي وَلَمْ يُقِمْ كَلاَمِي». فَاغْتَاظَ صَمُوئِيلُ وَصَرَخَ إِلَى الرَّبِّ اللَّيْلَ كُلَّهُ.
\par 12 فَبَكَّرَ صَمُوئِيلُ لِلِقَاءِ شَاوُلَ صَبَاحاً. فَأُخْبِرَ صَمُوئِيلُ: «قَدْ جَاءَ شَاوُلُ إِلَى الْكَرْمَلِ, وَهُوَذَا قَدْ نَصَبَ لِنَفْسِهِ نَصَباً وَدَارَ وَعَبَرَ وَنَزَلَ إِلَى الْجِلْجَالِ».
\par 13 وَلَمَّا جَاءَ صَمُوئِيلُ إِلَى شَاوُلَ قَالَ لَهُ شَاوُلُ: «مُبَارَكٌ أَنْتَ لِلرَّبِّ. قَدْ أَقَمْتُ كَلاَمَ الرَّبِّ».
\par 14 فَقَالَ صَمُوئِيلُ: «وَمَا هُوَ صَوْتُ الْغَنَمِ هَذَا فِي أُذُنَيَّ, وَصَوْتُ الْبَقَرِ الَّذِي أَنَا سَامِعٌ؟»
\par 15 فَقَالَ شَاوُلُ: «مِنَ الْعَمَالِقَةِ, قَدْ أَتُوا بِهَا لأَنَّ الشَّعْبَ قَدْ عَفَا عَنْ خِيَارِ الْغَنَمِ وَالْبَقَرِ لأَجْلِ الذَّبْحِ لِلرَّبِّ إِلَهِكَ. وَأَمَّا الْبَاقِي فَقَدْ حَرَّمْنَاهُ».
\par 16 فَقَالَ صَمُوئِيلُ لِشَاوُلَ: «كُفَّ فَأُخْبِرَكَ بِمَا تَكَلَّمَ بِهِ الرَّبُّ إِلَيَّ هَذِهِ اللَّيْلَةَ». فَقَالَ لَهُ: «تَكَلَّمْ».
\par 17 فَقَالَ صَمُوئِيلُ: «أَلَيْسَ إِذْ كُنْتَ صَغِيراً فِي عَيْنَيْكَ صِرْتَ رَأْسَ أَسْبَاطِ إِسْرَائِيلَ وَمَسَحَكَ الرَّبُّ مَلِكاً عَلَى إِسْرَائِيلَ,
\par 18 وَأَرْسَلَكَ الرَّبُّ فِي طَرِيقٍ وَقَالَ: اذْهَبْ وَحَرِّمِ الْخُطَاةَ عَمَالِيقَ وَحَارِبْهُمْ حَتَّى يَفْنُوا؟
\par 19 فَلِمَاذَا لَمْ تَسْمَعْ لِصَوْتِ الرَّبِّ, بَلْ ثُرْتَ عَلَى الْغَنِيمَةِ وَعَمِلْتَ الشَّرَّ فِي عَيْنَيِ الرَّبِّ؟»
\par 20 فَقَالَ شَاوُلُ لِصَمُوئِيلَ: «إِنِّي قَدْ سَمِعْتُ لِصَوْتِ الرَّبِّ وَذَهَبْتُ فِي الطَّرِيقِ الَّتِي أَرْسَلَنِي فِيهَا الرَّبُّ وَأَتَيْتُ بِأَجَاجَ مَلِكِ عَمَالِيقَ وَحَرَّمْتُ عَمَالِيقَ.
\par 21 فَأَخَذَ الشَّعْبُ مِنَ الْغَنِيمَةِ غَنَماً وَبَقَراً, أَوَائِلَ الْحَرَامِ لأَجْلِ الذَّبْحِ لِلرَّبِّ إِلَهِكَ فِي الْجِلْجَالِ».
\par 22 فَقَالَ صَمُوئِيلُ: «هَلْ مَسَرَّةُ الرَّبِّ بِالْمُحْرَقَاتِ وَالذَّبَائِحِ كَمَا بِاسْتِمَاعِ صَوْتِ الرَّبِّ؟ هُوَذَا الاِسْتِمَاعُ أَفْضَلُ مِنَ الذَّبِيحَةِ وَالْإِصْغَاءُ أَفْضَلُ مِنْ شَحْمِ الْكِبَاشِ.
\par 23 لأَنَّ التَّمَرُّدَ كَخَطِيَّةِ الْعِرَافَةِ, وَالْعِنَادُ كَالْوَثَنِ وَالتَّرَافِيمِ. لأَنَّكَ رَفَضْتَ كَلاَمَ الرَّبِّ رَفَضَكَ مِنَ الْمُلْكِ!».
\par 24 فَقَالَ شَاوُلُ لِصَمُوئِيلَ: «أَخْطَأْتُ لأَنِّي تَعَدَّيْتُ قَوْلَ الرَّبِّ وَكَلاَمَكَ, لأَنِّي خِفْتُ مِنَ الشَّعْبِ وَسَمِعْتُ لِصَوْتِهِمْ.
\par 25 وَالآنَ فَاغْفِرْ خَطِيَّتِي وَارْجِعْ مَعِي فَأَسْجُدَ لِلرَّبِّ».
\par 26 فَقَالَ صَمُوئِيلُ لِشَاوُلَ: «لاَ أَرْجِعُ مَعَكَ لأَنَّكَ رَفَضْتَ كَلاَمَ الرَّبِّ, فَرَفَضَكَ الرَّبُّ مِنْ أَنْ تَكُونَ مَلِكاً عَلَى إِسْرَائِيلَ».
\par 27 وَدَارَ صَمُوئِيلُ لِيَمْضِيَ, فَأَمْسَكَ بِذَيْلِ جُبَّتِهِ فَانْمَزَقَ.
\par 28 فَقَالَ لَهُ صَمُوئِيلُ: «يُمَزِّقُ الرَّبُّ مَمْلَكَةَ إِسْرَائِيلَ عَنْكَ الْيَوْمَ وَيُعْطِيهَا لِصَاحِبِكَ الَّذِي هُوَ خَيْرٌ مِنْكَ.
\par 29 وَأَيْضاً نَصِيحُ إِسْرَائِيلَ لاَ يَكْذِبُ وَلاَ يَنْدَمُ لأَنَّهُ لَيْسَ إِنْسَاناً لِيَنْدَمَ».
\par 30 فَقَالَ: «قَدْ أَخْطَأْتُ. وَالآنَ فَأَكْرِمْنِي أَمَامَ شُيُوخِ شَعْبِي وَأَمَامَ إِسْرَائِيلَ, وَارْجِعْ مَعِي فَأَسْجُدَ لِلرَّبِّ إِلَهِكَ».
\par 31 فَرَجَعَ صَمُوئِيلُ وَرَاءَ شَاوُلَ وَسَجَدَ شَاوُلُ لِلرَّبِّ.
\par 32 وَقَالَ صَمُوئِيلُ: «قَدِّمُوا إِلَيَّ أَجَاجَ مَلِكَ عَمَالِيقَ». فَذَهَبَ إِلَيْهِ أَجَاجُ فَرِحاً. وَقَالَ أَجَاجُ: «حَقّاً قَدْ زَالَتْ مَرَارَةُ الْمَوْتِ».
\par 33 فَقَالَ صَمُوئِيلُ: «كَمَا أَثْكَلَ سَيْفُكَ النِّسَاءَ كَذَلِكَ تُثْكَلُ أُمُّكَ بَيْنَ النِّسَاءِ». فَقَطَعَ صَمُوئِيلُ أَجَاجَ أَمَامَ الرَّبِّ فِي الْجِلْجَالِ.
\par 34 وَذَهَبَ صَمُوئِيلُ إِلَى الرَّامَةِ. وَأَمَّا شَاوُلُ فَصَعِدَ إِلَى بَيْتِهِ فِي جِبْعَةِ شَاوُلَ.
\par 35 وَلَمْ يَعُدْ صَمُوئِيلُ لِرُؤْيَةِ شَاوُلَ إِلَى يَوْمِ مَوْتِهِ, لأَنَّ صَمُوئِيلَ نَاحَ عَلَى شَاوُلَ, وَالرَّبُّ نَدِمَ لأَنَّهُ مَلَّكَ شَاوُلَ عَلَى إِسْرَائِيلَ.

\chapter{16}

\par 1 فَقَالَ الرَّبُّ لِصَمُوئِيلَ: «حَتَّى مَتَى تَنُوحُ عَلَى شَاوُلَ, وَأَنَا قَدْ رَفَضْتُهُ عَنْ أَنْ يَمْلِكَ عَلَى إِسْرَائِيلَ؟ امْلَأْ قَرْنَكَ دُهْناً وَتَعَالَ أُرْسِلْكَ إِلَى يَسَّى الْبَيْتَلَحْمِيِّ, لأَنِّي قَدْ رَأَيْتُ لِي فِي بَنِيهِ مَلِكاً».
\par 2 فَقَالَ صَمُوئِيلُ: «كَيْفَ أَذْهَبُ؟ إِنْ سَمِعَ شَاوُلُ يَقْتُلُنِي». فَقَالَ الرَّبُّ: «خُذْ بِيَدِكَ عِجْلَةً مِنَ الْبَقَرِ وَقُلْ: قَدْ جِئْتُ لأَذْبَحَ لِلرَّبِّ.
\par 3 وَادْعُ يَسَّى إِلَى الذَّبِيحَةِ, وَأَنَا أُعَلِّمُكَ مَاذَا تَصْنَعُ. وَامْسَحْ لِيَ الَّذِي أَقُولُ لَكَ عَنْهُ».
\par 4 فَفَعَلَ صَمُوئِيلُ كَمَا تَكَلَّمَ الرَّبُّ وَجَاءَ إِلَى بَيْتِ لَحْمٍ. فَارْتَعَدَ شُيُوخُ الْمَدِينَةِ عِنْدَ اسْتِقْبَالِهِ وَقَالُوا: «أَسَلاَمٌ مَجِيئُكَ؟»
\par 5 فَقَالَ: «سَلاَمٌ. قَدْ جِئْتُ لأَذْبَحَ لِلرَّبِّ. تَقَدَّسُوا وَتَعَالُوا مَعِي إِلَى الذَّبِيحَةِ». وَقَدَّسَ يَسَّى وَبَنِيهِ وَدَعَاهُمْ إِلَى الذَّبِيحَةِ.
\par 6 وَكَانَ لَمَّا جَاءُوا أَنَّهُ رَأَى أَلِيآبَ, فَقَالَ: «إِنَّ أَمَامَ الرَّبِّ مَسِيحَهُ».
\par 7 فَقَالَ الرَّبُّ لِصَمُوئِيلَ: «لاَ تَنْظُرْ إِلَى مَنْظَرِهِ وَطُولِ قَامَتِهِ لأَنِّي قَدْ رَفَضْتُهُ. لأَنَّهُ لَيْسَ كَمَا يَنْظُرُ الْإِنْسَانُ. لأَنَّ الْإِنْسَانَ يَنْظُرُ إِلَى الْعَيْنَيْنِ, وَأَمَّا الرَّبُّ فَإِنَّهُ يَنْظُرُ إِلَى الْقَلْبِ».
\par 8 فَدَعَا يَسَّى أَبِينَادَابَ وَعَبَّرَهُ أَمَامَ صَمُوئِيلَ, فَقَالَ: «وَهَذَا أَيْضاً لَمْ يَخْتَرْهُ الرَّبُّ».
\par 9 وَعَبَّرَ يَسَّى شَمَّةَ, فَقَالَ: «وَهَذَا أَيْضاً لَمْ يَخْتَرْهُ الرَّبُّ».
\par 10 وَعَبَّرَ يَسَّى بَنِيهِ السَّبْعَةَ أَمَامَ صَمُوئِيلَ, فَقَالَ صَمُوئِيلُ لِيَسَّى: «الرَّبُّ لَمْ يَخْتَرْ هَؤُلاَءِ».
\par 11 وَقَالَ صَمُوئِيلُ لِيَسَّى: «هَلْ كَمُلَ الْغِلْمَانُ؟» فَقَالَ: «بَقِيَ بَعْدُ الصَّغِيرُ وَهُوَذَا يَرْعَى الْغَنَمَ». فَقَالَ صَمُوئِيلُ لِيَسَّى: «أَرْسِلْ وَأْتِ بِهِ, لأَنَّنَا لاَ نَجْلِسُ حَتَّى يَأْتِيَ إِلَى هَهُنَا».
\par 12 فَأَرْسَلَ وَأَتَى بِهِ. وَكَانَ أَشْقَرَ مَعَ حَلاَوَةِ الْعَيْنَيْنِ وَحَسَنَ الْمَنْظَرِ. فَقَالَ الرَّبُّ: «قُمِ امْسَحْهُ لأَنَّ هَذَا هُوَ».
\par 13 فَأَخَذَ صَمُوئِيلُ قَرْنَ الدُّهْنِ وَمَسَحَهُ فِي وَسَطِ إِخْوَتِهِ. وَحَلَّ رُوحُ الرَّبِّ عَلَى دَاوُدَ مِنْ ذَلِكَ الْيَوْمِ فَصَاعِداً. ثُمَّ قَامَ صَمُوئِيلُ وَذَهَبَ إِلَى الرَّامَةِ.
\par 14 وَذَهَبَ رُوحُ الرَّبِّ مِنْ عِنْدِ شَاوُلَ, وَبَغَتَهُ رُوحٌ رَدِيءٌ مِنْ قِبَلِ الرَّبِّ.
\par 15 فَقَالَ عَبِيدُ شَاوُلَ لَهُ: «هُوَذَا رُوحٌ رَدِيءٌ مِنْ قِبَلِ اللَّهِ يَبْغَتُكَ.
\par 16 فَلْيَأْمُرْ سَيِّدُنَا عَبِيدَهُ قُدَّامَهُ أَنْ يُفَتِّشُوا عَلَى رَجُلٍ يُحْسِنُ الضَّرْبَ بِالْعُودِ. وَيَكُونُ إِذَا كَانَ عَلَيْكَ الرُّوحُ الرَّدِيءُ مِنْ قِبَلِ اللَّهِ أَنَّهُ يَضْرِبُ بِيَدِهِ فَتَطِيبُ».
\par 17 فَقَالَ شَاوُلُ لِعَبِيدِهِ: «انْظُرُوا لِي رَجُلاً يُحْسِنُ الضَّرْبَ وَأْتُوا بِهِ إِلَيَّ».
\par 18 فَأَجَابَ وَاحِدٌ مِنَ الْغِلْمَانِ: «هُوَذَا قَدْ رَأَيْتُ ابْناً لِيَسَّى الْبَيْتَلَحْمِيِّ يُحْسِنُ الضَّرْبَ, وَهُوَ جَبَّارُ بَأْسٍ وَرَجُلُ حَرْبٍ وَفَصِيحٌ وَرَجُلٌ جَمِيلٌ, وَالرَّبُّ مَعَهُ».
\par 19 فَأَرْسَلَ شَاوُلُ رُسُلاً إِلَى يَسَّى يَقُولُ: «أَرْسِلْ إِلَيَّ دَاوُدَ ابْنَكَ الَّذِي مَعَ الْغَنَمِ».
\par 20 فَأَخَذَ يَسَّى حِمَاراً حَامِلاً خُبْزاً وَزِقَّ خَمْرٍ وَجَدْيَ مِعْزىً وَأَرْسَلَهَا بِيَدِ دَاوُدَ ابْنِهِ إِلَى شَاوُلَ.
\par 21 فَجَاءَ دَاوُدُ إِلَى شَاوُلَ وَوَقَفَ أَمَامَهُ, فَأَحَبَّهُ جِدّاً وَكَانَ لَهُ حَامِلَ سِلاَحٍ.
\par 22 فَأَرْسَلَ شَاوُلُ إِلَى يَسَّى يَقُولُ: «لِيَقِفْ دَاوُدُ أَمَامِي لأَنَّهُ وَجَدَ نِعْمَةً فِي عَيْنَيَّ».
\par 23 وَكَانَ عِنْدَمَا جَاءَ الرُّوحُ مِنْ قِبَلِ اللَّهِ عَلَى شَاوُلَ أَنَّ دَاوُدَ أَخَذَ الْعُودَ وَضَرَبَ بِيَدِهِ, فَكَانَ شَاوُلُ يَرْتَاحُ وَيَطِيبُ وَيَذْهَبُ عَنْهُ الرُّوحُ الرَّدِيءُ.

\chapter{17}

\par 1 وَجَمَعَ الْفِلِسْطِينِيُّونَ جُيُوشَهُمْ لِلْحَرْبِ فَاجْتَمَعُوا فِي سُوكُوهَ الَّتِي لِيَهُوذَا, وَنَزَلُوا بَيْنَ سُوكُوهَ وَعَزِيقَةَ فِي أَفَسِ دَمِّيمَ.
\par 2 وَاجْتَمَعَ شَاوُلُ وَرِجَالُ إِسْرَائِيلَ وَنَزَلُوا فِي وَادِي الْبُطْمِ, وَاصْطَفُّوا لِلْحَرْبِ لِلِقَاءِ الْفِلِسْطِينِيِّينَ.
\par 3 وَكَانَ الْفِلِسْطِينِيُّونَ وُقُوفاً عَلَى جَبَلٍ مِنْ هُنَا وَإِسْرَائِيلُ وُقُوفاً عَلَى جَبَلٍ مِنْ هُنَاكَ, وَالْوَادِي بَيْنَهُمْ.
\par 4 فَخَرَجَ رَجُلٌ مُبَارِزٌ مِنْ جُيُوشِ الْفِلِسْطِينِيِّينَ اسْمُهُ جُلْيَاتُ, مِنْ جَتَّ, طُولُهُ سِتُّ أَذْرُعٍ وَشِبْرٌ,
\par 5 وَعَلَى رَأْسِهِ خُوذَةٌ مِنْ نُحَاسٍ, وَكَانَ لاَبِساً دِرْعاً حَرْشَفِيّاً وَزْنُهُ خَمْسَةُ آلاَفِ شَاقِلِ نُحَاسٍ.
\par 6 وَجُرْمُوقَا نُحَاسٍ عَلَى رِجْلَيْهِ, وَحَرْبَةُ نُحَاسٍ بَيْنَ كَتِفَيْهِ.
\par 7 وَقَنَاةُ رُمْحِهِ كَنَوْلِ النَّسَّاجِينَ, وَسِنَانُ رُمْحِهِ سِتُّ مِئَةِ شَاقِلِ حَدِيدٍ, وَحَامِلُ التُّرْسِ كَانَ يَمْشِي قُدَّامَهُ.
\par 8 فَوَقَفَ وَنَادَى صُفُوفَ إِسْرَائِيلَ: «لِمَاذَا تَخْرُجُونَ لِتَصْطَفُّوا لِلْحَرْبِ؟ أَمَا أَنَا الْفِلِسْطِينِيُّ, وَأَنْتُمْ عَبِيدٌ لِشَاوُلَ؟ اخْتَارُوا لأَنْفُسِكُمْ رَجُلاً وَلْيَنْزِلْ إِلَيَّ.
\par 9 فَإِنْ قَدِرَ أَنْ يُحَارِبَنِي وَيَقْتُلَنِي نَصِيرُ لَكُمْ عَبِيداً. وَإِنْ قَدِرْتُ أَنَا عَلَيْهِ وَقَتَلْتُهُ تَصِيرُونَ أَنْتُمْ لَنَا عَبِيداً وَتَخْدِمُونَنَا».
\par 10 وَقَالَ الْفِلِسْطِينِيُّ: «أَنَا عَيَّرْتُ صُفُوفَ إِسْرَائِيلَ هَذَا الْيَوْمَ. أَعْطُونِي رَجُلاً فَنَتَحَارَبَ مَعاً».
\par 11 وَلَمَّا سَمِعَ شَاوُلُ وَجَمِيعُ إِسْرَائِيلَ كَلاَمَ الْفِلِسْطِينِيِّ هَذَا ارْتَاعُوا وَخَافُوا جِدّاً.
\par 12 وَدَاوُدُ هُوَ ابْنُ ذَلِكَ الرَّجُلِ الأَفْرَاتِيِّ مِنْ بَيْتِ لَحْمِ يَهُوذَا الَّذِي اسْمُهُ يَسَّى وَلَهُ ثَمَانِيَةُ بَنِينَ. وَكَانَ الرَّجُلُ فِي أَيَّامِ شَاوُلَ قَدْ شَاخَ وَكَبِرَ بَيْنَ النَّاسِ.
\par 13 وَذَهَبَ بَنُو يَسَّى الثَّلاَثَةُ الْكِبَارُ وَتَبِعُوا شَاوُلَ إِلَى الْحَرْبِ وَأَسْمَاءُ بَنِيهِ الثَّلاَثَةِ الَّذِينَ ذَهَبُوا إِلَى الْحَرْبِ: أَلِيآبُ الْبِكْرُ, وَأَبِينَادَابُ ثَانِيهِ, وَشَمَّةُ ثَالِثُهُمَا.
\par 14 وَدَاوُدُ هُوَ الصَّغِيرُ وَالثَّلاَثَةُ الْكِبَارُ ذَهَبُوا وَرَاءَ شَاوُلَ.
\par 15 وَأَمَّا دَاوُدُ فَكَانَ يَذْهَبُ وَيَرْجِعُ مِنْ عِنْدِ شَاوُلَ لِيَرْعَى غَنَمَ أَبِيهِ فِي بَيْتِ لَحْمٍ.
\par 16 وَكَانَ الْفِلِسْطِينِيُّ يَتَقَدَّمُ وَيَقِفُ صَبَاحاً وَمَسَاءً أَرْبَعِينَ يَوْماً.
\par 17 فَقَالَ يَسَّى لِدَاوُدَ ابْنِهِ: «خُذْ لإِخْوَتِكَ إِيفَةً مِنْ هَذَا الْفَرِيكِ, وَهَذِهِ الْعَشَرَ الْخُبْزَاتِ وَارْكُضْ إِلَى الْمَحَلَّةِ إِلَى إِخْوَتِكَ.
\par 18 وَهَذِهِ الْعَشَرَ الْقِطْعَاتِ مِنَ الْجُبْنِ قَدِّمْهَا لِرَئِيسِ الأَلْفِ, وَافْتَقِدْ سَلاَمَةَ إِخْوَتِكَ وَخُذْ مِنْهُمْ عَرْبُوناً».
\par 19 وَكَانَ شَاوُلُ وَهُمْ وَجَمِيعُ رِجَالِ إِسْرَائِيلَ فِي وَادِي الْبُطْمِ يُحَارِبُونَ الْفِلِسْطِينِيِّينَ.
\par 20 فَبَكَّرَ دَاوُدُ صَبَاحاً وَتَرَكَ الْغَنَمَ مَعَ حَارِسٍ وَحَمَّلَ وَذَهَبَ كَمَا أَمَرَهُ يَسَّى, وَأَتَى إِلَى الْمِتْرَاسِ وَالْجَيْشُ خَارِجٌ إِلَى الاِصْطِفَافِ وَهَتَفُوا لِلْحَرْبِ.
\par 21 وَاصْطَفَّ إِسْرَائِيلُ وَالْفِلِسْطِينِيُّونَ صَفّاً مُقَابَِلَ صَفٍّ.
\par 22 فَتَرَكَ دَاوُدُ الأَمْتِعَةَ الَّتِي مَعَهُ بِيَدِ حَافِظِ الأَمْتِعَةِ وَرَكَضَ إِلَى الصَّفِّ وَأَتَى وَسَأَلَ عَنْ سَلاَمَةِ إِخْوَتِهِ.
\par 23 وَفِيمَا هُوَ يُكَلِّمُهُمْ إِذَا بِرَجُلٍ مُبَارِزٍ اسْمُهُ جُلْيَاتُ الْفِلِسْطِينِيُّ مِنْ جَتَّ صَاعِدٌ مِنْ صُفُوفِ الْفِلِسْطِينِيِّينَ وَتَكَلَّمَ بِمِثْلِ هَذَا الْكَلاَمِ, فَسَمِعَ دَاوُدُ.
\par 24 وَجَمِيعُ رِجَالِ إِسْرَائِيلَ لَمَّا رَأُوا الرَّجُلَ هَرَبُوا مِنْهُ وَخَافُوا جِدّاً.
\par 25 فَقَالَ رِجَالُ إِسْرَائِيلَ: «أَرَأَيْتُمْ هَذَا الرَّجُلَ الصَّاعِدَ؟ لِيُعَيِّرَ إِسْرَائِيلَ هُوَ صَاعِدٌ! فَيَكُونُ أَنَّ الرَّجُلَ الَّذِي يَقْتُلُهُ يُغْنِيهِ الْمَلِكُ غِنًى جَزِيلاً, وَيُعْطِيهِ ابْنَتَهُ, وَيَجْعَلُ بَيْتَ أَبِيهِ حُرّاً فِي إِسْرَائِيلَ».
\par 26 فَسَأَلَ دَاوُدُ الرِّجَالَ الْوَاقِفِينَ مَعَهُ: «مَاذَا يُفْعَلُ لِلرَّجُلِ الَّذِي يَقْتُلُ ذَلِكَ الْفِلِسْطِينِيَّ وَيُزِيلُ الْعَارَ عَنْ إِسْرَائِيلَ؟ لأَنَّهُ مَنْ هُوَ هَذَا الْفِلِسْطِينِيُّ الأَغْلَفُ حَتَّى يُعَيِّرَ صُفُوفَ اللَّهِ الْحَيِّ؟»
\par 27 فَكَلَّمَهُ الشَّعْبُ بِمِثْلِ هَذَا الْكَلاَمِ قَائِلِينَ: «كَذَا يُفْعَلُ لِلرَّجُلِ الَّذِي يَقْتُلُهُ».
\par 28 وَسَمِعَ أَخُوهُ الأَكْبَرُ أَلِيآبُ كَلاَمَهُ مَعَ الرِّجَالِ, فَحَمِيَ غَضَبُ أَلِيآبَ عَلَى دَاوُدَ وَقَالَ: «لِمَاذَا نَزَلْتَ, وَعَلَى مَنْ تَرَكْتَ تِلْكَ الْغُنَيْمَاتِ الْقَلِيلَةَ فِي الْبَرِّيَّةِ؟ أَنَا عَلِمْتُ كِبْرِيَاءَكَ وَشَرَّ قَلْبِكَ, لأَنَّكَ إِنَّمَا نَزَلْتَ لِتَرَى الْحَرْبَ».
\par 29 فَقَالَ دَاوُدُ: «مَاذَا عَمِلْتُ الآنَ؟ أَمَا هُوَ كَلاَمٌ؟»
\par 30 وَتَحَوَّلَ مِنْ عِنْدِهِ نَحْوَ آخَرَ وَتَكَلَّمَ بِمِثْلِ هَذَا الْكَلاَمِ, فَرَدَّ لَهُ الشَّعْبُ جَوَاباً كَالْجَوَابِ الأَوَّلِ.
\par 31 وَسُمِعَ الْكَلاَمُ الَّذِي تَكَلَّمَ بِهِ دَاوُدُ وَأَخْبَرُوا بِهِ أَمَامَ شَاوُلَ. فَاسْتَحْضَرَهُ.
\par 32 فَقَالَ دَاوُدُ لِشَاوُلَ: «لاَ يَسْقُطْ قَلْبُ أَحَدٍ بِسَبَبِهِ. عَبْدُكَ يَذْهَبُ وَيُحَارِبُ هَذَا الْفِلِسْطِينِيَّ».
\par 33 فَقَالَ شَاوُلُ لِدَاوُدَ: «لاَ تَسْتَطِيعُ أَنْ تَذْهَبَ إِلَى هَذَا الْفِلِسْطِينِيِّ لِتُحَارِبَهُ لأَنَّكَ غُلاَمٌ وَهُوَ رَجُلُ حَرْبٍ مُنْذُ صِبَاهُ».
\par 34 فَقَالَ دَاوُدُ لِشَاوُلَ: «كَانَ عَبْدُكَ يَرْعَى لأَبِيهِ غَنَماً, فَجَاءَ أَسَدٌ مَعَ دُبٍّ وَأَخَذَ شَاةً مِنَ الْقَطِيعِ.
\par 35 فَخَرَجْتُ وَرَاءَهُ وَقَتَلْتُهُ وَأَنْقَذْتُهَا مِنْ فَمِهِ. وَلَمَّا قَامَ عَلَيَّ أَمْسَكْتُهُ مِنْ ذَقْنِهِ وَضَرَبْتُهُ فَقَتَلْتُهُ.
\par 36 قَتَلَ عَبْدُكَ الأَسَدَ وَالدُّبَّ جَمِيعاً. وَهَذَا الْفِلِسْطِينِيُّ الأَغْلَفُ يَكُونُ كَوَاحِدٍ مِنْهُمَا لأَنَّهُ قَدْ عَيَّرَ صُفُوفَ اللَّهِ الْحَيِّ».
\par 37 وَقَالَ دَاوُدُ: «الرَّبُّ الَّذِي أَنْقَذَنِي مِنْ يَدِ الأَسَدِ وَمِنْ يَدِ الدُّبِّ هُوَ يُنْقِذُنِي مِنْ يَدِ هَذَا الْفِلِسْطِينِيِّ». فَقَالَ شَاوُلُ لِدَاوُدَ: «اذْهَبْ وَلْيَكُنِ الرَّبُّ مَعَكَ».
\par 38 وَأَلْبَسَ شَاوُلُ دَاوُدَ ثِيَابَهُ, وَجَعَلَ خُوذَةً مِنْ نُحَاسٍ عَلَى رَأْسِهِ وَأَلْبَسَهُ دِرْعاً.
\par 39 فَتَقَلَّدَ دَاوُدُ بِسَيْفِهِ فَوْقَ ثِيَابِهِ وَعَزَمَ أَنْ يَمْشِيَ لأَنَّهُ لَمْ يَكُنْ قَدْ جَرَّبَ. فَقَالَ دَاوُدُ لِشَاوُلَ: «لاَ أَقْدِرُ أَنْ أَمْشِيَ بِهَذِهِ لأَنِّي لَمْ أُجَرِّبْهَا». وَنَزَعَهَا دَاوُدُ عَنْهُ.
\par 40 وَأَخَذَ عَصَاهُ بِيَدِهِ, وَانْتَخَبَ لَهُ خَمْسَةَ حِجَارَةٍ مُلْسٍ مِنَ الْوَادِي وَجَعَلَهَا فِي كِنْفِ الرُّعَاةِ الَّذِي لَهُ (أَيْ فِي الْجِرَابِ) وَمِقْلاَعَهُ بِيَدِهِ وَتَقَدَّمَ نَحْوَ الْفِلِسْطِينِيِّ.
\par 41 وَاقْتَرَبَ الْفِلِسْطِينِيُّ إِلَى دَاوُدَ وَحَامِلُ التُّرْسِ أَمَامَهُ.
\par 42 وَلَمَّا رَأَى دَاوُدَ اسْتَحْقَرَهُ لأَنَّهُ كَانَ غُلاَماً وَأَشْقَرَ جَمِيلَ الْمَنْظَرِ.
\par 43 فَقَالَ لِدَاوُدَ: «أَلَعَلِّي أَنَا كَلْبٌ حَتَّى تَأْتِي إِلَيَّ بِعِصِيٍّ». وَلَعَنَ دَاوُدَ بِآلِهَتِهِ.
\par 44 وَقَالَ الْفِلِسْطِينِيُّ لِدَاوُدَ: «تَعَالَ إِلَيَّ فَأُعْطِيَ لَحْمَكَ لِطُيُورِ السَّمَاءِ وَوُحُوشِ الْبَرِّيَّةِ».
\par 45 فَقَالَ دَاوُدُ: «أَنْتَ تَأْتِي إِلَيَّ بِسَيْفٍ وَبِرُمْحٍ وَبِتُرْسٍ. وَأَنَا آتِي إِلَيْكَ بِاسْمِ رَبِّ الْجُنُودِ إِلَهِ صُفُوفِ إِسْرَائِيلَ الَّذِينَ عَيَّرْتَهُمْ.
\par 46 هَذَا الْيَوْمَ يَحْبِسُكَ الرَّبُّ فِي يَدِي فَأَقْتُلُكَ وَأَقْطَعُ رَأْسَكَ. وَأُعْطِي جُثَثَ جَيْشِ الْفِلِسْطِينِيِّينَ هَذَا الْيَوْمَ لِطُيُورِ السَّمَاءِ وَحَيَوَانَاتِ الأَرْضِ, فَتَعْلَمُ كُلُّ الأَرْضِ أَنَّهُ يُوجَدُ إِلَهٌ لإِسْرَائِيلَ.
\par 47 وَتَعْلَمُ هَذِهِ الْجَمَاعَةُ كُلُّهَا أَنَّهُ لَيْسَ بِسَيْفٍ وَلاَ بِرُمْحٍ يُخَلِّصُ الرَّبُّ, لأَنَّ الْحَرْبَ لِلرَّبِّ وَهُوَ يَدْفَعُكُمْ لِيَدِنَا».
\par 48 وَرَكَضَ نَحْوَ الصَّفِّ لِلِقَاءِ الْفِلِسْطِينِيِّ.
\par 49 وَمَدَّ دَاوُدُ يَدَهُ إِلَى الْكِنْفِ وَأَخَذَ مِنْهُ حَجَراً وَرَمَاهُ بِالْمِقْلاَعِ, وَضَرَبَ الْفِلِسْطِينِيَّ فِي جِبْهَتِهِ, فَانْغَزَرَ الْحَجَرُ فِي جِبْهَتِهِ وَسَقَطَ عَلَى وَجْهِهِ إِلَى الأَرْضِ.
\par 50 فَتَمَكَّنَ دَاوُدُ مِنَ الْفِلِسْطِينِيِّ بِالْمِقْلاَعِ وَالْحَجَرِ, وَضَرَبَ الْفِلِسْطِينِيَّ وَقَتَلَهُ. وَلَمْ يَكُنْ سَيْفٌ بِيَدِ دَاوُدَ.
\par 51 فَرَكَضَ دَاوُدُ وَوَقَفَ عَلَى الْفِلِسْطِينِيِّ وَأَخَذَ سَيْفَهُ وَاخْتَرَطَهُ مِنْ غِمْدِهِ وَقَتَلَهُ وَقَطَعَ بِهِ رَأْسَهُ. فَلَمَّا رَأَى الْفِلِسْطِينِيُّونَ أَنَّ جَبَّارَهُمْ قَدْ مَاتَ هَرَبُوا.
\par 52 فَقَامَ رِجَالُ إِسْرَائِيلَ وَيَهُوذَا وَهَتَفُوا وَلَحِقُوا الْفِلِسْطِينِيِّينَ حَتَّى مَجِيئِكَ إِلَى الْوَادِي وَحَتَّى أَبْوَابِ عَقْرُونَ. فَسَقَطَتْ قَتْلَى الْفِلِسْطِينِيِّينَ فِي طَرِيقِ شَعَرَايِمَ إِلَى جَتَّ وَإِلَى عَقْرُونَ.
\par 53 ثُمَّ رَجَعَ بَنُو إِسْرَائِيلَ مِنْ الاِحْتِمَاءِ وَرَاءَ الْفِلِسْطِينِيِّينَ وَنَهَبُوا مَحَلَّتَهُمْ.
\par 54 وَأَخَذَ دَاوُدُ رَأْسَ الْفِلِسْطِينِيِّ وَأَتَى بِهِ إِلَى أُورُشَلِيمَ, وَوَضَعَ أَدَوَاتِهِ فِي خَيْمَتِهِ.
\par 55 وَلَمَّا رَأَى شَاوُلُ دَاوُدَ خَارِجاً لِلِقَاءِ الْفِلِسْطِينِيِّ قَالَ لأَبْنَيْرَ رَئِيسِ الْجَيْشِ: «ابْنُ مَنْ هَذَا الْغُلاَمُ يَا أَبْنَيْرُ؟» فَقَالَ أَبْنَيْرُ: «وَحَيَاتِكَ أَيُّهَا الْمَلِكُ لَسْتُ أَعْلَمُ!»
\par 56 فَقَالَ الْمَلِكُ: «اسْأَلِ ابْنُ مَنْ هَذَا الْغُلاَمُ».
\par 57 وَلَمَّا رَجَعَ دَاوُدُ مِنْ قَتْلِ الْفِلِسْطِينِيِّ أَخَذَهُ أَبْنَيْرُ وَأَحْضَرَهُ أَمَامَ شَاوُلَ وَرَأْسُ الْفِلِسْطِينِيِّ بِيَدِهِ.
\par 58 فَقَالَ لَهُ شَاوُلُ: «ابْنُ مَنْ أَنْتَ يَا غُلاَمُ؟» فَقَالَ دَاوُدُ: «ابْنُ عَبْدِكَ يَسَّى الْبَيْتَلَحْمِيِّ».

\chapter{18}

\par 1 وَكَانَ لَمَّا فَرَغَ مِنَ الْكَلاَمِ مَعَ شَاوُلَ أَنَّ نَفْسَ يُونَاثَانَ تَعَلَّقَتْ بِنَفْسِ دَاوُدَ, وَأَحَبَّهُ يُونَاثَانُ كَنَفْسِهِ.
\par 2 فَأَخَذَهُ شَاوُلُ فِي ذَلِكَ الْيَوْمِ وَلَمْ يَدَعْهُ يَرْجِعُ إِلَى بَيْتِ أَبِيهِ.
\par 3 وَقَطَعَ يُونَاثَانُ وَدَاوُدُ عَهْداً لأَنَّهُ أَحَبَّهُ كَنَفْسِهِ.
\par 4 وَخَلَعَ يُونَاثَانُ الْجُبَّةَ الَّتِي عَلَيْهِ وَأَعْطَاهَا لِدَاوُدَ مَعَ ثِيَابِهِ وَسَيْفِهِ وَقَوْسِهِ وَمِنْطَقَتِهِ.
\par 5 وَكَانَ دَاوُدُ يَخْرُجُ إِلَى حَيْثُمَا أَرْسَلَهُ شَاوُلُ. كَانَ يُفْلِحُ. فَجَعَلَهُ شَاوُلُ عَلَى رِجَالِ الْحَرْبِ. وَحَسُنَ فِي أَعْيُنِ جَمِيعِ الشَّعْبِ وَفِي أَعْيُنِ عَبِيدِ شَاوُلَ أَيْضاً.
\par 6 وَكَانَ عِنْدَ مَجِيئِهِمْ حِينَ رَجَعَ دَاوُدُ مِنْ قَتْلِ الْفِلِسْطِينِيِّ أَنَّ النِّسَاءَ خَرَجَتْ مِنْ جَمِيعِ مُدُنِ إِسْرَائِيلَ بِالْغِنَاءِ وَالرَّقْصِ لِلِقَاءِ شَاوُلَ الْمَلِكِ بِدُفُوفٍ وَبِفَرَحٍ وَبِمُثَلَّثَاتٍ.
\par 7 فَغَنَّتِ النِّسَاءُ اللَّاعِبَاتُ وَقُلْنَ: «ضَرَبَ شَاوُلُ أُلُوفَهُ وَدَاوُدُ رَبَوَاتِهِ.
\par 8 فَغَضِبَ شَاوُلُ جِدّاً وَسَاءَ هَذَا الْكَلاَمُ فِي عَيْنَيْهِ, وَقَالَ: «أَعْطَيْنَ دَاوُدَ رَبَوَاتٍ وَأَمَّا أَنَا فَأَعْطَيْنَنِي الأُلُوفَ! وَبَعْدُ فَقَطْ تَبْقَى لَهُ الْمَمْلَكَةُ!»
\par 9 فَكَانَ شَاوُلُ يُعَايِنُ دَاوُدَ مِنْ ذَلِكَ الْيَوْمِ فَصَاعِداً.
\par 10 وَكَانَ فِي الْغَدِ أَنَّ الرُّوحَ الرَّدِيءَ مِنْ قِبَلِ اللَّهِ اقْتَحَمَ شَاوُلَ وَجُنَّ فِي وَسَطِ الْبَيْتِ. وَكَانَ دَاوُدُ يَضْرِبُ بِيَدِهِ كَمَا فِي يَوْمٍ فَيَوْمٍ, وَكَانَ الرُّمْحُ بِيَدِ شَاوُلَ.
\par 11 فَأَشْرَعَ شَاوُلُ الرُّمْحَ وَقَالَ: «أَضْرِبُ دَاوُدَ حَتَّى إِلَى الْحَائِطِ». فَتَحَوَّلَ دَاوُدُ مِنْ أَمَامِهِ مَرَّتَيْنِ.
\par 12 وَكَانَ شَاوُلُ يَخَافُ دَاوُدَ لأَنَّ الرَّبَّ كَانَ مَعَهُ وَقَدْ فَارَقَ شَاوُلَ.
\par 13 فَأَبْعَدَهُ شَاوُلُ عَنْهُ وَجَعَلَهُ لَهُ رَئِيسَ أَلْفٍ, فَكَانَ يَخْرُجُ وَيَدْخُلُ أَمَامَ الشَّعْبِ.
\par 14 وَكَانَ دَاوُدُ مُفْلِحاً فِي جَمِيعِ طُرُقِهِ وَالرَّبُّ مَعَهُ.
\par 15 فَلَمَّا رَأَى شَاوُلُ أَنَّهُ مُفْلِحٌ جِدّاً فَزِعَ مِنْهُ.
\par 16 وَكَانَ جَمِيعُ إِسْرَائِيلَ وَيَهُوذَا يُحِبُّونَ دَاوُدَ لأَنَّهُ كَانَ يَخْرُجُ وَيَدْخُلُ أَمَامَهُمْ.
\par 17 وَقَالَ شَاوُلُ لِدَاوُدَ: «هُوَذَا ابْنَتِي الْكَبِيرَةُ مَيْرَبُ أُعْطِيكَ إِيَّاهَا امْرَأَةً. إِنَّمَا كُنْ لِي ذَا بَأْسٍ وَحَارِبْ حُرُوبَ الرَّبِّ». فَإِنَّ شَاوُلَ قَالَ: «لاَ تَكُنْ يَدِي عَلَيْهِ, بَلْ لِتَكُنْ عَلَيْهِ يَدُ الْفِلِسْطِينِيِّينَ».
\par 18 فَقَالَ دَاوُدُ لِشَاوُلَ: «مَنْ أَنَا وَمَا هِيَ حَيَاتِي وَعَشِيرَةُ أَبِي فِي إِسْرَائِيلَ حَتَّى أَكُونَ صِهْرَ الْمَلِكِ!»
\par 19 وَكَانَ فِي وَقْتِ إِعْطَاءِ مَيْرَبَ ابْنَةِ شَاوُلَ لِدَاوُدَ أَنَّهَا أُعْطِيَتْ لِعَدْرِيئِيلَ الْمَحُولِيِّ امْرَأَةً.
\par 20 وَمِيكَالُ ابْنَةُ شَاوُلَ أَحَبَّتْ دَاوُدَ, فَأَخْبَرُوا شَاوُلَ, فَحَسُنَ الأَمْرُ فِي عَيْنَيْهِ.
\par 21 وَقَالَ شَاوُلُ: «أُعْطِيهِ إِيَّاهَا فَتَكُونُ لَهُ شَرَكاً وَتَكُونُ يَدُ الْفِلِسْطِينِيِّينَ عَلَيْهِ». وَقَالَ شَاوُلُ لِدَاوُدَ ثَانِيَةً: «تُصَاهِرُنِي الْيَوْمَ».
\par 22 وَأَمَرَ شَاوُلُ عَبِيدَهُ: «تَكَلَّمُوا مَعَ دَاوُدَ سِرّاً قَائِلِينَ: هُوَذَا قَدْ سُرَّ بِكَ الْمَلِكُ, وَجَمِيعُ عَبِيدِهِ قَدْ أَحَبُّوكَ. فَالآنَ صَاهِرِ الْمَلِكَ».
\par 23 فَتَكَلَّمَ عَبِيدُ شَاوُلَ فِي أُذُنَيْ دَاوُدَ بِهَذَا الْكَلاَمِ. فَقَالَ دَاوُدُ: «هَلْ هُوَ مُسْتَخَفٌّ فِي أَعْيُنِكُمْ مُصَاهَرَةُ الْمَلِكِ وَأَنَا رَجُلٌ مَِسْكِينٌ وَحَقِيرٌ؟»
\par 24 فَأَخْبَرَ شَاوُلَ عَبِيدُهُ: «بِمِثْلِ هَذَا الْكَلاَمِ تَكَلَّمَ دَاوُدُ».
\par 25 فَقَالَ شَاوُلُ: «هَكَذَا تَقُولُونَ لِدَاوُدَ: لَيْسَتْ مَسَرَّةُ الْمَلِكِ بِالْمَهْرِ, بَلْ بِمِئَةِ غُلْفَةٍ مِنَ الْفِلِسْطِينِيِّينَ لِلاِنْتِقَامِ مِنْ أَعْدَاءِ الْمَلِكِ». وَكَانَ شَاوُلُ يَتَفَكَّرُ أَنْ يُوقِعَ دَاوُدَ بِيَدِ الْفِلِسْطِينِيِّينَ.
\par 26 فَأَخْبَرَ عَبِيدُهُ دَاوُدَ بِهَذَا الْكَلاَمِ, فَحَسُنَ الْكَلاَمُ فِي عَيْنَيْ دَاوُدَ أَنْ يُصَاهِرَ الْمَلِكَ. وَلَمْ تَكْمُلِ الأَيَّامُ
\par 27 حَتَّى قَامَ دَاوُدُ وَذَهَبَ هُوَ وَرِجَالُهُ وَقَتَلَ مِنَ الْفِلِسْطِينِيِّينَ مِئَتَيْ رَجُلٍ, وَأَتَى دَاوُدُ بِغُلَفِهِمْ فَأَكْمَلُوهَا لِلْمَلِكِ لِمُصَاهَرَةِ الْمَلِكِ. فَأَعْطَاهُ شَاوُلُ مِيكَالَ ابْنَتَهُ امْرَأَةً.
\par 28 فَرَأَى شَاوُلُ وَعَلِمَ أَنَّ الرَّبَّ مَعَ دَاوُدَ. وَمِيكَالُ ابْنَةُ شَاوُلَ كَانَتْ تُحِبُّهُ.
\par 29 وَعَادَ شَاوُلُ يَخَافُ دَاوُدَ بَعْدُ, وَصَارَ شَاوُلُ عَدُوّاً لِدَاوُدَ كُلَّ الأَيَّامِ.
\par 30 وَخَرَجَ أَقْطَابُ الْفِلِسْطِينِيِّينَ. وَمِنْ حِينِ خُرُوجِهِمْ كَانَ دَاوُدُ يُفْلِحُ أَكْثَرَ مِنْ جَمِيعِ عَبِيدِ شَاوُلَ, فَتَوَقَّرَ اسْمُهُ جِدّاً.

\chapter{19}

\par 1 وَكَلَّمَ شَاوُلُ يُونَاثَانَ ابْنَهُ وَجَمِيعَ عَبِيدِهِ أَنْ يَقْتُلُوا دَاوُدَ.
\par 2 وَأَمَّا يُونَاثَانُ بْنُ شَاوُلَ فَسُرَّ بِدَاوُدَ جِدّاً. فَأَخْبَرَ يُونَاثَانُ دَاوُدَ: «شَاوُلُ أَبِي مُلْتَمِسٌ قَتْلَكَ, وَالآنَ فَاحْتَفِظْ عَلَى نَفْسِكَ إِلَى الصَّبَاحِ وَأَقِمْ فِي خُفْيَةٍ وَاخْتَبِئْ.
\par 3 وَأَنَا أَخْرُجُ وَأَقِفُ بِجَانِبِ أَبِي فِي الْحَقْلِ الَّذِي أَنْتَ فِيهِ, وَأُكَلِّمُ أَبِي عَنْكَ, وَأَرَى مَاذَا يَصِيرُ وَأُخْبِرُكَ».
\par 4 وَتَكَلَّمَ يُونَاثَانُ عَنْ دَاوُدَ حَسَناً مَعَ شَاوُلَ أَبِيهِ وَقَالَ لَهُ: «لاَ يُخْطِئِ الْمَلِكُ إِلَى عَبْدِهِ دَاوُدَ, لأَنَّهُ لَمْ يُخْطِئْ إِلَيْكَ, وَلأَنَّ أَعْمَالَهُ حَسَنَةٌ لَكَ جِدّاً.
\par 5 فَإِنَّهُ وَضَعَ نَفْسَهُ بِيَدِهِ وَقَتَلَ الْفِلِسْطِينِيَّ فَصَنَعَ الرَّبُّ خَلاَصاً عَظِيماً لِجَمِيعِ إِسْرَائِيلَ. أَنْتَ رَأَيْتَ وَفَرِحْتَ. فَلِمَاذَا تُخْطِئُ إِلَى دَمٍ بَرِيءٍ بِقَتْلِ دَاوُدَ بِلاَ سَبَبٍ؟»
\par 6 فَسَمِعَ شَاوُلُ لِصَوْتِ يُونَاثَانَ, وَحَلَفَ شَاوُلُ: «حَيٌّ هُوَ الرَّبُّ لاَ يُقْتَلُ».
\par 7 فَدَعَا يُونَاثَانُ دَاوُدَ وَأَخْبَرَهُ بِجَمِيعِ هَذَا الْكَلاَمِ. ثُمَّ جَاءَ يُونَاثَانُ بِدَاوُدَ إِلَى شَاوُلَ فَكَانَ أَمَامَهُ كَأَمْسٍ وَمَا قَبْلَهُ.
\par 8 وَعَادَتِ الْحَرْبُ تَحْدُثُ, فَخَرَجَ دَاوُدُ وَحَارَبَ الْفِلِسْطِينِيِّينَ وَضَرَبَهُمْ ضَرْبَةً عَظِيمَةً فَهَرَبُوا مِنْ أَمَامِهِ.
\par 9 وَكَانَ الرُّوحُ الرَّدِيءُ مِنْ قِبَلِ الرَّبِّ عَلَى شَاوُلَ وَهُوَ جَالِسٌ فِي بَيْتِهِ وَرُمْحُهُ بِيَدِهِ, وَكَانَ دَاوُدُ يَضْرِبُ بِالْيَدِ.
\par 10 فَالْتَمَسَ شَاوُلُ أَنْ يَطْعَنَ دَاوُدَ بِالرُّمْحِ حَتَّى إِلَى الْحَائِطِ, فَفَرَّ مِنْ أَمَامِ شَاوُلَ فَضَرَبَ الرُّمْحَ إِلَى الْحَائِطِ. فَهَرَبَ دَاوُدُ وَنَجَا تِلْكَ اللَّيْلَةَ.
\par 11 فَأَرْسَلَ شَاوُلُ رُسُلاً إِلَى بَيْتِ دَاوُدَ لِيُرَاقِبُوهُ وَيَقْتُلُوهُ فِي الصَّبَاحِ. فَأَخْبَرَتْ دَاوُدَ مِيكَالُ امْرَأَتُهُ: «إِنْ كُنْتَ لاَ تَنْجُو بِنَفْسِكَ هَذِهِ اللَّيْلَةَ فَإِنَّكَ تُقْتَلُ غَداً».
\par 12 فَأَنْزَلَتْ مِيكَالُ دَاوُدَ مِنَ الْكُوَّةِ فَذَهَبَ هَارِباً وَنَجَا.
\par 13 فَأَخَذَتْ مِيكَالُ التَّرَافِيمَ وَوَضَعَتْهُ فِي الْفِرَاشِ, وَوَضَعَتْ لُبْدَةَ الْمِعْزَى تَحْتَ رَأْسِهِ وَغَطَّتْهُ بِثَوْبٍ.
\par 14 وَأَرْسَلَ شَاوُلُ رُسُلاً لأَخْذِ دَاوُدَ, فَقَالَتْ: «هُوَ مَرِيضٌ».
\par 15 ثُمَّ أَرْسَلَ شَاوُلُ الرُّسُلَ لِيَرُوا دَاوُدَ قَائِلاً: «اصْعَدُوا بِهِ إِلَيَّ عَلَى الْفِرَاشِ لأَقْتُلَهُ».
\par 16 فَجَاءَ الرُّسُلُ وَإِذَا فِي الْفِرَاشِ التَّرَافِيمُ وَلِبْدَةُ الْمِعْزَى تَحْتَ رَأْسِهِ.
\par 17 فَقَالَ شَاوُلُ لِمِيكَالَ: «لِمَاذَا خَدَعْتِنِي, فَأَطْلَقْتِ عَدُوِّي حَتَّى نَجَا؟» فَقَالَتْ مِيكَالُ لِشَاوُلَ: «هُوَ قَالَ لِي: أَطْلِقِينِي, لِمَاذَا أَقْتُلُكِ؟».
\par 18 فَهَرَبَ دَاوُدُ وَنَجَا وَجَاءَ إِلَى صَمُوئِيلَ فِي الرَّامَةِ وَأَخْبَرَهُ بِكُلِّ مَا عَمِلَ بِهِ شَاوُلُ. وَذَهَبَ هُوَ وَصَمُوئِيلُ وَأَقَامَا فِي نَايُوتَ.
\par 19 فَأُخْبِرَ شَاوُلُ وَقِيلَ لَهُ: «هُوَذَا دَاوُدُ فِي نَايُوتَ فِي الرَّامَةِ».
\par 20 فَأَرْسَلَ شَاوُلُ رُسُلاً لأَخْذِ دَاوُدَ. وَلَمَّا رَأُوا جَمَاعَةَ الأَنْبِيَاءِ يَتَنَبَّأُونَ, وَصَمُوئِيلَ وَاقِفاً رَئِيساً عَلَيْهِمْ, كَانَ رُوحُ اللَّهِ عَلَى رُسُلِ شَاوُلَ فَتَنَبَّأُوا هُمْ أَيْضاً.
\par 21 وَأَخْبَرُوا شَاوُلَ, فَأَرْسَلَ رُسُلاً آخَرِينَ, فَتَنَبَّأُوا هُمْ أَيْضاً. ثُمَّ عَادَ شَاوُلُ فَأَرْسَلَ رُسُلاً ثَالِثَةً, فَتَنَبَّأُوا هُمْ أَيْضاً.
\par 22 فَذَهَبَ هُوَ أَيْضاً إِلَى الرَّامَةِ وَجَاءَ إِلَى الْبِئْرِ الْعَظِيمَةِ الَّتِي عِنْدَ سِيخُو وَسَأَلَ: «أَيْنَ صَمُوئِيلُ وَدَاوُدُ؟» فَقِيلَ: «هَا هُمَا فِي نَايُوتَ فِي الرَّامَةِ».
\par 23 فَذَهَبَ إِلَى هُنَاكَ إِلَى نَايُوتَ فِي الرَّامَةِ, فَكَانَ عَلَيْهِ أَيْضاً رُوحُ اللَّهِ فَكَانَ يَذْهَبُ وَيَتَنَبَّأُ حَتَّى جَاءَ إِلَى نَايُوتَ فِي الرَّامَةِ.
\par 24 فَخَلَعَ هُوَ أَيْضاً ثِيَابَهُ وَتَنَبَّأَ هُوَ أَيْضاً أَمَامَ صَمُوئِيلَ وَانْطَرَحَ عُرْيَاناً ذَلِكَ النَّهَارَ كُلَّهُ وَكُلَّ اللَّيْلِ. لِذَلِكَ يَقُولُونَ: «أَشَاوُلُ أَيْضاً بَيْنَ الأَنْبِيَاءِ؟».

\chapter{20}

\par 1 فَهَرَبَ دَاوُدُ مِنْ نَايُوتَ فِي الرَّامَةِ, وَجَاءَ وَقَالَ قُدَّامَ يُونَاثَانَ: «مَاذَا عَمِلْتُ وَمَا هُوَ إِثْمِي وَمَا هِيَ خَطِيَّتِي أَمَامَ أَبِيكَ حَتَّى يَطْلُبَ نَفْسِي؟»
\par 2 فَقَالَ لَهُ: «حَاشَا. لاَ تَمُوتُ. هُوَذَا أَبِي لاَ يَعْمَلُ أَمْراً كَبِيراً وَلاَ أَمْراً صَغِيراً إِلَّا وَيُخْبِرُنِي بِهِ. وَلِمَاذَا يُخْفِي عَنِّي أَبِي هَذَا الأَمْرَ؟ لَيْسَ كَذَا».
\par 3 فَحَلَفَ أَيْضاً دَاوُدُ وَقَالَ: «إِنَّ أَبَاكَ قَدْ عَلِمَ أَنِّي قَدْ وَجَدْتُ نِعْمَةً فِي عَيْنَيْكَ, فَقَالَ: لاَ يَعْلَمْ يُونَاثَانُ هَذَا لِئَلَّا يَغْتَمَّ. وَلَكِنْ حَيٌّ هُوَ الرَّبُّ وَحَيَّةٌ هِيَ نَفْسُكَ إِنَّهُ كَخَطْوَةٍ بَيْنِي وَبَيْنَ الْمَوْتِ».
\par 4 فَقَالَ يُونَاثَانُ لِدَاوُدَ: «مَهْمَا تَقُلْ نَفْسُكَ أَفْعَلْهُ لَكَ».
\par 5 فَقَالَ دَاوُدُ لِيُونَاثَانَ: «هُوَذَا الشَّهْرُ غَداً حِينَمَا أَجْلِسُ مَعَ الْمَلِكِ لِلأَكْلِ. وَلَكِنْ أَرْسِلْنِي فَأَخْتَبِئَ فِي الْحَقْلِ إِلَى مَسَاءِ الْيَوْمِ الثَّالِثِ.
\par 6 وَإِذَا افْتَقَدَنِي أَبُوكَ, فَقُلْ: قَدْ طَلَبَ دَاوُدُ مِنِّي طِلْبَةً أَنْ يَرْكُضَ إِلَى بَيْتِ لَحْمٍ مَدِينَتِهِ, لأَنَّ هُنَاكَ ذَبِيحَةً سَنَوِيَّةً لِكُلِّ الْعَشِيرَةِ.
\par 7 فَإِنْ قَالَ: حَسَناً. كَانَ سَلاَمٌ لِعَبْدِكَ. وَلَكِنْ إِنِ اغْتَاظَ غَيْظاً, فَاعْلَمْ أَنَّهُ قَدْ أُعِدَّ الشَّرُّ عِنْدَهُ.
\par 8 فَتَعْمَلُ مَعْرُوفاً مَعَ عَبْدِكَ, لأَنَّكَ بِعَهْدِ الرَّبِّ أَدْخَلْتَ عَبْدَكَ مَعَكَ. وَإِنْ كَانَ فِيَّ إِثْمٌ فَاقْتُلْنِي أَنْتَ, وَلِمَاذَا تَأْتِي بِي إِلَى أَبِيكَ؟»
\par 9 فَقَالَ يُونَاثَانُ: «حَاشَا لَكَ! لأَنَّهُ لَوْ عَلِمْتُ أَنَّ الشَّرَّ قَدْ أُعِدَّ عِنْدَ أَبِي لِيَأْتِيَ عَلَيْكَ, أَفَمَا كُنْتُ أُخْبِرُكَ بِهِ؟»
\par 10 فَقَالَ دَاوُدُ لِيُونَاثَانَ: «مَنْ يُخْبِرُنِي إِنْ جَاوَبَكَ أَبُوكَ شَيْئاً قَاسِياً؟»
\par 11 فَقَالَ يُونَاثَانُ لِدَاوُدَ: «تَعَالَ نَخْرُجُ إِلَى الْحَقْلِ». فَخَرَجَا كِلاَهُمَا إِلَى الْحَقْلِ.
\par 12 وَقَالَ يُونَاثَانُ لِدَاوُدَ: «يَا رَبُّ إِلَهَ إِسْرَائِيلَ, مَتَى اخْتَبَرْتُ أَبِي مِثْلَ الآنَ غَداً أَوْ بَعْدَ غَدٍ, فَإِنْ كَانَ خَيْرٌ لِدَاوُدَ وَلَمْ أُرْسِلْ حِينَئِذٍ فَأُخْبِرَهُ,
\par 13 فَهَكَذَا يَفْعَلُ الرَّبُّ لِيُونَاثَانَ وَهَكَذَا يَزِيدُ. وَإِنِ اسْتَحْسَنَ أَبِي الشَّرَّ نَحْوَكَ, فَإِنِّي أُخْبِرُكَ وَأُطْلِقُكَ فَتَذْهَبُ بِسَلاَمٍ. وَلْيَكُنِ الرَّبُّ مَعَكَ كَمَا كَانَ مَعَ أَبِي.
\par 14 وَلاَ وَأَنَا حَيٌّ بَعْدُ تَصْنَعُ مَعِي إِحْسَانَ الرَّبِّ حَتَّى لاَ أَمُوتَ,
\par 15 بَلْ لاَ تَقْطَعُ مَعْرُوفَكَ عَنْ بَيْتِي إِلَى الأَبَدِ, وَلاَ حِينَ يَقْطَعُ الرَّبُّ أَعْدَاءَ دَاوُدَ جَمِيعاً عَنْ وَجْهِ الأَرْضِ».
\par 16 فَعَاهَدَ يُونَاثَانُ بَيْتَ دَاوُدَ وَقَالَ: «لِيَطْلُبِ الرَّبُّ مِنْ يَدِ أَعْدَاءِ دَاوُدَ».
\par 17 ثُمَّ عَادَ يُونَاثَانُ وَاسْتَحْلَفَ دَاوُدَ بِمَحَبَّتِهِ لَهُ لأَنَّهُ أَحَبَّهُ مَحَبَّةَ نَفْسِهِ.
\par 18 وَقَالَ لَهُ يُونَاثَانُ: «غَداً الشَّهْرُ فَتُفْتَقَدُ لأَنَّ مَوْضِعَكَ يَكُونُ خَالِياً.
\par 19 وَفِي الْيَوْمِ الثَّالِثِ تَنْزِلُ سَرِيعاً وَتَأْتِي إِلَى الْمَوْضِعِ الَّذِي اخْتَبَأْتَ فِيهِ يَوْمَ الْعَمَلِ, وَتَجْلِسُ بِجَانِبِ حَجَرِ الاِفْتِرَاقِ.
\par 20 وَأَنَا أَرْمِي ثَلاَثَةَ سِهَامٍ إِلَى جَانِبِهِ كَأَنِّي أَرْمِي هَدَفاً.
\par 21 وَحِينَئِذٍ أُرْسِلُ الْغُلاَمَ قَائِلاً: اذْهَبِ الْتَقِطِ السِّهَامَ. فَإِنْ قُلْتُ لِلْغُلاَمِ: هُوَذَا السِّهَامُ دُونَكَ فَجَائِياً, خُذْهَا. فَتَعَالَ لأَنَّ لَكَ سَلاَماً. لاَ يُوجَدُ شَيْءٌ. حَيٌّ هُوَ الرَّبُّ.
\par 22 وَلَكِنْ إِنْ قُلْتُ هَكَذَا لِلْغُلاَمِ: هُوَذَا السِّهَامُ دُونَكَ فَصَاعِداً. فَاذْهَبْ لأَنَّ الرَّبَّ قَدْ أَطْلَقَكَ.
\par 23 وَأَمَّا الْكَلاَمُ الَّذِي تَكَلَّمْنَا بِهِ أَنَا وَأَنْتَ فَهُوَذَا الرَّبُّ بَيْنِي وَبَيْنَكَ إِلَى الأَبَدِ».
\par 24 فَاخْتَبَأَ دَاوُدُ فِي الْحَقْلِ. وَكَانَ الشَّهْرُ, فَجَلَسَ الْمَلِكُ عَلَى الطَّعَامِ لِيَأْكُلَ.
\par 25 فَجَلَسَ الْمَلِكُ فِي مَوْضِعِهِ حَسَبَ كُلِّ مَرَّةٍ عَلَى مَجْلِسٍ عِنْدَ الْحَائِطِ. وَقَامَ يُونَاثَانُ وَجَلَسَ أَبْنَيْرُ إِلَى جَانِبِ شَاوُلَ, وَخَلاَ مَوْضِعُ دَاوُدَ.
\par 26 وَلَمْ يَقُلْ شَاوُلُ شَيْئاً فِي ذَلِكَ الْيَوْمِ لأَنَّهُ قَالَ: «لَعَلَّهُ عَارِضٌ. غَيْرُ طَاهِرٍ هُوَ. إِنَّهُ لَيْسَ طَاهِراً».
\par 27 وَكَانَ فِي الْغَدِ الثَّانِي مِنَ الشَّهْرِ أَنَّ مَوْضِعَ دَاوُدَ خَلاَ, فَقَالَ شَاوُلُ لِيُونَاثَانَ ابْنِهِ: «لِمَاذَا لَمْ يَأْتِ ابْنُ يَسَّى إِلَى الطَّعَامِ لاَ أَمْسِ وَلاَ الْيَوْمَ؟»
\par 28 فَأَجَابَ يُونَاثَانُ شَاوُلَ: «إِنَّ دَاوُدَ طَلَبَ مِنِّي أَنْ يَذْهَبَ إِلَى بَيْتِ لَحْمٍ
\par 29 وَقَالَ: أَطْلِقْنِي لأَنَّ عِنْدَنَا ذَبِيحَةَ عَشِيرَةٍ فِي الْمَدِينَةِ, وَقَدْ أَوْصَانِي أَخِي بِذَلِكَ. وَالآنَ إِنْ وَجَدْتُ نِعْمَةً فِي عَيْنَيْكَ فَدَعْنِي أُفْلِتُ وَأَرَى إِخْوَتِي. لِذَلِكَ لَمْ يَأْتِ إِلَى مَائِدَةِ الْمَلِكِ».
\par 30 فَحَمِيَ غَضَبُ شَاوُلَ عَلَى يُونَاثَانَ وَقَالَ لَهُ: «يَا ابْنَ الْمُتَعَوِّجَةِ الْمُتَمَرِّدَةِ, أَمَا عَلِمْتُ أَنَّكَ قَدِ اخْتَرْتَ ابْنَ يَسَّى لِخِزْيِكَ وَخِزْيِ عَوْرَةِ أُمِّكَ؟
\par 31 لأَنَّهُ مَا دَامَ ابْنُ يَسَّى حَيّاً عَلَى الأَرْضِ لاَ تُثْبَتُ أَنْتَ وَلاَ مَمْلَكَتُكَ. وَالآنَ أَرْسِلْ وَأْتِ بِهِ إِلَيَّ لأَنَّهُ ابْنُ الْمَوْتِ هُوَ».
\par 32 فَأَجَابَ يُونَاثَانُ شَاوُلَ أَبَاهُ: «لِمَاذَا يُقْتَلُ؟ مَاذَا عَمِلَ؟»
\par 33 فَوَجَّهَ شَاوُلُ الرُّمْحَ نَحْوَهُ لِيَطْعَنَهُ. فَعَلِمَ يُونَاثَانُ أَنَّ أَبَاهُ قَدْ عَزَمَ عَلَى قَتْلِ دَاوُدَ.
\par 34 فَقَامَ يُونَاثَانُ عَنِ الْمَائِدَةِ بِحُمُوِّ غَضَبٍ وَلَمْ يَأْكُلْ خُبْزاً فِي الْيَوْمِ الثَّانِي مِنَ الشَّهْرِ, لأَنَّهُ اغْتَمَّ عَلَى دَاوُدَ, لأَنَّ أَبَاهُ قَدْ أَخْزَاهُ.
\par 35 وَكَانَ فِي الصَّبَاحِ أَنَّ يُونَاثَانَ خَرَجَ إِلَى الْحَقْلِ إِلَى مِيعَادِ دَاوُدَ وَغُلاَمٌ صَغِيرٌ مَعَهُ.
\par 36 وَقَالَ لِغُلاَمِهِ: «ارْكُضِ الْتَقِطِ السِّهَامَ الَّتِي أَنَا رَامِيهَا». وَبَيْنَمَا الْغُلاَمُ رَاكِضٌ رَمَى السَّهْمَ حَتَّى جَاوَزَهُ.
\par 37 وَلَمَّا جَاءَ الْغُلاَمُ إِلَى مَوْضِعِ السَّهْمِ الَّذِي رَمَاهُ يُونَاثَانُ, نَادَى يُونَاثَانُ وَرَاءَ الْغُلاَمِ: «أَلَيْسَ السَّهْمُ دُونَكَ فَصَاعِداً؟»
\par 38 وَنَادَى يُونَاثَانُ وَرَاءَ الْغُلاَمِ: «اعْجَلْ. أَسْرِعْ. لاَ تَقِفْ». فَالْتَقَطَ غُلاَمُ يُونَاثَانَ السَّهْمَ وَجَاءَ إِلَى سَيِّدِهِ.
\par 39 وَالْغُلاَمُ لَمْ يَكُنْ يَعْلَمُ شَيْئاً, وَأَمَّا يُونَاثَانُ وَدَاوُدُ فَكَانَا يَعْلَمَانِ الأَمْرَ.
\par 40 فَأَعْطَى يُونَاثَانُ سِلاَحَهُ لِلْغُلاَمِ الَّذِي لَهُ وَقَالَ لَهُ: «اذْهَبِ. ادْخُلْ بِهِ إِلَى الْمَدِينَةِ».
\par 41 اَلْغُلاَمُ ذَهَبَ وَدَاوُدُ قَامَ مِنْ جَانِبِ الْجَنُوبِ وَسَقَطَ عَلَى وَجْهِهِ إِلَى الأَرْضِ وَسَجَدَ ثَلاَثَ مَرَّاتٍ. وَقَبَّلَ كُلٌّ مِنْهُمَا صَاحِبَهُ, وَبَكَى كُلٌّ مِنْهُمَا مَعَ صَاحِبِهِ حَتَّى زَادَ دَاوُدُ.
\par 42 فَقَالَ يُونَاثَانُ لِدَاوُدَ: «اذْهَبْ بِسَلاَمٍ لأَنَّنَا كِلَيْنَا قَدْ حَلَفْنَا بِاسْمِ الرَّبِّ قَائِلَيْنِ: الرَّبُّ يَكُونُ بَيْنِي وَبَيْنَكَ وَبَيْنَ نَسْلِي وَنَسْلِكَ إِلَى الأَبَدِ». فَقَامَ وَذَهَبَ, وَأَمَّا يُونَاثَانُ فَجَاءَ إِلَى الْمَدِينَةِ.

\chapter{21}

\par 1 فَجَاءَ دَاوُدُ إِلَى نُوبٍ إِلَى أَخِيمَالِكَ الْكَاهِنِ. فَاضْطَرَبَ أَخِيمَالِكُ عِنْدَ لِقَاءِ دَاوُدَ وَقَالَ لَهُ: «لِمَاذَا أَنْتَ وَحْدَكَ وَلَيْسَ مَعَكَ أَحَدٌ؟»
\par 2 فَقَالَ دَاوُدُ لأَخِيمَالِكَ الْكَاهِنِ: «إِنَّ الْمَلِكَ أَمَرَنِي بِشَيْءٍ وَقَالَ لِي: لاَ يَعْلَمْ أَحَدٌ شَيْئاً مِنَ الأَمْرِ الَّذِي أَرْسَلْتُكَ فِيهِ وَأَمَرْتُكَ بِهِ. وَأَمَّا الْغِلْمَانُ فَقَدْ عَيَّنْتُ لَهُمُ الْمَوْضِعَ الْفُلاَنِيَّ وَالْفُلاَنِيَّ.
\par 3 وَالآنَ فَمَاذَا يُوجَدُ تَحْتَ يَدِكَ؟ أَعْطِ خَمْسَ خُبْزَاتٍ فِي يَدِي أَوِ الْمَوْجُودَ».
\par 4 فَأَجَابَ الْكَاهِنُ دَاوُدَ: «لاَ يُوجَدُ خُبْزٌ مُحَلَّلٌ تَحْتَ يَدِي, وَلَكِنْ يُوجَدُ خُبْزٌ مُقَدَّسٌ إِذَا كَانَ الْغِلْمَانُ قَدْ حَفِظُوا أَنْفُسَهُمْ لاَ سِيَّمَا مِنَ النِّسَاءِ».
\par 5 فَأَجَابَ دَاوُدُ الْكَاهِنَ: «إِنَّ النِّسَاءَ قَدْ مُنِعَتْ عَنَّا مُنْذُ أَمْسِ وَمَا قَبْلَهُ عِنْدَ خُرُوجِي وَأَمْتِعَةُ الْغِلْمَانِ مُقَدَّسَةٌ. وَهُوَ عَلَى نَوْعٍ مُحَلَّلٌ, وَالْيَوْمَ أَيْضاً يَتَقَدَّسُ بِالآنِيَةِ».
\par 6 فَأَعْطَاهُ الْكَاهِنُ الْمُقَدَّسَ, لأَنَّهُ لَمْ يَكُنْ هُنَاكَ خُبْزٌ إِلَّا خُبْزَ الْوُجُوهِ الْمَرْفُوعَ مِنْ أَمَامِ الرَّبِّ لِيُوضَعَ خُبْزٌ سُخْنٌ فِي يَوْمِ أَخْذِهِ.
\par 7 وَكَانَ هُنَاكَ رَجُلٌ مِنْ عَبِيدِ شَاوُلَ فِي ذَلِكَ الْيَوْمِ مَحْصُوراً أَمَامَ الرَّبِّ اسْمُهُ دُوَاغُ الأَدُومِيُّ رَئِيسُ رُعَاةِ شَاوُلَ.
\par 8 وَقَالَ دَاوُدُ لأَخِيمَالِكَ: «أَفَمَا يُوجَدُ هُنَا تَحْتَ يَدِكَ رُمْحٌ أَوْ سَيْفٌ, لأَنِّي لَمْ آخُذْ بِيَدِي سَيْفِي وَلاَ سِلاَحِي لأَنَّ أَمْرَ الْمَلِكِ كَانَ مُعَجِّلاً؟»
\par 9 فَقَالَ الْكَاهِنُ: «إِنَّ سَيْفَ جُلْيَاتَ الْفِلِسْطِينِيِّ الَّذِي قَتَلْتَهُ فِي وَادِي الْبُطْمِ هَا هُوَ مَلْفُوفٌ فِي ثَوْبٍ خَلْفَ الأَفُودِ, فَإِنْ شِئْتَ أَنْ تَأْخُذَهُ فَخُذْهُ, لأَنَّهُ لَيْسَ آخَرُ سِوَاهُ هُنَا». فَقَالَ دَاوُدُ: «لاَ يُوجَدُ مِثْلُهُ. أَعْطِنِي إِيَّاهُ».
\par 10 وَقَامَ دَاوُدُ وَهَرَبَ فِي ذَلِكَ الْيَوْمِ مِنْ أَمَامِ شَاوُلَ وَجَاءَ إِلَى أَخِيشَ مَلِكِ جَتَّ.
\par 11 فَقَالَ عَبِيدُ أَخِيشَ لَهُ: «أَلَيْسَ هَذَا دَاوُدَ مَلِكَ الأَرْضِ؟ أَلَيْسَ لِهَذَا كُنَّ يُغَنِّينَ فِي الرَّقْصِ قَائِلاَتٍ: ضَرَبَ شَاوُلُ أُلُوفَهُ وَدَاوُدُ رَبَوَاتِهِ؟».
\par 12 فَوَضَعَ دَاوُدُ هَذَا الْكَلاَمَ فِي قَلْبِهِ وَخَافَ جِدّاً مِنْ أَخِيشَ مَلِكِ جَتَّ.
\par 13 فَغَيَّرَ عَقْلَهُ فِي أَعْيُنِهِمْ, وَتَظَاهَرَ بِالْجُنُونِ بَيْنَ أَيْدِيهِمْ, وَأَخَذَ يُخَرْبِشُ عَلَى مَصَارِيعِ الْبَابِ وَيُسِيلُ رِيقَهُ عَلَى لِحْيَتِهِ.
\par 14 فَقَالَ أَخِيشُ لِعَبِيدِهِ: «هُوَذَا تَرَوْنَ الرَّجُلَ مَجْنُوناً, فَلِمَاذَا تَأْتُونَ بِهِ إِلَيَّ؟
\par 15 أَلَعَلِّي مُحْتَاجٌ إِلَى مَجَانِينَ حَتَّى أَتَيْتُمْ بِهَذَا لِيَتَجَنَّنَ عَلَيَّ؟ أَهَذَا يَدْخُلُ بَيْتِي؟».

\chapter{22}

\par 1 فَذَهَبَ دَاوُدُ مِنْ هُنَاكَ وَنَجَا إِلَى مَغَارَةِ عَدُلَّامَ. فَلَمَّا سَمِعَ إِخْوَتُهُ وَجَمِيعُ بَيْتِ أَبِيهِ نَزَلُوا إِلَيْهِ إِلَى هُنَاكَ.
\par 2 وَاجْتَمَعَ إِلَيْهِ كُلُّ رَجُلٍ مُتَضَايِقٍ, وَكُلُّ مَنْ كَانَ عَلَيْهِ دَيْنٌ, وَكُلُّ رَجُلٍ مُرِّ النَّفْسِ, فَكَانَ عَلَيْهِمْ رَئِيساً. وَكَانَ مَعَهُ نَحْوُ أَرْبَعِ مِئَةِ رَجُلٍ.
\par 3 وَذَهَبَ دَاوُدُ مِنْ هُنَاكَ إِلَى مِصْفَاةِ مُوآبَ وَقَالَ لِمَلِكِ مُوآبَ: «لِيَخْرُجْ أَبِي وَأُمِّي إِلَيْكُمْ حَتَّى أَعْلَمَ مَاذَا يَصْنَعُ لِيَ اللَّهُ».
\par 4 فَوَدَعَهُمَا عِنْدَ مَلِكِ مُوآبَ فَأَقَامَا عِنْدَهُ كُلَّ أَيَّامِ إِقَامَةِ دَاوُدَ فِي الْحِصْنِ.
\par 5 فَقَالَ جَادُ النَّبِيُّ لِدَاوُدَ: «لاَ تُقِمْ فِي الْحِصْنِ. اذْهَبْ وَادْخُلْ أَرْضَ يَهُوذَا». فَذَهَبَ دَاوُدُ وَجَاءَ إِلَى وَعْرِ حَارِثٍ.
\par 6 وَسَمِعَ شَاوُلُ أَنَّهُ قَدِ اشْتَهَرَ دَاوُدُ وَالرِّجَالُ الَّذِينَ مَعَهُ. وَكَانَ شَاوُلُ مُقِيماً فِي جِبْعَةَ تَحْتَ الأَثْلَةِ فِي الرَّامَةِ وَرُمْحُهُ بِيَدِهِ, وَجَمِيعُ عَبِيدِهِ وُقُوفاً لَدَيْهِ.
\par 7 فَقَالَ شَاوُلُ لِعَبِيدِهِ الْوَاقِفِينَ لَدَيْهِ: «اسْمَعُوا يَا بِنْيَامِينِيُّونَ. هَلْ يُعْطِيكُمْ جَمِيعَكُمُ ابْنُ يَسَّى حُقُولاً وَكُرُوماً, وَهَلْ يَجْعَلُكُمْ جَمِيعَكُمْ رُؤَسَاءَ أُلُوفٍ وَرُؤَسَاءَ مِئَاتٍ,
\par 8 حَتَّى فَتَنْتُمْ كُلُّكُمْ عَلَيَّ, وَلَيْسَ مَنْ يُخْبِرُنِي بِعَهْدِ ابْنِي مَعَ ابْنِ يَسَّى, وَلَيْسَ مِنْكُمْ مَنْ يَحْزَنُ عَلَيَّ أَوْ يُخْبِرُنِي بِأَنَّ ابْنِي قَدْ أَقَامَ عَبْدِي عَلَيَّ كَمِيناً كَهَذَا الْيَوْمِ؟»
\par 9 فَأَجَابَ دُوَاغُ الأَدُومِيُّ الَّذِي كَانَ مُوَكَّلاً عَلَى عَبِيدِ شَاوُلَ: «قَدْ رَأَيْتُ ابْنَ يَسَّى آتِياً إِلَى نُوبَ إِلَى أَخِيمَالِكَ بْنِ أَخِيطُوبَ.
\par 10 فَسَأَلَ لَهُ مِنَ الرَّبِّ وَأَعْطَاهُ زَاداً. وَسَيْفَ جُلْيَاتَ الْفِلِسْطِينِيِّ أَعْطَاهُ إِيَّاهُ».
\par 11 فَأَرْسَلَ الْمَلِكُ وَاسْتَدْعَى أَخِيمَالِكَ بْنَ أَخِيطُوبَ الْكَاهِنَ وَجَمِيعَ بَيْتِ أَبِيهِ, الْكَهَنَةَ الَّذِينَ فِي نُوبٍ. فَجَاءُوا كُلُّهُمْ إِلَى الْمَلِكِ.
\par 12 فَقَالَ شَاوُلُ: «اسْمَعْ يَا ابْنَ أَخِيطُوبَ». فَقَالَ: «هَئَنَذَا يَا سَيِّدِي».
\par 13 فَقَالَ لَهُ شَاوُلُ: «لِمَاذَا فَتَنْتُمْ عَلَيَّ أَنْتَ وَابْنُ يَسَّى بِإِعْطَائِكَ إِيَّاهُ خُبْزاً وَسَيْفاً, وَسَأَلْتَ لَهُ مِنَ اللَّهِ لِيَقُومَ عَلَيَّ كَامِناً كَهَذَا الْيَوْمِ؟»
\par 14 فَأَجَابَ أَخِيمَالِكُ الْمَلِكَ: «وَمَنْ مِنْ جَمِيعِ عَبِيدِكَ مِثْلُ دَاوُدَ, أَمِينٌ وَصِهْرُ الْمَلِكِ وَصَاحِبُ سِرِّكَ وَمُكَرَّمٌ فِي بَيْتِكَ؟
\par 15 فَهَلِ الْيَوْمَ ابْتَدَأْتُ أَسْأَلُ لَهُ مِنَ اللَّهِ؟ حَاشَا لِي! لاَ يَنْسِبِ الْمَلِكُ شَيْئاً لِعَبْدِهِ وَلاَ لِجَمِيعِ بَيْتِ أَبِي, لأَنَّ عَبْدَكَ لَمْ يَعْلَمْ شَيْئاً مِنْ كُلِّ هَذَا صَغِيراً أَوْ كَبِيراً».
\par 16 فَقَالَ الْمَلِكُ: «مَوْتاً تَمُوتُ يَا أَخِيمَالِكُ أَنْتَ وَكُلُّ بَيْتِ أَبِيكَ».
\par 17 وَقَالَ الْمَلِكُ لِلسُّعَاةِ الْوَاقِفِينَ لَدَيْهِ: «دُورُوا وَاقْتُلُوا كَهَنَةَ الرَّبِّ, لأَنَّ يَدَهُمْ أَيْضاً مَعَ دَاوُدَ, وَلأَنَّهُمْ عَلِمُوا أَنَّهُ هَارِبٌ وَلَمْ يُخْبِرُونِي». فَلَمْ يَرْضَ عَبِيدُ الْمَلِكِ أَنْ يَمُدُّوا أَيْدِيَهُمْ لِيَقَعُوا بِكَهَنَةِ الرَّبِّ.
\par 18 فَقَالَ الْمَلِكُ لِدُوَاغَ: «دُرْ أَنْتَ وَقَعْ بِالْكَهَنَةِ». فَدَارَ دُوَاغُ الأَدُومِيُّ وَوَقَعَ هُوَ بِالْكَهَنَةِ, وَقَتَلَ فِي ذَلِكَ الْيَوْمِ خَمْسَةً وَثَمَانِينَ رَجُلاً لاَبِسِي أَفُودِ كَتَّانٍ,
\par 19 وَضَرَبَ نُوبَ مَدِينَةَ الْكَهَنَةِ بِحَدِّ السَّيْفِ: الرِّجَالَ وَالنِّسَاءَ وَالأَطْفَالَ وَالرِّضْعَانَ وَالثِّيرَانَ وَالْحَمِيرَ وَالْغَنَمَ.
\par 20 فَنَجَا وَلَدٌ وَاحِدٌ لأَخِيمَالِكَ بْنِ أَخِيطُوبَ اسْمُهُ أَبِيَاثَارُ وَهَرَبَ إِلَى دَاوُدَ.
\par 21 وَأَخْبَرَ أَبِيَاثَارُ دَاوُدَ بِأَنَّ شَاوُلَ قَدْ قَتَلَ كَهَنَةَ الرَّبِّ.
\par 22 فَقَالَ دَاوُدُ لأَبِيَاثَارَ: «عَلِمْتُ فِي ذَلِكَ الْيَوْمِ الَّذِي فِيهِ كَانَ دُوَاغُ الأَدُومِيُّ هُنَاكَ أَنَّهُ يُخْبِرُ شَاوُلَ. أَنَا سَبَّبْتُ لِجَمِيعِ أَنْفُسِ بَيْتِ أَبِيكَ.
\par 23 أَقِمْ مَعِي. لاَ تَخَفْ, لأَنَّ الَّذِي يَطْلُبُ نَفْسِي يَطْلُبُ نَفْسَكَ, وَلَكِنَّكَ عِنْدِي مَحْفُوظٌ».

\chapter{23}

\par 1 فَأَخْبَرُوا دَاوُدَ: «هُوَذَا الْفِلِسْطِينِيُّونَ يُحَارِبُونَ قَعِيلَةَ وَيَنْهَبُونَ الْبَيَادِرَ».
\par 2 فَسَأَلَ دَاوُدُ مِنَ الرَّبِّ: «أَأَذْهَبُ وَأَضْرِبُ هَؤُلاَءِ الْفِلِسْطِينِيِّينَ؟» فَقَالَ الرَّبُّ لِدَاوُدَ: «اذْهَبْ وَاضْرِبِ الْفِلِسْطِينِيِّينَ وَخَلِّصْ قَعِيلَةَ».
\par 3 فَقَالَ رِجَالُ دَاوُدَ لَهُ: «هَا نَحْنُ هَهُنَا فِي يَهُوذَا خَائِفُونَ, فَكَمْ بِالْحَرِيِّ إِذَا ذَهَبْنَا إِلَى قَعِيلَةَ ضِدَّ صُفُوفِ الْفِلِسْطِينِيِّينَ؟»
\par 4 فَعَادَ أَيْضاً دَاوُدُ وَسَأَلَ مِنَ الرَّبِّ, فَأَجَابَهُ الرَّبُّ: «قُمِ انْزِلْ إِلَى قَعِيلَةَ, فَإِنِّي أَدْفَعُ الْفِلِسْطِينِيِّينَ لِيَدِكَ».
\par 5 فَذَهَبَ دَاوُدُ وَرِجَالُهُ إِلَى قَعِيلَةَ وَحَارَبَ الْفِلِسْطِينِيِّينَ وَسَاقَ مَوَاشِيَهُمْ وَضَرَبَهُمْ ضَرْبَةً عَظِيمَةً, وَخَلَّصَ دَاوُدُ سُكَّانَ قَعِيلَةَ.
\par 6 وَكَانَ لَمَّا هَرَبَ أَبِيَاثَارُ بْنُ أَخِيمَالِكَ إِلَى دَاوُدَ إِلَى قَعِيلَةَ نَزَلَ وَبِيَدِهِ أَفُودٌ,
\par 7 فَأُخْبِرَ شَاوُلُ بِأَنَّ دَاوُدَ قَدْ جَاءَ إِلَى قَعِيلَةَ. فَقَالَ شَاوُلُ: «قَدْ نَبَذَهُ اللَّهُ إِلَى يَدِي, لأَنَّهُ قَدْ أُغْلِقَ عَلَيْهِ بِالدُّخُولِ إِلَى مَدِينَةٍ لَهَا أَبْوَابٌ وَعَوَارِضُ».
\par 8 وَدَعَا شَاوُلُ جَمِيعَ الشَّعْبِ لِلْحَرْبِ لِلنُّزُولِ إِلَى قَعِيلَةَ لِمُحَاصَرَةِ دَاوُدَ وَرِجَالِهِ.
\par 9 فَلَمَّا عَرَفَ دَاوُدُ أَنَّ شَاوُلَ مُنْشِئٌ عَلَيْهِ الشَّرَّ, قَالَ لأَبِيَاثَارَ الْكَاهِنِ قَدِّمِ الأَفُودَ.
\par 10 ثُمَّ قَالَ دَاوُدُ: «يَا رَبُّ إِلَهَ إِسْرَائِيلَ, إِنَّ عَبْدَكَ قَدْ سَمِعَ بِأَنَّ شَاوُلَ يُحَاوِلُ أَنْ يَأْتِيَ إِلَى قَعِيلَةَ لِيُخْرِبَ الْمَدِينَةَ بِسَبَبِي.
\par 11 فَهَلْ يُسَلِّمُنِي أَهْلُ قَعِيلَةَ لِيَدِهِ؟ هَلْ يَنْزِلُ شَاوُلُ كَمَا سَمِعَ عَبْدُكَ؟ يَا رَبُّ إِلَهَ إِسْرَائِيلَ, أَخْبِرْ عَبْدَكَ». فَقَالَ الرَّبُّ: «يَنْزِلُ».
\par 12 فَقَالَ دَاوُدُ: «هَلْ يُسَلِّمُنِي أَهْلُ قَعِيلَةَ مَعَ رِجَالِي لِيَدِ شَاوُلَ؟» فَقَالَ الرَّبُّ: «يُسَلِّمُونَ».
\par 13 فَقَامَ دَاوُدُ وَرِجَالُهُ, نَحْوُ سِتِّ مِئَةِ رَجُلٍ, وَخَرَجُوا مِنْ قَعِيلَةَ وَذَهَبُوا حَيْثُمَا ذَهَبُوا. فَأُخْبِرَ شَاوُلُ بِأَنَّ دَاوُدَ قَدْ أَفْلَتَ مِنْ قَعِيلَةَ, فَعَدَلَ عَنِ الْخُرُوجِ.
\par 14 وَأَقَامَ دَاوُدُ فِي الْبَرِّيَّةِ فِي الْحُصُونِ وَمَكَثَ فِي الْجَبَلِ فِي بَرِّيَّةِ زِيفٍ. وَكَانَ شَاوُلُ يَطْلُبُهُ كُلَّ الأَيَّامِ, وَلَكِنْ لَمْ يَدْفَعْهُ اللَّهُ لِيَدِهِ.
\par 15 فَرَأَى دَاوُدُ أَنَّ شَاوُلَ قَدْ خَرَجَ يَطْلُبُ نَفْسَهُ. وَكَانَ دَاوُدُ فِي بَرِّيَّةِ زِيفٍ فِي الْغَابِ.
\par 16 فَقَامَ يُونَاثَانُ بْنُ شَاوُلَ وَذَهَبَ إِلَى دَاوُدَ إِلَى الْغَابِ وَشَدَّدَ يَدَهُ بِاللَّهِ.
\par 17 وَقَالَ لَهُ: «لاَ تَخَفْ لأَنَّ يَدَ شَاوُلَ أَبِي لاَ تَجِدُكَ, وَأَنْتَ تَمْلِكُ عَلَى إِسْرَائِيلَ, وَأَنَا أَكُونُ لَكَ ثَانِياً. وَشَاوُلُ أَبِي أَيْضاً يَعْلَمُ ذَلِكَ».
\par 18 فَقَطَعَا كِلاَهُمَا عَهْداً أَمَامَ الرَّبِّ. وَأَقَامَ دَاوُدُ فِي الْغَابِ, وَأَمَّا يُونَاثَانُ فَمَضَى إِلَى بَيْتِهِ.
\par 19 فَصَعِدَ الزِّيفِيُّونَ إِلَى شَاوُلَ إِلَى جِبْعَةَ قَائِلِينَ: «أَلَيْسَ دَاوُدُ مُخْتَبِئاً عِنْدَنَا فِي حُصُونٍ فِي الْغَابِ فِي تَلِّ حَخِيلَةَ الَّتِي إِلَى يَمِينِ الْقَفْرِ.
\par 20 فَالآنَ حَسَبَ كُلِّ شَهْوَةِ نَفْسِكَ أَيُّهَا الْمَلِكُ فِي النُّزُولِ انْزِلْ, وَعَلَيْنَا أَنْ نُسَلِّمَهُ لِيَدِ الْمَلِكِ».
\par 21 فَقَالَ شَاوُلُ: «مُبَارَكُونَ أَنْتُمْ مِنَ الرَّبِّ لأَنَّكُمْ قَدْ أَشْفَقْتُمْ عَلَيَّ.
\par 22 فَاذْهَبُوا أَكِّدُوا أَيْضاً وَاعْلَمُوا وَانْظُرُوا مَكَانَهُ حَيْثُ تَكُونُ رِجْلُهُ وَمَنْ رَآهُ هُنَاكَ. لأَنَّهُ قِيلَ لِي إِنَّهُ مَكْراً يَمْكُرُ.
\par 23 فَانْظُرُوا وَاعْلَمُوا جَمِيعَ الْمُخْتَبَئَاتِ الَّتِي يَخْتَبِئُ فِيهَا ثُمَّ ارْجِعُوا إِلَيَّ عَلَى تَأْكِيدٍ, فَأَسِيرَ مَعَكُمْ. وَيَكُونُ إِذَا وُجِدَ فِي الأَرْضِ أَنِّي أُفَتِّشُ عَلَيْهِ بِجَمِيعِ أُلُوفِ يَهُوذَا».
\par 24 فَقَامُوا وَذَهَبُوا إِلَى زِيفٍ قُدَّامَ شَاوُلَ. وَكَانَ دَاوُدُ وَرِجَالُهُ فِي بَرِّيَّةِ مَعُونٍ فِي السَّهْلِ عَنْ يَمِينِ الْقَفْرِ.
\par 25 وَذَهَبَ شَاوُلُ وَرِجَالُهُ لِلتَّفْتِيشِ, فَأَخْبَرُوا دَاوُدَ فَنَزَلَ إِلَى الصَّخْرِ وَأَقَامَ فِي بَرِّيَّةِ مَعُونٍ. فَلَمَّا سَمِعَ شَاوُلُ تَبِعَ دَاوُدَ إِلَى بَرِّيَّةِ مَعُونٍ.
\par 26 فَذَهَبَ شَاوُلُ عَنْ جَانِبِ الْجَبَلِ مِنْ هُنَا, وَدَاوُدُ وَرِجَالُهُ عَنْ جَانِبِ الْجَبَلِ مِنْ هُنَاكَ. وَكَانَ دَاوُدُ يَفِرُّ فِي الذَّهَابِ مِنْ أَمَامِ شَاوُلَ, وَكَانَ شَاوُلُ وَرِجَالُهُ يُحَاوِطُونَ دَاوُدَ وَرِجَالَهُ لِيَأْخُذُوهُمْ.
\par 27 فَجَاءَ رَسُولٌ إِلَى شَاوُلَ يَقُولُ: «أَسْرِعْ وَاذْهَبْ لأَنَّ الْفِلِسْطِينِيِّينَ قَدِ اقْتَحَمُوا الأَرْضَ».
\par 28 فَرَجَعَ شَاوُلُ عَنِ اتِّبَاعِ دَاوُدَ, وَذَهَبَ لِلِقَاءِ الْفِلِسْطِينِيِّينَ. لِذَلِكَ دُعِيَ ذَلِكَ الْمَوْضِعُ «صَخْرَةَ الزَّلَقَاتِ».
\par 29 وَصَعِدَ دَاوُدُ مِنْ هُنَاكَ وَأَقَامَ فِي حُصُونِ عَيْنِ جَدْيٍ.

\chapter{24}

\par 1 وَلَمَّا رَجَعَ شَاوُلُ مِنْ وَرَاءِ الْفِلِسْطِينِيِّينَ أَخْبَرُوهُ: «هُوَذَا دَاوُدُ فِي بَرِّيَّةِ عَيْنِ جَدْيٍ».
\par 2 فَأَخَذَ شَاوُلُ ثَلاَثَةَ آلاَفِ رَجُلٍ مُنْتَخَبِينَ مِنْ جَمِيعِ إِسْرَائِيلَ وَذَهَبَ يَطْلُبُ دَاوُدَ وَرِجَالَهُ عَلَى صُخُورِ الْوُعُولِ.
\par 3 وَجَاءَ إِلَى حَظَائِرِ الْغَنَمِ الَّتِي فِي الطَّرِيقِ. وَكَانَ هُنَاكَ كَهْفٌ فَدَخَلَ شَاوُلُ لِحَاجَةٍ لَهُ (وَدَاوُدُ وَرِجَالُهُ كَانُوا جُلُوساً فِي مُؤَخَّرَةِ الْكَهْفِ).
\par 4 فَقَالَ رِجَالُ دَاوُدَ لَهُ: «هُوَذَا الْيَوْمُ الَّذِي قَالَ لَكَ عَنْهُ الرَّبُّ:هَئَنَذَا أَدْفَعُ عَدُوَّكَ لِيَدِكَ فَتَفْعَلُ بِهِ مَا يَحْسُنُ فِي عَيْنَيْكَ». فَقَامَ دَاوُدُ وَقَطَعَ طَرَفَ جُبَّةِ شَاوُلَ سِرّاً.
\par 5 وَكَانَ بَعْدَ ذَلِكَ أَنَّ قَلْبَ دَاوُدَ ضَرَبَهُ عَلَى قَطْعِهِ طَرَفَ جُبَّةِ شَاوُلَ,
\par 6 فَقَالَ لِرِجَالِهِ: «حَاشَا لِي مِنْ قِبَلِ الرَّبِّ أَنْ أَعْمَلَ هَذَا الأَمْرَ بِسَيِّدِي بِمَسِيحِ الرَّبِّ, فَأَمُدَّ يَدِي إِلَيْهِ لأَنَّهُ مَسِيحُ الرَّبِّ هُوَ».
\par 7 فَوَبَّخَ دَاوُدُ رِجَالَهُ بِالْكَلاَمِ وَلَمْ يَدَعْهُمْ يَقُومُونَ عَلَى شَاوُلَ. وَأَمَّا شَاوُلُ فَقَامَ مِنَ الْكَهْفِ وَذَهَبَ فِي طَرِيقِهِ.
\par 8 ثُمَّ قَامَ دَاوُدُ بَعْدَ ذَلِكَ وَخَرَجَ مِنَ الْكَهْفِ وَنَادَى وَرَاءَ شَاوُلَ: «يَا سَيِّدِي الْمَلِكُ». وَلَمَّا الْتَفَتَ شَاوُلُ إِلَى وَرَائِهِ خَرَّ دَاوُدُ عَلَى وَجْهِهِ إِلَى الأَرْضِ وَسَجَدَ.
\par 9 وَقَالَ دَاوُدُ لِشَاوُلَ: «لِمَاذَا تَسْمَعُ كَلاَمَ النَّاسِ الْقَائِلِينَ: هُوَذَا دَاوُدُ يَطْلُبُ أَذِيَّتَكَ.
\par 10 هُوَذَا قَدْ رَأَتْ عَيْنَاكَ الْيَوْمَ هَذَا كَيْفَ دَفَعَكَ الرَّبُّ لِيَدِي فِي الْكَهْفِ, وَقِيلَ لِي أَنْ أَقْتُلَكَ, وَلَكِنَّنِي أَشْفَقْتُ عَلَيْكَ وَقُلْتُ: لاَ أَمُدُّ يَدِي إِلَى سَيِّدِي لأَنَّهُ مَسِيحُ الرَّبِّ هُوَ.
\par 11 فَانْظُرْ يَا أَبِي, انْظُرْ أَيْضاً طَرَفَ جُبَّتِكَ بِيَدِي. فَمِنْ قَطْعِي طَرَفَ جُبَّتِكَ وَعَدَمِ قَتْلِي إِيَّاكَ اعْلَمْ وَانْظُرْ أَنَّهُ لَيْسَ فِي يَدِي شَرٌّ وَلاَ جُرْمٌ, وَلَمْ أُخْطِئْ إِلَيْكَ, وَأَنْتَ تَصِيدُ نَفْسِي لِتَأْخُذَهَا.
\par 12 يَقْضِي الرَّبُّ بَيْنِي وَبَيْنَكَ وَيَنْتَقِمُ لِي الرَّبُّ مِنْكَ, وَلَكِنْ يَدِي لاَ تَكُونُ عَلَيْكَ.
\par 13 كَمَا يَقُولُ مَثَلُ الْقُدَمَاءِ: مِنَ الأَشْرَارِ يَخْرُجُ شَرٌّ. وَلَكِنْ يَدِي لاَ تَكُونُ عَلَيْكَ.
\par 14 وَرَاءَ مَنْ خَرَجَ مَلِكُ إِسْرَائِيلَ؟ وَرَاءَ مَنْ أَنْتَ مُطَارِدٌ؟ وَرَاءَ كَلْبٍ مَيِّتٍ! وَرَاءَ بُرْغُوثٍ وَاحِدٍ!
\par 15 فَيَكُونُ الرَّبُّ الدَّيَّانَ وَيَقْضِي بَيْنِي وَبَيْنَكَ, وَيَرَى وَيُحَاكِمُ مُحَاكَمَتِي وَيُنْقِذُنِي مِنْ يَدِكَ».
\par 16 فَلَمَّا فَرَغَ دَاوُدُ مِنَ التَّكَلُّمِ بِهَذَا الْكَلاَمِ إِلَى شَاوُلَ قَالَ شَاوُلُ: «أَهَذَا صَوْتُكَ يَا ابْنِي دَاوُدُ؟» وَرَفَعَ شَاوُلُ صَوْتَهُ وَبَكَى.
\par 17 ثُمَّ قَالَ لِدَاوُدَ: «أَنْتَ أَبَرُّ مِنِّي لأَنَّكَ جَازَيْتَنِي خَيْراً وَأَنَا جَازَيْتُكَ شَرّاً.
\par 18 وَقَدْ أَظْهَرْتَ الْيَوْمَ أَنَّكَ عَمِلْتَ بِي خَيْراً لأَنَّ الرَّبَّ قَدْ دَفَعَنِي بِيَدِكَ وَلَمْ تَقْتُلْنِي.
\par 19 فَإِذَا وَجَدَ رَجُلٌ عَدُوَّهُ, فَهَلْ يُطْلِقُهُ فِي طَرِيقِ خَيْرٍ؟ فَالرَّبُّ يُجَازِيكَ خَيْراً عَمَّا فَعَلْتَهُ لِي الْيَوْمَ هَذَا.
\par 20 وَالآنَ فَإِنِّي عَلِمْتُ أَنَّكَ تَكُونُ مَلِكاً وَتَثْبُتُ بِيَدِكَ مَمْلَكَةُ إِسْرَائِيلَ.
\par 21 فَاحْلِفْ لِي الآنَ بِالرَّبِّ إِنَّكَ لاَ تَقْطَعُ نَسْلِي مِنْ بَعْدِي, وَلاَ تُبِيدُ اسْمِي مِنْ بَيْتِ أَبِي».
\par 22 فَحَلَفَ دَاوُدُ لِشَاوُلَ. ثُمَّ ذَهَبَ شَاوُلُ إِلَى بَيْتِهِ, وَأَمَّا دَاوُدُ وَرِجَالُهُ فَصَعِدُوا إِلَى الْحِصْنِ.

\chapter{25}

\par 1 وَمَاتَ صَمُوئِيلُ فَاجْتَمَعَ جَمِيعُ إِسْرَائِيلَ وَنَدَبُوهُ وَدَفَنُوهُ فِي بَيْتِهِ فِي الرَّامَةِ. وَقَامَ دَاوُدُ وَنَزَلَ إِلَى بَرِّيَّةِ فَارَانَ.
\par 2 وَكَانَ رَجُلٌ فِي مَعُونٍ وَأَمْلاَكُهُ فِي الْكَرْمَلِ. وَكَانَ الرَّجُلُ عَظِيماً جِدّاً وَلَهُ ثَلاَثَةُ آلاَفٍ مِنَ الْغَنَمِ وَأَلْفٌ مِنَ الْمَعْزِ وَكَانَ يَجُزُّ غَنَمَهُ فِي الْكَرْمَلِ.
\par 3 وَاسْمُ الرَّجُلِ نَابَالُ وَاسْمُ امْرَأَتِهِ أَبِيجَايِلُ. وَكَانَتِ الْمَرْأَةُ جَيِّدَةَ الْفَهْمِ وَجَمِيلَةَ الصُّورَةِ. وَأَمَّا الرَّجُلُ فَكَانَ قَاسِياً وَرَدِيءَ الأَعْمَالِ. وَهُوَ كَالِبِيٌّ.
\par 4 فَسَمِعَ دَاوُدُ فِي الْبَرِّيَّةِ أَنَّ نَابَالَ يَجُزُّ غَنَمَهُ.
\par 5 فَأَرْسَلَ دَاوُدُ عَشَرَةَ غِلْمَانٍ وَقَالَ دَاوُدُ لِلْغِلْمَانِ: «اصْعَدُوا إِلَى الْكَرْمَلِ وَادْخُلُوا إِلَى نَابَالَ وَاسْأَلُوا بِاسْمِي عَنْ سَلاَمَتِهِ
\par 6 وَقُولُوا هَكَذَا: حَيِيتَ وَأَنْتَ سَالِمٌ وَبَيْتُكَ سَالِمٌ وَكُلُّ مَالِكَ سَالِمٌ.
\par 7 وَالآنَ قَدْ سَمِعْتُ أَنَّ عِنْدَكَ جَزَّازِينَ. حِينَ كَانَ رُعَاتُكَ مَعَنَا لَمْ نُؤْذِهِمْ وَلَمْ يُفْقَدْ لَهُمْ شَيْءٌ كُلَّ الأَيَّامِ الَّتِي كَانُوا فِيهَا فِي الْكَرْمَلِ.
\par 8 اِسْأَلْ غِلْمَانَكَ فَيُخْبِرُوكَ. فَلْيَجِدِ الْغِلْمَانُ نِعْمَةً فِي عَيْنَيْكَ لأَنَّنَا قَدْ جِئْنَا فِي يَوْمٍ طَيِّبٍ. فَأَعْطِ مَا وَجَدَتْهُ يَدُكَ لِعَبِيدِكَ وَلاِبْنِكَ دَاوُدَ».
\par 9 فَجَاءَ الْغِلْمَانُ وَكَلَّمُوا نَابَالَ حَسَبَ كُلِّ هَذَا الْكَلاَمِ بِاسْمِ دَاوُدَ وَكَفُّوا.
\par 10 فَأَجَابَ نَابَالُ عَبِيدَ دَاوُدَ: «وَقَالَ مَنْ هُوَ دَاوُدُ وَمَنْ هُوَ ابْنُ يَسَّى؟ قَدْ كَثُرَ الْيَوْمَ الْعَبِيدُ الَّذِينَ يَهْرُبُونَ كُلُّ وَاحِدٍ مِنْ أَمَامِ سَيِّدِهِ!
\par 11 أَآخُذُ خُبْزِي وَمَائِي وَذَبِيحِيَ الَّذِي ذَبَحْتُ لِجَازِّيَّ وَأُعْطِيهِ لِقَوْمٍ لاَ أَعْلَمُ مِنْ أَيْنَ هُمْ؟»
\par 12 فَتَحَوَّلَ غِلْمَانُ دَاوُدَ إِلَى طَرِيقِهِمْ وَرَجَعُوا وَجَاءُوا وَأَخْبَرُوهُ حَسَبَ كُلِّ هَذَا الْكَلاَمِ.
\par 13 فَقَالَ دَاوُدُ لِرِجَالِهِ: «لِيَتَقَلَّدْ كُلُّ وَاحِدٍ مِنْكُمْ سَيْفَهُ». فَتَقَلَّدَ كُلُّ وَاحِدٍ سَيْفَهُ. وَتَقَلَّدَ دَاوُدُ أَيْضاً سَيْفَهُ. وَصَعِدَ وَرَاءَ دَاوُدَ نَحْوُ أَرْبَعِ مِئَةِ رَجُلٍ, وَمَكَثَ مِئَتَانِ مَعَ الأَمْتِعَةِ.
\par 14 فَأَخْبَرَ أَبِيجَايِلَ امْرَأَةَ نَابَالَ غُلاَمٌ مِنَ الْغِلْمَانِ: «هُوَذَا دَاوُدُ أَرْسَلَ رُسُلاً مِنَ الْبَرِّيَّةِ لِيُبَارِكُوا سَيِّدَنَا فَثَارَ عَلَيْهِمْ.
\par 15 وَالرِّجَالُ مُحْسِنُونَ إِلَيْنَا جِدّاً, فَلَمْ نُؤْذَ وَلاَ فُقِدَ مِنَّا شَيْءٌ كُلَّ أَيَّامِ تَرَدُّدِنَا مَعَهُمْ وَنَحْنُ فِي الْحَقْلِ.
\par 16 كَانُوا سُوراً لَنَا لَيْلاً وَنَهَاراً كُلَّ الأَيَّامِ الَّتِي كُنَّا فِيهَا مَعَهُمْ نَرْعَى الْغَنَمَ.
\par 17 وَالآنَ اعْلَمِي وَانْظُرِي مَاذَا تَعْمَلِينَ, لأَنَّ الشَّرَّ قَدْ أُعِدَّ عَلَى سَيِّدِنَا وَعَلَى بَيْتِهِ, وَهُوَ ابْنُ لَئِيمٍ لاَ يُمْكِنُ الْكَلاَمُ مَعَهُ».
\par 18 فَبَادَرَتْ أَبِيجَايِلُ وَأَخَذَتْ مِئَتَيْ رَغِيفِ خُبْزٍ وَزِقَّيْ خَمْرٍ وَخَمْسَةَ خِرْفَانٍ مُهَيَّأَةً وَخَمْسَ كَيْلاَتٍ مِنَ الْفَرِيكِ وَمِئَتَيْ عُنْقُودٍ مِنَ الزَّبِيبِ وَمِئَتَيْ قُرْصٍ مِنَ التِّينِ وَوَضَعَتْهَا عَلَى الْحَمِيرِ
\par 19 وَقَالَتْ لِغِلْمَانِهَا: «اعْبُرُوا قُدَّامِي. هَئَنَذَا جَائِيَةٌ وَرَاءَكُمْ». وَلَمْ تُخْبِرْ رَجُلَهَا نَابَالَ.
\par 20 وَفِيمَا هِيَ رَاكِبَةٌ عَلَى الْحِمَارِ وَنَازِلَةٌ فِي سُتْرَةِ الْجَبَلِ إِذَا بِدَاوُدَ وَرِجَالِهُِ مُنْحَدِرُونَ لاِسْتِقْبَالِهَا, فَصَادَفَتْهُمْ.
\par 21 وَقَالَ دَاوُدُ: «إِنَّمَا بَاطِلاً حَفِظْتُ كُلَّ مَا لِهَذَا فِي الْبَرِّيَّةِ فَلَمْ يُفْقَدْ مِنْ كُلِّ مَا لَهُ شَيْءٌ, فَكَافَأَنِي شَرّاً بَدَلَ خَيْرٍ.
\par 22 هَكَذَا يَصْنَعُ اللَّهُ لأَعْدَاءِ دَاوُدَ وَهَكَذَا يَزِيدُ إِنْ أَبْقَيْتُ ذَكَراً مِنْ كُلِّ مَا لَهُ إِلَى ضُوءِ الصَّبَاحِ».
\par 23 وَلَمَّا رَأَتْ أَبِيجَايِلُ دَاوُدَ أَسْرَعَتْ وَنَزَلَتْ عَنِ الْحِمَارِ, وَسَقَطَتْ أَمَامَ دَاوُدَ عَلَى وَجْهِهَا وَسَجَدَتْ إِلَى الأَرْضِ,
\par 24 وَسَقَطَتْ عَلَى رِجْلَيْهِ وَقَالَتْ: «عَلَيَّ أَنَا يَا سَيِّدِي هَذَا الذَّنْبُ, وَدَعْ أَمَتَكَ تَتَكَلَّمُ فِي أُذُنَيْكَ وَاسْمَعْ كَلاَمَ أَمَتِكَ.
\par 25 لاَ يَضَعَنَّ سَيِّدِي قَلْبَهُ عَلَى الرَّجُلِ اللَّئِيمِ هَذَا, عَلَى نَابَالَ, لأَنَّ كَاسْمِهِ هَكَذَا هُوَ. نَابَالُ اسْمُهُ وَالْحَمَاقَةُ عِنْدَهُ. وَأَنَا أَمَتَكَ لَمْ أَرَ غِلْمَانَ سَيِّدِي الَّذِينَ أَرْسَلْتَهُمْ.
\par 26 وَالآنَ يَا سَيِّدِي حَيٌّ هُوَ الرَّبُّ وَحَيَّةٌ هِيَ نَفْسُكَ إِنَّ الرَّبَّ قَدْ مَنَعَكَ عَنْ إِتْيَانِ الدِّمَاءِ وَانْتِقَامِ يَدِكَ لِنَفْسِكَ. وَالآنَ فَلِْيَكُنْ كَنَابَالَ أَعْدَاؤُكَ وَالَّذِينَ يَطْلُبُونَ الشَّرَّ لِسَيِّدِي.
\par 27 وَالآنَ هَذِهِ الْبَرَكَةُ الَّتِي أَتَتْ بِهَا جَارِيَتُكَ إِلَى سَيِّدِي فَلْتُعْطَ لِلْغِلْمَانِ السَّائِرِينَ وَرَاءَ سَيِّدِي.
\par 28 وَاصْفَحْ عَنْ ذَنْبِ أَمَتِكَ لأَنَّ الرَّبَّ يَصْنَعُ لِسَيِّدِي بَيْتاً أَمِيناً, لأَنَّ سَيِّدِي يُحَارِبُ حُرُوبَ الرَّبِّ, وَلَمْ يُوجَدْ فِيكَ شَرٌّ كُلَّ أَيَّامِكَ.
\par 29 وَقَدْ قَامَ رَجُلٌ لِيُطَارِدَكَ وَيَطْلُبَ نَفْسَكَ, وَلَكِنْ نَفْسُ سَيِّدِي لِتَكُنْ مَحْزُومَةً فِي حُزْمَةِ الْحَيَاةِ مَعَ الرَّبِّ إِلَهِكَ. وَأَمَّا نَفْسُ أَعْدَائِكَ فَلْيَرْمِ بِهَا كَمَا مِنْ وَسَطِ كَفَّةِ الْمِقْلاَعِ.
\par 30 وَيَكُونُ عِنْدَمَا يَصْنَعُ الرَّبُّ لِسَيِّدِي حَسَبَ كُلِّ مَا تَكَلَّمَ بِهِ مِنَ الْخَيْرِ مِنْ أَجْلِكَ, وَيُقِيمُكَ رَئِيساً عَلَى إِسْرَائِيلَ,
\par 31 أَنَّهُ لاَ تَكُونُ لَكَ هَذِهِ مَصْدَمَةً وَمَعْثَرَةَ قَلْبٍ لِسَيِّدِي أَنَّكَ قَدْ سَفَكْتَ دَماً عَفْواً, أَوْ أَنَّ سَيِّدِي قَدِ انْتَقَمَ لِنَفْسِهِ. وَإِذَا أَحْسَنَ الرَّبُّ إِلَى سَيِّدِي فَاذْكُرْ أَمَتَكَ».
\par 32 فَقَالَ دَاوُدُ لأَبِيجَايِلَ: «مُبَارَكٌ الرَّبُّ إِلَهُ إِسْرَائِيلَ الَّذِي أَرْسَلَكِ هَذَا الْيَوْمَ لاِسْتِقْبَالِي,
\par 33 وَمُبَارَكٌ عَقْلُكِ وَمُبَارَكَةٌ أَنْتِ لأَنَّكِ مَنَعْتِنِي الْيَوْمَ مِنْ إِتْيَانِ الدِّمَاءِ وَانْتِقَامِ يَدِي لِنَفْسِي.
\par 34 وَلَكِنْ حَيٌّ هُوَ الرَّبُّ إِلَهُ إِسْرَائِيلَ الَّذِي مَنَعَنِي عَنْ أَذِيَّتِكِ, إِنَّكِ لَوْ لَمْ تُبَادِرِي وَتَأْتِي لاِسْتِقْبَالِي لَمَا أُبْقِيَ ذَكَرٌ لِنَابَالَ إِلَى ضُوءِ الصَّبَاحِ».
\par 35 فَأَخَذَ دَاوُدُ مِنْ يَدِهَا مَا أَتَتْ بِهِ إِلَيْهِ وَقَالَ لَهَا: «اصْعَدِي بِسَلاَمٍ إِلَى بَيْتِكِ. انْظُرِي. قَدْ سَمِعْتُ لِصَوْتِكِ وَرَفَعْتُ وَجْهَكِ».
\par 36 فَجَاءَتْ أَبِيجَايِلُ إِلَى نَابَالَ وَإِذَا وَلِيمَةٌ عِنْدَهُ فِي بَيْتِهِ كَوَلِيمَةِ مَلِكٍ. وَكَانَ نَابَالُ قَدْ طَابَ قَلْبُهُ وَكَانَ سَكْرَانَ جِدّاً, فَلَمْ تُخْبِرْهُ بِشَيْءٍ صَغِيرٍ أَوْ كَبِيرٍ إِلَى ضُوءِ الصَّبَاحِ.
\par 37 وَفِي الصَّبَاحِ عَُِنْدَ خُرُوجِ الْخَمْرِ مِنْ نَابَالَ أَخْبَرَتْهُ امْرَأَتُهُ بِهَذَا الْكَلاَمِ, فَمَاتَ قَلْبُهُ دَاخِلَهُ وَصَارَ كَحَجَرٍ.
\par 38 وَبَعْدَ نَحْوِ عَشَرَةِ أَيَّامٍ ضَرَبَ الرَّبُّ نَابَالَ فَمَاتَ.
\par 39 فَلَمَّا سَمِعَ دَاوُدُ أَنَّ نَابَالَ قَدْ مَاتَ قَالَ: «مُبَارَكٌ الرَّبُّ الَّذِي انْتَقَمَ نَقْمَةَ تَعْيِيرِي مِنْ يَدِ نَابَالَ, وَأَمْسَكَ عَبْدَهُ عَنِ الشَّرِّ, وَرَدَّ الرَّبُّ شَرَّ نَابَالَ عَلَى رَأْسِهِ». وَأَرْسَلَ دَاوُدُ وَتَكَلَّمَ مَعَ أَبِيجَايِلَ لِيَتَّخِذَهَا لَهُ امْرَأَةً.
\par 40 فَجَاءَ عَبِيدُ دَاوُدَ إِلَى أَبِيجَايِلَ إِلَى الْكَرْمَلِ وَقَالُوا لَهَا: «إِنَّ دَاوُدَ قَدْ أَرْسَلَنَا إِلَيْكِ لِنَتَّخِذَكِ لَهُ امْرَأَةً».
\par 41 فَقَامَتْ وَسَجَدَتْ عَلَى وَجْهِهَا إِلَى الأَرْضِ وَقَالَتْ: «هُوَذَا أَمَتُكَ جَارِيَةٌ لِغَسْلِ أَرْجُلِ عَبِيدِ سَيِّدِي».
\par 42 ثُمَّ بَادَرَتْ وَقَامَتْ وَرَكِبَتِ الْحِمَارَ مَعَ خَمْسِ فَتَيَاتٍ لَهَا ذَاهِبَاتٍ وَرَاءَهَا, وَسَارَتْ وَرَاءَ رُسُلِ دَاوُدَ وَصَارَتْ لَهُ امْرَأَةً.
\par 43 ثُمَّ أَخَذَ دَاوُدُ أَخِينُوعَمَ مِنْ يَزْرَعِيلَ فَكَانَتَا لَهُ كِلْتَاهُمَا امْرَأَتَيْنِ.
\par 44 فَأَعْطَى شَاوُلُ مِيكَالَ ابْنَتَهُ امْرَأَةَ دَاوُدَ لِفَلْطِي بْنِ لاَيِشَ الَّذِي مِنْ جَلِّيمَ.

\chapter{26}

\par 1 ثُمَّ جَاءَ الزِّيفِيُّونَ إِلَى شَاوُلَ إِلَى جِبْعَةَ قَائِلِينَ: «أَلَيْسَ دَاوُدُ مُخْتَفِياً فِي تَلِّ حَخِيلَةَ الَّذِي مُقَابَِلَ الْقَفْرِ؟»
\par 2 فَقَامَ شَاوُلُ وَنَزَلَ إِلَى بَرِّيَّةِ زِيفٍ وَمَعَهُ ثَلاَثَةُ آلاَفِ رَجُلٍ مُنْتَخَبِي إِسْرَائِيلَ لِيُفَتِّشَ عَلَى دَاوُدَ فِي بَرِّيَّةِ زِيفٍ.
\par 3 وَنَزَلَ شَاوُلُ فِي تَلِّ حَخِيلَةَ الَّذِي مُقَابَِلَ الْقَفْرِ عَلَى الطَّرِيقِ. وَكَانَ دَاوُدُ مُقِيماً فِي الْبَرِّيَّةِ. فَلَمَّا رَأَى أَنَّ شَاوُلَ قَدْ جَاءَ وَرَاءَهُ إِلَى الْبَرِّيَّةِ
\par 4 أَرْسَلَ دَاوُدُ جَوَاسِيسَ وَعَلِمَ بِالْيَقِينِ أَنَّ شَاوُلَ قَدْ جَاءَ.
\par 5 فَقَامَ دَاوُدُ وَجَاءَ إِلَى الْمَكَانِ الَّذِي نَزَلَ فِيهِ شَاوُلُ, وَنَظَرَ دَاوُدُ الْمَكَانَ الَّذِي اضْطَجَعَ فِيهِ شَاوُلُ وَأَبْنَيْرُ بْنُ نَيْرٍ رَئِيسُ جَيْشِهِ. وَكَانَ شَاوُلُ مُضْطَجِعاً عِنْدَ الْمِتْرَاسِ وَالشَّعْبُ نُزُولٌ حَوَالَيْهِ.
\par 6 فَقَالَ دَاوُدُ لأَخِيمَالِكَ الْحِثِّيَّ وَأَبِيشَايَ ابْنِ صَرُوِيَّةَ أَخِي يُوآبَ: «مَنْ يَنْزِلُ مَعِي إِلَى شَاوُلَ إِلَى الْمَحَلَّةِ؟» فَقَالَ أَبِيشَايُ: «أَنَا أَنْزِلُ مَعَكَ».
\par 7 فَجَاءَ دَاوُدُ وَأَبِيشَايُ إِلَى الشَّعْبِ لَيْلاً وَإِذَا بِشَاوُلَ مُضْطَجِعٌ نَائِمٌ عِنْدَ الْمِتْرَاسِ وَرُمْحُهُ مَرْكُوزٌ فِي الأَرْضِ عِنْدَ رَأْسِهِ وَأَبْنَيْرُ وَالشَّعْبُ مُضْطَجِعُونَ حَوَالَيْهِ.
\par 8 فَقَالَ أَبِيشَايُ لِدَاوُدَ: «قَدْ حَبَسَ اللَّهُ الْيَوْمَ عَدُوَّكَ فِي يَدِكَ. فَدَعْنِيَ الآنَ أَضْرِبْهُ بِالرُّمْحِ إِلَى الأَرْضِ دُفْعَةً وَاحِدَةً وَلاَ أُثَنِّي عَلَيْهِ».
\par 9 فَقَالَ دَاوُدُ لأَبِيشَايَ: «لاَ تُهْلِكْهُ, فَمَنِ الَّذِي يَمُدُّ يَدَهُ إِلَى مَسِيحِ الرَّبِّ وَيَتَبَرَّأُ؟»
\par 10 وَقَالَ دَاوُدُ: «حَيٌّ هُوَ الرَّبُّ, إِنَّ الرَّبَّ سَوْفَ يَضْرِبُهُ أَوْ يَأْتِي يَوْمُهُ فَيَمُوتُ أَوْ يَنْزِلُ إِلَى الْحَرْبِ وَيَهْلِكُ.
\par 11 حَاشَا لِي مِنْ قِبَلِ الرَّبِّ أَنْ أَمُدَّ يَدِي إِلَى مَسِيحِ الرَّبِّ! وَالآنَ فَخُذِ الرُّمْحَ الَّذِي عِنْدَ رَأْسِهِ وَكُوزَ الْمَاءِ وَهَلُمَّ».
\par 12 فَأَخَذَ دَاوُدُ الرُّمْحَ وَكُوزَ الْمَاءِ مِنْ عِنْدِ رَأْسِ شَاوُلَ وَذَهَبَا, وَلَمْ يَرَ وَلاَ عَلِمَ وَلاَ انْتَبَهَ أَحَدٌ لأَنَّهُمْ جَمِيعاً كَانُوا نِيَاماً, لأَنَّ سُبَاتَ الرَّبِّ وَقَعَ عَلَيْهِمْ.
\par 13 وَعَبَرَ دَاوُدُ إِلَى الْعَبْرِ وَوَقَفَ عَلَى رَأْسِ الْجَبَلِ عَنْ بُعْدٍ, وَالْمَسَافَةُ بَيْنَهُمْ كَبِيرَةٌ.
\par 14 وَنَادَى دَاوُدُ الشَّعْبَ وَأَبْنَيْرَ بْنَ نَيْرٍ: «أَمَا تُجِيبُ يَا أَبْنَيْرُ؟» فَأَجَابَ أَبْنَيْرُ: «مَنْ أَنْتَ الَّذِي يُنَادِي الْمَلِكَ؟»
\par 15 فَقَالَ دَاوُدُ لأَبْنَيْرَ: «أَمَا أَنْتَ رَجُلٌ, وَمَنْ مِثْلُكَ فِي إِسْرَائِيلَ؟ فَلِمَاذَا لَمْ تَحْرُسْ سَيِّدَكَ الْمَلِكَ؟ لأَنَّهُ قَدْ جَاءَ وَاحِدٌ مِنَ الشَّعْبِ لِيُهْلِكَ الْمَلِكَ سَيِّدَكَ!
\par 16 لَيْسَ حَسَناً هَذَا الأَمْرُ الَّذِي عَمِلْتَ! حَيٌّ هُوَ الرَّبُّ إِنَّكُمْ أَبْنَاءُ الْمَوْتِ أَنْتُمْ لأَنَّكُمْ لَمْ تُحَافِظُوا عَلَى سَيِّدِكُمْ, عَلَى مَسِيحِ الرَّبِّ. فَانْظُرِ الآنَ أَيْنَ هُوَ رُمْحُ الْمَلِكِ وَكُوزُ الْمَاءِ الَّذِي كَانَ عِنْدَ رَأْسِهِ».
\par 17 وَعَرَفَ شَاوُلُ صَوْتَ دَاوُدَ فَقَالَ: «أَهَذَا هُوَ صَوْتُكَ يَا ابْنِي دَاوُدُ؟» فَقَالَ دَاوُدُ: «إِنَّهُ صَوْتِي يَا سَيِّدِي الْمَلِكَ».
\par 18 ثُمَّ قَالَ: «لِمَاذَا يَسْعَى سَيِّدِي وَرَاءَ عَبْدِهِ, لأَنِّي مَاذَا عَمِلْتُ وَأَيُّ شَرٍّ بِيَدِي؟
\par 19 وَالآنَ فَلْيَسْمَعْ سَيِّدِي الْمَلِكُ كَلاَمَ عَبْدِهِ. فَإِنْ كَانَ الرَّبُّ قَدْ أَهَاجَكَ ضِدِّي فَلْيَشْتَمَّ تَقْدِمَةً. وَإِنْ كَانَ بَنُو النَّاسِ فَلْيَكُونُوا مَلْعُونِينَ أَمَامَ الرَّبِّ لأَنَّهُمْ قَدْ طَرَدُونِي الْيَوْمَ مِنَ الاِنْضِمَامِ إِلَى نَصِيبِ الرَّبِّ قَائِلِينَ: اذْهَبِ اعْبُدْ آلِهَةً أُخْرَى.
\par 20 وَالآنَ لاَ يَسْقُطْ دَمِي إِلَى الأَرْضِ أَمَامَ وَجْهِ الرَّبِّ. لأَنَّ مَلِكَ إِسْرَائِيلَ قَدْ خَرَجَ لِيُفَتِّشَ عَلَى بُرْغُوثٍ وَاحِدٍ! كَمَا يُتْبَعُ الْحَجَلُ فِي الْجِبَالِ!».
\par 21 فَقَالَ شَاوُلُ: «قَدْ أَخْطَأْتُ. ارْجِعْ يَا ابْنِي دَاوُدُ لأَنِّي لاَ أُسِيءُ إِلَيْكَ بَعْدُ مِنْ أَجْلِ أَنَّ نَفْسِي كَانَتْ كَرِيمَةً فِي عَيْنَيْكَ الْيَوْمَ. هُوَذَا قَدْ حَمِقْتُ وَضَلَلْتُ كَثِيراً جِدّاً».
\par 22 فَأَجَابَ دَاوُدُ: «هُوَذَا رُمْحُ الْمَلِكِ, فَلْيَعْبُرْ وَاحِدٌ مِنَ الْغِلْمَانِ وَيَأْخُذْهُ.
\par 23 وَالرَّبُّ يَرُدُّ عَلَى كُلِّ وَاحِدٍ بِرَّهُ وَأَمَانَتَهُ, لأَنَّهُ قَدْ دَفَعَكَ الرَّبُّ الْيَوْمَ لِيَدِي وَلَمْ أَشَأْ أَنْ أَمُدَّ يَدِي إِلَى مَسِيحِ الرَّبِّ.
\par 24 وَهُوَذَا كَمَا كَانَتْ نَفْسُكَ عَظِيمَةً الْيَوْمَ فِي عَيْنَيَّ, كَذَلِكَ لِتَعْظُمْ نَفْسِي فِي عَيْنَيِ الرَّبِّ فَيَنْقُذْنِي مِنْ كُلِّ ضِيقٍ».
\par 25 فَقَالَ شَاوُلُ لِدَاوُدَ: «مُبَارَكٌ أَنْتَ يَا ابْنِي دَاوُدُ فَإِنَّكَ تَفْعَلُ وَتَقْدِرُ. ثُمَّ ذَهَبَ دَاوُدُ فِي طَرِيقِهِ وَرَجَعَ شَاوُلُ إِلَى مَكَانِهِ.

\chapter{27}

\par 1 وَقَالَ دَاوُدُ فِي قَلْبِهِ: «إِنِّي سَأَهْلِكُ يَوْماً بِيَدِ شَاوُلَ, فَلاَ شَيْءَ خَيْرٌ لِي مِنْ أَنْ أُفْلِتَ إِلَى أَرْضِ الْفِلِسْطِينِيِّينَ فَيَيْأَسُ شَاوُلُ مِنِّي فَلاَ يُفَتِّشُ عَلَيَّ بَعْدُ فِي جَمِيعِ تُخُومِ إِسْرَائِيلَ, فَأَنْجُو مِنْ يَدِهِ».
\par 2 فَقَامَ دَاوُدُ وَعَبَرَ هُوَ وَالسِّتُّ مِئَةِ الرَّجُلِ الَّذِينَ مَعَهُ إِلَى أَخِيشَ بْنِ مَعُوكَ مَلِكِ جَتٍّ
\par 3 وَأَقَامَ دَاوُدُ عِنْدَ أَخِيشَ فِي جَتٍّ هُوَ وَرِجَالُهُ, كُلُّ وَاحِدٍ وَبَيْتُهُ, دَاوُدُ وَامْرَأَتَاهُ أَخِينُوعَمُ الْيَزْرَعِيلِيَّةُ وَأَبِيجَايِلُ امْرَأَةُ نَابَالَ الْكَرْمَلِيَّةُ.
\par 4 فَأُخْبِرَ شَاوُلُ أَنَّ دَاوُدَ قَدْ هَرَبَ إِلَى جَتٍّ فَلَمْ يَعُدْ أَيْضاً يُفَتِّشُ عَلَيْهِ.
\par 5 فَقَالَ دَاوُدُ لأَخِيشَ: «إِنْ كُنْتُ قَدْ وَجَدْتُ نِعْمَةً فِي عَيْنَيْكَ فَلْيُعْطُونِي مَكَاناً فِي إِحْدَى قُرَى الْحَقْلِ فَأَسْكُنَ هُنَاكَ. وَلِمَاذَا يَسْكُنُ عَبْدُكَ فِي مَدِينَةِ الْمَمْلَكَةِ مَعَكَ؟»
\par 6 فَأَعْطَاهُ أَخِيشُ فِي ذَلِكَ الْيَوْمِ صِقْلَغَ. لِذَلِكَ صَارَتْ صِقْلَغُ لِمُلُوكِ يَهُوذَا إِلَى هَذَا الْيَوْمِ.
\par 7 وَكَانَ عَدَدُ الأَيَّامِ الَّتِي سَكَنَ فِيهَا دَاوُدُ فِي بِلاَدِ الْفِلِسْطِينِيِّينَ سَنَةً وَأَرْبَعَةَ أَشْهُرٍ.
\par 8 وَصَعِدَ دَاوُدُ وَرِجَالُهُ وَغَزُوا الْجَشُورِيِّينَ وَالْجَرِزِّيِّينَ وَالْعَمَالِقَةَ لأَنَّ هَؤُلاَءِ مِنْ قَدِيمٍ سُكَّانُ الأَرْضِ مِنْ عِنْدِ شُورٍ إِلَى أَرْضِ مِصْرَ.
\par 9 وَضَرَبَ دَاوُدُ الأَرْضَ, وَلَمْ يَسْتَبْقِ رَجُلاً وَلاَ امْرَأَةً, وَأَخَذَ غَنَماً وَبَقَراً وَحَمِيراً وَجِمَالاً وَثِيَاباً وَرَجَعَ وَجَاءَ إِلَى أَخِيشَ.
\par 10 فَقَالَ أَخِيشُ: «إِذاً لَمْ تَغْزُوا الْيَوْمَ». فَقَالَ دَاوُدُ: «بَلَى. عَلَى جَنُوبِيِّ يَهُوذَا وَجَنُوبِيِّ الْيَرْحَمْئِيلِيِّينَ وَجَنُوبِيِّ الْقِينِيِّينَ».
\par 11 فَلَمْ يَسْتَبْقِ دَاوُدُ رَجُلاً وَلاَ امْرَأَةً حَتَّى يَأْتِيَ إِلَى جَتٍّ إِذْ قَالَ: «لِئَلَّا يُخْبِرُوا عَنَّا قَائِلِينَ: هَكَذَا فَعَلَ دَاوُدُ». وَهَكَذَا عَادَتُهُ كُلَّ أَيَّامِ إِقَامَتِهِ فِي بِلاَدِ الْفِلِسْطِينِيِّينَ.
\par 12 فَصَدَّقَ أَخِيشُ دَاوُدَ قَائِلاً: «قَدْ صَارَ مَكْرُوهاً لَدَى شَعْبِهِ إِسْرَائِيلَ, فَيَكُونُ لِي عَبْداً إِلَى الأَبَدِ».

\chapter{28}

\par 1 وَكَانَ فِي تِلْكَ الأَيَّامِ أَنَّ الْفِلِسْطِينِيِّينَ جَمَعُوا جُيُوشَهُمْ لِيُحَارِبُوا إِسْرَائِيلَ. فَقَالَ أَخِيشُ لِدَاوُدَ: «اعْلَمْ يَقِيناً أَنَّكَ سَتَخْرُجُ مَعِي فِي الْجَيْشِ أَنْتَ وَرِجَالُكَ».
\par 2 فَقَالَ دَاوُدُ لأَخِيشَ: «لِذَلِكَ أَنْتَ سَتَعْلَمُ مَا يَفْعَلُ عَبْدُكَ». فَقَالَ أَخِيشُ لِدَاوُدَ: «لِذَلِكَ أَجْعَلُكَ حَارِساً لِرَأْسِي كُلَّ الأَيَّامِ».
\par 3 وَمَاتَ صَمُوئِيلُ وَنَدَبَهُ كُلُّ إِسْرَائِيلَ وَدَفَنُوهُ فِي الرَّامَةِ فِي مَدِينَتِهِ. وَكَانَ شَاوُلُ قَدْ نَفَى أَصْحَابَ الْجَانِّ وَالتَّوَابِعِ مِنَ الأَرْضِ.
\par 4 فَاجْتَمَعَ الْفِلِسْطِينِيُّونَ وَجَاءُوا وَنَزَلُوا فِي شُونَمَ وَجَمَعَ شَاوُلُ جَمِيعَ إِسْرَائِيلَ وَنَزَلَ فِي جِلْبُوعَ.
\par 5 وَلَمَّا رَأَى شَاوُلُ جَيْشَ الْفِلِسْطِينِيِّينَ خَافَ وَاضْطَرَبَ قَلْبُهُ جِدّاً.
\par 6 فَسَأَلَ شَاوُلُ مِنَ الرَّبِّ, فَلَمْ يُجِبْهُ الرَّبُّ لاَ بِالأَحْلاَمِ وَلاَ بِالأُورِيمِ وَلاَ بِالأَنْبِيَاءِ.
\par 7 فَقَالَ شَاوُلُ لِعَبِيدِهِ: «فَتِّشُوا لِي عَلَى امْرَأَةٍ صَاحِبَةِ جَانٍّ فَأَذْهَبَ إِلَيْهَا وَأَسْأَلَهَا». فَقَالَ لَهُ عَبِيدُهُ: «هُوَذَا امْرَأَةٌ صَاحِبَةُ جَانٍّ فِي عَيْنِ دُورٍ».
\par 8 فَتَنَكَّرَ شَاوُلُ وَلَبِسَ ثِيَاباً أُخْرَى, وَذَهَبَ هُوَ وَرَجُلاَنِ مَعَهُ وَجَاءُوا إِلَى الْمَرْأَةِ لَيْلاً. وَقَالَ: «اعْرِفِي لِي بِالْجَانِّ وَأَصْعِدِي لِي مَنْ أَقُولُ لَكِ».
\par 9 فَقَالَتْ لَهُ الْمَرْأَةُ: «هُوَذَا أَنْتَ تَعْلَمُ مَا فَعَلَ شَاوُلُ, كَيْفَ قَطَعَ أَصْحَابَ الْجَانِّ وَالتَّوَابِعِ مِنَ الأَرْضِ. فَلِمَاذَا تَضَعُ شَرَكاً لِنَفْسِي لِتُمِيتَهَا؟»
\par 10 فَحَلَفَ لَهَا شَاوُلُ بِالرَّبِّ: «حَيٌّ هُوَ الرَّبُّ, إِنَّهُ لاَ يَلْحَقُكِ إِثْمٌ فِي هَذَا الأَمْرِ».
\par 11 فَقَالَتِ الْمَرْأَةُ: «مَنْ أُصْعِدُ لَكَ؟» فَقَالَ: «أَصْعِدِي لِي صَمُوئِيلَ».
\par 12 فَلَمَّا رَأَتِ الْمَرْأَةُ صَمُوئِيلَ صَرَخَتْ بِصَوْتٍ عَظِيمٍ, وَقَالَتِ لِشَاوُلَ: «لِمَاذَا خَدَعْتَنِي وَأَنْتَ شَاوُلُ؟»
\par 13 فَقَالَ لَهَا الْمَلِكُ: «لاَ تَخَافِي. فَمَاذَا رَأَيْتِ؟» فَقَالَتِ الْمَرْأَةُ لِشَاوُلَ: «رَأَيْتُ آلِهَةً يَصْعَدُونَ مِنَ الأَرْضِ».
\par 14 فَقَالَ لَهَا: «مَا هِيَ صُورَتُهُ؟» فَقَالَتْ: «رَجُلٌ شَيْخٌ صَاعِدٌ وَهُوَ مُغَطًّى بِجُبَّةٍ». فَعَلِمَ شَاوُلُ أَنَّهُ صَمُوئِيلُ, فَخَرَّ عَلَى وَجْهِهِ إِلَى الأَرْضِ وَسَجَدَ.
\par 15 فَقَالَ صَمُوئِيلُ لِشَاوُلَ: «لِمَاذَا أَقْلَقْتَنِي بِإِصْعَادِكَ إِيَّايَ؟» فَقَالَ شَاوُلُ: «قَدْ ضَاقَ بِي الأَمْرُ جِدّاً. الْفِلِسْطِينِيُّونَ يُحَارِبُونَنِي, وَالرَّبُّ فَارَقَنِي وَلَمْ يَعُدْ يُجِيبُنِي لاَ بِالأَنْبِيَاءِ وَلاَ بِالأَحْلاَمِ. فَدَعَوْتُكَ لِتُعْلِمَنِي مَاذَا أَصْنَعُ».
\par 16 فَقَالَ صَمُوئِيلُ: «وَلِمَاذَا تَسْأَلُنِي وَالرَّبُّ قَدْ فَارَقَكَ وَصَارَ عَدُوَّكَ؟
\par 17 وَقَدْ فَعَلَ الرَّبُّ لِنَفْسِهِ كَمَا تَكَلَّمَ عَنْ يَدِي, وَقَدْ شَقَّ الرَّبُّ الْمَمْلَكَةَ مِنْ يَدِكَ وَأَعْطَاهَا لِقَرِيبِكَ دَاوُدَ.
\par 18 لأَنَّكَ لَمْ تَسْمَعْ لِصَوْتِ الرَّبِّ وَلَمْ تَفْعَلْ حُمُوَّ غَضَبِهِ فِي عَمَالِيقَ, لِذَلِكَ قَدْ فَعَلَ الرَّبُّ بِكَ هَذَا الأَمْرَ الْيَوْمَ.
\par 19 وَيَدْفَعُ الرَّبُّ إِسْرَائِيلَ أَيْضاً مَعَكَ لِيَدِ الْفِلِسْطِينِيِّينَ. وَغَداً أَنْتَ وَبَنُوكَ تَكُونُونَ مَعِي, وَيَدْفَعُ الرَّبُّ جَيْشَ إِسْرَائِيلَ أَيْضاً لِيَدِ الْفِلِسْطِينِيِّينَ».
\par 20 فَأَسْرَعَ شَاوُلُ وَسَقَطَ عَلَى طُولِهِ إِلَى الأَرْضِ وَخَافَ جِدّاً مِنْ كَلاَمِ صَمُوئِيلَ, وَأَيْضاً لَمْ تَكُنْ فِيهِ قُوَّةٌ, لأَنَّهُ لَمْ يَأْكُلْ طَعَاماً النَّهَارَ كُلَّهُ وَاللَّيْلَ.
\par 21 ثُمَّ جَاءَتِ الْمَرْأَةُ إِلَى شَاوُلَ وَرَأَتْ أَنَّهُ مُرْتَاعٌ جِدّاً, فَقَالَتْ لَهُ: «هُوَذَا قَدْ سَمِعَتْ جَارِيَتُكَ لِصَوْتِكَ فَوَضَعْتُ نَفْسِي فِي كَفِّي وَسَمِعْتُ لِكَلاَمِكَ الَّذِي كَلَّمْتَنِي بِهِ.
\par 22 وَالآنَ اسْمَعْ أَنْتَ أَيْضاً لِصَوْتِ جَارِيَتِكَ فَأَضَعَ قُدَّامَكَ كِسْرَةَ خُبْزٍ وَكُلْ, فَتَكُونَ فِيكَ قُوَّةٌ إِذْ تَسِيرُ فِي الطَّرِيقِ».
\par 23 فَأَبَى وَقَالَ: «لاَ آكُلُ». فَأَلَحَّ عَلَيْهِ عَبْدَاهُ وَالْمَرْأَةُ أَيْضاً, فَسَمِعَ لِصَوْتِهِمْ وَقَامَ عَنِ الأَرْضِ وَجَلَسَ عَلَى السَّرِيرِ.
\par 24 وَكَانَ لِلْمَرْأَةِ عِجْلٌ مُسَمَّنٌ فِي الْبَيْتِ, فَأَسْرَعَتْ وَذَبَحَتْهُ وَأَخَذَتْ دَقِيقاً وَعَجَنَتْهُ وَخَبَزَتْ فَطِيراً,
\par 25 ثُمَّ قَدَّمَتْهُ أَمَامَ شَاوُلَ وَأَمَامَ عَبْدَيْهِ فَأَكَلُوا. وَقَامُوا وَذَهَبُوا فِي تِلْكَ اللَّيْلَةِ.

\chapter{29}

\par 1 وَجَمَعَ الْفِلِسْطِينِيُّونَ جَمِيعَ جُيُوشِهِمْ إِلَى أَفِيقَ. وَكَانَ الْإِسْرَائِيلِيُّونَ نَازِلِينَ عَلَى الْعَيْنِ الَّتِي فِي يَزْرَعِيلَ.
\par 2 وَعَبَرَ أَقْطَابُ الْفِلِسْطِينِيِّينَ مِئَاتٍ وَأُلُوفاً, وَعَبَرَ دَاوُدُ وَرِجَالُهُ فِي الْمُؤَخَّرَةِ مَعَ أَخِيشَ.
\par 3 فَقَالَ رُؤَسَاءُ الْفِلِسْطِينِيِّينَ: «مَا هَؤُلاَءِ الْعِبْرَانِيُّونَ؟» فَقَالَ أَخِيشُ لِرُؤَسَاءِ الْفِلِسْطِينِيِّينَ: «أَلَيْسَ هَذَا دَاوُدَ عَبْدَ شَاوُلَ مَلِكِ إِسْرَائِيلَ الَّذِي كَانَ مَعِي هَذِهِ الأَيَّامَ أَوْ هَذِهِ السِّنِينَ, وَلَمْ أَجِدْ فِيهِ شَيْئاً مِنْ يَوْمِ نُزُولِهِ إِلَى هَذَا الْيَوْمِ».
\par 4 وَسَخَطَ عَلَيْهِ رُؤَسَاءُ الْفِلِسْطِينِيِّينَ, وَقَالُوا لَهُ: «أَرْجِعِ الرَّجُلَ فَيَرْجِعَ إِلَى مَوْضِعِهِ الَّذِي عَيَّنْتَ لَهُ, وَلاَ يَنْزِلَ مَعَنَا إِلَى الْحَرْبِ وَلاَ يَكُونَ لَنَا عَدُوّاً فِي الْحَرْبِ. فَبِمَاذَا يُرْضِي هَذَا سَيِّدَهُ؟ أَلَيْسَ بِرُؤُوسِ أُولَئِكَ الرِّجَالِ؟
\par 5 أَلَيْسَ هَذَا هُوَ دَاوُدُ الَّذِي غَنَّيْنَ لَهُ بِالرَّقْصِ قَائِلاَتٍ: ضَرَبَ شَاوُلُ أُلُوفَهُ وَدَاوُدُ رَبَوَاتِهِ؟».
\par 6 فَدَعَا أَخِيشُ دَاوُدَ وَقَالَ لَهُ: «حَيٌّ هُوَ الرَّبُّ إِنَّكَ أَنْتَ مُسْتَقِيمٌ, وَخُرُوجُكَ وَدُخُولُكَ مَعِي فِي الْجَيْشِ صَالِحٌ فِي عَيْنَيَّ لأَنِّي لَمْ أَجِدْ فِيكَ شَرّاً مِنْ يَوْمِ جِئْتَ إِلَيَّ إِلَى الْيَوْمِ. وَأَمَّا فِي أَعْيُنِ الأَقْطَابِ فَلَسْتَ بِصَالِحٍ.
\par 7 فَالآنَ ارْجِعْ وَاذْهَبْ بِسَلاَمٍ, وَلاَ تَفْعَلْ سُوءاً فِي أَعْيُنِ أَقْطَابِ الْفِلِسْطِينِيِّينَ».
\par 8 فَقَالَ دَاوُدُ لأَخِيشَ: «فَمَاذَا عَمِلْتُ, وَمَاذَا وَجَدْتَ فِي عَبْدِكَ مِنْ يَوْمِ صِرْتُ أَمَامَكَ إِلَى الْيَوْمِ حَتَّى لاَ آتِيَ وَأُحَارِبَ أَعْدَاءَ سَيِّدِي الْمَلِكِ؟»
\par 9 فَأَجَابَ أَخِيشُ: «عَلِمْتُ أَنَّكَ صَالِحٌ فِي عَيْنَيَّ كَمَلاَكِ اللَّهِ. إِلَّا إِنَّ رُؤَسَاءَ الْفِلِسْطِينِيِّينَ قَالُوا: «لاَ يَصْعَدْ مَعَنَا إِلَى الْحَرْبِ.
\par 10 وَالآنَ فَبَكِّرْ صَبَاحاً مَعَ عَبِيدِ سَيِّدِكَ الَّذِينَ جَاءُوا مَعَكَ. وَإِذَا بَكَّرْتُمْ صَبَاحاً وَأَضَاءَ لَكُمْ فَاذْهَبُوا».
\par 11 فَبَكَّرَ دَاوُدُ هُوَ وَرِجَالُهُ لِيَذْهَبُوا صَبَاحاً وَيَرْجِعُوا إِلَى أَرْضِ الْفِلِسْطِينِيِّينَ. وَأَمَّا الْفِلِسْطِينِيُّونَ فَصَعِدُوا إِلَى يَزْرَعِيلَ.

\chapter{30}

\par 1 وَلَمَّا جَاءَ دَاوُدُ وَرِجَالُهُ إِلَى صِقْلَغَ فِي الْيَوْمِ الثَّالِثِ, كَانَ الْعَمَالِقَةُ قَدْ غَزُوا الْجَنُوبَ وَصِقْلَغَ, وَضَرَبُوا صِقْلَغَ وَأَحْرَقُوهَا بِالنَّارِ,
\par 2 وَسَبُوا النِّسَاءَ اللَّوَاتِي فِيهَا. لَمْ يَقْتُلُوا أَحَداً لاَ صَغِيراً وَلاَ كَبِيراً, بَلْ سَاقُوهُمْ وَمَضُوا فِي طَرِيقِهِمْ.
\par 3 فَدَخَلَ دَاوُدُ وَرِجَالُهُ الْمَدِينَةَ وَإِذَا هِيَ مُحْرَقَةٌ بِالنَّارِ, وَنِسَاؤُهُمْ وَبَنُوهُمْ وَبَنَاتُهُمْ قَدْ سُبُوا.
\par 4 فَرَفَعَ دَاوُدُ وَالشَّعْبُ الَّذِينَ مَعَهُ أَصْوَاتَهُمْ وَبَكُوا حَتَّى لَمْ تَبْقَ لَهُمْ قُوَّةٌ لِلْبُكَاءِ.
\par 5 وَسُبِيَتِ امْرَأَتَا دَاوُدَ: أَخِينُوعَمُ الْيَزْرَعِيلِيَّةُ وَأَبِيجَايِلُ امْرَأَةُ نَابَالَ الْكَرْمَلِيِّ.
\par 6 فَتَضَايَقَ دَاوُدُ جِدّاً لأَنَّ الشَّعْبَ قَالُوا بِرَجْمِهِ, لأَنَّ أَنْفُسَ جَمِيعِ الشَّعْبِ كَانَتْ مُرَّةً كُلُّ وَاحِدٍ عَلَى بَنِيهِ وَبَنَاتِهِ. وَأَمَّا دَاوُدُ فَتَشَدَّدَ بِالرَّبِّ إِلَهِهِ.
\par 7 ثُمَّ قَالَ دَاوُدُ لأَبِيَاثَارَ الْكَاهِنِ ابْنِ أَخِيمَالِكَ: «قَدِّمْ إِلَيَّ الأَفُودَ». فَقَدَّمَ أَبِيَاثَارُ الأَفُودَ إِلَى دَاوُدَ.
\par 8 فَسَأَلَ دَاوُدُ مِنَ الرَّبِّ: «إِذَا لَحِقْتُ هَؤُلاَءِ الْغُزَاةَ فَهَلْ أُدْرِكُهُمْ؟» فَقَالَ لَهُ: «الْحَقْهُمْ فَإِنَّكَ تُدْرِكُ وَتُنْقِذُ».
\par 9 فَذَهَبَ دَاوُدُ هُوَ وَالسِّتُّ مِئَةِ الرَّجُلِ الَّذِينَ مَعَهُ وَجَاءُوا إِلَى وَادِي الْبَسُورِ, وَالْمُتَخَلِّفُونَ وَقَفُوا.
\par 10 وَأَمَّا دَاوُدُ فَلَحِقَ هُوَ وَأَرْبَعُ مِئَةِ رَجُلٍ, وَوَقَفَ مِئَتَا رَجُلٍ لأَنَّهُمْ أَعْيُوا عَنْ أَنْ يَعْبُرُوا وَادِيَ الْبَسُورِ.
\par 11 فَصَادَفُوا رَجُلاً مِصْرِيّاً فِي الْحَقْلِ فَأَخَذُوهُ إِلَى دَاوُدَ, وَأَعْطُوهُ خُبْزاً فَأَكَلَ وَسَقُوهُ مَاءً,
\par 12 وَأَعْطُوهُ قُرْصاً مِنَ التِّينِ وَعُنْقُودَيْنِ مِنَ الزَّبِيبِ, فَأَكَلَ وَرَجَعَتْ رُوحُهُ إِلَيْهِ, لأَنَّهُ لَمْ يَأْكُلْ خُبْزاً وَلاَ شَرِبَ مَاءً فِي ثَلاَثَةِ أَيَّامٍ وَثَلاَثِ لَيَالٍ.
\par 13 فَقَالَ لَهُ دَاوُدُ: «لِمَنْ أَنْتَ وَمِنْ أَيْنَ أَنْتَ؟» فَقَالَ: «أَنَا غُلاَمٌ مِصْرِيٌّ عَبْدٌ لِرَجُلٍ عَمَالِيقِيٍّ, وَقَدْ تَرَكَنِي سَيِّدِي لأَنِّي مَرِضْتُ مُنْذُ ثَلاَثَةِ أَيَّامٍ.
\par 14 فَإِنَّنَا قَدْ غَزَوْنَا عَلَى جَنُوبِيِّ الْكَرِيتِيِّينَ, وَعَلَى مَا لِيَهُوذَا وَعَلَى جَنُوبِيِّ كَالِبَ وَأَحْرَقْنَا صِقْلَغَ بِالنَّارِ».
\par 15 فَقَالَ لَهُ دَاوُدُ: «هَلْ تَنْزِلُ بِي إِلَى هَؤُلاَءِ الْغُزَاةِ؟» فَقَالَ: «احْلِفْ لِي بِاللَّهِ أَنَّكَ لاَ تَقْتُلُنِي وَلاَ تُسَلِّمُنِي لِيَدِ سَيِّدِي فَأَنْزِلَ بِكَ إِلَى هَؤُلاَءِ الْغُزَاةِ».
\par 16 فَنَزَلَ بِهِ وَإِذَا بِهِمْ مُنْتَشِرُونَ عَلَى وَجْهِ كُلِّ الأَرْضِ, يَأْكُلُونَ وَيَشْرَبُونَ وَيَرْقُصُونَ بِسَبَبِ جَمِيعِ الْغَنِيمَةِ الْعَظِيمَةِ الَّتِي أَخَذُوا مِنْ أَرْضِ الْفِلِسْطِينِيِّينَ وَمِنْ أَرْضِ يَهُوذَا.
\par 17 فَضَرَبَهُمْ دَاوُدُ مِنَ الْعَتَمَةِ إِلَى مَسَاءِ غَدِهِمْ, وَلَمْ يَنْجُ مِنْهُمْ رَجُلٌ إِلَّا أَرْبَعَ مِئَةِ غُلاَمٍ الَّذِينَ رَكِبُوا جِمَالاً وَهَرَبُوا.
\par 18 وَاسْتَخْلَصَ دَاوُدُ كُلَّ مَا أَخَذَهُ عَمَالِيقُ, وَأَنْقَذَ دَاوُدُ امْرَأَتَيْهِ.
\par 19 وَلَمْ يُفْقَدْ لَهُمْ شَيْءٌ لاَ صَغِيرٌ وَلاَ كَبِيرٌ وَلاَ بَنُونَ وَلاَ بَنَاتٌ وَلاَ غَنِيمَةٌ, وَلاَ شَيْءٌ مِنْ جَمِيعِ مَا أَخَذُوا لَهُمْ, بَلْ رَدَّ دَاوُدُ الْجَمِيعَ.
\par 20 وَأَخَذَ دَاوُدُ الْغَنَمَ وَالْبَقَرَ. سَاقُوهَا أَمَامَ تِلْكَ الْمَاشِيَةِ وَقَالُوا: «هَذِهِ غَنِيمَةُ دَاوُدَ».
\par 21 وَجَاءَ دَاوُدُ إِلَى مِئَتَيِ الرَّجُلِ الَّذِينَ أَعْيُوا عَنِ الذَّهَابِ وَرَاءَ دَاوُدَ, فَأَرْجَعُوهُمْ فِي وَادِي الْبَسُورِ, فَخَرَجُوا لِلِقَاءِ دَاوُدَ وَلِقَاءِ الشَّعْبِ الَّذِينَ مَعَهُ. فَتَقَدَّمَ دَاوُدُ إِلَى الْقَوْمِ وَسَأَلَ عَنْ سَلاَمَتِهِمْ.
\par 22 فَقَالَ كُلُّ رَجُلٍ شِرِّيرٍ وَلَئِيمٍ مِنَ الرِّجَالِ الَّذِينَ سَارُوا مَعَ دَاوُدَ: «لأَجْلِ أَنَّهُمْ لَمْ يَذْهَبُوا مَعَنَا لاَ نُعْطِيهِمْ مِنَ الْغَنِيمَةِ الَّتِي اسْتَخْلَصْنَاهَا, بَلْ لِكُلِّ رَجُلٍ امْرَأَتَهُ وَبَنِيهِ, فَلْيَقْتَادُوهُمْ وَيَنْطَلِقُوا».
\par 23 فَقَالَ دَاوُدُ: «لاَ تَفْعَلُوا هَكَذَا يَا إِخْوَتِي, لأَنَّ الرَّبَّ قَدْ أَعْطَانَا وَحَفِظَنَا وَدَفَعَ لِيَدِنَا الْغُزَاةَ الَّذِينَ جَاءُوا عَلَيْنَا.
\par 24 وَمَنْ يَسْمَعُ لَكُمْ فِي هَذَا الأَمْرِ؟ لأَنَّهُ كَنَصِيبِ النَّازِلِ إِلَى الْحَرْبِ نَصِيبُ الَّذِي يُقِيمُ عِنْدَ الأَمْتِعَةِ, فَإِنَّهُمْ يَقْتَسِمُونَ بِالسَّوِيَّةِ».
\par 25 وَكَانَ مِنْ ذَلِكَ الْيَوْمِ فَصَاعِداً أَنَّهُ جَعَلَهَا فَرِيضَةً وَقَضَاءً لإِسْرَائِيلَ إِلَى هَذَا الْيَوْمِ.
\par 26 وَلَمَّا جَاءَ دَاوُدُ إِلَى صِقْلَغَ أَرْسَلَ مِنَ الْغَنِيمَةِ إِلَى شُيُوخِ يَهُوذَا إِلَى أَصْحَابِهِ قَائِلاً: «هَذِهِ لَكُمْ بَرَكَةٌ مِنْ غَنِيمَةِ أَعْدَاءِ الرَّبِّ».
\par 27 إِلَى الَّذِينَ فِي بَيْتِ إِيلٍ, وَالَّذِينَ فِي رَامُوتَ الْجَنُوبِ, وَالَّذِينَ فِي يَتِّيرَ,
\par 28 وَإِلَى الَّذِينَ فِي عَرُوعِيرَ, وَالَّذِينَ فِي سِفْمُوثَ, وَالَّذِينَ فِي أَشْتِمُوعَ,
\par 29 وَإِلَى الَّذِينَ فِي رَاخَالَ وَالَّذِينَ فِي مُدُنِ الْيَرْحَمْئِيلِيِّينَ وَالَّذِينَ فِي مُدُنِ الْقِينِيِّينَ
\par 30 وَإِلَى الَّذِينَ فِي حُرْمَةَ وَالَّذِينَ فِي كُورَِ عَاشَانَ وَالَّذِينَ فِي عَتَاكَ
\par 31 وَإِلَى الَّذِينَ فِي حَبْرُونَ وَإِلَى جَمِيعِ الأَمَاكِنِ الَّتِي تَرَدَّدَ فِيهَا دَاوُدُ وَرِجَالُهُ.

\chapter{31}

\par 1 وَحَارَبَ الْفِلِسْطِينِيُّونَ إِسْرَائِيلَ, فَهَرَبَ رِجَالُ إِسْرَائِيلَ مِنْ أَمَامِ الْفِلِسْطِينِيِّينَ وَسَقَطُوا قَتْلَى فِي جَبَلِ جِلْبُوعَ.
\par 2 فَشَدَّ الْفِلِسْطِينِيُّونَ وَرَاءَ شَاوُلَ وَبَنِيهِ, وَضَرَبَ الْفِلِسْطِينِيُّونَ يُونَاثَانَ وَأَبِينَادَابَ وَمَلْكِيشُوعَ أَبْنَاءَ شَاوُلَ.
\par 3 وَاشْتَدَّتِ الْحَرْبُ عَلَى شَاوُلَ فَأَصَابَهُ الرُّمَاةُ رِجَالُ الْقِسِيِّ, فَانْجَرَحَ جِدّاً مِنَ الرُّمَاةِ.
\par 4 فَقَالَ شَاوُلُ لِحَامِلِ سِلاَحِهِ: «اسْتَلَّ سَيْفَكَ وَاطْعَنِّي بِهِ لِئَلَّا يَأْتِيَ هَؤُلاَءِ الْغُلْفُ وَيَطْعَنُونِي وَيُقَبِّحُونِي». فَلَمْ يَشَأْ حَامِلُ سِلاَحِهِ لأَنَّهُ خَافَ جِدّاً. فَأَخَذَ شَاوُلُ السَّيْفَ وَسَقَطَ عَلَيْهِ.
\par 5 وَلَمَّا رَأَى حَامِلُ سِلاَحِهِ أَنَّهُ قَدْ مَاتَ شَاوُلُ, سَقَطَ هُوَ أَيْضاً عَلَى سَيْفِهِ وَمَاتَ مَعَهُ.
\par 6 فَمَاتَ شَاوُلُ وَبَنُوهُ الثَّلاَثَةُ وَحَامِلُ سِلاَحِهِ وَجَمِيعُ رِجَالِهِ فِي ذَلِكَ الْيَوْمِ مَعاً.
\par 7 وَلَمَّا رَأَى رِجَالُ إِسْرَائِيلَ الَّذِينَ فِي عَبْرِ الْوَادِي وَالَّذِينَ فِي عَبْرِ الأُرْدُنِّ أَنَّ رِجَالَ إِسْرَائِيلَ قَدْ هَرَبُوا, وَأَنَّ شَاوُلَ وَبَنِيهِ قَدْ مَاتُوا, تَرَكُوا الْمُدُنَ وَهَرَبُوا, فَأَتَى الْفِلِسْطِينِيُّونَ وَسَكَنُوا بِهَا.
\par 8 وَفِي الْغَدِ لَمَّا جَاءَ الْفِلِسْطِينِيُّونَ لِيُعَرُّوا الْقَتْلَى وَجَدُوا شَاوُلَ وَبَنِيهِ الثَّلاَثَةَ سَاقِطِينَ فِي جَبَلِ جِلْبُوعَ,
\par 9 فَقَطَعُوا رَأْسَهُ وَنَزَعُوا سِلاَحَهُ وَأَرْسَلُوا إِلَى أَرْضِ الْفِلِسْطِينِيِّينَ فِي كُلِّ جِهَةٍ لأَجْلِ التَّبْشِيرِ فِي بَيْتِ أَصْنَامِهِمْ وَفِي الشَّعْبِ.
\par 10 وَوَضَعُوا سِلاَحَهُ فِي بَيْتَِ عَشْتَارُوثَ, وَسَمَّرُوا جَسَدَهُ عَلَى سُورِ بَيْتِ شَانَ.
\par 11 وَلَمَّا سَمِعَ سُكَّانُ يَابِيشَ جِلْعَادَ بِمَا فَعَلَ الْفِلِسْطِينِيُّونَ بِشَاوُلَ,
\par 12 قَامَ كُلُّ ذِي بَأْسٍ وَسَارُوا اللَّيْلَ كُلَّهُ, وَأَخَذُوا جَسَدَ شَاوُلَ وَأَجْسَادَ بَنِيهِ عَنْ سُورِ بَيْتِ شَانَ, وَجَاءُوا بِهَا إِلَى يَابِيشَ وَأَحْرَقُوهَا هُنَاكَ
\par 13 وَأَخَذُوا عِظَامَهُمْ وَدَفَنُوهَا تَحْتَ الأَثْلَةِ فِي يَابِيشَ, وَصَامُوا سَبْعَةَ أَيَّامٍ.

\end{document}