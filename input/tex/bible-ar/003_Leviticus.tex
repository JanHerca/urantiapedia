\begin{document}

\title{لاويين}


\chapter{1}

\par 1 وَدَعَا الرَّبُّ مُوسَى وَكَلَّمَهُ مِنْ خَيْمَةِ الاجْتِمَاعِ قَائِلا:
\par 2 «قُلْ لِبَنِي اسْرَائِيلَ: اذَا قَرَّبَ انْسَانٌ مِنْكُمْ قُرْبَانا لِلرَّبِّ مِنَ الْبَهَائِمِ فَمِنَ الْبَقَرِ وَالْغَنَمِ تُقَرِّبُونَ قَرَابِينَكُمْ.
\par 3 انْ كَانَ قُرْبَانُهُ مُحْرَقَةً مِنَ الْبَقَرِ فَذَكَرا صَحِيحا يُقَرِّبْهُ. الَى بَابِ خَيْمَةِ الاجْتِمَاعِ يُقَدِّمُهُ لِلرِّضَا عَنْهُ امَامَ الرَّبِّ.
\par 4 وَيَضَعُ يَدَهُ عَلَى رَاسِ الْمُحْرَقَةِ فَيُرْضَى عَلَيْهِ لِلتَّكْفِيرِ عَنْهُ.
\par 5 وَيَذْبَحُ الْعِجْلَ امَامَ الرَّبِّ وَيُقَرِّبُ الْكَهَنَةُ بَنُو هَارُونَ الدَّمَ وَيَرُشُّونَهُ مُسْتَدِيرا عَلَى الْمَذْبَحِ الَّذِي لَدَى بَابِ خَيْمَةِ الاجْتِمَاعِ.
\par 6 وَيَسْلَُخُ الْمُحْرَقَةَ وَيُقَطِّعُهَا الَى قِطَعِهَا.
\par 7 وَيَجْعَلُ بَنُو هَارُونَ الْكَاهِنِ نَارا عَلَى الْمَذْبَحِ وَيُرَتِّبُونَ حَطَبا عَلَى النَّارِ.
\par 8 وَيُرَتِّبُ بَنُو هَارُونَ الْكَهَنَةُ الْقِطَعَ مَعَ الرَّاسِ وَالشَّحْمِ فَوْقَ الْحَطَبِ الَّذِي عَلَى النَّارِ الَّتِي عَلَى الْمَذْبَحِ.
\par 9 وَامَّا احْشَاؤُهُ وَاكَارِعُهُ فَيَغْسِلُهَا بِمَاءٍ وَيُوقِدُ الْكَاهِنُ الْجَمِيعَ عَلَى الْمَذْبَحِ مُحْرَقَةً وَقُودَ رَائِحَةِ سُرُورٍ لِلرَّبِّ.
\par 10 «وَانْ كَانَ قُرْبَانُهُ مِنَ الْغَنَمِ (الضَّانِ اوِ الْمَعْزِ) مُحْرَقَةً فَذَكَرا صَحِيحا يُقَرِّبُهُ.
\par 11 وَيَذْبَحُهُ عَلَى جَانِبِ الْمَذْبَحِ الَى الشِّمَالِ امَامَ الرَّبِّ. وَيَرُشُّ بَنُو هَارُونَ الْكَهَنَةُ دَمَهُ عَلَى الْمَذْبَحِ مُسْتَدِيرا.
\par 12 وَيُقَطِّعُهُ الَى قِطَعِهِ مَعَ رَاسِهِ وَشَحْمِهِ. وَيُرَتِّبُهُنَّ الْكَاهِنُ فَوْقَ الْحَطَبِ الَّذِي عَلَى النَّارِ الَّتِي عَلَى الْمَذْبَحِ.
\par 13 وَامَّا الاحْشَاءُ وَالاكَارِعُ فَيَغْسِلُهَا بِمَاءٍ وَيُقَرِّبُ الْكَاهِنُ الْجَمِيعَ وَيُوقِدُ عَلَى الْمَذْبَحِ. انَّهُ مُحْرَقَةٌ وَقُودُ رَائِحَةِ سُرُورٍ لِلرَّبِّ.
\par 14 «وَانْ كَانَ قُرْبَانُهُ لِلرَّبِّ مِنَ الطَّيْرِ مُحْرَقَةً يُقَرِّبُ قُرْبَانَهُ مِنَ الْيَمَامِ اوْ مِنْ افْرَاخِ الْحَمَامِ.
\par 15 يُقَدِّمُهُ الْكَاهِنُ الَى الْمَذْبَحِ وَيَحُزُّ رَاسَهُ وَيُوقِدُ عَلَى الْمَذْبَحِ وَيُعْصَرُ دَمُهُ عَلَى حَائِطِ الْمَذْبَحِ.
\par 16 وَيَنْزِعُ حَوْصَلَتَهُ بِفَرْثِهَا وَيَطْرَحُهَا الَى جَانِبِ الْمَذْبَحِ شَرْقا الَى مَكَانِ الرَّمَادِ.
\par 17 وَيَشُقُّهُ بَيْنَ جَنَاحَيْهِ. لا يَفْصِلُهُ. وَيُوقِدُهُ الْكَاهِنُ عَلَى الْمَذْبَحِ فَوْقَ الْحَطَبِ الَّذِي عَلَى النَّارِ. انَّهُ مُحْرَقَةٌ وَقُودُ رَائِحَةِ سُرُورٍ لِلرَّبِّ.

\chapter{2}

\par 1 «وَاذَا قَرَّبَ احَدٌ قُرْبَانَ تَقْدِمَةٍ لِلرَّبِّ يَكُونُ قُرْبَانُهُ مِنْ دَقِيقٍ. وَيَسْكُبُ عَلَيْهَا زَيْتا وَيَجْعَلُ عَلَيْهَا لُبَانا.
\par 2 وَيَاتِي بِهَا الَى بَنِي هَارُونَ الْكَهَنَةِ وَيَقْبِضُ مِنْهَا مِلْءَ قَبْضَتِهِ مِنْ دَقِيقِهَا وَزَيْتِهَا مَعَ كُلِّ لُبَانِهَا. وَيُوقِدُ الْكَاهِنُ تِذْكَارَهَا عَلَى الْمَذْبَحِ وَقُودَ رَائِحَةِ سُرُورٍ لِلرَّبِّ.
\par 3 وَالْبَاقِي مِنَ التَّقْدِمَةِ هُوَ لِهَارُونَ وَبَنِيهِ. قُدْسُ اقْدَاسٍ مِنْ وَقَائِدِ الرَّبِّ.
\par 4 «وَاذَا قَرَّبْتَ قُرْبَانَ تَقْدِمَةٍ مَخْبُوزَةٍ فِي تَنُّورٍ تَكُونُ اقْرَاصا مِنْ دَقِيقٍ فَطِيرا مَلْتُوتَةً بِزَيْتٍ وَرِقَاقا فَطِيرا مَدْهُونَةً بِزَيْتٍ.
\par 5 وَانْ كَانَ قُرْبَانُكَ تَقْدِمَةً عَلَى الصَّاجِ تَكُونُ مِنْ دَقِيقٍ مَلْتُوتَةً بِزَيْتٍ فَطِيرا.
\par 6 تَفُتُّهَا فُتَاتا وَتَسْكُبُ عَلَيْهَا زَيْتا. انَّهَا تَقْدِمَةٌ.
\par 7 «وَانْ كَانَ قُرْبَانُكَ تَقْدِمَةً مِنْ طَاجِنٍ فَمِنْ دَقِيقٍ بِزَيْتٍ تَعْمَلُهُ.
\par 8 فَتَاتِي بِالتَّقْدِمَةِ الَّتِي تُصْطَنَعُ مِنْ هَذِهِ الَى الرَّبِّ وَتُقَدِّمُهَا الَى الْكَاهِنِ فَيَدْنُو بِهَا الَى الْمَذْبَحِ.
\par 9 وَيَاخُذُ الْكَاهِنُ مِنَ التَّقْدِمَةِ تِذْكَارَهَا وَيُوقِدُ عَلَى الْمَذْبَحِ وَقُودَ رَائِحَةِ سَرُورٍ لِلرَّبِّ.
\par 10 وَالْبَاقِي مِنَ التَّقْدِمَةِ هُوَ لِهَارُونَ وَبَنِيهِ قُدْسُ اقْدَاسٍ مِنْ وَقَائِدِ الرَّبِّ.
\par 11 «كُلُّ التَّقْدِمَاتِ الَّتِي تُقَرِّبُونَهَا لِلرَّبِّ لا تُصْطَنَعُ خَمِيرا لانَّ كُلَّ خَمِيرٍ وَكُلَّ عَسَلٍ لا تُوقِدُوا مِنْهُمَا وَقُودا لِلرَّبِّ.
\par 12 قُرْبَانَ اوَائِلَ تُقَرِّبُونَهُمَا لِلرَّبِّ. لَكِنْ عَلَى الْمَذْبَحِ لا يَصْعَدَانِ لِرَائِحَةِ سُرُورٍ.
\par 13 وَكُلُّ قُرْبَانٍ مِنْ تَقَادِمِكَ بِالْمِلْحِ تُمَلِّحُهُ وَلا تُخْلِ تَقْدِمَتَكَ مِنْ مِلْحِ عَهْدِ الَهِكَ. عَلَى جَمِيعِ قَرَابِينِكَ تُقَرِّبُ مِلْحا.
\par 14 «وَانْ قَرَّبْتَ تَقْدِمَةَ بَاكُورَاتٍ لِلرَّبِّ فَفَرِيكا مَشْوِيّا بِالنَّارِ. جَرِيشا سَوِيقا تُقَرِّبُ تَقْدِمَةَ بَاكُورَاتِكَ.
\par 15 وَتَجْعَلُ عَلَيْهَا زَيْتا وَتَضَعُ عَلَيْهَا لُبَانا. انَّهَا تَقْدِمَةٌ.
\par 16 فَيُوقِدُ الْكَاهِنُ تِذْكَارَهَا مِنْ جَرِيشِهَا وَزَيْتِهَا مَعَ جَمِيعِ لُبَانِهَا وَقُودا لِلرَّبِّ.

\chapter{3}

\par 1 «وَانْ كَانَ قُرْبَانُهُ ذَبِيحَةَ سَلامَةٍ فَانْ قَرَّبَ مِنَ الْبَقَرِ ذَكَرا اوْ انْثَى فَصَحِيحا يُقَرِّبُهُ امَامَ الرَّبِّ.
\par 2 يَضَعُ يَدَهُ عَلَى رَاسِ قُرْبَانِهِ وَيَذْبَحُهُ لَدَى بَابِ خَيْمَةِ الاجْتِمَاعِ. وَيَرُشُّ بَنُو هَارُونَ الْكَهَنَةُ الدَّمَ عَلَى الْمَذْبَحِ مُسْتَدِيرا.
\par 3 وَيُقَرِّبُ مِنْ ذَبِيحَةِ السَّلامَةِ وَقُودا لِلرَّبِّ: الشَّحْمَ الَّذِي يُغَشِّي الاحْشَاءَ وَسَائِرَ الشَّحْمِ الَّذِي عَلَى الاحْشَاءِ
\par 4 وَالْكُلْيَتَيْنِ وَالشَّحْمَ الَّذِي عَلَيْهِمَا الَّذِي عَلَى الْخَاصِرَتَيْنِ وَزِيَادَةَ الْكَبِدِ مَعَ الْكُلْيَتَيْنِ يَنْزِعُهَا.
\par 5 وَيُوقِدُهَا بَنُو هَارُونَ عَلَى الْمَذْبَحِ عَلَى الْمُحْرَقَةِ الَّتِي فَوْقَ الْحَطَبِ الَّذِي عَلَى النَّارِ وَقُودَ رَائِحَةِ سَرُورٍ لِلرَّبِّ.
\par 6 «وَانْ كَانَ قُرْبَانُهُ مِنَ الْغَنَمِ ذَبِيحَةَ سَلامَةٍ لِلرَّبِّ ذَكَرا اوْ انْثَى فَصَحِيحا يُقَرِّبُهُ.
\par 7 انْ قَرَّبَ قُرْبَانَهُ مِنَ الضَّانِ يُقَدِّمُهُ امَامَ الرَّبِّ.
\par 8 يَضَعُ يَدَهُ عَلَى رَاسِ قُرْبَانِهِ وَيَذْبَحُهُ قُدَّامَ خَيْمَةِ الاجْتِمَاعِ. وَيَرُشُّ بَنُو هَارُونَ دَمَهُ عَلَى الْمَذْبَحِ مُسْتَدِيرا.
\par 9 وَيُقَرِّبُ مِنْ ذَبِيحَةِ السَّلامَةِ شَحْمَهَا وَقُودا لِلرَّبِّ: الالْيَةَ صَحِيحَةً مِنْ عِنْدِ الْعُصْعُصِ يَنْزِعُهَا وَالشَّحْمَ الَّذِي يُغَشِّي الاحْشَاءَ وَسَائِرَ الشَّحْمِ الَّذِي عَلَى الاحْشَاءِ
\par 10 وَالْكُلْيَتَيْنِ وَالشَّحْمَ الَّذِي عَلَيْهِمَا الَّذِي عَلَى الْخَاصِرَتَيْنِ وَزِيَادَةَ الْكَبِدِ مَعَ الْكُلْيَتَيْنِ يَنْزِعُهَا.
\par 11 وَيُوقِدُهَا الْكَاهِنُ عَلَى الْمَذْبَحِ طَعَامَ وَقُودٍ لِلرَّبِّ.
\par 12 «وَانْ كَانَ قُرْبَانُهُ مِنَ الْمَعْزِ يُقَدِّمُهُ امَامَ الرَّبِّ.
\par 13 يَضَعُ يَدَهُ عَلَى رَاسِهِ وَيَذْبَحُهُ قُدَّامَ خَيْمَةِ الاجْتِمَاعِ. وَيَرُشُّ بَنُو هَارُونَ دَمَهُ عَلَى الْمَذْبَحِ مُسْتَدِيرا.
\par 14 وَيُقَرِّبُ مِنْهُ قُرْبَانَهُ وَقُودا لِلرَّبِّ: الشَّحْمَ الَّذِي يُغَشِّي الاحْشَاءَ وَسَائِرَ الشَّحْمِ الَّذِي عَلَى الاحْشَاءِ
\par 15 وَالْكُلْيَتَيْنِ وَالشَّحْمَ الَّذِي عَلَيْهِمَا الَّذِي عَلَى الْخَاصِرَتَيْنِ وَزِيَادَةَ الْكَبِدِ مَعَ الْكُلْيَتَيْنِ يَنْزِعُهَا.
\par 16 وَيُوقِدُهُنَّ الْكَاهِنُ عَلَى الْمَذْبَحِ طَعَامَ وَقُودٍ لِرَائِحَةِ سُرُورٍ. كُلُّ الشَّحْمِ لِلرَّبِّ.
\par 17 فَرِيضَةً دَهْرِيَّةً فِي اجْيَالِكُمْ فِي جَمِيعِ مَسَاكِنِكُمْ: لا تَاكُلُوا شَيْئا مِنَ الشَّحْمِ وَلا مِنَ الدَّمِ».

\chapter{4}

\par 1 وَقَالَ الرَّبُّ لِمُوسَى:
\par 2 «قُلْ لِبَنِي اسْرَائِيلَ: اذَا اخْطَاتْ نَفْسٌ سَهْوا فِي شَيْءٍ مِنْ جَمِيعِ مَنَاهِي الرَّبِّ الَّتِي لا يَنْبَغِي عَمَلُهَا وَعَمِلَتْ وَاحِدَةً مِنْهَا -
\par 3 انْ كَانَ الْكَاهِنُ الْمَمْسُوحُ يُخْطِئُ لاثْمِ الشَّعْبِ يُقَرِّبُ عَنْ خَطِيَّتِهِ الَّتِي اخْطَا ثَوْرا ابْنَ بَقَرٍ صَحِيحا لِلرَّبِّ ذَبِيحَةَ خَطِيَّةٍ.
\par 4 يُقَدِّمُ الثَّوْرَ الَى بَابِ خَيْمَةِ الاجْتِمَاعِ امَامَ الرَّبِّ وَيَضَعُ يَدَهُ عَلَى رَاسِ الثَّوْرِ وَيَذْبَحُ الثَّوْرَ امَامَ الرَّبِّ.
\par 5 وَيَاخُذُ الْكَاهِنُ الْمَمْسُوحُ مِنْ دَمِ الثَّوْرِ وَيَدْخُلُ بِهِ الَى خَيْمَةِ الاجْتِمَاعِ
\par 6 وَيَغْمِسُ الْكَاهِنُ اصْبِعَهُ فِي الدَّمِ وَيَنْضِحُ مِنَ الدَّمِ سَبْعَ مَرَّاتٍ امَامَ الرَّبِّ لَدَى حِجَابِ الْقُدْسِ.
\par 7 وَيَجْعَلُ الْكَاهِنُ مِنَ الدَّمِ عَلَى قُرُونِ مَذْبَحِ الْبَخُورِ الْعَطِرِ الَّذِي فِي خَيْمَةِ الاجْتِمَاعِ امَامَ الرَّبِّ. وَسَائِرُ دَمِ الثَّوْرِ يَصُبُّهُ الَى اسْفَلِ مَذْبَحِ الْمُحْرَقَةِ الَّذِي لَدَى بَابِ خَيْمَةِ الاجْتِمَاعِ.
\par 8 وَجَمِيعُ شَحْمِ ثَوْرِ الْخَطِيَّةِ يَنْزِعُهُ عَنْهُ. الشَّحْمَ الَّذِي يُغَشِّي الاحْشَاءَ وَسَائِرَ الشَّحْمِ الَّذِي عَلَى الاحْشَاءِ
\par 9 وَالْكُلْيَتَيْنِ وَالشَّحْمَ الَّذِي عَلَيْهِمَا الَّذِي عَلَى الْخَاصِرَتَيْنِ وَزِيَادَةَ الْكَبِدِ مَعَ الْكُلْيَتَيْنِ يَنْزِعُهَا
\par 10 كَمَا تُنْزَعُ مِنْ ثَوْرِ ذَبِيحَةِ السَّلامَةِ. وَيُوقِدُهُنَّ الْكَاهِنُ عَلَى مَذْبَحِ الْمُحْرَقَةِ.
\par 11 وَامَّا جِلْدُ الثَّوْرِ وَكُلُّ لَحْمِهِ مَعَ رَاسِهِ وَاكَارِعِهِ وَاحْشَائِهِ وَفَرْثِهِ
\par 12 فَيُخْرِجُ سَائِرَ الثَّوْرِ الَى خَارِجِ الْمَحَلَّةِ الَى مَكَانٍ طَاهِرٍ الَى مَرْمَى الرَّمَادِ وَيُحْرِقُهَا عَلَى حَطَبٍ بِالنَّارِ. عَلَى مَرْمَى الرَّمَادِ تُحْرَقُ.
\par 13 «وَانْ سَهَا كُلُّ جَمَاعَةِ اسْرَائِيلَ وَاخْفِيَ امْرٌ عَنْ اعْيُنِ الْمَجْمَعِ وَعَمِلُوا وَاحِدَةً مِنْ جَمِيعِ مَنَاهِي الرَّبِّ الَّتِي لا يَنْبَغِي عَمَلُهَا وَاثِمُوا
\par 14 ثُمَّ عُرِفَتِ الْخَطِيَّةُ الَّتِي اخْطَاوا بِهَا يُقَرِّبُ الْمَجْمَعُ ثَوْرا ابْنَ بَقَرٍ ذَبِيحَةَ خَطِيَّةٍ. يَاتُونَ بِهِ الَى قُدَّامِ خَيْمَةِ الاجْتِمَاعِ
\par 15 وَيَضَعُ شُيُوخُ الْجَمَاعَةِ ايْدِيَهُمْ عَلَى رَاسِ الثَّوْرِ امَامَ الرَّبِّ وَيَذْبَحُوا الثَّوْرَ امَامَ الرَّبِّ.
\par 16 وَيُدْخِلُ الْكَاهِنُ الْمَمْسُوحُ مِنْ دَمِ الثَّوْرِ الَى خَيْمَةِ الاجْتِمَاعِ.
\par 17 وَيَغْمِسُ الْكَاهِنُ اصْبِعَهُ فِي الدَّمِ وَيَنْضِحُ سَبْعَ مَرَّاتٍ امَامَ الرَّبِّ لَدَى الْحِجَابِ.
\par 18 وَيَجْعَلُ مِنَ الدَّمِ عَلَى قُرُونِ الْمَذْبَحِ الَّذِي امَامَ الرَّبِّ فِي خَيْمَةِ الاجْتِمَاعِ. وَسَائِرَ الدَّمِ يَصُبُّهُ الَى اسْفَلِ مَذْبَحِ الْمُحْرَقَةِ الَّذِي لَدَى بَابِ خَيْمَةِ الاجْتِمَاعِ.
\par 19 وَجَمِيعَ شَحْمِهِ يَنْزِعُهُ عَنْهُ وَيُوقِدُهُ عَلَى الْمَذْبَحِ.
\par 20 وَيَفْعَلُ بِالثَّوْرِ كَمَا فَعَلَ بِثَوْرِ الْخَطِيَّةِ. كَذَلِكَ يَفْعَلُ بِهِ. وَيُكَفِّرُ عَنْهُمُ الْكَاهِنُ فَيُصْفَحُ عَنْهُمْ.
\par 21 ثُمَّ يُخْرِجُ الثَّوْرَ الَى خَارِجِ الْمَحَلَّةِ وَيُحْرِقُهُ كَمَا احْرَقَ الثَّوْرَ الاوَّلَ. انَّهُ ذَبِيحَةُ خَطِيَّةِ الْمَجْمَعِ.
\par 22 «اذَا اخْطَا رَئِيسٌ وَعَمِلَ بِسَهْوٍ وَاحِدَةً مِنْ جَمِيعِ مَنَاهِي الرَّبِّ الَهِهِ الَّتِي لا يَنْبَغِي عَمَلُهَا وَاثِمَ
\par 23 ثُمَّ اعْلِمَ بِخَطِيَّتِهِ الَّتِي اخْطَا بِهَا يَاتِي بِقُرْبَانِهِ تَيْسا مِنَ الْمَعْزِ ذَكَرا صَحِيحا.
\par 24 وَيَضَعُ يَدَهُ عَلَى رَاسِ التَّيْسِ وَيَذْبَحُهُ فِي الْمَوْضِعِ الَّذِي يَذْبَحُ فِيهِ الْمُحْرَقَةَ امَامَ الرَّبِّ. انَّهُ ذَبِيحَةُ خَطِيَّةٍ.
\par 25 وَيَاخُذُ الْكَاهِنُ مِنْ دَمِ ذَبِيحَةِ الْخَطِيَّةِ بِاصْبِعِهِ وَيَجْعَلُ عَلَى قُرُونِ مَذْبَحِ الْمُحْرَقَةِ ثُمَّ يَصُبُّ دَمَهُ الَى اسْفَلِ مَذْبَحِ الْمُحْرَقَةِ.
\par 26 وَجَمِيعَ شَحْمِهِ يُوقِدُهُ عَلَى الْمَذْبَحِ كَشَحْمِ ذَبِيحَةِ السَّلامَةِ وَيُكَفِّرُ الْكَاهِنُ عَنْهُ مِنْ خَطِيَّتِهِ فَيُصْفَحُ عَنْهُ.
\par 27 «وَانْ اخْطَا احَدٌ مِنْ عَامَّةِ الارْضِ سَهْوا بِعَمَلِهِ وَاحِدَةً مِنْ مَنَاهِي الرَّبِّ الَّتِي لا يَنْبَغِي عَمَلُهَا وَاثِمَ
\par 28 ثُمَّ اعْلِمَ بِخَطِيَّتِهِ الَّتِي اخْطَا بِهَا يَاتِي بِقُرْبَانِهِ عَنْزا مِنَ الْمَعْزِ انْثَى صَحِيحَةً عَنْ خَطِيَّتِهِ الَّتِي اخْطَا.
\par 29 وَيَضَعُ يَدَهُ عَلَى رَاسِ ذَبِيحَةِ الْخَطِيَّةِ وَيَذْبَحُ ذَبِيحَةَ الْخَطِيَّةِ فِي مَوْضِعِ الْمُحْرَقَةِ.
\par 30 وَيَاخُذُ الْكَاهِنُ مِنْ دَمِهَا بِاصْبَعِهِ وَيَجْعَلُ عَلَى قُرُونِ مَذْبَحِ الْمُحْرَقَةِ وَيَصُبُّ سَائِرَ دَمِهَا الَى اسْفَلِ الْمَذْبَحِ.
\par 31 وَجَمِيعَ شَحْمِهَا يَنْزِعُهُ كَمَا نُزِعَ الشَّحْمُ عَنْ ذَبِيحَةِ السَّلامَةِ وَيُوقِدُ الْكَاهِنُ عَلَى الْمَذْبَحِ رَائِحَةَ سُرُورٍ لِلرَّبِّ وَيُكَفِّرُ عَنْهُ الْكَاهِنُ فَيُصْفَحُ عَنْهُ.
\par 32 «وَانْ اتَى بِقُرْبَانِهِ مِنَ الضَّانِ ذَبِيحَةَ خَطِيَّةٍ يَاتِي بِهَا انْثَى صَحِيحَةً.
\par 33 وَيَضَعُ يَدَهُ عَلَى رَاسِ ذَبِيحَةِ الْخَطِيَّةِ وَيَذْبَحُهَا ذَبِيحَةَ خَطِيَّةٍ فِي الْمَوْضِعِ الَّذِي يَذْبَحُ فِيهِ الْمُحْرَقَةَ.
\par 34 وَيَاخُذُ الْكَاهِنُ مِنْ دَمِ ذَبِيحَةِ الْخَطِيَّةِ بِاصْبَعِهِ وَيَجْعَلُ عَلَى قُرُونِ مَذْبَحِ الْمُحْرَقَةِ وَيَصُبُّ سَائِرَ الدَّمِ الَى اسْفَلِ الْمَذْبَحِ.
\par 35 وَجَمِيعَ شَحْمِهِ يَنْزِعُهُ كَمَا يُنْزَعُ شَحْمُ الضَّانِ عَنْ ذَبِيحَةِ السَّلامَةِ وَيُوقِدُهُ الْكَاهِنُ عَلَى الْمَذْبَحِ عَلَى وَقَائِدِ الرَّبِّ. وَيُكَفِّرُ عَنْهُ الْكَاهِنُ مِنْ خَطِيَّتِهِ الَّتِي اخْطَا فَيُصْفَحُ عَنْهُ.

\chapter{5}

\par 1 «وَاذَا اخْطَا احَدٌ وَسَمِعَ صَوْتَ حَلْفٍ وَهُوَ شَاهِدٌ يُبْصِرُ اوْ يَعْرِفُ فَانْ لَمْ يُخْبِرْ بِهِ حَمَلَ ذَنْبَهُ.
\par 2 اوْ اذَا مَسَّ احَدٌ شَيْئا نَجِسا: جُثَّةَ وَحْشٍ نَجِسٍ اوْ جُثَّةَ بَهِيمَةٍ نَجِسَةٍ اوْ جُثَّةَ دَبِيبٍ نَجِسٍ وَاخْفِيَ عَنْهُ فَهُوَ نَجِسٌ وَمُذْنِبٌ.
\par 3 اوْ اذَا مَسَّ نَجَاسَةَ انْسَانٍ مِنْ جَمِيعِ نَجَاسَاتِهِ الَّتِي يَتَنَجَّسُ بِهَا وَاخْفِيَ عَنْهُ ثُمَّ عُلِمَ فَهُوَ مُذْنِبٌ.
\par 4 اوْ اذَا حَلَفَ احَدٌ مُفْتَرِطا بِشَفَتَيْهِ لِلاسَاءَةِ اوْ لِلاحْسَانِ مِنْ جَمِيعِ مَا يَفْتَرِطُ بِهِ الْانْسَانُ فِي الْيَمِينِ وَاخْفِيَ عَنْهُ ثُمَّ عُلِمَ فَهُوَ مُذْنِبٌ فِي شَيْءٍ مِنْ ذَلِكَ.
\par 5 فَانْ كَانَ يُذْنِبُ فِي شَيْءٍ مِنْ هَذِهِ يُقِرُّ بِمَا قَدْ اخْطَا بِهِ.
\par 6 وَيَاتِي الَى الرَّبِّ بِذَبِيحَةٍ لاثْمِهِ عَنْ خَطِيَّتِهِ الَّتِي اخْطَا بِهَا: انْثَى مِنَ الاغْنَامِ نَعْجَةً اوْ عَنْزا مِنَ الْمَعْزِ ذَبِيحَةَ خَطِيَّةٍ فَيُكَفِّرُ عَنْهُ الْكَاهِنُ مِنْ خَطِيَّتِهِ.
\par 7 وَانْ لَمْ تَنَلْ يَدُهُ كِفَايَةً لِشَاةٍ فَيَاتِي بِذَبِيحَةٍ لاثْمِهِ الَّذِي اخْطَا بِهِ يَمَامَتَيْنِ اوْ فَرْخَيْ حَمَامٍ الَى الرَّبِّ احَدُهُمَا ذَبِيحَةُ خَطِيَّةٍ وَالْاخَرُ مُحْرَقَةٌ.
\par 8 يَاتِي بِهِمَا الَى الْكَاهِنِ فَيُقَرِّبُ الَّذِي لِلْخَطِيَّةِ اوَّلا. يَحُزُّ رَاسَهُ مِنْ قَفَاهُ وَلا يَفْصِلُهُ.
\par 9 وَيَنْضِحُ مِنْ دَمِ ذَبِيحَةِ الْخَطِيَّةِ عَلَى حَائِطِ الْمَذْبَحِ. وَالْبَاقِي مِنَ الدَّمِ يُعْصَرُ الَى اسْفَلِ الْمَذْبَحِ. انَّهُ ذَبِيحَةُ خَطِيَّةٍ.
\par 10 وَامَّا الثَّانِي فَيَعْمَلُهُ مُحْرَقَةً كَالْعَادَةِ فَيُكَفِّرُ عَنْهُ الْكَاهِنُ مِنْ خَطِيَّتِهِ الَّتِي اخْطَا فَيُصْفَحُ عَنْهُ.
\par 11 وَانْ لَمْ تَنَلْ يَدُهُ يَمَامَتَيْنِ اوْ فَرْخَيْ حَمَامٍ فَيَاتِي بِقُرْبَانِهِ عَمَّا اخْطَا بِهِ عُشْرَ الْايفَةِ مِنْ دَقِيقٍ قُرْبَانَ خَطِيَّةٍ. لا يَضَعُ عَلَيْهِ زَيْتا وَلا يَجْعَلُ عَلَيْهِ لُبَانا لانَّهُ قُرْبَانُ خَطِيَّةٍ.
\par 12 يَاتِي بِهِ الَى الْكَاهِنِ فَيَقْبِضُ الْكَاهِنُ مِنْهُ مِلْءَ قَبْضَتِهِ تِذْكَارَهُ وَيُوقِدُهُ عَلَى الْمَذْبَحِ عَلَى وَقَائِدِ الرَّبِّ. انَّهُ قُرْبَانُ خَطِيَّةٍ.
\par 13 فَيُكَفِّرُ عَنْهُ الْكَاهِنُ مِنْ خَطِيَّتِهِ الَّتِي اخْطَا بِهَا فِي وَاحِدَةٍ مِنْ ذَلِكَ فَيُصْفَحُ عَنْهُ. وَيَكُونُ لِلْكَاهِنِ كَالتَّقْدِمَةِ».
\par 14 وَقَالَ الرَّبُّ لِمُوسَى:
\par 15 «اذَا خَانَ احَدٌ خِيَانَةً وَاخْطَا سَهْوا فِي اقْدَاسِ الرَّبِّ يَاتِي الَى الرَّبِّ بِذَبِيحَةٍ لاثْمِهِ: كَبْشا صَحِيحا مِنَ الْغَنَمِ بِتَقْوِيمِكَ مِنْ شَوَاقِلِ فِضَّةٍ عَلَى شَاقِلِ الْقُدْسِ ذَبِيحَةَ اثْمٍ.
\par 16 وَيُعَوِّضُ عَمَّا اخْطَا بِهِ مِنَ الْقُدْسِ وَيَزِيدُ عَلَيْهِ خُمْسَهُ وَيَدْفَعُهُ الَى الْكَاهِنِ فَيُكَفِّرُ الْكَاهِنُ عَنْهُ بِكَبْشِ الْاثْمِ فَيُصْفَحُ عَنْهُ.
\par 17 «وَاذَا اخْطَا احَدٌ وَعَمِلَ وَاحِدَةً مِنْ جَمِيعِ مَنَاهِي الرَّبِّ الَّتِي لا يَنْبَغِي عَمَلُهَا وَلَمْ يَعْلَمْ كَانَ مُذْنِبا وَحَمَلَ ذَنْبَهُ.
\par 18 فَيَاتِي بِكَبْشٍ صَحِيحٍ مِنَ الْغَنَمِ بِتَقْوِيمِكَ ذَبِيحَةَ اثْمٍ الَى الْكَاهِنِ فَيُكَفِّرُ عَنْهُ الْكَاهِنُ مِنْ سَهْوِهِ الَّذِي سَهَا وَهُوَ لا يَعْلَمُ فَيُصْفَحُ عَنْهُ.
\par 19 انَّهُ ذَبِيحَةُ اثْمٍ. قَدْ اثِمَ اثْما الَى الرَّبِّ».

\chapter{6}

\par 1 وَقَالَ الرَّبُّ لِمُوسَى:
\par 2 «اذَا اخْطَا احَدٌ وَخَانَ خِيَانَةً بِالرَّبِّ وَجَحَدَ صَاحِبَهُ وَدِيعَةً اوْ امَانَةً اوْ مَسْلُوبا اوِ اغْتَصَبَ مِنْ صَاحِبِهِ
\par 3 اوْ وَجَدَ لُقَطَةً وَجَحَدَهَا وَحَلَفَ كَاذِبا عَلَى شَيْءٍ مِنْ كُلِّ مَا يَفْعَلُهُ الْانْسَانُ مُخْطِئا بِهِ -
\par 4 فَاذَا اخْطَا وَاذْنَبَ يَرُدُّ الْمَسْلُوبَ الَّذِي سَلَبَهُ اوِ الْمُغْتَصَبَ الَّذِي اغْتَصَبَهُ اوِ الْوَدِيعَةَ الَّتِي اودِعَتْ عِنْدَهُ اوِ اللُّقَطَةَ الَّتِي وَجَدَهَا
\par 5 اوْ كُلَّ مَا حَلَفَ عَلَيْهِ كَاذِبا. يُعَوِّضُهُ بِرَاسِهِ وَيَزِيدُ عَلَيْهِ خُمْسَهُ. الَى الَّذِي هُوَ لَهُ يَدْفَعُهُ يَوْمَ ذَبِيحَةِ اثْمِهِ.
\par 6 وَيَاتِي الَى الرَّبِّ بِذَبِيحَةٍ لاثْمِهِ كَبْشا صَحِيحا مِنَ الْغَنَمِ بِتَقْوِيمِكَ ذَبِيحَةَ اثْمٍ الَى الْكَاهِنِ.
\par 7 فَيُكَفِّرُ عَنْهُ الْكَاهِنُ امَامَ الرَّبِّ فَيُصْفَحُ عَنْهُ فِي الشَّيْءِ مِنْ كُلِّ مَا فَعَلَهُ مُذْنِبا بِهِ».
\par 8 وَقَالَ الرَّبُّ لِمُوسَى:
\par 9 «اوْصِ هَارُونَ وَبَنِيهِ قَائِلا: هَذِهِ شَرِيعَةُ الْمُحْرَقَةِ: هِيَ الْمُحْرَقَةُ تَكُونُ عَلَى الْمَوْقِدَةِ فَوْقَ الْمَذْبَحِ كُلَّ اللَّيْلِ حَتَّى الصَّبَاحِ وَنَارُ الْمَذْبَحِ تَتَّقِدُ عَلَيْهِ.
\par 10 ثُمَّ يَلْبِسُ الْكَاهِنُ ثَوْبَهُ مِنْ كَتَّانٍ وَيَلْبِسُ سَرَاوِيلَ مِنْ كَتَّانٍ عَلَى جَسَدِهِ وَيَرْفَعُ الرَّمَادَ الَّذِي صَيَّرَتِ النَّارُ الْمُحْرَقَةَ ايَّاهُ عَلَى الْمَذْبَحِ وَيَضَعُهُ بِجَانِبِ الْمَذْبَحِ.
\par 11 ثُمَّ يَخْلَعُ ثِيَابَهُ وَيَلْبِسُ ثِيَابا اخْرَى وَيُخْرِجُ الرَّمَادَ الَى خَارِجِ الْمَحَلَّةِ الَى مَكَانٍ طَاهِرٍ.
\par 12 وَالنَّارُ عَلَى الْمَذْبَحِ تَتَّقِدُ عَلَيْهِ. لا تَطْفَا. وَيُشْعِلُ عَلَيْهَا الْكَاهِنُ حَطَبا كُلَّ صَبَاحٍ وَيُرَتِّبُ عَلَيْهَا الْمُحْرَقَةَ وَيُوقِدُ عَلَيْهَا شَحْمَ ذَبَائِحِ السَّلامَةِ.
\par 13 نَارٌ دَائِمَةٌ تَتَّقِدُ عَلَى الْمَذْبَحِ. لا تَطْفَا.
\par 14 «وَهَذِهِ شَرِيعَةُ التَّقْدِمَةِ: يُقَدِّمُهَا بَنُو هَارُونَ امَامَ الرَّبِّ الَى قُدَّامِ الْمَذْبَحِ
\par 15 وَيَاخُذُ مِنْهَا بِقَبْضَتِهِ بَعْضَ دَقِيقِ التَّقْدِمَةِ وَزَيْتِهَا وَكُلَّ اللُّبَانِ الَّذِي عَلَى التَّقْدِمَةِ وَيُوقِدُ عَلَى الْمَذْبَحِ رَائِحَةَ سُرُورٍ تِذْكَارَهَا لِلرَّبِّ.
\par 16 وَالْبَاقِي مِنْهَا يَاكُلُهُ هَارُونُ وَبَنُوهُ. فَطِيرا يُؤْكَلُ فِي مَكَانٍ مُقَدَّسٍ. فِي دَارِ خَيْمَةِ الاجْتِمَاعِ يَاكُلُونَهُ.
\par 17 لا يُخْبَزُ خَمِيرا. قَدْ جَعَلْتُهُ نَصِيبَهُمْ مِنْ وَقَائِدِي. انَّهَا قُدْسُ اقْدَاسٍ كَذَبِيحَةِ الْخَطِيَّةِ وَذَبِيحَةِ الْاثْمِ.
\par 18 كُلُّ ذَكَرٍ مِنْ بَنِي هَارُونَ يَاكُلُ مِنْهَا. فَرِيضَةً دَهْرِيَّةً فِي اجْيَالِكُمْ مِنْ وَقَائِدِ الرَّبِّ. كُلُّ مَنْ مَسَّهَا يَتَقَدَّسُ».
\par 19 وقَالَ الرَّبُّ لِمُوسَى:
\par 20 «هَذَا قُرْبَانُ هَارُونَ وَبَنِيهِ الَّذِي يُقَرِّبُونَهُ لِلرَّبِّ يَوْمَ مَسْحَتِهِ: عُشْرُ الْايفَةِ مِنْ دَقِيقٍ تَقْدِمَةً دَائِمَةً نِصْفُهَا صَبَاحا وَنِصْفُهَا مَسَاءً.
\par 21 عَلَى صَاجٍ تُعْمَلُ بِزَيْتٍ مَرْبُوكَةً تَاتِي بِهَا. ثَرَائِدَ تَقْدِمَةٍ فُتَاتا تُقَرِّبُهَا رَائِحَةَ سُرُورٍ لِلرَّبِّ.
\par 22 وَالْكَاهِنُ الْمَمْسُوحُ عِوَضا عَنْهُ مِنْ بَنِيهِ يَعْمَلُهَا فَرِيضَةً دَهْرِيَّةً لِلرَّبِّ تُوقَدُ بِكَمَالِهَا.
\par 23 وَكُلُّ تَقْدِمَةِ كَاهِنٍ تُحْرَقُ بِكَمَالِهَا. لا تُؤْكَلُ».
\par 24 وَقَالَ الرَّبُّ لِمُوسَى:
\par 25 «كَلِّمْ هَارُونَ وَبَنِيهِ قَائِلا: هَذِهِ شَرِيعَةُ ذَبِيحَةِ الْخَطِيَّةِ. فِي الْمَكَانِ الَّذِي تُذْبَحُ فِيهِ الْمُحْرَقَةُ تُذْبَحُ ذَبِيحَةُ الْخَطِيَّةِ امَامَ الرَّبِّ. انَّهَا قُدْسُ اقْدَاسٍ.
\par 26 الْكَاهِنُ الَّذِي يَعْمَلُهَا لِلْخَطِيَّةِ يَاكُلُهَا. فِي مَكَانٍ مُقَدَّسٍ تُؤْكَلُ فِي دَارِ خَيْمَةِ الاجْتِمَاعِ.
\par 27 كُلُّ مَنْ مَسَّ لَحْمَهَا يَتَقَدَّسُ. وَاذَا انْتَثَرَ مِنْ دَمِهَا عَلَى ثَوْبٍ تَغْسِلُ مَا انْتَثَرَ عَلَيْهِ فِي مَكَانٍ مُقَدَّسٍ.
\par 28 وَامَّا انَاءُ الْخَزَفِ الَّذِي تُطْبَخُ فِيهِ فَيُكْسَرُ. وَانْ طُبِخَتْ فِي انَاءِ نُحَاسٍ يُجْلَى وَيُشْطَفُ بِمَاءٍ.
\par 29 كُلُّ ذَكَرٍ مِنَ الْكَهَنَةِ يَاكُلُ مِنْهَا. انَّهَا قُدْسُ اقْدَاسٍ.
\par 30 وَكُلُّ ذَبِيحَةِ خَطِيَّةٍ يُدْخَلُ مِنْ دَمِهَا الَى خَيْمَةِ الاجْتِمَاعِ لِلتَّكْفِيرِ فِي الْقُدْسِ لا تُؤْكَلُ. تُحْرَقُ بِنَارٍ.

\chapter{7}

\par 1 «وَهَذِهِ شَرِيعَةُ ذَبِيحَةِ الْاثْمِ: انَّهَا قُدْسُ اقْدَاسٍ.
\par 2 فِي الْمَكَانِ الَّذِي يَذْبَحُونَ فِيهِ الْمُحْرَقَةَ يَذْبَحُونَ ذَبِيحَةَ الْاثْمِ. وَيَرُشُّ دَمَهَا عَلَى الْمَذْبَحِ مُسْتَدِيرا
\par 3 وَيُقَرِّبُ مِنْهَا كُلَّ شَحْمِهَا: الالْيَةَ وَالشَّحْمَ الَّذِي يُغَشِّي الاحْشَاءَ
\par 4 وَالْكُلْيَتَيْنِ وَالشَّحْمَ الَّذِي عَلَيْهِمَا الَّذِي عَلَى الْخَاصِرَتَيْنِ وَزِيَادَةَ الْكَبِدِ مَعَ الْكُلْيَتَيْنِ يَنْزِعُهَا.
\par 5 وَيُوقِدُهُنَّ الْكَاهِنُ عَلَى الْمَذْبَحِ وَقُودا لِلرَّبِّ. انَّهَا ذَبِيحَةُ اثْمٍ.
\par 6 كُلُّ ذَكَرٍ مِنَ الْكَهَنَةِ يَاكُلُ مِنْهَا. فِي مَكَانٍ مُقَدَّسٍ تُؤْكَلُ. انَّهَا قُدْسُ اقْدَاسٍ.
\par 7 ذَبِيحَةُ الْاثْمِ كَذَبِيحَةِ الْخَطِيَّةِ لَهُمَا شَرِيعَةٌ وَاحِدَةٌ. الْكَاهِنُ الَّذِي يُكَفِّرُ بِهَا تَكُونُ لَهُ.
\par 8 وَالْكَاهِنُ الَّذِي يُقَرِّبُ مُحْرَقَةَ انْسَانٍ فَجِلْدُ الْمُحْرَقَةِ الَّتِي يُقَرِّبُهَا يَكُونُ لَهُ.
\par 9 وَكُلُّ تَقْدِمَةٍ خُبِزَتْ فِي التَّنُّورِ وَكُلُّ مَا عُمِلَ فِي طَاجِنٍ اوْ عَلَى صَاجٍ يَكُونُ لِلْكَاهِنِ الَّذِي يُقَرِّبُهُ.
\par 10 وَكُلُّ تَقْدِمَةٍ مَلْتُوتَةً بِزَيْتٍ اوْ نَاشِفَةً تَكُونُ لِجَمِيعِ بَنِي هَارُونَ كُلِّ انْسَانٍ كَاخِيهِ.
\par 11 «وَهَذِهِ شَرِيعَةُ ذَبِيحَةِ السَّلامَةِ. الَّذِي يُقَرِّبُهَا لِلرَّبِّ
\par 12 انْ قَرَّبَهَا لاجْلِ الشُّكْرِ يُقَرِّبُ عَلَى ذَبِيحَةِ الشُّكْرِ اقْرَاصَ فَطِيرٍ مَلْتُوتَةً بِزَيْتٍ وَرِقَاقَ فَطِيرٍ مَدْهُونَةً بِزَيْتٍ وَدَقِيقا مَرْبُوكا اقْرَاصا مَلْتُوتَةً بِزَيْتٍ
\par 13 مَعَ اقْرَاصِ خُبْزٍ خَمِيرٍ يُقَرِّبُ قُرْبَانَهُ عَلَى ذَبِيحَةِ شُكْرِ سَلامَتِهِ.
\par 14 وَيُقَرِّبُ مِنْهُ وَاحِدا مِنْ كُلِّ قُرْبَانٍ رَفِيعَةً لِلرَّبِّ يَكُونُ لِلْكَاهِنِ الَّذِي يَرُشُّ دَمَ ذَبِيحَةِ السَّلامَةِ.
\par 15 وَلَحْمُ ذَبِيحَةِ شُكْرِ سَلامَتِهِ يُؤْكَلُ يَوْمَ قُرْبَانِهِ. لا يُبْقِي مِنْهُ شَيْئا الَى الصَّبَاحِ.
\par 16 وَانْ كَانَتْ ذَبِيحَةُ قُرْبَانِهِ نَذْرا اوْ نَافِلَةً فَفِي يَوْمِ تَقْرِيبِهِ ذَبِيحَتَهُ تُؤْكَلُ. وَفِي الْغَدِ يُؤْكَلُ مَا فَضَلَ مِنْهَا.
\par 17 وَامَّا الْفَاضِلُ مِنْ لَحْمِ الذَّبِيحَةِ فِي الْيَوْمِ الثَّالِثِ فَيُحْرَقُ بِالنَّارِ.
\par 18 وَانْ اكِلَ مِنْ لَحْمِ ذَبِيحَةِ سَلامَتِهِ فِي الْيَوْمِ الثَّالِثِ لا تُقْبَلُ. الَّذِي يُقَرِّبُهَا لا تُحْسَبُ لَهُ. تَكُونُ نَجَاسَةً. وَالنَّفْسُ الَّتِي تَاكُلُ مِنْهَا تَحْمِلُ ذَنْبَهَا.
\par 19 وَاللَّحْمُ الَّذِي مَسَّ شَيْئا مَا نَجِسا لا يُؤْكَلُ. يُحْرَقُ بِالنَّارِ. وَاللَّحْمُ يَاكُلُ كُلُّ طَاهِرٍ مِنْهُ.
\par 20 وَامَّا النَّفْسُ الَّتِي تَاكُلُ لَحْما مِنْ ذَبِيحَةِ السَّلامَةِ الَّتِي لِلرَّبِّ وَنَجَاسَتُهَا عَلَيْهَا فَتُقْطَعُ تِلْكَ النَّفْسُ مِنْ شَعْبِهَا.
\par 21 وَالنَّفْسُ الَّتِي تَمَسُّ شَيْئا مَا نَجِسا نَجَاسَةَ انْسَانٍ اوْ بَهِيمَةً نَجِسَةً اوْ مَكْرُوها مَا نَجِسا ثُمَّ تَاكُلُ مِنْ لَحْمِ ذَبِيحَةِ السَّلامَةِ الَّتِي لِلرَّبِّ تُقْطَعُ تِلْكَ النَّفْسُ مِنْ شَعْبِهَا».
\par 22 وَقَالَ الرَّبُّ لِمُوسَى:
\par 23 «قُلْ لِبَنِي اسْرَائِيلَ: كُلَّ شَحْمِ ثَوْرٍ اوْ كَبْشٍ اوْ مَاعِزٍ لا تَاكُلُوا.
\par 24 وَامَّا شَحْمُ الْمَيْتَةِ وَشَحْمُ الْمُفْتَرَسَةِ فَيُسْتَعْمَلُ لِكُلِّ عَمَلٍ. لَكِنْ اكْلا لا تَاكُلُوهُ.
\par 25 انَّ كُلَّ مَنْ اكَلَ شَحْما مِنَ الْبَهَائِمِ الَّتِي يُقَرِّبُ مِنْهَا وَقُودا لِلرَّبِّ تُقْطَعُ مِنْ شَعْبِهَا النَّفْسُ الَّتِي تَاكُلُ.
\par 26 وَكُلَّ دَمٍ لا تَاكُلُوا فِي جَمِيعِ مَسَاكِنِكُمْ مِنَ الطَّيْرِ وَمِنَ الْبَهَائِمِ.
\par 27 كُلُّ نَفْسٍ تَاكُلُ شَيْئا مِنَ الدَّمِ تُقْطَعُ تِلْكَ النَّفْسُ مِنْ شَعْبِهَا».
\par 28 وَقَال الرَّبُّ لِمُوسَى:
\par 29 «قُلْ لِبَنِي اسْرَائِيلَ: الَّذِي يُقَرِّبُ ذَبِيحَةَ سَلامَتِهِ لِلرَّبِّ يَاتِي بِقُرْبَانِهِ الَى الرَّبِّ مِنْ ذَبِيحَةِ سَلامَتِهِ.
\par 30 يَدَاهُ تَاتِيَانِ بِوَقَائِدِ الرَّبِّ. الشَّحْمُ يَاتِي بِهِ مَعَ الصَّدْرِ. امَّا الصَّدْرُ فَلِكَيْ يُرَدِّدَهُ تَرْدِيدا امَامَ الرَّبِّ.
\par 31 فَيُوقِدُ الْكَاهِنُ الشَّحْمَ عَلَى الْمَذْبَحِ وَيَكُونُ الصَّدْرُ لِهَارُونَ وَبَنِيهِ.
\par 32 وَالسَّاقُ الْيُمْنَى تُعْطُونَهَا رَفِيعَةً لِلْكَاهِنِ مِنْ ذَبَائِحِ سَلامَتِكُمْ.
\par 33 الَّذِي يُقَرِّبُ دَمَ ذَبِيحَةِ السَّلامَةِ وَالشَّحْمَ مِنْ بَنِي هَارُونَ تَكُونُ لَهُ السَّاقُ الْيُمْنَى نَصِيبا
\par 34 لانَّ صَدْرَ التَّرْدِيدِ وَسَاقَ الرَّفِيعَةِ قَدْ اخَذْتُهُمَا مِنْ بَنِي اسْرَائِيلَ مِنْ ذَبَائِحِ سَلامَتِهِمْ وَاعْطَيْتُهُمَا لِهَارُونَ الْكَاهِنِ وَلِبَنِيهِ فَرِيضَةً دَهْرِيَّةً مِنْ بَنِي اسْرَائِيلَ».
\par 35 تِلْكَ مَسْحَةُ هَارُونَ وَمَسْحَةُ بَنِيهِ مِنْ وَقَائِدِ الرَّبِّ يَوْمَ تَقْدِيمِهِمْ لِيَكْهَنُوا لِلرَّبِّ
\par 36 الَّتِي امَرَ الرَّبُّ انْ تُعْطَى لَهُمْ يَوْمَ مَسْحِهِ ايَّاهُمْ مِنْ بَنِي اسْرَائِيلَ فَرِيضَةً دَهْرِيَّةً فِي اجْيَالِهِمْ.
\par 37 تِلْكَ شَرِيعَةُ الْمُحْرَقَةِ وَالتَّقْدِمَةِ وَذَبِيحَةِ الْخَطِيَّةِ وَذَبِيحَةِ الْاثْمِ وَذَبِيحَةِ الْمِلْءِ وَذَبِيحَةِ السَّلامَةِ
\par 38 الَّتِي امَرَ الرَّبُّ بِهَا مُوسَى فِي جَبَلِ سِينَاءَ يَوْمَ امْرِهِ بَنِي اسْرَائِيلَ بِتَقْرِيبِ قَرَابِينِهِمْ لِلرَّبِّ فِي بَرِّيَّةِ سِينَاءَ.

\chapter{8}

\par 1 وَقَالَ الرَّبُّ لِمُوسَى:
\par 2 «خُذْ هَارُونَ وَبَنِيهِ مَعَهُ وَالثِّيَابَ وَدُهْنَ الْمَسْحَةِ وَثَوْرَ الْخَطِيَّةِ وَالْكَبْشَيْنِ وَسَلَّ الْفَطِيرِ
\par 3 وَاجْمَعْ كُلَّ الْجَمَاعَةِ الَى بَابِ خَيْمَةِ الاجْتِمَاعِ».
\par 4 فَفَعَلَ مُوسَى كَمَا امَرَهُ الرَّبُّ. فَاجْتَمَعَتِ الْجَمَاعَةُ الَى بَابِ خَيْمَةِ الاجْتِمَاعِ.
\par 5 ثُمَّ قَالَ مُوسَى لِلْجَمَاعَةِ: «هَذَا مَا امَرَ الرَّبُّ انْ يُفْعَلَ».
\par 6 فَقَدَّمَ مُوسَى هَارُونَ وَبَنِيهِ وَغَسَّلَهُمْ بِمَاءٍ.
\par 7 وَجَعَلَ عَلَيْهِ الْقَمِيصَ وَنَطَّقَهُ بِالْمِنْطَقَةِ وَالْبَسَهُ الْجُبَّةَ وَجَعَلَ عَلَيْهِ الرِّدَاءَ وَنَطَّقَهُ بِزُنَّارِ الرِّدَاءِ وَشَدَّهُ بِهِ.
\par 8 وَوَضَعَ عَلَيْهِ الصُّدْرَةَ وَجَعَلَ فِي الصُّدْرَةِ الاورِيمَ وَالتُّمِّيمَ.
\par 9 وَوَضَعَ الْعِمَامَةَ عَلَى رَاسِهِ وَوَضَعَ عَلَى الْعِمَامَةِ الَى جِهَةِ وَجْهِهِ صَفِيحَةَ الذَّهَبِ الْاكْلِيلَ الْمُقَدَّسَ كَمَا امَرَ الرَّبُّ مُوسَى.
\par 10 ثُمَّ اخَذَ مُوسَى دُهْنَ الْمَسْحَةِ وَمَسَحَ الْمَسْكَنَ وَكُلَّ مَا فِيهِ وَقَدَّسَهُ
\par 11 وَنَضَحَ مِنْهُ عَلَى الْمَذْبَحِ سَبْعَ مَرَّاتٍ وَمَسَحَ الْمَذْبَحَ وَجَمِيعَ انِيَتِهِ وَالْمِرْحَضَةَ وَقَاعِدَتَهَا لِتَقْدِيسِهَا.
\par 12 وَصَبَّ مِنْ دُهْنِ الْمَسْحَةِ عَلَى رَاسِ هَارُونَ وَمَسَحَهُ لِتَقْدِيسِهِ.
\par 13 ثُمَّ قَدَّمَ مُوسَى بَنِي هَارُونَ وَالْبَسَهُمْ اقْمِصَةً وَنَطَّقَهُمْ بِمَنَاطِقَ وَشَدَّ لَهُمْ قَلانِسَ - كَمَا امَرَ الرَّبُّ مُوسَى.
\par 14 ثُمَّ قَدَّمَ ثَوْرَ الْخَطِيَّةِ وَوَضَعَ هَارُونُ وَبَنُوهُ ايْدِيَهُمْ عَلَى رَاسِ ثَوْرِ الْخَطِيَّةِ
\par 15 فَذَبَحَهُ وَاخَذَ مُوسَى الدَّمَ وَجَعَلَهُ عَلَى قُرُونِ الْمَذْبَحِ مُسْتَدِيرا بِاصْبِعِهِ وَطَهَّرَ الْمَذْبَحَ ثُمَّ صَبَّ الدَّمَ الَى اسْفَلِ الْمَذْبَحِ وَقَدَّسَهُ تَكْفِيرا عَنْهُ.
\par 16 وَاخَذَ كُلَّ الشَّحْمِ الَّذِي عَلَى الاحْشَاءِ وَزِيَادَةَ الْكَبِدِ وَالْكُلْيَتَيْنِ وَشَحْمَهُمَا وَاوْقَدَهُ مُوسَى عَلَى الْمَذْبَحِ.
\par 17 وَامَّا الثَّوْرُ: جِلْدُهُ وَلَحْمُهُ وَفَرْثُهُ فَاحْرَقَهُ بِنَارٍ خَارِجَ الْمَحَلَّةِ كَمَا امَرَ الرَّبُّ مُوسَى.
\par 18 ثُمَّ قَدَّمَ كَبْشَ الْمُحْرَقَةِ فَوَضَعَ هَارُونُ وَبَنُوهُ ايْدِيَهُمْ عَلَى رَاسِ الْكَبْشِ.
\par 19 فَذَبَحَهُ وَرَشَّ مُوسَى الدَّمَ عَلَى الْمَذْبَحِ مُسْتَدِيرا.
\par 20 وَقَطَّعَ الْكَبْشَ الَى قِطَعِهِ. وَاوْقَدَ مُوسَى الرَّاسَ وَالْقِطَعَ وَالشَّحْمَ.
\par 21 وَامَّا الاحْشَاءُ وَالاكَارِعُ فَغَسَلَهَا بِمَاءٍ وَاوْقَدَ مُوسَى كُلَّ الْكَبْشِ عَلَى الْمَذْبَحِ. انَّهُ مُحْرَقَةٌ لِرَائِحَةِ سُرُورٍ. وَقُودٌ هُوَ لِلرَّبِّ - كَمَا امَرَ الرَّبُّ مُوسَى.
\par 22 ثُمَّ قَدَّمَ الْكَبْشَ الثَّانِيَ كَبْشَ الْمَلْءِ فَوَضَعَ هَارُونُ وَبَنُوهُ ايْدِيَهُمْ عَلَى رَاسِ الْكَبْشِ.
\par 23 فَذَبَحَهُ وَاخَذَ مُوسَى مِنْ دَمِهِ وَجَعَلَ عَلَى شَحْمَةِ اذُنِ هَارُونَ الْيُمْنَى وَعَلَى ابْهَامِ يَدِهِ الْيُمْنَى وَعَلَى ابْهَامِ رِجْلِهِ الْيُمْنَى.
\par 24 ثُمَّ قَدَّمَ مُوسَى بَنِي هَارُونَ وَجَعَلَ مِنَ الدَّمِ عَلَى شَحْمِ اذَانِهِمِ الْيُمْنَى وَعَلَى ابَاهِمِ ايْدِيهِمِ الْيُمْنَى وَعَلَى ابَاهِمِ ارْجُلِهِمِ الْيُمْنَى ثُمَّ رَشَّ مُوسَى الدَّمَ عَلَى الْمَذْبَحِ مُسْتَدِيرا.
\par 25 ثُمَّ اخَذَ الشَّحْمَ: الالْيَةَ وَكُلَّ الشَّحْمِ الَّذِي عَلَى الاحْشَاءِ وَزِيَادَةَ الْكَبِدِ وَالْكُلْيَتَيْنِ وَشَحْمَهُمَا وَالسَّاقَ الْيُمْنَى.
\par 26 وَمِنْ سَلِّ الْفَطِيرِ الَّذِي امَامَ الرَّبِّ اخَذَ قُرْصا وَاحِدا فَطِيرا وَقُرْصا وَاحِدا مِنَ الْخُبْزِ بِزَيْتٍ وَرُقَاقَةً وَاحِدَةً وَوَضَعَهَا عَلَى الشَّحْمِ وَعَلَى السَّاقِ الْيُمْنَى
\par 27 وَجَعَلَ الْجَمِيعَ عَلَى كَفَّيْ هَارُونَ وَكُفُوفِ بَنِيهِ وَرَدَّدَهَا تَرْدِيدا امَامَ الرَّبِّ.
\par 28 ثُمَّ اخَذَهَا مُوسَى عَنْ كُفُوفِهِمْ وَاوْقَدَهَا عَلَى الْمَذْبَحِ فَوْقَ الْمُحْرَقَةِ. انَّهَا قُرْبَانُ مَلْءٍ لِرَائِحَةِ سُرُورٍ. وَقُودٌ هِيَ لِلرَّبِّ.
\par 29 ثُمَّ اخَذَ مُوسَى الصَّدْرَ وَرَدَّدَهُ تَرْدِيدا امَامَ الرَّبِّ مِنْ كَبْشِ الْمَلْءِ. لِمُوسَى كَانَ نَصِيبا - كَمَا امَرَ الرَّبُّ مُوسَى.
\par 30 ثُمَّ اخَذَ مُوسَى مِنْ دُهْنِ الْمَسْحَةِ وَمِنَ الدَّمِ الَّذِي عَلَى الْمَذْبَحِ وَنَضَحَ عَلَى هَارُونَ وَعَلَى ثِيَابِهِ وَعَلَى بَنِيهِ وَعَلَى ثِيَابِ بَنِيهِ مَعَهُ. وَقَدَّسَ هَارُونَ وَثِيَابَهُ وَبَنِيهِ وَثِيَابَ بَنِيهِ مَعَهُ.
\par 31 ثُمَّ قَالَ مُوسَى لِهَارُونَ وَبَنِيهِ: «اطْبُخُوا اللَّحْمَ لَدَى بَابِ خَيْمَةِ الاجْتِمَاعِ وَهُنَاكَ تَاكُلُونَهُ وَالْخُبْزَ الَّذِي فِي سَلِّ قُرْبَانِ الْمَلْءِ كَمَا امَرْتُ قَائِلا: هَارُونُ وَبَنُوهُ يَاكُلُونَهُ.
\par 32 وَالْبَاقِي مِنَ اللَّحْمِ وَالْخُبْزِ تُحْرِقُونَهُ بِالنَّارِ.
\par 33 وَمِنْ لَدُنْ بَابِ خَيْمَةِ الاجْتِمَاعِ لا تَخْرُجُونَ سَبْعَةَ ايَّامٍ الَى يَوْمِ كَمَالِ ايَّامِ مَلْئِكُمْ لانَّهُ سَبْعَةَ ايَّامٍ يَمْلَا ايْدِيَكُمْ.
\par 34 كَمَا فَعَلَ فِي هَذَا الْيَوْمِ قَدْ امَرَ الرَّبُّ انْ يُفْعَلَ لِلتَّكْفِيرِ عَنْكُمْ.
\par 35 وَلَدَى بَابِ خَيْمَةِ الاجْتِمَاعِ تُقِيمُونَ نَهَارا وَلَيْلا سَبْعَةَ ايَّامٍ وَتَحْفَظُونَ شَعَائِرَ الرَّبِّ فَلا تَمُوتُونَ لانِّي هَكَذَا امِرْتُ».
\par 36 فَعَمِلَ هَارُونُ وَبَنُوهُ كُلَّ مَا امَرَ بِهِ الرَّبُّ عَلَى يَدِ مُوسَى.

\chapter{9}

\par 1 وَفِي الْيَوْمِ الثَّامِنِ دَعَا مُوسَى هَارُونَ وَبَنِيهِ وَشُيُوخَ اسْرَائِيلَ.
\par 2 وَقَالَ لِهَارُونَ: «خُذْ لَكَ عِجْلا ابْنَ بَقَرٍ لِذَبِيحَةِ خَطِيَّةٍ وَكَبْشا لِمُحْرَقَةٍ صَحِيحَيْنِ. وَقَدِّمْهُمَا امَامَ الرَّبِّ.
\par 3 وَقُلْ لِبَنِي اسْرَائِيلَ: خُذُوا تَيْسا مِنَ الْمَعْزِ لِذَبِيحَةِ خَطِيَّةٍ وَعِجْلا وَخَرُوفا حَوْلِيَّيْنِ صَحِيحَيْنِ لِمُحْرَقَةٍ
\par 4 وَثَوْرا وَكَبْشا لِذَبِيحَةِ سَلامَةٍ لِلذَّبْحِ امَامَ الرَّبِّ وَتَقْدِمَةً مَلْتُوتَةً بِزَيْتٍ. لانَّ الرَّبَّ الْيَوْمَ يَتَرَاءَى لَكُمْ».
\par 5 فَاخَذُوا مَا امَرَ بِهِ مُوسَى الَى قُدَّامِ خَيْمَةِ الاجْتِمَاعِ. وَتَقَدَّمَ كُلُّ الْجَمَاعَةِ وَوَقَفُوا امَامَ الرَّبِّ.
\par 6 فَقَالَ مُوسَى: «هَذَا مَا امَرَ بِهِ الرَّبُّ. تَعْمَلُونَهُ فَيَتَرَاءَى لَكُمْ مَجْدُ الرَّبِّ».
\par 7 ثُمَّ قَالَ مُوسَى لِهَارُونَ: «تَقَدَّمْ الَى الْمَذْبَحِ وَاعْمَلْ ذَبِيحَةَ خَطِيَّتِكَ وَمُحْرَقَتَكَ وَكَفِّرْ عَنْ نَفْسِكَ وَعَنِ الشَّعْبِ. وَاعْمَلْ قُرْبَانَ الشَّعْبِ وَكَفِّرْ عَنْهُمْ كَمَا امَرَ الرَّبُّ».
\par 8 فَتَقَدَّمَ هَارُونُ الَى الْمَذْبَحِ وَذَبَحَ عِجْلَ الْخَطِيَّةِ الَّذِي لَهُ.
\par 9 وَقَدَّمَ بَنُو هَارُونَ الَيْهِ الدَّمَ فَغَمَسَ اصْبِعَهُ فِي الدَّمِ وَجَعَلَ عَلَى قُرُونِ الْمَذْبَحِ. ثُمَّ صَبَّ الدَّمَ الَى اسْفَلِ الْمَذْبَحِ.
\par 10 وَالشَّحْمَ وَالْكُلْيَتَيْنِ وَزِيَادَةَ الْكَبِدِ مِنْ ذَبِيحَةِ الْخَطِيَّةِ اوْقَدَهَا عَلَى الْمَذْبَحِ - كَمَا امَرَ الرَّبُّ مُوسَى.
\par 11 وَامَّا اللَّحْمُ وَالْجِلْدُ فَاحْرَقَهُمَا بِنَارٍ خَارِجَ الْمَحَلَّةِ.
\par 12 ثُمَّ ذَبَحَ الْمُحْرَقَةَ فَنَاوَلَهُ بَنُو هَارُونَ الدَّمَ فَرَشَّهُ عَلَى الْمَذْبَحِ مُسْتَدِيرا.
\par 13 ثُمَّ نَاوَلُوهُ الْمُحْرَقَةَ بِقِطَعِهَا وَالرَّاسَ فَاوْقَدَهَا عَلَى الْمَذْبَحِ.
\par 14 وَغَسَّلَ الاحْشَاءَ وَالاكَارِعَ وَاوْقَدَهَا فَوْقَ الْمُحْرَقَةِ عَلَى الْمَذْبَحِ.
\par 15 ثُمَّ قَدَّمَ قُرْبَانَ الشَّعْبِ وَاخَذَ تَيْسَ الْخَطِيَّةِ الَّذِي لِلشَّعْبِ وَذَبَحَهُ وَعَمِلَهُ لِلْخَطِيَّةِ كَالاوَّلِ.
\par 16 ثُمَّ قَدَّمَ الْمُحْرَقَةَ وَعَمِلَهَا كَالْعَادَةِ.
\par 17 ثُمَّ قَدَّمَ التَّقْدِمَةَ وَمَلَا كَفَّهُ مِنْهَا وَاوْقَدَهَا عَلَى الْمَذْبَحِ عَدَا مُحْرَقَةِ الصَّبَاحِ.
\par 18 ثُمَّ ذَبَحَ الثَّوْرَ وَالْكَبْشَ ذَبِيحَةَ السَّلامَةِ الَّتِي لِلشَّعْبِ. وَنَاوَلَهُ بَنُو هَارُونَ الدَّمَ فَرَشَّهُ عَلَى الْمَذْبَحِ مُسْتَدِيرا.
\par 19 وَالشَّحْمَ مِنَ الثَّوْرِ وَمِنَ الْكَبْشِ: الالْيَةَ وَمَا يُغَشِّي وَالْكُلْيَتَيْنِ وَزِيَادَةَ الْكَبِدِ.
\par 20 وَوَضَعُوا الشَّحْمَ عَلَى الصَّدْرَيْنِ فَاوْقَدَ الشَّحْمَ عَلَى الْمَذْبَحِ.
\par 21 وَامَّا الصَّدْرَانِ وَالسَّاقُ الْيُمْنَى فَرَدَّدَهَا هَارُونُ تَرْدِيدا امَامَ الرَّبِّ - كَمَا امَرَ مُوسَى.
\par 22 ثُمَّ رَفَعَ هَارُونُ يَدَهُ نَحْوَ الشَّعْبِ وَبَارَكَهُمْ وَانْحَدَرَ مِنْ عَمَلِ ذَبِيحَةِ الْخَطِيَّةِ وَالْمُحْرَقَةِ وَذَبِيحَةِ السَّلامَةِ.
\par 23 وَدَخَلَ مُوسَى وَهَارُونُ الَى خَيْمَةِ الاجْتِمَاعِ ثُمَّ خَرَجَا وَبَارَكَا الشَّعْبَ. فَتَرَاءَى مَجْدُ الرَّبِّ لِكُلِّ الشَّعْبِ
\par 24 وَخَرَجَتْ نَارٌ مِنْ عِنْدِ الرَّبِّ وَاحْرَقَتْ عَلَى الْمَذْبَحِ الْمُحْرَقَةَ وَالشَّحْمَ. فَرَاى جَمِيعُ الشَّعْبِ وَهَتَفُوا وَسَقَطُوا عَلَى وُجُوهِهِمْ.

\chapter{10}

\par 1 وَاخَذَ ابْنَا هَارُونَ نَادَابُ وَابِيهُو كُلٌّ مِنْهُمَا مِجْمَرَتَهُ وَجَعَلا فِيهِمَا نَارا وَوَضَعَا عَلَيْهَا بَخُورا وَقَرَّبَا امَامَ الرَّبِّ نَارا غَرِيبَةً لَمْ يَامُرْهُمَا بِهَا.
\par 2 فَخَرَجَتْ نَارٌ مِنْ عِنْدِ الرَّبِّ وَاكَلَتْهُمَا فَمَاتَا امَامَ الرَّبِّ.
\par 3 فَقَالَ مُوسَى لِهَارُونَ: «هَذَا مَا تَكَلَّمَ بِهِ الرَّبُّ قَائِلا: فِي الْقَرِيبِينَ مِنِّي اتَقَدَّسُ وَامَامَ جَمِيعِ الشَّعْبِ اتَمَجَّدُ». فَصَمَتَ هَارُونُ.
\par 4 فَدَعَا مُوسَى مِيشَائِيلَ وَالْصَافَانَ ابْنَيْ عُزِّيئِيلَ عَمِّ هَارُونَ وَقَالَ لَهُمَا: «تَقَدَّمَا ارْفَعَا اخَوَيْكُمَا مِنْ قُدَّامِ الْقُدْسِ الَى خَارِجِ الْمَحَلَّةِ».
\par 5 فَتَقَدَّمَا وَرَفَعَاهُمَا فِي قَمِيصَيْهِمَا الَى خَارِجِ الْمَحَلَّةِ كَمَا قَالَ مُوسَى.
\par 6 وَقَالَ مُوسَى لِهَارُونَ وَالِعَازَارَ وَايثَامَارَ ابْنَيْهِ: «لا تَكْشِفُوا رُؤُوسَكُمْ وَلا تَشُقُّوا ثِيَابَكُمْ لِئَلَّا تَمُوتُوا وَيُسْخَطَ عَلَى كُلِّ الْجَمَاعَةِ. وَامَّا اخْوَتُكُمْ كُلُّ بَيْتِ اسْرَائِيلَ فَيَبْكُونَ عَلَى الْحَرِيقِ الَّذِي احْرَقَهُ الرَّبُّ.
\par 7 وَمِنْ بَابِ خَيْمَةِ الاجْتِمَاعِ لا تَخْرُجُوا لِئَلَّا تَمُوتُوا. لانَّ دُهْنَ مَسْحَةِ الرَّبِّ عَلَيْكُمْ». فَفَعَلُوا حَسَبَ كَلامِ مُوسَى.
\par 8 وَقَالَ الرَّبُّ لِهَارُونَ:
\par 9 «خَمْرا وَمُسْكِرا لا تَشْرَبْ انْتَ وَبَنُوكَ مَعَكَ عِنْدَ دُخُولِكُمْ الَى خَيْمَةِ الاجْتِمَاعِ لِكَيْ لا تَمُوتُوا. فَرْضا دَهْرِيّا فِي اجْيَالِكُمْ
\par 10 وَلِلتَّمْيِيزِ بَيْنَ الْمُقَدَّسِ وَالْمُحَلَّلِ وَبَيْنَ النَّجِسِ وَالطَّاهِرِ
\par 11 وَلِتَعْلِيمِ بَنِي اسْرَائِيلَ جَمِيعَ الْفَرَائِضِ الَّتِي كَلَّمَهُمُ الرَّبُّ بِهَا بِيَدِ مُوسَى».
\par 12 وَقَالَ مُوسَى لِهَارُونَ وَالِعَازَارَ وَايثَامَارَ ابْنَيْهِ الْبَاقِيَيْنِ: «خُذُوا التَّقْدِمَةَ الْبَاقِيَةَ مِنْ وَقَائِدِ الرَّبِّ وَكُلُوهَا فَطِيرا بِجَانِبِ الْمَذْبَحِ لانَّهَا قُدْسُ اقْدَاسٍ.
\par 13 كُلُوهَا فِي مَكَانٍ مُقَدَّسٍ لانَّهَا فَرِيضَتُكَ وَفَرِيضَةُ بَنِيكَ مِنْ وَقَائِدِ الرَّبِّ. فَانَّنِي هَكَذَا امِرْتُ.
\par 14 وَامَّا صَدْرُ التَّرْدِيدِ وَسَاقُ الرَّفِيعَةِ فَتَاكُلُونَهُمَا فِي مَكَانٍ طَاهِرٍ انْتَ وَبَنُوكَ وَبَنَاتُكَ مَعَكَ لانَّهُمَا جُعِلا فَرِيضَتَكَ وَفَرِيضَةَ بَنِيكَ مِنْ ذَبَائِحِ سَلامَةِ بَنِي اسْرَائِيلَ.
\par 15 سَاقُ الرَّفِيعَةِ وَصَدْرُ التَّرْدِيدِ يَاتُونَ بِهِمَا مَعَ وَقَائِدِ الشَّحْمِ لِيُرَدَّدَا تَرْدِيدا امَامَ الرَّبِّ فَيَكُونَانِ لَكَ وَلِبَنِيكَ مَعَكَ فَرِيضَةً دَهْرِيَّةً كَمَا امَرَ الرَّبُّ».
\par 16 وَامَّا تَيْسُ الْخَطِيَّةِ فَانَّ مُوسَى طَلَبَهُ فَاذَا هُوَ قَدِ احْتَرَقَ. فَسَخَطَ عَلَى الِعَازَارَ وَايثَامَارَ ابْنَيْ هَارُونَ الْبَاقِيَيْنِ وَقَالَ:
\par 17 «مَا لَكُمَا لَمْ تَاكُلا ذَبِيحَةَ الْخَطِيَّةِ فِي الْمَكَانِ الْمُقَدَّسِ لانَّهَا قُدْسُ اقْدَاسٍ وَقَدْ اعْطَاكُمَا ايَّاهَا لِتَحْمِلا اثْمَ الْجَمَاعَةِ تَكْفِيرا عَنْهُمْ امَامَ الرَّبِّ؟
\par 18 انَّهُ لَمْ يُؤْتَ بِدَمِهَا الَى الْقُدْسِ دَاخِلا. اكْلا تَاكُلانِهَا فِي الْقُدْسِ كَمَا امَرْتُ».
\par 19 فَقَالَ هَارُونُ لِمُوسَى: «انَّهُمَا الْيَوْمَ قَدْ قَرَّبَا ذَبِيحَةَ خَطِيَّتِهِمَا وَمُحْرَقَتَهُمَا امَامَ الرَّبِّ وَقَدْ اصَابَنِي مِثْلُ هَذِهِ. فَلَوْ اكَلْتُ ذَبِيحَةَ الْخَطِيَّةِ الْيَوْمَ هَلْ كَانَ يَحْسُنُ فِي عَيْنَيِ الرَّبِّ؟»
\par 20 فَلَمَّا سَمِعَ مُوسَى حَسُنَ فِي عَيْنَيْهِ.

\chapter{11}

\par 1 وَقَالَ الرَّبُّ لِمُوسَى وَهَارُونَ:
\par 2 «قُولا لِبَنِي اسْرَائِيلَ: هَذِهِ هِيَ الْحَيَوَانَاتُ الَّتِي تَاكُلُونَهَا مِنْ جَمِيعِ الْبَهَائِمِ الَّتِي عَلَى الارْضِ:
\par 3 كُلُّ مَا شَقَّ ظِلْفا وَقَسَمَهُ ظِلْفَيْنِ وَيَجْتَرُّ مِنَ الْبَهَائِمِ فَايَّاهُ تَاكُلُونَ.
\par 4 الَّا هَذِهِ فَلا تَاكُلُوهَا مِمَّا يَجْتَرُّ وَمِمَّا يَشُقُّ الظِّلْفَ: الْجَمَلَ لانَّهُ يَجْتَرُّ لَكِنَّهُ لا يَشُقُّ ظِلْفا فَهُوَ نَجِسٌ لَكُمْ.
\par 5 وَالْوَبْرَ لانَّهُ يَجْتَرُّ لَكِنَّهُ لا يَشُقُّ ظِلْفا فَهُوَ نَجِسٌ لَكُمْ.
\par 6 وَالارْنَبَ لانَّهُ يَجْتَرُّ لَكِنَّهُ لا يَشُقُّ ظِلْفا فَهُوَ نَجِسٌ لَكُمْ.
\par 7 وَالْخِنْزِيرَ لانَّهُ يَشُقُّ ظِلْفا وَيَقْسِمُهُ ظِلْفَيْنِ لَكِنَّهُ لا يَجْتَرُّ فَهُوَ نَجِسٌ لَكُمْ.
\par 8 مِنْ لَحْمِهَا لا تَاكُلُوا وَجُثَثَهَا لا تَلْمِسُوا. انَّهَا نَجِسَةٌ لَكُمْ.
\par 9 «وَهَذَا تَاكُلُونَهُ مِنْ جَمِيعِ مَا فِي الْمِيَاهِ: كُلُّ مَا لَهُ زَعَانِفُ وَحَرْشَفٌ فِي الْمِيَاهِ فِي الْبِحَارِ وَفِي الانْهَارِ فَايَّاهُ تَاكُلُونَ.
\par 10 لَكِنْ كُلُّ مَا لَيْسَ لَهُ زَعَانِفُ وَحَرْشَفٌ فِي الْبِحَارِ وَفِي الانْهَارِ مِنْ كُلِّ دَبِيبٍ فِي الْمِيَاهِ وَمِنْ كُلِّ نَفْسٍ حَيَّةٍ فِي الْمِيَاهِ فَهُوَ مَكْرُوهٌ لَكُمْ
\par 11 وَمَكْرُوها يَكُونُ لَكُمْ. مِنْ لَحْمِهِ لا تَاكُلُوا وَجُثَّتَهُ تَكْرَهُونَ.
\par 12 كُلُّ مَا لَيْسَ لَهُ زَعَانِفُ وَحَرْشَفٌ فِي الْمِيَاهِ فَهُوَ مَكْرُوهٌ لَكُمْ.
\par 13 «وَهَذِهِ تَكْرَهُونَهَا مِنَ الطُّيُورِ. لا تُؤْكَلْ. انَّهَا مَكْرُوهَةٌ: النَّسْرُ وَالانُوقُ وَالْعُقَابُ
\par 14 وَالْحِدَاةُ وَالْبَاشِقُ عَلَى اجْنَاسِهِ
\par 15 وَكُلُّ غُرَابٍ عَلَى اجْنَاسِهِ
\par 16 وَالنَّعَامَةُ وَالظَّلِيمُ وَالسَّافُ وَالْبَازُ عَلَى اجْنَاسِهِ
\par 17 وَالْبُومُ وَالْغَوَّاصُ وَالْكُرْكِيُّ
\par 18 وَالْبَجَعُ وَالْقُوقُ وَالرَّخَمُ
\par 19 وَاللَّقْلَقُ وَالْبَبْغَاءَ عَلَى اجْنَاسِهِ وَالْهُدْهُدُ وَالْخُفَّاشُ
\par 20 وَكُلُّ دَبِيبِ الطَّيْرِ الْمَاشِي عَلَى ارْبَعٍ. فَهُوَ مَكْرُوهٌ لَكُمْ.
\par 21 الَّا هَذَا تَاكُلُونَهُ مِنْ جَمِيعِ دَبِيبِ الطَّيْرِ الْمَاشِي عَلَى ارْبَعٍ: مَا لَهُ كُرَاعَانِ فَوْقَ رِجْلَيْهِ يَثِبُ بِهِمَا عَلَى الارْضِ.
\par 22 هَذَا مِنْهُ تَاكُلُونَ. الْجَرَادُ عَلَى اجْنَاسِهِ وَالدَّبَا عَلَى اجْنَاسِهِ وَالْحَرْجُوانُ عَلَى اجْنَاسِهِ وَالْجُنْدُبُ عَلَى اجْنَاسِهِ.
\par 23 لَكِنْ سَائِرُ دَبِيبِ الطَّيْرِ الَّذِي لَهُ ارْبَعُ ارْجُلٍ فَهُوَ مَكْرُوهٌ لَكُمْ.
\par 24 مِنْ هَذِهِ تَتَنَجَّسُونَ. كُلُّ مَنْ مَسَّ جُثَثَهَا يَكُونُ نَجِسا الَى الْمَسَاءِ
\par 25 وَكُلُّ مَنْ حَمَلَ مِنْ جُثَثِهَا يَغْسِلُ ثِيَابَهُ وَيَكُونُ نَجِسا الَى الْمَسَاءِ.
\par 26 وَجَمِيعُ الْبَهَائِمِ الَّتِي لَهَا ظِلْفٌ وَلَكِنْ لا تَشُقُّهُ شَقّا اوْ لا تَجْتَرُّ فَهِيَ نَجِسَةٌ لَكُمْ. كُلُّ مَنْ مَسَّهَا يَكُونُ نَجِسا.
\par 27 وَكُلُّ مَا يَمْشِي عَلَى كُفُوفِهِ مِنْ جَمِيعِ الْحَيَوَانَاتِ الْمَاشِيَةِ عَلَى ارْبَعٍ فَهُوَ نَجِسٌ لَكُمْ. كُلُّ مَنْ مَسَّ جُثَثَهَا يَكُونُ نَجِسا الَى الْمَسَاءِ.
\par 28 وَمَنْ حَمَلَ جُثَثَهَا يَغْسِلُ ثِيَابَهُ وَيَكُونُ نَجِسا الَى الْمَسَاءِ. انَّهَا نَجِسَةٌ لَكُمْ.
\par 29 «وَهَذَا هُوَ النَّجِسُ لَكُمْ مِنَ الدَّبِيبِ الَّذِي يَدِبُّ عَلَى الارْضِ: ابْنُ عِرْسٍ وَالْفَارُ وَالضَّبُّ عَلَى اجْنَاسِهِ
\par 30 وَالْحِرْذَوْنُ وَالْوَرَلُ وَالْوَزَغَةُ وَالْعِظَايَةُ وَالْحِرْبَاءُ.
\par 31 هَذِهِ هِيَ النَّجِسَةُ لَكُمْ مِنْ كُلِّ الدَّبِيبِ. كُلُّ مَنْ مَسَّهَا بَعْدَ مَوْتِهَا يَكُونُ نَجِسا الَى الْمَسَاءِ
\par 32 وَكُلُّ مَا وَقَعَ عَلَيْهِ وَاحِدٌ مِنْهَا بَعْدَ مَوْتِهَا يَكُونُ نَجِسا. مِنْ كُلِّ مَتَاعِ خَشَبٍ اوْ ثَوْبٍ اوْ جِلْدٍ اوْ بَلاسٍ. كُلُّ مَتَاعٍ يُعْمَلُ بِهِ عَمَلٌ يُلْقَى فِي الْمَاءِ وَيَكُونُ نَجِسا الَى الْمَسَاءِ ثُمَّ يَطْهُرُ.
\par 33 وَكُلُّ مَتَاعِ خَزَفٍ وَقَعَ فِيهِ مِنْهَا فَكُلُّ مَا فِيهِ يَتَنَجَّسُ وَامَّا هُوَ فَتَكْسِرُونَهُ.
\par 34 مَا يَاتِي عَلَيْهِ مَاءٌ مِنْ كُلِّ طَعَامٍ يُؤْكَلُ يَكُونُ نَجِسا. وَكُلُّ شَرَابٍ يُشْرَبُ فِي كُلِّ مَتَاعٍ يَكُونُ نَجِسا.
\par 35 وَكُلُّ مَا وَقَعَ عَلَيْهِ وَاحِدَةٌ مِنْ جُثَثِهَا يَكُونُ نَجِسا. التَّنُّورُ وَالْمَوْقِدَةُ يُهْدَمَانِ. انَّهَا نَجِسَةٌ وَتَكُونُ نَجِسَةً لَكُمْ.
\par 36 الَّا الْعَيْنَ وَالْبِئْرَ مُجْتَمَعَيِ الْمَاءِ تَكُونَانِ طَاهِرَتَيْنِ. لَكِنْ مَا مَسَّ جُثَثَهَا يَكُونُ نَجِسا.
\par 37 وَاذَا وَقَعَتْ وَاحِدَةٌ مِنْ جُثَثِهَا عَلَى شَيْءٍ مِنْ بِزْرِ زَرْعٍ يُزْرَعُ فَهُوَ طَاهِرٌ.
\par 38 لَكِنْ اذَا جُعِلَ مَاءٌ عَلَى بِزْرٍ فَوَقَعَ عَلَيْهِ وَاحِدَةٌ مِنْ جُثَثِهَا فَانَّهُ نَجِسٌ لَكُمْ.
\par 39 وَاذَا مَاتَ وَاحِدٌ مِنَ الْبَهَائِمِ الَّتِي هِيَ طَعَامٌ لَكُمْ فَمَنْ مَسَّ جُثَّتَهُ يَكُونُ نَجِسا الَى الْمَسَاءِ.
\par 40 وَمَنْ اكَلَ مِنْ جُثَّتِهِ يَغْسِلُ ثِيَابَهُ وَيَكُونُ نَجِسا الَى الْمَسَاءِ. وَمَنْ حَمَلَ جُثَّتَهُ يَغْسِلُ ثِيَابَهُ وَيَكُونُ نَجِسا الَى الْمَسَاءِ.
\par 41 «وَكُلُّ دَبِيبٍ يَدِبُّ عَلَى الارْضِ فَهُوَ مَكْرُوهٌ لا يُؤْكَلُ.
\par 42 كُلُّ مَا يَمْشِي عَلَى بَطْنِهِ وَكُلُّ مَا يَمْشِي عَلَى ارْبَعٍ مَعَ كُلِّ مَا كَثُرَتْ ارْجُلُهُ مِنْ كُلِّ دَبِيبٍ يَدِبُّ عَلَى الارْضِ لا تَاكُلُوهُ لانَّهُ مَكْرُوهٌ.
\par 43 لا تُدَنِّسُوا انْفُسَكُمْ بِدَبِيبٍ يَدِبُّ وَلا تَتَنَجَّسُوا بِهِ وَلا تَكُونُوا بِهِ نَجِسِينَ.
\par 44 انِّي انَا الرَّبُّ الَهُكُمْ فَتَتَقَدَّسُونَ وَتَكُونُونَ قِدِّيسِينَ لانِّي انَا قُدُّوسٌ. وَلا تُنَجِّسُوا انْفُسَكُمْ بِدَبِيبٍ يَدِبُّ عَلَى الارْضِ.
\par 45 انِّي انَا الرَّبُّ الَّذِي اصْعَدَكُمْ مِنْ ارْضِ مِصْرَ لِيَكُونَ لَكُمْ الَها. فَتَكُونُونَ قِدِّيسِينَ لانِّي انَا قُدُّوسٌ».
\par 46 هَذِهِ شَرِيعَةُ الْبَهَائِمِ وَالطُّيُورِ وَكُلِّ نَفْسٍ حَيَّةٍ تَسْعَى فِي الْمَاءِ وَكُلِّ نَفْسٍ تَدِبُّ عَلَى الارْضِ
\par 47 لِلتَّمْيِيزِ بَيْنَ النَّجِسِ وَالطَّاهِرِ وَبَيْنَ الْحَيَوَانَاتِ الَّتِي تُؤْكَلُ وَالْحَيَوَانَاتِ الَّتِي لا تُؤْكَلُ.

\chapter{12}

\par 1 وَقَالَ الرَّبُّ لِمُوسَى:
\par 2 «قُلْ لِبَنِي اسْرَائِيلَ: اذَا حَبِلَتِ امْرَاةٌ وَوَلَدَتْ ذَكَرا تَكُونُ نَجِسَةً سَبْعَةَ ايَّامٍ. كَمَا فِي ايَّامِ طَمْثِ عِلَّتِهَا تَكُونُ نَجِسَةً.
\par 3 وَفِي الْيَوْمِ الثَّامِنِ يُخْتَنُ لَحْمُ غُرْلَتِهِ.
\par 4 ثُمَّ تُقِيمُ ثَلاثَةً وَثَلاثِينَ يَوْما فِي دَمِ تَطْهِيرِهَا. كُلَّ شَيْءٍ مُقَدَّسٍ لا تَمَسَّ وَالَى الْمَقْدِسِ لا تَجِئْ حَتَّى تَكْمُلَ ايَّامُ تَطْهِيرِهَا.
\par 5 وَانْ وَلَدَتْ انْثَى تَكُونُ نَجِسَةً اسْبُوعَيْنِ كَمَا فِي طَمْثِهَا. ثُمَّ تُقِيمُ سِتَّةً وَسِتِّينَ يَوْما فِي دَمِ تَطْهِيرِهَا.
\par 6 وَمَتَى كَمِلَتْ ايَّامُ تَطْهِيرِهَا لاجْلِ ابْنٍ اوِ ابْنَةٍ تَاتِي بِخَرُوفٍ حَوْلِيٍّ مُحْرَقَةً وَفَرْخِ حَمَامَةٍ اوْ يَمَامَةٍ ذَبِيحَةَ خَطِيَّةٍ الَى بَابِ خَيْمَةِ الاجْتِمَاعِ الَى الْكَاهِنِ
\par 7 فَيُقَدِّمُهُمَا امَامَ الرَّبِّ وَيُكَفِّرُ عَنْهَا فَتَطْهَرُ مِنْ يَنْبُوعِ دَمِهَا. هَذِهِ شَرِيعَةُ الَّتِي تَلِدُ ذَكَرا اوْ انْثَى.
\par 8 وَانْ لَمْ تَنَلْ يَدُهَا كِفَايَةً لِشَاةٍ تَاخُذُ يَمَامَتَيْنِ اوْ فَرْخَيْ حَمَامٍ الْوَاحِدَ مُحْرَقَةً وَالْاخَرَ ذَبِيحَةَ خَطِيَّةٍ فَيُكَفِّرُ عَنْهَا الْكَاهِنُ فَتَطْهُرُ».

\chapter{13}

\par 1 وَقَالَ الرَّبُّ لِمُوسَى وَهَارُونَ:
\par 2 «اذَا كَانَ انْسَانٌ فِي جِلْدِ جَسَدِهِ نَاتِئٌ اوْ قُوبَاءُ اوْ لُمْعَةٌ تَصِيرُ فِي جِلْدِ جَسَدِهِ ضَرْبَةَ بَرَصٍ يُؤْتَى بِهِ الَى هَارُونَ الْكَاهِنِ اوْ الَى احَدِ بَنِيهِ الْكَهَنَةِ.
\par 3 فَانْ رَاى الْكَاهِنُ الضَّرْبَةَ فِي جِلْدِ الْجَسَدِ وَفِي الضَّرْبَةِ شَعَرٌ قَدِ ابْيَضَّ وَمَنْظَرُ الضَّرْبَةِ اعْمَقُ مِنْ جِلْدِ جَسَدِهِ فَهِيَ ضَرْبَةُ بَرَصٍ. فَمَتَى رَاهُ الْكَاهِنُ يَحْكُمُ بِنَجَاسَتِهِ.
\par 4 لَكِنْ انْ كَانَتِ الضَّرْبَةُ لُمْعَةً بَيْضَاءَ فِي جِلْدِ جَسَدِهِ وَلَمْ يَكُنْ مَنْظَرُهَا اعْمَقَ مِنَ الْجِلْدِ وَلَمْ يَبْيَضَّ شَعْرُهَا يَحْجِزُ الْكَاهِنُ الْمَضْرُوبَ سَبْعَةَ ايَّامٍ.
\par 5 فَانْ رَاهُ الْكَاهِنُ فِي الْيَوْمِ السَّابِعِ وَاذَا فِي عَيْنِهِ الضَّرْبَةُ قَدْ وَقَفَتْ وَلَمْ تَمْتَدَّ الضَّرْبَةُ فِي الْجِلْدِ يَحْجِزُهُ الْكَاهِنُ سَبْعَةَ ايَّامٍ ثَانِيَةً.
\par 6 فَانْ رَاهُ الْكَاهِنُ فِي الْيَوْمِ السَّابِعِ ثَانِيَةً وَاذَا الضَّرْبَةُ كَامِدَةُ اللَّوْنِ وَلَمْ تَمْتَدَّ الضَّرْبَةُ فِي الْجِلْدِ يَحْكُمُ الْكَاهِنُ بِطَهَارَتِهِ. انَّهَا حِزَازٌ. فَيَغْسِلُ ثِيَابَهُ وَيَكُونُ طَاهِرا.
\par 7 لَكِنْ انْ كَانَتِ الْقُوبَاءُ تَمْتَدُّ فِي الْجِلْدِ بَعْدَ عَرْضِهِ عَلَى الْكَاهِنِ لِتَطْهِيرِهِ يُعْرَضُ عَلَى الْكَاهِنِ ثَانِيَةً.
\par 8 فَانْ رَاى الْكَاهِنُ وَاذَا الْقُوبَاءُ قَدِ امْتَدَّتْ فِي الْجِلْدِ يَحْكُمُ الْكَاهِنُ بِنَجَاسَتِهِ. انَّهَا بَرَصٌ.
\par 9 «انْ كَانَتْ فِي انْسَانٍ ضَرْبَةُ بَرَصٍ فَيُؤْتَى بِهِ الَى الْكَاهِنِ.
\par 10 فَانْ رَاى الْكَاهِنُ وَاذَا فِي الْجِلْدِ نَاتِئٌ ابْيَضُ قَدْ صَيَّرَ الشَّعْرَ ابْيَضَ وَفِي النَّاتِئِ وَضَحٌ مِنْ لَحْمٍ حَيٍّ
\par 11 فَهُوَ بَرَصٌ مُزْمِنٌ فِي جِلْدِ جَسَدِهِ. فَيَحْكُمُ الْكَاهِنُ بِنَجَاسَتِهِ. لا يَحْجِزُهُ لانَّهُ نَجِسٌ.
\par 12 لَكِنْ انْ كَانَ الْبَرَصُ قَدْ افْرَخَ فِي الْجِلْدِ وَغَطَّى الْبَرَصُ كُلَّ جِلْدِ الْمَضْرُوبِ مِنْ رَاسِهِ الَى قَدَمَيْهِ حَسَبَ كُلِّ مَا تَرَاهُ عَيْنَا الْكَاهِنِ
\par 13 وَرَاى الْكَاهِنُ وَاذَا الْبَرَصُ قَدْ غَطَّى كُلَّ جِسْمِهِ يَحْكُمُ بِطَهَارَةِ الْمَضْرُوبِ. كُلُّهُ قَدِ ابْيَضَّ. انَّهُ طَاهِرٌ.
\par 14 لَكِنْ يَوْمَ يُرَى فِيهِ لَحْمٌ حَيٌّ يَكُونُ نَجِسا.
\par 15 فَمَتَى رَاى الْكَاهِنُ اللَّحْمَ الْحَيَّ يَحْكُمُ بِنَجَاسَتِهِ. اللَّحْمُ الْحَيُّ نَجِسٌ. انَّهُ بَرَصٌ.
\par 16 ثُمَّ انْ عَادَ اللَّحْمُ الْحَيُّ وَابْيَضَّ يَاتِي الَى الْكَاهِنِ.
\par 17 فَانْ رَاهُ الْكَاهِنُ وَاذَا الضَّرْبَةُ قَدْ صَارَتْ بَيْضَاءَ يَحْكُمُ الْكَاهِنُ بِطَهَارَةِ الْمَضْرُوبِ. انَّهُ طَاهِرٌ.
\par 18 «وَاذَا كَانَ الْجِسْمُ فِي جِلْدِهِ دُمَّلَةٌ قَدْ بَرِئَتْ
\par 19 وَصَارَ فِي مَوْضِعِ الدُّمَّلَةِ نَاتِئٌ ابْيَضُ اوْ لُمْعَةٌ بَيْضَاءُ ضَارِبَةٌ الَى الْحُمْرَةِ يُعْرَضُ عَلَى الْكَاهِنِ.
\par 20 فَانْ رَاى الْكَاهِنُ وَاذَا مَنْظَرُهَا اعْمَقُ مِنَ الْجِلْدِ وَقَدِ ابْيَضَّ شَعْرُهَا يَحْكُمُ الْكَاهِنُ بِنَجَاسَتِهِ. انَّهَا ضَرْبَةُ بَرَصٍ افْرَخَتْ فِي الدُّمَّلَةِ.
\par 21 لَكِنْ انْ رَاهَا الْكَاهِنُ وَاذَا لَيْسَ فِيهَا شَعْرٌ ابْيَضُ وَلَيْسَتْ اعْمَقَ مِنَ الْجِلْدِ وَهِيَ كَامِدَةُ اللَّوْنِ يَحْجِزُهُ الْكَاهِنُ سَبْعَةَ ايَّامٍ.
\par 22 فَانْ كَانَتْ قَدِ امْتَدَّتْ فِي الْجِلْدِ يَحْكُمُ الْكَاهِنُ بِنَجَاسَتِهِ. انَّهَا ضَرْبَةٌ.
\par 23 لَكِنْ انْ وَقَفَتِ اللُّمْعَةُ مَكَانَهَا وَلَمْ تَمْتَدَّ فَهِيَ اثَرُ الدُّمَّلَةِ. فَيَحْكُمُ الْكَاهِنُ بِطَهَارَتِهِ.
\par 24 «اوْ اذَا كَانَ الْجِسْمُ فِي جِلْدِهِ كَيُّ نَارٍ وَكَانَ حَيُّ الْكَيِّ لُمْعَةً بَيْضَاءَ ضَارِبَةً الَى الْحُمْرَةِ اوْ بَيْضَاءَ
\par 25 وَرَاهَا الْكَاهِنُ وَاذَا الشَّعْرُ فِي اللُّمْعَةِ قَدِ ابْيَضَّ وَمَنْظَرُهَا اعْمَقُ مِنَ الْجِلْدِ فَهِيَ بَرَصٌ قَدْ افْرَخَ فِي الْكَيِّ. فَيَحْكُمُ الْكَاهِنُ بِنَجَاسَتِهِ. انَّهَا ضَرْبَةُ بَرَصٍ.
\par 26 لَكِنْ انْ رَاهَا الْكَاهِنُ وَاذَا لَيْسَ فِي اللُّمْعَةِ شَعْرٌ ابْيَضُ وَلَيْسَتْ اعْمَقَ مِنَ الْجِلْدِ وَهِيَ كَامِدَةُ اللَّوْنِ يَحْجِزُهُ الْكَاهِنُ سَبْعَةَ ايَّامٍ
\par 27 ثُمَّ يَرَاهُ الْكَاهِنُ فِي الْيَوْمِ السَّابِعِ. فَانْ كَانَتْ قَدِ امْتَدَّتْ فِي الْجِلْدِ يَحْكُمُ الْكَاهِنُ بِنَجَاسَتِهِ. انَّهَا ضَرْبَةُ بَرَصٍ.
\par 28 لَكِنْ انْ وَقَفَتِ اللُّمْعَةُ مَكَانَهَا لَمْ تَمْتَدَّ فِي الْجِلْدِ وَكَانَتْ كَامِدَةَ اللَّوْنِ فَهِيَ نَاتِئُ الْكَيِّ فَالْكَاهِنُ يَحْكُمُ بِطَهَارَتِهِ لانَّهَا اثَرُ الْكَيِّ.
\par 29 «وَاذَا كَانَ رَجُلٌ اوِ امْرَاةٌ فِيهِ ضَرْبَةٌ فِي الرَّاسِ اوْ فِي الذَّقَنِ
\par 30 وَرَاى الْكَاهِنُ الضَّرْبَةَ وَاذَا مَنْظَرُهَا اعْمَقُ مِنَ الْجِلْدِ وَفِيهَا شَعْرٌ اشْقَرُ دَقِيقٌ يَحْكُمُ الْكَاهِنُ بِنَجَاسَتِهِ. انَّهَا قَرَعٌ. بَرَصُ الرَّاسِ اوِ الذَّقَنِ.
\par 31 لَكِنْ اذَا رَاى الْكَاهِنُ ضَرْبَةَ الْقَرَعِ وَاذَا مَنْظَرُهَا لَيْسَ اعْمَقَ مِنَ الْجِلْدِ لَكِنْ لَيْسَ فِيهَا شَعْرٌ اسْوَدُ يَحْجِزُ الْكَاهِنُ الْمَضْرُوبَ بِالْقَرَعِ سَبْعَةَ ايَّامٍ.
\par 32 فَانْ رَاى الْكَاهِنُ الضَّرْبَةَ فِي الْيَوْمِ السَّابِعِ وَاذَا الْقَرَعُ لَمْ يَمْتَدَّ وَلَمْ يَكُنْ فِيهِ شَعْرٌ اشْقَرُ وَلا مَنْظَرُ الْقَرَعِ اعْمَقُ مِنَ الْجِلْدِ
\par 33 فَلْيَحْلِقْ. لَكِنْ لا يَحْلِقِ الْقَرَعَ. وَيَحْجِزُ الْكَاهِنُ الاقْرَعَ سَبْعَةَ ايَّامٍ ثَانِيَةً.
\par 34 فَانْ رَاى الْكَاهِنُ الاقْرَعَ فِي الْيَوْمِ السَّابِعِ وَاذَا الْقَرَعُ لَمْ يَمْتَدَّ فِي الْجِلْدِ وَلَيْسَ مَنْظَرُهُ اعْمَقَ مِنَ الْجِلْدِ يَحْكُمُ الْكَاهِنُ بِطَهَارَتِهِ فَيَغْسِلُ ثِيَابَهُ وَيَكُونُ طَاهِرا.
\par 35 لَكِنْ انْ كَانَ الْقَرَعُ يَمْتَدُّ فِي الْجِلْدِ بَعْدَ الْحُكْمِ بِطَهَارَتِهِ
\par 36 وَرَاهُ الْكَاهِنُ وَاذَا الْقَرَعُ قَدِ امْتَدَّ فِي الْجِلْدِ فَلا يُفَتِّشُ الْكَاهِنُ عَلَى الشَّعْرِ الاشْقَرِ. انَّهُ نَجِسٌ.
\par 37 لَكِنْ انْ وَقَفَ فِي عَيْنَيْهِ وَنَبَتَ فِيهِ شَعْرٌ اسْوَدُ فَقَدْ بَرِئَ الْقَرَعُ. انَّهُ طَاهِرٌ فَيَحْكُمُ الْكَاهِنُ بِطَهَارَتِهِ.
\par 38 «وَاذَا كَانَ رَجُلٌ اوِ امْرَاةٌ فِي جِلْدِ جَسَدِهِ لُمَعٌ لُمَعٌ بِيضٌ
\par 39 وَرَاى الْكَاهِنُ وَاذَا فِي جِلْدِ جَسَدِهِ لُمَعٌ كَامِدَةُ اللَّوْنِ بَيْضَاءُ فَذَلِكَ بَهَقٌ قَدْ افْرَخَ فِي الْجِلْدِ. انَّهُ طَاهِرٌ.
\par 40 «وَاذَا كَانَ انْسَانٌ قَدْ ذَهَبَ شَعْرُ رَاسِهِ فَهُوَ اقْرَعُ. انَّهُ طَاهِرٌ.
\par 41 وَانْ ذَهَبَ شَعْرُ رَاسِهِ مِنْ جِهَةِ وَجْهِهِ فَهُوَ اصْلَعُ. انَّهُ طَاهِرٌ.
\par 42 لَكِنْ اذَا كَانَ فِي الْقَرَعَةِ اوْ فِي الصَّلْعَةِ ضَرْبَةٌ بَيْضَاءُ ضَارِبَةٌ الَى الْحُمْرَةِ فَهُوَ بَرَصٌ مُفْرِخٌ فِي قَرَعَتِهِ اوْ فِي صَلْعَتِهِ.
\par 43 فَانْ رَاهُ الْكَاهِنُ وَاذَا نَاتِئُ الضَّرْبَةِ ابْيَضُ ضَارِبٌ الَى الْحُمْرَةِ فِي قَرَعَتِهِ اوْ فِي صَلْعَتِهِ كَمَنْظَرِ الْبَرَصِ فِي جِلْدِ الْجَسَدِ
\par 44 فَهُوَ انْسَانٌ ابْرَصُ. انَّهُ نَجِسٌ. فَيَحْكُمُ الْكَاهِنُ بِنَجَاسَتِهِ. انَّ ضَرْبَتَهُ فِي رَاسِهِ.
\par 45 وَالابْرَصُ الَّذِي فِيهِ الضَّرْبَةُ تَكُونُ ثِيَابُهُ مَشْقُوقَةً وَرَاسُهُ يَكُونُ مَكْشُوفا وَيُغَطِّي شَارِبَيْهِ وَيُنَادِي: نَجِسٌ نَجِسٌ.
\par 46 كُلَّ الايَّامِ الَّتِي تَكُونُ الضَّرْبَةُ فِيهِ يَكُونُ نَجِسا. انَّهُ نَجِسٌ. يُقِيمُ وَحْدَهُ. خَارِجَ الْمَحَلَّةِ يَكُونُ مَقَامُهُ.
\par 47 «وَامَّا الثَّوْبُ فَاذَا كَانَ فِيهِ ضَرْبَةُ بَرَصٍ ثَوْبُ صُوفٍ اوْ ثَوْبُ كَتَّانٍ
\par 48 فِي السَّدَى اوِ اللُّحْمَةِ مِنَ الصُّوفِ اوِ الْكَتَّانِ اوْ فِي جِلْدٍ اوْ فِي كُلِّ مَصْنُوعٍ مِنْ جِلْدٍ
\par 49 وَكَانَتِ الضَّرْبَةُ ضَارِبَةً الَى الْخُضْرَةِ اوْ الَى الْحُمْرَةِ فِي الثَّوْبِ اوْ فِي الْجِلْدِ فِي السَّدَى اوِ اللُّحْمَةِ اوْ فِي مَتَاعٍ مَا مِنْ جِلْدٍ فَانَّهَا ضَرْبَةُ بَرَصٍ فَتُعْرَضُ عَلَى الْكَاهِنِ.
\par 50 فَيَرَى الْكَاهِنُ الضَّرْبَةَ وَيَحْجِزُ الْمَضْرُوبَ سَبْعَةَ ايَّامٍ.
\par 51 فَمَتَى رَاى الضَّرْبَةَ فِي الْيَوْمِ السَّابِعِ اذَا كَانَتِ الضَّرْبَةُ قَدِ امْتَدَّتْ فِي الثَّوْبِ فِي السَّدَى اوِ اللُّحْمَةِ اوْ فِي الْجِلْدِ مِنْ كُلِّ مَا يُصْنَعُ مِنْ جِلْدٍ لِلْعَمَلِ فَالضَّرْبَةُ بَرَصٌ مُفْسِدٌ. انَّهَا نَجِسَةٌ.
\par 52 فَيُحْرِقُ الثَّوْبَ اوِ السَّدَى اوِ اللُّحْمَةَ مِنَ الصُّوفِ اوِ الْكَتَّانِ اوْ مَتَاعِ الْجِلْدِ الَّذِي كَانَتْ فِيهِ الضَّرْبَةُ لانَّهَا بَرَصٌ مُفْسِدٌ. بِالنَّارِ يُحْرَقُ.
\par 53 لَكِنْ انْ رَاى الْكَاهِنُ وَاذَا الضَّرْبَةُ لَمْ تَمْتَدَّ فِي الثَّوْبِ فِي السَّدَى اوِ اللُّحْمَةِ اوْ فِي مَتَاعِ الْجِلْدِ
\par 54 يَامُرُ الْكَاهِنُ انْ يَغْسِلُوا مَا فِيهِ الضَّرْبَةُ وَيَحْجِزُهُ سَبْعَةَ ايَّامٍ ثَانِيَةً.
\par 55 فَانْ رَاى الْكَاهِنُ بَعْدَ غَسْلِ الْمَضْروبِ وَاذَا الضَّرْبَةُ لَمْ تُغَيِّرْ مَنْظَرَهَا وَلا امْتَدَّتِ الضَّرْبَةُ فَهُوَ نَجِسٌ. بِالنَّارِ تُحْرِقُهُ. انَّهَا نُخْرُوبٌ فِي جُرْدَةِ بَاطِنِهِ اوْ ظَاهِرِهِ.
\par 56 لَكِنْ انْ رَاى الْكَاهِنُ وَاذَا الضَّرْبَةُ كَامِدَةُ اللَّوْنِ بَعْدَ غَسْلِهِ يُمَزِّقُهَا مِنَ الثَّوْبِ اوِ الْجِلْدِ مِنَ السَّدَى اوِ اللُّحْمَةِ.
\par 57 ثُمَّ انْ ظَهَرَتْ ايْضا فِي الثَّوْبِ فِي السَّدَى اوِ اللُّحْمَةِ اوْ فِي مَتَاعِ الْجِلْدِ فَهِيَ مُفْرِخَةٌ. بِالنَّارِ تُحْرِقُ مَا فِيهِ الضَّرْبَةُ.
\par 58 وَامَّا الثَّوْبُ السَّدَى اوِ اللُّحْمَةُ اوْ مَتَاعُ الْجِلْدِ الَّذِي تَغْسِلُهُ وَتَزُولُ مِنْهُ الضَّرْبَةُ فَيُغْسَلُ ثَانِيَةً فَيَطْهُرُ.
\par 59 «هَذِهِ شَرِيعَةُ ضَرْبَةِ الْبَرَصِ فِي الصُّوفِ اوِ الْكَتَّانِ فِي السَّدَى اوِ اللُّحْمَةِ اوْ فِي كُلِّ مَتَاعٍ مِنْ جِلْدٍ لِلْحُكْمِ بِطَهَارَتِهِ اوْ نَجَاسَتِهِ».

\chapter{14}

\par 1 وَقَالَ الرَّبُّ لِمُوسَى:
\par 2 «هَذِهِ تَكُونُ شَرِيعَةَ الابْرَصِ: يَوْمَ طُهْرِهِ يُؤْتَى بِهِ الَى الْكَاهِنِ.
\par 3 وَيَخْرُجُ الْكَاهِنُ الَى خَارِجِ الْمَحَلَّةِ. فَانْ رَاى الْكَاهِنُ وَاذَا ضَرْبَةُ الْبَرَصِ قَدْ بَرِئَتْ مِنَ الابْرَصِ
\par 4 يَامُرُ الْكَاهِنُ انْ يُؤْخَذَ لِلْمُتَطَهِّرِ عُصْفُورَانِ حَيَّانِ طَاهِرَانِ وَخَشَبُ ارْزٍ وَقِرْمِزٌ وَزُوفَا.
\par 5 وَيَامُرُ الْكَاهِنُ انْ يُذْبَحَ الْعُصْفُورُ الْوَاحِدُ فِي انَاءِ خَزَفٍ عَلَى مَاءٍ حَيٍّ.
\par 6 امَّا الْعُصْفُورُ الْحَيُّ فَيَاخُذُهُ مَعَ خَشَبِ الارْزِ وَالْقِرْمِزِ وَالزُّوفَا وَيَغْمِسُهَا مَعَ الْعُصْفُورِ الْحَيِّ فِي دَمِ الْعُصْفُورِ الْمَذْبُوحِ عَلَى الْمَاءِ الْحَيِّ
\par 7 وَيَنْضِحُ عَلَى الْمُتَطَهِّرِ مِنَ الْبَرَصِ سَبْعَ مَرَّاتٍ فَيُطَهِّرُهُ ثُمَّ يُطْلِقُ الْعُصْفُورَ الْحَيَّ عَلَى وَجْهِ الصَّحْرَاءِ.
\par 8 فَيَغْسِلُ الْمُتَطَهِّرُ ثِيَابَهُ وَيَحْلِقُ كُلَّ شَعْرِهِ وَيَسْتَحِمُّ بِمَاءٍ فَيَطْهُرُ. ثُمَّ يَدْخُلُ الْمَحَلَّةَ لَكِنْ يُقِيمُ خَارِجَ خَيْمَتِهِ سَبْعَةَ ايَّامٍ.
\par 9 وَفِي الْيَوْمِ السَّابِعِ يَحْلِقُ كُلَّ شَعْرِهِ. رَاسَهُ وَلِحْيَتَهُ وَحَوَاجِبَ عَيْنَيْهِ وَجَمِيعَ شَعْرِهِ يَحْلِقُ. وَيَغْسِلُ ثِيَابَهُ وَيَرْحَضُ جَسَدَهُ بِمَاءٍ فَيَطْهُرُ.
\par 10 ثُمَّ فِي الْيَوْمِ الثَّامِنِ يَاخُذُ خَرُوفَيْنِ صَحِيحَيْنِ وَنَعْجَةً وَاحِدَةً حَوْلِيَّةً صَحِيحَةً وَثَلاثَةَ اعْشَارِ دَقِيقٍ تَقْدِمَةً مَلْتُوتَةً بِزَيْتٍ وَلُجَّ زَيْتٍ.
\par 11 فَيُوقِفُ الْكَاهِنُ الْمُطَهِّرُ الْانْسَانَ الْمُتَطَهِّرَ وَايَّاهَا امَامَ الرَّبِّ لَدَى بَابِ خَيْمَةِ الاجْتِمَاعِ.
\par 12 ثُمَّ يَاخُذُ الْكَاهِنُ الْخَرُوفَ الْوَاحِدَ وَيُقَرِّبُهُ ذَبِيحَةَ اثْمٍ مَعَ لُجِّ الزَّيْتِ. يُرَدِّدُهُمَا تَرْدِيدا امَامَ الرَّبِّ.
\par 13 وَيَذْبَحُ الْخَرُوفَ فِي الْمَوْضِعِ الَّذِي يَذْبَحُ فِيهِ ذَبِيحَةَ الْخَطِيَّةِ وَالْمُحْرَقَةَ فِي الْمَكَانِ الْمُقَدَّسِ لانَّ ذَبِيحَةَ الْاثْمِ كَذَبِيحَةِ الْخَطِيَّةِ لِلْكَاهِنِ. انَّهَا قُدْسُ اقْدَاسٍ.
\par 14 وَيَاخُذُ الْكَاهِنُ مِنْ دَمِ ذَبِيحَةِ الْاثْمِ وَيَجْعَلُ الْكَاهِنُ عَلَى شَحْمَةِ اذُنِ الْمُتَطَهِّرِ الْيُمْنَى وَعَلَى ابْهَامِ يَدِهِ الْيُمْنَى وَعَلَى ابْهَامِ رِجْلِهِ الْيُمْنَى.
\par 15 وَيَاخُذُ الْكَاهِنُ مِنْ لُجِّ الزَّيْتِ وَيَصُبُّ فِي كَفِّ الْكَاهِنِ الْيُسْرَى.
\par 16 وَيَغْمِسُ الْكَاهِنُ اصْبِعَهُ الْيُمْنَى فِي الزَّيْتِ الَّذِي عَلَى كَفِّهِ الْيُسْرَى وَيَنْضِحُ مِنَ الزَّيْتِ بِاصْبِعِهِ سَبْعَ مَرَّاتٍ امَامَ الرَّبِّ.
\par 17 وَمِمَّا فَضِلَ مِنَ الزَّيْتِ الَّذِي فِي كَفِّهِ يَجْعَلُ الْكَاهِنُ عَلَى شَحْمَةِ اذُنِ الْمُتَطَهِّرِ الْيُمْنَى وَعَلَى ابْهَامِ يَدِهِ الْيُمْنَى وَعَلَى ابْهَامِ رِجْلِهِ الْيُمْنَى عَلَى دَمِ ذَبِيحَةِ الْاثْمِ.
\par 18 وَالْفَاضِلُ مِنَ الزَّيْتِ الَّذِي فِي كَفِّ الْكَاهِنِ يَجْعَلُهُ عَلَى رَاسِ الْمُتَطَهِّرِ وَيُكَفِّرُ عَنْهُ الْكَاهِنُ امَامَ الرَّبِّ.
\par 19 ثُمَّ يَعْمَلُ الْكَاهِنُ ذَبِيحَةَ الْخَطِيَّةِ وَيُكَفِّرُ عَنِ الْمُتَطَهِّرِ مِنْ نَجَاسَتِهِ. ثُمَّ يَذْبَحُ الْمُحْرَقَةَ.
\par 20 وَيُصْعِدُ الْكَاهِنُ الْمُحْرَقَةَ وَالتَّقْدِمَةَ عَلَى الْمَذْبَحِ وَيُكَفِّرُ عَنْهُ الْكَاهِنُ فَيَطْهُرُ.
\par 21 «لَكِنْ انْ كَانَ فَقِيرا وَلا تَنَالُ يَدُهُ يَاخُذُ خَرُوفا وَاحِدا ذَبِيحَةَ اثْمٍ لِتَرْدِيدٍ تَكْفِيرا عَنْهُ وَعُشْرا وَاحِدا مِنْ دَقِيقٍ مَلْتُوتٍ بِزَيْتٍ لِتَقْدِمَةٍ وَلُجَّ زَيْتٍ
\par 22 وَيَمَامَتَيْنِ اوْ فَرْخَيْ حَمَامٍ كَمَا تَنَالُ يَدُهُ فَيَكُونُ الْوَاحِدُ ذَبِيحَةَ خَطِيَّةٍ وَالْاخَرُ مُحْرَقَةً.
\par 23 وَيَاتِي بِهَا فِي الْيَوْمِ الثَّامِنِ لِطُهْرِهِ الَى الْكَاهِنِ الَى بَابِ خَيْمَةِ الاجْتِمَاعِ امَامَ الرَّبِّ.
\par 24 فَيَاخُذُ الْكَاهِنُ كَبْشَ الْاثْمِ وَلُجَّ الزَّيْتِ وَيُرَدِّدُهُمَا الْكَاهِنُ تَرْدِيدا امَامَ الرَّبِّ.
\par 25 ثُمَّ يَذْبَحُ كَبْشَ الْاثْمِ وَيَاخُذُ الْكَاهِنُ مِنْ دَمِ ذَبِيحَةِ الْاثْمِ وَيَجْعَلُ عَلَى شَحْمَةِ اذُنِ الْمُتَطَهِّرِ الْيُمْنَى وَعَلَى ابْهَامِ يَدِهِ الْيُمْنَى وَعَلَى ابْهَامِ رِجْلِهِ الْيُمْنَى.
\par 26 وَيَصُبُّ الْكَاهِنُ مِنَ الزَّيْتِ فِي كَفِّ الْكَاهِنِ الْيُسْرَى
\par 27 وَيَنْضِحُ الْكَاهِنُ بِاصْبِعِهِ الْيُمْنَى مِنَ الزَّيْتِ الَّذِي فِي كَفِّهِ الْيُسْرَى سَبْعَ مَرَّاتٍ امَامَ الرَّبِّ.
\par 28 وَيَجْعَلُ الْكَاهِنُ مِنَ الزَّيْتِ الَّذِي فِي كَفِّهِ عَلَى شَحْمَةِ اذُنِ الْمُتَطَهِّرِ الْيُمْنَى وَعَلَى ابْهَامِ يَدِهِ الْيُمْنَى وَعَلَى ابْهَامِ رِجْلِهِ الْيُمْنَى عَلَى مَوْضِعِ دَمِ ذَبِيحَةِ الْاثْمِ.
\par 29 وَالْفَاضِلُ مِنَ الزَّيْتِ الَّذِي فِي كَفِّ الْكَاهِنِ يَجْعَلُهُ عَلَى رَاسِ الْمُتَطَهِّرِ تَكْفِيرا عَنْهُ امَامَ الرَّبِّ.
\par 30 ثُمَّ يَعْمَلُ وَاحِدَةً مِنَ الْيَمَامَتَيْنِ اوْ مِنْ فَرْخَيِ الْحَمَامِ مِمَّا تَنَالُ يَدُهُ
\par 31 مَا تَنَالُ يَدُهُ. الْوَاحِدَ ذَبِيحَةَ خَطِيَّةٍ وَالْاخَرَ مُحْرَقَةً مَعَ التَّقْدِمَةِ. وَيُكَفِّرُ الْكَاهِنُ عَنِ الْمُتَطَهِّرِ امَامَ الرَّبِّ.
\par 32 هَذِهِ شَرِيعَةُ الَّذِي فِيهِ ضَرْبَةُ بَرَصٍ الَّذِي لا تَنَالُ يَدُهُ فِي تَطْهِيرِهِ».
\par 33 وَقَالَ الرَّبُّ لِمُوسَى وَهَارُونَ:
\par 34 «مَتَى جِئْتُمْ الَى ارْضِ كَنْعَانَ الَّتِي اعْطِيكُمْ مُلْكا وَجَعَلْتُ ضَرْبَةَ بَرَصٍ فِي بَيْتٍ فِي ارْضِ مُلْكِكُمْ.
\par 35 يَاتِي الَّذِي لَهُ الْبَيْتُ وَيَقُولُ لِلْكَاهِنِ: قَدْ ظَهَرَ لِي شِبْهُ ضَرْبَةٍ فِي الْبَيْتِ.
\par 36 فَيَامُرُ الْكَاهِنُ انْ يُفْرِغُوا الْبَيْتَ قَبْلَ دُخُولِ الْكَاهِنِ لِيَرَى الضَّرْبَةَ لِئَلَّا يَتَنَجَّسَ كُلُّ مَا فِي الْبَيْتِ. وَبَعْدَ ذَلِكَ يَدْخُلُ الْكَاهِنُ لِيَرَى الْبَيْتَ.
\par 37 فَاذَا رَاى الضَّرْبَةَ وَاذَا الضَّرْبَةُ فِي حِيطَانِ الْبَيْتِ نُقَرٌ ضَارِبَةٌ الَى الْخُضْرَةِ اوْ الَى الْحُمْرَةِ وَمَنْظَرُهَا اعْمَقُ مِنَ الْحَائِطِ
\par 38 يَخْرُجُ الْكَاهِنُ مِنَ الْبَيْتِ الَى بَابِ الْبَيْتِ وَيُغْلِقُ الْبَيْتَ سَبْعَةَ ايَّامٍ.
\par 39 فَاذَا رَجَعَ الْكَاهِنُ فِي الْيَوْمِ السَّابِعِ وَرَاى وَاذَا الضَّرْبَةُ قَدِ امْتَدَّتْ فِي حِيطَانِ الْبَيْتِ
\par 40 يَامُرُ الْكَاهِنُ انْ يَقْلَعُوا الْحِجَارَةَ الَّتِي فِيهَا الضَّرْبَةُ وَيَطْرَحُوهَا خَارِجَ الْمَدِينَةِ فِي مَكَانٍ نَجِسٍ.
\par 41 وَيُقَشِّرُ الْبَيْتَ مِنْ دَاخِلٍ حَوَالَيْهِ وَيَطْرَحُونَ التُّرَابَ الَّذِي يُقَشِّرُونَهُ خَارِجَ الْمَدِينَةِ فِي مَكَانٍ نَجِسٍ.
\par 42 وَيَاخُذُونَ حِجَارَةً اخْرَى وَيُدْخِلُونَهَا فِي مَكَانِ الْحِجَارَةِ وَيَاخُذُ تُرَابا اخَرَ وَيُطَيِّنُ الْبَيْتَ.
\par 43 فَانْ رَجَعَتِ الضَّرْبَةُ وَافْرَخَتْ فِي الْبَيْتِ بَعْدَ قَلْعِ الْحِجَارَةِ وَقَشْرِ الْبَيْتِ وَتَطْيِينِهِ
\par 44 وَاتَى الْكَاهِنُ وَرَاى وَاذَا الضَّرْبَةُ قَدِ امْتَدَّتْ فِي الْبَيْتِ فَهِيَ بَرَصٌ مُفْسِدٌ فِي الْبَيْتِ. انَّهُ نَجِسٌ.
\par 45 فَيَهْدِمُ الْبَيْتَ: حِجَارَتَهُ وَاخْشَابَهُ وَكُلَّ تُرَابِ الْبَيْتِ وَيُخْرِجُهَا الَى خَارِجِ الْمَدِينَةِ الَى مَكَانٍ نَجِسٍ.
\par 46 وَمَنْ دَخَلَ الَى الْبَيْتِ فِي كُلِّ ايَّامِ انْغِلاقِهِ يَكُونُ نَجِسا الَى الْمَسَاءِ.
\par 47 وَمَنْ نَامَ فِي الْبَيْتِ يَغْسِلُ ثِيَابَهُ. وَمَنْ اكَلَ فِي الْبَيْتِ يَغْسِلُ ثِيَابَهُ.
\par 48 لَكِنْ انْ اتَى الْكَاهِنُ وَرَاى وَاذَا الضَّرْبَةُ لَمْ تَمْتَدَّ فِي الْبَيْتِ بَعْدَ تَطْيِينِ الْبَيْتِ يُطَهِّرُ الْكَاهِنُ الْبَيْتَ. لانَّ الضَّرْبَةَ قَدْ بَرِئَتْ.
\par 49 فَيَاخُذُ لِتَطْهِيرِ الْبَيْتِ عُصْفُورَيْنِ وَخَشَبَ ارْزٍ وَقِرْمِزا وَزُوفَا.
\par 50 وَيَذْبَحُ الْعُصْفُورَ الْوَاحِدَ فِي انَاءِ خَزَفٍ عَلَى مَاءٍ حَيٍّ
\par 51 وَيَاخُذُ خَشَبَ الارْزِ وَالزُّوفَا وَالْقِرْمِزَ وَالْعُصْفُورَ الْحَيَّ وَيَغْمِسُهَا فِي دَمِ الْعُصْفُورِ الْمَذْبُوحِ وَفِي الْمَاءِ الْحَيِّ وَيَنْضِحُ الْبَيْتَ سَبْعَ مَرَّاتٍ
\par 52 وَيُطَهِّرُ الْبَيْتَ بِدَمِ الْعُصْفُورِ وَبِالْمَاءِ الْحَيِّ وَبِالْعُصْفُورِ الْحَيِّ وَبِخَشَبِ الارْزِ وَبِالزُّوفَا وَبِالْقِرْمِزِ.
\par 53 ثُمَّ يُطْلِقُ الْعُصْفُورَ الْحَيَّ الَى خَارِجِ الْمَدِينَةِ عَلَى وَجْهِ الصَّحْرَاءِ وَيُكَفِّرُ عَنِ الْبَيْتِ فَيَطْهُرُ.
\par 54 «هَذِهِ هِيَ الشَّرِيعَةُ لِكُلِّ ضَرْبَةٍ مِنَ الْبَرَصِ وَلِلْقَرَعِ
\par 55 وَلِبَرَصِ الثَّوْبِ وَالْبَيْتِ
\par 56 وَلِلنَّاتِئِ وَلِلْقُوبَاءِ وَلِلُّمْعَةِ
\par 57 لِلتَّعْلِيمِ فِي يَوْمِ النَّجَاسَةِ وَيَوْمِ الطَّهَارَةِ. هَذِهِ شَرِيعَةُ الْبَرَصِ».

\chapter{15}

\par 1 وَقَالَ الرَّبُّ لِمُوسَى وَهَارُونَ:
\par 2 «قُولا لِبَنِي اسْرَائِيلَ: كُلُّ رَجُلٍ يَكُونُ لَهُ سَيْلٌ مِنْ لَحْمِهِ فَسَيْلُهُ نَجِسٌ.
\par 3 وَهَذِهِ تَكُونُ نَجَاسَتُهُ بِسَيْلِهِ: انْ كَانَ لَحْمُهُ يَبْصُقُ سَيْلَهُ اوْ يَحْتَبِسُ لَحْمُهُ عَنْ سَيْلِهِ فَذَلِكَ نَجَاسَتُهُ.
\par 4 كُلُّ فِرَاشٍ يَضْطَجِعُ عَلَيْهِ الَّذِي لَهُ السَّيْلُ يَكُونُ نَجِسا وَكُلُّ مَتَاعٍ يَجْلِسُ عَلَيْهِ يَكُونُ نَجِسا.
\par 5 وَمَنْ مَسَّ فِرَاشَهُ يَغْسِلُ ثِيَابَهُ وَيَسْتَحِمُّ بِمَاءٍ وَيَكُونُ نَجِسا الَى الْمَسَاءِ.
\par 6 وَمَنْ جَلَسَ عَلَى الْمَتَاعِ الَّذِي يَجْلِسُ عَلَيْهِ ذُو السَّيْلِ يَغْسِلُ ثِيَابَهُ وَيَسْتَحِمُّ بِمَاءٍ وَيَكُونُ نَجِسا الَى الْمَسَاءِ.
\par 7 وَمَنْ مَسَّ لَحْمَ ذِي السَّيْلِ يَغْسِلُ ثِيَابَهُ وَيَسْتَحِمُّ بِمَاءٍ وَيَكُونُ نَجِسا الَى الْمَسَاءِ.
\par 8 وَانْ بَصَقَ ذُو السَّيْلِ عَلَى طَاهِرٍ يَغْسِلُ ثِيَابَهُ وَيَسْتَحِمُّ بِمَاءٍ وَيَكُونُ نَجِسا الَى الْمَسَاءِ.
\par 9 وَكُلُّ مَا يَرْكَبُ عَلَيْهِ ذُو السَّيْلِ يَكُونُ نَجِسا.
\par 10 وَكُلُّ مَنْ مَسَّ كُلَّ مَا كَانَ تَحْتَهُ يَكُونُ نَجِسا الَى الْمَسَاءِ وَمَنْ حَمَلَهُنَّ يَغْسِلُ ثِيَابَهُ وَيَسْتَحِمُّ بِمَاءٍ وَيَكُونُ نَجِسا الَى الْمَسَاءِ.
\par 11 وَكُلُّ مَنْ مَسَّهُ ذُو السَّيْلِ وَلَمْ يَغْسِلْ يَدَيْهِ بِمَاءٍ يَغْسِلُ ثِيَابَهُ وَيَسْتَحِمُّ بِمَاءٍ وَيَكُونُ نَجِسا الَى الْمَسَاءِ.
\par 12 وَانَاءُ الْخَزَفِ الَّذِي يَمَسُّهُ ذُو السَّيْلِ يُكْسَرُ. وَكُلُّ انَاءِ خَشَبٍ يُغْسَلُ بِمَاءٍ.
\par 13 وَاذَا طَهُرَ ذُو السَّيْلِ مِنْ سَيْلِهِ يُحْسَبُ لَهُ سَبْعَةُ ايَّامٍ لِطُهْرِهِ وَيَغْسِلُ ثِيَابَهُ وَيَرْحَضُ جَسَدَهُ بِمَاءٍ حَيٍّ فَيَطْهُرُ.
\par 14 وَفِي الْيَوْمِ الثَّامِنِ يَاخُذُ لِنَفْسِهِ يَمَامَتَيْنِ اوْ فَرْخَيْ حَمَامٍ وَيَاتِي الَى امَامِ الرَّبِّ الَى بَابِ خَيْمَةِ الاجْتِمَاعِ وَيُعْطِيهِمَا لِلْكَاهِنِ
\par 15 فَيَعْمَلُهُمَا الْكَاهِنُ: الْوَاحِدَ ذَبِيحَةَ خَطِيَّةٍ وَالْاخَرَ مُحْرَقَةً. وَيُكَفِّرُ عَنْهُ الْكَاهِنُ امَامَ الرَّبِّ مِنْ سَيْلِهِ.
\par 16 «وَاذَا حَدَثَ مِنْ رَجُلٍ اضْطِجَاعُ زَرْعٍ يَرْحَضُ كُلَّ جَسَدِهِ بِمَاءٍ وَيَكُونُ نَجِسا الَى الْمَسَاءِ.
\par 17 وَكُلُّ ثَوْبٍ وَكُلُّ جِلْدٍ يَكُونُ عَلَيْهِ اضْطِجَاعُ زَرْعٍ يُغْسَلُ بِمَاءٍ وَيَكُونُ نَجِسا الَى الْمَسَاءِ.
\par 18 وَالْمَرْاةُ الَّتِي يَضْطَجِعُ مَعَهَا رَجُلٌ اضْطِجَاعَ زَرْعٍ يَسْتَحِمَّانِ بِمَاءٍ وَيَكُونَانِ نَجِسَيْنِ الَى الْمَسَاءِ.
\par 19 «وَاذَا كَانَتِ امْرَاةٌ لَهَا سَيْلٌ وَكَانَ سَيْلُهَا دَما فِي لَحْمِهَا فَسَبْعَةَ ايَّامٍ تَكُونُ فِي طَمْثِهَا. وَكُلُّ مَنْ مَسَّهَا يَكُونُ نَجِسا الَى الْمَسَاءِ.
\par 20 وَكُلُّ مَا تَضْطَجِعُ عَلَيْهِ فِي طَمْثِهَا يَكُونُ نَجِسا وَكُلُّ مَا تَجْلِسُ عَلَيْهِ يَكُونُ نَجِسا.
\par 21 وَكُلُّ مَنْ مَسَّ فِرَاشَهَا يَغْسِلُ ثِيَابَهُ وَيَسْتَحِمُّ بِمَاءٍ وَيَكُونُ نَجِسا الَى الْمَسَاءِ.
\par 22 وَكُلُّ مَنْ مَسَّ مَتَاعا تَجْلِسُ عَلَيْهِ يَغْسِلُ ثِيَابَهُ وَيَسْتَحِمُّ بِمَاءٍ وَيَكُونُ نَجِسا الَى الْمَسَاءِ.
\par 23 وَانْ كَانَ عَلَى الْفِرَاشِ اوْ عَلَى الْمَتَاعِ الَّذِي هِيَ جَالِسَةٌ عَلَيْهِ عِنْدَمَا يَمَسُّهُ يَكُونُ نَجِسا الَى الْمَسَاءِ.
\par 24 وَانِ اضْطَجَعَ مَعَهَا رَجُلٌ فَكَانَ طَمْثُهَا عَلَيْهِ يَكُونُ نَجِسا سَبْعَةَ ايَّامٍ. وَكُلُّ فِرَاشٍ يَضْطَجِعُ عَلَيْهِ يَكُونُ نَجِسا.
\par 25 «وَاذَا كَانَتِ امْرَاةٌ يَسِيلُ سَيْلُ دَمِهَا ايَّاما كَثِيرَةً فِي غَيْرِ وَقْتِ طَمْثِهَا اوْ اذَا سَالَ بَعْدَ طَمْثِهَا فَتَكُونُ كُلَّ ايَّامِ سَيَلانِ نَجَاسَتِهَا كَمَا فِي ايَّامِ طَمْثِهَا. انَّهَا نَجِسَةٌ.
\par 26 كُلُّ فِرَاشٍ تَضْطَجِعُ عَلَيْهِ كُلَّ ايَّامِ سَيْلِهَا يَكُونُ لَهَا كَفِرَاشِ طَمْثِهَا. وَكُلُّ الامْتِعَةِ الَّتِي تَجْلِسُ عَلَيْهَا تَكُونُ نَجِسَةً كَنَجَاسَةِ طَمْثِهَا.
\par 27 وَكُلُّ مَنْ مَسَّهُنَّ يَكُونُ نَجِسا فَيَغْسِلُ ثِيَابَهُ وَيَسْتَحِمُّ بِمَاءٍ وَيَكُونُ نَجِسا الَى الْمَسَاءِ.
\par 28 وَاذَا طَهُرَتْ مِنْ سَيْلِهَا تَحْسِبُ لِنَفْسِهَا سَبْعَةَ ايَّامٍ ثُمَّ تَطْهُرُ.
\par 29 وَفِي الْيَوْمِ الثَّامِنِ تَاخُذُ لِنَفْسِهَا يَمَامَتَيْنِ اوْ فَرْخَيْ حَمَامٍ وَتَاتِي بِهِمَا الَى الْكَاهِنِ الَى بَابِ خَيْمَةِ الاجْتِمَاعِ.
\par 30 فَيَعْمَلُ الْكَاهِنُ الْوَاحِدَ ذَبِيحَةَ خَطِيَّةٍ وَالْاخَرَ مُحْرَقَةً وَيُكَفِّرُ عَنْهَا الْكَاهِنُ امَامَ الرَّبِّ مِنْ سَيْلِ نَجَاسَتِهَا.
\par 31 فَتَعْزِلانِ بَنِي اسْرَائِيلَ عَنْ نَجَاسَتِهِمْ لِئَلَّا يَمُوتُوا فِي نَجَاسَتِهِمْ بِتَنْجِيسِهِمْ مَسْكَنِيَ الَّذِي فِي وَسَطِهِمْ.
\par 32 «هَذِهِ شَرِيعَةُ ذِي السَّيْلِ وَالَّذِي يَحْدُثُ مِنْهُ اضْطِجَاعُ زَرْعٍ فَيَتَنَجَّسُ بِهَا
\par 33 وَالْعَلِيلَةِ فِي طَمْثِهَا وَالسَّائِلِ سَيْلُهُ: الذَّكَرِ وَالانْثَى وَالرَّجُلِ الَّذِي يَضْطَجِعُ مَعَ نَجِسَةٍ».

\chapter{16}

\par 1 وَقَالَ الرَّبُّ لِمُوسَى بَعْدَ مَوْتِ ابْنَيْ هَارُونَ عِنْدَمَا اقْتَرَبَا امَامَ الرَّبِّ وَمَاتَا:
\par 2 «كَلِّمْ هَارُونَ اخَاكَ انْ لا يَدْخُلَ كُلَّ وَقْتٍ الَى الْقُدْسِ دَاخِلَ الْحِجَابِ امَامَ الْغِطَاءِ الَّذِي عَلَى التَّابُوتِ لِئَلَّا يَمُوتَ لانِّي فِي السَّحَابِ اتَرَاءَى عَلَى الْغِطَاءِ.
\par 3 بِهَذَا يَدْخُلُ هَارُونُ الَى الْقُدْسِ: بِثَوْرِ ابْنِ بَقَرٍ لِذَبِيحَةِ خَطِيَّةٍ وَكَبْشٍ لِمُحْرَقَةٍ.
\par 4 يَلْبَسُ قَمِيصَ كَتَّانٍ مُقَدَّسا وَتَكُونُ سَرَاوِيلُ كَتَّانٍ عَلَى جَسَدِهِ وَيَتَنَطَّقُ بِمِنْطَقَةِ كَتَّانٍ وَيَتَعَمَّمُ بِعِمَامَةِ كَتَّانٍ. انَّهَا ثِيَابٌ مُقَدَّسَةٌ. فَيَرْحَضُ جَسَدَهُ بِمَاءٍ وَيَلْبَسُهَا.
\par 5 وَمِنْ جَمَاعَةِ بَنِي اسْرَائِيلَ يَاخُذُ تَيْسَيْنِ مِنَ الْمَعْزِ لِذَبِيحَةِ خَطِيَّةٍ وَكَبْشا وَاحِدا لِمُحْرَقَةٍ.
\par 6 وَيُقَرِّبُ هَارُونُ ثَوْرَ الْخَطِيَّةِ الَّذِي لَهُ وَيُكَفِّرُ عَنْ نَفْسِهِ وَعَنْ بَيْتِهِ.
\par 7 وَيَاخُذُ التَّيْسَيْنِ وَيُوقِفُهُمَا امَامَ الرَّبِّ لَدَى بَابِ خَيْمَةِ الاجْتِمَاعِ.
\par 8 وَيُلْقِي هَارُونُ عَلَى التَّيْسَيْنِ قُرْعَتَيْنِ: قُرْعَةً لِلرَّبِّ وَقُرْعَةً لِعَزَازِيلَ.
\par 9 وَيُقَرِّبُ هَارُونُ التَّيْسَ الَّذِي خَرَجَتْ عَلَيْهِ الْقُرْعَةُ لِلرَّبِّ وَيَعْمَلُهُ ذَبِيحَةَ خَطِيَّةٍ.
\par 10 وَامَّا التَّيْسُ الَّذِي خَرَجَتْ عَلَيْهِ الْقُرْعَةُ لِعَزَازِيلَ فَيُوقَفُ حَيّا امَامَ الرَّبِّ لِيُكَفِّرَ عَنْهُ لِيُرْسِلَهُ الَى عَزَازِيلَ الَى الْبَرِّيَّةِ.
\par 11 «وَيُقَدِّمُ هَارُونُ ثَوْرَ الْخَطِيَّةِ الَّذِي لَهُ وَيُكَفِّرُ عَنْ نَفْسِهِ وَعَنْ بَيْتِهِ وَيَذْبَحُ ثَوْرَ الْخَطِيَّةِ الَّذِي لَهُ
\par 12 وَيَاخُذُ مِلْءَ الْمَجْمَرَةِ جَمْرَ نَارٍ عَنِ الْمَذْبَحِ مِنْ امَامِ الرَّبِّ وَمِلْءَ رَاحَتَيْهِ بَخُورا عَطِرا دَقِيقا وَيَدْخُلُ بِهِمَا الَى دَاخِلِ الْحِجَابِ
\par 13 وَيَجْعَلُ الْبَخُورَ عَلَى النَّارِ امَامَ الرَّبِّ فَتُغَشِّي سَحَابَةُ الْبَخُورِ الْغِطَاءَ الَّذِي عَلَى الشَّهَادَةِ فَلا يَمُوتُ.
\par 14 ثُمَّ يَاخُذُ مِنْ دَمِ الثَّوْرِ وَيَنْضِحُ بِاصْبِعِهِ عَلَى وَجْهِ الْغِطَاءِ الَى الشَّرْقِ. وَقُدَّامَ الْغِطَاءِ يَنْضِحُ سَبْعَ مَرَّاتٍ مِنَ الدَّمِ بِاصْبِعِهِ.
\par 15 «ثُمَّ يَذْبَحُ تَيْسَ الْخَطِيَّةِ الَّذِي لِلشَّعْبِ وَيَدْخُلُ بِدَمِهِ الَى دَاخِلِ الْحِجَابِ. وَيَفْعَلُ بِدَمِهِ كَمَا فَعَلَ بِدَمِ الثَّوْرِ: يَنْضِحُهُ عَلَى الْغِطَاءِ وَقُدَّامَ الْغِطَاءِ
\par 16 فَيُكَفِّرُ عَنِ الْقُدْسِ مِنْ نَجَاسَاتِ بَنِي اسْرَائِيلَ وَمِنْ سَيِّئَاتِهِمْ مَعَ كُلِّ خَطَايَاهُمْ. وَهَكَذَا يَفْعَلُ لِخَيْمَةِ الاجْتِمَاعِ الْقَائِمَةِ بَيْنَهُمْ فِي وَسَطِ نَجَاسَاتِهِمْ.
\par 17 وَلا يَكُنْ انْسَانٌ فِي خَيْمَةِ الاجْتِمَاعِ مِنْ دُخُولِهِ لِلتَّكْفِيرِ فِي الْقُدْسِ الَى خُرُوجِهِ. فَيُكَفِّرُ عَنْ نَفْسِهِ وَعَنْ بَيْتِهِ وَعَنْ كُلِّ جَمَاعَةِ اسْرَائِيلَ.
\par 18 ثُمَّ يَخْرُجُ الَى الْمَذْبَحِ الَّذِي امَامَ الرَّبِّ وَيُكَفِّرُ عَنْهُ. يَاخُذُ مِنْ دَمِ الثَّوْرِ وَمِنْ دَمِ التَّيْسِ وَيَجْعَلُ عَلَى قُرُونِ الْمَذْبَحِ مُسْتَدِيرا.
\par 19 وَيَنْضِحُ عَلَيْهِ مِنَ الدَّمِ بِاصْبِعِهِ سَبْعَ مَرَّاتٍ وَيُطَهِّرُهُ وَيُقَدِّسُهُ مِنْ نَجَاسَاتِ بَنِي اسْرَائِيلَ.
\par 20 «وَمَتَى فَرَغَ مِنَ التَّكْفِيرِ عَنِ الْقُدْسِ وَعَنْ خَيْمَةِ الاجْتِمَاعِ وَعَنِ الْمَذْبَحِ يُقَدِّمُ التَّيْسَ الْحَيَّ.
\par 21 وَيَضَعُ هَارُونُ يَدَيْهِ عَلَى رَاسِ التَّيْسِ الْحَيِّ وَيُقِرُّ عَلَيْهِ بِكُلِّ ذُنُوبِ بَنِي اسْرَائِيلَ وَكُلِّ سَيِّئَاتِهِمْ مَعَ كُلِّ خَطَايَاهُمْ وَيَجْعَلُهَا عَلَى رَاسِ التَّيْسِ وَيُرْسِلُهُ بِيَدِ مَنْ يُلاقِيهِ الَى الْبَرِّيَّةِ
\par 22 لِيَحْمِلَ التَّيْسُ عَلَيْهِ كُلَّ ذُنُوبِهِمْ الَى ارْضٍ مُقْفِرَةٍ فَيُطْلِقُ التَّيْسَ فِي الْبَرِّيَّةِ.
\par 23 ثُمَّ يَدْخُلُ هَارُونُ الَى خَيْمَةِ الاجْتِمَاعِ وَيَخْلَعُ ثِيَابَ الْكَتَّانِ الَّتِي لَبِسَهَا عِنْدَ دُخُولِهِ الَى الْقُدْسِ وَيَضَعُهَا هُنَاكَ.
\par 24 وَيَرْحَضُ جَسَدَهُ بِمَاءٍ فِي مَكَانٍ مُقَدَّسٍ ثُمَّ يَلْبَسُ ثِيَابَهُ وَيَخْرُجُ وَيَعْمَلُ مُحْرَقَتَهُ وَمُحْرَقَةَ الشَّعْبِ وَيُكَفِّرُ عَنْ نَفْسِهِ وَعَنِ الشَّعْبِ.
\par 25 وَشَحْمُ ذَبِيحَةِ الْخَطِيَّةِ يُوقِدُهُ عَلَى الْمَذْبَحِ.
\par 26 وَالَّذِي اطْلَقَ التَّيْسَ الَى عَزَازِيلَ يَغْسِلُ ثِيَابَهُ وَيَرْحَضُ جَسَدَهُ بِمَاءٍ وَبَعْدَ ذَلِكَ يَدْخُلُ الَى الْمَحَلَّةِ.
\par 27 وَثَوْرُ الْخَطِيَّةِ وَتَيْسُ الْخَطِيَّةِ اللَّذَانِ اتِيَ بِدَمِهِمَا لِلتَّكْفِيرِ فِي الْقُدْسِ يُخْرِجُهُمَا الَى خَارِجِ الْمَحَلَّةِ وَيُحْرِقُونَ بِالنَّارِ جِلْدَيْهِمَا وَلَحْمَهُمَا وَفَرْثَهُمَا.
\par 28 وَالَّذِي يُحْرِقُهُمَا يَغْسِلُ ثِيَابَهُ وَيَرْحَضُ جَسَدَهُ بِمَاءٍ وَبَعْدَ ذَلِكَ يَدْخُلُ الَى الْمَحَلَّةِ.
\par 29 «وَيَكُونُ لَكُمْ فَرِيضَةً دَهْرِيَّةً انَّكُمْ فِي الشَّهْرِ السَّابِعِ فِي عَاشِرِ الشَّهْرِ تُذَلِّلُونَ نُفُوسَكُمْ وَكُلَّ عَمَلٍ لا تَعْمَلُونَ: الْوَطَنِيُّ وَالْغَرِيبُ النَّازِلُ فِي وَسَطِكُمْ.
\par 30 لانَّهُ فِي هَذَا الْيَوْمِ يُكَفِّرُ عَنْكُمْ لِتَطْهِيرِكُمْ. مِنْ جَمِيعِ خَطَايَاكُمْ امَامَ الرَّبِّ تَطْهُرُونَ.
\par 31 سَبْتُ عُطْلَةٍ هُوَ لَكُمْ وَتُذَلِّلُونَ نُفُوسَكُمْ فَرِيضَةً دَهْرِيَّةً.
\par 32 وَيُكَفِّرُ الْكَاهِنُ الَّذِي يَمْسَحُهُ وَالَّذِي يَمْلَا يَدَهُ لِلْكَهَانَةِ عِوَضا عَنْ ابِيهِ. يَلْبَسُ ثِيَابَ الْكَتَّانِ الثِّيَابَ الْمُقَدَّسَةَ
\par 33 وَيُكَفِّرُ عَنْ مَقْدِسِ الْقُدْسِ. وَعَنْ خَيْمَةِ الاجْتِمَاعِ وَالْمَذْبَحِ يُكَفِّرُ. وَعَنِ الْكَهَنَةِ وَكُلِّ شَعْبِ الْجَمَاعَةِ يُكَفِّرُ.
\par 34 وَتَكُونُ هَذِهِ لَكُمْ فَرِيضَةً دَهْرِيَّةً لِلتَّكْفِيرِ عَنْ بَنِي اسْرَائِيلَ مِنْ جَمِيعِ خَطَايَاهُمْ مَرَّةً فِي السَّنَةِ». فَفَعَلَ كَمَا امَرَ الرَّبُّ مُوسَى.

\chapter{17}

\par 1 وَقَالَ الرَّبُّ لِمُوسَى:
\par 2 «قُلْ لِهَارُونَ وَبَنِيهِ وَجَمِيعِ بَنِي اسْرَائِيلَ: هَذَا هُوَ الامْرُ الَّذِي يُوصِي بِهِ الرَّبُّ:
\par 3 كُلُّ انْسَانٍ مِنْ بَيْتِ اسْرَائِيلَ يَذْبَحُ بَقَرا اوْ غَنَما اوْ مِعْزًى فِي الْمَحَلَّةِ اوْ يَذْبَحُ خَارِجَ الْمَحَلَّةِ
\par 4 وَالَى بَابِ خَيْمَةِ الاجْتِمَاعِ لا يَاتِي بِهِ لِيُقَرِّبَ قُرْبَانا لِلرَّبِّ امَامَ مَسْكَنِ الرَّبِّ يُحْسَبُ عَلَى ذَلِكَ الْانْسَانِ دَمٌ. قَدْ سَفَكَ دَما. فَيُقْطَعُ ذَلِكَ الْانْسَانُ مِنْ شَعْبِهِ.
\par 5 لِكَيْ يَاتِيَ بَنُو اسْرَائِيلَ بِذَبَائِحِهِمِ الَّتِي يَذْبَحُونَهَا عَلَى وَجْهِ الصَّحْرَاءِ وَيُقَدِّمُوهَا لِلرَّبِّ الَى بَابِ خَيْمَةِ الاجْتِمَاعِ الَى الْكَاهِنِ وَيَذْبَحُوهَا ذَبَائِحَ سَلامَةٍ لِلرَّبِّ.
\par 6 وَيَرُشُّ الْكَاهِنُ الدَّمَ عَلَى مَذْبَحِ الرَّبِّ لَدَى بَابِ خَيْمَةِ الاجْتِمَاعِ وَيُوقِدُ الشَّحْمَ لِرَائِحَةِ سُرُورٍ لِلرَّبِّ.
\par 7 وَلا يَذْبَحُوا بَعْدُ ذَبَائِحَهُمْ لِلتُّيُوسِ الَّتِي هُمْ يَزْنُونَ وَرَاءَهَا. فَرِيضَةً دَهْرِيَّةً تَكُونُ هَذِهِ لَهُمْ فِي اجْيَالِهِمْ.
\par 8 «وَتَقُولُ لَهُمْ: كُلُّ انْسَانٍ مِنْ بَيْتِ اسْرَائِيلَ وَمِنَ الْغُرَبَاءِ الَّذِينَ يَنْزِلُونَ فِي وَسَطِكُمْ يُصْعِدُ مُحْرَقَةً اوْ ذَبِيحَةً
\par 9 وَلا يَاتِي بِهَا الَى بَابِ خَيْمَةِ الاجْتِمَاعِ لِيَصْنَعَهَا لِلرَّبِّ يُقْطَعُ ذَلِكَ الْانْسَانُ مِنْ شَعْبِهِ.
\par 10 وَكُلُّ انْسَانٍ مِنْ بَيْتِ اسْرَائِيلَ وَمِنَ الْغُرَبَاءِ النَّازِلِينَ فِي وَسَطِكُمْ يَاكُلُ دَما اجْعَلُ وَجْهِي ضِدَّ النَّفْسِ الْاكِلَةِ الدَّمَِ وَاقْطَعُهَا مِنْ شَعْبِهَا
\par 11 لانَّ نَفْسَ الْجَسَدِ هِيَ فِي الدَّمِ فَانَا اعْطَيْتُكُمْ ايَّاهُ عَلَى الْمَذْبَحِ لِلتَّكْفِيرِ عَنْ نُفُوسِكُمْ لانَّ الدَّمَ يُكَفِّرُ عَنِ النَّفْسِ.
\par 12 لِذَلِكَ قُلْتُ لِبَنِي اسْرَائِيلَ: لا تَاكُلْ نَفْسٌ مِنْكُمْ دَما وَلا يَاكُلِ الْغَرِيبُ النَّازِلُ فِي وَسَطِكُمْ دَما.
\par 13 وَكُلُّ انْسَانٍ مِنْ بَنِي اسْرَائِيلَ وَمِنَ الْغُرَبَاءِ النَّازِلِينَ فِي وَسَطِكُمْ يَصْطَادُ صَيْدا وَحْشا اوْ طَائِرا يُؤْكَلُ يَسْفِكُ دَمَهُ وَيُغَطِّيهِ بِالتُّرَابِ.
\par 14 لانَّ نَفْسَ كُلِّ جَسَدٍ دَمُهُ هُوَ بِنَفْسِهِ. فَقُلْتُ لِبَنِي اسْرَائِيلَ: لا تَاكُلُوا دَمَ جَسَدٍ مَا لانَّ نَفْسَ كُلِّ جَسَدٍ هِيَ دَمُهُ. كُلُّ مَنْ اكَلَهُ يُقْطَعُ.
\par 15 وَكُلُّ انْسَانٍ يَاكُلُ مَيْتَةً اوْ فَرِيسَةً وَطَنِيّا كَانَ اوْ غَرِيبا يَغْسِلُ ثِيَابَهُ وَيَسْتَحِمُّ بِمَاءٍ وَيَبْقَى نَجِسا الَى الْمَسَاءِ ثُمَّ يَكُونُ طَاهِرا.
\par 16 وَانْ لَمْ يَغْسِلْ وَلَمْ يَرْحَضْ جَسَدَهُ يَحْمِلْ ذَنْبَهُ».

\chapter{18}

\par 1 وَقَالَ الرَّبُّ لِمُوسَى
\par 2 «قُلْ لِبَنِي اسْرَائِيلَ: انَا الرَّبُّ الَهُكُمْ.
\par 3 مِثْلَ عَمَلِ ارْضِ مِصْرَ الَّتِي سَكَنْتُمْ فِيهَا لا تَعْمَلُوا وَمِثْلَ عَمَلِ ارْضِ كَنْعَانَ الَّتِي انَا اتٍ بِكُمْ الَيْهَا لا تَعْمَلُوا وَحَسَبَ فَرَائِضِهِمْ لا تَسْلُكُوا.
\par 4 احْكَامِي تَعْمَلُونَ وَفَرَائِضِي تَحْفَظُونَ لِتَسْلُكُوا فِيهَا. انَا الرَّبُّ الَهُكُمْ.
\par 5 فَتَحْفَظُونَ فَرَائِضِي وَاحْكَامِي الَّتِي اذَا فَعَلَهَا الْانْسَانُ يَحْيَا بِهَا. انَا الرَّبُّ.
\par 6 «لا يَقْتَرِبْ انْسَانٌ الَى قَرِيبِ جَسَدِهِ لِيَكْشِفَ الْعَوْرَةَ. انَا الرَّبُّ.
\par 7 عَوْرَةَ ابِيكَ وَعَوْرَةَ امِّكَ لا تَكْشِفْ. انَّهَا امُّكَ لا تَكْشِفْ عَوْرَتَهَا.
\par 8 عَوْرَةَ امْرَاةِ ابِيكَ لا تَكْشِفْ. انَّهَا عَوْرَةُ ابِيكَ.
\par 9 عَوْرَةَ اخْتِكَ بِنْتِ ابِيكَ اوْ بِنْتِ امِّكَ الْمَوْلُودَةِ فِي الْبَيْتِ اوِ الْمَوْلُودَةِ خَارِجا لا تَكْشِفْ عَوْرَتَهَا.
\par 10 عَوْرَةَ ابْنَةِ ابْنِكَ اوِ ابْنَةِ ابْنَتِكَ لا تَكْشِفْ عَوْرَتَهَا. انَّهَا عَوْرَتُكَ.
\par 11 عَوْرَةَ ابْنَةِ امْرَاةِ ابِيكَ الْمَوْلُودَةِ مِنْ ابِيكَ لا تَكْشِفْ عَوْرَتَهَا انَّهَا اخْتُكَ.
\par 12 عَوْرَةَ اخْتِ ابِيكَ لا تَكْشِفْ. انَّهَا قَرِيبَةُ ابِيكَ.
\par 13 عَوْرَةَ اخْتِ امِّكَ لا تَكْشِفْ. انَّهَا قَرِيبَةُ امِّكَ.
\par 14 عَوْرَةَ اخِي ابِيكَ لا تَكْشِفْ. الَى امْرَاتِهِ لا تَقْتَرِبْ. انَّهَا عَمَّتُكَ.
\par 15 عَوْرَةَ كَنَّتِكَ لا تَكْشِفْ. انَّهَا امْرَاةُ ابْنِكَ. لا تَكْشِفْ عَوْرَتَهَا.
\par 16 عَوْرَةَ امْرَاةِ اخِيكَ لا تَكْشِفْ. انَّهَا عَوْرَةُ اخِيكَ.
\par 17 عَوْرَةَ امْرَاةٍ وَابْنَتِهَا لا تَكْشِفْ. وَلا تَاخُذِ ابْنَةَ ابْنِهَا اوِ ابْنَةَ ابْنَتِهَا لِتَكْشِفَ عَوْرَتَهَا. انَّهُمَا قَرِيبَتَاهَا. انَّهُ رَذِيلَةٌ.
\par 18 وَلا تَاخُذِ امْرَاةً عَلَى اخْتِهَا لِلضِّرِّ لِتَكْشِفَ عَوْرَتَهَا مَعَهَا فِي حَيَاتِهَا.
\par 19 «وَلا تَقْتَرِبْ الَى امْرَاةٍ فِي نَجَاسَةِ طَمْثِهَا لِتَكْشِفَ عَوْرَتَهَا.
\par 20 وَلا تَجْعَلْ مَعَ امْرَاةِ صَاحِبِكَ مَضْجَعَكَ لِزَرْعٍ فَتَتَنَجَّسَ بِهَا.
\par 21 وَلا تُعْطِ مِنْ زَرْعِكَ لِلْاجَازَةِ لِمُولَكَ لِئَلَّا تُدَنِّسَ اسْمَ الَهِكَ. انَا الرَّبُّ.
\par 22 وَلا تُضَاجِعْ ذَكَرا مُضَاجَعَةَ امْرَاةٍ. انَّهُ رِجْسٌ.
\par 23 وَلا تَجْعَلْ مَعَ بَهِيمَةٍ مَضْجَعَكَ فَتَتَنَجَّسَ بِهَا. وَلا تَقِفِ امْرَاةٌ امَامَ بَهِيمَةٍ لِنِزَائِهَا. انَّهُ فَاحِشَةٌ.
\par 24 «بِكُلِّ هَذِهِ لا تَتَنَجَّسُوا لانَّهُ بِكُلِّ هَذِهِ قَدْ تَنَجَّسَ الشُّعُوبُ الَّذِينَ انَا طَارِدُهُمْ مِنْ امَامِكُمْ
\par 25 فَتَنَجَّسَتِ الارْضُ. فَاجْتَزِي ذَنْبَهَا مِنْهَا فَتَقْذِفُ الارْضُ سُكَّانَهَا.
\par 26 لَكِنْ تَحْفَظُونَ انْتُمْ فَرَائِضِي وَاحْكَامِي وَلا تَعْمَلُونَ شَيْئا مِنْ جَمِيعِ هَذِهِ الرَّجَاسَاتِ لا الْوَطَنِيُّ وَلا الْغَرِيبُ النَّازِلُ فِي وَسَطِكُمْ
\par 27 (لانَّ جَمِيعَ هَذِهِ الرَّجَاسَاتِ قَدْ عَمِلَهَا اهْلُ الارْضِ الَّذِينَ قَبْلَكُمْ فَتَنَجَّسَتِ الارْضُ).
\par 28 فَلا تَقْذِفُكُمُ الارْضُ بِتَنْجِيسِكُمْ ايَّاهَا كَمَا قَذَفَتِ الشُّعُوبَ الَّتِي قَبْلَكُمْ.
\par 29 بَلْ كُلُّ مَنْ عَمِلَ شَيْئا مِنْ جَمِيعِ هَذِهِ الرَّجَاسَاتِ تُقْطَعُ الانْفُسُ الَّتِي تَعْمَلُهَا مِنْ شَعْبِهَا.
\par 30 فَتَحْفَظُونَ شَعَائِرِي لِكَيْ لا تَعْمَلُوا شَيْئا مِنَ الرُّسُومِ الرَّجِسَةِ الَّتِي عُمِلَتْ قَبْلَكُمْ وَلا تَتَنَجَّسُوا بِهَا. انَا الرَّبُّ الَهُكُمْ».

\chapter{19}

\par 1 وَقَالَ الرَّبُّ لِمُوسَى:
\par 2 «قُلْ لِكُلِّ جَمَاعَةِ بَنِي اسْرَائِيلَ: تَكُونُونَ قِدِّيسِينَ لانِّي قُدُّوسٌ الرَّبُّ الَهُكُمْ.
\par 3 تَهَابُونَ كُلُّ انْسَانٍ امَّهُ وَابَاهُ وَتَحْفَظُونَ سُبُوتِي. انَا الرَّبُّ الَهُكُمْ.
\par 4 لا تَلْتَفِتُوا الَى الاوْثَانِ وَالِهَةً مَسْبُوكَةً لا تَصْنَعُوا لانْفُسِكُمْ. انَا الرَّبُّ الَهُكُمْ.
\par 5 وَمَتَى ذَبَحْتُمْ ذَبِيحَةَ سَلامَةٍ لِلرَّبِّ فَلِلرِّضَا عَنْكُمْ تَذْبَحُونَهَا.
\par 6 يَوْمَ تَذْبَحُونَهَا تُؤْكَلُ وَفِي الْغَدِ. وَالْفَاضِلُ الَى الْيَوْمِ الثَّالِثِ يُحْرَقُ بِالنَّارِ.
\par 7 وَاذَا اكِلَتْ فِي الْيَوْمِ الثَّالِثِ فَذَلِكَ نَجَاسَةٌ لا يُرْضَى بِهِ.
\par 8 وَمَنْ اكَلَ مِنْهَا يَحْمِلُ ذَنْبَهُ لانَّهُ قَدْ دَنَّسَ قُدْسَ الرَّبِّ. فَتُقْطَعُ تِلْكَ النَّفْسُ مِنْ شَعْبِهَا.
\par 9 «وَعِنْدَمَا تَحْصُدُونَ حَصِيدَ ارْضِكُمْ لا تُكَمِّلْ زَوَايَا حَقْلِكَ فِي الْحَصَادِ. وَلُقَاطَ حَصِيدِكَ لا تَلْتَقِطْ.
\par 10 وَكَرْمَكَ لا تُعَلِّلْهُ وَنِثَارَ كَرْمِكَ لا تَلْتَقِطْ. لِلْمِسْكِينِ وَالْغَرِيبِ تَتْرُكُهُ. انَا الرَّبُّ الَهُكُمْ.
\par 11 «لا تَسْرِقُوا وَلا تَكْذِبُوا وَلا تَغْدُرُوا احَدُكُمْ بِصَاحِبِهِ.
\par 12 وَلا تَحْلِفُوا بِاسْمِي لِلْكَذِبِ فَتُدَنِّسَ اسْمَ الَهِكَ. انَا الرَّبُّ.
\par 13 «لا تَغْصِبْ قَرِيبَكَ وَلا تَسْلِبْ وَلا تَبِتْ اجْرَةُ اجِيرٍ عِنْدَكَ الَى الْغَدِ.
\par 14 لا تَشْتِمِ الاصَمَّ وَقُدَّامَ الاعْمَى لا تَجْعَلْ مَعْثَرَةً بَلِ اخْشَ الَهَكَ. انَا الرَّبُّ.
\par 15 لا تَرْتَكِبُوا جَوْرا فِي الْقَضَاءِ. لا تَاخُذُوا بِوَجْهِ مِسْكِينٍ وَلا تَحْتَرِمْ وَجْهَ كَبِيرٍ. بِالْعَدْلِ تَحْكُمُ لِقَرِيبِكَ.
\par 16 لا تَسْعَ فِي الْوِشَايَةِ بَيْنَ شَعْبِكَ. لا تَقِفْ عَلَى دَمِ قَرِيبِكَ. انَا الرَّبُّ.
\par 17 لا تُبْغِضْ اخَاكَ فِي قَلْبِكَ. انْذَارا تُنْذِرُ صَاحِبَكَ وَلا تَحْمِلْ لاجْلِهِ خَطِيَّةً.
\par 18 لا تَنْتَقِمْ وَلا تَحْقِدْ عَلَى ابْنَاءِ شَعْبِكَ بَلْ تُحِبُّ قَرِيبَكَ كَنَفْسِكَ. انَا الرَّبُّ.
\par 19 فَرَائِضِي تَحْفَظُونَ. لا تُنَزِّ بَهَائِمَكَ جِنْسَيْنِ وَحَقْلَكَ لا تَزْرَعْ صِنْفَيْنِ وَلا يَكُنْ عَلَيْكَ ثَوْبٌ مُصَنَّفٌ مِنْ صِنْفَيْنِ.
\par 20 وَاذَا اضْطَجَعَ رَجُلٌ مَعَ امْرَاةٍ اضْطِجَاعَ زَرْعٍ وَهِيَ امَةٌ مَخْطُوبَةٌ لِرَجُلٍ وَلَمْ تُفْدَ فِدَاءً وَلا اعْطِيَتْ حُرِّيَّتَهَا فَلْيَكُنْ تَادِيبٌ. لا يُقْتَلا لانَّهَا لَمْ تُعْتَقْ.
\par 21 وَيَاتِي الَى الرَّبِّ بِذَبِيحَةٍ لاثْمِهِ الَى بَابِ خَيْمَةِ الاجْتِمَاعِ: كَبْشا ذَبِيحَةَ اثْمٍ.
\par 22 فَيُكَفِّرُ عَنْهُ الْكَاهِنُ بِكَبْشِ الْاثْمِ امَامَ الرَّبِّ مِنْ خَطِيَّتِهِ الَّتِي اخْطَا فَيُصْفَحُ لَهُ عَنْ خَطِيَّتِهِ الَّتِي اخْطَا.
\par 23 «وَمَتَى دَخَلْتُمُ الارْضَ وَغَرَسْتُمْ كُلَّ شَجَرَةٍ لِلطَّعَامِ تَحْسِبُونَ ثَمَرَهَا غُرْلَتَهَا. ثَلاثَ سِنِينَ تَكُونُ لَكُمْ غَلْفَاءَ. لا يُؤْكَلْ مِنْهَا.
\par 24 وَفِي السَّنَةِ الرَّابِعَةِ يَكُونُ كُلُّ ثَمَرِهَا قُدْسا لِتَمْجِيدِ الرَّبِّ.
\par 25 وَفِي السَّنَةِ الْخَامِسَةِ تَاكُلُونَ ثَمَرَهَا لِتَزِيدَ لَكُمْ غَلَّتَهَا. انَا الرَّبُّ الَهُكُمْ.
\par 26 «لا تَاكُلُوا بِالدَّمِ. لا تَتَفَاءَلُوا وَلا تَعِيفُوا.
\par 27 لا تُقَصِّرُوا رُؤُوسَكُمْ مُسْتَدِيرا وَلا تُفْسِدْ عَارِضَيْكَ.
\par 28 وَلا تَجْرَحُوا اجْسَادَكُمْ لِمَيْتٍ. وَكِتَابَةَ وَسْمٍ لا تَجْعَلُوا فِيكُمْ. انَا الرَّبُّ.
\par 29 لا تُدَنِّسِ ابْنَتَكَ بِتَعْرِيضِهَا لِلزِّنَى لِئَلَّا تَزْنِيَ الارْضُ وَتَمْتَلِئَ الارْضُ رَذِيلَةً.
\par 30 سُبُوتِي تَحْفَظُونَ وَمَقْدِسِي تَهَابُونَ. انَا الرَّبُّ.
\par 31 لا تَلْتَفِتُوا الَى الْجَانِّ وَلا تَطْلُبُوا التَّوَابِعَ فَتَتَنَجَّسُوا بِهِمْ. انَا الرَّبُّ الَهُكُمْ.
\par 32 مِنْ امَامِ الاشْيَبِ تَقُومُ وَتَحْتَرِمُ وَجْهَ الشَّيْخِ وَتَخْشَى الَهَكَ. انَا الرَّبُّ.
\par 33 «وَاذَا نَزَلَ عِنْدَكَ غَرِيبٌ فِي ارْضِكُمْ فَلا تَظْلِمُوهُ.
\par 34 كَالْوَطَنِيِّ مِنْكُمْ يَكُونُ لَكُمُ الْغَرِيبُ النَّازِلُ عِنْدَكُمْ وَتُحِبُّهُ كَنَفْسِكَ لانَّكُمْ كُنْتُمْ غُرَبَاءَ فِي ارْضِ مِصْرَ. انَا الرَّبُّ الَهُكُمْ.
\par 35 لا تَرْتَكِبُوا جَوْرا فِي الْقَضَاءِ لا فِي الْقِيَاسِ وَلا فِي الْوَزْنِ وَلا فِي الْكَيْلِ.
\par 36 مِيزَانُ حَقٍّ وَوَزْنَاتُ حَقٍّ وَايفَةُ حَقٍّ وَهِينُ حَقٍّ تَكُونُ لَكُمْ. انَا الرَّبُّ الَهُكُمُ الَّذِي اخْرَجَكُمْ مِنْ ارْضِ مِصْرَ.
\par 37 فَتَحْفَظُونَ كُلَّ فَرَائِضِي وَكُلَّ احْكَامِي وَتَعْمَلُونَهَا. انَا الرَّبُّ».

\chapter{20}

\par 1 وَقَالَ الرَّبُّ لِمُوسَى:
\par 2 «وَتَقُولُ لِبَنِي اسْرَائِيلَ: كُلُّ انْسَانٍ مِنْ بَنِي اسْرَائِيلَ وَمِنَ الْغُرَبَاءِ النَّازِلِينَ فِي اسْرَائِيلَ اعْطَى مِنْ زَرْعِهِ لِمُولَكَ فَانَّهُ يُقْتَلُ. يَرْجِمُهُ شَعْبُ الارْضِ بِالْحِجَارَةِ.
\par 3 وَاجْعَلُ انَا وَجْهِي ضِدَّ ذَلِكَ الْانْسَانِ وَاقْطَعُهُ مِنْ شَعْبِهِ لانَّهُ اعْطَى مِنْ زَرْعِهِ لِمُولَكَ لِكَيْ يُنَجِّسَ مَقْدِسِي وَيُدَنِّسَ اسْمِيَ الْقُدُّوسَ.
\par 4 وَانْ غَمَّضَ شَعْبُ الارْضِ اعْيُنَهُمْ عَنْ ذَلِكَ الْانْسَانِ عِنْدَمَا يُعْطِي مِنْ زَرْعِهِ لِمُولَكَ فَلَمْ يَقْتُلُوهُ
\par 5 فَانِّي اضَعُ وَجْهِي ضِدَّ ذَلِكَ الْانْسَانِ وَضِدَّ عَشِيرَتِهِ وَاقْطَعُهُ وَجَمِيعَ الْفَاجِرِينَ وَرَاءَهُ بِالزِّنَى وَرَاءَ مُولَكَ مِنْ شَعْبِهِمْ.
\par 6 وَالنَّفْسُ الَّتِي تَلْتَفِتُ الَى الْجَانِّ وَالَى التَّوَابِعِ لِتَزْنِيَ وَرَاءَهُمْ اجْعَلُ وَجْهِي ضِدَّ تِلْكَ النَّفْسِ وَاقْطَعُهَا مِنْ شَعْبِهَا
\par 7 فَتَتَقَدَّسُونَ وَتَكُونُونَ قِدِّيسِينَ لانِّي انَا الرَّبُّ الَهُكُمْ
\par 8 وَتَحْفَظُونَ فَرَائِضِي وَتَعْمَلُونَهَا. انَا الرَّبُّ مُقَدِّسُكُمْ.
\par 9 «كُلُّ انْسَانٍ سَبَّ ابَاهُ اوْ امَّهُ فَانَّهُ يُقْتَلُ. قَدْ سَبَّ ابَاهُ اوْ امَّهُ. دَمُهُ عَلَيْهِ.
\par 10 وَاذَا زَنَى رَجُلٌ مَعَ امْرَاةٍ فَاذَا زَنَى مَعَ امْرَاةِ قَرِيبِهِ فَانَّهُ يُقْتَلُ الزَّانِي وَالزَّانِيَةُ.
\par 11 وَاذَا اضْطَجَعَ رَجُلٌ مَعَ امْرَاةِ ابِيهِ فَقَدْ كَشَفَ عَوْرَةَ ابِيهِ. انَّهُمَا يُقْتَلانِ كِلاهُمَا. دَمُهُمَا عَلَيْهِمَا.
\par 12 وَاذَا اضْطَجَعَ رَجُلٌ مَعَ كَنَّتِهِ فَانَّهُمَا يُقْتَلانِ كِلاهُمَا. قَدْ فَعَلا فَاحِشَةً. دَمُهُمَا عَلَيْهِمَا.
\par 13 وَاذَا اضْطَجَعَ رَجُلٌ مَعَ ذَكَرٍ اضْطِجَاعَ امْرَاةٍ فَقَدْ فَعَلا كِلاهُمَا رِجْسا. انَّهُمَا يُقْتَلانِ. دَمُهُمَا عَلَيْهِمَا.
\par 14 وَاذَا اتَّخَذَ رَجُلٌ امْرَاةً وَامَّهَا فَذَلِكَ رَذِيلَةٌ. بِالنَّارِ يُحْرِقُونَهُ وَايَّاهُمَا لِكَيْ لا يَكُونَ رَذِيلَةٌ بَيْنَكُمْ.
\par 15 وَاذَا جَعَلَ رَجُلٌ مَضْجَعَهُ مَعَ بَهِيمَةٍ فَانَّهُ يُقْتَلُ وَالْبَهِيمَةُ تُمِيتُونَهَا.
\par 16 وَاذَا اقْتَرَبَتِ امْرَاةٌ الَى بَهِيمَةٍ لِنِزَائِهَا تُمِيتُ الْمَرْاةَ وَالْبَهِيمَةَ. انَّهُمَا يُقْتَلانِ. دَمُهُمَا عَلَيْهِمَا.
\par 17 وَاذَا اخَذَ رَجُلٌ اخْتَهُ بِنْتَ ابِيهِ اوْ بِنْتَ امِّهِ وَرَاى عَوْرَتَهَا وَرَاتْ هِيَ عَوْرَتَهُ فَذَلِكَ عَارٌ. يُقْطَعَانِ امَامَ اعْيُنِ بَنِي شَعْبِهِمَا. قَدْ كَشَفَ عَوْرَةَ اخْتِهِ. يَحْمِلُ ذَنْبَهُ.
\par 18 وَاذَا اضْطَجَعَ رَجُلٌ مَعَ امْرَاةٍ طَامِثٍ وَكَشَفَ عَوْرَتَهَا عَرَّى يَنْبُوعَهَا وَكَشَفَتْ هِيَ يَنْبُوعَ دَمِهَا يُقْطَعَانِ كِلاهُمَا مِنْ شَعِبْهِمَا.
\par 19 عَوْرَةَ اخْتِ امِّكَ اوْ اخْتِ ابِيكَ لا تَكْشِفْ. انَّهُ قَدْ عَرَّى قَرِيبَتَهُ. يَحْمِلانِ ذَنْبَهُمَا.
\par 20 وَاذَا اضْطَجَعَ رَجُلٌ مَعَ امْرَاةِ عَمِّهِ فَقَدْ كَشَفَ عَوْرَةَ عَمِّهِ. يَحْمِلانِ ذَنْبَهُمَا. يَمُوتَانِ عَقِيمَيْنِ.
\par 21 وَاذَا اخَذَ رَجُلٌ امْرَاةَ اخِيهِ فَذَلِكَ نَجَاسَةٌ. قَدْ كَشَفَ عَوْرَةَ اخِيهِ. يَكُونَانِ عَقِيمَيْنِ.
\par 22 «فَتَحْفَظُونَ جَمِيعَ فَرَائِضِي وَجَمِيعَ احْكَامِي وَتَعْمَلُونَهَا لِكَيْ لا تَقْذِفَكُمُ الارْضُ الَّتِي انَا اتٍ بِكُمْ الَيْهَا لِتَسْكُنُوا فِيهَا.
\par 23 وَلا تَسْلُكُونَ فِي رُسُومِ الشُّعُوبِ الَّذِينَ انَا طَارِدُهُمْ مِنْ امَامِكُمْ. لانَّهُمْ قَدْ فَعَلُوا كُلَّ هَذِهِ فَكَرِهْتُهُمْ
\par 24 وَقُلْتُ لَكُمْ: تَرِثُونَ انْتُمْ ارْضَهُمْ وَانَا اعْطِيكُمْ ايَّاهَا لِتَرِثُوهَا ارْضا تَفِيضُ لَبَنا وَعَسَلا. انَا الرَّبُّ الَهُكُمُ الَّذِي مَيَّزَكُمْ مِنَ الشُّعُوبِ.
\par 25 فَتُمَيِّزُونَ بَيْنَ الْبَهَائِمِ الطَّاهِرَةِ وَالنَّجِسَةِ وَبَيْنَ الطُّيُورِ النَّجِسَةِ وَالطَّاهِرَةِ. فَلا تُدَنِّسُوا نُفُوسَكُمْ بِالْبَهَائِمِ وَالطُّيُورِ وَلا بِكُلِّ مَا يَدِبُّ عَلَى الارْضِ مِمَّا مَيَّزْتُهُ لَكُمْ لِيَكُونَ نَجِسا.
\par 26 وَتَكُونُونَ لِي قِدِّيسِينَ لانِّي قُدُّوسٌ انَا الرَّبُّ. وَقَدْ مَيَّزْتُكُمْ مِنَ الشُّعُوبِ لِتَكُونُوا لِي.
\par 27 «وَاذَا كَانَ فِي رَجُلٍ اوِ امْرَاةٍ جَانٌّ اوْ تَابِعَةٌ فَانَّهُ يُقْتَلُ. بِالْحِجَارَةِ يَرْجُمُونَهُ. دَمُهُ عَلَيْهِ».

\chapter{21}

\par 1 وَقَالَ الرَّبُّ لِمُوسَى: «قُلْ لِلْكَهَنَةَ بَنِي هَارُونَ: لا يَتَنَجَّسْ احَدٌ مِنْكُمْ لِمَيِّتٍ فِي قَوْمِهِ
\par 2 الَّا لاقْرِبَائِهِ الاقْرَبِ الَيْهِ: امِّهِ وَابِيهِ وَابْنِهِ وَابْنَتِهِ وَاخِيهِ
\par 3 وَاخْتِهِ الْعَذْرَاءِ الْقَرِيبَةِ الَيْهِ الَّتِي لَمْ تَصِرْ لِرَجُلٍ. لاجْلِهَا يَتَنَجَّسُ.
\par 4 كَزَوْجٍ لا يَتَنَجَّسْ بِاهْلِهِ لِتَدْنِيسِهِ.
\par 5 لا يَجْعَلُوا قَرْعَةً فِي رُؤُوسِهِمْ وَلا يَحْلِقُوا عَوَارِضَ لِحَاهُمْ وَلا يَجْرَحُوا جِرَاحَةً فِي اجْسَادِهِمْ.
\par 6 مُقَدَّسِينَ يَكُونُونَ لالَهِهِمْ وَلا يُدَنِّسُونَ اسْمَ الَهِهِمْ لانَّهُمْ يُقَرِّبُونَ وَقَائِدَ الرَّبِّ طَعَامَ الَهِهِمْ فَيَكُونُونَ قُدْسا.
\par 7 امْرَاةً زَانِيَةً اوْ مُدَنَّسَةً لا يَاخُذُوا وَلا يَاخُذُوا امْرَاةً مُطَلَّقَةً مِنْ زَوْجِهَا. لانَّهُ مُقَدَّسٌ لالَهِهِ.
\par 8 فَتَحْسِبُهُ مُقَدَّسا لانَّهُ يُقَرِّبُ خُبْزَ الَهِكَ. مُقَدَّسا يَكُونُ عِنْدَكَ لانِّي قُدُّوسٌ انَا الرَّبُّ مُقَدِّسُكُمْ.
\par 9 وَاذَا تَدَنَّسَتِ ابْنَةُ كَاهِنٍ بِالزِّنَى فَقَدْ دَنَّسَتْ ابَاهَا. بِالنَّارِ تُحْرَقُ.
\par 10 «وَالْكَاهِنُ الاعْظَمُ بَيْنَ اخْوَتِهِ الَّذِي صُبَّ عَلَى رَاسِهِ دُهْنُ الْمَسْحَةِ وَمُلِئَتْ يَدُهُ لِيَلْبِسَ الثِّيَابَ لا يَكْشِفُ رَاسَهُ وَلا يَشُقُّ ثِيَابَهُ
\par 11 وَلا يَاتِي الَى نَفْسٍ مَيِّتَةٍ وَلا يَتَنَجَّسُ لابِيهِ اوْ امِّهِ
\par 12 وَلا يَخْرُجُ مِنَ الْمَقْدِسِ لِئَلَّا يُدَنِّسَ مَقْدِسَ الَهِهِ لانَّ اكْلِيلَ دُهْنِ مَسْحَةِ الَهِهِ عَلَيْهِ. انَا الرَّبُّ.
\par 13 هَذَا يَاخُذُ امْرَاةً عَذْرَاءَ.
\par 14 امَّا الارْمَلَةُ وَالْمُطَلَّقَةُ وَالْمُدَنَّسَةُ وَالزَّانِيَةُ فَمِنْ هَؤُلاءِ لا يَاخُذُ بَلْ يَتَّخِذُ عَذْرَاءَ مِنْ قَوْمِهِ امْرَاةً.
\par 15 وَلا يُدَنِّسُ زَرْعَهُ بَيْنَ شَعْبِهِ لانِّي انَا الرَّبُّ مُقَدِّسُهُ».
\par 16 وَقَالَ الرَّبُّ لِمُوسَى:
\par 17 «قُلْ لِهَارُون: اذَا كَانَ رَجُلٌ مِنْ نَسْلِكَ فِي اجْيَالِهِمْ فِيهِ عَيْبٌ فَلا يَتَقَدَّمْ لِيُقَرِّبَ خُبْزَ الَهِهِ.
\par 18 لانَّ كُلَّ رَجُلٍ فِيهِ عَيْبٌ لا يَتَقَدَّمْ. لا رَجُلٌ اعْمَى وَلا اعْرَجُ وَلا افْطَسُ وَلا زَوَائِدِيٌّ
\par 19 وَلا رَجُلٌ فِيهِ كَسْرُ رِجْلٍ اوْ كَسْرُ يَدٍ
\par 20 وَلا احْدَبُ وَلا اكْثَمُ وَلا مَنْ فِي عَيْنِهِ بَيَاضٌ وَلا اجْرَبُ وَلا اكْلَفُ وَلا مَرْضُوضُ الْخُصَى.
\par 21 كُلُّ رَجُلٍ فِيهِ عَيْبٌ مِنْ نَسْلِ هَارُونَ الْكَاهِنِ لا يَتَقَدَّمْ لِيُقَرِّبَ وَقَائِدَ الرَّبِّ. فِيهِ عَيْبٌ لا يَتَقَدَّمْ لِيُقَرِّبَ خُبْزَ الَهِهِ.
\par 22 خُبْزَ الَهِهِ مِنْ قُدْسِ الاقْدَاسِ وَمِنَ الْقُدْسِ يَاكُلُ.
\par 23 لَكِنْ الَى الْحِجَابِ لا يَاتِي وَالَى الْمَذْبَحِ لا يَقْتَرِبُ لانَّ فِيهِ عَيْبا لِئَلَّا يُدَنِّسَ مَقْدِسِي لانِّي انَا الرَّبُّ مُقَدِّسُهُمْ».
\par 24 فَكَلَّمَ مُوسَى هَارُونَ وَبَنِيهِ وَكُلَّ بَنِي اسْرَائِيلَ.

\chapter{22}

\par 1 وَقَالَ الرَّبُّ لِمُوسَى:
\par 2 «قُلْ لِهَارُونَ وَبَنِيهِ انْ يَتَوَقُّوا اقْدَاسَ بَنِي اسْرَائِيلَ الَّتِي يُقَدِّسُونَهَا لِي وَلا يُدَنِّسُوا اسْمِي الْقُدُّوسَ. انَا الرَّبُّ.
\par 3 قُلْ لَهُمْ: فِي اجْيَالِكُمْ كُلُّ انْسَانٍ مِنْ جَمِيعِ نَسْلِكُمُ اقْتَرَبَ الَى الاقْدَاسِ الَّتِي يُقَدِّسُهَا بَنُو اسْرَائِيلَ لِلرَّبِّ وَنَجَاسَتُهُ عَلَيْهِ تُقْطَعُ تِلْكَ النَّفْسُ مِنْ امَامِي. انَا الرَّبُّ.
\par 4 كُلُّ انْسَانٍ مِنْ نَسْلِ هَارُونَ وَهُوَ ابْرَصُ اوْ ذُو سَيْلٍ لا يَاكُلْ مِنَ الاقْدَاسِ حَتَّى يَطْهُرَ. وَمَنْ مَسَّ شَيْئا نَجِسا لِمَيِّتٍ اوْ انْسَانٌ حَدَثَ مِنْهُ اضْطِجَاعُ زَرْعٍ
\par 5 اوْ انْسَانٌ مَسَّ دَبِيبا يَتَنَجَّسُ بِهِ اوْ انْسَانا يَتَنَجَّسُ بِهِ لِنَجَاسَةٍ فِيهِ
\par 6 فَالَّذِي يَمَسُّ ذَلِكَ يَكُونُ نَجِسا الَى الْمَسَاءِ وَلا يَاكُلْ مِنَ الاقْدَاسِ بَلْ يَرْحَضُ جَسَدَهُ بِمَاءٍ.
\par 7 فَمَتَى غَرَبَتِ الشَّمْسُ يَكُونُ طَاهِرا ثُمَّ يَاكُلُ مِنَ الاقْدَاسِ لانَّهَا طَعَامُهُ.
\par 8 مِيِّتَةً اوْ فَرِيسَةً لا يَاكُلْ فَيَتَنَجَّسَ بِهَا. انَا الرَّبُّ.
\par 9 فَيَحْفَظُونَ شَعَائِرِي لِكَيْ لا يَحْمِلُوا لاجْلِهَا خَطِيَّةً يَمُوتُونَ بِهَا لانَّهُمْ يُدَنِّسُونَهَا. انَا الرَّبُّ مُقَدِّسُهُمْ.
\par 10 «وَكُلُّ اجْنَبِيٍّ لا يَاكُلُ قُدْسا. نَزِيلُ كَاهِنٍ وَاجِيرُهُ لا يَاكُلُونَ قُدْسا.
\par 11 لَكِنْ اذَا اشْتَرَى كَاهِنٌ احَدا شِرَاءَ فِضَّةٍ فَهُوَ يَاكُلُ مِنْهُ وَالْمَوْلُودُ فِي بَيْتِهِ. هُمَا يَاكُلانِ مِنْ طَعَامِهِ.
\par 12 وَاذَا صَارَتِ ابْنَةُ كَاهِنٍ لِرَجُلٍ اجْنَبِيٍّ لا تَاكُلُ مِنْ رَفِيعَةِ الاقْدَاسِ.
\par 13 وَامَّا ابْنَةُ كَاهِنٍ قَدْ صَارَتْ ارْمَلَةً اوْ مُطَلَّقَةً وَلَمْ يَكُنْ لَهَا نَسْلٌ وَرَجَعَتْ الَى بَيْتِ ابِيهَا كَمَا فِي صِبَاهَا فَتَاكُلُ مِنْ طَعَامِ ابِيهَا. لَكِنَّ كُلَّ اجْنَبِيٍّ لا يَاكُلُ مِنْهُ.
\par 14 وَاذَا اكَلَ انْسَانٌ قُدْسا سَهْوا يَزِيدُ عَلَيْهِ خُمْسَهُ وَيَدْفَعُ الْقُدْسَ لِلْكَاهِنِ.
\par 15 فَلا يُدَنِّسُونَ اقْدَاسَ بَنِي اسْرَائِيلَ الَّتِي يَرْفَعُونَهَا لِلرَّبِّ
\par 16 فَيُحَمِّلُونَهُمْ ذَنْبَ اثْمٍ بِاكْلِهِمْ اقْدَاسَهُمْ. لانِّي انَا الرَّبُّ مُقَدِّسُهُمْ».
\par 17 وَقَالَ الرَّبُّ لِمُوسَى:
\par 18 «قُلْ لِهَارُونَ وَبَنِيهِ وَجَمِيعِ بَنِي اسْرَائِيلَ: كُلُّ انْسَانٍ مِنْ بَيْتِ اسْرَائِيلَ وَمِنَ الْغُرَبَاءِ فِي اسْرَائِيلَ قَرَّبَ قُرْبَانَهُ مِنْ جَمِيعِ نُذُورِهِمْ وَجَمِيعِ نَوَافِلِهِمِ الَّتِي يُقَرِّبُونَهَا لِلرَّبِّ مُحْرَقَةً
\par 19 فَلِلرِّضَا عَنْكُمْ يَكُونُ ذَكَرا صَحِيحا مِنَ الْبَقَرِ اوِ الْغَنَمِ اوِ الْمَعْزِ.
\par 20 كُلُّ مَا كَانَ فِيهِ عَيْبٌ لا تُقَرِّبُوهُ لانَّهُ لا يَكُونُ لِلرِّضَا عَنْكُمْ.
\par 21 وَاذَا قَرَّبَ انْسَانٌ ذَبِيحَةَ سَلامَةٍ لِلرَّبِّ وَفَاءً لِنَذْرٍ اوْ نَافِلَةً مِنَ الْبَقَرِ اوِ الاغْنَامِ تَكُونُ صَحِيحَةً لِلرِّضَا. كُلُّ عَيْبٍ لا يَكُونُ فِيهَا.
\par 22 الاعْمَى وَالْمَكْسُورُ وَالْمَجْرُوحُ وَالْبَثِيرُ وَالاجْرَبُ وَالاكْلَفُ هَذِهِ لا تُقَرِّبُوهَا لِلرَّبِّ وَلا تَجْعَلُوا مِنْهَا وَقُودا عَلَى الْمَذْبَحِ لِلرَّبِّ.
\par 23 وَامَّا الثَّوْرُ اوِ الشَّاةُ الزَّوَائِدِيُّ اوِ الْقُزُمُ فَنَافِلَةً تَعْمَلُهُ وَلَكِنْ لِنَذْرٍ لا يُرْضَى بِهِ.
\par 24 وَمَرْضُوضَ الْخِصْيَةِ وَمَسْحُوقَهَا وَمَقْطُوعَهَا لا تُقَرِّبُوا لِلرَّبِّ. وَفِي ارْضِكُمْ لا تَعْمَلُوهَا.
\par 25 وَمِنْ يَدِ ابْنِ الْغَرِيبِ لا تُقَرِّبُوا خُبْزَ الَهِكُمْ مِنْ جَمِيعِ هَذِهِ لانَّ فِيهَا فَسَادَهَا. فِيهَا عَيْبٌ لا يُرْضَى بِهَا عَنْكُمْ».
\par 26 وَقَالَ الرَّبُّ لِمُوسَى:
\par 27 «مَتَى وُلِدَ بَقَرٌ اوْ غَنَمٌ اوْ مِعْزًى يَكُونُ سَبْعَةَ ايَّامٍ تَحْتَ امِّهِ ثُمَّ مِنَ الْيَوْمِ الثَّامِنِ فَصَاعِدا يُرْضَى بِهِ قُرْبَانَ وَقُودٍ لِلرَّبِّ.
\par 28 وَامَّا الْبَقَرَةُ اوِ الشَّاةُ فَلا تَذْبَحُوهَا وَابْنَهَا فِي يَوْمٍ وَاحِدٍ.
\par 29 وَمَتَى ذَبَحْتُمْ ذَبِيحَةَ شُكْرٍ لِلرَّبِّ فَلِلرِّضَا عَنْكُمْ تَذْبَحُونَهَا.
\par 30 فِي ذَلِكَ الْيَوْمِ تُؤْكَلُ. لا تُبْقُوا مِنْهَا الَى الْغَدِ. انَا الرَّبُّ.
\par 31 فَتَحْفَظُونَ وَصَايَايَ وَتَعْمَلُونَهَا. انَا الرَّبُّ.
\par 32 وَلا تُدَنِّسُونَ اسْمِي الْقُدُّوسَ فَاتَقَدَّسُ فِي وَسَطِ بَنِي اسْرَائِيلَ. انَا الرَّبُّ مُقَدِّسُكُمُ
\par 33 الَّذِي اخْرَجَكُمْ مِنْ ارْضِ مِصْرَ لِيَكُونَ لَكُمْ الَها. انَا الرَّبُّ».

\chapter{23}

\par 1 وَقَالَ الرَّبُّ لِمُوسَى:
\par 2 «قُلْ لِبَنِي اسْرَائِيلَ: مَوَاسِمُ الرَّبِّ الَّتِي فِيهَا تُنَادُونَ مَحَافِلَ مُقَدَّسَةً. هَذِهِ هِيَ مَوَاسِمِي:
\par 3 سِتَّةَ ايَّامٍ يُعْمَلُ عَمَلٌ وَامَّا الْيَوْمُ السَّابِعُ فَفِيهِ سَبْتُ عُطْلَةٍ مَحْفَلٌ مُقَدَّسٌ. عَمَلا مَا لا تَعْمَلُوا. انَّهُ سَبْتٌ لِلرَّبِّ فِي جَمِيعِ مَسَاكِنِكُمْ.
\par 4 «هَذِهِ مَوَاسِمُ الرَّبِّ الْمَحَافِلُ الْمُقَدَّسَةُ الَّتِي تُنَادُونَ بِهَا فِي اوْقَاتِهَا.
\par 5 فِي الشَّهْرِ الاوَّلِ فِي الرَّابِعَِ عَشَرَ مِنَ الشَّهْرِ بَيْنَ الْعِشَاءَيْنِ فِصْحٌ لِلرَّبِّ.
\par 6 وَفِي الْيَوْمِ الْخَامِسَ عَشَرَ مِنْ هَذَا الشَّهْرِ عِيدُ الْفَطِيرِ لِلرَّبِّ. سَبْعَةَ ايَّامٍ تَاكُلُونَ فَطِيرا.
\par 7 فِي الْيَوْمِ الاوَّلِ يَكُونُ لَكُمْ مَحْفَلٌ مُقَدَّسٌ. عَمَلا مَا مِنَ الشُّغْلِ لا تَعْمَلُوا.
\par 8 وَسَبْعَةَ ايَّامٍ تُقَرِّبُونَ وَقُودا لِلرَّبِّ. فِي الْيَوْمِ السَّابِعِ يَكُونُ مَحْفَلٌ مُقَدَّسٌ. عَمَلا مَا مِنَ الشُّغْلِ لا تَعْمَلُوا».
\par 9 وقَالَ الرَّبُّ لِمُوسَى:
\par 10 «قُلْ لِبَنِي اسْرَائِيلَ: مَتَى جِئْتُمْ الَى الارْضِ الَّتِي انَا اعْطِيكُمْ وَحَصَدْتُمْ حَصِيدَهَا تَاتُونَ بِحُزْمَةِ اوَّلِ حَصِيدِكُمْ الَى الْكَاهِنِ.
\par 11 فَيُرَدِّدُ الْحُزْمَةَ امَامَ الرَّبِّ لِلرِّضَا عَنْكُمْ. فِي غَدِ السَّبْتِ يُرَدِّدُهَا الْكَاهِنُ.
\par 12 وَتَعْمَلُونَ يَوْمَ تَرْدِيدِكُمُ الْحُزْمَةَ خَرُوفا صَحِيحا حَوْلِيّا مُحْرَقَةً لِلرَّبِّ.
\par 13 وَتَقْدِمَتَهُ عُشْرَيْنِ مِنْ دَقِيقٍ مَلْتُوتٍ بِزَيْتٍ وَقُودا لِلرَّبِّ رَائِحَةَ سُرُورٍ وَسَكِيبَهُ رُبْعَ الْهِينِ مِنْ خَمْرٍ.
\par 14 وَخُبْزا وَفَرِيكا وَسَوِيقا لا تَاكُلُوا الَى هَذَا الْيَوْمِ عَيْنِهِ الَى انْ تَاتُوا بِقُرْبَانِ الَهِكُمْ فَرِيضَةً دَهْرِيَّةً فِي اجْيَالِكُمْ فِي جَمِيعِ مَسَاكِنِكُمْ.
\par 15 «ثُمَّ تَحْسِبُونَ لَكُمْ مِنْ غَدِ السَّبْتِ مِنْ يَوْمِ اتْيَانِكُمْ بِحُزْمَةِ التَّرْدِيدِ سَبْعَةَ اسَابِيعَ تَكُونُ كَامِلَةً.
\par 16 الَى غَدِ السَّبْتِ السَّابِعِ تَحْسِبُونَ خَمْسِينَ يَوْما ثُمَّ تُقَرِّبُونَ تَقْدِمَةً جَدِيدَةً لِلرَّبِّ.
\par 17 مِنْ مَسَاكِنِكُمْ تَاتُونَ بِخُبْزِ تَرْدِيدٍ. رَغِيفَيْنِ عُشْرَيْنِ يَكُونَانِ مِنْ دَقِيقٍ وَيُخْبَزَانِ خَمِيرا بَاكُورَةً لِلرَّبِّ.
\par 18 وَتُقَرِّبُونَ مَعَ الْخُبْزِ سَبْعَةَ خِرَافٍ صَحِيحَةٍ حَوْلِيَّةٍ وَثَوْرا وَاحِدا ابْنَ بَقَرٍ وَكَبْشَيْنِ مُحْرَقَةً لِلرَّبِّ مَعَ تَقْدِمَتِهَا وَسَكِيبِهَا وَقُودَ رَائِحَةِ سُرُورٍ لِلرَّبِّ.
\par 19 وَتَعْمَلُونَ تَيْسا وَاحِدا مِنَ الْمَعْزِ ذَبِيحَةَ خَطِيَّةٍ وَخَرُوفَيْنِ حَوْلِيَّيْنِ ذَبِيحَةَ سَلامَةٍ.
\par 20 فَيُرَدِّدُهَا الْكَاهِنُ مَعَ خُبْزِ الْبَاكُورَةِ تَرْدِيدا امَامَ الرَّبِّ مَعَ الْخَرُوفَيْنِ فَتَكُونُ لِلْكَاهِنِ قُدْسا لِلرَّبِّ.
\par 21 وَتُنَادُونَ فِي ذَلِكَ الْيَوْمِ عَيْنِهِ مَحْفَلا مُقَدَّسا يَكُونُ لَكُمْ. عَمَلا مَا مِنَ الشُّغْلِ لا تَعْمَلُوا. فَرِيضَةً دَهْرِيَّةً فِي جَمِيعِ مَسَاكِنِكُمْ فِي اجْيَالِكُمْ.
\par 22 وَعِنْدَمَا تَحْصُدُونَ حَصِيدَ ارْضِكُمْ لا تُكَمِّلْ زَوَايَا حَقْلِكَ فِي حَصَادِكَ وَلُقَاطَ حَصِيدِكَ لا تَلْتَقِطْ. لِلْمِسْكِينِ وَالْغَرِيبِ تَتْرُكُهُ. انَا الرَّبُّ الَهُكُمْ».
\par 23 وَقَالَ الرَّبُّ لِمُوسَى:
\par 24 «قُلْ لِبَنِي اسْرَائِيلَ: فِي الشَّهْرِ السَّابِعِ فِي اوَّلِ الشَّهْرِ يَكُونُ لَكُمْ عُطْلَةٌ تِذْكَارُ هُتَافِ الْبُوقِ مَحْفَلٌ مُقَدَّسٌ.
\par 25 عَمَلا مَا مِنَ الشُّغْلِ لا تَعْمَلُوا لَكِنْ تُقَرِّبُونَ وَقُودا لِلرَّبِّ».
\par 26 وَقَالَ الرَّبُّ لِمُوسَى:
\par 27 «امَّا الْعَاشِرُ مِنْ هَذَا الشَّهْرِ السَّابِعِ فَهُوَ يَوْمُ الْكَفَّارَةِ. مَحْفَلا مُقَدَّسا يَكُونُ لَكُمْ. تُذَلِّلُونَ نُفُوسَكُمْ وَتُقَرِّبُونَ وَقُودا لِلرَّبِّ.
\par 28 عَمَلا مَا لا تَعْمَلُوا فِي هَذَا الْيَوْمِ عَيْنِهِ لانَّهُ يَوْمُ كَفَّارَةٍ لِلتَّكْفِيرِ عَنْكُمْ امَامَ الرَّبِّ الَهِكُمْ.
\par 29 انَّ كُلَّ نَفْسٍ لا تَتَذَلَّلُ فِي هَذَا الْيَوْمِ عَيْنِهِ تُقْطَعُ مِنْ شَعْبِهَا.
\par 30 وَكُلَّ نَفْسٍ تَعْمَلُ عَمَلا مَا فِي هَذَا الْيَوْمِ عَيْنِهِ ابِيدُ تِلْكَ النَّفْسَ مِنْ شَعْبِهَا.
\par 31 عَمَلا مَا لا تَعْمَلُوا. فَرِيضَةً دَهْرِيَّةً فِي اجْيَالِكُمْ فِي جَمِيعِ مَسَاكِنِكُمْ.
\par 32 انَّهُ سَبْتُ عُطْلَةٍ لَكُمْ فَتُذَلِّلُونَ نُفُوسَكُمْ. فِي تَاسِعِ الشَّهْرِ عِنْدَ الْمَسَاءِ. مِنَ الْمَسَاءِ الَى الْمَسَاءِ تَسْبِتُونَ سَبْتَكُمْ».
\par 33 وَقَالَ الرَّبُّ لِمُوسَى:
\par 34 «قُلْ لِبَنِي اسْرَائِيلَ: فِي الْيَوْمِ الْخَامِسَ عَشَرَ مِنْ هَذَا الشَّهْرِ السَّابِعِ عِيدُ الْمَظَالِّ سَبْعَةَ ايَّامٍ لِلرَّبِّ.
\par 35 فِي الْيَوْمِ الاوَّلِ مَحْفَلٌ مُقَدَّسٌ. عَمَلا مَا مِنَ الشُّغْلِ لا تَعْمَلُوا.
\par 36 سَبْعَةَ ايَّامٍ تُقَرِّبُونَ وَقُودا لِلرَّبِّ. فِي الْيَوْمِ الثَّامِنِ يَكُونُ لَكُمْ مَحْفَلٌ مُقَدَّسٌ تُقَرِّبُونَ وَقُودا لِلرَّبِّ. انَّهُ اعْتِكَافٌ. كُلُّ عَمَلِ شَغْلٍ لا تَعْمَلُوا.
\par 37 «هَذِهِ هِيَ مَوَاسِمُ الرَّبِّ الَّتِي فِيهَا تُنَادُونَ مَحَافِلَ مُقَدَّسَةً لِتَقْرِيبِ وَقُودٍ لِلرَّبِّ مُحْرَقَةً وَتَقْدِمَةً وَذَبِيحَةً وَسَكِيبا امْرَ الْيَوْمِ بِيَوْمِهِ
\par 38 عَدَا سُبُوتَِ الرَّبِّ وَعَدَا عَطَايَاكُمْ وَجَمِيعِ نُذُورِكُمْ وَجَمِيعِ نَوَافِلِكُمُ الَّتِي تُعْطُونَهَا لِلرَّبِّ.
\par 39 امَّا الْيَوْمُ الْخَامِسَ عَشَرَ مِنَ الشَّهْرِ السَّابِعِ فَفِيهِ عِنْدَمَا تَجْمَعُونَ غَلَّةَ الارْضِ تُعَيِّدُونَ عِيدا لِلرَّبِّ سَبْعَةَ ايَّامٍ. فِي الْيَوْمِ الاوَّلِ عُطْلَةٌ وَفِي الْيَوْمِ الثَّامِنِ عُطْلَةٌ.
\par 40 وَتَاخُذُونَ لانْفُسِكُمْ فِي الْيَوْمِ الاوَّلِ ثَمَرَ اشْجَارٍ بَهِجَةٍ وَسَعَفَ النَّخْلِ وَاغْصَانَ اشْجَارٍ غَبْيَاءَ وَصَفْصَافَ الْوَادِي وَتَفْرَحُونَ امَامَ الرَّبِّ الَهِكُمْ سَبْعَةَ ايَّامٍ.
\par 41 تُعَيِّدُونَهُ عِيدا لِلرَّبِّ سَبْعَةَ ايَّامٍ فِي السَّنَةِ فَرِيضَةً دَهْرِيَّةً فِي اجْيَالِكُمْ. فِي الشَّهْرِ السَّابِعِ تُعَيِّدُونَهُ.
\par 42 فِي مَظَالَّ تَسْكُنُونَ سَبْعَةَ ايَّامٍ. كُلُّ الْوَطَنِيِّينَ فِي اسْرَائِيلَ يَسْكُنُونَ فِي الْمَظَالِّ.
\par 43 لِكَيْ تَعْلَمَ اجْيَالُكُمْ انِّي فِي مَظَالَّ اسْكَنْتُ بَنِي اسْرَائِيلَ لَمَّا اخْرَجْتُهُمْ مِنْ ارْضِ مِصْرَ. انَا الرَّبُّ الَهُكُمْ».
\par 44 فَاخْبَرَ مُوسَى بَنِي اسْرَائِيلَ بِمَوَاسِمِ الرَّبِّ.

\chapter{24}

\par 1 وَقَالَ الرَّبُّ لِمُوسَى:
\par 2 «اوْصِ بَنِي اسْرَائِيلَ انْ يُقَدِّمُوا الَيْكَ زَيْتَ زَيْتُونٍ مَرْضُوضٍ نَقِيّا لِلضُّوءِ لايقَادِ السُّرُجِ دَائِما.
\par 3 خَارِجَ حِجَابِ الشَّهَادَةِ فِي خَيْمَةِ الاجْتِمَاعِ يُرَتِّبُهَا هَارُونُ مِنَ الْمَسَاءِ الَى الصَّبَاحِ امَامَ الرَّبِّ دَائِما فَرِيضَةً دَهْرِيَّةً فِي اجْيَالِكُمْ.
\par 4 عَلَى الْمَنَارَةِ الطَّاهِرَةِ يُرَتِّبُ السُّرُجَ امَامَ الرَّبِّ دَائِما.
\par 5 «وَتَاخُذُ دَقِيقا وَتَخْبِزُهُ اثْنَيْ عَشَرَ قُرْصا. عُشْرَيْنِ يَكُونُ الْقُرْصُ الْوَاحِدُ.
\par 6 وَتَجْعَلُهَا صَفَّيْنِ كُلَّ صَفٍّ سِتَّةً عَلَى الْمَائِدَةِ الطَّاهِرَةِ امَامَ الرَّبِّ.
\par 7 وَتَجْعَلُ عَلَى كُلِّ صَفٍّ لُبَانا نَقِيّا فَيَكُونُ لِلْخُبْزِ تِذْكَارا وَقُودا لِلرَّبِّ.
\par 8 فِي كُلِّ يَوْمِ سَبْتٍ يُرَتِّبُهُ امَامَ الرَّبِّ دَائِما مِنْ عِنْدِ بَنِي اسْرَائِيلَ مِيثَاقا دَهْرِيّا.
\par 9 فَيَكُونُ لِهَارُونَ وَبَنِيهِ فَيَاكُلُونَهُ فِي مَكَانٍ مُقَدَّسٍ لانَّهُ قُدْسُ اقْدَاسٍ لَهُ مِنْ وَقَائِدِ الرَّبِّ فَرِيضَةً دَهْرِيَّةً».
\par 10 وَخَرَجَ ابْنُ امْرَاةٍ اسْرَائِيلِيَّةٍ وَهُوَ ابْنُ رَجُلٍ مِصْرِيٍّ فِي وَسَطِ بَنِي اسْرَائِيلَ. وَتَخَاصَمَ فِي الْمَحَلَّةِ ابْنُ الْاسْرَائِيلِيَّةِ وَرَجُلٌ اسْرَائِيلِيٌّ.
\par 11 فَجَدَّفَ ابْنُ الْاسْرَائِيلِيَّةِ عَلَى الاسْمِ وَسَبَّ. فَاتُوا بِهِ الَى مُوسَى. (وَكَانَ اسْمُ امِّهِ شَلُومِيَةَ بِنْتَ دِبْرِي مِنْ سِبْطِ دَانٍ).
\par 12 فَوَضَعُوهُ فِي الْمَحْرَسِ لِيُعْلَنَ لَهُمْ عَنْ فَمِ الرَّبِّ.
\par 13 فَقَالَ الرَّبُّ لِمُوسَى:
\par 14 «اخْرِجِ الَّذِي سَبَّ الَى خَارِجِ الْمَحَلَّةِ فَيَضَعَ جَمِيعُ السَّامِعِينَ ايْدِيَهُمْ عَلَى رَاسِهِ وَيَرْجُمَهُ كُلُّ الْجَمَاعَةِ.
\par 15 وَقُلْ لِبَنِي اسْرَائِيلَ: كُلُّ مَنْ سَبَّ الَهَهُ يَحْمِلُ خَطِيَّتَهُ
\par 16 وَمَنْ جَدَّفَ عَلَى اسْمِ الرَّبِّ فَانَّهُ يُقْتَلُ. يَرْجُمُهُ كُلُّ الْجَمَاعَةِ رَجْما. الْغَرِيبُ كَالْوَطَنِيِّ عِنْدَمَا يُجَدِّفُ عَلَى الاسْمِ يُقْتَلُ.
\par 17 وَاذَا امَاتَ احَدٌ انْسَانا فَانَّهُ يُقْتَلُ.
\par 18 وَمَنْ امَاتَ بَهِيمَةً يُعَوِّضُ عَنْهَا نَفْسا بِنَفْسٍ.
\par 19 وَاذَا احْدَثَ انْسَانٌ فِي قَرِيبِهِ عَيْبا فَكَمَا فَعَلَ كَذَلِكَ يُفْعَلُ بِهِ.
\par 20 كَسْرٌ بِكَسْرٍ وَعَيْنٌ بِعَيْنٍ وَسِنٌّ بِسِنٍّ. كَمَا احْدَثَ عَيْبا فِي الْانْسَانِ كَذَلِكَ يُحْدَثُ فِيهِ.
\par 21 مَنْ قَتَلَ بَهِيمَةً يُعَوِّضُ عَنْهَا وَمَنْ قَتَلَ انْسَانا يُقْتَلْ.
\par 22 حُكْمٌ وَاحِدٌ يَكُونُ لَكُمْ. الْغَرِيبُ يَكُونُ كَالْوَطَنِيِّ. انِّي انَا الرَّبُّ الَهُكُمْ».
\par 23 فَكَلَّمَ مُوسَى بَنِي اسْرَائِيلَ انْ يُخْرِجُوا الَّذِي سَبَّ الَى خَارِجِ الْمَحَلَّةِ وَيَرْجُمُوهُ بِالْحِجَارَةِ. فَفَعَلَ بَنُو اسْرَائِيلَ كَمَا امَرَ الرَّبُّ مُوسَى.

\chapter{25}

\par 1 وَقَالَ الرَّبُّ لِمُوسَى فِي جَبَلِ سِينَاءَ:
\par 2 «قُلْ لِبَنِي اسْرَائِيلَ: مَتَى اتَيْتُمْ الَى الارْضِ الَّتِي انَا اعْطِيكُمْ تَسْبِتُ الارْضُ سَبْتا لِلرَّبِّ.
\par 3 سِتَّ سِنِينَ تَزْرَعُ حَقْلَكَ وَسِتَّ سِنِينَ تَقْضِبُ كَرْمَكَ وَتَجْمَعُ غَلَّتَهُمَا.
\par 4 وَامَّا السَّنَةُ السَّابِعَةُ فَفِيهَا يَكُونُ لِلارْضِ سَبْتُ عُطْلَةٍ سَبْتا لِلرَّبِّ. لا تَزْرَعْ حَقْلَكَ وَلا تَقْضِبْ كَرْمَكَ.
\par 5 زِرِّيعَ حَصِيدِكَ لا تَحْصُدْ وَعِنَبَ كَرْمِكَ الْمُحْوِلِ لا تَقْطِفْ. سَنَةَ عُطْلَةٍ تَكُونُ لِلارْضِ.
\par 6 وَيَكُونُ سَبْتُ الارْضِ لَكُمْ طَعَاما. لَكَ وَلِعَبْدِكَ وَلامَتِكَ وَلاجِيرِكَ وَلِمُسْتَوْطِنِكَ النَّازِلِينَ عِنْدَكَ
\par 7 وَلِبَهَائِمِكَ وَلِلْحَيَوَانِ الَّذِي فِي ارْضِكَ تَكُونُ كُلُّ غَلَّتِهَا طَعَاما.
\par 8 «وَتَعُدُّ لَكَ سَبْعَةَ سُبُوتِ سِنِينَ. سَبْعَ سِنِينَ سَبْعَ مَرَّاتٍ. فَتَكُونُ لَكَ ايَّامُ السَّبْعَةِ السُّبُوتِ السَّنَوِيَّةِ تِسْعا وَارْبَعِينَ سَنَةً.
\par 9 ثُمَّ تُعَبِّرُ بُوقَ الْهُتَافِ فِي الشَّهْرِ السَّابِعِ فِي عَاشِرِ الشَّهْرِ. فِي يَوْمِ الْكَفَّارَةِ تُعَبِّرُونَ الْبُوقَ فِي جَمِيعِ ارْضِكُمْ.
\par 10 وَتُقَدِّسُونَ السَّنَةَ الْخَمْسِينَ وَتُنَادُونَ بِالْعِتْقِ فِي الارْضِ لِجَمِيعِ سُكَّانِهَا. تَكُونُ لَكُمْ يُوبِيلا وَتَرْجِعُونَ كُلٌّ الَى مُلْكِهِ وَتَعُودُونَ كُلٌّ الَى عَشِيرَتِهِ.
\par 11 يُوبِيلا تَكُونُ لَكُمُ السَّنَةُ الْخَمْسُونَ. لا تَزْرَعُوا وَلا تَحْصُدُوا زِرِّيعَهَا وَلا تَقْطِفُوا كَرْمَهَا الْمُحْوِلَ.
\par 12 انَّهَا يُوبِيلٌ. مُقَدَّسَةً تَكُونُ لَكُمْ. مِنَ الْحَقْلِ تَاكُلُونَ غَلَّتَهَا.
\par 13 فِي سَنَةِ الْيُوبِيلِ هَذِهِ تَرْجِعُونَ كُلٌّ الَى مُلْكِهِ.
\par 14 فَمَتَى بِعْتَ صَاحِبَكَ مَبِيعا اوِ اشْتَرَيْتَ مِنْ يَدِ صَاحِبِكَ فَلا يَغْبِنْ احَدُكُمْ اخَاهُ.
\par 15 حَسَبَ عَدَدِ السِّنِينَ بَعْدَ الْيُوبِيلِ تَشْتَرِي مِنْ صَاحِبِكَ وَحَسَبَ سِنِي الْغَلَّةِ يَبِيعُكَ.
\par 16 عَلَى قَدْرِ كَثْرَةِ السِّنِينَ تُكَثِّرُ ثَمَنَهُ وَعَلَى قَدْرِ قِلَّةِ السِّنِينَ تُقَلِّلُ ثَمَنَهُ لانَّهُ عَدَدَ الْغَلَّاتِ يَبِيعُكَ.
\par 17 فَلا يَغْبِنْ احَدُكُمْ صَاحِبَهُ بَلِ اخْشَ الَهَكَ. انِّي انَا الرَّبُّ الَهُكُمْ.
\par 18 فَتَعْمَلُونَ فَرَائِضِي وَتَحْفَظُونَ احْكَامِي وَتَعْمَلُونَهَا لِتَسْكُنُوا عَلَى الارْضِ امِنِينَ
\par 19 وَتُعْطِي الارْضُ ثَمَرَهَا فَتَاكُلُونَ لِلشَّبَعِ وَتَسْكُنُونَ عَلَيْهَا امِنِينَ.
\par 20 وَاذَا قُلْتُمْ: مَاذَا نَاكُلُ فِي السَّنَةِ السَّابِعَةِ انْ لَمْ نَزْرَعْ وَلَمْ نَجْمَعْ غَلَّتَنَا؟
\par 21 فَانِّي امُرُ بِبَرَكَتِي لَكُمْ فِي السَّنَةِ السَّادِسَةِ فَتَعْمَلُ غَلَّةً لِثَلاثِ سِنِينَ.
\par 22 فَتَزْرَعُونَ السَّنَةَ الثَّامِنَةَ وَتَاكُلُونَ مِنَ الْغَلَّةِ الْعَتِيقَةِ الَى السَّنَةِ التَّاسِعَةِ. الَى انْ تَاتِيَ غَلَّتُهَا تَاكُلُونَ عَتِيقا.
\par 23 «وَالارْضُ لا تُبَاعُ بَتَّةً لانَّ لِيَ الارْضَ وَانْتُمْ غُرَبَاءُ وَنُزَلاءُ عِنْدِي.
\par 24 بَلْ فِي كُلِّ ارْضِ مُلْكِكُمْ تَجْعَلُونَ فِكَاكا لِلارْضِ.
\par 25 اذَا افْتَقَرَ اخُوكَ فَبَاعَ مِنْ مُلْكِهِ يَاتِي وَلِيُّهُ الاقْرَبُ الَيْهِ وَيَفُكُّ مَبِيعَ اخِيهِ.
\par 26 وَمَنْ لَمْ يَكُنْ لَهُ وَلِيٌّ فَانْ نَالَتْ يَدُهُ وَوَجَدَ مِقْدَارَ فِكَاكِهِ
\par 27 يَحْسِبُ سِنِي بَيْعِهِ وَيَرُدُّ الْفَاضِلَ لِلْانْسَانِ الَّذِي بَاعَ لَهُ فَيَرْجِعُ الَى مُلْكِهِ.
\par 28 وَانْ لَمْ تَنَلْ يَدُهُ كِفَايَةً لِيَرُدَّ لَهُ يَكُونُ مَبِيعُهُ فِي يَدِ شَارِيهِ الَى سَنَةِ الْيُوبِيلِ ثُمَّ يَخْرُجُ فِي الْيُوبِيلِ فَيَرْجِعُ الَى مُلْكِهِ.
\par 29 «وَاذَا بَاعَ انْسَانٌ بَيْتَ سَكَنٍ فِي مَدِينَةٍ ذَاتِ سُورٍ فَيَكُونُ فِكَاكُهُ الَى تَمَامِ سَنَةِ بَيْعِهِ. سَنَةً يَكُونُ فِكَاكُهُ.
\par 30 وَانْ لَمْ يُفَكَّ قَبْلَ انْ تَكْمُلَ لَهُ سَنَةٌ تَامَّةٌ وَجَبَ الْبَيْتُ الَّذِي فِي الْمَدِينَةِ ذَاتِ السُّورِ بَتَّةً لِشَارِيهِ فِي اجْيَالِهِ. لا يَخْرُجُ فِي الْيُوبِيلِ.
\par 31 لَكِنَّ بُيُوتَ الْقُرَى الَّتِي لَيْسَ لَهَا سُورٌ حَوْلَهَا فَمَعَ حُقُولِ الارْضِ تُحْسَبُ. يَكُونُ لَهَا فِكَاكٌ وَفِي الْيُوبِيلِ تَخْرُجُ.
\par 32 وَامَّا مُدُنُ اللَّاوِيِّينَ بُيُوتُ مُدُنِ مُلْكِهِمْ فَيَكُونُ لَهَا فِكَاكٌ مُؤَبَّدٌ لِلَّاوِيِّينَ.
\par 33 وَالَّذِي يَفُكُّهُ مِنَ اللَّاوِيِّينَ الْمَبِيعَ مِنْ بَيْتٍ اوْ مِنْ مَدِينَةِ مُلْكِهِ يَخْرُجُ فِي الْيُوبِيلِ لانَّ بُيُوتَ مُدُنِ اللَّاوِيِّينَ هِيَ مُلْكُهُمْ فِي وَسَطِ بَنِي اسْرَائِيلَ.
\par 34 وَامَّا حُقُولُ الْمَسَارِحِ لِمُدُنِهِمْ فَلا تُبَاعُ لانَّهَا مُلْكٌ دَهْرِيٌّ لَهُمْ.
\par 35 «وَاذَا افْتَقَرَ اخُوكَ وَقَصُرَتْ يَدُهُ عِنْدَكَ فَاعْضُدْهُ غَرِيبا اوْ مُسْتَوْطِنا فَيَعِيشَ مَعَكَ.
\par 36 لا تَاخُذْ مِنْهُ رِبا وَلا مُرَابَحَةً بَلِ اخْشَ الَهَكَ فَيَعِيشَ اخُوكَ مَعَكَ.
\par 37 فِضَّتَكَ لا تُعْطِهِ بِالرِّبَا وَطَعَامَكَ لا تُعْطِ بِالْمُرَابَحَةِ.
\par 38 انَا الرَّبُّ الَهُكُمُ الَّذِي اخْرَجَكُمْ مِنْ ارْضِ مِصْرَ لِيُعْطِيَكُمْ ارْضَ كَنْعَانَ فَيَكُونَ لَكُمْ الَها.
\par 39 «وَاذَا افْتَقَرَ اخُوكَ عِنْدَكَ وَبِيعَ لَكَ فَلا تَسْتَعْبِدْهُ اسْتِعْبَادَ عَبْدٍ.
\par 40 كَاجِيرٍ كَنَزِيلٍ يَكُونُ عِنْدَكَ. الَى سَنَةِ الْيُوبِيلِ يَخْدِمُ عِنْدَكَ
\par 41 ثُمَّ يَخْرُجُ مِنْ عِنْدِكَ هُوَ وَبَنُوهُ مَعَهُ وَيَعُودُ الَى عَشِيرَتِهِ وَالَى مُلْكِ ابَائِهِ يَرْجِعُ.
\par 42 لانَّهُمْ عَبِيدِي الَّذِينَ اخْرَجْتُهُمْ مِنْ ارْضِ مِصْرَ لا يُبَاعُونَ بَيْعَ الْعَبِيدِ.
\par 43 لا تَتَسَلَّطْ عَلَيْهِ بِعُنْفٍ. بَلِ اخْشَ الَهَكَ.
\par 44 وَامَّا عَبِيدُكَ وَامَاؤُكَ الَّذِينَ يَكُونُونَ لَكَ فَمِنَ الشُّعُوبِ الَّذِينَ حَوْلَكُمْ. مِنْهُمْ تَقْتَنُونَ عَبِيدا وَامَاءً.
\par 45 وَايْضا مِنْ ابْنَاءِ الْمُسْتَوْطِنِينَ النَّازِلِينَ عِنْدَكُمْ مِنْهُمْ تَقْتَنُونَ وَمِنْ عَشَائِرِهِمِ الَّذِينَ عِنْدَكُمُ الَّذِينَ يَلِدُونَهُمْ فِي ارْضِكُمْ فَيَكُونُونَ مُلْكا لَكُمْ.
\par 46 وَتَسْتَمْلِكُونَهُمْ لابْنَائِكُمْ مِنْ بَعْدِكُمْ مِيرَاثَ مُلْكٍ. تَسْتَعْبِدُونَهُمْ الَى الدَّهْرِ. وَامَّا اخْوَتُكُمْ بَنُو اسْرَائِيلَ فَلا يَتَسَلَّطْ انْسَانٌ عَلَى اخِيهِ بِعُنْفٍ.
\par 47 «وَاذَا طَالَتْ يَدُ غَرِيبٍ اوْ نَزِيلٍ عِنْدَكَ وَافْتَقَرَ اخُوكَ عِنْدَهُ وَبِيعَ لِلْغَرِيبِ الْمُسْتَوْطِنِ عِنْدَكَ اوْ لِنَسْلِ عَشِيرَةِ الْغَرِيبِ
\par 48 فَبَعْدَ بَيْعِهِ يَكُونُ لَهُ فِكَاكٌ. يَفُكُّهُ وَاحِدٌ مِنْ اخْوَتِهِ
\par 49 اوْ يَفُكُّهُ عَمُّهُ اوِ ابْنُ عَمِّهِ اوْ يَفُكُّهُ وَاحِدٌ مِنْ اقْرِبَاءِ جَسَدِهِ مِنْ عَشِيرَتِهِ اوْ اذَا نَالَتْ يَدُهُ يَفُكُّ نَفْسَهُ.
\par 50 فَيُحَاسِبُ شَارِيَهُ مِنْ سَنَةِ بَيْعِهِ لَهُ الَى سَنَةِ الْيُوبِيلِ وَيَكُونُ ثَمَنُ بَيْعِهِ حَسَبَ عَدَدِ السِّنِينَ. كَايَّامِ اجِيرٍ يَكُونُ عِنْدَهُ.
\par 51 انْ بَقِيَ كَثِيرٌ مِنَ السِّنِينِ فَعَلَى قَدْرِهَا يَرُدُّ فِكَاكَهُ مِنْ ثَمَنِ شِرَائِهِ.
\par 52 وَانْ بَقِيَ قَلِيلٌ مِنَ السِّنِينَ الَى سَنَةِ الْيُوبِيلِ يَحْسِبُ لَهُ وَعَلَى قَدْرِ سِنِيهِ يَرُدُّ فِكَاكَهُ.
\par 53 كَاجِيرٍ مِنْ سَنَةٍ الَى سَنَةٍ يَكُونُ عِنْدَهُ. لا يَتَسَلَّطْ عَلَيْهِ بِعُنْفٍ امَامَ عَيْنَيْكَ.
\par 54 وَانْ لَمْ يُفَكَّ بِهَؤُلاءِ يَخْرُجُ فِي سَنَةِ الْيُوبِيلِ هُوَ وَبَنُوهُ مَعَهُ
\par 55 لانَّ بَنِي اسْرَائِيلَ لِي عَبِيدٌ. هُمْ عَبِيدِي الَّذِينَ اخْرَجْتُهُمْ مِنْ ارْضِ مِصْرَ. انَا الرَّبُّ الَهُكُمْ.

\chapter{26}

\par 1 «لا تَصْنَعُوا لَكُمْ اوْثَانا وَلا تُقِيمُوا لَكُمْ تِمْثَالا مَنْحُوتا اوْ نَصَبا وَلا تَجْعَلُوا فِي ارْضِكُمْ حَجَرا مُصَّوَرا لِتَسْجُدُوا لَهُ. لانِّي انَا الرَّبُّ الَهُكُمْ.
\par 2 سُبُوتِي تَحْفَظُونَ وَمَقْدِسِي تَهَابُونَ. انَا الرَّبُّ.
\par 3 «اذَا سَلَكْتُمْ فِي فَرَائِضِي وَحَفِظْتُمْ وَصَايَايَ وَعَمِلْتُمْ بِهَا
\par 4 اعْطِي مَطَرَكُمْ فِي حِينِهِ وَتُعْطِي الارْضُ غَلَّتَهَا وَتُعْطِي اشْجَارُ الْحَقْلِ اثْمَارَهَا
\par 5 وَيَلْحَقُ دِرَاسُكُمْ بِالْقِطَافِ وَيَلْحَقُ الْقِطَافُ بِالزَّرْعِ فَتَاكُلُونَ خُبْزَكُمْ لِلشَّبَعِ وَتَسْكُنُونَ فِي ارْضِكُمْ امِنِينَ.
\par 6 وَاجْعَلُ سَلاما فِي الارْضِ فَتَنَامُونَ وَلَيْسَ مَنْ يُزْعِجُكُمْ. وَابِيدُ الْوُحُوشَ الرَّدِيئَةَ مِنَ الارْضِ وَلا يَعْبُرُ سَيْفٌ فِي ارْضِكُمْ.
\par 7 وَتَطْرُدُونَ اعْدَاءَكُمْ فَيَسْقُطُونَ امَامَكُمْ بِالسَّيْفِ.
\par 8 يَطْرُدُ خَمْسَةٌ مِنْكُمْ مِئَةً وَمِئَةٌ مِنْكُمْ يَطْرُدُونَ رَبْوَةً وَيَسْقُطُ اعْدَاؤُكُمْ امَامَكُمْ بِالسَّيْفِ.
\par 9 وَالْتَفِتُ الَيْكُمْ وَاثْمِرُكُمْ وَاكَثِّرُكُمْ وَافِي مِيثَاقِي مَعَكُمْ
\par 10 فَتَاكُلُونَ الْعَتِيقَ الْمُعَتَّقَ وَتُخْرِجُونَ الْعَتِيقَ مِنْ وَجْهِ الْجَدِيدِ.
\par 11 وَاجْعَلُ مَسْكَنِي فِي وَسَطِكُمْ وَلا تَرْذُلُكُمْ نَفْسِي.
\par 12 وَاسِيرُ بَيْنَكُمْ وَاكُونُ لَكُمْ الَها وَانْتُمْ تَكُونُونَ لِي شَعْبا.
\par 13 انَا الرَّبُّ الَهُكُمُ الَّذِي اخْرَجَكُمْ مِنْ ارْضِ مِصْرَ مِنْ كَوْنِكُمْ لَهُمْ عَبِيدا وَقَطَّعَ قُيُودَ نِيرِكُمْ وَسَيَّرَكُمْ قِيَاما.
\par 14 «لَكِنْ انْ لَمْ تَسْمَعُوا لِي وَلَمْ تَعْمَلُوا كُلَّ هَذِهِ الْوَصَايَا
\par 15 وَانْ رَفَضْتُمْ فَرَائِضِي وَكَرِهَتْ انْفُسُكُمْ احْكَامِي فَمَا عَمِلْتُمْ كُلَّ وَصَايَايَ بَلْ نَكَثْتُمْ مِيثَاقِي
\par 16 فَانِّي اعْمَلُ هَذِهِ بِكُمْ: اسَلِّطُ عَلَيْكُمْ رُعْبا وَسِلا وَحُمَّى تُفْنِي الْعَيْنَيْنِ وَتُتْلِفُ النَّفْسَ. وَتَزْرَعُونَ بَاطِلا زَرْعَكُمْ فَيَاكُلُهُ اعْدَاؤُكُمْ.
\par 17 وَاجْعَلُ وَجْهِي ضِدَّكُمْ فَتَنْهَزِمُونَ امَامَ اعْدَائِكُمْ وَيَتَسَلَّطُ عَلَيْكُمْ مُبْغِضُوكُمْ وَتَهْرُبُونَ وَلَيْسَ مَنْ يَطْرُدُكُمْ.
\par 18 «وَانْ كُنْتُمْ مَعَ ذَلِكَ لا تَسْمَعُونَ لِي ازِيدُ عَلَى تَادِيبِكُمْ سَبْعَةَ اضْعَافٍ حَسَبَ خَطَايَاكُمْ
\par 19 فَاحَطِّمُ فَخَارَ عِزِّكُمْ وَاصَيِّرُ سَمَاءَكُمْ كَالْحَدِيدِ وَارْضَكُمْ كَالنُّحَاسِ
\par 20 فَتُفْرَغُ بَاطِلا قُوَّتُكُمْ وَارْضُكُمْ لا تُعْطِي غَلَّتَهَا وَاشْجَارُ الارْضِ لا تُعْطِي اثْمَارَهَا.
\par 21 «وَانْ سَلَكْتُمْ مَعِي بِالْخِلافِ وَلَمْ تَشَاءُوا انْ تَسْمَعُوا لِي ازِيدُ عَلَيْكُمْ ضَرْبَاتٍ سَبْعَةَ اضْعَافٍ حَسَبَ خَطَايَاكُمْ.
\par 22 اطْلِقُ عَلَيْكُمْ وُحُوشَ الْبَرِّيَّةِ فَتُعْدِمُكُمُ الاوْلادَ وَتَقْرِضُ بَهَائِمَكُمْ وَتُقَلِّلُكُمْ فَتُوحَشُ طُرُقُكُمْ.
\par 23 «وَانْ لَمْ تَتَادَّبُوا مِنِّي بِذَلِكَ بَلْ سَلَكْتُمْ مَعِي بِالْخِلافِ
\par 24 فَانِّي انَا اسْلُكُ مَعَكُمْ بِالْخِلافِ وَاضْرِبُكُمْ سَبْعَةَ اضْعَافٍ حَسَبَ خَطَايَاكُمْ.
\par 25 اجْلِبُ عَلَيْكُمْ سَيْفا يَنْتَقِمُ نَقْمَةَ الْمِيثَاقِ فَتَجْتَمِعُونَ الَى مُدُنِكُمْ وَارْسِلُ فِي وَسَطِكُمُ الْوَبَا فَتُدْفَعُونَ بِيَدِ الْعَدُوِّ.
\par 26 بِكَسْرِي لَكُمْ عَصَا الْخُبْزِ. تَخْبِزُ عَشَرُ نِسَاءٍ خُبْزَكُمْ فِي تَنُّورٍ وَاحِدٍ وَيَرْدُدْنَ خُبْزَكُمْ بِالْوَزْنِ فَتَاكُلُونَ وَلا تَشْبَعُونَ.
\par 27 «وَانْ كُنْتُمْ بِذَلِكَ لا تَسْمَعُونَ لِي بَلْ سَلَكْتُمْ مَعِي بِالْخِلافِ
\par 28 فَانَا اسْلُكُ مَعَكُمْ بِالْخِلافِ سَاخِطا وَاؤَدِّبُكُمْ سَبْعَةَ اضْعَافٍ حَسَبَ خَطَايَاكُمْ
\par 29 فَتَاكُلُونَ لَحْمَ بَنِيكُمْ وَلَحْمَ بَنَاتِكُمْ تَاكُلُونَ.
\par 30 وَاخْرِبُ مُرْتَفَعَاتِكُمْ وَاقْطَعُ شَمْسَاتِكُمْ وَالْقِي جُثَثَكُمْ عَلَى جُثَثِ اصْنَامِكُمْ وَتَرْذُلُكُمْ نَفْسِي.
\par 31 وَاصَيِّرُ مُدُنَكُمْ خَرِبَةً وَمَقَادِسَكُمْ مُوحِشَةً وَلا اشْتَمُّ رَائِحَةَ سُرُورِكُمْ.
\par 32 وَاوحِشُ الارْضَ فَيَسْتَوْحِشُ مِنْهَا اعْدَاؤُكُمُ السَّاكِنُونَ فِيهَا.
\par 33 وَاذَرِّيكُمْ بَيْنَ الامَمِ وَاجَرِّدُ وَرَاءَكُمُ السَّيْفَ فَتَصِيرُ ارْضُكُمْ مُوحِشَةً وَمُدُنُكُمْ تَصِيرُ خَرِبَةً.
\par 34 حِينَئِذٍ تَسْتَوْفِي الارْضُ سُبُوتَهَا كُلَّ ايَّامِ وَحْشَتِهَا وَانْتُمْ فِي ارْضِ اعْدَائِكُمْ. حِينَئِذٍ تَسْبِتُ الارْضُ وَتَسْتَوْفِي سُبُوتَهَا.
\par 35 كُلَّ ايَّامِ وَحْشَتِهَا تَسْبِتُ مَا لَمْ تَسْبِتْهُ مِنْ سُبُوتِكُمْ فِي سَكَنِكُمْ عَلَيْهَا.
\par 36 وَالْبَاقُونَ مِنْكُمْ الْقِي الْجَبَانَةَ فِي قُلُوبِهِمْ فِي ارَاضِي اعْدَائِهِمْ فَيَهْزِمُهُمْ صَوْتُ وَرَقَةٍ مُنْدَفِعَةٍ فَيَهْرُبُونَ كَالْهَرَبِ مِنَ السَّيْفِ وَيَسْقُطُونَ وَلَيْسَ طَارِدٌ.
\par 37 وَيَعْثُرُ بَعْضُهُمْ بِبَعْضٍ كَمَا مِنْ امَامِ السَّيْفِ وَلَيْسَ طَارِدٌ وَلا يَكُونُ لَكُمْ قِيَامٌ امَامَ اعْدَائِكُمْ
\par 38 فَتَهْلِكُونَ بَيْنَ الشُّعُوبِ وَتَاكُلُكُمْ ارْضُ اعْدَائِكُمْ.
\par 39 وَالْبَاقُونَ مِنْكُمْ يَفْنُونَ بِذُنُوبِهِمْ فِي ارَاضِي اعْدَائِكُمْ. وَايْضا بِذُنُوبِ ابَائِهِمْ مَعَهُمْ يَفْنُونَ.
\par 40 لَكِنْ انْ اقَرُّوا بِذُنُوبِهِمْ وَذُنُوبِ ابَائِهِمْ فِي خِيَانَتِهِمِ الَّتِي خَانُونِي بِهَا وَسُلُوكِهِمْ مَعِيَ الَّذِي سَلَكُوا بِالْخِلافِ
\par 41 وَانِّي ايْضا سَلَكْتُ مَعَهُمْ بِالْخِلافِ وَاتَيْتُ بِهِمْ الَى ارْضِ اعْدَائِهِمْ. الَّا انْ تَخْضَعَ حِينَئِذٍ قُلُوبُهُمُ الْغُلْفُ وَيَسْتَوْفُوا حِينَئِذٍ عَنْ ذُنُوبِهِمْ
\par 42 اذْكُرُ مِيثَاقِي مَعَ يَعْقُوبَ وَاذْكُرُ ايْضا مِيثَاقِي مَعَ اسْحَاقَ وَمِيثَاقِي مَعَ ابْرَاهِيمَ وَاذْكُرُ الارْضَ.
\par 43 وَالارْضُ تُتْرَكُ مِنْهُمْ وَتَسْتَوْفِي سُبُوتَهَا فِي وَحْشَتِهَا مِنْهُمْ وَهُمْ يَسْتَوْفُونَ عَنْ ذُنُوبِهِمْ لانَّهُمْ قَدْ ابُوا احْكَامِي وَكَرِهَتْ انْفُسُهُمْ فَرَائِضِي.
\par 44 وَلَكِنْ مَعَ ذَلِكَ ايْضا مَتَى كَانُوا فِي ارْضِ اعْدَائِهِمْ مَا ابَيْتُهُمْ وَلا كَرِهْتُهُمْ حَتَّى ابِيدَهُمْ وَانْكُثَ مِيثَاقِي مَعَهُمْ لانِّي انَا الرَّبُّ الَهُهُمْ.
\par 45 بَلْ اذْكُرُ لَهُمْ الْمِيثَاقَ مَعَ الاوَّلِينَ الَّذِينَ اخْرَجْتُهُمْ مِنْ ارْضِ مِصْرَ امَامَ اعْيُنِ الشُّعُوبِ لاكُونَ لَهُمْ الَها. انَا الرَّبُّ».
\par 46 هَذِهِ هِيَ الْفَرَائِضُ وَالاحْكَامُ وَالشَّرَائِعُ الَّتِي وَضَعَهَا الرَّبُّ بَيْنَهُ وَبَيْنَ بَنِي اسْرَائِيلَ فِي جَبَلِ سِينَاءَ بِيَدِ مُوسَى.

\chapter{27}

\par 1 وَقَالَ الرَّبُّ لِمُوسَى:
\par 2 «قُلْ لِبَنِي اسْرَائِيلَ: اذَا افْرَزَ انْسَانٌ نَذْرا حَسَبَ تَقْوِيمِكَ نُفُوسا لِلرَّبِّ
\par 3 فَانْ كَانَ تَقْوِيمُكَ لِذَكَرٍ مِنِ ابْنِ عِشْرِينَ سَنَةً الَى ابْنِ سِتِّينَ سَنَةً يَكُونُ تَقْوِيمُكَ خَمْسِينَ شَاقِلَ فِضَّةٍ عَلَى شَاقِلِ الْمَقْدِسِ.
\par 4 وَانْ كَانَ انْثَى يَكُونُ تَقْوِيمُكَ ثَلاثِينَ شَاقِلا.
\par 5 وَانْ كَانَ مِنِ ابْنِ خَمْسِ سِنِينَ الَى ابْنِ عِشْرِينَ سَنَةً يَكُونُ تَقْوِيمُكَ لِذَكَرٍ عِشْرِينَ شَاقِلا وَلانْثَى عَشَرَةَ شَوَاقِلَ.
\par 6 وَانْ كَانَ مِنِ ابْنِ شَهْرٍ الَى ابْنِ خَمْسِ سِنِينَ يَكُونُ تَقْوِيمُكَ لِذَكَرٍ خَمْسَةَ شَوَاقِلِ فِضَّةٍ وَلِانْثَى يَكُونُ تَقْوِيمُكَ ثَلاثَةَ شَوَاقِلِ فِضَّةٍ.
\par 7 وَانْ كَانَ مِنِ ابْنِ سِتِّينَ سَنَةً فَصَاعِدا فَانْ كَانَ ذَكَرا يَكُونُ تَقْوِيمُكَ خَمْسَةَ عَشَرَ شَاقِلا. وَامَّا لِلانْثَى فَعَشَرَةَ شَوَاقِلَ.
\par 8 وَانْ كَانَ فَقِيرا عَنْ تَقْوِيمِكَ يُوقِفُهُ امَامَ الْكَاهِنِ فَيُقَوِّمُهُ الْكَاهِنُ. عَلَى قَدْرِ مَا تَنَالُ يَدُ النَّاذِرِ يُقَوِّمُهُ الْكَاهِنُ.
\par 9 «وَانْ كَانَ بَهِيمَةً مِمَّا يُقَرِّبُونَهُ قُرْبَانا لِلرَّبِّ فَكُلُّ مَا يُعْطِي مِنْهُ لِلرَّبِّ يَكُونُ قُدْسا.
\par 10 لا يُغَيِّرُهُ وَلا يُبْدِلُهُ جَيِّدا بِرَدِيءٍ اوْ رَدِيئا بِجَيِّدٍ. وَانْ ابْدَلَ بَهِيمَةً بِبَهِيمَةٍ تَكُونُ هِيَ وَبَدِيلُهَا قُدْسا.
\par 11 وَانْ كَانَ بَهِيمَةً نَجِسَةً مِمَّا لا يُقَرِّبُونَهُ قُرْبَانا لِلرَّبِّ يُوقِفُ الْبَهِيمَةَ امَامَ الْكَاهِنِ
\par 12 فَيُقَوِّمُهَا الْكَاهِنُ جَيِّدَةً امْ رَدِيئَةً. فَحَسَبَ تَقْوِيمِكَ يَا كَاهِنُ هَكَذَا يَكُونُ.
\par 13 فَانْ فَكَّهَا يَزِيدُ خُمْسَهَا عَلَى تَقْوِيمِكَ.
\par 14 «وَاذَا قَدَّسَ انْسَانٌ بَيْتَهُ قُدْسا لِلرَّبِّ يُقَوِّمُهُ الْكَاهِنُ جَيِّدا امْ رَدِيئا. وَكَمَا يُقَوِّمُهُ الْكَاهِنُ هَكَذَا يَقُومُ.
\par 15 فَانْ كَانَ الْمُقَدِّسُ يَفُكُّ بَيْتَهُ يَزِيدُ خُمْسَ فِضَّةِ تَقْوِيمِكَ عَلَيْهِ فَيَكُونُ لَهُ.
\par 16 وَانْ قَدَّسَ انْسَانٌ بَعْضَ حَقْلِ مُلْكِهِ لِلرَّبِّ يَكُونُ تَقْوِيمُكَ عَلَى قَدَرِ بِذَارِهِ. بِذَارُ حُومَرٍ مِنَ الشَّعِيرِ بِخَمْسِينَ شَاقِلِ فِضَّةٍ.
\par 17 انْ قَدَّسَ حَقْلَهُ مِنْ سَنَةِ الْيُوبِيلِ فَحَسَبَ تَقْوِيمِكَ يَقُومُ.
\par 18 وَانْ قَدَّسَ حَقْلَهُ بَعْدَ سَنَةِ الْيُوبِيلِ يَحْسِبُ لَهُ الْكَاهِنُ الْفِضَّةَ عَلَى قَدَرِ السِّنِينَ الْبَاقِيَةِ الَى سَنَةِ الْيُوبِيلِ فَيُنَقَّصُ مِنْ تَقْوِيمِكَ.
\par 19 فَانْ فَكَّ الْحَقْلَ مُقَدِّسُهُ يَزِيدُ خُمْسَ فِضَّةِ تَقْوِيمِكَ عَلَيْهِ فَيَجِبُ لَهُ.
\par 20 لَكِنْ انْ لَمْ يَفُكَّ الْحَقْلَ وَبِيعَ الْحَقْلُ لانْسَانٍ اخَرَ لا يُفَكُّ بَعْدُ
\par 21 بَلْ يَكُونُ الْحَقْلُ عِنْدَ خُرُوجِهِ فِي الْيُوبِيلِ قُدْسا لِلرَّبِّ كَالْحَقْلِ الْمُحَرَّمِ. لِلْكَاهِنِ يَكُونُ مُلْكُهُ.
\par 22 «وَانْ قَدَّسَ لِلرَّبِّ حَقْلا مِنْ شِرَائِهِ لَيْسَ مِنْ حُقُولِ مُلْكِهِ
\par 23 يَحْسِبُ لَهُ الْكَاهِنُ مَبْلَغَ تَقْوِيمِكَ الَى سَنَةِ الْيُوبِيلِ فَيُعْطِي تَقْوِيمَكَ فِي ذَلِكَ الْيَوْمِ قُدْسا لِلرَّبِّ.
\par 24 وَفِي سَنَةِ الْيُوبِيلِ يَرْجِعُ الْحَقْلُ الَى الَّذِي اشْتَرَاهُ مِنْهُ الَى الَّذِي لَهُ مُلْكُ الارْضِ.
\par 25 وَكُلُّ تَقْوِيمِكَ يَكُونُ عَلَى شَاقِلِ الْمَقْدِسِ. عِشْرِينَ جِيرَةً يَكُونُ الشَّاقِلُ.
\par 26 «لَكِنَّ الْبِكْرَ الَّذِي يُفْرَزُ بِكْرا لِلرَّبِّ مِنَ الْبَهَائِمِ فَلا يُقَدِّسُهُ احَدٌ. ثَوْرا كَانَ اوْ شَاةً فَهُوَ لِلرَّبِّ.
\par 27 وَانْ كَانَ مِنَ الْبَهَائِمِ النَّجِسَةِ يَفْدِيهِ حَسَبَ تَقْوِيمِكَ وَيَزِيدُ خُمْسَهُ عَلَيْهِ. وَانْ لَمْ يُفَكَّ فَيُبَاعُ حَسَبَ تَقْوِيمِكَ.
\par 28 امَّا كُلُّ مُحَرَّمٍ يُحَرِّمُهُ انْسَانٌ لِلرَّبِّ مِنْ كُلِّ مَا لَهُ مِنَ النَّاسِ وَالْبَهَائِمِ وَمِنْ حُقُولِ مُلْكِهِ فَلا يُبَاعُ وَلا يُفَكُّ. انَّ كُلَّ مُحَرَّمٍ هُوَ قُدْسُ اقْدَاسٍ لِلرَّبِّ.
\par 29 كُلُّ مُحَرَّمٍ يُحَرَّمُ مِنَ النَّاسِ لا يُفْدَى. يُقْتَلُ قَتْلا.
\par 30 «وَكُلُّ عُشْرِ الارْضِ مِنْ حُبُوبِ الارْضِ وَاثْمَارِ الشَّجَرِ فَهُوَ لِلرَّبِّ. قُدْسٌ لِلرَّبِّ.
\par 31 وَانْ فَكَّ انْسَانٌ بَعْضَ عُشْرِهِ يَزِيدُ خُمْسَهُ عَلَيْهِ.
\par 32 وَامَّا كُلُّ عُشْرِ الْبَقَرِ وَالْغَنَمِ فَكُلُّ مَا يَعْبُرُ تَحْتَ الْعَصَا يَكُونُ الْعَاشِرُ قُدْسا لِلرَّبِّ.
\par 33 لا يُفْحَصُ اجَيِّدٌ هُوَ امْ رَدِيءٌ وَلا يُبْدِلُهُ. وَانْ ابْدَلَهُ يَكُونُ هُوَ وَبَدِيلُهُ قُدْسا. لا يُفَكُّ».
\par 34 هَذِهِ هِيَ الْوَصَايَا الَّتِي اوْصَى الرَّبُّ بِهَا مُوسَى الَى بَنِي اسْرَائِيلَ فِي جَبَلِ سِينَاءَ.


\end{document}