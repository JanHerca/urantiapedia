\begin{document}

\title{تكوين}


\chapter{1}

\par 1 فِي الْبَدْءِ خَلَقَ اللهُ السَّمَاوَاتِ وَالارْضَ.
\par 2 وَكَانَتِ الارْضُ خَرِبَةً وَخَالِيَةً وَعَلَى وَجْهِ الْغَمْرِ ظُلْمَةٌ وَرُوحُ اللهِ يَرِفُّ عَلَى وَجْهِ الْمِيَاهِ.
\par 3 وَقَالَ اللهُ: «لِيَكُنْ نُورٌ» فَكَانَ نُورٌ.
\par 4 وَرَاى اللهُ النُّورَ انَّهُ حَسَنٌ. وَفَصَلَ اللهُ بَيْنَ النُّورِ وَالظُّلْمَةِ.
\par 5 وَدَعَا اللهُ النُّورَ نَهَارا وَالظُّلْمَةُ دَعَاهَا لَيْلا. وَكَانَ مَسَاءٌ وَكَانَ صَبَاحٌ يَوْما وَاحِدا.
\par 6 وَقَالَ اللهُ: «لِيَكُنْ جَلَدٌ فِي وَسَطِ الْمِيَاهِ. وَلْيَكُنْ فَاصِلا بَيْنَ مِيَاهٍ وَمِيَاهٍ».
\par 7 فَعَمِلَ اللهُ الْجَلَدَ وَفَصَلَ بَيْنَ الْمِيَاهِ الَّتِي تَحْتَ الْجَلَدِ وَالْمِيَاهِ الَّتِي فَوْقَ الْجَلَدِ. وَكَانَ كَذَلِكَ.
\par 8 وَدَعَا اللهُ الْجَلَدَ سَمَاءً. وَكَانَ مَسَاءٌ وَكَانَ صَبَاحٌ يَوْما ثَانِيا.
\par 9 وَقَالَ اللهُ: «لِتَجْتَمِعِ الْمِيَاهُ تَحْتَ السَّمَاءِ الَى مَكَانٍ وَاحِدٍ وَلْتَظْهَرِ الْيَابِسَةُ». وَكَانَ كَذَلِكَ.
\par 10 وَدَعَا اللهُ الْيَابِسَةَ ارْضا وَمُجْتَمَعَ الْمِيَاهِ دَعَاهُ بِحَارا. وَرَاى اللهُ ذَلِكَ انَّهُ حَسَنٌ.
\par 11 وَقَالَ اللهُ: «لِتُنْبِتِ الارْضُ عُشْبا وَبَقْلا يُبْزِرُ بِزْرا وَشَجَرا ذَا ثَمَرٍ يَعْمَلُ ثَمَرا كَجِنْسِهِ بِزْرُهُ فِيهِ عَلَى الارْضِ». وَكَانَ كَذَلِكَ.
\par 12 فَاخْرَجَتِ الارْضُ عُشْبا وَبَقْلا يُبْزِرُ بِزْرا كَجِنْسِهِ وَشَجَرا يَعْمَلُ ثَمَرا بِزْرُهُ فِيهِ كَجِنْسِهِ. وَرَاى اللهُ ذَلِكَ انَّهُ حَسَنٌ.
\par 13 وَكَانَ مَسَاءٌ وَكَانَ صَبَاحٌ يَوْما ثَالِثا.
\par 14 وَقَالَ اللهُ: «لِتَكُنْ انْوَارٌ فِي جَلَدِ السَّمَاءِ لِتَفْصِلَ بَيْنَ النَّهَارِ وَاللَّيْلِ وَتَكُونَ لايَاتٍ وَاوْقَاتٍ وَايَّامٍ وَسِنِينٍ.
\par 15 وَتَكُونَ انْوَارا فِي جَلَدِ السَّمَاءِ لِتُنِيرَ عَلَى الارْضِ». وَكَانَ كَذَلِكَ.
\par 16 فَعَمِلَ اللهُ النُّورَيْنِ الْعَظِيمَيْنِ: النُّورَ الاكْبَرَ لِحُكْمِ النَّهَارِ وَالنُّورَ الاصْغَرَ لِحُكْمِ اللَّيْلِ وَالنُّجُومَ.
\par 17 وَجَعَلَهَا اللهُ فِي جَلَدِ السَّمَاءِ لِتُنِيرَ عَلَى الارْضِ
\par 18 وَلِتَحْكُمَ عَلَى النَّهَارِ وَاللَّيْلِ وَلِتَفْصِلَ بَيْنَ النُّورِ وَالظُّلْمَةِ. وَرَاى اللهُ ذَلِكَ انَّهُ حَسَنٌ.
\par 19 وَكَانَ مَسَاءٌ وَكَانَ صَبَاحٌ يَوْما رَابِعا.
\par 20 وَقَالَ اللهُ: «لِتَفِضِ الْمِيَاهُ زَحَّافَاتٍ ذَاتَ نَفْسٍ حَيَّةٍ وَلْيَطِرْ طَيْرٌ فَوْقَ الارْضِ عَلَى وَجْهِ جَلَدِ السَّمَاءِ».
\par 21 فَخَلَقَ اللهُ التَّنَانِينَ الْعِظَامَ وَكُلَّ نَفْسٍ حَيَّةٍ تَدِبُّ الَّتِي فَاضَتْ بِهَا الْمِيَاهُ كَاجْنَاسِهَا وَكُلَّ طَائِرٍ ذِي جَنَاحٍ كَجِنْسِهِ. وَرَاى اللهُ ذَلِكَ انَّهُ حَسَنٌ.
\par 22 وَبَارَكَهَا اللهُ قَائِلا: «اثْمِرِي وَاكْثُرِي وَامْلاي الْمِيَاهَ فِي الْبِحَارِ. وَلْيَكْثُرِ الطَّيْرُ عَلَى الارْضِ».
\par 23 وَكَانَ مَسَاءٌ وَكَانَ صَبَاحٌ يَوْما خَامِسا.
\par 24 وَقَالَ اللهُ: «لِتُخْرِجِ الارْضُ ذَوَاتِ انْفُسٍ حَيَّةٍ كَجِنْسِهَا: بَهَائِمَ وَدَبَّابَاتٍ وَوُحُوشَ ارْضٍ كَاجْنَاسِهَا». وَكَانَ كَذَلِكَ.
\par 25 فَعَمِلَ اللهُ وُحُوشَ الارْضِ كَاجْنَاسِهَا وَالْبَهَائِمَ كَاجْنَاسِهَا وَجَمِيعَ دَبَّابَاتِ الارْضِ كَاجْنَاسِهَا. وَرَاى اللهُ ذَلِكَ انَّهُ حَسَنٌ.
\par 26 وَقَالَ اللهُ: «نَعْمَلُ الانْسَانَ عَلَى صُورَتِنَا كَشَبَهِنَا فَيَتَسَلَّطُونَ عَلَى سَمَكِ الْبَحْرِ وَعَلَى طَيْرِ السَّمَاءِ وَعَلَى الْبَهَائِمِ وَعَلَى كُلِّ الارْضِ وَعَلَى جَمِيعِ الدَّبَّابَاتِ الَّتِي تَدِبُّ عَلَى الارْضِ».
\par 27 فَخَلَقَ اللهُ الانْسَانَ عَلَى صُورَتِهِ. عَلَى صُورَةِ اللهِ خَلَقَهُ. ذَكَرا وَانْثَى خَلَقَهُمْ.
\par 28 وَبَارَكَهُمُ اللهُ وَقَالَ لَهُمْ: «اثْمِرُوا وَاكْثُرُوا وَامْلاوا الارْضَ وَاخْضِعُوهَا وَتَسَلَّطُوا عَلَى سَمَكِ الْبَحْرِ وَعَلَى طَيْرِ السَّمَاءِ وَعَلَى كُلِّ حَيَوَانٍ يَدِبُّ عَلَى الارْضِ».
\par 29 وَقَالَ اللهُ: «انِّي قَدْ اعْطَيْتُكُمْ كُلَّ بَقْلٍ يُبْزِرُ بِزْرا عَلَى وَجْهِ كُلِّ الارْضِ وَكُلَّ شَجَرٍ فِيهِ ثَمَرُ شَجَرٍ يُبْزِرُ بِزْرا لَكُمْ يَكُونُ طَعَاما.
\par 30 وَلِكُلِّ حَيَوَانِ الارْضِ وَكُلِّ طَيْرِ السَّمَاءِ وَكُلِّ دَبَّابَةٍ عَلَى الارْضِ فِيهَا نَفْسٌ حَيَّةٌ اعْطَيْتُ كُلَّ عُشْبٍ اخْضَرَ طَعَاما». وَكَانَ كَذَلِكَ.
\par 31 وَرَاى اللهُ كُلَّ مَا عَمِلَهُ فَاذَا هُوَ حَسَنٌ جِدّا. وَكَانَ مَسَاءٌ وَكَانَ صَبَاحٌ يَوْما سَادِسا.

\chapter{2}

\par 1 فَاكْمِلَتِ السَّمَاوَاتُ وَالارْضُ وَكُلُّ جُنْدِهَا.
\par 2 وَفَرَغَ اللهُ فِي الْيَوْمِ السَّابِعِ مِنْ عَمَلِهِ الَّذِي عَمِلَ. فَاسْتَرَاحَ فِي الْيَوْمِ السَّابِعِ مِنْ جَمِيعِ عَمَلِهِ الَّذِي عَمِلَ.
\par 3 وَبَارَكَ اللهُ الْيَوْمَ السَّابِعَ وَقَدَّسَهُ لانَّهُ فِيهِ اسْتَرَاحَ مِنْ جَمِيعِ عَمَلِهِ الَّذِي عَمِلَ اللهُ خَالِقا.
\par 4 هَذِهِ مَبَادِئُ السَّمَاوَاتِ وَالارْضِ حِينَ خُلِقَتْ يَوْمَ عَمِلَ الرَّبُّ الالَهُ الارْضَ وَالسَّمَاوَاتِ
\par 5 كُلُّ شَجَرِ الْبَرِّيَّةِ لَمْ يَكُنْ بَعْدُ فِي الارْضِ وَكُلُّ عُشْبِ الْبَرِّيَّةِ لَمْ يَنْبُتْ بَعْدُ لانَّ الرَّبَّ الالَهَ لَمْ يَكُنْ قَدْ امْطَرَ عَلَى الارْضِ وَلا كَانَ انْسَانٌ لِيَعْمَلَ الارْضَ.
\par 6 ثُمَّ كَانَ ضَبَابٌ يَطْلَعُ مِنَ الارْضِ وَيَسْقِي كُلَّ وَجْهِ الارْضِ.
\par 7 وَجَبَلَ الرَّبُّ الالَهُ ادَمَ تُرَابا مِنَ الارْضِ وَنَفَخَ فِي انْفِهِ نَسَمَةَ حَيَاةٍ. فَصَارَ ادَمُ نَفْسا حَيَّةً.
\par 8 وَغَرَسَ الرَّبُّ الالَهُ جَنَّةً فِي عَدْنٍ شَرْقا وَوَضَعَ هُنَاكَ ادَمَ الَّذِي جَبَلَهُ.
\par 9 وَانْبَتَ الرَّبُّ الالَهُ مِنَ الارْضِ كُلَّ شَجَرَةٍ شَهِيَّةٍ لِلنَّظَرِ وَجَيِّدَةٍ لِلاكْلِ وَشَجَرَةَ الْحَيَاةِ فِي وَسَطِ الْجَنَّةِ وَشَجَرَةَ مَعْرِفَةِ الْخَيْرِ وَالشَّرِّ.
\par 10 وَكَانَ نَهْرٌ يَخْرُجُ مِنْ عَدْنٍ لِيَسْقِيَ الْجَنَّةَ وَمِنْ هُنَاكَ يَنْقَسِمُ فَيَصِيرُ ارْبَعَةَ رُؤُوسٍ:
\par 11 اسْمُ الْوَاحِدِ فِيشُونُ وَهُوَ الْمُحِيطُ بِجَمِيعِ ارْضِ الْحَوِيلَةِ حَيْثُ الذَّهَبُ.
\par 12 وَذَهَبُ تِلْكَ الارْضِ جَيِّدٌ. هُنَاكَ الْمُقْلُ وَحَجَرُ الْجَزْعِ.
\par 13 وَاسْمُ النَّهْرِ الثَّانِي جِيحُونُ. وَهُوَ الْمُحِيطُ بِجَمِيعِ ارْضِ كُوشٍ.
\par 14 وَاسْمُ النَّهْرِ الثَّالِثِ حِدَّاقِلُ. وَهُوَ الْجَارِي شَرْقِيَّ اشُّورَ. وَالنَّهْرُ الرَّابِعُ الْفُرَاتُ.
\par 15 وَاخَذَ الرَّبُّ الالَهُ ادَمَ وَوَضَعَهُ فِي جَنَّةِ عَدْنٍ لِيَعْمَلَهَا وَيَحْفَظَهَا.
\par 16 وَاوْصَى الرَّبُّ الالَهُ ادَمَ قَائِلا: «مِنْ جَمِيعِ شَجَرِ الْجَنَّةِ تَاكُلُ اكْلا
\par 17 وَامَّا شَجَرَةُ مَعْرِفَةِ الْخَيْرِ وَالشَّرِّ فَلا تَاكُلْ مِنْهَا لانَّكَ يَوْمَ تَاكُلُ مِنْهَا مَوْتا تَمُوتُ».
\par 18 وَقَالَ الرَّبُّ الالَهُ: «لَيْسَ جَيِّدا انْ يَكُونَ ادَمُ وَحْدَهُ فَاصْنَعَ لَهُ مُعِينا نَظِيرَهُ».
\par 19 وَجَبَلَ الرَّبُّ الالَهُ مِنَ الارْضِ كُلَّ حَيَوَانَاتِ الْبَرِّيَّةِ وَكُلَّ طُيُورِ السَّمَاءِ فَاحْضَرَهَا الَى ادَمَ لِيَرَى مَاذَا يَدْعُوهَا وَكُلُّ مَا دَعَا بِهِ ادَمُ ذَاتَ نَفْسٍ حَيَّةٍ فَهُوَ اسْمُهَا.
\par 20 فَدَعَا ادَمُ بِاسْمَاءٍ جَمِيعَ الْبَهَائِمِ وَطُيُورَ السَّمَاءِ وَجَمِيعَ حَيَوَانَاتِ الْبَرِّيَّةِ. وَامَّا لِنَفْسِهِ فَلَمْ يَجِدْ مُعِينا نَظِيرَهُ.
\par 21 فَاوْقَعَ الرَّبُّ الالَهُ سُبَاتا عَلَى ادَمَ فَنَامَ فَاخَذَ وَاحِدَةً مِنْ اضْلاعِهِ وَمَلَا مَكَانَهَا لَحْما.
\par 22 وَبَنَى الرَّبُّ الالَهُ الضِّلْعَ الَّتِي اخَذَهَا مِنْ ادَمَ امْرَاةً وَاحْضَرَهَا الَى ادَمَ.
\par 23 فَقَالَ ادَمُ: «هَذِهِ الْانَ عَظْمٌ مِنْ عِظَامِي وَلَحْمٌ مِنْ لَحْمِي. هَذِهِ تُدْعَى امْرَاةً لانَّهَا مِنِ امْرِءٍ اخِذَتْ».
\par 24 لِذَلِكَ يَتْرُكُ الرَّجُلُ ابَاهُ وَامَّهُ وَيَلْتَصِقُ بِامْرَاتِهِ وَيَكُونَانِ جَسَدا وَاحِدا.
\par 25 وَكَانَا كِلاهُمَا عُرْيَانَيْنِ ادَمُ وَامْرَاتُهُ وَهُمَا لا يَخْجَلانِ.

\chapter{3}

\par 1 وَكَانَتِ الْحَيَّةُ احْيَلَ جَمِيعِ حَيَوَانَاتِ الْبَرِّيَّةِ الَّتِي عَمِلَهَا الرَّبُّ الالَهُ فَقَالَتْ لِلْمَرْاةِ: «احَقّا قَالَ اللهُ لا تَاكُلا مِنْ كُلِّ شَجَرِ الْجَنَّةِ؟»
\par 2 فَقَالَتِ الْمَرْاةُ لِلْحَيَّةِ: «مِنْ ثَمَرِ شَجَرِ الْجَنَّةِ نَاكُلُ
\par 3 وَامَّا ثَمَرُ الشَّجَرَةِ الَّتِي فِي وَسَطِ الْجَنَّةِ فَقَالَ اللهُ: لا تَاكُلا مِنْهُ وَلا تَمَسَّاهُ لِئَلَّا تَمُوتَا».
\par 4 فَقَالَتِ الْحَيَّةُ لِلْمَرْاةِ: «لَنْ تَمُوتَا!
\par 5 بَلِ اللهُ عَالِمٌ انَّهُ يَوْمَ تَاكُلانِ مِنْهُ تَنْفَتِحُ اعْيُنُكُمَا وَتَكُونَانِ كَاللهِ عَارِفَيْنِ الْخَيْرَ وَالشَّرَّ».
\par 6 فَرَاتِ الْمَرْاةُ انَّ الشَّجَرَةَ جَيِّدَةٌ لِلاكْلِ وَانَّهَا بَهِجَةٌ لِلْعُيُونِ وَانَّ الشَّجَرَةَ شَهِيَّةٌ لِلنَّظَرِ. فَاخَذَتْ مِنْ ثَمَرِهَا وَاكَلَتْ وَاعْطَتْ رَجُلَهَا ايْضا مَعَهَا فَاكَلَ.
\par 7 فَانْفَتَحَتْ اعْيُنُهُمَا وَعَلِمَا انَّهُمَا عُرْيَانَانِ. فَخَاطَا اوْرَاقَ تِينٍ وَصَنَعَا لانْفُسِهِمَا مَازِرَ.
\par 8 وَسَمِعَا صَوْتَ الرَّبِّ الالَهِ مَاشِيا فِي الْجَنَّةِ عِنْدَ هُبُوبِ رِيحِ النَّهَارِ فَاخْتَبَا ادَمُ وَامْرَاتُهُ مِنْ وَجْهِ الرَّبِّ الالَهِ فِي وَسَطِ شَجَرِ الْجَنَّةِ.
\par 9 فَنَادَى الرَّبُّ الالَهُ ادَمَ: «ايْنَ انْتَ؟».
\par 10 فَقَالَ: «سَمِعْتُ صَوْتَكَ فِي الْجَنَّةِ فَخَشِيتُ لانِّي عُرْيَانٌ فَاخْتَبَاتُ».
\par 11 فَقَالَ: «مَنْ اعْلَمَكَ انَّكَ عُرْيَانٌ؟ هَلْ اكَلْتَ مِنَ الشَّجَرَةِ الَّتِي اوْصَيْتُكَ انْ لا تَاكُلَ مِنْهَا؟»
\par 12 فَقَالَ ادَمُ: «الْمَرْاةُ الَّتِي جَعَلْتَهَا مَعِي هِيَ اعْطَتْنِي مِنَ الشَّجَرَةِ فَاكَلْتُ».
\par 13 فَقَالَ الرَّبُّ الالَهُ لِلْمَرْاةِ: «مَا هَذَا الَّذِي فَعَلْتِ؟» فَقَالَتِ الْمَرْاةُ: «الْحَيَّةُ غَرَّتْنِي فَاكَلْتُ».
\par 14 فَقَالَ الرَّبُّ الالَهُ لِلْحَيَّةِ: «لانَّكِ فَعَلْتِ هَذَا مَلْعُونَةٌ انْتِ مِنْ جَمِيعِ الْبَهَائِمِ وَمِنْ جَمِيعِ وُحُوشِ الْبَرِّيَّةِ. عَلَى بَطْنِكِ تَسْعِينَ وَتُرَابا تَاكُلِينَ كُلَّ ايَّامِ حَيَاتِكِ.
\par 15 وَاضَعُ عَدَاوَةً بَيْنَكِ وَبَيْنَ الْمَرْاةِ وَبَيْنَ نَسْلِكِ وَنَسْلِهَا. هُوَ يَسْحَقُ رَاسَكِ وَانْتِ تَسْحَقِينَ عَقِبَهُ».
\par 16 وَقَالَ لِلْمَرْاةِ: «تَكْثِيرا اكَثِّرُ اتْعَابَ حَبَلِكِ. بِالْوَجَعِ تَلِدِينَ اوْلادا. وَالَى رَجُلِكِ يَكُونُ اشْتِيَاقُكِ وَهُوَ يَسُودُ عَلَيْكِ».
\par 17 وَقَالَ لِادَمَ: «لانَّكَ سَمِعْتَ لِقَوْلِ امْرَاتِكَ وَاكَلْتَ مِنَ الشَّجَرَةِ الَّتِي اوْصَيْتُكَ قَائِلا: لا تَاكُلْ مِنْهَا مَلْعُونَةٌ الارْضُ بِسَبَبِكَ. بِالتَّعَبِ تَاكُلُ مِنْهَا كُلَّ ايَّامِ حَيَاتِكَ.
\par 18 وَشَوْكا وَحَسَكا تُنْبِتُ لَكَ وَتَاكُلُ عُشْبَ الْحَقْلِ.
\par 19 بِعَرَقِ وَجْهِكَ تَاكُلُ خُبْزا حَتَّى تَعُودَ الَى الارْضِ الَّتِي اخِذْتَ مِنْهَا. لانَّكَ تُرَابٌ وَالَى تُرَابٍ تَعُودُ».
\par 20 وَدَعَا ادَمُ اسْمَ امْرَاتِهِ «حَوَّاءَ» لانَّهَا امُّ كُلِّ حَيٍّ.
\par 21 وَصَنَعَ الرَّبُّ الالَهُ لِادَمَ وَامْرَاتِهِ اقْمِصَةً مِنْ جِلْدٍ وَالْبَسَهُمَا.
\par 22 وَقَالَ الرَّبُّ الالَهُ: «هُوَذَا الانْسَانُ قَدْ صَارَ كَوَاحِدٍ مِنَّا عَارِفا الْخَيْرَ وَالشَّرَّ. وَالْانَ لَعَلَّهُ يَمُدُّ يَدَهُ وَيَاخُذُ مِنْ شَجَرَةِ الْحَيَاةِ ايْضا وَيَاكُلُ وَيَحْيَا الَى الابَدِ».
\par 23 فَاخْرَجَهُ الرَّبُّ الالَهُ مِنْ جَنَّةِ عَدْنٍ لِيَعْمَلَ الارْضَ الَّتِي اخِذَ مِنْهَا.
\par 24 فَطَرَدَ الانْسَانَ وَاقَامَ شَرْقِيَّ جَنَّةِ عَدْنٍ الْكَرُوبِيمَ وَلَهِيبَ سَيْفٍ مُتَقَلِّبٍ لِحِرَاسَةِ طَرِيقِ شَجَرَةِ الْحَيَاةِ.

\chapter{4}

\par 1 وَعَرَفَ ادَمُ حَوَّاءَ امْرَاتَهُ فَحَبِلَتْ وَوَلَدَتْ قَايِينَ. وَقَالَتِ: «اقْتَنَيْتُ رَجُلا مِنْ عِنْدِ الرَّبِّ».
\par 2 ثُمَّ عَادَتْ فَوَلَدَتْ اخَاهُ هَابِيلَ. وَكَانَ هَابِيلُ رَاعِيا لِلْغَنَمِ وَكَانَ قَايِينُ عَامِلا فِي الارْضِ.
\par 3 وَحَدَثَ مِنْ بَعْدِ ايَّامٍ انَّ قَايِينَ قَدَّمَ مِنْ اثْمَارِ الارْضِ قُرْبَانا لِلرَّبِّ
\par 4 وَقَدَّمَ هَابِيلُ ايْضا مِنْ ابْكَارِ غَنَمِهِ وَمِنْ سِمَانِهَا. فَنَظَرَ الرَّبُّ الَى هَابِيلَ وَقُرْبَانِهِ
\par 5 وَلَكِنْ الَى قَايِينَ وَقُرْبَانِهِ لَمْ يَنْظُرْ. فَاغْتَاظَ قَايِينُ جِدّا وَسَقَطَ وَجْهُهُ.
\par 6 فَقَالَ الرَّبُّ لِقَايِينَ: «لِمَاذَا اغْتَظْتَ وَلِمَاذَا سَقَطَ وَجْهُكَ؟
\par 7 انْ احْسَنْتَ افَلا رَفْعٌ. وَانْ لَمْ تُحْسِنْ فَعِنْدَ الْبَابِ خَطِيَّةٌ رَابِضَةٌ وَالَيْكَ اشْتِيَاقُهَا وَانْتَ تَسُودُ عَلَيْهَا».
\par 8 وَكَلَّمَ قَايِينُ هَابِيلَ اخَاهُ. وَحَدَثَ اذْ كَانَا فِي الْحَقْلِ انَّ قَايِينَ قَامَ عَلَى هَابِيلَ اخِيهِ وَقَتَلَهُ.
\par 9 فَقَالَ الرَّبُّ لِقَايِينَ: «ايْنَ هَابِيلُ اخُوكَ؟» فَقَالَ: «لا اعْلَمُ! احَارِسٌ انَا لاخِي؟»
\par 10 فَقَالَ: «مَاذَا فَعَلْتَ؟ صَوْتُ دَمِ اخِيكَ صَارِخٌ الَيَّ مِنَ الارْضِ.
\par 11 فَالْانَ مَلْعُونٌ انْتَ مِنَ الارْضِ الَّتِي فَتَحَتْ فَاهَا لِتَقْبَلَ دَمَ اخِيكَ مِنْ يَدِكَ!
\par 12 مَتَى عَمِلْتَ الارْضَ لا تَعُودُ تُعْطِيكَ قُوَّتَهَا. تَائِها وَهَارِبا تَكُونُ فِي الارْضِ».
\par 13 فَقَالَ قَايِينُ لِلرَّبِّ: «ذَنْبِي اعْظَمُ مِنْ انْ يُحْتَمَلَ.
\par 14 انَّكَ قَدْ طَرَدْتَنِي الْيَوْمَ عَنْ وَجْهِ الارْضِ وَمِنْ وَجْهِكَ اخْتَفِي وَاكُونُ تَائِها وَهَارِبا فِي الارْضِ فَيَكُونُ كُلُّ مَنْ وَجَدَنِي يَقْتُلُنِي».
\par 15 فَقَالَ لَهُ الرَّبُّ: «لِذَلِكَ كُلُّ مَنْ قَتَلَ قَايِينَ فَسَبْعَةَ اضْعَافٍ يُنْتَقَمُ مِنْهُ». وَجَعَلَ الرَّبُّ لِقَايِينَ عَلامَةً لِكَيْ لا يَقْتُلَهُ كُلُّ مَنْ وَجَدَهُ.
\par 16 فَخَرَجَ قَايِينُ مِنْ لَدُنِ الرَّبِّ وَسَكَنَ فِي ارْضِ نُودٍ شَرْقِيَّ عَدْنٍ.
\par 17 وَعَرَفَ قَايِينُ امْرَاتَهُ فَحَبِلَتْ وَوَلَدَتْ حَنُوكَ. وَكَانَ يَبْنِي مَدِينَةً فَدَعَا اسْمَ الْمَدِينَةِ كَاسْمِ ابْنِهِ حَنُوكَ.
\par 18 وَوُلِدَ لِحَنُوكَ عِيرَادُ. وَعِيرَادُ وَلَدَ مَحُويَائِيلَ. وَمَحُويَائِيلُ وَلَدَ مَتُوشَائِيلَ. وَمَتُوشَائِيلُ وَلَدَ لامَكَ.
\par 19 وَاتَّخَذَ لامَكُ لِنَفْسِهِ امْرَاتَيْنِ: اسْمُ الْوَاحِدَةِ عَادَةُ وَاسْمُ الاخْرَى صِلَّةُ.
\par 20 فَوَلَدَتْ عَادَةُ يَابَالَ الَّذِي كَانَ ابا لِسَاكِنِي الْخِيَامِ وَرُعَاةِ الْمَوَاشِي.
\par 21 وَاسْمُ اخِيهِ يُوبَالُ الَّذِي كَانَ ابا لِكُلِّ ضَارِبٍ بِالْعُودِ وَالْمِزْمَارِ.
\par 22 وَصِلَّةُ ايْضا وَلَدَتْ تُوبَالَ قَايِينَ الضَّارِبَ كُلَّ الَةٍ مِنْ نُحَاسٍ وَحَدِيدٍ. وَاخْتُ تُوبَالَ قَايِينَ نَعْمَةُ.
\par 23 وَقَالَ لامَكُ لِامْرَاتَيْهِ عَادَةَ وَصِلَّةَ: «اسْمَعَا قَوْلِي يَا امْرَاتَيْ لامَكَ وَاصْغِيَا لِكَلامِي. فَانِّي قَتَلْتُ رَجُلا لِجُرْحِي وَفَتىً لِشَدْخِي.
\par 24 انَّهُ يُنْتَقَمُ لِقَايِينَ سَبْعَةَ اضْعَافٍ وَامَّا لِلامَكَ فَسَبْعَةً وَسَبْعِينَ».
\par 25 وَعَرَفَ ادَمُ امْرَاتَهُ ايْضا فَوَلَدَتِ ابْنا وَدَعَتِ اسْمَهُ شِيثا قَائِلَةً: «لانَّ اللهَ قَدْ وَضَعَ لِي نَسْلا اخَرَ عِوَضا عَنْ هَابِيلَ». لانَّ قَايِينَ كَانَ قَدْ قَتَلَهُ.
\par 26 وَلِشِيثَ ايْضا وُلِدَ ابْنٌ فَدَعَا اسْمَهُ انُوشَ. حِينَئِذٍ ابْتُدِئَ انْ يُدْعَى بِاسْمِ الرَّبِّ.

\chapter{5}

\par 1 هَذَا كِتَابُ مَوَالِيدِ ادَمَ يَوْمَ خَلَقَ اللهُ الانْسَانَ. عَلَى شَبَهِ اللهِ عَمِلَهُ.
\par 2 ذَكَرا وَانْثَى خَلَقَهُ وَبَارَكَهُ وَدَعَا اسْمَهُ ادَمَ يَوْمَ خُلِقَ.
\par 3 وَعَاشَ ادَمُ مِئَةً وَثَلاثِينَ سَنَةً وَوَلَدَ وَلَدا عَلَى شَبَهِهِ كَصُورَتِهِ وَدَعَا اسْمَهُ شِيثا.
\par 4 وَكَانَتْ ايَّامُ ادَمَ بَعْدَ مَا وَلَدَ شِيثا ثَمَانِيَ مِئَةِ سَنَةٍ وَوَلَدَ بَنِينَ وَبَنَاتٍ.
\par 5 فَكَانَتْ كُلُّ ايَّامِ ادَمَ الَّتِي عَاشَهَا تِسْعَ مِئَةٍ وَثَلاثِينَ سَنَةً وَمَاتَ.
\par 6 وَعَاشَ شِيثُ مِئَةً وَخَمْسَ سِنِينَ وَوَلَدَ انُوشَ.
\par 7 وَعَاشَ شِيثُ بَعْدَ مَا وَلَدَ انُوشَ ثَمَانِيَ مِئَةٍ وَسَبْعَ سِنِينَ وَوَلَدَ بَنِينَ وَبَنَاتٍ.
\par 8 فَكَانَتْ كُلُّ ايَّامِ شِيثَ تِسْعَ مِئَةٍ وَاثْنَتَيْ عَشَرَةَ سَنَةً وَمَاتَ.
\par 9 وَعَاشَ انُوشُ تِسْعِينَ سَنَةً وَوَلَدَ قِينَانَ.
\par 10 وَعَاشَ انُوشُ بَعْدَ مَا وَلَدَ قِينَانَ ثَمَانِيَ مِئَةٍ وَخَمْسَ عَشَرَةَ سَنَةً وَوَلَدَ بَنِينَ وَبَنَاتٍ.
\par 11 فَكَانَتْ كُلُّ ايَّامِ انُوشَ تِسْعَ مِئَةٍ وَخَمْسَ سِنِينَ وَمَاتَ.
\par 12 وَعَاشَ قِينَانُ سَبْعِينَ سَنَةً وَوَلَدَ مَهْلَلْئِيلَ.
\par 13 وَعَاشَ قِينَانُ بَعْدَ مَا وَلَدَ مَهْلَلْئِيلَ ثَمَانِيَ مِئَةٍ وَارْبَعِينَ سَنَةً وَوَلَدَ بَنِينَ وَبَنَاتٍ.
\par 14 فَكَانَتْ كُلُّ ايَّامِ قِينَانَ تِسْعَ مِئَةٍ وَعَشَرَ سِنِينَ وَمَاتَ.
\par 15 وَعَاشَ مَهْلَلْئِيلُ خَمْسا وَسِتِّينَ سَنَةً وَوَلَدَ يَارِدَ.
\par 16 وَعَاشَ مَهْلَلْئِيلُ بَعْدَ مَا وَلَدَ يَارِدَ ثَمَانِيَ مِئَةٍ وَثَلاثِينَ سَنَةً وَوَلَدَ بَنِينَ وَبَنَاتٍ.
\par 17 فَكَانَتْ كُلُّ ايَّامِ مَهْلَلْئِيلَ ثَمَانِيَ مِئَةٍ وَخَمْسا وَتِسْعِينَ سَنَةً وَمَاتَ.
\par 18 وَعَاشَ يَارِدُ مِئَةً وَاثْنَتَيْنِ وَسِتِّينَ سَنَةً وَوَلَدَ اخْنُوخَ.
\par 19 وَعَاشَ يَارِدُ بَعْدَ مَا وَلَدَ اخْنُوخَ ثَمَانِيَ مِئَةِ سَنَةٍ وَوَلَدَ بَنِينَ وَبَنَاتٍ.
\par 20 فَكَانَتْ كُلُّ ايَّامِ يَارِدَ تِسْعَ مِئَةٍ وَاثْنَتَيْنِ وَسِتِّينَ سَنَةً وَمَاتَ.
\par 21 وَعَاشَ اخْنُوخُ خَمْسا وَسِتِّينَ سَنَةً وَوَلَدَ مَتُوشَالَحَ.
\par 22 وَسَارَ اخْنُوخُ مَعَ اللهِ بَعْدَ مَا وَلَدَ مَتُوشَالَحَ ثَلاثَ مِئَةِ سَنَةٍ وَوَلَدَ بَنِينَ وَبَنَاتٍ.
\par 23 فَكَانَتْ كُلُّ ايَّامِ اخْنُوخَ ثَلاثَ مِئَةٍ وَخَمْسا وَسِتِّينَ سَنَةً.
\par 24 وَسَارَ اخْنُوخُ مَعَ اللهِ وَلَمْ يُوجَدْ لانَّ اللهَ اخَذَهُ.
\par 25 وَعَاشَ مَتُوشَالَحُ مِئَةً وَسَبْعا وَثَمَانِينَ سَنَةً وَوَلَدَ لامَكَ.
\par 26 وَعَاشَ مَتُوشَالَحُ بَعْدَ مَا وَلَدَ لامَكَ سَبْعَ مِئَةٍ وَاثْنَتَيْنِ وَثَمَانِينَ سَنَةً وَوَلَدَ بَنِينَ وَبَنَاتٍ.
\par 27 فَكَانَتْ كُلُّ ايَّامِ مَتُوشَالَحَ تِسْعَ مِئَةٍ وَتِسْعا وَسِتِّينَ سَنَةً وَمَاتَ.
\par 28 وَعَاشَ لامَكُ مِئَةً وَاثْنَتَيْنِ وَثَمَانِينَ سَنَةً وَوَلَدَ ابْنا.
\par 29 وَدَعَا اسْمَهُ نُوحا قَائِلا: «هَذَا يُعَزِّينَا عَنْ عَمَلِنَا وَتَعَبِ ايْدِينَا بِسَبَبِ الارْضِ الَّتِي لَعَنَهَا الرَّبُّ».
\par 30 وَعَاشَ لامَكُ بَعْدَ مَا وَلَدَ نُوحا خَمْسَ مِئَةٍ وَخَمْسا وَتِسْعِينَ سَنَةً وَوَلَدَ بَنِينَ وَبَنَاتٍ.
\par 31 فَكَانَتْ كُلُّ ايَّامِ لامَكَ سَبْعَ مِئَةٍ وَسَبْعا وَسَبْعِينَ سَنَةً وَمَاتَ.
\par 32 وَكَانَ نُوحٌ ابْنَ خَمْسِ مِئَةِ سَنَةٍ. وَوَلَدَ نُوحٌ: سَاما وَحَاما وَيَافَثَ.

\chapter{6}

\par 1 وَحَدَثَ لَمَّا ابْتَدَا النَّاسُ يَكْثُرُونَ عَلَى الارْضِ وَوُلِدَ لَهُمْ بَنَاتٌ
\par 2 انَّ ابْنَاءَ اللهِ رَاوا بَنَاتِ النَّاسِ انَّهُنَّ حَسَنَاتٌ. فَاتَّخَذُوا لانْفُسِهِمْ نِسَاءً مِنْ كُلِّ مَا اخْتَارُوا.
\par 3 فَقَالَ الرَّبُّ: «لا يَدِينُ رُوحِي فِي الانْسَانِ الَى الابَدِ. لِزَيَغَانِهِ هُوَ بَشَرٌ وَتَكُونُ ايَّامُهُ مِئَةً وَعِشْرِينَ سَنَةً».
\par 4 كَانَ فِي الارْضِ طُغَاةٌ فِي تِلْكَ الايَّامِ. وَبَعْدَ ذَلِكَ ايْضا اذْ دَخَلَ بَنُو اللهِ عَلَى بَنَاتِ النَّاسِ وَوَلَدْنَ لَهُمْ اوْلادا - هَؤُلاءِ هُمُ الْجَبَابِرَةُ الَّذِينَ مُنْذُ الدَّهْرِ ذَوُو اسْمٍ.
\par 5 وَرَاى الرَّبُّ انَّ شَرَّ الانْسَانِ قَدْ كَثُرَ فِي الارْضِ وَانَّ كُلَّ تَصَوُّرِ افْكَارِ قَلْبِهِ انَّمَا هُوَ شِرِّيرٌ كُلَّ يَوْمٍ.
\par 6 فَحَزِنَ الرَّبُّ انَّهُ عَمِلَ الانْسَانَ فِي الارْضِ وَتَاسَّفَ فِي قَلْبِهِ.
\par 7 فَقَالَ الرَّبُّ: «امْحُو عَنْ وَجْهِ الارْضِ الانْسَانَ الَّذِي خَلَقْتُهُ: الانْسَانَ مَعَ بَهَائِمَ وَدَبَّابَاتٍ وَطُيُورِ السَّمَاءِ. لانِّي حَزِنْتُ انِّي عَمِلْتُهُمْ».
\par 8 وَامَّا نُوحٌ فَوَجَدَ نِعْمَةً فِي عَيْنَيِ الرَّبِّ.
\par 9 هَذِهِ مَوَالِيدُ نُوحٍ: كَانَ نُوحٌ رَجُلا بَارّا كَامِلا فِي اجْيَالِهِ. وَسَارَ نُوحٌ مَعَ اللهِ.
\par 10 وَوَلَدَ نُوحٌ ثَلاثَةَ بَنِينَ: سَاما وَحَاما وَيَافَثَ.
\par 11 وَفَسَدَتِ الارْضُ امَامَ اللهِ وَامْتَلَاتِ الارْضُ ظُلْما.
\par 12 وَرَاى اللهُ الارْضَ فَاذَا هِيَ قَدْ فَسَدَتْ اذْ كَانَ كُلُّ بَشَرٍ قَدْ افْسَدَ طَرِيقَهُ عَلَى الارْضِ.
\par 13 فَقَالَ اللهُ لِنُوحٍ: «نِهَايَةُ كُلِّ بَشَرٍ قَدْ اتَتْ امَامِي لانَّ الارْضَ امْتَلَاتْ ظُلْما مِنْهُمْ. فَهَا انَا مُهْلِكُهُمْ مَعَ الارْضِ.
\par 14 اصْنَعْ لِنَفْسِكَ فُلْكا مِنْ خَشَبِ جُفْرٍ. تَجْعَلُ الْفُلْكَ مَسَاكِنَ وَتَطْلِيهِ مِنْ دَاخِلٍ وَمِنْ خَارِجٍ بِالْقَارِ.
\par 15 وَهَكَذَا تَصْنَعُهُ: ثَلاثَ مِئَةِ ذِرَاعٍ يَكُونُ طُولُ الْفُلْكِ وَخَمْسِينَ ذِرَاعا عَرْضُهُ وَثَلاثِينَ ذِرَاعا ارْتِفَاعُهُ.
\par 16 وَتَصْنَعُ كَوّا لِلْفُلْكِ وَتُكَمِّلُهُ الَى حَدِّ ذِرَاعٍ مِنْ فَوْقُ. وَتَضَعُ بَابَ الْفُلْكِ فِي جَانِبِهِ. مَسَاكِنَ سُفْلِيَّةً وَمُتَوَسِّطَةً وَعُلْوِيَّةً تَجْعَلُهُ.
\par 17 فَهَا انَا اتٍ بِطُوفَانِ الْمَاءِ عَلَى الارْضِ لِاهْلِكَ كُلَّ جَسَدٍ فِيهِ رُوحُ حَيَاةٍ مِنْ تَحْتِ السَّمَاءِ. كُلُّ مَا فِي الارْضِ يَمُوتُ.
\par 18 وَلَكِنْ اقِيمُ عَهْدِي مَعَكَ فَتَدْخُلُ الْفُلْكَ انْتَ وَبَنُوكَ وَامْرَاتُكَ وَنِسَاءُ بَنِيكَ مَعَكَ.
\par 19 وَمِنْ كُلِّ حَيٍّ مِنْ كُلِّ ذِي جَسَدٍ اثْنَيْنِ مِنْ كُلٍّ تُدْخِلُ الَى الْفُلْكِ لِاسْتِبْقَائِهَا مَعَكَ. تَكُونُ ذَكَرا وَانْثَى.
\par 20 مِنَ الطُّيُورِ كَاجْنَاسِهَا وَمِنَ الْبَهَائِمَ كَاجْنَاسِهَا وَمِنْ كُلِّ دَبابَاتِ الارْضِ كَاجْنَاسِهِ. اثْنَيْنِ مِنْ كُلٍّ تُدْخِلُ الَيْكَ لِاسْتِبْقَائِهَا.
\par 21 وَانْتَ فَخُذْ لِنَفْسِكَ مِنْ كُلِّ طَعَامٍ يُؤْكَلُ وَاجْمَعْهُ عِنْدَكَ فَيَكُونَ لَكَ وَلَهَا طَعَاما».
\par 22 فَفَعَلَ نُوحٌ حَسَبَ كُلِّ مَا امَرَهُ بِهِ اللهُ. هَكَذَا فَعَلَ.

\chapter{7}

\par 1 وَقَالَ الرَّبُّ لِنُوحٍ: «ادْخُلْ انْتَ وَجَمِيعُ بَيْتِكَ الَى الْفُلْكِ لانِّي ايَّاكَ رَايْتُ بَارّا لَدَيَّ فِي هَذَا الْجِيلِ.
\par 2 مِنْ جَمِيعِ الْبَهَائِمِ الطَّاهِرَةِ تَاخُذُ مَعَكَ سَبْعَةً سَبْعَةً ذَكَرا وَانْثَى. وَمِنَ الْبَهَائِمِ الَّتِي لَيْسَتْ بِطَاهِرَةٍ اثْنَيْنِ: ذَكَرا وَانْثَى.
\par 3 وَمِنْ طُيُورِ السَّمَاءِ ايْضا سَبْعَةً سَبْعَةً: ذَكَرا وَانْثَى. لِاسْتِبْقَاءِ نَسْلٍ عَلَى وَجْهِ كُلِّ الارْضِ.
\par 4 لانِّي بَعْدَ سَبْعَةِ ايَّامٍ ايْضا امْطِرُ عَلَى الارْضِ ارْبَعِينَ يَوْما وَارْبَعِينَ لَيْلَةً. وَامْحُو عَنْ وَجْهِ الارْضِ كُلَّ قَائِمٍ عَمِلْتُهُ».
\par 5 فَفَعَلَ نُوحٌ حَسَبَ كُلِّ مَا امَرَهُ بِهِ الرَّبُّ.
\par 6 وَلَمَّا كَانَ نُوحٌ ابْنَ سِتِّ مِئَةِ سَنَةٍ صَارَ طُوفَانُ الْمَاءِ عَلَى الارْضِ
\par 7 فَدَخَلَ نُوحٌ وَبَنُوهُ وَامْرَاتُهُ وَنِسَاءُ بَنِيهِ مَعَهُ الَى الْفُلْكِ مِنْ وَجْهِ مِيَاهِ الطُّوفَانِ.
\par 8 وَمِنَ الْبَهَائِمِ الطَّاهِرَةِ وَالْبَهَائِمِ الَّتِي لَيْسَتْ بِطَاهِرَةٍ وَمِنَ الطُّيُورِ وَكُلِّ مَا يَدِبُّ عَلَى الارْضِ:
\par 9 دَخَلَ اثْنَانِ اثْنَانِ الَى نُوحٍ الَى الْفُلْكِ ذَكَرا وَانْثَى. كَمَا امَرَ اللهُ نُوحا.
\par 10 وَحَدَثَ بَعْدَ السَّبْعَةِ الايَّامِ انَّ مِيَاهَ الطُّوفَانِ صَارَتْ عَلَى الارْضِ.
\par 11 فِي سَنَةِ سِتِّ مِئَةٍ مِنْ حَيَاةِ نُوحٍ فِي الشَّهْرِ الثَّانِي فِي الْيَوْمِ السَّابِعَ عَشَرَ مِنَ الشَّهْرِ انْفَجَرَتْ كُلُّ يَنَابِيعِ الْغَمْرِ الْعَظِيمِ وَانْفَتَحَتْ طَاقَاتُ السَّمَاءِ.
\par 12 وَكَانَ الْمَطَرُ عَلَى الارْضِ ارْبَعِينَ يَوْما وَارْبَعِينَ لَيْلَةً.
\par 13 فِي ذَلِكَ الْيَوْمِ عَيْنِهِ دَخَلَ نُوحٌ وَسَامٌ وَحَامٌ وَيَافَثُ بَنُو نُوحٍ وَامْرَاةُ نُوحٍ وَثَلاثُ نِسَاءِ بَنِيهِ مَعَهُمْ الَى الْفُلْكِ.
\par 14 هُمْ وَكُلُّ الْوُحُوشِ كَاجْنَاسِهَا وَكُلُّ الْبَهَائِمِ كَاجْنَاسِهَا وَكُلُّ الدَبَّابَاتِ الَّتِي تَدُبُّ عَلَى الارْضِ كَاجْنَاسِهَا وَكُلُّ الطُّيُورِ كَاجْنَاسِهَا: كُلُّ عُصْفُورٍ كُلُّ ذِي جَنَاحٍ.
\par 15 وَدَخَلَتْ الَى نُوحٍ الَى الْفُلْكِ اثْنَيْنِ اثْنَيْنِ مِنْ كُلِّ جَسَدٍ فِيهِ رُوحُ حَيَاةٍ.
\par 16 وَالدَّاخِلاتُ دَخَلَتْ ذَكَرا وَانْثَى مِنْ كُلِّ ذِي جَسَدٍ كَمَا امَرَهُ اللهُ. وَاغْلَقَ الرَّبُّ عَلَيْهِ.
\par 17 وَكَانَ الطُّوفَانُ ارْبَعِينَ يَوْما عَلَى الارْضِ. وَتَكَاثَرَتِ الْمِيَاهُ وَرَفَعَتِ الْفُلْكَ فَارْتَفَعَ عَنِ الارْضِ.
\par 18 وَتَعَاظَمَتِ الْمِيَاهُ وَتَكَاثَرَتْ جِدّا عَلَى الارْضِ فَكَانَ الْفُلْكُ يَسِيرُ عَلَى وَجْهِ الْمِيَاهِ.
\par 19 وَتَعَاظَمَتِ الْمِيَاهُ كَثِيرا جِدّا عَلَى الارْضِ فَتَغَطَّتْ جَمِيعُ الْجِبَالِ الشَّامِخَةِ الَّتِي تَحْتَ كُلِّ السَّمَاءِ.
\par 20 خَمْسَ عَشَرَةَ ذِرَاعا فِي الِارْتِفَاعِ تَعَاظَمَتِ الْمِيَاهُ فَتَغَطَّتِ الْجِبَالُ.
\par 21 فَمَاتَ كُلُّ ذِي جَسَدٍ كَانَ يَدِبُّ عَلَى الارْضِ مِنَ الطُّيُورِ وَالْبَهَائِمِ وَالْوُحُوشِ وَكُلُّ الزَّحَّافَاتِ الَّتِي كَانَتْ تَزْحَفُ عَلَى الارْضِ وَجَمِيعُ النَّاسِ.
\par 22 كُلُّ مَا فِي انْفِهِ نَسَمَةُ رُوحِ حَيَاةٍ مِنْ كُلِّ مَا فِي الْيَابِسَةِ مَاتَ.
\par 23 فَمَحَا اللهُ كُلَّ قَائِمٍ كَانَ عَلَى وَجْهِ الارْضِ: النَّاسَ وَالْبَهَائِمَ وَالدَّبَّابَاتَ وَطُيُورَ السَّمَاءِ فَانْمَحَتْ مِنَ الارْضِ. وَتَبَقَّى نُوحٌ وَالَّذِينَ مَعَهُ فِي الْفُلْكِ فَقَطْ.
\par 24 وَتَعَاظَمَتِ الْمِيَاهُ عَلَى الارْضِ مِئَةً وَخَمْسِينَ يَوْما.

\chapter{8}

\par 1 ثُمَّ ذَكَرَ اللهُ نُوحا وَكُلَّ الْوُحُوشِ وَكُلَّ الْبَهَائِمِ الَّتِي مَعَهُ فِي الْفُلْكِ. وَاجَازَ اللهُ رِيحا عَلَى الارْضِ فَهَدَاتِ الْمِيَاهُ.
\par 2 وَانْسَدَّتْ يَنَابِيعُ الْغَمْرِ وَطَاقَاتُ السَّمَاءِ فَامْتَنَعَ الْمَطَرُ مِنَ السَّمَاءِ.
\par 3 وَرَجَعَتِ الْمِيَاهُ عَنِ الارْضِ رُجُوعا مُتَوَالِيا. وَبَعْدَ مِئَةٍ وَخَمْسِينَ يَوْما نَقَصَتِ الْمِيَاهُ
\par 4 وَاسْتَقَرَّ الْفُلْكُ فِي الشَّهْرِ السَّابِعِ فِي الْيَوْمِ السَّابِعَ عَشَرَ مِنَ الشَّهْرِ عَلَى جِبَالِ ارَارَاطَ.
\par 5 وَكَانَتِ الْمِيَاهُ تَنْقُصُ نَقْصا مُتَوَالِيا الَى الشَّهْرِ الْعَاشِرِ. وَفِي الْعَاشِرِ فِي اوَّلِ الشَّهْرِ ظَهَرَتْ رُؤُوسُ الْجِبَالِ.
\par 6 وَحَدَثَ مِنْ بَعْدِ ارْبَعِينَ يَوْما انَّ نُوحا فَتَحَ طَاقَةَ الْفُلْكِ الَّتِي كَانَ قَدْ عَمِلَهَا
\par 7 وَارْسَلَ الْغُرَابَ فَخَرَجَ مُتَرَدِّدا حَتَّى نَشِفَتِ الْمِيَاهُ عَنِ الارْضِ.
\par 8 ثُمَّ ارْسَلَ الْحَمَامَةَ مِنْ عِنْدِهِ لِيَرَى هَلْ قَلَّتِ الْمِيَاهُ عَنْ وَجْهِ الارْضِ
\par 9 فَلَمْ تَجِدِ الْحَمَامَةُ مَقَرّا لِرِجْلِهَا فَرَجَعَتْ الَيْهِ الَى الْفُلْكِ لانَّ مِيَاها كَانَتْ عَلَى وَجْهِ كُلِّ الارْضِ. فَمَدَّ يَدَهُ وَاخَذَهَا وَادْخَلَهَا عِنْدَهُ الَى الْفُلْكِ.
\par 10 فَلَبِثَ ايْضا سَبْعَةَ ايَّامٍ اخَرَ وَعَادَ فَارْسَلَ الْحَمَامَةَ مِنَ الْفُلْكِ
\par 11 فَاتَتْ الَيْهِ الْحَمَامَةُ عِنْدَ الْمَسَاءِ وَاذَا وَرَقَةُ زَيْتُونٍ خَضْرَاءُ فِي فَمِهَا. فَعَلِمَ نُوحٌ انَّ الْمِيَاهَ قَدْ قَلَّتْ عَنِ الارْضِ.
\par 12 فَلَبِثَ ايْضا سَبْعَةَ ايَّامٍ اخَرَ وَارْسَلَ الْحَمَامَةَ فَلَمْ تَعُدْ تَرْجِعُ الَيْهِ ايْضا.
\par 13 وَكَانَ فِي السَّنَةِ الْوَاحِدَةِ وَالسِّتِّ مِئَةٍ فِي الشَّهْرِ الاوَّلِ فِي اوَّلِ الشَّهْرِ انَّ الْمِيَاهَ نَشِفَتْ عَنِ الارْضِ. فَكَشَفَ نُوحٌ الْغِطَاءَ عَنِ الْفُلْكِ وَنَظَرَ فَاذَا وَجْهُ الارْضِ قَدْ نَشِفَ.
\par 14 وَفِي الشَّهْرِ الثَّانِي فِي الْيَوْمِ السَّابِعِ وَالْعِشْرِينَ مِنَ الشَّهْرِ جَفَّتِ الارْضُ.
\par 15 وَامَرَ اللهُ نُوحا:
\par 16 «اخْرُجْ مِنَ الْفُلْكِ انْتَ وَامْرَاتُكَ وَبَنُوكَ وَنِسَاءُ بَنِيكَ مَعَكَ.
\par 17 وَكُلَّ الْحَيَوَانَاتِ الَّتِي مَعَكَ مِنْ كُلِّ ذِي جَسَدٍ: الطُّيُورِ وَالْبَهَائِمِ وَكُلَّ الدَّبَّابَاتِ الَّتِي تَدُبُّ عَلَى الارْضِ اخْرِجْهَا مَعَكَ. وَلْتَتَوَالَدْ فِي الارْضِ وَتُثْمِرْ وَتَكْثُرْ عَلَى الارْضِ».
\par 18 فَخَرَجَ نُوحٌ وَبَنُوهُ وَامْرَاتُهُ وَنِسَاءُ بَنِيهِ مَعَهُ.
\par 19 وَكُلُّ الْحَيَوَانَاتِ وَكُلُّ الطُّيُورِ كُلُّ مَا يَدِبُّ عَلَى الارْضِ كَانْوَاعِهَا خَرَجَتْ مِنَ الْفُلْكِ.
\par 20 وَبَنَى نُوحٌ مَذْبَحا لِلرَّبِّ. وَاخَذَ مِنْ كُلِّ الْبَهَائِمِ الطَّاهِرَةِ وَمِنْ كُلِّ الطُّيُورِ الطَّاهِرَةِ وَاصْعَدَ مُحْرَقَاتٍ عَلَى الْمَذْبَحِ
\par 21 فَتَنَسَّمَ الرَّبُّ رَائِحَةَ الرِّضَا. وَقَالَ الرَّبُّ فِي قَلْبِهِ: «لا اعُودُ الْعَنُ الارْضَ ايْضا مِنْ اجْلِ الانْسَانِ لانَّ تَصَوُّرَ قَلْبِ الانْسَانِ شِرِّيرٌ مُنْذُ حَدَاثَتِهِ. وَلا اعُودُ ايْضا امِيتُ كُلَّ حَيٍّ كَمَا فَعَلْتُ.
\par 22 مُدَّةَ كُلِّ ايَّامِ الارْضِ زَرْعٌ وَحَصَادٌ وَبَرْدٌ وَحَرٌّ وَصَيْفٌ وَشِتَاءٌ وَنَهَارٌ وَلَيْلٌ لا تَزَالُ».

\chapter{9}

\par 1 وَبَارَكَ اللهُ نُوحا وَبَنِيهِ وَقَالَ لَهُمْ: «اثْمِرُوا وَاكْثُرُوا وَامْلَاوا الارْضَ.
\par 2 وَلْتَكُنْ خَشْيَتُكُمْ وَرَهْبَتُكُمْ عَلَى كُلِّ حَيَوَانَاتِ الارْضِ وَكُلِّ طُيُورِ السَّمَاءِ مَعَ كُلِّ مَا يَدِبُّ عَلَى الارْضِ وَكُلِّ اسْمَاكِ الْبَحْرِ. قَدْ دُفِعَتْ الَى ايْدِيكُمْ.
\par 3 كُلُّ دَابَّةٍ حَيَّةٍ تَكُونُ لَكُمْ طَعَاما. كَالْعُشْبِ الاخْضَرِ دَفَعْتُ الَيْكُمُ الْجَمِيعَ.
\par 4 غَيْرَ انَّ لَحْما بِحَيَاتِهِ دَمِهِ لا تَاكُلُوهُ.
\par 5 وَاطْلُبُ انَا دَمَكُمْ لانْفُسِكُمْ فَقَطْ. مِنْ يَدِ كُلِّ حَيَوَانٍ اطْلُبُهُ. وَمِنْ يَدِ الانْسَانِ اطْلُبُ نَفْسَ الانْسَانِ مِنْ يَدِ الانْسَانِ اخِيهِ.
\par 6 سَافِكُ دَمِ الانْسَانِ بِالانْسَانِ يُسْفَكُ دَمُهُ. لانَّ اللهَ عَلَى صُورَتِهِ عَمِلَ الانْسَانَ.
\par 7 فَاثْمِرُوا انْتُمْ وَاكْثُرُوا وَتَوَالَدُوا فِي الارْضِ وَتَكَاثَرُوا فِيهَا».
\par 8 وَقَالَ اللهُ لِنُوحٍ وَبَنِيهِ:
\par 9 «وَهَا انَا مُقِيمٌ مِيثَاقِي مَعَكُمْ وَمَعَ نَسْلِكُمْ مِنْ بَعْدِكُمْ
\par 10 وَمَعَ كُلِّ ذَوَاتِ الانْفُسِ الْحَيَّةِ الَّتِي مَعَكُمِْ: الطُّيُورِ وَالْبَهَائِمِ وَكُلِّ وُحُوشِ الارْضِ الَّتِي مَعَكُمْ مِنْ جَمِيعِ الْخَارِجِينَ مِنَ الْفُلْكِ حَتَّى كُلُّ حَيَوَانِ الارْضِ.
\par 11 اقِيمُ مِيثَاقِي مَعَكُمْ فَلا يَنْقَرِضُ كُلُّ ذِي جَسَدٍ ايْضا بِمِيَاهِ الطُّوفَانِ. وَلا يَكُونُ ايْضا طُوفَانٌ لِيُخْرِبَ الارْضَ».
\par 12 وَقَالَ اللهُ: «هَذِهِ عَلامَةُ الْمِيثَاقِ الَّذِي انَا وَاضِعُهُ بَيْنِي وَبَيْنَكُمْ وَبَيْنَ كُلِّ ذَوَاتِ الانْفُسِ الْحَيَّةِ الَّتِي مَعَكُمْ الَى اجْيَالِ الدَّهْرِ:
\par 13 وَضَعْتُ قَوْسِي فِي السَّحَابِ فَتَكُونُ عَلامَةَ مِيثَاقٍ بَيْنِي وَبَيْنَ الارْضِ.
\par 14 فَيَكُونُ مَتَى انْشُرْ سَحَابا عَلَى الارْضِ وَتَظْهَرِ الْقَوْسُ فِي السَّحَابِ
\par 15 انِّي اذْكُرُ مِيثَاقِي الَّذِي بَيْنِي وَبَيْنَكُمْ وَبَيْنَ كُلِّ نَفْسٍ حَيَّةٍ فِي كُلِّ جَسَدٍ. فَلا تَكُونُ ايْضا الْمِيَاهُ طُوفَانا لِتُهْلِكَ كُلَّ ذِي جَسَدٍ.
\par 16 فَمَتَى كَانَتِ الْقَوْسُ فِي السَّحَابِ ابْصِرُهَا لاذْكُرَ مِيثَاقا ابَدِيّا بَيْنَ اللهِ وَبَيْنَ كُلِّ نَفْسٍ حَيَّةٍ فِي كُلِّ جَسَدٍ عَلَى الارْضِ».
\par 17 وَقَالَ اللهُ لِنُوحٍ: «هَذِهِ عَلامَةُ الْمِيثَاقِ الَّذِي انَا اقَمْتُهُ بَيْنِي وَبَيْنَ كُلِّ ذِي جَسَدٍ عَلَى الارْضِ».
\par 18 وَكَانَ بَنُو نُوحٍ الَّذِينَ خَرَجُوا مِنَ الْفُلْكِ سَاما وَحَاما وَيَافَثَ. وَحَامٌ هُوَ ابُو كَنْعَانَ.
\par 19 هَؤُلاءِ الثَّلاثَةُ هُمْ بَنُو نُوحٍ. وَمِنْ هَؤُلاءِ تَشَعَّبَتْ كُلُّ الارْضِ.
\par 20 وَابْتَدَا نُوحٌ يَكُونُ فَلَّاحا وَغَرَسَ كَرْما.
\par 21 وَشَرِبَ مِنَ الْخَمْرِ فَسَكِرَ وَتَعَرَّى دَاخِلَ خِبَائِهِ.
\par 22 فَابْصَرَ حَامٌ ابُو كَنْعَانَ عَوْرَةَ ابِيهِ وَاخْبَرَ اخَوَيْهِ خَارِجا.
\par 23 فَاخَذَ سَامٌ وَيَافَثُ الرِّدَاءَ وَوَضَعَاهُ عَلَى اكْتَافِهِمَا وَمَشَيَا الَى الْوَرَاءِ وَسَتَرَا عَوْرَةَ ابِيهِمَا وَوَجْهَاهُمَا الَى الْوَرَاءِ. فَلَمْ يُبْصِرَا عَوْرَةَ ابِيهِمَا.
\par 24 فَلَمَّا اسْتَيْقَظَ نُوحٌ مِنْ خَمْرِهِ عَلِمَ مَا فَعَلَ بِهِ ابْنُهُ الصَّغِيرُ
\par 25 فَقَالَ: «مَلْعُونٌ كَنْعَانُ. عَبْدَ الْعَبِيدِ يَكُونُ لاخْوَتِهِ».
\par 26 وَقَالَ: «مُبَارَكٌ الرَّبُّ الَهُ سَامٍ. وَلْيَكُنْ كَنْعَانُ عَبْدا لَهُ.
\par 27 لِيَفْتَحِ اللهُ لِيَافَثَ فَيَسْكُنَ فِي مَسَاكِنِ سَامٍ. وَلْيَكُنْ كَنْعَانُ عَبْدا لَهُمْ».
\par 28 وَعَاشَ نُوحٌ بَعْدَ الطُّوفَانِ ثَلاثَ مِئَةٍ وَخَمْسِينَ سَنَةً.
\par 29 فَكَانَتْ كُلُّ ايَّامِ نُوحٍ تِسْعَ مِئَةٍ وَخَمْسِينَ سَنَةً وَمَاتَ.

\chapter{10}

\par 1 وَهَذِهِ مَوَالِيدُ بَنِي نُوحٍ: سَامٌ وَحَامٌ وَيَافَثُ. وَوُلِدَ لَهُمْ بَنُونَ بَعْدَ الطُّوفَانِ.
\par 2 بَنُو يَافَثَ: جُومَرُ وَمَاجُوجُ وَمَادَاي وَيَاوَانُ وَتُوبَالُ وَمَاشِكُ وَتِيرَاسُ.
\par 3 وَبَنُو جُومَرَ: اشْكَنَازُ وَرِيفَاثُ وَتُوجَرْمَةُ.
\par 4 وَبَنُو يَاوَانَ: الِيشَةُ وَتَرْشِيشُ وَكِتِّيمُ وَدُودَانِيمُ.
\par 5 مِنْ هَؤُلاءِ تَفَرَّقَتْ جَزَائِرُ الامَمِ بِارَاضِيهِمْ كُلُّ انْسَانٍ كَلِسَانِهِ حَسَبَ قَبَائِلِهِمْ بِامَمِهِمْ.
\par 6 وَبَنُو حَامٍ: كُوشُ وَمِصْرَايِمُ وَفُوطُ وَكَنْعَانُ.
\par 7 وَبَنُو كُوشَ: سَبَا وَحَوِيلَةُ وَسَبْتَةُ وَرَعْمَةُ وَسَبْتَكَا. وَبَنُو رَعْمَةَ: شَبَا وَدَدَانُ.
\par 8 وَكُوشُ وَلَدَ نِمْرُودَ الَّذِي ابْتَدَا يَكُونُ جَبَّارا فِي الارْضِ
\par 9 الَّذِي كَانَ جَبَّارَ صَيْدٍ امَامَ الرَّبِّ. لِذَلِكَ يُقَالُ: «كَنِمْرُودَ جَبَّارُ صَيْدٍ امَامَ الرَّبِّ».
\par 10 وَكَانَ ابْتِدَاءُ مَمْلَكَتِهِ بَابِلَ وَارَكَ وَاكَّدَ وَكَلْنَةَ فِي ارْضِ شِنْعَارَ.
\par 11 مِنْ تِلْكَ الارْضِ خَرَجَ اشُّورُ وَبَنَى نِينَوَى وَرَحُوبُوتَ عَيْرَ وَكَالَحَ
\par 12 وَرَسَنَ بَيْنَ نِينَوَى وَكَالَحَ. (هِيَ الْمَدِينَةُ الْكَبِيرَةُ).
\par 13 وَمِصْرَايِمُ وَلَدَ: لُودِيمَ وَعَنَامِيمَ وَلَهَابِيمَ وَنَفْتُوحِيمَ
\par 14 وَفَتْرُوسِيمَ وَكَسْلُوحِيمَ. (الَّذِينَ خَرَجَ مِنْهُمْ فِلِشْتِيمُ وَكَفْتُورِيمُ).
\par 15 وَكَنْعَانُ وَلَدَ: صَيْدُونَ بِكْرَهُ وَحِثَّ
\par 16 وَالْيَبُوسِيَّ وَالامُورِيَّ وَالْجِرْجَاشِيَّ
\par 17 وَالْحِوِّيَّ وَالْعَرْقِيَّ وَالسِّينِيَّ
\par 18 وَالارْوَادِيَّ وَالصَّمَارِيَّ وَالْحَمَاتِيَّ. وَبَعْدَ ذَلِكَ تَفَرَّقَتْ قَبَائِلُ الْكَنْعَانِيِّ.
\par 19 وَكَانَتْ تُخُومُ الْكَنْعَانِيِّ مِنْ صَيْدُونَ حِينَمَا تَجِيءُ نَحْوَ جَرَارَ الَى غَزَّةَ وَحِينَمَا تَجِيءُ نَحْوَ سَدُومَ وَعَمُورَةَ وَادْمَةَ وَصَبُويِيمَ الَى لاشَعَ.
\par 20 هَؤُلاءِ بَنُو حَامٍ حَسَبَ قَبَائِلِهِمْ كَالْسِنَتِهِمْ بِارَاضِيهِمْ وَامَمِهِمْ.
\par 21 وَسَامٌ ابُو كُلِّ بَنِي عَابِرَ اخُو يَافَثَ الْكَبِيرُ وُلِدَ لَهُ ايْضا بَنُونَ.
\par 22 بَنُو سَامَ: عِيلامُ وَاشُّورُ وَارْفَكْشَادُ وَلُودُ وَارَامُ.
\par 23 وَبَنُو ارَامَ: عُوصُ وَحُولُ وَجَاثَرُ وَمَاشُ.
\par 24 وَارْفَكْشَادُ وَلَدَ شَالَحَ وَشَالَحُ وَلَدَ عَابِرَ.
\par 25 وَلِعَابِرَ وُلِدَ ابْنَانِ: اسْمُ الْوَاحِدِ فَالَجُ لانَّ فِي ايَّامِهِ قُسِمَتِ الارْضُ. وَاسْمُ اخِيهِ يَقْطَانُ.
\par 26 وَيَقْطَانُ وَلَدَ الْمُودَادَ وَشَالَفَ وَحَضَرْمَوْتَ وَيَارَحَ
\par 27 وَهَدُورَامَ وَاوزَالَ وَدِقْلَةَ
\par 28 وَعُوبَالَ وَابِيمَايِلَ وَشَبَا
\par 29 وَاوفِيرَ وَحَوِيلَةَ وَيُوبَابَ. جَمِيعُ هَؤُلاءِ بَنُو يَقْطَانَ.
\par 30 وَكَانَ مَسْكَنُهُمْ مِنْ مِيشَا حِينَمَا تَجِيءُ نَحْوَ سَفَارَ جَبَلِ الْمَشْرِقِ.
\par 31 هَؤُلاءِ بَنُو سَامَ حَسَبَ قَبَائِلِهِمْ كَالْسِنَتِهِمْ بِارَاضِيهِمْ حَسَبَ امَمِهِمْ.
\par 32 هَؤُلاءِ قَبَائِلُ بَنِي نُوحٍ حَسَبَ مَوَالِيدِهِمْ بِامَمِهِمْ. وَمِنْ هَؤُلاءِ تَفَرَّقَتِ الامَمُ فِي الارْضِ بَعْدَ الطُّوفَانِ.

\chapter{11}

\par 1 وَكَانَتِ الارْضُ كُلُّهَا لِسَانا وَاحِدا وَلُغَةً وَاحِدَةً.
\par 2 وَحَدَثَ فِي ارْتِحَالِهِمْ شَرْقا انَّهُمْ وَجَدُوا بُقْعَةً فِي ارْضِ شِنْعَارَ وَسَكَنُوا هُنَاكَ.
\par 3 وَقَالَ بَعْضُهُمْ لِبَعْضٍ: «هَلُمَّ نَصْنَعُ لِبْنا وَنَشْوِيهِ شَيّا». فَكَانَ لَهُمُ اللِّبْنُ مَكَانَ الْحَجَرِ وَكَانَ لَهُمُ الْحُمَرُ مَكَانَ الطِّينِ.
\par 4 وَقَالُوا: «هَلُمَّ نَبْنِ لانْفُسِنَا مَدِينَةً وَبُرْجا رَاسُهُ بِالسَّمَاءِ. وَنَصْنَعُ لانْفُسِنَا اسْما لِئَلَّا نَتَبَدَّدَ عَلَى وَجْهِ كُلِّ الارْضِ».
\par 5 فَنَزَلَ الرَّبُّ لِيَنْظُرَ الْمَدِينَةَ وَالْبُرْجَ اللَّذَيْنِ كَانَ بَنُو ادَمَ يَبْنُونَهُمَا.
\par 6 وَقَالَ الرَّبُّ: «هُوَذَا شَعْبٌ وَاحِدٌ وَلِسَانٌ وَاحِدٌ لِجَمِيعِهِمْ وَهَذَا ابْتِدَاؤُهُمْ بِالْعَمَلِ. وَالْانَ لا يَمْتَنِعُ عَلَيْهِمْ كُلُّ مَا يَنْوُونَ انْ يَعْمَلُوهُ.
\par 7 هَلُمَّ نَنْزِلْ وَنُبَلْبِلْ هُنَاكَ لِسَانَهُمْ حَتَّى لا يَسْمَعَ بَعْضُهُمْ لِسَانَ بَعْضٍ».
\par 8 فَبَدَّدَهُمُ الرَّبُّ مِنْ هُنَاكَ عَلَى وَجْهِ كُلِّ الارْضِ فَكَفُّوا عَنْ بُنْيَانِ الْمَدِينَةِ
\par 9 لِذَلِكَ دُعِيَ اسْمُهَا «بَابِلَ» لانَّ الرَّبَّ هُنَاكَ بَلْبَلَ لِسَانَ كُلِّ الارْضِ. وَمِنْ هُنَاكَ بَدَّدَهُمُ الرَّبُّ عَلَى وَجْهِ كُلِّ الارْضِ.
\par 10 هَذِهِ مَوَالِيدُ سَامٍ: لَمَّا كَانَ سَامٌ ابْنَ مِئَةِ سَنَةٍ وَلَدَ ارْفَكْشَادَ بَعْدَ الطُّوفَانِ بِسَنَتَيْنِ.
\par 11 وَعَاشَ سَامٌ بَعْدَ مَا وَلَدَ ارْفَكْشَادَ خَمْسَ مِئَةِ سَنَةٍ وَوَلَدَ بَنِينَ وَبَنَاتٍ.
\par 12 وَعَاشَ ارْفَكْشَادُ خَمْسا وَثَلاثِينَ سَنَةً وَوَلَدَ شَالَحَ.
\par 13 وَعَاشَ ارْفَكْشَادُ بَعْدَ مَا وَلَدَ شَالَحَ ارْبَعَ مِئَةٍ وَثَلاثَ سِنِينَ وَوَلَدَ بَنِينَ وَبَنَاتٍ.
\par 14 وَعَاشَ شَالَحُ ثَلاثِينَ سَنَةً وَوَلَدَ عَابِرَ.
\par 15 وَعَاشَ شَالَحُ بَعْدَ مَا وَلَدَ عَابِرَ ارْبَعَ مِئَةٍ وَثَلاثَ سِنِينَ وَوَلَدَ بَنِينَ وَبَنَاتٍ.
\par 16 وَعَاشَ عَابِرُ ارْبَعا وَثَلاثِينَ سَنَةً وَوَلَدَ فَالَجَ.
\par 17 وَعَاشَ عَابِرُ بَعْدَ مَا وَلَدَ فَالَجَ ارْبَعَ مِئَةٍ وَثَلاثِينَ سَنَةً وَوَلَدَ بَنِينَ وَبَنَاتٍ.
\par 18 وَعَاشَ فَالَجُ ثَلاثِينَ سَنَةً وَوَلَدَ رَعُوَ.
\par 19 وَعَاشَ فَالَجُ بَعْدَ مَا وَلَدَ رَعُوَ مِئَتَيْنِ وَتِسْعَ سِنِينَ وَوَلَدَ بَنِينَ وَبَنَاتٍ.
\par 20 وَعَاشَ رَعُو اثْنَتَيْنِ وَثَلاثِينَ سَنَةً وَوَلَدَ سَرُوجَ.
\par 21 وَعَاشَ رَعُو بَعْدَ مَا وَلَدَ سَرُوجَ مِئَتَيْنِ وَسَبْعَ سِنِينَ وَوَلَدَ بَنِينَ وَبَنَاتٍ.
\par 22 وَعَاشَ سَرُوجُ ثَلاثِينَ سَنَةً وَوَلَدَ نَاحُورَ.
\par 23 وَعَاشَ سَرُوجُ بَعْدَ مَا وَلَدَ نَاحُورَ مِئَتَيْ سَنَةٍ وَوَلَدَ بَنِينَ وَبَنَاتٍ.
\par 24 وَعَاشَ نَاحُورُ تِسْعا وَعِشْرِينَ سَنَةً وَوَلَدَ تَارَحَ.
\par 25 وَعَاشَ نَاحُورُ بَعْدَ مَا وَلَدَ تَارَحَ مِئَةً وَتِسْعَ عَشَرَةَ سَنَةً وَوَلَدَ بَنِينَ وَبَنَاتٍ.
\par 26 وَعَاشَ تَارَحُ سَبْعِينَ سَنَةً وَوَلَدَ ابْرَامَ وَنَاحُورَ وَهَارَانَ.
\par 27 وَهَذِهِ مَوَالِيدُ تَارَحَ: وَلَدَ تَارَحُ ابْرَامَ وَنَاحُورَ وَهَارَانَ. وَوَلَدَ هَارَانُ لُوطا.
\par 28 وَمَاتَ هَارَانُ قَبْلَ تَارَحَ ابِيهِ فِي ارْضِ مِيلادِهِ فِي اورِ الْكِلْدَانِيِّينَ.
\par 29 وَاتَّخَذَ ابْرَامُ وَنَاحُورُ لَهُمَا امْرَاتَيْنِ: اسْمُ امْرَاةِ ابْرَامَ سَارَايُ وَاسْمُ امْرَاةِ نَاحُورَ مِلْكَةُ بِنْتُ هَارَانَ ابِي مِلْكَةَ وَابِي يِسْكَةَ.
\par 30 وَكَانَتْ سَارَايُ عَاقِرا لَيْسَ لَهَا وَلَدٌ.
\par 31 وَاخَذَ تَارَحُ ابْرَامَ ابْنَهُ وَلُوطا بْنَ هَارَانَ ابْنَ ابْنِهِ وَسَارَايَ كَنَّتَهُ امْرَاةَ ابْرَامَ ابْنِهِ فَخَرَجُوا مَعا مِنْ اورِ الْكِلْدَانِيِّينَ لِيَذْهَبُوا الَى ارْضِ كَنْعَانَ. فَاتُوا الَى حَارَانَ وَاقَامُوا هُنَاكَ.
\par 32 وَكَانَتْ ايَّامُ تَارَحَ مِئَتَيْنِ وَخَمْسَ سِنِينَ. وَمَاتَ تَارَحُ فِي حَارَانَ.

\chapter{12}

\par 1 وَقَالَ الرَّبُّ لابْرَامَ: «اذْهَبْ مِنْ ارْضِكَ وَمِنْ عَشِيرَتِكَ وَمِنْ بَيْتِ ابِيكَ الَى الارْضِ الَّتِي ارِيكَ.
\par 2 فَاجْعَلَكَ امَّةً عَظِيمَةً وَابَارِكَكَ وَاعَظِّمَ اسْمَكَ وَتَكُونَ بَرَكَةً.
\par 3 وَابَارِكُ مُبَارِكِيكَ وَلاعِنَكَ الْعَنُهُ. وَتَتَبَارَكُ فِيكَ جَمِيعُ قَبَائِلِ الارْضِ».
\par 4 فَذَهَبَ ابْرَامُ كَمَا قَالَ لَهُ الرَّبُّ وَذَهَبَ مَعَهُ لُوطٌ. وَكَانَ ابْرَامُ ابْنَ خَمْسٍ وَسَبْعِينَ سَنَةً لَمَّا خَرَجَ مِنْ حَارَانَ.
\par 5 فَاخَذَ ابْرَامُ سَارَايَ امْرَاتَهُ وَلُوطا ابْنَ اخِيهِ وَكُلَّ مُقْتَنَيَاتِهِمَا الَّتِي اقْتَنَيَا وَالنُّفُوسَ الَّتِي امْتَلَكَا فِي حَارَانَ. وَخَرَجُوا لِيَذْهَبُوا الَى ارْضِ كَنْعَانَ. فَاتُوا الَى ارْضِ كَنْعَانَ.
\par 6 وَاجْتَازَ ابْرَامُ فِي الارْضِ الَى مَكَانِ شَكِيمَ الَى بَلُّوطَةِ مُورَةَ. وَكَانَ الْكَنْعَانِيُّونَ حِينَئِذٍ فِي الارْضِ.
\par 7 وَظَهَرَ الرَّبُّ لابْرَامَ وَقَالَ: «لِنَسْلِكَ اعْطِي هَذِهِ الارْضَ». فَبَنَى هُنَاكَ مَذْبَحا لِلرَّبِّ الَّذِي ظَهَرَ لَهُ.
\par 8 ثُمَّ نَقَلَ مِنْ هُنَاكَ الَى الْجَبَلِ شَرْقِيَّ بَيْتِ ايلٍ وَنَصَبَ خَيْمَتَهُ. وَلَهُ بَيْتُ ايلَ مِنَ الْمَغْرِبِ وَعَايُ مِنَ الْمَشْرِقِ. فَبَنَى هُنَاكَ مَذْبَحا لِلرَّبِّ وَدَعَا بِاسْمِ الرَّبِّ.
\par 9 ثُمَّ ارْتَحَلَ ابْرَامُ ارْتِحَالا مُتَوَالِيا نَحْوَ الْجَنُوبِ.
\par 10 وَحَدَثَ جُوعٌ فِي الارْضِ فَانْحَدَرَ ابْرَامُ الَى مِصْرَ لِيَتَغَرَّبَ هُنَاكَ لانَّ الْجُوعَ فِي الارْضِ كَانَ شَدِيدا.
\par 11 وَحَدَثَ لَمَّا قَرُبَ انْ يَدْخُلَ مِصْرَ انَّهُ قَالَ لِسَارَايَ امْرَاتِهِ: «انِّي قَدْ عَلِمْتُ انَّكِ امْرَاةٌ حَسَنَةُ الْمَنْظَرِ.
\par 12 فَيَكُونُ اذَا رَاكِ الْمِصْرِيُّونَ انَّهُمْ يَقُولُونَ: هَذِهِ امْرَاتُهُ. فَيَقْتُلُونَنِي وَيَسْتَبْقُونَكِ.
\par 13 قُولِي انَّكِ اخْتِي لِيَكُونَ لِي خَيْرٌ بِسَبَبِكِ وَتَحْيَا نَفْسِي مِنْ اجْلِكِ».
\par 14 فَحَدَثَ لَمَّا دَخَلَ ابْرَامُ الَى مِصْرَ انَّ الْمِصْرِيِّينَ رَاوُا الْمَرْاةَ انَّهَا حَسَنَةٌ جِدّا.
\par 15 وَرَاهَا رُؤَسَاءُ فِرْعَوْنَ وَمَدَحُوهَا لَدَى فِرْعَوْنَ فَاخِذَتِ الْمَرْاةُ الَى بَيْتِ فِرْعَوْنَ
\par 16 فَصَنَعَ الَى ابْرَامَ خَيْرا بِسَبَبِهَا وَصَارَ لَهُ غَنَمٌ وَبَقَرٌ وَحَمِيرٌ وَعَبِيدٌ وَامَاءٌ وَاتُنٌ وَجِمَالٌ.
\par 17 فَضَرَبَ الرَّبُّ فِرْعَوْنَ وَبَيْتَهُ ضَرَبَاتٍ عَظِيمَةً بِسَبَبِ سَارَايَ امْرَاةِ ابْرَامَ.
\par 18 فَدَعَا فِرْعَوْنُ ابْرَامَ وَقَالَ: «مَا هَذَا الَّذِي صَنَعْتَ بِي؟ لِمَاذَا لَمْ تُخْبِرْنِي انَّهَا امْرَاتُكَ؟
\par 19 لِمَاذَا قُلْتَ هِيَ اخْتِي حَتَّى اخَذْتُهَا لِي لِتَكُونَ زَوْجَتِي؟ وَالْانَ هُوَذَا امْرَاتُكَ! خُذْهَا وَاذْهَبْ!».
\par 20 فَاوْصَى عَلَيْهِ فِرْعَوْنُ رِجَالا فَشَيَّعُوهُ وَامْرَاتَهُ وَكُلَّ مَا كَانَ لَهُ.

\chapter{13}

\par 1 فَصَعِدَ ابْرَامُ مِنْ مِصْرَ هُوَ وَامْرَاتُهُ وَكُلُّ مَا كَانَ لَهُ وَلُوطٌ مَعَهُ الَى الْجَنُوبِ.
\par 2 وَكَانَ ابْرَامُ غَنِيّا جِدّا فِي الْمَوَاشِي وَالْفِضَّةِ وَالذَّهَبِ.
\par 3 وَسَارَ فِي رِحْلاتِهِ مِنَ الْجَنُوبِ الَى بَيْتِ ايلَ الَى الْمَكَانِ الَّذِي كَانَتْ خَيْمَتُهُ فِيهِ فِي الْبَدَاءَةِ بَيْنَ بَيْتِ ايلَ وَعَايَ
\par 4 الَى مَكَانِ الْمَذْبَحِ الَّذِي عَمِلَهُ هُنَاكَ اوَّلا. وَدَعَا هُنَاكَ ابْرَامُ بِاسْمِ الرَّبِّ.
\par 5 وَلُوطٌ السَّائِرُ مَعَ ابْرَامَ كَانَ لَهُ ايْضا غَنَمٌ وَبَقَرٌ وَخِيَامٌ.
\par 6 وَلَمْ تَحْتَمِلْهُمَا الارْضُ انْ يَسْكُنَا مَعا اذْ كَانَتْ امْلاكُهُمَا كَثِيرَةً فَلَمْ يَقْدِرَا انْ يَسْكُنَا مَعا.
\par 7 فَحَدَثَتْ مُخَاصَمَةٌ بَيْنَ رُعَاةِ مَوَاشِي ابْرَامَ وَرُعَاةِ مَوَاشِي لُوطٍ. وَكَانَ الْكَنْعَانِيُّونَ وَالْفِرِزِّيُّونَ حِينَئِذٍ سَاكِنِينَ فِي الارْضِ.
\par 8 فَقَالَ ابْرَامُ لِلُوطٍ: «لا تَكُنْ مُخَاصَمَةٌ بَيْنِي وَبَيْنَكَ وَبَيْنَ رُعَاتِي وَرُعَاتِكَ لانَّنَا نَحْنُ اخَوَانِ.
\par 9 الَيْسَتْ كُلُّ الارْضِ امَامَكَ؟ اعْتَزِلْ عَنِّي. انْ ذَهَبْتَ شِمَالا فَانَا يَمِينا وَانْ يَمِينا فَانَا شِمَالا».
\par 10 فَرَفَعَ لُوطٌ عَيْنَيْهِ وَرَاى كُلَّ دَائِرَةِ الارْدُنِّ انَّ جَمِيعَهَا سَقْيٌ قَبْلَمَا اخْرَبَ الرَّبُّ سَدُومَ وَعَمُورَةَ كَجَنَّةِ الرَّبِّ كَارْضِ مِصْرَ. حِينَمَا تَجِيءُ الَى صُوغَرَ.
\par 11 فَاخْتَارَ لُوطٌ لِنَفْسِهِ كُلَّ دَائِرَةِ الارْدُنِّ وَارْتَحَلَ لُوطٌ شَرْقا. فَاعْتَزَلَ الْوَاحِدُ عَنِ الْاخَرِ.
\par 12 ابْرَامُ سَكَنَ فِي ارْضِ كَنْعَانَ وَلُوطٌ سَكَنَ فِي مُدُنِ الدَّائِرَةِ وَنَقَلَ خِيَامَهُ الَى سَدُومَ.
\par 13 وَكَانَ اهْلُ سَدُومَ اشْرَارا وَخُطَاةً لَدَى الرَّبِّ جِدّا.
\par 14 وَقَالَ الرَّبُّ لابْرَامَ بَعْدَ اعْتِزَالِ لُوطٍ عَنْهُ: «ارْفَعْ عَيْنَيْكَ وَانْظُرْ مِنَ الْمَوْضِعِ الَّذِي انْتَ فِيهِ شِمَالا وَجَنُوبا وَشَرْقا وَغَرْبا
\par 15 لانَّ جَمِيعَ الارْضِ الَّتِي انْتَ تَرَى لَكَ اعْطِيهَا وَلِنَسْلِكَ الَى الابَدِ.
\par 16 وَاجْعَلُ نَسْلَكَ كَتُرَابِ الارْضِ حَتَّى اذَا اسْتَطَاعَ احَدٌ انْ يَعُدَّ تُرَابَ الارْضِ فَنَسْلُكَ ايْضا يُعَدُّ.
\par 17 قُمِ امْشِ فِي الارْضِ طُولَهَا وَعَرْضَهَا لانِّي لَكَ اعْطِيهَا».
\par 18 فَنَقَلَ ابْرَامُ خِيَامَهُ وَاتَى وَاقَامَ عِنْدَ بَلُّوطَاتِ مَمْرَا الَّتِي فِي حَبْرُونَ وَبَنَى هُنَاكَ مَذْبَحا لِلرَّبِّ.

\chapter{14}

\par 1 وَحَدَثَ فِي ايَّامِ امْرَافَلَ مَلِكِ شِنْعَارَ وَارْيُوكَ مَلِكِ الَّاسَارَ وَكَدَرْلَعَوْمَرَ مَلِكِ عِيلامَ وَتِدْعَالَ مَلِكِ جُويِيمَ
\par 2 انَّ هَؤُلاءِ صَنَعُوا حَرْبا مَعَ بَارَعَ مَلِكِ سَدُومَ وَبِرْشَاعَ مَلِكِ عَمُورَةَ وَشِنْابَ مَلِكِ ادْمَةَ وَشِمْئِيبَرَ مَلِكِ صَبُويِيمَ وَمَلِكِ بَالَعَ (الَّتِي هِيَ صُوغَرُ).
\par 3 جَمِيعُ هَؤُلاءِ اجْتَمَعُوا مُتَعَاهِدِينَ الَى عُمْقِ السِّدِّيمِ (الَّذِي هُوَ بَحْرُ الْمِلْحِ).
\par 4 اثْنَتَيْ عَشَرَةَ سَنَةً اسْتُعْبِدُوا لِكَدَرْلَعَوْمَرَ وَالسَّنَةَ الثَّالِثَةَ عَشَرَةَ عَصُوا عَلَيْهِ.
\par 5 وَفِي السَّنَةِ الرَّابِعَةَ عَشَرَةَ اتَى كَدَرْلَعَوْمَرُ وَالْمُلُوكُ الَّذِينَ مَعَهُ وَضَرَبُوا الرَّفَائِيِّينَ فِي عَشْتَارُوثَ قَرْنَايِمَ وَالزُّوزِيِّينَ فِي هَامَ وَالايمِيِّينَ فِي شَوَى قَرْيَتَايِمَ
\par 6 وَالْحُورِيِّينَ فِي جَبَلِهِمْ سَعِيرَ الَى بُطْمَةِ فَارَانَ الَّتِي عِنْدَ الْبَرِّيَّةِ.
\par 7 ثُمَّ رَجَعُوا وَجَاءُوا الَى عَيْنِ مِشْفَاطَ (الَّتِي هِيَ قَادِشُ). وَضَرَبُوا كُلَّ بِلادِ الْعَمَالِقَةِ وَايْضا الامُورِيِّينَ السَّاكِنِينَ فِي حَصُّونَ تَامَارَ.
\par 8 فَخَرَجَ مَلِكُ سَدُومَ وَمَلِكُ عَمُورَةَ وَمَلِكُ ادْمَةَ وَمَلِكُ صَبُويِيمَ وَمَلِكُ بَالَعَ (الَّتِي هِيَ صُوغَرُ) وَنَظَمُوا حَرْبا مَعَهُمْ فِي عُمْقِ السِّدِّيمِ.
\par 9 مَعَ كَدَرْلَعَوْمَرَ مَلِكِ عِيلامَ وَتِدْعَالَ مَلِكِ جُويِيمَ وَامْرَافَلَ مَلِكِ شِنْعَارَ وَارْيُوكَ مَلِكِ الَّاسَارَ. ارْبَعَةُ مُلُوكٍ عَلَى خَمْسَةٍ.
\par 10 وَعُمْقُ السِّدِّيمِ كَانَ فِيهِ ابَارُ حُمَرٍ كَثِيرَةٌ. فَهَرَبَ مَلِكَا سَدُومَ وَعَمُورَةَ وَسَقَطَا هُنَاكَ وَالْبَاقُونَ هَرَبُوا الَى الْجَبَلِ.
\par 11 فَاخَذُوا جَمِيعَ امْلاكِ سَدُومَ وَعَمُورَةَ وَجَمِيعَ اطْعِمَتِهِمْ وَمَضُوا.
\par 12 وَاخَذُوا لُوطا ابْنَ اخِي ابْرَامَ وَامْلاكَهُ وَمَضُوا اذْ كَانَ سَاكِنا فِي سَدُومَ.
\par 13 فَاتَى مَنْ نَجَا وَاخْبَرَ ابْرَامَ الْعِبْرَانِيَّ. وَكَانَ سَاكِنا عِنْدَ بَلُّوطَاتِ مَمْرَا الامُورِيِّ اخِي اشْكُولَ وَاخِي عَانِرَ. وَكَانُوا اصْحَابَ عَهْدٍ مَعَ ابْرَامَ.
\par 14 فَلَمَّا سَمِعَ ابْرَامُ انَّ اخَاهُ سُبِيَ جَرَّ غِلْمَانَهُ الْمُتَمَرِّنِينَ وِلْدَانَ بَيْتِهِ ثَلاثَ مِئَةٍ وَثَمَانِيَةَ عَشَرَ وَتَبِعَهُمْ الَى دَانَ.
\par 15 وَانْقَسَمَ عَلَيْهِمْ لَيْلا هُوَ وَعَبِيدُهُ فَكَسَّرَهُمْ وَتَبِعَهُمْ الَى حُوبَةَ الَّتِي عَنْ شَِمَالِ دِمَشْقَ.
\par 16 وَاسْتَرْجَعَ كُلَّ الامْلاكِ وَاسْتَرْجَعَ لُوطا اخَاهُ ايْضا وَامْلاكَهُ وَالنِّسَاءَ ايْضا وَالشَّعْبَ.
\par 17 فَخَرَجَ مَلِكُ سَدُومَ لِاسْتِقْبَالِهِ بَعْدَ رُجُوعِهِ مِنْ كَسْرَةِ كَدَرْلَعَوْمَرَ وَالْمُلُوكِ الَّذِينَ مَعَهُ الَى عُمْقِ شَوَى (الَّذِي هُوَ عُمْقُ الْمَلِكِ).
\par 18 وَمَلْكِي صَادِقُ مَلِكُ شَالِيمَ اخْرَجَ خُبْزا وَخَمْرا. وَكَانَ كَاهِنا لِلَّهِ الْعَلِيِّ.
\par 19 وَبَارَكَهُ وَقَالَ: «مُبَارَكٌ ابْرَامُ مِنَ اللهِ الْعَلِيِّ مَالِكِ السَّمَاوَاتِ وَالارْضِ
\par 20 وَمُبَارَكٌ اللهُ الْعَلِيُّ الَّذِي اسْلَمَ اعْدَاءَكَ فِي يَدِكَ». فَاعْطَاهُ عُشْرا مِنْ كُلِّ شَيْءٍ.
\par 21 وَقَالَ مَلِكُ سَدُومَ لابْرَامَ: «اعْطِنِي النُّفُوسَ وَامَّا الامْلاكَ فَخُذْهَا لِنَفْسِكَ».
\par 22 فَقَالَ ابْرَامُ لِمَلِكِ سَدُومَ: «رَفَعْتُ يَدِي الَى الرَّبِّ الالَهِ الْعَلِيِّ مَالِكِ السَّمَاءِ وَالارْضِ
\par 23 لا اخُذَنَّ لا خَيْطا وَلا شِرَاكَ نَعْلٍ وَلا مِنْ كُلِّ مَا هُوَ لَكَ فَلا تَقُولُ: انَا اغْنَيْتُ ابْرَامَ.
\par 24 لَيْسَ لِي غَيْرَ الَّذِي اكَلَهُ الْغِلْمَانُ. وَامَّا نَصِيبُ الرِّجَالِ الَّذِينَ ذَهَبُوا مَعِي: عَانِرَ وَاشْكُولَ وَمَمْرَا فَهُمْ يَاخُذُونَ نَصِيبَهُمْ».

\chapter{15}

\par 1 بَعْدَ هَذِهِ الامُورِ صَارَ كَلامُ الرَّبِّ الَى ابْرَامَ فِي الرُّؤْيَا: «لا تَخَفْ يَا ابْرَامُ. انَا تُرْسٌ لَكَ. اجْرُكَ كَثِيرٌ جِدّا».
\par 2 فَقَالَ ابْرَامُ: «ايُّهَا السَّيِّدُ الرَّبُّ مَاذَا تُعْطِينِي وَانَا مَاضٍ عَقِيما وَمَالِكُ بَيْتِي هُوَ الِيعَازَرُ الدِّمَشْقِيُّ؟»
\par 3 وَقَالَ ابْرَامُ ايْضا: «انَّكَ لَمْ تُعْطِنِي نَسْلا وَهُوَذَا ابْنُ بَيْتِي وَارِثٌ لِي».
\par 4 فَاذَا كَلامُ الرَّبِّ الَيْهِ: «لا يَرِثُكَ هَذَا. بَلِ الَّذِي يَخْرُجُ مِنْ احْشَائِكَ هُوَ يَرِثُكَ».
\par 5 ثُمَّ اخْرَجَهُ الَى خَارِجٍ وَقَالَ: «انْظُرْ الَى السَّمَاءِ وَعُدَّ النُّجُومَ انِ اسْتَطَعْتَ انْ تَعُدَّهَا». وَقَالَ لَهُ: «هَكَذَا يَكُونُ نَسْلُكَ».
\par 6 فَامَنَ بِالرَّبِّ فَحَسِبَهُ لَهُ بِرّا.
\par 7 وَقَالَ لَهُ: «انَا الرَّبُّ الَّذِي اخْرَجَكَ مِنْ اورِ الْكِلْدَانِيِّينَ لِيُعْطِيَكَ هَذِهِ الارْضَ لِتَرِثَهَا».
\par 8 فَقَالَ: «ايُّهَا السَّيِّدُ الرَّبُّ بِمَاذَا اعْلَمُ انِّي ارِثُهَا؟»
\par 9 فَقَالَ لَهُ: «خُذْ لِي عِجْلَةً ثُلاثِيَّةً وَعَنْزَةً ثُلاثِيَّةً وَكَبْشا ثُلاثِيّا وَيَمَامَةً وَحَمَامَةً».
\par 10 فَاخَذَ هَذِهِ كُلَّهَا وَشَقَّهَا مِنَ الْوَسَطِ وَجَعَلَ شِقَّ كُلِّ وَاحِدٍ مُقَابِلَ صَاحِبِهِ. وَامَّا الطَّيْرُ فَلَمْ يَشُقَّهُ.
\par 11 فَنَزَلَتِ الْجَوَارِحُ عَلَى الْجُثَثِ وَكَانَ ابْرَامُ يَزْجُرُهَا.
\par 12 وَلَمَّا صَارَتِ الشَّمْسُ الَى الْمَغِيبِ وَقَعَ عَلَى ابْرَامَ سُبَاتٌ وَاذَا رُعْبَةٌ مُظْلِمَةٌ عَظِيمَةٌ وَاقِعَةٌ عَلَيْهِ.
\par 13 فَقَالَ لابْرَامَ: «اعْلَمْ يَقِينا انَّ نَسْلَكَ سَيَكُونُ غَرِيبا فِي ارْضٍ لَيْسَتْ لَهُمْ وَيُسْتَعْبَدُونَ لَهُمْ فَيُذِلُّونَهُمْ ارْبَعَ مِئَةِ سَنَةٍ.
\par 14 ثُمَّ الامَّةُ الَّتِي يُسْتَعْبَدُونَ لَهَا انَا ادِينُهَا. وَبَعْدَ ذَلِكَ يَخْرُجُونَ بِامْلاكٍ جَزِيلَةٍ.
\par 15 وَامَّا انْتَ فَتَمْضِي الَى ابَائِكَ بِسَلامٍ وَتُدْفَنُ بِشَيْبَةٍ صَالِحَةٍ.
\par 16 وَفِي الْجِيلِ الرَّابِعِ يَرْجِعُونَ الَى هَهُنَا لانَّ ذَنْبَ الامُورِيِّينَ لَيْسَ الَى الْانَ كَامِلا».
\par 17 ثُمَّ غَابَتِ الشَّمْسُ فَصَارَتِ الْعَتَمَةُ وَاذَا تَنُّورُ دُخَانٍ وَمِصْبَاحُ نَارٍ يَجُوزُ بَيْنَ تِلْكَ الْقِطَعِ.
\par 18 فِي ذَلِكَ الْيَوْمِ قَطَعَ الرَّبُّ مَعَ ابْرَامَ مِيثَاقا قَائِلا: «لِنَسْلِكَ اعْطِي هَذِهِ الارْضَ مِنْ نَهْرِ مِصْرَ الَى النَّهْرِ الْكَبِيرِ نَهْرِ الْفُرَاتِ.
\par 19 الْقِينِيِّينَ وَالْقَنِزِّيِّينَ وَالْقَدْمُونِيِّينَ
\par 20 وَالْحِثِّيِّينَ وَالْفِرِزِّيِّينَ وَالرَّفَائِيِّينَ
\par 21 وَالامُورِيِّينَ وَالْكَنْعَانِيِّينَ وَالْجِرْجَاشِيِّينَ وَالْيَبُوسِيِّينَ».

\chapter{16}

\par 1 وَامَّا سَارَايُ امْرَاةُ ابْرَامَ فَلَمْ تَلِدْ لَهُ. وَكَانَتْ لَهَا جَارِيَةٌ مِصْرِيَّةٌ اسْمُهَا هَاجَرُ
\par 2 فَقَالَتْ سَارَايُ لابْرَامَ: «هُوَذَا الرَّبُّ قَدْ امْسَكَنِي عَنِ الْوِلادَةِ. ادْخُلْ عَلَى جَارِيَتِي لَعَلِّي ارْزَقُ مِنْهَا بَنِينَ». فَسَمِعَ ابْرَامُ لِقَوْلِ سَارَايَ.
\par 3 فَاخَذَتْ سَارَايُ امْرَاةُ ابْرَامَ هَاجَرَ الْمِصْرِيَّةَ جَارِيَتَهَا مِنْ بَعْدِ عَشَرِ سِنِينَ لاقَامَةِ ابْرَامَ فِي ارْضِ كَنْعَانَ وَاعْطَتْهَا لابْرَامَ رَجُلِهَا زَوْجَةً لَهُ.
\par 4 فَدَخَلَ عَلَى هَاجَرَ فَحَبِلَتْ. وَلَمَّا رَاتْ انَّهَا حَبِلَتْ صَغُرَتْ مَوْلاتُهَا فِي عَيْنَيْهَا.
\par 5 فَقَالَتْ سَارَايُ لابْرَامَ: «ظُلْمِي عَلَيْكَ! انَا دَفَعْتُ جَارِيَتِي الَى حِضْنِكَ فَلَمَّا رَاتْ انَّهَا حَبِلَتْ صَغُرْتُ فِي عَيْنَيْهَا. يَقْضِي الرَّبُّ بَيْنِي وَبَيْنَكَ».
\par 6 فَقَالَ ابْرَامُ لِسَارَايَ: «هُوَذَا جَارِيَتُكِ فِي يَدِكِ. افْعَلِي بِهَا مَا يَحْسُنُ فِي عَيْنَيْكِ». فَاذَلَّتْهَا سَارَايُ فَهَرَبَتْ مِنْ وَجْهِهَا.
\par 7 فَوَجَدَهَا مَلاكُ الرَّبِّ عَلَى عَيْنِ الْمَاءِ فِي الْبَرِّيَّةِ عَلَى الْعَيْنِ الَّتِي فِي طَرِيقِ شُورَ.
\par 8 وَقَالَ: «يَا هَاجَرُ جَارِيَةَ سَارَايَ مِنْ ايْنَ اتَيْتِ وَالَى ايْنَ تَذْهَبِين؟». فَقَالَتْ: «انَا هَارِبَةٌ مِنْ وَجْهِ مَوْلاتِي سَارَايَ».
\par 9 فَقَالَ لَهَا مَلاكُ الرَّبِّ: «ارْجِعِي الَى مَوْلاتِكِ وَاخْضَعِي تَحْتَ يَدَيْهَا».
\par 10 وَقَالَ لَهَا مَلاكُ الرَّبِّ: «تَكْثِيرا اكَثِّرُ نَسْلَكِ فَلا يُعَدُّ مِنَ الْكَثْرَةِ».
\par 11 وَقَالَ لَهَا مَلاكُ الرَّبِّ: «هَا انْتِ حُبْلَى فَتَلِدِينَ ابْنا وَتَدْعِينَ اسْمَهُ اسْمَاعِيلَ لانَّ الرَّبَّ قَدْ سَمِعَ لِمَذَلَّتِكِ.
\par 12 وَانَّهُ يَكُونُ انْسَانا وَحْشِيّا يَدُهُ عَلَى كُلِّ وَاحِدٍ وَيَدُ كُلِّ وَاحِدٍ عَلَيْهِ وَامَامَ جَمِيعِ اخْوَتِهِ يَسْكُنُ».
\par 13 فَدَعَتِ اسْمَ الرَّبِّ الَّذِي تَكَلَّمَ مَعَهَا: «انْتَ ايلُ رُئِي». لانَّهَا قَالَتْ: «اهَهُنَا ايْضا رَايْتُ بَعْدَ رُؤْيَةٍ؟»
\par 14 لِذَلِكَ دُعِيَتِ الْبِئْرُ «بِئْرَ لَحَيْ رُئِي». هَا هِيَ بَيْنَ قَادِشَ وَبَارَدَ.
\par 15 فَوَلَدَتْ هَاجَرُ لابْرَامَ ابْنا. وَدَعَا ابْرَامُ اسْمَ ابْنِهِ الَّذِي وَلَدَتْهُ هَاجَرُ «اسْمَاعِيلَ».
\par 16 كَانَ ابْرَامُ ابْنَ سِتٍّ وَثَمَانِينَ سَنَةً لَمَّا وَلَدَتْ هَاجَرُ اسْمَاعِيلَ لابْرَامَ.

\chapter{17}

\par 1 وَلَمَّا كَانَ ابْرَامُ ابْنَ تِسْعٍ وَتِسْعِينَ سَنَةً ظَهَرَ الرَّبُّ لابْرَامَ وَقَالَ لَهُ: «انَا اللهُ الْقَدِيرُ. سِرْ امَامِي وَكُنْ كَامِلا
\par 2 فَاجْعَلَ عَهْدِي بَيْنِي وَبَيْنَكَ وَاكَثِّرَكَ كَثِيرا جِدّا».
\par 3 فَسَقَطَ ابْرَامُ عَلَى وَجْهِهِ. وَقَالَ اللهُ لَهُ:
\par 4 «امَّا انَا فَهُوَذَا عَهْدِي مَعَكَ وَتَكُونُ ابا لِجُمْهُورٍ مِنَ الامَمِ
\par 5 فَلا يُدْعَى اسْمُكَ بَعْدُ ابْرَامَ بَلْ يَكُونُ اسْمُكَ ابْرَاهِيمَ لانِّي اجْعَلُكَ ابا لِجُمْهُورٍ مِنَ الامَمِ.
\par 6 وَاثْمِرُكَ كَثِيرا جِدّا وَاجْعَلُكَ امَما وَمُلُوكٌ مِنْكَ يَخْرُجُونَ.
\par 7 وَاقِيمُ عَهْدِي بَيْنِي وَبَيْنَكَ وَبَيْنَ نَسْلِكَ مِنْ بَعْدِكَ فِي اجْيَالِهِمْ عَهْدا ابَدِيّا لاكُونَ الَها لَكَ وَلِنَسْلِكَ مِنْ بَعْدِكَ.
\par 8 وَاعْطِي لَكَ وَلِنَسْلِكَ مِنْ بَعْدِكَ ارْضَ غُرْبَتِكَ كُلَّ ارْضِ كَنْعَانَ مِلْكا ابَدِيّا. وَاكُونُ الَهَهُمْ».
\par 9 وَقَالَ اللهُ لابْرَاهِيمَ: «وَامَّا انْتَ فَتَحْفَظُ عَهْدِي انْتَ وَنَسْلُكَ مِنْ بَعْدِكَ فِي اجْيَالِهِمْ.
\par 10 هَذَا هُوَ عَهْدِي الَّذِي تَحْفَظُونَهُ بَيْنِي وَبَيْنَكُمْ وَبَيْنَ نَسْلِكَ مِنْ بَعْدِكَ: يُخْتَنُ مِنْكُمْ كُلُّ ذَكَرٍ
\par 11 فَتُخْتَنُونَ فِي لَحْمِ غُرْلَتِكُمْ فَيَكُونُ عَلامَةَ عَهْدٍ بَيْنِي وَبَيْنَكُمْ.
\par 12 ابْنَ ثَمَانِيَةِ ايَّامٍ يُخْتَنُ مِنْكُمْ كُلُّ ذَكَرٍ فِي اجْيَالِكُمْ: وَلِيدُ الْبَيْتِ وَالْمُبْتَاعُ بِفِضَّةٍ مِنْ كُلِّ ابْنِ غَرِيبٍ لَيْسَ مِنْ نَسْلِكَ.
\par 13 يُخْتَنُ خِتَانا وَلِيدُ بَيْتِكَ وَالْمُبْتَاعُ بِفِضَّتِكَ فَيَكُونُ عَهْدِي فِي لَحْمِكُمْ عَهْدا ابَدِيّا.
\par 14 وَامَّا الذَّكَرُ الاغْلَفُ الَّذِي لا يُخْتَنُ فِي لَحْمِ غُرْلَتِهِ فَتُقْطَعُ تِلْكَ النَّفْسُ مِنْ شَعْبِهَا. انَّهُ قَدْ نَكَثَ عَهْدِي».
\par 15 وَقَالَ اللهُ لابْرَاهِيمَ: «سَارَايُ امْرَاتُكَ لا تَدْعُو اسْمَهَا سَارَايَ بَلِ اسْمُهَا سَارَةُ.
\par 16 وَابَارِكُهَا وَاعْطِيكَ ايْضا مِنْهَا ابْنا. ابَارِكُهَا فَتَكُونُ امَما وَمُلُوكُ شُعُوبٍ مِنْهَا يَكُونُونَ».
\par 17 فَسَقَطَ ابْرَاهِيمُ عَلَى وَجْهِهِ وَضَحِكَ وَقَالَ فِي قَلْبِهِ: «هَلْ يُولَدُ لِابْنِ مِئَةِ سَنَةٍ؟ وَهَلْ تَلِدُ سَارَةُ وَهِيَ بِنْتُ تِسْعِينَ سَنَةً؟».
\par 18 وَقَالَ ابْرَاهِيمُ لِلَّهِ: «لَيْتَ اسْمَاعِيلَ يَعِيشُ امَامَكَ!»
\par 19 فَقَالَ اللهُ بَلْ سَارَةُ امْرَاتُكَ تَلِدُ لَكَ ابْنا وَتَدْعُو اسْمَهُ اسْحَاقَ. وَاقِيمُ عَهْدِي مَعَهُ عَهْدا ابَدِيّا لِنَسْلِهِ مِنْ بَعْدِهِ.
\par 20 وَامَّا اسْمَاعِيلُ فَقَدْ سَمِعْتُ لَكَ فِيهِ. هَا انَا ابَارِكُهُ وَاثْمِرُهُ وَاكَثِّرُهُ كَثِيرا جِدّا. اثْنَيْ عَشَرَ رَئِيسا يَلِدُ وَاجْعَلُهُ امَّةً كَبِيرَةً.
\par 21 وَلَكِنْ عَهْدِي اقِيمُهُ مَعَ اسْحَاقَ الَّذِي تَلِدُهُ لَكَ سَارَةُ فِي هَذَا الْوَقْتِ فِي السَّنَةِ الْاتِيَةِ».
\par 22 فَلَمَّا فَرَغَ مِنَ الْكَلامِ مَعَهُ صَعِدَ اللهُ عَنْ ابْرَاهِيمَ.
\par 23 فَاخَذَ ابْرَاهِيمُ اسْمَاعِيلَ ابْنَهُ وَجَمِيعَ وِلْدَانِ بَيْتِهِ وَجَمِيعَ الْمُبْتَاعِينَ بِفِضَّتِهِ كُلَّ ذَكَرٍ مِنْ اهْلِ بَيْتِ ابْرَاهِيمَ وَخَتَنَ لَحْمَ غُرْلَتِهِمْ فِي ذَلِكَ الْيَوْمِ عَيْنِهِ كَمَا كَلَّمَهُ اللهُ.
\par 24 وَكَانَ ابْرَاهِيمُ ابْنَ تِسْعٍ وَتِسْعِينَ سَنَةً حِينَ خُتِنَ فِي لَحْمِ غُرْلَتِهِ
\par 25 وَكَانَ اسْمَاعِيلُ ابْنُهُ ابْنَ ثَلاثَ عَشَرَةَ سَنَةً حِينَ خُتِنَ فِي لَحْمِ غُرْلَتِهِ.
\par 26 فِي ذَلِكَ الْيَوْمِ عَيْنِهِ خُتِنَ ابْرَاهِيمُ وَاسْمَاعِيلُ ابْنُهُ.
\par 27 وَكُلُّ رِجَالِ بَيْتِهِ وِلْدَانِ الْبَيْتِ وَالْمُبْتَاعِينَ بِالْفِضَّةِ مِنِ ابْنِ الْغَرِيبِ خُتِنُوا مَعَهُ.

\chapter{18}

\par 1 وَظَهَرَ لَهُ الرَّبُّ عِنْدَ بَلُّوطَاتِ مَمْرَا وَهُوَ جَالِسٌ فِي بَابِ الْخَيْمَةِ وَقْتَ حَرِّ النَّهَارِ
\par 2 فَرَفَعَ عَيْنَيْهِ وَنَظَرَ وَاذَا ثَلاثَةُ رِجَالٍ وَاقِفُونَ لَدَيْهِ. فَلَمَّا نَظَرَ رَكَضَ لِاسْتِقْبَالِهِمْ مِنْ بَابِ الْخَيْمَةِ وَسَجَدَ الَى الارْضِ
\par 3 وَقَالَ: «يَا سَيِّدُ انْ كُنْتُ قَدْ وَجَدْتُ نِعْمَةً فِي عَيْنَيْكَ فَلا تَتَجَاوَزْ عَبْدَكَ.
\par 4 لِيُؤْخَذْ قَلِيلُ مَاءٍ وَاغْسِلُوا ارْجُلَكُمْ وَاتَّكِئُوا تَحْتَ الشَّجَرَةِ
\par 5 فَاخُذَ كِسْرَةَ خُبْزٍ فَتُسْنِدُونَ قُلُوبَكُمْ ثُمَّ تَجْتَازُونَ لانَّكُمْ قَدْ مَرَرْتُمْ عَلَى عَبْدِكُمْ». فَقَالُوا: «هَكَذَا تَفْعَلُ كَمَا تَكَلَّمْتَ».
\par 6 فَاسْرَعَ ابْرَاهِيمُ الَى الْخَيْمَةِ الَى سَارَةَ وَقَالَ: «اسْرِعِي بِثَلاثِ كَيْلاتٍ دَقِيقا سَمِيذا. اعْجِنِي وَاصْنَعِي خُبْزَ مَلَّةٍ».
\par 7 ثُمَّ رَكَضَ ابْرَاهِيمُ الَى الْبَقَرِ وَاخَذَ عِجْلا رَخْصا وَجَيِّدا وَاعْطَاهُ لِلْغُلامِ فَاسْرَعَ لِيَعْمَلَهُ.
\par 8 ثُمَّ اخَذَ زُبْدا وَلَبَنا وَالْعِجْلَ الَّذِي عَمِلَهُ وَوَضَعَهَا قُدَّامَهُمْ. وَاذْ كَانَ هُوَ وَاقِفا لَدَيْهِمْ تَحْتَ الشَّجَرَةِ اكَلُوا.
\par 9 وَقَالُوا لَهُ: «ايْنَ سَارَةُ امْرَاتُكَ؟» فَقَالَ: «هَا هِيَ فِي الْخَيْمَةِ».
\par 10 فَقَالَ: «انِّي ارْجِعُ الَيْكَ نَحْوَ زَمَانِ الْحَيَاةِ وَيَكُونُ لِسَارَةَ امْرَاتِكَ ابْنٌ». وَكَانَتْ سَارَةُ سَامِعَةً فِي بَابِ الْخَيْمَةِ وَهُوَ وَرَاءَهُ -
\par 11 وَكَانَ ابْرَاهِيمُ وَسَارَةُ شَيْخَيْنِ مُتَقَدِّمَيْنِ فِي الايَّامِ وَقَدِ انْقَطَعَ انْ يَكُونَ لِسَارَةَ عَادَةٌ كَالنِّسَاءِ.
\par 12 فَضَحِكَتْ سَارَةُ فِي بَاطِنِهَا قَائِلَةً: «ابَعْدَ فَنَائِي يَكُونُ لِي تَنَعُّمٌ وَسَيِّدِي قَدْ شَاخَ!»
\par 13 فَقَالَ الرَّبُّ لابْرَاهِيمَ: «لِمَاذَا ضَحِكَتْ سَارَةُ قَائِلَةً: افَبِالْحَقِيقَةِ الِدُ وَانَا قَدْ شِخْتُ؟
\par 14 هَلْ يَسْتَحِيلُ عَلَى الرَّبِّ شَيْءٌ؟ فِي الْمِيعَادِ ارْجِعُ الَيْكَ نَحْوَ زَمَانِ الْحَيَاةِ وَيَكُونُ لِسَارَةَ ابْنٌ».
\par 15 فَانْكَرَتْ سَارَةُ قَائِلَةً: «لَمْ اضْحَكْ». (لانَّهَا خَافَتْ). فَقَالَ: «لا! بَلْ ضَحِكْتِ».
\par 16 ثُمَّ قَامَ الرِّجَالُ مِنْ هُنَاكَ وَتَطَلَّعُوا نَحْوَ سَدُومَ. وَكَانَ ابْرَاهِيمُ مَاشِيا مَعَهُمْ لِيُشَيِّعَهُمْ.
\par 17 فَقَالَ الرَّبُّ: «هَلْ اخْفِي عَنْ ابْرَاهِيمَ مَا انَا فَاعِلُهُ
\par 18 وَابْرَاهِيمُ يَكُونُ امَّةً كَبِيرَةً وَقَوِيَّةً وَيَتَبَارَكُ بِهِ جَمِيعُ امَمِ الارْضِ؟
\par 19 لانِّي عَرَفْتُهُ لِكَيْ يُوصِيَ بَنِيهِ وَبَيْتَهُ مِنْ بَعْدِهِ انْ يَحْفَظُوا طَرِيقَ الرَّبِّ لِيَعْمَلُوا بِرّا وَعَدْلا لِكَيْ يَاتِيَ الرَّبُّ لابْرَاهِيمَ بِمَا تَكَلَّمَ بِهِ».
\par 20 وَقَالَ الرَّبُّ: «انَّ صُرَاخَ سَدُومَ وَعَمُورَةَ قَدْ كَثُرَ وَخَطِيَّتُهُمْ قَدْ عَظُمَتْ جِدّا.
\par 21 انْزِلُ وَارَى هَلْ فَعَلُوا بِالتَّمَامِ حَسَبَ صُرَاخِهَا الْاتِي الَيَّ وَالَّا فَاعْلَمُ».
\par 22 وَانْصَرَفَ الرِّجَالُ مِنْ هُنَاكَ وَذَهَبُوا نَحْوَ سَدُومَ وَامَّا ابْرَاهِيمُ فَكَانَ لَمْ يَزَلْ قَائِما امَامَ الرَّبِّ.
\par 23 فَتَقَدَّمَ ابْرَاهِيمُ وَقَالَ: «افَتُهْلِكُ الْبَارَّ مَعَ الاثِيمِ؟
\par 24 عَسَى انْ يَكُونَ خَمْسُونَ بَارّا فِي الْمَدِينَةِ. افَتُهْلِكُ الْمَكَانَ وَلا تَصْفَحُ عَنْهُ مِنْ اجْلِ الْخَمْسِينَ بَارّا الَّذِينَ فِيهِ؟
\par 25 حَاشَا لَكَ انْ تَفْعَلَ مِثْلَ هَذَا الامْرِ انْ تُمِيتَ الْبَارَّ مَعَ الاثِيمِ فَيَكُونُ الْبَارُّ كَالاثِيمِ. حَاشَا لَكَ! ادَيَّانُ كُلِّ الارْضِ لا يَصْنَعُ عَدْلا؟»
\par 26 فَقَالَ الرَّبُّ: «انْ وَجَدْتُ فِي سَدُومَ خَمْسِينَ بَارّا فِي الْمَدِينَةِ فَانِّي اصْفَحُ عَنِ الْمَكَانِ كُلِّهِ مِنْ اجْلِهِمْ».
\par 27 فَقَالَ ابْرَاهِيمُ: «انِّي قَدْ شَرَعْتُ اكَلِّمُ الْمَوْلَى وَانَا تُرَابٌ وَرَمَادٌ.
\par 28 رُبَّمَا نَقَصَ الْخَمْسُونَ بَارّا خَمْسَةً. اتُهْلِكُ كُلَّ الْمَدِينَةِ بِالْخَمْسَةِ؟» فَقَالَ: «لا اهْلِكُ انْ وَجَدْتُ هُنَاكَ خَمْسَةً وَارْبَعِينَ».
\par 29 فَعَادَ يُكَلِّمُهُ ايْضا وَقَالَ: « عَسَى انْ يُوجَدَ هُنَاكَ ارْبَعُونَ». فَقَالَ: «لا افْعَلُ مِنْ اجْلِ الارْبَعِينَ».
\par 30 فَقَالَ: «لا يَسْخَطِ الْمَوْلَى فَاتَكَلَّمَ. عَسَى انْ يُوجَدَ هُنَاكَ ثَلاثُونَ». فَقَالَ: «لا افْعَلُ انْ وَجَدْتُ هُنَاكَ ثَلاثِينَ».
\par 31 فَقَالَ: «انِّي قَدْ شَرَعْتُ اكَلِّمُ الْمَوْلَى. عَسَى انْ يُوجَدَ هُنَاكَ عِشْرُونَ». فَقَالَ: «لا اهْلِكُ مِنْ اجْلِ الْعِشْرِينَ».
\par 32 فَقَالَ: «لا يَسْخَطِ الْمَوْلَى فَاتَكَلَّمَ هَذِهِ الْمَرَّةَ فَقَطْ. عَسَى انْ يُوجَدَ هُنَاكَ عَشَرَةٌ». فَقَالَ: «لا اهْلِكُ مِنْ اجْلِ الْعَشَرَةِ».
\par 33 وَذَهَبَ الرَّبُّ عِنْدَمَا فَرَغَ مِنَ الْكَلامِ مَعَ ابْرَاهِيمَ وَرَجَعَ ابْرَاهِيمُ الَى مَكَانِهِ.

\chapter{19}

\par 1 فَجَاءَ الْمَلاكَانِ الَى سَدُومَ مَسَاءً وَكَانَ لُوطٌ جَالِسا فِي بَابِ سَدُومَ. فَلَمَّا رَاهُمَا لُوطٌ قَامَ لِاسْتِقْبَالِهِمَا وَسَجَدَ بِوَجْهِهِ الَى الارْضِ.
\par 2 وَقَالَ: «يَا سَيِّدَيَّ مِيلا الَى بَيْتِ عَبْدِكُمَا وَبِيتَا وَاغْسِلا ارْجُلَكُمَا ثُمَّ تُبَكِّرَانِ وَتَذْهَبَانِ فِي طَرِيقِكُمَا». فَقَالا: «لا بَلْ فِي السَّاحَةِ نَبِيتُ».
\par 3 فَالَحَّ عَلَيْهِمَا جِدّا فَمَالا الَيْهِ وَدَخَلا بَيْتَهُ فَصَنَعَ لَهُمَا ضِيَافَةً وَخَبَزَ فَطِيرا فَاكَلا.
\par 4 وَقَبْلَمَا اضْطَجَعَا احَاطَ بِالْبَيْتِ رِجَالُ الْمَدِينَةِ رِجَالُ سَدُومَ مِنَ الْحَدَثِ الَى الشَّيْخِ كُلُّ الشَّعْبِ مِنْ اقْصَاهَا.
\par 5 فَنَادُوا لُوطا وَقَالُوا لَهُ: «ايْنَ الرَّجُلانِ اللَّذَانِ دَخَلا الَيْكَ اللَّيْلَةَ؟ اخْرِجْهُمَا الَيْنَا لِنَعْرِفَهُمَا».
\par 6 فَخَرَجَ الَيْهِمْ لُوطٌ الَى الْبَابِ وَاغْلَقَ الْبَابَ وَرَاءَهُ
\par 7 وَقَالَ: «لا تَفْعَلُوا شَرّا يَا اخْوَتِي.
\par 8 هُوَذَا لِي ابْنَتَانِ لَمْ تَعْرِفَا رَجُلا. اخْرِجُهُمَا الَيْكُمْ فَافْعَلُوا بِهِمَا كَمَا يَحْسُنُ فِي عُيُونِكُمْ. وَامَّا هَذَانِ الرَّجُلانِ فَلا تَفْعَلُوا بِهِمَا شَيْئا لانَّهُمَا قَدْ دَخَلا تَحْتَ ظِلِّ سَقْفِي».
\par 9 فَقَالُوا: «ابْعُدْ الَى هُنَاكَ». ثُمَّ قَالُوا: «جَاءَ هَذَا الانْسَانُ لِيَتَغَرَّبَ وَهُوَ يَحْكُمُ حُكْما. الْانَ نَفْعَلُ بِكَ شَرّا اكْثَرَ مِنْهُمَا». فَالَحُّوا عَلَى لُوطٍ جِدّا وَتَقَدَّمُوا لِيُكَسِّرُوا الْبَابَ
\par 10 فَمَدَّ الرَّجُلانِ ايْدِيَهُمَا وَادْخَلا لُوطا الَيْهِمَا الَى الْبَيْتِ وَاغْلَقَا الْبَابَ.
\par 11 وَامَّا الرِّجَالُ الَّذِينَ عَلَى بَابِ الْبَيْتِ فَضَرَبَاهُمْ بِالْعَمَى مِنَ الصَّغِيرِ الَى الْكَبِيرِ فَعَجِزُوا عَنْ انْ يَجِدُوا الْبَابَ.
\par 12 وَقَالَ الرَّجُلانِ لِلُوطٍ: «مَنْ لَكَ ايْضا هَهُنَا؟ اصْهَارَكَ وَبَنِيكَ وَبَنَاتِكَ وَكُلَّ مَنْ لَكَ فِي الْمَدِينَةِ اخْرِجْ مِنَ الْمَكَانِ
\par 13 لانَّنَا مُهْلِكَانِ هَذَا الْمَكَانَ اذْ قَدْ عَظُمَ صُرَاخُهُمْ امَامَ الرَّبِّ فَارْسَلَنَا الرَّبُّ لِنُهْلِكَهُ».
\par 14 فَخَرَجَ لُوطٌ وَكَلَّمَ اصْهَارَهُ الْاخِذِينَ بَنَاتِهِ وَقَالَ: «قُومُوا اخْرُجُوا مِنْ هَذَا الْمَكَانِ لانَّ الرَّبَّ مُهْلِكٌ الْمَدِينَةَ». فَكَانَ كَمَازِحٍ فِي اعْيُنِ اصْهَارِهِ.
\par 15 وَلَمَّا طَلَعَ الْفَجْرُ كَانَ الْمَلاكَانِ يُعَجِّلانِ لُوطا قَائِلَيْنِ: «قُمْ خُذِ امْرَاتَكَ وَابْنَتَيْكَ الْمَوْجُودَتَيْنِ لِئَلَّا تَهْلَِكَ بِاثْمِ الْمَدِينَةِ».
\par 16 وَلَمَّا تَوَانَى امْسَكَ الرَّجُلانِ بِيَدِهِ وَبِيَدِ امْرَاتِهِ وَبِيَدِ ابْنَتَيْهِ - لِشَفَقَةِ الرَّبِّ عَلَيْهِ - وَاخْرَجَاهُ وَوَضَعَاهُ خَارِجَ الْمَدِينَةِ.
\par 17 وَكَانَ لَمَّا اخْرَجَاهُمْ الَى خَارِجٍ انَّهُ قَالَ: «اهْرُبْ لِحَيَاتِكَ. لا تَنْظُرْ الَى وَرَائِكَ وَلا تَقِفْ فِي كُلِّ الدَّائِرَةِ. اهْرُبْ الَى الْجَبَلِ لِئَلَّا تَهْلَِكَ».
\par 18 فَقَالَ لَهُمَا لُوطٌ: «لا يَا سَيِّدُ.
\par 19 هُوَذَا عَبْدُكَ قَدْ وَجَدَ نِعْمَةً فِي عَيْنَيْكَ وَعَظَّمْتَ لُطْفَكَ الَّذِي صَنَعْتَ الَيَّ بِاسْتِبْقَاءِ نَفْسِي وَانَا لا اقْدِرُ انْ اهْرُبَ الَى الْجَبَلِ لَعَلَّ الشَّرَّ يُدْرِكُنِي فَامُوتَ.
\par 20 هُوَذَا الْمَدِينَةُ هَذِهِ قَرِيبَةٌ لِلْهَرَبِ الَيْهَا وَهِيَ صَغِيرَةٌ. اهْرُبُ الَى هُنَاكَ. (الَيْسَتْ هِيَ صَغِيرَةً؟) فَتَحْيَا نَفْسِي».
\par 21 فَقَالَ لَهُ: «انِّي قَدْ رَفَعْتُ وَجْهَكَ فِي هَذَا الامْرِ ايْضا انْ لا اقْلِبَ الْمَدِينَةَ الَّتِي تَكَلَّمْتَ عَنْهَا.
\par 22 اسْرِعِ اهْرُبْ الَى هُنَاكَ لانِّي لا اسْتَطِيعُ انْ افْعَلَ شَيْئا حَتَّى تَجِيءَ الَى هُنَاكَ». لِذَلِكَ دُعِيَ اسْمُ الْمَدِينَةِ «صُوغَرَ».
\par 23 وَاذْ اشْرَقَتِ الشَّمْسُ عَلَى الارْضِ دَخَلَ لُوطٌ الَى صُوغَرَ
\par 24 فَامْطَرَ الرَّبُّ عَلَى سَدُومَ وَعَمُورَةَ كِبْرِيتا وَنَارا مِنْ عِنْدِ الرَّبِّ مِنَ السَّمَاءِ.
\par 25 وَقَلَبَ تِلْكَ الْمُدُنَ وَكُلَّ الدَّائِرَةِ وَجَمِيعَ سُكَّانِ الْمُدُنِ وَنَبَاتَِ الارْضِ.
\par 26 وَنَظَرَتِ امْرَاتُهُ مِنْ وَرَائِهِ فَصَارَتْ عَمُودَ مِلْحٍ!
\par 27 وَبَكَّرَ ابْرَاهِيمُ فِي الْغَدِ الَى الْمَكَانِ الَّذِي وَقَفَ فِيهِ امَامَ الرَّبِّ
\par 28 وَتَطَلَّعَ نَحْوَ سَدُومَ وَعَمُورَةَ وَنَحْوَ كُلِّ ارْضِ الدَّائِرَةِ وَنَظَرَ وَاذَا دُخَانُ الارْضِ يَصْعَدُ كَدُخَانِ الاتُونِ.
\par 29 وَحَدَثَ لَمَّا اخْرَبَ اللهُ مُدُنَ الدَّائِرَةِ انَّ اللهَ ذَكَرَ ابْرَاهِيمَ وَارْسَلَ لُوطا مِنْ وَسَطِ الِانْقِلابِ. حِينَ قَلَبَ الْمُدُنَ الَّتِي سَكَنَ فِيهَا لُوطٌ.
\par 30 وَصَعِدَ لُوطٌ مِنْ صُوغَرَ وَسَكَنَ فِي الْجَبَلِ وَابْنَتَاهُ مَعَهُ لانَّهُ خَافَ انْ يَسْكُنَ فِي صُوغَرَ. فَسَكَنَ فِي الْمَغَارَةِ هُوَ وَابْنَتَاهُ.
\par 31 وَقَالَتِ الْبِكْرُ لِلصَّغِيرَةِ: «ابُونَا قَدْ شَاخَ وَلَيْسَ فِي الارْضِ رَجُلٌ لِيَدْخُلَ عَلَيْنَا كَعَادَةِ كُلِّ الارْضِ.
\par 32 هَلُمَّ نَسْقِي ابَانَا خَمْرا وَنَضْطَجِعُ مَعَهُ فَنُحْيِي مِنْ ابِينَا نَسْلا».
\par 33 فَسَقَتَا ابَاهُمَا خَمْرا فِي تِلْكَ اللَّيْلَةِ وَدَخَلَتِ الْبِكْرُ وَاضْطَجَعَتْ مَعَ ابِيهَا وَلَمْ يَعْلَمْ بِاضْطِجَاعِهَا وَلا بِقِيَامِهَا.
\par 34 وَحَدَثَ فِي الْغَدِ انَّ الْبِكْرَ قَالَتْ لِلصَّغِيرَةِ: «انِّي قَدِ اضْطَجَعْتُ الْبَارِحَةَ مَعَ ابِي. نَسْقِيهِ خَمْرا اللَّيْلَةَ ايْضا فَادْخُلِي اضْطَجِعِي مَعَهُ فَنُحْيِيَ مِنْ ابِينَا نَسْلا».
\par 35 فَسَقَتَا ابَاهُمَا خَمْرا فِي تِلْكَ اللَّيْلَةِ ايْضا وَقَامَتِ الصَّغِيرَةُ وَاضْطَجَعَتْ مَعَهُ وَلَمْ يَعْلَمْ بِاضْطِجَاعِهَا وَلا بِقِيَامِهَا
\par 36 فَحَبِلَتِ ابْنَتَا لُوطٍ مِنْ ابِيهِمَا.
\par 37 فَوَلَدَتِ الْبِكْرُ ابْنا وَدَعَتِ اسْمَهُ «مُوابَ» - وَهُوَ ابُو الْمُوابِيِّينَ الَى الْيَوْمِ.
\par 38 وَالصَّغِيرَةُ ايْضا وَلَدَتِ ابْنا وَدَعَتِ اسْمَهُ «بِنْ عَمِّي» - وَهُوَ ابُو بَنِي عَمُّونَ الَى الْيَوْمِ.

\chapter{20}

\par 1 وَانْتَقَلَ ابْرَاهِيمُ مِنْ هُنَاكَ الَى ارْضِ الْجَنُوبِ وَسَكَنَ بَيْنَ قَادِشَ وَشُورَ وَتَغَرَّبَ فِي جَرَارَ.
\par 2 وَقَالَ ابْرَاهِيمُ عَنْ سَارَةَ امْرَاتِهِ: «هِيَ اخْتِي». فَارْسَلَ ابِيمَالِكُ مَلِكُ جَرَارَ وَاخَذَ سَارَةَ.
\par 3 فَجَاءَ اللهُ الَى ابِيمَالِكَ فِي حُلْمِ اللَّيْلِ وَقَالَ لَهُ: «هَا انْتَ مَيِّتٌ مِنْ اجْلِ الْمَرْاةِ الَّتِي اخَذْتَهَا فَانَّهَا مُتَزَوِّجَةٌ بِبَعْلٍ».
\par 4 وَلَكِنْ لَمْ يَكُنْ ابِيمَالِكُ قَدِ اقْتَرَبَ الَيْهَا. فَقَالَ: «يَا سَيِّدُ اامَّةً بَارَّةً تَقْتُلُ؟
\par 5 الَمْ يَقُلْ هُوَ لِي انَّهَا اخْتِي وَهِيَ ايْضا نَفْسُهَا قَالَتْ هُوَ اخِي؟ بِسَلامَةِ قَلْبِي وَنَقَاوَةِ يَدَيَّ فَعَلْتُ هَذَا».
\par 6 فَقَالَ لَهُ اللهُ فِي الْحُلْمِ: «انَا ايْضا عَلِمْتُ انَّكَ بِسَلامَةِ قَلْبِكَ فَعَلْتَ هَذَا. وَانَا ايْضا امْسَكْتُكَ عَنْ انْ تُخْطِئَ الَيَّ لِذَلِكَ لَمْ ادَعْكَ تَمَسُّهَا.
\par 7 فَالْانَ رُدَّ امْرَاةَ الرَّجُلِ فَانَّهُ نَبِيٌّ فَيُصَلِّيَ لاجْلِكَ فَتَحْيَا. وَانْ كُنْتَ لَسْتَ تَرُدُّهَا فَاعْلَمْ انَّكَ مَوْتا تَمُوتُ انْتَ وَكُلُّ مَنْ لَكَ».
\par 8 فَبَكَّرَ ابِيمَالِكُ فِي الْغَدِ وَدَعَا جَمِيعَ عَبِيدِهِ وَتَكَلَّمَ بِكُلِّ هَذَا الْكَلامِ فِي مَسَامِعِهِمْ. فَخَافَ الرِّجَالُ جِدّا.
\par 9 ثُمَّ دَعَا ابِيمَالِكُ ابْرَاهِيمَ وَقَالَ لَهُ: «مَاذَا فَعَلْتَ بِنَا وَبِمَاذَا اخْطَاتُ الَيْكَ حَتَّى جَلَبْتَ عَلَيَّ وَعَلَى مَمْلَكَتِي خَطِيَّةً عَظِيمَةً؟ اعْمَالا لا تُعْمَلُ عَمِلْتَ بِي!».
\par 10 وَقَالَ ابِيمَالِكُ لابْرَاهِيمَ: «مَاذَا رَايْتَ حَتَّى عَمِلْتَ هَذَا الشَّيْءَ؟»
\par 11 فَقَالَ ابْرَاهِيمُ: «انِّي قُلْتُ: لَيْسَ فِي هَذَا الْمَوْضِعِ خَوْفُ اللهِ الْبَتَّةَ فَيَقْتُلُونَنِي لاجْلِ امْرَاتِي.
\par 12 وَبِالْحَقِيقَةِ ايْضا هِيَ اخْتِي ابْنَةُ ابِي غَيْرَ انَّهَا لَيْسَتِ ابْنَةَ امِّي فَصَارَتْ لِي زَوْجَةً.
\par 13 وَحَدَثَ لَمَّا اتَاهَنِي اللهُ مِنْ بَيْتِ ابِي انِّي قُلْتُ لَهَا: هَذَا مَعْرُوفُكِ الَّذِي تَصْنَعِينَ الَيَّ: فِي كُلِّ مَكَانٍ نَاتِي الَيْهِ قُولِي عَنِّي هُوَ اخِي».
\par 14 فَاخَذَ ابِيمَالِكُ غَنَما وَبَقَرا وَعَبِيدا وَامَاءً وَاعْطَاهَا لابْرَاهِيمَ وَرَدَّ الَيْهِ سَارَةَ امْرَاتَهُ.
\par 15 وَقَالَ ابِيمَالِكُ: «هُوَذَا ارْضِي قُدَّامَكَ. اسْكُنْ فِي مَا حَسُنَ فِي عَيْنَيْكَ».
\par 16 وَقَالَ لِسَارَةَ: «انِّي قَدْ اعْطَيْتُ اخَاكِ الْفا مِنَ الْفِضَّةِ. هَا هُوَ لَكِ غِطَاءُ عَيْنٍ مِنْ جِهَةِ كُلِّ مَا عِنْدَكِ وَعِنْدَ كُلِّ وَاحِدٍ فَانْصِفْتِ».
\par 17 فَصَلَّى ابْرَاهِيمُ الَى اللهِ فَشَفَى اللهُ ابِيمَالِكَ وَامْرَاتَهُ وَجَوَارِيَهُ فَوَلَدْنَ -
\par 18 لانَّ الرَّبَّ كَانَ قَدْ اغْلَقَ كُلَّ رَحِمٍ لِبَيْتِ ابِيمَالِكَ بِسَبَبِ سَارَةَ امْرَاةِ ابْرَاهِيمَ.

\chapter{21}

\par 1 وَافْتَقَدَ الرَّبُّ سَارَةَ كَمَا قَالَ وَفَعَلَ الرَّبُّ لِسَارَةَ كَمَا تَكَلَّمَ.
\par 2 فَحَبِلَتْ سَارَةُ وَوَلَدَتْ لابْرَاهِيمَ ابْنا فِي شَيْخُوخَتِهِ فِي الْوَقْتِ الَّذِي تَكَلَّمَ اللهُ عَنْهُ.
\par 3 وَدَعَا ابْرَاهِيمُ اسْمَ ابْنِهِ الْمَوْلُودِ لَهُ الَّذِي وَلَدَتْهُ لَهُ سَارَةُ «اسْحَاقَ».
\par 4 وَخَتَنَ ابْرَاهِيمُ اسْحَاقَ ابْنَهُ وَهُوَ ابْنُ ثَمَانِيَةِ ايَّامٍ كَمَا امَرَهُ اللهُ.
\par 5 وَكَانَ ابْرَاهِيمُ ابْنَ مِئَةِ سَنَةٍ حِينَ وُلِدَ لَهُ اسْحَاقُ ابْنُهُ.
\par 6 وَقَالَتْ سَارَةُ: «قَدْ صَنَعَ الَيَّ اللهُ ضِحْكا. كُلُّ مَنْ يَسْمَعُ يَضْحَكُ لِي».
\par 7 وَقَالَتْ: «مَنْ قَالَ لابْرَاهِيمَ: سَارَةُ تُرْضِعُ بَنِينَ حَتَّى وَلَدْتُ ابْنا فِي شَيْخُوخَتِهِ!»
\par 8 فَكَبِرَ الْوَلَدُ وَفُطِمَ. وَصَنَعَ ابْرَاهِيمُ وَلِيمَةً عَظِيمَةً يَوْمَ فِطَامِ اسْحَاقَ.
\par 9 وَرَاتْ سَارَةُ ابْنَ هَاجَرَ الْمِصْرِيَّةِ الَّذِي وَلَدَتْهُ لابْرَاهِيمَ يَمْزَحُ
\par 10 فَقَالَتْ لابْرَاهِيمَ: «اطْرُدْ هَذِهِ الْجَارِيَةَ وَابْنَهَا لانَّ ابْنَ هَذِهِ الْجَارِيَةِ لا يَرِثُ مَعَ ابْنِي اسْحَاقَ».
\par 11 فَقَبُحَ الْكَلامُ جِدّا فِي عَيْنَيْ ابْرَاهِيمَ لِسَبَبِ ابْنِهِ.
\par 12 فَقَالَ اللهُ لابْرَاهِيمَ: «لا يَقْبُحُ فِي عَيْنَيْكَ مِنْ اجْلِ الْغُلامِ وَمِنْ اجْلِ جَارِيَتِكَ. فِي كُلِّ مَا تَقُولُ لَكَ سَارَةُ اسْمَعْ لِقَوْلِهَا لانَّهُ بِاسْحَاقَ يُدْعَى لَكَ نَسْلٌ.
\par 13 وَابْنُ الْجَارِيَةِ ايْضا سَاجْعَلُهُ امَّةً لانَّهُ نَسْلُكَ».
\par 14 فَبَكَّرَ ابْرَاهِيمُ صَبَاحا وَاخَذَ خُبْزا وَقِرْبَةَ مَاءٍ وَاعْطَاهُمَا لِهَاجَرَ وَاضِعا ايَّاهُمَا عَلَى كَتِفِهَا وَالْوَلَدَ وَصَرَفَهَا. فَمَضَتْ وَتَاهَتْ فِي بَرِّيَّةِ بِئْرِ سَبْعٍ.
\par 15 وَلَمَّا فَرَغَ الْمَاءُ مِنَ الْقِرْبَةِ طَرَحَتِ الْوَلَدَ تَحْتَ احْدَى الاشْجَارِ
\par 16 وَمَضَتْ وَجَلَسَتْ مُقَابِلَهُ بَعِيدا نَحْوَ رَمْيَةِ قَوْسٍ لانَّهَا قَالَتْ: «لا انْظُرُ مَوْتَ الْوَلَدِ». فَجَلَسَتْ مُقَابِلَهُ وَرَفَعَتْ صَوْتَهَا وَبَكَتْ.
\par 17 فَسَمِعَ اللهُ صَوْتَ الْغُلامِ. وَنَادَى مَلاكُ اللهِ هَاجَرَ مِنَ السَّمَاءِ وَقَالَ لَهَا: «مَا لَكِ يَا هَاجَرُ؟ لا تَخَافِي لانَّ اللهَ قَدْ سَمِعَ لِصَوْتِ الْغُلامِ حَيْثُ هُوَ.
\par 18 قُومِي احْمِلِي الْغُلامَ وَشُدِّي يَدَكِ بِهِ لانِّي سَاجْعَلُهُ امَّةً عَظِيمَةً».
\par 19 وَفَتَحَ اللهُ عَيْنَيْهَا فَابْصَرَتْ بِئْرَ مَاءٍ فَذَهَبَتْ وَمَلَاتِ الْقِرْبَةَ مَاءً وَسَقَتِ الْغُلامَ.
\par 20 وَكَانَ اللهُ مَعَ الْغُلامِ فَكَبِرَ وَسَكَنَ فِي الْبَرِّيَّةِ وَكَانَ يَنْمُو رَامِيَ قَوْسٍ.
\par 21 وَسَكَنَ فِي بَرِّيَّةِ فَارَانَ. وَاخَذَتْ لَهُ امُّهُ زَوْجَةً مِنْ ارْضِ مِصْرَ.
\par 22 وَحَدَثَ فِي ذَلِكَ الزَّمَانِ انَّ ابِيمَالِكَ وَفِيكُولَ رَئِيسَ جَيْشِهِ قَالا لابْرَاهِيمَ: «اللهُ مَعَكَ فِي كُلِّ مَا انْتَ صَانِعٌ.
\par 23 فَالْانَ احْلِفْ لِي بِاللهِ هَهُنَا انَّكَ لا تَغْدُرُ بِي وَلا بِنَسْلِي وَذُرِّيَّتِي. كَالْمَعْرُوفِ الَّذِي صَنَعْتُ الَيْكَ تَصْنَعُ الَيَّ وَالَى الارْضِ الَّتِي تَغَرَّبْتَ فِيهَا».
\par 24 فَقَالَ ابْرَاهِيمُ: «انَا احْلِفُ».
\par 25 وَعَاتَبَ ابْرَاهِيمُ ابِيمَالِكَ لِسَبَبِ بِئْرِ الْمَاءِ الَّتِي اغْتَصَبَهَا عَبِيدُ ابِيمَالِكَ.
\par 26 فَقَالَ ابِيمَالِكُ: «لَمْ اعْلَمْ مَنْ فَعَلَ هَذَا الامْرَ. انْتَ لَمْ تُخْبِرْنِي وَلا انَا سَمِعْتُ سِوَى الْيَوْمِ».
\par 27 فَاخَذَ ابْرَاهِيمُ غَنَما وَبَقَرا وَاعْطَى ابِيمَالِكَ فَقَطَعَا كِلاهُمَا مِيثَاقا.
\par 28 وَاقَامَ ابْرَاهِيمُ سَبْعَ نِعَاجٍ مِنَ الْغَنَمِ وَحْدَهَا.
\par 29 فَقَالَ ابِيمَالِكُ لابْرَاهِيمَ: «مَا هِيَ هَذِهِ السَّبْعُ النِّعَاجِ الَّتِي اقَمْتَهَا وَحْدَهَا؟»
\par 30 فَقَالَ: «انَّكَ سَبْعَ نِعَاجٍ تَاخُذُ مِنْ يَدِي لِكَيْ تَكُونَ لِي شَهَادَةً بِانِّي حَفَرْتُ هَذِهِ الْبِئْرَ».
\par 31 لِذَلِكَ دَعَا ذَلِكَ الْمَوْضِعَ بِئْرَ سَبْعٍ. لانَّهُمَا هُنَاكَ حَلَفَا كِلاهُمَا.
\par 32 فَقَطَعَا مِيثَاقا فِي بِئْرِ سَبْعٍ. ثُمَّ قَامَ ابِيمَالِكُ وَفِيكُولُ رَئِيسُ جَيْشِهِ وَرَجَعَا الَى ارْضِ الْفَلَسْطِينِيِّينَ.
\par 33 وَغَرَسَ ابْرَاهِيمُ اثْلا فِي بِئْرِ سَبْعٍ وَدَعَا هُنَاكَ بِاسْمِ الرَّبِّ «الالَهِ السَّرْمَدِيِّ».
\par 34 وَتَغَرَّبَ ابْرَاهِيمُ فِي ارْضِ الْفَلَسْطِينِيِّينَ ايَّاما كَثِيرَةً.

\chapter{22}

\par 1 وَحَدَثَ بَعْدَ هَذِهِ الامُورِ انَّ اللهَ امْتَحَنَ ابْرَاهِيمَ فَقَالَ لَهُ: «يَا ابْرَاهِيمُ». فَقَالَ: «هَئَنَذَا».
\par 2 فَقَالَ: «خُذِ ابْنَكَ وَحِيدَكَ الَّذِي تُحِبُّهُ اسْحَاقَ وَاذْهَبْ الَى ارْضِ الْمُرِيَّا وَاصْعِدْهُ هُنَاكَ مُحْرَقَةً عَلَى احَدِ الْجِبَالِ الَّذِي اقُولُ لَكَ».
\par 3 فَبَكَّرَ ابْرَاهِيمُ صَبَاحا وَشَدَّ عَلَى حِمَارِهِ وَاخَذَ اثْنَيْنِ مِنْ غِلْمَانِهِ مَعَهُ وَاسْحَاقَ ابْنَهُ وَشَقَّقَ حَطَبا لِمُحْرَقَةٍ وَقَامَ وَذَهَبَ الَى الْمَوْضِعِ الَّذِي قَالَ لَهُ اللهُ.
\par 4 وَفِي الْيَوْمِ الثَّالِثِ رَفَعَ ابْرَاهِيمُ عَيْنَيْهِ وَابْصَرَ الْمَوْضِعَ مِنْ بَعِيدٍ
\par 5 فَقَالَ ابْرَاهِيمُ لِغُلامَيْهِ: «اجْلِسَا انْتُمَا هَهُنَا مَعَ الْحِمَارِ وَامَّا انَا وَالْغُلامُ فَنَذْهَبُ الَى هُنَاكَ وَنَسْجُدُ ثُمَّ نَرْجِعُ الَيْكُمَا».
\par 6 فَاخَذَ ابْرَاهِيمُ حَطَبَ الْمُحْرَقَةِ وَوَضَعَهُ عَلَى اسْحَاقَ ابْنِهِ وَاخَذَ بِيَدِهِ النَّارَ وَالسِّكِّينَ. فَذَهَبَا كِلاهُمَا مَعا.
\par 7 وَقَالَ اسْحَاقُ لابْرَاهِيمَ ابِيهِ: «يَا ابِي». فَقَالَ: «هَئَنَذَا يَا ابْنِي». فَقَالَ: «هُوَذَا النَّارُ وَالْحَطَبُ وَلَكِنْ ايْنَ الْخَرُوفُ لِلْمُحْرَقَةِ؟»
\par 8 فَقَالَ ابْرَاهِيمُ: «اللهُ يَرَى لَهُ الْخَرُوفَ لِلْمُحْرَقَةِ يَا ابْنِي». فَذَهَبَا كِلاهُمَا مَعا.
\par 9 فَلَمَّا اتَيَا الَى الْمَوْضِعِ الَّذِي قَالَ لَهُ اللهُ بَنَى هُنَاكَ ابْرَاهِيمُ الْمَذْبَحَ وَرَتَّبَ الْحَطَبَ وَرَبَطَ اسْحَاقَ ابْنَهُ وَوَضَعَهُ عَلَى الْمَذْبَحِ فَوْقَ الْحَطَبِ.
\par 10 ثُمَّ مَدَّ ابْرَاهِيمُ يَدَهُ وَاخَذَ السِّكِّينَ لِيَذْبَحَ ابْنَهُ.
\par 11 فَنَادَاهُ مَلاكُ الرَّبِّ مِنَ السَّمَاءِ وَقَالَ: «ابْرَاهِيمُ ابْرَاهِيمُ». فَقَالَ: «هَئَنَذَا»
\par 12 فَقَالَ: «لا تَمُدَّ يَدَكَ الَى الْغُلامِ وَلا تَفْعَلْ بِهِ شَيْئا لانِّي الْانَ عَلِمْتُ انَّكَ خَائِفٌ اللهَ فَلَمْ تُمْسِكِ ابْنَكَ وَحِيدَكَ عَنِّي».
\par 13 فَرَفَعَ ابْرَاهِيمُ عَيْنَيْهِ وَنَظَرَ وَاذَا كَبْشٌ وَرَاءَهُ مُمْسَكا فِي الْغَابَةِ بِقَرْنَيْهِ فَذَهَبَ ابْرَاهِيمُ وَاخَذَ الْكَبْشَ وَاصْعَدَهُ مُحْرَقَةً عِوَضا عَنِ ابْنِهِ.
\par 14 فَدَعَا ابْرَاهِيمُ اسْمَ ذَلِكَ الْمَوْضِعِ «يَهْوَهْ يِرْاهْ». حَتَّى انَّهُ يُقَالُ الْيَوْمَ: «فِي جَبَلِ الرَّبِّ يُرَى».
\par 15 وَنَادَى مَلاكُ الرَّبِّ ابْرَاهِيمَ ثَانِيَةً مِنَ السَّمَاءِ
\par 16 وَقَالَ: «بِذَاتِي اقْسَمْتُ يَقُولُ الرَّبُّ انِّي مِنْ اجْلِ انَّكَ فَعَلْتَ هَذَا الامْرَ وَلَمْ تُمْسِكِ ابْنَكَ وَحِيدَكَ
\par 17 ابَارِكُكَ مُبَارَكَةً وَاكَثِّرُ نَسْلَكَ تَكْثِيرا كَنُجُومِ السَّمَاءِ وَكَالرَّمْلِ الَّذِي عَلَى شَاطِئِ الْبَحْرِ وَيَرِثُ نَسْلُكَ بَابَ اعْدَائِهِ
\par 18 وَيَتَبَارَكُ فِي نَسْلِكَ جَمِيعُ امَمِ الارْضِ مِنْ اجْلِ انَّكَ سَمِعْتَ لِقَوْلِي».
\par 19 ثُمَّ رَجَعَ ابْرَاهِيمُ الَى غُلامَيْهِ فَقَامُوا وَذَهَبُوا مَعا الَى بِئْرِ سَبْعٍ. وَسَكَنَ ابْرَاهِيمُ فِي بِئْرِ سَبْعٍ.
\par 20 وَحَدَثَ بَعْدَ هَذِهِ الامُورِ انَّهُ قِيلَ لابْرَاهِيمَ: «هُوَذَا مِلْكَةُ قَدْ وَلَدَتْ هِيَ ايْضا بَنِينَ لِنَاحُورَ اخِيكَ:
\par 21 عُوصا بِكْرَهُ وَبُوزا اخَاهُ وَقَمُوئِيلَ ابَا ارَامَ
\par 22 وَكَاسَدَ وَحَزْوا وَفِلْدَاشَ وَيِدْلافَ وَبَتُوئِيلَ».
\par 23 وَوَلَدَ بَتُوئِيلُ رِفْقَةَ. هَؤُلاءِ الثَّمَانِيَةُ وَلَدَتْهُمْ مِلْكَةُ لِنَاحُورَ اخِي ابْرَاهِيمَ.
\par 24 وَامَّا سُرِّيَّتُهُ وَاسْمُهَا رَؤُومَةُ فَوَلَدَتْ هِيَ ايْضا طَابَحَ وَجَاحَمَ وَتَاحَشَ وَمَعْكَةَ.

\chapter{23}

\par 1 وَكَانَتْ حَيَاةُ سَارَةَ مِئَةً وَسَبْعا وَعِشْرِينَ سَنَةً سِنِي حَيَاةِ سَارَةَ.
\par 2 وَمَاتَتْ سَارَةُ فِي قَرْيَةِ ارْبَعَ (الَّتِي هِيَ حَبْرُونُ) فِي ارْضِ كَنْعَانَ. فَاتَى ابْرَاهِيمُ لِيَنْدُبَ سَارَةَ وَيَبْكِيَ عَلَيْهَا.
\par 3 وَقَامَ ابْرَاهِيمُ مِنْ امَامِ مَيِّتِهِ وَقَالَ لِبَنِي حِثَّ:
\par 4 «انَا غَرِيبٌ وَنَزِيلٌ عِنْدَكُمْ. اعْطُونِي مُلْكَ قَبْرٍ مَعَكُمْ لادْفِنَ مَيِّتِي مِنْ امَامِي».
\par 5 فَاجَابَ بَنُو حِثَّ ابْرَاهِيمَ:
\par 6 «اسْمَعْنَا يَا سَيِّدِي انْتَ رَئِيسٌ مِنَ اللهِ بَيْنَنَا. فِي افْضَلِ قُبُورِنَا ادْفِنْ مَيِّتَكَ. لا يَمْنَعُ احَدٌ مِنَّا قَبْرَهُ عَنْكَ حَتَّى لا تَدْفِنَ مَيِّتَكَ».
\par 7 فَقَامَ ابْرَاهِيمُ وَسَجَدَ لِشَعْبِ الارْضِ لِبَنِي حِثَّ
\par 8 وَقَالَ: «انْ كَانَ فِي نُفُوسِكُمْ انْ ادْفِنَ مَيِّتِي مِنْ امَامِي فَاسْمَعُونِي وَالْتَمِسُوا لِي مِنْ عِفْرُونَ بْنِ صُوحَرَ
\par 9 انْ يُعْطِيَنِي مَغَارَةَ الْمَكْفِيلَةِ الَّتِي لَهُ الَّتِي فِي طَرَفِ حَقْلِهِ. بِثَمَنٍ كَامِلٍ يُعْطِينِي ايَّاهَا فِي وَسَطِكُمْ مُلْكَ قَبْرٍ».
\par 10 وَكَانَ عِفْرُونُ جَالِسا بَيْنَ بَنِي حِثَّ. فَاجَابَ عِفْرُونُ الْحِثِّيُّ ابْرَاهِيمَ فِي مَسَامِعِ بَنِي حِثَّ لَدَى جَمِيعِ الدَّاخِلِينَ بَابَ مَدِينَتِهِ:
\par 11 «لا يَا سَيِّدِي اسْمَعْنِي. الْحَقْلُ وَهَبْتُكَ ايَّاهُ وَالْمَغَارَةُ الَّتِي فِيهِ لَكَ وَهَبْتُهَا. لَدَى عُيُونِ بَنِي شَعْبِي وَهَبْتُكَ ايَّاهَا. ادْفِنْ مَيِّتَكَ».
\par 12 فَسَجَدَ ابْرَاهِيمُ امَامَ شَعْبِ الارْضِ
\par 13 وَقَالَ لِعِفْرُونَ فِي مَسَامِعِ شَعْبِ الارْضِ: «بَلْ انْ كُنْتَ انْتَ ايَّاهُ فَلَيْتَكَ تَسْمَعُنِي. اعْطِيكَ ثَمَنَ الْحَقْلِ. خُذْ مِنِّي فَادْفِنَ مَيِّتِي هُنَاكَ».
\par 14 فَاجَابَ عِفْرُونُ ابْرَاهِيمَ:
\par 15 «يَا سَيِّدِي اسْمَعْنِي. ارْضٌ بِارْبَعِ مِئَةِ شَاقِلِ فِضَّةٍ مَا هِيَ بَيْنِي وَبَيْنَكَ؟ فَادْفِنْ مَيِّتَكَ».
\par 16 فَسَمِعَ ابْرَاهِيمُ لِعِفْرُونَ وَوَزَنَ ابْرَاهِيمُ لِعِفْرُونَ الْفِضَّةَ الَّتِي ذَكَرَهَا فِي مَسَامِعِ بَنِي حِثَّ. ارْبَعَ مِئَةِ شَاقِلِ فِضَّةٍ جَائِزَةٍ عِنْدَ التُّجَّارِ.
\par 17 فَوَجَبَ حَقْلُ عِفْرُونَ الَّذِي فِي الْمَكْفِيلَةِ الَّتِي امَامَ مَمْرَا الْحَقْلُ وَالْمَغَارَةُ الَّتِي فِيهِ وَجَمِيعُ الشَّجَرِ الَّذِي فِي الْحَقْلِ الَّذِي فِي جَمِيعِ حُدُودِهِ حَوَالَيْهِ
\par 18 لابْرَاهِيمَ مُلْكا لَدَى عُيُونِ بَنِي حِثَّ بَيْنَ جَمِيعِ الدَّاخِلِينَ بَابَ مَدِينَتِهِ.
\par 19 وَبَعْدَ ذَلِكَ دَفَنَ ابْرَاهِيمُ سَارَةَ امْرَاتَهُ فِي مَغَارَةِ حَقْلِ الْمَكْفِيلَةِ امَامَ مَمْرَا (الَّتِي هِيَ حَبْرُونُ) فِي ارْضِ كَنْعَانَ
\par 20 فَوَجَبَ الْحَقْلُ وَالْمَغَارَةُ الَّتِي فِيهِ لابْرَاهِيمَ مُلْكَ قَبْرٍ مِنْ عِنْدِ بَنِي حِثَّ.

\chapter{24}

\par 1 وَشَاخَ ابْرَاهِيمُ وَتَقَدَّمَ فِي الايَّامِ. وَبَارَكَ الرَّبُّ ابْرَاهِيمَ فِي كُلِّ شَيْءٍ.
\par 2 وَقَالَ ابْرَاهِيمُ لِعَبْدِهِ كَبِيرِ بَيْتِهِ الْمُسْتَوْلِي عَلَى كُلِّ مَا كَانَ لَهُ: «ضَعْ يَدَكَ تَحْتَ فَخْذِي
\par 3 فَاسْتَحْلِفَكَ بِالرَّبِّ الَهِ السَّمَاءِ وَالَهِ الارْضِ انْ لا تَاخُذَ زَوْجَةً لِابْنِي مِنْ بَنَاتِ الْكَنْعَانِيِّينَ الَّذِينَ انَا سَاكِنٌ بَيْنَهُمْ
\par 4 بَلْ الَى ارْضِي وَالَى عَشِيرَتِي تَذْهَبُ وَتَاخُذُ زَوْجَةً لِابْنِي اسْحَاقَ».
\par 5 فَقَالَ لَهُ الْعَبْدُ: «رُبَّمَا لا تَشَاءُ الْمَرْاةُ انْ تَتْبَعَنِي الَى هَذِهِ الارْضِ. هَلْ ارْجِعُ بِابْنِكَ الَى الارْضِ الَّتِي خَرَجْتَ مِنْهَا؟»
\par 6 فَقَالَ لَهُ ابْرَاهِيمُ: «احْتَرِزْ مِنْ انْ تَرْجِعَ بِابْنِي الَى هُنَاكَ.
\par 7 الرَّبُّ الَهُ السَّمَاءِ الَّذِي اخَذَنِي مِنْ بَيْتِ ابِي وَمِنْ ارْضِ مِيلادِي وَالَّذِي كَلَّمَنِي وَالَّذِي اقْسَمَ لِي قَائِلا: لِنَسْلِكَ اعْطِي هَذِهِ الارْضَ هُوَ يُرْسِلُ مَلاكَهُ امَامَكَ فَتَاخُذُ زَوْجَةً لِابْنِي مِنْ هُنَاكَ.
\par 8 وَانْ لَمْ تَشَا الْمَرْاةُ انْ تَتْبَعَكَ تَبَرَّاتَ مِنْ حَلْفِي هَذَا. امَّا ابْنِي فَلا تَرْجِعْ بِهِ الَى هُنَاكَ».
\par 9 فَوَضَعَ الْعَبْدُ يَدَهُ تَحْتَ فَخْذِ ابْرَاهِيمَ مَوْلاهُ وَحَلَفَ لَهُ عَلَى هَذَا الامْرِ.
\par 10 ثُمَّ اخَذَ الْعَبْدُ عَشَرَةَ جِمَالٍ مِنْ جِمَالِ مَوْلاهُ وَمَضَى وَجَمِيعُ خَيْرَاتِ مَوْلاهُ فِي يَدِهِ. فَقَامَ وَذَهَبَ الَى ارَامِ النَّهْرَيْنِ الَى مَدِينَةِ نَاحُورَ.
\par 11 وَانَاخَ الْجِمَالَ خَارِجَ الْمَدِينَةِ عِنْدَ بِئْرِ الْمَاءِ وَقْتَ الْمَسَاءِ وَقْتَ خُرُوجِ الْمُسْتَقِيَاتِ.
\par 12 وَقَالَ: «ايُّهَا الرَّبُّ الَهَ سَيِّدِي ابْرَاهِيمَ يَسِّرْ لِي الْيَوْمَ وَاصْنَعْ لُطْفا الَى سَيِّدِي ابْرَاهِيمَ.
\par 13 هَا انَا وَاقِفٌ عَلَى عَيْنِ الْمَاءِ وَبَنَاتُ اهْلِ الْمَدِينَةِ خَارِجَاتٌ لِيَسْتَقِينَ مَاءً.
\par 14 فَلْيَكُنْ انَّ الْفَتَاةَ الَّتِي اقُولُ لَهَا: امِيلِي جَرَّتَكِ لاشْرَبَ فَتَقُولَ: اشْرَبْ وَانَا اسْقِي جِمَالَكَ ايْضا هِيَ الَّتِي عَيَّنْتَهَا لِعَبْدِكَ اسْحَاقَ. وَبِهَا اعْلَمُ انَّكَ صَنَعْتَ لُطْفا الَى سَيِّدِي».
\par 15 وَاذْ كَانَ لَمْ يَفْرَغْ بَعْدُ مِنَ الْكَلامِ اذَا رِفْقَةُ الَّتِي وُلِدَتْ لِبَتُوئِيلَ ابْنِ مِلْكَةَ امْرَاةِ نَاحُورَ اخِي ابْرَاهِيمَ خَارِجَةٌ وَجَرَّتُهَا عَلَى كَتِفِهَا.
\par 16 وَكَانَتِ الْفَتَاةُ حَسَنَةَ الْمَنْظَرِ جِدّا وَعَذْرَاءَ لَمْ يَعْرِفْهَا رَجُلٌ. فَنَزَلَتْ الَى الْعَيْنِ وَمَلَاتْ جَرَّتَهَا وَطَلَعَتْ.
\par 17 فَرَكَضَ الْعَبْدُ لِلِقَائِهَا وَقَالَ: «اسْقِينِي قَلِيلَ مَاءٍ مِنْ جَرَّتِكِ».
\par 18 فَقَالَتِ: «اشْرَبْ يَا سَيِّدِي». وَاسْرَعَتْ وَانْزَلَتْ جَرَّتَهَا عَلَى يَدِهَا وَسَقَتْهُ.
\par 19 وَلَمَّا فَرَغَتْ مِنْ سَقْيِهِ قَالَتْ: «اسْتَقِي لِجِمَالِكَ ايْضا حَتَّى تَفْرَغَ مِنَ الشُّرْبِ».
\par 20 فَاسْرَعَتْ وَافْرَغَتْ جَرَّتَهَا فِي الْمَسْقَاةِ وَرَكَضَتْ ايْضا الَى الْبِئْرِ لِتَسْتَقِيَ. فَاسْتَقَتْ لِكُلِّ جِمَالِهِ.
\par 21 وَالرَّجُلُ يَتَفَرَّسُ فِيهَا صَامِتا لِيَعْلَمَ: هَلْ انْجَحَ الرَّبُّ طَرِيقَهُ امْ لا؟
\par 22 وَحَدَثَ عِنْدَمَا فَرَغَتِ الْجِمَالُ مِنَ الشُّرْبِ انَّ الرَّجُلَ اخَذَ خِزَامَةَ ذَهَبٍ وَزْنُهَا نِصْفُ شَاقِلٍ وَسِوَارَيْنِ عَلَى يَدَيْهَا وَزْنُهُمَا عَشَرَةُ شَوَاقِلِ ذَهَبٍ.
\par 23 وَقَالَ: «بِنْتُ مَنْ انْتِ؟ اخْبِرِينِي. هَلْ فِي بَيْتِ ابِيكِ مَكَانٌ لَنَا لِنَبِيتَ؟»
\par 24 فَقَالَتْ لَهُ: «انَا بِنْتُ بَتُوئِيلَ ابْنِ مِلْكَةَ الَّذِي وَلَدَتْهُ لِنَاحُورَ».
\par 25 وَقَالَتْ لَهُ: «عَِنْدَنَا تِبْنٌ وَعَلَفٌ كَثِيرٌ وَمَكَانٌ لِتَبِيتُوا ايْضا».
\par 26 فَخَرَّ الرَّجُلُ وَسَجَدَ لِلرَّبِّ
\par 27 وَقَالَ: «مُبَارَكٌ الرَّبُّ الَهُ سَيِّدِي ابْرَاهِيمَ الَّذِي لَمْ يَمْنَعْ لُطْفَهُ وَحَقَّهُ عَنْ سَيِّدِي. اذْ كُنْتُ انَا فِي الطَّرِيقِ هَدَانِي الرَّبُّ الَى بَيْتِ اخْوَةِ سَيِّدِي».
\par 28 فَرَكَضَتِ الْفَتَاةُ وَاخْبَرَتْ بَيْتَ امِّهَا بِحَسَبِ هَذِهِ الامُورِ.
\par 29 وَكَانَ لِرِفْقَةَ اخٌ اسْمُهُ لابَانُ. فَرَكَضَ لابَانُ الَى الرَّجُلِ خَارِجا الَى الْعَيْنِ.
\par 30 وَحَدَثَ انَّهُ اذْ رَاى الْخِزَامَةَ وَالسِّوَارَيْنِ عَلَى يَدَيْ اخْتِهِ وَاذْ سَمِعَ كَلامَ رِفْقَةَ اخْتِهِ قَائِلَةً: «هَكَذَا كَلَّمَنِي الرَّجُلُ» جَاءَ الَى الرَّجُلِ وَاذَا هُوَ وَاقِفٌ عِنْدَ الْجِمَالِ عَلَى الْعَيْنِ.
\par 31 فَقَالَ: «ادْخُلْ يَا مُبَارَكَ الرَّبِّ. لِمَاذَا تَقِفُ خَارِجا وَانَا قَدْ هَيَّاتُ الْبَيْتَ وَمَكَانا لِلْجِمَالِ؟»
\par 32 فَدَخَلَ الرَّجُلُ الَى الْبَيْتِ وَحَلَّ عَنِ الْجِمَالِ. فَاعْطَى تِبْنا وَعَلَفا لِلْجِمَالِ وَمَاءً لِغَسْلِ رِجْلَيْهِ وَارْجُلِ الرِّجَالِ الَّذِينَ مَعَهُ.
\par 33 وَوُضِعَ قُدَّامَهُ لِيَاكُلَ. فَقَالَ: «لا اكُلُ حَتَّى اتَكَلَّمَ كَلامِي». فَقَالَ: «تَكَلَّمْ».
\par 34 فَقَالَ: «انَا عَبْدُ ابْرَاهِيمَ.
\par 35 وَالرَّبُّ قَدْ بَارَكَ مَوْلايَ جِدّا فَصَارَ عَظِيما وَاعْطَاهُ غَنَما وَبَقَرا وَفِضَّةً وَذَهَبا وَعَبِيدا وَامَاءً وَجِمَالا وَحَمِيرا.
\par 36 وَوَلَدَتْ سَارَةُ امْرَاةُ سَيِّدِي ابْنا لِسَيِّدِي بَعْدَ مَا شَاخَتْ فَقَدْ اعْطَاهُ كُلَّ مَا لَهُ.
\par 37 وَاسْتَحْلَفَنِي سَيِّدِي قَائِلا:لا تَاخُذْ زَوْجَةً لِابْنِي مِنْ بَنَاتِ الْكَنْعَانِيِّينَ الَّذِينَ انَا سَاكِنٌ فِي ارْضِهِمْ
\par 38 بَلْ الَى بَيْتِ ابِي تَذْهَبُ وَالَى عَشِيرَتِي وَتَاخُذُ زَوْجَةً لِابْنِي.
\par 39 فَقُلْتُ لِسَيِّدِي: رُبَّمَا لا تَتْبَعُنِي الْمَرْاةُ.
\par 40 فَقَالَ لِي: انَّ الرَّبَّ الَّذِي سِرْتُ امَامَهُ يُرْسِلُ مَلاكَهُ مَعَكَ وَيُنْجِحُ طَرِيقَكَ فَتَاخُذُ زَوْجَةً لِابْنِي مِنْ عَشِيرَتِي وَمِنْ بَيْتِ ابِي.
\par 41 حِينَئِذٍ تَتَبَرَّا مِنْ حَلْفِي حِينَمَا تَجِيءُ الَى عَشِيرَتِي. وَانْ لَمْ يُعْطُوكَ فَتَكُونُ بَرِيئا مِنْ حَلْفِي.
\par 42 فَجِئْتُ الْيَوْمَ الَى الْعَيْنِ وَقُلْتُ: ايُّهَا الرَّبُّ الَهُ سَيِّدِي ابْرَاهِيمَ انْ كُنْتَ تُنْجِحُ طَرِيقِي الَّذِي انَا سَالِكٌ فِيهِ
\par 43 فَهَا انَا وَاقِفٌ عَلَى عَيْنِ الْمَاءِ وَلْيَكُنْ انَّ الْفَتَاةَ الَّتِي تَخْرُجُ لِتَسْتَقِيَ وَاقُولُ لَهَا: اسْقِينِي قَلِيلَ مَاءٍ مِنْ جَرَّتِكِ
\par 44 فَتَقُولَ لِيَ: اشْرَبْ انْتَ وَانَا اسْتَقِي لِجِمَالِكَ ايْضا هِيَ الْمَرْاةُ الَّتِي عَيَّنَهَا الرَّبُّ لِابْنِ سَيِّدِي.
\par 45 وَاذْ كُنْتُ انَا لَمْ افْرَغْ بَعْدُ مِنَ الْكَلامِ فِي قَلْبِي اذَا رِفْقَةُ خَارِجَةٌ وَجَرَّتُهَا عَلَى كَتِفِهَا فَنَزَلَتْ الَى الْعَيْنِ وَاسْتَقَتْ. فَقُلْتُ لَهَا: اسْقِينِي.
\par 46 فَاسْرَعَتْ وَانْزَلَتْ جَرَّتَهَا عَنْهَا وَقَالَتِ: اشْرَبْ وَانَا اسْقِي جِمَالَكَ ايْضا. فَشَرِبْتُ وَسَقَتِ الْجِمَالَ ايْضا.
\par 47 فَسَالْتُهَا: بِنْتُ مَنْ انْتِ؟ فَقَالَتْ: بِنْتُ بَتُوئِيلَ بْنِ نَاحُورَ الَّذِي وَلَدَتْهُ لَهُ مِلْكَةُ. فَوَضَعْتُ الْخِزَامَةَ فِي انْفِهَا وَالسِّوَارَيْنِ عَلَى يَدَيْهَا.
\par 48 وَخَرَرْتُ وَسَجَدْتُ لِلرَّبِّ وَبَارَكْتُ الرَّبَّ الَهَ سَيِّدِي ابْرَاهِيمَ الَّذِي هَدَانِي فِي طَرِيقٍ امِينٍ لِاخُذَ ابْنَةَ اخِي سَيِّدِي لِابْنِهِ.
\par 49 وَالْانَ انْ كُنْتُمْ تَصْنَعُونَ مَعْرُوفا وَامَانَةً الَى سَيِّدِي فَاخْبِرُونِي وَالَّا فَاخْبِرُونِي لانْصَرِفَ يَمِينا اوْ شِمَالا».
\par 50 فَاجَابَ لابَانُ وَبَتُوئِيلُ: «مِنْ عِنْدِ الرَّبِّ خَرَجَ الامْرُ. لا نَقْدِرُ انْ نُكَلِّمَكَ بِشَرٍّ اوْ خَيْرٍ.
\par 51 هُوَذَا رِفْقَةُ قُدَّامَكَ. خُذْهَا وَاذْهَبْ. فَلْتَكُنْ زَوْجَةً لِابْنِ سَيِّدِكَ كَمَا تَكَلَّمَ الرَّبُّ».
\par 52 وَكَانَ عِنْدَمَا سَمِعَ عَبْدُ ابْرَاهِيمَ كَلامَهُمْ انَّهُ سَجَدَ لِلرَّبِّ الَى الارْضِ.
\par 53 وَاخْرَجَ الْعَبْدُ انِيَةَ فِضَّةٍ وَانِيَةَ ذَهَبٍ وَثِيَابا وَاعْطَاهَا لِرِفْقَةَ وَاعْطَى تُحَفا لاخِيهَا وَلِامِّهَا.
\par 54 فَاكَلَ وَشَرِبَ هُوَ وَالرِّجَالُ الَّذِينَ مَعَهُ وَبَاتُوا. ثُمَّ قَامُوا صَبَاحا فَقَالَ: «اصْرِفُونِي الَى سَيِّدِي».
\par 55 فَقَالَ اخُوهَا وَامُّهَا: «لِتَمْكُثِ الْفَتَاةُ عِنْدَنَا ايَّاما اوْ عَشَرَةً بَعْدَ ذَلِكَ تَمْضِي».
\par 56 فَقَالَ لَهُمْ: «لا تُعَوِّقُونِي وَالرَّبُّ قَدْ انْجَحَ طَرِيقِي. اصْرِفُونِي لاذْهَبَ الَى سَيِّدِي».
\par 57 فَقَالُوا: «نَدْعُو الْفَتَاةَ وَنَسْالُهَا شِفَاها».
\par 58 فَدَعُوا رِفْقَةَ وَقَالُوا لَهَا: «هَلْ تَذْهَبِينَ مَعَ هَذَا الرَّجُلِ؟» فَقَالَتْ: «اذْهَبُ».
\par 59 فَصَرَفُوا رِفْقَةَ اخْتَهُمْ وَمُرْضِعَتَهَا وَعَبْدَ ابْرَاهِيمَ وَرِجَالَهُ.
\par 60 وَبَارَكُوا رِفْقَةَ وَقَالُوا لَهَا: «انْتِ اخْتُنَا. صِيرِي الُوفَ رَبَوَاتٍ وَلْيَرِثْ نَسْلُكِ بَابَ مُبْغِضِيهِ».
\par 61 فَقَامَتْ رِفْقَةُ وَفَتَيَاتُهَا وَرَكِبْنَ عَلَى الْجِمَالِ وَتَبِعْنَ الرَّجُلَ. فَاخَذَ الْعَبْدُ رِفْقَةَ وَمَضَى.
\par 62 وَكَانَ اسْحَاقُ قَدْ اتَى مِنْ وُرُودِ بِئْرِ لَحَيْ رُئِي - اذْ كَانَ سَاكِنا فِي ارْضِ الْجَنُوبِ.
\par 63 وَخَرَجَ اسْحَاقُ لِيَتَامَّلَ فِي الْحَقْلِ عِنْدَ اقْبَالِ الْمَسَاءِ فَرَفَعَ عَيْنَيْهِ وَنَظَرَ وَاذَا جِمَالٌ مُقْبِلَةٌ.
\par 64 وَرَفَعَتْ رِفْقَةُ عَيْنَيْهَا فَرَاتْ اسْحَاقَ فَنَزَلَتْ عَنِ الْجَمَلِ.
\par 65 وَقَالَتْ لِلْعَبْدِ: «مَنْ هَذَا الرَّجُلُ الْمَاشِي فِي الْحَقْلِ لِلِقَائِنَا؟» فَقَالَ الْعَبْدُ: «هُوَ سَيِّدِي». فَاخَذَتِ الْبُرْقُعَ وَتَغَطَّتْ.
\par 66 ثُمَّ حَدَّثَ الْعَبْدُ اسْحَاقَ بِكُلِّ الامُورِ الَّتِي صَنَعَ
\par 67 فَادْخَلَهَا اسْحَاقُ الَى خِبَاءِ سَارَةَ امِّهِ وَاخَذَ رِفْقَةَ فَصَارَتْ لَهُ زَوْجَةً وَاحَبَّهَا. فَتَعَزَّى اسْحَاقُ بَعْدَ مَوْتِ امِّهِ.

\chapter{25}

\par 1 وَعَادَ ابْرَاهِيمُ فَاخَذَ زَوْجَةً اسْمُهَا قَطُورَةُ
\par 2 فَوَلَدَتْ لَهُ زِمْرَانَ وَيَقْشَانَ وَمَدَانَ وَمِدْيَانَ وَيِشْبَاقَ وَشُوحا.
\par 3 وَوَلَدَ يَقْشَانُ: شَبَا وَدَدَانَ. وَكَانَ بَنُو دَدَانَ: اشُّورِيمَ وَلَطُوشِيمَ وَلَامِّيمَ.
\par 4 وَبَنُو مِدْيَانَ: عَيْفَةُ وَعِفْرُ وَحَنُوكُ وَابِيدَاعُ وَالْدَعَةُ. جَمِيعُ هَؤُلاءِ بَنُو قَطُورَةَ.
\par 5 وَاعْطَى ابْرَاهِيمُ اسْحَاقَ كُلَّ مَا كَانَ لَهُ.
\par 6 وَامَّا بَنُو السَّرَارِيِّ اللَّوَاتِي كَانَتْ لابْرَاهِيمَ فَاعْطَاهُمْ ابْرَاهِيمُ عَطَايَا وَصَرَفَهُمْ عَنْ اسْحَاقَ ابْنِهِ شَرْقا الَى ارْضِ الْمَشْرِقِ وَهُوَ بَعْدُ حَيٌّ.
\par 7 وَهَذِهِ ايَّامُ سِنِي حَيَاةِ ابْرَاهِيمَ الَّتِي عَاشَهَا: مِئَةٌ وَخَمْسٌ وَسَبْعُونَ سَنَةً.
\par 8 وَاسْلَمَ ابْرَاهِيمُ رُوحَهُ وَمَاتَ بِشَيْبَةٍ صَالِحَةٍ شَيْخا وَشَبْعَانَ ايَّاما وَانْضَمَّ الَى قَوْمِهِ.
\par 9 وَدَفَنَهُ اسْحَاقُ وَاسْمَاعِيلُ ابْنَاهُ فِي مَغَارَةِ الْمَكْفِيلَةِ فِي حَقْلِ عِفْرُونَ بْنِ صُوحَرَ الْحِثِّيِّ الَّذِي امَامَ مَمْرَا -
\par 10 الْحَقْلِ الَّذِي اشْتَرَاهُ ابْرَاهِيمُ مِنْ بَنِي حِثٍّ. هُنَاكَ دُفِنَ ابْرَاهِيمُ وَسَارَةُ امْرَاتُهُ.
\par 11 وَكَانَ بَعْدَ مَوْتِ ابْرَاهِيمَ انَّ اللهَ بَارَكَ اسْحَاقَ ابْنَهُ. وَسَكَنَ اسْحَاقُ عِنْدَ بِئْرِ لَحَيْ رُئِي.
\par 12 وَهَذِهِ مَوَالِيدُ اسْمَاعِيلَ بْنِ ابْرَاهِيمَ الَّذِي وَلَدَتْهُ هَاجَرُ الْمِصْرِيَّةُ جَارِيَةُ سَارَةَ لابْرَاهِيمَ.
\par 13 وَهَذِهِ اسْمَاءُ بَنِي اسْمَاعِيلَ بِاسْمَائِهِمْ حَسَبَ مَوَالِيدِهِمْ: نَبَايُوتُ بِكْرُ اسْمَاعِيلَ وَقِيدَارُ وَادَبْئِيلُ وَمِبْسَامُ
\par 14 وَمِشْمَاعُ وَدُومَةُ وَمَسَّا
\par 15 وَحَدَارُ وَتَيْمَا وَيَطُورُ وَنَافِيشُ وَقِدْمَةُ.
\par 16 هَؤُلاءِ هُمْ بَنُو اسْمَاعِيلَ وَهَذِهِ اسْمَاؤُهُمْ بِدِيَارِهِمْ وَحُصُونِهِمْ. اثْنَا عَشَرَ رَئِيسا حَسَبَ قَبَائِلِهِمْ.
\par 17 وَهَذِهِ سِنُو حَيَاةِ اسْمَاعِيلَ: مِئَةٌ وَسَبْعٌ وَثَلاثُونَ سَنَةً. وَاسْلَمَ رُوحَهُ وَمَاتَ وَانْضَمَّ الَى قَوْمِهِ.
\par 18 (وَسَكَنُوا مِنْ حَوِيلَةَ الَى شُورَ الَّتِي امَامَ مِصْرَ حِينَمَا تَجِيءُ نَحْوَ اشُّورَ). امَامَ جَمِيعِ اخْوَتِهِ نَزَلَ.
\par 19 وَهَذِهِ مَوَالِيدُ اسْحَاقَ بْنِ ابْرَاهِيمَ: وَلَدَ ابْرَاهِيمُ اسْحَاقَ.
\par 20 وَكَانَ اسْحَاقُ ابْنَ ارْبَعِينَ سَنَةً لَمَّا اتَّخَذَ لِنَفْسِهِ زَوْجَةً رِفْقَةَ بِنْتَ بَتُوئِيلَ الارَامِيِّ اخْتَ لابَانَ الارَامِيِّ مِنْ فَدَّانَِ ارَامَ.
\par 21 وَصَلَّى اسْحَاقُ الَى الرَّبِّ لاجْلِ امْرَاتِهِ لانَّهَا كَانَتْ عَاقِرا فَاسْتَجَابَ لَهُ الرَّبُّ فَحَبِلَتْ رِفْقَةُ امْرَاتُهُ.
\par 22 وَتَزَاحَمَ الْوَلَدَانِ فِي بَطْنِهَا فَقَالَتْ: «انْ كَانَ هَكَذَا فَلِمَاذَا انَا؟» فَمَضَتْ لِتَسْالَ الرَّبَّ.
\par 23 فَقَالَ لَهَا الرَّبُّ: «فِي بَطْنِكِ امَّتَانِ وَمِنْ احْشَائِكِ يَفْتَرِقُ شَعْبَانِ: شَعْبٌ يَقْوَى عَلَى شَعْبٍ وَكَبِيرٌ يُسْتَعْبَدُ لِصَغِيرٍ»
\par 24 فَلَمَّا كَمُلَتْ ايَّامُهَا لِتَلِدَ اذَا فِي بَطْنِهَا تَوْامَانِ.
\par 25 فَخَرَجَ الاوَّلُ احْمَرَ كُلُّهُ كَفَرْوَةِ شَعْرٍ فَدَعُوا اسْمَهُ عِيسُوَ.
\par 26 وَبَعْدَ ذَلِكَ خَرَجَ اخُوهُ وَيَدُهُ قَابِضَةٌ بِعَقِبِ عِيسُو فَدُعِيَ اسْمُهُ يَعْقُوبَ. وَكَانَ اسْحَاقُ ابْنَ سِتِّينَ سَنَةً لَمَّا وَلَدَتْهُمَا.
\par 27 فَكَبِرَ الْغُلامَانِ. وَكَانَ عِيسُو انْسَانا يَعْرِفُ الصَّيْدَ انْسَانَ الْبَرِّيَّةِ. وَيَعْقُوبُ انْسَانا كَامِلا يَسْكُنُ الْخِيَامَ.
\par 28 فَاحَبَّ اسْحَاقُ عِيسُوَ لانَّ فِي فَمِهِ صَيْدا وَامَّا رِفْقَةُ فَكَانَتْ تُحِبُّ يَعْقُوبَ.
\par 29 وَطَبَخَ يَعْقُوبُ طَبِيخا فَاتَى عِيسُو مِنَ الْحَقْلِ وَهُوَ قَدْ اعْيَا.
\par 30 فَقَالَ عِيسُو لِيَعْقُوبَ: «اطْعِمْنِي مِنْ هَذَا الاحْمَرِ لانِّي قَدْ اعْيَيْتُ. (لِذَلِكَ دُعِيَ اسْمُهُ ادُومَ).
\par 31 فَقَالَ يَعْقُوبُ: «بِعْنِي الْيَوْمَ بَكُورِيَّتَكَ».
\par 32 فَقَالَ عِيسُو: «هَا انَا مَاضٍ الَى الْمَوْتِ فَلِمَاذَا لِي بَكُورِيَّةٌ؟»
\par 33 فَقَالَ يَعْقُوبُ: «احْلِفْ لِيَ الْيَوْمَ». فَحَلَفَ لَهُ. فَبَاعَ بَكُورِيَّتَهُ لِيَعْقُوبَ.
\par 34 فَاعْطَى يَعْقُوبُ عِيسُوَ خُبْزا وَطَبِيخَ عَدَسٍ فَاكَلَ وَشَرِبَ وَقَامَ وَمَضَى. فَاحْتَقَرَ عِيسُو الْبَكُورِيَّةَ.

\chapter{26}

\par 1 وَكَانَ فِي الارْضِ جُوعٌ غَيْرُ الْجُوعِ الاوَّلِ الَّذِي كَانَ فِي ايَّامِ ابْرَاهِيمَ فَذَهَبَ اسْحَاقُ الَى ابِيمَالِكَ مَلِكِ الْفَلَسْطِينِيِّينَ الَى جَرَارَ.
\par 2 وَظَهَرَ لَهُ الرَّبُّ وَقَالَ: «لا تَنْزِلْ الَى مِصْرَ. اسْكُنْ فِي الارْضِ الَّتِي اقُولُ لَكَ.
\par 3 تَغَرَّبْ فِي هَذِهِ الارْضِ فَاكُونَ مَعَكَ وَابَارِكَكَ لانِّي لَكَ وَلِنَسْلِكَ اعْطِي جَمِيعَ هَذِهِ الْبِلادِ وَافِي بِالْقَسَمِ الَّذِي اقْسَمْتُ لابْرَاهِيمَ ابِيكَ.
\par 4 وَاكَثِّرُ نَسْلَكَ كَنُجُومِ السَّمَاءِ وَاعْطِي نَسْلَكَ جَمِيعَ هَذِهِ الْبِلادِ وَتَتَبَارَكُ فِي نَسْلِكَ جَمِيعُ امَمِ الارْضِ
\par 5 مِنْ اجْلِ انَّ ابْرَاهِيمَ سَمِعَ لِقَوْلِي وَحَفِظَ مَا يُحْفَظُ لِي: اوَامِرِي وَفَرَائِضِي وَشَرَائِعِي».
\par 6 فَاقَامَ اسْحَاقُ فِي جَرَارَ.
\par 7 وَسَالَهُ اهْلُ الْمَكَانِ عَنِ امْرَاتِهِ فَقَالَ: «هِيَ اخْتِي». لانَّهُ خَافَ انْ يَقُولَ «امْرَاتِي» لَعَلَّ اهْلَ الْمَكَانِ «يَقْتُلُونَنِي مِنْ اجْلِ رِفْقَةَ» لانَّهَا كَانَتْ حَسَنَةَ الْمَنْظَرِ.
\par 8 وَحَدَثَ اذْ طَالَتْ لَهُ الايَّامُ هُنَاكَ انَّ ابِيمَالِكَ مَلِكَ الْفِلِسْطِينِيِّينَ اشْرَفَ مِنَ الْكُوَّةِ وَنَظَرَ وَاذَا اسْحَاقُ يُلاعِبُ رِفْقَةَ امْرَاتَهُ.
\par 9 فَدَعَا ابِيمَالِكُ اسْحَاقَ وَقَالَ: «انَّمَا هِيَ امْرَاتُكَ! فَكَيْفَ قُلْتَ: هِيَ اخْتِي؟» فَقَالَ لَهُ اسْحَاقُ: «لانِّي قُلْتُ: لَعَلِّي امُوتُ بِسَبَبِهَا».
\par 10 فَقَالَ ابِيمَالِكُ: «مَا هَذَا الَّذِي صَنَعْتَ بِنَا؟ لَوْلا قَلِيلٌ لاضْطَجَعَ احَدُ الشَّعْبِ مَعَ امْرَاتِكَ فَجَلَبْتَ عَلَيْنَا ذَنْبا».
\par 11 فَاوْصَى ابِيمَالِكُ جَمِيعَ الشَّعْبِ: «الَّذِي يَمَسُّ هَذَا الرَّجُلَ اوِ امْرَاتَهُ مَوْتا يَمُوتُ».
\par 12 وَزَرَعَ اسْحَاقُ فِي تِلْكَ الارْضِ فَاصَابَ فِي تِلْكَ السَّنَةِ مِئَةَ ضِعْفٍ وَبَارَكَهُ الرَّبُّ.
\par 13 فَتَعَاظَمَ الرَّجُلُ وَكَانَ يَتَزَايَدُ فِي التَّعَاظُمِ حَتَّى صَارَ عَظِيما جِدّا.
\par 14 فَكَانَ لَهُ مَوَاشٍ مِنَ الْغَنَمِ وَمَوَاشٍ مِنَ الْبَقَرِ وَعَبِيدٌ كَثِيرُونَ. فَحَسَدَهُ الْفَلَسْطِينِيُّونَ.
\par 15 وَجَمِيعُ الْابَارِ الَّتِي حَفَرَهَا عَبِيدُ ابِيهِ فِي ايَّامِ ابْرَاهِيمَ ابِيهِ طَمَّهَا الْفَلَسْطِينِيُّونَ وَمَلَاوهَا تُرَابا.
\par 16 وَقَالَ ابِيمَالِكُ لاسْحَاقَ: «اذْهَبْ مِنْ عِنْدِنَا لانَّكَ صِرْتَ اقْوَى مِنَّا جِدّا».
\par 17 فَمَضَى اسْحَاقُ مِنْ هُنَاكَ. وَنَزَلَ فِي وَادِي جَرَارَ وَاقَامَ هُنَاكَ.
\par 18 فَعَادَ اسْحَاقُ وَنَبَشَ ابَارَ الْمَاءِ الَّتِي حَفَرُوهَا فِي ايَّامِ ابْرَاهِيمَ ابِيهِ وَطَمَّهَا الْفَلَسْطِينِيُّونَ بَعْدَ مَوْتِ ابِيهِ وَدَعَاهَا بِاسْمَاءٍ كَالاسْمَاءِ الَّتِي دَعَاهَا بِهَا ابُوهُ.
\par 19 وَحَفَرَ عَبِيدُ اسْحَاقَ فِي الْوَادِي فَوَجَدُوا هُنَاكَ بِئْرَ مَاءٍ حَيٍّ.
\par 20 فَخَاصَمَ رُعَاةُ جَرَارَ رُعَاةَ اسْحَاقَ قَائِلِينَ: «لَنَا الْمَاءُ». فَدَعَا اسْمَ الْبِئْرِ «عِسِقَ» لانَّهُمْ نَازَعُوهُ.
\par 21 ثُمَّ حَفَرُوا بِئْرا اخْرَى وَتَخَاصَمُوا عَلَيْهَا ايْضا فَدَعَا اسْمَهَا «سِطْنَةَ».
\par 22 ثُمَّ نَقَلَ مِنْ هُنَاكَ وَحَفَرَ بِئْرا اخْرَى وَلَمْ يَتَخَاصَمُوا عَلَيْهَا فَدَعَا اسْمَهَا «رَحُوبُوتَ» وَقَالَ: «انَّهُ الْانَ قَدْ ارْحَبَ لَنَا الرَّبُّ وَاثْمَرْنَا فِي الارْضِ».
\par 23 ثُمَّ صَعِدَ مِنْ هُنَاكَ الَى بِئْرِ سَبْعٍ.
\par 24 فَظَهَرَ لَهُ الرَّبُّ فِي تِلْكَ اللَّيْلَةِ وَقَالَ: «انَا الَهُ ابْرَاهِيمَ ابِيكَ. لا تَخَفْ لانِّي مَعَكَ وَابَارِكُكَ وَاكَثِّرُ نَسْلَكَ مِنْ اجْلِ ابْرَاهِيمَ عَبْدِي».
\par 25 فَبَنَى هُنَاكَ مَذْبَحا وَدَعَا بِاسْمِ الرَّبِّ. وَنَصَبَ هُنَاكَ خَيْمَتَهُ. وَحَفَرَ هُنَاكَ عَبِيدُ اسْحَاقَ بِئْرا.
\par 26 وَذَهَبَ الَيْهِ مِنْ جَرَارَ ابِيمَالِكُ وَاحُزَّاتُ مِنْ اصْحَابِهِ وَفِيكُولُ رَئِيسُ جَيْشِهِ.
\par 27 فَقَالَ لَهُمْ اسْحَاقُ: «مَا بَالُكُمْ اتَيْتُمْ الَيَّ وَانْتُمْ قَدْ ابْغَضْتُمُونِي وَصَرَفْتُمُونِي مِنْ عِنْدِكُمْ؟»
\par 28 فَقَالُوا: «انَّنَا قَدْ رَايْنَا انَّ الرَّبَّ كَانَ مَعَكَ فَقُلْنَا: لِيَكُنْ بَيْنَنَا حَلْفٌ بَيْنَنَا وَبَيْنَكَ وَنَقْطَعُ مَعَكَ عَهْدا:
\par 29 انْ لا تَصْنَعَ بِنَا شَرّا كَمَا لَمْ نَمَسَّكَ وَكَمَا لَمْ نَصْنَعْ بِكَ الَّا خَيْرا وَصَرَفْنَاكَ بِسَلامٍ. انْتَ الْانَ مُبَارَكُ الرَّبِّ!»
\par 30 فَصَنَعَ لَهُمْ ضِيَافَةً. فَاكَلُوا وَشَرِبُوا
\par 31 ثُمَّ بَكَّرُوا فِي الْغَدِ وَحَلَفُوا بَعْضُهُمْ لِبَعْضٍ وَصَرَفَهُمْ اسْحَاقُ. فَمَضُوا مِنْ عِنْدِهِ بِسَلامٍ.
\par 32 وَحَدَثَ فِي ذَلِكَ الْيَوْمِ انَّ عَبِيدَ اسْحَاقَ جَاءُوا وَاخْبَرُوهُ عَنِ الْبِئْرِ الَّتِي حَفَرُوا وَقَالُوا لَهُ: «قَدْ وَجَدْنَا مَاءً».
\par 33 فَدَعَاهَا «شِبْعَةَ». لِذَلِكَ اسْمُ الْمَدِينَةِ بِئْرُ سَبْعٍ الَى هَذَا الْيَوْمِ.
\par 34 وَلَمَّا كَانَ عِيسُو ابْنَ ارْبَعِينَ سَنَةً اتَّخَذَ زَوْجَةً: يَهُودِيتَ ابْنَةَ بِيرِي الْحِثِّيِّ وَبَسْمَةَ ابْنَةَ ايلُونَ الْحِثِّيِّ.
\par 35 فَكَانَتَا مَرَارَةَ نَفْسٍ لاسْحَاقَ وَرِفْقَةَ.

\chapter{27}

\par 1 وَحَدَثَ لَمَّا شَاخَ اسْحَاقُ وَكَلَّتْ عَيْنَاهُ عَنِ النَّظَرِ انَّهُ دَعَا عِيسُوَ ابْنَهُ الاكْبَرَ وَقَالَ لَهُ: «يَا ابْنِي». فَقَالَ لَهُ: «هَئَنَذَا».
\par 2 فَقَالَ: «انَّنِي قَدْ شِخْتُ وَلَسْتُ اعْرِفُ يَوْمَ وَفَاتِي.
\par 3 فَالْانَ خُذْ عُدَّتَكَ: جُعْبَتَكَ وَقَوْسَكَ وَاخْرُجْ الَى الْبَرِّيَّةِ وَتَصَيَّدْ لِي صَيْدا
\par 4 وَاصْنَعْ لِي اطْعِمَةً كَمَا احِبُّ وَاتِنِي بِهَا لِاكُلَ حَتَّى تُبَارِكَكَ نَفْسِي قَبْلَ انْ امُوتَ».
\par 5 وَكَانَتْ رِفْقَةُ سَامِعَةً اذْ تَكَلَّمَ اسْحَاقُ مَعَ عِيسُو ابْنِهِ. فَذَهَبَ عِيسُو الَى الْبَرِّيَّةِ لِيَصْطَادَ صَيْدا لِيَاتِيَ بِهِ.
\par 6 وَامَّا رِفْقَةُ فَقَالَتْ لِيَعْقُوبَ ابْنِهَا: «انِّي قَدْ سَمِعْتُ ابَاكَ يُكَلِّمُ عِيسُوَ اخَاكَ قَائِلا:
\par 7 ائْتِنِي بِصَيْدٍ وَاصْنَعْ لِي اطْعِمَةً لِاكُلَ وَابَارِكَكَ امَامَ الرَّبِّ قَبْلَ وَفَاتِي.
\par 8 فَالْانَ يَا ابْنِي اسْمَعْ لِقَوْلِي فِي مَا انَا امُرُكَ بِهِ:
\par 9 اذْهَبْ الَى الْغَنَمِ وَخُذْ لِي مِنْ هُنَاكَ جَدْيَيْنِ جَيِّدَيْنِ مِنَ الْمِعْزَى فَاصْنَعَهُمَا اطْعِمَةً لابِيكَ كَمَا يُحِبُّ
\par 10 فَتُحْضِرَهَا الَى ابِيكَ لِيَاكُلَ حَتَّى يُبَارِكَكَ قَبْلَ وَفَاتِهِ».
\par 11 فَقَالَ يَعْقُوبُ لِرِفْقَةَ امِّهِ: «هُوَذَا عِيسُو اخِي رَجُلٌ اشْعَرُ وَانَا رَجُلٌ امْلَسُ.
\par 12 رُبَّمَا يَجُسُّنِي ابِي فَاكُونُ فِي عَيْنَيْهِ كَمُتَهَاوِنٍ وَاجْلِبُ عَلَى نَفْسِي لَعْنَةً لا بَرَكَةً».
\par 13 فَقَالَتْ لَهُ امُّهُ: «لَعْنَتُكَ عَلَيَّ يَا ابْنِي. اسْمَعْ لِقَوْلِي فَقَطْ وَاذْهَبْ خُذْ لِي».
\par 14 فَذَهَبَ وَاخَذَ وَاحْضَرَ لِامِّهِ فَصَنَعَتْ امُّهُ اطْعِمَةً كَمَا كَانَ ابُوهُ يُحِبُّ.
\par 15 وَاخَذَتْ رِفْقَةُ ثِيَابَ عِيسُو ابْنِهَا الاكْبَرِ الْفَاخِرَةَ الَّتِي كَانَتْ عِنْدَهَا فِي الْبَيْتِ وَالْبَسَتْ يَعْقُوبَ ابْنَهَا الاصْغَرَ
\par 16 وَالْبَسَتْ يَدَيْهِ وَمَلاسَةَ عُنُقِهِ جُلُودَ جَدْيَيِ الْمِعْزَى.
\par 17 وَاعْطَتِ الاطْعِمَةَ وَالْخُبْزَ الَّتِي صَنَعَتْ فِي يَدِ يَعْقُوبَ ابْنِهَا.
\par 18 فَدَخَلَ الَى ابِيهِ وَقَالَ: «يَا ابِي». فَقَالَ: «هَئَنَذَا. مَنْ انْتَ يَا ابْنِي؟»
\par 19 فَقَالَ يَعْقُوبُ لابِيهِ: «انَا عِيسُو بِكْرُكَ. قَدْ فَعَلْتُ كَمَا كَلَّمْتَنِي. قُمِ اجْلِسْ وَكُلْ مِنْ صَيْدِي لِتُبَارِكَنِي نَفْسُكَ».
\par 20 فَقَالَ اسْحَاقُ لِابْنِهِ: «مَا هَذَا الَّذِي اسْرَعْتَ لِتَجِدَ يَا ابْنِي؟» فَقَالَ: «انَّ الرَّبَّ الَهَكَ قَدْ يَسَّرَ لِي».
\par 21 فَقَالَ اسْحَاقُ لِيَعْقُوبَ: «تَقَدَّمْ لاجُسَّكَ يَا ابْنِي. اانْتَ هُوَ ابْنِي عِيسُو امْ لا؟»
\par 22 فَتَقَدَّمَ يَعْقُوبُ الَى اسْحَاقَ ابِيهِ فَجَسَّهُ وَقَالَ: «الصَّوْتُ صَوْتُ يَعْقُوبَ وَلَكِنَّ الْيَدَيْنِ يَدَا عِيسُو».
\par 23 وَلَمْ يَعْرِفْهُ لانَّ يَدَيْهِ كَانَتَا مُشْعِرَتَيْنِ كَيَدَيْ عِيسُو اخِيهِ. فَبَارَكَهُ.
\par 24 وَقَالَ: «هَلْ انْتَ هُوَ ابْنِي عِيسُو؟» فَقَالَ: «انَا هُوَ».
\par 25 فَقَالَ: «قَدِّمْ لِي لِاكُلَ مِنْ صَيْدِ ابْنِي حَتَّى تُبَارِكَكَ نَفْسِي». فَقَدَّمَ لَهُ فَاكَلَ وَاحْضَرَ لَهُ خَمْرا فَشَرِبَ.
\par 26 فَقَالَ لَهُ اسْحَاقُ ابُوهُ: «تَقَدَّمْ وَقَبِّلْنِي يَا ابْنِي».
\par 27 فَتَقَدَّمَ وَقَبَّلَهُ. فَشَمَّ رَائِحَةَ ثِيَابِهِ وَبَارَكَهُ. وَقَالَ: «انْظُرْ! رَائِحَةُ ابْنِي كَرَائِحَةِ حَقْلٍ قَدْ بَارَكَهُ الرَّبُّ.
\par 28 فَلْيُعْطِكَ اللهُ مِنْ نَدَى السَّمَاءِ وَمِنْ دَسَمِ الارْضِ وَكَثْرَةَ حِنْطَةٍ وَخَمْرٍ.
\par 29 لِيُسْتَعْبَدْ لَكَ شُعُوبٌ وَتَسْجُدْ لَكَ قَبَائِلُ. كُنْ سَيِّدا لاخْوَتِكَ وَلْيَسْجُدْ لَكَ بَنُو امِّكَ. لِيَكُنْ لاعِنُوكَ مَلْعُونِينَ وَمُبَارِكُوكَ مُبَارَكِينَ».
\par 30 وَحَدَثَ عِنْدَمَا فَرَغَ اسْحَاقُ مِنْ بَرَكَةِ يَعْقُوبَ وَيَعْقُوبُ قَدْ خَرَجَ مِنْ لَدُنْ اسْحَاقَ ابِيهِ انَّ عِيسُوَ اخَاهُ اتَى مِنْ صَيْدِهِ
\par 31 فَصَنَعَ هُوَ ايْضا اطْعِمَةً وَدَخَلَ بِهَا الَى ابِيهِ وَقَالَ لابِيهِ: «لِيَقُمْ ابِي وَيَاكُلْ مِنْ صَيْدِ ابْنِهِ حَتَّى تُبَارِكَنِي نَفْسُكَ».
\par 32 فَقَالَ لَهُ اسْحَاقُ ابُوهُ: «مَنْ انْتَ؟» فَقَالَ: «انَا ابْنُكَ بِكْرُكَ عِيسُو».
\par 33 فَارْتَعَدَ اسْحَاقُ ارْتِعَادا عَظِيما جِدّا. وَقَالَ: «فَمَنْ هُوَ الَّذِي اصْطَادَ صَيْدا وَاتَى بِهِ الَيَّ فَاكَلْتُ مِنَ الْكُلِّ قَبْلَ انْ تَجِيءَ وَبَارَكْتُهُ؟ نَعَمْ وَيَكُونُ مُبَارَكا!»
\par 34 فَعِنْدَمَا سَمِعَ عِيسُو كَلامَ ابِيهِ صَرَخَ صَرْخَةً عَظِيمَةً وَمُرَّةً جِدّا وَقَالَ لابِيهِ: «بَارِكْنِي انَا ايْضا يَا ابِي!»
\par 35 فَقَالَ: «قَدْ جَاءَ اخُوكَ بِمَكْرٍ وَاخَذَ بَرَكَتَكَ».
\par 36 فَقَالَ: «الا انَّ اسْمَهُ دُعِيَ يَعْقُوبَ فَقَدْ تَعَقَّبَنِي الْانَ مَرَّتَيْنِ! اخَذَ بَكُورِيَّتِي وَهُوَذَا الْانَ قَدْ اخَذَ بَرَكَتِي». ثُمَّ قَالَ: «امَا ابْقَيْتَ لِي بَرَكَةً؟»
\par 37 فَقَالَ اسْحَاقُ لِعِيسُو: «انِّي قَدْ جَعَلْتُهُ سَيِّدا لَكَ وَدَفَعْتُ الَيْهِ جَمِيعَ اخْوَتِهِ عَبِيدا وَعَضَدْتُهُ بِحِنْطَةٍ وَخَمْرٍ. فَمَاذَا اصْنَعُ الَيْكَ يَا ابْنِي؟»
\par 38 فَقَالَ عِيسُو لابِيهِ: «الَكَ بَرَكَةٌ وَاحِدَةٌ فَقَطْ يَا ابِي؟ بَارِكْنِي انَا ايْضا يَا ابِي!» وَرَفَعَ عِيسُو صَوْتَهُ وَبَكَى.
\par 39 فَاجَابَ اسْحَاقُ ابُوهُ: «هُوَذَا بِلا دَسَمِ الارْضِ يَكُونُ مَسْكَنُكَ وَبِلا نَدَى السَّمَاءِ مِنْ فَوْقُ.
\par 40 وَبِسَيْفِكَ تَعِيشُ وَلاخِيكَ تُسْتَعْبَدُ. وَلَكِنْ يَكُونُ حِينَمَا تَجْمَحُ انَّكَ تُكَسِّرُ نِيرَهُ عَنْ عُنُقِكَ».
\par 41 فَحَقَدَ عِيسُو عَلَى يَعْقُوبَ مِنْ اجْلِ الْبَرَكَةِ الَّتِي بَارَكَهُ بِهَا ابُوهُ. وَقَالَ عِيسُو فِي قَلْبِهِ: «قَرُبَتْ ايَّامُ مَنَاحَةِ ابِي فَاقْتُلُ يَعْقُوبَ اخِي».
\par 42 فَاخْبِرَتْ رِفْقَةُ بِكَلامِ عِيسُوَ ابْنِهَا الاكْبَرِ فَارْسَلَتْ وَدَعَتْ يَعْقُوبَ ابْنَهَا الاصْغَرَ وَقَالَتْ لَهُ: «هُوَذَا عِيسُو اخُوكَ مُتَسَلٍّ مِنْ جِهَتِكَ بِانَّهُ يَقْتُلُكَ.
\par 43 فَالْانَ يَا ابْنِي اسْمَعْ لِقَوْلِي وَقُمِ اهْرُبْ الَى اخِي لابَانَ الَى حَارَانَ
\par 44 وَاقِمْ عِنْدَهُ ايَّاما قَلِيلَةً حَتَّى يَرْتَدَّ غَضَبُ اخِيكَ عَنْكَ
\par 45 وَيَنْسَى مَا صَنَعْتَ بِهِ. ثُمَّ ارْسِلُ فَاخُذُكَ مِنْ هُنَاكَ. لِمَاذَا اعْدَمُ اثْنَيْكُمَا فِي يَوْمٍ وَاحِدٍ؟».
\par 46 وَقَالَتْ رِفْقَةُ لاسْحَاقَ: «مَلِلْتُ حَيَاتِي مِنْ اجْلِ بَنَاتِ حِثَّ. انْ كَانَ يَعْقُوبُ يَاخُذُ زَوْجَةً مِنْ بَنَاتِ حِثَّ مِثْلَ هَؤُلاءِ مِنْ بَنَاتِ الارْضِ فَلِمَاذَا لِي حَيَاةٌ؟»

\chapter{28}

\par 1 فَدَعَا اسْحَاقُ يَعْقُوبَ وَبَارَكَهُ وَاوْصَاهُ وَقَالَ لَهُ: «لا تَاخُذْ زَوْجَةً مِنْ بَنَاتِ كَنْعَانَ.
\par 2 قُمِ اذْهَبْ الَى فَدَّانَِ ارَامَ الَى بَيْتِ بَتُوئِيلَ ابِي امِّكَ وَخُذْ لِنَفْسِكَ زَوْجَةً مِنْ هُنَاكَ مِنْ بَنَاتِ لابَانَ اخِي امِّكَ.
\par 3 وَاللهُ الْقَدِيرُ يُبَارِكُكَ وَيَجْعَلُكَ مُثْمِرا وَيُكَثِّرُكَ فَتَكُونُ جُمْهُورا مِنَ الشُّعُوبِ.
\par 4 وَيُعْطِيكَ بَرَكَةَ ابْرَاهِيمَ لَكَ وَلِنَسْلِكَ مَعَكَ لِتَرِثَ ارْضَ غُرْبَتِكَ الَّتِي اعْطَاهَا اللهُ لابْرَاهِيمَ».
\par 5 فَصَرَفَ اسْحَاقُ يَعْقُوبَ فَذَهَبَ الَى فَدَّانَِ ارَامَ الَى لابَانَ بْنِ بَتُوئِيلَ الارَامِيِّ اخِي رِفْقَةَ امِّ يَعْقُوبَ وَعِيسُوَ.
\par 6 فَلَمَّا رَاى عِيسُو انَّ اسْحَاقَ بَارَكَ يَعْقُوبَ وَارْسَلَهُ الَى فَدَّانَِ ارَامَ لِيَاخُذَ لِنَفْسِهِ مِنْ هُنَاكَ زَوْجَةً اذْ بَارَكَهُ وَاوْصَاهُ قَائِلا: «لا تَاخُذْ زَوْجَةً مِنْ بَنَاتِ كَنْعَانَ».
\par 7 وَانَّ يَعْقُوبَ سَمِعَ لابِيهِ وَامِّهِ وَذَهَبَ الَى فَدَّانَِ ارَامَ
\par 8 رَاى عِيسُو انَّ بَنَاتِ كَنْعَانَ شِرِّيرَاتٌ فِي عَيْنَيْ اسْحَاقَ ابِيهِ
\par 9 فَذَهَبَ عِيسُو الَى اسْمَاعِيلَ وَاخَذَ مَحْلَةَ بِنْتَ اسْمَاعِيلَ بْنِ ابْرَاهِيمَ اخْتَ نَبَايُوتَ زَوْجَةً لَهُ عَلَى نِسَائِهِ.
\par 10 فَخَرَجَ يَعْقُوبُ مِنْ بِئْرِ سَبْعٍ وَذَهَبَ نَحْوَ حَارَانَ.
\par 11 وَصَادَفَ مَكَانا وَبَاتَ هُنَاكَ لانَّ الشَّمْسَ كَانَتْ قَدْ غَابَتْ. وَاخَذَ مِنْ حِجَارَةِ الْمَكَانِ وَوَضَعَهُ تَحْتَ رَاسِهِ فَاضْطَجَعَ فِي ذَلِكَ الْمَكَانِ.
\par 12 وَرَاى حُلْما وَاذَا سُلَّمٌ مَنْصُوبَةٌ عَلَى الارْضِ وَرَاسُهَا يَمَسُّ السَّمَاءَ وَهُوَذَا مَلائِكَةُ اللهِ صَاعِدَةٌ وَنَازِلَةٌ عَلَيْهَا
\par 13 وَهُوَذَا الرَّبُّ وَاقِفٌ عَلَيْهَا فَقَالَ: «انَا الرَّبُّ الَهُ ابْرَاهِيمَ ابِيكَ وَالَهُ اسْحَاقَ. الارْضُ الَّتِي انْتَ مُضْطَجِعٌ عَلَيْهَا اعْطِيهَا لَكَ وَلِنَسْلِكَ.
\par 14 وَيَكُونُ نَسْلُكَ كَتُرَابِ الارْضِ وَتَمْتَدُّ غَرْبا وَشَرْقا وَشِمَالا وَجَنُوبا. وَيَتَبَارَكُ فِيكَ وَفِي نَسْلِكَ جَمِيعُ قَبَائِلِ الارْضِ.
\par 15 وَهَا انَا مَعَكَ وَاحْفَظُكَ حَيْثُمَا تَذْهَبُ وَارُدُّكَ الَى هَذِهِ الارْضِ لانِّي لا اتْرُكُكَ حَتَّى افْعَلَ مَا كَلَّمْتُكَ بِهِ».
\par 16 فَاسْتَيْقَظَ يَعْقُوبُ مِنْ نَوْمِهِ وَقَالَ: «حَقّا انَّ الرَّبَّ فِي هَذَا الْمَكَانِ وَانَا لَمْ اعْلَمْ!»
\par 17 وَخَافَ وَقَالَ: «مَا ارْهَبَ هَذَا الْمَكَانَ! مَا هَذَا الَّا بَيْتُ اللهِ وَهَذَا بَابُ السَّمَاءِ!»
\par 18 وَبَكَّرَ يَعْقُوبُ فِي الصَّبَاحِ وَاخَذَ الْحَجَرَ الَّذِي وَضَعَهُ تَحْتَ رَاسِهِ وَاقَامَهُ عَمُودا وَصَبَّ زَيْتا عَلَى رَاسِهِ
\par 19 وَدَعَا اسْمَ ذَلِكَ الْمَكَانِ «بَيْتَ ايلَ». وَلَكِنِ اسْمُ الْمَدِينَةِ اوَّلا كَانَ لُوزَ.
\par 20 وَنَذَرَ يَعْقُوبُ نَذْرا قَائِلا: «انْ كَانَ اللهُ مَعِي وَحَفِظَنِي فِي هَذَا الطَّرِيقِ الَّذِي انَا سَائِرٌ فِيهِ وَاعْطَانِي خُبْزا لِاكُلَ وَثِيَابا لالْبِسَ
\par 21 وَرَجَعْتُ بِسَلامٍ الَى بَيْتِ ابِي يَكُونُ الرَّبُّ لِي الَها
\par 22 وَهَذَا الْحَجَرُ الَّذِي اقَمْتُهُ عَمُودا يَكُونُ بَيْتَ اللهِ وَكُلُّ مَا تُعْطِينِي فَانِّي اعَشِّرُهُ لَكَ».

\chapter{29}

\par 1 ثُمَّ قَامَ يَعْقُوبُ وَذَهَبَ الَى ارْضِ بَنِي الْمَشْرِقِ.
\par 2 وَنَظَرَ وَاذَا فِي الْحَقْلِ بِئْرٌ وَهُنَاكَ ثَلاثَةُ قُطْعَانِ غَنَمٍ رَابِضَةٌ عِنْدَهَا لانَّهُمْ كَانُوا مِنْ تِلْكَ الْبِئْرِ يَسْقُونَ الْقُطْعَانَ وَالْحَجَرُ عَلَى فَمِ الْبِئْرِ كَانَ كَبِيرا.
\par 3 فَكَانَ يَجْتَمِعُ الَى هُنَاكَ جَمِيعُ الْقُطْعَانِ فَيُدَحْرِجُونَ الْحَجَرَ عَنْ فَمِ الْبِئْرِ وَيَسْقُونَ الْغَنَمَ ثُمَّ يَرُدُّونَ الْحَجَرَ عَلَى فَمِ الْبِئْرِ الَى مَكَانِهِ.
\par 4 فَقَالَ لَهُمْ يَعْقُوبُ: «يَا اخْوَتِي مِنْ ايْنَ انْتُمْ؟» فَقَالُوا: «نَحْنُ مِنْ حَارَانَ».
\par 5 فَقَالَ لَهُمْ: «هَلْ تَعْرِفُونَ لابَانَ ابْنَ نَاحُورَ؟» فَقَالُوا: «نَعْرِفُهُ».
\par 6 فَقَالَ لَهُمْ: «هَلْ لَهُ سَلامَةٌ؟» فَقَالُوا: «لَهُ سَلامَةٌ. وَهُوَذَا رَاحِيلُ ابْنَتُهُ اتِيَةٌ مَعَ الْغَنَمِ».
\par 7 فَقَالَ: «هُوَذَا النَّهَارُ بَعْدُ طَوِيلٌ. لَيْسَ وَقْتَ اجْتِمَاعِ الْمَوَاشِي. اسْقُوا الْغَنَمَ وَاذْهَبُوا ارْعُوا».
\par 8 فَقَالُوا: «لا نَقْدِرُ حَتَّى تَجْتَمِعَ جَمِيعُ الْقُطْعَانِ وَيُدَحْرِجُوا الْحَجَرَ عَنْ فَمِ الْبِئْرِ ثُمَّ نَسْقِي الْغَنَمَ».
\par 9 وَاذْ هُوَ بَعْدُ يَتَكَلَّمُ مَعَهُمْ اتَتْ رَاحِيلُ مَعَ غَنَمِ ابِيهَا لانَّهَا كَانَتْ تَرْعَى.
\par 10 فَكَانَ لَمَّا ابْصَرَ يَعْقُوبُ رَاحِيلَ بِنْتَ لابَانَ خَالِهِ وَغَنَمَ لابَانَ خَالِهِ انَّ يَعْقُوبَ تَقَدَّمَ وَدَحْرَجَ الْحَجَرَ عَنْ فَمِ الْبِئْرِ وَسَقَى غَنَمَ لابَانَ خَالِهِ.
\par 11 وَقَبَّلَ يَعْقُوبُ رَاحِيلَ وَرَفَعَ صَوْتَهُ وَبَكَى.
\par 12 وَاخْبَرَ يَعْقُوبُ رَاحِيلَ انَّهُ اخُو ابِيهَا وَانَّهُ ابْنُ رِفْقَةَ. فَرَكَضَتْ وَاخْبَرَتْ ابَاهَا.
\par 13 فَكَانَ حِينَ سَمِعَ لابَانُ خَبَرَ يَعْقُوبَ ابْنِ اخْتِهِ انَّهُ رَكَضَ لِلِقَائِهِ وَعَانَقَهُ وَقَبَّلَهُ وَاتَى بِهِ الَى بَيْتِهِ. فَحَدَّثَ لابَانَ بِجَمِيعِ هَذِهِ الامُورِ.
\par 14 فَقَالَ لَهُ لابَانُ: «انَّمَا انْتَ عَظْمِي وَلَحْمِي». فَاقَامَ عِنْدَهُ شَهْرا مِنَ الزَّمَانِ.
\par 15 ثُمَّ قَالَ لابَانُ لِيَعْقُوبَ: «الانَّكَ اخِي تَخْدِمُنِي مَجَّانا؟ اخْبِرْنِي مَا اجْرَتُكَ».
\par 16 وَكَانَ لِلابَانَ ابْنَتَانِ اسْمُ الْكُبْرَى لَيْئَةُ وَاسْمُ الصُّغْرَى رَاحِيلُ.
\par 17 وَكَانَتْ عَيْنَا لَيْئَةَ ضَعِيفَتَيْنِ وَامَّا رَاحِيلُ فَكَانَتْ حَسَنَةَ الصُّورَةِ وَحَسَنَةَ الْمَنْظَرِ.
\par 18 وَاحَبَّ يَعْقُوبُ رَاحِيلَ فَقَالَ: «اخْدِمُكَ سَبْعَ سِنِينٍ بِرَاحِيلَ ابْنَتِكَ الصُّغْرَى».
\par 19 فَقَالَ لابَانُ: «انْ اعْطِيَكَ ايَّاهَا احْسَنُ مِنْ انْ اعْطِيَهَا لِرَجُلٍ اخَرَ. اقِمْ عِنْدِي».
\par 20 فَخَدَمَ يَعْقُوبُ بِرَاحِيلَ سَبْعَ سِنِينٍ وَكَانَتْ فِي عَيْنَيْهِ كَايَّامٍ قَلِيلَةٍ بِسَبَبِ مَحَبَّتِهِ لَهَا.
\par 21 ثُمَّ قَالَ يَعْقُوبُ لِلابَانَ: «اعْطِنِي امْرَاتِي لانَّ ايَّامِي قَدْ كَمُلَتْ فَادْخُلَ عَلَيْهَا».
\par 22 فَجَمَعَ لابَانُ جَمِيعَ اهْلِ الْمَكَانِ وَصَنَعَ وَلِيمَةً.
\par 23 وَكَانَ فِي الْمَسَاءِ انَّهُ اخَذَ لَيْئَةَ ابْنَتَهُ وَاتَى بِهَا الَيْهِ فَدَخَلَ عَلَيْهَا.
\par 24 وَاعْطَى لابَانُ زِلْفَةَ جَارِيَتَهُ لِلَيْئَةَ ابْنَتِهِ جَارِيَةً.
\par 25 وَفِي الصَّبَاحِ اذَا هِيَ لَيْئَةُ. فَقَالَ لِلابَانَ: «مَا هَذَا الَّذِي صَنَعْتَ بِي! الَيْسَ بِرَاحِيلَ خَدَمْتُ عِنْدَكَ؟ فَلِمَاذَا خَدَعْتَنِي؟»
\par 26 فَقَالَ لابَانُ: «لا يُفْعَلُ هَكَذَا فِي مَكَانِنَا انْ تُعْطَى الصَّغِيرَةُ قَبْلَ الْبِكْرِ.
\par 27 اكْمِلْ اسْبُوعَ هَذِهِ فَنُعْطِيَكَ تِلْكَ ايْضا بِالْخِدْمَةِ الَّتِي تَخْدِمُنِي ايْضا سَبْعَ سِنِينٍ اخَرَ».
\par 28 فَفَعَلَ يَعْقُوبُ هَكَذَا. فَاكْمَلَ اسْبُوعَ هَذِهِ فَاعْطَاهُ رَاحِيلَ ابْنَتَهُ زَوْجَةً لَهُ.
\par 29 وَاعْطَى لابَانُ رَاحِيلَ ابْنَتَهُ بَلْهَةَ جَارِيَتَهُ جَارِيَةً لَهَا.
\par 30 فَدَخَلَ عَلَى رَاحِيلَ ايْضا. وَاحَبَّ ايْضا رَاحِيلَ اكْثَرَ مِنْ لَيْئَةَ. وَعَادَ فَخَدَمَ عِنْدَهُ سَبْعَ سِنِينٍ اخَرَ.
\par 31 وَرَاى الرَّبُّ انَّ لَيْئَةَ مَكْرُوهَةٌ فَفَتَحَ رَحِمَهَا. وَامَّا رَاحِيلُ فَكَانَتْ عَاقِرا.
\par 32 فَحَبِلَتْ لَيْئَةُ وَوَلَدَتِ ابْنا وَدَعَتِ اسْمَهُ رَاوبَيْنَ لانَّهَا قَالَتْ: «انَّ الرَّبَّ قَدْ نَظَرَ الَى مَذَلَّتِي. انَّهُ الْانَ يُحِبُّنِي رَجُلِي».
\par 33 وَحَبِلَتْ ايْضا وَوَلَدَتِ ابْنا وَقَالَتْ: «انَّ الرَّبَّ قَدْ سَمِعَ انِّي مَكْرُوهَةٌ فَاعْطَانِي هَذَا ايْضا». فَدَعَتِ اسْمَهُ «شَمْعُونَ».
\par 34 وَحَبِلَتْ ايْضا وَوَلَدَتِ ابْنا وَقَالَتِ: «الْانَ هَذِهِ الْمَرَّةَ يَقْتَرِنُ بِي رَجُلِي لانِّي وَلَدْتُ لَهُ ثَلاثَةَ بَنِينَ». لِذَلِكَ دُعِيَ اسْمُهُ «لاوِيَ».
\par 35 وَحَبِلَتْ ايْضا وَوَلَدَتِ ابْنا وَقَالَتْ: «هَذِهِ الْمَرَّةَ احْمَدُ الرَّبَّ». لِذَلِكَ دَعَتِ اسْمَهُ «يَهُوذَا». ثُمَّ تَوَقَّفَتْ عَنِ الْوِلادَةِ.

\chapter{30}

\par 1 فَلَمَّا رَاتْ رَاحِيلُ انَّهَا لَمْ تَلِدْ لِيَعْقُوبَ غَارَتْ رَاحِيلُ مِنْ اخْتِهَا وَقَالَتْ لِيَعْقُوبَ: «هَبْ لِي بَنِينَ وَالَّا فَانَا امُوتُ».
\par 2 فَحَمِيَ غَضَبُ يَعْقُوبَ عَلَى رَاحِيلَ وَقَالَ: «الَعَلِّي مَكَانَ اللهِ الَّذِي مَنَعَ عَنْكِ ثَمْرَةَ الْبَطْنِ؟»
\par 3 فَقَالَتْ: «هُوَذَا جَارِيَتِي بَلْهَةُ. ادْخُلْ عَلَيْهَا فَتَلِدَ عَلَى رُكْبَتَيَّ وَارْزَقُ انَا ايْضا مِنْهَا بَنِينَ».
\par 4 فَاعْطَتْهُ بَلْهَةَ جَارِيَتَهَا زَوْجَةً فَدَخَلَ عَلَيْهَا يَعْقُوبُ
\par 5 فَحَبِلَتْ بَلْهَةُ وَوَلَدَتْ لِيَعْقُوبَ ابْنا
\par 6 فَقَالَتْ رَاحِيلُ: «قَدْ قَضَى لِيَ اللهُ وَسَمِعَ ايْضا لِصَوْتِي وَاعْطَانِيَ ابْنا». لِذَلِكَ دَعَتِ اسْمَهُ «دَانا».
\par 7 وَحَبِلَتْ ايْضا بَلْهَةُ جَارِيَةُ رَاحِيلَ وَوَلَدَتِ ابْنا ثَانِيا لِيَعْقُوبَ
\par 8 فَقَالَتْ رَاحِيلُ: «قَدْ صَارَعْتُ اخْتِي مُصَارَعَاتِ اللهِ وَغَلَبْتُ». فَدَعَتِ اسْمَهُ «نَفْتَالِي».
\par 9 وَلَمَّا رَاتْ لَيْئَةُ انَّهَا تَوَقَّفَتْ عَنِ الْوِلادَةِ اخَذَتْ زِلْفَةَ جَارِيَتَهَا وَاعْطَتْهَا لِيَعْقُوبَ زَوْجَةً
\par 10 فَوَلَدَتْ زِلْفَةُ جَارِيَةُ لَيْئَةَ لِيَعْقُوبَ ابْنا.
\par 11 فَقَالَتْ لَيْئَةُ: «بِسَعْدٍ». فَدَعَتِ اسْمَهُ «جَادا».
\par 12 وَوَلَدَتْ زِلْفَةُ جَارِيَةُ لَيْئَةَ ابْنا ثَانِيا لِيَعْقُوبَ
\par 13 فَقَالَتْ لَيْئَةُ: «بِغِبْطَتِي لانَّهُ تُغَبِّطُنِي بَنَاتٌ». فَدَعَتِ اسْمَهُ «اشِيرَ».
\par 14 وَمَضَى رَاوبَيْنُ فِي ايَّامِ حَصَادِ الْحِنْطَةِ فَوَجَدَ لُفَّاحا فِي الْحَقْلِ وَجَاءَ بِهِ الَى لَيْئَةَ امِّهِ. فَقَالَتْ رَاحِيلُ لِلَيْئَةَ: «اعْطِينِي مِنْ لُفَّاحِ ابْنِكِ».
\par 15 فَقَالَتْ لَهَا: «اقَلِيلٌ انَّكِ اخَذْتِ رَجُلِي فَتَاخُذِينَ لُفَّاحَ ابْنِي ايْضا؟» فَقَالَتْ رَاحِيلُ: «اذا يَضْطَجِعُ مَعَكِ اللَّيْلَةَ عِوَضا عَنْ لُفَّاحِ ابْنِكِ».
\par 16 فَلَمَّا اتَى يَعْقُوبُ مِنَ الْحَقْلِ فِي الْمَسَاءِ خَرَجَتْ لَيْئَةُ لِمُلاقَاتِهِ وَقَالَتْ: «الَيَّ تَجِيءُ لانِّي قَدِ اسْتَاجَرْتُكَ بِلُفَّاحِ ابْنِي». فَاضْطَجَعَ مَعَهَا تِلْكَ اللَّيْلَةَ.
\par 17 وَسَمِعَ اللهُ لِلَيْئَةَ فَحَبِلَتْ وَوَلَدَتْ لِيَعْقُوبَ ابْنا خَامِسا.
\par 18 فَقَالَتْ لَيْئَةُ: «قَدْ اعْطَانِي اللهُ اجْرَتِي لانِّي اعْطَيْتُ جَارِيَتِي لِرَجُلِي». فَدَعَتِ اسْمَهُ «يَسَّاكَرَ».
\par 19 وَحَبِلَتْ ايْضا لَيْئَةُ وَوَلَدَتِ ابْنا سَادِسا لِيَعْقُوبَ
\par 20 فَقَالَتْ لَيْئَةُ: «قَدْ وَهَبَنِي اللهُ هِبَةً حَسَنَةً. الْانَ يُسَاكِنُنِي رَجُلِي لانِّي وَلَدْتُ لَهُ سِتَّةَ بَنِينَ». فَدَعَتِ اسْمَهُ «زَبُولُونَ».
\par 21 ثُمَّ وَلَدَتِ ابْنَةً وَدَعَتِ اسْمَهَا «دِينَةَ».
\par 22 وَذَكَرَ اللهُ رَاحِيلَ وَسَمِعَ لَهَا اللهُ وَفَتَحَ رَحِمَهَا
\par 23 فَحَبِلَتْ وَوَلَدَتِ ابْنا فَقَالَتْ: «قَدْ نَزَعَ اللهُ عَارِي».
\par 24 وَدَعَتِ اسْمَهُ «يُوسُفَ» قَائِلَةً: «يَزِيدُنِي الرَّبُّ ابْنا اخَرَ».
\par 25 وَحَدَثَ لَمَّا وَلَدَتْ رَاحِيلُ يُوسُفَ انَّ يَعْقُوبَ قَالَ لِلابَانَ: «اصْرِفْنِي لاذْهَبَ الَى مَكَانِي وَالَى ارْضِي.
\par 26 اعْطِنِي نِسَائِي وَاوْلادِي الَّذِينَ خَدَمْتُكَ بِهِمْ فَاذْهَبَ لانَّكَ انْتَ تَعْلَمُ خِدْمَتِي الَّتِي خَدَمْتُكَ».
\par 27 فَقَالَ لَهُ لابَانُ: «لَيْتَنِي اجِدُ نِعْمَةً فِي عَيْنَيْكَ. قَدْ تَفَاءَلْتُ فَبَارَكَنِي الرَّبُّ بِسَبَبِكَ».
\par 28 وَقَالَ: «عَيِّنْ لِي اجْرَتَكَ فَاعْطِيَكَ».
\par 29 فَقَالَ لَهُ: «انْتَ تَعْلَمُ مَاذَا خَدَمْتُكَ وَمَاذَا صَارَتْ مَوَاشِيكَ مَعِي
\par 30 لانَّ مَا كَانَ لَكَ قَبْلِي قَلِيلٌ فَقَدِ اتَّسَعَ الَى كَثِيرٍ وَبَارَكَكَ الرَّبُّ فِي اثَرِي. وَالْانَ مَتَى اعْمَلُ انَا ايْضا لِبَيْتِي؟»
\par 31 فَقَالَ: «مَاذَا اعْطِيكَ؟» فَقَالَ يَعْقُوبُ: «لا تُعْطِينِي شَيْئا. انْ صَنَعْتَ لِي هَذَا الامْرَ اعُودُ ارْعَى غَنَمَكَ وَاحْفَظُهَا:
\par 32 اجْتَازُ بَيْنَ غَنَمِكَ كُلِّهَا الْيَوْمَ وَاعْزِلْ انْتَ مِنْهَا كُلَّ شَاةٍ رَقْطَاءَ وَبَلْقَاءَ وَكُلَّ شَاةٍ سَوْدَاءَ بَيْنَ الْخِرْفَانِ وَبَلْقَاءَ وَرَقْطَاءَ بَيْنَ الْمِعْزَى. فَيَكُونَ مِثْلُ ذَلِكَ اجْرَتِي
\par 33 وَيَشْهَدُ فِيَّ بِرِّي يَوْمَ غَدٍ اذَا جِئْتَ مِنْ اجْلِ اجْرَتِي قُدَّامَكَ. كُلُّ مَا لَيْسَ ارْقَطَ اوْ ابْلَقَ بَيْنَ الْمِعْزَى وَاسْوَدَ بَيْنَ الْخِرْفَانِ فَهُوَ مَسْرُوقٌ عِنْدِي».
\par 34 فَقَالَ لابَانُ: «هُوَذَا لِيَكُنْ بِحَسَبِ كَلامِكَ».
\par 35 فَعَزَلَ فِي ذَلِكَ الْيَوْمِ التُّيُوسَ الْمُخَطَّطَةَ وَالْبَلْقَاءَ وَكُلَّ الْعِنَازِ الرَّقْطَاءِ وَالْبَلْقَاءِ كُلَّ مَا فِيهِ بَيَاضٌ وَكُلَّ اسْوَدَ بَيْنَ الْخِرْفَانِ وَدَفَعَهَا الَى ايْدِي بَنِيهِ.
\par 36 وَجَعَلَ مَسِيرَةَ ثَلاثَةِ ايَّامٍ بَيْنَهُ وَبَيْنَ يَعْقُوبَ. وَكَانَ يَعْقُوبُ يَرْعَى غَنَمَ لابَانَ الْبَاقِيَةَ.
\par 37 فَاخَذَ يَعْقُوبُ لِنَفْسِهِ قُضْبَانا خُضْرا مِنْ لُبْنَى وَلَوْزٍ وَدُلْبٍ وَقَشَّرَ فِيهَا خُطُوطا بِيضا كَاشِطا عَنِ الْبَيَاضِ الَّذِي عَلَى الْقُضْبَانِ.
\par 38 وَاوْقَفَ الْقُضْبَانَ الَّتِي قَشَّرَهَا فِي الاجْرَانِ فِي مَسَاقِي الْمَاءِ حَيْثُ كَانَتِ الْغَنَمُ تَجِيءُ لِتَشْرَبَ تُجَاهَ الْغَنَمِ لِتَتَوَحَّمَ عِنْدَ مَجِيئِهَا لِتَشْرَبَ.
\par 39 فَتَوَحَّمَتِ الْغَنَمُ عِنْدَ الْقُضْبَانِ وَوَلَدَتِ الْغَنَمُ مُخَطَّطَاتٍ وَرُقْطا وَبُلْقا.
\par 40 وَافْرَزَ يَعْقُوبُ الْخِرْفَانَ وَجَعَلَ وُجُوهَ الْغَنَمِ الَى الْمُخَطَّطِ وَكُلِّ اسْوَدَ بَيْنَ غَنَمِ لابَانَ. وَجَعَلَ لَهُ قُطْعَانا وَحْدَهُ وَلَمْ يَجْعَلْهَا مَعَ غَنَمِ لابَانَ.
\par 41 وَحَدَثَ كُلَّمَا تَوَحَّمَتِ الْغَنَمُ الْقَوِيَّةُ انَّ يَعْقُوبَ وَضَعَ الْقُضْبَانَ امَامَ عُيُونِ الْغَنَمِ فِي الاجْرَانِ لِتَتَوَحَّمَ بَيْنَ الْقُضْبَانِ.
\par 42 وَحِينَ اسْتَضْعَفَتِ الْغَنَمُ لَمْ يَضَعْهَا. فَصَارَتِ الضَّعِيفَةُ لِلابَانَ وَالْقَوِيَّةُ لِيَعْقُوبَ.
\par 43 فَاتَّسَعَ الرَّجُلُ كَثِيرا جِدّا وَكَانَ لَهُ غَنَمٌ كَثِيرٌ وَجَوَارٍ وَعَبِيدٌ وَجِمَالٌ وَحَمِيرٌ.

\chapter{31}

\par 1 فَسَمِعَ يَعْقُوبُ بَنِي لابَانَ يَقُولُونَ: «اخَذَ يَعْقُوبُ كُلَّ مَا كَانَ لابِينَا وَمِمَّا لابِينَا صَنَعَ كُلَّ هَذَا الْمَجْدِ».
\par 2 وَنَظَرَ يَعْقُوبُ وَجْهَ لابَانَ وَاذَا هُوَ لَيْسَ مَعَهُ كَامْسٍ وَاوَّلَ مِنْ امْسِ.
\par 3 وَقَالَ الرَّبُّ لِيَعْقُوبَ: «ارْجِعْ الَى ارْضِ ابَائِكَ وَالَى عَشِيرَتِكَ فَاكُونَ مَعَكَ».
\par 4 فَارْسَلَ يَعْقُوبُ وَدَعَا رَاحِيلَ وَلَيْئَةَ الَى الْحَقْلِ الَى غَنَمِهِ
\par 5 وَقَالَ لَهُمَا: «انَا ارَى وَجْهَ ابِيكُمَا انَّهُ لَيْسَ نَحْوِي كَامْسِ وَاوَّلَ مِنْ امْسِ. وَلَكِنْ الَهُ ابِي كَانَ مَعِي.
\par 6 وَانْتُمَا تَعْلَمَانِ انِّي بِكُلِّ قُوَّتِي خَدَمْتُ ابَاكُمَا
\par 7 وَامَّا ابُوكُمَا فَغَدَرَ بِي وَغَيَّرَ اجْرَتِي عَشَرَ مَرَّاتٍ. لَكِنَّ اللهَ لَمْ يَسْمَحْ لَهُ انْ يَصْنَعَ بِي شَرّا.
\par 8 انْ قَالَ: الرُّقْطُ تَكُونُ اجْرَتَكَ وَلَدَتْ كُلُّ الْغَنَمِ رُقْطا. وَانْ قَالَ: الْمُخَطَّطَةُ تَكُونُ اجْرَتَكَ وَلَدَتْ كُلُّ الْغَنَمِ مُخَطَّطَةً.
\par 9 فَقَدْ سَلَبَ اللهُ مَوَاشِيَ ابِيكُمَا وَاعْطَانِي.
\par 10 وَحَدَثَ فِي وَقْتِ تَوَحُّمِ الْغَنَمِ انِّي رَفَعْتُ عَيْنَيَّ وَنَظَرْتُ فِي حُلْمٍ وَاذَا الْفُحُولُ الصَّاعِدَةُ عَلَى الْغَنَمِ مُخَطَّطَةٌ وَرَقْطَاءُ وَمُنَمَّرَةٌ.
\par 11 وَقَالَ لِي مَلاكُ اللهِ فِي الْحُلْمِ: يَا يَعْقُوبُ. فَقُلْتُ: هَئَنَذَا.
\par 12 فَقَالَ: ارْفَعْ عَيْنَيْكَ وَانْظُرْ! جَمِيعُ الْفُحُولِ الصَّاعِدَةِ عَلَى الْغَنَمِ مُخَطَّطَةٌ وَرَقْطَاءُ وَمُنَمَّرَةٌ لانِّي قَدْ رَايْتُ كُلَّ مَا يَصْنَعُ بِكَ لابَانُ.
\par 13 انَا الَهُ بَيْتِ ايلَ حَيْثُ مَسَحْتَ عَمُودا. حَيْثُ نَذَرْتَ لِي نَذْرا. الْانَ قُمِ اخْرُجْ مِنْ هَذِهِ الارْضِ وَارْجِعْ الَى ارْضِ مِيلادِكَ».
\par 14 فَاجَابَتْ رَاحِيلُ وَلَيْئَةُ: «الَنَا ايْضا نَصِيبٌ وَمِيرَاثٌ فِي بَيْتِ ابِينَا؟
\par 15 الَمْ نُحْسَبْ مِنْهُ اجْنَبِيَّتَيْنِ لانَّهُ بَاعَنَا وَقَدْ اكَلَ ايْضا ثَمَنَنَا؟
\par 16 انَّ كُلَّ الْغِنَى الَّذِي سَلَبَهُ اللهُ مِنْ ابِينَا هُوَ لَنَا وَلاوْلادِنَا. فَالْانَ كُلَّ مَا قَالَ لَكَ اللهُ افْعَلْ».
\par 17 فَقَامَ يَعْقُوبُ وَحَمَلَ اوْلادَهُ وَنِسَاءَهُ عَلَى الْجِمَالِ
\par 18 وَسَاقَ كُلَّ مَوَاشِيهِ وَجَمِيعَ مُقْتَنَاهُ الَّذِي كَانَ قَدِ اقْتَنَى: مَوَاشِيَ اقْتِنَائِهِ الَّتِي اقْتَنَى فِي فَدَّانَِ ارَامَ لِيَجِيءَ الَى اسْحَاقَ ابِيهِ الَى ارْضِ كَنْعَانَ.
\par 19 وَامَّا لابَانُ فَكَانَ قَدْ مَضَى لِيَجُزَّ غَنَمَهُ فَسَرِقَتْ رَاحِيلُ اصْنَامَ ابِيهَا.
\par 20 وَخَدَعَ يَعْقُوبُ قَلْبَ لابَانَ الارَامِيِّ اذْ لَمْ يُخْبِرْهُ بِانَّهُ هَارِبٌ.
\par 21 فَهَرَبَ هُوَ وَكُلُّ مَا كَانَ لَهُ وَقَامَ وَعَبَرَ النَّهْرَ وَجَعَلَ وَجْهَهُ نَحْوَ جَبَلِ جِلْعَادَ.
\par 22 فَاخْبِرَ لابَانُ فِي الْيَوْمِ الثَّالِثِ بِانَّ يَعْقُوبَ قَدْ هَرَبَ
\par 23 فَاخَذَ اخْوَتَهُ مَعَهُ وَسَعَى وَرَاءَهُ مَسِيرَةَ سَبْعَةِ ايَّامٍ فَادْرَكَهُ فِي جَبَلِ جِلْعَادَ.
\par 24 وَاتَى اللهُ الَى لابَانَ الارَامِيِّ فِي حُلْمِ اللَّيْلِ وَقَالَ لَهُ: «احْتَرِزْ مِنْ انْ تُكَلِّمَ يَعْقُوبَ بِخَيْرٍ اوْ شَرٍّ».
\par 25 فَلَحِقَ لابَانُ يَعْقُوبَ وَيَعْقُوبُ قَدْ ضَرَبَ خَيْمَتَهُ فِي الْجَبَلِ. فَضَرَبَ لابَانُ مَعَ اخْوَتِهِ فِي جَبَلِ جِلْعَادَ.
\par 26 وَقَالَ لابَانُ لِيَعْقُوبَ: «مَاذَا فَعَلْتَ وَقَدْ خَدَعْتَ قَلْبِي وَسُقْتَ بَنَاتِي كَسَبَايَا السَّيْفِ؟
\par 27 لِمَاذَا هَرَبْتَ خُفْيَةً وَخَدَعْتَنِي وَلَمْ تُخْبِرْنِي حَتَّى اشَيِّعَكَ بِالْفَرَحِ وَالاغَانِيِّ بِالدُّفِّ وَالْعُودِ
\par 28 وَلَمْ تَدَعْنِي اقَبِّلُ بَنِيَّ وَبَنَاتِي؟ الْانَ بِغَبَاوَةٍ فَعَلْتَ!
\par 29 فِي قُدْرَةِ يَدِي انْ اصْنَعَ بِكُمْ شَرّا وَلَكِنْ الَهُ ابِيكُمْ كَلَّمَنِيَ الْبَارِحَةَ قَائِلا: احْتَرِزْ مِنْ انْ تُكَلِّمَ يَعْقُوبَ بِخَيْرٍ اوْ شَرٍّ.
\par 30 وَالْانَ انْتَ ذَهَبْتَ لانَّكَ قَدِ اشْتَقْتَ الَى بَيْتِ ابِيكَ وَلَكِنْ لِمَاذَا سَرِقْتَ الِهَتِي؟»
\par 31 فَاجَابَ يَعْقُوبُ: «انِّي خِفْتُ لانِّي قُلْتُ لَعَلَّكَ تَغْتَصِبُ ابْنَتَيْكَ مِنِّي.
\par 32 الَّذِي تَجِدُ الِهَتَكَ مَعَهُ لا يَعِيشُ. قُدَّامَ اخْوَتِنَا انْظُرْ مَاذَا مَعِي وَخُذْهُ لِنَفْسِكَ». (وَلَمْ يَكُنْ يَعْقُوبُ يَعْلَمُ انَّ رَاحِيلَ سَرَقَتْهَا).
\par 33 فَدَخَلَ لابَانُ خِبَاءَ يَعْقُوبَ وَخِبَاءَ لَيْئَةَ وَخِبَاءَ الْجَارِيَتَيْنِ وَلَمْ يَجِدْ. وَخَرَجَ مِنْ خِبَاءِ لَيْئَةَ وَدَخَلَ خِبَاءَ رَاحِيلَ.
\par 34 وَكَانَتْ رَاحِيلُ قَدْ اخَذَتِ الاصْنَامَ وَوَضَعَتْهَا فِي حِدَاجَةِ الْجَمَلِ وَجَلَسَتْ عَلَيْهَا. فَجَسَّ لابَانُ كُلَّ الْخِبَاءِ وَلَمْ يَجِدْ.
\par 35 وَقَالَتْ لابِيهَا: «لا يَغْتَظْ سَيِّدِي انِّي لا اسْتَطِيعُ انْ اقُومَ امَامَكَ لانَّ عَلَيَّ عَادَةَ النِّسَاءِ». فَفَتَّشَ وَلَمْ يَجِدِ الاصْنَامَ.
\par 36 فَاغْتَاظَ يَعْقُوبُ وَخَاصَمَ لابَانَ. وَقَالَ يَعْقُوبُ لِلابَانَ: «مَا جُرْمِي؟ مَا خَطِيَّتِي حَتَّى حَمِيتَ وَرَائِي؟
\par 37 انَّكَ جَسَسْتَ جَمِيعَ اثَاثِي. مَاذَا وَجَدْتَ مِنْ جَمِيعِ اثَاثِ بَيْتِكَ؟ ضَعْهُ هَهُنَا قُدَّامَ اخْوَتِي وَاخْوَتِكَ! فَلْيُنْصِفُوا بَيْنَنَا الِاثْنَيْنِ.
\par 38 الْانَ عِشْرِينَ سَنَةً انَا مَعَكَ. نِعَاجُكَ وَعِنَازُكَ لَمْ تُسْقِطْ. وَكِبَاشَ غَنَمِكَ لَمْ اكُلْ.
\par 39 فَرِيسَةً لَمْ احْضِرْ الَيْكَ. انَا كُنْتُ اخْسَرُهَا. مِنْ يَدِي كُنْتَ تَطْلُبُهَا. مَسْرُوقَةَ النَّهَارِ اوْ مَسْرُوقَةَ اللَّيْلِ.
\par 40 كُنْتُ فِي النَّهَارِ يَاكُلُنِي الْحَرُّ وَفِي اللَّيْلِ الْجَلِيدُ وَطَارَ نَوْمِي مِنْ عَيْنَيَّ.
\par 41 الْانَ لِي عِشْرُونَ سَنَةً فِي بَيْتِكَ. خَدَمْتُكَ ارْبَعَ عَشَرَةَ سَنَةً بَابْنَتَيْكَ وَسِتَّ سِنِينٍ بِغَنَمِكَ. وَقَدْ غَيَّرْتَ اجْرَتِي عَشَرَ مَرَّاتٍ!
\par 42 لَوْلا انَّ الَهَ ابِي الَهَ ابْرَاهِيمَ وَهَيْبَةَ اسْحَاقَ كَانَ مَعِي لَكُنْتَ الْانَ قَدْ صَرَفْتَنِي فَارِغا. قَدْ نَظَرَ اللهُ مَشَقَّتِي وَتَعَبَ يَدَيَّ فَوَبَّخَكَ الْبَارِحَةَ».
\par 43 فَاجَابَ لابَانُ: «الْبَنَاتُ بَنَاتِي وَالْبَنُونَ بَنِيَّ وَالْغَنَمُ غَنَمِي وَكُلُّ مَا انْتَ تَرَى فَهُوَ لِي. فَبَنَاتِي مَاذَا اصْنَعُ بِهِنَّ الْيَوْمَ اوْ بِاوْلادِهِنَّ الَّذِينَ وَلَدْنَ؟
\par 44 فَالْانَ هَلُمَّ نَقْطَعْ عَهْدا انَا وَانْتَ فَيَكُونُ شَاهِدا بَيْنِي وَبَيْنَكَ».
\par 45 فَاخَذَ يَعْقُوبُ حَجَرا وَاوْقَفَهُ عَمُودا
\par 46 وَقَالَ يَعْقُوبُ لاخْوَتِهِ: «الْتَقِطُوا حِجَارَةً». فَاخَذُوا حِجَارَةً وَعَمِلُوا رُجْمَةً وَاكَلُوا هُنَاكَ عَلَى الرُّجْمَةِ.
\par 47 وَدَعَاهَا لابَانُ «يَجَرْ سَهْدُوثَا» وَامَّا يَعْقُوبُ فَدَعَاهَا «جَلْعِيدَ»
\par 48 وَقَالَ لابَانُ: «هَذِهِ الرُّجْمَةُ هِيَ شَاهِدَةٌ بَيْنِي وَبَيْنَكَ الْيَوْمَ». لِذَلِكَ دُعِيَ اسْمُهَا «جَلْعِيدَ»
\par 49 وَ «الْمِصْفَاةَ» لانَّهُ قَالَ: «لِيُرَاقِبِ الرَّبُّ بَيْنِي وَبَيْنَكَ حِينَمَا نَتَوَارَى بَعْضُنَا عَنْ بَعْضٍ.
\par 50 انَّكَ لا تُذِلُّ بَنَاتِي وَلا تَاخُذُ نِسَاءً عَلَى بَنَاتِي. لَيْسَ انْسَانٌ مَعَنَا. انْظُرْ. اللهُ شَاهِدٌ بَيْنِي وَبَيْنَكَ».
\par 51 وَقَالَ لابَانُ لِيَعْقُوبَ: «هُوَذَا هَذِهِ الرُّجْمَةُ وَهُوَذَا الْعَمُودُ الَّذِي وَضَعْتُ بَيْنِي وَبَيْنَكَ.
\par 52 شَاهِدَةٌ هَذِهِ الرُّجْمَةُ وَشَاهِدٌ الْعَمُودُ انِّي لا اتَجَاوَزُ هَذِهِ الرُّجْمَةَ الَيْكَ وَانَّكَ لا تَتَجَاوَزُ هَذِهِ الرُّجْمَةَ وَهَذَا الْعَمُودَ الَيَّ لِلشَّرِّ.
\par 53 الَهُ ابْرَاهِيمَ وَالِهَةُ نَاحُورَ الِهَةُ ابِيهِمَا يَقْضُونَ بَيْنَنَا». وَحَلَفَ يَعْقُوبُ بِهَيْبَةِ ابِيهِ اسْحَاقَ.
\par 54 وَذَبَحَ يَعْقُوبُ ذَبِيحَةً فِي الْجَبَلِ وَدَعَا اخْوَتَهُ لِيَاكُلُوا طَعَاما. فَاكَلُوا طَعَاما وَبَاتُوا فِي الْجَبَلِ.
\par 55 ثُمَّ بَكَّرَ لابَانُ صَبَاحا وَقَبَّلَ بَنِيهِ وَبَنَاتِهِ وَبَارَكَهُمْ وَمَضَى. وَرَجَعَ لابَانُ الَى مَكَانِهِ.

\chapter{32}

\par 1 وَامَّا يَعْقُوبُ فَمَضَى فِي طَرِيقِهِ وَلاقَاهُ مَلائِكَةُ اللهِ.
\par 2 وَقَالَ يَعْقُوبُ اذْ رَاهُمْ: «هَذَا جَيْشُ اللهِ!» فَدَعَا اسْمَ ذَلِكَ الْمَكَانِ «مَحَنَايِمَ».
\par 3 وَارْسَلَ يَعْقُوبُ رُسُلا قُدَّامَهُ الَى عِيسُوَ اخِيهِ الَى ارْضِ سَعِيرَ بِلادِ ادُومَ
\par 4 وَامَرَهُمْ: «هَكَذَا تَقُولُونَ لِسَيِّدِي عِيسُوَ: هَكَذَا قَالَ عَبْدُكَ يَعْقُوبُ: تَغَرَّبْتُ عِنْدَ لابَانَ وَلَبِثْتُ الَى الْانَ.
\par 5 وَقَدْ صَارَ لِي بَقَرٌ وَحَمِيرٌ وَغَنَمٌ وَعَبِيدٌ وَامَاءٌ. وَارْسَلْتُ لِاخْبِرَ سَيِّدِي لِكَيْ اجِدَ نِعْمَةً فِي عَيْنَيْكَ».
\par 6 فَرَجَعَ الرُّسُلُ الَى يَعْقُوبَ قَائِلِينَ: «اتَيْنَا الَى اخِيكَ الَى عِيسُو وَهُوَ ايْضا قَادِمٌ لِلِقَائِكَ وَارْبَعُ مِئَةِ رَجُلٍ مَعَهُ».
\par 7 فَخَافَ يَعْقُوبُ جِدّا وَضَاقَ بِهِ الامْرُ. فَقَسَمَ الْقَوْمَ الَّذِينَ مَعَهُ وَالْغَنَمَ وَالْبَقَرَ وَالْجِمَالَ الَى جَيْشَيْنِ.
\par 8 وَقَالَ: «انْ جَاءَ عِيسُو الَى الْجَيْشِ الْوَاحِدِ وَضَرَبَهُ يَكُونُ الْجَيْشُ الْبَاقِي نَاجِيا».
\par 9 وَقَالَ يَعْقُوبُ: «يَا الَهَ ابِي ابْرَاهِيمَ وَالَهَ ابِي اسْحَاقَ الرَّبَّ الَّذِي قَالَ لِيَ: ارْجِعْ الَى ارْضِكَ وَالَى عَشِيرَتِكَ فَاحْسِنَ الَيْكَ.
\par 10 صَغِيرٌ انَا عَنْ جَمِيعِ الْطَافِكَ وَجَمِيعِ الامَانَةِ الَّتِي صَنَعْتَ الَى عَبْدِكَ. فَانِّي بِعَصَايَ عَبَرْتُ هَذَا الارْدُنَّ وَالْانَ قَدْ صِرْتُ جَيْشَيْنِ.
\par 11 نَجِّنِي مِنْ يَدِ اخِي مِنْ يَدِ عِيسُوَ لانِّي خَائِفٌ مِنْهُ انْ يَاتِيَ وَيَضْرِبَنِي الامَّ مَعَ الْبَنِينَ.
\par 12 وَانْتَ قَدْ قُلْتَ: انِّي احْسِنُ الَيْكَ وَاجْعَلُ نَسْلَكَ كَرَمْلِ الْبَحْرِ الَّذِي لا يُعَدُّ لِلْكَثْرَةِ».
\par 13 وَبَاتَ هُنَاكَ تِلْكَ اللَّيْلَةَ وَاخَذَ مِمَّا اتَى بِيَدِهِ هَدِيَّةً لِعِيسُو اخِيهِ:
\par 14 مِئَتَيْ عَنْزٍ وَعِشْرِينَ تَيْسا مِئَتَيْ نَعْجَةٍ وَعِشْرِينَ كَبْشا
\par 15 ثَلاثِينَ نَاقَةً مُرْضِعَةً وَاوْلادَهَا ارْبَعِينَ بَقَرَةً وَعَشَرَةَ ثِيرَانٍ عِشْرِينَ اتَانا وَعَشَرَةَ حَمِيرٍ
\par 16 وَدَفَعَهَا الَى يَدِ عَبِيدِهِ قَطِيعا قَطِيعا عَلَى حِدَةٍ. وَقَالَ لِعَبِيدِهِ: «اجْتَازُوا قُدَّامِي وَاجْعَلُوا فُسْحَةً بَيْنَ قَطِيعٍ وَقَطِيعٍ».
\par 17 وَامَرَ الاوَّلَ: «اذَا صَادَفَكَ عِيسُو اخِي وَسَالَك: لِمَنْ انْتَ وَالَى ايْنَ تَذْهَبُ وَلِمَنْ هَذَا الَّذِي قُدَّامَكَ؟
\par 18 تَقُولُ: لِعَبْدِكَ يَعْقُوبَ. هُوَ هَدِيَّةٌ مُرْسَلَةٌ لِسَيِّدِي عِيسُوَ وَهَا هُوَ ايْضا وَرَاءَنَا».
\par 19 وَامَرَ ايْضا الثَّانِيَ وَالثَّالِثَ وَجَمِيعَ السَّائِرِينَ وَرَاءَ الْقُطْعَانِ: «بِمِثْلِ هَذَا الْكَلامِ تُكَلِّمُونَ عِيسُوَ حِينَمَا تَجِدُونَهُ
\par 20 وَتَقُولُونَ: هُوَذَا عَبْدُكَ يَعْقُوبُ ايْضا وَرَاءَنَا». لانَّهُ قَالَ: «اسْتَعْطِفُ وَجْهَهُ بِالْهَدِيَّةِ السَّائِرَةِ امَامِي وَبَعْدَ ذَلِكَ انْظُرُ وَجْهَهُ عَسَى انْ يَرْفَعَ وَجْهِي».
\par 21 فَاجْتَازَتِ الْهَدِيَّةُ قُدَّامَهُ وَامَّا هُوَ فَبَاتَ تِلْكَ اللَّيْلَةَ فِي الْمَحَلَّةِ.
\par 22 ثُمَّ قَامَ فِي تِلْكَ اللَّيْلَةِ وَاخَذَ امْرَاتَيْهِ وَجَارِيَتَيْهِ وَاوْلادَهُ الاحَدَ عَشَرَ وَعَبَرَ مَخَاضَةَ يَبُّوقَ.
\par 23 اخَذَهُمْ وَاجَازَهُمُ الْوَادِيَ وَاجَازَ مَا كَانَ لَهُ.
\par 24 فَبَقِيَ يَعْقُوبُ وَحْدَهُ. وَصَارَعَهُ انْسَانٌ حَتَّى طُلُوعِ الْفَجْرِ.
\par 25 وَلَمَّا رَاى انَّهُ لا يَقْدِرُ عَلَيْهِ ضَرَبَ حُقَّ فَخْذِهِ فَانْخَلَعَ حُقُّ فَخْذِ يَعْقُوبَ فِي مُصَارَعَتِهِ مَعَهُ.
\par 26 وَقَالَ: «اطْلِقْنِي لانَّهُ قَدْ طَلَعَ الْفَجْرُ». فَقَالَ: «لا اطْلِقُكَ انْ لَمْ تُبَارِكْنِي».
\par 27 فَسَالَهُ: «مَا اسْمُكَ؟» فَقَالَ: «يَعْقُوبُ».
\par 28 فَقَالَ: «لا يُدْعَى اسْمُكَ فِي مَا بَعْدُ يَعْقُوبَ بَلْ اسْرَائِيلَ لانَّكَ جَاهَدْتَ مَعَ اللهِ وَالنَّاسِ وَقَدِرْتَ».
\par 29 وَسَالَهُ يَعْقُوبُ: «اخْبِرْنِي بِاسْمِكَ». فَقَالَ: «لِمَاذَا تَسْالُ عَنِ اسْمِي؟» وَبَارَكَهُ هُنَاكَ.
\par 30 فَدَعَا يَعْقُوبُ اسْمَ الْمَكَانِ «فَنِيئِيلَ» قَائِلا: «لانِّي نَظَرْتُ اللهَ وَجْها لِوَجْهٍ وَنُجِّيَتْ نَفْسِي».
\par 31 وَاشْرَقَتْ لَهُ الشَّمْسُ اذْ عَبَرَ فَنُوئِيلَ وَهُوَ يَخْمَعُ عَلَى فَخْذِهِ -
\par 32 لِذَلِكَ لا يَاكُلُ بَنُو اسْرَائِيلَ عِرْقَ النَّسَا الَّذِي عَلَى حُقِّ الْفَخِْذِ الَى هَذَا الْيَوْمِ لانَّهُ ضَرَبَ حُقَّ فَخْذِ يَعْقُوبَ عَلَى عِرْقِ النَّسَا.

\chapter{33}

\par 1 وَرَفَعَ يَعْقُوبُ عَيْنَيْهِ وَنَظَرَ وَاذَا عِيسُو مُقْبِلٌ وَمَعَهُ ارْبَعُ مِئَةِ رَجُلٍ فَقَسَمَ الاوْلادَ عَلَى لَيْئَةَ وَعَلَى رَاحِيلَ وَعَلَى الْجَارِيَتَيْنِ
\par 2 وَوَضَعَ الْجَارِيَتَيْنِ وَاوْلادَهُمَا اوَّلا وَلَيْئَةَ وَاوْلادَهَا وَرَاءَهُمْ وَرَاحِيلَ وَيُوسُفَ اخِيرا.
\par 3 وَامَّا هُوَ فَاجْتَازَ قُدَّامَهُمْ وَسَجَدَ الَى الارْضِ سَبْعَ مَرَّاتٍ حَتَّى اقْتَرَبَ الَى اخِيهِ.
\par 4 فَرَكَضَ عِيسُو لِلِقَائِهِ وَعَانَقَهُ وَوَقَعَ عَلَى عُنُقِهِ وَقَبَّلَهُ. وَبَكَيَا.
\par 5 ثُمَّ رَفَعَ عَيْنَيْهِ وَابْصَرَ النِّسَاءَ وَالاوْلادَ وَقَالَ: «مَا هَؤُلاءِ مِنْكَ؟» فَقَالَ: «الاوْلادُ الَّذِينَ انْعَمَ اللهُ بِهِمْ عَلَى عَبْدِكَ».
\par 6 فَاقْتَرَبَتِ الْجَارِيَتَانِ هُمَا وَاوْلادُهُمَا وَسَجَدَتَا
\par 7 ثُمَّ اقْتَرَبَتْ لَيْئَةُ ايْضا وَاوْلادُهَا وَسَجَدُوا وَبَعْدَ ذَلِكَ اقْتَرَبَ يُوسُفُ وَرَاحِيلُ وَسَجَدَا.
\par 8 فَقَالَ: «مَاذَا مِنْكَ كُلُّ هَذَا الْجَيْشِ الَّذِي صَادَفْتُهُ؟» فَقَالَ: «لاجِدَ نِعْمَةً فِي عَيْنَيْ سَيِّدِي».
\par 9 فَقَالَ عِيسُو: «لِي كَثِيرٌ. يَا اخِي لِيَكُنْ لَكَ الَّذِي لَكَ».
\par 10 فَقَالَ يَعْقُوبُ: «لا. انْ وَجَدْتُ نِعْمَةً فِي عَيْنَيْكَ تَاخُذْ هَدِيَّتِي مِنْ يَدِي لانِّي رَايْتُ وَجْهَكَ كَمَا يُرَى وَجْهُ اللهِ فَرَضِيتَ عَلَيَّ.
\par 11 خُذْ بَرَكَتِي الَّتِي اتِيَ بِهَا الَيْكَ لانَّ اللهَ قَدْ انْعَمَ عَلَيَّ وَلِي كُلُّ شَيْءٍ». وَالَحَّ عَلَيْهِ فَاخَذَ.
\par 12 ثُمَّ قَالَ: «لِنَرْحَلْ وَنَذْهَبْ وَاذْهَبُ انَا قُدَّامَكَ».
\par 13 فَقَالَ لَهُ: «سَيِّدِي عَالِمٌ انَّ الاوْلادَ رَخْصَةٌ وَالْغَنَمَ وَالْبَقَرَ الَّتِي عِنْدِي مُرْضِعَةٌ. فَانِ اسْتَكَدُّوهَا يَوْما وَاحِدا مَاتَتْ كُلُّ الْغَنَمِ.
\par 14 لِيَجْتَزْ سَيِّدِي قُدَّامَ عَبْدِهِ وَانَا اسْتَاقُ عَلَى مَهَلِي فِي اثَرِ الامْلاكِ الَّتِي قُدَّامِي وَفِي اثَرِ الاوْلادِ حَتَّى اجِيءَ الَى سَيِّدِي الَى سَعِيرَ».
\par 15 فَقَالَ عِيسُو: «اتْرُكُ عِنْدَكَ مِنَ الْقَوْمِ الَّذِينَ مَعِي». فَقَالَ: «لِمَاذَا؟ دَعْنِي اجِدْ نِعْمَةً فِي عَيْنَيْ سَيِّدِي».
\par 16 فَرَجَعَ عِيسُو ذَلِكَ الْيَوْمَ فِي طَرِيقِهِ الَى سَعِيرَ.
\par 17 وَامَّا يَعْقُوبُ فَارْتَحَلَ الَى سُكُّوتَ وَبَنَى لِنَفْسِهِ بَيْتا وَصَنَعَ لِمَوَاشِيهِ مِظَلَّاتٍ. لِذَلِكَ دَعَا اسْمَ الْمَكَانِ «سُكُّوتَ».
\par 18 ثُمَّ اتَى يَعْقُوبُ سَالِما الَى مَدِينَةِ شَكِيمَ الَّتِي فِي ارْضِ كَنْعَانَ حِينَ جَاءَ مِنْ فَدَّانِ ارَامَ. وَنَزَلَ امَامَ الْمَدِينَةِ.
\par 19 وَابْتَاعَ قِطْعَةَ الْحَقْلِ الَّتِي نَصَبَ فِيهَا خَيْمَتَهُ مِنْ يَدِ بَنِي حَمُورَ ابِي شَكِيمَ بِمِئَةِ قَسِيطَةٍ.
\par 20 وَاقَامَ هُنَاكَ مَذْبَحا وَدَعَاهُ «ايلَ الَهَ اسْرَائِيلَ».

\chapter{34}

\par 1 وَخَرَجَتْ دِينَةُ ابْنَةُ لَيْئَةَ الَّتِي وَلَدَتْهَا لِيَعْقُوبَ لِتَنْظُرَ بَنَاتِ الارْضِ
\par 2 فَرَاهَا شَكِيمُ ابْنُ حَمُورَ الْحِوِّيِّ رَئِيسِ الارْضِ وَاخَذَهَا وَاضْطَجَعَ مَعَهَا وَاذَلَّهَا.
\par 3 وَتَعَلَّقَتْ نَفْسُهُ بِدِينَةَ ابْنَةِ يَعْقُوبَ وَاحَبَّ الْفَتَاةَ وَلاطَفَها.
\par 4 فَقَالَ شَكِيمُ لِحَمُورَ ابِيهِ: «خُذْ لِي هَذِهِ الصَّبِيَّةَ زَوْجَةً».
\par 5 وَسَمِعَ يَعْقُوبُ انَّهُ نَجَّسَ دِينَةَ ابْنَتَهُ. وَامَّا بَنُوهُ فَكَانُوا مَعَ مَوَاشِيهِ فِي الْحَقْلِ فَسَكَتَ يَعْقُوبُ حَتَّى جَاءُوا.
\par 6 فَخَرَجَ حَمُورُ ابُو شَكِيمَ الَى يَعْقُوبَ لِيَتَكَلَّمَ مَعَهُ.
\par 7 وَاتَى بَنُو يَعْقُوبَ مِنَ الْحَقْلِ حِينَ سَمِعُوا. وَغَضِبَ الرِّجَالُ وَاغْتَاظُوا جِدّا لانَّهُ صَنَعَ قَبَاحَةً فِي اسْرَائِيلَ بِمُضَاجَعَةِ ابْنَةِ يَعْقُوبَ. وَ«هَكَذَا لا يُصْنَعُ».
\par 8 وَقَالَ لَهُمْ حَمُورُ: «شَكِيمُ ابْنِي قَدْ تَعَلَّقَتْ نَفْسُهُ بِابْنَتِكُمْ. اعْطُوهُ ايَّاهَا زَوْجَةً
\par 9 وَصَاهِرُونَا. تُعْطُونَنَا بَنَاتِكُمْ وَتَاخُذُونَ لَكُمْ بَنَاتِنَا
\par 10 وَتَسْكُنُونَ مَعَنَا وَتَكُونُ الارْضُ قُدَّامَكُمُ. اسْكُنُوا وَاتَّجِرُوا فِيهَا وَتَمَلَّكُوا بِهَا».
\par 11 ثُمَّ قَالَ شَكِيمُ لابِيهَا وَلاخْوَتِهَا: «دَعُونِي اجِدْ نِعْمَةً فِي اعْيُنِكُمْ. فَالَّذِي تَقُولُونَ لِي اعْطِي.
\par 12 كَثِّرُوا عَلَيَّ جِدّا مَهْرا وَعَطِيَّةً فَاعْطِيَ كَمَا تَقُولُونَ لِي. وَاعْطُونِي الْفَتَاةَ زَوْجَةً».
\par 13 فَاجَابَ بَنُو يَعْقُوبَ شَكِيمَ وَحَمُورَ ابَاهُ بِمَكْرٍ لانَّهُ كَانَ قَدْ نَجَّسَ دِينَةَ اخْتَهُمْ:
\par 14 «لا نَسْتَطِيعُ انْ نَفْعَلَ هَذَا الامْرَ انْ نُعْطِيَ اخْتَنَا لِرَجُلٍ اغْلَفَ لانَّهُ عَارٌ لَنَا.
\par 15 غَيْرَ انَّنَا بِهَذَا نُواتِيكُمْ: انْ صِرْتُمْ مِثْلَنَا بِخَتْنِكُمْ كُلَّ ذَكَرٍ.
\par 16 نُعْطِيكُمْ بَنَاتِنَا وَنَاخُذُ لَنَا بَنَاتِكُمْ وَنَسْكُنُ مَعَكُمْ وَنَصِيرُ شَعْبا وَاحِدا.
\par 17 وَانْ لَمْ تَسْمَعُوا لَنَا انْ تَخْتَتِنُوا نَاخُذُ ابْنَتَنَا وَنَمْضِي».
\par 18 فَحَسُنَ كَلامُهُمْ فِي عَيْنَيْ حَمُورَ وَفِي عَيْنَيْ شَكِيمَ بْنِ حَمُورَ.
\par 19 وَلَمْ يَتَاخَّرِ الْغُلامُ انْ يَفْعَلَ الامْرَ لانَّهُ كَانَ مَسْرُورا بِابْنَةِ يَعْقُوبَ. وَكَانَ اكْرَمَ جَمِيعِ بَيْتِ ابِيهِ.
\par 20 فَاتَى حَمُورُ وَشَكِيمُ ابْنُهُ الَى بَابِ مَدِينَتِهُِمَا وَقَالا لاهْلَ مَدِينَتِهُِمَا:
\par 21 «هَؤُلاءِ الْقَوْمُ مُسَالِمُونَ لَنَا. فَلْيَسْكُنُوا فِي الارْضِ وَيَتَّجِرُوا فِيهَا. وَهُوَذَا الارْضُ وَاسِعَةُ الطَّرَفَيْنِ امَامَهُمْ. نَاخُذُ لَنَا بَنَاتِهِمْ زَوْجَاتٍ وَنُعْطِيهِمْ بَنَاتِنَا.
\par 22 غَيْرَ انَّهُ بِهَذَا فَقَطْ يُواتِينَا الْقَوْمُ عَلَى السَّكَنِ مَعَنَا لِنَصِيرَ شَعْبا وَاحِدا: بِخَتْنِنَا كُلَّ ذَكَرٍ كَمَا هُمْ مَخْتُونُونَ.
\par 23 الا تَكُونُ مَوَاشِيهِمْ وَمُقْتَنَاهُمْ وَكُلُّ بَهَائِمِهِمْ لَنَا؟ نُواتِيهِمْ فَقَطْ فَيَسْكُنُونَ مَعَنَا».
\par 24 فَسَمِعَ لِحَمُورَ وَشَكِيمَ ابْنِهِ جَمِيعُ الْخَارِجِينَ مِنْ بَابِ الْمَدِينَةِ. وَاخْتَتَنَ كُلُّ ذَكَرٍ - كُلُّ الْخَارِجِينَ مِنْ بَابِ الْمَدِينَةِ.
\par 25 فَحَدَثَ فِي الْيَوْمِ الثَّالِثِ اذْ كَانُوا مُتَوَجِّعِينَ انَّ ابْنَيْ يَعْقُوبَ شِمْعُونَ وَلاوِيَ اخَوَيْ دِينَةَ اخَذَا كُلُّ وَاحِدٍ سَيْفَهُ وَاتَيَا عَلَى الْمَدِينَةِ بِامْنٍ وَقَتَلا كُلَّ ذَكَرٍ.
\par 26 وَقَتَلا حَمُورَ وَشَكِيمَ ابْنَهُ بِحَدِّ السَّيْفِ وَاخَذَا دِينَةَ مِنْ بَيْتِ شَكِيمَ وَخَرَجَا.
\par 27 ثُمَّ اتَى بَنُو يَعْقُوبَ عَلَى الْقَتْلَى وَنَهَبُوا الْمَدِينَةَ لانَّهُمْ نَجَّسُوا اخْتَهُمْ.
\par 28 غَنَمَهُمْ وَبَقَرَهُمْ وَحَمِيرَهُمْ وَكُلَُّ مَا فِي الْمَدِينَةِ وَمَا فِي الْحَقْلِ اخَذُوهُ.
\par 29 وَسَبُوا وَنَهَبُوا كُلَّ ثَرْوَتِهِمْ وَكُلَّ اطْفَالِهِمْ وَنِسَاءَهُمْ وَكُلَّ مَا فِي الْبُيُوتِ.
\par 30 فَقَالَ يَعْقُوبُ لِشَمْعُونَ وَلاوِي: «كَدَّرْتُمَانِي بِتَكْرِيهِكُمَا ايَّايَ عِنْدَ سُكَّانِ الارْضِ الْكَنْعَانِيِّينَ وَالْفِرِزِيِّينَ وَانَا نَفَرٌ قَلِيلٌ. فَيَجْتَمِعُونَ عَلَيَّ وَيَضْرِبُونَنِي فَابِيدُ انَا وَبَيْتِي».
\par 31 فَقَالا: «انَظِيرَ زَانِيَةٍ يَفْعَلُ بِاخْتِنَا؟».

\chapter{35}

\par 1 ثُمَّ قَالَ اللهُ لِيَعْقُوبَ: «قُمِ اصْعَدْ الَى بَيْتَِ ايلَ وَاقِمْ هُنَاكَ وَاصْنَعْ هُنَاكَ مَذْبَحا لِلَّهِ الَّذِي ظَهَرَ لَكَ حِينَ هَرَبْتَ مِنْ وَجْهِ عِيسُو اخِيكَ».
\par 2 فَقَالَ يَعْقُوبُ لِبَيْتِهِ وَلِكُلِّ مَنْ كَانَ مَعَهُ: «اعْزِلُوا الْالِهَةَ الْغَرِيبَةَ الَّتِي بَيْنَكُمْ وَتَطَهَّرُوا وَابْدِلُوا ثِيَابَكُمْ.
\par 3 وَلْنَقُمْ وَنَصْعَدْ الَى بَيْتِ ايلَ فَاصْنَعَ هُنَاكَ مَذْبَحا لِلَّهِ الَّذِي اسْتَجَابَ لِي فِي يَوْمِ ضِيقَتِي وَكَانَ مَعِي فِي الطَّرِيقِ الَّذِي ذَهَبْتُ فِيهِ».
\par 4 فَاعْطُوا يَعْقُوبَ كُلَّ الْالِهَةِ الْغَرِيبَةِ الَّتِي فِي ايْدِيهِمْ وَالاقْرَاطَِ الَّتِي فِي اذَانِهِمْ فَطَمَرَهَا يَعْقُوبُ تَحْتَ الْبُطْمَةِ الَّتِي عِنْدَ شَكِيمَ.
\par 5 ثُمَّ رَحَلُوا. وَكَانَ خَوْفُ اللهِ عَلَى الْمُدُنِ الَّتِي حَوْلَهُمْ فَلَمْ يَسْعُوا وَرَاءَ بَنِي يَعْقُوبَ.
\par 6 فَاتَى يَعْقُوبُ الَى لُوزَ الَّتِي فِي ارْضِ كَنْعَانَ (وَهِيَ بَيْتُ ايلَ) هُوَ وَجَمِيعُ الْقَوْمِ الَّذِينَ مَعَهُ.
\par 7 وَبَنَى هُنَاكَ مَذْبَحا وَدَعَا الْمَكَانَ «ايلَ بَيْتِ ايلَ» لانَّهُ هُنَاكَ ظَهَرَ لَهُ اللهُ حِينَ هَرَبَ مِنْ وَجْهِ اخِيهِ.
\par 8 وَمَاتَتْ دَبُورَةُ مُرْضِعَةُ رِفْقَةَ وَدُفِنَتْ تَحْتَ بَيْتَ ايلَ تَحْتَ الْبَلُّوطَةِ فَدَعَا اسْمَهَا «الُّونَ بَاكُوتَ».
\par 9 وَظَهَرَ اللهُ لِيَعْقُوبَ ايْضا حِينَ جَاءَ مِنْ فَدَّانَِ ارَامَ وَبَارَكَهُ.
\par 10 وَقَالَ لَهُ اللهُ: «اسْمُكَ يَعْقُوبُ. لا يُدْعَى اسْمُكَ فِيمَا بَعْدُ يَعْقُوبَ بَلْ يَكُونُ اسْمُكَ اسْرَائِيلَ». فَدَعَا اسْمَهُ اسْرَائِيلَ.
\par 11 وَقَالَ لَهُ اللهُ: «انَا اللهُ الْقَدِيرُ. اثْمِرْ وَاكْثُرْ. امَّةٌ وَجَمَاعَةُ امَمٍ تَكُونُ مِنْكَ. وَمُلُوكٌ سَيَخْرُجُونَ مِنْ صُلْبِكَ.
\par 12 وَالارْضُ الَّتِي اعْطَيْتُ ابْرَاهِيمَ وَاسْحَاقَ لَكَ اعْطِيهَا. وَلِنَسْلِكَ مِنْ بَعْدِكَ اعْطِي الارْضَ».
\par 13 ثُمَّ صَعِدَ اللهُ عَنْهُ فِي الْمَكَانِ الَّذِي فِيهِ تَكَلَّمَ مَعَهُ
\par 14 فَنَصَبَ يَعْقُوبُ عَمُودا فِي الْمَكَانِ الَّذِي فِيهِ تَكَلَّمَ مَعَهُ عَمُودا مِنْ حَجَرٍ وَسَكَبَ عَلَيْهِ سَكِيبا وَصَبَّ عَلَيْهِ زَيْتا
\par 15 وَدَعَا يَعْقُوبُ اسْمَ الْمَكَانِ الَّذِي فِيهِ تَكَلَّمَ اللهُ مَعَهُ «بَيْتَ ايلَ».
\par 16 ثُمَّ رَحَلُوا مِنْ بَيْتِ ايلَ. وَلَمَّا كَانَ مَسَافَةٌ مِنَ الارْضِ بَعْدُ حَتَّى يَاتُوا الَى افْرَاتَةَ وَلَدَتْ رَاحِيلُ وَتَعَسَّرَتْ وِلادَتُهَا.
\par 17 فَقَالَتِ الْقَابِلَةُ لَهَا: «لا تَخَافِي لانَّ هَذَا ايْضا ابْنٌ لَكِ».
\par 18 وَكَانَ عِنْدَ خُرُوجِ نَفْسِهَا (لانَّهَا مَاتَتْ) انَّهَا دَعَتِ اسْمَهُ «بِنْ اونِي». وَامَّا ابُوهُ فَدَعَاهُ بِنْيَامِينَ.
\par 19 فَمَاتَتْ رَاحِيلُ وَدُفِنَتْ فِي طَرِيقِ افْرَاتَةَ (الَّتِي هِيَ بَيْتُ لَحْمٍ).
\par 20 فَنَصَبَ يَعْقُوبُ عَمُودا عَلَى قَبْرِهَا. وَهُوَ «عَمُودُ قَبْرِ رَاحِيلَ» الَى الْيَوْمِ.
\par 21 ثُمَّ رَحَلَ اسْرَائِيلُ وَنَصَبَ خَيْمَتَهُ وَرَاءَ مَجْدَلَ عِدْرٍ.
\par 22 وَحَدَثَ اذْ كَانَ اسْرَائِيلُ سَاكِنا فِي تِلْكَ الارْضِ انَّ رَاوبَيْنَ ذَهَبَ وَاضْطَجَعَ مَعَ بِلْهَةَ سُرِّيَّةِ ابِيهِ. وَسَمِعَ اسْرَائِيلُ. وَكَانَ بَنُو يَعْقُوبَ اثْنَيْ عَشَرَ:
\par 23 بَنُو لَيْئَةَ: رَاوبَيْنُ بِكْرُ يَعْقُوبَ وَشَمْعُونُ وَلاوِي وَيَهُوذَا وَيَسَّاكَرُ وَزَبُولُونُ.
\par 24 وَابْنَا رَاحِيلَ؛ يُوسُفُ وَبِنْيَامِينُ.
\par 25 وَابْنَا بِلْهَةَ جَارِيَةِ رَاحِيلَ: دَانُ وَنَفْتَالِي.
\par 26 وَابْنَا زِلْفَةَ جَارِيَةِ لَيْئَةَ: جَادُ وَاشِيرُ. هَؤُلاءِ بَنُو يَعْقُوبَ الَّذِينَ وُلِدُوا لَهُ فِي فَدَّانَِ ارَامَ.
\par 27 وَجَاءَ يَعْقُوبُ الَى اسْحَاقَ ابِيهِ الَى مَمْرَا قَِرْيَةِ ارْبَعَ (الَّتِي هِيَ حَبْرُونُ) حَيْثُ تَغَرَّبَ ابْرَاهِيمُ وَاسْحَاقُ.
\par 28 وَكَانَتْ ايَّامُ اسْحَاقَ مِئَةً وَثَمَانِينَ سَنَةً.
\par 29 فَاسْلَمَ اسْحَاقُ رُوحَهُ وَمَاتَ وَانْضَمَّ الَى قَوْمِهِ شَيْخا وَشَبْعَانَ ايَّاما وَدَفَنَهُ عِيسُو وَيَعْقُوبُ ابْنَاهُ.

\chapter{36}

\par 1 وَهَذِهِ مَوَالِيدُ عِيسُوَ الَّذِي هُوَ ادُومُ:
\par 2 اخَذَ عِيسُو نِسَاءَهُ مِنْ بَنَاتِ كَنْعَانَ: عَدَا بِنْتَ ايلُونَ الْحِثِّيِّ وَاهُولِيبَامَةَ بِنْتَ عَنَى بِنْتِ صِبْعُونَ الْحِوِّيِّ
\par 3 وَبَسْمَةَ بِنْتَ اسْمَاعِيلَ اخْتَ نَبَايُوتَ.
\par 4 فَوَلَدَتْ عَدَا لِعِيسُو الِيفَازَ وَوَلَدَتْ بَسْمَةُ رَعُوئِيلَ
\par 5 وَوَلَدَتْ اهُولِيبَامَةُ: يَعُوشَ وَيَعْلامَ وَقُورَحَ. هَؤُلاءِ بَنُو عِيسُو الَّذِينَ وُلِدُوا لَهُ فِي ارْضِ كَنْعَانَ.
\par 6 ثُمَّ اخَذَ عِيسُو نِسَاءَهُ وَبَنِيهِ وَبَنَاتِهِ وَجَمِيعَ نُفُوسِ بَيْتِهِ وَمَوَاشِيَهُ وَكُلَّ بَهَائِمِهِ وَكُلَّ مُقْتَنَاهُ الَّذِي اقْتَنَى فِي ارْضِ كَنْعَانَ وَمَضَى الَى ارْضٍ اخْرَى مِنْ وَجْهِ يَعْقُوبَ اخِيهِ
\par 7 لانَّ امْلاكَهُمَا كَانَتْ كَثِيرَةً عَلَى السُّكْنَى مَعا وَلَمْ تَسْتَطِعْ ارْضُ غُرْبَتِهِمَا انْ تَحْمِلَهُمَا مِنْ اجْلِ مَوَاشِيهِمَا
\par 8 فَسَكَنَ عِيسُو فِي جَبَلِ سَعِيرَ. (وَعِيسُو هُوَ ادُومُ).
\par 9 وَهَذِهِ مَوَالِيدُ عِيسُو ابِي ادُومَ فِي جَبَلِ سَعِيرَ.
\par 10 هَذِهِ اسْمَاءُ بَنِي عِيسُو: الِيفَازُ ابْنُ عَدَا امْرَاةِ عِيسُو وَرَعُوئِيلُ ابْنُ بَسْمَةَ امْرَاةِ عِيسُو.
\par 11 وَكَانَ بَنُو الِيفَازَ: تَيْمَانَ وَاوْمَارَ وَصَفْوا وَجَعْثَامَ وَقَنَازَ.
\par 12 وَكَانَتْ تِمْنَاعُ سُرِّيَّةً لالِيفَازَ بْنِ عِيسُو فَوَلَدَتْ لالِيفَازَ عَمَالِيقَ. هَؤُلاءِ بَنُو عَدَا امْرَاةِ عِيسُو.
\par 13 وَهَؤُلاءِ بَنُو رَعُوئِيلَ: نَحَثُ وَزَارَحُ وَشَمَّةُ وَمِزَّةُ. هَؤُلاءِ كَانُوا بَنِي بَسْمَةَ امْرَاةِ عِيسُو.
\par 14 وَهَؤُلاءِ كَانُوا بَنِي اهُولِيبَامَةَ بِنْتِ عَنَى بِنْتِ صِبْعُونَ امْرَاةِ عِيسُو: وَلَدَتْ لِعِيسُو يَعُوشَ وَيَعْلامَ وَقُورَحَ.
\par 15 هَؤُلاءِ امَرَاءُ بَنِي عِيسُو: بَنُو الِيفَازَ بِكْرِ عِيسُو امِيرُ تَيْمَانَ وَامِيرُ اومَارَ وَامِيرُ صَفْوٍ وَامِيرُ قَنَازَ
\par 16 وَامِيرُ قُورَحَ وَامِيرُ جَعْثَامَ وَامِيرُ عَمَالِيقَ. هَؤُلاءِ امَرَاءُ الِيفَازَ فِي ارْضِ ادُومَ. هَؤُلاءِ بَنُو عَدَا.
\par 17 وَهَؤُلاءِ بَنُو رَعُوئِيلَ بْنِ عِيسُو: امِيرُ نَحَثَ وَامِيرُ زَارَحَ وَامِيرُ شَمَّةَ وَامِيرُ مِزَّةَ. هَؤُلاءِ امَرَاءُ رَعُوئِيلَ فِي ارْضِ ادُومَ. هَؤُلاءِ بَنُو بَسْمَةَ امْرَاةِ عِيسُو.
\par 18 وَهَؤُلاءِ بَنُو اهُولِيبَامَةَ امْرَاةِ عِيسُو: امِيرُ يَعُوشَ وَامِيرُ يَعْلامَ وَامِيرُ قُورَحَ. هَؤُلاءِ امَرَاءُ اهُولِيبَامَةَ بِنْتِ عَنَى امْرَاةِ عِيسُو.
\par 19 هَؤُلاءِ بَنُو عِيسُو الَّذِي هُوَ ادُومُ وَهَؤُلاءِ امَرَاؤُهُمْ.
\par 20 هَؤُلاءِ بَنُو سَعِيرَ الْحُورِيِّ سُكَّانُ الارْضِ: لُوطَانُ وَشُوبَالُ وَصِبْعُونُ وَعَنَى
\par 21 وَدِيشُونُ وَايصَرُ وَدِيشَانُ. هَؤُلاءِ امَرَاءُ الْحُورِيِّينَ بَنُو سَعِيرَ فِي ارْضِ ادُومَ.
\par 22 وَكَانَ ابْنَا لُوطَانَ: حُورِيَ وَهَيْمَامَ. وَكَانَتْ تِمْنَاعُ اخْتَ لُوطَانَ.
\par 23 وَهَؤُلاءِ بَنُو شُوبَالَ: عَلْوَانُ وَمَنَاحَةُ وَعَيْبَالُ وَشَفْوٌ وَاونَامُ.
\par 24 وَهَذَانِ ابْنَا صِبْعُونَ: ايَّةُ وَعَنَى. هَذَا هُوَ عَنَى الَّذِي وَجَدَ الْحَمَائِمَ فِي الْبَرِّيَّةِ اذْ كَانَ يَرْعَى حَمِيرَ صِبْعُونَ ابِيهِ.
\par 25 وَهَذَا ابْنُ عَنَى: دِيشُونُ. وَاهُولِيبَامَةُ هِيَ بِنْتُ عَنَى.
\par 26 وَهَؤُلاءِ بَنُو دِيشَانَ: حَمْدَانُ وَاشْبَانُ وَيِثْرَانُ وَكَرَانُ.
\par 27 هَؤُلاءِ بَنُو ايصَرَ: بَلْهَانُ وَزَعْوَانُ وَعَقَانُ.
\par 28 هَذَانِ ابْنَا دِيشَانَ: عُوصٌ وَارَانُ.
\par 29 هَؤُلاءِ امَرَاءُ الْحُورِيِّينَ: امِيرُ لُوطَانَ وَامِيرُ شُوبَالَ وَامِيرُ سِبْعُونَ وَامِيرُ عَنَى
\par 30 وَامِيرُ دِيشُونَ وَامِيرُ ايصَرَ وَامِيرُ دِيشَانَ. هَؤُلاءِ امَرَاءُ الْحُورِيِّينَ بِامَرَائِهِمْ فِي ارْضِ سَعِيرَ.
\par 31 وَهَؤُلاءِ هُمُ الْمُلُوكُ الَّذِينَ مَلَكُوا فِي ارْضِ ادُومَ قَبْلَمَا مَلَكَ مَلِكٌ لِبَنِي اسْرَائِيلَ.
\par 32 مَلَكَ فِي ادُومَ بَالَعُ بْنُ بَعُورَ. وَكَانَ اسْمُ مَدِينَتِهِ دِنْهَابَةَ.
\par 33 وَمَاتَ بَالَعُ فَمَلَكَ مَكَانَهُ يُوبَابُ بْنُ زَارَحَ مِنْ بُصْرَةَ.
\par 34 وَمَاتَ يُوبَابُ فَمَلَكَ مَكَانَهُ حُوشَامُ مِنْ ارْضِ التَّيْمَانِيِّ.
\par 35 وَمَاتَ حُوشَامُ فَمَلَكَ مَكَانَهُ هَدَادُ بْنُ بَدَادَ الَّذِي كَسَّرَ مِدْيَانَ فِي بِلادِ مُوابَ وَكَانَ اسْمُ مَدِينَتِهِ عَوِيتَ.
\par 36 وَمَاتَ هَدَادُ فَمَلَكَ مَكَانَهُ سَمْلَةُ مِنْ مَسْرِيقَةَ.
\par 37 وَمَاتَ سَمْلَةُ فَمَلَكَ مَكَانَهُ شَاولُ مِنْ رَحُوبُوتَ النَّهْرِ.
\par 38 وَمَاتَ شَاولُ فَمَلَكَ مَكَانَهُ بَعْلُ حَانَانَ بْنُ عَكْبُورَ.
\par 39 وَمَاتَ بَعْلُ حَانَانَ بْنُ عَكْبُورَ فَمَلَكَ مَكَانَهُ هَدَارُ. وَكَانَ اسْمُ مَدِينَتِهِ فَاعُوَ. وَاسْمُ امْرَاتِهِ مَهِيطَبْئِيلُ بِنْتُ مَطْرِدَ بِنْتِ مَاءِ ذَهَبٍ.
\par 40 وَهَذِهِ اسْمَاءُ امَرَاءِ عِيسُو حَسَبَ قَبَائِلِهِمْ وَامَاكِنِهِمْ بِاسْمَائِهِمْ: امِيرُ تِمْنَاعَ وَامِيرُ عَلْوَةَ وَامِيرُ يَتِيتَ
\par 41 وَامِيرُ اهُولِيبَامَةَ وَامِيرُ ايْلَةَ وَامِيرُ فِينُونَ
\par 42 وَامِيرُ قَنَازَ وَامِيرُ تَيْمَانَ وَامِيرُ مِبْصَارَ
\par 43 وَامِيرُ مَجْدِيئِيلَ وَامِيرُ عِيرَامَ. هَؤُلاءِ امَرَاءُ ادُومَ حَسَبَ مَسَاكِنِهِمْ فِي ارْضِ مُلْكِهِمْ. هَذَا هُوَ عِيسُو ابُو ادُومَ.

\chapter{37}

\par 1 وَسَكَنَ يَعْقُوبُ فِي ارْضِ غُرْبَةِ ابِيهِ فِي ارْضِ كَنْعَانَ.
\par 2 هَذِهِ مَوَالِيدُ يَعْقُوبَ: يُوسُفُ اذْ كَانَ ابْنَ سَبْعَ عَشَرَةَ سَنَةً كَانَ يَرْعَى مَعَ اخْوَتِهِ الْغَنَمَ وَهُوَ غُلامٌ عِنْدَ بَنِي بِلْهَةَ وَبَنِي زِلْفَةَ امْرَاتَيْ ابِيهِ. وَاتَى يُوسُفُ بِنَمِيمَتِهِمِ الرَّدِيئَةِ الَى ابِيهِمْ.
\par 3 وَامَّا اسْرَائِيلُ فَاحَبَّ يُوسُفَ اكْثَرَ مِنْ سَائِرِ بَنِيهِ لانَّهُ ابْنُ شَيْخُوخَتِهِ فَصَنَعَ لَهُ قَمِيصا مُلَوَّنا.
\par 4 فَلَمَّا رَاى اخْوَتُهُ انَّ ابَاهُمْ احَبَّهُ اكْثَرَ مِنْ جَمِيعِ اخْوَتِهِ ابْغَضُوهُ وَلَمْ يَسْتَطِيعُوا انْ يُكَلِّمُوهُ بِسَلامٍ.
\par 5 وَحَلُمَ يُوسُفُ حُلْما وَاخْبَرَ اخْوَتَهُ فَازْدَادُوا ايْضا بُغْضا لَهُ.
\par 6 فَقَالَ لَهُمُ: «اسْمَعُوا هَذَا الْحُلْمَ الَّذِي حَلُمْتُ.
\par 7 فَهَا نَحْنُ حَازِمُونَ حُزَما فِي الْحَقْلِ وَاذَا حُزْمَتِي قَامَتْ وَانْتَصَبَتْ فَاحْتَاطَتْ حُزَمُكُمْ وَسَجَدَتْ لِحُزْمَتِي».
\par 8 فَقَالَ لَهُ اخْوَتُهُ: «الَعَلَّكَ تَمْلِكُ عَلَيْنَا مُلْكا امْ تَتَسَلَّطُ عَلَيْنَا تَسَلُّطا؟» وَازْدَادُوا ايْضا بُغْضا لَهُ مِنْ اجْلِ احْلامِهِ وَمِنْ اجْلِ كَلامِهِ.
\par 9 ثُمَّ حَلُمَ ايْضا حُلْما اخَرَ وَقَصَّهُ عَلَى اخْوَتِهِ. فَقَالَ: «انِّي قَدْ حَلُمْتُ حُلْما ايْضا وَاذَا الشَّمْسُ وَالْقَمَرُ وَاحَدَ عَشَرَ كَوْكَبا سَاجِدَةٌ لِي».
\par 10 وَقَصَّهُ عَلَى ابِيهِ وَعَلَى اخْوَتِهِ فَانْتَهَرَهُ ابُوهُ وَقَالَ لَهُ: «مَا هَذَا الْحُلْمُ الَّذِي حَلُمْتَ! هَلْ نَاتِي انَا وَامُّكَ وَاخْوَتُكَ لِنَسْجُدَ لَكَ الَى الارْضِ؟»
\par 11 فَحَسَدَهُ اخْوَتُهُ وَامَّا ابُوهُ فَحَفِظَ الامْرَ.
\par 12 وَمَضَى اخْوَتُهُ لِيَرْعُوا غَنَمَ ابِيهِمْ عِنْدَ شَكِيمَ.
\par 13 فَقَالَ اسْرَائِيلُ لِيُوسُفَ: «الَيْسَ اخْوَتُكَ يَرْعُونَ عِنْدَ شَكِيمَ؟ تَعَالَ فَارْسِلَكَ الَيْهِمْ». فَقَالَ لَهُ: «هَئَنَذَا».
\par 14 فَقَالَ لَهُ: «اذْهَبِ انْظُرْ سَلامَةَ اخْوَتِكَ وَسَلامَةَ الْغَنَمِ وَرُدَّ لِي خَبَرا». فَارْسَلَهُ مِنْ وَطَاءِ حَبْرُونَ فَاتَى الَى شَكِيمَ.
\par 15 فَوَجَدَهُ رَجُلٌ وَاذَا هُوَ ضَالٌّ فِي الْحَقْلِ. فَسَالَهُ الرَّجُلُ: «مَاذَا تَطْلُبُ؟»
\par 16 فَقَالَ: «انَا طَالِبٌ اخْوَتِي. اخْبِرْنِي ايْنَ يَرْعُونَ».
\par 17 فَقَالَ الرَّجُلُ: «قَدِ ارْتَحَلُوا مِنْ هُنَا لانِّي سَمِعْتُهُمْ يَقُولُونَ: لِنَذْهَبْ الَى دُوثَان»َ. فَذَهَبَ يُوسُفُ وَرَاءَ اخْوَتِهِ فَوَجَدَهُمْ فِي دُوثَانَ.
\par 18 فَلَمَّا ابْصَرُوهُ مِنْ بَعِيدٍ قَبْلَمَا اقْتَرَبَ الَيْهِمِ احْتَالُوا لَهُ لِيُمِيتُوهُ.
\par 19 فَقَالَ بَعْضُهُمْ لِبَعْضٍ: «هُوَذَا هَذَا صَاحِبُ الاحْلامِ قَادِمٌ.
\par 20 فَالْانَ هَلُمَّ نَقْتُلْهُ وَنَطْرَحْهُ فِي احْدَى الْابَارِ وَنَقُولُ: وَحْشٌ رَدِيءٌ اكَلَهُ. فَنَرَى مَاذَا تَكُونُ احْلامُهُ».
\par 21 فَسَمِعَ رَاوبَيْنُ وَانْقَذَهُ مِنْ ايْدِيهِمْ وَقَالَ: «لا نَقْتُلُهُ».
\par 22 وَقَالَ لَهُمْ رَاوبَيْنُ: «لا تَسْفِكُوا دَما. اطْرَحُوهُ فِي هَذِهِ الْبِئْرِ الَّتِي فِي الْبَرِّيَّةِ وَلا تَمُدُّوا الَيْهِ يَدا» - لِكَيْ يُنْقِذَهُ مِنْ ايْدِيهِمْ لِيَرُدَّهُ الَى ابِيهِ.
\par 23 فَكَانَ لَمَّا جَاءَ يُوسُفُ الَى اخْوَتِهِ انَّهُمْ خَلَعُوا عَنْهُ قَمِيصَهُ الْمُلَوَّنَ الَّذِي عَلَيْهِ
\par 24 وَاخَذُوهُ وَطَرَحُوهُ فِي الْبِئْرِ. وَامَّا الْبِئْرُ فَكَانَتْ فَارِغَةً لَيْسَ فِيهَا مَاءٌ.
\par 25 ثُمَّ جَلَسُوا لِيَاكُلُوا طَعَاما. فَرَفَعُوا عُيُونَهُمْ وَنَظَرُوا وَاذَا قَافِلَةُ اسْمَاعِيلِيِّينَ مُقْبِلَةٌ مِنْ جِلْعَادَ وَجِمَالُهُمْ حَامِلَةٌ كَثِيرَاءَ وَبَلَسَانا وَلاذَنا ذَاهِبِينَ لِيَنْزِلُوا بِهَا الَى مِصْرَ.
\par 26 فَقَالَ يَهُوذَا لاخْوَتِهِ: «مَا الْفَائِدَةُ انْ نَقْتُلَ اخَانَا وَنُخْفِيَ دَمَهُ؟
\par 27 تَعَالُوا فَنَبِيعَهُ لِلاسْمَاعِيلِيِّينَ وَلا تَكُنْ ايْدِينَا عَلَيْهِ لانَّهُ اخُونَا وَلَحْمُنَا». فَسَمِعَ لَهُ اخْوَتُهُ.
\par 28 وَاجْتَازَ رِجَالٌ مِدْيَانِيُّونَ تُجَّارٌ فَسَحَبُوا يُوسُفَ وَاصْعَدُوهُ مِنَ الْبِئْرِ وَبَاعُوا يُوسُفَ لِلاسْمَاعِيلِيِّينَ بِعِشْرِينَ مِنَ الْفِضَّةِ. فَاتُوا بِيُوسُفَ الَى مِصْرَ.
\par 29 وَرَجَعَ رَاوبَيْنُ الَى الْبِئْرِ وَاذَا يُوسُفُ لَيْسَ فِي الْبِئْرِ فَمَزَّقَ ثِيَابَهُ.
\par 30 ثُمَّ رَجَعَ الَى اخْوَتِهِ وَقَالَ: «الْوَلَدُ لَيْسَ مَوْجُودا وَانَا الَى ايْنَ اذْهَبُ؟»
\par 31 فَاخَذُوا قَمِيصَ يُوسُفَ وَذَبَحُوا تَيْسا مِنَ الْمِعْزَى وَغَمَسُوا الْقَمِيصَ فِي الدَّمِ
\par 32 وَارْسَلُوا الْقَمِيصَ الْمُلَوَّنَ وَاحْضَرُوهُ الَى ابِيهِمْ وَقَالُوا: «وَجَدْنَا هَذَا. حَقِّقْ اقَمِيصُ ابْنِكَ هُوَ امْ لا؟»
\par 33 فَتَحَقَّقَهُ وَقَالَ: «قَمِيصُ ابْنِي. وَحْشٌ رَدِيءٌ اكَلَهُ! افْتُرِسَ يُوسُفُ افْتِرَاسا!»
\par 34 فَمَزَّقَ يَعْقُوبُ ثِيَابَهُ وَوَضَعَ مِسْحا عَلَى حَقَوَيْهِ وَنَاحَ عَلَى ابْنِهِ ايَّاما كَثِيرَةً.
\par 35 فَقَامَ جَمِيعُ بَنِيهِ وَجَمِيعُ بَنَاتِهِ لِيُعَزُّوهُ. فَابَى انْ يَتَعَزَّى وَقَالَ: «انِّي انْزِلُ الَى ابْنِي نَائِحا الَى الْهَاوِيَةِ». وَبَكَى عَلَيْهِ ابُوهُ.
\par 36 وَامَّا الْمِدْيَانِيُّونَ فَبَاعُوهُ فِي مِصْرَ لِفُوطِيفَارَ خَصِيِّ فِرْعَوْنَ رَئِيسِ الشُّرَطِ.

\chapter{38}

\par 1 وَحَدَثَ فِي ذَلِكَ الزَّمَانِ انَّ يَهُوذَا نَزَلَ مِنْ عِنْدِ اخْوَتِهِ وَمَالَ الَى رَجُلٍ عَدُلَّامِيٍّ اسْمُهُ حِيرَةُ.
\par 2 وَنَظَرَ يَهُوذَا هُنَاكَ ابْنَةَ رَجُلٍ كَنْعَانِيٍّ اسْمُهُ شُوعٌ فَاخَذَهَا وَدَخَلَ عَلَيْهَا
\par 3 فَحَبِلَتْ وَوَلَدَتِ ابْنا وَدَعَا اسْمَهُ عِيرا.
\par 4 ثُمَّ حَبِلَتْ ايْضا وَوَلَدَتِ ابْنا وَدَعَتِ اسْمَهُ اونَانَ.
\par 5 ثُمَّ عَادَتْ فَوَلَدَتْ ايْضا ابْنا وَدَعَتِ اسْمَهُ شِيلَةَ. وَكَانَ فِي كَزِيبَ حِينَ وَلَدَتْهُ.
\par 6 وَاخَذَ يَهُوذَا زَوْجَةً لِعِيرٍ بِكْرِهِ اسْمُهَا ثَامَارُ.
\par 7 وَكَانَ عِيرٌ بِكْرُ يَهُوذَا شِرِّيرا فِي عَيْنَيِ الرَّبِّ فَامَاتَهُ الرَّبُّ.
\par 8 فَقَالَ يَهُوذَا لِاونَانَ: «ادْخُلْ عَلَى امْرَاةِ اخِيكَ وَتَزَوَّجْ بِهَا وَاقِمْ نَسْلا لاخِيكَ».
\par 9 فَعَلِمَ اونَانُ انَّ النَّسْلَ لا يَكُونُ لَهُ. فَكَانَ اذْ دَخَلَ عَلَى امْرَاةِ اخِيهِ انَّهُ افْسَدَ عَلَى الارْضِ لِكَيْ لا يُعْطِيَ نَسْلا لاخِيهِ.
\par 10 فَقَبُحَ فِي عَيْنَيِ الرَّبِّ مَا فَعَلَهُ فَامَاتَهُ ايْضا.
\par 11 فَقَالَ يَهُوذَا لِثَامَارَ كَنَّتِهِ: «اقْعُدِي ارْمَلَةً فِي بَيْتِ ابِيكِ حَتَّى يَكْبُرَ شِيلَةُ ابْنِي». لانَّهُ قَالَ: «لَعَلَّهُ يَمُوتُ هُوَ ايْضا كَاخَوَيْهِ». فَمَضَتْ ثَامَارُ وَقَعَدَتْ فِي بَيْتِ ابِيهَا.
\par 12 وَلَمَّا طَالَ الزَّمَانُ مَاتَتِ ابْنَةُ شُوعٍ امْرَاةُ يَهُوذَا. ثُمَّ تَعَزَّى يَهُوذَا فَصَعِدَ الَى جُزَّازِ غَنَمِهِ الَى تِمْنَةَ هُوَ وَحِيرَةُ صَاحِبُهُ الْعَدُلَّامِيُّ.
\par 13 فَاخْبِرَتْ ثَامَارُ: «هُوَذَا حَمُوكِ صَاعِدٌ الَى تِمْنَةَ لِيَجُزَّ غَنَمَهُ».
\par 14 فَخَلَعَتْ عَنْهَا ثِيَابَ تَرَمُّلِهَا وَتَغَطَّتْ بِبُرْقُعٍ وَتَلَفَّفَتْ وَجَلَسَتْ فِي مَدْخَلِ عَيْنَايِمَ الَّتِي عَلَى طَرِيقِ تِمْنَةَ - لانَّهَا رَاتْ انَّ شِيلَةَ قَدْ كَبِرَ وَهِيَ لَمْ تُعْطَ لَهُ زَوْجَةً.
\par 15 فَنَظَرَهَا يَهُوذَا وَحَسِبَهَا زَانِيَةً لانَّهَا كَانَتْ قَدْ غَطَّتْ وَجْهَهَا.
\par 16 فَمَالَ الَيْهَا عَلَى الطَّرِيقِ وَقَالَ: «هَاتِي ادْخُلْ عَلَيْكِ». لانَّهُ لَمْ يَعْلَمْ انَّهَا كَنَّتُهُ. فَقَالَتْ: «مَاذَا تُعْطِينِي لِكَيْ تَدْخُلَ عَلَيَّ؟»
\par 17 فَقَالَ: «انِّي ارْسِلُ جَدْيَ مِعْزَى مِنَ الْغَنَمِ». فَقَالَتْ: «هَلْ تُعْطِينِي رَهْنا حَتَّى تُرْسِلَهُ؟»
\par 18 فَقَالَ: «مَا الرَّهْنُ الَّذِي اعْطِيكِ؟» فَقَالَتْ: «خَاتِمُكَ وَعِصَابَتُكَ وَعَصَاكَ الَّتِي فِي يَدِكَ». فَاعْطَاهَا وَدَخَلَ عَلَيْهَا. فَحَبِلَتْ مِنْهُ.
\par 19 ثُمَّ قَامَتْ وَمَضَتْ وَخَلَعَتْ عَنْهَا بُرْقُعَهَا وَلَبِسَتْ ثِيَابَ تَرَمُّلِهَا.
\par 20 فَارْسَلَ يَهُوذَا جَدْيَ الْمِعْزَى بِيَدِ صَاحِبِهِ الْعَدُلَّامِيِّ لِيَاخُذَ الرَّهْنَ مِنْ يَدِ الْمَرْاةِ فَلَمْ يَجِدْهَا.
\par 21 فَسَالَ اهْلَ مَكَانِهَا: «ايْنَ الزَّانِيَةُ الَّتِي كَانَتْ فِي عَيْنَايِمَ عَلَى الطَّرِيقِ؟» فَقَالُوا: «لَمْ تَكُنْ هَهُنَا زَانِيَةٌ».
\par 22 فَرَجَعَ الَى يَهُوذَا وَقَالَ: «لَمْ اجِدْهَا. وَاهْلُ الْمَكَانِ ايْضا قَالُوا: لَمْ تَكُنْ هَهُنَا زَانِيَةٌ».
\par 23 فَقَالَ يَهُوذَا: «لِتَاخُذْ لِنَفْسِهَا لِئَلَّا نَصِيرَ اهَانَةً. انِّي قَدْ ارْسَلْتُ هَذَا الْجَدْيَ وَانْتَ لَمْ تَجِدْهَا».
\par 24 وَلَمَّا كَانَ نَحْوُ ثَلاثَةِ اشْهُرٍ اخْبِرَ يَهُوذَا وَقِيلَ لَهُ: «قَدْ زَنَتْ ثَامَارُ كَنَّتُكَ. وَهَا هِيَ حُبْلَى ايْضا مِنَ الزِّنَا». فَقَالَ يَهُوذَا: «اخْرِجُوهَا فَتُحْرَقَ».
\par 25 امَّا هِيَ فَلَمَّا اخْرِجَتْ ارْسَلَتْ الَى حَمِيهَا قَائِلَةً: «مِنَ الرَّجُلِ الَّذِي هَذِهِ لَهُ انَا حُبْلَى!» وَقَالَتْ: «حَقِّقْ لِمَنِ الْخَاتِمُ وَالْعِصَابَةُ وَالْعَصَا هَذِهِ».
\par 26 فَتَحَقَّقَهَا يَهُوذَا وَقَالَ: «هِيَ ابَرُّ مِنِّي لانِّي لَمْ اعْطِهَا لِشِيلَةَ ابْنِي». فَلَمْ يَعُدْ يَعْرِفُهَا ايْضا.
\par 27 وَفِي وَقْتِ وِلادَتِهَا اذَا فِي بَطْنِهَا تَوْامَانِ.
\par 28 وَكَانَ فِي وِلادَتِهَا انَّ احَدَهُمَا اخْرَجَ يَدا فَاخَذَتِ الْقَابِلَةُ وَرَبَطَتْ عَلَى يَدِهِ قِرْمِزا قَائِلَةً: «هَذَا خَرَجَ اوَّلا».
\par 29 وَلَكِنْ حِينَ رَدَّ يَدَهُ اذَا اخُوهُ قَدْ خَرَجَ. فَقَالَتْ: «لِمَاذَا اقْتَحَمْتَ؟ عَلَيْكَ اقْتِحَامٌ». فَدُعِيَ اسْمُهُ «فَارِصَ».
\par 30 وَبَعْدَ ذَلِكَ خَرَجَ اخُوهُ الَّذِي عَلَى يَدِهِ الْقِرْمِزُ. فَدُعِيَ اسْمُهُ «زَارَحَ».

\chapter{39}

\par 1 وَامَّا يُوسُفُ فَانْزِلَ الَى مِصْرَ وَاشْتَرَاهُ فُوطِيفَارُ خَصِيُّ فِرْعَوْنَ رَئِيسُ الشُّرَطِ رَجُلٌ مِصْرِيٌّ مِنْ يَدِ الاسْمَاعِيلِيِّينَ الَّذِينَ انْزَلُوهُ الَى هُنَاكَ.
\par 2 وَكَانَ الرَّبُّ مَعَ يُوسُفَ فَكَانَ رَجُلا نَاجِحا. وَكَانَ فِي بَيْتِ سَيِّدِهِ الْمِصْرِيِّ.
\par 3 وَرَاى سَيِّدُهُ انَّ الرَّبَّ مَعَهُ وَانَّ كُلَّ مَا يَصْنَعُ كَانَ الرَّبُّ يُنْجِحُهُ بِيَدِهِ.
\par 4 فَوَجَدَ يُوسُفُ نِعْمَةً فِي عَيْنَيْهِ وَخَدَمَهُ فَوَكَّلَهُ عَلَى بَيْتِهِ وَدَفَعَ الَى يَدِهِ كُلَّ مَا كَانَ لَهُ.
\par 5 وَكَانَ مِنْ حِينَ وَكَّلَهُ عَلَى بَيْتِهِ وَعَلَى كُلِّ مَا كَانَ لَهُ انَّ الرَّبَّ بَارَكَ بَيْتَ الْمِصْرِيِّ بِسَبَبِ يُوسُفَ. وَكَانَتْ بَرَكَةُ الرَّبِّ عَلَى كُلِّ مَا كَانَ لَهُ فِي الْبَيْتِ وَفِي الْحَقْلِ
\par 6 فَتَرَكَ كُلَّ مَا كَانَ لَهُ فِي يَدِ يُوسُفَ. وَلَمْ يَكُنْ مَعَهُ يَعْرِفُ شَيْئا الَّا الْخُبْزَ الَّذِي يَاكُلُ. وَكَانَ يُوسُفُ حَسَنَ الصُّورَةِ وَحَسَنَ الْمَنْظَرِ.
\par 7 وَحَدَثَ بَعْدَ هَذِهِ الامُورِ انَّ امْرَاةَ سَيِّدِهِ رَفَعَتْ عَيْنَيْهَا الَى يُوسُفَ وَقَالَتِ: «اضْطَجِعْ مَعِي».
\par 8 فَابَى وَقَالَ لِامْرَاةِ سَيِّدِهِ: «هُوَذَا سَيِّدِي لا يَعْرِفُ مَعِي مَا فِي الْبَيْتِ وَكُلُّ مَا لَهُ قَدْ دَفَعَهُ الَى يَدِي.
\par 9 لَيْسَ هُوَ فِي هَذَا الْبَيْتِ اعْظَمَ مِنِّي. وَلَمْ يُمْسِكْ عَنِّي شَيْئا غَيْرَكِ لانَّكِ امْرَاتُهُ. فَكَيْفَ اصْنَعُ هَذَا الشَّرَّ الْعَظِيمَ وَاخْطِئُ الَى اللهِ؟»
\par 10 وَكَانَ اذْ كَلَّمَتْ يُوسُفَ يَوْما فَيَوْما انَّهُ لَمْ يَسْمَعْ لَهَا انْ يَضْطَجِعَ بِجَانِبِهَا لِيَكُونَ مَعَهَا.
\par 11 ثُمَّ حَدَثَ نَحْوَ هَذَا الْوَقْتِ انَّهُ دَخَلَ الْبَيْتَ لِيَعْمَلَ عَمَلَهُ وَلَمْ يَكُنْ انْسَانٌ مِنْ اهْلِ الْبَيْتِ هُنَاكَ فِي الْبَيْتِ.
\par 12 فَامْسَكَتْهُ بِثَوْبِهِ قَائِلَةً: «اضْطَجِعْ مَعِي». فَتَرَكَ ثَوْبَهُ فِي يَدِهَا وَهَرَبَ وَخَرَجَ الَى خَارِجٍ.
\par 13 وَكَانَ لَمَّا رَاتْ انَّهُ تَرَكَ ثَوْبَهُ فِي يَدِهَا وَهَرَبَ الَى خَارِجٍ
\par 14 انَّهَا نَادَتْ اهْلَ بَيْتِهَا وَقَالَتْ: «انْظُرُوا! قَدْ جَاءَ الَيْنَا بِرَجُلٍ عِبْرَانِيٍّ لِيُدَاعِبَنَا. دَخَلَ الَيَّ لِيَضْطَجِعَ مَعِي فَصَرَخْتُ بِصَوْتٍ عَظِيمٍ.
\par 15 وَكَانَ لَمَّا سَمِعَ انِّي رَفَعْتُ صَوْتِي وَصَرَخْتُ انَّهُ تَرَكَ ثَوْبَهُ بِجَانِبِي وَهَرَبَ وَخَرَجَ الَى خَارِجٍ».
\par 16 فَوَضَعَتْ ثَوْبَهُ بِجَانِبِهَا حَتَّى جَاءَ سَيِّدُهُ الَى بَيْتِهِ.
\par 17 فَكَلَّمَتْهُ بِمِثْلِ هَذَا الْكَلامِ قَائِلَةً: «دَخَلَ الَيَّ الْعَبْدُ الْعِبْرَانِيُّ الَّذِي جِئْتَ بِهِ الَيْنَا لِيُدَاعِبَنِي.
\par 18 وَكَانَ لَمَّا رَفَعْتُ صَوْتِي وَصَرَخْتُ انَّهُ تَرَكَ ثَوْبَهُ بِجَانِبِي وَهَرَبَ الَى خَارِجٍ».
\par 19 فَكَانَ لَمَّا سَمِعَ سَيِّدُهُ كَلامَ امْرَاتِهِ الَّذِي كَلَّمَتْهُ بِهِ قَائِلَةً: «بِحَسَبِ هَذَا الْكَلامِ صَنَعَ بِي عَبْدُكَ» انَّ غَضَبَهُ حَمِيَ.
\par 20 فَاخَذَ يُوسُفَ وَوَضَعَهُ فِي بَيْتِ السِّجْنِ الْمَكَانِ الَّذِي كَانَ اسْرَى الْمَلِكِ مَحْبُوسِينَ فِيهِ. وَكَانَ هُنَاكَ فِي بَيْتِ السِّجْنِ.
\par 21 وَلَكِنَّ الرَّبَّ كَانَ مَعَ يُوسُفَ وَبَسَطَ الَيْهِ لُطْفا وَجَعَلَ نِعْمَةً لَهُ فِي عَيْنَيْ رَئِيسِ بَيْتِ السِّجْنِ.
\par 22 فَدَفَعَ رَئِيسُ بَيْتِ السِّجْنِ الَى يَدِ يُوسُفَ جَمِيعَ الاسْرَى الَّذِينَ فِي بَيْتِ السِّجْنِ. وَكُلُّ مَا كَانُوا يَعْمَلُونَ هُنَاكَ كَانَ هُوَ الْعَامِلَ.
\par 23 وَلَمْ يَكُنْ رَئِيسُ بَيْتِ السِّجْنِ يَنْظُرُ شَيْئا الْبَتَّةَ مِمَّا فِي يَدِهِ لانَّ الرَّبَّ كَانَ مَعَهُ وَمَهْمَا صَنَعَ كَانَ الرَّبُّ يُنْجِحُهُ.

\chapter{40}

\par 1 وَحَدَثَ بَعْدَ هَذِهِ الامُورِ انَّ سَاقِيَ مَلِكِ مِصْرَ وَالْخَبَّازَ اذْنَبَا الَى سَيِّدِهِمَا مَلِكِ مِصْرَ
\par 2 فَسَخَطَ فِرْعَوْنُ عَلَى خَصِيَّيْهِ: رَئِيسِ السُّقَاةِ وَرَئِيسِ الْخَبَّازِينَ
\par 3 فَوَضَعَهُمَا فِي حَبْسِ بَيْتِ رَئِيسِ الشُّرَطِ فِي بَيْتِ السِّجْنِ الْمَكَانِ الَّذِي كَانَ يُوسُفُ مَحْبُوسا فِيهِ.
\par 4 فَاقَامَ رَئِيسُ الشُّرَطِ يُوسُفَ عِنْدَهُمَا فَخَدَمَهُمَا. وَكَانَا ايَّاما فِي الْحَبْسِ.
\par 5 وَحَلُمَا كِلاهُمَا حُلْما فِي لَيْلَةٍ وَاحِدَةٍ كُلُّ وَاحِدٍ حُلْمَهُ كُلُّ وَاحِدٍ بِحَسَبِ تَعْبِيرِ حُلْمِهِ: سَاقِي مَلِكِ مِصْرَ وَخَبَّازُهُ الْمَحْبُوسَانِ فِي بَيْتِ السِّجْنِ.
\par 6 فَدَخَلَ يُوسُفُ الَيْهِمَا فِي الصَّبَاحِ وَنَظَرَهُمَا وَاذَا هُمَا مُغْتَمَّانِ.
\par 7 فَسَالَ خَصِيَّيْ فِرْعَوْنَ اللَّذَيْنِ مَعَهُ فِي حَبْسِ بَيْتِ سَيِّدِهِ: «لِمَاذَا وَجْهَاكُمَا مُكْمَدَّانِ الْيَوْمَ؟»
\par 8 فَقَالا لَهُ: «حَلُمْنَا حُلْما وَلَيْسَ مَنْ يُعَبِّرُهُ». فَقَالَ لَهُمَا يُوسُفُ: «الَيْسَتْ لِلَّهِ التَّعَابِيرُ؟ قُصَّا عَلَيَّ».
\par 9 فَقَصَّ رَئِيسُ السُّقَاةِ حُلْمَهُ عَلَى يُوسُفَ وَقَالَ لَهُ: «كُنْتُ فِي حُلْمِي وَاذَا كَرْمَةٌ امَامِي.
\par 10 وَفِي الْكَرْمَةِ ثَلاثَةُ قُضْبَانٍ. وَهِيَ اذْ افْرَخَتْ طَلَعَ زَهْرُهَا وَانْضَجَتْ عَنَاقِيدُهَا عِنَبا.
\par 11 وَكَانَتْ كَاسُ فِرْعَوْنَ فِي يَدِي. فَاخَذْتُ الْعِنَبَ وَعَصَرْتُهُ فِي كَاسِ فِرْعَوْنَ وَاعْطَيْتُ الْكَاسَ فِي يَدِ فِرْعَوْنَ».
\par 12 فَقَالَ لَهُ يُوسُفُ: «هَذَا تَعْبِيرُهُ: الثَّلاثَةُ الْقُضْبَانِ هِيَ ثَلاثَةُ ايَّامٍ.
\par 13 فِي ثَلاثَةِ ايَّامٍ ايْضا يَرْفَعُ فِرْعَوْنُ رَاسَكَ وَيَرُدُّكَ الَى مَقَامِكَ فَتُعْطِي كَاسَ فِرْعَوْنَ فِي يَدِهِ كَالْعَادَةِ الاولَى حِينَ كُنْتَ سَاقِيَهُ.
\par 14 وَانَّمَا اذَا ذَكَرْتَنِي عِنْدَكَ حِينَمَا يَصِيرُ لَكَ خَيْرٌ تَصْنَعُ الَيَّ احْسَانا وَتَذْكُرُنِي لِفِرْعَوْنَ وَتُخْرِجُنِي مِنْ هَذَا الْبَيْتِ.
\par 15 لانِّي قَدْ سُرِقْتُ مِنْ ارْضِ الْعِبْرَانِيِّينَ. وَهُنَا ايْضا لَمْ افْعَلْ شَيْئا حَتَّى وَضَعُونِي فِي السِّجْنِ».
\par 16 فَلَمَّا رَاى رَئِيسُ الْخَبَّازِينَ انَّهُ عَبَّرَ جَيِّدا قَالَ لِيُوسُفَ: «كُنْتُ انَا ايْضا فِي حُلْمِي وَاذَا ثَلاثَةُ سِلالِ بَيْضَاءَ عَلَى رَاسِي.
\par 17 وَفِي السَّلِّ الاعْلَى مِنْ جَمِيعِ طَعَامِ فِرْعَوْنَ مِنْ صَنْعَةِ الْخَبَّازِ. وَالطُّيُورُ تَاكُلُهُ مِنَ السَّلِّ عَنْ رَاسِي».
\par 18 فَاجَابَ يُوسُفُ وَقَالَ: «هَذَا تَعْبِيرُهُ: الثَّلاثَةُ السِّلالِ هِيَ ثَلاثَةُ ايَّامٍ.
\par 19 فِي ثَلاثَةِ ايَّامٍ ايْضا يَرْفَعُ فِرْعَوْنُ رَاسَكَ عَنْكَ وَيُعَلِّقُكَ عَلَى خَشَبَةٍ وَتَاكُلُ الطُّيُورُ لَحْمَكَ عَنْكَ».
\par 20 فَحَدَثَ فِي الْيَوْمِ الثَّالِثِ يَوْمِ مِيلادِ فِرْعَوْنَ انَّهُ صَنَعَ وَلِيمَةً لِجَمِيعِ عَبِيدِهِ وَرَفَعَ رَاسَ رَئِيسِ السُّقَاةِ وَرَاسَ رَئِيسِ الْخَبَّازِينَ بَيْنَ عَبِيدِهِ.
\par 21 وَرَدَّ رَئِيسَ السُّقَاةِ الَى سَقْيِهِ. فَاعْطَى الْكَاسَ فِي يَدِ فِرْعَوْنَ.
\par 22 وَامَّا رَئِيسُ الْخَبَّازِينَ فَعَلَّقَهُ كَمَا عَبَّرَ لَهُمَا يُوسُفُ.
\par 23 وَلَكِنْ لَمْ يَذْكُرْ رَئِيسُ السُّقَاةِ يُوسُفَ بَلْ نَسِيَهُ.

\chapter{41}

\par 1 وَحَدَثَ مِنْ بَعْدِ سَنَتَيْنِ مِنَ الزَّمَانِ انَّ فِرْعَوْنَ رَاى حُلْما وَاذَا هُوَ وَاقِفٌ عِنْدَ النَّهْرِ.
\par 2 وَهُوَذَا سَبْعُ بَقَرَاتٍ طَالِعَةٍ مِنَ النَّهْرِ حَسَنَةِ الْمَنْظَرِ وَسَمِينَةِ اللَّحْمِ فَارْتَعَتْ فِي رَوْضَةٍ.
\par 3 ثُمَّ هُوَذَا سَبْعُ بَقَرَاتٍ اخْرَى طَالِعَةٍ وَرَاءَهَا مِنَ النَّهْرِ قَبِيحَةِ الْمَنْظَرِ وَرَقِيقَةِ اللَّحْمِ. فَوَقَفَتْ بِجَانِبِ الْبَقَرَاتِ الاولَى عَلَى شَاطِئِ النَّهْرِ.
\par 4 فَاكَلَتِ الْبَقَرَاتُ الْقَبِيحَةُ الْمَنْظَرِ وَالرَّقِيقَةُ اللَّحْمِ الْبَقَرَاتِ السَّبْعَ الْحَسَنَةَ الْمَنْظَرِ وَالسَّمِينَةَ. وَاسْتَيْقَظَ فِرْعَوْنُ.
\par 5 ثُمَّ نَامَ فَحَلُمَ ثَانِيَةً. وَهُوَذَا سَبْعُ سَنَابِلَ طَالِعَةٍ فِي سَاقٍ وَاحِدٍ سَمِينَةٍ وَحَسَنَةٍ.
\par 6 ثُمَّ هُوَذَا سَبْعُ سَنَابِلَ رَقِيقَةٍ وَمَلْفُوحَةٍ بِالرِّيحِ الشَّرْقِيَّةِ نَابِتَةٍ وَرَاءَهَا.
\par 7 فَابْتَلَعَتِ السَّنَابِلُ الرَّقِيقَةُ السَّنَابِلَ السَّبْعَ السَّمِينَةَ الْمُمْتَلِئَةَ. وَاسْتَيْقَظَ فِرْعَوْنُ وَاذَا هُوَ حُلْمٌ.
\par 8 وَكَانَ فِي الصَّبَاحِ انَّ نَفْسَهُ انْزَعَجَتْ فَارْسَلَ وَدَعَا جَمِيعَ سَحَرَةِ مِصْرَ وَجَمِيعَ حُكَمَائِهَا وَقَصَّ عَلَيْهِمْ فِرْعَوْنُ حُلْمَهُ. فَلَمْ يَكُنْ مَنْ يُعَبِّرُهُ لِفِرْعَوْنَ.
\par 9 ثُمَّ قَالَ رَئِيسُ السُّقَاةِ لِفِرْعَوْنَ: «انَا اتَذَكَّرُ الْيَوْمَ خَطَايَايَ.
\par 10 فِرْعَوْنُ سَخَطَ عَلَى عَبْدَيْهِ فَجَعَلَنِي فِي حَبْسِ بَيْتِ رَئِيسِ الشُّرَطِ انَا وَرَئِيسَ الْخَبَّازِينَ.
\par 11 فَحَلُمْنَا حُلْما فِي لَيْلَةٍ وَاحِدَةٍ انَا وَهُوَ. حَلُمْنَا كُلُّ وَاحِدٍ بِحَسَبِ تَعْبِيرِ حُلْمِهِ.
\par 12 وَكَانَ هُنَاكَ مَعَنَا غُلامٌ عِبْرَانِيٌّ عَبْدٌ لِرَئِيسِ الشُّرَطِ فَقَصَصْنَا عَلَيْهِ فَعَبَّرَ لَنَا حُلْمَيْنَا. عَبَّرَ لِكُلِّ وَاحِدٍ بِحَسَبِ حُلْمِهِ.
\par 13 وَكَمَا عَبَّرَ لَنَا هَكَذَا حَدَثَ. رَدَّنِي انَا الَى مَقَامِي وَامَّا هُوَ فَعَلَّقَهُ».
\par 14 فَارْسَلَ فِرْعَوْنُ وَدَعَا يُوسُفَ فَاسْرَعُوا بِهِ مِنَ السِّجْنِ. فَحَلَقَ وَابْدَلَ ثِيَابَهُ وَدَخَلَ عَلَى فِرْعَوْنَ.
\par 15 فَقَالَ فِرْعَوْنُ لِيُوسُفَ: «حَلُمْتُ حُلْما وَلَيْسَ مَنْ يُعَبِّرُهُ. وَانَا سَمِعْتُ عَنْكَ قَوْلا انَّكَ تَسْمَعُ احْلاما لِتُعَبِّرَهَا».
\par 16 فَاجَابَ يُوسُفُ فِرْعَوْنَ: «لَيْسَ لِي. اللهُ يُجِيبُ بِسَلامَةِ فِرْعَوْنَ».
\par 17 فَقَالَ فِرْعَوْنُ لِيُوسُفَ: «انِّي كُنْتُ فِي حُلْمِي وَاقِفا عَلَى شَاطِئِ النَّهْرِ
\par 18 وَهُوَذَا سَبْعُ بَقَرَاتٍ طَالِعَةٍ مِنَ النَّهْرِ سَمِينَةِ اللَّحْمِ وَحَسَنَةِ الصُّورَةِ. فَارْتَعَتْ فِي رَوْضَةٍ.
\par 19 ثُمَّ هُوَذَا سَبْعُ بَقَرَاتٍ اخْرَى طَالِعَةٍ وَرَاءَهَا مَهْزُولَةٍ وَقَبِيحَةِ الصُّورَةِ جِدّا وَرَقِيقَةِ اللَّحْمِ. لَمْ انْظُرْ فِي كُلِّ ارْضِ مِصْرَ مِثْلَهَا فِي الْقَبَاحَةِ.
\par 20 فَاكَلَتِ الْبَقَرَاتُ الرَّقِيقَةُ وَالْقَبِيحَةُ الْبَقَرَاتِ السَّبْعَ الاولَى السَّمِينَةَ.
\par 21 فَدَخَلَتْ اجْوَافَهَا. وَلَمْ يُعْلَمْ انَّهَا دَخَلَتْ فِي اجْوَافِهَا. فَكَانَ مَنْظَرُهَا قَبِيحا كَمَا فِي الاوَّلِ. وَاسْتَيْقَظْتُ.
\par 22 ثُمَّ رَايْتُ فِي حُلْمِي وَهُوَذَا سَبْعُ سَنَابِلَ طَالِعَةٍ فِي سَاقٍ وَاحِدٍ مُمْتَلِئَةٍ وَحَسَنَةِ.
\par 23 ثُمَّ هُوَذَا سَبْعُ سَنَابِلَ يَابِسَةٍ رَقِيقَةٍ مَلْفُوحَةٍ بِالرِّيحِ الشَّرْقِيَّةِ نَابِتَةٍ وَرَاءَهَا.
\par 24 فَابْتَلَعَتِ السَّنَابِلُ الرَّقِيقَةُ السَّنَابِلَ السَّبْعَ الْحَسَنَةَ. فَقُلْتُ لِلسَّحَرَةِ وَلَمْ يَكُنْ مَنْ يُخْبِرُنِي».
\par 25 فَقَالَ يُوسُفُ لِفِرْعَوْنَ: «حُلْمُ فِرْعَوْنَ وَاحِدٌ. قَدْ اخْبَرَ اللهُ فِرْعَوْنَ بِمَا هُوَ صَانِعٌ.
\par 26 الْبَقَرَاتُ السَّبْعُ الْحَسَنَةُ هِيَ سَبْعُ سِنِينَ. وَالسَّنَابِلُ السَّبْعُ الْحَسَنَةُ هِيَ سَبْعُ سِنِينَ. هُوَ حُلْمٌ وَاحِدٌ.
\par 27 وَالْبَقَرَاتُ السَّبْعُ الرَّقِيقَةُ الْقَبِيحَةُ الَّتِي طَلَعَتْ وَرَاءَهَا هِيَ سَبْعُ سِنِينَ. وَالسَّنَابِلُ السَّبْعُ الْفَارِغَةُ الْمَلْفُوحَةُ بِالرِّيحِ الشَّرْقِيَّةِ تَكُونُ سَبْعَ سِنِينَ جُوعا.
\par 28 هُوَ الامْرُ الَّذِي كَلَّمْتُ بِهِ فِرْعَوْنَ. قَدْ اظْهَرَ اللهُ لِفِرْعَوْنَ مَا هُوَ صَانِعٌ.
\par 29 هُوَذَا سَبْعُ سِنِينَ قَادِمَةٌ شَبَعا عَظِيما فِي كُلِّ ارْضِ مِصْرَ.
\par 30 ثُمَّ تَقُومُ بَعْدَهَا سَبْعُ سِنِينَ جُوعا فَيُنْسَى كُلُّ الشَّبَعِ فِي ارْضِ مِصْرَ وَيُتْلِفُ الْجُوعُ الارْضَ.
\par 31 وَلا يُعْرَفُ الشَّبَعُ فِي الارْضِ مِنْ اجْلِ ذَلِكَ الْجُوعِ بَعْدَهُ لانَّهُ يَكُونُ شَدِيدا جِدّا.
\par 32 وَامَّا عَنْ تَكْرَارِ الْحُلْمِ عَلَى فِرْعَوْنَ مَرَّتَيْنِ فَلانَّ الامْرَ مُقَرَّرٌ مِنْ قِبَلِ اللهِ وَاللهُ مُسْرِعٌ لِيَصْنَعَهُ.
\par 33 «فَالْانَ لِيَنْظُرْ فِرْعَوْنُ رَجُلا بَصِيرا وَحَكِيما وَيَجْعَلْهُ عَلَى ارْضِ مِصْرَ.
\par 34 يَفْعَلْ فِرْعَوْنُ فَيُوَكِّلْ نُظَّارا عَلَى الارْضِ وَيَاخُذْ خُمْسَ غَلَّةِ ارْضِ مِصْرَ فِي سَبْعِ سِنِي الشَّبَعِ
\par 35 فَيَجْمَعُونَ جَمِيعَ طَعَامِ هَذِهِ السِّنِينَ الْجَيِّدَةِ الْقَادِمَةِ وَيَخْزِنُونَ قَمْحا تَحْتَ يَدِ فِرْعَوْنَ طَعَاما. فِي الْمُدُنِ وَيَحْفَظُونَهُ.
\par 36 فَيَكُونُ الطَّعَامُ ذَخِيرَةً لِلارْضِ لِسَبْعِ سِنِي الْجُوعِ الَّتِي تَكُونُ فِي ارْضِ مِصْرَ. فَلا تَنْقَرِضُ الارْضُ بِالْجُوعِ».
\par 37 فَحَسُنَ الْكَلامُ فِي عَيْنَيْ فِرْعَوْنَ وَفِي عُيُونِ جَمِيعِ عَبِيدِهِ.
\par 38 فَقَالَ فِرْعَوْنُ لِعَبِيدِهِ: «هَلْ نَجِدُ مِثْلَ هَذَا رَجُلا فِيهِ رُوحُ اللهِ؟»
\par 39 ثُمَّ قَالَ فِرْعَوْنُ لِيُوسُفَ: «بَعْدَ مَا اعْلَمَكَ اللهُ كُلَّ هَذَا لَيْسَ بَصِيرٌ وَحَكِيمٌ مِثْلَكَ.
\par 40 انْتَ تَكُونُ عَلَى بَيْتِي وَعَلَى فَمِكَ يُقَبِّلُ جَمِيعُ شَعْبِي. الَّا انَّ الْكُرْسِيَّ اكُونُ فِيهِ اعْظَمَ مِنْكَ».
\par 41 ثُمَّ قَالَ فِرْعَوْنُ لِيُوسُفَ: «انْظُرْ. قَدْ جَعَلْتُكَ عَلَى كُلِّ ارْضِ مِصْرَ».
\par 42 وَخَلَعَ فِرْعَوْنُ خَاتِمَهُ مِنْ يَدِهِ وَجَعَلَهُ فِي يَدِ يُوسُفَ وَالْبَسَهُ ثِيَابَ بُوصٍ وَوَضَعَ طَوْقَ ذَهَبٍ فِي عُنُقِهِ
\par 43 وَارْكَبَهُ فِي مَرْكَبَتِهِ الثَّانِيَةِ وَنَادُوا امَامَهُ «ارْكَعُوا». وَجَعَلَهُ عَلَى كُلِّ ارْضِ مِصْرَ.
\par 44 وَقَالَ فِرْعَوْنُ لِيُوسُفَ: «انَا فِرْعَوْنُ. فَبِدُونِكَ لا يَرْفَعُ انْسَانٌ يَدَهُ وَلا رِجْلَهُ فِي كُلِّ ارْضِ مِصْرَ».
\par 45 وَدَعَا فِرْعَوْنُ اسْمَ يُوسُفَ «صَفْنَاتَ فَعْنِيحَ». وَاعْطَاهُ اسْنَاتَ بِنْتَ فُوطِي فَارَعَ كَاهِنِ اونَ زَوْجَةً. فَخَرَجَ يُوسُفُ عَلَى ارْضِ مِصْرَ.
\par 46 وَكَانَ يُوسُفُ ابْنَ ثَلاثِينَ سَنَةً لَمَّا وَقَفَ قُدَّامَ فِرْعَوْنَ مَلِكِ مِصْرَ. فَخَرَجَ يُوسُفُ مِنْ لَدُنْ فِرْعَوْنَ وَاجْتَازَ فِي كُلِّ ارْضِ مِصْرَ.
\par 47 وَاثْمَرَتِ الارْضُ فِي سَبْعِ سِنِي الشَّبَعِ بِحُزَمٍ.
\par 48 فَجَمَعَ كُلَّ طَعَامِ السَّبْعِ سِنِينَ الَّتِي كَانَتْ فِي ارْضِ مِصْرَ وَجَعَلَ طَعَاما فِي الْمُدُنِ. طَعَامَ حَقْلِ الْمَدِينَةِ الَّذِي حَوَالَيْهَا جَعَلَهُ فِيهَا.
\par 49 وَخَزَنَ يُوسُفُ قَمْحا كَرَمْلِ الْبَحْرِ كَثِيرا جِدّا حَتَّى تَرَكَ الْعَدَدَ اذْ لَمْ يَكُنْ لَهُ عَدَدٌ.
\par 50 وَوُلِدَ لِيُوسُفَ ابْنَانِ قَبْلَ انْ تَاتِيَ سَنَةُ الْجُوعِ وَلَدَتْهُمَا لَهُ اسْنَاتُ بِنْتُ فُوطِي فَارَعَ كَاهِنِ اونَ.
\par 51 وَدَعَا يُوسُفُ اسْمَ الْبِكْرِ مَنَسَّى قَائِلا: «لانَّ اللهَ انْسَانِي كُلَّ تَعَبِي وَكُلَّ بَيْتِ ابِي».
\par 52 وَدَعَا اسْمَ الثَّانِي افْرَايِمَ قَائِلا: «لانَّ اللهَ جَعَلَنِي مُثْمِرا فِي ارْضِ مَذَلَّتِي».
\par 53 ثُمَّ كَمِلَتْ سَبْعُ سِنِي الشَّبَعِ الَّذِي كَانَ فِي ارْضِ مِصْرَ.
\par 54 وَابْتَدَاتْ سَبْعُ سِنِي الْجُوعِ تَاتِي كَمَا قَالَ يُوسُفُ فَكَانَ جُوعٌ فِي جَمِيعِ الْبُلْدَانِ. وَامَّا جَمِيعُ ارْضِ مِصْرَ فَكَانَ فِيهَا خُبْزٌ.
\par 55 وَلَمَّا جَاعَتْ جَمِيعُ ارْضِ مِصْرَ وَصَرَخَ الشَّعْبُ الَى فِرْعَوْنَ لاجْلِ الْخُبْزِ قَالَ فِرْعَوْنُ لِكُلِّ الْمِصْرِيِّينَ: «اذْهَبُوا الَى يُوسُفَ وَالَّذِي يَقُولُ لَكُمُ افْعَلُوا».
\par 56 وَكَانَ الْجُوعُ عَلَى كُلِّ وَجْهِ الارْضِ. وَفَتَحَ يُوسُفُ جَمِيعَ مَا فِيهِ طَعَامٌ وَبَاعَ لِلْمِصْرِيِّينَ. وَاشْتَدَّ الْجُوعُ فِي ارْضِ مِصْرَ.
\par 57 وَجَاءَتْ كُلُّ الارْضِ الَى مِصْرَ الَى يُوسُفَ لِتَشْتَرِيَ قَمْحا لانَّ الْجُوعَ كَانَ شَدِيدا فِي كُلِّ الارْضِ.

\chapter{42}

\par 1 فَلَمَّا رَاى يَعْقُوبُ انَّهُ يُوجَدُ قَمْحٌ فِي مِصْرَ قَالَ يَعْقُوبُ لِبَنِيهِ: «لِمَاذَا تَنْظُرُونَ بَعْضُكُمْ الَى بَعْضٍ؟
\par 2 انِّي قَدْ سَمِعْتُ انَّهُ يُوجَدُ قَمْحٌ فِي مِصْرَ. انْزِلُوا الَى هُنَاكَ وَاشْتَرُوا لَنَا مِنْ هُنَاكَ لِنَحْيَا وَلا نَمُوتَ».
\par 3 فَنَزَلَ عَشَرَةٌ مِنْ اخْوَةِ يُوسُفَ لِيَشْتَرُوا قَمْحا مِنْ مِصْرَ.
\par 4 وَامَّا بِنْيَامِينُ اخُو يُوسُفَ فَلَمْ يُرْسِلْهُ يَعْقُوبُ مَعَ اخْوَتِهِ لانَّهُ قَالَ: «لَعَلَّهُ تُصِيبُهُ اذِيَّةٌ».
\par 5 فَاتَى بَنُو اسْرَائِيلَ لِيَشْتَرُوا بَيْنَ الَّذِينَ اتُوا لانَّ الْجُوعَ كَانَ فِي ارْضِ كَنْعَانَ.
\par 6 وَكَانَ يُوسُفُ هُوَ الْمُسَلَّطَ عَلَى الارْضِ وَهُوَ الْبَائِعَ لِكُلِّ شَعْبِ الارْضِ. فَاتَى اخْوَةُ يُوسُفَ وَسَجَدُوا لَهُ بِوُجُوهِهِمْ الَى الارْضِ.
\par 7 وَلَمَّا نَظَرَ يُوسُفُ اخْوَتَهُ عَرَفَهُمْ فَتَنَكَّرَ لَهُمْ وَتَكَلَّمَ مَعَهُمْ بِجَفَاءٍ وَقَالَ لَهُمْ: «مِنْ ايْنَ جِئْتُمْ؟» فَقَالُوا: «مِنْ ارْضِ كَنْعَانَ لِنَشْتَرِيَ طَعَاما».
\par 8 وَعَرَفَ يُوسُفُ اخْوَتَهُ وَامَّا هُمْ فَلَمْ يَعْرِفُوهُ.
\par 9 فَتَذَكَّرَ يُوسُفُ الاحْلامَ الَّتِي حَلُمَ عَنْهُمْ وَقَالَ لَهُمْ: «جَوَاسِيسُ انْتُمْ! لِتَرُوا عَوْرَةَ الارْضِ جِئْتُمْ!»
\par 10 فَقَالُوا لَهُ: «لا يَا سَيِّدِي. بَلْ عَبِيدُكَ جَاءُوا لِيَشْتَرُوا طَعَاما.
\par 11 نَحْنُ جَمِيعُنَا بَنُو رَجُلٍ وَاحِدٍ. نَحْنُ امَنَاءُ. لَيْسَ عَبِيدُكَ جَوَاسِيسَ».
\par 12 فَقَالَ لَهُمْ: «كَلَّا! بَلْ لِتَرُوا عَوْرَةَ الارْضِ جِئْتُمْ».
\par 13 فَقَالُوا: «عَبِيدُكَ اثْنَا عَشَرَ اخا. نَحْنُ بَنُو رَجُلٍ وَاحِدٍ فِي ارْضِ كَنْعَانَ. وَهُوَذَا الصَّغِيرُ عِنْدَ ابِينَا الْيَوْمَ وَالْوَاحِدُ مَفْقُودٌ».
\par 14 فَقَالَ لَهُمْ يُوسُفُ: «ذَلِكَ مَا كَلَّمْتُكُمْ بِهِ قَائِلا: جَوَاسِيسُ انْتُمْ.
\par 15 بِهَذَا تُمْتَحَنُونَ. وَحَيَاةِ فِرْعَوْنَ لا تَخْرُجُونَ مِنْ هُنَا الَّا بِمَجِيءِ اخِيكُمُ الصَّغِيرِ الَى هُنَا.
\par 16 ارْسِلُوا مِنْكُمْ وَاحِدا لِيَجِيءَ بِاخِيكُمْ وَانْتُمْ تُحْبَسُونَ فَيُمْتَحَنَ كَلامُكُمْ هَلْ عِنْدَكُمْ صِدْقٌ. وَالَّا فَوَحَيَاةِ فِرْعَوْنَ انَّكُمْ لَجَوَاسِيسُ!»
\par 17 فَجَمَعَهُمْ الَى حَبْسٍ ثَلاثَةَ ايَّامٍ.
\par 18 ثُمَّ قَالَ لَهُمْ يُوسُفُ فِي الْيَوْمِ الثَّالِثِ: «افْعَلُوا هَذَا وَاحْيُوا. انَا خَائِفُ اللهِ.
\par 19 انْ كُنْتُمْ امَنَاءَ فَلْيُحْبَسْ اخٌ وَاحِدٌ مِنْكُمْ فِي بَيْتِ حَبْسِكُمْ وَانْطَلِقُوا انْتُمْ وَخُذُوا قَمْحا لِمَجَاعَةِ بُيُوتِكُمْ.
\par 20 وَاحْضِرُوا اخَاكُمُ الصَّغِيرَ الَيَّ فَيَتَحَقَّقَ كَلامُكُمْ وَلا تَمُوتُوا». فَفَعَلُوا هَكَذَا.
\par 21 وَقَالُوا بَعْضُهُمْ لِبَعْضٍ: «حَقّا انَّنَا مُذْنِبُونَ الَى اخِينَا الَّذِي رَايْنَا ضِيقَةَ نَفْسِهِ لَمَّا اسْتَرْحَمَنَا وَلَمْ نَسْمَعْ. لِذَلِكَ جَاءَتْ عَلَيْنَا هَذِهِ الضِّيقَةُ».
\par 22 فَاجَابَهُمْ رَاوبَيْنُ: «الَمْ اكَلِّمْكُمْ قَائِلا: لا تَاثَمُوا بِالْوَلَدِ وَانْتُمْ لَمْ تَسْمَعُوا؟ فَهُوَذَا دَمُهُ يُطْلَبُ».
\par 23 وَهُمْ لَمْ يَعْلَمُوا انَّ يُوسُفَ فَاهِمٌ؛ لانَّ التُّرْجُمَانَ كَانَ بَيْنَهُمْ.
\par 24 فَتَحَوَّلَ عَنْهُمْ وَبَكَى. ثُمَّ رَجَعَ الَيْهِمْ وَكَلَّمَهُمْ وَاخَذَ مِنْهُمْ شَمْعُونَ وَقَيَّدَهُ امَامَ عُيُونِهِمْ.
\par 25 ثُمَّ امَرَ يُوسُفُ انْ تُمْلَا اوْعِيَتُهُمْ قَمْحا وَتُرَدَّ فِضَّةُ كُلِّ وَاحِدٍ الَى عِدْلِهِ وَانْ يُعْطَوْا زَادا لِلطَّرِيقِ. فَفُعِلَ لَهُمْ هَكَذَا.
\par 26 فَحَمَلُوا قَمْحَهُمْ عَلَى حَمِيرِهِمْ وَمَضُوا مِنْ هُنَاكَ.
\par 27 فَلَمَّا فَتَحَ احَدُهُمْ عِدْلَهُ لِيُعْطِيَ عَلِيقا لِحِمَارِهِ فِي الْمَنْزِلِ رَاى فِضَّتَهُ وَاذَا هِيَ فِي فَمِ عِدْلِهِ.
\par 28 فَقَالَ لاخْوَتِهِ: «رُدَّتْ فِضَّتِي وَهَا هِيَ فِي عِدْلِي». فَطَارَتْ قُلُوبُهُمْ وَارْتَعَدُوا بَعْضُهُمْ فِي بَعْضٍ قَائِلِينَ: «مَا هَذَا الَّذِي صَنَعَهُ اللهُ بِنَا؟».
\par 29 فَجَاءُوا الَى يَعْقُوبَ ابِيهِمْ الَى ارْضِ كَنْعَانَ وَاخْبَرُوهُ بِكُلِّ مَا اصَابَهُمْ قَائِلِينَ:
\par 30 «تَكَلَّمَ مَعَنَا الرَّجُلُ سَيِّدُ الارْضِ بِجَفَاءٍ وَحَسِبَنَا جَوَاسِيسَ الارْضِ.
\par 31 فَقُلْنَا لَهُ: نَحْنُ امَنَاءُ. لَسْنَا جَوَاسِيسَ.
\par 32 نَحْنُ اثْنَا عَشَرَ اخا بَنُو ابِينَا. الْوَاحِدُ مَفْقُودٌ وَالصَّغِيرُ الْيَوْمَ عِنْدَ ابِينَا فِي ارْضِ كَنْعَانَ.
\par 33 فَقَالَ لَنَا الرَّجُلُ سَيِّدُ الارْضِ: بِهَذَا اعْرِفُ انَّكُمْ امَنَاءُ. دَعُوا اخا وَاحِدا مِنْكُمْ عِنْدِي وَخُذُوا لِمَجَاعَةِ بُيُوتِكُمْ وَانْطَلِقُوا.
\par 34 وَاحْضِرُوا اخَاكُمُ الصَّغِيرَ الَيَّ فَاعْرِفَ انَّكُمْ لَسْتُمْ جَوَاسِيسَ بَلْ انَّكُمْ امَنَاءُ فَاعْطِيَكُمْ اخَاكُمْ وَتَتَّجِرُونَ فِي الارْضِ».
\par 35 وَاذْ كَانُوا يُفَرِّغُونَ عِدَالَهُمْ اذَا صُرَّةُ فِضَّةِ كُلِّ وَاحِدٍ فِي عِدْلِهِ. فَلَمَّا رَاوْا صُرَرَ فِضَّتِهِمْ هُمْ وَابُوهُمْ خَافُوا.
\par 36 فَقَالَ لَهُمْ يَعْقُوبُ: «اعْدَمْتُمُونِي الاوْلادَ! يُوسُفُ مَفْقُودٌ وَشَمْعُونُ مَفْقُودٌ وَبِنْيَامِينُ تَاخُذُونَهُ! صَارَ كُلُّ هَذَا عَلَيَّ!»
\par 37 وَقَالَ رَاوبَيْنُ لابِيهِ: «اقْتُلِ ابْنَيَّ انْ لَمْ اجِئْ بِهِ الَيْكَ. سَلِّمْهُ بِيَدِي وَانَا ارُدُّهُ الَيْكَ».
\par 38 فَقَالَ: «لا يَنْزِلُ ابْنِي مَعَكُمْ لانَّ اخَاهُ قَدْ مَاتَ وَهُوَ وَحْدَهُ بَاقٍ. فَانْ اصَابَتْهُ اذِيَّةٌ فِي الطَّرِيقِ الَّتِي تَذْهَبُونَ فِيهَا تُنْزِلُونَ شَيْبَتِي بِحُزْنٍ الَى الْهَاوِيَةِ».

\chapter{43}

\par 1 وَكَانَ الْجُوعُ شَدِيدا فِي الارْضِ.
\par 2 وَحَدَثَ لَمَّا فَرَغُوا مِنْ اكْلِ الْقَمْحِ الَّذِي جَاءُوا بِهِ مِنْ مِصْرَ انَّ ابَاهُمْ قَالَ لَهُمُ: «ارْجِعُوا اشْتَرُوا لَنَا قَلِيلا مِنَ الطَّعَامِ».
\par 3 فَقَالَ لَهُ يَهُوذَا: «انَّ الرَّجُلَ قَدْ اشْهَدَ عَلَيْنَا قَائِلا: لا تَرُونَ وَجْهِي بِدُونِ انْ يَكُونَ اخُوكُمْ مَعَكُمْ.
\par 4 انْ كُنْتَ تُرْسِلُ اخَانَا مَعَنَا نَنْزِلُ وَنَشْتَرِي لَكَ طَعَاما.
\par 5 وَلَكِنْ انْ كُنْتَ لا تُرْسِلُهُ لا نَنْزِلُ. لانَّ الرَّجُلَ قَالَ لَنَا: لا تَرُونَ وَجْهِي بِدُونِ انْ يَكُونَ اخُوكُمْ مَعَكُمْ».
\par 6 فَقَالَ اسْرَائِيلُ: «لِمَاذَا اسَاتُمْ الَيَّ حَتَّى اخْبَرْتُمُ الرَّجُلَ انَّ لَكُمْ اخا ايْضا؟»
\par 7 فَقَالُوا: «انَّ الرَّجُلَ قَدْ سَالَ عَنَّا وَعَنْ عَشِيرَتِنَا قَائِلا: هَلْ ابُوكُمْ حَيٌّ بَعْدُ؟ هَلْ لَكُمْ اخٌ؟ فَاخْبَرْنَاهُ بِحَسَبِ هَذَا الْكَلامِ. هَلْ كُنَّا نَعْلَمُ انَّهُ يَقُولُ انْزِلُوا بِاخِيكُمْ؟».
\par 8 وَقَالَ يَهُوذَا لاسْرَائِيلَ ابِيهِ: «ارْسِلِ الْغُلامَ مَعِي لِنَقُومَ وَنَذْهَبَ وَنَحْيَا وَلا نَمُوتَ نَحْنُ وَانْتَ وَاوْلادُنَا جَمِيعا.
\par 9 انَا اضْمَنُهُ. مِنْ يَدِي تَطْلُبُهُ. انْ لَمْ اجِئْ بِهِ الَيْكَ وَاوقِفْهُ قُدَّامَكَ اصِرْ مُذْنِبا الَيْكَ كُلَّ الايَّامِ.
\par 10 لانَّنَا لَوْ لَمْ نَتَوَانَ لَكُنَّا قَدْ رَجَعْنَا الْانَ مَرَّتَيْنِ».
\par 11 فَقَالَ لَهُمْ اسْرَائِيلُ ابُوهُمْ: «انْ كَانَ هَكَذَا فَافْعَلُوا هَذَا: خُذُوا مِنْ افْخَرِ جَنَى الارْضِ فِي اوْعِيَتِكُمْ وَانْزِلُوا لِلرَّجُلِ هَدِيَّةً. قَلِيلا مِنَ الْبَلَسَانِ وَقَلِيلا مِنَ الْعَسَلِ وَكَثِيرَاءَ وَلاذَنا وَفُسْتُقا وَلَوْزا.
\par 12 وَخُذُوا فِضَّةً اخْرَى فِي ايَادِيكُمْ. وَالْفِضَّةَ الْمَرْدُودَةَ فِي افْوَاهِ عِدَالِكُمْ رُدُّوهَا فِي ايَادِيكُمْ. لَعَلَّهُ كَانَ سَهْوا.
\par 13 وَخُذُوا اخَاكُمْ وَقُومُوا ارْجِعُوا الَى الرَّجُلِ.
\par 14 وَاللهُ الْقَدِيرُ يُعْطِيكُمْ رَحْمَةً امَامَ الرَّجُلِ حَتَّى يُطْلِقَ لَكُمْ اخَاكُمُ الْاخَرَ وَبِنْيَامِينَ. وَانَا اذَا عَدِمْتُ الاوْلادَ عَدِمْتُهُمْ».
\par 15 فَاخَذَ الرِّجَالُ هَذِهِ الْهَدِيَّةَ وَاخَذُوا ضِعْفَ الْفِضَّةِ فِي ايَادِيهِمْ وَبِنْيَامِينَ وَقَامُوا وَنَزَلُوا الَى مِصْرَ وَوَقَفُوا امَامَ يُوسُفَ.
\par 16 فَلَمَّا رَاى يُوسُفُ بِنْيَامِينَ مَعَهُمْ قَالَ لِلَّذِي عَلَى بَيْتِهِ: «ادْخِلِ الرِّجَالَ الَى الْبَيْتِ وَاذْبَحْ ذَبِيحَةً وَهَيِّئْ لانَّ الرِّجَالَ يَاكُلُونَ مَعِي عَِنْدَ الظُّهْرِ».
\par 17 فَفَعَلَ الرَّجُلُ كَمَا قَالَ يُوسُفُ. وَادْخَلَ الرَّجُلُ الرِّجَالَ الَى بَيْتِ يُوسُفَ.
\par 18 فَخَافَ الرِّجَالُ اذْ ادْخِلُوا الَى بَيْتِ يُوسُفَ وَقَالُوا: «لِسَبَبِ الْفِضَّةِ الَّتِي رَجَعَتْ اوَّلا فِي عِدَالِنَا نَحْنُ قَدْ ادْخِلْنَا لِيَهْجِمَ عَلَيْنَا وَيَقَعَ بِنَا وَيَاخُذَنَا عَبِيدا وَحَمِيرَنَا».
\par 19 فَتَقَدَّمُوا الَى الرَّجُلِ الَّذِي عَلَى بَيْتِ يُوسُفَ وَكَلَّمُوهُ فِي بَابِ الْبَيْتِ.
\par 20 وَقَالُوا: «اسْتَمِعْ يَا سَيِّدِي. انَّنَا قَدْ نَزَلْنَا اوَّلا لِنَشْتَرِيَ طَعَاما.
\par 21 وَكَانَ لَمَّا اتَيْنَا الَى الْمَنْزِلِ انَّنَا فَتَحْنَا عِدَالَنَا وَاذَا فِضَّةُ كُلِّ وَاحِدٍ فِي فَمِ عِدْلِهِ. فِضَّتُنَا بِوَزْنِهَا. فَقَدْ رَدَدْنَاهَا فِي ايَادِينَا.
\par 22 وَانْزَلْنَا فِضَّةً اخْرَى فِي ايَادِينَا لِنَشْتَرِيَ طَعَاما. لا نَعْلَمُ مَنْ وَضَعَ فِضَّتَنَا فِي عِدَالِنَا».
\par 23 فَقَالَ: «سَلامٌ لَكُمْ. لا تَخَافُوا. الَهُكُمْ وَالَهُ ابِيكُمْ اعْطَاكُمْ كَنْزا فِي عِدَالِكُمْ. فِضَّتُكُمْ وَصَلَتْ الَيَّ». ثُمَّ اخْرَجَ الَيْهِمْ شَمْعُونَ.
\par 24 وَادْخَلَ الرَّجُلُ الرِّجَالَ الَى بَيْتِ يُوسُفَ وَاعْطَاهُمْ مَاءً لِيَغْسِلُوا ارْجُلَهُمْ وَاعْطَى عَلِيقا لِحَمِيرِهِمْ.
\par 25 وَهَيَّاوا الْهَدِيَّةَ الَى انْ يَجِيءَ يُوسُفُ عَِنْدَ الظُّهْرِ. لانَّهُمْ سَمِعُوا انَّهُمْ هُنَاكَ يَاكُلُونَ طَعَاما.
\par 26 فَلَمَّا جَاءَ يُوسُفُ الَى الْبَيْتِ احْضَرُوا الَيْهِ الْهَدِيَّةَ الَّتِي فِي ايَادِيهِمْ الَى الْبَيْتِ وَسَجَدُوا لَهُ الَى الارْضِ.
\par 27 فَسَالَ عَنْ سَلامَتِهِمْ وَقَالَ: «اسَالِمٌ ابُوكُمُ الشَّيْخُ الَّذِي قُلْتُمْ عَنْهُ؟ احَيٌّ هُوَ بَعْدُ؟»
\par 28 فَقَالُوا: «عَبْدُكَ ابُونَا سَالِمٌ. هُوَ حَيٌّ بَعْدُ». وَخَرُّوا وَسَجَدُوا.
\par 29 فَرَفَعَ عَيْنَيْهِ وَنَظَرَ بِنْيَامِينَ اخَاهُ ابْنَ امِّهِ وَقَالَ: «اهَذَا اخُوكُمُ الصَّغِيرُ الَّذِي قُلْتُمْ لِي عَنْهُ؟» ثُمَّ قَالَ: «اللهُ يُنْعِمُ عَلَيْكَ يَا ابْنِي».
\par 30 وَاسْتَعْجَلَ يُوسُفُ لانَّ احْشَاءَهُ حَنَّتْ الَى اخِيهِ وَطَلَبَ مَكَانا لِيَبْكِيَ. فَدَخَلَ الْمَخْدَعَ وَبَكَى هُنَاكَ.
\par 31 ثُمَّ غَسَلَ وَجْهَهُ وَخَرَجَ وَتَجَلَّدَ وَقَالَ: «قَدِّمُوا طَعَاما».
\par 32 فَقَدَّمُوا لَهُ وَحْدَهُ وَلَهُمْ وَحْدَهُمْ وَلِلْمِصْرِيِّينَ الْاكِلِينَ عِنْدَهُ وَحْدَهُمْ لانَّ الْمِصْرِيِّينَ لا يَقْدِرُونَ انْ يَاكُلُوا طَعَاما مَعَ الْعِبْرَانِيِّينَ لانَّهُ رِجْسٌ عِنْدَ الْمِصْرِيِّينَ.
\par 33 فَجَلَسُوا قُدَّامَهُ: الْبِكْرُ بِحَسَبِ بَكُورِيَّتِهِ وَالصَّغِيرُ بِحَسَبِ صِغَرِهِ. فَبُهِتَ الرِّجَالُ بَعْضُهُمْ الَى بَعْضٍ.
\par 34 وَرَفَعَ حِصَصا مِنْ قُدَّامِهِ الَيْهِمْ. فَكَانَتْ حِصَّةُ بِنْيَامِينَ اكْثَرَ مِنْ حِصَصِ جَمِيعِهِمْ خَمْسَةَ اضْعَافٍ. وَشَرِبُوا وَرَوُوا مَعَهُ.

\chapter{44}

\par 1 ثُمَّ امَرَ الَّذِي عَلَى بَيْتِهِ قَائِلا: «امْلَا عِدَالَ الرِّجَالِ طَعَاما حَسَبَ مَا يُطِيقُونَ حِمْلَهُ وَضَعْ فِضَّةَ كُلِّ وَاحِدٍ فِي فَمِ عِدْلِهِ.
\par 2 وَطَاسِي طَاسَ الْفِضَّةِ تَضَعُ فِي فَمِ عِدْلِ الصَّغِيرِ وَثَمَنَ قَمْحِهِ». فَفَعَلَ بِحَسَبِ كَلامِ يُوسُفَ الَّذِي تَكَلَّمَ بِهِ.
\par 3 فَلَمَّا اضَاءَ الصُّبْحُ انْصَرَفَ الرِّجَالُ هُمْ وَحَمِيرُهُمْ.
\par 4 وَلَمَّا كَانُوا قَدْ خَرَجُوا مِنَ الْمَدِينَةِ وَلَمْ يَبْتَعِدُوا قَالَ يُوسُفُ لِلَّذِي عَلَى بَيْتِهِ: «قُمِ اسْعَ وَرَاءَ الرِّجَالِ وَمَتَى ادْرَكْتَهُمْ فَقُلْ لَهُمْ: لِمَاذَا جَازَيْتُمْ شَرّا عِوَضا عَنْ خَيْرٍ؟
\par 5 الَيْسَ هَذَا هُوَ الَّذِي يَشْرَبُ سَيِّدِي فِيهِ؟ وَهُوَ يَتَفَاءَلُ بِهِ. اسَاتُمْ فِي مَا صَنَعْتُمْ».
\par 6 فَادْرَكَهُمْ وَقَالَ لَهُمْ هَذَا الْكَلامَ.
\par 7 فَقَالُوا لَهُ: «لِمَاذَا يَتَكَلَّمُ سَيِّدِي مِثْلَ هَذَا الْكَلامِ؟ حَاشَا لِعَبِيدِكَ انْ يَفْعَلُوا مِثْلَ هَذَا الامْرِ!
\par 8 هُوَذَا الْفِضَّةُ الَّتِي وَجَدْنَا فِي افْوَاهِ عِدَالِنَا رَدَدْنَاهَا الَيْكَ مِنْ ارْضِ كَنْعَانَ. فَكَيْفَ نَسْرِقُ مِنْ بَيْتِ سَيِّدِكَ فِضَّةً اوْ ذَهَبا؟
\par 9 الَّذِي يُوجَدُ مَعَهُ مِنْ عَبِيدِكَ يَمُوتُ وَنَحْنُ ايْضا نَكُونُ عَبِيدا لِسَيِّدِي».
\par 10 فَقَالَ: «نَعَمِ الْانَ بِحَسَبِ كَلامِكُمْ هَكَذَا يَكُونُ. الَّذِي يُوجَدُ مَعَهُ يَكُونُ لِي عَبْدا وَامَّا انْتُمْ فَتَكُونُونَ ابْرِيَاءَ».
\par 11 فَاسْتَعْجَلُوا وَانْزَلُوا كُلُّ وَاحِدٍ عِدْلَهُ الَى الارْضِ وَفَتَحُوا كُلُّ وَاحِدٍ عِدْلَهُ.
\par 12 فَفَتَّشَ مُبْتَدِئا مِنَ الْكَبِيرِ حَتَّى انْتَهَى الَى الصَّغِيرِ. فَوُجِدَ الطَّاسُ فِي عِدْلِ بِنْيَامِينَ.
\par 13 فَمَزَّقُوا ثِيَابَهُمْ وَحَمَّلَ كُلُّ وَاحِدٍ عَلَى حِمَارِهِ وَرَجَعُوا الَى الْمَدِينَةِ.
\par 14 فَدَخَلَ يَهُوذَا وَاخْوَتُهُ الَى بَيْتِ يُوسُفَ وَهُوَ بَعْدُ هُنَاكَ وَوَقَعُوا امَامَهُ عَلَى الارْضِ.
\par 15 فَقَالَ لَهُمْ يُوسُفُ: «مَا هَذَا الْفَِعْلُ الَّذِي فَعَلْتُمْ؟ الَمْ تَعْلَمُوا انَّ رَجُلا مِثْلِي يَتَفَاءَلُ؟»
\par 16 فَقَالَ يَهُوذَا: «مَاذَا نَقُولُ لِسَيِّدِي؟ مَاذَا نَتَكَلَّم وَبِمَاذَا نَتَبَرَّرُ؟ اللهُ قَدْ وَجَدَ اثْمَ عَبِيدِكَ. هَا نَحْنُ عَبِيدٌ لِسَيِّدِي نَحْنُ وَالَّذِي وُجِدَ الطَّاسُ فِي يَدِهِ جَمِيعا».
\par 17 فَقَالَ: «حَاشَا لِي انْ افْعَلَ هَذَا! الرَّجُلُ الَّذِي وُجِدَ الطَّاسُ فِي يَدِهِ هُوَ يَكُونُ لِي عَبْدا وَامَّا انْتُمْ فَاصْعَدُوا بِسَلامٍ الَى ابِيكُمْ».
\par 18 ثُمَّ تَقَدَّمَ الَيْهِ يَهُوذَا وَقَالَ: «اسْتَمِعْ يَا سَيِّدِي. لِيَتَكَلَّمْ عَبْدُكَ كَلِمَةً فِي اذُنَيْ سَيِّدِي وَلا يَحْمَ غَضَبُكَ عَلَى عَبْدِكَ لانَّكَ مِثْلُ فِرْعَوْنَ.
\par 19 سَيِّدِي سَالَ عَبِيدَهُ: هَلْ لَكُمْ ابٌ اوْ اخٌ؟
\par 20 فَقُلْنَا لِسَيِّدِي: لَنَا ابٌ شَيْخٌ وَابْنُ شَيْخُوخَةٍ صَغِيرٌ مَاتَ اخُوهُ وَبَقِيَ هُوَ وَحْدَهُ لِامِّهِ وَابُوهُ يُحِبُّهُ.
\par 21 فَقُلْتَ لِعَبِيدِكَ: انْزِلُوا بِهِ الَيَّ فَاجْعَلَ نَظَرِي عَلَيْهِ.
\par 22 فَقُلْنَا لِسَيِّدِي: لا يَقْدِرُ الْغُلامُ انْ يَتْرُكَ ابَاهُ. وَانْ تَرَكَ ابَاهُ يَمُوتُ.
\par 23 فَقُلْتَ لِعَبِيدِكَ: انْ لَمْ يَنْزِلْ اخُوكُمُ الصَّغِيرُ مَعَكُمْ لا تَعُودُوا تَنْظُرُونَ وَجْهِي.
\par 24 فَكَانَ لَمَّا صَعِدْنَا الَى عَبْدِكَ ابِي انَّنَا اخْبَرْنَاهُ بِكَلامِ سَيِّدِي.
\par 25 ثُمَّ قَالَ ابُونَا: ارْجِعُوا اشْتَرُوا لَنَا قَلِيلا مِنَ الطَّعَامِ.
\par 26 فَقُلْنَا: لا نَقْدِرُ انْ نَنْزِلَ. وَانَّمَا اذَا كَانَ اخُونَا الصَّغِيرُ مَعَنَا نَنْزِلُ لانَّنَا لا نَقْدِرُ انْ نَنْظُرَ وَجْهَ الرَّجُلِ وَاخُونَا الصَّغِيرُ لَيْسَ مَعَنَا.
\par 27 فَقَالَ لَنَا عَبْدُكَ ابِي: انْتُمْ تَعْلَمُونَ انَّ امْرَاتِي وَلَدَتْ لِي اثْنَيْنِ
\par 28 فَخَرَجَ الْوَاحِدُ مِنْ عِنْدِي وَقُلْتُ: انَّمَا هُوَ قَدِ افْتُرِسَ افْتِرَاسا. وَلَمْ انْظُرْهُ الَى الْانَ.
\par 29 فَاذَا اخَذْتُمْ هَذَا ايْضا مِنْ امَامِ وَجْهِي وَاصَابَتْهُ اذِيَّةٌ تُنْزِلُونَ شَيْبَتِي بِشَرٍّ الَى الْهَاوِيَةِ.
\par 30 فَالْانَ مَتَى جِئْتُ الَى عَبْدِكَ ابِي وَالْغُلامُ لَيْسَ مَعَنَا وَنَفْسُهُ مُرْتَبِطَةٌ بِنَفْسِهِ
\par 31 يَكُونُ مَتَى رَاى انَّ الْغُلامَ مَفْقُودٌ انَّهُ يَمُوتُ فَيُنْزِلُ عَبِيدُكَ شَيْبَةَ عَبْدِكَ ابِينَا بِحُزْنٍ الَى الْهَاوِيَةِ
\par 32 لانَّ عَبْدَكَ ضَمِنَ الْغُلامَ لابِي قَائِلا: انْ لَمْ اجِئْ بِهِ الَيْكَ اصِرْ مُذْنِبا الَى ابِي كُلَّ الايَّامِ.
\par 33 فَالْانَ لِيَمْكُثْ عَبْدُكَ عِوَضا عَنِ الْغُلامِ عَبْدا لِسَيِّدِي وَيَصْعَدِ الْغُلامُ مَعَ اخْوَتِهِ.
\par 34 لانِّي كَيْفَ اصْعَدُ الَى ابِي وَالْغُلامُ لَيْسَ مَعِي؟ لِئَلَّا انْظُرَ الشَّرَّ الَّذِي يُصِيبُ ابِي!».

\chapter{45}

\par 1 فَلَمْ يَسْتَطِعْ يُوسُفُ انْ يَضْبِطَ نَفْسَهُ لَدَى جَمِيعِ الْوَاقِفِينَ عِنْدَهُ فَصَرَخَ: «اخْرِجُوا كُلَّ انْسَانٍ عَنِّي!» فَلَمْ يَقِفْ احَدٌ عِنْدَهُ حِينَ عَرَّفَ يُوسُفُ اخْوَتَهُ بِنَفْسِهِ.
\par 2 فَاطْلَقَ صَوْتَهُ بِالْبُكَاءِ. فَسَمِعَ الْمِصْرِيُّونَ وَسَمِعَ بَيْتُ فِرْعَوْنَ.
\par 3 وَقَالَ يُوسُفُ لاخْوَتِهِ: «انَا يُوسُفُ. احَيٌّ ابِي بَعْدُ؟» فَلَمْ يَسْتَطِعْ اخْوَتُهُ انْ يُجِيبُوهُ لانَّهُمُ ارْتَاعُوا مِنْهُ.
\par 4 فَقَالَ يُوسُفُ لاخْوَتِهِ: «تَقَدَّمُوا الَيَّ». فَتَقَدَّمُوا. فَقَالَ: «انَا يُوسُفُ اخُوكُمُ الَّذِي بِعْتُمُوهُ الَى مِصْرَ.
\par 5 وَالْانَ لا تَتَاسَّفُوا وَلا تَغْتَاظُوا لانَّكُمْ بِعْتُمُونِي الَى هُنَا لانَّهُ لِاسْتِبْقَاءِ حَيَاةٍ ارْسَلَنِيَ اللهُ قُدَّامَكُمْ.
\par 6 لانَّ لِلْجُوعِ فِي الارْضِ الْانَ سَنَتَيْنِ. وَخَمْسُ سِنِينَ ايْضا لا تَكُونُ فِيهَا فَلاحَةٌ وَلا حَصَادٌ.
\par 7 فَقَدْ ارْسَلَنِي اللهُ قُدَّامَكُمْ لِيَجْعَلَ لَكُمْ بَقِيَّةً فِي الارْضِ وَلِيَسْتَبْقِيَ لَكُمْ نَجَاةً عَظِيمَةً.
\par 8 فَالْانَ لَيْسَ انْتُمْ ارْسَلْتُمُونِي الَى هُنَا بَلِ اللهُ. وَهُوَ قَدْ جَعَلَنِي ابا لِفِرْعَوْنَ وَسَيِّدا لِكُلِّ بَيْتِهِ وَمُتَسَلِّطا عَلَى كُلِّ ارْضِ مِصْرَ.
\par 9 اسْرِعُوا وَاصْعَدُوا الَى ابِي وَقُولُوا لَهُ: هَكَذَا يَقُولُ ابْنُكَ يُوسُفُ: قَدْ جَعَلَنِيَ اللهُ سَيِّدا لِكُلِّ مِصْرَ. انْزِلْ الَيَّ. لا تَقِفْ.
\par 10 فَتَسْكُنَ فِي ارْضِ جَاسَانَ وَتَكُونَ قَرِيبا مِنِّي انْتَ وَبَنُوكَ وَبَنُو بَنِيكَ وَغَنَمُكَ وَبَقَرُكَ وَكُلُّ مَا لَكَ.
\par 11 وَاعُولُكَ هُنَاكَ لانَّهُ يَكُونُ ايْضا خَمْسُ سِنِينَ جُوعا. لِئَلَّا تَفْتَقِرَ انْتَ وَبَيْتُكَ وَكُلُّ مَا لَكَ.
\par 12 وَهُوَذَا عُِيُونُكُمْ تَرَى وَعَيْنَا اخِي بَنْيَامِينَ انَّ فَمِي هُوَ الَّذِي يُكَلِّمُكُمْ.
\par 13 وَتُخْبِرُونَ ابِي بِكُلِّ مَجْدِي فِي مِصْرَ وَبِكُلِّ مَا رَايْتُمْ وَتَسْتَعْجِلُونَ وَتَنْزِلُونَ بِابِي الَى هُنَا».
\par 14 ثُمَّ وَقَعَ عَلَى عُنُقِ بِنْيَامِينَ اخِيهِ وَبَكَى. وَبَكَى بِنْيَامِينُ عَلَى عُنُقِهِ.
\par 15 وَقَبَّلَ جَمِيعَ اخْوَتِهِ وَبَكَى عَلَيْهِمْ. وَبَعْدَ ذَلِكَ تَكَلَّمَ اخْوَتُهُ مَعَهُ.
\par 16 وَسُمِعَ الْخَبَرُ فِي بَيْتِ فِرْعَوْنَ وَقِيلَ: «جَاءَ اخْوَةُ يُوسُفَ». فَحَسُنَ فِي عَيْنَيْ فِرْعَوْنَ وَفِي عُيُونِ عَبِيدِهِ.
\par 17 فَقَالَ فِرْعَوْنُ لِيُوسُفَ: «قُلْ لاخْوَتِكَ: افْعَلُوا هَذَا. حَمِّلُوا دَوَابَّكُمْ وَانْطَلِقُوا اذْهَبُوا الَى ارْضِ كَنْعَانَ.
\par 18 وَخُذُوا ابَاكُمْ وَبُيُوتَكُمْ وَتَعَالُوا الَيَّ. فَاعْطِيَكُمْ خَيْرَاتِ ارْضِ مِصْرَ وَتَاكُلُوا دَسَمَ الارْضِ.
\par 19 فَانْتَ قَدْ امِرْتَ. افْعَلُوا هَذَا. خُذُوا لَكُمْ مِنْ ارْضِ مِصْرَ عَجَلاتٍ لاوْلادِكُمْ وَنِسَائِكُمْ وَاحْمِلُوا ابَاكُمْ وَتَعَالُوا.
\par 20 وَلا تَحْزَنْ عُيُونُكُمْ عَلَى اثَاثِكُمْ لانَّ خَيْرَاتِ جَمِيعِ ارْضِ مِصْرَ لَكُمْ».
\par 21 فَفَعَلَ بَنُو اسْرَائِيلَ هَكَذَا. وَاعْطَاهُمْ يُوسُفُ عَجَلاتٍ بِحَسَبِ امْرِ فِرْعَوْنَ. وَاعْطَاهُمْ زَادا لِلطَّرِيقِ.
\par 22 وَاعْطَى كُلَّ وَاحِدٍ مِنْهُمْ حُلَلَ ثِيَابٍ. وَامَّا بِنْيَامِينُ فَاعْطَاهُ ثَلاثَ مِئَةٍ مِنَ الْفِضَّةِ وَخَمْسَ حُلَلِ ثِيَابٍ.
\par 23 وَارْسَلَ لابِيهِ عَشَرَةَ حَمِيرٍ حَامِلَةً مِنْ خَيْرَاتِ مِصْرَ وَعَشَرَ اتُنٍ حَامِلَةً حِنْطَةً وَخُبْزا وَطَعَاما لابِيهِ لاجْلِ الطَّرِيقِ.
\par 24 ثُمَّ صَرَفَ اخْوَتَهُ فَانْطَلَقُوا وَقَالَ لَهُمْ: «لا تَتَغَاضَبُوا فِي الطَّرِيقِ».
\par 25 فَصَعِدُوا مِنْ مِصْرَ وَجَاءُوا الَى ارْضِ كَنْعَانَ الَى يَعْقُوبَ ابِيهِمْ
\par 26 وَاخْبَرُوهُ قَائِلِينَ: «يُوسُفُ حَيٌّ بَعْدُ وَهُوَ مُتَسَلِّطٌ عَلَى كُلِّ ارْضِ مِصْرَ!» فَجَمَدَ قَلْبُهُ لانَّهُ لَمْ يُصَدِّقْهُمْ.
\par 27 ثُمَّ كَلَّمُوهُ بِكُلِّ كَلامِ يُوسُفَ الَّذِي كَلَّمَهُمْ بِهِ وَابْصَرَ الْعَجَلاتِ الَّتِي ارْسَلَهَا يُوسُفُ لِتَحْمِلَهُ. فَعَاشَتْ رُوحُ يَعْقُوبَ ابِيهِمْ.
\par 28 فَقَالَ اسْرَائِيلُ: «كَفَى! يُوسُفُ ابْنِي حَيٌّ بَعْدُ. اذْهَبُ وَارَاهُ قَبْلَ انْ امُوتَ».

\chapter{46}

\par 1 فَارْتَحَلَ اسْرَائِيلُ وَكُلُّ مَا كَانَ لَهُ وَاتَى الَى بِئْرِ سَبْعَ وَذَبَحَ ذَبَائِحَ لالَهِ ابِيهِ اسْحَاقَ.
\par 2 فَكَلَّمَ اللهُ اسْرَائِيلَ فِي رُؤَى اللَّيْلِ وَقَالَ: «يَعْقُوبُ يَعْقُوبُ». فَقَالَ: «هَئَنَذَا».
\par 3 فَقَالَ: «انَا اللهُ الَهُ ابِيكَ. لا تَخَفْ مِنَ النُّزُولِ الَى مِصْرَ لانِّي اجْعَلُكَ امَّةً عَظِيمَةً هُنَاكَ.
\par 4 انَا انْزِلُ مَعَكَ الَى مِصْرَ وَانَا اصْعِدُكَ ايْضا. وَيَضَعُ يُوسُفُ يَدَهُ عَلَى عَيْنَيْكَ».
\par 5 فَقَامَ يَعْقُوبُ مِنْ بِئْرِ سَبْعٍ. وَحَمَلَ بَنُو اسْرَائِيلَ يَعْقُوبَ ابَاهُمْ وَاوْلادَهُمْ وَنِسَاءَهُمْ فِي الْعَجَلاتِ الَّتِي ارْسَلَ فِرْعَوْنُ لِحَمْلِهِ.
\par 6 وَاخَذُوا مَوَاشِيَهُمْ وَمُقْتَنَاهُمُ الَّذِي اقْتَنُوا فِي ارْضِ كَنْعَانَ وَجَاءُوا الَى مِصْرَ. يَعْقُوبُ وَكُلُّ نَسْلِهِ مَعَهُ.
\par 7 بَنُوهُ وَبَنُو بَنِيهِ مَعَهُ وَبَنَاتُهُ وَبَنَاتُ بَنِيهِ وَكُلُّ نَسْلِهِ جَاءَ بِهِمْ مَعَهُ الَى مِصْرَ.
\par 8 وَهَذِهِ اسْمَاءُ بَنِي اسْرَائِيلَ الَّذِينَ جَاءُوا الَى مِصْرَ: يَعْقُوبُ وَبَنُوهُ. بِكْرُ يَعْقُوبَ رَاوبَيْنُ.
\par 9 وَبَنُو رَاوبَيْنَ: حَنُوكُ وَفَلُّو وَحَصْرُونُ وَكَرْمِي.
\par 10 وَبَنُو شَمْعُونَ: يَمُوئِيلُ وَيَامِينُ وَاوهَدُ وَيَاكِينُ وَصُوحَرُ وَشَاولُ ابْنُ الْكَنْعَانِيَّةِ.
\par 11 وَبَنُو لاوِي: جَرْشُونُ وَقَهَاتُ وَمَرَارِي.
\par 12 وَبَنُو يَهُوذَا عِيرٌ وَاونَانُ وَشِيلَةُ وَفَارَصُ وَزَارَحُ. وَامَّا عِيرٌ وَاونَانُ فَمَاتَا فِي ارْضِ كَنْعَانَ. وَكَانَ ابْنَا فَارَصَ حَصْرُونَ وَحَامُولَ.
\par 13 وَبَنُو يَسَّاكَرَ: تُولاعُ وَفَوَّةُ وَيُوبُ وَشِمْرُونُ.
\par 14 وَبَنُو زَبُولُونَ: سَارَدُ وَايلُونُ وَيَاحَلْئِيلُ.
\par 15 هَؤُلاءِ بَنُو لَيْئَةَ الَّذِينَ وَلَدَتْهُمْ لِيَعْقُوبَ فِي فَدَّانَ ارَامَ مَعَ دِينَةَ ابْنَتِهِ. جَمِيعُ نُفُوسِ بَنِيهِ وَبَنَاتِهِ ثَلاثٌ وَثَلاثُونَ.
\par 16 وَبَنُو جَادَ: صِفْيُونُ وَحَجِّي وَشُونِي وَاصْبُونُ وَعِيرِي وَارُودِي وَارْئِيلِي.
\par 17 وَبَنُو اشِيرَ: يِمْنَةُ وَيِشْوَةُ وَيِشْوِي وَبَرِيعَةُ وَسَارَحُ هِيَ اخْتُهُمْ. وَابْنَا بَرِيعَةَ حَابِرُ وَمَلْكِيئِيلُ.
\par 18 هَؤُلاءِ بَنُو زِلْفَةَ الَّتِي اعْطَاهَا لابَانُ لِلَيْئَةَ ابْنَتِهِ فَوَلَدَتْ هَؤُلاءِ لِيَعْقُوبَ سِتَّ عَشَرَةَ نَفْسا.
\par 19 ابْنَا رَاحِيلَ امْرَاةِ يَعْقُوبَ: يُوسُفُ وَبَنْيَامِينُ.
\par 20 وَوُلِدَ لِيُوسُفَ فِي ارْضِ مِصْرَ: مَنَسَّى وَافْرَايِمُ اللَّذَانِ وَلَدَتْهُمَا لَهُ اسْنَاتُ بِنْتُ فُوطِي فَارَعَ كَاهِنِ اونٍ.
\par 21 وَبَنُو بِنْيَامِينَ: بَالَعُ وَبَاكَرُ وَاشْبِيلُ وَجِيرَا وَنَعْمَانُ وَايحِي وَرُوشُ وَمُفِّيمُ وَحُفِّيمُ وَارْدُ.
\par 22 هَؤُلاءِ بَنُو رَاحِيلَ الَّذِينَ وُلِدُوا لِيَعْقُوبَ. جَمِيعُ النُّفُوسِ ارْبَعَ عَشَرَةَ.
\par 23 وَابْنُ دَانَ حُوشِيمُ.
\par 24 وَبَنُو نَفْتَالِي: يَاحَصْئِيلُ وَجُونِي وَيِصْرُ وَشِلِّيمُ.
\par 25 هَؤُلاءِ بَنُو بِلْهَةَ الَّتِي اعْطَاهَا لابَانُ لِرَاحِيلَ ابْنَتِهِ. فَوَلَدَتْ هَؤُلاءِ لِيَعْقُوبَ. جَمِيعُ الانْفُسِ سَبْعٌ.
\par 26 جَمِيعُ النُّفُوسِ لِيَعْقُوبَ الَّتِي اتَتْ الَى مِصْرَ الْخَارِجَةِ مِنْ صُلْبِهِ مَا عَدَا نِسَاءَ بَنِي يَعْقُوبَ جَمِيعُ النُّفُوسِ سِتٌّ وَسِتُّونَ نَفْسا.
\par 27 وَابْنَا يُوسُفَ اللَّذَانِ وُلِدَا لَهُ فِي مِصْرَ نَفْسَانِ. جَمِيعُ نُفُوسِ بَيْتِ يَعْقُوبَ الَّتِي جَاءَتْ الَى مِصْرَ سَبْعُونَ.
\par 28 فَارْسَلَ يَهُوذَا امَامَهُ الَى يُوسُفَ لِيُرِيَ الطَّرِيقَ امَامَهُ الَى جَاسَانَ ثُمَّ جَاءُوا الَى ارْضِ جَاسَانَ.
\par 29 فَشَدَّ يُوسُفُ مَرْكَبَتَهُ وَصَعِدَ لِاسْتِقْبَالِ اسْرَائِيلَ ابِيهِ الَى جَاسَانَ. وَلَمَّا ظَهَرَ لَهُ وَقَعَ عَلَى عُنُقِهِ وَبَكَى عَلَى عُنُقِهِ زَمَانا.
\par 30 فَقَالَ اسْرَائِيلُ لِيُوسُفَ: «امُوتُ الْانَ بَعْدَ مَا رَايْتُ وَجْهَكَ انَّكَ حَيٌّ بَعْدُ».
\par 31 ثُمَّ قَالَ يُوسُفُ لاخْوَتِهِ وَلِبَيْتِ ابِيهِ: «اصْعَدُ وَاخْبِرُ فِرْعَوْنَ وَاقُولُ لَهُ: اخْوَتِي وَبَيْتُ ابِي الَّذِينَ فِي ارْضِ كَنْعَانَ جَاءُوا الَيَّ
\par 32 وَالرِّجَالُ رُعَاةُ غَنَمٍ فَانَّهُمْ كَانُوا اهْلَ مَوَاشٍ وَقَدْ جَاءُوا بِغَنَمِهِمْ وَبَقَرِهِمْ وَكُلِّ مَا لَهُمْ.
\par 33 فَيَكُونُ اذَا دَعَاكُمْ فِرْعَوْنُ وَقَالَ: مَا صِنَاعَتُكُمْ؟
\par 34 انْ تَقُولُوا: عَبِيدُكَ اهْلُ مَوَاشٍ مُنْذُ صِبَانَا الَى الْانَ نَحْنُ وَابَاؤُنَا جَمِيعا. لِكَيْ تَسْكُنُوا فِي ارْضِ جَاسَانَ. لانَّ كُلَّ رَاعِي غَنَمٍ رِجْسٌ لِلْمِصْرِيِّينَ».

\chapter{47}

\par 1 فَاتَى يُوسُفُ وَقَالَ لِفِرْعَوْنَ: «ابِي وَاخْوَتِي وَغَنَمُهُمْ وَبَقَرُهُمْ وَكُلُّ مَا لَهُمْ جَاءُوا مِنْ ارْضِ كَنْعَانَ. وَهُوَذَا هُمْ فِي ارْضِ جَاسَانَ».
\par 2 وَاخَذَ مِنْ جُمْلَةِ اخْوَتِهِ خَمْسَةَ رِجَالٍ وَاوْقَفَهُمْ امَامَ فِرْعَوْنَ.
\par 3 فَقَالَ فِرْعَوْنُ لاخْوَتِهِ: «مَا صِنَاعَتُكُمْ؟» فَقَالُوا لِفِرْعَوْنَ: «عَبِيدُكَ رُعَاةُ غَنَمٍ نَحْنُ وَابَاؤُنَا جَمِيعا».
\par 4 وَقَالُوا لِفِرْعَوْنَ: «جِئْنَا لِنَتَغَرَّبَ فِي الارْضِ اذْ لَيْسَ لِغَنَمِ عَبِيدِكَ مَرْعًى لانَّ الْجُوعَ شَدِيدٌ فِي ارْضِ كَنْعَانَ. فَالْانَ لِيَسْكُنْ عَبِيدُكَ فِي ارْضِ جَاسَانَ».
\par 5 فَقَالَ فِرْعَوْنُ لِيُوسُفَ: «ابُوكَ وَاخْوَتُكَ جَاءُوا الَيْكَ.
\par 6 ارْضُ مِصْرَ قُدَّامَكَ. فِي افْضَلِ الارْضِ اسْكِنْ ابَاكَ وَاخْوَتَكَ. لِيَسْكُنُوا فِي ارْضِ جَاسَانَ. وَانْ عَلِمْتَ انَّهُ يُوجَدُ بَيْنَهُمْ ذَوُو قُدْرَةٍ فَاجْعَلْهُمْ رُؤَسَاءَ مَوَاشٍ عَلَى الَّتِي لِي»
\par 7 ثُمَّ ادْخَلَ يُوسُفُ يَعْقُوبَ ابَاهُ وَاوْقَفَهُ امَامَ فِرْعَوْنَ. وَبَارَكَ يَعْقُوبُ فِرْعَوْنَ.
\par 8 فَقَالَ فِرْعَوْنُ لِيَعْقُوبَ: «كَمْ هِيَ ايَّامُ سِنِي حَيَاتِكَ؟»
\par 9 فَقَالَ يَعْقُوبُ لِفِرْعَوْنَ: «ايَّامُ سِنِي غُرْبَتِي مِئَةٌ وَثَلاثُونَ سَنَةً. قَلِيلَةً وَرَدِيَّةً كَانَتْ ايَّامُ سِنِي حَيَاتِي وَلَمْ تَبْلُغْ الَى ايَّامِ سِنِي حَيَاةِ ابَائِي فِي ايَّامِ غُرْبَتِهِمْ».
\par 10 وَبَارَكَ يَعْقُوبُ فِرْعَوْنَ وَخَرَجَ مِنْ لَدُنْ فِرْعَوْنَ.
\par 11 فَاسْكَنَ يُوسُفُ ابَاهُ وَاخْوَتَهُ وَاعْطَاهُمْ مُلْكا فِي ارْضِ مِصْرَ فِي افْضَلِ الارْضِ فِي ارْضِ رَعَمْسِيسَ كَمَا امَرَ فِرْعَوْنُ.
\par 12 وَعَالَ يُوسُفُ ابَاهُ وَاخْوَتَهُ وَكُلَّ بَيْتِ ابِيهِ بِطَعَامٍ عَلَى حَسَبِ الاوْلادِ.
\par 13 وَلَمْ يَكُنْ خُبْزٌ فِي كُلِّ الارْضِ لانَّ الْجُوعَ كَانَ شَدِيدا جِدّا. فَخَوَّرَتْ ارْضُ مِصْرَ وَارْضُ كَنْعَانَ مِنْ اجْلِ الْجُوعِ.
\par 14 فَجَمَعَ يُوسُفُ كُلَّ الْفِضَّةِ الْمَوْجُودَةِ فِي ارْضِ مِصْرَ وَفِي ارْضِ كَنْعَانَ بِالْقَمْحِ الَّذِي اشْتَرُوا. وَجَاءَ يُوسُفُ بِالْفِضَّةِ الَى بَيْتِ فِرْعَوْنَ.
\par 15 فَلَمَّا فَرَغَتِ الْفِضَّةُ مِنْ ارْضِ مِصْرَ وَمِنْ ارْضِ كَنْعَانَ اتَى جَمِيعُ الْمِصْرِيِّينَ الَى يُوسُفَ قَائِلِينَ: «اعْطِنَا خُبْزا فَلِمَاذَا نَمُوتُ قُدَّامَكَ؟ لانْ لَيْسَ فِضَّةٌ ايْضا».
\par 16 فَقَالَ يُوسُفُ: «هَاتُوا مَوَاشِيَكُمْ فَاعْطِيَكُمْ بِمَوَاشِيكُمْ انْ لَمْ يَكُنْ فِضَّةٌ ايْضا».
\par 17 فَجَاءُوا بِمَوَاشِيهِمْ الَى يُوسُفَ فَاعْطَاهُمْ يُوسُفُ خُبْزا بِالْخَيْلِ وَبِمَوَاشِي الْغَنَمِ وَالْبَقَرِ وَبِالْحَمِيرِ. فَقَاتَهُمْ بِالْخُبْزِ تِلْكَ السَّنَةَ بَدَلَ جَمِيعِ مَوَاشِيهِمْ.
\par 18 وَلَمَّا تَمَّتْ تِلْكَ السَّنَةُ اتُوا الَيْهِ فِي السَّنَةِ الثَّانِيَةِ وَقَالُوا لَهُ: «لا نُخْفِي عَنْ سَيِّدِي انَّهُ اذْ قَدْ فَرَغَتِ الْفِضَّةُ وَمَوَاشِي الْبَهَائِمِ عَِنْدَ سَيِّدِي لَمْ يَبْقَ قُدَّامَ سَيِّدِي الَّا اجْسَادُنَا وَارْضُنَا.
\par 19 لِمَاذَا نَمُوتُ امَامَ عَيْنَيْكَ نَحْنُ وَارْضُنَا جَمِيعا؟ اشْتَرِنَا وَارْضَنَا بِالْخُبْزِ فَنَصِيرَ نَحْنُ وَارْضُنَا عَبِيدا لِفِرْعَوْنَ. وَاعْطِ بِذَارا لِنَحْيَا وَلا نَمُوتَ وَلا تَصِيرَ ارْضُنَا قَفْرا».
\par 20 فَاشْتَرَى يُوسُفُ كُلَّ ارْضِ مِصْرَ لِفِرْعَوْنَ اذْ بَاعَ الْمِصْرِيُّونَ كُلُّ وَاحِدٍ حَقْلَهُ لانَّ الْجُوعَ اشْتَدَّ عَلَيْهِمْ. فَصَارَتِ الارْضُ لِفِرْعَوْنَ.
\par 21 وَامَّا الشَّعْبُ فَنَقَلَهُمْ الَى الْمُدُنِ مِنْ اقْصَى حَدِّ مِصْرَ الَى اقْصَاهُ.
\par 22 الَّا انَّ ارْضَ الْكَهَنَةِ لَمْ يَشْتَرِهَا اذْ كَانَتْ لِلْكَهَنَةِ فَرِيضَةٌ مِنْ قِبَلِ فِرْعَوْنَ. فَاكَلُوا فَرِيضَتَهُمُ الَّتِي اعْطَاهُمْ فِرْعَوْنُ. لِذَلِكَ لَمْ يَبِيعُوا ارْضَهُمْ.
\par 23 فَقَالَ يُوسُفُ لِلشَّعْبِ: «انِّي قَدِ اشْتَرَيْتُكُمُ الْيَوْمَ وَارْضَكُمْ لِفِرْعَوْنَ. هُوَذَا لَكُمْ بِذَارٌ فَتَزْرَعُونَ الارْضَ.
\par 24 وَيَكُونُ عَِنْدَ الْغَلَّةِ انَّكُمْ تُعْطُونَ خُمْسا لِفِرْعَوْنَ وَالارْبَعَةُ الاجْزَاءُ تَكُونُ لَكُمْ بِذَارا لِلْحَقْلِ وَطَعَاما لَكُمْ وَلِمَنْ فِي بُِيُوتِكُمْ وَطَعَاما لاوْلادِكُمْ».
\par 25 فَقَالُوا: «احْيَيْتَنَا. لَيْتَنَا نَجِدُ نِعْمَةً فِي عَيْنَيْ سَيِّدِي فَنَكُونَ عَبِيدا لِفِرْعَوْنَ».
\par 26 فَجَعَلَهَا يُوسُفُ فَرْضا عَلَى ارْضِ مِصْرَ الَى هَذَا الْيَوْمِ: لِفِرْعَوْنَ الْخُمْسُ. الَّا انَّ ارْضَ الْكَهَنَةِ وَحْدَهُمْ لَمْ تَصِرْ لِفِرْعَوْنَ.
\par 27 وَسَكَنَ اسْرَائِيلُ فِي ارْضِ مِصْرَ فِي ارْضِ جَاسَانَ وَتَمَلَّكُوا فِيهَا وَاثْمَرُوا وَكَثُرُوا جِدّا.
\par 28 وَعَاشَ يَعْقُوبُ فِي ارْضِ مِصْرَ سَبْعَ عَشَرَةَ سَنَةً. فَكَانَتْ ايَّامُ يَعْقُوبَ سِنُو حَيَاتِهِ مِئَةً وَسَبْعا وَارْبَعِينَ سَنَةً.
\par 29 وَلَمَّا قَرُبَتْ ايَّامُ اسْرَائِيلَ انْ يَمُوتَ دَعَا ابْنَهُ يُوسُفَ وَقَالَ لَهُ: «انْ كُنْتُ قَدْ وَجَدْتُ نِعْمَةً فِي عَيْنَيْكَ فَضَعْ يَدَكَ تَحْتَ فَخْذِي وَاصْنَعْ مَعِي مَعْرُوفا وَامَانَةً. لا تَدْفِنِّي فِي مِصْرَ.
\par 30 بَلْ اضْطَجِعُ مَعَ ابَائِي. فَتَحْمِلُنِي مِنْ مِصْرَ وَتَدْفِنُنِي فِي مَقْبَرَتِهِمْ». فَقَالَ: «انَا افْعَلُ بِحَسَبِ قَوْلِكَ».
\par 31 فَقَالَ: «احْلِفْ لِي». فَحَلَفَ لَهُ. فَسَجَدَ اسْرَائِيلُ عَلَى رَاسِ السَّرِيرِ.

\chapter{48}

\par 1 وَحَدَثَ بَعْدَ هَذِهِ الامُورِ انَّهُ قِيلَ لِيُوسُفَ: «هُوَذَا ابُوكَ مَرِيضٌ». فَاخَذَ مَعَهُ ابْنَيْهِ مَنَسَّى وَافْرَايِمَ.
\par 2 فَاخْبِرَ يَعْقُوبُ وَقِيلَ لَهُ: «هُوَذَا ابْنُكَ يُوسُفُ قَادِمٌ الَيْكَ». فَتَشَدَّدَ اسْرَائِيلُ وَجَلَسَ عَلَى السَّرِيرِ.
\par 3 وَقَالَ يَعْقُوبُ لِيُوسُفَ: «اللهُ الْقَادِرُ عَلَى كُلِّ شَيْءٍ ظَهَرَ لِي فِي لُوزَ فِي ارْضِ كَنْعَانَ وَبَارَكَنِي.
\par 4 وَقَالَ لِي: هَا انَا اجْعَلُكَ مُثْمِرا وَاكَثِّرُكَ وَاجْعَلُكَ جُمْهُورا مِنَ الامَمِ وَاعْطِي نَسْلَكَ هَذِهِ الارْضَ مِنْ بَعْدِكَ مُلْكا ابَدِيّا.
\par 5 وَالْانَ ابْنَاكَ الْمَوْلُودَانِ لَكَ فِي ارْضِ مِصْرَ قَبْلَمَا اتَيْتُ الَيْكَ الَى مِصْرَ هُمَا لِي. افْرَايِمُ وَمَنَسَّى كَرَاوبَيْنَ وَشَمْعُونَ يَكُونَانِ لِي.
\par 6 وَامَّا اوْلادُكَ الَّذِينَ تَلِدُ بَعْدَهُمَا فَيَكُونُونَ لَكَ. عَلَى اسْمِ اخَوَيْهِمْ يُسَمُّونَ فِي نَصِيبِهِمْ.
\par 7 وَانَا حِينَ جِئْتُ مِنْ فَدَّانَ مَاتَتْ عِنْدِي رَاحِيلُ فِي ارْضِ كَنْعَانَ فِي الطَّرِيقِ اذْ بَقِيَتْ مَسَافَةٌ مِنَ الارْضِ حَتَّى اتِيَ الَى افْرَاتَةَ. فَدَفَنْتُهَا هُنَاكَ فِي طَرِيقِ افْرَاتَةَ (الَّتِي هِيَ بَيْتُ لَحْمٍ)».
\par 8 وَرَاى اسْرَائِيلُ ابْنَيْ يُوسُفَ فَقَالَ: «مَنْ هَذَانِ؟».
\par 9 فَقَالَ يُوسُفُ لابِيهِ: «هُمَا ابْنَايَ اللَّذَانِ اعْطَانِيَ اللهُ هَهُنَا». فَقَالَ: «قَدِّمْهُمَا الَيَّ لِابَارِكَهُمَا».
\par 10 وَامَّا عَيْنَا اسْرَائِيلَ فَكَانَتَا قَدْ ثَقُلَتَا مِنَ الشَّيْخُوخَةِ لا يَقْدُِرُ انْ يُبْصِرَ فَقَرَّبَهُمَا الَيْهِ فَقَبَّلَهُمَا وَاحْتَضَنَهُمَا.
\par 11 وَقَالَ اسْرَائِيلُ لِيُوسُفَ: «لَمْ اكُنْ اظُنُّ انِّي ارَى وَجْهَكَ وَهُوَذَا اللهُ قَدْ ارَانِي نَسْلَكَ ايْضا».
\par 12 ثُمَّ اخْرَجَهُمَا يُوسُفُ مِنْ بَيْنَ رُكْبَتَيْهِ وَسَجَدَ امَامَ وَجْهِهِ الَى الارْضِ.
\par 13 وَاخَذَ يُوسُفُ الِاثْنَيْنِ افْرَايِمَ بِيَمِينِهِ عَنْ يَسَارِ اسْرَائِيلَ وَمَنَسَّى بِيَسَارِهِ عَنْ يَمِينِ اسْرَائِيلَ وَقَرَّبَهُمَا الَيْهِ.
\par 14 فَمَدَّ اسْرَائِيلُ يَمِينَهُ وَوَضَعَهَا عَلَى رَاسِ افْرَايِمَ وَهُوَ الصَّغِيرُ وَيَسَارَهُ عَلَى رَاسِ مَنَسَّى. وَضَعَ يَدَيْهِ بِفِطْنَةٍ فَانَّ مَنَسَّى كَانَ الْبِكْرَ.
\par 15 وَبَارَكَ يُوسُفَ وَقَالَ: «اللهُ الَّذِي سَارَ امَامَهُ ابَوَايَ ابْرَاهِيمُ وَاسْحَاقُ - اللهُ الَّذِي رَعَانِي مُنْذُ وُجُودِي الَى هَذَا الْيَوْمِ -
\par 16 الْمَلاكُ الَّذِي خَلَّصَنِي مِنْ كُلِّ شَرٍّ يُبَارِكُ الْغُلامَيْنِ. وَلْيُدْعَ عَلَيْهِمَا اسْمِي وَاسْمُ ابَوَيَّ ابْرَاهِيمَ وَاسْحَاقَ. وَلْيَكْثُرَا كَثِيرا فِي الارْضِ».
\par 17 فَلَمَّا رَاى يُوسُفُ انَّ ابَاهُ وَضَعَ يَدَهُ الْيُمْنَى عَلَى رَاسِ افْرَايِمَ سَاءَ ذَلِكَ فِي عَيْنَيْهِ فَامْسَكَ بِيَدِ ابِيهِ لِيَنْقُلَهَا عَنْ رَاسِ افْرَايِمَ الَى رَاسِ مَنَسَّى.
\par 18 وَقَالَ يُوسُفُ لابِيهِ: «لَيْسَ هَكَذَا يَا ابِي لانَّ هَذَا هُوَ الْبِكْرُ. ضَعْ يَمِينَكَ عَلَى رَاسِهِ».
\par 19 فَابَى ابُوهُ وَقَالَ: «عَلِمْتُ يَا ابْنِي عَلِمْتُ! هُوَ ايْضا يَكُونُ شَعْبا وَهُوَ ايْضا يَصِيرُ كَبِيرا. وَلَكِنَّ اخَاهُ الصَّغِيرَ يَكُونُ اكْبَرَ مِنْهُ وَنَسْلُهُ يَكُونُ جُمْهُورا مِنَ الامَمِ».
\par 20 وَبَارَكَهُمَا فِي ذَلِكَ الْيَوْمِ قَائِلا: «بِكَ يُبَارِكُ اسْرَائِيلُ قَائِلا: يَجْعَلُكَ اللهُ كَافْرَايِمَ وَكَمَنَسَّى». فَقَدَّمَ افْرَايِمَ عَلَى مَنَسَّى.
\par 21 وَقَالَ اسْرَائِيلُ لِيُوسُفَ: «هَا انَا امُوتُ وَلَكِنَّ اللهَ سَيَكُونُ مَعَكُمْ وَيَرُدُّكُمْ الَى ارْضِ ابَائِكُمْ.
\par 22 وَانَا قَدْ وَهَبْتُ لَكَ سَهْما وَاحِدا فَوْقَ اخْوَتِكَ اخَذْتُهُ مِنْ يَدِ الامُورِيِّينَ بِسَيْفِي وَقَوْسِي».

\chapter{49}

\par 1 وَدَعَا يَعْقُوبُ بَنِيهِ وَقَالَ: «اجْتَمِعُوا لِانْبِئَكُمْ بِمَا يُصِيبُكُمْ فِي اخِرِ الايَّامِ.
\par 2 اجْتَمِعُوا وَاسْمَعُوا يَا بَنِي يَعْقُوبَ وَاصْغُوا الَى اسْرَائِيلَ ابِيكُمْ.
\par 3 رَاوبَيْنُ انْتَ بِكْرِي قُوَّتِي وَاوَّلُ قُدْرَتِي فَضْلُ الرِّفْعَةِ وَفَضْلُ الْعِزِّ.
\par 4 فَائِرا كَالْمَاءِ لا تَتَفَضَّلُ لانَّكَ صَعِدْتَ عَلَى مَضْجَعِ ابِيكَ. حِينَئِذٍ دَنَّسْتَهُ. عَلَى فِرَاشِي صَعِدَ.
\par 5 شَمْعُونُ وَلاوِي اخَوَانِ. الاتُ ظُلْمٍ سُيُوفُهُمَا.
\par 6 فِي مَجْلِسِهِمَا لا تَدْخُلُ نَفْسِي. بِمَجْمَعِهِمَا لا تَتَّحِدُ كَرَامَتِي. لانَّهُمَا فِي غَضَبِهِمَا قَتَلا انْسَانا وَفِي رِضَاهُمَا عَرْقَبَا ثَوْرا.
\par 7 مَلْعُونٌ غَضَبُهُمَا فَانَّهُ شَدِيدٌ وَسَخَطُهُمَا فَانَّهُ قَاسٍ. اقَسِّمُهُمَا فِي يَعْقُوبَ وَافَرِّقُهُمَا فِي اسْرَائِيلَ.
\par 8 يَهُوذَا ايَّاكَ يَحْمَدُ اخْوَتُكَ. يَدُكَ عَلَى قَفَا اعْدَائِكَ. يَسْجُدُ لَكَ بَنُو ابِيكَ.
\par 9 يَهُوذَا جَرْوُ اسَدٍ. مِنْ فَرِيسَةٍ صَعِدْتَ يَا ابْنِي. جَثَا وَرَبَضَ كَاسَدٍ وَكَلَبْوَةٍ. مَنْ يُنْهِضُهُ؟
\par 10 لا يَزُولُ قَضِيبٌ مِنْ يَهُوذَا وَمُشْتَرِعٌ مِنْ بَيْنِ رِجْلَيْهِ حَتَّى يَاتِيَ شِيلُونُ وَلَهُ يَكُونُ خُضُوعُ شُعُوبٍ.
\par 11 رَابِطا بِالْكَرْمَةِ جَحْشَهُ وَبِالْجَفْنَةِ ابْنَ اتَانِهِ. غَسَلَ بِالْخَمْرِ لِبَاسَهُ وَبِدَمِ الْعِنَبِ ثَوْبَهُ.
\par 12 مُسْوَدُّ الْعَيْنَيْنِ مِنَ الْخَمْرِ وَمُبْيَضُّ الاسْنَانِ مِنَ اللَّبَنِ.
\par 13 زَبُولُونُ عِنْدَ سَاحِلِ الْبَحْرِ يَسْكُنُ وَهُوَ عِنْدَ سَاحِلِ السُّفُنِ وَجَانِبُهُ عِنْدَ صَيْدُونَ.
\par 14 يَسَّاكَرُ حِمَارٌ جَسِيمٌ رَابِضٌ بَيْنَ الْحَظَائِرِ.
\par 15 فَرَاى الْمَحَلَّ انَّهُ حَسَنٌ وَالارْضَ انَّهَا نَزِهَةٌ فَاحْنَى كَتِفَهُ لِلْحِمْلِ وَصَارَ لِلْجِزْيَةِ عَبْدا.
\par 16 دَانُ يَدِينُ شَعْبَهُ كَاحَدِ اسْبَاطِ اسْرَائِيلَ.
\par 17 يَكُونُ دَانُ حَيَّةً عَلَى الطَّرِيقِ افْعُوانا عَلَى السَّبِيلِ يَلْسَعُ عَقِبَيِ الْفَرَسِ فَيَسْقُطُ رَاكِبُهُ الَى الْوَرَاءِ.
\par 18 لِخَلاصِكَ انْتَظَرْتُ يَا رَبُّ.
\par 19 جَادُ يَزْحَمُهُ جَيْشٌ وَلَكِنَّهُ يَزْحَمُ مُؤَخَّرَهُ.
\par 20 اشِيرُ خُبْزُهُ سَمِينٌ وَهُوَ يُعْطِي لَذَّاتِ مُلُوكٍ.
\par 21 نَفْتَالِي ايِّلَةٌ مُسَيَّبَةٌ يُعْطِي اقْوَالا حَسَنَةً.
\par 22 يُوسُفُ غُصْنُ شَجَرَةٍ مُثْمِرَةٍ غُصْنُ شَجَرَةٍ مُثْمِرَةٍ عَلَى عَيْنٍ. اغْصَانٌ قَدِ ارْتَفَعَتْ فَوْقَ حَائِطٍ.
\par 23 فَمَرَّرَتْهُ وَرَمَتْهُ وَاضْطَهَدَتْهُ ارْبَابُ السِّهَامِ.
\par 24 وَلَكِنْ ثَبَتَتْ بِمَتَانَةٍ قَوْسُهُ وَتَشَدَّدَتْ سَوَاعِدُ يَدَيْهِ. مِنْ يَدَيْ عَزِيزِ يَعْقُوبَ مِنْ هُنَاكَ مِنَ الرَّاعِي صَخْرِ اسْرَائِيلَ
\par 25 مِنْ الَهِ ابِيكَ الَّذِي يُعِينُكَ وَمِنَ الْقَادِرِ عَلَى كُلِّ شَيْءٍ الَّذِي يُبَارِكُكَ تَاتِي بَرَكَاتُ السَّمَاءِ مِنْ فَوْقُ وَبَرَكَاتُ الْغَمْرِ الرَّابِضِ تَحْتُ. بَرَكَاتُ الثَّدْيَيْنِ وَالرَّحِمِ.
\par 26 بَرَكَاتُ ابِيكَ فَاقَتْ عَلَى بَرَكَاتِ ابَوَيَّ. الَى مُنْيَةِ الْاكَامِ الدَّهْرِيَّةِ تَكُونُ عَلَى رَاسِ يُوسُفَ وَعَلَى قِمَّةِ نَذِيرِ اخْوَتِهِ.
\par 27 بِنْيَامِينُ ذِئْبٌ يَفْتَرِسُ. فِي الصَّبَاحِ يَاكُلُ غَنِيمَةً وَعِنْدَ الْمَسَاءِ يُقَسِّمُ نَهْبا».
\par 28 جَمِيعُ هَؤُلاءِ هُمْ اسْبَاطُ اسْرَائِيلَ الِاثْنَا عَشَرَ. وَهَذَا مَا كَلَّمَهُمْ بِهِ ابُوهُمْ وَبَارَكَهُمْ. كُلُّ وَاحِدٍ بِحَسَبِ بَرَكَتِهِ بَارَكَهُمْ.
\par 29 وَاوْصَاهُمْ وَقَالَ لَهُمْ: «انَا انْضَمُّ الَى قَوْمِي. ادْفِنُونِي عِنْدَ ابَائِي فِي الْمَغَارَةِ الَّتِي فِي حَقْلِ عِفْرُونَ الْحِثِّيِّ.
\par 30 فِي الْمَغَارَةِ الَّتِي فِي حَقْلِ الْمَكْفِيلَةِ الَّتِي امَامَ مَمْرَا فِي ارْضِ كَنْعَانَ الَّتِي اشْتَرَاهَا ابْرَاهِيمُ مَعَ الْحَقْلِ مِنْ عِفْرُونَ الْحِثِّيِّ مُلْكَ قَبْرٍ.
\par 31 هُنَاكَ دَفَنُوا ابْرَاهِيمَ وَسَارَةَ امْرَاتَهُ. هُنَاكَ دَفَنُوا اسْحَاقَ وَرِفْقَةَ امْرَاتَهُ. وَهُنَاكَ دَفَنْتُ لَيْئَةَ.
\par 32 شِرَاءُ الْحَقْلِ وَالْمَغَارَةِ الَّتِي فِيهِ كَانَ مِنْ بَنِي حِثَّ».
\par 33 وَلَمَّا فَرَغَ يَعْقُوبُ مِنْ تَوْصِيَةِ بَنِيهِ ضَمَّ رِجْلَيْهِ الَى السَّرِيرِ وَاسْلَمَ الرُّوحَ وَانْضَمَّ الَى قَوْمِهِ.

\chapter{50}

\par 1 فَوَقَعَ يُوسُفُ عَلَى وَجْهِ ابِيهِ وَبَكَى عَلَيْهِ وَقَبَّلَهُ.
\par 2 وَامَرَ يُوسُفُ عَبِيدَهُ الاطِبَّاءَ انْ يُحَنِّطُوا ابَاهُ. فَحَنَّطَ الاطِبَّاءُ اسْرَائِيلَ.
\par 3 وَكَمِلَ لَهُ ارْبَعُونَ يَوْما لانَّهُ هَكَذَا تَكْمُلُ ايَّامُ الْمُحَنَّطِينَ. وَبَكَى عَلَيْهِ الْمِصْرِيُّونَ سَبْعِينَ يَوْما.
\par 4 وَبَعْدَ مَا مَضَتْ ايَّامُ بُكَائِهِ قَالَ يُوسُفُ لِبَيْتَ فِرْعَوْنَ: «انْ كُنْتُ قَدْ وَجَدْتُ نِعْمَةً فِي عُيُونِكُمْ فَتَكَلَّمُوا فِي مَسَامِعِ فِرْعَوْنَ قَائِلِينَ:
\par 5 ابِي اسْتَحْلَفَنِي قَائِلا: هَا انَا امُوتُ. فِي قَبْرِيَ الَّذِي حَفَرْتُ لِنَفْسِي فِي ارْضِ كَنْعَانَ هُنَاكَ تَدْفِنُنِي. فَالْانَ اصْعَدُ لادْفِنَ ابِي وَارْجِعُ».
\par 6 فَقَالَ فِرْعَوْنُ: «اصْعَدْ وَادْفِنْ ابَاكَ كَمَا اسْتَحْلَفَكَ».
\par 7 فَصَعِدَ يُوسُفُ لِيَدْفِنَ ابَاهُ وَصَعِدَ مَعَهُ جَمِيعُ عَبِيدِ فِرْعَوْنَ شُيُوخُ بَيْتِهِ وَجَمِيعُ شُيُوخِ ارْضِ مِصْرَ
\par 8 وَكُلُّ بَيْتِ يُوسُفَ وَاخْوَتُهُ وَبَيْتُ ابِيهِ. غَيْرَ انَّهُمْ تَرَكُوا اوْلادَهُمْ وَغَنَمَهُمْ وَبَقَرَهُمْ فِي ارْضِ جَاسَانَ.
\par 9 وَصَعِدَ مَعَهُ مَرْكَبَاتٌ وَفُرْسَانٌ. فَكَانَ الْجَيْشُ كَثِيرا جِدّا.
\par 10 فَاتُوا الَى بَيْدَرِ اطَادَ الَّذِي فِي عَبْرِ الارْدُنِّ وَنَاحُوا هُنَاكَ نَوْحا عَظِيما وَشَدِيدا جِدّا. وَصَنَعَ لابِيهِ مَنَاحَةً سَبْعَةَ ايَّامٍ.
\par 11 فَلَمَّا رَاى اهْلُ الْبِلادِ الْكَنْعَانِيُّونَ الْمَنَاحَةَ فِي بَيْدَرِ اطَادَ قَالُوا: «هَذِهِ مَنَاحَةٌ ثَقِيلَةٌ لِلْمِصْرِيِّينَ». لِ×لِكَ دُعِيَ اسْمُهُ «ابَلَ مِصْرَايِمَ». الَّذِي فِي عَبْرِ الارْدُنِّ.
\par 12 وَفَعَلَ لَهُ بَنُوهُ =كَذَا كَمَا اوْصَاهُمْ:
\par 13 حَمَلَهُ بَنُوهُ الَى ارْضِ كَنْعَانَ وَدَفَنُوهُ فِي مَغَارَةِ حَقْلِ الْمَكْفِيلَةِ الَّتِي اشْتَرَاهَا ابْرَاهِيمُ مَعَ الْحَقْلِ مُلْكَ قَبْرٍ مِنْ عِفْرُونَ الْحِثِّيِّ امَامَ مَمْرَا.
\par 14 ثُمَّ رَجَعَ يُوسُفُ الَى مِصْرَ هُوَ وَاخْوَتُهُ وَجَمِيعُ الَّذِينَ صَعِدُوا مَعَهُ لِدَفْنِ ابِيهِ بَعْدَ مَا دَفَنَ ابَاهُ.
\par 15 وَلَمَّا رَاى اخْوَةُ يُوسُفَ انَّ ابَاهُمْ قَدْ مَاتَ قَالُوا: «لَعَلَّ يُوسُفَ يَضْطَهِدُنَا وَيَرُدُّ عَلَيْنَا جَمِيعَ الشَّرِّ الَّذِي صَنَعْنَا بِهِ».
\par 16 فَاوْصُوا الَى يُوسُفَ قَائِلِينَ: «ابُوكَ اوْصَى قَبْلَ مَوْتِهِ قَائِلا:
\par 17 هَكَذَا تَقُولُونَ لِيُوسُفَ: اهِ! اصْفَحْ عَنْ ذَنْبِ اخْوَتِكَ وَخَطِيَّتِهِمْ فَانَّهُمْ صَنَعُوا بِكَ شَرّا. فَالْانَ اصْفَحْ عَنْ ذَنْبِ عَبِيدِ الَهِ ابِيكَ». فَبَكَى يُوسُفُ حِينَ كَلَّمُوهُ.
\par 18 وَاتَى اخْوَتُهُ ايْضا وَوَقَعُوا امَامَهُ وَقَالُوا: «هَا نَحْنُ عَبِيدُكَ».
\par 19 فَقَالَ لَهُمْ يُوسُفُ: «لا تَخَافُوا. لانَّهُ هَلْ انَا مَكَانَ اللهِ؟
\par 20 انْتُمْ قَصَدْتُمْ لِي شَرّا امَّا اللهُ فَقَصَدَ بِهِ خَيْرا لِكَيْ يَفْعَلَ كَمَا الْيَوْمَ لِيُحْيِيَ شَعْبا كَثِيرا.
\par 21 فَالْانَ لا تَخَافُوا. انَا اعُولُكُمْ وَاوْلادَكُمْ». فَعَزَّاهُمْ وَطَيَّبَ قُلُوبَهُمْ.
\par 22 وَسَكَنَ يُوسُفُ فِي مِصْرَ هُوَ وَبَيْتُ ابِيهِ. وَعَاشَ يُوسُفُ مِئَةً وَعَشَرَ سِنِينَ.
\par 23 وَرَاى يُوسُفُ لافْرَايِمَ اوْلادَ الْجِيلِ الثَّالِثِ. وَاوْلادُ مَاكِيرَ بْنِ مَنَسَّى ايْضا وُلِدُوا عَلَى رُكْبَتَيْ يُوسُفَ.
\par 24 وَقَالَ يُوسُفُ لاخْوَتِهِ: «انَا امُوتُ وَلَكِنَّ اللهَ سَيَفْتَقِدُكُمْ وَيُصْعِدُكُمْ مِنْ هَذِهِ الارْضِ الَى الارْضِ الَّتِي حَلَفَ لابْرَاهِيمَ وَاسْحَاقَ وَيَعْقُوبَ».
\par 25 وَاسْتَحْلَفَ يُوسُفُ بَنِي اسْرَائِيلَ قَائِلا: «اللهُ سَيَفْتَقِدُكُمْ فَتُصْعِدُونَ عِظَامِي مِنْ هُنَا».
\par 26 ثُمَّ مَاتَ يُوسُفُ وَهُوَ ابْنُ مِئَةٍ وَعَشَرِ سِنِينَ فَحَنَّطُوهُ وَوُضِعَ فِي تَابُوتٍ فِي مِصْرَ.

\end{document}