\begin{document}

\title{الكتاب الأول لآدم وحواء}

\chapter{1}

البحر البلوري. أمر الله آدم، المطرود من عدن، أن يسكن في كهف الكنوز.

\par 1 في اليوم الثالث، غرس الله الجنة في شرق الأرض، على حدود العالم شرقًا، وخلفها، نحو شروق الشمس، لا يجد المرء سوى الماء، الذي يحيط بالعالم كله، ويصل إلى حدود السماء

\par 2 وإلى الشمال من الحديقة يوجد بحر من الرقاقات، صافٍ ونقي المذاق، لا يشبه أي شيء آخر؛ بحيث يمكن للمرء من خلال صفائه أن ينظر إلى أعماق الأرض

\par 3 فإذا اغتسل فيه الإنسان، طهر من طهارته، وابيض من بياضه - وإن كان أسود

\par 4 وخلق الله ذلك البحر بمشيئته، لأنه كان يعلم ما سيحدث للإنسان الذي سيخلقه؛ حتى أنه بعد أن يغادر الجنة، بسبب معصيته، سيولد رجال على الأرض، ومن بينهم سيموت الصالحون، الذين سيقيم الله أرواحهم في اليوم الأخير؛ عندما يعودون إلى أجسادهم؛ ويستحمون في مياه ذلك البحر، ويتوبون جميعًا عن خطاياهم

\par 5 ولكن عندما أخرج الله آدم من الجنة، لم يضعه على حدودها الشمالية، لئلا يقترب من بحر الماء، فيغتسل هو وحواء فيه، ويتطهران من خطاياهما، وينسيان المعصية التي ارتكباها، ولا يعودان يتذكرانها عند التفكير في عقابهما

\par 6 ثم، مرة أخرى، فيما يتعلق بالجانب الجنوبي من الجنة، لم يكن الله مسرورًا بترك آدم يسكن هناك؛ لأنه عندما تهب الريح من الشمال، فإنها ستجلب له، على ذلك الجانب الجنوبي، رائحة أشجار الجنة اللذيذة

\par 7 لذلك لم يضع الله آدم هناك، لئلا يشم رائحة تلك الأشجار الطيبة، وينسى معصيته، ويجد عزاءً عما فعله، ويتلذذ برائحة الأشجار، ولا يتطهر من معصيته

\par 8 مرة أخرى، أيضًا، لأن الله رحيم وذو شفقة عظيمة، ويحكم كل الأشياء بطريقة يعرفها هو وحده - فقد جعل أبانا آدم يسكن في الحد الغربي من الجنة، لأن الأرض واسعة جدًا من ذلك الجانب

\par 9 وأمره الله أن يسكن هناك في مغارة في صخرة - مغارة الكنوز أسفل الجنة

\chapter{2}

\par \textit{أُغمي على آدم وحواء عند مغادرة الجنة. فأرسل الله كلمته لتشجيعهما.}

\par 1 ولكن لما خرج أبونا آدم وحواء من الجنة وطآ الأرض بأرجلهما ولم يعلما أنهما يدوسان.

\par 2 ولما وصلوا إلى مدخل باب الجنة ورأوا الأرض واسعة أمامهم مفتوحة وقد غطتها حجارة كبيرة وصغيرة ورمل، خافوا وارتعدوا وسقطوا على وجوههم من الخوف الذي أصابهم، وصاروا كالأموات.

\par 3 لأنهم كانوا حتى ذلك الحين في أرض الحديقة المزروعة بشكل جميل بجميع أنواع الأشجار، لكنهم الآن رأوا أنفسهم في أرض غريبة لم يعرفوها ولم يروها من قبل.

\par 4 ولأنهم كانوا في ذلك الوقت ممتلئين بنعمة الطبيعة المشرقة، ولم تكن قلوبهم متجهة نحو الأمور الأرضية

\par 5 لذلك أشفق الله عليهم، وعندما رآهم ساقطين أمام باب الجنة، أرسل كلمته إلى أبيهم آدم وحواء، وأقامهما من حالتهما الساقطة

\chapter{3}

\par \textit{فيما يتعلق بوعد الأيام الخمسة والنصف العظيمة.}

\par 1 قال الله لآدم: "لقد قدّرتُ على هذه الأرض أيامًا وسنين، وستسكن أنت ونسلك فيها وتسيرون فيها، حتى تتم الأيام والسنين؛ حين أرسل الكلمة التي خلقتك، والتي عصيتك، الكلمة التي أخرجتك من الجنة، والتي رفعتك عندما كنت ساقطًا."

\par 2 «نعم، الكلمة التي ستخلصك مرة أخرى عندما تتم الخمسة أيام والنصف.»

\par 3 ولكن عندما سمع آدم هذه الكلمات من الله، وعن الأيام الخمسة والنصف العظيمة، لم يفهم معناها

\par 4 فكان آدم يظن أنه لن يتبقى له إلا خمسة أيام ونصف حتى نهاية العالم.

\par 5 فبكى آدم، وطلب من الله أن يشرح له الأمر.

\par 6 ثم شرح الله لآدم، الذي خُلق على صورته ومثاله، أن هذه كانت 5000 و500 سنة؛ وكيف سيأتي واحد بعد ذلك ويخلصه هو ونسله

\par 7 لكن الله كان قد قطع هذا العهد مع أبينا آدم، بنفس الشروط، قبل أن يخرج من الجنة، عندما كان عند الشجرة التي أخذت حواء من ثمرها وأعطته إياها ليأكل

\par 8 فلما خرج أبونا آدم من الجنة، مر بتلك الشجرة، فرأى كيف غيّر الله منظرها إلى صورة أخرى، وكيف ذبلت

\par 9 ولما ذهب آدم إليه خاف وارتجف وسقط، لكن الله برحمته رفعه، ثم قطع معه هذا العهد

\par 10 وأيضًا، عندما كان آدم عند باب الجنة، ورأى الكروب وفي يده سيف من نار متقدة، فغضب الكروب وعبس في وجهه، فخاف منه كل من آدم وحواء، وظنّا أنه ينوي قتلهما. فسقطا على وجهيهما وارتعدا من الخوف

\par 11 فأشفق عليهم وأظهر لهم الرحمة، ثم انصرف عنهم وصعد إلى السماء وصلى إلى الرب وقال:

\par 12 «يا رب، لقد أرسلتني لأحرس باب الجنة بسيف من نار.»

\par 13 «ولكن لما رآني عبداك آدم وحواء، سقطا على وجهيهما كأنهما ميتان. يا سيدي، ماذا نفعل بعبديك؟»

\par 14 ثم أشفق الله عليهم، وأظهر لهم الرحمة، وأرسل ملاكه ليحرس الجنة

\par 15 وكانت كلمة الرب إلى آدم وحواء، وأقامهما.

\par 16 فقال الرب لآدم: لقد قلت لك أنه في نهاية الخمسة أيام والنصف، سأرسل كلمتي وأخلصك.

\par 17 «فشدّد قلبك، وأقم في كهف الكنوز، الذي سبق أن تحدثت إليك عنه.»

\par 18 ولما سمع آدم هذه الكلمة من الله، تعزى بما قاله الله له. لأنه أخبره كيف سيخلصه

\chapter{4}

\par \textit{ينوح آدم على الظروف المتغيرة. يدخل آدم وحواء كهف الكنوز.}

\par 1 لكن آدم وحواء بكيا لخروجهما من الجنة، مسكنهما الأول

\par 2 وبالفعل، عندما نظر آدم إلى جسده المتغير، بكى بمرارة هو وحواء على ما فعلاه. وسارا ونزلا بهدوء إلى كهف الكنوز

\par 3 وعندما وصلوا إليه، بكى آدم على نفسه وقال لحواء: "انظري إلى هذا الكهف الذي سيكون سجننا في هذا العالم، ومكانًا للعقاب!"

\par 4 «ما هي مقارنة بالحديقة؟ ما هو ضيقها مقارنة بمساحة الأخرى؟»

\par 5 «ما هذه الصخرة، بجانب تلك البساتين؟ ما ظلمة هذا الكهف، مقارنة بنور الحديقة؟»

\par 6 «ما قيمة هذه الصخرة البارزة التي تحمينا، مقارنةً برحمة الرب التي ظللتنا؟»

\par 7 «ما قيمة تربة هذا الكهف مقارنةً بأرض الحديقة؟ هذه الأرض، المتناثرة بالحجارة؛ وتلك، المزروعة بأشجار الفاكهة اللذيذة؟»

\par 8 وقال آدم لحواء: "انظري إلى عينيكِ، وإلى عينيّ، اللتين سبق أن رأيتا ملائكة في السماء يسبحون؛ وهم أيضًا، بلا انقطاع."

\par 9 «لكننا الآن لا نرى كما كنا نرى. لقد صارت عيوننا جسدًا. لا تستطيع أن تبصر كما كانت تبصر من قبل.»

\par 10 قال آدم مرة أخرى لحواء: "ما هو جسدنا اليوم مقارنةً بما كان عليه في الأيام السابقة، حين كنا نسكن في الجنة؟"

\par 11 بعد ذلك، لم يُرِد آدم دخول الكهف، تحت الصخرة البارزة؛ ولم يكن ليدخله أبدًا

\par 12 لكنه رضخ لأوامر الله، وقال في نفسه: "ما لم أدخل الكهف، فسأعود مذنبًا مرة أخرى."

\chapter{5}

\par \textit{حيث ​​تقوم حواء بشفاعة نبيلة ومؤثرة، وتتحمل اللوم على نفسها.}

\par 1 ثم دخل آدم وحواء الكهف، ووقفا يصليان بلغتهما الخاصة، التي لا نعرفها، ولكنهما يعرفانها جيدًا.

\par 2 وبينما هما يصليان، رفع آدم عينيه، فرأى الصخرة وسقف الكهف الذي كان يستره من فوق، فلم يستطع أن يرى السماء ولا مخلوقات الله. فبكى وضرب على صدره بشدة حتى سقط وصار ميتًا.

\par 3 وجلست حواء تبكي لأنها كانت تعتقد أنه مات.

\par 4 ثم قامت ومدت يديها إلى الله داعية إياه بالرحمة والشفقة وقالت: يا الله اغفر لي خطيئتي التي ارتكبتها ولا تذكرها علي.

\par 5 "لأني وحدي تسببت في سقوط عبدك من الجنة إلى هذه الدار الضائعة، ومن النور إلى هذه الظلمة، ومن دار الفرح إلى هذا السجن."

\par 6 "يا الله، انظر إلى عبدك هذا الساقط، وأقمه من موته، حتى يبكي ويتوب عن خطيئته التي ارتكبها من خلالي."

\par 7 «لا تسلب روحه هذه المرة، بل دعه يحيا حتى يقف بعد مقدار توبته، ويفعل مشيئتك كما كان قبل موته.»

\par 8 «ولكن إن لم تُقمه، يا الله، خذ روحي لأكون مثله؛ ولا تتركني في هذا السجن، وحدي؛ لأني لم أستطع الوقوف وحدي في هذا العالم، إلا معه وحده.»

\par 9 «لأنك يا الله، جلبت عليه سباتًا، وأخذت عظمًا من جنبه، وأعدت اللحم مكانه، بقدرتك الإلهية.»

\par 10 «وأخذتني، أنا العظم، وجعلتني امرأة، مشرقة مثله، في القلب والعقل والكلام؛ وفي الجسد، مثله؛ وجعلتني على شبه وجهه، برحمتك وقدرتك.»

\par 11 يا رب، أنا وهو واحد، وأنت يا الله خالقنا، أنت الذي خلقتنا في يوم واحد

\par 12 «لذلك يا الله، أعطه الحياة، ليكون معي في هذه الأرض الغريبة، بينما نحن ساكنون فيها بسبب معصيتنا.»

\par 13 «ولكن إن لم تُحيه، فخذني أنا مثله؛ لنموت كلانا في اليوم نفسه.»

\par 14 فبكت حواء بكاءً مرًا، ووقعت على أبينا آدم من شدة حزنها

\chapter{6}

\par \textit{نصيحة الله لآدم وحواء، حيث يوضح كيف ولماذا أخطأوا.}

\par 1 لكن الله نظر إليهم، لأنهم قتلوا أنفسهم من حزن عظيم

\par 2 لكنه كان سيقيمهم ويعزيهم.

\par 3 فأرسل إليهم كلمته لكي يقفوا ويقوموا في الحال.

\par 4 وقال الرب لآدم وحواء: «لقد تجاوزتما بإرادتكما الحرة، حتى خرجتما من الجنة التي وضعتكما فيها».

\par 5 «بمحض إرادتك الحرة، تجاوزتَ الحدود من خلال رغبتك في الألوهية والعظمة والحالة السامية، مثلي؛ حتى أنني حرمتك من الطبيعة المشرقة التي كنتَ فيها آنذاك، وجعلتك تخرج من الجنة إلى هذه الأرض الوعرة والمليئة بالمتاعب.»

\par 6 «لو لم تتعدَّ وصيتي وحفظتَ شريعتي، ولم تأكل من ثمر الشجرة التي قلتُ لك لا تقترب! وكان في الجنة أشجار مثمرة خير من تلك.»

\par 7 "ولكن الشيطان الشرير الذي لم يثبت في حالته الأولى، ولم يحافظ على إيمانه، والذي لم يكن له نية طيبة نحوي، والذي على الرغم من أنني خلقته، إلا أنه استخف بي، وسعى إلى اللاهوت، حتى ألقيته من السماء، هو الذي جعل الشجرة تبدو لطيفة في عينيك، حتى أكلت منها، من خلال الاستماع إليه."

\par 8 "لقد خالفت وصيتي، ولذلك جلبت عليك كل هذه الأحزان."

\par 9 لأني أنا الله الخالق، الذي لم أقصد إهلاك مخلوقاتي حين خلقتهم. ولكن بعد أن أغضبوني غضبًا شديدًا، عاقبتهم بضرباتٍ قاسية حتى يتوبوا.

\par 10 «ولكن إن استمروا على العكس، متقسين في معصيتهم، فسيكونون تحت لعنة إلى الأبد.»

\chapter{7}

\par \textit{تصالح الوحوش.}

\par 1 عندما سمع آدم وحواء هذه الكلمات من الله، بكيا وناحا أكثر؛ ولكنهما عززا قلبيهما في الله، لأنهما الآن شعرا أن الرب كان لهما مثل الأب والأم؛ ولهذا السبب بالذات، بكيا أمامه، وطلبا الرحمة منه.

\par 2 ثم شفق الله عليهما، وقال: "يا آدم، لقد قطعت عهدي معك، ولن أرجع عنه، ولن أدعك ترجع إلى الجنة حتى يتم الوفاء بعهدي الخمسة أيام والنصف العظيمة."

\par 3 ثم قال آدم لله: "يا رب، أنت خلقتنا وجعلتنا صالحين للعيش في الجنة، وقبل أن أعصيك، جعلت كل الحيوانات تأتي إليّ لأسميها."

\par 4 «ثم كانت نعمتك عليّ، وسمّيت كل واحد حسب رأيك، وأخضعتهم جميعًا لي.»

\par 5 «ولكن الآن يا رب الإله، لأني تجاوزت أمرك، فستقوم عليّ جميع الوحوش وتلتهمني أنا وحواء أمتك، وتقطع حياتنا عن وجه الأرض.»

\par 6 «لذلك أتوسل إليك يا الله، بما أنك أخرجتنا من الجنة، وجعلتنا في أرض غريبة، فلا تدع الوحوش تؤذينا.»

\par 7 عندما سمع الرب هذه الكلمات من آدم، أشفق عليه، وشعر أنه قال حقًا إن وحوش الحقل ستنهض وتلتهمه هو وحواء، لأنه، الرب، كان غاضبًا عليهما بسبب تعديهما

\par 8 ثم أمر الله البهائم والطيور وكل ما يدب على الأرض أن يأتي إلى آدم ويتعرف عليه، ولا يزعجه ولا حواء، ولا أحدًا من الصالحين والصالحين من ذريتهم

\par 9 فسجدت الحيوانات لآدم حسب أمر الله، إلا الحية التي غضب الله عليها، فلم تأتِ إلى آدم مع الحيوانات

\chapter{8}

\par \textit{تُنزع "الطبيعة المشرقة" من الإنسان.}

\par 1 ثم بكى آدم وقال: "يا الله، عندما كنا نسكن في الجنة، وارتفعت قلوبنا، رأينا الملائكة يسبحون في السماء، لكننا الآن لا نرى كما اعتدنا أن نفعل؛ بل عندما دخلنا الكهف، اختفت كل الخليقة عنا."

\par 2 ثم قال الرب لآدم: «حين كنتَ خاضعًا لي، كانت في داخلك طبيعةٌ مُشرقة، ولذلك كنتَ ترى الأشياء البعيدة. ولكن بعد معصيتك، سُحبت منك طبيعتك المُشرقة؛ ولم يبقَ لك أن ترى الأشياء البعيدة، بل القريبة فقط؛ على قدر قدرة الجسد؛ لأنه وحشيّ».

\par 3 عندما سمع آدم وحواء هذه الكلمات من الله، ذهبا في طريقهما، يسبحانه ويعبدانه بقلب حزين.

\par 4 وانقطع الله عن الكلام معهم.

\chapter{9}

\par \textit{ماء من شجرة الحياة. آدم وحواء على وشك الغرق.}

\par 1 ثم خرج آدم وحواء من كهف الكنوز، واقتربا من باب الحديقة، ووقفا هناك ينظران إليه، وبكيا لأنهما خرجا منه

\par 2 وذهب آدم وحواء من أمام باب الجنة إلى جانبها الجنوبي، ووجدا هناك الماء الذي يسقي الجنة، من أصل شجرة الحياة، والذي ينقسم من هناك إلى أربعة أنهار على الأرض

\par 3 ثم جاءوا واقتربوا من ذلك الماء ونظروا إليه، فرأوا أنه الماء الخارج من تحت أصل شجرة الحياة في الجنة

\par 4 فبكى آدم ونوح وضرب على صدره لأنه انفصل عن الجنة، وقال لحواء:

\par 5 «لماذا جلبت عليّ وعلى نفسك وعلى نسلنا كل هذه الآفات والعقوبات؟»

\par 6 فقالت له حواء: «ماذا رأيت حتى تبكي وتكلمني بهذا الشكل؟»

\par 7 وقال لحواء: «أما تنظرين إلى هذا الماء الذي كان عندنا في الجنة، الذي كان يسقي أشجار الجنة، ويتدفق من هناك؟»

\par 8 «ونحن، حين كنا في الجنة، لم نكن نهتم بها؛ ولكن منذ أن أتينا إلى هذه الأرض الغريبة، أحببناها، واستخدمناها لجسدنا.»

\par 9 ولكن عندما سمعت حواء هذه الكلمات منه، بكت؛ ومن شدة بكائهم، سقطوا في ذلك الماء؛ وكانوا سيضعون حدًا لأنفسهم فيه، حتى لا يعودوا أبدًا لينظروا إلى الخليقة؛ لأنهم عندما نظروا إلى عمل الخلق، شعروا أنه يجب عليهم وضع حد لأنفسهم

\chapter{10}

\par \textit{تحتاج أجسادهم إلى الماء بعد خروجهم من الجنة.}

\par 1 ثم نظر الله الرحيم والرؤوف إليهم وهم على هذه الحال في الماء، وعلى وشك الموت، فأرسل ملاكاً فأخرجهم من الماء ووضعهم على شاطئ البحر كأموات.

\par 2 ثم صعد الملاك إلى الله، فاستقبله، وقال: "يا الله، لقد لفظت مخلوقاتك أنفاسها الأخيرة".

\par 3 ثم أرسل الله كلمته إلى آدم وحواء، فأقامهما من موتهما.

\par 4 وقال آدم بعد رفعه: يا رب، حين كنا في الجنة لم نكن نحتاج إلى هذا الماء ولا نهتم به، ولكن منذ أن أتينا إلى هذه الأرض لا نستطيع الاستغناء عنه.

\par 5 ثم قال الله لآدم: "عندما كنت تحت قيادتي وكنت ملاكًا نبيلًا، لم تكن تعرف هذا الماء".

\par 6 "ولكن بعد أن خالفت وصيتي، لا يمكنك الاستغناء عن الماء، الذي تغسل به جسدك وتنميه؛ لأنه الآن مثل جسد البهائم، ويحتاج إلى الماء."

\par 7 عندما سمع آدم وحواء هذه الكلمات من الله، بكيا صرخة مريرة، وطلب آدم من الله أن يسمح له بالعودة إلى الجنة، وينظر إليها مرة ثانية.

\par 8 ولكن قال الله لآدم: «لقد قطعت لك وعداً، وعندما يتحقق هذا الوعد، سأعيدك إلى الجنة أنت ونسلك الصالح».

\par 9 وتوقف الله عن التواصل مع آدم.

\chapter{11}

\par \textit{ذكريات عن الأيام المجيدة في الحديقة.}

\par 1 ثم شعر آدم وحواء بأنهما يحترقان من العطش والحرارة والحزن

\par 2 وقال آدم لحواء: "لن نشرب من هذا الماء، حتى لو متنا. يا حواء، عندما يدخل هذا الماء إلى أحشائنا، فإنه سيزيد من عقابنا وعقاب أبنائنا الذين سيأتون بعدنا."

\par 3 ثم انسحب آدم وحواء من الماء، ولم يشربا منه إطلاقًا؛ بل أتيا ودخلا كهف الكنوز

\par 4 ولكن عندما كان آدم في الداخل، لم يستطع رؤية حواء؛ بل سمع فقط صوتها. كما لم تستطع هي رؤية آدم، لكنها سمعت صوته

\par 5 فبكى آدم بكاءً شديدًا، وقرع على صدره، وقام وقال لحواء: «أين أنتِ؟»

\par 6 فقالت له: «ها أنا واقفة في هذه الظلمة.»

\par 7 ثم قال لها: تذكري الطبيعة المشرقة التي كنا نعيش فيها عندما كنا نقيم في الجنة!

\par 8 «يا حواء! تذكري المجد الذي حل علينا في الجنة. يا حواء! تذكري الأشجار التي ظللتنا في الجنة بينما كنا نتحرك بينها.»

\par 9 يا حواء! تذكري أننا في الجنة، لم نعرف ليلًا ولا نهارًا. تذكري شجرة الحياة، التي تدفق الماء من تحتها، وأضاءت علينا! تذكري يا حواء أرض الجنة، وضيائها!

\par 10 «فكّر، فكّر في تلك الحديقة التي لم يكن فيها ظلام، حين كنا نعيش فيها.»

\par 11 «ما إن دخلنا كهف الكنوز هذا حتى أحاط بنا الظلام؛ حتى لم نعد نرى بعضنا البعض؛ وانتهت كل متعة هذه الحياة.»

\chapter{12}

\par \textit{كيف حل الظلام بين آدم وحواء.}

\par 1 ثم ضرب آدم على صدره، هو وحواء، وبكيا طوال الليل حتى اقترب الفجر، وتنهدا طوال الليل في ميازيا

\par 2 فضرب آدم نفسه، وألقى نفسه على الأرض في الكهف، من شدة الحزن، ومن الظلمة، وظل هناك كميت.

\par 3 وسمعت حواء صوت سقوطه على الأرض، فتحسسته بيديها، فوجدتْه كالجثة.

\par 4 ثم شعرت بالخوف، ولم تستطع النطق، وبقيت بجانبه.

\par 5 ولكن الرب الرحيم نظر إلى موت آدم، وإلى صمت حواء خوفاً من الظلمة.

\par 6 وجاءت كلمة الله إلى آدم وأقامته من موته، وفتحت فم حواء لتتكلم.

\par 7 ثم قام آدم في الكهف وقال: يا رب، لماذا غاب النور عنا، وحلت علينا الظلمة؟ لماذا تتركنا في هذا الظلام الطويل؟ لماذا تعذبنا هكذا؟

\par 8 يا رب، أين كان هذا الظلام حين حلّ بنا؟ إنه ظلامٌ لا نرى فيه بعضنا بعضًا.

\par 9 فطالما كنا في الجنة، لم نرَ ولا عرفنا ما هو الظلام. لم أكن مختبئًا عن حواء، ولا هي مختبئة عني، حتى الآن وقد عجزت عن رؤيتي؛ ولم يطل بنا الظلام ليفصلنا عن بعضنا البعض.

\par 10 «لكنني وهي كنا في ضوء واحد ساطع. رأيتها ورأتني. ولكن منذ أن دخلنا هذا الكهف، حلّ علينا الظلام، وفرقنا، حتى أنني لا أراها ولا تراني.»

\par 11 «يا رب، هل ستعذبنا بهذا الظلام؟»

\chapter{13}

سقوط آدم. لماذا خُلِقَ الليل والنهار؟

\par 1 ثم لما سمع الله الرحيم والرؤوف صوت آدم، قال له:

\par 2 «يا آدم، ما دام الملاك الصالح مطيعًا لي، فقد أضاء عليه وعلى جحافله نور ساطع.»

\par 3 «ولكن عندما تجاوز وصيتي، حرمته من تلك الطبيعة المشرقة، وأصبح مظلمًا.»

\par 4 «وعندما كان في السماوات، في عوالم النور، لم يكن يعرف شيئًا عن الظلمة.»

\par 5 «لكنه تعدى، فأسقطته من السماء إلى الأرض، فأتى عليه هذا الظلام.»

\par 6 «وعليك يا آدم، وأنت في جنتي مطيعًا لي، استقر ذلك النور الساطع أيضًا.»

\par 7 «ولكن عندما سمعتُ بذنوبك، حرمتك من ذلك النور الساطع. ومع ذلك، من رحمتي، لم أحوّلك إلى ظلمة، بل صنعتُ لك جسدًا من لحم، وبسطتُ عليه هذا الجلد، ليتحمل البرد والحرارة.»

\par 8 «لو أنزلت غضبي عليك بشدة، لكنت أهلكتك؛ ولو حولتك إلى ظلمات، لكان الأمر كما لو أنني قتلتك.»

\par 9 «ولكن برحمتي، خلقتك كما أنت؛ عندما خالفت وصيتي يا آدم، طردتك من الجنة، وأخرجتك إلى هذه الأرض؛ وأمرتك أن تسكن في هذا الكهف؛ فحل عليك الظلام كما حل على من خالف وصيتي.»

\par 10 هكذا يا آدم، خدعتك هذه الليلة. إنها لن تدوم إلى الأبد، بل هي اثنتي عشرة ساعة فقط، وعندما تنقضي، يعود ضوء النهار.

\par 11 "لذلك لا تتنهد ولا تتحرك، ولا تقل في قلبك أن هذا الظلام طويل ويستمر بإرهاق، ولا تقل في قلبك أنني أزعجك به."

\par 12 «شَدِّدْ قَلْبَكَ، وَلا تَخَفْ. هَذِهِ الظُّلْمَةُ لَيْسَتْ عَذَابًا. لَكِنْ يَا آدَمَ، قَدْ خَلَقْتُ النُّهَارَ، وَوَضَعْتُ الشَّمْسَ فِيهِ لِتُضِيئَهُ، لِتَقُومَ أَنْتَ وَأَبْنَاؤُكَ بِأَعْمَالِكُمْ.»

\par 13 «لأني كنت أعلم أنك ستخطئ وتتعدى، وتخرج إلى هذه الأرض. ومع ذلك، لن أجبرك، ولن يُسمع صوتك، ولن أحبسك؛ ولن أحكم عليك بسقوطك؛ ولا بخروجك من النور إلى الظلمة؛ ولا حتى بخروجك من الجنة إلى هذه الأرض.»

\par 14 «لأني خلقتك من النور، وأردت أن أخرج منك أبناء نور وأمثالك.»

\par 15 "ولكنك لم تحفظ وصيتي يومًا واحدًا، حتى أكملت الخليقة وباركت كل ما فيها."

\par 16 «ثم أوصيتك بشأن الشجرة أن لا تأكل منها. ولكني علمت أن الشيطان الذي خدع نفسه سيخدعك أيضًا.»

\par 17 «فأعلمتك من خلال الشجرة ألا تقترب منها. وقلت لك ألا تأكل من ثمرها، ولا تتذوق منه، ولا تجلس تحتها، ولا تلمسها.»

\par 18 «لو لم أكن قد كلمتك يا آدم بشأن الشجرة، وتركتك بلا أمر، وأخطأت، لكان ذلك إثمًا مني، لعدم إعطائي أي أمر؛ لَتَفَرَّغْتَ وَلُوْمتني عليه.»

\par 19 «لكنني أمرتك، وحذرتك، فسقطت. لذلك لا يمكن لخليقتي أن تلومني؛ بل يقع اللوم عليهم وحدهم.»

\par 20 «و يا آدم، لقد جعلت النهار لك ولأبنائك من بعدك، ليعملوا ويكدحوا فيه. وجعلت الليل ليسكنوا فيه من أعمالهم، وليخرج حيوان الحقل ليلًا ويطلب طعامه.»

\par 21 «لكن لم يبقَ الآن إلا القليل من الظلام يا آدم؛ وسيظهر ضوء النهار قريبًا.»

\chapter{14}

\par \textit{أقدم نبوءة عن مجيء المسيح.}

\par 1 ثم قال آدم لله: يا رب، خذ نفسي ولا تدعني أرى هذا الظلام بعد، أو انقلني إلى مكان ليس فيه ظلام.

\par 2 لكن الرب الإله قال لآدم: «الحق أقول لك: إن هذه الظلمة ستزول عنك كل يوم حددته لك، حتى يتم عهدي؛ عندما أخلصك وأعيدك إلى الجنة، إلى مسكن النور الذي تشتاق إليه، حيث لا ظلمة. سأعيدك إليه - إلى ملكوت السماوات».

\par 3 قال الله لآدم مرة أخرى: "كل هذا البؤس الذي أُجبرت على تحمله بسبب معصيتك، لن يحررك من يد الشيطان، ولن يخلصك."

\par 4 «لكنني سأفعل. عندما أنزل من السماء، وأصبح جسدًا من نسلك، وأتحمل الضعف الذي تعاني منه، فإن الظلمة التي حلت عليك في هذا الكهف ستحل عليّ في القبر، عندما أكون في جسد نسلك.»

\par 5 «وأنا، الذي ليس لي سنون، سأخضع لحساب السنين والأوقات والشهور والأيام، وسأُحسب كواحد من أبناء البشر، لكي أخلصك.»

\par 6 وتوقف الله عن التواصل مع آدم.

\chapter{15}

\par 1 فبكى آدم وحواء وحزنا بسبب كلمة الله لهما، أنهما لن يعودا إلى الجنة حتى تتم الأيام المحددة لهما؛ ولكن في الأغلب لأن الله قال لهما أنه سوف يتألم من أجل خلاصهما.

\chapter{16}

\par \textit{أول شروق شمس. يعتقد آدم وحواء أنها نار قادمة لتحرقهما.}

\par 1 وبعد ذلك ظل آدم وحواء واقفين في الكهف يصليان ويبكان حتى أشرق عليهما الصباح.

\par 2 فلما رأوا النور رجع إليهم، ارتعدوا خوفاً وشددوا قلوبهم.

\par 3 ثم بدأ آدم بالخروج من الكهف. وعندما وصل إلى فمه، ووقف وأدار وجهه نحو الشرق، ورأى الشمس تشرق بأشعتها المتوهجة، وشعر بحرارتها على جسده، خاف منها، وظن في قلبه أن هذا اللهب خرج ليعذبه

\par 4 فبكى ثم ضرب على صدره وسقط على وجهه على الأرض، وقدم طلبه قائلًا:

\par 5 «يا رب، لا تُوبِّخني، ولا تُفنيني، ولا تُزيل حياتي من الأرض.»

\par 6 لأنه كان يعتقد أن الشمس هي الله.

\par 7 بما أنه بينما كان آدم في الجنة وسمع صوت الله والصوت الذي أحدثه في الجنة وخافه، لم يرَ نور الشمس الساطع قط، ولم تمس حرارة الشمس الملتهبة جسده

\par 8 لذلك كان يخاف من الشمس عندما تصل إليه أشعتها المشتعلة. ظن أن الله ينوي أن يعذبه بها كل الأيام التي قدرها له

\par 9 لأن آدم قال أيضًا في أفكاره: كما أن الله لم يُعذبنا بالظلمة، هوذا قد جعل هذه الشمس تشرق وتُعذبنا بحرارة مُحرقة

\par 10 ولكن بينما كان يفكر هكذا في قلبه، جاءته كلمة الله وقالت:

\par 11 يا آدم، قم وقف. هذه الشمس ليست إلهًا؛ لكنها خُلقت لتنير نهارًا، وهو ما كلمتك عنه في الكهف قائلًا: «أن يشرق الفجر، ويكون هناك نور نهارًا».

\par 12 «لكنني الله الذي عزّيتك في الليل.»

\par 13 وتوقف الله عن التواصل مع آدم.



\chapter{17}

\par \textit{فصل الثعبان.}

\par 1 ثم خرج آدم وحواء من باب الكهف وذهبا نحو الجنة.

\par 2 ولكن عندما اقتربوا منها، أمام البوابة الغربية، التي خرج منها الشيطان عندما خدع آدم وحواء، وجدوا الحية التي أصبحت الشيطان آتية إلى البوابة، تلعق التراب بحزن، وتتلوى على صدرها على الأرض، بسبب اللعنة التي سقطت عليها من الله

\par 3 وبينما كان الثعبان في السابق أعظم كل الحيوانات، فقد تغير الآن وأصبح زلقًا وأحطهم جميعًا، ويزحف على صدره ويمشي على بطنه.

\par 4 وبينما كان أجمل الحيوانات، فقد تحوّل وأصبح أبشعها. فبدلاً من أن يتغذى على أجود الطعام، أصبح يأكل التراب. وبدلًا من أن يسكن، كما كان من قبل، في أرقى الأماكن، أصبح الآن يعيش في التراب.

\par 5 وبينما كان أجمل كل الحيوانات، والتي كانت كلها تقف صامتة أمام جماله، أصبح الآن مكروهًا منهم.

\par 6 "وبعد ذلك، بينما كان يسكن في مسكن جميل، كانت جميع الحيوانات الأخرى تأتي من أماكن أخرى؛ وكانت تشرب منه أيضًا؛ الآن، بعد أن أصبح سامًا، بسبب لعنة الله، هربت جميع الحيوانات من مسكنه، ولم تشرب من الماء الذي يشربه؛ بل هربت منه.

\chapter{18}

\par \textit{المعركة المميتة مع الثعبان.}

\par 1 عندما رأى الثعبان الملعون آدم وحواء، انتفخ رأسه، ووقف على ذيله، وبعينين حمراوين كالدم، فعل كما لو أنه سيقتلهما

\par 2 اتجهت مباشرة نحو حواء، وركضت خلفها؛ بينما كان آدم واقفًا يبكي لأنه لم يكن لديه عصا في يده ليضرب بها الحية، ولم يكن يعرف كيف يقتلها

\par 3 ولكن بقلبٍ يحترق لحواء، اقترب آدم من الحية، وأمسكها من ذيلها؛ عندما التفتت نحوه وقالت له:

\par 4 «يا آدم، بسببك وبسبب حواء، أنا زلق، وأسير على بطني.» ثم بسبب قوته العظيمة، ألقى آدم وحواء أرضًا وضغط عليهما، كما لو كان سيقتلهما

\par 5 لكن الله أرسل ملاكًا فطرح الحية عنهم وأقامهم

\par 6 ثم جاء كلام الله إلى الحية وقال لها: "في المرة الأولى جعلتك طليقة اللسان، وجعلتك تمشي على بطنك، لكنني لم أحرمك من النطق."

\par 7 «الآن، كن أخرسًا؛ ولا تتكلم أكثر من ذلك، أنت وجنسك؛ لأنه في المقام الأول، حدث هلاك مخلوقاتي بسببك، والآن تريد قتلهم.»

\par 8 فَضُرِبَتِ الْحَيَّةُ بِخَمْسٍ، وَلَمْ تَتَكَلَّمْ بَعْدُ.

\par 9 فجاءت ريح من السماء بأمر الله فحملت الحية من آدم وحواء وألقتها على شاطئ البحر فهبطت في الهند.

\chapter{19}

\par \textit{الوحوش الخاضعة لآدم.}

\par 1 لكن آدم وحواء بكيا أمام الله. وقال له آدم:

\par 2 يا رب، حين كنت في المغارة، قلت لك يا سيدي، أن وحوش الحقل ستقوم وتلتهمني، وتقطع حياتي من الأرض.

\par 3 "ثم ضرب آدم على صدره مما أصابه، فسقط على الأرض كالجثة، ثم أتاه كلمة الله فأقامه، وقال له:

\par 4 "يا آدم، لن يتمكن واحد من هذه الحيوانات من إيذائك؛ لأنه عندما جعلت الحيوانات والأشياء المتحركة الأخرى تأتي إليك في الكهف، لم أسمح للحية أن تأتي معها، لئلا تقوم عليك، وتجعلك ترتجف؛ ويقع الخوف منها في قلوبكم."

\par 5 "لأني كنت أعلم أن ذلك الملعون شرير، لذلك لم أسمح له أن يقترب إليك مع سائر الحيوانات."

\par 6 "ولكن الآن شدد قلبك ولا تخف. أنا معك إلى نهاية الأيام التي حددتها لك."

\chapter{20}

\par \textit{آدم يرغب في حماية حواء.}

\par 1 ثم بكى آدم وقال: "يا الله، انقلنا إلى مكان آخر، لئلا تقترب منا الحية مرة أخرى، وتثور علينا. لئلا تجد أمتك حواء وحدها وتقتلها؛ لأن عينيها قبيحتان وشريرتان."

\par 2 ولكن قال الله لآدم وحواء: لا تخافوا بعد الآن، لن أدعه يقترب منكم. لقد طردته عنكم من هذا الجبل، ولن أترك فيه شيئاً يؤذيكم.

\par 3 ثم سجد آدم وحواء أمام الله وشكراه وسبحاه لأنه خلصهما من الموت.

\chapter{21}

\par \textit{آدم وحواء يحاولان الانتحار.}

\par 1 ثم ذهب آدم وحواء بحثًا عن الجنة.

\par 2 وكانت الحرارة تضرب كلهب على وجوههم، وكانوا يعرقون من الحر ويبكون أمام الرب.

\par 3 وأما المكان الذي بكوا فيه فكان قريبًا من جبل عالٍ، مواجهًا للباب الغربي للبستان

\par 4 ثم ألقى آدم بنفسه من أعلى ذلك الجبل، وكان وجهه محمرًا ولحمه مسلوخًا، وسال منه دم كثير، وكان على وشك الموت

\par 5 في هذه الأثناء، بقيت حواء واقفة على الجبل تبكي عليه، مستلقية

\par 6 فقالت: «لا أريد أن أعيش بعده، لأن كل ما فعل بنفسه كان من خلالي».

\par 7 ثم ألقت بنفسها وراءه، فمزقتها الحجارة وخدشتها، وبقيت ملقاة كأنها ميتة

\par 8 لكن الله الرحيم، الذي ينظر إلى مخلوقاته، نظر إلى آدم وحواء وهما ميتان، وأرسل إليهما كلمته، فأقامهما

\par 9 وقال لآدم: "يا آدم، كل هذا البؤس الذي جلبته على نفسك لن ينفع ضد حكمي، ولن يغير عهد الـ 5500 سنة."

\chapter{22}

\par \textit{آدم في مزاج شهم.}

\par 1 ثم قال آدم لله: "أنا أذبل من الحر، وأضعف من المشي، وأكره هذا العالم. ولا أعلم متى تُخرجني منه لأرتاح."

\par 2 فقال له الرب الإله: يا آدم، لا يمكن أن يكون هذا الآن حتى تكمل أيامك. حينئذٍ أخرجك من هذه الأرض البائسة.

\par 3 فقال آدم لله: حين كنت في الجنة لم أعرف حراً ولا ضعفاً ولا ارتعاشاً ولا خوفاً، ولكن الآن منذ جئت إلى هذه الأرض أصابني كل هذا البؤس.

\par 4 ثم قال الله لآدم: «ما دمتَ تحفظ وصيتي، فقد استقرّ عليك نوري ونعمتي. ولكن عندما خالفتَ وصيتي، أصابك الحزن والبؤس في هذه الأرض».

\par 5 فبكى آدم وقال: يا رب لا تقطعني لهذا السبب، ولا تضربني بضربات ثقيلة، ولا تجازني حسب خطيئتي. لأننا خالفنا وصيتك بإرادتنا، وتركنا شريعتك، وسعينا إلى أن نصبح آلهة مثلك، عندما خدعنا الشيطان العدو.

\par 6 ثم قال الله لآدم أيضًا: "لأنك تحملت الخوف والرعدة في هذه الأرض، والضعف والمعاناة وأنت تطأ وتمشي، وتصعد هذا الجبل، وتموت منه، فإني آخذ كل هذا على نفسي لكي أخلصك".

\chapter{23}

\par \textit{آدم وحواء يحزمان أنفسهما ويصنعان أول مذبح بُني على الإطلاق.}

\par 1 فبكى آدم مرة أخرى وقال: يا الله ارحمني حتى أتحمل عليك ما أريد أن أفعله.

\par 2 لكن الله أخذ كلمته من آدم وحواء.

\par 3 فقام آدم وحواء على أقدامهما، فقال آدم لحواء: «تمنطقي فأتمنطق أنا أيضًا». فتمنطقت كما قال لها آدم.

\par 4 ثم أخذ آدم وحواء حجارة ووضعاها على شكل مذبح، وأخذا أوراقاً من الأشجار خارج الجنة، ومسحا بها عن وجه الصخرة الدم الذي سفكاه.

\par 5 وأما ما سقط على الرمل، فأخذوه مع التراب الذي امتزج به، وقدموه على المذبح قربانًا لله

\par 6 ثم وقف آدم وحواء تحت المذبح وبكيا، متوسلين إلى الله قائلين: "اغفر لنا ذنوبنا وخطايانا، وانظر إلينا بعين رحمتك. لأنه عندما كنا في الجنة، كانت تسابيحنا وترانيمنا ترتفع أمامك بلا انقطاع."

\par 7 «ولكن عندما أتينا إلى هذه الأرض الغريبة، لم يعد لنا التسبيح الخالص، ولا الصلاة الصالحة، ولا القلوب المتفهمة، ولا الأفكار الحلوة، ولا النصائح العادلة، ولا الفطنة الطويلة، ولا المشاعر المستقيمة، ولم تفارقنا طبيعتنا المشرقة. لكن جسدنا تغير عن الصورة التي كان عليها في البداية، عندما خُلقنا.»

\par 8 «ولكن انظر الآن إلى دمائنا التي تُقدم على هذه الحجارة، واقبلها من أيدينا، مثل التسبيح الذي اعتدنا أن نغنيه لك في البداية، عندما كنا في البستان.»

\par 9 وبدأ آدم يطلب المزيد من الله.

\par \textit{الحواشي السفلية}

16:1 الأصل في صلاة الرب يقال أنها استخدمت قبل ربنا بحوالي 150 عامًا: أبانا الذي في السماء، ارحمنا، يا رب إلهنا، ليتقدس اسمك، وليتمجد ذكرك في السماء من فوق وعلى الأرض هنا في الأسفل.

ليملك علينا ملكوتك الآن وإلى الأبد. قال القديسون القدماء: "اغفر واصفح لجميع الناس عما فعلوه بي. ولا تدخلنا في تجربة، لكن نجنا من الشرير؛ لأن لك الملك، وستملك بمجد إلى أبد الآبدين، آمين."

\chapter{24}

\par \textit{نبوءة حية عن حياة المسيح وموته.}

\par 1 ١٠ نظر الله الرحيم، الصالح ومحب البشر، إلى آدم وحواء، وإلى دميهما اللذين قدماهما قربانًا له دون أمر منه. لكنه تعجب منهما، وقبل قربانهما.

\par 2 فأرسل الله من حضرته نارًا مضيئة، فأكلت تقدمتهم

\par 3 شمّ رائحة قربانهم الطيبة، وأظهر لهم الرحمة.

\par 4 ثم جاء كلمة الله إلى آدم، وقال له: "يا آدم، كما سفكت دمك، سأسفك دمي عندما أصبح لحمًا من نسلك؛ وكما مت يا آدم، سأموت أيضًا. وكما بنيت مذبحًا، سأصنع لك مذبحًا على الأرض؛ وكما قدمت دمك عليه، سأقدم دمي أيضًا على مذبح على الأرض".

\par 5 "وكما طلبت المغفرة من خلال ذلك الدم، هكذا سأجعل دمي أيضًا غفرانًا للخطايا، وأمحو الخطايا التي به."

\par 6 والآن، ها أنا قد قبلتُ قربانك يا آدم، ولكن أيام العهد الذي ألزمتك به لم تكتمل. وعندما تكتمل، أُعيدك إلى الجنة.

\par 7 «والآن، شدد قلبك، وعندما يأتي عليك الحزن، قدم لي تقدمة، وسأكون راضيًا عنك.»

\chapter{25}

\par \textit{يُمَثَّل الله بالرحيم والمحب. إقامة العبادة.}

\par 1 لكن الله علم أن آدم كان يفكر في أن يقتل نفسه كثيرًا ويقدم دمه ذبيحة له

\par 2 لذلك قال له: «يا آدم، لا تقتل نفسك مرة أخرى كما فعلت، بإلقاء نفسك من ذلك الجبل».

\par 3 فقال آدم لله: «لقد كان في نيتي أن أقضي على نفسي فورًا، لأني خالفت وصاياك، ولأني خرجت من الجنة الجميلة، ولأني حرمتني من النور الساطع، ولأني كنت أسبح بحمدك، ولأني كنت أسبح بحمدك، ولأني كنت أسبح بحمدك.»

\par 4 «ولكن يا الله، لا تقتلني كليًا بفضلك، بل كن لي عونًا في كل مرة أموت فيها، وأحييني.»

\par 5 «وبذلك يُعرَف أنك إله رحيم، لا يشاء هلاك أحد؛ ولا يحب سقوط أحد؛ ولا يدين أحدًا بقسوة أو سوء أو بالهلاك الكامل.»

\par 6 ثم صمت آدم.

\par 7 وجاءت إليه كلمة الله، وباركته، وعزّاه، وعاهده أن يخلصه في نهاية الأيام المرسومة عليه

\par 8 كانت هذه إذن أول قربان قدمه آدم لله، وهكذا أصبحت عادته أن يفعل

\chapter{26}

\par \textit{نبوءة جميلة عن الحياة الأبدية والفرح (الآية 15). حلول الليل.}

\par 1 ثم أخذ آدم حواء، وشرعا في العودة إلى مغارة الكنوز حيث كانا يقيمان. ولكن عندما اقتربا منها ورأياها من بعيد، وقع على آدم وحواء حزن شديد عندما نظروا إليها.

\par 2 فقال آدم لحواء: «حين كنا على الجبل كنا نتعزى بكلمة الله التي كانت تكلّمنا، والنور الذي جاء من المشرق أشرق علينا».

\par 3 "ولكن الآن كلمة الله مخفية عنا، والنور الذي أشرق علينا تغير حتى اختفى، وسمح للظلام والحزن أن يحلا علينا."

\par 4 "ونحن مضطرون إلى دخول هذا الكهف الذي هو بمثابة سجن، حيث يغطينا الظلام، حتى انفصلنا عن بعضنا البعض؛ فلا تراني ولا أراك."

\par 5 ولما قال آدم هذه الكلمات بكى وبسطا أيديهما أمام الله، لأنهما كانا حزينين للغاية.

\par 6 فتوسلوا إلى الله أن يُشرق عليهم الشمس، فلا يعود عليهم الظلام، ولا يعودوا إلى هذا الغطاء الصخري. وفضلوا الموت على رؤية الظلام.

\par 7 ثم نظر الله إلى آدم وحواء، وإلى حزنهما العظيم، وإلى كل ما فعلاه بقلب حار، بسبب كل المتاعب التي كانا فيها، بدلاً من رفاهيتهما السابقة، وبسبب كل البؤس الذي حل بهما في أرض غريبة.

\par 8 لذلك لم يكن الله غاضبًا عليهم، ولا قليل الصبر عليهم، بل كان طويل الأناة وصابرًا عليهم، كما هو الحال مع الأطفال الذين خلقهم

\par 9 ثم جاء كلام الله إلى آدم، وقال له: "يا آدم، أما الشمس، فلو أخذتها وأحضرتها إليك، فستذهب الأيام والساعات والسنين والأشهر سدى، ولن يتم الوفاء بالعهد الذي قطعته معك أبدًا."

\par 10 «ولكنك حينها ستُترك في وباء طويل، ولن يبقى لك خلاص إلى الأبد.»

\par 11 «بل اصبر وهدئ نفسك ما دمت ليلًا ونهارًا، حتى تتم الأيام ويأتي وقت عهدي.»

\par 12 «حينئذٍ آتي وأخلصك يا آدم، لأني لا أريد أن تتألم.»

\par 13 «وعندما أنظر إلى كل الأشياء الجيدة التي عشت فيها، ولماذا خرجت منها، عندها أود أن أرحمك طوعًا.»

\par 14 «ولكن لا يمكنني تغيير العهد الذي خرج من فمي، وإلا لكنت أعدتك إلى الجنة.»

\par 15 «ولكن عندما يتم الوفاء بالعهد، سأريك ولنسلك الرحمة، وأدخلك إلى أرض فرح، حيث لا حزن ولا معاناة؛ بل فرح وسرور دائمان، ونور لا ينطفئ، وتسابيح لا تنقطع؛ وجنة جميلة لا تزول أبدًا.»

\par 16 وقال الله لآدم مرة أخرى: "تأنَّ وادخل الكهف، لأن الظلام الذي كنت تخاف منه لن يستمر سوى اثنتي عشرة ساعة، وعندما ينتهي، سيشرق النور."

\par 17 فلما سمع آدم هذه الكلمات من الله، سجد هو وحواء أمامه، فتعزت قلوبهما. رجعا إلى الكهف كعادتهما، والدموع تنهمر من أعينهما، والحزن والنحيب ينبعثان من قلوبهما، وتمنّيا لو أن روحهما تفارق جسدهما

\par 18 ووقف آدم وحواء يصليان، حتى أقبل عليهما ظلام الليل، فاختفى آدم عن حواء، واختفت هي عنه

\par 19 وظلوا قائمين يصلون.

\chapter{27}

الإغراء الثاني لآدم وحواء. الشيطان يتخذ شكل نور ساحر.

\par 1 عندما رأى الشيطان، كاره كل خير، كيف استمروا في الصلاة، وكيف تواصل الله معهم، وعزّاهم، وكيف قبل قربانهم، ظهر الشيطان

\par 2 بدأ بتحويل جيوشه؛ كانت في يديه نار متوهجة، وكانوا في نور عظيم

\par 3 ثم وضع عرشه بالقرب من فم الكهف لأنه لم يستطع دخوله بسبب صلواتهم. وألقى نورًا في الكهف، حتى أشرق الكهف على آدم وحواء؛ بينما بدأ جنوده يترنمون بالتسبيح

\par 4 وفعل الشيطان هذا، حتى عندما يرى آدم النور، يعتقد في نفسه أنه نور سماوي، وأن جنود الشيطان هم ملائكة، وأن الله أرسلهم ليراقبوا الكهف، وليعطوه نورًا في الظلمة

\par 5 حتى إذا خرج آدم من الكهف ورآهم، وسجد آدم وحواء للشيطان، فإنه بذلك يتغلب على آدم، ويذله مرة ثانية أمام الله

\par 6 لذلك، عندما رأى آدم وحواء النور، ظنّا أنه حقيقي، شددا قلبيهما؛ ومع ذلك، بينما كانا يرتجفان، قال آدم لحواء:

\par 7 «انظروا إلى ذلك النور العظيم، وإلى تلك الترانيم الكثيرة، وإلى ذلك الجيش الواقف في الخارج الذي لا يأتي إلينا، لا تخبرونا بما يقولون، أو من أين يأتون، أو ما معنى هذا النور؛ ما هي تلك الترانيم؛ لماذا أُرسلوا إلى هنا، ولماذا لا يأتون.»

\par 8 «لو كانوا من عند الله لأتونا في الكهف، وأخبرونا بأمرهم.»

\par 9 ثم قام آدم وصلى إلى الله بقلب حار، وقال:

\par 10 «يا رب، هل يوجد في العالم إله آخر غيرك، الذي خلق الملائكة وملأهم بالنور، وأرسلهم ليحفظونا، من سيأتي معهم؟»

\par 11 «لكن، انظروا، نرى هذه الجيوش تقف عند مدخل الكهف؛ إنهم في نور عظيم؛ وهم يُغنون تسبيحًا عاليًا. إن كانوا من إله غيرك، فأخبرني؛ وإن كانوا مرسلين منك، فأخبرني عن سبب إرسالهم.»

\par 12 ما إن قال آدم هذا، حتى ظهر له ملاك من الله في الكهف، وقال له: "يا آدم، لا تخف. هذا هو الشيطان وجنوده؛ يريد أن يخدعك كما خدعك في البداية. لأول مرة، كان مختبئًا في الحية؛ لكنه هذه المرة جاء إليك في صورة ملاك نور؛ حتى يفتنك عندما تعبده، في حضرة الله."

\par 13 ثم انطلق الملاك من عند آدم، وأمسك بالشيطان عند مدخل الكهف، وجرده من الخدعة التي كان يتخذها، وأحضره في صورته البشعة إلى آدم وحواء؛ اللذين كانا خائفين منه عندما رأياه

\par 14 فقال الملاك لآدم: "هذا الشكل البشع كان عليه منذ أن أسقطه الله من السماء. لم يكن ليقترب منك به؛ لذلك تحول إلى ملاك نور."

\par 15 ثم طرد الملاك الشيطان وجنوده من آدم وحواء، وقال لهما: "لا تخافا؛ الله الذي خلقكما هو الذي سيقويكما."

\par 16 ثم ذهب الملاك من عندهم.

\par 17 أما آدم وحواء فبقيا واقفين في الكهف، ولم يأتِهما أي عزاء، بل كانا منقسمين في أفكارهما

\par 18 ولما كان الصباح صلّوا، ثم خرجوا ليبحثوا عن الحديقة. لأن قلوبهم كانت متجهة إليها، ولم يجدوا أي عزاء لمغادرتهم إياها

\chapter{28}

\par \textit{يتظاهر الشيطان بإرشاد آدم وحواء إلى الماء للاستحمام.}

\par 1 ولكن لما رآهم الشيطان الماكر أنهم ذاهبون إلى الجنة، جمع جيشه وظهر على سحابة لكي يضلهم.

\par 2 ولكن عندما رآه آدم وحواء هكذا في رؤيا، ظنّا أنهما ملائكة الله جاءا لتعزيتهما بشأن مغادرتهما الجنة، أو لإعادتهما إليها مرة أخرى

\par 3 فبسط آدم يديه إلى الله، متوسلاً إليه أن يُفهمه ما هما

\par 4 ثم قال الشيطان، كاره كل خير، لآدم: "يا آدم، أنا ملاك الله العظيم، وانظر إلى الجنود الذين يحيطون بي."

\par 5 «لقد أرسلني الله وأرسلهم لأخذك وإحضارك إلى حدود الجنة شمالًا؛ إلى شاطئ البحر الصافي، ولأغتسل أنت وحواء فيه، وأعيدكما إلى سعادتكما السابقة، حتى تعودا مرة أخرى إلى الجنة.»

\par 6 غرقت هذه الكلمات في قلب آدم وحواء.

\par 7 ولكن الله حجب كلمته عن آدم، ولم يجعله يفهم على الفور، بل انتظر ليرى قوته؛ هل سيُهزم كما هُزمت حواء في الجنة، أم أنه سيسود.

\par 8 ثم نادى الشيطان آدم وحواء، وقال: «ها نحن ذاهبون إلى بحر الماء». فبدأوا بالذهاب

\par 9 وتبعهم آدم وحواء على مسافة قصيرة.

\par 10 ولكن عندما وصلوا إلى الجبل الذي يقع شمال الجنة، وهو جبل عالي جدًا، بدون أي درجات إلى قمته، اقترب الشيطان من آدم وحواء، وجعلهما يصعدان إلى القمة في الحقيقة، وليس في رؤيا؛ متمنيًا، كما فعل، أن يطرحهما إلى أسفل ويقتلهما، وأن يمحو اسمهما من الأرض؛ حتى تبقى هذه الأرض له ولجيشه وحدهم.



\chapter{29}

\par \textit{يخبر الله آدم عن غرض الشيطان. (الآية 4).}

\par 1 ولكن عندما رأى الله الرحيم أن الشيطان يريد قتل آدم بمكائده المتعددة، ورأى أن آدم وديع وبلا حيلة، تكلم الله إلى الشيطان بصوت عالٍ ولعنه

\par 2 ثم هرب هو وجيشه، وبقي آدم وحواء واقفين على قمة الجبل، حيث أبصرا من تحتهما العالم الواسع الذي كانا فوقه. لكنهما لم يريا أحدًا من الجيش الذي كان قريبًا منهما.

\par 3 فبكى آدم وحواء أمام الله وطلبا منه المغفرة.

\par 4 ثم جاء الكلمة من الله إلى آدم وقال له: "اعلم وافهم عن هذا الشيطان أنه يسعى إلى خداعك ونسلك من بعدك".

\par 5 فبكى آدم أمام الرب الإله، وطلب إليه أن يعطيه شيئًا من الجنة علامة له حتى يتعزى.

\par 6 ونظر الله إلى فكر آدم، فأرسل الملاك ميخائيل إلى البحر الذي يصل إلى الهند، ليأخذ من هناك قضبانًا ذهبية ويأتي بها إلى آدم.

\par 7 لقد فعل الله هذا بحكمته، حتى تشرق هذه القضبان الذهبية، التي كانت مع آدم في الكهف، بالنور في الليل حوله، وتنهي خوفه من الظلام.

\par 8 ثم نزل الملاك ميخائيل بأمر الله، وأخذ قضباناً من ذهب كما أمره الله، وأتى بها إلى الله.

\chapter{30}

\par \textit{آدم يتلقى أول خيرات الدنيا.}

\par 1 وبعد هذه الأمور، أمر الله الملاك جبرائيل أن ينزل إلى الجنة ويقول للكروب حارسها: "هوذا الله أمرني أن أدخل الجنة وأن آخذ من هناك بخورا طيب الرائحة وأعطيه لآدم".

\par 2 ثم نزل الملاك جبرائيل بأمر من الله إلى الجنة، وأخبر الكروب كما أمره الله

\par 3 فقال الكروب: "حسنًا". فدخل جبرائيل وأخذ البخور

\par 4 ثم أمر الله ملاكه رافائيل أن ينزل إلى الجنة، ويكلّم الكروب عن بعض المر ليعطيه لآدم

\par 5 فنزل الملاك رافائيل وأخبر الكروب كما أمره الله، فقال الكروب: «حسنًا». فدخل رافائيل وأخذ المر

\par 6 كانت القضبان الذهبية من بحر الهند، حيث توجد أحجار كريمة. وكان البخور من الحد الشرقي للجنة، والمر من الحد الغربي، حيث جاءت المرارة على آدم

\par 7 وأحضر الملائكة هذه الأشياء الثلاثة إلى الله، عند شجرة الحياة، في الجنة

\par 8 ثم قال الله للملائكة: "اغمسوهم في نبع الماء، ثم خذوهم ورشوا ماءهم على آدم وحواء، لكي يتعزيا قليلًا في حزنهما، وأعطوهم لآدم وحواء."

\par 9 ففعل الملائكة كما أمرهم الله، وأعطوا كل تلك الأشياء لآدم وحواء على قمة الجبل الذي وضعهم عليه الشيطان، عندما سعى إلى القضاء عليهم

\par 10 ولما رأى آدم قضبان الذهب والبخور والمر، فرح وبكى لأنه ظن أن الذهب علامة على الملكوت الذي جاء منه، وأن البخور علامة على النور الساطع الذي أُخذ منه، وأن المر علامة على الحزن الذي كان فيه

\chapter{31}

\par \textit{يجعلون أنفسهم أكثر راحة في كهف الكنوز في اليوم الثالث.}

\par 1 وبعد هذه الأمور قال الله لآدم: «لقد طلبت مني شيئاً من الجنة لتتعزى به، فأعطيتك هذه الثلاث علامات كتعزية لك، لكي تثق بي وفي عهدي معك».

\par 2 "لأني سآتي وأخلصك، وسيحضر لي الملوك عندما أكون في الجسد، ذهبًا وبخورًا ومُرًا؛ الذهب كعلامة على مملكتي، والبخور كعلامة على ألوهيتي، والمر كعلامة على معاناتي وموتى."

\par 3 "ولكن يا آدم، ضع هذه الأشياء معك في الكهف؛ الذهب لكي يضيء عليك في الليل، والبخور لكي تشم رائحته الطيبة، والمر لكي يعزيك في حزنك."

\par 4 فلما سمع آدم هذه الكلمات من الله، سجد أمامه، وسجد هو وحواء له وشكراه لأنه رحمهما.

\par 5 ثم أمر الله الملائكة الثلاثة، ميخائيل وجبرائيل ورافائيل، أن يحضروا كل واحد ما أحضره، ويسلموه إلى آدم. ففعلوا ذلك واحدًا واحدًا.

\par 6 وأمر الله سوريال وشألتيئيل أن يحملا آدم وحواء، وينزلاهما من قمة الجبل العالي، ويأخذاهما إلى مغارة الكنوز

\par 7 هناك وضعوا الذهب على الجانب الجنوبي من الكهف، والبخور على الجانب الشرقي، والمر على الجانب الغربي. لأن فم الكهف كان في الجانب الشمالي

\par 8 ثم عزّى الملائكة آدم وحواء، ثم انصرفوا.

\par 9 وكان الذهب سبعين قضيبا، والبخور اثنتا عشرة منا، والمر ثلاث منا.

\par 10 هذه الأشياء التي تركها آدم في بيت الكنوز؛ لذلك سُمي "مخبأ". لكن مفسرين آخرين يقولون إنه سُمي "كهف الكنوز"، بسبب جثث الرجال الصالحين التي كانت فيه

\par 11 هذه الأشياء الثلاثة أعطاها الله لآدم، في اليوم الثالث بعد خروجه من الجنة، رمزًا للأيام الثلاثة التي سيبقى فيها الرب في قلب الأرض

\par 12 وهذه الأشياء الثلاثة، بينما كانوا يواصلون مع آدم في الكهف، أعطته نورًا ليلًا، وفي النهار أعطته راحةً قليلةً من حزنه

\chapter{32}

\par \textit{آدم وحواء يذهبان إلى الماء للصلاة.}

\par 1 وأما آدم وحواء فبقيا في مغارة الكنوز إلى اليوم السابع، ولم يأكلا من ثمر الأرض، ولم يشربا ماء.

\par 2 ولما طلع فجر اليوم الثامن، قال آدم لحواء: "يا حواء، لقد صلينا إلى الله أن يعطينا شيئًا من الجنة، فأرسل ملائكته فأحضروا لنا ما رغبنا فيه."

\par 3 «لكن الآن، انهضوا، لنذهب إلى بحر الماء الذي رأيناه في البداية، ولنقف فيه، وندعو الله أن يرحمنا مرة أخرى ويعيدنا إلى الجنة؛ أو أن يمنحنا شيئًا؛ أو أن يمنحنا الراحة في أرض أخرى غير هذه التي نحن فيها.»

\par 4 ثم خرج آدم وحواء من الكهف، وذهبا ووقفا على حافة البحر الذي ألقيا نفسيهما فيه من قبل، وقال آدم لحواء:

\par 5 «تعالَ، انزل إلى هذا المكان، ولا تخرج منه حتى نهاية ثلاثين يومًا، حين آتي إليك. وادع الله بقلب حار وصوت عذب، أن يغفر لنا.»

\par 6 "وأنا أذهب إلى مكان آخر وأنزل إليه وأفعل مثلك."

\par 7 فنزلت حواء إلى الماء كما أمرها آدم. فنزل آدم أيضًا إلى الماء، ووقفا يصليان، وتوسلا إلى الرب أن يغفر لهما ذنبهما، وأن يردهما إلى حالتهما الأولى.

\par 8 وظلوا يصلون هكذا إلى نهاية الخمسة والثلاثين يوماً.

\chapter{33}

\par \textit{الشيطان يعد زورًا بـ"النور الساطع!"}

\par 1 لكن الشيطان، كاره كل خير، بحث عنهم في الكهف، لكنه لم يجدهم، مع أنه بحث عنهم بجد

\par 2 ولكنه وجدهما واقفين في الماء يصليان، ففكر في نفسه: "إن آدم وحواء واقفين في الماء يطلبان من الله أن يغفر لهما خطيئتهما، وأن يعيدهما إلى حالتهما الأولى، وأن يأخذهما من تحت يدي".

\par 3 «لكنني سأخدعهم حتى يخرجوا من الماء ولا يوفوا بنذرهم.»

\par 4 ثم لم يذهب كاره كل خير إلى آدم، بل ذهب إلى حواء، وتجسد في صورة ملاك الله، يسبح ويفرح، وقال لها:

\par 5 «السلام عليك! افرح وابتهج! الله راضٍ عنك، وقد أرسلني إلى آدم. لقد جلبت له بشرى الخلاص، وامتلاءه بنور ساطع كما كان في البداية.»

\par 6 «وآدم، في فرحه بعودته، أرسلني إليك، لكي تأتي إليّ، لكي أُكلّلك بنور مثله.»

\par 7 «وقال لي: «كلم حواء؛ إن لم تأتِ معك، فأخبرها عن العلامة التي رأيناها عندما كنا على قمة الجبل؛ كيف أرسل الله ملائكته الذين أخذونا وأحضرونا إلى مغارة الكنوز؛ ووضع الذهب على الجانب الجنوبي، والبخور على الجانب الشرقي، والمر على الجانب الغربي.» فالآن تعال إليه.»

\par 8 عندما سمعت حواء هذه الكلمات منه، فرحت فرحًا عظيمًا. وظنت أن ظهور الشيطان حقيقي، فخرجت من البحر

\par 9 سبقها، وتبعته حتى وصلا إلى آدم. ثم اختفى عنها الشيطان، فلم تره بعد ذلك

\par 10 ثم جاءت ووقفت أمام آدم، الذي كان واقفًا عند الماء يفرح بغفران الله

\par 11 ولما نادته، استدار فوجدها هناك، فبكى لما رآها، وضرب على صدره، ومن مرارة حزنه غرق في الماء

\par 12 لكن الله نظر إليه وإلى بؤسه، وإلى أنه على وشك أن يلفظ أنفاسه الأخيرة. فجاءت كلمة الله من السماء، وأخرجته من الماء، وقالت له: "اصعد الضفة العليا إلى حواء". وعندما صعد إلى حواء قال لها: "من قال لك تعالي إلى هنا؟"

\par 13 ثم أخبرته بكلام الملاك الذي ظهر لها وأعطاها آية

\par 14 لكن آدم حزن، وأعلمها أنه الشيطان. ثم أخذها وعادا كلاهما إلى الكهف

\par 15 حدث لهما هذا في المرة الثانية التي نزلا فيها إلى الماء، بعد سبعة أيام من خروجهما من الجنة

\par 16 صاموا في الماء خمسة وثلاثين يومًا، أي ما مجموعه اثنين وأربعين يومًا منذ خروجهم من الجنة

\chapter{34}

\par \textit{يتذكر آدم خلق حواء. ويدعو ببلاغة من أجل الطعام والشراب.}

\par 1 وفي صباح اليوم الثالث والأربعين، خرجوا من الكهف، حزينين وباكيين. كانت أجسادهم هزيلة، وقد جفّوا من الجوع والعطش، ومن الصيام والصلاة، ومن حزنهم الشديد على معصيتهم

\par 2 ولما خرجوا من المغارة صعدوا إلى الجبل غربي الجنة.

\par 3 هناك وقفوا وصلوا وتوسلوا إلى الله أن يمنحهم غفران خطاياهم

\par 4 وبعد صلاتهم، بدأ آدم يتوسل إلى الله قائلًا: "يا ربي وإلهي وخالقي، لقد أمرت بجمع العناصر الأربعة معًا، وقد جُمعت بأمرك."

\par 5 «ثم بسطت يدك فخلقتني من عنصر واحد، من تراب الأرض، وأدخلتني الجنة في الساعة الثالثة، يوم الجمعة، وأخبرتني بذلك في الكهف».

\par 6 «ثم، في البداية، لم أكن أعرف الليل ولا النهار، لأن طبيعتي كانت مشرقة؛ ولم يتركني النور الذي كنت أعيش فيه أعرف الليل أو النهار.»

\par 7 «ثم، يا رب، في تلك الساعة الثالثة التي خلقتني فيها، أحضرت إليّ جميع الوحوش، والأسود، والنعام، وطيور السماء، وكل ما يتحرك على الأرض، التي خلقتها في الساعة الأولى أمامي من يوم الجمعة.»

\par 8 «وكانت مشيئتك أن أُسمّيهم جميعًا، واحدًا تلو الآخر، باسمٍ مناسب. ولكنك أعطيتني فهمًا ومعرفةً، وقلبًا نقيًا وعقلًا مستقيمًا من عندك، لأُسمّيهم على حسب رأيك فيما يتعلق بتسميتهم.»

\par 9 «يا الله، لقد أطعتهم لي، وأمرت ألا يخرج أحد منهم عن سلطاني، وفقًا لأمرك، وللسلطان الذي أعطيتني إياه عليهم. لكنهم الآن جميعًا غرباء عني.»

\par 10 «ثم كانت تلك الساعة الثالثة من يوم الجمعة، التي خلقتني فيها، وأمرتني بشأن الشجرة التي لا يجب أن أقترب منها ولا آكل منها؛ لأنك قلت لي في الجنة: «عندما تأكل منها، تموت موتًا».»

\par 11 «وإذا عاقبتني بالموت كما قلت، لمتُّ في تلك اللحظة بالذات.»

\par 12 «وعلاوة على ذلك، عندما أمرتني بشأن الشجرة، لم يكن لي أن أقترب منها ولا أن أقطع منها، ولم تكن حواء معي؛ ولم تكن قد خلقتها بعد، ولم تكن قد أخرجتها من جانبي بعد؛ ولم تكن قد سمعت هذا الأمر منك بعد.»

\par 13 «ثم في نهاية الساعة الثالثة من ذلك الجمعة، يا رب، جعلتني أغفو ونائمًا، فنمت، وغلبني النوم.»

\par 14 «ثم أخذتَ ضلعًا من جنبي، وخلقتها على شبهي وصورتي. ثم استيقظتُ، فلما رأيتها وعرفتُ من هي، قلتُ: هذه عظم من عظامي، ولحم من لحمي؛ ومن الآن فصاعدًا تُدعى امرأة.»

\par 15 «لقد كان من حسن إرادتك يا الله أنك جلبت عليّ سباتًا ونومًا، وأنك أخرجت حواء من جنبي على الفور، حتى خرجت، حتى أنني لم أرَ كيف خُلقت؛ ولم أستطع أن أشهد يا ربي كم هو عظيم وعظيم صلاحك ومجدك.»

\par 16 "ومن فضلك يا رب، خلقتنا كلينا بأجساد من نور، وجعلتنا اثنين واحدًا، وأعطيتنا نعمتك، وملأتنا بتسبيح الروح القدس، حتى لا نجوع ولا نعطش، ولا نعرف ما هو الحزن، ولا ضعف القلب، ولا نتألم ولا نصوم ولا نتعب."

\par 17 «ولكن الآن يا الله، بما أننا تعدينا وصيتك وخالفنا شريعتك، فقد أخرجتنا إلى أرض غريبة، وسببت لنا المعاناة والضعف والجوع والعطش.»

\par 18 «الآن، يا الله، نسألك أن تعطينا شيئًا نأكله من الجنة، لنشبع به جوعنا؛ وشيئًا نروي به عطشنا.»

\par 19 «لأنه يا الله، هوذا أيام كثيرة لم نذق شيئًا ولم نشرب شيئًا، وقد جف لحمنا، وضاعت قوتنا، وطار النوم من أعيننا من التعب والبكاء.»

\par 20 «إذن، يا الله، لا نجرؤ على قطف أي شيء من ثمار الأشجار، خوفًا منك. لأنك عندما أخطأنا في البداية، أنقذتنا، ولم تمتنا.»

\par 21 «لكننا الآن، فكّرنا في قلوبنا، إذا أكلنا من ثمار الأشجار، بدون أمر الله، فسوف يُهلكنا هذه المرة، وسيمحونا عن وجه الأرض.»

\par 22 «وإذا شربنا من هذا الماء، دون أمر الله، فسوف يُهلكنا ويقتلعنا في الحال.»

\par 23 «والآن، يا الله، بما أنني وصلت إلى هذا المكان مع حواء، فإننا نطلب منك أن تعطينا من ثمار الجنة، حتى نشبع منها.»

\par 24 «لأننا نرغب في الثمر الذي على الأرض، وكل ما ينقصنا فيه.»

\chapter{35}

\par \textit{رد الله.}

\par 1 ثم نظر الله مرة أخرى إلى آدم وبكائه وأنينه، وجاء إليه كلمة الله وقال له:

\par 2 يا آدم، عندما كنت في جنتي، لم تكن تعرف الأكل ولا الشرب؛ ولا الضعف ولا المعاناة؛ ولا هزال الجسد ولا التغير؛ ولم يفارق النوم عينيك. ولكن منذ أن تجاوزت الحدود، وجئت إلى هذه الأرض الغريبة، حلت عليك كل هذه التجارب

\chapter{36}

\par \textit{Figs.}

\par 1 ثم أمر الله الكروب حارس باب الجنة وبيده سيف نار أن يأخذ من ثمر التينة ويعطيه لآدم

\par 2 فأطاع الكروب أمر الرب الإله، ودخل الجنة وأحضر تينتين على غصنين، كل تينة معلقة بورقة. كانتا من اثنتين من الأشجار التي اختبأ بينهما آدم وحواء عندما ذهب الله للمشي في الجنة، وجاءت كلمة الله إلى آدم وحواء وقالت لهما: "آدم، آدم، أين أنت؟"

\par 3 فأجاب آدم: «يا الله، ها أنا ذا. فلما سمعت صوتك وصوتك، اختبأت لأني عريان».

\par 4 ثم أخذ الكروب تينتين وقدمهما إلى آدم وحواء. لكنه طرحهما لهما من بعيد، لأنهما لم يستطيعا الاقتراب من الكروب بسبب لحمهما الذي لا يستطيع الاقتراب من النار

\par 5 في البداية، ارتجفت الملائكة من حضور آدم وخافوا منه. أما الآن، ارتجف آدم أمام الملائكة وخاف منهم.

\par 6 فتقدم آدم وأخذ تينة، ثم جاءت حواء أيضًا وأخذت الأخرى

\par 7 وبينما كانوا يحملونها بأيديهم، نظروا إليها، فعرفوا أنها من الأشجار التي أخفوا أنفسهم بينها

\chapter{37}

\par \textit{ثلاثة وأربعون يومًا من التوبة لا تكفر ساعة واحدة من الخطيئة (الآية 6).}

\par 1 ثم قال آدم لحواء: "ألا ترين هذا التين وأوراقه التي غطينا بها أنفسنا عندما جُرِّدنا من طبيعتنا المشرقة؟ لكننا الآن لا نعلم أي بؤس ومعاناة قد تصيبنا من أكله."

\par 2 «والآن يا حواء، دعينا نمتنع عن الأكل منها، أنا وأنتِ، ولنطلب من الله أن يعطينا من ثمرة شجرة الحياة.»

\par 3 وهكذا كتم آدم وحواء نفسيهما، ولم يأكلا من هذا التين

\par 4 لكن آدم بدأ يصلي إلى الله ويتوسل إليه أن يعطيه من ثمرة شجرة الحياة، قائلاً: "يا الله، عندما خالفنا وصيتك في الساعة السادسة من يوم الجمعة، جُرِّدنا من الطبيعة المشرقة التي كانت لدينا، ولم نبق في الجنة بعد خطئنا أكثر من ثلاث ساعات."

\par 5 «ولكن في المساء أخرجتنا منه. يا الله، لقد أخطأنا في حقك ساعة واحدة، وحلّ علينا كل هذه التجارب والأحزان حتى هذا اليوم.»

\par 6 "وهذه الأيام مع اليوم الثالث والأربعين لا تعوض تلك الساعة التي تعدينا فيها!"

\par 7 "اللهم انظر إلينا بعين الرأفة، ولا تجازنا حسب مخالفتنا لوصيتك أمامك."

\par 8 «يا الله، أعطنا من ثمرة شجرة الحياة، لنأكل منها ونحيا، ولا نلتفت فنرى معاناةً ومتاعب أخرى على هذه الأرض؛ لأنك أنت الله.»

\par 9 «لأننا خالفنا وصيتك، أخرجتنا من الجنة، وأرسلت كروبًا ليحرس شجرة الحياة، لئلا نأكل منها فنحيا، ولا نعرف شيئًا عن الضعف بعد أن خالفنا وصيتك.»

\par 10 «ولكن الآن يا رب، انظر، لقد صمدنا كل هذه الأيام، وتحملنا الآلام. اجعل هذه الأيام الثلاثة والأربعين معادلة للساعة الواحدة التي تعدينا فيها.»

\chapter{38}

\par \textit{"عندما تكتمل 5500 سنة..."}

\par 1 بعد هذه الأمور، جاء كلام الله إلى آدم، وقال له:—

\par 2 يا آدم، أما ثمرة شجرة الحياة التي تطلبها، فلن أعطيك إياها الآن، بل عندما تنقضي الخمسة والخمسمائة سنة. حينئذٍ أعطيك من ثمرة شجرة الحياة، فتأكلها وتحيا إلى الأبد، أنت وحواء ونسلك الصالح.

\par 3 "ولكن هذه الثلاثة والأربعين يومًا لا يمكنها أن تكفر عن الساعة التي خالفت فيها وصيتي."

\par 4 يا آدم، قد أطعمتك من شجرة التين التي اختبأت فيها. اذهب وكل منها أنت وحواء.

\par 5 "لن أرفض طلبك، ولن أخيب رجاءك. لذلك، اصبر حتى يتم الوفاء بالعهد الذي قطعته معك."

\par 6 فرفع الله كلمته عن آدم.

\chapter{39}

\par \textit{آدم حذر - ولكن بعد فوات الأوان.}

\par 1 ثم رجع آدم إلى حواء وقال لها: قومي وخذي لنفسك تينة وأنا آخذ أخرى، ونذهب إلى كهفنا

\par 2 ثم أخذ آدم وحواء كل منهما تينة وذهبا نحو الكهف. وكان الوقت قد اقترب من غروب الشمس، وجعلتهما أفكارهما يشتهيان أكل الثمرة

\par 3 فقال آدم لحواء: "إني أخاف أن آكل من هذه التين. لا أعلم ماذا قد يأتي عليّ منها."

\par 4 فبكى آدم، ووقف يصلي أمام الله قائلًا: "أشبع جوعي دون أن آكل هذه التين؛ فماذا يفيدني بعد أن آكلها؟ وماذا أريد وأطلب منك يا الله بعد أن تنتهي؟"

\par 5 فقال أيضًا: «أخاف أن آكل منه، لأني لا أعلم ما يصيبني منه».

\chapter{40}

\par \textit{الجوع البشري الأول.}

\par 1 ثم جاء كلمة الله إلى آدم، وقال له: "يا آدم، لماذا لم يكن لديك هذا الخوف، ولا هذا الصيام، ولا هذا الهم من قبل؟ ولماذا لم يكن لديك هذا الخوف قبل أن تعصي؟"

\par 2 "ولكن عندما أتيت لتسكن في هذه الأرض الغريبة، لم يكن جسمك الحيواني قادرًا على البقاء على الأرض بدون طعام أرضي، لتقويته واستعادة قواه."

\par 3 فرفع الله كلمته عن آدم.

\chapter{41}

\par \textit{العطش البشري الأول.}

\par 1 ثم أخذ آدم التين ووضعه على قضبان الذهب. وأخذت حواء أيضًا تينها ووضعته على البخور

\par 2 وكان وزن كل تينة مثل بطيخة، لأن ثمر الجنة كان أعظم من ثمر هذه الأرض.

\par 3 وأما آدم وحواء فبقيا واقفين صائمين تلك الليلة كلها حتى طلع الصباح.

\par 4 وعندما أشرقت الشمس كانوا في صلاتهم، وقال آدم لحواء بعد أن انتهوا من الصلاة:

\par 5 يا حواء، تعالي لنذهب إلى حدود الجنة، نحو الجنوب، إلى حيث ينبع النهر، وينقسم إلى أربعة رؤوس. هناك سنصلي إلى الله، ونسأله أن يسقينا من ماء الحياة.

\par 6 لأن الله لم يُطعمنا من شجرة الحياة حتى لا نعيش. لذلك، سنطلب منه أن يُعطينا من ماء الحياة، فنروي به عطشنا، لا أن نشرب من ماء هذه الأرض.

\par 7 عندما سمعت حواء هذه الكلمات من آدم، وافقت؛ وقاما كلاهما ووصلا إلى الحد الجنوبي للجنة، على حافة نهر الماء على مسافة قصيرة من الجنة

\par 8 ووقفوا وصلوا أمام الرب، وطلبوا منه أن ينظر إليهم هذه المرة، وأن يغفر لهم، وأن يستجيب لهم طلبهم

\par 9 بعد هذه الصلاة منهما، بدأ آدم بالصلاة بصوته أمام الله، وقال:

\par 10 "يا رب، حين كنت في الجنة ورأيت الماء يخرج من تحت شجرة الحياة، لم يشتهي قلبي، ولا احتاج جسدي أن يشرب منه، ولا عرفت العطش لأني كنت حياً، وفوق ما أنا عليه الآن."

\par 11 «لذلك، لكي أعيش، لم أكن بحاجة إلى أي طعام حي، ولم أشرب ماء الحياة.»

\par 12 «لكن الآن يا الله، أنا ميت؛ جسدي عطشان. أعطني من ماء الحياة لأشرب منه وأحيا.»

\par 13 «برحمتك يا الله، نجني من هذه الأوبئة والتجارب، وأدخلني إلى أرض أخرى مختلفة عن هذه، إن لم تدعني أسكن في جنتك.»

\chapter{42}

\par \textit{وعد بماء الحياة. النبوءة الثالثة عن مجيء المسيح.}

\par 1 ثم جاء كلمة الله إلى آدم وقال له:

\par 2 يا آدم، أما قولك: «أدخلني إلى أرض الراحة»، فهي ليست أرضًا أخرى غير هذه، بل هي ملكوت السماوات حيث الراحة وحدها.

\par 3 «لكنك لا تستطيع دخوله في الوقت الحاضر؛ ولكن فقط بعد أن ينقضي دينونتك ويتم تنفيذها.»

\par 4 «ثم أصعدك إلى ملكوت السماوات، أنت ونسلك البار، وأعطيك وإياهم ما تطلبه الآن.»

\par 5 «وإن قلتَ: أعطني من ماء الحياة لأشرب وأحيا، فلا يمكن أن يكون هذا اليوم، بل في اليوم الذي سأنزل فيه إلى الجحيم، وأكسر أبواب النحاس، وأسحق ممالك الحديد.»

\par 6 «حينئذٍ سأُخلِّص روحك ونفوس الصالحين برحمتي، لأُريحهم في جنتي. وسيكون ذلك عندما تأتي نهاية العالم.»

\par 7 «وأيضًا، فيما يتعلق بماء الحياة الذي تطلبه، فلن يُمنح لك اليوم؛ بل في اليوم الذي سأسفك فيه دمي على رأسك في أرض الجلجثة.»

\par 8 «لأن دمي سيكون ماء حياة لك، في ذلك الوقت، ليس لك وحدك، بل لجميع نسلك الذين يؤمنون بي؛ ليكون لهم راحة إلى الأبد.»

\par 9 قال الرب لآدم مرة أخرى: "يا آدم، عندما كنت في الجنة، لم تأت إليك هذه التجارب"

\par 10 «ولكن منذ أن خالفت وصيتي، حلت عليك كل هذه الآلام.»

\par 11. «والآن، فإن جسدك يحتاج إلى طعام وشراب، فاشرب إذن من ذلك الماء الذي يجري منك على وجه الأرض».

\par 12 ثم سحب الله كلمته من آدم.

\par 13 وسجد آدم وحواء للرب، ثم رجعا من نهر الماء إلى الكهف. وكان وقت الظهر، فلما اقتربا من الكهف، رأيا عنده نارًا عظيمة.

\chapter{43}

\par \textit{الشيطان يحاول إشعال النار.}

\par 1 ثم خاف آدم وحواء، ووقفا. وقال آدم لحواء: "ما هذه النار التي بجانب كهفنا؟ نحن لا نفعل فيها شيئًا لإشعال هذه النار."

\par 2 ليس لدينا خبزٌ نخبزه فيه، ولا مرقٌ نطبخه فيه. أما هذه النار، فلا نعرف لها مثيلًا، ولا نعرف ما نسميها.

\par 3 "ولكن منذ أرسل الله الكروب بسيف من نار يلمع ويخفت في يده، فسقطنا من الخوف وصرنا كالجثث، لم نرَ مثله."

\par 4 «ولكن الآن يا حواء، انظري، هذه هي نفس النار التي كانت في يد الكروب، التي أرسلها الله لحفظ الكهف الذي نسكن فيه.»

\par 5 «يا حواء، هذا لأن الله غاضب علينا، وسيطردنا منه.»

\par 6 «يا حواء، لقد خالفنا وصيته مرة أخرى في ذلك الكهف، فأرسل هذه النار لتحرقه، وتمنعنا من دخوله.»

\par 7 «إن كان الأمر كذلك حقًا يا حواء، فأين نسكن؟ ومن أين نهرب من وجه الرب؟ لأنه، فيما يتعلق بالجنة، لن يدعنا نقيم فيها، وقد حرمنا من خيراتها؛ بل وضعنا في هذا الكهف، الذي تحملنا فيه الظلمة والتجارب والمصاعب، حتى وجدنا فيه أخيرًا الراحة.»

\par 8 «ولكن الآن وقد أخرجنا إلى أرض أخرى، فمن يدري ما قد يحدث فيها؟ ومن يدري إلا أن ظلمة تلك الأرض قد تكون أعظم بكثير من ظلمة هذه الأرض؟»

\par 9 «من يدري ما قد يحدث في تلك الأرض نهارًا أو ليلًا؟ ومن يدري إن كان بعيدًا أم قريبًا يا حواء؟ حيث يُرضي الله أن يضعنا، قد يكون بعيدًا عن الجنة يا حواء! أو حيث يمنعنا الله من رؤيته، لأننا تجاوزنا وصيته، ولأننا كنا نطلب منه في كل وقت؟»

\par 10 «يا حواء، إذا كان الله سيُدخلنا إلى أرض غريبة غير هذه، نجد فيها العزاء، فلا بد أن يكون ذلك لإماتة أرواحنا، ومحو اسمنا من على وجه الأرض.»

\par 11 «يا حواء، إذا ابتعدنا أكثر عن الجنة وعن الله، فأين نجده مرة أخرى، ونطلب منه أن يعطينا ذهبًا وبخورًا ومرًّا وبعضًا من ثمار شجرة التين؟»

\par 12 «أين نجده ليعزينا مرة ثانية؟ أين نجده ليذكرنا بالعهد الذي قطعه من أجلنا»

\par 13 ثم سكت آدم. وظلا ينظران، هو وحواء، نحو الكهف، نحو النار المشتعلة حوله

\par 14 لكن تلك النار كانت من الشيطان. لأنه جمع أشجارًا وأعشابًا يابسة، وحملها وأتى بها إلى الكهف، وأشعل فيها النار، ليحرق الكهف وما فيه

\par 15 لكي يُترَك آدم وحواء في حزن، ويقطع ثقتهما بالله، ويجعلهما ينكرانه

\par 16 ولكن بفضل رحمة الله لم يستطع أن يحرق الكهف، لأن الله أرسل ملاكه حول الكهف ليحرسه من مثل هذه النار حتى انطفأ

\par 17 واستمرت هذه النار من الظهر حتى طلوع الفجر. كان ذلك اليوم الخامس والأربعين

\chapter{44}

\par \textit{قوة النار على الإنسان.}

\par 1 ولكن آدم وحواء كانا واقفين ينظران إلى النار، ولم يستطيعا أن يقتربا من الكهف من خوفهما من النار.

\par 2 وظل الشيطان يحضر الأشجار ويلقيها في النار، حتى ارتفع لهيبها عالياً وغطى الكهف كله، ظاناً، كما في نفسه، أن يلتهم الكهف بنار كثيرة. ولكن ملاك الرب كان يحرسه.

\par 3 ومع ذلك، لم يستطع أن يلعن الشيطان، ولا أن يؤذيه بالكلام، لأنه لم يكن له سلطان عليه، ولم يكن يفعل ذلك بكلمات من فمه

\par 4 لذلك تحمله الملاك، دون أن ينطق بكلمة سيئة واحدة، حتى جاءت كلمة الله التي قالت للشيطان: "اذهب من هنا؛ لقد خدعت عبيدي من قبل، وهذه المرة تسعى إلى إهلاكهم."

\par 5 «لولا رحمتي لأهلكتك أنت وجنودك من على وجه الأرض. لكنني صبرت عليك حتى نهاية العالم.»

\par 6 فهرب الشيطان من أمام الرب. لكن النار ظلت مشتعلة حول الكهف كنار فحم طوال اليوم؛ وهو اليوم السادس والأربعون الذي قضاه آدم وحواء منذ خروجهما من الجنة

\par 7 ولما رأى آدم وحواء أن حرارة النار قد خفّت قليلًا، بدأا بالسير نحو الكهف ليدخلاه كما اعتادا، لكنهما لم يستطيعا بسبب حرارة النار

\par 8 ثم أخذا يبكيان بسبب النار التي فصلت بينهما وبين الكهف، والتي جذبت إليهما مشتعلة. وكانا خائفين

\par 9 ثم قال آدم لحواء: "انظري إلى هذه النار التي لنا نصيب منها: التي كانت لنا سابقًا، ولكنها لم تعد كذلك الآن بعد أن تجاوزنا حدود الخلق، وغيرنا حالتنا، وتغيرت طبيعتنا. لكن النار لم تتغير في طبيعتها، ولم تتغير منذ خلقها. لذلك لها الآن سلطان علينا؛ وعندما نقترب منها، تحرق أجسادنا."

\chapter{45}

\par \textit{لماذا لم يُوفِ الشيطان بوعوده؟}

\par 1 فقام آدم وصلى إلى الله قائلاً: انظر، هذه النار قد فصلت بيننا وبين المغارة التي أمرتنا أن نسكن فيها، والآن هوذا لا نستطيع أن ندخلها.

\par 2 ثم سمع الله آدم، وأرسل إليه كلمته التي قالت:

\par 3 «يا آدم، انظر إلى هذه النار! ما الفرق بين لهيبها وحرارتها وجنة النعيم وما فيها من خيرات!»

\par 4 "عندما كنت تحت سيطرتي، خضعت لك جميع المخلوقات؛ ولكن بعد أن خالفت وصيتي، نهضوا جميعًا فوقك."

\par 5 قال له الله مرة أخرى: "انظر يا آدم، كيف رفعك الشيطان! لقد حرمك من الألوهية، ومن مكانة سامية مثلي، ولم يوفِ بوعده لك؛ بل أصبح عدوك. هو الذي أشعل هذه النار التي أراد أن يحرقك أنت وحواء فيها."

\par 6 "لماذا يا آدم لم يحفظ عهده معك ولو يوما واحدا، بل حرمك من المجد الذي كان عليك حين خضعت لأمره؟"

\par 7 أتظن يا آدم أنه أحبك حين عقد معك هذا العهد؟ أم أنه أحبك وأراد أن يرفعك إلى أعلى؟

\par 8 "ولكن يا آدم، لم يفعل كل ذلك من أجل محبته لك؛ بل أراد أن يخرجك من النور إلى الظلمة، ومن حالة الرفع إلى الانحطاط؛ ومن المجد إلى الذل؛ ومن الفرح إلى الحزن؛ ومن الراحة إلى الصوم والإغماء."

\par 9 قال الله أيضًا لآدم: "انظر إلى هذه النار التي أشعلها الشيطان حول كهفك؛ انظر إلى هذه العجائب التي تحيط بك؛ واعلم أنها ستحيط بك وبذريتك عندما تستجيب لأمره؛ وأنه سيعذبك بالنار؛ وأنك ستهبط إلى الجحيم بعد موتك."

\par 10 «حينئذٍ سترون لهيب ناره، التي ستشتعل حولكم وحول نسلكم. ولن يكون لكم خلاص منها إلا عند مجيئي؛ كما لا يمكنكم الآن دخول كهفكم بسبب النار العظيمة المحيطة به؛ ليس قبل أن تأتي كلمتي التي ستمهد لكم طريقًا يوم يتمم عهدي.»

\par 11 «لا سبيل لك الآن للمجيء من هنا إلى الراحة، ليس حتى تأتي كلمتي، التي هي كلمتي. حينئذٍ سيجعل لك طريقًا، وسترتاح.» ثم دعا الله بكلمته تلك النار التي كانت مشتعلة حول الكهف، أن تنقسم إلى أجزاء، حتى يمر آدم من خلالها. ثم انقسمت النار بأمر من الله، وفُتح طريق لآدم

\par 12 فرفع الله كلمته عن آدم.

\chapter{46}

"كم مرة أنقذتك من يده..."

\par 1 ثم بدأ آدم وحواء بالدخول مرة أخرى إلى الكهف. وعندما وصلا إلى الطريق بين النار، نفخ الشيطان في النار كالزوبعة، وأشعل على آدم وحواء جمرًا مشتعلًا؛ حتى احترقت أجسادهما؛ وأحرقتهما نار الجمر

\par 2 ومن بين نار الاحتراق، صرخ آدم وحواء بصوت عالٍ، وقالا: "يا رب، نجنا! لا تتركنا لنُهلك ونُبتلى بهذه النار المشتعلة؛ ولا تُحاسبنا على تجاوزنا وصيتك."

\par 3 ثم نظر الله إلى أجسادهم التي أشعل فيها الشيطان نارًا، فأرسل الله ملاكه فأوقف النار المشتعلة. لكن الجروح بقيت على أجسادهم

\par 4 فقال الله لآدم: "انظر إلى محبة الشيطان لك، الذي تظاهر بأنه يمنحك الألوهية والعظمة؛ وها هو يحرقك بالنار، ويسعى إلى إبادتك عن الأرض."

\par 5 «ثم انظر إليّ يا آدم، أنا خلقتك، فكم مرة أنقذتك من يده؟ لو لم أفعل، ألم يكن ليهلكك؟»

\par 6 قال الله مرة أخرى لحواء: "ما الذي وعدكِ به في الجنة قائلاً: في وقت أكلكما من الشجرة، تنفتح أعينكما، وتصيران كآلهة، عارفين الخير والشر. ولكن ها هوذا قد أحرق أجسادكما بالنار، وأذقاكما طعم النار، طعم الجنة، وأذقاكما لهيب النار، وشرها، وسلطانها عليكما."

\par 7 لقد رأت عيناكم الخير الذي أخذه منكم، وبالحق فتح أعينكم. ورأيتم الجنة التي كنتم فيها معي، ورأيتم أيضًا الشر الذي أصابكم من الشيطان. أما اللاهوت، فلا يقدر أن يعطيكم إياه، ولا أن يُتمم كلامه لكم. بل كان قاسيًا عليكم وعلى نسلكم الذي سيأتي بعدكم.

\par 8 وسحب الله كلمته عنهم.

\chapter{47}

\par \textit{مخططات الشيطان الخاصة.}

\par 1 ثم دخل آدم وحواء الكهف، وهما لا يزالان يرتجفان من النار التي أحرقت جسديهما. فقال آدم لحواء:

\par 2 «هوذا النار قد أحرقت أجسادنا في هذا العالم؛ ولكن كيف سيكون الحال عندما نموت، ويعاقب الشيطان أرواحنا؟ أليس خلاصنا طويلًا وبعيدًا، ما لم يأتِ الله، ويُتمم وعده برحمته لنا؟»

\par 3 ثم دخل آدم وحواء الكهف، مباركين نفسيهما لدخولهما إياه مرة أخرى. لأنه كان في فكرهما ألا يدخلاه أبدًا، عندما رأيا النار حوله

\par 4 ولكن مع غروب الشمس، كانت النار لا تزال مشتعلة وتقترب من آدم وحواء في الكهف، فلم يتمكنا من النوم فيه. وبعد غروب الشمس، خرجا منه. كان هذا هو اليوم السابع والأربعون بعد خروجهما من الجنة

\par 5 ثم جاء آدم وحواء تحت قمة التل بجوار الجنة للنوم، كما اعتادا

\par 6 ووقفوا وصلوا إلى الله أن يغفر لهم خطاياهم، ثم ناموا تحت قمة الجبل

\par 7 لكن الشيطان، كاره كل خير، فكر في نفسه: بما أن الله وعد آدم بالخلاص بالعهد، وأنه سينقذه من كل المصاعب التي حلت به، لكنه لم يعدني بالعهد، ولن ينقذني من مصاعبي؛ بل بما أنه وعده أنه سيجعله هو ونسله يسكنون في المملكة التي كنت فيها ذات يوم، فسأقتل آدم

\par 8 "ستتخلص الأرض منه، وستترك لي وحدي، حتى أنه عندما يموت لن يكون لديه أي بذرة متبقية ليرث المملكة التي ستبقى مملكتي الخاصة، وسوف يكون الله في حاجة إلي، وسوف يعيدني إليها مع جيوشى."

\chapter{48}

\par \textit{الظهور الخامس للشيطان لآدم وحواء.}

\par 1 بعد ذلك، نادى الشيطان على جيوشه، فجاءوا إليه جميعًا، وقالوا له:

\par 2 "يا ربنا ماذا تفعل؟"

\par 3 ثم قال لهم: «تعلمون أن آدم هذا الذي خلقه الله من التراب هو الذي أخذ ملكوتنا. هلموا نجتمع ونقتله، أو نرمي عليه وعلى حواء حجرًا فنسحقهما تحته».

\par 4 ولما سمع جنود الشيطان هذا الكلام، جاءوا إلى جزء الجبل الذي كان آدم وحواء نائمين فيه.

\par 5 فأخذ الشيطان وجنوده صخرة عظيمة عريضة مستوية بلا عيب، وقالوا في أنفسهم: "إن كان هناك ثقب في الصخرة، فعندما سقطت عليهم، سقط الثقب في الصخرة عليهم، فينجوون ولا يموتون".

\par 6 ثم قال لجنوده: «ارفعوا هذا الحجر وألقوه عليهم لئلا يتدحرج عنهم إلى مكان آخر. ومتى ألقيتموه فاهربوا ولا تبقوا».

\par 7 ففعلوا كما أمرهم. ولكن عندما سقطت الصخرة من الجبل على آدم وحواء، أمر الله أن تصبح نوعًا من السقيفة فوقهما، لا تسبب لهما أي ضرر. وهكذا كان بأمر الله

\par 8 ولكن عندما سقطت الصخرة، ارتجت الأرض كلها معها، واهتزت من حجم الصخرة

\par 9 وبينما كان يهتز ويرتجف، استيقظ آدم وحواء من النوم، فوجدا نفسيهما تحت صخرة كالسقيفة. لكنهما لم يعرفا كيف هي؛ لأنهما عندما ناما كانا تحت السماء، وليس تحت سقيفة؛ وعندما رأياها خافا

\par 10 ثم قال آدم لحواء: "لماذا انحنى الجبل، وارتجت الأرض وزلزلت بسببنا؟ ولماذا بسطت هذه الصخرة نفسها علينا كالخيمة؟"

\par 11 «هل ينوي الله أن يبتلينا ويحبسنا في هذا السجن؟ أم سيغلق الأرض علينا؟»

\par 12 «إنه غاضب علينا لخروجنا من الكهف دون أمره؛ ولأننا فعلنا ذلك من تلقاء أنفسنا، دون استشارته، عندما غادرنا الكهف وجئنا إلى هذا المكان.»

\par 13 ثم قالت حواء: "إن كانت الأرض قد زلزلت من أجلنا، وصارت هذه الصخرة خيمة فوقنا بسبب معصيتنا، فويل لنا يا آدم، لأن عقابنا سيطول."

\par 14 «لكن قوموا وصلوا إلى الله ليخبرنا عن هذا، وما هي هذه الصخرة التي هي منصوبة علينا كالخيمة.»

\par 15 فقام آدم وصلى أمام الرب ليُعلمه بهذا الضيق. ووقف آدم يصلي إلى الصباح

\chapter{49}

\par \textit{النبوءة الأولى عن القيامة.}

\par 1 ثم جاءت كلمة الله وقالت:

\par 2 «يا آدم، من أشار عليك، عندما خرجتَ من الكهف، أن تأتي إلى هذا المكان؟»

\par 3 فقال آدم لله: يا رب، لقد أتينا إلى هذا المكان بسبب حرارة النار التي أصابتنا في داخل الكهف

\par 4 ثم قال الرب الإله لآدم: "يا آدم، أنت تخاف من حر النار ليلة واحدة، ولكن كيف سيكون الأمر وأنت تسكن في الجحيم؟"

\par 5 «ولكن يا آدم، لا تخف، ولا تقل في قلبك إني بسطت هذه الصخرة عليك كظلة لأبتليك بها.»

\par 6 «لقد جاء من الشيطان، الذي وعدك بالألوهية والجلال. هو الذي ألقى هذه الصخرة ليقتلك تحتها، وحواء معك، وبالتالي ليمنعك من العيش على الأرض.»

\par 7 "ولكن رحمةً بك، حين سقطت تلك الصخرة عليك، أمرتها أن تشكل سقيفة فوقك، والصخرة التي تحتك أن تنزل."

\par 8 "وهذه الآية يا آدم ستحدث لي عند مجيئي إلى الأرض: سيقيم الشيطان شعب اليهود ويضعونني في صخرة ويختمون علي حجرا كبيرا، فأبقى داخل تلك الصخرة ثلاثة أيام وثلاث ليال."

\par 9 «ولكن في اليوم الثالث سأقوم، ويكون خلاصًا لك يا آدم ولنسلك أن تؤمن بي. ولكن يا آدم، لن أخرجك من تحت هذه الصخرة حتى تمضي ثلاثة أيام وثلاث ليالٍ.»

\par 10 فرفع الله كلمته عن آدم.

\par 11 وأما آدم وحواء، فأقاما تحت الصخرة ثلاثة أيام وثلاث ليالٍ كما قال لهما الله

\par 12 وفعل الله بهم ذلك لأنهم تركوا كهفهم وجاءوا إلى هذا المكان نفسه دون أمر من الله

\par 13 ولكن بعد ثلاثة أيام وثلاث ليالٍ، فتح الله الصخرة وأخرجهم من تحتها. وكان لحمهم قد جف، واضطربت عيونهم وقلوبهم من البكاء والحزن

\chapter{50}

\par \textit{آدم وحواء يسعيان إلى تغطية عريهما.}

\par 1 فخرج آدم وحواء ودخلا مغارة الكنوز، ووقفا يصليان فيها ذلك اليوم كله إلى المساء.

\par 2 وحدث هذا في نهاية خمسين يومًا بعد خروجهم من الجنة

\par 3 لكن آدم وحواء قاما مرة أخرى وصليا إلى الله في الكهف طوال تلك الليلة، وتوسلا إليه طالبين الرحمة

\par 4 ولما أشرق النهار، قال آدم لحواء: "تعالي! لنذهب ونعمل بعض الأعمال لأجسادنا."

\par 5 فخرجوا من الكهف، ووصلوا إلى الحد الشمالي للبستان، باحثين عن شيء يغطون به أجسادهم. لكنهم لم يجدوا شيئًا، ولم يعرفوا كيف يقومون بالعمل. ومع ذلك، كانت أجسادهم ملطخة، وكانوا عاجزين عن الكلام من البرد والحر

\par 6 ثم وقف آدم وطلب من الله أن يريه شيئًا يستر به جسديهما

\par 7 ثم جاءه كلام الله وقال له: "يا آدم، خذ حواء وتعالى إلى شاطئ البحر، حيث صمتما من قبل. هناك ستجدان جلود غنم أكلت الأسود لحومها، وبقيت جلودها. خذوها واصنعوا لأنفسكم ثيابًا، والبسوا بها."

\chapter{51}

\par \textit{"ما جماله حتى تتبعه؟"}

\par 1 عندما سمع آدم هذه الكلمات من الله، أخذ حواء ونقلها من الطرف الشمالي من الجنة إلى الجنوب منها، بجانب نهر الماء، حيث كانا يصومان ذات مرة.

\par 2 ولكن بينما كانوا سائرين في الطريق، وقبل أن يصلوا إلى ذلك المكان، سمع الشيطان الشرير كلمة الله تتكلم مع آدم بشأن غطائه.

\par 3 فحزن عليه، وأسرع إلى المكان الذي كانت فيه جلود الغنم، بقصد أن يأخذها ويلقيها في البحر، أو يحرقها بالنار، حتى لا يجدها آدم وحواء.

\par 4 ولكن بينما كان على وشك أن يأخذهما، نزلت كلمة الله من السماء، وقيدته بجانبي الجلدين حتى اقترب آدم وحواء منه. ولكن عندما اقتربا منه، خافا منه ومن منظره البشع.

\par 5 ثم جاء كلمة الله إلى آدم وحواء، وقال لهما: «هذا هو الذي كان مختبئًا في الحية، والذي خدعكما، وجردكما من ثوب النور والمجد الذي كنتما فيه».

\par 6 هذا هو الذي وعدكم بالجلال والألوهية. فأين الجمال الذي كان عليه؟ أين لاهوته؟ أين نوره؟ أين المجد الذي حل عليه؟

\par 7 «أما الآن فقد أصبحت صورته بشعة، وقد أصبح بغيضًا بين الملائكة، وأصبح يُدعى شيطانًا.»

\par 8 «يا آدم، لقد أراد أن ينزع منك هذا الثوب الأرضي من جلود الغنم، ويهلكه، ولا يدعك تلبسه.»

\par 9 «فما جماله إذًا حتى تتبعوه؟ وماذا ربحتم من الاستماع إليه؟ انظروا إلى أعماله الشريرة ثم انظروا إليّ؛ إليّ أنا خالقكم، وإلى الأعمال الصالحة التي أفعلها لكم.»

\par 10 «انظر، لقد قيدته حتى أتيت ورأيته ورأيت ضعفه، فلم يبق معه قوة.»

\par 11 فأطلقه الله من قيوده.

\chapter{52}

\par \textit{آدم وحواء يخيطان القميص الأول.}

\par 1 بعد ذلك، لم ينطق آدم وحواء بكلمة أخرى، بل بكيا أمام الله بسبب خلقهما، ولأن جسديهما كانا بحاجة إلى غطاء أرضي

\par 2 ثم قال آدم لحواء: "يا حواء، هذا هو جلد البهائم الذي سنُغطى به. ولكن عندما نلبسه، إذا بعلامة الموت قد حلت علينا، لأن أصحاب هذه الجلود قد ماتوا وذبُلوا. هكذا نموت نحن أيضًا ونزول."

\par 3 ثم أخذ آدم وحواء الجلود، وعادا إلى كهف الكنوز، وعندما كانا بداخله، وقفا وصليا كعادتهما

\par 4 وفكروا كيف يصنعون ثيابًا من تلك الجلود؛ إذ لم تكن لديهم مهارة في ذلك

\par 5 ثم أرسل الله إليهم ملاكه ليُريهم كيف يعملون ذلك. وقال الملاك لآدم: «اخرج وأحضر بعض شوك النخل». فخرج آدم وأحضر كما أمره الملاك

\par 6 ثم ابتدأ الملاك أمامهم يُخرج الجلود، كما يُخرج قميصًا. وأخذ الشوك وغرزه في الجلود أمام أعينهم

\par 7 ثم وقف الملاك مرة أخرى وصلى إلى الله أن يتم إخفاء الأشواك في تلك الجلود، بحيث يتم خياطتها بخيط واحد.

\par 8 فكان ذلك بأمر الله، فصارا لباساً لآدم وحواء، فألبسهما إياه.

\par 9 ومن ذلك الوقت أصبحت عورة أجسادهم مغطاة عن رؤية أعين بعضهم البعض.

\par 10 وكان هذا في نهاية اليوم الحادي والخمسين.

\par 11 ثم لما غطوا أجساد آدم وحواء، وقفا يصليان، ويطلبان رحمة الرب ومغفرته، ويشكرانه على رحمته لهما وستر عورتهما. ولم ينقطعا عن الصلاة تلك الليلة كلها.

\par 12 ثم لما أشرق الفجر مع طلوع الشمس، صلوا كعادتهم، ثم خرجوا من الكهف

\par 13 فقال آدم لحواء: «بما أننا لا نعرف ما يوجد غربي هذا الكهف، فلنخرج ونراه اليوم». ثم خرجا واتجها نحو الحد الغربي



\chapter{53}

\par \textit{نبوءة الأراضي الغربية.}

\par 1 ولم يكونوا بعيدين عن الكهف، عندما جاء الشيطان نحوهم، واختبأ بينهم وبين الكهف، تحت شكل أسدين جائعين لمدة ثلاثة أيام بلا طعام، جاءا نحو آدم وحواء، وكأنهما يريدان تحطيمهما والتهامهما.

\par 2 ثم بكى آدم وحواء، وصليا إلى الله أن ينقذهما من براثنهما

\par 3 ثم جاءت إليهم كلمة الله، وطردت الأسود عنهم

\par 4 فقال الله لآدم: «يا آدم، ماذا تطلب على الحد الغربي؟ ولماذا تركت من تلقاء نفسك الحد الشرقي الذي كان مسكنك فيه؟»

\par 5 «الآن، ارجع إلى كهفك، وابق فيه، لئلا يخدعك الشيطان، ولا يُنفذ غرضه عليك.»

\par 6 «لأنه في هذه الحدود الغربية، يا آدم، ستأتي منك بذرة تُكملها؛ وستُدنّس نفسها بخطاياها، وبخضوعها لأوامر الشيطان، وباتباع أعماله.»

\par 7 «لذلك أجلب عليهم مياه طوفان، وأغمرهم جميعًا. وأنقذ ما تبقى من الصالحين بينهم، وآتي بهم إلى أرض بعيدة، والأرض التي أنت ساكن فيها الآن تبقى خربة بلا ساكن فيها.»

\par 8 بعد أن خاطبهم الله هكذا، عادوا إلى كهف الكنوز. لكن لحمهم كان قد جف، وضعف قوتهم من الصوم والصلاة، ومن الحزن الذي شعروا به لتعديهم على الله

\chapter{54}

\par \textit{آدم وحواء يذهبان للاستكشاف.}

\par 1 ثم قام آدم وحواء في الكهف وصليا تلك الليلة كلها حتى طلع الصباح. ولما أشرقت الشمس خرجا كلاهما من الكهف، ورأساهما تائهان من ثقل الحزن، لا يدريان إلى أين يذهبان.

\par 2 وساروا هكذا إلى الحد الجنوبي للحديقة، وابتدأوا يصعدون ذلك الحد حتى وصلوا إلى الحد الشرقي الذي لم يكن وراءه مكان.

\par 3 وكان الكروب حارس الجنة واقفًا عند الباب الغربي، يحرسها من آدم وحواء لئلا يدخلا الجنة فجأة. فالتفت الكروب كأنه يريد قتلهما، حسب الوصية التي أعطاه إياها الله.

\par 4 عندما وصل آدم وحواء إلى الطرف الشرقي من الجنة، ظانّين في قلوبهما أن الكروب غافل، وبينما كانا واقفين عند الباب كأنهما يريدان الدخول، ظهر الكروب فجأةً وفي يده سيفٌ من نارٍ متقدة، فلما رآهما خرج ليقتلهما، لأنه خاف أن يُهلكه الله إن دخلا الجنة دون أمره.

\par 5 وبدا سيف الكروب وكأنه يشتعل من بعيد. ولكن عندما رفعه على آدم وحواء، لم يشتعل لهيبه.

\par 6 فظنّ الكروب أن الله قد رضي عنهم، وأنه يردّهم إلى الجنة. فوقف الكروب متعجبًا.

\par 7 ولم يكن بوسعه أن يصعد إلى السماء ليتأكد من أمر الله بشأن دخولهم الجنة؛ لذلك بقي واقفا بجانبهم، غير قادر على الانفصال عنهم؛ لأنه كان خائفا من أن يدخلوا الجنة بدون إذن من الله، فيهلكه الله حينئذ.

\par 8 عندما رأى آدم وحواء الكروب قادمًا نحوهما وفي يده سيف ملتهب من نار، سقطا على وجهيهما من الخوف، وكانا كموتى

\par 9 في ذلك الوقت اهتزت السماوات والأرض، ونزل كروبيم آخرون من السماء إلى الكروب حارس الجنة، ورأوه مذهولاً وصامتًا

\par 10 ثم نزل ملائكة آخرون بالقرب من المكان الذي كان فيه آدم وحواء. وكانوا منقسمين بين الفرح والحزن

\par 11 لقد فرحوا، لأنهم اعتقدوا أن الله كان مُرضيًا لآدم، وأراد له أن يعود إلى الجنة؛ وأرادوا أن يعيد إليه السعادة التي كان يتمتع بها ذات يوم

\par 12 لكنهم حزنوا على آدم، لأنه سقط كرجل ميت، هو وحواء؛ وقالوا في أفكارهم: "لم يمت آدم في هذا المكان؛ لكن الله قتله، لأنه جاء إلى هذا المكان، وأراد دخول الجنة دون إذنه."



\chapter{55}

\par \textit{صراع الشيطان.}

\par 1 ثم جاء كلمة الله إلى آدم وحواء، وأقامهما من حالتهما الميتة، قائلاً لهما: "لماذا صعدتما إلى هنا؟ يا سيد، هل تنويان دخول الجنة التي أخرجتكما منها؟ لا يمكن أن يكون ذلك اليوم، بل عندما يتم العهد الذي قطعته معكما."

\par 2 ثم لما سمع آدم كلمة الله، ورفرفة الملائكة الذين لم يرهم، بل سمع صوتهم فقط بأذنيه، بكى هو وحواء، وقالا للملائكة:

\par 3 «يا أيها الأرواح، يا من تنتظرون الله، انظروا إليّ، وإلى عدم قدرتي على رؤيتكم! فعندما كنت في طبيعتي المشرقة السابقة، كان بإمكاني رؤيتكم. كنتُ أُسبّح كما تفعلون؛ وكان قلبي فوقكم بكثير.»

\par 4 «لكن الآن، بعد أن تجاوزت الحدود، ذهبت تلك الطبيعة المشرقة مني، ووصلت إلى هذه الحالة البائسة. والآن وصلت إلى هذا، حيث لا أستطيع رؤيتك، وأنت لا تخدمني كما كنت معتادًا. لأنني أصبحت جسدًا حيوانيًا.»

\par 5 «والآن يا ملائكة الله، اطلبوا من الله معي أن يعيدني إلى ما كنت عليه سابقًا؛ وأن ينقذني من هذا البؤس، وأن يرفع عني حكم الموت الذي أصدره عليّ، لتعديي عليه.»

\par 6 ثم لما سمع الملائكة هذه الكلمات، حزنوا عليه جميعًا، ولعنوا الشيطان الذي أغوى آدم حتى خرج من الجنة إلى الشقاء، ومن الحياة إلى الموت، ومن السلام إلى المتاعب، ومن السعادة إلى أرض غريبة

\par 7 ثم قال الملائكة لآدم: "لقد استمعت للشيطان، وتركت كلمة الله الذي خلقك، وآمنت أن الشيطان سيحقق كل ما وعدك به."

\par 8 "ولكن الآن يا آدم سنعرفك بما حدث لنا بواسطته قبل سقوطه من السماء."

\par 9 "فجمع جيوشه وخدعهم، ووعدهم أن يعطيهم ملكوتًا عظيمًا، وطبيعة إلهية، ووعودًا أخرى صنعها لهم."

\par 10 «آمن جحافله أن كلمته صادقة، فاستسلموا له، ونبذوا مجد الله.»

\par 11 «ثم أرسل إلينا وفقًا للأوامر التي كانت علينا - أن نخضع لقيادته، وأن نستمع إلى وعده الباطل. لكننا لم نرد، ولم نأخذ بنصيحته.»

\par 12 «ثم بعد أن حارب الله، وتصرف معه بحزم، جمع جيوشه، وحاربنا. ولولا قوة الله التي كانت معنا، لما تغلبنا عليه حتى نطرحه من السماء.»

\par 13 «ولكن لما سقط من بيننا، كان فرح عظيم في السماء بسبب نزوله عنا. لأنه لو بقي في السماء، لما بقي فيها شيء، ولا ملاك واحد.»

\par 14 «لكن الله برحمته، طرده من بيننا إلى هذه الأرض المظلمة؛ لأنه أصبح هو نفسه الظلمة وفاعل إثم.»

\par 15 «ولقد استمر يا آدم في شن الحرب عليك، حتى أغواك وأخرجك من الجنة، إلى هذه الأرض الغريبة، حيث أتت إليك كل هذه التجارب. والموت الذي جلبه الله عليه جلبه لك أيضًا يا آدم، لأنك أطعته وعصيت الله.»

\par 16 ثم فرح الملائكة وسبّحوا الله، وطلبوا منه ألا يُهلك آدم هذه المرة، لأنه سعى لدخول الجنة؛ بل أن يتحمله حتى يتم الوعد؛ وأن يساعده في هذا العالم حتى يتحرر من يد الشيطان

\chapter{56}

\par \textit{فصل من التعزية الإلهية.}

\par 1 ثم جاء كلمة الله إلى آدم وقال له:

\par 2 يا آدم، انظر إلى حديقة الفرح هذه وإلى أرض العناء هذه، وانظر إلى الملائكة الذين في الجنة المليئة بهم، وانظر إلى نفسك وحدك على هذه الأرض، مع الشيطان الذي أطعته

\par 3 «ومع ذلك، لو كنت قد خضعت وأطعتني، وحفظت كلمتي، لكنت مع ملائكتي في حديقتي.»

\par 4 «ولكن عندما عصيتَ الشيطان وأصغيتَ إليه، أصبحتَ ضيفه بين ملائكته، الممتلئين بالشر؛ وأتيتَ إلى هذه الأرض، التي تُنبت لك الأشواك والحسك.»

\par 5 «يا آدم، اطلب ممن خدعك أن يمنحك الطبيعة الإلهية التي وعدك بها، أو أن يصنع لك جنة كما صنعتها لك؛ أو أن يملأك بنفس الطبيعة المشرقة التي ملأتك بها.»

\par 6 «اطلب منه أن يصنع لك جسدًا مثل الذي صنعته لك، أو أن يمنحك يومًا من الراحة كما منحتك إياه؛ أو أن يخلق في داخلك روحًا عاقلة كما خلقتك من أجلك؛ أو أن ينقلك من هنا إلى أرض أخرى غير هذه التي منحتك إياها. لكن يا آدم، لن يُنجز لك ولو شيئًا واحدًا مما قاله لك.»

\par 7 "اعترف إذن بفضلي عليك ورحمتي عليك يا مخلوقي؛ أنني لم أجازيك على معصيتك لي، ولكن في شفقتي عليك وعدتك أنه في نهاية الأيام الخمسة والنصف العظيمة سوف آتي وأخلصك."

\par 8 ثم قال الله مرة أخرى لآدم وحواء: «قوما وانزلا من هنا لئلا يهلككما الكروب وسيف النار في يده».

\par 9 لكن قلب آدم تعزى بكلام الله له، وسجد أمامه

\par 10 وأمر الله ملائكته بمرافقة آدم وحواء إلى الكهف بفرح، بدلًا من الخوف الذي أصابهما

\par 11 ثم أخذ الملائكة آدم وحواء، وأنزلوهما من الجبل بجوار الجنة، مع الترانيم والمزامير، حتى أحضروهما إلى الكهف. هناك بدأ الملائكة بتعزيتهما وتقويتهما، ثم انطلقوا عنهما نحو السماء، إلى خالقهما الذي أرسلهما

\par 12 ولكن بعد أن ذهبت الملائكة من عند آدم وحواء، جاء الشيطان، خجلاً، ووقف عند مدخل الكهف الذي كان فيه آدم وحواء. ثم نادى آدم، وقال: "يا آدم، تعال، دعني أكلمك."

\par 13 ثم خرج آدم من الكهف، ظانًا أنه أحد ملائكة الله الذين جاءوا ليقدموا له بعض النصائح الجيدة

\chapter{57}

\par \textit{"لذلك سقطت..."}

\par 1 ولكن لما خرج آدم ورأى صورته القبيحة خاف منه وقال له: من أنت؟

\par 2 فأجابه الشيطان وقال له: "أنا الذي اختبأت داخل الحية، وكلمت حواء، وخدعتها حتى استجابت لأمري. أنا الذي أرسلتها، من خلال مكائد كلامي، لتخدعك، حتى أكلت أنت وهي من ثمرة الشجرة، وخرجتما من تحت أمر الله."

\par 3 فلما سمع آدم هذه الكلمات منه، قال له: "أتستطيع أن تصنع لي جنة كما صنع الله لي؟ أم تستطيع أن تلبسني نفس الطبيعة المضيئة التي كساني الله إياها؟"

\par 4 «أين الطبيعة الإلهية التي وعدتني أن تمنحني إياها؟ أين كلامك الجميل الذي حملته معنا في البداية، عندما كنا في الجنة؟»

\par 5 ثم قال الشيطان لآدم: "أتظن أنه عندما أتحدث إلى أحد عن أي شيء، سأحضره إليه أو أفي بكلمتي؟ كلا. لأني لم أفكر قط في الحصول على ما طلبته."

\par 6 «لذلك سقطتُ، وجعلتُكم تسقطون بسبب ما سقطتُ بسببه أنا نفسي؛ ومعكم أيضًا، كل من يقبل نصيحتي يسقط بذلك.»

\par 7 «ولكن الآن يا آدم، بسبب سقوطك، أنت تحت حكمي، وأنا ملك عليك؛ لأنك سمعت لي، وعصيت إلهك. ولن يكون هناك خلاص من يدي حتى اليوم الذي وعدك به إلهك.»

\par 8 وقال أيضًا: «بما أننا لا نعرف اليوم المتفق عليه معك من إلهك، ولا الساعة التي تنقذ فيها، لذلك سنكثر الحرب والقتل عليك وعلى نسلك من بعدك».

\par 9 «هذه هي إرادتنا ومسرتنا، أن لا نترك أحدًا من أبناء البشر يرث رتبتنا في السماء.»

\par 10 «أما مسكننا يا آدم، فهو في نار متقدة؛ ولن نكف عن فعل الشر، لا يومًا ولا ساعة. وأنا يا آدم، سأزرع عليك نارًا عندما تدخل الكهف لتسكن هناك.»

\par 11 عندما سمع آدم هذه الكلمات، بكى ونوح، وقال لحواء: "اسمعي ما قاله؛ إنه لن يُتمم شيئًا مما قاله لكِ في الجنة. فهل أصبح حقًا ملكًا علينا؟"

\par 12 «لكننا سنطلب من الله الذي خلقنا أن ينقذنا من يديه.»

\chapter{58}

\par \textit{"حول غروب الشمس في اليوم الثالث والخمسين..."}

\par 1 ثم مد آدم وحواء أيديهما إلى الله، يصليان ويتوسلان إليه أن يطرد الشيطان عنهما، وألا يستخدم معهما أي عنف، وألا يرغمهما على إنكار الله.

\par 2 فأرسل الله إليهم في الحال ملاكه، فطرد الشيطان عنهم. حدث هذا عند غروب الشمس، في اليوم الثالث والخمسين بعد خروجهم من الجنة

\par 3 ثم دخل آدم وحواء الكهف، وقاما وأدارا وجهيهما إلى الأرض ليصليا إلى الله

\par 4 ولكن قبل أن يصليا، قال آدم لحواء: "انظري، لقد رأيتِ التجارب التي حلت بنا في هذه الأرض. تعالي، فلنقم، ونطلب من الله أن يغفر لنا خطايانا التي ارتكبناها؛ ولن نخرج حتى نهاية اليوم الذي يقارب الأربعين. وإذا متنا هنا، فسوف يخلصنا."

\par 5 ثم قام آدم وحواء، وانضما إلى التضرع إلى الله.

\par 6 فبقوا في المغارة يصلون، ولم يخرجوا منها ليلاً ولا نهاراً حتى ارتفعت صلاتهم من أفواههم كلهيب نار.

\chapter{59}

\par \textit{الظهور الثامن للشيطان لآدم وحواء.}

\par 1 لكن الشيطان، كاره كل خير، لم يسمح لهما بإنهاء صلاتهما. لأنه نادى على جنوده، فأتوا جميعًا. ثم قال لهم: "بما أن آدم وحواء، اللذين خدعناهما، قد اتفقا على الصلاة إلى الله ليلًا ونهارًا، والتضرع إليه لإنقاذهما، وبما أنهما لن يخرجا من الكهف حتى نهاية اليوم الأربعين."

\par 2 "ولما كانا سيواصلان صلاتهما كما اتفقا، لينقذهما من أيدينا ويعيدهما إلى حالتهما الأولى، فانظر ماذا نفعل بهما." فقال له جنوده: "لك القدرة يا سيدنا أن تفعل ما تشاء."

\par 3 ثم أخذ الشيطان العظيم جيشه ودخل المغارة في الليلة الثلاثين من الأربعين يوماً وواحداً، وضرب آدم وحواء حتى تركهما ميتين.

\par 4 ثم جاءت كلمة الله إلى آدم وحواء، فأقامهما من معاناتهما، وقال الله لآدم: "كن قوياً ولا تخف من الذي جاء إليك الآن".

\par 5 فبكى آدم وقال: "أين كنت يا إلهي حتى يضربوني بمثل هذه الضربات، ويحل بنا هذا العذاب، عليّ وعلى حواء أمتك؟"

\par 6 ثم قال له الله: "يا آدم، انظر، هو رب وسيد كل ما لديك، هو الذي قال إنه سيمنحك الألوهية. أين هذا الحب لك؟ وأين العطية التي وعد بها؟"

\par 7 «لمرة واحدة سُرّ يا آدم أن يأتي إليك، ليعزيك، ويقويك، ويفرح معك، ويرسل جنوده لحراستك؛ لأنك استمعت إليه، واستسلمت لمشورته؛ وتجاوزت وصيتي، بل اتبعت أمره؟»

\par 8 ثم بكى آدم أمام الرب، وقال: "يا رب، لأني تجاوزت قليلاً، فقد عاقبتني بشدة، أطلب منك أن تنقذني من يديه؛ أو ترحمني، وتخرج روحي من جسدي الآن في هذه الأرض الغريبة."

\par 9 ثم قال الله لآدم: "لو كان هناك هذا التنهد والصلاة من قبل، قبل أن ترتكب المعصية! لكنت استرحت من الضيق الذي أنت فيه الآن."

\par 10 لكن الله صبر على آدم، وتركه وحواء يبقيان في الكهف حتى أكملا الأربعين يومًا

\par 11 أما آدم وحواء، فقد ذبلت قوتهما وجسدهما من الصوم والصلاة، من الجوع والعطش؛ لأنهما لم يتذوقا طعامًا أو شرابًا منذ أن غادرا الجنة؛ ولم تكن وظائف جسديهما قد استقرت بعد؛ ولم تبق لهما قوة للاستمرار في الصلاة من الجوع، حتى نهاية اليوم التالي إلى الأربعين. لقد سقطا في الكهف؛ ومع ذلك، فإن ما خرج من أفواههما لم يكن سوى التسبيح

\chapter{60}

\par \textit{يظهر الشيطان كرجل عجوز. يقدم "مكانًا للراحة".}

\par 1 ثم في اليوم التاسع والثمانين، جاء الشيطان إلى الكهف، مرتديًا ثوبًا من نور، ومتمنطقًا بمنطقة مضيئة

\par 2 كان في يديه عصا من نور، وكان يبدو في غاية البشاعة: لكن وجهه كان لطيفًا وكلامه كان حلوًا،

\par 3 وهكذا غيّر نفسه ليخدع آدم وحواء، وليخرجهما من الكهف، قبل أن يُكملا الأربعين يومًا

\par 4 لأنه قال في نفسه: "الآن بعد أن أكملوا الأربعين يومًا من الصيام والصلاة، سيعيدهم الله إلى حالتهم الأولى؛ ولكن إن لم يفعل ذلك، فسيظل راضيًا عنهم؛ وحتى لو لم يرحمهم، فهل سيعطيهم شيئًا من الجنة ليعزيهم؛ كما فعل مرتين من قبل."

\par 5 ثم اقترب الشيطان من الكهف بهذا المظهر الجميل، وقال:

\par 6 يا آدم، انهض، وقف أنت وحواء، وتعاليا معي إلى أرض طيبة؛ ولا تخافا. أنا لحم وعظام مثلكما؛ وفي البداية كنت مخلوقًا خلقه الله

\par 7 "وكان كذلك، أنه لما خلقني، وضعني في جنة في الشمال، على أقاصي العالم."

\par 8 "فقال لي: امكث هنا، فأقمت هناك حسب قوله، ولم أتعدَّ وصيته."

\par 9 «ثم نام عليّ، فأخرجك يا آدم من جنبي، ولم يُثبّتك عندي».

\par 10 «لكن الله أخذك بيده الإلهية، ووضعك في جنة شرقًا.»

\par 11 «ثم حزنتُ عليك، لأنه بينما أخذك الله من جانبي، لم يدعك تبقى معي.»

\par 12 «لكن الله قال لي: لا تحزن بسبب آدم الذي أخرجته من ضلعك، فلن يصيبه أذى».

\par 13 «لأني الآن قد أخرجت له من جنبه معينًا، وأعطيته فرحًا بذلك.»

\par 14 ثم قال الشيطان مرة أخرى: "لم أكن أعرف كيف أنتم في هذا الكهف، ولا أي شيء عن هذه المحنة التي حلت بكم - حتى قال لي الله: هوذا آدم قد تعدى، الذي أخذته من ضلعك، وحواء أيضًا التي أخذتها من ضلعه؛ وطردتهما من الجنة؛ وأسكنتهما في أرض حزن وبؤس، لأنهما تعديا عليّ، وأنصتا إلى الشيطان. وها هما في معاناة إلى هذا اليوم، الثمانين."

\par 15 «ثم قال لي الله: قم، اذهب إليهم، واجعلهم يأتون إلى مكانك، ولا تدع الشيطان يقترب منهم ويؤذيهم. لأنهم الآن في بؤس عظيم، ويرقدون عاجزين من الجوع.»

\par 16 «ثم قال لي أيضًا: 'عندما تأخذهم إليك، أعطهم ليأكلوا من ثمرة شجرة الحياة، واسقهم من ماء السلام؛ وألبسهم ثوب النور، وأعدهم إلى حالتهم السابقة من النعمة، ولا تتركهم في بؤس، لأنهم خرجوا منك. ولكن لا تحزن عليهم، ولا تندم على ما أصابهم.'»

\par 17 «ولكن عندما سمعت هذا، حزنت؛ ولم يستطع قلبي أن يتحمله بصبر من أجلك يا بني.»

\par 18 «لكن يا آدم، عندما سمعت اسم الشيطان، خفت، وقلت في نفسي: لن أخرج، لئلا يوقعني في الفخ، كما فعل مع ابنيّ آدم وحواء.»

\par 19 فقلت: يا رب، عندما أذهب إلى أبنائي، سيقابلني الشيطان في الطريق، ويحاربني كما فعل معهم

\par 20 «ثم قال لي الله: لا تخف. متى وجدته فاضربه بالعصا التي في يدك، ولا تخف منه لأنك قديم، ولن يقدر عليك.»

\par 21 «ثم قلت: يا سيدي، أنا عجوز ولا أستطيع الذهاب. أرسل ملائكتك ليحضروهم.»

\par 22 «لكن الله قال لي: إن الملائكة، حقًا، ليسوا مثلهم، ولن يوافقوا على المجيء معهم. لكنني اخترتك، لأنهم ذريتك، ومثلك، وسيستمعون إلى ما تقوله.»

\par 23 قال لي الله أيضًا: «إن لم تكن لديك قوة على المشي، فسأرسل سحابة لتحملك وتنزلك عند مدخل كهفهم؛ ثم تعود السحابة وتتركك هناك».

\par 24 "وإن جاءوا معك أرسل سحابة تحملك وإياهم."

\par 25 «ثم أمر سحابة فحملتني وجاءت بي إليك، ثم رجعت».

\par 26 «والآن يا ولديّ، آدم وحواء، انظرا إلى شعري الأبيض، وإلى وضعي الهزيل، وإلى قدومي من ذلك المكان البعيد. تعالا، تعالا معي، إلى مكان الراحة.»

\par 27 ثم بدأ يبكي وينوح أمام آدم وحواء، وانهمرت دموعه على الأرض كالماء

\par 28 ولما رفع آدم وحواء أعينهما، ورأيا لحيته، وسمعا كلامه المعسول، لانت قلوبهما تجاهه، وأصغيا إليه، لأنهما آمنا أنه صادق

\par 29 فظهر لهم أنهم ذريته حقًا، لما رأوا وجهه يشبه وجوههم، فوثقوا به

\chapter{61}

\par \textit{يبدأون باتباع الشيطان.}

\par 1 ثم أخذ بيد آدم وحواء، وأخرجهما من الكهف.

\par 2 ولكن عندما ابتعدوا قليلاً، علم الله أن الشيطان قد غلبهم، وأخرجهم قبل انقضاء الأربعين يومًا، ليأخذهم إلى مكان بعيد، ويهلكهم

\par 3 ثم جاءت كلمة الرب الإله مرة أخرى ولعنت الشيطان وطردته عنهم

\par 4 وبدأ الله يُكلِّم آدم وحواء قائلًا لهما: «ما الذي أخرجكما من الكهف إلى هذا المكان؟»

\par 5 ثم قال آدم لله: "هل خلقت إنسانًا قبلنا؟ لأنه بينما كنا في الكهف، جاء إلينا فجأة رجل عجوز صالح وقال لنا: أنا رسول من الله إليكم لأعيدكم إلى مكان للراحة."

\par 6 «وآمنا يا الله أنه رسول منك، وخرجنا معه، ولم نكن نعلم إلى أين نذهب معه».

\par 7 ثم قال الله لآدم: "انظر، هذا هو أبو فنون الشر، الذي أخرجك أنت وحواء من جنة النعيم. والآن، في الواقع، عندما رأى أنك وحواء قد انضممتما معًا في الصوم والصلاة، وأنكما لم تخرجا من الكهف قبل نهاية الأربعين يومًا، أراد أن يجعل هدفكما باطلًا، وأن يقطع رباطكما المتبادل، وأن يقطع كل أمل عنكما، وأن يدفعكما إلى مكان قد يهلككما فيه."

\par 8 «لأنه لم يكن قادرًا على أن يفعل لك شيئًا، إلا إذا أظهر نفسه في شبهك.»

\par 9 «لذلك جاء إليك بوجه مثل وجهك، وبدأ يعطيك الرموز كما لو كانت جميعها حقيقية.»

\par 10 «لكنني، برحمتي وإحساني إليكم، لم أسمح له بإهلاككم، بل طردته عنكم.»

\par 11 «والآن يا آدم، خذ حواء، وارجع إلى كهفك، وابق فيه حتى غدٍ من اليوم الأربعين. وعندما تخرج، اذهب نحو البوابة الشرقية للجنة.»

\par 12 ثم سجد آدم وحواء لله، وسبحاه وباركاه على النجاة التي أتتهما منه. وعادا إلى الكهف. حدث ذلك في مساء اليوم التاسع والثلاثين

\par 13 ثم قام آدم وحواء، وصليا إلى الله بحرارة شديدة، ليُخرجهما من قلة قوتهما، فقد فارقتهما قوتهما بسبب الجوع والعطش والصلاة. لكنهما سهرا تلك الليلة كلها يصليان حتى الصباح.

\par 14 ثم قال آدم لحواء: «قومي، فلنذهب نحو الباب الشرقي للجنة كما أمرنا الله».

\par 15 وصلّوا كما اعتادوا أن يفعلوا كل يوم، وخرجوا من الكهف ليقتربوا من الباب الشرقي للبستان

\par 16 ثم قام آدم وحواء وصليا، وتوسلا إلى الله أن يقويهما، وأن يرسل لهما شيئًا يسد جوعهما

\par 17 ولكن عندما انتهوا من صلاتهم، بقوا حيث كانوا بسبب ضعف قوتهم

\par 18. ثم جاء كلمة الله مرة أخرى وقال لهم: "يا آدم، قم واذهب وأحضر هنا تينتين."

\par 19 ثم قام آدم وحواء، وذهبا حتى اقتربا من الكهف.

\chapter{62}

\par \textit{شجرتا فاكهة.}

\par 1 لكن الشيطان الشرير حسد بسبب التعزية التي منحها الله لهم

\par 2 فمنعهما، ودخل الكهف وأخذ التينتين، ودفنهما خارج الكهف، حتى لا يجدهما آدم وحواء. وكان في أفكاره أيضًا إهلاكهما

\par 3 ولكن بفضل رحمة الله، حالما أصبحت هاتان التينتان في الأرض، هزم الله مشورة الشيطان بشأنهما؛ وجعلهما شجرتي فاكهة، ظللتا الكهف. لأن الشيطان دفنهما على جانبه الشرقي

\par 4 فلما نمت الشجرتان وغطتا بالثمر، حزن الشيطان ونوح، وقال: «كان من الأفضل لو تركا التين كما هما؛ لأنه الآن ها قد صارتا شجرتي ثمر، يأكل منهما آدم كل أيام حياته. أما أنا فقد كنت أفكر، عندما دفنتهما، أن أهلكهما تمامًا وأخفيهما إلى الأبد».

\par 5 «لكن الله قلب مشورتي، ولم يُرِد أن تهلك هذه الثمرة المقدسة، وأوضح نيتي، وأبطل المشورة التي كنت قد كونتها ضد عباده.»

\par 6 ثم مضى الشيطان خجلاً لأنه لم يُتمم مشيئته.

\chapter{63}

\par \textit{الفرحة الأولى للأشجار.}

\par 1 ولكن آدم وحواء، عندما اقتربا من الكهف، رأيا شجرتي تين مملوءتين بالثمر وكانتا تظللان الكهف.

\par 2 ثم قال آدم لحواء: «يبدو لي أننا ضللنا الطريق. متى نمت هاتان الشجرتان هنا؟ يبدو لي أن العدو يريد أن يضلنا. أتقولين إن في الأرض مغارة أخرى غير هذه؟»

\par 3 يا حواء، لندخل الكهف ونجد فيه التينتين، فهذا كهفنا الذي كنا فيه. وإن لم نجد فيه التينتين، فلا يكون كهفنا.

\par 4 فدخلوا المغارة ونظروا في زواياها الأربع فلم يجدوا التينتين.

\par 5 فبكى آدم وقال لحواء: "ألم ندخل مغارةً يا حواء؟ يبدو لي أن هاتين التينتين هما التينتان اللتان كانتا في المغارة". فقالت حواء: "أنا لا أعرف".

\par 6 ثم قام آدم وصلى وقال: "يا رب، لقد أمرتنا بالعودة إلى الكهف، وأخذ التينتين، ثم العودة إليك."

\par 7 «لكننا الآن لم نجدهما. يا الله، هل أخذتهما وزرعت هاتين الشجرتين، أم ضللنا في الأرض، أم خدعنا العدو؟ إن كان ذلك حقيقيًا، يا الله، فاكشف لنا سر هاتين الشجرتين والتينتين.»

\par 8 ثم جاء كلام الله إلى آدم، وقال له: "يا آدم، عندما أرسلتك لتأخذ التين، سبقك الشيطان إلى الكهف، وأخذ التين، ودفنه خارجًا، شرقي الكهف، ظانًا أنه سيُهلكه، ولم يزرعه بنية حسنة."

\par 9 «ليس من أجله وحده، إذًا، نمت هذه الأشجار دفعة واحدة؛ بل رحمتك وأمرتها أن تنمو. فنمتا وصارتا شجرتين كبيرتين، لكي تظللكما أغصانهما، وتجدا راحة؛ ولكي أجعلكما ترى قوتي وأعمالي العجيبة.»

\par 10 «وأيضًا، لأريكم دناءة الشيطان وأعماله الشريرة، فمنذ أن خرجتم من الجنة، لم يتوقف، لا، ولا يومًا واحدًا، عن إيذائكم. لكنني لم أعطه سلطانًا عليكم.»

\par 11 وقال الله: "من الآن فصاعدًا يا آدم، افرح أنت وحواء بالأشجار؛ واستريحا تحتها عندما تشعران بالتعب. ولكن لا تأكلا من ثمارها، ولا تقتربا منها."

\par 12 فبكى آدم وقال: يا الله، أتقتلنا مرة أخرى، أم تطردنا من أمام وجهك، وتقطع حياتنا عن وجه الأرض؟

\par 13 «يا الله، أتوسل إليك، إن كنت تعلم أن في هذه الأشجار موتًا أو شرًا آخر، كما في المرة الأولى، فاقتلعها من قرب كهفنا، وأذبلها؛ واتركنا نموت من الحر والجوع والعطش.»

\par 14 «لأننا نعلم عجائبك يا الله، أنها عظيمة، وأنك بقدرتك تستطيع أن تُخرج شيئًا من شيء، دون إرادة أحد. لأن قدرتك قادرة على أن تجعل الصخور أشجارًا، والأشجار صخورًا.»

\chapter{64}

\par \textit{آدم وحواء يتناولان أول طعام أرضي.}

\par 1 ثم نظر الله إلى آدم، وإلى قوة نفسه، وإلى احتماله الجوع والعطش والحر. فحوّل التينتين إلى تينتين كما كانتا في البداية، ثم قال لآدم وحواء: «خذا كل واحد منكما تينة واحدة». فأخذاها كما أمرهما الرب.

\par 2 فقال لهم: «اذهبوا إلى المغارة وكلوا التين وأشبعوا لئلا تموتوا».

\par 3 فدخلا الكهف كما أمرهما الله، عند غروب الشمس تقريبًا. وقام آدم وحواء وصليا عند غروب الشمس

\par 4 ثم جلسوا ليأكلوا التين، لكنهم لم يعرفوا كيف يأكلونه، لأنهم لم يكونوا معتادين على أكل الطعام الأرضي. خافوا أيضًا أن تثقل بطونهم، ويغلظ لحمهم، فتميل قلوبهم إلى أكل الطعام الأرضي.

\par 5 ولكن بينما هم جالسون هكذا، أرسل الله ملاكه إليهم، شفقةً عليهم، لئلا يهلكوا من الجوع والعطش

\par 6 وقال الملاك لآدم وحواء: "يقول لكما الله إنكما لا تملكان القوة على الصيام حتى الموت؛ فكلا إذن، وقوّيا أجسادكما؛ لأنكما الآن جسد حيوان، لا تستطيعان البقاء بدون طعام وشراب."

\par 7 ثم أخذ آدم وحواء التين وبدءا يأكلان منه. لكن الله جعل فيهما مزيجًا من خبز لذيذ ودم

\par 8 ثم ذهب الملاك من عند آدم وحواء، فأكلا من التين حتى أشبعا جوعهما. ثم وضعا ما بقى؛ ولكن بقدرة الله، شبع التين كما كان من قبل، لأن الله باركهما

\par 9 بعد ذلك، قام آدم وحواء، وصليا بقلب فرح وقوة متجددة، وسبحا وفرحا كثيرًا طوال تلك الليلة. وكانت هذه نهاية اليوم الثالث والثمانين

\chapter{65}

\par \textit{اكتسب آدم وحواء أعضاءً هضمية. خمد الأمل الأخير في العودة إلى الجنة.}

\par 1 ولما كان النهار، قاموا وصلوا كعادتهم، ثم خرجوا من الكهف

\par 2 ولكن لما شعروا بضيق شديد من الطعام الذي أكلوه، والذي لم يعتادوا عليه، تجولوا في الكهف قائلين بعضهم لبعض:

\par 3 «ماذا حدث لنا من خلال الأكل حتى أصابنا هذا الألم؟ ويل لنا، سنموت! خير لنا أن نموت من أن نأكل؛ وأن نحافظ على أجسادنا نقية من أن ندنسها بالطعام.»

\par 4 ثم قال آدم لحواء: "لم ينزل بنا هذا الألم في الجنة، ولم نأكل هناك طعامًا رديئًا كهذا. أتظنين يا حواء أن الله سيعذبنا بالطعام الذي فينا، أو أن أحشائنا ستخرج، أو أن الله يقصد أن يقتلنا بهذا الألم قبل أن يفي بوعده لنا؟"

\par 5 ثم تضرع آدم إلى الرب وقال: «يا رب، لا نهلك بسبب الطعام الذي أكلناه. يا رب، لا تضربنا، بل عاملنا حسب رحمتك العظيمة، ولا تتركنا إلى يوم الموعد الذي وعدتنا به».

\par 6 ثم نظر الله إليهم، فأعدّهم للوقت ليأكلوا طعاماً إلى هذا اليوم، لكي لا يهلكوا.

\par 7 ثم عاد آدم وحواء إلى الكهف حزينين يبكيان على ما حلّ بهما من تغيير. ومنذ تلك الساعة، أدركا أنهما كائنان متغيران، وأن أملهما في العودة إلى الجنة قد انقطع، وأنهما لا يستطيعان دخولها.

\par 8 لأن أجسادهم الآن لها وظائف غريبة، وكل جسد يحتاج إلى طعام وشراب لوجوده لا يمكن أن يكون في الجنة.

\par 9 ثم قال آدم لحواء: «ها قد انقطع رجائنا، وانقطع رجاءنا في دخول الجنة. لم نعد من أهل الجنة، بل نحن من الآن فصاعدًا ترابيون من أهل الأرض، ولن نعود إلى الجنة إلا في اليوم الذي وعدنا الله فيه بخلاصنا وإعادتنا إليها كما وعدنا».

\par 10 ثم صلوا إلى الله أن يرحمهم؛ وبعد ذلك، هدأت عقولهم، وانكسرت قلوبهم، وبرد شوقهم؛ وأصبحوا كغرباء على الأرض. قضى آدم وحواء تلك الليلة في الكهف، حيث ناما نومًا عميقًا بسبب الطعام الذي أكلاه

\chapter{66}

\par \textit{آدم يقوم بعمله في يومه الأول.}

\par 1 ولما كان الصباح، في اليوم التالي لتناولهما الطعام، صلى آدم وحواء في الكهف، وقال آدم لحواء: "ها نحن طلبنا من الله طعامًا فأعطانا إياه. والآن فلنطلب منه أيضًا أن يسقينا ماءً".

\par 2 ثم نهضوا وذهبوا إلى ضفة جدول الماء، الذي كان على الحافة الجنوبية للحديقة، الذي ألقوا أنفسهم فيه سابقًا. ووقفوا على الضفة، وصلوا إلى الله أن يأمرهم بشرب الماء

\par 3 ثم جاء كلمة الله إلى آدم، وقال له: "يا آدم، لقد أصبح جسدك وحشيًا، ويحتاج إلى ماء ليشرب. خذ واشرب أنت وحواء؛ اشكرا وسبحا."

\par 4 ثم اقترب آدم وحواء منه، وشربا منه حتى انتعشت أجسادهما. وبعد أن شربا، سبّحا الله، ثم عادا إلى كهفهما، كعادتهما السابقة. حدث هذا في نهاية ثلاثة وثمانين يومًا

\par 5 ثم في اليوم الرابع والثمانين، أخذوا تينتين وعلقوهما في الكهف مع أوراقهما، لتكونا لهما علامة وبركة من الله. ووضعوهما هناك حتى ينشأ لهما نسل يرى العجائب التي صنعها الله بهم

\par 6 ثم وقف آدم وحواء مرة أخرى خارج الكهف، وطلبا من الله أن يريهما بعض الطعام الذي يغذي أجسادهما

\par 7 ثم جاء كلمة الله وقال له: "يا آدم، انزل إلى غرب الكهف، حتى تصل إلى أرض ذات تربة داكنة، وهناك تجد طعامًا."

\par 8 فاستمع آدم لكلمة الله، فأخذ حواء، ونزل إلى أرض مظلمة، فوجد هناك قمحًا نابتًا في سنبله، وتينًا ليأكله، ففرح آدم بذلك

\par 9 ثم جاء كلمة الله مرة أخرى إلى آدم، وقال له: "خذ من هذه القمحة واصنع لك منها خبزًا لتغذي به جسدك." وأعطى الله قلب آدم الحكمة، ليعمل الذرة حتى أصبحت خبزًا

\par 10 ففعل آدم كل ذلك حتى غلبه التعب والإرهاق. ثم عاد إلى الكهف فرحًا بما تعلمه مما يُصنع بالقمح حتى يُصنع منه خبزًا.

\chapter{67}

\par \textit{"ثم بدأ الشيطان في إضلال آدم وحواء..."}

\par 1 ولكن عندما نزل آدم وحواء إلى أرض الطين الأسود، واقتربا من القمح الذي أراهما الله إياه، ورأياه ناضجًا وجاهزًا للحصاد، حيث لم يكن لديهما منجل لحصاده، فقد حزما أنفسهما وبدءا في سحب القمح حتى انتهى كل شيء.

\par 2 ثم جمعوها إلى كومة، وإذ أضعفهم الحر والعطش، ذهبوا تحت شجرة ظليلة، حيث هبت عليهم نسمة الهواء ليناموا

\par 3 فرأى الشيطان ما فعله آدم وحواء. فدعا جنوده وقال لهم: "بما أن الله قد أظهر لآدم وحواء كل شيء عن هذه القمحة التي يقويان بها جسديهما - وها هما قد أتيا وصنعا منها كومة، وقد تعبا من التعب وهما الآن نائمان - تعالوا فلنشعل نارًا في هذه الكومة من القمح ونحرقها، ولنأخذ قربة الماء التي بجانبهما ونفرغها، فلا يجدان ما يشربانه، فنقتلهما جوعًا وعطشًا."

\par 4 «ثم إذا استيقظوا من نومهم، وحاولوا العودة إلى الكهف، سنأتيهم في الطريق، وسنضللهم؛ فيموتوا من الجوع والعطش؛ وربما ينكرون الله، فيُهلكهم. فسنتخلص منهم.»

\par 5 ثم ألقى الشيطان وجنوده نارًا على القمح فأكلته.

\par 6 ولكن من حرارة اللهيب استيقظ آدم وحواء من نومهما، فرأيا القمح يحترق، ودلو الماء الذي بجانبهما يسكب.

\par 7 ثم بكوا وعادوا إلى الكهف.

\par 8 ولكن بينما كانوا صاعدين من أسفل الجبل الذي كانوا فيه، استقبلهم الشيطان وجنوده في شكل ملائكة، وهم يسبحون الله.

\par 9 ثم قال الشيطان لآدم: "يا آدم، لماذا أنت متألم من الجوع والعطش؟ يبدو لي أن الشيطان قد أحرق القمح." فقال له آدم: "أجل."

\par 10 قال الشيطان لآدم مرة أخرى: "ارجع معنا؛ نحن ملائكة الله. أرسلنا الله إليك لنريك حقل قمح آخر، أفضل من ذلك؛ وخلفه ينبوع ماء جيد، وأشجار كثيرة، حيث ستسكن بالقرب منه، وتعمل في حقل القمح لغرض أفضل من ذلك الذي استهلكه الشيطان."

\par 11 ظن آدم أنه صادق، وأنهم ملائكة تحدثوا معه، فعاد معهم

ثم بدأ الشيطان يُضل آدم وحواء ثمانية أيام، حتى سقطا كالميتين من الجوع والعطش والضعف. ثم هرب مع جيشه وتركهم.

\chapter{68}

\par \textit{كم هو دمار ومتاعب الشيطان عندما يكون السيد. أسس آدم وحواء عادة العبادة.}

\par 1 ثم نظر الله إلى آدم وحواء، وما جاء عليهما من الشيطان، وكيف أهلكهما

\par 2 فأرسل الله كلمته، وأقام آدم وحواء من حالة الموت.

\par 3 ثم قال آدم لما رُفع: يا رب، لقد أحرقت وسلبت منا القمح الذي وهبتنا إياه، وأفرغت دلو الماء. وأرسلت ملائكتك الذين ضلوا طريقنا في الحقل. أتريد أن تهلكنا؟ إن كان هذا منك يا رب، فاقتلع أرواحنا، ولا تعذبنا.

\par 4 ثم قال الله لآدم: "لم أحرق القمح، ولم أسكب الماء من الدلو، ولم أرسل ملائكتي لإضلالك."

\par 5 «لكن الشيطان، سيدك، هو من فعل ذلك؛ هو الذي أخضعت نفسك له؛ وقد أُلغيت وصيتي في هذه الأثناء. هو الذي أحرق القمح، وسكب الماء، وأضلك؛ وكل الوعود التي قطعها لك، ليست في الحقيقة سوى خداع وخداع وكذب.»

\par 6 «ولكن الآن يا آدم، عليك أن تعترف بأعمالي الصالحة التي فعلتها لك.»

\par 7 وأمر الله ملائكته أن يأخذوا آدم وحواء ويحملوهما إلى حقل القمح الذي وجدوه كما في السابق مع الدلو المملوء ماء.

\par 8 هناك رأوا شجرة، فوجدوا عليها منًّا صلبًا، فتعجبوا من قدرة الله. وأمرهم الملائكة أن يأكلوا من المن عندما يجوعون

\par 9 وأقسم الله على الشيطان بلعنة ألا يعود مرة أخرى، فيُهلك حقل القمح

\par 10 ثم أخذ آدم وحواء من القمح، وجعلا منه تقدمة، ورفعاها على الجبل، المكان الذي قدما فيه تقدمتهما الأولى من الدم

\par 11 وقدموا هذه التقدمة مرة أخرى على المذبح الذي بنوه أولًا. وقاموا وصلوا وتضرعوا إلى الرب قائلين: "هكذا يا الله، عندما كنا في الجنة، صعدت تسابيحنا إليك مثل هذه التقدمة؛ وصعدت براءتنا إليك مثل البخور. والآن يا الله، اقبل منا هذه التقدمة، ولا تردنا محرومين من رحمتك."

\par 12 ثم قال الله لآدم وحواء: "بما أنكما قدمتما هذه القربانة وقدمتها لي، فسأجعلها جسدي، عندما أنزل إلى الأرض لإنقاذكما؛ وسأجعلها تُقدم باستمرار على مذبح، من أجل المغفرة والرحمة، لمن يتناول منها كما ينبغي."

\par 13 وأرسل الله نارًا مُشرقة على ذبيحة آدم وحواء، فملأها سطوعًا ونعمة ونورًا؛ ونزل الروح القدس على تلك الذبيحة

\par 14 ثم أمر الله ملاكًا أن يأخذ ملقطًا ناريًا، مثل الملعقة، وأن يأخذ معه قربانًا ويقدمه لآدم وحواء. ففعل الملاك ذلك كما أمره الله، وقدمه لهما

\par 15 وأشرقت نفوس آدم وحواء، وامتلأت قلوبهما بالفرح والسرور وتسبيح الله

\par 16 وقال الله لآدم: «هكذا تكون عادتك أن تفعل، حين يصيبك الضيق والحزن. أما نجاتك ودخولك الجنة، فلن يكونا إلا بعد انقضاء الأيام التي اتفقت عليها بيني وبينك. ولولا ذلك، لرحمتك وشفقتي عليك، لأعدتك إلى جنتي ورضا عني، من أجل قربانك الذي قدمته لاسمي».

\par 17 فرح آدم بهذه الكلمات التي سمعها من الله، وسجد هو وحواء أمام المذبح، وانحنوا له، ثم عادا إلى كهف الكنوز

\par 18 وحدث هذا في نهاية اليوم الثاني عشر بعد اليوم الثمانين، من خروج آدم وحواء من الجنة

\par 19 وقاموا الليل كله يصلون حتى الصباح، ثم خرجوا من الكهف

\par 20 ثم قال آدم لحواء، بفرح قلب، بسبب التقدمة التي قدموها لله، والتي قبلها: "لنفعل هذا ثلاث مرات في كل أسبوع: في اليوم الرابع الأربعاء، وفي يوم الاستعداد الجمعة، وفي يوم السبت الأحد، كل أيام حياتنا."

\par 21 ولما اتفقوا على هذه الكلمات فيما بينهم، سُرَّ الله بأفكارهم وبالقرار الذي اتخذه كلٌّ منهم مع الآخر

\par 22 بعد ذلك، جاء كلمة الله إلى آدم، وقال: "يا آدم، لقد حددت مسبقًا الأيام التي ستأتي فيها الآلام عليّ، عندما أصير جسدًا؛ لأنها الأربعاء الرابع، ويوم الاستعداد الجمعة."

\par 23 «أما في اليوم الأول، ففيه خلقتُ كل شيء، ورفعتُ السماوات. وأيضًا، من خلال قيامتي مرة أخرى في هذا اليوم، سأخلق الفرح، وأرفع المؤمنين بي إلى الأعالي؛ يا آدم، قدم هذه القربانة، كل أيام حياتك.»

\par 24 ثم سحب الله كلمته من آدم.

\par 25 لكن آدم استمر في تقديم هذه التقدمة هكذا، كل أسبوع ثلاث مرات، حتى نهاية الأسابيع السبعة. وفي اليوم الأول، وهو اليوم الخمسون، قدم آدم تقدمة كعادته، فأخذها هو وحواء وجاءا إلى المذبح أمام الله، كما علمهما



\chapter{69}

\par \textit{الظهور الثاني عشر للشيطان لآدم وحواء، بينما كان آدم يصلي على القربان على المذبح؛ عندما ضربه الشيطان.}

\par 1 ثم أسرع الشيطان، كاره كل خير، حاسدًا آدم وقربانه الذي نال به نعمة عند الله، وأخذ حجرًا حادًا من بين حجارة الحديد الحادة، وظهر في صورة رجل، وذهب ووقف بجانب آدم وحواء

\par 2 كان آدم حينها يقدم القربان على المذبح، وبدأ بالصلاة، ويداه ممدودتان إلى الله

\par 3 فأسرع الشيطان بحجر الحديد الحاد الذي كان معه، وطعن به آدم في جنبه الأيمن، فسال دم وماء، فسقط آدم على المذبح كالجثة. وهرب الشيطان

\par 4 ثم جاءت حواء، وأخذت آدم ووضعته تحت المذبح. وبقيت هناك تبكي عليه، بينما كان سيل من الدم يسيل من جنب آدم على قربانه

\par 5 لكن الله نظر إلى موت آدم، فأرسل كلمته، وأقامه، وقال له: «أوفِ بتقدمتك يا آدم، فإنها عظيمة، ولا عيب فيها».

\par 6 قال الله لآدم أيضًا: "هكذا سيحدث لي أيضًا على الأرض، عندما أُطعن، ويسيل دم وماء من جنبي ويسيل على جسدي، وهي الذبيحة الحقيقية، والتي تُقدم على المذبح كذبيحة كاملة."

\par 7 ثم أمر الله آدم بإكمال تقدمته، وعندما انتهى منها سجد أمام الله، وحمده على الآيات التي أراها له

\par 8 وشفى الله آدم في يوم واحد، وهو نهاية الأسابيع السبعة، وهو اليوم الخمسون

\par 9 ثم رجع آدم وحواء من الجبل، ودخلا مغارة الكنوز، كما اعتادا أن يفعلا. واكتمل بذلك لآدم وحواء مئة وأربعين يومًا منذ خروجهما من الجنة

\par 10 ثم قاما كلاهما في تلك الليلة وصليا إلى الله. ولما كان الصباح، خرجا ونزلا غربي الكهف، إلى المكان الذي كانت فيه قمحهما، وهناك استراحا تحت ظل شجرة، كما كانا معتادين

\par 11 ولكن عندما أحاط بهم جمهور من الوحوش، كان ذلك من فعل الشيطان، في شره، لشن حرب على آدم من خلال الزواج

\chapter{70}

\par \textit{الظهور الثالث عشر للشيطان لآدم وحواء، لشن حرب ضده، من خلال زواجه من حواء.}

\par 1 بعد هذا، اتخذ الشيطان، كاره كل خير، شكل ملاك، ومعه اثنان آخران، بحيث بدوا مثل الملائكة الثلاثة الذين أحضروا لآدم الذهب والبخور والمر

\par 2 مروا أمام آدم وحواء وهما تحت الشجرة، وحيّوا آدم وحواء بكلمات طيبة مليئة بالمكر

\par 3 ولكن عندما رأى آدم وحواء مظهرهما الجميل، وسمعا كلامهما العذب، قام آدم، ورحب بهما، وأحضرهما إلى حواء، وبقوا جميعًا معًا؛ وكان قلب آدم في تلك الأثناء سعيدًا لأنه فكر فيهما، وأنهما نفس الملائكة الذين أحضروا له الذهب والبخور والمر

\par 4 لأنهم عندما أتوا إلى آدم في المرة الأولى، نزل عليه منهم السلام والفرح، من خلال إحضارهم له علامات جيدة؛ لذلك ظن آدم أنهم جاءوا مرة ثانية ليعطوه علامات أخرى ليفرح بها. لأنه لم يكن يعلم أنه الشيطان؛ لذلك استقبلهم بفرح ورافقهم

\par 5 ثم قال الشيطان، أطولهم: "افرح يا آدم وابتهج. هوذا الله قد أرسلنا إليك لنخبرك بشيء."

\par 6 فقال آدم: "ما هو؟" فأجاب الشيطان: "إنه أمر هين، ولكنه كلام الله، فهل تسمعه منا وتعمل به؟ وإن لم تسمع، فسنرجع إلى الله ونخبره أنك لن تقبل كلامه."

\par 7 فقال الشيطان لآدم أيضًا: لا تخف ولا يأتك الرعد، أما تعرفنا؟

\par 8 فقال آدم: لا أعرفك.

\par 9 فقال له الشيطان: أنا الملاك الذي أحضر لك الذهب وأدخلك إلى المغارة، وهذا الآخر هو الذي أحضر لك البخور، والثالث هو الذي أحضر لك المر حين كنت على قمة الجبل وأدخلك إلى المغارة.

\par 10 «أما بالنسبة للملائكة الآخرين رفقائنا، الذين حملوكم إلى الكهف، فلم يرسلهم الله معنا هذه المرة؛ لأنه قال لنا: تكفيكم».

\par 11 فلما سمع آدم هذه الكلمات آمن بها، وقال لهؤلاء الملائكة: "تكلموا بكلمة الله حتى أقبلها."

\par 12 فقال له الشيطان: "أقسم، وعدني أنك ستنالها."

\par 13 ثم قال آدم: "لا أعرف أن أقسم وأعد."

\par 14 فقال له الشيطان: هات يدك وضعها في يدي.

\par 15 ثم مدّ آدم يده ووضعها في يد الشيطان، فقال له الشيطان: "قل الآن - بما أن الله حيّ عاقل ناطق، حقّ، هو الذي رفع السماوات في الفضاء، وأسّس الأرض على المياه، وخلقني من العناصر الأربعة، ومن تراب الأرض - لن أخلف وعدي، ولن أنكر كلمتي."

\par 16 فأقسم آدم هكذا.

\par 17 فقال له الشيطان: "هوذا قد مضى زمن منذ أن خرجت من الجنة، وأنت لا تعرف الشر ولا الشر. ولكن الآن يقول لك الله: خذ حواء التي خرجت من جنبك، وتزوجها، فتلد لك أولادًا، لتعزيك، وتطرد عنك الضيق والحزن. الآن هذا الأمر ليس صعبًا، ولا فيه عيب عليك."

\chapter{71}

\par \textit{آدم منزعج من زواجه من حواء.}

\par 1 "ولكن عندما سمع آدم هذه الكلمات من الشيطان، حزن جدًا من أجل قسمه ووعده، وقال: ""أزني بلحمي وعظامي وأخطئ إلى نفسي حتى يهلكني الله ويمحوني عن وجه الأرض؟""

\par 2 «منذ أن أكلت من الشجرة في البداية، أخرجني من الجنة إلى هذه الأرض الغريبة، وحرمني من طبيعتي المشرقة، وجلب عليّ الموت. إذا فعلت هذا، فسوف يقطع حياتي عن الأرض، ويلقي بي في الجحيم، وسيعذبني هناك طويلًا.»

\par 3 «لكن الله لم يتكلم قط بالكلام الذي قلته لي؛ وأنتم لستم ملائكة الله، ولا مُرسلين منه. بل أنتم شياطين، تعالوا إليّ في هيئة ملائكة كاذبة. ابتعدوا عني؛ يا ملعونين من الله!»

\par 4 ثم هربت تلك الشياطين من أمام آدم. وقام هو وحواء، وعادا إلى مغارة الكنوز، ودخلاها

\par 5 ثم قال آدم لحواء: "إذا رأيتِ ما فعلتُ فلا تخبري به؛ لأني أخطأتُ إلى الله عندما أقسمتُ باسمه العظيم، ووضعتُ يدي مرةً أخرى في يد الشيطان." ثم التزمت حواء الصمت، كما أخبرها آدم

\par 6 فقام آدم، وبسط يديه إلى الله، متضرعًا إليه بدموع، أن يغفر له ما فعله. وظل آدم قائمًا يصلي أربعين يومًا وأربعين ليلة، لا يأكل ولا يشرب حتى سقط على الأرض من الجوع والعطش.

\par 7 ثم أرسل الله كلمته إلى آدم، فأقامه من حيث كان يرقد، وقال له: "يا آدم، لماذا أقسمت باسمي، ولماذا اتفقت مع الشيطان مرة أخرى؟"

\par 8 فبكى آدم وقال: "يا رب، اغفر لي، فقد فعلت هذا دون قصد، معتقدًا أنهم ملائكة الله."

\par 9 فغفر الله لآدم، وقال له: «احذر الشيطان».

\par 10 وسحب كلمته من آدم.

\par 11 ثم تعزى قلب آدم، فأخذ حواء، وخرجا من الكهف ليصنعا طعامًا لجسديهما

\par 12 ولكن منذ ذلك اليوم، كان آدم يتصارع في ذهنه بشأن زواجه حواء؛ إذ كان خائفًا من فعل ذلك خشية أن يغضب الله عليه

\par 13 ثم ذهب آدم وحواء إلى نهر الماء، وجلسا على ضفته، كما يفعل الناس عندما يستمتعون

\par 14 لكن الشيطان حسدهم وأراد أن يُهلكهم.

\chapter{72}

\par \textit{قلب آدم مشتعل بالنار.}

\par 1 ثم قام الشيطان، وعشرة من جنوده، بتحويل أنفسهم إلى عذارى، على عكس أي شخص آخر في العالم كله من أجل النعمة

\par 2 صعدوا من النهر أمام آدم وحواء، وقالوا فيما بينهم: "هلموا ننظر إلى وجوه آدم وحواء، اللذين هما من البشر على الأرض. ما أجملهما، وما أغرب شكلهما عن وجوهنا." ثم جاءوا إلى آدم وحواء، وسلموا عليهما، ووقفوا متعجبين منهما

\par 3 نظر آدم وحواء إليهما أيضًا، وتعجبا من جمالهما، وقالا: "هل يوجد إذن، تحتنا، عالم آخر، فيه مخلوقات جميلة مثل هذه؟"

\par 4 فقالت تلك العذارى لآدم وحواء: "نعم، نحن خلق كثير."

\par 5 فقال لهم آدم: "ولكن كيف تتكاثرون؟"

\par 6 فأجابوه: «لنا أزواج تزوجونا، فننجب لهم أولادًا، فيكبرون، ويتزوجون بدورهم، وينجبون أولادًا أيضًا، وهكذا نكبر. وإن لم تصدقنا يا آدم، فسنريك أزواجنا وأولادنا».

\par 7 ثم صاحوا فوق النهر كأنهم ينادون على أزواجهم وأولادهم الذين صعدوا من النهر، رجالاً وأطفالاً، وجاء كل واحد إلى زوجته، وأولاده معه.

\par 8 ولكن عندما رآهم آدم وحواء وقفا صامتين وتعجبا منهم.

\par 9 ثم قالوا لآدم وحواء: «انظروا إلى أزواجنا وأولادنا، تزوجوا حواء كما تزوجنا نحن نساءنا، وستنجبون أولادًا مثلنا». كانت هذه حيلة من الشيطان لخداع آدم.

\par 10 فكّر الشيطان في نفسه أيضًا: «أولًا، أوصى الله آدم بشأن ثمرة الشجرة قائلًا: لا تأكل منها، وإلا فستموت موتًا». فأكل آدم منها، ومع ذلك لم يقتله الله، بل كتب عليه الموت والأوبئة والتجارب إلى يوم خروجه من جسده.

\par 11 «والآن، إذا خدعته ليفعل هذا الأمر، ويتزوج حواء بدون أمر الله، فسيقتله الله حينئذٍ.»

\par 12 لذلك، صنع الشيطان هذا الظهور أمام آدم وحواء؛ لأنه سعى لقتله وإخفائه عن وجه الأرض

\par 13 في هذه الأثناء، اشتعلت نار الخطيئة في آدم، ففكر في ارتكاب الخطيئة. لكنه كبح جماح نفسه، خوفًا من أن يقتله الله إذا اتبع نصيحة الشيطان هذه

\par 14 ثم قام آدم وحواء وصليا إلى الله، بينما نزل الشيطان وجنوده إلى النهر، بحضور آدم وحواء، ليعلما أنهما عائدان إلى أرضيهما

\par 15 ثم عاد آدم وحواء إلى كهف الكنوز، كما كانا معتادين، حوالي وقت المساء

\par 16 فقاما كلاهما وصليا إلى الله في تلك الليلة. وظل آدم قائمًا يصلي، ولكنه لم يكن يعرف كيف يصلي، بسبب أفكار قلبه بشأن ليلة زفافه، واستمر على ذلك حتى الصباح

\par 17 ولما أشرق النور، قال آدم لحواء: "قومي، لننزل إلى أسفل الجبل حيث أحضروا لنا الذهب، ولنسأل الرب في هذا الأمر."

\par 18 فقالت حواء: "ما هذا الأمر يا آدم؟"

\par 19 فأجابها: «لأسأل الرب أن يُخبرني بأمر زواجكِ، فأنا لا أفعل ذلك إلا بأمره، لئلا يُهلكنا، أنتِ وأنا. فقد أشعلت تلك الشياطين قلبي بأفكارٍ عمّا أرونا إياه في ظهوراتهم الآثمة».

\par 20 ثم قالت حواء لآدم: "لماذا نحتاج أن ننزل إلى أسفل الجبل؟ فلنقم ونصلي في كهفنا إلى الله، ليخبرنا إن كانت هذه النصيحة جيدة أم لا."

\par 21 ثم قام آدم في الصلاة وقال: "يا رب، أنت تعلم أننا تجاوزنا عنك، ومنذ اللحظة التي تجاوزنا فيها، حُرمنا من طبيعتنا المشرقة؛ وأصبح جسدنا وحشيًا، يحتاج إلى الطعام والشراب؛ وله شهوات حيوانية."

\par 22 «أوصنا يا الله ألا نستسلم لهم دون أمرك، لئلا تهلكنا. لأنه إن لم تأمرنا، فسوف نُقهر، ونتبع نصيحة الشيطان؛ وستُهلكنا مرة أخرى.»

\par 23 «وإن لم يكن كذلك، فخذ أرواحنا منا؛ ولنتخلص من هذه الشهوة الحيوانية. وإن لم تأمرنا بهذا الأمر، فافصل حواء عني، وافصلني عنها؛ واجعل كل منا بعيدًا عن الآخر.»

\par 24 يا رب، مرة أخرى، عندما تُفرّقنا، تُضلّنا الشياطين بظهوراتها، وتُدمّر قلوبنا، وتُشوّه أفكارنا تجاه بعضنا البعض. وإن لم يكن كلٌّ منا تجاه الآخر، فسيكون ذلك، على أي حال، من خلال ظهورهم لنا. هنا أنهى آدم صلاته.

\chapter{73}

\par \textit{خطوبة آدم وحواء.}

\par 1 ثم نظر الله إلى كلمات آدم ووجدها صادقة، وأنه يستطيع أن ينتظر أمره طويلاً، فيما يتعلق بمشورة الشيطان.

\par 2 ووافق الله آدم على ما فكر فيه بشأن هذا الأمر، وفي الصلاة التي قدمها أمامه؛ وجاءت كلمة الله إلى آدم وقالت له: "يا آدم، لو كنت قد اتخذت هذا الحذر في البداية، قبل أن تخرج من الجنة إلى هذه الأرض!"

\par 3 بعد ذلك، أرسل الله ملاكه الذي أحضر الذهب، والملاك الذي أحضر البخور، والملاك الذي أحضر المر إلى آدم، ليخبروه بشأن عشية زفافه

\par 4 ثم قال أولئك الملائكة لآدم: "خذ الذهب وأعطه لحواء هدية زفاف، وخطبها؛ ثم أعطها بعض البخور والمر هدية؛ وكونوا أنتم وهي جسدًا واحدًا."

\par 5 استمع آدم للملائكة، وأخذ الذهب ووضعه في حضن حواء في ثوبها، وخطبها بيده

\par 6 ثم أمر الملائكة آدم وحواء أن يقوما ويصليا أربعين يومًا وأربعين ليلة؛ وبعد ذلك، أن يدخل آدم على امرأته؛ لأنه حينئذٍ سيكون هذا عملاً طاهرًا لا دنس فيه؛ وأن ينجب أولادًا يتكاثرون ويملؤون وجه الأرض

\par 7 ثم تلقى آدم وحواء كلام الملائكة، فانصرف عنهما الملائكة

\par 8 ثم بدأ آدم وحواء بالصوم والصلاة حتى نهاية الأربعين يومًا؛ ثم اجتمعا كما أخبرتهما الملائكة. ومن وقت خروج آدم من الجنة حتى زواجه من حواء، كان مئتان وثلاثة وعشرون يومًا، أي سبعة أشهر وثلاثة عشر يومًا

\par 9 وهكذا هُزمت حرب الشيطان مع آدم.



\chapter{74}

ولادة قابيل ولولوة. لماذا سُمّيا بهذه الأسماء؟

\par 1 "وسكنوا على الأرض يعملون، من أجل استمرار سلامة أجسادهم؛ وظلوا كذلك حتى انتهت الأشهر التسعة من حمل حواء، واقترب الوقت الذي يجب أن تولد فيه.

\par 2 ثم قالت لآدم: "هذه المغارة بقعة نقية بفضل العلامات التي أُجريت فيها منذ أن غادرنا الجنة، وسنصلي فيها مرة أخرى. لذا، لا يليق بي أن أُخرجها؛ فلنلجأ إلى صخرة الحماية التي رماها الشيطان علينا حين أراد قتلنا بها، ولكنها رُفعت وبسطت كظلة فوقنا بأمر الله، وشكلت مغارة."

\par 3 ثم نقل آدم حواء إلى ذلك الكهف، وعندما حان وقت ولادتها، عانت كثيرًا. فحزن آدم، وتألم قلبه من أجلها، لأنها كانت على وشك الموت، ليتم قول الله لها: "في الشقاء تلد، وفي الحزن تلد".

\par 4 ولكن عندما رأى آدم الضيق الذي كانت فيه حواء، قام وصلى إلى الله، وقال: "يا رب، انظر إليّ بعين رحمتك، وأخرجها من ضيقها."

\par 5 ونظر الله إلى أمته حواء وخلصها، فولدت ابنها البكر، ومعه ابنة

\par 6 ثم فرح آدم بخلاص حواء، وكذلك بالأولاد الذين ولدتهم له. وخدم آدم حواء في الكهف حتى نهاية ثمانية أيام، حين سميا الابن قابيل، والابنة لولوة

\par 7 معنى قابيل هو "الكراهية"، لأنه كان يكره أخته في بطن أمهما؛ قبل أن يخرجا منه. لذلك سماه آدم قابيل

\par 8 لكن لولوة تعني "جميلة"، لأنها كانت أجمل من والدتها

\par 9 ثم انتظر آدم وحواء حتى بلغ قابيل وأخته أربعين يومًا، فقال آدم لحواء: «سنقدم ذبيحة ونقدمها عن الأولاد».

\par 10 فقالت حواء: «سنقدم ذبيحة واحدة عن الابن البكر، وبعد ذلك سنقدم ذبيحة واحدة عن الابنة».

\chapter{75}

\par \textit{تزور العائلة كهف الكنوز مرة أخرى. ولادة هابيل وأكليميا.}

\par 1 ثم أعد آدم تقدمة، فرفعها هو وحواء عن أولادهما، وأتيا بها إلى المذبح الذي بنوه أولاً

\par 2 فقدّم آدم القربان، وطلب من الله أن يقبل قربانه

\par 3 ثم قبل الله تقدمة آدم، وأرسل نورًا من السماء أضاء على التقدمة. فتقدم آدم والابن إلى التقدمة، وأما حواء والابنة فلم تقتربا منها

\par 4 فنزل آدم عن المذبح وفرحوا. وانتظر آدم وحواء حتى صارت ابنتهما ابنة ثمانين يوما. ثم أعد آدم تقدمة وجاء بها إلى حواء والأولاد. فجاءوا إلى المذبح وقدمها آدم كما كان عادة طالبا من الرب أن يقبل تقدمته.

\par 5 فقبل ​​الرب تقدمة آدم وحواء. فتقدم آدم وحواء والأبناء معًا ونزلوا من الجبل فرحين.

\par 6 ولكنهم لم يعودوا إلى الكهف الذي ولدوا فيه، بل جاؤوا إلى كهف الكنوز، لكي يدور الأطفال حوله، ويتباركوا بالرموز التي أحضروها من الحديقة.

\par 7 ولكن بعد أن باركهم الله بهذه الرموز، عادوا إلى الكهف الذي ولدوا فيه.

\par 8 قبل أن تُقدّم حواء ذبيحتها، أخذها آدم وذهب معها إلى نهر الماء الذي ألقيا فيه أولًا، وهناك اغتسلا. غسل آدم جسده، وحواء جسدها أيضًا، بعد المعاناة والضيق اللذين أصاباهما.

\par 9 "ولكن آدم وحواء، بعد أن اغتسلا في نهر الماء، عادا كل ليلة إلى كهف الكنوز، حيث صليا وتباركا؛ ثم عادا إلى كهفهما حيث ولد الأطفال."

\par 10 وهكذا فعل آدم وحواء حتى انتهى الطفلان من الرضاعة. ثم، عندما فُطما، قدم آدم ذبيحة عن أرواح أطفاله؛ بخلاف المرات الثلاث التي قدم فيها ذبيحة عنهم، كل أسبوع

\par 11 ولما انقضت أيام الرضاعة، حملت حواء أيضًا، ولما كملت أيامها ولدت ابنًا وابنة آخرين، فسمتا الابن هابيل والبنت أكلية

\par 12 ثم في نهاية أربعين يومًا، قدم آدم ذبيحة عن الابن، وفي نهاية ثمانين يومًا قدم ذبيحة أخرى عن الابنة، وفعل بهما كما فعل من قبل بقابيل وأخته لولوة

\par 13 أحضرهم إلى كهف الكنوز، حيث نالوا البركة، ثم عادوا إلى الكهف الذي ولدوا فيه. بعد ولادة هؤلاء، توقفت حواء عن الإنجاب

\chapter{76}

\par \textit{يغار قابيل بسبب أخواته.}

\par 1 وبدأ الأطفال ينمون ويكبرون، ولكن قابيل كان قاسي القلب، وسيطر على أخيه الأصغر.

\par 2 وفي كثير من الأحيان، عندما كان والده يقدم قربانًا، كان يبقى ولا يذهب معهم لتقديمه

\par 3 أما هابيل، فكان وديع القلب، مطيعًا لأبيه وأمه، اللذين كان كثيرًا ما يحركهما لتقديم التقدمة، لأنه كان يحبها؛ وكان يصلي ويصوم كثيرًا

\par 4 ثم ظهرت هذه الآية على هابيل. وبينما كان يدخل مغارة الكنوز، ورأى قضبان الذهب والبخور والمر، سأل أبويه آدم وحواء عنها، وقال لهما: "كيف حصلتما على هذه؟"

\par 5 ثم أخبره آدم بكل ما حدث لهما. وتأثر هابيل بشدة بما أخبره به أبوه

\par 6 علاوة على ذلك، أخبره والده آدم عن أعمال الله وعن الجنة؛ وبعد ذلك، بقي خلف والده طوال تلك الليلة في كهف الكنوز

\par 7 وفي تلك الليلة، بينما كان يصلي، ظهر له الشيطان في صورة رجل، وقال له: "لقد حثثت والدك مرارًا على تقديم قربان، وعلى الصوم والصلاة، لذلك سأقتلك، وأهلكك من هذا العالم."

\par 8 أما هابيل، فقد صلى إلى الله، وطرد الشيطان عنه، ولم يُصدّق كلام إبليس. ثم لما كان النهار، ظهر له ملاك الله، وقال له: «لا تُقَصِّر الصوم والصلاة، ولا تُقَدِّم قربانًا لإلهك. لأن الرب قد قبل صلاتك. لا تخف من الشبح الذي ظهر لك في الليل، والذي لعنك حتى الموت». فانصرف عنه الملاك

\par 9 فلما طلع النهار، جاء هابيل إلى آدم وحواء، وأخبرهما بالرؤيا التي رآها. فلما سمعا حزنا عليها حزنًا شديدًا، ولم يخبراه عنها شيئًا، بل عزياه فقط.

\par 10 أما قابيل القاسي القلب، فقد جاءه الشيطان ليلًا، وظهر له، وقال له: "بما أن آدم وحواء يحبان أخاك هابيل أكثر مما يحبانك، ويريدان أن يتزوجاه من أختك الجميلة، لأنهما يحبانه؛ ولكنهما يريدان أن يتزوجاك من أخته سيئة الحظ، لأنهما يكرهانك؛"

\par 11 «الآن، إذن، أنصحك، عندما يفعلون ذلك، أن تقتل أخاك؛ عندها ستُترك لك أختك؛ وستُطرح أخته بعيدًا.»

\par 12 فانصرف عنه الشيطان. لكن الشرير بقي في قلب قابيل، الذي سعى مرارًا لقتل أخيه



\chapter{77}

\par \textit{قابيل، 15 عامًا، وهابيل، 12 عامًا، يكبران بعيدًا عن بعضهما البعض.}

\par 1 ولكن لما رأى آدم أن الأخ الأكبر يكره الأصغر، سعى إلى تليين قلوبهما، وقال لقابيل: «خذ يا ابني من ثمر زرعك وقدم ذبيحة لله، لكي يغفر لك شرك وخطيئتك».

\par 2 وقال أيضًا لهابيل: «خذ من زرعك وقدّم تقدمة وقدمها لله، فيغفر لك ذنبك وخطيئتك».

\par 3 فسمع هابيل لقول أبيه، وأخذ من زرعه، وقدم تقدمة جيدة، وقال لأبيه آدم: «تعال معي لتريني كيف أقدمها».

\par 4 فذهب آدم وحواء معه، وأراهما كيف يقدم قربانه على المذبح. ثم بعد ذلك، قاما وصليا أن يقبل الله قربان هابيل

\par 5 ثم نظر الله إلى هابيل فقبل قربانه. ورضي الله بهابيل أكثر من قربانه، وذلك لطيب قلبه ونقاء جسده. لم يكن فيه أثر غش

\par 6 ثم نزلوا من المذبح، وذهبوا إلى الكهف الذي كانوا يسكنون فيه. أما هابيل، فمن شدة فرحه بتقديم تقدمته، كان يكررها ثلاث مرات في الأسبوع، اقتداءً بأبيه آدم

\par 7 أما قابيل، فلم يُسرّ بالتقدمة، بل بعد غضب شديد من أبيه، قدّم قربانه مرة واحدة، وعندما قدّم، كانت عينه على التقدمة التي قدّمها، فأخذ أصغر غنمه تقدمة، وكانت عينه عليها أيضًا

\par 8 لذلك لم يقبل الله تقدمته، لأن قلبه كان مليئًا بأفكار قاتلة

\par 9 وهكذا عاشوا جميعًا معًا في الكهف الذي ولدت فيه حواء، حتى بلغ قابيل خمس عشرة سنة، وهابيل اثنتي عشرة سنة

\chapter{78}

\par \textit{تتغلب الغيرة على قابيل. يُسبب مشاكل في العائلة. كيف تم التخطيط لأول جريمة قتل.}

\par 1 ثم قال آدم لحواء: هوذا الأبناء قد كبروا، فلنفكر في إيجاد زوجات لهم.

\par 2 فأجابت حواء: "كيف يمكننا أن نفعل ذلك؟"

\par 3 فقال لها آدم: نزوج أخت هابيل إلى قابيل، وأخت قابيل إلى هابيل.

\par 4 فقالت حواء لآدم: «أنا لا أحب قابيل لأنه قاسٍ القلب، ولكن دعهم ينتظرون حتى نذبح للرب عنهم».

\par 5 ولم يزد آدم على ذلك.

\par 6 في هذه الأثناء، جاء الشيطان إلى قابيل في صورة رجل من الحقل، وقال له: «هوذا آدم وحواء قد تشاورا بشأن زواجكما، واتفقا على أن يزوجاك أخت هابيل، وأختك له».

\par 7 «لكن لو لم أكن أحبك، لما أخبرتك بهذا الأمر. ولكن إذا أخذتِ بنصيحتي وأصغيتِ إليّ، فسأحضر لكِ في يوم زفافكِ ثيابًا جميلة، من الذهب والفضة بكثرة، وسيخدمكِ أقاربي.»

\par 8 فقال قابيل بفرح: «أين أقاربك؟»

\par 9 فأجاب الشيطان: «إن أقاربي في جنة في الشمال، حيث كنت أقصد أن أحضر أباك آدم إليها، ولكنه لم يقبل عرضي».

\par 10 «لكنك إن قبلت كلامي وأتيت إليّ بعد زفافك، فسترتاح من البؤس الذي أنت فيه؛ وسترتاح وستكون أفضل حالًا من أبيك آدم.»

\par 11 عند هذه الكلمات من الشيطان، فتح قابيل أذنيه، وانحنى نحو كلامه

\par 12 ولم يبقَ في الحقل، بل ذهب إلى حواء أمه، وضربها، ولعنها، وقال لها: "لماذا أنتم مُزمعون على أن تأخذوا أختي لتُزوجوها لأخي؟ هل أنا ميت؟"

\par 13 لكن أمه هدأته وأرسلته إلى الحقل الذي كان فيه

\par 14 فلما جاء آدم، أخبرته بما فعل قابيل.

\par 15 ولكن آدم حزن وسكت ولم يتكلم بكلمة.

\par 16 ثم في الغد قال آدم لقابيل ابنه: «خذ من غنمك صغارًا وحسنات وقدمها لإلهك، وأنا أُكلم أخاك أن يُقدم لإلهه تقدمة قمح».

\par 17 فسمعا كلاهما لأبيهما آدم، وأخذا قرابينهما، وأصعداها على الجبل عند المذبح

\par 18 لكن قابيل تكبر على أخيه، ودفعه عن المذبح، ولم يدعه يقدم قربانه عليه، بل قدم قربانه عليه بقلب متكبر، مملوء مكرًا وغشًا

\par 19 أما هابيل، فقد نصب حجارة قريبة، وعلى ذلك قدم قربانه بقلب متواضع وخالٍ من الغش

\par 20 كان قابيل واقفًا عند المذبح الذي قدم عليه قربانه، فصرخ إلى الله أن يقبل قربانه، لكن الله لم يقبله منه، ولم تنزل نار إلهية لتأكل قربانه

\par 21 لكنه بقي واقفًا قبالة المذبح، من فرط السخرية والغضب، ناظرًا إلى أخيه هابيل، ليرى هل يقبل الله تقدمته أم لا

\par 22 وصلى هابيل إلى الله أن يقبل قربانه. فنزلت نار إلهية وأكلت قربانه. فتشمم الله رائحة قربانه الطيبة، لأن هابيل أحبه وفرح به.

\par 23 ولأن الله سُرَّ به، أرسل إليه ملاك نور في صورة إنسان تناول من قربانه، لأنه اشتم رائحة قربانه الطيبة، فعزوا هابيل وشددوا قلبه

\par 24 وكان قابيل ينظر إلى كل ما حدث عند قربان أخيه، فغضب لأجله

\par 25 ثم فتح فمه وجدف على الله، لأنه لم يقبل قربانه

\par 26 فقال الله لقابيل: «لماذا وجهك مكتئب؟ كن بارًا حتى أقبل قربانك. لم تتذمر عليّ، بل على نفسك.»

\par 27 فقال الله هذا لقابيل توبيخًا، ولأنه كرهه هو وقربانه

\par 28 فنزل قابيل من على المذبح، وقد تغير لونه، وكئيب المنظر، وجاء إلى أبيه وأمه، وأخبرهما بكل ما أصابه. فحزن آدم حزنًا شديدًا لأن الله لم يقبل قربان قابيل

\par 29 فنزل هابيل فرحًا وقلبًا طيبًا، وأخبر أباه وأمه كيف قبل الله تقدمته. ففرحا بذلك وقبلا وجهه

\par 30 فقال هابيل لأبيه: «لأن قابيل دفعني عن المذبح ولم يدعني أقدم قرباني عليه، صنعت لنفسي مذبحًا وقدمت قرباني عليه».

\par 31 ولكن عندما سمع آدم ذلك، حزن بشدة، لأنه كان المذبح الذي بناه أولاً، والذي قدم عليه عطاياه الخاصة

\par 32 أما قابيل، فقد كان متجهمًا وغاضبًا جدًا لدرجة أنه ذهب إلى الحقل، حيث جاء إليه الشيطان وقال له: "منذ أن لجأ أخوك هابيل إلى أبيك آدم، لأنك دفعته عن المذبح، قبلوا وجهه، وفرحوا به أكثر مما فرحوا بك."

\par 33 عندما سمع قابيل كلمات الشيطان هذه، امتلأ غضبًا، ولم يُعلم أحدًا. لكنه كان يُكمن ليقتل أخاه، حتى أدخله إلى الكهف، ثم قال له:

\par 34 يا أخي، البلد جميل جدًا، وفيه أشجار جميلة وممتعة، وساحرة للنظر! لكن يا أخي، لم تذهب يومًا واحدًا إلى الحقل لتستمتع به

\par 35 «اليوم يا أخي، أتمنى بشدة أن تأتي معي إلى الحقل، لتستمتع بوقتك وتبارك حقولنا وقطعاننا، لأنك بار، وأنا أحبك كثيرًا يا أخي! لكنك ابتعدت عني.»

\par 36 فوافق هابيل على الذهاب مع أخيه قابيل إلى الحقل.

\par 37 ولكن قبل أن يخرج قال قابيل لهابيل: انتظرني حتى آتي بعصا بسبب الوحوش.

\par 38 ثم وقف هابيل منتظرًا في براءته. أما قابيل، المهاجم، فأخذ عصا وخرج

\par 39 فابتدأ قابيل وهابيل أخوه يمشيان في الطريق وكان قابيل يكلم هابيل ويعزيه حتى أنساه كل شيء.

\chapter{79}

\par \textit{خطة شريرة تنتهي بنهاية مأساوية. قابيل خائف. "هل أنا حارس لأخي؟" العقوبات السبعة. السلام محطم.}

\par 1 وهكذا ساروا حتى وصلوا إلى مكانٍ قفرٍ لا خراف فيه. فقال هابيل لقابيل: "انظر يا أخي، لقد تعبنا من المشي؛ لأننا لا نرى شيئًا من الأشجار، ولا من الثمار، ولا من الخضرة، ولا من الخراف، ولا شيئًا مما أخبرتني عنه. أين خرافك التي طلبت مني أن أباركها؟"

\par 2 فقال له قابيل: «هلم، سترى الآن أشياءً جميلة كثيرة، ولكن قدِّم أمامي حتى أصعد إليك».

\par 3 ثم تقدم هابيل، وبقي قابيل خلفه.

\par 4 وكان هابيل يمشي في طهارته بلا غش، ولم يكن يعتقد أن أخاه سيقتله.

\par 5 فلما اقترب إليه قابيل، عزاه بكلامه، وسار خلفه قليلاً، ثم أسرع وضربه بالعصا ضربة تلو الأخرى حتى صعق،

\par 6 فلما سقط هابيل على الأرض، إذ رأى أن أخاه ينوي قتله، قال لقابيل: "يا أخي، ارحمني. بالثديين اللذين رضعناهما لا تضربني! وبالبطن الذي حملنا وأتى بنا إلى العالم، لا تضربني حتى الموت بتلك العصا! إن كنت تريد أن تقتلني، فخذ أحد هذه الحجارة الكبيرة واقتلني فورًا."

\par 7 ثم أخذ قابيل، القاتل القاسي القلب، حجرًا كبيرًا، وضرب به أخاه على رأسه، حتى سال دماغه، واختنق في دمه أمامه

\par 8 ولم يندم قابيل على ما فعل.

\par 9 ولكن الأرض، عندما وقع عليها دم هابيل الصديق، ارتجفت لأنها شربت دمه، وكانت ستجعل قابيل لا شيء بسبب ذلك.

\par 10 وصرخ دم هابيل في سرٍّ إلى الله، لينتقم له من قاتله

\par 11 فبدأ قابيل للوقت يحفر الأرض ليضع فيها أخاه، لأنه كان يرتعد من الخوف الذي أصابه لما رأى الأرض تهتز بسببه

\par 12 ثم ألقى أخاه في الحفرة التي حفرها، وغطّاه بالتراب. لكن الأرض لم تقبله، بل قذفته في الحال

\par 13 حفر قابيل الأرض مرة أخرى وأخفى أخاه فيها، ولكن الأرض قذفته على نفسها مرة أخرى، حتى قذفت الأرض جسد هابيل على نفسها ثلاث مرات

\par 14 قذفته الأرض الوحلة في المرة الأولى، لأنه لم يكن الخليقة الأولى؛ وقذفته في المرة الثانية ولم تقبله، لأنه كان بارًا وصالحًا، وقد قُتل بلا سبب؛ وقذفته الأرض في المرة الثالثة ولم تقبله، لكي يبقى أمام أخيه شاهدًا عليه

\par 15 وهكذا سخرت الأرض من قابيل، حتى أتته كلمة الله بشأن أخيه

\par 16 فغضب الله واستاء كثيرًا لموت هابيل، فارعد من السماء، ومرت البروق أمامه، وجاءت كلمة الرب الإله من السماء إلى قابيل وقالت له: «أين هابيل أخوك؟»

\par 17 فأجاب قابيل بقلبٍ مُتكبرٍ وصوتٍ أجشٍّ: «يا الله، كيف أكون حارسًا لأخي؟»

\par 18 ثم قال الله لقابيل: «ملعونة الأرض التي شربت دم هابيل أخيك، وأنت ترتعد وترتجف، وهذه لك علامة أن كل من وجدك يقتلك».

\par 19 لكن قابيل بكى لأن الله قال له تلك الكلمات، وقال له قابيل: "يا الله، كل من وجدني يقتلني، فأُمحى عن وجه الأرض".

\par 20 ثم قال الله لقابيل: «كل من وجدك لا يقتلك»، لأنه قبل ذلك كان الله يقول لقابيل: «سأتنازل عن سبع عقوبات على من قتل قابيل». أما بالنسبة لكلمة الله لقابيل: «أين أخوك؟» فقد قالها الله رحمةً له، ليحاول أن يجعله يتوب

\par 21 لأنه لو تاب قابيل في ذلك الوقت، وقال: "يا رب، اغفر لي خطيئتي وقتل أخي"، لكان الله قد غفر له خطيئته

\par 22 وأما قول الله لقابيل: "ملعونة الأرض التي شربت دم أخيك"، فكان ذلك أيضًا رحمة من الله على قابيل. لأن الله لم يلعنه، بل لعن الأرض؛ مع أنها لم تكن الأرض التي قتلت هابيل وارتكبت الإثم

\par 23 لأنه كان من اللائق أن تقع اللعنة على القاتل؛ ولكن في رحمته دبّر الله أفكاره بحيث لا يعلم بها أحد، فيبتعد عن قابيل

\par 24 فقال له: «أين أخوك؟» فأجاب وقال: «لا أعلم». فقال له الخالق: «ارتعد وارتعد».

\par 25 فارتعد قابيل وارتاع. ومن خلال هذه الآية جعله الله عبرة أمام كل الخليقة، كقاتل أخيه. وأنزل الله عليه الرعدة والرعب، ليرى السلام الذي كان فيه في البداية، ويرى أيضًا الرعدة والرعب اللذين تحملهما في النهاية؛ حتى يتواضع أمام الله، ويتوب عن خطيئته، ويسعى إلى السلام الذي تمتع به في البداية

\par 26 وفي كلمة الله التي قالت: "سأتنازل عن سبع عقوبات عن كل من يقتل قابيل"، لم يكن الله يسعى إلى قتل قابيل بالسيف، بل سعى إلى جعله يموت صائمًا، وصلاة، وبكاءً، بقاعدة صارمة، حتى يحين الوقت الذي تحرر فيه من خطيئته

\par 27 والعقوبات السبع هي الأجيال السبعة التي انتظر الله خلالها قابيل لقتله أخيه

\par 28 أما قابيل، فمنذ أن قتل أخاه، لم يجد راحة في أي مكان؛ بل عاد إلى آدم وحواء، مرتجفًا، مرعوبًا، وملطخًا بالدماء...


\end{document}