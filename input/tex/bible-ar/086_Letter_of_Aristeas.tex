\begin{document}

\title{رسالة أريستياس}

\chapter{1}

\par \textit{في زمن سبي اليهود في مصر، كشف بطليموس فيلادلفوس عن نفسه كأول عاشق للكتب. رغب في أن تكون جميع كتب العالم في مكتبته؛ ومن أجل الحصول على شريعة موسى، عرض مقايضة 100,000 أسير مقابل هذا العمل، قائلاً: "إنها نعمة صغيرة حقًا!"}

\par 1 بما أنني جمعتُ موادًا لتاريخٍ لا يُنسى لزيارتي إلى إليعازار رئيس كهنة اليهود، ولأنك يا فيلوكراتس، كما لا تُفوِّت فرصةً لتذكيري، قد أوليتَ أهميةً كبيرةً لتلقي وصفٍ لدوافع مهمتي وهدفها، فقد حاولتُ أن أُعِدَّ لك شرحًا واضحًا للأمر، لأني أُدرك أنك تمتلك حبًا فطريًا للتعلم، وهي صفةٌ تُعَدُّ أسمى ما يمتلكه الإنسان - أن تحاول باستمرار "إضافة المزيد إلى مخزونه من المعرفة والمكتسبات" سواءً من خلال دراسة التاريخ أو من خلال المشاركة الفعلية في الأحداث نفسها

\par 2 بهذه الوسيلة، من خلال استيعابها لأسمى العناصر، تُرسّخ الروح في النقاء، وبعد أن تُثبّت هدفها على التقوى، وهي أسمى الغايات، فإنها تستخدمها كدليل معصوم لها، وبالتالي تكتسب غرضًا محددًا

\par 3 كان إخلاصي في السعي وراء المعرفة الدينية هو ما دفعني إلى تولي مهمة السفارة إلى الرجل الذي ذكرته، والذي كان يحظى بأعلى درجات التقدير من قبل مواطنيه والآخرين، لفضيلته وجلالته، والذي كان يمتلك وثائق ذات قيمة عالية لليهود في بلده وفي البلاد الأجنبية لتفسير الشريعة الإلهية، لأن قوانينهم مكتوبة على رقوق جلدية بأحرف يهودية

\par 4 لقد قمتُ بهذه السفارة بحماس، بعد أن وجدتُ في البداية فرصةً للتوسل إلى الملك نيابةً عن الأسرى اليهود الذين نُقلوا من يهودا إلى مصر على يد والد الملك، عندما استولى على هذه المدينة لأول مرة وغزا أرض مصر

\par 5 من الجدير أن أخبرك بهذه القصة أيضًا، لأنني مقتنع بأنك، بميلك نحو القداسة وتعاطفك مع الرجال الذين يعيشون وفقًا للشريعة المقدسة، ستستمع بسهولة أكبر إلى الرواية التي أعتزم عرضها، لأنك أنت نفسك أتيت إلينا مؤخرًا من الجزيرة وترغب بشدة في سماع كل ما يساهم في بناء الروح

\par 6 في مناسبة سابقة أيضًا، أرسلتُ إليكم سجلًا بالحقائق التي رأيتُ أنها جديرة بالسرد عن العرق اليهودي، وهو السجل الذي حصلتُ عليه من أرقى كهنة مصر وأكثرهم علمًا

\par 7 بما أنكم حريصون جدًا على اكتساب المعرفة بالأشياء التي يمكن أن تفيد العقل، أشعر أنه من واجبي أن أنقل إليكم كل المعلومات التي في وسعي

\par 8 يجب أن أشعر بنفس الواجب تجاه كل من يمتلك نفس التصرف، ولكنني أشعر به بشكل خاص تجاهك لأن لديك تطلعات نبيلة للغاية، ولأنك لست أخي في الشخصية فحسب، ولا تقل عن أخي في الدم، بل أنت واحد معي أيضًا في السعي وراء الخير.

\par 9 فلا المتعة المستمدة من الذهب ولا من أي ممتلكات أخرى تُقدّرها العقول السطحية تمنح نفس الفائدة التي يمنحها السعي وراء الثقافة والدراسة التي نبذلها في سبيل الحصول عليها

\par 10 ولكن لكي لا أزعجكم بمقدمة طويلة جدًا، سأنتقل على الفور إلى جوهر سردتي

\par 11 تلقى ديمتريوس الفاليرمي، رئيس مكتبة الملك، مبلغًا ضخمًا من المال، لغرض جمع كل الكتب في العالم قدر استطاعته

\par 12 من خلال الشراء والنسخ، نفّذ، على أفضل وجه ممكن، هدف الملك

\par 13 في إحدى المرات عندما كنت حاضرًا، سُئل: كم ألف كتاب في المكتبة؟ فأجاب: "أكثر من مئتي ألف، أيها الملك، وسأبذل قصارى جهدي في المستقبل القريب لجمع الباقي أيضًا، حتى يصل العدد الإجمالي إلى خمسمائة ألف. قيل لي إن قوانين اليهود تستحق النسخ وتستحق مكانًا في مكتبتك!"

\par 14 أجاب الملك: «ما الذي يمنعك من فعل هذا؟ كل ما هو ضروري قد وُضع تحت تصرفك!»

\par 15 أجاب ديمتريوس: «يجب ترجمتها، لأنهم في بلاد اليهود يستخدمون أبجدية خاصة (كما أن للمصريين أيضًا شكلًا خاصًا من الحروف) ويتحدثون لهجة خاصة».

\par 16 «من المفترض أن يستخدموا اللغة السريانية، لكن هذا ليس هو الحال؛ لغتهم مختلفة تمامًا.»

\par 17 ولما فهم الملك جميع حقائق القضية، أمر بكتابة رسالة إلى رئيس الكهنة اليهودي حتى يتم تحقيق غرضه (الذي سبق وصفه).

\par 18 ظنًا مني أن الوقت قد حان للضغط من أجل تحقيق المطلب، الذي طرحته مرارًا وتكرارًا على سوسيبيوس التارانتمي وأندرياس، رئيس الحرس الشخصي، لتحرير اليهود الذين نقلهم والد الملك من يهودا - لأنه عندما نجح في هجومه على كامل منطقة سورية الجوفاء وفينيقيا، في عملية إرهاب البلاد وإخضاعها، نقل بعض أعدائه وأسر آخرين

\par 19 بلغ عدد الذين نقلهم من بلاد اليهود إلى مصر ما لا يقل عن مائة ألف

\par 20 من بين هؤلاء، سلّح ثلاثين ألف رجل مختار، وأسكنهم في حاميات في المناطق الريفية

\par 21 (وحتى قبل هذا الوقت، كانت أعداد كبيرة من اليهود قد جاءت إلى مصر مع الفرس، وفي فترة سابقة، تم إرسال آخرين إلى مصر لمساعدة بسامتيك في حملته ضد ملك الأحباش. ولكن هؤلاء لم يكونوا في عدد كبير مثل الأسرى الذين نقلهم بطليموس بن لاجوس.)

\par 22 كما قلتُ سابقًا، اختار بطليموس أفضل هؤلاء، الرجال الذين كانوا في ريعان شبابهم والمتميزين بشجاعتهم، وسلحهم، لكن الغالبية العظمى من الآخرين، أولئك الذين كانوا كبارًا أو صغارًا جدًا لهذا الغرض، والنساء أيضًا، استعبدهم، ليس لأنه أراد أن يفعل ذلك بمحض إرادته، ولكن لأنه أُجبر على ذلك من قبل جنوده الذين طالبوا بهم كمكافأة على الخدمات التي قدموها في الحرب

\par 23 بعد أن حصلنا، كما سبق ذكره، على فرصة لتأمين تحريرهم، خاطبت الملك بالحجج التالية: "دعونا لا نكون غير عقلانيين لدرجة أن نسمح لأفعالنا أن تكذب أقوالنا."

\par 24 بما أن الشريعة التي لا نرغب في نسخها فحسب، بل في ترجمتها أيضًا، تنتمي إلى العرق اليهودي بأكمله، فما هو المبرر الذي سنتمكن من إيجاده لسفارتنا بينما لا تزال أعداد هائلة منهم في حالة عبودية في مملكتك؟

\par 25 «في كمال رحمتك ووفرتها، أطلق سراح أولئك الذين وقعوا في مثل هذه العبودية البائسة، لأنه كما حرصت على اكتشافه، فإن الإله الذي أعطاهم شريعتهم هو الإله الذي يحافظ على مملكتك.»

\par 26 «إنهم يعبدون نفس الإله - رب الكون وخالقه، مثل جميع البشر الآخرين، مثلنا نحن، أيها الملك، على الرغم من أننا نسميه بأسماء مختلفة، مثل زيوس 1 أو ديس.»

\par 27 «لقد أُطلق عليه هذا الاسم بشكل مناسب جدًا من قِبل أسلافنا الأوائل، وذلك للدلالة على أنه، الذي من خلاله تُمنح كل الأشياء الحياة وتأتي إلى الوجود، هو بالضرورة الفارس وسيد الكون.»

\par 28 «اجعلوا البشرية جمعاء قدوة في الكرم بإطلاق سراح أولئك المأسورين.»

\par 29 بعد فترة وجيزة، بينما كنت أدعو الله بإخلاص أن يوجه الملك إلى إطلاق سراح جميع الأسرى - (لأن الجنس البشري، كونه من خلق الله، يتأثر به ويتأثر به

\par 30 لذلك، بصلوات عديدة ومتنوعة، دعوتُ من يحكم القلب أن يُجبر الملك على تلبية طلبي

\par 31 لأنه كانت لدي آمال كبيرة فيما يتعلق بخلاص الرجال، لأنني كنت متأكدًا من أن الله سيحقق صلاتي

\par 32 لأنه عندما يخطط الرجال بدوافع نقية لعمل ما من أجل البر وأداء الأعمال النبيلة، فإن الله القدير يُكلل جهودهم وأغراضهم بالنجاح) - رفع الملك رأسه ونظر إليّ بوجه مرح وسأل: "كم ألفًا تعتقد أنهم سيبلغون؟"

\par 33 أجاب أندرياس، الذي كان يقف بالقرب منه، "أكثر بقليل من مائة ألف."

\par 34 «إنها نعمة صغيرة حقًا»، قال الملك، «ما يطلبه أريستياس منا!»

\par 35 ثم قال سوسيبيوس وبعض الحاضرين: "نعم، ولكن سيكون من دواعي سروركم أن تقدموا تحرير هؤلاء الرجال كعمل من أعمال التفاني للإله الأعظم."

\par 36 لقد كرمك الله القدير تكريمًا عظيمًا، ورفعك فوق جميع أجدادك في المجد، ومن المناسب أن تقدم له أعظم ذبيحة شكر في قدرتك

\par 37 سُرَّ بهذه الحجج للغاية، فأصدر أوامره بإضافة مبلغ الفداء إلى أجور الجنود، ودفع عشرين دراخما للمالكين عن كل عبد، وإصدار أمر عام، وإرفاق سجلات الأسرى به

\par 38 لقد أظهر أكبر قدر من الحماس في العمل، لأن الله هو الذي حقق هدفنا بالكامل وأجبره على فداء ليس فقط أولئك الذين جاءوا إلى مصر مع جيش أبيه، ولكن أي شخص جاء قبل ذلك الوقت أو تم جلبه لاحقًا إلى المملكة

\par 39 أُشير إليه أن مبلغ الفدية سيتجاوز أربعمائة تالنت

\par 40 أعتقد أنه سيكون من المفيد إدراج نسخة من المرسوم، لأنه بهذه الطريقة سيتضح جليًا كرم الملك، الذي فوضه الله لإنقاذ هذه الجموع الغفيرة

\par 41 جاء في مرسوم الملك ما يلي: «كل من خدم في جيش أبينا في الحملة على سوريا وفينيقيا وفي الهجوم على بلاد اليهود، واستولى على أسرى يهود وأعادهم إلى مدينة الإسكندرية وأرض مصر أو باعهم لآخرين - وبنفس الطريقة، يُطلب من كل من استولى على مثل هؤلاء الأسرى أن يطلق سراحهم على الفور، ويحصل على عشرين درهمًا لكل رأس كفدية».

\par 42 «سيحصل الجنود على هذه الأموال كهدية تُضاف إلى أجورهم، أما الباقي فسيحصلون عليه من خزينة الملك.»

\par 43 «نعتقد أن أسرهم كان ضد إرادة أبينا وضد كل حق، وأن تدمير أرضهم ونقل اليهود إلى مصر كان عملاً من أعمال الفوضى العسكرية.»

\par 44 «كانت الغنائم التي سقطت على الجنود في ساحة المعركة هي كل الغنائم التي كان ينبغي لهم المطالبة بها.»

\par 45 «إن استعباد الناس كان أيضًا عملاً من أعمال الظلم المطلق.»

\par 46 "ولذلك، بما أنه من المعترف به أننا معتادون على تحقيق العدالة لجميع البشر وخاصة أولئك الذين هم في حالة عبودية غير عادلة، وحيث أننا نسعى جاهدين للتعامل بشكل عادل مع جميع البشر وفقًا لمتطلبات العدالة والتقوى، فقد قررنا، فيما يتعلق بأشخاص اليهود الذين هم في أي حالة من العبودية في أي جزء من سيطرتنا، أن أولئك الذين يمتلكونهم سوف يتلقون المبلغ المحدد من المال ويطلقون سراحهم وألا يظهر أي رجل أي تأخير في الوفاء بالتزاماته. "

\par 47 «في غضون ثلاثة أيام من نشر هذا المرسوم، يجب عليهم إعداد قوائم بالعبيد للضباط المعينين لتنفيذ إرادتنا، وتقديم أسماء الأسرى على الفور.»

\par 48 «لأننا نرى أنه سيكون من المفيد لنا ولأمورنا أن يُحسم الأمر.»

\par 49 «يُمكن لأي شخص يرغب في ذلك أن يُقدم معلومات عن أي شخص يُخالف المرسوم، بشرط أنه إذا ثبتت إدانة الرجل، فسيُصبح عبدًا له؛ ومع ذلك، تُسلم ممتلكاته إلى الخزانة الملكية.»

\par 50 عندما عُرض المرسوم على الملك ليُقرأ عليه للموافقة عليه، احتوى على جميع الأحكام الأخرى باستثناء عبارة "أي أسرى كانوا في الأرض قبل ذلك الوقت أو جُلبوا إلى هنا بعد ذلك"، وفي كرمه وسعة قلبه، أضاف الملك هذا البند وأصدر أوامر بإيداع منحة المال المطلوبة للفداء بالكامل لدى صرافين القوات والمصرفيين الملكيين، وهكذا حُسم الأمر وتم التصديق على المرسوم في غضون سبعة أيام

\par 51 بلغت منحة الفداء أكثر من ستمائة وستين وزنة؛ لأن العديد من الأطفال الرضع قد تحرروا مع أمهاتهم

\par 52 عندما طُرح السؤال عما إذا كان سيتم دفع مبلغ عشرين وزنة مقابل هذه الأشياء، أمر الملك بذلك، وبالتالي نفذ قراره على أكمل وجه

\par \textit{الحواشي السفلية}
\par \textit{143:1 مقارنة مهمة بين الله وزيوس.}

\chapter{2}

\par \textit{يُظهر كيف تم الاحتفاظ بسجلات دقيقة للغاية لشؤون الدولة. البيروقراطية الحكومية. تم تعيين لجنة من ستة أفراد للذهاب إلى رئيس الكهنة في القدس وترتيب التبادل. تم تكليف أريستياس بقيادة الوفد.}

\par 1 ولما انتهى من ذلك، أمر ديمتريوس بإعداد مذكرة بشأن نسخ الكتب اليهودية

\par 2 لأن جميع شؤون الدولة كانت تُنفذ عن طريق المراسيم وبأقصى درجات الدقة من قبل هؤلاء الملوك المصريين، ولم يكن يتم فعل أي شيء بطريقة عشوائية أو عشوائية

\par 3 ولذلك أدرجت نسخًا من النصب التذكاري والرسائل، وعدد الهدايا المرسلة، وطبيعة كل منها، حيث تفوقت كل واحدة منها في الروعة والمهارة الفنية

\par 4 فيما يلي نسخة من النصب التذكاري. ذكرى ديمتريوس للملك العظيم. «بما أنك أصدرتَ لي أيها الملك تعليماتٍ بجمع الكتب اللازمة لإكمال مكتبتك، وإصلاح ما بها من عيوب، فقد كرّستُ نفسي بكل عناية لتحقيق رغباتك، والآن لديّ الاقتراح التالي لأضعه بين يديك.»

\par 5 «كتب شريعة اليهود (مع بعض الكتب الأخرى) غائبة عن المكتبة.»

\par 6 إنها مكتوبة بالأحرف واللغة العبرية، وقد فُسِّرت بإهمال، ولا تمثل النص الأصلي كما أبلغني أولئك الذين يعرفون؛ لأنها لم تحظَ قط برعاية ملك لحمايتها

\par 7 «من الضروري أن تكون هذه النصوص دقيقة لمكتبتكم، لأن القانون الذي تحتويه، بما أنه من أصل إلهي، مليء بالحكمة وخالٍ من كل عيب.»

\par 8 «لهذا السبب، امتنع الأدباء والشعراء وجمهور الكتاب التاريخيين عن الإشارة إلى هذه الكتب والرجال الذين عاشوا ويعيشون وفقًا لها، لأن مفهومهم عن الحياة مقدس وديني للغاية، كما يقول هيكاتيوس الأبديري.»

\par 9 «إن شئت، أيها الملك، تُكتب رسالة إلى رئيس الكهنة في أورشليم، نطلب منه فيها إرسال ستة شيوخ من كل سبط - رجال عاشوا أشرف حياة وأكثرهم مهارة في شريعتهم - حتى نتمكن من معرفة النقاط التي يتفق عليها معظمهم، وبالتالي، بعد الحصول على ترجمة دقيقة، يمكننا وضعها في مكان بارز بطريقة تليق بالعمل نفسه وغرضك.»

\par 10 «ليكن لك الرخاء الدائم!»

\par 11 عندما قُدِّم هذا النصب التذكاري، أمر الملك بكتابة رسالة إلى ألعازار بهذا الشأن، تتضمن أيضًا وصفًا لتحرير الأسرى اليهود

\par 12 وأعطى خمسين وزنة من الذهب وسبعين وزنة من الفضة وحجارة كريمة كثيرة لصنع أوانٍ وجامات ومائدة وكؤوس سكائب

\par 13 كما أصدر أوامره لأولئك الذين كانوا مسؤولين عن خزائنه بالسماح للحرفيين باختيار أي مواد قد يحتاجونها لهذا الغرض، وأنه يجب إرسال مائة وزنة من المال لتقديم القرابين للمعبد ولتلبية الاحتياجات الأخرى

\par 14 سأقدم لكم وصفًا كاملاً للعمل بعد أن أعرض عليكم نسخًا من الرسائل. وجاء في رسالة الملك ما يلي:

\par 15 «الملك بطليموس يرسل التحية والسلام إلى رئيس الكهنة العازار.»

\par 16 "ولأن هناك العديد من اليهود المقيمين في مملكتنا والذين أخذهم الفرس من القدس في زمن قوتهم والعديد غيرهم الذين جاءوا مع والدي إلى مصر كأسرى - فقد وضع أعدادًا كبيرة منهم في الجيش ودفع لهم أجورًا أعلى من المعتاد، وعندما أثبت ولاء قادتهم بنى الحصون ووضعهم تحت مسؤوليتهم حتى يخاف منهم المصريون الأصليون."

\par 17 «وأنا، عندما اعتليت العرش، اتخذت موقفًا لطيفًا تجاه جميع رعيتي، وبشكل أكثر تحديدًا تجاه أولئك الذين كانوا من مواطنيك - لقد أطلقت سراح أكثر من مائة ألف أسير، ودفعت لأصحابهم السعر السوقي المناسب لهم، وإذا حدث أي شر لشعبك بسبب أهواء الغوغاء، فقد قدمت لهم تعويضات.»

\par 18 «كان الدافع الذي دفعني إلى هذا العمل هو الرغبة في التصرف بتقوى وتقديم ذبيحة شكر لله الأعظم على الحفاظ على مملكتي في سلام ومجد عظيم في جميع أنحاء العالم.»

\par 19 «وعلاوة على ذلك، قمتُ بتجنيد أولئك من شعبك الذين كانوا في ريعان شبابهم في جيشي، وأولئك الذين كانوا لائقين بالارتباط بشخصي ومستحقين لثقة البلاط، وقد ثبتتهم في مناصب رسمية.»

\par 20 «والآن، بما أنني حريص على إظهار امتناني لهؤلاء الرجال ولليهود في جميع أنحاء العالم وللأجيال القادمة، فقد قررت أن يُترجم شريعتكم من اللغة العبرية المستخدمة بينكم إلى اللغة اليونانية، حتى تُضاف هذه الكتب إلى الكتب الملكية الأخرى في مكتبتي.»

\par 21 «سيكون من لطفك ومكافأةً على حماسي أن تختار ستة شيوخ من كل سبط من أسباطك، رجالًا ذوي حياة نبيلة وخبراء في شريعتك وقادرين على تفسيرها، حتى نتمكن في مسائل الخلاف من اكتشاف الحكم الذي تتفق عليه الأغلبية، لأن التحقيق له أهمية قصوى.»

\par 22 آمل أن أحظى بشهرة كبيرة بإنجاز هذا العمل.

\par 23 "لقد أرسلت أندرياس، رئيس حرسى، وأريستياس - وهما رجلان أكن لهما تقديرًا كبيرًا - لعرض الأمر عليك وتقديم مائة وزنة من الفضة، وهي باكورة تقدمتى للمعبد والتضحيات والطقوس الدينية الأخرى."

\par 24 «إذا كتبتَ لي بشأن رغباتك في هذه الأمور، فستُقدِّم لي معروفًا عظيمًا وتمنحني عهدًا جديدًا بالصداقة، حيث سيتم تنفيذ جميع رغباتك في أسرع وقت ممكن. وداعًا!»

\par 25 ردَّ أليعازار على هذه الرسالة بما يلي: «يرسل أليعازار رئيس الكهنة تحياته إلى الملك بطليموس صديقه الحقيقي».

\par 26 أطيب تمنياتي لك ولرفاهيتك ولرفاهيتك الملكة أرسينوي، أختك، وأطفالك

\par 27 أنا أيضًا بخير. لقد تلقيت رسالتك، وأنا سعيد جدًا بهدفك ومشورتك النبيلة

\par 28 "لقد جمعت كل الشعب وقرأت عليهم الكتاب حتى يعرفوا مدى تقواكم لإلهنا."

\par 29 «أريتهم أيضًا الكؤوس التي أرسلتموها، عشرين من ذهب وثلاثين من فضة، والخمس جامات ومائدة التدشين، ومئة وزنة من الفضة لتقديم الذبائح وتوفير ما يحتاجه الهيكل.»

\par 30 «لقد أُحضرت لي هذه الهدايا من قِبل أندرياس، أحد أكثر خدامك تكريمًا، وأريستياس، وكلاهما رجلان صالحان وصادقان، يتميزان بعلمهما، وجديران بكل الطرق أن يكونا ممثلين لمبادئك السامية ومقاصدك الصالحة.»

\par 31 «لقد نقل إليّ هؤلاء الرجال رسالتك، وتلقوا مني ردًا يتفق مع رسالتك. سأوافق على كل ما هو مفيد لك، حتى لو كان طلبك غير عادي للغاية.»

\par 32 «لأنكم أنعمتم على مواطنينا بفوائد عظيمة لا تُنسى أبدًا بطرق عديدة.»

\par 33 «لذلك قدمتُ في الحال ذبائح عنك وعن أختك وعن أولادك وعن أصدقائك، وصلى كل الشعب أن تنجح خططك دائمًا، وأن يحفظ الله القدير مملكتك بسلام وبكرامة، وأن تكون ترجمة الشريعة المقدسة مفيدة لك وتُنفذ بنجاح.»

\par 34 «بحضور جميع الناس، اخترت ستة شيوخ من كل سبط، رجالاً صالحين وأمناء، وأرسلتهم إليك مع نسخة من شريعتنا.»

\par 35 «سيكون من لطفك، أيها الملك الصالح، أن تُصدر تعليمات بأنه بمجرد الانتهاء من ترجمة القانون، سيعود الرجال إلينا سالمين. وداعًا!»

\par 36 فيما يلي أسماء الشيوخ: من السبط الأول: يوسف، حزقيا، زكريا، يوحنا، حزقيا، أليشع

\par 37 من السبط الثاني: يهوذا، سمعان، صموئيل، أداوس، متثيا، إشليميا

\par 38 من السبط الثالث: نحميا، يوسف، ثيودوسيوس، بسيس، أرنياس، داكيس

\par 39 من السبط الرابع: يونثان، أبراهيم، إليشع، حنانيا، خابرياس...

\par 40 من السبط الخامس: إسحاق، يعقوب، يسوع، سباتيوس، سمعان، لاوي.

\par 41 من السبط السادس يهوذا، يوسف، سمعان، زكريا، صموئيل، سيلام.

\par 42 من السبط السابع: سبتيوس، صدقيا، يعقوب، إسحاق، يشعيا، نثاوس

\par 43 من السبط الثامن: ثيودوسيوس، ياسون، يسوع، ثيودوتوس، يوحنا، يوناثان

\par 44 من السبط التاسع: ثاوفيلس، إبراهيم، أرساموس، ياسون، إنديمياس، دانيال

\par 45 من السبط العاشر: إرميا، أليعازار، زكريا، بانياس، أليشع، داثايوس

\par 46 من السبط الحادي عشر: صموئيل، يوسف، يهوذا، يوناثيس، حابو، دوسيثيوس

\par 47 من السبط الثاني عشر: إشعياء، يوحنا، ثيودوسيوس، أرساموس، أبييتس، حزقيال

\par 48 كان عددهم اثنين وسبعين. هذا هو الجواب الذي قدمه أليعازار وأصدقاؤه لرسالة الملك

\chapter{3}

\par \textit{يُوصف فيه أروع وأجمل مائدة صُنعت على الإطلاق. بالإضافة إلى هدايا ثمينة أخرى، مثيرة للاهتمام في ضوء الحفريات الحديثة في مصر.}

\par 1 سأشرع الآن في الوفاء بوعدي وتقديم وصف للأعمال الفنية

\par 2 لقد صُنعت بمهارة استثنائية، لأن الملك لم يدخر أي جهد، وأشرف شخصيًا على العمال بشكل فردي

\par 3 لذلك، لم يتمكنوا من التهرب من أي جزء من العمل أو إنهائه بإهمال

\par 4 أولاً، سأقدم لكم وصفًا للجدول.

\par 5 وكان الملك حريصًا على أن تكون هذه القطعة من العمل ذات أبعاد كبيرة بشكل استثنائي، وأمر بإجراء استفسارات لدى اليهود في المنطقة فيما يتعلق بحجم المائدة الموجودة بالفعل في هيكل القدس.

\par 6 وعندما وصفوا القياسات، شرع في السؤال عما إذا كان بإمكانه صنع هيكل أكبر

\par 7 فأجاب بعض الكهنة واليهود الآخرون أنه لا يوجد ما يمنعه

\par 8 وقال إنه كان حريصًا على جعله أكبر بخمسة أضعاف، لكنه تردد خشية أن يكون عديم الفائدة لخدمات المعبد

\par 9 كان راغبًا في ألا تُوضع هديته في الهيكل فحسب، لأنه سيُتيح له متعة أكبر إذا تمكن الرجال الذين كان من واجبهم تقديم الذبائح المناسبة من القيام بذلك بشكل مناسب على المائدة التي صنعها

\par 10 لم يفترض أن سبب صنع الطاولة السابقة بحجم صغير كان نقص الذهب، ولكن يبدو، كما قال، أن هناك سببًا ما لصنعها بهذا الحجم

\par 11 فلو صدر الأمر، لما كان هناك نقص في الوسائل

\par 12 لذلك يجب ألا نتجاوز أو نتعدى الحد المناسب.

\par 13 وفي الوقت نفسه، أمرهم بالاستعانة بجميع أشكال الفن المتنوعة، لأنه كان رجلاً يتمتع بمفاهيم سامية، وقد وهبت الطبيعة خيالاً حادًا مكنه من تصوير المظهر الذي سيظهره العمل النهائي.

\par 14 وأصدر أوامر أيضًا، حيث لم تكن هناك تعليمات منصوص عليها في الكتب المقدسة اليهودية، يجب أن يكون كل شيء جميلًا قدر الإمكان

\par 15 عندما توضع مثل هذه التعليمات، يجب تنفيذها حرفيًا

\par 16 وصنعوا المائدة طولها ذراعان، وعرضها ذراع، وارتفاعها ذراع ونصف، وصوّروها من ذهب خالص

\par 17 ما أصفه لم يكن ذهبًا رقيقًا موضوعًا فوق أساس آخر، بل كان الهيكل بأكمله من ذهب ضخم ملحوم معًا

\par 18 وصنعوا له حاشية حولها شبرًا.

\par 19 وكانت هناك إكليل من زخارف الأمواج، محفورة بشكل بارز في شكل حبال مصنوعة بشكل رائع على جوانبها الثلاثة.

\par 20 لأنه كان مثلث الشكل، وكان أسلوب العمل متطابقًا تمامًا على كل جانب من الجوانب، لذلك أيًا كان الجانب الذي تم تدويره، كان يُظهر نفس المظهر

\par 21 من بين الجانبين تحت الحدود، كان الجانب المائل إلى أسفل الطاولة قطعة عمل جميلة للغاية، لكن الجانب الخارجي هو الذي جذب انتباه المتفرج.

\par 22 الآن، كانت الحافة العلوية للجانبين، نظرًا لارتفاعها، حادة لأنه، كما قلنا، كانت الحافة ثلاثية الجوانب، من أي وجهة نظر اقترب منها المرء

\par 23 وكانت هناك طبقات من الأحجار الكريمة عليها وسط زخارف الحبال المنقوشة، وقد تشابكت مع بعضها البعض بأسلوب فني لا يُضاهى

\par 24 من أجل الأمان، تم تثبيتها جميعًا بإبر ذهبية تم إدخالها في ثقوب الأحجار

\par 25 تم تثبيتها معًا على الجانبين بواسطة مثبتات لتثبيتها بإحكام

\par 26 على جزء الإطار المحيط بالطاولة المائل لأعلى والمُقابل للعينين، كان هناك نمط من البيض المُرصّع بالأحجار الكريمة، محفور بشكل متقن بقطعة متصلة من النقش البارز المُضلّع، متصلة بشكل وثيق حول الطاولة بأكملها

\par 27 وتحت الأحجار التي تم ترتيبها لتمثيل البيض، صنع الفنانون تاجًا يحتوي على جميع أنواع الفاكهة، وفي قمته عناقيد من العنب وكوابيس الذرة، والتمر أيضًا والتفاح، والرمان وما شابه ذلك، مرتبة بشكل واضح

\par 28 كانت هذه الفاكهة مصنوعة من أحجار كريمة، من نفس لون الفاكهة نفسها، وثبتوها على حواف جميع جوانب الطاولة بشريط من الذهب

\par 29 وبعد وضع تاج الفاكهة، أُدخل أسفله نمط آخر من البيض المرصع بالأحجار الكريمة، وأعمال أخرى من الزخرفة والنحت، بحيث يمكن استخدام جانبي الطاولة، وفقًا لرغبات المالكين، ولهذا السبب تم تمديد الزخرفة المتموجة والحافة إلى أسفل حتى أقدام الطاولة

\par 30 صنعوا وثبتوا تحت كامل عرض الطاولة صفيحة ضخمة سمكها أربعة أصابع، بحيث يمكن إدخال الأرجل فيها، وتثبيتها بدبابيس خطافية تُركب في فتحات أسفل الحافة، بحيث يمكن استخدام أي جانب من جوانب الطاولة يفضله الناس

\par 31 وهكذا أصبح من الواضح تمامًا أن العمل كان مخصصًا للاستخدام بأي طريقة

\par 32 على الطاولة نفسها نقشوا "مِندرًا"، وفي وسطه أحجار كريمة، ياقوت وزمرد وعقيق أيضًا، وأنواع أخرى كثيرة من الأحجار التي تتفوق في الجمال

\par 33 وبجانب "المايندر" وُضعت قطعة رائعة من الشبكة، جعلت مركز الطاولة يبدو وكأنه معين الشكل، وقد صُنعت عليها بلورة وعنبر، كما يُطلق عليها، مما ترك انطباعًا لا يُضاهى على الناظرين

\par 34 صنعوا أقدام المائدة برؤوس تشبه الزنابق، بحيث بدت وكأنها زنابق منحنية أسفل المائدة، والأجزاء المرئية تمثل أوراقًا منتصبة

\par 35 كانت قاعدة القدم على الأرض تتكون من ياقوتة، وكان ارتفاعها من جميع الجوانب عرض اليد

\par 36 كان شكله مثل الحذاء وكان عرضه ثمانية أصابع.

\par 37 عليها استقرت القدم بأكملها.

\par 38 وجعلوا القدم تبدو مثل اللبلاب النابت في الحجر، متشابكًا مع الأقنثوس، ومحاطًا بكرمة تحيط بها عناقيد من العنب، مصنوعة من الحجارة، إلى أعلى القدم.

\par 39 صُنعت جميع الأرجل الأربعة بنفس الأسلوب، وصُنع كل شيء ورُكّب بمهارة شديدة، وبُذلت مهارة ومعرفة رائعتان لجعلها مطابقة للطبيعة، لدرجة أنه عندما تحرك الهواء بنسمة من الرياح، انتقلت الحركة إلى الأوراق، وصُنع كل شيء ليتوافق مع الواقع الفعلي الذي يمثله

\par 40 وصنعوا سطح الطاولة من ثلاثة أجزاء مثل الثلاثي، وكانت متناسقة ومتشابكة مع بعضها البعض باستخدام صنابير على طول عرض العمل بالكامل، بحيث لم يكن من الممكن رؤية التقاء المفاصل أو حتى اكتشافها

\par 41 لم يكن سمك الطاولة أقل من نصف ذراع، لذا لا بد أن العمل بأكمله قد كلف العديد من المواهب

\par 42 بما أن الملك لم يرغب في زيادة حجمها، فقد أنفق على التفاصيل نفس المبلغ من المال الذي كان سيلزم لو كانت الطاولة ذات أبعاد أكبر

\par 43 وقد أُكمل كل شيء وفقًا لخطته، بطريقة رائعة ومميزة للغاية، بفن لا يُضاهى وجمال لا يُضاهى

\par 44 من بين أوعية الخلط، كان اثنان مصنوعين من الذهب، ومن القاعدة إلى المنتصف نُقش عليهما نقش بارز على شكل قشور، وبين القشور أُدخلت أحجار كريمة بمهارة فنية كبيرة

\par 45 ثم كان هناك "مِعْرَض" ارتفاعه ذراع، سطحه مصنوع من أحجار كريمة متعددة الألوان، مما يُظهر جهدًا فنيًا وجمالًا عظيمين

\par 46 فوق هذا كانت هناك فسيفساء، مصنوعة على شكل معين، ذات مظهر يشبه الشبكة وتمتد حتى حافتها

\par 47 في المنتصف، دروع صغيرة مصنوعة من أحجار كريمة مختلفة، موضوعة بالتناوب، ومختلفة في النوع، لا يقل عرضها عن أربعة أصابع، مما زاد من جمال مظهرها

\par 48 على قمة الحافة، كانت هناك زخرفة من زنابق متفتحة، وعناقيد عنب متشابكة محفورة في كل مكان

\par 49 هكذا كان بناء الأوعية الذهبية، وكان كل منها يتسع لأكثر من كأسين

\par 50 كانت الأوعية الفضية ذات سطح أملس، وصُنعت بشكل رائع كما لو كانت مخصصة للمرآة، بحيث كان كل ما يُقرب منها ينعكس بشكل أوضح من المرايا

\par 51 ولكن من المستحيل وصف الانطباع الحقيقي الذي تركته هذه الأعمال الفنية في الأذهان عند الانتهاء منها

\par 52 لأنه عندما تم الانتهاء من هذه الأوعية ووضعها جنبًا إلى جنب، أولاً وعاء من الفضة ثم وعاء ذهبي، ثم آخر فضي، ثم آخر ذهبي، فإن المظهر الذي قدمته لا يمكن وصفه على الإطلاق، ولم يتمكن أولئك الذين جاءوا لرؤيتها من تمزيق أنفسهم من المنظر الرائع والمشهد الأخاذ.

\par 53 كانت الانطباعات التي أحدثها المشهد متنوعة في نوعها.

\par 54 عندما نظر الرجال إلى الأوعية الذهبية، وقامت عقولهم بإجراء مسح كامل لكل تفاصيل الصنعة، كانت أرواحهم تتشوّق للدهشة.

\par 55 مرة أخرى، عندما أراد رجل أن يوجه نظره إلى الأواني الفضية، وهي واقفة أمامه، بدا أن كل شيء يلمع بالضوء حول المكان الذي كان يقف فيه، مما وفر متعة أكبر للناظرين

\par 56 بحيث يستحيل وصف الجمال الفني للأعمال

\par 57 نقشوا في المنتصف القوارير الذهبية بأكاليل من الكرمة.

\par 58 وعلى الحواف نسجوا إكليلا من اللبلاب والآس والزيتون بشكل بارز ووضعوا فيه أحجارا كريمة.

\par 59 أما الأجزاء الأخرى من أعمال الإغاثة فقد صنعوها بأنماط مختلفة، حيث جعلوا من شرفهم إكمال كل شيء بطريقة تليق بجلالة الملك

\par 60 باختصار، يمكن القول إنه لم يكن في خزانة الملك ولا في أي خزانة أخرى أي أعمال تضاهي هذه من حيث التكلفة أو المهارة الفنية

\par 61 لأن الملك لم يفكر فيهم كثيرًا، لأنه كان يحب أن ينال المجد لتميز خططه

\par 62 لأنه كان في كثير من الأحيان يهمل أعماله الرسمية، ويقضي وقته مع الفنانين في قلقه من أن يُكملوا كل شيء بطريقة تليق بالمكان الذي ستُرسل إليه الهدايا

\par 63 وهكذا تم تنفيذ كل شيء على نطاق واسع، بطريقة تليق بالملك الذي أرسل الهدايا، وبرئيس الكهنة الذي كان حاكمًا للأرض

\par 64 لم يكن هناك نقص في الأحجار الكريمة، حيث تم استخدام ما لا يقل عن خمسة آلاف حجر، وكانت جميعها كبيرة الحجم

\par 65 تم استخدام مهارة فنية استثنائية، بحيث كانت تكلفة الأحجار والصناعة خمسة أضعاف تكلفة الذهب

\par \textit{الحواشي السفلية}

\par \textit{148:1 الذراع يساوي 18 بوصة.}

\chapter{4}

تفاصيل حية للتضحية. دقة الكهنة في وصفهم جديرة بالملاحظة. حفلة عربدة وحشية. وصف للمعبد ومنشآته المائية.

\par 1 لقد أعطيتك هذا الوصف للهدايا لأنني اعتقدت أنه ضروري

\par 2 النقطة التالية في السرد هي سرد ​​رحلتنا إلى إليعازار، لكنني سأقدم لكم أولًا وصفًا للبلد بأكمله

\par 3 عندما وصلنا إلى أرض اليهود، رأينا المدينة تقع في وسط يهودا بأكملها على قمة جبل مرتفع للغاية

\par 4 على القمة، بُني المعبد بكل روعته.

\par 5 وكان محاطاً بثلاثة أسوار ارتفاعها أكثر من سبعين ذراعاً، وطولها وعرضها يتناسب مع بناء المبنى.

\par 6 تميزت جميع المباني بفخامة وباهظة غير مسبوقة

\par 7 كان من الواضح أنه لم يتم ادخار أي نفقات على الباب والمثبتات التي تربطه بقوائم الباب، وعلى ثبات العتب

\par 8 كان نمط الستارة أيضًا متناسبًا تمامًا مع نمط المدخل

\par 9 كان نسيجها في حركة دائمة بسبب تيار الريح، وبما أن هذه الحركة كانت تنتقل من الأسفل وانتفاخ الستارة إلى أقصى حد لها، فقد وفرت مشهدًا ممتعًا لا يكاد يستطيع الإنسان أن ينزع نفسه عنه

\par 10 كان بناء المذبح متناسبًا مع المكان نفسه ومع المحرقات التي أُحرقت عليه، وكان الاقتراب منه على نطاق مماثل

\par 11 كان هناك منحدر تدريجي للصعود إليه، مُرتَّب بشكل ملائم لغرض الحشمة، وكان الكهنة الخادمون يرتدون ملابس من الكتان، حتى كواحلهم

\par 12 يواجه المعبد الشرق وظهره نحو الغرب.

\par 13 وتُرصَف الأرض كلها بالحجارة وتنحدر إلى الأماكن المحددة، حتى يمكن نقل الماء لغسل الدم من الذبائح، حيث يتم التضحية بآلاف الحيوانات هناك في أيام الأعياد.

\par 14 وهناك مصدر لا ينضب من المياه، لأن نبعًا طبيعيًا وفيرًا يتدفق من داخل منطقة المعبد

\par 15 علاوة على ذلك، توجد صهاريج رائعة لا توصف تحت الأرض، كما أشاروا لي، على مسافة خمسة فيرلنغ حول موقع المعبد، ولكل منها عدد لا يحصى من الأنابيب بحيث تتقارب الجداول المختلفة معًا

\par 16 وقد ثُبّتت جميعها بالرصاص من الأسفل وعلى الجدران الجانبية، ورُشّت فوقها كمية كبيرة من الجص، ونُفِّذ كل جزء من العمل بعناية فائقة

\par 17 توجد فتحات عديدة للماء عند قاعدة المذبح، وهي غير مرئية للجميع إلا لمن يقومون بالخدمة، بحيث يُغسل كل دم الذبائح الذي يُجمع بكميات كبيرة في لمح البصر

\par 18 هذا هو رأيي فيما يتعلق بطبيعة الخزانات، وسأوضح لكم الآن كيف تم تأكيد ذلك

\par 19 أخذوني لأكثر من أربعة فيرلنغ خارج المدينة وطلبوا مني أن أنظر إلى أسفل نحو مكان معين وأن أستمع إلى الضجيج الذي أحدثه التقاء المياه، حتى أصبح الحجم الكبير للخزانات واضحًا لي، كما سبق أن أشرت

\par 20 إن خدمة الكهنة لا مثيل لها من جميع النواحي، سواء من حيث قدرتها على التحمل البدني أو من حيث خدمتها المنظمة والصامتة

\par 21 لأنهم جميعًا يعملون بشكل عفوي، على الرغم من أن ذلك يستلزم الكثير من الجهد المؤلم، ولكل منهم مهمة خاصة موكلة إليه

\par 22 وتستمر الخدمة دون انقطاع - يقدم البعض الخشب، والبعض الآخر الزيت، والبعض الآخر دقيق القمح، والبعض الآخر التوابل؛ ويقدم آخرون أيضًا قطع اللحم للمحرقة، مما يُظهر درجة رائعة من القوة.

\par 23 فإنهم يحملون بكلتا يديها ساقي عجل، يزن كل منهما أكثر من وزنتين، ويطرحانهما بكلتا يديها بطريقة عجيبة على مرتفع المذبح، ولا يخطئان في وضعهما في المكان المناسب

\par 24 كذلك فإن قطع الغنم والماعز رائعة من حيث وزنها ودسمها

\par 25 بالنسبة لأولئك الذين يعملون في هذا المجال، فإنهم يختارون دائمًا الحيوانات التي لا عيب فيها وخاصةً تلك التي تكون سمينة، وبالتالي يتم تنفيذ الذبيحة التي وصفتها

\par 26 هناك مكان خاص مخصص لهم للراحة، حيث يجلس المعفون من الخدمة

\par 27 عندما يحدث هذا، فإن أولئك الذين استراحوا بالفعل وهم مستعدون لاستئناف واجباتهم ينهضون تلقائيًا لأنه لا يوجد من يعطي الأوامر فيما يتعلق بترتيب القرابين

\par 28 يسود الصمت التام بحيث قد يتخيل المرء أنه لم يكن هناك شخص واحد حاضر، على الرغم من وجود سبعمائة رجل منخرطين في العمل، بالإضافة إلى العدد الهائل من المنشغلين بتقديم القرابين

\par 29 يتم تنفيذ كل شيء باحترام وبطريقة تليق بالله العظيم

\par 30 لقد دهشنا بشدة عندما رأينا أليعازار منشغلاً بالخدمة، من طريقة لباسه، وعظمة مظهره، التي انكشفت في الرداء الذي كان يرتديه والأحجار الكريمة التي كانت على جسده

\par 31 كانت هناك أجراس ذهبية على الثوب تصل إلى قدميه، تُصدر نوعًا غريبًا من اللحن، وعلى جانبيها كان هناك رمان بأزهار متنوعة ذات لون رائع

\par 32 كان مُزَرَّقًا بِزِنادَةٍ جَمالٍ بَارِزٍ، منسوجةٍ بأجملِ الألوان

\par 33 كان يرتدي على صدره وحي الله، كما يُسمى، والذي نُقش عليه اثنا عشر حجرًا، من أنواع مختلفة، مثبتة معًا بالذهب، تحتوي على أسماء زعماء القبائل، وفقًا لترتيبهم الأصلي، كل واحد منهم يلمع بطريقة لا توصف بلونه الخاص

\par 34 كان يرتدي على رأسه تاجًا، كما يُطلق عليه، وفوقه في منتصف جبهته عمامة لا تُضاهى، وهي الإكليل الملكي الممتلئ بالمجد مع اسم الله محفورًا بأحرف مقدسة على صفيحة من الذهب... بعد أن اعتُبر جديرًا بارتداء هذه الرموز في الخدمات

\par 35 خلق مظهرهم رهبةً وارتباكًا ذهنيًا لدرجة جعلت المرء يشعر وكأنه أمام رجل ينتمي إلى عالم مختلف

\par 36 أنا مقتنع بأن أي شخص يشارك في المشهد الذي وصفته سوف يمتلئ بالدهشة والدهشة التي لا توصف، وسوف يتأثر بشدة في ذهنه عند التفكير في القداسة المرتبطة بكل تفاصيل الخدمة.

\par 37 ولكن لكي نتمكن من الحصول على معلومات كاملة، صعدنا إلى قمة القلعة المجاورة ونظرنا حولنا

\par 38 يقع في مكان مرتفع للغاية، وهو محصن بأبراج عديدة، بُنيت حتى قمته، من حجارة ضخمة، بهدف، كما أُبلغنا، حراسة حرم المعبد، بحيث إذا وقع هجوم، أو تمرد، أو هجوم من العدو، فلن يتمكن أحد من اقتحام الأسوار المحيطة بالمعبد

\par 39 وُضعت على أبراج القلعة محركات حربية وأنواع مختلفة من الآلات، وكان موقعها أعلى بكثير من دائرة الأسوار التي ذكرتها

\par 40 كانت الأبراج أيضًا تحت حراسة رجال موثوق بهم للغاية، قدموا أقصى درجات البرهان على ولائهم لبلدهم

\par 41 لم يُسمح لهؤلاء الرجال أبدًا بمغادرة القلعة، إلا في أيام الأعياد، وفي مفارز فقط، ولم يسمحوا لأي غريب بدخولها

\par 42 كانوا أيضًا حذرين للغاية عند صدور أي أمر من كبير الضباط للسماح لأي زائر بتفتيش المكان، كما علمتنا تجربتنا الخاصة

\par 43 كانوا مترددين للغاية في السماح لنا - على الرغم من أننا كنا مجرد رجلين أعزلين - بمشاهدة تقديم القرابين

\par 44 وأكدوا أنهم كانوا ملزمين بقسم عندما أُوكلت إليهم الأمانة، لأنهم جميعًا أقسموا وكانوا ملزمين بتنفيذ القسم حرفيًا، وأنه على الرغم من أن عددهم كان خمسمائة، إلا أنهم لن يسمحوا لأكثر من خمسة رجال بالدخول في وقت واحد

\par 45 كانت القلعة بمثابة الحماية الخاصة للمعبد، وقد حصنها مؤسسها بقوة لتتمكن من حمايته بكفاءة

\chapter{5}

\par \textit{وصف للمدينة والريف. قارن الآية 11 بظروف اليوم. تكشف الآيات 89-41 كيف كان القدماء يقدرون العالم والرجل النبيل.}

\par 1 حجم المدينة ذو أبعاد معتدلة.

\par 2 يبلغ محيطه حوالي أربعين فرلنغًا، على حد ما يمكن للمرء أن يخمنه.

\par 3 أبراجه مرتبة على شكل مسرح، مع طرق رئيسية تؤدي بينها. الآن، يمكن رؤية تقاطعات الأبراج السفلية، لكن تقاطعات الأبراج العلوية أكثر ازدحامًا

\par 4 لأن الأرض ترتفع، لأن المدينة مبنية على جبل.

\par 5 هناك أيضًا درجات تؤدي إلى مفترق الطرق، وبعض الناس يصعدون دائمًا، والبعض الآخر ينزل، ويبقون على مسافة بعيدة عن بعضهم البعض قدر الإمكان على الطريق من أجل أولئك الذين يلتزمون بقواعد الطهارة، حتى لا يلمسوا أي شيء غير قانوني.

\par 6 ولم يكن من دون سبب أن قام المؤسسون الأصليون للمدينة ببنائها بالتناسب المناسب، إذ كانوا يمتلكون رؤية واضحة فيما يتعلق بما هو مطلوب.

\par 7 فالبلاد واسعة وجميلة.

\par 8 وبعض أجزائها مستوية، ولا سيما المناطق التي تنتمي إلى السامرة، كما تسمى، والتي تجاور أرض الأدوميين، وبعضها جبلية، ولا سيما تلك التي تجاور أرض يهوذا.

\par 9 لذلك، فإن الناس ملزمون بتكريس أنفسهم للزراعة وزراعة التربة حتى يتمكنوا من خلال هذه الوسيلة من الحصول على إمدادات وفيرة من المحاصيل

\par 10 وبهذه الطريقة، تتم الزراعة بجميع أنواعها، ويُحصد حصاد وفير في كامل الأرض المذكورة أعلاه

\par 11 المدن الكبيرة التي تتمتع بازدهار مماثل تكون مكتظة بالسكان، لكنها تهمل المناطق الريفية، لأن جميع الناس يميلون إلى حياة المتعة، لأن كل شخص لديه ميل طبيعي نحو السعي وراء المتعة

\par 12 حدث الشيء نفسه في الإسكندرية، التي تتفوق على جميع المدن من حيث الحجم والازدهار

\par 13 أدى هجرة سكان الريف من المناطق الريفية واستقرارهم في المدينة إلى تشويه سمعة الزراعة: ولمنعهم من الاستقرار في المدينة، أصدر الملك أوامر بعدم بقائهم فيها لأكثر من عشرين يومًا. 2

\par 14 وبنفس الطريقة، أعطى القضاة تعليمات مكتوبة بأنه إذا كان من الضروري إصدار استدعاء ضد أي شخص يعيش في البلاد، فيجب تسوية القضية في غضون خمسة أيام

\par 15 ولأنه اعتبر الأمر ذا أهمية كبيرة، فقد عيّن أيضًا مسؤولين قانونيين لكل منطقة مع مساعديهم، حتى لا يضطر المزارعون ومحاموهم، في مصلحة العمل، إلى إفراغ مخازن حبوب المدينة، أعني من إنتاج الزراعة

\par 16 لقد سمحتُ بهذا الاستطراد لأن إليعازار هو من أشار بوضوح كبير إلى النقاط التي ذُكرت

\par 17 لأن الطاقة التي يبذلونها في حرث التربة عظيمة

\par 18 فالأرض مزروعة بكثافة بأشجار الزيتون، وبمحاصيل الذرة والبقوليات، وبالكروم أيضًا، وهناك وفرة من العسل

\par 19 لا تُحتسب أنواع أشجار الفاكهة والتمور الأخرى مقارنة بهذه

\par 20 توجد ماشية من جميع الأنواع بكميات كبيرة ومراعي غنية لها

\par 21 ولذلك، فهم يدركون بحق أن المناطق الريفية تحتاج إلى عدد كبير من السكان، وأن العلاقات بين المدينة والقرى منظمة بشكل صحيح

\par 22 يُحضر العرب كمية كبيرة من التوابل والأحجار الكريمة والذهب إلى البلاد

\par 23 فالبلاد مهيأة جيدًا ليس فقط للزراعة ولكن أيضًا للتجارة، والمدينة غنية بالفنون ولا تفتقر إلى أي من البضائع التي تُجلب عبر البحر

\par 24 وهي تمتلك موانئ مناسبة وواسعة في عسقلان ويافا وغزة، وكذلك في بطليموس التي أسسها الملك والتي تحتل موقعًا مركزيًا مقارنة بالأماكن الأخرى المذكورة، حيث أنها ليست بعيدة عن أي منها.

\par 25 تنتج البلاد كل شيء بوفرة، حيث تتمتع بسقي جيد في جميع الاتجاهات ومحمية جيدًا من العواصف

\par 26 نهر الأردن، كما يُطلق عليه، والذي لا يجف أبدًا، يتدفق عبر الأرض

\par 27 في الأصل، كانت مساحة البلاد لا تقل عن 60 مليون فدان - على الرغم من أن الشعوب المجاورة قامت بغزوات عليها بعد ذلك - وتم توطين 600,000 رجل فيها في مزارع تبلغ مساحة كل منها مائة فدان

\par 28 يرتفع النهر، مثل نهر النيل، في وقت الحصاد ويروي جزءًا كبيرًا من الأرض

\par 29 بالقرب من المنطقة التابعة لشعب بطليموس، ينبع في نهر آخر ويتدفق هذا النهر إلى البحر

\par 30 تتدفق سيول جبلية أخرى، كما تُسمى، إلى السهل وتحيط بالأجزاء المحيطة بغزة وقضاء أشدود

\par 31 البلاد محاطة بسياج طبيعي، ومن الصعب جدًا مهاجمتها ولا يمكن اقتحامها بقوات كبيرة، بسبب الممرات الضيقة، والمنحدرات المتدلية والوديان العميقة، والطبيعة الوعرة للمناطق الجبلية التي تحيط بالأرض بأكملها

\par 32 قيل لنا إنه كان يتم الحصول على النحاس والحديد سابقًا من جبال شبه الجزيرة العربية المجاورة

\par 33 ومع ذلك، فقد توقف هذا في زمن الحكم الفارسي، حيث نشرت السلطات في ذلك الوقت تقريرًا كاذبًا في الخارج مفاده أن العمل في المناجم عديم الفائدة ومكلف، وذلك لمنع تدمير بلادهم بسبب التعدين في هذه المناطق، وربما انتزاعها منهم بسبب الحكم الفارسي، حيث وجدوا بمساعدة هذا التقرير الكاذب ذريعة لدخول المنطقة

\par 34 لقد قدمت لك الآن، أخي العزيز فيلوكراتيس، جميع المعلومات الأساسية حول هذا الموضوع بشكل موجز

\par 35 سأصف عمل الترجمة في الجزء الثاني.

\par 36 كان رئيس الكهنة يختار الرجال من ذوي أفضل الأخلاق وأعلى مستويات الثقافة، كما هو متوقع من والديهم النبيل.

\par 37 كانوا رجالاً لم يكتسبوا إتقانًا في الأدب اليهودي فحسب، بل درسوا أيضًا الأدب اليوناني بعناية فائقة

\par 38 لذلك كانوا مؤهلين بشكل خاص للخدمة في السفارات، وقاموا بهذه المهمة كلما كان ذلك ضروريًا

\par 39 كانوا يمتلكون قدرة كبيرة على عقد المؤتمرات ومناقشة المشكلات المتعلقة بالقانون

\par 40 لقد تبنوا المسار الوسطي - وهذا هو دائمًا المسار الأفضل الذي يجب اتباعه

\par 41 لقد نبذوا الأسلوب الخشن والفظ، لكنهم كانوا فوق الكبرياء تمامًا ولم يتباهوا أبدًا بهالة من التفوق على الآخرين، وفي المحادثة كانوا مستعدين للاستماع وإعطاء إجابة مناسبة لكل سؤال

\par 42 وكانوا جميعًا يلتزمون بهذه القاعدة بعناية، وكانوا حريصين فوق كل شيء آخر على التفوق على بعضهم البعض في مراعاتها، وكانوا جميعًا جديرين بزعيمهم وفضيلته.

\par 43 ويمكن للمرء أن يلاحظ كيف أحبوا أليعازار من خلال عدم رغبتهم في الانفصال عنه وكيف أحبهم

\par 44 بالإضافة إلى الرسالة التي كتبها إلى الملك بشأن عودتهم سالمين، فقد توسل أيضًا إلى أندرياس بجدية للعمل من أجل نفس الغاية، وحثني أيضًا على المساعدة بأفضل ما في وسعي

\par 45 وعلى الرغم من أننا وعدنا بإيلاء أفضل اهتمامنا للأمر، إلا أنه قال إنه لا يزال يشعر بحزن شديد، لأنه كان يعلم أن الملك، بدافع طيبة طبعه، كان يعتبر أن من أعظم امتيازاته، كلما سمع عن رجل يتفوق على رفاقه في الثقافة والحكمة، أن يستدعيه إلى بلاطه

\par 46 فقد سمعتُ له قولاً حسناً مفاده أنه بتأمين رجال عادلين وحكماء لشخصه، فإنه سيضمن أعظم حماية لمملكته، لأن هؤلاء الأصدقاء سيقدمون له بلا تحفظ أفضل النصائح

\par 47 ولا شك أن الرجال الذين أرسلهم إليه أليعازار كانوا يمتلكون هذه الصفات

\par 48 وكثيرًا ما كان يؤكد تحت القسم أنه لن يسمح أبدًا للرجال بالرحيل إذا كانت مجرد مصلحة خاصة به هي الدافع وراء ذلك - ولكن من أجل المصلحة المشتركة لجميع المواطنين كان يرسلهم

\par 49 لأنه أوضح أن الحياة الطيبة تتمثل في حفظ أحكام الشريعة، وأن هذه الغاية تتحقق بالسماع أكثر بكثير من القراءة

\par 50 من هذه التصريحات وغيرها من التصريحات المماثلة، اتضحت مشاعره تجاههم

\par \textit{الحواشي السفلية}

\par \textit{154:1 الفرلنغ هو 1/8 ميل (أي 220 ياردة).}

\par \textit{154:2 إن هذا الحساب للإجراءات المتخذة في الإسكندرية لمنع هجرة سكان الريف إلى المدينة يشكل كشفاً مثيراً للاهتمام بأن المسألة كانت حادة قبل 2000 عام كما هي اليوم.}

\chapter{6}

\par \textit{شروحات لعادات الناس توضح المقصود بكلمة "نجس". جوهر وأصل "الإيمان بالله". تقدم الآيتان 48-44 وصفًا بديعًا لألوهية علم وظائف الأعضاء.}

\par 1 يجدر بنا أن نذكر بإيجاز المعلومات التي قدمها ردًا على أسئلتنا

\par 2 لأني أظن أن معظم الناس يشعرون بالفضول تجاه بعض أحكام الشريعة، وخاصة تلك المتعلقة باللحوم والأشربة والحيوانات التي تُعتبر نجسة

\par 3 عندما سألنا لماذا، بما أن هناك شكلًا واحدًا فقط من الخلق، تُعتبر بعض الحيوانات نجسة للأكل، وبعضها الآخر نجسة حتى للمس (فمع أن القانون دقيق في معظم النقاط، إلا أنه دقيق بشكل خاص في مثل هذه الأمور)، بدأ رده على النحو التالي:

\par 4 «لاحظ،» قال، «ما هو التأثير الذي تحدثه أنماط حياتنا وارتباطاتنا علينا؛ فمن خلال الارتباط بالسيئ، يصاب الرجال بانحرافاتهم ويصبحون بائسين طوال حياتهم؛ ولكن إذا عاشوا مع الحكماء والحكماء، فإنهم يجدون وسائل الهروب من الجهل وإصلاح حياتهم».

\par 5 وضع مُشرِّعنا، أولاً وقبل كل شيء، مبادئ التقوى والصلاح، وغرسها نقطة بنقطة، ليس فقط من خلال المحظورات، بل من خلال استخدام الأمثلة أيضًا، موضحًا الآثار الضارة للخطيئة والعقوبات التي يوقعها الله على المذنبين

\par 6 لأنه أثبت أولاً وقبل كل شيء أنه لا يوجد سوى إله واحد وأن قدرته تتجلى في جميع أنحاء الكون، حيث أن كل مكان مليء بسيادته، ولا يغيب عن علمه أي شيء مما يصنعه البشر سراً على الأرض

\par 7 لأن كل ما يفعله الإنسان وكل ما سيحدث في المستقبل ظاهر له

\par 8 بعد أن عمل على هذه الحقائق بعناية وأوضحها، أظهر أنه حتى لو فكر رجل في فعل الشر - ناهيك عن تنفيذه بالفعل - فلن يفلت من الكشف، لأنه أوضح أن قوة الله تتخلل الشريعة بأكملها

\par 9 انطلاقًا من نقطة بدايته، واصل توضيح أن جميع البشر باستثنائي يؤمنون بوجود العديد من الآلهة، على الرغم من أنهم أنفسهم أقوى بكثير من الكائنات التي يعبدونها عبثًا

\par 10 فعندما يصنعون تماثيل من الحجر والخشب، يقولون إنها صور لأولئك الذين اخترعوا شيئًا مفيدًا للحياة، ويعبدونها، مع أن لديهم دليلًا واضحًا على أنهم لا يمتلكون أي شعور

\par 11 لأنه سيكون من الحماقة تمامًا افتراض أن أي شخص أصبح إلهًا بفضل اختراعاته

\par 12 لأن المخترعين ببساطة أخذوا أشياءً معينةً تم إنشاؤها بالفعل، ومن خلال دمجها معًا، أظهروا أنها تمتلك منفعة جديدة: لم يخلقوا بأنفسهم جوهر الشيء، ولذلك فمن العبث والحماقة أن يصنع الناس آلهةً من بشر مثلهم

\par 13 ففي عصرنا هذا، يوجد كثيرون أكثر إبداعًا وعلمًا من رجال الأيام السابقة الذين تم تأليههم، ومع ذلك لم يأتوا أبدًا لعبادتهم

\par 14 يعتقد صانعو ومؤلفو هذه الأساطير أنهم أحكم الإغريق

\par 15 لماذا نحتاج إلى الحديث عن شعوب أخرى مفتونة، كالمصريين وأمثالهم، الذين يعتمدون على الوحوش البرية ومعظم أنواع الزاحفات والماشية، ويعبدونها، ويقدمون لها الذبائح في حياتهم وأمواتهم؟

\par 16 "ولأن مشرعنا كان رجلاً حكيماً وموهوباً من الله بشكل خاص لفهم كل الأشياء، فقد نظر إلى كل التفاصيل الدقيقة نظرة شاملة، وسيّجنا بأسوار منيعة وجدران من الحديد، حتى لا نختلط على الإطلاق بأي من الأمم الأخرى، بل نبقى طاهرين في الجسد والروح، أحرارًا من كل التخيلات الباطلة، ونعبد الله الواحد القدير فوق الخليقة كلها.

\par 17 ومن ثم، بعد أن نظر كبار الكهنة المصريين بعناية في العديد من الأمور، وكانوا على دراية بشؤوننا، أطلقوا علينا اسم "رجال الله".

\par 18 هذا لقب لا يخص بقية البشر، بل يخص فقط أولئك الذين يعبدون الإله الحقيقي

\par 19 أما الباقون فهم رجال ليسوا من الله، بل من الأطعمة والمشروبات والملابس.

\par 20 لأن تصرفاتهم كلها تقودهم إلى إيجاد العزاء في هذه الأشياء التي لا يُحسب لها حساب، بل في كل ما يخصهم.

\par 21 إن اعتبار شعبنا الرئيسي طوال حياتهم هو سيادة الله

\par 22 لذلك، لئلا نفسد بأي رجس، أو تنحرف حياتنا بالمعاملات الشريرة، فقد أحاطنا من جميع الجوانب بقواعد الطهارة، مما يؤثر على ما نأكله، أو نشربه، أو نلمسه، أو نسمعه، أو نراه

\par 23 فمع أن جميع الأشياء، بشكل عام، متشابهة في تركيبها الطبيعي، لأنها جميعًا تحكمها قوة واحدة، إلا أن هناك سببًا عميقًا في كل حالة فردية يجعلنا نمتنع عن استخدام أشياء معينة ونستمتع بالاستخدام المشترك لأشياء أخرى

\par 24 من أجل التوضيح، سأتناول نقطة أو نقطتين وأشرحهما لكم

\par 25 فلا يجب أن تقع في فخ الفكرة المهينة القائلة بأن موسى وضع قوانينه بعناية فائقة من باب مراعاة الفئران وابن عرس وأشياء أخرى من هذا القبيل. 1

\par 26 وُضعت كل هذه المراسيم من أجل البر للمساعدة في السعي وراء الفضيلة وتكميل الشخصية

\par 27 فجميع الطيور التي نستخدمها أليفة وتتميز بنظافتها، وتتغذى على أنواع مختلفة من الحبوب والبقوليات، مثل الحمام، واليمام، والجراد، والحجل، والإوز أيضًا، وجميع الطيور الأخرى من هذه الفئة

\par 28 لكن الطيور الممنوعة ستجدها برية وآكلة للحوم، تتسلط على الآخرين بالقوة التي تمتلكها، وتحصل على طعامها بقسوة عن طريق افتراس الطيور الأليفة المذكورة أعلاه

\par 29 وليس هذا فحسب، بل إنهم يختطفون الحملان والجداء، ويؤذون البشر أيضًا، سواء أكانوا أمواتًا أم أحياءً، ولذا بتسميتهم نجسين، فقد أعطى إشارة من خلالهم إلى أن أولئك الذين وُضع لهم التشريع يجب أن يمارسوا البر في قلوبهم، وألا يستبدوا بأحد بالاعتماد على قوتهم الخاصة، ولا يسرقوا منه أي شيء، بل أن يوجهوا مسار حياتهم وفقًا للعدل، تمامًا كما تلتهم الطيور الأليفة، التي سبق ذكرها، أنواعًا مختلفة من البقول التي تنمو على الأرض، ولا تستبد بأهلاك أقاربها

\par 30 لقد علمنا مشرعنا أنه من خلال مثل هذه الأساليب يتم إعطاء الإرشادات للحكماء، وأنه يجب عليهم أن يكونوا عادلين وألا يفعلوا شيئًا بالعنف، وأن يمتنعوا عن الاستبداد على الآخرين بالاعتماد على قوتهم الخاصة.

\par 31 بما أنه يُعتبر من غير اللائق حتى لمس مثل هذه الحيوانات النجسة، كما ذُكر، بسبب عاداتها الخاصة، ألا ينبغي لنا اتخاذ كل الاحتياطات خشية أن تُدمر شخصياتنا بنفس القدر؟

\par 32 لذلك، فإن جميع القواعد التي وضعها فيما يتعلق بما هو مسموح به في حالة هذه الطيور والحيوانات الأخرى، قد سنها بهدف تعليمنا درسًا أخلاقيًا

\par 33 لأن تقسيم الحافر وفصل المخالب يهدفان إلى تعليمنا أنه يجب علينا التمييز بين أفعالنا الفردية بهدف ممارسة الفضيلة

\par 34 لأن قوة جسدنا كله ونشاطه يعتمدان على أكتافنا وأطرافنا

\par 35 لذلك، فهو يُلزمنا بالاعتراف بأنه يجب علينا القيام بجميع أفعالنا بتمييز وفقًا لمعيار البر، وخاصةً لأننا انفصلنا بشكل واضح عن بقية البشر

\par 36 لأن معظم الرجال الآخرين ينجسون أنفسهم بالعلاقات غير الشرعية، مرتكبين بذلك إثمًا عظيمًا، وتفتخر دول ومدن بأكملها بمثل هذه الرذائل

\par 37 لأنهم لا يكتفون بمعاشرة الرجال، بل ينجسون أمهاتهم، وحتى بناتهم

\par 38 ولكننا قد حُفظنا منفصلين عن مثل هذه الخطايا.

\par 39 والناس الذين انفصلوا بالطريقة المذكورة، وصفهم المشرع أيضًا بأنهم يمتلكون موهبة الذاكرة.

\par 40 فجميع الحيوانات "ذات الأرجل المشقوقة والمجترة" تمثل للمبتدئ رمزًا للذاكرة

\par 41 ففعل الاجترار ليس إلا تذكُّرًا للحياة والوجود

\par 42 لأن الحياة عادةً ما تُدعم بالطعام، ولذلك يحثنا في الكتاب المقدس أيضًا بهذه الكلمات: "تذكر الرب الذي صنع فيك تلك الأمور العظيمة والعجائب".

\par 43 لأنه عندما يتم تصورها بشكل صحيح، فإنها تكون عظيمة ومجيدة بشكل واضح؛ أولاً، بناء الجسم وتوزيع الطعام وفصل كل عضو على حدة، والأكثر من ذلك، تنظيم الحواس، وعمل العقل وحركته غير المرئية، وسرعة أفعاله الخاصة واكتشافه للفنون، تُظهر براعة لا نهائية

\par 44 لذلك، يحثنا على أن نتذكر أن الأجزاء المذكورة أعلاه تُحفظ معًا بقوة إلهية بمهارة فائقة

\par 45 لأنه حدد كل زمان ومكان حتى نتذكر باستمرار الله الذي يحكمنا ويحفظنا

\par 46 أما فيما يتعلق باللحوم والمشروبات، فيطلب منا أولاً تقديم جزء منها كذبيحة، ثم الاستمتاع بوجبتنا على الفور

\par 47 وعلاوة على ذلك، فقد أعطانا على ملابسنا رمزًا للذكرى، وبالمثل أمرنا بوضع الوحي الإلهي على أبوابنا وبواباتنا كذكرى لله.

\par 48 وعلى أيدينا أيضًا، يأمر صراحةً بتثبيت الرمز، موضحًا بوضوح أنه يجب علينا القيام بكل عمل باستقامة، متذكرين خلقنا، وقبل كل شيء مخافة الله

\par 49 إنه يأمر البشر أيضًا، عند الاستلقاء للنوم والاستيقاظ مجددًا، بالتأمل في أعمال الله، ليس فقط بالكلام، ولكن من خلال ملاحظة التغيير والانطباع الذي يحدث عليهم بوضوح، عندما ينامون، وأيضًا عند استيقاظهم، كم هو إلهي وغير مفهوم التغيير من إحدى هاتين الحالتين إلى الأخرى

\par 50 لقد تم الآن توضيح روعة القياس فيما يتعلق بالتمييز والذاكرة، وفقًا لتفسيرنا لـ "الظلف المشقوق والاجترار".

\par 51 لأن قوانيننا لم تُوضع عشوائيًا أو وفقًا لأول فكرة عابرة تخطر على البال، بل بهدف الحقيقة ودلالة العقل السليم

\par 52 فمن خلال التوجيهات التي يقدمها فيما يتعلق باللحوم والمشروبات وحالات اللمس الخاصة، فإنه يأمرنا ألا نفعل أي شيء أو نستمع إليه دون تفكير، وألا نلجأ إلى الظلم من خلال إساءة استخدام قوة العقل

\par 53 في حالة الحيوانات البرية أيضًا، قد يتم اكتشاف المبدأ نفسه

\par 54 لأن شخصية ابن عرس والفئران والحيوانات المماثلة، المذكورة صراحةً، مدمرة

\par 55 تُدنس الفئران كل شيء وتُتلفه، ليس فقط من أجل غذائها، بل حتى إلى حد جعل كل ما يقع في طريقها لإتلافه عديم الفائدة تمامًا للإنسان

\par 56 إن فئة ابن عرس غريبة أيضًا: فبالإضافة إلى ما قيل، لها سمة نجسة: فهي تحمل من الأذنين وتلد من الفم

\par 57 ولهذا السبب تُعتبر ممارسة مماثلة نجسة عند الرجال

\par 58 لأنهم بتجسيدهم في الكلام كل ما يتلقونه من خلال الآذان، فإنهم يشركون الآخرين في الشرور ولا يرتكبون نجاسة عادية، إذ هم أنفسهم مدنسون تمامًا بنجاسة الفجور

\par 59 وملككم، كما علمنا، يفعل الصواب تمامًا في تدمير هؤلاء الرجال

\par 60 ثم قلت: "أفترض أنك تقصد المخبرين، لأنه يعرضهم باستمرار للتعذيب ولأشكال مؤلمة من الموت."

\par 61 أجاب: «نعم، هؤلاء هم الرجال الذين أقصدهم؛ لأن مراقبة هلاك البشر أمر غير مقدس».

\par 62 ويمنعنا قانوننا من إيذاء أي شخص سواء بالقول أو بالفعل.

\par 63 إن شرحي الموجز لهذه الأمور كان ينبغي أن يقنعك بأن جميع لوائحنا قد وضعت بهدف البر، ولم يتم سن أي شيء في الكتاب المقدس دون تفكير أو سبب مناسب، ولكن غرضه هو تمكيننا طوال حياتنا وفي جميع أفعالنا من ممارسة البر أمام جميع الناس، مع مراعاة الله القدير.

\par 64 وهكذا فيما يتعلق باللحوم والأشياء النجسة، والزواحف، والوحوش البرية، فإن النظام بأكمله يهدف إلى البر والعلاقات الصالحة بين الإنسان والإنسان

\par 65 بدا لي أنه قدم دفاعًا جيدًا عن جميع النقاط؛ لأنه فيما يتعلق أيضًا بالعجول والكباش والماعز التي تُقدم، قال إنه من الضروري أخذها من القطعان والقطعان، والتضحية بالحيوانات الأليفة وعدم تقديم أي شيء بري، حتى يتمكن مقدمو الذبائح من فهم المعنى الرمزي للمشرع ولا يقعون تحت تأثير وعي ذاتي متغطرس

\par 66 لأن من يقدم ذبيحة، فإنه يقدم ذبيحة نفسه أيضًا بكل أمزجتها

\par 67 أعتقد أن هذه التفاصيل المتعلقة بمناقشتنا تستحق السرد، ونظرًا لقدسية القانون ومعناه الطبيعي، فقد حُثِلتُ على شرحها لك بوضوح يا فيلوكراتس، نظرًا لإخلاصك للتعلم

\par \textit{الحواشي السفلية}

\par \textit{158:1 قارن هذه الفكرة الغريبة مع رسالة كورنثوس الأولى، 9: 9.}

\chapter{7}

\par \textit{وصول المبعوثين مع مخطوطة الكتاب الثمين والهدايا. الاستعدادات لمأدبة ملكية. يقوم المضيف فور جلوسه على المائدة بتسلية ضيوفه بالأسئلة والأجوبة. بعض التعليقات الحكيمة على علم الاجتماع.}

\par 1 وبعد أن قدم أليعازار الذبيحة، واختار الرسل، وجهّز هدايا كثيرة للملك، أرسلنا في رحلتنا بأمان عظيم

\par 2 وعندما وصلنا إلى الإسكندرية، أُبلغ الملك على الفور بوصولنا

\par 3 عند دخولنا القصر، استقبلنا أنا وأندرياس الملك بحرارة، وسلمناه الرسالة التي كتبها إليعازار

\par 4 كان الملك متلهفًا جدًا للقاء المبعوثين، وأصدر أوامره بطرد جميع المسؤولين الآخرين واستدعاء المبعوثين إلى حضرته على الفور

\par 5 الآن، أثارت هذه المفاجأة العامة المثيرة، فمن المعتاد أن يُسمح لمن يأتون للقاء الملك في أمور ذات أهمية بحضوره في اليوم الخامس، بينما يُؤمَّن للمبعوثين من الملوك أو المدن المهمة جدًا دخول البلاط بصعوبة في ثلاثين يومًا - لكنه اعتبر هؤلاء الرجال جديرين بشرف أكبر، لأنه كان يُقدِّر سيدهم تقديرًا كبيرًا، ولذلك صرف على الفور أولئك الذين اعتبر حضورهم غير ضروري، واستمر في التجول حتى دخلوا وتمكن من الترحيب بهم

\par 6 وعندما دخلوا بالهدايا التي أرسلت معهم والرقوق الثمينة التي نقش عليها القانون بالذهب بأحرف يهودية، لأن الرقوق كانت معدة بشكل رائع وكان الارتباط بين الصفحات قد تم بطريقة غير مرئية، بمجرد أن رآهم الملك بدأ يسألهم عن الكتب.

\par 7 ولما أخرجوا اللفائف من أغطيتها وفتحوا الصفحات، وقف الملك طويلاً ثم سجد نحو سبع مرات، وقال:

\par 8 أشكركم يا أصدقائي، وأشكر من أرسلكم أكثر، والأهم من ذلك كله الله، الذي هذه هي نبوءاته

\par 9 وعندما صرخ الجميع، المبعوثون والآخرون الذين كانوا حاضرين أيضًا، في وقت واحد وبصوت واحد: "حفظ الله الملك!" انفجر في البكاء من الفرح

\par 10 لأن سمو روحه والشعور بالشرف الساحق الذي مُنح له أجبراه على البكاء على حظه السعيد

\par 11 أمرهم بإعادة اللفائف إلى أماكنها، ثم بعد أن سلم على الرجال، قال: "كان من الصواب، يا رجال الله، أن أُقدّر أولاً الكتب التي من أجلها استدعيتكم إلى هنا، وبعد أن فعلت ذلك، أن أمد إليكم يمين الصداقة."

\par 12 «لهذا السبب فعلتُ هذا أولًا.»

\par 13 "لقد قررت أن هذا اليوم الذي وصلت فيه يجب أن يظل يومًا عظيمًا ويتم الاحتفال به سنويًا طوال حياتي."

\par 14 «ويصادف أيضًا أن اليوم هو ذكرى انتصاري البحري على أنتيجونوس. لذلك سأكون سعيدًا بتناول الطعام معك اليوم.»

\par 15 قال: «كل ما قد تحتاجون إلى استخدامه، سيُهيأ لكم بالطريقة المناسبة، ولي أيضًا معكم».

\par 16 بعد أن أعربوا عن سعادتهم، أصدر أمرًا بتخصيص أفضل المساكن بالقرب من القلعة لهم، والتحضير للمأدبة

\par 17 واستدعى نيكانور رئيس الوكلاء، دوروثيوس، الذي كان الضابط الخاص المعين لرعاية اليهود، وأمره بإعداد الاستعدادات اللازمة لكل واحد منهم

\par 18 لأن هذا الترتيب قد اتخذه الملك، وهو ترتيب ترونه قائمًا حتى اليوم

\par 19 بالنسبة للعديد من المدن التي لديها عادات خاصة في أمور الشرب والأكل والاستلقاء، يتم تعيين ضباط خاصين لرعاية متطلباتها

\par 20 وكلما جاؤوا لزيارة الملوك، تُجرى الاستعدادات وفقًا لعاداتهم الخاصة، حتى لا يكون هناك أي إزعاج يعكر صفو استمتاعهم بزيارتهم

\par 21 تم اتخاذ نفس الاحتياط في حالة المبعوثين اليهود.

\par 22 وكان دوروثيوس الذي تم تعيينه لرعاية الضيوف اليهود رجلاً ضميريًا للغاية.

\par 23 جميع المؤن التي كانت تحت سيطرته والمخصصة لاستقبال هؤلاء الضيوف، أخرجها للعيد

\par 24 قام بترتيب المقاعد في صفين وفقًا لتعليمات الملك.

\par 25 لأنه أمره أن يجعل نصف الرجال يجلسون عن يمينه والبقية خلفه، حتى لا يحجب عنهم أعلى شرف ممكن

\par 26 عندما جلسوا، أمر دوروثيوس بتنفيذ كل شيء وفقًا للعادات المتبعة بين ضيوفه اليهود

\par 27 لذلك، استغنى عن خدمات المبشرين المقدسين والكهنة الذين يقدمون القرابين وغيرهم ممن اعتادوا تقديم الصلوات، ودعا أحدنا، أليعازار، أكبر الكهنة اليهود سنًا، لتقديم الصلاة بدلاً من ذلك

\par 28 ثم نهض ودعا دعاءً رائعًا. «أسأل الله العظيم أن يغنيك أيها الملك بكل ما صنع من خير، وأن يرزقك أنت وزوجتك وأولادك ورفاقك امتلاكها دائمًا ما دمت حيًا!»

\par 29 عند هذه الكلمات، اندلع تصفيق حار وفرح استمر لفترة طويلة، ثم تحولوا إلى الاستمتاع بالمأدبة التي تم إعدادها

\par 30 تم تنفيذ جميع ترتيبات الخدمة على المائدة وفقًا لأمر دوروثيوس

\par 31 كان من بين الحاضرين الخدم الملكيون وغيرهم ممن شغلوا مناصب شرف في بلاط الملك

\par 32 انتهز الملك فرصةً أُتيحت له خلال فترة استراحة في المأدبة، فسأل المبعوث الذي جلس على كرسي الشرف (لأنهم كانوا مُرتَّبين حسب الأقدمية)، كيف يُمكنه الحفاظ على مملكته سليمةً حتى النهاية؟

\par 33 بعد أن فكر للحظة، أجاب: "يمكنكم ترسيخ أمنه على أفضل وجه إذا كنتم تحاكون لطف الله الدائم. لأنه إذا أظهرتم الرأفة وفرضتم عقوبات خفيفة على من يستحقونها وفقًا لاستحقاقاتهم، فسوف تحولونهم عن الشر وتقودونهم إلى التوبة."

\par 34 أشاد الملك بالإجابة ثم سأل الرجل التالي: كيف يمكنه أن يفعل كل شيء على أكمل وجه في جميع أفعاله؟

\par 35 فأجاب: "إذا حافظ الإنسان على موقف عادل تجاه الجميع، فسوف يتصرف دائمًا بشكل صحيح في كل مناسبة، متذكرًا أن كل فكرة معروفة لدى الله. إذا اتخذت مخافة الله نقطة انطلاقك، فلن تخطئ الهدف أبدًا."

\par 36 أثنى الملك على هذا الرجل أيضًا على إجابته وسأل آخر: كيف يمكن أن يكون له أصدقاء يشاركونه نفس التفكير؟

\par 37 أجاب: «إذا رأوك تدرس مصالح الجماهير التي تحكمها، فمن الجيد أن تلاحظ كيف يمنح الله بركاته للبشرية، فيوفر لهم الصحة والطعام وكل الأشياء الأخرى في حينه».

\par 38 بعد أن أعرب الملك عن موافقته على الرد، سأل الضيف التالي: كيف يمكنه، من خلال إلقاء المحاضرات وإصدار الأحكام، أن يحظى بالثناء حتى من أولئك الذين فشلوا في الفوز بدعواتهم؟

\par 39 وقال: «إذا كنتَ منصفًا في حديثك مع الجميع على حد سواء، ولم تكن متعجرفًا ولا جائرًا في معاملتك للمسيئين. وستفعل ذلك إذا راقبتَ تصرفات الله. إن دعوات المستحقين تُستجاب دائمًا، بينما يُخبر من لا يُستجاب دعاؤهم من خلال الأحلام أو الأحداث بما كان فيه من ضرر في طلباتهم، وأن الله لا يُصيبهم حسب خطاياهم أو عظمة قوته، بل يتصرف معهم بصبر».

\par 40 أثنى الملك على الرجل بحرارة على إجابته وسأل التالي بالترتيب: كيف يمكن أن يكون لا يقهر في الشؤون العسكرية؟

\par 41 فأجاب: «لو لم يثق كليًا بجماهيره أو قواته المحاربة، بل دعا الله باستمرار أن يُكلل مشاريعه بالنجاح، بينما كان هو نفسه يؤدي جميع واجباته بروح العدل».

\par 42 رحّب بهذه الإجابة، وسأل آخر كيف يُمكن أن يُصبح موضع خوف لأعدائه

\par 43 فأجاب: «إذا تذكر، وهو يحتفظ بمخزون هائل من الأسلحة والقوات، أن هذه الأشياء عاجزة عن تحقيق نتيجة دائمة وحاسمة. فحتى الله يغرس الخوف في عقول البشر بمنح فترات راحة وإظهار عظمة قدرته فحسب.»

\par 44 أثنى الملك على هذا الرجل ثم قال للذي يليه: "ما هو أسمى خير في الحياة؟"

\par 45 فأجاب: «أن نعرف أن الله هو رب الكون، وأننا في أروع إنجازاتنا لسنا نحن من نحقق النجاح، بل الله هو الذي بقدرته يُحقق كل شيء ويقودنا إلى الهدف».

\par 46 هتف الملك قائلاً إن الرجل قد أجاب جيدًا، ثم سأل التالي: كيف يمكنه الحفاظ على جميع ممتلكاته سليمة، وفي النهاية يسلمها إلى خلفائه بنفس الحالة؟

\par 47 فأجاب: «بالدعاء المستمر إلى الله أن يلهمكم دوافع نبيلة في جميع مساعيكم، وبتحذير أحفادكم من الانبهار بالشهرة أو الثروة، لأن الله هو الذي يمنح كل هذه المواهب، والبشر لا يفوزون بالسيادة بمفردهم أبدًا».

\par 48 أعرب الملك عن موافقته على الإجابة وسأل الضيف التالي: كيف يمكنه أن يتحمل بهدوء ما يصيبه؟

\par 49 فقال: "إذا كان لديك فهم راسخ لفكرة أن جميع البشر معينون من قبل الله للمشاركة في أعظم الشرور وكذلك أعظم الخيرات، لأنه من المستحيل على الإنسان أن يكون معفيًا من هذه. لكن الله الذي يجب أن نصلي إليه دائمًا، يلهمنا الشجاعة للتحمل."

\par 50 سُرّ الملك برد الرجل، وقال إن جميع إجاباتهم كانت جيدة. وأضاف: "سأطرح سؤالاً على أحدهم الآخر، ثم سأتوقف الآن: حتى نتمكن من توجيه انتباهنا إلى الاستمتاع بالوليمة وقضاء وقت ممتع."

\par 51 ثم سأل الرجل: "ما هو الهدف الحقيقي من الشجاعة؟"

\par 52 فأجاب: «إذا نُفِّذت خطة صائبة في ساعة الخطر، وفقًا للقصد الأصلي. فكل شيء يُتمه الله لمصلحتك، أيها الملك، لأن قصدك صالح».

\par 53 عندما أبدى الجميع بتصفيقهم موافقتهم على الإجابة، قال الملك للفلاسفة (لأن عددًا ليس قليلًا منهم كان حاضرًا): "أعتقد أن هؤلاء الرجال يتفوقون في الفضيلة ويمتلكون معرفة غير عادية، لأنهم قدموا إجابات مناسبة على هذه الأسئلة التي طرحتها عليهم على الفور، وجعلوا الله جميعًا نقطة انطلاق لكلماتهم."

\par 54 وقال مينيديموس، فيلسوف إريتريا: "صحيح أيها الملك، فبما أن الكون يُدار بالعناية الإلهية، وبما أننا ندرك بحق أن الإنسان هو خلق الله، فمن الطبيعي أن تكون كل قوة وجمال الكلام من الله."

\par 55 عندما أومأ الملك موافقًا على هذا الرأي، توقف الحديث وشرعوا في الاستمتاع. وعندما حل المساء، انتهت الوليمة

\par \textit{الحواشي السفلية}

\par \textit{162:1 قارن هذا الموقف تجاه المجرمين بموقف ما يسمى بالنظرة الإنسانية الحديثة. انظر أيضًا الفصل الثامن، 11.}

\chapter{8}

\par \textit{المزيد من الأسئلة والأجوبة. لاحظ الآية 20 مع إشارتها إلى الطيران في الهواء المكتوبة عام 150 قبل الميلاد}

\par 1 في اليوم التالي، جلسوا على المائدة مرة أخرى، وواصلوا المأدبة وفقًا لنفس الترتيبات

\par 2 عندما رأى الملك أن فرصة مناسبة قد حانت لطرح الأسئلة على ضيوفه، شرع في طرح المزيد من الأسئلة على الرجال الجالسين بجواره بالترتيب، ثم على أولئك الذين قدموا إجابات في اليوم السابق

\par 3 بدأ بفتح الحوار مع الرجل الحادي عشر، إذ كان هناك عشرة أشخاص طُرحت عليهم أسئلة في المرة السابقة

\par 4 عندما ساد الصمت، سأل كيف يمكنه الاستمرار في الثراء؟

\par 5 بعد تفكير وجيز، أجاب الرجل الذي طُرح عليه السؤال: «إذا لم يفعل شيئًا لا يليق بمنصبه، ولم يتصرف أبدًا بفجور، ولم يبذر الإنفاق على مساعي فارغة وباطلة، بل من خلال أعمال الخير جعل جميع رعيته على علاقة طيبة معه. لأن الله هو مصدر كل الخير، ويجب على الإنسان طاعته.»

\par 6 أثنى عليه الملك ثم سأل آخر كيف يمكنه الحفاظ على الحقيقة؟

\par 7 قال ردًا على السؤال: "بإدراكي أن الكذب يجلب عارًا كبيرًا على جميع البشر، وخاصةً على الملوك. فبما أن لديهم القدرة على فعل ما يريدون، فلماذا يلجأون إلى الكذب؟ بالإضافة إلى ذلك، يجب أن تتذكر دائمًا، أيها الملك، أن الله محب للحقيقة."

\par 8 استقبل الملك الجواب بسرور عظيم، ونظر إلى آخر وقال: "ما هو تعليم الحكمة؟"

\par 9 فأجاب الآخر: «كما أنك لا تريد أن يصيبك شر، بل أن تشارك في كل خير، فكذلك يجب أن تتصرف على نفس المبدأ مع رعيتك والمسيئين، وأن تحذر الشرفاء والصالحين برفق. فإن الله يجذب إليه جميع الناس بلطفه».

\par 10 أثنى عليه الملك وسأل التالي بالترتيب: كيف يُمكن أن يكون صديقًا للبشر؟

\par 11 فأجاب: «بملاحظة أن الجنس البشري يتزايد ويولد مع الكثير من المتاعب والمعاناة الشديدة: لذلك يجب ألا تعاقبهم أو تعذبهم باستخفاف، لأنك تعلم أن حياة البشر تتكون من آلام وعقوبات. لأنه إذا فهمت كل شيء، فستمتلئ بالشفقة، لأن الله أيضًا مثير للشفقة!»

\par 12 تلقى الملك الإجابة باستحسان، وسأل التالي: "ما هو أهم مؤهلات الحكم؟"

\par 13 أجاب: «أن يحفظ الإنسان نفسه من الرشوة، وأن يمارس الاعتدال خلال الجزء الأكبر من حياته، وأن يكرم البر فوق كل شيء، وأن يصادق رجالًا من هذا النوع. فالله أيضًا محب للعدل! وبعد أن أبدى الملك موافقته، قال لآخر: ما هي العلامة الحقيقية للتقوى؟»

\par 14 فأجاب: «أن تدرك أن الله يعمل باستمرار في الكون ويعلم كل شيء، ولا يمكن لأي إنسان يتصرف بظلم أو شر أن يفلت من ملاحظته. فكما أن الله هو فاعل الخير للعالم أجمع، فكذلك يجب عليك أنت أيضًا أن تحذو حذوه وتكون خاليًا من الإساءة!»

\par 15 أشار الملك إلى موافقته وقال لآخر: "ما هو جوهر الملكية؟"

\par 16 فأجاب: «أن يحكم المرء نفسه جيدًا ولا ينجرف وراء الغنى أو الشهرة في رغبات مفرطة أو غير لائقة، فهذه هي الطريقة الصحيحة للحكم إذا تأملت الأمر جيدًا. فكل ما تحتاجه حقًا هو لك، والله خالٍ من الحاجة ورحيم في ذات الوقت. لتكن أفكارك كما يليق بالإنسان، ولا ترغب في أشياء كثيرة، بل فقط ما هو ضروري للحكم!»

\par 17 أثنى عليه الملك وسأل رجلاً آخر: كيف يمكن أن تكون مداولاته للأفضل؟

\par 18 فأجاب: «إذا كان يضع العدل أمامه دائمًا في كل شيء، ويظن أن الظلم يعادل الحرمان من الحياة. فالله يعد دائمًا بأعظم النعم للعادلين!»

\par 19 بعد أن أثنى عليه الملك، سأل التالي: كيف يمكنه أن يتحرر من الأفكار المزعجة في نومه؟

\par 20 فأجاب: "لقد سألتني سؤالاً يصعب جدًا الإجابة عليه، لأننا لا نستطيع أن نجعل ذواتنا الحقيقية تلعب دورًا خلال ساعات النوم، بل تقيدنا فيها خيالات لا يمكن للعقل التحكم فيها. لأن أرواحنا تمتلك الشعور بأنها ترى بالفعل الأشياء التي تدخل إلى وعينا أثناء النوم. لكننا نخطئ إذا افترضنا أننا نبحر في البحر في قوارب أو نطير في الهواء 1 أو نسافر إلى مناطق أخرى أو أي شيء آخر من هذا القبيل. ومع ذلك فإننا نتخيل بالفعل حدوث مثل هذه الأشياء."

\par 21 بقدر ما أستطيع أن أقرر، فقد توصلت إلى الاستنتاج التالي. يجب عليك، أيها الملك، أن تحكم أقوالك وأفعالك بكل طريقة ممكنة بقاعدة التقوى حتى تدرك أنك تحافظ على الفضيلة وأنك لا تختار أبدًا إرضاء نفسك على حساب العقل، ولا تسيء استخدام سلطتك أبدًا للاحتقار للصلاح

\par 22 لأن العقل ينشغل في الغالب أثناء النوم بنفس الأشياء التي ينشغل بها أثناء اليقظة. ومن يوجه جميع أفكاره وأفعاله نحو أسمى الغايات، فإنه يرسخ نفسه في الاستقامة سواءً كان مستيقظًا أو نائمًا. لذلك يجب أن تكون ثابتًا في الانضباط الذاتي المستمر

\par 23 أثنى الملك على الرجل وقال لآخر: "بما أنك العاشر في الإجابة، فعندما تتكلم، سنتفرغ للوليمة." ثم طرح السؤال: كيف يمكنني تجنب فعل أي شيء لا يليق بي؟

\par 24 فأجاب: «انظُر دائمًا إلى شهرتك ومكانتك المرموقة، حتى لا تتحدث وتفكر إلا في الأمور التي تتفق معها، عالمًا أن جميع رعاياك يفكرون ويتحدثون عنك. فلا يجب أن تبدو أسوأ من الممثلين، الذين يدرسون بعناية الدور الذي من الضروري أن يلعبوه، ويشكلون جميع أفعالهم وفقًا له. أنت لست تمثل دورًا، بل أنت ملك حقًا، لأن الله قد منح لك سلطة ملكية تتناسب مع شخصيتك.»

\par 25 عندما صفق الملك بصوت عالٍ ومطولًا بألطف طريقة، حثّ الضيوف على طلب الراحة. لذلك عندما توقف الحديث، انصرفوا إلى الطبق التالي من الوليمة

\par 26 في اليوم التالي، تم اتباع نفس الترتيب، وعندما وجد الملك فرصة لطرح الأسئلة على الرجال، سأل أول من تبقى للاستجواب التالي: ما هو أعلى شكل من أشكال الحكم؟

\par 27 فأجاب: «أن يحكم المرء نفسه ولا ينساق وراء النزوات. فجميع الرجال لديهم ميول طبيعية معينة في العقل. ومن المحتمل أن معظم الرجال لديهم ميل نحو الطعام والشراب والملذات، وأن الملوك لديهم ميل نحو الاستحواذ على الأراضي والشهرة العظيمة. ولكن من الجيد أن يكون هناك اعتدال في كل شيء.»

\par 28 «ما يمنحك إياه الله، يجب أن تأخذه وتحتفظ به، ولكن لا تتوق أبدًا إلى أشياء بعيدة المنال.»

\par 29 سُرَّ الملك بهذه الكلمات، وسأل التالي: كيف يُمكنني التحرر من الحسد؟

\par 30 فأجاب بعد صمت قصير: "إذا فكرتَ أولًا وقبل كل شيء أن الله هو الذي يمنح جميع الملوك المجد والثروة العظيمة، ولا أحد ملك بقوته الخاصة. فجميع الناس يرغبون في مشاركة هذا المجد لكنهم لا يستطيعون، لأنه هبة الله!"

\par 31 «أشاد الملك بالرجل في خطاب طويل ثم سأل آخر: كيف يمكنه أن يحتقر أعداءه؟»

\par 32 فأجاب: «إذا أظهرتَ اللطفَ لجميع الناس وكسبتَ صداقتهم، فلا تخشَ أحدًا. إنَّ محبَّةَ الناسِ أسمى عطايا الله!»

\par 33 بعد أن أشاد الملك بهذه الإجابة، أمر الرجل التالي بالرد على السؤال: كيف يمكنه الحفاظ على شهرته العظيمة؟

\par 34 فأجاب: «إذا كنت كريمًا وواسع القلب في منح اللطف وأعمال النعمة للآخرين، فلن تفقد شهرتك أبدًا، ولكن إذا كنت ترغب في استمرار النعم المذكورة، فعليك أن تدعو الله باستمرار».

\par 35 أعرب الملك عن موافقته وسأل التالي: لمن ينبغي للإنسان أن يُظهر الكرم؟

\par 36 فأجاب: «يُقر جميع الناس بأنه يجب علينا أن نُظهر الكرم لمن يُحسنون إلينا، ولكن أعتقد أنه يجب علينا أن نُظهر نفس روح الكرم الشديد لمن يُعارضوننا حتى نتمكن بهذه الوسيلة من كسبهم إلى الحق وإلى ما هو مُفيد لنا. ولكن يجب أن نصلي إلى الله أن يتم ذلك، لأنه هو الذي يُسيطر على عقول جميع البشر».

\par 37 بعد أن أبدى الملك موافقته على الإجابة، طلب من السادس أن يجيب على السؤال: لمن يجب أن نظهر الامتنان؟

\par 38 فأجاب: «إلى والدينا دائمًا، لأن الله قد أعطانا وصية بالغة الأهمية فيما يتعلق بإكرام الوالدين. ثم ذكر موقف الصديق تجاه الصديق، إذ تحدث عن «الصديق الذي هو مثل نفسك». أحسنتَ صنعًا في محاولتك جعل جميع الناس في صداقة معك.»

\par 39 تحدث الملك معه بلطف ثم سأل التالي: ما الذي يشبه الجمال في القيمة؟

\par 40 فقال: «التقوى، فهي أسمى صور الجمال، وقوتها تكمن في المحبة، وهي هبة الله. لقد اكتسبتِها بالفعل ومعها كل بركات الحياة».

\par 41 صفق الملك للإجابة بأقصى درجات اللطف وسأل آخر: كيف، إذا فشل، يمكنه استعادة سمعته مرة أخرى بنفس الدرجة؟

\par 42 وقال: "ليس من الممكن أن تفشل، لأنك زرعت في جميع الناس بذور الامتنان التي تُنتج حصادًا من حسن النية، وهذا أقوى من أقوى الأسلحة ويضمن أكبر قدر من الأمان. ولكن إذا فشل أي إنسان، فعليه ألا يفعل مرة أخرى تلك الأشياء التي تسببت في فشله، بل يجب عليه تكوين صداقات والتصرف بعدل. لأن القدرة على فعل الخير هي هبة من الله وليس العكس."

\par 43 سُرَّ الملك بهذه الكلمات، فسأل آخر: كيف يُمكنه التحرر من الحزن؟

\par 44 فأجاب: «إن كان لم يؤذِ أحدًا قط، بل أحسن إلى الجميع وسلك طريق البر، فإن ثماره تجلب التحرر من الحزن. ولكن يجب أن ندعو الله ألا تصيبنا شرور غير متوقعة كالموت أو المرض أو الألم أو أي شيء من هذا القبيل ويؤذينا. ولكن بما أنك ملتزم بالتقوى، فلن تصيبك مثل هذه المصيبة أبدًا.»

\par 45 فأثنى عليه الملك مديحاً عظيماً وسأل العاشر: ما هو أعلى المجد؟

\par 46 فقال: «لتكريم الله، وهذا لا يتم بالقرابين والتضحيات، بل بنقاء النفس واليقين المقدس، لأن كل الأشياء تُشكل وتُحكم من قبل الله وفقًا لإرادته. وأنتم في حيازة دائمة لهذا الغرض، كما يستطيع أي رجل، من إنجازاتكم في الماضي والحاضر.»

\par 47 سلم عليهم الملك بصوت عالٍ وتحدث إليهم بلطف، وأعرب جميع الحاضرين عن موافقتهم، وخاصة الفلاسفة. لأنهم كانوا متفوقين عليهم بكثير [أي الفلاسفة] في السلوك والجدل، لأنهم كانوا دائمًا يتخذون الله نقطة انطلاقهم

\par 48 بعد ذلك، شرع الملك في شرب مشروبات صحية لضيوفه، وذلك لإظهار حسن ظنه

\par \textit{الحواشي السفلية}

\par \textit{165:1 كُتب حوالي 150 قبل الميلاد!}

\chapter{9}

الآية ٨ تُجسّد قيمة المعرفة. الآية ٢٨ تُجسّد عطف الوالدين. انتبه تحديدًا للسؤال في الآية ٢٦ وإجابته. انتبه أيضًا للسؤال في الآية ٤٧ وإجابته. هذه نصيحة حكيمة لرجال الأعمال.

\par 1 في اليوم التالي، تم وضع نفس الترتيبات للمأدبة، وبمجرد أن سنحت الفرصة، بدأ الملك في طرح الأسئلة على الرجال الذين جلسوا بجانب أولئك الذين أجابوا بالفعل، وقال للأول: "هل الحكمة قابلة للتعليم؟"

\par 2 وقال: «إن الروح مُكَوَّنة بحيث إنها قادرة بالقدرة الإلهية على استقبال كل الخير ورفض ما هو عكسه».

\par 3 أبدى الملك موافقته وسأل الرجل التالي: ما هو الشيء الأكثر فائدة للصحة؟

\par 4 فقال: «الاعتدال، ولا يمكن اكتسابه إلا إذا خلق الله له خلقًا».

\par 5 تحدث الملك بلطف مع الرجل، وقال لآخر: "كيف يمكن للرجل أن يفي بدين الشكر لوالديه؟"

\par 6 وقال: «بعدم التسبب لهم بالألم أبدًا، وهذا غير ممكن إلا إذا وجّه الله العقل إلى السعي وراء أنبل الغايات».

\par 7 أعرب الملك عن موافقته وسأل التالي: كيف يُمكنني أن أصبح مستمعًا شغوفًا؟

\par 8 وقال: «بتذكر أن كل معرفة مفيدة، لأنها تمكنك بمعونة الله في وقت الطوارئ من اختيار بعض الأشياء التي تعلمتها وتطبيقها على الأزمة التي تواجهك. وهكذا تتحقق جهود البشر بمعونة الله.»

\par 9 أثنى عليه الملك وسأل التالي: كيف يمكنه تجنب فعل أي شيء مخالف للقانون؟

\par 10 فقال: «إن كنتم تعلمون أن الله هو الذي وضع الأفكار في قلوب المشرعين لحفظ أرواح الناس، فسوف تتبعونهم».

\par 11 أقر الملك بإجابة الرجل وقال لآخر: ما فائدة القرابة؟

\par 12 فأجاب: «إذا تبيّن لنا أننا نُبتلى بالمصائب التي تُصيب أقاربنا، وأن معاناتهم تُصيبنا، فإن قوة القرابة تتجلى فورًا، لأنه لا يُظهِر هذا الشعور إلا ما نُكنّ له الاحترام والتقدير في أعينهم. فالمساعدة، إذا اقترنت باللطف، هي في حد ذاتها رباط لا ينفصم. وفي يوم رخائهم، لا ينبغي لنا أن نطمع في ممتلكاتهم، بل أن ندعو الله أن يُنعم عليهم بكل خير».

\par 13 وبعد أن منحه الملك نفس الثناء الذي منحه للبقية، سأل آخر: كيف يمكنه أن ينال التحرر من الخوف؟

\par 14 وقال: «عندما يدرك العقل أنه لم يفعل شرًا، وعندما يوجهه الله إلى جميع النصائح النبيلة».

\par 15 أعرب الملك عن موافقته وسأل آخر: كيف يمكنه دائمًا الحفاظ على حكم صحيح؟

\par 16 فأجاب: «إذا وضع أمام عينيه باستمرار المصائب التي تصيب البشر، وأدرك أن الله هو الذي يسلب الرخاء من البعض، ويجلب للآخرين شرفًا ومجدًا عظيمين».

\par 17 استقبل الملك الرجل استقبالًا كريمًا وطلب من التالي أن يجيب على السؤال: كيف يمكنه تجنب حياة الرخاء والمتعة؟

\par 18 فأجاب: «إذا تذكر باستمرار أنه حاكم إمبراطورية عظيمة وسيد جماهير غفيرة، وأنه لا ينبغي أن ينشغل عقله بأشياء أخرى، بل ينبغي أن يفكر دائمًا في أفضل السبل لتعزيز رفاهيتهم. فعليه أيضًا أن يصلي إلى الله ألا يهمل أي واجب».

\par 19 بعد أن أثنى عليه الملك، سأل العاشر: كيف يمكنه التعرف على أولئك الذين يخونونه؟

\par 20 فأجاب على السؤال: "إذا لاحظ ما إذا كانت تصرفات من حوله طبيعية، وما إذا كانوا يحافظون على قاعدة الأسبقية الصحيحة في حفلات الاستقبال والمجالس، وفي تعاملاتهم العامة، ولا يتجاوزون حدود اللياقة في التهاني أو في غيرها من أمور السلوك. لكن الله سيميل عقلك، أيها الملك، إلى كل ما هو نبيل."

\par 21 عندما أعرب الملك عن موافقته الصاخبة وأثنى عليهم جميعًا فرديًا (وسط تصفيق جميع الحاضرين)، تحولوا إلى الاستمتاع بالوليمة

\par 22 وفي اليوم التالي، عندما سنحت الفرصة، سأل الملك الرجل التالي: ما هو أشد أشكال الإهمال؟

\par 23 فأجاب: «إذا لم يهتم الرجل بأولاده ولم يبذل كل جهد في تعليمهم، فإننا ندعو الله دائمًا ليس لأنفسنا بقدر ما ندعوه لأولادنا أن تكون لهم كل نعمة. إن رغبتنا في أن يمتلك أبناؤنا ضبط النفس لا تتحقق إلا بقدرة الله».

\par 24 قال الملك إنه تحدث جيدًا، ثم سأل آخر: كيف يُمكن أن يكون وطنيًا؟

\par 25 أجاب: «بأن تضع في ذهنك فكرة أن العيش والموت في وطنك خيرٌ لك. فالإقامة في الخارج تجلب الاحتقار للفقراء والعار للأغنياء، كما لو كانوا منفيين لارتكابهم جريمة. إذا أسديتَ معروفًا للجميع، كما تفعل دائمًا، فسيمنحك الله حظوة عند الجميع، وستُحسب وطنيًا».

\par 26 بعد الاستماع إلى هذا الرجل، سأل الملك الرجل التالي في الترتيب: كيف يمكنه العيش بودّ مع زوجته؟

\par 27 فأجاب: «بإدراك أن النساء بطبيعتهن عنيدات ونشيطات في سعيهن وراء رغباتهن، وعرضة للتغيرات المفاجئة في الرأي من خلال التفكير الخاطئ، وأن طبيعتهن ضعيفة في جوهرها. من الضروري التعامل معهن بحكمة وعدم إثارة الصراع. من أجل إدارة ناجحة للحياة، يجب على القائد أن يعرف الهدف الذي يجب أن يوجه مساره نحوه. فقط من خلال استدعاء عون الله يمكن للبشر توجيه مسار الحياة الصحيح في جميع الأوقات.»

\par 28 أعرب الملك عن موافقته وسأل التالي: كيف يمكن أن يكون خاليًا من الخطأ؟

\par 29 فأجاب: «إذا تصرفت دائمًا بتأنٍّ ولم تُصدِّق الافتراءات، بل أثبتَّ بنفسك ما يُقال لك، وحكمتَ برأيك الخاص في الطلبات المُقدَّمة إليك، ونفَّذتَ كل شيء في ضوء رأيكَ، فستكون بريئًا من الخطأ، أيها الملك. لكن معرفة هذه الأشياء وممارستها هي عمل القدرة الإلهية».

\par 30 سُرَّ الملك بهذه الكلمات، فسأل آخر: كيف يُمكنه أن يتحرر من الغضب؟

\par 31 فقال ردًا على السؤال: "إذا اعترف بأن لديه القدرة على الجميع حتى على إيقاع الموت بهم، وإذا استسلم للغضب، وأنه سيكون من غير المجدي والمثير للشفقة أن يحرم الكثيرين من الحياة لمجرد أنه سيد."

\par 32 «ما الحاجة إلى الغضب، عندما كان جميع الناس خاضعين ولم يكن أحد معاديًا له؟ من الضروري أن ندرك أن الله يحكم العالم كله بروح اللطف ودون غضب على الإطلاق، وأنت،» قال، «أيها الملك، يجب عليك بالضرورة أن تحذو حذوه.»

\par 33 قال الملك إنه أجاب جيدًا، ثم سأل الرجل التالي: ما هي المشورة الجيدة؟

\par 34 أوضح قائلًا: "أن نتصرف بشكل جيد في جميع الأوقات وبتفكير مناسب، ونقارن ما هو مفيد لسياستنا بالآثار الضارة التي قد تنتج عن تبني الرأي المعاكس، حتى نتمكن من خلال وزن كل نقطة من الحصول على نصيحة جيدة ويمكن تحقيق هدفنا. والأهم من ذلك كله، بقوة الله، ستتحقق كل خطة من خططكم لأنكم تمارسون التقوى."

\par 35 قال الملك إن هذا الرجل أجاب جيدًا، وسأل آخر: ما هي الفلسفة؟

\par 36 وأوضح قائلاً: «أن نُحسن التدبير في أي مسألة تُطرح، وألا ننجرف وراء النزوات، بل أن نُمعن النظر في الأضرار الناجمة عن الأهواء، وأن نتصرف بما تقتضيه الظروف، مُتَّسمين بالاعتدال. ولكن يجب أن ندعو الله أن يُلهمنا الاهتمام بهذه الأمور».

\par 37 أبدى الملك موافقته وسأل آخر: كيف يُمكنه أن يحظى بالتقدير عند السفر إلى الخارج؟

\par 38 أجاب: «بإنصافه جميع الناس، وبظهوره بمظهر أدنى من أولئك الذين كان يسافر بينهم، لا متفوقًا عليهم. فمن المبادئ المعترف بها أن الله بطبيعته يقبل المتواضعين. والبشر يحبون أولئك الذين هم على استعداد للخضوع لهم.»

\par 39 بعد أن أبدى الملك موافقته على هذا الرد، سأل آخر: كيف يمكنه البناء بطريقة تجعل مبانيه تدوم من بعده؟

\par 40 فأجاب على السؤال: "لو كانت إبداعاته عظيمة ونبيلة، بحيث يحفظها الناظرون لجمالها، ولو لم يطرد أحدًا ممن صنعوا مثل هذه الأعمال، ولم يُجبر الآخرين على خدمته دون أجر."

\par 41 إذ لاحظ كيف يُعيل الله الجنس البشري، مانحًا إياهم الصحة والقدرة العقلية وجميع المواهب الأخرى، ينبغي عليه هو نفسه أن يحذو حذوه من خلال تقديم مكافأة للبشر على تعبهم الشاق. 1 لأن الأعمال التي تُصنع بالبر هي التي تبقى دائمًا!

\par 42 قال الملك إن هذا الرجل أيضًا قد أجاب جيدًا، وسأل العاشر: ما هي ثمرة الحكمة؟

\par 43 فأجاب: «أن يدرك الإنسان في نفسه أنه لم يرتكب شرًا، وأن يعيش حياته بالحق. فمن هؤلاء، أيها الملك العظيم، ينال أعظم فرح وثبات النفس وإيمان قوي بالله إذا حكمت مملكتك بالتقوى».

\par 44 وعندما سمعوا الجواب، هتف الجميع بصوت عالٍ، وبعد ذلك بدأ الملك في ملء فرحه يشرب مشروباتهم الصحية

\par 45 وفي اليوم التالي، سارت المأدبة على نفس النهج الذي سارت عليه في المناسبات السابقة، وعندما سنحت الفرصة، شرع الملك في طرح الأسئلة على الضيوف المتبقين، وقال للأول: "كيف يمكن للإنسان أن يحفظ نفسه من الكبرياء؟"

\par 46 فأجاب: «إذا حافظ على المساواة وتذكر في كل مناسبة أنه رجل يحكم الناس. والله يُبيد المتكبرين، ويرفع الوديع والمتواضع!»

\par 47 تحدث الملك معه بلطف وسأل التالي: من ينبغي للرجل أن يختار مستشاريه؟

\par 48 فأجاب: «أولئك الذين اختبروا في أمور كثيرة، وحافظوا على حسن نية غير مختلط تجاهه، وشاركوا في تصرفاته. والله يتجلى لمن يستحقون لتحقيق هذه الغايات».

\par 49 أثنى عليه الملك وسأل آخر: ما هي أهم ممتلكات الملك؟

\par 50 «صداقة رعيته ومحبتهم»، أجاب، «فبها تدوم رابطة النوايا الحسنة. والله هو الذي يضمن أن يتم ذلك وفقًا لرغبتك».

\par 51 أثنى عليه الملك وسأل آخر: ما هو هدف الكلام؟ فأجاب: «إقناع خصمك بإظهار أخطائه بأسلوب منظم من الحجج».

\par 52 «لأنك بهذه الطريقة تكسب سامعك، ليس بمعارضته، بل بإغداق الثناء عليه بهدف إقناعه. وبقدرة الله يتم الإقناع.»

\par 53 قال الملك إنه أعطى إجابة جيدة، وسأل آخر، كيف يمكنه العيش بود مع الأعراق المختلفة العديدة التي تشكل سكان مملكته؟

\par 54 أجاب: «بتصرفكم بما يليق بكل واحد، واتخاذكم البر دليلاً لكم، كما تفعلون الآن بمساعدة البصيرة التي ينعم الله عليكم بها».

\par 55 سُرّ الملك بهذا الرد، وسأل آخر: "في أي ظروف يجب على الإنسان أن يعاني الحزن؟"

\par 56 أجاب: «في المصائب التي تصيب أصدقائنا، عندما نرى أنها طويلة الأمد ولا يمكن علاجها. لا يسمح لنا العقل بالحزن على أولئك الذين ماتوا وتحرروا من الشر، لكن جميع البشر يحزنون عليهم لأنهم لا يفكرون إلا في أنفسهم ومصلحتهم. فبقدرة الله وحدها يمكننا النجاة من كل شر!»

\par 57 قال الملك إنه أعطى إجابة مناسبة، وسأل آخر: كيف تُفقَد السمعة؟

\par 58 فأجاب: «عندما يسود الكبرياء والثقة الجامحة بالنفس، يتولد العار وفقدان السمعة. لأن الله هو رب كل سمعة، ويهبها حيث يشاء».

\par 59 أكد الملك الإجابة، وسأل الرجل التالي: إلى من ينبغي أن يأتمن الناس أنفسهم؟

\par 60 أجاب: «إلى أولئك الذين يخدمونك بحسن نية، لا بدافع الخوف أو المصلحة الذاتية، ولا يفكرون إلا في مصلحتهم الخاصة. فهؤلاء علامة الحب، والبعض الآخر علامة سوء النية والتقاعس عن العمل.»

\par 61 «لأن الرجل الذي يسعى دائمًا لمصلحته الخاصة هو خائن في قلبه. لكنك تمتلك محبة جميع رعيتك بمساعدة المشورة الصالحة التي يمنحك إياها الله.»

\par 62 قال الملك إنه أجاب بحكمة، وسأل آخر: ما الذي يحافظ على سلامة المملكة؟

\par 63 فأجاب على السؤال قائلًا: «الحرص والتفكير المسبق حتى لا يرتكب أصحاب السلطة على الناس أي شر، وهذا ما تفعله دائمًا بمعونة الله الذي يلهمك حكمًا خطيرًا».

\par 64 خاطبه الملك بكلمات تشجيعية، وسأل آخر: ما الذي يحافظ على الامتنان والشرف؟

\par 65 فأجاب: «الفضيلة، فهي خالقة الأعمال الصالحة، وبها يُباد الشر، كما تُظهِر نبل الأخلاق تجاه الجميع بالهبة التي ينعم بها الله عليك».

\par 66 فأقر الملك بالإجابة بكل لطف وسأل الحادي عشر (بما أن عددهم كان اثنين أكثر من سبعين)، كيف يستطيع في زمن الحرب أن يحافظ على هدوء النفس؟

\par 67 فأجاب: «بتذكرك أنه لم يُلحق أيًا من رعيته ضررًا، وأن الجميع سيقاتلون من أجله مقابل المنافع التي حصلوا عليها، مع علمك أنه حتى لو فقدوا حياتهم، فإنك ستعتني بمن يعولونهم. لأنك لا تقصر أبدًا في تعويض أي شخص - هذه هي لطف القلب الذي ألهمك الله به».

\par 68 صفق الملك لهم جميعًا بصوت عالٍ وتحدث إليهم بلطف شديد، ثم شرب جرعة طويلة من الشراب من أجل صحة كل واحد منهم، منغمسًا في الاستمتاع، وأغدق على ضيوفه أكرم وأسعد صداقة

\par \textit{الحواشي السفلية}

\par \textit{169:1 كان هناك مقيمون أجانب في تلك الأيام أيضًا.}

\par \textit{170:1 إن سياسة الأجر العادل ليوم عمل عادل لا تبدو هنا حديثة كما نعتقد أحيانًا في ما يسعدنا أن نسميه هذا العصر المستنير.}

\chapter{10}

\par \textit{تستمر الأسئلة والأجوبة. توضح كيفية اختيار ضباط الجيش. من هو الرجل الذي يستحق الإعجاب، ومشاكل أخرى في الحياة اليومية لا تزال صحيحة اليوم كما كانت قبل 2000 عام. تتميز الآيات 15-17 بتوصيتها بالمسرح. تصف الآيات 21-22 حكمة انتخاب رئيس أو وجود ملك.}

\par 1 في اليوم السابع، تم إجراء استعدادات أكثر شمولاً، وكان هناك العديد من الأشخاص الآخرين حاضرين من مدن مختلفة (من بينهم عدد كبير من السفراء).

\par 2 عندما سنحت الفرصة، سأل الملك أول من لم يُسأل بعد، كيف يمكنه تجنب الانخداع بالمنطق الخاطئ؟

\par 3 فأجاب: «بملاحظة المتحدث بعناية، والشيء المنطوق، والموضوع قيد المناقشة، وطرح نفس الأسئلة مرة أخرى بعد فترة بأشكال مختلفة. لكن امتلاك عقل يقظ والقدرة على تكوين حكم سليم في كل حالة هو إحدى عطايا الله الصالحة، وأنت تمتلكها أيها الملك.»

\par 4 صفق الملك بصوت عالٍ للإجابة وسأل آخر: لماذا لا يصبح غالبية الرجال فاضلين أبدًا؟

\par 5 أجاب: «لأن جميع البشر بطبيعتهم غير معتدلين ويميلون إلى اللذة. ومن ثم، ينشأ الظلم ويتدفق الجشع. إن عادة الفضيلة عائق أمام أولئك الذين يكرسون أنفسهم لحياة المتعة لأنها تفرض عليهم تفضيل الاعتدال والاستقامة. لأن الله هو سيد هذه الأشياء.»

\par 6 قال الملك إنه أجاب جيدًا، وسأل: ما الذي يجب على الملوك طاعته؟ فقال: "القوانين، حتى يتمكنوا من استعادة حياة البشر من خلال التشريعات الصالحة. كما أنك بمثل هذا السلوك طاعةً للأمر الإلهي، قد ادخرت لنفسك تذكارًا دائمًا."

\par 7 قال الملك إن هذا الرجل أيضًا قد تحدث جيدًا، وسأل التالي: من ينبغي لنا أن نعينه حكامًا؟

\par 8 فأجاب: «كل من يبغض الشر ويقتدي بسيرتك، فليعمل بالصلاح ليحافظ على سمعته الطيبة دائمًا. فهذا ما تفعله أيها الملك العظيم، والله هو الذي وهبك إكليل البر».

\par 9 هتف الملك بصوت عالٍ بالإجابة، ثم نظر إلى الرجل التالي وقال: "من ينبغي لنا أن نعينه ضباطًا على القوات؟"

\par 10 وأوضح قائلًا: «أولئك الذين يتفوقون في الشجاعة والاستقامة، وأولئك الذين يهتمون بسلامة رجالهم أكثر من تحقيق النصر بالمخاطرة بحياتهم بالتهور. فكما أن الله يُحسن إلى جميع الناس، فكذلك أنتم، اقتداءً به، تُحسنون إلى جميع رعيتكم».

\par 11 قال الملك إنه أعطى إجابة جيدة وسأل آخر: أي رجل يستحق الإعجاب؟

\par 12 فأجاب: «الرجل الذي يتمتع بالسمعة والثروة والسلطة، ويملك روحًا تعادل كل ذلك. أنت نفسك تُظهر بأفعالك أنك جدير بالإعجاب بمعونة الله الذي يجعلك تهتم بهذه الأشياء.»

\par 13 أعرب الملك عن موافقته وقال لآخر: "لأي الشؤون ينبغي للملوك أن يكرسوا معظم وقتهم؟"

\par 14 فأجاب: «لقراءة ودراسة سجلات الرحلات الرسمية، المكتوبة فيما يتعلق بالممالك المختلفة، بهدف إصلاح الرعايا والحفاظ عليهم. ومن خلال هذا النشاط وصلتم إلى مجد لم يبلغه أحد من قبل، بعون الله الذي يحقق جميع رغباتكم.»

\par 15 تحدث الملك بحماس إلى الرجل وسأل آخر: كيف ينبغي للإنسان أن يشغل نفسه خلال ساعات الاسترخاء والاستجمام؟

\par 16 فأجاب: «إن مشاهدة تلك المسرحيات التي يمكن تمثيلها بشكل لائق، ووضع مشاهد مأخوذة من الحياة أمام أعيننا، وممثلة بكرامة ونزاهة، أمر مفيد ومناسب».

\par 17 «لأنه يمكن العثور على بعض التثقيف حتى في هذه التسلية، فغالبًا ما تُعلّمنا أبسط أمور الحياة درسًا مرغوبًا. ولكن بممارسة أقصى درجات اللياقة في جميع أفعالك، فقد أظهرت أنك فيلسوف وأن الله يكرمك بسبب فضيلتك.»

\par 18 سُرَّ الملك بالكلام الذي قيل للتو، فقال للرجل التاسع: كيف ينبغي للإنسان أن يتصرف في الولائم؟

\par 19 فأجاب: «يجب عليك أن تستدعي رجال العلم والقادرين على إعطائك نصائح مفيدة فيما يتعلق بشؤون مملكتك وحياة رعيتك (فإنك لم تجد موضوعًا أنسب أو أكثر تعليمًا من هذا)، فإن هؤلاء الرجال أعزاء على الله لأنهم دربوا عقولهم على التأمل في أنبل المواضيع - كما تفعل أنت بنفسك، لأن جميع أفعالك موجهة من الله».

\par 20 سُرَّ الملك بالجواب، فسأل الرجل التالي: ما هو الأفضل للشعب؟ أن يُنصَّب مواطنٌ ملكًا عليهم، أم أن يُنصَّب أحد أفراد العائلة المالكة؟

\par 21 فأجاب: «من هو الأفضل بطبيعته. فالملوك الذين ينحدرون من سلالة ملكية غالبًا ما يكونون قساة وصارمين تجاه رعاياهم. وهذا ينطبق أكثر على بعض الذين ارتقوا من صفوف المواطنين العاديين، الذين بعد أن عانوا الشر وتحملوا نصيبهم من الفقر، عندما يحكمون جموعًا، يتبين أنهم أكثر قسوة من الطغاة الملحدين.»

\par 22 «ولكن، كما قلتُ، فإن الطبيعة الصالحة التي تم تدريبها بشكل صحيح قادرة على الحكم، وأنت ملك عظيم، ليس لأنك تتفوق في مجد حكمك وثروتك، بل لأنك تفوقت على جميع الرجال في الرأفة والإحسان، بفضل الله الذي وهبك هذه الصفات.»

\par 23 أمضى الملك بعض الوقت في مدح هذا الرجل ثم سأل الأخير: ما هو أعظم إنجاز في حكم إمبراطورية؟

\par 24 فأجاب: «أن يعيش الرعايا في سلام دائم، وأن يُطبق العدل بسرعة في حالات النزاع».

\par 25 «تتحقق هذه النتائج من خلال تأثير الحاكم، عندما يكون رجلاً يكره الشر ويحب الخير ويكرس طاقاته لإنقاذ أرواح الناس، تمامًا كما تعتبر الظلم أسوأ أشكال الشر، وبفضل إدارتك العادلة، صنعت لنفسك سمعة خالدة، لأن الله يمنحك عقلًا نقيًا لا يشوبه أي شر.»

\par 26 وعندما توقف، اندلع تصفيق حار ومبهج لبعض الوقت. وعندما توقف، أخذ الملك كأسًا وألقى نخبًا تكريمًا لجميع ضيوفه والكلمات التي نطقوا بها

\par 27 ثم قال في الختام: لقد استفدت كثيرًا من وجودك. لقد استفدت كثيرًا من النصائح الحكيمة التي قدمتها لي فيما يتعلق بفن الحكم

\par 28 ثم أمر بتقديم ثلاث وزنات من الفضة لكل واحد منهم، وعيّن أحد عبيده لتسليم المال

\par 29 هتف الجميع بالموافقة على الفور، وأصبحت الوليمة مشهدًا من الفرح، بينما استسلم الملك لجولة متواصلة من الاحتفالات

\chapter{11}

\par \textit{للحصول على تعليق على الاختزال القديم، انظر الآية 7. تُقدم الترجمة للموافقة عليها وتُقبل كما هي مقروءة، و(الآية 23) يُجرى تصويت بالموافقة ويُنفذ بالإجماع.}

\par 1 لقد كتبتُ مطولاً، وأرجو عفوك يا فيلوكراتس.

\par 2 لقد أذهلني الرجال إلى حد لا يمكن قياسه والطريقة التي أعطوا بها الإجابات في اللحظة الراهنة والتي كانت في الواقع تحتاج إلى وقت طويل لوضعها.

\par 3 فبالرغم من أن السائل قد فكر ملياً في كل سؤال على حدة، فإن أولئك الذين أجابوا واحداً تلو الآخر كانت إجاباتهم على الأسئلة جاهزة على الفور، ولذلك بدا لي ولكل من حضر وخاصة الفلاسفة أنهم يستحقون الإعجاب.

\par 4 وأفترض أن الأمر سيبدو مذهلاً لأولئك الذين سيقرأون روايتي في المستقبل

\par 5 ولكن من غير اللائق تحريف الحقائق المسجلة في الأرشيفات العامة

\par 6 ولن يكون من الصواب لي أن أتجاوز في أمر كهذا. أروي القصة كما حدثت تمامًا، متجنبًا أي خطأ بضمير حي

\par 7 لقد أُعجبتُ بشدة بقوة أقوالهم، لدرجة أنني بذلتُ جهدًا لاستشارة أولئك الذين كان من واجبهم تسجيل كل ما حدث في المقابلات الملكية والمآدب

\par 8 فمن المعتاد، كما تعلمون، من اللحظة التي يبدأ فيها الملك بإنجاز أعماله حتى وقت تقاعده للراحة، أن يُسجل جميع أقواله وأفعاله - وهو ترتيب ممتاز ومفيد للغاية

\par 9 ففي اليوم التالي، تُقرأ محاضر أعمال وأقوال اليوم السابق قبل بدء العمل، وإذا كان هناك أي مخالفة، يُصحح الأمر على الفور

\par 10 لذلك، كما قيل، حصلت على معلومات دقيقة من السجلات العامة، وقد عرضت الحقائق بالترتيب الصحيح لأنني أعرف مدى حرصكم على الحصول على معلومات مفيدة

\par 11 بعد ثلاثة أيام، أخذ ديمتريوس الرجال، وعبر على طول جدار البحر، بطول سبعة استاديات، إلى الجزيرة، وعبر الجسر واتجه إلى المناطق الشمالية من فاروس

\par 12 هناك جمعهم في منزل بُني على شاطئ البحر، في غاية الجمال وفي مكان منعزل، ودعاهم للقيام بعمل الترجمة، حيث وُضع تحت تصرفهم كل ما يحتاجونه لهذا الغرض

\par 13 فشرعوا في مقارنة نتائجهم المختلفة وجعلها متوافقة، وكل ما اتفقوا عليه تم نسخه بشكل مناسب تحت إشراف ديمتريوس

\par 14 واستمرت الجلسة حتى الساعة التاسعة، وبعد ذلك أُطلق سراحهم لخدمة احتياجاتهم الجسدية

\par 15 تم توفير كل ما أرادوه لهم على نطاق واسع. بالإضافة إلى ذلك، قام دوروثيوس بنفس الاستعدادات اليومية التي كانت تُجرى للملك نفسه - لأنه كان قد أمر بذلك من قبل الملك

\par 16 في الصباح الباكر، كانوا يظهرون يوميًا في البلاط، وبعد تحية الملك، يعودون إلى مكانهم

\par 17 وكما هي عادة جميع اليهود، غسلوا أيديهم في البحر وصلوا إلى الله ثم انصرفوا إلى قراءة وترجمة المقطع الذي كانوا مشغولين به، وطرحت عليهم السؤال: لماذا غسلوا أيديهم قبل أن يصلوا؟

\par 18 وأوضحوا أن ذلك كان علامة على أنهم لم يفعلوا شرًا (لأن كل شكل من أشكال النشاط يتم بواسطة الأيدي) لأنهم في طريقهم النبيل والمقدس يعتبرون كل شيء رمزًا للبر والحقيقة.

\par 19 كما قلتُ سابقًا، كانوا يجتمعون يوميًا في المكان الذي كان ساحرًا بهدوئه وإشراقه، وينكبّون على مهمتهم

\par 20 وتصادف أن عمل الترجمة قد اكتمل في اثنين وسبعين يومًا، كما لو كان قد تم ترتيبه لغرض محدد

\par 21 ولما اكتمل العمل، جمع ديمتريوس السكان اليهود في المكان الذي تمت فيه الترجمة، وقرأها على الجميع، بحضور المترجمين، الذين لاقوا أيضًا استقبالًا كبيرًا من الشعب، بسبب الفوائد العظيمة التي أنعموا بها عليهم

\par 22 كما أثنوا على ديمتريوس بشدة، وحثوه على نسخ القانون بأكمله وتقديم نسخة منه إلى قادتهم

\par 23 بعد قراءة الكتب، وقف الكهنة وشيوخ المترجمين والجماعة اليهودية وزعماء الشعب وقالوا إنه بما أن هذه الترجمة ممتازة ومقدسة ودقيقة، فمن الصواب أن تبقى كما هي ولا يُجرى عليها أي تغيير

\par 24 وعندما أعربت المجموعة بأكملها عن موافقتها، أمروا بإصدار لعنة، وفقًا لعاداتهم، على أي شخص يُجري أي تغيير، سواءً بإضافة أي شيء أو بتغيير أي كلمة من الكلمات التي كُتبت أو بحذف أي شيء

\par 25 كان هذا احتياطًا حكيمًا للغاية لضمان الحفاظ على الكتاب دون تغيير طوال الوقت في المستقبل

\par 26 عندما أُبلغ الملك بالأمر، فرح فرحًا عظيمًا، لأنه شعر أن التصميم الذي وضعه قد نُفِّذ بأمان

\par 27 قُرئ عليه الكتاب بأكمله، فاندهش بشدة من روح المشرع

\par 28 وقال لديمتريوس: "كيف لم يعتقد أي من المؤرخين أو الشعراء أن الإشارة إلى مثل هذا الإنجاز الرائع تستحق العناء؟"

\par 29 فأجاب: «لأن الشريعة مقدسة ومصدرها إلهي. وبعض الذين عزموا على التعامل معها قد أصابهم الله فعدلوا عن ذلك».

\par 30 قال إنه سمع من ثيوبومبوس أنه فقد عقله لأكثر من ثلاثين يومًا لأنه كان ينوي أن يُدرج في تاريخه بعض الحوادث من الترجمات السابقة وغير الموثوقة إلى حد ما للقانون

\par 31 عندما تعافى قليلًا، توسل إلى الله أن يوضح له سبب المصيبة التي حلت به

\par 32 وأُوحي إليه في المنام أنه من باب الفضول العاطل كان يرغب في توصيل الحقائق المقدسة لعامة الناس، وأنه إذا كف عن ذلك فسوف يستعيد عافيته

\par 33 سمعت أيضًا من شفتي ثيوديكتس، أحد الشعراء المأساويين، أنه عندما كان على وشك تكييف بعض الأحداث المسجلة في الكتاب لإحدى مسرحياته، أصيب بإعتام عدسة العين في كلتا عينيه.

\par 34 ولما أدرك سبب المصيبة التي أصابته، صلى إلى الله أيامًا عديدة، ثم شُفي بعد ذلك

\par 35 وبعد أن تلقى الملك، كما قلتُ سابقًا، شرح ديمتريوس لهذه النقطة، قدم ولاءه وأمر بالعناية الفائقة بالكتب، وأن تُحفظ بقداسة

\par 36 وحثّ المترجمين على زيارته كثيرًا بعد عودتهم إلى اليهودية، لأنه قال إنه من الصواب أن يرسلهم الآن إلى أوطانهم

\par 37 ولكن عندما يعودون، سيعاملهم كأصدقاء، كما هو الصواب، وسيتلقون منه هدايا ثمينة

\par 38 أمر بالاستعدادات اللازمة لعودتهم إلى ديارهم، وعاملهم بسخاء بالغ

\par 39 أهدى لكل واحد منهم ثلاثة أثواب من أجود الأنواع، ووزنتين من الذهب، وخزانة جانبية وزنها وزنة واحدة، وجميع أثاث ثلاث أرائك

\par 40 وأرسل مع الحرس إلى ألعازار عشرة أسرّة بأرجل من فضة، وكل ما يلزم من أدوات، وخزانة جانبية بقيمة ثلاثين وزنة، وعشرة أثواب أرجوانية، وتاجًا فاخرًا، ومئة قطعة من أجود أنواع الكتان المنسوج، وأوعية وأطباقًا، وكأسين من ذهب لتقديمهما لله

\par 41 وحثه أيضًا في رسالة على ألا يعيق أيًا من الرجال إذا فضل العودة إليه

\par 42 لأنه كان يعد الاستمتاع بصحبة هؤلاء الرجال المتعلمين امتيازًا كبيرًا، وكان يفضل أن يبذر ثروته عليهم بدلاً من أن يبذرها على الغرور

\par 43 والآن يا فيلوكراتس، لديك القصة الكاملة وفقًا لوعدي

\par 44 أعتقد أنك تجد متعة أكبر في هذه الأمور من كتابات علماء الأساطير

\par 45 لأنك مُكرس لدراسة تلك الأشياء التي يمكن أن تُفيد الروح، وتُقضي وقتًا طويلاً في ذلك. سأحاول سرد أي أحداث أخرى تستحق التسجيل، حتى تتمكن من خلال قراءتها من الحصول على أعلى مكافأة لحماستك


\end{document}