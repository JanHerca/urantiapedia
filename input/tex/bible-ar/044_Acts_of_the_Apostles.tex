\begin{document}

\title{اعمال}


\chapter{1}

\par 1 اَلْكَلاَمُ الأَوَّلُ أَنْشَأْتُهُ يَا ثَاوُفِيلُسُ عَنْ جَمِيعِ مَا ابْتَدَأَ يَسُوعُ يَفْعَلُهُ وَيُعَلِّمُ بِهِ
\par 2 إِلَى الْيَوْمِ الَّذِي ارْتَفَعَ فِيهِ بَعْدَ مَا أَوْصَى بِالرُّوحِ الْقُدُسِ الرُّسُلَ الَّذِينَ اخْتَارَهُمْ.
\par 3 اَلَّذِينَ أَرَاهُمْ أَيْضاً نَفْسَهُ حَيّاً بِبَرَاهِينَ كَثِيرَةٍ بَعْدَ مَا تَأَلَّمَ وَهُوَ يَظْهَرُ لَهُمْ أَرْبَعِينَ يَوْماً وَيَتَكَلَّمُ عَنِ الْأُمُورِ الْمُخْتَصَّةِ بِمَلَكُوتِ اللهِ.
\par 4 وَفِيمَا هُوَ مُجْتَمِعٌ مَعَهُمْ أَوْصَاهُمْ أَنْ لاَ يَبْرَحُوا مِنْ أُورُشَلِيمَ بَلْ يَنْتَظِرُوا «مَوْعِدَ الآبِ الَّذِي سَمِعْتُمُوهُ مِنِّي
\par 5 لأَنَّ يُوحَنَّا عَمَّدَ بِالْمَاءِ وَأَمَّا أَنْتُمْ فَسَتَتَعَمَّدُونَ بِالرُّوحِ الْقُدُسِ لَيْسَ بَعْدَ هَذِهِ الأَيَّامِ بِكَثِيرٍ».
\par 6 أَمَّا هُمُ الْمُجْتَمِعُونَ فَسَأَلُوهُ: «يَا رَبُّ هَلْ فِي هَذَا الْوَقْتِ تَرُدُّ الْمُلْكَ إِلَى إِسْرَائِيلَ؟»
\par 7 فَقَالَ لَهُمْ: «لَيْسَ لَكُمْ أَنْ تَعْرِفُوا الأَزْمِنَةَ وَالأَوْقَاتَ الَّتِي جَعَلَهَا الآبُ فِي سُلْطَانِهِ
\par 8 لَكِنَّكُمْ سَتَنَالُونَ قُوَّةً مَتَى حَلَّ الرُّوحُ الْقُدُسُ عَلَيْكُمْ وَتَكُونُونَ لِي شُهُوداً فِي أُورُشَلِيمَ وَفِي كُلِّ الْيَهُودِيَّةِ وَالسَّامِرَةِ وَإِلَى أَقْصَى الأَرْضِ».
\par 9 وَلَمَّا قَالَ هَذَا ارْتَفَعَ وَهُمْ يَنْظُرُونَ وَأَخَذَتْهُ سَحَابَةٌ عَنْ أَعْيُنِهِمْ.
\par 10 وَفِيمَا كَانُوا يَشْخَصُونَ إِلَى السَّمَاءِ وَهُوَ مُنْطَلِقٌ إِذَا رَجُلاَنِ قَدْ وَقَفَا بِهِمْ بِلِبَاسٍ أَبْيَضَ
\par 11 وَقَالاَ: «أَيُّهَا الرِّجَالُ الْجَلِيلِيُّونَ مَا بَالُكُمْ وَاقِفِينَ تَنْظُرُونَ إِلَى السَّمَاءِ؟ إِنَّ يَسُوعَ هَذَا الَّذِي ارْتَفَعَ عَنْكُمْ إِلَى السَّمَاءِ سَيَأْتِي هَكَذَا كَمَا رَأَيْتُمُوهُ مُنْطَلِقاً إِلَى السَّمَاءِ».
\par 12 حِينَئِذٍ رَجَعُوا إِلَى أُورُشَلِيمَ مِنَ الْجَبَلِ الَّذِي يُدْعَى جَبَلَ الزَّيْتُونِ الَّذِي هُوَ بِالْقُرْبِ مِنْ أُورُشَلِيمَ عَلَى سَفَرِ سَبْتٍ.
\par 13 وَلَمَّا دَخَلُوا صَعِدُوا إِلَى الْعِلِّيَّةِ الَّتِي كَانُوا يُقِيمُونَ فِيهَا: بُطْرُسُ وَيَعْقُوبُ وَيُوحَنَّا وَأَنْدَرَاوُسُ وَفِيلُبُّسُ وَتُومَا وَبَرْثُولَمَاوُسُ وَمَتَّى وَيَعْقُوبُ بْنُ حَلْفَى وَسِمْعَانُ الْغَيُورُ وَيَهُوذَا بْنُ يَعْقُوبَ.
\par 14 هَؤُلاَءِ كُلُّهُمْ كَانُوا يُواظِبُونَ بِنَفْسٍ وَاحِدَةٍ عَلَى الصَّلاَةِ وَالطِّلْبَةِ مَعَ النِّسَاءِ وَمَرْيَمَ أُمِّ يَسُوعَ وَمَعَ إِخْوَتِهِ.
\par 15 وَفِي تِلْكَ الأَيَّامِ قَامَ بُطْرُسُ فِي وَسَطِ التَّلاَمِيذِ وَكَانَ عِدَّةُ أَسْمَاءٍ مَعاً نَحْوَ مِئَةٍ وَعِشْرِينَ. فَقَالَ:
\par 16 «أَيُّهَا الرِّجَالُ الإِخْوَةُ كَانَ يَنْبَغِي أَنْ يَتِمَّ هَذَا الْمَكْتُوبُ الَّذِي سَبَقَ الرُّوحُ الْقُدُسُ فَقَالَهُ بِفَمِ دَاوُدَ عَنْ يَهُوذَا الَّذِي صَارَ دَلِيلاً لِلَّذِينَ قَبَضُوا عَلَى يَسُوعَ
\par 17 إِذْ كَانَ مَعْدُوداً بَيْنَنَا وَصَارَ لَهُ نَصِيبٌ فِي هَذِهِ الْخِدْمَةِ.
\par 18 فَإِنَّ هَذَا اقْتَنَى حَقْلاً مِنْ أُجْرَةِ الظُّلْمِ وَإِذْ سَقَطَ عَلَى وَجْهِهِ انْشَقَّ مِنَ الْوَسَطِ فَانْسَكَبَتْ أَحْشَاؤُهُ كُلُّهَا.
\par 19 وَصَارَ ذَلِكَ مَعْلُوماً عِنْدَ جَمِيعِ سُكَّانِ أُورُشَلِيمَ حَتَّى دُعِيَ ذَلِكَ الْحَقْلُ فِي لُغَتِهِمْ «حَقْلَ دَمَا» (أَيْ: حَقْلَ دَمٍ).
\par 20 لأَنَّهُ مَكْتُوبٌ فِي سِفْرِ الْمَزَامِيرِ: لِتَصِرْ دَارُهُ خَرَاباً وَلاَ يَكُنْ فِيهَا سَاكِنٌ وَلْيَأْخُذْ وَظِيفَتَهُ آَخَرُ.
\par 21 فَيَنْبَغِي أَنَّ الرِّجَالَ الَّذِينَ اجْتَمَعُوا مَعَنَا كُلَّ الزَّمَانِ الَّذِي فِيهِ دَخَلَ إِلَيْنَا الرَّبُّ يَسُوعُ وَخَرَجَ
\par 22 مُنْذُ مَعْمُودِيَّةِ يُوحَنَّا إِلَى الْيَوْمِ الَّذِي ارْتَفَعَ فِيهِ عَنَّا يَصِيرُ وَاحِدٌ مِنْهُمْ شَاهِداً مَعَنَا بِقِيَامَتِهِ».
\par 23 فَأَقَامُوا اثْنَيْنِ: يُوسُفَ الَّذِي يُدْعَى بَارْسَابَا الْمُلَقَّبَ يُوسْتُسَ وَمَتِّيَاسَ.
\par 24 وَصَلَّوْا قَائِلِينَ: «أَيُّهَا الرَّبُّ الْعَارِفُ قُلُوبَ الْجَمِيعِ عَيِّنْ أَنْتَ مِنْ هَذَيْنِ الاِثْنَيْنِ أَيّاً اخْتَرْتَهُ
\par 25 لِيَأْخُذَ قُرْعَةَ هَذِهِ الْخِدْمَةِ وَالرِّسَالَةِ الَّتِي تَعَدَّاهَا يَهُوذَا لِيَذْهَبَ إِلَى مَكَانِهِ».
\par 26 ثُمَّ أَلْقَوْا قُرْعَتَهُمْ فَوَقَعَتِ الْقُرْعَةُ عَلَى مَتِّيَاسَ فَحُسِبَ مَعَ الأَحَدَ عَشَرَ رَسُولاً.

\chapter{2}

\par 1 وَلَمَّا حَضَرَ يَوْمُ الْخَمْسِينَ كَانَ الْجَمِيعُ مَعاً بِنَفْسٍ وَاحِدَةٍ
\par 2 وَصَارَ بَغْتَةً مِنَ السَّمَاءِ صَوْتٌ كَمَا مِنْ هُبُوبِ رِيحٍ عَاصِفَةٍ وَمَلأَ كُلَّ الْبَيْتِ حَيْثُ كَانُوا جَالِسِينَ
\par 3 وَظَهَرَتْ لَهُمْ أَلْسِنَةٌ مُنْقَسِمَةٌ كَأَنَّهَا مِنْ نَارٍ وَاسْتَقَرَّتْ عَلَى كُلِّ وَاحِدٍ مِنْهُمْ.
\par 4 وَامْتَلأَ الْجَمِيعُ مِنَ الرُّوحِ الْقُدُسِ وَابْتَدَأُوا يَتَكَلَّمُونَ بِأَلْسِنَةٍ أُخْرَى كَمَا أَعْطَاهُمُ الرُّوحُ أَنْ يَنْطِقُوا.
\par 5 وَكَانَ يَهُودٌ رِجَالٌ أَتْقِيَاءُ مِنْ كُلِّ أُمَّةٍ تَحْتَ السَّمَاءِ سَاكِنِينَ فِي أُورُشَلِيمَ.
\par 6 فَلَمَّا صَارَ هَذَا الصَّوْتُ اجْتَمَعَ الْجُمْهُورُ وَتَحَيَّرُوا لأَنَّ كُلَّ وَاحِدٍ كَانَ يَسْمَعُهُمْ يَتَكَلَّمُونَ بِلُغَتِهِ.
\par 7 فَبُهِتَ الْجَمِيعُ وَتَعَجَّبُوا قَائِلِينَ بَعْضُهُمْ لِبَعْضٍ: «أَتُرَى لَيْسَ جَمِيعُ هَؤُلاَءِ الْمُتَكَلِّمِينَ جَلِيلِيِّينَ؟
\par 8 فَكَيْفَ نَسْمَعُ نَحْنُ كُلُّ وَاحِدٍ مِنَّا لُغَتَهُ الَّتِي وُلِدَ فِيهَا:
\par 9 فَرْتِيُّونَ وَمَادِيُّونَ وَعِيلاَمِيُّونَ وَالسَّاكِنُونَ مَا بَيْنَ النَّهْرَيْنِ وَالْيَهُودِيَّةَ وَكَبَّدُوكِيَّةَ وَبُنْتُسَ وَأَسِيَّا
\par 10 وَفَرِيجِيَّةَ وَبَمْفِيلِيَّةَ وَمِصْرَ وَنَوَاحِيَ لِيبِيَّةَ الَّتِي نَحْوَ الْقَيْرَوَانِ وَالرُّومَانِيُّونَ الْمُسْتَوْطِنُونَ يَهُودٌ وَدُخَلاَءُ
\par 11 كِرِيتِيُّونَ وَعَرَبٌ نَسْمَعُهُمْ يَتَكَلَّمُونَ بِأَلْسِنَتِنَا بِعَظَائِمِ اللهِ؟».
\par 12 فَتَحَيَّرَ الْجَمِيعُ وَارْتَابُوا قَائلِينَ بَعْضُهُمْ لِبَعْضٍ: «مَا عَسَى أَنْ يَكُونَ هَذَا؟».
\par 13 وَكَانَ آخَرُونَ يَسْتَهْزِئُونَ قَائِلِينَ: «إِنَّهُمْ قَدِ امْتَلأُوا سُلاَفَةً».
\par 14 فَوَقَفَ بُطْرُسُ مَعَ الأَحَدَ عَشَرَ وَرَفَعَ صَوْتَهُ وَقَالَ لَهُمْ: «أَيُّهَا الرِّجَالُ الْيَهُودُ وَالسَّاكِنُونَ فِي أُورُشَلِيمَ أَجْمَعُونَ لِيَكُنْ هَذَا مَعْلُوماً عِنْدَكُمْ وَأَصْغُوا إِلَى كَلاَمِي
\par 15 لأَنَّ هَؤُلاَءِ لَيْسُوا سُكَارَى كَمَا أَنْتُمْ تَظُنُّونَ لأَنَّهَا السَّاعَةُ الثَّالِثَةُ مِنَ النَّهَارِ.
\par 16 بَلْ هَذَا مَا قِيلَ بِيُوئِيلَ النَّبِيِّ.
\par 17 يَقُولُ اللهُ: وَيَكُونُ فِي الأَيَّامِ الأَخِيرَةِ أَنِّي أَسْكُبُ مِنْ رُوحِي عَلَى كُلِّ بَشَرٍ فَيَتَنَبَّأُ بَنُوكُمْ وَبَنَاتُكُمْ وَيَرَى شَبَابُكُمْ رُؤىً وَيَحْلُمُ شُيُوخُكُمْ أَحْلاَماً.
\par 18 وَعَلَى عَبِيدِي أَيْضاً وَإِمَائِي أَسْكُبُ مِنْ رُوحِي فِي تِلْكَ الأَيَّامِ فَيَتَنَبَّأُونَ.
\par 19 وَأُعْطِي عَجَائِبَ فِي السَّمَاءِ مِنْ فَوْقُ وَآيَاتٍ عَلَى الأَرْضِ مِنْ أَسْفَلُ: دَماً وَنَاراً وَبُخَارَ دُخَانٍ.
\par 20 تَتَحَوَّلُ الشَّمْسُ إِلَى ظُلْمَةٍ وَالْقَمَرُ إِلَى دَمٍ قَبْلَ أَنْ يَجِيءَ يَوْمُ الرَّبِّ الْعَظِيمُ الشَّهِيرُ.
\par 21 وَيَكُونُ كُلُّ مَنْ يَدْعُو بِاسْمِ الرَّبِّ يَخْلُصُ».
\par 22 «أَيُّهَا الرِّجَالُ الإِسْرَائِيلِيُّونَ اسْمَعُوا هَذِهِ الأَقْوَالَ: يَسُوعُ النَّاصِرِيُّ رَجُلٌ قَدْ تَبَرْهَنَ لَكُمْ مِنْ قِبَلِ اللهِ بِقُوَّاتٍ وَعَجَائِبَ وَآيَاتٍ صَنَعَهَا اللهُ بِيَدِهِ فِي وَسَطِكُمْ كَمَا أَنْتُمْ أَيْضاً تَعْلَمُونَ.
\par 23 هَذَا أَخَذْتُمُوهُ مُسَلَّماً بِمَشُورَةِ اللهِ الْمَحْتُومَةِ وَعِلْمِهِ السَّابِقِ وَبِأَيْدِي أَثَمَةٍ صَلَبْتُمُوهُ وَقَتَلْتُمُوهُ.
\par 24 اَلَّذِي أَقَامَهُ اللهُ نَاقِضاً أَوْجَاعَ الْمَوْتِ إِذْ لَمْ يَكُنْ مُمْكِناً أَنْ يُمْسَكَ مِنْهُ.
\par 25 لأَنَّ دَاوُدَ يَقُولُ فِيهِ: كُنْتُ أَرَى الرَّبَّ أَمَامِي فِي كُلِّ حِينٍ أَنَّهُ عَنْ يَمِينِي لِكَيْ لاَ أَتَزَعْزَعَ.
\par 26 لِذَلِكَ سُرَّ قَلْبِي وَتَهَلَّلَ لِسَانِي. حَتَّى جَسَدِي أَيْضاً سَيَسْكُنُ عَلَى رَجَاءٍ.
\par 27 لأَنَّكَ لَنْ تَتْرُكَ نَفْسِي فِي الْهَاوِيَةِ وَلاَ تَدَعَ قُدُّوسَكَ يَرَى فَسَاداً.
\par 28 عَرَّفْتَنِي سُبُلَ الْحَيَاةِ وَسَتَمْلأُنِي سُرُوراً مَعَ وَجْهِكَ.
\par 29 أَيُّهَا الرِّجَالُ الإِخْوَةُ يَسُوغُ أَنْ يُقَالَ لَكُمْ جِهَاراً عَنْ رَئِيسِ الآبَاءِ دَاوُدَ إِنَّهُ مَاتَ وَدُفِنَ وَقَبْرُهُ عِنْدَنَا حَتَّى هَذَا الْيَوْمِ.
\par 30 فَإِذْ كَانَ نَبِيّاً وَعَلِمَ أَنَّ اللهَ حَلَفَ لَهُ بِقَسَمٍ أَنَّهُ مِنْ ثَمَرَةِ صُلْبِهِ يُقِيمُ الْمَسِيحَ حَسَبَ الْجَسَدِ لِيَجْلِسَ عَلَى كُرْسِيِّهِ
\par 31 سَبَقَ فَرَأَى وَتَكَلَّمَ عَنْ قِيَامَةِ الْمَسِيحِ أَنَّهُ لَمْ تُتْرَكْ نَفْسُهُ فِي الْهَاوِيَةِ وَلاَ رَأَى جَسَدُهُ فَسَاداً.
\par 32 فَيَسُوعُ هَذَا أَقَامَهُ اللهُ وَنَحْنُ جَمِيعاً شُهُودٌ لِذَلِكَ.
\par 33 وَإِذِ ارْتَفَعَ بِيَمِينِ اللهِ وَأَخَذَ مَوْعِدَ الرُّوحِ الْقُدُسِ مِنَ الآبِ سَكَبَ هَذَا الَّذِي أَنْتُمُ الآنَ تُبْصِرُونَهُ وَتَسْمَعُونَهُ.
\par 34 لأَنَّ دَاوُدَ لَمْ يَصْعَدْ إِلَى السَّمَاوَاتِ. وَهُوَ نَفْسُهُ يَقُولُ: قَالَ الرَّبُّ لِرَبِّي اجْلِسْ عَنْ يَمِينِي
\par 35 حَتَّى أَضَعَ أَعْدَاءَكَ مَوْطِئاً لِقَدَمَيْكَ.
\par 36 فَلْيَعْلَمْ يَقِيناً جَمِيعُ بَيْتِ إِسْرَائِيلَ أَنَّ اللهَ جَعَلَ يَسُوعَ هَذَا الَّذِي صَلَبْتُمُوهُ أَنْتُمْ رَبّاً وَمَسِيحاً».
\par 37 فَلَمَّا سَمِعُوا نُخِسُوا فِي قُلُوبِهِمْ وَسَأَلُوا بُطْرُسَ وَسَائِرَ الرُّسُلِ: «مَاذَا نَصْنَعُ أَيُّهَا الرِّجَالُ الإِخْوَةُ؟»
\par 38 فَقَالَ لَهُمْ بُطْرُسُ: «تُوبُوا وَلْيَعْتَمِدْ كُلُّ وَاحِدٍ مِنْكُمْ عَلَى اسْمِ يَسُوعَ الْمَسِيحِ لِغُفْرَانِ الْخَطَايَا فَتَقْبَلُوا عَطِيَّةَ الرُّوحِ الْقُدُسِ.
\par 39 لأَنَّ الْمَوْعِدَ هُوَ لَكُمْ وَلأَوْلاَدِكُمْ وَلِكُلِّ الَّذِينَ عَلَى بُعْدٍ كُلِّ مَنْ يَدْعُوهُ الرَّبُّ إِلَهُنَا».
\par 40 وَبِأَقْوَالٍ أُخَرَ كَثِيرَةٍ كَانَ يَشْهَدُ لَهُمْ وَيَعِظُهُمْ قَائِلاً: «اخْلُصُوا مِنْ هَذَا الْجِيلِ الْمُلْتَوِي».
\par 41 فَقَبِلُوا كَلاَمَهُ بِفَرَحٍ وَاعْتَمَدُوا وَانْضَمَّ فِي ذَلِكَ الْيَوْمِ نَحْوُ ثَلاَثَةِ آلاَفِ نَفْسٍ.
\par 42 وَكَانُوا يُواظِبُونَ عَلَى تَعْلِيمِ الرُّسُلِ وَالشَّرِكَةِ وَكَسْرِ الْخُبْزِ وَالصَّلَوَاتِ.
\par 43 وَصَارَ خَوْفٌ فِي كُلِّ نَفْسٍ. وَكَانَتْ عَجَائِبُ وَآيَاتٌ كَثِيرَةٌ تُجْرَى عَلَى أَيْدِي الرُّسُلِ.
\par 44 وَجَمِيعُ الَّذِينَ آمَنُوا كَانُوا مَعاً وَكَانَ عِنْدَهُمْ كُلُّ شَيْءٍ مُشْتَرَكاً.
\par 45 وَالأَمْلاَكُ وَالْمُقْتَنَيَاتُ كَانُوا يَبِيعُونَهَا وَيَقْسِمُونَهَا بَيْنَ الْجَمِيعِ كَمَا يَكُونُ لِكُلِّ وَاحِدٍ احْتِيَاجٌ.
\par 46 وَكَانُوا كُلَّ يَوْمٍ يُواظِبُونَ فِي الْهَيْكَلِ بِنَفْسٍ وَاحِدَةٍ. وَإِذْ هُمْ يَكْسِرُونَ الْخُبْزَ فِي الْبُيُوتِ كَانُوا يَتَنَاوَلُونَ الطَّعَامَ بِابْتِهَاجٍ وَبَسَاطَةِ قَلْبٍ
\par 47 مُسَبِّحِينَ اللهَ وَلَهُمْ نِعْمَةٌ لَدَى جَمِيعِ الشَّعْبِ. وَكَانَ الرَّبُّ كُلَّ يَوْمٍ يَضُمُّ إِلَى الْكَنِيسَةِ الَّذِينَ يَخْلُصُونَ.

\chapter{3}

\par 1 وَصَعِدَ بُطْرُسُ وَيُوحَنَّا مَعاً إِلَى الْهَيْكَلِ فِي سَاعَةِ الصَّلاَةِ التَّاسِعَةِ.
\par 2 وَكَانَ رَجُلٌ أَعْرَجُ مِنْ بَطْنِ أُمِّهِ يُحْمَلُ كَانُوا يَضَعُونَهُ كُلَّ يَوْمٍ عِنْدَ بَابِ الْهَيْكَلِ الَّذِي يُقَالُ لَهُ «الْجَمِيلُ» لِيَسْأَلَ صَدَقَةً مِنَ الَّذِينَ يَدْخُلُونَ الْهَيْكَلَ.
\par 3 فَهَذَا لَمَّا رَأَى بُطْرُسَ وَيُوحَنَّا مُزْمِعَيْنِ أَنْ يَدْخُلاَ الْهَيْكَلَ سَأَلَ لِيَأْخُذَ صَدَقَةً.
\par 4 فَتَفَرَّسَ فِيهِ بُطْرُسُ مَعَ يُوحَنَّا وَقَالَ: «انْظُرْ إِلَيْنَا!»
\par 5 فَلاَحَظَهُمَا مُنْتَظِراً أَنْ يَأْخُذَ مِنْهُمَا شَيْئاً.
\par 6 فَقَالَ بُطْرُسُ: «لَيْسَ لِي فِضَّةٌ وَلاَ ذَهَبٌ وَلَكِنِ الَّذِي لِي فَإِيَّاهُ أُعْطِيكَ: بِاسْمِ يَسُوعَ الْمَسِيحِ النَّاصِرِيِّ قُمْ وَامْشِ».
\par 7 وَأَمْسَكَهُ بِيَدِهِ الْيُمْنَى وَأَقَامَهُ فَفِي الْحَالِ تَشَدَّدَتْ رِجْلاَهُ وَكَعْبَاهُ
\par 8 فَوَثَبَ وَوَقَفَ وَصَارَ يَمْشِي وَدَخَلَ مَعَهُمَا إِلَى الْهَيْكَلِ وَهُوَ يَمْشِي وَيَطْفُرُ وَيُسَبِّحُ اللهَ
\par 9 وَأَبْصَرَهُ جَمِيعُ الشَّعْبِ وَهُوَ يَمْشِي وَيُسَبِّحُ اللهَ.
\par 10 وَعَرَفُوهُ أَنَّهُ هُوَ الَّذِي كَانَ يَجْلِسُ لأَجْلِ الصَّدَقَةِ عَلَى بَابِ الْهَيْكَلِ الْجَمِيلِ فَامْتَلأُوا دَهْشَةً وَحَيْرَةً مِمَّا حَدَثَ لَهُ.
\par 11 وَبَيْنَمَا كَانَ الرَّجُلُ الأَعْرَجُ الَّذِي شُفِيَ مُتَمَسِّكاً بِبُطْرُسَ وَيُوحَنَّا تَرَاكَضَ إِلَيْهِمْ جَمِيعُ الشَّعْبِ إِلَى الرِّوَاقِ الَّذِي يُقَالُ لَهُ «رِوَاقُ سُلَيْمَانَ» وَهُمْ مُنْدَهِشُونَ.
\par 12 فَلَمَّا رَأَى بُطْرُسُ ذَلِكَ قَالَ لِلْشَّعْبَ: «أَيُّهَا الرِّجَالُ الإِسْرَائِيلِيُّونَ مَا بَالُكُمْ تَتَعَجَّبُونَ مِنْ هَذَا وَلِمَاذَا تَشْخَصُونَ إِلَيْنَا كَأَنَّنَا بِقُوَّتِنَا أَوْ تَقْوَانَا قَدْ جَعَلْنَا هَذَا يَمْشِي؟
\par 13 إِنَّ إِلَهَ إِبْرَاهِيمَ وَإِسْحَاقَ وَيَعْقُوبَ إِلَهَ آبَائِنَا مَجَّدَ فَتَاهُ يَسُوعَ الَّذِي أَسْلَمْتُمُوهُ أَنْتُمْ وَأَنْكَرْتُمُوهُ أَمَامَ وَجْهِ بِيلاَطُسَ وَهُوَ حَاكِمٌ بِإِطْلاَقِهِ.
\par 14 وَلَكِنْ أَنْتُمْ أَنْكَرْتُمُ الْقُدُّوسَ الْبَارَّ وَطَلَبْتُمْ أَنْ يُوهَبَ لَكُمْ رَجُلٌ قَاتِلٌ.
\par 15 وَرَئِيسُ الْحَيَاةِ قَتَلْتُمُوهُ الَّذِي أَقَامَهُ اللهُ مِنَ الأَمْوَاتِ وَنَحْنُ شُهُودٌ لِذَلِكَ.
\par 16 وَبِالإِيمَانِ بِاسْمِهِ شَدَّدَ اسْمُهُ هَذَا الَّذِي تَنْظُرُونَهُ وَتَعْرِفُونَهُ وَالإِيمَانُ الَّذِي بِوَاسِطَتِهِ أَعْطَاهُ هَذِهِ الصِّحَّةَ أَمَامَ جَمِيعِكُمْ.
\par 17 «وَالآنَ أَيُّهَا الإِخْوَةُ أَنَا أَعْلَمُ أَنَّكُمْ بِجَهَالَةٍ عَمِلْتُمْ كَمَا رُؤَسَاؤُكُمْ أَيْضاً.
\par 18 وَأَمَّا اللهُ فَمَا سَبَقَ وَأَنْبَأَ بِهِ بِأَفْوَاهِ جَمِيعِ أَنْبِيَائِهِ أَنْ يَتَأَلَّمَ الْمَسِيحُ قَدْ تَمَّمَهُ هَكَذَا.
\par 19 فَتُوبُوا وَارْجِعُوا لِتُمْحَى خَطَايَاكُمْ لِكَيْ تَأْتِيَ أَوْقَاتُ الْفَرَجِ مِنْ وَجْهِ الرَّبِّ.
\par 20 وَيُرْسِلَ يَسُوعَ الْمَسِيحَ الْمُبَشَّرَ بِهِ لَكُمْ قَبْلُ.
\par 21 الَّذِي يَنْبَغِي أَنَّ السَّمَاءَ تَقْبَلُهُ إِلَى أَزْمِنَةِ رَدِّ كُلِّ شَيْءٍ الَّتِي تَكَلَّمَ عَنْهَا اللهُ بِفَمِ جَمِيعِ أَنْبِيَائِهِ الْقِدِّيسِينَ مُنْذُ الدَّهْرِ.
\par 22 فَإِنَّ مُوسَى قَالَ لِلآبَاءِ: إِنَّ نَبِيّاً مِثْلِي سَيُقِيمُ لَكُمُ الرَّبُّ إِلَهُكُمْ مِنْ إِخْوَتِكُمْ. لَهُ تَسْمَعُونَ فِي كُلِّ مَا يُكَلِّمُكُمْ بِهِ.
\par 23 وَيَكُونُ أَنَّ كُلَّ نَفْسٍ لاَ تَسْمَعُ لِذَلِكَ النَّبِيِّ تُبَادُ مِنَ الشَّعْبِ.
\par 24 وَجَمِيعُ الأَنْبِيَاءِ أَيْضاً مِنْ صَمُوئِيلَ فَمَا بَعْدَهُ جَمِيعُ الَّذِينَ تَكَلَّمُوا سَبَقُوا وَأَنْبَأُوا بِهَذِهِ الأَيَّامِ.
\par 25 أَنْتُمْ أَبْنَاءُ الأَنْبِيَاءِ وَالْعَهْدِ الَّذِي عَاهَدَ بِهِ اللهُ آبَاءَنَا قَائِلاً لِإِبْراهِيمَ: وَبِنَسْلِكَ تَتَبَارَكُ جَمِيعُ قَبَائِلِ الأَرْضِ.
\par 26 إِلَيْكُمْ أَوَّلاً إِذْ أَقَامَ اللهُ فَتَاهُ يَسُوعَ أَرْسَلَهُ يُبَارِكُكُمْ بِرَدِّ كُلِّ وَاحِدٍ مِنْكُمْ عَنْ شُرُورِهِ».

\chapter{4}

\par 1 وَبَيْنَمَا هُمَا يُخَاطِبَانِ الشَّعْبَ أَقْبَلَ عَلَيْهِمَا الْكَهَنَةُ وَقَائِدُ جُنْدِ الْهَيْكَلِ وَالصَّدُّوقِيُّونَ
\par 2 مُتَضَجِّرِينَ مِنْ تَعْلِيمِهِمَا الشَّعْبَ وَنِدَائِهِمَا فِي يَسُوعَ بِالْقِيَامَةِ مِنَ الأَمْوَاتِ.
\par 3 فَأَلْقَوْا عَلَيْهِمَا الأَيَادِيَ وَوَضَعُوهُمَا فِي حَبْسٍ إِلَى الْغَدِ لأَنَّهُ كَانَ قَدْ صَارَ الْمَسَاءُ.
\par 4 وَكَثِيرُونَ مِنَ الَّذِينَ سَمِعُوا الْكَلِمَةَ آمَنُوا وَصَارَ عَدَدُ الرِّجَالِ نَحْوَ خَمْسَةِ آلاَفٍ.
\par 5 وَحَدَثَ فِي الْغَدِ أَنَّ رُؤَسَاءَهُمْ وَشُيُوخَهُمْ وَكَتَبَتَهُمُ اجْتَمَعُوا إِلَى أُورُشَلِيمَ
\par 6 مَعَ حَنَّانَ رَئِيسِ الْكَهَنَةِ وَقَيَافَا وَيُوحَنَّا وَالإِسْكَنْدَرِ وَجَمِيعِ الَّذِينَ كَانُوا مِنْ عَشِيرَةِ رُؤَسَاءِ الْكَهَنَةِ.
\par 7 وَلَمَّا أَقَامُوهُمَا فِي الْوَسَطِ جَعَلُوا يَسْأَلُونَهُمَا: «بِأَيَّةِ قُوَّةٍ وَبِأَيِّ اسْمٍ صَنَعْتُمَا أَنْتُمَا هَذَا؟»
\par 8 حِينَئِذٍ امْتَلأَ بُطْرُسُ مِنَ الرُّوحِ الْقُدُسِ وَقَالَ لَهُمْ: «يَا رُؤَسَاءَ الشَّعْبِ وَشُيُوخَ إِسْرَائِيلَ
\par 9 إِنْ كُنَّا نُفْحَصُ الْيَوْمَ عَنْ إِحْسَانٍ إِلَى إِنْسَانٍ سَقِيمٍ بِمَاذَا شُفِيَ هَذَا
\par 10 فَلْيَكُنْ مَعْلُوماً عِنْدَ جَمِيعِكُمْ وَجَمِيعِ شَعْبِ إِسْرَائِيلَ أَنَّهُ بِاسْمِ يَسُوعَ الْمَسِيحِ النَّاصِرِيِّ الَّذِي صَلَبْتُمُوهُ أَنْتُمُ الَّذِي أَقَامَهُ اللهُ مِنَ الأَمْوَاتِ بِذَاكَ وَقَفَ هَذَا أَمَامَكُمْ صَحِيحاً.
\par 11 هَذَا هُوَ الْحَجَرُ الَّذِي احْتَقَرْتُمُوهُ أَيُّهَا الْبَنَّاؤُونَ الَّذِي صَارَ رَأْسَ الزَّاوِيَةِ.
\par 12 وَلَيْسَ بِأَحَدٍ غَيْرِهِ الْخَلاَصُ. لأَنْ لَيْسَ اسْمٌ آخَرُ تَحْتَ السَّمَاءِ قَدْ أُعْطِيَ بَيْنَ النَّاسِ بِهِ يَنْبَغِي أَنْ نَخْلُصَ».
\par 13 فَلَمَّا رَأَوْا مُجَاهَرَةَ بُطْرُسَ وَيُوحَنَّا وَوَجَدُوا أَنَّهُمَا إِنْسَانَانِ عَدِيمَا الْعِلْمِ وَعَامِّيَّانِ تَعَجَّبُوا. فَعَرَفُوهُمَا أَنَّهُمَا كَانَا مَعَ يَسُوعَ.
\par 14 وَلَكِنْ إِذْ نَظَرُوا الإِنْسَانَ الَّذِي شُفِيَ وَاقِفاً مَعَهُمَا لَمْ يَكُنْ لَهُمْ شَيْءٌ يُنَاقِضُونَ بِهِ.
\par 15 فَأَمَرُوهُمَا أَنْ يَخْرُجَا إِلَى خَارِجِ الْمَجْمَعِ وَتَآمَرُوا فِيمَا بَيْنَهُمْ
\par 16 قَائِلِينَ: «مَاذَا نَفْعَلُ بِهَذَيْنِ الرَّجُلَيْنِ؟ لأَنَّهُ ظَاهِرٌ لِجَمِيعِ سُكَّانِ أُورُشَلِيمَ أَنَّ آيَةً مَعْلُومَةً قَدْ جَرَتْ بِأَيْدِيهِمَا وَلاَ نَقْدِرُ أَنْ نُنْكِرَ.
\par 17 وَلَكِنْ لِئَلاَّ تَشِيعَ أَكْثَرَ فِي الشَّعْبِ لِنُهَدِّدْهُمَا تَهْدِيداً أَنْ لاَ يُكَلِّمَا أَحَداً مِنَ النَّاسِ فِيمَا بَعْدُ بِهَذَا الاِسْمِ».
\par 18 فَدَعُوهُمَا وَأَوْصُوهُمَا أَنْ لاَ يَنْطِقَا الْبَتَّةَ وَلاَ يُعَلِّمَا بِاسْمِ يَسُوعَ.
\par 19 فَأَجَابَهُمْ بُطْرُسُ وَيُوحَنَّا: «إِنْ كَانَ حَقّاً أَمَامَ اللهِ أَنْ نَسْمَعَ لَكُمْ أَكْثَرَ مِنَ اللهِ فَاحْكُمُوا.
\par 20 لأَنَّنَا نَحْنُ لاَ يُمْكِنُنَا أَنْ لاَ نَتَكَلَّمَ بِمَا رَأَيْنَا وَسَمِعْنَا».
\par 21 وَبَعْدَمَا هَدَّدُوهُمَا أَيْضاً أَطْلَقُوهُمَا إِذْ لَمْ يَجِدُوا الْبَتَّةَ كَيْفَ يُعَاقِبُونَهُمَا بِسَبَبِ الشَّعْبِ لأَنَّ الْجَمِيعَ كَانُوا يُمَجِّدُونَ اللهَ عَلَى مَا جَرَى
\par 22 لأَنَّ الإِنْسَانَ الَّذِي صَارَتْ فِيهِ آيَةُ الشِّفَاءِ هَذِهِ كَانَ لَهُ أَكْثَرُ مِنْ أَرْبَعِينَ سَنَةً.
\par 23 وَلَمَّا أُطْلِقَا أَتَيَا إِلَى رُفَقَائِهِمَا وَأَخْبَرَاهُمْ بِكُلِّ مَا قَالَهُ لَهُمَا رُؤَسَاءُ الْكَهَنَةِ وَالشُّيُوخُ.
\par 24 فَلَمَّا سَمِعُوا رَفَعُوا بِنَفْسٍ وَاحِدَةٍ صَوْتاً إِلَى اللهِ وَقَالُوا: «أَيُّهَا السَّيِّدُ أَنْتَ هُوَ الإِلَهُ الصَّانِعُ السَّمَاءَ وَالأَرْضَ وَالْبَحْرَ وَكُلَّ مَا فِيهَا
\par 25 الْقَائِلُ بِفَمِ دَاوُدَ فَتَاكَ: لِمَاذَا ارْتَجَّتِ الْأُمَمُ وَتَفَكَّرَ الشُّعُوبُ بِالْبَاطِلِ؟
\par 26 قَامَتْ مُلُوكُ الأَرْضِ وَاجْتَمَعَ الرُّؤَسَاءُ مَعاً عَلَى الرَّبِّ وَعَلَى مَسِيحِهِ.
\par 27 لأَنَّهُ بِالْحَقِيقَةِ اجْتَمَعَ عَلَى فَتَاكَ الْقُدُّوسِ يَسُوعَ الَّذِي مَسَحْتَهُ هِيرُودُسُ وَبِيلاَطُسُ الْبُنْطِيُّ مَعَ أُمَمٍ وَشُعُوبِ إِسْرَائِيلَ
\par 28 لِيَفْعَلُوا كُلَّ مَا سَبَقَتْ فَعَيَّنَتْ يَدُكَ وَمَشُورَتُكَ أَنْ يَكُونَ.
\par 29 وَالآنَ يَا رَبُّ انْظُرْ إِلَى تَهْدِيدَاتِهِمْ وَامْنَحْ عَبِيدَكَ أَنْ يَتَكَلَّمُوا بِكَلاَمِكَ بِكُلِّ مُجَاهَرَةٍ
\par 30 بِمَدِّ يَدِكَ لِلشِّفَاءِ وَلْتُجْرَ آيَاتٌ وَعَجَائِبُ بِاسْمِ فَتَاكَ الْقُدُّوسِ يَسُوعَ».
\par 31 وَلَمَّا صَلَّوْا تَزَعْزَعَ الْمَكَانُ الَّذِي كَانُوا مُجْتَمِعِينَ فِيهِ وَامْتَلأَ الْجَمِيعُ مِنَ الرُّوحِ الْقُدُسِ وَكَانُوا يَتَكَلَّمُونَ بِكَلاَمِ اللهِ بِمُجَاهَرَةٍ.
\par 32 وَكَانَ لِجُمْهُورِ الَّذِينَ آمَنُوا قَلْبٌ وَاحِدٌ وَنَفْسٌ وَاحِدَةٌ وَلَمْ يَكُنْ أَحَدٌ يَقُولُ إِنَّ شَيْئاً مِنْ أَمْوَالِهِ لَهُ بَلْ كَانَ عِنْدَهُمْ كُلُّ شَيْءٍ مُشْتَرَكاً.
\par 33 وَبِقُوَّةٍ عَظِيمَةٍ كَانَ الرُّسُلُ يُؤَدُّونَ الشَّهَادَةَ بِقِيَامَةِ الرَّبِّ يَسُوعَ وَنِعْمَةٌ عَظِيمَةٌ كَانَتْ عَلَى جَمِيعِهِمْ
\par 34 إِذْ لَمْ يَكُنْ فِيهِمْ أَحَدٌ مُحْتَاجاً لأَنَّ كُلَّ الَّذِينَ كَانُوا أَصْحَابَ حُقُولٍ أَوْ بُيُوتٍ كَانُوا يَبِيعُونَهَا وَيَأْتُونَ بِأَثْمَانِ الْمَبِيعَاتِ
\par 35 وَيَضَعُونَهَا عِنْدَ أَرْجُلِ الرُّسُلِ فَكَانَ يُوزَّعُ عَلَى كُلِّ أَحَدٍ كَمَا يَكُونُ لَهُ احْتِيَاجٌ.
\par 36 وَيُوسُفُ الَّذِي دُعِيَ مِنَ الرُّسُلِ بَرْنَابَا الَّذِي يُتَرْجَمُ ابْنَ الْوَعْظِ وَهُوَ لاَوِيٌّ قُبْرُسِيُّ الْجِنْسِ
\par 37 إِذْ كَانَ لَهُ حَقْلٌ بَاعَهُ وَأَتَى بِالدَّرَاهِمِ وَوَضَعَهَا عِنْدَ أَرْجُلِ الرُّسُلِ.

\chapter{5}

\par 1 وَرَجُلٌ اسْمُهُ حَنَانِيَّا وَامْرَأَتُهُ سَفِّيرَةُ بَاعَ مُلْكاً
\par 2 وَاخْتَلَسَ مِنَ الثَّمَنِ وَامْرَأَتُهُ لَهَا خَبَرُ ذَلِكَ وَأَتَى بِجُزْءٍ وَوَضَعَهُ عِنْدَ أَرْجُلِ الرُّسُلِ.
\par 3 فَقَالَ بُطْرُسُ: «يَا حَنَانِيَّا لِمَاذَا مَلأَ الشَّيْطَانُ قَلْبَكَ لِتَكْذِبَ عَلَى الرُّوحِ الْقُدُسِ وَتَخْتَلِسَ مِنْ ثَمَنِ الْحَقْلِ؟
\par 4 أَلَيْسَ وَهُوَ بَاقٍ كَانَ يَبْقَى لَكَ؟ وَلَمَّا بِيعَ أَلَمْ يَكُنْ فِي سُلْطَانِكَ؟ فَمَا بَالُكَ وَضَعْتَ فِي قَلْبِكَ هَذَا الأَمْرَ؟ أَنْتَ لَمْ تَكْذِبْ عَلَى النَّاسِ بَلْ عَلَى اللهِ».
\par 5 فَلَمَّا سَمِعَ حَنَانِيَّا هَذَا الْكَلاَمَ وَقَعَ وَمَاتَ. وَصَارَ خَوْفٌ عَظِيمٌ عَلَى جَمِيعِ الَّذِينَ سَمِعُوا بِذَلِكَ.
\par 6 فَنَهَضَ الأَحْدَاثُ وَلَفُّوهُ وَحَمَلُوهُ خَارِجاً وَدَفَنُوهُ.
\par 7 ثُمَّ حَدَثَ بَعْدَ مُدَّةِ نَحْوِ ثَلاَثِ سَاعَاتٍ أَنَّ امْرَأَتَهُ دَخَلَتْ وَلَيْسَ لَهَا خَبَرُ مَا جَرَى.
\par 8 فَسَأَلَهَا بُطْرُسُ: «قُولِي لِي أَبِهَذَا الْمِقْدَارِ بِعْتُمَا الْحَقْلَ؟» فَقَالَتْ: «نَعَمْ بِهَذَا الْمِقْدَارِ».
\par 9 فَقَالَ لَهَا بُطْرُسُ: «مَا بَالُكُمَا اتَّفَقْتُمَا عَلَى تَجْرِبَةِ رُوحِ الرَّبِّ؟ هُوَذَا أَرْجُلُ الَّذِينَ دَفَنُوا رَجُلَكِ عَلَى الْبَابِ وَسَيَحْمِلُونَكِ خَارِجاً».
\par 10 فَوَقَعَتْ فِي الْحَالِ عِنْدَ رِجْلَيْهِ وَمَاتَتْ. فَدَخَلَ الشَّبَابُ وَوَجَدُوهَا مَيْتَةً فَحَمَلُوهَا خَارِجاً وَدَفَنُوهَا بِجَانِبِ رَجُلِهَا.
\par 11 فَصَارَ خَوْفٌ عَظِيمٌ عَلَى جَمِيعِ الْكَنِيسَةِ وَعَلَى جَمِيعِ الَّذِينَ سَمِعُوا بِذَلِكَ.
\par 12 وَجَرَتْ عَلَى أَيْدِي الرُّسُلِ آيَاتٌ وَعَجَائِبُ كَثِيرَةٌ فِي الشَّعْبِ. وَكَانَ الْجَمِيعُ بِنَفْسٍ وَاحِدَةٍ فِي رِوَاقِ سُلَيْمَانَ.
\par 13 وَأَمَّا الآخَرُونَ فَلَمْ يَكُنْ أَحَدٌ مِنْهُمْ يَجْسُرُ أَنْ يَلْتَصِقَ بِهِمْ لَكِنْ كَانَ الشَّعْبُ يُعَظِّمُهُمْ.
\par 14 وَكَانَ مُؤْمِنُونَ يَنْضَمُّونَ لِلرَّبِّ أَكْثَرَ جَمَاهِيرُ مِنْ رِجَالٍ وَنِسَاءٍ
\par 15 حَتَّى إِنَّهُمْ كَانُوا يَحْمِلُونَ الْمَرْضَى خَارِجاً فِي الشَّوَارِعِ وَيَضَعُونَهُمْ عَلَى فُرُشٍ وَأَسِرَّةٍ حَتَّى إِذَا جَاءَ بُطْرُسُ يُخَيِّمُ وَلَوْ ظِلُّهُ عَلَى أَحَدٍ مِنْهُمْ.
\par 16 وَاجْتَمَعَ جُمْهُورُ الْمُدُنِ الْمُحِيطَةِ إِلَى أُورُشَلِيمَ حَامِلِينَ مَرْضَى وَمُعَذَّبِينَ مِنْ أَرْوَاحٍ نَجِسَةٍ وَكَانُوا يُبْرَأُونَ جَمِيعُهُمْ.
\par 17 فَقَامَ رَئِيسُ الْكَهَنَةِ وَجَمِيعُ الَّذِينَ مَعَهُ الَّذِينَ هُمْ شِيعَةُ الصَّدُّوقِيِّينَ وَامْتَلأُوا غَيْرَةً
\par 18 فَأَلْقَوْا أَيْدِيَهُمْ عَلَى الرُّسُلِ وَوَضَعُوهُمْ فِي حَبْسِ الْعَامَّةِ.
\par 19 وَلَكِنَّ مَلاَكَ الرَّبِّ فِي اللَّيْلِ فَتَحَ أَبْوَابَ السِّجْنِ وَأَخْرَجَهُمْ وَقَالَ:
\par 20 «اذْهَبُوا قِفُوا وَكَلِّمُوا الشَّعْبَ فِي الْهَيْكَلِ بِجَمِيعِ كَلاَمِ هَذِهِ الْحَيَاةِ».
\par 21 فَلَمَّا سَمِعُوا دَخَلُوا الْهَيْكَلَ نَحْوَ الصُّبْحِ وَجَعَلُوا يُعَلِّمُونَ. ثُمَّ جَاءَ رَئِيسُ الْكَهَنَةِ وَالَّذِينَ مَعَهُ وَدَعَوُا الْمَجْمَعَ وَكُلَّ مَشْيَخَةِ بَنِي إِسْرَائِيلَ فَأَرْسَلُوا إِلَى الْحَبْسِ لِيُؤْتَى بِهِمْ.
\par 22 وَلَكِنَّ الْخُدَّامَ لَمَّا جَاءُوا لَمْ يَجِدُوهُمْ فِي السِّجْنِ فَرَجَعُوا وَأَخْبَرُوا
\par 23 قَائِلِينَ: «إِنَّنَا وَجَدْنَا الْحَبْسَ مُغْلَقاً بِكُلِّ حِرْصٍ وَالْحُرَّاسَ وَاقِفِينَ خَارِجاً أَمَامَ الأَبْوَابِ وَلَكِنْ لَمَّا فَتَحْنَا لَمْ نَجِدْ فِي الدَّاخِلِ أَحَداً».
\par 24 فَلَمَّا سَمِعَ الْكَاهِنُ وَقَائِدُ جُنْدِ الْهَيْكَلِ وَرُؤَسَاءُ الْكَهَنَةِ هَذِهِ الأَقْوَالَ ارْتَابُوا مِنْ جِهَتِهِمْ: مَا عَسَى أَنْ يَصِيرَ هَذَا؟
\par 25 ثُمَّ جَاءَ وَاحِدٌ وَأَخْبَرَهُمْ قَائِلاً: «هُوَذَا الرِّجَالُ الَّذِينَ وَضَعْتُمُوهُمْ فِي السِّجْنِ هُمْ فِي الْهَيْكَلِ وَاقِفِينَ يُعَلِّمُونَ الشَّعْبَ».
\par 26 حِينَئِذٍ مَضَى قَائِدُ الْجُنْدِ مَعَ الْخُدَّامِ فَأَحْضَرَهُمْ لاَ بِعُنْفٍ لأَنَّهُمْ كَانُوا يَخَافُونَ الشَّعْبَ لِئَلاَّ يُرْجَمُوا.
\par 27 فَلَمَّا أَحْضَرُوهُمْ أَوْقَفُوهُمْ فِي الْمَجْمَعِ. فَسَأَلَهُمْ رَئِيسُ الْكَهَنَةِ:
\par 28 «أَمَا أَوْصَيْنَاكُمْ وَصِيَّةً أَنْ لاَ تُعَلِّمُوا بِهَذَا الاِسْمِ؟ وَهَا أَنْتُمْ قَدْ مَلَأْتُمْ أُورُشَلِيمَ بِتَعْلِيمِكُمْ وَتُرِيدُونَ أَنْ تَجْلِبُوا عَلَيْنَا دَمَ هَذَا الإِنْسَانِ».
\par 29 فَأَجَابَ بُطْرُسُ وَالرُّسُلُ: «يَنْبَغِي أَنْ يُطَاعَ اللهُ أَكْثَرَ مِنَ النَّاسِ.
\par 30 إِلَهُ آبَائِنَا أَقَامَ يَسُوعَ الَّذِي أَنْتُمْ قَتَلْتُمُوهُ مُعَلِّقِينَ إِيَّاهُ عَلَى خَشَبَةٍ.
\par 31 هَذَا رَفَّعَهُ اللهُ بِيَمِينِهِ رَئِيساً وَمُخَلِّصاً لِيُعْطِيَ إِسْرَائِيلَ التَّوْبَةَ وَغُفْرَانَ الْخَطَايَا.
\par 32 وَنَحْنُ شُهُودٌ لَهُ بِهَذِهِ الْأُمُورِ وَالرُّوحُ الْقُدُسُ أَيْضاً الَّذِي أَعْطَاهُ اللهُ لِلَّذِينَ يُطِيعُونَهُ».
\par 33 فَلَمَّا سَمِعُوا حَنِقُوا وَجَعَلُوا يَتَشَاوَرُونَ أَنْ يَقْتُلُوهُمْ.
\par 34 فَقَامَ فِي الْمَجْمَعِ رَجُلٌ فَرِّيسِيٌّ اسْمُهُ غَمَالاَئِيلُ مُعَلِّمٌ لِلنَّامُوسِ مُكَرَّمٌ عِنْدَ جَمِيعِ الشَّعْبِ وَأَمَرَ أَنْ يُخْرَجَ الرُّسُلُ قَلِيلاً.
\par 35 ثُمَّ قَالَ لَهُمْ: « أَيُّهَا الرِّجَالُ الإِسْرَائِيلِيُّونَ احْتَرِزُوا لأَنْفُسِكُمْ مِنْ جِهَةِ هَؤُلاَءِ النَّاسِ فِي مَا أَنْتُمْ مُزْمِعُونَ أَنْ تَفْعَلُوا.
\par 36 لأَنَّهُ قَبْلَ هَذِهِ الأَيَّامِ قَامَ ثُودَاسُ قَائِلاً عَنْ نَفْسِهِ إِنَّهُ شَيْءٌ الَّذِي الْتَصَقَ بِهِ عَدَدٌ مِنَ الرِّجَالِ نَحْوُ أَرْبَعِمِئَةٍ الَّذِي قُتِلَ وَجَمِيعُ الَّذِينَ انْقَادُوا إِلَيْهِ تَبَدَّدُوا وَصَارُوا لاَ شَيْءَ.
\par 37 بَعْدَ هَذَا قَامَ يَهُوذَا الْجَلِيلِيُّ فِي أَيَّامِ الاِكْتِتَابِ وَأَزَاغَ وَرَاءَهُ شَعْباً غَفِيراً. فَذَاكَ أَيْضاً هَلَكَ وَجَمِيعُ الَّذِينَ انْقَادُوا إِلَيْهِ تَشَتَّتُوا.
\par 38 وَالآنَ أَقُولُ لَكُمْ: تَنَحَّوْا عَنْ هَؤُلاَءِ النَّاسِ وَاتْرُكُوهُمْ! لأَنَّهُ إِنْ كَانَ هَذَا الرَّأْيُ أَوْ هَذَا الْعَمَلُ مِنَ النَّاسِ فَسَوْفَ يَنْتَقِضُ
\par 39 وَإِنْ كَانَ مِنَ اللهِ فَلاَ تَقْدِرُونَ أَنْ تَنْقُضُوهُ لِئَلاَّ تُوجَدُوا مُحَارِبِينَ لِلَّهِ أَيْضاً».
\par 40 فَانْقَادُوا إِلَيْهِ. وَدَعُوا الرُّسُلَ وَجَلَدُوهُمْ وَأَوْصُوهُمْ أَنْ لاَ يَتَكَلَّمُوا بِاسْمِ يَسُوعَ ثُمَّ أَطْلَقُوهُمْ.
\par 41 وَأَمَّا هُمْ فَذَهَبُوا فَرِحِينَ مِنْ أَمَامِ الْمَجْمَعِ لأَنَّهُمْ حُسِبُوا مُسْتَأْهِلِينَ أَنْ يُهَانُوا مِنْ أَجْلِ اسْمِهِ.
\par 42 وَكَانُوا لاَ يَزَالُونَ كُلَّ يَوْمٍ فِي الْهَيْكَلِ وَفِي الْبُيُوتِ مُعَلِّمِينَ وَمُبَشِّرِينَ بِيَسُوعَ الْمَسِيحِ.

\chapter{6}

\par 1 وَفِي تِلْكَ الأَيَّامِ إِذْ تَكَاثَرَ التَّلاَمِيذُ حَدَثَ تَذَمُّرٌ مِنَ الْيُونَانِيِّينَ عَلَى الْعِبْرَانِيِّينَ أَنَّ أَرَامِلَهُمْ كُنَّ يُغْفَلُ عَنْهُنَّ فِي الْخِدْمَةِ الْيَوْمِيَّةِ.
\par 2 فَدَعَا الاِثْنَا عَشَرَ جُمْهُورَ التَّلاَمِيذِ وَقَالُوا: «لاَ يُرْضِي أَنْ نَتْرُكَ نَحْنُ كَلِمَةَ اللهِ وَنَخْدِمَ مَوَائِدَ.
\par 3 فَانْتَخِبُوا أَيُّهَا الإِخْوَةُ سَبْعَةَ رِجَالٍ مِنْكُمْ مَشْهُوداً لَهُمْ وَمَمْلُوِّينَ مِنَ الرُّوحِ الْقُدُسِ وَحِكْمَةٍ فَنُقِيمَهُمْ عَلَى هَذِهِ الْحَاجَةِ.
\par 4 وَأَمَّا نَحْنُ فَنُواظِبُ عَلَى الصَّلاَةِ وَخِدْمَةِ الْكَلِمَةِ».
\par 5 فَحَسُنَ هَذَا الْقَوْلُ أَمَامَ كُلِّ الْجُمْهُورِ فَاخْتَارُوا اسْتِفَانُوسَ رَجُلاً مَمْلُوّاً مِنَ الإِيمَانِ وَالرُّوحِ الْقُدُسِ وَفِيلُبُّسَ وَبُرُوخُورُسَ وَنِيكَانُورَ وَتِيمُونَ وَبَرْمِينَاسَ وَنِيقُولاَوُسَ دَخِيلاً أَنْطَاكِيّاً.
\par 6 اَلَّذِينَ أَقَامُوهُمْ أَمَامَ الرُّسُلِ فَصَلُّوا وَوَضَعُوا عَلَيْهِمِ الأَيَادِيَ.
\par 7 وَكَانَتْ كَلِمَةُ اللهِ تَنْمُو وَعَدَدُ التَّلاَمِيذِ يَتَكَاثَرُ جِدّاً فِي أُورُشَلِيمَ وَجُمْهُورٌ كَثِيرٌ مِنَ الْكَهَنَةِ يُطِيعُونَ الإِيمَانَ.
\par 8 وَأَمَّا اسْتِفَانُوسُ فَإِذْ كَانَ مَمْلُوّاً إِيمَاناً وَقُوَّةً كَانَ يَصْنَعُ عَجَائِبَ وَآيَاتٍ عَظِيمَةً فِي الشَّعْبِ.
\par 9 فَنَهَضَ قَوْمٌ مِنَ الْمَجْمَعِ الَّذِي يُقَالُ لَهُ مَجْمَعُ اللِّيبَرْتِينِيِّينَ وَالْقَيْرَوَانِيِّينَ وَالإِسْكَنْدَرِيِّينَ وَمِنَ الَّذِينَ مِنْ كِيلِيكِيَّا وَأَسِيَّا يُحَاوِرُونَ اسْتِفَانُوسَ.
\par 10 وَلَمْ يَقْدِرُوا أَنْ يُقَاوِمُوا الْحِكْمَةَ وَالرُّوحَ الَّذِي كَانَ يَتَكَلَّمُ بِهِ.
\par 11 حِينَئِذٍ دَسُّوا لِرِجَالٍ يَقُولُونَ: «إِنَّنَا سَمِعْنَاهُ يَتَكَلَّمُ بِكَلاَمِ تَجْدِيفٍ عَلَى مُوسَى وَعَلَى اللهِ».
\par 12 وَهَيَّجُوا الشَّعْبَ وَالشُّيُوخَ وَالْكَتَبَةَ فَقَامُوا وَخَطَفُوهُ وَأَتَوْا بِهِ إِلَى الْمَجْمَعِ
\par 13 وَأَقَامُوا شُهُوداً كَذَبَةً يَقُولُونَ: «هَذَا الرَّجُلُ لاَ يَفْتُرُ عَنْ أَنْ يَتَكَلَّمَ تَجْدِيفاً ضِدَّ هَذَا الْمَوْضِعِ الْمُقَدَّسِ وَالنَّامُوسِ
\par 14 لأَنَّنَا سَمِعْنَاهُ يَقُولُ: إِنَّ يَسُوعَ النَّاصِرِيَّ هَذَا سَيَنْقُضُ هَذَا الْمَوْضِعَ وَيُغَيِّرُ الْعَوَائِدَ الَّتِي سَلَّمَنَا إِيَّاهَا مُوسَى».
\par 15 فَشَخَصَ إِلَيْهِ جَمِيعُ الْجَالِسِينَ فِي الْمَجْمَعِ وَرَأَوْا وَجْهَهُ كَأَنَّهُ وَجْهُ مَلاَكٍ.

\chapter{7}

\par 1 فَسَأَلَ رَئِيسُ الْكَهَنَةِ: «أَتُرَى هَذِهِ الْأُمُورُ هَكَذَا هِيَ؟»
\par 2 فَأَجَابَ: «أَيُّهَا الرِّجَالُ الإِخْوَةُ وَالآبَاءُ اسْمَعُوا. ظَهَرَ إِلَهُ الْمَجْدِ لأَبِينَا إِبْرَاهِيمَ وَهُوَ فِي مَا بَيْنَ النَّهْرَيْنِ قَبْلَمَا سَكَنَ فِي حَارَانَ
\par 3 وَقَالَ لَهُ: اخْرُجْ مِنْ أَرْضِكَ وَمِنْ عَشِيرَتِكَ وَهَلُمَّ إِلَى الأَرْضِ الَّتِي أُرِيكَ
\par 4 فَخَرَجَ حِينَئِذٍ مِنْ أَرْضِ الْكَلْدَانِيِّينَ وَسَكَنَ فِي حَارَانَ. وَمِنْ هُنَاكَ نَقَلَهُ بَعْدَ مَا مَاتَ أَبُوهُ إِلَى هَذِهِ الأَرْضِ الَّتِي أَنْتُمُ الآنَ سَاكِنُونَ فِيهَا.
\par 5 وَلَمْ يُعْطِهِ فِيهَا مِيرَاثاً وَلاَ وَطْأَةَ قَدَمٍ وَلَكِنْ وَعَدَ أَنْ يُعْطِيَهَا مُلْكاً لَهُ وَلِنَسْلِهِ مِنْ بَعْدِهِ وَلَمْ يَكُنْ لَهُ بَعْدُ وَلَدٌ.
\par 6 وَتَكَلَّمَ اللهُ هَكَذَا: أَنْ يَكُونَ نَسْلُهُ مُتَغَرِّباً فِي أَرْضٍ غَرِيبَةٍ فَيَسْتَعْبِدُوهُ وَيُسِيئُوا إِلَيْهِ أَرْبَعَ مِئَةِ سَنَةٍ
\par 7 وَالْأُمَّةُ الَّتِي يُسْتَعْبَدُونَ لَهَا سَأَدِينُهَا أَنَا يَقُولُ اللهُ. وَبَعْدَ ذَلِكَ يَخْرُجُونَ وَيَعْبُدُونَنِي فِي هَذَا الْمَكَانِ.
\par 8 وَأَعْطَاهُ عَهْدَ الْخِتَانِ وَهَكَذَا وَلَدَ إِسْحَاقَ وَخَتَنَهُ فِي الْيَوْمِ الثَّامِنِ. وَإِسْحَاقُ وَلَدَ يَعْقُوبَ وَيَعْقُوبُ وَلَدَ رُؤَسَاءَ الآبَاءِ الاِثْنَيْ عَشَرَ.
\par 9 وَرُؤَسَاءُ الآبَاءِ حَسَدُوا يُوسُفَ وَبَاعُوهُ إِلَى مِصْرَ وَكَانَ اللهُ مَعَهُ
\par 10 وَأَنْقَذَهُ مِنْ جَمِيعِ ضِيقَاتِهِ وَأَعْطَاهُ نِعْمَةً وَحِكْمَةً أَمَامَ فِرْعَوْنَ مَلِكِ مِصْرَ فَأَقَامَهُ مُدَبِّراً عَلَى مِصْرَ وَعَلَى كُلِّ بَيْتِهِ.
\par 11 ثُمَّ أَتَى جُوعٌ عَلَى كُلِّ أَرْضِ مِصْرَ وَكَنْعَانَ وَضِيقٌ عَظِيمٌ فَكَانَ آبَاؤُنَا لاَ يَجِدُونَ قُوتاً.
\par 12 وَلَمَّا سَمِعَ يَعْقُوبُ أَنَّ فِي مِصْرَ قَمْحاً أَرْسَلَ آبَاءَنَا أَوَّلَ مَرَّةٍ.
\par 13 وَفِي الْمَرَّةِ الثَّانِيَةِ اسْتَعْرَفَ يُوسُفُ إِلَى إِخْوَتِهِ وَاسْتَعْلَنَتْ عَشِيرَةُ يُوسُفَ لِفِرْعَوْنَ.
\par 14 فَأَرْسَلَ يُوسُفُ وَاسْتَدْعَى أَبَاهُ يَعْقُوبَ وَجَمِيعَ عَشِيرَتِهِ خَمْسَةً وَسَبْعِينَ نَفْساً.
\par 15 فَنَزَلَ يَعْقُوبُ إِلَى مِصْرَ وَمَاتَ هُوَ وَآبَاؤُنَا
\par 16 وَنُقِلُوا إِلَى شَكِيمَ وَوُضِعُوا فِي الْقَبْرِ الَّذِي اشْتَرَاهُ إِبْرَاهِيمُ بِثَمَنٍ فِضَّةٍ مِنْ بَنِي حَمُورَ أَبِي شَكِيمَ.
\par 17 وَكَمَا كَانَ يَقْرُبُ وَقْتُ الْمَوْعِدِ الَّذِي أَقْسَمَ اللهُ عَلَيْهِ لِإِبْرَاهِيمَ كَانَ الشَّعْبُ يَنْمُو وَيَكْثُرُ فِي مِصْرَ
\par 18 إِلَى أَنْ قَامَ مَلِكٌ آخَرُ لَمْ يَكُنْ يَعْرِفُ يُوسُفَ.
\par 19 فَاحْتَالَ هَذَا عَلَى جِنْسِنَا وَأَسَاءَ إِلَى آبَائِنَا حَتَّى جَعَلُوا أَطْفَالَهُمْ مَنْبُوذِينَ لِكَيْ لاَ يَعِيشُوا.
\par 20 «وَفِي ذَلِكَ الْوَقْتِ وُلِدَ مُوسَى وَكَانَ جَمِيلاً جِدّاً فَرُبِّيَ هَذَا ثَلاَثَةَ أَشْهُرٍ فِي بَيْتِ أَبِيهِ.
\par 21 وَلَمَّا نُبِذَ اتَّخَذَتْهُ ابْنَةُ فِرْعَوْنَ وَرَبَّتْهُ لِنَفْسِهَا ابْناً.
\par 22 فَتَهَذَّبَ مُوسَى بِكُلِّ حِكْمَةِ الْمِصْرِيِّينَ وَكَانَ مُقْتَدِراً فِي الأَقْوَالِ وَالأَعْمَالِ.
\par 23 وَلَمَّا كَمِلَتْ لَهُ مُدَّةُ أَرْبَعِينَ سَنَةً خَطَرَ عَلَى بَالِهِ أَنْ يَفْتَقِدَ إِخْوَتَهُ بَنِي إِسْرَائِيلَ.
\par 24 وَإِذْ رَأَى وَاحِداً مَظْلُوماً حَامَى عَنْهُ وَأَنْصَفَ الْمَغْلُوبَ إِذْ قَتَلَ الْمِصْرِيَّ.
\par 25 فَظَنَّ أَنَّ إِخْوَتَهُ يَفْهَمُونَ أَنَّ اللهَ عَلَى يَدِهِ يُعْطِيهِمْ نَجَاةً وَأَمَّا هُمْ فَلَمْ يَفْهَمُوا.
\par 26 وَفِي الْيَوْمِ الثَّانِي ظَهَرَ لَهُمْ وَهُمْ يَتَخَاصَمُونَ فَسَاقَهُمْ إِلَى السَّلاَمَةِ قَائِلاً: أَيُّهَا الرِّجَالُ أَنْتُمْ إِخْوَةٌ. لِمَاذَا تَظْلِمُونَ بَعْضُكُمْ بَعْضاً؟
\par 27 فَالَّذِي كَانَ يَظْلِمُ قَرِيبَهُ دَفَعَهُ قَائِلاً: مَنْ أَقَامَكَ رَئِيساً وَقَاضِياً عَلَيْنَا؟
\par 28 أَتُرِيدُ أَنْ تَقْتُلَنِي كَمَا قَتَلْتَ أَمْسَ الْمِصْرِيَّ؟
\par 29 فَهَرَبَ مُوسَى بِسَبَبِ هَذِهِ الْكَلِمَةِ وَصَارَ غَرِيباً فِي أَرْضِ مَدْيَانَ حَيْثُ وَلَدَ ابْنَيْنِ.
\par 30 «وَلَمَّا كَمِلَتْ أَرْبَعُونَ سَنَةً ظَهَرَ لَهُ مَلاَكُ الرَّبِّ فِي بَرِّيَّةِ جَبَلِ سِينَاءَ فِي لَهِيبِ نَارِ عُلَّيْقَةٍ.
\par 31 فَلَمَّا رَأَى مُوسَى ذَلِكَ تَعَجَّبَ مِنَ الْمَنْظَرِ. وَفِيمَا هُوَ يَتَقَدَّمُ لِيَتَطَلَّعَ صَارَ إِلَيْهِ صَوْتُ الرَّبِّ:
\par 32 أَنَا إِلَهُ آبَائِكَ إِلَهُ إِبْرَاهِيمَ وَإِلَهُ إِسْحَاقَ وَإِلَهُ يَعْقُوبَ. فَارْتَعَدَ مُوسَى وَلَمْ يَجْسُرْ أَنْ يَتَطَلَّعَ.
\par 33 فَقَالَ لَهُ الرَّبُّ: اخْلَعْ نَعْلَ رِجْلَيْكَ لأَنَّ الْمَوْضِعَ الَّذِي أَنْتَ وَاقِفٌ عَلَيْهِ أَرْضٌ مُقَدَّسَةٌ.
\par 34 إِنِّي رَأَيْتُ مَشَقَّةَ شَعْبِي الَّذِينَ فِي مِصْرَ وَسَمِعْتُ أَنِينَهُمْ وَنَزَلْتُ لِأُنْقِذَهُمْ. فَهَلُمَّ الآنَ أُرْسِلُكَ إِلَى مِصْرَ.
\par 35 «هَذَا مُوسَى الَّذِي أَنْكَرُوهُ قَائِلِينَ: مَنْ أَقَامَكَ رَئِيساً وَقَاضِياً؟ هَذَا أَرْسَلَهُ اللهُ رَئِيساً وَفَادِياً بِيَدِ الْمَلاَكِ الَّذِي ظَهَرَ لَهُ فِي الْعُلَّيْقَةِ.
\par 36 هَذَا أَخْرَجَهُمْ صَانِعاً عَجَائِبَ وَآيَاتٍ فِي أَرْضِ مِصْرَ وَفِي الْبَحْرِ الأَحْمَرِ وَفِي الْبَرِّيَّةِ أَرْبَعِينَ سَنَةً.
\par 37 «هَذَا هُوَ مُوسَى الَّذِي قَالَ لِبَنِي إِسْرَائِيلَ: نَبِيّاً مِثْلِي سَيُقِيمُ لَكُمُ الرَّبُّ إِلَهُكُمْ مِنْ إِخْوَتِكُمْ. لَهُ تَسْمَعُونَ.
\par 38 هَذَا هُوَ الَّذِي كَانَ فِي الْكَنِيسَةِ فِي الْبَرِّيَّةِ مَعَ الْمَلاَكِ الَّذِي كَانَ يُكَلِّمُهُ فِي جَبَلِ سِينَاءَ وَمَعَ آبَائِنَا. الَّذِي قَبِلَ أَقْوَالاً حَيَّةً لِيُعْطِيَنَا إِيَّاهَا.
\par 39 الَّذِي لَمْ يَشَأْ آبَاؤُنَا أَنْ يَكُونُوا طَائِعِينَ لَهُ بَلْ دَفَعُوهُ وَرَجَعُوا بِقُلُوبِهِمْ إِلَى مِصْرَ
\par 40 قَائِلِينَ لِهَارُونَ: اعْمَلْ لَنَا آلِهَةً تَتَقَدَّمُ أَمَامَنَا لأَنَّ هَذَا مُوسَى الَّذِي أَخْرَجَنَا مِنْ أَرْضِ مِصْرَ لاَ نَعْلَمُ مَاذَا أَصَابَهُ.
\par 41 فَعَمِلُوا عِجْلاً فِي تِلْكَ الأَيَّامِ وَأَصْعَدُوا ذَبِيحَةً لِلصَّنَمِ وَفَرِحُوا بِأَعْمَالِ أَيْدِيهِمْ.
\par 42 فَرَجَعَ اللهُ وَأَسْلَمَهُمْ لِيَعْبُدُوا جُنْدَ السَّمَاءِ كَمَا هُوَ مَكْتُوبٌ فِي كِتَابِ الأَنْبِيَاءِ: هَلْ قَرَّبْتُمْ لِي ذَبَائِحَ وَقَرَابِينَ أَرْبَعِينَ سَنَةً فِي الْبَرِّيَّةِ يَا بَيْتَ إِسْرَائِيلَ؟
\par 43 بَلْ حَمَلْتُمْ خَيْمَةَ مُولُوكَ وَنَجْمَ إِلَهِكُمْ رَمْفَانَ التَّمَاثِيلَ الَّتِي صَنَعْتُمُوهَا لِتَسْجُدُوا لَهَا. فَأَنْقُلُكُمْ إِلَى مَا وَرَاءَ بَابِلَ.
\par 44 «وَأَمَّا خَيْمَةُ الشَّهَادَةِ فَكَانَتْ مَعَ آبَائِنَا فِي الْبَرِّيَّةِ كَمَا أَمَرَ الَّذِي كَلَّمَ مُوسَى أَنْ يَعْمَلَهَا عَلَى الْمِثَالِ الَّذِي كَانَ قَدْ رَآهُ
\par 45 الَّتِي أَدْخَلَهَا أَيْضاً آبَاؤُنَا إِذْ تَخَلَّفُوا عَلَيْهَا مَعَ يَشُوعَ فِي مُلْكِ الْأُمَمِ الَّذِينَ طَرَدَهُمُ اللهُ مِنْ وَجْهِ آبَائِنَا إِلَى أَيَّامِ دَاوُدَ
\par 46 الَّذِي وَجَدَ نِعْمَةً أَمَامَ اللهِ وَالْتَمَسَ أَنْ يَجِدَ مَسْكَناً لِإِلَهِ يَعْقُوبَ.
\par 47 وَلَكِنَّ سُلَيْمَانَ بَنَى لَهُ بَيْتاً.
\par 48 لَكِنَّ الْعَلِيَّ لاَ يَسْكُنُ فِي هَيَاكِلَ مَصْنُوعَةٍ بِالأَيَادِي كَمَا يَقُولُ النَّبِيُّ:
\par 49 السَّمَاءُ كُرْسِيٌّ لِي وَالأَرْضُ مَوْطِئٌ لِقَدَمَيَّ. أَيَّ بَيْتٍ تَبْنُونَ لِي يَقُولُ الرَّبُّ وَأَيٌّ هُوَ مَكَانُ رَاحَتِي؟
\par 50 أَلَيْسَتْ يَدِي صَنَعَتْ هَذِهِ الأَشْيَاءَ كُلَّهَا؟
\par 51 «يَا قُسَاةَ الرِّقَابِ وَغَيْرَ الْمَخْتُونِينَ بِالْقُلُوبِ وَالآذَانِ أَنْتُمْ دَائِماً تُقَاوِمُونَ الرُّوحَ الْقُدُسَ. كَمَا كَانَ آبَاؤُكُمْ كَذَلِكَ أَنْتُمْ.
\par 52 أَيُّ الأَنْبِيَاءِ لَمْ يَضْطَهِدْهُ آبَاؤُكُمْ وَقَدْ قَتَلُوا الَّذِينَ سَبَقُوا فَأَنْبَأُوا بِمَجِيءِ الْبَارِّ الَّذِي أَنْتُمُ الآنَ صِرْتُمْ مُسَلِّمِيهِ وَقَاتِلِيهِ
\par 53 الَّذِينَ أَخَذْتُمُ النَّامُوسَ بِتَرْتِيبِ مَلاَئِكَةٍ وَلَمْ تَحْفَظُوهُ؟».
\par 54 فَلَمَّا سَمِعُوا هَذَا حَنِقُوا بِقُلُوبِهِمْ وَصَرُّوا بِأَسْنَانِهِمْ عَلَيْهِ.
\par 55 وَأَمَّا هُوَ فَشَخَصَ إِلَى السَّمَاءِ وَهُوَ مُمْتَلِئٌ مِنَ الرُّوحِ الْقُدُسِ فَرَأَى مَجْدَ اللهِ وَيَسُوعَ قَائِماً عَنْ يَمِينِ اللهِ.
\par 56 فَقَالَ: «هَا أَنَا أَنْظُرُ السَّمَاوَاتِ مَفْتُوحَةً وَابْنَ الإِنْسَانِ قَائِماً عَنْ يَمِينِ اللهِ».
\par 57 فَصَاحُوا بِصَوْتٍ عَظِيمٍ وَسَدُّوا آذَانَهُمْ وَهَجَمُوا عَلَيْهِ بِنَفْسٍ وَاحِدَةٍ
\par 58 وَأَخْرَجُوهُ خَارِجَ الْمَدِينَةِ وَرَجَمُوهُ. وَالشُّهُودُ خَلَعُوا ثِيَابَهُمْ عِنْدَ رِجْلَيْ شَابٍّ يُقَالُ لَهُ شَاوُلُ.
\par 59 فَكَانُوا يَرْجُمُونَ اسْتِفَانُوسَ وَهُوَ يَدْعُو وَيَقُولُ: «أَيُّهَا الرَّبُّ يَسُوعُ اقْبَلْ رُوحِي».
\par 60 ثُمَّ جَثَا عَلَى رُكْبَتَيْهِ وَصَرَخَ بِصَوْتٍ عَظِيمٍ: «يَا رَبُّ لاَ تُقِمْ لَهُمْ هَذِهِ الْخَطِيَّةَ». وَإِذْ قَالَ هَذَا رَقَدَ.

\chapter{8}

\par 1 وَكَانَ شَاوُلُ رَاضِياً بِقَتْلِهِ. وَحَدَثَ فِي ذَلِكَ الْيَوْمِ اضْطِهَادٌ عَظِيمٌ عَلَى الْكَنِيسَةِ الَّتِي فِي أُورُشَلِيمَ فَتَشَتَّتَ الْجَمِيعُ فِي كُوَرِ الْيَهُودِيَّةِ وَالسَّامِرَةِ مَا عَدَا الرُّسُلَ.
\par 2 وَحَمَلَ رِجَالٌ أَتْقِيَاءُ اسْتِفَانُوسَ وَعَمِلُوا عَلَيْهِ مَنَاحَةً عَظِيمَةً.
\par 3 وَأَمَّا شَاوُلُ فَكَانَ يَسْطُو عَلَى الْكَنِيسَةِ وَهُوَ يَدْخُلُ الْبُيُوتَ وَيَجُرُّ رِجَالاً وَنِسَاءً وَيُسَلِّمُهُمْ إِلَى السِّجْنِ.
\par 4 فَالَّذِينَ تَشَتَّتُوا جَالُوا مُبَشِّرِينَ بِالْكَلِمَةِ.
\par 5 فَانْحَدَرَ فِيلُبُّسُ إِلَى مَدِينَةٍ مِنَ السَّامِرَةِ وَكَانَ يَكْرِزُ لَهُمْ بِالْمَسِيحِ.
\par 6 وَكَانَ الْجُمُوعُ يُصْغُونَ بِنَفْسٍ وَاحِدَةٍ إِلَى مَا يَقُولُهُ فِيلُبُّسُ عِنْدَ اسْتِمَاعِهِمْ وَنَظَرِهِمُ الآيَاتِ الَّتِي صَنَعَهَا
\par 7 لأَنَّ كَثِيرِينَ مِنَ الَّذِينَ بِهِمْ أَرْوَاحٌ نَجِسَةٌ كَانَتْ تَخْرُجُ صَارِخَةً بِصَوْتٍ عَظِيمٍ وَكَثِيرُونَ مِنَ الْمَفْلُوجِينَ وَالْعُرْجِ شُفُوا.
\par 8 فَكَانَ فَرَحٌ عَظِيمٌ فِي تِلْكَ الْمَدِينَةِ.
\par 9 وَكَانَ قَبْلاً فِي الْمَدِينَةِ رَجُلٌ اسْمُهُ سِيمُونُ يَسْتَعْمِلُ السِّحْرَ وَيُدْهِشُ شَعْبَ السَّامِرَةِ قَائِلاً: «إِنَّهُ شَيْءٌ عَظِيمٌ!».
\par 10 وَكَانَ الْجَمِيعُ يَتْبَعُونَهُ مِنَ الصَّغِيرِ إِلَى الْكَبِيرِ قَائِلِينَ: «هَذَا هُوَ قُوَّةُ اللهِ الْعَظِيمَةُ».
\par 11 وَكَانُوا يَتْبَعُونَهُ لِكَوْنِهِمْ قَدِ انْدَهَشُوا زَمَاناً طَوِيلاً بِسِحْرِهِ.
\par 12 وَلَكِنْ لَمَّا صَدَّقُوا فِيلُبُّسَ وَهُوَ يُبَشِّرُ بِالْأُمُورِ الْمُخْتَصَّةِ بِمَلَكُوتِ اللهِ وَبِاسْمِ يَسُوعَ الْمَسِيحِ اعْتَمَدُوا رِجَالاً وَنِسَاءً.
\par 13 وَسِيمُونُ أَيْضاً نَفْسُهُ آمَنَ. وَلَمَّا اعْتَمَدَ كَانَ يُلاَزِمُ فِيلُبُّسَ وَإِذْ رَأَى آيَاتٍ وَقُوَّاتٍ عَظِيمَةً تُجْرَى انْدَهَشَ.
\par 14 وَلَمَّا سَمِعَ الرُّسُلُ الَّذِينَ فِي أُورُشَلِيمَ أَنَّ السَّامِرَةَ قَدْ قَبِلَتْ كَلِمَةَ اللهِ أَرْسَلُوا إِلَيْهِمْ بُطْرُسَ وَيُوحَنَّا
\par 15 اللَّذَيْنِ لَمَّا نَزَلاَ صَلَّيَا لأَجْلِهِمْ لِكَيْ يَقْبَلُوا الرُّوحَ الْقُدُسَ
\par 16 لأَنَّهُ لَمْ يَكُنْ قَدْ حَلَّ بَعْدُ عَلَى أَحَدٍ مِنْهُمْ - غَيْرَ أَنَّهُمْ كَانُوا مُعْتَمِدِينَ بِاسْمِ الرَّبِّ يَسُوعَ.
\par 17 حِينَئِذٍ وَضَعَا الأَيَادِيَ عَلَيْهِمْ فَقَبِلُوا الرُّوحَ الْقُدُسَ.
\par 18 وَلَمَّا رَأَى سِيمُونُ أَنَّهُ بِوَضْعِ أَيْدِي الرُّسُلِ يُعْطَى الرُّوحُ الْقُدُسُ قَدَّمَ لَهُمَا دَرَاهِمَ
\par 19 قَائِلاً: «أَعْطِيَانِي أَنَا أَيْضاً هَذَا السُّلْطَانَ حَتَّى أَيُّ مَنْ وَضَعْتُ عَلَيْهِ يَدَيَّ يَقْبَلُ الرُّوحَ الْقُدُسَ».
\par 20 فَقَالَ لَهُ بُطْرُسُ: «لِتَكُنْ فِضَّتُكَ مَعَكَ لِلْهَلاَكِ لأَنَّكَ ظَنَنْتَ أَنْ تَقْتَنِيَ مَوْهِبَةَ اللهِ بِدَرَاهِمَ.
\par 21 لَيْسَ لَكَ نَصِيبٌ وَلاَ قُرْعَةٌ فِي هَذَا الأَمْرِ لأَنَّ قَلْبَكَ لَيْسَ مُسْتَقِيماً أَمَامَ اللهِ.
\par 22 فَتُبْ مِنْ شَرِّكَ هَذَا وَاطْلُبْ إِلَى اللهِ عَسَى أَنْ يُغْفَرَ لَكَ فِكْرُ قَلْبِكَ
\par 23 لأَنِّي أَرَاكَ فِي مَرَارَةِ الْمُرِّ وَرِبَاطِ الظُّلْمِ».
\par 24 فَأَجَابَ سِيمُونُ: «اطْلُبَا أَنْتُمَا إِلَى الرَّبِّ مِنْ أَجْلِي لِكَيْ لاَ يَأْتِيَ عَلَيَّ شَيْءٌ مِمَّا ذَكَرْتُمَا».
\par 25 ثُمَّ إِنَّهُمَا بَعْدَ مَا شَهِدَا وَتَكَلَّمَا بِكَلِمَةِ الرَّبِّ رَجَعَا إِلَى أُورُشَلِيمَ وَبَشَّرَا قُرىً كَثِيرَةً لِلسَّامِرِيِّينَ.
\par 26 ثُمَّ إِنَّ مَلاَكَ الرَّبِّ قَالَ لِفِيلُبُّسَ: «قُمْ وَاذْهَبْ نَحْوَ الْجَنُوبِ عَلَى الطَّرِيقِ الْمُنْحَدِرَةِ مِنْ أُورُشَلِيمَ إِلَى غَزَّةَ» الَّتِي هِيَ بَرِّيَّةٌ.
\par 27 فَقَامَ وَذَهَبَ. وَإِذَا رَجُلٌ حَبَشِيٌّ خَصِيٌّ وَزِيرٌ لِكَنْدَاكَةَ مَلِكَةِ الْحَبَشَةِ كَانَ عَلَى جَمِيعِ خَزَائِنِهَا - فَهَذَا كَانَ قَدْ جَاءَ إِلَى أُورُشَلِيمَ لِيَسْجُدَ.
\par 28 وَكَانَ رَاجِعاً وَجَالِساً عَلَى مَرْكَبَتِهِ وَهُوَ يَقْرَأُ النَّبِيَّ إِشَعْيَاءَ.
\par 29 فَقَالَ الرُّوحُ لِفِيلُبُّسَ: «تَقَدَّمْ وَرَافِقْ هَذِهِ الْمَرْكَبَةَ».
\par 30 فَبَادَرَ إِلَيْهِ فِيلُبُّسُ وَسَمِعَهُ يَقْرَأُ النَّبِيَّ إِشَعْيَاءَ فَسَأَلَهُ: «أَلَعَلَّكَ تَفْهَمُ مَا أَنْتَ تَقْرَأُ؟»
\par 31 فَأَجَابَ: «كَيْفَ يُمْكِنُنِي إِنْ لَمْ يُرْشِدْنِي أَحَدٌ؟». وَطَلَبَ إِلَى فِيلُبُّسَ أَنْ يَصْعَدَ وَيَجْلِسَ مَعَهُ.
\par 32 وَأَمَّا فَصْلُ الْكِتَابِ الَّذِي كَانَ يَقْرَأُهُ فَكَانَ هَذَا: «مِثْلَ شَاةٍ سِيقَ إِلَى الذَّبْحِ وَمِثْلَ خَرُوفٍ صَامِتٍ أَمَامَ الَّذِي يَجُزُّهُ هَكَذَا لَمْ يَفْتَحْ فَاهُ.
\par 33 فِي تَوَاضُعِهِ انْتَزَعَ قَضَاؤُهُ وَجِيلُهُ مَنْ يُخْبِرُ بِهِ لأَنَّ حَيَاتَهُ تُنْتَزَعُ مِنَ الأَرْضِ؟»
\par 34 فَسَأَلَ الْخَصِيُّ فِيلُبُّسَ: «أَطْلُبُ إِلَيْكَ: عَنْ مَنْ يَقُولُ النَّبِيُّ هَذَا؟ عَنْ نَفْسِهِ أَمْ عَنْ وَاحِدٍ آخَرَ؟»
\par 35 فَابْتَدَأَ فِيلُبُّسُ مِنْ هَذَا الْكِتَابِ يُبَشِّرَهُ بِيَسُوعَ.
\par 36 وَفِيمَا هُمَا سَائِرَانِ فِي الطَّرِيقِ أَقْبَلاَ عَلَى مَاءٍ فَقَالَ الْخَصِيُّ: «هُوَذَا مَاءٌ. مَاذَا يَمْنَعُ أَنْ أَعْتَمِدَ؟»
\par 37 فَقَالَ فِيلُبُّسُ: «إِنْ كُنْتَ تُؤْمِنُ مِنْ كُلِّ قَلْبِكَ يَجُوزُ». فَأَجَابَ: «أَنَا أُومِنُ أَنَّ يَسُوعَ الْمَسِيحَ هُوَ ابْنُ اللهِ».
\par 38 فَأَمَرَ أَنْ تَقِفَ الْمَرْكَبَةُ فَنَزَلاَ كِلاَهُمَا إِلَى الْمَاءِ فِيلُبُّسُ وَالْخَصِيُّ فَعَمَّدَهُ.
\par 39 وَلَمَّا صَعِدَا مِنَ الْمَاءِ خَطَفَ رُوحُ الرَّبِّ فِيلُبُّسَ فَلَمْ يُبْصِرْهُ الْخَصِيُّ أَيْضاً وَذَهَبَ فِي طَرِيقِهِ فَرِحاً.
\par 40 وَأَمَّا فِيلُبُّسُ فَوُجِدَ فِي أَشْدُودَ. وَبَيْنَمَا هُوَ مُجْتَازٌ كَانَ يُبَشِّرُ جَمِيعَ الْمُدُنِ حَتَّى جَاءَ إِلَى قَيْصَرِيَّةَ.

\chapter{9}

\par 1 أَمَّا شَاوُلُ فَكَانَ لَمْ يَزَلْ يَنْفُثُ تَهَدُّداً وَقَتْلاً عَلَى تَلاَمِيذِ الرَّبِّ فَتَقَدَّمَ إِلَى رَئِيسِ الْكَهَنَةِ
\par 2 وَطَلَبَ مِنْهُ رَسَائِلَ إِلَى دِمَشْقَ إِلَى الْجَمَاعَاتِ حَتَّى إِذَا وَجَدَ أُنَاساً مِنَ الطَّرِيقِ رِجَالاً أَوْ نِسَاءً يَسُوقُهُمْ مُوثَقِينَ إِلَى أُورُشَلِيمَ.
\par 3 وَفِي ذَهَابِهِ حَدَثَ أَنَّهُ اقْتَرَبَ إِلَى دِمَشْقَ فَبَغْتَةً أَبْرَقَ حَوْلَهُ نُورٌ مِنَ السَّمَاءِ
\par 4 فَسَقَطَ عَلَى الأَرْضِ وَسَمِعَ صَوْتاً قَائِلاً لَهُ: «شَاوُلُ شَاوُلُ لِمَاذَا تَضْطَهِدُنِي؟»
\par 5 فَسَأَلَهُ: «مَنْ أَنْتَ يَا سَيِّدُ؟» فَقَالَ الرَّبُّ: «أَنَا يَسُوعُ الَّذِي أَنْتَ تَضْطَهِدُهُ. صَعْبٌ عَلَيْكَ أَنْ تَرْفُسَ مَنَاخِسَ».
\par 6 فَسَأَلَ وَهُوَ مُرْتَعِدٌ وَمُتَحَيِّرٌ: «يَا رَبُّ مَاذَا تُرِيدُ أَنْ أَفْعَلَ؟»فَقَالَ لَهُ الرَّبُّ: «قُم وَادْخُلِ الْمَدِينَةَ فَيُقَالَ لَكَ مَاذَا يَنْبَغِي أَنْ تَفْعَلَ».
\par 7 وَأَمَّا الرِّجَالُ الْمُسَافِرُونَ مَعَهُ فَوَقَفُوا صَامِتِينَ يَسْمَعُونَ الصَّوْتَ وَلاَ يَنْظُرُونَ أَحَداً.
\par 8 فَنَهَضَ شَاوُلُ عَنِ الأَرْضِ وَكَانَ وَهُوَ مَفْتُوحُ الْعَيْنَيْنِ لاَ يُبْصِرُ أَحَداً. فَاقْتَادُوهُ بِيَدِهِ وَأَدْخَلُوهُ إِلَى دِمَشْقَ.
\par 9 وَكَانَ ثَلاَثَةَ أَيَّامٍ لاَ يُبْصِرُ فَلَمْ يَأْكُلْ وَلَمْ يَشْرَبْ.
\par 10 وَكَانَ فِي دِمَشْقَ تِلْمِيذٌ اسْمُهُ حَنَانِيَّا فَقَالَ لَهُ الرَّبُّ فِي رُؤْيَا: «يَا حَنَانِيَّا». فَقَالَ: «هَأَنَذَا يَا رَبُّ».
\par 11 فَقَالَ لَهُ الرَّبُّ: «قُمْ وَاذْهَبْ إِلَى الزُّقَاقِ الَّذِي يُقَالُ لَهُ الْمُسْتَقِيمُ وَاطْلُبْ فِي بَيْتِ يَهُوذَا رَجُلاً طَرْسُوسِيّاً اسْمُهُ شَاوُلُ - لأَنَّهُ هُوَذَا يُصَلِّي.
\par 12 وَقَدْ رَأَى فِي رُؤْيَا رَجُلاً اسْمُهُ حَنَانِيَّا دَاخِلاً وَوَاضِعاً يَدَهُ عَلَيْهِ لِكَيْ يُبْصِرَ».
\par 13 فَأَجَابَ حَنَانِيَّا: «يَا رَبُّ قَدْ سَمِعْتُ مِنْ كَثِيرِينَ عَنْ هَذَا الرَّجُلِ كَمْ مِنَ الشُّرُورِ فَعَلَ بِقِدِّيسِيكَ فِي أُورُشَلِيمَ.
\par 14 وَهَهُنَا لَهُ سُلْطَانٌ مِنْ رُؤَسَاءِ الْكَهَنَةِ أَنْ يُوثِقَ جَمِيعَ الَّذِينَ يَدْعُونَ بِاسْمِكَ».
\par 15 فَقَالَ لَهُ الرَّبُّ: «اذْهَبْ لأَنَّ هَذَا لِي إِنَاءٌ مُخْتَارٌ لِيَحْمِلَ اسْمِي أَمَامَ أُمَمٍ وَمُلُوكٍ وَبَنِي إِسْرَائِيلَ.
\par 16 لأَنِّي سَأُرِيهِ كَمْ يَنْبَغِي أَنْ يَتَأَلَّمَ مِنْ أَجْلِ اسْمِي».
\par 17 فَمَضَى حَنَانِيَّا وَدَخَلَ الْبَيْتَ وَوَضَعَ عَلَيْهِ يَدَيْهِ وَقَالَ: «أَيُّهَا الأَخُ شَاوُلُ قَدْ أَرْسَلَنِي الرَّبُّ يَسُوعُ الَّذِي ظَهَرَ لَكَ فِي الطَّرِيقِ الَّذِي جِئْتَ فِيهِ لِكَيْ تُبْصِرَ وَتَمْتَلِئَ مِنَ الرُّوحِ الْقُدُسِ».
\par 18 فَلِلْوَقْتِ وَقَعَ مِنْ عَيْنَيْهِ شَيْءٌ كَأَنَّهُ قُشُورٌ فَأَبْصَرَ فِي الْحَالِ وَقَامَ وَاعْتَمَدَ.
\par 19 وَتَنَاوَلَ طَعَاماً فَتَقَوَّى. وَكَانَ شَاوُلُ مَعَ التَّلاَمِيذِ الَّذِينَ فِي دِمَشْقَ أَيَّاماً.
\par 20 وَلِلْوَقْتِ جَعَلَ يَكْرِزُ فِي الْمَجَامِعِ بِالْمَسِيحِ «أَنْ هَذَا هُوَ ابْنُ اللهِ».
\par 21 فَبُهِتَ جَمِيعُ الَّذِينَ كَانُوا يَسْمَعُونَ وَقَالُوا: «أَلَيْسَ هَذَا هُوَ الَّذِي أَهْلَكَ فِي أُورُشَلِيمَ الَّذِينَ يَدْعُونَ بِهَذَا الاِسْمِ وَقَدْ جَاءَ إِلَى هُنَا: لِيَسُوقَهُمْ مُوثَقِينَ إِلَى رُؤَسَاءِ الْكَهَنَةِ؟».
\par 22 وَأَمَّا شَاوُلُ فَكَانَ يَزْدَادُ قُوَّةً وَيُحَيِّرُ الْيَهُودَ السَّاكِنِينَ فِي دِمَشْقَ مُحَقِّقاً «أَنَّ هَذَا هُوَ الْمَسِيحُ».
\par 23 وَلَمَّا تَمَّتْ أَيَّامٌ كَثِيرَةٌ تَشَاوَرَ الْيَهُودُ لِيَقْتُلُوهُ
\par 24 فَعَلِمَ شَاوُلُ بِمَكِيدَتِهِمْ. وَكَانُوا يُرَاقِبُونَ الأَبْوَابَ أَيْضاً نَهَاراً وَلَيْلاً لِيَقْتُلُوهُ.
\par 25 فَأَخَذَهُ التَّلاَمِيذُ لَيْلاً وَأَنْزَلُوهُ مِنَ السُّورِ مُدَلِّينَ إِيَّاهُ فِي سَلٍّ.
\par 26 وَلَمَّا جَاءَ شَاوُلُ إِلَى أُورُشَلِيمَ حَاوَلَ أَنْ يَلْتَصِقَ بِالتَّلاَمِيذِ وَكَانَ الْجَمِيعُ يَخَافُونَهُ غَيْرَ مُصَدِّقِينَ أَنَّهُ تِلْمِيذٌ.
\par 27 فَأَخَذَهُ بَرْنَابَا وَأَحْضَرَهُ إِلَى الرُّسُلِ وَحَدَّثَهُمْ كَيْفَ أَبْصَرَ الرَّبَّ فِي الطَّرِيقِ وَأَنَّهُ كَلَّمَهُ وَكَيْفَ جَاهَرَ فِي دِمَشْقَ بِاسْمِ يَسُوعَ.
\par 28 فَكَانَ مَعَهُمْ يَدْخُلُ وَيَخْرُجُ فِي أُورُشَلِيمَ وَيُجَاهِرُ بِاسْمِ الرَّبِّ يَسُوعَ.
\par 29 وَكَانَ يُخَاطِبُ وَيُبَاحِثُ الْيُونَانِيِّينَ فَحَاوَلُوا أَنْ يَقْتُلُوهُ.
\par 30 فَلَمَّا عَلِمَ الإِخْوَةُ أَحْدَرُوهُ إِلَى قَيْصَرِيَّةَ وَأَرْسَلُوهُ إِلَى طَرْسُوسَ.
\par 31 وَأَمَّا الْكَنَائِسُ فِي جَمِيعِ الْيَهُودِيَّةِ وَالْجَلِيلِ وَالسَّامِرَةِ فَكَانَ لَهَا سَلاَمٌ وَكَانَتْ تُبْنَى وَتَسِيرُ فِي خَوْفِ الرَّبِّ وَبِتَعْزِيَةِ الرُّوحِ الْقُدُسِ كَانَتْ تَتَكَاثَرُ.
\par 32 وَحَدَثَ أَنَّ بُطْرُسَ وَهُوَ يَجْتَازُ بِالْجَمِيعِ نَزَلَ أَيْضاً إِلَى الْقِدِّيسِينَ السَّاكِنِينَ فِي لُدَّةَ
\par 33 فَوَجَدَ هُنَاكَ إِنْسَاناً اسْمُهُ إِينِيَاسُ مُضْطَجِعاً عَلَى سَرِيرٍ مُنْذُ ثَمَانِي سِنِينَ وَكَانَ مَفْلُوجاً.
\par 34 فَقَالَ لَهُ بُطْرُسُ: «يَا إِينِيَاسُ يَشْفِيكَ يَسُوعُ الْمَسِيحُ. قُمْ وَافْرُشْ لِنَفْسِكَ». فَقَامَ لِلْوَقْتِ.
\par 35 وَرَآهُ جَمِيعُ السَّاكِنِينَ فِي لُدَّةَ وَسَارُونَ الَّذِينَ رَجَعُوا إِلَى الرَّبِّ.
\par 36 وَكَانَ فِي يَافَا تِلْمِيذَةٌ اسْمُهَا طَابِيثَا الَّذِي تَرْجَمَتُهُ غَزَالَةُ. هَذِهِ كَانَتْ مُمْتَلِئَةً أَعْمَالاً صَالِحَةً وَإِحْسَانَاتٍ كَانَتْ تَعْمَلُهَا.
\par 37 وَحَدَثَ فِي تِلْكَ الأَيَّامِ أَنَّهَا مَرِضَتْ وَمَاتَتْ فَغَسَّلُوهَا وَوَضَعُوهَا فِي عِلِّيَّةٍ.
\par 38 وَإِذْ كَانَتْ لُدَّةُ قَرِيبَةً مِنْ يَافَا وَسَمِعَ التَّلاَمِيذُ أَنَّ بُطْرُسَ فِيهَا أَرْسَلُوا رَجُلَيْنِ يَطْلُبَانِ إِلَيْهِ أَنْ لاَ يَتَوَانَى عَنْ أَنْ يَجْتَازَ إِلَيْهِمْ.
\par 39 فَقَامَ بُطْرُسُ وَجَاءَ مَعَهُمَا. فَلَمَّا وَصَلَ صَعِدُوا بِهِ إِلَى الْعِلِّيَّةِ فَوَقَفَتْ لَدَيْهِ جَمِيعُ الأَرَامِلِ يَبْكِينَ وَيُرِينَ أَقْمِصَةً وَثِيَاباً مِمَّا كَانَتْ تَعْمَلُ غَزَالَةُ وَهِيَ مَعَهُنَّ.
\par 40 فَأَخْرَجَ بُطْرُسُ الْجَمِيعَ خَارِجاً وَجَثَا عَلَى رُكْبَتَيْهِ وَصَلَّى ثُمَّ الْتَفَتَ إِلَى الْجَسَدِ وَقَالَ: «يَا طَابِيثَا قُومِي!» فَفَتَحَتْ عَيْنَيْهَا. وَلَمَّا أَبْصَرَتْ بُطْرُسَ جَلَسَتْ
\par 41 فَنَاوَلَهَا يَدَهُ وَأَقَامَهَا. ثُمَّ نَادَى الْقِدِّيسِينَ وَالأَرَامِلَ وَأَحْضَرَهَا حَيَّةً.
\par 42 فَصَارَ ذَلِكَ مَعْلُوماً فِي يَافَا كُلِّهَا فَآمَنَ كَثِيرُونَ بِالرَّبِّ.
\par 43 وَمَكَثَ أَيَّاماً كَثِيرَةً فِي يَافَا عِنْدَ سِمْعَانَ رَجُلٍ دَبَّاغٍ.

\chapter{10}

\par 1 وَكَانَ فِي قَيْصَرِيَّةَ رَجُلٌ اسْمُهُ كَرْنِيلِيُوسُ قَائِدُ مِئَةٍ مِنَ الْكَتِيبَةِ الَّتِي تُدْعَى الإِيطَالِيَّةَ.
\par 2 وَهُوَ تَقِيٌّ وَخَائِفُ اللهِ مَعَ جَمِيعِ بَيْتِهِ يَصْنَعُ حَسَنَاتٍ كَثِيرَةً لِلشَّعْبِ وَيُصَلِّي إِلَى اللهِ فِي كُلِّ حِينٍ.
\par 3 فَرَأَى ظَاهِراً فِي رُؤْيَا نَحْوَ السَّاعَةِ التَّاسِعَةِ مِنَ النَّهَارِ مَلاَكاً مِنَ اللهِ دَاخِلاً إِلَيْهِ وَقَائِلاً لَهُ: «يَا كَرْنِيلِيُوسُ».
\par 4 فَلَمَّا شَخَصَ إِلَيْهِ وَدَخَلَهُ الْخَوْفُ قَالَ: «مَاذَا يَا سَيِّدُ؟» فَقَالَ لَهُ: «صَلَوَاتُكَ وَصَدَقَاتُكَ صَعِدَتْ تَذْكَاراً أَمَامَ اللهِ.
\par 5 وَالآنَ أَرْسِلْ إِلَى يَافَا رِجَالاً وَاسْتَدْعِ سِمْعَانَ الْمُلَقَّبَ بُطْرُسَ.
\par 6 إِنَّهُ نَازِلٌ عِنْدَ سِمْعَانَ رَجُلٍ دَبَّاغٍ بَيْتُهُ عِنْدَ الْبَحْرِ. هُوَ يَقُولُ لَكَ مَاذَا يَنْبَغِي أَنْ تَفْعَلَ».
\par 7 فَلَمَّا انْطَلَقَ الْمَلاَكُ الَّذِي كَانَ يُكَلِّمُ كَرْنِيلِيُوسَ نَادَى اثْنَيْنِ مِنْ خُدَّامِهِ وَعَسْكَرِيّاً تَقِيّاً مِنَ الَّذِينَ كَانُوا يُلاَزِمُونَهُ
\par 8 وَأَخْبَرَهُمْ بِكُلِّ شَيْءٍ وَأَرْسَلَهُمْ إِلَى يَافَا.
\par 9 ثُمَّ فِي الْغَدِ فِيمَا هُمْ يُسَافِرُونَ وَيَقْتَرِبُونَ إِلَى الْمَدِينَةِ صَعِدَ بُطْرُسُ عَلَى السَّطْحِ لِيُصَلِّيَ نَحْوَ السَّاعَةِ السَّادِسَةِ.
\par 10 فَجَاعَ كَثِيراً وَاشْتَهَى أَنْ يَأْكُلَ. وَبَيْنَمَا هُمْ يُهَيِّئُونَ لَهُ وَقَعَتْ عَلَيْهِ غَيْبَةٌ
\par 11 فَرَأَى السَّمَاءَ مَفْتُوحَةً وَإِنَاءً نَازِلاً عَلَيْهِ مِثْلَ مُلاَءَةٍ عَظِيمَةٍ مَرْبُوطَةٍ بِأَرْبَعَةِ أَطْرَافٍ وَمُدَلاَّةٍ عَلَى الأَرْضِ.
\par 12 وَكَانَ فِيهَا كُلُّ دَوَابِّ الأَرْضِ وَالْوُحُوشِ وَالزَّحَّافَاتِ وَطُيُورِ السَّمَاءِ.
\par 13 وَصَارَ إِلَيْهِ صَوْتٌ: «قُمْ يَا بُطْرُسُ اذْبَحْ وَكُلْ».
\par 14 فَقَالَ بُطْرُسُ: «كَلاَّ يَا رَبُّ لأَنِّي لَمْ آكُلْ قَطُّ شَيْئاً دَنِساً أَوْ نَجِساً».
\par 15 فَصَارَ إِلَيْهِ أَيْضاً صَوْتٌ ثَانِيَةً: «مَا طَهَّرَهُ اللهُ لاَ تُدَنِّسْهُ أَنْتَ!»
\par 16 وَكَانَ هَذَا عَلَى ثَلاَثِ مَرَّاتٍ ثُمَّ ارْتَفَعَ الإِنَاءُ أَيْضاً إِلَى السَّمَاءِ.
\par 17 وَإِذْ كَانَ بُطْرُسُ يَرْتَابُ فِي نَفْسِهِ: مَاذَا عَسَى أَنْ تَكُونَ الرُّؤْيَا الَّتِي رَآهَا؟ إِذَا الرِّجَالُ الَّذِينَ أَرْسَلَهُمْ كَرْنِيلِيُوسُ كَانُوا قَدْ سَأَلُوا عَنْ بَيْتِ سِمْعَانَ وَوَقَفُوا عَلَى الْبَابِ
\par 18 وَنَادَوْا يَسْتَخْبِرُونَ: هَلْ سِمْعَانُ الْمُلَقَّبُ بُطْرُسَ نَازِلٌ هُنَاكَ؟
\par 19 وَبَيْنَمَا بُطْرُسُ مُتَفَكِّرٌ فِي الرُّؤْيَا قَالَ لَهُ الرُّوحُ: «هُوَذَا ثَلاَثَةُ رِجَالٍ يَطْلُبُونَكَ.
\par 20 لَكِنْ قُمْ وَانْزِلْ وَاذْهَبْ مَعَهُمْ غَيْرَ مُرْتَابٍ فِي شَيْءٍ لأَنِّي أَنَا قَدْ أَرْسَلْتُهُمْ».
\par 21 فَنَزَلَ بُطْرُسُ إِلَى الرِّجَالِ الَّذِينَ أَرْسَلَهُمْ إِلَيْهِ كَرْنِيلِيُوسُ وَقَالَ: «هَا أَنَا الَّذِي تَطْلُبُونَهُ. مَا هُوَ السَّبَبُ الَّذِي حَضَرْتُمْ لأَجْلِهِ؟»
\par 22 فَقَالُوا: «إِنَّ كَرْنِيلِيُوسَ قَائِدَ مِئَةٍ رَجُلاً بَارّاً وَخَائِفَ اللهِ وَمَشْهُوداً لَهُ مِنْ كُلِّ أُمَّةِ الْيَهُودِ أُوحِيَ إِلَيْهِ بِمَلاَكٍ مُقَدَّسٍ أَنْ يَسْتَدْعِيَكَ إِلَى بَيْتِهِ وَيَسْمَعَ مِنْكَ كَلاَماً».
\par 23 فَدَعَاهُمْ إِلَى دَاخِلٍ وَأَضَافَهُمْ. ثُمَّ فِي الْغَدِ خَرَجَ بُطْرُسُ مَعَهُمْ وَأُنَاسٌ مِنَ الإِخْوَةِ الَّذِينَ مِنْ يَافَا رَافَقُوهُ.
\par 24 وَفِي الْغَدِ دَخَلُوا قَيْصَرِيَّةَ. وَأَمَّا كَرْنِيلِيُوسُ فَكَانَ يَنْتَظِرُهُمْ وَقَدْ دَعَا أَنْسِبَاءَهُ وَأَصْدِقَاءَهُ الأَقْرَبِينَ.
\par 25 وَلَمَّا دَخَلَ بُطْرُسُ اسْتَقْبَلَهُ كَرْنِيلِيُوسُ وَسَجَدَ وَاقِعاً عَلَى قَدَمَيْهِ.
\par 26 فَأَقَامَهُ بُطْرُسُ قَائِلاً: «قُمْ أَنَا أَيْضاً إِنْسَانٌ».
\par 27 ثُمَّ دَخَلَ وَهُوَ يَتَكَلَّمُ مَعَهُ وَوَجَدَ كَثِيرِينَ مُجْتَمِعِينَ.
\par 28 فَقَالَ لَهُمْ: «أَنْتُمْ تَعْلَمُونَ كَيْفَ هُوَ مُحَرَّمٌ عَلَى رَجُلٍ يَهُودِيٍّ أَنْ يَلْتَصِقَ بِأَحَدٍ أَجْنَبِيٍّ أَوْ يَأْتِيَ إِلَيْهِ. وَأَمَّا أَنَا فَقَدْ أَرَانِي اللهُ أَنْ لاَ أَقُولَ عَنْ إِنْسَانٍ مَا إِنَّهُ دَنِسٌ أَوْ نَجِسٌ.
\par 29 فَلِذَلِكَ جِئْتُ مِنْ دُونِ مُنَاقَضَةٍ إِذِ اسْتَدْعَيْتُمُونِي. فَأَسْتَخْبِرُكُمْ: لأَيِّ سَبَبٍ اسْتَدْعَيْتُمُونِي؟».
\par 30 فَقَالَ كَرْنِيلِيُوسُ: «مُنْذُ أَرْبَعَةِ أَيَّامٍ إِلَى هَذِهِ السَّاعَةِ كُنْتُ صَائِماً. وَفِي السَّاعَةِ التَّاسِعَةِ كُنْتُ أُصَلِّي فِي بَيْتِي وَإِذَا رَجُلٌ قَدْ وَقَفَ أَمَامِي بِلِبَاسٍ لاَمِعٍ
\par 31 وَقَالَ: يَا كَرْنِيلِيُوسُ سُمِعَتْ صَلاَتُكَ وَذُكِرَتْ صَدَقَاتُكَ أَمَامَ اللهِ.
\par 32 فَأَرْسِلْ إِلَى يَافَا وَاسْتَدْعِ سِمْعَانَ الْمُلَقَّبَ بُطْرُسَ. إِنَّهُ نَازِلٌ فِي بَيْتِ سِمْعَانَ رَجُلٍ دَبَّاغٍ عِنْدَ الْبَحْرِ. فَهُوَ مَتَى جَاءَ يُكَلِّمُكَ.
\par 33 فَأَرْسَلْتُ إِلَيْكَ حَالاً. وَأَنْتَ فَعَلْتَ حَسَناً إِذْ جِئْتَ. وَالآنَ نَحْنُ جَمِيعاً حَاضِرُونَ أَمَامَ اللهِ لِنَسْمَعَ جَمِيعَ مَا أَمَرَكَ بِهِ اللهُ».
\par 34 فَقَالَ بُطْرُسُ: «بِالْحَقِّ أَنَا أَجِدُ أَنَّ اللهَ لاَ يَقْبَلُ الْوُجُوهَ.
\par 35 بَلْ فِي كُلِّ أُمَّةٍ الَّذِي يَتَّقِيهِ وَيَصْنَعُ الْبِرَّ مَقْبُولٌ عِنْدَهُ.
\par 36 الْكَلِمَةُ الَّتِي أَرْسَلَهَا إِلَى بَنِي إِسْرَائِيلَ يُبَشِّرُ بِالسَّلاَمِ بِيَسُوعَ الْمَسِيحِ. هَذَا هُوَ رَبُّ الْكُلِّ.
\par 37 أَنْتُمْ تَعْلَمُونَ الأَمْرَ الَّذِي صَارَ فِي كُلِّ الْيَهُودِيَّةِ مُبْتَدِئاً مِنَ الْجَلِيلِ بَعْدَ الْمَعْمُودِيَّةِ الَّتِي كَرَزَ بِهَا يُوحَنَّا.
\par 38 يَسُوعُ الَّذِي مِنَ النَّاصِرَةِ كَيْفَ مَسَحَهُ اللهُ بِالرُّوحِ الْقُدُسِ وَالْقُوَّةِ الَّذِي جَالَ يَصْنَعُ خَيْراً وَيَشْفِي جَمِيعَ الْمُتَسَلِّطِ عَلَيْهِمْ إِبْلِيسُ لأَنَّ اللهَ كَانَ مَعَهُ.
\par 39 وَنَحْنُ شُهُودٌ بِكُلِّ مَا فَعَلَ فِي كُورَةِ الْيَهُودِيَّةِ وَفِي أُورُشَلِيمَ. الَّذِي أَيْضاً قَتَلُوهُ مُعَلِّقِينَ إِيَّاهُ عَلَى خَشَبَةٍ.
\par 40 هَذَا أَقَامَهُ اللهُ فِي الْيَوْمِ الثَّالِثِ وَأَعْطَى أَنْ يَصِيرَ ظَاهِراً
\par 41 لَيْسَ لِجَمِيعِ الشَّعْبِ بَلْ لِشُهُودٍ سَبَقَ اللهُ فَانْتَخَبَهُمْ. لَنَا نَحْنُ الَّذِينَ أَكَلْنَا وَشَرِبْنَا مَعَهُ بَعْدَ قِيَامَتِهِ مِنَ الأَمْوَاتِ.
\par 42 وَأَوْصَانَا أَنْ نَكْرِزَ لِلشَّعْبِ وَنَشْهَدَ بِأَنَّ هَذَا هُوَ الْمُعَيَّنُ مِنَ اللهِ دَيَّاناً لِلأَحْيَاءِ وَالأَمْوَاتِ.
\par 43 لَهُ يَشْهَدُ جَمِيعُ الأَنْبِيَاءِ أَنَّ كُلَّ مَنْ يُؤْمِنُ بِهِ يَنَالُ بِاسْمِهِ غُفْرَانَ الْخَطَايَا».
\par 44 فَبَيْنَمَا بُطْرُسُ يَتَكَلَّمُ بِهَذِهِ الْأُمُورِ حَلَّ الرُّوحُ الْقُدُسُ عَلَى جَمِيعِ الَّذِينَ كَانُوا يَسْمَعُونَ الْكَلِمَةَ.
\par 45 فَانْدَهَشَ الْمُؤْمِنُونَ الَّذِينَ مِنْ أَهْلِ الْخِتَانِ كُلُّ مَنْ جَاءَ مَعَ بُطْرُسَ لأَنَّ مَوْهِبَةَ الرُّوحِ الْقُدُسِ قَدِ انْسَكَبَتْ عَلَى الْأُمَمِ أَيْضاً -
\par 46 لأَنَّهُمْ كَانُوا يَسْمَعُونَهُمْ يَتَكَلَّمُونَ بِأَلْسِنَةٍ وَيُعَظِّمُونَ اللهَ. حِينَئِذٍ قَالَ بُطْرُسُ:
\par 47 «أَتُرَى يَسْتَطِيعُ أَحَدٌ أَنْ يَمْنَعَ الْمَاءَ حَتَّى لاَ يَعْتَمِدَ هَؤُلاَءِ الَّذِينَ قَبِلُوا الرُّوحَ الْقُدُسَ كَمَا نَحْنُ أَيْضاً؟»
\par 48 وَأَمَرَ أَنْ يَعْتَمِدُوا بِاسْمِ الرَّبِّ. حِينَئِذٍ سَأَلُوهُ أَنْ يَمْكُثَ أَيَّاماً.

\chapter{11}

\par 1 فَسَمِعَ الرُّسُلُ وَالإِخْوَةُ الَّذِينَ كَانُوا فِي الْيَهُودِيَّةِ أَنَّ الْأُمَمَ أَيْضاً قَبِلُوا كَلِمَةَ اللهِ.
\par 2 وَلَمَّا صَعِدَ بُطْرُسُ إِلَى أُورُشَلِيمَ خَاصَمَهُ الَّذِينَ مِنْ أَهْلِ الْخِتَانِ
\par 3 قَائِلِينَ: «إِنَّكَ دَخَلْتَ إِلَى رِجَالٍ ذَوِي غُلْفَةٍ وَأَكَلْتَ مَعَهُمْ».
\par 4 فَابْتَدَأَ بُطْرُسُ يَشْرَحُ لَهُمْ بِالتَّتَابُعِ قَائِلاً:
\par 5 «أَنَا كُنْتُ فِي مَدِينَةِ يَافَا أُصَلِّي فَرَأَيْتُ فِي غَيْبَةٍ رُؤْيَا: إِنَاءً نَازِلاً مِثْلَ مُلاَءَةٍ عَظِيمَةٍ مُدَلاَّةٍ بِأَرْبَعَةِ أَطْرَافٍ مِنَ السَّمَاءِ فَأَتَى إِلَيَّ.
\par 6 فَتَفَرَّسْتُ فِيهِ مُتَأَمِّلاً فَرَأَيْتُ دَوَابَّ الأَرْضِ وَالْوُحُوشَ وَالزَّحَّافَاتِ وَطُيُورَ السَّمَاءِ.
\par 7 وَسَمِعْتُ صَوْتاً قَائِلاً لِي: قُمْ يَا بُطْرُسُ اذْبَحْ وَكُلْ.
\par 8 فَقُلْتُ: كَلاَّ يَا رَبُّ لأَنَّهُ لَمْ يَدْخُلْ فَمِي قَطُّ دَنِسٌ أَوْ نَجِسٌ.
\par 9 فَأَجَابَنِي صَوْتٌ ثَانِيَةً مِنَ السَّمَاءِ: مَا طَهَّرَهُ اللهُ لاَ تُنَجِّسْهُ أَنْتَ.
\par 10 وَكَانَ هَذَا عَلَى ثَلاَثِ مَرَّاتٍ ثُمَّ انْتُشِلَ الْجَمِيعُ إِلَى السَّمَاءِ أَيْضاً.
\par 11 وَإِذَا ثَلاَثَةُ رِجَالٍ قَدْ وَقَفُوا لِلْوَقْتِ عِنْدَ الْبَيْتِ الَّذِي كُنْتُ فِيهِ مُرْسَلِينَ إِلَيَّ مِنْ قَيْصَرِيَّةَ.
\par 12 فَقَالَ لِي الرُّوحُ أَنْ أَذْهَبَ مَعَهُمْ غَيْرَ مُرْتَابٍ فِي شَيْءٍ. وَذَهَبَ مَعِي أَيْضاً هَؤُلاَءِ الإِخْوَةُ السِّتَّةُ. فَدَخَلْنَا بَيْتَ الرَّجُلِ
\par 13 فَأَخْبَرَنَا كَيْفَ رَأَى الْمَلاَكَ فِي بَيْتِهِ قَائِماً وَقَائِلاً لَهُ: أَرْسِلْ إِلَى يَافَا رِجَالاً وَاسْتَدْعِ سِمْعَانَ الْمُلَقَّبَ بُطْرُسَ
\par 14 وَهُوَ يُكَلِّمُكَ كَلاَماً بِهِ تَخْلُصُ أَنْتَ وَكُلُّ بَيْتِكَ.
\par 15 فَلَمَّا ابْتَدَأْتُ أَتَكَلَّمُ حَلَّ الرُّوحُ الْقُدُسُ عَلَيْهِمْ كَمَا عَلَيْنَا أَيْضاً فِي الْبَدَاءَةِ.
\par 16 فَتَذَكَّرْتُ كَلاَمَ الرَّبِّ كَيْفَ قَالَ: إِنَّ يُوحَنَّا عَمَّدَ بِمَاءٍ وَأَمَّا أَنْتُمْ فَسَتُعَمَّدُونَ بِالرُّوحِ الْقُدُسِ.
\par 17 فَإِنْ كَانَ اللهُ قَدْ أَعْطَاهُمُ الْمَوْهِبَةَ كَمَا لَنَا أَيْضاً بِالسَّوِيَّةِ مُؤْمِنِينَ بِالرَّبِّ يَسُوعَ الْمَسِيحِ فَمَنْ أَنَا؟ أَقَادِرٌ أَنْ أَمْنَعَ اللهَ؟».
\par 18 فَلَمَّا سَمِعُوا ذَلِكَ سَكَتُوا وَكَانُوا يُمَجِّدُونَ اللهَ قَائِلِينَ: «إِذاً أَعْطَى اللهُ الْأُمَمَ أَيْضاً التَّوْبَةَ لِلْحَيَاةِ!».
\par 19 أَمَّا الَّذِينَ تَشَتَّتُوا مِنْ جَرَّاءِ الضِّيقِ الَّذِي حَصَلَ بِسَبَبِ إِسْتِفَانُوسَ فَاجْتَازُوا إِلَى فِينِيقِيَةَ وَقُبْرُسَ وَأَنْطَاكِيَةَ وَهُمْ لاَ يُكَلِّمُونَ أَحَداً بِالْكَلِمَةِ إِلاَّ الْيَهُودَ فَقَطْ.
\par 20 وَلَكِنْ كَانَ مِنْهُمْ قَوْمٌ وَهُمْ رِجَالٌ قُبْرُسِيُّونَ وَقَيْرَوَانِيُّونَ الَّذِينَ لَمَّا دَخَلُوا أَنْطَاكِيَةَ كَانُوا يُخَاطِبُونَ الْيُونَانِيِّينَ مُبَشِّرِينَ بِالرَّبِّ يَسُوعَ.
\par 21 وَكَانَتْ يَدُ الرَّبِّ مَعَهُمْ فَآمَنَ عَدَدٌ كَثِيرٌ وَرَجَعُوا إِلَى الرَّبِّ.
\par 22 فَسُمِعَ الْخَبَرُ عَنْهُمْ فِي آذَانِ الْكَنِيسَةِ الَّتِي فِي أُورُشَلِيمَ فَأَرْسَلُوا بَرْنَابَا لِكَيْ يَجْتَازَ إِلَى أَنْطَاكِيَةَ.
\par 23 الَّذِي لَمَّا أَتَى وَرَأَى نِعْمَةَ اللهِ فَرِحَ وَوَعَظَ الْجَمِيعَ أَنْ يَثْبُتُوا فِي الرَّبِّ بِعَزْمِ الْقَلْبِ
\par 24 لأَنَّهُ كَانَ رَجُلاً صَالِحاً وَمُمْتَلِئاً مِنَ الرُّوحِ الْقُدُسِ وَالإِيمَانِ. فَانْضَمَّ إِلَى الرَّبِّ جَمْعٌ غَفِيرٌ.
\par 25 ثُمَّ خَرَجَ بَرْنَابَا إِلَى طَرْسُوسَ لِيَطْلُبَ شَاوُلَ. وَلَمَّا وَجَدَهُ جَاءَ بِهِ إِلَى أَنْطَاكِيَةَ.
\par 26 فَحَدَثَ أَنَّهُمَا اجْتَمَعَا فِي الْكَنِيسَةِ سَنَةً كَامِلَةً وَعَلَّمَا جَمْعاً غَفِيراً. وَدُعِيَ التَّلاَمِيذُ «مَسِيحِيِّينَ» فِي أَنْطَاكِيَةَ أَوَّلاً.
\par 27 وَفِي تِلْكَ الأَيَّامِ انْحَدَرَ أَنْبِيَاءُ مِنْ أُورُشَلِيمَ إِلَى أَنْطَاكِيَةَ.
\par 28 وَقَامَ وَاحِدٌ مِنْهُمُ اسْمُهُ أَغَابُوسُ وَأَشَارَ بِالرُّوحِ أَنَّ جُوعاً عَظِيماً كَانَ عَتِيداً أَنْ يَصِيرَ عَلَى جَمِيعِ الْمَسْكُونَةِ - الَّذِي صَارَ أَيْضاً فِي أَيَّامِ كُلُودِيُوسَ قَيْصَرَ.
\par 29 فَحَتَمَ التَّلاَمِيذُ حَسْبَمَا تَيَسَّرَ لِكُلٍّ مِنْهُمْ أَنْ يُرْسِلَ كُلُّ وَاحِدٍ شَيْئاً خِدْمَةً إِلَى الإِخْوَةِ السَّاكِنِينَ فِي الْيَهُودِيَّةِ.
\par 30 فَفَعَلُوا ذَلِكَ مُرْسِلِينَ إِلَى الْمَشَايِخِ بِيَدِ بَرْنَابَا وَشَاوُلَ.

\chapter{12}

\par 1 وَفِي ذَلِكَ الْوَقْتِ مَدَّ هِيرُودُسُ الْمَلِكُ يَدَيْهِ لِيُسِيئَ إِلَى أُنَاسٍ مِنَ الْكَنِيسَةِ
\par 2 فَقَتَلَ يَعْقُوبَ أَخَا يُوحَنَّا بِالسَّيْفِ.
\par 3 وَإِذْ رَأَى أَنَّ ذَلِكَ يُرْضِي الْيَهُودَ عَادَ فَقَبَضَ عَلَى بُطْرُسَ أَيْضاً. وَكَانَتْ أَيَّامُ الْفَطِيرِ.
\par 4 وَلَمَّا أَمْسَكَهُ وَضَعَهُ فِي السِّجْنِ مُسَلِّماً إِيَّاهُ إِلَى أَرْبَعَةِ أَرَابِعَ مِنَ الْعَسْكَرِ لِيَحْرُسُوهُ نَاوِياً أَنْ يُقَدِّمَهُ بَعْدَ الْفِصْحِ إِلَى الشَّعْبِ.
\par 5 فَكَانَ بُطْرُسُ مَحْرُوساً فِي السِّجْنِ وَأَمَّا الْكَنِيسَةُ فَكَانَتْ تَصِيرُ مِنْهَا صَلاَةٌ بِلَجَاجَةٍ إِلَى اللهِ مِنْ أَجْلِهِ.
\par 6 وَلَمَّا كَانَ هِيرُودُسُ مُزْمِعاً أَنْ يُقَدِّمَهُ كَانَ بُطْرُسُ فِي تِلْكَ اللَّيْلَةِ نَائِماً بَيْنَ عَسْكَرِيَّيْنِ مَرْبُوطاً بِسِلْسِلَتَيْنِ وَكَانَ قُدَّامَ الْبَابِ حُرَّاسٌ يَحْرُسُونَ السِّجْنَ.
\par 7 وَإِذَا مَلاَكُ الرَّبِّ أَقْبَلَ وَنُورٌ أَضَاءَ فِي الْبَيْتِ فَضَرَبَ جَنْبَ بُطْرُسَ وَأَيْقَظَهُ قَائِلاً: «قُمْ عَاجِلاً». فَسَقَطَتِ السِّلْسِلَتَانِ مِنْ يَدَيْهِ.
\par 8 وَقَالَ لَهُ الْمَلاَكُ: «تَمَنْطَقْ وَالْبَسْ نَعْلَيْكَ». فَفَعَلَ هَكَذَا. فَقَالَ لَهُ: «الْبَسْ رِدَاءَكَ وَاتْبَعْنِي».
\par 9 فَخَرَجَ يَتْبَعُهُ - وَكَانَ لاَ يَعْلَمُ أَنَّ الَّذِي جَرَى بِوَاسِطَةِ الْمَلاَكِ هُوَ حَقِيقِيٌّ بَلْ يَظُنُّ أَنَّهُ يَنْظُرُ رُؤْيَا.
\par 10 فَجَازَا الْمَحْرَسَ الأَوَّلَ وَالثَّانِيَ وَأَتَيَا إِلَى بَابِ الْحَدِيدِ الَّذِي يُؤَدِّي إِلَى الْمَدِينَةِ فَانْفَتَحَ لَهُمَا مِنْ ذَاتِهِ فَخَرَجَا وَتَقَدَّمَا زُقَاقاً وَاحِداً وَلِلْوَقْتِ فَارَقَهُ الْمَلاَكُ.
\par 11 فَقَالَ بُطْرُسُ وَهُوَ قَدْ رَجَعَ إِلَى نَفْسِهِ: «الآنَ عَلِمْتُ يَقِيناً أَنَّ الرَّبَّ أَرْسَلَ مَلاَكَهُ وَأَنْقَذَنِي مِنْ يَدِ هِيرُودُسَ وَمِنْ كُلِّ انْتِظَارِ شَعْبِ الْيَهُودِ».
\par 12 ثُمَّ جَاءَ وَهُوَ مُنْتَبِهٌ إِلَى بَيْتِ مَرْيَمَ أُمِّ يُوحَنَّا الْمُلَقَّبِ مَرْقُسَ حَيْثُ كَانَ كَثِيرُونَ مُجْتَمِعِينَ وَهُمْ يُصَلُّونَ.
\par 13 فَلَمَّا قَرَعَ بُطْرُسُ بَابَ الدِّهْلِيزِ جَاءَتْ جَارِيَةٌ اسْمُهَا رَوْدَا لِتَسْمَعَ.
\par 14 فَلَمَّا عَرَفَتْ صَوْتَ بُطْرُسَ لَمْ تَفْتَحِ الْبَابَ مِنَ الْفَرَحِ بَلْ رَكَضَتْ إِلَى دَاخِلٍ وَأَخْبَرَتْ أَنَّ بُطْرُسَ وَاقِفٌ قُدَّامَ الْبَابِ.
\par 15 فَقَالُوا لَهَا: «أَنْتِ تَهْذِينَ!». وَأَمَّا هِيَ فَكَانَتْ تُؤَكِّدُ أَنَّ هَكَذَا هُوَ. فَقَالُوا: «إِنَّهُ مَلاَكُهُ!».
\par 16 وَأَمَّا بُطْرُسُ فَلَبِثَ يَقْرَعُ. فَلَمَّا فَتَحُوا وَرَأَوْهُ انْدَهَشُوا.
\par 17 فَأَشَارَ إِلَيْهِمْ بِيَدِهِ لِيَسْكُتُوا وَحَدَّثَهُمْ كَيْفَ أَخْرَجَهُ الرَّبُّ مِنَ السِّجْنِ. وَقَالَ: «أَخْبِرُوا يَعْقُوبَ وَالإِخْوَةَ بِهَذَا». ثُمَّ خَرَجَ وَذَهَبَ إِلَى مَوْضِعٍ آخَرَ.
\par 18 فَلَمَّا صَارَ النَّهَارُ حَصَلَ اضْطِرَابٌ لَيْسَ بِقَلِيلٍ بَيْنَ الْعَسْكَرِ: تُرَى مَاذَا جَرَى لِبُطْرُسَ؟
\par 19 وَأَمَّا هِيرُودُسُ فَلَمَّا طَلَبَهُ وَلَمْ يَجِدْهُ فَحَصَ الْحُرَّاسَ وَأَمَرَ أَنْ يَنْقَادُوا إِلَى الْقَتْلِ. ثُمَّ نَزَلَ مِنَ الْيَهُودِيَّةِ إِلَى قَيْصَرِيَّةَ وَأَقَامَ هُنَاكَ.
\par 20 وَكَانَ هِيرُودُسُ سَاخِطاً عَلَى الصُّورِيِّينَ وَالصَّيْدَاوِيِّينَ فَحَضَرُوا إِلَيْهِ بِنَفْسٍ وَاحِدَةٍ وَاسْتَعْطَفُوا بَلاَسْتُسَ النَّاظِرَ عَلَى مَضْجَعِ الْمَلِكِ ثُمَّ صَارُوا يَلْتَمِسُونَ الْمُصَالَحَةَ لأَنَّ كُورَتَهُمْ تَقْتَاتُ مِنْ كُورَةِ الْمَلِكِ.
\par 21 فَفِي يَوْمٍ مُعَيَّنٍ لَبِسَ هِيرُودُسُ الْحُلَّةَ الْمُلُوكِيَّةَ وَجَلَسَ عَلَى كُرْسِيِّ الْمُلْكِ وَجَعَلَ يُخَاطِبُهُمْ.
\par 22 فَصَرَخَ الشَّعْبُ: «هَذَا صَوْتُ إِلَهٍ لاَ صَوْتُ إِنْسَانٍ!»
\par 23 فَفِي الْحَالِ ضَرَبَهُ مَلاَكُ الرَّبِّ لأَنَّهُ لَمْ يُعْطِ الْمَجْدَ لِلَّهِ فَصَارَ يَأْكُلُهُ الدُّودُ وَمَاتَ.
\par 24 وَأَمَّا كَلِمَةُ اللهِ فَكَانَتْ تَنْمُو وَتَزِيدُ.
\par 25 وَرَجَعَ بَرْنَابَا وَشَاوُلُ مِنْ أُورُشَلِيمَ بَعْدَ مَا كَمَّلاَ الْخِدْمَةَ وَأَخَذَا مَعَهُمَا يُوحَنَّا الْمُلَقَّبَ مَرْقُسَ.

\chapter{13}

\par 1 وَكَانَ فِي أَنْطَاكِيَةَ فِي الْكَنِيسَةِ هُنَاكَ أَنْبِيَاءُ وَمُعَلِّمُونَ: بَرْنَابَا وَسِمْعَانُ الَّذِي يُدْعَى نِيجَرَ وَلُوكِيُوسُ الْقَيْرَوَانِيُّ وَمَنَايِنُ الَّذِي تَرَبَّى مَعَ هِيرُودُسَ رَئِيسِ الرُّبْعِ وَشَاوُلُ.
\par 2 وَبَيْنَمَا هُمْ يَخْدِمُونَ الرَّبَّ وَيَصُومُونَ قَالَ الرُّوحُ الْقُدُسُ: «أَفْرِزُوا لِي بَرْنَابَا وَشَاوُلَ لِلْعَمَلِ الَّذِي دَعَوْتُهُمَا إِلَيْهِ».
\par 3 فَصَامُوا حِينَئِذٍ وَصَلُّوا وَوَضَعُوا عَلَيْهِمَا الأَيَادِيَ ثُمَّ أَطْلَقُوهُمَا.
\par 4 فَهَذَانِ إِذْ أُرْسِلاَ مِنَ الرُّوحِ الْقُدُسِ انْحَدَرَا إِلَى سَلُوكِيَةَ وَمِنْ هُنَاكَ سَافَرَا فِي الْبَحْرِ إِلَى قُبْرُسَ.
\par 5 وَلَمَّا صَارَا فِي سَلاَمِيسَ نَادَيَا بِكَلِمَةِ اللهِ فِي مَجَامِعِ الْيَهُودِ. وَكَانَ مَعَهُمَا يُوحَنَّا خَادِماً.
\par 6 وَلَمَّا اجْتَازَا الْجَزِيرَةَ إِلَى بَافُوسَ وَجَدَا رَجُلاً سَاحِراً نَبِيّاً كَذَّاباً يَهُودِيّاً اسْمُهُ بَارْيَشُوعُ
\par 7 كَانَ مَعَ الْوَالِي سَرْجِيُوسَ بُولُسَ وَهُوَ رَجُلٌ فَهِيمٌ. فَهَذَا دَعَا بَرْنَابَا وَشَاوُلَ وَالْتَمَسَ أَنْ يَسْمَعَ كَلِمَةَ اللهِ.
\par 8 فَقَاوَمَهُمَا عَلِيمٌ السَّاحِرُ لأَنْ هَكَذَا يُتَرْجَمُ اسْمُهُ طَالِباً أَنْ يُفْسِدَ الْوَالِيَ عَنِ الإِيمَانِ.
\par 9 وَأَمَّا شَاوُلُ الَّذِي هُوَ بُولُسُ أَيْضاً فَامْتَلأَ مِنَ الرُّوحِ الْقُدُسِ وَشَخَصَ إِلَيْهِ
\par 10 وَقَالَ: «أَيُّهَا الْمُمْتَلِئُ كُلَّ غِشٍّ وَكُلَّ خُبْثٍ! يَا ابْنَ إِبْلِيسَ! يَا عَدُوَّ كُلِّ بِرٍّ! أَلاَ تَزَالُ تُفْسِدُ سُبُلَ اللهِ الْمُسْتَقِيمَةَ؟
\par 11 فَالآنَ هُوَذَا يَدُ الرَّبِّ عَلَيْكَ فَتَكُونُ أَعْمَى لاَ تُبْصِرُ الشَّمْسَ إِلَى حِينٍ». فَفِي الْحَالِ سَقَطَ عَلَيْهِ ضَبَابٌ وَظُلْمَةٌ فَجَعَلَ يَدُورُ مُلْتَمِساً مَنْ يَقُودُهُ بِيَدِهِ.
\par 12 فَالْوَالِي حِينَئِذٍ لَمَّا رَأَى مَا جَرَى آمَنَ مُنْدَهِشاً مِنْ تَعْلِيمِ الرَّبِّ.
\par 13 ثُمَّ أَقْلَعَ بُولُسُ وَمَنْ مَعَهُ مِنْ بَافُوسَ وَأَتَوْا إِلَى بَرْجَةَ بَمْفِيلِيَّةَ. وَأَمَّا يُوحَنَّا فَفَارَقَهُمْ وَرَجَعَ إِلَى أُورُشَلِيمَ.
\par 14 وَأَمَّا هُمْ فَجَازُوا مِنْ بَرْجَةَ وَأَتَوْا إِلَى أَنْطَاكِيَةَ بِيسِيدِيَّةَ وَدَخَلُوا الْمَجْمَعَ يَوْمَ السَّبْتِ وَجَلَسُوا.
\par 15 وَبَعْدَ قِرَاءَةِ النَّامُوسِ وَالأَنْبِيَاءِ أَرْسَلَ إِلَيْهِمْ رُؤَسَاءُ الْمَجْمَعِ قَائِلِينَ: «أَيُّهَا الرِّجَالُ الإِخْوَةُ إِنْ كَانَتْ عِنْدَكُمْ كَلِمَةُ وَعْظٍ لِلشَّعْبِ فَقُولُوا».
\par 16 فَقَامَ بُولُسُ وَأَشَارَ بِيَدِهِ وَقَالَ: «أَيُّهَا الرِّجَالُ الإِسْرَائِيلِيُّونَ وَالَّذِينَ يَتَّقُونَ اللهَ اسْمَعُوا.
\par 17 إِلَهُ شَعْبِ إِسْرَائِيلَ هَذَا اخْتَارَ آبَاءَنَا وَرَفَعَ الشَّعْبَ فِي الْغُرْبَةِ فِي أَرْضِ مِصْرَ وَبِذِرَاعٍ مُرْتَفِعَةٍ أَخْرَجَهُمْ مِنْهَا.
\par 18 وَنَحْوَ مُدَّةِ أَرْبَعِينَ سَنَةً احْتَمَلَ عَوَائِدَهُمْ فِي الْبَرِّيَّةِ.
\par 19 ثُمَّ أَهْلَكَ سَبْعَ أُمَمٍ فِي أَرْضِ كَنْعَانَ وَقَسَمَ لَهُمْ أَرْضَهُمْ بِالْقُرْعَةِ.
\par 20 وَبَعْدَ ذَلِكَ فِي نَحْوِ أَرْبَعِمِئَةٍ وَخَمْسِينَ سَنَةً أَعْطَاهُمْ قُضَاةً حَتَّى صَمُوئِيلَ النَّبِيِّ.
\par 21 وَمِنْ ثَمَّ طَلَبُوا مَلِكاً فَأَعْطَاهُمُ اللهُ شَاوُلَ بْنَ قَيْسٍ رَجُلاً مِنْ سِبْطِ بِنْيَامِينَ أَرْبَعِينَ سَنَةً.
\par 22 ثُمَّ عَزَلَهُ وَأَقَامَ لَهُمْ دَاوُدَ مَلِكاً الَّذِي شَهِدَ لَهُ أَيْضاً إِذْ قَالَ: وَجَدْتُ دَاوُدَ بْنَ يَسَّى رَجُلاً حَسَبَ قَلْبِي الَّذِي سَيَصْنَعُ كُلَّ مَشِيئَتِي.
\par 23 مِنْ نَسْلِ هَذَا حَسَبَ الْوَعْدِ أَقَامَ اللهُ لِإِسْرَائِيلَ مُخَلِّصاً يَسُوعَ.
\par 24 إِذْ سَبَقَ يُوحَنَّا فَكَرَزَ قَبْلَ مَجِيئِهِ بِمَعْمُودِيَّةِ التَّوْبَةِ لِجَمِيعِ شَعْبِ إِسْرَائِيلَ.
\par 25 وَلَمَّا صَارَ يُوحَنَّا يُكَمِّلُ سَعْيَهُ جَعَلَ يَقُولُ: «مَنْ تَظُنُّونَ أَنِّي أَنَا؟ لَسْتُ أَنَا إِيَّاهُ لَكِنْ هُوَذَا يَأْتِي بَعْدِي الَّذِي لَسْتُ مُسْتَحِقّاً أَنْ أَحُلَّ حِذَاءَ قَدَمَيْهِ.
\par 26 «أَيُّهَا الرِّجَالُ الإِخْوَةُ بَنِي جِنْسِ إِبْرَاهِيمَ وَالَّذِينَ بَيْنَكُمْ يَتَّقُونَ اللهَ إِلَيْكُمْ أُرْسِلَتْ كَلِمَةُ هَذَا الْخَلاَصِ.
\par 27 لأَنَّ السَّاكِنِينَ فِي أُورُشَلِيمَ وَرُؤَسَاءَهُمْ لَمْ يَعْرِفُوا هَذَا. وَأَقْوَالُ الأَنْبِيَاءِ الَّتِي تُقْرَأُ كُلَّ سَبْتٍ تَمَّمُوهَا إِذْ حَكَمُوا عَلَيْهِ.
\par 28 وَمَعْ أَنَّهُمْ لَمْ يَجِدُوا عِلَّةً وَاحِدَةً لِلْمَوْتِ طَلَبُوا مِنْ بِيلاَطُسَ أَنْ يُقْتَلَ.
\par 29 وَلَمَّا تَمَّمُوا كُلَّ مَا كُتِبَ عَنْهُ أَنْزَلُوهُ عَنِ الْخَشَبَةِ وَوَضَعُوهُ فِي قَبْرٍ.
\par 30 وَلَكِنَّ اللهَ أَقَامَهُ مِنَ الأَمْوَاتِ.
\par 31 وَظَهَرَ أَيَّاماً كَثِيرَةً لِلَّذِينَ صَعِدُوا مَعَهُ مِنَ الْجَلِيلِ إِلَى أُورُشَلِيمَ الَّذِينَ هُمْ شُهُودُهُ عِنْدَ الشَّعْبِ.
\par 32 وَنَحْنُ نُبَشِّرُكُمْ بِالْمَوْعِدِ الَّذِي صَارَ لِآبَائِنَا
\par 33 إِنَّ اللهَ قَدْ أَكْمَلَ هَذَا لَنَا نَحْنُ أَوْلاَدَهُمْ إِذْ أَقَامَ يَسُوعَ كَمَا هُوَ مَكْتُوبٌ أَيْضاً فِي الْمَزْمُورِ الثَّانِي: أَنْتَ ابْنِي أَنَا الْيَوْمَ وَلَدْتُكَ.
\par 34 إِنَّهُ أَقَامَهُ مِنَ الأَمْوَاتِ غَيْرَ عَتِيدٍ أَنْ يَعُودَ أَيْضاً إِلَى فَسَادٍ فَهَكَذَا قَالَ: إِنِّي سَأُعْطِيكُمْ مَرَاحِمَ دَاوُدَ الصَّادِقَةَ.
\par 35 وَلِذَلِكَ قَالَ أَيْضاً فِي مَزْمُورٍ آخَرَ:لَنْ تَدَعَ قُدُّوسَكَ يَرَى فَسَاداً.
\par 36 لأَنَّ دَاوُدَ بَعْدَ مَا خَدَمَ جِيلَهُ بِمَشُورَةِ اللهِ رَقَدَ وَانْضَمَّ إِلَى آبَائِهِ وَرَأَى فَسَاداً.
\par 37 وَأَمَّا الَّذِي أَقَامَهُ اللهُ فَلَمْ يَرَ فَسَاداً.
\par 38 فَلْيَكُنْ مَعْلُوماً عِنْدَكُمْ أَيُّهَا الرِّجَالُ الإِخْوَةُ أَنَّهُ بِهَذَا يُنَادَى لَكُمْ بِغُفْرَانِ الْخَطَايَا
\par 39 وَبِهَذَا يَتَبَرَّرُ كُلُّ مَنْ يُؤْمِنُ مِنْ كُلِّ مَا لَمْ تَقْدِرُوا أَنْ تَتَبَرَّرُوا مِنْهُ بِنَامُوسِ مُوسَى.
\par 40 فَانْظُرُوا لِئَلاَّ يَأْتِيَ عَلَيْكُمْ مَا قِيلَ فِي الأَنْبِيَاءِ:
\par 41 اُنْظُرُوا أَيُّهَا الْمُتَهَاوِنُونَ وَتَعَجَّبُوا وَاهْلِكُوا لأَنَّنِي عَمَلاً أَعْمَلُ فِي أَيَّامِكُمْ عَمَلاً لاَ تُصَدِّقُونَ إِنْ أَخْبَرَكُمْ أَحَدٌ بِهِ».
\par 42 وَبَعْدَمَا خَرَجَ الْيَهُودُ مِنَ الْمَجْمَعِ جَعَلَ الْأُمَمُ يَطْلُبُونَ إِلَيْهِمَا أَنْ يُكَلِّمَاهُمْ بِهَذَا الْكَلاَمِ فِي السَّبْتِ الْقَادِمِ.
\par 43 وَلَمَّا انْفَضَّتِ الْجَمَاعَةُ تَبِعَ كَثِيرُونَ مِنَ الْيَهُودِ وَالدُّخَلاَءِ الْمُتَعَبِّدِينَ بُولُسَ وَبَرْنَابَا اللَّذَيْنِ كَانَا يُكَلِّمَانِهِمْ وَيُقْنِعَانِهِمْ أَنْ يَثْبُتُوا فِي نِعْمَةِ اللهِ.
\par 44 وَفِي السَّبْتِ التَّالِي اجْتَمَعَتْ كُلُّ الْمَدِينَةِ تَقْرِيباً لِتَسْمَعَ كَلِمَةَ اللهِ.
\par 45 فَلَمَّا رَأَى الْيَهُودُ الْجُمُوعَ امْتَلأُوا غَيْرَةً وَجَعَلُوا يُقَاوِمُونَ مَا قَالَهُ بُولُسُ مُنَاقِضِينَ وَمُجَدِّفِينَ.
\par 46 فَجَاهَرَ بُولُسُ وَبَرْنَابَا وَقَالاَ: «كَانَ يَجِبُ أَنْ تُكَلَّمُوا أَنْتُمْ أَوَّلاً بِكَلِمَةِ اللهِ وَلَكِنْ إِذْ دَفَعْتُمُوهَا عَنْكُمْ وَحَكَمْتُمْ أَنَّكُمْ غَيْرُ مُسْتَحِقِّينَ لِلْحَيَاةِ الأَبَدِيَّةِ هُوَذَا نَتَوَجَّهُ إِلَى الْأُمَمِ.
\par 47 لأَنْ هَكَذَا أَوْصَانَا الرَّبُّ: قَدْ أَقَمْتُكَ نُوراً لِلْأُمَمِ لِتَكُونَ أَنْتَ خَلاَصاً إِلَى أَقْصَى الأَرْضِ».
\par 48 فَلَمَّا سَمِعَ الْأُمَمُ ذَلِكَ كَانُوا يَفْرَحُونَ وَيُمَجِّدُونَ كَلِمَةَ الرَّبِّ وَآمَنَ جَمِيعُ الَّذِينَ كَانُوا مُعَيَّنِينَ لِلْحَيَاةِ الأَبَدِيَّةِ
\par 49 وَانْتَشَرَتْ كَلِمَةُ الرَّبِّ فِي كُلِّ الْكُورَةِ.
\par 50 وَلَكِنَّ الْيَهُودَ حَرَّكُوا النِّسَاءَ الْمُتَعَبِّدَاتِ الشَّرِيفَاتِ وَوُجُوهَ الْمَدِينَةِ وَأَثَارُوا اضْطِهَاداً عَلَى بُولُسَ وَبَرْنَابَا وَأَخْرَجُوهُمَا مِنْ تُخُومِهِمْ.
\par 51 أَمَّا هُمَا فَنَفَضَا غُبَارَ أَرْجُلِهِمَا عَلَيْهِمْ وَأَتَيَا إِلَى إِيقُونِيَةَ.
\par 52 وَأَمَّا التَّلاَمِيذُ فَكَانُوا يَمْتَلِئُونَ مِنَ الْفَرَحِ وَالرُّوحِ الْقُدُسِ.

\chapter{14}

\par 1 وَحَدَثَ فِي إِيقُونِيَةَ أَنَّهُمَا دَخَلاَ مَعاً إِلَى مَجْمَعِ الْيَهُودِ وَتَكَلَّمَا حَتَّى آمَنَ جُمْهُورٌ كَثِيرٌ مِنَ الْيَهُودِ وَالْيُونَانِيِّينَ.
\par 2 وَلَكِنَّ الْيَهُودَ غَيْرَ الْمُؤْمِنِينَ غَرُّوا وَأَفْسَدُوا نُفُوسَ الْأُمَمِ عَلَى الإِخْوَةِ.
\par 3 فَأَقَامَا زَمَاناً طَوِيلاً يُجَاهِرَانِ بِالرَّبِّ الَّذِي كَانَ يَشْهَدُ لِكَلِمَةِ نِعْمَتِهِ وَيُعْطِي أَنْ تُجْرَى آيَاتٌ وَعَجَائِبُ عَلَى أَيْدِيهِمَا.
\par 4 فَانْشَقَّ جُمْهُورُ الْمَدِينَةِ فَكَانَ بَعْضُهُمْ مَعَ الْيَهُودِ وَبَعْضُهُمْ مَعَ الرَّسُولَيْنِ.
\par 5 فَلَمَّا حَصَلَ مِنَ الْأُمَمِ وَالْيَهُودِ مَعَ رُؤَسَائِهِمْ هُجُومٌ لِيَبْغُوا عَلَيْهِمَا وَيَرْجُمُوهُمَا
\par 6 شَعَرَا بِهِ فَهَرَبَا إِلَى مَدِينَتَيْ لِيكَأُونِيَّةَ: لِسْتِرَةَ وَدَرْبَةَ وَإِلَى الْكُورَةِ الْمُحِيطَةِ.
\par 7 وَكَانَا هُنَاكَ يُبَشِّرَانِ.
\par 8 وَكَانَ يَجْلِسُ فِي لِسْتِرَةَ رَجُلٌ عَاجِزُ الرِّجْلَيْنِ مُقْعَدٌ مِنْ بَطْنِ أُمِّهِ وَلَمْ يَمْشِ قَطُّ.
\par 9 هَذَا كَانَ يَسْمَعُ بُولُسَ يَتَكَلَّمُ فَشَخَصَ إِلَيْهِ وَإِذْ رَأَى أَنَّ لَهُ إِيمَاناً لِيُشْفَى
\par 10 قَالَ بِصَوْتٍ عَظِيمٍ: «قُمْ عَلَى رِجْلَيْكَ مُنْتَصِباً». فَوَثَبَ وَصَارَ يَمْشِي.
\par 11 فَالْجُمُوعُ لَمَّا رَأَوْا مَا فَعَلَ بُولُسُ رَفَعُوا صَوْتَهُمْ بِلُغَةِ لِيكَأُونِيَّةَ قَائِلِينَ: «إِنَّ الآلِهَةَ تَشَبَّهُوا بِالنَّاسِ وَنَزَلُوا إِلَيْنَا».
\par 12 فَكَانُوا يَدْعُونَ بَرْنَابَا «زَفْسَ» وَبُولُسَ «هَرْمَسَ» إِذْ كَانَ هُوَ الْمُتَقَدِّمَ فِي الْكَلاَمِ.
\par 13 فَأَتَى كَاهِنُ زَفْسَ الَّذِي كَانَ قُدَّامَ الْمَدِينَةِ بِثِيرَانٍ وَأَكَالِيلَ عِنْدَ الأَبْوَابِ مَعَ الْجُمُوعِ وَكَانَ يُرِيدُ أَنْ يَذْبَحَ.
\par 14 فَلَمَّا سَمِعَ الرَّسُولاَنِ بَرْنَابَا وَبُولُسُ مَزَّقَا ثِيَابَهُمَا وَانْدَفَعَا إِلَى الْجَمْعِ صَارِخَيْنِ:
\par 15 «أَيُّهَا الرِّجَالُ لِمَاذَا تَفْعَلُونَ هَذَا؟ نَحْنُ أَيْضاً بَشَرٌ تَحْتَ آلاَمٍ مِثْلُكُمْ نُبَشِّرُكُمْ أَنْ تَرْجِعُوا مِنْ هَذِهِ الأَبَاطِيلِ إِلَى الإِلَهِ الْحَيِّ الَّذِي خَلَقَ السَّمَاءَ وَالأَرْضَ وَالْبَحْرَ وَكُلَّ مَا فِيهَا
\par 16 الَّذِي فِي الأَجْيَالِ الْمَاضِيَةِ تَرَكَ جَمِيعَ الْأُمَمِ يَسْلُكُونَ فِي طُرُقِهِمْ -
\par 17 مَعَ أَنَّهُ لَمْ يَتْرُكْ نَفْسَهُ بِلاَ شَاهِدٍ - وَهُوَ يَفْعَلُ خَيْراً يُعْطِينَا مِنَ السَّمَاءِ أَمْطَاراً وَأَزْمِنَةً مُثْمِرَةً وَيَمْلأُ قُلُوبَنَا طَعَاماً وَسُرُوراً».
\par 18 وَبِقَوْلِهِمَا هَذَا كَفَّا الْجُمُوعَ بِالْجَهْدِ عَنْ أَنْ يَذْبَحُوا لَهُمَا.
\par 19 ثُمَّ أَتَى يَهُودٌ مِنْ أَنْطَاكِيَةَ وَإِيقُونِيَةَ وَأَقْنَعُوا الْجُمُوعَ فَرَجَمُوا بُولُسَ وَجَرُّوهُ خَارِجَ الْمَدِينَةِ ظَانِّينَ أَنَّهُ قَدْ مَاتَ.
\par 20 وَلَكِنْ إِذْ أَحَاطَ بِهِ التَّلاَمِيذُ قَامَ وَدَخَلَ الْمَدِينَةَ وَفِي الْغَدِ خَرَجَ مَعَ بَرْنَابَا إِلَى دَرْبَةَ.
\par 21 فَبَشَّرَا فِي تِلْكَ الْمَدِينَةِ وَتَلْمَذَا كَثِيرِينَ ثُمَّ رَجَعَا إِلَى لِسْتِرَةَ وَإِيقُونِيَةَ وَأَنْطَاكِيَةَ
\par 22 يُشَدِّدَانِ أَنْفُسَ التَّلاَمِيذِ وَيَعِظَانِهِمْ أَنْ يَثْبُتُوا فِي الإِيمَانِ وَأَنَّهُ بِضِيقَاتٍ كَثِيرَةٍ يَنْبَغِي أَنْ نَدْخُلَ مَلَكُوتَ اللهِ.
\par 23 وَانْتَخَبَا لَهُمْ قُسُوساً فِي كُلِّ كَنِيسَةٍ ثُمَّ صَلَّيَا بِأَصْوَامٍ وَاسْتَوْدَعَاهُمْ لِلرَّبِّ الَّذِي كَانُوا قَدْ آمَنُوا بِهِ.
\par 24 وَلَمَّا اجْتَازَا فِي بِيسِيدِيَّةَ أَتَيَا إِلَى بَمْفِيلِيَّةَ
\par 25 وَتَكَلَّمَا بِالْكَلِمَةِ فِي بَرْجَةَ ثُمَّ نَزَلاَ إِلَى أَتَّالِيَةَ
\par 26 وَمِنْ هُنَاكَ سَافَرَا فِي الْبَحْرِ إِلَى أَنْطَاكِيَةَ حَيْثُ كَانَا قَدْ أُسْلِمَا إِلَى نِعْمَةِ اللهِ لِلْعَمَلِ الَّذِي أَكْمَلاَهُ.
\par 27 وَلَمَّا حَضَرَا وَجَمَعَا الْكَنِيسَةَ أَخْبَرَا بِكُلِّ مَا صَنَعَ اللهُ مَعَهُمَا وَأَنَّهُ فَتَحَ لِلْأُمَمِ بَابَ الإِيمَانِ.
\par 28 وَأَقَامَا هُنَاكَ زَمَاناً لَيْسَ بِقَلِيلٍ مَعَ التَّلاَمِيذِ.

\chapter{15}

\par 1 وَانْحَدَرَ قَوْمٌ مِنَ الْيَهُودِيَّةِ وَجَعَلُوا يُعَلِّمُونَ الإِخْوَةَ أَنَّهُ «إِنْ لَمْ تَخْتَتِنُوا حَسَبَ عَادَةِ مُوسَى لاَ يُمْكِنُكُمْ أَنْ تَخْلُصُوا».
\par 2 فَلَمَّا حَصَلَ لِبُولُسَ وَبَرْنَابَا مُنَازَعَةٌ وَمُبَاحَثَةٌ لَيْسَتْ بِقَلِيلَةٍ مَعَهُمْ رَتَّبُوا أَنْ يَصْعَدَ بُولُسُ وَبَرْنَابَا وَأُنَاسٌ آخَرُونَ مِنْهُمْ إِلَى الرُّسُلِ وَالْمَشَايِخِ إِلَى أُورُشَلِيمَ مِنْ أَجْلِ هَذِهِ الْمَسْأَلَةِ.
\par 3 فَهَؤُلاَءِ بَعْدَ مَا شَيَّعَتْهُمُ الْكَنِيسَةُ اجْتَازُوا فِي فِينِيقِيَةَ وَالسَّامِرَةِ يُخْبِرُونَهُمْ بِرُجُوعِ الْأُمَمِ وَكَانُوا يُسَبِّبُونَ سُرُوراً عَظِيماً لِجَمِيعِ الإِخْوَةِ.
\par 4 وَلَمَّا حَضَرُوا إِلَى أُورُشَلِيمَ قَبِلَتْهُمُ الْكَنِيسَةُ وَالرُّسُلُ وَالْمَشَايِخُ فَأَخْبَرُوهُمْ بِكُلِّ مَا صَنَعَ اللهُ مَعَهُمْ.
\par 5 وَلَكِنْ قَامَ أُنَاسٌ مِنَ الَّذِينَ كَانُوا قَدْ آمَنُوا مِنْ مَذْهَبِ الْفَرِّيسِيِّينَ وَقَالُوا: «إِنَّهُ يَنْبَغِي أَنْ يُخْتَنُوا وَيُوصَوْا بِأَنْ يَحْفَظُوا نَامُوسَ مُوسَى».
\par 6 فَاجْتَمَعَ الرُّسُلُ وَالْمَشَايِخُ لِيَنْظُرُوا فِي هَذَا الأَمْرِ.
\par 7 فَبَعْدَ مَا حَصَلَتْ مُبَاحَثَةٌ كَثِيرَةٌ قَامَ بُطْرُسُ وَقَالَ لَهُمْ: «أَيُّهَا الرِّجَالُ الإِخْوَةُ أَنْتُمْ تَعْلَمُونَ أَنَّهُ مُنْذُ أَيَّامٍ قَدِيمَةٍ اخْتَارَ اللهُ بَيْنَنَا أَنَّهُ بِفَمِي يَسْمَعُ الْأُمَمُ كَلِمَةَ الإِنْجِيلِ وَيُؤْمِنُونَ.
\par 8 وَاللَّهُ الْعَارِفُ الْقُلُوبَ شَهِدَ لَهُمْ مُعْطِياً لَهُمُ الرُّوحَ الْقُدُسَ كَمَا لَنَا أَيْضاً.
\par 9 وَلَمْ يُمَيِّزْ بَيْنَنَا وَبَيْنَهُمْ بِشَيْءٍ إِذْ طَهَّرَ بِالإِيمَانِ قُلُوبَهُمْ.
\par 10 فَالآنَ لِمَاذَا تُجَرِّبُونَ اللهَ بِوَضْعِ نِيرٍ عَلَى عُنُقِ التَّلاَمِيذِ لَمْ يَسْتَطِعْ آبَاؤُنَا وَلاَ نَحْنُ أَنْ نَحْمِلَهُ؟
\par 11 لَكِنْ بِنِعْمَةِ الرَّبِّ يَسُوعَ الْمَسِيحِ نُؤْمِنُ أَنْ نَخْلُصَ كَمَا أُولَئِكَ أَيْضاً».
\par 12 فَسَكَتَ الْجُمْهُورُ كُلُّهُ. وَكَانُوا يَسْمَعُونَ بَرْنَابَا وَبُولُسَ يُحَدِّثَانِ بِجَمِيعِ مَا صَنَعَ اللهُ مِنَ الآيَاتِ وَالْعَجَائِبِ فِي الْأُمَمِ بِوَاسِطَتِهِمْ.
\par 13 وَبَعْدَمَا سَكَتَا قَالَ يَعْقُوبُ: «أَيُّهَا الرِّجَالُ الإِخْوَةُ اسْمَعُونِي.
\par 14 سِمْعَانُ قَدْ أَخْبَرَ كَيْفَ افْتَقَدَ اللهُ أَوَّلاً الْأُمَمَ لِيَأْخُذَ مِنْهُمْ شَعْباً عَلَى اسْمِهِ.
\par 15 وَهَذَا تُوافِقُهُ أَقْوَالُ الأَنْبِيَاءِ كَمَا هُوَ مَكْتُوبٌ:
\par 16 سَأَرْجِعُ بَعْدَ هَذَا وَأَبْنِي أَيْضاً خَيْمَةَ دَاوُدَ السَّاقِطَةَ وَأَبْنِي أَيْضاً رَدْمَهَا وَأُقِيمُهَا ثَانِيَةً
\par 17 لِكَيْ يَطْلُبَ الْبَاقُونَ مِنَ النَّاسِ الرَّبَّ وَجَمِيعُ الْأُمَمِ الَّذِينَ دُعِيَ اسْمِي عَلَيْهِمْ يَقُولُ الرَّبُّ الصَّانِعُ هَذَا كُلَّهُ.
\par 18 مَعْلُومَةٌ عِنْدَ الرَّبِّ مُنْذُ الأَزَلِ جَمِيعُ أَعْمَالِهِ.
\par 19 لِذَلِكَ أَنَا أَرَى أَنْ لاَ يُثَقَّلَ عَلَى الرَّاجِعِينَ إِلَى اللهِ مِنَ الْأُمَمِ
\par 20 بَلْ يُرْسَلْ إِلَيْهِمْ أَنْ يَمْتَنِعُوا عَنْ نَجَاسَاتِ الأَصْنَامِ وَالزِّنَا وَالْمَخْنُوقِ وَالدَّمِ.
\par 21 لأَنَّ مُوسَى مُنْذُ أَجْيَالٍ قَدِيمَةٍ لَهُ فِي كُلِّ مَدِينَةٍ مَنْ يَكْرِزُ بِهِ إِذْ يُقْرَأُ فِي الْمَجَامِعِ كُلَّ سَبْتٍ».
\par 22 حِينَئِذٍ رَأَى الرُّسُلُ وَالْمَشَايِخُ مَعَ كُلِّ الْكَنِيسَةِ أَنْ يَخْتَارُوا رَجُلَيْنِ مِنْهُمْ فَيُرْسِلُوهُمَا إِلَى أَنْطَاكِيَةَ مَعَ بُولُسَ وَبَرْنَابَا: يَهُوذَا الْمُلَقَّبَ بَرْسَابَا وَسِيلاَ رَجُلَيْنِ مُتَقَدِّمَيْنِ فِي الإِخْوَةِ.
\par 23 وَكَتَبُوا بِأَيْدِيهِمْ هَكَذَا: «اَلرُّسُلُ وَالْمَشَايِخُ وَالإِخْوَةُ يُهْدُونَ سَلاَماً إِلَى الإِخْوَةِ الَّذِينَ مِنَ الْأُمَمِ فِي أَنْطَاكِيَةَ وَسُورِيَّةَ وَكِيلِيكِيَّةَ:
\par 24 إِذْ قَدْ سَمِعْنَا أَنَّ أُنَاساً خَارِجِينَ مِنْ عِنْدِنَا أَزْعَجُوكُمْ بِأَقْوَالٍ مُقَلِّبِينَ أَنْفُسَكُمْ وَقَائِلِينَ أَنْ تَخْتَتِنُوا وَتَحْفَظُوا النَّامُوسَ - الَّذِينَ نَحْنُ لَمْ نَأْمُرْهُمْ.
\par 25 رَأَيْنَا وَقَدْ صِرْنَا بِنَفْسٍ وَاحِدَةٍ أَنْ نَخْتَارَ رَجُلَيْنِ وَنُرْسِلَهُمَا إِلَيْكُمْ مَعَ حَبِيبَيْنَا بَرْنَابَا وَبُولُسَ
\par 26 رَجُلَيْنِ قَدْ بَذَلاَ نَفْسَيْهِمَا لأَجْلِ اسْمِ رَبِّنَا يَسُوعَ الْمَسِيحِ -
\par 27 فَقَدْ أَرْسَلْنَا يَهُوذَا وَسِيلاَ وَهُمَا يُخْبِرَانِكُمْ بِنَفْسِ الْأُمُورِ شِفَاهاً.
\par 28 لأَنَّهُ قَدْ رَأَى الرُّوحُ الْقُدُسُ وَنَحْنُ أَنْ لاَ نَضَعَ عَلَيْكُمْ ثِقْلاً أَكْثَرَ غَيْرَ هَذِهِ الأَشْيَاءِ الْوَاجِبَةِ:
\par 29 أَنْ تَمْتَنِعُوا عَمَّا ذُبِحَ لِلأَصْنَامِ وَعَنِ الدَّمِ وَالْمَخْنُوقِ وَالزِّنَا الَّتِي إِنْ حَفِظْتُمْ أَنْفُسَكُمْ مِنْهَا فَنِعِمَّا تَفْعَلُونَ. كُونُوا مُعَافَيْنَ».
\par 30 فَهَؤُلاَءِ لَمَّا أُطْلِقُوا جَاءُوا إِلَى أَنْطَاكِيَةَ وَجَمَعُوا الْجُمْهُورَ وَدَفَعُوا الرِّسَالَةَ.
\par 31 فَلَمَّا قَرَأُوهَا فَرِحُوا لِسَبَبِ التَّعْزِيَةِ.
\par 32 وَيَهُوذَا وَسِيلاَ إِذْ كَانَا هُمَا أَيْضاً نَبِيَّيْنِ وَعَظَا الإِخْوَةَ بِكَلاَمٍ كَثِيرٍ وَشَدَّدَاهُمْ.
\par 33 ثُمَّ بَعْدَ مَا صَرَفَا زَمَاناً أُطْلِقَا بِسَلاَمٍ مِنَ الإِخْوَةِ إِلَى الرُّسُلِ.
\par 34 وَلَكِنَّ سِيلاَ رَأَى أَنْ يَلْبَثَ هُنَاكَ.
\par 35 أَمَّا بُولُسُ وَبَرْنَابَا فَأَقَامَا فِي أَنْطَاكِيَةَ يُعَلِّمَانِ وَيُبَشِّرَانِ مَعَ آخَرِينَ كَثِيرِينَ أَيْضاً بِكَلِمَةِ الرَّبِّ.
\par 36 ثُمَّ بَعْدَ أَيَّامٍ قَالَ بُولُسُ لِبَرْنَابَا: «لِنَرْجِعْ وَنَفْتَقِدْ إِخْوَتَنَا فِي كُلِّ مَدِينَةٍ نَادَيْنَا فِيهَا بِكَلِمَةِ الرَّبِّ كَيْفَ هُمْ».
\par 37 فَأَشَارَ بَرْنَابَا أَنْ يَأْخُذَا مَعَهُمَا أَيْضاً يُوحَنَّا الَّذِي يُدْعَى مَرْقُسَ
\par 38 وَأَمَّا بُولُسُ فَكَانَ يَسْتَحْسِنُ أَنَّ الَّذِي فَارَقَهُمَا مِنْ بَمْفِيلِيَّةَ وَلَمْ يَذْهَبْ مَعَهُمَا لِلْعَمَلِ لاَ يَأْخُذَانِهِ مَعَهُمَا.
\par 39 فَحَصَلَ بَيْنَهُمَا مُشَاجَرَةٌ حَتَّى فَارَقَ أَحَدُهُمَا الآخَرَ. وَبَرْنَابَا أَخَذَ مَرْقُسَ وَسَافَرَ فِي الْبَحْرِ إِلَى قُبْرُسَ.
\par 40 وَأَمَّا بُولُسُ فَاخْتَارَ سِيلاَ وَخَرَجَ مُسْتَوْدَعاً مِنَ الإِخْوَةِ إِلَى نِعْمَةِ اللهِ.
\par 41 فَاجْتَازَ فِي سُورِيَّةَ وَكِيلِيكِيَّةَ يُشَدِّدُ الْكَنَائِسَ.

\chapter{16}

\par 1 ثُمَّ وَصَلَ إِلَى دَرْبَةَ وَلِسْتِرَةَ وَإِذَا تِلْمِيذٌ كَانَ هُنَاكَ اسْمُهُ تِيمُوثَاوُسُ ابْنُ امْرَأَةٍ يَهُودِيَّةٍ مُؤْمِنَةٍ وَلَكِنَّ أَبَاهُ يُونَانِيٌّ
\par 2 وَكَانَ مَشْهُوداً لَهُ مِنَ الإِخْوَةِ الَّذِينَ فِي لِسْتِرَةَ وَإِيقُونِيَةَ.
\par 3 فَأَرَادَ بُولُسُ أَنْ يَخْرُجَ هَذَا مَعَهُ فَأَخَذَهُ وَخَتَنَهُ مِنْ أَجْلِ الْيَهُودِ الَّذِينَ فِي تِلْكَ الأَمَاكِنِ لأَنَّ الْجَمِيعَ كَانُوا يَعْرِفُونَ أَبَاهُ أَنَّهُ يُونَانِيٌّ.
\par 4 وَإِذْ كَانُوا يَجْتَازُونَ فِي الْمُدُنِ كَانُوا يُسَلِّمُونَهُمُ الْقَضَايَا الَّتِي حَكَمَ بِهَا الرُّسُلُ وَالْمَشَايِخُ الَّذِينَ فِي أُورُشَلِيمَ لِيَحْفَظُوهَا.
\par 5 فَكَانَتِ الْكَنَائِسُ تَتَشَدَّدُ فِي الإِيمَانِ وَتَزْدَادُ فِي الْعَدَدِ كُلَّ يَوْمٍ.
\par 6 وَبَعْدَ مَا اجْتَازُوا فِي فِرِيجِيَّةَ وَكُورَةِ غَلاَطِيَّةَ مَنَعَهُمُ الرُّوحُ الْقُدُسُ أَنْ يَتَكَلَّمُوا بِالْكَلِمَةِ فِي أَسِيَّا.
\par 7 فَلَمَّا أَتَوْا إِلَى مِيسِيَّا حَاوَلُوا أَنْ يَذْهَبُوا إِلَى بِثِينِيَّةَ فَلَمْ يَدَعْهُمُ الرُّوحُ.
\par 8 فَمَرُّوا عَلَى مِيسِيَّا وَانْحَدَرُوا إِلَى تَرُوَاسَ.
\par 9 وَظَهَرَتْ لِبُولُسَ رُؤْيَا فِي اللَّيْلِ: رَجُلٌ مَكِدُونِيٌّ قَائِمٌ يَطْلُبُ إِلَيْهِ وَيَقُولُ: «اعْبُرْ إِلَى مَكِدُونِيَّةَ وَأَعِنَّا!».
\par 10 فَلَمَّا رَأَى الرُّؤْيَا لِلْوَقْتِ طَلَبْنَا أَنْ نَخْرُجَ إِلَى مَكِدُونِيَّةَ مُتَحَقِّقِينَ أَنَّ الرَّبَّ قَدْ دَعَانَا لِنُبَشِّرَهُمْ.
\par 11 فَأَقْلَعْنَا مِنْ تَرُوَاسَ وَتَوَجَّهْنَا بِالاِسْتِقَامَةِ إِلَى سَامُوثْرَاكِي وَفِي الْغَدِ إِلَى نِيَابُولِيسَ.
\par 12 وَمِنْ هُنَاكَ إِلَى فِيلِبِّي الَّتِي هِيَ أَوَّلُ مَدِينَةٍ مِنْ مُقَاطَعَةِ مَكِدُونِيَّةَ وَهِيَ كُولُونِيَّةُ. فَأَقَمْنَا فِي هَذِهِ الْمَدِينَةِ أَيَّاماً.
\par 13 وَفِي يَوْمِ السَّبْتِ خَرَجْنَا إِلَى خَارِجِ الْمَدِينَةِ عِنْدَ نَهْرٍ حَيْثُ جَرَتِ الْعَادَةُ أَنْ تَكُونَ صَلاَةٌ فَجَلَسْنَا وَكُنَّا نُكَلِّمُ النِّسَاءَ اللَّوَاتِي اجْتَمَعْنَ.
\par 14 فَكَانَتْ تَسْمَعُ امْرَأَةٌ اسْمُهَا لِيدِيَّةُ بَيَّاعَةُ أُرْجُوانٍ مِنْ مَدِينَةِ ثَيَاتِيرَا مُتَعَبِّدَةٌ لِلَّهِ فَفَتَحَ الرَّبُّ قَلْبَهَا لِتُصْغِيَ إِلَى مَا كَانَ يَقُولُهُ بُولُسُ.
\par 15 فَلَمَّا اعْتَمَدَتْ هِيَ وَأَهْلُ بَيْتِهَا طَلَبَتْ قَائِلَةً: «إِنْ كُنْتُمْ قَدْ حَكَمْتُمْ أَنِّي مُؤْمِنَةٌ بِالرَّبِّ فَادْخُلُوا بَيْتِي وَامْكُثُوا». فَأَلْزَمَتْنَا.
\par 16 وَحَدَثَ بَيْنَمَا كُنَّا ذَاهِبِينَ إِلَى الصَّلاَةِ أَنَّ جَارِيَةً بِهَا رُوحُ عِرَافَةٍ اسْتَقْبَلَتْنَا. وَكَانَتْ تُكْسِبُ مَوَالِيَهَا مَكْسَباً كَثِيراً بِعِرَافَتِهَا.
\par 17 هَذِهِ اتَّبَعَتْ بُولُسَ وَإِيَّانَا وَصَرَخَتْ قَائِلَةً: «هَؤُلاَءِ النَّاسُ هُمْ عَبِيدُ اللهِ الْعَلِيِّ الَّذِينَ يُنَادُونَ لَكُمْ بِطَرِيقِ الْخَلاَصِ».
\par 18 وَكَانَتْ تَفْعَلُ هَذَا أَيَّاماً كَثِيرَةً. فَضَجِرَ بُولُسُ وَالْتَفَتَ إِلَى الرُّوحِ وَقَالَ: «أَنَا آمُرُكَ بِاسْمِ يَسُوعَ الْمَسِيحِ أَنْ تَخْرُجَ مِنْهَا». فَخَرَجَ فِي تِلْكَ السَّاعَةِ.
\par 19 فَلَمَّا رَأَى مَوَالِيهَا أَنَّهُ قَدْ خَرَجَ رَجَاءُ مَكْسَبِهِمْ أَمْسَكُوا بُولُسَ وَسِيلاَ وَجَرُّوهُمَا إِلَى السُّوقِ إِلَى الْحُكَّامِ.
\par 20 وَإِذْ أَتَوْا بِهِمَا إِلَى الْوُلاَةِ قَالُوا: «هَذَانِ الرَّجُلاَنِ يُبَلْبِلاَنِ مَدِينَتَنَا وَهُمَا يَهُودِيَّانِ
\par 21 وَيُنَادِيَانِ بِعَوَائِدَ لاَ يَجُوزُ لَنَا أَنْ نَقْبَلَهَا وَلاَ نَعْمَلَ بِهَا إِذْ نَحْنُ رُومَانِيُّونَ».
\par 22 فَقَامَ الْجَمْعُ مَعاً عَلَيْهِمَا وَمَزَّقَ الْوُلاَةُ ثِيَابَهُمَا وَأَمَرُوا أَنْ يُضْرَبَا بِالْعِصِيِّ.
\par 23 فَوَضَعُوا عَلَيْهِمَا ضَرَبَاتٍ كَثِيرَةً وَأَلْقُوهُمَا فِي السِّجْنِ وَأَوْصُوا حَافِظَ السِّجْنِ أَنْ يَحْرُسَهُمَا بِضَبْطٍ.
\par 24 وَهُوَ إِذْ أَخَذَ وَصِيَّةً مِثْلَ هَذِهِ أَلْقَاهُمَا فِي السِّجْنِ الدَّاخِلِيِّ وَضَبَطَ أَرْجُلَهُمَا فِي الْمِقْطَرَةِ.
\par 25 وَنَحْوَ نِصْفِ اللَّيْلِ كَانَ بُولُسُ وَسِيلاَ يُصَلِّيَانِ وَيُسَبِّحَانِ اللهَ وَالْمَسْجُونُونَ يَسْمَعُونَهُمَا.
\par 26 فَحَدَثَ بَغْتَةً زَلْزَلَةٌ عَظِيمَةٌ حَتَّى تَزَعْزَعَتْ أَسَاسَاتُ السِّجْنِ فَانْفَتَحَتْ فِي الْحَالِ الأَبْوَابُ كُلُّهَا وَانْفَكَّتْ قُيُودُ الْجَمِيعِ.
\par 27 وَلَمَّا اسْتَيْقَظَ حَافِظُ السِّجْنِ وَرَأَى أَبْوَابَ السِّجْنِ مَفْتُوحَةً اسْتَلَّ سَيْفَهُ وَكَانَ مُزْمِعاً أَنْ يَقْتُلَ نَفْسَهُ ظَانّاً أَنَّ الْمَسْجُونِينَ قَدْ هَرَبُوا.
\par 28 فَنَادَى بُولُسُ بِصَوْتٍ عَظِيمٍ قَائِلاً: «لاَ تَفْعَلْ بِنَفْسِكَ شَيْئاً رَدِيّاً لأَنَّ جَمِيعَنَا هَهُنَا».
\par 29 فَطَلَبَ ضَوْءاً وَانْدَفَعَ إِلَى دَاخِلٍ وَخَرَّ لِبُولُسَ وَسِيلاَ وَهُوَ مُرْتَعِدٌ
\par 30 ثُمَّ أَخْرَجَهُمَا وَقَالَ: «يَا سَيِّدَيَّ مَاذَا يَنْبَغِي أَنْ أَفْعَلَ لِكَيْ أَخْلُصَ؟»
\par 31 فَقَالاَ: «آمِنْ بِالرَّبِّ يَسُوعَ الْمَسِيحِ فَتَخْلُصَ أَنْتَ وَأَهْلُ بَيْتِكَ».
\par 32 وَكَلَّمَاهُ وَجَمِيعَ مَنْ فِي بَيْتِهِ بِكَلِمَةِ الرَّبِّ.
\par 33 فَأَخَذَهُمَا فِي تِلْكَ السَّاعَةِ مِنَ اللَّيْلِ وَغَسَّلَهُمَا مِنَ الْجِرَاحَاتِ وَاعْتَمَدَ فِي الْحَالِ هُوَ وَالَّذِينَ لَهُ أَجْمَعُونَ.
\par 34 وَلَمَّا أَصْعَدَهُمَا إِلَى بَيْتِهِ قَدَّمَ لَهُمَا مَائِدَةً وَتَهَلَّلَ مَعَ جَمِيعِ بَيْتِهِ إِذْ كَانَ قَدْ آمَنَ بِاللَّهِ.
\par 35 وَلَمَّا صَارَ النَّهَارُ أَرْسَلَ الْوُلاَةُ الْجَلاَّدِينَ قَائِلِينَ: «أَطْلِقْ ذَيْنِكَ الرَّجُلَيْنِ».
\par 36 فَأَخْبَرَ حَافِظُ السِّجْنِ بُولُسَ أَنَّ الْوُلاَةَ قَدْ أَرْسَلُوا أَنْ تُطْلَقَا فَاخْرُجَا الآنَ وَاذْهَبَا بِسَلاَمٍ.
\par 37 فَقَالَ لَهُمْ بُولُسُ: «ضَرَبُونَا جَهْراً غَيْرَ مَقْضِيٍّ عَلَيْنَا وَنَحْنُ رَجُلاَنِ رُومَانِيَّانِ وَأَلْقَوْنَا فِي السِّجْنِ - أَفَالآنَ يَطْرُدُونَنَا سِرّاً؟ كَلاَّ! بَلْ لِيَأْتُوا هُمْ أَنْفُسُهُمْ وَيُخْرِجُونَا».
\par 38 فَأَخْبَرَ الْجَلاَّدُونَ الْوُلاَةَ بِهَذَا الْكَلاَمِ فَاخْتَشَوْا لَمَّا سَمِعُوا أَنَّهُمَا رُومَانِيَّانِ.
\par 39 فَجَاءُوا وَتَضَرَّعُوا إِلَيْهِمَا وَأَخْرَجُوهُمَا وَسَأَلُوهُمَا أَنْ يَخْرُجَا مِنَ الْمَدِينَةِ.
\par 40 فَخَرَجَا مِنَ السِّجْنِ وَدَخَلاَ عِنْدَ لِيدِيَّةَ فَأَبْصَرَا الإِخْوَةَ وَعَزَّيَاهُمْ ثُمَّ خَرَجَا.

\chapter{17}

\par 1 فَاجْتَازَا فِي أَمْفِيبُولِيسَ وَأَبُولُونِيَّةَ وَأَتَيَا إِلَى تَسَالُونِيكِي حَيْثُ كَانَ مَجْمَعُ الْيَهُودِ.
\par 2 فَدَخَلَ بُولُسُ إِلَيْهِمْ حَسَبَ عَادَتِهِ وَكَانَ يُحَاجُّهُمْ ثَلاَثَةَ سُبُوتٍ مِنَ الْكُتُبِ
\par 3 مُوَضِّحاً وَمُبَيِّناً أَنَّهُ كَانَ يَنْبَغِي أَنَّ الْمَسِيحَ يَتَأَلَّمُ وَيَقُومُ مِنَ الأَمْوَاتِ وَأَنَّ هَذَا هُوَ الْمَسِيحُ يَسُوعُ الَّذِي أَنَا أُنَادِي لَكُمْ بِهِ.
\par 4 فَاقْتَنَعَ قَوْمٌ مِنْهُمْ وَانْحَازُوا إِلَى بُولُسَ وَسِيلاَ وَمِنَ الْيُونَانِيِّينَ الْمُتَعَبِّدِينَ جُمْهُورٌ كَثِيرٌ وَمِنَ النِّسَاءِ الْمُتَقَدِّمَاتِ عَدَدٌ لَيْسَ بِقَلِيلٍ.
\par 5 فَغَارَ الْيَهُودُ غَيْرُ الْمُؤْمِنِينَ وَاتَّخَذُوا رِجَالاً أَشْرَاراً مِنْ أَهْلِ السُّوقِ وَتَجَمَّعُوا وَسَجَّسُوا الْمَدِينَةَ وَقَامُوا عَلَى بَيْتِ يَاسُونَ طَالِبِينَ أَنْ يُحْضِرُوهُمَا إِلَى الشَّعْبِ.
\par 6 وَلَمَّا لَمْ يَجِدُوهُمَا جَرُّوا يَاسُونَ وَأُنَاساً مِنَ الإِخْوَةِ إِلَى حُكَّامِ الْمَدِينَةِ صَارِخِينَ: «إِنَّ هَؤُلاَءِ الَّذِينَ فَتَنُوا الْمَسْكُونَةَ حَضَرُوا إِلَى هَهُنَا أَيْضاً.
\par 7 وَقَدْ قَبِلَهُمْ يَاسُونُ. وَهَؤُلاَءِ كُلُّهُمْ يَعْمَلُونَ ضِدَّ أَحْكَامِ قَيْصَرَ قَائِلِينَ إِنَّهُ يُوجَدُ مَلِكٌ آخَرُ: يَسُوعُ!»
\par 8 فَأَزْعَجُوا الْجَمْعَ وَحُكَّامَ الْمَدِينَةِ إِذْ سَمِعُوا هَذَا.
\par 9 فَأَخَذُوا كَفَالَةً مِنْ يَاسُونَ وَمِنَ الْبَاقِينَ ثُمَّ أَطْلَقُوهُمْ.
\par 10 وَأَمَّا الإِخْوَةُ فَلِلْوَقْتِ أَرْسَلُوا بُولُسَ وَسِيلاَ لَيْلاً إِلَى بِيرِيَّةَ. وَهُمَا لَمَّا وَصَلاَ مَضَيَا إِلَى مَجْمَعِ الْيَهُودِ.
\par 11 وَكَانَ هَؤُلاَءِ أَشْرَفَ مِنَ الَّذِينَ فِي تَسَالُونِيكِي فَقَبِلُوا الْكَلِمَةَ بِكُلِّ نَشَاطٍ فَاحِصِينَ الْكُتُبَ كُلَّ يَوْمٍ: هَلْ هَذِهِ الْأُمُورُ هَكَذَا؟
\par 12 فَآمَنَ مِنْهُمْ كَثِيرُونَ وَمِنَ النِّسَاءِ الْيُونَانِيَّاتِ الشَّرِيفَاتِ وَمِنَ الرِّجَالِ عَدَدٌ لَيْسَ بِقَلِيلٍ.
\par 13 فَلَمَّا عَلِمَ الْيَهُودُ الَّذِينَ مِنْ تَسَالُونِيكِي أَنَّهُ فِي بِيرِيَّةَ أَيْضاً نَادَى بُولُسُ بِكَلِمَةِ اللهِ جَاءُوا يُهَيِّجُونَ الْجُمُوعَ هُنَاكَ أَيْضاً.
\par 14 فَحِينَئِذٍ أَرْسَلَ الإِخْوَةُ بُولُسَ لِلْوَقْتِ لِيَذْهَبَ كَمَا إِلَى الْبَحْرِ وَأَمَّا سِيلاَ وَتِيمُوثَاوُسُ فَبَقِيَا هُنَاكَ.
\par 15 وَالَّذِينَ صَاحَبُوا بُولُسَ جَاءُوا بِهِ إِلَى أَثِينَا. وَلَمَّا أَخَذُوا وَصِيَّةً إِلَى سِيلاَ وَتِيمُوثَاوُسَ أَنْ يَأْتِيَا إِلَيْهِ بِأَسْرَعِ مَا يُمْكِنُ مَضَوْا.
\par 16 وَبَيْنَمَا بُولُسُ يَنْتَظِرُهُمَا فِي أَثِينَا احْتَدَّتْ رُوحُهُ فِيهِ إِذْ رَأَى الْمَدِينَةَ مَمْلُوءَةً أَصْنَاماً.
\par 17 فَكَانَ يُكَلِّمُ فِي الْمَجْمَعِ الْيَهُودَ الْمُتَعَبِّدِينَ وَالَّذِينَ يُصَادِفُونَهُ فِي السُّوقِ كُلَّ يَوْمٍ.
\par 18 فَقَابَلَهُ قَوْمٌ مِنَ الْفَلاَسِفَةِ الأَبِيكُورِيِّينَ وَالرِّوَاقِيِّينَ وَقَالَ بَعْضٌ: «تُرَى مَاذَا يُرِيدُ هَذَا الْمِهْذَارُ أَنْ يَقُولَ؟» وَبَعْضٌ: «إِنَّهُ يَظْهَرُ مُنَادِياً بِآلِهَةٍ غَرِيبَةٍ» - لأَنَّهُ كَانَ يُبَشِّرُهُمْ بِيَسُوعَ وَالْقِيَامَةِ.
\par 19 فَأَخَذُوهُ وَذَهَبُوا بِهِ إِلَى أَرِيُوسَ بَاغُوسَ قَائِلِينَ: «هَلْ يُمْكِنُنَا أَنْ نَعْرِفَ مَا هُوَ هَذَا التَّعْلِيمُ الْجَدِيدُ الَّذِي تَتَكَلَّمُ بِهِ.
\par 20 لأَنَّكَ تَأْتِي إِلَى مَسَامِعِنَا بِأُمُورٍ غَرِيبَةٍ فَنُرِيدُ أَنْ نَعْلَمَ مَا عَسَى أَنْ تَكُونَ هَذِهِ».
\par 21 أَمَّا الأَثِينِيُّونَ أَجْمَعُونَ وَالْغُرَبَاءُ الْمُسْتَوْطِنُونَ فَلاَ يَتَفَرَّغُونَ لِشَيْءٍ آخَرَ إِلاَّ لأَنْ يَتَكَلَّمُوا أَوْ يَسْمَعُوا شَيْئاً حَديثاً.
\par 22 فَوَقَفَ بُولُسُ فِي وَسَطِ أَرِيُوسَ بَاغُوسَ وَقَالَ: «أَيُّهَا الرِّجَالُ الأَثِينِيُّونَ أَرَاكُمْ مِنْ كُلِّ وَجْهٍ كَأَنَّكُمْ مُتَدَيِّنُونَ كَثِيراً
\par 23 لأَنَّنِي بَيْنَمَا كُنْتُ أَجْتَازُ وَأَنْظُرُ إِلَى مَعْبُودَاتِكُمْ وَجَدْتُ أَيْضاً مَذْبَحاً مَكْتُوباً عَلَيْهِ: «لِإِلَهٍ مَجْهُولٍ». فَالَّذِي تَتَّقُونَهُ وَأَنْتُمْ تَجْهَلُونَهُ هَذَا أَنَا أُنَادِي لَكُمْ بِهِ.
\par 24 الإِلَهُ الَّذِي خَلَقَ الْعَالَمَ وَكُلَّ مَا فِيهِ هَذَا إِذْ هُوَ رَبُّ السَّمَاءِ وَالأَرْضِ لاَ يَسْكُنُ فِي هَيَاكِلَ مَصْنُوعَةٍ بِالأَيَادِي
\par 25 وَلاَ يُخْدَمُ بِأَيَادِي النَّاسِ كَأَنَّهُ مُحْتَاجٌ إِلَى شَيْءٍ إِذْ هُوَ يُعْطِي الْجَمِيعَ حَيَاةً وَنَفْساً وَكُلَّ شَيْءٍ.
\par 26 وَصَنَعَ مِنْ دَمٍ وَاحِدٍ كُلَّ أُمَّةٍ مِنَ النَّاسِ يَسْكُنُونَ عَلَى كُلِّ وَجْهِ الأَرْضِ وَحَتَمَ بِالأَوْقَاتِ الْمُعَيَّنَةِ وَبِحُدُودِ مَسْكَنِهِمْ
\par 27 لِكَيْ يَطْلُبُوا اللهَ لَعَلَّهُمْ يَتَلَمَّسُونَهُ فَيَجِدُوهُ مَعَ أَنَّهُ عَنْ كُلِّ وَاحِدٍ مِنَّا لَيْسَ بَعِيداً.
\par 28 لأَنَّنَا بِهِ نَحْيَا وَنَتَحَرَّكُ وَنُوجَدُ. كَمَا قَالَ بَعْضُ شُعَرَائِكُمْ أَيْضاً: لأَنَّنَا أَيْضاً ذُرِّيَّتُهُ.
\par 29 فَإِذْ نَحْنُ ذُرِّيَّةُ اللهِ لاَ يَنْبَغِي أَنْ نَظُنَّ أَنَّ اللاَّهُوتَ شَبِيهٌ بِذَهَبٍ أَوْ فِضَّةٍ أَوْ حَجَرٍ نَقْشِ صِنَاعَةِ وَاخْتِرَاعِ إِنْسَانٍ.
\par 30 فَاللَّهُ الآنَ يَأْمُرُ جَمِيعَ النَّاسِ فِي كُلِّ مَكَانٍ أَنْ يَتُوبُوا مُتَغَاضِياً عَنْ أَزْمِنَةِ الْجَهْلِ.
\par 31 لأَنَّهُ أَقَامَ يَوْماً هُوَ فِيهِ مُزْمِعٌ أَنْ يَدِينَ الْمَسْكُونَةَ بِالْعَدْلِ بِرَجُلٍ قَدْ عَيَّنَهُ مُقَدِّماً لِلْجَمِيعِ إِيمَاناً إِذْ أَقَامَهُ مِنَ الأَمْوَاتِ».
\par 32 وَلَمَّا سَمِعُوا بِالْقِيَامَةِ مِنَ الأَمْوَاتِ كَانَ الْبَعْضُ يَسْتَهْزِئُونَ وَالْبَعْضُ يَقُولُونَ: «سَنَسْمَعُ مِنْكَ عَنْ هَذَا أَيْضاً!».
\par 33 وَهَكَذَا خَرَجَ بُولُسُ مِنْ وَسَطِهِمْ.
\par 34 وَلَكِنَّ أُنَاساً الْتَصَقُوا بِهِ وَآمَنُوا مِنْهُمْ دِيُونِيسِيُوسُ الأَرِيُوبَاغِيُّ وَامْرَأَةٌ اسْمُهَا دَامَرِسُ وَآخَرُونَ مَعَهُمَا.

\chapter{18}

\par 1 وَبَعْدَ هَذَا مَضَى بُولُسُ مِنْ أَثِينَا وَجَاءَ إِلَى كُورِنْثُوسَ
\par 2 فَوَجَدَ يَهُودِيّاً اسْمُهُ أَكِيلاَ بُنْطِيَّ الْجِنْسِ كَانَ قَدْ جَاءَ حَدِيثاً مِنْ إِيطَالِيَا وَبِرِيسْكِلاَّ امْرَأَتَهُ - لأَنَّ كُلُودِيُوسَ كَانَ قَدْ أَمَرَ أَنْ يَمْضِيَ جَمِيعُ الْيَهُودِ مِنْ رُومِيَةَ. فَجَاءَ إِلَيْهِمَا.
\par 3 وَلِكَوْنِهِ مِنْ صِنَاعَتِهِمَا أَقَامَ عِنْدَهُمَا وَكَانَ يَعْمَلُ لأَنَّهُمَا كَانَا فِي صِنَاعَتِهِمَا خِيَامِيَّيْنِ.
\par 4 وَكَانَ يُحَاجُّ فِي الْمَجْمَعِ كُلَّ سَبْتٍ وَيُقْنِعُ يَهُوداً وَيُونَانِيِّينَ.
\par 5 وَلَمَّا انْحَدَرَ سِيلاَ وَتِيمُوثَاوُسُ مِنْ مَكِدُونِيَّةَ كَانَ بُولُسُ مُنْحَصِراً بِالرُّوحِ وَهُوَ يَشْهَدُ لِلْيَهُودِ بِالْمَسِيحِ يَسُوعَ.
\par 6 وَإِذْ كَانُوا يُقَاوِمُونَ وَيُجَدِّفُونَ نَفَضَ ثِيَابَهُ وَقَالَ لَهُمْ: «دَمُكُمْ عَلَى رُؤُوسِكُمْ. أَنَا بَرِيءٌ. مِنَ الآنَ أَذْهَبُ إِلَى الْأُمَمِ».
\par 7 فَانْتَقَلَ مِنْ هُنَاكَ وَجَاءَ إِلَى بَيْتِ رَجُلٍ اسْمُهُ يُوسْتُسُ كَانَ مُتَعَبِّداً لِلَّهِ وَكَانَ بَيْتُهُ مُلاَصِقاً لِلْمَجْمَعِ.
\par 8 وَكِرِيسْبُسُ رَئِيسُ الْمَجْمَعِ آمَنَ بِالرَّبِّ مَعَ جَمِيعِ بَيْتِهِ وَكَثِيرُونَ مِنَ الْكُورِنْثِيِّينَ إِذْ سَمِعُوا آمَنُوا وَاعْتَمَدُوا.
\par 9 فَقَالَ الرَّبُّ لِبُولُسَ بِرُؤْيَا فِي اللَّيْلِ: «لاَ تَخَفْ بَلْ تَكَلَّمْ وَلاَ تَسْكُتْ
\par 10 لأَنِّي أَنَا مَعَكَ وَلاَ يَقَعُ بِكَ أَحَدٌ لِيُؤْذِيَكَ لأَنَّ لِي شَعْباً كَثِيراً فِي هَذِهِ الْمَدِينَةِ».
\par 11 فَأَقَامَ سَنَةً وَسِتَّةَ أَشْهُرٍ يُعَلِّمُ بَيْنَهُمْ بِكَلِمَةِ اللهِ.
\par 12 وَلَمَّا كَانَ غَالِيُونُ يَتَوَلَّى أَخَائِيَةَ قَامَ الْيَهُودُ بِنَفْسٍ وَاحِدَةٍ عَلَى بُولُسَ وَأَتَوْا بِهِ إِلَى كُرْسِيِّ الْوِلاَيَةِ
\par 13 قَائِلِينَ: «إِنَّ هَذَا يَسْتَمِيلُ النَّاسَ أَنْ يَعْبُدُوا اللهَ بِخِلاَفِ النَّامُوسِ».
\par 14 وَإِذْ كَانَ بُولُسُ مُزْمِعاً أَنْ يَتَكَلَّمَ قَالَ غَالِيُونُ لِلْيَهُودِ: «لَوْ كَانَ ظُلْماً أَوْ خُبْثاً رَدِيّاً أَيُّهَا الْيَهُودُ لَكُنْتُ بِالْحَقِّ قَدِ احْتَمَلْتُكُمْ.
\par 15 وَلَكِنْ إِذَا كَانَ مَسْأَلَةً عَنْ كَلِمَةٍ وَأَسْمَاءٍ وَنَامُوسِكُمْ فَتُبْصِرُونَ أَنْتُمْ. لأَنِّي لَسْتُ أَشَاءُ أَنْ أَكُونَ قَاضِياً لِهَذِهِ الْأُمُورِ».
\par 16 فَطَرَدَهُمْ مِنَ الْكُرْسِيِّ.
\par 17 فَأَخَذَ جَمِيعُ الْيُونَانِيِّينَ سُوسْتَانِيسَ رَئِيسَ الْمَجْمَعِ وَضَرَبُوهُ قُدَّامَ الْكُرْسِيِّ وَلَمْ يَهُمَّ غَالِيُونَ شَيْءٌ مِنْ ذَلِكَ.
\par 18 وَأَمَّا بُولُسُ فَلَبِثَ أَيْضاً أَيَّاماً كَثِيرَةً ثُمَّ وَدَّعَ الإِخْوَةَ وَسَافَرَ فِي الْبَحْرِ إِلَى سُورِيَّةَ وَمَعَهُ بِرِيسْكِلاَّ وَأَكِيلاَ بَعْدَمَا حَلَقَ رَأْسَهُ فِي كَنْخَرِيَا - لأَنَّهُ كَانَ عَلَيْهِ نَذْرٌ.
\par 19 فَأَقْبَلَ إِلَى أَفَسُسَ وَتَرَكَهُمَا هُنَاكَ. وَأَمَّا هُوَ فَدَخَلَ الْمَجْمَعَ وَحَاجَّ الْيَهُودَ.
\par 20 وَإِذْ كَانُوا يَطْلُبُونَ أَنْ يَمْكُثَ عِنْدَهُمْ زَمَاناً أَطْوَلَ لَمْ يُجِبْ.
\par 21 بَلْ وَدَّعَهُمْ قَائِلاً: «يَنْبَغِي عَلَى كُلِّ حَالٍ أَنْ أَعْمَلَ الْعِيدَ الْقَادِمَ فِي أُورُشَلِيمَ. وَلَكِنْ سَأَرْجِعُ إِلَيْكُمْ أَيْضاً إِنْ شَاءَ اللهُ». فَأَقْلَعَ مِنْ أَفَسُسَ.
\par 22 وَلَمَّا نَزَلَ فِي قَيْصَرِيَّةَ صَعِدَ وَسَلَّمَ عَلَى الْكَنِيسَةِ ثُمَّ انْحَدَرَ إِلَى أَنْطَاكِيَةَ.
\par 23 وَبَعْدَمَا صَرَفَ زَمَاناً خَرَجَ وَاجْتَازَ بِالتَّتَابُعِ فِي كُورَةِ غَلاَطِيَّةَ وَفِرِيجِيَّةَ يُشَدِّدُ جَمِيعَ التَّلاَمِيذِ.
\par 24 ثُمَّ أَقْبَلَ إِلَى أَفَسُسَ يَهُودِيٌّ اسْمُهُ أَبُلُّوسُ إِسْكَنْدَرِيُّ الْجِنْسِ رَجُلٌ فَصِيحٌ مُقْتَدِرٌ فِي الْكُتُبِ.
\par 25 كَانَ هَذَا خَبِيراً فِي طَرِيقِ الرَّبِّ. وَكَانَ وَهُوَ حَارٌّ بِالرُّوحِ يَتَكَلَّمُ وَيُعَلِّمُ بِتَدْقِيقٍ مَا يَخْتَصُّ بِالرَّبِّ. عَارِفاً مَعْمُودِيَّةَ يُوحَنَّا فَقَطْ.
\par 26 وَابْتَدَأَ هَذَا يُجَاهِرُ فِي الْمَجْمَعِ. فَلَمَّا سَمِعَهُ أَكِيلاَ وَبِرِيسْكِلاَّ أَخَذَاهُ إِلَيْهِمَا وَشَرَحَا لَهُ طَرِيقَ الرَّبِّ بِأَكْثَرِ تَدْقِيقٍ.
\par 27 وَإِذْ كَانَ يُرِيدُ أَنْ يَجْتَازَ إِلَى أَخَائِيَةَ كَتَبَ الإِخْوَةُ إِلَى التَّلاَمِيذِ يَحُضُّونَهُمْ أَنْ يَقْبَلُوهُ. فَلَمَّا جَاءَ سَاعَدَ كَثِيراً بِالنِّعْمَةِ الَّذِينَ كَانُوا قَدْ آمَنُوا
\par 28 لأَنَّهُ كَانَ بِاشْتِدَادٍ يُفْحِمُ الْيَهُودَ جَهْراً مُبَيِّناً بِالْكُتُبِ أَنَّ يَسُوعَ هُوَ الْمَسِيحُ.

\chapter{19}

\par 1 فَحَدَثَ فِيمَا كَانَ أَبُلُّوسُ فِي كُورِنْثُوسَ أَنَّ بُولُسَ بَعْدَ مَا اجْتَازَ فِي النَّوَاحِي الْعَالِيَةِ جَاءَ إِلَى أَفَسُسَ. فَإِذْ وَجَدَ تَلاَمِيذَ
\par 2 سَأَلَهُمْ: «هَلْ قَبِلْتُمُ الرُّوحَ الْقُدُسَ لَمَّا آمَنْتُمْ؟» قَالُوا لَهُ: «وَلاَ سَمِعْنَا أَنَّهُ يُوجَدُ الرُّوحُ الْقُدُسُ».
\par 3 فَسَأَلَهُمْ: «فَبِمَاذَا اعْتَمَدْتُمْ؟» فَقَالُوا: «بِمَعْمُودِيَّةِ يُوحَنَّا».
\par 4 فَقَالَ بُولُسُ: «إِنَّ يُوحَنَّا عَمَّدَ بِمَعْمُودِيَّةِ التَّوْبَةِ قَائِلاً لِلشَّعْبِ أَنْ يُؤْمِنُوا بِالَّذِي يَأْتِي بَعْدَهُ أَيْ بِالْمَسِيحِ يَسُوعَ».
\par 5 فَلَمَّا سَمِعُوا اعْتَمَدُوا بِاسْمِ الرَّبِّ يَسُوعَ.
\par 6 وَلَمَّا وَضَعَ بُولُسُ يَدَيْهِ عَلَيْهِمْ حَلَّ الرُّوحُ الْقُدُسُ عَلَيْهِمْ فَطَفِقُوا يَتَكَلَّمُونَ بِلُغَاتٍ وَيَتَنَبَّأُونَ.
\par 7 وَكَانَ جَمِيعُ الرِّجَالِ نَحْوَ اثْنَيْ عَشَرَ.
\par 8 ثُمَّ دَخَلَ الْمَجْمَعَ وَكَانَ يُجَاهِرُ مُدَّةَ ثَلاَثَةِ أَشْهُرٍ مُحَاجّاً وَمُقْنِعاً فِي مَا يَخْتَصُّ بِمَلَكُوتِ اللهِ.
\par 9 وَلَمَّا كَانَ قَوْمٌ يَتَقَسُّونَ وَلاَ يَقْنَعُونَ شَاتِمِينَ الطَّرِيقَ أَمَامَ الْجُمْهُورِ اعْتَزَلَ عَنْهُمْ وَأَفْرَزَ التَّلاَمِيذَ مُحَاجّاً كُلَّ يَوْمٍ فِي مَدْرَسَةِ إِنْسَانٍ اسْمُهُ تِيرَانُّسُ -
\par 10 وَكَانَ ذَلِكَ مُدَّةَ سَنَتَيْنِ حَتَّى سَمِعَ كَلِمَةَ الرَّبِّ يَسُوعَ جَمِيعُ السَّاكِنِينَ فِي أَسِيَّا مِنْ يَهُودٍ وَيُونَانِيِّينَ.
\par 11 وَكَانَ اللهُ يَصْنَعُ عَلَى يَدَيْ بُولُسَ قُوَّاتٍ غَيْرَ الْمُعْتَادَةِ
\par 12 حَتَّى كَانَ يُؤْتَى عَنْ جَسَدِهِ بِمَنَادِيلَ أَوْ مَآزِرَ إِلَى الْمَرْضَى فَتَزُولُ عَنْهُمُ الأَمْرَاضُ وَتَخْرُجُ الأَرْوَاحُ الشِّرِّيرَةُ مِنْهُمْ.
\par 13 فَشَرَعَ قَوْمٌ مِنَ الْيَهُودِ الطَّوَّافِينَ الْمُعَزِّمِينَ أَنْ يُسَمُّوا عَلَى الَّذِينَ بِهِمِ الأَرْوَاحُ الشِّرِّيرَةُ بِاسْمِ الرَّبِّ يَسُوعَ قَائِلِينَ: «نُقْسِمُ عَلَيْكَ بِيَسُوعَ الَّذِي يَكْرِزُ بِهِ بُولُسُ!»
\par 14 وَكَانَ الَّذِينَ فَعَلُوا هَذَا سَبْعَةَ بَنِينَ لِسَكَاوَا رَجُلٍ يَهُودِيٍّ رَئِيسِ كَهَنَةٍ.
\par 15 فَقَالَ الرُّوحُ الشِّرِّيرُ لَهُمْ: «أَمَّا يَسُوعُ فَأَنَا أَعْرِفُهُ وَبُولُسُ أَنَا أَعْلَمُهُ وَأَمَّا أَنْتُمْ فَمَنْ أَنْتُمْ؟»
\par 16 فَوَثَبَ عَلَيْهِمُ الإِنْسَانُ الَّذِي كَانَ فِيهِ الرُّوحُ الشِّرِّيرُ وَغَلَبَهُمْ وَقَوِيَ عَلَيْهِمْ حَتَّى هَرَبُوا مِنْ ذَلِكَ الْبَيْتِ عُرَاةً وَمُجَرَّحِينَ.
\par 17 وَصَارَ هَذَا مَعْلُوماً عِنْدَ جَمِيعِ الْيَهُودِ وَالْيُونَانِيِّينَ السَّاكِنِينَ فِي أَفَسُسَ. فَوَقَعَ خَوْفٌ عَلَى جَمِيعِهِمْ وَكَانَ اسْمُ الرَّبِّ يَسُوعَ يَتَعَظَّمُ.
\par 18 وَكَانَ كَثِيرُونَ مِنَ الَّذِينَ آمَنُوا يَأْتُونَ مُقِرِّينَ وَمُخْبِرِينَ بِأَفْعَالِهِمْ
\par 19 وَكَانَ كَثِيرُونَ مِنَ الَّذِينَ يَسْتَعْمِلُونَ السِّحْرَ يَجْمَعُونَ الْكُتُبَ وَيُحَرِّقُونَهَا أَمَامَ الْجَمِيعِ. وَحَسَبُوا أَثْمَانَهَا فَوَجَدُوهَا خَمْسِينَ أَلْفاً مِنَ الْفِضَّةِ.
\par 20 هَكَذَا كَانَتْ كَلِمَةُ الرَّبِّ تَنْمُو وَتَقْوَى بِشِدَّةٍ.
\par 21 وَلَمَّا كَمِلَتْ هَذِهِ الْأُمُورُ وَضَعَ بُولُسُ فِي نَفْسِهِ أَنَّهُ بَعْدَمَا يَجْتَازُ فِي مَكِدُونِيَّةَ وَأَخَائِيَةَ يَذْهَبُ إِلَى أُورُشَلِيمَ قَائِلاً: «إِنِّي بَعْدَ مَا أَصِيرُ هُنَاكَ يَنْبَغِي أَنْ أَرَى رُومِيَةَ أَيْضاً».
\par 22 فَأَرْسَلَ إِلَى مَكِدُونِيَّةَ اثْنَيْنِ مِنَ الَّذِينَ كَانُوا يَخْدِمُونَهُ: تِيمُوثَاوُسَ وَأَرَسْطُوسَ وَلَبِثَ هُوَ زَمَاناً فِي أَسِيَّا.
\par 23 وَحَدَثَ فِي ذَلِكَ الْوَقْتِ شَغَبٌ لَيْسَ بِقَلِيلٍ بِسَبَبِ هَذَا الطَّرِيقِ
\par 24 لأَنَّ إِنْسَاناً اسْمُهُ دِيمِتْرِيُوسُ صَائِغٌ صَانِعُ هَيَاكِلِ فِضَّةٍ لأَرْطَامِيسَ كَانَ يُكَسِّبُ الصُّنَّاعَ مَكْسَباً لَيْسَ بِقَلِيلٍ.
\par 25 فَجَمَعَهُمْ وَالْفَعَلَةَ فِي مِثْلِ ذَلِكَ الْعَمَلِ وَقَالَ: «أَيُّهَا الرِّجَالُ أَنْتُمْ تَعْلَمُونَ أَنَّ سِعَتَنَا إِنَّمَا هِيَ مِنْ هَذِهِ الصِّنَاعَةِ.
\par 26 وَأَنْتُمْ تَنْظُرُونَ وَتَسْمَعُونَ أَنَّهُ لَيْسَ مِنْ أَفَسُسَ فَقَطْ بَلْ مِنْ جَمِيعِ أَسِيَّا تَقْرِيباً اسْتَمَالَ وَأَزَاغَ بُولُسُ هَذَا جَمْعاً كَثِيراً قَائِلاً: إِنَّ الَّتِي تُصْنَعُ بِالأَيَادِي لَيْسَتْ آلِهَةً.
\par 27 فَلَيْسَ نَصِيبُنَا هَذَا وَحْدَهُ فِي خَطَرٍ مِنْ أَنْ يَحْصُلَ فِي إِهَانَةٍ بَلْ أَيْضاً هَيْكَلُ أَرْطَامِيسَ - الإِلَهَةِ الْعَظِيمَةِ - أَنْ يُحْسَبَ لاَ شَيْءَ وَأَنْ سَوْفَ تُهْدَمُ عَظَمَتُهَا هِيَ الَّتِي يَعْبُدُهَا جَمِيعُ أَسِيَّا وَالْمَسْكُونَةِ».
\par 28 فَلَمَّا سَمِعُوا امْتَلأُوا غَضَباً وَطَفِقُوا يَصْرُخُونَ قَائِلِينَ: «عَظِيمَةٌ هِيَ أَرْطَامِيسُ الأَفَسُسِيِّينَ».
\par 29 فَامْتَلأَتِ الْمَدِينَةُ كُلُّهَا اضْطِرَاباً وَانْدَفَعُوا بِنَفْسٍ وَاحِدَةٍ إِلَى الْمَشْهَدِ خَاطِفِينَ مَعَهُمْ غَايُوسَ وَأَرِسْتَرْخُسَ الْمَكِدُونِيَّيْنِ رَفِيقَيْ بُولُسَ فِي السَّفَرِ.
\par 30 وَلَمَّا كَانَ بُولُسُ يُرِيدُ أَنْ يَدْخُلَ بَيْنَ الشَّعْبِ لَمْ يَدَعْهُ التَّلاَمِيذُ.
\par 31 وَأُنَاسٌ مِنْ وُجُوهِ أَسِيَّا - كَانُوا أَصْدِقَاءَهُ - أَرْسَلُوا يَطْلُبُونَ إِلَيْهِ أَنْ لاَ يُسَلِّمَ نَفْسَهُ إِلَى الْمَشْهَدِ.
\par 32 وَكَانَ الْبَعْضُ يَصْرُخُونَ بِشَيْءٍ وَالْبَعْضُ بِشَيْءٍ آخَرَ لأَنَّ الْمَحْفَلَ كَانَ مُضْطَرِباً وَأَكْثَرُهُمْ لاَ يَدْرُونَ لأَيِّ شَيْءٍ كَانُوا قَدِ اجْتَمَعُوا!
\par 33 فَاجْتَذَبُوا إِسْكَنْدَرَ مِنَ الْجَمْعِ وَكَانَ الْيَهُودُ يَدْفَعُونَهُ. فَأَشَارَ إِسْكَنْدَرُ بِيَدِهِ يُرِيدُ أَنْ يَحْتَجَّ لِلشَّعْبِ.
\par 34 فَلَمَّا عَرَفُوا أَنَّهُ يَهُودِيٌّ صَارَ صَوْتٌ وَاحِدٌ مِنَ الْجَمِيعِ صَارِخِينَ نَحْوَ مُدَّةِ سَاعَتَيْنِ: «عَظِيمَةٌ هِيَ أَرْطَامِيسُ الأَفَسُسِيِّينَ!».
\par 35 ثُمَّ سَكَّنَ الْكَاتِبُ الْجَمْعَ وَقَالَ: «أَيُّهَا الرِّجَالُ الأَفَسُسِيُّونَ مَنْ هُوَ الإِنْسَانُ الَّذِي لاَ يَعْلَمُ أَنَّ مَدِينَةَ الأَفَسُسِيِّينَ مُتَعَبِّدَةٌ لأَرْطَامِيسَ الإِلَهَةِ الْعَظِيمَةِ وَالتِّمْثَالِ الَّذِي هَبَطَ مِنْ زَفْسَ؟
\par 36 فَإِذْ كَانَتْ هَذِهِ الأَشْيَاءُ لاَ تُقَاوَمُ يَنْبَغِي أَنْ تَكُونُوا هَادِئِينَ وَلاَ تَفْعَلُوا شَيْئاً اقْتِحَاماً.
\par 37 لأَنَّكُمْ أَتَيْتُمْ بِهَذَيْنِ الرَّجُلَيْنِ وَهُمَا لَيْسَا سَارِقَيْ هَيَاكِلَ وَلاَ مُجَدِّفَيْنِ عَلَى إِلَهَتِكُمْ.
\par 38 فَإِنْ كَانَ دِيمِتْرِيُوسُ وَالصُّنَّاعُ الَّذِينَ مَعَهُ لَهُمْ دَعْوَى عَلَى أَحَدٍ فَإِنَّهُ تُقَامُ أَيَّامٌ لِلْقَضَاءِ وَيُوجَدُ وُلاَةٌ فَلْيُرَافِعُوا بَعْضُهُمْ بَعْضاً.
\par 39 وَإِنْ كُنْتُمْ تَطْلُبُونَ شَيْئاً مِنْ جِهَةِ أُمُورٍ أُخَرَ فَإِنَّهُ يُقْضَى فِي مَحْفِلٍ شَرْعِيٍّ.
\par 40 لأَنَّنَا فِي خَطَرٍ أَنْ نُحَاكَمَ مِنْ أَجْلِ فِتْنَةِ هَذَا الْيَوْمِ. وَلَيْسَ عِلَّةٌ يُمْكِنُنَا مِنْ أَجْلِهَا أَنْ نُقَدِّمَ حِسَاباً عَنْ هَذَا التَّجَمُّعِ».
\par 41 وَلَمَّا قَالَ هَذَا صَرَفَ الْمَحْفَلَ.

\chapter{20}

\par 1 وَبَعْدَمَا انْتَهَى الشَّغَبُ دَعَا بُولُسُ التَّلاَمِيذَ وَوَدَّعَهُمْ وَخَرَجَ لِيَذْهَبَ إِلَى مَكِدُونِيَّةَ.
\par 2 وَلَمَّا كَانَ قَدِ اجْتَازَ فِي تِلْكَ النَّوَاحِي وَوَعَظَهُمْ بِكَلاَمٍ كَثِيرٍ جَاءَ إِلَى هَلاَّسَ
\par 3 فَصَرَفَ ثَلاَثَةَ أَشْهُرٍ. ثُمَّ إِذْ حَصَلَتْ مَكِيدَةٌ مِنَ الْيَهُودِ عَلَيْهِ - وَهُوَ مُزْمِعٌ أَنْ يَصْعَدَ إِلَى سُورِيَّةَ - صَارَ رَأْيٌ أَنْ يَرْجِعَ عَلَى طَرِيقِ مَكِدُونِيَّةَ.
\par 4 فَرَافَقَهُ إِلَى أَسِيَّا سُوبَاتَرُسُ الْبِيرِيُّ وَمِنْ أَهْلِ تَسَالُونِيكِي: أَرِسْتَرْخُسُ وَسَكُونْدُسُ وَغَايُسُ الدَّرْبِيُّ وَتِيمُوثَاوُسُ. وَمِنْ أَهْلِ أَسِيَّا: تِيخِيكُسُ وَتُرُوفِيمُسُ.
\par 5 هَؤُلاَءِ سَبَقُوا وَانْتَظَرُونَا فِي تَرُواسَ.
\par 6 وَأَمَّا نَحْنُ فَسَافَرْنَا فِي الْبَحْرِ بَعْدَ أَيَّامِ الْفَطِيرِ مِنْ فِيلِبِّي وَوَافَيْنَاهُمْ فِي خَمْسَةِ أَيَّامٍ إِلَى تَرُواسَ حَيْثُ صَرَفْنَا سَبْعَةَ أَيَّامٍ.
\par 7 وَفِي أَوَّلِ الْأُسْبُوعِ إِذْ كَانَ التَّلاَمِيذُ مُجْتَمِعِينَ لِيَكْسِرُوا خُبْزاً خَاطَبَهُمْ بُولُسُ وَهُوَ مُزْمِعٌ أَنْ يَمْضِيَ فِي الْغَدِ وَأَطَالَ الْكَلاَمَ إِلَى نِصْفِ اللَّيْلِ.
\par 8 وَكَانَتْ مَصَابِيحُ كَثِيرَةٌ فِي الْعِلِّيَّةِ الَّتِي كَانُوا مُجْتَمِعِينَ فِيهَا.
\par 9 وَكَانَ شَابٌّ اسْمُهُ أَفْتِيخُوسُ جَالِساً فِي الطَّاقَةِ مُتَثَقِّلاً بِنَوْمٍ عَمِيقٍ. وَإِذْ كَانَ بُولُسُ يُخَاطِبُ خِطَاباً طَوِيلاً غَلَبَ عَلَيْهِ النَّوْمُ فَسَقَطَ مِنَ الطَّبَقَةِ الثَّالِثَةِ إِلَى أَسْفَلُ وَحُمِلَ مَيِّتاً.
\par 10 فَنَزَلَ بُولُسُ وَوَقَعَ عَلَيْهِ وَاعْتَنَقَهُ قَائِلاً: «لاَ تَضْطَرِبُوا لأَنَّ نَفْسَهُ فِيهِ».
\par 11 ثُمَّ صَعِدَ وَكَسَّرَ خُبْزاً وَأَكَلَ وَتَكَلَّمَ كَثِيراً إِلَى الْفَجْرِ. وَهَكَذَا خَرَجَ.
\par 12 وَأَتُوا بِالْفَتَى حَيّاً وَتَعَزُّوا تَعْزِيَةً لَيْسَتْ بِقَلِيلَةٍ.
\par 13 وَأَمَّا نَحْنُ فَسَبَقْنَا إِلَى السَّفِينَةِ وَأَقْلَعْنَا إِلَى أَسُّوسَ مُزْمِعِينَ أَنْ نَأْخُذَ بُولُسَ مِنْ هُنَاكَ لأَنَّهُ كَانَ قَدْ رَتَّبَ هَكَذَا مُزْمِعاً أَنْ يَمْشِيَ.
\par 14 فَلَمَّا وَافَانَا إِلَى أَسُّوسَ أَخَذْنَاهُ وَأَتَيْنَا إِلَى مِيتِيلِينِي.
\par 15 ثُمَّ سَافَرْنَا مِنْ هُنَاكَ فِي الْبَحْرِ وَأَقْبَلْنَا فِي الْغَدِ إِلَى مُقَابِلِ خِيُوسَ. وَفِي الْيَوْمِ الآخَرِ وَصَلْنَا إِلَى سَامُوسَ وَأَقَمْنَا فِي تُرُوجِيلِيُّونَ ثُمَّ فِي الْيَوْمِ التَّالِي جِئْنَا إِلَى مِيلِيتُسَ
\par 16 لأَنَّ بُولُسَ عَزَمَ أَنْ يَتَجَاوَزَ أَفَسُسَ فِي الْبَحْرِ لِئَلاَّ يَعْرِضَ لَهُ أَنْ يَصْرِفَ وَقْتاً فِي أَسِيَّا لأَنَّهُ كَانَ يُسْرِعُ حَتَّى إِذَا أَمْكَنَهُ يَكُونُ فِي أُورُشَلِيمَ فِي يَوْمِ الْخَمْسِينَ.
\par 17 وَمِنْ مِيلِيتُسَ أَرْسَلَ إِلَى أَفَسُسَ وَاسْتَدْعَى قُسُوسَ الْكَنِيسَةِ.
\par 18 فَلَمَّا جَاءُوا إِلَيْهِ قَالَ لَهُمْ: «أَنْتُمْ تَعْلَمُونَ مِنْ أَوَّلِ يَوْمٍ دَخَلْتُ أَسِيَّا كَيْفَ كُنْتُ مَعَكُمْ كُلَّ الزَّمَانِ
\par 19 أَخْدِمُ الرَّبَّ بِكُلِّ تَوَاضُعٍ وَدُمُوعٍ كَثِيرَةٍ وَبِتَجَارِبَ أَصَابَتْنِي بِمَكَايِدِ الْيَهُودِ.
\par 20 كَيْفَ لَمْ أُؤَخِّرْ شَيْئاً مِنَ الْفَوَائِدِ إِلاَّ وَأَخْبَرْتُكُمْ وَعَلَّمْتُكُمْ بِهِ جَهْراً وَفِي كُلِّ بَيْتٍ
\par 21 شَاهِداً لِلْيَهُودِ وَالْيُونَانِيِّينَ بِالتَّوْبَةِ إِلَى اللهِ وَالإِيمَانِ الَّذِي بِرَبِّنَا يَسُوعَ الْمَسِيحِ.
\par 22 وَالآنَ هَا أَنَا أَذْهَبُ إِلَى أُورُشَلِيمَ مُقَيَّداً بِالرُّوحِ لاَ أَعْلَمُ مَاذَا يُصَادِفُنِي هُنَاكَ.
\par 23 غَيْرَ أَنَّ الرُّوحَ الْقُدُسَ يَشْهَدُ فِي كُلِّ مَدِينَةٍ قَائِلاً: إِنَّ وُثُقاً وَشَدَائِدَ تَنْتَظِرُنِي.
\par 24 وَلَكِنَّنِي لَسْتُ أَحْتَسِبُ لِشَيْءٍ وَلاَ نَفْسِي ثَمِينَةٌ عِنْدِي حَتَّى أُتَمِّمَ بِفَرَحٍ سَعْيِي وَالْخِدْمَةَ الَّتِي أَخَذْتُهَا مِنَ الرَّبِّ يَسُوعَ لأَشْهَدَ بِبِشَارَةِ نِعْمَةِ اللهِ.
\par 25 وَالآنَ هَا أَنَا أَعْلَمُ أَنَّكُمْ لاَ تَرَوْنَ وَجْهِي أَيْضاً أَنْتُمْ جَمِيعاً الَّذِينَ مَرَرْتُ بَيْنَكُمْ كَارِزاً بِمَلَكُوتِ اللهِ.
\par 26 لِذَلِكَ أُشْهِدُكُمُ الْيَوْمَ هَذَا أَنِّي بَرِيءٌ مِنْ دَمِ الْجَمِيعِ
\par 27 لأَنِّي لَمْ أُؤَخِّرْ أَنْ أُخْبِرَكُمْ بِكُلِّ مَشُورَةِ اللهِ.
\par 28 اِحْتَرِزُوا اذاً لأَنْفُسِكُمْ وَلِجَمِيعِ الرَّعِيَّةِ الَّتِي أَقَامَكُمُ الرُّوحُ الْقُدُسُ فِيهَا أَسَاقِفَةً لِتَرْعُوا كَنِيسَةَ اللهِ الَّتِي اقْتَنَاهَا بِدَمِهِ.
\par 29 لأَنِّي أَعْلَمُ هَذَا: أَنَّهُ بَعْدَ ذِهَابِي سَيَدْخُلُ بَيْنَكُمْ ذِئَابٌ خَاطِفَةٌ لاَ تُشْفِقُ عَلَى الرَّعِيَّةِ.
\par 30 وَمِنْكُمْ أَنْتُمْ سَيَقُومُ رِجَالٌ يَتَكَلَّمُونَ بِأُمُورٍ مُلْتَوِيَةٍ لِيَجْتَذِبُوا التَّلاَمِيذَ وَرَاءَهُمْ.
\par 31 لِذَلِكَ اسْهَرُوا مُتَذَكِّرِينَ أَنِّي ثَلاَثَ سِنِينَ لَيْلاً وَنَهَاراً لَمْ أَفْتُرْ عَنْ أَنْ أُنْذِرَ بِدُمُوعٍ كُلَّ وَاحِدٍ.
\par 32 وَالآنَ أَسْتَوْدِعُكُمْ يَا إِخْوَتِي لِلَّهِ وَلِكَلِمَةِ نِعْمَتِهِ الْقَادِرَةِ أَنْ تَبْنِيَكُمْ وَتُعْطِيَكُمْ مِيرَاثاً مَعَ جَمِيعِ الْمُقَدَّسِينَ.
\par 33 فِضَّةَ أَوْ ذَهَبَ أَوْ لِبَاسَ أَحَدٍ لَمْ أَشْتَهِ.
\par 34 أَنْتُمْ تَعْلَمُونَ أَنَّ حَاجَاتِي وَحَاجَاتِ الَّذِينَ مَعِي خَدَمَتْهَا هَاتَانِ الْيَدَانِ.
\par 35 فِي كُلِّ شَيْءٍ أَرَيْتُكُمْ أَنَّهُ هَكَذَا يَنْبَغِي أَنَّكُمْ تَتْعَبُونَ وَتَعْضُدُونَ الضُّعَفَاءَ مُتَذَكِّرِينَ كَلِمَاتِ الرَّبِّ يَسُوعَ أَنَّهُ قَالَ: مَغْبُوطٌ هُوَ الْعَطَاءُ أَكْثَرُ مِنَ الأَخْذِ».
\par 36 وَلَمَّا قَالَ هَذَا جَثَا عَلَى رُكْبَتَيْهِ مَعَ جَمِيعِهِمْ وَصَلَّى.
\par 37 وَكَانَ بُكَاءٌ عَظِيمٌ مِنَ الْجَمِيعِ وَوَقَعُوا عَلَى عُنُقِ بُولُسَ يُقَبِّلُونَهُ
\par 38 مُتَوَجِّعِينَ وَلاَ سِيَّمَا مِنَ الْكَلِمَةِ الَّتِي قَالَهَا: إِنَّهُمْ لَنْ يَرَوْا وَجْهَهُ أَيْضاً. ثُمَّ شَيَّعُوهُ إِلَى السَّفِينَةِ.

\chapter{21}

\par 1 وَلَمَّا انْفَصَلْنَا عَنْهُمْ أَقْلَعْنَا وَجِئْنَا مُتَوَجِّهِينَ بِالاِسْتِقَامَةِ إِلَى كُوسَ وَفِي الْيَوْمِ التَّالِي إِلَى رُودُسَ وَمِنْ هُنَاكَ إِلَى بَاتَرَا.
\par 2 فَإِذْ وَجَدْنَا سَفِينَةً عَابِرَةً إِلَى فِينِيقِيَةَ صَعِدْنَا إِلَيْهَا وَأَقْلَعْنَا.
\par 3 ثُمَّ اطَّلَعْنَا عَلَى قُبْرُسَ وَتَرَكْنَاهَا يَسْرَةً وَسَافَرْنَا إِلَى سُورِيَّةَ وَأَقْبَلْنَا إِلَى صُورَ لأَنَّ هُنَاكَ كَانَتِ السَّفِينَةُ تَضَعُ وَسْقَهَا.
\par 4 وَإِذْ وَجَدْنَا التَّلاَمِيذَ مَكَثْنَا هُنَاكَ سَبْعَةَ أَيَّامٍ. وَكَانُوا يَقُولُونَ لِبُولُسَ بِالرُّوحِ أَنْ لاَ يَصْعَدَ إِلَى أُورُشَلِيمَ.
\par 5 وَلَكِنْ لَمَّا اسْتَكْمَلْنَا الأَيَّامَ خَرَجْنَا ذَاهِبِينَ وَهُمْ جَمِيعاً يُشَيِّعُونَنَا مَعَ النِّسَاءِ وَالأَوْلاَدِ إِلَى خَارِجِ الْمَدِينَةِ. فَجَثَوْنَا عَلَى رُكَبِنَا عَلَى الشَّاطِئِ وَصَلَّيْنَا.
\par 6 وَلَمَّا وَدَّعْنَا بَعْضُنَا بَعْضاً صَعِدْنَا إِلَى السَّفِينَةِ. وَأَمَّا هُمْ فَرَجَعُوا إِلَى خَاصَّتِهِمْ.
\par 7 وَلَمَّا أَكْمَلْنَا السَّفَرَ فِي الْبَحْرِ مِنْ صُورَ أَقْبَلْنَا إِلَى بُتُولِمَايِسَ فَسَلَّمْنَا عَلَى الإِخْوَةِ وَمَكَثْنَا عِنْدَهُمْ يَوْماً وَاحِداً.
\par 8 ثُمَّ خَرَجْنَا فِي الْغَدِ نَحْنُ رُفَقَاءَ بُولُسَ وَجِئْنَا إِلَى قَيْصَرِيَّةَ فَدَخَلْنَا بَيْتَ فِيلُبُّسَ الْمُبَشِّرِ إِذْ كَانَ وَاحِداً مِنَ السَّبْعَةِ وَأَقَمْنَا عِنْدَهُ.
\par 9 وَكَانَ لِهَذَا أَرْبَعُ بَنَاتٍ عَذَارَى كُنَّ يَتَنَبَّأْنَ.
\par 10 وَبَيْنَمَا نَحْنُ مُقِيمُونَ أَيَّاماً كَثِيرَةً انْحَدَرَ مِنَ الْيَهُودِيَّةِ نَبِيٌّ اسْمُهُ أَغَابُوسُ.
\par 11 فَجَاءَ إِلَيْنَا وَأَخَذَ مِنْطَقَةَ بُولُسَ وَرَبَطَ يَدَيْ نَفْسِهِ وَرِجْلَيْهِ وَقَالَ: «هَذَا يَقُولُهُ الرُّوحُ الْقُدُسُ: الرَّجُلُ الَّذِي لَهُ هَذِهِ الْمِنْطَقَةُ هَكَذَا سَيَرْبُطُهُ الْيَهُودُ فِي أُورُشَلِيمَ وَيُسَلِّمُونَهُ إِلَى أَيْدِي الْأُمَمِ».
\par 12 فَلَمَّا سَمِعْنَا هَذَا طَلَبْنَا إِلَيْهِ نَحْنُ وَالَّذِينَ مِنَ الْمَكَانِ أَنْ لاَ يَصْعَدَ إِلَى أُورُشَلِيمَ.
\par 13 فَأَجَابَ بُولُسُ: «مَاذَا تَفْعَلُونَ؟ تَبْكُونَ وَتَكْسِرُونَ قَلْبِي. لأَنِّي مُسْتَعِدٌّ لَيْسَ أَنْ أُرْبَطَ فَقَطْ بَلْ أَنْ أَمُوتَ أَيْضاً فِي أُورُشَلِيمَ لأَجْلِ اسْمِ الرَّبِّ يَسُوعَ».
\par 14 وَلَمَّا لَمْ يُقْنَعْ سَكَتْنَا قَائِلِينَ: «لِتَكُنْ مَشِيئَةُ الرَّبِّ».
\par 15 وَبَعْدَ تِلْكَ الأَيَّامِ تَأَهَّبْنَا وَصَعِدْنَا إِلَى أُورُشَلِيمَ.
\par 16 وَجَاءَ أَيْضاً مَعَنَا مِنْ قَيْصَرِيَّةَ أُنَاسٌ مِنَ التَّلاَمِيذِ ذَاهِبِينَ بِنَا إِلَى مَنَاسُونَ وَهُوَ رَجُلٌ قُبْرُسِيٌّ تِلْمِيذٌ قَدِيمٌ لِنَنْزِلَ عِنْدَهُ.
\par 17 وَلَمَّا وَصَلْنَا إِلَى أُورُشَلِيمَ قَبِلَنَا الإِخْوَةُ بِفَرَحٍ.
\par 18 وَفِي الْغَدِ دَخَلَ بُولُسُ مَعَنَا إِلَى يَعْقُوبَ وَحَضَرَ جَمِيعُ الْمَشَايِخِ.
\par 19 فَبَعْدَ مَا سَلَّمَ عَلَيْهِمْ طَفِقَ يُحَدِّثُهُمْ شَيْئاً فَشَيْئاً بِكُلِّ مَا فَعَلَهُ اللهُ بَيْنَ الْأُمَمِ بِوَاسِطَةِ خِدْمَتِهِ.
\par 20 فَلَمَّا سَمِعُوا كَانُوا يُمَجِّدُونَ الرَّبَّ. وَقَالُوا لَهُ: «أَنْتَ تَرَى أَيُّهَا الأَخُ كَمْ يُوجَدُ رَبْوَةً مِنَ الْيَهُودِ الَّذِينَ آمَنُوا وَهُمْ جَمِيعاً غَيُورُونَ لِلنَّامُوسِ.
\par 21 وَقَدْ أُخْبِرُوا عَنْكَ أَنَّكَ تُعَلِّمُ جَمِيعَ الْيَهُودِ الَّذِينَ بَيْنَ الْأُمَمِ الاِرْتِدَادَ عَنْ مُوسَى قَائِلاً أَنْ لاَ يَخْتِنُوا أَوْلاَدَهُمْ وَلاَ يَسْلُكُوا حَسَبَ الْعَوَائِدِ.
\par 22 فَإِذاً مَاذَا يَكُونُ؟ لاَ بُدَّ عَلَى كُلِّ حَالٍ أَنْ يَجْتَمِعَ الْجُمْهُورُ لأَنَّهُمْ سَيَسْمَعُونَ أَنَّكَ قَدْ جِئْتَ.
\par 23 فَافْعَلْ هَذَا الَّذِي نَقُولُ لَكَ: عِنْدَنَا أَرْبَعَةُ رِجَالٍ عَلَيْهِمْ نَذْرٌ.
\par 24 خُذْ هَؤُلاَءِ وَتَطهَّرْ مَعَهُمْ وَأَنْفِقْ عَلَيْهِمْ لِيَحْلِقُوا رُؤُوسَهُمْ فَيَعْلَمَ الْجَمِيعُ أَنْ لَيْسَ شَيْءٌ مِمَّا أُخْبِرُوا عَنْكَ بَلْ تَسْلُكُ أَنْتَ أَيْضاً حَافِظاً لِلنَّامُوسِ.
\par 25 وَأَمَّا مِنْ جِهَةِ الَّذِينَ آمَنُوا مِنَ الْأُمَمِ فَأَرْسَلْنَا نَحْنُ إِلَيْهِمْ وَحَكَمْنَا أَنْ لاَ يَحْفَظُوا شَيْئاً مِثْلَ ذَلِكَ سِوَى أَنْ يُحَافِظُوا عَلَى أَنْفُسِهِمْ مِمَّا ذُبِحَ لِلأَصْنَامِ وَمِنَ الدَّمِ وَالْمَخْنُوقِ وَالزِّنَا».
\par 26 حِينَئِذٍ أَخَذَ بُولُسُ الرِّجَالَ فِي الْغَدِ وَتَطَهَّرَ مَعَهُمْ وَدَخَلَ الْهَيْكَلَ مُخْبِراً بِكَمَالِ أَيَّامِ التَّطْهِيرِ إِلَى أَنْ يُقَرَّبَ عَنْ كُلِّ وَاحِدٍ مِنْهُمُ الْقُرْبَانُ.
\par 27 وَلَمَّا قَارَبَتِ الأَيَّامُ السَّبْعَةُ أَنْ تَتِمَّ رَآهُ الْيَهُودُ الَّذِينَ مِنْ أَسِيَّا فِي الْهَيْكَلِ فَأَهَاجُوا كُلَّ الْجَمْعِ وَأَلْقَوْا عَلَيْهِ الأَيَادِيَ
\par 28 صَارِخِينَ: «يَا أَيُّهَا الرِّجَالُ الإِسْرَائِيلِيُّونَ أَعِينُوا! هَذَا هُوَ الرَّجُلُ الَّذِي يُعَلِّمُ الْجَمِيعَ فِي كُلِّ مَكَانٍ ضِدّاً لِلشَّعْبِ وَالنَّامُوسِ وَهَذَا الْمَوْضِعِ حَتَّى أَدْخَلَ يُونَانِيِّينَ أَيْضاً إِلَى الْهَيْكَلِ وَدَنَّسَ هَذَا الْمَوْضِعَ الْمُقَدَّسَ».
\par 29 لأَنَّهُمْ كَانُوا قَدْ رَأَوْا مَعَهُ فِي الْمَدِينَةِ تُرُوفِيمُسَ الأَفَسُسِيَّ فَكَانُوا يَظُنُّونَ أَنَّ بُولُسَ أَدْخَلَهُ إِلَى الْهَيْكَلِ.
\par 30 فَهَاجَتِ الْمَدِينَةُ كُلُّهَا وَتَرَاكَضَ الشَّعْبُ وَأَمْسَكُوا بُولُسَ وَجَرُّوهُ خَارِجَ الْهَيْكَلِ. وَلِلْوَقْتِ أُغْلِقَتِ الأَبْوَابُ.
\par 31 وَبَيْنَمَا هُمْ يَطْلُبُونَ أَنْ يَقْتُلُوهُ نَمَا خَبَرٌ إِلَى أَمِيرِ الْكَتِيبَةِ أَنَّ أُورُشَلِيمَ كُلَّهَا قَدِ اضْطَرَبَتْ
\par 32 فَلِلْوَقْتِ أَخَذَ عَسْكَراً وَقُوَّادَ مِئَاتٍ وَرَكَضَ إِلَيْهِمْ. فَلَمَّا رَأُوا الأَمِيرَ وَالْعَسْكَرَ كَفُّوا عَنْ ضَرْبِ بُولُسَ.
\par 33 حِينَئِذٍ اقْتَرَبَ الأَمِيرُ وَأَمْسَكَهُ وَأَمَرَ أَنْ يُقَيَّدَ بِسِلْسِلَتَيْنِ وَطَفِقَ يَسْتَخْبِرُ: تُرَى مَنْ يَكُونُ وَمَاذَا فَعَلَ؟
\par 34 وَكَانَ الْبَعْضُ يَصْرُخُونَ بِشَيْءٍ وَالْبَعْضُ بِشَيْءٍ آخَرَ فِي الْجَمْعِ. وَلَمَّا لَمْ يَقْدِرْ أَنْ يَعْلَمَ الْيَقِينَ لِسَبَبِ الشَّغَبِ أَمَرَ أَنْ يُذْهَبَ بِهِ إِلَى الْمُعَسْكَرِ.
\par 35 وَلَمَّا صَارَ عَلَى الدَّرَجِ اتَّفَقَ أَنَّ الْعَسْكَرَ حَمَلَهُ بِسَبَبِ عُنْفِ الْجَمْعِ
\par 36 لأَنَّ جُمْهُورَ الشَّعْبِ كَانُوا يَتْبَعُونَهُ صَارِخِينَ: «خُذْهُ!».
\par 37 وَإِذْ قَارَبَ بُولُسُ أَنْ يَدْخُلَ الْمُعَسْكَرَ قَالَ لِلأَمِيرِ: «أَيَجُوزُ لِي أَنْ أَقُولَ لَكَ شَيْئاً؟» فَقَالَ: «أَتَعْرِفُ الْيُونَانِيَّةَ؟
\par 38 أَفَلَسْتَ أَنْتَ الْمِصْرِيَّ الَّذِي صَنَعَ قَبْلَ هَذِهِ الأَيَّامِ فِتْنَةً وَأَخْرَجَ إِلَى الْبَرِّيَّةِ أَرْبَعَةَ الآلاَفِ الرَّجُلِ مِنَ الْقَتَلَةِ؟».
\par 39 فَقَالَ بُولُسُ: «أَنَا رَجُلٌ يَهُودِيٌّ طَرْسُوسِيٌّ مِنْ أَهْلِ مَدِينَةٍ غَيْرِ دَنِيَّةٍ مِنْ كِيلِيكِيَّةَ. وَأَلْتَمِسُ مِنْكَ أَنْ تَأْذَنَ لِي أَنْ أُكَلِّمَ الشَّعْبَ».
\par 40 فَلَمَّا أَذِنَ لَهُ وَقَفَ بُولُسُ عَلَى الدَّرَجِ وَأَشَارَ بِيَدِهِ إِلَى الشَّعْبِ فَصَارَ سُكُوتٌ عَظِيمٌ. فَنَادَى بِاللُّغَةِ الْعِبْرَانِيَّةِ قَائلاً:

\chapter{22}

\par 1 «أَيُّهَا الرِّجَالُ الإِخْوَةُ وَالآبَاءُ اسْمَعُوا احْتِجَاجِي الآنَ لَدَيْكُمْ».
\par 2 فَلَمَّا سَمِعُوا أَنَّهُ يُنَادِي لَهُمْ بِاللُّغَةِ الْعِبْرَانِيَّةِ أَعْطُوا سُكُوتاً أَحْرَى. فَقَالَ:
\par 3 «أَنَا رَجُلٌ يَهُودِيٌّ وُلِدْتُ فِي طَرْسُوسَ كِيلِيكِيَّةَ وَلَكِنْ رَبَيْتُ فِي هَذِهِ الْمَدِينَةِ مُؤَدَّباً عِنْدَ رِجْلَيْ غَمَالاَئِيلَ عَلَى تَحْقِيقِ النَّامُوسِ الأَبَوِيِّ. وَكُنْتُ غَيُوراً لِلَّهِ كَمَا أَنْتُمْ جَمِيعُكُمُ الْيَوْمَ.
\par 4 وَاضْطَهَدْتُ هَذَا الطَّرِيقَ حَتَّى الْمَوْتِ مُقَيِّداً وَمُسَلِّماً إِلَى السُّجُونِ رِجَالاً وَنِسَاءً
\par 5 كَمَا يَشْهَدُ لِي أَيْضاً رَئِيسُ الْكَهَنَةِ وَجَمِيعُ الْمَشْيَخَةِ الَّذِينَ إِذْ أَخَذْتُ أَيْضاً مِنْهُمْ رَسَائِلَ لِلإِخْوَةِ إِلَى دِمَشْقَ ذَهَبْتُ لِآتِيَ بِالَّذِينَ هُنَاكَ إِلَى أُورُشَلِيمَ مُقَيَّدِينَ لِكَيْ يُعَاقَبُوا.
\par 6 فَحَدَثَ لِي وَأَنَا ذَاهِبٌ وَمُتَقَرِّبٌ إِلَى دِمَشْقَ أَنَّهُ نَحْوَ نِصْفِ النَّهَارِ بَغْتَةً أَبْرَقَ حَوْلِي مِنَ السَّمَاءِ نُورٌ عَظِيمٌ.
\par 7 فَسَقَطْتُ عَلَى الأَرْضِ وَسَمِعْتُ صَوْتاً قَائِلاً لِي: شَاوُلُ شَاوُلُ لِمَاذَا تَضْطَهِدُنِي؟
\par 8 فَأَجَبْتُ: مَنْ أَنْتَ يَا سَيِّدُ؟ فَقَالَ لِي: أَنَا يَسُوعُ النَّاصِرِيُّ الَّذِي أَنْتَ تَضْطَهِدُهُ.
\par 9 وَالَّذِينَ كَانُوا مَعِي نَظَرُوا النُّورَ وَارْتَعَبُوا وَلَكِنَّهُمْ لَمْ يَسْمَعُوا صَوْتَ الَّذِي كَلَّمَنِي.
\par 10 فَقُلْتُ: مَاذَا أَفْعَلُ يَا رَبُّ؟ فَقَالَ لِي الرَّبُّ: قُمْ وَاذْهَبْ إِلَى دِمَشْقَ وَهُنَاكَ يُقَالُ لَكَ عَنْ جَمِيعِ مَا تَرَتَّبَ لَكَ أَنْ تَفْعَلَ.
\par 11 وَإِذْ كُنْتُ لاَ أُبْصِرُ مِنْ أَجْلِ بَهَاءِ ذَلِكَ النُّورِ اقْتَادَنِي بِيَدِي الَّذِينَ كَانُوا مَعِي فَجِئْتُ إِلَى دِمَشْقَ.
\par 12 «ثُمَّ إِنَّ حَنَانِيَّا رَجُلاً تَقِيّاً حَسَبَ النَّامُوسِ وَمَشْهُوداً لَهُ مِنْ جَمِيعِ الْيَهُودِ السُّكَّانِ
\par 13 أَتَى إِلَيَّ وَوَقَفَ وَقَالَ لِي: أَيُّهَا الأَخُ شَاوُلُ أَبْصِرْ! فَفِي تِلْكَ السَّاعَةِ نَظَرْتُ إِلَيْهِ
\par 14 فَقَالَ: إِلَهُ آبَائِنَا انْتَخَبَكَ لِتَعْلَمَ مَشِيئَتَهُ وَتُبْصِرَ الْبَارَّ وَتَسْمَعَ صَوْتاً مِنْ فَمِهِ.
\par 15 لأَنَّكَ سَتَكُونُ لَهُ شَاهِداً لِجَمِيعِ النَّاسِ بِمَا رَأَيْتَ وَسَمِعْتَ.
\par 16 وَالآنَ لِمَاذَا تَتَوَانَى؟ قُمْ وَاعْتَمِدْ وَاغْسِلْ خَطَايَاكَ دَاعِياً بِاسْمِ الرَّبِّ.
\par 17 وَحَدَثَ لِي بَعْدَ مَا رَجَعْتُ إِلَى أُورُشَلِيمَ وَكُنْتُ أُصَلِّي فِي الْهَيْكَلِ أَنِّي حَصَلْتُ فِي غَيْبَةٍ
\par 18 فَرَأَيْتُهُ قَائِلاً لِي: أَسْرِعْ وَاخْرُجْ عَاجِلاً مِنْ أُورُشَلِيمَ لأَنَّهُمْ لاَ يَقْبَلُونَ شَهَادَتَكَ عَنِّي.
\par 19 فَقُلْتُ: يَا رَبُّ هُمْ يَعْلَمُونَ أَنِّي كُنْتُ أَحْبِسُ وَأَضْرِبُ فِي كُلِّ مَجْمَعٍ الَّذِينَ يُؤْمِنُونَ بِكَ.
\par 20 وَحِينَ سُفِكَ دَمُ اسْتِفَانُوسَ شَهِيدِكَ كُنْتُ أَنَا وَاقِفاً وَرَاضِياً بِقَتْلِهِ وَحَافِظاً ثِيَابَ الَّذِينَ قَتَلُوهُ.
\par 21 فَقَالَ لِي: اذْهَبْ فَإِنِّي سَأُرْسِلُكَ إِلَى الْأُمَمِ بَعِيداً».
\par 22 فَسَمِعُوا لَهُ حَتَّى هَذِهِ الْكَلِمَةَِ ثُمَّ صَرَخُوا قَائِلِينَ: «خُذْ مِثْلَ هَذَا مِنَ الأَرْضِ لأَنَّهُ كَانَ لاَ يَجُوزُ أَنْ يَعِيشَ».
\par 23 وَإِذْ كَانُوا يَصِيحُونَ وَيَطْرَحُونَ ثِيَابَهُمْ وَيَرْمُونَ غُبَاراً إِلَى الْجَوِّ
\par 24 أَمَرَ الأَمِيرُ أَنْ يُذْهَبَ بِهِ إِلَى الْمُعَسْكَرِ قَائِلاً أَنْ يُفْحَصَ بِضَرَبَاتٍ لِيَعْلَمَ لأَيِّ سَبَبٍ كَانُوا يَصْرُخُونَ عَلَيْهِ هَكَذَا.
\par 25 فَلَمَّا مَدُّوهُ لِلسِّيَاطِ قَالَ بُولُسُ لِقَائِدِ الْمِئَةِ الْوَاقِفِ: «أَيَجُوزُ لَكُمْ أَنْ تَجْلِدُوا إِنْسَاناً رُومَانِيّاً غَيْرَ مَقْضِيٍّ عَلَيْهِ؟»
\par 26 فَإِذْ سَمِعَ قَائِدُ الْمِئَةِ ذَهَبَ إِلَى الأَمِيرِ وَأَخْبَرَهُ قَائِلاً: «انْظُرْ مَاذَا أَنْتَ مُزْمِعٌ أَنْ تَفْعَلَ! لأَنَّ هَذَا الرَّجُلَ رُومَانِيٌّ».
\par 27 فَجَاءَ الأَمِيرُ وَقَالَ لَهُ: «قُلْ لِي. أَأَنْتَ رُومَانِيٌّ؟» فَقَالَ: «نَعَمْ».
\par 28 فَأَجَابَ الأَمِيرُ: «أَمَّا أَنَا فَبِمَبْلَغٍ كَبِيرٍ اقْتَنَيْتُ هَذِهِ الرَّعَوِيَّةَ». فَقَالَ بُولُسُ: «أَمَّا أَنَا فَقَدْ وُلِدْتُ فِيهَا».
\par 29 وَلِلْوَقْتِ تَنَحَّى عَنْهُ الَّذِينَ كَانُوا مُزْمِعِينَ أَنْ يَفْحَصُوهُ. وَاخْتَشَى الأَمِيرُ لَمَّا عَلِمَ أَنَّهُ رُومَانِيٌّ وَلأَنَّهُ قَدْ قَيَّدَهُ.
\par 30 وَفِي الْغَدِ إِذْ كَانَ يُرِيدُ أَنْ يَعْلَمَ الْيَقِينَ: لِمَاذَا يَشْتَكِي الْيَهُودُ عَلَيْهِ؟ حَلَّهُ مِنَ الرِّبَاطِ وَأَمَرَ أَنْ يَحْضُرَ رُؤَسَاءُ الْكَهَنَةِ وَكُلُّ مَجْمَعِهِمْ. فَأَحْضَرَ بُولُسَ وَأَقَامَهُ لَدَيْهِمْ.

\chapter{23}

\par 1 فَتَفَرَّسَ بُولُسُ فِي الْمَجْمَعِ وَقَالَ: «أَيُّهَا الرِّجَالُ الإِخْوَةُ إِنِّي بِكُلِّ ضَمِيرٍ صَالِحٍ قَدْ عِشْتُ لِلَّهِ إِلَى هَذَا الْيَوْمِ».
\par 2 فَأَمَرَ حَنَانِيَّا رَئِيسُ الْكَهَنَةِ الْوَاقِفِينَ عِنْدَهُ أَنْ يَضْرِبُوهُ عَلَى فَمِهِ.
\par 3 حِينَئِذٍ قَالَ لَهُ بُولُسُ: «سَيَضْرِبُكَ اللهُ أَيُّهَا الْحَائِطُ الْمُبَيَّضُ! أَفَأَنْتَ جَالِسٌ تَحْكُمُ عَلَيَّ حَسَبَ النَّامُوسِ وَأَنْتَ تَأْمُرُ بِضَرْبِي مُخَالِفاً لِلنَّامُوسِ؟»
\par 4 فَقَالَ الْوَاقِفُونَ: «أَتَشْتِمُ رَئِيسَ كَهَنَةِ اللهِ؟»
\par 5 فَقَالَ بُولُسُ: «لَمْ أَكُنْ أَعْرِفُ أَيُّهَا الإِخْوَةُ أَنَّهُ رَئِيسُ كَهَنَةٍ لأَنَّهُ مَكْتُوبٌ: رَئِيسُ شَعْبِكَ لاَ تَقُلْ فِيهِ سُوءاً».
\par 6 وَلَمَّا عَلِمَ بُولُسُ أَنَّ قِسْماً مِنْهُمْ صَدُّوقِيُّونَ وَالآخَرَ فَرِّيسِيُّونَ صَرَخَ فِي الْمَجْمَعِ: «أَيُّهَا الرِّجَالُ الإِخْوَةُ أَنَا فَرِّيسِيٌّ ابْنُ فَرِّيسِيٍّ. عَلَى رَجَاءِ قِيَامَةِ الأَمْوَاتِ أَنَا أُحَاكَمُ».
\par 7 وَلَمَّا قَالَ هَذَا حَدَثَتْ مُنَازَعَةٌ بَيْنَ الْفَرِّيسِيِّينَ وَالصَّدُّوقِيِّينَ وَانْشَقَّتِ الْجَمَاعَةُ
\par 8 لأَنَّ الصَّدُّوقِيِّينَ يَقُولُونَ إِنَّهُ لَيْسَ قِيَامَةٌ وَلاَ مَلاَكٌ وَلاَ رُوحٌ وَأَمَّا الْفَرِّيسِيُّونَ فَيُقِرُّونَ بِكُلِّ ذَلِكَ.
\par 9 فَحَدَثَ صِيَاحٌ عَظِيمٌ وَنَهَضَ كَتَبَةُ قِسْمِ الْفَرِّيسِيِّينَ وَطَفِقُوا يُخَاصِمُونَ قَائِلِينَ: «لَسْنَا نَجِدُ شَيْئاً رَدِيّاً فِي هَذَا الإِنْسَانِ! وَإِنْ كَانَ رُوحٌ أَوْ مَلاَكٌ قَدْ كَلَّمَهُ فَلاَ نُحَارِبَنَّ اللهَ».
\par 10 وَلَمَّا حَدَثَتْ مُنَازَعَةٌ كَثِيرَةٌ اخْتَشَى الأَمِيرُ أَنْ يَفْسَخُوا بُولُسَ فَأَمَرَ الْعَسْكَرَ أَنْ يَنْزِلُوا وَيَخْتَطِفُوهُ مِنْ وَسَطِهِمْ وَيَأْتُوا بِهِ إِلَى الْمُعَسْكَرِ.
\par 11 وَفِي اللَّيْلَةِ التَّالِيَةِ وَقَفَ بِهِ الرَّبُّ وَقَالَ: «ثِقْ يَا بُولُسُ لأَنَّكَ كَمَا شَهِدْتَ بِمَا لِي فِي أُورُشَلِيمَ هَكَذَا يَنْبَغِي أَنْ تَشْهَدَ فِي رُومِيَةَ أَيْضاً».
\par 12 وَلَمَّا صَارَ النَّهَارُ صَنَعَ بَعْضُ الْيَهُودِ اتِّفَاقاً وَحَرَمُوا أَنْفُسَهُمْ قَائِلِينَ إِنَّهُمْ لاَ يَأْكُلُونَ وَلاَ يَشْرَبُونَ حَتَّى يَقْتُلُوا بُولُسَ.
\par 13 وَكَانَ الَّذِينَ صَنَعُوا هَذَا التَّحَالُفَ أَكْثَرَ مِنْ أَرْبَعِينَ.
\par 14 فَتَقَدَّمُوا إِلَى رُؤَسَاءِ الْكَهَنَةِ وَالشُّيُوخِ وَقَالُوا: «قَدْ حَرَمْنَا أَنْفُسَنَا حِرْماً أَنْ لاَ نَذُوقَ شَيْئاً حَتَّى نَقْتُلَ بُولُسَ.
\par 15 وَالآنَ أَعْلِمُوا الأَمِيرَ أَنْتُمْ مَعَ الْمَجْمَعِ لِكَيْ يُنْزِلَهُ إِلَيْكُمْ غَداً كَأَنَّكُمْ مُزْمِعُونَ أَنْ تَفْحَصُوا بِأَكْثَرِ تَدْقِيقٍ عَمَّا لَهُ. وَنَحْنُ قَبْلَ أَنْ يَقْتَرِبَ مُسْتَعِدُّونَ لِقَتْلِهِ».
\par 16 وَلَكِنَّ ابْنَ أُخْتِ بُولُسَ سَمِعَ بِالْكَمِينِ فَجَاءَ وَدَخَلَ الْمُعَسْكَرَ وَأَخْبَرَ بُولُسَ.
\par 17 فَاسْتَدْعَى بُولُسُ وَاحِداً مِنْ قُوَّادِ الْمِئَاتِ وَقَالَ: «اذْهَبْ بِهَذَا الشَّابِّ إِلَى الأَمِيرِ لأَنَّ عِنْدَهُ شَيْئاً يُخْبِرُهُ بِهِ».
\par 18 فَأَخَذَهُ وَأَحْضَرَهُ إِلَى الأَمِيرِ وَقَالَ: «اسْتَدْعَانِي الأَسِيرُ بُولُسُ وَطَلَبَ أَنْ أُحْضِرَ هَذَا الشَّابَّ إِلَيْكَ وَهُوَ عِنْدَهُ شَيْءٌ لِيَقُولَهُ لَكَ».
\par 19 فَأَخَذَ الأَمِيرُ بِيَدِهِ وَتَنَحَّى بِهِ مُنْفَرِداً وَاسْتَخْبَرَهُ: «مَا هُوَ الَّذِي عِنْدَكَ لِتُخْبِرَنِي بِهِ؟»
\par 20 فَقَالَ: «إِنَّ الْيَهُودَ تَعَاهَدُوا أَنْ يَطْلُبُوا مِنْكَ أَنْ تُنْزِلَ بُولُسَ غَداً إِلَى الْمَجْمَعِ كَأَنَّهُمْ مُزْمِعُونَ أَنْ يَسْتَخْبِرُوا عَنْهُ بِأَكْثَرِ تَدْقِيقٍ.
\par 21 فَلاَ تَنْقَدْ إِلَيْهِمْ لأَنَّ أَكْثَرَ مِنْ أَرْبَعِينَ رَجُلاً مِنْهُمْ كَامِنُونَ لَهُ قَدْ حَرَمُوا أَنْفُسَهُمْ أَنْ لاَ يَأْكُلُوا وَلاَ يَشْرَبُوا حَتَّى يَقْتُلُوهُ. وَهُمُ الآنَ مُسْتَعِدُّونَ مُنْتَظِرُونَ الْوَعْدَ مِنْكَ».
\par 22 فَأَطْلَقَ الأَمِيرُ الشَّابَّ مُوصِياً إِيَّاهُ أَنْ: «لاَ تَقُلْ لأَحَدٍ إِنَّكَ أَعْلَمْتَنِي بِهَذَا».
\par 23 ثُمَّ دَعَا اثْنَيْنِ مِنْ قُوَّادِ الْمِئَاتِ وَقَالَ: «أَعِدَّا مِئَتَيْ عَسْكَرِيٍّ لِيَذْهَبُوا إِلَى قَيْصَرِيَّةَ وَسَبْعِينَ فَارِساً وَمِئَتَيْ رَامِحٍ مِنَ السَّاعَةِ الثَّالِثَةِ مِنَ اللَّيْلِ.
\par 24 وَأَنْ يُقَدِّمَا دَوَابَّ لِيُرْكِبَا بُولُسَ وَيُوصِلاَهُ سَالِماً إِلَى فِيلِكْسَ الْوَالِي».
\par 25 وَكَتَبَ رِسَالَةً حَاوِيَةً هَذِهِ الصُّورَةَ:
\par 26 «كُلُودِيُوسُ لِيسِيَاسُ يُهْدِي سَلاَماً إِلَى الْعَزِيزِ فِيلِكْسَ الْوَالِي.
\par 27 هَذَا الرَّجُلُ لَمَّا أَمْسَكَهُ الْيَهُودُ وَكَانُوا مُزْمِعِينَ أَنْ يَقْتُلُوهُ أَقْبَلْتُ مَعَ الْعَسْكَرِ وَأَنْقَذْتُهُ إِذْ أُخْبِرْتُ أَنَّهُ رُومَانِيٌّ.
\par 28 وَكُنْتُ أُرِيدُ أَنْ أَعْلَمَ الْعِلَّةَ الَّتِي لأَجْلِهَا كَانُوا يَشْتَكُونَ عَلَيْهِ فَأَنْزَلْتُهُ إِلَى مَجْمَعِهِمْ
\par 29 فَوَجَدْتُهُ مَشْكُوّاً عَلَيْهِ مِنْ جِهَةِ مَسَائِلِ نَامُوسِهِمْ. وَلَكِنَّ شَكْوَى تَسْتَحِقُّ الْمَوْتَ أَوِ الْقُيُودَ لَمْ تَكُنْ عَلَيْهِ.
\par 30 ثُمَّ لَمَّا أُعْلِمْتُ بِمَكِيدَةٍ عَتِيدَةٍ أَنْ تَصِيرَ عَلَى الرَّجُلِ مِنَ الْيَهُودِ أَرْسَلْتُهُ لِلْوَقْتِ إِلَيْكَ آمِراً الْمُشْتَكِينَ أَيْضاً أَنْ يَقُولُوا لَدَيْكَ مَا عَلَيْهِ. كُنْ مُعَافىً».
\par 31 فَالْعَسْكَرُ أَخَذُوا بُولُسَ كَمَا أُمِرُوا وَذَهَبُوا بِهِ لَيْلاً إِلَى أَنْتِيبَاتْرِيسَ.
\par 32 وَفِي الْغَدِ تَرَكُوا الْفُرْسَانَ يَذْهَبُونَ مَعَهُ وَرَجَعُوا إِلَى الْمُعَسْكَرِ.
\par 33 وَأُولَئِكَ لَمَّا دَخَلُوا قَيْصَرِيَّةَ وَدَفَعُوا الرِّسَالَةَ إِلَى الْوَالِي أَحْضَرُوا بُولُسَ أَيْضاً إِلَيْهِ.
\par 34 فَلَمَّا قَرَأَ الْوَالِي الرِّسَالَةَ وَسَأَلَ مِنْ أَيَّةِ وِلاَيَةٍ هُوَ وَوَجَدَ أَنَّهُ مِنْ كِيلِيكِيَّةَ
\par 35 قَالَ: «سَأَسْمَعُكَ مَتَى حَضَرَ الْمُشْتَكُونَ عَلَيْكَ أَيْضاً». وَأَمَرَ أَنْ يُحْرَسَ فِي قَصْرِ هِيرُودُسَ.

\chapter{24}

\par 1 وَبَعْدَ خَمْسَةِ أَيَّامٍ انْحَدَرَ حَنَانِيَّا رَئِيسُ الْكَهَنَةِ مَعَ الشُّيُوخِ وَخَطِيبٍ اسْمُهُ تَرْتُلُّسُ. فَعَرَضُوا لِلْوَالِي ضِدَّ بُولُسَ.
\par 2 فَلَمَّا دُعِيَ ابْتَدَأَ تَرْتُلُّسُ فِي الشِّكَايَةِ قَائِلاً:
\par 3 «إِنَّنَا حَاصِلُونَ بِوَاسِطَتِكَ عَلَى سَلاَمٍ جَزِيلٍ وَقَدْ صَارَتْ لِهَذِهِ الْأُمَّةِ مَصَالِحُ بِتَدْبِيرِكَ. فَنَقْبَلُ ذَلِكَ أَيُّهَا الْعَزِيزُ فِيلِكْسُ بِكُلِّ شُكْرٍ فِي كُلِّ زَمَانٍ وَكُلِّ مَكَانٍ.
\par 4 وَلَكِنْ لِئَلاَّ أُعَوِّقَكَ أَكْثَرَ أَلْتَمِسُ أَنْ تَسْمَعَنَا بِالاِخْتِصَارِ بِحِلْمِكَ.
\par 5 فَإِنَّنَا إِذْ وَجَدْنَا هَذَا الرَّجُلَ مُفْسِداً وَمُهَيِّجَ فِتْنَةٍ بَيْنَ جَمِيعِ الْيَهُودِ الَّذِينَ فِي الْمَسْكُونَةِ وَمِقْدَامَ شِيعَةِ النَّاصِرِيِّينَ
\par 6 وَقَدْ شَرَعَ أَنْ يُنَجِّسَ الْهَيْكَلَ أَيْضاً أَمْسَكْنَاهُ وَأَرَدْنَا أَنْ نَحْكُمَ عَلَيْهِ حَسَبَ نَامُوسِنَا.
\par 7 فَأَقْبَلَ لِيسِيَاسُ الأَمِيرُ بِعُنْفٍ شَدِيدٍ وَأَخَذَهُ مِنْ بَيْنِ أَيْدِينَا
\par 8 وَأَمَرَ الْمُشْتَكِينَ عَلَيْهِ أَنْ يَأْتُوا إِلَيْكَ. وَمِنْهُ يُمْكِنُكَ إِذَا فَحَصْتَ أَنْ تَعْلَمَ جَمِيعَ هَذِهِ الْأُمُورِ الَّتِي نَشْتَكِي بِهَا عَلَيْهِ».
\par 9 ثُمَّ وَافَقَهُ الْيَهُودُ أَيْضاً قَائِلِينَ: «إِنَّ هَذِهِ الْأُمُورَ هَكَذَا».
\par 10 فَأَجَابَ بُولُسُ إِذْ أَوْمَأَ إِلَيْهِ الْوَالِي أَنْ يَتَكَلَّمَ: «إِنِّي إِذْ قَدْ عَلِمْتُ أَنَّكَ مُنْذُ سِنِينَ كَثِيرَةٍ قَاضٍ لِهَذِهِ الْأُمَّةِ أَحْتَجُّ عَمَّا فِي أَمْرِي بِأَكْثَرِ سُرُورٍ.
\par 11 وَأَنْتَ قَادِرٌ أَنْ تَعْرِفَ أَنَّهُ لَيْسَ لِي أَكْثَرُ مِنِ اثْنَيْ عَشَرَ يَوْماً مُنْذُ صَعِدْتُ لأَسْجُدَ فِي أُورُشَلِيمَ.
\par 12 وَلَمْ يَجِدُونِي فِي الْهَيْكَلِ أُحَاجُّ أَحَداً أَوْ أَصْنَعُ تَجَمُّعاً مِنَ الشَّعْبِ وَلاَ فِي الْمَجَامِعِ وَلاَ فِي الْمَدِينَةِ.
\par 13 وَلاَ يَسْتَطِيعُونَ أَنْ يُثْبِتُوا مَا يَشْتَكُونَ بِهِ الآنَ عَلَيَّ.
\par 14 وَلَكِنَّنِي أُقِرُّ لَكَ بِهَذَا: أَنَّنِي حَسَبَ الطَّرِيقِ الَّذِي يَقُولُونَ لَهُ «شِيعَةٌ» هَكَذَا أَعْبُدُ إِلَهَ آبَائِي مُؤْمِناً بِكُلِّ مَا هُوَ مَكْتُوبٌ فِي النَّامُوسِ وَالأَنْبِيَاءِ.
\par 15 وَلِي رَجَاءٌ بِاللَّهِ فِي مَا هُمْ أَيْضاً يَنْتَظِرُونَهُ: أَنَّهُ سَوْفَ تَكُونُ قِيَامَةٌ لِلأَمْوَاتِ الأَبْرَارِ وَالأَثَمَةِ.
\par 16 لِذَلِكَ أَنَا أَيْضاً أُدَرِّبُ نَفْسِي لِيَكُونَ لِي دَائِماً ضَمِيرٌ بِلاَ عَثْرَةٍ مِنْ نَحْوِ اللهِ وَالنَّاسِ.
\par 17 وَبَعْدَ سِنِينَ كَثِيرَةٍ جِئْتُ أَصْنَعُ صَدَقَاتٍ لِأُمَّتِي وَقَرَابِينَ.
\par 18 وَفِي ذَلِكَ وَجَدَنِي مُتَطَهِّراً فِي الْهَيْكَلِ - لَيْسَ مَعَ جَمْعٍ وَلاَ مَعَ شَغَبٍ - قَوْمٌ هُمْ يَهُودٌ مِنْ أَسِيَّا
\par 19 كَانَ يَنْبَغِي أَنْ يَحْضُرُوا لَدَيْكَ وَيَشْتَكُوا إِنْ كَانَ لَهُمْ عَلَيَّ شَيْءٌ.
\par 20 أَوْ لِيَقُلْ هَؤُلاَءِ أَنْفُسُهُمْ مَاذَا وَجَدُوا فِيَّ مِنَ الذَّنْبِ وَأَنَا قَائِمٌ أَمَامَ الْمَجْمَعِ
\par 21 إِلاَّ مِنْ جِهَةِ هَذَا الْقَوْلِ الْوَاحِدِ الَّذِي صَرَخْتُ بِهِ وَاقِفاً بَيْنَهُمْ: أَنِّي مِنْ أَجْلِ قِيَامَةِ الأَمْوَاتِ أُحَاكَمُ مِنْكُمُ الْيَوْمَ».
\par 22 فَلَمَّا سَمِعَ هَذَا فِيلِكْسُ أَمْهَلَهُمْ إِذْ كَانَ يَعْلَمُ بِأَكْثَرِ تَحْقِيقٍ أُمُورَ هَذَا الطَّرِيقِ قَائِلاً: «مَتَى انْحَدَرَ لِيسِيَاسُ الأَمِيرُ أَفْحَصُ عَنْ أُمُورِكُمْ».
\par 23 وَأَمَرَ قَائِدَ الْمِئَةِ أَنْ يُحْرَسَ بُولُسُ وَتَكُونَ لَهُ رُخْصَةٌ وَأَنْ لاَ يَمْنَعَ أَحَداً مِنْ أَصْحَابِهِ أَنْ يَخْدِمَهُ أَوْ يَأْتِيَ إِلَيْهِ.
\par 24 ثُمَّ بَعْدَ أَيَّامٍ جَاءَ فِيلِكْسُ مَعَ دُرُوسِلاَّ امْرَأَتِهِ وَهِيَ يَهُودِيَّةٌ. فَاسْتَحْضَرَ بُولُسَ وَسَمِعَ مِنْهُ عَنِ الإِيمَانِ بِالْمَسِيحِ.
\par 25 وَبَيْنَمَا كَانَ يَتَكَلَّمُ عَنِ الْبِرِّ وَالتَّعَفُّفِ وَالدَّيْنُونَةِ الْعَتِيدَةِ أَنْ تَكُونَ ارْتَعَبَ فِيلِكْسُ وَأَجَابَ: «أَمَّا الآنَ فَاذْهَبْ وَمَتَى حَصَلْتُ عَلَى وَقْتٍ أَسْتَدْعِيكَ».
\par 26 وَكَانَ أَيْضاً يَرْجُو أَنْ يُعْطِيَهُ بُولُسُ دَرَاهِمَ لِيُطْلِقَهُ وَلِذَلِكَ كَانَ يَسْتَحْضِرُهُ مِرَاراً أَكْثَرَ وَيَتَكَلَّمُ مَعَهُ.
\par 27 وَلَكِنْ لَمَّا كَمِلَتْ سَنَتَانِ قَبِلَ فِيلِكْسُ بُورْكِيُوسَ فَسْتُوسَ خَلِيفَةً لَهُ. وَإِذْ كَانَ فِيلِكْسُ يُرِيدُ أَنْ يُودِعَ الْيَهُودَ مِنَّةً تَرَكَ بُولُسَ مُقَيَّداً.

\chapter{25}

\par 1 فَلَمَّا قَدِمَ فَسْتُوسُ إِلَى الْوِلاَيَةِ صَعِدَ بَعْدَ ثَلاَثَةِ أَيَّامٍ مِنْ قَيْصَرِيَّةَ إِلَى أُورُشَلِيمَ.
\par 2 فَعَرَضَ لَهُ رَئِيسُ الْكَهَنَةِ وَوُجُوهُ الْيَهُودِ ضِدَّ بُولُسَ وَالْتَمَسُوا مِنْهُ
\par 3 طَالِبِينَ عَلَيْهِ مِنَّةً أَنْ يَسْتَحْضِرَهُ إِلَى أُورُشَلِيمَ وَهُمْ صَانِعُونَ كَمِيناً لِيَقْتُلُوهُ فِي الطَّرِيقِ.
\par 4 فَأَجَابَ فَسْتُوسُ أَنْ يُحْرَسَ بُولُسُ فِي قَيْصَرِيَّةَ وَأَنَّهُ هُوَ مُزْمِعٌ أَنْ يَنْطَلِقَ عَاجِلاً.
\par 5 وَقَالَ: «فَلْيَنْزِلْ مَعِي الَّذِينَ هُمْ بَيْنَكُمْ مُقْتَدِرُونَ. وَإِنْ كَانَ فِي هَذَا الرَّجُلِ شَيْءٌ فَلْيَشْتَكُوا عَلَيْهِ».
\par 6 وَبَعْدَ مَا صَرَفَ عِنْدَهُمْ أَكْثَرَ مِنْ عَشَرَةِ أَيَّامٍ انْحَدَرَ إِلَى قَيْصَرِيَّةَ. وَفِي الْغَدِ جَلَسَ عَلَى كُرْسِيِّ الْوِلاَيَةِ وَأَمَرَ أَنْ يُؤْتَى بِبُولُسَ.
\par 7 فَلَمَّا حَضَرَ وَقَفَ حَوْلَهُ الْيَهُودُ الَّذِينَ كَانُوا قَدِ انْحَدَرُوا مِنْ أُورُشَلِيمَ وَقَدَّمُوا عَلَى بُولُسَ دَعَاوِيَ كَثِيرَةً وَثَقِيلَةً لَمْ يَقْدِرُوا أَنْ يُبَرْهِنُوهَا.
\par 8 إِذْ كَانَ هُوَ يَحْتَجُّ: «أَنِّي مَا أَخْطَأْتُ بِشَيْءٍ لاَ إِلَى نَامُوسِ الْيَهُودِ وَلاَ إِلَى الْهَيْكَلِ وَلاَ إِلَى قَيْصَرَ».
\par 9 وَلَكِنَّ فَسْتُوسَ إِذْ كَانَ يُرِيدُ أَنْ يُودِعَ الْيَهُودَ مِنَّةً قَالَ لِبُولُسَ: «أَتَشَاءُ أَنْ تَصْعَدَ إِلَى أُورُشَلِيمَ لِتُحَاكَمَ هُنَاكَ لَدَيَّ مِنْ جِهَةِ هَذِهِ الْأُمُورِ؟»
\par 10 فَقَالَ بُولُسُ: «أَنَا وَاقِفٌ لَدَى كُرْسِيِّ وِلاَيَةِ قَيْصَرَ حَيْثُ يَنْبَغِي أَنْ أُحَاكَمَ. أَنَا لَمْ أَظْلِمِ الْيَهُودَ بِشَيْءٍ كَمَا تَعْلَمُ أَنْتَ أَيْضاً جَيِّداً.
\par 11 لأَنِّي إِنْ كُنْتُ آثِماً أَوْ صَنَعْتُ شَيْئاً يَسْتَحِقُّ الْمَوْتَ فَلَسْتُ أَسْتَعْفِي مِنَ الْمَوْتِ. وَلَكِنْ إِنْ لَمْ يَكُنْ شَيْءٌ مِمَّا يَشْتَكِي عَلَيَّ بِهِ هَؤُلاَءِ فَلَيْسَ أَحَدٌ يَسْتَطِيعُ أَنْ يُسَلِّمَنِي لَهُمْ. إِلَى قَيْصَرَ أَنَا رَافِعٌ دَعْوَايَ».
\par 12 حِينَئِذٍ تَكَلَّمَ فَسْتُوسُ مَعَ أَرْبَابِ الْمَشُورَةِ فَأَجَابَ: «إِلَى قَيْصَرَ رَفَعْتَ دَعْوَاكَ. إِلَى قَيْصَرَ تَذْهَبُ».
\par 13 وَبَعْدَمَا مَضَتْ أَيَّامٌ أَقْبَلَ أَغْرِيبَاسُ الْمَلِكُ وَبَرْنِيكِي إِلَى قَيْصَرِيَّةَ لِيُسَلِّمَا عَلَى فَسْتُوسَ.
\par 14 وَلَمَّا كَانَا يَصْرِفَانِ هُنَاكَ أَيَّاماً كَثِيرَةً عَرَضَ فَسْتُوسُ عَلَى الْمَلِكِ أَمْرَ بُولُسَ قَائِلاً: «يُوجَدُ رَجُلٌ تَرَكَهُ فِيلِكْسُ أَسِيراً
\par 15 وَعَرَضَ لِي عَنْهُ رُؤَسَاءُ الْكَهَنَةِ وَمَشَايِخُ الْيَهُودِ لَمَّا كُنْتُ فِي أُورُشَلِيمَ طَالِبِينَ حُكْماً عَلَيْهِ.
\par 16 فَأَجَبْتُهُمْ أَنْ لَيْسَ لِلرُّومَانِ عَادَةٌ أَنْ يُسَلِّمُوا أَحَداً لِلْمَوْتِ قَبْلَ أَنْ يَكُونَ الْمَشْكُّوُ عَلَيْهِ مُواجَهَةً مَعَ الْمُشْتَكِينَ فَيَحْصُلُ عَلَى فُرْصَةٍ لِلاِحْتِجَاجِ عَنِ الشَّكْوَى.
\par 17 فَلَمَّا اجْتَمَعُوا إِلَى هُنَا جَلَسْتُ مِنْ دُونِ إِمْهَالٍ فِي الْغَدِ عَلَى كُرْسِيِّ الْوِلاَيَةِ وَأَمَرْتُ أَنْ يُؤْتَى بِالرَّجُلِ.
\par 18 فَلَمَّا وَقَفَ الْمُشْتَكُونَ حَوْلَهُ لَمْ يَأْتُوا بِعِلَّةٍ وَاحِدَةٍ مِمَّا كُنْتُ أَظُنُّ.
\par 19 لَكِنْ كَانَ لَهُمْ عَلَيْهِ مَسَائِلُ مِنْ جِهَةِ دِيَانَتِهِمْ وَعَنْ وَاحِدٍ اسْمُهُ يَسُوعُ قَدْ مَاتَ وَكَانَ بُولُسُ يَقُولُ إِنَّهُ حَيٌّ.
\par 20 وَإِذْ كُنْتُ مُرْتَاباً فِي الْمَسْأَلَةِ عَنْ هَذَا قُلْتُ: أَلَعَلَّهُ يَشَاءُ أَنْ يَذْهَبَ إِلَى أُورُشَلِيمَ وَيُحَاكَمَ هُنَاكَ مِنْ جِهَةِ هَذِهِ الْأُمُورِ؟
\par 21 وَلَكِنْ لَمَّا رَفَعَ بُولُسُ دَعْوَاهُ لِكَيْ يُحْفَظَ لِفَحْصِ أُوغُسْطُسَ أَمَرْتُ بِحِفْظِهِ إِلَى أَنْ أُرْسِلَهُ إِلَى قَيْصَرَ».
\par 22 فَقَالَ أَغْرِيبَاسُ لِفَسْتُوسَ: «كُنْتُ أُرِيدُ أَنَا أَيْضاً أَنْ أَسْمَعَ الرَّجُلَ». فَقَالَ: «غَداً تَسْمَعُهُ».
\par 23 فَفِي الْغَدِ لَمَّا جَاءَ أَغْرِيبَاسُ وَبَرْنِيكِي فِي احْتِفَالٍ عَظِيمٍ وَدَخَلاَ إِلَى دَارِ الاِسْتِمَاعِ مَعَ الْأُمَرَاءِ وَرِجَالِ الْمَدِينَةِ الْمُقَدَّمِينَ أَمَرَ فَسْتُوسُ فَأُتِيَ بِبُولُسَ.
\par 24 فَقَالَ فَسْتُوسُ: «أَيُّهَا الْمَلِكُ أَغْرِيبَاسُ وَالرِّجَالُ الْحَاضِرُونَ مَعَنَا أَجْمَعُونَ أَنْتُمْ تَنْظُرُونَ هَذَا الَّذِي تَوَسَّلَ إِلَيَّ مِنْ جِهَتِهِ كُلُّ جُمْهُورِ الْيَهُودِ فِي أُورُشَلِيمَ وَهُنَا صَارِخِينَ أَنَّهُ لاَ يَنْبَغِي أَنْ يَعِيشَ بَعْدُ.
\par 25 وَأَمَّا أَنَا فَلَمَّا وَجَدْتُ أَنَّهُ لَمْ يَفْعَلْ شَيْئاً يَسْتَحِقُّ الْمَوْتَ وَهُوَ قَدْ رَفَعَ دَعْوَاهُ إِلَى أُوغُسْطُسَ عَزَمْتُ أَنْ أُرْسِلَهُ.
\par 26 وَلَيْسَ لِي شَيْءٌ يَقِينٌ مِنْ جِهَتِهِ لأَكْتُبَ إِلَى السَّيِّدِ. لِذَلِكَ أَتَيْتُ بِهِ لَدَيْكُمْ وَلاَ سِيَّمَا لَدَيْكَ أَيُّهَا الْمَلِكُ أَغْرِيبَاسُ حَتَّى إِذَا صَارَ الْفَحْصُ يَكُونُ لِي شَيْءٌ لأَكْتُبَ.
\par 27 لأَنِّي أَرَى حَمَاقَةً أَنْ أُرْسِلَ أَسِيراً وَلاَ أُشِيرَ إِلَى الدَّعَاوِي الَّتِي عَلَيْهِ».

\chapter{26}

\par 1 فَقَالَ أَغْرِيبَاسُ لِبُولُسَ: «مَأْذُونٌ لَكَ أَنْ تَتَكَلَّمَ لأَجْلِ نَفْسِكَ». حِينَئِذٍ بَسَطَ بُولُسُ يَدَهُ وَجَعَلَ يَحْتَجُّ:
\par 2 «إِنِّي أَحْسِبُ نَفْسِي سَعِيداً أَيُّهَا الْمَلِكُ أَغْرِيبَاسُ إِذْ أَنَا مُزْمِعٌ أَنْ أَحْتَجَّ الْيَوْمَ لَدَيْكَ عَنْ كُلِّ مَا يُحَاكِمُنِي بِهِ الْيَهُودُ.
\par 3 لاَ سِيَّمَا وَأَنْتَ عَالِمٌ بِجَمِيعِ الْعَوَائِدِ وَالْمَسَائِلِ الَّتِي بَيْنَ الْيَهُودِ. لِذَلِكَ أَلْتَمِسُ مِنْكَ أَنْ تَسْمَعَنِي بِطُولِ الأَنَاةِ.
\par 4 فَسِيرَتِي مُنْذُ حَدَاثَتِي الَّتِي مِنَ الْبُدَاءَةِ كَانَتْ بَيْنَ أُمَّتِي فِي أُورُشَلِيمَ يَعْرِفُهَا جَمِيعُ الْيَهُودِ
\par 5 عَالِمِينَ بِي مِنَ الأَوَّلِ - إِنْ أَرَادُوا أَنْ يَشْهَدُوا - أَنِّي حَسَبَ مَذْهَبِ عِبَادَتِنَا الأَضْيَقِ عِشْتُ فَرِّيسِيّاً.
\par 6 وَالآنَ أَنَا وَاقِفٌ أُحَاكَمُ عَلَى رَجَاءِ الْوَعْدِ الَّذِي صَارَ مِنَ اللهِ لِآبَائِنَا
\par 7 الَّذِي أَسْبَاطُنَا الاِثْنَا عَشَرَ يَرْجُونَ نَوَالَهُ عَابِدِينَ بِالْجَهْدِ لَيْلاً وَنَهَاراً. فَمِنْ أَجْلِ هَذَا الرَّجَاءِ أَنَا أُحَاكَمُ مِنَ الْيَهُودِ أَيُّهَا الْمَلِكُ أَغْرِيبَاسُ.
\par 8 لِمَاذَا يُعَدُّ عِنْدَكُمْ أَمْراً لاَ يُصَدَّقُ إِنْ أَقَامَ اللهُ أَمْوَاتاً؟
\par 9 فَأَنَا ارْتَأَيْتُ فِي نَفْسِي أَنَّهُ يَنْبَغِي أَنْ أَصْنَعَ أُمُوراً كَثِيرَةً مُضَادَّةً لاِسْمِ يَسُوعَ النَّاصِرِيِّ.
\par 10 وَفَعَلْتُ ذَلِكَ أَيْضاً فِي أُورُشَلِيمَ فَحَبَسْتُ فِي سُجُونٍ كَثِيرِينَ مِنَ الْقِدِّيسِينَ آخِذاً السُّلْطَانَ مِنْ قِبَلِ رُؤَسَاءِ الْكَهَنَةِ. وَلَمَّا كَانُوا يُقْتَلُونَ أَلْقَيْتُ قُرْعَةً بِذَلِكَ.
\par 11 وَفِي كُلِّ الْمَجَامِعِ كُنْتُ أُعَاقِبُهُمْ مِرَاراً كَثِيرَةً وَأَضْطَرُّهُمْ إِلَى التَّجْدِيفِ. وَإِذْ أَفْرَطَ حَنَقِي عَلَيْهِمْ كُنْتُ أَطْرُدُهُمْ إِلَى الْمُدُنِ الَّتِي فِي الْخَارِجِ.
\par 12 «وَلَمَّا كُنْتُ ذَاهِباً فِي ذَلِكَ إِلَى دِمَشْقَ بِسُلْطَانٍ وَوَصِيَّةٍ مِنْ رُؤَسَاءِ الْكَهَنَةِ
\par 13 رَأَيْتُ فِي نِصْفِ النَّهَارِ فِي الطَّرِيقِ أَيُّهَا الْمَلِكُ نُوراً مِنَ السَّمَاءِ أَفْضَلَ مِنْ لَمَعَانِ الشَّمْسِ قَدْ أَبْرَقَ حَوْلِي وَحَوْلَ الذَّاهِبِينَ مَعِي.
\par 14 فَلَمَّا سَقَطْنَا جَمِيعُنَا عَلَى الأَرْضِ سَمِعْتُ صَوْتاً يُكَلِّمُنِي بِاللُّغَةِ الْعِبْرَانِيَّةِ: شَاوُلُ شَاوُلُ لِمَاذَا تَضْطَهِدُنِي؟ صَعْبٌ عَلَيْكَ أَنْ تَرْفُسَ مَنَاخِسَ
\par 15 فَقُلْتُ أَنَا: مَنْ أَنْتَ يَا سَيِّدُ؟ فَقَالَ: أَنَا يَسُوعُ الَّذِي أَنْتَ تَضْطَهِدُهُ.
\par 16 وَلَكِنْ قُمْ وَقِفْ عَلَى رِجْلَيْكَ لأَنِّي لِهَذَا ظَهَرْتُ لَكَ لأَنْتَخِبَكَ خَادِماً وَشَاهِداً بِمَا رَأَيْتَ وَبِمَا سَأَظْهَرُ لَكَ بِهِ
\par 17 مُنْقِذاً إِيَّاكَ مِنَ الشَّعْبِ وَمِنَ الْأُمَمِ الَّذِينَ أَنَا الآنَ أُرْسِلُكَ إِلَيْهِمْ
\par 18 لِتَفْتَحَ عُيُونَهُمْ كَيْ يَرْجِعُوا مِنْ ظُلُمَاتٍ إِلَى نُورٍ وَمِنْ سُلْطَانِ الشَّيْطَانِ إِلَى اللهِ حَتَّى يَنَالُوا بِالإِيمَانِ بِي غُفْرَانَ الْخَطَايَا وَنَصِيباً مَعَ الْمُقَدَّسِينَ.
\par 19 «مِنْ ثَمَّ أَيُّهَا الْمَلِكُ أَغْرِيبَاسُ لَمْ أَكُنْ مُعَانِداً لِلرُّؤْيَا السَّمَاوِيَّةِ
\par 20 بَلْ أَخْبَرْتُ أَوَّلاً الَّذِينَ فِي دِمَشْقَ وَفِي أُورُشَلِيمَ حَتَّى جَمِيعِ كُورَةِ الْيَهُودِيَّةِ ثُمَّ الْأُمَمَ أَنْ يَتُوبُوا وَيَرْجِعُوا إِلَى اللهِ عَامِلِينَ أَعْمَالاً تَلِيقُ بِالتَّوْبَةِ.
\par 21 مِنْ أَجْلِ ذَلِكَ أَمْسَكَنِي الْيَهُودُ فِي الْهَيْكَلِ وَشَرَعُوا فِي قَتْلِي.
\par 22 فَإِذْ حَصَلْتُ عَلَى مَعُونَةٍ مِنَ اللهِ بَقِيتُ إِلَى هَذَا الْيَوْمِ شَاهِداً لِلصَّغِيرِ وَالْكَبِيرِ. وَأَنَا لاَ أَقُولُ شَيْئاً غَيْرَ مَا تَكَلَّمَ الأَنْبِيَاءُ وَمُوسَى أَنَّهُ عَتِيدٌ أَنْ يَكُونَ:
\par 23 إِنْ يُؤَلَّمِ الْمَسِيحُ يَكُنْ هُوَ أَوَّلَ قِيَامَةِ الأَمْوَاتِ مُزْمِعاً أَنْ يُنَادِيَ بِنُورٍ لِلشَّعْبِ وَلِلْأُمَمِ».
\par 24 وَبَيْنَمَا هُوَ يَحْتَجُّ بِهَذَا قَالَ فَسْتُوسُ بِصَوْتٍ عَظِيمٍ: «أَنْتَ تَهْذِي يَا بُولُسُ! الْكُتُبُ الْكَثِيرَةُ تُحَوِّلُكَ إِلَى الْهَذَيَانِ».
\par 25 فَقَالَ: «لَسْتُ أَهْذِي أَيُّهَا الْعَزِيزُ فَسْتُوسُ بَلْ أَنْطِقُ بِكَلِمَاتِ الصِّدْقِ وَالصَّحْوِ.
\par 26 لأَنَّهُ مِنْ جِهَةِ هَذِهِ الْأُمُورِ عَالِمٌ الْمَلِكُ الَّذِي أُكَلِّمُهُ جِهَاراً إِذْ أَنَا لَسْتُ أُصَدِّقُ أَنْ يَخْفَى عَلَيْهِ شَيْءٌ مِنْ ذَلِكَ لأَنَّ هَذَا لَمْ يُفْعَلْ فِي زَاوِيَةٍ.
\par 27 أَتُؤْمِنُ أَيُّهَا الْمَلِكُ أَغْرِيبَاسُ بِالأَنْبِيَاءِ؟ أَنَا أَعْلَمُ أَنَّكَ تُؤْمِنُ».
\par 28 فَقَالَ أَغْرِيبَاسُ لِبُولُسَ: «بِقَلِيلٍ تُقْنِعُنِي أَنْ أَصِيرَ مَسِيحِيّاً».
\par 29 فَقَالَ بُولُسُ: «كُنْتُ أُصَلِّي إِلَى اللهِ أَنَّهُ بِقَلِيلٍ وَبِكَثِيرٍ لَيْسَ أَنْتَ فَقَطْ بَلْ أَيْضاً جَمِيعُ الَّذِينَ يَسْمَعُونَنِي الْيَوْمَ يَصِيرُونَ هَكَذَا كَمَا أَنَا مَا خَلاَ هَذِهِ الْقُيُودَ».
\par 30 فَلَمَّا قَالَ هَذَا قَامَ الْمَلِكُ وَالْوَالِي وَبَرْنِيكِي وَالْجَالِسُونَ مَعَهُمْ
\par 31 وَانْصَرَفُوا وَهُمْ يُكَلِّمُونَ بَعْضُهُمْ بَعْضاً قَائِلِينَ: «إِنَّ هَذَا الإِنْسَانَ لَيْسَ يَفْعَلُ شَيْئاً يَسْتَحِقُّ الْمَوْتَ أَوِ الْقُيُودَ».
\par 32 وَقَالَ أَغْرِيبَاسُ لِفَسْتُوسَ: «كَانَ يُمْكِنُ أَنْ يُطْلَقَ هَذَا الإِنْسَانُ لَوْ لَمْ يَكُنْ قَدْ رَفَعَ دَعْوَاهُ إِلَى قَيْصَرَ».

\chapter{27}

\par 1 فَلَمَّا اسْتَقَرَّ الرَّأْيُ أَنْ نُسَافِرَ فِي الْبَحْرِ إِلَى إِيطَالِيَا سَلَّمُوا بُولُسَ وَأَسْرَى آخَرِينَ إِلَى قَائِدِ مِئَةٍ مِنْ كَتِيبَةِ أُوغُسْطُسَ اسْمُهُ يُولِيُوسُ.
\par 2 فَصَعِدْنَا إِلَى سَفِينَةٍ أَدْرَامِيتِينِيَّةٍ وَأَقْلَعْنَا مُزْمِعِينَ أَنْ نُسَافِرَ مَارِّينَ بِالْمَوَاضِعِ الَّتِي فِي أَسِيَّا. وَكَانَ مَعَنَا أَرِسْتَرْخُسُ رَجُلٌ مَكِدُونِيٌّ مِنْ تَسَالُونِيكِي.
\par 3 وَفِي الْيَوْمِ الآخَرِ أَقْبَلْنَا إِلَى صَيْدَاءَ فَعَامَلَ يُولِيُوسُ بُولُسَ بِالرِّفْقِ وَأَذِنَ أَنْ يَذْهَبَ إِلَى أَصْدِقَائِهِ لِيَحْصُلَ عَلَى عِنَايَةٍ مِنْهُمْ.
\par 4 ثُمَّ أَقْلَعْنَا مِنْ هُنَاكَ وَسَافَرْنَا فِي الْبَحْرِ مِنْ تَحْتِ قُبْرُسَ لأَنَّ الرِّيَاحَ كَانَتْ مُضَادَّةً.
\par 5 وَبَعْدَ مَا عَبَرْنَا الْبَحْرَ الَّذِي بِجَانِبِ كِيلِيكِيَّةَ وَبَمْفِيلِيَّةَ نَزَلْنَا إِلَى مِيرَا لِيكِيَّةَ.
\par 6 فَإِذْ وَجَدَ قَائِدُ الْمِئَةِ هُنَاكَ سَفِينَةً إِسْكَنْدَرِيَّةً مُسَافِرَةً إِلَى إِيطَالِيَا أَدْخَلَنَا فِيهَا.
\par 7 وَلَمَّا كُنَّا نُسَافِرُ رُوَيْداً أَيَّاماً كَثِيرَةً وَبِالْجَهْدِ صِرْنَا بِقُرْبِ كِنِيدُسَ وَلَمْ تُمَكِّنَّا الرِّيحُ أَكْثَرَ سَافَرْنَا مِنْ تَحْتِ كِرِيتَ بِقُرْبِ سَلْمُونِي.
\par 8 وَلَمَّا تَجَاوَزْنَاهَا بِالْجَهْدِ جِئْنَا إِلَى مَكَانٍ يُقَالُ لَهُ «الْمَوَانِي الْحَسَنَةُ» الَّتِي بِقُرْبِهَا مَدِينَةُ لَسَائِيَةَ.
\par 9 وَلَمَّا مَضَى زَمَانٌ طَوِيلٌ وَصَارَ السَّفَرُ فِي الْبَحْرِ خَطِراً إِذْ كَانَ الصَّوْمُ أَيْضاً قَدْ مَضَى جَعَلَ بُولُسُ يُنْذِرُهُمْ
\par 10 قَائِلاً: «أَيُّهَا الرِّجَالُ أَنَا أَرَى أَنَّ هَذَا السَّفَرَ عَتِيدٌ أَنْ يَكُونَ بِضَرَرٍ وَخَسَارَةٍ كَثِيرَةٍ لَيْسَ لِلشَّحْنِ وَالسَّفِينَةِ فَقَطْ بَلْ لأَنْفُسِنَا أَيْضاً».
\par 11 وَلَكِنْ كَانَ قَائِدُ الْمِئَةِ يَنْقَادُ إِلَى رُبَّانِ السَّفِينَةِ وَإِلَى صَاحِبِهَا أَكْثَرَ مِمَّا إِلَى قَوْلِ بُولُسَ.
\par 12 وَلأَنَّ مَوْقِعَ الْمِينَا لَمْ يَكُنْ صَالِحاً لِلْمَشْتَى اسْتَقَرَّ رَأْيُ أَكْثَرِهِمْ أَنْ يُقْلِعُوا مِنْ هُنَاكَ أَيْضاً عَسَى أَنْ يُمْكِنَهُمُ الإِقْبَالُ إِلَى فِينِكْسَ لِيَشْتُوا فِيهَا. وَهِيَ مِينَا فِي كِرِيتَ تَنْظُرُ نَحْوَ الْجَنُوبِ وَالشَّمَالِ الْغَرْبِيَّيْنِ.
\par 13 فَلَمَّا نَسَّمَتْ رِيحٌ جَنُوبٌ ظَنُّوا أَنَّهُمْ قَدْ مَلَكُوا مَقْصَدَهُمْ فَرَفَعُوا الْمِرْسَاةَ وَطَفِقُوا يَتَجَاوَزُونَ كِرِيتَ عَلَى أَكْثَرِ قُرْبٍ.
\par 14 وَلَكِنْ بَعْدَ قَلِيلٍ هَاجَتْ عَلَيْهَا رِيحٌ زَوْبَعِيَّةٌ يُقَالُ لَهَا «أُورُوكْلِيدُونُ».
\par 15 فَلَمَّا خُطِفَتِ السَّفِينَةُ وَلَمْ يُمْكِنْهَا أَنْ تُقَابِلَ الرِّيحَ سَلَّمْنَا فَصِرْنَا نُحْمَلُ.
\par 16 فَجَرَيْنَا تَحْتَ جَزِيرَةٍ يُقَالُ لَهَا «كَلَوْدِي» وَبِالْجَهْدِ قَدِرْنَا أَنْ نَمْلِكَ الْقَارِبَ.
\par 17 وَلَمَّا رَفَعُوهُ طَفِقُوا يَسْتَعْمِلُونَ مَعُونَاتٍ حَازِمِينَ السَّفِينَةَ وَإِذْ كَانُوا خَائِفِينَ أَنْ يَقَعُوا فِي السِّيرْتِسِ أَنْزَلُوا الْقُلُوعَ وَهَكَذَا كَانُوا يُحْمَلُونَ.
\par 18 وَإِذْ كُنَّا فِي نَوْءٍ عَنِيفٍ جَعَلُوا يُفَرِّغُونَ فِي الْغَدِ.
\par 19 وَفِي الْيَوْمِ الثَّالِثِ رَمَيْنَا بِأَيْدِينَا أَثَاثَ السَّفِينَةِ.
\par 20 وَإِذْ لَمْ تَكُنِ الشَّمْسُ وَلاَ النُّجُومُ تَظْهَرُ أَيَّاماً كَثِيرَةً وَاشْتَدَّ عَلَيْنَا نَوْءٌ لَيْسَ بِقَلِيلٍ انْتُزِعَ أَخِيراً كُلُّ رَجَاءٍ فِي نَجَاتِنَا.
\par 21 فَلَمَّا حَصَلَ صَوْمٌ كَثِيرٌ حِينَئِذٍ وَقَفَ بُولُسُ فِي وَسَطِهِمْ وَقَالَ: «كَانَ يَنْبَغِي أَيُّهَا الرِّجَالُ أَنْ تُذْعِنُوا لِي وَلاَ تُقْلِعُوا مِنْ كِرِيتَ فَتَسْلَمُوا مِنْ هَذَا الضَّرَرِ وَالْخَسَارَةِ.
\par 22 وَالآنَ أُنْذِرُكُمْ أَنْ تُسَرُّوا لأَنَّهُ لاَ تَكُونُ خَسَارَةُ نَفْسٍ وَاحِدَةٍ مِنْكُمْ إِلاَّ السَّفِينَةَ.
\par 23 لأَنَّهُ وَقَفَ بِي هَذِهِ اللَّيْلَةَ مَلاَكُ الإِلَهِ الَّذِي أَنَا لَهُ وَالَّذِي أَعْبُدُهُ
\par 24 قَائِلاً: لاَ تَخَفْ يَا بُولُسُ. يَنْبَغِي لَكَ أَنْ تَقِفَ أَمَامَ قَيْصَرَ. وَهُوَذَا قَدْ وَهَبَكَ اللهُ جَمِيعَ الْمُسَافِرِينَ مَعَكَ.
\par 25 لِذَلِكَ سُرُّوا أَيُّهَا الرِّجَالُ لأَنِّي أُومِنُ بِاللَّهِ أَنَّهُ يَكُونُ هَكَذَا كَمَا قِيلَ لِي.
\par 26 وَلَكِنْ لاَ بُدَّ أَنْ نَقَعَ عَلَى جَزِيرَةٍ».
\par 27 فَلَمَّا كَانَتِ اللَّيْلَةُ الرَّابِعَةُ عَشْرَةُ وَنَحْنُ نُحْمَلُ تَائِهِينَ فِي بَحْرِ أَدْرِيَا ظَنَّ النُّوتِيَّةُ نَحْوَ نِصْفِ اللَّيْلِ أَنَّهُمُ اقْتَرَبُوا إِلَى بَرٍّ.
\par 28 فَقَاسُوا وَوَجَدُوا عِشْرِينَ قَامَةً. وَلَمَّا مَضَوْا قَلِيلاً قَاسُوا أَيْضاً فَوَجَدُوا خَمْسَ عَشْرَةَ قَامَةً.
\par 29 وَإِذْ كَانُوا يَخَافُونَ أَنْ يَقَعُوا عَلَى مَوَاضِعَ صَعْبَةٍ رَمَوْا مِنَ الْمُؤَخَّرِ أَرْبَعَ مَرَاسٍ وَكَانُوا يَطْلُبُونَ أَنْ يَصِيرَ النَّهَارُ.
\par 30 وَلَمَّا كَانَ النُّوتِيَّةُ يَطْلُبُونَ أَنْ يَهْرُبُوا مِنَ السَّفِينَةِ وَأَنْزَلُوا الْقَارِبَ إِلَى الْبَحْرِ بِعِلَّةِ أَنَّهُمْ مُزْمِعُونَ أَنْ يَمُدُّوا مَرَاسِيَ مِنَ الْمُقَدَّمِ
\par 31 قَالَ بُولُسُ لِقَائِدِ الْمِئَةِ وَالْعَسْكَرِ: «إِنْ لَمْ يَبْقَ هَؤُلاَءِ فِي السَّفِينَةِ فَأَنْتُمْ لاَ تَقْدِرُونَ أَنْ تَنْجُوا».
\par 32 حِينَئِذٍ قَطَعَ الْعَسْكَرُ حِبَالَ الْقَارِبِ وَتَرَكُوهُ يَسْقُطُ.
\par 33 وَحَتَّى قَارَبَ أَنْ يَصِيرَ النَّهَارُ كَانَ بُولُسُ يَطْلُبُ إِلَى الْجَمِيعِ أَنْ يَتَنَاوَلُوا طَعَاماً قَائِلاً: «هَذَا هُوَ الْيَوْمُ الرَّابِعُ عَشَرَ وَأَنْتُمْ مُنْتَظِرُونَ لاَ تَزَالُونَ صَائِمِينَ وَلَمْ تَأْخُذُوا شَيْئاً.
\par 34 لِذَلِكَ أَلْتَمِسُ مِنْكُمْ أَنْ تَتَنَاوَلُوا طَعَاماً لأَنَّ هَذَا يَكُونُ مُفِيداً لِنَجَاتِكُمْ لأَنَّهُ لاَ تَسْقُطُ شَعْرَةٌ مِنْ رَأْسِ وَاحِدٍ مِنْكُمْ».
\par 35 وَلَمَّا قَالَ هَذَا أَخَذَ خُبْزاً وَشَكَرَ اللهَ أَمَامَ الْجَمِيعِ وَكَسَّرَ وَابْتَدَأَ يَأْكُلُ.
\par 36 فَصَارَ الْجَمِيعُ مَسْرُورِينَ وَأَخَذُوا هُمْ أَيْضاً طَعَاماً.
\par 37 وَكُنَّا فِي السَّفِينَةِ جَمِيعُ الأَنْفُسِ مِئَتَيْنِ وَسِتَّةً وَسَبْعِينَ.
\par 38 وَلَمَّا شَبِعُوا مِنَ الطَّعَامِ طَفِقُوا يُخَفِّفُونَ السَّفِينَةَ طَارِحِينَ الْحِنْطَةَ فِي الْبَحْرِ.
\par 39 وَلَمَّا صَارَ النَّهَارُ لَمْ يَكُونُوا يَعْرِفُونَ الأَرْضَ وَلَكِنَّهُمْ أَبْصَرُوا خَلِيجاً لَهُ شَاطِئٌ فَأَجْمَعُوا أَنْ يَدْفَعُوا إِلَيْهِ السَّفِينَةَ إِنْ أَمْكَنَهُمْ.
\par 40 فَلَمَّا نَزَعُوا الْمَرَاسِيَ تَارِكِينَ إِيَّاهَا فِي الْبَحْرِ وَحَلُّوا رُبُطَ الدَّفَّةِ أَيْضاً رَفَعُوا قِلْعاً لِلرِّيحِ الْهَابَّةِ وَأَقْبَلُوا إِلَى الشَّاطِئِ.
\par 41 وَإِذْ وَقَعُوا عَلَى مَوْضِعٍ بَيْنَ بَحْرَيْنِ شَطَّطُوا السَّفِينَةَ فَارْتَكَزَ الْمُقَدَّمُ وَلَبِثَ لاَ يَتَحَرَّكُ. وَأَمَّا الْمؤَخَّرُ فَكَانَ يَنْحَلُّ مِنْ عُنْفِ الأَمْوَاجِ.
\par 42 فَكَانَ رَأْيُ الْعَسْكَرِ أَنْ يَقْتُلُوا الأَسْرَى لِئَلاَّ يَسْبَحَ أَحَدٌ مِنْهُمْ فَيَهْرُبَ.
\par 43 وَلَكِنَّ قَائِدَ الْمِئَةِ إِذْ كَانَ يُرِيدُ أَنْ يُخَلِّصَ بُولُسَ مَنَعَهُمْ مِنْ هَذَا الرَّأْيِ وَأَمَرَ أَنَّ الْقَادِرِينَ عَلَى السِّبَاحَةِ يَرْمُونَ أَنْفُسَهُمْ أَوَّلاً فَيَخْرُجُونَ إِلَى الْبَرِّ
\par 44 وَالْبَاقِينَ بَعْضُهُمْ عَلَى أَلْوَاحٍ وَبَعْضُهُمْ عَلَى قِطَعٍ مِنَ السَّفِينَةِ. فَهَكَذَا حَدَثَ أَنَّ الْجَمِيعَ نَجَوْا إِلَى الْبَرِّ.

\chapter{28}

\par 1 وَلَمَّا نَجَوْا وَجَدُوا أَنَّ الْجَزِيرَةَ تُدْعَى مَلِيطَةَ.
\par 2 فَقَدَّمَ أَهْلُهَا الْبَرَابِرَةُ لَنَا إِحْسَاناً غَيْرَ الْمُعْتَادِ لأَنَّهُمْ أَوْقَدُوا نَاراً وَقَبِلُوا جَمِيعَنَا مِنْ أَجْلِ الْمَطَرِ الَّذِي أَصَابَنَا وَمِنْ أَجْلِ الْبَرْدِ.
\par 3 فَجَمَعَ بُولُسُ كَثِيراً مِنَ الْقُضْبَانِ وَوَضَعَهَا عَلَى النَّارِ فَخَرَجَتْ مِنَ الْحَرَارَةِ أَفْعَى وَنَشِبَتْ فِي يَدِهِ.
\par 4 فَلَمَّا رَأَى الْبَرَابِرَةُ الْوَحْشَ مُعَلَّقاً بِيَدِهِ قَالَ بَعْضُهُمْ لِبَعْضٍ: «لاَ بُدَّ أَنَّ هَذَا الإِنْسَانَ قَاتِلٌ لَمْ يَدَعْهُ الْعَدْلُ يَحْيَا وَلَوْ نَجَا مِنَ الْبَحْرِ».
\par 5 فَنَفَضَ هُوَ الْوَحْشَ إِلَى النَّارِ وَلَمْ يَتَضَرَّرْ بِشَيْءٍ رَدِيءٍ.
\par 6 وَأَمَّا هُمْ فَكَانُوا يَنْتَظِرُونَ أَنَّهُ عَتِيدٌ أَنْ يَنْتَفِخَ أَوْ يَسْقُطَ بَغْتَةً مَيْتاً. فَإِذِ انْتَظَرُوا كَثِيراً وَرَأَوْا أَنَّهُ لَمْ يَعْرِضْ لَهُ شَيْءٌ مُضِرٌّ تَغَيَّرُوا وَقَالُوا: «هُوَ إِلَهٌ!».
\par 7 وَكَانَ فِي مَا حَوْلَ ذَلِكَ الْمَوْضِعِ ضِيَاعٌ لِمُقَدَّمِ الْجَزِيرَةِ الَّذِي اسْمُهُ بُوبْلِيُوسُ. فَهَذَا قَبِلَنَا وَأَضَافَنَا بِمُلاَطَفَةٍ ثَلاَثَةَ أَيَّامٍ.
\par 8 فَحَدَثَ أَنَّ أَبَا بُوبْلِيُوسَ كَانَ مُضْطَجِعاً مُعْتَرًى بِحُمَّى وَسَحْجٍ. فَدَخَلَ إِلَيْهِ بُولُسُ وَصَلَّى وَوَضَعَ يَدَيْهِ عَلَيْهِ فَشَفَاهُ.
\par 9 فَلَمَّا صَارَ هَذَا كَانَ الْبَاقُونَ الَّذِينَ بِهِمْ أَمْرَاضٌ فِي الْجَزِيرَةِ يَأْتُونَ وَيُشْفَوْنَ.
\par 10 فَأَكْرَمَنَا هَؤُلاَءِ إِكْرَامَاتٍ كَثِيرَةً. وَلَمَّا أَقْلَعْنَا زَوَّدُونَا بِمَا يُحْتَاجُ إِلَيْهِ.
\par 11 وَبَعْدَ ثَلاَثَةِ أَشْهُرٍ أَقْلَعْنَا فِي سَفِينَةٍ إِسْكَنْدَرِيَّةٍ مَوْسُومَةٍ بِعَلاَمَةِ الْجَوْزَاءِ كَانَتْ قَدْ شَتَتْ فِي الْجَزِيرَةِ.
\par 12 فَنَزَلْنَا إِلَى سِيرَاكُوسَ وَمَكَثْنَا ثَلاَثَةَ أَيَّامٍ.
\par 13 ثُمَّ مِنْ هُنَاكَ دُرْنَا وَأَقْبَلْنَا إِلَى رِيغِيُونَ. وَبَعْدَ يَوْمٍ وَاحِدٍ حَدَثَتْ رِيحٌ جَنُوبٌ فَجِئْنَا فِي الْيَوْمِ الثَّانِي إِلَى بُوطِيُولِي
\par 14 حَيْثُ وَجَدْنَا إِخْوَةً فَطَلَبُوا إِلَيْنَا أَنْ نَمْكُثَ عِنْدَهُمْ سَبْعَةَ أَيَّامٍ. وَهَكَذَا أَتَيْنَا إِلَى رُومِيَةَ.
\par 15 وَمِنْ هُنَاكَ لَمَّا سَمِعَ الإِخْوَةُ بِخَبَرِنَا خَرَجُوا لاِسْتِقْبَالِنَا إِلَى فُورُنِ أَبِّيُوسَ وَالثَّلاَثَةِ الْحَوَانِيتِ. فَلَمَّا رَآهُمْ بُولُسُ شَكَرَ اللهَ وَتَشَجَّعَ.
\par 16 وَلَمَّا أَتَيْنَا إِلَى رُومِيَةَ سَلَّمَ قَائِدُ الْمِئَةِ الأَسْرَى إِلَى رَئِيسِ الْمُعَسْكَرِ وَأَمَّا بُولُسُ فَأُذِنَ لَهُ أَنْ يُقِيمَ وَحْدَهُ مَعَ الْعَسْكَرِيِّ الَّذِي كَانَ يَحْرُسُهُ.
\par 17 وَبَعْدَ ثَلاَثَةِ أَيَّامٍ اسْتَدْعَى بُولُسُ الَّذِينَ كَانُوا وُجُوهَ الْيَهُودِ. فَلَمَّا اجْتَمَعُوا قَالَ لَهُمْ: «أَيُّهَا الرِّجَالُ الإِخْوَةُ مَعَ أَنِّي لَمْ أَفْعَلْ شَيْئاً ضِدَّ الشَّعْبِ أَوْ عَوَائِدِ الآبَاءِ أُسْلِمْتُ مُقَيَّداً مِنْ أُورُشَلِيمَ إِلَى أَيْدِي الرُّومَانِ
\par 18 الَّذِينَ لَمَّا فَحَصُوا كَانُوا يُرِيدُونَ أَنْ يُطْلِقُونِي لأَنَّهُ لَمْ تَكُنْ فِيَّ عِلَّةٌ وَاحِدَةٌ لِلْمَوْتِ.
\par 19 وَلَكِنْ لَمَّا قَاوَمَ الْيَهُودُ اضْطُرِرْتُ أَنْ أَرْفَعَ دَعْوَايَ إِلَى قَيْصَرَ - لَيْسَ كَأَنَّ لِي شَيْئاً لأَشْتَكِيَ بِهِ عَلَى أُمَّتِي.
\par 20 فَلِهَذَا السَّبَبِ طَلَبْتُكُمْ لأَرَاكُمْ وَأُكَلِّمَكُمْ لأَنِّي مِنْ أَجْلِ رَجَاءِ إِسْرَائِيلَ مُوثَقٌ بِهَذِهِ السِّلْسِلَةِ».
\par 21 فَقَالُوا لَهُ: «نَحْنُ لَمْ نَقْبَلْ كِتَابَاتٍ فِيكَ مِنَ الْيَهُودِيَّةِ وَلاَ أَحَدٌ مِنَ الإِخْوَةِ جَاءَ فَأَخْبَرَنَا أَوْ تَكَلَّمَ عَنْكَ بِشَيْءٍ رَدِيٍّ.
\par 22 وَلَكِنَّنَا نَسْتَحْسِنُ أَنْ نَسْمَعَ مِنْكَ مَاذَا تَرَى لأَنَّهُ مَعْلُومٌ عِنْدَنَا مِنْ جِهَةِ هَذَا الْمَذْهَبِ أَنَّهُ يُقَاوَمُ فِي كُلِّ مَكَانٍ».
\par 23 فَعَيَّنُوا لَهُ يَوْماً فَجَاءَ إِلَيْهِ كَثِيرُونَ إِلَى الْمَنْزِلِ فَطَفِقَ يَشْرَحُ لَهُمْ شَاهِداً بِمَلَكُوتِ اللهِ وَمُقْنِعاً إِيَّاهُمْ مِنْ نَامُوسِ مُوسَى وَالأَنْبِيَاءِ بِأَمْرِ يَسُوعَ مِنَ الصَّبَاحِ إِلَى الْمَسَاءِ.
\par 24 فَاقْتَنَعَ بَعْضُهُمْ بِمَا قِيلَ وَبَعْضُهُمْ لَمْ يُؤْمِنُوا.
\par 25 فَانْصَرَفُوا وَهُمْ غَيْرُ مُتَّفِقِينَ بَعْضُهُمْ مَعَ بَعْضٍ لَمَّا قَالَ بُولُسُ كَلِمَةً وَاحِدَةً: «إِنَّهُ حَسَناً كَلَّمَ الرُّوحُ الْقُدُسُ آبَاءَنَا بِإِشَعْيَاءَ النَّبِيِّ
\par 26 قَائِلاً: اذْهَبْ إِلَى هَذَا الشَّعْبِ وَقُلْ: سَتَسْمَعُونَ سَمْعاً وَلاَ تَفْهَمُونَ وَسَتَنْظُرُونَ نَظَراً وَلاَ تُبْصِرُونَ.
\par 27 لأَنَّ قَلْبَ هَذَا الشَّعْبِ قَدْ غَلُظَ وَبِآذَانِهِمْ سَمِعُوا ثَقِيلاً وَأَعْيُنُهُمْ أَغْمَضُوهَا. لِئَلاَّ يُبْصِرُوا بِأَعْيُنِهِمْ وَيَسْمَعُوا بِآذَانِهِمْ وَيَفْهَمُوا بِقُلُوبِهِمْ وَيَرْجِعُوا فَأَشْفِيَهُمْ.
\par 28 فَلْيَكُنْ مَعْلُوماً عِنْدَكُمْ أَنَّ خَلاَصَ اللهِ قَدْ أُرْسِلَ إِلَى الْأُمَمِ وَهُمْ سَيَسْمَعُونَ».
\par 29 وَلَمَّا قَالَ هَذَا مَضَى الْيَهُودُ وَلَهُمْ مُبَاحَثَةٌ كَثِيرَةٌ فِيمَا بَيْنَهُمْ.
\par 30 وَأَقَامَ بُولُسُ سَنَتَينِ كَامِلَتَينِ فِي بَيْتٍ اسْتَأْجَرَهُ لِنَفْسِهِ. وَكَانَ يَقْبَلُ جَمِيعَ الَّذِينَ يَدْخُلُونَ إِلَيْهِ
\par 31 كَارِزاً بِمَلَكُوتِ اللهِ وَمُعَلِّماً بِأَمْرِ الرَّبِّ يَسُوعَ الْمَسِيحِ بِكُلِّ مُجَاهَرَةٍ بِلاَ مَانِعٍ.


\end{document}