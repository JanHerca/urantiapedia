\begin{document}

\title{سفر المكابيين الثالث}


\chapter{1}

\par 1 ولما علم فيلوباتور من العائدين أن المناطق التي كان يسيطر عليها قد استولى عليها أنطيوخس، أصدر أوامره إلى جميع قواته، من المشاة والفرسان، وأخذ معه أخته أرسينوي، وسار إلى المنطقة القريبة من رافيا، حيث كان أنصار أنطيوخس معسكرين.
\par 2 لكن رجلاً يُدعى ثيودوتوس، مصممًا على تنفيذ المؤامرة التي دبرها، أخذ معه أفضل الأسلحة البطلمية التي سُلّمت إليه سابقًا، وعبر ليلًا إلى خيمة بطليموس، عازمًا على قتله بمفرده وبالتالي إنهاء الحرب
\par 3 لكن دوسيثيوس، المعروف بابن دريميلوس، وهو يهودي المولد غيّر دينه لاحقًا وارتد عن التقاليد الأجدادية، كان قد اقتاد الملك بعيدًا ورتب أن ينام رجل تافه في الخيمة؛ وهكذا اتضح أن هذا الرجل جلب الانتقام المقصود للملك
\par 4 عندما انتهى قتال مرير، واتجهت الأمور لصالح أنطيوخس، ذهبت أرسينوي إلى الجنود وهي تبكي وتبكي، وشعرها أشعث، وحثتهم على الدفاع عن أنفسهم وأطفالهم وزوجاتهم بشجاعة، ووعدتهم بمنح كل منهم منينتين من الذهب إذا فازوا في المعركة
\par 5 وهكذا هُزم العدو في المعركة، وأُسر العديد من الأسرى أيضًا
\par 6 بعد أن أحبط المؤامرة، قرر بطليموس زيارة المدن المجاورة وتشجيعها
\par 7 ومن خلال القيام بذلك، ومن خلال منح حرماتهم المقدسة بالهدايا، عزز معنويات رعيته
\par 8 بما أن اليهود أرسلوا بعضًا من مجلسهم وشيوخهم للترحيب به، وتقديم هدايا الترحيب له، وتهنئته بما حدث، فقد كان أكثر حرصًا على زيارتهم في أقرب وقت ممكن
\par 9 بعد وصوله إلى أورشليم، قدّم ذبيحة للإله الأعظم وقدم ذبائح شكر وفعل ما يليق بالمكان المقدس. ثم، عند دخوله المكان وانبهاره بجودته وجماله،
\par 10 تعجب من حسن ترتيب الهيكل، وشعر برغبة في دخول قدس الأقداس
\par 11 عندما قالوا إن هذا غير مسموح به، لأنه لا يُسمح حتى لأفراد أمتهم بالدخول، ولا حتى لجميع الكهنة، بل فقط لرئيس الكهنة الذي كان متسلطًا على الجميع، وكان يدخل مرة واحدة فقط في السنة، لم يقتنع الملك بأي حال من الأحوال
\par 12 حتى بعد أن قُرئت عليه الشريعة، لم يكف عن الإصرار على أنه يجب عليه الدخول، قائلاً: "حتى لو حُرم هؤلاء الرجال من هذا الشرف، فلا ينبغي لي ذلك."
\par 13 وسأل لماذا، عندما دخل كل معبد آخر، لم يوقفه أحد هناك
\par 14 وقال أحدهم بتهور إنه من الخطأ اعتبار هذا علامة في حد ذاته
\par 15 قال الملك: "ولكن بما أن هذا قد حدث، فلماذا لا أدخل على الأقل، سواء رغبوا في ذلك أم لا؟"
\par 16 ثم سجد الكهنة بكل ثيابهم وتوسلوا إلى الإله الأعظم أن يساعد في الوضع الحاضر وأن يمنع عنف هذا المخطط الشرير، وملأوا الهيكل بالصراخ والدموع.
\par 17 واضطرب من بقي في المدينة وسارعوا إلى الخروج، ظانين أن شيئًا غامضًا يحدث
\par 18 هرعت العذارى اللواتي كنّ محبوسات في حجراتهن مع أمهاتهن، ونثرن الغبار على شعورهن، وملأن الشوارع بالآهات والنحيب
\par 19 تركت النساء اللواتي تم تجهيزهن مؤخرًا للزواج غرف الزفاف المجهزة للزواج، وأهملن الحياء اللائق، وتدفقن معًا في المدينة في اندفاع فوضوي
\par 20 تخلت الأمهات والممرضات حتى عن الأطفال حديثي الولادة هنا وهناك، بعضهم في المنازل وبعضهم في الشوارع، ودون أن ينظروا إلى الوراء، تجمعوا معًا في أعلى معبد
\par 21 تنوعت دعاءات المجتمعين هناك بسبب ما كان الملك يدبره ببذاءة
\par 22 بالإضافة إلى ذلك، فإن أكثر المواطنين جرأة لن يتسامحوا مع استكمال خططه أو تحقيق غرضه المقصود
\par 23 صرخوا على رفاقهم ليحملوا السلاح ويموتوا بشجاعة من أجل شريعة الأجداد، وأحدثوا اضطرابًا كبيرًا في المكان المقدس؛ ولأنهم بالكاد تمكنوا من السيطرة عليهم من قبل الشيوخ والشيوخ، لجأوا إلى نفس وضعية التوسّل التي اتخذها الآخرون
\par 24 في هذه الأثناء، كان الحشد، كما في السابق، منشغلاً بالصلاة،
\par 25 بينما حاول الشيوخ القريبون من الملك بطرق مختلفة أن يغيروا تفكيره المتغطرس عن الخطة التي خطط لها.
\par 26 لكنه، في غطرسته، لم يكترث بشيء، وبدأ الآن في الاقتراب، مصممًا على إنهاء الخطة المذكورة أعلاه
\par 27 عندما لاحظ من حوله ذلك، التفتوا، مع شعبنا، ليتوسلوا إلى من لديه كل القوة للدفاع عنهم في هذه المحنة الحالية، وألا يتغاضى عن هذا الفعل غير القانوني والمتغطرس
\par 28 أدى صراخ الحشود المستمر والقوي والمتضافر إلى ضجة هائلة؛
\par 29 لأنه بدا أن الصدى لم يكن للرجال فقط، بل للجدران والأرض كلها المحيطة، لأن الجميع في ذلك الوقت فضلوا الموت على تدنيس المكان

\chapter{2}

\par 1 ثم وقف رئيس الكهنة سمعان مواجهًا للهيكل، راكعًا على ركبتيه ومدّ يديه بوقار وهدوء، وصلى هكذا:
\par 2 «يا رب، يا رب، ملك السماوات، وسيد الخليقة كلها، القدوس بين القديسين، الحاكم الوحيد، القادر على كل شيء، انتبه لنا نحن الذين نعاني بشدة من رجل غير تقيّ وفاسق، مغرور في جرأته وسلطانه.»
\par 3 «لأنك أنت، خالق كل شيء وحاكم كل شيء، حاكم عادل، وأنت تدين من فعلوا أي شيء بوقاحة وغرور.»
\par 4 "لقد دمرت أولئك الذين ارتكبوا الظلم في الماضي، ومن بينهم حتى العمالقة الذين وثقوا بقوتهم وجرأتهم، الذين دمرتهم بإحضار طوفان لا حدود له عليهم."
\par 5 «أحرقتَ بالنار والكبريت رجال سدوم المتكبرين، المعروفين برذائلهم، وجعلتَهم عبرةً لمن يأتي بعدهم».
\par 6 «لقد أظهرتَ قوتك العظيمة بإنزال عقوباتٍ كثيرة ومتنوعة على الفرعون الجريء الذي استعبد شعبك المقدس إسرائيل.»
\par 7 «وعندما طاردهم بمركبات وحشد من الجنود، غمرته في أعماق البحر، لكنك أخرجت بسلام أولئك الذين وضعوا ثقتهم فيك، أنت سيد الخليقة كلها.»
\par 8 «ولما رأوا أعمال يديك، سبحوك أيها القادر على كل شيء.»
\par 9 «أنت أيها الملك، عندما خلقت الأرض التي لا حدود لها ولا قياس، اخترت هذه المدينة وقدست هذا المكان لاسمك، مع أنك لست بحاجة إلى أي شيء؛ وعندما مجدتها بظهورك العظيم، جعلتها أساسًا متينًا لمجد اسمك العظيم والمكرم.»
\par 10 «ولأنك تحب بيت إسرائيل، فقد وعدت أنه إذا حدثت لنا انتكاسات أو أصابتنا ضيقات، فستسمع دعائنا عندما نأتي إلى هذا المكان ونصلي.»
\par 11 «وإنك لأمين وصادق.»
\par 12 "ومن أجل أنك في كثير من الأحيان عندما كان آباؤنا مظلومين كنت تساعدهم في إذلالهم، وتنقذهم من شرور عظيمة،"
\par 13 «انظر الآن، أيها الملك القدوس، أنه بسبب خطايانا الكثيرة والعظيمة، فإننا نُسحق بالمعاناة، ونخضع لأعدائنا، ويغلب علينا العجز.»
\par 14 «في سقوطنا، يتعهد هذا الرجل الجريء والمدنس بانتهاك المكان المقدس على الأرض المخصص لاسمك المجيد.»
\par 15 «لأن مسكنك، سماء السموات، لا يُدنى منه إنسان.»
\par 16 «ولكن لأنك أنعمت بمجدك على شعبك إسرائيل، فقد قدستَ هذا المكان.»
\par 17 «لا تعاقبنا على نجاسة هؤلاء الرجال، ولا تحاسبنا على هذا التدنيس، لئلا يفتخر المذنبون بغضبهم أو يفرحوا بغطرسة ألسنتهم قائلين،»
\par 18 «لقد داسنا بيت المقدس كما تُداس البيوت المهينة.»
\par 19 «امسح خطايانا وشتت أخطاءنا، وأظهر رحمتك في هذه الساعة.»
\par 20 «فلتُدركنا رحمتك سريعًا، واجعل التسبيح في أفواه المحزونين ومنكسري الروح، وامنحنا السلام.»
\par 21 عندئذٍ، سمع الله، الذي يُشرف على كل شيء، وهو الآب الأول للجميع، والقدوس بين القديسين، الدعاء الشرعي، فجلد من رفع نفسه بوقاحة ووقاحة
\par 22 فهزه من هنا ومن هناك كما تهز الريح القصبة، حتى إنه كان ملقى على الأرض عاجزاً، وبالإضافة إلى أنه كان مشلولاً في أطرافه، لم يكن قادراً حتى على التكلم، لأنه أصيب بحكم عادل.
\par 23 ثم رأى الأصدقاء والحراس الشخصيون العقاب الشديد الذي حل به، وخافوا أن يفقد حياته، فسحبوه بسرعة إلى الخارج، وقد أصيبوا بالذعر من خوفهم الشديد
\par 24 بعد فترة، تعافى، ورغم أنه عوقب، إلا أنه لم يتب أبدًا، بل ذهب وهو يطلق تهديدات مريرة
\par 25 ولما وصل إلى مصر، ازداد في أفعاله الخبيثة، بتحريض من رفاقه ورفاقه في الشرب المذكورين سابقًا، والذين كانوا غرباء عن كل شيء عادل
\par 26 لم يكن راضيًا عن أفعاله الفاسقة التي لا تُحصى، بل استمر أيضًا بجرأة كبيرة لدرجة أنه لفّق تقارير شريرة في مختلف المناطق؛ والعديد من أصدقائه، الذين كانوا يراقبون غرض الملك باهتمام، اتبعوا إرادته أيضًا
\par 27 اقترح إلحاق العار العام بالمجتمع اليهودي، ونصب حجرًا على البرج في الفناء مكتوبًا عليه:
\par 28 «لا يجوز لأي شخص لا يقدم تضحيات أن يدخل مقدساته، ويخضع جميع اليهود لتسجيل يتضمن ضريبة رأس المال ولوضع العبيد. أما من يعترض على ذلك، فيؤخذ بالقوة ويُقتل؛»
\par 29 «يجب أيضًا وسم أجساد المسجلين بالنار برمز ورقة اللبلاب لديونيسوس، ويجب أيضًا إعادتهم إلى وضعهم المحدود السابق.»
\par 30 لكي لا يبدو عدوًا للجميع، كتب أدناه: "ولكن إذا فضل أي منهم الانضمام إلى أولئك الذين بدأوا في الأسرار، فسيكون لهم مواطنة متساوية مع الإسكندريين."
\par 31 ومع ذلك، فقد استسلم البعض، مع اشمئزازهم الواضح من الثمن الذي سيُطلب للحفاظ على دين مدينتهم، لأنهم توقعوا تعزيز سمعتهم من خلال ارتباطهم المستقبلي بالملك
\par 32 لكن الأغلبية تصرفت بحزم وبروح شجاعة ولم تحيد عن دينها؛ وبدفع المال مقابل الحياة حاولوا بثقة إنقاذ أنفسهم من التسجيل
\par 33 ظلوا متمسكين بأمل الحصول على المساعدة، وكرهوا أولئك الذين انفصلوا عنهم، واعتبروهم أعداءً للأمة اليهودية، وحرموهم من الرفقة المشتركة والمساعدة المتبادلة

\chapter{3}

\par 1 عندما أدرك الملك الفاجر هذا الوضع، غضب بشدة لدرجة أنه لم يقتصر على غضبه على اليهود الذين يعيشون في الإسكندرية، بل ازداد عداؤه تجاه أولئك الذين يعيشون في الريف؛ وأمر بجمع الجميع على الفور في مكان واحد، وإعدامهم بأقسى الوسائل
\par 2 وبينما كانت هذه الأمور تُرتَّب، انتشرت شائعة معادية ضد الأمة اليهودية من قِبَل رجال تآمروا لإيذائهم، وكانت الذريعة هي تقرير مفاده أنهم يمنعون الآخرين من مراعاة عاداتهم.
\par 3 ومع ذلك، استمر اليهود في الحفاظ على حسن النية والولاء الثابت تجاه السلالة؛
\par 4 ولكن لأنهم كانوا يعبدون الله ويسلكون وفقًا لشريعته، فقد حافظوا على انفصالهم فيما يتعلق بالأطعمة. لهذا السبب بدوا مكروهين للبعض؛
\par 5 ولكن بما أنهم زيّنوا أسلوب حياتهم بالأعمال الصالحة التي يقوم بها الصالحون، فقد اكتسبوا سمعة طيبة بين جميع الناس
\par 6 ومع ذلك، لم يُعرِ أبناء الأعراق الأخرى اهتمامًا لخدمتهم الجيدة لأمتهم، وهو أمر كان شائعًا بين الجميع؛
\par 7 بدلًا من ذلك، كانوا يثرثرون حول الاختلافات في العبادة والأطعمة، زاعمين أن هؤلاء الناس لم يكونوا موالين للملك ولا لسلطاته، بل كانوا معادين ومعارضين بشدة لحكومته. لذلك لم يوجهوا إليهم أي لوم عادي
\par 8 على الرغم من أن اليونانيين في المدينة لم يُظلموا بأي شكل من الأشكال، إلا أنهم عندما رأوا ضجة غير متوقعة حول هؤلاء الناس والحشود التي كانت تتشكل فجأة، لم يكونوا أقوياء بما يكفي لمساعدتهم، لأنهم كانوا يعيشون تحت الاستبداد. لقد حاولوا مواساتهم، إذ كانوا حزينين على الوضع، وتوقعوا أن الأمور ستتغير؛
\par 9 لأنه لا ينبغي ترك مثل هذا المجتمع العظيم لمصيره عندما لم يرتكب أي جريمة
\par 10 وبالفعل، أخذ بعض جيرانهم وأصدقائهم وشركائهم التجاريين بعضهم جانبًا على انفراد، وتعهدوا بحمايتهم وبذل المزيد من الجهود الجادة لمساعدتهم
\par 11 ثم تفاخر الملك بحظه السعيد الحالي، ولم يضع في اعتباره قوة الله الأعظم، بل افترض أنه سيواصل غرضه نفسه باستمرار، فكتب هذه الرسالة ضدهم:
\par 12 «من الملك بطليموس فيلوباتور إلى قواده وجنوده في مصر وجميع مقاطعاتها، تحياتي وتمنياتي لكم بالصحة والعافية.»
\par 13 «أنا وحكومتنا بخير.»
\par 14 "عندما انطلقت حملتنا إلى آسيا، كما تعلمون أنفسكم، فقد انتهت، وفقًا للخطة، بفضل تحالف الآلهة المتعمد معنا في المعركة،"
\par 15 «ورأينا أنه لا ينبغي لنا أن نحكم الأمم التي تسكن سورية الجوفاء وفينيقيا بقوة الرمح، بل ينبغي أن نعتني بها برحمة وإحسان كبير، ونعاملها بكل سرور معاملة حسنة.»
\par 16 «وبعد أن منحنا إيراداتٍ جزيلةً جدًا للهياكل في المدن، أتينا إلى أورشليم أيضًا، وصعدنا لنكرم هيكل أولئك الأشرار الذين لا يكفون عن جهالتهم.»
\par 17 «لقد قبلوا حضورنا بالكلام، ولكنهم لم يتقبلوا ذلك بالفعل، لأنه عندما اقترحنا دخول معبدهم الداخلي وتكريمه بقرابين رائعة وجميله،»
\par 18 "لقد انجرفوا وراء غرورهم التقليدي، ومنعونا من الدخول؛ ولكنهم نجوا من ممارسة قوتنا بسبب الإحسان الذي لدينا تجاه الجميع."
\par 19 «بإصرارهم على إظهار سوء نيتهم ​​تجاهنا، يصبحون الشعب الوحيد بين جميع الأمم الذي يرفع رؤوسه عالياً في تحدٍ للملوك ومحسنيهم، وغير مستعد لاعتبار أي عمل صادقاً.»
\par 20 «ولكننا، عندما وصلنا إلى مصر منتصرين، تكيفنا مع حماقتهم وفعلنا ما هو لائق، لأننا نعامل جميع الأمم بإحسان.»
\par 21 «من بين أمور أخرى، أعلنا للجميع عفونا عن مواطنيهم هنا، سواء بسبب تحالفهم معنا أو بسبب الشؤون العديدة التي عُهد بها إليهم بسخاء منذ البداية؛ وتجرأنا على إجراء تغيير، من خلال اتخاذ قرار باعتبارهم جديرين بالمواطنة السكندرية وجعلهم مشاركين في طقوسنا الدينية المعتادة.»
\par 22 «لكنهم في خبثهم الفطري أخذوا هذا بروح معاكسة، واحتقروا الخير. لأنهم يميلون باستمرار إلى الشر،»
\par 23 «إنهم لا يرفضون المواطنة التي لا تقدر بثمن فحسب، بل يكرهون أيضًا، بالكلام والصمت، أولئك القلائل منهم الذين يميلون إلينا بصدق؛ في كل موقف، ووفقًا لأسلوب حياتهم سيئ السمعة، يشتبهون سرًا في أننا قد نغير سياستنا قريبًا.»
\par 24 «لذلك، واقتناعًا تامًا من هذه المؤشرات على أنهم سيئو التصرف تجاهنا بكل الطرق، فقد اتخذنا احتياطاتنا خشية أن نجعل هؤلاء الأشرار خلف ظهورنا، إذا ما نشأ اضطراب مفاجئ ضدنا لاحقًا، خونة وأعداء همجيين.»
\par 25 «لذلك أصدرنا أوامرنا بأنه بمجرد وصول هذه الرسالة، يجب عليكم إرسال من يعيشون بينكم، مع زوجاتهم وأطفالهم، إلى بلادنا، ليُعاملوا معاملة مهينة وقاسية، ومقيدين بإحكام بأغلال من حديد، ليعانوا الموت المؤكد والمخزي الذي يليق بالأعداء.»
\par 26 «لأنه عندما يُعاقب هؤلاء جميعًا، فنحن على يقين من أن الحكومة ستُقام لنا في الوقت المتبقي في حالة جيدة وفي أفضل حال.»
\par 27 «ولكن من آوى أحدًا من اليهود، شيخًا كان أو طفلًا أو حتى رُضّعًا، فإنه يُعذب حتى الموت بأبشع العذابات، مع أهله.»
\par 28 «أي شخص يرغب في تقديم معلومات سيحصل على ممتلكات الشخص الذي يتحمل العقوبة، بالإضافة إلى ألفي دراخما من الخزانة الملكية، وسيُمنح حريته.»
\par 29 «كل مكان يُكتشف أنه يؤوي يهوديًا يجب أن يُجعل منيعًا ويُحرق بالنار، ويصبح عديم الفائدة إلى الأبد لأي مخلوق بشري.»
\par 30 كُتبت الرسالة بالشكل المذكور أعلاه.

\chapter{4}

\par 1 ففي كل مكان وصل إليه هذا المرسوم، تم ترتيب وليمة على نفقة عامة للأمم مع هتافات وفرح، لأن العداوة المتأصلة التي كانت في أذهانهم منذ زمن طويل أصبحت الآن واضحة وصريحة
\par 2 لكن بين اليهود كان هناك حزن ونحيب وصراخ دامع لا ينقطع؛ كانت قلوبهم تحترق في كل مكان، وكانوا يتأوهون بسبب الدمار غير المتوقع الذي قُدِّر لهم فجأة
\par 3 أي حي أو مدينة، أو أي مكان صالح للسكن على الإطلاق، أو أي شوارع لم تكن مليئة بالحزن والنحيب عليهم؟
\par 4 لأنه بمثل هذه الروح القاسية والقاسية، تم طردهم جميعًا معًا من قبل الجنرالات في المدن المختلفة، لدرجة أنه عند رؤية عقوباتهم غير العادية، حتى أن بعض أعدائهم، الذين أدركوا هدف الشفقة المشترك أمام أعينهم، تأملوا في عدم يقين الحياة وذرفوا الدموع على أبشع طرد لهؤلاء الناس
\par 5 لأنه كان يتم اقتياد حشد من الرجال المسنين ذوي الرؤوس الرمادية، البطيئين والمنحنين بسبب التقدم في السن، مجبرين على السير بخطى سريعة بسبب العنف الذي كانوا يُساقون به بهذه الطريقة المخزية
\par 6 واستبدلت الشابات اللواتي دخلن لتوه حجرة العرس للمشاركة في الحياة الزوجية الفرح بالبكاء، وشعرهن المعطر بالمر مرشوش بالرماد، وحملن بعيدًا سافرات، ورفعن جميعًا رثاءً بدلًا من أغنية زفاف، إذ تمزقهن المعاملة القاسية للوثنيين
\par 7 تم جرهم بعنف إلى مكان الصعود، مقيدين ومعروضين على الملأ
\par 8 كان أزواجهن، في ريعان شبابهم، محاطين بالحبال بدلاً من الأكاليل، يقضون الأيام المتبقية من مهرجان زواجهم في الرثاء بدلاً من البهجة والمرح الشبابي، ويرون الموت أمامهم مباشرة
\par 9 تم جلبهم على متن السفن كحيوانات برية، مدفوعين تحت قيود حديدية؛ تم ربط بعضهم من أعناقهم بمقاعد القوارب، بينما تم تأمين أقدام آخرين بقيود غير قابلة للكسر،
\par 10 بالإضافة إلى ذلك، تم حبسهم تحت سطح سفينة مصمت، بحيث تكون أعينهم في ظلام دامس، ويتلقون معاملة تليق بالخونة طوال الرحلة
\par 11 عندما تم إحضار هؤلاء الرجال إلى المكان المسمى شيديا، وانتهت الرحلة كما قرر الملك، أمر باحتجازهم في ميدان سباق الخيل الذي تم بناؤه بسور محيطي ضخم أمام المدينة، والذي كان مناسبًا تمامًا لجعلهم مشهدًا واضحًا لجميع العائدين إلى المدينة ولأولئك القادمين من المدينة إلى الريف، بحيث لا يمكنهم التواصل مع قوات الملك ولا الادعاء بأي شكل من الأشكال بأنهم داخل محيط المدينة
\par 12 ولما حدث هذا، سمع الملك أن أبناء وطن اليهود من المدينة كانوا يخرجون سرًا في كثير من الأحيان ليندبوا بمرارة سوء حظ إخوانهم الحقير،
\par 13 أمر في غضبه بأن يتم التعامل مع هؤلاء الرجال بنفس الطريقة تمامًا مثل الآخرين، دون إغفال أي تفاصيل عن عقوبتهم
\par 14 وكان من المقرر تسجيل السباق بأكمله بشكل فردي، وليس للأعمال الشاقة التي تم ذكرها بإيجاز من قبل، ولكن للتعذيب بالفظائع التي أمر بها، وفي النهاية ليتم تدميره في غضون يوم واحد.
\par 15 لذلك، تم تسجيل هؤلاء الأشخاص بسرعة مريرة وعزم متحمس من شروق الشمس إلى غروبها، وعلى الرغم من عدم اكتماله، فقد توقف بعد أربعين يومًا
\par 16 كان الملك يمتلئ فرحًا شديدًا ومستمرًا، ويقيم ولائم تكريمًا لجميع أصنامه، بعقل منفصل عن الحقيقة وفمٍ دنيء، يمتدح أشياءً صامتة لا تستطيع حتى التواصل أو مساعدة المرء، وينطق بكلمات غير لائقة ضد الله الأعظم
\par 17 ولكن بعد انقضاء الفترة الزمنية المذكورة سابقًا، أعلن الكتبة للملك أنهم لم يعودوا قادرين على إحصاء اليهود بسبب كثرة عددهم،
\par 18 على الرغم من أن معظمهم كانوا لا يزالون في البلاد، وبعضهم لا يزال يقيم في منازلهم، وبعضهم في المكان؛ كانت المهمة مستحيلة لجميع الجنرالات في مصر
\par 19 بعد أن هددهم بشدة، متهمًا إياهم بالرشوة للتوصل إلى وسيلة للهروب، كان مقتنعًا تمامًا بالأمر
\par 20 عندما قالوا وأثبتوا أن الورق والأقلام التي استخدموها للكتابة قد نفدت بالفعل
\par 21 لكن هذا كان فعلًا من أفعال العناية الإلهية التي لا تُقهر لمن كان يُعين اليهود من السماء

\chapter{5}

\par 1 ثم امتلأ الملك غضبًا وغيظًا شديدين، فاستدعى حرمون، حارس الفيلة،
\par 2 وأمره في اليوم التالي بتخدير جميع الفيلة - وعددها خمسمائة - بحفنات كبيرة من اللبان والكثير من النبيذ غير المخلوط، وقيادتها، وقد أصابها الجنون من وفرة الخمور، حتى يلاقي اليهود حتفهم
\par 3 بعد أن أصدر هذه الأوامر، عاد إلى وليمة، مع أصدقائه وأفراد الجيش الذين كانوا معادين لليهود بشكل خاص
\par 4 وشرع حرمون، حارس الفيلة، في تنفيذ الأوامر بإخلاص
\par 5 خرج الخدم المسؤولون عن اليهود في المساء وقيدوا أيدي الشعب البائس ورتبوا لاستمرار احتجازهم طوال الليل، مقتنعين بأن الأمة بأكملها ستشهد دمارها النهائي
\par 6 لأنه بالنسبة للأمم، بدا أن اليهود تُركوا دون أي مساعدة،
\par 7 لأنهم كانوا مقيدين بالقوة من كل جانب. ولكن بدموع وصوت يصعب إسكاته، دعوا جميعًا الرب القدير وحاكم كل قوة، إلههم وأبوهم الرحيم، مصلين
\par 8 أنه ينتقم من المؤامرة الشريرة ضدهم، وينقذهم في تجلي مجيد من المصير المُعد لهم الآن
\par 9 فصعدت توسلاتهم بحرارة إلى السماء.
\par 10 ولكن حرمون، بعد أن خدر الفيلة القاسية حتى امتلأت بكمية كبيرة من النبيذ وشبعت بالبخور، حضر إلى الفناء في الصباح الباكر ليخبر الملك عن هذه الاستعدادات.
\par 11 لكن الرب أرسل على الملك نصيبًا من النوم، تلك النعمة التي يُمنحها من البداية، ليلًا ونهارًا، من يمنحها لمن يشاء
\par 12 وبفضل عمل الرب، غلب عليه نوم عميق وممتع لدرجة أنه فشل تمامًا في تحقيق هدفه الخارج عن القانون، وأحبط تمامًا في خطته الجامدة
\par 13 ثم إن اليهود، إذ هربوا من الساعة المعينة، سبّحوا إلههم القدوس، وتوسلوا أيضًا إلى الذي يُصالح بسهولة أن يُظهر قدرة يده القوية للأمم المتغطرسين
\par 14 ولكن الآن، وبما أن الوقت كان يقترب من منتصف الساعة العاشرة، فإن الشخص المسؤول عن الدعوات، عندما رأى أن الضيوف قد اجتمعوا، اقترب من الملك ودفعه
\par 15 وبعد أن أيقظه بصعوبة، أشار إلى أن ساعة الوليمة قد اقتربت بالفعل، وقدم له وصفًا للموقف
\par 16 بعد أن فكر الملك في هذا، عاد إلى شربه، وأمر الحاضرين في المأدبة بالاتكاء مقابله
\par 17 عندما انتهى من ذلك، حثهم على الانغماس في المرح والاحتفال، وجعل الجزء الحالي من المأدبة مبهجًا من خلال الاحتفال أكثر فأكثر
\par 18 بعد أن استمر الحفل لبعض الوقت، استدعى الملك حرمون، وطالبه بتهديدات حادة بمعرفة سبب السماح لليهود بالبقاء على قيد الحياة حتى يومنا هذا
\par 19 ولكن عندما أشار، بتأييد من أصدقائه، إلى أنه بينما كان الليل لا يزال يسدل ستاره، نفذ الأمر الصادر له بالكامل،
\par 20 قال الملك، الذي كان ممسوسًا بوحشية أسوأ من وحشية فالاريس، إن اليهود قد استفادوا من نوم اليوم، "لكن"، أضاف، "غدًا دون تأخير، جهزوا الأفيال بنفس الطريقة لتدمير اليهود الخارجين عن القانون!"
\par 21 ولما تكلم الملك، وافق جميع الحاضرين بفرح وسرور، وانصرف كل واحد إلى منزله
\par 22 لكنهم لم يستغلوا مدة الليل في النوم بقدر ما استغلوها في ابتكار كل أنواع الإهانات لأولئك الذين ظنوا أنهم محكوم عليهم بالهلاك
\par 23 ثم، بمجرد أن صاح الديك في الصباح الباكر، قام حرمون بتجهيز الحيوانات، وبدأ في تحريكها في الرواق الكبير
\par 24 كانت حشود المدينة قد تجمعت لمشاهدة هذا المشهد المثير للشفقة، وكانوا ينتظرون بفارغ الصبر بزوغ الفجر
\par 25 لكن اليهود، في أنفاسهم الأخيرة، وبما أن الوقت قد انتهى، مدوا أيديهم نحو السماء، وتوسلوا إلى الله الأعظم بدعاء دامع ومرثيات حزينة أن يساعدهم مرة أخرى على الفور
\par 26 ولم تكن أشعة الشمس قد أشرقت بعد، وبينما كان الملك يستقبل أصدقاءه، وصل حرمون ودعاه للخروج، مشيراً إلى أن ما يطلبه الملك أصبح جاهزاً للعمل.
\par 27 لكنه، عند استلامه التقرير وتأثره بالدعوة غير العادية للخروج - لأنه كان غارقًا تمامًا في عدم الفهم - سأل عن السبب الذي من أجله تم إنجاز هذا العمل بحماس من أجله
\par 28 كان هذا فعل الله الذي يحكم كل شيء، لأنه غرس في عقل الملك نسيانًا للأشياء التي ابتكرها سابقًا
\par 29 ثم أشار حرمون وجميع أصدقاء الملك إلى أن الوحوش والقوات المسلحة جاهزة، "أيها الملك، حسب مشيئتك الحارة."
\par 30 ولكن عند هذه الكلمات امتلأ غضبًا عارمًا، لأنه بفضل عناية الله كان عقله كله قد اختل فيما يتعلق بهذه الأمور؛ وبنظرة تهديد قال:
\par 31 «لو كان والداكم أو أبناؤكم حاضرين، لكنت أعددتهم ليكونوا وليمة دسمة للوحوش المتوحشة بدلاً من اليهود، الذين لا يمنحونني أي سبب للشكوى، وقد أظهروا إلى درجة غير عادية ولاءً كاملاً وثابتًا لأسلافي.»
\par 32 «في الواقع، كنت ستُحرم من الحياة بدلًا من هذه، لولا المودة التي تنشأ من تربيتنا المشتركة وفائدتك.»
\par 33 فتعرض حرمون لتهديد غير متوقع وخطير، فارتعشت عيناه وسقط وجهه
\par 34 انسحب أصدقاء الملك واحدًا تلو الآخر على مضض، وصرفوا الناس المجتمعين، كلٌّ إلى عمله
\par 35 فلما سمع اليهود ما قاله الملك، مدحوا الرب الإله الظاهر، ملك الملوك، لأن هذا أيضًا كان عونه الذي نالوه
\par 36 ومع ذلك، أعاد الملك دعوة الحفل بنفس الطريقة وحث الضيوف على العودة إلى احتفالاتهم
\par 37 بعد استدعاء حرمون، قال بنبرة تهديد: "كم مرة، أيها المسكين، يجب أن أعطيك أوامر بشأن هذه الأمور؟"
\par 38 «جهزوا الأفيال الآن مرة أخرى لتدمير اليهود غدًا!»
\par 39 لكن المسؤولين الذين كانوا على المائدة معه، تعجبوا من عدم استقراره العقلي، واحتجوا على النحو التالي:
\par 40 «أيها الملك، إلى متى ستُجرِّبنا كما لو كنا أغبياء، فتأمر الآن للمرة الثالثة بإبادتهم، وتُلغي مجددًا مرسومك في هذا الأمر؟»
\par 41 «ونتيجة لذلك، تعيش المدينة حالة من الاضطراب بسبب توقعاتها؛ فهي مكتظة بالناس، كما أنها معرضة دائمًا لخطر النهب.»
\par 42 عند هذا، لم يُعر الملك، وهو فالاريس في كل شيء ومليء بالجنون، أي اهتمام للتغيرات التي طرأت على رأيه بشأن حماية اليهود، وأقسم يمينًا لا رجعة فيه أنه سيرسلهم إلى الموت دون تأخير، مشوهين من ركب وأقدام الوحوش،
\par 43 "وسوف يزحف أيضًا على يهودا ويسويها بالأرض بسرعة بالنار والرماح، وبحرق الهيكل الذي كان من غير الممكن الوصول إليه، فإنه سيجعله سريعًا فارغًا إلى الأبد من أولئك الذين قدموا الذبائح هناك.
\par 44 ثم غادر الأصدقاء والضباط بفرح عظيم، ونشروا القوات المسلحة بثقة في الأماكن الأكثر ملاءمة للحراسة في المدينة
\par 45 بعد أن أُصيبت الحيوانات بالجنون تقريبًا، إن صح التعبير، بسبب جرعات النبيذ العطري الممزوج باللبان، وتم تزويدها بأدوات مخيفة، قام حارس الفيلة
\par 46 دخل عند الفجر تقريبًا إلى الفناء - حيث كانت المدينة الآن مليئة بحشود لا حصر لها من الناس يتزاحمون في طريقهم إلى ميدان سباق الخيل - وحث الملك على معالجة الأمر المطروح
\par 47 لذلك، عندما ملأ عقله غير التقي غضبًا عميقًا، اندفع بكامل قوته مع الوحوش، راغبًا في أن يشهد، بقلب لا يقهر وبعينيه، الدمار المؤلم والمؤسف للشعب المذكور
\par 48 ولما رأى اليهود الغبار الذي أثارته الفيلة الخارجة من الباب والقوات المسلحة التي تلتها، وكذلك وطء الجموع، وسمعوا الضجيج العالي والهائج،
\par 49 ظنوا أن هذه هي آخر لحظة في حياتهم، نهاية أكثر لحظات ترقبهم بؤسًا، وفسحوا المجال للرثاء والتأوه، فقبلوا بعضهم البعض، واحتضنوا الأقارب، وسقطوا في أحضان بعضهم البعض - الآباء والأطفال، والأمهات والبنات، وآخرون يحملون أطفالًا رضعًا على صدورهم، ويسحبون آخر حليب لهم
\par 50 ليس هذا فحسب، بل عندما فكروا في المساعدة التي تلقوها سابقًا من السماء، سجدوا على الأرض بنفس واحدة، وأخرجوا الأطفال من صدورهم،
\par 51 وصرخوا بصوت عالٍ جدًا، متوسلين إلى الحاكم على كل قوة أن يُظهر نفسه ويرحمهم، وهم واقفون الآن على أبواب الموت

\chapter{6}

\par 1 ثم أمر رجل يُدعى أليعازار، وكان مشهورًا بين كهنة البلاد، وكان قد بلغ شيخوخةً عظيمة، وكان طوال حياته مزينًا بكل فضيلة، الشيوخ من حوله بالتوقف عن الدعاء إلى الله القدوس، وصلى على النحو التالي:
\par 2 «ملك ذو قوة عظيمة، الله القدير العلي، يحكم كل الخليقة بالرحمة،»
\par 3 «انظر إلى أحفاد إبراهيم، يا أبتاه، إلى أبناء يعقوب القديس، شعب من نصيبك المقدس يهلكون كغرباء في أرض غريبة.»
\par 4 «فرعون بمركباته الكثيرة، الحاكم السابق لمصر، المتعالي بوقاحة خارجة عن القانون ولسان متبجح، أهلكته هو وجيشه المتغطرس بإغراقهم في البحر، مظهرًا نور رحمتك على أمة إسرائيل.»
\par 5 "سنحاريب، الذي يفتخر بجيوشه التي لا تعد ولا تحصى، أيها الملك الظالم للآشوريين، الذي استولى على العالم كله بالرمح، ورفع على مدينتك المقدسة، متكلمًا بكلمات مؤلمة بتباهي ووقاحة، أنت يا رب حطمت، وأظهرت قوتك لأمم كثيرة."
\par 6 «الرفاق الثلاثة في بابل الذين سلموا حياتهم طواعيةً للنيران حتى لا يخدموا أشياءً باطلة، أنقذتهم سالمين، حتى شعرة واحدة، بللت أتون النار بالندى وحولت لهيبه ضد جميع أعدائهم.»
\par 7 «دانيال، الذي طُرح إلى الأرض للأسود بسبب افتراءات حسدية ليكون طعامًا للوحوش، أصعدته إلى النور سالمًا.»
\par 8 «ويونان، وهو يذبل في بطن وحش بحري ضخم، راقبته أنت يا أبتاه، وأعدته سالمًا إلى جميع أفراد عائلته.»
\par 9 «والآن، يا من تكره الوقاحة، يا رحيمًا وحاميًا للجميع، اكشف عن نفسك سريعًا لأبناء أمة إسرائيل - الذين يُعاملون معاملة شنيعة من قِبَل الأمم البغيضة الخارجة عن القانون.»
\par 10 «حتى لو تورطت حياتنا في الفجور في منفانا، أنقذنا من يد العدو، ودمرنا يا رب، بأي مصير تختاره.»
\par 11 «لا يمدح المتكبرون غرورهم عند هلاك شعبكم الحبيب، قائلين: حتى إلههم لم ينقذهم».
\par 12 «لكنك أنت، أيها الأزلي، الذي لك كل القدرة وكل القدرة، احرسنا الآن وارحمنا نحن الذين، بسبب وقاحة الخارجين عن القانون الحمقاء، نُحرم من الحياة على غرار الخونة.»
\par 13 «وليرتعد الأمم اليوم خوفًا من قوتك التي لا تُقهر، أيها المُكرَّم، الذي لك القدرة على إنقاذ أمة يعقوب.»
\par 14 «كل حشد الأطفال وآبائهم يتوسلون إليك بالدموع.»
\par 15 «ليُعلَم جميع الأمم أنك معنا يا رب، ولم تصرف وجهك عنا. بل كما قلت: حتى حين كانوا في أرض أعدائهم لم أهملهم، هكذا تمم يا رب.»
\par 16 بينما كان أليعازار ينهي صلاته، وصل الملك إلى ميدان سباق الخيل مع الوحوش وكل غطرسة قواته
\par 17 وعندما لاحظ اليهود ذلك، أطلقوا صرخات عظيمة إلى السماء حتى أن الوديان القريبة دوّت بها، وجلبت رعبًا لا يمكن السيطرة عليه على الجيش
\par 18 ثم كشف الله المجيد والقدير والحقيقي عن وجهه القدوس وفتح الأبواب السماوية، فنزل منها ملاكان مجيدان مهيبان، كانا مرئيين للجميع ما عدا اليهود
\par 19 عارضوا قوات العدو وملأوها بالفوضى والرعب، وقيدوها بأغلال ثابتة
\par 20 حتى الملك بدأ يرتجف جسديًا، ونسي وقاحته الكئيبة
\par 21 انقلبت الوحوش على القوات المسلحة التي كانت تتبعها وبدأت تدوسها وتدمرها
\par 22 ثم تحول غضب الملك إلى شفقة وبكاء بسبب الأمور التي دبرها من قبل.
\par 23 لأنه عندما سمع الصراخ ورأهم جميعًا يسقطون على رؤوسهم نحو الهلاك، بكى وهدد أصدقاءه بغضب قائلًا:
\par 24 «أنت ترتكب الخيانة وتتفوق على الطغاة في القسوة؛ وحتى أنا، وليّ أمرك، تحاول الآن حرمانك من السيادة والحياة من خلال التخطيط سرًا لأعمال لا تعود بالنفع على المملكة.»
\par 25 «من هو الذي أخذ كل رجل من منزله وجمع هنا بلا معنى أولئك الذين دافعوا بإخلاص عن حصون بلدنا؟»
\par 26 «من ذا الذي حاصر بمثل هذه المعاملة الفظيعة أولئك الذين اختلفوا منذ البداية عن جميع الأمم في حسن نيتهم ​​تجاهنا، والذين غالبًا ما قبلوا طواعية أسوأ المخاطر البشرية؟»
\par 27 «أحلوا قيودهم الظالمة! وأعيدوهم إلى ديارهم بسلام، متوسلين بالصفح عن أفعالكم السابقة!»
\par 28 «أطلق سراح أبناء الإله القدير الحي في السماء، الذي منح حكومتنا، منذ زمن أجدادنا وحتى الآن، استقرارًا ملحوظًا ودون عوائق.»
\par 29 هذه كانت الأشياء التي قالها، فأطلق سراح اليهود على الفور، وسبحوا إلههم القدوس ومخلصهم، لأنهم الآن قد نجوا من الموت
\par 30 ثم عاد الملك إلى المدينة، واستدعى المسؤول عن الإيرادات وأمره أن يوفر لليهود الخمر وكل ما يحتاجون إليه لعيد مدته سبعة أيام، وقرر أن يحتفلوا بإنقاذهم بكل فرح في نفس المكان الذي توقعوا أن يلقوا فيه هلاكهم
\par 31 وبناءً على ذلك، قام أولئك الذين عوملوا معاملة مهينة وكانوا على وشك الموت، أو بالأحرى، الذين وقفوا على أبوابه، بترتيب وليمة خلاص بدلاً من موت مرير ومؤسف، وبفرح كبير وزعوا على المحتفلين المكان الذي أُعدّ لهلاكهم ودفنهم
\par 32 توقفوا عن ترديد المراثي، واستأنفوا ترنيمة آبائهم، مُسبّحين الله، مُخلّصهم وصانع العجائب. وأنهوا كل حزن وعويل، وشكّلوا جوقات كعلامة على الفرح السلمي
\par 33 وبالمثل، بعد أن أقام الملك وليمة عظيمة للاحتفال بهذه الأحداث، شكر السماء بلا انقطاع وبسخاء على الإنقاذ غير المتوقع الذي شهده
\par 34 وأولئك الذين كانوا يعتقدون سابقًا أن اليهود سيهلكون ويصبحون طعامًا للطيور، وسجلوا ذلك بفرح، تأوهوا لأنهم غلب عليهم العار، وأخمدت جرأتهم في إطلاق النار بشكل مخزٍ
\par 35 أما اليهود، فلما رتبوا المجموعة الكورالية المذكورة آنفًا، كما قلنا سابقًا، فقد أمضوا الوقت في الاحتفال على أنغام الشكر البهيج والمزامير
\par 36 "ولما رسموا لهذه الأمور طقسًا عامًا في مجتمعهم كله وفي نسلهم، أسسوا الاحتفال بالأيام المذكورة كعيد، ليس للشرب والشراهة، بل بسبب النجاة التي أتتهم من خلال الله.
\par 37 ثم تقدموا بالتماس إلى الملك، طالبين إعادتهم إلى ديارهم.
\par 38 فكان إحصاءهم من الخامس والعشرين من شهر باخون إلى الرابع من شهر أبيف أربعين يوما، وكان هلاكهم محددا من الخامس إلى السابع من شهر أبيف ثلاثة أيام.
\par 39 الذي فيه أظهر رب الجميع رحمته المجيدة وأنقذهم جميعًا معًا سالمين
\par 40 ثم تناولوا وليمة، وقد زودهم الملك بكل شيء، حتى اليوم الرابع عشر، وفيه أيضًا قدموا التماسًا لفصلهم
\par 41 وافق الملك على طلبهم على الفور وكتب لهم الرسالة التالية إلى القادة في المدن، معربًا بسخاء عن قلقه:

\chapter{7}

\par 1 «من الملك بطليموس فيلوباتور إلى القادة في مصر وجميع من هم في السلطة في حكومته، تحياتي وتمنياتي لكم بالصحة والعافية.»
\par 2 «نحن وأطفالنا بخير، والله العظيم يوجه شؤوننا وفقًا لرغباتنا.»
\par 3 «كان بعض أصدقائنا، الذين يلحّون علينا مرارًا وتكرارًا بنية خبيثة، يقنعوننا بجمع يهود المملكة في هيئة ومعاقبتهم بعقوبات وحشية كخونة؛»
\par 4 «لأنهم أعلنوا أن حكومتنا لن تتأسس أبدًا حتى يتم إنجاز ذلك، بسبب سوء النية الذي يكنه هؤلاء الناس تجاه جميع الدول.»
\par 5 «كما اقتادوهم بمعاملة قاسية كعبيد، أو بالأحرى كخونة، وتسلحوا بقسوة أكثر وحشية من تلك التي اعتاد عليها السكيثيون، وحاولوا دون أي تحقيق أو فحص قتلهم.»
\par 6 «لكننا هددناهم بشدة على هذه الأفعال، ووفقًا للرأفة التي لدينا تجاه جميع البشر، بالكاد نجونا بحياتهم. منذ أن أدركنا أن إله السماء يدافع بالتأكيد عن اليهود، ويأخذ دورهم دائمًا كما يفعل الأب مع أبنائه،»
\par 7 «وبما أننا أخذنا في الاعتبار حسن النية الودود والراسخ الذي كان لديهم تجاهنا وتجاه أسلافنا، فقد برّأناهم بحق من كل تهمة مهما كان نوعها.»
\par 8 «وأمرنا كل واحد بالعودة إلى داره، وألا يؤذيه أحد في أي مكان، أو يوبخه على ما حدث من أمور غير منطقية.»
\par 9 «لأنه يجب أن تعلموا أنه إذا دبرنا أي شر ضدهم أو تسببنا لهم في أي حزن على الإطلاق، فلن يكون لدينا دائمًا إنسان، بل الحاكم على كل سلطة، الله العلي، في كل شيء، ولا مفر منه كخصم للانتقام من مثل هذه الأفعال. وداعًا.»
\par 10 ولما تلقوا هذه الرسالة لم يسارعوا إلى الرحيل فورًا، بل طلبوا من الملك أن يعاقب على أيديهم أولئك من الأمة اليهودية الذين خالفوا عمدًا الله القدوس وشريعة الله بالعقوبة التي يستحقونها.
\par 11 لأنهم أعلنوا أن أولئك الذين خالفوا الوصايا الإلهية من أجل البطن لن يكونوا أبدًا مؤيدين لحكومة الملك
\par 12 ثم أقر الملك بصحة ما قالوه ووافق عليه، ومنحهم ترخيصًا عامًا حتى يتمكنوا بحرية ودون سلطة أو إشراف ملكي من تدمير أولئك الذين انتهكوا شريعة الله في كل مكان في مملكته
\par 13 وبعد أن صفقوا له كما ينبغي، هتف كهنتهم والجمع كله بالهللويا وانصرفوا بفرح
\par 14 وهكذا في طريقهم عاقبوا وأعدموا علنًا ومخزيًا أي شخص يقابلونه من مواطنيهم الذين أصبحوا مدنسين
\par 15 في ذلك اليوم قتلوا أكثر من ثلاثمائة رجل، وأقاموا ذلك اليوم عيدًا بهيجًا، لأنهم أهلكوا المدنسين
\par 16 أما أولئك الذين تمسكوا بالله حتى الموت، ونالوا التمتع الكامل بالخلاص، فقد بدأوا رحيلهم من المدينة، متوجين بجميع أنواع الزهور العطرة، شاكرين بفرح وبصوت عالٍ إله آبائهم الوحيد، مخلص إسرائيل الأبدي، بكلمات التسبيح وجميع أنواع الأغاني الشجية
\par 17 عندما وصلوا إلى بطليموس، التي تُسمى "حاملة الورد" بسبب سمة من سمات المكان، انتظرهم الأسطول، وفقًا للرغبة المشتركة، لمدة سبعة أيام
\par 18 هناك احتفلوا بخلاصهم، لأن الملك كان قد وفر لهم بسخاء كل ما يلزم لرحلتهم، لكل واحد منهم حتى بيته
\par 19 وعندما وصلوا بسلام مع الشكر المناسب، قرروا هناك أيضًا الاحتفال بهذه الأيام كعيد بهيج خلال فترة إقامتهم
\par 20 ثم، بعد أن نقشوا على عمود أنهم مقدسون، وخصصوا مكانًا للصلاة في موقع المهرجان، غادروا سالمين، أحرارًا، وفرحين للغاية، لأنه بأمر الملك، تم نقلهم بأمان عن طريق البر والبحر والنهر، كلٌّ إلى مكانه
\par 21 كما كانوا يتمتعون بمكانة أكبر بين أعدائهم، حيث كانوا يُعاملون بتكريم ورهبة؛ ولم يكونوا عرضة على الإطلاق لمصادرة ممتلكاتهم من قبل أي شخص
\par 22 إلى جانب ذلك، استعادوا جميعًا جميع ممتلكاتهم، وفقًا للتسجيل، حتى أن أولئك الذين كانوا يملكون أيًا منها أعادوها إليهم بخوف شديد. وهكذا، قام الله الأعظم بأعمال عظيمة على أكمل وجه من أجل خلاصهم
\par 23 تبارك مخلص إسرائيل إلى الأبد! آمين.

\end{document}