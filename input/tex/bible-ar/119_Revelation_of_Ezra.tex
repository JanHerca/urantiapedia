\begin{document}

\title{رؤيا عزرا}

\chapter{1}

\par textit{الوحي الذي أُنزل على عزرا وبني إسرائيل بشأن طبيعة السنة حتى بداية شهر يناير.}

\par 1 إذا صادف اليوم الأول من شهر يناير يوم الرب، فإنه يُنتج شتاءً دافئًا، وربيعًا ماطرًا، وخريفًا عاصفًا، ومحاصيل جيدة، ووفرة من الماشية، وعسلًا كافيًا، ومحاصيل جيدة، ووفرة من الفاصوليا، وحدائق ناجحة. (لكن) سيموت الشباب، وستكون هناك معارك وسرقات كبيرة، (وسيُسمع) شيء جديد عن الملوك والحكام

\par 2 يوم القمر (الاثنين) يجعل كلاً من الشتاء (و) الصيف معتدلين. ستكون هناك فيضانات كبيرة وأمراض، وحروب مشاة، وتغييرات في الحكام، وستجلس العديد من الزوجات في رثاء، وسيكون هناك الكثير من الجليد، وسيموت الملوك، (سيكون هناك) موسم حصاد جيد، وسيموت النحل

\par 3 يوم المريخ (الثلاثاء) يُنتج شتاءً قاسيًا وكئيبًا، وربيعًا ثلجيًا، وصيفًا ممطرًا، وخريفًا جافًا. ستكون الحبوب باهظة الثمن. (سيكون هناك) هلاك للخنازير، (و) وباء مفاجئ بين الماشية. (سيكون) الإبحار خطيرًا، (و) العسل كافٍ؛ (سيكون) الكتان باهظ الثمن، (ستكون) الحرائق كثيرة، (ستكون) الفاصوليا، وخضروات الحدائق، (و) الزيت وفيرًا. ستموت النساء، (وسيموت الملوك أيضًا). (سيكون) المحصول مضطربًا

\par 4 يوم عطارد (الأربعاء): إنتاجية المحاصيل، حصاد جيد، قلة الثمار، نجاح في الأعمال، هلاك الرجال، شتاء دافئ. سيكون الخريف معتدلاً. (ستكون هناك) مخاطر من السيف، ووفرة من الزيت، وإسهال في الأمعاء. ستموت النساء، وستكون هناك مجاعة في أماكن مختلفة (و) صيف جيد. سيُسمع شيء جديد (و) لن يكون هناك عسل

\par 5 يوم المشتري (الخميس): (سيكون هناك) انعدام قيمة الحبوب؛ (سيكون) اللحم غالي الثمن (وسيكون هناك) وفرة من الفاكهة. سيكون هناك عسل أقل؛ (سيكون) الشتاء معتدلاً، والربيع عاصفًا، والخريف جيدًا، والصيف جيدًا. (سيكون هناك) هلاك للخنازير (و) أمطار غزيرة؛ ستفيض الأنهار. (و) سيكون هناك ما يكفي من الزيت، وسيفسد المحصول، وستختلط الفاصوليا، (و) سيكون هناك) سلام

\par 6 يوم الزهرة (الجمعة) يُنتج شتاءً معتدلاً، وصيفًا سيئًا، وخريفًا جافًا، وحبوبًا لا قيمة لها، ومحصولًا جيدًا، والتهابًا في العيون. سيموت الأطفال، وسيكون هناك زلزال، و(سيكون) خطر على الملوك؛ وسيكون الزيت وفيرًا، وستهلك الأغنام والنحل

\par 7 يوم زحل (السبت) يُحدث شتاءً عاصفًا، وربيعًا سيئًا، وصيفًا يتغير بسبب العواصف، وخريفًا جافًا، وندرة في الحبوب والكتان باهظ الثمن. ستنتشر الحمى، وسيُصاب الناس بأمراض مختلفة، وسيموت كبار السن


\end{document}