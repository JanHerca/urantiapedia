\begin{document}

\title{2 صموئيل}


\chapter{1}

\par 1 وَكَانَ بَعْدَ مَوْتِ شَاوُلَ وَرُجُوعِ دَاوُدَ مِنْ مُضَارَبَةِ الْعَمَالِقَةِ أَنَّ دَاوُدَ أَقَامَ فِي صِقْلَغَ يَوْمَيْنِ.
\par 2 وَفِي الْيَوْمِ الثَّالِثِ إِذَا بِرَجُلٍ أَتَى مِنَ الْمَحَلَّةِ مِنْ عِنْدِ شَاوُلَ وَثِيَابُهُ مُمَزَّقَةٌ وَعَلَى رَأْسِهِ تُرَابٌ. فَلَمَّا جَاءَ إِلَى دَاوُدَ خَرَّ إِلَى الأَرْضِ وَسَجَدَ.
\par 3 فَقَالَ لَهُ دَاوُدُ: «مِنْ أَيْنَ أَتَيْتَ؟» فَقَالَ لَهُ: «مِنْ مَحَلَّةِ إِسْرَائِيلَ نَجَوْتُ».
\par 4 فَقَالَ لَهُ دَاوُدُ: «كَيْفَ كَانَ الأَمْرُ؟ أَخْبِرْنِي». فَقَالَ: «إِنَّ الشَّعْبَ قَدْ هَرَبَ مِنَ الْقِتَالِ، وَسَقَطَ أَيْضاً كَثِيرُونَ مِنَ الشَّعْبِ وَمَاتُوا، وَمَاتَ شَاوُلُ وَيُونَاثَانُ ابْنُهُ أَيْضاً».
\par 5 فَقَالَ دَاوُدُ لِلْغُلاَمِ الَّذِي أَخْبَرَهُ: «كَيْفَ عَرَفْتَ أَنَّهُ قَدْ مَاتَ شَاوُلُ وَيُونَاثَانُ ابْنُهُ؟»
\par 6 فَقَالَ الْغُلاَمُ الَّذِي أَخْبَرَهُ: «اتَّفَقَ أَنِّي كُنْتُ فِي جَبَلِ جِلْبُوعَ وَإِذَا شَاوُلُ يَتَوَكَّأُ عَلَى رُمْحِهِ، وَإِذَا بِالْمَرْكَبَاتِ وَالْفُرْسَانِ يَشُدُّونَ وَرَاءَهُ.
\par 7 فَالْتَفَتَ إِلَى وَرَائِهِ فَرَآنِي وَدَعَانِي فَقُلْتُ: هَئَنَذَا.
\par 8 فَقَالَ لِي: مَنْ أَنْتَ؟ فَقُلْتُ لَهُ: عَمَالِيقِيٌّ أَنَا.
\par 9 فَقَالَ لِي: قِفْ عَلَيَّ وَاقْتُلْنِي لأَنَّهُ قَدِ اعْتَرَانِيَ الدُّوَارُ لأَنَّ كُلَّ نَفْسِي بَعْدُ فِيَّ.
\par 10 فَوَقَفْتُ عَلَيْهِ وَقَتَلْتُهُ لأَنِّي عَلِمْتُ أَنَّهُ لاَ يَعِيشُ بَعْدَ سُقُوطِهِ، وَأَخَذْتُ الإِكْلِيلَ الَّذِي عَلَى رَأْسِهِ وَالسِّوارَ الَّذِي عَلَى ذِرَاعِهِ وَأَتَيْتُ بِهِمَا إِلَى سَيِّدِي هَهُنَا».
\par 11 فَأَمْسَكَ دَاوُدُ ثِيَابَهُ وَمَزَّقَهَا وَكَذَا جَمِيعُ الرِّجَالِ الَّذِينَ مَعَهُ.
\par 12 وَنَدَبُوا وَبَكُوا وَصَامُوا إِلَى الْمَسَاءِ عَلَى شَاوُلَ وَعَلَى يُونَاثَانَ ابْنِهِ وَعَلَى شَعْبِ الرَّبِّ وَعَلَى بَيْتِ إِسْرَائِيلَ لأَنَّهُمْ سَقَطُوا بِالسَّيْفِ.
\par 13 ثُمَّ قَالَ دَاوُدُ لِلْغُلاَمِ الَّذِي أَخْبَرَهُ: «مِنْ أَيْنَ أَنْتَ؟» فَقَالَ: «أَنَا ابْنُ رَجُلٍ غَرِيبٍ عَمَالِيقِيٌٍّ».
\par 14 فَقَالَ لَهُ دَاوُدُ: «كَيْفَ لَمْ تَخَفْ أَنْ تَمُدَّ يَدَكَ لِتُهْلِكَ مَسِيحَ الرَّبِّ؟»
\par 15 ثُمَّ دَعَا دَاوُدُ وَاحِداً مِنَ الْغِلْمَانِ وَقَالَ: «تَقَدَّمْ. أَوْقِعْ بِهِ». فَضَرَبَهُ فَمَاتَ.
\par 16 فَقَالَ لَهُ دَاوُدُ: «دَمُكَ عَلَى رَأْسِكَ لأَنَّ فَمَكَ شَهِدَ عَلَيْكَ قَائِلاً: أَنَا قَتَلْتُ مَسِيحَ الرَّبِّ».
\par 17 وَرَثَا دَاوُدُ بِهَذِهِ الْمَرْثَاةِ شَاوُلَ وَيُونَاثَانَ ابْنَهُ،
\par 18 وَقَالَ أَنْ يَتَعَلَّمَ بَنُو يَهُوذَا «نَشِيدَ الْقَوْسِ». هُوَذَا ذَلِكَ مَكْتُوبٌ فِي «سِفْرِ يَاشَرَ»:
\par 19 «اَلظَّبْيُ يَا إِسْرَائِيلُ مَقْتُولٌ عَلَى شَوَامِخِكَ. كَيْفَ سَقَطَ الْجَبَابِرَةُ!
\par 20 لاَ تُخْبِرُوا فِي جَتَّ. لاَ تُبَشِّرُوا فِي أَسْوَاقِ أَشْقَلُونَ، لِئَلاَّ تَفْرَحَ بَنَاتُ الْفِلِسْطِينِيِّينَ، لِئَلاَّ تَشْمَتَ بَنَاتُ الْغُلْفِ.
\par 21 يَا جِبَالَ جِلْبُوعَ لاَ يَكُنْ طَلٌّ وَلاَ مَطَرٌ عَلَيْكُنَّ وَلاَ حُقُولُ تَقْدِمَاتٍ، لأَنَّهُ هُنَاكَ طُرِحَ مِجَنُّ الْجَبَابِرَةِ، مِجَنُّ شَاوُلَ بِلاَ مَسْحٍ بِالدُّهْنِ.
\par 22 مِنْ دَمِ الْقَتْلَى مِنْ شَحْمِ الْجَبَابِرَةِ لَمْ تَرْجِعْ قَوْسُ يُونَاثَانَ إِلَى الْوَرَاءِ، وَسَيْفُ شَاوُلَ لَمْ يَرْجِعْ خَائِباً.
\par 23 شَاوُلُ وَيُونَاثَانُ الْمَحْبُوبَانِ وَالْحُلْوَانِ فِي حَيَاتِهِمَا لَمْ يَفْتَرِقَا فِي مَوْتِهِمَا. أَخَفُّ مِنَ النُّسُورِ وَأَشَدُّ مِنَ الأُسُودِ.
\par 24 يَا بَنَاتِ إِسْرَائِيلَ، ابْكِينَ شَاوُلَ الَّذِي أَلْبَسَكُنَّ قِرْمِزاً بِالتَّنَعُّمِ، وَجَعَلَ حُلِيَّ الذَّهَبِ عَلَى مَلاَبِسِكُنَّ.
\par 25 كَيْفَ سَقَطَ الْجَبَابِرَةُ فِي وَسَطِ الْحَرْبِ! يُونَاثَانُ عَلَى شَوَامِخِكَ مَقْتُولٌ.
\par 26 قَدْ تَضَايَقْتُ عَلَيْكَ يَا أَخِي يُونَاثَانُ. كُنْتَ حُلْواً لِي جِدّاً. مَحَبَّتُكَ لِي أَعْجَبُ مِنْ مَحَبَّةِ النِّسَاءِ.
\par 27 كَيْفَ سَقَطَ الْجَبَابِرَةُ وَبَادَتْ آلاَتُ الْحَرْبِ».

\chapter{2}

\par 1 وَكَانَ بَعْدَ ذَلِكَ أَنَّ دَاوُدَ سَأَلَ الرَّبَّ: «أَأَصْعَدُ إِلَى إِحْدَى مَدَائِنِ يَهُوذَا؟» فَقَالَ لَهُ الرَّبُّ: «اصْعَدْ». فَقَالَ دَاوُدُ: «إِلَى أَيْنَ أَصْعَدُ؟» فَقَالَ: «إِلَى حَبْرُونَ».
\par 2 فَصَعِدَ دَاوُدُ إِلَى هُنَاكَ هُوَ وَامْرَأَتَاهُ أَخِينُوعَمُ الْيَزْرَعِيلِيَّةُ وَأَبِيجَايِلُ امْرَأَةُ نَابَالَ الْكَرْمَلِيِّ.
\par 3 وَأَصْعَدَ دَاوُدُ رِجَالَهُ الَّذِينَ مَعَهُ كُلَّ وَاحِدٍ وَبَيْتَهُ وَسَكَنُوا فِي مُدُنِ حَبْرُونَ.
\par 4 وَأَتَى رِجَالُ يَهُوذَا وَمَسَحُوا هُنَاكَ دَاوُدَ مَلِكاً عَلَى بَيْتِ يَهُوذَا. وَأَخْبَرُوا دَاوُدَ: «إِنَّ رِجَالَ يَابِيشَ جِلْعَادَ هُمُ الَّذِينَ دَفَنُوا شَاوُلَ».
\par 5 فَأَرْسَلَ دَاوُدُ رُسُلاً إِلَى أَهْلِ يَابِيشَ جِلْعَادَ يَقُولُ لَهُمْ: «مُبَارَكُونَ أَنْتُمْ مِنَ الرَّبِّ إِذْ قَدْ فَعَلْتُمْ هَذَا الْمَعْرُوفَ بِسَيِّدِكُمْ شَاوُلَ فَدَفَنْتُمُوهُ.
\par 6 وَالآنَ لِيَصْنَعِ الرَّبُّ مَعَكُمْ إِحْسَاناً وَحَقّاً، وَأَنَا أَيْضاً أَفْعَلُ مَعَكُمْ هَذَا الْخَيْرَ لأَنَّكُمْ فَعَلْتُمْ هَذَا الأَمْرَ.
\par 7 وَالآنَ فَلْتَتَشَدَّدْ أَيْدِيكُمْ وَكُونُوا ذَوِي بَأْسٍ، لأَنَّهُ قَدْ مَاتَ سَيِّدُكُمْ شَاوُلُ، وَإِيَّايَ مَسَحَ بَيْتُ يَهُوذَا مَلِكاً عَلَيْهِمْ».
\par 8 وَأَمَّا أَبْنَيْرُ بْنُ نَيْرٍ، رَئِيسُ جَيْشِ شَاوُلَ، فَأَخَذَ إِيشْبُوشَثَ بْنَ شَاوُلَ وَعَبَرَ بِهِ إِلَى مَحَنَايِمَ
\par 9 وَجَعَلَهُ مَلِكاً عَلَى جِلْعَادَ وَعَلَى الأَشُّورِيِّينَ وَعَلَى يَزْرَعِيلَ وَعَلَى أَفْرَايِمَ وَعَلَى بِنْيَامِينَ وَعَلَى كُلِّ إِسْرَائِيلَ.
\par 10 وَكَانَ إِيشْبُوشَثُ بْنُ شَاوُلَ ابْنَ أَرْبَعِينَ سَنَةً حِينَ مَلَكَ عَلَى إِسْرَائِيلَ، وَمَلَكَ سَنَتَيْنِ. وَأَمَّا بَيْتُ يَهُوذَا فَإِنَّمَا اتَّبَعُوا دَاوُدَ.
\par 11 وَكَانَتِ الْمُدَّةُ الَّتِي مَلَكَ فِيهَا دَاوُدُ فِي حَبْرُونَ عَلَى بَيْتِ يَهُوذَا سَبْعَ سِنِينَ وَسِتَّةَ أَشْهُرٍ.
\par 12 وَخَرَجَ أَبْنَيْرُ بْنُ نَيْرٍ وَعَبِيدُ إِيشْبُوشَثَ بْنِ شَاوُلَ مِنْ مَحَنَايِمَ إِلَى جِبْعُونَ.
\par 13 وَخَرَجَ يُوآبُ بْنُ صَرُويَةَ وَعَبِيدُ دَاوُدَ، فَالْتَقُوا جَمِيعاً عَلَى بِرْكَةِ جِبْعُونَ. وَجَلَسُوا هَؤُلاَءِ عَلَى الْبِرْكَةِ مِنْ هُنَا وَهَؤُلاَءِ عَلَى الْبِرْكَةِ مِنْ هُنَاكَ.
\par 14 فَقَالَ أَبْنَيْرُ لِيُوآبَ: «لِيَقُمِ الْغِلْمَانُ وَيَتَكَافَحُوا أَمَامَنَا». فَقَالَ يُوآبُ: «لِيَقُومُوا».
\par 15 فَقَامُوا وَعَبَرُوا بِالْعَدَدِ، اثْنَا عَشَرَ لأَجْلِ بِنْيَامِينَ وَإِيشْبُوشَثَ بْنِ شَاوُلَ، وَاثْنَا عَشَرَ مِنْ عَبِيدِ دَاوُدَ.
\par 16 وَأَمْسَكَ كُلُّ وَاحِدٍ بِرَأْسِ صَاحِبِهِ وَضَرَبَ سَيْفَهُ فِي جَنْبِ صَاحِبِهِ وَسَقَطُوا جَمِيعاً. فَدُعِيَ ذَلِكَ الْمَوْضِعُ «حِلْقَثَ هَصُّورِيمَ» الَّتِي هِيَ فِي جِبْعُونَ.
\par 17 وَكَانَ الْقِتَالُ شَدِيداً جِدّاً فِي ذَلِكَ الْيَوْمِ، وَانْكَسَرَ أَبْنَيْرُ وَرِجَالُ إِسْرَائِيلَ أَمَامَ عَبِيدِ دَاوُدَ.
\par 18 وَكَانَ هُنَاكَ بَنُو صَرُويَةَ الثَّلاَثَةُ: يُوآبُ، وَأَبِيشَايُ، وَعَسَائِيلُ. وَكَانَ عَسَائِيلُ خَفِيفَ الرَِّجْلَيْنِ كَظَبْيِ الْبَرِّ.
\par 19 فَسَعَى عَسَائِيلُ وَرَاءَ أَبْنَيْرَ، وَلَمْ يَمِلْ فِي السَّيْرِ يَمْنَةً وَلاَ يَسْرَةً مِنْ وَرَاءِ أَبْنَيْرَ.
\par 20 فَالْتَفَتَ أَبْنَيْرُ إِلَى وَرَائِهِ وَقَالَ: «أَأَنْتَ عَسَائِيلُ؟» فَقَالَ: «أَنَا هُوَ».
\par 21 فَقَالَ لَهُ أَبْنَيْرُ: «مِلْ إِلَى يَمِينِكَ أَوْ إِلَى يَسَارِكَ وَاقْبِضْ عَلَى أَحَدِ الْغِلْمَانِ وَخُذْ لِنَفْسِكَ سَلَبَهُ». فَلَمْ يَشَأْ عَسَائِيلُ أَنْ يَمِيلَ مِنْ وَرَائِهِ.
\par 22 ثُمَّ عَادَ أَبْنَيْرُ وَقَالَ لِعَسَائِيلَ: «مِلْ مِنْ وَرَائِي. لِمَاذَا أَضْرِبُكَ إِلَى الأَرْضِ؟ فَكَيْفَ أَرْفَعُ وَجْهِي لَدَى يُوآبَ أَخِيكَ؟»
\par 23 فَأَبَى أَنْ يَمِيلَ، فَضَرَبَهُ أَبْنَيْرُ بِطَرَفِ الرُّمْحِ فِي بَطْنِهِ، فَخَرَجَ الرُّمْحُ مِنْ خَلْفِهِ فَسَقَطَ هُنَاكَ وَمَاتَ فِي مَكَانِهِ. وَكَانَ كُلُّ مَنْ يَأْتِي إِلَى الْمَوْضِعِ الَّذِي سَقَطَ فِيهِ عَسَائِيلُ وَمَاتَ يَقِفُ.
\par 24 وَسَعَى يُوآبُ وَأَبِيشَايُ وَرَاءَ أَبْنَيْرَ، وَغَابَتِ الشَّمْسُ عِنْدَمَا أَتَيَا إِلَى تَلِّ أَمَّةَ الَّذِي تُجَاهَ جِيحَ فِي طَرِيقِ بَرِّيَّةِ جِبْعُونَ.
\par 25 فَاجْتَمَعَ بَنُو بِنْيَامِينَ وَرَاءَ أَبْنَيْرَ وَصَارُوا جَمَاعَةً وَاحِدَةً، وَوَقَفُوا عَلَى رَأْسِ تَلٍّ وَاحِدٍ.
\par 26 فَنَادَى أَبْنَيْرُ يُوآبَ: «هَلْ إِلَى الأَبَدِ يَأْكُلُ السَّيْفُ؟ أَلَمْ تَعْلَمْ أَنَّهَا تَكُونُ مَرَارَةً فِي الأَخِيرِ؟ فَحَتَّى مَتَى لاَ تَقُولُ لِلشَّعْبِ أَنْ يَرْجِعُوا مِنْ وَرَاءِ إِخْوَتِهِمْ؟»
\par 27 فَقَالَ يُوآبُ: «حَيٌّ هُوَ اللَّهُ إِنَّهُ لَوْ لَمْ تَتَكَلَّمْ لَكَانَ الشَّعْبُ فِي الصَّبَاحِ قَدْ صَعِدَ كُلُّ وَاحِدٍ مِنْ وَرَاءِ أَخِيهِ».
\par 28 وَضَرَبَ يُوآبُ بِالْبُوقِ فَوَقَفَ جَمِيعُ الشَّعْبِ وَلَمْ يَسْعُوا بَعْدُ وَرَاءَ إِسْرَائِيلَ وَلاَ عَادُوا إِلَى الْمُحَارَبَةِ.
\par 29 فَسَارَ أَبْنَيْرُ وَرِجَالُهُ فِي الْعَرَبَةِ ذَلِكَ اللَّيْلَ كُلَّهُ وَعَبَرُوا الأُرْدُنَّ، وَسَارُوا فِي كُلِّ الشُّعَبِ وَجَاءُوا إِلَى مَحَنَايِمَ.
\par 30 وَرَجَعَ يُوآبُ مِنْ وَرَاءِ أَبْنَيْرَ وَجَمَعَ كُلَّ الشَّعْبِ. وَفُقِدَ مِنْ عَبِيدِ دَاوُدَ تِسْعَةَ عَشَرَ رَجُلاً وَعَسَائِيلُ.
\par 31 وَضَرَبَ عَبِيدُ دَاوُدَ مِنْ بِنْيَامِينَ وَمِنْ رِجَالِ أَبْنَيْرَ، فَمَاتَ ثَلاَثُ مِئَةٍ وَسِتُّونَ رَجُلاً.
\par 32 وَرَفَعُوا عَسَائِيلَ وَدَفَنُوهُ فِي قَبْرِ أَبِيهِ الَّذِي فِي بَيْتِ لَحْمٍ. وَسَارَ يُوآبُ وَرِجَالُهُ اللَّيْلَ كُلَّهُ وَأَصْبَحُوا فِي حَبْرُونَ.

\chapter{3}

\par 1 وَكَانَتِ الْحَرْبُ طَوِيلَةً بَيْنَ بَيْتِ شَاوُلَ وَبَيْتِ دَاوُدَ، وَكَانَ دَاوُدُ يَذْهَبُ يَتَقَوَّى وَبَيْتُ شَاوُلَ يَذْهَبُ يَضْعُفُ.
\par 2 وَوُلِدَ لِدَاوُدَ بَنُونَ فِي حَبْرُونَ. وَكَانَ بِكْرُهُ أَمْنُونَ مِنْ أَخِينُوعَمَ الْيَزْرَعِيلِيَّةِ،
\par 3 وَثَانِيهِ كِيلآبَ مِنْ أَبِيجَايِلَ امْرَأَةِ نَابَالَ الْكَرْمَلِيِّ. وَالثَّالِثُ أَبْشَالُومَ ابْنَ مَعْكَةَ بِنْتِ تَلْمَايَ مَلِكِ جَشُورَ،
\par 4 وَالرَّابِعُ أَدُونِيَّا ابْنَ حَجِّيثَ، وَالْخَامِسُ شَفَطْيَا ابْنَ أَبِيطَالَ،
\par 5 وَالسَّادِسُ يَثْرَعَامَ مِنْ عَجْلَةَ امْرَأَةِ دَاوُدَ. هَؤُلاَءِ وُلِدُوا لِدَاوُدَ فِي حَبْرُونَ.
\par 6 وَكَانَ فِي وُقُوعِ الْحَرْبِ بَيْنَ بَيْتِ شَاوُلَ وَبَيْتِ دَاوُدَ أَنَّ أَبْنَيْرَ تَشَدَّدَ لأَجْلِ بَيْتِ شَاوُلَ.
\par 7 وَكَانَتْ لِشَاوُلَ سُرِّيَّةٌ اسْمُهَا رِصْفَةُ بِنْتُ أَيَّةَ. فَقَالَ إِيشْبُوشَثُ لأَبْنَيْرَ: «لِمَاذَا دَخَلْتَ إِلَى سُرِّيَّةِ أَبِي؟»
\par 8 فَاغْتَاظَ أَبْنَيْرُ جِدّاً مِنْ كَلاَمِ إِيشْبُوشَثَ وَقَالَ: «أَلَعَلِّي رَأْسُ كَلْبٍ لِيَهُوذَا؟ الْيَوْمَ أَصْنَعُ مَعْرُوفاً مَعَ بَيْتِ شَاوُلَ أَبِيكَ، مَعَ إِخْوَتِهِ وَمَعَ أَصْحَابِهِ، وَلَمْ أُسَلِّمْكَ لِيَدِ دَاوُدَ، وَتُطَالِبُنِي الْيَوْمَ بِإِثْمِ الْمَرْأَةِ!
\par 9 هَكَذَا يَصْنَعُ اللَّهُ بِأَبْنَيْرَ وَهَكَذَا يَزِيدُهُ إِنَّهُ كَمَا حَلَفَ الرَّبُّ لِدَاوُدَ كَذَلِكَ أَصْنَعُ لَهُ
\par 10 لِنَقْلِ الْمَمْلَكَةِ مِنْ بَيْتِ شَاوُلَ وَإِقَامَةِ كُرْسِيِّ دَاوُدَ عَلَى إِسْرَائِيلَ وَعَلَى يَهُوذَا مِنْ دَانَ إِلَى بِئْرِ سَبْعٍ».
\par 11 وَلَمْ يَقْدِرْ بَعْدُ أَنْ يُجَاوِبَ أَبْنَيْرَ بِكَلِمَةٍ لأَجْلِ خَوْفِهِ مِنْهُ.
\par 12 فَأَرْسَلَ أَبْنَيْرُ مِنْ فَوْرِهِ رُسُلاً إِلَى دَاوُدَ قَائِلاً: «لِمَنْ هِيَ الأَرْضُ؟ يَقُولُونَ: اقْطَعْ عَهْدَكَ مَعِي، وَهُوَذَا يَدِي مَعَكَ لِرَدِّ جَمِيعِ إِسْرَائِيلَ إِلَيْكَ».
\par 13 فَقَالَ: «حَسَناً. أَنَا أَقْطَعُ مَعَكَ عَهْداً، إِلاَّ إِنِّي أَطْلُبُ مِنْكَ أَمْراً وَاحِداً، وَهُوَ أَنْ لاَ تَرَى وَجْهِي مَا لَمْ تَأْتِ أَوَّلاً بِمِيكَالَ بِنْتِ شَاوُلَ حِينَ تَأْتِي لِتَرَى وَجْهِي».
\par 14 وَأَرْسَلَ دَاوُدُ رُسُلاً إِلَى إِيشْبُوشَثَ بْنِ شَاوُلَ يَقُولُ: «أَعْطِنِي امْرَأَتِي مِيكَالَ الَّتِي خَطَبْتُهَا لِنَفْسِي بِمِئَةِ غُلْفَةٍ مِنَ الْفِلِسْطِينِيِّينَ».
\par 15 فَأَرْسَلَ إِيشْبُوشَثُ وَأَخَذَهَا مِنْ عِنْدِ رَجُلِهَا، مِنْ فَلْطِيئِيلَ بْنِ لاَيِشَ.
\par 16 وَكَانَ رَجُلُهَا يَسِيرُ مَعَهَا وَيَبْكِي وَرَاءَهَا إِلَى بَحُورِيمَ. فَقَالَ لَهُ أَبْنَيْرُ: «اذْهَبِ ارْجِعْ». فَرَجَعَ.
\par 17 وَكَانَ كَلاَمُ أَبْنَيْرَ إِلَى شُيُوخِ إِسْرَائِيلَ: «قَدْ كُنْتُمْ مُنْذُ أَمْسٍ وَمَا قَبْلَهُ تَطْلُبُونَ دَاوُدَ لِيَكُونَ مَلِكاً عَلَيْكُمْ.
\par 18 فَالآنَ افْعَلُوا. لأَنَّ الرَّبَّ قَالَ لِدَاوُدَ: «إِنِّي بِيَدِ دَاوُدَ عَبْدِي أُخَلِّصُ شَعْبِي إِسْرَائِيلَ مِنْ يَدِ الْفِلِسْطِينِيِّينَ وَمِنْ أَيْدِي جَمِيعِ أَعْدَائِهِمْ».
\par 19 وَتَكَلَّمَ أَبْنَيْرُ أَيْضاً فِي مَسَامِعِ بِنْيَامِينَ، وَذَهَبَ أَبْنَيْرُ لِيَتَكَلَّمَ فِي مَسَامِعِ دَاوُدَ أَيْضاً فِي حَبْرُونَ بِكُلِّ مَا حَسُنَ فِي أَعْيُنِ إِسْرَائِيلَ وَفِي أَعْيُنِ جَمِيعِ بَيْتِ بِنْيَامِينَ.
\par 20 فَجَاءَ أَبْنَيْرُ إِلَى دَاوُدَ إِلَى حَبْرُونَ وَمَعَهُ عِشْرُونَ رَجُلاً. فَصَنَعَ دَاوُدُ لأَبْنَيْرَ وَلِلرِّجَالِ الَّذِينَ مَعَهُ وَلِيمَةً.
\par 21 وَقَالَ أَبْنَيْرُ لِدَاوُدَ: «أَقُومُ وَأَذْهَبُ وَأَجْمَعُ إِلَى سَيِّدِي الْمَلِكِ جَمِيعَ إِسْرَائِيلَ، فَيَقْطَعُونَ مَعَكَ عَهْداً، وَتَمْلِكُ حَسَبَ كُلِّ مَا تَشْتَهِي نَفْسُكَ». فَأَرْسَلَ دَاوُدُ أَبْنَيْرَ فَذَهَبَ بِسَلاَمٍ.
\par 22 وَإِذَا بِعَبِيدِ دَاوُدَ وَيُوآبُ قَدْ جَاءُوا مِنَ الْغَزْوِ وَأَتُوا بِغَنِيمَةٍ كَثِيرَةٍ مَعَهُمْ، وَلَمْ يَكُنْ أَبْنَيْرُ مَعَ دَاوُدَ فِي حَبْرُونَ، لأَنَّهُ كَانَ قَدْ أَرْسَلَهُ فَذَهَبَ بِسَلاَمٍ.
\par 23 وَجَاءَ يُوآبُ وَكُلُّ الْجَيْشِ الَّذِي مَعَهُ. فَأَخْبَرُوا يُوآبَ: «قَدْ جَاءَ أَبْنَيْرُ بْنُ نَيْرٍ إِلَى الْمَلِكِ فَأَرْسَلَهُ فَذَهَبَ بِسَلاَمٍ».
\par 24 فَدَخَلَ يُوآبُ إِلَى الْمَلِكِ وَقَالَ: «مَاذَا فَعَلْتَ؟ هُوَذَا قَدْ جَاءَ أَبْنَيْرُ إِلَيْكَ. لِمَاذَا أَرْسَلْتَهُ فَذَهَبَ؟
\par 25 أَنْتَ تَعْلَمُ أَبْنَيْرَ بْنَ نَيْرٍ أَنَّهُ إِنَّمَا جَاءَ لِيُمَلِّقَكَ وَلِيَعْلَمَ خُرُوجَكَ وَدُخُولَكَ وَلِيَعْلَمَ كُلَّ مَا تَصْنَعُ».
\par 26 ثُمَّ خَرَجَ يُوآبُ مِنْ عِنْدِ دَاوُدَ وَأَرْسَلَ رُسُلاً وَرَاءَ أَبْنَيْرَ، فَرَدُّوهُ مِنْ بِئْرِ السِّيرَةِ وَدَاوُدُ لاَ يَعْلَمُ.
\par 27 وَلَمَّا رَجَعَ أَبْنَيْرُ إِلَى حَبْرُونَ مَالَ بِهِ يُوآبُ إِلَى وَسَطِ الْبَابِ لِيُكَلِّمَهُ سِرّاً، وَضَرَبَهُ هُنَاكَ فِي بَطْنِهِ فَمَاتَ بِدَمِ عَسَائِيلَ أَخِيهِ.
\par 28 فَسَمِعَ دَاوُدُ بَعْدَ ذَلِكَ فَقَالَ: «إِنِّي بَرِيءٌ أَنَا وَمَمْلَكَتِي لَدَى الرَّبِّ إِلَى الأَبَدِ مِنْ دَمِ أَبْنَيْرَ بْنِ نَيْرٍ.
\par 29 فَلْيَحِلَّ عَلَى رَأْسِ يُوآبَ وَعَلَى كُلِّ بَيْتِ أَبِيهِ، وَلاَ يَنْقَطِعُ مِنْ بَيْتِ يُوآبَ ذُو سَيْلٍ وَأَبْرَصُ وَعَاكِزٌ عَلَى الْعُكَّازَةِ وَسَاقِطٌ بِالسَّيْفِ وَمُحْتَاجُ الْخُبْزِ».
\par 30 فَقَتَلَ يُوآبُ وَأَبِيشَايُ أَخُوهُ أَبْنَيْرَ لأَنَّهُ قَتَلَ عَسَائِيلَ أَخَاهُمَا فِي جِبْعُونَ فِي الْحَرْبِ.
\par 31 فَقَالَ دَاوُدُ لِيُوآبَ وَلِجَمِيعِ الشَّعْبِ الَّذِي مَعَهُ: «مَزِّقُوا ثِيَابَكُمْ وَتَنَطَّقُوا بِالْمُسُوحِ وَالْطِمُوا أَمَامَ أَبْنَيْرَ». وَكَانَ دَاوُدُ الْمَلِكُ يَمْشِي وَرَاءَ النَّعْشِ.
\par 32 وَدَفَنُوا أَبْنَيْرَ فِي حَبْرُونَ. وَرَفَعَ الْمَلِكُ صَوْتَهُ وَبَكَى عَلَى قَبْرِ أَبْنَيْرَ، وَبَكَى جَمِيعُ الشَّعْبِ.
\par 33 وَرَثَا الْمَلِكُ أَبْنَيْرَ وَقَالَ: «هَلْ كَمَوْتِ أَحْمَقَ يَمُوتُ أَبْنَيْرُ؟
\par 34 يَدَاكَ لَمْ تَكُونَا مَرْبُوطَتَيْنِ، وَرِجْلاَكَ لَمْ تُوضَعَا فِي سَلاَسِلِ نُحَاسٍ. كَالسُّقُوطِ أَمَامَ بَنِي الإِثْمِ سَقَطْتَ». وَعَادَ جَمِيعُ الشَّعْبِ يَبْكُونَ عَلَيْهِ.
\par 35 وَجَاءَ جَمِيعُ الشَّعْبِ لِيُطْعِمُوا دَاوُدَ خُبْزاً، وَكَانَ بَعْدُ نَهَارٌ. فَحَلَفَ دَاوُدُ: «هَكَذَا يَفْعَلُ لِيَ اللَّهُ وَهَكَذَا يَزِيدُ إِنْ كُنْتُ أَذُوقُ خُبْزاً أَوْ شَيْئاً آخَرَ قَبْلَ غُرُوبِ الشَّمْسِ».
\par 36 فَعَرَفَ جَمِيعُ الشَّعْبِ وَحَسُنَ فِي أَعْيُنِهِمْ، كَمَا أَنَّ كُلَّ مَا صَنَعَ الْمَلِكُ كَانَ حَسَناً فِي أَعْيُنِ جَمِيعِ الشَّعْبِ.
\par 37 وَعَلِمَ كُلُّ الشَّعْبِ وَجَمِيعُ إِسْرَائِيلَ فِي ذَلِكَ الْيَوْمِ أَنَّهُ لَمْ يَكُنْ مِنَ الْمَلِكِ قَتْلُ أَبْنَيْرَ بْنِ نَيْرٍ.
\par 38 وَقَالَ الْمَلِكُ لِعَبِيدِهِ: «أَلاَ تَعْلَمُونَ أَنَّ رَئِيساً وَعَظِيماً سَقَطَ الْيَوْمَ فِي إِسْرَائِيلَ؟
\par 39 وَأَنَا الْيَوْمَ ضَعِيفٌ وَمَمْسُوحٌ مَلِكاً، وَهَؤُلاَءِ الرِّجَالُ بَنُو صَرُويَةَ أَقْوَى مِنِّي. يُجَازِي الرَّبُّ فَاعِلَ الشَّرِّ كَشَرِّهِ».

\chapter{4}

\par 1 وَلَمَّا سَمِعَ ابْنُ شَاوُلَ أَنَّ أَبْنَيْرَ قَدْ مَاتَ فِي حَبْرُونَ ارْتَخَتْ يَدَاهُ، وَارْتَاعَ جَمِيعُ إِسْرَائِيلَ.
\par 2 وَكَانَ لاِبْنِ شَاوُلَ رَجُلاَنِ رَئِيسَا غُزَاةٍ، اسْمُ الْوَاحِدِ بَعْنَةُ وَاسْمُ الآخَرِ رَكَابُ، ابْنَا رِمُّونَ الْبَئِيرُوتِيِّ مِنْ بَنِي بِنْيَامِينَ (لأَنَّ بَئِيرُوتَ حُسِبَتْ لِبِنْيَامِينَ.
\par 3 وَهَرَبَ الْبَئِيرُوتِيُّونَ إِلَى جَتَّايِمَ وَتَغَرَّبُوا هُنَاكَ إِلَى هَذَا الْيَوْمِ).
\par 4 وَكَانَ لِيُونَاثَانَ بْنِ شَاوُلَ ابْنٌ مَضْرُوبُ الرِّجْلَيْنِ، كَانَ ابْنَ خَمْسِ سِنِينٍ عِنْدَ مَجِيءِ خَبَرِ شَاوُلَ وَيُونَاثَانَ مِنْ يَزْرَعِيلَ، فَحَمَلَتْهُ مُرَبِّيَتُهُ وَهَرَبَتْ. وَلَمَّا كَانَتْ مُسْرِعَةً لِتَهْرُبَ وَقَعَ وَصَارَ أَعْرَجَ. وَاسْمُهُ مَفِيبُوشَثُ.
\par 5 وَسَارَ ابْنَا رِمُّونَ الْبَئِيرُوتِيِّ، رَكَابُ وَبَعْنَةُ، وَدَخَلاَ عِنْدَ حَرِّ النَّهَارِ إِلَى بَيْتِ إِيشْبُوشَثَ وَهُوَ نَائِمٌ نَوْمَةَ الظَّهِيرَةِ.
\par 6 فَدَخَلاَ إِلَى وَسَطِ الْبَيْتِ لِيَأْخُذَا حِنْطَةً، وَضَرَبَاهُ فِي بَطْنِهِ. ثُمَّ أَفْلَتَ رَكَابُ وَبَعْنَةُ أَخُوهُ.
\par 7 فَعِنْدَ دُخُولِهِمَا الْبَيْتَ كَانَ هُوَ مُضْطَجِعاً عَلَى سَرِيرِهِ فِي مِخْدَعِ نَوْمِهِ فَضَرَبَاهُ وَقَتَلاَهُ وَقَطَعَا رَأْسَهُ، وَأَخَذَا رَأْسَهُ وَسَارَا فِي طَرِيقِ الْعَرَبَةِ اللَّيْلَ كُلَّهُ.
\par 8 وَأَتَيَا بِرَأْسِ إِيشْبُوشَثَ إِلَى دَاوُدَ إِلَى حَبْرُونَ، وَقَالاَ لِلْمَلِكِ: «هُوَذَا رَأْسُ إِيشْبُوشَثَ بْنِ شَاوُلَ عَدُوِّكَ الَّذِي كَانَ يَطْلُبُ نَفْسَكَ. وَقَدْ أَعْطَى الرَّبُّ لِسَيِّدِي الْمَلِكِ انْتِقَاماً فِي هَذَا الْيَوْمِ مِنْ شَاوُلَ وَمِنْ نَسْلِهِ».
\par 9 فَأَجَابَ دَاوُدُ رَكَابَ وَبَعْنَةَ أَخَاهُ، ابْنَيْ رِمُّونَ الْبَئِيرُوتِيِّ: «حَيٌّ هُوَ الرَّبُّ الَّذِي فَدَى نَفْسِي مِنْ كُلِّ ضِيقٍ
\par 10 إِنَّ الَّذِي أَخْبَرَنِي قَائِلاً: هُوَذَا قَدْ مَاتَ شَاوُلُ وَكَانَ فِي عَيْنَيْ نَفْسِهِ كَمُبَشِّرٍ قَبَضْتُ عَلَيْهِ وَقَتَلْتُهُ فِي صِقْلَغَ. ذَلِكَ أَعْطَيْتُهُ بِشَارَةً.
\par 11 فَكَمْ بِالْحَرِيِّ إِذَا كَانَ رَجُلاَنِ بَاغِيَانِ يَقْتُلاَنِ رَجُلاً صِدِّيقاً فِي بَيْتِهِ عَلَى سَرِيرِهِ! فَالآنَ أَمَا أَطْلُبُ دَمَهُ مِنْ أَيْدِيكُمَا وَأَنْزِعُكُمَا مِنَ الأَرْضِ؟»
\par 12 وَأَمَرَ دَاوُدُ الْغِلْمَانَ فَقَتَلُوهُمَا، وَقَطَعُوا أَيْدِيَهُمَا وَأَرْجُلَهُمَا وَعَلَّقُوهُمَا عَلَى الْبِرْكَةِ فِي حَبْرُونَ. وَأَمَّا رَأْسُ إِيشْبُوشَثَ فَأَخَذُوهُ وَدَفَنُوهُ فِي قَبْرِ أَبْنَيْرَ فِي حَبْرُونَ.

\chapter{5}

\par 1 وَجَاءَ جَمِيعُ أَسْبَاطِ إِسْرَائِيلَ إِلَى دَاوُدَ إِلَى حَبْرُونَ قَائِلِينَ: «هُوَذَا عَظْمُكَ وَلَحْمُكَ نَحْنُ.
\par 2 وَمُنْذُ أَمْسِ وَمَا قَبْلَهُ، حِينَ كَانَ شَاوُلُ مَلِكاً عَلَيْنَا، قَدْ كُنْتَ أَنْتَ تُخْرِجُ وَتُدْخِلُ إِسْرَائِيلَ. وَقَدْ قَالَ لَكَ الرَّبُّ: أَنْتَ تَرْعَى شَعْبِي إِسْرَائِيلَ، وَأَنْتَ تَكُونُ رَئِيساً عَلَى إِسْرَائِيلَ».
\par 3 وَجَاءَ جَمِيعُ شُيُوخِ إِسْرَائِيلَ إِلَى الْمَلِكِ إِلَى حَبْرُونَ، فَقَطَعَ الْمَلِكُ دَاوُدُ مَعَهُمْ عَهْداً فِي حَبْرُونَ أَمَامَ الرَّبِّ. وَمَسَحُوا دَاوُدَ مَلِكاً عَلَى إِسْرَائِيلَ.
\par 4 كَانَ دَاوُدُ ابْنَ ثَلاَثِينَ سَنَةً حِينَ مَلَكَ، وَمَلَكَ أَرْبَعِينَ سَنَةً.
\par 5 فِي حَبْرُونَ مَلَكَ عَلَى يَهُوذَا سَبْعَ سِنِينٍ وَسِتَّةَ أَشْهُرٍ. وَفِي أُورُشَلِيمَ مَلَكَ ثَلاَثاً وَثَلاَثِينَ سَنَةً عَلَى جَمِيعِ إِسْرَائِيلَ وَيَهُوذَا.
\par 6 وَذَهَبَ الْمَلِكُ وَرِجَالُهُ إِلَى أُورُشَلِيمَ إِلَى الْيَبُوسِيِّينَ سُكَّانِ الأَرْضِ. فَقَالُوا لِدَاوُدَ: «لاَ تَدْخُلْ إِلَى هُنَا مَا لَمْ تَنْزِعِ الْعُمْيَانَ وَالْعُرْجَ». (أَيْ لاَ يَدْخُلُ دَاوُدُ إِلَى هُنَا).
\par 7 وَأَخَذَ دَاوُدُ حِصْنَ صِهْيَوْنَ (هِيَ مَدِينَةُ دَاوُدَ).
\par 8 وَقَالَ دَاوُدُ فِي ذَلِكَ الْيَوْمِ: «إِنَّ الَّذِي يَضْرِبُ الْيَبُوسِيِّينَ وَيَبْلُغُ إِلَى الْقَنَاةِ (وَالْعُرْجِ وَالْعُمْيِ الْمُبْغَضِينَ مِنْ نَفْسِ دَاوُدَ) لِذَلِكَ يَقُولُونَ: لاَ يَدْخُلِ الْبَيْتَ أَعْمَى أَوْ أَعْرَجُ».
\par 9 وَأَقَامَ دَاوُدُ فِي الْحِصْنِ وَسَمَّاهُ «مَدِينَةَ دَاوُدَ». وَبَنَى دَاوُدُ مُسْتَدِيراً مِنَ الْقَلْعَةِ فَدَاخِلاً.
\par 10 وَكَانَ دَاوُدُ يَتَزَايَدُ مُتَعَظِّماً وَالرَّبُّ إِلَهُ الْجُنُودِ مَعَهُ.
\par 11 وَأَرْسَلَ حِيرَامُ مَلِكُ صُورَ رُسُلاً إِلَى دَاوُدَ وَخَشَبَ أَرْزٍ وَنَجَّارِينَ وَبَنَّائِينَ فَبَنُوا لِدَاوُدَ بَيْتاً.
\par 12 وَعَلِمَ دَاوُدُ أَنَّ الرَّبَّ قَدْ أَثْبَتَهُ مَلِكاً عَلَى إِسْرَائِيلَ، وَأَنَّهُ قَدْ رَفَّعَ مُلْكَهُ مِنْ أَجْلِ شَعْبِهِ إِسْرَائِيلَ.
\par 13 وَأَخَذَ دَاوُدُ أَيْضاً سَرَارِيَ وَنِسَاءً مِنْ أُورُشَلِيمَ بَعْدَ مَجِيئِهِ مِنْ حَبْرُونَ، فَوُلِدَ أَيْضاً لِدَاوُدَ بَنُونَ وَبَنَاتٌ.
\par 14 وَهَذِهِ أَسْمَاءُ الَّذِينَ وُلِدُوا لَهُ فِي أُورُشَلِيمَ: شَمُّوعُ وَشُوبَابُ وَنَاثَانُ وَسُلَيْمَانُ
\par 15 وَيِبْحَارُ وَأَلِيشُوعُ وَنَافَجُ وَيَافِيعُ
\par 16 وَأَلِيشَمَعُ وَأَلِيدَاعُ وَأَلِيفَلَطُ.
\par 17 وَسَمِعَ الْفِلِسْطِينِيُّونَ أَنَّهُمْ قَدْ مَسَحُوا دَاوُدَ مَلِكاً عَلَى إِسْرَائِيلَ، فَصَعِدَ جَمِيعُ الْفِلِسْطِينِيِّينَ لِيُفَتِّشُوا عَلَى دَاوُدَ. وَلَمَّا سَمِعَ دَاوُدُ نَزَلَ إِلَى الْحِصْنِ.
\par 18 وَجَاءَ الْفِلِسْطِينِيُّونَ وَانْتَشَرُوا فِي وَادِي الرَّفَائِيِّينَ.
\par 19 وَسَأَلَ دَاوُدُ مِنَ الرَّبِّ: «أَأَصْعَدُ إِلَى الْفِلِسْطِينِيِّينَ؟ أَتَدْفَعُهُمْ لِيَدِي؟» فَقَالَ الرَّبُّ لِدَاوُدَ: «اصْعَدْ لأَنِّي دَفْعاً أَدْفَعُ الْفِلِسْطِينِيِّينَ لِيَدِكَ».
\par 20 فَجَاءَ دَاوُدُ إِلَى بَعْلِ فَرَاصِيمَ وَضَرَبَهُمْ دَاوُدُ هُنَاكَ، وَقَالَ: «قَدِ اقْتَحَمَ الرَّبُّ أَعْدَائِي أَمَامِي كَاقْتِحَامِ الْمِيَاهِ». لِذَلِكَ دَعَى اسْمَ ذَلِكَ الْمَوْضِعِ «بَعْلَ فَرَاصِيمَ».
\par 21 وَتَرَكُوا هُنَاكَ أَصْنَامَهُمْ فَنَزَعَهَا دَاوُدُ وَرِجَالُهُ.
\par 22 ثُمَّ عَادَ الْفِلِسْطِينِيُّونَ فَصَعِدُوا أَيْضاً وَانْتَشَرُوا فِي وَادِي الرَّفَائِيِّينَ.
\par 23 فَسَأَلَ دَاوُدُ مِنَ الرَّبِّ فَقَالَ: «لاَ تَصْعَدْ، بَلْ دُرْ مِنْ وَرَائِهِمْ وَهَلُمَّ عَلَيْهِمْ مُقَابِلَ أَشْجَارِ الْبُكَا
\par 24 وَعِنْدَمَا تَسْمَعُ صَوْتَ خَطَوَاتٍ فِي رُؤُوسِ أَشْجَارِ الْبُكَا حِينَئِذٍ احْتَرِصْ، لأَنَّهُ إِذْ ذَاكَ يَخْرُجُ الرَّبُّ أَمَامَكَ لِضَرْبِ مَحَلَّةِ الْفِلِسْطِينِيِّينَ».
\par 25 فَفَعَلَ دَاوُدُ كَذَلِكَ كَمَا أَمَرَهُ الرَّبُّ، وَضَرَبَ الْفِلِسْطِينِيِّينَ مِنْ جَبْعٍَ إِلَى مَدْخَلِ جَازَرَ.

\chapter{6}

\par 1 وَجَمَعَ دَاوُدُ أَيْضاً جَمِيعَ الْمُنْتَخَبِينَ فِي إِسْرَائِيلَ، ثَلاَثِينَ أَلْفاً.
\par 2 وَقَامَ دَاوُدُ وَذَهَبَ هُوَ وَجَمِيعُ الشَّعْبِ الَّذِي مَعَهُ مِنْ بَعَلَةِ يَهُوذَا لِيُصْعِدُوا مِنْ هُنَاكَ تَابُوتَ اللَّهِ الَّذِي يُدْعَى عَلَيْهِ بِاسْمِ رَبِّ الْجُنُودِ الْجَالِسِ عَلَى الْكَرُوبِيمِ.
\par 3 فَأَرْكَبُوا تَابُوتَ اللَّهِ عَلَى عَجَلَةٍ جَدِيدَةٍ، وَحَمَلُوهُ مِنْ بَيْتِ أَبِينَادَابَ الَّذِي فِي الأَكَمَةِ. وَكَانَ عُزَّةُ وَأَخِيُو ابْنَا أَبِينَادَابَ يَسُوقَانِ الْعَجَلَةَ الْجَدِيدَةَ.
\par 4 فَأَخَذُوهَا مِنْ بَيْتِ أَبِينَادَابَ الَّذِي فِي الأَكَمَةِ مَعَ تَابُوتِ اللَّهِ. وَكَانَ أَخِيُو يَسِيرُ أَمَامَ التَّابُوتِ
\par 5 وَدَاوُدُ وَكُلُّ بَيْتِ إِسْرَائِيلَ يَلْعَبُونَ أَمَامَ الرَّبِّ بِكُلِّ أَنْوَاعِ الآلاَتِ مِنْ خَشَبِ السَّرْوِ بِالْعِيدَانِ وَبِالرَّبَابِ وَبِالدُّفُوفِ وَبِالْجُنُوكِ وَبِالصُّنُوجِ.
\par 6 وَلَمَّا انْتَهُوا إِلَى بَيْدَرِ نَاخُونَ مَدَّ عُزَّةُ يَدَهُ إِلَى تَابُوتِ اللَّهِ وَأَمْسَكَهُ، لأَنَّ الثِّيرَانَ تَعَثَّرَتْ.
\par 7 فَحَمِيَ غَضَبُ الرَّبِّ عَلَى عُزَّةَ وَضَرَبَهُ اللَّهُ هُنَاكَ لأَجْلِ غَفَلِهِ، فَمَاتَ هُنَاكَ لَدَى تَابُوتِ اللَّهِ.
\par 8 فَاغْتَاظَ دَاوُدُ لأَنَّ الرَّبَّ اقْتَحَمَ عُزَّةَ اقْتِحَاماً، وَسَمَّى ذَلِكَ الْمَوْضِعَ «فَارِصَ عُزَّةَ» إِلَى هَذَا الْيَوْمِ.
\par 9 وَخَافَ دَاوُدُ مِنَ الرَّبِّ فِي ذَلِكَ الْيَوْمِ وَقَالَ: «كَيْفَ يَأْتِي إِلَيَّ تَابُوتُ الرَّبِّ؟»
\par 10 وَلَمْ يَشَأْ دَاوُدُ أَنْ يَنْقُلَ تَابُوتَ الرَّبِّ إِلَيْهِ إِلَى مَدِينَةِ دَاوُدَ، فَمَالَ بِهِ دَاوُدُ إِلَى بَيْتِ عُوبِيدَ أَدُومَ الْجَتِّيِّ.
\par 11 وَبَقِيَ تَابُوتُ الرَّبِّ فِي بَيْتِ عُوبِيدَ أَدُومَ الْجَتِّيِّ ثَلاَثَةَ أَشْهُرٍ. وَبَارَكَ الرَّبُّ عُوبِيدَ أَدُومَ وَكُلَّ بَيْتِهِ.
\par 12 فَأُخْبِرَ الْمَلِكُ دَاوُدُ: «قَدْ بَارَكَ الرَّبُّ بَيْتَ عُوبِيدَ أَدُومَ وَكُلَّ مَا لَهُ بِسَبَبِ تَابُوتِ اللَّهِ». فَذَهَبَ دَاوُدُ وَأَصْعَدَ تَابُوتَ اللَّهِ مِنْ بَيْتِ عُوبِيدَ أَدُومَ إِلَى مَدِينَةِ دَاوُدَ بِفَرَحٍ.
\par 13 وَكَانَ كُلَّمَا خَطَا حَامِلُو تَابُوتِ الرَّبِّ سِتَّ خَطَوَاتٍ يَذْبَحُ ثَوْراً وَعِجْلاً مَعْلُوفاً.
\par 14 وَكَانَ دَاوُدُ يَرْقُصُ بِكُلِّ قُوَّتِهِ أَمَامَ الرَّبِّ. وَكَانَ دَاوُدُ مُتَنَطِّقاً بِأَفُودٍ مِنْ كَتَّانٍ.
\par 15 فَأَصْعَدَ دَاوُدُ وَجَمِيعُ بَيْتِ إِسْرَائِيلَ تَابُوتَ الرَّبِّ بِالْهُتَافِ وَبِصَوْتِ الْبُوقِ.
\par 16 وَلَمَّا دَخَلَ تَابُوتُ الرَّبِّ مَدِينَةَ دَاوُدَ، أَشْرَفَتْ مِيكَالُ بِنْتُ شَاوُلَ مِنَ الْكُوَّةِ وَرَأَتِ الْمَلِكَ دَاوُدَ يَطْفُرُ وَيَرْقُصُ أَمَامَ الرَّبِّ، فَاحْتَقَرَتْهُ فِي قَلْبِهَا.
\par 17 فَأَدْخَلُوا تَابُوتَ الرَّبِّ وَأَوْقَفُوهُ فِي مَكَانِهِ فِي وَسَطِ الْخَيْمَةِ الَّتِي نَصَبَهَا لَهُ دَاوُدُ، وَأَصْعَدَ دَاوُدُ مُحْرَقَاتٍ أَمَامَ الرَّبِّ وَذَبَائِحَ سَلاَمَةٍ،
\par 18 وَلَمَّا انْتَهَى دَاوُدُ مِنْ إِصْعَادِ الْمُحْرَقَاتِ وَذَبَائِحِ السَّلاَمَةِ بَارَكَ الشَّعْبَ بِاسْمِ رَبِّ الْجُنُودِ.
\par 19 وَقَسَمَ عَلَى جَمِيعِ الشَّعْبِ، عَلَى كُلِّ جُمْهُورِ إِسْرَائِيلَ رِجَالاً وَنِسَاءً، عَلَى كُلِّ وَاحِدٍ رَغِيفَ خُبْزٍ وَكَأْسَ خَمْرٍ وَقُرْصَ زَبِيبٍ. ثُمَّ ذَهَبَ كُلُّ الشَّعْبِ كُلُّ وَاحِدٍ إِلَى بَيْتِهِ
\par 20 وَرَجَعَ دَاوُدُ لِيُبَارِكَ بَيْتَهُ. فَخَرَجَتْ مِيكَالُ بِنْتُ شَاوُلَ لاِسْتِقْبَالِ دَاوُدَ، وَقَالَتْ: «مَا كَانَ أَكْرَمَ مَلِكَ إِسْرَائِيلَ الْيَوْمَ حَيْثُ تَكَشَّفَ الْيَوْمَ فِي أَعْيُنِ إِمَاءِ عَبِيدِهِ كَمَا يَتَكَشَّفُ أَحَدُ السُّفَهَاءِ!»
\par 21 فَقَالَ دَاوُدُ لِمِيكَالَ: «إِنَّمَا أَمَامَ الرَّبِّ الَّذِي اخْتَارَنِي دُونَ أَبِيكِ وَدُونَ كُلَّ بَيْتِهِ لِيُقِيمَنِي رَئِيساً عَلَى شَعْبِ الرَّبِّ إِسْرَائِيلَ، فَلَعِبْتُ أَمَامَ الرَّبِّ.
\par 22 وَإِنِّي أَتَصَاغَرُ دُونَ ذَلِكَ وَأَكُونُ وَضِيعاً فِي عَيْنَيْ نَفْسِي. وَأَمَّا عِنْدَ الإِمَاءِ الَّتِي ذَكَرْتِ فَأَتَمَجَّدُ».
\par 23 وَلَمْ يَكُنْ لِمِيكَالَ بِنْتِ شَاوُلَ وَلَدٌ إِلَى يَوْمِ مَوْتِهَا.

\chapter{7}

\par 1 وَكَانَ لَمَّا سَكَنَ الْمَلِكُ فِي بَيْتِهِ وَأَرَاحَهُ الرَّبُّ مِنْ كُلِّ الْجِهَاتِ مِنْ جَمِيعِ أَعْدَائِهِ
\par 2 أَنَّ الْمَلِكَ قَالَ لِنَاثَانَ النَّبِيِّ: «انْظُرْ. إِنِّي سَاكِنٌ فِي بَيْتٍ مِنْ أَرْزٍ، وَتَابُوتُ اللَّهِ سَاكِنٌ دَاخِلَ الشُّقَقِ».
\par 3 فَقَالَ نَاثَانُ لِلْمَلِكِ: «اذْهَبِ افْعَلْ كُلَّ مَا بِقَلْبِكَ، لأَنَّ الرَّبَّ مَعَكَ».
\par 4 وَفِي تِلْكَ اللَّيْلَةِ كَانَ كَلاَمُ الرَّبِّ إِلَى نَاثَانَ:
\par 5 «اِذْهَبْ وَقُلْ لِعَبْدِي دَاوُدَ: هَكَذَا قَالَ الرَّبُّ: أَأَنْتَ تَبْنِي لِي بَيْتاً لِسُكْنَايَ؟
\par 6 لأَنِّي لَمْ أَسْكُنْ فِي بَيْتٍ مُنْذُ يَوْمَ أَصْعَدْتُ بَنِي إِسْرَائِيلَ مِنْ مِصْرَ إِلَى هَذَا الْيَوْمِ، بَلْ كُنْتُ أَسِيرُ فِي خَيْمَةٍ وَفِي مَسْكَنٍ.
\par 7 فِي كُلِّ مَا سِرْتُ مَعَ جَمِيعِ بَنِي إِسْرَائِيلَ، هَلْ قُلْتُ لأَحَدِ قُضَاةِ إِسْرَائِيلَ الَّذِينَ أَمَرْتُهُمْ أَنْ يَرْعُوا شَعْبِي إِسْرَائِيلَ: لِمَاذَا لَمْ تَبْنُوا لِي بَيْتاً مِنَ الأَرْزِ؟
\par 8 وَالآنَ فَهَكَذَا تَقُولُ لِعَبْدِي دَاوُدَ: هَكَذَا قَالَ رَبُّ الْجُنُودِ: أَنَا أَخَذْتُكَ مِنَ الْمَرْبَضِ مِنْ وَرَاءِ الْغَنَمِ لِتَكُونَ رَئِيساً عَلَى شَعْبِي إِسْرَائِيلَ.
\par 9 وَكُنْتُ مَعَكَ حَيْثُمَا تَوَجَّهْتَ، وَقَرَضْتُ جَمِيعَ أَعْدَائِكَ مِنْ أَمَامِكَ، وَعَمِلْتُ لَكَ اسْماً عَظِيماً كَاسْمِ الْعُظَمَاءِ الَّذِينَ فِي الأَرْضِ.
\par 10 وَعَيَّنْتُ مَكَاناً لِشَعْبِي إِسْرَائِيلَ وَغَرَسْتُهُ، فَسَكَنَ فِي مَكَانِهِ، وَلاَ يَضْطَرِبُ بَعْدُ وَلاَ يَعُودُ بَنُو الإِثْمِ يُذَلِّلُونَهُ كَمَا فِي الأَوَّلِ
\par 11 وَمُنْذُ يَوْمَ أَقَمْتُ فِيهِ قُضَاةً عَلَى شَعْبِي إِسْرَائِيلَ. وَقَدْ أَرَحْتُكَ مِنْ جَمِيعِ أَعْدَائِكَ. وَالرَّبُّ يُخْبِرُكَ أَنَّ الرَّبَّ يَصْنَعُ لَكَ بَيْتاً.
\par 12 مَتَى كَمِلَتْ أَيَّامُكَ وَاضْطَجَعْتَ مَعَ آبَائِكَ أُقِيمُ بَعْدَكَ نَسْلَكَ الَّذِي يَخْرُجُ مِنْ أَحْشَائِكَ وَأُثَبِّتُ مَمْلَكَتَهُ.
\par 13 هُوَ يَبْنِي بَيْتاً لاِسْمِي، وَأَنَا أُثَبِّتُ كُرْسِيَّ مَمْلَكَتِهِ إِلَى الأَبَدِ.
\par 14 أَنَا أَكُونُ لَهُ أَباً وَهُوَ يَكُونُ لِيَ ابْناً. إِنْ تَعَوَّجَ أُؤَدِّبْهُ بِقَضِيبِ النَّاسِ وَبِضَرَبَاتِ بَنِي آدَمَ.
\par 15 وَلَكِنَّ رَحْمَتِي لاَ تُنْزَعُ مِنْهُ كَمَا نَزَعْتُهَا مِنْ شَاوُلَ الَّذِي أَزَلْتُهُ مِنْ أَمَامِكَ.
\par 16 وَيَأْمَنُ بَيْتُكَ وَمَمْلَكَتُكَ إِلَى الأَبَدِ أَمَامَكَ. كُرْسِيُّكَ يَكُونُ ثَابِتاً إِلَى الأَبَدِ».
\par 17 فَحَسَبَ جَمِيعِ هَذَا الْكَلاَمِ وَحَسَبَ كُلِّ هَذِهِ الرُّؤْيَا كَذَلِكَ كَلَّمَ نَاثَانُ دَاوُدَ.
\par 18 فَدَخَلَ الْمَلِكُ دَاوُدُ وَجَلَسَ أَمَامَ الرَّبِّ وَقَالَ: «مَنْ أَنَا يَا سَيِّدِي الرَّبَّ، وَمَا هُوَ بَيْتِي حَتَّى أَوْصَلْتَنِي إِلَى هَهُنَا؟
\par 19 وَقَلَّ هَذَا أَيْضاً فِي عَيْنَيْكَ يَا سَيِّدِي الرَّبَّ فَتَكَلَّمْتَ أَيْضاً مِنْ جِهَةِ بَيْتِ عَبْدِكَ إِلَى زَمَانٍ طَوِيلٍ. وَهَذِهِ عَادَةُ الإِنْسَانِ يَا سَيِّدِي الرَّبَّ.
\par 20 وَبِمَاذَا يَعُودُ دَاوُدُ يُكَلِّمُكَ وَأَنْتَ قَدْ عَرَفْتَ عَبْدَكَ يَا سَيِّدِي الرَّبَّ؟
\par 21 فَمِنْ أَجْلِ كَلِمَتِكَ وَحَسَبَ قَلْبِكَ فَعَلْتَ هَذِهِ الْعَظَائِمَ كُلَّهَا لِتُعَرِّفَ عَبْدَكَ.
\par 22 لِذَلِكَ قَدْ عَظُمْتَ أَيُّهَا الرَّبُّ الإِلَهُ، لأَنَّهُ لَيْسَ مِثْلُكَ وَلَيْسَ إِلَهٌ غَيْرَكَ حَسَبَ كُلِّ مَا سَمِعْنَاهُ بِآذَانِنَا.
\par 23 وَأَيَّةُ أُمَّةٍ عَلَى الأَرْضِ مِثْلُ شَعْبِكَ إِسْرَائِيلَ الَّذِي سَارَ اللَّهُ لِيَفْتَدِيَهُ لِنَفْسِهِ شَعْباً، وَيَجْعَلَ لَهُ اسْماً، وَيَعْمَلَ لَكُمُ الْعَظَائِمَ وَالتَّخَاوِيفَ لأَرْضِكَ أَمَامَ شَعْبِكَ الَّذِي افْتَدَيْتَهُ لِنَفْسِكَ مِنْ مِصْرَ مِنَ الشُّعُوبِ وَآلِهَتِهِمْ.
\par 24 وَثَبَّتَّ شَعْبَكَ إِسْرَائِيلَ شَعْباً لِنَفْسِكَ إِلَى الأَبَدِ، وَأَنْتَ يَا رَبُّ صِرْتَ لَهُمْ إِلَهاً.
\par 25 وَالآنَ أَيُّهَا الرَّبُّ الإِلَهُ أَقِمْ إِلَى الأَبَدِ الْكَلاَمَ الَّذِي تَكَلَّمْتَ بِهِ عَنْ عَبْدِكَ وَعَنْ بَيْتِهِ، وَافْعَلْ كَمَا نَطَقْتَ.
\par 26 وَلِْيَتَعَظَّمِ اسْمُكَ إِلَى الأَبَدِ، فَيُقَالَ: رَبُّ الْجُنُودِ إِلَهٌ عَلَى إِسْرَائِيلَ. وَلْيَكُنْ بَيْتُ عَبْدِكَ دَاوُدَ ثَابِتاً أَمَامَكَ.
\par 27 لأَنَّكَ أَنْتَ يَا رَبَّ الْجُنُودِ إِلَهَ إِسْرَائِيلَ قَدْ أَعْلَنْتَ لِعَبْدِكَ قَائِلاً إِنِّي أَبْنِي لَكَ بَيْتاً. لِذَلِكَ وَجَدَ عَبْدُكَ فِي قَلْبِهِ أَنْ يُصَلِّيَ لَكَ هَذِهِ الصَّلاَةَ.
\par 28 وَالآنَ يَا سَيِّدِي الرَّبَّ أَنْتَ هُوَ اللَّهُ وَكَلاَمُكَ هُوَ حَقٌّ، وَقَدْ كَلَّمْتَ عَبْدَكَ بِهَذَا الْخَيْرِ.
\par 29 فَالآنَ ارْتَضِ وَبَارِكْ بَيْتَ عَبْدِكَ لِيَكُونَ إِلَى الأَبَدِ أَمَامَكَ، لأَنَّكَ أَنْتَ يَا سَيِّدِي الرَّبَّ قَدْ تَكَلَّمْتَ. فَلْيُبَارَكْ بَيْتُ عَبْدِكَ بِبَرَكَتِكَ إِلَى الأَبَدِ».

\chapter{8}

\par 1 وَبَعْدَ ذَلِكَ ضَرَبَ دَاوُدُ الْفِلِسْطِينِيِّينَ وَذَلَّلَهُمْ، وَأَخَذَ دَاوُدُ «زِمَامَ الْقَصَبَةِ» مِنْ يَدِ الْفِلِسْطِينِيِّينَ.
\par 2 وَضَرَبَ الْمُوآبِيِّينَ وَقَاسَهُمْ بِالْحَبْلِ. أَضْجَعَهُمْ عَلَى الأَرْضِ، فَقَاسَ بِحَبْلَيْنِ لِلْقَتْلِ وَبِحَبْلٍ لِلاِسْتِحْيَاءِ. وَصَارَ الْمُوآبِيُّونَ عَبِيداً لِدَاوُدَ يُقَدِّمُونَ هَدَايَا.
\par 3 وَضَرَبَ دَاوُدُ هَدَدَ عَزَرَ بْنَ رَحُوبَ مَلِكَ صُوبَةَ حِينَ ذَهَبَ لِيَرُدَّ سُلْطَتَهُ عِنْدَ نَهْرِ الْفُرَاتِ.
\par 4 فَأَخَذَ دَاوُدُ مِنْهُ أَلْفاً وَسَبْعَ مِئَةِ فَارِسٍ وَعِشْرِينَ أَلْفَ رَاجِلٍ. وَعَرْقَبَ دَاوُدُ جَمِيعَ خَيْلِ الْمَرْكَبَاتِ وَأَبْقَى مِنْهَا مِئَةَ مَرْكَبَةٍ.
\par 5 فَجَاءَ أَرَامُ دِمَشْقَ لِنَجْدَةِ هَدَدَ عَزَرَ مَلِكِ صُوبَةَ، فَضَرَبَ دَاوُدُ مِنْ أَرَامَ اثْنَيْنَ وَعِشْرِينَ أَلْفَ رَجُلٍ.
\par 6 وَجَعَلَ دَاوُدُ مُحَافِظِينَ فِي أَرَامِ دِمَشْقَ وَصَارَ الأَرَامِيُّونَ لِدَاوُدَ عَبِيداً يُقَدِّمُونَ هَدَايَا. وَكَانَ الرَّبُّ يُخَلِّصُ دَاوُدَ حَيْثُمَا تَوَجَّهَ.
\par 7 وَأَخَذَ دَاوُدُ أَتْرَاسَ الذَّهَبِ الَّتِي كَانَتْ عَلَى عَبِيدِ هَدَدَ عَزَرَ وَأَتَى بِهَا إِلَى أُورُشَلِيمَ.
\par 8 وَمِنْ بَاطِحَ وَمِنْ بِيرَوَثَايَ مَدِينَتَيْ هَدَدَ عَزَرَ أَخَذَ الْمَلِكُ دَاوُدُ نُحَاساً كَثِيراً جِدّاً.
\par 9 وَسَمِعَ تُوعِي مَلِكُ حَمَاةَ أَنَّ دَاوُدَ قَدْ ضَرَبَ كُلَّ جَيْشِ هَدَدَ عَزَرَ،
\par 10 فَأَرْسَلَ تُوعِي يُورَامَ ابْنَهُ إِلَى الْمَلِكِ دَاوُدَ لِيَسْأَلَ عَنْ سَلاَمَتِهِ وَيُبَارِكَهُ لأَنَّهُ حَارَبَ هَدَدَ عَزَرَ وَضَرَبَهُ، لأَنَّ هَدَدَ عَزَرَ كَانَتْ لَهُ حُرُوبٌ مَعَ تُوعِي. وَكَانَ بِيَدِهِ آنِيَةُ فِضَّةٍ وَآنِيَةُ ذَهَبٍ وَآنِيَةُ نُحَاسٍ.
\par 11 وَهَذِهِ أَيْضاً قَدَّسَهَا الْمَلِكُ دَاوُدُ لِلرَّبِّ مَعَ الْفِضَّةِ وَالذَّهَبِ الَّذِي قَدَّسَهُ مِنْ جَمِيعِ الشُّعُوبِ الَّذِينَ أَخْضَعَهُمْ.
\par 12 مِنْ أَرَامَ وَمِنْ مُوآبَ وَمِنْ بَنِي عَمُّونَ وَمِنَ الْفِلِسْطِينِيِّينَ وَمِنْ عَمَالِيقَ وَمِنْ غَنِيمَةِ هَدَدَ عَزَرَ بْنِ رَحُوبَ مَلِكِ صُوبَةَ.
\par 13 وَنَصَبَ دَاوُدُ تِذْكَاراً عِنْدَ رُجُوعِهِ مِنْ ضَرْبِهِ ثَمَانِيَةَ عَشَرَ أَلْفاً مِنْ أَرَامَ فِي وَادِي الْمِلْحِ.
\par 14 وَجَعَلَ فِي أَدُومَ مُحَافِظِينَ. وَضَعَ مُحَافِظِينَ فِي أَدُومَ كُلِّهَا. وَكَانَ جَمِيعُ الأَدُومِيِّينَ عَبِيداً لِدَاوُدَ. وَكَانَ الرَّبُّ يُخَلِّصُ دَاوُدَ حَيْثُمَا تَوَجَّهَ.
\par 15 وَمَلَكَ دَاوُدُ عَلَى جَمِيعِ إِسْرَائِيلَ. وَكَانَ دَاوُدُ يُجْرِي قَضَاءً وَعَدْلاً لِكُلِّ شَعْبِهِ.
\par 16 وَكَانَ يُوآبُ ابْنُ صَرُويَةَ عَلَى الْجَيْشِ، وَيَهُوشَافَاطُ بْنُ أَخِيلُودَ مُسَجِّلاً،
\par 17 وَصَادُوقُ بْنُ أَخِيطُوبَ وَأَخِيمَالِكُ بْنُ أَبِيَاثَارَ كَاهِنَيْنِ، وَسَرَايَا كَاتِباً،
\par 18 وَبَنَايَاهُو بْنُ يَهُويَادَاعَ عَلَى الْجَلاَّدِينَ وَالسُّعَاةِ، وَبَنُو دَاوُدَ كَانُوا كَهَنَةً.

\chapter{9}

\par 1 وَقَالَ دَاوُدُ: «هَلْ يُوجَدُ بَعْدُ أَحَدٌ قَدْ بَقِيَ مِنْ بَيْتِ شَاوُلَ فَأَصْنَعَ مَعَهُ مَعْرُوفاً مِنْ أَجْلِ يُونَاثَانَ؟»
\par 2 وَكَانَ لِبَيْتِ شَاوُلَ عَبْدٌ اسْمُهُ صِيبَا، فَاسْتَدْعُوهُ إِلَى دَاوُدَ، وَقَالَ لَهُ الْمَلِكُ: «أَأَنْتَ صِيبَا؟» فَقَالَ: «عَبْدُكَ».
\par 3 فَقَالَ الْمَلِكُ: «أَلاَ يُوجَدُ بَعْدُ أَحَدٌ لِبَيْتِ شَاوُلَ فَأَصْنَعَ مَعَهُ إِحْسَانَ اللَّهِ؟» فَقَالَ صِيبَا لِلْمَلِكِ: «بَعْدُ ابْنٌ لِيُونَاثَانَ أَعْرَجُ الرِّجْلَيْنِ».
\par 4 فَقَالَ لَهُ الْمَلِكُ: «أَيْنَ هُوَ؟» فَقَالَ صِيبَا لِلْمَلِكِ: «هُوَذَا هُوَ فِي بَيْتِ مَاكِيرَ بْنِ عَمِّيئِيلَ فِي لُودَبَارَ».
\par 5 فَأَرْسَلَ الْمَلِكُ دَاوُدُ وَأَخَذَهُ مِنْ بَيْتِ مَاكِيرَ بْنِ عَمِّيئِيلَ مِنْ لُودَبَارَ.
\par 6 فَجَاءَ مَفِيبُوشَثُ بْنُ يُونَاثَانَ بْنِ شَاوُلَ إِلَى دَاوُدَ وَخَرَّ عَلَى وَجْهِهِ وَسَجَدَ. فَقَالَ دَاوُدُ: «يَا مَفِيبُوشَثُ». فَقَالَ: «هَئَنَذَا عَبْدُكَ».
\par 7 فَقَالَ لَهُ دَاوُدُ: «لاَ تَخَفْ. فَإِنِّي لَأَعْمَلَنَّ مَعَكَ مَعْرُوفاً مِنْ أَجْلِ يُونَاثَانَ أَبِيكَ، وَأَرُدُّ لَكَ كُلَّ حُقُولِ شَاوُلَ أَبِيكَ، وَأَنْتَ تَأْكُلُ خُبْزاً عَلَى مَائِدَتِي دَائِماً».
\par 8 فَسَجَدَ وَقَالَ: «مَنْ هُوَ عَبْدُكَ حَتَّى تَلْتَفِتَ إِلَى كَلْبٍ مَيِّتٍ مِثْلِي؟».
\par 9 وَدَعَا الْمَلِكُ صِيبَا غُلاَمَ شَاوُلَ وَقَالَ لَهُ: «كُلُّ مَا كَانَ لِشَاوُلَ وَلِكُلِّ بَيْتِهِ قَدْ دَفَعْتُهُ لاِبْنِ سَيِّدِكَ.
\par 10 فَتَشْتَغِلُ لَهُ فِي الأَرْضِ أَنْتَ وَبَنُوكَ وَعَبِيدُكَ، وَتَسْتَغِلُّ لِيَكُونَ لاِبْنِ سَيِّدِكَ خُبْزٌ لِيَأْكُلَ. وَمَفِيبُوشَثُ ابْنُ سَيِّدِكَ يَأْكُلُ دَائِماً خُبْزاً عَلَى مَائِدَتِي». وَكَانَ لِصِيبَا خَمْسَةَ عَشَرَ ابْناً وَعِشْرُونَ عَبْداً.
\par 11 فَقَالَ صِيبَا لِلْمَلِكِ: «حَسَبَ كُلِّ مَا يَأْمُرُ بِهِ سَيِّدِي الْمَلِكُ عَبْدَهُ كَذَلِكَ يَصْنَعُ عَبْدُكَ». «فَيَأْكُلُ مَفِيبُوشَثُ عَلَى مَائِدَتِي كَوَاحِدٍ مِنْ بَنِي الْمَلِكِ».
\par 12 وَكَانَ لِمَفِيبُوشَثَ ابْنٌ صَغِيرٌ اسْمُهُ مِيخَا. وَكَانَ جَمِيعُ سَاكِنِي بَيْتِ صِيبَا عَبِيداً لِمَفِيبُوشَثَ.
\par 13 فَسَكَنَ مَفِيبُوشَثُ فِي أُورُشَلِيمَ لأَنَّهُ كَانَ يَأْكُلُ دَائِماً عَلَى مَائِدَةِ الْمَلِكِ. وَكَانَ أَعْرَجَ مِنْ رِجْلَيْهِ كِلْتَيْهِمَا.

\chapter{10}

\par 1 وَكَانَ بَعْدَ ذَلِكَ أَنَّ مَلِكَ بَنِي عَمُّونَ مَاتَ، وَمَلَكَ حَانُونُ ابْنُهُ عِوَضاً عَنْهُ.
\par 2 فَقَالَ دَاوُدُ: «أَصْنَعُ مَعْرُوفاً مَعَ حَانُونَ بْنِ نَاحَاشَ كَمَا صَنَعَ أَبُوهُ مَعِي مَعْرُوفاً». فَأَرْسَلَ دَاوُدُ بِيَدِ عَبِيدِهِ يُعَزِّيهِ عَنْ أَبِيهِ. فَجَاءَ عَبِيدُ دَاوُدَ إِلَى أَرْضِ بَنِي عَمُّونَ.
\par 3 فَقَالَ رُؤَسَاءُ بَنِي عَمُّونَ لِحَانُونَ سَيِّدِهِمْ: «هَلْ يُكْرِمُ دَاوُدُ أَبَاكَ فِي عَيْنَيْكَ حَتَّى أَرْسَلَ إِلَيْكَ مُعَزِّينَ؟ أَلَيْسَ لأَجْلِ فَحْصِ الْمَدِينَةِ وَتَجَسُّسِهَا وَقَلْبِهَا أَرْسَلَ دَاوُدُ عَبِيدَهُ إِلَيْكَ؟»
\par 4 فَأَخَذَ حَانُونُ عَبِيدَ دَاوُدَ وَحَلَقَ أَنْصَافَ لِحَاهُمْ، وَقَصَّ ثِيَابَهُمْ مِنَ الْوَسَطِ إِلَى أَسْتَاهِهِمْ، ثُمَّ أَطْلَقَهُمْ.
\par 5 وَلَمَّا أَخْبَرُوا دَاوُدَ أَرْسَلَ لِلِقَائِهِمْ لأَنَّ الرِّجَالَ كَانُوا خَجِلِينَ جِدّاً. وَقَالَ الْمَلِكُ: «أَقِيمُوا فِي أَرِيحَا حَتَّى تَنْبُتَ لِحَاكُمْ ثُمَّ ارْجِعُوا».
\par 6 وَلَمَّا رَأَى بَنُو عَمُّونَ أَنَّهُمْ قَدْ أَنْتَنُوا عِنْدَ دَاوُدَ أَرْسَلَ بَنُو عَمُّونَ وَاسْتَأْجَرُوا أَرَامَ بَيْتِ رَحُوبَ وَأَرَامَ صُوبَا، عِشْرِينَ أَلْفَ رَاجِلٍ، وَمِنْ مَلِكِ مَعْكَةَ أَلْفَ رَجُلٍ، وَرِجَالَ طُوبَ اثْنَيْ عَشَرَ أَلْفَ رَجُلٍ.
\par 7 فَلَمَّا سَمِعَ دَاوُدُ أَرْسَلَ يُوآبَ وَكُلَّ جَيْشِ الْجَبَابِرَةِ.
\par 8 وَخَرَجَ بَنُو عَمُّونَ وَاصْطَفُّوا لِلْحَرْبِ عَُِنْدَ مَدْخَلِ الْبَابِ، وَكَانَ أَرَامُ صُوبَا وَرَحُوبُ وَرِجَالُ طُوبَ وَمَعْكَةَ وَحْدَهُمْ فِي الْحَقْلِ.
\par 9 فَلَمَّا رَأَى يُوآبُ أَنَّ مُقَدَّمَةَ الْحَرْبِ كَانَتْ نَحْوَهُ مِنْ قُدَّامٍ وَمِنْ وَرَاءٍ، اخْتَارَ مِنْ جَمِيعِ مُنْتَخَبِي إِسْرَائِيلَ وَصَفَّهُمْ لِلِقَاءِ أَرَامَ
\par 10 وَسَلَّمَ بَقِيَّةَ الشَّعْبِ لِيَدِ أَخِيهِ أَبِيشَايَ فَصَفَّهُمْ لِلِقَاءِ بَنِي عَمُّونَ.
\par 11 وَقَالَ: «إِنْ قَوِيَ أَرَامُ عَلَيَّ تَكُونُ لِي مُنْجِداً. وَإِنْ قَوِيَ عَلَيْكَ بَنُو عَمُّونَ أَذْهَبُ لِنَجْدَتِكَ.
\par 12 تَجَلَّدْ وَلْنَتَشَدَّدْ مِنْ أَجْلِ شَعْبِنَا وَمِنْ أَجْلِ مُدُنِ إِلَهِنَا، وَالرَّبُّ يَفْعَلُ مَا يَحْسُنُ فِي عَيْنَيْهِ».
\par 13 فَتَقَدَّمَ يُوآبُ وَالشَّعْبُ الَّذِينَ مَعَهُ لِمُحَارَبَةِ أَرَامَ فَهَرَبُوا مِنْ أَمَامِهِ.
\par 14 وَلَمَّا رَأَى بَنُو عَمُّونَ أَنَّهُ قَدْ هَرَبَ أَرَامُ هَرَبُوا مِنْ أَمَامِ أَبِيشَايَ وَدَخَلُوا الْمَدِينَةَ. فَرَجَعَ يُوآبُ عَنْ بَنِي عَمُّونَ وَأَتَى إِلَى أُورُشَلِيمَ.
\par 15 وَلَمَّا رَأَى أَرَامُ أَنَّهُمْ قَدِ انْكَسَرُوا أَمَامَ إِسْرَائِيلَ اجْتَمَعُوا مَعاً.
\par 16 وَأَرْسَلَ هَدَدُ عَزَرُ فَأَبْرَزَ أَرَامَ الَّذِي فِي عَبْرِ النَّهْرِ، فَأَتُوا إِلَى حِيلاَمَ وَأَمَامَهُمْ شُوبَكُ رَئِيسُ جَيْشِ هَدَدَ عَزَرَ.
\par 17 وَلَمَّا أُخْبِرَ دَاوُدُ جَمَعَ كُلَّ إِسْرَائِيلَ وَعَبَرَ الأُرْدُنَّ وَجَاءَ إِلَى حِيلاَمَ، فَاصْطَفَّ أَرَامُ لِلِقَاءِ دَاوُدَ وَحَارَبُوهُ.
\par 18 وَهَرَبَ أَرَامُ مِنْ أَمَامِ إِسْرَائِيلَ، وَقَتَلَ دَاوُدُ مِنْ أَرَامَ سَبْعَ مِئَةِ مَرْكَبَةٍ وَأَرْبَعِينَ أَلْفَ فَارِسٍ، وَضَرَبَ شُوبَكَ رَئِيسَ جَيْشِهِ فَمَاتَ هُنَاكَ.
\par 19 وَلَمَّا رَأَى جَمِيعُ الْمُلُوكِ، عَبِيدُ هَدَدَ عَزَرَ أَنَّهُمُ انْكَسَرُوا أَمَامَ إِسْرَائِيلَ صَالَحُوا إِسْرَائِيلَ وَاسْتُعْبِدُوا لَهُمْ، وَخَافَ أَرَامُ أَنْ يُنْجِدُوا بَنِي عَمُّونَ بَعْدُ.

\chapter{11}

\par 1 وَكَانَ عِنْدَ تَمَامِ السَّنَةِ فِي وَقْتِ خُرُوجِ الْمُلُوكِ أَنَّ دَاوُدَ أَرْسَلَ يُوآبَ وَعَبِيدَهُ مَعَهُ وَجَمِيعَ إِسْرَائِيلَ، فَأَخْرَبُوا بَنِي عَمُّونَ وَحَاصَرُوا رَبَّةَ. وَأَمَّا دَاوُدُ فَأَقَامَ فِي أُورُشَلِيمَ.
\par 2 وَكَانَ فِي وَقْتِ الْمَسَاءِ أَنَّ دَاوُدَ قَامَ عَنْ سَرِيرِهِ وَتَمَشَّى عَلَى سَطْحِ بَيْتِ الْمَلِكِ، فَرَأَى مِنْ عَلَى السَّطْحِ امْرَأَةً تَسْتَحِمُّ. وَكَانَتِ الْمَرْأَةُ جَمِيلَةَ الْمَنْظَرِ جِدّاً.
\par 3 فَأَرْسَلَ دَاوُدُ وَسَأَلَ عَنِ الْمَرْأَةِ، فَقَالَ وَاحِدٌ: «أَلَيْسَتْ هَذِهِ بَثْشَبَعَ بِنْتَ أَلِيعَامَ امْرَأَةَ أُورِيَّا الْحِثِّيِّ؟»
\par 4 فَأَرْسَلَ دَاوُدُ رُسُلاً وَأَخَذَهَا، فَدَخَلَتْ إِلَيْهِ فَاضْطَجَعَ مَعَهَا وَهِيَ مُطَهَّرَةٌ مِنْ طَمْثِهَا. ثُمَّ رَجَعَتْ إِلَى بَيْتِهَا.
\par 5 وَحَبِلَتِ الْمَرْأَةُ، فَأَرْسَلَتْ وَأَخْبَرَتْ دَاوُدَ وَقَالَتْ: «إِنِّي حُبْلَى».
\par 6 فَأَرْسَلَ دَاوُدُ إِلَى يُوآبَ يَقُولُ: «أَرْسِلْ إِلَيَّ أُورِيَّا الْحِثِّيَّ». فَأَرْسَلَ يُوآبُ أُورِيَّا إِلَى دَاوُدَ.
\par 7 فَأَتَى أُورِيَّا إِلَيْهِ، فَسَأَلَ دَاوُدُ عَنْ سَلاَمَةِ يُوآبَ وَسَلاَمَةِ الشَّعْبِ وَنَجَاحِ الْحَرْبِ.
\par 8 وَقَالَ دَاوُدُ لِأُورِيَّا: «انْزِلْ إِلَى بَيْتِكَ وَاغْسِلْ رِجْلَيْكَ». فَخَرَجَ أُورِيَّا مِنْ بَيْتِ الْمَلِكِ، وَخَرَجَتْ وَرَاءَهُ حِصَّةٌ مِنْ عِنْدِ الْمَلِكِ.
\par 9 وَنَامَ أُورِيَّا عَلَى بَابِ بَيْتِ الْمَلِكِ مَعَ جَمِيعِ عَبِيدِ سَيِّدِهِ وَلَمْ يَنْزِلْ إِلَى بَيْتِهِ.
\par 10 فَقَالُوا لِدَاوُدَ: «لَمْ يَنْزِلْ أُورِيَّا إِلَى بَيْتِهِ». فَقَالَ دَاوُدُ لِأُورِيَّا: «أَمَا جِئْتَ مِنَ السَّفَرِ؟ فَلِمَاذَا لَمْ تَنْزِلْ إِلَى بَيْتِكَ؟»
\par 11 فَقَالَ أُورِيَّا لِدَاوُدَ: «إِنَّ التَّابُوتَ وَإِسْرَائِيلَ وَيَهُوذَا سَاكِنُونَ فِي الْخِيَامِ، وَسَيِّدِي يُوآبُ وَعَبِيدُ سَيِّدِي نَازِلُونَ عَلَى وَجْهِ الصَّحْرَاءِ، وَأَنَا آتِي إِلَى بَيْتِي لِآكُلَ وَأَشْرَبَ وَأَضْطَجِعَ مَعَ امْرَأَتِي! وَحَيَاتِكَ وَحَيَاةِ نَفْسِكَ لاَ أَفْعَلُ هَذَا الأَمْرَ».
\par 12 فَقَالَ دَاوُدُ لِأُورِيَّا: «أَقِمْ هُنَا الْيَوْمَ أَيْضاً، وَغَداً أُطْلِقُكَ». فَأَقَامَ أُورِيَّا فِي أُورُشَلِيمَ ذَلِكَ الْيَوْمَ وَغَدَهُ.
\par 13 وَدَعَاهُ دَاوُدُ فَأَكَلَ أَمَامَهُ وَشَرِبَ وَأَسْكَرَهُ. وَخَرَجَ عِنْدَ الْمَسَاءِ لِيَضْطَجِعَ فِي مَضْجَعِهِ مَعَ عَبِيدِ سَيِّدِهِ، وَإِلَى بَيْتِهِ لَمْ يَنْزِلْ.
\par 14 وَفِي الصَّبَاحِ كَتَبَ دَاوُدُ مَكْتُوباً إِلَى يُوآبَ وَأَرْسَلَهُ بِيَدِ أُورِيَّا.
\par 15 وَكَتَبَ فِي الْمَكْتُوبِ يَقُولُ: «اجْعَلُوا أُورِيَّا فِي وَجْهِ الْحَرْبِ الشَّدِيدَةِ، وَارْجِعُوا مِنْ وَرَائِهِ فَيُضْرَبَ وَيَمُوتَ».
\par 16 وَكَانَ فِي مُحَاصَرَةِ يُوآبَ الْمَدِينَةَ أَنَّهُ جَعَلَ أُورِيَّا فِي الْمَوْضِعِ الَّذِي عَلِمَ أَنَّ رِجَالَ الْبَأْسِ فِيهِ.
\par 17 فَخَرَجَ رِجَالُ الْمَدِينَةِ وَحَارَبُوا يُوآبَ، فَسَقَطَ بَعْضُ الشَّعْبِ مِنْ عَبِيدِ دَاوُدَ، وَمَاتَ أُورِيَّا الْحِثِّيُّ أَيْضاً.
\par 18 فَأَرْسَلَ يُوآبُ وَأَخْبَرَ دَاوُدَ بِجَمِيعِ أُمُورِ الْحَرْبِ.
\par 19 وَأَوْصَى الرَّسُولَ: «عِنْدَمَا تَفْرَغُ مِنَ الْكَلاَمِ مَعَ الْمَلِكِ عَنْ جَمِيعِ أُمُورِ الْحَرْبِ،
\par 20 فَإِنِ اشْتَعَلَ غَضَبُ الْمَلِكِ، وَقَالَ لَكَ: لِمَاذَا دَنَوْتُمْ مِنَ الْمَدِينَةِ لِلْقِتَالِ؟ أَمَا عَلِمْتُمْ أَنَّهُمْ يَرْمُونَ مِنْ عَلَى السُّورِ؟
\par 21 مَنْ قَتَلَ أَبِيمَالِكَ بْنَ يَرُبُّوشَثَ؟ أَلَمْ تَرْمِهِ امْرَأَةٌ بِقِطْعَةِ رَحًى مِنْ عَلَى السُّورِ فَمَاتَ فِي تَابَاصَ؟ لِمَاذَا دَنَوْتُمْ مِنَ السُّورِ؟ فَقُلْ: قَدْ مَاتَ عَبْدُكَ أُورِيَّا الْحِثِّيُّ أَيْضاً».
\par 22 فَذَهَبَ الرَّسُولُ وَدَخَلَ وَأَخْبَرَ دَاوُدَ بِكُلِّ مَا أَرْسَلَهُ فِيهِ يُوآبُ.
\par 23 وَقَالَ الرَّسُولُ لِدَاوُدَ: «قَدْ تَجَبَّرَ عَلَيْنَا الْقَوْمُ وَخَرَجُوا إِلَيْنَا إِلَى الْحَقْلِ فَكُنَّا عَلَيْهِمْ إِلَى مَدْخَلِ الْبَابِ.
\par 24 فَرَمَى الرُّمَاةُ عَبِيدَكَ مِنْ عَلَى السُّورِ، فَمَاتَ الْبَعْضُ مِنْ عَبِيدِ الْمَلِكِ، وَمَاتَ عَبْدُكَ أُورِيَّا الْحِثِّيُّ أَيْضاً».
\par 25 فَقَالَ دَاوُدُ لِلرَّسُولِ: « هَكَذَا تَقُولُ لِيُوآبَ: لاَ يَسُؤْ فِي عَيْنَيْكَ هَذَا الأَمْرُ، لأَنَّ السَّيْفَ يَأْكُلُ هَذَا وَذَاكَ. شَدِّدْ قِتَالَكَ عَلَى الْمَدِينَةِ وَأَخْرِبْهَا. وَشَدِّدْهُ».
\par 26 فَلَمَّا سَمِعَتِ امْرَأَةُ أُورِيَّا أَنَّهُ قَدْ مَاتَ أُورِيَّا رَجُلُهَا نَدَبَتْ بَعْلَهَا.
\par 27 وَلَمَّا مَضَتِ الْمَنَاحَةُ أَرْسَلَ دَاوُدُ وَضَمَّهَا إِلَى بَيْتِهِ، وَصَارَتْ لَهُ امْرَأَةً وَوَلَدَتْ لَهُ ابْناً. وَأَمَّا الأَمْرُ الَّذِي فَعَلَهُ دَاوُدُ فَقَبُحَ فِي عَيْنَيِ الرَّبِّ.

\chapter{12}

\par 1 فَأَرْسَلَ الرَّبُّ نَاثَانَ إِلَى دَاوُدَ. فَجَاءَ إِلَيْهِ وَقَالَ لَهُ: «كَانَ رَجُلاَنِ فِي مَدِينَةٍ وَاحِدَةٍ، وَاحِدٌ مِنْهُمَا غَنِيٌّ وَالآخَرُ فَقِيرٌ.
\par 2 وَكَانَ لِلْغَنِيِّ غَنَمٌ وَبَقَرٌ كَثِيرَةٌ جِدّاً.
\par 3 وَأَمَّا الْفَقِيرُ فَلَمْ يَكُنْ لَهُ شَيْءٌ إِلاَّ نَعْجَةٌ وَاحِدَةٌ صَغِيرَةٌ قَدِ اقْتَنَاهَا وَرَبَّاهَا وَكَبِرَتْ مَعَهُ وَمَعَ بَنِيهِ جَمِيعاً. تَأْكُلُ مِنْ لُقْمَتِهِ وَتَشْرَبُ مِنْ كَأْسِهِ وَتَنَامُ فِي حِضْنِهِ، وَكَانَتْ لَهُ كَابْنَةٍ.
\par 4 فَجَاءَ ضَيْفٌ إِلَى الرَّجُلِ الْغَنِيِّ فَعَفَا أَنْ يَأْخُذَ مِنْ غَنَمِهِ وَمِنْ بَقَرِهِ لِيُهَيِّئَ لِلضَّيْفِ الَّذِي جَاءَ إِلَيْهِ، فَأَخَذَ نَعْجَةَ الرَّجُلِ الْفَقِيرِ وَهَيَّأَ لِلرَّجُلِ الَّذِي جَاءَ إِلَيْهِ».
\par 5 فَحَمِيَ غَضَبُ دَاوُدَ عَلَى الرَّجُلِ جِدّاً، وَقَالَ لِنَاثَانَ: «حَيٌّ هُوَ الرَّبُّ إِنَّهُ يُقْتَلُ الرَّجُلُ الْفَاعِلُ ذَلِكَ،
\par 6 وَيَرُدُّ النَّعْجَةَ أَرْبَعَةَ أَضْعَافٍ لأَنَّهُ فَعَلَ هَذَا الأَمْرَ وَلأَنَّهُ لَمْ يُشْفِقْ».
\par 7 فَقَالَ نَاثَانُ لِدَاوُدَ: «أَنْتَ هُوَ الرَّجُلُ! هَكَذَا قَالَ الرَّبُّ إِلَهُ إِسْرَائِيلَ: أَنَا مَسَحْتُكَ مَلِكاً عَلَى إِسْرَائِيلَ وَأَنْقَذْتُكَ مِنْ يَدِ شَاوُلَ
\par 8 وَأَعْطَيْتُكَ بَيْتَ سَيِّدِكَ وَنِسَاءَ سَيِّدِكَ فِي حِضْنِكَ، وَأَعْطَيْتُكَ بَيْتَ إِسْرَائِيلَ وَيَهُوذَا. وَإِنْ كَانَ ذَلِكَ قَلِيلاً كُنْتُ أَزِيدُ لَكَ كَذَا وَكَذَا.
\par 9 لِمَاذَا احْتَقَرْتَ كَلاَمَ الرَّبِّ لِتَعْمَلَ الشَّرَّ فِي عَيْنَيْهِ؟ قَدْ قَتَلْتَ أُورِيَّا الْحِثِّيَّ بِالسَّيْفِ، وَأَخَذْتَ امْرَأَتَهُ لَكَ امْرَأَةً، وَإِيَّاهُ قَتَلْتَ بِسَيْفِ بَنِي عَمُّونَ.
\par 10 وَالآنَ لاَ يُفَارِقُ السَّيْفُ بَيْتَكَ إِلَى الأَبَدِ، لأَنَّكَ احْتَقَرْتَنِي وَأَخَذْتَ امْرَأَةَ أُورِيَّا الْحِثِّيِّ لِتَكُونَ لَكَ امْرَأَةً.
\par 11 هَكَذَا قَالَ الرَّبُّ: هَئَنَذَا أُقِيمُ عَلَيْكَ الشَّرَّ مِنْ بَيْتِكَ، وَآخُذُ نِسَاءَكَ أَمَامَ عَيْنَيْكَ وَأُعْطِيهِنَّ لِقَرِيبِكَ، فَيَضْطَجِعُ مَعَ نِسَائِكَ فِي عَيْنِ هَذِهِ الشَّمْسِ.
\par 12 لأَنَّكَ أَنْتَ فَعَلْتَ بِالسِّرِّ وَأَنَا أَفْعَلُ هَذَا الأَمْرَ قُدَّامَ جَمِيعِ إِسْرَائِيلَ وَقُدَّامَ الشَّمْسِ».
\par 13 فَقَالَ دَاوُدُ لِنَاثَانَ: «قَدْ أَخْطَأْتُ إِلَى الرَّبِّ». فَقَالَ نَاثَانُ لِدَاوُدَ: «الرَّبُّ أَيْضاً قَدْ نَقَلَ عَنْكَ خَطِيَّتَكَ. لاَ تَمُوتُ.
\par 14 غَيْرَ أَنَّهُ مِنْ أَجْلِ أَنَّكَ قَدْ جَعَلْتَ بِهَذَا الأَمْرِ أَعْدَاءَ الرَّبِّ يَشْمَتُونَ فَالاِبْنُ الْمَوْلُودُ لَكَ يَمُوتُ».
\par 15 وَذَهَبَ نَاثَانُ إِلَى بَيْتِهِ. وَضَرَبَ الرَّبُّ الْوَلَدَ الَّذِي وَلَدَتْهُ امْرَأَةُ أُورِيَّا لِدَاوُدَ فَثَقِلَ.
\par 16 فَسَأَلَ دَاوُدُ اللَّهَ مِنْ أَجْلِ الصَّبِيِّ، وَصَامَ دَاوُدُ صَوْماً، وَدَخَلَ وَبَاتَ مُضْطَجِعاً عَلَى الأَرْضِ.
\par 17 فَقَامَ شُيُوخُ بَيْتِهِ عَلَيْهِ لِيُقِيمُوهُ عَنِ الأَرْضِ فَلَمْ يَشَأْ، وَلَمْ يَأْكُلْ مَعَهُمْ خُبْزاً.
\par 18 وَكَانَ فِي الْيَوْمِ السَّابِعِ أَنَّ الْوَلَدَ مَاتَ، فَخَافَ عَبِيدُ دَاوُدَ أَنْ يُخْبِرُوهُ بِأَنَّ الْوَلَدَ قَدْ مَاتَ لأَنَّهُمْ قَالُوا: «هُوَذَا لَمَّا كَانَ الْوَلَدُ حَيّاً كَلَّمْنَاهُ فَلَمْ يَسْمَعْ لِصَوْتِنَا. فَكَيْفَ نَقُولُ لَهُ قَدْ مَاتَ الْوَلَدُ؟ يَعْمَلُ أَشَرَّ!».
\par 19 وَرَأَى دَاوُدُ عَبِيدَهُ يَتَنَاجُونَ، فَفَطِنَ دَاوُدُ أَنَّ الْوَلَدَ قَدْ مَاتَ. فَقَالَ دَاوُدُ لِعَبِيدِهِ: «هَلْ مَاتَ الْوَلَدُ؟» فَقَالُوا: «مَاتَ».
\par 20 فَقَامَ دَاوُدُ عَنِ الأَرْضِ وَاغْتَسَلَ وَادَّهَنَ وَبَدَّلَ ثِيَابَهُ وَدَخَلَ بَيْتَ الرَّبِّ وَسَجَدَ، ثُمَّ جَاءَ إِلَى بَيْتِهِ وَطَلَبَ فَوَضَعُوا لَهُ خُبْزاً فَأَكَلَ.
\par 21 فَقَالَ لَهُ عَبِيدُهُ: «مَا هَذَا الأَمْرُ الَّذِي فَعَلْتَ؟ لَمَّا كَانَ الْوَلَدُ حَيّاً صُمْتَ وَبَكَيْتَ، وَلَمَّا مَاتَ الْوَلَدُ قُمْتَ وَأَكَلْتَ خُبْزاً».
\par 22 فَقَالَ: «لَمَّا كَانَ الْوَلَدُ حَيّاً صُمْتُ وَبَكَيْتُ لأَنِّي قُلْتُ: مَنْ يَعْلَمُ؟ رُبَّمَا يَرْحَمُنِي الرَّبُّ وَيَحْيَا الْوَلَدُ.
\par 23 وَالآنَ قَدْ مَاتَ، فَلِمَاذَا أَصُومُ؟ هَلْ أَقْدِرُ أَنْ أَرُدَّهُ بَعْدُ؟ أَنَا ذَاهِبٌ إِلَيْهِ وَأَمَّا هُوَ فَلاَ يَرْجِعُ إِلَيَّ».
\par 24 وَعَزَّى دَاوُدُ بَثْشَبَعَ امْرَأَتَهُ وَدَخَلَ إِلَيْهَا وَاضْطَجَعَ مَعَهَا فَوَلَدَتِ ابْناً، فَدَعَا اسْمَهُ سُلَيْمَانَ، وَالرَّبُّ أَحَبَّهُ،
\par 25 وَأَرْسَلَ بِيَدِ نَاثَانَ النَّبِيِّ وَدَعَا اسْمَهُ «يَدِيدِيَّا» مِنْ أَجْلِ الرَّبِّ.
\par 26 وَحَارَبَ يُوآبُ رَبَّةَ بَنِي عَمُّونَ وَأَخَذَ مَدِينَةَ الْمَمْلَكَةِ.
\par 27 وَأَرْسَلَ يُوآبُ رُسُلاً إِلَى دَاوُدَ يَقُولُ: «قَدْ حَارَبْتُ رَبَّةَ وَأَخَذْتُ أَيْضاً مَدِينَةَ الْمِيَاهِ.
\par 28 فَالآنَ اجْمَعْ بَقِيَّةَ الشَّعْبِ وَانْزِلْ عَلَى الْمَدِينَةِ وَخُذْهَا لِئَلاَّ آخُذَ أَنَا الْمَدِينَةَ فَيُدْعَى بِاسْمِي عَلَيْهَا».
\par 29 فَجَمَعَ دَاوُدُ كُلَّ الشَّعْبِ وَذَهَبَ إِلَى رَبَّةَ وَحَارَبَهَا وَأَخَذَهَا.
\par 30 وَأَخَذَ تَاجَ مَلِكِهِمْ عَنْ رَأْسِهِ وَوَزْنُهُ وَزْنَةٌ مِنَ الذَّهَبِ مَعَ حَجَرٍ كَرِيمٍ، وَكَانَ عَلَى رَأْسِ دَاوُدَ. وَأَخْرَجَ غَنِيمَةَ الْمَدِينَةِ كَثِيرَةً جِدّاً.
\par 31 وَأَخْرَجَ الشَّعْبَ الَّذِي فِيهَا وَوَضَعَهُمْ تَحْتَ مَنَاشِيرَ وَنَوَارِجِ حَدِيدٍ وَفُؤُوسِ حَدِيدٍ وَأَمَرَّهُمْ فِي أَتُونِ الآجُرِّ، وَهَكَذَا صَنَعَ بِجَمِيعِ مُدُنِ بَنِي عَمُّونَ. ثُمَّ رَجَعَ دَاوُدُ وَجَمِيعُ الشَّعْبِ إِلَى أُورُشَلِيمَ.

\chapter{13}

\par 1 وَجَرَى بَعْدَ ذَلِكَ أَنَّهُ كَانَ لأَبْشَالُومَ بْنِ دَاوُدَ أُخْتٌ جَمِيلَةٌ اسْمُهَا ثَامَارُ، فَأَحَبَّهَا أَمْنُونُ بْنُ دَاوُدَ.
\par 2 وَأُحْصِرَ أَمْنُونُ لِلسُّقْمِ مِنْ أَجْلِ ثَامَارَ أُخْتِهِ لأَنَّهَا كَانَتْ عَذْرَاءَ، وَعَسُرَ فِي عَيْنَيْ أَمْنُونَ أَنْ يَفْعَلَ لَهَا شَيْئاً.
\par 3 وَكَانَ لأَمْنُونَ صَاحِبٌ اسْمُهُ يُونَادَابُ بْنُ شَمْعَى أَخِي دَاوُدَ. وَكَانَ يُونَادَابُ رَجُلاً حَكِيماً جِدّاً.
\par 4 فَقَالَ لَهُ: «لِمَاذَا يَا ابْنَ الْمَلِكِ أَنْتَ ضَعِيفٌ هَكَذَا مِنْ صَبَاحٍ إِلَى صَبَاحٍ؟ أَمَا تُخْبِرُنِي؟» فَقَالَ لَهُ أَمْنُونُ: «إِنِّي أُحِبُّ ثَامَارَ أُخْتَ أَبْشَالُومَ أَخِي».
\par 5 فَقَالَ يُونَادَابُ: «اضْطَجِعْ عَلَى سَرِيرِكَ وَتَمَارَضْ. وَإِذَا جَاءَ أَبُوكَ لِيَرَاكَ فَقُلْ لَهُ: دَعْ ثَامَارَ أُخْتِي فَتَأْتِيَ وَتُطْعِمَنِي خُبْزاً وَتَعْمَلَ أَمَامِي الطَّعَامَ لأَرَى فَآكُلَ مِنْ يَدِهَا».
\par 6 فَاضْطَجَعَ أَمْنُونُ وَتَمَارَضَ، فَجَاءَ الْمَلِكُ لِيَرَاهُ. فَقَالَ أَمْنُونُ لِلْمَلِكِ: «دَعْ ثَامَارَ أُخْتِي فَتَأْتِيَ وَتَصْنَعَ أَمَامِي كَعْكَتَيْنِ فَآكُلَ مِنْ يَدِهَا».
\par 7 فَأَرْسَلَ دَاوُدُ إِلَى ثَامَارَ إِلَى الْبَيْتِ قَائِلاً: «اذْهَبِي إِلَى بَيْتِ أَمْنُونَ أَخِيكِ وَاعْمَلِي لَهُ طَعَاماً».
\par 8 فَذَهَبَتْ ثَامَارُ إِلَى بَيْتِ أَمْنُونَ أَخِيهَا وَهُوَ مُضْطَجِعٌ. وَأَخَذَتِ الْعَجِينَ وَعَجَنَتْ وَعَمِلَتْ كَعْكاً أَمَامَهُ وَخَبَزَتِ الْكَعْكَ
\par 9 وَأَخَذَتِ الْمِقْلاَةَ وَسَكَبَتْ أَمَامَهُ، فَأَبَى أَنْ يَأْكُلَ. وَقَالَ أَمْنُونُ: «أَخْرِجُوا كُلَّ إِنْسَانٍ عَنِّي». فَخَرَجَ كُلُّ إِنْسَانٍ عَنْهُ.
\par 10 ثُمَّ قَالَ أَمْنُونُ لِثَامَارَ: «اِيتِي بِالطَّعَامِ إِلَى الْمَخْدَعِ فَآكُلَ مِنْ يَدِكِ». فَأَخَذَتْ ثَامَارُ الْكَعْكَ الَّذِي عَمِلَتْهُ وَأَتَتْ بِهِ أَمْنُونَ أَخَاهَا إِلَى الْمَخْدَعِ.
\par 11 وَقَدَّمَتْ لَهُ لِيَأْكُلَ، فَأَمْسَكَهَا وَقَالَ لَهَا: «تَعَالَيِ اضْطَجِعِي مَعِي يَا أُخْتِي».
\par 12 فَقَالَتْ لَهُ: «لاَ يَا أَخِي، لاَ تُذِلَّنِي لأَنَّهُ لاَ يُفْعَلُ هَكَذَا فِي إِسْرَائِيلَ. لاَ تَعْمَلْ هَذِهِ الْقَبَاحَةَ.
\par 13 أَمَّا أَنَا فَأَيْنَ أَذْهَبُ بِعَارِي، وَأَمَّا أَنْتَ فَتَكُونُ كَوَاحِدٍ مِنَ السُّفَهَاءِ فِي إِسْرَائِيلَ! وَالآنَ كَلِّمِ الْمَلِكَ لأَنَّهُ لاَ يَمْنَعُنِي مِنْكَ».
\par 14 فَلَمْ يَشَأْ أَنْ يَسْمَعَ لِصَوْتِهَا، بَلْ تَمَكَّنَ مِنْهَا وَقَهَرَهَا وَاضْطَجَعَ مَعَهَا.
\par 15 ثُمَّ أَبْغَضَهَا أَمْنُونُ بُغْضَةً شَدِيدَةً جِدّاً حَتَّى إِنَّ الْبُغْضَةَ الَّتِي أَبْغَضَهَا إِيَّاهَا كَانَتْ أَشَدَّ مِنَ الْمَحَبَّةِ الَّتِي أَحَبَّهَا إِيَّاهَا. وَقَالَ لَهَا أَمْنُونُ: «قُومِي انْطَلِقِي!»
\par 16 فَقَالَتْ لَهُ: «لاَ سَبَبَ! هَذَا الشَّرُّ بِطَرْدِكَ إِيَّايَ هُوَ أَعْظَمُ مِنَ الآخَرِ الَّذِي عَمِلْتَهُ بِي». فَلَمْ يَشَأْ أَنْ يَسْمَعَ لَهَا،
\par 17 بَلْ دَعَا غُلاَمَهُ الَّذِي كَانَ يَخْدِمُهُ وَقَالَ: «اطْرُدْ هَذِهِ عَنِّي خَارِجاً وَأَقْفِلِ الْبَابَ وَرَاءَهَا».
\par 18 وَكَانَ عَلَيْهَا ثَوْبٌ مُلوَّنٌ، لأَنَّ بَنَاتِ الْمَلِكِ الْعَذَارَى كُنَّ يَلْبِسْنَ جُبَّاتٍ مِثْلَ هَذِهِ. فَأَخْرَجَهَا خَادِمُهُ إِلَى الْخَارِجِ وَأَقْفَلَ الْبَابَ وَرَاءَهَا.
\par 19 فَجَعَلَتْ ثَامَارُ رَمَاداً عَلَى رَأْسِهَا، وَمَزَّقَتِ الثَّوْبَ الْمُلَوَّنَ الَّذِي عَلَيْهَا، وَوَضَعَتْ يَدَهَا عَلَى رَأْسِهَا وَكَانَتْ تَذْهَبُ صَارِخَةً.
\par 20 فَقَالَ لَهَا أَبْشَالُومُ أَخُوهَا: «هَلْ كَانَ أَمْنُونُ أَخُوكِ مَعَكِ؟ فَالآنَ يَا أُخْتِي اسْكُتِي. أَخُوكِ هُوَ. لاَ تَضَعِي قَلْبَكِ عَلَى هَذَا الأَمْرِ». فَأَقَامَتْ ثَامَارُ مُسْتَوْحِشَةً فِي بَيْتِ أَبْشَالُومَ أَخِيهَا.
\par 21 وَلَمَّا سَمِعَ الْمَلِكُ دَاوُدُ بِجَمِيعِ هَذِهِ الأُمُورِ اغْتَاظَ جِدّاً.
\par 22 وَلَمْ يُكَلِّمْ أَبْشَالُومُ أَمْنُونَ بِشَرٍّ وَلاَ بِخَيْرٍ، لأَنَّ أَبْشَالُومَ أَبْغَضَ أَمْنُونَ مِنْ أَجْلِ أَنَّهُ أَذَلَّ ثَامَارَ أُخْتَهُ.
\par 23 وَكَانَ بَعْدَ سَنَتَيْنِ مِنَ الزَّمَانِ أَنَّهُ كَانَ لأَبْشَالُومَ جَزَّازُونَ فِي بَعْلَ حَاصُورَ الَّتِي عِنْدَ أَفْرَايِمَ. فَدَعَا أَبْشَالُومُ جَمِيعَ بَنِي الْمَلِكِ.
\par 24 وَجَاءَ أَبْشَالُومُ إِلَى الْمَلِكِ وَقَالَ: «هُوَذَا لِعَبْدِكَ جَزَّازُونَ. فَلْيَذْهَبِ الْمَلِكُ وَعَبِيدُهُ مَعَ عَبْدِكَ».
\par 25 فَقَالَ الْمَلِكُ لأَبْشَالُومَ: «لاَ يَا ابْنِي. لاَ نَذْهَبْ كُلُّنَا لِئَلاَّ نُثَقِّلَ عَلَيْكَ». فَأَلَحَّ عَلَيْهِ، فَلَمْ يَشَأْ أَنْ يَذْهَبَ بَلْ بَارَكَهُ.
\par 26 فَقَالَ أَبْشَالُومُ: «إِذاً دَعْ أَخِي أَمْنُونَ يَذْهَبْ مَعَنَا». فَقَالَ الْمَلِكُ: «لِمَاذَا يَذْهَبُ مَعَكَ؟»
\par 27 فَأَلَحَّ عَلَيْهِ أَبْشَالُومُ، فَأَرْسَلَ مَعَهُ أَمْنُونَ وَجَمِيعَ بَنِي الْمَلِكِ.
\par 28 فَأَوْصَى أَبْشَالُومُ غِلْمَانَهُ: «انْظُرُوا. مَتَى طَابَ قَلْبُ أَمْنُونَ بِالْخَمْرِ وَقُلْتُ لَكُمُ اضْرِبُوا أَمْنُونَ فَاقْتُلُوهُ. لاَ تَخَافُوا. أَلَيْسَ أَنِّي أَنَا أَمَرْتُكُمْ؟ فَتَشَدَّدُوا وَكُونُوا ذَوِي بَأْسٍ».
\par 29 فَفَعَلَ غِلْمَانُ أَبْشَالُومَ بِأَمْنُونَ كَمَا أَمَرَ أَبْشَالُومُ. فَقَامَ جَمِيعُ بَنِي الْمَلِكِ وَرَكِبُوا كُلُّ وَاحِدٍ عَلَى بَغْلِهِ وَهَرَبُوا.
\par 30 وَفِيمَا هُمْ فِي الطَّرِيقِ وَصَلَ الْخَبَرُ إِلَى دَاوُدَ: «قَدْ قَتَلَ أَبْشَالُومُ جَمِيعَ بَنِي الْمَلِكِ، وَلَمْ يَتَبَقَّ مِنْهُمْ أَحَدٌ».
\par 31 فَقَامَ الْمَلِكُ وَمَزَّقَ ثِيَابَهُ وَاضْطَجَعَ عَلَى الأَرْضِ وَجَمِيعُ عَبِيدِهِ وَاقِفُونَ وَثِيَابُهُمْ مُمَزَّقَةٌ.
\par 32 فَقَالبَ يُونَادَابُ بْنُ شَمْعَى أَخِي دَاوُدَ: «لاَ يَظُنَّ سَيِّدِي أَنَّهُمْ قَتَلُوا جَمِيعَ الْفِتْيَانِ بَنِي الْمَلِكِ. إِنَّمَا أَمْنُونُ وَحْدَهُ مَاتَ، لأَنَّ ذَلِكَ قَدْ وُضِعَ عِنْدَ أَبْشَالُومَ مُنْذُ يَوْمَ أَذَلَّ ثَامَارَ أُخْتَهُ.
\par 33 وَالآنَ لاَ يَضَعَنَّ سَيِّدِي الْمَلِكُ فِي قَلْبِهِ شَيْئاً قَائِلاً إِنَّ جَمِيعَ بَنِي الْمَلِكِ قَدْ مَاتُوا. إِنَّمَا أَمْنُونُ وَحْدَهُ مَاتَ».
\par 34 وَهَرَبَ أَبْشَالُومُ. وَرَفَعَ الرَّقِيبُ طَرْفَهُ وَنَظَرَ وَإِذَا بِشَعْبٍ كَثِيرٍ يَسِيرُونَ عَلَى الطَّرِيقِ وَرَاءَهُ بِجَانِبِ الْجَبَلِ.
\par 35 فَقَالَ يُونَادَابُ لِلْمَلِكِ: «هُوَذَا بَنُو الْمَلِكِ قَدْ جَاءُوا. كَمَا قَالَ عَبْدُكَ كَذَلِكَ صَارَ».
\par 36 وَلَمَّا فَرَغَ مِنَ الْكَلاَمِ إِذَا بِبَنِي الْمَلِكِ قَدْ جَاءُوا، وَرَفَعُوا أَصْوَاتَهُمْ وَبَكُوا وَكَذَلِكَ بَكَى الْمَلِكُ وَعَبِيدُهُ بُكَاءً عَظِيماً جِدّاً.
\par 37 فَهَرَبَ أَبْشَالُومُ وَذَهَبَ إِلَى تَلْمَايَ بْنِ عَمِّيهُودَ مَلِكِ جَشُورَ. وَنَاحَ دَاوُدُ عَلَى ابْنِهِ الأَيَّامَ كُلَّهَا.
\par 38 وَهَرَبَ أَبْشَالُومُ وَذَهَبَ إِلَى جَشُورَ وَكَانَ هُنَاكَ ثَلاَثَ سِنِينَ.
\par 39 وَكَانَ دَاوُدُ يَتُوقُ إِلَى الْخُرُوجِ إِلَى أَبْشَالُومَ لأَنَّهُ تَعَزَّى عَنْ أَمْنُونَ حَيْثُ إِنَّهُ مَاتَ.

\chapter{14}

\par 1 وَعَلِمَ يُوآبُ ابْنُ صَرُويَةَ أَنَّ قَلْبَ الْمَلِكِ عَلَى أَبْشَالُومَ.
\par 2 فَأَرْسَلَ يُوآبُ إِلَى تَقُوعَ وَأَخَذَ مِنْ هُنَاكَ امْرَأَةً حَكِيمَةً وَقَالَ لَهَا: «تَظَاهَرِي بِالْحُزْنِ وَالْبَسِي ثِيَابَ الْحُزْنِ، وَلاَ تَدَّهِنِي بِزَيْتٍ بَلْ كُونِي كَامْرَأَةٍ لَهَا أَيَّامٌ كَثِيرَةٌ وَهِيَ تَنُوحُ عَلَى مَيِّتٍ.
\par 3 وَادْخُلِي إِلَى الْمَلِكِ وَكَلِّمِيهِ بِهَذَا الْكَلاَمِ». وَجَعَلَ يُوآبُ الْكَلاَمَ فِي فَمِهَا.
\par 4 وَكَلَّمَتِ الْمَرْأَةُ التَّقُوعِيَّةُ الْمَلِكَ وَخَرَّتْ عَلَى وَجْهِهَا إِلَى الأَرْضِ وَسَجَدَتْ وَقَالَتْ: «أَعِنْ أَيُّهَا الْمَلِكُ!».
\par 5 فَقَالَ لَهَا الْمَلِكُ: «مَا بَالُكِ؟» فَقَالَتْ: «إِنِّي أَرْمَلَةٌ قَدْ مَاتَ رَجُلِي.
\par 6 وَلِجَارِيَتِكَ ابْنَانِ، فَتَخَاصَمَا فِي الْحَقْلِ وَلَيْسَ مَنْ يَفْصِلُ بَيْنَهُمَا، فَضَرَبَ أَحَدُهُمَا الآخَرَ وَقَتَلَهُ.
\par 7 وَهُوَذَا الْعَشِيرَةُ كُلُّهَا قَدْ قَامَتْ عَلَى جَارِيَتِكَ وَقَالُوا: سَلِّمِي ضَارِبَ أَخِيهِ لِنَقْتُلَهُ بِنَفْسِ أَخِيهِ الَّذِي قَتَلَهُ، فَنُهْلِكَ الْوَارِثَ أَيْضاً. فَيُطْفِئُونَ جَمْرَتِي الَّتِي بَقِيَتْ، وَلاَ يَتْرُكُونَ لِرَجُلِي اسْماً وَلاَ بَقِيَّةً عَلَى وَجْهِ الأَرْضِ».
\par 8 فَقَالَ الْمَلِكُ لِلْمَرْأَةِ: «اذْهَبِي إِلَى بَيْتِكِ وَأَنَا أُوصِي فِيكِ».
\par 9 فَقَالَتِ الْمَرْأَةُ التَّقُوعِيَّةُ لِلْمَلِكِ: «عَلَيَّ الإِثْمُ يَا سَيِّدِي الْمَلِكَ وَعَلَى بَيْتِ أَبِي، وَالْمَلِكُ وَكُرْسِيُّهُ نَقِيَّانِ».
\par 10 فَقَالَ الْمَلِكُ: «إِذَا كَلَّمَكِ أَحَدٌ فَأْتِي بِهِ إِلَيَّ فَلاَ يَعُودَ يَمَسُّكِ بَعْدُ».
\par 11 فَقَالَتِ: «اذْكُرْ أَيُّهَا الْمَلِكُ الرَّبَّ إِلَهَكَ حَتَّى لاَ يُكَثِّرَ وَلِيُّ الدَّمِ الْقَتْلَ لِئَلاَّ يُهْلِكُوا ابْنِي». فَقَالَ: «حَيٌّ هُوَ الرَّبُّ إِنَّهُ لاَ تَسْقُطُ شَعْرَةٌ مِنْ شَعْرِ ابْنِكِ إِلَى الأَرْضِ».
\par 12 فَقَالَتِ الْمَرْأَةُ: «لِتَتَكَلَّمْ جَارِيَتُكَ كَلِمَةً إِلَى سَيِّدِي الْمَلِكِ». فَقَالَ: «تَكَلَّمِي»
\par 13 فَقَالَتِ الْمَرْأَةُ: «وَلِمَاذَا افْتَكَرْتَ بِمِثْلِ هَذَا الأَمْرِ عَلَى شَعْبِ اللَّهِ؟ وَيَتَكَلَّمُ الْمَلِكُ بِهَذَا الْكَلاَمِ كَمُذْنِبٍ بِمَا أَنَّ الْمَلِكَ لاَ يَرُدُّ مَنْفِيَّهُ.
\par 14 لأَنَّهُ لاَ بُدَّ أَنْ نَمُوتَ وَنَكُونَ كَالْمَاءِ الْمَُهْرَاقِ عَلَى الأَرْضِ الَّذِي لاَ يُجْمَعُ أَيْضاً. وَلاَ يَنْزِعُ اللَّهُ نَفْساً بَلْ يُفَكِّرُ أَفْكَاراً حَتَّى لاَ يُطْرَدَ عَنْهُ مَنْفِيُّهُ.
\par 15 وَالآنَ حَيْثُ إِنِّي جِئْتُ لِأُكَلِّمَ الْمَلِكَ سَيِّدِي بِهَذَا الأَمْرِ، لأَنَّ الشَّعْبَ أَخَافَنِي، فَقَالَتْ جَارِيَتُكَ أُكَلِّمُ الْمَلِكَ لَعَلَّ الْمَلِكَ يَفْعَلُ كَقَوْلِ أَمَتِهِ.
\par 16 لأَنَّ الْمَلِكَ يَسْمَعُ لِيُنْقِذَ أَمَتَهُ مِنْ يَدِ الرَّجُلِ الَّذِي يُرِيدُ أَنْ يُهْلِكَنِي أَنَا وَابْنِي مَعاً مِنْ نَصِيبِ اللَّهِ.
\par 17 فَقَالَتْ جَارِيَتُكَ لِيَكُنْ كَلاَمُ سَيِّدِي الْمَلِكِ عَزَاءً، لأَنَّهُ سَيِّدِي الْمَلِكُ إِنَّمَا هُوَ كَمَلاَكِ اللَّهِ لِفَهْمِ الْخَيْرِ وَالشَّرِّ، وَالرَّبُّ إِلَهُكَ يَكُونُ مَعَكَ».
\par 18 فَقَالَ الْمَلِكُ لِلْمَرْأَةِ: «لاَ تَكْتُمِي عَنِّي أَمْراً أَسْأَلُكِ عَنْهُ». فَقَالَتِ الْمَرْأَةُ: «لِيَتَكَلَّمْ سَيِّدِي الْمَلِكُ».
\par 19 فَقَالَ الْمَلِكُ: «هَلْ يَدُ يُوآبَ مَعَكِ فِي هَذَا كُلِّهِ؟» فَأَجَابَتِ الْمَرْأَةُ: «حَيَّةٌ هِيَ نَفْسُكَ يَا سَيِّدِي الْمَلِكَ، لاَ يُحَادُ يَمِيناً أَوْ يَسَاراً عَنْ كُلِّ مَا تَكَلَّمَ بِهِ سَيِّدِي الْمَلِكُ، لأَنَّ عَبْدَكَ يُوآبَ هُوَ أَوْصَانِي وَهُوَ وَضَعَ فِي فَمِ جَارِيَتِكَ كُلَّ هَذَا الْكَلاَمِ.
\par 20 لأَجْلِ تَحْوِيلِ وَجْهِ الْكَلاَمِ فَعَلَ عَبْدُكَ يُوآبُ هَذَا الأَمْرَ، وَسَيِّدِي حَكِيمٌ كَحِكْمَةِ مَلاَكِ اللَّهِ لِيَعْلَمَ كُلَّ مَا فِي الأَرْضِ».
\par 21 فَقَالَ الْمَلِكُ لِيُوآبَ: «هَئَنَذَ قَدْ فَعَلْتُ هَذَا الأَمْرَ، فَاذْهَبْ رُدَّ الْفَتَى أَبْشَالُومَ».
\par 22 فَسَقَطَ يُوآبُ عَلَى وَجْهِهِ إِلَى الأَرْضِ وَسَجَدَ وَبَارَكَ الْمَلِكَ، وَقَالَ يُوآبُ: «الْيَوْمَ عَلِمَ عَبْدُكَ أَنِّي قَدْ وَجَدْتُ نِعْمَةً فِي عَيْنَيْكَ يَا سَيِّدِي الْمَلِكَ إِذْ فَعَلَ الْمَلِكُ قَوْلَ عَبْدِهِ».
\par 23 ثُمَّ قَامَ يُوآبُ وَذَهَبَ إِلَى جَشُورَ وَأَتَى بِأَبْشَالُومَ إِلَى أُورُشَلِيمَ.
\par 24 فَقَالَ الْمَلِكُ: «لِيَنْصَرِفْ إِلَى بَيْتِهِ وَلاَ يَرَ وَجْهِي». فَانْصَرَفَ أَبْشَالُومُ إِلَى بَيْتِهِ وَلَمْ يَرَ وَجْهَ الْمَلِكِ.
\par 25 وَلَمْ يَكُنْ فِي كُلِّ إِسْرَائِيلَ رَجُلٌ جَمِيلٌ وَمَمْدُوحٌ جِدّاً كَأَبْشَالُومَ، مِنْ بَاطِنِ قَدَمِهِ حَتَّى هَامَتِهِ لَمْ يَكُنْ فِيهِ عَيْبٌ.
\par 26 وَعِنْدَ حَلْقِهِ رَأْسَهُ، إِذْ كَانَ يَحْلِقُهُ فِي آخِرِ كُلِّ سَنَةٍ، لأَنَّهُ كَانَ يَثْقُلُ عَلَيْهِ فَيَحْلِقُهُ، كَانَ يَزِنُ شَعْرَ رَأْسِهِ مِئَتَيْ شَاقِلٍ بِوَزْنِ الْمَلِكِ.
\par 27 وَوُلِدَ لأَبْشَالُومَ ثَلاَثَةُ بَنِينَ وَبِنْتٌ وَاحِدَةٌ اسْمُهَا ثَامَارُ، وَكَانَتِ امْرَأَةً جَمِيلَةَ الْمَنْظَرِ.
\par 28 وَأَقَامَ أَبْشَالُومُ فِي أُورُشَلِيمَ سَنَتَيْنِ وَلَمْ يَرَ وَجْهَ الْمَلِكِ.
\par 29 فَأَرْسَلَ أَبْشَالُومُ إِلَى يُوآبَ لِيُرْسِلَهُ إِلَى الْمَلِكِ فَلَمْ يَشَأْ أَنْ يَأْتِيَ إِلَيْهِ. ثُمَّ أَرْسَلَ أَيْضاً ثَانِيَةً فَلَمْ يَشَأْ أَنْ يَأْتِيَ.
\par 30 فَقَالَ لِعَبِيدِهِ: «انْظُرُوا. حَقْلَةَ يُوآبَ بِجَانِبِي، وَلَهُ هُنَاكَ شَعِيرٌ. اذْهَبُوا وَأَحْرِقُوهُ بِالنَّارِ». فَأَحْرَقَ عَبِيدُ أَبْشَالُومَ الْحَقْلَةَ بِالنَّارِ.
\par 31 فَقَامَ يُوآبُ وَجَاءَ إِلَى أَبْشَالُومَ إِلَى الْبَيْتِ وَقَالَ لَهُ: «لِمَاذَا أَحْرَقَ عَبِيدُكَ حَقْلَتِي بِالنَّارِ؟»
\par 32 فَقَالَ أَبْشَالُومُ لِيُوآبَ: «هَئَنَذَا قَدْ أَرْسَلْتُ إِلَيْكَ قَائِلاً: تَعَالَ إِلَى هُنَا فَأُرْسِلَكَ إِلَى الْمَلِكِ لِتَسْأَلَهُ: لِمَاذَا جِئْتُ مِنْ جَشُورَ؟ خَيْرٌ لِي لَوْ كُنْتُ بَاقِياً هُنَاكَ. فَالآنَ إِنِّي أَرَى وَجْهَ الْمَلِكِ، وَإِنْ وُجِدَ فِيَّ إِثْمٌ فَلْيَقْتُلْنِي».
\par 33 فَجَاءَ يُوآبُ إِلَى الْمَلِكِ وَأَخْبَرَهُ. وَدَعَا أَبْشَالُومَ فَأَتَى إِلَى الْمَلِكِ وَسَجَدَ عَلَى وَجْهِهِ إِلَى الأَرْضِ قُدَّامَ الْمَلِكِ، فَقَبَّلَ الْمَلِكُ أَبْشَالُومَ.

\chapter{15}

\par 1 وَكَانَ بَعْدَ ذَلِكَ أَنَّ أَبْشَالُومَ اتَّخَذَ مَرْكَبَةً وَخَيْلاً وَخَمْسِينَ رَجُلاً يَجْرُونَ قُدَّامَهُ.
\par 2 وَكَانَ أَبْشَالُومُ يُبَكِّرُ وَيَقِفُ بِجَانِبِ طَرِيقِ الْبَابِ، وَكُلُّ صَاحِبِ دَعْوَى آتٍ إِلَى الْمَلِكِ لأَجْلِ الْحُكْمِ كَانَ أَبْشَالُومُ يَدْعُوهُ إِلَيْهِ وَيَقُولُ: «مِنْ أَيَّةِ مَدِينَةٍ أَنْتَ؟» فَيَقُولُ: «مِنْ أَحَدِ أَسْبَاطِ إِسْرَائِيلَ عَبْدُكَ».
\par 3 فَيَقُولُ أَبْشَالُومُ لَهُ: «انْظُرْ. أُمُورُكَ صَالِحَةٌ وَمُسْتَقِيمَةٌ، وَلَكِنْ لَيْسَ مَنْ يَسْمَعُ لَكَ مِنْ قِبَلِ الْمَلِكِ».
\par 4 ثُمَّ يَقُولُ أَبْشَالُومُ: «مَنْ يَجْعَلُنِي قَاضِياً فِي الأَرْضِ فَيَأْتِيَ إِلَيَّ كُلُّ إِنْسَانٍ لَهُ خُصُومَةٌ وَدَعْوَى فَأُنْصِفَهُ؟»
\par 5 وَكَانَ إِذَا تَقَدَّمَ أَحَدٌ لِيَسْجُدَ لَهُ يَمُدُّ يَدَهُ وَيُمْسِكُهُ وَيُقَبِّلُهُ.
\par 6 وَكَانَ أَبْشَالُومُ يَفْعَلُ مِثْلَ هَذَا الأَمْرِ لِجَمِيعِ إِسْرَائِيلَ الَّذِينَ كَانُوا يَأْتُونَ لأَجْلِ الْحُكْمِ إِلَى الْمَلِكِ، فَاسْتَرَقَ أَبْشَالُومُ قُلُوبَ رِجَالِ إِسْرَائِيلَ.
\par 7 وَفِي نِهَايَةِ أَرْبَعِينَ سَنَةً قَالَ أَبْشَالُومُ لِلْمَلِكِ: «دَعْنِي فَأَذْهَبَ وَأُوفِيَ نَذْرِي الَّذِي نَذَرْتُهُ لِلرَّبِّ فِي حَبْرُونَ،
\par 8 لأَنَّ عَبْدَكَ نَذَرَ نَذْراً عِنْدَ سُكْنَايَ فِي جَشُورَ فِي أَرَامَ قَائِلاً: إِنْ أَرْجَعَنِي الرَّبُّ إِلَى أُورُشَلِيمَ فَإِنِّي أَعْبُدُ الرَّبَّ».
\par 9 فَقَالَ لَهُ الْمَلِكُ: «اذْهَبْ بِسَلاَمٍ». فَقَامَ وَذَهَبَ إِلَى حَبْرُونَ.
\par 10 وَأَرْسَلَ أَبْشَالُومُ جَوَاسِيسَ فِي جَمِيعِ أَسْبَاطِ إِسْرَائِيلَ قَائِلاً: «إِذَا سَمِعْتُمْ صَوْتَ الْبُوقِ فَقُولُوا: قَدْ مَلَكَ أَبْشَالُومُ فِي حَبْرُونَ!»
\par 11 وَانْطَلَقَ مَعَ أَبْشَالُومَ مِئَتَا رَجُلٍ مِنْ أُورُشَلِيمَ قَدْ دُعُوا وَذَهَبُوا بِبَسَاطَةٍ، وَلَمْ يَكُونُوا يَعْلَمُونَ شَيْئاً.
\par 12 وَأَرْسَلَ أَبْشَالُومُ إِلَى أَخِيتُوفَلَ الْجِيلُونِيِّ مُشِيرِ دَاوُدَ مِنْ مَدِينَتِهِ جِيلُوهَ إِذْ كَانَ يَذْبَحُ ذَبَائِحَ. وَكَانَتِ الْفِتْنَةُ شَدِيدَةً وَكَانَ الشَّعْبُ لاَ يَزَالُ يَتَزَايَدُ مَعَ أَبْشَالُومَ.
\par 13 فَأَتَى مُخَبِّرٌ إِلَى دَاوُدَ قَائِلاً: «إِنَّ قُلُوبَ رِجَالِ إِسْرَائِيلَ صَارَتْ وَرَاءَ أَبْشَالُومَ».
\par 14 فَقَالَ دَاوُدُ لِجَمِيعِ عَبِيدِهِ الَّذِينَ مَعَهُ فِي أُورُشَلِيمَ: «قُومُوا بِنَا نَهْرُبُ، لأَنَّهُ لَيْسَ لَنَا نَجَاةٌ مِنْ وَجْهِ أَبْشَالُومَ. أَسْرِعُوا لِلذَّهَابِ لِئَلاَّ يُبَادِرَ وَيُدْرِكَنَا وَيُنْزِلَ بِنَا الشَّرَّ وَيَضْرِبَ الْمَدِينَةَ بِحَدِّ السَّيْفِ».
\par 15 فَقَالَ عَبِيدُ الْمَلِكِ لِلْمَلِكِ: «حَسَبَ كُلِّ مَا يَخْتَارُهُ سَيِّدُنَا الْمَلِكُ نَحْنُ عَبِيدُهُ».
\par 16 فَخَرَجَ الْمَلِكُ وَجَمِيعُ بَيْتِهِ وَرَاءَهُ. وَتَرَكَ الْمَلِكُ عَشَرَ نِسَاءٍ سَرَارِيَّ لِحِفْظِ الْبَيْتِ.
\par 17 وَخَرَجَ الْمَلِكُ وَكُلُّ الشَّعْبِ فِي أَثَرِهِ وَوَقَفُوا عَُِنْدَ الْبَيْتِ الأَبْعَدِ.
\par 18 وَجَمِيعُ عَبِيدِهِ كَانُوا يَعْبُرُونَ بَيْنَ يَدَيْهِ مَعَ جَمِيعِ الْجَلاَّدِينَ وَالسُّعَاةِ وَجَمِيعُ الْجَتِّيِّينَ، سِتُّ مِئَةِ رَجُلٍ أَتُوا وَرَاءَهُ مِنْ جَتَّ، وَكَانُوا يَعْبُرُونَ بَيْنَ يَدَيِ الْمَلِكِ.
\par 19 فَقَالَ الْمَلِكُ لإِتَّايَ الْجَتِّيِّ: «لِمَاذَا تَذْهَبُ أَنْتَ أَيْضاً مَعَنَا؟ ارْجِعْ وَأَقِمْ مَعَ الْمَلِكِ لأَنَّكَ غَرِيبٌ وَمَنْفِيٌّ أَيْضاً مِنْ وَطَنِكَ.
\par 20 أَمْساً جِئْتَ وَالْيَوْمَ أُتِيهُكَ بِالذَّهَابِ مَعَنَا وَأَنَا أَنْطَلِقُ إِلَى حَيْثُ أَنْطَلِقُ؟ ارْجِعْ وَرَجِّعْ إِخْوَتَكَ. الرَّحْمَةُ وَالْحَقُّ مَعَكَ».
\par 21 فَأَجَابَ إِتَّايُ الْمَلِكَ: «حَيٌّ هُوَ الرَّبُّ وَحَيٌّ سَيِّدِي الْمَلِكُ، إِنَّهُ حَيْثُمَا كَانَ سَيِّدِي الْمَلِكُ - إِنْ كَانَ لِلْمَوْتِ أَوْ لِلْحَيَاةِ - فَهُنَاكَ يَكُونُ عَبْدُكَ أَيْضاً».
\par 22 فَقَالَ دَاوُدُ لإِتَّايَ: «اذْهَبْ وَاعْبُرْ». فَعَبَرَ إِتَّايُ الْجَتِّيُّ وَجَمِيعُ رِجَالِهِ وَجَمِيعُ الأَطْفَالِ الَّذِينَ مَعَهُ.
\par 23 وَكَانَتْ جَمِيعُ الأَرْضِ تَبْكِي بِصَوْتٍ عَظِيمٍ وَجَمِيعُ الشَّعْبِ يَعْبُرُونَ. وَعَبَرَ الْمَلِكُ فِي وَادِي قَدْرُونَ وَعَبَرَ جَمِيعُ الشَّعْبِ نَحْوَ طَرِيقِ الْبَرِّيَّةِ.
\par 24 وَإِذَا بِصَادُوقَ أَيْضاً وَجَمِيعُ اللاَّوِيِّينَ مَعَهُ يَحْمِلُونَ تَابُوتَ عَهْدِ اللَّهِ. فَوَضَعُوا تَابُوتَ اللَّهِ، وَصَعِدَ أَبِيَاثَارُ حَتَّى انْتَهَى جَمِيعُ الشَّعْبِ مِنَ الْعُبُورِ مِنَ الْمَدِينَةِ.
\par 25 فَقَالَ الْمَلِكُ لِصَادُوقَ: «أَرْجِعْ تَابُوتَ اللَّهِ إِلَى الْمَدِينَةِ، فَإِنْ وَجَدْتُ نِعْمَةً فِي عَيْنَيِ الرَّبِّ فَإِنَّهُ يُرْجِعُنِي وَيُرِينِي إِيَّاهُ وَمَسْكَنَهُ.
\par 26 وَإِنْ قَالَ: «إِنِّي لَمْ أُسَرَّ بِكَ، فَهَئَنَذَا. فَلْيَفْعَلْ بِي حَسَبَمَا يَحْسُنُ فِي عَيْنَيْهِ».
\par 27 ثُمَّ قَالَ الْمَلِكُ لِصَادُوقَ الْكَاهِنِ: «أَأَنْتَ رَاءٍ؟ فَارْجِعْ إِلَى الْمَدِينَةِ بِسَلاَمٍ أَنْتَ وَأَخِيمَعَصُ ابْنُكَ وَيُونَاثَانُ بْنُ أَبِيَاثَارَ. ابْنَاكُمَا كِلاَهُمَا مَعَكُمَا.
\par 28 انْظُرُوا. أَنِّي أَتَوَانَى فِي سُهُولِ الْبَرِّيَّةِ حَتَّى تَأْتِيَ كَلِمَةٌ مِنْكُمْ لِتَخْبِيرِي».
\par 29 فَأَرْجَعَ صَادُوقُ وَأَبِيَاثَارُ تَابُوتَ اللَّهِ إِلَى أُورُشَلِيمَ وَأَقَامَا هُنَاكَ.
\par 30 وَأَمَّا دَاوُدُ فَصَعِدَ فِي مَصْعَدِ جَبَلِ الزَّيْتُونِ. كَانَ يَصْعَدُ بَاكِياً وَرَأْسُهُ مُغَطَّى وَيَمْشِي حَافِياً، وَجَمِيعُ الشَّعْبِ الَّذِينَ مَعَهُ غَطُّوا كُلُّ وَاحِدٍ رَأْسَهُ، وَكَانُوا يَصْعَدُونَ وَهُمْ يَبْكُونَ.
\par 31 وَأُخْبِرَ دَاوُدُ إِنَّ أَخِيتُوفَلَ بَيْنَ الْفَاتِنِينَ مَعَ أَبْشَالُومَ، فَقَالَ دَاوُدُ: «حَمِّقْ يَا رَبُّ مَشُورَةَ أَخِيتُوفَلَ».
\par 32 وَلَمَّا وَصَلَ دَاوُدُ إِلَى الْقِمَّةِ حَيْثُ سَجَدَ لِلَّهِ، إِذَا بِحُوشَايَ الأَرْكِيِّ قَدْ لَقِيَهُ مُمَزَّقَ الثَّوْبِ وَالتُّرَابُ عَلَى رَأْسِهِ.
\par 33 فَقَالَ لَهُ دَاوُدُ: «إِذَا عَبَرْتَ مَعِي تَكُونُ عَلَيَّ حِمْلاً.
\par 34 وَلَكِنْ إِذَا رَجَعْتَ إِلَى الْمَدِينَةِ وَقُلْتَ لأَبْشَالُومَ: أَنَا أَكُونُ عَبْدَكَ أَيُّهَا الْمَلِكُ. أَنَا عَبْدُ أَبِيكَ مُنْذُ زَمَانٍ وَالآنَ أَنَا عَبْدُكَ. فَإِنَّكَ تُبْطِلُ لِي مَشُورَةَ أَخِيتُوفَلَ.
\par 35 أَلَيْسَ مَعَكَ هُنَاكَ صَادُوقُ وَأَبِيَاثَارُ الْكَاهِنَانِ. فَكُلُّ مَا تَسْمَعُهُ مِنْ بَيْتِ الْمَلِكِ فَأَخْبِرْ بِهِ صَادُوقَ وَأَبِيَاثَارَ الْكَاهِنَيْنِ.
\par 36 هُوَذَا هُنَاكَ مَعَهُمَا ابْنَاهُمَا أَخِيمَعَصُ لِصَادُوقَ وَيُونَاثَانُ لأَبِيَاثَارَ. فَتُرْسِلُونَ عَلَى أَيْدِيهِمَا إِلَيَّ كُلَّ كَلِمَةٍ تَسْمَعُونَهَا».
\par 37 فَأَتَى حُوشَايُ صَاحِبُ دَاوُدَ إِلَى الْمَدِينَةِ وَأَبْشَالُومُ يَدْخُلُ أُورُشَلِيمَ.

\chapter{16}

\par 1 وَلَمَّا عَبَرَ دَاوُدُ قَلِيلاً عَنِ الْقِمَّةِ إِذَا بِصِيبَا غُلاَمِ مَفِيبُوشَثَ قَدْ لَقِيَهُ بِحِمَارَيْنِ مَشْدُودَيْنِ، عَلَيْهِمَا مِئَتَا رَغِيفِ خُبْزٍ وَمِئَةُ عُنْقُودِ زَبِيبٍ وَمِئَةُ قُرْصِ تِينٍ وَزِقُّ خَمْرٍ.
\par 2 فَقَالَ الْمَلِكُ لِصِيبَا: «مَا لَكَ وَهَذِهِ؟» فَقَالَ صِيبَا: «اَلْحِمَارَانِ لِبَيْتِ الْمَلِكِ لِلرُّكُوبِ، وَالْخُبْزُ وَالتِّينُ لِلْغِلْمَانِ لِيَأْكُلُوا، وَالْخَمْرُ لِيَشْرَبَهُ مَنْ أَعْيَا فِي الْبَرِّيَّةِ».
\par 3 فَقَالَ الْمَلِكُ: «وَأَيْنَ ابْنُ سَيِّدِكَ؟» فَقَالَ صِيبَا لِلْمَلِكِ: «هُوَذَا هُوَ مُقِيمٌ فِي أُورُشَلِيمَ، لأَنَّهُ قَالَ: الْيَوْمَ يَرُدُّ لِي بَيْتُ إِسْرَائِيلَ مَمْلَكَةَ أَبِي».
\par 4 فَقَالَ الْمَلِكُ لِصِيبَا: «هُوَذَا لَكَ كُلُّ مَا لِمَفِيبُوشَثَ». فَقَالَ صِيبَا: «سَجَدْتُ! لَيْتَنِي أَجِدُ نِعْمَةً فِي عَيْنَيْكَ يَا سَيِّدِي الْمَلِكَ».
\par 5 وَلَمَّا جَاءَ الْمَلِكُ دَاوُدُ إِلَى بَحُورِيمَ إِذَا بِرَجُلٍ خَارِجٍ مِنْ هُنَاكَ مِنْ عَشِيرَةِ بَيْتِ شَاوُلَ اسْمُهُ شَمْعِي بْنُ جِيرَا، يَسُبُّ وَهُوَ يَخْرُجُ،
\par 6 وَيَرْشُقُ بِالْحِجَارَةِ دَاوُدَ وَجَمِيعَ عَبِيدِ الْمَلِكِ دَاوُدَ وَجَمِيعُ الشَّعْبِ وَجَمِيعُ الْجَبَابِرَةِ عَنْ يَمِينِهِ وَعَنْ يَسَارِهِ.
\par 7 وَهَكَذَا كَانَ شَمْعِي يَقُولُ فِي سَبِّهِ: «اخْرُجِ اخْرُجْ يَا رَجُلَ الدِّمَاءِ وَرَجُلَ بَلِيَّعَالَ!
\par 8 قَدْ رَدَّ الرَّبُّ عَلَيْكَ كُلَّ دِمَاءِ بَيْتِ شَاوُلَ الَّذِي مَلَكْتَ عِوَضاً عَنْهُ، وَقَدْ دَفَعَ الرَّبُّ الْمَمْلَكَةَ لِيَدِ أَبْشَالُومَ ابْنِكَ، وَهَا أَنْتَ وَاقِعٌ بِشَرِّكَ لأَنَّكَ رَجُلُ دِمَاءٍ!»
\par 9 فَقَالَ أَبِيشَايُ ابْنُ صَرُويَةَ لِلْمَلِكِ: «لِمَاذَا يَسُبُّ هَذَا الْكَلْبُ الْمَيِّتُ سَيِّدِي الْمَلِكَ؟ دَعْنِي أَعْبُرْ فَأَقْطَعَ رَأْسَهُ».
\par 10 فَقَالَ الْمَلِكُ: «مَا لِي وَلَكُمْ يَا بَنِي صَرُويَةَ؟ دَعُوهُ يَسُبَّ لأَنَّ الرَّبَّ قَالَ لَهُ: سُبَّ دَاوُدَ. وَمَنْ يَقُولُ: لِمَاذَا تَفْعَلُ هَكَذَا؟»
\par 11 وَقَالَ دَاوُدُ لأَبِيشَايَ وَلِجَمِيعِ عَبِيدِهِ: «هُوَذَا ابْنِي الَّذِي خَرَجَ مِنْ أَحْشَائِي يَطْلُبُ نَفْسِي، فَكَمْ بِالْحَرِيِّ الآنَ بِنْيَامِينِيٌّ؟ دَعُوهُ يَسُبَّ لأَنَّ الرَّبَّ قَالَ لَهُ.
\par 12 لَعَلَّ الرَّبَّ يَنْظُرُ إِلَى مَذَلَّتِي وَيُكَافِئَنِي الرَّبُّ خَيْراً عِوَضَ مَسَبَّتِهِ بِهَذَا الْيَوْمِ».
\par 13 وَإِذْ كَانَ دَاوُدُ وَرِجَالُهُ يَسِيرُونَ فِي الطَّرِيقِ كَانَ شَمْعِي يَسِيرُ فِي جَانِبِ الْجَبَلِ مُقَابِلَهُ وَيَسُبُّ وَهُوَ سَائِرٌ وَيَرْشُقُ بِالْحِجَارَةِ مُقَابِلَهُ وَيَذْرِي التُّرَابَ.
\par 14 وَجَاءَ الْمَلِكُ وَكُلُّ الشَّعْبِ الَّذِينَ مَعَهُ وَقَدْ أَعْيُوا فَاسْتَرَاحُوا هُنَاكَ.
\par 15 وَأَمَّا أَبْشَالُومُ وَجَمِيعُ الشَّعْبِ رِجَالُ إِسْرَائِيلَ فَأَتُوا إِلَى أُورُشَلِيمَ وَأَخِيتُوفَلُ مَعَهُمْ.
\par 16 وَلَمَّا جَاءَ حُوشَايُ الأَرْكِيُّ صَاحِبُ دَاوُدَ إِلَى أَبْشَالُومَ، قَالَ: «لِيَحْيَ الْمَلِكُ! لِيَحْيَ الْمَلِكُ!»
\par 17 فَقَالَ أَبْشَالُومُ لِحُوشَايَ: «أَهَذَا مَعْرُوفُكَ مَعَ صَاحِبِكَ؟ لِمَاذَا لَمْ تَذْهَبْ مَعَ صَاحِبِكَ؟»
\par 18 فَقَالَ حُوشَايُ لأَبْشَالُومَ: «كَلاَّ، وَلَكِنِ الَّذِي اخْتَارَهُ الرَّبُّ وَهَذَا الشَّعْبُ وَكُلُّ رِجَالِ إِسْرَائِيلَ فَلَهُ أَكُونُ وَمَعَهُ أُقِيمُ.
\par 19 وَثَانِياً: مَنْ أَخْدِمُ؟ أَلَيْسَ بَيْنَ يَدَيِ ابْنِهِ؟ كَمَا خَدَمْتُ بَيْنَ يَدَيْ أَبِيكَ كَذَلِكَ أَكُونُ بَيْنَ يَدَيْكَ».
\par 20 وَقَالَ أَبْشَالُومُ لأَخِيتُوفَلَ: «أَعْطُوا مَشُورَةً مَاذَا نَفْعَلُ».
\par 21 فَقَالَ أَخِيتُوفَلُ لأَبْشَالُومَ: «ادْخُلْ إِلَى سَرَارِيِّ أَبِيكَ اللَّوَاتِي تَرَكَهُنَّ لِحِفْظِ الْبَيْتِ، فَيَسْمَعَ كُلُّ إِسْرَائِيلَ أَنَّكَ قَدْ صِرْتَ مَكْرُوهاً مِنْ أَبِيكَ، فَتَتَشَدَّدَ أَيْدِي جَمِيعِ الَّذِينَ مَعَكَ».
\par 22 فَنَصَبُوا لأَبْشَالُومَ الْخَيْمَةَ عَلَى السَّطْحِ، وَدَخَلَ أَبْشَالُومُ إِلَى سَرَارِيِّ أَبِيهِ أَمَامَ جَمِيعِ إِسْرَائِيلَ.
\par 23 وَكَانَتْ مَشُورَةُ أَخِيتُوفَلَ الَّتِي كَانَ يُشِيرُ بِهَا فِي تِلْكَ الأَيَّامِ كَمَنْ يَسْأَلُ بِكَلاَمِ اللَّهِ. هَكَذَا كُلُّ مَشُورَةِ أَخِيتُوفَلَ عَلَى دَاوُدَ وَعَلَى أَبْشَالُومَ جَمِيعاً.

\chapter{17}

\par 1 وَقَالَ أَخِيتُوفَلُ لأَبْشَالُومَ: «دَعْنِي أَنْتَخِبُ اثْنَيْ عَشَرَ أَلْفَ رَجُلٍ وَأَقُومُ وَأَسْعَى وَرَاءَ دَاوُدَ هَذِهِ اللَّيْلَةَ
\par 2 فَآتِيَ عَلَيْهِ وَهُوَ مُتْعَبٌ وَمُرْتَخِي الْيَدَيْنِ فَأُزْعِجُهُ، فَيَهْرُبَ كُلُّ الشَّعْبِ الَّذِي مَعَهُ، وَأَضْرِبُ الْمَلِكَ وَحْدَهُ
\par 3 وَأَرُدُّ جَمِيعَ الشَّعْبِ إِلَيْكَ. كَرُجُوعِ الْجَمِيعِ هُوَ الرَّجُلُ الَّذِي تَطْلُبُهُ، فَيَكُونُ كُلُّ الشَّعْبِ فِي سَلاَمٍ».
\par 4 فَحَسُنَ الأَمْرُ فِي عَيْنَيْ أَبْشَالُومَ وَأَعْيُنِ جَمِيعِ شُيُوخِ إِسْرَائِيلَ.
\par 5 فَقَالَ أَبْشَالُومُ: «ادْعُ أَيْضاً حُوشَايَ الأَرْكِيَّ فَنَسْمَعَ مَا يَقُولُ هُوَ أَيْضاً».
\par 6 فَلَمَّا جَاءَ حُوشَايُ إِلَى أَبْشَالُومَ قَالَ أَبْشَالُومُ: «بِمِثْلِ هَذَا الْكَلاَمِ تَكَلَّمَ أَخِيتُوفَلُ. أَنَعْمَلُ حَسَبَ كَلاَمِهِ أَمْ لاَ؟ تَكَلَّمْ أَنْتَ».
\par 7 فَقَالَ حُوشَايُ لأَبْشَالُومَ: «لَيْسَتْ حَسَنَةً الْمَشُورَةُ الَّتِي أَشَارَ بِهَا أَخِيتُوفَلُ هَذِهِ الْمَرَّةَ.
\par 8 أَنْتَ تَعْلَمُ أَبَاكَ وَرِجَالَهُ أَنَّهُمْ جَبَابِرَةٌ، وَأَنَّ أَنْفُسَهُمْ مُرَّةٌ كَدُبَّةٍ مُثْكِلٍ فِي الْحَقْلِ. وَأَبُوكَ رَجُلُ قِتَالٍ وَلاَ يَبِيتُ مَعَ الشَّعْبِ.
\par 9 هَا هُوَ الآنَ مُخْتَبِئٌ فِي إِحْدَى الْحُفَرِ أَوْ أَحَدِ الأَمَاكِنِ. وَيَكُونُ إِذَا سَقَطَ بَعْضُهُمْ فِي الاِبْتِدَاءِ أَنَّ السَّامِعَ يَسْمَعُ فَيَقُولُ: قَدْ صَارَتْ كَسْرَةٌ فِي الشَّعْبِ الَّذِي وَرَاءَ أَبْشَالُومَ.
\par 10 أَيْضاً ذُو الْبَأْسِ الَّذِي قَلْبُهُ كَقَلْبِ الأَسَدِ يَذُوبُ ذَوَبَاناً، لأَنَّ جَمِيعَ إِسْرَائِيلَ يَعْلَمُونَ أَنَّ أَبَاكَ جَبَّارٌ، وَالَّذِينَ مَعَهُ ذَوُو بَأْسٍ.
\par 11 لِذَلِكَ أُشِيرُ بِأَنْ يَجْتَمِعَ إِلَيْكَ كُلُّ إِسْرَائِيلَ مِنْ دَانَ إِلَى بِئْرِ سَبْعٍ، كَالرَّمْلِ الَّذِي عَلَى الْبَحْرِ فِي الْكَثْرَةِ، وَحَضْرَتُكَ سَائِرٌ فِي الْوَسَطِ.
\par 12 وَنَأْتِيَ إِلَيْهِ إِلَى أَحَدِ الأَمَاكِنِ حَيْثُ هُوَ، وَنَنْزِلَ عَلَيْهِ نُزُولَ الطَّلِّ عَلَى الأَرْضِ، وَلاَ يَبْقَى مِنْهُ وَلاَ مِنْ جَمِيعِ الرِّجَالِ الَّذِينَ مَعَهُ وَاحِدٌ.
\par 13 وَإِذَا انْحَازَ إِلَى مَدِينَةٍ، يَحْمِلُ جَمِيعُ إِسْرَائِيلَ إِلَى تِلْكَ الْمَدِينَةِ حِبَالاً، فَنَجُرُّهَا إِلَى الْوَادِي حَتَّى لاَ تَبْقَى هُنَاكَ وَلاَ حَصَاةٌ».
\par 14 فَقَالَ أَبْشَالُومُ وَكُلُّ رِجَالِ إِسْرَائِيلَ: «إِنَّ مَشُورَةَ حُوشَايَ الأَرْكِيِّ أَحْسَنُ مِنْ مَشُورَةِ أَخِيتُوفَلَ». فَإِنَّ الرَّبَّ أَمَرَ بِإِبْطَالِ مَشُورَةِ أَخِيتُوفَلَ الصَّالِحَةِ لِيُنْزِلَ الرَّبُّ الشَّرَّ بِأَبْشَالُومَ.
\par 15 وَقَالَ حُوشَايُ لِصَادُوقَ وَأَبِيَاثَارَ الْكَاهِنَيْنِ: «كَذَا وَكَذَا أَشَارَ أَخِيتُوفَلُ عَلَى أَبْشَالُومَ وَعَلَى شُيُوخِ إِسْرَائِيلَ، وَكَذَا وَكَذَا أَشَرْتُ أَنَا.
\par 16 فَالآنَ أَرْسِلُوا عَاجِلاً وَأَخْبِرُوا دَاوُدَ: لاَ تَبِتْ هَذِهِ اللَّيْلَةَ فِي سُهُولِ الْبَرِّيَّةِ، بَلِ اعْبُرْ لِئَلاَّ يُبْتَلَعَ الْمَلِكُ وَجَمِيعُ الشَّعْبِ الَّذِي مَعَهُ».
\par 17 وَكَانَ يُونَاثَانُ وَأَخِيمَعَصُ وَاقِفَيْنِ عِنْدَ عَيْنِ رُوجَلَ، فَانْطَلَقَتِ الْجَارِيَةُ وَأَخْبَرَتْهُمَا، وَهُمَا ذَهَبَا وَأَخْبَرَا الْمَلِكَ دَاوُدَ، لأَنَّهُمَا لَمْ يَقْدِرَا أَنْ يُرَيَا دَاخِلَيْنِ الْمَدِينَةَ.
\par 18 فَرَآهُمَا غُلاَمٌ وَأَخْبَرَ أَبْشَالُومَ. فَذَهَبَا كِلاَهُمَا عَاجِلاً وَدَخَلاَ بَيْتَ رَجُلٍ فِي بَحُورِيمَ وَلَهُ بِئْرٌ فِي دَارِهِ، فَنَزَلاَ إِلَيْهَا.
\par 19 فَأَخَذَتِ الْمَرْأَةُ وَفَرَشَتْ سَجْفاً عَلَى فَمِ الْبِئْرِ وَسَطَحَتْ عَلَيْهِ سَمِيذاً فَلَمْ يُعْلَمِ الأَمْرُ.
\par 20 فَجَاءَ عَبِيدُ أَبْشَالُومَ إِلَى الْمَرْأَةِ إِلَى الْبَيْتِ وَقَالُوا: «أَيْنَ أَخِيمَعَصُ وَيُونَاثَانُ؟» فَقَالَتْ لَهُمُ الْمَرْأَةُ: «قَدْ عَبَرَا قَنَاةَ الْمَاءِ». وَلَمَّا فَتَّشُوا وَلَمْ يَجِدُوهُمَا رَجَعُوا إِلَى أُورُشَلِيمَ.
\par 21 وَبَعْدَ ذِهَابِهِمْ خَرَجَا مِنَ الْبِئْرِ وَذَهَبَا وَقَالاَ لِدَاوُدَ: «قُومُوا وَاعْبُرُوا سَرِيعاً الْمَاءَ، لأَنَّ هَكَذَا أَشَارَ عَلَيْكُمْ أَخِيتُوفَلُ».
\par 22 فَقَامَ دَاوُدُ وَجَمِيعُ الشَّعْبِ الَّذِي مَعَهُ وَعَبَرُوا الأُرْدُنَّ. وَعِنْدَ ضُوءِ الصَّبَاحِ لَمْ يَبْقَ أَحَدٌ لَمْ يَعْبُرِ الأُرْدُنَّ.
\par 23 وَأَمَّا أَخِيتُوفَلُ فَلَمَّا رَأَى أَنَّ مَشُورَتَهُ لَمْ يُعْمَلْ بِهَا، شَدَّ عَلَى الْحِمَارِ وَقَامَ وَانْطَلَقَ إِلَى بَيْتِهِ إِلَى مَدِينَتِهِ، وَأَوْصَى لِبَيْتِهِ، وَخَنَقَ نَفْسَهُ وَمَاتَ وَدُفِنَ فِي قَبْرِ أَبِيهِ.
\par 24 وَجَاءَ دَاوُدُ إِلَى مَحَنَايِمَ. وَعَبَرَ أَبْشَالُومُ الأُرْدُنَّ هُوَ وَجَمِيعُ رِجَالِ إِسْرَائِيلَ مَعَهُ.
\par 25 وَأَقَامَ أَبْشَالُومُ عَمَاسَا بَدَلَ يُوآبَ عَلَى الْجَيْشِ. وَكَانَ عَمَاسَا ابْنَ رَجُلٍ اسْمُهُ يِثْرَا الإِسْرَائِيلِيُّ الَّذِي دَخَلَ إِلَى أَبِيجَايِلَ بِنْتِ نَاحَاشَ أُخْتِ صَرُويَةَ أُمِّ يُوآبَ.
\par 26 وَنَزَلَ إِسْرَائِيلُ وَأَبْشَالُومُ فِي أَرْضِ جِلْعَادَ.
\par 27 وَكَانَ لَمَّا جَاءَ دَاوُدُ إِلَى مَحَنَايِمَ أَنَّ شُوبِيَ بْنَ نَاحَاشَ مِنْ رَبَّةِ بَنِي عَمُّونَ، وَمَاكِيرَ بْنُ عَمِّيئِيلَ مِنْ لُودَبَارَ، وَبَرْزِلاَّيَ الْجِلْعَادِيَّ مِنْ رُوجَلِيمَ،
\par 28 قَدَّمُوا فَرْشاً وَطُسُوساً وَآنِيَةَ خَزَفٍ وَحِنْطَةً وَشَعِيراً وَدَقِيقاً وَفَرِيكاً وَفُولاً وَعَدَساً وَحِمِّصاً مَشْوِيّاً
\par 29 وَعَسَلاً وَزُبْدَةً وَضَأْناً وَجُبْنَ بَقَرٍ لِدَاوُدَ وَلِلشَّعْبِ الَّذِي مَعَهُ لِيَأْكُلُوا. لأَنَّهُمْ قَالُوا: «الشَّعْبُ جَوْعَانُ وَمُتْعَبٌ وَعَطْشَانُ فِي الْبَرِّيَّةِ».

\chapter{18}

\par 1 وَأَحْصَى دَاوُدُ الشَّعْبَ الَّذِي مَعَهُ، وَجَعَلَ عَلَيْهِمْ رُؤَسَاءَ أُلُوفٍ وَرُؤَسَاءَ مِئَاتٍ.
\par 2 وَأَرْسَلَ دَاوُدُ الشَّعْبَ ثُلْثاً بِيَدِ يُوآبَ وَثُلْثاً بِيَدِ أَبِيشَايَ ابْنِ صَرُويَةَ أَخِي يُوآبَ وَثُلْثاً بِيَدِ إِتَّايَ الْجَتِّيِّ. وَقَالَ الْمَلِكُ لِلشَّعْبِ: «إِنِّي أَنَا أَيْضاً أَخْرُجُ مَعَكُمْ».
\par 3 فَقَالَ الشَّعْبُ: «لاَ تَخْرُجْ، لأَنَّنَا إِذَا هَرَبْنَا لاَ يُبَالُونَ بِنَا، وَإِذَا مَاتَ نِصْفُنَا لاَ يُبَالُونَ بِنَا. وَالآنَ أَنْتَ كَعَشَرَةِ آلاَفٍ مِنَّا. وَالآنَ الأَصْلَحُ أَنْ تَكُونَ لَنَا نَجْدَةً مِنَ الْمَدِينَةِ».
\par 4 فَقَالَ لَهُمُ الْمَلِكُ: «مَا يَحْسُنُ فِي أَعْيُنِكُمْ أَفْعَلُهُ». فَوَقَفَ الْمَلِكُ بِجَانِبِ الْبَابِ وَخَرَجَ جَمِيعُ الشَّعْبِ مِئَاتٍ وَأُلُوفاً.
\par 5 وَأَوْصَى الْمَلِكُ يُوآبَ وَأَبِيشَايَ وَإِتَّايَ: «تَرَفَّقُوا لِي بِالْفَتَى أَبْشَالُومَ». وَسَمِعَ جَمِيعُ الشَّعْبِ حِينَ أَوْصَى الْمَلِكُ جَمِيعَ الرُّؤَسَاءِ بِأَبْشَالُومَ.
\par 6 وَخَرَجَ الشَّعْبُ إِلَى الْحَقْلِ لِلِقَاءِ إِسْرَائِيلَ. وَكَانَ الْقِتَالُ فِي وَعْرِ أَفْرَايِمَ،
\par 7 فَانْكَسَرَ هُنَاكَ شَعْبُ إِسْرَائِيلَ أَمَامَ عَبِيدِ دَاوُدَ، وَكَانَتْ هُنَاكَ مَقْتَلَةٌ عَظِيمَةٌ فِي ذَلِكَ الْيَوْمِ. قُتِلَ عِشْرُونَ أَلْفاً.
\par 8 وَكَانَ الْقِتَالُ هُنَاكَ مُنْتَشِراً عَلَى وَجْهِ كُلِّ الأَرْضِ، وَزَادَ الَّذِينَ أَكَلَهُمُ الْوَعْرُ مِنَ الشَّعْبِ عَلَى الَّذِينَ أَكَلَهُمُ السَّيْفُ فِي ذَلِكَ الْيَوْمِ.
\par 9 وَصَادَفَ أَبْشَالُومُ عَبِيدَ دَاوُدَ، وَكَانَ أَبْشَالُومُ رَاكِباً عَلَى بَغْلٍ، فَدَخَلَ الْبَغْلُ تَحْتَ أَغْصَانِ الْبُطْمَةِ الْعَظِيمَةِ الْمُلْتَفَّةِ، فَتَعَلَّقَ رَأْسُهُ بِالْبُطْمَةِ وَعُلِّقَ بَيْنَ السَّمَاءِ وَالأَرْضِ، وَالْبَغْلُ الَّذِي تَحْتَهُ مَرَّ.
\par 10 فَرَآهُ رَجُلٌ وَأَخْبَرَ يُوآبَ: «إِنِّي قَدْ رَأَيْتُ أَبْشَالُومَ مُعَلَّقاً بِالْبُطْمَةِ».
\par 11 فَقَالَ يُوآبُ لِلرَّجُلِ الَّذِي أَخْبَرَهُ: «إِنَّكَ قَدْ رَأَيْتَهُ، فَلِمَاذَا لَمْ تَضْرِبْهُ هُنَاكَ إِلَى الأَرْضِ، وَعَلَيَّ أَنْ أُعْطِيَكَ عَشَرَةً مِنَ الْفِضَّةِ وَمِنْطَقَةً؟»
\par 12 فَقَالَ الرَّجُلُ لِيُوآبَ: «فَلَوْ وُزِنَ فِي يَدِي أَلْفٌ مِنَ الْفِضَّةِ لَمَا كُنْتُ أَمُدُّ يَدِي إِلَى ابْنِ الْمَلِكِ، لأَنَّ الْمَلِكَ أَوْصَاكَ فِي آذَانِنَا أَنْتَ وَأَبِيشَايَ وَإِتَّايَ قَائِلاً: احْتَرِزُوا أَيّاً كَانَ مِنْكُمْ عَلَى الْفَتَى أَبْشَالُومَ.
\par 13 وَإَِّلا فَكُنْتُ فَعَلْتُ بِنَفْسِي زُوراً، إِذْ لاَ يَخْفَى عَنِ الْمَلِكِ شَيْءٌ، وأَنْتَ كُنْتَ وَقَفْتَ ضِدِّي».
\par 14 فَقَالَ يُوآبُ: «إِنِّي لاَ أَصْبِرُ هَكَذَا أَمَامَكَ». فَأَخَذَ ثَلاَثَةَ سِهَامٍ بِيَدِهِ وَنَشَّبَهَا فِي قَلْبِ أَبْشَالُومَ وَهُوَ بَعْدُ حَيٌّ فِي قَلْبِ الْبُطْمَةِ،
\par 15 وَأَحَاطَ بِهَا عَشَرَةُ غِلْمَانٍ حَامِلُو سِلاَحِ يُوآبَ وَضَرَبُوا أَبْشَالُومَ وَأَمَاتُوهُ.
\par 16 وَضَرَبَ يُوآبُ بِالْبُوقِ فَرَجَعَ الشَّعْبُ عَنِ اتِّبَاعِ إِسْرَائِيلَ، لأَنَّ يُوآبَ مَنَعَ الشَّعْبَ.
\par 17 وَأَخَذُوا أَبْشَالُومَ وَطَرَحُوهُ فِي الْوَعْرِ فِي الْجُبِّ الْعَظِيمِ وَأَقَامُوا عَلَيْهِ رُجْمَةً عَظِيمَةً جِدّاً مِنَ الْحِجَارَةِ. وَهَرَبَ كُلُّ إِسْرَائِيلَ، كُلُّ وَاحِدٍ إِلَى خَيْمَتِهِ.
\par 18 وَكَانَ أَبْشَالُومُ قَدْ أَخَذَ وَأَقَامَ لِنَفْسِهِ وَهُوَ حَيٌّ النَّصَبَ الَّذِي فِي وَادِي الْمَلِكِ، لأَنَّهُ قَالَ: «لَيْسَ لِيَ ابْنٌ لأَجْلِ تَذْكِيرِ اسْمِي». وَدَعَا النَّصَبَ بِاسْمِهِ، وَهُوَ يُدْعَى «يَدَ أَبْشَالُومَ» إِلَى هَذَا الْيَوْمِ.
\par 19 وَقَالَ أَخِيمَعَصُ بْنُ صَادُوقَ: «دَعْنِي أَجْرِ فَأُبَشِّرَ الْمَلِكَ، لأَنَّ اللَّهَ قَدِ انْتَقَمَ لَهُ مِنْ أَعْدَائِهِ».
\par 20 فَقَالَ لَهُ يُوآبُ: «مَا أَنْتَ صَاحِبُ بِشَارَةٍ فِي هَذَا الْيَوْمِ. فِي يَوْمٍ آخَرَ تُبَشِّرُ، وَهَذَا الْيَوْمَ لاَ تُبَشِّرُ مِنْ أَجْلِ أَنَّ ابْنَ الْمَلِكِ قَدْ مَاتَ».
\par 21 وَقَالَ يُوآبُ لِكُوشِي: «اذْهَبْ وَأَخْبِرِ الْمَلِكَ بِمَا رَأَيْتَ». فَسَجَدَ كُوشِي لِيُوآبَ وَرَكَضَ.
\par 22 وَعَادَ أَيْضاً أَخِيمَعَصُ بْنُ صَادُوقَ فَقَالَ لِيُوآبَ: «مَهْمَا كَانَ فَدَعْنِي أَجْرِ أَنَا أَيْضاً وَرَاءَ كُوشِي». فَقَالَ يُوآبُ: «لِمَاذَا تَجْرِي أَنْتَ يَا ابْنِي وَلَيْسَ لَكَ بِشَارَةٌ تُجَازَى؟»
\par 23 قَالَ: «مَهْمَا كَانَ أَجْرِي». فَقَالَ لَهُ: «اجْرِ». فَجَرَى أَخِيمَعَصُ فِي طَرِيقِ الْغَوْرِ وَسَبَقَ كُوشِيَ.
\par 24 وَكَانَ دَاوُدُ جَالِساً بَيْنَ الْبَابَيْنِ، وَطَلَعَ الرَّقِيبُ إِلَى سَطْحِ الْبَابِ إِلَى السُّورِ وَرَفَعَ عَيْنَيْهِ وَنَظَرَ وَإِذَا بِرَجُلٍ يَجْرِي وَحْدَهُ.
\par 25 فَنَادَى الرَّقِيبُ وَأَخْبَرَ الْمَلِكَ. فَقَالَ الْمَلِكُ: «إِنْ كَانَ وَحْدَهُ فَفِي فَمِهِ بِشَارَةٌ». وَكَانَ يَسْعَى وَيَقْرُبُ.
\par 26 ثُمَّ رَأَى الرَّقِيبُ رَجُلاً آخَرَ يَجْرِي، فَنَادَى الرَّقِيبُ الْبَوَّابَ وَقَالَ: «هُوَذَا رَجُلٌ يَجْرِي وَحْدَهُ». فَقَالَ الْمَلِكُ: «وَهَذَا أَيْضاً مُبَشِّرٌ».
\par 27 وَقَالَ الرَّقِيبُ: «إِنِّي أَرَى جَرْيَ الأَوَّلِ كَجَرْيِ أَخِيمَعَصَ بْنِ صَادُوقَ». فَقَالَ الْمَلِكُ: «هَذَا رَجُلٌ صَالِحٌ وَيَأْتِي بِبِشَارَةٍ صَالِحَةٍ».
\par 28 فَنَادَى أَخِيمَعَصُ وَقَالَ لِلْمَلِكِ: «السَّلاَمُ». وَسَجَدَ لِلْمَلِكِ عَلَى وَجْهِهِ إِلَى الأَرْضِ. وَقَالَ: «مُبَارَكٌ الرَّبُّ إِلَهُكَ الَّذِي دَفَعَ الْقَوْمَ الَّذِينَ رَفَعُوا أَيْدِيَهُمْ عَلَى سَيِّدِي الْمَلِكِ».
\par 29 فَقَالَ الْمَلِكُ: «أَسَلاَمٌ لِلْفَتَى أَبْشَالُومَ؟» فَقَالَ أَخِيمَعَصُ: «قَدْ رَأَيْتُ جُمْهُوراً عَظِيماً عِنْدَ إِرْسَالِ يُوآبَ عَبْدَ الْمَلِكِ وَعَبْدَكَ، وَلَمْ أَعْلَمْ مَاذَا».
\par 30 فَقَالَ الْمَلِكُ: «دُرْ وَقِفْ هَهُنَا». فَدَارَ وَوَقَفَ.
\par 31 وَإِذَا بِكُوشِي قَدْ أَتَى، وَقَالَ كُوشِي: « لِيُبَشَّرْ سَيِّدِي الْمَلِكُ لأَنَّ الرَّبَّ قَدِ انْتَقَمَ لَكَ الْيَوْمَ مِنْ جَمِيعِ الْقَائِمِينَ عَلَيْكَ».
\par 32 فَقَالَ الْمَلِكُ لِكُوشِي: «أَسَلاَمٌ لِلْفَتَى أَبْشَالُومَ؟» فَقَالَ كُوشِي: «لِيَكُنْ كَالْفَتَى أَعْدَاءُ سَيِّدِي الْمَلِكِ وَجَمِيعُ الَّذِينَ قَامُوا عَلَيْكَ لِلشَّرِّ».
\par 33 فَانْزَعَجَ الْمَلِكُ وَصَعِدَ إِلَى عِلِّيَّةِ الْبَابِ وَكَانَ يَبْكِي وَيَقُولُ وَهُوَ يَتَمَشَّى: «يَا ابْنِي أَبْشَالُومُ، يَا ابْنِي يَا ابْنِي! أَبْشَالُومُ، يَا لَيْتَنِي مُتُّ عِوَضاً عَنْكَ! يَا أَبْشَالُومُ ابْنِي يَا ابْنِي».

\chapter{19}

\par 1 فَأُخْبِرَ يُوآبُ: «هُوَذَا الْمَلِكُ يَبْكِي وَيَنُوحُ عَلَى أَبْشَالُومَ».
\par 2 فَصَارَتِ الْغَلَبَةُ فِي ذَلِكَ الْيَوْمِ مَنَاحَةً عِنْدَ جَمِيعِ الشَّعْبِ، لأَنَّ الشَّعْبَ سَمِعُوا فِي ذَلِكَ الْيَوْمِ مَنْ يَقُولُ إِنَّ الْمَلِكَ قَدْ تَأَسَّفَ عَلَى ابْنِهِ.
\par 3 وَتَسَلَّلَ الشَّعْبُ فِي ذَلِكَ الْيَوْمِ لِلدُّخُولِ إِلَى الْمَدِينَةِ كَمَا يَتَسَلَّلُ الْقَوْمُ الْخَجِلُونَ عِنْدَمَا يَهْرُبُونَ فِي الْقِتَالِ.
\par 4 وَسَتَرَ الْمَلِكُ وَجْهَهُ وَصَرَخَ بِصَوْتٍ عَظِيمٍ: «يَا ابْنِي أَبْشَالُومُ، يَا أَبْشَالُومُ ابْنِي يَا ابْنِي!»
\par 5 فَدَخَلَ يُوآبُ إِلَى الْمَلِكِ إِلَى الْبَيْتِ وَقَالَ: «قَدْ أَخْزَيْتَ الْيَوْمَ وُجُوهَ جَمِيعِ عَبِيدِكَ، مُنْقِذِي نَفْسِكَ الْيَوْمَ وَأَنْفُسِ بَنِيكَ وَبَنَاتِكَ وَأَنْفُسِ نِسَائِكَ وَأَنْفُسِ سَرَارِيِّكَ،
\par 6 بِمَحَبَّتِكَ لِمُبْغِضِيكَ وَبُغْضِكَ لِمُحِبِّيكَ. لأَنَّكَ أَظْهَرْتَ الْيَوْمَ أَنَّهُ لَيْسَ لَكَ رُؤَسَاءُ وَلاَ عَبِيدٌ، لأَنِّي عَلِمْتُ الْيَوْمَ أَنَّهُ لَوْ كَانَ أَبْشَالُومُ حَيّاً وَكُلُّنَا الْيَوْمَ مَوْتَى لَحَسُنَ حِينَئِذٍ الأَمْرُ فِي عَيْنَيْكَ.
\par 7 فَالآنَ قُمْ وَاخْرُجْ وَطَيِّبْ قُلُوبَ عَبِيدِكَ. لأَنِّي قَدْ أَقْسَمْتُ بِالرَّبِّ إِنَّهُ إِنْ لَمْ تَخْرُجْ لاَ يَبِيتُ أَحَدٌ مَعَكَ هَذِهِ اللَّيْلَةَ، وَيَكُونُ ذَلِكَ أَشَرَّ عَلَيْكَ مِنْ كُلِّ شَرٍّ أَصَابَكَ مُنْذُ صِبَاكَ إِلَى الآنَ!»
\par 8 فَقَامَ الْمَلِكُ وَجَلَسَ فِي الْبَابِ. فَأَخْبَرُوا جَمِيعَ الشَّعْبِ: «هُوَذَا الْمَلِكُ جَالِسٌ فِي الْبَابِ». فَأَتَى جَمِيعُ الشَّعْبِ أَمَامَ الْمَلِكِ. وَأَمَّا إِسْرَائِيلُ فَهَرَبُوا كُلُّ وَاحِدٍ إِلَى خَيْمَتِهِ.
\par 9 وَكَانَ جَمِيعُ الشَّعْبِ فِي خِصَامٍ فِي جَمِيعِ أَسْبَاطِ إِسْرَائِيلَ قَائِلِينَ: «إِنَّ الْمَلِكَ قَدْ أَنْقَذَنَا مِنْ يَدِ أَعْدَائِنَا وَهُوَ نَجَّانَا مِنْ يَدِ الْفِلِسْطِينِيِّينَ. وَالآنَ قَدْ هَرَبَ مِنَ الأَرْضِ لأَجْلِ أَبْشَالُومَ
\par 10 وَأَبْشَالُومُ الَّذِي مَسَحْنَاهُ عَلَيْنَا قَدْ مَاتَ فِي الْحَرْبِ. فَالآنَ لِمَاذَا أَنْتُمْ سَاكِتُونَ عَنْ إِرْجَاعِ الْمَلِكِ؟»
\par 11 وَأَرْسَلَ الْمَلِكُ دَاوُدُ إِلَى صَادُوقَ وَأَبِيَاثَارَ الْكَاهِنَيْنِ قَائِلاً: «قُولاَ لِشُيُوخِ يَهُوذَا: لِمَاذَا تَكُونُونَ آخِرِينَ فِي إِرْجَاعِ الْمَلِكِ إِلَى بَيْتِهِ، وَقَدْ أَتَى كَلاَمُ جَمِيعِ إِسْرَائِيلَ إِلَى الْمَلِكِ فِي بَيْتِهِ؟
\par 12 أَنْتُمْ إِخْوَتِي. أَنْتُمْ عَظْمِي وَلَحْمِي. فَلِمَاذَا تَكُونُونَ آخِرِينَ فِي إِرْجَاعِ الْمَلِكِ؟
\par 13 وَقُولاَ لِعَمَاسَا: أَمَا أَنْتَ عَظْمِي وَلَحْمِي؟ هَكَذَا يَفْعَلُ بِيَ اللَّهُ وَهَكَذَا يَزِيدُ إِنْ كُنْتَ لاَ تَصِيرُ رَئِيسَ جَيْشٍ عِنْدِي كُلَّ الأَيَّامِ بَدَلَ يُوآبَ».
\par 14 فَاسْتَمَالَ قُلُوبَ جَمِيعِ رِجَالِ يَهُوذَا كَرَجُلٍ وَاحِدٍ، فَأَرْسَلُوا إِلَى الْمَلِكِ قَائِلِينَ: «ارْجِعْ أَنْتَ وَجَمِيعُ عَبِيدِكَ».
\par 15 فَرَجَعَ الْمَلِكُ وَأَتَى إِلَى الأُرْدُنِّ، وَأَتَى يَهُوذَا إِلَى الْجِلْجَالِ سَائِراً لِمُلاَقَاةِ الْمَلِكِ لِيُعَبِّرَ الْمَلِكَ الأُرْدُنَّ.
\par 16 فَبَادَرَ شَمْعِي بْنُ جِيْرَا الْبِنْيَامِينِيُّ الَّذِي مِنْ بَحُورِيمَ وَنَزَلَ مَعَ رِجَالِ يَهُوذَا لِلِقَاءِ الْمَلِكِ دَاوُدَ
\par 17 وَمَعَهُ أَلْفُ رَجُلٍ مِنْ بِنْيَامِينَ، وَصِيبَا غُلاَمُ بَيْتِ شَاوُلَ وَبَنُوهُ الْخَمْسَةَ عَشَرَ وَعَبِيدُهُ الْعِشْرُونَ مَعَهُ، فَخَاضُوا الأُرْدُنَّ أَمَامَ الْمَلِكِ.
\par 18 وَعَبَرَ الْقَارِبُ لِتَعْبِيرِ بَيْتِ الْمَلِكِ وَلِعَمَلِ مَا يَحْسُنُ فِي عَيْنَيْهِ. وَسَقَطَ شَمْعِي بْنُ جِيْرَا أَمَامَ الْمَلِكِ عِنْدَمَا عَبَرَ الأُرْدُنَّ
\par 19 وَقَالَ لِلْمَلِكِ: «لاَ يَحْسِبْ لِي سَيِّدِي إِثْماً، وَلاَ تَذْكُرْ مَا افْتَرَى بِهِ عَبْدُكَ يَوْمَ خُرُوجِ سَيِّدِي الْمَلِكِ مِنْ أُورُشَلِيمَ حَتَّى يَضَعَ الْمَلِكُ ذَلِكَ فِي قَلْبِهِ.
\par 20 لأَنَّ عَبْدَكَ يَعْلَمُ أَنِّي قَدْ أَخْطَأْتُ، وَهَئَنَذَا قَدْ جِئْتُ الْيَوْمَ أَوَّلَ كُلِّ بَيْتِ يُوسُفَ وَنَزَلْتُ لِلِقَاءِ سَيِّدِي الْمَلِكِ».
\par 21 فَقَالَ أَبِيشَايُ ابْنُ صَرُويَةَ: «أَلاَ يُقْتَلُ شَمْعِي لأَنَّهُ سَبَّ مَسِيحَ الرَّبِّ؟»
\par 22 فَقَالَ دَاوُدُ: «مَا لِي وَلَكُمْ يَا بَنِي صَرُويَةَ حَتَّى تَكُونُوا لِيَ الْيَوْمَ مُقَاوِمِينَ؟ آلْيَوْمَ يُقْتَلُ أَحَدٌ فِي إِسْرَائِيلَ؟ أَفَمَا عَلِمْتُ أَنِّي الْيَوْمَ مَلِكٌ عَلَى إِسْرَائِيلَ؟»
\par 23 ثُمَّ قَالَ الْمَلِكُ لِشَمْعِي: «لاَ تَمُوتُ». وَحَلَفَ لَهُ الْمَلِكُ.
\par 24 وَنَزَلَ مَفِيبُوشَثُ ابْنُ شَاوُلَ لِلِقَاءِ الْمَلِكِ وَلَمْ يَعْتَنِ بِرِجْلَيْهِ وَلاَ اعْتَنَى بِلِحْيَتِهِ وَلاَ غَسَلَ ثِيَابَهُ مِنَ الْيَوْمِ الَّذِي ذَهَبَ فِيهِ الْمَلِكُ إِلَى الْيَوْمِ الَّذِي أَتَى فِيهِ بِسَلاَمٍ.
\par 25 فَلَمَّا جَاءَ إِلَى أُورُشَلِيمَ لِلِقَاءِ الْمَلِكِ قَالَ لَهُ الْمَلِكُ: «لِمَاذَا لَمْ تَذْهَبْ مَعِي يَا مَفِيبُوشَثُ؟»
\par 26 فَقَالَ: «يَا سَيِّدِي الْمَلِكُ إِنَّ عَبْدِي قَدْ خَدَعَنِي، لأَنَّ عَبْدَكَ قَالَ: أَشُدُّ لِنَفْسِيَ الْحِمَارَ فَأَرْكَبُ عَلَيْهِ وَأَذْهَبُ مَعَ الْمَلِكِ، لأَنَّ عَبْدَكَ أَعْرَجُ.
\par 27 وَوَشَى بِعَبْدِكَ إِلَى سَيِّدِي الْمَلِكِ، وَسَيِّدِي الْمَلِكُ كَمَلاَكِ اللَّهِ. فَافْعَلْ مَا يَحْسُنُ فِي عَيْنَيْكَ.
\par 28 لأَنَّ كُلَّ بَيْتِ أَبِي لَمْ يَكُنْ إِلاَّ أُنَاساً مَوْتَى لِسَيِّدِي الْمَلِكِ، وَقَدْ جَعَلْتَ عَبْدَكَ بَيْنَ الآكِلِينَ عَلَى مَائِدَتِكَ. فَأَيُّ حَقٍّ لِي بَعْدُ حَتَّى أَصْرُخَ أَيْضاً إِلَى الْمَلِكِ؟»
\par 29 فَقَالَ لَهُ الْمَلِكُ: «لِمَاذَا تَتَكَلَّمُ بَعْدُ بِأُمُورِكَ؟ قَدْ قُلْتُ إِنَّكَ أَنْتَ وَصِيبَا تَقْسِمَانِ الْحَقْلَ».
\par 30 فَقَالَ مَفِيبُوشَثُ لِلْمَلِكِ: «فَلْيَأْخُذِ الْكُلَّ أَيْضاً بَعْدَ أَنْ جَاءَ سَيِّدِي الْمَلِكُ بِسَلاَمٍ إِلَى بَيْتِهِ».
\par 31 وَنَزَلَ بَرْزِلاَّيُ الْجِلْعَادِيُّ مِنْ رُوجَلِيمَ وَعَبَرَ الأُرْدُنَّ مَعَ الْمَلِكِ لِيُشَيِّعَهُ عِنْدَ الأُرْدُنِّ.
\par 32 وَكَانَ بَرْزِلاَّيُ قَدْ شَاخَ جِدّاً - كَانَ ابْنَ ثَمَانِينَ سَنَةً. وَهُوَ عَالَ الْمَلِكَ عِنْدَ إِقَامَتِهِ فِي مَحَنَايِمَ لأَنَّهُ كَانَ رَجُلاً عَظِيماً جِدّاً.
\par 33 فَقَالَ الْمَلِكُ لِبَرْزِلاَّيَ: «اعْبُرْ أَنْتَ مَعِي وَأَنَا أَعُولُكَ مَعِي فِي أُورُشَلِيمَ».
\par 34 فَقَالَ بَرْزِلاَّيُ لِلْمَلِكِ: «كَمْ أَيَّامُ سِنِي حَيَاتِي حَتَّى أَصْعَدَ مَعَ الْمَلِكِ إِلَى أُورُشَلِيمَ؟
\par 35 أَنَا الْيَوْمَ ابْنُ ثَمَانِينَ سَنَةً. هَلْ أُمَيِّزُ بَيْنَ الطَّيِّبِ وَالرَّدِيءِ، وَهَلْ يَسْتَطْعِمُ عَبْدُكَ بِمَا آكُلُ وَمَا أَشْرَبُ، وَهَلْ أَسْمَعُ أَيْضاً أَصْوَاتَ الْمُغَنِّينَ وَالْمُغَنِّيَاتِ؟ فَلِمَاذَا يَكُونُ عَبْدُكَ أَيْضاً ثِقْلاً عَلَى سَيِّدِي الْمَلِكِ؟
\par 36 يَعْبُرُ عَبْدُكَ قَلِيلاً الأُرْدُنَّ مَعَ الْمَلِكِ. وَلِمَاذَا يُكَافِئُنِي الْمَلِكُ بِهَذِهِ الْمُكَافَأَةِ؟
\par 37 دَعْ عَبْدَكَ يَرْجِعُ فَأَمُوتَ فِي مَدِينَتِي عِنْدَ قَبْرِ أَبِي وَأُمِّي. وَهُوَذَا عَبْدُكَ كِمْهَامُ يَعْبُرُ مَعَ سَيِّدِي الْمَلِكِ فَافْعَلْ لَهُ مَا يَحْسُنُ فِي عَيْنَيْكَ».
\par 38 فَأَجَابَ الْمَلِكُ: «إِنَّ كِمْهَامَ يَعْبُرُ مَعِي فَأَفْعَلُ لَهُ مَا يَحْسُنُ فِي عَيْنَيْكَ، وَكُلُّ مَا تَتَمَنَّاهُ مِنِّي أَفْعَلُهُ لَكَ».
\par 39 فَعَبَرَ جَمِيعُ الشَّعْبِ الأُرْدُنَّ، وَالْمَلِكُ عَبَرَ. وَقَبَّلَ الْمَلِكُ بَرْزِلاَّيَ وَبَارَكَهُ فَرَجَعَ إِلَى مَكَانِهِ.
\par 40 وَعَبَرَ الْمَلِكُ إِلَى الْجِلْجَالِ وَعَبَرَ كِمْهَامُ مَعَهُ، وَكُلُّ شَعْبِ يَهُوذَا عَبَّرُوا الْمَلِكَ، وَكَذَلِكَ نِصْفُ شَعْبِ إِسْرَائِيلَ.
\par 41 وَإِذَا بِجَمِيعِ رِجَالِ إِسْرَائِيلَ جَاءُونَ إِلَى الْمَلِكِ، وَقَالُوا لِلْمَلِكِ: «لِمَاذَا سَرِقَكَ إِخْوَتُنَا رِجَالُ يَهُوذَا وَعَبَرُوا الأُرْدُنَّ بِالْمَلِكِ وَبَيْتِهِ وَكُلِّ رِجَالِ دَاوُدَ مَعَهُ؟»
\par 42 فَأَجَابَ كُلُّ رِجَالِ يَهُوذَا رِجَالَ إِسْرَائِيلَ: «لأَنَّ الْمَلِكَ قَرِيبٌ إِلَيَّ. وَلِمَاذَا تَغْتَاظُ مِنْ هَذَا الأَمْرِ؟ هَلْ أَكَلْنَا شَيْئاً مِنَ الْمَلِكِ أَوْ وَهَبَنَا هِبَةً؟»
\par 43 فَأَجَابَ رِجَالُ إِسْرَائِيلَ رِجَالَ يَهُوذَا: «لِي عَشَرَةُ أَسْهُمٍ فِي الْمَلِكِ، وَأَنَا أَحَقُّ مِنْكَ بِدَاوُدَ. فَلِمَاذَا اسْتَخْفَفْتَ بِي وَلَمْ يَكُنْ كَلاَمِي أَوَّلاً فِي إِرْجَاعِ مَلِكِي؟» وَكَانَ كَلاَمُ رِجَالِ يَهُوذَا أَقْسَى مِنْ كَلاَمِ رِجَالِ إِسْرَائِيلَ.

\chapter{20}

\par 1 وَاتَّفَقَ هُنَاكَ رَجُلٌ لَئِيمٌ اسْمُهُ شَبَعُ بْنُ بِكْرِي رَجُلٌ بِنْيَامِينِيٌّ، فَضَرَبَ بِالْبُوقِ وَقَالَ: «لَيْسَ لَنَا قِسْمٌ فِي دَاوُدَ وَلاَ لَنَا نَصِيبٌ فِي ابْنِ يَسَّى. كُلُّ رَجُلٍ إِلَى خَيْمَتِهِ يَا إِسْرَائِيلُ».
\par 2 فَصَعِدَ كُلُّ رِجَالِ إِسْرَائِيلَ مِنْ وَرَاءِ دَاوُدَ إِلَى وَرَاءِ شَبَعَ بْنِ بِكْرِي. وَأَمَّا رِجَالُ يَهُوذَا فَلاَزَمُوا مَلِكَهُمْ مِنَ الأُرْدُنِّ إِلَى أُورُشَلِيمَ.
\par 3 وَجَاءَ دَاوُدُ إِلَى بَيْتِهِ فِي أُورُشَلِيمَ. وَأَخَذَ الْمَلِكُ النِّسَاءَ السَّرَارِيَّ الْعَشَرَ اللَّوَاتِي تَرَكَهُنَّ لِحِفْظِ الْبَيْتِ، وَجَعَلَهُنَّ تَحْتَ حَجْزٍ، وَكَانَ يَعُولُهُنَّ وَلَكِنْ لَمْ يَدْخُلْ إِلَيْهِنَّ، بَلْ كُنَّ مَحبُوسَاتٍ إِلَى يَوْمِ مَوْتِهِنَّ فِي عِيشَةِ الْعُزُوبَةِ.
\par 4 وَقَالَ الْمَلِكُ لِعَمَاسَا: «اجْمَعْ لِي رِجَالَ يَهُوذَا فِي ثَلاَثَةِ أَيَّامٍ، وَاحْضُرْ أَنْتَ هُنَا».
\par 5 فَذَهَبَ عَمَاسَا لِيَجْمَعَ يَهُوذَا، وَلَكِنَّهُ تأَخَّرَ عَنِ الْمِيقَاتِ الَّذِي عَيَّنَهُ.
\par 6 فَقَالَ دَاوُدُ لأَبِيشَايَ: «الآنَ يُسِيءُ إِلَيْنَا شَبَعُ بْنُ بِكْرِي أَكْثَرَ مِنْ أَبْشَالُومَ. فَخُذْ أَنْتَ عَبِيدَ سَيِّدِكَ وَاتْبَعْهُ لِئَلاَّ يَجِدَ لِنَفْسِهِ مُدُناً حَصِينَةً وَيَنْفَلِتَ مِنْ أَمَامِ أَعْيُنِنَا».
\par 7 فَخَرَجَ وَرَاءَهُ رِجَالُ يُوآبَ: الْجَلاَّدُونَ وَالسُّعَاةُ وَجَمِيعُ الأَبْطَالِ، وَخَرَجُوا مِن أُورُشَلِيمَ لِيَتْبَعُوا شَبَعَ بْنَ بِكْرِي.
\par 8 وَلَمَّا كَانُوا عِنْدَ الصَّخْرَةِ الْعَظِيمَةِ الَّتِي فِي جِبْعُونَ جَاءَ عَمَاسَا قُدَّامَهُمْ. وَكَانَ يُوآبُ مُتَنَطِّقاً عَلَى ثَوْبِهِ الَّذِي كَانَ لاَبِسَهُ، وَفَوْقَهُ مِنْطَقَةُ سَيْفٍ فِي غِمْدِهِ مَشْدُودَةٌ عَلَى حَقَوَيْهِ، فَلَمَّا خَرَجَ انْدَلَقَ السَّيْفُ.
\par 9 فَقَالَ يُوآبُ لِعَمَاسَا: «أَسَالِمٌ أَنْتَ يَا أَخِي؟ وَأَمْسَكَتْ يَدُ يُوآبَ الْيُمْنَى بِلِحْيَةِ عَمَاسَا لِيُقَبِّلَهُ».
\par 10 وَأَمَّا عَمَاسَا فَلَمْ يَحْتَرِزْ مِنَ السَّيْفِ الَّذِي بِيَدِ يُوآبَ، فَضَرَبَهُ بِهِ فِي بَطْنِهِ فَدَلَقَ أَمْعَاءَهُ إِلَى الأَرْضِ وَلَمْ يُثَنِّ عَلَيْهِ، فَمَاتَ. وَأَمَّا يُوآبُ وَأَبِيشَايُ أَخُوهُ فَتَبِعَا شَبَعَ بْنَ بِكْرِي.
\par 11 وَوَقَفَ عَُِنْدَهُ وَاحِدٌ مِنْ غِلْمَانِ يُوآبَ، فَقَالَ: «مَنْ سُرَّ بِيُوآبَ، وَمَنْ هُوَ لِدَاوُدَ، فَوَرَاءَ يُوآبَ».
\par 12 وَكَانَ عَمَاسَا يَتَمَرَّغُ فِي الدَّمِ فِي وَسَطِ السِّكَّةِ. وَلَمَّا رَأَى الرَّجُلُ أَنَّ كُلَّ الشَّعْبِ يَقِفُونَ، نَقَلَ عَمَاسَا مِنَ السِّكَّةِ إِلَى الْحَقْلِ وَطَرَحَ عَلَيْهِ ثَوْباً، لَمَّا رَأَى أَنَّ كُلَّ مَنْ يَصِلُ إِلَيْهِ يَقِفُ.
\par 13 فَلَمَّا نُقِلَ عَنِ السِّكَّةِ عَبَرَ كُلُّ إِنْسَانٍ وَرَاءَ يُوآبَ لاِتِّبَاعِ شَبَعَ بْنِ بِكْرِي.
\par 14 وَعَبَرَ فِي جَمِيعِ أَسْبَاطِ إِسْرَائِيلَ إِلَى آبَلَ وَبَيْتِ مَعْكَةَ وَجَمِيعِ الْبِيرِيِّينَ، فَاجْتَمَعُوا وَخَرَجُوا أَيْضاً وَرَاءَهُ.
\par 15 وَجَاءُوا وَحَاصَرُوهُ فِي آبَلِ بَيْتِ مَعْكَةَ وَأَقَامُوا مِتْرَسَةً حَوْلَ الْمَدِينَةِ فَأَقَامَتْ فِي الْحِصَارِ، وَجَمِيعُ الشَّعْبِ الَّذِينَ مَعَ يُوآبَ كَانُوا يُخْرِبُونَ لأَجْلِ إِسْقَاطِ السُّورِ.
\par 16 فَنَادَتِ امْرَأَةٌ حَكِيمَةٌ مِنَ الْمَدِينَةِ: «اِسْمَعُوا. اسْمَعُوا. قُولُوا لِيُوآبَ تَقَدَّمْ إِلَى هَهُنَا فَأُكَلِّمَكَ».
\par 17 فَتَقَدَّمَ إِلَيْهَا، فَقَالَتِ الْمَرْأَةُ: «أَأَنْتَ يُوآبُ؟» فَقَالَ: «أَنَا هُوَ». فَقَالَتْ لَهُ: «اسْمَعْ كَلاَمَ أَمَتِكَ». فَقَالَ: «أَنَا سَامِعٌ».
\par 18 فَقَالَتْ: «كَانُوا يَتَكَلَّمُونَ أَوَّلاً قَائِلِينَ: سُؤَالاً يَسْأَلُونَ فِي آبَلَ. وَهَكَذَا كَانُوا انْتَهَوْا.
\par 19 أَنَا مُسَالِمَةٌ أَمِينَةٌ فِي إِسْرَائِيلَ. أَنْتَ طَالِبٌ أَنْ تُمِيتَ مَدِينَةً وَأُمّاً فِي إِسْرَائِيلَ. لِمَاذَا تَبْلَعُ نَصِيبَ الرَّبِّ؟»
\par 20 فَأَجَابَ يُوآبُ: «حَاشَايَ! حَاشَايَ أَنْ أَبْلَعَ وَأَنْ أُهْلِكَ.
\par 21 الأَمْرُ لَيْسَ كَذَلِكَ. لأَنَّ رَجُلاً مِنْ جَبَلِ أَفْرَايِمَ اسْمُهُ شَبَعُ بْنُ بِكْرِي رَفَعَ يَدَهُ عَلَى الْمَلِكِ دَاوُدَ. سَلِّمُوهُ وَحْدَهُ فَأَنْصَرِفَ عَنِ الْمَدِينَةِ». فَقَالَتِ الْمَرْأَةُ لِيُوآبَ: «هُوَذَا رَأْسُهُ يُلْقَى إِلَيْكَ عَنِ السُّورِ».
\par 22 فَأَتَتِ الْمَرْأَةُ إِلَى جَمِيعِ الشَّعْبِ بِحِكْمَتِهَا فَقَطَعُوا رَأْسَ شَبَعَ بْنِ بِكْرِي وَأَلْقُوهُ إِلَى يُوآبَ، فَضَرَبَ بِالْبُوقِ فَانْصَرَفُوا عَنِ الْمَدِينَةِ كُلُّ وَاحِدٍ إِلَى خَيْمَتِهِ. وَأَمَّا يُوآبُ فَرَجَعَ إِلَى أُورُشَلِيمَ إِلَى الْمَلِكِ.
\par 23 وَكَانَ يُوآبُ عَلَى جَمِيعِ جَيْشِ إِسْرَائِيلَ، وَبَنَايَا بْنُ يَهُويَادَاعَ عَلَى الْجَلاَّدِينَ وَالسُّعَاةِ،
\par 24 وَأَدُورَامُ عَلَى الْجِزْيَةِ، وَيَهُوشَافَاطُ بْنُ أَخِيلُودَ مُسَجِّلاً،
\par 25 وَشِيوَا كَاتِباً، وَصَادُوقُ وَأَبِيَاثَارُ كَاهِنَيْنِ،
\par 26 وَعِيْرَا الْيَائِيرِيُّ أَيْضاً كَانَ كَاهِناً لِدَاوُدَ.

\chapter{21}

\par 1 وَكَانَ جُوعٌ فِي أَيَّامِ دَاوُدَ ثَلاَثَ سِنِينَ، سَنَةً بَعْدَ سَنَةٍ. فَطَلَبَ دَاوُدُ وَجْهَ الرَّبِّ. فَقَالَ الرَّبُّ: «هُوَ لأَجْلِ شَاوُلَ وَلأَجْلِ بَيْتِ الدِّمَاءِ، لأَنَّهُ قَتَلَ الْجِبْعُونِيِّينَ».
\par 2 (وَالْجِبْعُونِيُّونَ لَيْسُوا مِنْ بَنِي إِسْرَائِيلَ بَلْ مِنْ بَقَايَا الأَمُورِيِّينَ، وَقَدْ حَلَفَ لَهُمْ بَنُو إِسْرَائِيلَ، وَطَلَبَ شَاوُلُ أَنْ يَقْتُلَهُمْ لأَجْلِ غَيْرَتِهِ عَلَى بَنِي إِسْرَائِيلَ وَيَهُوذَا)
\par 3 فَدَعَا الْمَلِكُ الْجِبْعُونِيِّينَ وَقَالَ لَهُمْ: «مَاذَا أَفْعَلُ لَكُمْ وَبِمَاذَا أُكَفِّرُ فَتُبَارِكُوا نَصِيبَ الرَّبِّ؟»
\par 4 فَقَالَ لَهُ الْجِبْعُونِيُّونَ: «لَيْسَ لَنَا فِضَّةٌ وَلاَ ذَهَبٌ عِنْدَ شَاوُلَ وَلاَ عِنْدَ بَيْتِهِ، وَلَيْسَ لَنَا أَنْ نُمِيتَ أَحَداً فِي إِسْرَائِيلَ». فَقَالَ: «مَهْمَا قُلْتُمْ أَفْعَلُهُ لَكُمْ».
\par 5 فَقَالُوا لِلْمَلِكِ: «الرَّجُلُ الَّذِي أَفْنَانَا وَالَّذِي تَآمَرَ عَلَيْنَا لِيُبِيدَنَا لِكَيْ لاَ نُقِيمَ فِي كُلِّ تُخُومِ إِسْرَائِيلَ،
\par 6 فَلْنُعْطَ سَبْعَةَ رِجَالٍ مِنْ بَنِيهِ فَنَصْلِبَهُمْ لِلرَّبِّ فِي جِبْعَةِ شَاوُلَ مُخْتَارِ الرَّبِّ». فَقَالَ الْمَلِكُ: «أَنَا أُعْطِي».
\par 7 وَأَشْفَقَ الْمَلِكُ عَلَى مَفِيبُوشَثَ بْنِ يُونَاثَانَ بْنِ شَاوُلَ مِنْ أَجْلِ يَمِينِ الرَّبِّ الَّتِي بَيْنَ دَاوُدَ وَيُونَاثَانَ بْنِ شَاوُلَ.
\par 8 فَأَخَذَ الْمَلِكُ ابْنَيْ رِصْفَةَ ابْنَةِ أَيَّةَ اللَّذَيْنِ وَلَدَتْهُمَا لِشَاوُلَ: أَرْمُونِيَ وَمَفِيبُوشَثَ، وَبَنِي مِيكَالَ ابْنَةِ شَاوُلَ الْخَمْسَةَ الَّذِينَ وَلَدَتْهُمْ لِعَدْرِئِيلَ بْنِ بَرْزِلاَّيَ الْمَحُولِيِّ،
\par 9 وَسَلَّمَهُمْ إِلَى يَدِ الْجِبْعُونِيِّينَ فَصَلَبُوهُمْ عَلَى الْجَبَلِ أَمَامَ الرَّبِّ. فَسَقَطَ السَّبْعَةُ مَعاً وَقُتِلُوا فِي أَيَّامِ الْحَصَادِ فِي أَوَّلِهَا فِي ابْتِدَاءِ حَصَادِ الشَّعِيرِ.
\par 10 فَأَخَذَتْ رِصْفَةُ ابْنَةُ أَيَّةَ مِسْحاً وَفَرَشَتْهُ لِنَفْسِهَا عَلَى الصَّخْرِ مِنِ ابْتِدَاءِ الْحَصَادِ حَتَّى انْصَبَّ الْمَاءُ عَلَيْهِمْ مِنَ السَّمَاءِ، وَلَمْ تَدَعْ طُيُورَ السَّمَاءِ تَنْزِلُ عَلَيْهِمْ نَهَاراً وَلاَ حَيَوَانَاتِ الْحَقْلِ لَيْلاً.
\par 11 فَأُخْبِرَ دَاوُدُ بِمَا فَعَلَتْ رِصْفَةُ ابْنَةُ أَيَّةَ سُرِّيَّةُ شَاوُلَ.
\par 12 فَذَهَبَ دَاوُدُ وَأَخَذَ عِظَامَ شَاوُلَ وَعِظَامَ يُونَاثَانَ ابْنِهِ مِنْ أَهْلِ يَابِيشِ جِلْعَادَ الَّذِينَ سَرِقُوهَا مِنْ شَارِعِ بَيْتِ شَانَ، حَيْثُ عَلَّقَهُمَا الْفِلِسْطِينِيُّونَ يَوْمَ ضَرَبَ الْفِلِسْطِينِيُّونَ شَاوُلَ فِي جِلْبُوعَ.
\par 13 فَأَصْعَدَ مِنْ هُنَاكَ عِظَامَ شَاوُلَ وَعِظَامَ يُونَاثَانَ ابْنِهِ، وَجَمَعُوا عِظَامَ الْمَصْلُوبِينَ،
\par 14 وَدَفَنُوا عِظَامَ شَاوُلَ وَيُونَاثَانَ ابْنِهِ فِي أَرْضِ بِنْيَامِينَ فِي صَيْلَعَ فِي قَبْرِ قَيْسَ أَبِيهِ، وَعَمِلُوا كُلَّ مَا أَمَرَ بِهِ الْمَلِكُ. وَبَعْدَ ذَلِكَ اسْتَجَابَ اللَّهُ مِنْ أَجْلِ الأَرْضِ.
\par 15 وَكَانَتْ أَيْضاً حَرْبٌ بَيْنَ الْفِلِسْطِينِيِّينَ وَإِسْرَائِيلَ، فَانْحَدَرَ دَاوُدُ وَعَبِيدُهُ مَعَهُ وَحَارَبُوا الْفِلِسْطِينِيِّينَ، فَأَعْيَا دَاوُدُ.
\par 16 وَيِشْبِي بَنُوبُ الَّذِي مِنْ أَوْلاَدِ رَافَا، وَوَزْنُ رُمْحِهِ ثَلاَثُ مِئَةِ شَاقِلِ نُحَاسٍ وَقَدْ تَقَلَّدَ جَدِيداً، افْتَكَرَ أَنْ يَقْتُلَ دَاوُدَ.
\par 17 فَأَنْجَدَهُ أَبِيشَايُ ابْنُ صَرُويَةَ فَضَرَبَ الْفِلِسْطِينِيَّ وَقَتَلَهُ. حِينَئِذٍ حَلَفَ رِجَالُ دَاوُدَ لَهُ قَائِلِينَ: «لاَ تَخْرُجُ أَيْضاً مَعَنَا إِلَى الْحَرْبِ، وَلاَ تُطْفِئُ سِرَاجَ إِسْرَائِيلَ».
\par 18 ثُمَّ بَعْدَ ذَلِكَ كَانَتْ أَيْضاً حَرْبٌ فِي جُوبَ مَعَ الْفِلِسْطِينِيِّينَ. حِينَئِذٍ سَبْكَايُ الْحُوشِيُّ قَتَلَ سَافَ الَّذِي هُوَ مِنْ أَوْلاَدِ رَافَا.
\par 19 ثُمَّ كَانَتْ أَيْضاً حَرْبٌ فِي جُوبَ مَعَ الْفِلِسْطِينِيِّينَ. فَأَلْحَانَانُ بْنُ يَعْرِي أُرَجِيمَ الْبَيْتَلَحْمِيُّ قَتَلَ جُلْيَاتَ الْجَتِّيَّ، وَكَانَتْ قَنَاةُ رُمْحِهِ كَنَوْلِ النَّسَّاجِينَ.
\par 20 وَكَانَتْ أَيْضاً حَرْبٌ فِي جَتَّ، وَكَانَ رَجُلٌ طَوِيلَ الْقَامَةِ أَصَابِعُ كُلٍّ مِنْ يَدَيْهِ سِتٌّ، وَأَصَابِعُ كُلٍّ مِنْ رِجْلَيْهِ سِتٌّ (عَدَدُهَا أَرْبَعٌ وَعِشْرُونَ) وَهُوَ أَيْضاً وُلِدَ لِرَافَا.
\par 21 وَلَمَّا عَيَّرَ إِسْرَائِيلَ ضَرَبَهُ يُونَاثَانُ بْنُ شَمْعَى أَخِي دَاوُدَ.
\par 22 هَؤُلاَءِ الأَرْبَعَةُ وُلِدُوا لِرَافَا فِي جَتَّ وَسَقَطُوا بِيَدِ دَاوُدَ وَبِيَدِ عَبِيدِهِ.

\chapter{22}

\par 1 وَكَلَّمَ دَاوُدُ الرَّبَّ بِكَلاَمِ هَذَا النَّشِيدِ فِي الْيَوْمِ الَّذِي أَنْقَذَهُ فِيهِ الرَّبُّ مِنْ أَيْدِي كُلِّ أَعْدَائِهِ وَمِنْ يَدِ شَاوُلَ،
\par 2 فَقَالَ: «اَلرَّبُّ صَخْرَتِي وَحِصْنِي وَمُنْقِذِي،
\par 3 إِلَهُ صَخْرَتِي بِهِ أَحْتَمِي. تُرْسِي وَقَرْنُ خَلاَصِي. مَلْجَإِي وَمَنَاصِي. مُخَلِّصِي، مِنَ الظُّلْمِ تُخَلِّصُنِي.
\par 4 أَدْعُو الرَّبَّ الْحَمِيدَ فَأَتَخَلَّصُ مِنْ أَعْدَائِي.
\par 5 لأَنَّ أَمْوَاجَ الْمَوْتِ اكْتَنَفَتْنِي. سُيُولُ الْهَلاَكِ أَفْزَعَتْنِي.
\par 6 حِبَالُ الْهَاوِيَةِ أَحَاطَتْ بِي. شُرُكُ الْمَوْتِ أَصَابَتْنِي.
\par 7 فِي ضِيقِي دَعَوْتُ الرَّبَّ وَإِلَى إِلَهِي صَرَخْتُ، فَسَمِعَ مِنْ هَيْكَلِهِ صَوْتِي وَصُرَاخِي دَخَلَ أُذُنَيْهِ.
\par 8 فَارْتَجَّتِ الأَرْضُ وَارْتَعَشَتْ. أُسُسُ السَّمَوَاتِ ارْتَعَدَتْ وَارْتَجَّتْ، لأَنَّهُ غَضِبَ.
\par 9 صَعِدَ دُخَانٌ مِنْ أَنْفِهِ، وَنَارٌ مِنْ فَمِهِ أَكَلَتْ. جَمْرٌ اشْتَعَلَتْ مِنْهُ.
\par 10 طَأْطَأَ السَّمَاوَاتِ وَنَزَلَ وَضَبَابٌ تَحْتَ رِجْلَيْهِ.
\par 11 رَكِبَ عَلَى كَرُوبٍ وَطَارَ، وَرُئِيَ عَلَى أَجْنِحَةِ الرِّيحِ.
\par 12 جَعَلَ الظُّلْمَةَ حَوْلَهُ مَظَلاَّتٍ، مِيَاهاً مُتَجَمِّعَةً وَظَلاَمَ الْغَمَامِ.
\par 13 مِنَ الشُّعَاعِ قُدَّامَهُ اشْتَعَلَتْ جَمْرُ نَارٍ.
\par 14 أَرْعَدَ الرَّبُّ مِنَ السَّمَاوَاتِ، وَالْعَلِيُّ أَعْطَى صَوْتَهُ.
\par 15 أَرْسَلَ سِهَاماً فَشَتَّتَهُمْ، بَرْقاً فَأَزْعَجَهُمْ.
\par 16 فَظَهَرَتْ أَعْمَاقُ الْبَحْرِ، وَانْكَشَفَتْ أُسُسُ الْمَسْكُونَةِ مِنْ زَجْرِ الرَّبِّ، مِنْ نَسْمَةِ رِيحِ أَنْفِهِ.
\par 17 أَرْسَلَ مِنَ الْعُلَى فَأَخَذَنِي. نَشَلَنِي مِنْ مِيَاهٍ كَثِيرَةٍ.
\par 18 أَنْقَذَنِي مِنْ عَدُوِّيَ الْقَوِيِّ، مِنْ مُبْغِضِيَّ لأَنَّهُمْ أَقْوَى مِنِّي.
\par 19 أَصَابُونِي فِي يَوْمِ بَلِيَّتِي وَكَانَ الرَّبُّ سَنَدِي.
\par 20 أَخْرَجَنِي إِلَى الرُّحْبِ. خَلَّصَنِي لأَنَّهُ سُرَّ بِي.
\par 21 يُكَافِئُنِي الرَّبُّ حَسَبَ بِرِّي. حَسَبَ طَهَارَةِ يَدَيَّ يَرُدُّ عَلَيَّ.
\par 22 لأَنِّي حَفِظْتُ طُرُقَ الرَّبِّ وَلَمْ أَعْصِ إِلَهِي.
\par 23 لأَنَّ جَمِيعَ أَحْكَامِهِ أَمَامِي وَفَرَائِضُهُ لاَ أَحِيدُ عَنْهَا.
\par 24 وَأَكُونُ كَامِلاً لَدَيْهِ وَأَتَحَفَّظُ مِنْ إِثْمِي.
\par 25 فَيَرُدُّ الرَّبُّ عَلَيَّ كَبِرِّي وَكَطَهَارَتِي أَمَامَ عَيْنَيْهِ.
\par 26 «مَعَ الرَّحِيمِ تَكُونُ رَحِيماً. مَعَ الرَّجُلِ الْكَامِلِ تَكُونُ كَامِلاً.
\par 27 مَعَ الطَّاهِرِ تَكُونُ طَاهِراً وَمَعَ الأَعْوَجِ تَكُونُ مُلْتَوِياً.
\par 28 وَتُخَلِّصُ الشَّعْبَ الْبَائِسَ، وَعَيْنَاكَ عَلَى الْمُتَرَفِّعِينَ فَتَضَعُهُمْ.
\par 29 لأَنَّكَ أَنْتَ سِرَاجِي يَا رَبُّ، وَالرَّبُّ يُضِيءُ ظُلْمَتِي.
\par 30 لأَنِّي بِكَ اقْتَحَمْتُ جَيْشاً. بِإِلَهِي تَسَوَّرْتُ أَسْوَاراً.
\par 31 اَللَّهُ طَرِيقُهُ كَامِلٌ وَقَوْلُ الرَّبِّ نَقِيٌّ. تُرْسٌ هُوَ لِجَمِيعِ الْمُحْتَمِينَ بِهِ.
\par 32 لأَنَّهُ مَنْ هُوَ إِلَهٌ غَيْرُ الرَّبِّ، وَمَنْ هُوَ صَخْرَةٌ غَيْرُ إِلَهِنَا؟
\par 33 الإِلَهُ الَّذِي يُعَزِّزُنِي بِالْقُوَّةِ، وَيُصَيِّرُ طَرِيقِي كَامِلاً.
\par 34 الَّذِي يَجْعَلُ رِجْلَيَّ كَالإِيَّلِ وَعَلَى مُرْتَفَعَاتِي يُقِيمُنِي
\par 35 الَّذِي يُعَلِّمُ يَدَيَّ الْقِتَالَ فَتُحْنَى بِذِرَاعَيَّ قَوْسٌ مِنْ نُحَاسٍ.
\par 36 وَتَجْعَلُ لِي تُرْسَ خَلاَصِكَ وَلُطْفُكَ يُعَظِّمُنِي.
\par 37 تُوَسِّعُ خَطَوَاتِي تَحْتِي فَلَمْ تَتَقَلْقَلْ كَعْبَايَ.
\par 38 أَلْحَقُ أَعْدَائِي فَأُهْلِكُهُمْ، وَلاَ أَرْجِعُ حَتَّى أُفْنِيَهُمْ.
\par 39 أُفْنِيهِمْ وَأَسْحَقُهُمْ فَلاَ يَقُومُونَ، بَلْ يَسْقُطُونَ تَحْتَ رِجْلَيَّ.
\par 40 «تُنَطِّقُنِي قُوَّةً لِلْقِتَالِ، وَتَصْرَعُ الْقَائِمِينَ عَلَيَّ تَحْتِي.
\par 41 وَتُعْطِينِي أَقْفِيَةَ أَعْدَائِي وَمُبْغِضِيَّ فَأُفْنِيهِمْ.
\par 42 يَتَطَلَّعُونَ فَلَيْسَ مُخَلِّصٌ، إِلَى الرَّبِّ فَلاَ يَسْتَجِيبُهُمْ.
\par 43 فَأَسْحَقُهُمْ كَغُبَارِ الأَرْضِ. مِثْلَ طِينِ الأَسْوَاقِ أَدُقُّهُمْ وَأَدُوسُهُمْ.
\par 44 وَتُنْقِذُنِي مِنْ مُخَاصَمَاتِ شَعْبِي وَتَحْفَظُنِي رَأْساً لِلأُمَمِ. شَعْبٌ لَمْ أَعْرِفْهُ يَتَعَبَّدُ لِي.
\par 45 بَنُو الْغُرَبَاءِ يَتَذَلَّلُونَ لِي. مِنْ سَمَاعِ الأُذُنِ يَسْمَعُونَ لِي.
\par 46 بَنُو الْغُرَبَاءِ يَبْلُونَ وَيَزْحَفُونَ مِنْ حُصُونِهِمْ.
\par 47 حَيٌّ هُوَ الرَّبُّ وَمُبَارَكٌ صَخْرَتِي، وَمُرْتَفَعٌ إِلَهُ صَخْرَةِ خَلاَصِي،
\par 48 الإِلَهُ الْمُنْتَقِمُ لِي وَالْمُخْضِعُ شُعُوباً تَحْتِي،
\par 49 وَالَّذِي يُخْرِجُنِي مِنْ بَيْنِ أَعْدَائِي وَيَرْفَعُنِي فَوْقَ الْقَائِمِينَ عَلَيَّ، وَيُنْقِذُنِي مِنْ رَجُلِ الظُّلْمِ.
\par 50 لِذَلِكَ أَحْمَدُكَ يَا رَبُّ فِي الأُمَمِ وَلاِسْمِكَ أُرَنِّمُ.
\par 51 بُرْجُ خَلاَصٍ لِمَلِكِهِ وَالصَّانِعُ رَحْمَةً لِمَسِيحِهِ، لِدَاوُدَ وَنَسْلِهِ إِلَى الأَبَدِ».

\chapter{23}

\par 1 فَهَذِهِ هِيَ كَلِمَاتُ دَاوُدَ الأَخِيرَةُ: «وَحْيُ دَاوُدَ بْنِ يَسَّى، وَوَحْيُ الرَّجُلِ الْقَائِمِ فِي الْعُلاَ، مَسِيحِ إِلَهِ يَعْقُوبَ، وَمُرَنِّمِ إِسْرَائِيلَ الْحُلْوِ:
\par 2 رُوحُ الرَّبِّ تَكَلَّمَ بِي وَكَلِمَتُهُ عَلَى لِسَانِي.
\par 3 قَالَ إِلَهُ إِسْرَائِيلَ. إِلَيَّ تَكَلَّمَ صَخْرَةُ إِسْرَائِيلَ. إِذَا تَسَلَّطَ عَلَى النَّاسِ بَارٌّ يَتَسَلَّطُ بِخَوْفِ اللَّهِ،
\par 4 وَكَنُورِ الصَّبَاحِ إِذَا أَشْرَقَتِ الشَّمْسُ. كَعُشْبٍ مِنَ الأَرْضِ فِي صَبَاحٍ صَحْوٍ مُضِيءٍ غِبَّ الْمَطَرِ.
\par 5 أَلَيْسَ هَكَذَا بَيْتِي عِنْدَ اللَّهِ لأَنَّهُ وَضَعَ لِي عَهْداً أَبَدِيّاً مُتْقَناً فِي كُلِّ شَيْءٍ وَمَحْفُوظاً؟ أَفَلاَ يُثْبِتُ كُلَّ خَلاَصِي وَكُلَّ مَسَرَّتِي؟
\par 6 وَلَكِنَّ بَنِي بَلِيَّعَالَ جَمِيعَهُمْ كَشَوْكٍ مَطْرُوحٍ لأَنَّهُمْ لاَ يُؤْخَذُونَ بِيَدٍ.
\par 7 وَالرَّجُلُ الَّذِي يَمَسُّهُمْ يَتَسَلَّحُ بِحَدِيدٍ وَعَصَا رُمْحٍ. فَيَحْتَرِقُونَ بِالنَّارِ فِي مَكَانِهِمْ».
\par 8 هَذِهِ أَسْمَاءُ الأَبْطَالِ الَّذِينَ لِدَاوُدَ: يُشَيْبَ بَشَّبَثُ التَّحْكَمُونِيُّ رَئِيسُ الثَّلاَثَةِ. هُوَ هَزَّ رُمْحَهُ عَلَى ثَمَانِ مِئَةٍ قَتَلَهُمْ دُفْعَةً وَاحِدَةً.
\par 9 وَبَعْدَهُ أَلِعَازَارُ بْنُ دُودُو بْنِ أَخُوخِي أَحَدُ الثَّلاَثَةِ الأَبْطَالِ الَّذِينَ كَانُوا مَعَ دَاوُدَ حِينَمَا عَيَّرُوا الْفِلِسْطِينِيِّينَ الَّذِينَ اجْتَمَعُوا هُنَاكَ لِلْحَرْبِ وَصَعِدَ رِجَالُ إِسْرَائِيلَ.
\par 10 أَمَّا هُوَ فَأَقَامَ وَضَرَبَ الْفِلِسْطِينِيِّينَ حَتَّى كَلَّتْ يَدُهُ، وَلَصِقَتْ يَدُهُ بِالسَّيْفِ، وَصَنَعَ الرَّبُّ خَلاَصاً عَظِيماً فِي ذَلِكَ الْيَوْمِ، وَرَجَعَ الشَّعْبُ وَرَاءَهُ لِلنَّهْبِ فَقَطْ.
\par 11 وَبَعْدَهُ شَمَّةُ بْنُ أَجِي الْهَرَارِيُّ. فَاجْتَمَعَ الْفِلِسْطِينِيُّونَ جَيْشاً وَكَانَتْ هُنَاكَ قِطْعَةُ حَقْلٍ مَمْلُوءةً عَدَساً، فَهَرَبَ الشَّعْبُ مِنْ أَمَامِ الْفِلِسْطِينِيِّينَ.
\par 12 فَوَقَفَ فِي وَسَطِ الْقِطْعَةِ وَأَنْقَذَهَا، وَضَرَبَ الْفِلِسْطِينِيِّينَ فَصَنَعَ الرَّبُّ خَلاَصاً عَظِيماً.
\par 13 وَنَزَلَ الثَّلاَثَةُ مِنَ الثَّلاَثِينَ رَئِيساً وَأَتُوا فِي الْحَصَادِ إِلَى دَاوُدَ إِلَى مَغَارَةِ عَدُلاَّمَ، وَجَيْشُ الْفِلِسْطِينِيِّينَ نَازِلٌ فِي وَادِي الرَّفَائِيِّينَ.
\par 14 وَكَانَ دَاوُدُ حِينَئِذٍ فِي الْحِصْنِ، وَحَفَظَةُ الْفِلِسْطِينِيِّينَ حِينَئِذٍ فِي بَيْتِ لَحْمٍ.
\par 15 فَتَأَوَّهَ دَاوُدُ وَقَالَ: «مَنْ يَسْقِينِي مَاءً مِنْ بِئْرِ بَيْتِ لَحْمٍ الَّتِي عِنْدَ الْبَابِ؟»
\par 16 فَشَقَّ الأَبْطَالُ الثَّلاَثَةُ مَحَلَّةَ الْفِلِسْطِينِيِّينَ وَاسْتَقُوا مَاءً مِنْ بِئْرِ بَيْتِ لَحْمٍ الَّتِي عِنْدَ الْبَابِ، وَحَمَلُوهُ وَأَتُوا بِهِ إِلَى دَاوُدَ، فَلَمْ يَشَأْ أَنْ يَشْرَبَهُ بَلْ سَكَبَهُ لِلرَّبِّ
\par 17 وَقَالَ: «حَاشَا لِي يَا رَبُّ أَنْ أَفْعَلَ ذَلِكَ. هَذَا دَمُ الرِّجَالِ الَّذِينَ خَاطَرُوا بِأَنْفُسِهِمْ». فَلَمْ يَشَأْ أَنْ يَشْرَبَهُ. هَذَا مَا فَعَلَهُ الثَّلاَثَةُ الأَبْطَالُ.
\par 18 وَأَبِيشَايُ أَخُو يُوآبَ ابْنُِ صَرُويَةَ هُوَ رَئِيسُ ثَلاَثَةٍ. هَذَا هَزَّ رُمْحَهُ عَلَى ثَلاَثِ مِئَةٍ قَتَلَهُمْ، فَكَانَ لَهُ اسْمٌ بَيْنَ الثَّلاَثَةِ.
\par 19 أَلَمْ يُكْرَمْ عَلَى الثَّلاَثَةِ فَكَانَ لَهُمْ رَئِيساً، إِلاَّ أَنَّهُ لَمْ يَصِلْ إِلَى الثَّلاَثَةِ الأُوَلِ.
\par 20 وَبَنَايَاهُو بْنُ يَهُويَادَاعَ، ابْنُ ذِي بَأْسٍ، كَثِيرُ الأَفْعَالِ، مِنْ قَبْصِئِيلَ، هُوَ الَّذِي ضَرَبَ أَسَدَيْ مُوآبَ، وَهُوَ الَّذِي نَزَلَ وَضَرَبَ أَسَداً فِي وَسَطِ جُبٍّ يَوْمَ الثَّلْجِ.
\par 21 وَهُوَ ضَرَبَ رَجُلاً مِصْرِيّاً ذَا مَنْظَرٍ، وَكَانَ بِيَدِ الْمِصْرِيِّ رُمْحٌ، فَنَزَلَ إِلَيْهِ بِعَصاً وَخَطَفَ الرُّمْحَ مِنْ يَدِ الْمِصْرِيِّ وَقَتَلَهُ بِرُمْحِهِ.
\par 22 هَذَا مَا فَعَلَهُ بَنَايَاهُو بْنُ يَهُويَادَاعَ، فَكَانَ لَهُ اسْمٌ بَيْنَ الثَّلاَثَةِ الأَبْطَالِ،
\par 23 وَأُكْرِمَ عَلَى الثَّلاَثِينَ، إِلاَّ أَنَّهُ لَمْ يَصِلْ إِلَى الثَّلاَثَةِ. فَجَعَلَهُ دَاوُدُ مِنْ أَصْحَابِ سِرِّهِ.
\par 24 وَعَسَائِيلُ أَخُو يُوآبَ كَانَ مِنَ الثَّلاَثِينَ، وَأَلْحَانَانُ بْنُ دُودُو مِنْ بَيْتِ لَحْمٍ.
\par 25 وَشَمَّةُ الْحَرُودِيُّ، وَأَلِيقَا الْحَرُودِيُّ،
\par 26 وَحَالَصُ الْفَلْطِيُّ، وَعِيرَا بْنُ عِقِّيشَ التَّقُوعِيُّ،
\par 27 وَأَبِيعَزَرُ الْعَنَاثُوثِيُّ، وَمَبُونَايُ الْحُوشَاتِيُّ،
\par 28 وَصَلْمُونُ الأَخُوخِيُّ، وَمَهْرَايُ النَّطُوفَاتِيُّ،
\par 29 وَخَالَبُ بْنُ بَعْنَةَ النَّطُوفَاتِيُّ، وَإِتَّايُ بْنُ رِيبَايَ مِنْ جِبْعَةِ بَنِي بِنْيَامِينَ،
\par 30 وَبَنَايَا الْفَرْعَتُونِيُّ، وَهِدَّايُ مِنْ أَوْدِيَةِ جَاعَشَ،
\par 31 وَأَبُو عَلْبُونَ الْعَرَبَاتِيُّ، وَعَزْمُوتُ الْبَرْحُومِيُّ،
\par 32 وَأَلْيَحْبَا الشَّعْلُبُونِيُّ وَمِنْ بَنِي يَاشَنَ: يُونَاثَانُ.
\par 33 وَشَمَّةُ الْهَرَارِيُّ، وَأَخِيآمُ بْنُ شَارَارَ الأَرَارِيُّ،
\par 34 وَأَلِيفَلَطُ بْنُ أَحَسْبَايَ ابْنُ الْمَعْكِيِّ، وَأَلِيعَامُ بْنُ أَخِيتُوفَلَ الْجِيلُونِيُّ،
\par 35 وَحَصْرَايُ الْكَرْمَلِيُّ، وَفَعْرَايُ الأَرَبِيُّ،
\par 36 وَيَجْآلُ بْنُ نَاثَانَ مِنْ صُوبَةَ، وَبَانِي الْجَادِيُّ،
\par 37 وَصَالَقُ الْعَمُّونِيُّ، وَنَحْرَايُ الْبَئِيرُوتِيُّ (حَامِلُ سِلاَحِ يُوآبَ بْنِ صَرُويَةَ)
\par 38 وَعِيرَا الْيِثْرِيُّ، وَجَارَبُ الْيَثْرِيُّ،
\par 39 وَأُورِيَّا الْحِثِّيُّ. الْجَمِيعُ سَبْعَةٌ وَثَلاَثُونَ.

\chapter{24}

\par 1 وَعَادَ فَحَمِيَ غَضَبُ الرَّبِّ عَلَى إِسْرَائِيلَ فَأَهَاجَ عَلَيْهِمْ دَاوُدَ قَائِلاً: «امْضِ وَأَحْصِ إِسْرَائِيلَ وَيَهُوذَا».
\par 2 فَقَالَ الْمَلِكُ لِيُوآبَ رَئِيسِ الْجَيْشِ الَّذِي عِنْدَهُ: «طُفْ فِي جَمِيعِ أَسْبَاطِ إِسْرَائِيلَ مِنْ دَانَ إِلَى بِئْرِ سَبْعٍ وَعُدُّوا الشَّعْبَ، فَأَعْلَمَ عَدَدَ الشَّعْبِ».
\par 3 فَقَالَ يُوآبُ لِلْمَلِكِ: «لِيَزِدِ الرَّبُّ إِلَهُكَ الشَّعْبَ أَمْثَالَهُمْ مِئَةَ ضِعْفٍ، وَعَيْنَا سَيِّدِي الْمَلِكِ نَاظِرَتَانِ. وَلَكِنْ لِمَاذَا يُسَرُّ سَيِّدِي الْمَلِكُ بِهَذَا الأَمْرِ؟»
\par 4 فَاشْتَدَّ كَلاَمُ الْمَلِكِ عَلَى يُوآبَ وَعَلَى رُؤَسَاءِ الْجَيْشِ، فَخَرَجَ يُوآبُ وَرُؤَسَاءُ الْجَيْشِ مِنْ عِنْدِ الْمَلِكِ لِيَعُدُّوا إِسْرَائِيلَ.
\par 5 فَعَبَرُوا الأُرْدُنَّ وَنَزَلُوا فِي عَرُوعِيرَ عَنْ يَمِينِ الْمَدِينَةِ الَّتِي فِي وَسَطِ وَادِي جَادَ وَتُجَاهَ يَعْزِيرَ،
\par 6 وَأَتُوا إِلَى جِلْعَادَ وَإِلَى أَرْضِ تَحْتِيمَ إِلَى حُدْشِي، ثُمَّ أَتُوا إِلَى دَانِ يَعَنَ وَاسْتَدَارُوا إِلَى صَيْدُونَ،
\par 7 ثُمَّ أَتُوا إِلَى حِصْنِ صُورٍ وَجَمِيعِ مُدُنِ الْحِوِّيِّينَ وَالْكَنْعَانِيِّينَ، ثُمَّ خَرَجُوا إِلَى جَنُوبِيِّ يَهُوذَا إِلَى بِئْرِ سَبْعٍ
\par 8 وَطَافُوا كُلَّ الأَرْضِ، وَجَاءُوا فِي نِهَايَةِ تِسْعَةِ أَشْهُرٍ وَعِشْرِينَ يَوْماً إِلَى أُورُشَلِيمَ.
\par 9 فَدَفَعَ يُوآبُ جُمْلَةَ عَدَدِ الشَّعْبِ إِلَى الْمَلِكِ، فَكَانَ إِسْرَائِيلُ ثَمَانَ مِئَةِ أَلْفِ رَجُلٍ ذِي بَأْسٍ مُسْتَلِّ السَّيْفِ، وَرِجَالُ يَهُوذَا خَمْسَ مِئَةِ أَلْفِ رَجُلٍ.
\par 10 وَضَرَبَ دَاوُدَ قَلْبُهُ بَعْدَمَا عَدَّ الشَّعْبَ. فَقَالَ دَاوُدُ لِلرَّبِّ: «لَقَدْ أَخْطَأْتُ جِدّاً فِي مَا فَعَلْتُ، وَالآنَ يَا رَبُّ أَزِلْ إِثْمَ عَبْدِكَ لأَنِّي انْحَمَقْتُ جِدّاً».
\par 11 وَلَمَّا قَامَ دَاوُدُ صَبَاحاً كَانَ كَلاَمُ الرَّبِّ إِلَى جَادٍ النَّبِيِّ رَائِي دَاوُدَ:
\par 12 «اِذْهَبْ وَقُلْ لِدَاوُدَ: هَكَذَا قَالَ الرَّبُّ: ثَلاَثَةً أَنَا عَارِضٌ عَلَيْكَ، فَاخْتَرْ لِنَفْسِكَ وَاحِداً مِنْهَا فَأَفْعَلَهُ بِكَ».
\par 13 فَأَتَى جَادُ إِلَى دَاوُدَ وَقَالَ لَهُ: «أَتَأْتِي عَلَيْكَ سَبْعُ سِنِي جُوعٍ فِي أَرْضِكَ، أَمْ تَهْرُبُ ثَلاَثَةَ أَشْهُرٍ أَمَامَ أَعْدَائِكَ وَهُمْ يَتْبَعُونَكَ، أَمْ يَكُونُ ثَلاَثَةَ أَيَّامٍ وَبَأٌ فِي أَرْضِكَ؟ فَالآنَ اعْرِفْ وَانْظُرْ مَاذَا أَرُدُّ جَوَاباً عَلَى مُرْسِلِي».
\par 14 فَقَالَ دَاوُدُ لِجَادٍ: «قَدْ ضَاقَ بِيَ الأَمْرُ جِدّاً. فَلْنَسْقُطْ فِي يَدِ الرَّبِّ لأَنَّ مَرَاحِمَهُ كَثِيرَةٌ وَلاَ أَسْقُطْ فِي يَدِ إِنْسَانٍ».
\par 15 فَجَعَلَ الرَّبُّ وَبَأً فِي إِسْرَائِيلَ مِنَ الصَّبَاحِ إِلَى الْمِيعَادِ، فَمَاتَ مِنَ الشَّعْبِ مِنْ دَانٍ إِلَى بِئْرِ سَبْعٍ سَبْعُونَ أَلْفَ رَجُلٍ.
\par 16 وَبَسَطَ الْمَلاَكُ يَدَهُ عَلَى أُورُشَلِيمَ لِيُهْلِكَهَا، فَنَدِمَ الرَّبُّ عَنِ الشَّرِّ وَقَالَ لِلْمَلاَكِ الْمُهْلِكِ الشَّعْبَ: «كَفَى! الآنَ رُدَّ يَدَكَ». وَكَانَ مَلاَكُ الرَّبِّ عِنْدَ بَيْدَرِ أَرُونَةَ الْيَبُوسِيِّ.
\par 17 فَقَالَ دَاوُدُ لِلرَّبَّ عِنْدَمَا رَأَى الْمَلاَكَ الضَّارِبَ الشَّعْبَ: «هَا أَنَا أَخْطَأْتُ وَأَنَا أَذْنَبْتُ، وَأَمَّا هَؤُلاَءِ الْخِرَافُ فَمَاذَا فَعَلُوا؟ فَلْتَكُنْ يَدُكَ عَلَيَّ وَعَلَى بَيْتِ أَبِي».
\par 18 فَجَاءَ جَادُ فِي ذَلِكَ الْيَوْمِ إِلَى دَاوُدَ وَقَالَ لَهُ: «اصْعَدْ وَأَقِمْ لِلرَّبِّ مَذْبَحاً فِي بَيْدَرِ أَرُونَةَ الْيَبُوسِيِّ».
\par 19 فَصَعِدَ دَاوُدُ حَسَبَ كَلاَمِ جَادَ كَمَا أَمَرَ الرَّبُّ.
\par 20 فَتَطَلَّعَ أَرُونَةُ وَرَأَى الْمَلِكَ وَعَبِيدَهُ يُقْبِلُونَ إِلَيْهِ، فَخَرَجَ أَرُونَةُ وَسَجَدَ لِلْمَلِكِ عَلَى وَجْهِهِ إِلَى الأَرْضِ.
\par 21 وَقَالَ أَرُونَةُ: «لِمَاذَا جَاءَ سَيِّدِي الْمَلِكُ إِلَى عَبْدِهِ؟» فَقَالَ دَاوُدُ: «لأَشْتَرِيَ مِنْكَ الْبَيْدَرَ لأَبْنِيَ مَذْبَحاً لِلرَّبِّ فَتَكُفَّ الضَّرْبَةُ عَنِ الشَّعْبِ».
\par 22 فَقَالَ أَرُونَةُ لِدَاوُدَ: «فَلْيَأْخُذْهُ سَيِّدِي الْمَلِكُ وَيُصْعِدْ مَا يَحْسُنُ فِي عَيْنَيْهِ. انْظُرْ. الْبَقَرُ لِلْمُحْرَقَةِ، وَالنَّوَارِجُ وَأَدَوَاتُ الْبَقَرِ حَطَباً».
\par 23 اَلْكُلُّ دَفَعَهُ أَرُونَةُ الْمَالِكُ إِلَى الْمَلِكِ. وَقَالَ أَرُونَةُ لِلْمَلِكِ: «الرَّبُّ إِلَهُكَ يَرْضَى عَنْكَ».
\par 24 فَقَالَ الْمَلِكُ لأَرُونَةَ: «لاَ. بَلْ أَشْتَرِي مِنْكَ بِثَمَنٍ وَلاَ أُصْعِدُ لِلرَّبِّ إِلَهِي مُحْرَقَاتٍ مَجَّانِيَّةً». فَاشْتَرَى دَاوُدُ الْبَيْدَرَ وَالْبَقَرَ بِخَمْسِينَ شَاقِلاً مِنَ الْفِضَّةِ.
\par 25 وَبَنَى دَاوُدُ هُنَاكَ مَذْبَحاً لِلرَّبِّ وَأَصْعَدَ مُحْرَقَاتٍ وَذَبَائِحَ سَلاَمَةٍ. وَاسْتَجَابَ الرَّبُّ مِنْ أَجْلِ الأَرْضِ، فَكَفَّتِ الضَّرْبَةُ عَنْ إِسْرَائِيلَ.


\end{document}