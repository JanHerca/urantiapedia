\begin{document}

\title{راعوث}


\chapter{1}

\par 1 حَدَثَ فِي أَيَّامِ حُكْمِ الْقُضَاةِ أَنَّهُ صَارَ جُوعٌ فِي الأَرْضِ, فَذَهَبَ رَجُلٌ مِنْ بَيْتِ لَحْمِ يَهُوذَا لِيَتَغَرَّبَ فِي بِلاَدِ مُوآبَ هُوَ وَامْرَأَتُهُ وَابْنَاهُ.
\par 2 وَاسْمُ الرَّجُلِ أَلِيمَالِكُ, وَاسْمُ امْرَأَتِهِ نُعْمِي, وَاسْمَا ابْنَيْهِ مَحْلُونُ وَكِلْيُونُ - أَفْرَاتِيُّونَ مِنْ بَيْتِ لَحْمِ يَهُوذَا. فَأَتُوا إِلَى بِلاَدِ مُوآبَ وَكَانُوا هُنَاكَ.
\par 3 وَمَاتَ أَلِيمَالِكُ رَجُلُ نُعْمِي, وَبَقِيَتْ هِيَ وَابْنَاهَا.
\par 4 فَأَخَذَا لَهُمَا امْرَأَتَيْنِ مُوآبِيَّتَيْنِ, اسْمُ إِحْدَاهُمَا عُرْفَةُ وَاسْمُ الأُخْرَى رَاعُوثُ. وَأَقَامَا هُنَاكَ نَحْوَ عَشَرِ سِنِينٍ.
\par 5 ثُمَّ مَاتَا كِلاَهُمَا مَحْلُونُ وَكِلْيُونُ, فَتُرِكَتِ الْمَرْأَةُ مِنِ ابْنَيْهَا وَمِنْ رَجُلِهَا.
\par 6 فَقَامَتْ هِيَ وَكَنَّتَاهَا وَرَجَعَتْ مِنْ بِلاَدِ مُوآبَ, لأَنَّهَا سَمِعَتْ فِي بِلاَدِ مُوآبَ أَنَّ الرَّبَّ قَدِ افْتَقَدَ شَعْبَهُ لِيُعْطِيَهُمْ خُبْزاً.
\par 7 وَخَرَجَتْ مِنَ الْمَكَانِ الَّذِي كَانَتْ فِيهِ وَكَنَّتَاهَا مَعَهَا, وَسِرْنَ فِي الطَّرِيقِ لِلرُّجُوعِ إِلَى أَرْضِ يَهُوذَا.
\par 8 فَقَالَتْ نُعْمِي لِكَنَّتَيْهَا: «اذْهَبَا ارْجِعَا كُلُّ وَاحِدَةٍ إِلَى بَيْتِ أُمِّهَا. وَلْيَصْنَعِ الرَّبُّ مَعَكُمَا إِحْسَاناً كَمَا صَنَعْتُمَا بِالْمَوْتَى وَبِي.
\par 9 وَلْيُعْطِكُمَا الرَّبُّ أَنْ تَجِدَا رَاحَةً كُلُّ وَاحِدَةٍ فِي بَيْتِ رَجُلِهَا». فَقَبَّلَتْهُمَا, وَرَفَعْنَ أَصْوَاتَهُنَّ وَبَكَيْنَ.
\par 10 فَقَالَتَا لَهَا: «إِنَّنَا نَرْجِعُ مَعَكِ إِلَى شَعْبِكِ».
\par 11 فَقَالَتْ نُعْمِي: «ارْجِعَا يَا بِنْتَيَّ. لِمَاذَا تَذْهَبَانِ مَعِي؟ هَلْ فِي أَحْشَائِي بَنُونَ بَعْدُ حَتَّى يَكُونُوا لَكُمَا رِجَالاً؟
\par 12 اِرْجِعَا يَا بِنْتَيَّ وَاذْهَبَا لأَنِّي قَدْ شِخْتُ عَنْ أَنْ أَكُونَ لِرَجُلٍ. وَإِنْ قُلْتُ لِي رَجَاءٌ أَيْضاً بِأَنِّي أَصِيرُ هَذِهِ اللَّيْلَةَ لِرَجُلٍ وَأَلِدُ بَنِينَ أَيْضاً,
\par 13 هَلْ تَصْبِرَانِ لَهُمْ حَتَّى يَكْبُرُوا؟ هَلْ تَنْحَجِزَانِ مِنْ أَجْلِهِمْ عَنْ أَنْ تَكُونَا لِرَجُلٍ؟ لاَ يَا بِنْتَيَّ. فَإِنِّي مَغْمُومَةٌ جِدّاً مِنْ أَجْلِكُمَا لأَنَّ يَدَ الرَّبِّ قَدْ خَرَجَتْ عَلَيَّ».
\par 14 ثُمَّ رَفَعْنَ أَصْوَاتَهُنَّ وَبَكَيْنَ أَيْضاً. فَقَبَّلَتْ عُرْفَةُ حَمَاتَهَا, وَأَمَّا رَاعُوثُ فَلَصِقَتْ بِهَا.
\par 15 فَقَالَتْ: «هُوَذَا قَدْ رَجَعَتْ سِلْفَتُكِ إِلَى شَعْبِهَا وَآلِهَتِهَا. ارْجِعِي أَنْتِ وَرَاءَ سِلْفَتِكِ.
\par 16 فَقَالَتْ رَاعُوثُ: «لاَ تُلِحِّي عَلَيَّ أَنْ أَتْرُكَكِ وَأَرْجِعَ عَنْكِ, لأَنَّهُ حَيْثُمَا ذَهَبْتِ أَذْهَبُ وَحَيْثُمَا بِتِّ أَبِيتُ. شَعْبُكِ شَعْبِي وَإِلَهُكِ إِلَهِي.
\par 17 حَيْثُمَا مُتِّ أَمُوتُ وَهُنَاكَ أَنْدَفِنُ. هَكَذَا يَفْعَلُ الرَّبُّ بِي وَهَكَذَا يَزِيدُ. إِنَّمَا الْمَوْتُ يَفْصِلُ بَيْنِي وَبَيْنَكِ».
\par 18 فَلَمَّا رَأَتْ أَنَّهَا مُشَدِّدَةٌ عَلَى الذَّهَابِ مَعَهَا كَفَّتْ عَنِ الْكَلاَمِ إِلَيْهَا.
\par 19 فَذَهَبَتَا كِلْتَاهُمَا حَتَّى دَخَلَتَا بَيْتَ لَحْمٍ. وَكَانَ عِنْدَ دُخُولِهِمَا بَيْتَ لَحْمٍ أَنَّ الْمَدِينَةَ كُلَّهَا تَحَرَّكَتْ بِسَبَبِهِمَا, وَقَالُوا: «أَهَذِهِ نُعْمِي؟»
\par 20 فَقَالَتْ لَهُمْ: «لاَ تَدْعُونِي نُعْمِيَ بَلِ ادْعُونِي مُرَّةَ, لأَنَّ الْقَدِيرَ قَدْ أَمَرَّنِي جِدّاً.
\par 21 إِنِّي ذَهَبْتُ مُمْتَلِئَةً وَأَرْجَعَنِيَ الرَّبُّ فَارِغَةً. لِمَاذَا تَدْعُونَنِي «نُعْمِيَ» وَالرَّبُّ قَدْ أَذَلَّنِي وَالْقَدِيرُ قَدْ كَسَّرَنِي؟»
\par 22 فَرَجَعَتْ نُعْمِي وَرَاعُوثُ الْمُوآبِيَّةُ كَنَّتُهَا مَعَهَا, الَّتِي رَجَعَتْ مِنْ بِلاَدِ مُوآبَ, وَدَخَلَتَا بَيْتَ لَحْمٍ فِي ابْتِدَاءِ حَصَادِ الشَّعِيرِ.

\chapter{2}

\par 1 وَكَانَ لِنُعْمِي ذُو قَرَابَةٍ لِرَجُلِهَا, جَبَّارُ بَأْسٍ مِنْ عَشِيرَةِ أَلِيمَالِكَ, اسْمُهُ بُوعَزُ.
\par 2 فَقَالَتْ رَاعُوثُ الْمُوآبِيَّةُ لِنُعْمِي: «دَعِينِي أَذْهَبْ إِلَى الْحَقْلِ وَأَلْتَقِطْ سَنَابِلَ وَرَاءَ مَنْ أَجِدُ نِعْمَةً فِي عَيْنَيْهِ». فَقَالَتْ لَهَا: «اذْهَبِي يَا ابْنَتِي».
\par 3 فَذَهَبَتْ وَجَاءَتْ وَالْتَقَطَتْ فِي الْحَقْلِ وَرَاءَ الْحَصَّادِينَ. فَاتَّفَقَ نَصِيبُهَا فِي قِطْعَةِ حَقْلٍ لِبُوعَزَ الَّذِي مِنْ عَشِيرَةِ أَلِيمَالِكَ.
\par 4 وَإِذَا بِبُوعَزَ قَدْ جَاءَ مِنْ بَيْتِ لَحْمٍ وَقَالَ لِلْحَصَّادِينَ: «الرَّبُّ مَعَكُمْ». فَقَالُوا لَهُ: «يُبَارِكُكَ الرَّبُّ».
\par 5 فَقَالَ بُوعَزُ لِغُلاَمِهِ الْمُوَكَّلِ عَلَى الْحَصَّادِينَ: «لِمَنْ هَذِهِ الْفَتَاةُ؟»
\par 6 فَأَجَابَ: «هِيَ فَتَاةٌ مُوآبِيَّةٌ قَدْ رَجَعَتْ مَعَ نُعْمِي مِنْ بِلاَدِ مُوآبَ,
\par 7 وَقَالَتْ: دَعُونِي أَلْتَقِطْ وَأَجْمَعْ بَيْنَ الْحُزَمِ وَرَاءَ الْحَصَّادِينَ. فَجَاءَتْ وَمَكَثَتْ مِنَ الصَّبَاحِ إِلَى الآنَ. قَلِيلاً مَّا لَبِثَتْ فِي الْبَيْتِ».
\par 8 فَقَالَ بُوعَزُ لِرَاعُوثَ: «أَلاَ تَسْمَعِينَ يَا ابْنَتِي؟ لاَ تَذْهَبِي لِتَلْتَقِطِي فِي حَقْلِ آخَرَ, وَأَيْضاً لاَ تَبْرَحِي مِنْ هَهُنَا, بَلْ هُنَا لاَزِمِي فَتَيَاتِي.
\par 9 عَيْنَاكِ عَلَى الْحَقْلِ الَّذِي يَحْصُدُونَ وَاذْهَبِي وَرَاءَهُمْ. أَلَمْ أُوصِ الْغِلْمَانَ أَنْ لاَ يَمَسُّوكِ؟ وَإِذَا عَطِشْتِ فَاذْهَبِي إِلَى الآنِيَةِ وَاشْرَبِي مِمَّا اسْتَقَاهُ الْغِلْمَانُ».
\par 10 فَسَقَطَتْ عَلَى وَجْهِهَا وَسَجَدَتْ إِلَى الأَرْضِ وَقَالَتْ لَهُ: «كَيْفَ وَجَدْتُ نِعْمَةً فِي عَيْنَيْكَ حَتَّى تَنْظُرَ إِلَيَّ وَأَنَا غَرِيبَةٌ!»
\par 11 فَأَجَابَ بُوعَزُ: «إِنَّنِي قَدْ أُخْبِرْتُ بِكُلِّ مَا فَعَلْتِ بِحَمَاتِكِ بَعْدَ مَوْتِ رَجُلِكِ, حَتَّى تَرَكْتِ أَبَاكِ وَأُمَّكِ وَأَرْضَ مَوْلِدِكِ وَسِرْتِ إِلَى شَعْبٍ لَمْ تَعْرِفِيهِ مِنْ قَبْلُ.
\par 12 لِيُكَافِئِ الرَّبُّ عَمَلَكِ, وَلْيَكُنْ أَجْرُكِ كَامِلاً مِنْ عِنْدِ الرَّبِّ إِلَهِ إِسْرَائِيلَ الَّذِي جِئْتِ لِكَيْ تَحْتَمِي تَحْتَ جَنَاحَيْهِ».
\par 13 فَقَالَت: «لَيْتَنِي أَجِدُ نِعْمَةً فِي عَيْنَيْكَ يَا سَيِّدِي لأَنَّكَ قَدْ عَزَّيْتَنِي وَطَيَّبْتَ قَلْبَ جَارِيَتِكَ, وَأَنَا لَسْتُ كَوَاحِدَةٍ مِنْ جَوَارِيكَ».
\par 14 فَقَالَ لَهَا بُوعَزُ: «عِنْدَ وَقْتِ الأَكْلِ تَقَدَّمِي إِلَى هَهُنَا وَكُلِي مِنَ الْخُبْزِ وَاغْمِسِي لُقْمَتَكِ فِي الْخَلِّ». فَجَلَسَتْ بِجَانِبِ الْحَصَّادِينَ فَنَاوَلَهَا فَرِيكاً, فَأَكَلَتْ وَشَبِعَتْ وَفَضَلَ عَنْهَا.
\par 15 ثُمَّ قَامَتْ لِتَلْتَقِطَ. فَأَمَرَ بُوعَزُ غِلْمَانَهُ: «دَعُوهَا تَلْتَقِطْ بَيْنَ الْحُزَمِ أَيْضاً وَلاَ تُؤْذُوهَا.
\par 16 وَأَنْسِلُوا أَيْضاً لَهَا مِنَ الْحُزَمِ وَدَعُوهَا تَلْتَقِطْ وَلاَ تَنْتَهِرُوهَا».
\par 17 فَالْتَقَطَتْ فِي الْحَقْلِ إِلَى الْمَسَاءِ, وَخَبَطَتْ مَا الْتَقَطَتْهُ فَكَانَ نَحْوَ إِيفَةِ شَعِيرٍ.
\par 18 فَحَمَلَتْهُ وَدَخَلَتِ الْمَدِينَةَ. فَرَأَتْ حَمَاتُهَا مَا الْتَقَطَتْهُ. وَأَخْرَجَتْ وَأَعْطَتْهَا مَا فَضَلَ عَنْهَا بَعْدَ شِبَعِهَا.
\par 19 فَقَالَتْ لَهَا حَمَاتُهَا: «أَيْنَ الْتَقَطْتِ الْيَوْمَ وَأَيْنَ اشْتَغَلْتِ؟ لِيَكُنِ النَّاظِرُ إِلَيْكِ مُبَارَكاً». فَأَخْبَرَتْ حَمَاتَهَا بِالَّذِي اشْتَغَلَتْ مَعَهُ وَقَالَتِ: «اسْمُ الرَّجُلِ الَّذِي اشْتَغَلْتُ مَعَهُ الْيَوْمَ بُوعَزُ».
\par 20 فَقَالَتْ نُعْمِي لِكَنَّتِهَا: «مُبَارَكٌ هُوَ مِنَ الرَّبِّ لأَنَّهُ لَمْ يَتْرُكِ الْمَعْرُوفَ مَعَ الأَحْيَاءِ وَالْمَوْتَى». ثُمَّ قَالَتْ لَهَا نُعْمِي: «الرَّجُلُ ذُو قَرَابَةٍ لَنَا. هُوَ ثَانِي وَلِيِّنَا».
\par 21 فَقَالَتْ رَاعُوثُ الْمُوآبِيَّةُ: «إِنَّهُ قَالَ لِي أَيْضاً لاَزِمِي فِتْيَانِي حَتَّى يُكَمِّلُوا جَمِيعَ حَصَادِي».
\par 22 فَقَالَتْ نُعْمِي لِرَاعُوثَ كَنَّتِهَا: «إِنَّهُ حَسَنٌ يَا ابْنَتِي أَنْ تَخْرُجِي مَعَ فَتَيَاتِهِ حَتَّى لاَ يَقَعُوا بِكِ فِي حَقْلِ آخَرَ».
\par 23 فَلاَزَمَتْ فَتَيَاتِ بُوعَزَ فِي الاِلْتِقَاطِ حَتَّى انْتَهَى حَصَادُ الشَّعِيرِ وَحَصَادُ الْحِنْطَةِ. وَسَكَنَتْ مَعَ حَمَاتِهَا.

\chapter{3}

\par 1 وَقَالَتْ لَهَا نُعْمِي حَمَاتُهَا: «يَا ابْنَتِي أَلاَ أَلْتَمِسُ لَكِ رَاحَةً لِيَكُونَ لَكِ خَيْرٌ؟
\par 2 فَالآنَ أَلَيْسَ بُوعَزُ ذَا قَرَابَةٍ لَنَا, الَّذِي كُنْتِ مَعَ فَتَيَاتِهِ؟ هَا هُوَ يُذَرِّي بَيْدَرَ الشَّعِيرِ اللَّيْلَةَ.
\par 3 فَاغْتَسِلِي وَتَدَهَّنِي وَالْبَسِي ثِيَابَكِ وَانْزِلِي إِلَى الْبَيْدَرِ, وَلَكِنْ لاَ تُعْرَفِي عِنْدَ الرَّجُلِ حَتَّى يَفْرَغَ مِنَ الأَكْلِ وَالشُّرْبِ.
\par 4 وَمَتَى اضْطَجَعَ فَاعْلَمِي الْمَكَانَ الَّذِي يَضْطَجِعُ فِيهِ وَادْخُلِي وَاكْشِفِي نَاحِيَةَ رِجْلَيْهِ وَاضْطَجِعِي, وَهُوَ يُخْبِرُكِ بِمَا تَعْمَلِينَ».
\par 5 فَقَالَتْ لَهَا: «كُلَّ مَا قُلْتِ أَصْنَعُ».
\par 6 فَنَزَلَتْ إِلَى الْبَيْدَرِ وَعَمِلَتْ حَسَبَ كُلِّ مَا أَمَرَتْهَا بِهِ حَمَاتُهَا.
\par 7 فَأَكَلَ بُوعَزُ وَشَرِبَ وَطَابَ قَلْبُهُ وَدَخَلَ لِيَضْطَجِعَ فِي طَرَفِ الْعَرَمَةِ. فَدَخَلَتْ سِرّاً وَكَشَفَتْ نَاحِيَةَ رِجْلَيْهِ وَاضْطَجَعَتْ.
\par 8 وَكَانَ عِنْدَ انْتِصَافِ اللَّيْلِ أَنَّ الرَّجُلَ اضْطَرَبَ, وَالْتَفَتَ وَإِذَا بِامْرَأَةٍ مُضْطَجِعَةٍ عِنْدَ رِجْلَيْهِ.
\par 9 فَقَالَ: «مَنْ أَنْتِ؟» فَقَالَتْ: «أَنَا رَاعُوثُ أَمَتُكَ. فَابْسُطْ ذَيْلَ ثَوْبِكَ عَلَى أَمَتِكَ لأَنَّكَ وَلِيٌّ».
\par 10 فَقَالَ: «إِنَّكِ مُبَارَكَةٌ مِنَ الرَّبِّ يَا ابْنَتِي لأَنَّكِ قَدْ أَحْسَنْتِ مَعْرُوفَكِ فِي الأَخِيرِ أَكْثَرَ مِنَ الأَوَّلِ, إِذْ لَمْ تَسْعِي وَرَاءَ الشُّبَّانِ, فُقَرَاءَ كَانُوا أَوْ أَغْنِيَاءَ.
\par 11 وَالآنَ يَا ابْنَتِي لاَ تَخَافِي. كُلُّ مَا تَقُولِينَ أَفْعَلُ لَكِ, لأَنَّ جَمِيعَ أَبْوَابِ شَعْبِي تَعْلَمُ أَنَّكِ امْرَأَةٌ فَاضِلَةٌ.
\par 12 وَالآنَ صَحِيحٌ أَنِّي وَلِيٌّ, وَلَكِنْ يُوجَدُ وَلِيٌّ أَقْرَبُ مِنِّي.
\par 13 بِيتِي اللَّيْلَةَ, وَيَكُونُ فِي الصَّبَاحِ أَنَّهُ إِنْ قَضَى لَكِ حَقَّ الْوَلِيِّ فَحَسَناً. لِيَقْضِ. وَإِنْ لَمْ يَشَأْ أَنْ يَقْضِيَ لَكِ حَقَّ الْوَلِيِّ, فَأَنَا أَقْضِي لَكِ. حَيٌّ هُوَ الرَّبُّ. اضْطَجِعِي إِلَى الصَّبَاحِ».
\par 14 فَاضْطَجَعَتْ عِنْدَ رِجْلَيْهِ إِلَى الصَّبَاحِ. ثُمَّ قَامَتْ قَبْلَ أَنْ يَقْدِرَ الْوَاحِدُ عَلَى مَعْرِفَةِ صَاحِبِهِ. وَقَالَ: «لاَ يُعْلَمْ أَنَّ الْمَرْأَةَ جَاءَتْ إِلَى الْبَيْدَرِ».
\par 15 ثُمَّ قَالَ: «هَاتِي الرِّدَاءَ الَّذِي عَلَيْكِ وَأَمْسِكِيهِ». فَأَمْسَكَتْهُ, فَاكْتَالَ سِتَّةً مِنَ الشَّعِيرِ وَوَضَعَهَا عَلَيْهَا. ثُمَّ دَخَلَ الْمَدِينَةَ.
\par 16 فَجَاءَتْ إِلَى حَمَاتِهَا فَقَالَتْ: «مَنْ أَنْتِ يَا ابْنَتِي؟» فَأَخْبَرَتْهَا بِكُلِّ مَا فَعَلَ لَهَا الرَّجُلُ.
\par 17 وَقَالَتْ: «هَذِهِ السِّتَّةَ مِنَ الشَّعِيرِ أَعْطَانِي, لأَنَّهُ قَالَ: لاَ تَجِيئِي فَارِغَةً إِلَى حَمَاتِكِ».
\par 18 فَقَالَتِ: «اجْلِسِي يَا ابْنَتِي حَتَّى تَعْلَمِي كَيْفَ يَقَعُ الأَمْرُ, لأَنَّ الرَّجُلَ لاَ يَهْدَأُ حَتَّى يُتَمِّمَ الأَمْرَ الْيَوْمَ».

\chapter{4}

\par 1 فَصَعِدَ بُوعَزُ إِلَى الْبَابِ وَجَلَسَ هُنَاكَ. وَإِذَا بِالْوَلِيِّ الَّذِي تَكَلَّمَ عَنْهُ بُوعَزُ عَابِرٌ. فَقَالَ: «مِلْ وَاجْلِسْ هُنَا أَنْتَ يَا فُلاَنُ الْفُلاَنِيُّ». فَمَالَ وَجَلَسَ.
\par 2 ثُمَّ أَخَذَ عَشَرَةَ رِجَالٍ مِنْ شُيُوخِ الْمَدِينَةِ وَقَالَ لَهُمُ: «اجْلِسُوا هُنَا». فَجَلَسُوا.
\par 3 ثُمَّ قَالَ لِلْوَلِيِّ: «إِنَّ نُعْمِيَ الَّتِي رَجَعَتْ مِنْ بِلاَدِ مُوآبَ تَبِيعُ قِطْعَةَ الْحَقْلِ الَّتِي لأَخِينَا أَلِيمَالِكَ.
\par 4 فَقُلْتُ إِنِّي أُخْبِرُكَ: «اشْتَرِ قُدَّامَ الْجَالِسِينَ وَقُدَّامَ شُيُوخِ شَعْبِي. فَإِنْ كُنْتَ تَفُكُّ فَفُكَّ. وَإِنْ كُنْتَ لاَ تَفُكُّ فَأَخْبِرْنِي لأَعْلَمَ. لأَنَّهُ لَيْسَ غَيْرُكَ يَفُكُّ وَأَنَا بَعْدَكَ». فَقَالَ: «إِنِّي أَفُكُّ».
\par 5 فَقَالَ بُوعَزُ: «يَوْمَ تَشْتَرِي الْحَقْلَ مِنْ يَدِ نُعْمِي تَشْتَرِي أَيْضاً مِنْ يَدِ رَاعُوثَ الْمُوآبِيَّةِ امْرَأَةِ الْمَيِّتِ لِتُقِيمَ اسْمَ الْمَيِّتِ عَلَى مِيرَاثِهِ».
\par 6 فَقَالَ الْوَلِيُّ: «لاَ أَقْدِرُ أَنْ أَفُكَّ لِنَفْسِي لِئَلَّا أُفْسِدَ مِيرَاثِي. فَفُكَّ أَنْتَ لِنَفْسِكَ فِكَاكِي لأَنِّي لاَ أَقْدِرُ أَنْ أَفُكَّ».
\par 7 وَهَذِهِ هِيَ الْعَادَةُ سَابِقاً فِي إِسْرَائِيلَ فِي أَمْرِ الْفِكَاكِ وَالْمُبَادَلَةِ, لأَجْلِ إِثْبَاتِ كُلِّ أَمْرٍ. يَخْلَعُ الرَّجُلُ نَعْلَهُ وَيُعْطِيهِ لِصَاحِبِهِ. فَهَذِهِ هِيَ الْعَادَةُ فِي إِسْرَائِيلَ.
\par 8 فَقَالَ الْوَلِيُّ لِبُوعَزَ: «اشْتَرِ لِنَفْسِكَ». وَخَلَعَ نَعْلَهُ.
\par 9 فَقَالَ بُوعَزُ لِلشُّيُوخِ وَلِجَمِيعِ الشَّعْبِ: «أَنْتُمْ شُهُودٌ الْيَوْمَ أَنِّي قَدِ اشْتَرَيْتُ كُلَّ مَا لأَلِيمَالِكَ وَكُلَّ مَا لِكِلْيُونَ وَمَحْلُونَ مِنْ يَدِ نُعْمِي.
\par 10 وَكَذَا رَاعُوثُ الْمُوآبِيَّةُ امْرَأَةُ مَحْلُونَ قَدِ اشْتَرَيْتُهَا لِيَ امْرَأَةً, لِأُقِيمَ اسْمَ الْمَيِّتِ عَلَى مِيرَاثِهِ وَلاَ يَنْقَرِضُ اسْمُ الْمَيِّتِ مِنْ بَيْنِ إِخْوَتِهِ وَمِنْ بَابِ مَكَانِهِ. أَنْتُمْ شُهُودٌ الْيَوْمَ».
\par 11 فَقَالَ جَمِيعُ الشَّعْبِ الَّذِينَ فِي الْبَابِ وَالشُّيُوخُ: «نَحْنُ شُهُودٌ. فَلْيَجْعَلِ الرَّبُّ الْمَرْأَةَ الدَّاخِلَةَ إِلَى بَيْتِكَ كَرَاحِيلَ وَكَلَيْئَةَ اللَّتَيْنِ بَنَتَا بَيْتَ إِسْرَائِيلَ. فَاصْنَعْ بِبَأْسٍ فِي أَفْرَاتَةَ وَكُنْ ذَا اسْمٍ فِي بَيْتِ لَحْمٍ.
\par 12 وَلْيَكُنْ بَيْتُكَ كَبَيْتِ فَارِصَ الَّذِي وَلَدَتْهُ ثَامَارُ لِيَهُوذَا, مِنَ النَّسْلِ الَّذِي يُعْطِيكَ الرَّبُّ مِنْ هَذِهِ الْفَتَاةِ».
\par 13 فَأَخَذَ بُوعَزُ رَاعُوثَ امْرَأَةً وَدَخَلَ عَلَيْهَا, فَأَعْطَاهَا الرَّبُّ حَبَلاً فَوَلَدَتِ ابْناً.
\par 14 فَقَالَتِ النِّسَاءُ لِنُعْمِي: «مُبَارَكٌ الرَّبُّ الَّذِي لَمْ يُعْدِمْكِ وَلِيّاً الْيَوْمَ لِكَيْ يُدْعَى اسْمُهُ فِي إِسْرَائِيلَ.
\par 15 وَيَكُونُ لَكِ لإِرْجَاعِ نَفْسٍ وَإِعَالَةِ شَيْبَتِكِ. لأَنَّ كَنَّتَكِ الَّتِي أَحَبَّتْكِ قَدْ وَلَدَتْهُ, وَهِيَ خَيْرٌ لَكِ مِنْ سَبْعَةِ بَنِينَ».
\par 16 فَأَخَذَتْ نُعْمِي الْوَلَدَ وَوَضَعَتْهُ فِي حِضْنِهَا وَصَارَتْ لَهُ مُرَبِّيَةً.
\par 17 وَسَمَّتْهُ الْجَارَاتُ اسْماً قَائِلاَتٍ: «قَدْ وُلِدَ ابْنٌ لِنُعْمِي» وَدَعَوْنَ اسْمَهُ عُوبِيدَ. هُوَ أَبُو يَسَّى أَبِي دَاوُدَ.
\par 18 وَهَذِهِ مَوَالِيدُ فَارِصَ: فَارِصُ وَلَدَ حَصْرُونَ,
\par 19 وَحَصْرُونُ وَلَدَ رَامَ, وَرَامُ وَلَدَ عَمِّينَادَابَ,
\par 20 وَعَمِّينَادَابُ وَلَدَ نَحْشُونَ, وَنَحْشُونُ وَلَدَ سَلْمُونَ,
\par 21 وَسَلْمُونُ وَلَدَ بُوعَزَ, وَبُوعَزُ وَلَدَ عُوبِيدَ,
\par 22 وَعُوبِيدُ وَلَدَ يَسَّى, وَيَسَّى وَلَدَ دَاوُدَ.

\end{document}