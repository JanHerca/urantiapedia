\begin{document}

\title{سفر أخنوخ الثالث}

\chapter{1}

\par \textit{مقدمة: صعد ر. إسماعيل إلى السماء ليرى رؤية المركبة، وأُسندت إليه مسؤولية ميتاترون}

\par \textit{وكان أخنوخ يسير مع الله ولم يوجد لأن الله أخذه (تكوين 4)}

\par 1 قال الحاخام إسماعيل: عندما صعدتُ إلى العُلى لأرى رؤيا المركبة ودخلتُ القاعات الست، واحدة داخل الأخرى:

\par 2 بمجرد أن وصلتُ إلى باب القاعة السابعة، وقفتُ في الصلاة أمام القدوس، تبارك اسمه، ورفعتُ عينيّ إلى الأعلى (أي نحو الجلالة الإلهية)، وقلتُ:

\par 3 «يا رب الكون، أدعوك أن تكون فضل هارون بن عمرام، محب السلام وساعيه، الذي نال إكليل الكهنوت من مجدك على جبل سيناء، صالحًا لي في هذه الساعة، حتى لا يتمكن قفصائيل، الأمير، والملائكة معه من السيطرة عليّ، ولا أن يطردوني من السماء».

\par 4 على الفور، أرسل لي القدوس، تبارك اسمه، ميتاترون، خادمه ('إبيد) الملاك، أمير الحضور، فبسط جناحيه، وجاء بفرح عظيم للقائي لإنقاذي من أيديهم

\par 5 وأخذني بيدي أمام أنظارهم، وقال لي: "ادخل بسلام أمام الملك العلي السامي، وانظر إلى صورة المركبة".

\par 6 ثم دخلت القاعة السابعة، وقادني إلى معسكر (معسكرات) الشاكينا ووضعني أمام القدوس، تبارك اسمه، لأرى المركبة

\par 7 بمجرد أن رآني أمراء المركبة والسيرافيم المشتعلون، حدقوا بي. وفي الحال، انتابني ارتجاف ورعدة، وسقطت أرضًا، وخدرتني صورة عيونهم المشعة ومظهر وجوههم البهي؛ حتى وبخهم القدوس، تبارك اسمه، قائلاً:

\par 8 «يا عبيدي، يا سيرافيم، يا كروبيم، يا أوفانيم! غطوا أعينكم عن إسماعيل، ابني، يا صديقي، يا حبيبي، يا مجدي، لئلا يرتعد ولا يرتعد!»

\par 9 على الفور، جاء ميتاترون، أمير الوجود، وأعاد روحي وأوقفني على قدمي

\par 10 بعد تلك (اللحظة) لم تكن لديّ القوة الكافية لقول أغنية أمام عرش مجد الملك المجيد، أعظم الملوك، وأعظم الأمراء، إلا بعد مرور الساعة

\par 11 بعد مرور ساعة، فتح لي القدوس، تبارك اسمه، أبواب الشكينة، أبواب السلام، أبواب الحكمة، أبواب القوة، أبواب القدرة، أبواب الكلام (الديبور)> أبواب الغناء، أبواب قدوشا، أبواب الترانيم

\par 12 وأنار عينيّ وقلبي بكلمات المزامير والأناشيد والتسبيح والتمجيد والشكر والتهليل والتمجيد والترنيم والمديح. وبينما كنت أفتح فمي أنشد ترنيمة أمام القدوس، تبارك هو، أجابتني الروح القدس من تحت عرش المجد وفوقه قائلة: "قدوس" و"تبارك مجد الرب من مكانه!" (أي ترنيمة "قدوشا").

\chapter{2}

\par \textit{أعلى طبقات الملائكة تسأل عن ر. إسماعيل، ويجيب عليها ميتاترون}

\par 1 قال ر. إسماعيل: في تلك الساعة، سأل نسور المركبة، العوفانيم المشتعلون، والسيرافيم آكلو النار، ميتاترون، قائلين له:

\par 2 «يا فتى! لماذا تسمح لامرأة مولودة أن تدخل وترى المركبة؟ من أي أمة، ومن أي قبيلة هذا؟ ما هي شخصيته؟»

\par 3 أجاب ميتاترون وقال لهم: "من أمة إسرائيل التي اختارها القدوس تبارك لشعبه من بين سبعين لسانًا (أمة)، ومن سبط لاوي الذي قدمه تقدمة لاسمه، ومن نسل هارون الذي اختاره القدوس تبارك لخادمه ووضع عليه إكليل الكهنوت على سيناء".

\par 4 فتكلموا على الفور وقالوا: "حقًا، هذه جديرة برؤية المركبة". وقالوا: "طوبى للشعب الذي هو في مثل هذه الحالة!"

\chapter{3}

\par \textit{ميتاترون لديه 70 اسمًا، لكن الله يناديه "شبابًا"}

\par 1 قال ر. إسماعيل: في تلك الساعة سألت ميتاترون، الملاك، أمير الحضور: "ما اسمك؟"

\par 2 فأجابني: "لدي سبعون اسمًا، تتوافق مع ألسنة العالم السبعين وكلها مستندة إلى اسم ميتاترون، ملاك الحضور؛ لكن ملكي يناديني "الشباب" (ناار)".

\chapter{4}

\par \textit{ميتاترون هو نفسه أخنوخ الذي نُقل إلى السماء وقت الطوفان}

\par 1 قال ر. إسماعيل: سألتُ ميتاترون وقلتُ له: "لماذا يُدعى عليك اسم خالقك، بسبعين اسمًا؟ أنت أعظم من جميع الأمراء، وأعلى من جميع الملائكة، وأحبُّ من جميع الخدم، ومُكرَّمٌ فوق جميع الأقوياء في الملك والعظمة والمجد: لماذا يُنادونك "شابًا" في السماوات العلى؟"

\par 2 فأجابني وقال لي: لأني أنا حنوك بن يارد.

\par 3 "فإن جيل الطوفان حين أخطأ وارتبك في أعماله قائلين لله: اذهب عنا لأننا لا نرغب في معرفة طرقك، فإن القدوس المبارك أخرجني من وسطهم لأكون شاهداً عليهم في السماوات العليا أمام كل سكان العالم، حتى لا يقولوا: الرحيم قاسٍ".

\par 4 [عادل:] ما الذي أخطأت به كل تلك الجموع، زوجاتهم، أبنائهم وبناتهم، خيولهم، بغالهم وماشيتهم وممتلكاتهم، وكل طيور العالم، التي أهلكها القدوس المبارك من العالم معهم في مياه الطوفان؟ ولا يجوز أن يقول: "ماذا لو أخطأ جيل الطوفان؛ الوحوش والطيور، ماذا أخطأت حتى تهلك معهم؟" [قبل الميلاد:] "ما هي الخطايا التي ارتكبتها كل تلك الجموع؟ أو، فليكن أنهم أخطأوا، ماذا أخطأ أبناؤهم وبناتهم وبغالهم وماشيتهم؟ وكذلك، كل الحيوانات، الداجنة والبرية، والطيور في العالم التي أهلكها الله من العالم؟"

\par 5 ولذلك رفعني القدوس، تبارك اسمه، في حياتهم أمام أعينهم لأكون شاهدًا عليهم في العالم المستقبلي. وعيّنني القدوس، تبارك اسمه، أميرًا وحاكمًا بين الملائكة الخادمين

\par 6 في تلك الساعة، خرج ثلاثة من الملائكة الخادمين، عزة وعزة وعزائيل، واتهموني في السماوات العلى، قائلين أمام القدوس، تبارك اسمه: "ألم يقل القدماء (الأوائل) أمامك حقًا: لا تخلق الإنسان!" أجاب القدوس، تبارك اسمه، وقال لهم: "لقد خلقت وسأحمل، نعم، سأحمل وسأنقذ".

\par 7 فلما رأوني قالوا أمامه: "يا رب العالمين! ما هذا الذي يصعد إلى علو الأعالي؟ أليس من بين أبناء الذين هلكوا في أيام الطوفان؟ ماذا يفعل في الرقي؟"

\par 8 فأجابهم القدوس، تبارك اسمه، وقال لهم: "ما أنتم حتى تدخلوا وتتكلموا أمامي؟ إني أُسر بهذا أكثر منكم جميعًا، ولذلك سيكون عليكم رئيسًا وحاكمًا في السماوات العليا."

\par 9 على الفور وقف الجميع وخرجوا للقائي، وسجدوا أمامي وقالوا: "طوبى لك، وطوبى لأبيك، لأن خالقك يرضيك".

\par 10 ولأنني صغير وشاب بينهم في الأيام والشهور والسنين، لذلك يسمونني "شبابًا" (ناعر).

\chapter{5}

\par \textit{عبادة جيل أنوش تدفع الله إلى إزالة الشكينة من الأرض. عبادة الأصنام المستوحاة من عزة وعزى وعزيئيل}

\par 1 قال ر. إسماعيل: قال لي ميتاترون، أمير الحضور: منذ اليوم الذي طرد فيه القدوس، تبارك اسمه، آدم الأول من جنة عدن (وما بعده)، كانت شكينة تسكن على كروب تحت شجرة الحياة

\par 2 وكان الملائكة الخادمون يتجمعون وينزلون من السماء في جماعات، ومن الرقعة في جماعات، ومن السماء في معسكرات، ليفعلوا مشيئته في العالم كله

\par 3 وكان الإنسان الأول وجيله جالسين خارج باب الجنة لينظروا إلى مظهر الشكينة المتألق

\par 4 لأن بهاء الشكينة جاب العالم من أقصاه إلى أقصاه (ببهاء) 365,000 ضعف كرة الشمس. وكل من استفاد من بهاء الشكينة، لم يسكنه ذباب ولا بعوض، ولم يمرض ولم يتألم. لم يتسلّط عليه شياطين، ولم يستطيعوا أن يؤذوه.

\par 5 عندما خرج القدوس، تبارك اسمه، ودخل من الجنة إلى عدن، ومن عدن إلى الجنة، ومن الجنة إلى رقية، ومن رقية إلى جنة عدن، رأى الجميع روعة شكينته ولم يصبهم أذى؛

\par 6 حتى زمن جيل أنوش الذي كان رأس جميع عبدة الأصنام في العالم

\par 7 وماذا فعل جيل أنوش؟ لقد جابوا من أقاصي الأرض إلى أقاصيها، وأتى كل واحد منهم بالفضة والذهب والأحجار الكريمة واللؤلؤ في أكوام كالجبال والتلال، وصنعوا منها أصنامًا في كل أنحاء الأرض. وأقاموا الأصنام في كل أرجاء الأرض: كان حجم كل صنم فراسخ

\par 8 وأنزلوا الشمس والقمر والكواكب والأبراج، ووضعوها أمام الأصنام عن يمينها وعن يسارها، لتُخدمها كما تُخدم القدوس تبارك اسمه، كما هو مكتوب: "وكان كل جند السماء واقفا لديه عن يمينه وعن يساره".

\par 9 ما القوة التي كانت فيهم حتى تمكنوا من إسقاطهم؟ ما كانوا ليتمكنوا من إسقاطهم لولا العزى والعزى والعزائيل الذين علموهم السحر فأسقطوهم به واستخدموه

\par 10 في ذلك الوقت، قدّم الملائكة الخادمون شكاوى (ضدهم) أمام القدوس، تبارك اسمه، قائلين أمامه: "يا سيد العالم! ما لك وأبناء البشر؟ كما هو مكتوب: ما هو الإنسان (أنوش) حتى تذكره؟" لم يُكتب هنا "ماه آدم"، بل "ماه أنوش"، لأنه (أنوش) هو رأس عبدة الأصنام

\par 11 لماذا تركت [أد: أعلى السماوات العلى، ومسكن اسمك المجيد، والعرش العالي والمرتفع في عربوت في الأعالي] [ب: رقية العربوت الممتلئة بجلال مجدك، العظيمة والعالية على حد سواء، والعرش العالي والمرتفع في رقية العربوت في الأعالي] [كل: أعلى السماوات العلى الممتلئة بجلال مجدك والعالية والمرتفعة والمرتفعة، والعرش العالي والمرتفع في رقية العربوت في الأعالي] وذهبت وسكنت مع أبناء البشر الذين يعبدون الأصنام ويساوونك بالأصنام

\par 12 أنت الآن على الأرض والأصنام كذلك. ما شأنك بسكان الأرض الذين يعبدون الأصنام؟

\par 13 على الفور، رفع القدوس، تبارك اسمه، شكينته من الأرض، من وسطهم

\par 14 في تلك اللحظة، جاء الملائكة الخادمون، وجنود الجيوش، وجيوش العربوت، آلاف المعسكرات وعشرة آلاف جيش. أحضروا الأبواق، وأخذوا القرون بأيديهم، وأحاطوا الشكينة بجميع أنواع الأغاني. وصعد إلى السماوات العليا، كما هو مكتوب: «صعد الله بهتاف، والرب بصوت بوق».

\chapter{6}

\par \textit{رُفِعَ حَنوخ إلى السماء مع الشكينة. استجاب الله لاحتجاجات الملائكة}

\par 1 قال ر. إسماعيل: قال لي ميتاترون، الملاك، أمير الحضور: عندما أراد القدوس، تبارك اسمه، أن يرفعني إلى العلاء، أرسل أولًا أنافيد (تتركغرغماتفم)، الأمير، وأخذني من وسطهم أمام أنظارهم وحملني في مجد عظيم على عربة نارية مع خيول نارية، خدام المجد. ورفعني إلى السماوات العالية مع الشكينة

\par 2 بمجرد أن وصلتُ إلى السماوات العالية، شمّت الأرواح المقدسة، والأوفانيم، والسيرافيم، والكروبيم، وعجلات المركبة (الجلجاليم)، وخدام النار الآكلة رائحتي من مسافة 365000 ألف فرسخ، وقالوا: [أ: "ما رائحة مولود امرأة، وما طعم قطرة بيضاء هذه التي تصعد إلى السماء، وها هو مجرد بعوضة بين الذين يقسمون لهيب النار؟"] [ب: "ما هو مولود امرأة بيننا؟ طعم قطرة بيضاء تصعد إلى السماء لتخدم الذين يقسمون لهيب النار".] [CDEL: "ما رائحة مولود امرأة هذه، وما طعم قطرة بيضاء هذه التي تصعد إلى السماء لتخدم الذين يقسمون لهيب النار.]

\par 3 أجاب القدوس، تبارك اسمه، وخاطبهم: "يا خدامي، يا جيوشِي، يا كروبيمي، يا عوفانيمي، يا سيرافيم! لا تغضبوا من هذا! بما أن جميع أبناء البشر قد أنكروا مملكتي العظيمة وذهبوا يعبدون الأصنام، فقد نزعتُ شاكينا من بينهم ورفعتها عالياً. لكن هذا الذي أخذته من بينهم هو مختار من بين (سكان) العالم، وهو مساوٍ لهم جميعًا في الإيمان والصلاح وكمال العمل، وقد اتخذته جزيةً من عالمي تحت كل السماوات".

\chapter{7}

\par \textit{رفع أخنوخ على أجنحة الشكينة إلى مكان العرش والمركبات والجيوش الملائكية}

\par 1 قال ر. إسماعيل: قال لي ميتاترون، الملاك، أمير الحضور: عندما أخذني القدوس، تبارك اسمه، بعيدًا عن جيل الطوفان، رفعني على أجنحة ريح الشكينة إلى أعلى السماء وأدخلني إلى القصور العظيمة لـ "عربوث راقة" في الأعالي، حيث يوجد عرش الشكينة المجيد، والمركابا، وجنود الغضب، وجيوش العنف، والشنانيم الناريين، والكروبيم المشتعلين، والأوفانيم المشتعلين، والخدم المشتعلين، والحششماليم اللامعين، والسيرافيم المضيئين. ووضعني (هناك) لحضور عرش المجد يومًا بعد يوم.

\chapter{8}

\par \textit{انفتحت أبواب (خزائن السماء) أمام ميتاترون}

\par 1 قال ر. إسماعيل: قال لي ميتاترون، أمير الحضور: قبل أن يعينني لحضور عرش المجد، فتح لي القدوس تبارك وتعالى

\par ثلاثمائة ألف بوابة للفهم

\par ثلاثمائة ألف بوابة من الدقة

\par ثلاثمائة ألف بوابة للحياة

\par ثلاثمائة ألف بوابة من "النعمة واللطف المحب"

\par ثلاثمائة ألف بوابة للحب

\par ثلاثمائة ألف بوابة تورا

\par ثلاثمائة ألف باب من أبواب الوداعة

\par ثلاثمائة ألف بوابة صيانة

\par ثلاثمائة ألف باب رحمة

\بار ثلاثمائة ألف باب من أبواب خوف السماء.

\par 2 في تلك الساعة، أضاف القدوس، تبارك اسمه، إليّ حكمة إلى حكمة، وفهمًا إلى فهم، ودهاءً إلى دهاء، ومعرفة إلى معرفة، ورحمة إلى رحمة، وتعليمًا إلى تعليم، ومحبة إلى محبة، ولطفًا إلى لطف، وخيرًا إلى خير، ووداعة إلى وداعة، وقدرة إلى قدرة، وقوة إلى قوة، وقدرة إلى قوة، وتألقًا إلى تألق، وجمالًا إلى جمال، وبهاءً إلى بهاء، وتشرفت وتزينت بكل هذه الأشياء الصالحة والجديرة بالثناء أكثر من جميع أبناء السماء


\chapter{9}

\par \textit{ينال أخنوخ بركات من العلي ويتزين بصفات ملائكية}

\par 1 قال ر. إسماعيل: قال لي ميتاترون أمير الحضور: بعد هذه الأمور كلها وضع القدوس تبارك اسمه يده علي وباركني بالبركات.

\par 2 ونشأت وكبرت إلى حجم طول وعرض العالم.

\par 3 وأنبت لي جناحين من كل جانب. وكان كل جناح كالعالم كله

\par 4 وثبّت عليّ 365 عينًا: كل عين كانت كالنور العظيم.

\par 5 ولم يترك في كل أنوار الكون أي نوع من الروعة أو التألق أو الإشراق أو الجمال إلا وسلطه عليّ.

\chapter{10}

\par \textit{يضع الله ميتاترون على عرش عند باب القاعة السابعة ويعلن من خلال المنادي أن ميتاترون هو من الآن فصاعدًا ممثل الله وحاكمه على جميع أمراء الممالك وجميع أبناء السماء، باستثناء الأمراء الثمانية العظام الذين يُدعون يهوه باسم ملكهم}

\par 1 قال ر. إسماعيل: قال لي ميتاترون، أمير الحضور: كل هذه الأشياء صنعها لي القدوس تبارك وتعالى: صنع لي عرشًا يشبه عرش المجد. وبسط عليّ ستارًا من البهاء والمظهر الباهر، من الجمال والنعمة والرحمة، يشبه ستار عرش المجد؛ وعليه ثُبّتت جميع أنواع الأنوار في الكون.

\par 2 ووضعه عند باب القاعة السابعة وأجلسني عليه

\par 3 وخرج المنادي إلى كل سماء قائلاً: هذا ميتاترون، خادمي. لقد جعلته أميرًا وحاكمًا على جميع أمراء ممالكي وعلى جميع أبناء السماء، باستثناء الأمراء الثمانية العظام، المكرمين والمبجلين الذين يُدعون يهوه باسم ملكهم

\par 4 وكل ملاك وكل أمير لديه كلمة ليقولها في حضوري (أمامي) سيدخل إلى حضوره (أمامه) ويتحدث إليه (بدلاً من ذلك).

\par 5 وكل ما ينطق به لكم باسمي فاحفظوه وأتموه. لأني أوكلت إليه أمير الحكمة وأمير الفهم ليعلمه حكمة الأمور السماوية والأرضية، حكمة هذا العالم والعالم الآتي

\par 6 علاوة على ذلك، فقد جعلته على جميع خزائن قصور العرب، وعلى جميع مخازن الحياة التي في السماوات العليا

\chapter{11}

\par \textit{يكشف الله كل الأسرار والأسرار لميتاترون}

\par 1 قال ر. إسماعيل: قال لي ميتاترون الملاك أمير الحضور: منذ ذلك الحين كشف لي القدوس المبارك كل أسرار التوراة وكل أسرار الحكمة وكل أعماق القانون الكامل؛ وكشفت لي أفكار قلب جميع الكائنات الحية وكل أسرار الكون وكل أسرار الخلق كما كشفت لخالق الخلق.

\par 2 وراقبتُ باهتمامٍ لأُدرك أسرار العمق والغموض العجيب. [ABL: قبل أن يُفكّر الإنسان في الخفاء، رأيتُه، وقبل أن يصنع الإنسان شيئًا، رأيتُه.] [C: قبل أن يُفكّر الإنسان، عرفتُ ما يدور في خلده.]

\par 3 [ABL: ولم يكن شيء في الأعلى ولا في أعماق العالم مخفيًا عني.] [C: ولم يكن شيء في الأعلى ولا في أعماق العالم مخفيًا عني.]

\chapter{12}

\par \textit{ألبس الله ميتاترون ثوب المجد، ووضع تاجًا ملكيًا على رأسه، ودعاه "يهوه الأصغر"}

\par 1 قال ر. إسماعيل: قال لي ميتاترون أمير الحضور: بسبب المحبة التي أحبني بها القدوس المبارك أكثر من جميع أبناء السماء، صنع لي ثوب مجد مثبت عليه جميع أنواع الأضواء، وألبسني إياه.

\par 2 وألبسني ثوب شرف مثبت عليه كل أنواع الجمال والبهاء والتألق والجلال.

\par 3 وصنع لي تاجًا ملكيًا مثبتًا فيه تسعة وأربعون حجرًا كريمًا مثل ضوء كرة الشمس.

\par 4 فإن بهاءه قد خرج في أرباع العرب رقية، وفي السماوات السبع، وفي أرباع الأرض. ووضعه على رأسي.

\par 5 ودعاني يهوه الأصغر أمام جميع بيته السماوي، كما هو مكتوب: «لأن اسمي فيه».

\chapter{13}

\par \textit{يكتب الله بخط ملتهب على تاج ميتاترون الحروف الكونية التي خُلقت بها السماء والأرض}

\par 1 قال ر. إسماعيل: قال لي ميتاترون، الملاك، أمير الحضور، مجد كل السماوات: بسبب المحبة العظيمة والرحمة التي أحبني بها القدوس، تبارك اسمه، ورعاني أكثر من جميع أبناء السماء، كتب بإصبعه بقلم ملتهب على تاج رأسي الحروف التي بها خُلقت السماء والأرض، والبحار والأنهار، والجبال والتلال، والكواكب والأبراج، والبروق والرياح والزلازل والأصوات (الرعد)، والثلج والبرد، والرياح العاتية والعاصفة؛ الحروف التي بها خُلقت جميع احتياجات العالم وجميع أوامر الخلق

\par 2 وكل حرف يُرسل مرة بعد مرة كالبروق، مرة بعد مرة كالمصابيح، مرة بعد مرة كلهب النار، مرة بعد مرة (أشعة) كشروق الشمس والقمر والكواكب

\chapter{14}

\par \textit{جميع الأمراء الأعلى، والملائكة الابتدائية، والملائكة الكوكبية والنجمية يخافون ويرتعدون عند رؤية ميتاترون متوجًا}

\par 1 قال ر. إسماعيل: قال لي ميتاترون الملاك أمير الحضور: عندما وضع القدوس المبارك هذا التاج على رأسي، ارتعد من أمامي جميع أمراء الممالك الذين في ارتفاع أرابوث رقية وجميع جيوش كل السماء؛ وحتى أمراء العليم وأمراء الإريليم وأمراء التفساريم، الذين هم أعظم من جميع الملائكة الخادمين الذين يخدمون أمام عرش المجد، ارتعدوا وخافوا وارتجفوا من أمامي عندما رأوني.

\par 2 حتى صموئيل، رئيس المشتكين، الذي هو أعظم من كل رؤساء الممالك العليا، خاف وارتعد مني.

\par 3 وحتى ملاك النار، وملاك البَرَد، وملاك الريح، وملاك البرق، وملاك الغضب، وملاك الرعد، وملاك الثلج، وملاك المطر؛ وملاك النهار، وملاك الليل، وملاك الشمس، وملاك القمر، وملاك الكواكب، وملاك الأبراج الذين يحكمون العالم تحت أيديهم، خافوا وارتعدوا وفزعوا مني عندما رأوني

\par 4 هذه هي أسماء حكام العالم: جبرائيل ملاك النار، براديئيل ملاك البرد، روخيل الذي وُضع على الريح، باراكيل الذي وُضع على البرق، زئييل الذي وُضع على الشرر، زيئيل الذي وُضع على الهيجان، زئييل الذي وُضع على الريح العاتية، رعيئيل الذي وُضع على الرعود، رعئيل الذي وُضع على الزلزال، شلجيل الذي وُضع على الثلج، ماتاريئيل الذي وُضع على المطر، شيمشيئيل الذي وُضع على النهار، ليليلييل الذي وُضع على الليل، جلجالييل الذي وُضع على كرة الشمس، عوفانيئيل الذي وُضع على كرة القمر، كوكبيئيل الذي وُضع على الكواكب، راحتيل الذي وُضع على الأبراج.

\par 5 فسجدوا جميعهم عندما رأوني. ولم يستطيعوا أن ينظروا إليّ من بهاء المجد وجمال منظر نور إكليل المجد الذي على رأسي

\chapter{15}

\par \textit{ميتاترون يتحول إلى نار}

\par 1 قال ر. إسماعيل: قال لي ميتاترون الملاك أمير الحضور مجد كل السماوات: بمجرد أن أخذني القدوس المبارك في خدمته لحضور عرش المجد وعجلات (جالجاليم) المركبة واحتياجات الشكينة، تحول جسدي على الفور إلى لهيب، وأعصابي إلى نار ملتهبة، وعظامي إلى جمر من العرعر المحترق، وضوء جفوني إلى روعة البرق، ومقلة عيني إلى جمر نار، وشعر رأسي إلى لهب ساخن، وجميع أعضائي إلى أجنحة من نار مشتعلة وجسدي كله إلى نار متوهجة.

\par 2 وعلى يميني كانت هناك فرق من النيران المشتعلة، وعلى يساري كانت جمر النار مشتعلة، ومن حولي كانت رياح عاصفة وعواصف تهب، وأمامي وخلفي كان هدير الرعد مصحوبًا بزلزال

\chapter{15b}

\par \textit{إضافة في ب و ل}

\par 1 [ب: قال ر. إسماعيل: قال لي ميتاترون، أمير الحضور والأمير على جميع الأمراء - وهو يقف أمام] [ل: ميتاترون، هو أمير على جميع الأمراء - وهو يقف أمام] من هو أعظم من كل الإلوهيم. ويدخل تحت عرش المجد. وله خيمة نور عظيمة في الأعالي. ويُخرج نار الصمم ويضعها في آذان الكائنات المقدسة، حتى لا يسمعوا صوت الكلمة (الدبور) التي تخرج من فم الجلالة الإلهية

\par 2 ولما صعد موسى إلى الأعالي، صام صيامًا حتى انفتحت له مساكن الحشمال؛ ورأى [ب: القلب داخل قلب الأسد] [ل: رأى أنه أبيض كقلب الأسد] ورأى جيشًا لا يُحصى من حوله. وأرادوا أن يحرقوه. لكن موسى صلى من أجل الرحمة، أولًا لإسرائيل وبعد ذلك لنفسه: ففتح الجالس على المركبة النوافذ التي فوق رؤوس الكروبيم. وخرج حشد من المحامين -ورئيس الحضور، ميتاترون، معهم- للقاء موسى. وأخذوا صلوات إسرائيل ووضعوها كإكليل على رأس القدوس، تبارك اسمه.

\par 3 وقالوا: "اسمع يا إسرائيل، الرب إلهنا رب واحد" [ب: وأشرقت وجوههم وفرحوا على الشكينة] [ل: وأشرق وجه الشكينة وفرح] وقالوا لميترون: "ما هؤلاء؟ ولمن يعطون كل هذا الشرف والمجد؟" فأجابوا: "لرب إسرائيل المجيد". وتكلموا: [ب: "اسمع يا إسرائيل، الرب إلهنا رب واحد. لمن يُعطى وفرة الشرف والجلال إلا لك يا يهوه، الجلالة الإلهية، الملك الحي الأبدي".] [ل: "يهوه الحي الأبدي".]

\par 4 في تلك اللحظة، تكلم أكاتريل ياه يهود سباؤوث وقال لميتاترون، أمير الحضور: "لا تدع أي صلاة يصليها أمامي تعود (إليه) باطلة. اسمع صلاته وحقق رغبته سواء كانت كبيرة أم صغيرة".

\par 5 على الفور، قال ميتاترون، أمير الحضور، لموسى: "يا ابن عمرام! لا تخف، لأن الله الآن يُسر بك. واطلب ما تشتهيه من المجد والجلال. لأن وجهك يُشرق من أقاصي الأرض إلى أقاصيها". لكن موسى أجابه: "(أخشى) أن أجلب على نفسي إثمًا". قال له ميتاترون: "خذ حروف القسم، التي لا نقض فيها للعهد" (مما يمنع أي خرق للعهد).

\chapter{16}

\par \textit{تجريد ميتاترون من امتيازه في رئاسة عرش خاص به بسبب سوء فهم آكر في اعتباره قوة إلهية ثانية}

\par 1 قال ر. إسماعيل: قال لي ميتاترون، الملاك، أمير الحضور، مجد كل السماء: في البداية كنت جالسًا على عرش عظيم عند باب القاعة السابعة؛ وكنت أحكم على أبناء السماء، أهل البيت في الأعالي بسلطة القدوس، تبارك اسمه. وقسمت العظمة والملك والكرامة والحكم والشرف والثناء، والإكليل وتاج المجد لجميع أمراء الممالك، بينما كنت أترأس (حرفيًا: جالسًا) في البلاط السماوي (يشيبا)، وكان أمراء الممالك واقفين أمامي، عن يميني وعن يساري - بسلطة القدوس، تبارك اسمه

\par 2 "ولكن عندما جاء آكر ليرى رؤيا المركبة وثبت عينيه عليّ، خاف وارتجف أمامي وكانت روحه مرعوبة حتى أنها ابتعدت عنه بسبب الخوف والرعب والرعب مني، عندما رآني جالسًا على عرش مثل الملك مع جميع الملائكة الخادمين واقفين بجانبي كخدم لي وجميع أمراء الممالك مزينين بالتيجان المحيطة بي.

\par 3 في تلك اللحظة فتح فمه وقال: "في الواقع، هناك قوتان إلهيتان في السماء!"

\par 4 على الفور خرج باث قول (الصوت الإلهي) من السماء من أمام الشاكينا وقال: "ارجعوا أيها الأبناء المرتدون، إلا آخر!"

\par 5 ثم جاء أنييل، الأمير، المُبجَّل، المحبوب، الرائع، الموقر، والمهيب، بتكليف من القدوس، تبارك اسمه، وضربني ستين جلدة بسياط نار، وأوقفني على قدمي

\chapter{17}

\par \textit{أمراء السماوات السبع، والشمس والقمر، والكواكب والأبراج، وجنودها من الملائكة}

\par 1 قال ر. إسماعيل: قال لي ميتاترون، الملاك، أمير الحضور، مجد كل السماوات: سبعة أمراء، عظماء، جميلون، مبجلون، رائعون، ومكرمون، مُعيَّنون على السماوات السبع. وهؤلاء هم: [أ: ميكائيل، جبرائيل، شاتقيل، شاحقييل، بكارييل، بدارييل، فشرئيل.] [د: ميكائيل وجبرائيل، شاتقيل وباراديل، شاحقييل وباراديل، وسدرئيل.]

\par 2 وكل واحد منهم هو أمير جند السماء الواحدة. وكل واحد منهم يرافقه 496,000 ربوة من الملائكة الخادمين

\par 3 وميكائيل الأمير العظيم هو الذي يتولى السماء السابعة وهي السماء العليا التي في العربوت.

\par جبرائيل، أمير الجيش، مُعيَّن على السماء السادسة التي في ماكون.

\par شاتاقيل، أمير الجيش، هو المعين على السماء الخامسة التي في ماعون.

\par شحاقيئيل رئيس الجيش هو الذي يتولى السماء الرابعة التي في زبول.

\par بداريئيل رئيس الجيش هو الذي يتولى السماء الثالثة التي في شهاقيم.

\par باراكيل، رئيس الجيش، هو المعين على السماء الثانية التي في ارتفاع (ميروم) رقيا.

\par بازرئيل رئيس الجيش مُوَكَّل على السماء الأولى التي في ويلون التي في شمائيم.

\par 4 وتحتهم جلجاليل الأمير المعين على كرة الشمس (جلجال)، ومعه ملائكة عظماء ومكرمون يحركون الشمس في رقية.

\par 5 تحتهم عوفانئيل، الأمير المُقام على كوكب القمر. ومعه ملائكة يُحركون كوكب القمر ألف فرسخ كل ليلة، حين يكون القمر في الشرق عند دورته. ومتى يكون القمر في الشرق عند دورته؟ الجواب: في اليوم الخامس عشر من كل شهر.

\par 6 تحتهم راحتيئيل، الأمير المُكلَّف بالأبراج. ويرافقه اثنان وسبعون ملكًا عظيمًا مُكرَّمًا. ولماذا سُمِّي راحتيئيل؟ لأنه يُجري النجوم (مارهيت) في مداراتها ودوراتها 339 ألف فرسخ كل ليلة من الشرق إلى الغرب، ومن الغرب إلى الشرق. لأن القدوس، تبارك اسمه، قد نصب لهم جميعًا خيمةً للشمس والقمر والكواكب والنجوم التي يسافرون فيها ليلًا من الغرب إلى الشرق.

\par 7 تحتهم كوكبيل، الأمير المُعيَّن على جميع الكواكب. ومعه 365,000 عدد لا يحصى من الملائكة الخادمين، العظماء والمُكرَّمين الذين يُحرِّكون الكواكب من مدينة إلى أخرى ومن مقاطعة إلى مقاطعة في رقية السماوات

\par 8 وفوقهم اثنان وسبعون أميرًا من أمراء الممالك العليا، على غرار ألسنة العالم. وجميعهم متوجون بتيجان ملكية، ويرتدون ثيابًا ملكية، ويرتدون عباءات ملكية. وجميعهم يركبون خيولًا ملكية، ويحملون صولجانات ملكية في أيديهم. وأمام كل واحد منهم، عندما يسافر في رقية، يركض الخدم الملكيون بمجد وجلال عظيمين [أ: كما هو الحال على الأرض، يسافر الأمراء في مركبات مع فرسان وجيوش عظيمة، وفي مجد وعظمة مع تسبيح وغناء وشرف.] [د: وأمام كل واحد منهم، عندما يسافر في رقية، تركض جيوش عظيمة، كما هو الحال على الأرض، مع مركبات، في مجد وعظمة، تسبيح وغناء وشرف.]


\chapter{18}

\par \textit{ترتيب مراتب الملائكة والولاء الذي يتلقاه أصحاب المراتب العليا من أصحاب المراتب الدنيا}

\par 1 قال ر. إسماعيل: قال لي ميتاترون، الملاك، أمير الحضور، مجد السماء كلها: ملائكة السماء الأولى، كلما رأوا أميرهم، ترجلوا عن خيولهم وسقطوا على وجوههم

\par وأميرة السماء الأولى إذا رأت أميرة السماء الثانية نزلت وخلعت إكليل المجد عن رأسه وسقطت على وجهه.

وأمّا رئيس السماء الثانية، فإذا رأى رئيس السماء الثالثة، نزع إكليل المجد عن رأسه وسقط على وجهه.

وأمّا أمير السماء الثالثة، فإذا رأى أمير السماء الرابعة، نزع إكليل المجد عن رأسه وسقط على وجهه.

\par وأميرة السماء الرابعة إذا رأت أميرة السماء الخامسة نزعت إكليل المجد عن رأسها وسقطت على وجهه.

\par وأميرة السماء الخامسة إذا رأت أميرة السماء السادسة نزعت إكليل المجد عن رأسها وسقطت على وجهه.

\par وأميرة السماء السادسة إذا رأت أميرة السماء السابعة نزعت إكليل المجد عن رأسها وسقطت على وجهه.

\par 2 وأمّا رئيس السماء السابعة، فإذا رأى الاثنين والسبعين أميراً من أمراء الممالك، نزع إكليل المجد عن رأسه وسقط على وجهه.

\par وأمراء الممالك السبعون، عندما يرون حراس الباب الأول في رقية العربوت في الأعلى، ينزعون التاج الملكي عن رؤوسهم ويسقطون على وجوههم.

\par 3 وحراس القاعة الأولى، عندما يرون حراس القاعة الثانية، ينزعون إكليل المجد عن رؤوسهم ويسقطون على وجوههم

وأما بوابو الدار الثانية فلما رأوا بوابي الدار الثالثة نزعوا إكليل المجد عن رؤوسهم وسقطوا على وجوههم.

وأما بوابو الدار الثالثة فلما رأوا بوابي الدار الرابعة نزعوا إكليل المجد عن رؤوسهم وسقطوا على وجوههم.

وأما حراس الباب الرابع فلما رأوا حراس الباب الخامس نزعوا إكليل المجد عن رؤوسهم وسقطوا على وجوههم.

\par وأما حراس الباب الخامس، فلما رأوا حراس الباب السادس، نزعوا إكليل المجد عن رؤوسهم وسقطوا على وجوههم.

وأما بوابو الدار السادسة فلما رأوا بوابي الدار السابعة نزعوا إكليل المجد عن رؤوسهم وسقطوا على وجوههم.

\par 4 وعندما رأى حراس القاعة السابعة الأمراء الأربعة العظام، المكرمين، المعينين على معسكرات الشكينة الأربعة، نزعوا تاج (أكاليل) المجد عن رؤوسهم وسقطوا على وجوههم

\par 5 والأمراء الأربعة العظام، عندما يرون تاج، الأمير العظيم والمُكرّم بالأغاني والتسبيح، على رأس جميع أبناء السماء، ينزعون إكليل المجد عن رؤوسهم ويسقطون على وجوههم

\par 6 وتاغاس، الأمير العظيم الموقر، عندما رأى باراتيئيل، الأمير العظيم ذو الثلاثة أصابع في علو العربوت، أعلى السماء، نزع تاج المجد عن رأسه وسقط على وجهه

\par 7 وباراتيئيل، الأمير العظيم، عندما يرى هامون، الأمير العظيم، المخيف والمُكرّم، اللطيف والمخيف - الذي يجعل جميع أبناء السماء يرتعدون، عندما يقترب الوقت (المحدد) لقول القدوس (مثلث القدوس) كما هو مكتوب: "عند ضجيج الضجيج (هامون) تهرب الشعوب؛ وعند رفعك تتشتت الأمم" - يخلع إكليل المجد عن رأسه ويسقط على وجهه

\par 8 وهامون، الأمير العظيم، عندما رأى توترسيل، الأمير العظيم، نزع تاج المجد عن رأسه وسقط على وجهه

\par 9 وعندما رأى توترسيل، الأمير العظيم، أتروجيل، الأمير العظيم، نزع تاج المجد عن رأسه وسقط على وجهه

\par 10 وأتروجيل الأمير العظيم عندما يرى نعريرئيل الأمير العظيم ينزع تاج المجد عن رأسه ويسقط على وجهه.

\par 11 وعندما رأى نآريئيل هـ، الأمير العظيم، ساسنيجيل، الأمير العظيم، نزع تاج المجد عن رأسه وسقط على وجهه

\par 12 وساسنيجيل هـ، عندما رأى زازرئيل هـ، الأمير العظيم، نزع تاج المجد عن رأسه وسقط على وجهه

\par 13 فزرئيئيل هـ، الأمير، عندما رأى جبرائيل هـ، الأمير، نزع إكليل المجد عن رأسه وسقط على وجهه

\par 14 وعندما رأى جبرائيل هـ، الأمير، أرافائيل هـ، نزع تاج المجد عن رأسه وسقط على وجهه

\par 15 وعندما رأى الأمير "أرافائيل هـ" الأمير "أشرويلو" الذي يترأس جميع جلسات أبناء السماء، نزع تاج المجد عن رأسه وسقط على وجهه

\par 16 وعندما رأى أشرويلو هـ، الأمير، جاليسور هـ، الأمير، الذي يكشف جميع أسرار الشريعة (التوراة)، نزع تاج المجد عن رأسه وسقط على وجهه

\par 17 وعندما رأى غالسور هـ، الأمير، زكزاكييل هـ، الأمير المعين لكتابة فضائل إسرائيل على عرش المجد، نزع تاج المجد عن رأسه وسقط على وجهه

\par 18 وزكزاكييل هـ، الأمير العظيم، عندما رأى 'أنف(ي)هـ'، الأمير الذي يحمل مفاتيح القاعات السماوية، نزع تاج المجد عن رأسه وسقط على وجهه. لماذا يُدعى باسم 'أنف(ي)هـ؟ لأن غصن شرفه وجلاله وتاجه وروعته وبهائه يغطي (يُظلل) جميع غرف 'أرابوث راقِيَة' في الأعالي، كما يُظللها خالق العالم. وكما هو مكتوب عن خالق العالم: "غطّى مجده السماوات، وامتلأت الأرض من تسبيحه"، كذلك يغطي شرف وجلال 'أنف(ي)هـ جميع أمجاد 'أرابوث الأعالي

\par 19 وعندما يرى سوثر "أشيل هـ"، الأمير، العظيم، المهيب والمُكرّم، ينزع إكليل المجد عن رأسه ويسقط على وجهه. لماذا يُدعى سوثر أشيل؟ لأنه مُعيّن على رؤوس النهر الناري الأربعة مقابل عرش المجد؛ وكل أمير يخرج أو يدخل أمام الشكينة، لا يخرج أو يدخل إلا بإذنه. لأن أختام النهر الناري مُوكلة إليه. وعلاوة على ذلك، يبلغ طوله 7000 ربوة من الفراسخ. ويُثير نار النهر؛ ويخرج ويدخل أمام الشكينة ليشرح ما هو مكتوب (مُسجّل) عن سكان العالم. كما هو مكتوب: "وُضع الحكم، وفُتحت الأسفار".

\par 20 وسوذر أشيل الأمير، عندما رأى شوقد حزي، الأمير العظيم، الجبار، المهيب، المكرم، نزع تاج المجد عن رأسه وسقط على وجهه. ولماذا سمي شوقد حزي؟ لأنه يزن جميع محاسن الإنسان في ميزان أمام القدوس، تبارك اسمه.

\par 21 وعندما يرى زيهانبوريو، الأمير العظيم، الجبار والمهيب، مُكرّمًا وممجّدًا ومهيبًا في كل البيت السماوي، ينزع إكليل المجد عن رأسه ويسقط على وجهه. لماذا يُدعى زيهانبوريو؟ لأنه يوبخ النهر الناري ويدفعه إلى مكانه

\par 22 وعندما يرى "أزبوغا هـ"، الأمير العظيم، المُمجَّد، المُبجَّل، المُكرَّم، المُزيَّن، الرائع، المُمجَّد، المحبوب، والمهيب بين جميع الأمراء العظماء الذين يعرفون سر عرش المجد، يخلع تاج المجد عن رأسه ويسقط على وجهه. لماذا يُدعى "أزبوغا"؟ لأنه في المستقبل سيُزيِّن (يكسو) صالحي العالم وأتقيائه بثياب الحياة، ويلفُّهم بعباءة الحياة، ليحيوا فيها حياة أبدية

\par 23 وعندما يرى الأميرين العظيمين، القويين والممجدين، الواقفين فوقه، ينزع إكليل المجد عن رأسه ويسقط على وجهه. وهذان اسما الأميرين:

\بار سوفرييل هـ (الذي) يقتل، (سوفرييل هـ القاتل)، الأمير العظيم، المكرم، المجيد، الذي لا عيب فيه، الجليل، القديم والقوي؛ (و) سوفرييل هـ (الذي) يحيي (سوفرييل هـ المعطي للحياة)، الأمير العظيم، المكرم، المجيد، الذي لا عيب فيه، القديم والقوي.

\par 24 لماذا يُدعى سوفرييل القاتل (سوفرييل القاتل)؟ لأنه مُكلَّف بسجلات الموتى: [حتى] يكتب كل شخص، عندما يقترب يوم وفاته، اسمه في سجلات الموتى

لماذا يُدعى صوفريل هـ المُحيي (صوفريل هـ المُحيي)؟ لأنه مُكلّفٌ بكتب الأحياء، فكل من يُحييه القدوس تبارك اسمه، يكتبه في كتاب الأحياء، بسلطة مقام. قد تقول: "بما أن القدوس تبارك اسمه جالسٌ على عرش، فهم أيضًا جالسون عند الكتابة". (الجواب): يُعلّمنا الكتاب المقدس: "وكلّ جند السماء واقفون عنده". يُقال: "جند السماء" ليُظهر لنا أن حتى الأمراء العظام، الذين لا مثيل لهم في السموات العليا، لا يُلبّون طلبات الشكينة إلا بالوقوف. ولكن كيف يُمكنهم الكتابة وهم واقفون؟ الأمر كالتالي:

\par 25 يقف أحدهما على عجلات العاصفة، والآخر على عجلات رياح العاصفة

\par الواحد يرتدي ثيابًا ملكية، والآخر يرتدي ثيابًا ملكية.

\par أحدهما ملفوف في عباءة الجلالة والآخر ملفوف في عباءة الجلالة.

\par أحدهما متوج بتاج ملكي، والآخر متوج بتاج ملكي.

\par جسد أحدهما مليء بالعيون، وجسد الآخر مليء بالعيون.

\par منظر الواحد كمنظر البرق ومنظر الآخر كمنظر البرق.

\par عينا أحدهما كالشمس في قوتها، وعين الآخر كالشمس في قوتها.

\par ارتفاع أحدهما كارتفاع سبع سماوات، وارتفاع الآخر كارتفاع سبع سماوات.

\par أجنحة الواحد عدد أيام السنة، وأجنحة الآخر عدد أيام السنة.

\par جناحا أحدهما يمتدان إلى عرض راقيت، وجناحا الآخر يمتدان إلى عرض راقية!.

\par شفتا الواحد كأبواب الشرق، وشفتا الآخر كأبواب الشرق.

لسان أحدهما كأمواج البحر، ولسان الآخر كأمواج البحر.

من فم الواحد يخرج لهيب، ومن فم الآخر يخرج لهيب.

\par من فم الواحد يخرج بروق ومن فم الآخر يخرج بروق.

من عرق أحدهما تشتعل النار، ومن عرق الآخر تشتعل النار.

من لسان أحدهما يشتعل شعلة، ومن لسان الآخر يشتعل شعلة.

\par على رأس أحدهما حجر الياقوت، وعلى رأس الآخر حجر الياقوت.

\par على كتفي الواحد عجلة كروب سريع، وعلى كتفي الآخر عجلة كروب سريع.

\par أحدهما يحمل في يده مخطوطة مشتعلة، والآخر يحمل في يده مخطوطة مشتعلة.

\par أحدهما في يده قلم ملتهب، والآخر في يده قلم ملتهب.

طول المخطوطة آلاف من الفراسخ، وحجم الخط ثلاثة آلاف ألف فرسخ، وحجم كل حرف يكتبونه ثلاثمائة وخمسة وستين فرسخاً.

\chapter{19}

\par \textit{ريكبيل، أمير عجلات المركبة. محيط المركبة. الاضطراب بين الجيوش الملائكية في زمن القدوشا}

\par 1 قال ر. إسماعيل: قال لي ميتاترون، الملاك، أمير الحضور: فوق هؤلاء الملائكة الثلاثة، هؤلاء الأمراء العظماء، أمير واحد، مُمَيَّز، مُشَرَّف، نبيل، مُمَجَّد، مُزَيَّن، مُهيب، شجاع، قوي، عظيم، مُبجَّل، مُجيد، مُتَوَّج، رائع، مُمَجَّد، بلا لوم، محبوب، مُلَك، عالٍ وشامخ، قديم وعظيم، ليس مثله أحد بين الأمراء. اسمه ريكبيل هـ، الأمير العظيم والمُبجَّل الذي يقف بجانب المركبة

\par 2 ولماذا يُسمّى ريكبيل؟ لأنه مُكلّفٌ بعجلات المركبة، وهي مُسلَّمةٌ إليه.

\par 3 وكم عدد العجلات؟ ثماني عجلات؛ اثنتان في كل اتجاه. وتحيط بها أربع رياح. وهذه أسماؤها: "ريح العاصفة"، و"العاصفة"، و"الريح الشديدة"، و"ريح الزلزال".

\par 4 وتحتها أربعة أنهار نارية تجري باستمرار، نهر ناري واحد على كل جانب. وحولها، بين الأنهار، أربع سحب مزروعة، وهي: "سحب من نار"، و"سحب من مصابيح"، و"سحب من فحم"، و"سحب من كبريت"، وهي واقفة مقابل عجلاتها

\par 5 وأقدامُ الحِيوثِ تستقرُّ على البكرات. وبينَ البكرةِ والأخرى زلزلةٌ هديرٌ ورعدٌ هادر

\par 6 وعندما يقترب وقت تلاوة النشيد، تتحرك حشود العجلات، وترتجف حشود السحب، ويخاف جميع الرؤساء (شاليشيم)، ويثور جميع الفرسان (باراشيم)، ويهتاج جميع الجبابرة (جتبوريم)، وترتجف جميع الجيوش (سيبايم)، وتخاف جميع القوات (جدوديم)، ويسرع جميع المعينين (ميمونيم)، ويرتجف جميع الأمراء (ساريم) والجيوش (حاييليم)، ويضعف جميع الخدم (مشيراتيم)، وتعاني جميع الملائكة (مافاكيم) والفرق (ديغاليم) من الألم

\par 7 وتُصدر عجلة واحدة صوتًا يُسمع للأخرى، وكروب واحد لآخر، وحايا واحد لآخر، وسراف واحد لآخر (قائلين) "سبّحوا للراكب في العربوت باسمه ياه، وافرحوا أمامه!"

\chapter{20}

\par \textit{حييل، أمير الحيوت}

\par 1 قال ر. إسماعيل: قال لي ميتاترون، الملاك، أمير الحضور: فوق هؤلاء أمير عظيم وعظيم. اسمه شايليل هـ، أمير نبيل ومبجل، أمير مجيد وقوي، أمير عظيم ومبجل، أمير يرتجف أمامه جميع أبناء السماء، أمير قادر على ابتلاع الأرض كلها في لحظة (بملعقته).

\par 2 ولماذا يُدعى حييليئيل هـ؟ لأنه مُعيّن على الحيوت المقدسة ويضرب الحيوت بسياط من نار: ويمجدهم عندما يُسبّحون ويُمجّدون ويفرحون، ويجعلهم يُسارعون إلى قول "قدوس" و"مبارك مجد هـ من مكانه!" (أي القدوشة).

\chapter{21}

\par \textit{الخيوث}

\par 1 قال رَ. إسماعيل: قال لي ميتاترون، الملاك، أمير الحضور: أربعة (هم) الخيوث، أي الرياح الأربع. كل خية هي بمثابة فضاء العالم كله. ولكل منها أربعة وجوه؛ وكل وجه كوجه الشرق

\par 2 كل واحد منهم لديه أربعة أجنحة، وكل جناح هو بمثابة غطاء (سقف) للكون.

\par 3 ولكل واحد منها وجوه في وسط وجوه وأجنحة في وسط أجنحة، وحجم الوجوه كحجم الوجوه وحجم الأجنحة كحجم الأجنحة.

\par 4 وكل واحد مُتوّج بتيجان على رأسه. وكل تاج كالقوس في السحاب. وبهاؤه كبهاء كرة الشمس. والشرر المنبعث من كل واحد كبهاء كوكب الزهرة في المشرق.

\chapter{22}

\par \textit{كروبيئيل، أمير الكروبيم. وصف الكروبيم}

\par 1 قال ر. إسماعيل: قال لي ميتاترون، الملاك، أمير الحضور: فوق هؤلاء أمير واحد، نبيل، عجيب، قوي، ومُمدح بكل أنواع المديح. اسمه كروبيئيل هـ، أمير عظيم، مليء بالقوة والقدرة

\par [AD: أمير ذو جلالة، وسمو (معه)، أمير بار، وبر (معه)، أمير مقدس، وقداسة (معه)، أمير] [B: أمير ذو جلالة، ومعه (هناك) أمير بار، من البر، ومعه أمير مقدس، من القداسة، ومعه (هناك) أمير] ممجد في (بواسطة) آلاف جيش، ومرتفع بعشرة آلاف جيش.

\par 2 من غضبه ترتجف الأرض، ومن غضبه تتهاوى المعسكرات، ومن خوفه تهتز الأساسات، ومن توبيخه يرتجف العرب

\par 3 قامته مملوءة جمرًا. طول قامته كارتفاع سبع سماوات، وعرض قامته كعرض سبع سماوات، وسمك قامته كسبع سماوات

\par 4 فتحة فمه كمصباح نار. لسانه نار آكلة. حاجباه كبهاء البرق. عيناه كشرارات من تألق. وجهه كنار مشتعلة

\par 5 وعلى رأسه إكليل قداسة منقوش عليه الاسم الصريح، وتخرج منه بروق. وقوس الشكينة بين كتفيه

\par 6 [إعلان: وسيفه على حقويه، وسهامه كالبرق في منطقته. وعلى عنقه درع من نار آكلة، وحوله جمر من عرعر.] [ب: وسيفه كالبرق، وعلى حقويه سهام كاللهيب، وعلى درعه وترسه نار آكلة، وعلى عنقه جمر من عرعر مشتعل، وحوله جمر من عرعر مشتعل.]

\par 7 وبهاء الشكينة على وجهه، وقرون الجلالة على عجلاته، وإكليل ملكي على جمجمته

\par 8 وجسمه مليء بالعيون. والأجنحة تغطي كل قامته الطويلة (حرفيًا: طول قامته هو كل الأجنحة).

\par 9 على يمينه لهب مشتعل، وعلى يساره نار متوهجة، وجمر يحترق منها. وتخرج جمر من جسده. وتنطلق البرق من وجهه. معه رعد على رعد، وبجانبه زلزلة على زلزلة

\par 10 وأمراء المركبة معه.

\par 11 لماذا يُدعى كروبيئيل هـ، أي الأمير؟ لأنه مُكلَّفٌ على مركبة الكروبيم، ويُوكَل إليه الكروبيم الأقوياء، فيزيِّن التيجان على رؤوسهم، ويُصَقِّل التيجان على جماجمهم.

\par 12 يُعَظِّمُ مَجْدَ ظَاهِرَتِهِمْ. ويمجِّدُ بَهَاءَ جَلالِهِمْ. ويزِيدُ عَظَمَةَ مَجْدِهِمْ. يُسْمِعُ أُنْشِيدَةَ تَسْبِيحِهِمْ. يُشَدِّدُ قُوَّتِهِمْ الْجَمِيلِ. يُظْهِرُ بُهَاءَ مَجْدِهِمْ. يُجَمِّلُ رَحْمَتَهُمْ الْحَسَنَةَ وَرَحْمَتَهُمْ. يُؤَطِّرُ بَهاءَ شُعْرِهِمْ. يَجْعَلُ جَمالَهُمْ الرَّحِيمَ أَكْثَرَ جَمالاً. يُمَجِّدُ تَسْبِيحَهِمْ، لِيُثَبِّتَ مَسْكَنَهُ "الْجَالِسَ عَلَى الْكُرُوبِيم".

\par 13 والكروبيم واقفون عند الحيوت المقدسة، وأجنحتهم مرفوعة إلى رؤوسهم (حرفيًا: كارتفاع رؤوسهم)
\par و شكينة (تستقر) عليهم
\par و بريق المجد على وجوههم
\par والغناء والتسبيح في أفواههم
\par وأيديهم تحت أجنحتهم
\par وأقدامهم مغطاة بأجنحتهم
\par وقرون المجد على رؤوسهم
\par وروعة الشكينة على وجوههم
\par و شكينة (تستقر) عليهم
\par وأحجار الياقوت تحيط بهم
\par وأعمدة من النار على جوانبها الأربعة
\par وأعمدة من المشاعل بجانبها.

\par 14 هناك ياقوتة واحدة على جانب وياقوتة أخرى على الجانب الآخر، وتحت الياقوتات توجد جمر من شجر العرعر المشتعل

\par 15 وكروب واحد يقف في كل اتجاه، لكن أجنحة الكروبيم تحيط بعضها ببعض فوق جماجمهم في مجد، وينشرونها ليُغنوا معهم ترنيمة للساكن في السحاب، وليُسبّحوا معهم جلال ملك الملوك المهيب

\par 16 وكروبيئيل هـ، الأمير المُعيّن عليهم، يُزيّنهم بأزياء جميلة وجميلة ومُبهجة، ويرفعهم بكل أنواع السمو والكرامة والمجد. ويُعجّلهم - بمجد وقوة - ليفعلوا مشيئة خالقهم في كل لحظة. لأن فوق رؤوسهم العالية يستقرّ دائمًا مجد الملك العلي "الساكن على الكروبيم".

\chapter{22b}

\par 1 [L(mr)، بعد وصف الفصل 22 ج، الآيات 1-3 (الوسط): وأمام عرش المجد،] [B: قال لي ر. إسماعيل: قال لي ميتاترون، الملاك، أمير الحضور: كيف تقف الملائكة في الأعالي؟ قال: مثل جسر يوضع فوق نهر حتى يتمكن الجميع من المرور فوقه، كذلك يوضع جسر من بداية المدخل إلى نهايته.]

\par 2 [لمَر الذي لا يستطيع السرافيم ولا الملائكة دخوله، وهو ستة آلاف ربوة فراسخ، كما هو مكتوب: "والسيرافيم واقفون فوقه" (آخر كلمة في النص الكتابي هي [القيمة العددية: ٣٦]).] [ب: ويحيط به ثلاثة ملائكة خادمين، ويرددون ترنيمة أمام الرب إله إسرائيل. وهناك يقف أمامه رؤساء رعب وقادة خوف، آلاف وعشرات الآلاف، وهم يغنون التسبيح والترانيم أمام الرب إله إسرائيل.]

\par 3 [Lmr: بما أن القيمة العددية لـ (36) هي عدد الجسور هناك.] [B: هناك العديد من الجسور: جسور من النار وجسور من البَرَد. أيضًا العديد من أنهار البَرَد، وخزانات الثلج العديدة، وعجلات النار العديدة.]

\par 4 [Lmr: وهناك ربوات من عجلات النار. والملائكة الخادمون هم 12000 ربوة. وهناك 12000 نهر من البرد، و12000 خزانة من الثلج. وفي القاعات السبع عربات من نار ولهب، بلا حساب، ولا نهاية، ولا بحث. (ينتهي Lmr هنا.)] [B: وكم عدد الملائكة الخادمين؟ 12000 ربوة: ستة (آلاف ربوة) فوق وستة (آلاف ربوة) تحت. و12000 هي خزائن الثلج، ستة فوق وستة تحت. وربوات من عجلات النار، 12 (آلاف) فوق و12 (آلاف) تحت. وهم يحيطون بالجسور وأنهار النار وأنهار البرد وهناك عدد كبير من الملائكة الخادمين، الذين يشكلون مداخل، لجميع المخلوقات التي تقف في وسطها \ تتوافق (مقابل) مع طرق رقطة شميم. ]

\par 5 ماذا يفعل الرب إله إسرائيل ملك المجد؟ الإله العظيم المهيب، الشديد القوة، يغطي وجهه

\par 6 في العربوث ستمائة وستون ألفًا من ملائكة المجد واقفين مقابل عرش المجد وأقسام نار ملتهبة. ويغطي ملك المجد وجهه، وإلا لتمزقت العربوث رقية في وسطها من جلال وبهاء وجمال وإشراق وجمال وبهاء ولمعان وإشراق وفخامة ظهور القدوس تبارك وتعالى

\par 7 هناك العديد من الملائكة الخادمين الذين ينفذون إرادته، والعديد من الملوك، والعديد من الأمراء في أرض سروره، ملائكة محترمون بين حكام السماء، متميزون، مزينون بالأغاني، ويذكرون الحب: (الذين) يخافون من روعة الشكينة، وتبهر عيونهم بجمال ملكهم المتألق، وتسود وجوههم وتضعف قوتهم

\par 8 تتدفق أنهار الفرح، وجداول البهجة، وأنهار الابتهاج، وأنهار النصر، وأنهار الحب، وأنهار الصداقة - (قراءة أخرى:) من الاضطراب - وتتدفق وتخرج أمام عرش المجد وتزداد عظمة وتمر عبر أبواب دروب عربوث رقية على صوت هتاف وأغاني الحيوت، وعلى صوت فرح دفوف أوفانيمه وعلى لحن صنج كروبيمه. وتزداد عظمة وتخرج بضجة مع صوت الترنيمة: "قدوس، قدوس، قدوس، رب الجنود؛ الأرض كلها مملوءة من مجده!"

\chapter{22c}

\par \textit{في ب، لو، ولمر)}

\par 1 قال إسماعيل: قال لي ميتاترون أمير الحضرة: ما المسافة بين جسر وآخر؟ اثنا عشر ألف فراسخ. صعودهم آلاف فراسخ، ونزولهم اثنا عشر ألف فراسخ

\par 2 المسافة بين أنهار الرعب وأنهار الخوف 22 ألف فراسخ؛ بين أنهار البرد وأنهار الظلام 36 ألف فراسخ؛ بين غرف البرق وسحب الرحمة 42 ألف فراسخ؛ بين سحب الرحمة والمركبة 148 ألف فراسخ؛ بين المركبة والكروب 148 ألف فراسخ؛ بين الكروبيم والأوفانيم 24 ألف فراسخ؛ بين الأوفانيم وغرف الغرف 24 ألف فراسخ؛ بين غرف الغرف والحيوت المقدسة 40,000 ألف فراسخ؛ "وبين جناح وآخر اثني عشر ألف فرسخ، وعرض كل جناح منها مقدار ذلك، والمسافة بين الجناح المقدس وعرش المجد ثلاثون ألف فرسخ."

\par 3 ومن أسفل العرش إلى المقعد أربعون ألف ربوة فراسخ. واسم الجالس عليه: ليتقدس الاسم!

\par 4 [وأقواس القوس منصوبة فوق العربوت، وارتفاعها ألف ألف وعشرة آلاف فراسخ. قياسها كقياس العيرين والقاديشين (المراقبين والقديسين). كما هو مكتوب: "وضعت قوسي في السحاب". لم يُكتب هنا "سأضع" بل "وضعت"، أي: سحبًا تُحيط بعرش المجد. وعندما تمر سحبه، تتحول ملائكة البَرَد إلى جمر متقد

\par 5 وتنزل نار الصوت من عند الحويوث المقدسة. وبسبب نَفَس ذلك الصوت "يركضون" إلى مكان آخر، خائفين من أن يأمرهم بالذهاب؛ و"يعودون" لئلا يؤذيهم من الجانب الآخر. لذلك "يركضون ويرجعون".

\par 6 وهذه الأقواس من القوس أجمل وأكثر إشراقًا من إشراق الشمس خلال الانقلاب الصيفي. وهي أكثر بياضًا من نار مشتعلة، وهي عظيمة وجميلة

\par 7 فوق أقواس القوس عجلات العُفّان. ارتفاعها ألف وعشرة آلاف وحدة قياس، قياس السرافيم والجُدوديم.]

\chapter{23}

\par \textit{الرياح تهب "تحت أجنحة الكروبيم"}

\par 1 قال ر. إسماعيل: قال لي ميتاترون، الملاك، أمير الحضور: هناك رياح عديدة تهب تحت أجنحة الكروبيم. هناك تهب "الريح الحاضنة"، كما هو مكتوب: "وكانت ريح الله تهب على وجه المياه".

\par 2 تهب "ريح قوية"، كما قيل: "فجعل الرب البحر يرجع بريح شرقية قوية طوال تلك الليلة".

\par 3 تهب "ريح الشرق" كما هو مكتوب: "جلبت ريح الشرق الجراد".

\par 4 تهب "ريح السلوى" كما هو مكتوب: "وخرجت ريح من عند الرب وجلبت السلوى".

\par 5 تهب "ريح الغيرة" كما هو مكتوب: "فجاءت عليه ريح الغيرة".

\par 6 تهب "ريح الزلزلة" كما هو مكتوب: "وبعد ذلك هبت ريح الزلزلة، ولكن الرب لم يكن في الزلزلة".

\par 7 تهب "ريح هـ" كما هو مكتوب: "فحملني بريح هـ وأنزلني".

\par 8 تهب "الريح الشريرة" كما هو مكتوب: "فَذَهَبَتْ عَنْهُ الرِّيحُ الشِّرِّيرَةُ".

\par 9 تهب "ريح الحكمة" و"ريح الفهم" و"ريح المعرفة" و"ريح الخوف من الله" كما هو مكتوب: "وتستقر عليه ريح الله؛ ريح الحكمة والفهم، ريح المشورة والقوة، ريح المعرفة والخوف من الله".

\par 10 تهب "ريح المطر"، كما هو مكتوب: "ريح الشمال تُعطي مطرًا".

\par 11 تهب "ريح البرق"، كما هو مكتوب: "يصنع بروقًا للمطر ويخرج الريح من خزائنه".

\par 12 تهب "الريح، فتكسر الصخور"، كما هو مكتوب: "مرّ الرب وريح عظيمة وقوية (تشقّ الجبال وتكسر الصخور أمام الرب)".

\par 13 تهب "ريح تهدئة البحر"، كما هو مكتوب: "فأجاز الله ريحًا على الأرض، فهدأت المياه".

\par 14 تهب "ريح الغضب"، كما هو مكتوب: "وإذا ريح عظيمة جاءت من البرية وضربت زوايا البيت الأربع فسقط".

\par 15 تهب "ريح العاصفة"، كما هو مكتوب: "ريح العاصفة، تُتمم كلمته".

\par 16 والشيطان يقف بين هذه الرياح، لأن "ريح العاصفة" ليست سوى "الشيطان"، وكل هذه الرياح لا تهب إلا تحت أجنحة الكروبيم، كما هو مكتوب: "وركب على كروب وطار، نعم، وطار سريعًا على أجنحة الريح".

\par 17 وإلى أين تذهب كل هذه الرياح؟ يُعلّمنا الكتاب أنها تخرج من تحت أجنحة الكروبيم وتهبط على كرة الشمس، كما هو مكتوب: "الريح تذهب نحو الجنوب وتدور نحو الشمال، تدور باستمرار في مجراها ثم تعود إلى مداراتها". ومن كرة الشمس تعود وتهبط على الأنهار والبحار، على الجبال والتلال، كما هو مكتوب: "فهوذا صانع الجبال وخالق الريح".

\par 18 ومن الجبال والتلال يعودون وينزلون إلى البحار والأنهار؛ ومن البحار والأنهار يعودون وينزلون على المدن والأقاليم؛ ومن المدن والأقاليم يعودون وينزلون إلى الجنة، ومن الجنة يعودون وينزلون إلى عدن، كما هو مكتوب: "متمشين في الجنة في ريح النهار". وفي وسط الجنة يجتمعون وينفخون من جانب إلى آخر ويتعطرون بأطياب الجنة حتى من أقصى أجزائها، حتى ينفصلون عن بعضهم البعض، ويمتلئون برائحة الأطياب النقية، ويحملون الرائحة من أقصى أرجاء عدن وأطياب الجنة إلى الأبرار والأتقياء الذين سيرثون في المستقبل جنة عدن وشجرة الحياة، كما هو مكتوب: "استيقظي يا ريح الشمال؛ وتعالي جنوبًا؛ هبّي على جنتي، فيفيض أطيابها. ليأتِ حبيبي إلى جنته ويأكل ثمرها الثمين".

\chapter{24}

\par \textit{مركبات القدوس المختلفة، تبارك اسمه}

\par 1 قال ر. إسماعيل: قال لي ميتاترون، الملاك، أمير الحضور، مجد كل السماء: لدى القدوس، تبارك اسمه، مركبات عديدة: لديه "مركبات الكروبيم"، كما هو مكتوب: "وركب على كروب وطار".

\par 2 لديه "مركبات الريح"، كما هو مكتوب (ib.): "وطار سريعًا على أجنحة الريح"

\par 3 لديه "مركبات السحاب السريع"، كما هو مكتوب: "هوذا الرب راكب على سحابة سريعة".

\par 4 لديه "مركبات السحاب"، كما هو مكتوب: "ها أنا آتي إليك في سحابة".

\par 5 لديه "مركبات المذبح"، كما هو مكتوب: "رأيت الرب واقفًا على المذبح".

\par 6 لديه "مركبات ريبوتايم"، كما هو مكتوب: "مركبات الله ريبوتايم؛ آلاف من الملائكة".

\par 7 لديه "مركبات الخيمة"، كما هو مكتوب: "وظهر الرب في الخيمة في عمود سحاب".

\par 8 لديه "مركبات المسكن"، كما هو مكتوب: "وكلمه الرب من المسكن".

\par 9 لديه "مركبات غطاء الرحمة"، كما هو مكتوب: "ثم سمع صوتًا يكلمه من على غطاء الرحمة".

\par 10 كان لديه "عربات من حجر الياقوت"، كما هو مكتوب: "وكان تحت قدميه كأنها عمل مرصوف من حجر الياقوت".

\par 11 لديه "مركبات النسور"، كما هو مكتوب: "أحملكم على أجنحة النسور". النسور هنا ليست المقصودة حرفيًا، بل "الطائرة كالنسور".

\par 12وله «مركبات الهتاف»، كما هو مكتوب: «صعد الله بهتاف».

\par 13 لديه "مركبات العربات"، كما هو مكتوب: "سبّحوا الراكب على العربات".

\par 14 لديه "مركبات السحاب الكثيف"، كما هو مكتوب: "الذي يجعل السحاب الكثيف مركبته".

\par 15 لديه "مركبات الحيوث" كما هو مكتوب: "وركض الحيوث ورجعوا". يركضون بإذن ويعودون بإذن، لأن الشكينة فوق رؤوسهم

\par 16 لديه "عربات العجلات (جلجاليم)"، كما هو مكتوب: "وقال: ادخل بين العجلات الدوارة".

\par 17 لديه "مركبات كروب سريع"، كما هو مكتوب (؟): "راكبًا على كروب سريع".

وعندما يركب على جواد سريع، يضع إحدى قدميه عليه، قبل أن يضع الأخرى على ظهره، ينظر إلى ثمانية عشر ألف عالم في لمحة واحدة. فيميز ويبصر في كل منها، ويعرف ما فيها جميعًا، ثم يضع الأخرى عليه، كما هو مكتوب: "حوالي ثمانية عشر ألفًا".

من أين لنا أن نعرف أنه كان ينظر إلى كل واحد منهم يوميًا؟ مكتوب: «نظر من السماء إلى بني البشر ليرى إن كان هناك من يفهم ويسعى إلى الله».

\par 18 لديه "مركبات الأوفنيم"، كما هو مكتوب: "وكانت الأوفنيم مليئة بالعيون من حولها".

\par 19 لديه "مركبات عرشه المقدس"، كما هو مكتوب: "الله جالس على عرشه المقدس".

\par 20 لديه "مركبات عرش يهوه"، كما هو مكتوب: "لأن يدًا مرفوعة على عرش يهوه".

\par 21 لديه "مركبات عرش الدينونة"، كما هو مكتوب: "ولكن رب الجنود يرتفع في القضاء".

\par 22 لديه "مركبات عرش المجد"، كما هو مكتوب: "عرش المجد، الموضوع عالياً منذ البدء، هو موضع مقدسنا".

\par 23 لديه "مركبات العرش العالي والمرتفع"، كما هو مكتوب: "رأيت الرب جالسًا على العرش العالي والمرتفع".

\chapter{25}

\par \textit{عوفانييل أمير العوفان. وصف العوفانيم}

\par 1 قال ر. إسماعيل: قال لي ميتاترون، الملاك، أمير الحضور: فوق هؤلاء يوجد أمير عظيم، مُبجل، عالٍ، مُهيب، مُهيب، قديم وقوي، اسمه "أفانيل هـ".

\par 2 له ستة عشر وجهًا، أربعة أوجه على كل جانب، (وأيضًا) مئة جناح على كل جانب. وله عيون، تُقابل أيام السنة. [أ: ٢١٩٠ - وبعضهم يقول ٢١١٦ - على كل جانب.] [د: ٢١٩١ (هـ: ١٩٦) وستة عشر على كل جانب.]

\par 3 وتلك العينان في وجهه، في كل واحدة منهما تشرق البرق، ومن كل واحدة منهما جمر متقد، ولا يستطيع أي مخلوق أن ينظر إليهما، لأنه من ينظر إليهما يحترق في لحظة.

\par 4 طوله مسيرة سنوات. لا عين تراه، ولا فم يروي عظمة قوته إلا ملك الملوك، القدوس، تبارك هو وحده

\par 5 لماذا يُدعى عوفنيئيل؟

\par لأنه مُكلَّف على العوفنيم، والعوفنيم مُكلَّفون تحت رعايتِه. يقف كل يوم ويعتني بهم ويُجمِّلهم. ويُعلي ويُرتِّب مساكنهم (بالألمانية: جريًا) ويُلمِّع مكان وقوفهم، ويُنير مساكنهم، ويجعل زواياهم مستوية، ويُطهِّر مقاعدهم. ويخدمهم في الصباح الباكر وفي وقت متأخر، ليلًا ونهارًا، ليزيد من جمالهم، ويعظم كرامتهم، ويجعلهم مُجتهدين في تسبيح خالقهم

\par 6 وجميع العوفانيين ممتلئون عيونًا، وكلهم ممتلئون تألقًا؛ اثنان وسبعون حجر ياقوت أزرق مثبت على ثيابهم عن يمينهم، واثنان وسبعون حجر ياقوت أزرق مثبت على ثيابهم عن يسارهم

\par 7 وأربعة أحجار كريمة مثبتة على تاج كل منها، يمتد روعتها في جهات العربوت الأربع كما يمتد روع كرة الشمس في جميع جهات الكون. ولماذا سميت بالجمشت؟ لأن روعتها تشبه مظهر البرق. وتحيط بها خيام من الجلال، خيام من التألق، خيام من الياقوت والجمشت بسبب لمعان عيونها

\chapter{26}

\par \textit{سرافيل، أمير السيرافيم. وصف السيرافيم}

\par 1 قال ر. إسماعيل: قال لي ميتاترون، الملاك، أمير الحضور: فوق هؤلاء أمير واحد، عجيب، نبيل، عظيم، شريف، جبار، رهيب، رئيس وقائد x وكاتب سريع، ممجد، مكرم، ومحبوب

\par 2 إنه ممتلئ بالروعة، ممتلئ بالثناء والتألق؛ وهو ممتلئ بالتألق والنور والجمال؛ وكله ممتلئ بالصلاح والعظمة

\par 3 وجهه كله كوجه الملائكة، لكن جسده كجسد نسر

\par 4 بهاؤه كالبرق، ومظهره كأضرحة النار، وجماله كالشرر، وجلاله كالجمر، وجلاله كالجمر، وإشراقه كنور كوكب الزهرة. صورته كالنور الأعظم. طوله كالسماوات السبع. نور حاجبيه كالنور السباعي.

\par 5 حجر الياقوت على رأسه عظيم مثل الكون بأكمله ويشبه روعة السماوات في إشعاعها.

\par 6 جسده مليء بعيون كنجوم السماء، لا تُحصى ولا تُحصى. كل عين ككوكب الزهرة. ومع ذلك، هناك منها ما يشبه النور الأصغر، وبعضها يشبه النور الأكبر. من كاحليه إلى ركبتيه كنجوم البرق، ومن ركبتيه إلى فخذيه ككوكب الزهرة، ومن فخذيه إلى حقويه كالقمر، ومن حقويه إلى عنقه كالشمس، ومن عنقه إلى جمجمته كالنور الخالد.

\par 7 التاج على رأسه كبهاء عرش المجد. مقياس التاج هو مسافة رحلة سنوات. لا يوجد نوع من الروعة، ولا نوع من التألق، ولا نوع من الإشراق، ولا نوع من النور في الكون إلا وهو مثبت على ذلك التاج

\par 8 اسم ذلك الأمير هو سيرافيل هـ. والتاج على رأسه اسمه "أمير السلام". ولماذا يُدعى باسم سيرافيل هـ؟ لأنه مُعيّن على السيرافيم. والسيرافيم المتأججون مُكلَّفون تحت رعايته. وهو يرأسهم نهارًا وليلًا ويُعلِّمهم الغناء والتسبيح وإعلان الجمال والقوة والجلال؛ حتى يُعلنوا جمال ملكهم بكل أنواع التسبيح والتقديس (قدوشا).

\par 9 كم عدد السرافيم؟ أربعة، يقابلون رياح العالم الأربع. وكم جناحًا لكل واحد منهم؟ ستة، يقابلون أيام الخلق الستة. وكم وجهًا لكل واحد منهم؟ أربعة وجوه لكل واحد منهم

\par 10 يقابل طول السيرافيم وارتفاع كل واحد منهم ارتفاع السماوات السبع. وحجم كل جناح مثل قياس جميع الرقايا. وحجم كل وجه مثل حجم وجه المشرق

\par 11 وكل واحد منها يُصدر نورًا كبهاء عرش المجد: حتى أن الحيوت المقدسة، والأوفانيم المكرمين، والكروبيم المهيبين لا يستطيعون رؤيته. لأن كل من ينظر إليه، تُظلم عيناه من شدة بهائه

\par 12 لماذا يُطلق عليهم اسم السيرافيم؟ لأنهم يحرقون (ساراف) ألواح كتابة الشيطان: يجلس الشيطان كل يوم، مع صموئيل، أمير روما، ودبئيل، أمير فارس، ويكتبون آثام إسرائيل على ألواح الكتابة التي يسلمونها إلى السيرافيم، ليقدموها أمام القدوس، تبارك هو، حتى يُبيد إسرائيل من العالم. لكن السيرافيم يعرفون من أسرار القدوس، تبارك هو، أنه لا يريد أن يهلك هذا الشعب إسرائيل. ماذا يفعل السيرافيم؟ يستقبلونهم كل يوم من يد الشيطان ويحرقونهم في النار المشتعلة أمام العرش العالي والممجد حتى لا يأتوا أمام القدوس، تبارك هو، في الوقت الذي يجلس فيه على عرش الدينونة، ويحكم العالم كله بالحق



\chapter{27}

\par \textit{رادويريل، حارس كتاب السجلات}

\par 1 قال ر. إسماعيل: قال لي ميتاترون، ملاك هـ، أمير الحضور: فوق السيرافيم أمير واحد، مُتعالٍ على جميع الأمراء، أعجوبة من جميع الخدم. اسمه رادويريل هـ، المُعيَّن على خزائن الكتب.

\par 2 يُحضر علبة الكتابات (و) فيها سفر السجلات، ويُحضرها أمام القدوس، تبارك اسمه. ويكسر أختام العلبة، ويفتحها، ويُخرج الكتب ويُسلمها أمام القدوس، تبارك اسمه. فيأخذها القدوس، تبارك اسمه، من يده ويسلمها أمامه للكتبة، ليقرأوها في بيت الدين العظيم في علو "عربوت رقية"، أمام البيت السماوي

\par 3 ولماذا يُدعى رادويريل؟ لأنه من كل كلمة تخرج من فمه يُخلق ملاك: وهو يقف في ترانيم (في فرقة الغناء) للملائكة الخادمين وينطق بترنيمة أمام القدوس، تبارك اسمه عندما يقترب وقت تلاوة القداسة (ثلاث مرات).

\chapter{28}

\par \textit{عيرين وقديشين}

\par 1 قال ر. إسماعيل: قال لي ميتاترون، الملاك، أمير الحضور: فوق هؤلاء جميعًا، أربعة أمراء عظماء، إيرين وقاديشين بالاسم: عظماء، مُكرّمون، مُبجّلون، محبوبون، رائعون، مُجيدون، أعظم من جميع أبناء السماء. لا مثيل لهم بين جميع الأمراء السماويين، ولا نظير لهم بين جميع الخدم. فكل واحد منهم يُساوي جميع الباقين معًا.

\par 2 ومسكنهم مقابل عرش المجد، ومقامهم مقابل القدوس تبارك اسمه، حتى أن بهاء مسكنهم هو انعكاس بهاء عرش المجد، وبهاء وجوههم هو انعكاس بهاء الشكينة.

\par 3 وهم ممجدون بمجد الجلالة الإلهية (جبورا) وممدوحون بحمد الشكينة.

\par 4 وليس هذا فحسب، بل إن القدوس، تبارك اسمه، لا يفعل شيئًا في عالمه دون استشارتهم أولًا، بل يفعله بعد ذلك. كما هو مكتوب: "الحكم بقضاء العيرين، والأمر بكلمة القديشين"

\par 5 العيرين اثنان والقاديشين اثنان. وكيف يقفان أمام القدوس تبارك وتعالى؟ يجب أن يُفهم أن عيرًا واحدًا يقف على جانب والآخر عيرًا على الجانب الآخر، وقاديشًا واحدًا يقف على جانب والآخر على الجانب الآخر

\par 6 ودائمًا ما يرفعون المتواضعين، ويضعون المتكبرين، ويرفعون المتواضعين

\par 7 وكل يوم، بينما يجلس القدوس، تبارك اسمه، على عرش الدينونة ويحكم على العالم أجمع، وتُفتح أمامه كتب الأحياء وكتب الأموات، يقف جميع أبناء السماء أمامه في خوف ورعب ورعب ورعدة. في ذلك الوقت، يجلس القدوس، تبارك اسمه، على عرش الدينونة ليُجري الحكم، ثوبه أبيض كالثلج، وشعر رأسه كالصوف النقي، وكامل ردائه كالنور الساطع. وهو مُغطى بالبر كله كما لو كان بدرع.

\par 8 ويقف أمامه أولئك العيرين والقاديشين كموظفي محكمة أمام القاضي. ويرفعون ويجادلون في كل قضية ويغلقون القضية التي تأتي أمام القدوس، تبارك اسمه، في الحكم، كما هو مكتوب: "الحكم بقضاء العيرين والطلب بكلمة القاديشين"

\par 9 بعضهم يتجادل والبعض الآخر يُصدر الحكم في بيت الدين الكبير في العربوت. بعضهم يُقدم الطلبات أمام الجلالة الإلهية، والبعض الآخر يُغلق القضايا أمام العلي. وينتهي آخرون بالنزول (والتأكيد =) وتنفيذ الأحكام على الأرض. وكما هو مكتوب: "هوذا عير وقديش نزلا من السماء وصرخا بصوت عالٍ وقالا هكذا: اقطعوا الشجرة واقطعوا أغصانها، وانفضوا أوراقها، وانثروا ثمرها: لتهرب الحيوانات من تحتها، والطيور من أغصانها".

\par 10 لماذا يُطلق عليهما اسم عيرين وقديشين؟ لأنهما يُقدسان الجسد والروح بسياط نار في اليوم الثالث من الدينونة، كما هو مكتوب: "بعد يومين يُحيينا. في اليوم الثالث يُقيمنا فنحيا أمامه".

\chapter{29}

\par \textit{وصف فئة من الملائكة}

\par 1 قال ر. إسماعيل: قال لي ميتاترون، الملاك، أمير الحضور: لكل اسم من أسماء الثيتا سبعون اسمًا تُقابل ألسنة العالم السبعين. وكلها (مبنية) على اسم القدوس، تبارك اسمه. وكل اسم مكتوب بخط ملتهب على التاج المهيب (كثِر نورا) الذي على رأس الملك العظيم المتعال.

\par 2 ومن كل واحدة منها يخرج شرارات وبروق. وكل واحدة منها محاطة بقرون بهاء من حولها. من كل واحدة تشرق الأنوار، وكل واحدة محاطة بخيام من التألق بحيث لا يستطيع حتى السيرافيم والحويوث الذين هم أعظم من كل أبناء السماء أن يروها

\chapter{30}

\par \textit{أمراء الممالك وأمير العالم يؤدون الخدمة في السنهدرين العظيم في السماء}

\par 1 قال ر. إسماعيل: قال لي ميتاترون الملاك أمير الحضور: كلما جلس بيت دين العظيم في 'عربوث راقة في الأعلى، لا يكون هناك فتح فم لأحد في العالم إلا لأولئك الأمراء العظماء الذين يُدعون هـ' باسم القدوس المبارك.

\par 2 كم عدد هؤلاء الأمراء؟ اثنان وسبعون أميرًا من ممالك العالم، بالإضافة إلى أمير العالم الذي يتحدث (يدافع) لصالح العالم أمام القدوس، تبارك اسمه، كل يوم، في الساعة التي يُفتح فيها الكتاب الذي تُسجل فيه جميع أعمال العالم، كما هو مكتوب: "وُضِعَ الدينونة وفُتِحَت الأسفار".

\chapter{31}

\par \textit{(صفات) العدل والرحمة والحق عند عرش الدين}

\par 1 قال ر. إسماعيل: قال لي ميتاترون، الملاك، أمير الحضور: عندما يجلس القدوس، تبارك اسمه، على عرش الدينونة، (حينئذٍ) تقف العدالة عن يمينه والرحمة عن يساره والحق أمام وجهه

\par 2 وعندما يدخل الإنسان أمامه للدينونة، يخرج من بهاء الرحمة نحوه كالعصا ويقف أمامه. فيسقط الإنسان على وجهه، ويرتجف جميع ملائكة الهلاك أمامه، كما هو مكتوب: "وبالرحمة يُثبت العرش، ويجلس عليه بالحق".

\chapter{32}

\par \textit{تنفيذ الحكم على الأشرار. سيف الله}

\par 1 قال ر. إسماعيل: قال لي ميتاترون الملاك أمير الحضور: عندما يفتح القدوس المبارك الكتاب الذي نصفه نار ونصفه لهب، يخرجون من أمامه في كل لحظة لتنفيذ الحكم على الأشرار بسيفه (أي) المسلول من غمده والذي يتألق بريقه مثل البرق وينتشر في العالم من طرف إلى آخر، كما هو مكتوب: "لأنه بالنار سيحاكم الرب (وبسيفه مع كل بشر)".

\par 2 "وكل سكان العالم (حرفيًا، أولئك الذين يأتون إلى العالم) يخافون ويرتعدون أمامه، عندما ينظرون إلى سيفه الحاد مثل البرق من أقاصي العالم إلى أقاصيه، والشرر والومضات بحجم نجوم راقدت تخرج منه؛ كما هو مكتوب: "إذا شحذت برق سيفي".


\chapter{33}

\par \textit{ملائكة الرحمة والسلام والدمار عند عرش الدينونة. الكتبة (الآية 1)، والملائكة عند عرش المجد والأنهار النارية التي تحته. (الآية 5)}

\par 1 قال ر. إسماعيل: قال لي ميتاترون الملاك أمير الحضور: في الوقت الذي يجلس فيه القدوس تبارك وتعالى على عرش الدينونة، فإن ملائكة الرحمة تقف عن يمينه، وملائكة السلام تقف عن يساره، وملائكة الهلاك تقف أمامه.

\par 2 ويكون كاتب واحد واقفًا تحته، وكاتب آخر فوقه.

\par 3 والسيرافيم المجيدين [أ: يحيطون بهم كجمر حول عرش المجد.] [هـ: يحيطون بالعرش من جوانبه الأربعة بجدران من البرق، ويحيط بهم العوفانيم بجمر حول عرش المجد.] وسحب من النار وسحب من اللهب تحيط بهم من اليمين واليسار؛ والحيوت المقدس يحمل عرش المجد من الأسفل: كل واحد بثلاثة أصابع. ومقياس أصابع كل واحد هو 800000 ومئة مرة، (و) 66000 فرسخ.

\par 4 وتحت أقدام الحيوت سبعة أنهار نارية تجري وتتدفق. وعرض كل نهر ألف فرسخ، وعمقه آلاف ربوات من الفراسخ. وطوله لا يُستقصى ولا يُقاس

\par 5 ويدور كل نهر في منحنى في الجهات الأربع لعربوت رقية، ومن هناك ينحدر إلى موتون ويستقر، ومن معون إلى زبول، ومن زبول إلى شكيم، ومن شكيم إلى رقية، ومن رقية إلى شمايم، ومن شمايم على رؤوس الأشرار الذين في جهنم، كما هو مكتوب: "هوذا زوبعة الرب، غضبه، قد مضت، نعم، عاصفة عاتية؛ تنفجر على رؤوس الأشرار".

\chapter{34}

\par \textit{الدوائر المختلفة متحدة المركز حول الحيوت، والتي تتكون من النار والماء وحجارة البرد وما إلى ذلك، والملائكة الذين ينطقون بـ Qedushsha responsorium}

\par 1 قال ر. إسماعيل: قال لي ميتاترون، الملاك، أمير الحضور: حوافر الحوت محاطة بسبعة سحب من الجمر المشتعل. سحب الجمر المشتعلة محاطة من الخارج بسبعة جدران من اللهب. جدران اللهب السبعة محاطة من الخارج بسبعة جدران من أحجار البَرَد (أحجار إيل-غابيشي). أحجار البَرَد محاطة من الخارج بأحجار البَرَد (حجر باراد). أحجار البَرَد محاطة من الخارج بأحجار "أجنحة العاصفة". أحجار "أجنحة العاصفة" محاطة من الخارج بلهيب النار. لهيب النار محاط بغرف الزوبعة. غرف الزوبعة محاطة من الخارج بالنار والماء

\par 2 حول النار والماء أولئك الذين ينطقون بكلمة "قدوس". حول أولئك الذين ينطقون بكلمة "قدوس" أولئك الذين ينطقون بكلمة "مبارك". حول أولئك الذين ينطقون بكلمة "مبارك" السحب المضيئة. السحب المضيئة محاطة من الخارج بجمر من شجر العرعر المشتعل؛ وعلى الجانب الخارجي المحيط بجمر العرعر المشتعل، هناك آلاف معسكرات النار وعشرات الآلاف من جحافل اللهب. وبين كل عدة معسكرات وكل عدة جحافل هناك سحابة، حتى لا يحترقوا بالنار

\chapter{35}

\par \textit{معسكرات الملائكة في "عربوث رقية": ملائكة يؤدون القدوشة}

\par 1 قال ر. إسماعيل: قال لي ميتاترون، الملاك، أمير الحضرة: إن للقدوس تبارك وتعالى في أعالي رقية العرب، خمسمائة وستة آلاف جيش، وفي كل جيش أربعمائة وستة وتسعون ألف ملك.

\par 2 وكل ملاك واحد، طول قامته كالبحر العظيم، ومنظر وجوههم كمنظر البرق، وعيونهم كمصابيح نار، وأذرعهم وأرجلهم كلون النحاس المصقول، وصوت كلامهم كصوت جمع

\par 3 وهم جميعًا واقفون أمام عرش المجد في أربعة صفوف. ويقف أمراء الجيش على رأس كل صف

\par 4 وبعضهم ينطق بـ "القدوس" والبعض الآخر ينطق بـ "المبارك"، وبعضهم يركض كرسل، والبعض الآخر يقف في الحضور، كما هو مكتوب: "خدمه آلاف الآلاف، ووقف أمامه ربوات الآلاف. فُتح الدينونة وفُتحت الأسفار".

\par 5 وفي الساعة التي يقترب فيها وقت قول "القدوس"، (حينئذٍ) تخرج أولًا زوبعة من أمام القدوس، تبارك اسمه، وتقتحم معسكر الشكينة، ويحدث اضطراب كبير بينهم، كما هو مكتوب: "هوذا زوبعة الرب تخرج بغضب، اضطراب مستمر".

\par 6 في تلك اللحظة، يتحول آلاف الآلاف منهم إلى شرارات، آلاف الآلاف منهم إلى جمر، آلاف الآلاف إلى ومضات، آلاف الآلاف إلى لهب، آلاف الآلاف إلى ذكور، آلاف الآلاف إلى إناث، آلاف الآلاف إلى رياح، آلاف الآلاف إلى نيران مشتعلة، آلاف الآلاف إلى لهب، آلاف الآلاف إلى شرارات، آلاف الآلاف إلى أعمدة من النور؛ حتى يحملوا على عاتقهم نير ملكوت السماوات، العالي والمرتفع، نير خالقهم جميعًا بخوف ورعب ورهبة ورعدة، باضطراب وقلق ورعب وخوف. ثم يتحولون مرة أخرى إلى شكلهم السابق ليكون خوف ملكهم أمامهم دائمًا، لأنهم وضعوا قلوبهم على ترديد النشيد باستمرار، كما هو مكتوب: "ونادى هذا ذاك وقال (قدوس، قدوس، قدوس، إلخ)".


\chapter{36}

\par \textit{الملائكة تستحم في النهر الناري قبل تلاوة "الترنيمة"}

\par 1 قال ر. إسماعيل: قال لي ميتاترون، الملاك، أمير الحضور: في الوقت الذي يرغب فيه الملائكة الخادمون في قول الأغنية، يرتفع نهر النور (التيار الناري) مع "آلاف الآلاف وآلاف الآلاف" (من الملائكة) من القوة وقوة النار، ويجري ويمر تحت عرش المجد، بين معسكرات الملائكة الخادمين وجنود العربوث

\par 2 "وينزل جميع الملائكة الخادمين أولاً إلى نهار دي نور، ويغمسون أنفسهم في النار ويغمسون ألسنتهم وأفواههم سبع مرات؛ وبعد ذلك يصعدون ويرتدون ثوب "مشق سامال" ويغطون أنفسهم بأردية من حشمال ويقفون في أربعة صفوف مقابل عرش المجد، في كل السماوات.


\chapter{37}

\par \textit{معسكرات الشاكينا الأربعة ومحيطها}

\par 1 قال ر. إسماعيل: قال لي ميتاترون، الملاك، أمير الحضور: في القاعات السبع، تقف أربع عربات شيكينا، وأمام كل عربة تقف معسكرات شيكينا الأربعة. وبين كل معسكر وآخر، يتدفق نهر من النار باستمرار.

\par 2 بين كل نهر غيوم لامعة [تحيط بها]، وبين كل سحابة أعمدة من الكبريت. وبين عمود وآخر عجلات ملتهبة قائمة، تحيط بها. وبين عجلة وأخرى لهيب نار حولها. وبين لهيب وآخر مخازن بروق؛ وخلف مخازن البرق أجنحة ريح العاصفة. وخلف أجنحة ريح العاصفة حجرات العاصفة؛ وخلف حجرات العاصفة رياح وأصوات ورعود وشرر فوق شرر وزلازل فوق زلازل


\chapter{38}

\par \textit{الخوف الذي يصيب السماوات كلها عند سماع صوت "القدوس"، وخاصة الأجرام السماوية. هذه يهدئها أمير العالم}

\par 1 قال ر. إسماعيل: قال لي ميتاترون الملاك أمير الحضور: في ذلك الوقت، عندما ينطق الملائكة الخادمون بالقداسة (ثلاث مرات)، ترتجف جميع أعمدة السماوات وقواعدها، وتهتز أبواب قاعات عربوث راقة وتتحرك أسس شكيم والكون (تيبل)، وترتجف أوامر ماعون وغرف ماكون، وتضطرب جميع أوامر راقة والأبراج والكواكب، وتسرع كرات الشمس والقمر بعيدًا وتفر من مساراتها وتجري 12000 فرسخ وتحاول إلقاء نفسها من السماء،

\par 2 "بسبب صوت هدير ترانيمهم، وضجيج تسبيحهم، والشرر والبروق التي تخرج من وجوههم، كما هو مكتوب: ""كان صوت رعدك في السماء (أضاءت البروق العالم، وارتجفت الأرض وارتجفت)""."

\par 3 حتى يناديهم أمير العالم قائلًا: "اهدأوا في أماكنكم! لا تخافوا من الملائكة الخادمين الذين يغنون ترنيمة أمام القدوس، تبارك هو". كما هو مكتوب: "عندما ترنمت نجوم الصبح معًا، وهتف جميع أبناء السماء فرحًا".



\chapter{39}

\par \textit{تنطلق الأسماء الصريحة من العرش، وتسجد جميع الجيوش الملائكية المختلفة أمامه في وقت القدوششا}

\par 1 قال ر. إسماعيل: قال لي ميتاترون، الملاك، أمير الحضور: عندما ينطق الملائكة الخادمون بكلمة "قدوس"، فإن جميع الأسماء الصريحة المنقوشة بأسلوب ملتهب على عرش المجد تطير كالنسور، بستة عشر جناحًا. وتحيط بالقدوس، تبارك اسمه، من الجوانب الأربعة لمكان شكينته.

\par 2 وملائكة الجيش، والخدام المشتعلون، والأوفانم الأقوياء، وكروبيم الشكينة، والحيّوت المقدسة، والسيرافيم، والأريليم، والتافساريم، وجنود النار الآكلة، والجيوش النارية، والجيوش المشتعلة، والأمراء المقدسون، المزينون بالتيجان، والمتشحون بالجلال الملكي، والمغطون بالمجد، والمتمنطقون بالعلو، يخرّون على وجوههم ثلاث مرات قائلين: "ليكن اسم مملكته المجيدة مباركًا إلى الأبد".


\chapter{40}

\par \textit{يُكافأ الملائكة الخادمون بالتيجان عند نطقهم بالقداس الثاني بترتيبه الصحيح، ويُعاقبون بالنار الآكلة إن لم يفعلوا. ويُخلق ملائكة جدد بدلًا من الملائكة الآكلة}

\par 1 قال ر. إسماعيل: قال لي ميتاترون، الملاك، أمير الحضور: عندما يقول الملائكة الخادمون "قدوس" أمام القدوس، تبارك اسمه، بالطريقة الصحيحة، فإن خدام عرشه، خدام مجده، يخرجون بفرح عظيم من تحت عرش المجد

\par 2 ويحملون جميعًا في أيديهم، كل واحد منهم ألف ألف وعشرة آلاف مرة عشرة آلاف تاج من النجوم، تشبه في مظهرها كوكب الزهرة، ويضعونها على الملائكة الخادمين والأمراء العظام الذين ينطقون بكلمة "قدوس". يضعون على كل واحد منهم ثلاثة تيجان: تاج لأنهم يقولون "قدوس"، وتاج آخر لأنهم يقولون "قدوس، قدوس"، وتاج ثالث لأنهم يقولون "قدوس، قدوس، قدوس، رب الجنود".

\par 3 وفي اللحظة التي لا ينطقون فيها بكلمة "قدوس" بالترتيب الصحيح، تخرج نار آكلة من خنصر القدوس، تبارك اسمه، وتسقط في وسط صفوفهم وتنقسم إلى ألف جزء يتوافق مع المعسكرات الأربعة للملائكة الخادمين، وتستهلكهم في لحظة واحدة، كما هو مكتوب: "نار تسير أمامه وتحرق أعداءه من حوله".

\par 4 بعد ذلك، يفتح القدوس، تبارك اسمه، فمه وينطق بكلمة واحدة ويخلق أخرى بدلًا منها، جديدة مثلها. ويقف كل واحد أمام عرش مجده، ينطق بكلمة "قدوس"، كما هو مكتوب (مراثي إرميا 3: 3): "هي جديدة كل صباح؛ عظيمة هي أمانتك".


\chapter{41}

\par \textit{يُظهر ميتاترون لـ ر. إسماعيل الحروف المنقوشة على عرش المجد والتي بها خُلِقَ كل شيء في السماء والأرض}

\par 1 قال ر. إسماعيل: قال لي ميتاترون الملاك أمير الحضور: تعال وانظر إلى الحروف التي خلقت بها السماء والأرض،
\par الحروف التي خلقت بها الجبال والتلال،
\par الحروف التي خلقت بها البحار والأنهار،
\par الحروف التي خلقت بها الأشجار والأعشاب،
\par الحروف التي خلقت بها الكواكب والأبراج،
\par الحروف التي خلقت بها كرة القمر وكرة الشمس والجبار والثريا وجميع النجوم المختلفة في الراقية.

\par 2 الحروف التي خُلِقَ بها عرش المجد وعجلات المركبة، الحروف التي خُلِقَت بها ضروريات العوالم،

\par 3 الحروف التي خُلقت بها الحكمة والفهم والمعرفة والفطنة والوداعة والصلاح التي بها يُبنى العالم كله

\par 4 وسرت بجانبه، فأخذني بيدي ورفعني على جناحيه، وأراني تلك الحروف، كلها، منقوشة بقلم ملتهب على عرش المجد، ويخرج منها شرار ويغطي جميع حجرات العربوت


\chapter{42}

\par \textit{أمثلة على الأضداد القطبية التي تحافظ عليها متوازنة من خلال العديد من الأسماء الإلهية وعجائب أخرى مماثلة}

\par 1 قال ر. إسماعيل: قال لي ميتاترون، الملاك، أمير الحضور: تعالَ وسأريك، أين المياه معلقة في الأعالي، أين النار مشتعلة وسط البَرَد، أين تشرق البرق من وسط الجبال الثلجية، أين تزأر الرعود في المرتفعات السماوية، أين يشتعل اللهب وسط النار المشتعلة، أين تُسمع الأصوات وسط الرعد والزلزال

\par 2 ثم ذهبتُ بجانبه، فأخذني بيده ورفعني على جناحيه وأراني كل تلك الأشياء. رأيتُ المياهَ المعلقةَ في أعالي عربوث رقيةً باسم يهوه يي آشر يهوه يي (ياه، أنا هو)، وثمارها تنزل من السماء وتسقي وجه العالم، كما هو مكتوب: «(يسقي الجبال من حجائره): تشبع الأرض من ثمرة عملك».

\par 3 "ورأيت النار والثلج والبرد مختلطين ببعضهما البعض ومع ذلك لم يتضرروا، بقوة اسم "إش أوكيلا" (النار الآكلة)، كما هو مكتوب: "لأن الرب إلهك نار آكلة".

\par 4 "ورأيت بروقاً تشرق من جبال الثلج ولم تنطفئ بقوة اسم يهوه الصخرة الأبدية كما هو مكتوب لأنه في يهوه الصخرة الأبدية."

\par 5 ورأيت رعودًا وأصواتًا تزأر وسط لهيب ناري، ولم تُكسر (تُسكت) بقوة اسم إيل شداي رابا (الإله العظيم القدير) كما هو مكتوب: "أنا الله القدير".

\par 6 ورأيتُ لهبًا وهجًا (ألسنة لهب متوهجة) كانا يشتعلان ويتوهجان في وسط نار مشتعلة، ومع ذلك لم يتضررا (يُلتهما)، بقوة اسم ياد القيسياه (اليد على عرش الرب) كما هو مكتوب: "وقال: لأن اليد على عرش الرب".

\par 7 ورأيتُ أنهارًا من نارٍ في وسط أنهار ماء، ولم تُخمد (تُطفأ) بقوة اسم "صانع السلام"، كما هو مكتوب: "يصنع السلام في أعاليه". فهو يصنع السلام بين النار والماء، وبين البَرَد والنار، وبين الريح والسحاب، وبين الزلزال والشرر.

\chapter{43}

\par \textit{يُظهر ميتاترون لـ ر. إسماعيل مسكن الأرواح التي لم تولد بعد وأرواح الموتى الصالحين}

\par 1 قال ر. إسماعيل: قال لي ميتاترون: تعالَ وسأريكَ أين توجد أرواح الصالحين الذين خُلقوا وعادوا، وأرواح الصالحين الذين لم يُخلقوا بعد

\par 2 ورفعني إلى جانبه، وأخذني بيده، ورفعني بالقرب من عرش المجد عند موضع الشكينة؛ وكشف لي عن عرش المجد، وأراني الأرواح التي خُلقت وعادت: وكانت تحلق فوق عرش المجد أمام القدوس، تبارك اسمه

\par 3 بعد ذلك، ذهبتُ لتفسير الآية التالية من الكتاب المقدس، ووجدتُ فيما هو مكتوب: "لأن الروح لبس أمامي، والنفوس التي خلقتها" أن ("لأن الروح لبس أمامي") تعني الأرواح التي خُلقت في حجرة خلق الصالحين، والتي عادت أمام القدوس، تبارك اسمه؛ (والكلمات:) والنفوس التي خلقتها تشير إلى أرواح الصالحين التي لم تُخلق بعد في الحجرة (GUPH).


\chapter{44}

\par \textit{يُظهر ميتاترون للحاخام إسماعيل مسكن الأشرار والوسيط في الهاوية (الآيات 6-10)} يصلي الآباء من أجل خلاص إسرائيل (الآيات 7-10)}

\par 1 قال ر. إسماعيل: قال لي ميتاترون الملاك أمير الحضور: تعال فأريك أرواح الأشرار وأرواح الوسطاء أين هم واقفون، وأرواح الوسطاء إلى أين ينزلون، وأرواح الأشرار إلى أين ينزلون.

\par 2 فقال لي: إن أرواح الأشرار تنزل إلى الهاوية على يدي ملاكين من ملائكة الهلاك اسمهما زعفيل وسمكيئيل.

\par 3 عُيّن سماكيل على الوسطاء ليُعينهم ويُطهّرهم بفضل رحمة رئيس المكان العظيمة. وعُيّن زعفيل على أرواح الأشرار ليُنزلهم من حضرة القدوس، تبارك اسمه، ومن بهاء الشكينة إلى الهاوية، ليُعاقَبوا في نار جهنم بعصيّ جمر مُشتعلة.

\par 4 وذهبت بجانبه، وأخذني بيدي وأراني إياهم جميعًا بأصابعه

\par 5 ورأيتُ منظر وجوههم (وإذا هي) كمنظر بني البشر، وأجسامهم كالنسور. بل كان لون وجه الوسيط كالشيب الشاحب من جراء أعمالهم، إذ عليهم لطخات حتى يُطهروا من إثمهم في النار.

\par 6 وكان لون الأشرار كقاع القدر من شرور أعمالهم

\par 7 ورأيت أرواح الآباء إبراهيم وإسحاق ويعقوب وسائر الصالحين الذين أخرجوهم من قبورهم وصعدوا إلى السماء (رقية). وكانوا يصلون أمام القدوس، تبارك اسمه، قائلين في صلاتهم: "يا رب الكون! إلى متى ستجلس على عرشك كالحزين في أيام حزنه ويمينك خلفك ولا تخلص أولادك وتكشف ملكوتك في العالم؟ وإلى متى لن ترحم أولادك الذين أصبحوا عبيدًا بين أمم العالم؟ ولا على يمينك التي خلفك التي بسطت بها السماوات والأرض وسماء السماوات؟ متى ترحم؟"

\par 8 ثم أجاب القدوس، تبارك اسمه، كل واحد منهم قائلاً: "بما أن هؤلاء الأشرار يخطئون كذا وكذا، ويرتكبون كذا وكذا من المعاصي ضدي، فكيف يمكنني أن أنقذ يدي اليمنى العظيمة من السقوط بأيديهم (التي تسببت فيها).

\par 9 في تلك اللحظة، ناداني ميتاترون وقال لي: "يا خادمي! خذ الكتب، واقرأ أفعالهم الشريرة!" على الفور، أخذت الكتب وقرأت أفعالهم، فوجدت 36 مخالفة (مدونةً) فيما يتعلق بكل شرير، بالإضافة إلى أنهم انتهكوا جميع حروف التوراة، كما هو مكتوب: "نعم، لقد انتهك جميع إسرائيل شريعتك". لم يُكتب "al torateka" بل "et torateka"، لأنهم انتهكوا من الألف إلى التاء، 40 قانونًا انتهكوا لكل حرف

\par 10 فبكى إبراهيم وإسحاق ويعقوب في الحال. ثم قال لهم القدوس تبارك: "يا إبراهيم حبيبي، يا إسحاق مختاري، يا يعقوب بكر! كيف يمكنني الآن أن أخلصهم من بين أمم العالم؟" وفي الحال صرخ ميكائيل، أمير إسرائيل، وبكى بصوت عظيم وقال: "لماذا تقف بعيدًا يا رب؟"


\chapter{45}

\par \textit{يُظهر ميتاترون لـ ر. إسماعيل أحداث الماضي والمستقبل المسجلة على ستار العرش}

\par 1 قال ر. إسماعيل: قال لي ميتاترون: تعالَ، وسأريكَ حجاب المقام (الجلالة الإلهية) المَبسوط أمام القدوس، تبارك وتعالى، والذي نُقش عليه جميع أجيال العالم وجميع أعمالهم، ما فعلوه وما سيفعلونه حتى نهاية جميع الأجيال

\par 2 فذهبتُ، فأراه لي وأشار إليه بأصابعه كأب يُعلّم أولاده حروف التوراة. ورأيتُ كل جيل، حكام كل جيل،
\par ورؤساء كل جيل،
\par رعاة كل جيل،
\par الظالمين (السائقين) لكل جيل،
\par حراس كل جيل،
\par سواط كل جيل،
\par المشرفين على كل جيل،
\par قضاة كل جيل،
\par ضباط المحكمة من كل جيل،
\par المعلمين من كل جيل،
\par من المؤيدين لكل جيل،
\par زعماء كل جيل،
\par رؤساء الأكاديميات من كل جيل،
\par قضاة كل جيل،
\par أمراء كل جيل،
\par مستشاري كل جيل،
\par النبلاء من كل جيل،
\par ورجال القوة في كل جيل،
\par شيوخ كل جيل،
\par وأدلة كل جيل.

\par 3 ورأيت آدم وجيله، أعمالهم وأفكارهم،
\par نوح وجيله، أعمالهم وأفكارهم،
\par وتوليد الطوفان، وأفعالهم وأفكارهم،
\بار سام وجيله، أعمالهم وأفكارهم،
\بار نمرود ونشأة بلبلة الألسنة وملكه
\par الجيل، أفعالهم وأفكارهم،
\par إبراهيم وجيله، أعمالهم وأفكارهم،
\par إسحاق وجيله، أعمالهم وأفكارهم،
\par إسماعيل وجيله، أعمالهم وأفكارهم،
\par يعقوب وجيله، أعمالهم وأفكارهم،
\بار يوسف وجيله، أعمالهم وأفكارهم،
\par القبائل وأجيالهم، أعمالهم وأفكارهم،
\par عمرام وجيله، أعمالهم وأفكارهم،
\par موسى وجيله، أعمالهم وأفكارهم،

\par 4 هارون ومريم أعمالهما وتصرفاتهما،
\par الرؤساء والشيوخ أعمالهم وتصرفاتهم،
\par يشوع وجيله، أعمالهم وتصرفاتهم،
\par القضاة وجيلهم، أعمالهم وتصرفاتهم،
\par إيلي وجيله، أعمالهم وأفعالهم،
\par فينحاس، أعمالهم وأفعالهم (؟)
\par إلكانة وجيله، أعمالهم وأفعالهم،
\par صموئيل وجيله، أعمالهم وتصرفاتهم،
\par ملوك يهوذا مع أجيالهم وأعمالهم وتصرفاتهم،
\par ملوك إسرائيل وأجيالهم وأعمالهم وتصرفاتهم،
\لأمراء إسرائيل وأعمالهم وتصرفاتهم؛ أمراء أمم العالم وأعمالهم وتصرفاتهم،
\عن رؤساء مجالس إسرائيل وأعمالهم وتصرفاتهم؛
\لرؤساء (المجالس) في دول العالم، وأجيالهم، وأعمالهم، وتصرفاتهم؛
\عن رؤساء إسرائيل وأجيالهم وأعمالهم وتصرفاتهم؛
\par نبلاء إسرائيل وجيلهم، أعمالهم وأفعالهم؛ نبلاء أمم العالم وجيلهم (أجيالهم)، أعمالهم وأفعالهم؛
\par رجال السمعة في إسرائيل، أجيالهم، أعمالهم وتصرفاتهم؛
\عن قضاة إسرائيل، وأجيالهم، وأعمالهم، وتصرفاتهم؛
\من قضاة أمم العالم وأجيالهم وأعمالهم وتصرفاتهم؛
\par معلمي الأطفال في إسرائيل، أجيالهم، أعمالهم وتصرفاتهم؛ معلمي الأطفال في أمم العالم، أجيالهم، أعمالهم وتصرفاتهم؛
\من مستشاري (مترجمي) إسرائيل، وجيلهم، وأعمالهم، وأفعالهم؛ مستشاري (مترجمي) أمم العالم، وجيلهم، وأعمالهم، وأفعالهم؛
\لكل أنبياء إسرائيل وأجيالهم وأعمالهم وتصرفاتهم؛ كل أنبياء أمم العالم وأجيالهم وأعمالهم وتصرفاتهم؛

\par 5 وجميع المعارك والحروب التي شنتها دول العالم ضد شعب إسرائيل في زمن مملكتهم

ورأيتُ المسيحَ ابنَ يوسفَ وجيلَه، وأعمالَهم وأفعالَهم التي سيفعلونها ضدَّ أممِ العالم. ورأيتُ المسيحَ ابنَ داودَ وجيلَه، وكلَّ المعاركِ والحروبِ، وأعمالَهم وأفعالَهم التي سيفعلونها بإسرائيلَ، خيرًا كان أم شرًّا. ورأيتُ كلَّ المعاركِ والحروبِ التي سيُقاتلُها جوجُ وماجوجُ في أيامِ المسيح، وكلَّ ما سيفعلُه القدوسُ تباركَ اللهُ بهم في الزمانِ الآتي.

\par 6 وجميع بقية قادة الأجيال وجميع أعمال الأجيال في إسرائيل وفي أمم العالم، سواء ما تم وما سيتم فعله فيما بعد لجميع الأجيال حتى نهاية الزمان، (جميعها) نُقشت على حجاب المقام. ورأيت كل هذه الأشياء بعيني؛ وبعد أن رأيتها، فتحت فمي في مدح المقام (الجلالة الإلهية) (قائلًا هكذا): "لأن كلمة الملك لها سلطان (ومن يقول له: ماذا تفعل؟) من يحفظ الوصايا لن يعرف شيئًا شريرًا". وقلت: "يا رب، ما أعظم أعمالك!"



\chapter{46}

\par \textit{مكان النجوم كما أُظهر لـ ر. إسماعيل}

\par 1 قال ر. إسماعيل: قال لي متاترون: (تعال أريك) فضاء النجوم التي تقوم في الرقعة ليلاً خوفاً من المقام (أريك) أين تذهب وأين تقف.

\par 2 مشيت بجانبه، فأمسك بي من يدي وأشار إليّ جميعها بأصابعه. وكانوا واقفين على شرارات من لهب حول مركبة القدير (مقام). ماذا فعل ميتاترون؟ في تلك اللحظة صفق بيديه وطردهم من مكانهم. طاروا على الفور بأجنحة ملتهبة، ونهضوا وفرّوا من الجوانب الأربعة لعرش المركبة، و(أثناء طيرانهم) أخبرني بأسماء كل واحد منهم. كما هو مكتوب: "يُحصي عدد النجوم؛ يُعطيها جميعها أسمائها"، مُعلّمًا أن القدوس، تبارك اسمه، قد أعطى اسمًا لكل واحد منهم.

\par 3 ويدخلون جميعًا بترتيب مُحصَى بتوجيه (حرفيًا من خلال، على يدي) راحاتيئيل إلى رقيا ها-ششامايم لخدمة العالم. ويخرجون بترتيب مُحصَى لتمجيد القدوس، تبارك اسمه، بالأغاني والترانيم، كما هو مكتوب: "السماوات تُعلن مجد الله".

\par 4 ولكن في الزمان القادم، سيخلقهم القدوس، تبارك اسمه، من جديد 9، كما هو مكتوب: "هم جدد في كل صباح". ويفتحون أفواههم وينطقون بترنيمة. ما هي الترنيمة التي ينطقون بها؟ "عندما أفكر في سماواتك".

\chapter{47}

\par \textit{ميتاترون يُظهر لـ ر. إسماعيل أرواح الملائكة المعاقبين}

\par 1 قال ر. إسماعيل: قال لي ميتاترون: تعالَ وسأريك أرواح الملائكة وأرواح الخدم الذين احترقت أجسادهم في نار المقام (القدير) التي تخرج من إصبعه الصغير. وقد تحولوا إلى جمر ناري في وسط النهر الناري (نهار دي نور). لكن أرواحهم ونفوسهم تقف خلف الشكينة

\par 2 كلما نطق الملائكة الخادمون بأغنية في وقت غير مناسب أو غير مخصص للغناء، فإنهم يحترقون ويستهلكون بنار خالقهم ولهيب من صانعهم، [أ: في أماكن (غرف) الزوبعة، لأنها تهب عليهم وتدفعهم] [هـ: في مكانهم (= في الحال)؛ وتهب عليهم زوبعة وتلقي بهم] في نهر النور؛ وهناك يتحولون إلى جبال عديدة من الفحم المشتعل. لكن روحهم ونفسهم تعود إلى خالقهم، وجميعهم يقفون خلف سيدهم

\par 3 وذهبت بجانبه فأخذني بيدي وأراني جميع أرواح الملائكة وأرواح الخدم الذين كانوا واقفين خلف الشكينة على أجنحة الزوبعة وجدران النار المحيطة بهم.

\par 4 في تلك اللحظة، فتح لي ميتاترون أبواب الأسوار التي كانوا يقفون خلف الشكينة، فرفعتُ عينيّ ورأيتهم، فإذا بشبه كل واحد منهم يشبه الملائكة، وأجنحتهم كأجنحة الطيور، مصنوعة من لهيب، من صنع نار مشتعلة. في تلك اللحظة، فتحتُ فمي مُشيدًا بمقام وقلت: "ما أعظم أعمالك يا رب".

\chapter{48أ}

\par \textit{يُظهِر ميتاترون لـ ر. إسماعيل اليد اليمنى للعلي، غير النشطة الآن خلفه، ولكن في المستقبل مقدر لها أن تعمل على تحرير إسرائيل}

\par 1 قال ر. إسماعيل: قال لي ميتاترون: تعال، وسأريك اليد اليمنى للمقام، الموضوعة خلفه بسبب دمار الهيكل المقدس؛ الذي منه تشرق كل أنواع الروعة والأنوار التي خُلقت بها السماوات؛ والذي لا يُسمح حتى للسيرافيم والعوفانيم برؤيته، حتى يأتي يوم الخلاص

\par 2 وذهبتُ بجانبه، فأخذني بيدي وأراني (اليد اليمنى للمقام) بكل أنواع التسبيح والابتهاج والغناء: ولا يستطيع فم أن ينطق بحمده، ولا تستطيع عين أن تراه، لعظمته وكرامته وجلاله ومجده وجماله

\par 3 وليس ذلك فحسب، بل جميع نفوس الأبرار الذين يُحسبون أهلاً لأن يروا فرح أورشليم، وهم واقفون عندها، يسبحون ويصلون أمامها ثلاث مرات كل يوم، قائلين: «استيقظي، استيقظي، البسي قوة، يا ذراع الرب» كما هو مكتوب: «سيّر ذراع مجده عن يمين موسى».

\par 4 في تلك اللحظة، كانت يد ماقوم اليمنى تبكي. وخرجت من أصابعها الخمسة خمسة أنهار من الدموع وسقطت في البحر العظيم وهزت العالم أجمع، كما هو مكتوب: "لقد تحطمت الأرض تمامًا، وذابت الأرض تمامًا، وتحركت الأرض بشدة، وستترنح الأرض كرجل ثمل، وستتحرك ذهابًا وإيابًا مثل كوخ"، خمس مرات تتوافق مع أصابع يده اليمنى العظيمة

\par 5 ولكن عندما يرى القدوس، تبارك اسمه، أنه لا يوجد رجل صالح في الجيل، ولا رجل تقيّ (حسيدي) على الأرض، ولا عدل في أيدي البشر؛ وأنه لا يوجد رجل مثل موسى، ولا شفيع مثل صموئيل الذي يستطيع الصلاة أمام المقام من أجل الخلاص والنجاة، ومن أجل ملكوته، ليُعلن في العالم أجمع؛ ومن أجل يمينه العظيمة التي وضعها أمام نفسه مرة أخرى ليعمل بها خلاصًا عظيمًا لإسرائيل،

\par 6 حينئذٍ سيتذكر القدوس، تبارك اسمه، عدله ورضاه ورحمته ونعمته، وسينقذ ذراعه العظيمة بنفسه، وسيؤيده بره. كما هو مكتوب: "فرأى أنه ليس إنسان" - (أي:) مثل موسى الذي صلى مرات لا تحصى من أجل إسرائيل في البرية وصرف عنهم المراسيم (الإلهية) - "وتعجب أنه ليس هناك شفيع" - مثل صموئيل الذي توسل إلى القدوس، تبارك اسمه، ودعا إليه، فاستجاب له وحقق رغبته، حتى لو لم تكن مناسبة (وفقًا للخطة الإلهية)، كما هو مكتوب: "أليس هو حصاد قمح اليوم؟ إلى الرب أدعو".

\par 7 وليس هذا فقط، بل كان شريكاً لموسى في كل مكان، كما هو مكتوب: «موسى وهارون بين كهنته». وأيضاً مكتوب: «وإن وقف موسى وصموئيل أمامي»، «خلصتني ذراعي».

\par 8 قال القدوس، تبارك اسمه، في تلك الساعة: "إلى متى أنتظر أبناء من عنده ليصنعوا الخلاص حسب برهم لذراعي؟ من أجل نفسي ومن أجل استحقاقي وبرّي سأنقذ ذراعي وبها أفدي أبنائي من بين أمم العالم." كما هو مكتوب: "من أجل نفسي سأفعل ذلك. لأنه كيف يُدنس اسمي؟"

\par 9 في تلك اللحظة، سيكشف القدوس، تبارك اسمه، عن ذراعه العظيمة ويُريها لأمم العالم: لأن طولها كطول العالم وعرضها كعرض العالم. ومظهر بهائها يشبه بهاء أشعة الشمس في قوتها، في الانقلاب الصيفي

\par 10 في الحال سيُخلَّص بنو إسرائيل من بين أمم العالم. وسيظهر لهم المسيح، وسيصعدهم إلى أورشليم بفرح عظيم. وليس ذلك فحسب، بل [أ: سيأكلون ويشربون لأنهم سيمجِّدون مملكة المسيح، من بيت داود، في أنحاء العالم الأربع. ولن تقوى عليهم أمم العالم،] [هـ: سيأتي بنو إسرائيل من أنحاء العالم الأربع ويأكلون مع المسيح. لكن أمم العالم لن تأكل معهم،] كما هو مكتوب: "قد كشف الرب عن ذراعه المقدسة أمام عيون كل الأمم، وسترى كل أقاصي الأرض خلاص إلهنا". ومرة ​​أخرى: "الرب وحده قاده، ولم يكن معه إله غريب".: "ويكون الرب ملكًا على كل الأرض".



\chapter{48b}

\par \textit{الأسماء الإلهية التي تخرج من عرش المجد، متوجة ومُرافقة من قِبل العديد من الجحافل الملائكية عبر السماوات وتعود مرة أخرى إلى العرش - الملائكة تُنشد "القدوس" و"المبارك"}

\par 1 [AEFGH: هذه هي أسماء القدوس، تبارك اسمه] [K: هذه هي الأسماء الاثنين والسبعين المكتوبة على قلب القدوس، تبارك اسمه: SS، SeDeQ (البر)، SaHI'eL SUR، SBI، SaDdlQ {البار}، S'Ph، SHN، SeBa'oTh (رب الجنود)، ShaDdaY (الله القدير)، 'eLoHIM (الله)، YHWH، SH، DGUL، WDOM، SSS''، 'YW، y T، 'HW، HB، YaH، HW، WWW، SSS، PPP، NN، HH، HaY (حي)، HaY، ROKeB 'aRaBOTh (راكبًا على العربوت)، YH، HH، WH، MMM، NNN، HWW، YH، YHH، HPhS، H'S، W، S'Z' QQQ (قدوس، قدوس، قدوس)، QShR، BW، ZK، GINUR، GINURYa'، T، YOD، 'aLePh، H'N، P'P، R'W، YYW، YYW، BBB، DDD، TTT، KKK، KLL، SYS، TT BShKMLW (= مبارك اسم مملكته المجيدة إلى الأبد)، مكتمل لـ MeLeK Ha'OLaM (ملك الكون)، BRH LB' (بداية الحكمة لأبناء البشر)، BNLK W''Y (مبارك هو الذي يعطي القوة للمتعبين ويزيد القوة لأولئك الذين ليس لديهم قوة.) ]الذين يخرجون (مزينين) بالعديد من تيجان النار مع العديد من تيجان اللهب، مع العديد من تيجان chashmal، مع العديد من تيجان البرق من أمام عرش المجد. ومعهم آلاف المئات من القوة (أي الملائكة الأقوياء) الذين يرافقونهم مثل الملك [AE: مع شرف وأعمدة من نار وسحابة، وأعمدة من لهب، وبروق من إشعاع ومع شبه (ال) تشاشمال.] [FG: مع ارتعاش ورعب، مع رهبة ورعدة، مع شرف وجلالة وخوف، مع رعب، مع عظمة وكرامة، مع مجد وقوة، مع فهم ومعرفة ومع عمود من نار وعمود من لهب وبرق - ونورهم كبروق من نور - ومع شبه تشاشمال.]

\par 2 ويُمجِّدونهم ويُجيبون ويُنادون أمامهم: قدوس، قدوس، قدوس. ويُطوفون بهم في كل سماء كأمراء أقوياء مُكرَّمين. وعندما يُعيدونهم جميعًا إلى عرش المجد، يفتح جميع الأحباء بجانب المركبة أفواههم في تسبيح اسمه المجيد، قائلين: "ليكن اسم مملكته المجيدة مباركًا إلى الأبد".



\chapter{48c}

\par \textit{قطعة من أخنوخ-ميتاترون}

\par 1 [AEFGH: ألفٌ قوّيته، أخذته، عيّنته: (أي) ميتاترون، خادمي الذي هو فريدٌ من بين جميع أبناء السماء. قوّيته في جيل آدم الأول. ولكن لما رأيتُ رجال جيل الطوفان فاسدين، ذهبتُ ونزعتُ شكينتي من بينهم. ورفعتُها عالياً بصوت بوقٍ وهتاف، كما هو مكتوب: "صعد الله بهتاف، الرب بصوت بوق".] [K: "قبضتُ عليه، وأخذتُه، وعيّنتُه" - أي أخنوخ،]


\par 2 [AEFGH: "وأخذته": (أي) أخنوخ، ابن يارد، من بينهم. ورفعته بصوت بوق وبتروعة (صيحة) إلى السماوات العالية، ليكون شاهدًا لي مع Chayyoth، من قبل المركبة في العالم القادم. ] [K: ابن يارد، واسمه ميتاترون (2) وأخذته من بين أبناء البشر (5) وجعلته عرشًا مقابل عرشي. ما هو حجم هذا العرش؟ سبعون ألف فرسخ (كلها) من نار. 9 لقد أوكلت إليه 70 ملائكة يتوافقون مع الأمم (في العالم) وسلمت إلى عهدته جميع أهل البيت أعلاه وأسفل. (7) وأوكلت إليه الحكمة والذكاء أكثر من (إلى) جميع الملائكة. وسميته "ياه الأصغر"، واسمه حسب الجيماتريا ٧١. ورتبت له جميع أعمال الخلق. وجعلت قدرته تفوق (أي جعلته قوة تفوق) جميع الملائكة الخادمين. (ينتهي K).]

\par 3 [AEFGH: لقد عيّنته على جميع الخزائن والمخازن التي أملكها في كل سماء. وسلمت في يده مفاتيح كل واحدة منها.] [Lm (يبدأ هنا): لقد سلم إلى ميتاترون - أي أخنوخ، ابن يارد - جميع الخزائن. وعيّنته على جميع المخازن التي أملكها في كل سماء. وسلمت في يديه مفاتيح كل مخزن سماوي.]

\par 4 [AEFGH: جعلته أميرًا على جميع الأمراء ووزيرًا لعرش المجد وقاعات العربوت: ليفتح لي أبوابها، ولعرش المجد، ليرفعه ويرتبه؛ (وعيّنته على) الحيوت المقدسة ليضع التيجان على رؤوسهم؛ والأوفانيم المهيبين ليتوجهم بالقوة والمجد؛ والكروبيم المكرمين، ليلبسهم الجلالة؛ وعلى الشرارات المشعة، ليجعلهم يتألقون بروعة وتألق؛ وعلى السيرافيم المشتعلين، ليغطوهم بالسموّ؛ حششماليم النور، ليجعلهم مشعين بالنور وليهيئوا المقعد لكل صباح] [لم: جعلته أميرًا على جميع الأمراء، وجعلته وزيرًا لعرشي المجد، ليوفر ويرتب الكائنات المقدسة، وليُكللهم بالتيجان (لتتويجهم بالتيجان)، وليُلبسهم الشرف والجلال وليُهيئ لهم مقعدًا] [أ: وأنا أجلس على عرش المجد. ولأُعظم وأُعظم مجدي في أوج قوتي؛ (وقد عهدت إليه) بأسرار ما فوق وأسرار ما تحت (أسرار سماوية وأسرار أرضية).] [فغ: عندما أجلس على عرشي في مجد وكرامة حتى يرى مجدي في أوج قوتي، في أسرار ما فوق وفي أسرار ما تحت.] [لم: عندما يجلس على عرشه ليعظم مجده في الأعالي]

\par 5 [أفغ: جعلته أعلى من الجميع. طول قامته بين جميع الطولاء سبعون ألف فرسخ. عظمت عرشه بجلال عرشي. وزدت مجده بشرف مجدي.] [لم: طول قامته بين جميع الطولاء سبعون ألف فرسخ. وعظمت مجده كجلال مجدي.]

\par 6 [AFGH: حولتُ لحمه إلى مشاعل نار، وجميع عظام جسده إلى جمر ناري؛ وجعلتُ منظر عينيه كالبرق، ونور حاجبيه كالنور الخالد. جعلتُ وجهه يلمع كبهاء الشمس، وعينيه كبهاء عرش المجد.] [Lm: وبريق عينيه كبهاء عرش المجد]

\par 7 [AFGH: لقد شرفت وجلال ملابسه، وجماله وسموه عباءته وتاجًا ملكيًا من 500 فرسخ (مرات) فرسخ (له).] [Lm: ثوبه شرف وجلال، تاجه الملكي 500 في 500 فرسخ.] [AFGHLm: وألبسته من شرفي وجلالي وبهائي، من مجدي الذي على عرشي المجيد. دعوته الرب الأصغر، أمير الحضور، عالم الأسرار: لأني كشفت له عن كل سر5a كأب، وأعلنت له جميع الأسرار باستقامة.]

\par 8 نصبت عرشه على باب قاعتي ليجلس ويحكم على أهل البيت السماويين في الأعالي. ووضعت كل أمير أمامه، ليأخذ منه السلطة، ولينفذ إرادته

\par 9 سبعون اسمًا أخذتها من أسمائي ودعوته بها ليعظم مجده

\par 10 سبعون أميراً استسلموا في يده، لكي يأمرهم بأوامري وكلماتي بكل لسان؛ [AFGH: لإذلال المتكبرين إلى الأرض بكلمته، ورفع المتواضعين إلى العلاء بنطق شفتيه؛ "ليضرب الملوك بكلامه، ويرد الملوك عن سبلهم، ويقيم حكامًا على سلطانهم كما هو مكتوب: "ويغير الأوقات والأوقات، ويعطي حكمة لكل حكماء العالم، وفهمًا ومعرفة لكل من يفهم المعرفة"، كما هو مكتوب: "والمعرفة للذين يعرفون الفهم"، ليكشف لهم أسرار كلماتي، ويعلمهم قضاء حكمي العادل، كما هو مكتوب: ] [ل م: ولإذلال المتكبرين إلى الأرض، ورفع المتواضعين إلى العلاء، وضرب الملوك، وإخضاع الحكام، وإقامة الملوك والحكام، ويغير الأوقات والأوقات، ويعزل الملوك، ويقيم الملوك، ويعطي حكمة للحكماء، ومعرفة للذين يعرفون الفهم، وقد جعلته ليكشف الأسرار، ويعلم الحق والعدل، ] "هكذا تكون كلمتي التي تخرج من فمي، لا ترجع إلي فارغة، بل تنجز (ما أنا من فضلك، لم يُكتب هنا "إييسه ي" (سأُنجز)، بل "إل آساه ي" (سيُنجز)، أي أن أي كلمة أو لفظ يخرج من أمام القدوس، تبارك اسمه، يقف ميتاترون وينفذه. ويُرسي أحكام القدوس، تبارك اسمه. (هنا تنتهي النسخة "ل" من الجزء "ج").

\par 11 ["ويُنجِحُ مَا أَرْسَلْتُهُ". لم يُكتب هنا "أَسْلَعَاهُ" (سأُنجِحُه)، بل "ويُنجِحُه" (سيُنجِحُه)، مُعلِّمًا أن أيَّ حُكمٍ يَصْدُرُ مِن قَبْلِ القُدُّوسِ، تبارك اسمه، على إنسان، فبمجرد أن يُتاب، لا "يُنفِّذُونَهُ (عليه) بل على آخر، شرير، كما هو مكتوب: "يُنْجَى الصِّدِّيقُ مِنَ الضِّيقِ، وَيَأْتِي الشِّرِّيرُ مَكَانَهُ".]

\par 12 وليس هذا فحسب، بل يجلس ميتاترون ثلاث ساعات كل يوم في السماوات العليا، ويجمع كل أرواح الموتى الذين ماتوا في بطون أمهاتهم، والرضع الذين ماتوا على صدور أمهاتهم، والعلماء الذين ماتوا على أسفار الشريعة الخمسة. ويضعهم تحت عرش المجد ويضعهم في فرق وأقسام وصفوف حول الحضرة: ويعلمهم الشريعة، و(كتب) الحكمة، والهاجادا والتقاليد، ويكمل (يكمل) تعليمهم [لهم]. كما هو مكتوب: "من سيعلم المعرفة؟ ومن سيفهم التقاليد؟ أولئك الذين فُطموا عن الحليب ورُضِعوا عن الصدور".


\chapter{48د}

أسماء ميتاترون. فُتحت كنوز الحكمة لموسى على جبل سيناء. احتجّت الملائكة على ميتاترون لإفصاحه الأسرار لموسى، فأجابهم الله ووبخهم. سلسلة التقاليد وقوة الأسرار المتناقلة في شفاء الأمراض.

\par 1 سبعون اسمًا لميتاترون أخذها القدوس، تبارك اسمه، من اسمه ووضعها عليه. وهذه هي:

\par يهوئيل ياه، يهوئيل، يوفيل ويوففيل، و5 'أفيئيل ومارجزيل، جبوييل، بعزئيل، آه، بيريئيل، تاتريل، تبقيل، 'W، يهوه، DH لماذا،'eBeD، DiBbbURIEL، 'aPH'aPIEL، SPPIEL، PaSPaSIEL، سينيجرون، ميتاترون، سوغدين، 'أ- دراغون، أسوم، ساكبام، ساكتام، ميغون، ميتون، موترون، روسفيم، كينوث، تشاتاتياه، ديغازياه، بسبياه، بسكنيه، مزرغ، باراد، MKRKK، MSPRD، ChShG، ChShB، MNRTTT، بسيريم، متيمون، تيمون، بيسكون، سافسافياه، زرتش، زرشياه، بياه، هبه بياه، بيليت، بلتياه، راباياه، تشاس، تشاسياه، تافتافياه، تمياه، سيهاسياه، إيروريا، لالياه، بازريديه، ساتسكياه، ساسديه، رارزازيه، بريزيه، اريميه، صبحيه، سبخيه، ستمكام، يحسيه، سبيبيه، سبكسبيه، قليلقاليه، كيهه، هيه، زوه، لماذا، زكلياه، توتريسياه، سوريا، زِهْ، بِنِرْهَياه، زِرْهْ، جَالْ رَزَايَا، مِمِلْيَكِيَّة، تِيَاهْ، إِمِقْ، قَامِيَاهْ، مِكَابَرِ يَاهُ، بِرْشِيَاهُ، سِفَامْ، جِبِير، جِبْرُيَاهُ، جَور، جَور يَاهُ، زِو، عُقْبَار، الرَّبُّ الأَصْغَرُ، بَعْدَ اسْمِ سِيرَتِهِ، "لأَنَّ اسْمِي فِيهِ"، رَبيبِيلِ، تَمِيئِيل، سِغَانْسَاكِيلِ، أَمْرِ الْحِكْمَةِ.

\par 2 ولماذا يُدعى ساغنيساكيل؟ لأن جميع كنوز الحكمة محفوظة في يده

\par 3 وفُتح جميعها لموسى على جبل سيناء، فتعلمها خلال الأربعين يومًا، وهو واقف (باقي): التوراة في سبعين وجهًا من السبعين لسانًا، والأنبياء في سبعين وجهًا من السبعين لسانًا، والكتابات في سبعين وجهًا من السبعين لسانًا، والهالاكا في سبعين وجهًا من السبعين لسانًا، والتقاليد في سبعين وجهًا من السبعين لسانًا، والهاجاداس في سبعين وجهًا من السبعين لسانًا، والتوسفتاس في سبعين وجهًا من السبعين لسانًا

\par 4 ولكن بمجرد انتهاء الأربعين يومًا، نسيها جميعًا في لحظة واحدة. ثم دعا القدوس، تبارك اسمه، يفيعيا، رئيس الشريعة، ومن خلاله أُعطيت لموسى كهدية. كما هو مكتوب: "وأعطاني الرب إياها". وبعد ذلك بقيت معه. ومن أين نعرف أنها بقيت (في ذاكرته)؟ لأنه مكتوب: "اذكروا شريعة موسى عبدي12 التي أمرته بها في حوريب على كل إسرائيل، فرائضي وأحكامي". "شريعة موسى": أي التوراة، والأنبياء والكتب، "الفرائض": أي الشريعة والتقاليد، "الأحكام"؛ أي الهاجادات والتوسفتات. وأُعطيت جميعها لموسى في الأعالي على جبل سيناء،

\par 5 هذه الأسماء السبعون هي انعكاس للاسم (أو الأسماء) الصريح على المركبة المنقوشة على عرش المجد. فالقدوس، تبارك اسمه، أخذ من اسمه (أو أسماءه) الصريح ووضع على اسم ميتاترون: سبعون اسمًا له، ينادي بها الملائكة الخادمون ملك ملوك الملوك، تبارك اسمه، في السماوات العلى، واثنان وعشرون حرفًا على خاتم إصبعه، تُختم بها أقدار أمراء الممالك العليا في العظمة والقوة، وتُختم بها قرعة ملك الموت، ومصائر كل أمة ولسان.

\par 6 قال ميتاترون، الملاك، أمير الحضور؛ الملاك، أمير الحكمة؛ الملاك، أمير الفهم؛ الملاك، أمير الملوك؛ الملاك، أمير الحكام؛ الملاك، أمير المجد؛ الملاك، أمير العليَّين، والأمراء، العظماء والمكرمين، في السماء وعلى الأرض:

\par 7 ح، إله إسرائيل، شاهدي على هذا الأمر، (أنه) عندما كشفت هذا السر لموسى، ثار عليّ كل الجيوش في كل سماء عالية وقالوا لي:

\par 8 لماذا تكشف هذا السر لابن الإنسان، المولود من امرأة، دنس ونجس، رجل من قطرة متعفنة، السر الذي به خُلقت السماء والأرض، والبحر واليابسة، والجبال والتلال، والأنهار والينابيع، وجهنم النار والبرد، وجنة عدن وشجرة الحياة؛ والذي به خُلقت آدم وحواء، والبهائم، والوحوش البرية، وطيور السماء، وأسماك البحر، وبهيمو وليفياثان، والزحافات، والديدان، وتنانين البحر، وزحافات الصحاري؛ والتوراة والحكمة والمعرفة والفكر ومعرفة الأشياء العليا ومخافة السماء. لماذا تكشف هذا للجسد والدم؟ [أ: هل حصلت على سلطة من مقام؟ ومرة ​​أخرى: هل حصلت على إذن؟ [فج: أجبتهم: لأن القدوس تبارك اسمه قد أعطاني السلطة، وعلاوة على ذلك، فقد حصلت على إذن من العرش العالي والمرتفع، الذي تخرج منه جميع الأسماء الظاهرة] مع برق من نار وحرق ملتهب.

\par 9 لكنهم لم يهدأوا حتى وبخهم القدوس، تبارك اسمه، وطردهم بالتوبيخ من أمامه، قائلاً لهم: "أنا أُسرُّ بميتاترون، وقد وضعتُ محبتي عليه، وأوكلتُه إليه، وأودعته، لأنه واحد (فريد) بين جميع أبناء السماء

\par 10 وأخرجها ميتاترون من بيت خزائنه وسلمها إلى موسى، وموسى إلى يشوع، ويشوع إلى الشيوخ، والشيوخ إلى الأنبياء، والأنبياء إلى رجال المجمع العظيم، ورجال المجمع العظيم إلى عزرا، وعزرا الكاتب إلى هليل الشيخ، وهليل الشيخ إلى ربي أباهو، وربي أباهو إلى ربي زرع، وربي زرع إلى رجال الإيمان، وسلمها رجال الإيمان لتحذير الناس وشفاء جميع الأمراض المستشرية في العالم، كما هو مكتوب: "إن سمعت لصوت الرب إلهك، وعملت ما هو مستقيم في عينيه، وأصغيت إلى وصاياه، وحفظت جميع فرائضه، فلن أضع عليك شيئًا من الأمراض التي وضعتها على المصريين، لأني أنا الرب شافيك". (انتهى وانتهى. الحمد لله رب العالمين.)




\end{document}