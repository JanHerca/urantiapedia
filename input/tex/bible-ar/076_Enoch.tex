\begin{document}

\title{سفر أخنوخ}

\part{I. Introduction}

\chapter{1}

\par 1 كلمات بركة أخنوخ، التي بارك بها المختارين والأبرار، الذين سيعيشون في يوم الضيق، عندما يُزال جميع الأشرار والكافرين
\par 2 ثم أخذ مثله وقال: «إن أخنوخ البار الذي فتح الله عينيه، رأى رؤيا القدوس في السماء التي أراني إياها الملائكة، ومنهم سمعت كل شيء، ومنهم فهمت كما رأيت، ولكن ليس لهذا الجيل، بل للجيل البعيد الذي هو آتٍ.»
\par 3 «أما المختارون فقلتُ وأخذتُ مثلي عنهم: سيخرج القدوس العظيم من مسكنه،»
\par 4 «وسيسير الإله الأزلي على الأرض، (حتى) على جبل سيناء، [ويظهر من معسكره] ويظهر بقوة قدرته من سماء السموات.»
\par 5 «وسيُصاب الجميع بالخوف، وسيرتعد المراقبون، وسيُسيطر عليهم خوف ورعدة عظيمان حتى أقاصي الأرض.»
\par 6 «وتهتز الجبال الشامخة، وتنخفض التلال الشامخة، وتذوب كالشمع أمام اللهيب.»
\par 7 «وتتشقق الأرض كلها، ويهلك كل ما عليها، ويكون هناك دينونة على الجميع».
\par 8 «لكنه سيصنع السلام مع الأبرار. وسيحمي المختارين، وستكون عليهم الرحمة. وسيكونون جميعًا لله، وسيزدهرون، وسيُباركون جميعًا. وسيعينهم جميعًا، وسيظهر لهم النور، وسيصنع معهم السلام.»
\par 9 «وهوذا يأتي مع ربوات من قديسيه ليُجري دينونة على الجميع، ويهلك جميع الأشرار، ويُدين كل بشر على جميع أعمال فجورهم التي ارتكبوها، وعلى جميع الأمور الصعبة التي تكلم بها عليه الخطاة الأشرار.»

\chapter{2}

\par 1 لاحظوا كل ما يحدث في السماء، كيف أنها لا تغير مداراتها، والنيّرات التي في السماء، كيف تشرق وتغرب جميعها بالترتيب، كلٌ في موسمه، ولا تتعدى على ترتيبها المعين
\par 2 انظروا إلى الأرض، وانظروا إلى الأشياء التي تحدث فيها من البداية إلى النهاية، كيف هي ثابتة، وكيف لا يتغير شيء من الأشياء على الأرض، لكن جميع أعمال الله تظهر لكم.
\par 3 انظر إلى الصيف والشتاء، كيف تمتلئ الأرض كلها بالماء، ويسقط عليها السحاب والندى والمطر.

\chapter{3}

\par 1 لاحظ وانظر كيف تبدو جميع الأشجار (في الشتاء) وكأنها ذبلت وتساقطت أوراقها، باستثناء أربع عشرة شجرة، لا تفقد أوراقها بل تحتفظ بأوراقها القديمة من سنتين إلى ثلاث سنوات حتى يأتي الجديد

\chapter{4}

\par 1 وانظروا أيضًا إلى أيام الصيف، كيف تكون الشمس فوق الأرض مقابلها. وأنتم تبحثون عن ظلٍّ ومأوى بسبب حرارة الشمس، والأرض أيضًا تحترق بحرارة متزايدة، ولذلك لا يمكنكم أن تطأوا الأرض ولا الصخر من شدة حرارتها.

\chapter{5}

\par 1 لاحظوا كيف تغطي الأشجار نفسها بأوراق خضراء وتثمر، لذلك انتبهوا واعرفوا جميع أعماله، واعرفوا كيف أن الحي إلى الأبد قد خلقها كذلك
\par 2 وجميع أعماله تستمر هكذا من سنة إلى سنة إلى الأبد، وجميع المهام التي يقومون بها من أجله، ومهامهم لا تتغير، بل كما قدّر الله هكذا يتم
\par 3 وانظر كيف أن البحر والأنهار على نفس المنوال تُنجز مهامها ولا تبتعد عن وصاياه
\par 4 لكنكم لم تكونوا ثابتين، ولم تعملوا بوصايا الرب، بل ارتددتم وتكلمتم بكلام متكبر وقاسٍ بأفواهكم النجسة ضد عظمته. يا قساة القلوب، لن تجدوا سلامًا
\par 5 لذلك ستلعنون أيامكم، وستفنى سنو حياتكم، وستكثر سنو هلاككم في لعنة أبدية، ولن تجدوا رحمة
\par 6 [أ] في تلك الأيام تجعلون أسماءكم لعنة أبدية لجميع الصالحين. [ب] وبكم يلعن جميع اللاعنين، وجميع الخطاة والكافرين. ويفرح جميع ... ، ويكون هناك غفران للخطايا، وكل رحمة وسلام وصبر؛ سيكون لهم خلاص، نور صالح. ولكم جميعًا أيها الخطاة لن يكون هناك خلاص، بل ستبقى عليكم جميعًا لعنة
\par 7 أما المختارون فسيكون لهم نور وفرح وسلام، وهم يرثون الأرض. وأما أنتم أيها الكافرون فستكون لكم لعنة
\par 8 وحينئذٍ يُمنح المختارون الحكمة، فيعيشون جميعًا ولن يخطئوا مرة أخرى، سواءً بسبب الإلحاد أو بسبب الكبرياء. أما الحكماء فيتواضعون
\par 9 ولن يعودوا إلى التعدي، ولن يخطئوا كل أيام حياتهم، ولن يموتوا من الغضب (الإلهي) أو السخط، بل سيكملون عدد أيام حياتهم. وستزداد حياتهم في سلام، وستتضاعف سنو فرحهم، في سعادة وسلام أبديين، كل أيام حياتهم

\chapter{6}

\par 1 ولما تكاثر بنو البشر، وُلدت لهم في تلك الأيام بنات جميلات ورائعات
\par 2 فرأى الملائكة أبناء السماء فاشتاقوا إليهم وقال بعضهم لبعض: هلموا نختار لأنفسنا زوجات من بين أبناء البشر ونلد لأنفسنا أولاداً.
\par 3 وقال لهم سمجازا، الذي كان زعيمهم: "أخشى أنكم لن توافقوا على القيام بهذا العمل، وسأضطر وحدي إلى دفع عقوبة الخطيئة الكبرى".
\par 4 فأجابوه جميعهم وقالوا: "لنحلف جميعًا يمينًا ونلتزم جميعًا باللعنات المتبادلة ألا نترك هذا الأمر بل أن نفعل هذا الأمر".
\par 5 ثم أقسموا جميعًا معًا وربطوا أنفسهم باللعنات المتبادلة عليه
\par 6 وكانوا جميعًا مئتين، الذين نزلوا في أيام يارد على قمة جبل حرمون، ودعوه جبل حرمون، لأنهم أقسموا عليه وربطوا أنفسهم باللعنات المتبادلة
\par 7 وهذه أسماء قادتهم: سيميزاز، قائدهم، أراكيبة، راميل، كوكبييل، تاميل، راميل، دانيل، حزقيال، باراكيجال، أسائيل، أرماروس، باتاريل، أنانيل، زاقيل، سامسابيل، ساتاريل، توريل، جومجيل، سارييل
\par 8 هؤلاء هم رؤساء عشراتهم.

\chapter{7}

\par 1 "وأخذ كل الآخرين معهم نساء، واختار كل واحد لنفسه واحدة، وبدأوا يدخلون إليهن وينجسون بهن، وعلموهن السحر والتعاويذ وقطع الجذور، وعرفوهن على النباتات."
\par 2 فحملتا، وولدتا عمالقة عظامًا، بلغ طولهم ثلاثة آلاف ذراع،
\par 3 الذين استهلكوا جميع مقتنيات الرجال. وعندما لم يعد الرجال قادرين على تحملها،
\par 4 انقلب عليهم العمالقة والتهموا البشر.
\par 5 وبدأوا يخطئون إلى الطيور والوحوش والزواحف والأسماك، ويأكلون لحوم بعضهم البعض، ويشربون دم بعضهم البعض
\par 6 ثم وجهت الأرض اتهاما ضد الأشرار.

\chapter{8}

\par 1 وعلم عزازيل الرجال صنع السيوف والسكاكين والدروع والدروع، وعرّفهم على معادن الأرض وفن تشكيلها، والأساور والحلي، واستخدام الإثمد، وتجميل الجفون، وجميع أنواع الأحجار الكريمة، وجميع صبغات التلوين
\par 2 فكثرت الفجور، وزنوا، وضلوا، وفسدوا في كل طرقهم.
\par 3 كان سيمجازا يعلم السحر وقطع الجذور، وكان أرماروس يعلم حل السحر، وكان باراكيجال يعلم علم التنجيم، وكان كوكابيل يعلم الأبراج، وكان حزقيال يعلم معرفة السحب، وكان أراكيل يعلم علامات الأرض، وكان شامسيل يعلم علامات الشمس، وكان سارييل يعلم مسار القمر.
\par 4 "ولما هلك الرجال صرخوا، وارتفع صراخهم إلى السماء..."

\chapter{9}

\par 1 ثم نظر ميخائيل وأورييل ورافائيل وجبرائيل من السماء ورأوا دماءً كثيرة تُسفك على الأرض، وكل إثم يُرتكب على الأرض
\par 2 وقال بعضهم لبعض: «الأرض الخاوية تصرخ بصوت صراخهم إلى أبواب السماء».
\par 3 "والآن إليكم يا قديسي السماء تتقدم أرواح البشر بطلبها قائلة: أحضروا قضيتنا أمام العلي."
\par 4 فقالوا لرب الدهور: رب الأرباب، إله الآلهة، ملك الملوك، وإله الدهور.
\par 5 عرش مجدك قائمٌ إلى جميع أجيال الدهور، واسمك قدوسٌ ومجيدٌ ومباركٌ إلى جميع الدهور! أنت خلقتَ كلَّ شيء، ولك سلطانٌ على كلِّ شيء، وكلُّ شيءٍ عريانٌ ومكشوفٌ أمامَ نظرك، وأنت ترى كلَّ شيء، ولا شيءَ يختفي عنك.
\par 6 «أنت ترى ما فعله عزازيل، الذي علّم كل إثم على الأرض وكشف الأسرار الأبدية التي كانت (محفوظة) في السماء، والتي كان البشر يسعون جاهدين لتعلمها.»
\par 7 «وسمجازا الذي وهبت له سلطانًا على شركائه.»
\par 8 «وذهبوا إلى بنات الناس على الأرض، وضاجعوا النساء، ونجّسوا أنفسهم، وكشفوا لهن كل أنواع الخطايا.»
\par 9 «وأنجبت النساء عمالقة، وامتلأت الأرض كلها بالدماء والظلم.»
\par 10 «والآن، ها هي أرواح الذين ماتوا تبكي وتتوجه إلى أبواب السماء، وقد تصاعدت نواحهم، ولا يمكن أن تتوقف بسبب الأعمال غير القانونية التي ارتُكبت على الأرض.»
\par 11 «وأنت تعلم كل الأشياء قبل أن تحدث، وترى هذه الأشياء وتتحملها، ولا تخبرنا بما يجب أن نفعله بها بشأنها.»

\chapter{10}

\par 1 ثم قال العلي، القدوس العظيم، وأرسل أوريئيل إلى ابن لامك، وقال له:
\par 2 «اذهب إلى نوح وقل له باسمي: «اختبئ!» واكشف له عن النهاية التي تقترب، وهي أن الأرض كلها ستُدمر، وأن طوفانًا على وشك أن يأتي على الأرض كلها، وسيدمر كل ما عليها.»
\par 3 «والآن أوصه بأن ينجو ويُحفظ نسله لجميع أجيال العالم.»
\par 4 وقال الرب أيضًا لرافائيل: "اربط عزازيل بيديه ورجليه، وألقه في الظلمة، واصنع له فتحة في البرية التي في دودائيل، وألقه فيها."
\par 5 «وضعوا عليه صخورًا خشنة ومسننة، وغطوه بالظلام، وليبق هناك إلى الأبد، وغطوا وجهه حتى لا يرى النور.»
\par 6 "وفي يوم الدينونة العظيمة سيُلقى في النار."
\par 7 "وشفاء الأرض التي أفسدها الملائكة، وإعلان شفاء الأرض، حتى يتمكنوا من شفاء الطاعون، وحتى لا يهلك جميع أبناء البشر من خلال كل الأشياء السرية التي كشفها المراقبون وعلموها لأبنائهم."
\par 8 "وفسدت الأرض كلها بسبب الأعمال التي علمها عزازيل. إليه ارجع كل الخطيئة."
\par 9 وقال الرب لجبرائيل: "اذهب إلى الأوغاد والمنبوذين وضد أبناء الزنا ودمر [أبناء الزنا] وأبناء المراقبين من بين الرجال [وجعلهم يخرجون]، أرسلهم بعضهم على بعض حتى يدمروا بعضهم بعضًا في المعركة، لأنه لن يكون لهم طول الأيام".
\par 10 "ولا يعطى لآبائهم من أجلهم كل ما يطلبونه منك، لأنهم يرجون حياة أبدية، وأن يعيش كل واحد منهم خمسمائة سنة."
\par 11 وقال الرب لميخائيل: "اذهب، اربط سمجازا ورفاقه الذين اتحدوا مع النساء حتى تنجسوا معهن بكل نجاستهم."
\par 12 «وعندما يقتل أبناؤهم بعضهم بعضًا، ويشاهدون هلاك أحبائهم، اربطوهم بقوة سبعين جيلًا في أودية الأرض، إلى يوم دينونتهم وفنائهم، إلى أن يتم الدينونة التي هي إلى الأبد.»
\par 13 «في تلك الأيام، سيُساقون إلى هاوية النار، وإلى العذاب والسجن الذي سيُحبسون فيه إلى الأبد.»
\par 14 «وكل من يُدان ويُهلك، سيُربط معهم من الآن فصاعدًا إلى نهاية كل الأجيال.»
\par 15 «ودمر جميع أرواح الأشرار وأبناء المراقبين، لأنهم ظلموا البشرية.»
\par 16 «أُدمِّر كلَّ خطأٍ من على وجه الأرض، وليُقضِ على كلِّ عملٍ شرير، وليظهر غرسُ البرِّ والحقِّ، فيكون بركةً؛ وستُغرس أعمالُ البرِّ والحقِّ في الحقِّ والفرحِ إلى الأبد.»
\par 17 وحينئذٍ ينجو جميع الصالحين، ويعيشون حتى ينجبوا آلاف الأطفال، ويُكملوا كل أيام شبابهم وشيخوختهم بسلام
\par 18 وحينئذٍ تُفلح الأرض كلها بالبر، وتُغرس جميعها بالأشجار، وتكون مليئة بالبركة
\par 19 ويُغرس فيها كل شجرة شهية، ويُغرس فيها كروم، والكرمة التي يغرسونها عليها تُعطي خمرًا بكثرة، وكل بذر يُزرع فيها يُنتج كل مكيال منها ألفًا، وكل مكيال زيتون يُنتج عشر معاصر زيت
\par 20 وطهّر الأرض من كل ظلم، ومن كل إثم، ومن كل خطيئة، ومن كل إثم، ومن كل نجاسة تُصنع على الأرض، وأزلها من على وجه الأرض
\par 21 ويصبح جميع أبناء البشر أبرارًا، وستُقدِّم جميع الأمم عبادةً وتُسبِّحني، وسيعبدني الجميع
\par 22 وستُطهَّر الأرض من كل دنس، ومن كل خطيئة، ومن كل عقاب، ومن كل عذاب، ولن أُرسِلهم عليها مرة أخرى من جيل إلى جيل وإلى الأبد

\chapter{11}

\par 1 وفي تلك الأيام سأفتح خزائن البركة التي في السماء، لأرسل
\par 2 لهم على الأرض من أجل عمل وكدح أبناء البشر. وسيرتبط الحق والسلام معًا طوال أيام العالم وعبر جميع أجيال البشر

\chapter{12}

\par 1 قبل هذه الأمور كان حنوك مختبئا، ولم يكن أحد من بني البشر يعلم أين كان مختبئا، وأين كان يقيم، وماذا حدث له.
\par 2 وكانت أنشطته تتعلق بالمراقبين، وكانت أيامه مع القديسين
\par 3 وكنت أنا أخنوخ أبارك رب الجلالة وملك الدهور، وإذا بالمراقبين ينادونني - أخنوخ الكاتب - ويقولون لي:
\par 4 «يا أخنوخ، يا كاتب البر، اذهب، وأخبر حراس السماء الذين تركوا السماء العالية، المكان الأبدي المقدس، ودنسوا أنفسهم بالنساء، وفعلوا كما يفعل أبناء الأرض، واتخذوا لأنفسهم نساء
\par 5 لقد أحدثتم دمارًا عظيمًا على الأرض. ولن يكون لكم سلام ولا غفران للخطايا،
\par 6 وبقدر ما يتلذذون بأطفالهم، سيرون مقتل أحبائهم، وعلى هلاك أطفالهم سيندبون، ويتضرعون إلى الأبد، لكن الرحمة والسلام لن تنالواهما

\chapter{13}

\par 1 فذهب أخنوخ وقال: "عزازيل، لن يكون لك سلام. لقد صدر عليك حكم شديد لربطك بالقيود."
\par 2 "ولن يُقبل منك عذر ولا طلب بسبب الإثم الذي علمته وبسبب كل أعمال الإثم والظلم والخطية التي أظهرتها للناس."
\par 3 ثم ذهبت وكلمتهم كلهم ​​جميعا، فخافوا جميعا وأخذهم الخوف والرعدة.
\par 4 وطلبوا مني أن أكتب لهم طلبًا لكي يجدوا المغفرة، وأن أقرأ طلبهم أمام رب السماء.
\par 5 فمنذ ذلك الحين لم يستطيعوا أن يتكلموا (معه) ولا أن يرفعوا أعينهم إلى السماء خجلاً من خطاياهم التي حكم عليهم بسببها.
\par 6 ثم كتبت دعوانا، والدعاء فيما يتعلق بأرواحهم وأعمالهم الفردية وفيما يتعلق بطلباتهم أن يمنحوا المغفرة والطول.
\par 7 ثم انطلقت وجلست عند مياه دان، في أرض دان، إلى الجنوب الغربي من حرمون. قرأت دعواهم حتى نمت
\par 8 وإذا حلمٌ أتاني، ووقعت عليّ رؤى، ورأيت رؤى عقاب، وجاءني صوت يأمرني أن أخبر به أبناء السماء، وأنبذهم
\par 9 ولما استيقظت، أتيتهم، فإذا هم جميعًا جالسون مجتمعون معًا، يبكون في سجن آبل الذي بين لبنان وسنصر، ووجوههم مغطاة
\par 10 وقصتُ عليهم جميع الرؤى التي رأيتها في النوم، وبدأتُ أتكلم بكلمات البر، وأوبخ المراقبين السماويين

\chapter{14}

\par 1 كتاب كلمات البر، وتوبيخ المراقبين الأبديين وفقًا لأمر القدوس العظيم في تلك الرؤية
\par 2 لقد رأيت في نومي ما سأقوله الآن بلسان من لحم ونفس فمي: وهو ما أعطاه العظيم للرجال للتحدث به وفهمه بالقلب.
\par 3 كما خلق الله الإنسان وأعطاه القدرة على فهم كلمة الحكمة، فكذلك خلقني أنا أيضًا وأعطاني القدرة على توبيخ المراقبين، أبناء السماء
\par 4 «كتبتُ عريضتك، وفي رؤيتي ظهر لي أن عريضتك لن تُمنح لك طوال أيام الأبدية، وأن الحكم قد صدر عليك أخيرًا: نعم (عريضتك) لن تُمنح لك.»
\par 5 «ومن الآن فصاعدًا لن تصعدوا إلى السماء إلى الأبد، وقد صدر المرسوم ليربطكم بقيود الأرض طوال أيام العالم.»
\par 6 «وأنكم سترون من قبل هلاك أبنائكم الأحباء، ولن تسروا بهم، بل سيسقطون أمامكم بالسيف.»
\par 7 «ولن يُستجاب لطلبك من أجلهم، ولا حتى من أجلك وحدك، حتى لو بكيت وصليتَ ونطقتَ بجميع الكلمات الواردة في الكتابة التي كتبتها.»
\par 8 وأُريتُ الرؤيا هكذا. إذا في الرؤيا دعتني السحب، واستدعاني الضباب، وتسارعت حركة النجوم والبروق، وجعلتني الرياح في الرؤيا أطير، ورفعتني إلى أعلى، وحملتني إلى السماء
\par 9 ودخلت حتى اقتربت من جدار مبني من البلورات ومحاط بألسنة من نار، وبدأ يخيفني
\par 10 ودخلتُ في ألسنة النار واقتربتُ من بيتٍ كبيرٍ مبنيٍّ من البلورات، وكانت جدران البيت كأرضيةٍ فسيفسائيةٍ من البلورات، وأرضيته من البلورات
\par 11 كان سقفها كمسار النجوم والبروق، وبينها كروبيم ناري، وسماؤها (صافية كالماء).
\par 12 أحاطت نار مشتعلة بالأسوار، واشتعلت بواباتها بالنار
\par 13 ودخلت ذلك البيت، وكان حارًا كالنار وباردًا كالجليد: لم تكن فيه ملذات الحياة؛ غمرني الخوف، وأصابني الرعشة
\par 14 وبينما كنت أرتجف وأرتجف، سقطت على وجهي. ورأيت رؤيا،
\par 15 وإذا ببيت ثانٍ، أعظم من الأول، وكانت البوابة بأكملها مفتوحة أمامي، وكانت مبنية من لهيب نار
\par 16 وقد تفوق في كل شيء من حيث الروعة والعظمة والاتساع لدرجة أنني لا أستطيع وصف روعته ومداه لكم
\par 17 وكانت أرضيتها من نار، وفوقها بروق ومسارات نجوم، وكان سقفها أيضًا نارًا ملتهبة
\par 18 ونظرت فرأيت فيه عرشًا عاليًا، منظره كالبلور، وعجلاته كالشمس المضيئة، وهناك كانت رؤيا الكروبيم
\par 19 ومن تحت العرش خرجت تيارات من نار ملتهبة حتى لم أستطع النظر إليها
\par 20 وجلس عليه المجد العظيم، وأشرقت ثيابه أكثر إشراقًا من الشمس، وكانت أشد بياضا من أي ثلج
\par 21 لم يستطع أي من الملائكة الدخول ورؤية وجهه بسبب العظمة والمجد، ولم يستطع أي جسد أن يراه
\par 22 كانت النار المشتعلة تحيط به، ونار عظيمة واقفة أمامه، ولم يستطع أحد من حوله أن يقترب منه: عشرات الآلاف أمامه، ومع ذلك لم يكن بحاجة إلى مشير
\par 23 والأطهار المقربون منه لم يفارقوه ليلاً ولم يفارقوه
\par 24 وحتى ذلك الحين كنتُ ساجدًا على وجهي، مرتجفًا، فناداني الرب بفمه، وقال لي: "تعالَ إلى هنا يا أخنوخ، واسمع كلامي."
\par 25 وجاء إليّ أحد القديسين وأيقظني، وجعلني أنهض وأقترب من الباب، فانحنيت وجهي إلى أسفل

\chapter{15}

\par 1 فأجاب وقال لي، فسمعت صوته: «لا تخف يا أخنوخ، أيها الرجل البار وكاتب البر. تقدم إلى هنا واسمع صوتي».
\par 2 «واذهب، وقل لحراس السماء، الذين أرسلوك للتشفع لهم: ينبغي أن تشفع للناس، لا للناس لك»
\par 3 «لماذا تركتم السماء العليا المقدسة الأبدية، وضاجعتم النساء، ودنستم أنفسكم ببنات الرجال، واتخذتم لأنفسكم نساءً، وفعلتم مثل أبناء الأرض، وأنجبتم عمالقة كأبنائكم.»
\par 4 «ومع أنكم كنتم قديسين، روحانيين، تعيشون الحياة الأبدية، فقد نجستم أنفسكم بدم النساء، وولدتم (أطفالًا) بدم لحم، وكأبناء البشر، اشتهيتم لحمًا ودمًا كما يفعل أيضًا أولئك الذين يموتون ويهلكون.»
\par 5 «لذلك أعطيتهم زوجات أيضًا ليحملوا وينجبوهن، حتى لا ينقصهم شيء على الأرض.»
\par 6 «لكنكم كنتم سابقًا روحانيين، تحيون الحياة الأبدية، وخالدين لجميع أجيال العالم.»
\par 7 «ولذلك لم أُعيِّن لكم زوجات، لأن الروحانيين في السماء، ففي السماء مسكنهم.»
\par 8 «والآن، سيُطلق على العمالقة، المولودين من الأرواح والجسد، أرواحًا شريرة على الأرض، وسيكون مسكنهم على الأرض.»
\par 9 «لقد خرجت الأرواح الشريرة من أجسادهم؛ لأنهم ولدوا من البشر، ومن المراقبين المقدسين هي بدايتهم وأصلهم البدائي؛ سيكونون أرواحًا شريرة على الأرض، وسيُطلق عليهم اسم الأرواح الشريرة.»
\par 10 «[أما أرواح السماء، ففي السماء سيكون مسكنها، وأما أرواح الأرض التي وُلدت على الأرض، فعلى الأرض سيكون مسكنها.]»
\par 11 «وأرواح العمالقة تُزعج، وتُضطهد، وتُدمر، وتُهاجم، وتُحارب، وتُسبب الدمار على الأرض، وتُسبب المتاعب: فهم لا يأخذون طعامًا، ومع ذلك يجوعون ويعطشون، ويُسببون الإساءات.»
\par 12 "وتقوم هذه الأرواح على بني البشر وعلى النساء لأنها خرجت منهم."

\chapter{16}

\par 1 من أيام ذبح وتدمير وموت العمالقة، من أرواح أجسادهم التي ستُدمرها الأرواح، بعد خروجها، دون أن تُحاسب - هكذا سيُدمرون حتى يوم القيامة، يوم الدينونة العظيمة التي سيُحاسب فيها العصر، على المراقبين والكافرين، نعم، سيُحاسبون بالكامل
\par 2 وأما الناظرون الذين أرسلوك للشفاعة فيهم، والذين كانوا في السماء من قبل، (فقل لهم):
\par 3 «لقد كنتم في السماء، ولكن لم تُكشف لكم جميع الأسرار بعد، وعرفتم الأسرار الباطلة، وقد كشفتم هذه الأسرار للنساء في قساوة قلوبكم، ومن خلال هذه الأسرار يفعل النساء والرجال الكثير من الشر على الأرض.»
\par 4 فقل لهم: «ليس لكم سلام».

\chapter{17}

\par 1 فأخذوني وأتوا بي إلى مكان كان فيه من كانوا مثل نار ملتهبة، وعندما أرادوا ظهروا كأنهم رجال.
\par 2 وأتوا بي إلى مكان الظلمة، وإلى جبل رأسه يمس السماء.
\par 3 ورأيت أماكن النيّرات وخزائن النجوم والرعود وفي الأعماق القصوى حيث القوس الناري والسهام وجعبتها والسيف الناري وجميع البروق.
\par 4 وأخذوني إلى المياه الحية، وإلى نار الغرب التي تستقبل كل غروب للشمس.
\par 5 ووصلتُ إلى نهر من نار تتدفق فيه النار كالماء وتصب في البحر الكبير باتجاه الغرب
\par 6 رأيت الأنهار العظيمة، وجئت إلى النهر العظيم والظلام العظيم، وذهبت إلى المكان الذي لا يمشي فيه بشر
\par 7 رأيت جبال ظلمة الشتاء والمكان الذي تتدفق منه جميع مياه الأعماق
\par 8 رأيتُ مصبات جميع أنهار الأرض وفم الغمر

\chapter{18}

\par 1 رأيتُ كنوزَ جميعِ الرياح. رأيتُ كيفَ جهّزَ بها الخليقةَ كلها وأسسَ الأرضِ الثابتة
\par 2 ورأيت حجر زاوية الأرض، ورأيت الرياح الأربع التي تحمل الأرض وجلد السماء.
\par 3 ورأيت كيف أن الرياح تمد أقبية السماء، وتجعل مركزها بين السماء والأرض: هذه هي أعمدة السماء.
\par 4 لقد رأيت رياح السماء وهي تدور وتجلب محيط الشمس وجميع النجوم إلى غروبها.
\par 5 رأيتُ الرياح على الأرض تحملُ الغيوم. رأيتُ مساراتِ الملائكة. رأيتُ في أقصى الأرضِ قبةَ السماءِ في الأعلى.
\par 6 ثم تقدمت فرأيت مكانًا يحترق ليلًا ونهارًا، حيث توجد سبعة جبال من الحجارة الرائعة، ثلاثة نحو الشرق، وثلاثة نحو الجنوب.
\par 7 وأما التي نحو المشرق فكانت من حجر ملون، وواحدة من لؤلؤ، وواحدة من أسمانجوني، والتي نحو الجنوب من حجر أحمر.
\par 8 وأما الأوسط فكان يصل إلى السماء مثل عرش الله من المرمر، وكان رأس العرش من الياقوت الأزرق.
\par 9 ورأيتُ نارًا مُلتهبة. وخلف هذه الجبال
\par 10 هي منطقة نهاية الأرض العظيمة: هناك اكتملت السماوات.
\par 11 ورأيت هاوية عميقة، بأعمدة من نار سماوية، ومن بينها رأيت أعمدة من نار تتساقط، كانت بلا حدود على حد سواء نحو الارتفاع ونحو العمق
\par 12 وخلف تلك الهاوية، رأيت مكانًا ليس فيه سماء من فوق، ولا أرض راسخة تحته: لم يكن عليه ماء، ولا طيور، بل كان مكانًا قفرًا ومرعبًا
\par 13 رأيت هناك سبعة نجوم كجبال عظيمة مشتعلة، وعندما سألت عنها،
\par 14 قال الملاك: "هذا المكان هو نهاية السماء والأرض: لقد أصبح هذا سجنًا للنجوم وجنود السماء."
\par 15 «والنجوم التي تتدحرج فوق النار هي التي تعدت أمر الرب في بداية طلوعها، إذ لم تظهر في أوقاتها المعينة.»
\par 16 «وغضب عليهم، وقيدهم حتى يتم إثمهم عشرة آلاف سنة».

\chapter{19}

\par 1 وقال لي أورييل: "هنا سيقف الملائكة الذين ارتبطوا بالنساء، وأرواحهم التي تتخذ أشكالًا مختلفة تُدنس البشرية وتقودهم إلى التضحية بالشياطين كآلهة، (هنا سيقفون) حتى يوم الدينونة العظيمة التي سيُحاكمون فيها حتى يتم القضاء عليهم."
\par 2 «وأيضًا ستُصبح نساء الملائكة الضالات صفارات إنذار.»
\par 3 «وأنا، حنوك، رأيتُ الرؤيا وحدي، نهايات كل الأشياء: ولن يرى أحدٌ كما رأيتُ.»

\chapter{20}

\par 1 وهذه أسماء الملائكة القديسين الذين يراقبون.
\par 2 أورييل، أحد الملائكة القديسين، الذي يسود العالم وفوق تارتاروس
\par 3 رافائيل، أحد الملائكة القديسين، الذي يرأس أرواح البشر.
\par 4 راجويل، أحد الملائكة القديسين الذي ينتقم من عالم المُنيرين
\par 5 ميخائيل، أحد الملائكة القديسين، أي الذي وُضِعَ على أفضل جزء من البشرية وعلى الفوضى
\par 6 سراقائيل، أحد الملائكة القديسين، المُكلَّف بالأرواح التي تُخطئ في الروح
\par 7 جبرائيل، أحد الملائكة القديسين، الذي هو فوق الفردوس والأفاعي والكاروبيم
\par 8 رميئيل، أحد الملائكة القديسين، الذين أقامهم الله على القائمين.

\chapter{21}

\par 1 وتوجهت إلى حيث كانت الأمور فوضوية.
\par 2 ورأيت هناك شيئًا فظيعًا: لم أرَ سماءً في الأعلى ولا أرضًا راسخة، بل مكانًا فوضويًا ومرعبًا.
\par 3 وهناك رأيت سبع نجوم من السماء مرتبطة ببعضها مثل جبال عظيمة ومتقدة بالنار.
\par 4 فقلت: "بأي خطيئة هم مقيدون، ولماذا ألقوا هنا؟"
\par 5 "ثم قال أورييل، أحد الملائكة القديسين، الذي كان معي، وكان رئيسًا عليهم، وقال: ""يا حنوك، لماذا تسأل، ولماذا أنت حريص على الحقيقة؟
\par 6 هؤلاء هم من عدد نجوم السماء، الذين خالفوا أمر الرب، وهم مقيدين هنا حتى تكتمل عشرة آلاف سنة، وهي المدة التي استلزمتها خطاياهم
\par 7 ومن هناك ذهبت إلى مكان آخر، كان أكثر فظاعة من السابق، ورأيت شيئًا مروعًا: نار عظيمة هناك اشتعلت واشتعلت، وكان المكان مشقوقًا حتى الهاوية، ممتلئًا بأعمدة نار عظيمة متساقطة: لم أستطع رؤية مداها أو حجمها، ولا يمكنني التخمين
\par 8 ثم قلت: "ما أشد رعب المكان وما أشد رعب المنظر!"
\par 9 ثم أجابني أورييل، أحد الملائكة القديسين الذي كان معي، وقال لي: "يا أخنوخ، لماذا تشعر بكل هذا الخوف والرعب؟" فأجبت: "بسبب هذا المكان المخيف، وبسبب مشهد الألم."
\par 10 وقال لي: "هذا المكان سجن الملائكة، وهنا سيُسجنون إلى الأبد."

\chapter{22}

\par 1 ومن ثم ذهبت إلى مكان آخر، وهو جبل من الصخور الصلبة
\par 2 وكان فيه أربعة أماكن جوف، عميقة وواسعة وناعمة جدًا. "ما أكثر نعومة هذه الأماكن الجوفاء وعمقها ومظلمتها للنظر."
\par 3 ثم أجاب رافائيل، أحد الملائكة القديسين الذي كان معي، وقال لي: "لقد خُلقت هذه الأماكن المجوفة لهذا الغرض بالذات، لكي تتجمع فيها أرواح الموتى، نعم لكي تتجمع هنا جميع أرواح أبناء البشر."
\par 4 «وجعلت هذه الأماكن لاستقبالهم إلى يوم دينونتهم وإلى أجلهم المسمى، إلى أن يأتي عليهم الدينونة الكبرى».
\par 5 رأيت (روح) رجل ميت يصنع بدلة، وصوته خرج إلى السماء وصنع بدلة
\par 6 فسألت رافائيل الملاك الذي كان معي وقلت له: "هذه الروح التي تناسب من هي، التي يخرج صوتها وتتناسب مع السماء"
\par 7 فأجابني قائلاً: «هذه هي الروح التي خرجت من هابيل الذي قتله قابيل أخوه، وهو يرفع دعواه عليه حتى يباد نسله عن وجه الأرض، ويباد نسله من بين نسل البشر».
\par 8 فسألت عنه وعن جميع الفراغات: لماذا انفصل أحدهما عن الآخر؟
\par 9 فأجابني وقال لي: «هذه الثلاثة خُلقت لفصل أرواح الموتى. ومثل هذا الفصل خُلقت لأرواح الأبرار، حيث ينبوع الماء المضيء».
\par 10 "ومثل ذلك قد صنع للخطاة حين ماتوا ودُفنوا في الأرض ولم يُنفذ عليهم الحكم في حياتهم."
\par 11 هنا ستُعزل أرواحهم في هذا العذاب العظيم إلى يوم الدينونة العظيم، وعقاب وعذاب من يلعنون إلى الأبد، وجزاءً لأرواحهم. هناك سيربطهم إلى الأبد.
\par 12 «وقد وُضع مثل هذا التقسيم لأرواح أولئك الذين يرفعون دعوى قضائية، والذين يكشفون عن هلاكهم، عندما قُتلوا في أيام الخطاة.»
\par 13 «لقد خُلِقَ هذا لأرواح البشر الذين لم يكونوا أبرارًا بل خطاة، الذين كانوا كاملين في المعصية، وسيكونون رفقاء للمذنبين: لكن أرواحهم لن تُقتل في يوم الدينونة ولن تُبعث من هناك.»
\par 14 ثم باركت رب المجد وقلت: "تبارك ربي، رب البر، الحاكم إلى الأبد."

\chapter{23}

\par 1 ومن هناك ذهبت إلى مكان آخر غرب أقاصي الأرض
\par 2 ورأيت نارًا متقدة لا تهدأ، ولا تتوقف عن مجراها ليلًا أو نهارًا، بل (تستمر) بانتظام
\par 3 فسألتُ قائلًا: "ما هذا الذي لا يستقر؟"
\par 4 "ثم أجابني راجويل، أحد الملائكة القديسين الذي كان معي، وقال لي: "إن مسار النار هذا الذي رأيته هو النار في الغرب التي تضطهد جميع نجوم السماء".

\chapter{24}

\par 1 ومن هناك ذهبت إلى مكان آخر من الأرض، وأراني سلسلة جبال من النار تحترق ليلًا ونهارًا
\par 2 "ثم ذهبت إلى ما وراءها فرأيت سبعة جبال عظيمة، كل منها يختلف عن الآخر، وكانت أحجارها عظيمة وجميلة، عظيمة في مجموعها، ذات مظهر رائع وخارج جميل: ثلاثة نحو الشرق، واحد قائم على الآخر، وثلاثة نحو الجنوب، واحد على الآخر، ووديان عميقة وعرة، لا يلتقي أحد منها مع الآخر.
\par 3 وكان الجبل السابع في وسط هذه، وكان يفوقها طولاً، يشبه كرسي العرش، وأشجار عطرية محيطة بالعرش.
\par 4 وكان من بينها شجرة لم أشم مثلها قط، لم يكن بينها ولا مثلها، لها رائحة لا تزول، وورقها وزهرها وخشبها لا يذبل إلى الأبد، وثمرها جميل، وثمرها يشبه التمر.
\par 5 ثم قلت: "ما أجمل هذه الشجرة، وعبيرها زكي، وأوراقها جميلة، وأزهارها بديعة المنظر."
\par 6 فأجاب ميخائيل، أحد الملائكة القديسين والمكرمين الذي كان معي، وكان قائدهم

\chapter{25}

\par 1 فقال لي: "يا أخنوخ، لماذا تسألني عن رائحة الشجرة، ولماذا تريد أن تعرف الحقيقة؟"
\par 2 فأجبته قائلاً: "أريد أن أعرف كل شيء، وخاصة عن هذه الشجرة".
\par 3 فأجاب قائلا: «هذا الجبل العالي الذي رأيته، الذي قمته مثل عرش الله، هو عرشه، حيث يجلس القدوس العظيم، رب المجد، الملك الأبدي، عندما ينزل لزيارة الأرض بالخير».
\par 4 وأما هذه الشجرة العطرة، فلا يجوز لبشر أن يمسها حتى يوم القيامة، حين ينتقم من الجميع ويُهلك كل شيء إلى الأبد. ثم تُعطى للأبرار والقديسين.
\par 5 «يكون ثمرها طعامًا للمختارين: تُزرع في المكان المقدس، في هيكل الرب، الملك الأبدي.»
\par 6 «حينئذٍ يبتهجون ويفرحون، ويدخلون القدس، ويكون رائحته في عظامهم، ويعيشون حياة طويلة على الأرض كما عاش آباؤك، ولا يمسهم في أيامهم حزن ولا وباء ولا عذاب ولا مصيبة.»
\par 7 ثم تباركت أنا إله المجد، الملك الأبدي، الذي أعد مثل هذه الأشياء للأبرار، وخلقها، ووعد بإعطائهم إياها

\chapter{26}

\par 1 ثم انطلقت من هناك إلى وسط الأرض، فرأيت مكانًا مباركًا فيه أشجار ذات أغصان باقية مزهرة [لشجرة مقطوعة الأوصال].
\par 2 وهناك رأيت جبلًا مقدسًا، وتحت الجبل إلى الشرق كان هناك جدول يتدفق نحو الجنوب
\par 3 ورأيت نحو الشرق جبلًا آخر أعلى من هذا، وبينهما وادٍ عميق وضيق، وكان يجري فيه أيضًا جدول من أسفل الجبل
\par 4 وإلى الغرب منه كان هناك جبل آخر، أدنى من الجبل السابق، قليل الارتفاع، ووادٍ عميق وجاف بينهما: وكان هناك وادٍ آخر عميق وجاف عند أطراف الجبال الثلاثة
\par 5 وكانت جميع الأودية عميقة وضيقة، (مكونة) من صخور صلبة، ولم تكن الأشجار مزروعة عليها.
\par 6 وتعجبت من الصخور، وتعجبت من الوادي، نعم، تعجبت كثيرا.

\chapter{27}

\par 1 ثم قلت: "ما غرض هذه الأرض المباركة، المليئة بالأشجار، وهذا الوادي الملعون بينهما"
\par 2 فأجابني أورييل، أحد الملائكة القديسين الذين كانوا معي، وقال: «هذا الوادي الملعون للملعونين إلى الأبد: هنا يُجمع كل الملعونين الذين ينطقون بألفاظ بذيئة ضد الرب، ويتكلمون بكلمات قاسية عن مجده. هنا يُجمعون، وهنا يكون مقام دينونتهم».
\par 3 "وفي الأيام الأخيرة سيكون عليهم مشهد الدينونة العادلة أمام الأبرار إلى الأبد. هنا يبارك الرحماء رب المجد الملك الأبدي."
\par 4 "وفي أيام الدينونة على أولئك يباركونه على الرحمة التي قسمها لهم."
\par 5 ثم باركت رب المجد وبينت مجده وسبحته مجيدا.

\chapter{28}

\par 1 ومن ثم اتجهت نحو الشرق، إلى وسط سلسلة جبال الصحراء، ورأيت برية، وكانت منعزلة،
\par 2 مليئة بالأشجار والنباتات.
\par 3 فانفجرت المياه من فوق. اندفعت كجدول غزير نحو الشمال الغربي، فتسببت في صعود السحب والندى من كل جانب.

\chapter{29}

\par 1 ومن ثم ذهبت إلى مكان آخر في الصحراء، واقتربت من شرق هذه السلسلة الجبلية.
\par 2 وهناك رأيت أشجارًا عطرية تفوح منها رائحة اللبان والمر، وكانت الأشجار أيضًا تشبه شجرة اللوز

\chapter{30}

\par 1 وخلف هذه، ذهبتُ بعيدًا إلى الشرق، ورأيتُ مكانًا آخر، واديًا مليئًا بالماء
\par 2 وكان فيها شجرة لونها () مثل لون الشجر الطيب كالمستكا.
\par 3 وعلى جوانب تلك الوديان رأيتُ رائحة قرفة عطرة. وخلفها اتجهتُ شرقًا

\chapter{31}

\par 1 ورأيت جبالًا أخرى، ومن بينها بساتين أشجار، وكان يتدفق منها رحيق يُسمى السرارة والجلبانوم
\par 2 وخلف هذه الجبال رأيت جبلاً آخر إلى شرقي أقاصي الأرض، عليه أشجار الصبر، وكانت جميع الأشجار مليئة ببراعم تشبه أشجار اللوز.
\par 3 وعندما أحرقته، كان رائحته أحلى من أي رائحة عطرية.

\chapter{32}

\par 1 وبعد هذه الروائح العطرة، بينما كنت أنظر نحو الشمال فوق الجبال، رأيت سبعة جبال مليئة بالنارد المختار والأشجار العطرة والقرفة والفلفل
\par 2 ومن ثم ذهبت فوق قمم كل هذه الجبال، بعيدًا نحو شرق الأرض، وعبرت فوق البحر الإريتري وذهبت بعيدًا عنه، وعبرت الملاك زوتييل.
\par 3 "وجئت إلى جنة البر، ومن بعيد أشجار أكثر عدداً مني، هذه الأشجار عظيمة - شجرتان هناك، عظيمة جداً وجميلة ومجيدة ورائعة، وشجرة المعرفة التي يأكلون من ثمرها المقدس ويعرفون حكمة عظيمة."
\par 4 تلك الشجرة طولها كشجرة التنوب، وأوراقها كأوراق شجرة الخروب، وثمرها كعناقيد الكرمة، جميلة جدًا، ورائحة الشجرة تفوح من بعيد
\par 5 ثم قلت: "ما أجمل الشجرة، وما أروع منظرها!"
\par 6 فأجابني رافائيل الملاك المقدس، الذي كان معي، وقال: "هذه هي شجرة الحكمة التي أكل منها أبوك الشيخ وأمك المسنة اللذان كانا قبلك، وتعلما الحكمة وانفتحت أعينهما، وعلما أنهما عريانان، فطُردا من الجنة."

\chapter{33}

\par 1 ومن هناك ذهبتُ إلى أقاصي الأرض، فرأيتُ هناك وحوشًا عظيمة، كلٌّ منها يختلف عن الآخر؛ ورأيتُ طيورًا أيضًا تختلف في المظهر والجمال والصوت، كلٌّ منها يختلف عن الآخر
\par 2 وإلى الشرق من تلك الحيوانات رأيت أطراف الأرض حيث تستقر السماء، وأبواب السماء مفتوحة.
\par 3 "ورأيت كيف تخرج نجوم السماء، وأحصيت الأبواب التي تخرج منها، وكتبت جميع منافذها، لكل نجم على حدة، حسب عددها وأسمائها، ومساراتها ومواقعها، وأوقاتها وأشهرها، كما أراني أورييل الملاك المقدس الذي كان معي.
\par 4 وأراني كل شيء وكتبه لي، وكتب لي أسماءهم وأحكامهم وشركاتهم.

\chapter{34}

\par 1 ومن هناك ذهبتُ نحو الشمال إلى أقاصي الأرض، وهناك رأيتُ تدبيرًا عظيمًا ومجيدًا في أقاصي الأرض كلها
\par 2 وهنا رأيت ثلاثة أبواب للسماء مفتوحة في السماء: من كل منها تمر رياح الشمال: عندما تهب يكون هناك برد، وبَرَد، وصقيع، وثلج، وندى، ومطر
\par 3 وينفخون من بوابة واحدة للخير: ولكن عندما ينفخون من البوابتين الأخريين، يكون ذلك بعنف وبؤس على الأرض، وينفخون بعنف

\chapter{35}

\par 1 ومن هناك اتجهت نحو الغرب إلى أقاصي الأرض، ورأيت هناك ثلاثة أبواب للسماء مفتوحة كما رأيتها في الشرق، نفس عدد الأبواب، ونفس عدد المنافذ

\chapter{36}

\par 1 ومن هناك ذهبتُ إلى الجنوب إلى أقاصي الأرض، ورأيتُ هناك ثلاثة أبواب مفتوحة للسماء: ومن هناك يأتي الندى والمطر والريح
\par 2 ومن هناك ذهبت إلى الشرق إلى أقاصي السماء، ورأيت هنا أبواب السماء الشرقية الثلاثة مفتوحة وأبوابًا صغيرة فوقها.
\par 3 ومن خلال كل من هذه البوابات الصغيرة تمر نجوم السماء وتسير في طريقها نحو الغرب على المسار الذي يظهر لها.
\par 4 وكلما رأيتُ باركتُ دائمًا رب المجد، وواصلتُ مباركة رب المجد الذي صنع عجائب عظيمة ومجيدة، لإظهار عظمة عمله للملائكة والأرواح والبشر، لكي يُسبّحوا عمله وكل خليقته: لكي يروا عمل قدرته ويُسبّحوا عمل يديه العظيم ويباركوه إلى الأبد

\part{القسم الثاني. الفصول من السابع والثلاثين إلى الحادي والسبعين. الأمثال}

\chapter{37}

\par 1 الرؤيا الثانية التي رآها، رؤيا الحكمة - التي رآها حنوك بن يارد، بن مهللئيل، بن قينان، بن أنوش، بن شيث، بن آدم
\par 2 وهذه هي بداية كلمات الحكمة التي رفعت صوتي لأتكلم بها وأقولها لساكني الأرض:
\par 3 اسمعوا يا رجال الأزمنة القديمة، وانظروا يا من يأتون بعدي، كلام القدوس الذي سأنطق به أمام رب الأرواح. كان من الأفضل أن نعلنه لرجال الأزمنة القديمة فقط، ولكن حتى من يأتون بعدي لن نحجب عنهم بداية الحكمة.
\par 4 "حتى يومنا هذا لم يمنحني رب الأرواح مثل هذه الحكمة التي تلقيتها وفقًا لبصيرتي، وفقًا لسر رب الأرواح الذي منحني نصيب الحياة الأبدية."
\par 5 «ثم أُعطيت لي ثلاثة أمثال، ورفعت صوتي وقصتها على الساكنين على الأرض.»

\chapter{38}
\par 1 المثل الأول: عندما تظهر جماعة الصالحين، ويُدان الخطاة على خطاياهم، ويُطردون من على وجه الأرض:
\par 2 وعندما يظهر البار أمام أعين الأبرار، الذي تُعلّق أعماله المُختارة على رب الأرواح، ويظهر النور للأبرار والمختارين الساكنين على الأرض، فأين سيكون مسكن الخطاة، وأين سيكون مسكن الذين أنكروا رب الأرواح؟ كان خيرًا لهم لو لم يولدوا.
\par 3 عندما تُكشف أسرار الصالحين ويُدان الخطاة، ويُطرد الكافرون من أمام الصالحين والمختارين،
\par 4 من ذلك الوقت، لن يكون أولئك الذين يمتلكون الأرض أقوياء ومرتفعين بعد الآن: ولن يتمكنوا من رؤية وجه القديسين، لأن رب الأرواح قد جعل نوره يظهر على وجه القديسين والأبرار والمختارين
\par 5 حينئذٍ يهلك الملوك والأقوياء ويُسلمون إلى أيدي الأبرار والقديسين
\par 6 ومن الآن فصاعدًا، لن يطلب أحد الرحمة من رب الأرواح، لأن حياتهم قد شارفت على الانتهاء

\chapter{39}

\par 1 [ويحدث في تلك الأيام أن أبناءً مختارين ومقدسين سينزلون من السماء العليا، وسيصبح نسلهم واحدًا مع أبناء البشر
\par 2 وفي تلك الأيام، تلقى أخنوخ كتب الغيرة والغضب، وكتب الاضطراب والطرد.] ولن تُمنح لهم الرحمة، يقول رب الأرواح
\par 3 وفي تلك الأيام حملتني عاصفة عن الأرض وأنزلتني في أقصى السماوات.
\par 4 وهناك رأيت رؤيا أخرى مساكن القديسين ومساكن الصديقين.
\par 5 هنا رأت عيناي مساكنهم مع ملائكته الأبرار، ومساكنهم مع القديسين. توسّلوا وتشفّعوا وصلّوا لأجل بني البشر، ففاض البرّ أمامهم كالماء، والرحمة كالندى على الأرض: هكذا هم إلى الأبد.
\par 6 [أ] وفي ذلك المكان رأت عينيّ مختار البر والإيمان، [ب] وسيسود البر في أيامه، وسيكون الأبرار والمختارون بلا عدد أمامه إلى أبد الآبدين.
\par 7 [أ] ورأيت مسكنه تحت جناحي رب الأرواح. [ب] ويكون جميع الأبرار والمختارين أمامه أقوياء كأنوار نارية، وتكون أفواههم مليئة بالبركة، وشفاههم تسبح اسم رب الأرواح، والبر أمامه لا يزول أبدًا، [والاستقامة لا تزول أمامه أبدًا.]
\par 8 هناك تمنيتُ السكنى، واشتاقت روحي إلى ذلك المسكن: وهناك كان نصيبي حتى الآن، لأنه هكذا حُدد لي أمام رب الأرواح
\par 9 في تلك الأيام، سبّحتُ وسبّحتُ اسم رب الأرواح بالبركات والتسبيحات، لأنه قدّر لي البركة والمجد حسب مسرة رب الأرواح
\par 10 لفترة طويلة نظرت عيني إلى ذلك المكان، وباركته وسبحته قائلاً: "تبارك هو، وليكن مباركًا من البدء وإلى الأبد".
\par 11 «وأمّا هو فلا انقضاء أمامه. وهو يعلم قبل خلق العالم ما هو للأبد وما سيكون من جيل إلى جيل.»
\par 12 «الذين لا ينامون يباركونك: يقفون أمام مجدك ويباركون ويسبحون ويسبحون قائلين: قدوس، قدوس، قدوس، رب الأرواح: يملأ الأرض بالأرواح.»
\par 13 وهنا رأت عينيّ كل الذين لا ينامون: يقفون أمامه ويباركون ويقولون: "مبارك أنت، ومبارك اسم الرب من الأبد."
\par 14 وتغير وجهي، لأني لم أعد أستطيع النظر.

\chapter{40}

\par 1 وبعد ذلك رأيت آلاف الآلاف وعشرة آلاف مرة عشرة آلاف، رأيت جمعًا لا يُحصى ولا يُحصى، وقفوا أمام رب الأرواح
\par 2 وعلى الجوانب الأربعة لرب الأرواح، رأيت أربعة وجودات، مختلفة عن تلك التي لا تنام، وتعلمت أسماءها: لأن الملاك الذي ذهب معي عرّفني على أسماءها، وأراني كل الأشياء المخفية
\par 3 وسمعت أصوات تلك الوجودات الأربعة وهم ينطقون بالتسبيح أمام رب المجد
\par 4 الصوت الأول يبارك رب الأرواح إلى الأبد.
\par 5 والصوت الثاني سمعته يبارك المختار والمختارين المعلقين على رب الأرواح.
\par 6 والصوت الثالث الذي سمعته يصلي ويشفع لأولئك الذين يسكنون على الأرض ويتوسلون باسم رب الأرواح
\par 7 وسمعت الصوت الرابع يصد الشياطين ويمنعهم من المجيء أمام رب الأرواح لاتهام سكان الأرض
\par 8 بعد ذلك سألتُ ملاك السلام الذي رافقني، والذي أراني كل ما هو مخفي: "من هم هؤلاء الأربعة الذين رأيتهم، والذين سمعت كلماتهم وكتبتها."
\par 9 فقال لي: "هذا الأول هو ميخائيل، الرحيم وطويل الأناة. والثاني، المُكلَّف بجميع أمراض وجروح بني البشر، هو رافائيل. والثالث، المُكلَّف بجميع القوات، هو جبرائيل. والرابع، المُكلَّف بتوبة رجاء الذين يرثون الحياة الأبدية، اسمه فانوئيل."
\par 10 وهؤلاء هم ملائكة رب الأرواح الأربعة والأصوات الأربعة التي سمعتها في تلك الأيام

\chapter{41}

\par 1 وبعد ذلك رأيتُ جميع أسرار السماوات، وكيف يُقسَّم الملكوت، وكيف تُوزن أفعال البشر في الميزان
\par 2 وهناك رأيت منازل المختارين ومنازل القديسين، ورأت عيني هناك كل الخطاة مطرودين من هناك، الذين ينكرون اسم رب الأرواح، ويجرون بعيداً: ولم يستطيعوا أن يصمدوا بسبب العقاب الذي يأتي من رب الأرواح.
\par 3 وهناك رأت عيني أسرار البرق والرعد، وأسرار الرياح، وكيف انقسمت لتهب على الأرض، وأسرار السحب والندى، وهناك رأيت من أين تأتي في ذلك المكان ومن أين تشبع الأرض الترابية.
\par 4 وهناك رأيتُ غرفًا مغلقةً تنقسم إليها الرياح، غرفة البَرَد والرياح، وغرفة الضباب والسحب، وسحابته تحوم فوق الأرض منذ بداية العالم
\par 5 ورأيت حجرات الشمس والقمر، من أين ينطلقان وإلى أين يعودان، وعودتهما المجيدة، وكيف أن أحدهما أسمى من الآخر، ومدارهما الجليل، وكيف لا يغادران مدارهما، ولا يضيفان شيئًا إلى مدارهما ولا ينقصان منه شيئًا، ويحافظان على العهد فيما بينهما، وفقًا للقسم الذي يرتبطان به معًا
\par 6 وأولاً تشرق الشمس وتسير في طريقها وفقًا لأمر رب الأرواح، ويكون اسمه عظيمًا إلى الأبد
\par 7 وبعد ذلك رأيتُ مسار القمر الخفي والمرئي، وهو يُكمل مساره في ذلك المكان نهارًا وليلًا - أحدهما يتخذ موقعًا مُقابلًا للآخر أمام رب الأرواح. وهم يشكرون ويسبحون ولا يهدأون؛ لأن لهم راحة الشكر
\par 8 لأن الشمس تتغير كثيرًا إما إلى نعمة أو نقمة، ومسير القمر نور للأبرار وظلمة للخطاة باسم الرب، الذي فصل بين النور والظلمة، وفرق أرواح البشر، وقوّى أرواح الأبرار باسم بره
\par 9 لأنه لا ملاك يعيق ولا قوة تستطيع أن تعيق، لأنه يُعيّن قاضيًا للجميع، وهو يحكم عليهم جميعًا أمامه

\chapter{42}

\par 1 لم تجد الحكمة مكانًا تسكن فيه، فعُيّن لها مسكن في السماوات
\par 2 خرجت الحكمة لتجعل مسكنها بين أبناء البشر، ولم تجد مسكنًا. عادت الحكمة إلى مكانها، وجلست بين الملائكة
\par 3 فخرج الإثم من مخادعها. من لم تطلبه وجدته وسكنت معهم، كالمطر في البرية والندى على أرض عطشى

\chapter{43}

\par 1 ورأيت بروقًا أخرى ونجومًا في السماء، ورأيت كيف دعاها كلها بأسمائها، فأصغت إليه
\par 2 ورأيت كيف يوزنون في الميزان حسب نورهم وعرض أفلاكهم ويوم ظهورهم ودورانهم ينجم عنه البرق ودورانهم حسب عدد الملائكة وهم يوفون ببعضهم.
\par 3 فسألت الملاك الذي كان معي والذي أراني ما كان مخفيًا: "ما هؤلاء؟"
\par 4 وقال لي: "لقد أظهر لك رب الأرواح معناها الرمزي (حرفيًا مثلهم): هذه هي أسماء القديسين الذين يسكنون الأرض ويؤمنون باسم رب الأرواح إلى الأبد".

\chapter{44}

\par 1 رأيتُ أيضًا ظاهرة أخرى تتعلق بالبروق: كيف تنشأ بعض النجوم وتصبح بروقًا ولا تستطيع الانفصال عن شكلها الجديد

\chapter{45}

\par 1 وهذا هو المثل الثاني عن أولئك الذين ينكرون اسم مسكن القديسين ورب الأرواح
\par 2 وإلى السماء لن يصعدوا، وعلى الأرض لن يأتوا: هكذا سيكون نصيب الخطاة الذين أنكروا اسم رب الأرواح، والذين يُحفظون هكذا ليوم المعاناة والضيق
\par 3 في ذلك اليوم، سيجلس مختاري على عرش المجد، وسيختبر أعمالهم، وستكون أماكن راحتهم لا تُحصى. وستقوى أرواحهم في داخلهم عندما يرون مختاري، وأولئك الذين دعوا باسمي المجيد:
\par 4 ثم سأُسكن مختاري بينهم. وسأُغير شكل السماء وأجعلها نعمة ونورًا أبديًا
\par 5 وأُغيّر الأرض وأجعلها بركة، وأُسكن عليها مختاريّ، أما الخطاة والأشرار فلا تطأها أقدامهم
\par 6 لأني قد هيأت وأشبعت أبريائي بالسلام وأسكنتهم أمامي. أما الخطاة فلهم دينونة قريبة مني، لأبيدهم عن وجه الأرض

\chapter{46}

\par 1 وهناك رأيتُ من كان له "رأس أيام"، وكان رأسه أبيض كالصوف، ومعه كائن آخر كان وجهه يشبه وجه إنسان، وكان وجهه مليئًا بالنعمة، كأحد الملائكة القديسين
\par 2 فسألت الملاك الذي ذهب معي وأراني كل الأمور الخفية المتعلقة بابن الإنسان، من هو، ومن أين هو، ولماذا ذهب مع "رأس الأيام".
\par 3 فأجاب وقال لي: هذا هو ابن الإنسان الذي له البر، والذي يسكن البر، والذي يكشف كل كنوز ما هو مخفي، لأن رب الأرواح اختاره، والذي له نصيبه الأولوية أمام رب الأرواح في الاستقامة إلى الأبد.
\par 4 "وهذا ابن الإنسان الذي رأيته يقيم الملوك والأقوياء عن كراسيهم، [والأقوياء عن عروشهم]، ويحل قيود الأقوياء، ويكسر أسنان الخطاة."
\par 5 "[ويضع الملوك عن كراسيهم وممالكهم] لأنهم لا يسبحونه ولا يحمدونه، ولا يعترفون بتواضع من أين أعطي لهم الملك."
\par 6 "ويُحطِّم وجه الأقوياء، ويملأهم خزيًا. ويكون الظلام مسكنهم، والدود فراشهم، ولا أمل لهم في النهوض من مضاجعهم، لأنهم لا يُسبِّحون اسم رب الأرواح."
\par 7 وهؤلاء هم الذين يحكمون على نجوم السماء، ويرفعون أيديهم على العلي، ويدوسون على الأرض ويسكنونها. وكل أعمالهم تُظهِر إثمًا، وسلطانهم على ثرواتهم، وإيمانهم بالآلهة التي صنعوها بأيديهم، وينكرون اسم رب الأرواح.
\par 8 «ويضطهدون بيوت كنائسه، والمؤمنين المعلقين باسم رب الأرواح.»

\chapter{47}

\par 1 وفي تلك الأيام تصعد صلاة الأبرار، ودم الأبرار من الأرض أمام رب الأرواح
\par 2 في تلك الأيام، سيتحد القديسون الساكنون في السماء بصوت واحد، ويتوسلون ويصلون ويسبحون، ويشكرون ويباركون اسم رب الأرواح من أجل دم الصديق الذي سُفك، وحتى لا تذهب صلاة الصديقين سدى أمام رب الأرواح، وحتى يُحكم عليهم، ولا يضطرون إلى المعاناة إلى الأبد
\par 3 في تلك الأيام رأيت "رأس الأيام" وهو جالس على عرش مجده، وانفتحت أمامه أسفار الأحياء، ووقف أمامه كل جنده الذي في السماء من فوق، ومشيروه،
\par 4 وامتلأت قلوب القديسين فرحًا؛ لأن عدد الأبرار قد قُدِّم، وصلاة الأبرار استُجيبت، وطُلب دم الأبرار أمام رب الأرواح

\chapter{48}

\par 1 وفي ذلك المكان رأيت ينبوع البر الذي لا ينضب، وحوله ينابيع حكمة كثيرة، فشرب منها كل العطاش، وامتلأوا حكمة، وكانت مساكنهم مع الأبرار والقديسين والمختارين
\par 2 وفي تلك الساعة سمي ابن الإنسان أمام رب الأرواح، واسمه أمام رأس الأيام.
\par 3 نعم، قبل أن تُخلق الشمس والآيات، قبل أن تُصنع نجوم السماء، سُمِّي اسمه أمام رب الأرواح.
\par 4 فيكون للصديق عصا يستندون عليها فلا يعثرون، ويكون نوراً للأمم ورجاءً لمن تعوزهم القلوب.
\par 5 كل من يسكن على الأرض سيخر ويسجد أمامه، وسيسبح ويبارك ويحتفل بالترانيم لرب الأرواح
\par 6 ولهذا السبب اختير وأُخفي أمامه، قبل إنشاء العالم وإلى الأبد
\par 7 وقد كشفت حكمة رب الأرواح عن ذلك للقديسين والأبرار؛ لأنه حفظ نصيب الأبرار، لأنهم كرهوا واحتقروا هذا العالم الظالم، وكرهوا كل أعماله وطرقه باسم رب الأرواح: لأنهم باسمه يخلصون، وحسب مسرته يكون ذلك في حياتهم
\par 8 في هذه الأيام يصير ملوك الأرض والأقوياء الذين يملكون الأرض منكسري الوجوه بسبب أعمال أيديهم، لأنه في يوم ضيقهم وبؤسهم لا يستطيعون أن يخلصوا أنفسهم.
\par 9 وسأسلمهم إلى أيدي مختاريّ: كالقش في النار، هكذا يحترقون أمام وجه القديسين، كالرصاص في الماء، يغرقون أمام وجه الصديقين، ولن يوجد لهم أثر بعد
\par 10 وفي يوم بلائهم تكون راحة على الأرض، وأمامهم يسقطون ولا يقومون: ولا يكون من يأخذهم بيديه ويقيمهم: لأنهم أنكروا رب الأرواح ومسيحه. فليتبارك اسم رب الأرواح

\chapter{49}

\par 1 لأن الحكمة تُسكب كالماء، والمجد لا يزول أمامه إلى الأبد
\par 2 لأنه قدير في كل أسرار البر، والإثم سيختفي كالظل، ولا يكون له استمرار؛ لأن المختار يقف أمام رب الأرواح، ومجده إلى أبد الآبدين، وقدرته إلى كل الأجيال
\par 3 وفيه يسكن روح الحكمة، وروح الفهم، وروح الفهم والقوة، وروح الراقدين في البر
\par 4 ويحكم في الأمور الخفية، ولا يستطيع أحد أن ينطق بكلمة كذب أمامه؛ لأنه هو المختار أمام رب الأرواح حسب مسرته

\chapter{50}

\par 1 وفي تلك الأيام سيحدث تغيير للقديسين والمختارين، وسيُضيء عليهم نور الأيام، ويتحول المجد والكرامة للقديسين،
\par 2 في يوم الضيق الذي يُخزَّن فيه الشرور على الخطاة، وينتصر الأبرار باسم رب الأرواح، ويُشهد الآخرين على ذلك حتى يتوبوا ويتركوا أعمال أيديهم.
\par 3 لا يكون لهم كرامة باسم رب الأرواح، ولكن باسمه يخلصون، ورب الأرواح يرحمهم لأن رحمته عظيمة.
\par 4 وهو بار أيضاً في دينونته، وفي حضرة مجده لا يبقى الظلم. في دينونته يهلك أمامه غير التائبين.
\par 5 ومن الآن فصاعدًا لن أرحمهم، يقول رب الأرواح

\chapter{51}

\par 1 وفي تلك الأيام، ستعيد الأرض أيضًا ما أُودِعَت إليه، وستُعيد الهاوية أيضًا ما أخذته، وستُعيد الجحيم ما عليها
\par 2 ويختار من بينهم الصديقين والقديسين لأنه قد اقترب يوم خلاصهم.
\par 3 "ويجلس المختار في تلك الأيام على عرشي، ويفيض فمه بجميع أسرار الحكمة والمشورة، لأن رب الأرواح أعطاها له ومجده."
\par 4 وفي تلك الأيام تقفز الجبال مثل الكباش، وتقفز التلال مثل الحملان التي تشبع من اللبن، وتضيء وجوه ملائكة السماء فرحاً.
\par 5 [أ] لأنه في تلك الأيام سيقوم المختار، [ب] وتبتهج الأرض، [ج] ويسكنها الأبرار، [د] ويمشي عليها المختارون

\chapter{52}

\par 1 وبعد تلك الأيام في ذلك المكان الذي رأيت فيه كل رؤى ما هو مخفي - فقد خُطفت في عاصفة وحملتني نحو الغرب -،
\par 2 هناك رأت عيني كل أسرار السماء التي ستكون، جبل من حديد، وجبل من نحاس، وجبل من فضة، وجبل من ذهب، وجبل من معدن لين، وجبل من رصاص
\par 3 فسألتُ الملاك الذي كان يرافقني قائلًا: "ما هذه الأشياء التي رأيتها سرًا؟"
\par 4 وقال لي: "كل هذه الأشياء التي رأيتها ستخدم سلطان مسيحه ليكون مقتدرًا وقويًا على الأرض."
\par 5 فأجابني ملاك السلام قائلاً: "انتظر قليلاً، وسوف ينكشف لك كل الأمور السرية التي تحيط برب الأرواح".
\par 6 «وهذه الجبال التي رأتها عيناك، جبل الحديد، وجبل النحاس، وجبل الفضة، وجبل الذهب، وجبل المعدن اللين، وجبل الرصاص، كل هذه ستكون في حضرة المختار كالشمع: أمام النار، وكالماء الذي يتدفق من فوق [على تلك الجبال]، وستصبح عاجزة أمام قدميه.»
\par 7 «وسيحدث في تلك الأيام أنه لن ينجو أحد، لا بالذهب ولا بالفضة، ولن يتمكن أحد من النجاة.»
\par 8 «ولن يكون هناك حديد للحرب، ولا يلبس أحد درعًا. لن يكون البرونز مفيدًا، ولن يُقدَّر القصدير، ولن يكون الرصاص مرغوبًا فيه.»
\par 9 «وسيتم [إنكار] وتدمير كل هذه الأشياء من على سطح الأرض، عندما يظهر المختار أمام وجه رب الأرواح.»

\chapter{53}

\par 1 هناك رأت عيني واديًا عميقًا مفتوح الأفواه، وسيقدم له كل من يسكن الأرض والبحر والجزر هدايا وعطايا ورموز احترام، لكن ذلك الوادي العميق لن يمتلئ
\par 2 "وأيديهم تعمل أعمالاً غير قانونية، والخطاة يلتهمون كل من يظلمونه بغير قانون: ولكن الخطاة سوف يُهلكوا أمام وجه رب الأرواح، وسوف يُنفون عن وجه أرضه، وسوف يهلكون إلى الأبد."
\par 3 لأني رأيت كل ملائكة العقاب واقفين هناك ويجهزون كل أدوات الشيطان.
\par 4 وسألت ملاك السلام الذي ذهب معي: "لمن يُعِدّون هذه الآلات؟"
\par 5 فقال لي: إنهم يعدون هذه لملوك وأقوياء هذه الأرض لكي يهلكوا بها.
\par 6 "وبعد هذا يجعل البار والمختار بيت جماعته يظهر: ومن الآن فصاعدًا لن يكون هناك أي عائق أمامهم باسم رب الأرواح."
\par 7 «ولن تثبت هذه الجبال كالأرض أمام بره، بل تكون قمم التلال كنبع ماء، ويحصل الصديق على راحة من ظلم الخطاة.»

\chapter{54}

\par 1 ثم نظرت والتفت إلى جزء آخر من الأرض، فرأيت هناك واديًا عميقًا ونارًا مشتعلة
\par 2 وأحضروا الملوك والأقوياء، وبدأوا في طرحهم في هذا الوادي العميق
\par 3 وهناك رأت عيني كيف صنعوا هذه أدواتهم، سلاسل حديدية ذات وزن لا يقاس
\par 4 فسألتُ ملاك السلام الذي كان معي قائلًا: "لمن تُعدّ هذه السلاسل؟"
\par 5 فقال لي: "هذه تُعدّ لجيوش عزازيل، ليأخذوها ويلقوها في هاوية الهلاك الكامل، ويغطون أفواههم بحجارة خشنة كما أمر رب الأرواح."
\par 6 «ويُمسك بهم ميخائيل وجبرائيل ورافائيل وفانوئيل في ذلك اليوم العظيم، ويطرحونهم في ذلك اليوم في أتون النار، لكي ينتقم منهم رب الأرواح على إثمهم بخضوعهم للشيطان وإضلالهم سكان الأرض.»
\par 7 "وفي تلك الأيام يأتي العقاب من رب الأرواح، ويفتح جميع حجرات المياه التي فوق السماء، والينابيع التي تحت الأرض."
\par 8 وتتحد جميع المياه مع المياه: ما فوق السماء هو المذكر، وما تحت الأرض هو المؤنث
\par 9 ويهلك جميع سكان الأرض والساكنين تحت أقاصي السماء
\par 10 ومتى أدركوا إثمهم الذي فعلوه في الأرض، فبه يهلكون

\chapter{55}

\par 1 وبعد ذلك تاب رأس الأيام وقال: "عبثًا أهلكت جميع سكان الأرض".
\par 2 وأقسم باسمه العظيم: «لا أفعل هكذا بعد الآن لجميع سكان الأرض، وسأضع آية في السماء، وستكون هذه عهد أمانة بيني وبينهم إلى الأبد ما دامت السماء فوق الأرض».
\par 3 وهذا بأمري. فإذا أردتُ أن أُمسك بهم بأيدي الملائكة يوم الضيق والألم بسبب ذلك، سأُنزل عليهم عقابي وغضبي، يقول الله رب الأرواح.
\par 4 "أيها الملوك الأقوياء الذين يسكنون الأرض، يجب أن تنظروا إلى مختاري، كيف يجلس على عرش المجد ويحكم عزازيل، وكل شركائه، وكل جيوشه باسم رب الأرواح."

\chapter{56}

\par 1 ورأيت هناك جيوش ملائكة العذاب تنطلق، وهم يحملون سياطًا وسلاسل من حديد وبرونز
\par 2 فسألت ملاك السلام الذي كان معي قائلاً: "إلى من يذهب هؤلاء الذين يحملون السياط؟".
\par 3 وقال لي: «إلى مختاريهم وأحبائهم، لكي يُلقوا في هاوية هاوية الوادي».
\par 4 «وبعد ذلك سيمتلئ ذلك الوادي بمختاريهم وأحبائهم، وستنتهي أيام حياتهم، ولن تُحسب أيام ضلالهم من الآن فصاعدًا.»
\par 5 «وفي تلك الأيام، يعود الملائكة ويقذفون أنفسهم شرقًا على البارثيين والميديين: يهيجون الملوك، فتأتي عليهم روح اضطراب، فينهضونهم من عروشهم، فيخرجون كالأسود من أوكارها، وكذئاب جائعة بين قطعانهم.»
\par 6 «ويصعدون ويدوسون أرض مختاريه [وتكون أرض مختاريه أمامهم بيدرًا وطريقًا:]»
\par 7 «لكن مدينة أبيي تكون عائقًا لخيلهم. فيبدأون في القتال فيما بينهم، وتكون يمينهم قوية عليهم، ولا يعرف الإنسان أخاه، ولا الابن أباه أو أمه، حتى لا يبقى عدد من الجثث من جراء قتلهم، ولا يذهب عقابهم سدى.»
\par 8 «في تلك الأيام، ستفتح الهاوية فكيها، فيُبتلعون فيها، وينتهي هلاكهم. ستبتلع الهاوية الخطاة أمام المختارين.»

\chapter{57}

\par 1 وحدث بعد ذلك أنني رأيت جيشًا آخر من العربات، ورجالًا يركبونها، قادمين مع الرياح من الشرق، ومن الغرب إلى الجنوب
\par 2 وسُمع صوت عرباتهم، وعندما حدث هذا الاضطراب، لاحظه القديسون من السماء، وتزحزحت أعمدة الأرض من مكانها، وسُمع صوتها من أقصاء السماء إلى أقصاها، في يوم واحد
\par 3 ويسجدون جميعًا ويسجدون لرب الأرواح. وهذه هي نهاية المثل الثاني.

\chapter{58}

\par 1 وبدأتُ أُلقي المثل الثالث عن الأبرار والمختارين
\par 2 طوبى لكم أيها الأبرار والمختارون، لأن نصيبكم سيكون مجيدًا.
\par 3 ويكون الصديقون في نور الشمس، والمختارون في نور الحياة الأبدية، وتكون أيام حياتهم بلا نهاية، وأيام القديسين بلا عدد.
\par 4 "ويطلبون النور ويجدون البر عند رب الأرواح. ويكون السلام للصديقين باسم الرب الأبدي."
\par 5 وبعد هذا يقال للقديسين في السماء أن يطلبوا أسرار البر وميراث الإيمان. لأنه قد صار مضيئا كالشمس على الأرض والظلام مضى.
\par 6 ويكون هناك نور لا ينتهي، وإلى حد (حرفيًا) عدد من الأيام لن يأتوا، لأن الظلمة ستُدمر أولاً، [ويُقام النور أمام رب الأرواح] ونور الاستقامة يُقام إلى الأبد أمام رب الأرواح.

\chapter{59}

\par 1 في تلك الأيام، رأت عيني أسرار البرق، والأضواء، والأحكام التي ينفذونها (حرفيًا: حكمهم): وهي تضيء نعمة أو نقمة كما يشاء رب الأرواح
\par 2 وهناك رأيت أسرار الرعد، وكيف أنه عندما يتردد صداه في السماء، يُسمع صوته، وجعلني أرى الأحكام تُنفذ على الأرض، سواء كانت من أجل الخير والبركة، أو من أجل اللعنة وفقًا لكلمة رب الأرواح
\par 3 وبعد ذلك، أُظهِرت لي جميع أسرار الأنوار والبروق، وهي تُضيء للبركة وللإرضاء

\chapter{60} جزء من سفر نوح

\par 1 في سنة خمسمائة، في الشهر السابع، في اليوم الرابع عشر من حياة أخنوخ. في ذلك المثل، رأيتُ كيف ارتجفت سماء السماوات ارتجافًا شديدًا، وجيش العلي، والملائكة، آلافًا وعشرات الآلاف، اضطربوا اضطرابًا عظيمًا.
\par 2 وجلس رئيس الأيام على كرسي مجده، والملائكة والأبرار واقفون حوله.
\par 3 فأخذني رعدة عظيمة، وأخذني خوف، وارتخت حقوي، وذابت كليتاي، فسقطت على وجهي.
\par 4 وأرسل ميخائيل ملاكًا آخر من بين القديسين فأقامني، وعندما أقامني عادت روحي، لأني لم أستطع أن أتحمل منظر هذا الجيش، ولا اضطراب السماء وارتعاشها
\par 5 فقال لي ميخائيل: "لماذا أنت منزعج من هذه الرؤيا إلى أن يدوم هذا اليوم يوم رحمته، وقد كان رحيمًا وطويل الأناة على سكان الأرض."
\par 6 «وعندما يأتي اليوم، والقوة، والعقاب، والدينونة، التي أعدها رب الأرواح لأولئك الذين لا يعبدون الناموس الصالح، ولأولئك الذين ينكرون الدينونة العادلة، ولأولئك الذين يستخدمون اسمه عبثًا - يكون ذلك اليوم مُعدًّا، للمختارين عهدًا، وللخطاة محاكم تفتيش.»
\par 7 وفي ذلك اليوم انفصل وحشان، وحش أنثى يُدعى ليفياثان، ليعيش في هاويات المحيط فوق ينابيع المياه
\par 8 لكن الذكر يُدعى بهيموث، الذي سكن بصدره برية قاحلة تُدعى دويداين، شرقي الجنة حيث يسكن المختارون والصالحون، حيث أُخذ جدي، السابع من آدم، أول رجل خلقه رب الأرواح
\par 9 وتوسلت إلى الملاك الآخر أن يريني قوة تلك الوحوش، وكيف انفصلت في يوم واحد وألقيت، واحدة في هاوية البحر، والأخرى إلى أرض البرية الجافة
\par 10 وقال لي: "يا ابن الإنسان، هنا تسعى لمعرفة ما هو مخفي." وقال لي ملاك السلام الذي كان معي: "هذان الوحشان، المُهيآن بما يتناسب مع عظمة الله، سيتغذيان... وعندما يحل عليهما عقاب رب الأرواح، فإنه سيحل لكي لا يأتي عقاب رب الأرواح عبثًا، ويقتل الأطفال مع أمهاتهم والأطفال مع آبائهم. وبعد ذلك سيجري الحكم وفقًا لرحمته وصبره."
\par 11 والملاك الآخر الذي ذهب معي وأراني ما كان مخفيًا، أخبرني ما هو الأول والأخير في السماء في العلو، وتحت الأرض في العمق، وفي أقاصي السماء، وعلى أساس السماء
\par 12 وغرف الرياح، وكيف تُقسّم الرياح، وكيف تُوزن، وكيف تُحسب أبواب الرياح، كلٌّ حسب قوة الريح، وقوة أضواء القمر، وحسب القوة المناسبة: وأقسام النجوم حسب أسمائها، وكيف تُقسّم جميع الأقسام
\par 13 والرعود حسب الأماكن التي تقع فيها، وجميع التقسيمات التي تُصنع بين البروق حتى تضيء، وجيشها حتى تطيع في الحال
\par 14 لأن للرعد أماكن راحة مخصصة له أثناء انتظاره دويه؛ والرعد والبرق لا ينفصلان، وعلى الرغم من أنهما ليسا واحدًا وغير منقسمين، إلا أنهما يسيران معًا عبر الروح ولا ينفصلان
\par 15 لأنه عندما يضيء البرق، يُطلق الرعد صوته، ويفرض الروح توقفًا أثناء دويه، ويقسم بينهما بالتساوي؛ لأن كنز دويهاتهما يشبه الرمل، وكل واحد منهما أثناء دويه يُمسك بلجام، ويُعاد إلى الوراء بقوة الروح، ويُدفع إلى الأمام وفقًا لجوانب الأرض العديدة
\par 16 وروح البحر رجولية وقوية، وحسب قوة قوته يجذبها للخلف بلجام، وبالمثل تُدفع للأمام وتنتشر وسط كل جبال الأرض
\par 17 وروح الصقيع ملاكه الخاص، وروح البَرَد ملاك صالح
\par 18 وروح الثلج قد هجرت غرفها بسبب قوتها - هناك روح خاصة فيها، وما يصعد منها يشبه الدخان، واسمه الصقيع
\par 19 وروح الضباب ليست متحدة بهم في غرفهم، بل لها غرفة خاصة؛ لأن مسارها مجيد في النور والظلام، وفي الشتاء والصيف، وفي غرفتها ملاك
\par 20 وروح الندى تسكن في أقاصي السماء، وهي متصلة بغرف المطر، ومجراها في الشتاء والصيف: وسُحبها وسُحب الضباب متصلة، وكلٌّ منهما يُعطي الآخر
\par 21 وعندما تخرج روح المطر من حجرتها، تأتي الملائكة وتفتح الحجرة وتخرجها، وعندما تنتشر في جميع أنحاء الأرض، تتحد مع الماء على الأرض. وكلما اتحدت مع الماء على الأرض...
\par 22 لأن المياه هي لساكني الأرض، لأنها غذاء للأرض من العلي الذي في السماء، لذلك هناك قدر للمطر، 22 والملائكة يتحكمون فيه
\par 23 ورأيتُ هذه الأشياءَ نحوَ جنةِ الأبرار.

\chapter{61}

\par 1 ورأيت في تلك الأيام كيف أُعطيت حبال طويلة لأولئك الملائكة، فأخذوا لأنفسهم أجنحة وطاروا، واتجهوا نحو الشمال
\par 2 فسألتُ الملاكَ قائلًا له: "لماذا أخذ هؤلاء (الملائكة) هذه الحبالَ وانطلقوا؟"
\par 3 وقال لي: "لقد ذهبوا للقياس". وقال لي الملاك الذي ذهب معي: "هؤلاء سيأتون بمكاييل الصالحين، وحبال الصالحين إلى الصالحين، لكي يثبتوا على اسم رب الأرواح إلى أبد الآبدين".
\par 4 «سيبدأ المختارون السكنى مع المختارين، وهذه هي المقاييس التي ستُعطى للإيمان والتي ستقوي البر.»
\par 5 «وستكشف هذه الإجراءات جميع أسرار أعماق الأرض، وأولئك الذين أهلكتهم الصحراء، وأولئك الذين التهمتهم الوحوش، وأولئك الذين التهمتهم أسماك البحر، حتى يتمكنوا من العودة والبقاء هم أنفسهم في يوم المختار؛ لأنه لن يُهلك أحد أمام رب الأرواح، ولا يمكن إهلاك أحد.»
\par 6 «وجميع الساكنين فوق في السماء أخذوا أمرًا وسلطانًا وصوتًا واحدًا ونورًا واحدًا مثل نار.»
\par 7 «وذلك الذي (بكلماتهم) الأولى باركوه، ومجدوه، وأثنوا عليه بحكمة، وكانوا حكماء في النطق وفي روح الحياة.»
\par 8 «ووضع رب الأرواح المختار على عرش المجد. وسيدين جميع أعمال القديسين في السماء من فوق، وفي الميزان توزن أعمالهم.»
\par 9 «وعندما يرفع وجهه ليحكم على طرقهم الخفية وفقًا لكلمة اسم رب الأرواح، وسبيلهم وفقًا لطريق القضاء العادل لرب الأرواح، فحينئذٍ سيتكلمون جميعًا بصوت واحد ويباركون، ويمجدون، ويرفعون، ويقدسون اسم رب الأرواح.»
\par 10 «وسيدعو كل جند السماوات، وكل القديسين في السماء، وجند الله، الكروبيم، والسيرافين، والأوفان، وكل ملائكة القوة، وكل ملائكة الرئاسات، والمختار، والقوات الأخرى على الأرض (و) فوق الماء.»
\par 11 "في ذلك اليوم يرفع الجميع صوتًا واحدًا، ويباركون ويمجدون ويرتفعون بروح الإيمان، وبروح الحكمة، وبروح الصبر، وبروح الرحمة، وبروح الحكمة والسلام، وبروح الخير، ويقولون جميعًا بصوت واحد: تبارك هو، وليكن اسم رب الأرواح مباركًا إلى الأبد."
\par 12 سيباركه كل من ليس نائمًا في السماء، وسيباركه جميع القديسين الذين في السماء، وجميع المختارين الساكنين في جنة الحياة، وكل روح نورانية قادرة على أن تبارك و تمجد و تعظم و تقدس اسمك المبارك، وكل بشر سيمجدون ويباركون اسمك إلى أبد الآبدين
\par 13 لأن رحمة رب الأرواح عظيمة، وهو طويل الأناة، وقد كشف جميع أعماله وكل ما خلقه للأبرار والمختارين باسم رب الأرواح

\chapter{62}

\par 1 وهكذا أمر الرب الملوك والأقوياء والمتسامين وسكان الأرض، وقال: «افتحوا أعينكم وارفعوا قرونكم إن استطعتم أن تعرفوا المختار».
\par 2 وأجلسه رب الأرواح على عرش مجده، وانسكب عليه روح البر، وكلمة فمه تقتل كل الخطاة، ويُباد جميع الأشرار من أمام وجهه
\par 3 ويقوم في ذلك اليوم جميع الملوك والأقوياء والمتسامون ومالكو الأرض، وينظرون ويعرفون كيف يجلس على كرسي مجده، ويُحكم على البر أمامه، ولا يُقال أمامه كلام كذب
\par 4 حينئذٍ يأتيهم وجع كالمخاض، [وتتألم في الولادة] عندما يدخل طفلها فم الرحم، وتتألم في الولادة
\par 5 وينظر قسم منهم إلى القسم الآخر، فيرتاعون، وتنهار وجوههم، ويصيبهم الألم، عندما يرون ابن الإنسان جالسًا على عرش مجده
\par 6 ويبارك الملوك والأقوياء وكل من يملك الأرض ويمجدون ويرفعون سلطان الجميع، الذي كان مخفيًا
\par 7 لأنه منذ البدء كان ابن الإنسان مخفيًا، وحفظه العلي أمام قدرته، وأعلنه للمختارين
\par 8 وتُزرع جماعة المختارين والقديسين، ويقف جميع المختارين أمامه في ذلك اليوم
\par 9 ويخر جميع الملوك والأقوياء والمرتفعين وحكام الأرض أمامه على وجوههم، ويسجدون، ويضعون رجاءهم على ابن الإنسان، ويتوسلون إليه ويتوسلون إليه طالبين الرحمة من يديه
\par 10 ومع ذلك، سيضغط عليهم رب الأرواح حتى يخرجوا مسرعين من حضرته، وستمتلئ وجوههم بالخزي، ويزداد الظلام عمقًا على وجوههم
\par 11 "وسيسلمهم إلى الملائكة للعقاب، لينفذوا الانتقام منهم لأنهم ظلموا أولاده ومختاريه."
\par 12 ويكونون مشهدًا للأبرار ولمختاريه. يفرحون بهم، لأن غضب رب الأرواح يحل عليهم، وسيفه قد سكر من دمائهم
\par 13 وسيخلص الأبرار والمختارون في ذلك اليوم، ولن يروا بعد ذلك وجه الخطاة والأشرار
\par 14 وسيدبر عليهم رب الأرواح، ومع ابن الإنسان يأكلون ويرقدون ويقومون إلى أبد الآبدين
\par 15 وسيقوم الأبرار والمختارون من الأرض، ويكفون عن أن يكونوا منكسري الوجوه. وسيلبسون ثياب المجد،
\par 16 وستكون هذه ثياب الحياة من رب الأرواح: ولن تبلى ثيابك، ولن يزول مجدك أمام رب الأرواح

\chapter{63}

\par 1 في تلك الأيام، سيتوسل إليه الأقوياء والملوك الذين يملكون الأرض أن يمنحهم مهلة قصيرة من ملائكة العقاب الذين أُسلموا إليهم، حتى يخروا ويسجدوا أمام رب الأرواح، ويعترفوا بخطاياهم أمامه
\par 2 ويباركون ويمجدون رب الأرواح، ويقولون: مبارك رب الأرواح ورب الملوك ورب الأقوياء ورب الأغنياء ورب المجد ورب الحكمة،
\par 3 وبهية في كل سرٍّ هي قدرتك من جيل إلى جيل، ومجدك إلى أبد الآبدين: عميقة هي جميع أسرارك ولا تُحصى، وبرك لا يُحصى
\par 4 لقد تعلمنا الآن أنه يجب علينا أن نمجد ونبارك رب الملوك، والذي هو ملك على كل الملوك
\par 5 وسيقولون: يا ليتنا استرحنا لنمجّد ونشكر ونعترف بإيماننا أمام مجده!
\par 6 «والآن نتوق إلى قسط من الراحة لكننا لا نجده: نسعى جاهدين وراءه ولا ننالها: وقد اختفى النور من أمامنا، والظلام مسكننا إلى الأبد:»
\par 7 «لأننا لم نؤمن قبله، ولا مجدنا اسم رب الأرواح، [ولا مجدنا ربنا]، لكن رجاؤنا كان على صولجان ملكنا، وعلى مجدنا.»
\par 8 «وفي يوم معاناتنا وضيقنا لا يخلصنا، ولا نجد أي مجال للاعتراف بأن ربنا صادق في جميع أعماله، وفي أحكامه وعدله، وأن أحكامه لا تحترم الأشخاص.»
\par 9 «ونمضي من أمام وجهه بسبب أعمالنا، وتُحسب جميع خطايانا في البر.»
\par 10 الآن سيقولون لأنفسهم: "نفوسنا مليئة بالكسب غير المشروع، لكن هذا لا يمنعنا من النزول من وسطها إلى عبء الهاوية."
\par 11 وبعد ذلك تمتلئ وجوههم ظلاماً وخجلاً أمام ابن الإنسان، ويطردون من أمامه، ويقف السيف أمام وجهه في وسطهم.
\par 12 هكذا تكلم رب الأرواح: "هذا هو النظام والحكم فيما يتعلق بالأقوياء والملوك والمرتفعين وأولئك الذين يملكون الأرض أمام رب الأرواح."

\chapter{64}

\par 1 وأشكال أخرى رأيتها مختبئة في ذلك المكان
\par 2 سمعت صوت الملاك يقول: "هؤلاء هم الملائكة الذين نزلوا إلى الأرض، وكشفوا ما كان مخفيًا لأبناء البشر، وأغووا أبناء البشر لارتكاب الخطيئة."

\chapter{65}

\par 1 وفي تلك الأيام رأى نوح الأرض أنها قد غرقت وأن هلاكها قد اقترب
\par 2 ثم قام من هناك وذهب إلى أقاصي الأرض ونادى جده حنوك بصوت عظيم، فقال نوح ثلاث مرات بصوت مرير: "اسمعني، اسمعني، اسمعني".
\par 3 فقلت له: "أخبرني ما الذي يسقط على الأرض حتى أصبحت في مثل هذا الوضع السيئ ومهتزة، خشية أن أهلك معها."
\par 4 فحدث اضطراب عظيم على الأرض، وسمع صوت من السماء، فسقطت على وجهي
\par 5 فجاء جدي حنوك ووقف بجانبي، وقال لي: "لماذا صرخت إليّ صرخة مريرة وبكاء؟"
\par 6 «وقد صدر أمر من حضرة الرب بشأن أولئك الذين يسكنون الأرض بأن هلاكهم قد تم لأنهم تعلموا كل أسرار الملائكة، وكل عنف الشياطين، وكل قدراتهم - الأكثر سرية - وكل قوة أولئك الذين يمارسون السحر، وقوة السحر، وقوة أولئك الذين يصنعون صورًا مسبوكة للأرض كلها؛»
\par 7 «وكيف تُستخرج الفضة من تراب الأرض، وكيف ينشأ المعدن اللين في الأرض.»
\par 8 «لأن الرصاص والقصدير لا يُستخرجان من الأرض مثل الأول: بل ينبوع يُنتجهما، ويقف فيه ملاك، وهذا الملاك هو الأبرز.»
\par 9 وبعد ذلك أمسكني جدي حنوك بيدي ورفعني، وقال لي: "اذهب، فقد طلبت من رب الأرواح أن يأمرني بهذا الاضطراب على الأرض."
\par 10 «فقال لي: «بسبب إثمهم، قد حُكم عليهم، ولن أُمسك عنهم إلى الأبد. بسبب السحر الذي بحثوا عنه وتعلموه، ستُهلك الأرض وساكنوها.»
\par 11 «وهؤلاء ليس لهم مكان للتوبة إلى الأبد، لأنهم أظهروا لهم ما كان مخفيًا، وهم الملعونون: أما أنت يا بني، فإن رب الأرواح يعلم أنك طاهر، وبرئ من هذا العار المتعلق بالأسرار.»
\par 12 "وقد قدر اسمك بين القديسين، وسيحفظك بين سكان الأرض، وقد قدر نسلك البار للملك وللكرامات العظيمة، ومن نسلك يخرج ينبوع الصديقين والقديسين بلا عدد إلى الأبد."

\chapter{66}

\par 1 وبعد ذلك أراني ملائكة العقاب المستعدين للمجيء وإطلاق جميع قوى المياه التي تحت الأرض من أجل جلب الدينونة والهلاك على كل من يسكن على الأرض
\par 2 وأعطى رب الأرواح أمرًا للملائكة الخارجين ألا يتسببوا في ارتفاع المياه، بل أن يكبحوها؛ لأن هؤلاء الملائكة كانوا على سلطات المياه
\par 3 وانصرفتُ من أمام حنوك.

\chapter{67}

\par 1 وفي تلك الأيام جاءتني كلمة الله وقال لي: "يا نوح، لقد صعدت قرعتك أمامي كثيرًا بلا لوم، كثيرًا من المحبة والاستقامة".
\par 2 "والآن الملائكة يصنعون بناءً خشبيًا، وعندما ينتهون من هذه المهمة سأضع يدي عليه وأحفظه، وسيخرج منه بذرة الحياة، وسيحدث تغيير بحيث لا تبقى الأرض بلا ساكن."
\par 3 "وأثبت عشبك أمامي إلى الأبد، وأنشر سكانك الساكنين معك. لا يكون بلا ثمر على وجه الأرض، بل يتبارك ويكثر على الأرض باسم الرب."
\par 4 وسيسجن أولئك الملائكة الذين أظهروا الظلم في ذلك الوادي المحترق الذي أراني إياه جدي أخنوخ سابقًا في الغرب بين جبال الذهب والفضة والحديد والمعادن اللينة والقصدير
\par 5 ورأيت ذلك الوادي وقد حدث فيه اضطراب عظيم واضطراب في المياه
\par 6 وعندما حدث كل هذا، انبعثت من ذلك المعدن المنصهر الناري ومن تشنجه في ذلك المكان رائحة كبريت، وارتبطت بتلك المياه، واحترق وادي الملائكة الذين أضلوا البشرية تحت تلك الأرض
\par 7 ومن خلال وديانها تجري تيارات من نار، حيث يُعاقب هؤلاء الملائكة الذين أضلوا من يسكنون الأرض
\par 8 لكن تلك المياه ستكون في تلك الأيام للملوك والأقوياء والمتسامين، ولسكان الأرض، لشفاء الجسد، ولكن لمعاقبة الروح. الآن أرواحهم مليئة بالشهوة، لكي يُعاقبوا في أجسادهم، لأنهم أنكروا رب الأرواح ويرون عقابهم يوميًا، ومع ذلك لا يؤمنون باسمه
\par 9 ومع ازدياد شدة احتراق أجسادهم، سيحدث تغيير مماثل في أرواحهم إلى الأبد؛ لأنه أمام رب الأرواح لن ينطق أحد بكلمة بطالة
\par 10 لأن الدينونة ستأتي عليهم، لأنهم يؤمنون بشهوات أجسادهم وينكرون روح الرب
\par 11 وتلك المياه نفسها سوف تخضع للتغيير في تلك الأيام؛ لأنه عندما يعاقب أولئك الملائكة في هذه المياه، فإن ينابيع المياه هذه سوف تغير درجة حرارتها، وعندما يصعد الملائكة، فإن مياه الينابيع هذه سوف تتغير وتصبح باردة.
\par 12 وسمعتُ ميخائيل يُجيب ويقول: "هذه الدينونة التي يُحاكم بها الملائكة هي شهادة للملوك والأقوياء الذين يملكون الأرض."
\par 13 «لأن مياه الدينونة هذه تعمل على شفاء أجساد الملوك وشهواتهم؛ لذلك لن يروا ولن يؤمنوا أن تلك المياه ستتغير وتتحول إلى نار متقدة إلى الأبد.»

\chapter{68}

\par 1 وبعد ذلك أعطاني جدي أخنوخ تعليم جميع الأسرار الموجودة في كتاب الأمثال الذي أُعطي له، وجمعها لي في كلمات كتاب الأمثال
\par 2 وفي ذلك اليوم أجاب ميخائيل رافائيل وقال: "إن قوة الروح تنقلني وتجعلني أرتجف بسبب شدة دينونة الأسرار، دينونة الملائكة: من يستطيع أن يتحمل الدينونة القاسية التي نُفِّذت، والتي يذوبون أمامها؟"
\par 3 فأجاب ميخائيل مرة أخرى، وقال لرافائيل: "من هو الذي لم يلين قلبه بشأن هذا الأمر، ولم تضطرب لجامه بسبب كلمة الدينونة هذه التي صدرت عليهم بسبب أولئك الذين أخرجوهم؟"
\par 4 وحدث عندما وقف أمام رب الأرواح أن ميخائيل قال لرافائيل هكذا: "لن أتحمل نصيبهم تحت نظر الرب؛ لأن رب الأرواح قد غضب عليهم لأنهم يتصرفون كما لو كانوا الرب."
\par 5 «لذلك، فإن كل ما هو خفي سيأتي عليهم إلى أبد الآبدين؛ لأنه لا ملاك ولا إنسان سيكون له نصيبه (فيه)، لكنهم وحدهم نالوا دينونتهم إلى أبد الآبدين.»

\chapter{69}

\par 1 وبعد هذا الدينونة، سيُرهبونهم ويرعبونهم لأنهم أظهروا ذلك لساكني الأرض
\par 2 "وانظر إلى أسماء أولئك الملائكة [وهذه أسماؤهم: أولهم سامجازا، والثاني أرتاقيفا، والثالث أرمين، والرابع كوكابيل، والخامس تورائيل، والسادس رومجال، والسابع دانجال، والثامن نقائل، والتاسع باراقيل، والعاشر أزازيل، والحادي عشر أرماروس، والثاني عشر باترجال، والثالث عشر بوساسجال، والرابع عشر حنانيل، والخامس عشر توريل، والسادس عشر سيمابسييل، والسابع عشر جيترل، والثامن عشر تومائيل، والتاسع عشر توريل، والعشرون رومائيل، والحادي والعشرون أزازيل."
\par 3 وهؤلاء هم رؤساء ملائكتهم وأسماؤهم، ورؤساء مئة وخمسين وعشرة].
\par 4 اسم يقون الأول أي الذي أضل جميع أبناء الله وأنزلهم إلى الأرض وأضلهم بواسطة بنات الناس.
\par 5 والثاني اسمه أسبيل، هذا أوصى أبناء الله القديسين بمشورة رديئة، وأضلهم حتى نجّسوا أجسادهم مع بنات الناس.
\par 6 والثالث كان اسمه غادرئيل: وهو الذي أظهر لأبناء البشر جميع ضربات الموت، وأضل حواء، وأظهر [أسلحة الموت لأبناء البشر] الدرع والدرع، والسيف للمعركة، وجميع أسلحة الموت لأبناء البشر
\par 7 ومن يده خرجوا على سكان الأرض من ذلك اليوم وإلى الأبد
\par 8 والرابع كان اسمه بينيموي: علّم أبناء البشر المر والحلو، وعلمهم كل أسرار حكمتهم
\par 9 وعلّم البشرية الكتابة بالحبر والورق، وبذلك أخطأ كثيرون من الأزل إلى الأبد وإلى يومنا هذا
\par 10 فلم يُخلق البشر لمثل هذا الغرض، أي لتأكيد حسن نيتهم ​​بالقلم والحبر
\par 11 لأن البشر خُلقوا تمامًا مثل الملائكة، بقصد أن يبقوا طاهرين وأبرارًا، والموت، الذي يدمر كل شيء، لم يكن ليُمسك بهم، ولكن من خلال هذه المعرفة فإنهم يهلكون، ومن خلال هذه القوة يستهلكني
\par 12 والخامس كان اسمه كاسديجا: هذا هو الذي أظهر لأبناء البشر جميع الضربات الشريرة للأرواح والشياطين، وضربات الجنين في الرحم حتى يزول، و[ضربات الروح] لدغات الحية، والضربات التي تصيبهم خلال حرارة الظهيرة، وهو ابن الحية المسمى تابايت
\par 13 وهذه هي مهمة كسبيل، رأس القسم الذي أقسم به للقديسين عندما سكن عالياً في المجد، واسمه بقاع
\par 14 طلب هذا (الملاك) من ميخائيل أن يريه الاسم المخفي، لينطق به في القسم، حتى يرتعد أولئك الذين كشفوا كل ما هو سر لأبناء البشر أمام ذلك الاسم وهذا القسم
\par 15 وهذه هي قوة هذا القسم، لأنه قوي ومتين، وقد وضع هذا القسم أكاي في يد ميخائيل
\par 16 وهذه هي أسرار هذا القسم... وهي قوية بقسمه: وكانت السماء معلقة قبل خلق العالم، وإلى الأبد
\par 17 ومن خلاله تأسست الأرض على الماء، ومن أعماق الجبال تأتي مياه جميلة، منذ خلق العالم وإلى الأبد
\par 18 ومن خلال هذا القسم خُلق البحر، وجعل له أساسًا الرمال ضد وقت غضبه، ولا يجرؤ على تجاوزه من خلق العالم إلى الأبد
\par 19 ومن خلال هذا القسم تُثبّت الأعماق، فلا تزول ولا تتحرك من مكانها من أبد إلى أبد
\par 20 ومن خلال هذا القسم، تُكمل الشمس والقمر مسارهما، ولا يحيدان عن نظامهما من الأزل إلى الأبد
\par 21 ومن خلال هذا القسم تكمل النجوم مسارها، ويناديها بأسمائها، فتجيبه من الأزل إلى الأبد.
\par 22 [وكذلك أرواح الماء، والرياح، وجميع النسمات، ومساراتها من جميع جهات الرياح
\par 23 وهناك محفوظة أصوات الرعد ونور البروق، وهناك محفوظة حجرات البرد وحجرات الصقيع، وحجرات الضباب، وحجرات المطر والندى
\par 24 وكل هؤلاء يؤمنون ويشكرون رب الأرواح، ويمجدونه بكل قوتهم، وطعامهم في كل شكر: يشكرون ويمجدون ويمجدون اسم رب الأرواح إلى أبد الآبدين.]
\par 25 وهذا القسم عظيم عليهم، ومن خلاله يُحفظون، وتُحفظ مساراتهم، ولا يُهلك مسارهم
\par 26 وكان فرح عظيم بينهم، وباركوا ومجدوا وسبحوا، لأنه أُظهر لهم اسم ابن الإنسان
\par 27 وجلس على عرش مجده، وأُعطيت جملة الدينونة لابن الإنسان، وجعل الخطاة يهلكون ويهلكون عن وجه الأرض، والذين أضلوا العالم
\par 28 سيُقيَّدون بالسلاسل، ويُسجنون في مكان تجمعهم المهلك، وتزول جميع أعمالهم عن وجه الأرض
\par 29 ومن الآن فصاعدًا لن يكون هناك شيء فاسد، لأن ابن الإنسان قد ظهر، وجلس على عرش مجده، وسيزول كل شر من أمام وجهه، وتخرج كلمة ابن الإنسان. وتقووا أمام رب الأرواح

\chapter{70}

\par 1 وحدث بعد ذلك أن اسمه رُفع في حياته إلى ابن الإنسان وإلى رب الأرواح من بين الساكنين على الأرض
\par 2 ورُفع عالياً على مركبات الروح واختفى اسمه من بينها
\par 3 ومن ذلك اليوم لم أعد معدودًا بينهم، ووضعني بين الريحين، بين الشمال والغرب، حيث أخذ الملائكة الحبال ليقيسوا لي مكان المختارين والصالحين
\par 4 وهناك رأيت الآباء الأوائل والأبرار الذين من البدء يسكنون في ذلك المكان

\chapter{71}

\par 1 وحدث بعد ذلك أن روحي انتقلت وصعدت إلى السموات: ورأيت أبناء الله القديسين يدوسون على لهيب نار: كانت ثيابهم بيضاء [وملابسهم]، ووجوههم تلمع كالثلج
\par 2 ورأيت جدولين من النار، وكان نور تلك النار يتلألأ كالزنبق، وسقطت على وجهي أمام رب الأرواح.
\par 3 وأمسكني الملاك ميخائيل بيدي اليمنى ورفعني وأخرجني إلى جميع الأسرار وأراني جميع أسرار البر.
\par 4 وأراني كل أسرار أقاصي السماء، وكل حجرات كل النجوم، وكل المنيرات التي تخرج منها أمام وجه القديسين.
\par 5 ونقل روحي إلى سماء السماوات، ورأيت هناك كما لو كان هيكلًا مبنيًا من البلورات، وبين تلك البلورات ألسنة من نار حية
\par 6 ورأى روحي المنطقة التي تُحيط ببيت النار، وعلى جوانبه الأربعة جداول ممتلئة من نار حية، وهي تُحيط بذلك البيت
\par 7 وحوله سيرافين وكروبيم وأوفانين. وهؤلاء هم الذين لا ينامون ويحرسون عرش مجده
\par 8 ورأيت ملائكة لا يُحصى عددهم، ألف ألف، وعشرة آلاف، يحيطون بذلك البيت. وميخائيل، ورافائيل، وجبرائيل، وفانوئيل، والملائكة القديسون الذين فوق السماوات، يدخلون ويخرجون من ذلك البيت
\par 9 وخرج من ذلك البيت ميخائيل وجبرائيل ورافائيل وفانوئيل والعديد من الملائكة القديسين بلا عدد
\par 10 ومعهم "رأس الأيام"، رأسه أبيض نقي كالصوف، ولباسه لا يوصف
\par 11 فسقطت على وجهي، واسترخى جسدي كله، وتغيرت هيئتي، وصرخت بصوت عظيم ... بروح القوة، ومباركة وممجدة ومرتفعة
\par 12 وكانت هذه البركات التي خرجت من فمي مرضية للغاية أمام "رأس الأيام". وجاء "رأس الأيام" مع ميخائيل وجبرائيل ورافائيل وفانوئيل، آلاف وعشرات الآلاف من الملائكة بلا عدد
\par 13 [مقطع مفقود يوصف فيه ابن الإنسان بأنه مرافق لـ "رأس الأيام"، ويسأل أخنوخ أحد الملائكة (كما في XLVI: 3) عن ابن الإنسان من هو.]
\par 14 فجاء إليّ (أي الملاك) وسلم عليّ بصوته، وقال لي: "هذا هو ابن الإنسان المولود للبر، والبر يثبت عليه، وبر "رأس الأيام" لا يتركه."
\par 15 وقال لي: «إنه يُعلن لك السلام باسم العالم الآتي؛ لأنه من هنا انبثق السلام منذ خلق العالم، وهكذا سيكون لك إلى أبد الآبدين.»
\par 16 «وسيسلك كل واحد في طرقه، لأن البر لا يتخلى عنه. معه تكون مساكنهم، ومعه ميراثهم، ولن ينفصلوا عنه إلى أبد الآبدين.»
\par 17 «وهكذا ستكون هناك أيام طويلة مع ابن الإنسان، ويكون للأبرار سلام وطريق مستقيم باسم رب الأرواح إلى أبد الآبدين.»

\part {القسم الثالث. الفصول من ٧٢ إلى ٨٢. كتاب الأنوار السماوية.}

\chapter{72}

\par 1 "كتاب مسارات الكواكب السماوية، وعلاقات كل منها، حسب طبقاتها، وسلطانها، وأوقاتها، وأسمائها وأماكن نشأتها، وأشهرها، التي أراني إياها أورييل، الملاك المقدس، الذي كان معي، وهو مرشدهم؛ وأراني جميع قوانينهم كما هي بالضبط، وكيف هي بالنسبة لجميع سنوات العالم وإلى الأبد، حتى يتم الخلق الجديد الذي يستمر إلى الأبد.
\par 2 وهذا هو القانون الأول للنيّرات: للنيّر - الشمس - شروقه في بوابات السماء الشرقية، وغروبه في بوابات السماء الغربية
\par 3 ورأيت ستة بوابات تشرق منها الشمس، وستة بوابات تغرب فيها الشمس ويشرق القمر ويغرب في هذه البوابات، وقادة النجوم ومن تقودهم: ستة في الشرق وستة في الغرب، وكلها تتبع بعضها البعض بترتيب دقيق ومتطابق: وأيضًا العديد من النوافذ على يمين ويسار هذه البوابات
\par 4 وأولاً يخرج النجم العظيم، المسمى الشمس، ومحيطه يشبه محيط السماء، وهو ممتلئ تمامًا بالنار المضيئة والمُسخِّنة
\par 5 المركبة التي يصعد عليها، تدفعها الرياح، وتغرب الشمس من السماء وتعود عبر الشمال لتصل إلى الشرق، وتُهدى بحيث تصل إلى البوابة المناسبة (حرفيًا: تلك) وتشرق على وجه السماء
\par 6 بهذه الطريقة يرتفع في الشهر الأول في البوابة العظيمة، وهي الرابعة [تلك البوابات الستة في القالب].
\par 7 وفي تلك البوابة الرابعة التي تشرق منها الشمس في الشهر الأول، توجد اثنا عشر نافذة، يخرج منها لهب عندما تُفتح في موسمها
\par 8 عندما تشرق الشمس في السماء، فإنها تخرج من تلك البوابة الرابعة ثلاثين صباحًا متتالية، وتغرب بدقة في البوابة الرابعة في غرب السماء
\par 9 وخلال هذه الفترة، يصبح النهار أطول يوميًا والليل أقصر ليلًا حتى صباح الثلاثين
\par 10 في ذلك اليوم يكون النهار أطول من الليل بتسعة أجزاء، ويكون النهار عشرة أجزاء بالضبط والليل ثمانية أجزاء
\par 11 وتشرق الشمس من تلك البوابة الرابعة، وتغرب في الرابعة، ثم تعود إلى البوابة الخامسة من الشرق بعد ثلاثين صباحًا، وتشرق منها وتغرب في البوابة الخامسة
\par 12 ثم يطول النهار جزأين فيكون أحد عشر جزءًا، ويقصر الليل سبعة أجزاء
\par 13 ويعود إلى الشرق ويدخل من الباب السادس، ويشرق ويغرب في الباب السادس بعد واحد وثلاثين صباحًا بسبب علامته
\par 14 في ذلك اليوم يطول النهار أكثر من الليل، ويصبح النهار ضعف الليل، ويصبح النهار اثني عشر جزءًا، ويقصر الليل فيصبح ستة أجزاء
\par 15 وتشرق الشمس لتقصر النهار وتطول الليل، وتعود الشمس إلى المشرق وتدخل من الباب السادس، وتشرق منه وتغرب ثلاثين صباحا.
\par 16 وعندما يتم ثلاثون صباحًا، ينقص النهار جزءًا واحدًا بالضبط، فيصبح أحد عشر جزءًا، والليل سبعة أجزاء
\par 17 وتخرج الشمس من تلك البوابة السادسة في الغرب، وتتجه شرقًا وتشرق في البوابة الخامسة لمدة ثلاثين صباحًا، ثم تغرب في الغرب مرة أخرى في البوابة الغربية الخامسة
\par 18 في ذلك اليوم ينقص النهار جزأين، فيصبح عشرة أجزاء، والليل ثمانية أجزاء
\par 19 وتخرج الشمس من ذلك الباب الخامس وتغرب في الباب الخامس من الغرب، وتشرق في الباب الرابع لمدة واحد وثلاثين صباحًا بسبب علامتها، وتغرب في الغرب
\par 20 في ذلك اليوم يتساوى النهار مع الليل، [ويصبحان متساويين في الطول]، فيكون الليل تسعة أجزاء والنهار تسعة أجزاء
\par 21 وتشرق الشمس من تلك البوابة وتغرب في الغرب، ثم تعود إلى الشرق وتشرق ثلاثين صباحًا في البوابة الثالثة وتغرب في الغرب في البوابة الثالثة
\par 22 وفي ذلك اليوم يطول الليل أكثر من النهار، ويطول الليل أكثر من الليل، ويقصر النهار عن النهار إلى صباح الثلاثين، فيكون الليل عشرة أجزاء بالضبط والنهار ثمانية أجزاء
\par 23 وتشرق الشمس من ذلك الباب الثالث وتغرب في الباب الثالث في الغرب ثم تعود إلى الشرق، وتشرق لمدة ثلاثين صباحًا في الباب الثاني في الشرق، وبالمثل تغرب في الباب الثاني في غرب السماء
\par 24 وفي ذلك اليوم يكون الليل أحد عشر جزءًا والنهار سبعة أجزاء
\par 25 وتشرق الشمس في ذلك اليوم من ذلك الباب الثاني وتغرب في الغرب في الباب الثاني، وتعود إلى الشرق في الباب الأول لمدة صباح وثلاثين دقيقة، وتغرب في الباب الأول في غرب السماء
\par 26 وفي ذلك اليوم يطول الليل ويصبح ضعف النهار: فيكون الليل اثني عشر جزءًا بالضبط والنهار ستة أجزاء
\par 27 وقد عبرت الشمس أقسام مدارها، ثم دارت مرة أخرى على تلك الأقسام، ودخلت تلك البوابة ثلاثين صباحًا، وغروبت أيضًا في الغرب المقابل لها
\par 28 وفي تلك الليلة نقص الليل تسعًا، فأصبح الليل أحد عشر جزءًا والنهار سبعة أجزاء
\par 29 وعادت الشمس ودخلت من البوابة الثانية في الشرق، وتعود على تلك الأقسام من مدارها لمدة ثلاثين صباحًا، تشرق وتغرب
\par 30 وفي ذلك اليوم يتناقص الليل، فيكون الليل عشرة أجزاء والنهار ثمانية أجزاء
\par 31 وفي ذلك اليوم تشرق الشمس من تلك البوابة، وتغرب في الغرب، ثم تعود إلى الشرق، وتشرق في البوابة الثالثة لمدة واحد وثلاثين صباحًا، وتغرب في غرب السماء
\par 32 وفي ذلك اليوم ينقص الليل فيكون تسعة أجزاء، والنهار تسعة أجزاء، ويكون الليل مثل النهار، وتكون السنة مثل أيامها ثلاثمائة وأربعة وستين.
\par 33 وطول الليل والنهار، وقصر الليل والنهار - من خلال مسار الشمس، يتم تمييز هذه الفروق (حرفيًا "إنهم منفصلون").
\par 34 فيصبح مساره يوميًا أطول، ومساره ليلًا أقصر
\par 35 وهذا هو قانون الشمس ومجرىها، وعودتها كلما عادت ستين مرة وطلعت، أي النجم العظيم الذي يُدعى الشمس، إلى أبد الآبدين
\par 36 والذي يرتفع (هكذا) هو النجم العظيم، ويُسمى كذلك حسب مظهره، كما أمر الرب
\par 37 كما يشرق، كذلك يغرب، لا ينقص، ولا يهدأ، بل يجري ليلًا ونهارًا، ونوره أشد سطوعًا من نور القمر بسبعة أضعاف؛ ولكن من حيث الحجم، كلاهما متساويان

\chapter{73}

\par 1 وبعد هذا القانون، رأيت قانونًا آخر يتعلق بالنجم الأصغر، وهو القمر
\par 2 ومحيطها كمحيط السماء، ومركبتها التي تركبها تدفعها الريح، ويُعطى لها النور بقدر (محدد).
\par 3 ويتغير طلوعها وغروبها كل شهر، وأيامها كأيام الشمس، وعندما يكون ضوءها موحدًا (أي كاملًا) فإنه يعادل سابع ضوء الشمس
\par 4 وهكذا تشرق. ويظهر أول طور لها في الشرق في صباح الثلاثين: وفي ذلك اليوم تصبح مرئية، وتشكل بالنسبة لكم أول طور للقمر في اليوم الثلاثين مع الشمس في البوابة التي تشرق منها الشمس
\par 5 ونصفها يخرج سبعًا، ومحيطها كله فارغ، لا نور فيه، إلا سبعه، وربعه عشر من نوره
\par 6 فإذا استقبلت سُبع نصف نورها، كان نورها سُبعًا ونصفه
\par 7 وتغرب مع الشمس، فإذا طلعت الشمس طلع القمر معها، واستقبل نصف جزء من الضوء، وفي تلك الليلة في أول صباحها [في بداية اليوم القمري] يغرب القمر مع الشمس، ويغيب في تلك الليلة مع الأجزاء الأربعة عشر ونصف جزء منها
\par 8 وتشرق في ذلك اليوم بسبعها تمامًا، وتخرج وتتأخر عن طلوع الشمس، وفي بقية أيامها تشرق في الثلاثة عشر جزءًا (الباقية).

\chapter{74}

\par 1 ورأيتُ مسارًا آخر، وقانونًا لها، وكيف تُجري دورتها الشهرية وفقًا لهذا القانون
\par 2 "وهؤلاء كلهم ​​أراني إياهم أورييل الملاك المقدس الذي هو زعيمهم جميعًا، ومواقعهم، وكتبت مواقعهم كما أراني إياهم، وكتبت أشهرهم كما كانت، وظهور أنوارهم حتى انقضت خمسة عشر يومًا."
\par 3 في سبعة أجزاء واحدة تُكمل كل نورها في الشرق، وفي سبعة أجزاء واحدة تُكمل كل ظلامها في الغرب
\par 4 وفي أشهر معينة تُغير إعداداتها، وفي أشهر معينة تتبع مسارها الخاص
\par 5 في غضون شهرين، يغرب القمر مع الشمس: في هاتين البوابتين الأوسطتين، تغرب البوابتان الثالثة والرابعة
\par 6 تخرج لمدة سبعة أيام، ثم تدور وتعود مرة أخرى من خلال البوابة التي تشرق منها الشمس، وتُكمل كل نورها: ثم تبتعد عن الشمس، وفي ثمانية أيام تدخل البوابة السادسة التي تخرج منها الشمس
\par 7 وعندما تخرج الشمس من البوابة الرابعة، فإنها تخرج بعد سبعة أيام، حتى تخرج من البوابة الخامسة وتعود مرة أخرى بعد سبعة أيام إلى البوابة الرابعة وتُكمل كل نورها: ثم تتراجع وتدخل البوابة الأولى بعد ثمانية أيام
\par 8 وتعود مرة أخرى بعد سبعة أيام إلى البوابة الرابعة التي تخرج منها الشمس
\par 9 وهكذا رأيت موقعهم - كيف تشرق الأقمار وتغرب الشمس في تلك الأيام
\par 10 وإذا جُمعت خمس سنوات، يكون للشمس فائض قدره ثلاثون يومًا، وجميع الأيام التي تتراكم عليها في إحدى تلك السنوات الخمس، عندما تكتمل، تبلغ 364 يومًا
\par 11 ويكون فائض الشمس والنجوم ستة أيام: في خمس سنوات تصبح ستة أيام كل عام 30 يومًا: ويتأخر القمر عن الشمس والنجوم بمقدار 30 يومًا
\par 12 والشمس والنجوم تُدخل جميع السنين بدقة، فلا تتقدم ولا تؤخر موقعها يومًا واحدًا إلى الأبد؛ بل تُكمل السنين بعدل كامل في 364 يومًا
\par 13 في 3 سنوات يوجد 1092 يومًا، وفي 5 سنوات يوجد 1820 يومًا، بحيث يكون في 8 سنوات 2912 يومًا
\par 14 بالنسبة للقمر وحده، يبلغ مجموع الأيام في 3 سنوات 1062 يومًا،
\par 15 وفي 5 سنوات تتأخر 50 يومًا: [أي إلى المجموع (1770) هناك 5 يجب إضافتها (1000 و) 62 يومًا.] وفي 5 سنوات هناك 1770 يومًا، بحيث تبلغ أيام القمر 6 في 8 سنوات 21832 يومًا.
\par 16 [لأنها في 8 سنوات تتأخر بمقدار 80 يومًا]، فإن جميع الأيام الـ 17 التي تتأخر عنها في 8 سنوات هي 80 يومًا. وتكتمل السنة بدقة وفقًا لمحطاتها العالمية ومحطات الشمس، التي تشرق من البوابات التي تشرق وتغرب من خلالها 30 يومًا

\chapter{75}

\par 1 "وأما رؤساء الآلاف الذين وضعوا على الخليقة كلها وعلى كل النجوم، فلهم أيضاً علاقة بالأيام الأربعة الإضافية، كونها غير قابلة للفصل عن مناصبهم، حسب حساب السنة، وهؤلاء يقدمون الخدمة في الأيام الأربعة التي لا تحسب في حساب السنة.
\par 2 وبسببهم يخطئ البشر فيها، لأن هؤلاء النجوم يخدمون حقًا محطات العالم، واحد في البوابة الأولى، وواحد في البوابة الثالثة من السماء، وواحد في البوابة الرابعة، وواحد في البوابة السادسة، ويتم تحقيق دقة السنة من خلال محطاتها الثلاثمائة والأربعة والستين المنفصلة
\par 3 لأن العلامات والأوقات والسنين والأيام التي أراني إياها الملاك أورييل، الذي جعله رب المجد إلى الأبد على جميع كواكب السماء، في السماء وفي العالم، لكي يحكموا على وجه السماء ويُرى على الأرض، ويكونوا قادة النهار والليل، أي الشمس والقمر والنجوم، وجميع المخلوقات الخادمة التي تدور في جميع مركبات السماء
\par 4 وبالمثل، أراني أورييل اثنا عشر بابًا، مفتوحة في محيط عربة الشمس في السماء، تنبثق من خلالها أشعة الشمس: ومنها تنتشر الدفء على الأرض، عندما تُفتح في مواسمها المحددة
\par 5 [وللرياح وروح الندى عندما تنفتح، تقف مفتوحة في السماوات عند الأطراف.]
\par 6 وأما الأبواب الاثني عشر في السماء، في أقاصي الأرض، التي تخرج منها الشمس والقمر والنجوم، وجميع أعمال السماء في المشرق والمغرب،
\par 7 هناك العديد من النوافذ المفتوحة على يسارها ويمينها، ونافذة واحدة في موسمها (المحدد) تُنتج دفئًا، يُطابق (كما تفعل) تلك الأبواب التي تخرج منها النجوم كما أمرها، والتي تغرب فيها بما يتناسب مع عددها
\par 8 ورأيت عربات في السماء، تجري في العالم، فوق تلك البوابات التي تدور فيها النجوم التي لا تغرب أبدًا
\par 9 وواحد أكبر من كل الباقي، وهو الذي يشق طريقه عبر العالم بأسره

\chapter{76}

\par 1 وفي أقاصي الأرض رأيت اثني عشر بابًا مفتوحة لجميع أركان السماء، ومنها تخرج الرياح وتهب على الأرض
\par 2 ثلاثة منها مفتوحة على وجه السماء (أي الشرق)، وثلاثة في الغرب، وثلاثة عن يمين السماء (أي الجنوب)، وثلاثة عن يسار السماء (أي الشمال).
\par 3 والثلاثة الأوائل هم من المشرق، وثلاثة من الشمال، وثلاثة [بعد الذين على اليسار] من الجنوب، وثلاثة من الغرب.
\par 4 ومن خلال أربعة منها تأتي رياح البركة والازدهار، ومن الثمانية تأتي رياح ضارة: عندما يتم إرسالها، فإنها تجلب الدمار على كل الأرض وعلى الماء عليها، وعلى كل من يسكن عليها، وعلى كل ما هو في الماء وعلى الأرض.
\par 5 "والريح الأولى من تلك الأبواب، المسماة الريح الشرقية، تخرج من الباب الأول الذي في الشرق، المائل نحو الجنوب: ومنها يخرج الخراب والجفاف والحرارة والدمار."
\par 6 ومن خلال البوابة الثانية في المنتصف يأتي ما هو مناسب، ومنه يأتي المطر والخصوبة والرخاء والندى؛ ومن خلال البوابة الثالثة التي تقع نحو الشمال يأتي البرد والجفاف
\par 7 وبعد هذه تخرج رياح الجنوب من خلال ثلاث بوابات: من البوابة الأولى منها المائلة نحو الشرق تخرج ريح حارة
\par 8 ومن خلال البوابة الوسطى المجاورة لها تخرج روائح عطرة، وندى ومطر، ورخاء وصحة
\par 9 ومن خلال البوابة الثالثة الواقعة إلى الغرب يخرج الندى والمطر والجراد والخراب
\par 10 وبعد هذه الرياح الشمالية: من الباب السابع في الشرق يأتي الندى والمطر والجراد والخراب
\par 11 ومن البوابة الوسطى تأتي في اتجاه مباشر الصحة والمطر والندى والرخاء؛ ومن البوابة الثالثة في الغرب تأتي السحابة والصقيع، والثلج والمطر، والندى والجراد
\par 12 وبعد هذه [الأربع] تأتي الرياح الغربية: من خلال البوابة الأولى المجاورة للشمال يخرج الندى والصقيع، والبرد والثلج والصقيع
\par 13 ومن الباب الأوسط يخرج الندى والمطر، والرخاء والبركة؛ ومن الباب الأخير المجاور للجنوب يخرج الجفاف والخراب، والحرق والدمار
\par 14 وقد أُكملت بذلك أبواب أركان السماء الاثنتي عشرة، وجميع شرائعها وجميع آثامها وجميع حسناتها قد أريتك يا ابني متوشالح

\chapter{77}

\par 1 ويُسمى الربع الأول الشرق، لأنه الأول: ويُسمى الربع الثاني الجنوب، لأن العلي سينزل هناك، نعم، هناك بمعنى خاص جدًا سينزل من هو مبارك إلى الأبد
\par 2 ويُسمى الربع الغربي بالربع المتناقص، لأن جميع أنوار السماء تتضاءل هناك وتهبط
\par 3 والربع الرابع، المسمى بالشمال، مقسم إلى ثلاثة أجزاء: الأول منها لسكنى البشر؛ والثاني يحتوي على بحار من الماء، وهاويات وغابات وأنهار، وظلام وسحب؛ والثالث يحتوي على حديقة البر
\par 4 رأيت سبعة جبال عالية، أعلى من جميع الجبال التي على الأرض، ومن هناك يخرج صقيع، وتمضي الأيام والفصول والسنين
\par 5 رأيت سبعة أنهار على الأرض أكبر من جميع الأنهار: أحدها قادم من الغرب يصب مياهه في البحر الكبير
\par 6 وهذان الاثنان يأتيان من الشمال إلى البحر ويصبان مياههما في البحر الإريتري في الشرق.
\par 7 والباقون أربعة يخرجون من جهة الشمال إلى بحرهم، اثنان منهم إلى بحر إريثر، واثنان إلى البحر الكبير ويفرغون أنفسهم هناك [ويقول البعض: في الصحراء].
\par 8 رأيت سبع جزر عظيمة في البحر وفي البر الرئيسي: اثنتان في البر الرئيسي وخمس في البحر الكبير.

\chapter{78}

\par 1 وأسماء الشمس هي التالية: الأولى أورخاريس، والثانية توماس
\par 2 وللقمر أربعة أسماء: الاسم الأول أسونجا، والثاني إيبلا، والثالث بيناس، والرابع إيراي
\par 3 هذان هما النوران العظيمان: محيطهما كمحيط السماء، وحجم محيط كليهما متساوٍ
\par 4 يوجد في محيط الشمس سبعة أجزاء من الضوء، تُضاف إليها أكثر من القمر، وبمقاييس محددة، تنتقل حتى ينفد الجزء السابع من الشمس
\par 5  وينطلقون ويدخلون من بوابات الغرب، ويدورون باتجاه الشمال، ويخرجون من البوابات الشرقية على وجه السماء
\par 6 وعندما يشرق القمر يظهر جزء من أربعة عشر جزءًا في السماء: [يكتمل نوره فيه]: في اليوم الرابع عشر يُكمل نوره
\par 7 وينتقل إليها خمسة عشر جزءًا من النور حتى اليوم الخامس عشر (عندما) يكتمل نورها، حسب علامة السنة، فتصبح خمسة عشر جزءًا، وينمو القمر بمقدار أربعة عشر جزءًا
\par 8 وفي تناقصه، يتناقص (القمر) في اليوم الأول إلى أربعة عشر جزءًا من نوره، وفي اليوم الثاني إلى ثلاثة عشر جزءًا من نوره، وفي اليوم الثالث إلى اثني عشر جزءًا، وفي اليوم الرابع إلى أحد عشر جزءًا، وفي اليوم الخامس إلى عشرة أجزاء، وفي اليوم السادس إلى تسعة أجزاء، وفي اليوم السابع إلى ثمانية أجزاء، وفي اليوم الثامن إلى سبعة أجزاء، وفي اليوم التاسع إلى ستة أجزاء، وفي اليوم العاشر إلى خمسة أجزاء، وفي اليوم الحادي عشر إلى أربعة أجزاء، وفي اليوم الثاني عشر إلى ثلاثة أجزاء، وفي اليوم الثالث عشر إلى جزأين، وفي اليوم الرابع عشر إلى نصف جزء من سبعة أجزاء، ويختفي كل نوره المتبقي تمامًا في اليوم الخامس عشر
\par 9 وفي بعض الأشهر يكون عدد أيام الشهر تسعة وعشرون يومًا، وفي مرة يكون ثمانية وعشرون يومًا
\par 10 وأراني أورييل قانونًا آخر: متى ينتقل الضوء إلى القمر، وعلى أي جانب ينتقل إليه بواسطة الشمس
\par 11 خلال كل الفترة التي ينمو فيها القمر في ضوئه، فإنه ينقله إلى نفسه. عندما يكون في مواجهة الشمس لمدة أربعة عشر يومًا، يكتمل ضوؤه في السماء، وعندما يكون مضاءً بالكامل، يكتمل ضوؤه في السماء
\par 12 وفي اليوم الأول يُطلق عليه اسم الهلال، لأنه في ذلك اليوم يشرق عليه النور
\par 13 يصبح القمر بدرًا تمامًا في اليوم الذي تغرب فيه الشمس في الغرب، ويشرق من الشرق ليلًا، ويظل القمر ساطعًا طوال الليل حتى تشرق الشمس في مقابله ويُرى القمر في مقابل الشمس
\par 14 على الجانب الذي ينبعث منه ضوء القمر، يتضاءل مرة أخرى حتى يختفي كل الضوء وتنتهي جميع أيام الشهر، ويصبح محيطه فارغًا، خاليًا من الضوء
\par 15 وثلاثة أشهر تكون ثلاثين يوما، وفي وقتها تكون ثلاثة أشهر كل شهر تسعة وعشرون يوما، وفيها تكمل نقصانها في الفترة الأولى من الزمن، وفي الباب الأول مائة وسبعة وسبعين يوما.
\par 16 وفي وقت خروجها تظهر ثلاثة أشهر كل منها ثلاثون يومًا، وثلاثة أشهر كل منها تسعة وعشرون
\par 17 في الليل تظهر كرجل لمدة عشرين يومًا في كل مرة، وفي النهار تظهر كالسماء، ولا يوجد فيها شيء آخر سوى نورها

\chapter{79}

\par 1 والآن يا بني، لقد أريتك كل شيء، واكتمل قانون جميع نجوم السماء
\par 2 وأراني جميع قوانين هذه لكل يوم، ولكل فصل من فصول السنة، ولكل سنة، ولخروجها، وللترتيب الموصوف لها كل شهر وكل أسبوع:
\par 3 وتناقص القمر الذي يحدث في البوابة السادسة: ففي هذه البوابة السادسة يكتمل نوره، وبعد ذلك يبدأ التناقص:
\par 4 (والتناقص) الذي يحدث في الباب الأول في موسمه، حتى يتم مائة وسبعة وسبعين يومًا: محسوبة بالأسابيع، خمسة وعشرون (أسبوعًا) ويومان
\par 5 إنها تتخلف عن الشمس وعن ترتيب النجوم خمسة أيام بالضبط في فترة زمنية واحدة، وذلك عندما يتم اجتياز هذا المكان الذي تراه
\par 6 هذه هي الصورة والرسم التخطيطي لكل منير الذي أراني إياه أورييل رئيس الملائكة، وهو قائدهم

\chapter{80}

\par 1 وفي تلك الأيام أجابني الملاك أورييل وقال لي: "انظر، لقد أريتك كل شيء يا أخنوخ، وكشفت لك كل شيء حتى ترى هذه الشمس وهذا القمر، وقادة نجوم السماء وكل من يديرها، ومهامهم وأوقاتهم ومغادراتهم
\par 2 وفي أيام الخطاة تقصر السنين، ويتأخر زرعهم في أراضيهم وحقولهم، وكل شيء على الأرض يتغير، ولا يظهر في وقته: ويمتنع المطر وتمنعه ​​السماء.
\par 3 وفي تلك الأوقات تتأخر ثمار الأرض، فلا تنمو في وقتها، وتتوقف ثمار الأشجار في وقتها.
\par 4 والقمر يغير ترتيبه فلا يظهر في وقته.
\par 5 [وفي تلك الأيام تظهر الشمس وتسير في المساء على طرف المركبة العظيمة في الغرب] وتشرق أكثر إشراقا مما يتوافق مع نظام النور.
\par 6 وكثير من رؤساء الكواكب يتعدون على النظام (المنصوص عليه)، ويغيرون مداراتهم ومهامهم، ولا يظهرون في المواسم المحددة لهم.
\par 7 وسيُخفى نظام النجوم بأكمله عن الخطاة، وستضل أفكار أهل الأرض بشأنها، [وسيتغيرون عن كل طرقهم]، نعم، سيضلون ويتخذونها آلهة
\par 8 "ويكثر عليهم الشر، ويأتي عليهم العقاب حتى يهلكون الجميع."

\chapter{81}

\par 1 وقال لي: "انظر يا أخنوخ إلى هذه الألواح السماوية، واقرأ ما هو مكتوب عليها، وحدد كل حقيقة على حدة."
\par 2 ونظرتُ إلى الألواح السماوية، وقرأتُ كل ما هو مكتوب عليها، وفهمتُ كل شيء، وقرأتُ كتاب جميع أعمال البشرية، وجميع أبناء الجسد الذين سيكونون على الأرض إلى أبعد الأجيال
\par 3 وباركتُ في الحال الرب العظيم ملك المجد إلى الأبد، لأنه صنع كل أعمال العالم، وسبّحتُ الرب بسبب صبره، وباركته بسبب بني البشر
\par 4 وبعد ذلك قلت: «طوبى للرجل الذي يموت في بر وصلاح، الذي لم يُكتب عنه كتاب إثم، ولن يوجد يوم دينونة عليه».
\par 5 فأحضرني أولئك القديسون السبعة ووضعوني على الأرض أمام باب بيتي، وقالوا لي: "أخبر ابنك متوشالح بكل شيء، وأر جميع أبنائك أنه ليس هناك جسد بار في نظر الرب، لأنه هو خالقهم."
\par 6 «سنتركك مع ابنك لمدة عام، حتى تعطي أوامرك (الأخيرة)، لكي تُعلّم أولادك وتُدوّنها لهم، وتشهد على جميع أبنائك؛ وفي السنة الثانية سيأخذونك من بينهم.»
\par 7 «ليكن قلبك قويًا، لأن الصالحين يُعلنون البر للصالحين، والبار مع الصالحين يفرح، ويُهنئ بعضهم بعضًا.»
\par 8 «لكن الخطاة يموتون مع الخطاة، والمرتد يهبط مع المرتد.»
\par 9 «والذين يفعلون البر سيموتون بسبب أعمال الناس، ويؤخذون بسبب أعمال الأشرار.»
\par 10 وفي تلك الأيام كفوا عن مخاطبتي، فجئت إلى شعبي، مباركًا رب العالمين

\chapter{82}

\par 1 والآن يا ابني متوشالح، كل هذه الأمور أرويها لك وأكتبها لك! وقد كشفت لك كل شيء، وأعطيتك كتبًا تتعلق بكل هذه الأمور: فاحفظ يا ابني متوشالح الكتب من يد أبيك، وتأكد من تسليمها لأجيال العالم
\par 2 لقد أعطيتك ولأبنائك الحكمة، [وأبنائك الذين سيكونون لك]، لكي يعطوها لأبنائهم مدى أجيال، هذه الحكمة (أي) التي تفوق فكرهم.
\par 3 وأما الذين يفهمونها فلا ينامون، بل يستمعون بالأذن لكي يتعلموا هذه الحكمة، وهي تطيب للذين يأكلونها أكثر من الطعام الجيد.
\par 4 طوبى لجميع الصالحين، طوبى لجميع الذين يسلكون في طريق البر ولا يخطئون كالخطاة، في حساب كل أيامهم التي فيها تجوب الشمس السماء، تدخل وتخرج من الأبواب لمدة ثلاثين يوما مع رؤوس آلاف من رتبة النجوم، مع الأربعة المتداخلة التي تقسم الأجزاء الأربعة من السنة، التي تقودهم وتدخل معهم أربعة أيام.
\par 5 بسببهم، سيُخطئ الناس ولن يُحاسبوهم في حساب السنة بأكملها: نعم، سيُخطئ الناس ولن يُدركوهم بدقة
\par 6 لأنها تنتمي إلى حساب السنة، وهي مسجلة حقًا (عليها) إلى الأبد، واحدة في البوابة الأولى، وواحدة في الثالثة، وواحدة في الرابعة، وواحدة في السادسة، وتكتمل السنة في ثلاثمائة وأربعة وستين يومًا
\par 7 ورواية ذلك دقيقة، والحساب المسجل لها دقيق؛ لأن النجوم، والأشهر، والأعياد، والسنين والأيام، أراني إياها وكشفها لي أورييل، الذي أخضع له رب كل خليقة العالم جند السماء
\par 8 وله سلطان على الليل والنهار في السماء ليجعل النور ينير البشر - الشمس والقمر والنجوم، وجميع قوى السماء التي تدور في مركباتها الدائرية
\par 9 وهذه هي ترتيبات النجوم التي تغرب في أماكنها، وفي فصولها وأعيادها وأشهرها
\par 10 وهذه أسماء من يقودهم، ويراقب دخولهم في أوقاتهم، وفي أوامرهم، وفي مواسمهم، وفي أشهرهم، وفي فترات سلطانهم، وفي مناصبهم
\par 11 يدخل قادتهم الأربعة الذين يقسمون أجزاء السنة الأربعة أولاً؛ وبعدهم قادة الرتب الاثنا عشر الذين يقسمون الأشهر؛ وبالنسبة للثلاثمائة وستين (يومًا) هناك رؤساء على الآلاف الذين يقسمون الأيام؛ وبالنسبة للأيام الأربعة الإضافية هناك القادة الذين يفصلون أجزاء السنة الأربعة
\par 12 وهؤلاء الرؤوس التي تزيد عن الآلاف موزعة بين قائد وقائد، كل منهم خلف محطة، لكن قادتهم هم من يقومون بالتقسيم
\par 13 وهذه هي أسماء القادة الذين يقسمون أجزاء السنة الأربعة التي تُرسم: ملكيئيل، وهلمالك، ومليعال، وناريل
\par 14 وأسماء من يقودهم: أدنارائيل، وإيجاسوسائيل، وإيلوميل - هؤلاء الثلاثة يتبعون قادة الطوائف، وهناك واحد يتبع قادة الطوائف الثلاثة الذين يتبعون قادة المحطات التي تقسم أجزاء السنة الأربعة
\par 15 في بداية العام، ينهض ملكجال أولاً ويحكم، ويُدعى تمعيني والشمس، وتكون جميع أيام حكمه أثناء توليه الحكم واحدًا وتسعين يومًا
\par 16 وهذه هي علامات الأيام التي تظهر على الأرض في أيام سلطانه: العرق والحر والهدوء، وجميع الأشجار تحمل ثمارًا، وتنمو الأوراق على جميع الأشجار، وحصاد القمح، وزهور الورد، وجميع الزهور التي تنبت في الحقل، ولكن أشجار فصل الشتاء تذبل.
\par 17 وهذه أسماء القادة الذين تحتهم: بركائيل، وزالبئيل، وآخر يضاف إليه رئيس ألف، يُدعى هيلوجاسف. وقد انتهت أيام سلطان هذا (القائد).
\par 18 القائد التالي بعده هو هلمالك، الذي يُسمى الشمس الساطعة، وكل أيام نوره واحد وتسعون يومًا
\par 19 وهذه علامات أيامه على الأرض: حر شديد وجفاف، وأشجار تنضج ثمارها وتعطي كل ثمارها ناضجة وجاهزة، وغنم تتزوج وتحمل، وتجمع كل ثمار الأرض، وكل ما في الحقول، ومعصرة العنب: هذه الأشياء تحدث في أيام سلطانه
\par 20 هذه هي أسماء ورتب ورؤساء أولئك رؤساء الألوف: جيدالجال، وقئيل، وهائيل، واسم رئيس الألف الذي يضاف إليهم، أسفائيل. وقد انتهت أيام ملكه

\part {القسم الرابع. الفصول ٨٣-XC. رؤى الأحلام.}

\chapter{83}

\par 1 والآن يا ابني متوشالح، سأريك كل رؤياي التي رأيتها، وأرويها أمامك
\par 2 رأيت رؤيتين قبل أن أتزوج، وكانت إحداهما مختلفة تمامًا عن الأخرى: الأولى عندما كنت أتعلم الكتابة، والثانية قبل أن أتزوج أمك، (عندما) رأيت رؤيا مروعة. وبخصوصهما صليت إلى الرب
\par 3 كنت قد استلقيت في بيت جدي مهللئيل، (عندما) رأيت في رؤيا كيف انهارت السماء وحُملت عنها وسقطت على الأرض
\par 4 وعندما سقطت على الأرض، رأيت كيف ابتُلِعَت الأرض في هاوية عظيمة، وجبال معلقة على جبال، وتلال غارقة على تلال، وأشجار عالية تمزقت من جذوعها، وأُلقيت وغرقت في الهاوية
\par 5 ثم وقعت كلمة في فمي، فرفعت صوتي لأصرخ بصوت عالٍ، وقلت: "دمرت الأرض".
\par 6 وأيقظني جدي مهللئيل وأنا مستلقٍ بالقرب منه، وقال لي:
\par 7 «لماذا تبكي هكذا يا بني، ولماذا تبكي هكذا؟» ورويت له الرؤيا كاملة التي رأيتها، فقال لي: «لقد رأيت شيئًا فظيعًا يا بني، ورؤياك في الحلم خطيرة للغاية فيما يتعلق بأسرار خطيئة الأرض كلها: يجب أن تغرق في الهاوية وتُدمر تدميرًا عظيمًا.»
\par 8 «والآن يا بني، قم وتضرع إلى رب المجد، بما أنك مؤمن، أن تبقى بقية على الأرض، وأن لا يُهلك الأرض كلها.»
\par 9 "يا ابني، من السماء سيأتي كل هذا على الأرض، وعلى الأرض يكون دمار عظيم."
\par 10 بعد ذلك، نهضتُ وصليتُ وتوسلتُ وتضرعتُ، وكتبتُ صلاتي من أجل أجيال العالم، وسأريك كل شيء يا ابني متوشالح
\par 11 وعندما خرجتُ إلى الأسفل ورأيتُ السماء، والشمس تشرق في الشرق، والقمر يغرب في الغرب، وبعض النجوم، والأرض كلها، وكل شيء كما عرفه في البدء، باركتُ ربَّ الدين وسبَّحتُه لأنه جعل الشمس تخرج من نوافذ الشرق، فصعد وارتفع على وجه السماء، وانطلق وظلَّ سائرًا في الطريق الذي أُظهر له

\chapter{84}

\par 1 ورفعت يدي بالبر وباركت القدوس العظيم، وتكلمت بنسمة فمي، وبلسان اللحم الذي صنعه الله لبني جسد البشر ليتكلموا به، وأعطاهم نفسًا ولسانًا وفمًا ليتكلموا به
\par 2 «تبارك أنت أيها الرب الملك، العظيم الجبار في عظمتك، رب كل خليقة السماء، ملك الملوك وإله العالم أجمع. وقوتك وملكك وعظمتك تدوم إلى أبد الآبدين، وعلى مر الأجيال كلها سلطانك، وكل السماوات عرشك إلى الأبد، والأرض كلها موطئ قدميك إلى أبد الآبدين.»
\par 3 «لأنك أنت خلقت كل شيء وتحكمه، ولا شيء يعسر عليك، والحكمة لا تفارق مكان عرشك، ولا تبتعد عن حضرتك. وأنت تعلم وترى وتسمع كل شيء، ولا يخفى عليك شيء [لأنك ترى كل شيء]».
\par 4 «والآن ملائكة سماواتك مذنبون بالتعدي، وعلى أجساد البشر يبقى غضبك إلى يوم الدينونة العظيم.»
\par 5 «والآن، يا الله والرب والملك العظيم، أتوسل إليك وأتوسل إليك أن تُلبي صلاتي، وأن تترك لي ذرية على الأرض، ولا تُهلك كل جسد بشري، وتجعل الأرض بلا ساكن، فيكون هناك دمار أبدي.»
\par 6 «والآن يا سيدي، أمحِ من الأرض الجسد الذي أثار غضبك، لكن جسد البر والاستقامة ثبّته كنبتة من البذرة الأبدية، ولا تحجب وجهك عن صلاة عبدك يا ​​رب.»

\chapter{85}

\par 1 وبعد هذا رأيتُ حلمًا آخر، وسأُريك الحلم كله يا ابني. ورفع أخنوخ صوته وكلم ابنه متوشالح:
\par 2 "إليك يا ابني أتكلم. اسمع كلامي، أمل أذنك إلى رؤيا أبيك."
\par 3 "قبل أن أتخذ أمك إيدنا، رأيت في رؤيا على فراشي، وإذا بثور يخرج من الأرض، وكان ذلك الثور أبيض؛ وبعدها خرجت عجلة، ومعها (الأخيرة) خرجت ثوران، أحدهما أسود والآخر أحمر."
\par 4 "وذلك الثور الأسود نطح الثور الأحمر وطارده فوق الأرض، وبعد ذلك لم أعد أستطيع رؤية ذلك الثور الأحمر."
\par 5 «لكن ذلك الثور الأسود كبر، وذهبت تلك العجلة معه، ورأيت أن ثيرانًا كثيرة خرجت منه تشبهه وتتبعه.»
\par 6 «وذهبت تلك البقرة، تلك الأولى، من أمام ذلك الثور الأول بحثًا عن ذلك الثور الأحمر، لكنها لم تجده، فناحت عليه نحيبًا عظيمًا وبحثت عنه.»
\par 7 «ونظرتُ حتى جاءها ذلك الثور الأول وأسكتها، ومنذ ذلك الوقت لم تعد تبكي.»
\par 8 «وبعد ذلك أنجبت ثورًا أبيض آخر، وبعده أنجبت العديد من الثيران والأبقار السوداء.»
\par 9 «ورأيت في نومي أن الثور الأبيض ينمو أيضًا ويصبح ثورًا أبيض كبيرًا، ومنه خرج العديد من الثيران البيضاء، وكانت تشبهه.»
\par 10 «وبدأوا يلدون ثيرانًا بيضًا كثيرة تشبههم، واحدًا تلو الآخر، كثيرًا.»

\chapter{86}

\par 1 ثم نظرت بعينيَّ أيضًا وأنا نائم، ورأيت السماء من فوق، وإذا نجمة سقطت من السماء، فقامت وأكلت ورعت بين تلك الثيران
\par 2 وبعد ذلك رأيت الثيران الكبيرة والسوداء، وإذا بها جميعًا قد غيرت حظائرها ومراعيها وماشيتها، وبدأت تعيش مع بعضها البعض
\par 3 ثم رأيت في الرؤيا مرة أخرى، ونظرت نحو السماء، وإذا بي أرى نجومًا كثيرة تنزل وتسقط من السماء على ذلك النجم الأول، فصارت ثيرانًا بين تلك الماشية وترعى معها
\par 4 ونظرت إليهم ورأيت، وإذا بهم جميعًا قد أخرجوا أعضاءهم التناسلية، مثل الخيول، وبدأوا في تغطية أبقار الثيران، فحملوا جميعًا وولدوا فيلة وجمالًا وحميرًا
\par 5 فخافتهم كل الثيران وارتاعوا منهم وابتدأت تعض بأسنانها وتأكل وتنطح بقرونها.
\par 6 ثم ابتدأوا يأكلون تلك الثيران، وإذا كل بني الأرض يرتعدون ويرتعدون أمامهم ويهربون منهم.

\chapter{87}

\par 1 ورأيت مرة أخرى كيف بدأوا ينطحون بعضهم بعضًا ويلتهمون بعضهم بعضًا، وبدأت الأرض تصرخ بصوت عالٍ
\par 2 ورفعت عيني إلى السماء أيضاً، ورأيت في الرؤيا، وإذا بكائنات خرجت من السماء مثل الرجال البيض، وخرج أربعة من ذلك المكان وثلاثة معهم.
\par 3 وأولئك الثلاثة الذين خرجوا أخيرا أمسكوا بيدي ورفعوني من بين أجيال الأرض ورفعوني إلى مكان عالٍ وأروني برجا عاليا فوق الأرض وكانت كل الجبال منخفضة.
\par 4 فقال لي أحدهم: «ابق هنا حتى ترى كل ما يحدث لتلك الأفيال، والجمال، والحمير، والنجوم، والثيران، وكلها».

\chapter{88}

\par 1 ورأيت واحداً من هؤلاء الأربعة الذين خرجوا أولاً، فأمسك بالنجم الأول الذي سقط من السماء، وقيده من يديه وقدميه، وألقاه في الهاوية: وكانت تلك الهاوية ضيقة وعميقة، ورهيبة ومظلمة.
\par 2 فاستل أحدهم سيفًا وأعطاه لتلك الفيلة والجمال والحمير، فبدأوا يضربون بعضهم بعضًا، وارتجت الأرض كلها منهم
\par 3 وبينما كنت أنظر في الرؤيا، إذا بواحد من أولئك الأربعة الذين خرجوا يرجمهم من السماء، وجمع وأخذ كل النجوم العظيمة التي كانت أعضاؤها التناسلية مثل أعضائها التناسلية للخيول، وقيدها جميعًا من الأيدي والأرجل، وألقاها في هاوية الأرض

\chapter{89}

\par 1 وذهب أحد هؤلاء الأربعة إلى ذلك الثور الأبيض وأرشده سرًا، دون أن يخاف: وُلد ثورًا وأصبح رجلًا، وبنى لنفسه إناءً عظيمًا وسكن عليه؛ وسكن معه ثلاثة ثيران في ذلك الإناء، وكانوا مغطون به
\par 2 ثم رفعت عيني نحو السماء فرأيت سقفا عاليا وعليه سبعة سيول ماء، وكانت تلك السيول تتدفق بماء كثير إلى مكان مسور.
\par 3 ثم نظرت أيضا وإذا ينابيع قد انفتحت على سطح ذلك المكان العظيم، وبدأ الماء يتضخم ويرتفع على السطح، ونظرت ذلك المكان حتى غطى الماء كل سطحه.
\par 4 وازداد عليه الماء والظلام والضباب، وعندما نظرت إلى ارتفاع ذلك الماء، رأيته قد ارتفع فوق ارتفاع ذلك المكان المغلق، وكان يتدفق فوق ذلك المكان المغلق، ووقف على الأرض
\par 5 فجمعت كل ماشية تلك الحظيرة حتى رأيتها تغرق وتبتلع وتفنى في تلك المياه
\par 6 لكن تلك السفينة طفت على الماء، بينما غرقت جميع الثيران والفيلة والجمال والحمير في القاع مع جميع الحيوانات، حتى لم أعد أستطيع رؤيتها، ولم يتمكنوا من الهرب، (بل) هلكوا وغرقوا في الأعماق
\par 7 ورأيت مرة أخرى في الرؤيا حتى أُزيلت تلك السيول المائية من ذلك السقف العالي، وسُوّيت هاويات الأرض وانفتحت هاويات أخرى
\par 8 ثم بدأ الماء يتدفق إلى داخلها، حتى أصبحت الأرض مرئية؛ لكن ذلك الإناء استقر على الأرض، فانسحب الظلام وظهر النور
\par 9 لكن ذلك الثور الأبيض الذي أصبح إنسانًا خرج من ذلك الإناء، والثيران الثلاثة التي معه، وكان واحد من هؤلاء الثلاثة أبيض مثل ذلك الثور، وواحد أحمر كالدم، وواحد أسود. فانصرف عنهم ذلك الثور الأبيض
\par 10 وابتدأوا يلدون وحوش البرية والطيور، فظهرت أجناس مختلفة: أسود، ونمور، وذئاب، وكلاب، وضباع، وخنازير برية، وثعالب، وسناجب، وخنازير، وصقور، ونسور، وحداوات، ونسور، وغربان؛ وولد بينها ثور أبيض
\par 11 فبدأوا يعضون بعضهم بعضا، ولكن الثور الأبيض الذي ولد بينهم ولد حماراً وحشياً وثوراً أبيض معه، فتكاثرت الحمير الوحشية.
\par 12 أما ذلك الثور الذي ولد منه فولد خنزيرًا بريًا أسود وخروفًا أبيض؛ فالأول ولد خنازير برية كثيرة، أما ذلك الخروف فولد اثني عشر خروفًا
\par 13 ولما كبرت تلك الخراف الاثنا عشر، سلموا واحدًا منها للحمير، فسلمت تلك الحمير ذلك الخروف مرة أخرى للذئاب، ونشأ ذلك الخروف بين الذئاب
\par 14 فأحضر الرب الإحدى عشر خروفًا ليعيشوا معها ويرعوا معها بين الذئاب، فتكاثرت وصارت قطعان غنم كثيرة
\par 15 فخافتهم الذئاب، واضطهدوهم حتى أهلكوا صغارهم، وألقوا صغارهم في نهر كثير الماء، فبدأت تلك الغنم تصرخ بصوت عالٍ على صغارها، وتشتكي إلى ربها
\par 16 وهربت خروفة نجت من الذئاب إلى الحمير الوحشية؛ ورأيت الخراف كيف كانت تبكي وتبكي، وتتوسل إلى ربها بكل قوتها، حتى نزل سيد الخراف عند صوت الخراف من مسكن عالٍ، وجاء إليها ورعاها
\par 17 فدعا الخروف الذي نجا من الذئاب، وتكلم معه عن الذئاب لكي يحذرهم من لمس الخراف
\par 18 فذهبت الخراف إلى الذئاب حسب كلمة الرب، فالتقاها خروف آخر وذهب معها، وذهب الاثنان ودخلا معًا في جماعة أولئك الذئاب، وتكلما معهم وحذروهم من أن يمسوا الخراف من الآن فصاعدًا
\par 19 وعندئذٍ رأيت الذئاب، وكيف اضطهدت الخراف بشدة بكل قوتها، وصاحت الخراف بصوت عالٍ
\par 20 فجاء الرب إلى الخراف فبدأت تضرب تلك الذئاب، فبدأت الذئاب بالنواح، أما الخراف فسكتت وتوقفت في الحال عن الصراخ
\par 21 ورأيت الخراف حتى انصرفت من بين الذئاب؛ لكن عيون الذئاب أُعميت، فانصرفت تلك الذئاب في مطاردة الخراف بكل قوتها
\par 22 وكان رب الغنم يسير معهم قائدًا لهم، وكانت جميع خرافه تتبعه، وكان وجهه باهرًا ومجيدًا ومرعبًا للنظر
\par 23 لكن الذئاب بدأت بمطاردة تلك الأغنام حتى وصلت إلى بحر من الماء
\par 24 وانشق البحر، فوقفت المياه من هنا ومن هناك أمام وجوههم، وقادهم ربهم ووضع نفسه بينهم وبين الذئاب
\par 25 وبما أن تلك الذئاب لم ترَ الخراف بعد، فقد سارت إلى وسط ذلك البحر، وتبعت الذئاب الخراف، وركضت [تلك الذئاب] وراءها إلى ذلك البحر
\par 26 ولما رأوا رب الغنم، التفتوا ليهربوا من وجهه، لكن ذلك البحر استجمع قواه، وصار كما خُلِق، وتضخم الماء وارتفع حتى غطى تلك الذئاب
\par 27 ورأيت حتى هلكت جميع الذئاب التي طاردت تلك الخراف وغرقت
\par 28 لكن الخراف هربت من ذلك الماء وخرجت إلى برية لا ماء فيها ولا عشب، فابتدأت تفتح أعينها وتبصر، فرأيت رب الخراف يرعاها ويسقيها وكلبها، وتلك الخراف ماشية وتقودها
\par 29 وصعد ذلك الخروف إلى قمة تلك الصخرة العالية، فأرسله رب الخراف إليهم
\par 30 وبعد ذلك رأيتُ ربَّ الخراف الواقف أمامهم، وكان منظره عظيمًا ومهيبًا ومهيبًا، فرأته كل تلك الخراف وخافت من وجهه
\par 31 فخافوا جميعًا وارتعدوا منه، وصرخوا إلى الخروف الذي كان معهم [الذي كان بينهم]: «لا نستطيع أن نقف أمام ربنا أو أن ننظر إليه».
\par 32 وصعد الخروف الذي قادهم مرة أخرى إلى قمة تلك الصخرة، لكن الخروف بدأ يُعمى ويضل عن الطريق الذي أراهم إياه، لكن الخروف لم يعرف ذلك
\par 33 فغضب رب الغنم عليهم غضبًا شديدًا، فاكتشفت تلك الغنم ذلك، ونزل من قمة الصخرة، وجاء إلى الغنم، فوجد أكثرهم أعمى وساقطًا
\par 34 فلما رأوه خافوا وارتعدوا من وجوده، وأرادوا الرجوع إلى حظائرهم
\par 35 فأخذت تلك الخروف معها خرافًا أخرى، وجاءت إلى تلك الخراف التي سقطت، وبدأت تقتلها؛ فخافت الخراف من وجودها، وهكذا أعادت تلك الخروف تلك الخراف التي سقطت، فعادت إلى حظائرها
\par 36 ورأيت في هذه الرؤيا حتى صار ذلك الخروف رجلاً وبنى بيتًا لسيد الخراف، ووضع جميع الخراف في ذلك البيت
\par 37 ونظرت حتى نام هذا الخروف الذي التقى بالخروف الذي قادهم. ونظرت حتى هلكت جميع الخراف الكبيرة وقامت صغارها في مكانها، وجاءت إلى مرعى واقتربت من جدول ماء
\par 38 ثم انصرف ذلك الخروف، قائدهم الذي صار إنسانًا، عنهم ونام، فطلبته كل الخراف وصرخت عليه صراخًا عظيمًا
\par 39 ورأيت حتى توقفوا عن البكاء على ذلك الخروف وعبروا ذلك الجدول المائي، وهناك نهض الخروفان قائدين في مكان الخروفين اللذين قاداهما وناموا (حرفيًا "ناموا وقادوهم").
\par 40 ونظرت حتى أتت الخراف إلى مكان جيد، وأرض جميلة ومجيدة، ونظرت حتى شبعت تلك الخراف، ووقف ذلك البيت بينها في الأرض الجميلة
\par 41 وكانت أعينهم تنفتح أحيانًا، وتعمى أحيانًا أخرى، حتى يقوم خروف آخر ويقودهم ويعيدهم جميعًا، فتنفتح أعينهم
\par 42 وبدأت الكلاب والثعالب والخنازير البرية تأكل تلك الغنم حتى أقام سيد الغنم كبشًا من وسطها، فقادها
\par 43 وبدأ ذلك الكبش ينطح الكلاب والثعالب والخنازير البرية من كلا الجانبين حتى أهلكهم جميعًا.
\par 44 فأبصرت تلك الخروف المفتوحة العينين ذلك الكبش الذي كان بين الخراف، حتى تخلى عن مجده وبدأ ينطح تلك الخراف، ويدوسها، ويتصرف بشكل غير لائق
\par 45 وأرسل رب الخراف الخروف إلى خروف آخر ورفعه ليكون كبشًا وقائدًا للخراف بدلًا من ذلك الكبش الذي ترك مجده
\par 46 فذهب إليه وتحدث إليه وحده، ورفعه إلى كبش، وجعله أميرًا وقائدًا للخراف؛ ولكن خلال كل هذه الأمور، كانت تلك الكلاب تضايق الخراف
\par 47 فطارد الكبش الأول ذلك الكبش الثاني، فقام الكبش الثاني وهرب أمامه، ونظرت حتى جذبت تلك الكلاب الكبش الأول
\par 48 فقام الكبش الثاني وقاد الخراف الصغيرة. فولد ذلك الكبش خرافًا كثيرة ونام، فصار خروف صغير كبشًا مكانه، وأصبح أميرًا وقائدًا لتلك الخراف
\par 49 ونمت تلك الأغنام وتكاثرت، لكن جميع الكلاب والثعالب والخنازير البرية خافت وهربت من أمامه، ونطح ذلك الكبش الوحوش وقتلها، ولم يعد لتلك الوحوش أي سلطان بين الأغنام ولم تعد تسلبها ما تستحقه
\par 50 وأصبح ذلك البيت عظيمًا وواسعًا، وبُني لتلك الغنم: وبُني برج عالٍ وعظيم على البيت لرب الغنم، وكان ذلك البيت منخفضًا، لكن البرج كان مرتفعًا وعاليًا، ووقف رب الغنم على ذلك البرج، وقدموا أمامه مائدة كاملة
\par 51 ورأيت مرة أخرى تلك الخراف أنها ضلت طريقها وذهبت في طرق كثيرة، وتركت بيتها، فدعا سيد الخراف بعضًا من بين الخراف وأرسلهم إلى الخراف، لكن الخراف بدأت تقتلهم
\par 52 فنجا واحد منها ولم يُقتل، فانطلق وصرخ بصوت عالٍ على الخراف، فطلبوا قتله، لكن رب الخراف أنقذه من الخراف، وأصعده إليّ، وأسكنه هناك
\par 53 وأرسل خرافًا أخرى كثيرة إلى أولئك الخراف لتشهد لهم وتندبهم
\par 54 وبعد ذلك رأيت أنهم لما تركوا بيت الرب وبرجه سقطوا تمامًا، وأُعميَت أعينهم؛ ورأيت رب الخراف كيف كان يُكثر من الذبح بينهم في قطعانهم حتى دعت تلك الخراف إلى تلك الذبح وخانت مكانه
\par 55 وأسلمهم إلى أيدي الأسود والنمور والذئاب والضباع، وإلى أيدي الثعالب، وإلى كل الوحوش، فابتدأت تلك الوحوش تمزق تلك الغنم
\par 56 ورأيت أنه ترك بيتهم وبرجهم وأسلمهم جميعًا إلى أيدي الأسود لتمزقهم وتفترسهم، إلى أيدي كل الوحوش البرية
\par 57 فبدأت أصرخ بكل قوتي وأتوسل إلى رب الغنم وأعرض عليه عن الغنم أنها افترست من قبل كل الوحوش البرية.
\par 58 لكنه ظل ثابتًا، مع أنه رأى ذلك، وفرح لأنهم التُهموا وابتُلعوا وسُلِبوا، وتركهم ليُلتهموا في أيدي جميع الوحوش
\par 59 فدعا سبعين راعيًا، وألقى إليهم تلك الغنم ليرعوها، وقال للرعاة ورفقائهم: «ليرع كل واحد منكم الغنم من الآن فصاعدًا، وافعلوا كل ما أوصيكم به».
\par 60 «وسأسلمهم إليكم معدودين، وأخبركم من منهم يُهلك، فتُهلكوهم». فأسلم إليهم تلك الخراف
\par 61 ثم دعا آخر وقال له: «انظر ولاحظ كل ما سيفعله الرعاة بأولئك الخراف؛ لأنهم سيهلكون منهم أكثر مما أمرتهم به».
\par 62 «وكل إفراط ودمار سيحدث من خلال الرعاة، سجل عدد الذين يدمرونهم وفقًا لأمري، وعدد الذين يدمرونهم وفقًا لأهوائهم: سجل ضد كل راعٍ كل الدمار الذي يسببه.»
\par 63 «واقرأوا أمامي بالعدد كم يهلكون، وكم يسلمون للهلاك، ليكون ذلك شهادةً عليهم، وأعرف كل عمل من أعمال الرعاة، حتى أفهم وأرى ما يفعلونه، سواءً التزموا بأمري الذي أمرتهم به أم لا.»
\par 64 «لكنهم لن يعرفوا ذلك، ولن تخبرهم به، ولن تحذرهم، بل سجل فقط ضد كل فرد كل الدمار الذي أحدثه الرعاة كلٌ في وقته، واعرضه أمامي كله.»
\par 65 ونظرت حتى رعى أولئك الرعاة في أوانهم، فبدأوا يقتلون ويهلكون أكثر مما أُمروا به، وأسلموا تلك الخراف إلى أيدي الأسود
\par 66 وأكلت الأسود والنمور وأكلت الجزء الأكبر من تلك الأغنام، وأكلت معها الخنازير البرية، وأحرقت ذلك البرج وهدمت ذلك المنزل
\par 67 وحزنتُ حزنًا شديدًا على ذلك البرج لأن بيت الخراف هُدم، وبعد ذلك لم أتمكن من رؤية ما إذا كانت تلك الخراف قد دخلت ذلك البيت
\par 68 فسلم الرعاة وشركاؤهم تلك الغنم إلى جميع الوحوش لتفترسها، وكان كل واحد منهم يأخذ في وقته عددًا محددًا: وقد كتب الآخر في كتاب عدد ما أهلكه كل واحد منهم
\par 69 وكل واحد منهم قتل وأهلك أكثر بكثير مما هو مقرر، وبدأت أبكي وأنوح على تلك الخراف
\par 70 وهكذا في الرؤيا رأيت ذلك الكاتب، كيف كتب كل ما أهلكه أولئك الرعاة، يومًا بعد يوم، وحمله ووضعه وأظهره بالفعل لرب الخراف - (حتى) كل ما فعلوه، وكل ما تخلص منه كل واحد منهم، وكل ما قدموه للهلاك.
\par 71 وقُرئ السفر أمام رب الخراف، فأخذ السفر من يده وقرأه وختمه ووضعه
\par 72 وفي الحال رأيتُ كيف رعى الرعاة اثنتي عشرة ساعة، وإذا بثلاثة من تلك الخراف قد رجعوا وجاءوا ودخلوا وبدأوا في بناء كل ما سقط من ذلك البيت؛ لكن الخنازير البرية حاولت منعهم، لكنها لم تستطع
\par 73 وبدأوا مرة أخرى في البناء كما في السابق، وبنوا ذلك البرج، وسُمي البرج العالي؛ وبدأوا مرة أخرى في وضع مائدة أمام البرج، لكن كل الخبز عليها كان نجسًا وغير نقي
\par 74 ومن جهة كل هذا، عمى عيون أولئك الخراف فلم يبصروا، وكذلك عيون رعاتهم، فسلموهم بأعداد كبيرة إلى رعاتهم للهلاك، فداسوا الخراف بأقدامهم وأكلوها
\par 75 وظل سيد الغنم ثابتًا حتى تفرقت جميع الغنم في الحقل واختلطت بها (أي بالحيوانات)، ولم ينقذوها (أي الرعاة) من أيدي الوحوش
\par 76 فحمله كاتب السفر وأراه وقرأه أمام رب الخراف، وتضرع إليه لأجلهم، وتضرع إليه لأجلهم، وأراه جميع أعمال الرعاة، وشهد أمامه على جميع الرعاة
\par 77 فأخذ الكتاب الحقيقي ووضعه بجانبه وانصرف

\chapter{90}

\par 1 ورأيت حتى ذلك الحين أن خمسة وثلاثين راعيًا تولوا رعي الأغنام بهذه الطريقة، وأكملوا فتراتهم على حدة كما فعل الأولون؛ وتسلمهم آخرون في أيديهم، لرعيهم في فتراتهم، كل راعٍ في فتراته
\par 2 وبعد ذلك رأيت في رؤياي جميع طيور السماء قادمة، النسور، النسور، الحدأة، الغربان؛ لكن النسور قادت جميع الطيور؛ وبدأت في التهام تلك الأغنام، وتفقيع عيونها، والتهام لحومها
\par 3 وصاحت الخراف لأن الطيور كانت تأكل لحمها، أما أنا فنظرت ورثيت في نومي على ذلك الراعي الذي كان يرعى الخراف
\par 4 ونظرت حتى أكلت الكلاب والنسور والحدأة تلك الغنم، ولم يتركوا عليها لحمًا ولا جلدًا ولا عصبًا حتى وقفت عظامها فقط، وسقطت عظامها أيضًا على الأرض، وقلّت الغنم
\par 5 ورأيت حتى أن ثلاثة وعشرين قد تولوا الرعي وأكملوا في فتراتهم المختلفة ثمانية وخمسين مرة
\par 6 فإذا بالخراف البيضاء تحمل خرافاً، فبدأت تفتح عيونها وتبصر وتصرخ إلى الخراف.
\par 7 نعم، صرخوا إليهم، لكنهم لم يصغوا إلى ما قالوه لهم، بل كانوا صُمًّا جدًّا، وعميت أعينهم جدًّا جدًّا
\par 8 ورأيت في الرؤيا كيف طارت الغربان على تلك الحملان، وأخذت أحد تلك الحملان، وحطمت الخراف وأكلتها
\par 9 ونظرت حتى نمت قرون على تلك الحملان، فألقت الغربان قرونها. ونظرت حتى أنبت قرنًا عظيمًا على أحد تلك الخراف، فانفتحت أعينها
\par 10 فنظر إليهم [وانفتحت أعينهم]، ونادى الخراف، فرأته الكباش وركضت إليه كلها
\par 11 وعلى الرغم من كل هذا، استمرت تلك النسور والعقبان والغربان والحدأة في تمزيق الأغنام والانقضاض عليها وافتراسها: ظلت الأغنام صامتة، لكن الكباش ندبت وصرخت
\par 12 وقاتلته تلك الغربان وحاربته وحاولت أن تخفض قرنه، لكن لم يكن لها عليه سلطان
\par 13 اجتمعت كل النسور والعقبان والغربان والحدأة، وجاءت معها كل غنم الحقل، نعم، اجتمعوا جميعًا، وساعدوا بعضهم البعض على كسر قرن الكبش
\par 14 [الآية 17 مكررة]
\par 15 [الآية 18 مكررة]
\par 16 [الآية 13 مكررة]
\par 17 ورأيت ذلك الرجل الذي كتب السفر حسب أمر الرب، حتى فتح ذلك السفر المتعلق بالتدمير الذي أحدثه أولئك الرعاة الاثنا عشر الأخيرون، وأظهر أنهم أهلكوا أكثر بكثير من أسلافهم، أمام رب الخراف
\par 18 ونظرت حتى جاء إليهم رب الغنم وأخذ في يده عصا غضبه، وضرب الأرض، فانشقت الأرض، وسقطت كل الوحوش وكل طيور السماء من بين تلك الغنم، وابتلعتها الأرض وغطتها
\par 19 ونظرت حتى أُعطي سيف عظيم للخراف، فقامت الخراف على جميع وحوش البرية لتقتلها، فهربت جميع الوحوش وطيور السماء من أمامها
\par 20 ونظرت حتى نُصب عرش في الأرض الهنيئة، وجلس عليه رب الخراف، وأخذ الآخر الأسفار المختومة وفتح تلك الأسفار أمام رب الخراف
\par 21 ودعا الرب أولئك الرجال البيض السبعة الأوائل، وأمرهم أن يحضروا أمامه، بدءًا من النجم الأول الذي كان يقود الطريق، جميع النجوم التي كانت أعضاؤها السرية كأعضائها الخاصة للخيول، وأحضروها جميعًا أمامه
\par 22 فقال لذلك الرجل الذي كان يكتب أمامه، وكان أحد أولئك السبعة البيض، وقال له: «خذ أولئك السبعين راعيًا الذين سلمت إليهم الخراف، والذين أخذوها على أنفسهم قتلوا أكثر مما أمرتهم».
\par 23 وإذا هم جميعًا مقيدين، رأيتهم، ووقفوا جميعًا أمامه
\par 24 وأُجري الحكم أولاً على النجوم، فحُكم عليهم ووجدوا مذنبين، وذهبوا إلى مكان الدينونة، وأُلقوا في هاوية مليئة بالنار واللهب، ومليئة بأعمدة من نار.
\par 25 وحُكم على هؤلاء الرعاة السبعين ووُجدوا مذنبين، وأُلقوا في تلك الهاوية النارية
\par 26 ورأيت في ذلك الوقت كيف انفتحت هاوية مماثلة في وسط الأرض، مملوءة نارًا، وأحضروا أولئك الخراف العمياء، وحوكموا جميعًا ووجدوا مذنبين وأُلقوا في هذه الهاوية النارية، واحترقوا. وكانت هذه الهاوية على يمين ذلك البيت
\par 27 ورأيت تلك الأغنام تحترق وعظامها تحترق.
\par 28 ثم وقفت لأنظر حتى طواوا ذلك البيت العتيق، وحملوا جميع الأعمدة، وطوا جميع عوارض البيت وزخارفه في وقت واحد، وحملوه ووضعوه في مكان في جنوب الأرض.
\par 29 ونظرت حتى أتى رب الغنم ببيت جديد أعظم وأعلا من الأول، وأقامه في موضع الأول الذي طوى: جميع أعمدته جديدة، وزينته جديدة وأعظم من زينة الأول العتيق الذي أزاله، وجميع الغنم كانت في داخله
\par 30 ورأيت جميع الخراف التي بقيت، وجميع حيوانات الأرض، وجميع طيور السماء، خرّت وسجدت لتلك الخراف، وتوسلت إليها، وأطاعتها في كل شيء
\par 31 وبعد ذلك، أمسك بي أولئك الثلاثة الذين كانوا لابسين ثيابًا بيضًا، وأمسكوا بيدي [الذين حملوني من قبل]، وأمسكتني يد ذلك الكبش أيضًا، وأقاموني وأنزلوني في وسط تلك الخراف قبل أن يأتي يوم الدينونة
\par 32 وكانت تلك الأغنام كلها بيضاء، وكان صوفها وفيرًا ونظيفًا
\par 33 واجتمع كل الذين هلكوا وتبددوا، وكل وحوش الحقل، وكل طيور السماء، في ذلك البيت، وفرح رب الغنم فرحًا عظيمًا لأنهم جميعًا صالحون ورجعوا إلى بيته
\par 34 ونظرت حتى وضعوا ذلك السيف الذي أُعطي للخراف، وأعادوه إلى البيت، وخُتم أمام الرب، ودُعيت جميع الخراف إلى ذلك البيت، لكنه لم يمسكها
\par 35 فانفتحت أعينهم جميعًا، وأبصروا الصالح، ولم يكن فيهم أحد إلا وهو مبصر
\par 36 ورأيت أن ذلك البيت كان كبيرًا وواسعًا وممتلئًا جدًا.
\par 37 ورأيت أنه قد ولد ثور أبيض ذو قرون كبيرة، وكانت كل وحوش الحقل وكل طيور السماء تخافه وتتضرع إليه كل الوقت.
\par 38 ونظرت حتى تحولت جميع أجيالهم، وصاروا جميعًا ثيرانًا بيضاء، وصار أولهم حملًا، وصار ذلك الحمل حيوانًا عظيمًا، وله قرون سوداء عظيمة على رأسه، وفرح به رب الخراف وبجميع الثيران
\par 39 ونمت في وسطهم، ثم استيقظت فرأيت كل شيء.
\par 40 هذه هي الرؤيا التي رأيتها وأنا نائم، فاستيقظت وباركت رب البر ومجدته
\par 41 ثم بكيت بكاءً عظيمًا، ولم تتوقف دموعي حتى لم أعد أستطيع تحملها: عندما رأيت، انهمرت دموعي بسبب ما رأيته؛ لأن كل شيء سيأتي ويتحقق، وقد أُظهرت لي جميع أعمال البشر بترتيبها
\par 42 في تلك الليلة تذكرت الحلم الأول، وبسببه بكيت واضطربت - لأني رأيت تلك الرؤية

\part {القسم الخامس. XCI-CIV (أي XCII، XCI.1-10، 18-19، XCIII.1-10، XCI.12-17، XCIV-CIV.). كتاب موعظة و طوبى للمتقين و لعنة و ويل للمذنبين } .

\chapter{91}

\par 1 «والآن يا ابني متوشالح، ادعُ إليّ جميع إخوتك، واجمع إليّ جميع أبناء أمك؛ لأن الكلمة تدعوني، والروح يُسكب عليّ، لأريك كل ما سيصيبك إلى الأبد.»
\par 2 فذهب متوشالح ودعا جميع إخوته وجمع أقرباءه
\par 3 وكلم جميع أبناء البر وقال: «اسمعوا يا بني أخنوخ جميع أقوال أبيكم، وأصغوا جيدًا إلى صوت فمي. لأني أحثكم وأقول لكم أيها الأحباء: أحبوا الاستقامة وامشوا بها».
\par 4 «ولا تقتربوا إلى الاستقامة بقلبٍ ذي قلبٍ مُنثني، ولا تُخالطوا ذوي القلوب المُنثنية، بل اسلكوا في البر يا أبنائي، فيهديكم إلى سبل الخير، ويكون البر رفيقكم.»
\par 5 «لأني أعلم أنه لا بد أن يكثر العنف على الأرض، وأن يُنفذ على الأرض تأديب عظيم، وأن ينتهي كل إثم. نعم، سيُقطع من جذوره، ويُهلك هيكله كله.»
\par 6 «وسيُستكمل الإثم مرة أخرى على الأرض، وستسود جميع أعمال الإثم والعنف والتعدي بدرجتين.»
\par 7 «وعندما تكثر الخطيئة والإثم والتجديف والعنف في جميع أنواع الأعمال، ويكثر الردة والمعصية والنجاسة، سيأتي عقاب عظيم من السماء على كل هؤلاء، وسيخرج الرب القدوس بغضب وتأديب ليُنفِّذ الدينونة على الأرض.»
\par 8 «في تلك الأيام يُقطع الظلم من جذوره، وجذور الإثم والغش، ويُبادون من تحت السماء.»
\par 9 «وسيتم التخلي عن جميع أصنام الوثنيين، وحرق معابدهم بالنار، وإزالتها من كل الأرض، وإلقائهم (أي الوثنيين) في دينونة النار، ويهلكون في غضب ودينونة مؤلمة إلى الأبد.»
\par 10 «ويستيقظ الصديقون من نومهم، وتستيقظ الحكمة وتُعطى لهم.»
\par 11 "[وبعد ذلك تُقطع جذور الإثم، ويُهلك الخطاة بالسيف... ويُقطع المجدفون في كل مكان، ويهلك بالسيف أولئك الذين يخططون للعنف والذين يرتكبون التجديف.]"
\par 12 [يجب قراءة هذه الآية بعد الإصحاح 93] 12 وبعد ذلك يكون أسبوع آخر، وهو الأسبوع الثامن، أسبوع البر، ويُعطى له سيف ليُنفذ حكم عادل على الظالمين، ويُسلم الخطاة إلى أيدي الأبرار
\par 13 [يجب قراءة هذه الآية بعد الإصحاح 93] 13 وفي ختامها يكتسبون بيوتًا ببرهم، ويُبنى بيت للملك العظيم في مجد إلى الأبد، 14د، ويتطلع كل البشر إلى طريق الاستقامة
\par 14 [يجب قراءة هذه الآية بعد الإصحاح 93] [أ] وبعد ذلك، في الأسبوع التاسع، سيُكشف عن الدينونة العادلة للعالم أجمع، [ب] وستختفي جميع أعمال الكافرين من كل الأرض، [ج] وسيُكتب العالم للهلاك
\par 15 [يجب قراءة هذه الآية بعد الإصحاح 93] وبعد ذلك، في الأسبوع العاشر في الجزء السابع، ستكون هناك الدينونة الأبدية العظيمة، والتي سينفذ فيها الانتقام بين الملائكة
\par 16 [يجب قراءة هذه الآية بعد الإصحاح 93] والسماء الأولى ستزول وتزول، وستظهر سماء جديدة، وستعطي جميع قوات السماوات نورًا سبعة أضعاف
\par 17 [يجب قراءة هذه الآية بعد الإصحاح 93] وبعد ذلك ستكون هناك أسابيع عديدة لا تُحصى إلى الأبد، وسيكون الجميع في خير وبر، ولن تُذكر الخطيئة بعد الآن إلى الأبد
\par 18 والآن أخبركم يا أبنائي، وأريكم سبل البر وسبل الظلم. نعم، سأريكم إياها أيضًا لتعلموا ما سيحدث
\par 19 والآن، اسمعوا لي يا أبنائي، وامشوا في سبل البر، ولا تسيروا في سبل الظلم، لأن كل من يسير في سبل الإثم يهلكون إلى الأبد

\chapter{92}

\par 1 الكتاب الذي كتبه أخنوخ - لقد كتب أخنوخ بالفعل هذه العقيدة الكاملة للحكمة، (التي) يُمدحها جميع البشر، وهو ديان كل الأرض - لجميع أبنائي الذين سيسكنون على الأرض. وللأجيال القادمة التي ستراعي الاستقامة والسلام
\par 2 لا تضطرب أرواحكم بسبب الأوقات، لأن القدوس العظيم قد عيّن أيامًا لكل شيء
\par 3 ويستيقظ البار من نومه، ويسلك في سبل البر، ويكون كل طريقه وسلوكه في صلاح ونعمة أبديين
\par 4 سيُنعم على البار ويمنحه استقامة أبدية، وسيمنحه قوةً ليكون (مُنوحًا) بالصلاح والاستقامة. وسيسير في النور الأبدي
\par 5 والخطيئة تهلك في الظلمة إلى الأبد، ولن تظهر بعد ذلك من ذلك اليوم إلى الأبد.

\chapter{93}

\par 1 وبعد ذلك أعطى أخنوخ وبدأ يروي من الكتب
\par 2 وقال أخنوخ: "فيما يتعلق بأبناء البر، ومختاري العالم، وغرس الاستقامة، سأتحدث بهذه الأمور، نعم، سأخبركم بها أنا أخنوخ يا أبنائي: وفقًا لما ظهر لي في الرؤيا السماوية، وما عرفته من خلال كلمة الملائكة القديسين، وتعلمته من الألواح السماوية."
\par 3 وبدأ أخنوخ يروي من الكتب وقال: "لقد وُلدتُ في السابع من الأسبوع الأول، بينما كان العدل والبر لا يزالان قائمين."
\par 4 «وبعدي سيقوم في الأسبوع الثاني شر عظيم، وينمو الغش، وفيه تكون النهاية الأولى. وفيه يخلص الإنسان، وبعد انتهائه ينمو الإثم، ويُسن قانون للخطاة.»
\par 5 «وبعد ذلك، في الأسبوع الثالث عند نهايته، يُنتخب رجلٌ ليكون غرسًا للحكم الصالح، وتصبح نسله غرسًا للبر إلى الأبد.»
\par 6 «وبعد ذلك في الأسبوع الرابع، عند ختامه، ستظهر رؤى القديسين والأبرار، وسيُصنع لهم شريعة لجميع الأجيال وسياج.»
\par 7 «وبعد ذلك، في الأسبوع الخامس، عند ختامه، يُبنى بيت المجد والسلطان إلى الأبد.»
\par 8 «وبعد ذلك في الأسبوع السادس، سيُصاب كل من يعيش فيه بالعمى، وستتخلى قلوبهم جميعًا عن الحكمة بغير إخلاص. وفيه يصعد إنسان؛ وفي ختامه يُحرق بيت السيادة بالنار، ويتبدد كل جنس الأصل المختار.»
\par 9 «وبعد ذلك في الأسبوع السابع يقوم جيل مرتد، وتكون أعماله كثيرة، وتكون جميع أعماله مرتدة.»
\par 10 «وفي ختامها يُنتخب الأبرار المختارون من غرس البر الأبدي، ليتلقوا تعليمًا سبعة أضعاف بشأن كل خليقته.»
\par 11 «فمن من جميع بني البشر من يستطيع أن يسمع صوت القدوس دون أن ينزعج؟ ومن يستطيع أن يفكر في أفكاره؟ ومن يستطيع أن يرى جميع أعمال السماء؟»
\par 12 «وكيف يمكن أن يوجد من يستطيع أن ينظر إلى السماء، ومن يستطيع أن يفهم أمور السماء ويرى نفسًا أو روحًا ويستطيع أن يخبر عنها، أو يصعد ويرى جميع غاياتها ويفكر فيها أو يفعل مثلها؟»
\par 13 «ومن من كل البشر يستطيع أن يعرف ما هو عرض الأرض وطولها، ولمن أُري قياسها كلها؟»
\par 14 «أم من أحد يستطيع أن يميز طول السماء، وكم يبلغ ارتفاعها، وعلى ماذا قامت، وكم يبلغ عدد النجوم، وأين تستقر جميع النيّرات؟»



\chapter{94}

\par 1 والآن أقول لكم يا أبنائي: أحبوا البر وامشوا فيه، لأن سبل البر جديرة بالقبول، أما سبل الإثم فسوف تُدمر فجأة وتختفي.
\par 2 ولرجال من جيلٍ ما، ستُكشف لهم سبل العنف والموت، فيبتعدون عنها ولا يتبعونها
\par 3 والآن أقول لكم أيها الأبرار: لا تسيروا في سبل الشر، ولا في سبل الموت، ولا تقتربوا منها لئلا تهلكوا
\par 4 بل اطلبوا واختاروا لأنفسكم البر والحياة المختارة، وامشوا في سبل السلام، فتحيون وتفلحون
\par 5 واحفظوا كلامي في أفكار قلوبكم، ولا تدعوه يمحى من قلوبكم؛ لأني أعلم أن الخطاة سيُغريون الناس بطلب الحكمة بالشر، فلا يوجد لها مكان، ولا ينقص أي نوع من الإغراء
\par 6 ويلٌ للذين يبنون الإثم والظلم ويضعون الغش أساسًا؛ لأنهم سيُقلبون فجأةً، ولن يكون لهم سلام
\par 7 ويلٌ للذين يبنون بيوتهم بالخطيئة، لأنهم سيهدمون من جميع أساساتهم، وبالسيف يسقطون. [والذي يقتني الذهب والفضة في الدينونة يبيد فجأة.]
\par 8 ويل لكم أيها الأغنياء، لأنكم توكلتم على غناكم، ومن غناكم تخرجون، لأنكم لم تذكروا العلي في أيام غناكم
\par 9 لقد ارتكبتم التجديف والإثم، وأصبحتم مستعدين ليوم القتل، ويوم الظلمة، ويوم الدينونة العظيمة
\par 10 هكذا أقول وأعلن لكم: الذي خلقكم سيُسقطكم، ولن تكون هناك رحمة لسقوطكم، وسيفرح خالقكم بهلاككم
\par 11 ويكون أبرارك في تلك الأيام عارا للخطاة والكافرين

\chapter{95}

\par 1 ليت عينيّ كانتا سحابة من مياه فأبكي عليك، وأسكب دموعي كسحابة من مياه: فأستريح من تعب قلبي!
\par 2 من سمح لكم بممارسة العار والشر؟ وهكذا ينزل عليكم العقاب أيها الخطاة!
\par 3 لا تخافوا من الخطاة أيها الصديقون، لأن الرب سيسلمهم أيضاً إلى أيديكم، لكي تجروا عليهم حكماً حسب رغباتكم.
\par 4 ويلٌ لكم أيها الذين تُشنّون حُرُومًا لا تُردّ! فالشفاءُ بعيدٌ عنكم بسبب خطاياكم!
\par 5 ويل لكم أيها الذين تجازون قريبكم بالشر، لأنكم ستُجازون حسب أعمالكم!
\par 6 ويل لكم أيها الشهود الكاذبون، ويا ​​من يزنون الظلم، لأنكم تهلكون فجأة!
\par 7 ويل لكم أيها الخطاة لأنكم تضطهدون الصديقين، لأنكم ستُسلمون وتُضطهدون بسبب الظلم، ويكون نيره ثقيلاً عليكم.

\chapter{96}

\par 1 إنتظروا أيها الصديقون، لأنه بغتة يهلك الخطاة أمامكم، وتتسلطون عليهم حسب رغباتكم.
\par 2 [وفي يوم ضيق الخطاة، سيرتفع أبناؤكم ويرتفعون كالنسور، وسيكون عشكم أعلى من النسور، وستصعدون وتدخلون شقوق الأرض، وشقوق الصخور إلى الأبد كالأرنب أمام الأشرار، وستتنهد صفارات الإنذار بسببكم وتبكي.]
\par 3 لذلك لا تخافوا أيها الذين عانيتم؛ لأن الشفاء سيكون نصيبكم، ونور ساطع سينيركم، وصوت الراحة ستسمعونه من السماء
\par 4 ويل لكم أيها الخطاة، فإن ثرواتكم تجعلكم تبدون كالأبرار، لكن قلوبكم تدينكم بأنكم خطاة، وستكون هذه الحقيقة شهادة عليكم لتذكرة أعمالكم الشريرة!
\par 5 ويل لكم أيها الذين تأكلون شحم الحنطة، وتشربون الخمر في أوانٍ كبيرة، وتدوسون المساكين بقوتكم!
\par 6 ويل لكم أيها الشاربون للماء من كل ينبوع، لأنكم فجأةً ستهلكون وتذبلون، لأنكم تركتم ينبوع الحياة!
\par 7 ويل لكم أيها العاملون في الإثم والغش والتجديف! سيكون لكم تذكارًا على الشر!
\par 8 ويل لكم أيها الجبابرة، الذين تظلمون الصديق بقوة؛ لأن يوم هلاككم آتٍ! في تلك الأيام، ستأتي أيام كثيرة وجيدة على الصديقين - في يوم دينونتكم

\chapter{97}

\par 1 آمنوا أيها الأبرار أن الخطاة سيصيرون عارًا ويهلكون في يوم الإثم
\par 2 فليكن معلوماً لديكم (أيها الخطاة) أن العلي يذكر هلاككم، وملائكة السماء تفرح بهلاككم.
\par 3 ماذا ستفعلون أيها الخطاة وإلى أين ستهربون في يوم الدينونة حين تسمعون صوت صلاة الصديقين؟
\par 4 نعم، سوف تعاملون مثل أولئك الذين ستكون هذه الكلمة شهادة ضدهم: "لقد كنتم رفقاء الخطاة".
\par 5 وفي تلك الأيام تصل صلاة الصديقين إلى الرب، وتأتي عليكم أيام دينونتكم.
\par 6 "وتُقرأ جميع أقوال إثمكم أمام القدوس العظيم، وتُغطى وجوهكم بالخزي، ويرفض كل عمل مؤسس على الإثم.
\par 7 ويل لكم أيها الخطاة الساكنون في وسط البحر وعلى اليابسة، الذين ذكراهم شر عليكم!
\par 8 ويل لكم أيها الذين تقتنون الفضة والذهب بالظلم وتقولون: لقد أغنينا بالثروات وامتلكنا الممتلكات وحصلنا على كل ما تمنيناه.
\par 9 «والآن فلنفعل ما نوينا، فقد جمعنا فضة!» [ج] «والكرارون كثيرون في بيوتنا!» [د] «وصوامعنا ممتلئة كالماء!»
\par 10 نعم، ومثل الماء، ستتدفق أكاذيبكم بعيدًا؛ لأن ثرواتكم لن تدوم، بل سترتفع عنكم سريعًا؛ لأنكم اكتسبتموها كلها بالإثم، وستُسلمون إلى لعنة عظيمة

\chapter{98}

\par 1 والآن أقسم لكم، للحكماء والجهلاء، لأنكم ستخوضون تجارب متعددة على الأرض
\par 2 لأنكم ستلبسون زينة أكثر من المرأة، وملابس ملونة أكثر من العذراء. في الملوكية والعظمة والسلطة، والفضة والذهب والأرجوان، والبهاء والطعام، سيُسكبون كالماء
\par 3 لذلك سيفتقرون إلى التعليم والحكمة، وسيهلكون بذلك مع ممتلكاتهم؛ ومع كل مجدهم وروعتهم، وفي العار والقتل والعوز الشديد، ستُلقى أرواحهم في أتون النار!
\par 4 لقد أقسمت لكم أيها الخطاة، كما أن الجبل لم يصر عبدًا، والتل لم يصبح خادمًا لامرأة، كذلك لم تُرسل الخطيئة إلى الأرض، بل الإنسان نفسه خلقها، وتحت لعنة عظيمة سيسقط من يرتكبها
\par 5 ولم تُعطَ المرأة عقمًا، بل بسبب أعمال يديها تموت بلا أولاد
\par 6 لقد أقسمت لكم، أيها الخطاة، بالقدوس العظيم، أن جميع أعمالكم الشريرة ستنكشف في السماوات، وأن لا شيء من أعمالكم الظالمة سيُغطى أو يُخفى
\par 7 ولا تفكر في روحك ولا تقل في قلبك إنك لا تعلم ولا ترى أن كل خطيئة تُسجل كل يوم في السماء أمام العلي
\par 8 من الآن فصاعدًا، تعلمون أن كل ظلمكم الذي تظلمون به مكتوب كل يوم إلى يوم دينونتكم
\par 9 ويل لكم أيها الجهال، لأنكم بحماقتكم تهلكون، وتتجاوزون عن الحكماء، فلا يكون لكم حظ سعيد!
\par 10 والآن، اعلموا أنكم مستعدون ليوم الهلاك: لذلك لا تأملوا في الحياة أيها الخطاة، بل سترحلون وتموتون؛ لأنكم لا تعرفون فدية؛ لأنكم مستعدون ليوم الدينونة العظيمة، ليوم الضيق والعار العظيم لأرواحكم
\par 11 ويل لكم أيها المعاندون في القلوب، فاعلو الشر وآكلي الدماء! من أين لكم الخير لتأكلوه وتشربوه وتشبعوا من جميع الخيرات التي كثرها الرب العلي على الأرض؛ لذلك لن يكون لكم سلام
\par 12 ويل لكم أيها الذين تحبون أعمال الإثم! لماذا ترجون الخير لأنفسكم؟ اعلموا أنكم ستُسلمون إلى أيدي الصالحين، فيقطعون أعناقكم ويقتلونكم، ولا يرحمونكم
\par 13 ويل لكم أيها الفرحون بضيق الصديقين، لأنه لن يُحفر لكم قبر!
\par 14 ويل لكم أيها الذين تستهينون بأقوال الصديقين، لأنه لن يكون لكم رجاء في الحياة
\par 15 ويل لكم أيها الذين تكتبون كلامًا كاذبًا وحاقدًا؛ فإنهم يكتبون أكاذيبهم ليسمعها الناس فيتصرفون بغير ورع مع قريبهم
\par 16 لذلك لن ينعموا بالسلام بل سيموتون موتًا مفاجئًا.

\chapter{99}

\par 1 ويل لكم أيها الذين يعملون الإثم ويفتخرون بالكذب ويفتخرون به. ستهلكون ولن تكون لكم حياة سعيدة.
\par 2 ويلٌ للذين يُحرِّفون كلمات الاستقامة، ويتعدون على الناموس الأبدي، ويغيرون أنفسهم إلى ما لم يكونوا عليه [إلى خطاة]! سيُداسون على الأرض بالأقدام!
\par 3 في تلك الأيام، استعدوا أيها الأبرار لرفع صلواتكم تذكارًا، ووضعها شهادة أمام الملائكة، حتى يضعوا خطيئة الخطاة تذكارًا أمام العلي
\par 4 في تلك الأيام تهتز الأمم، وتقوم قبائل الأمم في يوم الهلاك
\par 5 وفي تلك الأيام يخرج المساكين ويأخذون أولادهم، ويتركونهم، فيهلك أولادهم بسببهم. نعم، يتركون أولادهم الرضع، ولا يعودون إليهم، ولا يرحمون أحباءهم
\par 6 وأقسم لكم مرة أخرى، أيها الخطاة، أن الخطيئة مُعدّة ليوم سفك دماء لا يتوقف
\par 7 وأولئك الذين يعبدون الحجارة، والتماثيل المقبرة من الذهب والفضة والخشب (والحجر) والطين، وأولئك الذين يعبدون الأرواح النجسة والشياطين، وجميع أنواع الأصنام التي لا تتفق مع المعرفة، لن ينالوا أي مساعدة منهم
\par 8 وسيصبحون بلا دين بسبب جهل قلوبهم، وستعمى عيونهم من خوف قلوبهم ومن خلال رؤى أحلامهم
\par 9 بسبب هذه الأمور، سيصبحون بلا دين وخائفين؛ لأنهم سيصنعون كل أعمالهم بالكذب، وسيعبدون حجرًا. لذلك في لحظة سيهلكون!
\par 10 ولكن في تلك الأيام، طوبى لجميع الذين يقبلون كلمات الحكمة، ويفهمونها، ويتبعون سبل العلي، ويسلكون في طريق بره، ولا يصبحون بلا إله مع الأشرار؛ لأنهم سيخلصون
\par 11 ويل لكم أيها الذين تنشرون الشر لجيرانكم؛ لأنكم ستُقتلون في الهاوية!
\par 12 ويلٌ لكم أيها الذين تتخذون إجراءاتٍ خادعةً وكاذبةً، والذين يُسببون مرارةً في الأرض؛ فإنهم بذلك يُفنونَ إلى الأبد!
\par 13 ويل لكم أيها الذين تبنون بيوتكم من خلال تعب الآخرين الشاق، وكل مواد بنائهم هي طوب وحجارة الخطيئة؛ أقول لكم إنه لن يكون لكم سلام!
\par 14 ويلٌ للذين يرفضون نصيب آبائهم وميراثهم الأبدي، والذين تتبع نفوسهم الأصنام؛ لأنهم لن يهدأ لهم بال!
\par 15 ويلٌ للذين يعملون الإثم، ويساعدون على الظلم، ويقتلون جيرانهم إلى يوم الدينونة العظيمة!
\par 16 لأنه سيُسقط مجدكم، ويجلب الضيق على قلوبكم، ويُثير غضبه الشديد، ويُهلككم جميعًا بالسيف، وسيتذكر جميع القديسين والصالحين خطاياكم

\chapter{100}

\par 1 وفي تلك الأيام في مكان واحد يُضرب الآباء مع أبنائهم، ويسقط الإخوة بعضهم مع بعض في الموت حتى تجري الأنهار من دمائهم.
\par 2 لأنه لا يكف الإنسان عن قتل أبنائه وأبناء أبنائه، ولا يكف الخاطئ عن أخيه المكرم: من الفجر إلى غروب الشمس يقتلون بعضهم بعضًا
\par 3 ويصعد الحصان إلى الصدر في دم الخطاة، وتُغمر المركبة حتى ارتفاعها
\par 4 في تلك الأيام، ستنزل الملائكة إلى الأماكن السرية وتجمع في مكان واحد كل أولئك الذين جلبوا الخطيئة، وسيقوم العلي في يوم الدينونة ذلك ليُنفِّذ دينونة عظيمة بين الخطاة
\par 5 وسيُعيِّن على جميع الصالحين والقديسين حُرَّاسًا من بين الملائكة القديسين ليحرسوهم كحدقة عين، حتى يُنهي كل شر وكل خطيئة، ورغم أن الصالحين ينامون نومًا طويلًا، فليس لديهم ما يخشونه
\par 6 وحينئذٍ سيرى بنو الأرض الحكماء في أمان، ويفهمون جميع كلمات هذا الكتاب، ويدركون أن ثرواتهم لن تستطيع إنقاذهم في إسقاط خطاياهم
\par 7 ويل لكم أيها الخطاة في يوم الضيق الشديد، يا من تضايقون الصديق وتحرقونه بالنار: ستُجازون حسب أعمالكم!
\par 8 ويل لكم أيها المعاندون القلب، الساهرون لتدبروا الشر: لذلك سيأتي عليكم الخوف، ولن يكون لكم معين!
\par 9 ويل لكم أيها الخطاة، من أجل أقوال أفواهكم، ومن أجل أعمال أيديكم التي صنعتها إثمكم، ستحرقون بلهيب متقد أشد من النار!
\par 10 والآن، اعلموا أنه سيسألكم من الملائكة عن أعمالكم في السماء، ومن الشمس والقمر والنجوم عن خطاياكم، لأنكم على الأرض تدينون الأبرار
\par 11 ويشهد عليك كل سحابة وضباب وندى ومطر، لأنها ستُمنع جميعها بسببك من النزول عليك، وستذكر خطاياك
\par 12 والآن قدموا هدايا للمطر حتى لا يُمنع من النزول عليكم، ولا للندى أيضًا عندما يتلقى منكم ذهبًا وفضةً حتى ينزل عليكم
\par 13 عندما يقع عليكم الصقيع والثلج مع برودتهما، وجميع العواصف الثلجية بكل أوبئتها، فلن تتمكنوا من الوقوف أمامها في تلك الأيام

\chapter{101}

\par 1 راقبوا السماء يا أبناء السماء وكل عمل العلي، واتقوه ولا تعملوا شرًا في حضرته
\par 2 إذا أغلق نوافذ السماء، ومنع المطر والندى من النزول على الأرض بسببكم، فماذا ستفعلون حينئذ؟
\par 3 وإن أرسل غضبه عليكم بسبب أعمالكم فلا تستطيعون أن تطلبوا منه لأنكم تكلمتم بكلام عظائم وقبيح ضد بره لذلك لا يكون لكم سلام.
\par 4 ألا ترى إلى بحارة السفن، كيف تتقاذف الأمواج سفنهم، وتهزها الرياح، وهي في ضيق شديد؟
\par 5 ولذلك يخافون لأن جميع ممتلكاتهم الثمينة تذهب معهم إلى البحر، ولديهم نذير شؤم في قلوبهم بأن البحر سيبتلعهم ويهلكون فيه
\par 6 أليس البحر كله وكل مياهه وكل تحركاته من عمل العلي، ألم يضع حدودًا لأعماله ويحصره بالرمال من كل جانب؟
\par 7 وعند توبيخه يخاف فيجف، ويموت كل سمكه وكل ما فيه. أما أنتم الخطاة الذين على الأرض فلا تخافوه
\par 8 أليس هو الذي خلق السماء والأرض وكل ما فيهما، والذي أعطى الفهم والحكمة لكل ما يدب في الأرض وفي البحر؟
\par 9 ألا يخاف بحارة السفن البحر؟ أما الخطاة فلا يخافون العلي؟

\chapter{102}

\par 1 في تلك الأيام التي يُنزل عليكم فيها نارًا أليمة، إلى أين تهربون، وأين تجدون الخلاص؟ وعندما يُطلق كلمته عليكم، ألا ترتعبون وتخافون؟
\par 2 وسترتعب جميع النجوم خوفًا عظيمًا، وسترتعب كل الأرض وترتعد وترتعب
\par 3 وسينفذ جميع الملائكة أوامرهم. وسيسعون إلى إخفاء أنفسهم من حضرة المجد العظيم، وسيرتجف أبناء الأرض ويرتجفون؛ وستُلعنون أنتم الخطاة إلى الأبد، ولن يكون لكم سلام
\par 4 لا تخافوا يا نفوس الأبرار، وآملوا أنتم الذين متم في البر.
\par 5 ولا تحزن إذا نزلت نفسك إلى الهاوية بالحزن، وإذا لم يكن جسدك في حياتك صالحًا حسب صلاحك، بل انتظر يوم دينونة الخطاة ويوم اللعنة والتأديب.
\par 6 ومع ذلك، عندما تموت، يتحدث الخطاة عنك: "كما نموت، يموت الصالحون، فما الفائدة التي يحصدونها من أعمالهم؟"
\par 7 "انظر، كما نحن، كذلك يموتون في الحزن والظلام، وماذا لديهم أكثر مما لدينا؟ من الآن فصاعدا نحن متساوون؟"
\par 8 "وماذا سيحصلون وماذا سيبصرون إلى الأبد هوذا هم أيضًا ماتوا ولن يروا النور إلى الأبد."
\par 9 أقول لكم أيها الخطاة، أنتم ترضون أن تأكلوا وتشربوا وتسرقوا وتخطئوا وتعروا الناس وتكتسبوا ثروة وتشاهدوا أياماً طيبة.
\par 10 هل رأيتم الصديقين كيف تكون نهايتهم، حيث لا يوجد فيهم أي نوع من العنف حتى موتهم؟
\par 11 «ومع ذلك هلكوا وأصبحوا كأنهم لم يكونوا، ونزلت أرواحهم إلى الهاوية في ضيق.»

\chapter{103}

\par 1 والآن أقسم لكم أيها الصديقون بمجد العظيم المكرم القادر على كل شيء، وأقسم لكم بعظمته.
\par 2 أعرف لغزًا، وقد قرأت الألواح السماوية، ورأيت الكتب المقدسة، ووجدت مكتوبًا فيها ومنقوشًا عليها:
\par 3 أن كل الخير والفرح والمجد مُعدّ لهم، ومُدوّن لأرواح أولئك الذين ماتوا في البر، وأن الخير المُضاعف سيُعطى لكم مكافأةً على أعمالكم، وأن نصيبكم يفوق نصيب الأحياء بكثير
\par 4 وستحيا أرواحكم التي ماتت في البر وتفرح، ولن تهلك أرواحهم ولا ذكراهم من أمام وجه العظيم إلى جميع أجيال العالم: لذلك لا تخشوا احتقارهم بعد الآن
\par 5 ويل لكم أيها الخطاة، إذا متم، إذا متم في غنى خطاياكم، وقال عنكم أمثالكم: "طوبى للخطاة، فقد رأوا كل أيامهم!"
\par 6 «وكيف ماتوا في رخاء وغنى، ولم يروا فتنة ولا قتلًا في حياتهم؛ وماتوا بشرف، ولم يُنفذ عليهم حكم في حياتهم.»
\par 7 اعلموا أن أرواحهم ستُنزل إلى الهاوية، وسيشقون في ضيقهم العظيم
\par 8 وستدخل أرواحكم إلى الظلمة والسلاسل ولهيب متقد حيث يوجد دينونة مؤلمة؛ وسيكون الدينونة العظيمة لجميع أجيال العالم. ويل لكم، لأنه لن يكون لكم سلام
\par 9 لا تقل عن الصالحين والصالحين في هذه الحياة: "في أيامنا المضطربة، تعبنا كثيرًا وعانينا من كل ضيق، وقابلنا الكثير من الشرور واهلكنا، وأصبحنا قليلين وأرواحنا صغيرة."
\par 10 «ولقد أهلكنا ولم نجد من ينصرنا ولو بكلمة. لقد عذبنا [وأهلكنا]، ولم نرجو أن نرى حياة من يوم إلى يوم.»
\par 11 «كنا نرجو أن نكون رأسًا فصرنا ذنبًا. لقد تعبنا كثيرًا ولم يكن لنا شبع في تعبنا. وصرنا طعامًا للخطاة والأشرار، وقد وضعوا نيرهم علينا بثقل.»
\par 12 «لقد سيطر علينا من أبغضونا وضربونا، وأحنينا أعناقنا لمن أبغضونا، لكنهم لم يشفقوا علينا.»
\par 13 «أردنا أن نبتعد عنهم لننجو ونستريح، لكننا لم نجد مكانًا نهرب إليه ونكون آمنين منهم.»
\par 14 «ولقد تذمرنا إلى الحكام في ضيقتنا، وصرخنا على الذين افترسونا، لكنهم لم يصغوا إلى صراخنا ولم يسمعوا لصوتنا.»
\par 15 "وساعدوا الذين سلبونا وأكلونا والذين استخفوا بنا، وأخفوا ظلمهم، ولم يرفعوا عنا نير الذين أكلونا وفرقونا وقتلونا، وأخفوا قتلهم، ولم يذكروا أنهم رفعوا أيديهم علينا."

\chapter{104}

\par 1 أقسم لكم أن الملائكة في السماء يتذكرونكم جيدًا أمام مجد العظيم، وأسماؤكم مكتوبة أمام مجد العظيم
\par 2 كونوا متفائلين؛ لأنكم كنتم في السابق تُخزون بسبب المرض والبؤس؛ لكنكم الآن ستُضيئون كأنوار السماء، ستُضيئون وستُرى، وستُفتح لكم أبواب السماء
\par 3 وفي صراخكم، اصرخوا من أجل الدينونة، وستظهر لكم؛ لأن كل ضيقكم سيقع على الحكام وعلى كل من ساعد ناهبيكم
\par 4 كن متفائلاً، ولا تتخلى عن آمالك، فستحصل على فرح عظيم كملائكة السماء
\par 5 ماذا ستُلزمون بفعله؟ لن تضطروا للاختباء في يوم الدينونة العظيمة، ولن تُوجدوا كخطاة، وسيكون الدينونة الأبدية بعيدة عنكم لجميع أجيال العالم
\par 6 والآن لا تخافوا أيها الأبرار، عندما ترون الخطاة ينمون أقوياء ويزدهرون في طرقهم. لا تكونوا رفقاء لهم، بل ابتعدوا عن عنفهم؛ لأنكم ستصبحون رفقاء لجنود السماء
\par 7 وعلى الرغم من أنكم أيها الخطاة تقولون: "لن تُبحث جميع خطايانا وتُكتب"، إلا أنهم سيكتبون جميع خطاياكم كل يوم
\par 8 والآن أُريكم أن النور والظلام، والنهار والليل، يرون كل خطاياكم
\par 9 لا تكونوا غير متدينين في قلوبكم، ولا تكذبوا ولا تغيروا كلام الاستقامة، ولا تتهموا كلام القدوس العظيم بالكذب، ولا تأخذوا في الاعتبار أصنامكم؛ لأن كل كذبكم وكل إثمكم لا يؤدي إلى البر بل إلى خطيئة عظيمة
\par 10 والآن أعلم هذا السر، أن الخطاة سيغيرون ويحرفون أقوال البر بطرق كثيرة، ويتكلمون بكلام رديء، ويكذبون، ويمارسون خداعًا عظيمًا، ويكتبون عن أقوالهم كتبًا
\par 11 ولكن عندما يكتبون كل كلامي بصدق بلغاتهم، ولا يغيرون أو ينقصون من كلامي شيئًا، بل يكتبونه كله بصدق - كل ما شهدت به عليهم أولًا
\par 12 ثم أعلم سرًا آخر، وهو أن الكتب ستُعطى للأبرار والحكماء لتكون سببًا للفرح والاستقامة والحكمة الكثيرة
\par 13 ويُعطى لهم الكتب، فيؤمنون بها ويفرحون بها، وحينئذٍ يُجازى جميع الصالحين الذين تعلموا منها جميع سبل الاستقامة

\chapter{105}

\par 1 في تلك الأيام، أمرهم الرب باستدعاء أبناء الأرض والإدلاء بشهادتهم بشأن حكمتهم: أروهم إياها؛ لأنكم أنتم مرشدوهم، وجزاء على كل الأرض
\par 2 لأني وابني سنكون معهم إلى الأبد في طريق الاستقامة في حياتهم، وسيكون لكم السلام. افرحوا يا أبناء الاستقامة. آمين.

\part{جزء من كتاب نوح}

\chapter{106}

\par 1 وبعد أيام اتخذ ابني متوشالح زوجة لابنه لامك، فحملت منه وولدت ابنًا
\par 2 وكان جسده أبيض كالثلج وأحمر كزهر الورد، وكان شعر رأسه وخصلات شعره الطويلة بيضاء كالصوف، وعيناه جميلتان. وعندما فتح عينيه، أضاء المنزل كله كالشمس، وكان المنزل كله ساطعًا جدًا
\par 3 ثم قام بين يدي القابلة، وفتح فمه، وتحدث مع رب البر
\par 4 فخاف منه لامك أبوه فهرب، وجاء إلى متوشالح أبيه
\par 5 فقال له: «لقد أنجبت ابنًا غريبًا، مختلفًا عن البشر، لا يشبههم، ويشبه أبناء إله السماء؛ وطبيعته مختلفة، وهو ليس مثلنا، وعيناه كأشعة الشمس، ووجهه مجيد».
\par 6 «ويبدو لي أنه لم ينبع مني بل من الملائكة، وأخشى أن تحدث في أيامه آية على الأرض.»
\par 7 «والآن يا أبي، أنا هنا لأطلب منك وأتوسل إليك أن تذهب إلى أخنوخ، أبينا، وتتعلم منه الحقيقة، لأن مسكنه بين الملائكة.»
\par 8 ولما سمع متوشالح كلام ابنه، جاء إليّ إلى أقاصي الأرض، لأنه سمع أن هناك، فصرخ بصوت عالٍ، فسمعت صوته فجئت إليه. وقلت له: «ها أنا ذا يا ابني، لماذا أتيت إليّ».
\par 9 فأجاب وقال: «لسببٍ عظيمٍ من القلق أتيتُ إليك، ولسببٍ من رؤيا مُقلقةٍ اقتربتُ منك».
\par 10 «والآن يا أبي، اسمعني: لقد وُلد للامك ابني ابن ليس له مثل، وطبيعته ليست كطبيعته البشرية، ولون جسده أشد بياضًا من الثلج وأشد حمرة من زهرة الورد، وشعر رأسه أشد بياضًا من الصوف الأبيض، وعيناه كأشعة الشمس، ففتح عينيه فأضاء البيت كله.»
\par 11 «فقام في يدي القابلة وفتح فمه وبارك رب السماء.»
\par 12 «فخاف لامك أبوه وهرب إليّ، ولم يُصدّق أنه وُلد منه، بل إنه على شبه ملائكة السماء. وها أنا قد أتيت إليك لكي تُعرّفني بالحق.»
\par 13 فأجبتُ أنا أخنوخ وقلتُ له: «سيفعل الرب شيئًا جديدًا على الأرض، وقد رأيتُ هذا بالفعل في رؤيا، وسأُعلمك أنه في جيل أبي يارد، تعدّى بعض ملائكة السماء كلمة الرب».
\par 14 "وإذا هم يرتكبون الخطيئة ويتعدون على الناموس، ويرتبطون بالنساء ويرتكبون الخطيئة معهن، ويتزوجون من بعضهن، وأنجبوا منهن أولادًا."
\par 15 «نعم، سيأتي دمار عظيم على الأرض كلها، وسيكون هناك طوفان ودمار عظيم لمدة عام واحد.»
\par 16 «وهذا الابن المولود لك سيُترك على الأرض، وسيخلص أولاده الثلاثة معه: عندما يموت جميع البشر الذين على الأرض [سيخلص هو وأبناؤه]».
\par 17 «وسيُنتجون على الأرض عمالقة ليس حسب الروح، بل حسب الجسد، وسيكون هناك عقاب عظيم على الأرض، وستُطهر الأرض من كل دنس.»
\par 18 «والآن أعلم ابنك لامك أن المولود هو ابنه حقًا، وادع اسمه نوحًا؛ لأنه سيُترك لك، وسيُنجى هو وأبناؤه من الهلاك الذي سيحل على الأرض بسبب كل الخطيئة وكل الإثم، اللذين سيُستكملان على الأرض في أيامه.»
\par 19 «وبعد ذلك سيكون هناك إثم أكثر مما حدث أولًا على الأرض؛ لأني أعرف أسرار القديسين؛ لأنه هو الرب أراني وأعلمني، وقرأت (لها) في الألواح السماوية.»

\chapter{107}

\par 1 «ورأيت مكتوبًا عليها أن جيلًا فجيلًا سيتعدى، حتى يقوم جيل البر، ويُباد المعصية، وتزول الخطيئة من الأرض، ويأتي عليها كل خير.»
\par 2 «والآن يا ابني، اذهب وأخبر ابنك لامك أن هذا الابن الذي وُلد هو ابنه حقًا، وأن هذا ليس كذبًا.»
\par 3 ولما سمع متوشالح كلام أبيه حنوك - لأنه كان قد أراه كل شيء سرًا - عاد وأراه إياه ودعا اسم ذلك الابن نوحًا؛ لأنه هو الذي سيُعزي الأرض بعد كل الدمار

\chapter{108}

\par 1 كتاب آخر كتبه أخنوخ لابنه متوشالح ولمن سيأتون بعده، ويحفظون الشريعة في الأيام الأخيرة
\par 2 أما أنتم الذين فعلتم الخير فانتظروا تلك الأيام حتى تنتهي قوى الشر وقوة المذنبين.
\par 3 "وانتظروا حتى تمر الخطيئة، لأن أسماءهم ستُمحى من سفر الحياة ومن الكتب المقدسة، وستدمر نسلهم إلى الأبد، وستُقتل أرواحهم، وسيصرخون وينوحون في مكان هو برية خربة، وفي النار سيحترقون؛ لأنه لا توجد أرض هناك.
\par 4 ورأيت هناك شيئًا يشبه سحابة غير مرئية؛ لأنه بسبب عمقها لم أستطع النظر إليها، ورأيت لهبًا من نار مشتعلًا بشدة، وأشياء مثل جبال لامعة تدور وتتحرك ذهابًا وإيابًا
\par 5 فسألت أحد الملائكة القديسين الذي كان معي وقلت له: ما هذا الشيء المضيء، فهو ليس سماء بل هو فقط لهيب نار مشتعلة وصوت بكاء وعويل ورثاء وألم شديد؟
\par 6 فقال لي: «هذا المكان الذي تراه - هنا تُطرد أرواح الخطاة والمجدفين، وفاعلي الشر، والذين يُحرّفون كل ما تكلم به الرب بفم الأنبياء - (حتى) ما سيكون».
\par 7 «لأن بعضها مكتوب ومنقوش فوق في السماء، لكي يقرأه الملائكة ويعرفوا ما سيصيب الخطاة، وأرواح المتواضعين، والذين أذلوا أجسادهم، فكافأهم الله، والذين أخزاهم الأشرار،»
\par 8 «الذين أحبوا الله، ولم يحبوا الذهب ولا الفضة ولا شيئًا من خيرات العالم، بل أسلموا أجسادهم للعذاب.»
\par 9 «الذين منذ وجودهم، لم يتوقوا إلى الطعام الأرضي، بل اعتبروا كل شيء نَفَسًا عابرًا، وعاشوا وفقًا لذلك، وقد امتحنهم الرب كثيرًا، فوجدت أرواحهم نقية ليباركوا اسمه.»
\par 10 «وكل البركات المخصصة لهم قد ذكرتها في الكتب. وقد خصص لهم جزاءهم، لأنهم وُجدوا ممن أحبوا السماء أكثر من حياتهم في الدنيا، ومع أنهم دُوسوا تحت أقدام الأشرار، وتعرضوا للإساءة والشتائم منهم، وأُخزوا، إلا أنهم باركوني.»
\par 11 «والآن سأستدعي أرواح الصالحين الذين ينتمون إلى جيل النور، وسأحول أولئك الذين ولدوا في الظلمة، والذين لم يُكافأوا في الجسد بمثل هذا الشرف الذي تستحقه أمانتهم.»
\par 12 «وسأُخرج في نورٍ مُشرق أولئك الذين أحبوا اسمي القدوس، وسأُجلس كل واحدٍ منهم على عرش مجده.»
\par 13 «ويُبْهِجُونَ أَزْمِنَةً بِلاَ عَدَدٍ، لأَنَّ الْبِرَّ هُوَ حُكْمُ اللهِ، لأَنَّهُ يُعْطِي الأَمَانَةَ فِي مَسْكَنِ السُّبُلِ الْمُسْتَقِيمَةِ.»
\par 14 «وسيرون الذين ولدوا في الظلمة يُقادون إلى الظلمة، بينما يتألق الأبرار.»
\par 15 «وسيصرخ الخطاة بصوت عالٍ ويرونهم متألقين، وسيذهبون بالفعل إلى حيث تُحدد لهم الأيام والأوقات.»

\end{document}