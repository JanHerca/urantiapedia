\begin{document}

\title{عدد}


\chapter{1}

\par 1 وَقَال الرَّبُّ لِمُوسَى فِي بَرِّيَّةِ سِينَاءَ فِي خَيْمَةِ الاِجْتِمَاعِ فِي أَوَّلِ الشَّهْرِ الثَّانِي فِي السَّنَةِ الثَّانِيَةِ لِخُرُوجِهِمْ مِنْ أَرْضِ مِصْرَ:
\par 2 «أَحْصُوا كُل جَمَاعَةِ بَنِي إِسْرَائِيل بِعَشَائِرِهِمْ وَبُيُوتِ آبَائِهِمْ بِعَدَدِ الأَسْمَاءِ كُل ذَكَرٍ بِرَأْسِهِ
\par 3 مِنِ ابْنِ عِشْرِينَ سَنَةً فَصَاعِداً كُل خَارِجٍ لِلحَرْبِ فِي إِسْرَائِيل. تَحْسِبُهُمْ أَنْتَ وَهَارُونُ حَسَبَ أَجْنَادِهِمْ.
\par 4 وَيَكُونُ مَعَكُمَا رَجُلٌ لِكُلِّ سِبْطٍ هُوَ رَأْسٌ لِبَيْتِ آبَائِهِ.
\par 5 وَهَذِهِ أَسْمَاءُ الرِّجَالِ الذِينَ يَقِفُونَ مَعَكُمَا. لِرَأُوبَيْنَ أَلِيصُورُ بْنُ شَدَيْئُورَ.
\par 6 لِشَمْعُونَ شَلُومِيئِيلُ بْنُ صُورِيشَدَّاي.
\par 7 لِيَهُوذَا نَحْشُونُ بْنُ عَمِّينَادَابَ.
\par 8 لِيَسَّاكَرَ نَثَنَائِيلُ بْنُ صُوغَرَ.
\par 9 لِزَبُولُونَ أَلِيآبُ بْنُ حِيلُونَ.
\par 10 لاِبْنَيْ يُوسُفَ: لأَفْرَايِمَ أَلِيشَمَعُ بْنُ عَمِّيهُودَ وَلِمَنَسَّى جَمْلِيئِيلُ بْنُ فَدَهْصُورَ.
\par 11 لِبِنْيَامِينَ أَبِيدَنُ بْنُ جِدْعُونِي.
\par 12 لِدَانَ أَخِيعَزَرُ بْنُ عَمِّيشَدَّاي.
\par 13 لأَشِيرَ فَجْعِيئِيلُ بْنُ عُكْرَنَ.
\par 14 لِجَادَ أَلِيَاسَافُ بْنُ دَعُوئِيل.
\par 15 لِنَفْتَالِي أَخِيرَعُ بْنُ عِينَنَ».
\par 16 هَؤُلاءِ هُمْ مَشَاهِيرُ الجَمَاعَةِ رُؤَسَاءُ أَسْبَاطِ آبَائِهِمْ. رُؤُوسُ أُلُوفِ إِسْرَائِيل.
\par 17 فَأَخَذَ مُوسَى وَهَارُونُ هَؤُلاءِ الرِّجَال الذِينَ تَعَيَّنُوا بِأَسْمَائِهِمْ
\par 18 وَجَمَعَا كُل الجَمَاعَةِ فِي أَوَّلِ الشَّهْرِ الثَّانِي فَانْتَسَبُوا إِلى عَشَائِرِهِمْ وَبُيُوتِ آبَائِهِمْ بِعَدَدِ الأَسْمَاءِ مِنِ ابْنِ عِشْرِينَ سَنَةً فَصَاعِداً بِرُؤُوسِهِمْ
\par 19 كَمَا أَمَرَ الرَّبُّ مُوسَى. فَعَدَّهُمْ فِي بَرِّيَّةِ سِينَاءَ.
\par 20 فَكَانَ بَنُو رَأُوبَيْنَ بِكْرِ إِسْرَائِيل تَوَالِيدُهُمْ حَسَبَ عَشَائِرِهِمْ وَبُيُوتِ آبَائِهِمْ بِعَدَدِ الأَسْمَاءِ بِرُؤُوسِهِمْ كُلُّ ذَكَرٍ مِنِ ابْنِ عِشْرِينَ سَنَةً فَصَاعِداً كُلُّ خَارِجٍ لِلحَرْبِ
\par 21 كَانَ المَعْدُودُونَ مِنْهُمْ لِسِبْطِ رَأُوبَيْنَ سِتَّةً وَأَرْبَعِينَ أَلفاً وَخَمْسَ مِئَةٍ.
\par 22 بَنُو شَمْعُونَ تَوَالِيدُهُمْ حَسَبَ عَشَائِرِهِمْ وَبُيُوتِ آبَائِهِمِ المَعْدُودُونَ مِنْهُمْ بِعَدَدِ الأَسْمَاءِ بِرُؤُوسِهِمْ كُلُّ ذَكَرٍ مِنِ ابْنِ عِشْرِينَ سَنَةً فَصَاعِداً كُلُّ خَارِجٍ لِلحَرْبِ
\par 23 المَعْدُودُونَ مِنْهُمْ لِسِبْطِ شَمْعُونَ تِسْعَةٌ وَخَمْسُونَ أَلفاً وَثَلاثُ مِئَةٍ.
\par 24 بَنُو جَادَ تَوَالِيدُهُمْ حَسَبَ عَشَائِرِهِمْ وَبُيُوتِ آبَائِهِمْ بِعَدَدِ الأَسْمَاءِ مِنِ ابْنِ عِشْرِينَ سَنَةً فَصَاعِداً كُلُّ خَارِجٍ لِلحَرْبِ
\par 25 المَعْدُودُونَ مِنْهُمْ لِسِبْطِ جَادَ خَمْسَةٌ وَأَرْبَعُونَ أَلفاً وَسِتُّ مِئَةٍ وَخَمْسُونَ.
\par 26 بَنُو يَهُوذَا تَوَالِيدُهُمْ حَسَبَ عَشَائِرِهِمْ وَبُيُوتِ آبَائِهِمْ بِعَدَدِ الأَسْمَاءِ مِنِ ابْنِ عِشْرِينَ سَنَةً فَصَاعِداً كُلُّ خَارِجٍ لِلحَرْبِ
\par 27 المَعْدُودُونَ مِنْهُمْ لِسِبْطِ يَهُوذَا أَرْبَعَةٌ وَسَبْعُونَ أَلفاً وَسِتُّ مِئَةٍ.
\par 28 بَنُو يَسَّاكَرَ تَوَالِيدُهُمْ حَسَبَ عَشَائِرِهِمْ وَبُيُوتِ آبَائِهِمْ بِعَدَدِ الأَسْمَاءِ مِنِ ابْنِ عِشْرِينَ سَنَةً فَصَاعِداً كُلُّ خَارِجٍ لِلحَرْبِ
\par 29 المَعْدُودُونَ مِنْهُمْ لِسِبْطِ يَسَّاكَرَ أَرْبَعَةٌ وَخَمْسُونَ أَلفاً وَأَرْبَعُ مِئَةٍ.
\par 30 بَنُو زَبُولُونَ تَوَالِيدُهُمْ حَسَبَ عَشَائِرِهِمْ وَبُيُوتِ آبَائِهِمْ بِعَدَدِ الأَسْمَاءِ مِنِ ابْنِ عِشْرِينَ سَنَةً فَصَاعِداً كُلُّ خَارِجٍ لِلحَرْبِ
\par 31 المَعْدُودُونَ مِنْهُمْ لِسِبْطِ زَبُولُونَ سَبْعَةٌ وَخَمْسُونَ أَلفاً وَأَرْبَعُ مِئَةٍ.
\par 32 بَنُو يُوسُفَ: بَنُو أَفْرَايِمَ تَوَالِيدُهُمْ حَسَبَ عَشَائِرِهِمْ وَبُيُوتِ آبَائِهِمْ بِعَدَدِ الأَسْمَاءِ مِنِ ابْنِ عِشْرِينَ سَنَةً فَصَاعِداً كُلُّ خَارِجٍ لِلحَرْبِ
\par 33 المَعْدُودُونَ مِنْهُمْ لِسِبْطِ أَفْرَايِمَ أَرْبَعُونَ أَلفاً وَخَمْسُ مِئَةٍ.
\par 34 بَنُو مَنَسَّى تَوَالِيدُهُمْ حَسَبَ عَشَائِرِهِمْ وَبُيُوتِ آبَائِهِمْ بِعَدَدِ الأَسْمَاءِ مِنِ ابْنِ عِشْرِينَ سَنَةً فَصَاعِداً كُلُّ خَارِجٍ لِلحَرْبِ
\par 35 المَعْدُودُونَ مِنْهُمْ لِسِبْطِ مَنَسَّى اثْنَانِ وَثَلاثُونَ أَلفاً وَمِئَتَانِ.
\par 36 بَنُو بِنْيَامِينَ تَوَالِيدُهُمْ حَسَبَ عَشَائِرِهِمْ وَبُيُوتِ آبَائِهِمْ بِعَدَدِ الأَسْمَاءِ مِنِ ابْنِ عِشْرِينَ سَنَةً فَصَاعِداً كُلُّ خَارِجٍ لِلحَرْبِ
\par 37 المَعْدُودُونَ مِنْهُمْ لِسِبْطِ بَنْيَامِينَ خَمْسَةٌ وَثَلاثُونَ أَلفاً وَأَرْبَعُ مِئَةٍ.
\par 38 بَنُو دَانَ تَوَالِيدُهُمْ حَسَبَ عَشَائِرِهِمْ وَبُيُوتِ آبَائِهِمْ بِعَدَدِ الأَسْمَاءِ مِنِ ابْنِ عِشْرِينَ سَنَةً فَصَاعِداً كُلُّ خَارِجٍ لِلحَرْبِ
\par 39 المَعْدُودُونَ مِنْهُمْ لِسِبْطِ دَانَ اثْنَانِ وَسِتُّونَ أَلفاً وَسَبْعُ مِئَةٍ.
\par 40 بَنُو أَشِيرَ تَوَالِيدُهُمْ حَسَبَ عَشَائِرِهِمْ وَبُيُوتِ آبَائِهِمْ بِعَدَدِ الأَسْمَاءِ مِنِ ابْنِ عِشْرِينَ سَنَةً فَصَاعِداً كُلُّ خَارِجٍ لِلحَرْبِ
\par 41 المَعْدُودُونَ مِنْهُمْ لِسِبْطِ أَشِيرَ وَاحِدٌ وأَرْبَعُونَ أَلفاً وَخَمْسُ مِئَةٍ.
\par 42 بَنُو نَفْتَالِي تَوَالِيدُهُمْ حَسَبَ عَشَائِرِهِمْ وَبُيُوتِ آبَائِهِمْ بِعَدَدِ الأَسْمَاءِ مِنِ ابْنِ عِشْرِينَ سَنَةً فَصَاعِداً كُلُّ خَارِجٍ لِلحَرْبِ
\par 43 المَعْدُودُونَ مِنْهُمْ لِسِبْطِ نَفْتَالِي ثَلاثَةٌ وَخَمْسُونَ أَلفاً وَأَرْبَعُ مِئَةٍ.
\par 44 هَؤُلاءِ هُمُ المَعْدُودُونَ الذِينَ عَدَّهُمْ مُوسَى وَهَارُونُ وَرُؤَسَاءُ إِسْرَائِيل اثْنَا عَشَرَ رَجُلاً رَجُلٌ وَاحِدٌ لِبَيْتِ آبَائِهِ.
\par 45 فَكَانَ جَمِيعُ المَعْدُودِينَ مِنْ بَنِي إِسْرَائِيل حَسَبَ بُيُوتِ آبَائِهِمْ مِنِ ابْنِ عِشْرِينَ سَنَةً فَصَاعِداً كُلُّ خَارِجٍ لِلحَرْبِ فِي إِسْرَائِيل
\par 46 سِتَّ مِئَةِ أَلفٍ وَثَلاثَةَ آلافٍ وَخَمْسَ مِئَةٍ وَخَمْسِينَ.
\par 47 وَأَمَّا اللاوِيُّونَ حَسَبَ سِبْطِ آبَائِهِمْ فَلمْ يُعَدُّوا بَيْنَهُمْ
\par 48 إِذْ قَال الرَّبُّ لِمُوسَى:
\par 49 «أَمَّا سِبْطُ لاوِي فَلا تَحْسِبْهُ وَلا تَعُدَّهُ بَيْنَ بَنِي إِسْرَائِيل.
\par 50 بَل وَكِّلِ اللاوِيِّينَ عَلى مَسْكَنِ الشَّهَادَةِ وَعَلى جَمِيعِ أَمْتِعَتِهِ وَعَلى كُلِّ مَا لهُ. هُمْ يَحْمِلُونَ المَسْكَنَ وَكُل أَمْتِعَتِهِ وَهُمْ يَخْدِمُونَهُ وَحَوْل المَسْكَنِ يَنْزِلُونَ.
\par 51 فَعِنْدَ ارْتِحَالِ المَسْكَنِ يُنَزِّلُهُ اللاوِيُّونَ وَعِنْدَ نُزُولِ المَسْكَنِ يُقِيمُهُ اللاوِيُّونَ. وَالأَجْنَبِيُّ الذِي يَقْتَرِبُ يُقْتَلُ.
\par 52 وَيَنْزِلُ بَنُو إِسْرَائِيل كُلٌّ فِي مَحَلتِهِ وَكُلٌّ عِنْدَ رَايَتِهِ بِأَجْنَادِهِمْ.
\par 53 وَأَمَّا اللاوِيُّونَ فَيَنْزِلُونَ حَوْل مَسْكَنِ الشَّهَادَةِ لِكَيْ لا يَكُونَ سَخَطٌ عَلى جَمَاعَةِ بَنِي إِسْرَائِيل فَيَحْفَظُ اللاوِيُّونَ شَعَائِرَ مَسْكَنِ الشَّهَادَةِ».
\par 54 فَفَعَل بَنُو إِسْرَائِيل حَسَبَ كُلِّ مَا أَمَرَ الرَّبُّ مُوسَى.

\chapter{2}

\par 1 وَقَال الرَّبُّ لِمُوسَى وَهَارُونَ:
\par 2 «يَنْزِلُ بَنُو إِسْرَائِيل كُلٌّ عِنْدَ رَايَتِهِ بِأَعْلامٍ لِبُيُوتِ آبَائِهِمْ. قُبَالةَ خَيْمَةِ الاِجْتِمَاعِ حَوْلهَا يَنْزِلُونَ.
\par 3 فَالنَّازِلُونَ إِلى الشَّرْقِ نَحْوَ الشُّرُوقِ رَايَةُ مَحَلةِ يَهُوذَا حَسَبَ أَجْنَادِهِمْ. وَالرَّئِيسُ لِبَنِي يَهُوذَا نَحْشُونُ بْنُ عَمِّينَادَابَ.
\par 4 وَجُنْدُهُ المَعْدُودُونَ مِنْهُمْ أَرْبَعَةٌ وَسَبْعُونَ أَلفاً وَسِتُّ مِئَةٍ.
\par 5 وَالنَّازِلُونَ مَعَهُ سِبْطُ يَسَّاكَرَ. وَالرَّئِيسُ لِبَنِي يَسَّاكَرَ نَثَنَائِيلُ بْنُ صُوغَرَ.
\par 6 وَجُنْدُهُ المَعْدُودُونَ مِنْهُ أَرْبَعَةٌ وَخَمْسُونَ أَلفاً وَأَرْبَعُ مِئَةٍ.
\par 7 وَسِبْطُ زَبُولُونَ. وَالرَّئِيسُ لِبَنِي زَبُولُونَ أَلِيآبُ بْنُ حِيلُونَ.
\par 8 وَجُنْدُهُ المَعْدُودُونَ مِنْهُ سَبْعَةٌ وَخَمْسُونَ أَلفاً وَأَرْبَعُ مِئَةٍ.
\par 9 جَمِيعُ المَعْدُودِينَ لِمَحَلةِ يَهُوذَا مِئَةُ أَلفٍ وَسِتَّةٌ وَثَمَانُونَ أَلفاً وَأَرْبَعُ مِئَةٍ بِأَجْنَادِهِمْ. يَرْتَحِلُونَ أَوَّلاً.
\par 10 «رَايَةُ مَحَلةِ رَأُوبَيْنَ إِلى التَّيْمَنِ حَسَبَ أَجْنَادِهِمْ. وَالرَّئِيسُ لِبَنِي رَأُوبَيْنَ أَلِيصُورُ بْنُ شَدَيْئُورَ
\par 11 وَجُنْدُهُ المَعْدُودُونَ مِنْهُ سِتَّةٌ وَأَرْبَعُونَ أَلفاً وَخَمْسُ مِئَةٍ.
\par 12 وَالنَّازِلُونَ مَعَهُ سِبْطُ شَمْعُونَ. وَالرَّئِيسُ لِبَنِي شَمْعُونَ شَلُومِيئِيلُ بْنُ صُورِيشَدَّاي.
\par 13 وَجُنْدُهُ المَعْدُودُونَ مِنْهُمْ تِسْعَةٌ وَخَمْسُونَ أَلفاً وَثَلاثُ مِئَةٍ.
\par 14 وَسِبْطُ جَادَ. وَالرَّئِيسُ لِبَنِي جَادٍَ أَلِيَاسَافُ بْنُ رَعُوئِيل.
\par 15 وَجُنْدُهُ المَعْدُودُونَ مِنْهُمْ خَمْسَةٌ وَأَرْبَعُونَ أَلفاً وَسِتُّ مِئَةٍ وَخَمْسُونَ.
\par 16 جَمِيعُ المَعْدُودِينَ لِمَحَلةِ رَأُوبَيْنَ مِئَةُ أَلفٍ وَوَاحِدٌ وَخَمْسُونَ أَلفاً وَأَرْبَعُ مِئَةٍ وَخَمْسُونَ بِأَجْنَادِهِمْ وَيَرْتَحِلُونَ ثَانِيَةً.
\par 17 «ثُمَّ تَرْتَحِلُ خَيْمَةُ الاِجْتِمَاعِ. مَحَلةُ اللاوِيِّينَ فِي وَسَطِ المَحَلاتِ. كَمَا يَنْزِلُونَ كَذَلِكَ يَرْتَحِلُونَ. كُلٌّ فِي مَوْضِعِهِ بِرَايَاتِهِمْ.
\par 18 «رَايَةُ مَحَلةِ أَفْرَايِمَ حَسَبَ أَجْنَادِهِمْ إِلى الغَرْبِ. وَالرَّئِيسُ لِبَنِي أَفْرَايِمَ أَلِيشَمَعُ بْنُ عَمِّيهُودَ.
\par 19 وَجُنْدُهُ المَعْدُودُونَ مِنْهُمْ أَرْبَعُونَ أَلفاً وَخَمْسُ مِئَةٍ.
\par 20 وَمَعَهُ سِبْطُ مَنَسَّى. وَالرَّئِيسُ لِبَنِي مَنَسَّى جَمْلِيئِيلُ بْنُ فَدَهْصُورَ.
\par 21 وَجُنْدُهُ المَعْدُودُونَ مِنْهُمُ اثْنَانِ وَثَلاثُونَ أَلفاً وَمِئَتَانِ.
\par 22 وَسِبْطُ بِنْيَامِينَ. وَالرَّئِيسُ لِبَنِي بِنْيَامِينَ أَبِيدَنُ بْنُ جِدْعُونِي.
\par 23 وَجُنْدُهُ المَعْدُودُونَ مِنْهُمْ خَمْسَةٌ وَثَلاثُونَ أَلفاً وَأَرْبَعُ مِئَةٍ.
\par 24 جَمِيعُ المَعْدُودِينَ لِمَحَلةِ أَفْرَايِمَ مِئَةُ أَلفٍ وَثَمَانِيَةُ آلافٍ وَمِئَةٌ بِأَجْنَادِهِمْ. وَيَرْتَحِلُونَ ثَالِثَةً.
\par 25 «رَايَةُ مَحَلةِ دَانَ إِلى الشِّمَالِ حَسَبَ أَجْنَادِهِمْ. وَالرَّئِيسُ لِبَنِي دَانَ أَخِيعَزَرُ بْنُ عَمِّيشَدَّاي.
\par 26 وَجُنْدُهُ المَعْدُودُونَ مِنْهُمُ اثْنَانِ وَسِتُّونَ أَلفاً وَسَبْعُ مِئَةٍ.
\par 27 وَالنَّازِلُونَ مَعَهُ سِبْطُ أَشِيرَ. وَالرَّئِيسُ لِبَنِي أَشِيرَ فَجْعِيئِيلُ بْنُ عُكْرَنَ.
\par 28 وَجُنْدُهُ المَعْدُودُونَ مِنْهُمْ وَاحِدٌ وَأَرْبَعُونَ أَلفاً وَخَمْسُ مِئَةٍ.
\par 29 وَسِبْطُ نَفْتَالِي. وَالرَّئِيسُ لِبَنِي نَفْتَالِي أَخِيرَعُ بْنُ عِينَنَ.
\par 30 وَجُنْدُهُ المَعْدُودُونَ مِنْهُمْ ثَلاثَةٌ وَخَمْسُونَ أَلفاً وَأَرْبَعُ مِئَةٍ.
\par 31 جَمِيعُ المَعْدُودِينَ لِمَحَلةِ دَانٍَ مِئَةُ أَلفٍ وَسَبْعَةٌ وَخَمْسُونَ أَلفاً وَسِتُّ مِئَةٍ. يَرْتَحِلُونَ أَخِيراً بِرَايَاتِهِمْ».
\par 32 هَؤُلاءِ هُمُ المَعْدُودُونَ مِنْ بَنِي إِسْرَائِيل حَسَبَ بُيُوتِ آبَائِهِمْ. جَمِيعُ المَعْدُودِينَ مِنَ المَحَلاتِ بِأَجْنَادِهِمْ سِتُّ مِئَةِ أَلفٍ وَثَلاثَةُ آلافٍ وَخَمْسُ مِئَةٍ وَخَمْسُونَ.
\par 33 وَأَمَّا اللاوِيُّونَ فَلمْ يُعَدُّوا بَيْنَ بَنِي إِسْرَائِيل كَمَا أَمَرَ الرَّبُّ مُوسَى.
\par 34 فَفَعَل بَنُو إِسْرَائِيل حَسَبَ كُلِّ مَا أَمَرَ بِهِ الرَّبُّ مُوسَى. هَكَذَا نَزَلُوا بِرَايَاتِهِمْ وَهَكَذَا ارْتَحَلُوا. كُلٌّ حَسَبَ عَشَائِرِهِ مَعَ بَيْتِ آبَائِهِ.

\chapter{3}

\par 1 وَهَذِهِ تَوَالِيدُ هَارُونَ وَمُوسَى يَوْمَ كَلمَ الرَّبُّ مُوسَى فِي جَبَلِ سِينَاءَ.
\par 2 وَهَذِهِ أَسْمَاءُ بَنِي هَارُونَ: نَادَابُ البِكْرُ وَأَبِيهُو وَأَلِعَازَارُ وَإِيثَامَارُ.
\par 3 هَذِهِ أَسْمَاءُ بَنِي هَارُونَ الكَهَنَةِ المَمْسُوحِينَ الذِينَ مَلأَ أَيْدِيَهُمْ لِلكَهَانَةِ.
\par 4 وَلكِنْ مَاتَ نَادَابُ وَأَبِيهُو أَمَامَ الرَّبِّ عِنْدَمَا قَرَّبَا نَاراً غَرِيبَةً أَمَامَ الرَّبِّ فِي بَرِّيَّةِ سِينَاءَ وَلمْ يَكُنْ لهُمَا بَنُونَ. وَأَمَّا أَلِعَازَارُ وَإِيثَامَارُ فَكَهَنَا أَمَامَ هَارُونَ أَبِيهِمَا.
\par 5 وَقَال الرَّبُّ لِمُوسَى:
\par 6 «قَدِّمْ سِبْطَ لاوِي وَأَوْقِفْهُمْ قُدَّامَ هَارُونَ الكَاهِنِ وَليَخْدِمُوهُ.
\par 7 فَيَحْفَظُونَ شَعَائِرَهُ وَشَعَائِرَ كُلِّ الجَمَاعَةِ قُدَّامَ خَيْمَةِ الاِجْتِمَاعِ وَيَخْدِمُونَ خِدْمَةَ المَسْكَنِ
\par 8 فَيَحْرُسُونَ كُل أَمْتِعَةِ خَيْمَةِ الاِجْتِمَاعِ وَحِرَاسَةِ بَنِي إِسْرَائِيل وَيَخْدِمُونَ خِدْمَةَ المَسْكَنِ.
\par 9 فَتُعْطِي اللاوِيِّينَ لِهَارُونَ وَلِبَنِيهِ. إِنَّهُمْ مَوْهُوبُونَ لهُ هِبَةً مِنْ عِنْدِ بَنِي إِسْرَائِيل.
\par 10 وَتُوَكِّلُ هَارُونَ وَبَنِيهِ فَيَحْرُسُونَ كَهَنُوتَهُمْ. وَالأَجْنَبِيُّ الذِي يَقْتَرِبُ يُقْتَلُ».
\par 11 وَقَال الرَّبُّ لِمُوسَى:
\par 12 «وَهَا إِنِّي قَدْ أَخَذْتُ اللاوِيِّينَ مِنْ بَيْنِ بَنِي إِسْرَائِيل بَدَل كُلِّ بِكْرٍ فَاتِحِ رَحِمٍ مِنْ بَنِي إِسْرَائِيل فَيَكُونُ اللاوِيُّونَ لِي.
\par 13 لأَنَّ لِي كُل بِكْرٍ. يَوْمَ ضَرَبْتُ كُل بِكْرٍ فِي أَرْضِ مِصْرٍَ قَدَّسْتُ لِي كُل بِكْرٍ فِي إِسْرَائِيل مِنَ النَّاسِ وَالبَهَائِمِ. لِي يَكُونُونَ. أَنَا الرَّبُّ».
\par 14 وَأَمَرَ الرَّبُّ مُوسَى فِي بَرِّيَّةِ سِينَاءَ:
\par 15 «عُدَّ بَنِي لاوِي حَسَبَ بُيُوتِ آبَائِهِمْ وَعَشَائِرِهِمْ. كُل ذَكَرٍ مِنِ ابْنِ شَهْرٍ فَصَاعِداً تَعُدُّهُمْ».
\par 16 فَعَدَّهُمْ مُوسَى حَسَبَ قَوْلِ الرَّبِّ كَمَا أُمِرَ.
\par 17 وَكَانَ هَؤُلاءِ بَنِي لاوِي بِأَسْمَائِهِمْ: جَرْشُونُ وَقَهَاتُ وَمَرَارِي.
\par 18 وَهَذَانِ اسْمَا ابْنَيْ جَرْشُونَ حَسَبَ عَشَائِرِهِمَا: لِبْنِي وَشَمْعِي.
\par 19 وَبَنُو قَهَاتَ حَسَبَ عَشَائِرِهِمْ: عَمْرَامُ وَيِصْهَارُ وَحَبْرُونُ وَعُزِّيئِيلُ.
\par 20 وَابْنَا مَرَارِي حَسَبَ عَشَائِرِهِمَا: مَحْلِي وَمُوشِي. هَذِهِ هِيَ عَشَائِرُ اللاوِيِّينَ حَسَبَ بُيُوتِ آبَائِهِمْ.
\par 21 لِجَرْشُونَ عَشِيرَةُ اللِّبْنِيِّينَ وَعَشِيرَةُ الشَّمْعِيِّينَ. هَذِهِ هِيَ عَشَائِرُ الجَرْشُونِيِّينَ.
\par 22 المَعْدُودُونَ مِنْهُمْ بِعَدَدِ كُلِّ ذَكَرٍ مِنِ ابْنِ شَهْرٍ فَصَاعِداً المَعْدُودُونَ مِنْهُمْ سَبْعَةُ آلافٍ وَخَمْسُ مِئَةٍ.
\par 23 عَشَائِرُ الجَرْشُونِيِّينَ يَنْزِلُونَ وَرَاءَ المَسْكَنِ إِلى الغَرْبِ.
\par 24 وَالرَّئِيسُ لِبَيْتِ أَبِي الجَرْشُونِيِّينَ أَلِيَاسَافُ بْنُ لايِل.
\par 25 وَحِرَاسَةُ بَنِي جَرْشُونَ فِي خَيْمَةِ الاِجْتِمَاعِ: المَسْكَنُ وَالخَيْمَةُ وَغِطَاؤُهَا وَسَجْفُ بَابِ خَيْمَةِ الاِجْتِمَاعِ
\par 26 وَأَسْتَارُ الدَّارِ وَسَجْفُ بَابِ الدَّارِ اللوَاتِي حَوْل المَسْكَنِ وَحَوْل المَذْبَحِ مُحِيطاً وَأَطْنَابُهُ مَعَ كُلِّ خِدْمَتِهِ.
\par 27 وَلِقَهَاتَ عَشِيرَةُ العَمْرَامِيِّينَ وَعَشِيرَةُ اليِصْهَارِيِّينَ وَعَشِيرَةُ الحَبْرُونِيِّينَ وَعَشِيرَةُ العُزِّيئِيلِيِّينَ. هَذِهِ عَشَائِرُ القَهَاتِيِّينَ
\par 28 بِعَدَدِ كُلِّ ذَكَرٍ مِنِ ابْنِ شَهْرٍ فَصَاعِداً ثَمَانِيَةُ آلافٍ وَسِتُّ مِئَةٍ حَارِسِينَ حِرَاسَةَ القُدْسِ.
\par 29 وَعَشَائِرُ بَنِي قَهَاتَ يَنْزِلُونَ عَلى جَانِبِ المَسْكَنِ إِلى التَّيْمَنِ.
\par 30 وَالرَّئِيسُ لِبَيْتِ أَبِي عَشِيرَةِ القَهَاتِيِّينَ أَلِيصَافَانُ بْنُ عُزِّيئِيل.
\par 31 وَحِرَاسَتُهُمُ التَّابُوتُ وَالمَائِدَةُ وَالمَنَارَةُ وَالمَذْبَحَانِ وَأَمْتِعَةُ القُدْسِ التِي يَخْدِمُونَ بِهَا وَالحِجَابُ وَكُلُّ خِدْمَتِهِ.
\par 32 وَلِرَئِيسِ رُؤَسَاءِ اللاوِيِّينَ أَلِعَازَارَ بْنِ هَارُونَ الكَاهِنِ وَكَالةُ حُرَّاسِ حِرَاسَةِ القُدْسِ.
\par 33 وَلِمَرَارِي عَشِيرَةُ المَحْلِيِّينَ وَعَشِيرَةُ المُوشِيِّينَ. هَذِهِ هِيَ عَشَائِرُ مَرَارِي.
\par 34 وَالمَعْدُودُونَ مِنْهُمْ بِعَدَدِ كُلِّ ذَكَرٍ مِنِ ابْنِ شَهْرٍ فَصَاعِداً سِتَّةُ آلافٍ وَمِئَتَانِ.
\par 35 وَالرَّئِيسُ لِبَيْتِ أَبِي عَشَائِرِ مَرَارِي صُورِيئِيلُ بْنُ أَبِيجَايِل. يَنْزِلُونَ عَلى جَانِبِ المَسْكَنِ إِلى الشِّمَالِ.
\par 36 وَوَكَالةُ حِرَاسَةِ بَنِي مَرَارِي أَلوَاحُ المَسْكَنِ وَعَوَارِضُهُ وَأَعْمِدَتُهُ وَفُرَضُهُ وَكُلُّ أَمْتِعَتِهِ وَكُلُّ خِدْمَتِهِ
\par 37 وَأَعْمِدَةُ الدَّارِ حَوَاليْهَا وَفُرَضُهَا وَأَوْتَادُهَا وَأَطْنَابُهَا.
\par 38 وَالنَّازِلُونَ قُدَّامَ المَسْكَنِ إِلى الشَّرْقِ قُدَّامَ خَيْمَةِ الاِجْتِمَاعِ نَحْوَ الشُّرُوقِ هُمْ مُوسَى وَهَارُونُ وَبَنُوهُ حَارِسِينَ حِرَاسَةَ المَقْدِسِ لِحِرَاسَةِ بَنِي إِسْرَائِيل. وَالأَجْنَبِيُّ الذِي يَقْتَرِبُ يُقْتَلُ.
\par 39 جَمِيعُ المَعْدُودِينَ مِنَ اللاوِيِّينَ الذِينَ عَدَّهُمْ مُوسَى وَهَارُونُ حَسَبَ قَوْلِ الرَّبِّ بِعَشَائِرِهِمْ كُلُّ ذَكَرٍ مِنِ ابْنِ شَهْرٍ فَصَاعِداً اثْنَانِ وَعِشْرُونَ أَلفاً.
\par 40 وَقَال الرَّبُّ لِمُوسَى: «عُدَّ كُل بِكْرٍ ذَكَرٍ مِنْ بَنِي إِسْرَائِيل مِنِ ابْنِ شَهْرٍ فَصَاعِداً وَخُذْ عَدَدَ أَسْمَائِهِمْ.
\par 41 فَتَأْخُذُ اللاوِيِّينَ لِي. أَنَا الرَّبُّ. بَدَل كُلِّ بِكْرٍ فِي بَنِي إِسْرَائِيل. وَبَهَائِمَ اللاوِيِّينَ بَدَل كُلِّ بِكْرٍ فِي بَهَائِمِ بَنِي إِسْرَائِيل».
\par 42 فَعَدَّ مُوسَى كَمَا أَمَرَهُ الرَّبُّ كُل بِكْرٍ فِي بَنِي إِسْرَائِيل.
\par 43 فَكَانَ جَمِيعُ الأَبْكَارِ الذُّكُورِ بِعَدَدِ الأَسْمَاءِ مِنِ ابْنِ شَهْرٍ فَصَاعِداً المَعْدُودِينَ مِنْهُمُ اثْنَيْنِ وَعِشْرِينَ أَلفاً وَمِئَتَيْنِ وَثَلاثَةً وَسَبْعِينَ.
\par 44 وَقَال الرَّبُّ لِمُوسَى:
\par 45 «خُذِ اللاوِيِّينَ بَدَل كُلِّ بِكْرٍ فِي بَنِي إِسْرَائِيل وَبَهَائِمَ اللاوِيِّينَ بَدَل بَهَائِمِهِمْ فَيَكُونَ لِيَ اللاوِيُّونَ. أَنَا الرَّبُّ.
\par 46 وَأَمَّا فِدَاءُ المِئَتَيْنِ وَالثَّلاثَةِ وَالسَّبْعِينَ الزَّائِدِينَ عَلى اللاوِيِّينَ مِنْ أَبْكَارِ بَنِي إِسْرَائِيل
\par 47 فَتَأْخُذُ خَمْسَةَ شَوَاقِل لِكُلِّ رَأْسٍ. عَلى شَاقِلِ القُدْسِ تَأْخُذُهَا. عِشْرُونَ جِيرَةً الشَّاقِلُ.
\par 48 وَتُعْطِي الفِضَّةَ لِهَارُونَ وَبَنِيهِ فِدَاءَ الزَّائِدِينَ عَليْهِمْ».
\par 49 فَأَخَذَ مُوسَى فِضَّةَ فِدَائِهِمْ مِنَ الزَّائِدِينَ عَلى فِدَاءِ اللاوِيِّينَ.
\par 50 مِنْ أَبْكَارِ بَنِي إِسْرَائِيل أَخَذَ الفِضَّةَ أَلفاً وَثَلاثَ مِئَةٍ وَخَمْسَةً وَسِتِّينَ عَلى شَاقِلِ القُدْسِ
\par 51 وَأَعْطَى مُوسَى فِضَّةَ الفِدَاءِ لِهَارُونَ وَبَنِيهِ حَسَبَ قَوْلِ الرَّبِّ كَمَا أَمَرَ الرَّبُّ مُوسَى.

\chapter{4}

\par 1 وَقَال الرَّبُّ لِمُوسَى وَهَارُونَ:
\par 2 «خُذْ عَدَدَ بَنِي قَهَاتَ مِنْ بَيْنِ بَنِي لاوِي حَسَبَ عَشَائِرِهِمْ وَبُيُوتِ آبَائِهِمْ
\par 3 مِنِ ابْنِ ثَلاثِينَ سَنَةً فَصَاعِداً إِلى ابْنِ خَمْسِينَ سَنَةً كُلِّ دَاخِلٍ فِي الجُنْدِ لِيَعْمَل عَمَلاً فِي خَيْمَةِ الاِجْتِمَاعِ.
\par 4 هَذِهِ خِدْمَةُ بَنِي قَهَاتَ فِي خَيْمَةِ الاِجْتِمَاعِ: قُدْسُ الأَقْدَاسِ.
\par 5 يَأْتِي هَارُونُ وَبَنُوهُ عِنْدَ ارْتِحَالِ المَحَلةِ وَيُنَزِّلُونَ حِجَابَ السَّجْفِ وَيُغَطُّونَ بِهِ تَابُوتَ الشَّهَادَةِ
\par 6 وَيَجْعَلُونَ عَليْهِ غِطَاءً مِنْ جِلدِ تُخَسٍ وَيَبْسُطُونَ مِنْ فَوْقُ ثَوْباً كُلُّهُ أَسْمَانْجُونِيٌّ وَيَضَعُونَ عِصِيَّهُ.
\par 7 وَعَلى مَائِدَةِ الوُجُوهِ يَبْسُطُونَ ثَوْبَ أَسْمَانْجُونٍ وَيَضَعُونَ عَليْهِ الصِّحَافَ وَالصُّحُونَ وَالأَقْدَاحَ وَكَاسَاتِ السَّكِيبِ. وَيَكُونُ الخُبْزُ الدَّائِمُ عَليْهِ.
\par 8 وَيَبْسُطُونَ عَليْهَا ثَوْبَ قِرْمِزٍ وَيُغَطُّونَهُ بِغِطَاءٍ مِنْ جِلدِ تُخَسٍ وَيَضَعُونَ عِصِيَّهُ.
\par 9 وَيَأْخُذُونَ ثَوْبَ أَسْمَانْجُونٍ وَيُغَطُّونَ مَنَارَةَ الضُّوءِ وَسُرُجَهَا وَمَلاقِطَهَا وَمَنَافِضَهَا وَجَمِيعَ آنِيَةِ زَيْتِهَا التِي يَخْدِمُونَهَا بِهَا.
\par 10 وَيَجْعَلُونَهَا وَجَمِيعَ آنِيَتَهَا فِي غِطَاءٍ مِنْ جِلدِ تُخَسٍ وَيَجْعَلُونَهُ عَلى العَتَلةِ.
\par 11 وَعَلى مَذْبَحِ الذَّهَبِ يَبْسُطُونَ ثَوْبَ أَسْمَانْجُونٍ وَيُغَطُّونَهُ بِغِطَاءٍ مِنْ جِلدِ تُخَسٍ وَيَضَعُونَ عِصِيَّهُ.
\par 12 وَيَأْخُذُونَ جَمِيعَ أَمْتِعَةِ الخِدْمَةِ التِي يَخْدِمُونَ بِهَا فِي القُدْسِ وَيَجْعَلُونَهَا فِي ثَوْبِ أَسْمَانْجُونٍ وَيُغَطُّونَهَا بِغِطَاءٍ مِنْ جِلدِ تُخَسٍ وَيَجْعَلُونَهَا عَلى العَتَلةِ.
\par 13 وَيَرْفَعُونَ رَمَادَ المَذْبَحِ وَيَبْسُطُونَ عَليْهِ ثَوْبَ أُرْجُوانٍ.
\par 14 وَيَجْعَلُونَ عَليْهِ جَمِيعَ أَمْتِعَتِهِ التِي يَخْدِمُونَ عَليْهِ بِهَا المَجَامِرَ وَالمَنَاشِل وَالرُّفُوشَ وَالمَنَاضِحَ كُل أَمْتِعَةِ المَذْبَحِ. وَيَبْسُطُونَ عَليْهِ غِطَاءً مِنْ جِلدِ تُخَسٍ وَيَضَعُونَ عِصِيَّهُ.
\par 15 وَمَتَى فَرَغَ هَارُونُ وَبَنُوهُ مِنْ تَغْطِيَةِ القُدْسِ وَجَمِيعِ أَمْتِعَةِ القُدْسِ عِنْدَ ارْتِحَالِ المَحَلةِ يَأْتِي بَعْدَ ذَلِكَ بَنُو قَهَاتَ لِلحِمْلِ وَلكِنْ لا يَمَسُّوا القُدْسَ لِئَلا يَمُوتُوا. ذَلِكَ حِمْلُ بَنِي قَهَاتَ فِي خَيْمَةِ الاِجْتِمَاعِ.
\par 16 وَوِكَالةُ أَلِعَازَارَ بْنِ هَارُونَ الكَاهِنِ هِيَ زَيْتُ الضُّوءِ وَالبَخُورُ العَطِرُ وَالتَّقْدِمَةُ الدَّائِمَةُ وَدُهْنُ المَسْحَةِ وَوِكَالةُ كُلِّ المَسْكَنِ وَكُلِّ مَا فِيهِ بِالقُدْسِ وَأَمْتِعَتِهِ».
\par 17 وَأَمَرَ الرَّبُّ مُوسَى وَهَارُونَ:
\par 18 «لا تَقْرِضَا سِبْطَ عَشَائِرِ القَهَاتِيِّينَ مِنْ بَيْنِ اللاوِيِّينَ.
\par 19 بَلِ افْعَلا لهُمْ هَذَا فَيَعِيشُوا وَلا يَمُوتُوا عِنْدَ اقْتِرَابِهِمْ إِلى قُدْسِ الأَقْدَاسِ. يَدْخُلُ هَارُونُ وَبَنُوهُ وَيُقِيمُونَهُمْ كُل إِنْسَانٍ عَلى خِدْمَتِهِ وَحِمْلِهِ
\par 20 وَلا يَدْخُلُوا لِيَرُوا القُدْسَ لحْظَةً لِئَلا يَمُوتُوا».
\par 21 وَأَمَرَ الرَّبُّ مُوسَى:
\par 22 «خُذْ عَدَدَ بَنِي جَرْشُونَ أَيْضاً حَسَبَ بُيُوتِ آبَائِهِمْ وَعَشَائِرِهِمْ
\par 23 مِنِ ابْنِ ثَلاثِينَ سَنَةً فَصَاعِداً إِلى ابْنِ خَمْسِينَ سَنَةً تَعُدُّهُمْ. كُل الدَّاخِلِينَ لِيَتَجَنَّدُوا أَجْنَاداً لِيَخْدِمُوا خِدْمَةً فِي خَيْمَةِ الاِجْتِمَاعِ.
\par 24 هَذِهِ خِدْمَةُ عَشَائِرِ الجَرْشُونِيِّينَ مِنَ الخِدْمَةِ وَالحِمْلِ:
\par 25 يَحْمِلُونَ شُقَقَ المَسْكَنِ وَخَيْمَةَ الاِجْتِمَاعِ وَغِطَاءَهَا وَغِطَاءَ التُّخَسِ الذِي عَليْهَا مِنْ فَوْقُ وَسَجْفَ بَابِ خَيْمَةِ الاِجْتِمَاعِ
\par 26 وَأَسْتَارَ الدَّارِ وَسَجْفَ مَدْخَلِ بَابِ الدَّارِ اللوَاتِي حَوْل المَسْكَنِ وَحَوْل المَذْبَحِ مُحِيطَةً وَأَطْنَابَهُنَّ وَكُل أَمْتِعَةِ خِدْمَتِهِنَّ. وَكُلُّ مَا يُعْمَلُ لهُنَّ فَهُمْ يَصْنَعُونَهُ
\par 27 حَسَبَ قَوْلِ هَارُونَ وَبَنِيهِ تَكُونُ جَمِيعُ خِدْمَةِ بَنِي الجَرْشُونِيِّينَ مِنْ كُلِّ حِمْلِهِمْ وَمِنْ كُلِّ خِدْمَتِهِمْ. وَتُوَكِّلُهُمْ بِحِرَاسَةِ كُلِّ أَحْمَالِهِمْ.
\par 28 هَذِهِ خِدْمَةُ عَشَائِرِ بَنِي الجَرْشُونِيِّينَ فِي خَيْمَةِ الاِجْتِمَاعِ وَحِرَاسَتُهُمْ بِيَدِ إِيثَامَارَ بْنِ هَارُونَ الكَاهِنِ.
\par 29 «بَنُو مَرَارِي حَسَبَ عَشَائِرِهِمْ وَبُيُوتِ آبَائِهِمْ تَعُدُّهُمْ.
\par 30 مِنِ ابْنِ ثَلاثِينَ سَنَةً فَصَاعِداً إِلى ابْنِ خَمْسِينَ سَنَةً تَعُدُّهُمْ كُل الدَّاخِلِينَ فِي الجُنْدِ لِيَخْدِمُوا خِدْمَةَ خَيْمَةِ الاِجْتِمَاعِ.
\par 31 وَهَذِهِ حِرَاسَةُ حِمْلِهِمْ وَكُلُّ خِدْمَتِهِمْ فِي خَيْمَةِ الاِجْتِمَاعِ: أَلوَاحُ المَسْكَنِ وَعَوَارِضُهُ وَأَعْمِدَتُهُ وَفُرَضُهُ
\par 32 وَأَعْمِدَةُ الدَّارِ حَوَاليْهَا وَفُرَضُهَا وَأَوْتَادُهَا وَأَطْنَابُهَا مَعَ كُلِّ أَمْتِعَتِهَا وَكُلِّ خِدْمَتِهَا. وَبِالأَسْمَاءِ تَعُدُّونَ أَمْتِعَةَ حِرَاسَةِ حِمْلِهِمْ.
\par 33 هَذِهِ خِدْمَةُ عَشَائِرِ بَنِي مَرَارِي. كُلُّ خِدْمَتِهِمْ فِي خَيْمَةِ الاِجْتِمَاعِ بِيَدِ إِيثَامَارَ بْنِ هَارُونَ الكَاهِنِ».
\par 34 فَعَدَّ مُوسَى وَهَارُونُ وَرُؤَسَاءُ الجَمَاعَةِ بَنِي القَهَاتِيِّينَ حَسَبَ عَشَائِرِهِمْ وَبُيُوتِ آبَائِهِمْ
\par 35 مِنِ ابْنِ ثَلاثِينَ سَنَةً فَصَاعِداً إِلى ابْنِ خَمْسِينَ سَنَةً كُل الدَّاخِلِينَ فِي الجُنْدِ لِلخِدْمَةِ فِي خَيْمَةِ الاِجْتِمَاعِ.
\par 36 فَكَانَ المَعْدُودُونَ مِنْهُمْ حَسَبَ عَشَائِرِهِمْ أَلفَيْنِ وَسَبْعَ مِئَةٍ وَخَمْسِينَ.
\par 37 هَؤُلاءِ هُمُ المَعْدُودُونَ مِنْ عَشَائِرِ القَهَاتِيِّينَ كُلُّ الخَادِمِينَ فِي خَيْمَةِ الاِجْتِمَاعِ الذِينَ عَدَّهُمْ مُوسَى وَهَارُونُ حَسَبَ قَوْلِ الرَّبِّ عَنْ يَدِ مُوسَى.
\par 38 وَالمَعْدُودُونَ مِنْ بَنِي جَرْشُونَ حَسَبَ عَشَائِرِهِمْ وَبُيُوتِ آبَائِهِمْ
\par 39 مِنِ ابْنِ ثَلاثِينَ سَنَةً فَصَاعِداً إِلى ابْنِ خَمْسِينَ سَنَةً كُلُّ الدَّاخِلِينَ فِي الجُنْدِ لِلخِدْمَةِ فِي خَيْمَةِ الاِجْتِمَاعِ
\par 40 كَانَ المَعْدُودُونَ مِنْهُمْ حَسَبَ عَشَائِرِهِمْ وَبُيُوتِ آبَائِهِمْ أَلفَيْنِ وَسِتَّ مِئَةٍ وَثَلاثِينَ.
\par 41 هَؤُلاءِ هُمُ المَعْدُودُونَ مِنْ عَشَائِرِ بَنِي جَرْشُونَ كُلُّ الخَادِمِينَ فِي خَيْمَةِ الاِجْتِمَاعِ الذِينَ عَدَّهُمْ مُوسَى وَهَارُونُ حَسَبَ قَوْلِ الرَّبِّ.
\par 42 وَالمَعْدُودُونَ مِنْ عَشَائِرِ بَنِي مَرَارِي حَسَبَ عَشَائِرِهِمْ وَبُيُوتِ آبَائِهِمْ
\par 43 مِنِ ابْنِ ثَلاثِينَ سَنَةً فَصَاعِداً إِلى ابْنِ خَمْسِينَ سَنَةً كُل الدَّاخِلينَ فِي الجُنْدِ لِلخِدْمَةِ فِي خَيْمَةِ الاِجْتِمَاعِ
\par 44 كَانَ المَعْدُودُونَ مِنْهُمْ حَسَبَ عَشَائِرِهِمْ ثَلاثَةَ آلافٍ وَمِئَتَيْنِ.
\par 45 هَؤُلاءِ هُمُ المَعْدُودُونَ مِنْ عَشَائِرِ بَنِي مَرَارِي الذِينَ عَدَّهُمْ مُوسَى وَهَارُونُ حَسَبَ قَوْلِ الرَّبِّ عَنْ يَدِ مُوسَى.
\par 46 جَمِيعُ المَعْدُودِينَ اللاوِيِّينَ الذِينَ عَدَّهُمْ مُوسَى وَهَارُونُ وَرُؤَسَاءُ إِسْرَائِيل حَسَبَ عَشَائِرِهِمْ وَبُيُوتِ آبَائِهِمْ
\par 47 مِنِ ابْنِ ثَلاثِينَ سَنَةً فَصَاعِداً إِلى ابْنِ خَمْسِينَ سَنَةً كُلُّ الدَّاخِلِينَ لِيَعْمَلُوا عَمَل الخِدْمَةِ وَعَمَل الحِمْلِ فِي خَيْمَةِ الاِجْتِمَاعِ
\par 48 كَانَ المَعْدُودُونَ مِنْهُمْ ثَمَانِيَةَ آلافٍ وَخَمْسَ مِئَةٍ وَثَمَانِينَ.
\par 49 حَسَبَ قَوْلِ الرَّبِّ عَنْ يَدِ مُوسَى عُدَّ كُلُّ إِنْسَانٍ عَلى خِدْمَتِهِ وَعَلى حِمْلِهِ الذِينَ عَدَّهُمْ مُوسَى كَمَا أَمَرَهُ الرَّبُّ.

\chapter{5}

\par 1 وَأَمَرَ الرَّبُّ مُوسَى:
\par 2 «أَوْصِ بَنِي إِسْرَائِيل أَنْ يَنْفُوا مِنَ المَحَلةِ كُل أَبْرَصَ وَكُل ذِي سَيْلٍ وَكُل مُتَنَجِّسٍ لِمَيِّتٍ.
\par 3 الذَّكَرَ وَالأُنْثَى تَنْفُونَ. إِلى خَارِجِ المَحَلةِ تَنْفُونَهُمْ لِكَيْلا يُنَجِّسُوا مَحَلاتِهِمْ حَيْثُ أَنَا سَاكِنٌ فِي وَسَطِهِمْ».
\par 4 فَفَعَل هَكَذَا بَنُو إِسْرَائِيل وَنَفُوهُمْ إِلى خَارِجِ المَحَلةِ. كَمَا كَلمَ الرَّبُّ مُوسَى هَكَذَا فَعَل بَنُو إِسْرَائِيل.
\par 5 وَأَمَرَ الرَّبُّ مُوسَى:
\par 6 «قُل لِبَنِي إِسْرَائِيل: إِذَا عَمِل رَجُلٌ أَوِ امْرَأَةٌ شَيْئاً مِنْ جَمِيعِ خَطَايَا الإِنْسَانِ وَخَانَ خِيَانَةً بِالرَّبِّ فَقَدْ أَذْنَبَتْ تِلكَ النَّفْسُ.
\par 7 فَلتُقِرَّ بِخَطِيَّتِهَا التِي عَمِلتْ وَتَرُدَّ مَا أَذْنَبَتْ بِهِ بِعَيْنِهِ وَتَزِدْ عَليْهِ خُمْسَهُ وَتَدْفَعْهُ لِلذِي أَذْنَبَتْ إِليْهِ.
\par 8 وَإِنْ كَانَ ليْسَ لِلرَّجُلِ وَلِيٌّ لِيَرُدَّ إِليْهِ المُذْنَبَ بِهِ فَالمُذْنَبُ بِهِ المَرْدُودُ يَكُونُ لِلرَّبِّ لأَجْلِ الكَاهِنِ فَضْلاً عَنْ كَبْشِ الكَفَّارَةِ الذِي يُكَفِّرُ بِهِ عَنْهُ.
\par 9 وَكُلُّ رَفِيعَةٍ مَعَ كُلِّ أَقْدَاسِ بَنِي إِسْرَائِيل التِي يُقَدِّمُونَهَا لِلكَاهِنِ تَكُونُ لهُ.
\par 10 وَالإِنْسَانُ أَقْدَاسُهُ تَكُونُ لهُ. إِذَا أَعْطَى إِنْسَانٌ شَيْئاً لِلكَاهِنِ فَلهُ يَكُونُ».
\par 11 وَأَمَرَ الرَّبُّ مُوسَى:
\par 12 «قُل لِبَنِي إِسْرَائِيل: إِذَا زَاغَتِ امْرَأَةُ رَجُلٍ وَخَانَتْهُ خِيَانَةً
\par 13 وَاضْطَجَعَ مَعَهَا رَجُلٌ اضْطِجَاعَ زَرْعٍ وَأُخْفِيَ ذَلِكَ عَنْ عَيْنَيْ رَجُلِهَا وَاسْتَتَرَتْ وَهِيَ نَجِسَةٌ وَليْسَ شَاهِدٌ عَليْهَا وَهِيَ لمْ تُؤْخَذْ
\par 14 فَاعْتَرَاهُ رُوحُ الغَيْرَةِ وَغَارَ عَلى امْرَأَتِهِ وَهِيَ نَجِسَةٌ أَوِ اعْتَرَاهُ رُوحُ الغَيْرَةِ وَغَارَ عَلى امْرَأَتِهِ وَهِيَ ليْسَتْ نَجِسَةً
\par 15 يَأْتِي الرَّجُلُ بَامْرَأَتِهِ إِلى الكَاهِنِ وَيَأْتِي بِقُرْبَانِهَا مَعَهَا: عُشْرِ الإِيفَةِ مِنْ طَحِينِ شَعِيرٍ لا يَصُبُّ عَليْهِ زَيْتاً وَلا يَجْعَلُ عَليْهِ لُبَاناً لأَنَّهُ تَقْدِمَةُ غَيْرَةٍ تَقْدِمَةُ تِذْكَارٍ تُذَكِّرُ ذَنْباً.
\par 16 فَيُقَدِّمُهَا الكَاهِنُ وَيُوقِفُهَا أَمَامَ الرَّبِّ
\par 17 وَيَأْخُذُ الكَاهِنُ مَاءً مُقَدَّساً فِي إِنَاءِ خَزَفٍ وَيَأْخُذُ الكَاهِنُ مِنَ الغُبَارِ الذِي فِي أَرْضِ المَسْكَنِ وَيَجْعَلُ فِي المَاءِ
\par 18 وَيُوقِفُ الكَاهِنُ المَرْأَةَ أَمَامَ الرَّبِّ وَيَكْشِفُ رَأْسَ المَرْأَةِ وَيَجْعَلُ فِي يَدَيْهَا تَقْدِمَةَ التِّذْكَارِ التِي هِيَ تَقْدِمَةُ الغَيْرَةِ وَفِي يَدِ الكَاهِنِ يَكُونُ مَاءُ اللعْنَةِ المُرُّ.
\par 19 وَيَسْتَحْلِفُ الكَاهِنُ المَرْأَةَ وَيَقُولُ لهَا: إِنْ كَانَ لمْ يَضْطَجِعْ مَعَكِ رَجُلٌ وَإِنْ كُنْتِ لمْ تَزِيغِي إِلى نَجَاسَةٍ مِنْ تَحْتِ رَجُلِكِ فَكُونِي بَرِيئَةً مِنْ مَاءِ اللعْنَةِ هَذَا المُرِّ.
\par 20 وَلكِنْ إِنْ كُنْتِ قَدْ زُغْتِ مِنْ تَحْتِ رَجُلِكِ وَتَنَجَّسْتِ وَجَعَل مَعَكِ رَجُلٌ غَيْرُ رَجُلِكِ مَضْجَعَهُ.
\par 21 يَسْتَحْلِفُ الكَاهِنُ المَرْأَةَ بِحَلفِ اللعْنَةِ وَيَقُولُ الكَاهِنُ لِلمَرْأَةِ: يَجْعَلُكِ الرَّبُّ لعْنَةً وَحَلفاً بَيْنَ شَعْبِكِ بِأَنْ يَجْعَل الرَّبُّ فَخْذَكِ سَاقِطَةً وَبَطْنَكِ وَارِماً.
\par 22 وَيَدْخُلُ مَاءُ اللعْنَةِ هَذَا فِي أَحْشَائِكِ لِوَرَمِ البَطْنِ وَلِإِسْقَاطِ الفَخْذِ. فَتَقُولُ المَرْأَةُ: آمِينَ آمِينَ.
\par 23 وَيَكْتُبُ الكَاهِنُ هَذِهِ اللعْنَاتِ فِي الكِتَابِ ثُمَّ يَمْحُوهَا فِي المَاءِ المُرِّ
\par 24 وَيَسْقِي المَرْأَةَ مَاءَ اللعْنَةِ المُرَّ فَيَدْخُلُ فِيهَا مَاءُ اللعْنَةِ لِلمَرَارَةِ.
\par 25 وَيَأْخُذُ الكَاهِنُ مِنْ يَدِ المَرْأَةِ تَقْدِمَةَ الغَيْرَةِ وَيُرَدِّدُ التَّقْدِمَةَ أَمَامَ الرَّبِّ وَيُقَدِّمُهَا إِلى المَذْبَحِ.
\par 26 وَيَقْبِضُ الكَاهِنُ مِنَ التَّقْدِمَةِ تِذْكَارَهَا وَيُوقِدُهُ عَلى المَذْبَحِ وَبَعْدَ ذَلِكَ يَسْقِي المَرْأَةَ المَاءَ.
\par 27 وَمَتَى سَقَاهَا المَاءَ فَإِنْ كَانَتْ قَدْ تَنَجَّسَتْ وَخَانَتْ رَجُلهَا يَدْخُلُ فِيهَا مَاءُ اللعْنَةِ لِلمَرَارَةِ فَيَرِمُ بَطْنُهَا وَتَسْقُطُ فَخْذُهَا فَتَصِيرُ المَرْأَةُ لعْنَةً فِي وَسَطِ شَعْبِهَا.
\par 28 وَإِنْ لمْ تَكُنِ المَرْأَةُ قَدْ تَنَجَّسَتْ بَل كَانَتْ طَاهِرَةً تَتَبَرَّأُ وَتَحْبَلُ بِزَرْعٍ».
\par 29 هَذِهِ شَرِيعَةُ الغَيْرَةِ. إِذَا زَاغَتِ امْرَأَةٌ مِنْ تَحْتِ رَجُلِهَا وَتَنَجَّسَتْ
\par 30 أَوْ إِذَا اعْتَرَى رَجُلاً رُوحُ غَيْرَةٍ فَغَارَ عَلى امْرَأَتِهِ يُوقِفُ المَرْأَةَ أَمَامَ الرَّبِّ وَيَعْمَلُ لهَا الكَاهِنُ كُل هَذِهِ الشَّرِيعَةِ
\par 31 فَيَتَبَرَّأُ الرَّجُلُ مِنَ الذَّنْبِ وَتِلكَ المَرْأَةُ تَحْمِلُ ذَنْبَهَا.

\chapter{6}

\par 1 وَأَمَرَ الرَّبُّ مُوسَى:
\par 2 «قُل لِبَنِي إِسْرَائِيل: إِذَا انْفَرَزَ رَجُلٌ أَوِ امْرَأَةٌ لِيَنْذُرَ نَذْرَ النَّذِيرِ لِيَنْتَذِرَ لِلرَّبِّ
\par 3 فَعَنِ الخَمْرِ وَالمُسْكِرِ يَفْتَرِزُ وَلا يَشْرَبْ خَل الخَمْرِ وَلا خَل المُسْكِرِ وَلا يَشْرَبْ مِنْ نَقِيعِ العِنَبِ وَلا يَأْكُل عِنَباً رَطْباً وَلا يَابِساً.
\par 4 كُل أَيَّامِ نَذْرِهِ لا يَأْكُل مِنْ كُلِّ مَا يُعْمَلُ مِنْ جَفْنَةِ الخَمْرِ مِنَ العَجَمِ حَتَّى القِشْرِ.
\par 5 كُل أَيَّامِ نَذْرِ افْتِرَازِهِ لا يَمُرُّ مُوسَى عَلى رَأْسِهِ. إِلى كَمَالِ الأَيَّامِ التِي انْتَذَرَ فِيهَا لِلرَّبِّ يَكُونُ مُقَدَّساً وَيُرَبِّي خُصَل شَعْرِ رَأْسِهِ.
\par 6 كُل أَيَّامِ انْتِذَارِهِ لِلرَّبِّ لا يَأْتِي إِلى جَسَدِ مَيِّتٍ.
\par 7 أَبُوهُ وَأُمُّهُ وَأَخُوهُ وَأُخْتُهُ لا يَتَنَجَّسْ مِنْ أَجْلِهِمْ عِنْدَ مَوْتِهِمْ لأَنَّ انْتِذَارَ إِلهِهِ عَلى رَأْسِهِ.
\par 8 إِنَّهُ كُل أَيَّامِ انْتِذَارِهِ مُقَدَّسٌ لِلرَّبِّ.
\par 9 وَإِذَا مَاتَ مَيِّتٌ عِنْدَهُ بَغْتَةً عَلى فَجْأَةٍ فَنَجَّسَ رَأْسَ انْتِذَارِهِ يَحْلِقُ رَأْسَهُ يَوْمَ طُهْرِهِ. فِي اليَوْمِ السَّابِعِ يَحْلِقُهُ.
\par 10 وَفِي اليَوْمِ الثَّامِنِ يَأْتِي بِيَمَامَتَيْنِ أَوْ بِفَرْخَيْ حَمَامٍ إِلى الكَاهِنِ إِلى بَابِ خَيْمَةِ الاِجْتِمَاعِ
\par 11 فَيَعْمَلُ الكَاهِنُ وَاحِداً ذَبِيحَةَ خَطِيَّةٍ وَالآخَرَ مُحْرَقَةً وَيُكَفِّرُ عَنْهُ مَا أَخْطَأَ بِسَبَبِ الميِّتِ وَيُقَدِّسُ رَأْسَهُ فِي ذَلِكَ اليَوْمِ.
\par 12 فَمَتَى نَذَرَ لِلرَّبِّ أَيَّامَ انْتِذَارِهِ يَأْتِي بِخَرُوفٍ حَوْلِيٍّ ذَبِيحَةَ إِثْمٍ وَأَمَّا الأَيَّامُ الأُولى فَتَسْقُطُ لأَنَّهُ نَجَّسَ انْتِذَارَهُ.
\par 13 «وَهَذِهِ شَرِيعَةُ النَّذِيرِ: يَوْمَ تَكْمُلُ أَيَّامُ انْتِذَارِهِ يُؤْتَى بِهِ إِلى بَابِ خَيْمَةِ الاِجْتِمَاعِ
\par 14 فَيُقَرِّبُ قُرْبَانَهُ لِلرَّبِّ خَرُوفاً وَاحِداً حَوْلِيّاً صَحِيحاً مُحْرَقَةً وَنَعْجَةً وَاحِدَةً حَوْلِيَّةً صَحِيحَةً ذَبِيحَةَ خَطِيَّةٍ وَكَبْشاً وَاحِداً صَحِيحاً ذَبِيحَةَ سَلامَةٍ
\par 15 وَسَل فَطِيرٍ مِنْ دَقِيقٍ أَقْرَاصاً مَلتُوتَةً بِزَيْتٍ وَرِقَاقَ فَطِيرٍ مَدْهُونَةً بِزَيْتٍ مَعَ تَقْدِمَتِهَا وَسَكَائِبِهَا
\par 16 فَيُقَدِّمُهَا الكَاهِنُ أَمَامَ الرَّبِّ وَيَعْمَلُ ذَبِيحَةَ خَطِيَّتِهِ وَمُحْرَقَتَهُ.
\par 17 وَالكَبْشُ يَعْمَلُهُ ذَبِيحَةَ سَلامَةٍ لِلرَّبِّ مَعَ سَلِّ الفَطِيرِ وَيَعْمَلُ الكَاهِنُ تَقْدِمَتَهُ وَسَكِيبَهُ.
\par 18 وَيَحْلِقُ النَّذِيرُ لدَى بَابِ خَيْمَةِ الاِجْتِمَاعِ رَأْسَ انْتِذَارِهِ وَيَأْخُذُ شَعْرَ رَأْسِ انْتِذَارِهِ وَيَجْعَلُهُ عَلى النَّارِ التِي تَحْتَ ذَبِيحَةِ السَّلامَةِ.
\par 19 وَيَأْخُذُ الكَاهِنُ السَّاعِدَ مَسْلُوقاً مِنَ الكَبْشِ وَقُرْصَ فَطِيرٍ وَاحِداً مِنَ السَّلِّ وَرُقَاقَةَ فَطِيرٍ وَاحِدَةً وَيَجْعَلُهَا فِي يَدَيِ النَّذِيرِ بَعْدَ حَلقِهِ شَعْرَ انْتِذَارِهِ
\par 20 وَيُرَدِّدُهَا الكَاهِنُ تَرْدِيداً أَمَامَ الرَّبِّ. إِنَّهُ قُدْسٌ لِلكَاهِنِ مَعَ صَدْرِ التَّرْدِيدِ وَسَاقِ الرَّفِيعَةِ. وَبَعْدَ ذَلِكَ يَشْرَبُ النَّذِيرُ خَمْراً.
\par 21 هَذِهِ شَرِيعَةُ النَّذِيرِ الذِي يَنْذُرُ. قُرْبَانُهُ لِلرَّبِّ عَنِ انْتِذَارِهِ فَضْلاً عَمَّا تَنَالُ يَدُهُ. حَسَبَ نَذْرِهِ الذِي نَذَرَ كَذَلِكَ يَعْمَلُ حَسَبَ شَرِيعَةِ انْتِذَارِهِ».
\par 22 وَأَمَرَ الرَّبُّ مُوسَى:
\par 23 «قُل لِهَارُونَ وَبَنِيهِ: هَكَذَا تُبَارِكُونَ بَنِي إِسْرَائِيل:
\par 24 يُبَارِكُكَ الرَّبُّ وَيَحْرُسُكَ.
\par 25 يُضِيءُ الرَّبُّ بِوَجْهِهِ عَليْكَ وَيَرْحَمُكَ.
\par 26 يَرْفَعُ الرَّبُّ وَجْهَهُ عَليْكَ وَيَمْنَحُكَ سَلاماً.
\par 27 فَيَجْعَلُونَ اسْمِي عَلى بَنِي إِسْرَائِيل وَأَنَا أُبَارِكُهُمْ».

\chapter{7}

\par 1 وَيَوْمَ فَرَغَ مُوسَى مِنْ إِقَامَةِ المَسْكَنِ وَمَسَحَهُ وَقَدَّسَهُ وَجَمِيعَ أَمْتِعَتِهِ وَالمَذْبَحَ وَجَمِيعَ أَمْتِعَتِهِ وَمَسَحَهَا وَقَدَّسَهَا
\par 2 قَرَّبَ رُؤَسَاءُ إِسْرَائِيل رُؤُوسُ بُيُوتِ آبَائِهِمْ هُمْ رُؤَسَاءُ الأَسْبَاطِ الذِينَ وَقَفُوا عَلى المَعْدُودِينَ.
\par 3 أَتُوا بِقَرَابِينِهِمْ أَمَامَ الرَّبِّ: سِتَّ عَجَلاتٍ مُغَطَّاةً وَاثْنَيْ عَشَرَ ثَوْراً. لِكُلِّ رَئِيسَيْنِ عَجَلةٌ وَلِكُلِّ وَاحِدٍ ثَوْرٌ وَقَدَّمُوهَا أَمَامَ المَسْكَنِ.
\par 4 فَقَال الرَّبُّ لِمُوسَى:
\par 5 «خُذْهَا مِنْهُمْ فَتَكُونَ لِعَمَلِ خِدْمَةِ خَيْمَةِ الاِجْتِمَاعِ وَأَعْطِهَا لِلاوِيِّينَ لِكُلِّ وَاحِدٍ حَسَبَ خِدْمَتِهِ».
\par 6 فَأَخَذَ مُوسَى العَجَلاتِ وَالثِّيرَانَ وَأَعْطَاهَا لِلاوِيِّينَ.
\par 7 اثْنَتَانِ مِنَ العَجَلاتِ وَأَرْبَعَةٌ مِنَ الثِّيرَانِ أَعْطَاهَا لِبَنِي جَرْشُونَ حَسَبَ خِدْمَتِهِمْ.
\par 8 وَأَرْبَعٌ مِنَ العَجَلاتِ وَثَمَانِيَةٌ مِنَ الثِّيرَانِ أَعْطَاهَا لِبَنِي مَرَارِي حَسَبَ خِدْمَتِهِمْ بِيَدِ إِيثَامَارَ بْنِ هَارُونَ الكَاهِنِ.
\par 9 وَأَمَّا بَنُو قَهَاتَ فَلمْ يُعْطِهِمْ لأَنَّ خِدْمَةَ القُدْسِ كَانَتْ عَليْهِمْ. عَلى الأَكْتَافِ كَانُوا يَحْمِلُونَ.
\par 10 وَقَرَّبَ الرُّؤَسَاءُ لِتَدْشِينِ المَذْبَحِ يَوْمَ مَسْحِهِ. وَقَدَّمَ الرُّؤَسَاءُ قَرَابِينَهُمْ أَمَامَ المَذْبَحِ.
\par 11 فَقَال الرَّبُّ لِمُوسَى: «رَئِيساً رَئِيساً فِي كُلِّ يَوْمٍ يُقَرِّبُونَ قَرَابِينَهُمْ لِتَدْشِينِ المَذْبَحِ».
\par 12 وَالذِي قَرَّبَ قُرْبَانَهُ فِي اليَوْمِ الأَوَّلِ نَحْشُونُ بْنُ عَمِّينَادَابَ مِنْ سِبْطِ يَهُوذَا.
\par 13 وَقُرْبَانُهُ طَبَقٌ وَاحِدٌ مِنْ فِضَّةٍ وَزْنُهُ مِئَةٌ وَثَلاثُونَ شَاقِلاً وَمِنْضَحَةٌ وَاحِدَةٌ مِنْ فِضَّةٍ سَبْعُونَ شَاقِلاً عَلى شَاقِلِ القُدْسِ كِلتَاهُمَا مَمْلُوءَتَانِ دَقِيقاً مَلتُوتاً بِزَيْتٍ لِتَقْدِمَةٍ
\par 14 وَصَحْنٌ وَاحِدٌ عَشَرَةُ شَوَاقِل مِنْ ذَهَبٍ مَمْلُوءٌ بَخُوراً
\par 15 وَثَوْرٌ وَاحِدٌ ابْنُ بَقَرٍ وَكَبْشٌ وَاحِدٌ وَخَرُوفٌ وَاحِدٌ حَوْلِيٌّ لِمُحْرَقَةٍ
\par 16 وَتَيْسٌ وَاحِدٌ مِنَ المَعَزِ لِذَبِيحَةِ خَطِيَّةٍ.
\par 17 وَلِذَبِيحَةِ السَّلامَةِ ثَوْرَانِ وَخَمْسَةُ كِبَاشٍ وَخَمْسَةُ تُيُوسٍ وَخَمْسَةُ خِرَافٍ حَوْلِيَّةٍ. هَذَا قُرْبَانُ نَحْشُونَ بْنِ عَمِّينَادَابَ.
\par 18 وَفِي اليَوْمِ الثَّانِي قَرَّبَ نَثَنَائِيلُ بْنُ صُوغَرَ رَئِيسُ يَسَّاكَرَ.
\par 19 قَرَّبَ قُرْبَانَهُ طَبَقاً وَاحِداً مِنْ فِضَّةٍ وَزْنُهُ مِئَةٌ وَثَلاثُونَ شَاقِلاً وَمِنْضَحَةً وَاحِدَةً مِنْ فِضَّةٍ سَبْعِينَ شَاقِلاً عَلى شَاقِلِ القُدْسِ كِلتَاهُمَا مَمْلُوءَتَانِ دَقِيقاً مَلتُوتاً بِزَيْتٍ لِتَقْدِمَةٍ
\par 20 وَصَحْناً وَاحِداً عَشَرَةَ شَوَاقِل مِنْ ذَهَبٍ مَمْلُوّاً بَخُوراً
\par 21 وَثَوْراً وَاحِداً ابْنَ بَقَرٍ وَكَبْشاً وَاحِداً وَخَرُوفاً وَاحِداً حَوْلِيّاً لِمُحْرَقَةٍ
\par 22 وَتَيْساً وَاحِداً مِنَ المَعْزِ لِذَبِيحَةِ خَطِيَّةٍ.
\par 23 وَلِذَبِيحَةِ السَّلامَةِ ثَوْرَيْنِ وَخَمْسَةَ كِبَاشٍ وَخَمْسَةَ تُيُوسٍ وَخَمْسَةَ خِرَافٍ حَوْلِيَّةً. هَذَا قُرْبَانُ نَثَنَائِيل بْنِ صُوغَرَ.
\par 24 وَفِي اليَوْمِ الثَّالِثِ رَئِيسُ بَنِي زَبُولُونَ أَلِيآبُ بْنُ حِيلُونَ.
\par 25 قُرْبَانُهُ طَبَقٌ وَاحِدٌ مِنْ فِضَّةٍ وَزْنُهُ مِئَةٌ وَثَلاثُونَ شَاقِلاً وَمِنْضَحَةٌ وَاحِدَةٌ مِنْ فِضَّةٍ سَبْعُونَ شَاقِلاً عَلى شَاقِلِ القُدْسِ كِلتَاهُمَا مَمْلُوءَتَانِ دَقِيقاً مَلتُوتاً بِزَيْتٍ لِتَقْدِمَةٍ
\par 26 وَصَحْنٌ وَاحِدٌ عَشَرَةُ شَوَاقِل مِنْ ذَهَبٍ مَمْلُوءٌ بَخُوراً
\par 27 وَثَوْرٌ وَاحِدٌ ابْنُ بَقَرٍ وَكَبْشٌ وَاحِدٌ وَخَرُوفٌ وَاحِدٌ حَوْلِيٌّ لِمُحْرَقَةٍ
\par 28 وَتَيْسٌ وَاحِدٌ مِنَ المَعْزِ لِذَبِيحَةِ خَطِيَّةٍ
\par 29 وَلِذَبِيحَةِ السَّلامَةِ ثَوْرَانِ وَخَمْسَةُ كِبَاشٍ وَخَمْسَةُ تُيُوسٍ وَخَمْسَةُ خِرَافٍ حَوْلِيَّةٍ. هَذَا قُرْبَانُ أَلِيآبَ بْنِ حِيلُونَ.
\par 30 وَفِي اليَوْمِ الرَّابِعِ رَئِيسُ بَنِي رَأُوبَيْنَ أَلِيصُورُ بْنُ شَدَيْئُورَ.
\par 31 قُرْبَانُهُ طَبَقٌ وَاحِدٌ مِنْ فِضَّةٍ وَزْنُهُ مِئَةٌ وَثَلاثُونَ شَاقِلاً وَمِنْضَحَةٌ وَاحِدَةٌ مِنْ فِضَّةٍ سَبْعُونَ شَاقِلاً عَلى شَاقِلِ القُدْسِ كِلتَاهُمَا مَمْلُوءَتَانِ دَقِيقاً مَلتُوتاً بِزَيْتٍ لِتَقْدِمَةٍ
\par 32 وَصَحْنٌ وَاحِدٌ عَشَرَةُ شَوَاقِل مِنْ ذَهَبٍ مَمْلُوءٌ بَخُوراً
\par 33 وَثَوْرٌ وَاحِدٌ ابْنُ بَقَرٍ وَكَبْشٌ وَاحِدٌ وَخَرُوفٌ وَاحِدٌ حَوْلِيٌّ لِمُحْرَقَةٍ
\par 34 وَتَيْسٌ وَاحِدٌ مِنَ المَعْزِ لِذَبِيحَةِ خَطِيَّةٍ.
\par 35 وَلِذَبِيحَةِ السَّلامَةِ ثَوْرَانِ وَخَمْسَةُ كِبَاشٍ وَخَمْسَةُ تُيُوسٍ وَخَمْسَةُ خِرَافٍ حَوْلِيَّةٍ. هَذَا قُرْبَانُ أَلِيصُورَ بْنِ شَدَيْئُورَ.
\par 36 وَفِي اليَوْمِ الخَامِسِ رَئِيسُ بَنِي شَمْعُونَ شَلُومِيئِيلُ بْنُ صُورِيشَدَّاي.
\par 37 قُرْبَانُهُ طَبَقٌ وَاحِدٌ مِنْ فِضَّةٍ وَزْنُهُ مِئَةٌ وَثَلاثُونَ شَاقِلاً وَمِنْضَحَةٌ وَاحِدَةٌ مِنْ فِضَّةٍ سَبْعُونَ شَاقِلاً عَلى شَاقِلِ القُدْسِ كِلتَاهُمَا مَمْلُوءَتَانِ دَقِيقاً مَلتُوتاً بِزَيْتٍ لِتَقْدِمَةٍ
\par 38 وَصَحْنٌ وَاحِدٌ عَشَرَةُ شَوَاقِل مِنْ ذَهَبٍ مَمْلُوءٌ بَخُوراً
\par 39 وَثَوْرٌ وَاحِدٌ ابْنُ بَقَرٍ وَكَبْشٌ وَاحِدٌ وَخَرُوفٌ وَاحِدٌ حَوْلِيٌّ لِمُحْرَقَةٍ
\par 40 وَتَيْسٌ وَاحِدٌ مِنَ المَعْزِ لِذَبِيحَةِ خَطِيَّةٍ.
\par 41 وَلِذَبِيحَةِ السَّلامَةِ ثَوْرَانِ وَخَمْسَةُ كِبَاشٍ وَخَمْسَةُ تُيُوسٍ وَخَمْسَةُ خِرَافٍ حَوْلِيَّةٍ. هَذَا قُرْبَانُ شَلُومِيئِيل بْنِ صُورِيشَدَّاي.
\par 42 وَفِي اليَوْمِ السَّادِسِ رَئِيسُ بَنِي جَادَ أَلِيَاسَافُ بْنُ دَعُوئِيل.
\par 43 قُرْبَانُهُ طَبَقٌ وَاحِدٌ مِنْ فِضَّةٍ وَزْنُهُ مِئَةٌ وَثَلاثُونَ شَاقِلاً وَمِنْضَحَةٌ وَاحِدَةٌ مِنْ فِضَّةٍ سَبْعُونَ شَاقِلاً عَلى شَاقِلِ القُدْسِ كِلتَاهُمَا مَمْلُوءَتَانِ دَقِيقاً مَلتُوتاً بِزَيْتٍ لِتَقْدِمَةٍ
\par 44 وَصَحْنٌ وَاحِدٌ عَشَرَةُ شَوَاقِل مِنْ ذَهَبٍ مَمْلُوءٌ بَخُوراً
\par 45 وَثَوْرٌ وَاحِدٌ ابْنُ بَقَرٍ وَكَبْشٌ وَاحِدٌ وَخَرُوفٌ وَاحِدٌ حَوْلِيٌّ لِمُحْرَقَةٍ
\par 46 وَتَيْسٌ وَاحِدٌ مِنَ المَعْزِ لِذَبِيحَةِ خَطِيَّةٍ.
\par 47 وَلِذَبِيحَةِ السَّلامَةِ ثَوْرَانِ وَخَمْسَةُ كِبَاشٍ وَخَمْسَةُ تُيُوسٍ وَخَمْسَةُ خِرَافٍ حَوْلِيَّةٍ. هَذَا قُرْبَانُ أَلِيَاسَافَ بْنِ دَعُوئِيل.
\par 48 وَفِي اليَوْمِ السَّابِعِ رَئِيسُ بَنِي أَفْرَايِمَ أَلِيشَمَعُ بْنُ عَمِّيهُودَ.
\par 49 قُرْبَانُهُ طَبَقٌ وَاحِدٌ مِنْ فِضَّةٍ وَزْنُهُ مِئَةٌ وَثَلاثُونَ شَاقِلاً وَمِنْضَحَةٌ وَاحِدَةٌ مِنْ فِضَّةٍ سَبْعُونَ شَاقِلاً عَلى شَاقِلِ القُدْسِ كِلتَاهُمَا مَمْلُوءَتَانِ دَقِيقاً مَلتُوتاً بِزَيْتٍ لِتَقْدِمَةٍ
\par 50 وَصَحْنٌ وَاحِدٌ عَشَرَةُ شَوَاقِل مِنْ ذَهَبٍ مَمْلُوءٌ بَخُوراً
\par 51 وَثَوْرٌ وَاحِدٌ ابْنُ بَقَرٍ وَكَبْشٌ وَاحِدٌ وَخَرُوفٌ وَاحِدٌ حَوْلِيٌّ لِمُحْرَقَةٍ
\par 52 وَتَيْسٌ وَاحِدٌ مِنَ المَعْزِ لِذَبِيحَةِ خَطِيَّةٍ
\par 53 وَلِذَبِيحَةِ السَّلامَةِ ثَوْرَانِ وَخَمْسَةُ كِبَاشٍ وَخَمْسَةُ تُيُوسٍ وَخَمْسَةُ خِرَافٍ حَوْلِيَّةٍ. هَذَا قُرْبَانُ أَلِيشَمَعَ بْنِ عَمِّيهُودَ.
\par 54 وَفِي اليَوْمِ الثَّامِنِ رَئِيسُ بَنِي مَنَسَّى جَمْلِيئِيلُ بْنُ فَدَهْصُورَ.
\par 55 قُرْبَانُهُ طَبَقٌ وَاحِدٌ مِنْ فِضَّةٍ وَزْنُهُ مِئَةٌ وَثَلاثُونَ شَاقِلاً وَمِنْضَحَةٌ وَاحِدَةٌ مِنْ فِضَّةٍ سَبْعُونَ شَاقِلاً عَلى شَاقِلِ القُدْسِ كِلتَاهُمَا مَمْلُوءَتَانِ دَقِيقاً مَلتُوتاً بِزَيْتٍ لِتَقْدِمَةٍ
\par 56 وَصَحْنٌ وَاحِدٌ عَشَرَةُ شَوَاقِل مِنْ ذَهَبٍ مَمْلُوءٌ بَخُوراً
\par 57 وَثَوْرٌ وَاحِدٌ ابْنُ بَقَرٍ وَكَبْشٌ وَاحِدٌ وَخَرُوفٌ وَاحِدٌ حَوْلِيٌّ لِمُحْرَقَةٍ
\par 58 وَتَيْسٌ وَاحِدٌ مِنَ المَعْزِ لِذَبِيحَةِ خَطِيَّةٍ.
\par 59 وَلِذَبِيحَةِ السَّلامَةِ ثَوْرَانِ وَخَمْسَةُ كِبَاشٍ وَخَمْسَةُ تُيُوسٍ وَخَمْسَةُ خِرَافٍ حَوْلِيَّةٍ. هَذَا قُرْبَانُ جَمْلِيئِيل بْنِ فَدَهْصُورَ.
\par 60 وَفِي اليَوْمِ التَّاسِعِ رَئِيسُ بَنِي بِنْيَامِينَ أَبِيدَنُ بْنُ جِدْعُونِي.
\par 61 قُرْبَانُهُ طَبَقٌ وَاحِدٌ مِنْ فِضَّةٍ وَزْنُهُ مِئَةٌ وَثَلاثُونَ شَاقِلاً وَمِنْضَحَةٌ وَاحِدَةٌ مِنْ فِضَّةٍ سَبْعُونَ شَاقِلاً عَلى شَاقِلِ القُدْسِ كِلتَاهُمَا مَمْلُوءَتَانِ دَقِيقاً مَلتُوتاً بِزَيْتٍ لِتَقْدِمَةٍ
\par 62 وَصَحْنٌ وَاحِدٌ عَشَرَةُ شَوَاقِل مِنْ ذَهَبٍ مَمْلُوءٌ بَخُوراً
\par 63 وَثَوْرٌ وَاحِدٌ ابْنُ بَقَرٍ وَكَبْشٌ وَاحِدٌ وَخَرُوفٌ وَاحِدٌ حَوْلِيٌّ لِمُحْرَقَةٍ
\par 64 وَتَيْسٌ وَاحِدٌ مِنَ المَعْزِ لِذَبِيحَةِ خَطِيَّةٍ
\par 65 وَلِذَبِيحَةِ السَّلامَةِ ثَوْرَانِ وَخَمْسَةُ كِبَاشٍ وَخَمْسَةُ تُيُوسٍ وَخَمْسَةُ خِرَافٍ حَوْلِيَّةٍ. هَذَا قُرْبَانُ أَبِيدَنَ بْنِ جِدْعُونِي.
\par 66 وَفِي اليَوْمِ العَاشِرِ رَئِيسُ بَنِي دَانَ أَخِيعَزَرُ بْنُ عَمِّيشَدَّاي.
\par 67 قُرْبَانُهُ طَبَقٌ وَاحِدٌ مِنْ فِضَّةٍ وَزْنُهُ مِئَةٌ وَثَلاثُونَ شَاقِلاً وَمِنْضَحَةٌ وَاحِدَةٌ مِنْ فِضَّةٍ سَبْعُونَ شَاقِلاً عَلى شَاقِلِ القُدْسِ كِلتَاهُمَا مَمْلُوءَتَانِ دَقِيقاً مَلتُوتاً بِزَيْتٍ لِتَقْدِمَةٍ
\par 68 وَصَحْنٌ وَاحِدٌ عَشَرَةُ شَوَاقِل مِنْ ذَهَبٍ مَمْلُوءٌ بَخُوراً
\par 69 وَثَوْرٌ وَاحِدٌ ابْنُ بَقَرٍ وَكَبْشٌ وَاحِدٌ وَخَرُوفٌ وَاحِدٌ حَوْلِيٌّ لِمُحْرَقَةٍ
\par 70 وَتَيْسٌ وَاحِدٌ مِنَ المَعْزِ لِذَبِيحَةِ خَطِيَّةٍ.
\par 71 وَلِذَبِيحَةِ السَّلامَةِ ثَوْرَانِ وَخَمْسَةُ كِبَاشٍ وَخَمْسَةُ تُيُوسٍ وَخَمْسَةُ خِرَافٍ حَوْلِيَّةٍ. هَذَا قُرْبَانُ أَخِيعَزَرَ بْنِ عَمِّيشَدَّاي.
\par 72 وَفِي اليَوْمِ الحَادِي عَشَرَ رَئِيسُ بَنِي أَشِيرَ فَجْعِيئِيلُ بْنُ عُكْرَنَ.
\par 73 قُرْبَانُهُ طَبَقٌ وَاحِدٌ مِنْ فِضَّةٍ وَزْنُهُ مِئَةٌ وَثَلاثُونَ شَاقِلاً وَمِنْضَحَةٌ وَاحِدَةٌ مِنْ فِضَّةٍ سَبْعُونَ شَاقِلاً عَلى شَاقِلِ القُدْسِ كِلتَاهُمَا مَمْلُوءَتَانِ دَقِيقاً مَلتُوتاً بِزَيْتٍ لِتَقْدِمَةٍ
\par 74 وَصَحْنٌ وَاحِدٌ عَشَرَةُ شَوَاقِل مِنْ ذَهَبٍ مَمْلُوءٌ بَخُوراً
\par 75 وَثَوْرٌ وَاحِدٌ ابْنُ بَقَرٍ وَكَبْشٌ وَاحِدٌ وَخَرُوفٌ وَاحِدٌ حَوْلِيٌّ لِمُحْرَقَةٍ
\par 76 وَتَيْسٌ وَاحِدٌ مِنَ المَعْزِ لِذَبِيحَةِ خَطِيَّةٍ.
\par 77 وَلِذَبِيحَةِ السَّلامَةِ ثَوْرَانِ وَخَمْسَةُ كِبَاشٍ وَخَمْسَةُ تُيُوسٍ وَخَمْسَةُ خِرَافٍ حَوْلِيَّةٍ. هَذَا قُرْبَانُ فَجْعِيئِيل بْنِ عُكْرَنَ.
\par 78 وَفِي اليَوْمِ الثَّانِي عَشَرَ رَئِيسُ بَنِي نَفْتَالِي أَخِيرَعُ بْنُ عِينَنَ.
\par 79 قُرْبَانُهُ طَبَقٌ وَاحِدٌ مِنْ فِضَّةٍ وَزْنُهُ مِئَةٌ وَثَلاثُونَ شَاقِلاً وَمِنْضَحَةٌ وَاحِدَةٌ مِنْ فِضَّةٍ سَبْعُونَ شَاقِلاً عَلى شَاقِلِ القُدْسِ كِلتَاهُمَا مَمْلُوءَتَانِ دَقِيقاً مَلتُوتاً بِزَيْتٍ لِتَقْدِمَةٍ
\par 80 وَصَحْنٌ وَاحِدٌ عَشَرَةُ شَوَاقِل مِنْ ذَهَبٍ مَمْلُوءٌ بَخُوراً
\par 81 وَثَوْرٌ وَاحِدٌ ابْنُ بَقَرٍ وَكَبْشٌ وَاحِدٌ وَخَرُوفٌ وَاحِدٌ حَوْلِيٌّ لِمُحْرَقَةٍ
\par 82 وَتَيْسٌ وَاحِدٌ مِنَ المَعْزِ لِذَبِيحَةِ خَطِيَّةٍ.
\par 83 وَلِذَبِيحَةِ السَّلامَةِ ثَوْرَانِ وَخَمْسَةُ كِبَاشٍ وَخَمْسَةُ تُيُوسٍ وَخَمْسَةُ خِرَافٍ حَوْلِيَّةٍ. هَذَا قُرْبَانُ أَخِيرَعَ بْنِ عِينَنَ.
\par 84 هَذَا تَدْشِينُ المَذْبَحِ يَوْمَ مَسْحِهِ مِنْ رُؤَسَاءِ إِسْرَائِيل. أَطْبَاقُ فِضَّةٍ اثْنَا عَشَرَ وَمَنَاضِحُ فِضَّةٍ اثْنَتَا عَشرَةَ وَصُحُونُ ذَهَبٍ اثْنَا عَشَرَ
\par 85 كُلُّ طَبَقٍ مِئَةٌ وَثَلاثُونَ شَاقِل فِضَّةٍ وَكُلُّ مِنْضَحَةٍ سَبْعُونَ. جَمِيعُ فِضَّةِ الآنِيَةِ أَلفَانِ وَأَرْبَعُ مِئَةٍ عَلى شَاقِلِ القُدْسِ.
\par 86 وَصُحُونُ الذَّهَبِ اثْنَا عَشَرَ مَمْلُوءَةٌ بَخُوراً كُلُّ صَحْنٍ عَشَرَةٌ عَلى شَاقِلِ القُدْسِ. جَمِيعُ ذَهَبِ الصُّحُونِ مِئَةٌ وَعِشْرُونَ شَاقِلاً.
\par 87 كُلُّ الثِّيرَانِ لِلمُحْرَقَةِ اثْنَا عَشَرَ ثَوْراً وَالكِبَاشُ اثْنَا عَشَرَ وَالخِرَافُ الحَوْلِيَّةُ اثْنَا عَشَرَ مَعَ تَقْدِمَتِهَا وَتُيُوسُ المَعْزِ اثْنَا عَشَرَ لِذَبِيحَةِ الخَطِيَّةِ.
\par 88 وَكُلُّ الثِّيرَانِ لِذَبِيحَةِ السَّلامَةِ أَرْبَعَةٌ وَعِشْرُونَ ثَوْراً وَالكِبَاشُ سِتُّونَ وَالتُّيُوسُ سِتُّونَ وَالخِرَافُ الحَوْلِيَّةُ سِتُّونَ. هَذَا تَدْشِينُ المَذْبَحِ بَعْدَ مَسْحِهِ.
\par 89 فَلمَّا دَخَل مُوسَى إِلى خَيْمَةِ الاِجْتِمَاعِ لِيَتَكَلمَ مَعَهُ كَانَ يَسْمَعُ الصَّوْتَ يُكَلِّمُهُ مِنْ عَلى الغِطَاءِ الذِي عَلى تَابُوتِ الشَّهَادَةِ مِنْ بَيْنِ الكَرُوبَيْنِ فَكَلمَهُ.

\chapter{8}

\par 1 وَقَال الرَّبُّ لِمُوسَى:
\par 2 «قُل لِهَارُونَ: مَتَى رَفَعْتَ السُّرُجَ فَإِلى قُدَّامِ المَنَارَةِ تُضِيءُ السُّرُجُ السَّبْعَةُ».
\par 3 فَفَعَل هَارُونُ هَكَذَا. إِلى قُدَّامِ المَنَارَةِ رَفَعَ سُرُجَهَا كَمَا أَمَرَ الرَّبُّ مُوسَى.
\par 4 وَهَذِهِ هِيَ صَنْعَةُ المَنَارَةِ: مَسْحُولةٌ مِنْ ذَهَبٍ. حَتَّى سَاقُهَا وَزَهْرُهَا هِيَ مَسْحُولةٌ. حَسَبَ المَنْظَرِ الذِي أَرَاهُ الرَّبُّ مُوسَى هَكَذَا عَمِل المَنَارَةَ.
\par 5 وَقَال الرَّبُّ لِمُوسَى:
\par 6 «خُذِ اللاوِيِّينَ مِنْ بَيْنِ بَنِي إِسْرَائِيل وَطَهِّرْهُمْ
\par 7 وَهَكَذَا تَفْعَلُ لهُمْ لِتَطْهِيرِهِمِ: انْضِحْ عَليْهِمْ مَاءَ الخَطِيَّةِ وَليُمِرُّوا مُوسَى عَلى كُلِّ جَسَدِهِمْ وَيَغْسِلُوا ثِيَابَهُمْ فَيَتَطَهَّرُوا.
\par 8 ثُمَّ يَأْخُذُوا ثَوْراً ابْنَ بَقَرٍ وَتَقْدِمَتَهُ دَقِيقاً مَلتُوتاً بِزَيْتٍ. وَثَوْراً آخَرَ ابْنَ بَقَرٍ تَأْخُذُ لِذَبِيحَةِ خَطِيَّةٍ.
\par 9 فَتُقَدِّمُ اللاوِيِّينَ أَمَامَ خَيْمَةِ الاِجْتِمَاعِ وَتَجْمَعُ كُل جَمَاعَةِ بَنِي إِسْرَائِيل
\par 10 وَتُقَدِّمُ اللاوِيِّينَ أَمَامَ الرَّبِّ فَيَضَعُ بَنُو إِسْرَائِيل أَيْدِيَهُمْ عَلى اللاوِيِّينَ.
\par 11 وَيُرَدِّدُ هَارُونُ اللاوِيِّينَ تَرْدِيداً أَمَامَ الرَّبِّ مِنْ عِنْدِ بَنِي إِسْرَائِيل فَيَكُونُونَ لِخِدْمَةِ الرَّبِّ.
\par 12 ثُمَّ يَضَعُ اللاوِيُّونَ أَيْدِيَهُمْ عَلى رَأْسَيِ الثَّوْرَيْنِ فَتُقَرِّبُ الوَاحِدَ ذَبِيحَةَ خَطِيَّةٍ وَالآخَرَ مُحْرَقَةً لِلرَّبِّ لِلتَّكْفِيرِ عَنِ اللاوِيِّينَ.
\par 13 فَتُوقِفُ اللاوِيِّينَ أَمَامَ هَارُونَ وَبَنِيهِ وَتُرَدِّدُهُمْ تَرْدِيداً لِلرَّبِّ.
\par 14 وَتُفْرِزُ اللاوِيِّينَ مِنْ بَيْنِ بَنِي إِسْرَائِيل فَيَكُونُ اللاوِيُّونَ لِي.
\par 15 وَبَعْدَ ذَلِكَ يَأْتِي اللاوِيُّونَ لِيَخْدِمُوا خَيْمَةَ الاِجْتِمَاعِ فَتُطَهِّرُهُمْ وَتُرَدِّدُهُمْ تَرْدِيداً
\par 16 لأَنَّهُمْ مَوْهُوبُونَ لِي هِبَةً مِنْ بَيْنِ بَنِي إِسْرَائِيل. بَدَل كُلِّ فَاتِحِ رَحِمٍ بِكْرِ كُلٍّ مِنْ إِسْرَائِيل قَدِ اتَّخَذْتُهُمْ لِي.
\par 17 لأَنَّ لِي كُل بِكْرٍ فِي بَنِي إِسْرَائِيل مِنَ النَّاسِ وَمِنَ البَهَائِمِ. يَوْمَ ضَرَبْتُ كُل بِكْرٍ فِي أَرْضِ مِصْرَ قَدَّسْتُهُمْ لِي.
\par 18 فَاتَّخَذْتُ اللاوِيِّينَ بَدَل كُلِّ بِكْرٍ فِي بَنِي إِسْرَائِيل.
\par 19 وَوَهَبْتُ اللاوِيِّينَ هِبَةً لِهَارُونَ وَبَنِيهِ مِنْ بَيْنِ بَنِي إِسْرَائِيل لِيَخْدِمُوا خِدْمَةَ بَنِي إِسْرَائِيل فِي خَيْمَةِ الاِجْتِمَاعِ وَلِلتَّكْفِيرِ عَنْ بَنِي إِسْرَائِيل لِكَيْ لا يَكُونَ فِي بَنِي إِسْرَائِيل وَبَأٌ عِنْدَ اقْتِرَابِ بَنِي إِسْرَائِيل إِلى القُدْسِ».
\par 20 فَفَعَل مُوسَى وَهَارُونُ وَكُلُّ جَمَاعَةِ بَنِي إِسْرَائِيل لِلاوِيِّينَ حَسَبَ كُلِّ مَا أَمَرَ الرَّبُّ مُوسَى عَنِ اللاوِيِّينَ. هَكَذَا فَعَل لهُمْ بَنُو إِسْرَائِيل.
\par 21 فَتَطَهَّرَ اللاوِيُّونَ وَغَسَّلُوا ثِيَابَهُمْ وَرَدَّدَهُمْ هَارُونُ تَرْدِيداً أَمَامَ الرَّبِّ وَكَفَّرَ عَنْهُمْ هَارُونُ لِتَطْهِيرِهِمْ.
\par 22 وَبَعْدَ ذَلِكَ أَتَى اللاوِيُّونَ لِيَخْدِمُوا خِدْمَتَهُمْ فِي خَيْمَةِ الاِجْتِمَاعِ أَمَامَ هَارُونَ وَأَمَامَ بَنِيهِ - كَمَا أَمَرَ الرَّبُّ مُوسَى عَنِ اللاوِيِّينَ هَكَذَا فَعَلُوا لهُمْ.
\par 23 وَقَال الرَّبُّ لِمُوسَى:
\par 24 «هَذَا مَا لِلاوِيِّينَ: مِنِ ابْنِ خَمْسٍ وَعِشْرِينَ سَنَةً فَصَاعِداً يَأْتُونَ لِيَتَجَنَّدُوا أَجْنَاداً فِي خِدْمَةِ خَيْمَةِ الاِجْتِمَاعِ.
\par 25 وَمِنِ ابْنِ خَمْسِينَ سَنَةً يَرْجِعُونَ مِنْ جُنْدِ الخِدْمَةِ وَلا يَخْدِمُونَ بَعْدُ.
\par 26 يُوازِرُونَ إِخْوَتَهُمْ فِي خَيْمَةِ الاِجْتِمَاعِ لِحَرَسِ حِرَاسَةٍ. لكِنْ خِدْمَةً لا يَخْدِمُونَ. هَكَذَا تَعْمَلُ لِلاوِيِّينَ فِي حِرَاسَاتِهِمْ».

\chapter{9}

\par 1 وَقَال الرَّبُّ لِمُوسَى فِي بَرِّيَّةِ سِينَاءَ فِي السَّنَةِ الثَّانِيَةِ لِخُرُوجِهِمْ مِنْ أَرْضِ مِصْرَ فِي الشَّهْرِ الأَوَّلِ:
\par 2 «وَليَعْمَل بَنُو إِسْرَائِيل الفِصْحَ فِي وَقْتِهِ.
\par 3 فِي اليَوْمِ الرَّابِعَ عَشَرَ مِنْ هَذَا الشَّهْرِ بَيْنَ العِشَاءَيْنِ تَعْمَلُونَهُ فِي وَقْتِهِ. حَسَبَ كُلِّ فَرَائِضِهِ وَكُلِّ أَحْكَامِهِ تَعْمَلُونَهُ».
\par 4 فَكَلمَ مُوسَى بَنِي إِسْرَائِيل أَنْ يَعْمَلُوا الفِصْحَ.
\par 5 فَعَمِلُوا الفِصْحَ فِي الشَّهْرِ الأَوَّلِ فِي اليَوْمِ الرَّابِعَ عَشَرَ مِنَ الشَّهْرِ بَيْنَ العِشَاءَيْنِ فِي بَرِّيَّةِ سِينَاءَ - حَسَبَ كُلِّ مَا أَمَرَ الرَّبُّ مُوسَى هَكَذَا فَعَل بَنُو إِسْرَائِيل.
\par 6 لكِنْ كَانَ قَوْمٌ قَدْ تَنَجَّسُوا لِإِنْسَانٍ مَيِّتٍ فَلمْ يَحِل لهُمْ أَنْ يَعْمَلُوا الفِصْحَ فِي ذَلِكَ اليَوْمِ. فَتَقَدَّمُوا أَمَامَ مُوسَى وَهَارُونَ فِي ذَلِكَ اليَوْمِ
\par 7 وَقَالُوا لهُ: «إِنَّنَا مُتَنَجِّسُونَ لِإِنْسَانٍ مَيِّتٍ. لِمَاذَا نُتْرَكُ حَتَّى لا نُقَرِّبَ قُرْبَانَ الرَّبِّ فِي وَقْتِهِ بَيْنَ بَنِي إِسْرَائِيل؟»
\par 8 فَقَال لهُمْ مُوسَى: «قِفُوا لأَسْمَعَ مَا يَأْمُرُ بِهِ الرَّبُّ مِنْ جِهَتِكُمْ».
\par 9 فَأَمَرَ الرَّبُّ مُوسَى:
\par 10 «قُل لِبَنِي إِسْرَائِيل: كُلُّ إِنْسَانٍ مِنْكُمْ أَوْ مِنْ أَجْيَالِكُمْ كَانَ نَجِساً لِمَيِّتٍ أَوْ فِي سَفَرٍ بَعِيدٍ فَليَعْمَلِ الفِصْحَ لِلرَّبِّ.
\par 11 فِي الشَّهْرِ الثَّانِي فِي اليَوْمِ الرَّابِعَ عَشَرَ بَيْنَ العِشَاءَيْنِ يَعْمَلُونَهُ. عَلى فَطِيرٍ وَمُرَارٍ يَأْكُلُونَهُ.
\par 12 لا يُبْقُوا مِنْهُ إِلى الصَّبَاحِ وَلا يَكْسِرُوا عَظْماً مِنْهُ. حَسَبَ كُلِّ فَرَائِضِ الفِصْحِ يَعْمَلُونَهُ.
\par 13 لكِنْ مَنْ كَانَ طَاهِراً وَليْسَ فِي سَفَرٍ وَتَرَكَ عَمَل الفِصْحِ تُقْطَعُ تِلكَ النَّفْسُ مِنْ شَعْبِهَا لأَنَّهَا لمْ تُقَرِّبْ قُرْبَانَ الرَّبِّ فِي وَقْتِهِ. ذَلِكَ الإِنْسَانُ يَحْمِلُ خَطِيَّتَهُ.
\par 14 وَإِذَا نَزَل عِنْدَكُمْ غَرِيبٌ فَليَعْمَل فِصْحاً لِلرَّبِّ. حَسَبَ فَرِيضَةِ الفِصْحِ وَحُكْمِهِ كَذَلِكَ يَعْمَلُ. فَرِيضَةٌ وَاحِدَةٌ تَكُونُ لكُمْ لِلغَرِيبِ وَلِوَطَنِيِّ الأَرْضِ».
\par 15 وَفِي يَوْمِ إِقَامَةِ المَسْكَنِ غَطَّتِ السَّحَابَةُ المَسْكَنَ خَيْمَةَ الشَّهَادَةِ. وَفِي المَسَاءِ كَانَ عَلى المَسْكَنِ كَمَنْظَرِ نَارٍ إِلى الصَّبَاحِ.
\par 16 هَكَذَا كَانَ دَائِماً. السَّحَابَةُ تُغَطِّيهِ وَمَنْظَرُ النَّارِ ليْلاً.
\par 17 وَمَتَى ارْتَفَعَتِ السَّحَابَةُ عَنِ الخَيْمَةِ كَانَ بَعْدَ ذَلِكَ بَنُو إِسْرَائِيل يَرْتَحِلُونَ. وَفِي المَكَانِ حَيْثُ حَلتِ السَّحَابَةُ هُنَاكَ كَانَ بَنُو إِسْرَائِيل يَنْزِلُونَ.
\par 18 حَسَبَ قَوْلِ الرَّبِّ كَانَ بَنُو إِسْرَائِيل يَرْتَحِلُونَ وَحَسَبَ قَوْلِ الرَّبِّ كَانُوا يَنْزِلُونَ. جَمِيعَ أَيَّامِ حُلُولِ السَّحَابَةِ عَلى المَسْكَنِ كَانُوا يَنْزِلُونَ.
\par 19 وَإِذَا تَمَادَتِ السَّحَابَةُ عَلى المَسْكَنِ أَيَّاماً كَثِيرَةً كَانَ بَنُو إِسْرَائِيل يَحْرُسُونَ حِرَاسَةَ الرَّبِّ وَلا يَرْتَحِلُونَ.
\par 20 وَإِذَا كَانَتِ السَّحَابَةُ أَيَّاماً قَلِيلةً عَلى المَسْكَنِ فَحَسَبَ قَوْلِ الرَّبِّ كَانُوا يَنْزِلُونَ وَحَسَبَ قَوْلِ الرَّبِّ كَانُوا يَرْتَحِلُونَ.
\par 21 وَإِذَا كَانَتِ السَّحَابَةُ مِنَ المَسَاءِ إِلى الصَّبَاحِ ثُمَّ ارْتَفَعَتِ السَّحَابَةُ فِي الصَّبَاحِ كَانُوا يَرْتَحِلُونَ. أَوْ يَوْماً وَليْلةً ثُمَّ ارْتَفَعَتِ السَّحَابَةُ كَانُوا يَرْتَحِلُونَ.
\par 22 أَوْ يَوْمَيْنِ أَوْ شَهْراً أَوْ سَنَةً مَتَى تَمَادَتِ السَّحَابَةُ عَلى المَسْكَنِ حَالةً عَليْهِ كَانَ بَنُو إِسْرَائِيل يَنْزِلُونَ وَلا يَرْتَحِلُونَ. وَمَتَى ارْتَفَعَتْ كَانُوا يَرْتَحِلُونَ.
\par 23 حَسَبَ قَوْلِ الرَّبِّ كَانُوا يَنْزِلُونَ وَحَسَبَ قَوْلِ الرَّبِّ كَانُوا يَرْتَحِلُونَ. وَكَانُوا يَحْرُسُونَ حِرَاسَةَ الرَّبِّ حَسَبَ قَوْلِ الرَّبِّ بِيَدِ مُوسَى.

\chapter{10}

\par 1 وَأَمَرَ الرَّبُّ مُوسَى:
\par 2 «اصْنَعْ لكَ بُوقَيْنِ مِنْ فِضَّةٍ. مَسْحُوليْنِ تَعْمَلُهُمَا فَيَكُونَانِ لكَ لِمُنَادَاةِ الجَمَاعَةِ وَلاِرْتِحَالِ المَحَلاتِ.
\par 3 فَإِذَا ضَرَبُوا بِهِمَا يَجْتَمِعُ إِليْكَ كُلُّ الجَمَاعَةِ إِلى بَابِ خَيْمَةِ الاِجْتِمَاعِ.
\par 4 وَإِذَا ضَرَبُوا بِوَاحِدٍ يَجْتَمِعُ إِليْكَ الرُّؤَسَاءُ رُؤُوسُ أُلُوفِ إِسْرَائِيل.
\par 5 وَإِذَا ضَرَبْتُمْ هُتَافاً تَرْتَحِلُ المَحَلاتُ النَّازِلةُ إِلى الشَّرْقِ.
\par 6 وَإِذَا ضَرَبْتُمْ هُتَافاً ثَانِيَةً تَرْتَحِلُ المَحَلاتُ النَّازِلةُ إِلى الجَنُوبِ. هُتَافاً يَضْرِبُونَ لِرِحْلاتِهِمْ.
\par 7 وَأَمَّا عِنْدَمَا تَجْمَعُونَ الجَمَاعَةَ فَتَضْرِبُونَ وَلا تَهْتِفُونَ.
\par 8 وَبَنُو هَارُونَ الكَهَنَةُ يَضْرِبُونَ بِالأَبْوَاقِ. فَتَكُونُ لكُمْ فَرِيضَةً أَبَدِيَّةً فِي أَجْيَالِكُمْ.
\par 9 وَإِذَا ذَهَبْتُمْ إِلى حَرْبٍ فِي أَرْضِكُمْ عَلى عَدُوٍّ يَضُرُّ بِكُمْ تَهْتِفُونَ بِالأَبْوَاقِ فَتُذْكَرُونَ أَمَامَ الرَّبِّ إِلهِكُمْ وَتُخَلصُونَ مِنْ أَعْدَائِكُمْ.
\par 10 وَفِي يَوْمِ فَرَحِكُمْ وَفِي أَعْيَادِكُمْ وَرُؤُوسِ شُهُورِكُمْ تَضْرِبُونَ بِالأَبْوَاقِ عَلى مُحْرَقَاتِكُمْ وَذَبَائِحِ سَلامَتِكُمْ فَتَكُونُ لكُمْ تِذْكَاراً أَمَامَ إِلهِكُمْ. أَنَا الرَّبُّ إِلهُكُمْ».
\par 11 وَفِي السَّنَةِ الثَّانِيَةِ فِي الشَّهْرِ الثَّانِي فِي العِشْرِينَ مِنَ الشَّهْرِ ارْتَفَعَتِ السَّحَابَةُ عَنْ مَسْكَنِ الشَّهَادَةِ.
\par 12 فَارْتَحَل بَنُو إِسْرَائِيل فِي رِحْلاتِهِمْ مِنْ بَرِّيَّةِ سِينَاءَ فَحَلتِ السَّحَابَةُ فِي بَرِّيَّةِ فَارَانَ.
\par 13 ارْتَحَلُوا أَوَّلاً حَسَبَ قَوْلِ الرَّبِّ عَنْ يَدِ مُوسَى.
\par 14 فَارْتَحَلتْ رَايَةُ مَحَلةِ بَنِي يَهُوذَا أَوَّلاً حَسَبَ أَجْنَادِهِمْ وَعَلى جُنْدِهِ نَحْشُونُ بْنُ عَمِّينَادَابَ.
\par 15 وَعَلى جُنْدِ سِبْطِ بَنِي يَسَّاكَرَ نَثَنَائِيلُ بْنُ صُوغَرَ.
\par 16 وَعَلى جُنْدِ سِبْطِ بَنِي زَبُولُونَ أَلِيآبُ بْنُ حِيلُونَ.
\par 17 ثُمَّ أُنْزِل المَسْكَنُ فَارْتَحَل بَنُو جَرْشُونَ وَبَنُو مَرَارِي حَامِلِينَ المَسْكَنَ.
\par 18 ثُمَّ ارْتَحَلتْ رَايَةُ مَحَلةِ رَأُوبَيْنَ حَسَبَ أَجْنَادِهِمْ وَعَلى جُنْدِهِ أَلِيصُورُ بْنُ شَدَيْئُورَ.
\par 19 وَعَلى جُنْدِ سِبْطِ بَنِي شَمْعُونَ شَلُومِيئِيلُ بْنُ صُورِيشَدَّاي.
\par 20 وَعَلى جُنْدِ سِبْطِ بَنِي جَادَ أَلِيَاسَافُ بْنُ دَعُوئِيل.
\par 21 ثُمَّ ارْتَحَل القَهَاتِيُّونَ حَامِلِينَ المَقْدِسَ. (وَأُقِيمَ المَسْكَنُ إِلى أَنْ جَاءُوا)
\par 22 ثُمَّ ارْتَحَلتْ رَايَةُ مَحَلةِ بَنِي أَفْرَايِمَ حَسَبَ أَجْنَادِهِمْ وَعَلى جُنْدِهِ أَلِيشَمَعُ بْنُ عَمِّيهُودَ.
\par 23 وَعَلى جُنْدِ سِبْطِ بَنِي مَنَسَّى جَمْلِيئِيلُ بْنُ فَدَهْصُورَ.
\par 24 وَعَلى جُنْدِ سِبْطِ بَنِي بِنْيَامِينَ أَبِيدَنُ بْنُ جِدْعُونِي.
\par 25 ثُمَّ ارْتَحَلتْ رَايَةُ مَحَلةِ بَنِي دَانَ سَاقَةِ جَمِيعِ المَحَلاتِ حَسَبَ أَجْنَادِهِمْ وَعَلى جُنْدِهِ أَخِيعَزَرُ بْنُ عَمِّيشَدَّاي.
\par 26 وَعَلى جُنْدِ سِبْطِ بَنِي أَشِيرَ فَجْعِيئِيلُ بْنُ عُكْرَنَ.
\par 27 وَعَلى جُنْدِ سِبْطِ بَنِي نَفْتَالِي أَخِيرَعُ بْنُ عِينَنَ.
\par 28 هَذِهِ رِحْلاتُ بَنِي إِسْرَائِيل بِأَجْنَادِهِمْ حِينَ ارْتَحَلُوا.
\par 29 وَقَال مُوسَى لِحُوبَابَ بْنِ رَعُوئِيل المِدْيَانِيِّ حَمِي مُوسَى: «إِنَّنَا رَاحِلُونَ إِلى المَكَانِ الذِي قَال الرَّبُّ أُعْطِيكُمْ إِيَّاهُ. اذْهَبْ مَعَنَا فَنُحْسِنَ إِليْكَ لأَنَّ الرَّبَّ قَدْ تَكَلمَ عَنْ إِسْرَائِيل بِالإِحْسَانِ».
\par 30 فَقَال لهُ: «لا أَذْهَبُ بَل إِلى أَرْضِي وَإِلى عَشِيرَتِي أَمْضِي».
\par 31 فَقَال: «لا تَتْرُكْنَا لأَنَّهُ بِمَا أَنَّكَ تَعْرِفُ مَنَازِلنَا فِي البَرِّيَّةِ تَكُونُ لنَا كَعُيُونٍ.
\par 32 وَإِنْ ذَهَبْتَ مَعَنَا فَبِنَفْسِ الإِحْسَانِ الذِي يُحْسِنُ الرَّبُّ إِليْنَا نُحْسِنُ نَحْنُ إِليْكَ».
\par 33 فَارْتَحَلُوا مِنْ جَبَلِ الرَّبِّ مَسِيرَةَ ثَلاثَةِ أَيَّامٍ وَتَابُوتُ عَهْدِ الرَّبِّ رَاحِلٌ أَمَامَهُمْ مَسِيرَةَ ثَلاثَةِ أَيَّامٍ لِيَلتَمِسَ لهُمْ مَنْزِلاً.
\par 34 وَكَانَتْ سَحَابَةُ الرَّبِّ عَليْهِمْ نَهَاراً فِي ارْتِحَالِهِمْ مِنَ المَحَلةِ.
\par 35 وَعِنْدَ ارْتِحَالِ التَّابُوتِ كَانَ مُوسَى يَقُولُ: «قُمْ يَا رَبُّ فَلتَتَبَدَّدْ أَعْدَاؤُكَ وَيَهْرُبْ مُبْغِضُوكَ مِنْ أَمَامِكَ».
\par 36 وَعِنْدَ حُلُولِهِ كَانَ يَقُولُ: «ارْجِعْ يَا رَبُّ إِلى رَبَوَاتِ أُلُوفِ إِسْرَائِيل».

\chapter{11}

\par 1 وَكَانَ الشَّعْبُ كَأَنَّهُمْ يَشْتَكُونَ شَرّاً فِي أُذُنَيِ الرَّبِّ. وَسَمِعَ الرَّبُّ فَحَمِيَ غَضَبُهُ فَاشْتَعَلتْ فِيهِمْ نَارُ الرَّبِّ وَأَحْرَقَتْ فِي طَرَفِ المَحَلةِ.
\par 2 فَصَرَخَ الشَّعْبُ إِلى مُوسَى فَصَلى مُوسَى إِلى الرَّبِّ فَخَمَدَتِ النَّارُ.
\par 3 فَدُعِيَ اسْمُ ذَلِكَ المَوْضِعِ «تَبْعِيرَةَ» لأَنَّ نَارَ الرَّبِّ اشْتَعَلتْ فِيهِمْ.
\par 4 وَاللفِيفُ الذِي فِي وَسَطِهِمِ اشْتَهَى شَهْوَةً. فَعَادَ بَنُو إِسْرَائِيل أَيْضاً وَبَكُوا وَقَالُوا: «مَنْ يُطْعِمُنَا لحْماً؟
\par 5 قَدْ تَذَكَّرْنَا السَّمَكَ الذِي كُنَّا نَأْكُلُهُ فِي مِصْرَ مَجَّاناً وَالقِثَّاءَ وَالبَطِّيخَ وَالكُرَّاثَ وَالبَصَل وَالثُّومَ.
\par 6 وَالآنَ قَدْ يَبِسَتْ أَنْفُسُنَا. ليْسَ شَيْءٌ غَيْرَ أَنَّ أَعْيُنَنَا إِلى هَذَا المَنِّ!»
\par 7 وَأَمَّا المَنُّ فَكَانَ كَبِزْرِ الكُزْبَرَةِ وَمَنْظَرُهُ كَمَنْظَرِ المُقْلِ.
\par 8 كَانَ الشَّعْبُ يَطُوفُونَ لِيَلتَقِطُوهُ ثُمَّ يَطْحَنُونَهُ بِالرَّحَى أَوْ يَدُقُّونَهُ فِي الهَاوَنِ وَيَطْبُخُونَهُ فِي القُدُورِ وَيَعْمَلُونَهُ مَلاتٍ. وَكَانَ طَعْمُهُ كَطَعْمِ قَطَائِفَ بِزَيْتٍ.
\par 9 وَمَتَى نَزَل النَّدَى عَلى المَحَلةِ ليْلاً كَانَ يَنْزِلُ المَنُّ مَعَهُ.
\par 10 فَلمَّا سَمِعَ مُوسَى الشَّعْبَ يَبْكُونَ بِعَشَائِرِهِمْ كُل وَاحِدٍ فِي بَابِ خَيْمَتِهِ وَحَمِيَ غَضَبُ الرَّبِّ جِدّاً سَاءَ ذَلِكَ فِي عَيْنَيْ مُوسَى.
\par 11 فَقَال مُوسَى لِلرَّبِّ: «لِمَاذَا أَسَأْتَ إِلى عَبْدِكَ وَلِمَاذَا لمْ أَجِدْ نِعْمَةً فِي عَيْنَيْكَ حَتَّى أَنَّكَ وَضَعْتَ ثِقْل جَمِيعِ هَذَا الشَّعْبِ عَليَّ؟
\par 12 أَلعَلِّي حَبِلتُ بِجَمِيعِ هَذَا الشَّعْبِ أَوْ لعَلِّي وَلدْتُهُ حَتَّى تَقُول لِي احْمِلهُ فِي حِضْنِكَ كَمَا يَحْمِلُ المُرَبِّي الرَّضِيعَ إِلى الأَرْضِ التِي حَلفْتَ لِآبَائِهِ؟
\par 13 مِنْ أَيْنَ لِي لحْمٌ حَتَّى أُعْطِيَ جَمِيعَ هَذَا الشَّعْبِ. لأَنَّهُمْ يَبْكُونَ عَليَّ قَائِلِينَ: أَعْطِنَا لحْماً لِنَأْكُل.
\par 14 لا أَقْدِرُ أَنَا وَحْدِي أَنْ أَحْمِل جَمِيعَ هَذَا الشَّعْبِ لأَنَّهُ ثَقِيلٌ عَليَّ.
\par 15 فَإِنْ كُنْتَ تَفْعَلُ بِي هَكَذَا فَاقْتُلنِي قَتْلاً إِنْ وَجَدْتُ نِعْمَةً فِي عَيْنَيْكَ فَلا أَرَى بَلِيَّتِي».
\par 16 فَقَال الرَّبُّ لِمُوسَى: «اجْمَعْ إِليَّ سَبْعِينَ رَجُلاً مِنْ شُيُوخِ إِسْرَائِيل الذِينَ تَعْلمُ أَنَّهُمْ شُيُوخُ الشَّعْبِ وَعُرَفَاؤُهُ وَأَقْبِل بِهِمْ إِلى خَيْمَةِ الاِجْتِمَاعِ فَيَقِفُوا هُنَاكَ مَعَكَ.
\par 17 فَأَنْزِل أَنَا وَأَتَكَلمَ مَعَكَ هُنَاكَ وَآخُذَ مِنَ الرُّوحِ الذِي عَليْكَ وَأَضَعَ عَليْهِمْ فَيَحْمِلُونَ مَعَكَ ثِقْل الشَّعْبِ فَلا تَحْمِلُ أَنْتَ وَحْدَكَ.
\par 18 وَلِلشَّعْبِ تَقُولُ: تَقَدَّسُوا لِلغَدِ فَتَأْكُلُوا لحْماً. لأَنَّكُمْ قَدْ بَكَيْتُمْ فِي أُذُنَيِ الرَّبِّ قَائِلِينَ: مَنْ يُطْعِمُنَا لحْماً؟ إِنَّهُ كَانَ لنَا خَيْرٌ فِي مِصْرَ! فَيُعْطِيكُمُ الرَّبُّ لحْماً فَتَأْكُلُونَ.
\par 19 تَأْكُلُونَ لا يَوْماً وَاحِداً وَلا يَوْمَيْنِ وَلا خَمْسَةَ أَيَّامٍ وَلا عَشَرَةَ أَيَّامٍ وَلا عِشْرِينَ يَوْماً
\par 20 بَل شَهْراً مِنَ الزَّمَانِ حَتَّى يَخْرُجَ مِنْ مَنَاخِرِكُمْ وَيَصِيرَ لكُمْ كَرَاهَةً لأَنَّكُمْ رَفَضْتُمُ الرَّبَّ الذِي فِي وَسَطِكُمْ وَبَكَيْتُمْ أَمَامَهُ قَائِلِينَ: لِمَاذَا خَرَجْنَا مِنْ مِصْرَ؟»
\par 21 فَقَال مُوسَى: «سِتُّ مِئَةِ أَلفِ مَاشٍ هُوَ الشَّعْبُ الذِي أَنَا فِي وَسَطِهِ وَأَنْتَ قَدْ قُلتَ: أُعْطِيهِمْ لحْماً لِيَأْكُلُوا شَهْراً مِنَ الزَّمَانِ.
\par 22 أَيُذْبَحُ لهُمْ غَنَمٌ وَبَقَرٌ لِيَكْفِيَهُمْ أَمْ يُجْمَعُ لهُمْ كُلُّ سَمَكِ البَحْرِ لِيَكْفِيَهُمْ؟»
\par 23 فَقَال الرَّبُّ لِمُوسَى: «هَل تَقْصُرُ يَدُ الرَّبِّ؟ الآنَ تَرَى أَيُوافِيكَ كَلامِي أَمْ لا».
\par 24 فَخَرَجَ مُوسَى وَكَلمَ الشَّعْبَ بِكَلامِ الرَّبِّ وَجَمَعَ سَبْعِينَ رَجُلاً مِنْ شُيُوخِ الشَّعْبِ وَأَوْقَفَهُمْ حَوَاليِ الخَيْمَةِ.
\par 25 فَنَزَل الرَّبُّ فِي سَحَابَةٍ وَتَكَلمَ مَعَهُ وَأَخَذَ مِنَ الرُّوحِ الذِي عَليْهِ وَجَعَل عَلى السَّبْعِينَ رَجُلاً الشُّيُوخَ. فَلمَّا حَلتْ عَليْهِمِ الرُّوحُ تَنَبَّأُوا وَلكِنَّهُمْ لمْ يَزِيدُوا.
\par 26 وَبَقِيَ رَجُلانِ فِي المَحَلةِ اسْمُ الوَاحِدِ أَلدَادُ وَاسْمُ الآخَرِ مِيدَادُ فَحَل عَليْهِمَا الرُّوحُ. وَكَانَا مِنَ المَكْتُوبِينَ لكِنَّهُمَا لمْ يَخْرُجَا إِلى الخَيْمَةِ. فَتَنَبَّئَا فِي المَحَلةِ.
\par 27 فَرَكَضَ غُلامٌ وَأَخْبَرَ مُوسَى وَقَال: «أَلدَادُ وَمِيدَادُ يَتَنَبَّئَانِ فِي المَحَلةِ».
\par 28 فَقَال يَشُوعُ بْنُ نُونَ خَادِمُ مُوسَى (مِنْ حَدَاثَتِهِ): «يَا سَيِّدِي مُوسَى ارْدَعْهُمَا!»
\par 29 فَقَال لهُ مُوسَى: «هَل تَغَارُ أَنْتَ لِي؟ يَا ليْتَ كُل شَعْبِ الرَّبِّ كَانُوا أَنْبِيَاءَ إِذَا جَعَل الرَّبُّ رُوحَهُ عَليْهِمْ!».
\par 30 ثُمَّ انْحَازَ مُوسَى إِلى المَحَلةِ هُوَ وَشُيُوخُ إِسْرَائِيل.
\par 31 فَخَرَجَتْ رِيحٌ مِنْ قِبَلِ الرَّبِّ وَسَاقَتْ سَلوَى مِنَ البَحْرِ وَأَلقَتْهَا عَلى المَحَلةِ نَحْوَ مَسِيرَةِ يَوْمٍ مِنْ هُنَا وَمَسِيرَةِ يَوْمٍ مِنْ هُنَاكَ حَوَاليِ المَحَلةِ وَنَحْوَ ذِرَاعَيْنِ فَوْقَ وَجْهِ الأَرْضِ.
\par 32 فَقَامَ الشَّعْبُ كُل ذَلِكَ النَّهَارِ وَكُل الليْلِ وَكُل يَوْمِ الغَدِ وَجَمَعُوا السَّلوَى. (الذِي قَلل جَمَعَ عَشَرَةَ حَوَامِرَ). وَسَطَّحُوهَا لهُمْ مَسَاطِحَ حَوَاليِ المَحَلةِ.
\par 33 وَإِذْ كَانَ اللحْمُ بَعْدُ بَيْنَ أَسْنَانِهِمْ قَبْل أَنْ يَنْقَطِعَ حَمِيَ غَضَبُ الرَّبِّ عَلى الشَّعْبِ وَضَرَبَ الرَّبُّ الشَّعْبَ ضَرْبَةً عَظِيمَةً جِدّاً.
\par 34 فَدُعِيَ اسْمُ ذَلِكَ المَوْضِعِ «قَبَرُوتَ هَتَّأَوَةَ» لأَنَّهُمْ هُنَاكَ دَفَنُوا القَوْمَ الذِينَ اشْتَهُوا.
\par 35 وَمِنْ قَبَرُوتَ هَتَّأَوَةَ ارْتَحَل الشَّعْبُ إِلى حَضَيْرُوتَ فَكَانُوا فِي حَضَيْرُوتَ.

\chapter{12}

\par 1 وَتَكَلمَتْ مَرْيَمُ وَهَارُونُ عَلى مُوسَى بِسَبَبِ المَرْأَةِ الكُوشِيَّةِ التِي اتَّخَذَهَا (لأَنَّهُ كَانَ قَدِ اتَّخَذَ امْرَأَةً كُوشِيَّةً)
\par 2 فَقَالا: «هَل كَلمَ الرَّبُّ مُوسَى وَحْدَهُ؟ أَلمْ يُكَلِّمْنَا نَحْنُ أَيْضاً؟» فَسَمِعَ الرَّبُّ.
\par 3 وَأَمَّا الرَّجُلُ مُوسَى فَكَانَ حَلِيماً جِدّاً أَكْثَرَ مِنْ جَمِيعِ النَّاسِ الذِينَ عَلى وَجْهِ الأَرْضِ.
\par 4 فَقَال الرَّبُّ حَالاً لِمُوسَى وَهَارُونَ وَمَرْيَمَ: «اخْرُجُوا أَنْتُمُ الثَّلاثَةُ إِلى خَيْمَةِ الاِجْتِمَاعِ». فَخَرَجُوا هُمُ الثَّلاثَةُ.
\par 5 فَنَزَل الرَّبُّ فِي عَمُودِ سَحَابٍ وَوَقَفَ فِي بَابِ الخَيْمَةِ وَدَعَا هَارُونَ وَمَرْيَمَ فَخَرَجَا كِلاهُمَا.
\par 6 فَقَال: «اسْمَعَا كَلامِي. إِنْ كَانَ مِنْكُمْ نَبِيٌّ لِلرَّبِّ فَبِالرُّؤْيَا أَسْتَعْلِنُ لهُ. فِي الحُلمِ أُكَلِّمُهُ.
\par 7 وَأَمَّا عَبْدِي مُوسَى فَليْسَ هَكَذَا بَل هُوَ أَمِينٌ فِي كُلِّ بَيْتِي.
\par 8 فَماً إِلى فَمٍ وَعَيَاناً أَتَكَلمُ مَعَهُ لا بِالأَلغَازِ. وَشِبْهَ الرَّبِّ يُعَايِنُ. فَلِمَاذَا لا تَخْشَيَانِ أَنْ تَتَكَلمَا عَلى عَبْدِي مُوسَى؟».
\par 9 فَحَمِيَ غَضَبُ الرَّبِّ عَليْهِمَا وَمَضَى.
\par 10 فَلمَّا ارْتَفَعَتِ السَّحَابَةُ عَنِ الخَيْمَةِ إِذَا مَرْيَمُ بَرْصَاءُ كَالثَّلجِ. فَالتَفَتَ هَارُونُ إِلى مَرْيَمَ وَإِذَا هِيَ بَرْصَاءُ.
\par 11 فَقَال هَارُونُ لِمُوسَى: «أَسْأَلُكَ يَا سَيِّدِي لا تَجْعَل عَليْنَا الخَطِيَّةَ التِي حَمِقْنَا وَأَخْطَأْنَا بِهَا.
\par 12 فَلا تَكُنْ كَالمَيِّتِ الذِي يَكُونُ عِنْدَ خُرُوجِهِ مِنْ رَحِمِ أُمِّهِ قَدْ أُكِل نِصْفُ لحْمِهِ».
\par 13 فَصَرَخَ مُوسَى إِلى الرَّبِّ: «اللهُمَّ اشْفِهَا».
\par 14 فَقَال الرَّبُّ لِمُوسَى: «وَلوْ بَصَقَ أَبُوهَا بَصْقاً فِي وَجْهِهَا أَمَا كَانَتْ تَخْجَلُ سَبْعَةَ أَيَّامٍ؟ تُحْجَزُ سَبْعَةَ أَيَّامٍ خَارِجَ المَحَلةِ وَبَعْدَ ذَلِكَ تُرْجَعُ».
\par 15 فَحُجِزَتْ مَرْيَمُ خَارِجَ المَحَلةِ سَبْعَةَ أَيَّامٍ وَلمْ يَرْتَحِلِ الشَّعْبُ حَتَّى أُرْجِعَتْ مَرْيَمُ.
\par 16 وَبَعْدَ ذَلِكَ ارْتَحَل الشَّعْبُ مِنْ حَضَيْرُوتَ وَنَزَلُوا فِي بَرِّيَّةِ فَارَانَ.

\chapter{13}

\par 1 ثُمَّ قَال الرَّبُّ لِمُوسَى:
\par 2 «أَرْسِل رِجَالاً لِيَتَجَسَّسُوا أَرْضَ كَنْعَانَ التِي أَنَا مُعْطِيهَا لِبَنِي إِسْرَائِيل. رَجُلاً وَاحِداً لِكُلِّ سِبْطٍ مِنْ آبَائِهِ تُرْسِلُونَ. كُلُّ وَاحِدٍ رَئِيسٌ فِيهِمْ».
\par 3 فَأَرْسَلهُمْ مُوسَى مِنْ بَرِّيَّةِ فَارَانَ حَسَبَ قَوْلِ الرَّبِّ. كُلُّهُمْ رِجَالٌ هُمْ رُؤَسَاءُ بَنِي إِسْرَائِيل
\par 4 وَهَذِهِ أَسْمَاؤُهُمْ: مِنْ سِبْطِ رَأُوبَيْنَ شَمُّوعُ بْنُ زَكُّورَ.
\par 5 مِنْ سِبْطِ شَمْعُونَ شَافَاطُ ابْنُ حُورِي.
\par 6 مِنْ سِبْطِ يَهُوذَا كَالِبُ بْنُ يَفُنَّةَ.
\par 7 مِنْ سِبْطِ يَسَّاكَرَ يَجْآلُ بْنُ يُوسُفَ.
\par 8 مِنْ سِبْطِ أَفْرَايِمَ هُوشَعُ بْنُ نُونَ.
\par 9 مِنْ سِبْطِ بِنْيَامِينَ فَلطِي بْنُ رَافُو.
\par 10 مِنْ سِبْطِ زَبُولُونَ جَدِّيئِيلُ بْنُ سُودِي.
\par 11 مِنْ سِبْطِ يُوسُفَ: مِنْ سِبْطِ مَنَسَّى جِدِّي بْنُ سُوسِي.
\par 12 مِنْ سِبْطِ دَانَ عَمِّيئِيلُ بْنُ جَمَلِّي.
\par 13 مِنْ سِبْطِ أَشِيرَ سَتُورُ بْنُ مِيخَائِيل.
\par 14 مِنْ سِبْطِ نَفْتَالِي نَحْبِي بْنُ وَفْسِي.
\par 15 مِنْ سِبْطِ جَادَ جَأُوئِيلُ بْنُ مَاكِي.
\par 16 هَذِهِ أَسْمَاءُ الرِّجَالِ الذِينَ أَرْسَلهُمْ مُوسَى لِيَتَجَسَّسُوا الأَرْضَ. وَدَعَا مُوسَى هُوشَعَ بْنَ نُونَ «يَشُوعَ».
\par 17 فَأَرْسَلهُمْ مُوسَى لِيَتَجَسَّسُوا أَرْضَ كَنْعَانَ وَقَال لهُمُ: «اصْعَدُوا مِنْ هُنَا إِلى الجَنُوبِ وَاطْلعُوا إِلى الجَبَلِ
\par 18 وَانْظُرُوا الأَرْضَ مَا هِيَ؟ وَالشَّعْبَ السَّاكِنَ فِيهَا أَقَوِيٌّ هُوَ أَمْ ضَعِيفٌ؟ قَلِيلٌ أَمْ كَثِيرٌ؟
\par 19 وَكَيْفَ هِيَ الأَرْضُ التِي هُوَ سَاكِنٌ فِيهَا أَجَيِّدَةٌ أَمْ رَدِيئَةٌ؟ وَمَا هِيَ المُدُنُ التِي هُوَ سَاكِنٌ فِيهَا أَمُخَيَّمَاتٌ أَمْ حُصُونٌ؟
\par 20 وَكَيْفَ هِيَ الأَرْضُ أَسَمِينَةٌ أَمْ هَزِيلةٌ؟ أَفِيهَا شَجَرٌ أَمْ لا؟ وَتَشَدَّدُوا فَخُذُوا مِنْ ثَمَرِ الأَرْضِ». وَأَمَّا الأَيَّامُ فَكَانَتْ أَيَّامَ بَاكُورَاتِ العِنَبِ.
\par 21 فَصَعِدُوا وَتَجَسَّسُوا الأَرْضَ مِنْ بَرِّيَّةِ صِينَ إِلى رَحُوبَ فِي مَدْخَلِ حَمَاةَ.
\par 22 صَعِدُوا إِلى الجَنُوبِ وَأَتُوا إِلى حَبْرُونَ. وَكَانَ هُنَاكَ أَخِيمَانُ وَشِيشَايُ وَتَلمَايُ بَنُو عَنَاقٍ. (وَأَمَّا حَبْرُونُ فَبُنِيَتْ قَبْل صُوعَنِ مِصْرَ بِسَبْعِ سِنِينَ).
\par 23 وَأَتُوا إِلى وَادِي أَشْكُول وَقَطَفُوا مِنْ هُنَاكَ زَرَجُونَةً بِعُنْقُودٍ وَاحِدٍ مِنَ العِنَبِ وَحَمَلُوهُ بِالدُّقْرَانَةِ بَيْنَ اثْنَيْنِ مَعَ شَيْءٍ مِنَ الرُّمَّانِ وَالتِّينِ.
\par 24 فَدُعِيَ ذَلِكَ المَوْضِعُ «وَادِيَ أَشْكُول» بِسَبَبِ العُنْقُودِ الذِي قَطَعَهُ بَنُو إِسْرَائِيل مِنْ هُنَاكَ.
\par 25 ثُمَّ رَجَعُوا مِنْ تَجَسُّسِ الأَرْضِ بَعْدَ أَرْبَعِينَ يَوْماً.
\par 26 فَسَارُوا حَتَّى أَتُوا إِلى مُوسَى وَهَارُونَ وَكُلِّ جَمَاعَةِ بَنِي إِسْرَائِيل إِلى بَرِّيَّةِ فَارَانَ إِلى قَادِشَ وَرَدُّوا إِليْهِمَا خَبَراً وَإِلى كُلِّ الجَمَاعَةِ وَأَرُوهُمْ ثَمَرَ الأَرْضِ
\par 27 وَقَالُوا: «قَدْ ذَهَبْنَا إِلى الأَرْضِ التِي أَرْسَلتَنَا إِليْهَا وَحَقّاً إِنَّهَا تَفِيضُ لبَناً وَعَسَلاً وَهَذَا ثَمَرُهَا.
\par 28 غَيْرَ أَنَّ الشَّعْبَ السَّاكِنَ فِي الأَرْضِ مُعْتَزٌّ وَالمُدُنُ حَصِينَةٌ عَظِيمَةٌ جِدّاً. وَأَيْضاً قَدْ رَأَيْنَا بَنِي عَنَاقَ هُنَاكَ.
\par 29 العَمَالِقَةُ سَاكِنُونَ فِي أَرْضِ الجَنُوبِ وَالحِثِّيُّونَ وَاليَبُوسِيُّونَ وَالأَمُورِيُّونَ سَاكِنُونَ فِي الجَبَلِ وَالكَنْعَانِيُّونَ سَاكِنُونَ عِنْدَ البَحْرِ وَعَلى جَانِبِ الأُرْدُنِّ.
\par 30 لكِنْ كَالِبُ أَنْصَتَ الشَّعْبَ إِلى مُوسَى وَقَال: «إِنَّنَا نَصْعَدُ وَنَمْتَلِكُهَا لأَنَّنَا قَادِرُونَ عَليْهَا».
\par 31 وَأَمَّا الرِّجَالُ الذِينَ صَعِدُوا مَعَهُ فَقَالُوا: «لا نَقْدِرْ أَنْ نَصْعَدَ إِلى الشَّعْبِ لأَنَّهُمْ أَشَدُّ مِنَّا».
\par 32 فَأَشَاعُوا مَذَمَّةَ الأَرْضِ التِي تَجَسَّسُوهَا فِي بَنِي إِسْرَائِيل قَائِلِينَ: «الأَرْضُ التِي مَرَرْنَا فِيهَا لِنَتَجَسَّسَهَا هِيَ أَرْضٌ تَأْكُلُ سُكَّانَهَا. وَجَمِيعُ الشَّعْبِ الذِي رَأَيْنَا فِيهَا أُنَاسٌ طِوَالُ القَامَةِ.
\par 33 وَقَدْ رَأَيْنَا هُنَاكَ الجَبَابِرَةَ (بَنِي عَنَاقٍ مِنَ الجَبَابِرَةِ). فَكُنَّا فِي أَعْيُنِنَا كَالجَرَادِ وَهَكَذَا كُنَّا فِي أَعْيُنِهِمْ».

\chapter{14}

\par 1 فَرَفَعَتْ كُلُّ الجَمَاعَةِ صَوْتَهَا وَصَرَخَتْ. وَبَكَى الشَّعْبُ تِلكَ الليْلةَ.
\par 2 وَتَذَمَّرَ عَلى مُوسَى وَعَلى هَارُونَ جَمِيعُ بَنِي إِسْرَائِيل وَقَال لهُمَا كُلُّ الجَمَاعَةِ: «ليْتَنَا مُتْنَا فِي أَرْضِ مِصْرَ أَوْ ليْتَنَا مُتْنَا فِي هَذَا القَفْرِ!
\par 3 وَلِمَاذَا أَتَى بِنَا الرَّبُّ إِلى هَذِهِ الأَرْضِ لِنَسْقُطَ بِالسَّيْفِ؟ تَصِيرُ نِسَاؤُنَا وَأَطْفَالُنَا غَنِيمَةً. أَليْسَ خَيْراً لنَا أَنْ نَرْجِعَ إِلى مِصْرَ؟»
\par 4 فَقَال بَعْضُهُمْ لِبَعْضٍ: «نُقِيمُ رَئِيساً وَنَرْجِعُ إِلى مِصْرَ».
\par 5 فَسَقَطَ مُوسَى وَهَارُونُ عَلى وَجْهَيْهِمَا أَمَامَ كُلِّ مَعْشَرِ جَمَاعَةِ بَنِي إِسْرَائِيل.
\par 6 وَيَشُوعُ بْنُ نُونَ وَكَالِبُ بْنُ يَفُنَّةَ مِنَ الذِينَ تَجَسَّسُوا الأَرْضَ مَزَّقَا ثِيَابَهُمَا
\par 7 وَقَالا لِكُلِّ جَمَاعَةِ بَنِي إِسْرَائِيل: «الأَرْضُ التِي مَرَرْنَا فِيهَا لِنَتَجَسَّسَهَا جَيِّدَةٌ جِدّاً جِدّاً.
\par 8 إِنْ سُرَّ بِنَا الرَّبُّ يُدْخِلنَا إِلى هَذِهِ الأَرْضِ وَيُعْطِينَا إِيَّاهَا أَرْضاً تَفِيضُ لبَناً وَعَسَلاً.
\par 9 إِنَّمَا لا تَتَمَرَّدُوا عَلى الرَّبِّ وَلا تَخَافُوا مِنْ شَعْبِ الأَرْضِ لأَنَّهُمْ خُبْزُنَا. قَدْ زَال عَنْهُمْ ظِلُّهُمْ وَالرَّبُّ مَعَنَا. لا تَخَافُوهُمْ».
\par 10 وَلكِنْ قَال كُلُّ الجَمَاعَةِ أَنْ يُرْجَمَا بِالحِجَارَةِ. ثُمَّ ظَهَرَ مَجْدُ الرَّبِّ فِي خَيْمَةِ الاِجْتِمَاعِ لِكُلِّ بَنِي إِسْرَائِيل.
\par 11 وَقَال الرَّبُّ لِمُوسَى: «حَتَّى مَتَى يُهِينُنِي هَذَا الشَّعْبُ وَحَتَّى مَتَى لا يُصَدِّقُونَنِي بِجَمِيعِ الآيَاتِ التِي عَمِلتُ فِي وَسَطِهِمْ؟
\par 12 إِنِّي أَضْرِبُهُمْ بِالوَبَإِ وَأُبِيدُهُمْ وَأُصَيِّرُكَ شَعْباً أَكْبَرَ وَأَعْظَمَ مِنْهُمْ».
\par 13 فَقَال مُوسَى لِلرَّبِّ: «فَيَسْمَعُ المِصْرِيُّونَ الذِينَ أَصْعَدْتَ بِقُوَّتِكَ هَذَا الشَّعْبَ مِنْ وَسَطِهِمْ
\par 14 وَيَقُولُونَ لِسُكَّانِ هَذِهِ الأَرْضِ الذِينَ قَدْ سَمِعُوا أَنَّكَ يَا رَبُّ فِي وَسَطِ هَذَا الشَّعْبِ الذِينَ أَنْتَ يَا رَبُّ قَدْ ظَهَرْتَ لهُمْ عَيْناً لِعَيْنٍ وَسَحَابَتُكَ وَاقِفَةٌ عَليْهِمْ وَأَنْتَ سَائِرٌ أَمَامَهُمْ بِعَمُودِ سَحَابٍ نَهَاراً وَبِعَمُودِ نَارٍ ليْلاً.
\par 15 فَإِنْ قَتَلتَ هَذَا الشَّعْبَ كَرَجُلٍ وَاحِدٍ يَقُولُ الشُّعُوبُ الذِينَ سَمِعُوا بِخَبَرِكَ:
\par 16 لأَنَّ الرَّبَّ لمْ يَقْدِرْ أَنْ يُدْخِل هَذَا الشَّعْبَ إِلى الأَرْضِ التِي حَلفَ لهُمْ قَتَلهُمْ فِي القَفْرِ.
\par 17 فَالآنَ لِتَعْظُمْ قُدْرَةُ سَيِّدِي كَمَا قُلتَ:
\par 18 الرَّبُّ طَوِيلُ الرُّوحِ كَثِيرُ الإِحْسَانِ يَغْفِرُ الذَّنْبَ وَالسَّيِّئَةَ لكِنَّهُ لا يُبْرِئُ. بَل يَجْعَلُ ذَنْبَ الآبَاءِ عَلى الأَبْنَاءِ إِلى الجِيلِ الثَّالِثِ وَالرَّابِعِ.
\par 19 اِصْفَحْ عَنْ ذَنْبِ هَذَا الشَّعْبِ كَعَظَمَةِ نِعْمَتِكَ وَكَمَا غَفَرْتَ لِهَذَا الشَّعْبِ مِنْ مِصْرَ إِلى هَهُنَا».
\par 20 فَقَال الرَّبُّ: «قَدْ صَفَحْتُ حَسَبَ قَوْلِكَ.
\par 21 وَلكِنْ حَيٌّ أَنَا فَتُمْلأُ كُلُّ الأَرْضِ مِنْ مَجْدِ الرَّبِّ
\par 22 إِنَّ جَمِيعَ الرِّجَالِ الذِينَ رَأُوا مَجْدِي وَآيَاتِي التِي عَمِلتُهَا فِي مِصْرَ وَفِي البَرِّيَّةِ وَجَرَّبُونِي الآنَ عَشَرَ مَرَّاتٍ وَلمْ يَسْمَعُوا لِقَوْلِي
\par 23 لنْ يَرُوا الأَرْضَ التِي حَلفْتُ لِآبَائِهِمْ. وَجَمِيعُ الذِينَ أَهَانُونِي لا يَرُونَهَا.
\par 24 وَأَمَّا عَبْدِي كَالِبُ فَمِنْ أَجْلِ أَنَّهُ كَانَتْ مَعَهُ رُوحٌ أُخْرَى وَقَدِ اتَّبَعَنِي تَمَاماً أُدْخِلُهُ إِلى الأَرْضِ التِي ذَهَبَ إِليْهَا وَزَرْعُهُ يَرِثُهَا.
\par 25 وَإِذِ العَمَالِقَةُ وَالكَنْعَانِيُّونَ سَاكِنُونَ فِي الوَادِي فَانْصَرِفُوا غَداً وَارْتَحِلُوا إِلى القَفْرِ فِي طَرِيقِ بَحْرِ سُوفَ».
\par 26 وَقَال الرَّبُّ لِمُوسَى وَهَارُونَ:
\par 27 «حَتَّى مَتَى أَغْفِرُ لِهَذِهِ الجَمَاعَةِ الشِّرِّيرَةِ المُتَذَمِّرَةِ عَليَّ؟ قَدْ سَمِعْتُ تَذَمُّرَ بَنِي إِسْرَائِيل الذِي يَتَذَمَّرُونَهُ عَليَّ.
\par 28 قُل لهُمْ: حَيٌّ أَنَا يَقُولُ الرَّبُّ لأَفْعَلنَّ بِكُمْ كَمَا تَكَلمْتُمْ فِي أُذُنَيَّ.
\par 29 فِي هَذَا القَفْرِ تَسْقُطُ جُثَثُكُمْ جَمِيعُ المَعْدُودِينَ مِنْكُمْ حَسَبَ عَدَدِكُمْ مِنِ ابْنِ عِشْرِينَ سَنَةً فَصَاعِداً الذِينَ تَذَمَّرُوا عَليَّ.
\par 30 لنْ تَدْخُلُوا الأَرْضَ التِي رَفَعْتُ يَدِي لِأُسْكِنَنَّكُمْ فِيهَا مَا عَدَا كَالِبَ بْنَ يَفُنَّةَ وَيَشُوعَ بْنَ نُونٍ.
\par 31 وَأَمَّا أَطْفَالُكُمُ الذِينَ قُلتُمْ يَكُونُونَ غَنِيمَةً فَإِنِّي سَأُدْخِلُهُمْ فَيَعْرِفُونَ الأَرْضَ التِي احْتَقَرْتُمُوهَا.
\par 32 فَجُثَثُكُمْ أَنْتُمْ تَسْقُطُ فِي هَذَا القَفْرِ
\par 33 وَبَنُوكُمْ يَكُونُونَ رُعَاةً فِي القَفْرِ أَرْبَعِينَ سَنَةً وَيَحْمِلُونَ فُجُورَكُمْ حَتَّى تَفْنَى جُثَثُكُمْ فِي القَفْرِ.
\par 34 كَعَدَدِ الأَيَّامِ التِي تَجَسَّسْتُمْ فِيهَا الأَرْضَ أَرْبَعِينَ يَوْماً لِلسَّنَةِ يَوْمٌ. تَحْمِلُونَ ذُنُوبَكُمْ أَرْبَعِينَ سَنَةً فَتَعْرِفُونَ ابْتِعَادِي.
\par 35 أَنَا الرَّبُّ قَدْ تَكَلمْتُ. لأَفْعَلنَّ هَذَا بِكُلِّ هَذِهِ الجَمَاعَةِ الشِّرِّيرَةِ المُتَّفِقَةِ عَليَّ. فِي هَذَا القَفْرِ يَفْنُونَ وَفِيهِ يَمُوتُونَ».
\par 36 أَمَّا الرِّجَالُ الذِينَ أَرْسَلهُمْ مُوسَى لِيَتَجَسَّسُوا الأَرْضَ وَرَجَعُوا وَسَجَّسُوا عَليْهِ كُل الجَمَاعَةِ لِإِشَاعَةِ المَذَمَّةِ عَلى الأَرْضِ
\par 37 فَمَاتَ الرِّجَالُ الذِينَ أَشَاعُوا المَذَمَّةَ الرَّدِيئَةَ عَلى الأَرْضِ بِالوَبَإِ أَمَامَ الرَّبِّ.
\par 38 وَأَمَّا يَشُوعُ بْنُ نُونَ وَكَالِبُ بْنُ يَفُنَّةَ مِنْ أُولئِكَ الرِّجَالِ الذِينَ ذَهَبُوا لِيَتَجَسَّسُوا الأَرْضَ فَعَاشَا.
\par 39 وَلمَّا تَكَلمَ مُوسَى بِهَذَا الكَلامِ إِلى جَمِيعِ بَنِي إِسْرَائِيل بَكَى الشَّعْبُ جِدّاً.
\par 40 ثُمَّ بَكَّرُوا صَبَاحاً وَصَعِدُوا إِلى رَأْسِ الجَبَلِ قَائِلِينَ: «هُوَذَا نَحْنُ! نَصْعَدُ إِلى المَوْضِعِ الذِي قَال الرَّبُّ عَنْهُ فَإِنَّنَا قَدْ أَخْطَأْنَا».
\par 41 فَقَال مُوسَى: «لِمَاذَا تَتَجَاوَزُونَ قَوْل الرَّبِّ؟ فَهَذَا لا يَنْجَحُ.
\par 42 لا تَصْعَدُوا لأَنَّ الرَّبَّ ليْسَ فِي وَسَطِكُمْ لِئَلا تَنْهَزِمُوا أَمَامَ أَعْدَائِكُمْ.
\par 43 لأَنَّ العَمَالِقَةَ وَالكَنْعَانِيِّينَ هُنَاكَ قُدَّامَكُمْ تَسْقُطُونَ بِالسَّيْفِ. إِنَّكُمْ قَدِ ارْتَدَدْتُمْ عَنِ الرَّبِّ فَالرَّبُّ لا يَكُونُ مَعَكُمْ».
\par 44 لكِنَّهُمْ تَجَبَّرُوا وَصَعِدُوا إِلى رَأْسِ الجَبَلِ. وَأَمَّا تَابُوتُ عَهْدِ الرَّبِّ وَمُوسَى فَلمْ يَبْرَحَا مِنْ وَسَطِ المَحَلةِ.
\par 45 فَنَزَل العَمَالِقَةُ وَالكَنْعَانِيُّونَ السَّاكِنُونَ فِي ذَلِكَ الجَبَلِ وَضَرَبُوهُمْ وَكَسَّرُوهُمْ إِلى حُرْمَةَ.

\chapter{15}

\par 1 وَقَال الرَّبُّ لِمُوسَى:
\par 2 «قُل لِبَنِي إِسْرَائِيل: مَتَى جِئْتُمْ إِلى أَرْضِ مَسْكَنِكُمُ التِي أَنَا أُعْطِيكُمْ
\par 3 وَعَمِلتُمْ وَقُوداً لِلرَّبِّ مُحْرَقَةً أَوْ ذَبِيحَةً وَفَاءً لِنَذْرٍ أَوْ نَافِلةً أَوْ فِي أَعْيَادِكُمْ لِعَمَلِ رَائِحَةِ سُرُورٍ لِلرَّبِّ مِنَ البَقَرِ أَوْ مِنَ الغَنَمِ
\par 4 يُقَرِّبُ الذِي قَرَّبَ قُرْبَانَهُ لِلرَّبِّ تَقْدِمَةً مِنْ دَقِيقٍ عُشْراً مَلتُوتاً بِرُبْعِ الهِينِ مِنَ الزَّيْتِ
\par 5 وَخَمْراً لِلسَّكِيبِ رُبْعَ الهِينِ. تَعْمَلُ عَلى المُحْرَقَةِ أَوِ الذَّبِيحَةِ لِلخَرُوفِ الوَاحِدِ.
\par 6 لكِنْ لِلكَبْشِ تَعْمَلُ تَقْدِمَةً مِنْ دَقِيقٍ عُشْرَيْنِ مَلتُوتَيْنِ بِثُلثِ الهِينِ مِنَ الزَّيْتِ
\par 7 وَخَمْراً لِلسَّكِيبِ ثُلثَ الهِينِ تُقَرِّبُ لِرَائِحَةِ سَرُورٍ لِلرَّبِّ.
\par 8 وَإِذَا عَمِلتَ ابْنَ بَقَرٍ مُحْرَقَةً أَوْ ذَبِيحَةً وَفَاءً لِنَذْرٍ أَوْ ذَبِيحَةَ سَلامَةٍ لِلرَّبِّ
\par 9 تُقَرِّبُ عَلى ابْنِ البَقَرِ تَقْدِمَةً مِنْ دَقِيقٍ ثَلاثَةَ أَعْشَارٍ مَلتُوتَةً بِنِصْفِ الهِينِ مِنَ الزَّيْتِ
\par 10 وَخَمْراً تُقَرِّبُ لِلسَّكِيبِ نِصْفَ الهِينِ وَقُودَ رَائِحَةِ سَرُورٍ لِلرَّبِّ.
\par 11 هَكَذَا يُعْمَلُ لِلثَّوْرِ الوَاحِدِ أَوْ لِلكَبْشِ الوَاحِدِ أَوْ لِلشَّاةِ مِنَ الضَّأْنِ أَوْ مِنَ المَعْزِ.
\par 12 كَالعَدَدِ الذِي تَعْمَلُونَ هَكَذَا تَعْمَلُونَ لِكُلِّ وَاحِدٍ حَسَبَ عَدَدِهِنَّ.
\par 13 كُلُّ وَطَنِيٍّ يَعْمَلُ هَذِهِ هَكَذَا لِتَقْرِيبِ وَقُودِ رَائِحَةِ سُرُورٍ لِلرَّبِّ.
\par 14 وَإِذَا نَزَل عِنْدَكُمْ غَرِيبٌ أَوْ كَانَ أَحَدٌ فِي وَسَطِكُمْ فِي أَجْيَالِكُمْ وَعَمِل وَقُودَ رَائِحَةِ سُرُورٍ لِلرَّبِّ فَكَمَا تَفْعَلُونَ كَذَلِكَ يَفْعَلُ.
\par 15 أَيَّتُهَا الجَمَاعَةُ لكُمْ وَلِلغَرِيبِ النَّازِلِ عِنْدَكُمْ فَرِيضَةٌ وَاحِدَةٌ دَهْرِيَّةٌ فِي أَجْيَالِكُمْ. مَثَلُكُمْ يَكُونُ مَثَل الغَرِيبِ أَمَامَ الرَّبِّ.
\par 16 شَرِيعَةٌ وَاحِدَةٌ وَحُكْمٌ وَاحِدٌ يَكُونُ لكُمْ وَلِلغَرِيبِ النَّازِلِ عِنْدَكُمْ.
\par 17 وَقَال الرَّبُّ لِمُوسَى:
\par 18 «قُل لِبَنِي إِسْرَائِيل: مَتَى دَخَلتُمُ الأَرْضَ التِي أَنَا آتٍ بِكُمْ إِليْهَا
\par 19 فَعِنْدَمَا تَأْكُلُونَ مِنْ خُبْزِ الأَرْضِ تَرْفَعُونَ رَفِيعَةً لِلرَّبِّ.
\par 20 أَوَّل عَجِينِكُمْ تَرْفَعُونَ قُرْصاً رَفِيعَةً. كَرَفِيعَةِ البَيْدَرِ هَكَذَا تَرْفَعُونَهُ.
\par 21 مِنْ أَوَّلِ عَجِينِكُمْ تُعْطُونَ لِلرَّبِّ رَفِيعَةً فِي أَجْيَالِكُمْ.
\par 22 «وَإِذَا سَهَوْتُمْ وَلمْ تَعْمَلُوا جَمِيعَ هَذِهِ الوَصَايَا التِي كَلمَ بِهَا الرَّبُّ مُوسَى
\par 23 جَمِيعَ مَا أَمَرَكُمْ بِهِ الرَّبُّ عَنْ يَدِ مُوسَى مِنَ اليَوْمِ الذِي أَمَرَ فِيهِ الرَّبُّ فَصَاعِداً فِي أَجْيَالِكُمْ
\par 24 فَإِنْ عُمِل خُفْيَةً عَنْ أَعْيُنِ الجَمَاعَةِ سَهْواً يَعْمَلُ كُلُّ الجَمَاعَةِ ثَوْراً وَاحِداً ابْنَ بَقَرٍ مُحْرَقَةً لِرَائِحَةِ سُرُورٍ لِلرَّبِّ مَعَ تَقْدِمَتِهِ وَسَكِيبِهِ كَالعَادَةِ وَتَيْساً وَاحِداً مِنَ المَعْزِ ذَبِيحَةَ خَطِيَّةٍ.
\par 25 فَيُكَفِّرُ الكَاهِنُ عَنْ كُلِّ جَمَاعَةِ بَنِي إِسْرَائِيل فَيُصْفَحُ عَنْهُمْ لأَنَّهُ كَانَ سَهْواً. فَإِذَا أَتُوا بِقُرْبَانِهِمْ وَقُوداً لِلرَّبِّ وَبِذَبِيحَةِ خَطِيَّتِهِمْ أَمَامَ الرَّبِّ لأَجْلِ سَهْوِهِمْ
\par 26 يُصْفَحُ عَنْ كُلِّ جَمَاعَةِ بَنِي إِسْرَائِيل وَالغَرِيبِ النَّازِلِ بَيْنَهُمْ لأَنَّهُ حَدَثَ لِجَمِيعِ الشَّعْبِ بِسَهْوٍ.
\par 27 «وَإِنْ أَخْطَأَتْ نَفْسٌ وَاحِدَةٌ سَهْواً تُقَرِّبْ عَنْزاً حَوْلِيَّةً ذَبِيحَةَ خَطِيَّةٍ
\par 28 فَيُكَفِّرُ الكَاهِنُ عَنِ النَّفْسِ التِي سَهَتْ عِنْدَمَا أَخْطَأَتْ بِسَهْوٍ أَمَامَ الرَّبِّ لِلتَّكْفِيرِ عَنْهَا فَيُصْفَحُ عَنْهَا.
\par 29 لِلوَطَنِيِّ فِي بَنِي إِسْرَائِيل وَلِلغَرِيبِ النَّازِلِ بَيْنَهُمْ تَكُونُ شَرِيعَةٌ وَاحِدَةٌ لِلعَامِلِ بِسَهْوٍ.
\par 30 وَأَمَّا النَّفْسُ التِي تَعْمَلُ بِيَدٍ رَفِيعَةٍ مِنَ الوَطَنِيِّينَ أَوْ مِنَ الغُرَبَاءِ فَهِيَ تَزْدَرِي بِالرَّبِّ. فَتُقْطَعُ تِلكَ النَّفْسُ مِنْ بَيْنِ شَعْبِهَا
\par 31 لأَنَّهَا احْتَقَرَتْ كَلامَ الرَّبِّ وَنَقَضَتْ وَصِيَّتَهُ. قَطْعاً تُقْطَعُ تِلكَ النَّفْسُ. ذَنْبُهَا عَليْهَا».
\par 32 وَلمَّا كَانَ بَنُو إِسْرَائِيل فِي البَرِّيَّةِ وَجَدُوا رَجُلاً يَحْتَطِبُ حَطَباً فِي يَوْمِ السَّبْتِ.
\par 33 فَقَدَّمَهُ الذِينَ وَجَدُوهُ يَحْتَطِبُ حَطَباً إِلى مُوسَى وَهَارُونَ وَكُلِّ الجَمَاعَةِ.
\par 34 فَوَضَعُوهُ فِي المَحْرَسِ لأَنَّهُ لمْ يُعْلنْ مَاذَا يُفْعَلُ بِهِ.
\par 35 فَقَال الرَّبُّ لِمُوسَى: «قَتْلاً يُقْتَلُ الرَّجُلُ. يَرْجُمُهُ بِحِجَارَةٍ كُلُّ الجَمَاعَةِ خَارِجَ المَحَلةِ».
\par 36 فَأَخْرَجَهُ كُلُّ الجَمَاعَةِ إِلى خَارِجِ المَحَلةِ وَرَجَمُوهُ بِحِجَارَةٍ فَمَاتَ كَمَا أَمَرَ الرَّبُّ مُوسَى.
\par 37 وَقَال الرَّبُّ لِمُوسَى:
\par 38 «قُل لِبَنِي إِسْرَائِيل أَنْ يَصْنَعُوا لهُمْ أَهْدَاباً فِي أَذْيَالِ ثِيَابِهِمْ فِي أَجْيَالِهِمْ وَيَجْعَلُوا عَلى هُدْبِ الذَّيْلِ عِصَابَةً مِنْ أَسْمَانْجُونِيٍّ.
\par 39 فَتَكُونُ لكُمْ هُدْباً فَتَرُونَهَا وَتَذْكُرُونَ كُل وَصَايَا الرَّبِّ وَتَعْمَلُونَهَا وَلا تَطُوفُونَ وَرَاءَ قُلُوبِكُمْ وَأَعْيُنِكُمُ التِي أَنْتُمْ فَاسِقُونَ وَرَاءَهَا
\par 40 لِكَيْ تَذْكُرُوا وَتَعْمَلُوا كُل وَصَايَايَ وَتَكُونُوا مُقَدَّسِينَ لِإِلهِكُمْ.
\par 41 أَنَا الرَّبُّ إِلهُكُمُ الذِي أَخْرَجَكُمْ مِنْ أَرْضِ مِصْرَ لِيَكُونَ لكُمْ إِلهاً. أَنَا الرَّبُّ إِلهُكُمْ».

\chapter{16}

\par 1 وَأَخَذَ قُورَحُ بْنُ يِصْهَارَ بْنِ قَهَاتَ بْنِ لاوِي وَدَاثَانُ وَأَبِيرَامُ ابْنَا أَلِيآبَ وَأُونُ بْنُ فَالتَ بَنُو رَأُوبَيْنَ
\par 2 يُقَاوِمُونَ مُوسَى مَعَ أُنَاسٍ مِنْ بَنِي إِسْرَائِيل مِئَتَيْنِ وَخَمْسِينَ رُؤَسَاءِ الجَمَاعَةِ مَدْعُوِّينَ لِلاِجْتِمَاعِ ذَوِي اسْمٍ.
\par 3 فَاجْتَمَعُوا عَلى مُوسَى وَهَارُونَ وَقَالُوا لهُمَا: «كَفَاكُمَا! إِنَّ كُل الجَمَاعَةِ بِأَسْرِهَا مُقَدَّسَةٌ وَفِي وَسَطِهَا الرَّبُّ. فَمَا بَالُكُمَا تَرْتَفِعَانِ عَلى جَمَاعَةِ الرَّبِّ؟».
\par 4 فَلمَّا سَمِعَ مُوسَى سَقَطَ عَلى وَجْهِهِ.
\par 5 ثُمَّ قَال لِقُورَحَ وَجَمِيعِ قَوْمِهِ: «غَداً يُعْلِنُ الرَّبُّ مَنْ هُوَ لهُ وَمَنِ المُقَدَّسُ حَتَّى يُقَرِّبَهُ إِليْهِ. فَالذِي يَخْتَارُهُ يُقَرِّبُهُ إِليْهِ.
\par 6 اِفْعَلُوا هَذَا: خُذُوا لكُمْ مَجَامِرَ. قُورَحُ وَكُلُّ جَمَاعَتِهِ.
\par 7 وَاجْعَلُوا فِيهَا نَاراً وَضَعُوا عَليْهَا بَخُوراً أَمَامَ الرَّبِّ غَداً. فَالرَّجُلُ الذِي يَخْتَارُهُ الرَّبُّ هُوَ المُقَدَّسُ. كَفَاكُمْ يَا بَنِي لاوِي!»
\par 8 وَقَال مُوسَى لِقُورَحَ: «اسْمَعُوا يَا بَنِي لاوِي.
\par 9 أَقَلِيلٌ عَليْكُمْ أَنَّ إِلهَ إِسْرَائِيل أَفْرَزَكُمْ مِنْ جَمَاعَةِ إِسْرَائِيل لِيُقَرِّبَكُمْ إِليْهِ لِكَيْ تَعْمَلُوا خِدْمَةَ مَسْكَنِ الرَّبِّ وَتَقِفُوا قُدَّامَ الجَمَاعَةِ لِخِدْمَتِهَا؟
\par 10 فَقَرَّبَكَ وَجَمِيعَ إِخْوَتِكَ بَنِي لاوِي مَعَكَ وَتَطْلُبُونَ أَيْضاً كَهَنُوتاً!
\par 11 إِذَنْ أَنْتَ وَكُلُّ جَمَاعَتِكَ مُتَّفِقُونَ عَلى الرَّبِّ. وَأَمَّا هَارُونُ فَمَا هُوَ حَتَّى تَتَذَمَّرُوا عَليْهِ؟»
\par 12 فَأَرْسَل مُوسَى لِيَدْعُوَ دَاثَانَ وَأَبِيرَامَ ابْنَيْ أَلِيآبَ. فَقَالا: «لا نَصْعَدُ!
\par 13 أَقَلِيلٌ أَنَّكَ أَصْعَدْتَنَا مِنْ أَرْضٍ تَفِيضُ لبَناً وَعَسَلاً لِتُمِيتَنَا فِي البَرِّيَّةِ حَتَّى تَتَرَأَّسَ عَليْنَا تَرَؤُّساً؟
\par 14 كَذَلِكَ لمْ تَأْتِ بِنَا إِلى أَرْضٍ تَفِيضُ لبَناً وَعَسَلاً وَلا أَعْطَيْتَنَا نَصِيبَ حُقُولٍ وَكُرُومٍ. هَل تَقْلعُ أَعْيُنَ هَؤُلاءِ القَوْمِ؟ لا نَصْعَدُ!».
\par 15 فَاغْتَاظَ مُوسَى جِدّاً وَقَال لِلرَّبِّ: «لا تَلتَفِتْ إِلى تَقْدِمَتِهِمَا. حِمَاراً وَاحِداً لمْ آخُذْ مِنْهُمْ وَلا أَسَأْتُ إِلى أَحَدٍ مِنْهُمْ».
\par 16 وَقَال مُوسَى لِقُورَحَ: «كُنْ أَنْتَ وَكُلُّ جَمَاعَتِكَ أَمَامَ الرَّبِّ أَنْتَ وَهُمْ وَهَارُونُ غَداً
\par 17 وَخُذُوا كُلُّ وَاحِدٍ مِجْمَرَتَهُ وَاجْعَلُوا فِيهَا بَخُوراً وَقَدِّمُوا أَمَامَ الرَّبِّ كُلُّ وَاحِدٍ مِجْمَرَتَهُ.مِئَتَيْنِ وَخَمْسِينَ مَجْمَرَةً. وَأَنْتَ وَهَارُونُ كُلُّ وَاحِدٍ مَجْمَرَتَهُ».
\par 18 فَأَخَذُوا كُلُّ وَاحِدٍ مِجْمَرَتَهُ وَجَعَلُوا فِيهَا نَاراً وَوَضَعُوا عَليْهَا بَخُوراً وَوَقَفُوا لدَى بَابِ خَيْمَةِ الاِجْتِمَاعِ مَعَ مُوسَى وَهَارُونَ.
\par 19 وَجَمَعَ عَليْهِمَا قُورَحُ كُل الجَمَاعَةِ إِلى بَابِ خَيْمَةِ الاِجْتِمَاعِ فَتَرَاءَى مَجْدُ الرَّبِّ لِكُلِّ الجَمَاعَةِ.
\par 20 وَقَال الرَّبُّ لِمُوسَى وَهَارُونَ:
\par 21 «افْتَرِزَا مِنْ بَيْنِ هَذِهِ الجَمَاعَةِ فَإِنِّي أُفْنِيهِمْ فِي لحْظَةٍ!»
\par 22 فَخَرَّا عَلى وَجْهَيْهِمَا وَقَالا: «اللهُمَّ إِلهَ أَرْوَاحِ جَمِيعِ البَشَرِ هَل يُخْطِئُ رَجُلٌ وَاحِدٌ فَتَسْخَطَ عَلى كُلِّ الجَمَاعَةِ؟»
\par 23 فَقَال الرَّبُّ لِمُوسَى:
\par 24 «كَلِّمِ الجَمَاعَةَ قَائِلاً اطْلعُوا مِنْ حَوَاليْ مَسْكَنِ قُورَحَ وَدَاثَانَ وَأَبِيرَامَ».
\par 25 فَقَامَ مُوسَى وَذَهَبَ إِلى دَاثَانَ وَأَبِيرَامَ وَذَهَبَ وَرَاءَهُ شُيُوخُ إِسْرَائِيل.
\par 26 فَقَال لِلجَمَاعَةِ: «اعْتَزِلُوا عَنْ خِيَامِ هَؤُلاءِ القَوْمِ البُغَاةِ وَلا تَمَسُّوا شَيْئاً مِمَّا لهُمْ لِئَلا تَهْلكُوا بِجَمِيعِ خَطَايَاهُمْ».
\par 27 فَطَلعُوا مِنْ حَوَاليْ مَسْكَنِ قُورَحَ وَدَاثَانَ وَأَبِيرَامَ وَخَرَجَ دَاثَانُ وَأَبِيرَامُ وَوَقَفَا فِي بَابِ خَيْمَتَيْهِمَا مَعَ نِسَائِهِمَا وَبَنِيهِمَا وَأَطْفَالِهِمَا.
\par 28 فَقَال مُوسَى: «بِهَذَا تَعْلمُونَ أَنَّ الرَّبَّ قَدْ أَرْسَلنِي لأَعْمَل كُل هَذِهِ الأَعْمَالِ وَأَنَّهَا ليْسَتْ مِنْ نَفْسِي.
\par 29 إِنْ مَاتَ هَؤُلاءِ كَمَوْتِ كُلِّ إِنْسَانٍ وَأَصَابَتْهُمْ مَصِيبَةُ كُلِّ إِنْسَانٍ فَليْسَ الرَّبُّ قَدْ أَرْسَلنِي.
\par 30 وَلكِنْ إِنِ ابْتَدَعَ الرَّبُّ بِدْعَةً وَفَتَحَتِ الأَرْضُ فَاهَا وَابْتَلعَتْهُمْ وَكُل مَا لهُمْ فَهَبَطُوا أَحْيَاءً إِلى الهَاوِيَةِ تَعْلمُونَ أَنَّ هَؤُلاءِ القَوْمَ قَدِ ازْدَرُوا بِالرَّبِّ».
\par 31 فَلمَّا فَرَغَ مِنَ التَّكَلُّمِ بِكُلِّ هَذَا الكَلامِ انْشَقَّتِ الأَرْضُ التِي تَحْتَهُمْ
\par 32 وَفَتَحَتِ الأَرْضُ فَاهَا وَابْتَلعَتْهُمْ وَبُيُوتَهُمْ وَكُل مَنْ كَانَ لِقُورَحَ مَعَ كُلِّ الأَمْوَالِ
\par 33 فَنَزَلُوا هُمْ وَكُلُّ مَا كَانَ لهُمْ أَحْيَاءً إِلى الهَاوِيَةِ وَانْطَبَقَتْ عَليْهِمِ الأَرْضُ فَبَادُوا مِنْ بَيْنِ الجَمَاعَةِ.
\par 34 وَكُلُّ إِسْرَائِيل الذِينَ حَوْلهُمْ هَرَبُوا مِنْ صَوْتِهِمْ لأَنَّهُمْ قَالُوا: «لعَل الأَرْضَ تَبْتَلِعُنَا».
\par 35 وَخَرَجَتْ نَارٌ مِنْ عِنْدِ الرَّبِّ وَأَكَلتِ المِئَتَيْنِ وَالخَمْسِينَ رَجُلاً الذِينَ قَرَّبُوا البَخُورَ.
\par 36 ثُمَّ قَال الرَّبُّ لِمُوسَى:
\par 37 «قُل لأَلِعَازَارَ بْنِ هَارُونَ الكَاهِنِ أَنْ يَرْفَعَ المَجَامِرَ مِنَ الحَرِيقِ وَاذْرِ النَّارَ هُنَاكَ فَإِنَّهُنَّ قَدْ تَقَدَّسْنَ.
\par 38 مَجَامِرَ هَؤُلاءِ المُخْطِئِينَ ضِدَّ نُفُوسِهِمْ فَليَعْمَلُوهَا صَفَائِحَ مَطْرُوقَةً غِشَاءً لِلمَذْبَحِ لأَنَّهُمْ قَدْ قَدَّمُوهَا أَمَامَ الرَّبِّ فَتَقَدَّسَتْ. فَتَكُونُ عَلامَةً لِبَنِي إِسْرَائِيل».
\par 39 فَأَخَذَ أَلِعَازَارُ الكَاهِنُ مَجَامِرَ النُّحَاسِ التِي قَدَّمَهَا المُحْتَرِقُونَ وَطَرَقُوهَا غِشَاءً لِلمَذْبَحِ
\par 40 تِذْكَاراً لِبَنِي إِسْرَائِيل لِكَيْ لا يَقْتَرِبَ رَجُلٌ أَجْنَبِيٌّ ليْسَ مِنْ نَسْلِ هَارُونَ لِيُبَخِّرَ بَخُوراً أَمَامَ الرَّبِّ فَيَكُونَ مِثْل قُورَحَ وَجَمَاعَتِهِ كَمَا كَلمَهُ الرَّبُّ عَنْ يَدِ مُوسَى.
\par 41 فَتَذَمَّرَ كُلُّ جَمَاعَةِ بَنِي إِسْرَائِيل فِي الغَدِ عَلى مُوسَى وَهَارُونَ قَائِلِينَ: «أَنْتُمَا قَدْ قَتَلتُمَا شَعْبَ الرَّبِّ».
\par 42 وَلمَّا اجْتَمَعَتِ الجَمَاعَةُ عَلى مُوسَى وَهَارُونَ انْصَرَفَا إِلى خَيْمَةِ الاِجْتِمَاعِ وَإِذَا هِيَ قَدْ غَطَّتْهَا السَّحَابَةُ وَتَرَاءَى مَجْدُ الرَّبِّ.
\par 43 فَجَاءَ مُوسَى وَهَارُونُ إِلى قُدَّامِ خَيْمَةِ الاِجْتِمَاعِ.
\par 44 فَقَال الرَّبُّ لِمُوسَى:
\par 45 «اِطْلعَا مِنْ وَسَطِ هَذِهِ الجَمَاعَةِ فَإِنِّي أُفْنِيهِمْ بِلحْظَةٍ». فَخَرَّا عَلى وَجْهَيْهِمَا.
\par 46 ثُمَّ قَال مُوسَى لِهَارُونَ: «خُذِ المَجْمَرَةَ وَاجْعَل فِيهَا نَاراً مِنْ عَلى المَذْبَحِ وَضَعْ بَخُوراً وَاذْهَبْ بِهَا مُسْرِعاً إِلى الجَمَاعَةِ وَكَفِّرْ عَنْهُمْ لأَنَّ السَّخَطَ قَدْ خَرَجَ مِنْ قِبَلِ الرَّبِّ. قَدِ ابْتَدَأَ الوَبَأُ».
\par 47 فَأَخَذَ هَارُونُ كَمَا قَال مُوسَى وَرَكَضَ إِلى وَسَطِ الجَمَاعَةِ وَإِذَا الوَبَأُ قَدِ ابْتَدَأَ فِي الشَّعْبِ. فَوَضَعَ البَخُورَ وَكَفَّرَ عَنِ الشَّعْبِ.
\par 48 وَوَقَفَ بَيْنَ المَوْتَى وَالأَحْيَاءِ فَامْتَنَعَ الوَبَأُ.
\par 49 فَكَانَ الذِينَ مَاتُوا بِالوَبَإِ أَرْبَعَةَ عَشَرَ أَلفاً وَسَبْعَ مِئَةٍ عَدَا الذِينَ مَاتُوا بِسَبَبِ قُورَحَ.
\par 50 ثُمَّ رَجَعَ هَارُونُ إِلى مُوسَى إِلى بَابِ خَيْمَةِ الاِجْتِمَاعِ وَالوَبَأُ قَدِ امْتَنَعَ.

\chapter{17}

\par 1 وَقَال الرَّبُّ لِمُوسَى:
\par 2 «كَلِّمْ بَنِي إِسْرَائِيل وَخُذْ مِنْهُمْ عَصاً عَصاً لِكُلِّ بَيْتِ أَبٍ مِنْ جَمِيعِ رُؤَسَائِهِمْ حَسَبَ بُيُوتِ آبَائِهِمِ. اثْنَتَيْ عَشَرَةَ عَصاً. وَاسْمُ كُلِّ وَاحِدٍ تَكْتُبُهُ عَلى عَصَاهُ.
\par 3 وَاسْمُ هَارُونَ تَكْتُبُهُ عَلى عَصَا لاوِي لأَنَّ لِرَأْسِ بَيْتِ آبَائِهِمْ عَصاً وَاحِدَةً.
\par 4 وَضَعْهَا فِي خَيْمَةِ الاِجْتِمَاعِ أَمَامَ الشَّهَادَةِ حَيْثُ أَجْتَمِعُ بِكُمْ.
\par 5 فَالرَّجُلُ الذِي أَخْتَارُهُ تُفْرِخُ عَصَاهُ فَأُسَكِّنُ عَنِّي تَذَمُّرَاتِ بَنِي إِسْرَائِيل التِي يَتَذَمَّرُونَهَا عَليْكُمَا».
\par 6 فَكَلمَ مُوسَى بَنِي إِسْرَائِيل فَأَعْطَاهُ جَمِيعُ رُؤَسَائِهِمْ عَصاً عَصاً لِكُلِّ رَئِيسٍ حَسَبَ بُيُوتِ آبَائِهِمِ. اثْنَتَيْ عَشَرَةَ عَصاً. وَعَصَا هَارُونَ بَيْنَ عِصِيِّهِمْ.
\par 7 فَوَضَعَ مُوسَى العِصِيَّ أَمَامَ الرَّبِّ فِي خَيْمَةِ الشَّهَادَةِ.
\par 8 وَفِي الغَدِ دَخَل مُوسَى إِلى خَيْمَةِ الشَّهَادَةِ وَإِذَا عَصَا هَارُونَ لِبَيْتِ لاوِي قَدْ أَفْرَخَتْ. أَخْرَجَتْ فُرُوخاً وَأَزْهَرَتْ زَهْراً وَأَنْضَجَتْ لوْزاً.
\par 9 فَأَخْرَجَ مُوسَى جَمِيعَ العِصِيِّ مِنْ أَمَامِ الرَّبِّ إِلى جَمِيعِ بَنِي إِسْرَائِيل فَنَظَرُوا وَأَخَذَ كُلُّ وَاحِدٍ عَصَاهُ.
\par 10 وَقَال الرَّبُّ لِمُوسَى: «رُدَّ عَصَا هَارُونَ إِلى أَمَامِ الشَّهَادَةِ لأَجْلِ الحِفْظِ عَلامَةً لِبَنِي التَّمَرُّدِ فَتَكُفَّ تَذَمُّرَاتُهُمْ عَنِّي لِكَيْ لا يَمُوتُوا».
\par 11 فَفَعَل مُوسَى كَمَا أَمَرَهُ الرَّبُّ. كَذَلِكَ فَعَل.
\par 12 فَقَال بَنُو إِسْرَائِيل لِمُوسَى: «إِنَّنَا فَنِينَا وَهَلكْنَا. قَدْ هَلكْنَا جَمِيعاً.
\par 13 كُلُّ مَنِ اقْتَرَبَ إِلى مَسْكَنِ الرَّبِّ يَمُوتُ! أَمَا فَنِيْنَا تَمَاماً؟».

\chapter{18}

\par 1 وَقَال الرَّبُّ لِهَارُونَ: «أَنْتَ وَبَنُوكَ وَبَيْتُ أَبِيكَ مَعَكَ تَحْمِلُونَ ذَنْبَ المَقْدِسِ. وَأَنْتَ وَبَنُوكَ مَعَكَ تَحْمِلُونَ ذَنْبَ كَهَنُوتِكُمْ.
\par 2 وَأَيْضاً إِخْوَتَكَ سِبْطَ لاوِي سِبْطُ أَبِيكَ قَرِّبْهُمْ مَعَكَ فَيَقْتَرِنُوا بِكَ وَيُوازِرُوكَ وَأَنْتَ وَبَنُوكَ قُدَّامَ خَيْمَةِ الشَّهَادَةِ
\par 3 فَيَحْفَظُونَ حِرَاسَتَكَ وَحِرَاسَةَ الخَيْمَةِ كُلِّهَا. وَلكِنْ إِلى أَمْتِعَةِ القُدْسِ وَإِلى المَذْبَحِ لا يَقْتَرِبُونَ لِئَلا يَمُوتُوا هُمْ وَأَنْتُمْ جَمِيعاً.
\par 4 يَقْتَرِنُونَ بِكَ وَيَحْفَظُونَ حِرَاسَةَ خَيْمَةِ الاِجْتِمَاعِ مَعَ كُلِّ خِدْمَةِ الخَيْمَةِ. وَالأَجْنَبِيُّ لا يَقْتَرِبْ إِليْكُمْ.
\par 5 بَل تَحْفَظُونَ أَنْتُمْ حِرَاسَةَ القُدْسِ وَحِرَاسَةَ المَذْبَحِ لِكَيْ لا يَكُونَ أَيْضاً سَخَطٌ عَلى بَنِي إِسْرَائِيل.
\par 6 هَئَنَذَا قَدْ أَخَذْتُ إِخْوَتَكُمُ اللاوِيِّينَ مِنْ بَيْنِ بَنِي إِسْرَائِيل عَطِيَّةً لكُمْ مُعْطَيْنَ لِلرَّبِّ لِيَخْدِمُوا خِدْمَةَ خَيْمَةِ الاِجْتِمَاعِ.
\par 7 وَأَمَّا أَنْتَ وَبَنُوكَ مَعَكَ فَتَحْفَظُونَ كَهَنُوتَكُمْ مَعَ مَا لِلمَذْبَحِ وَمَا هُوَ دَاخِل الحِجَابِ وَتَخْدِمُونَ خِدْمَةً. عَطِيَّةً أَعْطَيْتُ كَهَنُوتَكُمْ. وَالأَجْنَبِيُّ الذِي يَقْتَرِبُ يُقْتَلُ».
\par 8 وَقَال الرَّبُّ لِهَارُونَ: «وَهَئَنَذَا قَدْ أَعْطَيْتُكَ حِرَاسَةَ رَفَائِعِي. مَعَ جَمِيعِ أَقْدَاسِ بَنِي إِسْرَائِيل لكَ أَعْطَيْتُهَا حَقَّ المَسْحَةِ وَلِبَنِيكَ فَرِيضَةً دَهْرِيَّةً.
\par 9 هَذَا يَكُونُ لكَ مِنْ قُدْسِ الأَقْدَاسِ مِنَ النَّارِ كُلُّ قَرَابِينِهِمْ مَعَ كُلِّ تَقْدِمَاتِهِمْ وَكُلِّ ذَبَائِحِ خَطَايَاهُمْ وَكُلِّ ذَبَائِحِ آثَامِهِمُِ التِي يَرُدُّونَهَا لِي. قُدْسُ أَقْدَاسٍ هِيَ لكَ وَلِبَنِيكَ.
\par 10 فِي قُدْسِ الأَقْدَاسِ تَأْكُلُهَا. كُلُّ ذَكَرٍ يَأْكُلُهَا. قُدْساً تَكُونُ لكَ.
\par 11 وَهَذِهِ لكَ: الرَّفِيعَةُ مِنْ عَطَايَاهُمْ مَعَ كُلِّ تَرْدِيدَاتِ بَنِي إِسْرَائِيل. لكَ أَعْطَيْتُهَا وَلِبَنِيكَ وَبَنَاتِكَ مَعَكَ فَرِيضَةً دَهْرِيَّةً. كُلُّ طَاهِرٍ فِي بَيْتِكَ يَأْكُلُ مِنْهَا.
\par 12 كُلُّ دَسَمِ الزَّيْتِ وَكُلُّ دَسَمِ المِسْطَارِ وَالحِنْطَةِ أَبْكَارُهُنَّ التِي يُعْطُونَهَا لِلرَّبِّ لكَ أَعْطَيْتُهَا.
\par 13 أَبْكَارُ كُلِّ مَا فِي أَرْضِهِمِ التِي يُقَدِّمُونَهَا لِلرَّبِّ لكَ تَكُونُ. كُلُّ طَاهِرٍ فِي بَيْتِكَ يَأْكُلُهَا.
\par 14 كُلُّ مُحَرَّمٍ فِي إِسْرَائِيل يَكُونُ لكَ.
\par 15 كُلُّ فَاتِحِ رَحِمٍ مِنْ كُلِّ جَسَدٍ يُقَدِّمُونَهُ لِلرَّبِّ مِنَ النَّاسِ وَمِنَ البَهَائِمِ يَكُونُ لكَ. غَيْرَ أَنَّكَ تَقْبَلُ فِدَاءَ بِكْرِ الإِنْسَانِ وَبِكْرُ البَهِيمَةِ النَّجِسَةِ تَقْبَلُ فِدَاءَهُ.
\par 16 وَفِدَاؤُهُ مِنِ ابْنِ شَهْرٍ تَقْبَلُهُ حَسَبَ تَقْوِيمِكَ فِضَّةً خَمْسَةَ شَوَاقِل عَلى شَاقِلِ القُدْسِ. هُوَ عِشْرُونَ جِيرَةً.
\par 17 لكِنْ بِكْرُ البَقَرِ أَوْ بِكْرُ الضَّأْنِ أَوْ بِكْرُ المَعْزِ لا تَقْبَل فِدَاءَهُ. إِنَّهُ قُدْسٌ. بَل تَرُشُّ دَمَهُ عَلى المَذْبَحِ وَتُوقِدُ شَحْمَهُ وَقُوداً رَائِحَةَ سُرُورٍ لِلرَّبِّ.
\par 18 وَلحْمُهُ يَكُونُ لكَ كَصَدْرِ التَّرْدِيدِ وَالسَّاقِ اليُمْنَى يَكُونُ لكَ.
\par 19 جَمِيعُ رَفَائِعِ الأَقْدَاسِ التِي يَرْفَعُهَا بَنُو إِسْرَائِيل لِلرَّبِّ أَعْطَيْتُهَا لكَ وَلِبَنِيكَ وَبَنَاتِكَ مَعَكَ حَقّاً دَهْرِيّاً. مِيثَاقَ مِلحٍ دَهْرِيّاً أَمَامَ الرَّبِّ لكَ وَلِزَرْعِكَ مَعَكَ».
\par 20 وَقَال الرَّبُّ لِهَارُونَ: «لا تَنَالُ نَصِيباً فِي أَرْضِهِمْ وَلا يَكُونُ لكَ قِسْمٌ فِي وَسَطِهِمْ. أَنَا قِسْمُكَ وَنَصِيبُكَ فِي وَسَطِ بَنِي إِسْرَائِيل.
\par 21 «وَأَمَّا بَنُو لاوِي فَإِنِّي قَدْ أَعْطَيْتُهُمْ كُل عُشْرٍ فِي إِسْرَائِيل مِيرَاثاً عِوَضَ خِدْمَةِ خَيْمَةِ الاِجْتِمَاعِ التِي يَخْدِمُونَهَا.
\par 22 فَلا يَقْتَرِبُ أَيْضاً بَنُو إِسْرَائِيل إِلى خَيْمَةِ الاِجْتِمَاعِ لِيَحْمِلُوا خَطِيَّةً لِلمَوْتِ
\par 23 بَلِ اللاوِيُّونَ يَخْدِمُونَ خِدْمَةَ خَيْمَةِ الاِجْتِمَاعِ وَهُمْ يَحْمِلُونَ ذَنْبَهُمْ فَرِيضَةً دَهْرِيَّةً فِي أَجْيَالِكُمْ. وَفِي وَسَطِ إِسْرَائِيل لا يَنَالُونَ نَصِيباً.
\par 24 إِنَّ عُشُورَ بَنِي إِسْرَائِيل التِي يَرْفَعُونَهَا لِلرَّبِّ رَفِيعَةً قَدْ أَعْطَيْتُهَا لِلاوِيِّينَ نَصِيباً. لِذَلِكَ قُلتُ لهُمْ: فِي وَسَطِ بَنِي إِسْرَائِيل لا يَنَالُونَ نَصِيباً».
\par 25 وَقَال الرَّبُّ لِمُوسَى:
\par 26 «لِلاوِيِّينَ تَقُولُ: مَتَى أَخَذْتُمْ مِنْ بَنِي إِسْرَائِيل العُشْرَ الذِي أَعْطَيْتُكُمْ إِيَّاهُ مِنْ عِنْدِهِمْ نَصِيباً لكُمْ تَرْفَعُونَ مِنْهُ رَفِيعَةَ الرَّبِّ: عُشْراً مِنَ العُشْرِ
\par 27 فَيُحْسَبُ لكُمْ. إِنَّهُ رَفِيعَتُكُمْ كَالحِنْطَةِ مِنَ البَيْدَرِ وَكَالمِلءِ مِنَ المِعْصَرَةِ.
\par 28 فَهَكَذَا تَرْفَعُونَ أَنْتُمْ أَيْضاً رَفِيعَةَ الرَّبِّ مِنْ جَمِيعِ عُشُورِكُمُ التِي تَأْخُذُونَ مِنْ بَنِي إِسْرَائِيل. تُعْطُونَ مِنْهَا رَفِيعَةَ الرَّبِّ لِهَارُونَ الكَاهِنِ.
\par 29 مِنْ جَمِيعِ عَطَايَاكُمْ تَرْفَعُونَ كُل رَفِيعَةِ الرَّبِّ مِنَ الكُلِّ دَسَمَهُ المُقَدَّسَ مِنْهُ.
\par 30 وَتَقُولُ لهُمْ: حِينَ تَرْفَعُونَ دَسَمَهُ مِنْهُ يُحْسَبُ لِلاوِيِّينَ كَمَحْصُولِ البَيْدَرِ وَكَمَحْصُولِ المِعْصَرَةِ.
\par 31 وَتَأْكُلُونَهُ فِي كُلِّ مَكَانٍ أَنْتُمْ وَبُيُوتُكُمْ لأَنَّهُ أُجْرَةٌ لكُمْ عِوَضَ خِدْمَتِكُمْ فِي خَيْمَةِ الاِجْتِمَاعِ.
\par 32 وَلا تَتَحَمَّلُونَ بِسَبَبِهِ خَطِيَّةً إِذَا رَفَعْتُمْ دَسَمَهُ مِنْهُ. وَأَمَّا أَقْدَاسُ بَنِي إِسْرَائِيل فَلا تُدَنِّسُوهَا لِئَلا تَمُوتُوا».

\chapter{19}

\par 1 وَقَال الرَّبُّ لِمُوسَى وَهَارُونَ:
\par 2 «هَذِهِ فَرِيضَةُ الشَّرِيعَةِ التِي أَمَرَ بِهَا الرَّبُّ: كَلِّمْ بَنِي إِسْرَائِيل أَنْ يَأْخُذُوا إِليْكَ بَقَرَةً حَمْرَاءَ صَحِيحَةً لا عَيْبَ فِيهَا وَلمْ يَعْلُ عَليْهَا نِيرٌ
\par 3 فَتُعْطُونَهَا لأَلِعَازَارَ الكَاهِنِ فَتُخْرَجُ إِلى خَارِجِ المَحَلةِ وَتُذْبَحُ قُدَّامَهُ.
\par 4 وَيَأْخُذُ أَلِعَازَارُ الكَاهِنُ مِنْ دَمِهَا بِإِصْبِعِهِ وَيَنْضِحُ مِنْ دَمِهَا إِلى جِهَةِ وَجْهِ خَيْمَةِ الاِجْتِمَاعِ سَبْعَ مَرَّاتٍ.
\par 5 وَتُحْرَقُ البَقَرَةُ أَمَامَ عَيْنَيْهِ. يُحْرَقُ جِلدُهَا وَلحْمُهَا وَدَمُهَا مَعَ فَرْثِهَا.
\par 6 وَيَأْخُذُ الكَاهِنُ خَشَبَ أَرْزٍ وَزُوفَا وَقِرْمِزاً وَيَطْرَحُهُنَّ فِي وَسَطِ حَرِيقِ البَقَرَةِ
\par 7 ثُمَّ يَغْسِلُ الكَاهِنُ ثِيَابَهُ وَيَرْحَضُ جَسَدَهُ بِمَاءٍ وَبَعْدَ ذَلِكَ يَدْخُلُ المَحَلةَ. وَيَكُونُ الكَاهِنُ نَجِساً إِلى المَسَاءِ.
\par 8 وَالذِي أَحْرَقَهَا يَغْسِلُ ثِيَابَهُ بِمَاءٍ وَيَرْحَضُ جَسَدَهُ بِمَاءٍ وَيَكُونُ نَجِساً إِلى المَسَاءِ.
\par 9 وَيَجْمَعُ رَجُلٌ طَاهِرٌ رَمَادَ البَقَرَةِ وَيَضَعُهُ خَارِجَ المَحَلةِ فِي مَكَانٍ طَاهِرٍ فَتَكُونُ لِجَمَاعَةِ بَنِي إِسْرَائِيل فِي حِفْظٍ مَاءَ نَجَاسَةٍ. إِنَّهَا ذَبِيحَةُ خَطِيَّةٍ.
\par 10 وَالذِي جَمَعَ رَمَادَ البَقَرَةِ يَغْسِلُ ثِيَابَهُ وَيَكُونُ نَجِساً إِلى المَسَاءِ. فَتَكُونُ لِبَنِي إِسْرَائِيل وَلِلغَرِيبِ النَّازِلِ فِي وَسَطِهِمْ فَرِيضَةً دَهْرِيَّةً.
\par 11 «مَنْ مَسَّ مَيِّتاً مَيِّتَةَ إِنْسَانٍ مَا يَكُونُ نَجِساً سَبْعَةَ أَيَّامٍ.
\par 12 يَتَطَهَّرُ بِهِ فِي اليَوْمِ الثَّالِثِ وَفِي اليَوْمِ السَّابِعِ يَكُونُ طَاهِراً. وَإِنْ لمْ يَتَطَهَّرْ فِي اليَوْمِ الثَّالِثِ فَفِي اليَوْمِ السَّابِعِ لا يَكُونُ طَاهِراً.
\par 13 كُلُّ مَنْ مَسَّ مَيِّتاً مَيِّتَةَ إِنْسَانٍ قَدْ مَاتَ وَلمْ يَتَطَهَّرْ يُنَجِّسُ مَسْكَنَ الرَّبِّ. فَتُقْطَعُ تِلكَ النَّفْسُ مِنْ إِسْرَائِيل. لأَنَّ مَاءَ النَّجَاسَةِ لمْ يُرَشَّ عَليْهَا تَكُونُ نَجِسَةً. نَجَاسَتُهَا لمْ تَزَل فِيهَا.
\par 14 «هَذِهِ هِيَ الشَّرِيعَةُ: إِذَا مَاتَ إِنْسَانٌ فِي خَيْمَةٍ فَكُلُّ مَنْ دَخَل الخَيْمَةَ وَكُلُّ مَنْ كَانَ فِي الخَيْمَةِ يَكُونُ نَجِساً سَبْعَةَ أَيَّامٍ.
\par 15 وَكُلُّ إِنَاءٍ مَفْتُوحٍ ليْسَ عَليْهِ سِدَادٌ بِعِصَابَةٍ فَإِنَّهُ نَجِسٌ.
\par 16 وَكُلُّ مَنْ مَسَّ عَلى وَجْهِ الصَّحْرَاءِ قَتِيلاً بِالسَّيْفِ أَوْ مَيِّتاً أَوْ عَظْمَ إِنْسَانٍ أَوْ قَبْراً يَكُونُ نَجِساً سَبْعَةَ أَيَّامٍ.
\par 17 فَيَأْخُذُونَ لِلنَّجِسِ مِنْ غُبَارِ حَرِيقِ ذَبِيحَةِ الخَطِيَّةِ وَيَجْعَلُ عَليْهِ مَاءً حَيّاً فِي إِنَاءٍ.
\par 18 وَيَأْخُذُ رَجُلٌ طَاهِرٌ زُوفَا وَيَغْمِسُهَا فِي المَاءِ وَيَنْضِحُهُ عَلى الخَيْمَةِ وَعَلى جَمِيعِ الأَمْتِعَةِ وَعَلى الأَنْفُسِ الذِينَ كَانُوا هُنَاكَ وَعَلى الذِي مَسَّ العَظْمَ أَوِ القَتِيل أَوِ المَيِّتَ أَوِ القَبْرَ.
\par 19 يَنْضِحُ الطَّاهِرُ عَلى النَّجِسِ فِي اليَوْمِ الثَّالِثِ وَاليَوْمِ السَّابِعِ. وَيُطَهِّرُهُ فِي اليَوْمِ السَّابِعِ فَيَغْسِلُ ثِيَابَهُ وَيَرْحَضُ بِمَاءٍ فَيَكُونُ طَاهِراً فِي المَسَاءِ.
\par 20 وَأَمَّا الإِنْسَانُ الذِي يَتَنَجَّسُ وَلا يَتَطَهَّرُ فَتُبَادُ تِلكَ النَّفْسُ مِنْ بَيْنِ الجَمَاعَةِ لأَنَّهُ نَجَّسَ مَقْدِسَ الرَّبِّ. مَاءُ النَّجَاسَةِ لمْ يُرَشَّ عَليْهِ. إِنَّهُ نَجِسٌ.
\par 21 فَتَكُونُ لهُمْ فَرِيضَةً دَهْرِيَّةً. وَالذِي رَشَّ مَاءَ النَّجَاسَةِ يَغْسِلُ ثِيَابَهُ وَالذِي مَسَّ مَاءَ النَّجَاسَةِ يَكُونُ نَجِساً إِلى المَسَاءِ.
\par 22 وَكُلُّ مَا مَسَّهُ النَّجِسُ يَتَنَجَّسُ وَالنَّفْسُ التِي تَمَسُّ تَكُونُ نَجِسَةً إِلى المَسَاءِ».

\chapter{20}

\par 1 وَأَتَى بَنُو إِسْرَائِيل الجَمَاعَةُ كُلُّهَا إِلى بَرِّيَّةِ صِينَ فِي الشَّهْرِ الأَوَّلِ. وَأَقَامَ الشَّعْبُ فِي قَادِشَ. وَمَاتَتْ هُنَاكَ مَرْيَمُ وَدُفِنَتْ هُنَاكَ.
\par 2 وَلمْ يَكُنْ مَاءٌ لِلجَمَاعَةِ فَاجْتَمَعُوا عَلى مُوسَى وَهَارُونَ.
\par 3 وَخَاصَمَ الشَّعْبُ مُوسَى وَقَالُوا لهُ: «ليْتَنَا فَنِينَا فَنَاءَ إِخْوَتِنَا أَمَامَ الرَّبِّ.
\par 4 لِمَاذَا أَتَيْتُمَا بِجَمَاعَةِ الرَّبِّ إِلى هَذِهِ البَرِّيَّةِ لِكَيْ نَمُوتَ فِيهَا نَحْنُ وَمَوَاشِينَا؟
\par 5 وَلِمَاذَا أَصْعَدْتُمَانَا مِنْ مِصْرَ لِتَأْتِيَا بِنَا إِلى هَذَا المَكَانِ الرَّدِيءِ؟ ليْسَ هُوَ مَكَانَ زَرْعٍ وَتِينٍ وَكَرْمٍ وَرُمَّانٍ وَلا فِيهِ مَاءٌ لِلشُّرْبِ».
\par 6 فَأَتَى مُوسَى وَهَارُونُ مِنْ أَمَامِ الجَمَاعَةِ إِلى بَابِ خَيْمَةِ الاِجْتِمَاعِ وَسَقَطَا عَلى وَجْهَيْهِمَا. فَتَرَاءَى لهُمَا مَجْدُ الرَّبِّ.
\par 7 وَأَمَرَ الرَّبُّ مُوسَى:
\par 8 «خُذِ العَصَا وَاجْمَعِ الجَمَاعَةَ أَنْتَ وَهَارُونُ أَخُوكَ وَكَلِّمَا الصَّخْرَةَ أَمَامَ أَعْيُنِهِمْ أَنْ تُعْطِيَ مَاءَهَا فَتُخْرِجُ لهُمْ مَاءً مِنَ الصَّخْرَةِ وَتَسْقِي الجَمَاعَةَ وَمَوَاشِيَهُمْ».
\par 9 فَأَخَذَ مُوسَى العَصَا مِنْ أَمَامِ الرَّبِّ كَمَا أَمَرَهُ
\par 10 وَجَمَعَ مُوسَى وَهَارُونُ الجُمْهُورَ أَمَامَ الصَّخْرَةِ فَقَال لهُمُ: «اسْمَعُوا أَيُّهَا المَرَدَةُ! أَمِنْ هَذِهِ الصَّخْرَةِ نُخْرِجُ لكُمْ مَاءً؟»
\par 11 وَرَفَعَ مُوسَى يَدَهُ وَضَرَبَ الصَّخْرَةَ بِعَصَاهُ مَرَّتَيْنِ فَخَرَجَ مَاءٌ غَزِيرٌ فَشَرِبَتِ الجَمَاعَةُ وَمَوَاشِيهَا.
\par 12 فَقَال الرَّبُّ لِمُوسَى وَهَارُونَ: «مِنْ أَجْلِ أَنَّكُمَا لمْ تُؤْمِنَا بِي حَتَّى تُقَدِّسَانِي أَمَامَ أَعْيُنِ بَنِي إِسْرَائِيل لِذَلِكَ لا تُدْخِلانِ هَذِهِ الجَمَاعَةَ إِلى الأَرْضِ التِي أَعْطَيْتُهُمْ إِيَّاهَا».
\par 13 هَذَا مَاءُ مَرِيبَةَ حَيْثُ خَاصَمَ بَنُو إِسْرَائِيل الرَّبَّ فَتَقَدَّسَ فِيهِمْ.
\par 14 وَأَرْسَل مُوسَى رُسُلاً مِنْ قَادِشَ إِلى مَلِكِ أَدُومَ: «هَكَذَا يَقُولُ أَخُوكَ إِسْرَائِيلُ قَدْ عَرَفْتَ كُل المَشَقَّةِ التِي أَصَابَتْنَا.
\par 15 إِنَّ آبَاءَنَا انْحَدَرُوا إِلى مِصْرَ وَأَقَمْنَا فِي مِصْرَ أَيَّاماً كَثِيرَةً وَأَسَاءَ المِصْرِيُّونَ إِليْنَا وَإِلى آبَائِنَا
\par 16 فَصَرَخْنَا إِلى الرَّبِّ فَسَمِعَ صَوْتَنَا وَأَرْسَل مَلاكاً وَأَخْرَجَنَا مِنْ مِصْرَ. وَهَا نَحْنُ فِي قَادِشَ مَدِينَةٍ فِي طَرَفِ تُخُومِكَ.
\par 17 دَعْنَا نَمُرَّ فِي أَرْضِكَ. لا نَمُرُّ فِي حَقْلٍ وَلا فِي كَرْمٍ وَلا نَشْرَبُ مَاءَ بِئْرٍ. فِي طَرِيقِ المَلِكِ نَمْشِي لا نَمِيلُ يَمِيناً وَلا يَسَاراً حَتَّى نَتَجَاوَزَ تُخُومَكَ».
\par 18 فَقَال لهُ أَدُومُ: «لا تَمُرُّ بِي لِئَلا أَخْرُجَ لِلِقَائِكَ بِالسَّيْفِ».
\par 19 فَقَال لهُ بَنُو إِسْرَائِيل: «فِي السِّكَّةِ نَصْعَدُ. وَإِذَا شَرِبْنَا أَنَا وَمَوَاشِيَّ مِنْ مَائِكَ أَدْفَعُ ثَمَنَهُ. لا شَيْءَ. أَمُرُّ بِرِجْليَّ فَقَطْ.
\par 20 فَقَال: «لا تَمُرُّ». وَخَرَجَ أَدُومُ لِلِقَائِهِ بِشَعْبٍ غَفِيرٍ وَبِيَدٍ شَدِيدَةٍ.
\par 21 وَأَبَى أَدُومُ أَنْ يَسْمَحَ لِإِسْرَائِيل بِالمُرُورِ فِي تُخُومِهِ فَتَحَوَّل إِسْرَائِيلُ عَنْهُ.
\par 22 فَارْتَحَل بَنُو إِسْرَائِيل الجَمَاعَةُ كُلُّهَا مِنْ قَادِشَ وَأَتُوا إِلى جَبَلِ هُورٍ.
\par 23 وَقَال الرَّبُّ لِمُوسَى وَهَارُونَ فِي جَبَلِ هُورٍ عَلى تُخُمِ أَرْضِ أَدُومَ:
\par 24 «يُضَمُّ هَارُونُ إِلى قَوْمِهِ لأَنَّهُ لا يَدْخُلُ الأَرْضَ التِي أَعْطَيْتُ لِبَنِي إِسْرَائِيل لأَنَّكُمْ عَصَيْتُمْ قَوْلِي عِنْدَ مَاءِ مَرِيبَةَ.
\par 25 خُذْ هَارُونَ وَأَلِعَازَارَ ابْنَهُ وَاصْعَدْ بِهِمَا إِلى جَبَلِ هُورٍ
\par 26 وَاخْلعْ عَنْ هَارُونَ ثِيَابَهُ وَأَلبِسْ أَلِعَازَارَ ابْنَهُ إِيَّاهَا. فَيُضَمُّ هَارُونُ وَيَمُوتُ هُنَاكَ».
\par 27 فَفَعَل مُوسَى كَمَا أَمَرَ الرَّبُّ وَصَعِدُوا إِلى جَبَلِ هُورٍ أَمَامَ أَعْيُنِ كُلِّ الجَمَاعَةِ.
\par 28 فَخَلعَ مُوسَى عَنْ هَارُونَ ثِيَابَهُ وَأَلبَسَ أَلِعَازَارَ ابْنَهُ إِيَّاهَا. فَمَاتَ هَارُونُ هُنَاكَ عَلى رَأْسِ الجَبَلِ. ثُمَّ انْحَدَرَ مُوسَى وَأَلِعَازَارُ عَنِ الجَبَلِ.
\par 29 فَلمَّا رَأَى كُلُّ الجَمَاعَةِ أَنَّ هَارُونَ قَدْ مَاتَ بَكَى جَمِيعُ بَيْتِ إِسْرَائِيل عَلى هَارُونَ ثَلاثِينَ يَوْماً.

\chapter{21}

\par 1 وَلمَّا سَمِعَ الكَنْعَانِيُّ مَلِكُ عَرَادَ السَّاكِنُ فِي الجَنُوبِ أَنَّ إِسْرَائِيل جَاءَ فِي طَرِيقِ أَتَارِيمَ حَارَبَ إِسْرَائِيل وَسَبَى مِنْهُمْ سَبْياً.
\par 2 فَنَذَرَ إِسْرَائِيلُ نَذْراً لِلرَّبِّ وَقَال: «إِنْ دَفَعْتَ هَؤُلاءِ القَوْمَ إِلى يَدِي أُحَرِّمُ مُدُنَهُمْ».
\par 3 فَسَمِعَ الرَّبُّ لِقَوْلِ إِسْرَائِيل وَدَفَعَ الكَنْعَانِيِّينَ فَحَرَّمُوهُمْ وَمُدُنَهُمْ. فَدُعِيَ اسْمُ المَكَانِ «حُرْمَةَ».
\par 4 وَارْتَحَلُوا مِنْ جَبَلِ هُورٍ فِي طَرِيقِ بَحْرِ سُوفٍ لِيَدُورُوا بِأَرْضِ أَدُومَ فَضَاقَتْ نَفْسُ الشَّعْبِ فِي الطَّرِيقِ.
\par 5 وَتَكَلمَ الشَّعْبُ عَلى اللهِ وَعَلى مُوسَى قَائِلِينَ: «لِمَاذَا أَصْعَدْتُمَانَا مِنْ مِصْرَ لِنَمُوتَ فِي البَرِّيَّةِ! لأَنَّهُ لا خُبْزَ وَلا مَاءَ وَقَدْ كَرِهَتْ أَنْفُسُنَا الطَّعَامَ السَّخِيفَ».
\par 6 فَأَرْسَل الرَّبُّ عَلى الشَّعْبِ الحَيَّاتِ المُحْرِقَةَ فَلدَغَتِ الشَّعْبَ فَمَاتَ قَوْمٌ كَثِيرُونَ مِنْ إِسْرَائِيل.
\par 7 فَأَتَى الشَّعْبُ إِلى مُوسَى وَقَالُوا: «قَدْ أَخْطَأْنَا إِذْ تَكَلمْنَا عَلى الرَّبِّ وَعَليْكَ فَصَلِّ إِلى الرَّبِّ لِيَرْفَعَ عَنَّا الحَيَّاتِ». فَصَلى مُوسَى لأَجْلِ الشَّعْبِ.
\par 8 فَقَال الرَّبُّ لِمُوسَى: «اصْنَعْ لكَ حَيَّةً مُحْرِقَةً وَضَعْهَا عَلى رَايَةٍ فَكُلُّ مَنْ لُدِغَ وَنَظَرَ إِليْهَا يَحْيَا».
\par 9 فَصَنَعَ مُوسَى حَيَّةً مِنْ نُحَاسٍ وَوَضَعَهَا عَلى الرَّايَةِ فَكَانَ مَتَى لدَغَتْ حَيَّةٌ إِنْسَاناً وَنَظَرَ إِلى حَيَّةِ النُّحَاسِ يَحْيَا.
\par 10 وَارْتَحَل بَنُو إِسْرَائِيل وَنَزَلُوا فِي أُوبُوتَ.
\par 11 وَارْتَحَلُوا مِنْ أُوبُوتَ وَنَزَلُوا فِي عَيِّي عَبَارِيمَ فِي البَرِّيَّةِ التِي قُبَالةَ مُوآبَ إِلى شُرُوقِ الشَّمْسِ.
\par 12 مِنْ هُنَاكَ ارْتَحَلُوا وَنَزَلُوا فِي وَادِي زَارَدَ.
\par 13 مِنْ هُنَاكَ ارْتَحَلُوا وَنَزَلُوا فِي عَبْرِ أَرْنُونَ الذِي فِي البَرِّيَّةِ خَارِجاً عَنْ تُخُمِ الأَمُورِيِّينَ. لأَنَّ أَرْنُونَ هُوَ تُخُمُ مُوآبَ بَيْنَ مُوآبَ وَالأَمُورِيِّينَ.
\par 14 لِذَلِكَ يُقَالُ فِي كِتَابِ «حُرُوبِ الرَّبِّ»: «وَاهِبٌ فِي سُوفَةَ وَأَوْدِيَةِ أَرْنُونَ
\par 15 وَمَصَبِّ الأَوْدِيَةِ الذِي مَال إِلى مَسْكَنِ عَارَ وَاسْتَنَدَ إِلى تُخُمِ مُوآبَ».
\par 16 وَمِنْ هُنَاكَ إِلى بِئْرٍ. وَهِيَ البِئْرُ حَيْثُ قَال الرَّبُّ لِمُوسَى: «اجْمَعِ الشَّعْبَ فَأُعْطِيَهُمْ مَاءً».
\par 17 حِينَئِذٍ تَرَنَّمَ إِسْرَائِيلُ بِهَذَا النَّشِيدِ: «اِصْعَدِي أَيَّتُهَا البِئْرُ! أَجِيبُوا لهَا.
\par 18 بِئْرٌ حَفَرَهَا رُؤَسَاءُ حَفَرَهَا شُرَفَاءُ الشَّعْبِ بِصَوْلجَانٍ بِعِصِيِّهِمْ». وَمِنَ البَرِّيَّةِ إِلى مَتَّانَةَ
\par 19 وَمِنْ مَتَّانَةَ إِلى نَحْلِيئِيل وَمِنْ نَحْلِيئِيل إِلى بَامُوتَ
\par 20 وَمِنْ بَامُوتَ إِلى الجِوَاءِ التِي فِي صَحْرَاءِ مُوآبَ عِنْدَ رَأْسِ الفِسْجَةِ التِي تُشْرِفُ عَلى وَجْهِ البَرِّيَّةِ.
\par 21 وَأَرْسَل إِسْرَائِيلُ رُسُلاً إِلى سِيحُونَ مَلِكِ الأَمُورِيِّينَ قَائِلاً:
\par 22 «دَعْنِي أَمُرَّ فِي أَرْضِكَ. لا نَمِيلُ إِلى حَقْلٍ وَلا إِلى كَرْمٍ وَلا نَشْرَبُ مَاءَ بِئْرٍ. فِي طَرِيقِ المَلِكِ نَمْشِي حَتَّى نَتَجَاوَزَ تُخُومَكَ».
\par 23 فَلمْ يَسْمَحْ سِيحُونُ لِإِسْرَائِيل بِالمُرُورِ فِي تُخُومِهِ بَل جَمَعَ سِيحُونُ جَمِيعَ قَوْمِهِ وَخَرَجَ لِلِقَاءِ إِسْرَائِيل إِلى البَرِّيَّةِ فَأَتَى إِلى يَاهَصَ وَحَارَبَ إِسْرَائِيل.
\par 24 فَضَرَبَهُ إِسْرَائِيلُ بِحَدِّ السَّيْفِ وَمَلكَ أَرْضَهُ مِنْ أَرْنُونَ إِلى يَبُّوقَ إِلى بَنِي عَمُّونَ. لأَنَّ تُخُمَ بَنِي عَمُّونَ كَانَ قَوِيّاً.
\par 25 فَأَخَذَ إِسْرَائِيلُ كُل هَذِهِ المُدُنِ وَأَقَامَ إِسْرَائِيلُ فِي جَمِيعِ مُدُنِ الأَمُورِيِّينَ فِي حَشْبُونَ وَفِي كُلِّ قُرَاهَا.
\par 26 لأَنَّ حَشْبُونَ كَانَتْ مَدِينَةَ سِيحُونَ مَلِكِ الأَمُورِيِّينَ وَكَانَ قَدْ حَارَبَ مَلِكَ مُوآبَ الأَوَّل وَأَخَذَ كُل أَرْضِهِ مِنْ يَدِهِ حَتَّى أَرْنُونَ.
\par 27 لِذَلِكَ يَقُولُ أَصْحَابُ الأَمْثَالِ: «اِيتُوا إِلى حَشْبُونَ فَتُبْنَى وَتُصْلحَ مَدِينَةُ سِيحُونَ.
\par 28 لأَنَّ نَاراً خَرَجَتْ مِنْ حَشْبُونَ. لهِيباً مِنْ قَرْيَةِ سِيحُونَ. أَكَلتْ عَارَ مُوآبَ. أَهْل مُرْتَفَعَاتِ أَرْنُونَ.
\par 29 وَيْلٌ لكَ يَا مُوآبُ. هَلكْتِ يَا أُمَّةَ كَمُوشَ. قَدْ صَيَّرَ بَنِيهِ هَارِبِينَ وَبَنَاتِهِ فِي السَّبْيِ لِمَلِكِ الأَمُورِيِّينَ سِيحُونَ.
\par 30 لكِنْ قَدْ رَمَيْنَاهُمْ. هَلكَتْ حَشْبُونُ إِلى دِيبُونَ. وَأَخْرَبْنَا إِلى نُوفَحَ التِي إِلى مَيْدَبَا».
\par 31 فَأَقَامَ إِسْرَائِيلُ فِي أَرْضِ الأَمُورِيِّينَ.
\par 32 وَأَرْسَل مُوسَى لِيَتَجَسَّسَ يَعْزِيرَ فَأَخَذُوا قُرَاهَا وَطَرَدُوا الأَمُورِيِّينَ الذِينَ هُنَاكَ.
\par 33 ثُمَّ تَحَوَّلُوا وَصَعِدُوا فِي طَرِيقِ بَاشَانَ. فَخَرَجَ عُوجُ مَلِكُ بَاشَانَ لِلِقَائِهِمْ هُوَ وَجَمِيعُ قَوْمِهِ إِلى الحَرْبِ فِي إِذْرَعِي.
\par 34 فَقَال الرَّبُّ لِمُوسَى: «لا تَخَفْ مِنْهُ لأَنِّي قَدْ دَفَعْتُهُ إِلى يَدِكَ مَعَ جَمِيعِ قَوْمِهِ وَأَرْضِهِ فَتَفْعَلُ بِهِ كَمَا فَعَلتَ بِسِيحُونَ مَلِكِ الأَمُورِيِّينَ السَّاكِنِ فِي حَشْبُونَ».
\par 35 فَضَرَبُوهُ وَبَنِيهِ وَجَمِيعَ قَوْمِهِ حَتَّى لمْ يَبْقَ لهُ شَارِدٌ وَمَلكُوا أَرْضَهُ.

\chapter{22}

\par 1 وَارْتَحَل بَنُو إِسْرَائِيل وَنَزَلُوا فِي عَرَبَاتِ مُوآبَ مِنْ عَبْرِ أُرْدُنِّ أَرِيحَا.
\par 2 وَلمَّا رَأَى بَالاقُ بْنُ صِفُّورَ جَمِيعَ مَا فَعَل إِسْرَائِيلُ بِالأَمُورِيِّينَ
\par 3 فَزَِعَ مُوآبُ مِنَ الشَّعْبِ جِدّاً لأَنَّهُ كَثِيرٌ وَضَجَِرَ مُوآبُ مِنْ قِبَل بَنِي إِسْرَائِيل.
\par 4 فَقَال مُوآبُ لِشُيُوخِ مِدْيَانَ: «الآنَ يَلحَسُ الجُمْهُورُ كُل مَا حَوْلنَا كَمَا يَلحَسُ الثَّوْرُ خُضْرَةَ الحَقْلِ». وَكَانَ بَالاقُ بْنُ صِفُّورَ مَلِكاً لِمُوآبَ فِي ذَلِكَ الزَّمَانِ.
\par 5 فَأَرْسَل رُسُلاً إِلى بَلعَامَ بْنِ بَعُورَ إِلى فَتُورَ التِي عَلى النَّهْرِ فِي أَرْضِ بَنِي شَعْبِهِ لِيَدْعُوَهُ قَائِلاً: «هُوَذَا شَعْبٌ قَدْ خَرَجَ مِنْ مِصْرَ. هُوَذَا قَدْ غَشَّى وَجْهَ الأَرْضِ وَهُوَ مُقِيمٌ مُقَابَِلِي.
\par 6 فَالآنَ تَعَال وَالعَنْ لِي هَذَا الشَّعْبَ لأَنَّهُ أَعْظَمُ مِنِّي. لعَلهُ يُمْكِنُنَا أَنْ نَكْسِرَهُ فَأَطْرُدَهُ مِنَ الأَرْضِ. لأَنِّي عَرَفْتُ أَنَّ الذِي تُبَارِكُهُ مُبَارَكٌ وَالذِي تَلعَنُهُ مَلعُونٌ».
\par 7 فَانْطَلقَ شُيُوخُ مُوآبَ وَشُيُوخُ مِدْيَانَ وَحُلوَانُ العِرَافَةِ فِي أَيْدِيهِمْ وَأَتُوا إِلى بَلعَامَ وَكَلمُوهُ بِكَلامِ بَالاقَ.
\par 8 فَقَال لهُمْ: «بِيتُوا هُنَا الليْلةَ فَأَرُدَّ عَليْكُمْ جَوَاباً كَمَا يُكَلِّمُنِي الرَّبُّ». فَمَكَثَ رُؤَسَاءُ مُوآبَ عِنْدَ بَلعَامَ.
\par 9 فَأَتَى اللهُ إِلى بَلعَامَ وَقَال: «مَنْ هُمْ هَؤُلاءِ الرِّجَالُ الذِينَ عِنْدَكَ؟»
\par 10 فَقَال بَلعَامُ لِلهِ: «بَالاقُ بْنُ صِفُّورَ مَلِكُ مُوآبَ قَدْ أَرْسَل إِليَّ يَقُولُ:
\par 11 هُوَذَا الشَّعْبُ الخَارِجُ مِنْ مِصْرَ قَدْ غَشَّى وَجْهَ الأَرْضِ. تَعَال الآنَ العَنْ لِي إِيَّاهُ لعَلِّي أَقْدِرُ أَنْ أُحَارِبَهُ وَأَطْرُدَهُ».
\par 12 فَقَال اللهُ لِبَلعَامَ: «لا تَذْهَبْ مَعَهُمْ وَلا تَلعَنِ الشَّعْبَ لأَنَّهُ مُبَارَكٌ».
\par 13 فَقَامَ بَلعَامُ صَبَاحاً وَقَال لِرُؤَسَاءِ بَالاقَ: «انْطَلِقُوا إِلى أَرْضِكُمْ لأَنَّ الرَّبَّ أَبَى أَنْ يَسْمَحَ لِي بِالذَّهَابِ مَعَكُمْ».
\par 14 فَقَامَ رُؤَسَاءُ مُوآبَ وَأَتُوا إِلى بَالاقَ وَقَالُوا: «أَبَى بَلعَامُ أَنْ يَأْتِيَ مَعَنَا».
\par 15 فَعَادَ بَالاقُ وَأَرْسَل أَيْضاً رُؤَسَاءَ أَكْثَرَ وَأَعْظَمَ مِنْ أُولئِكَ.
\par 16 فَأَتُوا إِلى بَلعَامَ وَقَالُوا لهُ: «هَكَذَا قَال بَالاقُ بْنُ صِفُّورَ: لا تَمْتَنِعْ مِنَ الإِتْيَانِ إِليَّ
\par 17 لأَنِّي أُكْرِمُكَ إِكْرَاماً عَظِيماً وَكُل مَا تَقُولُ لِي أَفْعَلُهُ. فَتَعَال الآنَ العَنْ لِي هَذَا الشَّعْبَ».
\par 18 فَأَجَابَ بَلعَامُ عَبِيدَ بَالاقَ: «وَلوْ أَعْطَانِي بَالاقُ مِلءَ بَيْتِهِ فِضَّةً وَذَهَباً لا أَقْدِرُ أَنْ أَتَجَاوَزَ قَوْل الرَّبِّ إِلهِي لأَعْمَل صَغِيراً أَوْ كَبِيراً.
\par 19 فَالآنَ امْكُثُوا هُنَا أَنْتُمْ أَيْضاً هَذِهِ الليْلةَ لأَعْلمَ مَاذَا يَعُودُ الرَّبُّ يُكَلِّمُنِي بِهِ».
\par 20 فَأَتَى اللهُ إِلى بَلعَامَ ليْلاً وَقَال لهُ: «إِنْ أَتَى الرِّجَالُ لِيَدْعُوكَ فَقُمِ اذْهَبْ مَعَهُمْ. إِنَّمَا تَعْمَلُ الأَمْرَ الذِي أُكَلِّمُكَ بِهِ فَقَطْ».
\par 21 فَقَامَ بَلعَامُ صَبَاحاً وَشَدَّ عَلى أَتَانِهِ وَانْطَلقَ مَعَ رُؤَسَاءِ مُوآبَ.
\par 22 فَحَمِيَ غَضَبُ اللهِ لأَنَّهُ مُنْطَلِقٌ وَوَقَفَ مَلاكُ الرَّبِّ فِي الطَّرِيقِ لِيُقَاوِمَهُ وَهُوَ رَاكِبٌ عَلى أَتَانِهِ وَغُلامَاهُ مَعَهُ.
\par 23 فَأَبْصَرَتِ الأَتَانُ مَلاكَ الرَّبِّ وَاقِفاً فِي الطَّرِيقِ وَسَيْفُهُ مَسْلُولٌ فِي يَدِهِ فَمَالتِ الأَتَانُ عَنِ الطَّرِيقِ وَمَشَتْ فِي الحَقْلِ. فَضَرَبَ بَلعَامُ الأَتَانَ لِيَرُدَّهَا إِلى الطَّرِيقِ.
\par 24 ثُمَّ وَقَفَ مَلاكُ الرَّبِّ فِي خَنْدَقٍ لِلكُرُومِ لهُ حَائِطٌ مِنْ هُنَا وَحَائِطٌ مِنْ هُنَاكَ.
\par 25 فَلمَّا أَبْصَرَتِ الأَتَانُ مَلاكَ الرَّبِّ زَحَمَتِ الحَائِطَ وَضَغَطَتْ رِجْل بَلعَامَ بِالحَائِطِ فَضَرَبَهَا أَيْضاً.
\par 26 ثُمَّ اجْتَازَ مَلاكُ الرَّبِّ أَيْضاً وَوَقَفَ فِي مَكَانٍ ضَيِّقٍ حَيْثُ ليْسَ سَبِيلٌ لِلنُّكُوبِ يَمِيناً أَوْ شِمَالاً.
\par 27 فَلمَّا أَبْصَرَتِ الأَتَانُ مَلاكَ الرَّبِّ رَبَضَتْ تَحْتَ بَلعَامَ. فَحَمِيَ غَضَبُ بَلعَامَ وَضَرَبَ الأَتَانَ بِالقَضِيبِ.
\par 28 فَفَتَحَ الرَّبُّ فَمَ الأَتَانِ فَقَالتْ لِبَلعَامَ: «مَاذَا صَنَعْتُ بِكَ حَتَّى ضَرَبْتَنِي الآنَ ثَلاثَ دَفَعَاتٍ؟»
\par 29 فَقَال بَلعَامُ لِلأَتَانِ: «لأَنَّكِ ازْدَرَيْتِ بِي. لوْ كَانَ فِي يَدِي سَيْفٌ لكُنْتُ الآنَ قَدْ قَتَلتُكِ».
\par 30 فَقَالتِ الأَتَانُ لِبَلعَامَ: «أَلسْتُ أَنَا أَتَانَكَ التِي رَكِبْتَ عَليْهَا مُنْذُ وُجُودِكَ إِلى هَذَا اليَوْمِ؟ هَل تَعَوَّدْتُ أَنْ أَفْعَل بِكَ هَكَذَا؟» فَقَال: «لا».
\par 31 ثُمَّ كَشَفَ الرَّبُّ عَنْ عَيْنَيْ بَلعَامَ فَأَبْصَرَ مَلاكَ الرَّبِّ وَاقِفاً فِي الطَّرِيقِ وَسَيْفُهُ مَسْلُولٌ فِي يَدِهِ فَخَرَّ سَاجِداً عَلى وَجْهِهِ.
\par 32 فَقَال لهُ مَلاكُ الرَّبِّ: «لِمَاذَا ضَرَبْتَ أَتَانَكَ الآنَ ثَلاثَ دَفَعَاتٍ؟ هَئَنَذَا قَدْ خَرَجْتُ لِلمُقَاوَمَةِ لأَنَّ الطَّرِيقَ وَرْطَةٌ أَمَامِي
\par 33 فَأَبْصَرَتْنِي الأَتَانُ وَمَالتْ مِنْ قُدَّامِي الآنَ ثَلاثَ دَفَعَاتٍ. وَلوْ لمْ تَمِل مِنْ قُدَّامِي لكُنْتُ الآنَ قَدْ قَتَلتُكَ وَاسْتَبْقَيْتُهَا».
\par 34 فَقَال بَلعَامُ لِمَلاكِ الرَّبِّ: «أَخْطَأْتُ. إِنِّي لمْ أَعْلمْ أَنَّكَ وَاقِفٌ تِلقَائِي فِي الطَّرِيقِ. وَالآنَ إِنْ قَبُحَ فِي عَيْنَيْكَ فَإِنِّي أَرْجِعُ».
\par 35 فَقَال مَلاكُ الرَّبِّ لِبَلعَامَ: «اذْهَبْ مَعَ الرِّجَالِ وَإِنَّمَا تَتَكَلمُ بِالكَلامِ الذِي أُكَلِّمُكَ بِهِ فَقَطْ». فَانْطَلقَ بَلعَامُ مَعَ رُؤَسَاءِ بَالاقَ.
\par 36 فَلمَّا سَمِعَ بَالاقُ أَنَّ بَلعَامَ جَاءَ خَرَجَ لاِسْتِقْبَالِهِ إِلى مَدِينَةِ مُوآبَ التِي عَلى تُخُمِ أَرْنُونَ الذِي فِي أَقْصَى التُّخُومِ.
\par 37 فَقَال بَالاقُ لِبَلعَامَ: «أَلمْ أُرْسِل إِليْكَ لأَدْعُوَكَ؟ لِمَاذَا لمْ تَأْتِ إِليَّ؟ أَحَقّاً لا أَقْدِرُ أَنْ أُكْرِمَكَ؟»
\par 38 فَقَال بَلعَامُ لِبَالاقَ: «هَئَنَذَا قَدْ جِئْتُ إِليْكَ. أَلعَلِّي الآنَ أَسْتَطِيعُ أَنْ أَتَكَلمَ بِشَيْءٍ؟ الكَلامُ الذِي يَضَعُهُ اللهُ فِي فَمِي بِهِ أَتَكَلمُ».
\par 39 فَانْطَلقَ بَلعَامُ مَعَ بَالاقَ وَأَتَيَا إِلى قَرْيَةِ حَصُوتَ.
\par 40 فَذَبَحَ بَالاقُ بَقَراً وَغَنَماً وَأَرْسَل إِلى بَلعَامَ وَإِلى الرُّؤَسَاءِ الذِينَ مَعَهُ.
\par 41 وَفِي الصَّبَاحِ أَخَذَ بَالاقُ بَلعَامَ وَأَصْعَدَهُ إِلى مُرْتَفَعَاتِ بَعْلٍ فَرَأَى مِنْ هُنَاكَ أَقْصَى الشَّعْبِ.

\chapter{23}

\par 1 فَقَال بَلعَامُ لِبَالاقَ: «ابْنِ لِي هَهُنَا سَبْعَةَ مَذَابِحَ وَهَيِّئْ لِي هَهُنَا سَبْعَةَ ثِيرَانٍ وَسَبْعَةَ كِبَاشٍ».
\par 2 فَفَعَل بَالاقُ كَمَا تَكَلمَ بَلعَامُ. وَأَصْعَدَ بَالاقُ وَبَلعَامُ ثَوْراً وَكَبْشاً عَلى كُلِّ مَذْبَحٍ.
\par 3 فَقَال بَلعَامُ لِبَالاقَ: «قِفْ عِنْدَ مُحْرَقَتِكَ فَأَنْطَلِقَ أَنَا لعَل الرَّبَّ يُوافِي لِلِقَائِي فَمَهْمَا أَرَانِي أُخْبِرْكَ بِهِ». ثُمَّ انْطَلقَ إِلى رَابِيَةٍ.
\par 4 فَوَافَى اللهُ بَلعَامَ. فَقَال لهُ: «قَدْ رَتَّبْتُ سَبْعَةَ مَذَابِحَ وَأَصْعَدْتُ ثَوْراً وَكَبْشاً عَلى كُلِّ مَذْبَحٍ».
\par 5 فَوَضَعَ الرَّبُّ كَلاماً فِي فَمِ بَلعَامَ وَقَال: «ارْجِعْ إِلى بَالاقَ وَتَكَلمْ هَكَذَا».
\par 6 فَرَجَعَ إِليْهِ وَإِذَا هُوَ وَاقِفٌ عِنْدَ مُحْرَقَتِهِ هُوَ وَجَمِيعُ رُؤَسَاءِ مُوآبَ.
\par 7 فَنَطَقَ بِمَثَلِهِ وَقَال: «مِنْ أَرَامَ أَتَى بِي بَالاقُ مَلِكُ مُوآبَ مِنْ جِبَالِ المَشْرِقِ. تَعَال العَنْ لِي يَعْقُوبَ وَهَلُمَّ اشْتِمْ إِسْرَائِيل.
\par 8 كَيْفَ أَلعَنُ مَنْ لمْ يَلعَنْهُ اللهُ وَكَيْفَ أَشْتِمُ مَنْ لمْ يَشْتِمْهُ الرَّبُّ؟
\par 9 إِنِّي مِنْ رَأْسِ الصُّخُورِ أَرَاهُ. وَمِنَ الآكَامِ أُبْصِرُهُ. هُوَذَا شَعْبٌ يَسْكُنُ وَحْدَهُ. وَبَيْنَ الشُّعُوبِ لا يُحْسَبُ.
\par 10 مَنْ أَحْصَى تُرَابَ يَعْقُوبَ وَرُبْعَ إِسْرَائِيل بِعَدَدٍ؟ لِتَمُتْ نَفْسِي مَوْتَ الأَبْرَارِ وَلتَكُنْ آخِرَتِي كَآخِرَتِهِمْ».
\par 11 فَقَال بَالاقُ لِبَلعَامَ: «مَاذَا فَعَلتَ بِي؟ لِتَشْتِمَ أَعْدَائِي أَخَذْتُكَ وَهُوَذَا أَنْتَ قَدْ بَارَكْتَهُمْ».
\par 12 فَأَجَابَ: «أَمَا الذِي يَضَعُهُ الرَّبُّ فِي فَمِي أَحْتَرِصُ أَنْ أَتَكَلمَ بِهِ؟»
\par 13 فَقَال لهُ بَالاقُ: «هَلُمَّ مَعِي إِلى مَكَانٍ آخَرَ تَرَاهُ مِنْهُ. إِنَّمَا تَرَى أَقْصَاءَهُ فَقَطْ وَكُلهُ لا تَرَى. فَالعَنْهُ لِي مِنْ هُنَاكَ».
\par 14 فَأَخَذَهُ إِلى حَقْلِ صُوفِيمَ إِلى رَأْسِ الفِسْجَةِ وَبَنَى سَبْعَةَ مَذَابِحَ وَأَصْعَدَ ثَوْراً وَكَبْشاً عَلى كُلِّ مَذْبَحٍ.
\par 15 فَقَال لِبَالاقَ: «قِفْ هُنَا عِنْدَ مُحْرَقَتِكَ وَأَنَا أُوافِي هُنَاكَ».
\par 16 فَوَافَى الرَّبُّ بَلعَامَ وَوَضَعَ كَلاماً فِي فَمِهِ وَقَال: «ارْجِعْ إِلى بَالاقَ وَتَكَلمْ هَكَذَا».
\par 17 فَأَتَى إِليْهِ وَإِذَا هُوَ وَاقِفٌ عِنْدَ مُحْرَقَتِهِ وَرُؤَسَاءُ مُوآبَ مَعَهُ. فَسَأَلهُ بَالاقُ: «مَاذَا تَكَلمَ بِهِ الرَّبُّ؟»
\par 18 فَنَطَقَ بِمَثَلِهِ وَقَال: «قُمْ يَا بَالاقُ وَاسْمَعْ. اصْغَ إِليَّ يَا ابْنَ صِفُّورَ.
\par 19 ليْسَ اللهُ إِنْسَاناً فَيَكْذِبَ وَلا ابْنَ إِنْسَانٍ فَيَنْدَمَ. هَل يَقُولُ وَلا يَفْعَلُ؟ أَوْ يَتَكَلمُ وَلا يَفِي؟
\par 20 إِنِّي قَدْ أُمِرْتُ أَنْ أُبَارِكَ. فَإِنَّهُ قَدْ بَارَكَ فَلا أَرُدُّهُ.
\par 21 لمْ يُبْصِرْ إِثْماً فِي يَعْقُوبَ وَلا رَأَى سُوءاً فِي إِسْرَائِيل. الرَّبُّ إِلهُهُ مَعَهُ. وَهُتَافُ مَلِكٍ فِيهِ.
\par 22 اَللهُ أَخْرَجَهُ مِنْ مِصْرَ. لهُ مِثْلُ سُرْعَةِ الرِّئْمِ.
\par 23 إِنَّهُ ليْسَ عِيَافَةٌ عَلى يَعْقُوبَ وَلا عِرَافَةٌ عَلى إِسْرَائِيل. فِي الوَقْتِ يُقَالُ عَنْ يَعْقُوبَ وَعَنْ إِسْرَائِيل مَا فَعَل اللهُ.
\par 24 هُوَذَا شَعْبٌ يَقُومُ كَلبْوَةٍ وَيَرْتَفِعُ كَأَسَدٍ. لا يَنَامُ حَتَّى يَأْكُل فَرِيسَةً وَيَشْرَبَ دَمَ قَتْلى».
\par 25 فَقَال بَالاقُ لِبَلعَامَ: «لا تَلعَنْهُ لعْنَةً وَلا تُبَارِكْهُ بَرَكَةً».
\par 26 فَأَجَابَ بَلعَامُ: «أَلمْ أُكَلِّمْكَ قَائِلاً: كُلُّ مَا يَتَكَلمُ بِهِ الرَّبُّ فَإِيَّاهُ أَفْعَلُ؟»
\par 27 فَقَال بَالاقُ لِبَلعَامَ: «هَلُمَّ آخُذْكَ إِلى مَكَانٍ آخَرَ؟ عَسَى أَنْ يَصْلُحَ فِي عَيْنَيِ اللهِ أَنْ تَلعَنَهُ لِي مِنْ هُنَاكَ».
\par 28 فَأَخَذَ بَالاقُ بَلعَامَ إِلى رَأْسِ فَغُورَ المُشْرِفِ عَلى وَجْهِ البَرِّيَّةِ.
\par 29 فَقَال بَلعَامُ لِبَالاقَ: «ابْنِ لِي هَهُنَا سَبْعَةَ مَذَابِحَ وَهَيِّئْ لِي هَهُنَا سَبْعَةَ ثِيرَانٍ وَسَبْعَةَ كِبَاشٍ».
\par 30 فَفَعَل بَالاقُ كَمَا قَال بَلعَامُ وَأَصْعَدَ ثَوْراً وَكَبْشاً عَلى كُلِّ مَذْبَحٍ.

\chapter{24}

\par 1 فَلمَّا رَأَى بَلعَامُ أَنَّهُ يَحْسُنُ فِي عَيْنَيِ الرَّبِّ أَنْ يُبَارِكَ إِسْرَائِيل لمْ يَنْطَلِقْ كَالمَرَّةِ الأُولى وَالثَّانِيَةِ لِيُوافِيَ فَأْلاً بَل جَعَل نَحْوَ البَرِّيَّةِ وَجْهَهُ.
\par 2 وَرَفَعَ بَلعَامُ عَيْنَيْهِ وَرَأَى إِسْرَائِيل حَالاًّ حَسَبَ أَسْبَاطِهِ فَكَانَ عَليْهِ رُوحُ اللهِ
\par 3 فَنَطَقَ بِمَثَلِهِ وَقَال: «وَحْيُ بَلعَامَ بْنِ بَعُورَ. وَحْيُ الرَّجُلِ المَفْتُوحِ العَيْنَيْنِ.
\par 4 وَحْيُ الذِي يَسْمَعُ أَقْوَال اللهِ. الذِي يَرَى رُؤْيَا القَدِيرِ مَطْرُوحاً وَهُوَ مَكْشُوفُ العَيْنَيْنِ.
\par 5 مَا أَحْسَنَ خِيَامَكَ يَا يَعْقُوبُ مَسَاكِنَكَ يَا إِسْرَائِيلُ!
\par 6 كَأَوْدِيَةٍ مُمْتَدَّةٍ. كَجَنَّاتٍ عَلى نَهْرٍ. كَشَجَرَاتِ عُودٍ غَرَسَهَا الرَّبُّ. كَأَرْزَاتٍ عَلى مِيَاهٍ.
\par 7 يَجْرِي مَاءٌ مِنْ دِلائِهِ وَيَكُونُ زَرْعُهُ عَلى مِيَاهٍ غَزِيرَةٍ وَيَتَسَامَى مَلِكُهُ عَلى أَجَاجَ وَتَرْتَفِعُ مَمْلكَتُهُ.
\par 8 اَللهُ أَخْرَجَهُ مِنْ مِصْرَ. لهُ مِثْلُ سُرْعَةِ الرِّئْمِ. يَأْكُلُ أُمَماً مُضَايِقِيهِ. وَيَقْضِمُ عِظَامَهُمْ وَيُحَطِّمُ سِهَامَهُ.
\par 9 جَثَمَ كَأَسَدٍ. رَبَضَ كَلبْوَةٍ. مَنْ يُقِيمُهُ! مُبَارِكُكَ مُبَارَكٌ وَلاعِنُكَ مَلعُونٌ».
\par 10 فَاشْتَعَل غَضَبُ بَالاقَ عَلى بَلعَامَ وَصَفَّقَ بِيَدَيْهِ وَقَال بَالاقُ لِبَلعَامَ: «لِتَشْتِمَ أَعْدَائِي دَعَوْتُكَ وَهُوَذَا أَنْتَ قَدْ بَارَكْتَهُمُ الآنَ ثَلاثَ دَفَعَاتٍ.
\par 11 فَالآنَ اهْرُبْ إِلى مَكَانِكَ. قُلتُ أُكْرِمُكَ إِكْرَاماً وَهُوَذَا الرَّبُّ قَدْ مَنَعَكَ عَنِ الكَرَامَةِ».
\par 12 فَقَال بَلعَامُ لِبَالاقَ: «أَلمْ أُكَلِّمْ أَيْضاً رُسُلكَ الذِينَ أَرْسَلتَ إِليَّ قَائِلاً:
\par 13 وَلوْ أَعْطَانِي بَالاقُ مِلءَ بَيْتِهِ فِضَّةً وَذَهَباً لا أَقْدِرُ أَنْ أَتَجَاوَزَ قَوْل الرَّبِّ لأَعْمَل خَيْراً أَوْ شَرّاً مِنْ نَفْسِي. الذِي يَتَكَلمُهُ الرَّبُّ إِيَّاهُ أَتَكَلمُ.
\par 14 وَالآنَ هُوَذَا أَنَا مُنْطَلِقٌ إِلى شَعْبِي. هَلُمَّ أُنْبِئْكَ بِمَا يَفْعَلُهُ هَذَا الشَّعْبُ بِشَعْبِكَ فِي آخِرِ الأَيَّامِ».
\par 15 ثُمَّ نَطَقَ بِمَثَلِهِ وَقَال: «وَحْيُ بَلعَامَ بْنِ بَعُورَ. وَحْيُ الرَّجُلِ المَفْتُوحِ العَيْنَيْنِ.
\par 16 وَحْيُ الذِي يَسْمَعُ أَقْوَال اللهِ وَيَعْرِفُ مَعْرِفَةَ العَلِيِّ. الذِي يَرَى رُؤْيَا القَدِيرِ سَاقِطاً وَهُوَ مَكْشُوفُ العَيْنَيْنِ.
\par 17 أَرَاهُ وَلكِنْ ليْسَ الآنَ. أُبْصِرُهُ وَلكِنْ ليْسَ قَرِيباً. يَبْرُزُ كَوْكَبٌ مِنْ يَعْقُوبَ وَيَقُومُ قَضِيبٌ مِنْ إِسْرَائِيل فَيُحَطِّمُ طَرَفَيْ مُوآبَ وَيُهْلِكُ كُل بَنِي الوَغَى.
\par 18 وَيَكُونُ أَدُومُ مِيرَاثاً وَيَكُونُ سَعِيرُ أَعْدَاؤُهُ مِيرَاثاً. وَيَصْنَعُ إِسْرَائِيلُ بِبَأْسٍ.
\par 19 وَيَتَسَلطُ الذِي مِنْ يَعْقُوبَ وَيَهْلِكُ الشَّارِدُ مِنْ مَدِينَةٍ».
\par 20 ثُمَّ رَأَى عَمَالِيقَ فَنَطَقَ بِمَثَلِهِ وَقَال: «عَمَالِيقُ أَوَّلُ الشُّعُوبِ وَأَمَّا آخِرَتُهُ فَإِلى الهَلاكِ».
\par 21 ثُمَّ رَأَى القِينِيَّ فَنَطَقَ بِمَثَلِهِ وَقَال: «لِيَكُنْ مَسْكَنُكَ مَتِيناً وَعُشُّكَ مَوْضُوعاً فِي صَخْرَةٍ.
\par 22 لكِنْ يَكُونُ قَايِنُ لِلدَّمَارِ. حَتَّى مَتَى يَسْتَأْسِرُكَ أَشُّورُ؟»
\par 23 ثُمَّ نَطَقَ بِمَثَلِهِ وَقَال: «آهِْ! مَنْ يَعِيشُ حِينَ يَفْعَلُ ذَلِكَ.
\par 24 وَتَأْتِي سُفُنٌ مِنْ نَاحِيَةِ كِتِّيمَ وَتُخْضِعُ أَشُّورَ وَتُخْضِعُ عَابِرَ فَهُوَ أَيْضاً إِلى الهَلاكِ».
\par 25 ثُمَّ قَامَ بَلعَامُ وَانْطَلقَ وَرَجَعَ إِلى مَكَانِهِ. وَبَالاقُ أَيْضاً ذَهَبَ فِي طَرِيقِهِ.

\chapter{25}

\par 1 وَأَقَامَ إِسْرَائِيلُ فِي شِطِّيمَ وَابْتَدَأَ الشَّعْبُ يَزْنُونَ مَعَ بَنَاتِ مُوآبَ.
\par 2 فَدَعَوْنَ الشَّعْبَ إِلى ذَبَائِحِ آلِهَتِهِنَّ فَأَكَل الشَّعْبُ وَسَجَدُوا لِآلِهَتِهِنَّ.
\par 3 وَتَعَلقَ إِسْرَائِيلُ بِبَعْلِ فَغُورَ. فَحَمِيَ غَضَبُ الرَّبِّ عَلى إِسْرَائِيل.
\par 4 فَقَال الرَّبُّ لِمُوسَى: «خُذْ جَمِيعَ رُؤُوسِ الشَّعْبِ وَعَلِّقْهُمْ لِلرَّبِّ مُقَابِل الشَّمْسِ فَيَرْتَدَّ حُمُوُّ غَضَبِ الرَّبِّ عَنْ إِسْرَائِيل».
\par 5 فَقَال مُوسَى لِقُضَاةِ إِسْرَائِيل: «اقْتُلُوا كُلُّ وَاحِدٍ قَوْمَهُ المُتَعَلِّقِينَ بِبَعْلِ فَغُورَ».
\par 6 وَإِذَا رَجُلٌ مِنْ بَنِي إِسْرَائِيل جَاءَ وَقَدَّمَ إِلى إِخْوَتِهِ المِدْيَانِيَّةَ أَمَامَ عَيْنَيْ مُوسَى وَأَعْيُنِ كُلِّ جَمَاعَةِ بَنِي إِسْرَائِيل وَهُمْ بَاكُونَ لدَى بَابِ خَيْمَةِ الاِجْتِمَاعِ.
\par 7 فَلمَّا رَأَى ذَلِكَ فِينَحَاسُ بْنُ أَلِعَازَارَ بْنِ هَارُونَ الكَاهِنُِ قَامَ مِنْ وَسَطِ الجَمَاعَةِ وَأَخَذَ رُمْحاً بِيَدِهِ
\par 8 وَدَخَل وَرَاءَ الرَّجُلِ الإِسْرَائِيلِيِّ إِلى القُبَّةِ وَطَعَنَ كِليْهِمَا الرَّجُل الإِسْرَائِيلِيَّ وَالمَرْأَةَ فِي بَطْنِهَا. فَامْتَنَعَ الوَبَأُ عَنْ بَنِي إِسْرَائِيل.
\par 9 وَكَانَ الذِينَ مَاتُوا بِالوَبَإِ أَرْبَعَةً وَعِشْرِينَ أَلفاً.
\par 10 فَقَال الرَّبُّ لِمُوسَى:
\par 11 «فِينَحَاسُ بْنُ أَلِعَازَارَ بْنِ هَارُونَ الكَاهِنُِ قَدْ رَدَّ سَخَطِي عَنْ بَنِي إِسْرَائِيل بِكَوْنِهِ غَارَ غَيْرَتِي فِي وَسَطِهِمْ حَتَّى لمْ أُفْنِ بَنِي إِسْرَائِيل بِغَيْرَتِي.
\par 12 لِذَلِكَ قُل هَئَنَذَا أُعْطِيهِ مِيثَاقِي مِيثَاقَ السَّلامِ
\par 13 فَيَكُونُ لهُ وَلِنَسْلِهِ مِنْ بَعْدِهِ مِيثَاقَ كَهَنُوتٍ أَبَدِيٍّ لأَجْلِ أَنَّهُ غَارَ لِلهِ وَكَفَّرَ عَنْ بَنِي إِسْرَائِيل».
\par 14 وَكَانَ اسْمُ الرَّجُلِ الإِسْرَائِيلِيِّ الذِي قُتِل مَعَ المِدْيَانِيَّةِ زِمْرِيَ بْنَ سَالُو رَئِيسَ بَيْتِ أَبٍ مِنَ الشَّمْعُونِيِّينَ.
\par 15 وَاسْمُ المَرْأَةِ المِدْيَانِيَّةِ المَقْتُولةِ كُزْبِيَ بِنْتَ صُورٍ. هُوَ رَئِيسُ قَبَائِلِ بَيْتِ أَبٍ فِي مِدْيَانَ.
\par 16 ثُمَّ قَال الرَّبُّ لِمُوسَى:
\par 17 «ضَايِقُوا المِدْيَانِيِّينَ وَاضْرِبُوهُمْ
\par 18 لأَنَّهُمْ ضَايَقُوكُمْ بِمَكَايِدِهِمِ التِي كَادُوكُمْ بِهَا فِي أَمْرِ فَغُورَ وَأَمْرِ كُزْبِي أُخْتِهِمْ بِنْتِ رَئِيسٍ لِمِدْيَانَ التِي قُتِلتْ يَوْمَ الوَبَإِ بِسَبَبِ فَغُورَ».

\chapter{26}

\par 1 ثُمَّ بَعْدَ الوَبَإِ أَمَرَ الرَّبُّ مُوسَى وَأَلِعَازَارَ بْنَ هَارُونَ الكَاهِنِ:
\par 2 «خُذَا عَدَدَ كُلِّ جَمَاعَةِ بَنِي إِسْرَائِيل مِنِ ابْنِ عِشْرِينَ سَنَةً فَصَاعِداً حَسَبَ بُيُوتِ آبَائِهِمْ كُلِّ خَارِجٍ لِلجُنْدِ فِي إِسْرَائِيل».
\par 3 فَقَال مُوسَى وَأَلِعَازَارُ الكَاهِنُ فِي عَرَبَاتِ مُوآبَ عَلى أُرْدُنِّ أَرِيحَا:
\par 4 «مِنِ ابْنِ عِشْرِينَ سَنَةً فَصَاعِداً. كَمَا أَمَرَ الرَّبُّ مُوسَى وَبَنِي إِسْرَائِيل الخَارِجِينَ مِنْ أَرْضِ مِصْرَ:
\par 5 رَأُوبَيْنُ بِكْرُ إِسْرَائِيل: بَنُو رَأُوبَيْنَ. لِحَنُوكَ عَشِيرَةُ الحَنُوكِيِّينَ. لِفَلُّو عَشِيرَةُ الفَلُّوِيِّينَ.
\par 6 لِحَصْرُونَ عَشِيرَةُ الحَصْرُونِيِّينَ. لِكَرْمِي عَشِيرَةُ الكَرْمِيِّينَ.
\par 7 هَذِهِ عَشَائِرُ الرَّأُوبَيْنِيِّينَ. وَكَانَ المَعْدُودُونَ مِنْهُمْ ثَلاثَةً وَأَرْبَعِينَ أَلفاً وَسَبْعَ مِئَةٍ وَثلاثِينَ.
\par 8 وَابْنُ فَلُّو أَلِيآبُ.
\par 9 وَبَنُو أَلِيآبَ نَمُوئِيلُ وَدَاثَانُ وَأَبِيرَامُ وَهُمَا دَاثَانُ وَأَبِيرَامُ المَدْعُوَّانِ مِنَ الجَمَاعَةِ اللذَانِ خَاصَمَا مُوسَى وَهَارُونَ فِي جَمَاعَةِ قُورَحَ حِينَ خَاصَمُوا الرَّبَّ
\par 10 فَفَتَحَتِ الأَرْضُ فَاهَا وَابْتَلعَتْهُمَا مَعَ قُورَحَ حِينَ مَاتَ القَوْمُ بِإِحْرَاقِ النَّارِ مِئَتَيْنِ وَخَمْسِينَ رَجُلاً. فَصَارُوا عِبْرَةً.
\par 11 وَأَمَّا بَنُو قُورَحَ فَلمْ يَمُوتُوا.
\par 12 بَنُو شَمْعُونَ حَسَبَ عَشَائِرِهِمْ: لِنَمُوئِيل عَشِيرَةُ النَّمُوئِيلِيِّينَ. لِيَامِينَ عَشِيرَةُ اليَامِينِيِّينَ. لِيَاكِينَ عَشِيرَةُ اليَاكِينِيِّينَ.
\par 13 لِزَارَحَ عَشِيرَةُ الزَّارَحِيِّينَ. لِشَأُول عَشِيرَةُ الشَّأُولِيِّينَ.
\par 14 هَذِهِ عَشَائِرُ الشَّمْعُونِيِّينَ اثْنَانِ وَعِشْرُونَ أَلفاً وَمِئَتَانِ.
\par 15 بَنُو جَادَ حَسَبَ عَشَائِرِهِمْ: لِصِفُونَ عَشِيرَةُ الصِّفُونِيِّينَ. لِحَجِّي عَشِيرَةُ الحَجِّيِّينَ. لِشُونِي عَشِيرَةُ الشُّونِيِّينَ.
\par 16 لِأُزْنِي عَشِيرَةُ الأُزْنِيِّينَ. لِعِيرِي عَشِيرَةُ العِيرِيِّينَ
\par 17 لأَرُودَ عَشِيرَةُ الأَرُودِيِّينَ. لأَرْئِيلِي عَشِيرَةُ الأَرْئِيلِيِّينَ.
\par 18 هَذِهِ عَشَائِرُ بَنِي جَادَ حَسَبَ عَدَدِهِمْ أَرْبَعُونَ أَلفاً وَخَمْسُ مِئَةٍ.
\par 19 اِبْنَا يَهُوذَا عِيرُ وَأُونَانُ: وَمَاتَ عِيرُ وَأُونَانُ فِي أَرْضِ كَنْعَانَ.
\par 20 فَكَانَ بَنُو يَهُوذَا حَسَبَ عَشَائِرِهِمْ: لِشِيلةَ عَشِيرَةُ الشِّيلِيِّينَ. وَلِفَارَصَ عَشِيرَةُ الفَارَصِيِّينَ. وَلِزَارَحَ عَشِيرَةُ الزَّارَحِيِّينَ.
\par 21 وَكَانَ بَنُو فَارَصَ: لِحَصْرُونَ عَشِيرَةُ الحَصْرُونِيِّينَ. وَلِحَامُول عَشِيرَةُ الحَامُولِيِّينَ.
\par 22 هَذِهِ عَشَائِرُ يَهُوذَا حَسَبَ عَدَدِهِمْ سِتَّةٌ وَسَبْعُونَ أَلفاً وَخَمْسُ مِئَةٍ.
\par 23 بَنُو يَسَّاكَرَ حَسَبَ عَشَائِرِهِمْ: لِتُولاعَ عَشِيرَةُ التُّولاعِيِّينَ. وَلِفُوَّةَ عَشِيرَةُ الفُوِّيِّينَ.
\par 24 وَلِيَاشُوبَ عَشِيرَةُ اليَاشُوبِيِّينَ. وَلِشِمْرُونَ عَشِيرَةُ الشِّمْرُونِيِّينَ.
\par 25 هَذِهِ عَشَائِرُ يَسَّاكَرَ حَسَبَ عَدَدِهِمْ أَرْبَعَةٌ وَسِتُّونَ أَلفاً وَثَلاثُ مِئَةٍ.
\par 26 بَنُو زَبُولُونَ حَسَبَ عَشَائِرِهِمْ: لِسَارَدَ عَشِيرَةُ السَّارَدِيِّينَ. وَلِإِيلُونَ عَشِيرَةُ الإِيلُونِيِّينَ. وَلِيَاحِلئِيل عَشِيرَةُ اليَاحِلئِيلِيِّينَ.
\par 27 هَذِهِ عَشَائِرُ الزَّبُولُونِيِّينَ حَسَبَ عَدَدِهِمْ سِتُّونَ أَلفاً وَخَمْسُ مِئَةٍ.
\par 28 اِبْنَا يُوسُفَ حَسَبَ عَشَائِرِهِمَا مَنَسَّى وَأَفْرَايِمُ:
\par 29 بَنُو مَنَسَّى لِمَاكِيرَ عَشِيرَةُ المَاكِيرِيِّينَ. وَمَاكِيرُ وَلدَ جِلعَادَ. وَلِجِلعَادَ عَشِيرَةُ الجِلعَادِيِّينَ.
\par 30 هَؤُلاءِ بَنُو جِلعَادَ. لِإِيعَزَرَ عَشِيرَةُ الإِيعَزَرِيِّينَ. لِحَالقَ عَشِيرَةُ الحَالقِيِّينَ
\par 31 لأَسْرِيئِيل عَشِيرَةُ الأَسْرِيئِيلِيِّينَ. لِشَكَمَ عَشِيرَةُ الشَّكَمِيِّينَ
\par 32 لِشَمِيدَاعَ عَشِيرَةُ الشَّمِيدَاعِيِّينَ. لِحَافَرَ عَشِيرَةُ الحَافَرِيِّينَ.
\par 33 وَأَمَّا صَلُفْحَادُ بْنُ حَافَرَ فَلمْ يَكُنْ لهُ بَنُونَ بَل بَنَاتٌ. وَأَسْمَاءُ بَنَاتِ صَلُفْحَادَ مَحْلةُ وَنُوعَةُ وَحُجْلةُ وَمِلكَةُ وَتِرْصَةُ.
\par 34 هَذِهِ عَشَائِرُ مَنَسَّى. وَالمَعْدُودُونَ مِنْهُمُ اثْنَانِ وَخَمْسُونَ أَلفاً وَسَبْعُ مِئَةٍ.
\par 35 وَهَؤُلاءِ بَنُو أَفْرَايِمَ حَسَبَ عَشَائِرِهِمْ: لِشُوتَالحَ عَشِيرَةُ الشُّوتَالحِيِّينَ. لِبَاكَرَ عَشِيرَةُ البَاكَرِيِّينَ. لِتَاحَنَ عَشِيرَةُ التَّاحَنِيِّينَ.
\par 36 وَهَؤُلاءِ بَنُو شُوتَالحَ. لِعِيرَانَ عَشِيرَةُ العِيرَانِيِّينَ.
\par 37 هَذِهِ عَشَائِرُ بَنِي أَفْرَايِمَ حَسَبَ عَدَدِهِمِ اثْنَانِ وَثَلاثُونَ أَلفاً وَخَمْسُ مِئَةٍ. هَؤُلاءِ بَنُو يُوسُِفَ حَسَبَ عَشَائِرِهِمْ.
\par 38 بَنُو بِنْيَامِينَ حَسَبَ عَشَائِرِهِمْ. لِبَالعَ عَشِيرَةُ البَالعِيِّينَ. لأَشْبِيل عَشِيرَةُ الأَشْبِيلِيِّينَ. لأَحِيرَامَ عَشِيرَةُ الأَحِيرَامِيِّينَ.
\par 39 لِشَفُوفَامَ عَشِيرَةُ الشَّفُوفَامِيِّينَ. لِحُوفَامَ عَشِيرَةُ الحُوفَامِيِّينَ.
\par 40 وَكَانَ ابْنَا بَالعَ: أَرْدَ وَنُعْمَانَ. لأَرْدَ عَشِيرَةُ الأَرْدِيِّينَ وَلِنُعْمَانَ عَشِيرَةُ النُّعْمَانِيِّينَ.
\par 41 هَؤُلاءِ بَنُو بِنْيَامِينَ حَسَبَ عَشَائِرِهِمْ. وَالمَعْدُودُونَ مِنْهُمْ خَمْسَةٌ وَأَرْبَعُونَ أَلفاً وَسِتُّ مِئَةٍ.
\par 42 هَؤُلاءِ بَنُو دَانَ حَسَبَ عَشَائِرِهِمْ: لِشُوحَامَ عَشِيرَةُ الشُّوحَامِيِّينَ. هَذِهِ قَبَائِلُ دَانَ حَسَبَ عَشَائِرِهِمْ.
\par 43 جَمِيعُ عَشَائِرِ الشُّوحَامِيِّينَ حَسَبَ عَدَدِهِمْ أَرْبَعَةٌ وَسِتُّونَ أَلفاً وَأَرْبَعُ مِئَةٍ.
\par 44 بَنُو أَشِيرَ حَسَبَ عَشَائِرِهِمْ: لِيِمْنَةَ عَشِيرَةُ اليِمْنِيِّينَ. لِيِشْوِي عَشِيرَةُ اليِشْوِيِّينَ. لِبَرِيعَةَ عَشِيرَةُ البَرِيعِيِّينَ.
\par 45 لِبَنِي بَرِيعَةَ لِحَابَرَ عَشِيرَةُ الحَابَرِيِّينَ. لِمَلكِيئِيل عَشِيرَةُ المَلكِيئِيلِيِّينَ.
\par 46 وَاسْمُ ابْنَةِ أَشِيرَ سَارَحُ.
\par 47 هَذِهِ عَشَائِرُ بَنِي أَشِيرَ حَسَبَ عَدَدِهِمْ ثَلاثَةٌ وَخَمْسُونَ أَلفاً وَأَرْبَعُ مِئَةٍ.
\par 48 بَنُو نَفْتَالِي حَسَبَ عَشَائِرِهِمْ: لِيَاحَصْئِيل عَشِيرَةُ اليَاحَصْئِيلِيِّينَ. لِجُونِي عَشِيرَةُ الجُونِيِّينَ.
\par 49 لِيِصِرَ عَشِيرَةُ اليِصِرِيِّينَ. لِشِلِّيمَ عَشِيرَةُ الشِّلِّيمِيِّينَ.
\par 50 هَذِهِ قَبَائِلُ نَفْتَالِي حَسَبَ عَشَائِرِهِمْ. وَالمَعْدُودُونَ مِنْهُمْ خَمْسَةٌ وَأَرْبَعُونَ أَلفاً وَأَرْبَعُ مِئَةٍ.
\par 51 هَؤُلاءِ المَعْدُودُونَ مِنْ بَنِي إِسْرَائِيل سِتُّ مِئَةِ أَلفٍ وَأَلفٌ وَسَبْعُ مِئَةٍ وَثَلاثُونَ.
\par 52 ثُمَّ أَمَرَ الرَّبُّ مُوسَى:
\par 53 «لِهَؤُلاءِ تُقْسَمُ الأَرْضُ نَصِيباً عَلى عَدَدِ الأَسْمَاءِ.
\par 54 اَلكَثِيرُ تُكَثِّرُ لهُ نَصِيبَهُ وَالقَلِيلُ تُقَلِّلُ لهُ نَصِيبَهُ. كُلُّ وَاحِدٍ حَسَبَ المَعْدُودِينَ مِنْهُ يُعْطَى نَصِيبَهُ.
\par 55 إِنَّمَا بِالقُرْعَةِ تُقْسَمُ الأَرْضُ. حَسَبَ أَسْمَاءِ أَسْبَاطِ آبَائِهِمْ يَمْلِكُونَ.
\par 56 حَسَبَ القُرْعَةِ يُقْسَمُ نَصِيبُهُمْ بَيْنَ كَثِيرٍ وَقَلِيلٍ».
\par 57 وَهَؤُلاءِ المَعْدُودُونَ مِنَ اللاوِيِّينَ حَسَبَ عَشَائِرِهِمْ: لِجَرْشُونَ عَشِيرَةُ الجَرْشُونِيِّينَ. لِقَهَاتَ عَشِيرَةُ القَهَاتِيِّينَ. لِمَرَارِي عَشِيرَةُ المَرَارِيِّينَ.
\par 58 هَذِهِ عَشَائِرُ لاوِي. عَشِيرَةُ اللِّبْنِيِّينَ وَعَشِيرَةُ الحَبْرُونِيِّينَ وَعَشِيرَةُ المَحْلِيِّينَ وَعَشِيرَةُ المُوشِيِّينَ وَعَشِيرَةُ القُورَحِيِّينَ. وَأَمَّا قَهَاتُ فَوَلدَ عَمْرَامَ.
\par 59 وَاسْمُ امْرَأَةِ عَمْرَامَ يُوكَابَدُ بِنْتُ لاوِي التِي وُلِدَتْ لِلاوِي فِي مِصْرَ. فَوَلدَتْ لِعَمْرَامَ هَارُونَ وَمُوسَى وَمَرْيَمَ أُخْتَهُمَا.
\par 60 وَلِهَارُونَ وُلِدَ نَادَابُ وَأَبِيهُو وَأَلِعَازَارُ وَإِيثَامَارُ.
\par 61 وَأَمَّا نَادَابُ وَأَبِيهُو فَمَاتَا عِنْدَمَا قَرَّبَا نَاراً غَرِيبَةً أَمَامَ الرَّبِّ.
\par 62 وَكَانَ المَعْدُودُونَ مِنْهُمْ ثَلاثَةً وَعِشْرِينَ أَلفاً كُل ذَكَرٍ مِنِ ابْنِ شَهْرٍ فَصَاعِداً. لأَنَّهُمْ لمْ يُعَدُّوا بَيْنَ بَنِي إِسْرَائِيل إِذْ لمْ يُعْطَ لهُمْ نَصِيبٌ بَيْنَ بَنِي إِسْرَائِيل.
\par 63 هَؤُلاءِ هُمُ الذِينَ عَدَّهُمْ مُوسَى وَأَلِعَازَارُ الكَاهِنُ حِينَ عَدَّا بَنِي إِسْرَائِيل فِي عَرَبَاتِ مُوآبَ عَلى أُرْدُنِّ أَرِيحَا.
\par 64 وَفِي هَؤُلاءِ لمْ يَكُنْ إِنْسَانٌ مِنَ الذِينَ عَدَّهُمْ مُوسَى وَهَارُونُ الكَاهِنُ حِينَ عَدَّا بَنِي إِسْرَائِيل فِي بَرِّيَّةِ سِينَاءَ
\par 65 لأَنَّ الرَّبَّ قَال لهُمْ إِنَّهُمْ يَمُوتُونَ فِي البَرِّيَّةِ فَلمْ يَبْقَ مِنْهُمْ إِنْسَانٌ إِلا كَالِبُ بْنُ يَفُنَّةَ وَيَشُوعُ بْنُ نُونَ.

\chapter{27}

\par 1 فَتَقَدَّمَتْ بَنَاتُ صَلُفْحَادَ بْنِ حَافَرَ بْنِ جِلعَادَ بْنِ مَاكِيرَ بْنِ مَنَسَّى مِنْ عَشَائِرِ مَنَسَّى بْنِ يُوسُفَ. وَهَذِهِ أَسْمَاءُ بَنَاتِهِ: مَحْلةُ وَنُوعَةُ وَحُجْلةُ وَمِلكَةُ وَتِرْصَةُ.
\par 2 وَوَقَفْنَ أَمَامَ مُوسَى وَأَلِعَازَارَ الكَاهِنِ وَأَمَامَ الرُّؤَسَاءِ وَكُلِّ الجَمَاعَةِ لدَى بَابِ خَيْمَةِ الاِجْتِمَاعِ قَائِلاتٍ:
\par 3 أَبُونَا مَاتَ فِي البَرِّيَّةِ وَلمْ يَكُنْ فِي القَوْمِ الذِينَ اجْتَمَعُوا عَلى الرَّبِّ فِي جَمَاعَةِ قُورَحَ بَل بِخَطِيَّتِهِ مَاتَ وَلمْ يَكُنْ لهُ بَنُونَ.
\par 4 لِمَاذَا يُحْذَفُ اسْمُ أَبِينَا مِنْ بَيْنِ عَشِيرَتِهِ لأَنَّهُ ليْسَ لهُ ابْنٌ؟ أَعْطِنَا مُلكاً بَيْنَ أَعْمَامِنَا».
\par 5 فَقَدَّمَ مُوسَى دَعْوَاهُنَّ أَمَامَ الرَّبِّ.
\par 6 فَقَال الرَّبُّ لِمُوسَى:
\par 7 «بِحَقٍّ تَكَلمَتْ بَنَاتُ صَلُفْحَادَ فَتُعْطِيهِنَّ مُلكَ نَصِيبٍ بَيْنَ أَعْمَامِهِنَّ وَتَنْقُلُ نَصِيبَ أَبِيهِنَّ إِليْهِنَّ.
\par 8 وَتَقُول لِبَنِي إِسْرَائِيل: أَيُّمَا رَجُلٍ مَاتَ وَليْسَ لهُ ابْنٌ تَنْقُلُونَ مُلكَهُ إِلى ابْنَتِهِ.
\par 9 وَإِنْ لمْ تَكُنْ لهُ ابْنَةٌ تُعْطُوا مُلكَهُ لِإِخْوَتِهِ.
\par 10 وَإِنْ لمْ يَكُنْ لهُ إِخْوَةٌ تُعْطُوا مُلكَهُ لأَعْمَامِهِ.
\par 11 وَإِنْ لمْ يَكُنْ لأَبِيهِ إِخْوَةٌ تُعْطُوا مُلكَهُ لِنَسِيبِهِ الأَقْرَبِ إِليْهِ مِنْ عَشِيرَتِهِ فَيَرِثُهُ». فَصَارَتْ لِبَنِي إِسْرَائِيل فَرِيضَةَ قَضَاءٍ كَمَا أَمَرَ الرَّبُّ مُوسَى.
\par 12 وَقَال الرَّبُّ لِمُوسَى: «اصْعَدْ إِلى جَبَلِ عَبَارِيمَ هَذَا وَانْظُرِ الأَرْضَ التِي أَعْطَيْتُ بَنِي إِسْرَائِيل.
\par 13 وَمَتَى نَظَرْتَهَا تُضَمُّ إِلى قَوْمِكَ أَنْتَ أَيْضاً كَمَا ضُمَّ هَارُونُ أَخُوكَ.
\par 14 لأَنَّكُمَا فِي بَرِّيَّةِ صِينَ عِنْدَ مُخَاصَمَةِ الجَمَاعَةِ عَصَيْتُمَا قَوْلِي أَنْ تُقَدِّسَانِي بِالمَاءِ أَمَامَ أَعْيُنِهِمْ». (ذَلِكَ مَاءُ مَرِيبَةِ قَادِشَ فِي بَرِّيَّةِ صِينَ).
\par 15 فَقَال مُوسَى لِلرَّبِّ:
\par 16 «لِيُوَكِّلِ الرَّبُّ إِلهُ أَرْوَاحِ جَمِيعِ البَشَرِ رَجُلاً عَلى الجَمَاعَةِ
\par 17 يَخْرُجُ أَمَامَهُمْ وَيَدْخُلُ أَمَامَهُمْ وَيُخْرِجُهُمْ وَيُدْخِلُهُمْ لِكَيْلا تَكُونَ جَمَاعَةُ الرَّبِّ كَالغَنَمِ التِي لا رَاعِيَ لهَا».
\par 18 فَقَال الرَّبُّ لِمُوسَى: «خُذْ يَشُوعَ بْنَ نُونَ رَجُلاً فِيهِ رُوحٌ وَضَعْ يَدَكَ عَليْهِ
\par 19 وَأَوْقِفْهُ قُدَّامَ أَلِعَازَارَ الكَاهِنِ وَقُدَّامَ كُلِّ الجَمَاعَةِ وَأَوْصِهِ أَمَامَ أَعْيُنِهِمْ.
\par 20 وَاجْعَل مِنْ هَيْبَتِكَ عَليْهِ لِيَسْمَعَ لهُ كُلُّ جَمَاعَةِ بَنِي إِسْرَائِيل
\par 21 فَيَقِفَ أَمَامَ أَلِعَازَارَ الكَاهِنِ فَيَسْأَلُ لهُ بِقَضَاءِ الأُورِيمِ أَمَامَ الرَّبِّ. حَسَبَ قَوْلِهِ يَخْرُجُونَ وَحَسَبَ قَوْلِهِ يَدْخُلُونَ هُوَ وَكُلُّ بَنِي إِسْرَائِيل مَعَهُ كُلُّ الجَمَاعَةِ
\par 22 فَفَعَل مُوسَى كَمَا أَمَرَهُ الرَّبُّ. أَخَذَ يَشُوعَ وَأَوْقَفَهُ قُدَّامَ أَلِعَازَارَ الكَاهِنِ وَقُدَّامَ كُلِّ الجَمَاعَةِ
\par 23 وَوَضَعَ يَدَيْهِ عَليْهِ وَأَوْصَاهُ كَمَا تَكَلمَ الرَّبُّ عَنْ يَدِ مُوسَى.

\chapter{28}

\par 1 وَأَمَرَ الرَّبُّ مُوسَى:
\par 2 «أَوْصِ بَنِي إِسْرَائِيل: قُرْبَانِي طَعَامِي مَعَ وَقَائِدِي رَائِحَةُ سُرُورِي تَحْرَِصُونَ أَنْ تُقَرِّبُوهُ لِي فِي وَقْتِهِ.
\par 3 وَقُل لهُمْ. هَذَا هُوَ الوَقُودُ الذِي تُقَرِّبُونَ لِلرَّبِّ: خَرُوفَانِ حَوْلِيَّانِ صَحِيحَانِ لِكُلِّ يَوْمٍ مُحْرَقَةً دَائِمَةً.
\par 4 الخَرُوفُ الوَاحِدُ تَعْمَلُهُ صَبَاحاً وَالخَرُوفُ الثَّانِي تَعْمَلُهُ بَيْنَ العِشَاءَيْنِ.
\par 5 وَعُشْرَ الإِيفَةِ مِنْ دَقِيقٍ مَلتُوتٍ بِرُبْعِ الهِينِ مِنْ زَيْتِ الرَّبِّ تَقْدِمَةً.
\par 6 مُحْرَقَةٌ دَائِمَةٌ. هِيَ المَعْمُولةُ فِي جَبَلِ سِينَاءَ. لِرَائِحَةِ سُرُورٍ وَقُوداً لِلرَّبِّ.
\par 7 وَسَكِيبُهَا رُبْعُ الهِينِ لِلخَرُوفِ الوَاحِدِ. فِي القُدْسِ اسْكُبْ سَكِيبَ مُسْكِرٍ لِلرَّبِّ.
\par 8 وَالخَرُوفُ الثَّانِي تَعْمَلُهُ بَيْنَ العِشَاءَيْنِ كَتَقْدِمَةِ الصَّبَاحِ وَكَسَكِيبِهِ تَعْمَلُهُ وَقُودَ رَائِحَةِ سُرُورٍ لِلرَّبِّ.
\par 9 «وَفِي يَوْمِ السَّبْتِ خَرُوفَانِ حَوْلِيَّانِ صَحِيحَانِ وَعُشْرَانِ مِنْ دَقِيقٍ مَلتُوتٍ بِزَيْتٍ تَقْدِمَةً مَعَ سَكِيبِهِ
\par 10 مُحْرَقَةُ كُلِّ سَبْتٍ فَضْلاً عَنِ المُحْرَقَةِ الدَّائِمَةِ وَسَكِيبِهَا.
\par 11 «وَفِي رُؤُوسِ شُهُورِكُمْ تُقَرِّبُونَ مُحْرَقَةً لِلرَّبِّ: ثَوْرَيْنِ ابْنَيْ بَقَرٍ وَكَبْشاً وَاحِداً وَسَبْعَةَ خِرَافٍ حَوْلِيَّةٍ صَحِيحَةٍ
\par 12 وَثَلاثَةَ أَعْشَارٍ مِنْ دَقِيقٍ مَلتُوتٍ بِزَيْتٍ تَقْدِمَةً لِكُلِّ ثَوْرٍ. وَعُشْرَيْنِ مِنْ دَقِيقٍ مَلتُوتٍ بِزَيْتٍ تَقْدِمَةً لِلكَبْشِ الوَاحِدِ.
\par 13 وَعُشْراً وَاحِداً مِنْ دَقِيقٍ مَلتُوتٍ بِزَيْتٍ تَقْدِمَةً لِكُلِّ خَرُوفٍ. مُحْرَقَةً رَائِحَةَ سُرُورٍ وَقُوداً لِلرَّبِّ.
\par 14 وَسَكَائِبُهُنَّ تَكُونُ نِصْفَ الهِينِ لِلثَّوْرِ وَثُلثَ الهِينِ لِلكَبْشِ وَرُبْعَ الهِينِ لِلخَرُوفِ مِنْ خَمْرٍ. هَذِهِ مُحْرَقَةُ كُلِّ شَهْرٍ مِنْ أَشْهُرِ السَّنَةِ.
\par 15 وَتَيْساً وَاحِداً مِنَ المَعْزِ ذَبِيحَةَ خَطِيَّةٍ لِلرَّبِّ. فَضْلاً عَنِ المُحْرَقَةِ الدَّائِمَةِ يُقَرَّبُ مَعَ سَكِيبِهِ.
\par 16 «وَفِي الشَّهْرِ الأَوَّلِ فِي اليَوْمِ الرَّابِعَ عَشَرَ مِنَ الشَّهْرِ فِصْحٌ لِلرَّبِّ.
\par 17 وَفِي اليَوْمِ الخَامِسَ عَشَرَ مِنْ هَذَا الشَّهْرِ عِيدٌ. سَبْعَةَ أَيَّامٍ يُؤْكَلُ فَطِيرٌ.
\par 18 فِي اليَوْمِ الأَوَّلِ مَحْفَلٌ مُقَدَّسٌ. عَمَلاً مَا مِنَ الشُّغْلِ لا تَعْمَلُوا.
\par 19 وَتُقَرِّبُونَ وَقُوداً مُحْرَقَةً لِلرَّبِّ ثَوْرَيْنِ ابْنَيْ بَقَرٍ وَكَبْشاً وَاحِداً وَسَبْعَةَ خِرَافٍ حَوْلِيَّةٍ. صَحِيحَةً تَكُونُ لكُمْ.
\par 20 وَتَقْدِمَتُهُنَّ مِنْ دَقِيقٍ مَلتُوتٍ بِزَيْتٍ. ثَلاثَةَ أَعْشَارٍ تَعْمَلُونَ لِلثَّوْرِ وَعُشْرَيْنِ لِلكَبْشِ
\par 21 وَعُشْراً وَاحِداً تَعْمَلُ لِكُلِّ خَرُوفٍ مِنَ السَّبْعَةِ الخِرَافِ
\par 22 وَتَيْساً وَاحِداً ذَبِيحَةَ خَطِيَّةٍ لِلتَّكْفِيرِ عَنْكُمْ.
\par 23 فَضْلاً عَنْ مُحْرَقَةِ الصَّبَاحِ التِي لِمُحْرَقَةٍ دَائِمَةٍ تَعْمَلُونَ هَذِهِ.
\par 24 هَكَذَا تَعْمَلُونَ كُل يَوْمٍ سَبْعَةَ أَيَّامٍ طَعَامَ وَقُودِ رَائِحَةِ سُرُورٍ لِلرَّبِّ. فَضْلاً عَنِ المُحْرَقَةِ الدَّائِمَةِ يُعْمَلُ مَعَ سَكِيبِهِ.
\par 25 وَفِي اليَوْمِ السَّابِعِ يَكُونُ لكُمْ مَحْفَلٌ مُقَدَّسٌ. عَمَلاً مَا مِنَ الشُّغْلِ لا تَعْمَلُوا.
\par 26 «وَفِي يَوْمِ البَاكُورَةِ حِينَ تُقَرِّبُونَ تَقْدِمَةً جَدِيدَةً لِلرَّبِّ فِي أَسَابِيعِكُمْ يَكُونُ لكُمْ مَحْفَلٌ مُقَدَّسٌ. عَمَلاً مَا مِنَ الشُّغْلِ لا تَعْمَلُوا.
\par 27 وَتُقَرِّبُونَ مُحْرَقَةً لِرَائِحَةِ سُرُورٍ لِلرَّبِّ: ثَوْرَيْنِ ابْنَيْ بَقَرٍ وَكَبْشاً وَاحِداً وَسَبْعَةَ خِرَافٍ حَوْلِيَّةٍ.
\par 28 وَتَقْدِمَتُهُنَّ مِنْ دَقِيقٍ مَلتُوتٍ بِزَيْتٍ ثَلاثَةَ أَعْشَارٍ لِكُلِّ ثَوْرٍ وَعُشْرَيْنِ لِلكَبْشِ الوَاحِدِ
\par 29 وَعُشْراً وَاحِداً لِكُلِّ خَرُوفٍ مِنَ السَّبْعَةِ الخِرَافِ
\par 30 وَتَيْساً وَاحِداً مِنَ المَعْزِ لِلتَّكْفِيرِ عَنْكُمْ.
\par 31 فَضْلاً عَنِ المُحْرَقَةِ الدَّائِمَةِ وَتَقْدِمَتِهَا تَعْمَلُونَ. مَعَ سَكَائِبِهِنَّ صَحِيحَاتٍ تَكُونُ لكُمْ».

\chapter{29}

\par 1 «وَفِي الشَّهْرِ السَّابِعِ فِي الأَوَّلِ مِنَ الشَّهْرِ يَكُونُ لكُمْ مَحْفَلٌ مُقَدَّسٌ. عَمَلاً مَا مِنَ الشُّغْلِ لا تَعْمَلُوا. يَوْمَ هُتَافِ بُوقٍ يَكُونُ لكُمْ.
\par 2 وَتَعْمَلُونَ مُحْرَقَةً لِرَائِحَةِ سُرُورٍ لِلرَّبِّ ثَوْراً وَاحِداً ابْنَ بَقَرٍ وَكَبْشاً وَاحِداً وَسَبْعَةَ خِرَافٍ حَوْلِيَّةٍ صَحِيحَةٍ.
\par 3 وَتَقْدِمَتَهُنَّ مِنْ دَقِيقٍ مَلتُوتٍ بِزَيْتٍ ثَلاثَةَ أَعْشَارٍ لِلثَّوْرِ وَعُشْرَيْنِ لِلكَبْشِ.
\par 4 وَعُشْراً وَاحِداً لِكُلِّ خَرُوفٍ مِنَ السَّبْعَةِ الخِرَافِ.
\par 5 وَتَيْساً وَاحِداً مِنَ المَعْزِ ذَبِيحَةَ خَطِيَّةٍ لِلتَّكْفِيرِ عَنْكُمْ.
\par 6 فَضْلاً عَنْ مُحْرَقَةِ الشَّهْرِ وَتَقْدِمَتِهَا وَالمُحْرَقَةِ الدَّائِمَةِ وَتَقْدِمَتِهَا مَعَ سَكَائِبِهِنَّ كَعَادَتِهِنَّ رَائِحَةَ سُرُورٍ وَقُوداً لِلرَّبِّ.
\par 7 «وَفِي عَاشِرِ هَذَا الشَّهْرِ السَّابِعِ يَكُونُ لكُمْ مَحْفَلٌ مُقَدَّسٌ وَتُذَلِّلُونَ أَنْفُسَكُمْ. عَمَلاً مَا لا تَعْمَلُوا.
\par 8 وَتُقَرِّبُونَ مُحْرَقَةً لِلرَّبِّ رَائِحَةَ سُرُورٍ: ثَوْراً وَاحِداً ابْنَ بَقَرٍ وَكَبْشاً وَاحِداً وَسَبْعَةَ خِرَافٍ حَوْلِيَّةٍ. صَحِيحَةً تَكُونُ لكُمْ.
\par 9 وَتَقْدِمَتُهُنَّ مِنْ دَقِيقٍ مَلتُوتٍ بِزَيْتٍ ثَلاثَةُ أَعْشَارٍ لِلثَّوْرِ وَعُشْرَانِ لِلكَبْشِ الوَاحِدِ
\par 10 وَعُشْرٌ وَاحِدٌ لِكُلِّ خَرُوفٍ مِنَ السَّبْعَةِ الخِرَافِ.
\par 11 وَتَيْساً وَاحِداً مِنَ المَعْزِ ذَبِيحَةَ خَطِيَّةٍ فَضْلاً عَنْ ذَبِيحَةِ الخَطِيَّةِ لِلكَفَّارَةِ وَالمُحْرَقَةِ الدَّائِمَةِ وَتَقْدِمَتِهَا مَعَ سَكَائِبِهِنَّ.
\par 12 «وَفِي اليَوْمِ الخَامِسَ عَشَرَ مِنَ الشَّهْرِ السَّابِعِ يَكُونُ لكُمْ مَحْفَلٌ مُقَدَّسٌ. عَمَلاً مَا مِنَ الشُّغْلِ لا تَعْمَلُوا. وَتُعَيِّدُونَ عِيداً لِلرَّبِّ سَبْعَةَ أَيَّامٍ.
\par 13 وَتُقَرِّبُونَ مُحْرَقَةً وَقُودَ رَائِحَةِ سُرُورٍ لِلرَّبِّ ثَلاثَةَ عَشَرَ ثَوْراً أَبْنَاءَ بَقَرٍ وَكَبْشَيْنِ وَأَرْبَعَةَ عَشَرَ خَرُوفاً حَوْلِيّاً. صَحِيحَةً تَكُونُ لكُمْ.
\par 14 وَتَقْدِمَتُهُنَّ مِنْ دَقِيقٍ مَلتُوتٍ بِزَيْتٍ ثَلاثَةُ أَعْشَارٍ لِكُلِّ ثَوْرٍ مِنَ الثَّلاثَةَ عَشَرَ ثَوْراً وَعُشْرَانِ لِكُلِّ كَبْشٍ مِنَ الكَبْشَيْنِ
\par 15 وَعُشْرٌ وَاحِدٌ لِكُلِّ خَرُوفٍ مِنَ الأَرْبَعَةَ عَشَرَ خَرُوفاً
\par 16 وَتَيْساً وَاحِداً مِنَ المَعْزِ ذَبِيحَةَ خَطِيَّةٍ فَضْلاً عَنِ المُحْرَقَةِ الدَّائِمَةِ وَتَقْدِمَتِهَا وَسَكِيبِهَا.
\par 17 «وَفِي اليَوْمِ الثَّانِي اثْنَيْ عَشَرَ ثَوْراً أَبْنَاءَ بَقَرٍ وَكَبْشَيْنِ وَأَرْبَعَةَ عَشَرَ خَرُوفاً حَوْلِيّاً صَحِيحاً.
\par 18 وَتَقْدِمَتَهُنَّ وَسَكَائِبَهُنَّ لِلثِّيرَانِ وَالكَبْشَيْنِ وَالخِرَافِ حَسَبَ عَدَدِهِنَّ كَالعَادَةِ.
\par 19 وَتَيْساً وَاحِداً مِنَ المَعْزِ ذَبِيحَةَ خَطِيَّةٍ فَضْلاً عَنِ المُحْرَقَةِ الدَّائِمَةِ وَتَقْدِمَتِهَا مَعَ سَكَائِبِهِنَّ.
\par 20 «وَفِي اليَوْمِ الثَّالِثِ أَحَدَ عَشَرَ ثَوْراً وَكَبْشَيْنِ وَأَرْبَعَةَ عَشَرَ خَرُوفاً حَوْلِيّاً صَحِيحاً.
\par 21 وَتَقْدِمَتَهُنَّ وَسَكَائِبَهُنَّ لِلثِّيرَانِ وَالكَبْشَيْنِ وَالخِرَافِ حَسَبَ عَدَدِهِنَّ كَالعَادَةِ.
\par 22 وَتَيْساً وَاحِداً لِذَبِيحَةِ خَطِيَّةٍ فَضْلاً عَنِ المُحْرَقَةِ الدَّائِمَةِ وَتَقْدِمَتِهَا وَسَكِيبِهَا.
\par 23 «وَفِي اليَوْمِ الرَّابِعِ عَشَرَةَ ثِيرَانٍ وَكَبْشَيْنِ وَأَرْبَعَةَ عَشَرَ خَرُوفاً حَوْلِيّاً صَحِيحاً.
\par 24 وَتَقْدِمَتَهُنَّ وَسَكَائِبَهُنَّ لِلثِّيرَانِ وَالكَبْشَيْنِ وَالخِرَافِ حَسَبَ عَدَدِهِنَّ كَالعَادَةِ.
\par 25 وَتَيْساً وَاحِداً مِنَ المَعْزِ لِذَبِيحَةِ خَطِيَّةٍ فَضْلاً عَنِ المُحْرَقَةِ الدَّائِمَةِ وَتَقْدِمَتِهَا وَسَكِيبِهَا.
\par 26 «وَفِي اليَوْمِ الخَامِسِ تِسْعَةَ ثِيرَانٍ وَكَبْشَيْنِ وَأَرْبَعَةَ عَشَرَ خَرُوفاً حَوْلِيّاً صَحِيحاً.
\par 27 وَتَقْدِمَتَهُنَّ وَسَكَائِبَهُنَّ لِلثِّيرَانِ وَالكَبْشَيْنِ وَالخِرَافِ حَسَبَ عَدَدِهِنَّ كَالعَادَةِ.
\par 28 وَتَيْساً وَاحِداً لِذَبِيحَةِ خَطِيَّةٍ فَضْلاً عَنِ المُحْرَقَةِ الدَّائِمَةِ وَتَقْدِمَتِهَا وَسَكِيبِهَا.
\par 29 «وَفِي اليَوْمِ السَّادِسِ ثَمَانِيَةَ ثِيرَانٍ وَكَبْشَيْنِ وَأَرْبَعَةَ عَشَرَ خَرُوفاً حَوْلِيّاً صَحِيحاً.
\par 30 وَتَقْدِمَتَهُنَّ وَسَكَائِبَهُنَّ لِلثِّيرَانِ وَالكَبْشَيْنِ وَالخِرَافِ حَسَبَ عَدَدِهِنَّ كَالعَادَةِ.
\par 31 وَتَيْساً وَاحِداً لِذَبِيحَةِ خَطِيَّةٍ فَضْلاً عَنِ المُحْرَقَةِ الدَّائِمَةِ وَتَقْدِمَتِهَا وَسَكِيبِهَا.
\par 32 «وَفِي اليَوْمِ السَّابِعِ سَبْعَةَ ثِيرَانٍ وَكَبْشَيْنِ وَأَرْبَعَةَ عَشَرَ خَرُوفاً حَوْلِيّاً صَحِيحاً.
\par 33 وَتَقْدِمَتَهُنَّ وَسَكَائِبَهُنَّ لِلثِّيرَانِ وَالكَبْشَيْنِ وَالخِرَافِ حَسَبَ عَدَدِهِنَّ كَعَادَتِهِنَّ.
\par 34 وَتَيْساً وَاحِداً لِذَبِيحَةِ خَطِيَّةٍ فَضْلاً عَنِ المُحْرَقَةِ الدَّائِمَةِ وَتَقْدِمَتِهَا وَسَكِيبِهَا.
\par 35 «فِي اليَوْمِ الثَّامِنِ يَكُونُ لكُمُ اعْتِكَافٌ. عَمَلاً مَا مِنَ الشُّغْلِ لا تَعْمَلُوا.
\par 36 وَتُقَرِّبُونَ مُحْرَقَةً وَقُوداً رَائِحَةَ سُرُورٍ لِلرَّبِّ ثَوْراً وَاحِداً وَكَبْشاً وَاحِداً وَسَبْعَةَ خِرَافٍ حَوْلِيَّةٍ صَحِيحَةٍ.
\par 37 وَتَقْدِمَتَهُنَّ وَسَكَائِبَهُنَّ لِلثَّوْرِ وَالكَبْشِ وَالخِرَافِ حَسَبَ عَدَدِهِنَّ كَالعَادَةِ.
\par 38 وَتَيْساً وَاحِداً لِذَبِيحَةِ خَطِيَّةٍ فَضْلاً عَنِ المُحْرَقَةِ الدَّائِمَةِ وَتَقْدِمَتِهَا وَسَكِيبِهَا.
\par 39 هَذِهِ تُقَرِّبُونَهَا لِلرَّبِّ فِي مَوَاسِمِكُمْ فَضْلاً عَنْ نُذُورِكُمْ وَنَوَافِلِكُمْ مِنْ مُحْرَقَاتِكُمْ وَتَقْدِمَاتِكُمْ وَسَكَائِبِكُمْ وَذَبَائِحِ سَلامَتِكُمْ».
\par 40 فَكَلمَ مُوسَى بَنِي إِسْرَائِيل حَسَبَ كُلِّ مَا أَمَرَهُ بِهِ الرَّبُّ.

\chapter{30}

\par 1 وَقَال مُوسَى لِرُؤُوسِ أَسْبَاطِ بَنِي إِسْرَائِيل: «هَذَا مَا أَمَرَ بِهِ الرَّبُّ:
\par 2 إِذَا نَذَرَ رَجُلٌ نَذْراً لِلرَّبِّ أَوْ أَقْسَمَ قَسَماً أَنْ يُلزِمَ نَفْسَهُ بِلازِمٍ فَلا يَنْقُضْ كَلامَهُ. حَسَبَ كُلِّ مَا خَرَجَ مِنْ فَمِهِ يَفْعَلُ.
\par 3 وَأَمَّا المَرْأَةُ فَإِذَا نَذَرَتْ نَذْراً لِلرَّبِّ وَالتَزَمَتْ بِلازِمٍ فِي بَيْتِ أَبِيهَا فِي صِبَاهَا
\par 4 وَسَمِعَ أَبُوهَا نَذْرَهَا وَاللازِمَ الذِي أَلزَمَتْ نَفْسَهَا بِهِ فَإِنْ سَكَتَ أَبُوهَا لهَا ثَبَتَتْ كُلُّ نُذُورِهَا. وَكُلُّ لوَازِمِهَا التِي أَلزَمَتْ نَفْسَهَا بِهَا تَثْبُتُ.
\par 5 وَإِنْ نَهَاهَا أَبُوهَا يَوْمَ سَمْعِهِ فَكُلُّ نُذُورِهَا وَلوَازِمِهَا التِي أَلزَمَتْ نَفْسَهَا بِهَا لا تَثْبُتُ وَالرَّبُّ يَصْفَحُ عَنْهَا لأَنَّ أَبَاهَا قَدْ نَهَاهَا.
\par 6 وَإِنْ كَانَتْ لِزَوْجٍ وَنُذُورُهَا عَليْهَا أَوْ نُطْقُ شَفَتَيْهَا الذِي أَلزَمَتْ نَفْسَهَا بِهِ
\par 7 وَسَمِعَ زَوْجُهَا فَإِنْ سَكَتَ فِي يَوْمِ سَمْعِهِ ثَبَتَتْ نُذُورُهَا. وَلوَازِمُهَا التِي أَلزَمَتْ نَفْسَهَا بِهَا تَثْبُتُ.
\par 8 وَإِنْ نَهَاهَا رَجُلُهَا فِي يَوْمِ سَمْعِهِ فَسَخَ نَذْرَهَا الذِي عَليْهَا وَنُطْقَ شَفَتَيْهَا الذِي أَلزَمَتْ نَفْسَهَا بِهِ وَالرَّبُّ يَصْفَحُ عَنْهَا.
\par 9 وَأَمَّا نَذْرُ أَرْمَلةٍ أَوْ مُطَلقَةٍ فَكُلُّ مَا أَلزَمَتْ نَفْسَهَا بِهِ يَثْبُتُ عَليْهَا.
\par 10 وَلكِنْ إِنْ نَذَرَتْ فِي بَيْتِ زَوْجِهَا أَوْ أَلزَمَتْ نَفْسَهَا بِلازِمٍ بِقَسَمٍ
\par 11 وَسَمِعَ زَوْجُهَا فَإِنْ سَكَتَ لهَا وَلمْ يَنْهَهَا ثَبَتَتْ كُلُّ نُذُورِهَا. وَكُلُّ لازِمٍ أَلزَمَتْ نَفْسَهَا بِهِ يَثْبُتُ.
\par 12 وَإِنْ فَسَخَهَا زَوْجُهَا فِي يَوْمِ سَمْعِهِ فَكُلُّ مَا خَرَجَ مِنْ شَفَتَيْهَا مِنْ نُذُورِهَا أَوْ لوَازِمِ نَفْسِهَا لا يَثْبُتُ. قَدْ فَسَخَهَا زَوْجُهَا. وَالرَّبُّ يَصْفَحُ عَنْهَا.
\par 13 كُلُّ نَذْرٍ وَكُلُّ قَسَمِ التِزَامٍ لِإِذْلالِ النَّفْسِ زَوْجُهَا يُثْبِتُهُ وَزَوْجُهَا يَفْسَخُهُ.
\par 14 وَإِنْ سَكَتَ لهَا زَوْجُهَا مِنْ يَوْمٍ إِلى يَوْمٍ فَقَدْ أَثْبَتَ كُل نُذُورِهَا أَوْ كُل لوَازِمِهَا التِي عَليْهَا. أَثْبَتَهَا لأَنَّهُ سَكَتَ لهَا فِي يَوْمِ سَمْعِهِ.
\par 15 فَإِنْ فَسَخَهَا بَعْدَ سَمْعِهِ فَقَدْ حَمَل ذَنْبَهَا».
\par 16 هَذِهِ هِيَ الفَرَائِضُ التِي أَمَرَ بِهَا الرَّبُّ مُوسَى بَيْنَ الزَّوْجِ وَزَوْجَتِهِ وَبَيْنَ الأَبِ وَابْنَتِهِ فِي صِبَاهَا فِي بَيْتِ أَبِيهَا.

\chapter{31}

\par 1 وَأَمَرَ الرَّبُّ مُوسَى:
\par 2 «اِنْتَقِمْ نَقْمَةً لِبَنِي إِسْرَائِيل مِنَ المِدْيَانِيِّينَ ثُمَّ تُضَمُّ إِلى قَوْمِكَ».
\par 3 فَقَال مُوسَى لِلشَّعْبِ: «جَرِّدُوا مِنْكُمْ رِجَالاً لِلجُنْدِ فَيَكُونُوا عَلى مِدْيَانَ لِيَجْعَلُوا نَقْمَةَ الرَّبِّ عَلى مِدْيَانَ.
\par 4 أَلفاً وَاحِداً مِنْ كُلِّ سِبْطٍ مِنْ جَمِيعِ أَسْبَاطِ إِسْرَائِيل تُرْسِلُونَ لِلحَرْبِ».
\par 5 فَاخْتِيرَ مِنْ أُلُوفِ إِسْرَائِيل أَلفٌ مِنْ كُلِّ سِبْطٍ. اثْنَا عَشَرَ أَلفاً مُجَرَّدُونَ لِلحَرْبِ.
\par 6 فَأَرْسَلهُمْ مُوسَى أَلفاً مِنْ كُلِّ سِبْطٍ إِلى الحَرْبِ هُمْ وَفِينَحَاسَ بْنَ أَلِعَازَارَ الكَاهِنِ إِلى الحَرْبِ وَأَمْتِعَةُ القُدْسِ وَأَبْوَاقُ الهُتَافِ فِي يَدِهِ.
\par 7 فَتَجَنَّدُوا عَلى مِدْيَانَ كَمَا أَمَرَ الرَّبُّ وَقَتَلُوا كُل ذَكَرٍ.
\par 8 وَمُلُوكُ مِدْيَانَ قَتَلُوهُمْ فَوْقَ قَتْلاهُمْ. أَوِيَ وَرَاقِمَ وَصُورَ وَحُورَ وَرَابِعَ. خَمْسَةَ مُلُوكِ مِدْيَانَ. وَبَلعَامَ بْنَ بَعُورَ قَتَلُوهُ بِالسَّيْفِ.
\par 9 وَسَبَى بَنُو إِسْرَائِيل نِسَاءَ مِدْيَانَ وَأَطْفَالهُمْ وَنَهَبُوا جَمِيعَ بَهَائِمِهِمْ وَجَمِيعَ مَوَاشِيهِمْ وَكُل أَمْلاكِهِمْ.
\par 10 وَأَحْرَقُوا جَمِيعَ مُدُنِهِمْ بِمَسَاكِنِهِمْ وَجَمِيعَ حُصُونِهِمْ بِالنَّارِ.
\par 11 وَأَخَذُوا كُل الغَنِيمَةِ وَكُل النَّهْبِ مِنَ النَّاسِ وَالبَهَائِمِ
\par 12 وَأَتُوا إِلى مُوسَى وَأَلِعَازَارَ الكَاهِنِ وَإِلى جَمَاعَةِ بَنِي إِسْرَائِيل بِالسَّبْيِ وَالنَّهْبِ وَالغَنِيمَةِ إِلى المَحَلةِ إِلى عَرَبَاتِ مُوآبَ التِي عَلى أُرْدُنِّ أَرِيحَا.
\par 13 فَخَرَجَ مُوسَى وَأَلِعَازَارُ الكَاهِنُ وَكُلُّ رُؤَسَاءِ الجَمَاعَةِ لاِسْتِقْبَالِهِمْ إِلى خَارِجِ المَحَلةِ.
\par 14 فَسَخَطَ مُوسَى عَلى وُكَلاءِ الجَيْشِ رُؤَسَاءِ الأُلُوفِ وَرُؤَسَاءِ المِئَاتِ القَادِمِينَ مِنْ جُنْدِ الحَرْبِ.
\par 15 وَقَال لهُمْ مُوسَى: «هَل أَبْقَيْتُمْ كُل أُنْثَى حَيَّةً؟
\par 16 إِنَّ هَؤُلاءِ كُنَّ لِبَنِي إِسْرَائِيل حَسَبَ كَلامِ بَلعَامَ سَبَبَ خِيَانَةٍ لِلرَّبِّ فِي أَمْرِ فَغُورَ فَكَانَ الوَبَأُ فِي جَمَاعَةِ الرَّبِّ.
\par 17 فَالآنَ اقْتُلُوا كُل ذَكَرٍ مِنَ الأَطْفَالِ. وَكُل امْرَأَةٍ عَرَفَتْ رَجُلاً بِمُضَاجَعَةِ ذَكَرٍ اقْتُلُوهَا.
\par 18 لكِنْ جَمِيعُ الأَطْفَالِ مِنَ النِّسَاءِ اللوَاتِي لمْ يَعْرِفْنَ مُضَاجَعَةَ ذَكَرٍ أَبْقُوهُنَّ لكُمْ حَيَّاتٍ.
\par 19 وَأَمَّا أَنْتُمْ فَانْزِلُوا خَارِجَ المَحَلةِ سَبْعَةَ أَيَّامٍ. وَتَطَهَّرُوا كُلُّ مَنْ قَتَل نَفْساً وَكُلُّ مَنْ مَسَّ قَتِيلاً فِي اليَوْمِ الثَّالِثِ وَفِي السَّابِعِ أَنْتُمْ وَسَبْيُكُمْ.
\par 20 وَكُلُّ ثَوْبٍ وَكُلُّ مَتَاعٍ مِنْ جِلدٍ وَكُلُّ مَصْنُوعٍ مِنْ شَعْرِ مَعْزٍ وَكُلُّ مَتَاعٍ مِنْ خَشَبٍ تُطَهِّرُونَهُ».
\par 21 وَقَال أَلِعَازَارُ الكَاهِنُ لِرِجَالِ الجُنْدِ الذِينَ ذَهَبُوا لِلحَرْبِ: «هَذِهِ فَرِيضَةُ الشَّرِيعَةِ التِي أَمَرَ بِهَا الرَّبُّ مُوسَى.
\par 22 اَلذَّهَبُ وَالفِضَّةُ وَالنُّحَاسُ وَالحَدِيدُ وَالقَصْدِيرُ وَالرَّصَاصُ
\par 23 كُلُّ مَا يَدْخُلُ النَّارَ تُجِيزُونَهُ فِي النَّارِ فَيَكُونُ طَاهِراً غَيْرَ أَنَّهُ يَتَطَهَّرُ بِمَاءِ النَّجَاسَةِ. وَأَمَّا كُلُّ مَا لا يَدْخُلُ النَّارَ فَتُجِيزُونَهُ فِي المَاءِ.
\par 24 وَتَغْسِلُونَ ثِيَابَكُمْ فِي اليَوْمِ السَّابِعِ فَتَكُونُونَ طَاهِرِينَ وَبَعْدَ ذَلِكَ تَدْخُلُونَ المَحَلةَ».
\par 25 وَقَال الرَّبُّ لِمُوسَى:
\par 26 «أَحْصِ النَّهْبَ المَسْبِيَّ مِنَ النَّاسِ وَالبَهَائِمِ أَنْتَ وَأَلِعَازَارُ الكَاهِنُ وَرُؤُوسُ آبَاءِ الجَمَاعَةِ.
\par 27 وَنَصِّفِ النَّهْبَ بَيْنَ الذِينَ بَاشَرُوا القِتَال الخَارِجِينَ إِلى الحَرْبِ وَبَيْنَ كُلِّ الجَمَاعَةِ.
\par 28 وَارْفَعْ زَكَاةً لِلرَّبِّ. مِنْ رِجَالِ الحَرْبِ الخَارِجِينَ إِلى القِتَالِ وَاحِدَةً. نَفْساً مِنْ كُلِّ خَمْسِ مِئَةٍ مِنَ النَّاسِ وَالبَقَرِ وَالحَمِيرِ وَالغَنَمِ.
\par 29 مِنْ نِصْفِهِمْ تَأْخُذُونَهَا وَتُعْطُونَهَا لأَلِعَازَارَ الكَاهِنِ رَفِيعَةً لِلرَّبِّ.
\par 30 وَمِنْ نِصْفِ بَنِي إِسْرَائِيل تَأْخُذُ وَاحِدَةً مَأْخُوذَةً مِنْ كُلِّ خَمْسِينَ مِنَ النَّاسِ وَالبَقَرِ وَالحَمِيرِ وَالغَنَمِ مِنْ جَمِيعِ البَهَائِمِ وَتُعْطِيهَا لِلاوِيِّينَ الحَافِظِينَ شَعَائِرَ مَسْكَنِ الرَّبِّ».
\par 31 فَفَعَل مُوسَى وَأَلِعَازَارُ الكَاهِنُ كَمَا أَمَرَ الرَّبُّ مُوسَى.
\par 32 وَكَانَ النَّهْبُ فَضْلةُ الغَنِيمَةِ التِي اغْتَنَمَهَا رِجَالُ الجُنْدِ مِنَ الغَنَمِ سِتَّ مِئَةٍ وَخَمْسَةً وَسَبْعِينَ أَلفاً.
\par 33 وَمِنَ البَقَرِ اثْنَيْنِ وَسَبْعِينَ أَلفاً.
\par 34 وَمِنَ الحَمِيرِ وَاحِداً وَسِتِّينَ أَلفاً.
\par 35 وَمِنْ نُفُوسِ النَّاسِ مِنَ النِّسَاءِ اللوَاتِي لمْ يَعْرِفْنَ مُضَاجَعَةَ ذَكَرٍ جَمِيعِ النُّفُوسِ اثْنَيْنِ وَثَلاثِينَ أَلفاً.
\par 36 وَكَانَ النِّصْفُ نَصِيبُ الخَارِجِينَ إِلى الحَرْبِ: عَدَدُ الغَنَمِ ثَلاثَ مِئَةٍ وَسَبْعَةً وَثَلاثِينَ أَلفاً وَخَمْسَ مِئَةٍ.
\par 37 وَكَانَتِ الزَّكَاةُ لِلرَّبِّ مِنَ الغَنَمِ سِتَّ مِئَةٍ وَخَمْسَةً وَسَبْعِينَ.
\par 38 وَالبَقَرُ سِتَّةً وَثَلاثِينَ أَلفاً وَزَكَاتُهَا لِلرَّبِّ اثْنَيْنِ وَسَبْعِينَ.
\par 39 وَالحَمِيرُ ثَلاثِينَ أَلفاً وَخَمْسَ مِئَةٍ وَزَكَاتُهَا لِلرَّبِّ وَاحِداً وَسِتِّينَ.
\par 40 وَنُفُوسُ النَّاسِ سِتَّةَ عَشَرَ أَلفاً وَزَكَاتُهَا لِلرَّبِّ اثْنَيْنِ وَثَلاثِينَ نَفْساً.
\par 41 فَأَعْطَى مُوسَى الزَّكَاةَ رَفِيعَةَ الرَّبِّ لأَلِعَازَارَ الكَاهِنِ كَمَا أَمَرَ الرَّبُّ مُوسَى.
\par 42 وَأَمَّا نِصْفُ إِسْرَائِيل الذِي قَسَمَهُ مُوسَى مِنَ الرِّجَالِ المُتَجَنِّدِينَ
\par 43 فَكَانَ نِصْفُ الجَمَاعَةِ مِنَ الغَنَمِ ثَلاثَ مِئَةٍ وَسَبْعَةً وَثَلاثِينَ أَلفاً وَخَمْسَ مِئَةٍ.
\par 44 وَمِنَ البَقَرِ سِتَّةً وَثَلاثِينَ أَلفاً.
\par 45 وَمِنَ الحَمِيرِ ثَلاثِينَ أَلفاً وَخَمْسَ مِئَةٍ.
\par 46 وَمِنْ نُفُوسِ النَّاسِ سِتَّةَ عَشَرَ أَلفاً.
\par 47 فَأَخَذَ مُوسَى مِنْ نِصْفِ بَنِي إِسْرَائِيل المَأْخُوذِ وَاحِداً مِنْ كُلِّ خَمْسِينَ مِنَ النَّاسِ وَمِنَ البَهَائِمِ وَأَعْطَاهَا لِلاوِيِّينَ الحَافِظِينَ شَعَائِرَ مَسْكَنِ الرَّبِّ كَمَا أَمَرَ الرَّبُّ مُوسَى.
\par 48 ثُمَّ تَقَدَّمَ إِلى مُوسَى الوُكَلاءُ الذِينَ عَلى أُلُوفِ الجُنْدِ رُؤَسَاءُ الأُلُوفِ وَرُؤَسَاءُ المِئَاتِ
\par 49 وَقَالُوا لِمُوسَى: «عَبِيدُكَ قَدْ أَخَذُوا عَدَدَ رِجَالِ الحَرْبِ الذِينَ فِي أَيْدِينَا فَلمْ يُفْقَدْ مِنَّا إِنْسَانٌ.
\par 50 فَقَدْ قَدَّمْنَا قُرْبَانَ الرَّبِّ كُلُّ وَاحِدٍ مَا وَجَدَهُ أَمْتِعَةَ ذَهَبٍ حُجُولاً وَأَسَاوِرَ وَخَوَاتِمَ وَأَقْرَاطاً وَقَلائِدَ لِلتَّكْفِيرِ عَنْ أَنْفُسِنَا أَمَامَ الرَّبِّ».
\par 51 فَأَخَذَ مُوسَى وَأَلِعَازَارُ الكَاهِنُ الذَّهَبَ مِنْهُمْ كُل أَمْتِعَةٍ مَصْنُوعَةٍ.
\par 52 وَكَانَ كُلُّ ذَهَبِ الرَّفِيعَةِ التِي رَفَعُوهَا لِلرَّبِّ سِتَّةَ عَشَرَ أَلفاً وَسَبْعَ مِئَةٍ وَخَمْسِينَ شَاقِلاً مِنْ عِنْدِ رُؤَسَاءِ الأُلُوفِ وَرُؤَسَاءِ المِئَاتِ.
\par 53 (أَمَّا رِجَالُ الجُنْدِ فَاغْتَنَمُوا كُلُّ وَاحِدٍ لِنَفْسِهِ).
\par 54 فَأَخَذَ مُوسَى وَأَلِعَازَارُ الكَاهِنُ الذَّهَبَ مِنْ رُؤَسَاءِ الأُلُوفِ وَالمِئَاتِ وَأَتَيَا بِهِ إِلى خَيْمَةِ الاِجْتِمَاعِ تِذْكَاراً لِبَنِي إِسْرَائِيل أَمَامَ الرَّبِّ.

\chapter{32}

\par 1 وَأَمَّا بَنُو رَأُوبَيْنَ وَبَنُو جَادَ فَكَانَ لهُمْ مَوَاشٍ كَثِيرَةٌ وَافِرَةٌ جِدّاً. فَلمَّا رَأُوا أَرْضَ يَعْزِيرَ وَأَرْضَ جِلعَادَ وَإِذَا المَكَانُ مَكَانُ مَوَاشٍ
\par 2 أَتَى بَنُو جَادَ وَبَنُو رَأُوبَيْنَ وَقَالُوا لِمُوسَى وَأَلِعَازَارَ الكَاهِنِ وَرُؤَسَاءِ الجَمَاعَةِ:
\par 3 «عَطَارُوتُ وَدِيبُونُ وَيَعْزِيرُ وَنِمْرَةُ وَحَشْبُونُ وَأَلِعَالةُ وَشَبَامُ وَنَبُو وَبَعُونُ
\par 4 الأَرْضُ التِي ضَرَبَهَا الرَّبُّ قُدَّامَ بَنِي إِسْرَائِيل هِيَ أَرْضُ مَوَاشٍ وَلِعَبِيدِكَ مَوَاشٍ».
\par 5 ثُمَّ قَالُوا: «إِنْ وَجَدْنَا نِعْمَةً فِي عَيْنَيْكَ فَلتُعْطَ هَذِهِ الأَرْضُ لِعَبِيدِكَ مُلكاً وَلا تُعَبِّرْنَا الأُرْدُنَّ».
\par 6 فَقَال مُوسَى لِبَنِي جَادٍ وَبَنِي رَأُوبَيْنَ: «هَل يَنْطَلِقُ إِخْوَتُكُمْ إِلى الحَرْبِ وَأَنْتُمْ تَقْعُدُونَ هَهُنَا؟
\par 7 فَلِمَاذَا تَصُدُّونَ قُلُوبَ بَنِي إِسْرَائِيل عَنِ العُبُورِ إِلى الأَرْضِ التِي أَعْطَاهُمُ الرَّبُّ؟
\par 8 هَكَذَا فَعَل آبَاؤُكُمْ حِينَ أَرْسَلتُهُمْ مِنْ قَادِشَ بَرْنِيعَ لِيَنْظُرُوا الأَرْضَ.
\par 9 صَعِدُوا إِلى وَادِي أَشْكُول وَنَظَرُوا الأَرْضَ وَصَدُّوا قُلُوبَ بَنِي إِسْرَائِيل عَنْ دُخُولِ الأَرْضِ التِي أَعْطَاهُمُ الرَّبُّ.
\par 10 فَحَمِيَ غَضَبُ الرَّبِّ فِي ذَلِكَ اليَوْمِ وَأَقْسَمَ قَائِلاً:
\par 11 لنْ يَرَى النَّاسُ الذِينَ صَعِدُوا مِنْ مِصْرَ مِنِ ابْنِ عِشْرِينَ سَنَةً فَصَاعِداً الأَرْضَ التِي أَقْسَمْتُ لِإِبْرَاهِيمَ وَإِسْحَاقَ وَيَعْقُوبَ لأَنَّهُمْ لمْ يَتَّبِعُونِي تَمَاماً
\par 12 مَا عَدَا كَالِبَ بْنَ يَفُنَّةَ القِنِزِّيَّ وَيَشُوعَ بْنَ نُونَ لأَنَّهُمَا اتَّبَعَا الرَّبَّ تَمَاماً.
\par 13 فَحَمِيَ غَضَبُ الرَّبِّ عَلى إِسْرَائِيل وَأَتَاهَهُمْ فِي البَرِّيَّةِ أَرْبَعِينَ سَنَةً حَتَّى فَنِيَ كُلُّ الجِيلِ الذِي فَعَل الشَّرَّ فِي عَيْنَيِ الرَّبِّ.
\par 14 فَهُوَذَا أَنْتُمْ قَدْ قُمْتُمْ عِوَضاً عَنْ آبَائِكُمْ تَرْبِيَةَ أُنَاسٍ خُطَاةٍ لِتَزِيدُوا أَيْضاً حُمُوَّ غَضَبِ الرَّبِّ عَلى إِسْرَائِيل.
\par 15 إِذَا ارْتَدَدْتُمْ مِنْ وَرَائِهِ يَعُودُ يَتْرُكُهُ أَيْضاً فِي البَرِّيَّةِ فَتُهْلِكُونَ كُل هَذَا الشَّعْبِ».
\par 16 فَاقْتَرَبُوا إِليْهِ وَقَالُوا: «نَبْنِي حَظَائِرَ غَنَمٍ لِمَوَاشِينَا هَهُنَا وَمُدُناً لأَطْفَالِنَا.
\par 17 وَأَمَّا نَحْنُ فَنَتَجَرَّدُ مُسْرِعِينَ قُدَّامَ بَنِي إِسْرَائِيل حَتَّى نَأْتِيَ بِهِمْ إِلى مَكَانِهِمْ. وَيَلبَثُ أَطْفَالُنَا فِي مُدُنٍ مُحَصَّنَةٍ مِنْ وَجْهِ سُكَّانِ الأَرْضِ.
\par 18 لا نَرْجِعُ إِلى بُيُوتِنَا حَتَّى يَقْتَسِمَ بَنُو إِسْرَائِيل كُلُّ وَاحِدٍ نَصِيبَهُ.
\par 19 إِنَّنَا لا نَمْلِكُ مَعَهُمْ فِي عَبْرِ الأُرْدُنِّ وَمَا وَرَاءَهُ لأَنَّ نَصِيبَنَا قَدْ حَصَل لنَا فِي عَبْرِ الأُرْدُنِّ إِلى الشَّرْقِ».
\par 20 فَقَال لهُمْ مُوسَى: «إِنْ فَعَلتُمْ هَذَا الأَمْرَ إِنْ تَجَرَّدْتُمْ أَمَامَ الرَّبِّ لِلحَرْبِ
\par 21 وَعَبَرَ الأُرْدُنَّ كُلُّ مُتَجَرِّدٍ مِنْكُمْ أَمَامَ الرَّبِّ حَتَّى طَرَدَ أَعْدَاءَهُ مِنْ أَمَامِهِ
\par 22 وَأُخْضِعَتِ الأَرْضُ أَمَامَ الرَّبِّ وَبَعْدَ ذَلِكَ رَجَعْتُمْ فَتَكُونُونَ أَبْرِيَاءَ مِنْ نَحْوِ الرَّبِّ وَمِنْ نَحْوِ إِسْرَائِيل وَتَكُونُ هَذِهِ الأَرْضُ مُلكاً لكُمْ أَمَامَ الرَّبِّ.
\par 23 وَلكِنْ إِنْ لمْ تَفْعَلُوا هَكَذَا فَإِنَّكُمْ تُخْطِئُونَ إِلى الرَّبِّ. وَتَعْلمُونَ خَطِيَّتَكُمُ التِي تُصِيبُكُمْ.
\par 24 اِبْنُوا لأَنْفُسِكُمْ مُدُناً لأَطْفَالِكُمْ وَحَظَائِرَ لِغَنَمِكُمْ. وَمَا خَرَجَ مِنْ أَفْوَاهِكُمُ افْعَلُوا».
\par 25 فَقَال بَنُو جَادَ وَبَنُو رَأُوبَيْنَ لِمُوسَى: «عَبِيدُكَ يَفْعَلُونَ كَمَا أَمَرَ سَيِّدِي.
\par 26 أَطْفَالُنَا وَنِسَاؤُنَا وَمَوَاشِينَا وَكُلُّ بَهَائِمِنَا تَكُونُ هُنَاكَ فِي مُدُنِ جِلعَادَ.
\par 27 وَعَبِيدُكَ يَعْبُرُونَ كُلُّ مُتَجَرِّدٍ لِلجُنْدِ أَمَامَ الرَّبِّ لِلحَرْبِ كَمَا تَكَلمَ سَيِّدِي».
\par 28 فَأَوْصَى بِهِمْ مُوسَى أَلِعَازَارَ الكَاهِنَ وَيَشُوعَ بْنَ نُونٍَ وَرُؤُوسَ آبَاءِ الأَسْبَاطِ مِنْ بَنِي إِسْرَائِيل.
\par 29 وَقَال لهُمْ مُوسَى: «إِنْ عَبَرَ الأُرْدُنَّ مَعَكُمْ بَنُو جَادَ وَبَنُو رَأُوبَيْنَ كُلُّ مُتَجَرِّدٍ لِلحَرْبِ أَمَامَ الرَّبِّ فَمَتَى أُخْضِعَتِ الأَرْضُ أَمَامَكُمْ تُعْطُونَهُمْ أَرْضَ جِلعَادَ مُلكاً.
\par 30 وَلكِنْ إِنْ لمْ يَعْبُرُوا مُتَجَرِّدِينَ مَعَكُمْ يَتَمَلكُوا فِي وَسَطِكُمْ فِي أَرْضِ كَنْعَانَ».
\par 31 فَأَجَابَ بَنُو جَادٍ وَبَنُو رَأُوبَيْنَ: «الذِي تَكَلمَ بِهِ الرَّبُّ عَنْ عَبِيدِكَ كَذَلِكَ نَفْعَلُ.
\par 32 نَحْنُ نَعْبُرُ مُتَجَرِّدِينَ أَمَامَ الرَّبِّ إِلى أَرْضِ كَنْعَانَ وَلكِنْ نُعْطَى مُلكَ نَصِيبِنَا فِي عَبْرِ الأُرْدُنِّ».
\par 33 فَأَعْطَى مُوسَى لهُمْ لِبَنِي جَادٍ وَبَنِي رَأُوبَيْنَ وَنِصْفِ سِبْطِ مَنَسَّى بْنِ يُوسُفَ مَمْلكَةَ سِيحُونَ مَلِكِ الأَمُورِيِّينَ وَمَمْلكَةَ عُوجٍ مَلِكِ بَاشَانَ الأَرْضَ مَعَ مُدُنِهَا بِتُخُومِ مُدُنِ الأَرْضِ حَوَاليْهَا.
\par 34 فَبَنَى بَنُو جَادَ دِيبُونَ وَعَطَارُوتَ وَعَرُوعِيرَ
\par 35 وَعَطْرُوتَ شُوفَانَ وَيَعْزِيرَ وَيُجْبَهَةَ
\par 36 وَبَيْتَ نِمْرَةَ وَبَيْتَ هَارَانَ مُدُناً مُحَصَّنَةً مَعَ حَظَائِرِ غَنَمٍ.
\par 37 وَبَنَى بَنُو رَأُوبَيْنَ حَشْبُونَ وَأَلِعَالةَ وَقَرْيَتَايِمَ
\par 38 وَنَبُوَ وَبَعْل مَعُونَ (مُغَيَّرَتَيِ الاِسْمِ) وَسَبْمَةَ وَدَعُوا بِأَسْمَاءٍ أَسْمَاءَ المُدُنِ التِي بَنُوا.
\par 39 وَذَهَبَ بَنُو مَاكِيرَ بْنِ مَنَسَّى إِلى جِلعَادَ وَأَخَذُوهَا وَطَرَدُوا الأَمُورِيِّينَ الذِينَ فِيهَا.
\par 40 فَأَعْطَى مُوسَى جِلعَادَ لِمَاكِيرَ بْنِ مَنَسَّى فَسَكَنَ فِيهَا.
\par 41 وَذَهَبَ يَائِيرُ ابْنُ مَنَسَّى وَأَخَذَ مَزَارِعَهَا وَدَعَاهُنَّ حَوُّوثَ يَائِيرَ.
\par 42 وَذَهَبَ نُوبَحُ وَأَخَذَ قَنَاةَ وَقُرَاهَا وَدَعَاهَا نُوبَحَ بِاسْمِهِ.

\chapter{33}

\par 1 هَذِهِ رِحْلاتُ بَنِي إِسْرَائِيل الذِينَ خَرَجُوا مِنْ أَرْضِ مِصْرَ بِجُنُودِهِمْ عَنْ يَدِ مُوسَى وَهَارُونَ.
\par 2 وَكَتَبَ مُوسَى مَخَارِجَهُمْ بِرِحْلاتِهِمْ حَسَبَ قَوْلِ الرَّبِّ. وَهَذِهِ رِحْلاتُهُمْ بِمَخَارِجِهِمْ:
\par 3 اِرْتَحَلُوا مِنْ رَعَمْسِيسَ فِي الشَّهْرِ الأَوَّلِ فِي اليَوْمِ الخَامِسِ عَشَرَ مِنَ الشَّهْرِ الأَوَّلِ فِي غَدِ الفِصْحِ. خَرَجَ بَنُو إِسْرَائِيل بِيَدٍ رَفِيعَةٍ أَمَامَ أَعْيُنِ جَمِيعِ المِصْرِيِّينَ
\par 4 إِذْ كَانَ المِصْرِيُّونَ يَدْفِنُونَ الذِينَ ضَرَبَ مِنْهُمُ الرَّبُّ مِنْ كُلِّ بِكْرٍ. وَالرَّبُّ قَدْ صَنَعَ بِآلِهَتِهِمْ أَحْكَاماً.
\par 5 فَارْتَحَل بَنُو إِسْرَائِيل مِنْ رَعَمْسِيسَ وَنَزَلُوا فِي سُكُّوتَ.
\par 6 ثُمَّ ارْتَحَلُوا مِنْ سُكُّوتَ وَنَزَلُوا فِي إِيثَامَ التِي فِي طَرَفِ البَرِّيَّةِ.
\par 7 ثُمَّ ارْتَحَلُوا مِنْ إِيثَامَ وَرَجَعُوا عَلى فَمِ الحِيرُوثِ التِي قُبَالةَ بَعْل صَفُونَ وَنَزَلُوا أَمَامَ مَجْدَلٍ.
\par 8 ثُمَّ ارْتَحَلُوا مِنْ أَمَامِ الحِيرُوثِ وَعَبَرُوا فِي وَسَطِ البَحْرِ إِلى البَرِّيَّةِ وَسَارُوا مَسِيرَةَ ثَلاثَةِ أَيَّامٍ فِي بَرِّيَّةِ إِيثَامَ وَنَزَلُوا فِي مَارَّةَ.
\par 9 ثُمَّ ارْتَحَلُوا مِنْ مَارَّةَ وَأَتُوا إِلى إِيلِيمَ. وَكَانَ فِي إِيلِيمَ اثْنَتَا عَشَرَةَ عَيْنَ مَاءٍ وَسَبْعُونَ نَخْلةً. فَنَزَلُوا هُنَاكَ.
\par 10 ثُمَّ ارْتَحَلُوا مِنْ إِيلِيمَ وَنَزَلُوا عَلى بَحْرِ سُوفَ.
\par 11 ثُمَّ ارْتَحَلُوا مِنْ بَحْرِ سُوفَ وَنَزَلُوا فِي بَرِّيَّةِ سِينٍ.
\par 12 ثُمَّ ارْتَحَلُوا مِنْ بَرِّيَّةِ سِينٍ وَنَزَلُوا فِي دُفْقَةَ.
\par 13 ثُمَّ ارْتَحَلُوا مِنْ دُفْقَةَ وَنَزَلُوا فِي أَلُوشَ.
\par 14 ثُمَّ ارْتَحَلُوا مِنْ أَلُوشَ وَنَزَلُوا فِي رَفِيدِيمَ. وَلمْ يَكُنْ هُنَاكَ مَاءٌ لِلشَّعْبِ لِيَشْرَبَ.
\par 15 ثُمَّ ارْتَحَلُوا مِنْ رَفِيدِيمَ وَنَزَلُوا فِي بَرِّيَّةِ سِينَاءَ.
\par 16 ثُمَّ ارْتَحَلُوا مِنْ بَرِّيَّةِ سِينَاءَ وَنَزَلُوا فِي قَبَرُوتَ هَتَّأَوَةَ.
\par 17 ثُمَّ ارْتَحَلُوا مِنْ قَبَرُوتَ هَتَّأَوَةَ وَنَزَلُوا فِي حَضَيْرُوتَ.
\par 18 ثُمَّ ارْتَحَلُوا مِنْ حَضَيْرُوتَ وَنَزَلُوا فِي رِثْمَةَ.
\par 19 ثُمَّ ارْتَحَلُوا مِنْ رِثْمَةَ وَنَزَلُوا فِي رِمُّونَ فَارِصَ.
\par 20 ثُمَّ ارْتَحَلُوا مِنْ رِمُّونَ فَارِصَ وَنَزَلُوا فِي لِبْنَةَ.
\par 21 ثُمَّ ارْتَحَلُوا مِنْ لِبْنَةَ وَنَزَلُوا فِي رِسَّةَ.
\par 22 ثُمَّ ارْتَحَلُوا مِنْ رِسَّةَ وَنَزَلُوا فِي قُهَيْلاتَةَ.
\par 23 ثُمَّ ارْتَحَلُوا مِنْ قُهَيْلاتَةَ وَنَزَلُوا فِي جَبَلِ شَافَرَ.
\par 24 ثُمَّ ارْتَحَلُوا مِنْ جَبَلِ شَافَرَ وَنَزَلُوا فِي حَرَادَةَ.
\par 25 ثُمَّ ارْتَحَلُوا مِنْ حَرَادَةَ وَنَزَلُوا فِي مَقْهَيْلُوتَ.
\par 26 ثُمَّ ارْتَحَلُوا مِنْ مَقْهَيْلُوتَ وَنَزَلُوا فِي تَاحَتَ.
\par 27 ثُمَّ ارْتَحَلُوا مِنْ تَاحَتَ وَنَزَلُوا فِي تَارَحَ.
\par 28 ثُمَّ ارْتَحَلُوا مِنْ تَارَحَ وَنَزَلُوا فِي مِثْقَةَ.
\par 29 ثُمَّ ارْتَحَلُوا مِنْ مِثْقَةَ وَنَزَلُوا فِي حَشْمُونَةَ.
\par 30 ثُمَّ ارْتَحَلُوا مِنْ حَشْمُونَةَ وَنَزَلُوا فِي مُسِيرُوتَ.
\par 31 ثُمَّ ارْتحَلُوا مِن مُسِيرُوتَ وَنَزَلُوا فِي بَنِي يَعْقَانَ.
\par 32 ثُّمَ ارْتَحَلُوا مِنْ بَنِي يَعْقَانَ وَنَزَلُوا فِي حُورِ الجِدْجَادِ.
\par 33 ثمَّ ارْتَحَلُوا مِنْ حُورِ الجِدْجَادِ وَنَزَلُوا فِي يُطْبَاتَ.
\par 34 ثمَّ ارْتَحَلُوا مِنْ يُطْبَاتَ وَنَزَلُوا فِي عَبْرُونَةَ.
\par 35 ثمَّ ارْتَحَلُوا مِنْ عَبْرُونَةَ وَنَزَلُوا فِي عِصْيُونَ جَابِرَ.
\par 36 ثمَّ ارْتَحَلُوا مِنْ عِصْيُونَ جَابِرَ وَنَزَلُوا فِي برِّيّةِ صِينٍ (وَهِيَ قَادِشُ).
\par 37 ثمَّ ارْتَحَلُوا مِنْ قَادِشَ وَنَزَلُوا فِي جَبَلِ هُورٍ فِي طَرَفِ أَرْضِ أَدُومَ.
\par 38 فَصَعِدَ هَارُونُ الكَاهِنُ إِلى جَبَلِ هُورٍ حَسَبَ قَوْلِ الرَّبِّ وَمَاتَ هُنَاكَ فِي السَّنَةِ الأَرْبَعِينَ لِخُرُوجِ بَنِي إِسْرَائِيل مِنْ أَرْضِ مِصْرَ في الشَّهْرِ الخَامِسِ فِي الأَوَّلِ مِنَ الشَّهْرِ.
\par 39 وَكَانَ هَارُونُ ابْنَ مِئَةٍ وثَلاثٍ وَعِشْرِينَ سَنَةً حِينَ مَاتَ فِي جَبَلِ هُورٍ.
\par 40 وَسَمِعَ الكَنْعَانِيُّ مَلِكُ عَرَادَ وَهُوَ سَاكِنٌ فِي الجَنُوبِ فِي أَرْضِ كَنْعَانَ بِمَجِيءِ بَنِي إِسْرَائِيل.
\par 41 ثُمَّ ارْتَحَلُوا مِنْ جَبَلِ هُورٍ وَنَزَلُوا فِي صَلمُونَةَ.
\par 42 ثُمَّ ارْتَحَلُوا مِنْ صَلمُونَةَ وَنَزَلُوا فِي فُونُونَ.
\par 43 ثُمَّ ارْتَحَلُوا مِنْ فُونُونَ وَنَزَلُوا فِي أُوبُوتَ.
\par 44 ثُمَّ ارْتَحَلُوا مِنْ أُوبُوتَ وَنَزَلُوا فِي عَيِّي عَبَارِيمَ فِي تُخُمِ مُوآبَ.
\par 45 ثُمَّ ارْتَحَلُوا مِنْ عَيِّيمَ وَنَزَلُوا فِي دِيبُونَ جَادَ.
\par 46 ثُمَّ ارْتَحَلُوا مِنْ دِيبُونَ جَادَ وَنَزَلُوا فِي عَلمُونَ دِبْلاتَايِمَ.
\par 47 ثُمَّ ارْتَحَلُوا مِنْ عَلمُونَ دِبْلاتَايِمَ وَنَزَلُوا فِي جِبَالِ عَبَارِيمَ أَمَامَ نَبُو.
\par 48 ثُمَّ ارْتَحَلُوا مِنْ جِبَالِ عَبَارِيمَ وَنَزَلُوا فِي عَرَبَاتِ مُوآبَ عَلى أُرْدُنِّ أَرِيحَا.
\par 49 نَزَلُوا عَلى الأُرْدُنِّ مِنْ بَيْتِ يَشِيمُوتَ إِلى آبَل شِطِّيمَ فِي عَرَبَاتِ مُوآبَ.
\par 50 وَقَال الرَّبُّ لِمُوسَى فِي عَرَبَاتِ مُوآبَ عَلى أُرْدُنِّ أَرِيحَا:
\par 51 «قُل لِبَنِي إِسْرَائِيل: إِنَّكُمْ عَابِرُونَ الأُرْدُنَّ إِلى أَرْضِ كَنْعَانَ
\par 52 فَتَطْرُدُونَ كُل سُكَّانِ الأَرْضِ مِنْ أَمَامِكُمْ وَتَمْحُونَ جَمِيعَ تَصَاوِيرِهِمْ وَتُبِيدُونَ كُل أَصْنَامِهِمِ المَسْبُوكَةِ وَتُخْرِبُونَ جَمِيعَ مُرْتَفَعَاتِهِمْ.
\par 53 تَمْلِكُونَ الأَرْضَ وَتَسْكُنُونَ فِيهَا لأَنِّي قَدْ أَعْطَيْتُكُمُ الأَرْضَ لِكَيْ تَمْلِكُوهَا
\par 54 وَتَقْتَسِمُونَ الأَرْضَ بِالقُرْعَةِ حَسَبَ عَشَائِرِكُمْ. الكَثِيرُ تُكَثِّرُونَ لهُ نَصِيبَهُ وَالقَلِيلُ تُقَلِّلُونَ لهُ نَصِيبَهُ. حَيْثُ خَرَجَتْ لهُ القُرْعَةُ فَهُنَاكَ يَكُونُ لهُ. حَسَبَ أَسْبَاطِ آبَائِكُمْ تَقْتَسِمُونَ.
\par 55 وَإِنْ لمْ تَطْرُدُوا سُكَّانَ الأَرْضِ مِنْ أَمَامِكُمْ يَكُونُ الذِينَ تَسْتَبْقُونَ مِنْهُمْ أَشْوَاكاً فِي أَعْيُنِكُمْ وَمَنَاخِسَ فِي جَوَانِبِكُمْ وَيُضَايِقُونَكُمْ عَلى الأَرْضِ التِي أَنْتُمْ سَاكِنُونَ فِيهَا.
\par 56 فَيَكُونُ أَنِّي أَفْعَلُ بِكُمْ كَمَا هَمَمْتُ أَنْ أَفْعَل بِهِمْ».

\chapter{34}

\par 1 وَأَمَر الرَّبُّ مُوسَى:
\par 2 «قُل لِبَنِي إِسْرَائِيل: إِنَّكُمْ دَاخِلُونَ إِلى أَرْضِ كَنْعَانَ. هَذِهِ هِيَ الأَرْضُ التِي تَقَعُ لكُمْ نَصِيباً. أَرْضُ كَنْعَانَ بِتُخُومِهَا.
\par 3 تَكُونُ لكُمْ نَاحِيَةُ الجَنُوبِ مِنْ بَرِّيَّةِ صِينَ عَلى جَانِبِ أَدُومَ. وَيَكُونُ لكُمْ تُخُمُ الجَنُوبِ مِنْ طَرَفِ بَحْرِ المِلحِ إِلى الشَّرْقِ
\par 4 وَيَدُورُ لكُمُ التُّخُمُ مِنْ جَنُوبِ عَقَبَةِ عَقْرِبِّيمَ وَيَعْبُرُ إِلى صِينَ وَتَكُونُ مَخَارِجُهُ مِنْ جَنُوبِ قَادِشَ بَرْنِيعَ وَيَخْرُجُ إِلى حَصَرِ أَدَّارَ وَيَعْبُرُ إِلى عَصْمُونَ.
\par 5 ثُمَّ يَدُورُ التُّخُمُ مِنْ عَصْمُونَ إِلى وَادِي مِصْرَ وَتَكُونُ مَخَارِجُهُ عِنْدَ البَحْرِ.
\par 6 وَأَمَّا تُخُمُ الغَرْبِ فَيَكُونُ البَحْرُ الكَبِيرُ لكُمْ تُخُماً. هَذَا يَكُونُ لكُمْ تُخُمُ الغَرْبِ.
\par 7 وَهَذَا يَكُونُ لكُمْ تُخُمُ الشِّمَالِ. مِنَ البَحْرِ الكَبِيرِ تَرْسُمُونَ لكُمْ إِلى جَبَلِ هُورَ.
\par 8 وَمِنْ جَبَلِ هُورَ تَرْسُمُونَ إِلى مَدْخَلِ حَمَاةَ وَتَكُونُ مَخَارِجُ التُّخُمِ إِلى صَدَدَ.
\par 9 ثُمَّ يَخْرُجُ التُّخُمُ إِلى زِفْرُونَ وَتَكُونُ مَخَارِجُهُ عِنْدَ حَصَرِ عِينَانَ. هَذَا يَكُونُ لكُمْ تُخُمُ الشِّمَالِ.
\par 10 وَتَرْسُمُونَ لكُمْ تُخُماً إِلى الشَّرْقِ مِنْ حَصَرِ عِينَانَ إِلى شَفَامَ.
\par 11 وَيَنْحَدِرُ التُّخُمُ مِنْ شَفَامَ إِلى رَبْلةَ شَرْقِيَّ عَيْنٍ. ثُمَّ يَنْحَدِرُ التُّخُمُ وَيَمَسُّ جَانِبَ بَحْرِ كِنَّارَةَ إِلى الشَّرْقِ.
\par 12 ثُمَّ يَنْحَدِرُ التُّخُمُ إِلى الأُرْدُنِّ وَتَكُونُ مَخَارِجُهُ عِنْدَ بَحْرِ المِلحِ. هَذِهِ تَكُونُ لكُمُ الأَرْضُ بِتُخُومِهَا حَوَاليْهَا».
\par 13 فَأَمَرَ مُوسَى بَنِي إِسْرَائِيل: «هَذِهِ هِيَ الأَرْضُ التِي تَقْتَسِمُونَهَا بِالقُرْعَةِ التِي أَمَرَ الرَّبُّ أَنْ تُعْطَى لِلتِّسْعَةِ الأَسْبَاطِ وَنِصْفِ السِّبْطِ.
\par 14 لأَنَّهُ قَدْ أَخَذَ سِبْطُ بَنِي رَأُوبَيْنَ حَسَبَ بُيُوتِ آبَائِهِمْ وَسِبْطُ بَنِي جَادَ حَسَبَ بُيُوتِ آبَائِهِمْ وَنِصْفُ سِبْطِ مَنَسَّى. قَدْ أَخَذُوا نَصِيبَهُمْ.
\par 15 اَلسِّبْطَانِ وَنِصْفُ السِّبْطِ قَدْ أَخَذُوا نَصِيبَهُمْ فِي عَبْرِ أُرْدُنِّ أَرِيحَا شَرْقاً نَحْوَ الشُّرُوقِ».
\par 16 وَقَال الرَّبُّ لِمُوسَى:
\par 17 «هَذَانِ اسْمَا الرَّجُليْنِ اللذَيْنِ يَقْسِمَانِ لكُمُ الأَرْضَ: أَلِعَازَارُ الكَاهِنُ وَيَشُوعُ بْنُ نُونَ.
\par 18 وَرَئِيساً وَاحِداً مِنْ كُلِّ سِبْطٍ تَأْخُذُونَ لِقِسْمَةِ الأَرْضِ.
\par 19 وَهَذِهِ أَسْمَاءُ الرِّجَالِ. مِنْ سِبْطِ يَهُوذَا كَالِبُ بْنُ يَفُنَّةَ.
\par 20 وَمِنْ سِبْطِ بَنِي شَمْعُونَ شَمُوئِيلُ بْنُ عَمِّيهُودَ.
\par 21 وَمِنْ سِبْطِ بِنْيَامِينَ أَلِيدَادُ بْنُ كَسْلُونَ.
\par 22 وَمِنْ سِبْطِ بَنِي دَانَ الرَّئِيسُ بُقِّي بْنُ يُجْلِي.
\par 23 وَمِنْ بَنِي يُوسُفَ: مِنْ سِبْطِ بَنِي مَنَسَّى الرَّئِيسُ حَنِّيئِيلُ بْنُ إِيفُودَ.
\par 24 وَمِنْ سِبْطِ بَنِي أَفْرَايِمَ الرَّئِيسُ قَمُوئِيلُ بْنُ شِفْطَانَ.
\par 25 وَمِنْ سِبْطِ بَنِي زَبُولُونَ الرَّئِيسُ أَلِيصَافَانُ بْنُ فَرْنَاخَ.
\par 26 وَمِنْ سِبْطِ بَنِي يَسَّاكَرَ الرَّئِيسُ فَلطِيئِيلُ بْنُ عَزَّانَ.
\par 27 وَمِنْ سِبْطِ بَنِي أَشِيرَ الرَّئِيسُ أَخِيهُودُ بْنُ شَلُومِي.
\par 28 وَمِنْ سِبْطِ بَنِي نَفْتَالِي الرَّئِيسُ فَدَهْئِيلُ بْنُ عَمِّيهُودَ».
\par 29 هَؤُلاءِ هُمُ الذِينَ أَمَرَهُمُ الرَّبُّ أَنْ يَقْسِمُوا لِبَنِي إِسْرَائِيل فِي أَرْضِ كَنْعَانَ.

\chapter{35}

\par 1 ثُمَّ أَمَرَ الرَّبُّ مُوسَى فِي عَرَبَاتِ مُوآبَ عَلى أُرْدُنِّ أَرِيحَا:
\par 2 «أَوْصِ بَنِي إِسْرَائِيل أَنْ يُعْطُوا اللاوِيِّينَ مِنْ نَصِيبِ مُلكِهِمْ مُدُناً لِلسَّكَنِ وَمَرَاعِيَ لِلمُدُنِ حَوَاليْهَا تُعْطُونَ اللاوِيِّينَ.
\par 3 فَتَكُونُ المُدُنُ لهُمْ لِلسَّكَنِ وَمَرَاعِيَهَا لِبَهَائِمِهِمْ وَأَمْوَالِهِمْ وَلِسَائِرِ حَيَوَانَاتِهِمْ.
\par 4 وَمَرَاعِي المُدُنِ التِي تُعْطُونَ اللاوِيِّينَ تَكُونُ مِنْ سُورِ المَدِينَةِ إِلى جِهَةِ الخَارِجِ أَلفَ ذِرَاعٍ حَوَاليْهَا.
\par 5 فَتَقِيسُونَ مِنْ خَارِجِ المَدِينَةِ جَانِبَ الشَّرْقِ أَلفَيْ ذِرَاعٍ وَجَانِبَ الجَنُوبِ أَلفَيْ ذِرَاعٍ وَجَانِبَ الغَرْبِ أَلفَيْ ذِرَاعٍ وَجَانِبَ الشِّمَالِ أَلفَيْ ذِرَاعٍ وَتَكُونُ المَدِينَةُ فِي الوَسَطِ. هَذِهِ تَكُونُ لهُمْ مَرَاعِي المُدُنِ.
\par 6 وَالمُدُنُ التِي تُعْطُونَ اللاوِيِّينَ تَكُونُ سِتٌّ مِنْهَا مُدُناً لِلمَلجَإِ. تُعْطُونَهَا لِكَيْ يَهْرُبَ إِليْهَا القَاتِلُ. وَفَوْقَهَا تُعْطُونَ اثْنَتَيْنِ وَأَرْبَعِينَ مَدِينَةً.
\par 7 جَمِيعُ المُدُنِ التِي تُعْطُونَ اللاوِيِّينَ ثَمَانِي وَأَرْبَعُونَ مَدِينَةً مَعَ مَرَاعِيَهَا.
\par 8 وَالمُدُنُ التِي تُعْطُونَ مِنْ مُلكِ بَنِي إِسْرَائِيل مِنَ الكَثِيرِ تُكَثِّرُونَ وَمِنَ القَلِيلِ تُقَلِّلُونَ. كُلُّ وَاحِدٍ حَسَبَ نَصِيبِهِ الذِي مَلكَهُ يُعْطِي مِنْ مُدُنِهِ لِلاوِيِّينَ».
\par 9 وَأَمَرَ الرَّبُّ مُوسَى:
\par 10 «قُل لِبَنِي إِسْرَائِيل: إِنَّكُمْ عَابِرُونَ الأُرْدُنَّ إِلى أَرْضِ كَنْعَانَ.
\par 11 فَتُعَيِّنُونَ لأَنْفُسِكُمْ مُدُناً تَكُونُ مُدُنَ مَلجَأٍ لكُمْ لِيَهْرُبَ إِليْهَا القَاتِلُ الذِي قَتَل نَفْساً سَهْواً.
\par 12 فَتَكُونُ لكُمُ المُدُنُ مَلجَأً مِنَ الوَلِيِّ لِكَيْلا يَمُوتَ القَاتِلُ حَتَّى يَقِفَ أَمَامَ الجَمَاعَةِ لِلقَضَاءِ.
\par 13 وَالمُدُنُ التِي تُعْطُونَ تَكُونُ سِتَّ مُدُنِ مَلجَأٍ لكُمْ.
\par 14 ثَلاثاً مِنَ المُدُنِ تُعْطُونَ فِي عَبْرِ الأُرْدُنِّ وَثَلاثاً مِنَ المُدُنِ تُعْطُونَ فِي أَرْضِ كَنْعَانَ. مُدُنَ مَلجَأٍ تَكُونُ
\par 15 لِبَنِي إِسْرَائِيل وَلِلغَرِيبِ وَلِلمُسْتَوْطِنِ فِي وَسَطِهِمْ تَكُونُ هَذِهِ السِّتُّ المُدُنِ لِلمَلجَإِ لِكَيْ يَهْرُبَ إِليْهَا كُلُّ مَنْ قَتَل نَفْساً سَهْواً.
\par 16 «إِنْ ضَرَبَهُ بِأَدَاةِ حَدِيدٍ فَمَاتَ فَهُوَ قَاتِلٌ. إِنَّ القَاتِل يُقْتَلُ.
\par 17 وَإِنْ ضَرَبَهُ بِحَجَرِ يَدٍ مِمَّا يُقْتَلُ بِهِ فَمَاتَ فَهُوَ قَاتِلٌ. إِنَّ القَاتِل يُقْتَلُ.
\par 18 أَوْ ضَرَبَهُ بِأَدَاةِ يَدٍ مِنْ خَشَبٍ مِمَّا يُقْتَلُ بِهِ فَهُوَ قَاتِلٌ. إِنَّ القَاتِل يُقْتَلُ.
\par 19 وَلِيُّ الدَّمِ يَقْتُلُ القَاتِل. حِينَ يُصَادِفُهُ يَقْتُلُهُ.
\par 20 وَإِنْ دَفَعَهُ بِبُغْضَةٍ أَوْ أَلقَى عَليْهِ شَيْئاً بِتَعَمُّدٍ فَمَاتَ
\par 21 أَوْ ضَرَبَهُ بِيَدِهِ بِعَدَاوَةٍ فَمَاتَ فَإِنَّهُ يُقْتَلُ الضَّارِبُ لأَنَّهُ قَاتِلٌ. وَلِيُّ الدَّمِ يَقْتُلُ القَاتِل حِينَ يُصَادِفُهُ.
\par 22 وَلكِنْ إِنْ دَفَعَهُ بَغْتَةً بِلا عَدَاوَةٍ أَوْ أَلقَى عَليْهِ أَدَاةً مَا بِلا تَعَمُّدٍ
\par 23 أَوْ حَجَراً مَا مِمَّا يُقْتَلُ بِهِ بِلا رُؤْيَةٍ. أَسْقَطَهُ عَليْهِ فَمَاتَ وَهُوَ ليْسَ عَدُوّاً لهُ وَلا طَالِباً أَذِيَّتَهُ
\par 24 تَقْضِي الجَمَاعَةُ بَيْنَ القَاتِلِ وَبَيْنَ وَلِيِّ الدَّمِ حَسَبَ هَذِهِ الأَحْكَامِ.
\par 25 وَتُنْقِذُ الجَمَاعَةُ القَاتِل مِنْ يَدِ وَلِيِّ الدَّمِ وَتَرُدُّهُ الجَمَاعَةُ إِلى مَدِينَةِ مَلجَئِهِ التِي هَرَبَ إِليْهَا فَيُقِيمُ هُنَاكَ إِلى مَوْتِ الكَاهِنِ العَظِيمِ الذِي مُسِحَ بِالدُّهْنِ المُقَدَّسِ.
\par 26 وَلكِنْ إِنْ خَرَجَ القَاتِلُ مِنْ حُدُودِ مَدِينَةِ مَلجَئِهِ التِي هَرَبَ إِليْهَا
\par 27 وَوَجَدَهُ وَلِيُّ الدَّمِ خَارِجَ حُدُودِ مَدِينَةِ مَلجَئِهِ وَقَتَل وَلِيُّ الدَّمِ القَاتِل فَليْسَ لهُ دَمٌ
\par 28 لأَنَّهُ فِي مَدِينَةِ مَلجَئِهِ يُقِيمُ إِلى مَوْتِ الكَاهِنِ العَظِيمِ. وَأَمَّا بَعْدَ مَوْتِ الكَاهِنِ العَظِيمِ فَيَرْجِعُ القَاتِلُ إِلى أَرْضِ مُلكِهِ.
\par 29 «فَتَكُونُ هَذِهِ لكُمْ فَرِيضَةَ حُكْمٍ إِلى أَجْيَالِكُمْ فِي جَمِيعِ مَسَاكِنِكُمْ.
\par 30 كُلُّ مَنْ قَتَل نَفْساً فَعَلى فَمِ شُهُودٍ يُقْتَلُ القَاتِلُ. وَشَاهِدٌ وَاحِدٌ لا يَشْهَدْ عَلى نَفْسٍ لِلمَوْتِ.
\par 31 وَلا تَأْخُذُوا فِدْيَةً عَنْ نَفْسِ القَاتِلِ المُذْنِبِ لِلمَوْتِ بَل إِنَّهُ يُقْتَلُ.
\par 32 وَلا تَأْخُذُوا فِدْيَةً لِيَهْرُبَ إِلى مَدِينَةِ مَلجَئِهِ فَيَرْجِعَ وَيَسْكُنَ فِي الأَرْضِ بَعْدَ مَوْتِ الكَاهِنِ.
\par 33 لا تُدَنِّسُوا الأَرْضَ التِي أَنْتُمْ فِيهَا لأَنَّ الدَّمَ يُدَنِّسُ الأَرْضَ. وَعَنِ الأَرْضِ لا يُكَفَّرُ لأَجْلِ الدَّمِ الذِي سُفِكَ فِيهَا إِلا بِدَمِ سَافِكِهِ.
\par 34 وَلا تُنَجِّسُوا الأَرْضَ التِي أَنْتُمْ مُقِيمُونَ فِيهَا التِي أَنَا سَاكِنٌ فِي وَسَطِهَا. إِنِّي أَنَا الرَّبُّ سَاكِنٌ فِي وَسَطِ بَنِي إِسْرَائِيل».

\chapter{36}

\par 1 وَتَقَدَّمَ رُؤُوسُ الآبَاءِ مِنْ عَشِيرَةِ بَنِي جِلعَادَ بْنِ مَاكِيرَ بْنِ مَنَسَّى مِنْ عَشَائِرِ بَنِي يُوسُفَ: وَتَكَلمُوا قُدَّامَ مُوسَى وَقُدَّامَ رُؤَسَاءِ الآبَاءِ مِنْ بَنِي إِسْرَائِيل
\par 2 وَقَالُوا: «قَدْ أَمَرَ الرَّبُّ سَيِّدِي أَنْ يُعْطِيَ الأَرْضَ بِقِسْمَةٍ بِالقُرْعَةِ لِبَنِي إِسْرَائِيل. وَقَدْ أَمَرَ الرَّبُّ سَيِّدِي أَنْ يُعْطِيَ نَصِيبَ صَلُفْحَادَ أَخِينَا لِبَنَاتِهِ.
\par 3 فَإِنْ صِرْنَ نِسَاءً لأَحَدٍ مِنْ بَنِي أَسْبَاطِ بَنِي إِسْرَائِيل يُؤْخَذُ نَصِيبُهُنَّ مِنْ نَصِيبِ آبَائِنَا وَيُضَافُ إِلى نَصِيبِ السِّبْطِ الذِي صِرْنَ لهُ. فَمِنْ قُرْعَةِ نَصِيبِنَا يُؤْخَذُ.
\par 4 وَمَتَى كَانَ اليُوبِيلُ لِبَنِي إِسْرَائِيل يُضَافُ نَصِيبُهُنَّ إِلى نَصِيبِ السِّبْطِ الذِي صِرْنَ لهُ وَمِنْ نَصِيبِ سِبْطِ آبَائِنَا يُؤْخَذُ نَصِيبُهُنَّ».
\par 5 فَأَمَرَ مُوسَى بَنِي إِسْرَائِيل حَسَبَ قَوْلِ الرَّبِّ: «بِحَقٍّ تَكَلمَ سِبْطُ بَنِي يُوسُفَ.
\par 6 هَذَا مَا أَمَرَ بِهِ الرَّبُّ عَنْ بَنَاتِ صَلُفْحَادَ: مَنْ حَسُنَ فِي أَعْيُنِهِنَّ يَكُنَّ لهُ نِسَاءً وَلكِنْ لِعَشِيرَةِ سِبْطِ آبَائِهِنَّ يَكُنَّ نِسَاءً.
\par 7 فَلا يَتَحَوَّلُ نَصِيبٌ لِبَنِي إِسْرَائِيل مِنْ سِبْطٍ إِلى سِبْطٍ بَل يُلازِمُ بَنُو إِسْرَائِيل كُلُّ وَاحِدٍ نَصِيبَ سِبْطِ آبَائِهِ.
\par 8 وَكُلُّ بِنْتٍ وَرَثَتْ نَصِيباً مِنْ أَسْبَاطِ بَنِي إِسْرَائِيل تَكُونُ امْرَأَةً لِوَاحِدٍ مِنْ عَشِيرَةِ سِبْطِ أَبِيهَا لِيَرِثَ بَنُو إِسْرَائِيل كُلُّ وَاحِدٍ نَصِيبَ آبَائِهِ
\par 9 فَلا يَتَحَوَّل نَصِيبٌ مِنْ سِبْطٍ إِلى سِبْطٍ آخَرَ بَل يُلازِمُ أَسْبَاطُ بَنِي إِسْرَائِيل كُلُّ وَاحِدٍ نَصِيبَهُ».
\par 10 كَمَا أَمَرَ الرَّبُّ مُوسَى كَذَلِكَ فَعَلتْ بَنَاتُ صَلُفْحَادَ.
\par 11 فَصَارَتْ مَحْلةُ وَتِرْصَةُ وَحَجْلةُ وَمِلكَةُ وَنُوعَةُ بَنَاتُ صَلُفْحَادَ نِسَاءً لِبَنِي أَعْمَامِهِنَّ.
\par 12 صِرْنَ نِسَاءً مِنْ عَشَائِرِ بَنِي مَنَسَّى بْنِ يُوسُفَ فَبَقِيَ نَصِيبُهُنَّ فِي سِبْطِ عَشِيرَةِ أَبِيهِنَّ.
\par 13 هَذِهِ هِيَ الوَصَايَا وَالأَحْكَامُ التِي أَوْصَى بِهَا الرَّبُّ إِلى بَنِي إِسْرَائِيل عَنْ يَدِ مُوسَى فِي عَرَبَاتِ مُوآبَ عَلى أُرْدُنِّ أَرِيحَا.

\end{document}