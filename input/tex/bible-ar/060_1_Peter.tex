\begin{document}

\title{1 بطرس}


\chapter{1}

\par 1 بُطْرُسُ، رَسُولُ يَسُوعَ الْمَسِيحِ، إِلَى الْمُتَغَرِّبِينَ مِنْ شَتَاتِ بُنْتُسَ وَغَلاَطِيَّةَ وَكَبَّدُوكِيَّةَ وَأَسِيَّا وَبِيثِينِيَّةَ، الْمُخْتَارِينَ
\par 2 بِمُقْتَضَى عِلْمِ اللهِ الآبِ السَّابِقِ، فِي تَقْدِيسِ الرُّوحِ لِلطَّاعَةِ، وَرَشِّ دَمِ يَسُوعَ الْمَسِيحِ. لِتُكْثَرْ لَكُمُ النِّعْمَةُ وَالسَّلاَمُ.
\par 3 مُبَارَكٌ اللهُ أَبُو رَبِّنَا يَسُوعَ الْمَسِيحِ، الَّذِي حَسَبَ رَحْمَتِهِ الْكَثِيرَةِ وَلَدَنَا ثَانِيَةً لِرَجَاءٍ حَيٍّ، بِقِيَامَةِ يَسُوعَ الْمَسِيحِ مِنَ الأَمْوَاتِ،
\par 4 لِمِيرَاثٍ لاَ يَفْنَى وَلاَ يَتَدَنَّسُ وَلاَ يَضْمَحِلُّ، مَحْفُوظٌ فِي السَّمَاوَاتِ لأَجْلِكُمْ،
\par 5 أَنْتُمُ الَّذِينَ بِقُوَّةِ اللهِ مَحْرُوسُونَ، بِإِيمَانٍ، لِخَلاَصٍ مُسْتَعَدٍّ أَنْ يُعْلَنَ فِي الزَّمَانِ الأَخِيرِ.
\par 6 الَّذِي بِهِ تَبْتَهِجُونَ، مَعَ أَنَّكُمُ الآنَ - إِنْ كَانَ يَجِبُ - تُحْزَنُونَ يَسِيراً بِتَجَارِبَ مُتَنَوِّعَةٍ،
\par 7 لِكَيْ تَكُونَ تَزْكِيَةُ إِيمَانِكُمْ، وَهِيَ أَثْمَنُ مِنَ الذَّهَبِ الْفَانِي، مَعَ أَنَّهُ يُمْتَحَنُ بِالنَّارِ، تُوجَدُ لِلْمَدْحِ وَالْكَرَامَةِ وَالْمَجْدِ عِنْدَ اسْتِعْلاَنِ يَسُوعَ الْمَسِيحِ،
\par 8 الَّذِي وَإِنْ لَمْ تَرَوْهُ تُحِبُّونَهُ. ذَلِكَ وَإِنْ كُنْتُمْ لاَ تَرَوْنَهُ الآنَ لَكِنْ تُؤْمِنُونَ بِهِ، فَتَبْتَهِجُونَ بِفَرَحٍ لاَ يُنْطَقُ بِهِ وَمَجِيدٍ،
\par 9 نَائِلِينَ غَايَةَ إِيمَانِكُمْ خَلاَصَ النُّفُوسِ.
\par 10 الْخَلاَصَ الَّذِي فَتَّشَ وَبَحَثَ عَنْهُ أَنْبِيَاءُ، الَّذِينَ تَنَبَّأُوا عَنِ النِّعْمَةِ الَّتِي لأَجْلِكُمْ،
\par 11 بَاحِثِينَ أَيُّ وَقْتٍ أَوْ مَا الْوَقْتُ الَّذِي كَانَ يَدُلُّ عَلَيْهِ رُوحُ الْمَسِيحِ الَّذِي فِيهِمْ، إِذْ سَبَقَ فَشَهِدَ بِالآلاَمِ الَّتِي لِلْمَسِيحِ وَالأَمْجَادِ الَّتِي بَعْدَهَا.
\par 12 الَّذِينَ أُعْلِنَ لَهُمْ أَنَّهُمْ لَيْسَ لأَنْفُسِهِمْ، بَلْ لَنَا كَانُوا يَخْدِمُونَ بِهَذِهِ الأُمُورِ الَّتِي أُخْبِرْتُمْ بِهَا أَنْتُمُ الآنَ بِوَاسِطَةِ الَّذِينَ بَشَّرُوكُمْ فِي الرُّوحِ الْقُدُسِ الْمُرْسَلِ مِنَ السَّمَاءِ. الَّتِي تَشْتَهِي الْمَلاَئِكَةُ أَنْ تَطَّلِعَ عَلَيْهَا.
\par 13 لِذَلِكَ مَنْطِقُوا أَحْقَاءَ ذِهْنِكُمْ صَاحِينَ، فَأَلْقُوا رَجَاءَكُمْ بِالتَّمَامِ عَلَى النِّعْمَةِ الَّتِي يُؤْتَى بِهَا إِلَيْكُمْ عِنْدَ اسْتِعْلاَنِ يَسُوعَ الْمَسِيحِ.
\par 14 كَأَوْلاَدِ الطَّاعَةِ لاَ تُشَاكِلُوا شَهَوَاتِكُمُ السَّابِقَةَ فِي جَهَالَتِكُمْ،
\par 15 بَلْ نَظِيرَ الْقُدُّوسِ الَّذِي دَعَاكُمْ، كُونُوا أَنْتُمْ أَيْضاً قِدِّيسِينَ فِي كُلِّ سِيرَةٍ.
\par 16 لأَنَّهُ مَكْتُوبٌ: «كُونُوا قِدِّيسِينَ لأَنِّي أَنَا قُدُّوسٌ».
\par 17 وَإِنْ كُنْتُمْ تَدْعُونَ أَباً الَّذِي يَحْكُمُ بِغَيْرِ مُحَابَاةٍ حَسَبَ عَمَلِ كُلِّ وَاحِدٍ، فَسِيرُوا زَمَانَ غُرْبَتِكُمْ بِخَوْفٍ،
\par 18 عَالِمِينَ أَنَّكُمُ افْتُدِيتُمْ لاَ بِأَشْيَاءَ تَفْنَى، بِفِضَّةٍ أَوْ ذَهَبٍ، مِنْ سِيرَتِكُمُ الْبَاطِلَةِ الَّتِي تَقَلَّدْتُمُوهَا مِنَ الآبَاءِ،
\par 19 بَلْ بِدَمٍ كَرِيمٍ، كَمَا مِنْ حَمَلٍ بِلاَ عَيْبٍ وَلاَ دَنَسٍ، دَمِ الْمَسِيحِ،
\par 20 مَعْرُوفاً سَابِقاً قَبْلَ تَأْسِيسِ الْعَالَمِ، وَلَكِنْ قَدْ أُظْهِرَ فِي الأَزْمِنَةِ الأَخِيرَةِ مِنْ أَجْلِكُمْ،
\par 21 أَنْتُمُ الَّذِينَ بِهِ تُؤْمِنُونَ بِاللهِ الَّذِي أَقَامَهُ مِنَ الأَمْوَاتِ وَأَعْطَاهُ مَجْداً، حَتَّى إِنَّ إِيمَانَكُمْ وَرَجَاءَكُمْ هُمَا فِي اللهِ.
\par 22 طَهِّرُوا نُفُوسَكُمْ فِي طَاعَةِ الْحَقِّ بِالرُّوحِ لِلْمَحَبَّةِ الأَخَوِيَّةِ الْعَدِيمَةِ الرِّيَاءِ، فَأَحِبُّوا بَعْضُكُمْ بَعْضاً مِنْ قَلْبٍ طَاهِرٍ بِشِدَّةٍ.
\par 23 مَوْلُودِينَ ثَانِيَةً، لاَ مِنْ زَرْعٍ يَفْنَى، بَلْ مِمَّا لاَ يَفْنَى، بِكَلِمَةِ اللهِ الْحَيَّةِ الْبَاقِيَةِ إِلَى الأَبَدِ.
\par 24 لأَنَّ كُلَّ جَسَدٍ كَعُشْبٍ، وَكُلَّ مَجْدِ إِنْسَانٍ كَزَهْرِ عُشْبٍ. الْعُشْبُ يَبِسَ وَزَهْرُهُ سَقَطَ،
\par 25 وَأَمَّا كَلِمَةُ الرَّبِّ فَتَثْبُتُ إِلَى الأَبَدِ. وَهَذِهِ هِيَ الْكَلِمَةُ الَّتِي بُشِّرْتُمْ بِهَا.

\chapter{2}

\par 1 فَاطْرَحُوا كُلَّ خُبْثٍ وَكُلَّ مَكْرٍ وَالرِّيَاءَ وَالْحَسَدَ وَكُلَّ مَذَمَّةٍ،
\par 2 وَكَأَطْفَالٍ مَوْلُودِينَ الآنَ اشْتَهُوا اللَّبَنَ الْعَقْلِيَّ الْعَدِيمَ الْغِشِّ لِكَيْ تَنْمُوا بِهِ -
\par 3 إِنْ كُنْتُمْ قَدْ ذُقْتُمْ أَنَّ الرَّبَّ صَالِحٌ.
\par 4 الَّذِي إِذْ تَأْتُونَ إِلَيْهِ، حَجَراً حَيّاً مَرْفُوضاً مِنَ النَّاسِ، وَلَكِنْ مُخْتَارٌ مِنَ اللهِ كَرِيمٌ،
\par 5 كُونُوا أَنْتُمْ أَيْضاً مَبْنِيِّينَ كَحِجَارَةٍ حَيَّةٍ، بَيْتاً رُوحِيّاً، كَهَنُوتاً مُقَدَّساً، لِتَقْدِيمِ ذَبَائِحَ رُوحِيَّةٍ مَقْبُولَةٍ عِنْدَ اللهِ بِيَسُوعَ الْمَسِيحِ.
\par 6 لِذَلِكَ يُتَضَمَّنُ أَيْضاً فِي الْكِتَابِ: «هَئَنَذَا أَضَعُ فِي صِهْيَوْنَ حَجَرَ زَاوِيَةٍ مُخْتَاراً كَرِيماً، وَالَّذِي يُؤْمِنُ بِهِ لَنْ يُخْزَى».
\par 7 فَلَكُمْ أَنْتُمُ الَّذِينَ تُؤْمِنُونَ الْكَرَامَةُ، وَأَمَّا لِلَّذِينَ لاَ يُطِيعُونَ فَالْحَجَرُ الَّذِي رَفَضَهُ الْبَنَّاؤُونَ هُوَ قَدْ صَارَ رَأْسَ الزَّاوِيَةِ،
\par 8 وَحَجَرَ صَدْمَةٍ وَصَخْرَةَ عَثْرَةٍ. الَّذِينَ يَعْثُرُونَ غَيْرَ طَائِعِينَ لِلْكَلِمَةِ، الأَمْرُ الَّذِي جُعِلُوا لَهُ.
\par 9 وَأَمَّا أَنْتُمْ فَجِنْسٌ مُخْتَارٌ، وَكَهَنُوتٌ مُلُوكِيٌّ، أُمَّةٌ مُقَدَّسَةٌ، شَعْبُ اقْتِنَاءٍ، لِكَيْ تُخْبِرُوا بِفَضَائِلِ الَّذِي دَعَاكُمْ مِنَ الظُّلْمَةِ إِلَى نُورِهِ الْعَجِيبِ.
\par 10 الَّذِينَ قَبْلاً لَمْ تَكُونُوا شَعْباً، وَأَمَّا الآنَ فَأَنْتُمْ شَعْبُ اللهِ. الَّذِينَ كُنْتُمْ غَيْرَ مَرْحُومِينَ، وَأَمَّا الآنَ فَمَرْحُومُونَ.
\par 11 أَيُّهَا الأَحِبَّاءُ، أَطْلُبُ إِلَيْكُمْ كَغُرَبَاءَ وَنُزَلاَءَ أَنْ تَمْتَنِعُوا عَنِ الشَّهَوَاتِ الْجَسَدِيَّةِ الَّتِي تُحَارِبُ النَّفْسَ،
\par 12 وَأَنْ تَكُونَ سِيرَتُكُمْ بَيْنَ الأُمَمِ حَسَنَةً، لِكَيْ يَكُونُوا فِي مَا يَفْتَرُونَ عَلَيْكُمْ كَفَاعِلِي شَرٍّ يُمَجِّدُونَ اللهَ فِي يَوْمِ الاِفْتِقَادِ، مِنْ أَجْلِ أَعْمَالِكُمُ الْحَسَنَةِ الَّتِي يُلاَحِظُونَهَا.
\par 13 فَاخْضَعُوا لِكُلِّ تَرْتِيبٍ بَشَرِيٍّ مِنْ أَجْلِ الرَّبِّ. إِنْ كَانَ لِلْمَلِكِ فَكَمَنْ هُوَ فَوْقَ الْكُلِّ،
\par 14 أَوْ لِلْوُلاَةِ فَكَمُرْسَلِينَ مِنْهُ لِلاِنْتِقَامِ مِنْ فَاعِلِي الشَّرِّ، وَلِلْمَدْحِ لِفَاعِلِي الْخَيْرِ.
\par 15 لأَنَّ هَكَذَا هِيَ مَشِيئَةُ اللهِ أَنْ تَفْعَلُوا الْخَيْرَ فَتُسَكِّتُوا جَهَالَةَ النَّاسِ الأَغْبِيَاءِ.
\par 16 كَأَحْرَارٍ، وَلَيْسَ كَالَّذِينَ الْحُرِّيَّةُ عِنْدَهُمْ سُتْرَةٌ لِلشَّرِّ، بَلْ كَعَبِيدِ اللهِ.
\par 17 أَكْرِمُوا الْجَمِيعَ. أَحِبُّوا الإِخْوَةَ. خَافُوا اللهَ. أَكْرِمُوا الْمَلِكَ.
\par 18 أَيُّهَا الْخُدَّامُ، كُونُوا خَاضِعِينَ بِكُلِّ هَيْبَةٍ لِلسَّادَةِ، لَيْسَ لِلصَّالِحِينَ الْمُتَرَفِّقِينَ فَقَطْ، بَلْ لِلْعُنَفَاءِ أَيْضاً.
\par 19 لأَنَّ هَذَا فَضْلٌ إِنْ كَانَ أَحَدٌ مِنْ أَجْلِ ضَمِيرٍ نَحْوَ اللهِ يَحْتَمِلُ أَحْزَاناً مُتَأَلِّماً بِالظُّلْمِ.
\par 20 لأَنَّهُ أَيُّ مَجْدٍ هُوَ إِنْ كُنْتُمْ تُلْطَمُونَ مُخْطِئِينَ فَتَصْبِرُونَ؟ بَلْ إِنْ كُنْتُمْ تَتَأَلَّمُونَ عَامِلِينَ الْخَيْرَ فَتَصْبِرُونَ، فَهَذَا فَضْلٌ عِنْدَ اللهِ،
\par 21 لأَنَّكُمْ لِهَذَا دُعِيتُمْ. فَإِنَّ الْمَسِيحَ أَيْضاً تَأَلَّمَ لأَجْلِنَا، تَارِكاً لَنَا مِثَالاً لِكَيْ تَتَّبِعُوا خُطُواتِهِ.
\par 22 الَّذِي لَمْ يَفْعَلْ خَطِيَّةً، وَلاَ وُجِدَ فِي فَمِهِ مَكْرٌ،
\par 23 الَّذِي إِذْ شُتِمَ لَمْ يَكُنْ يَشْتِمُ عِوَضاً وَإِذْ تَأَلَّمَ لَمْ يَكُنْ يُهَدِّدُ بَلْ كَانَ يُسَلِّمُ لِمَنْ يَقْضِي بِعَدْلٍ.
\par 24 الَّذِي حَمَلَ هُوَ نَفْسُهُ خَطَايَانَا فِي جَسَدِهِ عَلَى الْخَشَبَةِ، لِكَيْ نَمُوتَ عَنِ الْخَطَايَا فَنَحْيَا لِلْبِرِّ. الَّذِي بِجَلْدَتِهِ شُفِيتُمْ.
\par 25 لأَنَّكُمْ كُنْتُمْ كَخِرَافٍ ضَالَّةٍ، لَكِنَّكُمْ رَجَعْتُمُ الآنَ إِلَى رَاعِي نُفُوسِكُمْ وَأُسْقُفِهَا.

\chapter{3}

\par 1 كَذَلِكُنَّ أَيَّتُهَا النِّسَاءُ كُنَّ خَاضِعَاتٍ لِرِجَالِكُنَّ، حَتَّى وَإِنْ كَانَ الْبَعْضُ لاَ يُطِيعُونَ الْكَلِمَةَ، يُرْبَحُونَ بِسِيرَةِ النِّسَاءِ بِدُونِ كَلِمَةٍ،
\par 2 مُلاَحِظِينَ سِيرَتَكُنَّ الطَّاهِرَةَ بِخَوْفٍ.
\par 3 وَلاَ تَكُنْ زِينَتُكُنَّ الزِّينَةَ الْخَارِجِيَّةَ مِنْ ضَفْرِ الشَّعْرِ وَالتَّحَلِّي بِالذَّهَبِ وَلِبْسِ الثِّيَابِ،
\par 4 بَلْ إِنْسَانَ الْقَلْبِ الْخَفِيَّ فِي الْعَدِيمَةِ الْفَسَادِ، زِينَةَ الرُّوحِ الْوَدِيعِ الْهَادِئِ، الَّذِي هُوَ قُدَّامَ اللهِ كَثِيرُ الثَّمَنِ.
\par 5 فَإِنَّهُ هَكَذَا كَانَتْ قَدِيماً النِّسَاءُ الْقِدِّيسَاتُ أَيْضاً الْمُتَوَكِّلاَتُ عَلَى اللهِ، يُزَيِّنَّ أَنْفُسَهُنَّ خَاضِعَاتٍ لِرِجَالِهِنَّ،
\par 6 كَمَا كَانَتْ سَارَةُ تُطِيعُ إِبْرَاهِيمَ دَاعِيَةً إِيَّاهُ «سَيِّدَهَا». الَّتِي صِرْتُنَّ أَوْلاَدَهَا، صَانِعَاتٍ خَيْراً، وَغَيْرَ خَائِفَاتٍ خَوْفاً الْبَتَّةَ.
\par 7 كَذَلِكُمْ أَيُّهَا الرِّجَالُ كُونُوا سَاكِنِينَ بِحَسَبِ الْفِطْنَةِ مَعَ الإِنَاءِ النِّسَائِيِّ كَالأَضْعَفِ، مُعْطِينَ إِيَّاهُنَّ كَرَامَةً كَالْوَارِثَاتِ أَيْضاً مَعَكُمْ نِعْمَةَ الْحَيَاةِ، لِكَيْ لاَ تُعَاقَ صَلَوَاتُكُمْ.
\par 8 وَالنِّهَايَةُ، كُونُوا جَمِيعاً مُتَّحِدِي الرَّأْيِ بِحِسٍّ وَاحِدٍ، ذَوِي مَحَبَّةٍ أَخَوِيَّةٍ، مُشْفِقِينَ، لُطَفَاءَ،
\par 9 غَيْرَ مُجَازِينَ عَنْ شَرٍّ بِشَرٍّ أَوْ عَنْ شَتِيمَةٍ بِشَتِيمَةٍ، بَلْ بِالْعَكْسِ مُبَارِكِينَ، عَالِمِينَ أَنَّكُمْ لِهَذَا دُعِيتُمْ لِكَيْ تَرِثُوا بَرَكَةً.
\par 10 لأَنَّ مَنْ أَرَادَ أَنْ يُحِبَّ الْحَيَاةَ وَيَرَى أَيَّاماً صَالِحَةً، فَلْيَكْفُفْ لِسَانَهُ عَنِ الشَّرِّ وَشَفَتَيْهِ أَنْ تَتَكَلَّمَا بِالْمَكْرِ،
\par 11 لِيُعْرِضْ عَنِ الشَّرِّ وَيَصْنَعِ الْخَيْرَ، لِيَطْلُبِ السَّلاَمَ وَيَجِدَّ فِي أَثَرِهِ.
\par 12 لأَنَّ عَيْنَيِ الرَّبِّ عَلَى الأَبْرَارِ وَأُذْنَيْهِ إِلَى طَلِبَتِهِمْ، وَلَكِنَّ وَجْهَ الرَّبِّ ضِدُّ فَاعِلِي الشَّرِّ.
\par 13 فَمَنْ يُؤْذِيكُمْ إِنْ كُنْتُمْ مُتَمَثِّلِينَ بِالْخَيْرِ؟
\par 14 وَلَكِنْ وَإِنْ تَأَلَّمْتُمْ مِنْ أَجْلِ الْبِرِّ فَطُوبَاكُمْ. وَأَمَّا خَوْفَهُمْ فَلاَ تَخَافُوهُ وَلاَ تَضْطَرِبُوا،
\par 15 بَلْ قَدِّسُوا الرَّبَّ الإِلَهَ فِي قُلُوبِكُمْ، مُسْتَعِدِّينَ دَائِماً لِمُجَاوَبَةِ كُلِّ مَنْ يَسْأَلُكُمْ عَنْ سَبَبِ الرَّجَاءِ الَّذِي فِيكُمْ بِوَدَاعَةٍ وَخَوْفٍ،
\par 16 وَلَكُمْ ضَمِيرٌ صَالِحٌ، لِكَيْ يَكُونَ الَّذِينَ يَشْتِمُونَ سِيرَتَكُمُ الصَّالِحَةَ فِي الْمَسِيحِ يُخْزَوْنَ فِي مَا يَفْتَرُونَ عَلَيْكُمْ كَفَاعِلِي شَرٍّ.
\par 17 لأَنَّ تَأَلُّمَكُمْ إِنْ شَاءَتْ مَشِيئَةُ اللهِ وَأَنْتُمْ صَانِعُونَ خَيْراً، أَفْضَلُ مِنْهُ وَأَنْتُمْ صَانِعُونَ شَرّاً.
\par 18 فَإِنَّ الْمَسِيحَ أَيْضاً تَأَلَّمَ مَرَّةً وَاحِدَةً مِنْ أَجْلِ الْخَطَايَا، الْبَارُّ مِنْ أَجْلِ الأَثَمَةِ، لِكَيْ يُقَرِّبَنَا إِلَى اللهِ، مُمَاتاً فِي الْجَسَدِ وَلَكِنْ مُحْيىً فِي الرُّوحِ،
\par 19 الَّذِي فِيهِ أَيْضاً ذَهَبَ فَكَرَزَ لِلأَرْوَاحِ الَّتِي فِي السِّجْنِ،
\par 20 إِذْ عَصَتْ قَدِيماً، حِينَ كَانَتْ أَنَاةُ اللهِ تَنْتَظِرُ مَرَّةً فِي أَيَّامِ نُوحٍ، إِذْ كَانَ الْفُلْكُ يُبْنَى، الَّذِي فِيهِ خَلَصَ قَلِيلُونَ، أَيْ ثَمَانِي أَنْفُسٍ بِالْمَاءِ.
\par 21 الَّذِي مِثَالُهُ يُخَلِّصُنَا نَحْنُ الآنَ، أَيِ الْمَعْمُودِيَّةُ. لاَ إِزَالَةُ وَسَخِ الْجَسَدِ، بَلْ سُؤَالُ ضَمِيرٍ صَالِحٍ عَنِ اللهِ بِقِيَامَةِ يَسُوعَ الْمَسِيحِ،
\par 22 الَّذِي هُوَ فِي يَمِينِ اللهِ، إِذْ قَدْ مَضَى إِلَى السَّمَاءِ، وَمَلاَئِكَةٌ وَسَلاَطِينُ وَقُوَّاتٌ مُخْضَعَةٌ لَهُ.

\chapter{4}

\par 1 فَإِذْ قَدْ تَأَلَّمَ الْمَسِيحُ لأَجْلِنَا بِالْجَسَدِ، تَسَلَّحُوا أَنْتُمْ أَيْضاً بِهَذِهِ النِّيَّةِ. فَإِنَّ مَنْ تَأَلَّمَ فِي الْجَسَدِ كُفَّ عَنِ الْخَطِيَّةِ،
\par 2 لِكَيْ لاَ يَعِيشَ أَيْضاً الزَّمَانَ الْبَاقِيَ فِي الْجَسَدِ لِشَهَوَاتِ النَّاسِ، بَلْ لِإِرَادَةِ اللهِ.
\par 3 لأَنَّ زَمَانَ الْحَيَاةِ الَّذِي مَضَى يَكْفِينَا لِنَكُونَ قَدْ عَمِلْنَا إِرَادَةَ الأُمَمِ، سَالِكِينَ فِي الدَّعَارَةِ وَالشَّهَوَاتِ، وَإِدْمَانِ الْخَمْرِ، وَالْبَطَرِ، وَالْمُنَادَمَاتِ، وَعِبَادَةِ الأَوْثَانِ الْمُحَرَّمَةِ،
\par 4 الأَمْرُ الَّذِي فِيهِ يَسْتَغْرِبُونَ أَنَّكُمْ لَسْتُمْ تَرْكُضُونَ مَعَهُمْ إِلَى فَيْضِ هَذِهِ الْخَلاَعَةِ عَيْنِهَا، مُجَدِّفِينَ.
\par 5 الَّذِينَ سَوْفَ يُعْطُونَ حِسَاباً لِلَّذِي هُوَ عَلَى اسْتِعْدَادٍ أَنْ يَدِينَ الأَحْيَاءَ وَالأَمْوَاتَ.
\par 6 فَإِنَّهُ لأَجْلِ هَذَا بُشِّرَ الْمَوْتَى أَيْضاً، لِكَيْ يُدَانُوا حَسَبَ النَّاسِ بِالْجَسَدِ، وَلَكِنْ لِيَحْيُوا حَسَبَ اللهِ بِالرُّوحِ.
\par 7 وَإِنَّمَا نِهَايَةُ كُلِّ شَيْءٍ قَدِ اقْتَرَبَتْ، فَتَعَقَّلُوا وَاصْحُوا لِلصَّلَوَاتِ.
\par 8 وَلَكِنْ قَبْلَ كُلِّ شَيْءٍ لِتَكُنْ مَحَبَّتُكُمْ بَعْضِكُمْ لِبَعْضٍ شَدِيدَةً، لأَنَّ الْمَحَبَّةَ تَسْتُرُ كَثْرَةً مِنَ الْخَطَايَا.
\par 9 كُونُوا مُضِيفِينَ بَعْضُكُمْ بَعْضاً بِلاَ دَمْدَمَةٍ.
\par 10 لِيَكُنْ كُلُّ وَاحِدٍ بِحَسَبِ مَا أَخَذَ مَوْهِبَةً يَخْدِمُ بِهَا بَعْضُكُمْ بَعْضاً، كَوُكَلاَءَ صَالِحِينَ عَلَى نِعْمَةِ اللهِ الْمُتَنَوِّعَةِ.
\par 11 إِنْ كَانَ يَتَكَلَّمُ أَحَدٌ فَكَأَقْوَالِ اللهِ، وَإِنْ كَانَ يَخْدِمُ أَحَدٌ فَكَأَنَّهُ مِنْ قُوَّةٍ يَمْنَحُهَا اللهُ، لِكَيْ يَتَمَجَّدَ اللهُ فِي كُلِّ شَيْءٍ بِيَسُوعَ الْمَسِيحِ، الَّذِي لَهُ الْمَجْدُ وَالسُّلْطَانُ إِلَى أَبَدِ الآبِدِينَ. آمِينَ.
\par 12 أَيُّهَا الأَحِبَّاءُ، لاَ تَسْتَغْرِبُوا الْبَلْوَى الْمُحْرِقَةَ الَّتِي بَيْنَكُمْ حَادِثَةٌ، لأَجْلِ امْتِحَانِكُمْ، كَأَنَّهُ أَصَابَكُمْ أَمْرٌ غَرِيبٌ،
\par 13 بَلْ كَمَا اشْتَرَكْتُمْ فِي آلاَمِ الْمَسِيحِ افْرَحُوا لِكَيْ تَفْرَحُوا فِي اسْتِعْلاَنِ مَجْدِهِ أَيْضاً مُبْتَهِجِينَ.
\par 14 إِنْ عُيِّرْتُمْ بِاسْمِ الْمَسِيحِ فَطُوبَى لَكُمْ، لأَنَّ رُوحَ الْمَجْدِ وَاللهِ يَحِلُّ عَلَيْكُمْ. أَمَّا مِنْ جِهَتِهِمْ فَيُجَدَّفُ عَلَيْهِ، وَأَمَّا مِنْ جِهَتِكُمْ فَيُمَجَّدُ.
\par 15 فَلاَ يَتَأَلَّمْ أَحَدُكُمْ كَقَاتِلٍ، أَوْ سَارِقٍ، أَوْ فَاعِلِ شَرٍّ، أَوْ مُتَدَاخِلٍ فِي أُمُورِ غَيْرِهِ.
\par 16 وَلَكِنْ إِنْ كَانَ كَمَسِيحِيٍّ فَلاَ يَخْجَلْ، بَلْ يُمَجِّدُ اللهَ مِنْ هَذَا الْقَبِيلِ.
\par 17 لأَنَّهُ الْوَقْتُ لاِبْتِدَاءِ الْقَضَاءِ مِنْ بَيْتِ اللهِ. فَإِنْ كَانَ أَوَّلاً مِنَّا، فَمَا هِيَ نِهَايَةُ الَّذِينَ لاَ يُطِيعُونَ إِنْجِيلَ اللهِ؟
\par 18 وَإِنْ كَانَ الْبَارُّ بِالْجَهْدِ يَخْلُصُ، فَالْفَاجِرُ وَالْخَاطِئُ أَيْنَ يَظْهَرَانِ؟
\par 19 فَإِذاً، الَّذِينَ يَتَأَلَّمُونَ بِحَسَبِ مَشِيئَةِ اللهِ فَلْيَسْتَوْدِعُوا أَنْفُسَهُمْ كَمَا لِخَالِقٍ أَمِينٍ فِي عَمَلِ الْخَيْرِ.

\chapter{5}

\par 1 أَطْلُبُ إِلَى الشُّيُوخِ الَّذِينَ بَيْنَكُمْ، أَنَا الشَّيْخَ رَفِيقَهُمْ، وَالشَّاهِدَ لآلاَمِ الْمَسِيحِ، وَشَرِيكَ الْمَجْدِ الْعَتِيدِ أَنْ يُعْلَنَ،
\par 2 ارْعَوْا رَعِيَّةَ اللهِ الَّتِي بَيْنَكُمْ نُظَّاراً، لاَ عَنِ اضْطِرَارٍ بَلْ بِالاِخْتِيَارِ، وَلاَ لِرِبْحٍ قَبِيحٍ بَلْ بِنَشَاطٍ،
\par 3 وَلاَ كَمَنْ يَسُودُ عَلَى الأَنْصِبَةِ بَلْ صَائِرِينَ أَمْثِلَةً لِلرَّعِيَّةِ،
\par 4 وَمَتَى ظَهَرَ رَئِيسُ الرُّعَاةِ تَنَالُونَ إِكْلِيلَ الْمَجْدِ الَّذِي لاَ يَبْلَى.
\par 5 كَذَلِكَ أَيُّهَا الأَحْدَاثُ اخْضَعُوا لِلشُّيُوخِ، وَكُونُوا جَمِيعاً خَاضِعِينَ بَعْضُكُمْ لِبَعْضٍ، وَتَسَرْبَلُوا بِالتَّوَاضُعِ، لأَنَّ اللهَ يُقَاوِمُ الْمُسْتَكْبِرِينَ، وَأَمَّا الْمُتَوَاضِعُونَ فَيُعْطِيهِمْ نِعْمَةً.
\par 6 فَتَوَاضَعُوا تَحْتَ يَدِ اللهِ الْقَوِيَّةِ لِكَيْ يَرْفَعَكُمْ فِي حِينِهِ،
\par 7 مُلْقِينَ كُلَّ هَمِّكُمْ عَلَيْهِ لأَنَّهُ هُوَ يَعْتَنِي بِكُمْ.
\par 8 اُصْحُوا وَاسْهَرُوا لأَنَّ إِبْلِيسَ خَصْمَكُمْ كَأَسَدٍ زَائِرٍ، يَجُولُ مُلْتَمِساً مَنْ يَبْتَلِعُهُ هُوَ.
\par 9 فَقَاوِمُوهُ رَاسِخِينَ فِي الإِيمَانِ، عَالِمِينَ أَنَّ نَفْسَ هَذِهِ الآلاَمِ تُجْرَى عَلَى إِخْوَتِكُمُ الَّذِينَ فِي الْعَالَمِ.
\par 10 وَإِلَهُ كُلِّ نِعْمَةٍ الَّذِي دَعَانَا إِلَى مَجْدِهِ الأَبَدِيِّ فِي الْمَسِيحِ يَسُوعَ، بَعْدَمَا تَأَلَّمْتُمْ يَسِيراً، هُوَ يُكَمِّلُكُمْ، وَيُثَبِّتُكُمْ، وَيُقَوِّيكُمْ، وَيُمَكِّنُكُمْ.
\par 11 لَهُ الْمَجْدُ وَالسُّلْطَانُ إِلَى أَبَدِ الآبِدِينَ. آمِينَ.
\par 12 بِيَدِ سِلْوَانُسَ الأَخِ الأَمِينِ، كَمَا أَظُنُّ كَتَبْتُ إِلَيْكُمْ بِكَلِمَاتٍ قَلِيلَةٍ وَاعِظاً وَشَاهِداً، أَنَّ هَذِهِ هِيَ نِعْمَةُ اللهِ الْحَقِيقِيَّةُ الَّتِي فِيهَا تَقُومُونَ.
\par 13 تُسَلِّمُ عَلَيْكُمُ الَّتِي فِي بَابِلَ الْمُخْتَارَةُ مَعَكُمْ، وَمَرْقُسُ ابْنِي.
\par 14 سَلِّمُوا بَعْضُكُمْ عَلَى بَعْضٍ بِقُبْلَةِ الْمَحَبَّةِ. سَلاَمٌ لَكُمْ جَمِيعِكُمُ الَّذِينَ فِي الْمَسِيحِ يَسُوعَ. آمِينَ.


\end{document}