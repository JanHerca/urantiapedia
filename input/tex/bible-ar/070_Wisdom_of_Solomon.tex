\begin{document}

\title{حكمة سليمان}


\chapter{1}

\par 1 أحبوا البر يا قضاة الأرض. فكروا في الرب بقلب طيب، واطلبوه ببساطة قلب.
\par 2 لأنه يُوجد لدى الذين لا يُجربونه، ويُظهر نفسه لمن لا يشككون فيه
\par 3 لأن الأفكار الملتوية تبتعد عن الله، وقدرته إذا امتُحنت توبخ الجاهلين
\par 4 لأن الحكمة لن تدخل النفس الخبيثة، ولن تسكن الجسد الخاضع للخطيئة
\par 5 لأن روح التأديب القدوس يهرب من الخداع، ويبتعد عن الأفكار التي لا فهم لها، ولا يثبت عندما يدخل الإثم
\par 6 لأن الحكمة روح محبة، ولا تبرئ المجدف من كلامه. لأن الله شاهد على كليتيه، وناظر صادق إلى قلبه، وسامع للسانه
\par 7 لأن روح الرب يملأ العالم، والذي يحتوي كل الأشياء لديه معرفة الصوت
\par 8 لذلك لا يمكن إخفاء من يتكلم بالظلم، ولا يمر عليه الانتقام عندما يعاقب
\par 9 لأنه سيُفتش في أفكار الأشرار، وسيأتي صوت أقوالهم إلى الرب لإظهار أعمالهم الشريرة
\par 10 لأن أُذُنُ الحَسَدِ تَسْمَعُ كُلَّ شَيْءٍ، وَضَجْجُ التَّذَمَّرِ لا يُخْفَى
\par 11 لذلك احذروا التذمر، فإنه لا ينفع، وامتنعوا عن الغيبة، لأنه لا توجد كلمة سرية إلا وتذهب سدى، والفم الكاذب يقتل النفس
\par 12 لا تطلبوا الموت في ضلال حياتكم، ولا تجلبوا على أنفسكم الهلاك بأعمال أيديكم
\par 13 لأن الله لم يصنع الموت، ولا يُسر بهلاك الأحياء
\par 14 لأنه خلق كل الأشياء لتكون، وكانت أجيال العالم سليمة، وليس فيها سمّ الهلاك، ولا مملكة الموت على الأرض
\par 15 (لأن البر خالد)
\par 16 لكن الناس الأشرار استدعوه بأعمالهم وأقوالهم، لأنهم لما ظنوا أنه صديقهم، استهلكوه، وقطعوا معه عهدًا، لأنهم مستحقون أن يشاركوا فيه

\chapter{2}

\par 1 قال الأشرار، وهم يتجادلون مع أنفسهم، ولكن ليس على حق: إن حياتنا قصيرة ومملة، وفي موت الإنسان لا يوجد علاج: ولم يُعرف عن أي رجل أنه عاد من القبر
\par 2 فإننا نولد في كل مغامرة: وسوف نكون فيما بعد كما لو أننا لم نكن أبدًا: لأن النفس في أنوفنا كالدخان، وشرارة صغيرة في تحريك قلبنا:
\par 3 والتي عند إطفائها، سوف يتحول جسدنا إلى رماد، وسوف تختفي روحنا مثل الهواء النقي،
\par 4 وسوف يُنسى اسمنا مع مرور الوقت، ولن يتذكر أحد أعمالنا، وسوف تزول حياتنا كأثر سحابة، وتنتشر مثل الضباب الذي طردته أشعة الشمس، وتغلب عليه حرارتها.
\par 5 لأن زمننا ظلٌّ زائل، وبعد نهايتنا لا رجوع، لأنه مختومٌ بإحكام، ولن يعود أحدٌ مرةً أخرى
\par 6 هيا بنا إذن، لنستمتع بالخيرات الحاضرة، ولنستخدم المخلوقات بسرعة كما في شبابنا
\par 7 لنملأ أنفسنا بالخمر الغالي والمراهم، ولا ندع زهرة الربيع تمر بنا
\par 8 دعونا نُتوِّج أنفسنا ببراعم الورد، قبل أن تذبل:
\par 9 لا يترك أحد منا نصيبه من شهواتنا، ولنترك علامات فرحنا في كل مكان، لأن هذا هو نصيبنا، وهذا هو نصيبنا.
\par 10 دعونا نظلم الرجل الصالح الفقير، ولا نشفق على الأرملة، ولا نحترم شيب الشيوخ
\par 11 لتكن قوتنا قانون العدل: لأن ما هو ضعيف لا قيمة له
\par 12 لذلك فلنكمن للبار، لأنه ليس في دورنا، وهو نقيض أفعالنا: يوبخنا على مخالفتنا للشريعة، ويعترض على عارنا بسبب تجاوزاتنا في تعليمنا
\par 13 يزعم أن لديه معرفة الله، ويسمي نفسه ابن الرب
\par 14 لقد خُلِقَ لتوبيخ أفكارنا.
\par 15 إنه لأمرٌ مؤلمٌ علينا حتى أن نراه: لأن حياته ليست كحياة الآخرين، وطرقه مختلفة
\par 16 يُنظر إلينا على أننا مُزيَّفون: فهو يمتنع عن طرقنا كما عن القذارة، ويُعلن أن نهاية البار هي أن يكون مباركًا، ويفتخر بأن الله هو أبوه
\par 17 دعونا نرى ما إذا كانت كلماته صحيحة: ودعونا نثبت ما سيحدث في نهايته
\par 18 لأنه إن كان البار ابن الله، فهو يعينه وينقذه من أيدي أعدائه
\par 19 فلنمتحنه بالكراهية والتعذيب، لنعرف وداعته، ونختبر صبره
\par 20 فلنحكم عليه بموتة مخزية، لأنه بأقواله سيُحترم
\par 21 مثل هذه الأمور التي تخيلوها، فانخدعوا بها، لأن شرهم قد أعماهم
\par 22 أما أسرار الله فلم يعرفوها، ولم يرجوا أجر البر، ولم يعرفوا جزاء النفوس التي لا لوم عليها
\par 23 لأن الله خلق الإنسان ليكون خالدًا، وجعله صورة أبديته
\par 24 ولكن بحسد إبليس دخل الموت إلى العالم، والذين يتمسكون بجانبه يجدونه

\chapter{3}

\par 1 أما نفوس الأبرار فهي في يد الله، فلا يمسها عذاب
\par 2 في نظر الجهال، بدوا وكأنهم يموتون: ويُعتبر رحيلهم بؤسًا،
\par 3 "ويذهبون عنا إلى الهلاك، وأما هم في سلام."
\par 4 فرغم أنهم يُعاقَبون في نظر الناس، إلا أن رجاءهم مملوء بالخلود
\par 5 وبعد أن يُؤدَّبوا قليلًا، سينالون مكافأة عظيمة: لأن الله امتحنهم، ووجدهم أهلًا لنفسه
\par 6 امتحنهم كالذهب في الكوة، وقبلهم محرقة
\par 7 وفي وقت افتقادهم يلمعون، ويركضون جيئة وذهابًا كالشرار بين القش
\par 8 فيحكمون الأمم، ويتسلطون على الشعوب، ويملك ربهم إلى الأبد
\par 9 من يتوكل عليه سيفهم الحق، ومن يكون أمينًا في المحبة يمكث معه، لأن النعمة والرحمة لقديسيه، وهو يهتم بمختاريه
\par 10 أما الأشرار فسيُعاقَبون حسب أفكارهم، الذين أهملوا الصديق وتركوا الرب
\par 11 لأن من يحتقر الحكمة والتأديب فهو شقي، ورجاءه باطل، وأتعابه غير مثمرة، وأعماله غير نافعة
\par 12 نساؤهم حمقاوات، وأولادهم أشرار:
\par 13 ذريتهم ملعونة. لذلك طوبى للعاقر التي لم تعرف فراش الخطيئة، لأنها ستُثمر عند زيارة النفوس.
\par 14 وطوبى للخصي الذي لم يفعل بيديه إثمًا، ولم يفكر في أمور رديئة ضد الله. لأنه سيُعطى عطية الإيمان الخاصة، وميراثًا في هيكل الرب أكثر قبولًا لعقله
\par 15 لأن ثمرة الأعمال الصالحة مجيدة، وأصل الحكمة لا يزول أبدًا
\par 16 أما أولاد الزناة، فلن يبلغوا كمالهم، وبذرة فراش الظلم ستُقتلع
\par 17 فإنهم وإن عاشوا طويلاً، فلن يُعتد بهم، وستكون آخرتهم بلا شرف
\par 18 أو إذا ماتوا سريعًا، فلا رجاء لهم ولا عزاء في يوم المحنة
\par 19 لأن نهاية الجيل الظالم رهيبة.

\chapter{4}

\par 1 من الأفضل ألا يكون لديك أطفال، وأن تكون لديك فضيلة، لأن ذكرها خالد، لأنها معروفة عند الله وعند الناس
\par 2 عندما يكون موجودًا، يتخذه الناس قدوة؛ وعندما يرحل، يشتهونه: يرتدي تاجًا، وينتصر إلى الأبد، بعد أن حقق النصر، ويسعى للحصول على مكافآت نقية
\par 3 لكن نسل الأشرار المتكاثر لن ينجح، ولن يتأصل في زلات غير شرعية، ولن يضع أساسًا ثابتًا
\par 4 فإنها وإن كانت تزدهر في الفروع إلى حين، ولكنها لا تقف في النهاية، فسوف تهزها الريح، ومن خلال قوة الرياح سيتم اقتلاعها.
\par 5 والأغصان الناقصة تنكسر، وثمرها غير صالح، وغير ناضج للأكل، بل لا يصلح للشيء.
\par 6 فإن الأطفال المولودين من فراش غير شرعي هم شهود على الشر ضد والديهم في محاكمتهم.
\par 7 ولكن حتى لو منع الموت الصديق فإنه سيكون في راحة.
\par 8 فالعمر المحترم ليس ما يقاس بطول الزمن، ولا ما يقاس بعدد السنين
\par 9 لكن الحكمة هي الشيب عند الناس، والحياة النقية هي الشيخوخة
\par 10 أرضى الله وكان محبوبًا لديه، حتى إنه انتقل بين الخطاة
\par 11 نعم، لقد أُخذ بسرعة، لئلا يُغير الشر فهمه، أو يُخدع روحه
\par 12 لأن سحر الشقاوة يحجب الأمور الصادقة، وتيه الشهوة يقوض العقل البسيط
\par 13 إذ كُمِّل في وقت قصير، أتمّ زمانًا طويلًا:
\par 14 لأن نفسه رضيت أمام الرب، لذلك أسرع ليأخذه من بين الأشرار.
\par 15 هذا ما رآه الناس ولم يفهموه، ولم يضعوه في أذهانهم، أن نعمته ورحمته مع قديسيه، وأنه يحترم مختاريه
\par 16 هكذا يدين البار الميت الأشرار الأحياء، والشباب الذي يكتمل سريعًا السنين الكثيرة وشيخوخة الأشرار
\par 17 لأنهم سينظرون نهاية الحكيم، ولن يفهموا ماذا قضى الله عليه في مشورته، ولماذا جعله الرب آمنًا
\par 18 سيرونه ويحتقرونه، لكن الله سيسخر منهم، وسيكونون بعد ذلك جثثًا دنيئة، وعارًا بين الأموات إلى الأبد
\par 19 لأنه سيمزقهم ويطرحهم أرضًا، فيصبحون صامتين، وسيهزهم من الأساس، وسيخربون تمامًا، ويكونون في حزن، وسيبيد ذكرهم
\par 20 وعندما يُبلغون عن خطاياهم، يأتون خائفين، وآثامهم تُقنعهم في وجوههم

\chapter{5}

\par 1 حينئذٍ يقف البار بجرأة عظيمة أمام وجه الذين أذلوه ولم يحسبوا أتعابه حسابًا
\par 2 عندما يرون ذلك، سيضطربون خوفًا شديدًا، وسيُدهشون من غرابة خلاصه، الذي يفوق كل ما كانوا ينتظرونه
\par 3 فيقولون في أنفسهم وهم يتوبون ويتأوهون من ضيق الروح: هذا هو الذي كان لنا أحيانًا سخريةً ومثلًا للعار
\par 4 نحن الحمقى حسبنا حياته جنونًا، ونهايته بلا شرف
\par 5 كيف يُحصى بين أبناء الله، ويكون نصيبه بين القديسين!
\par 6 لذلك ضللنا عن طريق الحق، ولم يُشرق علينا نور البر، ولم تُشرق علينا شمس البر
\par 7 لقد تعبنا في طريق الشر والهلاك، نعم، عبرنا قفارًا لا طريق فيها، وأما طريق الرب فلم نعرفه
\par 8 ماذا نفعنا الكبرياء؟ وماذا جلب لنا الغنى مع افتخارنا؟
\par 9 لقد مرت كل هذه الأشياء مثل الظل، وكرسالة مرت مسرعة؛
\par 10 ومثل سفينة تمر فوق أمواج الماء، وعندما تمر، لا يمكن العثور على أثر لها، ولا مسار عارضة السفينة في الأمواج؛
\par 11 أو كما لو أن طائرًا يطير في الهواء، فلا توجد علامة على طريقه، لكن الهواء الخفيف الذي يضربه جناحاه وينفصل مع ضجيجهما وحركتهما العنيفة، يمر، ولا يوجد بعد ذلك أي علامة على المكان الذي ذهب إليه؛
\par 12 أو كما هو الحال عندما يُطلق سهم على هدف، فإنه يشق الهواء، والذي يعود على الفور إلى مكانه، بحيث لا يستطيع الإنسان أن يعرف من أين مر:
\par 13 هكذا نحن أيضًا، بمجرد ولادتنا، بدأنا نقترب من نهايتها، ولم تكن لدينا أي علامة على الفضيلة لنظهرها؛ بل استهلكنا في شرورنا
\par 14 لأن رجاء التقي كالغبار الذي تذروه الريح، وكزبد رقيق تذروه العاصفة، وكدخان يتناثر هنا وهناك مع العاصفة، فيزول كذكرى ضيف لم يمكث إلا يومًا واحدًا
\par 15 أما الصديق فيحيا إلى الأبد، وأجره عند الرب، وعنايته عند العلي
\par 16 لذلك سينالون ملكوتًا مجيدًا، وتاجًا جميلًا من يد الرب، لأنه بيمينه يسترهم، وبذراعه يحميهم
\par 17 سيأخذ لنفسه غيرته ليصنع درعًا كاملاً، ويجعل المخلوق سلاحه للانتقام من أعدائه
\par 18 يلبس البر درعًا، والحق بدلًا من خوذة
\par 19 يتخذ القداسة درعًا لا يُقهر.
\par 20 فيحد غضبه الشديد سيفًا، ويقاتل العالم معه ضد الجهلاء.
\par 21 حينئذٍ ستنطلق الصواعق الموجهة إلى الخارج؛ ومن السحب، كما لو كانت من قوس مرسوم جيدًا، ستطير إلى الهدف
\par 22 ويقذف عليهم بردٌ مملوءٌ غضبًا كأنه من قوس حجري، وتثور عليهم مياه البحر، وتغرقهم السيول بقسوة
\par 23 نعم، ستقف ضدهم ريح شديدة، وتدفعهم بعيدًا كالعاصفة: وهكذا يُخرب الإثم الأرض كلها، ويقلب سوء المعاملة عروش الأقوياء

\chapter{6}

\par 1 فاسمعوا أيها الملوك وافهموا. تعلموا أيها قضاة أقاصي الأرض
\par 2 أصغوا يا حكام الشعب، وافتخروا في كثرة الأمم
\par 3 لأنكم قد أُعطيتم سلطانًا من الرب، وسلطانًا من العلي، الذي سيختبر أعمالكم، ويفحص مشورتكم
\par 4 لأنكم وأنتم خدام ملكوته، لم تحكموا بالاستقامة، ولم تحفظوا الشريعة، ولم تسلكوا في مشورة الله
\par 5 سيأتي عليكِ مروعًا وسريعًا، لأن دينونة قاسية تكون على الذين في المناصب العليا
\par 6 لأن الرحمة ستغفر سريعًا للأضعف، أما الرجال الأقوياء فسوف يتعذبون بشدة.
\par 7 لأن من هو سيد الجميع لا يخاف من شخص أحد، ولا يهاب عظمة أحد: لأنه خلق الصغير والكبير، ويهتم بالجميع على حد سواء
\par 8 ولكن ستأتي بلاءٌ شديدٌ على الأقوياء.
\par 9 إليكم أيها الملوك أتكلم لكي تتعلموا الحكمة ولا ترتدوا.
\par 10 لأن الذين يحفظون القداسة يُدانون قديسين، والذين تعلموا مثل هذه الأمور سيجدون ما يجيبون به
\par 11 لذلك ضعوا عاطفتكم على كلماتي؛ ارغبوا فيها، وسوف تتعلمون
\par 12 الحكمة مجيدة، ولا تذبل أبدًا: نعم، يسهل رؤيتها لمن يحبونها، وتوجد لمن يطلبونها
\par 13 إنها تمنع أولئك الذين يرغبون فيها، من خلال تعريفهم بنفسها أولاً
\par 14 من يبكر إليها فلن يشق عليه مشقة كبيرة، لأنه يجدها جالسة على أبوابه
\par 15 لذلك فإن التفكير فيها هو كمال الحكمة، ومن يسهر عليها سرعان ما يفقد الهم
\par 16 لأنها تجوب بحثًا عن أولئك الذين يستحقونها، وتُظهر لهم قبولها في الطرق، وتُقابلهم في كل فكر
\par 17 لأن بدايتها الحقيقية هي الرغبة في التأديب، والاهتمام بالتأديب هو المحبة؛
\par 18 والمحبة هي حفظ شرائعها، والاهتمام بشرائعها هو ضمان عدم الفساد
\par 19 والخلود يقربنا من الله:
\par 20 لذلك فإن الرغبة في الحكمة تؤدي إلى الملكوت.
\par 21 إن كانت مسرتكم في العروش والصولجانات، يا ملوك الشعوب، فأكرموا الحكمة لكي تحكموا إلى الأبد
\par 22 أما الحكمة، فما هي وكيف خرجت، فسأخبركم بها، ولن أخفي عنكم الأسرار، بل سأطلبها من بداية ميلادها، وأُظهر معرفتها، ولن أتجاوز الحقيقة
\par 23 ولا أذهب بحسدٍ مُستهلك، لأن مثل هذا الرجل لن يكون له رفقةٌ مع الحكمة
\par 24 لكن كثرة الحكماء هي رفاهية العالم، والملك الحكيم هو سند الشعب
\par 25 فاقبلوا إذن التعليم من خلال كلماتي، فينفعكم ذلك

\chapter{7}

\par 1 أنا أيضًا إنسان فانٍ، مثل الجميع، وذرية من خُلِقَ أولًا من الأرض،
\par 2 وفي بطن أمي خُلقت لحمًا في مدة عشرة أشهر، مضغوطًا بالدم، من نسل الإنسان، واللذة التي تأتي مع النوم
\par 3 وعندما وُلدت، استنشقت الهواء العادي، وسقطت على الأرض، التي هي من طبيعة مماثلة، وكان الصوت الأول الذي نطقته يبكي، كما يفعل الآخرون
\par 4 لقد رُضعتُ في قماط، وكان ذلك مع هموم.
\par 5 لأنه ليس هناك ملك له بداية ميلاد أخرى.
\par 6 لكل إنسان مدخل واحد إلى الحياة، ومخرج واحد مثله
\par 7 لذلك صليت فأعطيت فهماً. دعوت الله فأتى إلي روح الحكمة.
\par 8 فضّلتها على الصولجانات والعروش، ولم أعتبر الغنى شيئًا مقارنةً بها
\par 9 ولا يُقارن بها أي حجر كريم، لأن كل الذهب عندها كقليل من الرمل، والفضة تُحسب طينًا أمامها
\par 10 أحببتها فوق الصحة والجمال، واخترت أن أمتلكها بدلًا من النور: لأن النور الذي ينبعث منها لا ينطفئ أبدًا
\par 11 جاءت إليّ معها كل الأشياء الجيدة، وثروات لا تُحصى بين يديها
\par 12 وفرحتُ بهم جميعًا، لأن الحكمة تسبقهم، ولم أكن أعلم أنها أمهم
\par 13 لقد تعلمتها بجد، وأتواصل معها بسخاء: أنا لا أخفي ثرواتها
\par 14 لأنها كنزٌ للبشر لا ينضب أبدًا: ومن يستخدمها يصبح صديقًا لله، إذ يُمدح على المواهب التي تأتي من التعلم
\par 15 لقد منحني الله أن أتكلم كما أريد، وأن أتصور كما يليق بالأشياء التي أُعطيت لي، لأنه هو الذي يقود إلى الحكمة، ويرشد الحكماء
\par 16 لأن في يده نحن وكلماتنا، وكل الحكمة ومعرفة الصنعة
\par 17 لأنه أعطاني معرفة مؤكدة بالأشياء الموجودة، أي معرفة كيفية خلق العالم، وعمل العناصر:
\par 18 بداية الأزمنة ونهايتها ومنتصفها: تغيرات دوران الشمس، وتغير الفصول:
\par 19 دوائر السنين، ومواقع النجوم:
\par 20 طبيعة الكائنات الحية، وغضب الوحوش البرية: عنف الرياح، ومنطق البشر: تنوع النباتات وفضائل الجذور:
\par 21 وكل ما هو سر أو ظاهر، فأنا أعرفه.
\par 22 "لأن الحكمة التي هي عاملة كل شيء هي التي علمتني، ففيها روح فهم قدوس واحد، متعدد، دقيق، حي، نقي، بلا دنس، واضح، غير قابل للأذى، يحب الخير سريعًا، الذي لا يمكن تركه، مستعد لفعل الخير،
\par 23 لطيف مع الإنسان، ثابت، واثق، متحرر من الهموم، يمتلك كل القوة، يشرف على كل الأشياء، ويجتاز كل الفهم، أرواح نقية، وأكثر دقة
\par 24 لأن الحكمة أكثر تأثيرًا من أي حركة: فهي تمر وتجتاز كل الأشياء بسبب نقائها
\par 25 لأنها نسمة قوة الله، وتأثير نقي ينبع من مجد القدير: لذلك لا يمكن لأي شيء نجس أن يسقط فيها
\par 26 لأنها سطوع النور الأبدي، والمرآة النقية لقوة الله، وصورة صلاحه
\par 27 ولأنها واحدة، فهي قادرة على كل شيء. وببقائها في ذاتها، فإنها تجعل كل شيء جديدًا. وفي كل العصور، تدخل في النفوس المقدسة، وتجعلها أصدقاء لله وأنبياء
\par 28 لأن الله لا يحب أحدًا إلا من يسلك مع الحكمة.
\par 29 فإنها أجمل من الشمس، وفوق كل رتبة من النجوم، وإذا قورنت بالنور وجدته أمامه.
\par 30 فبعد هذا يأتي الليل، لكن الرذيلة لن تتغلب على الحكمة

\chapter{8}

\par 1 تمتد الحكمة من طرف إلى آخر بقوة، وتُرتب كل الأشياء بلطف
\par 2 أحببتها، وبحثت عنها منذ شبابي، وتمنيتُ أن أجعلها زوجتي، وكنتُ عاشقًا لجمالها
\par 3 في معرفتها بالله، فإنها تُعظم نبلها: نعم، لقد أحبها رب كل شيء بنفسه
\par 4 لأنها مطلعة على أسرار معرفة الله، ومحبة لأعماله
\par 5 إذا كانت الثروة من الأشياء المرغوبة في هذه الحياة، فما هو أغنى من الحكمة التي تعمل كل شيء؟
\par 6 وإذا كانت الحكمة تعمل، فمن منا أذكى منها في العمل؟
\par 7 وإن أحب الرجل البر، فإن أعمالها فضائل، لأنها تُعلّم التعفف والحكمة، والعدل والشجاعة، وهي أمور لا يمكن أن يكون لها شيء أنفع في حياتها
\par 8 إذا رغب الرجل في اكتساب خبرة كبيرة، فإنها تعرف الأشياء القديمة، وتخمّن بشكل صحيح ما سيأتي: إنها تعرف دقائق الخطب، ويمكنها تفسير الجمل الغامضة: إنها تتنبأ بالعلامات والعجائب، وأحداث الفصول والأزمنة
\par 9 لذلك عزمت على اصطحابها إليّ لتعيش معي، عالمًا أنها ستكون مرشدة لي في الأمور الجيدة، وعزاءً لي في الهموم والأحزان
\par 10 من أجلها، سيكون لي تقدير بين الجموع، وإكرام لدى الشيوخ، مع أنني شاب
\par 11 سأُوجَد سريع الغرور في الحكم، وسأُعجَب في نظر العظماء
\par 12 عندما أمسك لساني، ينتظرون راحتي، وعندما أتحدث، ينصتون إليّ جيدًا: إذا تكلمت كثيرًا، يضعون أيديهم على أفواههم
\par 13 علاوة على ذلك، سأنال بواسطتها الخلود، وأترك ​​ورائي ذكرى أبدية لمن يأتون بعدي
\par 14 أُنظِّمُ الشُّعوبَ، وَتَخْضَعُ لِي الأُمَمُ
\par 15 سيخاف الطغاة المروعون بمجرد سماعهم عني؛ وسأكون صالحًا بين الجموع، وشجاعًا في الحرب
\par 16 بعد أن أدخل بيتي، سأرتاح معها؛ لأن حديثها ليس فيه مرارة، والعيش معها ليس فيه حزن، بل مرح وفرح
\par 17 الآن، عندما فكرت في هذه الأمور في نفسي، وتأملتها في قلبي، كيف أن التحالف مع الحكمة هو الخلود؛
\par 18 وإنه لمن دواعي سروري الكبير أن أحظى بصداقتها؛ وفي أعمال يديها ثروات لا حصر لها؛ وفي ممارسة المشورة معها الحكمة؛ وفي الحديث معها، تقرير جيد؛ لقد بحثت عن كيفية أخذها إلي
\par 19 لأني كنت طفلاً ذكياً، وروحي طيبة.
\par 20 بل بالحري، وأنا صالح، أتيت إلى جسد غير نجس.
\par 21 ومع ذلك، عندما أدركت أنني لا أستطيع الحصول عليها بطريقة أخرى، إلا إذا أعطاني الله إياها؛ وكان من الحكمة أيضًا أن أعرف لمن كانت هذه الهدية؛ صليت إلى الرب، وتوسلت إليه، ومن كل قلبي قلت،

\chapter{9}

\par 1 يا إله آبائي، ورب الرحمة، الذي صنعت كل شيء بكلمتك،
\par 2 وقدرتَ الإنسانَ بحكمتكَ، ليكونَ له سلطانٌ على المخلوقاتِ التي خلقتَها،
\par 3 وأحكم العالم بالعدل والاستقامة، وأجرِ الحكم بقلب مستقيم
\par 4 أعطني الحكمة، أنا الجالس بجانب عرشك، ولا ترفضني من بين أبنائك
\par 5 لأني أنا عبدك وابن أمتك، رجل ضعيف، قصير العمر، وصغير على فهم الحكم والشرائع
\par 6 فرغم أن الإنسان لن يكون كاملاً بين أبناء البشر، إلا أنه إن لم تكن حكمتك معه، فلن يُعتد به
\par 7 لقد اخترتني ملكًا على شعبك، وقاضيًا لأبنائك وبناتك
\par 8 لقد أمرتني أن أبني هيكلاً على جبل قدسك، ومذبحاً في المدينة التي تسكن فيها، على صورة المسكن المقدس الذي أعددته منذ البدء
\par 9 وكانت معك الحكمة، التي تعرف أعمالك، وكانت حاضرة عندما خلقت العالم، وعرفت ما هو مرضي أمامك ومستقيم في وصاياك
\par 10 أرسلها من سماواتك المقدسة، ومن عرش مجدك، حتى وهي حاضرة تعمل معي، لأعرف ما يرضيك
\par 11 لأنها تعرف كل شيء وتفهمه، وستقودني بعقلانية في أعمالي، وتحفظني في قوتها
\par 12 هكذا تكون أعمالي مقبولة، وحينئذ أحكم على شعبك بالعدل، وأكون مستحقًا للجلوس على كرسي أبي
\par 13 لأنه من هو الإنسان الذي يستطيع أن يعرف مشورة الله؟ أو من يستطيع أن يفكر ما هي مشيئة الرب؟
\par 14 لأن أفكار البشر الفانين بائسة، ومكائدنا غير مؤكدة
\par 15 لأن الجسد الفاسد يثقل كاهل النفس، والمسكن الترابي يثقل كاهل العقل الذي يتأمل في أشياء كثيرة
\par 16 ونادرًا ما نُخمن الأشياء التي على الأرض بشكل صحيح، ومع الجهد نجد الأشياء التي أمامنا: إلا الأشياء التي في السماء، فمن الذي بحث؟
\par 17 ومن عرف مشورتك، إن لم تُعطِ حكمة، وتُرسل روحك القدوس من العلاء؟
\par 18 لأنه هكذا أُصلِحَت طرق الذين عاشوا على الأرض، وتعلَّم الناس ما يُرضيك، وخلصوا بالحكمة

\chapter{10}

\par 1 لقد حفظت أول أب للعالم، الذي خُلق وحيدًا، وأخرجته من سقوطه،
\par 2 وأعطاه القدرة على حكم كل شيء.
\par 3 ولكن عندما ابتعد عنها الأشرار بغضبه، هلك هو أيضًا في الغضب الذي قتل به أخاه.
\par 4 من أجل هذا السبب غرقت الأرض بالطوفان، فحفظتها الحكمة مرة أخرى، وأرشدت مسار الصالحين في قطعة من الخشب ذات القيمة الصغيرة.
\par 5 علاوة على ذلك، عندما خجلت الأمم في مؤامرتهم الشريرة، اكتشفت الصديق، وحفظته بلا لوم أمام الله، وحافظت عليه قويًا ضد رحمته تجاه ابنه
\par 6 عندما هلك الأشرار، أنقذت الرجل الصالح، الذي هرب من النار التي سقطت على المدن الخمس
\par 7 الذي يشهد على شره حتى يومنا هذا، الأرض الخربة التي يدخن أهلها، والنباتات التي لا تنضج أبدًا: وعمود الملح القائم هو نصب تذكاري للنفس الكافرة
\par 8 لأنهم لم يهتموا بالحكمة، لم ينالوا هذا الألم فقط، وهو أنهم لم يعرفوا الأمور الصالحة؛ بل تركوا للعالم أيضًا ذكرى جهالتهم: حتى إنه لم يكن من الممكن إخفاؤهم فيما أخطأوا فيه
\par 9 حكمة شاقة تحرر من ألم أولئك الذين رافقوها.
\par 10 عندما هرب الصديق من غضب أخيه، أرشدته إلى الطرق الصحيحة، وأرته ملكوت الله، وأعطته معرفة الأشياء المقدسة، وجعلته غنيًا في أسفاره، وضاعفت ثمار أعماله.
\par 11 في جشع من ظلموه، وقفت بجانبه، وجعلته غنيًا
\par 12 دافعت عنه من أعدائه، وحفظته آمنًا من الكامنين، وفي صراع عنيف منحته النصر؛ ليعلم أن الخير أقوى من كل شيء
\par 13 عندما بِيعَ البار، لم تتخلَّ عنه، بل خلَّصته من الخطيئة: نزلت معه إلى الجب،
\par 14 ولم تتركه في القيود حتى أتت له بصولجان الملك وسلطانًا على الذين ظلموه. أما الذين اتهموه فكذبتهم وأعطته مجدًا أبديًا
\par 15 أنقذت الشعب الصالح والذرية الطاهرة من الأمة التي اضطهدتهم
\par 16 دخلت في نفس خادم الرب، وقاومت ملوكًا مخيفين بعجائب وآيات؛
\par 17 جُزِيَ لِلْأَبْرَارِ جَزَاءَ أَعْمَالِهِمْ، وَهَدَىٰهُمْ هَدًى عَجَيبًا، وَكَانَ لَهُمْ سَتَارًا نَهَارًا، وَضَوْءَ كُتُبٍ فِي اللَّيْلِ
\par 18 عبر بهم البحر الأحمر، وقادهم عبر مياه كثيرة:
\par 19 ولكنها أغرقت أعداءهم وأخرجتهم من قاع البحر.
\par 20 لذلك نهب الأبرار الأشرار، وسبحوا اسمك القدوس يا رب، وعظّموا بقلب واحد يدك التي حاربت عنهم
\par 21 لأن الحكمة فتحت أفواه البكم، وجعلت ألسنة العاجزين عن التكلم فصاحة

\chapter{11}

\par 1 لقد نجحت أعمالهم على يد النبي المقدس
\par 2 ساروا في البرية غير المأهولة، ونصبوا الخيام في الأماكن التي لم يكن فيها طريق
\par 3 صمدوا في وجه أعدائهم، وانتقموا من خصومهم.
\par 4 عندما عطشوا، دعواك، فأعطي لهم ماء من الصخر الصوان، وأُروِي عطشهم من الحجر الصلب
\par 5 لأن ما عوقب به أعداؤهم، استفادوا به هم في حاجتهم
\par 6 فبدلاً من نهرٍ دائم الجريان مُضطربٍ بالدم النتن،
\par 7 "ومن أجل توبيخ واضح لتلك الوصية التي قُتل بها الأطفال، أعطيتهم الكثير من الماء بطريقة لم يأملوها.
\par 8 مُعلنًا بذلك العطش كيف عاقبت خصومهم
\par 9 لأنهم عندما جُرِّبوا، وإن كان ذلك برحمة، عرفوا كيف يُدان الأشرار بغضب ويُعذبون، متعطشين بطريقة أخرى غير الأبرار
\par 10 لأن هؤلاء وعظتهم وجربتهم كأب، أما الآخرون، كملك صارم، فقد أدانتهم وعاقبتهم
\par 11 سواء كانوا غائبين أو حاضرين، فقد كانوا منزعجين على حد سواء.
\par 12 لأنه أصابهم حزن مضاعف وأنين بسبب تذكر الأمور الماضية.
\par 13 لأنه عندما سمعوا أن عقوباتهم الخاصة ستُفيد الآخرين، كان لديهم شعور ما بالرب
\par 14 الذي كانوا يحترمونه بازدراء، عندما طُرد منذ فترة طويلة عند طرد الأطفال، في النهاية، عندما رأوا ما حدث، أعجبوا به
\par 15 ولكن بسبب مكائدهم الحمقاء الشريرة، التي خُدعوا بها وعبدوا ثعابين فارغة العقل ووحوشًا دنيئة، أرسلت عليهم عددًا كبيرًا من الوحوش غير العاقلة للانتقام؛
\par 16 لكي يعلموا أنه بما يخطئ به الإنسان، فبذلك أيضًا يعاقب
\par 17 لأن يدك القديرة، التي خلقت العالم من مادة بلا شكل، لم تكن بحاجة إلى وسيلة لإرسال حشد من الدببة أو الأسود الشرسة بينهم،
\par 18 أو وحوش برية مجهولة، مليئة بالغضب، حديثة الخلق، تنفث إما بخارًا ناريًا، أو روائح كريهة من الدخان المتناثر، أو تطلق شرارات مروعة من عيونها:
\par 19 حيث لا يقتصر الضرر على إبادتهم على الفور، بل قد يؤدي المنظر المروع أيضًا إلى تدميرهم تمامًا
\par 20 نعم، ولولا هذه لكانوا قد سقطوا دفعة واحدة، اضطهدوا انتقامًا، وتشتتوا بنسمة قوتك: لكنك رتبت كل الأشياء بقدر وعدد ووزن
\par 21 لأنك تستطيع إظهار قوتك العظيمة في كل الأوقات عندما تريد، ومن يستطيع أن يقاوم قوة ذراعك؟
\par 22 لأن العالم كله أمامك كحبة صغيرة في الميزان، نعم، كقطرة من ندى الصباح التي تسقط على الأرض
\par 23 لكنك ترحم الجميع، لأنك قادر على كل شيء، وتتغاضى عن خطايا البشر، لأنهم يجب أن يتصالحوا
\par 24 لأنك تحب كل الأشياء الموجودة، ولا تكره شيئًا مما صنعته: لأنك لم تكن لتصنع شيئًا لو كنت تكرهه
\par 25 فكيف كان من الممكن أن يستمر أي شيء إذا لم تكن إرادتك؟ أو أن يُحفظ إذا لم تدع إليه؟
\par 26 لكنك تشفق على الجميع، لأنهم لك يا رب، يا محب النفوس

\chapter{12}

\par 1 لأن روحك الذي لا يفنى هو في كل شيء
\par 2 لذلك أدب الذين يخطئون شيئًا فشيئًا، وأنذرهم بتذكيرهم بما أخطأوا به، حتى يتركوا شرهم ويؤمنوا بك يا رب
\par 3 لأنه كان من مشيئتك أن تُهلك على أيدي آبائنا كلا هذين السكان القدامى لأرضك المقدسة،
\par 4 الذي كرهته لقيامه بأعمال السحر البغيضة والذبائح الشريرة؛
\par 5 وأيضًا أولئك قتلة الأطفال عديمو الرحمة، ومُلتهمو لحوم البشر، ومُحترفو ولائم الدماء،
\par 6 مع كهنتهم من بين طاقمهم الوثني، والآباء الذين قتلوا بأيديهم أرواحًا محرومة من المساعدة:
\par 7 لكي تستقبل الأرض، التي قدّرتها فوق كل شيء آخر، مستعمرة جديرة من أبناء الله
\par 8 ومع ذلك، فقد أبقيتَ على أولئك كالبشر، وأرسلتَ الدبابير، سابقةً لجيشك، لتُبيدهم شيئًا فشيئًا
\par 9 ليس أنك لم تكن قادرًا على إخضاع الأشرار تحت أيدي الصالحين في المعركة، أو تدميرهم على الفور بالوحوش القاسية، أو بكلمة واحدة قاسية:
\par 10 لكنك إذ نفذت أحكامك عليهم شيئًا فشيئًا، منحتهم مكانًا للتوبة، غير جاهل أنهم جيل شقي، وأن شرهم قد نشأ فيهم، وأن تفكيرهم لن يتغير أبدًا
\par 11 لأنها كانت بذرة ملعونة منذ البداية؛ ولم تغفر لهم خوفًا من أحد عن الأشياء التي أخطأوا بها
\par 12 لأنه من يقول: ماذا فعلت؟ أو من يقاوم حكمك؟ أو من يشكو إليك من أجل الأمم التي تهلك التي صنعتها؟ أو من يأتي ليقف ضدك لينتقم من الناس الظالمين؟
\par 13 لأنه لا يوجد إله غيرك يهتم بالجميع، والذي قد تُظهر له أن حكمك ليس ظالمًا
\par 14 لن يستطيع ملك ولا طاغية أن يوجه وجهه ضدك بسبب أي شخص عاقبته
\par 15 فبما أنك بارٌّ بنفسك، فإنك تُدبّر كل شيء باستقامة، إذ تظن أنه لا يتفق مع قدرتك على إدانة من لا يستحق العقاب
\par 16 لأن قدرتك هي رأس البر، ولأنك رب الجميع، فهي تجعلك كريمًا مع الجميع
\par 17 لأنه عندما لا يصدق الناس أنك ذو قوة كاملة، فإنك تُظهر قوتك، وبين أولئك الذين يعرفونها تُظهر جرأتهم
\par 18 لكنك أنت، إذ تتقن سلطتك، تحكم بالعدل، وتأمرنا برضا عظيم: لأنك تستطيع استخدام القوة متى شئت
\par 19 ولكن بمثل هذه الأعمال علمت شعبك أن الإنسان البار يجب أن يكون رحيماً، وجعلت أولادك على رجاء صالح في أن تعطيهم التوبة عن الخطايا.
\par 20 لأنه إذا عاقبت أعداء أبنائك، والمحكوم عليهم بالإعدام، بمثل هذا التروي، وأعطيتهم الوقت والمكان اللذين يمكن بهما تخليصهم من شرورهم:
\par 21 بأي حذرٍ عظيمٍ حكمتَ على أبنائك الذين أقسمتَ لآبائهم، وقطعتَ لهم عهودًا بوعودٍ صالحة؟
\par 22 لذلك، بينما تؤدبنا، فإنك تجلد أعداءنا ألف مرة أكثر، بحيث عندما نحكم، يجب أن نفكر بعناية في صلاحك، وعندما نحكم علينا، يجب أن نبحث عن الرحمة
\par 23 لذلك، بينما عاش الناس في فجور وظلم، عذبتهم برجاساتهم
\par 24 لأنهم ضلوا طريقهم في الضلال، واتخذوها آلهة، حتى بين وحوش أعدائهم، إذ خُدعوا كأطفال بلا فهم
\par 25 لذلك، أرسلت إليهم حكمًا للسخرية منهم، كما لو كانوا أطفالًا بلا عقل
\par 26 أما أولئك الذين لم يُصلحوا من خلال ذلك التصحيح الذي عبث بهم فيه، فسيشعرون بدينونة تليق بالله
\par 27 انظروا، على ما كانوا يحقدون عليه عندما عوقبوا، أي على من ظنوا أنهم آلهة؛ [الآن] وقد عوقبوا فيهم، عندما رأوا ذلك، اعترفوا به على أنه الإله الحقيقي، الذي كانوا ينكرون معرفته من قبل: ولذلك حلت عليهم لعنة شديدة

\chapter{13}

\par 1 حقًا إن جميع البشر باطلون بطبيعتهم، الذين يجهلون الله، ولم يستطيعوا أن يعرفوا الكائن من الخيرات التي يرونها، ولم يعترفوا بالصانع من خلال النظر في الأعمال
\par 2 لكنهم اعتبروا إما النار، أو الريح، أو الهواء السريع، أو دائرة النجوم، أو الماء العنيف، أو أضواء السماء، هي الآلهة التي تحكم العالم
\par 3 الذي إذا أُعجبوا بجماله اتخذوه آلهة؛ فليعلموا كم هو أفضل ربهم: لأن أول من خلق الجمال هو من خلقهم
\par 4 ولكن إذا اندهشوا من قوتهم وفضيلتهم، فليفهموا من خلالهم كم هو أعظم من الذي خلقهم
\par 5 لأنه من خلال عظمة وجمال المخلوقات، يُرى خالقها بنفس القدر
\par 6 ولكنهم أقل لومًا على هذا: فقد يضلُّون، إذ يبحثون عن الله ويرغبون في إيجاده
\par 7 لأنهم مطلعون على أعماله، فإنهم يفحصونه باجتهاد، ويصدقون ما يرونه، لأن الأشياء التي تُرى جميلة
\par 8 ومع ذلك فلا يُغفر لهم أيضًا.
\par 9 لأنه إذا كانوا قادرين على معرفة كل هذا، بحيث يمكنهم استهداف العالم؛ فكيف لم يكتشفوا ربه عاجلاً؟
\par 10 ولكنهم بائسون، وفي الأشياء الميتة رجاءهم، أولئك الذين يدعون أنفسهم آلهة، وهي أعمال أيدي البشر، الذهب والفضة، لإظهار الفن، وشبه الوحوش، أو الحجر الذي لا يصلح لأي شيء، عمل يد قديمة.
\par 11 الآن، النجار الذي يقطع الأخشاب، بعد أن ينشر شجرة مناسبة لهذا الغرض، ويزيل كل اللحاء المحيط بها بمهارة، ويصنعه بشكل جميل، ويصنع منه إناءً صالحًا لخدمة حياة الإنسان؛
\par 12 وبعد أن قضى فضلات عمله في تجهيز طعامه، شبع؛
\par 13 وأخذ نفس النفايات من بين تلك التي لم تُستخدم، وهي قطعة خشب ملتوية ومليئة بالعقد، ونحتها باجتهاد، عندما لم يكن لديه ما يفعله آخر، وشكلها بمهارة فهمه، وصاغها على صورة إنسان؛
\par 14 أو جعلوه كوحش حقير، فطلوه باللون القرمزي، وصبغوه بالطلاء الأحمر، وغطى كل بقعة فيه؛
\par 15 ولما هيأ لها موضعًا مناسبًا، وضعها في حائط، وثبتها بالحديد
\par 16 لأنه هيأ لها حتى لا تسقط، عالمًا أنها غير قادرة على مساعدة نفسها؛ لأنها صورة، وتحتاج إلى مساعدة
\par 17 ثم يدعو من أجل ماله، ومن أجل زوجته وأولاده، ولا يخجل من التحدث إلى من لا حياة له
\par 18 من أجل الصحة، يستغيث بالضعيف؛ فالحياة تستغيث بالميت؛ ومن أجل العون، يتوسل بتواضع لمن لا يملك إلا أقل الوسائل للمساعدة؛ ومن أجل رحلة طيبة، يستغيث بمن لا يستطيع أن يخطو خطوة للأمام
\par 19 وللحصول على الكسب والحصول عليه، وللنجاح في عمله، يطلب القدرة على القيام بشيء ما، وهو الأعجز عن فعل أي شيء

\chapter{14}

\par 1 مرة أخرى، من يستعد للإبحار، وعلى وشك اجتياز الأمواج الهائجة، يستدعي قطعة خشب أكثر تآكلًا من السفينة التي تحمله
\par 2 فإن الرغبة في الربح هي التي ابتكرت ذلك، وبناه الصانع بمهارته
\par 3 لكن عنايتك يا أبتاه هي التي تحكمها، لأنك جعلت في البحر طريقًا، وفي الأمواج سبيلًا آمنًا
\par 4 مُظهِرًا أنك قادر على الإنقاذ من كل خطر: نعم، حتى لو ذهب رجل إلى البحر بدون فن
\par 5 ومع ذلك، فأنت لا تريد أن تكون أعمال حكمتك عاطلة، ولذلك يُسلم الرجال حياتهم إلى قطعة صغيرة من الخشب، وينجو من يعبر البحر الهائج في سفينة ضعيفة
\par 6 لأنه في القديم أيضًا، عندما هلك العمالقة المتكبرون، هرب أمل العالم الذي تحكمه يدك في إناء ضعيف، وترك لجميع العصور بذرة جيل
\par 7 لأنَّ العودَ الذي بهِ يأتي البرُّ مباركٌ.
\par 8 "ولكن ما هو مصنوع بالأيدي فهو ملعون هو وصانعه. هو لأنه صنعه، وهو لأنه كان فاسداً فدعي إلهاً."
\par 9 لأن الفاجر وفجوره كلاهما مكروهان عند الله على حد سواء.
\par 10 لأن المصنوع سوف يعاقب مع صانعه.
\par 11 لذلك حتى على أصنام الأمم سيكون هناك زيارة، لأنها أصبحت في خليقة الله رجسًا، وعثرة لنفوس البشر، وفخًا لأقدام الجهال.
\par 12 لأن اختراع الأصنام كان بداية الزنا الروحي، واختراعها فساد الحياة
\par 13 لأنهم لم يكونوا موجودين منذ البدء، ولن يكونوا إلى الأبد
\par 14 لأنهم دخلوا العالم بمجد البشر الباطل، ولذلك سيأتون إلى نهايتهم قريبًا
\par 15 لأن أبًا مصابًا بالحزن المبكر، عندما صنع تمثالًا لابنه الذي اختُطف قريبًا، كرّمه الآن كإله، الذي كان آنذاك رجلاً ميتًا، وقدم لمن كانوا تحت إمرته طقوسًا وتضحيات
\par 16 وهكذا، مع مرور الوقت، ترسخت عادة شريرة كقانون، وعُبدت الصور المنحوتة بوصايا الملوك
\par 17 الذي لم يستطع الرجال تكريمه في حضوره، لأنهم كانوا يسكنون بعيدًا، أخذوا صورة مزيفة لوجهه من بعيد، وصنعوا صورة واضحة لملك يكرمونه، حتى يتمكنوا من خلال جرأتهم هذه من تملق الغائب كما لو كان حاضرًا
\par 18 كما أن الاجتهاد الفريد للصانع ساعد في دفع الجهلاء إلى المزيد من الخرافات
\par 19 لأنه، ربما كان راغبًا في إرضاء شخص ذي سلطة، أجبر كل مهاراته على صنع تشابه بأفضل الأزياء
\par 20 وهكذا، انبهر الجموع بنعمة العمل، فاعتبروه الآن إلهًا، والذي لم يكن يُكرَّم إلا قبل قليل
\par 21 وكانت هذه فرصة لخداع العالم: فالرجال، الذين يخدمون إما الكارثة أو الطغيان، نسبوا إلى الحجارة والأشجار الاسم الذي لا يُنطق به
\par 22 علاوة على ذلك، لم يكن هذا كافيًا بالنسبة لهم، أنهم أخطأوا في معرفة الله؛ ولكن بينما كانوا يعيشون في حرب الجهل العظيمة، فإن تلك الأوبئة العظيمة دعتهم سلامًا
\par 23 فبينما كانوا يذبحون أطفالهم كذبائح، أو يمارسون طقوسًا سرية، أو يحتفلون بطقوس غريبة؛
\par 24 لم يعودوا يحافظون على حياة أو زيجات نقية: بل إما أن يقتل أحدهم الآخر غدرًا، أو يحزنه بالزنى
\par 25 حتى ساد بين جميع البشر بلا استثناء الدم، والقتل غير العمد، والسرقة، والنفاق، والفساد، والخيانة، والاضطرابات، وشهادة الزور،
\par 26 إزعاج الصالحين، ونسيان الأعمال الصالحة، ودنس النفوس، وتغيير النوع، واضطراب الزيجات، والزنا، والنجاسة الوقحة
\par 27 إن عبادة الأصنام التي لا تُذكر هي بداية كل شر وسببه ونهاية كل شر
\par 28 إما أنهم مجانين عندما يفرحون، أو يتنبأون بالكذب، أو يعيشون ظلماً، أو ينكرون أنفسهم بسهولة
\par 29 لأن ثقتهم هي في الأصنام التي لا حياة لها، وإن أقسموا كذبًا، فهم لا ينظرون إلى الأذى
\par 30 ولكن لكلا السببين سوف يعاقبون عقابًا عادلًا: أولاً لأنهم لم يعتقدوا جيدًا عن الله، وأصغوا إلى الأصنام، وأيضًا لأنهم أقسموا ظلمًا بالغش، واستهزأوا بالقداسة.
\par 31 لأنه ليس سلطان من يقسمون به، بل انتقام الخطاة العادل، هو الذي يعاقب دائمًا على إثم الأشرار

\chapter{15}

\par 1 وأنت يا الله، رؤوف وصادق، طويل الأناة، ومدبر كل شيء بالرحمة،
\par 2 لأنه إن أخطأنا فنحن لك، عالمين قدرتك. لكننا لا نخطئ، عالمين أننا نُحسب لك
\par 3 لأن معرفتك هي بر كامل، نعم، معرفة قدرتك هي أصل الخلود
\par 4 فلم يخدعنا اختراع البشر المؤذي، ولا صورة ملطخة بألوان متنوعة، ولا عمل الرسام العقيم؛
\par 5 إن منظرها يغري الحمقى بالشهوة، ولذلك يشتهون صورة تمثال ميت لا روح فيه
\par 6 كل من يصنعها، والذين يرغبون فيها، والذين يعبدونها، هم محبون للشر، وهم جديرون بأن يثقوا في مثل هذه الأشياء
\par 7 فالخزاف، الذي يُصقل الطين اللين، يصنع كل إناء بجهد كبير لخدمتنا: نعم، من نفس الطين يصنع كلاً من الأواني التي تُستخدم لأغراض نظيفة، وكذلك جميع الأواني التي تُستخدم على العكس: ولكن ما فائدة أي من النوعين، الخزاف نفسه هو الحكم
\par 8 وباستخدام أعماله بشكل فاحش، فإنه يصنع من نفس الطين إلهًا مغرورًا، حتى هو نفسه الذي كان مصنوعًا من التراب قبل قليل، وبعد فترة وجيزة يعود إلى نفس الطين، عندما تُطالب حياته التي أُقرضت له
\par 9 على الرغم من اهتمامه، ليس لأنه سيعمل كثيرًا، ولا لأن حياته قصيرة: بل إنه يسعى جاهدًا للتفوق على صائغي الذهب والفضة، ويسعى جاهدًا ليفعل مثل عمال النحاس، ويعتبر صنع الأشياء المزيفة فخرًا له
\par 10 قلبه رماد، وأمله أحقر من التراب، وحياته أقل قيمة من الطين:
\par 11 إذ لم يكن يعرف خالقه، والذي ألهم فيه نفسًا فعّالة، ونفخ فيه روحًا حية
\par 12 لكنهم اعتبروا حياتنا هواية، ووقتنا هنا سوقًا للربح: لأنهم يقولون إنه يجب علينا أن نحصل على كل شيء، حتى لو كان ذلك بوسائل شريرة
\par 13 لأن هذا الإنسان، الذي يصنع من مادة أرضية أواني هشة وصورًا منحوتة، يعلم أنه يُسيء إلى الآخرين أكثر من أي شخص آخر
\par 14 وجميع أعداء شعبك الذين يستعبدونهم هم في غاية الحماقة، وهم أشقى من الأطفال
\par 15 لأنهم حسبوا جميع أصنام الأمم آلهة، لا عيون لها لتبصر، ولا أنوف لتتنفس، ولا آذان لتسمع، ولا أصابع أيدي لتلمس، وأما أقدامها فهي بطيئة في المشي
\par 16 لأن الإنسان خلقها، والذي استعار روحه صاغها. لكن لا يستطيع أحد أن يصنع إلهًا مثله
\par 17 لأنه لكونه فان، فهو يصنع شيئًا ميتًا بأيدي شريرة، لأنه هو أفضل من الأشياء التي يعبدها، بينما هو عاش مرة واحدة، وأما هم فلن يعيشوا أبدًا.
\par 18 نعم، لقد عبدوا أيضًا تلك الوحوش الأكثر بغيضًا: فعند مقارنتها ببعضها، يكون بعضها أسوأ من البعض الآخر
\par 19 كما أنها ليست جميلة، لدرجة أنها مرغوبة بين الوحوش: لكنها لم تنل مدح الله وبركاته

\chapter{16}

\par 1 لذلك عوقبوا بمثل ذلك، وعوقبوا بكثرتهم من الوحوش
\par 2 بدلًا من هذا العقاب، وأنت تتعامل بلطف مع شعبك، أعددت لهم لحمًا غريب المذاق، حتى السمان لإثارة شهيتهم:
\par 3 حتى أنهم، وهم يشتهون الطعام، قد يكرهون، بسبب المنظر القبيح للوحوش المرسلة بينهم، حتى ما يحتاجون إلى رغبتهم فيه؛ أما هؤلاء، فقد يعانون من الفقر لفترة قصيرة، فيصبحون شركاء في طعم غريب
\par 4 لأنه كان من الضروري أن يأتي على من يمارسون الطغيان الفقر، وهو ما لا يمكنهم تجنبه: ولكن لهؤلاء فقط يجب أن يُظهروا كيف يُعذب أعداؤهم
\par 5 لأنه عندما هاجمتهم شراسة الوحوش المروعة، وهلكوا بلسعات الثعابين الملتوية، لم يدم غضبك إلى الأبد
\par 6 لكنهم اضطربوا إلى زمان قصير، لكي يُنذروا، ويكون لهم علامة خلاص، لكي يذكروا وصية شريعتك
\par 7 لأن من التفت إليه لم يخلص بما رآه، بل بك أنت مخلص الجميع
\par 8 وبهذا جعلت أعداءك يعترفون بأنك أنت الذي ينجّي من كل شر:
\par 9 لقد قتلتهم لدغات الجراد والذباب، ولم يوجد علاج لحياتهم، لأنهم كانوا يستحقون العقاب بأمثال هذه
\par 10 لكن أبناءك لم يتغلبوا على أنياب التنانين السامة، لأن رحمتك كانت دائمًا معهم، وشفيتهم
\par 11 لأنهم نُخِسوا ليتذكروا كلماتك، وخلصوا سريعًا، حتى لا يقعوا في نسيان عميق، فيتذكروا صلاحك باستمرار
\par 12 فإنه لم يكن عشبًا، ولا ضمادة ملطفة، هي التي أعادتهم إلى الصحة، بل كلمتك يا رب، التي تشفي كل شيء
\par 13 لأن لك سلطان الحياة والموت، أنت تقود إلى أبواب الجحيم، وتصعدها مرة أخرى
\par 14 إن الإنسان يقتل بحقدِه، والروح إذا خرجت لا ترجع، والنفس التي استُلْقِدَت لا تعود
\par 15 لكن ليس من الممكن الهروب من يدك.
\par 16 لأن الأشرار الذين أنكروا معرفتك جلدوا بقوة ذراعك، بأمطار غريبة وبَرَد وأمطار غزيرة، لم يستطيعوا أن يتجنبوها، وبالنار أُفنوا.
\par 17 لأن الأمر الأكثر إثارة للدهشة هو أن النار كانت لها قوة أكبر في الماء الذي يطفئ كل شيء: لأن العالم يقاتل من أجل الصالحين
\par 18 لقد خفّت حدة اللهب إلى حد ما، حتى لا تحرق الوحوش التي أُرسلت ضد الأشرار؛ ولكنهم هم أنفسهم قد رأوا ولاحظوا أنهم اضطهدوا بحكم الله.
\par 19 وفي وقت آخر، يحترق حتى في وسط الماء فوق قوة النار، ليدمر ثمار أرض ظالمة
\par 20 بدلاً من ذلك، أطعمت شعبك طعام الملائكة، وأرسلت لهم من السماء خبزًا مُعدًّا بدون تعبهم، قادرًا على إرضاء كل سرور، ويوافق كل ذوق
\par 21 لأن قوتكِ أعلن حلاوتك لأطفالكِ، وأشبع شهية الآكل، وعدل نفسه حسب ذوق كل إنسان
\par 22 لكن الثلج والجليد صمد أمام النار، ولم يذبا، لكي يعلموا أن النار المشتعلة في البَرَد، والمتلألئة في المطر، قد أتلفت ثمار الأعداء
\par 23 لكن هذا أيضًا نسي قوته، لكي يتغذّى الصالحون
\par 24 لأن المخلوق الذي يخدمك، الذي أنت الخالق، يزيد قوته ضد الظالمين من أجل عقابهم، ويضعف قوته من أجل من يتوكلون عليك
\par 25 لذلك حتى في ذلك الوقت، تم تغييره إلى جميع الأشكال، وكان مطيعًا لنعمتك التي تغذي كل الأشياء، وفقًا لرغبة المحتاجين
\par 26 لكي يعلم أولادك، يا رب، الذين تحبهم، أن ليس نمو الثمار هو الذي يغذي الإنسان، بل كلمتك هي التي تحفظ الذين يتوكلون عليك
\par 27 لأن ما لم يُدمر بالنار، إذا دُفِئ بقليل من شعاع الشمس، سرعان ما ذاب:
\par 28 لكي يُعلم أنه يجب علينا أن نمنع الشمس من شكرك، وعند طلوع الفجر نصلي لك
\par 29 لأن رجاء غير الشاكرين سيتلاشى كصقيع الشتاء، وسيجري كالمياه غير النافعة

\chapter{17}

\par 1 لأن أحكامك عظيمة، ولا يمكن التعبير عنها: لذلك أخطأت النفوس غير المغذية
\par 2 لأنه عندما فكر الرجال الأشرار في قمع الأمة المقدسة، كانوا محبوسين في منازلهم، أسرى الظلام، ومقيدين بقيود ليل طويل، ومنفيين عن العناية الأبدية
\par 3 فبينما كان من المفترض أن يكونوا مختبئين في خطاياهم السرية، تشتتوا تحت ستار مظلم من النسيان، وقد أصيبوا بدهشة مروعة، واضطربتهم أطياف [غريبة].
\par 4 لأن الزاوية التي كانت تحتجزهم لم تستطع أن تمنعهم من الخوف، بل دوّت حولهم أصوات [كأصوات مياه] متساقطة، وظهرت لهم رؤى حزينة بوجوه ثقيلة
\par 5 لم تستطع أي قوة من قوة النار أن تمنحهم النور: ولم تستطع ألسنة اللهب الساطعة للنجوم أن تضيء تلك الليلة الرهيبة
\par 6 لم تظهر لهم إلا نار مشتعلة من تلقاء نفسها، مخيفة جدًا: لأنهم كانوا مرعوبين للغاية، فظنوا أن الأشياء التي رأوها أسوأ من المنظر الذي لم يروا
\par 7 وأما أوهام السحر الفني فقد تم دحضها، وتم توبيخ ادعائهم بالحكمة بالعار.
\par 8 لأن أولئك الذين وعدوا بطرد الأهوال والمتاعب من نفس مريضة، كانوا هم أنفسهم مرضى من الخوف، ويستحقون أن يُسخر منهم
\par 9 فمع أنهم لم يخشوا شيئًا رهيبًا، إلا أنهم كانوا يخافون من الوحوش المارة، ومن صفير الثعابين،
\par 10 ماتوا خوفًا، منكرين رؤية الهواء، الذي لا يمكن تجنبه من أي جانب
\par 11 لأن الشر، الذي يُدان بشهادته الخاصة، يكون خائفًا جدًا، ولأنه مضغوط بالضمير، فإنه يتنبأ دائمًا بأمور خطيرة
\par 12 فالخوف ليس إلا خيانة للمساعدات التي يقدمها العقل
\par 13 ولأن التوقعات من الداخل أقل، فإنها تحسب الجهل أكثر من السبب الذي يجلب العذاب
\par 14 لكنهم ناموا نفس النوم في تلك الليلة، والذي كان لا يُطاق حقًا، والذي جاءهم من أعماق الجحيم المحتوم،
\par 15 كانوا منزعجين جزئيًا من الأشباح الوحشية، ومنشغلين جزئيًا، وقلوبهم تخفق: إذ فاجأهم خوف مفاجئ لم يتوقعوه
\par 16 فكل من سقط كان يُحجز بصرامة في سجن بلا قضبان حديدية،
\par 17 لأنه سواء كان فلاحًا، أو راعيًا، أو عاملًا في الحقل، فقد أُدرك، وتحمل تلك الضرورة التي لا يمكن تجنبها: لأنهم جميعًا كانوا مقيدين بسلسلة واحدة من الظلام
\par 18 سواء كانت ريحًا صفيرًا، أو ضجيجًا شجيًا للطيور بين الأغصان الممتدة، أو سقوطًا جميلًا لماء يتدفق بعنف،
\par 19 أو صوت حجارة مُرعبة تُلقى، أو ركض لا يُرى لوجوه تقفز، أو صوت زئير وحوش برية شرسة، أو صدى مرتد من الجبال الجوفاء؛ هذه الأشياء جعلتهم يُغمى عليهم من الخوف
\par 20 لأن العالم كله أشرق بنور ساطع، ولم يُعاق أحد في عمله:
\par 21 لم يبقَ عليهم إلا ليلٌ ثقيل، صورةٌ لذلك الظلام الذي سيحلُّ بهم فيما بعد: لكنهم كانوا مع ذلك أشدَّ وطأةً على أنفسهم من الظلام

\chapter{18}

\par 1 مع ذلك، كان لقديسيك نور عظيم جدًا، الذين سمعوا صوتهم، ولم يروا شكلهم، لأنهم لم يعانوا نفس الأشياء، فقد حسبوهم سعداء
\par 2 ولكن لأنهم لم يؤذوا الآن من ظلموهم من قبل، فقد شكروهم، وطلبوا منهم العفو عن كونهم أعداءً
\par 3 بدلًا من ذلك، أعطيتهم عمودًا من نار مشتعلة، ليكونوا دليلاً في الرحلة المجهولة، وشمسًا غير ضارة لتسليةهم بشرف
\par 4 لأنهم كانوا يستحقون أن يُحرموا من النور ويُسجنوا في الظلمة، أولئك الذين حبسوا أبناءك، الذين بواسطتهم كان من المقرر أن يُعطى نور الشريعة غير الفاسد للعالم
\par 5 ولما عزموا على قتل أطفال القديسين، ألقي طفل واحد وأنقذ لتوبيخهم، أخذت جمع أطفالهم وأهلكتهم جميعاً في ماء هائج.
\par 6 كان آباؤنا قد أُقرّوا بتلك الليلة مسبقًا، حتى يعرفوا يقينًا أي قسمٍ أقسموا عليه، فيكونوا في سرور فيما بعد
\par 7 وهكذا قُبل من شعبك خلاص الأبرار وهلاك الأعداء
\par 8 لأن ما عاقبت به أعداءنا، فبنفس الطريقة مجدتنا نحن الذين دعوتهم
\par 9 لأن أبناء الصالحين من الرجال الصالحين ضحوا سرًا، وبموافقة واحدة وضعوا قانونًا مقدسًا، حتى يكون القديسون شركاء في نفس الخير والشر، والآباء الآن يغنون أغاني التسبيح
\par 10 ولكن على الجانب الآخر، دوى صراخ الأعداء، وحمل ضجيج مؤسف إلى الخارج من أجل الأطفال الذين كانوا ينتحبون
\par 11 عوقب السيد والخادم بطريقة واحدة؛ وكما عوقب الملك، عوقب عامة الناس أيضًا
\par 12 لذلك كان لديهم جميعًا عددًا لا يحصى من الموتى بنوع واحد من الموت؛ ولم يكن الأحياء كافيين لدفنهم: ففي لحظة واحدة هلك أنبل نسلهم
\par 13 فبينما لم يصدقوا شيئًا بسبب السحر، فإنه عند هلاك الأبكار، اعترفوا بأن هذا الشعب هم أبناء الله
\par 14 فبينما كان كل شيء في صمت هادئ، وكانت تلك الليلة في خضم مسارها السريع،
\par 15 قفزت كلمتك القديرة من السماء من عرشك الملكي، كرجل حرب شرس إلى وسط أرض دمار،
\par 16 وجاء بوصيتك الصادقة كسيف حاد، وقام وملأ كل شيء موتًا، ومس السماء، لكنه وقف على الأرض
\par 17 ثم فجأة أزعجتهم رؤى الأحلام المرعبة بشدة، وفاجأتهم أهوال لم يتوقعوها
\par 18 وألقي واحد هنا، وآخر هناك، نصف ميت، أظهر سبب وفاته
\par 19 لأن الأحلام التي أزعجتهم أنبأت بذلك، لئلا يهلكوا ولا يعلموا سبب بلائهم
\par 20 نعم، لقد مسّ طعم الموت الأبرار أيضًا، وكان هناك هلاك للجموع في البرية، ولكن الغضب لم يدم طويلًا
\par 21 فحينئذٍ سارع الرجل الذي لا عيب فيه، ووقف للدفاع عنهم، حاملاً درع خدمته الخاصة، أي الصلاة، وكفارة البخور، ووقف ضد الغضب، وبذلك أنهى الكارثة، معلنًا أنه خادمك
\par 22 لذلك تغلب على المهلك، ليس بقوة الجسد ولا بقوة السلاح، بل بكلمة أخضع من عاقب، مدعيًا القسم والعهود التي قطعها مع الآباء
\par 23 لأنه عندما سقط الموتى الآن في أكوام فوق بعضهم البعض، وقف بينهم، ووقف الغضب، وفتح الطريق للأحياء.
\par 24 لأنه في الرداء الطويل كان العالم كله، وفي صفوف الحجارة الأربعة كان مجد الآباء محفورًا، وجلالتك على تاج رأسه
\par 25 فتنازل المهلك عن مكانه لهؤلاء، وخاف منهم: لأنه كان يكفيهم أن يذوقوا الغضب فقط

\chapter{19}

\par 1 أما الأشرار، فجاء عليهم الغضب بلا رحمة إلى النهاية، لأنه علم مسبقًا ما سيفعلونه
\par 2 كيف أنهم بعد أن أذنوا لهم بالمغادرة، وأرسلوهم على عجل، سيتوبون ويطاردونهم
\par 3 فبينما كانوا لا يزالون ينوحون وينوحون على قبور الموتى، أضافوا حيلة حمقاء أخرى، وطاردوهم كهاربين، بعد أن توسلوا إليهم بالرحيل
\par 4 لأن القدر الذي كانوا يستحقونه هو الذي دفعهم إلى هذه الغاية، وجعلهم ينسون ما حدث بالفعل، حتى يتمكنوا من إتمام العقوبة التي كانت ناقصة لعذاباتهم:
\par 5 وأن يسلك شعبك طريقًا عجيبًا، ولكنهم قد يجدون موتًا غريبًا
\par 6 لأن الخليقة كلها خُلقت من جديد في نوعها الخاص، خادمةً الوصايا الخاصة التي أُعطيت لها، لكي يُحفظ أولادك دون أذى
\par 7 كأنها سحابة تُظلل المخيم، وحيث وقف الماء، ظهرت أرض يابسة، ومن البحر الأحمر طريق بلا عوائق، ومن النهر العنيف حقل أخضر:
\par 8 التي مرّ بها جميع الشعوب الذين كانوا محميين بيدك، ورأوا عجائبك العجيبة الغريبة
\par 9 لأنهم انطلقوا كالخيل، وقفزوا كالحملان، يسبحونك يا رب، يا من خلصتهم
\par 10 لأنهم كانوا لا يزالون يتذكرون الأمور التي حدثت أثناء إقامتهم في الأرض الغريبة، كيف أن الأرض أخرجت الذباب بدلًا من الماشية، وكيف أن النهر أخرج ضفادعًا كثيرة بدلًا من الأسماك
\par 11 ولكن بعد ذلك رأوا جيلًا جديدًا من الطيور، عندما انقادوا لشهيتهم، فطلبوا لحومًا لذيذة
\par 12 لأن السلوى صعدت إليهم من البحر لشِبَاعهم.
\par 13 "ولقد جاءت العقوبات على الخطاة ليس بدون علامات سابقة بقوة الرعد: لأنهم عانوا بحق حسب شرورهم، بقدر ما استخدموا سلوكًا أكثر قسوة وكراهية تجاه الغرباء.
\par 14 لأن أهل سدوم لم يقبلوا أولئك الذين لم يعرفوهم عند مجيئهم، بل هؤلاء استعبدوا أصدقاءهم الذين استحقوهم بجدارة
\par 15 وليس ذلك فحسب، بل ربما يُنظر إليهم باحترام، لأنهم استخدموا غرباء غير ودودين:
\par 16 لكن هؤلاء أحزنوا بشدة أولئك الذين استقبلوهم بالولائم، وكانوا قد أصبحوا بالفعل شركاء معهم في نفس الشرائع
\par 17 لذلك أصيب هؤلاء بالعمى، كما أصيب أولئك الذين كانوا على أبواب الرجل البار: عندما أحاط بهم ظلام عظيم رهيب، كان كل واحد يبحث عن طريق من أبوابه.
\par 18 لأن العناصر تغيرت في ذاتها بنوع من الانسجام، كما هو الحال في نغمات المزامير التي تغير اسم اللحن، ومع ذلك فهي دائمًا أصوات؛ والتي يمكن إدراكها بسهولة من خلال رؤية الأشياء التي تم القيام بها
\par 19 لأن الأشياء الأرضية تحولت إلى مائية، والأشياء التي كانت تسبح في الماء من قبل، أصبحت الآن على الأرض
\par 20 كانت للنار قوة في الماء، ناسيةً فضيلتها: ونسي الماء طبيعته المُطفِئة
\par 21 على الجانب الآخر، لم تُهدر النيران لحم الكائنات الحية القابلة للفساد، على الرغم من سيرها فيه؛ كما أنها لم تُذيب النوع الجليدي من اللحم السماوي الذي كان بطبيعته قابلاً للذوبان
\par 22 لأنه في كل شيء، يا رب، عظمت شعبك ومجدتهم، ولم تستخف بهم، بل ساعدتهم في كل زمان ومكان


\end{document}