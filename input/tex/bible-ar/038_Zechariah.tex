\begin{document}

\title{زكريا}


\chapter{1}

\par 1 فِي الشَّهْرِ الثَّامِنِ فِي السَّنَةِ الثَّانِيَةِ لِدَارِيُوسَ كَانَتْ كَلِمَةُ الرَّبِّ إِلَى زَكَرِيَّا بْنِ بَرَخِيَّا بْنِ عِدُّو النَّبِيِّ:
\par 2 قَدْ غَضِبَ الرَّبُّ غَضَباً عَلَى آبَائِكُمْ.
\par 3 فَقُلْ لَهُمْ: [هَكَذَا قَالَ رَبُّ الْجُنُودِ: ارْجِعُوا إِلَيَّ يَقُولُ رَبُّ الْجُنُودِ فَأَرْجِعَ إِلَيْكُمْ يَقُولُ رَبُّ الْجُنُودِ.
\par 4 لاَ تَكُونُوا كَآبَائِكُمُ الَّذِينَ نَادَاهُمُ الأَنْبِيَاءُ الأَوَّلُونَ: هَكَذَا قَالَ رَبُّ الْجُنُودِ: ارْجِعُوا عَنْ طُرُقِكُمُ الشِّرِّيرَةِ وَعَنْ أَعْمَالِكُمُ الشِّرِّيرَةِ. فَلَمْ يَسْمَعُوا وَلَمْ يُصْغُوا إِلَيَّ يَقُولُ رَبُّ الْجُنُودِ.
\par 5 آبَاؤُكُمْ أَيْنَ هُمْ؟ وَالأَنْبِيَاءُ هَلْ أَبَداً يَحْيُونَ.
\par 6 وَلَكِنْ كَلاَمِي وَفَرَائِضِي الَّتِي أَوْصَيْتُ بِهَا عَبِيدِي الأَنْبِيَاءَ أَفَلَمْ تُدْرِكْ آبَاءَكُمْ؟ فَرَجَعُوا وَقَالُوا: كَمَا قَصَدَ رَبُّ الْجُنُودِ أَنْ يَصْنَعَ بِنَا كَطُرُقِنَا وَكَأَعْمَالِنَا كَذَلِكَ فَعَلَ بِنَا].
\par 7 فِي الْيَوْمِ الرَّابِعِ وَالْعِشْرِينَ مِنَ الشَّهْرِ الْحَادِي عَشَرَ (هُوَ شَهْرُ شَبَاطَ). فِي السَّنَةِ الثَّانِيَةِ لِدَارِيُوسَ كَانَتْ كَلِمَةُ الرَّبِّ إِلَى زَكَرِيَّا بْنِ بَرَخِيَّا بْنِ عِدُّو النَّبِيِّ:
\par 8 رَأَيْتُ فِي اللَّيْلِ وَإِذَا بِرَجُلٍ رَاكِبٍ عَلَى فَرَسٍ أَحْمَرَ وَهُوَ وَاقِفٌ بَيْنَ الآسِ الَّذِي فِي الظِّلِّ وَخَلْفَهُ خَيْلٌ حُمْرٌ وَشُقْرٌ وَشُهْبٌ.
\par 9 فَقُلْتُ: [يَا سَيِّدِي مَا هَؤُلاَءِ؟] فَقَالَ لِي الْمَلاَكُ الَّذِي كَلَّمَنِي: [أَنَا أُرِيكَ مَا هَؤُلاَءِ].
\par 10 فَأَجَابَ الرَّجُلُ الْوَاقِفُ بَيْنَ الآسِ: [هَؤُلاَءِ هُمُ الَّذِينَ أَرْسَلَهُمُ الرَّبُّ لِلْجَوَلاَنِ فِي الأَرْضِ].
\par 11 فَأَجَابُوا مَلاَكَ الرَّبِّ الْوَاقِفِ بَيْنَ الآسِ: [قَدْ جُلْنَا فِي الأَرْضِ وَإِذَا الأَرْضُ كُلُّهَا مُسْتَرِيحَةٌ وَسَاكِنَةٌ].
\par 12 فَقَالَ مَلاَكُ الرَّبِّ: [يَا رَبَّ الْجُنُودِ إِلَى مَتَى أَنْتَ لاَ تَرْحَمُ أُورُشَلِيمَ وَمُدُنَ يَهُوذَا الَّتِي غَضِبْتَ عَلَيْهَا هَذِهِ السَّبْعِينَ سَنَةً؟]
\par 13 فَأَجَابَ الرَّبُّ الْمَلاَكَ الَّذِي كَلَّمَنِي بِكَلاَمٍ طَيِّبٍ وَكَلاَمِ تَعْزِيَةٍ.
\par 14 فَقَالَ لِي الْمَلاَكُ الَّذِي كَلَّمَنِي: [نَادِ قَائِلاً: هَكَذَا قَالَ رَبُّ الْجُنُودِ: غِرْتُ عَلَى أُورُشَلِيمَ وَعَلَى صِهْيَوْنَ غَيْرَةً عَظِيمَةً.
\par 15 وَأَنَا مُغْضِبٌ بِغَضَبٍ عَظِيمٍ عَلَى الأُمَمِ الْمُطْمَئِنِّينَ. لأَنِّي غَضِبْتُ قَلِيلاً وَهُمْ أَعَانُوا الشَّرَّ.
\par 16 لِذَلِكَ هَكَذَا قَالَ الرَّبُّ: قَدْ رَجَعْتُ إِلَى أُورُشَلِيمَ بِالْمَرَاحِمِ فَبَيْتِي يُبْنَى فِيهَا يَقُولُ رَبُّ الْجُنُودِ وَيُمَدُّ الْمِطْمَارُ عَلَى أُورُشَلِيمَ.
\par 17 نَادِ أَيْضاً وَقُلْ: هَكَذَا قَالَ رَبُّ الْجُنُودِ: إِنَّ مُدُنِي تَفِيضُ بَعْدُ خَيْراً وَالرَّبُّ يُعَزِّي صِهْيَوْنَ بَعْدُ وَيَخْتَارُ بَعْدُ أُورُشَلِيمَ].
\par 18 فَرَفَعْتُ عَيْنَيَّ وَنَظَرْتُ وَإِذَا بِأَرْبَعَةِ قُرُونٍ.
\par 19 فَقُلْتُ لِلْمَلاَكِ الَّذِي كَلَّمَنِي: [مَا هَذِهِ؟] فَقَالَ لِي: [هَذِهِ هِيَ الْقُرُونُ الَّتِي بَدَّدَتْ يَهُوذَا وَإِسْرَائِيلَ وَأُورُشَلِيمَ].
\par 20 فَأَرَانِي الرَّبُّ أَرْبَعَةَ صُنَّاعٍ.
\par 21 فَقُلْتُ: [جَاءَ هَؤُلاَءِ مَاذَا يَفْعَلُونَ؟] فَأَجَابَ: [هَذِهِ هِيَ الْقُرُونُ الَّتِي بَدَّدَتْ يَهُوذَا حَتَّى لَمْ يَرْفَعْ إِنْسَانٌ رَأْسَهُ. وَقَدْ جَاءَ هَؤُلاَءِ لِيُرْعِبُوهُمْ وَلِيَطْرُدُوا قُرُونَ الأُمَمِ الرَّافِعِينَ قَرْناً عَلَى أَرْضِ يَهُوذَا لِتَبْدِيدِهَا].

\chapter{2}

\par 1 فَرَفَعْتُ عَيْنَيَّ وَنَظَرْتُ وَإِذَا رَجُلٌ وَبِيَدِهِ حَبْلُ قِيَاسٍ.
\par 2 فَقُلْتُ: [إِلَى أَيْنَ أَنْتَ ذَاهِبٌ؟] فَقَالَ لِي: [لأَقِيسَ أُورُشَلِيمَ لأَرَى كَمْ عَرْضُهَا وَكَمْ طُولُهَا].
\par 3 وَإِذَا بِالْمَلاَكِ الَّذِي كَلَّمَنِي قَدْ خَرَجَ وَخَرَجَ مَلاَكٌ آخَرُ لِلِقَائِهِ.
\par 4 فَقَالَ لَهُ: [اجْرِ وَقُلْ لِهَذَا الْغُلاَمِ: كَالأَعْرَاءِ تُسْكَنُ أُورُشَلِيمُ مِنْ كَثْرَةِ النَّاسِ وَالْبَهَائِمِ فِيهَا.
\par 5 وَأَنَا يَقُولُ الرَّبُّ أَكُونُ لَهَا سُورَ نَارٍ مِنْ حَوْلِهَا وَأَكُونُ مَجْداً فِي وَسَطِهَا.
\par 6 [يَا يَا اهْرُبُوا مِنْ أَرْضِ الشِّمَالِ يَقُولُ الرَّبُّ. فَإِنِّي قَدْ فَرَّقْتُكُمْ كَرِيَاحِ السَّمَاءِ الأَرْبَعِ يَقُولُ الرَّبُّ.
\par 7 تَنَجَّيْ يَا صِهْيَوْنُ السَّاكِنَةُ فِي بِنْتِ بَابِلَ
\par 8 لأَنَّهُ هَكَذَا قَالَ رَبُّ الْجُنُودِ: بَعْدَ الْمَجْدِ أَرْسَلَنِي إِلَى الأُمَمِ الَّذِينَ سَلَبُوكُمْ لأَنَّهُ مَنْ يَمَسُّكُمْ يَمَسُّ حَدَقَةَ عَيْنِهِ.
\par 9 لأَنِّي هَئَنَذَا أُحَرِّكُ يَدِي عَلَيْهِمْ فَيَكُونُونَ سَلَباً لِعَبِيدِهِمْ. فَتَعْلَمُونَ أَنَّ رَبَّ الْجُنُودِ قَدْ أَرْسَلَنِي.
\par 10 [تَرَنَّمِي وَافْرَحِي يَا بِنْتَ صِهْيَوْنَ لأَنِّي هَئَنَذَا آتِي وَأَسْكُنُ فِي وَسَطِكِ يَقُولُ الرَّبُّ.
\par 11 فَيَتَّصِلُ أُمَمٌ كَثِيرَةٌ بِالرَّبِّ فِي ذَلِكَ الْيَوْمِ وَيَكُونُونَ لِي شَعْباً فَأَسْكُنُ فِي وَسَطِكِ فَتَعْلَمِينَ أَنَّ رَبَّ الْجُنُودِ قَدْ أَرْسَلَنِي إِلَيْكِ.
\par 12 وَالرَّبُّ يَرِثُ يَهُوذَا نَصِيبَهُ فِي الأَرْضِ الْمُقَدَّسَةِ وَيَخْتَارُ أُورُشَلِيمَ بَعْدُ.
\par 13 اُسْكُتُوا يَا كُلَّ الْبَشَرِ قُدَّامَ الرَّبِّ لأَنَّهُ قَدِ اسْتَيْقَظَ مِنْ مَسْكَنِ قُدْسِهِ].

\chapter{3}

\par 1 وَأَرَانِي يَهُوشَعَ الْكَاهِنَ الْعَظِيمَ قَائِماً قُدَّامَ مَلاَكِ الرَّبِّ وَالشَّيْطَانُ قَائِمٌ عَنْ يَمِينِهِ لِيُقَاوِمَهُ.
\par 2 فَقَالَ الرَّبُّ لِلشَّيْطَانِ: [لِيَنْتَهِرْكَ الرَّبُّ يَا شَيْطَانُ. لِيَنْتَهِرْكَ الرَّبُّ الَّذِي اخْتَارَ أُورُشَلِيمَ. أَفَلَيْسَ هَذَا شُعْلَةً مُنْتَشَلَةً مِنَ النَّارِ؟].
\par 3 وَكَانَ يَهُوشَعُ لاَبِساً ثِيَاباً قَذِرَةً وَوَاقِفاً قُدَّامَ الْمَلاَكِ.
\par 4 فَقَالَ لِلْوَاقِفِينَ قُدَّامَهُ: [انْزِعُوا عَنْهُ الثِّيَابَ الْقَذِرَةَ]. وَقَالَ لَهُ: [انْظُرْ. قَدْ أَذْهَبْتُ عَنْكَ إِثْمَكَ وَأُلْبِسُكَ ثِيَاباً مُزَخْرَفَةً].
\par 5 فَقُلْتُ: [لِيَضَعُوا عَلَى رَأْسِهِ عِمَامَةً طَاهِرَةً]. فَوَضَعُوا عَلَى رَأْسِهِ الْعِمَامَةَ الطَّاهِرَةَ وَأَلْبَسُوهُ ثِيَاباً وَمَلاَكُ الرَّبِّ وَاقِفٌ.
\par 6 فَأَشْهَدَ مَلاَكُ الرَّبِّ عَلَى يَهُوشَعَ قَائِلاً:
\par 7 [هَكَذَا قَالَ رَبُّ الْجُنُودِ: إِنْ سَلَكْتَ فِي طُرُقِي وَإِنْ حَفِظْتَ شَعَائِرِي فَأَنْتَ أَيْضاً تَدِينُ بَيْتِي وَتُحَافِظُ أَيْضاً عَلَى دِيَارِي وَأُعْطِيكَ مَسَالِكَ بَيْنَ هَؤُلاَءِ الْوَاقِفِينَ.
\par 8 فَاسْمَعْ يَا يَهُوشَعُ الْكَاهِنُ الْعَظِيمُ أَنْتَ وَرُفَقَاؤُكَ الْجَالِسُونَ أَمَامَكَ (لأَنَّهُمْ رِجَالُ آيَةٍ) لأَنِّي هَئَنَذَا آتِي بِعَبْدِي [الْغُصْنِ].
\par 9 فَهُوَذَا الْحَجَرُ الَّذِي وَضَعْتُهُ قُدَّامَ يَهُوشَعَ عَلَى حَجَرٍ وَاحِدٍ سَبْعُ أَعْيُنٍ. هَئَنَذَا نَاقِشٌ نَقْشَهُ يَقُولُ رَبُّ الْجُنُودِ وَأُزِيلُ إِثْمَ تِلْكَ الأَرْضِ فِي يَوْمٍ وَاحِدٍ.
\par 10 فِي ذَلِكَ الْيَوْمِ يَقُولُ رَبُّ الْجُنُودِ يُنَادِي كُلُّ إِنْسَانٍ قَرِيبَهُ تَحْتَ الْكَرْمَةَ وَتَحْتَ التِّينَةِ].

\chapter{4}

\par 1 فَرَجَعَ الْمَلاَكُ الَّذِي كَلَّمَنِي وَأَيْقَظَنِي كَرَجُلٍ أُوقِظَ مِنْ نَوْمِهِ.
\par 2 وَقَالَ لِي: [مَاذَا تَرَى؟] فَقُلْتُ: [قَدْ نَظَرْتُ وَإِذَا بِمَنَارَةٍ كُلُّهَا ذَهَبٌ وَكُوزُهَا عَلَى رَأْسِهَا وَسَبْعَةُ سُرُجٍ عَلَيْهَا وَسَبْعُ أَنَابِيبَ لِلسُّرْجِ الَّتِي عَلَى رَأْسِهَا.
\par 3 وَعِنْدَهَا زَيْتُونَتَانِ إِحْدَاهُمَا عَنْ يَمِينِ الْكُوزِ وَالأُخْرَى عَنْ يَسَارِهِ].
\par 4 فَسَأَلْتُ الْمَلاَكِ الَّذِي كَلَّمَنِي: [مَا هَذِهِ يَا سَيِّدِي؟]
\par 5 فَأَجَابَ الْمَلاَكُ الَّذِي كَلَّمَنِي: [أَمَا تَعْلَمُ مَا هَذِهِ؟] فَقُلْتُ: [لاَ يَا سَيِّدِي].
\par 6 فَقَالَ: [هَذِهِ كَلِمَةُ الرَّبِّ إِلَى زَرُبَّابِلَ: لاَ بِالْقُدْرَةِ وَلاَ بِالْقُوَّةِ بَلْ بِرُوحِي قَالَ رَبُّ الْجُنُودِ.
\par 7 مَنْ أَنْتَ أَيُّهَا الْجَبَلُ الْعَظِيمُ؟ أَمَامَ زَرُبَّابِلَ تَصِيرُ سَهْلاً! فَيُخْرِجُ حَجَرَ الزَّاوِيَةِ بَيْنَ الْهَاتِفِينَ: كَرَامَةً كَرَامَةً لَهُ].
\par 8 وَكَانَتْ إِلَيَّ كَلِمَةُ الرَّبِّ:
\par 9 [إِنَّ يَدَيْ زَرُبَّابِلَ قَدْ أَسَّسَتَا هَذَا الْبَيْتَ فَيَدَاهُ تُتَمِّمَانِهِ فَتَعْلَمُ أَنَّ رَبَّ الْجُنُودِ أَرْسَلَنِي إِلَيْكُمْ].
\par 10 لأَنَّهُ مَنِ ازْدَرَى بِيَوْمِ الأُمُورِ الصَّغِيرَةِ. فَتَفْرَحُ أُولَئِكَ السَّبْعُ وَيَرُونَ الزِّيجَ بِيَدِ زَرُبَّابِلَ. إِنَّمَا هِيَ أَعْيُنُ الرَّبِّ الْجَائِلَةُ فِي الأَرْضِ كُلِّهَا.
\par 11 فَسَأَلْتُهُ: [مَا هَاتَانِ الزَّيْتُونَتَانِ عَنْ يَمِينِ الْمَنَارَةِ وَعَنْ يَسَارِهَا؟]
\par 12 وَسَأَلْتُهُ ثَانِيَةً: [مَا فَرْعَا الزَّيْتُونِ اللَّذَانِ بِجَانِبِ الأَنَابِيبِ مِنْ ذَهَبٍ الْمُفْرِغَانِ مِنْ أَنْفُسِهِمَا الذَّهَبِيَّ؟]
\par 13 فَأَجَابَنِي: [أَمَا تَعْلَمُ مَا هَاتَانِ؟] فَقُلْتُ: [لاَ يَا سَيِّدِي].
\par 14 فَقَالَ: [هَاتَانِ هُمَا ابْنَا الزَّيْتِ الْوَاقِفَانِ عِنْدَ سَيِّدِ الأَرْضِ كُلِّهَا].

\chapter{5}

\par 1 فَعُدْتُ وَرَفَعْتُ عَيْنَيَّ وَنَظَرْتُ وَإِذَا بِدَرْجٍ طَائِرٍ.
\par 2 فَقَالَ لِي: [مَاذَا تَرَى؟] فَقُلْتُ: [إِنِّي أَرَى دَرْجاً طَائِراً طُولُهُ عِشْرُونَ ذِرَاعاً وَعَرْضُهُ عَشَرُ أَذْرُعٍ].
\par 3 فَقَالَ لِي: [هَذِهِ هِيَ اللَّعْنَةُ الْخَارِجَةُ عَلَى وَجْهِ كُلِّ الأَرْضِ. لأَنَّ كُلَّ سَارِقٍ يُبَادُ مِنْ هُنَا بِحَسَبِهَا وَكُلَّ حَالِفٍ يُبَادُ مِنْ هُنَاكَ بِحَسَبِهَا.
\par 4 إِنِّي أُخْرِجُهَا يَقُولُ رَبُّ الْجُنُودِ فَتَدْخُلُ بَيْتَ السَّارِقِ وَبَيْتَ الْحَالِفِ بِاسْمِي زُوراً وَتَبِيتُ فِي وَسَطِ بَيْتِهِ وَتُفْنِيهِ مَعَ خَشَبِهِ وَحِجَارَتِهِ].
\par 5 ثُمَّ خَرَجَ الْمَلاَكُ الَّذِي كَلَّمَنِي وَقَالَ لِي: [ارْفَعْ عَيْنَيْكَ وَانْظُرْ مَا هَذَا الْخَارِجُ].
\par 6 فَقُلْتُ: [مَا هُوَ؟] فَقَالَ: [هَذِهِ هِيَ الإِيفَةُ الْخَارِجَةُ]. وَقَالَ: [هَذِهِ عَيْنُهُمْ فِي كُلِّ الأَرْضِ].
\par 7 وَإِذَا بِوَزْنَةِ رَصَاصٍ رُفِعَتْ. وَكَانَتِ امْرَأَةٌ جَالِسَةٌ فِي وَسَطِ الإِيفَةِ.
\par 8 فَقَالَ: [هَذِهِ هِيَ الشَّرُّ]. فَطَرَحَهَا إِلَى وَسَطِ الإِيفَةِ وَطَرَحَ ثِقْلَ الرَّصَاصِ عَلَى فَمِهَا.
\par 9 وَرَفَعْتُ عَيْنَيَّ وَنَظَرْتُ وَإِذَا بِامْرَأَتَيْنِ خَرَجَتَا وَالرِّيحُ فِي أَجْنِحَتِهِمَا. وَلَهُمَا أَجْنِحَةٌ كَأَجْنِحَةِ اللَّقْلَقِ فَرَفَعَتَا الإِيفَةَ بَيْنَ الأَرْضِ وَالسَّمَاءِ.
\par 10 فَقُلْتُ لِلْمَلاَكِ الَّذِي كَلَّمَنِي: [إِلَى أَيْنَ هُمَا ذَاهِبَتَانِ بِالإِيفَةِ؟]
\par 11 فَقَالَ لِي: [لِتَبْنِيَا لَهَا بَيْتاً فِي أَرْضِ شِنْعَارَ. وَإِذَا تَهَيَّأَ تَقِرُّ هُنَاكَ عَلَى قَاعِدَتِهَا].

\chapter{6}

\par 1 فَعُدْتُ وَرَفَعْتُ عَيْنَيَّ وَنَظَرْتُ وَإِذَا بِأَرْبَعِ مَرْكَبَاتٍ خَارِجَاتٍ مِنْ بَيْنِ جَبَلَيْنِ وَالْجَبَلاَنِ جَبَلاَ نُحَاسٍ.
\par 2 فِي الْمَرْكَبَةِ الأُولَى خَيْلٌ حُمْرٌ وَفِي الْمَرْكَبَةِ الثَّانِيَةِ خَيْلٌ دُهْمٌ
\par 3 وَفِي الْمَرْكَبَةِ الثَّالِثَةِ خَيْلٌ شُهْبٌ وَفِي الْمَرْكَبَةِ الرَّابِعَةِ خَيْلٌ مُنَمَّرَةٌ شُقْرٌ.
\par 4 فَسَأَلْتُ الْمَلاَكِ الَّذِي كَلَّمَنِي: [مَا هَذِهِ يَا سَيِّدِي؟]
\par 5 فَأَجَابَ الْمَلاَكُ: [هَذِهِ هِيَ أَرْوَاحُ السَّمَاءِ الأَرْبَعُ خَارِجَةٌ مِنَ الْوُقُوفِ لَدَى سَيِّدِ الأَرْضِ كُلِّهَا.
\par 6 الَّتِي فِيهَا الْخَيْلُ الدُّهْمُ تَخْرُجُ إِلَى أَرْضِ الشِّمَالِ وَالشُّهْبُ خَارِجَةٌ وَرَاءَهَا وَالْمُنَمَّرَةُ تَخْرُجُ نَحْوَ أَرْضِ الْجَنُوبِ].
\par 7 أَمَّا الشُّقْرُ فَخَرَجَتْ وَالْتَمَسَتْ أَنْ تَذْهَبَ لِتَتَمَشَّى فِي الأَرْضِ فَقَالَ: [اذْهَبِي وَتَمَشِّي فِي الأَرْضِ]. فَتَمَشَّتْ فِي الأَرْضِ.
\par 8 فَصَرَخَ عَلَيَّ وَقَالَ: [هُوَذَا الْخَارِجُونَ إِلَى أَرْضِ الشِّمَالِ قَدْ سَكَّنُوا رُوحِي فِي أَرْضِ الشِّمَالِ].
\par 9 وَكَانَ إِلَيَّ كَلاَمُ الرَّبِّ:
\par 10 [خُذْ مِنْ أَهْلِ السَّبْيِ مِنْ حَلْدَايَ وَمِنْ طُوبِيَّا وَمِنْ يَدَعْيَا الَّذِينَ جَاءُوا مِنْ بَابِلَ وَتَعَالَ أَنْتَ فِي ذَلِكَ الْيَوْمِ وَادْخُلْ إِلَى بَيْتِ يُوشِيَّا بْنِ صَفَنْيَا.
\par 11 ثُمَّ خُذْ فِضَّةً وَذَهَباً وَاعْمَلْ تِيجَاناً وَضَعْهَا عَلَى رَأْسِ يَهُوشَعَ بْنِ يَهُوصَادَاقَ الْكَاهِنِ الْعَظِيمِ.
\par 12 وَقُلْ لَهُ: هَكَذَا قَالَ رَبُّ الْجُنُودِ: هُوَذَا الرَّجُلُ [الْغُصْنُ] اسْمُهُ. وَمِنْ مَكَانِهِ يَنْبُتُ وَيَبْنِي هَيْكَلَ الرَّبِّ.
\par 13 فَهُوَ يَبْنِي هَيْكَلَ الرَّبِّ وَهُوَ يَحْمِلُ الْجَلاَلَ وَيَجْلِسُ وَيَتَسَلَّطُ عَلَى كُرْسِيِّهِ وَيَكُونُ كَاهِناً عَلَى كُرْسِيِّهِ وَتَكُونُ مَشُورَةُ السَّلاَمِ بَيْنَهُمَا كِلَيْهِمَا.
\par 14 وَتَكُونُ التِّيجَانُ لِحَالِمَ وَلِطُوبِيَّا وَلِيَدَعْيَا وَلِحَيْنِ بْنِ صَفَنْيَا تَذْكَاراً فِي هَيْكَلِ الرَّبِّ.
\par 15 وَالْبَعِيدُونَ يَأْتُونَ وَيَبْنُونَ فِي هَيْكَلِ الرَّبِّ فَتَعْلَمُونَ أَنَّ رَبَّ الْجُنُودِ أَرْسَلَنِي إِلَيْكُمْ. وَيَكُونُ إِذَا سَمِعْتُمْ سَمَعاً صَوْتَ الرَّبِّ إِلَهِكُمْ].

\chapter{7}

\par 1 وَكَانَ فِي السَّنَةِ الرَّابِعَةِ لِدَارِيُوسَ الْمَلِكِ أَنَّ كَلاَمَ الرَّبِّ صَارَ إِلَى زَكَرِيَّا فِي الرَّابِعِ مِنَ الشَّهْرِ التَّاسِعِ فِي كِسْلُو
\par 2 لَمَّا أَرْسَلَ أَهْلُ بَيْتَِ إِيلَ شَرَاصَرَ وَرَجَمَ مَلِكَ وَرِجَالَهُمْ لِيُصَلُّوا قُدَّامَ الرَّبِّ
\par 3 وَلِيَسْأَلُوا الْكَهَنَةَ الَّذِينَ فِي بَيْتِ رَبِّ الْجُنُودِ وَالأَنْبِيَاءَ: [أَأَبْكِي فِي الشَّهْرِ الْخَامِسِ مُنْفَصِلاً كَمَا فَعَلْتُ كَمْ مِنَ السِّنِينَ هَذِهِ؟]
\par 4 ثُمَّ صَارَ إِلَيَّ كَلاَمُ رَبِّ الْجُنُودِ:
\par 5 [اِسْأَلْ جَمِيعِ شَعْبِ الأَرْضِ وَالْكَهَنَةِ: لَمَّا صُمْتُمْ وَنُحْتُمْ فِي الشَّهْرِ الْخَامِسِ وَالشَّهْرِ السَّابِعِ وَذَلِكَ هَذِهِ السَّبْعِينَ سَنَةً فَهَلْ صُمْتُمْ صَوْماً لِي أَنَا؟
\par 6 وَلَمَّا أَكَلْتُمْ وَلَمَّا شَرِبْتُمْ أَفَمَا كُنْتُمْ أَنْتُمُ الآكِلِينَ وَأَنْتُمُ الشَّارِبِينَ؟
\par 7 أَلَيْسَ هَذَا هُوَ الْكَلاَمُ الَّذِي نَادَى بِهِ الرَّبُّ عَنْ يَدِ الأَنْبِيَاءِ الأَوَّلِينَ حِينَ كَانَتْ أُورُشَلِيمُ مَعْمُورَةً وَمُسْتَرِيحَةً وَمُدُنُهَا حَوْلَهَا وَالْجَنُوبُ وَالسَّهْلُ مَعْمُورَيْنِ؟].
\par 8 وَكَانَ كَلاَمُ الرَّبِّ إِلَى زَكَرِيَّا:
\par 9 [هَكَذَا قَالَ رَبُّ الْجُنُودِ: اقْضُوا قَضَاءَ الْحَقِّ وَاعْمَلُوا إِحْسَاناً وَرَحْمَةً كُلُّ إِنْسَانٍ مَعَ أَخِيهِ.
\par 10 وَلاَ تَظْلِمُوا الأَرْمَلَةَ وَلاَ الْيَتِيمَ وَلاَ الْغَرِيبَ وَلاَ الْفَقِيرَ وَلاَ يُفَكِّرْ أَحَدٌ مِنْكُمْ شَرّاً عَلَى أَخِيهِ فِي قَلْبِهِ.
\par 11 فَأَبُوا أَنْ يُصْغُوا وَأَعْطُوا كَتِفاً مُعَانِدَةً وَثَقَّلُوا آذَانَهُمْ عَنِ السَّمْعِ.
\par 12 بَلْ جَعَلُوا قَلْبَهُمْ مَاساً لِئَلاَّ يَسْمَعُوا الشَّرِيعَةَ وَالْكَلاَمَ الَّذِي أَرْسَلَهُ رَبُّ الْجُنُودِ بِرُوحِهِ عَنْ يَدِ الأَنْبِيَاءِ الأَوَّلِينَ. فَجَاءَ غَضَبٌ عَظِيمٌ مِنْ عِنْدِ رَبِّ الْجُنُودِ.
\par 13 فَكَانَ كَمَا نَادَى هُوَ فَلَمْ يَسْمَعُوا كَذَلِكَ يُنَادُونَ هُمْ فَلاَ أَسْمَعُ قَالَ رَبُّ الْجُنُودِ.
\par 14 وَأَعْصِفُهُمْ إِلَى كُلِّ الأُمَمِ الَّذِينَ لَمْ يَعْرِفُوهُمْ. فَخَرِبَتِ الأَرْضُ وَرَاءَهُمْ لاَ ذَاهِبَ وَلاَ آئِبَ. فَجَعَلُوا الأَرْضَ الْبَهِجَةَ خَرَاباً].

\chapter{8}

\par 1 وَكَانَ كَلاَمُ رَبِّ الْجُنُودِ:
\par 2 [هَكَذَا قَالَ رَبُّ الْجُنُودِ: غِرْتُ عَلَى صِهْيَوْنَ غَيْرَةً عَظِيمَةً وَبِسَخَطٍ عَظِيمٍ غِرْتُ عَلَيْهَا].
\par 3 هَكَذَا قَالَ الرَّبُّ: [قَدْ رَجَعْتُ إِلَى صِهْيَوْنَ وَأَسْكُنُ فِي وَسَطِ أُورُشَلِيمَ فَتُدْعَى أُورُشَلِيمُ مَدِينَةَ الْحَقِّ وَجَبَلُ رَبِّ الْجُنُودِ الْجَبَلَ الْمُقَدَّسَ].
\par 4 هَكَذَا قَالَ رَبُّ الْجُنُودِ: [سَيَجْلِسُ بَعْدُ الشُّيُوخُ وَالشَّيْخَاتُ فِي أَسْوَاقِ أُورُشَلِيمَ كُلُّ إِنْسَانٍ مِنْهُمْ عَصَاهُ بِيَدِهِ مِنْ كَثْرَةِ الأَيَّامِ.
\par 5 وَتَمْتَلِئُ أَسْوَاقُ الْمَدِينَةِ مِنَ الصِّبْيَانِ وَالْبَنَاتِ لاَعِبِينَ فِي أَسْوَاقِهَا].
\par 6 هَكَذَا قَالَ رَبُّ الْجُنُودِ: [إِنْ يَكُنْ ذَلِكَ عَجِيباً فِي أَعْيُنِ بَقِيَّةِ هَذَا الشَّعْبِ فِي هَذِهِ الأَيَّامِ أَفَيَكُونُ أَيْضاً عَجِيباً فِي عَيْنَيَّ يَقُولُ رَبُّ الْجُنُودِ؟].
\par 7 هَكَذَا قَالَ رَبُّ الْجُنُودِ: [هَئَنَذَا أُخَلِّصُ شَعْبِي مِنْ أَرْضِ الْمَشْرِقِ وَمِنْ أَرْضِ مَغْرِبِ الشَّمْسِ.
\par 8 وَآتِي بِهِمْ فَيَسْكُنُونَ فِي وَسَطِ أُورُشَلِيمَ وَيَكُونُونَ لِي شَعْباًوَأَنَا أَكُونُ لَهُمْ إِلَهاً بِالْحَقِّ وَالْبِرِّ].
\par 9 هَكَذَا قَالَ رَبُّ الْجُنُودِ: [لِتَتَشَدَّدْ أَيْدِيكُمْ أَيُّهَا السَّامِعُونَ فِي هَذِهِ الأَيَّامِ هَذَا الْكَلاَمَ مِنْ أَفْوَاهِ الأَنْبِيَاءِ الَّذِي كَانَ يَوْمَ أُسِّسَ بَيْتُ رَبِّ الْجُنُودِ لِبِنَاءِ الْهَيْكَلِ.
\par 10 لأَنَّهُ قَبْلَ هَذِهِ الأَيَّامِ لَمْ تَكُنْ لِلإِنْسَانِ أُجْرَةٌ وَلاَ لِلْبَهِيمَةِ أُجْرَةٌ وَلاَ سَلاَمٌ لِمَنْ خَرَجَ أَوْ دَخَلَ مِنْ قِبَلِ الضِّيقِ. وَأَطْلَقْتُ كُلَّ إِنْسَانٍ الرَّجُلَ عَلَى قَرِيبِهِ.
\par 11 أَمَّا الآنَ فَلاَ أَكُونُ أَنَا لِبَقِيَّةِ هَذَا الشَّعْبِ كَمَا فِي الأَيَّامِ الأُولَى: يَقُولُ رَبُّ الْجُنُودِ
\par 12 بَلْ زَرْعُ السَّلاَمِ. الْكَرْمُ يُعْطِي ثَمَرَهُ وَالأَرْضُ تُعْطِي غَلَّتَهَا وَالسَّمَاوَاتُ تُعْطِي نَدَاهَا وَأُمَلِّكُ بَقِيَّةَ هَذَا الشَّعْبِ هَذِهِ كُلَّهَا.
\par 13 وَيَكُونُ كَمَا أَنَّكُمْ كُنْتُمْ لَعْنَةً بَيْنَ الأُمَمِ يَا بَيْتَ يَهُوذَا وَيَا بَيْتَ إِسْرَائِيلَ كَذَلِكَ أُخَلِّصُكُمْ فَتَكُونُونَ بَرَكَةً فَلاَ تَخَافُوا. لِتَتَشَدَّدْ أَيْدِيكُمْ].
\par 14 لأَنَّهُ هَكَذَا قَالَ رَبُّ الْجُنُودِ: [كَمَا أَنِّي فَكَّرْتُ فِي أَنْ أُسِيءَ إِلَيْكُمْ حِينَ أَغْضَبَنِي آبَاؤُكُمْ وَلَمْ أَنْدَمْ
\par 15 هَكَذَا عُدْتُ وَفَكَّرْتُ فِي هَذِهِ الأَيَّامِ فِي أَنْ أُحْسِنَ إِلَى أُورُشَلِيمَ وَبَيْتَِ يَهُوذَا. لاَ تَخَافُوا.
\par 16 هَذِهِ هِيَ الأُمُورُ الَّتِي تَفْعَلُونَهَا. لِيُكَلِّمْ كُلُّ إِنْسَانٍ قَرِيبَهُ بِالْحَقِّ. اقْضُوا بِالْحَقِّ وَقَضَاءِ السَّلاَمِ فِي أَبْوَابِكُمْ.
\par 17 وَلاَ يُفَكِّرَنَّ أَحَدٌ فِي السُّوءِ عَلَى قَرِيبِهِ فِي قُلُوبِكُمْ. وَلاَ تُحِبُّوا يَمِينَ الزُّورِ. لأَنَّ هَذِهِ جَمِيعَهَا أَكْرَهُهَا يَقُولُ الرَّبُّ].
\par 18 وَكَانَ إِلَيَّ كَلاَمُ رَبِّ الْجُنُودِ:
\par 19 [هَكَذَا قَالَ رَبُّ الْجُنُودِ: إِنَّ صَوْمَ الشَّهْرِ الرَّابِعِ وَصَوْمَ الْخَامِسِ وَصَوْمَ السَّابِعِ وَصَوْمَ الْعَاشِرِ يَكُونُ لِبَيْتِ يَهُوذَا ابْتِهَاجاً وَفَرَحاً وَأَعْيَاداً طَيِّبَةً. فَأَحِبُّوا الْحَقَّ وَالسَّلاَمَ].
\par 20 هَكَذَا قَالَ رَبُّ الْجُنُودِ: [سَيَأْتِي شُعُوبٌ بَعْدُ وَسُكَّانُ مُدُنٍ كَثِيرَةٍ.
\par 21 وَسُكَّانُ وَاحِدَةٍ يَسِيرُونَ إِلَى أُخْرَى قَائِلِينَ: لِنَذْهَبْ ذَهَاباً لِنَتَرَضَّى وَجْهَ الرَّبِّ وَنَطْلُبَ رَبَّ الْجُنُودِ. أَنَا أَيْضاً أَذْهَبُ].
\par 22 فَتَأْتِي شُعُوبٌ كَثِيرَةٌ وَأُمَمٌ قَوِيَّةٌ لِيَطْلُبُوا رَبَّ الْجُنُودِ فِي أُورُشَلِيمَ وَلْيَتَرَضُّوا وَجْهَ الرَّبِّ.
\par 23 هَكَذَا قَالَ رَبُّ الْجُنُودِ: [فِي تِلْكَ الأَيَّامِ يُمْسِكُ عَشَرَةُ رِجَالٍ مِنْ جَمِيعِ أَلْسِنَةِ الأُمَمِ يَتَمَسَّكُونَ بِذَيْلِ رَجُلٍ يَهُودِيٍّ قَائِلِينَ: نَذْهَبُ مَعَكُمْ لأَنَّنَا سَمِعْنَا أَنَّ اللَّهَ مَعَكُمْ].

\chapter{9}

\par 1 وَحْيُ كَلِمَةِ الرَّبِّ فِي أَرْضِ حَدْرَاخَ وَدِمَشْقَ مَحَلُّهُ. (لأَنَّ لِلرَّبِّ عَيْنَ الإِنْسَانِ وَكُلَّ أَسْبَاطِ إِسْرَائِيلَ).
\par 2 وَحَمَاةُ أَيْضاً تُتَاخِمُهَا وَصُورُ وَصَيْدُونُ وَإِنْ تَكُنْ حَكِيمَةً جِدّاً.
\par 3 وَقَدْ بَنَتْ صُورُ حِصْناً لِنَفْسِهَا وَكَوَّمَتِ الْفِضَّةَ كَالتُّرَابِ وَالذَّهَبَ كَطِينِ الأَسْوَاقِ.
\par 4 هُوَذَا السَّيِّدُ يَمْتَلِكُهَا وَيَضْرِبُ فِي الْبَحْرِ قُوَّتَهَا وَهِيَ تُؤْكَلُ بِالنَّارِ.
\par 5 تَرَى أَشْقَلُونُ فَتَخَافُ وَغَزَّةُ فَتَتَوَجَّعُ جِدّاً وَعَقْرُونُ. لأَنَّهُ يُخْزِيهَا انْتِظَارُهَا وَالْمَلِكُ يَبِيدُ مِنْ غَزَّةَ وَأَشْقَلُونُ لاَ تُسْكَنُ.
\par 6 وَيَسْكُنُ فِي أَشْدُودَ زَنِيمٌ وَأَقْطَعُ كِبْرِيَاءَ الْفِلِسْطِينِيِّينَ.
\par 7 وَأَنْزِعُ دِمَاءَهُ مِنْ فَمِهِ وَرِجْسَهُ مِنْ بَيْنِ أَسْنَانِهِ فَيَبْقَى هُوَ أَيْضاً لإِلَهِنَا وَيَكُونُ كَأَمِيرٍ فِي يَهُوذَا وَعَقْرُونُ كَيَبُوسِيٍّ.
\par 8 وَأَحُلُّ حَوْلَ بَيْتِي بِسَبَبِ الْجَيْشِ الذَّاهِبِ وَالآئِبِ فَلاَ يَعْبُرُ عَلَيْهِمْ بَعْدُ جَابِي الْجِزْيَةِ. فَإِنِّي الآنَ رَأَيْتُ بِعَيْنَيَّ.
\par 9 [اِبْتَهِجِي جِدّاً يَا ابْنَةَ صِهْيَوْنَ اهْتِفِي يَا بِنْتَ أُورُشَلِيمَ. هُوَذَا مَلِكُكِ يَأْتِي إِلَيْكِ. هُوَ عَادِلٌ وَمَنْصُورٌ وَدِيعٌ وَرَاكِبٌ عَلَى حِمَارٍ وَعَلَى جَحْشٍ ابْنِ أَتَانٍ.
\par 10 وَأَقْطَعُ الْمَرْكَبَةَ مِنْ أَفْرَايِمَ وَالْفَرَسَ مِنْ أُورُشَلِيمَ وَتُقْطَعُ قَوْسُ الْحَرْبِ. وَيَتَكَلَّمُ بِالسَّلاَمِ لِلأُمَمِ وَسُلْطَانُهُ مِنَ الْبَحْرِ إِلَى الْبَحْرِ وَمِنَ النَّهْرِ إِلَى أَقَاصِي الأَرْضِ.
\par 11 وَأَنْتِ أَيْضاً فَإِنِّي بِدَمِ عَهْدِكِ قَدْ أَطْلَقْتُ أَسْرَاكِ مِنَ الْجُبِّ الَّذِي لَيْسَ فِيهِ مَاءٌ.
\par 12 ارْجِعُوا إِلَى الْحِصْنِ يَا أَسْرَى الرَّجَاءِ. الْيَوْمَ أَيْضاً أُصَرِّحُ أَنِّي أَرُدُّ عَلَيْكِ ضِعْفَيْنِ.
\par 13 [لأَنِّي أَوْتَرْتُ يَهُوذَا لِنَفْسِي وَمَلَأْتُ الْقَوْسَ أَفْرَايِمَ وَأَنْهَضْتُ أَبْنَاءَكِ يَا صِهْيَوْنُ عَلَى بَنِيكِ يَا يَاوَانُ وَجَعَلْتُكِ كَسَيْفِ جَبَّارٍ].
\par 14 وَيُرَى الرَّبُّ فَوْقَهُمْ وَسَهْمُهُ يَخْرُجُ كَالْبَرْقِ وَالسَّيِّدُ الرَّبُّ يَنْفُخُ فِي الْبُوقِ وَيَسِيرُ فِي زَوَابِعِ الْجَنُوبِ.
\par 15 رَبُّ الْجُنُودِ يُحَامِي عَنْهُمْ فَيَأْكُلُونَ وَيَدُوسُونَ حِجَارَةَ الْمِقْلاَعِ وَيَشْرَبُونَ وَيَضِجُّونَ كَمَا مِنَ الْخَمْرِ وَيَمْتَلِئُونَ كَالْمَنْضَحِ وَكَزَوَايَا الْمَذْبَحِ.
\par 16 وَيُخَلِّصُهُمُ الرَّبُّ إِلَهُهُمْ فِي ذَلِكَ الْيَوْمِ. كَقَطِيعٍ شَعْبَهُ بَلْ كَحِجَارَةِ التَّاجِ مَرْفُوعَةً عَلَى أَرْضِهِ.
\par 17 مَا أَجْوَدَهُ وَمَا أَجْمَلَهُ! الْحِنْطَةُ تُنْمِي الْفِتْيَانَ وَالْمِسْطَارُ الْعَذَارَى.

\chapter{10}

\par 1 اُطْلُبُوا مِنَ الرَّبِّ الْمَطَرَ فِي أَوَانِ الْمَطَرِ الْمُتَأَخِّرِ فَيَصْنَعَ الرَّبُّ بُرُوقاً وَيُعْطِيَهُمْ مَطَرَ الْوَبْلِ. لِكُلِّ إِنْسَانٍ عُشْباً فِي الْحَقْلِ.
\par 2 لأَنَّ التَّرَافِيمَ قَدْ تَكَلَّمُوا بِالْبَاطِلِ وَالْعَرَّافِيِنَ رَأَوُا الْكَذِبَ وَأَخْبَرُوا بِأَحْلاَمِ كَذِبٍ. يُعَزُّونَ بِالْبَاطِلِ. لِذَلِكَ رَحَلُوا كَغَنَمٍ. ذَلُّوا إِذْ لَيْسَ رَاعٍ.
\par 3 [عَلَى الرُّعَاةِ اشْتَعَلَ غَضَبِي فَعَاقَبْتُ الأَعْتِدَةَ. لأَنَّ رَبَّ الْجُنُودِ قَدْ تَعَهَّدَ قَطِيعَهُ بَيْتَ يَهُوذَا وَجَعَلَهُمْ كَفَرَسِ جَلاَلِهِ فِي الْقِتَالِ.
\par 4 مِنْهُ الزَّاوِيَةُ. مِنْهُ الْوَتَدُ. مِنْهُ قَوْسُ الْقِتَالِ. مِنْهُ يَخْرُجُ كُلُّ ظَالِمٍ جَمِيعاً.
\par 5 وَيَكُونُونَ كَالْجَبَابِرَةِ الدَّائِسِينَ طِينَ الأَسْوَاقِ فِي الْقِتَالِ وَيُحَارِبُونَ لأَنَّ الرَّبَّ مَعَهُمْ وَالرَّاكِبُونَ الْخَيْلَ يَخْزُونَ.
\par 6 وَأُقَوِّي بَيْتَ يَهُوذَا وَأُخَلِّصُ بَيْتَ يُوسُِفَ وَأُرَجِّعُهُمْ لأَنِّي قَدْ رَحِمْتُهُمْ. وَيَكُونُونَ كَأَنِّي لَمْ أَرْفُضْهُمْ لأَنِّي أَنَا الرَّبُّ إِلَهُهُمْ فَأُجِيبُهُمْ.
\par 7 وَيَكُونُ أَفْرَايِمُ كَجَبَّارٍ وَيَفْرَحُ قَلْبُهُمْ كَأَنَّهُ بِالْخَمْرِ وَيَنْظُرُ بَنُوهُمْ فَيَفْرَحُونَ وَيَبْتَهِجُ قَلْبُهُمْ بِالرَّبِّ.
\par 8 أَصْفِرُ لَهُمْ وَأَجْمَعُهُمْ لأَنِّي قَدْ فَدَيْتُهُمْ وَيَكْثُرُونَ كَمَا كَثُرُوا.
\par 9 وَأَزْرَعُهُمْ بَيْنَ الشُّعُوبِ فَيَذْكُرُونَنِي فِي الأَرَاضِي الْبَعِيدَةِ وَيَحْيُونَ مَعَ بَنِيهِمْ وَيَرْجِعُونَ.
\par 10 وَأُرْجِعُهُمْ مِنْ أَرْضِ مِصْرَ وَأَجْمَعُهُمْ مِنْ أَشُّورَ وَآتِي بِهِمْ إِلَى أَرْضِ جِلْعَادَ وَلُبْنَانَ وَلاَ يُوجَدُ لَهُمْ مَكَانٌ.
\par 11 وَيَعْبُرُ فِي بَحْرِ الضِّيقِ وَيَضْرِبُ اللُّجَجَ فِي الْبَحْرِ وَتَجِفُّ كُلُّ أَعْمَاقِ النَّهْرِ وَتُخْفَضُ كِبْرِيَاءُ أَشُّورَ وَيَزُولُ قَضِيبُ مِصْرَ.
\par 12 وَأُقَوِّيهِمْ بِالرَّبِّ فَيَسْلُكُونَ بِاسْمِهِ] يَقُولُ الرَّبُّ.

\chapter{11}

\par 1 اِفْتَحْ أَبْوَابَكَ يَا لُبْنَانُ فَتَأْكُلَ النَّارُ أَرْزَكَ.
\par 2 وَلْوِلْ يَا سَرْوُ لأَنَّ الأَرْزَ سَقَطَ لأَنَّ الأَعِزَّاءَ قَدْ خَرِبُوا. وَلْوِلْ يَا بَلُّوطَ بَاشَانَ لأَنَّ الْوَعْرَ الْمَنِيعَ قَدْ هَبَطَ.
\par 3 صَوْتُ وَلْوَلَةِ الرُّعَاةِ لأَنَّ فَخْرَهُمْ خَرِبَ. صَوْتُ زَمْجَرَةِ الأَشْبَالِ لأَنَّ كِبْرِيَاءَ الأُرْدُنِّ خَرِبَتْ.
\par 4 هَكَذَا قَالَ الرَّبُّ إِلَهِي: [ارْعَ غَنَمَ الذَّبْحِ
\par 5 الَّذِينَ يَذْبَحُهُمْ مَالِكُوهُمْ وَلاَ يَأْثَمُونَ وَبَائِعُوهُمْ يَقُولُونَ: مُبَارَكٌ الرَّبُّ! قَدِ اسْتَغْنَيْتُ. وَرُعَاتُهُمْ لاَ يُشْفِقُونَ عَلَيْهِمْ.
\par 6 لأَنِّي لاَ أُشْفِقُ بَعْدُ عَلَى سُكَّانِ الأَرْضِ يَقُولُ الرَّبُّ بَلْ هَئَنَذَا مُسَلِّمٌ الإِنْسَانَ كُلَّ رَجُلٍ لِيَدِ قَرِيبِهِ وَلِيَدِ مَلِكِهِ فَيَضْرِبُونَ الأَرْضَ وَلاَ أُنْقِذُ مِنْ يَدِهِمْ].
\par 7 فَرَعَيْتُ غَنَمَ الذَّبْحِ. لَكِنَّهُمْ أَذَلُّ الْغَنَمِ. وَأَخَذْتُ لِنَفْسِي عَصَوَيْنِ فَسَمَّيْتُ الْوَاحِدَةَ [نِعْمَةَ] وَسَمَّيْتُ الأُخْرَى [حِبَالاً] وَرَعَيْتُ الْغَنَمَ.
\par 8 وَأَبَدْتُ الرُّعَاةَ الثَّلاَثَةَ فِي شَهْرٍ وَاحِدٍ وَضَاقَتْ نَفْسِي بِهِمْ وَكَرِهَتْنِي أَيْضاً نَفْسُهُمْ.
\par 9 فَقُلْتُ: [لاَ أَرْعَاكُمْ. مَنْ يَمُتْ فَلْيَمُتْ وَمَنْ يُبَدْ فَلْيُبَدْ. وَالْبَقِيَّةُ فَلْيَأْكُلْ بَعْضُهَا لَحْمَ بَعْضٍ!].
\par 10 فَأَخَذْتُ عَصَايَ [نِعْمَةَ] وَقَصَفْتُهَا لأَنْقُضَ عَهْدِي الَّذِي قَطَعْتُهُ مَعَ كُلِّ الأَسْبَاطِ.
\par 11 فَنُقِضَ فِي ذَلِكَ الْيَوْمِ. وَهَكَذَا عَلِمَ أَذَلُّ الْغَنَمِ الْمُنْتَظِرُونَ لِي أَنَّهَا كَلِمَةُ الرَّبِّ.
\par 12 فَقُلْتُ لَهُمْ: [إِنْ حَسُنَ فِي أَعْيُنِكُمْ فَأَعْطُونِي أُجْرَتِي وَإِلاَّ فَامْتَنِعُوا]. فَوَزَنُوا أُجْرَتِي ثَلاَثِينَ مِنَ الْفِضَّةِ.
\par 13 فَقَالَ لِي الرَّبُّ: [أَلْقِهَا إِلَى الْفَخَّارِيِّ الثَّمَنَ الْكَرِيمَ الَّذِي ثَمَّنُونِي بِهِ]. فَأَخَذْتُ الثَّلاَثِينَ مِنَ الْفِضَّةِ وَأَلْقَيْتُهَا إِلَى الْفَخَّارِيِّ فِي بَيْتِ الرَّبِّ.
\par 14 ثُمَّ قَصَفْتُ عَصَايَ الأُخْرَى [حِبَالاً] لأَنْقُضَ الإِخَاءَ بَيْنَ يَهُوذَا وَإِسْرَائِيلَ.
\par 15 فَقَالَ لِي الرَّبُّ: [خُذْ لِنَفْسِكَ بَعْدُ أَدَوَاتِ رَاعٍ أَحْمَقَ
\par 16 لأَنِّي هَئَنَذَا مُقِيمٌ رَاعِياً فِي الأَرْضِ لاَ يَفْتَقِدُ الْمُنْقَطِعِينَ وَلاَ يَطْلُبُ الْمُنْسَاقَ وَلاَ يَجْبُرُ الْمُنْكَسِرَ وَلاَ يُرَبِّي الْقَائِمَ. وَلَكِنْ يَأْكُلُ لَحْمَ السِّمَانِ وَيَنْزِعُ أَظْلاَفَهَا].
\par 17 وَيْلٌ لِلرَّاعِي الْبَاطِلِ التَّارِكِ الْغَنَمِ! السَّيْفُ عَلَى ذِرَاعِهِ وَعَلَى عَيْنِهِ الْيُمْنَى. ذِرَاعُهُ تَيْبَسُ يَبْساً وَعَيْنُهُ الْيُمْنَى تَكِلُّ كُلُولاً!

\chapter{12}

\par 1 وَحْيُ كَلاَمِ الرَّبِّ عَلَى إِسْرَائِيلَ. يَقُولُ الرَّبُّ بَاسِطُ السَّمَاوَاتِ وَمُؤَسِّسُ الأَرْضِ وَجَابِلُ رُوحِ الإِنْسَانِ فِي دَاخِلِهِ:
\par 2 [هَئَنَذَا أَجْعَلُ أُورُشَلِيمَ كَأْسَ تَرَنُّحٍ لِجَمِيعِ الشُّعُوبِ حَوْلَهَا وَأَيْضاً عَلَى يَهُوذَا تَكُونُ فِي حِصَارِ أُورُشَلِيمَ.
\par 3 وَيَكُونُ فِي ذَلِكَ الْيَوْمِ أَنِّي أَجْعَلُ أُورُشَلِيمَ حَجَراً مِشْوَالاً لِجَمِيعِ الشُّعُوبِ وَكُلُّ الَّذِينَ يَشِيلُونَهُ يَنْشَقُّونَ شَقّاً. وَيَجْتَمِعُ عَلَيْهَا كُلُّ أُمَمِ الأَرْضِ.
\par 4 فِي ذَلِكَ الْيَوْمِ يَقُولُ الرَّبُّ أَضْرِبُ كُلَّ فَرَسٍ بِالْحَِيْرَةِ وَرَاكِبَهُ بِالْجُنُونِ. وَأَفْتَحُ عَيْنَيَّ عَلَى بَيْتِ يَهُوذَا وَأَضْرِبُ كُلَّ خَيْلِ الشُّعُوبِ بِالْعَمَى.
\par 5 فَتَقُولُ أُمَرَاءُ يَهُوذَا فِي قَلْبِهِمْ: إِنَّ سُكَّانَ أُورُشَلِيمَ قُوَّةٌ لِي بِرَبِّ الْجُنُودِ إِلَهِهِمْ.
\par 6 فِي ذَلِكَ الْيَوْمِ أَجْعَلُ أُمَرَاءَ يَهُوذَا كَمِصْبَاحِ نَارٍ بَيْنَ الْحَطَبِ وَكَمِشْعَلِ نَارٍ بَيْنَ الْحُزَمِ فَيَأْكُلُونَ كُلَّ الشُّعُوبِ حَوْلَهُمْ عَنِ الْيَمِينِ وَعَنِ الْيَسَارِ فَتَثْبُتُ أُورُشَلِيمُ أَيْضاً فِي مَكَانِهَا بِأُورُشَلِيمَ.
\par 7 وَيُخَلِّصُ الرَّبُّ خِيَامَ يَهُوذَا أَوَّلاً لِكَيْلاَ يَتَعَاظَمَ افْتِخَارُ بَيْتِ دَاوُدَ وَافْتِخَارُ سُكَّانِ أُورُشَلِيمَ عَلَى يَهُوذَا.
\par 8 فِي ذَلِكَ الْيَوْمِ يَسْتُرُ الرَّبُّ سُكَّانَ أُورُشَلِيمَ فَيَكُونُ الْعَاثِرُ مِنْهُمْ فِي ذَلِكَ الْيَوْمِ مِثْلَ دَاوُدَ وَبَيْتُ دَاوُدَ مِثْلَ اللَّهِ مِثْلَ مَلاَكِ الرَّبِّ أَمَامَهُمْ.
\par 9 وَيَكُونُ فِي ذَلِكَ الْيَوْمِ أَنِّي أَلْتَمِسُ هَلاَكَ كُلِّ الأُمَمِ الآتِينَ عَلَى أُورُشَلِيمَ.
\par 10 [وَأُفِيضُ عَلَى بَيْتِ دَاوُدَ وَعَلَى سُكَّانِ أُورُشَلِيمَ رُوحَ النِّعْمَةِ وَالتَّضَرُّعَاتِ فَيَنْظُرُونَ إِلَيَّ الَّذِي طَعَنُوهُ وَيَنُوحُونَ عَلَيْهِ كَنَائِحٍ عَلَى وَحِيدٍ لَهُ وَيَكُونُونَ فِي مَرَارَةٍ عَلَيْهِ كَمَنْ هُوَ فِي مَرَارَةٍ عَلَى بِكْرِهِ.
\par 11 فِي ذَلِكَ الْيَوْمِ يَعْظُمُ النَّوْحُ فِي أُورُشَلِيمَ كَنَوْحِ هَدَدْرِمُّونَ فِي بُقْعَةِ مَجِدُّونَ.
\par 12 وَتَنُوحُ الأَرْضُ عَشَائِرَ عَشَائِرَ عَلَى حِدَتِهَا: عَشِيرَةُ بَيْتِ دَاوُدَ عَلَى حِدَتِهَا وَنِسَاؤُهُمْ عَلَى حِدَتِهِنَّ. عَشِيرَةُ بَيْتِ نَاثَانَ عَلَى حِدَتِهَا وَنِسَاؤُهُمْ عَلَى حِدَتِهِنَّ.
\par 13 عَشِيرَةُ بَيْتِ لاَوِي عَلَى حِدَتِهَا وَنِسَاؤُهُمْ عَلَى حِدَتِهِنَّ. عَشِيرَةُ شَمْعِي عَلَى حِدَتِهَا وَنِسَاؤُهُمْ عَلَى حِدَتِهِنَّ.
\par 14 كُلُّ الْعَشَائِرِ الْبَاقِيَةِ عَشِيرَةٌ عَشِيرَةٌ عَلَى حِدَتِهَا وَنِسَاؤُهُمْ عَلَى حِدَتِهِنَّ].

\chapter{13}

\par 1 [فِي ذَلِكَ الْيَوْمِ يَكُونُ يَنْبُوعٌ مَفْتُوحاً لِبَيْتِ دَاوُدَ وَلِسُكَّانِ أُورُشَلِيمَ لِلْخَطِيَّةِ وَلِلْنَجَاسَةِ.
\par 2 وَيَكُونُ فِي ذَلِكَ الْيَوْمِ يَقُولُ رَبُّ الْجُنُودِ أَنِّي أَقْطَعُ أَسْمَاءَ الأَصْنَامِ مِنَ الأَرْضِ فَلاَ تُذْكَرُ بَعْدُ وَأُزِيلُ الأَنْبِيَاءَ أَيْضاً وَالرُّوحَ النَّجِسَ مِنَ الأَرْضِ.
\par 3 وَيَكُونُ إِذَا تَنَبَّأَ أَحَدٌ بَعْدُ أَنَّ أَبَاهُ وَأُمَّهُ (وَالِدَيْهِ) يَقُولاَنِ لَهُ: لاَ تَعِيشُ لأَنَّكَ تَكَلَّمْتَ بِالْكَذِبِ بِاسْمِ الرَّبِّ. فَيَطْعَنُهُ أَبُوهُ وَأُمُّهُ (وَالِدَاهُ) عِنْدَمَا يَتَنَبَّأُ.
\par 4 وَيَكُونُ فِي ذَلِكَ الْيَوْمِ أَنَّ الأَنْبِيَاءَ يَخْزُونَ كُلُّ وَاحِدٍ مِنْ رُؤْيَاهُ إِذَا تَنَبَّأَ وَلاَ يَلْبِسُونَ ثَوْبَ شَعْرٍ لأَجْلِ الْغِشِّ.
\par 5 بَلْ يَقُولُ: لَسْتُ أَنَا نَبِيّاً. أَنَا إِنْسَانٌ فَالِحُ الأَرْضِ لأَنَّ إِنْسَاناً اقْتَنَانِي مِنْ صِبَايَ.
\par 6 فَيَسْأَلَهُ: مَا هَذِهِ الْجُرُوحُ فِي يَدَيْكَ؟ فَيَقُولُ: هِيَ الَّتِي جُرِحْتُ بِهَا فِي بَيْتِ أَحِبَّائِي.
\par 7 [اِسْتَيْقِظْ يَا سَيْفُ عَلَى رَاعِيَّ وَعَلَى رَجُلِ رِفْقَتِي يَقُولُ رَبُّ الْجُنُودِ. اضْرِبِ الرَّاعِيَ فَتَتَشَتَّتَ الْغَنَمُ وَأَرُدُّ يَدِي عَلَى الصِّغَارِ.
\par 8 وَيَكُونُ فِي كُلِّ الأَرْضِ يَقُولُ الرَّبُّ أَنَّ ثُلْثَيْنِ مِنْهَا يُقْطَعَانِ وَيَمُوتَانِ وَالثُّلْثَ يَبْقَى فِيهَا.
\par 9 وَأُدْخِلُ الثُّلْثَ فِي النَّارِ وَأَمْحَصُهُمْ كَمَحْصِ الْفِضَّةِ وَأَمْتَحِنُهُمُ امْتِحَانَ الذَّهَبِ. هُوَ يَدْعُو بِاسْمِي وَأَنَا أُجِيبُهُ. أَقُولُ: هُوَ شَعْبِي وَهُوَ يَقُولُ: الرَّبُّ إِلَهِي].

\chapter{14}

\par 1 هُوَذَا يَوْمٌ لِلرَّبِّ يَأْتِي فَيُقْسَمُ سَلَبُكِ فِي وَسَطِكِ.
\par 2 وَأَجْمَعُ كُلَّ الأُمَمِ عَلَى أُورُشَلِيمَ لِلْمُحَارَبَةِ فَتُؤْخَذُ الْمَدِينَةُ وَتُنْهَبُ الْبُيُوتُ وَتُفْضَحُ النِّسَاءُ وَيَخْرُجُ نِصْفُ الْمَدِينَةِ إِلَى السَّبْيِ وَبَقِيَّةُ الشَّعْبِ لاَ تُقْطَعُ مِنَ الْمَدِينَةِ.
\par 3 فَيَخْرُجُ الرَّبُّ وَيُحَارِبُ تِلْكَ الأُمَمَ كَمَا فِي يَوْمِ حَرْبِهِ يَوْمَ الْقِتَالِ.
\par 4 وَتَقِفُ قَدَمَاهُ فِي ذَلِكَ الْيَوْمِ عَلَى جَبَلِ الزَّيْتُونِ الَّذِي قُدَّامَ أُورُشَلِيمَ مِنَ الشَّرْقِ فَيَنْشَقُّ جَبَلُ الزَّيْتُونِ مِنْ وَسَطِهِ نَحْوَ الشَّرْقِ وَنَحْوَ الْغَرْبِ وَادِياً عَظِيماً جِدّاً وَيَنْتَقِلُ نِصْفُ الْجَبَلِ نَحْوَ الشِّمَالِ وَنِصْفُهُ نَحْوَ الْجَنُوبِ.
\par 5 وَتَهْرُبُونَ فِي جِوَاءِ جِبَالِي لأَنَّ جِوَاءَ الْجِبَالِ يَصِلُ إِلَى آصَلَ. وَتَهْرُبُونَ كَمَا هَرَبْتُمْ مِنَ الزَّلْزَلَةِ فِي أَيَّامِ عُزِّيَّا مَلِكِ يَهُوذَا. وَيَأْتِي الرَّبُّ إِلَهِي وَجَمِيعُ الْقِدِّيسِينَ مَعَكَ.
\par 6 وَيَكُونُ فِي ذَلِكَ الْيَوْمِ أَنَّهُ لاَ يَكُونُ نُورٌ. الدَّرَارِي تَنْقَبِضُ.
\par 7 وَيَكُونُ يَوْمٌ وَاحِدٌ مَعْرُوفٌ لِلرَّبِّ. لاَ نَهَارَ وَلاَ لَيْلَ بَلْ يَحْدُثُ أَنَّهُ فِي وَقْتِ الْمَسَاءِ يَكُونُ نُورٌ.
\par 8 وَيَكُونُ فِي ذَلِكَ الْيَوْمِ أَنَّ مِيَاهاً حَيَّةً تَخْرُجُ مِنْ أُورُشَلِيمَ نِصْفُهَا إِلَى الْبَحْرِ الشَّرْقِيِّ وَنِصْفُهَا إِلَى الْبَحْرِ الْغَرْبِيِّ. فِي الصَّيْفِ وَفِي الْخَرِيفِ تَكُونُ.
\par 9 وَيَكُونُ الرَّبُّ مَلِكاً عَلَى كُلِّ الأَرْضِ. فِي ذَلِكَ الْيَوْمِ يَكُونُ الرَّبُّ وَحْدَهُ وَاسْمُهُ وَحْدَهُ.
\par 10 وَتَتَحَوَّلُ الأَرْضُ كُلُّهَا كَالْعَرَبَةِ مِنْ جَبْعَ إِلَى رِمُّونَ جَنُوبَ أُورُشَلِيمَ. وَتَرْتَفِعُ وَتُعْمَرُ فِي مَكَانِهَا مِنْ بَابِ بِنْيَامِينَ إِلَى مَكَانِ الْبَابِ الأَوَّلِ إِلَى بَابِ الزَّوَايَا وَمِنْ بُرْجِ حَنَنْئِيلَ إِلَى مَعَاصِرِ الْمَلِكِ.
\par 11 فَيَسْكُنُونَ فِيهَا وَلاَ يَكُونُ بَعْدُ لَعْنٌ. فَتُعْمَرُ أُورُشَلِيمُ بِالأَمْنِ.
\par 12 وَهَذِهِ تَكُونُ الضَّرْبَةُ الَّتِي يَضْرِبُ بِهَا الرَّبُّ كُلَّ الشُّعُوبِ الَّذِينَ تَجَنَّدُوا عَلَى أُورُشَلِيمَ. لَحْمُهُمْ يَذُوبُ وَهُمْ وَاقِفُونَ عَلَى أَقْدَامِهِمْ وَعُيُونُهُمْ تَذُوبُ فِي أَوْقَابِهَا وَلِسَانُهُمْ يَذُوبُ فِي فَمِهِمْ.
\par 13 وَيَكُونُ فِي ذَلِكَ الْيَوْمِ أَنَّ اضْطِرَاباً عَظِيماً مِنَ الرَّبِّ يَحْدُثُ فِيهِمْ فَيُمْسِكُ الرَّجُلُ بِيَدِ قَرِيبِهِ وَتَعْلُو يَدُهُ عَلَى يَدِ قَرِيبِهِ.
\par 14 وَيَهُوذَا أَيْضاً تُحَارِبُ أُورُشَلِيمَ وَتُجْمَعُ ثَرْوَةُ كُلِّ الأُمَمِ مِنْ حَوْلِهَا: ذَهَبٌ وَفِضَّةٌ وَمَلاَبِسُ كَثِيرَةٌ جِدّاً.
\par 15 وَكَذَا تَكُونُ ضَرْبَةُ الْخَيْلِ وَالْبِغَالِ وَالْجِمَالِ وَالْحَمِيرِ وَكُلِّ الْبَهَائِمِ الَّتِي تَكُونُ فِي هَذِهِ الْمَحَالِّ. كَهَذِهِ الضَّرْبَةِ.
\par 16 وَيَكُونُ أَنَّ كُلَّ الْبَاقِي مِنْ جَمِيعِ الأُمَمِ الَّذِينَ جَاءُوا عَلَى أُورُشَلِيمَ يَصْعَدُونَ مِنْ سَنَةٍ إِلَى سَنَةٍ لِيَسْجُدُوا لِلْمَلِكِ رَبِّ الْجُنُودِ وَلِيُعَيِّدُوا عِيدَ الْمَظَالِّ.
\par 17 وَيَكُونُ أَنَّ كُلَّ مَنْ لاَ يَصْعَدُ مِنْ قَبَائِلِ الأَرْضِ إِلَى أُورُشَلِيمَ لِيَسْجُدَ لِلْمَلِكِ رَبِّ الْجُنُودِ لاَ يَكُونُ عَلَيْهِمْ مَطَرٌ.
\par 18 وَإِنْ لاَ تَصْعَدْ وَلاَ تَأْتِ قَبِيلَةُ مِصْرَ وَلاَ مَطَرٌ عَلَيْهَا تَكُنْ عَلَيْهَا الضَّرْبَةُ الَّتِي يَضْرِبُ بِهَا الرَّبُّ الأُمَمَ الَّذِينَ لاَ يَصْعَدُونَ لِيُعَيِّدُوا عِيدَ الْمَظَالِّ.
\par 19 هَذَا يَكُونُ قِصَاصُ مِصْرَ وَقِصَاصُ كُلِّ الأُمَمِ الَّذِينَ لاَ يَصْعَدُونَ لِيُعَيِّدُوا عِيدَ الْمَظَالِّ.
\par 20 فِي ذَلِكَ الْيَوْمِ يَكُونُ عَلَى أَجْرَاسِ الْخَيْلِ [قُدْسٌ لِلرَّبِّ]. وَالْقُدُورُ فِي بَيْتِ الرَّبِّ تَكُونُ كَالْمَنَاضِحِ أَمَامَ الْمَذْبَحِ.
\par 21 وَكُلُّ قِدْرٍ فِي أُورُشَلِيمَ وَفِي يَهُوذَا تَكُونُ قُدْساً لِرَبِّ الْجُنُودِ وَكُلُّ الذَّابِحِينَ يَأْتُونَ وَيَأْخُذُونَ مِنْهَا وَيَطْبُخُونَ فِيهَا. وَفِي ذَلِكَ الْيَوْمِ لاَ يَكُونُ بَعْدُ كَنْعَانِيٌّ فِي بَيْتِ رَبِّ الْجُنُودِ.

\end{document}