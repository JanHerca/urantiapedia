\begin{document}

\title{فيلبي}


\chapter{1}

\par 1 بُولُسُ وَتِيمُوثَاوُسُ عَبْدَا يَسُوعَ الْمَسِيحِ، إِلَى جَمِيعِ الْقِدِّيسِينَ فِي الْمَسِيحِ يَسُوعَ، الَّذِينَ فِي فِيلِبِّي، مَعَ أَسَاقِفَةٍ وَشَمَامِسَةٍ.
\par 2 نِعْمَةٌ لَكُمْ وَسَلاَمٌ مِنَ اللهِ أَبِينَا وَالرَّبِّ يَسُوعَ الْمَسِيحِ.
\par 3 أَشْكُرُ إِلَهِي عِنْدَ كُلِّ ذِكْرِي إِيَّاكُمْ
\par 4 دَائِماً فِي كُلِّ أَدْعِيَتِي، مُقَدِّماً الطِّلْبَةَ لأَجْلِ جَمِيعِكُمْ بِفَرَحٍ،
\par 5 لِسَبَبِ مُشَارَكَتِكُمْ فِي الإِنْجِيلِ مِنْ أَوَّلِ يَوْمٍ إِلَى الآنَ.
\par 6 وَاثِقاً بِهَذَا عَيْنِهِ أَنَّ الَّذِي ابْتَدَأَ فِيكُمْ عَمَلاً صَالِحاً يُكَمِّلُ إِلَى يَوْمِ يَسُوعَ الْمَسِيحِ.
\par 7 كَمَا يَحِقُّ لِي أَنْ أَفْتَكِرَ هَذَا مِنْ جِهَةِ جَمِيعِكُمْ، لأَنِّي حَافِظُكُمْ فِي قَلْبِي، فِي وُثُقِي، وَفِي الْمُحَامَاةِ عَنِ الإِنْجِيلِ وَتَثْبِيتِهِ، أَنْتُمُ الَّذِينَ جَمِيعُكُمْ شُرَكَائِي فِي النِّعْمَةِ.
\par 8 فَإِنَّ اللهَ شَاهِدٌ لِي كَيْفَ أَشْتَاقُ إِلَى جَمِيعِكُمْ فِي أَحْشَاءِ يَسُوعَ الْمَسِيحِ.
\par 9 وَهَذَا أُصَلِّيهِ: أَنْ تَزْدَادَ مَحَبَّتُكُمْ أَيْضاً أَكْثَرَ فَأَكْثَرَ فِي الْمَعْرِفَةِ وَفِي كُلِّ فَهْمٍ،
\par 10 حَتَّى تُمَيِّزُوا الْأُمُورَ الْمُتَخَالِفَةَ، لِكَيْ تَكُونُوا مُخْلِصِينَ وَبِلاَ عَثْرَةٍ إِلَى يَوْمِ الْمَسِيحِ،
\par 11 مَمْلُوئِينَ مِنْ ثَمَرِ الْبِرِّ الَّذِي بِيَسُوعَ الْمَسِيحِ لِمَجْدِ اللهِ وَحَمْدِهِ.
\par 12 ثُمَّ أُرِيدُ أَنْ تَعْلَمُوا أَيُّهَا الإِخْوَةُ أَنَّ أُمُورِي قَدْ آلَتْ أَكْثَرَ إِلَى تَقَدُّمِ الإِنْجِيلِ،
\par 13 حَتَّى إِنَّ وُثُقِي صَارَتْ ظَاهِرَةً فِي الْمَسِيحِ فِي كُلِّ دَارِ الْوِلاَيَةِ وَفِي بَاقِي الأَمَاكِنِ أَجْمَعَ.
\par 14 وَأَكْثَرُ الإِخْوَةِ، وَهُمْ وَاثِقُونَ فِي الرَّبِّ بِوُثُقِي، يَجْتَرِئُونَ أَكْثَرَ عَلَى التَّكَلُّمِ بِالْكَلِمَةِ بِلاَ خَوْفٍ.
\par 15 أَمَّا قَوْمٌ فَعَنْ حَسَدٍ وَخِصَامٍ يَكْرِزُونَ بِالْمَسِيحِ، وَأَمَّا قَوْمٌ فَعَنْ مَسَرَّةٍ.
\par 16 فَهَؤُلاَءِ عَنْ تَحَزُّبٍ يُنَادُونَ بِالْمَسِيحِ لاَ عَنْ إِخْلاَصٍ، ظَانِّينَ أَنَّهُمْ يُضِيفُونَ إِلَى وُثُقِي ضِيقاً.
\par 17 وَأُولَئِكَ عَنْ مَحَبَّةٍ، عَالِمِينَ أَنِّي مَوْضُوعٌ لِحِمَايَةِ الإِنْجِيلِ.
\par 18 فَمَاذَا؟ غَيْرَ أَنَّهُ عَلَى كُلِّ وَجْهٍ سَوَاءٌ كَانَ بِعِلَّةٍ أَمْ بِحَقٍّ يُنَادَى بِالْمَسِيحِ، وَبِهَذَا أَنَا أَفْرَحُ. بَلْ سَأَفْرَحُ أَيْضاً.
\par 19 لأَنِّي أَعْلَمُ أَنَّ هَذَا يَؤُولُ لِي إِلَى خَلاَصٍ بِطِلْبَتِكُمْ وَمُؤَازَرَةِ رُوحِ يَسُوعَ الْمَسِيحِ،
\par 20 حَسَبَ انْتِظَارِي وَرَجَائِي أَنِّي لاَ أُخْزَى فِي شَيْءٍ، بَلْ بِكُلِّ مُجَاهَرَةٍ كَمَا فِي كُلِّ حِينٍ، كَذَلِكَ الآنَ، يَتَعَظَّمُ الْمَسِيحُ فِي جَسَدِي، سَوَاءٌ كَانَ بِحَيَاةٍ أَمْ بِمَوْتٍ.
\par 21 لأَنَّ لِيَ الْحَيَاةَ هِيَ الْمَسِيحُ وَالْمَوْتُ هُوَ رِبْحٌ.
\par 22 وَلَكِنْ إِنْ كَانَتِ الْحَيَاةُ فِي الْجَسَدِ هِيَ لِي ثَمَرُ عَمَلِي، فَمَاذَا أَخْتَارُ؟ لَسْتُ أَدْرِي!
\par 23 فَإِنِّي مَحْصُورٌ مِنْ الاِثْنَيْنِ: لِيَ اشْتِهَاءٌ أَنْ أَنْطَلِقَ وَأَكُونَ مَعَ الْمَسِيحِ. ذَاكَ أَفْضَلُ جِدّاً.
\par 24 وَلَكِنْ أَنْ أَبْقَى فِي الْجَسَدِ أَلْزَمُ مِنْ أَجْلِكُمْ.
\par 25 فَإِذْ أَنَا وَاثِقٌ بِهَذَا أَعْلَمُ أَنِّي أَمْكُثُ وَأَبْقَى مَعَ جَمِيعِكُمْ لأَجْلِ تَقَدُّمِكُمْ وَفَرَحِكُمْ فِي الإِيمَانِ،
\par 26 لِكَيْ يَزْدَادَ افْتِخَارُكُمْ فِي الْمَسِيحِ يَسُوعَ فِيَّ، بِوَاسِطَةِ حُضُورِي أَيْضاً عِنْدَكُمْ.
\par 27 فَقَطْ عِيشُوا كَمَا يَحِقُّ لإِنْجِيلِ الْمَسِيحِ، حَتَّى إِذَا جِئْتُ وَرَأَيْتُكُمْ، أَوْ كُنْتُ غَائِباً أَسْمَعُ أُمُورَكُمْ أَنَّكُمْ تَثْبُتُونَ فِي رُوحٍ وَاحِدٍ، مُجَاهِدِينَ مَعاً بِنَفْسٍ وَاحِدَةٍ لإِيمَانِ الإِنْجِيلِ،
\par 28 غَيْرَ مُخَوَّفِينَ بِشَيْءٍ مِنَ الْمُقَاوِمِينَ، الأَمْرُ الَّذِي هُوَ لَهُمْ بَيِّنَةٌ لِلْهَلاَكِ، وَأَمَّا لَكُمْ فَلِلْخَلاَصِ، وَذَلِكَ مِنَ اللهِ.
\par 29 لأَنَّهُ قَدْ وُهِبَ لَكُمْ لأَجْلِ الْمَسِيحِ لاَ أَنْ تُؤْمِنُوا بِهِ فَقَطْ، بَلْ أَيْضاً أَنْ تَتَأَلَّمُوا لأَجْلِهِ.
\par 30 إِذْ لَكُمُ الْجِهَادُ عَيْنُهُ الَّذِي رَأَيْتُمُوهُ فِيَّ، وَالآنَ تَسْمَعُونَ فِيَّ.

\chapter{2}

\par 1 فَإِنْ كَانَ وَعْظٌ مَا فِي الْمَسِيحِ. إِنْ كَانَتْ تَسْلِيَةٌ مَا لِلْمَحَبَّةِ. إِنْ كَانَتْ شَرِكَةٌ مَا فِي الرُّوحِ. إِنْ كَانَتْ أَحْشَاءٌ وَرَأْفَةٌ،
\par 2 فَتَمِّمُوا فَرَحِي حَتَّى تَفْتَكِرُوا فِكْراً وَاحِداً وَلَكُمْ مَحَبَّةٌ وَاحِدَةٌ بِنَفْسٍ وَاحِدَةٍ، مُفْتَكِرِينَ شَيْئاً وَاحِداً،
\par 3 لاَ شَيْئاً بِتَحَزُّبٍ أَوْ بِعُجْبٍ، بَلْ بِتَوَاضُعٍ، حَاسِبِينَ بَعْضُكُمُ الْبَعْضَ أَفْضَلَ مِنْ أَنْفُسِهِمْ.
\par 4 لاَ تَنْظُرُوا كُلُّ وَاحِدٍ إِلَى مَا هُوَ لِنَفْسِهِ، بَلْ كُلُّ وَاحِدٍ إِلَى مَا هُوَ لآخَرِينَ أَيْضاً.
\par 5 فَلْيَكُنْ فِيكُمْ هَذَا الْفِكْرُ الَّذِي فِي الْمَسِيحِ يَسُوعَ أَيْضاً:
\par 6 الَّذِي إِذْ كَانَ فِي صُورَةِ اللهِ، لَمْ يَحْسِبْ خُلْسَةً أَنْ يَكُونَ مُعَادِلاً لِلَّهِ.
\par 7 لَكِنَّهُ أَخْلَى نَفْسَهُ، آخِذاً صُورَةَ عَبْدٍ، صَائِراً فِي شِبْهِ النَّاسِ.
\par 8 وَإِذْ وُجِدَ فِي الْهَيْئَةِ كَإِنْسَانٍ، وَضَعَ نَفْسَهُ وَأَطَاعَ حَتَّى الْمَوْتَ مَوْتَ الصَّلِيبِ.
\par 9 لِذَلِكَ رَفَّعَهُ اللهُ أَيْضاً، وَأَعْطَاهُ اسْماً فَوْقَ كُلِّ اسْمٍ
\par 10 لِكَيْ تَجْثُوَ بِاسْمِ يَسُوعَ كُلُّ رُكْبَةٍ مِمَّنْ فِي السَّمَاءِ وَمَنْ عَلَى الأَرْضِ وَمَنْ تَحْتَ الأَرْضِ،
\par 11 وَيَعْتَرِفَ كُلُّ لِسَانٍ أَنَّ يَسُوعَ الْمَسِيحَ هُوَ رَبٌّ لِمَجْدِ اللهِ الآبِ.
\par 12 إِذاً يَا أَحِبَّائِي، كَمَا أَطَعْتُمْ كُلَّ حِينٍ، لَيْسَ كَمَا فِي حُضُورِي فَقَطْ، بَلِ الآنَ بِالأَوْلَى جِدّاً فِي غِيَابِي، تَمِّمُوا خَلاَصَكُمْ بِخَوْفٍ وَرِعْدَةٍ،
\par 13 لأَنَّ اللهَ هُوَ الْعَامِلُ فِيكُمْ أَنْ تُرِيدُوا وَأَنْ تَعْمَلُوا مِنْ أَجْلِ الْمَسَرَّةِ.
\par 14 اِفْعَلُوا كُلَّ شَيْءٍ بِلاَ دَمْدَمَةٍ وَلاَ مُجَادَلَةٍ،
\par 15 لِكَيْ تَكُونُوا بِلاَ لَوْمٍ، وَبُسَطَاءَ، أَوْلاَداً للهِ بِلاَ عَيْبٍ فِي وَسَطِ جِيلٍ مُعَوَّجٍ وَمُلْتَوٍ، تُضِيئُونَ بَيْنَهُمْ كَأَنْوَارٍ فِي الْعَالَمِ.
\par 16 مُتَمَسِّكِينَ بِكَلِمَةِ الْحَيَاةِ لاِفْتِخَارِي فِي يَوْمِ الْمَسِيحِ بِأَنِّي لَمْ أَسْعَ بَاطِلاً وَلاَ تَعِبْتُ بَاطِلاً.
\par 17 لَكِنَّنِي وَإِنْ كُنْتُ أَنْسَكِبُ أَيْضاً عَلَى ذَبِيحَةِ إِيمَانِكُمْ وَخِدْمَتِهِ، أُسَرُّ وَأَفْرَحُ مَعَكُمْ أَجْمَعِينَ.
\par 18 وَبِهَذَا عَيْنِهِ كُونُوا أَنْتُمْ مَسْرُورِينَ أَيْضاً وَافْرَحُوا مَعِي.
\par 19 عَلَى أَنِّي أَرْجُو فِي الرَّبِّ يَسُوعَ أَنْ أُرْسِلَ إِلَيْكُمْ سَرِيعاً تِيمُوثَاوُسَ لِكَيْ تَطِيبَ نَفْسِي إِذَا عَرَفْتُ أَحْوَالَكُمْ.
\par 20 لأَنْ لَيْسَ لِي أَحَدٌ آخَرُ نَظِيرُ نَفْسِي يَهْتَمُّ بِأَحْوَالِكُمْ بِإِخْلاَصٍ،
\par 21 إِذِ الْجَمِيعُ يَطْلُبُونَ مَا هُوَ لأَنْفُسِهِمْ لاَ مَا هُوَ لِيَسُوعَ الْمَسِيحِ.
\par 22 وَأَمَّا اخْتِبَارُهُ فَأَنْتُمْ تَعْرِفُونَ أَنَّهُ كَوَلَدٍ مَعَ أَبٍ خَدَمَ مَعِي لأَجْلِ الإِنْجِيلِ.
\par 23 هَذَا أَرْجُو أَنْ أُرْسِلَهُ أَوَّلَ مَا أَرَى أَحْوَالِي حَالاً.
\par 24 وَأَثِقُ بِالرَّبِّ أَنِّي أَنَا أَيْضاً سَآتِي إِلَيْكُمْ سَرِيعاً.
\par 25 وَلَكِنِّي حَسِبْتُ مِنَ اللاَّزِمِ أَنْ أُرْسِلَ إِلَيْكُمْ أَبَفْرُودِتُسَ أَخِي، وَالْعَامِلَ مَعِي، وَالْمُتَجَنِّدَ مَعِي، وَرَسُولَكُمْ، وَالْخَادِمَ لِحَاجَتِي.
\par 26 إِذْ كَانَ مُشْتَاقاً إِلَى جَمِيعِكُمْ وَمَغْمُوماً، لأَنَّكُمْ سَمِعْتُمْ أَنَّهُ كَانَ مَرِيضاً.
\par 27 فَإِنَّهُ مَرِضَ قَرِيباً مِنَ الْمَوْتِ، لَكِنَّ اللهَ رَحِمَهُ. وَلَيْسَ إِيَّاهُ وَحْدَهُ بَلْ إِيَّايَ أَيْضاً لِئَلاَّ يَكُونَ لِي حُزْنٌ عَلَى حُزْنٍ.
\par 28 فَأَرْسَلْتُهُ إِلَيْكُمْ بِأَوْفَرِ سُرْعَةٍ، حَتَّى إِذَا رَأَيْتُمُوهُ تَفْرَحُونَ أَيْضاً وَأَكُونُ أَنَا أَقَلَّ حُزْناً.
\par 29 فَاقْبَلُوهُ فِي الرَّبِّ بِكُلِّ فَرَحٍ، وَلْيَكُنْ مِثْلُهُ مُكَرَّماً عِنْدَكُمْ.
\par 30 لأَنَّهُ مِنْ أَجْلِ عَمَلِ الْمَسِيحِ قَارَبَ الْمَوْتَ، مُخَاطِراً بِنَفْسِهِ، لِكَيْ يَجْبُرَ نُقْصَانَ خِدْمَتِكُمْ لِي.

\chapter{3}

\par 1 أَخِيراً يَا إِخْوَتِي افْرَحُوا فِي الرَّبِّ. كِتَابَةُ هَذِهِ الْأُمُورِ إِلَيْكُمْ لَيْسَتْ عَلَيَّ ثَقِيلَةً، وَأَمَّا لَكُمْ فَهِيَ مُؤَمِّنَةٌ.
\par 2 اُنْظُرُوا الْكِلاَبَ. انْظُرُوا فَعَلَةَ الشَّرِّ. انْظُرُوا الْقَطْعَ.
\par 3 لأَنَّنَا نَحْنُ الْخِتَانَ، الَّذِينَ نَعْبُدُ اللهَ بِالرُّوحِ، وَنَفْتَخِرُ فِي الْمَسِيحِ يَسُوعَ، وَلاَ نَتَّكِلُ عَلَى الْجَسَدِ -
\par 4 مَعَ أَنَّ لِي أَنْ أَتَّكِلَ عَلَى الْجَسَدِ أَيْضاً. إِنْ ظَنَّ وَاحِدٌ آخَرُ أَنْ يَتَّكِلَ عَلَى الْجَسَدِ فَأَنَا بِالأَوْلَى.
\par 5 مِنْ جِهَةِ الْخِتَانِ مَخْتُونٌ فِي الْيَوْمِ الثَّامِنِ، مِنْ جِنْسِ إِسْرَائِيلَ، مِنْ سِبْطِ بِنْيَامِينَ، عِبْرَانِيٌّ مِنَ الْعِبْرَانِيِّينَ. مِنْ جِهَةِ النَّامُوسِ فَرِّيسِيٌّ.
\par 6 مِنْ جِهَةِ الْغَيْرَةِ مُضْطَهِدُ الْكَنِيسَةِ. مِنْ جِهَةِ الْبِرِّ الَّذِي فِي النَّامُوسِ بِلاَ لَوْمٍ.
\par 7 لَكِنْ مَا كَانَ لِي رِبْحاً فَهَذَا قَدْ حَسِبْتُهُ مِنْ أَجْلِ الْمَسِيحِ خَسَارَةً.
\par 8 بَلْ إِنِّي أَحْسِبُ كُلَّ شَيْءٍ أَيْضاً خَسَارَةً مِنْ أَجْلِ فَضْلِ مَعْرِفَةِ الْمَسِيحِ يَسُوعَ رَبِّي، الَّذِي مِنْ أَجْلِهِ خَسِرْتُ كُلَّ الأَشْيَاءِ، وَأَنَا أَحْسِبُهَا نُفَايَةً لِكَيْ أَرْبَحَ الْمَسِيحَ
\par 9 وَأُوجَدَ فِيهِ، وَلَيْسَ لِي بِرِّي الَّذِي مِنَ النَّامُوسِ، بَلِ الَّذِي بِإِيمَانِ الْمَسِيحِ، الْبِرُّ الَّذِي مِنَ اللهِ بِالإِيمَانِ.
\par 10 لأَعْرِفَهُ، وَقُوَّةَ قِيَامَتِهِ، وَشَرِكَةَ آلاَمِهِ، مُتَشَبِّهاً بِمَوْتِهِ،
\par 11 لَعَلِّي أَبْلُغُ إِلَى قِيَامَةِ الأَمْوَاتِ.
\par 12 لَيْسَ أَنِّي قَدْ نِلْتُ أَوْ صِرْتُ كَامِلاً، وَلَكِنِّي أَسْعَى لَعَلِّي أُدْرِكُ الَّذِي لأَجْلِهِ أَدْرَكَنِي أَيْضاً الْمَسِيحُ يَسُوعُ.
\par 13 أَيُّهَا الإِخْوَةُ، أَنَا لَسْتُ أَحْسِبُ نَفْسِي أَنِّي قَدْ أَدْرَكْتُ. وَلَكِنِّي أَفْعَلُ شَيْئاً وَاحِداً: إِذْ أَنَا أَنْسَى مَا هُوَ وَرَاءُ وَأَمْتَدُّ إِلَى مَا هُوَ قُدَّامُ.
\par 14 أَسْعَى نَحْوَ الْغَرَضِ لأَجْلِ جَعَالَةِ دَعْوَةِ اللهِ الْعُلْيَا فِي الْمَسِيحِ يَسُوعَ.
\par 15 فَلْيَفْتَكِرْ هَذَا جَمِيعُ الْكَامِلِينَ مِنَّا، وَإِنِ افْتَكَرْتُمْ شَيْئاً بِخِلاَفِهِ فَاللهُ سَيُعْلِنُ لَكُمْ هَذَا أَيْضاً.
\par 16 وَأَمَّا مَا قَدْ أَدْرَكْنَاهُ، فَلْنَسْلُكْ بِحَسَبِ ذَلِكَ الْقَانُونِ عَيْنِهِ، وَنَفْتَكِرْ ذَلِكَ عَيْنَهُ.
\par 17 كُونُوا مُتَمَثِّلِينَ بِي مَعاً أَيُّهَا الإِخْوَةُ، وَلاَحِظُوا الَّذِينَ يَسِيرُونَ هَكَذَا كَمَا نَحْنُ عِنْدَكُمْ قُدْوَةٌ.
\par 18 لأَنَّ كَثِيرِينَ يَسِيرُونَ مِمَّنْ كُنْتُ أَذْكُرُهُمْ لَكُمْ مِرَاراً، وَالآنَ أَذْكُرُهُمْ أَيْضاً بَاكِياً، وَهُمْ أَعْدَاءُ صَلِيبِ الْمَسِيحِ،
\par 19 الَّذِينَ نِهَايَتُهُمُ الْهَلاَكُ، الَّذِينَ إِلَهُهُمْ بَطْنُهُمْ وَمَجْدُهُمْ فِي خِزْيِهِمِ، الَّذِينَ يَفْتَكِرُونَ فِي الأَرْضِيَّاتِ.
\par 20 فَإِنَّ سِيرَتَنَا نَحْنُ هِيَ فِي السَّمَاوَاتِ، الَّتِي مِنْهَا أَيْضاً نَنْتَظِرُ مُخَلِّصاً هُوَ الرَّبُّ يَسُوعُ الْمَسِيحُ،
\par 21 الَّذِي سَيُغَيِّرُ شَكْلَ جَسَدِ تَوَاضُعِنَا لِيَكُونَ عَلَى صُورَةِ جَسَدِ مَجْدِهِ، بِحَسَبِ عَمَلِ اسْتِطَاعَتِهِ أَنْ يُخْضِعَ لِنَفْسِهِ كُلَّ شَيْءٍ.

\chapter{4}

\par 1 إِذاً يَا إِخْوَتِي الأَحِبَّاءَ وَالْمُشْتَاقَ إِلَيْهِمْ، يَا سُرُورِي وَإِكْلِيلِي، اثْبُتُوا هَكَذَا فِي الرَّبِّ أَيُّهَا الأَحِبَّاءُ.
\par 2 أَطْلُبُ إِلَى أَفُودِيَةَ وَأَطْلُبُ إِلَى سِنْتِيخِي أَنْ تَفْتَكِرَا فِكْراً وَاحِداً فِي الرَّبِّ.
\par 3 نَعَمْ أَسْأَلُكَ أَنْتَ أَيْضاً، يَا شَرِيكِي الْمُخْلِصَ، سَاعِدْ هَاتَيْنِ اللَّتَيْنِ جَاهَدَتَا مَعِي فِي الإِنْجِيلِ، مَعَ أَكْلِيمَنْدُسَ أَيْضاً وَبَاقِي الْعَامِلِينَ مَعِي، الَّذِينَ أَسْمَاؤُهُمْ فِي سِفْرِ الْحَيَاةِ.
\par 4 اِفْرَحُوا فِي الرَّبِّ كُلَّ حِينٍ وَأَقُولُ أَيْضاً افْرَحُوا.
\par 5 لِيَكُنْ حِلْمُكُمْ مَعْرُوفاً عِنْدَ جَمِيعِ النَّاسِ. الرَّبُّ قَرِيبٌ.
\par 6 لاَ تَهْتَمُّوا بِشَيْءٍ، بَلْ فِي كُلِّ شَيْءٍ بِالصَّلاَةِ وَالدُّعَاءِ مَعَ الشُّكْرِ، لِتُعْلَمْ طِلْبَاتُكُمْ لَدَى اللهِ.
\par 7 وَسَلاَمُ اللهِ الَّذِي يَفُوقُ كُلَّ عَقْلٍ يَحْفَظُ قُلُوبَكُمْ وَأَفْكَارَكُمْ فِي الْمَسِيحِ يَسُوعَ.
\par 8 أَخِيراً أَيُّهَا الإِخْوَةُ كُلُّ مَا هُوَ حَقٌّ، كُلُّ مَا هُوَ جَلِيلٌ، كُلُّ مَا هُوَ عَادِلٌ، كُلُّ مَا هُوَ طَاهِرٌ، كُلُّ مَا هُوَ مُسِرٌّ، كُلُّ مَا صِيتُهُ حَسَنٌ - إِنْ كَانَتْ فَضِيلَةٌ وَإِنْ كَانَ مَدْحٌ، فَفِي هَذِهِ افْتَكِرُوا.
\par 9 وَمَا تَعَلَّمْتُمُوهُ، وَتَسَلَّمْتُمُوهُ، وَسَمِعْتُمُوهُ، وَرَأَيْتُمُوهُ فِيَّ، فَهَذَا افْعَلُوا، وَإِلَهُ السَّلاَمِ يَكُونُ مَعَكُمْ.
\par 10 ثُمَّ إِنِّي فَرِحْتُ بِالرَّبِّ جِدّاً لأَنَّكُمُ الآنَ قَدْ أَزْهَرَ أَيْضاً مَرَّةً اعْتِنَاؤُكُمْ بِي الَّذِي كُنْتُمْ تَعْتَنُونَهُ وَلَكِنْ لَمْ تَكُنْ لَكُمْ فُرْصَةٌ.
\par 11 لَيْسَ أَنِّي أَقُولُ مِنْ جِهَةِ احْتِيَاجٍ، فَإِنِّي قَدْ تَعَلَّمْتُ أَنْ أَكُونَ مُكْتَفِياً بِمَا أَنَا فِيهِ.
\par 12 أَعْرِفُ أَنْ أَتَّضِعَ وَأَعْرِفُ أَيْضاً أَنْ أَسْتَفْضِلَ. فِي كُلِّ شَيْءٍ وَفِي جَمِيعِ الأَشْيَاءِ قَدْ تَدَرَّبْتُ أَنْ أَشْبَعَ وَأَنْ أَجُوعَ، وَأَنْ أَسْتَفْضِلَ وَأَنْ أَنْقُصَ.
\par 13 أَسْتَطِيعُ كُلَّ شَيْءٍ فِي الْمَسِيحِ الَّذِي يُقَوِّينِي.
\par 14 غَيْرَ أَنَّكُمْ فَعَلْتُمْ حَسَناً إِذِ اشْتَرَكْتُمْ فِي ضِيقَتِي.
\par 15 وَأَنْتُمْ أَيْضاً تَعْلَمُونَ أَيُّهَا الْفِيلِبِّيُّونَ أَنَّهُ فِي بَدَاءَةِ الإِنْجِيلِ، لَمَّا خَرَجْتُ مِنْ مَكِدُونِيَّةَ، لَمْ تُشَارِكْنِي كَنِيسَةٌ وَاحِدَةٌ فِي حِسَابِ الْعَطَاءِ وَالأَخْذِ إِلاَّ أَنْتُمْ وَحْدَكُمْ.
\par 16 فَإِنَّكُمْ فِي تَسَالُونِيكِي أَيْضاً أَرْسَلْتُمْ إِلَيَّ مَرَّةً وَمَرَّتَيْنِ لِحَاجَتِي.
\par 17 لَيْسَ أَنِّي أَطْلُبُ الْعَطِيَّةَ، بَلْ أَطْلُبُ الثَّمَرَ الْمُتَكَاثِرَ لِحِسَابِكُمْ.
\par 18 وَلَكِنِّي قَدِ اسْتَوْفَيْتُ كُلَّ شَيْءٍ وَاسْتَفْضَلْتُ. قَدِ امْتَلَأْتُ إِذْ قَبِلْتُ مِنْ أَبَفْرُودِتُسَ الأَشْيَاءَ الَّتِي مِنْ عِنْدِكُمْ، نَسِيمَ رَائِحَةٍ طَيِّبَةٍ، ذَبِيحَةً مَقْبُولَةً مَرْضِيَّةً عِنْدَ اللهِ.
\par 19 فَيَمْلأُ إِلَهِي كُلَّ احْتِيَاجِكُمْ بِحَسَبِ غِنَاهُ فِي الْمَجْدِ فِي الْمَسِيحِ يَسُوعَ.
\par 20 وَلِلَّهِ وَأَبِينَا الْمَجْدُ إِلَى دَهْرِ الدَّاهِرِينَ. آمِينَ.
\par 21 سَلِّمُوا عَلَى كُلِّ قِدِّيسٍ فِي الْمَسِيحِ يَسُوعَ. يُسَلِّمُ عَلَيْكُمُ الإِخْوَةُ الَّذِينَ مَعِي.
\par 22 يُسَلِّمُ عَلَيْكُمْ جَمِيعُ الْقِدِّيسِينَ وَلاَ سِيَّمَا الَّذِينَ مِنْ بَيْتِ قَيْصَرَ.
\par 23 نِعْمَةُ رَبِّنَا يَسُوعَ الْمَسِيحِ مَعَ جَمِيعِكُمْ. آمِينَ.

\end{document}