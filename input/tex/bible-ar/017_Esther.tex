\begin{document}

\title{استير}


\chapter{1}

\par 1 وَحَدَثَ فِي أَيَّامِ أَحْشَوِيرُوشَ. (هُوَ أَحْشَوِيرُوشُ الَّذِي مَلَكَ مِنَ الْهِنْدِ إِلَى كُوشٍ عَلَى مِئَةٍ وَسَبْعٍ وَعِشْرِينَ كُورَةً)
\par 2 أَنَّهُ فِي تِلْكَ الأَيَّامِ حِينَ جَلَسَ الْمَلِكُ أَحْشَوِيرُوشُ عَلَى كُرْسِيِّ مُلْكِهِ الَّذِي فِي شُوشَنَ الْقَصْرِ
\par 3 فِي السَّنَةِ الثَّالِثَةِ مِنْ مُلْكِهِ عَمِلَ وَلِيمَةً لِجَمِيعِ رُؤَسَائِهِ وَعَبِيدِهِ جَيْشِ فَارِسَ وَمَادِي وَأَمَامَهُ شُرَفَاءُ الْبُلْدَانِ وَرُؤَسَاؤُهَا
\par 4 حِينَ أَظْهَرَ غِنَى مَجْدِ مُلْكِهِ وَوَقَارَ جَلاَلِ عَظَمَتِهِ أَيَّاماً كَثِيرَةً مِئَةً وَثَمَانِينَ يَوْماً.
\par 5 وَعِنْدَ انْقِضَاءِ هَذِهِ الأَيَّامِ عَمِلَ الْمَلِكُ لِجَمِيعِ الشَّعْبِ الْمَوْجُودِينَ فِي شُوشَنَ الْقَصْرِ مِنَ الْكَبِيرِ إِلَى الصَّغِيرِ وَلِيمَةً سَبْعَةَ أَيَّامٍ فِي دَارِ جَنَّةِ قَصْرِ الْمَلِكِ
\par 6 بِأَنْسِجَةٍ بَيْضَاءَ وَخَضْرَاءَ وَأَسْمَانْجُونِيَّةٍ مُعَلَّقَةٍ بِحِبَالٍ مِنْ بَزٍّ وَأُرْجُوانٍ فِي حَلَقَاتٍ مِنْ فِضَّةٍ وَأَعْمِدَةٍ مِنْ رُخَامٍ وَأَسِرَّةٍ مِنْ ذَهَبٍ وَفِضَّةٍ عَلَى مُجَزَّعٍ مِنْ بَهْتٍ وَمَرْمَرٍ وَدُرٍّ وَرُخَامٍ أَسْوَدَ.
\par 7 وَكَانَ السِّقَاءُ مِنْ ذَهَبٍ وَالآنِيَةُ مُخْتَلِفَةُ الأَشْكَالِ وَالْخَمْرُ الْمَلِكِيُّ بِكَثْرَةٍ حَسَبَ كَرَمِ الْمَلِكِ.
\par 8 وَكَانَ الشُّرْبُ حَسَبَ الأَمْرِ. لَمْ يَكُنْ غَاصِبٌ لأَنَّهُ هَكَذَا رَسَمَ الْمَلِكُ عَلَى كُلِّ عَظِيمٍ فِي بَيْتِهِ أَنْ يَعْمَلُوا حَسَبَ رِضَا كُلِّ وَاحِدٍ.
\par 9 وَوَشْتِي الْمَلِكَةُ عَمِلَتْ أَيْضاً وَلِيمَةً لِلنِّسَاءِ فِي بَيْتِ الْمُلْكِ الَّذِي لِلْمَلِكِ أَحْشَوِيرُوشَ.
\par 10 فِي الْيَوْمِ السَّابِعِ لَمَّا طَابَ قَلْبُ الْمَلِكِ بِالْخَمْرِ قَالَ لِمَهُومَانَ وَبِزْثَا وَحَرْبُونَا وَبِغْثَا وَأَبَغْثَا وَزِيثَارَ وَكَرْكَسَ الْخِصْيَانِ السَّبْعَةِ الَّذِينَ كَانُوا يَخْدِمُونَ بَيْنَ يَدَيِ الْمَلِكِ أَحْشَوِيرُوشَ
\par 11 أَنْ يَأْتُوا بِوَشْتِي الْمَلِكَةِ إِلَى أَمَامِ الْمَلِكِ بِتَاجِ الْمُلْكِ لِيُرِيَ الشُّعُوبَ وَالرُّؤَسَاءَ جَمَالَهَا لأَنَّهَا كَانَتْ حَسَنَةَ الْمَنْظَرِ.
\par 12 فَأَبَتِ الْمَلِكَةُ وَشْتِي أَنْ تَأْتِيَ حَسَبَ أَمْرِ الْمَلِكِ عَنْ يَدِ الْخِصْيَانِ. فَاغْتَاظَ الْمَلِكُ جِدّاً وَاشْتَعَلَ غَضَبُهُ فِيهِ.
\par 13 وَقَالَ الْمَلِكُ لِلْحُكَمَاءِ الْعَارِفِينَ بِالأَزْمِنَةِ (لأَنَّهُ هَكَذَا كَانَ أَمْرُ الْمَلِكِ نَحْوَ جَمِيعِ الْعَارِفِينَ بِالسُّنَّةِ وَالْقَضَاءِ.
\par 14 وَكَانَ الْمُقَرِّبُونَ إِلَيْهِ كَرْشَنَا وَشِيثَارَ وَأَدْمَاثَا وَتَرْشِيشَ وَمَرَسَ وَمَرْسَنَا وَمَمُوكَانَ سَبْعَةَ رُؤَسَاءِ فَارِسَ وَمَادِي الَّذِينَ يَرُونَ وَجْهَ الْمَلِكِ وَيَجْلِسُونَ أَوَّلاً فِي الْمُلْكِ):
\par 15 [حَسَبَ السُّنَّةِ مَاذَا يُعْمَلُ بِالْمَلِكَةِ وَشْتِي لأَنَّهَا لَمْ تَعْمَلْ كَقَوْلِ الْمَلِكِ أَحْشَوِيرُوشَ عَنْ يَدِ الْخِصْيَانِ؟]
\par 16 فَقَالَ مَمُوكَانُ أَمَامَ الْمَلِكِ وَالرُّؤَسَاءِ: [لَيْسَ إِلَى الْمَلِكِ وَحْدَهُ أَذْنَبَتْ وَشْتِي الْمَلِكَةُ بَلْ إِلَى جَمِيعِ الرُّؤَسَاءِ وَجَمِيعِ الشُّعُوبِ الَّذِينَ فِي كُلِّ بُلْدَانِ الْمَلِكِ أَحْشَوِيرُوشَ.
\par 17 لأَنَّهُ سَوْفَ يَبْلُغُ خَبَرُ الْمَلِكَةِ إِلَى جَمِيعِ النِّسَاءِ حَتَّى يُحْتَقَرَ أَزْوَاجُهُنَّ فِي أَعْيُنِهِنَّ عِنْدَمَا يُقَالُ إِنَّ الْمَلِكَ أَحْشَوِيرُوشَ أَمَرَ أَنْ يُؤْتَى بِوَشْتِي الْمَلِكَةِ إِلَى أَمَامِهِ فَلَمْ تَأْتِ.
\par 18 وَفِي هَذَا الْيَوْمِ تَقُولُهُ رَئِيسَاتُ فَارِسَ وَمَادِي اللَّوَاتِي سَمِعْنَ خَبَرَ الْمَلِكَةِ لِجَمِيعِ رُؤَسَاءِ الْمَلِكِ. وَمِثْلُ ذَلِكَ احْتِقَارٌ وَغَضَبٌ.
\par 19 فَإِذَا حَسُنَ عِنْدَ الْمَلِكِ فَلْيَخْرُجْ أَمْرٌ مَلِكِيٌّ مِنْ عِنْدِهِ وَلْيُكْتَبْ فِي سُنَنِ فَارِسَ وَمَادِي فَلاَ يَتَغَيَّرَ أَنْ لاَ تَأْتِ وَشْتِي إِلَى أَمَامِ الْمَلِكِ أَحْشَوِيرُوشَ وَلْيُعْطِ الْمَلِكُ مُلْكَهَا لِمَنْ هِيَ أَحْسَنُ مِنْهَا.
\par 20 فَيُسْمَعُ أَمْرُ الْمَلِكِ الَّذِي يُخْرِجُهُ فِي كُلِّ مَمْلَكَتِهِ (لأَنَّهَا عَظِيمَةٌ) فَتُعْطِي جَمِيعُ النِّسَاءِ الْوَقَارَ لأَزْوَاجِهِنَّ مِنَ الْكَبِيرِ إِلَى الصَّغِيرِ].
\par 21 فَحَسُنَ الْكَلاَمُ فِي أَعْيُنِ الْمَلِكِ وَالرُّؤَسَاءِ وَعَمِلَ الْمَلِكُ حَسَبَ قَوْلِ مَمُوكَانَ.
\par 22 وَأَرْسَلَ رَسَائِلَ إِلَى كُلِّ بُلْدَانِ الْمَلِكِ إِلَى كُلِّ بِلاَدٍ حَسَبَ كِتَابَتِهَا وَإِلَى كُلِّ شَعْبٍ حَسَبَ لِسَانِهِ لِيَكُونَ كُلُّ رَجُلٍ مُتَسَلِّطاً فِي بَيْتِهِ وَيُتَكَلَّمَ بِذَلِكَ بِلِسَانِ شَعْبِهِ.

\chapter{2}

\par 1 بَعْدَ هَذِهِ الأُمُورِ لَمَّا خَمِدَ غَضَبُ الْمَلِكِ أَحْشَوِيرُوشَ ذَكَرَ وَشْتِي وَمَا عَمِلَتْهُ وَمَا حُتِمَ بِهِ عَلَيْهَا.
\par 2 فَقَالَ غِلْمَانُ الْمَلِكِ الَّذِينَ يَخْدِمُونَهُ: [لِيُطْلَبْ لِلْمَلِكِ فَتَيَاتٌ عَذَارَى حَسَنَاتُ الْمَنْظَرِ
\par 3 وَلْيُوَكِّلِ الْمَلِكُ وُكَلاَءَ فِي كُلِّ بِلاَدِ مَمْلَكَتِهِ لِيَجْمَعُوا كُلَّ الْفَتَيَاتِ الْعَذَارَى الْحَسَنَاتِ الْمَنْظَرِ إِلَى شُوشَنَ الْقَصْرِ إِلَى بَيْتِ النِّسَاءِ إِلَى يَدِ هَيْجَايَ خَصِيِّ الْمَلِكِ حَارِسِ النِّسَاءِ وَلْيُعْطَيْنَ أَدْهَانَ عِطْرِهِنَّ.
\par 4 وَالْفَتَاةُ الَّتِي تَحْسُنُ فِي عَيْنَيِ الْمَلِكِ فَلْتَمْلُكْ مَكَانَ وَشْتِي]. فَحَسُنَ الْكَلاَمُ فِي عَيْنَيِ الْمَلِكِ فَعَمِلَ هَكَذَا.
\par 5 كَانَ فِي شُوشَنَ الْقَصْرِ رَجُلٌ يَهُودِيٌّ اسْمُهُ مُرْدَخَايُ بْنُ يَائِيرَ بْنِ شَمْعِي بْنِ قَيْسٍ رَجُلٌ بِنْيَامِينِيٌّ
\par 6 قَدْ سُبِيَ مِنْ أُورُشَلِيمَ مَعَ السَّبْيِ الَّذِي سُبِيَ مَعَ يَكُنْيَا مَلِكِ يَهُوذَا الَّذِي سَبَاهُ نَبُوخَذْنَصَّرُ مَلِكُ بَابَِلَ.
\par 7 وَكَانَ مُرَبِّياً لِهَدَسَّةَ (أَيْ أَسْتِيرَ) بِنْتِ عَمِّهِ لأَنَّهُ لَمْ يَكُنْ لَهَا أَبٌ وَلاَ أُمٌّ. وَكَانَتِ الْفَتَاةُ جَمِيلَةَ الصُّورَةِ وَحَسَنَةَ الْمَنْظَرِ وَعِنْدَ مَوْتِ أَبِيهَا وَأُمِّهَا اتَّخَذَهَا مُرْدَخَايُ لِنَفْسِهِ ابْنَةً.
\par 8 فَلَمَّا سُمِعَ كَلاَمُ الْمَلِكِ وَأَمْرُهُ وَجُمِعَتْ فَتَيَاتٌ كَثِيرَاتٌ إِلَى شُوشَنَ الْقَصْرِ إِلَى يَدِ هَيْجَايَ أُخِذَتْ أَسْتِيرُ إِلَى بَيْتِ الْمَلِكِ إِلَى يَدِ هَيْجَايَ حَارِسِ النِّسَاءِ.
\par 9 وَحَسُنَتِ الْفَتَاةُ فِي عَيْنَيْهِ وَنَالَتْ نِعْمَةً بَيْنَ يَدَيْهِ فَبَادَرَ بِأَدْهَانِ عِطْرِهَا وَأَنْصِبَتِهَا لِيَعْطِيَهَا إِيَّاهَا مَعَ السَّبْعِ الْفَتَيَاتِ الْمُخْتَارَاتِ لِتُعْطَى لَهَا مِنْ بَيْتِ الْمَلِكِ وَنَقَلَهَا مَعَ فَتَيَاتِهَا إِلَى أَحْسَنِ مَكَانٍ فِي بَيْتِ النِّسَاءِ.
\par 10 وَلَمْ تُخْبِرْ أَسْتِيرُ عَنْ شَعْبِهَا وَجِنْسِهَا لأَنَّ مُرْدَخَايَ أَوْصَاهَا أَنْ لاَ تُخْبِرَ.
\par 11 وَكَانَ مُرْدَخَايُ يَتَمَشَّى يَوْماً فَيَوْماً أَمَامَ دَارِ بَيْتِ النِّسَاءِ لِيَسْتَعْلِمَ عَنْ سَلاَمَةِ أَسْتِيرَ وَعَمَّا يُصْنَعُ بِهَا.
\par 12 وَلَمَّا بَلَغَتْ نَوْبَةُ فَتَاةٍ فَفَتَاةٍ لِلدُّخُولِ إِلَى الْمَلِكِ أَحْشَوِيرُوشَ بَعْدَ أَنْ يَكُونَ لَهَا حَسَبَ سُنَّةِ النِّسَاءِ اثْنَا عَشَرَ شَهْراً لأَنَّهُ هَكَذَا كَانَتْ تُكْمَلُ أَيَّامُ تَعَطُّرِهِنَّ سِتَّةَ أَشْهُرٍ بِزَيْتِ الْمُرِّ وَسِتَّةَ أَشْهُرٍ بِالأَطْيَابِ وَأَدْهَانِ تَعَطُّرِ النِّسَاءِ
\par 13 وَهَكَذَا كَانَتْ كُلُّ فَتَاةٍ تَدْخُلُ إِلَى الْمَلِكِ. وَكُلُّ مَا قَالَتْ عَنْهُ أُعْطِيَ لَهَا لِلدُّخُولِ مَعَهَا مِنْ بَيْتِ النِّسَاءِ إِلَى بَيْتِ الْمَلِكِ.
\par 14 فِي الْمَسَاءِ دَخَلَتْ وَفِي الصَّبَاحِ رَجَعَتْ إِلَى بَيْتِ النِّسَاءِ الثَّانِي إِلَى يَدِ شَعَشْغَازَ خَصِيِّ الْمَلِكِ حَارِسِ السَّرَارِيِّ. لَمْ تَعُدْ تَدْخُلْ إِلَى الْمَلِكِ إِلاَّ إِذَا سُرَّ بِهَا الْمَلِكُ وَدُعِيَتْ بِاسْمِهَا.
\par 15 وَلَمَّا بَلَغَتْ نَوْبَةُ أَسْتِيرَ ابْنَةِ أَبَيِحَائِلَ عَمِّ مُرْدَخَايَ الَّذِي اتَّخَذَهَا لِنَفْسِهِ ابْنَةً لِلدُّخُولِ إِلَى الْمَلِكِ لَمْ تَطْلُبْ شَيْئاً إِلاَّ مَا قَالَ عَنْهُ هَيْجَايُ خَصِيُّ الْمَلِكِ حَارِسُ النِّسَاءِ. وَكَانَتْ أَسْتِيرُ تَنَالُ نِعْمَةً فِي عَيْنَيْ كُلِّ مَنْ رَآهَا.
\par 16 وَأُخِذَتْ أَسْتِيرُ إِلَى الْمَلِكِ أَحْشَوِيرُوشَ إِلَى بَيْتِ مُلْكِهِ فِي الشَّهْرِ الْعَاشِرِ (هُوَ شَهْرُ طِيبِيتَ) فِي السَّنَةِ السَّابِعَةِ لِمُلْكِهِ.
\par 17 فَأَحَبَّ الْمَلِكُ أَسْتِيرَ أَكْثَرَ مِنْ جَمِيعِ النِّسَاءِ وَوَجَدَتْ نِعْمَةً وَإِحْسَاناً قُدَّامَهُ أَكْثَرَ مِنْ جَمِيعِ الْعَذَارَى فَوَضَعَ تَاجَ الْمُلْكِ عَلَى رَأْسِهَا وَمَلَّكَهَا مَكَانَ وَشْتِي.
\par 18 وَعَمِلَ الْمَلِكُ وَلِيمَةً عَظِيمَةً لِجَمِيعِ رُؤَسَائِهِ وَعَبِيدِهِ وَلِيمَةَ أَسْتِيرَ. وَعَمِلَ رَاحَةً لِلْبِلاَدِ وَأَعْطَى عَطَايَا حَسَبَ كَرَمِ الْمَلِكِ.
\par 19 وَلَمَّا جُمِعَتِ الْعَذَارَى ثَانِيَةً كَانَ مُرْدَخَايُ جَالِساً بِبَابِ الْمَلِكِ.
\par 20 وَلَمْ تَكُنْ أَسْتِيرُ أَخْبَرَتْ عَنْ جِنْسِهَا وَشَعْبِهَا كَمَا أَوْصَاهَا مُرْدَخَايُ. وَكَانَتْ أَسْتِيرُ تَعْمَلُ حَسَبَ قَوْلِ مُرْدَخَايَ كَمَا كَانَتْ فِي تَرْبِيَتِهَا عِنْدَهُ.
\par 21 فِي تِلْكَ الأَيَّامِ بَيْنَمَا كَانَ مُرْدَخَايُ جَالِساً فِي بَابِ الْمَلِكِ غَضِبَ بِغْثَانُ وَتَرَشُ خَصِيَّا الْمَلِكِ حَارِسَا الْبَابِ وَطَلَبَا أَنْ يَمُدَّا أَيْدِيَهُمَا إِلَى الْمَلِكِ أَحْشَوِيرُوشَ.
\par 22 فَعُلِمَ الأَمْرُ عِنْدَ مُرْدَخَايَ فَأَخْبَرَ أَسْتِيرَ الْمَلِكَةَ فَأَخْبَرَتْ أَسْتِيرُ الْمَلِكَ بِاسْمِ مُرْدَخَايَ.
\par 23 فَفُحِصَ عَنِ الأَمْرِ وَوُجِدَ فَصُلِبَا كِلاَهُمَا عَلَى خَشَبَةٍ وَكُتِبَ ذَلِكَ فِي سِفْرِ أَخْبَارِ الأَيَّامِ أَمَامَ الْمَلِكِ.

\chapter{3}

\par 1 بَعْدَ هَذِهِ الأُمُورِ عَظَّمَ الْمَلِكُ أَحْشَوِيرُوشُ هَامَانَ بْنَ هَمَدَاثَا الأَجَاجِيَّ وَرَقَّاهُ وَجَعَلَ كُرْسِيَّهُ فَوْقَ جَمِيعِ الرُّؤَسَاءِ الَّذِينَ مَعَهُ.
\par 2 فَكَانَ كُلُّ عَبِيدِ الْمَلِكِ الَّذِينَ بِبَابِ الْمَلِكِ يَجْثُونَ وَيَسْجُدُونَ لِهَامَانَ لأَنَّهُ هَكَذَا أَوْصَى بِهِ الْمَلِكُ. وَأَمَّا مُرْدَخَايُ فَلَمْ يَجْثُ وَلَمْ يَسْجُدْ.
\par 3 فَقَالَ عَبِيدُ الْمَلِكِ الَّذِينَ بِبَابِ الْمَلِكِ لِمُرْدَخَايَ: [لِمَاذَا تَتَعَدَّى أَمْرَ الْمَلِكِ؟]
\par 4 وَإِذْ كَانُوا يُكَلِّمُونَهُ يَوْماً فَيَوْماً وَلَمْ يَكُنْ يَسْمَعْ لَهُمْ أَخْبَرُوا هَامَانَ لِيَرُوا هَلْ يَقُومُ كَلاَمُ مُرْدَخَايَ لأَنَّهُ أَخْبَرَهُمْ بِأَنَّهُ يَهُودِيٌّ.
\par 5 وَلَمَّا رَأَى هَامَانُ أَنَّ مُرْدَخَايَ لاَ يَجْثُو وَلاَ يَسْجُدُ لَهُ امْتَلَأَ هَامَانُ غَضَباً.
\par 6 وَازْدُرِيَ فِي عَيْنَيْهِ أَنْ يَمُدَّ يَدَهُ إِلَى مُرْدَخَايَ وَحْدَهُ لأَنَّهُمْ أَخْبَرُوهُ عَنْ شَعْبِ مُرْدَخَايَ. فَطَلَبَ هَامَانُ أَنْ يُهْلِكَ جَمِيعَ الْيَهُودِ الَّذِينَ فِي كُلِّ مَمْلَكَةِ أَحْشَوِيرُوشَ شَعْبَ مُرْدَخَايَ.
\par 7 فِي الشَّهْرِ الأَوَّلِ (أَيْ شَهْرِ نِيسَانَ) فِي السَّنَةِ الثَّانِيَةِ عَشَرَةَ لِلْمَلِكِ أَحْشَوِيرُوشَ كَانُوا يُلْقُونَ فُوراً (أَيْ قُرْعَةً) أَمَامَ هَامَانَ مِنْ يَوْمٍ إِلَى يَوْمٍ وَمِنْ شَهْرٍ إِلَى شَهْرٍ إِلَى الثَّانِي عَشَرَ (أَيْ شَهْرِ أَذَارَ).
\par 8 فَقَالَ هَامَانُ لِلْمَلِكِ أَحْشَوِيرُوشَ: [إِنَّهُ مَوْجُودٌ شَعْبٌ مَّا مُتَشَتِّتٌ وَمُتَفَرِّقٌ بَيْنَ الشُّعُوبِ فِي كُلِّ بِلاَدِ مَمْلَكَتِكَ وَسُنَنُهُمْ مُغَايِرَةٌ لِجَمِيعِ الشُّعُوبِ وَهُمْ لاَ يَعْمَلُونَ سُنَنَ الْمَلِكِ فَلاَ يَلِيقُ بِالْمَلِكِ تَرْكُهُمْ.
\par 9 فَإِذَا حَسُنَ عِنْدَ الْمَلِكِ فَلْيُكْتَبْ أَنْ يُبَادُوا وَأَنَا أَزِنُ عَشَرَةَ آلاَفِ وَزْنَةٍ مِنَ الْفِضَّةِ فِي أَيْدِي الَّذِينَ يَعْمَلُونَ الْعَمَلَ لِيُؤْتَى بِهَا إِلَى خَزَائِنِ الْمَلِكِ.
\par 10 فَنَزَعَ الْمَلِكُ خَاتِمَهُ مِنْ يَدِهِ وَأَعْطَاهُ لِهَامَانَ بْنِ هَمَدَاثَا الأَجَاجِيِّ عَدُوِّ الْيَهُودِ.
\par 11 وَقَالَ الْمَلِكُ لِهَامَانَ: [الْفِضَّةُ قَدْ أُعْطِيَتْ لَكَ وَالشَّعْبُ أَيْضاً لِتَفْعَلَ بِهِ مَا يَحْسُنُ فِي عَيْنَيْكَ].
\par 12 فَدُعِيَ كُتَّابُ الْمَلِكِ فِي الشَّهْرِ الأَوَّلِ فِي الْيَوْمِ الثَّالِثَ عَشَرَ مِنْهُ وَكُتِبَ حَسَبَ كُلِّ مَا أَمَرَ بِهِ هَامَانُ إِلَى مَرَازِبَةِ الْمَلِكِ وَإِلَى وُلاَةِ بِلاَدٍ فَبِلاَدٍ وَإِلَى رُؤَسَاءِ شَعْبٍ فَشَعْبٍ كُلِّ بِلاَدٍ كَكِتَابَتِهَا وَكُلِّ شَعْبٍ كَلِسَانِهِ كُتِبَ بِاسْمِ الْمَلِكِ أَحْشَوِيرُوشَ وَخُتِمَ بِخَاتِمِ الْمَلِكِ
\par 13 وَأُرْسِلَتِ الْكِتَابَاتُ بِيَدِ السُّعَاةِ إِلَى كُلِّ بُلْدَانِ الْمَلِكِ لإِهْلاَكِ وَقَتْلِ وَإِبَادَةِ جَمِيعِ الْيَهُودِ مِنَ الْغُلاَمِ إِلَى الشَّيْخِ وَالأَطْفَالِ وَالنِّسَاءِ فِي يَوْمٍ وَاحِدٍ فِي الثَّالِثَ عَشَرَ مِنَ الشَّهْرِ الثَّانِي عَشَرَ (أَيْ شَهْرِ أَذَارَ) وَأَنْ يَسْلِبُوا غَنِيمَتَهُمْ.
\par 14 صُورَةُ الْكِتَابَةِ الْمُعْطَاةِ سُنَّةً فِي كُلِّ الْبُلْدَانِ أُشْهِرَتْ بَيْنَ جَمِيعِ الشُّعُوبِ لِيَكُونُوا مُسْتَعِدِّينَ لِهَذَا الْيَوْمِ.
\par 15 فَخَرَجَ السُّعَاةُ وَأَمْرُ الْمَلِكِ يَحِثُّهُمْ وَأُعْطِيَ الأَمْرُ فِي شُوشَنَ الْقَصْرِ. وَجَلَسَ الْمَلِكُ وَهَامَانُ لِلشُّرْبِ وَأَمَّا الْمَدِينَةُ شُوشَنُ فَارْتَبَكَتْ.

\chapter{4}

\par 1 وَلَمَّا عَلِمَ مُرْدَخَايُ كُلَّ مَا عُمِلَ شَقَّ ثِيَابَهُ وَلَبِسَ مِسْحاً بِرَمَادٍ وَخَرَجَ إِلَى وَسَطِ الْمَدِينَةِ وَصَرَخَ صَرْخَةً عَظِيمَةً مُرَّةً
\par 2 وَجَاءَ إِلَى قُدَّامِ بَابِ الْمَلِكِ لأَنَّهُ لاَ يَدْخُلُ أَحَدٌ بَابَ الْمَلِكِ وَهُوَ لاَبِسٌ مِسْحاً.
\par 3 وَفِي كُلِّ كُورَةٍ حَيْثُمَا وَصَلَ إِلَيْهَا أَمْرُ الْمَلِكِ وَسُنَّتُهُ كَانَتْ مَنَاحَةٌ عَظِيمَةٌ عِنْدَ الْيَهُودِ وَصَوْمٌ وَبُكَاءٌ وَنَحِيبٌ. وَانْفَرَشَ مِسْحٌ وَرَمَادٌ لِكَثِيرِينَ.
\par 4 فَدَخَلَتْ جَوَارِي أَسْتِيرَ وَخِصْيَانُهَا وَأَخْبَرُوهَا فَاغْتَمَّتِ الْمَلِكَةُ جِدّاً وَأَرْسَلَتْ ثِيَاباً لإِلْبَاسِ مُرْدَخَايَ وَلأَجْلِ نَزْعِ مِسْحِهِ عَنْهُ فَلَمْ يَقْبَلْ.
\par 5 فَدَعَتْ أَسْتِيرُ هَتَاخَ وَاحِداً مِنْ خِصْيَانِ الْمَلِكِ الَّذِي أَوْقَفَهُ بَيْنَ يَدَيْهَا وَأَعْطَتْهُ وَصِيَّةً إِلَى مُرْدَخَايَ لِتَعْلَمَ مَاذَا وَلِمَاذَا.
\par 6 فَخَرَجَ هَتَاخُ إِلَى مُرْدَخَايَ إِلَى سَاحَةِ الْمَدِينَةِ الَّتِي أَمَامَ بَابِ الْمَلِكِ
\par 7 فَأَخْبَرَهُ مُرْدَخَايُ بِكُلِّ مَا أَصَابَهُ وَعَنْ مَبْلَغِ الْفِضَّةِ الَّذِي وَعَدَ هَامَانُ بِوَزْنِهِ لِخَزَائِنِ الْمَلِكِ عَنِ الْيَهُودِ لإِبَادَتِهِمْ
\par 8 وَأَعْطَاهُ صُورَةَ كِتَابَةِ الأَمْرِ الَّذِي أُعْطِيَ فِي شُوشَنَ لإِهْلاَكِهِمْ لِيُرِيَهَا لأَسْتِيرَ وَيُخْبِرَهَا وَيُوصِيَهَا أَنْ تَدْخُلَ إِلَى الْمَلِكِ وَتَتَضَرَّعَ إِلَيْهِ وَتَطْلُبَ مِنْهُ لأَجْلِ شَعْبِهَا.
\par 9 فَأَتَى هَتَاخُ وَأَخْبَرَ أَسْتِيرَ بِكَلاَمِ مُرْدَخَايَ.
\par 10 فَكَلَّمَتْ أَسْتِيرُ هَتَاخَ وَأَعْطَتْهُ وَصِيَّةً إِلَى مُرْدَخَايَ:
\par 11 [إِنَّ كُلَّ عَبِيدِ الْمَلِكِ وَشُعُوبِ بِلاَدِ الْمَلِكِ يَعْلَمُونَ أَنَّ كُلَّ رَجُلٍ دَخَلَ أَوِ امْرَأَةٍ إِلَى الْمَلِكِ إِلَى الدَّارِ الدَّاخِلِيَّةِ وَلَمْ يُدْعَ فَشَرِيعَتُهُ وَاحِدَةٌ أَنْ يُقْتَلَ إِلاَّ الَّذِي يَمُدُّ لَهُ الْمَلِكُ قَضِيبَ الذَّهَبِ فَإِنَّهُ يَحْيَا. وَأَنَا لَمْ أُدْعَ لأَدْخُلَ إِلَى الْمَلِكِ هَذِهِ الثَّلاَثِينَ يَوْماً].
\par 12 فَأَخْبَرُوا مُرْدَخَايَ بِكَلاَمِ أَسْتِيرَ.
\par 13 فَقَالَ مُرْدَخَايُ أَنْ تُجَاوَبَ أَسْتِيرُ: [لاَ تَفْتَكِرِي فِي نَفْسِكِ أَنَّكِ تَنْجِينَ فِي بَيْتِ الْمَلِكِ دُونَ جَمِيعِ الْيَهُودِ.
\par 14 لأَنَّكِ إِنْ سَكَتِّ سُكُوتاً فِي هَذَا الْوَقْتِ يَكُونُ الْفَرَجُ وَالنَّجَاةُ لِلْيَهُودِ مِنْ مَكَانٍ آخَرَ وَأَمَّا أَنْتِ وَبَيْتُ أَبِيكِ فَتَبِيدُونَ. وَمَنْ يَعْلَمُ إِنْ كُنْتِ لِوَقْتٍ مِثْلِ هَذَا وَصَلْتِ إِلَى الْمُلْكِ!]
\par 15 فَقَالَتْ أَسْتِيرُ أَنْ يُجَاوَبَ مُرْدَخَايُ:
\par 16 [اذْهَبِ اجْمَعْ جَمِيعَ الْيَهُودِ الْمَوْجُودِينَ فِي شُوشَنَ وَصُومُوا مِنْ جِهَتِي وَلاَ تَأْكُلُوا وَلاَ تَشْرَبُوا ثَلاَثَةَ أَيَّامٍ لَيْلاً وَنَهَاراً. وَأَنَا أَيْضاً وَجَوَارِيَّ نَصُومُ كَذَلِكَ. وَهَكَذَا أَدْخُلُ إِلَى الْمَلِكِ خِلاَفَ السُّنَّةِ. فَإِذَا هَلَكْتُ هَلَكْتُ].
\par 17 فَانْصَرَفَ مُرْدَخَايُ وَعَمِلَ حَسَبَ كُلِّ مَا أَوْصَتْهُ بِهِ أَسْتِيرُ.

\chapter{5}

\par 1 وَفِي الْيَوْمِ الثَّالِثِ لَبِسَتْ أَسْتِيرُ ثِيَاباً مَلَكِيَّةً وَوَقَفَتْ فِي دَارِ بَيْتِ الْمَلِكِ الدَّاخِلِيَّةِ مُقَابَِلَ بَيْتِ الْمَلِكِ وَالْمَلِكُ جَالِسٌ عَلَى كُرْسِيِّ مُلْكِهِ فِي بَيْتِ الْمُلْكِ مُقَابَِلَ مَدْخَلِ الْبَيْتِ.
\par 2 فَلَمَّا رَأَى الْمَلِكُ أَسْتِيرَ الْمَلِكَةَ وَاقِفَةً فِي الدَّارِ نَالَتْ نِعْمَةً فِي عَيْنَيْهِ فَمَدَّ الْمَلِكُ لأَسْتِيرَ قَضِيبَ الذَّهَبِ الَّذِي بِيَدِهِ فَدَنَتْ أَسْتِيرُ وَلَمَسَتْ رَأْسَ الْقَضِيبِ.
\par 3 فَقَالَ لَهَا الْمَلِكُ: [مَا لَكِ يَا أَسْتِيرُ الْمَلِكَةُ وَمَا هِيَ طِلْبَتُكِ؟ إِلَى نِصْفِ الْمَمْلَكَةِ تُعْطَى لَكِ].
\par 4 فَقَالَتْ أَسْتِيرُ: [إِنْ حَسُنَ عِنْدَ الْمَلِكِ فَلْيَأْتِ الْمَلِكُ وَهَامَانُ الْيَوْمَ إِلَى الْوَلِيمَةِ الَّتِي عَمِلْتُهَا لَهُ].
\par 5 فَقَالَ الْمَلِكُ: [أَسْرِعُوا بِهَامَانَ لِيُفْعَلَ كَلاَمُ أَسْتِيرَ]. فَأَتَى الْمَلِكُ وَهَامَانُ إِلَى الْوَلِيمَةِ الَّتِي عَمِلَتْهَا أَسْتِيرُ.
\par 6 فَقَالَ الْمَلِكُ لأَسْتِيرَ عِنْدَ شُرْبِ الْخَمْرِ: [مَا هُوَ سُؤْلُكِ فَيُعْطَى لَكِ وَمَا هِيَ طِلْبَتُكِ؟ إِلَى نَُِصْفِ الْمَمْلَكَةِ تُقْضَى].
\par 7 فَأَجَابَتْ أَسْتِيرُ: [إِنَّ سُؤْلِي وَطِلْبَتِي
\par 8 إِنْ وَجَدْتُ نِعْمَةً فِي عَيْنَيِ الْمَلِكِ وَإِذَا حَسُنَ عِنْدَ الْمَلِكِ أَنْ يُعْطَى سُؤْلِي وَتُقْضَى طِلْبَتِي أَنْ يَأْتِيَ الْمَلِكُ وَهَامَانُ إِلَى الْوَلِيمَةِ الَّتِي أَعْمَلُهَا لَهُمَا وَغَداً أَفْعَلُ حَسَبَ أَمْرِ الْمَلِكِ].
\par 9 فَخَرَجَ هَامَانُ فِي ذَلِكَ الْيَوْمِ فَرِحاً وَطَيِّبَ الْقَلْبِ. وَلَكِنْ لَمَّا رَأَى هَامَانُ مُرْدَخَايَ فِي بَابِ الْمَلِكِ وَلَمْ يَقُمْ وَلاَ تَحَرَّكَ لَهُ امْتَلَأَ هَامَانُ غَيْظاً عَلَى مُرْدَخَايَ.
\par 10 وَتَجَلَّدَ هَامَانُ وَدَخَلَ بَيْتَهُ وَأَرْسَلَ فَاسْتَحْضَرَ أَحِبَّاءَهُ وَزَرَشَ زَوْجَتَهُ
\par 11 وَعَدَّدَ لَهُمْ هَامَانُ عَظَمَةَ غِنَاهُ وَكَثْرَةَ بَنِيهِ وَكُلَّ مَا عَظَّمَهُ الْمَلِكُ بِهِ وَرَقَّاهُ عَلَى الرُّؤَسَاءِ وَعَبِيدِ الْمَلِكِ.
\par 12 وَقَالَ هَامَانُ: [حَتَّى إِنَّ أَسْتِيرَ الْمَلِكَةَ لَمْ تُدْخِلْ مَعَ الْمَلِكِ إِلَى الْوَلِيمَةِ الَّتِي عَمِلَتْهَا إِلاَّ إِيَّايَ. وَأَنَا غَداً أَيْضاً مَدْعُوٌّ إِلَيْهَا مَعَ الْمَلِكِ.
\par 13 وَكُلُّ هَذَا لاَ يُسَاوِي عِنْدِي شَيْئاً كُلَّمَا أَرَى مُرْدَخَايَ الْيَهُودِيَّ جَالِساً فِي بَابِ الْمَلِكِ].
\par 14 فَقَالَتْ لَهُ زَرَشُ زَوْجَتُهُ وَكُلُّ أَحِبَّائِهِ: [فَلْيَعْمَلُوا خَشَبَةً ارْتِفَاعُهَا خَمْسُونَ ذِرَاعاً وَفِي الصَّبَاحِ قُلْ لِلْمَلِكِ أَنْ يَصْلِبُوا مُرْدَخَايَ عَلَيْهَا ثُمَّ ادْخُلْ مَعَ الْمَلِكِ إِلَى الْوَلِيمَةِ فَرِحاً]. فَحَسُنَ الْكَلاَمُ عِنْدَ هَامَانَ وَعَمِلَ الْخَشَبَةَ.

\chapter{6}

\par 1 فِي تِلْكَ اللَّيْلَةِ طَارَ نَوْمُ الْمَلِكِ فَأَمَرَ بِأَنْ يُؤْتَى بِسِفْرِ تِذْكَارِ أَخْبَارِ الأَيَّامِ فَقُرِئَتْ أَمَامَ الْمَلِكِ.
\par 2 فَوُجِدَ مَكْتُوباً مَا أَخْبَرَ بِهِ مُرْدَخَايُ عَنْ بِغْثَانَا وَتَرَشَ خَصِيَّيِ الْمَلِكِ حَارِسَيِ الْبَابِ اللَّذَيْنِ طَلَبَا أَنْ يَمُدَّا أَيْدِيَهُمَا إِلَى الْمَلِكِ أَحْشَوِيرُوشَ.
\par 3 فَقَالَ الْمَلِكُ: [أَيَّةُ كَرَامَةٍ وَعَظَمَةٍ عُمِلَتْ لِمُرْدَخَايَ لأَجْلِ هَذَا؟] فَقَالَ غِلْمَانُ الْمَلِكِ الَّذِينَ يَخْدِمُونَهُ: [لَمْ يُعْمَلْ مَعَهُ شَيْءٌ].
\par 4 فَقَالَ الْمَلِكُ: [مَنْ في الدَّارِ؟] وَكَانَ هَامَانُ قَدْ دَخَلَ دَارَ بَيْتِ الْمَلِكِ الْخَارِجِيَّةَ لِيُكَلِّمَ الْمَلِكَ أَنْ يُصْلَبَ مُرْدَخَايُ عَلَى الْخَشَبَةِ الَّتِي أَعَدَّهَا لَهُ.
\par 5 فَقَالَ غِلْمَانُ الْمَلِكِ لَهُ: [هُوَذَا هَامَانُ وَاقِفٌ فِي الدَّارِ]. فَقَالَ الْمَلِكُ: [لِيَدْخُلْ].
\par 6 وَلَمَّا دَخَلَ هَامَانُ قَالَ لَهُ الْمَلِكُ: [مَاذَا يُعْمَلُ لِرَجُلٍ يُسَرُّ الْمَلِكُ بِأَنْ يُكْرِمَهُ؟] فَقَالَ هَامَانُ فِي قَلْبِهِ: [مَنْ يُسَرُّ الْمَلِكُ بِأَنْ يُكْرِمَهُ أَكْثَرَ مِنِّي؟]
\par 7 فَقَالَ هَامَانُ لِلْمَلِكِ: [إِنَّ الرَّجُلَ الَّذِي يُسَرُّ الْمَلِكُ بِأَنْ يُكْرِمَهُ
\par 8 يَأْتُونَ بِاللِّبَاسِ السُّلْطَانِيِّ الَّذِي يَلْبِسُهُ الْمَلِكُ وَبِالْفَرَسِ الَّذِي يَرْكَبُهُ الْمَلِكُ وَبِتَاجِ الْمُلْكِ الَّذِي يُوضَعُ عَلَى رَأْسِهِ
\par 9 وَيُدْفَعُ اللِّبَاسُ وَالْفَرَسُ لِرَجُلٍ مِنْ رُؤَسَاءِ الْمَلِكِ الأَشْرَافِ وَيُلْبِسُونَ الرَّجُلَ الَّذِي سُرَّ الْمَلِكُ بِأَنْ يُكْرِمَهُ وَيُرَكِّبُونَهُ عَلَى الْفَرَسِ فِي سَاحَةِ الْمَدِينَةِ وَيُنَادُونَ قُدَّامَهُ: هَكَذَا يُصْنَعُ لِلرَّجُلِ الَّذِي يُسَرُّ الْمَلِكُ بِأَنْ يُكْرِمَهُ].
\par 10 فَقَالَ الْمَلِكُ لِهَامَانَ: [أَسْرِعْ وَخُذِ اللِّبَاسَ وَالْفَرَسَ كَمَا تَكَلَّمْتَ وَافْعَلْ هَكَذَا لِمُرْدَخَايَ الْيَهُودِيِّ الْجَالِسِ فِي بَابِ الْمَلِكِ! لاَ يَسْقُطْ شَيْءٌ مِنْ جَمِيعِ مَا قُلْتَهُ].
\par 11 فَأَخَذَ هَامَانُ اللِّبَاسَ وَالْفَرَسَ وَأَلْبَسَ مُرْدَخَايَ وَأَرْكَبَهُ فِي سَاحَةِ الْمَدِينَةِ وَنَادَى قُدَّامَهُ: [هَكَذَا يُصْنَعُ لِلرَّجُلِ الَّذِي يُسَرُّ الْمَلِكُ بِأَنْ يُكْرِمَهُ].
\par 12 وَرَجَعَ مُرْدَخَايُ إِلَى بَابِ الْمَلِكِ. وَأَمَّا هَامَانُ فَأَسْرَعَ إِلَى بَيْتِهِ نَائِحاً وَمُغَطَّى الرَّأْسِ.
\par 13 وَقَصَّ هَامَانُ عَلَى زَرَشَ زَوْجَتِهِ وَجَمِيعِ أَحِبَّائِهِ كُلَّ مَا أَصَابَهُ. فَقَالَ لَهُ حُكَمَاؤُهُ وَزَرَشُ زَوْجَتُهُ: [إِذَا كَانَ مُرْدَخَايُ الَّذِي ابْتَدَأْتَ تَسْقُطُ قُدَّامَهُ مِنْ نَسْلِ الْيَهُودِ فَلاَ تَقْدِرُ عَلَيْهِ بَلْ تَسْقُطُ قُدَّامَهُ سُقُوطاً].
\par 14 وَفِيمَا هُمْ يُكَلِّمُونَهُ وَصَلَ خِصْيَانُ الْمَلِكِ وَأَسْرَعُوا لِلإِتْيَانِ بِهَامَانَ إِلَى الْوَلِيمَةِ الَّتِي عَمِلَتْهَا أَسْتِيرُ.

\chapter{7}

\par 1 فَجَاءَ الْمَلِكُ وَهَامَانُ لِيَشْرَبَا عِنْدَ أَسْتِيرَ الْمَلِكَةِ.
\par 2 فَقَالَ الْمَلِكُ لأَسْتِيرَ فِي الْيَوْمِ الثَّانِي أَيْضاً عِنْدَ شُرْبِ الْخَمْرِ: [مَا هُوَ سُؤْلُكِ يَا أَسْتِيرُ الْمَلِكَةُ فَيُعْطَى لَكِ وَمَا هِيَ طِلْبَتُكِ؟ وَلَوْ إِلَى نِصْفِ الْمَمْلَكَةِ تُقْضَى].
\par 3 فَأَجَابَتْ أَسْتِيرُ الْمَلِكَةُ: [إِنْ كُنْتُ قَدْ وَجَدْتُ نِعْمَةً فِي عَيْنَيْكَ أَيُّهَا الْمَلِكُ وَإِذَا حَسُنَ عِنْدَ الْمَلِكِ فَلِْتُعْطَ لِي نَفْسِي بِسُؤْلِي وَشَعْبِي بِطِلْبَتِي.
\par 4 لأَنَّنَا قَدْ بِعْنَا أَنَا وَشَعْبِي لِلْهَلاَكِ وَالْقَتْلِ وَالإِبَادَةِ. وَلَوْ بِعْنَا عَبِيداً وَإِمَاءً لَكُنْتُ سَكَتُّ مَعَ أَنَّ الْعَدُوَّ لاَ يُعَوِّضُ عَنْ خَسَارَةِ الْمَلِكِ].
\par 5 فَقَالَ الْمَلِكُ أَحْشَوِيرُوشُ لأَسْتِيرَ الْمَلِكَةِ: [مَنْ هُوَ وَأَيْنَ هُوَ هَذَا الَّذِي يَتَجَاسَرُ بِقَلْبِهِ عَلَى أَنْ يَعْمَلَ هَكَذَا؟]
\par 6 فَقَالَتْ أَسْتِيرُ: [هُوَ رَجُلٌ خَصْمٌ وَعَدُوٌّ! هَذَا هَامَانُ الرَّدِيءُ]. فَارْتَاعَ هَامَانُ أَمَامَ الْمَلِكِ وَالْمَلِكَةِ.
\par 7 فَقَامَ الْمَلِكُ بِغَيْظِهِ عَنْ شُرْبِ الْخَمْرِ إِلَى جَنَّةِ الْقَصْرِ وَوَقَفَ هَامَانُ لِيَتَوَسَّلَ عَنْ نَفْسِهِ إِلَى أَسْتِيرَ الْمَلِكَةِ لأَنَّهُ رَأَى أَنَّ الشَّرَّ قَدْ أُعِدَّ عَلَيْهِ مِنْ قِبَلِ الْمَلِكِ.
\par 8 وَلَمَّا رَجَعَ الْمَلِكُ مِنْ جَنَّةِ الْقَصْرِ إِلَى بَيْتِ شُرْبِ الْخَمْرِ وَهَامَانُ مُتَوَاقِعٌ عَلَى السَّرِيرِ الَّذِي كَانَتْ أَسْتِيرُ عَلَيْهِ قَالَ الْمَلِكُ: [هَلْ أَيْضاً يَكْبِسُ الْمَلِكَةَ فِي الْبَيْتِ؟] وَلَمَّا خَرَجَتِ الْكَلِمَةُ مِنْ فَمِ الْمَلِكِ غَطُّوا وَجْهَ هَامَانَ.
\par 9 فَقَالَ حَرْبُونَا وَاحِدٌ مِنَ الْخِصْيَانِ الَّذِينَ بَيْنَ يَدَيِ الْمَلِكِ: [هُوَذَا الْخَشَبَةُ أَيْضاً الَّتِي عَمِلَهَا هَامَانُ لِمُرْدَخَايَ الَّذِي تَكَلَّمَ بِالْخَيْرِ نَحْوَ الْمَلِكِ قَائِمَةٌ فِي بَيْتِ هَامَانَ ارْتِفَاعُهَا خَمْسُونَ ذِرَاعاً]. فَقَالَ الْمَلِكُ: [اصْلِبُوهُ عَلَيْهَا].
\par 10 فَصَلَبُوا هَامَانَ عَلَى الْخَشَبَةِ الَّتِي أَعَدَّهَا لِمُرْدَخَايَ. ثُمَّ سَكَنَ غَضَبُ الْمَلِكِ.

\chapter{8}

\par 1 فِي ذَلِكَ الْيَوْمِ أَعْطَى الْمَلِكُ أَحْشَوِيرُوشُ لأَسْتِيرَ الْمَلِكَةِ بَيْتَ هَامَانَ عَدُوِّ الْيَهُودِ. وَأَتَى مُرْدَخَايُ إِلَى أَمَامِ الْمَلِكِ لأَنَّ أَسْتِيرَ أَخْبَرَتْهُ بِقَرَابَتِهِ.
\par 2 وَنَزَعَ الْمَلِكُ خَاتِمَهُ الَّذِي أَخَذَهُ مِنْ هَامَانَ وَأَعْطَاهُ لِمُرْدَخَايَ. وَأَقَامَتْ أَسْتِيرُ مُرْدَخَايَ عَلَى بَيْتِ هَامَانَ.
\par 3 ثُمَّ عَادَتْ أَسْتِيرُ وَتَكَلَّمَتْ أَمَامَ الْمَلِكِ وَسَقَطَتْ عِنْدَ رِجْلَيْهِ وَبَكَتْ وَتَضَرَّعَتْ إِلَيْهِ أَنْ يُزِيلَ شَرَّ هَامَانَ الأَجَاجِيِّ وَتَدْبِيرَهُ الَّذِي دَبَّرَهُ عَلَى الْيَهُودِ.
\par 4 فَمَدَّ الْمَلِكُ لأَسْتِيرَ قَضِيبَ الذَّهَبِ فَقَامَتْ أَسْتِيرُ وَوَقَفَتْ أَمَامَ الْمَلِكِ
\par 5 وَقَالَتْ: [إِذَا حَسُنَ عِنْدَ الْمَلِكِ وَإِنْ كُنْتُ قَدْ وَجَدْتُ نِعْمَةً أَمَامَهُ وَاسْتَقَامَ الأَمْرُ أَمَامَ الْمَلِكِ وَحَسُنْتُ أَنَا لَدَيْهِ فَلْيُكْتَبْ لِتُرَدَّ كِتَابَاتُ تَدْبِيرِ هَامَانَ بْنِ هَمَدَاثَا الأَجَاجِيِّ الَّتِي كَتَبَهَا لإِبَادَةِ الْيَهُودِ الَّذِينَ فِي كُلِّ بِلاَدِ الْمَلِكِ.
\par 6 لأَنَّنِي كَيْفَ أَسْتَطِيعُ أَنْ أَرَى الشَّرَّ الَّذِي يُصِيبُ شَعْبِي وَكَيْفَ أَسْتَطِيعُ أَنْ أَرَى هَلاَكَ جِنْسِي؟].
\par 7 فَقَالَ الْمَلِكُ أَحْشَوِيرُوشُ لأَسْتِيرَ الْمَلِكَةِ وَمُرْدَخَايَ الْيَهُودِيِّ: [هُوَذَا قَدْ أَعْطَيْتُ بَيْتَ هَامَانَ لأَسْتِيرَ أَمَّا هُوَ فَقَدْ صَلَبُوهُ عَلَى الْخَشَبَةِ مِنْ أَجْلِ أَنَّهُ مَدَّ يَدَهُ إِلَى الْيَهُودِ.
\par 8 فَاكْتُبَا أَنْتُمَا إِلَى الْيَهُودِ مَا يَحْسُنُ فِي أَعْيُنِكُمَا بِاسْمِ الْمَلِكِ وَاخْتُمَاهُ بِخَاتِمِ الْمَلِكِ لأَنَّ الْكِتَابَةَ الَّتِي تُكْتَبُ بِاسْمِ الْمَلِكِ وَتُخْتَمُ بِخَاتِمِهِ لاَ تُرَدُّ].
\par 9 فَدُعِيَ كُتَّابُ الْمَلِكِ فِي ذَلِكَ الْوَقْتِ فِي الشَّهْرِ الثَّالِثِ (أَيْ شَهْرِ سِيوَانَ) فِي الثَّالِثِ وَالْعِشْرِينَ مِنْهُ وَكُتِبَ حَسَبَ كُلِّ مَا أَمَرَ بِهِ مُرْدَخَايُ إِلَى الْيَهُودِ وَإِلَى الْمَرَازِبَةِ وَالْوُلاَةِ وَرُؤَسَاءِ الْبُلْدَانِ الَّتِي مِنَ الْهِنْدِ إِلَى كُوشَ؛ مِئَةٍ وَسَبْعٍ وَعِشْرِينَ كُورَةً؛ إِلَى كُلِّ كُورَةٍ بِكِتَابَتِهَا وَكُلِّ شَعْبٍ بِلِسَانِهِ؛ وَإِلَى الْيَهُودِ بِكِتَابَتِهِمْ وَلِسَانِهِمْ.
\par 10 فَكَتَبَ بِاسْمِ الْمَلِكِ أَحْشَوِيرُوشَ وَخَتَمَ بِخَاتِمِ الْمَلِكِ؛ وَأَرْسَلَ رَسَائِلَ بِأَيْدِي بَرِيدِ الْخَيْلِ رُكَّابِ الْجِيَادِ وَالْبِغَالِ بَنِي الْجِيَادِ الأَصِيلَةِ
\par 11 الَّتِي بِهَا أَعْطَى الْمَلِكُ الْيَهُودَ فِي مَدِينَةٍ فَمَدِينَةٍ أَنْ يَجْتَمِعُوا وَيَقِفُوا لأَجْلِ أَنْفُسِهِم؛ْ وَيُهْلِكُوا وَيَقْتُلُوا وَيُبِيدُوا قُوَّةَ كُلِّ شَعْبٍ وَكُورَةٍ تُضَادُّهُمْ حَتَّى الأَطْفَالَ وَالنِّسَاءَ؛ وَأَنْ يَسْلُبُوا غَنِيمَتَهُمْ
\par 12 فِي يَوْمٍ وَاحِدٍ فِي كُلِّ كُوَرِ الْمَلِكِ أَحْشَوِيرُوشَ؛ فِي الثَّالِثَ عَشَرَ مِنَ الشَّهْرِ الثَّانِي عَشَرَ (أَيْ شَهْرِ أَذَارَ).
\par 13 صُورَةُ الْكِتَابَةِ الْمُعْطَاةِ سُنَّةً فِي كُلِّ الْبُلْدَانِ أُشْهِرَتْ عَلَى جَمِيعِ الشُّعُوبِ أَنْ يَكُونَ الْيَهُودُ مُسْتَعِدِّينَ لِهَذَا الْيَوْمِ لِيَنْتَقِمُوا مِنْ أَعْدَائِهِمْ.
\par 14 فَخَرَجَ السُّعَاةُ رُكَّابُ الْجِيَادِ وَالْبِغَالِ وَأَمْرُ الْمَلِكِ يَحُثُّهُمْ وَيُعَجِّلُهُمْ وَأُعْطِيَ الأَمْرُ فِي شُوشَنَ الْقَصْرِ.
\par 15 وَخَرَجَ مُرْدَخَايُ مِنْ أَمَامِ الْمَلِكِ بِلِبَاسٍ مَلِكِيٍّ أَسْمَانْجُونِيٍّ وَأَبْيَضَ وَتَاجٌ عَظِيمٌ مِنْ ذَهَبٍ وَحُلَّةٌ مِنْ بَزٍّ وَأُرْجُوَانٍ. وَكَانَتْ مَدِينَةُ شُوشَنَ مُتَهَلِّلَةً وَفَرِحَةً.
\par 16 وَكَانَ لِلْيَهُودِ نُورٌ وَفَرَحٌ وَبَهْجَةٌ وَكَرَامَةٌ.
\par 17 وَفِي كُلِّ بِلاَدٍ وَمَدِينَةٍ كُلِّ مَكَانٍ وَصَلَ إِلَيْهِ كَلاَمُ الْمَلِكِ وَأَمْرُهُ كَانَ فَرَحٌ وَبَهْجَةٌ عِنْدَ الْيَهُودِ وَوَلاَئِمُ وَيَوْمٌ طَيِّبٌ. وَكَثِيرُونَ مِنْ شُعُوبِ الأَرْضِ تَهَوَّدُوا لأَنَّ رُعْبَ الْيَهُودِ وَقَعَ عَلَيْهِمْ.

\chapter{9}

\par 1 وَفِي الشَّهْرِ الثَّانِي عَشَرَ (أَيْ شَهْرِ أَذَارَ) فِي الْيَوْمِ الثَّالِثَ عَشَرَ مِنْهُ حِينَ قَرُبَ كَلاَمُ الْمَلِكِ وَأَمْرُهُ مِنَ التَّنْفِيذِ فِي الْيَوْمِ الَّذِي انْتَظَرَ فِيهِ أَعْدَاءُ الْيَهُودِ أَنْ يَتَسَلَّطُوا عَلَيْهِمْ فَتَحَوَّلَ ذَلِكَ حَتَّى إِنَّ الْيَهُودَ تَسَلَّطُوا عَلَى مُبْغِضِيهِمِ
\par 2 اجْتَمَعَ الْيَهُودُ فِي مُدُنِهِمْ فِي كُلِّ بِلاَدِ الْمَلِكِ أَحْشَوِيرُوشَ لِيَمُدُّوا أَيْدِيَهُمْ إِلَى طَالِبِي أَذِيَّتِهِمْ فَلَمْ يَقِفْ أَحَدٌ قُدَّامَهُمْ لأَنَّ رُعْبَهُمْ سَقَطَ عَلَى جَمِيعِ الشُّعُوبِ.
\par 3 وكُلُّ رُؤَسَاءِ الْبُلْدَانِ وَالْمَرَازِبَةُ وَالْوُلاَةُ وَعُمَّالُ الْمَلِكِ سَاعَدُوا الْيَهُودَ لأَنَّ رُعْبَ مُرْدَخَايَ سَقَطَ عَلَيْهِمْ.
\par 4 لأَنَّ مُرْدَخَايَ كَانَ عَظِيماً فِي بَيْتِ الْمَلِكِ وَسَارَ خَبَرُهُ فِي كُلِّ الْبُلْدَانِ لأَنَّ الرَّجُلَ مُرْدَخَايَ كَانَ يَتَزَايَدُ عَظَمَةً.
\par 5 فَضَرَبَ الْيَهُودُ جَمِيعَ أَعْدَائِهِمْ ضَرْبَةَ سَيْفٍ وَقَتْلٍ وَهَلاَكٍ وَعَمِلُوا بِمُبْغِضِيهِمْ مَا أَرَادُوا.
\par 6 وَقَتَلَ الْيَهُودُ فِي شُوشَنَ الْقَصْرِ وَأَهْلَكُوا خَمْسَ مِئَةِ رَجُلٍ.
\par 7 وَفَرْشَنْدَاثَا وَدَلْفُونَ وَأَسْفَاثَا
\par 8 وَفُورَاثَا وَأَدَلْيَا وَأَرِيدَاثَا
\par 9 وَفَرْمَشْتَا وَأَرِيسَايَ وَأَرِيدَايَ وَيِزَاثَا
\par 10 عَشَرَةَ بَنِي هَامَانَ بْنِ هَمَدَاثَا عَدُوِّ الْيَهُودِ قَتَلُوهُمْ وَلَكِنَّهُمْ لَمْ يَمُدُّوا أَيْدِيَهُمْ إِلَى النَّهْبِ.
\par 11 فِي ذَلِكَ الْيَوْمِ أُتِيَ بِعَدَدِ الْقَتْلَى فِي شُوشَنَ الْقَصْرِ إِلَى بَيْنِ يَدَيِ الْمَلِكِ.
\par 12 فَقَالَ الْمَلِكُ لأَسْتِيرَ الْمَلِكَةِ فِي شُوشَنَ الْقَصْرِ: [قَدْ قَتَلَ الْيَهُودُ وَأَهْلَكُوا خَمْسَ مِئَةِ رَجُلٍ وَبَنِي هَامَانَ الْعَشَرَةَ فَمَاذَا عَمِلُوا فِي بَاقِي بُلْدَانِ الْمَلِكِ! فَمَا هُوَ سُؤْلُكِ فَيُعْطَى لَكِ وَمَا هِيَ طِلْبَتُكِ بَعْدُ فَتُقْضَى؟].
\par 13 فَقَالَتْ أَسْتِيرُ: [إِنْ حَسُنَ عِنْدَ الْمَلِكِ فَلْيُعْطَ غَداً أَيْضاً لِلْيَهُودِ الَّذِينَ فِي شُوشَنَ أَنْ يَعْمَلُوا كَمَا فِي هَذَا الْيَوْمِ وَيَصْلِبُوا بَنِي هَامَانَ الْعَشَرَةَ عَلَى الْخَشَبَةِ].
\par 14 فَأَمَرَ الْمَلِكُ أَنْ يَعْمَلُوا هَكَذَا وَأُعْطِيَ الأَمْرُ فِي شُوشَنَ. فَصَلَبُوا بَنِي هَامَانَ الْعَشَرَةَ.
\par 15 ثُمَّ اجْتَمَعَ الْيَهُودُ الَّذِينَ فِي شُوشَنَ فِي الْيَوْمِ الرَّابِعِ عَشَرَ أَيْضاً مِنْ شَهْرِ أَذَارَ وَقَتَلُوا فِي شُوشَنَ ثَلاَثَ مِئَةِ رَجُلٍ وَلَكِنَّهُمْ لَمْ يَمُدُّوا أَيْدِيَهُمْ إِلَى النَّهْبِ.
\par 16 وَبَاقِي الْيَهُودِ الَّذِينَ فِي بُلْدَانِ الْمَلِكِ اجْتَمَعُوا وَوَقَفُوا لأَجْلِ أَنْفُسِهِمْ وَاسْتَرَاحُوا مِنْ أَعْدَائِهِمْ وَقَتَلُوا مِنْ مُبْغِضِيهِمْ خَمْسَةً وَسَبْعِينَ أَلْفاً. وَلَكِنَّهُمْ لَمْ يَمُدُّوا أَيْدِيَهُمْ إِلَى النَّهْبِ.
\par 17 فِي الْيَوْمِ الثَّالِثِ عَشَرَ مِنْ شَهْرِ أَذَارَ. وَاسْتَرَاحُوا فِي الْيَوْمِ الرَّابِعِ عَشَرَ مِنْهُ وَجَعَلُوهُ يَوْمَ شُرْبٍ وَفَرَحٍ.
\par 18 وَالْيَهُودُ الَّذِينَ فِي شُوشَنَ اجْتَمَعُوا فِي الثَّالِثِ عَشَرَ وَالرَّابِعِ عَشَرَ مِنْهُ وَاسْتَرَاحُوا فِي الْخَامِسِ عَشَرَ وَجَعَلُوهُ يَوْمَ شُرْبٍ وَفَرَحٍ.
\par 19 لِذَلِكَ يَهُودُ الأَعْرَاءِ السَّاكِنُونَ فِي مُدُنِ الأَعْرَاءِ جَعَلُوا الْيَوْمَ الرَّابِعَ عَشَرَ مِنْ شَهْرِ أَذَارَ لِلْفَرَحِ وَالشُّرْبِ وَيَوْماً طَيِّباً وَلإِرْسَالِ أَنْصِبَةٍ مِنْ كُلِّ وَاحِدٍ إِلَى صَاحِبِهِ.
\par 20 وَكَتَبَ مُرْدَخَايُ هَذِهِ الأُمُورَ وَأَرْسَلَ رَسَائِلَ إِلَى جَمِيعِ الْيَهُودِ الَّذِينَ فِي كُلِّ بُلْدَانِ الْمَلِكِ أَحْشَوِيرُوشَ الْقَرِيبِينَ وَالْبَعِيدِينَ
\par 21 لِيُوجِبَ عَلَيْهِمْ أَنْ يُعَيِّدُوا فِي الْيَوْمِ الرَّابِعِ عَشَرَ مِنْ شَهْرِ أَذَارَ وَالْيَوْمِ الْخَامِسِ عَشَرَ مِنْهُ فِي كُلِّ سَنَةٍ
\par 22 حَسَبَ الأَيَّامِ الَّتِي اسْتَرَاحَ فِيهَا الْيَهُودُ مِنْ أَعْدَائِهِمْ وَالشَّهْرِ الَّذِي تَحَوَّلَ عِنْدَهُمْ مِنْ حُزْنٍ إِلَى فَرَحٍ وَمِنْ نَوْحٍ إِلَى يَوْمٍ طَيِّبٍ لِيَجْعَلُوهَا أَيَّامَ شُرْبٍ وَفَرَحٍ وَإِرْسَالِ أَنْصِبَةٍ مِنْ كُلِّ وَاحِدٍ إِلَى صَاحِبِهِ وَعَطَايَا لِلْفُقَرَاءِ.
\par 23 فَقَبِلَ الْيَهُودُ مَا ابْتَدَأُوا يَعْمَلُونَهُ وَمَا كَتَبَهُ مُرْدَخَايُ إِلَيْهِمْ.
\par 24 وَلأَنَّ هَامَانَ بْنَ هَمَدَاثَا الأَجَاجِيَّ عَدُوَّ الْيَهُودِ جَمِيعاً تَفَكَّرَ عَلَى الْيَهُودِ لِيُبِيدَهُمْ وَأَلْقَى فُوراً (أَيْ قُرْعَةً) لإِفْنَائِهِمْ وَإِبَادَتِهِمْ.
\par 25 وَعِنْدَ دُخُولِهَا إِلَى أَمَامِ الْمَلِكِ أَمَرَ بِكِتَابَةٍ أَنْ يُرَدَّ تَدْبِيرُهُ الرَّدِيءُ الَّذِي دَبَّرَهُ ضِدَّ الْيَهُودِ عَلَى رَأْسِهِ وَأَنْ يَصْلِبُوهُ هُوَ وَبَنِيهِ عَلَى الْخَشَبَةِ.
\par 26 لِذَلِكَ دَعُوا تِلْكَ الأَيَّامِ [فُورِيمَ] عَلَى اسْمِ الْفُورِ. لِذَلِكَ مِنْ أَجْلِ جَمِيعِ كَلِمَاتِ هَذِهِ الرِّسَالَةِ وَمَا رَأُوهُ مِنْ ذَلِكَ وَمَا أَصَابَهُمْ
\par 27 أَوْجَبَ الْيَهُودُ وَقَبِلُوا عَلَى أَنْفُسِهِمْ وَعَلَى نَسْلِهِمْ وَعَلَى جَمِيعِ الَّذِينَ يَلْتَصِقُونَ بِهِمْ حَتَّى لاَ يَزُولَ أَنْ يُعَيِّدُوا هَذَيْنِ الْيَوْمَيْنِ حَسَبَ كِتَابَتِهِمَا وَحَسَبَ أَوْقَاتِهِمَا كُلَّ سَنَةٍ
\par 28 وَأَنْ يُذْكَرَ هَذَانِ الْيَوْمَانِ وَيُحْفَظَا فِي دَوْرٍ فَدَوْرٍ وَعَشِيرَةٍ فَعَشِيرَةٍ وَبِلاَدٍ فَبِلاَدٍ وَمَدِينَةٍ فَمَدِينَةٍ. وَيَوْمَا الْفُورِ هَذَانِ لاَ يَزُولاَنِ مِنْ وَسَطِ الْيَهُودِ وَذِكْرُهُمَا لاَ يَفْنَى مِنْ نَسْلِهِمْ.
\par 29 وَكَتَبَتْ أَسْتِيرُ الْمَلِكَةُ بِنْتُ أَبِيحَائِلَ وَمُرْدَخَايُ الْيَهُودِيُّ بِكُلِّ سُلْطَانٍ بِإِيجَابِ رِسَالَةِ الْفُورِيمِ هَذِهِ ثَانِيَةً.
\par 30 وَأَرْسَلَ الْكِتَابَاتِ إِلَى جَمِيعِ الْيَهُودِ إِلَى كُوَرِ مَمْلَكَةِ أَحْشَوِيرُوشَ الْمِئَةِ وَالسَّبْعِ وَالْعِشْرِينَ بِكَلاَمِ سَلاَمٍ وَأَمَانَةٍ
\par 31 لإِيجَابِ يَوْمَيِ الْفُورِيمِ هَذَيْنِ فِي أَوْقَاتِهِمَا كَمَا أَوْجَبَ عَلَيْهِمْ مُرْدَخَايُ الْيَهُودِيُّ وَأَسْتِيرُ الْمَلِكَةُ وَكَمَا أَوْجَبُوا عَلَى أَنْفُسِهِمْ وَعَلَى نَسْلِهِمْ أُمُورَ الأَصْوَامِ وَصُرَاخِهِمْ.
\par 32 وَأَمْرُ أَسْتِيرَ أَوْجَبَ أُمُورَ الْفُورِيمِ هَذِهِ فَكُتِبَتْ فِي السِّفْرِ.

\chapter{10}

\par 1 وَوَضَعَ الْمَلِكُ أَحْشَوِيرُوشُ جِزْيَةً عَلَى الأَرْضِ وَجَزَائِرِ الْبَحْرِ.
\par 2 وَكُلُّ عَمَلِ سُلْطَانِهِ وَجَبَرُوتِهِ وَإِذَاعَةُ عَظَمَةِ مُرْدَخَايَ الَّذِي عَظَّمَهُ الْمَلِكُ مَكْتُوبَةٌ فِي سِفْرِ أَخْبَارِ الأَيَّامِ لِمُلُوكِ مَادِي وَفَارِسَ.
\par 3 لأَنَّ مُرْدَخَايَ الْيَهُودِيَّ كَانَ ثَانِىَ الْمَلِكِ أَحْشَوِيرُوشَ وَعَظِيماً بَيْنَ الْيَهُودِ وَمَقْبُولاً عِنْدَ كَثْرَةِ إِخْوَتِهِ طَالِباً الْخَيْرَ لِشَعْبِهِ وَمُتَكَلِّماً بِالسَّلاَمِ لِكُلِّ نَسْلِهِ.

\end{document}