\begin{document}


\title{وصية أيوب}

\chapter{1}

\par 1 وفي اليوم الذي مرض فيه وعلم أنه سيضطر إلى مغادرة مسكنه الجسدي، جمع أبناءه السبعة وبناته الثلاث وتحدث إليهم على النحو التالي:

\par 2 «شكّلوا دائرة حولي، يا أطفال، واستمعوا، وسأروي لكم ما فعله الرب من أجلي وكل ما حدث لي

\par 3 لأني أنا أيوب أبوكم.

\par 4 فاعلموا إذن يا أبنائي أنكم جيل مختار، فاحذروا من ميلادكم الكريم

\par 5 لأني من أبناء عيسو. أخي هو ناحور، وأمك هي دينة. منها صرت أباك

\par 6 لأن زوجتي الأولى ماتت مع أطفالي العشرة الآخرين في موت مرير.

\par 7 اسمعوا الآن يا أولادي، وسأخبركم بما حدث لي.

\par 8 كنتُ رجلاً غنيًا جدًا أعيش في الشرق في أرض أوسيتيس (أوتز)، وقبل أن يُسمّيني الرب أيوب، كنتُ أُدعى يوباب

\par 9 كانت بداية محنتي هكذا. بالقرب من منزلي كان هناك صنم لشخص يعبده الناس؛ وكنت أرى باستمرار محرقات تُقدم إليه كإله

\par 10 ثم تأملت وقلت لنفسي: "أهذا هو الذي خلق السماء والأرض والبحر وكلنا؟ كيف سأعرف الحقيقة؟"

\par 11 وفي تلك الليلة، بينما كنت نائمًا، جاءني صوت ينادي: "يوباب! يوباب! قم، وسأخبرك من هو الشخص الذي تريد أن تعرفه."

\par 12 لكن هذا الذي يقدم له الناس المحرقات والسفك ليس هو الله، بل هذه هي قوة وعمل المُغوي (الشيطان) الذي يخدع به الناس

\par 13 فلما سمعتُ ذلك، خَرَرْتُ على الأرض وسجدتُ قائلًا:

\par 14 يا ربي الذي يتكلم من أجل خلاص روحي. أتوسل إليك، إذا كان هذا صنمًا للشيطان، أتوسل إليك، دعني أذهب من هنا وأدمره وأطهر هذه البقعة

\par 15 لأنه لا يوجد من يمنعني من فعل هذا، فأنا ملك هذه الأرض، حتى لا يضل سكانها بعد الآن

\par 16 وأجابني الصوت الذي تكلم من اللهب: "يمكنك تطهير هذه البقعة."

\par 17 لكن ها أنا أعلن لك ما أمرني الرب أن أخبرك به، لأني رئيس ملائكة الله

\par 18 فقلت: «مهما قيل لعبده، سأسمع».

\par 19 "فقال لي رئيس الملائكة: هكذا يقول الرب: إذا عزمت على تدمير وإزالة صورة الشيطان، فإنه سيبدأ حربًا ضدك بغضب، وسيظهر عليك كل شره.

\par 20 سيجلب عليك ضربات شديدة كثيرة، ويأخذ منك كل ما لديك

\par 21 سيأخذ أولادك، وسيُنزل عليك شرورًا كثيرة

\par 22 إذن عليك أن تصارع كالرياضي وتقاوم الألم، واثقًا من مكافأتك، وتتغلب على التجارب والمحن

\par 23 ولكن عندما تصبر، سأجعل اسمك معروفًا في جميع أجيال الأرض حتى نهاية العالم

\par 24 وسأعيد لك كل ما كان لديك، وسوف يُعطى لك ضعف ما ستفقده، لكي تعرف أن الله لا ينظر إلى الأشخاص، بل يعطي لكل من يستحق الخير.

\par 25 ويُعطى لك أيضًا، وتضع على رأسك إكليلًا من حجر العقيق

\par 26 وعند القيامة ستستيقظ للحياة الأبدية. حينئذٍ ستعرف أن الرب عادل وصادق وقدير

\par 27 عندها، أجبتُ يا أبنائي: «بسبب محبتي لله، سأتحمل حتى الموت كل ما سيأتي عليّ، ولن أتراجع».

\par 28 ثم وضع الملاك ختمه عليّ وتركني.

\chapter{2}

\par 1 بعد ذلك، استيقظتُ في الليل وأخذتُ خمسين عبدًا وذهبتُ إلى معبد الصنم ودمرتُه بالكامل

\par 2 وهكذا عدت إلى منزلي وأمرت بإغلاق الباب بإحكام، وقلت لبوابيّ:

\par 3 «إذا سألني أحد، فلا تحضر لي تقريرًا، بل قل له: إنه يحقق في أمور عاجلة. إنه بالداخل».

\par 4 ثم تنكر الشيطان في صورة متسول وطرق الباب بشدة، قائلاً للبواب:

\par 5 «أبلغ أيوب وقل إني أرغب في لقائه»،

\par 6 ودخل الحارس وأخبرني بذلك، لكنه سمع مني أنني أدرس.

\par 7 بعد أن فشل الشرير في ذلك، ذهب وأخذ على كتفه سلة قديمة ممزقة ودخل وتحدث إلى البواب قائلاً: "قل لأيوب: أعطني خبزًا من يديك لآكل".

\par 8 فلما سمعت ذلك، أعطيتها خبزًا محترقًا لتعطيه إياه، وقلت له: لا تنتظر أن تأكل من خبزي، فهو ممنوع عليك

\par 9 لكن البواب، إذ خجلت من أن تعطيه الخبز المحروق والرماد، لأنها لم تكن تعلم أنه شيطان، أخذت من خبزها الفاخر وأعطته إياه

\par 10 فأخذه، وعلم ما حدث، فقال للفتاة: «اذهبي من هنا أيتها الخادمة الشريرة، وأحضري لي الخبز الذي أُعطي لكِ».

\par 11 فصرخ العبد وتكلم بحزن: «أنت تقول الحق، إذ تقول إني عبد سيئ لأني لم أفعل كما أمرني سيدي».

\par 12 فالتفت وأحضر له الخبز المحروق وقال له: هكذا قال سيدي: لا تأكل من خبزي بعد الآن لأنه ممنوع عليك

\par 13 وهذا أعطانيه [قائلا: هذا أعطيه] لكي لا تُوجَّه عليَّ تهمة أنني لم أُعطِ للعدو الذي سأل.)

\par 14 ولما سمع الشيطان ذلك، أعاد الخادم إليّ قائلاً: "كما ترى هذا الخبز محترقًا بالكامل، سأحرق جسدك قريبًا لأجعله هكذا".

\par 15 فأجبت: «افعل ما تريد أن تفعله وأنجز كل ما تخطط له. فأنا مستعد لتحمل أي شيء تجلبه عليّ».

\par 16 ولما سمع إبليس هذا، تركني وصعد إلى أسفل السماء، وأخذ من الرب يميناً أن يكون له السلطان على جميع أموالي.

\par 17 وبعد أن استولى على السلطة، ذهب على الفور وأخذ كل ثروتي

\chapter{3}

\par 1 لأنه كان لي مئة وثلاثون ألف خروف، ومن هذه خصصت سبعة آلاف لكسوة الأيتام والأرامل والمساكين والمرضى

\par 2 كان لدي قطيع من ثمانمائة كلب يحرس أغنامي، بالإضافة إلى مائتي كلب يحرسان منزلي

\par 3 وكان لديّ تسعة طواحين تعمل للمدينة بأكملها، وسفن لنقل البضائع، وأُرسِلها إلى كل مدينة وقرى للضعفاء والمرضى والمحتاجين

\par 4 وكان لدي ثلاثمائة وأربعون ألف حمار بدوي، فخصصت منها خمسمائة، وأمرت ببيع ذريتها وتوزيع ثمنها على الفقراء والمساكين

\par 5 لأنه من جميع الأراضي جاء الفقراء لمقابلتي.

\par 6 "فإن أبواب بيتي الأربعة كانت مفتوحة، وكان على كل باب حارس لينظر: هل يأتي أحد يطلب صدقة، وهل يراني جالساً على أحد الأبواب فيخرجوا من الآخر ويأخذوا ما يحتاجون إليه."

\par 7 كان لدي أيضًا ثلاثون طاولة ثابتة مُجهزة في جميع الأوقات للغرباء فقط، وكان لدي أيضًا اثنتي عشرة طاولة مُجهزة للأرامل

\par 8 وإذا جاء أحد يطلب صدقة، وجد على مائدتي طعامًا يأخذ منه كل ما يحتاجه، ولم أرد أحدًا ليخرج من بابي ومعدته فارغة

\par 9 كان لدي أيضًا ثلاثة آلاف وخمسمائة نير من الثيران، واخترت من بين هذه الخمسمائة وجعلتها تعتني بالحرث

\par 10 وباستخدام هذه الأموال، كنتُ قد أنجزتُ جميع الأعمال في كل حقل، وكان من المقرر أن يتولى أولئك الذين سيتولون المسؤولية عنه، وكنتُ أخصص دخل محاصيلهم للفقراء على مائدتهم

\par 11 كان لدي أيضًا خمسون مخبزًا كنت أرسل منها [الخبز] إلى مائدة الفقراء

\par 12 وقد اخترتُ عبيدًا لخدمتهم.

\par 13 وكان هناك أيضًا بعض الغرباء الذين رأوا حسن نيتي، وأرادوا أن يعملوا بأنفسهم كنادلين.

\par 14 وجاء آخرون، وهم في محنة وغير قادرين على الحصول على لقمة العيش، بطلب قائلين:

\par 15 نسألك، بما أننا أيضًا نستطيع أن نشغل وظيفة الخدام (الشمامسة) وليس لدينا أي ممتلكات، أن ترحمنا وتدفع لنا نقودًا حتى نتمكن من الذهاب إلى المدن الكبرى وبيع البضائع

\par 16 ونعطي ما فائض ربحنا مساعدةً للفقراء، ثم نرد إليك مالك

\par 17 ولما سمعت هذا فرحت أن يأخذوا مني هذا كله من أجل أعمال البر للفقراء.

\par 18 وبقلب راغب، أعطيتهم ما أرادوا، وقبلت تعهدهم المكتوب، لكنني لم أقبل أي ضمان آخر منهم سوى الوثيقة المكتوبة

\par 19 وسافروا إلى الخارج وأعطوا فقراء الوقت بقدر ما نجحوا

\par 20 لكن في كثير من الأحيان، كانت بعض بضائعهم تضيع على الطريق أو في البحر، أو كانوا يتعرضون للسرقة.

\par 21 ثم يأتون ويقولون: "نرجوك، أن تتصرف بكرم تجاهنا حتى نرى كيف يمكننا أن نعيد إليك ما لديك".

\par 22 ولما سمعتُ ذلك، تعاطفتُ معهم، وسلمتُ لهم سنداتهم، وبعد أن قرأتُها أمامهم مرارًا، مزقتُها وأخليتُ سبيلهم من دينهم قائلًا لهم:

\par 23 «ما خصصته للفقراء، لن آخذه منكم».

\par 24 ولذلك لم أقبل شيئًا من مديني.

\par 25 وإذ جاءني رجل طيب القلب يقول: لا أحتاج إلى أن أكون عاملاً مأجوراً للفقراء.

\par 26 ولكني أريد أن أخدم المحتاجين على مائدتك، فوافق على العمل، وأكل نصيبه

\par 27 فأعطيته أجره مع ذلك، وذهبت إلى منزلي فرحًا.

\par 28 وعندما لم يرغب في أخذه، أجبرته على ذلك قائلاً: "أنا أعلم أنك رجل مجتهد يبحث عن أجره وينتظره، ويجب أن تأخذه".

\par 29 لم أؤجل قط دفع أجرة أجير أو غيره، ولا حجبت في بيتي ليلة واحدة أجرته المستحقة

\par 30 أشار أولئك الذين حلَبوا الأبقار والنعاج إلى المارة بأن عليهم أن يأخذوا حصتهم

\par 31 لأن الحليب تدفق بغزارة حتى أنه تجلط وتحول إلى زبدة على التلال وعلى جانب الطريق؛ وعلى الصخور والتلال رقدت الماشية التي ولدت ذريتها

\par 32 لأن خدامي قد ملّوا من حفظ طعام الأرامل والفقراء وتقطيعه إلى قطع صغيرة

\par 33 لأنهم كانوا يلعنون ويقولون: "ليتنا أخذنا من جسده حتى نشبع"، مع أنني كنت لطيفًا جدًا معهم،

\par 34 كان لدي أيضًا ست قيثارات [وستة عبيد ليعزفوا عليها]، بالإضافة إلى قيثارة، وهي آلة موسيقية ذات عشر نغمات، وكنت أعزف عليها خلال النهار

\par 35 وأخذتُ القيثارة، فأجابت الأرامل بعد وجباتهن.

\par 36 وبالآلة الموسيقية ذكّرتهم بالله لكي يحمدوا الرب.

\par 37 وعندما كانت إمائي تتمتم، كنت آخذ الآلات الموسيقية وأعزف عليها بقدر ما كن يفعلن مقابل أجورهن، وأريحهن من تعبهن وتنهداتهن

\chapter{4}

\par 1 وبعد أن تولوا مسؤولية الخدمة، تناولوا وجباتهم كل يوم مع أخواتهم الثلاث بدءًا من الأخ الأكبر، وأقاموا وليمة

\par 2 فقمت في الصباح وقدمت لهم خمسين كبشًا وتسعة عشر خروفًا ذبيحة خطية، وما فضل فقد قدسه للفقراء

\par 3 وقلت لهم: "خذوا هذه البقايا وصلوا من أجل أطفالي."

\par 4 لعلّ أبنائي أخطأوا أمام الربّ، قائلين بغطرسة: نحن أبناء هذا الغنيّ. لنا كلّ هذه الخيرات، فلماذا نكون عبيدًا للفقراء؟

\par 5 "وإذا تكلموا هكذا بروح متكبرة ربما أثاروا غضب الله، لأن الكبرياء المتعالي هو رجس أمام الرب."

\par 6 فأحضرت ثيرانًا كذبيحة للكاهن على المذبح قائلًا: "لا يفكر أبنائي بالشر تجاه الله في قلوبهم".

\par 7 بينما كنت أعيش بهذه الطريقة، لم يستطع المُغوي أن يتحمل رؤية الخير [الذي فعلته]، وطالب بحرب الله ضدي

\par 8 وهاجمني بقسوة.

\par 9 أحرق أولًا عددًا كبيرًا من الأغنام، ثم الإبل، ثم أحرق الماشية وجميع قطعاني؛ أو تم الاستيلاء عليها ليس فقط من قبل الأعداء، ولكن أيضًا من قبل أولئك الذين تلقوا منافع مني

\par 10 وجاء الرعاة وأخبروني بذلك.

\par 11 ولكن عندما سمعت ذلك، حمدت الله ولم أجدف.

\par 12 وعندما علم المُغوي بقوتي، دبر أمورًا جديدة ضدي

\par 13 تنكر في زي ملك فارس وحاصر مدينتي، وبعد أن قتل كل من فيها، خاطبهم بخبث قائلًا بلغة متباهية:

\par 14 «هذا الرجل أيوب الذي استحوذ على جميع خيرات الأرض ولم يترك شيئًا للآخرين، فقد دمر وهدم هيكل الله

\par 15 لذلك سأكافئه على ما فعله ببيت الإله العظيم

\par 16 الآن تعال معي وسننهب كل ما تبقى في منزله

\par 17 فأجابوا وقالوا له: «له سبعة بنين وثلاث بنات

\par 18 احذروا لئلا يفروا إلى بلاد أخرى، فيصبحوا طغاة علينا، ثم يأتون إلينا بالقوة ويقتلوننا

\par 19 فقال: لا تخافوا البتة. لقد أهلكت قطعانه وثروته بالنار، وسبيت الباقي، وها أنا أقتل أولاده

\par 20 وبعد أن قال هذا، ذهب وألقى المنزل على أولادي وقتلهم

\par 21 ولما رأى مواطنو بلدي أن ما قاله قد صار صحيحًا، جاءوا وطاردوني، ونهبوا مني كل ما كان في بيتي

\par 22 ورأيت بأم عيني نهب بيتي، وجلس على مائدتي وعلى أرائكي رجال بلا ثقافة ولا شرف، ولم أستطع الاعتراض عليهم

\par 23 لأني كنت منهكة كامرأة مرتخية من كثرة الآلام، متذكرة بشكل رئيسي أن الرب قد تنبأ لي بهذه الحرب من خلال ملاكه

\par 24 وأصبحتُ كمن يرى البحر الهائج والرياح العاتية، وحمولة السفينة في وسط المحيط ثقيلة جدًا، فيلقي بحمله في البحر قائلًا:

\par 25 «أريد تدمير كل هذا فقط لكي أصل إلى المدينة بأمان، حتى أتمكن من جني ثمار السفينة التي تم إنقاذها وأفضل ما لديّ.»

\par 26 هكذا دبّرتُ أموري.

\par 27 ولكن جاء رسول آخر وأعلن لي هلاك أبنائي، فارتعدتُ رعبًا

\par 28 فمزقتُ ثيابي وقلتُ: الربُّ أعطى، الربُّ أخذ. وكما شاء الربُّ، هكذا صار. فليتبارك اسم الرب.

\chapter{5}

\par 1 ولما رأى الشيطان أنه لا يستطيع أن يجعلني أشعر باليأس، ذهب وطلب جسدي من الرب لكي ينزل بي الوباء، لأن الشرير لم يستطع أن يتحمل صبري

\par 2 ثم أسلمني الرب بين يديه ليستخدم جسدي كما يشاء، لكنه لم يمنحه أي سلطة على روحي

\par 3 وجاء إليّ وأنا جالسة على عرشي، لا أزال حزينة على أطفالي

\par 4 وشبه إعصارًا عظيمًا، وقلب عرشي وألقى بي على الأرض

\par 5 وظللتُ مستلقيًا على الأرض لمدة ثلاث ساعات، وضربني بضربة قوية من أعلى رأسي إلى أصابع قدمي

\par 6 وغادرت المدينة في رعب شديد وحزن شديد، وجلست على مزبلة وجسدي قد أكله الدود

\par 7 وأبللت الأرض برطوبة جسدي المتألم، لأن المادة تدفقت من جسدي، وغطته ديدان كثيرة

\par 8 وعندما زحفت دودة واحدة من جسدي، أعدتها قائلة: "ابقَ في المكان الذي وُضعتَ فيه حتى يأمرك الذي أرسلك إلى مكان آخر."

\par 9 وهكذا صبرت لعدة سنوات، جالسًا على تلة روث خارج المدينة، مصابًا بالطاعون

\par 10 ورأيت بأم عيني أطفالي الذين تمنيتهم ​​[يحملهم الملائكة إلى السماء]

\par 11 وزوجتي المتواضعة التي أُحضرت إلى حجرة زفافها في رفاهية عظيمة ومع حراس شخصيين من حاملي الرماح. رأيتها تقوم بعمل حامل الماء كعبد في منزل رجل عادي من أجل كسب بعض الخبز وإحضاره لي

\par 12 وفي حزني الشديد قلت: "يا ليت حكام المدينة المتبجحين هؤلاء، الذين لم أظن أنهم مساوون لكلابي الراعية، يستخدمون زوجتي الآن كخادمة!"

\par 13 وبعد ذلك استعدتُ شجاعتي مرة أخرى.

\par 14 ومع ذلك، فقد حجبوا بعد ذلك حتى الخبز ليكون له قوت خاص به فقط

\par 15 لكنها أخذته وقسمته بيني وبينها، وقالت بحزن: "ويل لي! الآن لن يستطيع أن يأكل الخبز، ولا يستطيع أن يذهب إلى السوق ليطلب خبزًا من بائعي الخبز ليحضره ليأكله."

\par 16 وعندما علم الشيطان بذلك، اتخذ شكل بائع خبز، وكان الأمر كما لو أن زوجتي قابلته بالصدفة وطلبت منه خبزًا ظانةً أنه من هذا النوع من الرجال

\par 17 فقال لها الشيطان: "أعطني القيمة، ثم خذي ما تريدين."

\par 18 فأجابت قائلة: من أين لي بالمال؟ ألا تعلم ما أصابني من مصيبة؟ إن كنت تشفق فأرِني إياه، وإن لم تكن كذلك، فسوف ترى

\par 19 فأجاب قائلًا: «لو لم تستحق هذا البلاء لما عانيت كل هذا

\par 20 والآن إن لم يكن في يدك قطعة فضة فأعطني شعر رأسك وخذ بدله ثلاثة أرغفة خبز، لكي تعيش هناك ثلاثة أيام.

\par 21 ثم قالت في نفسها: "ما شعر رأسي بالمقارنة مع زوجي الجائع"

\par 22 وبعد أن فكرت في الأمر، قالت له: "قم وقص شعري".

\par 23 ثم أخذ مقصًا وحلق شعر رأسها أمام الجميع، وأعطاها ثلاثة أرغفة خبز

\par 24 ثم أخذتهما وأحضرتهما إليّ. وذهب الشيطان خلفها في الطريق، مختبئًا في طريقه، وقلق قلبها كثيرًا

\chapter{6}

\par 1 وفي الحال اقتربت مني زوجتي وهي تبكي بصوت عالٍ وتقول: "يا أيوب! يا أيوب! إلى متى ستجلس على كومة الزبل خارج المدينة، تفكر قليلاً وتنتظر خلاصك المرجو!"

\par 2 وكنتُ أتجول من مكان إلى آخر، أتجول كخادم أجير، وها هي الذاكرة قد ماتت بالفعل من على الأرض

\par 3 وأبنائي وبناتي الذين حملتهم على صدري، والتعب والآلام التي تحملتها، ذهبت سدىً

\par 4 وأنت تجلس في حالة كريهة من الألم والديدان، تقضي الليالي في الهواء البارد

\par 5 وقد تحملت كل التجارب والمتاعب والآلام، ليلًا ونهارًا حتى نجحت في إحضار الخبز إليك

\par 6 لأن فائض خبزك لم يعد مسموحًا لي به؛ ولأنني بالكاد أستطيع أن آخذ طعامي وأقسمه بيننا، فكرت في قلبي أنه ليس من الصواب أن تتألم وتجوع من أجل الخبز

\par 7 وهكذا تجرأت على الذهاب إلى السوق دون خجل، وعندما قال لي بائع الخبز: "أعطني نقودًا وستحصل على خبز"، كشفت له عن حالتنا المزرية

\par 8 ثم سمعته يقول: "إذا لم يكن لديك مال، فناولني شعر رأسك، وخذ ثلاثة أرغفة من الخبز لتعيش عليها ثلاثة أيام".

\par 9 واستسلمتُ للظلم وقلتُ له: "قم واقطع شعري!" فقام وقصَّ بالمقص شعر رأسي في السوق، خجلاً، بينما وقف الحشد متعجبًا

\par 10 من ذا الذي لا يندهش إذًا ويقول: "أهذه سيتيس، زوجة أيوب، التي كان لها أربعة عشر ستارة لتغطية غرفة جلوسها الداخلية، وأبواب داخل أبواب، فكان يُكرم كثيرًا من يُحضر إليها، والآن ها هي تقايض شعرها بالخبز!

\par 11 من كانت تملك جمالًا محملة بالبضائع، وتُجلب إلى الأراضي النائية للفقراء، والآن تبيع شعرها مقابل الخبز!

\par 12 انظروا إلى تلك التي كان لها سبع موائد ثابتة في منزلها يأكل عليها كل فقير وكل غريب، والآن تبيع شعرها مقابل الخبز!

\par 13 انظروا إلى تلك التي كان معها مِغسلٌ من ذهبٍ وفضةٍ لتغسل به قدميها، وهي الآن تمشي على الأرض وتبيع شعرها مقابل خبز!

\par 14 انظروا إليها التي كانت ثيابها مصنوعة من القماش المنسوج بالذهب، والآن تستبدل شعرها بالخبز!

\par 15 انظروا إلى تلك التي كانت لها أسِرّة من ذهب وفضة، والآن تبيع شعرها مقابل الخبز!

\par 16 باختصار، يا أيوب، بعد كل ما قيل لي، أقول لك الآن بكلمة واحدة:

\par 17 «بما أن ضعف قلبي قد سحق عظامي، فقم إذن وخذ أرغفة الخبز هذه واستمتع بها، ثم تكلم بكلمة ضد الرب ومت!»

\par 18 فأنا أيضًا أرغب في استبدال خمول الموت بقوت جسدي

\par 19 فأجبتها: "ها أنا ذا مصاب بالطاعون منذ سبع سنوات، وقد صمدتُ أمام ديدان جسدي، ولم أثقل روحي بكل هذه الآلام

\par 20 وأما الكلمة التي تقولها: «تكلم بكلمة ضد الله ومت!»، فسأتحمل معك الشر الذي تراه، ودعنا نتحمل دمار كل ما لدينا

\par 21 ومع ذلك، فأنت تريد منا أن نقول كلمة ضد الله وأن يتم استبداله ببلوتو العظيم [إله العالم السفلي].

\par 22 لماذا لا تتذكر تلك الخيرات العظيمة التي امتلكناها؟ إذا كانت هذه الخيرات تأتي من أراضي الرب، ألا ينبغي لنا أيضًا أن نتحمل الشرور ونكون متعاليين في كل شيء حتى يرحمنا الرب مرة أخرى ويشفق علينا؟

\par 23 ألا ترى المُغوي يقف خلفك ويُربك أفكارك لكي تُضلني؟

\par 24 والتفت إلى الشيطان وقال: "لماذا لا تأتي إليّ علانية؟ توقف عن إخفاء نفسك أيها البائس،

\par 25 هل يُظهر الأسد قوته في قفص ابن عرس أم يطير الطائر في السلة؟ أقول لك الآن: اذهب وشن حربك ضدي

\par 26 ثم انصرف من خلف زوجتي ووقف أمامي يبكي وقال: هوذا يا أيوب، أستسلم وأفسح لك الطريق، أنت لست سوى جسد وأنا روح

\par 27 أنت مصاب بالطاعون، لكنني في ورطة كبيرة.

\par 28 فإني أشبه مصارعاً يتنافس مع مصارع آخر هزم خصمه في قتال فردي وغطاه بالتراب وكسر كل عضو من أعضائه، في حين أن الآخر الذي يرقد في الأسفل، بعد أن أظهر شجاعته، يصدر أصوات انتصار تشهد على تفوقه.

\par 29 وهكذا أنت يا أيوب، مُصابٌ بالوباء والألم، ومع ذلك فقد حققت النصر في المصارعة معي، وها أنا أستسلم لك

\par 30 ثم تركني خجولًا.

\par 31 الآن يا أبنائي، أظهروا أنتم أيضًا قلبًا ثابتًا في كل الشرور التي تصيبكم، لأن ثبات القلب أعظم من كل شيء

\chapter{7}

\par 1 في ذلك الوقت سمع الملوك بما حدث لي، فقاموا وجاءوا إليّ كلٌّ من أرضه ليزورني ويعزيني

\par 2 ولما اقتربوا مني صرخوا بصوت عظيم ومزق كل واحد ثيابه

\par 3 وبعد أن سجدوا ووضعوا رؤوسهم على الأرض، جلسوا بجانبي سبعة أيام وسبع ليال، ولم يتكلم أحد منهم بكلمة.

\par 4 كانوا أربعة: أليبلاس، ملك تيمان، وبلد، وصوفر، وإليلهو

\par 5 ولما جلسوا تحدثوا عما حدث لي

\par 6 عندما أتوا إليّ لأول مرة وأريتهم أحجاري الثمينة، اندهشوا وقالوا:

\par 7 لو جُمعت كل ممتلكاتنا نحن الملوك الثلاثة في واحدة، لما وصل ذلك إلى الأحجار الكريمة لمملكة يوباب (التاج). لأنك أشرف من جميع أهل المشرق

\par 8 ولما جاءوا الآن إلى أرض عوص لزيارتي، سألوا في المدينة: "أين يوباب، حاكم هذه الأرض كلها؟"

\par 9 وأخبروهم عني قائلين: «إنه جالس على كومة المزبلة خارج المدينة، لأنه لم يدخل المدينة منذ سبع سنين».

\par 10 ثم سألوني مرة أخرى عن ممتلكاتي، وكشف لهم كل ما حدث لي

\par 11 ولما علموا ذلك، خرجوا مع السكان خارج المدينة، وأشار إليّ رفاقي لديهم

\par 12 فاعترضوا وقالوا: «حقًا، هذا ليس يوبابًا».

\par 13 وفيما هم مترددون، قال أليفاز ملك تيمان: "هلم نتقدم وننظر".

\par 14 وعندما اقتربوا تذكرتهم، وبكيت كثيرًا عندما علمت بالغرض من رحلتهم

\par 15 وألقيتُ التراب على رأسي، وبينما كنتُ أهز رأسي، كشفتُ لهم أنني [أيوب].

\par 16 وعندما رأوني أهز رأسي، ألقوا بأنفسهم على الأرض، وقد غلب عليهم الانفعال

\par 17 وبينما كان جيوشهم واقفين، رأيت الملوك الثلاثة مستلقين على الأرض لمدة ثلاث ساعات كالأموات

\par 18 ثم قاموا وقال بعضهم لبعض: لا نصدق أن هذا يوباب

\par 19 وأخيرًا، بعد أن استفسروا لمدة سبعة أيام عن كل ما يخصني، وبحثوا عن أغنامي وممتلكاتي الأخرى، قالوا:

\par 20 «ألا نعلم كم من البضائع أرسلها إلى المدن والقرى المجاورة ليوزعها على الفقراء، عدا كل ما وزعه داخل بيته؟ فكيف إذًا وقع في مثل هذه الحالة من الهلاك والبؤس!»

\par 21 وبعد الأيام السبعة قال إيليهو للملوك: «هلموا نتقدم ونفحصه بدقة، هل هو يوباب حقًا أم لا».

\par 22 ولما كانوا على بُعد أقل من نصف ميل (ملعب) من جثته النتنة، نهضوا وتقدموا نحوي، حاملين العطر في أيديهم، بينما كان جنودهم يسيرون معهم وينثرون حولهم بخورًا عطرًا حتى يتمكنوا من الاقتراب مني

\par 23 وبعد أن قضوا ثلاث ساعات على هذا النحو، وهم يغطون الطريق بالرائحة، اقتربوا

\par 24 فبدأ أليفاز وقال: "أنت حقًا يا أيوب رفيقنا في الملك، أنت الذي ملك المجد العظيم

\par 25 هل أنت الذي أشرق ذات يوم كشمس النهار على الأرض كلها؟ هل أنت الذي شبَّهتَ ذات يوم القمر والنجوم المتألقة طوال الليل؟

\par 26 فأجبته وقلت: "أنا هو"، وعندها بكى الجميع وناحوا، وغنوا أغنية رثاء ملكية، وانضم إليهم جيشهم بأكمله في جوقة

\par 27 وقال لي أليفاز مرة أخرى: "أنت الذي أمرت بإعطاء سبعة آلاف شاة لكسوة الفقراء؟ إلى أين ذهب مجد عرشك إذن؟"

\par 28 هل أنت الذي أمر بثلاثة آلاف رأس من الماشية بحراثة الحقل للفقراء؟ إذا ذبل، فقد ذهب مجدك!

\par 29 أأنت الذي كان له أسِرّة من ذهب، والآن تجلس على تلة روث ["إلى أين ذهب مجدك إذن!"]

\par 30 أأنت الذي جهز ستين مائدة للفقراء؟ أأنت الذي كان لديه مباخر للعطور الفاخرة المصنوعة من الأحجار الكريمة، والآن أنت في حالة كريهة الرائحة؟ إلى أين ذهب مجدك إذن!

\par 31 هل أنت الذي وضع شمعدانات ذهبية على حوامل فضية؟ والآن يجب أن تتوق إلى بريق القمر الطبيعي ["إلى أين ذهب مجدك إذن!"]

\par 32 أأنتِ من كان لديكِ مرهم مصنوع من أطياب اللبان، والآن أنتِ في حالة من الاشمئزاز! ["إلى أين ذهب مجدكِ إذن!"]

\par 33 أأنت الذي سخرتَ من المخطئين والخطاة، والآن أصبحتَ أضحوكة للجميع! ["إلى أين ذهب مجدك إذن؟"]

\par 34 ولما بكى أليفاز ونوح طويلاً، وانضم إليه الآخرون جميعًا، حتى كان الاضطراب عظيمًا جدًا، قلت لهم:

\par 35 اصمت، فأريك عرشي ومجد بهائه. مجدي يكون إلى الأبد

\par 36 سيفنى العالم كله، ويتلاشى مجده، وكل من يتمسك به سيبقى في الأسفل، لكن عرشي في العالم العلوي، ومجده وبهاؤه سيكونان عن يمين المخلص في السماوات

\par 37 عرشي موجود في حياة "القديسين" ومجده في العالم الخالد

\par 38 لأن الأنهار ستجف، وسينحدر كبرياؤها إلى عمق الهاوية، أما جداول أرضي التي أقيم عليها عرشي، فلن تجف، بل ستبقى قوية لا تنكسر

\par 39 يزول الملوك ويزول الحكام، ويكون مجدهم وكبرياؤهم كظل في مرآة، لكن مملكتي تدوم إلى الأبد، ومجدها وجمالها في عربة أبي

\chapter{8}

\par 1 ولما كلمتهم هكذا، غضب إيهيفاز وقال للأصدقاء الآخرين: "لماذا جئنا إلى هنا مع جيوشنا لنعزيه؟ هوذا يوبخنا. فلنرجع إلى بلادنا."

\par 2 يجلس هذا الرجل هنا في بؤس، متآكلًا بالديدان، وسط حالة لا تطاق من التعفن، ومع ذلك فهو يتحدى إنقاذه: "ستهلك الممالك وحكامها، ولكن مملكتي، كما يقول، ستبقى إلى الأبد".

\par 3 فنهض أليفاز في اضطراب شديد، وابتعد عنهم في غضب شديد، وقال: "أنا ذاهب من هنا. لقد جئنا بالفعل لنعزيه، لكنه أعلن الحرب علينا بسبب جيوشنا".

\par 4 ثم أمسكه بلداد بيده وقال: "لا ينبغي لأحد أن يتكلم هكذا مع رجل مصاب، وخاصة مع شخص مصاب بكل هذه الضربات

\par 5 هوذا نحن، ونحن في صحة جيدة، لم نجرؤ على الاقتراب منه بسبب الرائحة الكريهة، إلا بمساعدة الكثير من الرائحة العطرة. أما أنت يا أليفاز، فقد نسيت كل هذا

\par 6 دعوني أتحدث بصراحة. دعونا نكون كرماء ونعرف ما هو السبب. ألا يجب عليه عند تذكر أيام سعادته السابقة أن يُصاب بالجنون؟

\par 7 من لا ينبغي أن يكون في حيرة من أمره وهو يرى نفسه ينزلق إلى هذه المتاعب والأوبئة؟ ولكن دعني أقترب منه لأعرف سبب انزلاقه إلى هذه الحال

\par 8 فقام بلداد واقترب مني قائلًا: "أنت أيوب؟" فقال: "هل لا يزال قلبك سليمًا؟"

\par 9 فقلت: "لم أتمسك بالأشياء الأرضية، لأن الأرض بكل ما يسكنها غير مستقرة. لكن قلبي متمسك بالسماء، لأنه لا يوجد اضطراب في السماء".

\par 10 ثم ردّ بلداد وقال: "نعلم أن الأرض غير مستقرة، لأنها تتغير بتغير الفصول. أحيانًا تكون في حالة سلام، وأحيانًا تكون في حالة حرب. أما السماء فنسمع أنها ثابتة تمامًا

\par 11 لكن هل أنت في حالة هدوء حقًا؟ لذلك دعني أسأل وأتحدث، وعندما تجيبني على كلمتي الأولى، سيكون لدي سؤال ثانٍ لأطرحه، وإذا أجبت مرة أخرى بكلمات جيدة، فسيكون من الواضح أن قلبك لم يكن مختلًا

\par 12 فقلت: «على ماذا تَرجُون؟» فقلت: «على الله الحي».

\par 13 فقال لي: "من حرمك من كل ما كنت تملك، ومن أصابك بهذه الآفات؟" فقلت: "الله".

\par 14 فقال: «إذا كنت لا تزال تضع رجاءك على الله، فكيف يخطئ في الحكم، وقد جلب عليك هذه الأوبئة والمصائب، وأخذ منك كل ما تملك؟»

\par 15 وبما أنه أخذ هذه، فمن الواضح أنه لم يُعطِك شيئًا. لن يُخزي أي ملك جنديه الذي خدمه جيدًا كحارس شخصي

\par 16 [فأجبتُ قائلاً]: «من يفهم أعماق الرب وحكمته حتى يستطيع أن يتهم الله بالظلم؟»

\par 17 [وقال بلداد]: «أجبني يا أيوب على هذا. أقول لك أيضًا: إن كنت في حالة من الهدوء العقلي، فعلمني إن كانت لديك حكمة

\par 18 لماذا نرى الشمس تشرق في الشرق وتغرب في الغرب؟ ومرة ​​أخرى عندما تشرق في الصباح نجدها تشرق في الشرق؟ أخبرني برأيك في هذا

\par 19 ثم قلت: لماذا أخون (أثرثر) بأسرار الله العظيمة، ولماذا يتعثر فمي في الكشف عن أشياء تخص السيد أبدًا!

\par 20 من نحن حتى نتطفل على أمور تتعلق بالعالم العلوي ونحن من لحم ودم، بل من تراب ورماد!

\par 21 لكي تعلم أن قلبي سليم، اسمع ما أسألك عنه:

\par 22 "من المعدة يأتي الطعام، والماء الذي تشربه من الفم، ثم يتدفق عبر نفس الحلق، وعندما ينزل الاثنان ليصبحا فضلات، ينفصلان مرة أخرى؛ من الذي يقوم بهذا الانفصال؟"

\par 23 فقال بلداد: "لا أعرف". فرددتُ عليه وقلتُ: "إذا كنتَ لا تفهم حتى مخارج الجسد، فكيف يمكنكَ فهم الدوائر السماوية؟"

\par 24 ثم أجاب صوفر وقال: "نحن لا نسأل عن أحوالنا الخاصة، لكننا نرغب في معرفة ما إذا كنت في حالة سليمة، وها نحن نرى أن عقلك لم يتزعزع

\par 25 ماذا تريد الآن أن نفعل لك؟ انظر، لقد أتينا إلى هنا وأحضرنا أطباء ثلاثة ملوك، وإذا أردت، فيمكنك الشفاء منهم

\par 26 لكني أجبت وقلت: "شفائي وشفائي من الله، صانع الأطباء".

\chapter{9}

\par 1 ولما قلت لهم هذا، إذا بزوجتي سيتيس قد أتت راكضة، مرتدية ثيابًا رثة، من خدمة سيدها الذي كانت تعمل لديه كعبدة، على الرغم من أنها مُنعت من المغادرة، خشية أن يأخذها الملوك أسيرةً عند رؤيتها

\par 2 ولما أتت، ألقت بنفسها ساجدة عند أقدامهم، تبكي وتقول: "تذكروا يا أليفاز وأصدقائي الآخرين، كيف كنت معكم من قبل، وكيف تغيرت، وكيف أرتدي الآن ملابسي لاستقبالكم"

\par 3 فانفجر الملوك بكاءً شديدًا، وصمتوا في حيرة مضاعفة. فأخذ أليفاز رداءه الأرجواني وألقاه عليها لتلف نفسها به

\par 4 لكنها سألته قائلة: "أطلب منكم، يا سادة، أن تأمروا جنودكم بالحفر بين أنقاض منزلنا الذي سقط على أطفالي، حتى يمكن نقل عظامهم في حالة جيدة إلى القبور

\par 5 نظرًا لأننا، بسبب سوء حظنا، لا نملك أي قوة على الإطلاق، فقد نتمكن على الأقل من رؤية عظامهم

\par 6 هل لي، كوحش، أن أمتلك شعور الأمومة الذي تشعر به الوحوش البرية، حتى يموت أطفالي العشرة في يوم واحد، ولا أستطيع أن أدفن أحدهم دفنًا لائقًا؟

\par 7 وأمر الملوك بحفر أنقاض بيتي. لكنني منعتهم من ذلك، وأنقذت

\par 8 «لا تُضِلّوا في العناء عبثًا؛ لأن أبنائي لن يُوجدوا، لأنهم في حِفظة خالقهم وحاكمهم».

\par 9 فأجاب الملوك وقالوا: "من ينكر أنه مجنون ويهذي؟"

\par 10 فبينما نرغب في استعادة عظام أبنائه، ينهانا عن ذلك قائلاً: «لقد أُخذت ووُضعت في حراسة خالقها». فأثبت لنا الحقيقة.

\par 11 فقلت لهم: «أقيموني لأقوم»، فحملوني رافعين ذراعي من جانبي

\par 12 ووقفتُ منتصبًا، ونطقتُ أولًا بحمد الله، وبعد الصلاة قلتُ لهم: "انظروا بأعينكم إلى الشرق".

\par 13 ونظروا ورأوا أبنائي يحملون أكاليل بالقرب من مجد الملك، حاكم السماء

\par 14 ولما رأت زوجتي سيتيس ذلك، سقطت على الأرض وسجدت لله قائلة: "الآن أعلم أن ذكراي تبقى مع الرب".

\par 15 وبعد أن قالت هذا، وحل المساء، ذهبت إلى المدينة، عائدة إلى سيدها الذي كانت تخدمه أمةً، وألقت نفسها عند مذود الماشية وماتت هناك من التعب

\par 16 ولما بحث عنها سيدها المستبد ولم يجدها، جاء إلى حظيرة قطعانه، وهناك رآها ممددة على المذود ميتة، بينما كانت جميع الحيوانات حولها تبكي عليها

\par 17 وكل من رآها بكى وناح، وامتد الصراخ في جميع أنحاء المدينة

\par 18 فأنزلها الشعب وكفنوها ودفنوها عند البيت الذي سقط على أولادها

\par 19 وأقام فقراء المدينة حزنًا عظيمًا عليها وقالوا: "انظروا إلى هذه السيتيس التي لا يوجد مثلها في النبل والمجد في أي امرأة. يا للأسف! لم تُوجد جديرة بقبر لائق!"

\par 20 ستجد النشيد الخاص بها في السجل.

\chapter{10}

\par 1 لكن أليفاز والذين معه تعجبوا من هذه الأمور، وجلسوا معي وأجابوني وتكلموا عليّ بكلام فاحش سبعة وعشرين يومًا

\par 2 كرروا مرارًا وتكرارًا أنني عانيت بجدارة بسبب ارتكابي العديد من الخطايا، وأنه لم يعد هناك أمل لي، لكنني رددت على هؤلاء الرجال بحماسة من الجدل بنفسي

\par 3 فقاموا بغضب، مستعدين للانفصال بروح غاضبة. لكن إيليهو استدرجهم للبقاء قليلًا حتى يُريهم ما هو

\par 4 قال: «لأنك قضيت أيامًا كثيرة تسمح لأيوب أن يفتخر بأنه عادل. لكنني لن أتحمل ذلك بعد الآن

\par 5 لأني منذ البدء كنت أبكي عليه، متذكرًا سعادته السابقة. أما الآن فهو يتكلم بتباهي، وفي كبرياء متغطرس يقول إن عرشه في السماء

\par 6 لذلك، اسمعني، وسأخبرك ما هو سبب مصيره

\par 7 ثم، وقد تشبع بروح الشيطان، تكلم إيليهو بكلمات قاسية مكتوبة في السجلات التي على يسار إيليهو

\par 8 وبعد أن انتهى، ظهر لي الله في عاصفة وفي سحاب، وتكلم لومًا على إيليهو، وأراني أن الذي تكلم لم يكن إنسانًا، بل وحشًا بريًا

\par 9 ولما فرغ الله من الكلام معي، تكلم الرب مع إليفاز قائلاً: «أنت وأصدقاؤك أخطأتم إذ لم تتكلموا بالحق في شأن عبدي أيوب.

\par 10 لذلك قم واجعله يقدم لك ذبيحة خطية لكي تُغفر خطاياك، لأنه لولا هو لكنت أهلكتك

\par 11 فأحضروا إليّ كل ما هو ذبيحة، فأخذته وقدمت لهم ذبيحة خطية، فقبلها الرب وغفر لهم ذنبهم

\par 12 ثم عندما رأى أليفاز وبلداد وصوفر أن الله قد غفر خطيئتهم بنعمته من خلال خادمه أيوب، لكنه لم يتنازل ويغفر لإيليهو، بدأ أليفاز في ترنيمة، بينما استجاب الآخرون، وانضم إليهم جنودهم وهم واقفون بجانب المذبح

\par 13 فقال أليفاز: "لقد رُفعت الخطيئة وزال ظلمنا؛

\par 14 وأما إيليهو الشرير فلا يكون له ذكر بين الأحياء، فقد انطفأ نوره وفقد ضوئه.

\par 15 سيُعلن مجد مصباحه عن نفسه له، لأنه ابن الظلمة لا ابن النور

\par 16 سيُعطيه حُرَّاسُ مكان الظلمة مجدهم وجمالهم نصيبًا؛ فقد زال ملكوته، وتآكل عرشه، وكرامة قامته في (شيول) الهاوية

\par 17 لأنه أحب جمال الحية وقشور (جلود) التنين، فمرارته وسمه ينتميان إلى الشمالي (زفوني = أفعى).

\par 18 لأنه لم يكن متمسكًا بالرب ولا يخافه، بل كان يبغض الذين اختارهم (يعرفهم).

\par 19 وهكذا نسيه الله، وتركه "القديسون"، فيكون غضبه وغضبه عليه خرابًا، ولن يكون في قلبه رحمة ولا سلام، لأنه كان على لسانه سم أفعى

\par 20 بار هو الرب، وأحكامه حق، لا تفضيل عنده على أحد، لأنه يحكم على الجميع على قدم المساواة

\par 21 هوذا الرب قادم! هوذا "القديسون" قد أُعدّوا: أكاليل وجوائز المنتصرين تسبقهم!

\par 22 ليفرح القديسون، ولتبتهج قلوبهم فرحًا، لأنهم سينالون المجد المُعد لهم

\par \textit{كورس.}

\par 23 غُفرت خطايانا، وطُهِّر ظلمنا، لكن إيليهو ليس له ذكر بين الأحياء

\par 24 بعد أن انتهى أليفاز من الترنيمة، نهضنا وعدنا إلى المدينة، كلٌّ منا إلى البيت الذي كان يسكنه

\par 25 وأقام الناس لي وليمةً امتنانًا لله ورضًا عنه، وعاد إليّ جميع أصدقائي

\par 26 وكل من رآني في حالتي السابقة من السعادة سألني قائلين: "ما هذه الأشياء الثلاثة هنا بيننا؟"

\chapter{11}

\par 1 ولكنني، إذ كنت أرغب في استئناف عملي الخيري تجاه الفقراء، سألتهم قائلاً:

\par 2 «أعطوني كل واحدٍ خروفًا لكسوة الفقراء في حالة عريهم، وأربعة دراخمات من الفضة أو الذهب»

\par 3 ثم بارك الرب كل ما بقي لي، وبعد أيام قليلة أصبحت غنيًا مرة أخرى في التجارة وفي الأغنام وفي كل الأشياء التي فقدتها، واستردت كل شيء ضعف العدد مرة أخرى.

\par 4 ثم تزوجتُ أمكم أيضًا، وأصبحتُ أبًا لكم العشرة بدلًا من الأطفال العشرة الذين ماتوا

\par 5 والآن يا أبنائي، دعوني أنصحكم: "ها أنا أموت. ستأخذون مكاني

\par 6 إنما لا تتركوا الرب. كونوا خيرين على الفقراء؛ لا تحتقروا الضعفاء. لا تتخذوا لكم نساءً من أجنبيات

\par 7 انظروا يا أبنائي، سأقسم بينكم ما أملك، حتى يكون لكل واحدٍ سلطان على ما يملكه، ويكون له كامل السلطة على فعل الخير بنصيبه

\par 8 وبعد أن قال هذا، أحضر جميع ممتلكاته وقسمها على أبنائه السبعة، ولكنه لم يُعطِ بناته شيئًا من ممتلكاته

\par 9 ثم قالوا لأبيهم: «سيدنا وأبانا! ألسنا أيضًا أولادك، فلماذا لا تعطينا أيضًا نصيبًا من أموالك؟»

\par 10 ثم قال أيوب لبناته: «لا تغضبن يا بناتي، لم أنسكن. هوذا قد حفظت لكنّ ملكًا خيرًا مما أخذه إخوتكن».

\par 11 ونادى ابنته التي كانت اسمها داي (يميما) وقال لها: "خذي هذا الخاتم المزدوج الذي يُستخدم كمفتاح واذهبي إلى بيت الكنز وأحضري لي الصندوق الذهبي، لأعطيكِ ممتلكاتكِ".

\par 12 فذهبت وأحضرته إليه، ففتحه وأخرج منه مناطق ذات ثلاثة أوتار لا يستطيع أحد أن يتكلم عنها

\par 13 لأنها لم تكن عملاً أرضيًا، بل كانت شرارات سماوية من الضوء تتلألأ من خلالها مثل أشعة الشمس

\par 14 وأعطى كل واحدة من بناته خيطًا وقال: «ضعي هذه كمناطق حولك لكي تحيط بك كل أيام حياتك وتمنحك كل خير».

\par 15 فقالت الابنة الأخرى التي اسمها كاسيا: "هل هذه هي الممتلكات التي تقولين إنها أفضل من ممتلكات إخوتنا؟ ماذا يمكننا أن نعيش عليه الآن؟"

\par 16 فقال لهم أبوهم: "ليس لديكم هنا ما يكفي للعيش فحسب، بل هذه الأشياء ستنقلكم إلى عالم أفضل للعيش فيه، في السماوات

\par 17 أم لا تعرفون يا أبنائي قيمة هذه الأشياء هنا؟ اسمعوا إذًا! لما حسبني الرب أهلاً أن يرحمني وينزع عني الأوبئة والديدان، دعاني وسلمني هذه الأوتار الثلاثة

\par 18 فقال لي: «قم وشد حقويك كما أطلب من الرجل وأخبرني».

\par 19 فأخذتها وشددتها حول حقوي، وفي الحال غادرت الديدان جسدي، وكذلك فعلت الأوبئة، وأخذ جسدي كله قوة جديدة من خلال الرب، وهكذا مررت، كما لو أنني لم أعاني قط

\par 20 ولكنني نسيت الآلام في قلبي أيضًا. ثم كلمني الرب بقدرته العظيمة وأراني كل ما كان وما سيكون

\par 21 والآن يا أبنائي، إذا حافظتم على هذه الأمور، فلن يكون هناك مؤامرة من جانب العدو ضدكم ولا نوايا شريرة في أذهانكم لأن هذه تعويذة (فيلاكتيريون) من الرب.

\par 22 انهضوا إذن وشدوا هذه حولكم قبل أن أموت حتى تتمكنوا من رؤية الملائكة قادمة عند فراقني، وحتى تتمكنوا من رؤية قوى الله بدهشة

\par 23 ثم نهضت التي كان اسمها داي (يميما) وحزمت نفسها، وغادرت جسدها على الفور، كما قال والدها، وارتدت قلبًا آخر، كما لو أنها لم تهتم أبدًا بالأمور الأرضية

\par 24 وغنّت ترانيم ملائكية بصوت الملائكة، وغنّت تسبيحًا ملائكيًا لله وهي ترقص

\par 25 ثم لبست الابنة الأخرى، كاسيا، المنطقة، وتغير قلبها، فلم تعد ترغب في الأشياء الدنيوية

\par 26 واتخذ فمها لهجة الحكام السماويين (الأرخونت) وغنّت ترانيم عمل المكان العالي، وإذا رغب أي شخص في معرفة عمل السماوات، فيمكنه أن يأخذ نظرة ثاقبة في ترانيم كاسيا

\par 27 ثم شددت الابنة الأخرى، التي تُدعى قرن أملثيا (كيرين هابوخ)، نفسها وتكلم فمها بلغة الأعالي؛ لأن قلبها قد تغير، إذ ارتفع فوق الأمور الدنيوية

\par 28 كانت تتحدث بلهجة الكروبيم، وتغني بتسبيح حاكم القوى الكونية (الفضائل) وتمجد مجدها (مجده).

\par 29 ومن يرغب في اتباع آثار "مجد الآب" سيجدها مكتوبة في صلوات قرن أملثيا

\chapter{12}

\par 1 بعد أن انتهى هؤلاء الثلاثة من غناء الترانيم، جلست أنا ناحور (نيروس) أخو أيوب بجانبه وهو مضطجع

\par 2 وسمعت عجائب بنات أخي الثلاث، واحدة تلو الأخرى وسط صمت رهيب

\par 3 وكتبتُ هذا الكتاب الذي يحتوي على الترانيم، باستثناء تراتيل وعلامات الكلمة [المقدسة]، لأن هذه كانت عظائم الله

\par 4 فاضطجع أيوب من المرض على فراشه، ولكن دون ألم أو معاناة، لأن ألمه لم يُسيطر عليه بسبب سحر المنطقة التي لفها حول نفسه

\par 5 ولكن بعد ثلاثة أيام رأى أيوب الملائكة القديسين يأتون لأخذ روحه، فقام على الفور وأخذ القيثارة وأعطاها لابنته داي (يميما).

\par 6 وأعطى كاسيا مبخرة (بعطر = كاسيا)، وأعطى بوق أملثيا (= موسيقى) دفًا ليباركا الملائكة القديسين الذين جاؤوا من أجل روحه

\par 7 فأخذوا هذه، وغنوا، وعزفوا على القيثارة، وسبحوا ومجدوا الله باللهجة المقدسة

\par 8 وبعد هذا جاء الجالس على المركبة العظيمة وقبل أيوب، بينما كانت بناته الثلاث ينظرن، لكن الأخريات لم يرين ذلك

\par 9 وأخذ نفس أيوب وصعد إلى فوق، وأخذها من ذراعها وحملها على المركبة، وانطلق نحو المشرق

\par 10 وأُحضِر جسده إلى القبر، بينما سارت بناته الثلاث في المقدمة، وقد وضعن مناطقهن وغنين ترانيم تسبيحًا لله.

\par 11 ثم أقام ناحور (نيريوس) أخاه وأبناؤه السبعة، مع بقية الشعب والفقراء والأيتام والضعفاء، حزنًا عظيمًا عليه، قائلين:

\par 12 «ويل لنا، لأنه اليوم قد سُلب منا قوة الضعفاء، ونور العميان، وأبو الأيتام؛

\par 13 لقد رُفع عن كاهل الغرباء، قائد الضالين، وسترة العراة، وترس الأرامل. من ذا الذي لا يبكي على رجل الله؟

\par 14 وبينما كانوا ينوحون بهذه الصورة وتلك، لم يسمحوا له بوضعه في القبر

\par 15 بعد ثلاثة أيام، مع ذلك، وُضع أخيرًا في القبر، كمن في سبات هنيء، وحصل على اسم الصالح (الجميل) الذي سيبقى مشهورًا عبر جميع أجيال العالم

\par 16 ترك سبعة أبناء وثلاث بنات، ولم توجد على الأرض بنات جميلات كبنات أيوب

\par 17 كان اسم أيوب سابقًا يوباب، ودعاه الرب أيوب

\par 18 لقد عاش قبل طاعونه خمسة وثمانين عامًا، وبعد الطاعون أخذ ضعف نصيبه من الكل؛ ومن ثم ضاعف سنين عمره أيضًا، أي 170 عامًا. وهكذا عاش إجمالًا 255 عامًا

\par 19 ورأى أبناءً من أبنائه إلى الجيل الرابع. مكتوب أنه سيقوم مع الذين يوقظهم الرب. لربنا بالمجد. آمين

\end{document}