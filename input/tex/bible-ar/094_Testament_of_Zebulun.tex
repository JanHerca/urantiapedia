\begin{document}

\title{وصية زبولون}

\chapter{1}

زبولون، الابن السادس ليعقوب وليئة. المخترع والمُحسن، ما تعلمه من المؤامرة ضد يوسف.

\par 1 نسخة أقوال زبولون التي أوصى بها أبنائه قبل وفاته في السنة الرابعة عشرة بعد المائة من حياته، بعد عامين من وفاة يوسف

\par 2 فقال لهم: اسمعوا لي يا بني زبولون، وأصغوا إلى كلام أبيكم

\par 3 أنا زبولون، وُلدتُ هبةً طيبةً لوالديّ.

\par 4 لأنه عندما ولدت أنا، كان أبي ينمو كثيرًا جدًا في القطعان والبقر، عندما كان يأخذ نصيبه من القضبان المخططة.

\par 5 أنا لا أدرك أنني أخطأت طوال أيامي، إلا في الفكر.

\par 6 ولا أذكر بعد أني عملت إثماً إلا خطيئة الجهل التي أخطأتها مع يوسف، لأني عاهدت إخوتي أن لا أخبر أبي بما حدث.

\par 7 ولكني بكيت في الخفاء أيامًا كثيرة بسبب يوسف، لأني كنت خائفًا من إخوتي، لأنهم جميعًا اتفقوا على أنه إذا كشف أحد السر، يُقتل

\par 8 ولكن عندما أرادوا قتله، أقسمت عليهم كثيرًا وبدموع أن لا يكونوا مذنبين بهذه الخطيئة

\par 9 فتقدم شمعون وجاد على يوسف ليقتلاه، فقال لهما بدموع: اشفقوا عليّ يا إخوتي، ارحموا أحشاء يعقوب أبينا. لا تضعوا عليّ أيديكم لسفك دم بريء، لأني لم أخطئ إليكم

\par 10 وإن كنت قد أخطأت، فأدبوني يا إخوتي، ولكن لا تضعوا أيديكم عليّ من أجل يعقوب أبينا،

\par 11 وبينما كان ينطق بهذه الكلمات، وهو ينوح، لم أستطع تحمل نواحه، وبدأت أبكي، وانسكب كبدي، وانحلت كل مادة أحشائي

\par 12 وبكيت مع يوسف، وخفق قلبي، وارتجفت مفاصلي، ولم أستطع الوقوف

\par 13 فلما رآني يوسف أبكي معه، وهم مقبلون عليه ليقتلوه، هرب خلفي متضرعًا إليهم

\par 14 وفي أثناء ذلك قام رأوبين وقال: هلموا يا إخوتي، لا نقتله، بل نلقه في إحدى هذه الآبار اليابسة التي حفرها آباؤنا ولم يجدوا ماءً

\par 15 لأجل هذا منع الرب أن يرتفع الماء فيهما حفاظًا على يوسف

\par 16 ففعلوا ذلك حتى باعوه للإسماعيليين.

\par 17 لأنه لم يكن لي في ثمنه نصيب يا أبنائي.

\par 18 فأخذ شمعون وجاد وستة آخرون من إخوتنا ثمن يوسف، واشتروا لأنفسهم ولنسائهم وأولادهم أحذية، قائلين:

\par 19 لن نأكل منه، لأنه ثمن دم أخينا، لكننا سندوسه حتمًا، لأنه قال إنه سيكون ملكًا علينا، فلنرَ ماذا سيحدث لأحلامه

\par 20 لذلك مكتوب في كتابة ناموس موسى أن كل من لا يقيم نسلا لأخيه تحل حذاؤه ويبصقون في وجهه.

\par 21 ولم يشأ إخوة يوسف أن يعيش أخاهم، فحل الرب عنهم النعل الذي كانوا يلبسونه ضد يوسف أخيهم

\par 22 لأنه عندما دخلوا مصر، أطلقهم عبيد يوسف خارج الباب، فسجدوا ليوسف على طريقة الملك فرعون

\par 23 ولم يكتفوا بالسجود له، بل بصقوا عليهم أيضًا، وسقطوا أمامه في الحال، وهكذا أخجلوا أمام المصريين

\par 24 فبعد ذلك سمع المصريون بكل الشرور التي فعلوها بيوسف

\par 25 وبعد أن بيع، جلس إخوتي ليأكلوا ويشربوا.

\par 26 وأما أنا، شفقة على يوسف، فلم آكل، بل راقبت البئر، لأن يهوذا خاف أن ينقض عليه شمعون ودان وجاد ويقتلوه.

\par 27 ولكن لما رأوني لا آكل، جعلوني أراقبه حتى بيع للإسماعيليين

\par 28 ولما جاء رأوبين وسمع أن يوسف قد بيع في غيابه، مزق ثيابه وناح وقال:

\par 29 كيف أبدو في وجه أبي يعقوب؟ فأخذ الفضة وركض وراء التجار، فلما لم يجدهم عاد حزينًا

\par 30 لكن التجار تركوا الطريق العريض وساروا عبر سكان الكهوف عبر طريق مختصر

\par 31 أما رأوبين فحزن، ولم يأكل طعامًا في ذلك اليوم.

\par 32 فتقدم إليه دان وقال له: لا تبك ولا تحزن، فإننا قد وجدنا ما نقوله لأبينا يعقوب.

\par 33 لنذبح جديًا من المعزى، ونغمس فيه قميص يوسف، ثم نرسله إلى يعقوب قائلين: أعلم، هل هذا قميص ابنك؟

\par 34 ففعلوا كذلك. إذ خلعوا عن يوسف قميصه حين باعوه، وألبسوه ثوب العبد

\par 35 فأخذ شمعون القميص ولم يرد أن يُعطيه، لأنه أراد أن يمزقه بسيفه، لأنه كان غاضبًا لأن يوسف حيّ ولأنه لم يقتله

\par 36 ثم نهضنا جميعًا وقلنا له: إن لم تتنازل عن القميص، فسنقول لأبينا إنك أنت وحدك فعلت هذا الأمر الشرير في إسرائيل

\par 37 فأعطاهما إياه، ففعلا كما قال دان.

\chapter{2}

\par \textit{إنه يحث على التعاطف الإنساني والتفهم تجاه إخوانه البشر.}

\par 1 والآن أيها الأولاد، أُوصيكم (هكذا) أن تحفظوا وصايا الرب، وأن تُظهروا الرحمة لجيرانكم، وأن ترحموا الجميع، ليس البشر فقط، بل أيضًا البهائم

\par 2 من أجل كل هذا باركني الرب، وعندما مرض جميع إخوتي، نجوت دون مرض، لأن الرب يعلم مقاصد كل واحد

\par 3 فليكن لديكم رحماء في قلوبكم يا أبنائي، لأنه كما يفعل الإنسان بقريبه، هكذا يفعل الرب به أيضًا

\par 4 لأن أبناء إخوتي كانوا يمرضون ويموتون بسبب يوسف، لأنهم لم يظهروا الرحمة في قلوبهم. وأما ابنائي فقد نجاوا بلا مرض كما تعلمون.

\par 5 ولما كنت في أرض كنعان عند ساحل البحر، اصطدت سمكًا ليعقوب أبي. وعندما اختنق كثيرون في البحر، بقيت سالمًا

\par 6 كنتُ أول من صنع قاربًا ليُبحر في البحر، لأن الرب أعطاني فهمًا وحكمة فيه

\par 7 وأنزلت دفة خلفها، ومددت شراعًا على قطعة أخرى من الخشب منتصبة في المنتصف

\par 8 وأبحرت فيها على طول الشواطئ، أصطاد السمك لبيت أبي حتى وصلنا إلى مصر

\par 9 ومن خلال التعاطف، شاركت صيدي مع كل غريب.

\par 10 وإذا كان الرجل غريبًا أو مريضًا أو كبيرًا في السن، كنت أسلق السمك وأهيئه جيدًا وأقدمه لجميع الناس، كما يحتاج كل إنسان، حزينًا معهم ومشفقًا عليهم.

\par 11 لذلك أشبعني الرب أيضًا بكثرة من السمك عند صيد السمك، لأن من يشارك قريبه ينال من الرب أضعافًا مضاعفة

\par 12 لمدة خمس سنوات، كنت أصطاد سمكًا وأعطيه لكل إنسان أراه، وكفى كل بيت أبي

\par 13 وفي الصيف كنت أصطاد السمك، وفي الشتاء كنت أرعى الأغنام مع إخوتي

\par 14 الآن سأخبركم بما فعلت.

\par 15 ورأيت رجلاً متضايقاً عرياناً في الشتاء، فشفقت عليه، وسرقت ثوباً من بيت أبي خفية وأعطيته لرجل متضايق.

\par 16 لذلك، يا أبنائي، أظهروا مما أنعم الله عليكم بالرحمة والعطف دون تردد لجميع الناس، وأعطوا كل إنسان ذي قلب طيب

\par 17 وإن لم يكن لديكم ما تعطونه للمحتاج، فارحموه بأحشاء الرحمة

\par 18 أعلم أن يدي لم تجد ما يكفي لأعطيه لمن يحتاجه، ومشيت معه أبكي مسافة سبعة فيرلنغ، واشتاقت أحشائي إليه شفقةً

\par 19 فليكن لكم أنتم أيضًا، يا أبنائي، رأفةٌ على كل إنسان، لكي يرحمكم الرب أيضًا ويرحمكم

\par 20 لأنه في الأيام الأخيرة أيضًا، سيرسل الله رحمته على الأرض، وحيثما يجد أحشاء رحمة فإنه يسكن فيه

\par 21 لأنه بقدر ما يرحم الإنسان جيرانه، بنفس القدر يرحمه الرب أيضًا

\par 22 ولما نزلنا إلى مصر، لم يكن يوسف يحمل علينا ضغينة.

\par 23 فاحذروا أنتم أيضاً يا أولادي، وعرفوا أنفسكم بلا حسد، واحبوا بعضكم بعضاً، ولا يحسب كل واحد منكم شراً على أخيه.

\par 24 لأن هذا يكسر الوحدة ويفرق كل القبائل، ويزعج النفس، ويذبل الوجه

\par 25 لاحظ، إذن، المياه، واعلم أنها عندما تتدفق معًا، فإنها تكتسح الأحجار والأشجار والأرض وأشياء أخرى.

\par 26 ولكن إذا انقسموا إلى جداول كثيرة، فإن الأرض تبتلعهم، فيختفون

\par 27 هكذا تكونون أنتم أيضًا إذا انقسمتم. لا تنقسموا إذن إلى رأسين، لأن كل ما صنعه الرب له رأس واحد، وكتفان، ويدان، ورجلان، وجميع الأعضاء الباقية

\par 28 لأني علمت من كتابة آبائي أنكم ستنقسمون في إسرائيل، وتسيرون وراء ملكين، وتفعلون كل رجس

\par 29 ويسبيكم أعداؤكم، وتُساء معاملتكم بين الأمم، بأمراض وضيقات كثيرة

\par 30 وبعد هذه الأمور تذكرون الرب وتتوبون، فيرحمكم، لأنه رحيم ورؤوف

\par 31 ولا يحسب شرًا على بني البشر، لأنهم بشر، وقد انخدعوا بأعمالهم الشريرة

\par 32 وبعد هذه الأمور يشرق لكم الرب نفسه، نور البر، فتعودون إلى أرضكم

\par 33 وسوف ترونه في أورشليم من أجل اسمه.

\par 34 وأيضاً من خلال أعمالكم الشريرة تغضبونه،

\par 35 وسوف يُطرحكم بعيدًا عنه إلى وقت الانتهاء.

\par 36 والآن يا أبنائي، لا تحزنوا على موتي، ولا تحزنوا على اقتراب نهايتي.

\par 37 لأني سأقوم في وسطكم، كرئيس في وسط أبنائه، وأفرح في وسط أسباطي، كل من يحفظ شريعة الرب ووصايا زبولون أبيهم

\par 38 وأما الأشرار، فيجلب الرب عليهم نارًا أبدية، ويهلكهم طوال الأجيال

\par 39 لكنني الآن أسرع إلى راحتي، كما فعل آبائي أيضًا.

\par 40 بل تخافون الرب إلهنا بكل قوتكم كل أيام حياتكم.

\par 41 وبعد أن قال هذه الأشياء، نام في شيخوخته الصالحة.

\par 42 فوضعه بنوه في نعش من خشب، ثم حملوه ودفنوه في حبرون مع آبائه.



\end{document}