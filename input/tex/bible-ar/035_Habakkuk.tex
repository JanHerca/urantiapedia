\begin{document}

\title{حبقوق}


\chapter{1}

\par 1 اَلْوَحْيُ الَّذِي رَآهُ حَبَقُّوقُ النَّبِيُّ:
\par 2 حَتَّى مَتَى يَا رَبُّ أَدْعُو وَأَنْتَ لاَ تَسْمَعُ؟ أَصْرُخُ إِلَيْكَ مِنَ الظُّلْمِ وَأَنْتَ لاَ تُخَلِّصُ؟
\par 3 لِمَ تُرِينِي إِثْماً وَتُبْصِرُ جَوْراً وَقُدَّامِي اغْتِصَابٌ وَظُلْمٌ وَيَحْدُثُ خِصَامٌ وَتَرْفَعُ الْمُخَاصَمَةُ نَفْسَهَا؟
\par 4 لِذَلِكَ جَمَدَتِ الشَّرِيعَةُ وَلاَ يَخْرُجُ الْحُكْمُ بَتَّةً لأَنَّ الشِّرِّيرَ يُحِيطُ بِالصِّدِّيقِ فَلِذَلِكَ يَخْرُجُ الْحُكْمُ مُعَوَّجاً.
\par 5 «اُنْظُرُوا بَيْنَ الأُمَمِ وَأَبْصِرُوا وَتَحَيَّرُوا حَيْرَةً. لأَنِّي عَامِلٌ عَمَلاً فِي أَيَّامِكُمْ لاَ تُصَدِّقُونَ بِهِ إِنْ أُخْبِرَ بِهِ.
\par 6 فَهَئَنَذَا مُقِيمٌ الْكِلْدَانِيِّينَ الأُمَّةَ الْمُرَّةَ الْقَاحِمَةَ السَّالِكَةَ فِي رِحَابِ الأَرْضِ لِتَمْلِكَ مَسَاكِنَ لَيْسَتْ لَهَا.
\par 7 هِيَ هَائِلَةٌ وَمَخُوفَةٌ. مِنْ قِبَلِ نَفْسِهَا يَخْرُجُ حُكْمُهَا وَجَلاَلُهَا.
\par 8 وَخَيْلُهَا أَسْرَعُ مِنَ النُّمُورِ وَأَحَدُّ مِنْ ذِئَابِ الْمَسَاءِ وَفُرْسَانُهَا يَنْتَشِرُونَ وَيَأْتُونَ مِنْ بَعِيدٍ وَيَطِيرُونَ كَالنَّسْرِ الْمُسْرِعِ إِلَى الأَكْلِ.
\par 9 يَأْتُونَ كُلُّهُمْ لِلظُّلْمِ. مَنْظَرُ وُجُوهِهِمْ إِلَى قُدَّامٍ وَيَجْمَعُونَ سَبْياً كَالرَّمْلِ.
\par 10 وَهِيَ تَسْخَرُ مِنَ الْمُلُوكِ وَالرُّؤَسَاءُ ضِحْكَةٌ لَهَا. وَتَضْحَكُ عَلَى كُلِّ حِصْنٍ وَتُكَوِّمُ التُّرَابَ وَتَأْخُذُهُ.
\par 11 ثُمَّ تَتَعَدَّى رُوحُهَا فَتَعْبُرُ وَتَأْثَمُ. هَذِهِ قُوَّتُهَا إِلَهُهَا».
\par 12 أَلَسْتَ أَنْتَ مُنْذُ الأَزَلِ يَا رَبُّ إِلَهِي قُدُّوسِي؟ لاَ نَمُوتُ. يَا رَبُّ لِلْحُكْمِ جَعَلْتَهَا وَيَا صَخْرُ لِلتَّأْدِيبِ أَسَّسْتَهَا.
\par 13 عَيْنَاكَ أَطْهَرُ مِنْ أَنْ تَنْظُرَا الشَّرَّ وَلاَ تَسْتَطِيعُ النَّظَرَ إِلَى الْجَوْرِ فَلِمَ تَنْظُرُ إِلَى النَّاهِبِينَ وَتَصْمُتُ حِينَ يَبْلَعُ الشِّرِّيرُ مَنْ هُوَ أَبَرُّ مِنْهُ؟
\par 14 وَتَجْعَلُ النَّاسَ كَسَمَكِ الْبَحْرِ كَدَبَّابَاتٍ لاَ سُلْطَانَ لَهَا.
\par 15 تُطْلِعُ الْكُلَّ بِشِصِّهَا وَتَصْطَادُهُمْ بِشَبَكَتِهَا وَتَجْمَعُهُمْ فِي مِصْيَدَتِهَا فَلِذَلِكَ تَفْرَحُ وَتَبْتَهِجُ.
\par 16 لِذَلِكَ تَذْبَحُ لِشَبَكَتِهَا وَتُبَخِّرُ لِمِصْيَدَتِهَا لأَنَّهُ بِهِمَا سَمِنَ نَصِيبُهَا وَطَعَامُهَا مُسَمَّنٌ.
\par 17 أَفَلأَجْلِ هَذَا تَفْرَغُ شَبَكَتُهَا وَلاَ تَعْفُو عَنْ قَتْلِ الأُمَمِ دَائِماً؟

\chapter{2}

\par 1 عَلَى مَرْصَدِي أَقِفُ وَعَلَى الْحِصْنِ أَنْتَصِبُ وَأُرَاقِبُ لأَرَى مَاذَا يَقُولُ لِي وَمَاذَا أُجِيبُ عَنْ شَكْوَايَ.
\par 2 فَأَجَابَنِي الرَّبُّ: «اكْتُبِ الرُّؤْيَا وَانْقُشْهَا عَلَى الأَلْوَاحِ لِيَرْكُضَ قَارِئُهَا
\par 3 لأَنَّ الرُّؤْيَا بَعْدُ إِلَى الْمِيعَادِ وَفِي النِّهَايَةِ تَتَكَلَّمُ وَلاَ تَكْذِبُ. إِنْ تَوَانَتْ فَانْتَظِرْهَا لأَنَّهَا سَتَأْتِي إِتْيَاناً وَلاَ تَتَأَخَّرُ.
\par 4 «هُوَذَا مُنْتَفِخَةٌ غَيْرُ مُسْتَقِيمَةٍ نَفْسُهُ فِيهِ. وَالْبَارُّ بِإِيمَانِهِ يَحْيَا.
\par 5 وَحَقّاً إِنَّ الْخَمْرَ غَادِرَةٌ. الرَّجُلَ مُتَكَبِّرٌ وَلاَ يَهْدَأُ. الَّذِي قَدْ وَسَّعَ نَفْسَهُ كَالْهَاوِيَةِ وَهُوَ كَالْمَوْتِ فَلاَ يَشْبَعُ بَلْ يَجْمَعُ إِلَى نَفْسِهِ كُلَّ الأُمَمِ وَيَضُمُّ إِلَى نَفْسِهِ جَمِيعَ الشُّعُوبِ.
\par 6 فَهَلاَّ يَنْطِقُ هَؤُلاَءِ كُلُّهُمْ بِهَجْوٍ عَلَيْهِ وَلُغْزِ شَمَاتَةٍ بِهِ وَيَقُولُونَ: وَيْلٌ لِلْمُكَثِّرِ مَا لَيْسَ لَهُ. إِلَى مَتَى؟ وَلِلْمُثَقِّلِ نَفْسَهُ رُهُوناً؟
\par 7 أَلاَ يَقُومُ بَغْتَةً مُقَارِضُوكَ وَيَسْتَيْقِظُ مُزَعْزِعُوكَ فَتَكُونُ غَنِيمَةً لَهُمْ؟
\par 8 لأَنَّكَ سَلَبْتَ أُمَماً كَثِيرَةً فَبَقِيَّةُ الشُّعُوبِ كُلِّهَا تَسْلِبُكَ لِدِمَاءِ النَّاسِ وَظُلْمِ الأَرْضِ وَالْمَدِينَةِ وَجَمِيعِ السَّاكِنِينَ فِيهَا.
\par 9 «وَيْلٌ لِلْمُكْسِبِ بَيْتَهُ كَسْباً شِرِّيراً لِيَجْعَلَ عُشَّهُ فِي الْعُلُوِّ لِيَنْجُوَ مِنْ كَفِّ الشَّرِّ.
\par 10 تَآمَرْتَ الْخِزْيَ لِبَيْتِكَ. إِبَادَةَ شُعُوبٍ كَثِيرَةٍ وَأَنْتَ مُخْطِئٌ لِنَفْسِكَ.
\par 11 لأَنَّ الْحَجَرَ يَصْرُخُ مِنَ الْحَائِطِ فَيُجِيبُهُ الْجَائِزُ مِنَ الْخَشَبِ.
\par 12 «وَيْلٌ لِلْبَانِي مَدِينَةً بِالدِّمَاءِ وَلِلْمُؤَسِّسِ قَرْيَةً بِالْإِثْمِ.
\par 13 أَلَيْسَ مِنْ قِبَلِ رَبِّ الْجُنُودِ أَنَّ الشُّعُوبَ يَتْعَبُونَ لِلنَّارِ وَالأُمَمَ لِلْبَاطِلِ يُعْيُونَ؟
\par 14 لأَنَّ الأَرْضَ تَمْتَلِئُ مِنْ مَعْرِفَةِ مَجْدِ الرَّبِّ كَمَا تُغَطِّي الْمِيَاهُ الْبَحْرَ.
\par 15 «وَيْلٌ لِمَنْ يَسْقِي صَاحِبَهُ سَافِحاً حُمُوَّكَ وَمُسْكِراً أَيْضاً لِلنَّظَرِ إِلَى عَوْرَاتِهِمْ.
\par 16 قَدْ شَبِعْتَ خِزْياً عِوَضاً عَنِ الْمَجْدِ. فَاشْرَبْ أَنْتَ أَيْضاً وَاكْشِفْ غُرْلَتَكَ! تَدُورُ إِلَيْكَ كَأْسُ يَمِينِ الرَّبِّ وَقُيَاءُ الْخِزْيِ عَلَى مَجْدِكَ.
\par 17 لأَنَّ ظُلْمَ لُبْنَانَ يُغَطِّيكَ وَاغْتِصَابَ الْبَهَائِمِ الَّذِي رَوَّعَهَا لأَجْلِ دِمَاءِ النَّاسِ وَظُلْمِ الأَرْضِ وَالْمَدِينَةِ وَجَمِيعِ السَّاكِنِينَ فِيهَا.
\par 18 «مَاذَا نَفَعَ التِّمْثَالُ الْمَنْحُوتُ حَتَّى نَحَتَهُ صَانِعُهُ أَوِ الْمَسْبُوكُ وَمُعَلِّمُ الْكَذِبِ حَتَّى إِنَّ الصَّانِعَ صَنْعَةً يَتَّكِلُ عَلَيْهَا فَيَصْنَعُ أَوْثَاناً بُكْماً؟
\par 19 وَيْلٌ لِلْقَائِلِ لِلْعُودِ: اسْتَيْقِظْ! وَلِلْحَجَرِ الأَصَمِّ: انْتَبِهْ! أَهُوَ يُعَلِّمُ؟ هَا هُوَ مَطْلِيٌّ بِالذَّهَبِ وَالْفِضَّةِ وَلاَ رُوحَ الْبَتَّةَ فِي دَاخِلِهِ!
\par 20 أَمَّا الرَّبُّ فَفِي هَيْكَلِ قُدْسِهِ. فَاسْكُتِي قُدَّامَهُ يَا كُلَّ الأَرْضِ».

\chapter{3}

\par 1 صَلاَةٌ لِحَبَقُّوقَ النَّبِيِّ عَلَى الشَّجَوِيَّةِ:
\par 2 يَا رَبُّ قَدْ سَمِعْتُ خَبَرَكَ فَجَزِعْتُ. يَا رَبُّ عَمَلَكَ فِي وَسَطِ السِّنِينَ أَحْيِهِ. فِي وَسَطِ السِّنِينَ عَرِّفْ. فِي الْغَضَبِ اذْكُرِ الرَّحْمَةَ.
\par 3 اَللَّهُ جَاءَ مِنْ تِيمَانَ وَالْقُدُّوسُ مِنْ جَبَلِ فَارَانَ. سِلاَهْ. جَلاَلُهُ غَطَّى السَّمَاوَاتِ وَالأَرْضُ امْتَلَأَتْ مِنْ تَسْبِيحِهِ.
\par 4 وَكَانَ لَمَعَانٌ كَالنُّورِ. لَهُ مِنْ يَدِهِ شُعَاعٌ وَهُنَاكَ اسْتِتَارُ قُدْرَتِهِ.
\par 5 قُدَّامَهُ ذَهَبَ الْوَبَأُ وَعِنْدَ رِجْلَيْهِ خَرَجَتِ الْحُمَّى.
\par 6 وَقَفَ وَقَاسَ الأَرْضَ. نَظَرَ فَرَجَفَ الأُمَمُ وَدُكَّتِ الْجِبَالُ الدَّهْرِيَّةُ وَخَسَفَتْ آكَامُ الْقِدَمِ. مَسَالِكُ الأَزَلِ لَهُ.
\par 7 رَأَيْتُ خِيَامَ كُوشَانَ تَحْتَ بَلِيَّةٍ. رَجَفَتْ شُقَقُ أَرْضِ مِدْيَانَ.
\par 8 هَلْ عَلَى الأَنْهَارِ حَمِيَ يَا رَبُّ هَلْ عَلَى الأَنْهَارِ غَضَبُكَ أَوْ عَلَى الْبَحْرِ سَخَطُكَ حَتَّى أَنَّكَ رَكِبْتَ خَيْلَكَ مَرْكَبَاتِكَ مَرْكَبَاتِ الْخَلاَصِ؟
\par 9 عُرِّيَتْ قَوْسُكَ تَعْرِيَةً. سُبَاعِيَّاتُ سِهَامٍ كَلِمَتُكَ. سِلاَهْ. شَقَّقْتَ الأَرْضَ أَنْهَاراً.
\par 10 أَبْصَرَتْكَ فَفَزِعَتِ الْجِبَالُ. سَيْلُ الْمِيَاهِ طَمَا. أَعْطَتِ اللُّجَّةُ صَوْتَهَا. رَفَعَتْ يَدَيْهَا إِلَى الْعَلاَءِ.
\par 11 اَلشَّمْسُ وَالْقَمَرُ وَقَفَا فِي بُرُوجِهِمَا لِنُورِ سِهَامِكَ الطَّائِرَةِ لِلَمَعَانِ بَرْقِ مَجْدِكَ.
\par 12 بِغَضَبٍ خَطَرْتَ فِي الأَرْضِ بِسَخَطٍ دُسْتَ الأُمَمَ.
\par 13 خَرَجْتَ لِخَلاَصِ شَعْبِكَ لِخَلاَصِ مَسِيحِكَ. سَحَقْتَ رَأْسَ بَيْتِ الشِّرِّيرِ مُعَرِّياً الأَسَاسَ حَتَّى الْعُنُقِ. سِلاَهْ.
\par 14 ثَقَبْتَ بِسِهَامِهِ رَأْسَ قَبَائِلِهِ. عَصَفُوا لِتَشْتِيتِي. ابْتِهَاجُهُمْ كَمَا لأَكْلِ الْمِسْكِينِ فِي الْخُفْيَةِ.
\par 15 سَلَكْتَ الْبَحْرَ بِخَيْلِكَ كُوَمَ الْمِيَاهِ الْكَثِيرَةِ.
\par 16 سَمِعْتُ فَارْتَعَدَتْ أَحْشَائِي. مِنَ الصَّوْتِ رَجَفَتْ شَفَتَايَ. دَخَلَ النَّخْرُ فِي عِظَامِي وَارْتَعَدْتُ فِي مَكَانِي لأَسْتَرِيحَ فِي يَوْمِ الضَِّيقِ عِنْدَ صُعُودِ الشَّعْبِ الَّذِي يَزْحَمُنَا.
\par 17 فَمَعَ أَنَّهُ لاَ يُزْهِرُ التِّينُ وَلاَ يَكُونُ حَمْلٌ فِي الْكُرُومِ يَكْذِبُ عَمَلُ الزَّيْتُونَةِ وَالْحُقُولُ لاَ تَصْنَعُ طَعَاماً. يَنْقَطِعُ الْغَنَمُ مِنَ الْحَظِيرَةِ وَلاَ بَقَرَ فِي الْمَذَاوِدِ
\par 18 فَإِنِّي أَبْتَهِجُ بِالرَّبِّ وَأَفْرَحُ بِإِلَهِ خَلاَصِي.
\par 19 اَلرَّبُّ السَّيِّدُ قُوَّتِي وَيَجْعَلُ قَدَمَيَّ كَالأَيَائِلِ وَيُمَشِّينِي عَلَى مُرْتَفَعَاتِي. لِرَئِيسِ الْمُغَنِّينَ عَلَى آلاَتِي ذَوَاتِ الأَوْتَارِ.


\end{document}