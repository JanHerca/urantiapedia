\begin{document}

\title{سفر أخنوخ الثاني}

\chapter{1}

وصفٌ لآلية العالم، يُظهر آلية عمل الشمس والقمر. علم الفلك وتقويمٌ قديمٌ شيق. انظر الفصول ١٥-١٧، وكذلك الفصل ٢١. كيف كان العالم قبل الخلق، انظر الفصل ٢٤. الفصل ٢٦ رائعٌ بشكلٍ خاص. وصفٌ فريدٌ لكيفية خلق الشيطان (الفصل ٢٩).

\par 1 كان هناك رجل حكيم، صانع عظيم، وقد أحبه الرب واستقبله، لكي يرى المساكن العليا ويكون شاهد عيان على عالم الله القدير الحكيم والعظيم الذي لا يُدرك ولا يتغير، وعلى مقام خدام الرب الرائع والمجيد واللامع والواسع الأفق، وعلى عرش الرب الذي لا يُدرك، وعلى درجات ومظاهر الجيوش غير الجسدية، وعلى الخدمة التي لا توصف للعديد من العناصر، وعلى الظهورات المتنوعة والغناء الذي لا يُوصف لجيش الكروبيم، وعلى النور اللامحدود

\par 2 في ذلك الوقت، كما قال، عندما أكملت عامي الـ 165، أنجبت ابني ماثوسال

\par 3 بعد هذا أيضًا عشت مئتي عام، وأكملت من كل سنوات عمري ثلاثمائة وخمسة وستين عامًا

\par 4 في اليوم الأول من الشهر الأول، كنت في منزلي وحدي، مستريحًا على أريكتي ونمت

\par 5 وبينما كنت نائمًا، صعد ضيق عظيم إلى قلبي، وكنت أبكي وعيني في النوم، ولم أستطع أن أفهم ما هو هذا الضيق، أو ما الذي سيحدث لي

\par 6 وظهر لي رجلان ضخمان جدًا، لم أرَ مثلهما قط على الأرض؛ كانت وجوههما تتألق كالشمس، وعيناهما أيضًا كنور متقد، ومن شفتيهما خرجت نار، وملابسهما وغناءهما متنوع، وكانت أجنحتهما أكثر إشراقًا من الذهب، وأيديهما أكثر بياضًا من الثلج

\par 7 كانوا يقفون على رأس أريكتي وبدأوا ينادونني باسمي

\par 8 واستيقظت من نومي ورأيت بوضوح هذين الرجلين واقفين أمامي

\par 9 فسلمت عليهم، فأخذني الخوف، وتغيرت ملامح وجهي من الرعب، وقال لي أولئك الرجال:

\par 10 تشجع يا أخنوخ، لا تخف؛ لقد أرسلنا الله الأزلي إليك، وها أنت اليوم تصعد معنا إلى السماء، وتخبر أبناءك وجميع أهل بيتك بكل ما سيفعلونه بدونك على الأرض في بيتك، ولا تدع أحدًا يبحث عنك حتى يعيدك الرب إليهم

\par 11 فأسرعت لأطيعهم وخرجت من منزلي، وتوجهت إلى الأبواب كما أُمرت، واستدعيت أبنائي ماثوسال وريجيم وجايداد، وأخبرتهم بكل العجائب التي أخبرني بها أولئك الرجال

\chapter{2}

\par \textit{التعليمات. كيف علّم أخنوخ أبناءه.}

\par 1 اسمعوا لي يا أبنائي، لا أعلم إلى أين أذهب ولا ماذا يُصيبني. والآن يا أبنائي، أقول لكم: لا تبتعدوا عن الله أمام وجه الباطل الذي لم يخلق السماء والأرض، فإن هؤلاء يهلكون والذين يعبدونهم يهلكون. وليُثبّت الرب قلوبكم في مخافته. والآن يا أبنائي، لا يفكّر أحدٌ في طلبي حتى يُعيدني الرب إليكم.

\chapter{3}

\par \textit{عن صعود أخنوخ؛ كيف أخذه الملائكة إلى السماء الأولى.}

\par 1 وحدث، عندما أخبر أخنوخ أبناءه، أن الملائكة أخذوه على أجنحتهم وحملوه إلى السماء الأولى ووضعوه على السحاب. وهناك نظرت، ونظرت إلى أعلى مرة أخرى، فرأيت الأثير، ووضعوني في السماء الأولى وأروني بحرًا عظيمًا جدًا، أعظم من بحر الأرض

\chapter{4}

\par \textit{عن الملائكة الذين يحكمون النجوم.}

\par 1 أحضروا أمامي شيوخ وحكام الرتب النجمية، وأروني مئتي ملاك، يحكمون النجوم وخدماتهم للسماوات، ويطيرون بأجنحتهم ويحيطون بكل من يبحر

\chapter{5}

\par \textit{كيف يحتفظ الملائكة بمخازن الثلج.}

\par 1 وهنا نظرت إلى أسفل فرأيت مخازن الثلج، والملائكة الذين يحتفظون بمخازنهم الرهيبة، والسحب التي يخرجون منها والتي يذهبون إليها.

\chapter{6}

\par \textit{من الندى وزيت الزيتون وأزهار متنوعة.}

\par 1 وأروني كنز الندى الذي يشبه زيت الزيتون ومنظر شكله كجميع زهور الأرض، وملائكة كثيرون يحرسون كنوز هذه الأشياء، وكيف تُغلق وتُفتح.

\chapter{7}

\par \textit{كيف أُخذ أخنوخ إلى السماء الثانية.}

\par 1 وأخذني أولئك الرجال وقادوني إلى السماء الثانية، وأروني ظلامًا أعظم من ظلام الأرض، وهناك رأيت سجناء معلقين، يُراقبون، ينتظرون الدينونة العظيمة التي لا حدود لها، وكان هؤلاء الملائكة داكني المظهر، أكثر من ظلام الأرض، ويبكون بلا انقطاع طوال الساعات

\par 2 فقلت للرجال الذين كانوا معي: "لماذا يُعذب هؤلاء بلا انقطاع؟" فأجابوني: "هؤلاء مرتدون عن الله، لم يطيعوا أوامر الله، بل تشاوروا بإرادتهم الخاصة، وانصرفوا مع أميرهم، وهو أيضًا معلق في السماء الخامسة."

\par 3 وشعرتُ بشفقةٍ عظيمةٍ عليهم، فسلموا عليّ، وقالوا لي: يا رجل الله، صلِّ لأجلنا إلى الرب. فأجبتُهم: من أنا، الإنسان الفاني، حتى أصلي للملائكة؟ من يعلم إلى أين أذهب، أو ماذا سيصيبني؟ أو من سيصلي لأجلي؟

\chapter{8}

\par \textit{عن صعود أخنوخ إلى السماء الثالثة.}

\par 1 وأخذني أولئك الرجال من هناك، وصعدوا بي إلى السماء الثالثة، ووضعوني هناك؛ ونظرت إلى الأسفل، فرأيت نتاج هذه الأماكن، الذي لم يُعرف قط بجودته

\par 2 ورأيت كل الأشجار المزهرة الجميلة، ونظرت إلى ثمارها التي كانت ذات رائحة طيبة، وكل الأطعمة التي تحملها كانت تغلي برائحة زكية.

\par 3 وفي وسط الأشجار شجرة الحياة، في ذلك المكان الذي يستريح عليه الرب، عندما يصعد إلى الفردوس؛ وهذه الشجرة ذات جودة ورائحة لا توصف، ومزينة أكثر من كل شيء موجود؛ وهي من جميع الجوانب في شكل ذهبي اللون وقرمزي اللون وناري، وتغطي كل شيء، ولها ثمار من جميع الثمار

\par 4 جذرها في الجنة في طرف الأرض.

\par 5 والجنة بين الفساد والخلود.

\par 6 ويخرج ينبوعان يُخرجان عسلاً ولبناً، وينابيعهما تُخرج زيتاً وخمراً، وينفصلان إلى أربعة أقسام، ويدوران بهدوء، وينزلان إلى فردوس عدن، بين الفساد وفي الفساد

\par 7 ومن ثم ينطلقون على طول الأرض، ويدورون حول دائرتهم تمامًا مثل العناصر الأخرى

\par 8 ولا توجد هنا شجرة غير مثمرة، وكل مكان مبارك.

\par 9 وهناك ثلاثمائة ملاك مشرقين جدًا، يحفظون الحديقة، ويخدمون الرب بغناء حلو متواصل وأصوات لا تصمت أبدًا طوال الأيام والساعات.

\par 10 وقلت: ما أجمل هذا المكان، فقال لي أولئك الرجال:

\chapter{9}

\par \textit{إظهار مكانة الصديقين والرحماء لأخنوخ}

\par 1 هذا المكان، يا أخنوخ، مُهيأ للأبرار، الذين يتحملون كل أنواع الإساءات من أولئك الذين يُغيظون نفوسهم، والذين يصرفون أعينهم عن الإثم، ويُصدرون حكمًا عادلًا، ويُعطون الخبز للجائع، ويُغطون العريان بالملابس، ويُقيمون الساقطين، ويساعدون الأيتام المصابين، والذين يسلكون بلا عيب أمام وجه الرب، ويخدمونه وحده، ولهم مُهيأ هذا المكان للميراث الأبدي

\chapter{10}

\par \textit{هنا أظهروا لأخنوخ المكان الرهيب والتعذيب المتنوع.}

\par 1 "وأخذني هذان الرجلان إلى الجانب الشمالي، وأراني هناك مكانًا رهيبًا للغاية، وكانت هناك كل أنواع التعذيب في ذلك المكان: ظلام قاسٍ وكآبة غير مضاءة، ولا يوجد ضوء هناك، ولكن نارًا عكرة تشتعل باستمرار في الأعلى، وهناك نهر ناري يخرج، وهذا المكان كله في كل مكان نار، وفي كل مكان هناك صقيع وجليد، وعطش ورعدة، بينما القيود قاسية للغاية، والملائكة خائفون لا يرحمون، يحملون أسلحة غاضبة، وتعذيب لا يرحم، وقلت:

\par 2 «يا إلهي، يا إلهي، ما أشد فظاعة هذا المكان.»

\par 3 "وقال لي أولئك الرجال: هذا المكان يا أخنوخ مُهيأ لأولئك الذين يسيئون إلى الله، والذين يمارسون على الأرض الخطيئة ضد الطبيعة، والتي هي إفساد الأطفال على الطريقة اللواطية، وصنع السحر، والسحر والشعوذة الشيطانية، والذين يتباهون بأعمالهم الشريرة، والسرقة، والكذب، والافتراء، والحسد، والحقد، والزنا، والقتل، والذين، ملعونون، يسرقون أرواح الرجال، والذين، عندما يرون الفقراء يأخذون ممتلكاتهم ويصبحون أغنياء، يؤذونهم من أجل ممتلكات الآخرين؛ والذين، وهم قادرون على إشباع الفراغ، جعلوا الجائع يموت؛ ولأنهم قادرون على الكسوة، جردوا العاري؛ والذين لم يعرفوا خالقهم، وانحنوا لآلهة بلا روح (sc. بلا حياة)، الذين لا يستطيعون الرؤية ولا السمع، آلهة باطلة، والذين بنوا أيضًا صورًا منحوتة وسجدوا لعمل يدوي نجس، لأن كل هؤلاء مُهيأ هذا المكان بين هؤلاء، للميراث الأبدي.

\chapter{11}

\par \textit{هنا أخذوا أخنوخ إلى السماء الرابعة حيث مدار الشمس والقمر.}

\par 1 أخذني أولئك الرجال، وقادوني إلى السماء الرابعة، وأروني جميع المسارات المتتالية، وجميع أشعة ضوء الشمس والقمر

\par 2 وقمت بقياس خطواتهم وقارنت ضوءهم، فرأيت أن ضوء الشمس أقوى من ضوء القمر

\par 3 دائرتها والعجلات التي تدور عليها دائمًا، مثل ريح تمر بسرعة مذهلة للغاية، ولا راحة لها ليلًا ونهارًا. 1

\par 4 يرافق مرورها وعودتها أربعة نجوم عظيمة، وكل نجم تحته ألف نجمة، على يمين عجلة الشمس، وأربعة على يسارها، وكل منها تحته ألف نجمة، بإجمالي ثمانية آلاف، تنبعث مع الشمس باستمرار

\par 5 وفي النهار يحضرها خمسة عشر ربوة من الملائكة، وفي الليل ألف

\par 6 ويخرج ذوو الأجنحة الستة مع الملائكة أمام عجلة الشمس إلى ألسنة اللهب النارية، ويشعل مئة ملاك الشمس ويشعلونها

\par \textit{85:1 راجع "النقل السريع".}

\chapter{12}

\par \textit{من عناصر الشمس العجيبة جدًا.}

\par 1 ونظرت فرأيت عناصر أخرى طائرة للشمس، أسماؤها فينيكس وخالكيدري، عجيبة ورائعة، بأقدام وذيول على شكل أسد، ورأس تمساح، مظهرها أرجواني، مثل قوس قزح؛ حجمها تسعمائة قياس، أجنحتها مثل أجنحه الملائكة، لكل منها اثنا عشر، وهي ترافق الشمس وترافقها، تحمل الحرارة والندى، كما أمرها الله

\par 2 وهكذا تدور الشمس وتذهب، وتشرق تحت السماء، ويسير مسارها تحت الأرض بضوء أشعتها بلا انقطاع.

\chapter{13}

\par \textit{أخذ الملائكة أخنوخ ووضعوه في الشرق عند أبواب الشمس.}

\par 1 "حملني أولئك الرجال إلى الشرق، ووضعوني عند أبواب الشمس، حيث تخرج الشمس وفقًا لقانون الفصول ودورة أشهر السنة بأكملها، وعدد ساعات النهار والليل،

\par 2 "ورأيت ستة أبواب مفتوحة، كل باب فيه واحد وستون ملعبا وربع ملعب، وقمت بقياسها بدقة، وفهمت أن حجمها هو ذلك القدر الذي تخرج منه الشمس، وتذهب إلى الغرب، وتستوي، وتشرق في جميع الأشهر، وتعود مرة أخرى من الأبواب الستة حسب تعاقب الفصول؛ وبالتالي تنتهي فترة السنة بأكملها بعد عودة الفصول الأربعة،"

\chapter{14}

\par \textit{أخذوا أخنوخ إلى الغرب.}

\par 1 "ثم أخذني أولئك الرجال أيضًا إلى الأجزاء الغربية، وأروني ستة أبواب عظيمة مفتوحة تتوافق مع الأبواب الشرقية، قبالة حيث تغرب الشمس، وفقًا لعدد الأيام الثلاثمائة والخمسة والستين والربع."

\par 2 وهكذا تنزل مرة أخرى إلى البوابات الغربية، وتسحب نورها، وعظمة سطوعها، تحت الأرض؛ لأنه بما أن تاج تألقها في السماء مع الرب، ويحرسه أربعمائة ملاك، بينما تدور الشمس على عجلة تحت الأرض، وتقف سبع ساعات عظيمة في الليل، وتقضي نصف مسارها تحت الأرض، فعندما تصل إلى المدخل الشرقي في الساعة الثامنة من الليل، فإنها تجلب أنوارها، وتاج تألقها، وتشتعل الشمس أكثر من النار

\chapter{15}

\par \textit{انفجرت عناصر الشمس، والعنقاء، والخالكيدري في الغناء.}

\par 1 ثم تبدأ عناصر الشمس، التي تسمى فينيكس وكالكيديري، في الغناء، لذلك يرفرف كل طائر بجناحيه، فرحًا بواهب النور، ويبدأون في الغناء بأمر الرب.

\par 2 ويأتي واهب النور ليعطي سطوعًا للعالم أجمع، ويتشكل حارس الصباح، وهو أشعة الشمس، وتخرج شمس الأرض، وتستقبل سطوعها لإضاءة وجه الأرض بأكمله، وأروني هذا الحساب لذهاب الشمس.

\par 3 والأبواب التي تدخلها هي الأبواب العظيمة لحساب ساعات السنة؛ ولهذا السبب تُعد الشمس خليقة عظيمة، تستغرق دورتها ثمانية وعشرين عامًا، وتبدأ من جديد من البداية

\chapter{16}

\par \textit{فأخذوا حنوك ووضعوه مرة أخرى في الشرق عند مسار القمر.}

\par 1 لقد أراني هؤلاء الرجال المسار الآخر، مسار القمر، اثني عشر بوابة عظيمة، متوجة من الغرب إلى الشرق، والتي يدخل منها القمر ويخرج في الأوقات المعتادة.

\par 2 يدخل من البوابة الأولى إلى الأماكن الغربية للشمس، من البوابات الأولى مع واحد وثلاثين يومًا بالضبط، من البوابات الثانية مع واحد وثلاثين يومًا بالضبط، من الثالثة مع ثلاثين يومًا بالضبط، من الرابعة مع ثلاثين يومًا بالضبط، من الخامسة مع واحد وثلاثين يومًا بالضبط، من السادسة مع واحد وثلاثين يومًا بالضبط، من السابعة مع ثلاثين يومًا بالضبط، من الثامنة مع واحد وثلاثين يومًا تمامًا، من التاسعة مع واحد وثلاثين يومًا بالضبط، من العاشرة مع ثلاثين يومًا تمامًا، من الحادية عشرة مع واحد وثلاثين يومًا بالضبط، من الثانية عشرة مع ثمانية وعشرين يومًا بالضبط.

\par 3 ويمر عبر الأبواب الغربية حسب ترتيب وعدد الأبواب الشرقية، ويكمل ثلاثمائة وخمسة وستين وربع يوم من السنة الشمسية، بينما السنة القمرية فيها ثلاثمائة وأربعة وخمسون، وينقص منها اثنا عشر يومًا من الدورة الشمسية، وهي الأجزاء القمرية من السنة كلها.

\par 4 (وهكذا أيضًا، تحتوي الدائرة العظمى على خمسمائة واثنين وثلاثين عامًا.)

\par 5 يُحذف ربع يوم لمدة ثلاث سنوات، والرابع يُكمله تمامًا

\par 6 لذلك تُؤخذ خارج السماء لمدة ثلاث سنوات ولا تُضاف إلى عدد الأيام، لأنها تُغير وقت السنين إلى شهرين جديدين نحو الاكتمال، وإلى شهرين آخرين نحو النقصان

\par 7 وعندما تُبنى البوابات الغربية، تعود وتذهب إلى الشرق إلى الأضواء، وهكذا تدور ليلًا ونهارًا حول الدوائر السماوية، أدنى من جميع الدوائر، وأسرع من الرياح السماوية، والأرواح والعناصر والملائكة الطائرة؛ لكل ملاك ستة أجنحة

\par 8 له مسار سبعة أضعاف في تسعة عشر عامًا.

\chapter{17}

\par \textit{من تسابيح الملائكة التي لا يمكن وصفها.}

\par 1 في وسط السماوات، رأيت جنودًا مسلحين، يخدمون الرب، بطبلات وأورغن، بأصوات متواصلة، بصوت عذب، بصوت عذب ومتواصل، وغناء متنوع، يستحيل وصفه، ويذهل كل عقل، كم هو رائع وعجيب غناء هؤلاء الملائكة، وقد سررت بالاستماع إليه

\chapter{18}

\par \textit{عن أخذ أخنوخ إلى السماء الخامسة.}

\par 1 أخذني الرجال إلى السماء الخامسة ووضعوني، وهناك رأيت العديد من الجنود الذين لا حصر لهم، يُدعون غريغوري، بمظهر بشري، وكان حجمهم أكبر من حجم العمالقة العظام، وكانت وجوههم ذابلة، وصمت أفواههم أبديًا، ولم تكن هناك خدمة في السماء الخامسة، وقلت للرجال الذين كانوا معي:

\par 2 لماذا هؤلاء ذابلون جدًا، ووجوههم كئيبة، وأفواههم صامتة، ولماذا لا توجد خدمة في هذه السماء؟

\par 3 فقالوا لي: هؤلاء هم الجريجوريون الذين رفضوا مع أميرهم ساتنيل رب النور، وبعدهم أولئك المحتجزون في ظلام دامس في السماء الثانية، وثلاثة منهم نزلوا إلى الأرض من عرش الرب، إلى مكان إرمون، ونكثوا عهودهم على كتف تل إرمون ورأوا بنات البشر كم هن صالحات، واتخذوا لأنفسهم زوجات، ودنسوا الأرض بأفعالهم، الذين في كل أوقات عصرهم صنعوا الفوضى والاختلاط، وولد عمالقة ورجال عظماء عجيبون وعداوة عظيمة

\par 4 ولذلك حكم الله عليهم بالدينونة العظيمة، وهم يبكون على إخوتهم، وسوف يعاقبون في يوم الرب العظيم.

\par 5 فقلت للغريغوري: لقد رأيت إخوتك وأعمالهم وعذاباتهم العظيمة وصليت من أجلهم، لكن الرب حكم عليهم بالبقاء تحت الأرض حتى تنتهي السماء والأرض إلى الأبد.

\par 6 فقلت: «لماذا تنتظرون أيها الإخوة، ولا تخدمون أمام وجه الرب، ولم تقدموا خدماتكم أمام وجه الرب، لئلا تغضبوا سيدكم غضبًا؟»

\par 7 فاستمعوا إلى تحذيري، وكلموا الصفوف الأربعة في السماء، وإذا بي أقف مع هذين الرجلين، إذ بأربعة أبواق تنفخ معًا بصوت عظيم، وبدأ الغريغوري في الغناء بصوت واحد، وارتفع صوتهم أمام الرب بشكل مثير للشفقة ومؤثر

\par \textit{87:1 قارن الكتاب الثاني لآدم وحواء، الإصحاح العشرون.}

\chapter{19}

\par \textit{عن صعود أخنوخ إلى السماء السادسة.}

\par 1 "ومن ثم أخذني أولئك الرجال وحملوني إلى السماء السادسة، وهناك رأيت سبع فرق من الملائكة، لامعين للغاية ومجيدين للغاية، ووجوههم تلمع أكثر من سطوع الشمس، ولا يوجد فرق في وجوههم، أو سلوكهم، أو طريقة لباسهم؛ وهم يصدرون الأوامر، ويتعلمون مسارات النجوم، وتغير القمر، أو دوران الشمس، والحكم الصالح للعالم.

\par 2 وعندما يرون شرًا، يصدرون وصايا وتعليمًا، وغناءً حلوًا بصوت عالٍ، وجميع أغاني التسبيح

\par 3 هؤلاء هم رؤساء الملائكة الذين هم فوق الملائكة، ويقيسون كل حياة في السماء وعلى الأرض، والملائكة المعينون على الفصول والسنين، والملائكة الذين هم على الأنهار والبحار، والذين هم على ثمار الأرض، والملائكة الذين هم على كل عشب، ويعطون الطعام للجميع، لكل كائن حي، والملائكة الذين يكتبون جميع نفوس البشر، وكل أعمالهم، وحياتهم أمام وجه الرب؛ في وسطهم ستة طيور فينيكس وستة كروبيم وستة أجنحة، يغنون باستمرار بصوت واحد، ولا يمكن وصف غنائهم، وهم يفرحون أمام الرب عند موطئ قدميه

\chapter{20}

\par \textit{ومن ثم أخذوا أخنوخ إلى السماء السابعة.}

\par 1 ورفعني هذان الرجلان من هناك إلى السماء السابعة، ورأيت هناك نورًا عظيمًا جدًا، وجحافل نارية من رؤساء ملائكة عظماء، وقوى غير مادية، وسيادات، وأوامر وحكومات، وكروبيم وسيرافيم، وعروشًا وذوي عيون كثيرة، وتسعة أفواج، ومحطات نور يوانيت، فخفت، وبدأت أرتجف من رعب عظيم، فأخذني هذان الرجلان، وقاداني وراءهما، وقالا لي:

\par 2 «تشجع يا أخنوخ، لا تخف»، وأراني الرب من بعيد جالسًا على عرشه العالي. فماذا يوجد في السماء العاشرة، والرب يسكن فيها؟

\par 3 وفي السماء العاشرة يوجد الله، ويسمى في اللغة العبرية عرفات.

\par 4 "وكانت كل القوات السماوية تأتي وتقف على الدرجات العشر حسب رتبها، ويسجدون للرب، ويعودون إلى أماكنهم بفرح وسعادة، ويغنون الأغاني في النور اللامحدود بأصوات صغيرة وحنونة، ويخدمونه بمجد.

\chapter{21}

\par \textit{كيف غادر الملائكة هنا أخنوخ، في نهاية السماء السابعة، وذهبوا بعيدًا عنه دون أن يُرى.}

\par 1 والكروبيم والسيرافيم الواقفون حول العرش، ذوو الأجنحة الستة والعيون الكثيرة، لا يرحلون، واقفون أمام وجه الرب يفعلون مشيئته، ويغطون كل عرشه، وهم يترنمون بصوت وديع أمام وجه الرب: «قدوس، قدوس، قدوس، يا رب رئيس الجنود، السموات والأرض مملوءتان من مجدك».

\par 2 عندما رأيت كل هذه الأشياء، قال لي أولئك الرجال: "يا أخنوخ، إلى هذه النقطة أُمرنا أن نسافر معك." وابتعد أولئك الرجال عني، وعندها لم أرهم

\par 3 وبقيت وحدي في نهاية السماء السابعة، ففزعت، وسقطت على وجهي، وقلت في نفسي: يا ويلتي، ما الذي أصابني؟

\par 4 وأرسل الرب أحد مجيديه، رئيس الملائكة جبرائيل، وقال لي: "تشجع يا أخنوخ، لا تخف، قم أمام وجه الرب إلى الأبد، قم، وتعال معي."

\par 5 فأجبته وقلت في نفسي: يا سيدي، نفسي قد خرجت مني من الرعب والرعدة، وناديت الرجال الذين صعدوني إلى هذا المكان، عليهم توكلت، ومعهم أسير أمام وجه الرب

\par 6 فأخذني جبرائيل كما تخطف الريح ورقة، ووضعني أمام وجه الرب

\par 7 ورأيت السماء الثامنة، التي تُسمى في اللغة العبرية موزالوت، مُغيرة الفصول، الجفاف، والرطوبة، وعلامات البروج الاثنتي عشرة، التي فوق السماء السابعة

\par 8 ورأيت السماء التاسعة، التي تُسمى بالعبرية كوخافيم، حيث توجد المنازل السماوية لعلامات البروج الاثنتي عشرة

\chapter{22}

\par \textit{في السماء العاشرة، قاد رئيس الملائكة ميخائيل أخنوخ إلى وجه الرب.}

\par 1 وفي السماء العاشرة، أرافوت، رأيت منظر وجه الرب، مثل الحديد المتوهج في النار، ثم خرج، وأصدر شررًا، فاحترق.

\par 2 فرأيت وجه الرب، ولكن وجه الرب لا يوصف، عجيب ومرعب للغاية، ورهيب للغاية.

\par 3 ومن أنا لأُخبر عن ماهية الرب التي لا تُوصف، وعن وجهه البديع؟ ولا أستطيع أن أُحصي عدد تعاليمه الكثيرة، وأصواته المتنوعة، وعرش الرب العظيم غير المصنوع بالأيادي، ولا عدد الواقفين حوله، من كتائب الكروبيم والسيرافيم، ولا ترنيمهم المستمر، ولا جماله الذي لا يتغير، ومن سيُخبر عن عظمة مجده التي لا تُوصف؟

\par 4 فخررت وسجدت للرب، وقال لي الرب بشفتيه:

\par 5 "تشجع يا أخنوخ، لا تخف، قم وقف أمام وجهي إلى الأبد."

\par 6 فرفعني رئيس القضاة ميخائيل وأصعدني إلى أمام وجه الرب.

\par 7 وقال الرب لعبيده الذين جربهم: «ليقف أخنوخ أمام وجهي إلى الأبد». فسجد المجيدين للرب وقالوا: «ليذهب أخنوخ حسب قولك».

\par 8 فقال الرب لميخائيل: «اذهب وخذ حنوك من ثيابه الأرضية، وادهنه بدهني الحلو، وألبسه ثياب مجدي».

\par 9 ففعل ميخائيل هكذا كما قال له الرب. مسحني وألبسني، وكان منظر ذلك المرهم أعظم من النور العظيم، ومرهمه كالندى الحلو، ورائحته خفيفة، تشرق كشعاع الشمس، فنظرت إلى نفسي، فكنت كواحد من مجيديه

\par 10 فدعا الرب أحد رؤساء ملائكته اسمه برافوئيل، الذي كانت معرفته أسرع حكمة من رؤساء الملائكة الآخرين، فكتب جميع أعمال الرب. وقال الرب لبرافوئيل:

\par 11 «أخرج الكتب من مخازني، وقصبةً ذات كتابة سريعة، وأعطها لحنوك، وسلم له الكتب المختارة والمعزية من يدك.»



\chapter{23}

\par \textit{عن كتابات أخنوخ، وكيف كتب رحلاته الرائعة والظهورات السماوية، وكيف كتب بنفسه ثلاثمائة وستة وستين كتابًا.}

\par 1 وكان يخبرني بكل أعمال السماء والأرض والبحر، وجميع العناصر، وممراتها وجريانها، ورعود الرعود، والشمس والقمر، وجريان النجوم وتغيراتها، والفصول، والسنين، والأيام، والساعات، وشروق الرياح، وأعداد الملائكة، وتكوين أغانيهم، وكل الأشياء البشرية، ولسان كل أغنية وحياة بشرية، والوصايا، والتعليمات، والأغاني العذبة، وكل الأشياء التي من المناسب تعلمها

\par 2 وقال لي برافويل: "كل ما أخبرتك به، كتبناه. اجلس واكتب كل أرواح البشر، مهما كان عددهم المولودين، والأماكن المُعدّة لهم إلى الأبد؛ لأن جميع الأرواح مُعدّة إلى الأبد، قبل تكوين العالم."

\par 3 وكلها ثلاثون يومًا وثلاثون ليلة، وكتبت كل شيء بدقة، وكتبت ثلاثمائة وستة وستين كتابًا

\chapter{24}

\par \textit{من أسرار الله العظيمة، التي كشفها الله وأخبر بها أخنوخ، وتحدث معه وجهًا لوجه.}

\par 1 ودعاني الرب وقال لي: يا حنوك اجلس عن يساري مع جبرائيل.

\par 2 "وسجدت للرب، وكلمني الرب: يا حنوك الحبيب، كل ما تراه، كل الأشياء القائمة والمكتملة، أخبرك بها قبل البدء، كل ما خلقته من العدم، والمرئي مما لا يرى.

\par 3 اسمع يا أخنوخ، وافهم كلماتي هذه، لأني لم أخبر ملائكتي بسري، ولم أخبرهم بصعودهم، ولا بعالمي الذي لا نهاية له، ولم يفهموا خلقتي التي أخبرك بها اليوم.

\par 4 لأنه قبل أن تصبح كل الأشياء مرئية، كنت وحدي أتجول في الأشياء غير المرئية، مثل الشمس من الشرق إلى الغرب، ومن الغرب إلى الشرق.

\par 5 لكن حتى الشمس لديها سلام في حد ذاتها، بينما لم أجد سلامًا، لأني كنت أخلق كل الأشياء، وفكرت في وضع الأساسات، وخلق الخلق المرئي

\chapter{25}

\par \textit{يروي الله لأخنوخ كيف ينزل المرئي وغير المرئي من أدنى الظلمات.}

\par 1 أمرتُ في أدنى الأجزاء أن تنزل الأشياء المرئية من غير المرئية، فنزل أدويل عظيمًا جدًا، ونظرتُ إليه، وإذا ببطن من نور عظيم

\par 2 فقلت له: "انحطّ يا أدويل، ودع المرئي يخرج منك."

\par 3 ثم انحل، وخرج نور عظيم. وكنت في وسط النور العظيم، وكما يولد نور من نور، خرج عصر عظيم، وأظهر كل الخليقة التي فكرت في خلقها

\par 4 ورأيت أنه حسن.

\par 5 ووضعت لنفسي عرشًا، وجلست عليه، وقلت للنور: "اصعد إلى أعلى، وثبّت نفسك عاليًا فوق العرش، وكن أساسًا للأشياء العليا."

\par 6 وفوق النور لا يوجد شيء آخر، ثم انحنيت ونظرت من عرشي

\chapter{26}

\par \textit{يدعو الله من أدنى مكان للمرة الثانية أن يخرج أركاس، الثقيل والأحمر جدًا.}

\par 1 "و دعوت الأدنى مرة ثانية وقلت: فليخرج أركاس بقوة، فخرج بقوة من غير المرئي."

\par 2 وخرج أركاس، صلبًا، ثقيلًا، وأحمر اللون جدًا.

\par 3 فقلت: "انفتح يا أركاس، وليولد منك"، فانفتح، وظهر عصر عظيم ومظلم للغاية، يحمل خلق كل الأشياء الدنيا، ورأيت أنه جيد وقلت له:

\par 4 "انزل إلى الأسفل، وثبت نفسك، وكن أساسًا للأشياء السفلية"، فحدث ذلك، فنزل وثبت نفسه، وأصبح أساسًا للأشياء السفلية، وليس هناك شيء آخر تحت الظلمة.



\chapter{27}

\par \textit{كيف أسس الله الماء، وأحاطه بالنور، وأقام عليه سبع جزر.}

\par 1 وأمرت أن يؤخذ من النور والظلمة، وقلت: كن كثيفًا، فصار كذلك، ونشرته مع النور، فصار ماءً، ونشرته فوق الظلمة، تحت النور، ثم ثبتت المياه، أي التي لا قاع لها، وجعلت أساسًا من نور حول الماء، وخلقت سبع دوائر من داخل، وصورتها (أي الماء) كالبلور الرطب واليابس، أي كالزجاج، ومحيط المياه والعناصر الأخرى، وأريت كل واحد منها طريقه، والسبعة نجوم كل واحد منها في سمائها أنها تسير هكذا، ورأيت أنه كان حسنًا.

\par 2 وفصلت بين النور وبين الظلمة، أي في وسط الماء هنا وهناك، وقلت للنور أنه يكون نهاراً وللظلمة أنه يكون ليلاً، فكان مساء وكان صباح يوماً واحداً.

\chapter{28}

\par \textit{الأسبوع الذي أظهر فيه الله لأخنوخ كل حكمته وقدرته، طوال الأيام السبعة، كيف خلق كل القوى السماوية والأرضية وكل الأشياء المتحركة حتى الإنسان.}

\par 1 ثم ثبّتُ الدائرة السماوية، وجعلتُ الماء السفلي الذي تحت السماء يتجمع معًا في كل واحد، وأن الفوضى تجف، وأصبح كذلك

\par 2 من الأمواج، خلقتُ صخرةً صلبةً وكبيرةً، ومن الصخر كدستُ الجاف، وسمّيتُ الجاف أرضًا، وسمّيتُ وسط الأرض هاويةً، أي بلا قاع، جمعتُ البحر في مكانٍ واحدٍ وربطته بنير

\par 3 وقلت للبحر: ها أنا أعطيك حدودك الأبدية، ولن تنفصل عن أجزائك

\par 4 وهكذا ثبّتُ السماء. في هذا اليوم دعوتُ نفسي أول المخلوقات

\chapter{29}

\par \textit{ثم كان مساء، ثم كان صباح أيضًا، وكان يومًا ثانيًا. (الاثنين هو اليوم الأول.) الجوهر الناري.}

\par 1 "ومن أجل كل القوات السماوية، تصورت صورة وجوهر النار، ونظرت عيني إلى الصخرة الصلبة جدًا، ومن بريق عيني تلقى البرق طبيعته العجيبة، وهي نار في ماء وماء في نار، ولا يطفئ أحدهما الآخر، ولا يجفف أحدهما الآخر، لذلك يكون البرق أشد إشراقًا من الشمس، وأنعم من الماء، وأكثر صلابة من الصخرة الصلبة.

\par 2 وأشعلتُ من الصخرة نارًا عظيمة، ومن النار خلقتُ رتب عشرة ملائكة غير متجسدين، أسلحتهم نارية وملابسهم لهيبٌ مُتقد، وأمرتُ كل واحدٍ منهم بالوقوف في رتبته. وهنا أُلقي شيطانيل مع ملائكته من العُليا.

\par 3 "وبعد أن انحرف أحد الملائكة عن النظام الذي تحته، خطرت في ذهنه فكرة مستحيلة، وهي أن يضع عرشه أعلى من السحاب فوق الأرض، حتى يصبح مساويا في رتبته لقدرتي.

\par 4 وألقيته من العلو مع ملائكته، وكان يطير في الهواء باستمرار فوق القاع.

\chapter{30}

\par \textit{ثم خلقت السماوات كلها، وكان اليوم الثالث (الثلاثاء)}

\par 1 وفي اليوم الثالث أمرت الأرض بأن تنبت أشجارًا عظيمة ومثمرة، وتلالًا، وبذورًا تزرع، وغرستُ الجنة وأحاطتها، ووضعتُ حراسًا مسلحين من الملائكة المشتعلة، وهكذا خلقتُ التجديد.

\par 2 ثم جاء المساء وجاء الصباح اليوم الرابع.

\par 3 (الأربعاء). وفي اليوم الرابع أمرتُ بأن تكون هناك أنوار عظيمة على الدوائر السماوية.

\par 4 على الدائرة الأولى العليا وضعت النجوم كرونو، وعلى الثانية أفروديت، وعلى الثالثة أريس، وعلى الخامسة زيوس، وعلى السادسة إرميس، وعلى السابعة الأصغر القمر، وزينتها بالنجوم الأصغر.

\par 5 وفي الأسفل وضعت الشمس لإضاءة النهار، والقمر والنجوم لإضاءة الليل

\par 6 الشمس التي يجب أن تسير وفقًا لكل حيوان (علامات البروج)، اثني عشر، وعينت تعاقب الأشهر وأسمائها وحياتها، وعواصفها، وعلامات ساعاتها، وكيف ينبغي أن تنجح

\par 7 ثم جاء المساء وجاء صباح اليوم الخامس.

\par 8 (الخميس). وفي اليوم الخامس أمرت البحر أن يخرج سمكًا وطيورًا بأنواع مختلفة، وكل الحيوانات التي تدب على الأرض، وتسير على الأرض على أربع، وتحلق في الهواء، ذكرًا وأنثى، وكل نفس تتنفس روح الحياة.

\par 9 وكان مساء، وكان صباح يومًا سادسًا.

\par 10 (الجمعة). في اليوم السادس أمرت حكمتي أن تخلق الإنسان من سبعة أشياء: الأول لحمه من الأرض، والثاني دمه من الندى، والثالث عينيه من الشمس، والرابع عظامه من الحجر، والخامس عقله من سرعة الملائكة ومن السحاب، والسادس عروقه وشعره من عشب الأرض، والسابع روحه من أنفاسي ومن الريح.

\par 11 وأعطيته سبع طبائع: للجسد السمع، وللعينين البصر، وللنفس الشم، وللعروق اللمس، وللدم التذوق، وللعظام التحمل، وللعقل الحلاوة (المتعة).

\par 12 لقد فكرت في مقولة ماكرة تقول: لقد خلقت الإنسان من طبيعة غير مرئية ومن طبيعة مرئية، وكلاهما موته وحياته وصورته، إنه يعرف الكلام مثل شيء مخلوق، صغير في العظمة وكبير في الصغر، ووضعته على الأرض، ملاكًا ثانيًا، شريفًا وعظيمًا ومجيدًا، وعينته حاكمًا ليحكم على الأرض وليحصل على حكمتي، ولم يكن هناك مثله على الأرض من بين جميع مخلوقاتي الموجودة

\par 13 وعينت له اسمًا من الأجزاء الأربعة، من الشرق، ومن الغرب، ومن الجنوب، ومن الشمال، وعينت له أربعة نجوم خاصة، وسميته آدم، وأريته الطريقين، النور والظلمة، وقلت له:

\par 14 «هذا جيد، وهذا سيء»، حتى أعلم ما إذا كان يكنّ لي الحب أم الكراهية، حتى يتضح أيٌّ من عِرقه يحبني

\par 15 لأني رأيت طبيعته، لكنه لم يرَ طبيعته، لذلك من خلال عدم الرؤية سيخطئ أكثر، وقلت: "بعد الخطيئة ماذا يوجد سوى الموت؟"

\par 16 ووضعتُ فيه نومًا فنام. وأخذتُ منه ضلعًا، وخلقتُ له زوجةً، ليأتيه الموت عن طريق زوجته، وأخذتُ كلمته الأخيرة وسميتها أمًّا، أي إيفا

\chapter{31}

\par \textit{يسلم الله الجنة لآدم، ويعطيه أمرًا برؤية السماوات مفتوحة، وأن يرى الملائكة يغنون أغنية النصر.}

\par 1 كان لآدم حياة على الأرض، وخلقتُ جنة في عدن في الشرق، لكي يحفظ الوصية ويحفظ الوصية

\par 2 فتحت له السماوات، ليرى الملائكة يغنون أغنية النصر، والنور البهي

\par 3 وكان في الجنة باستمرار، وفهم الشيطان أنني أريد خلق عالم آخر، لأن آدم كان سيدًا على الأرض، ليحكمها ويسيطر عليها

\par 4 الشيطان هو الروح الشريرة للأماكن السفلية، كهارب صنع سوتونا من السماء وكان اسمه ساتنيل، وهكذا أصبح مختلفًا عن الملائكة، لكن طبيعته لم تغير ذكاءه فيما يتعلق بفهمه للأمور الصالحة والخاطئة

\par 5 وفهم إدانته والخطيئة التي أخطأها من قبل، لذلك حمل فكرًا ضد آدم، وفي هذه الصورة دخل وأغوى حواء، لكنه لم يلمس آدم

\par 6 لكنني لعنت الجهل، لكن ما باركته سابقًا، الذي لم ألعنه، لم ألعن الإنسان، ولا الأرض، ولا المخلوقات الأخرى، بل ثمر الإنسان الشرير وأعماله

\chapter{32}

\par \textit{بعد خطيئة آدم، أرسله الله إلى الأرض "حيث أخذتك"، لكنه لا يريد أن يُهلكه طوال السنوات القادمة.}

\par 1 فقلت له: أنت من التراب، وإلى التراب الذي أخذتك تذهب، ولا أهلكك، بل أرسلك من حيث أخذتك.

\par 2 ثم أستطيع أن آخذك مرة أخرى في مجيئي الثاني!

\par 3 وباركتُ جميع مخلوقاتي، المرئية وغير المرئية. ومكث آدم خمس ساعات ونصفًا في الجنة.

\par 4 وباركت اليوم السابع الذي هو السبت الذي فيه استراح من جميع أعماله.

\chapter{33}

\par \textit{أظهر الله لأخنوخ عمر هذا العالم، ووجوده لمدة سبعة آلاف سنة، والألف سنة الثامنة هي النهاية، لا سنوات، ولا أشهر، ولا أسابيع، ولا أيام.}

\par 1 وعينت اليوم الثامن أيضًا، حتى يكون اليوم الثامن هو اليوم الأول الذي يتم خلقه بعد عملي، وأن السبعة الأوائل تدور في شكل الألف السابع، وأن يكون في بداية الألف الثامن وقت لا عد فيه، لا نهاية له، ليس فيه سنوات ولا أشهر ولا أسابيع ولا أيام ولا ساعات.

\par 2 والآن يا أخنوخ، كل ما أخبرتك به، كل ما فهمته، كل ما رأيت من الأشياء السماوية، كل ما رأيت على الأرض، وكل ما كتبته في كتب بحكمتي العظيمة، كل هذه الأشياء ابتكرتها وخلقتها من الأساس العلوي إلى الأساس السفلي وإلى النهاية، وليس هناك مستشار ولا وارث لمخلوقاتي.

\par 3 أنا أبدي، لم أصنع بأيدي، ولا أتغير.

\par 4 فكري هو مستشاري، وحكمتي وكلمتي مصنوعة، وعيناي تراقب كل الأشياء كيف تقف هنا وترتجف من الرعب.

\par 5 إذا أدرت وجهي، فسوف يدمر كل شيء.

\par 6 "وطبق عقلك يا حنوك، واعرف من يكلمك، وخذ الكتب التي كتبتها بنفسك."

\par 7 وأعطيك صموئيل وراغويل اللذين قاداك، والكتب، وانزل إلى الأرض، وأخبر أبنائك بكل ما أخبرتك به، وكل ما رأيت، من السماء السفلى إلى عرشي، وكل الجيش

\par 8 لأني خلقت كل القوى، وليس هناك من يقاومني أو لا يخضع لي. فالجميع يخضعون لمملكتي، ويعملون من أجل حكمي وحدي

\par 9 أعطهم كتب الخط، فيقرأونها ويعرفونني خالق كل الأشياء، ويفهمون أنه لا يوجد إله آخر غيري

\par 10 وليوزعوا كتب خط يدك - أطفالًا على أطفال، وجيلًا على جيل، وأممًا على أمم

\par 11 وسأعطيك يا حنوك، شفيعي، رئيس القضاة ميخائيل، بدلًا من كتابات آبائك آدم، وشيث، وأنوش، وقينان، ومهللئيل، ويارد أبيك

\chapter{34}

\par \textit{يُدين الله المشركين والزناة اللواطيين، ولذلك يُنزل عليهم طوفانًا.}

\par 1 لقد رفضوا وصاياي ونيري، ونشأت نسل لا قيمة له، لا يخافون الله، ولم يسجدوا لي، بل بدأوا يسجدون لآلهة باطلة، وأنكروا وحدانيتي، وحملوا الأرض كلها بالأكاذيب والتعديات والفجور البغيض، كل واحد مع الآخر، وجميع أنواع الشرور النجسة الأخرى، التي من المثير للاشمئزاز سردها

\par 2 ولذلك سأُنزل طوفانًا على الأرض، وسأُهلك جميع البشر، وستنهار الأرض كلها في ظلام دامس

\chapter{35}

\par \textit{ترك الله رجلاً بارًا واحدًا من سبط أخنوخ مع بيته بأكمله، والذي فعل مسرة الله حسب مشيئته.}

\par 1 هوذا من نسلهم يقوم جيل آخر بعد زمن طويل، ولكن كثيرين منهم سيكونون غير شبعانين.

\par 2 "فمن يربي هذا الجيل، سيكشف لهم كتب خط يدك، كتب آبائك، لأولئك الذين يجب أن يشير إليهم إلى حراسة العالم، إلى الرجال المؤمنين والعاملين في سعادتي، الذين لا يعترفون باسمي عبثًا.

\par 3 وسيخبرون جيلاً آخر، والذين يقرؤون سيتمجدون فيما بعد أكثر من الأولين

\chapter{36}

\par \textit{أمر الله أخنوخ أن يعيش على الأرض ثلاثين يومًا، ليُعلّم أبنائه وأبناء أبنائه. وبعد ثلاثين يومًا، أُخذ مرة أخرى إلى السماء.}

\par 1 والآن يا أخنوخ أعطيك مهلة ثلاثين يوماً لتقضيها في بيتك، وتخبر بها أبنائك وكل أهل بيتك، لكي يسمع الجميع من وجهي ما تقوله لهم، فيقرأون ويفهمون أنه لا يوجد إله آخر غيري.

\par 2 "لكي يحفظوا وصاياي دائماً، ويبدأوا في القراءة ويأخذوا كتب خط يدك."

\par 3 وبعد ثلاثين يومًا سأرسل إليك ملاكي، وسيأخذك من الأرض ومن أبنائك إليّ.

\chapter{37}

\par 1 "فدعا الرب أحد الملائكة الأكبر سنا، وكان رهيبا ومهددا، ووضعه بجانبي، وكان منظره أبيض كالثلج، ويداه كالجليد، ومنظرهما مثل الصقيع العظيم، فجمد وجهي، لأني لم أستطع أن أتحمل رعب الرب، كما أنه لا يمكن أن أتحمل نار الموقد وحرارة الشمس وصقيع الهواء.

\par 2 فقال لي الرب: يا أخنوخ، إن لم يكن وجهك متجمدًا هنا، فلن يتمكن أحد من رؤية وجهك

\chapter{38}

\par \textit{استمر ماثوسال في الأمل وانتظار والده حنوك على فراشه ليلًا ونهارًا.}

\par 1 وقال الرب لأولئك الرجال الذين قادوني أولاً: «لينزل أخنوخ معكم إلى الأرض، وانتظروه إلى اليوم المحدد».

\par 2 ووضعوني على أريكتي ليلا.

\par 3 وكان متوشال ينتظر مجيئي، ويحرس فراشي ليلًا ونهارًا، وقد امتلأ رهبةً عندما سمع مجيئي، فقلت له: «ليجتمع كل أهل بيتي، لأخبرهم بكل شيء».

\chapter{39}

\par \textit{نصيحة حنوك الحزينة لأبنائه بالبكاء والنحيب الشديد، وهو يخاطبهم.}

\par 1 يا أبنائي، أحبائي، اسمعوا تحذير أبيكم، بقدر ما هو بحسب مشيئة الرب.

\par 2 لقد أتي إليكم اليوم وأعلن لكم، ليس من شفتي، بل من شفتي الرب، كل ما هو كائن وكل ما هو الآن وكل ما سيكون إلى يوم الدينونة.

\par 3 "لأن الرب سمح لي أن آتي إليكم، فأنتم تسمعون كلام شفتي، رجل عظيم لكم، ولكني رأيت وجه الرب، مثل الحديد المتوهج من النار يرسل شررًا ويحرق،

\par 4 تنظر الآن إلى عينيّ، عينيّ رجل كبيرتين تحملان معنىً لك، لكنني رأيت عينيّ الربّ، تتألقان كأشعة الشمس وتملأان عينيّ الإنسان بالرهبة

\par 5 ترون الآن، يا أبنائي، اليد اليمنى لرجل تساعدكم، لكنني رأيت يمين الرب تملأ السماء وهي تساعدني

\par 6 أنت ترى بوصلة عملي كعملك، لكنني رأيت بوصلة الرب اللامحدودة والكاملة، والتي ليس لها نهاية

\par 7 تسمع كلام شفتي، كما سمعت كلام الرب، كرعد عظيم لا يتوقف مع هدير السحاب

\par 8 والآن يا أبنائي، اسمعوا خطابات أبي الأرض، كم هو مخيف ومرعب الوقوف أمام وجه حاكم الأرض، وكم هو أكثر رعبًا ورعبًا الوقوف أمام وجه حاكم السماء، المتحكم بالأحياء والأموات، والجنود السماوية. من يستطيع أن يتحمل هذا الألم الذي لا ينتهي؟

\chapter{40}

\par \textit{ينصح أخنوخ أبناءه حقًا بكل ما جاء على لسان الرب، كيف رأى وسمع وكتب.}

\par 1 والآن يا أبنائي أعلم كل شيء، لأن هذا من فم الرب، وهذا ما رأته عيني من البداية إلى النهاية.

\par 2 أنا أعلم كل شيء، وقد كتبت كل شيء في الكتب، السماوات ونهايتها، وملؤها، وكل الجيوش وزحفها.

\par 3 لقد قمت بقياس ووصف النجوم، العدد الكبير الذي لا يحصى منها

\par 4 أي إنسان رأى ثوراتهم ودخولهم؟ حتى الملائكة لا يرون عددهم، بينما كتبتُ جميع أسمائهم

\par 5 وقمتُ بقياس دائرة الشمس، وقستُ أشعتها، وحسبتُ الساعات، وكتبتُ أيضًا كل ما يدور على الأرض. كتبتُ الأشياء التي تُغذّى، وكل البذور المزروعة وغير المزروعة، التي تُنتجها الأرض، وكل النباتات، وكل عشب وكل زهرة، وروائحها الزكية، وأسمائها، ومساكن السحب، وتركيبها، وأجنحتها، وكيف تحمل المطر وقطرات المطر

\par 6 وبحثتُ في كل شيء، وكتبتُ طريق الرعد والبرق، فأرَوني المفاتيح وحراسها، وصعودها، والطريق الذي تسلكه؛ يُطلَق باعتدال (برفق) بواسطة سلسلة، خشية أن يُلقي بسلسلة ثقيلة وعنف الغيوم الغاضبة ويدمر كل شيء على الأرض

\par 7 كتبتُ كنوز الثلج، ومخازن البرد والهواء المتجمد، ولاحظتُ حامل مفاتيح موسمها، فهو يملأ السحب بها، ولا يستنفد كنوزها

\par 8 وكتبتُ أماكن استراحة الرياح، ولاحظتُ ورأيتُ كيف يحمل حاملو مفاتيحها الموازين والمقاييس؛ أولًا، يضعونها في ميزان واحد، ثم في الآخر الأوزان، ويطلقونها حسب القياس بمهارة على الأرض كلها، لئلا تهتز الأرض بسبب تنفسها الثقيل

\par 9 وقمت بقياس الأرض كلها، وجبالها، وجميع التلال، والحقول، والأشجار، والأحجار، والأنهار، وكل الأشياء الموجودة، وكتبتها، الارتفاع من الأرض إلى السماء السابعة، ونزولاً إلى أدنى الجحيم، ومكان الدينونة، والجحيم العظيم المفتوح والباكى

\par 10 ورأيت كيف يتألم السجناء، ويتوقعون الحكم اللامحدود

\par 11 وكتبت جميع الذين يحكم عليهم القاضي، وجميع أحكامهم (أحكامهم) وجميع أعمالهم

\chapter{41}

\par \textit{كيف رثى أخنوخ خطيئة آدم.}

\par 1 ورأيت جميع الأجداد من كل العصور مع آدم وحواء، وتنهدت وانفجرت في البكاء وقلت عن خراب عارهم:

\par 2. «ويل لي لضعفي وضعف آبائي»، وفكرت في قلبي وقلت:

\par 3 طوبى للرجل الذي لم يولد أو الذي يولد ولا يخطئ أمام الرب حتى لا يأتي إلى هذا المكان ولا يحمل نير هذا المكان.

\chapter{42}

\par \textit{كيف رأى أخنوخ حاملي المفاتيح وحراس أبواب الجحيم واقفين.}

\par 1 "رأيت حاملي المفاتيح وحراس أبواب الجحيم واقفين مثل الحيات العظيمة ووجوههم مثل المصابيح المنطفئة وعيونهم من نار وأسنانهم الحادة ورأيت كل أعمال الرب كيف أنها صالحة وأعمال الإنسان بعضها صالح وبعضها رديء وفي أعمالهم يعرف الكاذبون.

\chapter{43}

\par \textit{يُظهر أخنوخ لأولاده كيف كان يقيس ويكتب أحكام الله.}

\par 1 أنا أبنائي، قست وكتبت كل عمل وكل قياس وكل حكم عادل

\par 2 كما أن سنةً ما أشرف من أخرى، فكذلك يكون رجلٌ أشرف من آخر، بعضهم لكثرة ممتلكاتهم، وبعضهم لحكمة قلبهم، وبعضهم لذكائهم الخاص، وبعضهم لدهائهم، وواحدٌ لصمت شفاههم، وآخرٌ للنظافة، وواحدٌ للقوة، وآخرٌ للجمال، وواحدٌ لشبابهم، وآخرٌ لذكائهم الحاد، وواحدٌ لشكل جسدهم، وآخرٌ لحساسيتهم، فليُسمع هذا في كل مكان، ولكن لا أحد أفضل ممن يتقي الله، سيكون أكثر مجدًا في المستقبل

\chapter{44}

\par \textit{يوصي أخنوخ أبناءه ألا يحتقروا وجه الإنسان، صغيرًا كان أم كبيرًا.}

\par 1 فخلق الرب الإنسان بيديه، على صورة وجهه، صغيراً وكبيراً صنعه.

\par 2 من شتم وجه الحاكم وأبغض وجه الرب فقد أبغض وجه الرب. ومن يصب غضبه على أحد بلا ضرر فغضب الرب العظيم يقطعه. ومن يبصق على وجه الإنسان عيباً يقطع عند دينونة الرب العظيمة.

\par 3 طوبى للرجل الذي لا يوجه قلبه بالحقد ضد أي إنسان، ويساعد المصاب والمدان، ويرفع المنكسر، ويتصدق على المحتاجين، لأنه في يوم الدينونة العظيمة، سيكون كل وزن وكل مكيال وكل وزن كما في السوق، أي أنها ستُعلق على ميزان وتقف في السوق، وسيتعلم كل واحد مكياله، وحسب مكياله سيأخذ أجره



\chapter{45}

\par \textit{يُظهِر الله أنه لا يريد من الناس ذبائح ولا محرقات، بل قلوبًا طاهرة ومنسحقة.}

\par 1 من يسرع في تقديم التقدمة أمام وجه الرب، فإن الرب يسرع في تقديمها بإعطائه من عمله.

\par 2 "ولكن من يكثر سراجه أمام الرب ولا يحكم بالحق فلا يكثر الرب كنزه في ملكوت الأعالي."

\par 3 عندما يطلب الرب الخبز، أو الشموع، أو اللحم (أي الماشية)، أو أي تضحية أخرى، فهذا لا شيء؛ لكن الله يطلب قلوبًا نقية، ومع كل ذلك يختبر قلب الإنسان فقط.

\chapter{46}

\par \textit{كيف لا يقبل الحاكم الأرضي من الإنسان الهدايا البغيضة والنجسة، فكم بالحري يكره الله الهدايا النجسة، بل يطردها بغضب ولا يقبل هداياه.}

\par 1 اسمعوا يا شعبي، وأنصتوا إلى أقوال شفتي.

\par 2 إن قدم أحد هدية إلى حاكم أرضي، وكان في قلبه أفكار غير مخلصة، وعلم الحاكم ذلك، أفلا يغضب عليه ولا يرفض هداياه ولا يسلمه إلى الحكم؟

\par 3 أو إذا أظهر رجل نفسه جيدًا لآخر بخداع اللسان، وكان الشر في قلبه، أفلا يفهم الآخر خيانة قلبه، ويُدان هو نفسه، لأن كذبه كان واضحًا للجميع؟

\par 4 وعندما يرسل الرب نورًا عظيمًا، فسيكون هناك دينونة للأبرار والأشرار، ولن يفلت أحد من العقاب

\chapter{47}

\par \textit{يُعلّم أخنوخ أبناءه من فم الله، ويُسلّمهم خط هذا الكتاب.}

\par 1 والآن يا أبنائي، ضعوا أفكارًا في قلوبكم، ولاحظوا جيدًا كلمات أبيكم، التي جاءت إليكم جميعًا من فم الرب

\par 2 خذ هذه الكتب التي بخط يد أبيك واقرأها.

\par 3 لأن الكتب كثيرة، وفيها ستتعلم جميع أعمال الرب، كل ما كان من بدء الخليقة، وسيكون إلى نهاية الزمان

\par 4 "وإذا لاحظتم خط يدي، فلن تخطئوا إلى الرب، لأنه ليس آخر إلا الرب، لا في السماء، ولا في الأرض، ولا في الأماكن السفلى، ولا في الأساس الواحد."

\par 5 وضع الرب الأساسات في المجهول، وبسط السماوات المرئية وغير المرئية؛ ثبّت الأرض على المياه، وخلق مخلوقات لا تُحصى، ومن أحصى الماء وأساس ما هو غير ثابت، أو تراب الأرض، أو رمل البحر، أو قطرات المطر، أو ندى الصباح، أو أنفاس الريح؟ من ملأ الأرض والبحر، والشتاء الذي لا ينفصم؟

\par 6 قطعتُ النجوم من النار، وزيّنتُ السماء، ووضعتها في وسطها

\chapter{48}

\par \textit{حول مرور الشمس على طول الدوائر السبع.}

\par 1 أن الشمس تسير على طول الدوائر السماوية السبع، وهي تعيين مائة واثنين وثمانين عرشًا، وأنها تغرب في يوم قصير، ومرة ​​أخرى مائة واثنين وثمانين عرشًا، وأنها تغرب في يوم عظيم، ولها عرشان تستريح عليهما، تدوران هنا وهناك فوق عروش الأشهر،

\par 2 من اليوم السابع عشر من شهر تسيفان ينزل إلى شهر ثيفان، ومن اليوم السابع عشر من ثيفان يصعد.

\par 3 وهكذا يقترب من الأرض، ثم تكون الأرض وتنمو ثمارها، وعندما يذهب بعيدًا، تكون الأرض حزينة، والأشجار وجميع الثمار ليس لها إزهار.

\par 4 لقد قاس كل هذا بمقياس جيد للساعات، ووضع مقياسًا بحكمته للمرئي وغير المرئي.

\par 5 فمن غير المنظور خلق كل شيء مرئيًا، وهو نفسه غير مرئي.

\par 6 "لذلك أعلمكم يا أبنائي، وأوزع الكتب على أبنائكم، في كل أجيالكم، وبين الأمم الذين لديهم الشعور بمخافة الله، فليقبلوها، ولعلهم يحبونها أكثر من أي طعام أو حلوى أرضية، ويقرؤونها ويطبقونها.

\par 7 وأولئك الذين لا يفهمون الرب، والذين لا يخافون الله، والذين لا يقبلونها بل يرفضونها، والذين لا يتقبلونها (أي الكتب)، فإن دينونة رهيبة تنتظرهم

\par 8 طوبى للرجل الذي يحمل نيرهم ويجرهم، لأنه سيُطلق سراحه في يوم الدينونة العظيمة

\chapter{49}

\par \textit{يُوصي أخنوخ أبناءه ألا يُقسموا بالسماء ولا بالأرض، ويُظهر وعد الله، حتى في رحم أمهاتهم.}

\par 1 أقسم لكم يا أبنائي، ولكنني لا أقسم بأي قسم، لا بالسماء ولا بالأرض، ولا بأي مخلوق آخر خلقه الله

\par 2 قال الرب: «ليس فيَّ قَسَمٌ ولا ظُلْمٌ، بل الحَقُّ».

\par 3 إذا لم يكن هناك حقيقة في الرجال، فعليهم أن يقسموا بكلمات "نعم، نعم"، أو "لا، لا!"

\par 4 وأقسم لكم، نعم، نعم، أنه لم يوجد رجل في بطن أمه، إلا أنه من قبل، لكل واحد منهم مكان مُعدّ لراحة الروح، ومقياس محدد لمقدار ما يُراد به اختبار الإنسان في هذا العالم

\par 5 نعم، أيها الأطفال، لا تخدعوا أنفسكم، لأنه قد أُعِدَّ مسبقًا مكان لكل نفس بشرية

\chapter{50}

\par \textit{كيف لا يمكن لأي مولود على الأرض أن يظل مخفيًا ولا أن يظل عمله مخفيًا، لكنه (الله) يأمرنا بأن نكون ودعاء، وأن نتحمل الهجوم والإهانة، ولا نسيء إلى الأرامل والأيتام.}

\par 1 لقد وضعت عمل كل إنسان في الكتابة ولا يمكن لأي إنسان ولد على الأرض أن يظل مخفيًا ولا يمكن لأعماله أن تظل مخفية.

\par 2 أرى كل الأشياء.

\par 3 والآن يا أبنائي، اقضوا عدد أيامكم في الصبر والوداعة، لكي ترثوا الحياة التي لا نهاية لها.

\par 4 اصبر من أجل الرب على كل جرح وكل أذى وكل كلمة شريرة وكل هجوم.

\par 5 وإن أصابك سوء فلا ترده لا إلى قريبك ولا إلى عدوك، لأن الرب يرده لك ويكون منتقماً لك في يوم الدينونة العظيمة، حتى لا يكون انتقام هنا بين الناس.

\par 6 من أنفق منكم ذهبًا أو فضة من أجل أخيه، فسينال كنزًا وافرًا في الآخرة

\par 7 لا تؤذوا الأرامل ولا اليتامى ولا الغرباء لئلا يأتي عليكم غضب الله.

\chapter{51}

\par \textit{يُوصي أخنوخ أبناءه ألا يخبئوا كنوزًا في الأرض، بل يأمرهم بإعطاء الصدقات للفقراء.}

\par 1 مد يديك للفقراء حسب قوتك.

\par 2 لا تخبئ فضتك في الأرض.

\par 3 أعن الرجل الأمين في الضيق، فلا يجدك الضيق في وقت ضيقك.

\par 4 وكل نير ثقيل وقاس يأتي عليكم، احملوه كله من أجل الرب، وهكذا تجدون أجركم في يوم الدينونة.

\par 5 من الجيد أن تذهب صباحًا وظهرًا ومساءً إلى مسكن الرب، لمجد خالقك.

\par 6 لأن كل نفس تمجده، وكل مخلوق، مرئي وغير مرئي، يرد له الحمد

\chapter{52}

\par \textit{الله يُعلّم أتباعه كيف يُسبّحون اسمه.}

\par 1 طوبى للرجل الذي يفتح شفتيه بتسبيح إله الصباؤوت، ويسبح الرب بقلبه

\par 2 ملعون كل إنسان يفتح شفتيه لاحتقار قريبه وافتراءه، لأنه يحتقر الله

\par 3 طوبى لمن يفتح شفتيه ويبارك الله ويسبحه.

\par 4 ملعون أمام الرب كل أيام حياته من يفتح شفتيه باللعن والشتم.

\par 5 طوبى لمن يبارك جميع أعمال الرب.

\par 6 ملعون من يحتقر خليقة الرب

\par 7 طوبى لمن ينظر إلى أسفل ويقيم الساقطين.

\par 8 ملعون من ينظر إلى ما ليس له ويسعى إلى إهلاكه.

\par 9 طوبى لمن يحفظ أسس آبائه ثابتة منذ البدء

\par 10 ملعون من يحرّف أحكام آبائه.

\par 11 طوبى لمن يزرع السلام والمحبة.

\par 12 ملعون من يزعج من يحبون جيرانهم.

\par 13 طوبى لمن يتكلم بلسان متواضع وقلب متواضع مع الجميع.

\par 14 ملعون من يتكلم بالسلام بلسانه، ولا سلام في قلبه إلا بالسيف

\par 15 لأن كل هذه الأشياء ستُكشف في الموازين وفي الكتب، يوم الدينونة العظيمة

\chapter{53}

\par \textit{(لا نقول: «أبانا أمام الله، سيقف لنا يوم القيامة»، لأن الأب لا يستطيع مساعدة ابنه، ولا الابن يستطيع مساعدة والده.)}

\par 1 والآن يا أبنائي لا تقولوا: إن أبانا واقف أمام الله، ويصلي من أجل خطايانا، لأنه ليس هناك معين لأي إنسان أخطأ.

\par 2 "أنت ترى كيف كتبت كل أعمال كل إنسان، قبل خلقه، كل ما يتم بين جميع البشر طوال الوقت، ولا يستطيع أحد أن يخبر أو يروي خط يدي، لأن الرب يرى كل تصورات الإنسان، وكيف أنها باطلة، وأين تكمن في كنوز القلب.

\par 3 والآن يا أبنائي، انتبهوا جيداً لكل كلام أبيكم الذي أقوله لكم، لئلا تندموا وتقولوا: لماذا لم يخبرنا أبونا؟

\chapter{54}

\par \textit{يُوصي حنوك أبناءه بأن يُسلموا الكتب للآخرين أيضًا.}

\par 1 "في ذلك الوقت لم تفهموا هذا، فلتكن لكم هذه الكتب التي أعطيتكم إياها ميراثاً للسلام.

\par 2 أعطها لكل من يريدها، وأرشدهم، لكي يروا أعمال الرب العظيمة والعجيبة.



\chapter{55}

\par \textit{هنا يُظهر أخنوخ أبناءه، ويقول لهم بدموع: "يا أبنائي، لقد اقتربت الساعة لأصعد إلى السماء؛ هوذا الملائكة واقفون أمامي."}

\par 1 يا أبنائي، هوذا يوم ولايتي ووقت ولايتي قد اقترب.

\par 2 لأن الملائكة الذين يذهبون معي هم واقفون أمامي ويحثونني على الرحيل عنك، وهم واقفون هنا على الأرض ينتظرون ما قيل لهم.

\par 3 لأني غدًا سأصعد إلى السماء، إلى أورشليم العليا، إلى ميراثي الأبدي

\par 4 لذلك، أطلب منكم أن تفعلوا أمام وجه الرب كل مسرته

\chapter{56}

\par \textit{يطلب متوشالام من أبيه البركة، لكي يصنع له (متوشالام) (أخنوخ) طعامًا ليأكله.}

\par 1 أجاب متوشالام أباه حنوك، وقال: "ما الذي يُحسن إليك يا أبي أن أصنعه أمام وجهك، حتى تُبارك مساكننا وأولادك، ويمجد شعبك من خلالك، ثم ترحل هكذا كما قال الرب؟"

\par 2 أجاب أخنوخ ابنه متوشالام وقال: «اسمع يا بني، منذ أن مسحني الرب بدهن مجده، لم يكن فيّ طعام، ولا تتذكر نفسي ملذات الأرض، ولا أريد أي شيء أرضي!»

\chapter{57}

\par \textit{أمر أخنوخ ابنه متوشالام باستدعاء جميع إخوته.}

\par 1 يا ابني متوشالام، استدع جميع إخوتك وأهل بيتنا وشيوخ الشعب، لكي أتحدث إليهم وأنطلق كما هو مخطط لي.

\par 2 فأسرع متوشالام ودعا إخوته ريجيم وريمان وأوخان وكرميون وجيداد وكل شيوخ الشعب أمام أبيه حنوك، وباركهم وقال لهم:

\chapter{58}

\par \textit{تعليمات أخنوخ لأبنائه.}

\par 1 استمعوا لي، يا أبنائي، اليوم

\par 2 في تلك الأيام، عندما نزل الرب إلى الأرض من أجل آدم، وزار كل خليقته التي خلقها بنفسه، بعد كل هذا خلق آدم، ودعا الرب كل وحوش الأرض، وكل الزواحف، وكل الطيور التي تطير في الهواء، وأحضرها كلها أمام وجه أبينا آدم.

\par 3 وأعطى آدم الأسماء لكل ما يعيش على الأرض.

\par 4 وجعله الرب رئيساً على الجميع، وأخضع له كل الأشياء تحت يديه، وجعلها صامتة وغير قادرة على الكلام لكي تأمر بها الناس وتكون خاضعة له وتطيعه.

\par 5 وهكذا خلق الرب أيضًا كل إنسان سيدًا على جميع ممتلكاته.

\par 6 لن يحكم الرب على نفس حيوان من أجل الإنسان، بل يحكم على نفوس البشر لحيواناتهم في هذا العالم؛ لأن البشر لديهم مكانة خاصة.

\par 7 وكما أن لكل نفس من نفس الإنسان عدد، كذلك لن تهلك الوحوش، ولا جميع أرواح الوحوش التي خلقها الرب، حتى يوم الدينونة العظيمة، وسوف يتهمون الإنسان إذا أطعمهم سوءًا

\chapter{59}

\par \textit{يُعلِّم أخنوخ أبناءه أنه لا يجوز لهم لمس لحم البقر بسبب ما يخرج منه.}

\par 1 من ينجس نفس البهائم ينجس نفسه.

\par 2 فإن الإنسان يأتي بحيوانات طاهرة ليذبحها ذبيحة للخطية، لكي يحصل على شفاء نفسه.

\par 3 وإذا قدموا للتضحية بالحيوانات الطاهرة والطيور، فإن الإنسان يشفي، يشفي نفسه.

\par 4 كل شيء يُعطى لك كطعام، اربطه بالأقدام الأربعة، أي لجعل العلاج جيدًا، فهو يشفي روحه.

\par 5 وأما من قتل بهيمة بغير جرح، فقد قتل نفسه ونجّس جسده

\par 6 ومن أذى بهيمةً أيًّا كان، سرًّا، فهو فعلٌ رديء، وهو ينجس نفسه

\chapter{60}

\par \textit{من يؤذي نفس الإنسان، يؤذي نفسه، وليس لجسده شفاء، ولا غفران إلى الأبد. كيف لا يليق قتل الإنسان لا بالسلاح ولا باللسان.}

\par 1 من يقتل نفس الإنسان، يقتل نفسه، ويقتل جسده، ولا علاج له إلى الأبد

\par 2 من يضع رجلاً في أي فخ، سيعلق فيه بنفسه، وليس له علاج إلى الأبد

\par 3 من وضع إنسانًا في أي إناء، فلن ينقصه جزاءه في يوم الدينونة العظيمة إلى الأبد

\par 4 من يعمل بشكل ملتوٍ أو يتكلم بالشر ضد أي نفس، فلن يُنصف نفسه إلى الأبد

\chapter{61}

\par \textit{يُوصي أخنوخ أبناءه بحماية أنفسهم من الظلم، وكثيراً ما يمدون أيديهم للفقراء، ليُعطوا نصيباً من أعمالهم.}

\par 1 والآن يا أبنائي، احفظوا قلوبكم من كل ظلم يبغضه الرب. فكما يطلب الإنسان من الله نفسه، فليفعل بكل نفس حية، لأني أعلم كل شيء، كيف أنه في الزمان العظيم (المقبل) تُهيأ منازل كثيرة للناس، خيرٌ للصالحين، وشرٌ للأشرار، لا يُحصى عددهم.

\par 2 طوبى للذين يدخلون البيوت الصالحة، فإنه في البيوت السيئة لا سلام ولا رجوع منها.

\par 3 اسمعوا يا أبنائي الصغار والكبار! عندما يضع الإنسان فكرة جيدة في قلبه، ويقدم عطايا تعبه أمام وجه الرب ولم تصنعها يداه، فإن الرب يحول وجهه عن تعب يديه، ولا يجد الإنسان تعب يديه.

\par 4 وإن كانت يداه هي التي صنعته، وقلبه يتذمر، وقلبه لا يكف عن التذمر بلا انقطاع، فلا فائدة له.

\chapter{62}

\par \textit{كيف يكون من المناسب أن يُحضر المرء هديته بالإيمان، لأنه لا توبة بعد الموت.}

\par 1 طوبى للرجل الذي يقدم مواهبه بصبر وإيمان أمام وجه الرب، فإنه يجد غفران الخطايا.

\par 2 فإن رجع عن كلامه قبل الوقت فلا توبة له، وإن مضى الوقت ولم يفعل ما وعد به من تلقاء نفسه فلا توبة بعد الموت.

\par 3 لأن كل عمل يعمله الإنسان قبل الوقت، هو خديعة أمام الناس، وخطيئة أمام الله.



\chapter{63}

\par \textit{كيف لا نحتقر الفقراء، بل نتقاسم معهم بالتساوي، لئلا نتذمر عليك أمام الله.}

\par 1 عندما يكسو الإنسان العريان ويشبع الجائع فإنه سيجد المكافأة من الله.

\par 2 ولكن إذا تهكم قلبه فقد أخطأ مرتين: هلاك نفسه وما أعطى، ولن يجد له جزاءً في ذلك.

\par 3 وإذا امتلأ قلبه بطعامه ولحمه بلباسه فقد ارتكب الاحتقار وخسر كل صبره على الفقر ولم يجد جزاء أعماله الصالحة.

\par 4 كل رجل متكبر ومتكبر مكروه لدى الرب، وكل كلام كاذب متسربل بالكذب يُقطع بحد سيف الموت، ويُلقى في النار، ويحترق إلى الأبد

\chapter{64}

\par \textit{كيف دعا الرب أخنوخ، وتشاور الناس للذهاب وتقبيله في المكان الذي يُدعى أخوزان.}

\par 1 عندما تكلم أخنوخ بهذه الكلمات لأبنائه، سمع جميع الناس، في كل مكان، كيف كان الرب يدعو أخنوخ. فتشاوروا معًا:

\par 2 «هلم نذهب ونقبل حنوك» فاجتمع ألفا رجل وجاءوا إلى مكان أخوزان حيث كان حنوك وأبناؤه

\par 3 فجاء شيوخ الشعب، كل الجماعة، وانحنوا وبدأوا يقبِّلون حنوك وقالوا له:

\par 4 "يا أبانا حنوك، كن مباركًا من الرب، الحاكم الأبدي، وبارك الآن أبناءك وكل الشعب، حتى نتمجد اليوم أمام وجهك.

\par 5 لأنك ستُمجَّد أمام وجه الرب إلى الأبد، لأن الرب اختارك بدلًا من جميع البشر على الأرض، وعيّنك كاتبًا لكل خليقته، المرئية وغير المرئية، وفاديًا لخطايا الإنسان، ومعينًا لأهل بيتك

\chapter{65}

\par \textit{من تعليم أخنوخ لأبنائه.}

\par 1 فأجاب أخنوخ جميع قومه قائلاً: «اسمعوا يا أبنائي، قبل أن تُخلق جميع المخلوقات، خلق الرب الأشياء المرئية وغير المرئية

\par 2 "وكلما مضى من الزمان، أدرك أنه بعد ذلك خلق الإنسان على شبه صورته، وجعل فيه عيوناً ليبصر، وآذاناً ليسمع، وقلباً ليفكر، وعقلاً ليتدبر.

\par 3 ورأى الرب جميع أعمال الإنسان، فخلق كل خلائقه، وقسم الزمان، من الزمان جعل السنين، ومن السنين عين الأشهر، ومن الشهور عين الأيام، ومن الأيام عين سبعة أيام.

\par 4 وفي تلك الساعات عينها وقاسها بدقة، لكي يتأمل الإنسان في الوقت ويحسب السنوات والأشهر والساعات وتعاقبها وبدايتها ونهايتها، ولكي يحسب حياته من البداية إلى الموت ويتأمل في خطيئته ويكتب عمله سيئًا أو جيدًا؛ لأنه ليس عمل مخفيًا أمام الرب، لكي يعرف كل إنسان أعماله ولا يتعدى جميع وصاياه، ويحفظ خط يدي إلى جيل فجيل.

\par 5 عندما تنتهي كل الخليقة المرئية وغير المرئية، كما خلقها الرب، فإن كل إنسان يذهب إلى الدينونة العظيمة، وبعد ذلك سوف يهلك كل الزمن، والسنين، ومن ثم لن يكون هناك أشهر ولا أيام ولا ساعات، سوف تلتصق ببعضها البعض ولن يتم حسابها.

\par 6 سيكون هناك دهر واحد، وجميع الصالحين الذين يهربون من دينونة الرب العظيمة، سيتم جمعهم في الدهر العظيم، لأن الصالحين سيبدأ الدهر العظيم، وسيعيشون إلى الأبد، وحينها أيضًا لن يكون بينهم عمل، ولا مرض، ولا إذلال، ولا قلق، ولا حاجة، ولا عنف، ولا ليل، ولا ظلام، بل نور عظيم.

\par 7 وسيكون لهم سور عظيم لا يهدم، وجنة منيرة لا تفنى، لأن كل الأشياء القابلة للفناء ستزول، وستكون هناك حياة أبدية

\chapter{66}

\par \textit{يُعلّم أخنوخ أبناءه وجميع شيوخ الشعب كيف يمشون برعب ورعدة أمام الرب، ويعبدونه وحده، ولا يسجدون للأصنام، بل لله الذي خلق السماء والأرض وكل خليقة، وعلى صورته.}

\par 1 والآن يا أبنائي، احفظوا أرواحكم من كل ظلم يكرهه الرب

\par 2 امشِ أمام وجهه برعب وارتجاف، واخدمه وحدك.

\par 3 اسجدوا لله الحق، لا للأصنام الصامتة، بل لصورته، وقدموا كل تقدمة صالحة أمام وجه الرب. الرب يبغض الظلم.

\par 4 لأن الرب يرى كل شيء؛ عندما يضع الإنسان فكرة في قلبه، فإنه ينصح العقول، وتكون كل فكرة دائمًا أمام الرب، الذي ثبّت الأرض ووضع عليها كل المخلوقات

\par 5 إذا نظرت إلى السماء، فالرب هناك؛ وإذا فكرت في أعماق البحر وكل ما تحت الأرض، فالرب هناك

\par 6 لأن الرب خلق كل شيء. لا تسجدوا للأشياء التي صنعها الإنسان، تاركين رب الخليقة كلها، لأنه لا يمكن لأي عمل أن يبقى مخفيًا أمام وجه الرب

\par 7 "امشوا يا أبنائي في الصبر، في الوداعة، في الصدق، في الاستفزاز، في الحزن، في الإيمان والحقيقة، في الاعتماد على الوعود، في المرض، في الإساءة، في الجروح، في الإغراء، في العري، في الحرمان، محبين بعضكم البعض، حتى تخرجوا من هذا العصر من الأمراض، لتصبحوا ورثة لزمن لا نهاية له.

\par 8 طوبى للأبرار الذين سيفلتون من الدينونة العظيمة، لأنهم سيضيئون أكثر من الشمس سبعة أضعاف، لأنه في هذا العالم يُنزع الجزء السابع من كل شيء: النور، والظلام، والطعام، والمتعة، والحزن، والجنة، والعذاب، والنار، والصقيع، وأشياء أخرى؛ وقد دوّن كل شيء، حتى تتمكنوا من القراءة والفهم

\chapter{67}

\par \textit{أطلق الرب ظلمة على الأرض وغطى الشعب وحنوك، ثم ارتفع إلى الأعالي، وعاد النور إلى السماء.}

\par 1 وبعد أن تكلم حنوك مع الشعب أرسل الرب ظلمة على الأرض، فكانت ظلمة وغطت الرجال الواقفين مع حنوك، فصعدوا حنوك إلى أعلى السماء حيث الرب، فأخذه ووضعه أمامه، فانقشعت الظلمة عن الأرض، وجاء النور أيضاً.

\par 2 ولما رأى الشعب ولم يفهموا كيف أُخذ حنوك، مجدوا الله، ووجدوا درجًا مكتوبًا فيه "الله غير المنظور"، وذهب الجميع إلى بيوتهم.

\chapter{68}

\par 1 وُلِد أخنوخ في اليوم السادس من شهر تْسيڤان، وعاش ثلاثمائة وخمسة وستين عامًا

\par 2 رُفع إلى السماء في اليوم الأول من شهر تسيفان، وبقي في السماء ستين يومًا

\par 3 كتب كل هذه العلامات لكل الخليقة التي خلقها الرب، وكتب ثلاثمائة وستة وستين سفرًا، وسلمها لأبنائه وبقي على الأرض ثلاثين يومًا، ثم رُفع مرة أخرى إلى السماء في اليوم السادس من شهر تسيفان، في نفس اليوم والساعة التي وُلد فيها

\par 4 كما أن طبيعة كل إنسان في هذه الحياة مظلمة، فكذلك هي الحال بالنسبة لحبله وولادته ومغادرته لهذه الحياة

\par 5 في أي ساعة حُبل به، وفي تلك الساعة وُلد، وفي تلك الساعة أيضًا مات

\par 6 فأسرع متوشالام وإخوته، جميع أبناء حنوك، وأقاموا مذبحًا في المكان الذي يُدعى أخوزان، وهو المكان الذي رُفع منه حنوك إلى السماء

\par 7 فأخذوا ثيرانًا للذبائح، ودعوا جميع الشعب، وذبحوا الذبيحة أمام وجه الرب

\par 8 جاء جميع الشعب وشيوخ الشعب والجماعة كلها إلى الوليمة وقدموا هدايا لأبناء حنوك

\par 9 وصنعوا وليمة عظيمة، فرحوا وابتهجوا ثلاثة أيام، وسبّحوا الله الذي أعطاهم مثل هذه الآية عن طريق حنوك الذي نال نعمة عنده، وأن يورثوها لأبنائهم من جيل إلى جيل، من جيل إلى جيل

\par 10 آمين.



\end{document}