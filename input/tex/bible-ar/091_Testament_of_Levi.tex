\begin{document}

\title{وصية لاوي}

\chapter{1}

\par \textit{لاوي، الابن الثالث ليعقوب وليا. عارفٌ بالأحلام، نبيٌّ.}

\par 1 نسخة أقوال لاوي، الأمور التي أمر بها بنيه، حسب كل ما يجب أن يفعلوه، وما سيحدث لهم إلى يوم الدين

\par 2 كان سليمًا معافى عندما دعاهم إليه؛ لأنه كان قد أوحي إليه أنه سيموت

\par 3 ولما جُمِعوا قال لهم:

\par 4 وأنا لاوي ولدت في حاران، وأتيت مع أبي إلى شكيم.

\par 5 وكنتُ شابًا في نحو عشرين سنة، حين انتقمتُ مع شمعون من حمور لأجل أختنا دينة

\par 6 وبينما كنت أرعى القطعان في آبل ماول، حل عليّ روح فهم الرب، فرأيت كل إنسان يفسد طريقه، وأن الإثم قد بنى لنفسه أسوارًا، والظلم جلس على الأبراج

\par 7 وكنتُ حزينًا على جنس بني البشر، وصليتُ إلى الرب لكي أخلص

\par 8 ثم غلبني النوم، فرأيت جبلًا عاليًا، وكنت عليه

\par 9 وإذا السماوات قد انفتحت، وقال لي ملاك الله: يا لاوي، ادخل

\par 10 ودخلتُ من السماء الأولى، فرأيتُ هناك بحرًا عظيمًا معلقًا

\par 11 ثم رأيت سماءً ثانيةً أكثر إشراقًا وتألقًا، إذ كان فيها أيضًا نورٌ لا حدود له،

\par 12 فقلت للملاك: لماذا هذا؟ فقال لي الملاك: لا تتعجب من هذا، لأنك سترى سماء أخرى أكثر إشراقًا ولا تُضاهى

\par 13 ومتى صعدت إلى هناك، تقف بالقرب من الرب، وتكون خادمًا له، وتُخبر الناس بأسراره، وتُخبر عن الذي سيخلص إسرائيل

\par 14 وبواسطتك وبواسطة يهوذا، سيظهر الرب بين الناس، ويخلص كل عرق من البشر

\par 15 ومن نصيب الرب تكون حياتك، وهو يكون لك حقلك وكرمك وثمرك وذهبك وفضتك

\par 16 فاسمع إذًا عن السماوات التي أُريت لك

\par 17 لهذا السبب، فإن الدنيا هي الكئيبة بالنسبة لك، لأنها ترى جميع أعمال البشر غير الصالحة

\par 18 وفيه نار وثلج وجليد مُهيأ ليوم القيامة، في دينونة الله العادلة؛ ففيه جميع أرواح القصاص للانتقام من البشر

\par 19 وفي الثانية جيوش الجيوش المُعدّة ليوم الدينونة، للانتقام من أرواح الخداع وبليعار

\par 20 وفوقهم القديسون.

\par 21 وفي أعالي الكل يسكن المجد العظيم، فوق كل قداسة

\par 22 في السماء المجاورة لها يوجد رؤساء الملائكة، الذين يخدمون الرب ويكفرون عنه عن جميع خطايا جهل الصالحين؛

\par 23 يُقَرِّبُونَ لِلرَّبِّ رَائِحَةً طَيِّبَةً، تَقْدِيمًا عَادِلًا بِلا دَمٍ

\par 24 وفي السماء أسفلها الملائكة الذين يحملون الإجابات لملائكة حضور الرب.

\par 25 وفي السماء التي بجانبها عروش وسيادات، فيها يُسبِّحون الله دائمًا

\par 26 لذلك، عندما ينظر الرب إلينا، نهتز جميعًا؛ نعم، تهتز السماوات والأرض والهاوية من أمام جلاله

\par 27 لكن أبناء البشر، إذ لا يدركون هذه الأمور، يخطئون ويغيظون العلي

\chapter{2}

\par \textit{ليفي يحث على التقوى والتعليم.}

\par 1 فاعلموا الآن أن الرب يجري حكما على بني البشر.

\par 2 لأنه عندما تتمزق الصخور، وتنطفئ الشمس، وتجف المياه، وتنكمش النار، وتضطرب الخليقة كلها، وتذوب الأرواح غير المرئية، ويأخذ الجحيم الغنائم من خلال زيارات العلي، فإن البشر سيظلون غير مؤمنين ويصرون على إثمهم

\par 3 على هذا الأساس سيُعاقبون.

\par 4 لذلك سمع العلي صلاتك ليفصلك عن الإثم ولكي تصير له ابنا وخادما وخادما لحضرته.

\par 5 سيُنير نور المعرفة في يعقوب، وستكون كالشمس لجميع نسل إسرائيل

\par 6 وتُعطى لك ولجميع نسلك بركة، حتى يفتقد الرب جميع الأمم برحمته إلى الأبد

\par 7 ولذلك أُعطيتَ المشورة والفهم، لكي تُعلِّم أبناءك في هذا الشأن؛

\par 8 لأن الذين يباركونه يتباركون، والذين يلعنونه يهلكون

\par 9 وعندئذٍ فتح لي الملاك أبواب السماء، فرأيت الهيكل المقدس، وعلى عرش المجد العلي

\par 10 فقال لي: يا لاوي، قد أعطيتك بركة الكهنوت حتى آتي وأتغرب في وسط إسرائيل

\par 11 ثم أنزلني الملاك إلى الأرض، وأعطاني ترسًا وسيفًا، وقال لي: انتقم من شكيم بسبب دينة أختك، فأكون معك لأن الرب أرسلني

\par 12 وأهلكتُ في ذلك الوقت بني حمور، كما هو مكتوب في الألواح السماوية

\par 13 فقلت له: يا رب، أخبرني باسمك لأدعوك في يوم الضيق

\par 14 فقال: أنا الملاك الذي يشفع لأمة إسرائيل لكي لا تُضرب ضربًا مبرحًا، لأن كل روح شرير يهاجمها

\par 15 وبعد هذه الأمور استيقظتُ، وباركت العلي، والملاك الذي يشفع لأمة إسرائيل ولجميع الأبرار

\chapter{3}

\par \textit{لدى ليوي رؤى ويُظهر المكافآت التي تنتظر الصالحين.}

\par 1 "وعندما ذهبت إلى أبي وجدت درعا من نحاس. لذلك أيضا اسم الجبل هو أسبيس الذي بالقرب من جبيل إلى الجنوب من أبيلا."

\par 2 وحفظتُ هذه الكلمات في قلبي. وبعد ذلك أشرتُ على أبي ورأوبين أخي أن يأمرا بني حمور ألا يختتنوا، لأني غارتُ بسبب الرجس الذي صنعوه بأختي.

\par 3 فقتلتُ شكيم أولًا، وقتل شمعون حمور. وبعد ذلك جاء إخوتي وضربوا تلك المدينة بحد السيف

\par 4 وسمع أبي هذه الأمور فغضب، وحزن لأنهم نالوا الختان، وبعد ذلك قُتلوا، وفي بركاته نظر إلينا بالسوء

\par 5 لأننا أخطأنا لأننا فعلنا هذا الأمر ضد إرادته، وكان مريضًا في ذلك اليوم

\par 6 ولكني رأيت أن حكم الله كان شرًا على شكيم، لأنهم حاولوا أن يفعلوا بسارة ورفقة كما فعلوا بدينة أختنا، لكن الرب منعهم

\par 7 واضطهدوا إبراهيم أبانا وهو غريب، وضايقوا غنمه وهي حامل، وأبلاء المولودة في بيته عاملوها معاملة مهينة

\par 8 وهكذا فعلوا بجميع الغرباء، أخذوا زوجاتهم بالقوة، ونفوهم

\par 9 ولكن غضب الرب جاء عليهم إلى أقصى حد.

\par 10 فقلت لأبي يعقوب: بواسطتك ينهب الرب الكنعانيين ويعطي أرضهم لك ولنسلك من بعدك.

\par 11 لأنه من هذا اليوم فصاعدًا ستُدعى شكيم مدينة البلهاء، لأنه كما يسخر الإنسان من الجاهل، كذلك سخرنا منهم

\par 12 لأنهم أيضًا صنعوا قباحة في إسرائيل بتدنيسهم أختي. فخرجنا وأتينا إلى بيت إيل

\par 13 وهناك رأيت رؤيا أخرى كالأولى، بعد أن قضينا هناك سبعين يومًا

\par 14 ورأيت سبعة رجال بثياب بيض يقولون لي: قم والبس رداء الكهنوت، وإكليل البر، وصدرة الفهم، وثوب الحق، ورداء الإيمان، وعمامة الرأس، ورداء النبوة

\par 15 فحملوا هذه الأشياء كل على حدة ووضعوها عليّ، وقالوا لي: من الآن فصاعدًا كن كاهنًا للرب أنت ونسلك إلى الأبد

\par 16 ومسحني الأول بزيت مقدس، وأعطاني عصا القضاء

\par 17 أما الثاني فقد غسلني بماء نقي، وأطعمني خبزًا وخمرًا حتى أقدس الأشياء، وألبسني ثوبًا مقدسًا ومجيدًا

\par 18 وألبسني الثالث ثوبًا من كتان يشبه الرداء.

\par 19 والرابع وضع حولي منطقة مثل الأرجوان.

\par 20 أعطاني الخامس غصن زيتون غني.

\par 21 السادس وضع تاجا على رأسي.

\par 22 ووضع السابع على رأسي إكليل كهنوت، وملأ يدي بخورًا، لأكون كاهنًا للرب الإله

\par 23 فقالوا لي: يا لاوي، سيُقسَّم نسلك إلى ثلاث وظائف، علامةً على مجد الرب الآتي

\par 24 ويكون الجزء الأول عظيمًا، نعم، لن يكون أعظم منه شيء

\par 25 والثانية تكون في الكهنوت.

\par 26 ويُدعى الثالث باسم جديد، لأنه سيقوم ملك في يهوذا، ويقيم كهنوتًا جديدًا على نمط الأمم

\par 27 وحضوره محبوب، كنبي العلي، من نسل إبراهيم أبينا

\par 28 لذلك، يكون لك ولنسلك كل ما هو شهي في إسرائيل، وتأكل كل ما هو شهي المنظر، وتقسم مائدة الرب نسلك

\par 29 ويكون بعضهم رؤساء كهنة وقضاة وكتبة، لأن القدس يُحفظ بأفواههم

\par 30 ولما استيقظت، فهمت أن هذا الحلم كان مثل الحلم الأول. فأخفيت هذا أيضًا في قلبي، ولم أخبر به أحدًا على الأرض

\par 31 وبعد يومين، صعدت أنا ويهوذا مع أبينا يعقوب إلى إسحاق أبي أبينا

\par 32 وباركني والد أبي حسب جميع أقوال الرؤى التي رأيتها. ولم يشأ أن يأتي معنا إلى بيت إيل

\par 33 ولما أتينا إلى بيت إيل، رأى أبي رؤيا عني، أن أكون لهم كاهنًا لله

\par 34 فبكر في الصباح، وأعطى عشر كل شيء للرب بواسطتي. وهكذا أتينا إلى الخليل لنقيم هناك

\par 35 وكان إسحاق يدعوني دائمًا ليذكرني بشريعة الرب، كما أراني ملاك الرب

\par 36 وعلمني ناموس كهنوت الذبائح، والمحرقات، والبواكير، والذبائح الطوعية، وذبائح السلامة

\par 37 وكان يُعلّمني كل يوم، وكان مشغولاً عني أمام الرب، وقال لي: احذر من روح الزنا؛ لأنه سيستمر، وسيُدنّس بنسلك المكان المقدس

\par 38 لذا، خذ لنفسك زوجة بلا عيب ولا دنس، وأنت لا تزال شابًا، وليست من جنس الأمم الغريبة

\par 39 وقبل دخولك إلى المكان المقدس، اغتسل، وعند تقديم الذبيحة، اغتسل، وعند الانتهاء من الذبيحة، اغتسل أيضًا

\par 40 من اثنتي عشرة شجرة ذات ورق، تُقَرِّب للرب، كما علَّمني إبراهيم أيضًا

\par 41 ومن كل البهائم الطاهرة والطيور الطاهرة تُقرِّب ذبيحة للرب.

\par 42 ومن كل بواكيرك ومن الخمر تقرب البواكير ذبيحة للرب الإله وكل ذبيحة تملّح بالملح.

\par 43 والآن، فاحفظوا ما أوصيكم به أيها الأبناء، لأني أخبرتكم بكل ما سمعته من آبائي

\par 44 وها أنا بريء من كفركم ومعصيتكم التي سترتكبونها في نهاية الدهور ضد مخلص العالم، المسيح، إذ تتصرفون بإلحاد، وتخدعون إسرائيل، وتهيجون عليها شرورًا عظيمة من الرب

\par 45 وتتصرفون بظلم مع إسرائيل، فلا يتسامح مع أورشليم بسبب شركم، بل يتمزق حجاب الهيكل، لئلا يغطي عورتكم

\par 46 "وتتشتتون بين الأمم وتكونون هناك عارا ولعنة."

\par 47 لأن البيت الذي يختاره الرب يُدعى أورشليم، كما هو وارد في سفر أخنوخ البار

\par 48 لذلك عندما تزوجت كنت في الثامنة والعشرين من عمري، وكان اسمها مَلِخَة

\par 49 فحبلت وولدت ابنًا، فدعوت اسمه جرشام، لأننا كنا غرباء في أرضنا

\par 50 ورأيت فيه أنه لن يكون في الصف الأول.

\par 51 وولد قهات في السنة الخامسة والثلاثين من حياتي عند شروق الشمس.

\par 52 ورأيت في رؤيا أنه واقف في مكان مرتفع في وسط كل الجماعة

\par 53 لذلك دعوت اسمه قهات، أي: بداية الجلال والأدب

\par 54 فولدت لي ابنًا ثالثًا، في السنة الأربعين من عمري؛ ولأن أمه ولدته بصعوبة، فقد دعوته مراري، أي "مرارتي"، لأنه كان على وشك الموت أيضًا

\par 55 وُلدت يوكابد في مصر، في سن الرابعة والستين من عمري، لأني كنت مشهورة آنذاك بين إخوتي

\par 56 واتخذ جرشام زوجة، فولدت له لمني وشمعي. وأبناء قهات: عمرام، ويساكر، وحبرون، وأوزيل. وأبناء مراري: مولي، وموسى

\par 57 وفي السنة الرابعة والتسعين، اتخذ عمرام يوكابد ابنتي زوجة له، لأنهما وُلدا في يوم واحد، هو وابنتي

\par 58 كنت ابن ثماني سنوات عندما دخلت أرض كنعان، وثماني عشرة سنة عندما قتلت شكيم، وفي التاسعة عشرة من عمري أصبحت كاهنًا، وفي الثامنة والعشرين من عمري تزوجت، وفي الثامنة والأربعين دخلت مصر

\par 59 وها أنتم يا أبنائي، أنتم الجيل الثالث. في سنتي المئة والثامنة عشرة مات يوسف

\chapter{4}

\par \textit{يُظهِر ليفي كيف تنجو الحكمة من الدمار. فهو لا يجد غضاضة في الناس المستهزئين.}

\par 1 والآن يا أبنائي، أوصيكم: اتقوا الرب إلهكم بكل قلوبكم، واسلكوا ببساطة حسب كل شريعته

\par 2 وعلّمو أولادكم أيضًا الحروف، ليكون لهم فهم طوال حياتهم، ويقرأون شريعة الله بلا انقطاع

\par 3 لأن كل من يعرف شريعة الرب يُكرَّم، ولا يكون غريبًا أينما ذهب

\par 4 نعم، سيكسب أصدقاءً كثيرين أكثر من والديه، وسيرغب كثير من الرجال في خدمته، وسماع الشريعة من فمه

\par 5 فاعملوا البر، يا أبنائي، على الأرض، ليكون لكم كنزًا في السماء

\par 6 وازرعوا الخير في نفوسكم، لتجدوه في حياتكم

\par 7 ولكن إن زرعتم شرورًا، فستحصدون كل ضيق وبلاء

\par 8 احصل على الحكمة في مخافة الله باجتهاد؛ لأنه وإن كان هناك سبي، وتدمير مدن وأراضي، وهلاك الذهب والفضة وكل ممتلكات، فإن حكمة الحكماء لا شيء يستطيع أن يسلبها، إلا عمى الإلحاد، والقسوة التي تأتي من الخطيئة

\par 9 لأنه إذا حَفِظَ الإنسان نفسه عن هذه الأمور الشريرة، فستكون الحكمة مجدًا له حتى بين أعدائه، وفي بلد غريب وطنًا، وفي وسط الأعداء صديقًا

\par 10 كل من علم أمورًا شريفة وعمل بها، سيُتوَّج مع الملوك، كما كان يوسف أخي أيضًا

\par 11 لذلك، يا أبنائي، لقد تعلمت أنه في نهاية الدهور ستعصيون الرب، وتمددون أيديكم إلى الشر ضده، وستكونون موضع ازدراء لجميع الأمم

\par 12 لأن أبانا إسرائيل طاهر من ذنوب رؤساء الكهنة [الذين سيضعون أيديهم على مخلص العالم].

\par 13 لأنه كما أن السماء أطهر في نظر الرب من الأرض، فكذلك كونوا أنتم أيضًا، أنوار إسرائيل، أطهر من جميع الأمم

\par 14 ولكن إن كنتم مظلمين بسبب التعديات، فماذا يفعل جميع الأمم وهم يعيشون في عمى؟

\par 15 نعم، ستجلبون لعنة على جنسنا، لأن نور الناموس الذي أُعطي لإضاءة كل إنسان، ترغبون في تدميره بتعليم وصايا تخالف أحكام الله

\par 16 تسلبون قرابين الرب، ومن نصيبه تسرقون نصيبًا مختارًا، وتأكلونه بازدراء مع الزواني

\par 17 ومن الطمع تُعلّمون وصايا الرب، وتُدنّسون المتزوجات، وتُنجّسون عذارى أورشليم، وترتبطون بالزواني والزواني، وتتزوجون بنات الأمم، وتُطهّرونهنّ تطهيرًا غير شرعي، ويكون اتحادكم مثل اتحاد سدوم وعمورة،

\par 18 وسترتفعون بسبب كهنوتكم، وترتفعون على الناس، وليس ذلك فقط، بل أيضًا على أوامر الله

\par 19 لأنكم ستحتقرون المقدسات بالهزء والضحك.

\par 20 لذلك فإن الهيكل الذي يختاره الرب يصبح خراباً بسبب نجاستكم وتكونون أسرى في جميع الأمم.

\par 21 وتكونون رجسًا لهم، وتنالون العار والعار الأبدي من دينونة الله العادلة

\par 22 وسيفرح بهلاكك كل من يبغضك.

\par 23 وإن لم تنالوا رحمة من إبراهيم وإسحق ويعقوب آباءنا، لم يبق من ذريتنا أحد على الأرض.

\par 24 والآن علمتُ أنكم ستضلون سبعين أسبوعًا، وتدنسون الكهنوت، وتنجسون الذبائح

\par 25 وتُبطلون الشريعة، وتُبطلون أقوال الأنبياء بالضلال الشرير

\par 26 وتضطهدون الصديقين، وتبغضون الأتقياء، وتبغضون كلام المؤمنين

\par 27 والرجل الذي يجدد الشريعة بقوة العلي، ستسمونه مخادعًا؛ وفي النهاية ستندفعون إليه لقتله، غير عارفين كرامته، آخذين دمًا بريئًا بالشر على رؤوسكم

\par 28 وتصير مقدساتكم خرابا إلى الأرض بسببه.

\par 29 ولن يكون لكم مكان طاهر، بل ستكونون بين الأمم لعنة وتشتتًا حتى يزوركم مرة أخرى، ويقبلكم برحمته بالإيمان والماء

\chapter{5}

\par \textit{يتنبأ بمجيء المسيح. كُتب هذا قبل المسيح بـ 100 عام.}

\par 1 وكما سمعتم عن السبعين أسبوعًا، فاسمعوا أيضًا عن الكهنوت. لأنه في كل يوبيل يكون كهنوت

\par 2 وفي اليوبيل الأول، يكون الأول الذي يُمسح للكهنوت عظيمًا، ويكلم الله كما يُكلم أبًا

\par 3 ويكون كهنوته كاملاً لدى الرب، وفي يوم فرحه يقوم لخلاص العالم

\par 4 في اليوبيل الثاني، يُحمل بالممسوح في حزن الأحباء، ويُكرَّم كهنوته ويُمجَّد من الجميع

\par 5 ويُقبض على الكاهن الثالث بالحزن.

\par 6 والرابع يتألم لأن الإثم يتجمع عليه جدا، وكل إسرائيل يبغض كل واحد قريبه.

\par 7 الخامس سيُمسك به الظلام. وكذلك السادس والسابع

\par 8 وفي اليوم السابع، يكون دنس لا أستطيع التعبير عنه أمام الناس، لأنهم سيعرفون من يفعل هذه الأشياء

\par 9 لذلك يُؤخذون أسرى ويصبحون غنيمة، وتُدمر أرضهم وممتلكاتهم

\par 10 وفي الأسبوع الخامس يعودون إلى أرضهم الخربة ويجددون بيت الرب

\par 11 وفي الأسبوع السابع يصيرون كهنة، وهم عبدة أوثان، زناة، محبو مال، متكبرون، خارجون عن القانون، فاسقون، مسيئون للأطفال والوحوش

\par 12 وبعد أن يأتي عقابهم من الرب، يفشل الكهنوت

\par 13 فيُقيم الرب كاهنًا جديدًا.

\par 14 وله تنكشف جميع أقوال الرب، ويجري دينونة عادلة على الأرض إلى كثرة الأيام.

\par 15 ويشرق نجمه في السماء كملك.

\par 16 أضاء نور المعرفة كما أضاءت شمس النهار، فيتعظم في العالم.

\par 17 سيشرق كالشمس على الأرض، ويزيل كل ظلمة من تحت السماء، ويكون هناك سلام في كل الأرض

\par 18 ستبتهج السماوات في أيامه، وتبتهج الأرض، وتبتهج السحب

\par 19 وتُسكب معرفة الرب على الأرض كمياه البحار

\par 20 ويفرح به ملائكة مجد وجه الرب

\par 21 ستُفتح السماوات، وسيأتي عليه التقديس من هيكل المجد، بصوت الآب كما من إبراهيم إلى إسحاق

\par 22 وينطق بمجد العلي عليه، ويحل عليه روح الفهم والتقديس في الماء.

\par 23 لأنه سيعطي عظمة الرب لأبنائه بالحق إلى الأبد

\par 24 ولن يخلفه أحد إلى الأبد.

\par 25 وفي كهنوته، يكثر الأمم في المعرفة على الأرض، ويستنيرون بنعمة الرب. وفي كهنوته، تنتهي الخطيئة، ويكفّ الأشرار عن فعل الشر.

\par 26 ويفتح أبواب الفردوس، وينزع السيف المهدد عن آدم، ويعطي القديسين أن يأكلوا من شجرة الحياة، ويكون عليهم روح القداسة

\par 27 وسيُقيّد بليعار به، وسيمنح أبنائه سلطانًا على دوس الأرواح الشريرة

\par 28 ويفرح الرب بأولاده، ويسعد بأحبائه إلى الأبد

\par 29 حينئذٍ يبتهج إبراهيم وإسحاق ويعقوب، وأفرح أنا، ويلبس جميع القديسين فرحًا

\par 30 والآن يا أبنائي، لقد سمعتم كل شيء؛ فاختاروا لأنفسكم إما النور أو الظلمة، إما شريعة الرب أو أعمال بليعار

\par 31 فأجابه بنوه قائلين: أمام الرب نسلك حسب شريعته

\par 32 فقال لهم أبوهم: الرب شاهد، وملائكته شهود، وأنتم شهود، وأنا شاهد على كلام فمكم

\par 33 فقال له بنوه: نحن شهود.

\par 34 وهكذا كف لاوي عن أن يأمر بنيه، ومد رجليه على السرير وانضم إلى آبائه بعدما عاش مائة وسبعا وثلاثين سنة.

\par 35 فوضعوه في نعش، وبعد ذلك دفنوه في حبرون مع إبراهيم وإسحاق ويعقوب


\end{document}