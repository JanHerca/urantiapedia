\begin{document}

\title{عهد جاد}

\chapter{1}

\par \textit{جاد، الابن التاسع ليعقوب وزلفة. راعي غنم ورجل قوي لكنه قاتل في القلب. الآية 25 هي تعريف بارز للكراهية.}

\par 1 نسخة وصية جاد، ما كلم به بنيه، في السنة المائة والخامسة والعشرين من حياته، قائلاً لهم:

\par 2 اسمعوا يا أبنائي، كنتُ الابن التاسع ليعقوب، وكنتُ شجاعًا في رعاية القطعان

\par 3 لذلك، كنت أحرس القطيع ليلًا، وكلما جاء أسد، أو ذئب، أو أي وحش بري إلى الحظيرة، كنت أطارده، وعندما أدركته، أمسكت بقدمه بيدي وقذفته على مرمى حجر، فقتلته

\par 4 وكان يوسف أخي يرعى الغنم معنا أكثر من ثلاثين يومًا، ولما كان صغيرًا مرض من الحر

\par 5 ورجع إلى الخليل إلى أبينا، فأضجعه عنده لأنه كان يحبه كثيرًا

\par 6 وأخبر يوسف أبانا أن ابني زلفة وبلهة كانا يقتلان خير الغنم ويأكلانها احتجاجًا على حكم رأوبين ويهوذا

\par 7 لأنه رأى أنني أنقذت خروفًا من فم دب، وقتلت الدب، ولكني ذبحت الخروف، حزنًا عليه لأنه لا يستطيع أن يعيش، فأكلناه

\par 8 وفي هذا الأمر غضبتُ على يوسف إلى يوم بيعه

\par 9 وكانت روح البغضاء فيّ، فلم أُرِد أن أسمع عن يوسف بالأذنين، ولا أن أراه بعيني، لأنه وبخنا في وجوهنا قائلاً إننا نأكل من غنم دون يهوذا

\par 10 لأنه كان يؤمن بكل ما قاله لأبينا.

\par 11 أعترف الآن لأطفالي، ولأبنائي، أنني في كثير من الأحيان كنت أتمنى قتله، لأنني كنت أكرهه من قلبي.

\par 12 علاوة على ذلك، كرهته أكثر بسبب أحلامه؛ وتمنيت أن ألعقه من أرض الأحياء، كما يلحس الثور عشب الحقل

\par 13 وباعه يهوذا سرًا للإسماعيليين.

\par 14 فأنقذه إله آبائنا من أيدينا لكي لا نعمل إثماً عظيماً في إسرائيل.

\par 15 والآن يا أبنائي، أصغوا إلى كلمات الحق لتعملوا بالبر، وكل شريعة العلي، ولا تضلوا بروح الكراهية، لأنها شريرة في كل أفعال البشر

\par 16 كل ما يفعله الإنسان يكرهه المبغض. وإن عمل الإنسان بشريعة الرب فلا يمدحه. وإن اتقى الإنسان الرب وسر بالبر فلا يحبه

\par 17 إنه يحتقر الحق، ويحسد الناجح، ويرحب بالتجديف، ويحب الكبرياء، لأن الكراهية تعمي نفسه؛ كما نظرتُ أنا أيضًا آنذاك إلى يوسف

\par 18 احذروا إذن يا أبنائي من الكراهية، لأنها تُسبب الفوضى حتى ضد الرب نفسه

\par 19 لأنه لا يسمع أقوال وصاياه فيما يتعلق بمحبة القريب، ويخطئ إلى الله.

\par 20 لأنه إن عثر أخ، فإنه يسره أن يُعلن ذلك لجميع الناس في الحال، ويلزم أن يُحاكم على ذلك ويُعاقب ويُقتل

\par 21 وإن كان عبدًا، فإنه يُثيره ضد سيده، ويُدبّر ضده كل ضيق، إن أمكن قتله

\par 22 لأن الكراهية تعمل مع الحسد أيضًا ضد الناجحين: فما دامت تسمع أو ترى نجاحهم فإنها تذبل دائمًا

\par 23 فكما أن الحب يُحيي حتى الموتى، ويُعيد المحكوم عليهم بالموت، كذلك فإن الكراهية تقتل الأحياء، وأولئك الذين أخطأوا خطأً طفيفًا لن تسمح لهم بالعيش

\par 24 لأن روح الكراهية تعمل مع الشيطان، من خلال تسرع الأرواح، في كل شيء لموت البشر؛ أما روح المحبة فتعمل مع ناموس الله في طول الأناة لخلاص البشر

\par 25 لذلك، فإن الكراهية شريرة، لأنها تتزاوج باستمرار مع الكذب، والتحدث ضد الحقيقة؛ وتجعل الأشياء الصغيرة عظيمة، وتجعل النور ظلامًا، وتسمي الحلو مرًا، وتعلم الافتراء، وتشعل الغضب، وتثير الحرب والعنف وكل طمع؛ إنها تملأ القلب بالشرور والسموم الشيطانية

\par 26 لذلك، أقول لكم هذه الأشياء من واقع تجربتي، يا أبنائي، حتى تطردوا الكراهية التي من الشيطان، وتتمسكوا بمحبة الله

\par 27 البر يطرد الكراهية، والتواضع يدمر الحسد.

\par 28 لأن البار والمتواضع يخجل من فعل الظلم، إذ لا يوبخ من قبل الآخرين، بل من قبل قلبه، لأن الرب ينظر إلى ميوله.

\par 29 لا يتكلم ضد رجل قديس، لأن مخافة الله تغلب الكراهية

\par 30 لأنه يخشى أن يسيء إلى الرب، فلن يسيء إلى أي إنسان، حتى في الفكر

\par 31 هذه الأشياء تعلمتها أخيرًا، بعد أن تبت عن يوسف

\par 32 لأن التوبة الصادقة على طريقة الله تدمر الجهل، وتطرد الظلمة، وتنير العيون، وتعطي المعرفة للنفس، وتقود العقل إلى الخلاص

\par 33 وتلك الأشياء التي لم يتعلمها من الإنسان، يعرفها بالتوبة

\par 34 لأن الله جلب عليّ مرضًا في الكبد، ولولا صلوات أبي يعقوب أنقذتني، لما فشل المرض، بل فارقت روحي

\par 35 لأن ما يتعداه الإنسان يُعاقب به أيضًا

\par 36 وبما أن كبدي قد أُلقي بلا رحمة على يوسف، فقد عانيت في كبدي أيضًا بلا رحمة، وحُكم عليّ لمدة أحد عشر شهرًا، وهي المدة التي كنت غاضبًا فيها على يوسف

\par \textit{الحواشي السفلية}

\par \textit{254:1 حتى لغتنا العامية الحالية عمرها قرون.}

\chapter{2}

يحثّ جاد مستمعيه على تجنّب الكراهية، موضحًا كيف أوقعته في مشاكل جمّة. الآيات من ٨ إلى ١١ لا تُنسى.

\par 1 والآن يا أبنائي، أحثكم على أن يحب كل واحد منكم أخاه، وأن يزيل الكراهية من قلوبكم، وأن يحب بعضكم بعضًا في الفعل، وفي القول، وفي نية النفس

\par 2 لأني في حضرة أبي تكلمت مع يوسف بسلام، وعندما خرجت، أظلمت روح الكراهية في ذهني، وأثارت نفسي لأقتله

\par 3 أحبوا بعضكم بعضًا من القلب. وإن أخطأ إليكم أحد، فتكلموا معه بسلام، ولا تكن لكم مكرًا في أنفسكم. وإن تاب واعترف، فاغفر له

\par 4 ولكن إذا أنكر ذلك، فلا تدخل في شجار معه، لئلا تلتقط السم الذي قد يلحقه بك بسبب شتمه، فتُخطئ إثمًا مضاعفًا

\par 5 لا تدع رجلاً آخر يسمع أسرارك عندما تنخرط في نزاع قانوني، لئلا يكرهك ويصبح عدوك، ويرتكب خطيئة عظيمة ضدك؛ لأنه غالبًا ما يخاطبك بمكر أو ينشغل بك بنية شريرة

\par 6 وحتى لو أنكر ذلك ومع ذلك يشعر بالخجل عند توبيخه، فتوقف عن توبيخه

\par 7 فمن ينكر فليتوب لئلا يخطئ إليك مرة أخرى؛ بل ليكرمك أيضًا ويخافك ويسكن في سلام معك

\par 8 وإن كان وقحًا وأصر على ظلمه، فاغفر له من قلبك، واترك المنتقم لله

\par 9 إن كان أحدٌ ينجح أكثر منك، فلا تحزن، بل صلِّ أيضًا لأجله، لكي يكون له نجاحٌ كامل

\par 10 لأنه هكذا خير لكم.

\par 11 وإن ارتفع أكثر، فلا تحسدوه، متذكرين أن كل ذي جسد سيموت، وقدموا الحمد لله الذي يعطي الخير والنافع لجميع الناس

\par 12 اطلب أحكام الرب، فيرتاح عقلك ويطمئن

\par 13 وإن اغتنى رجل بوسائل شريرة، مثل عيسو أخي أبي، فلا يغار، بل ينتظر نهاية الرب

\par 14 لأنه إذا أخذ من إنسان ثروة اكتسبها بطرق شريرة، فإنه يغفر له إذا تاب، أما غير التائب فيُحفظ للعقاب الأبدي

\par 15 لأن الفقير، إذا كان خاليًا من الحسد، يُرضي الرب في كل شيء، يكون مباركًا أكثر من جميع الناس، لأنه لا يعاني من تعب الناس الباطلين

\par 16 لذلك، انزعوا الغيرة من نفوسكم، وأحبوا بعضكم بعضًا باستقامة قلب

\par 17 فأخبروا أولادكم أيضًا بهذه الأمور، حتى يكرمون يهوذا ولاوي، لأنه منهما يقيم الرب خلاصًا لإسرائيل

\par 18 لأني أعلم أنه في النهاية سيبتعد أولادكم عنه، ويسلكون في الشر والضيق والفساد أمام الرب

\par 19 وبعد أن استراح قليلًا، قال أيضًا: يا أبنائي، أطيعوا أبيكم وادفنوني بقرب آبائي

\par 20 ورفع قدميه ونام بسلام.

\par 21 وبعد خمس سنين حملوه إلى حبرون وأضجعوه مع آبائه.



\end{document}