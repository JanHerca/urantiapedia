\begin{document}

\title{مراثي}


\chapter{1}

\par 1 كَيْفَ جَلَسَتْ وَحْدَهَا الْمَدِينَةُ الْكَثِيرَةُ الشَّعْبِ؟ كَيْفَ صَارَتْ كَأَرْمَلَةٍ الْعَظِيمَةُ فِي الأُمَمِ؟ السَّيِّدَةُ في الْبُلْدَانِ صَارَتْ تَحْتَ الْجِزْيَةِ!
\par 2 تَبْكِي في اللَّيْلِ بُكَاءً وَدُمُوعُهَا علَى خَدَّيْهَا. لَيْسَ لَهَا مُعَزٍّ مِن كُلِّ مُحِبِّيهَا. كُلُّ أَصْحَابِهَا غَدَرُوا بِهَا. صَارُوا لهَا أَعْدَاءً.
\par 3 قَد سُبِيَتْ يَهُوذَا مِنَ الْمَذَلَّةِ وَمِنْ كَثْرَةِ الْعُبُودِيَّةِ. هِيَ تَسْكُنُ بَيْنَ الأُمَمِ. لاَ تَجِدُ رَاحَةً. قَدْ أَدْرَكَهَا كُلُّ طَارِدِيهَا بَيْنَ الضِّيقَاتِ.
\par 4 طُرُقُ صِهْيَوْنَ نَائِحَةٌ لِعَدَمِ الآتِينَ إِلَى الْعِيدِ. كُلُّ أَبْوَابِهَا خَرِبَةٌ. كَهَنَتُهَا يَتَنَهَّدُونَ. عَذَارَاهَا مُذَلَّلَةٌ وَهِيَ فِي مَرَارَةٍ.
\par 5 صَارَ مُضَايِقُوهَا رَأْساً. نَجَحَ أَعْدَاؤُهَا لأَنَّ الرَّبَّ قَدْ أَذَلَّهَا لأَجْلِ كَثْرَةِ ذُنُوبِهَا. ذَهَبَ أَوْلاَدُهَا إِلَى السَّبْيِ قُدَّامَ الْعَدُوِّ.
\par 6 وَقَدْ خَرَجَ مِنْ بِنْتِ صِهْيَوْنَ كُلُّ بَهَائِهَا. صَارَتْ رُؤَسَاؤُهَا كَأَيَائِلَ لاَ تَجِدُ مَرْعًى فَيَسِيرُونَ بِلاَ قُوَّةٍ أَمَامَ الطَّارِدِ.
\par 7 قَدْ ذَكَرَتْ أُورُشَلِيمُ فِي أَيَّامِ مَذَلَّتِهَا وَتَطَوُّحِهَا كُلَّ مُشْتَهَيَاتِهَا الَّتِي كَانَتْ فِي أَيَّامِ الْقِدَمِ. عِنْدَ سُقُوطِ شَعْبِهَا بِيَدِ الْعَدُوِّ وَلَيْسَ مَنْ يُسَاعِدُهَا. رَأَتْهَا الأَعْدَاءُ. ضَحِكُوا عَلَى هَلاَكِهَا.
\par 8 قَدْ أَخْطَأَتْ أُورُشَلِيمُ خَطِيَّةً مِنْ أَجْلِ ذَلِكَ صَارَتْ رَجِسَةً. كُلُّ مُكَرِّمِيهَا يَحْتَقِرُونَهَا لأَنَّهُمْ رَأُوا عَوْرَتَهَا وَهِيَ أَيْضاً تَتَنَهَّدُ وَتَرْجِعُ إِلَى الْوَرَاءِ.
\par 9 نَجَاسَتُهَا فِي أَذْيَالِهَا. لَمْ تَذْكُرْ آخِرَتَهَا وَقَدِ انْحَطَّتِ انْحِطَاطاً عَجِيباً. لَيْسَ لَهَا مُعَزٍّ. انْظُرْ يَا رَبُّ إِلَى مَذَلَّتِي لأَنَّ الْعَدُوَّ قَدْ تَعَظَّمَ.
\par 10 بَسَطَ الْعَدُوُّ يَدَهُ عَلَى كُلِّ مُشْتَهَيَاتِهَا فَإِنَّهَا رَأَتِ الأُمَمَ دَخَلُوا مَقْدِسَهَا الَّذِينَ أَمَرْتَ أَنْ لاَ يَدْخُلُوا فِي جَمَاعَتِكَ.
\par 11 كُلُّ شَعْبِهَا يَتَنَهَّدُونَ يَطْلُبُونَ خُبْزاً. دَفَعُوا مُشْتَهَيَاتِهِمْ لِلأَكْلِ لأَجْلِ رَدِّ النَّفْسِ. «انْظُرْ يَا رَبُّ وَتَطَلَّعْ لأَنِّي قَدْ صِرْتُ مُحْتَقَرَةً].
\par 12 «أَمَا إِلَيْكُمْ يَا جَمِيعَ عَابِرِي الطَّرِيقِ؟ تَطَلَّعُوا وَانْظُرُوا إِنْ كَانَ حُزْنٌ مِثْلُ حُزْنِي الَّذِي صُنِعَ بِي الَّذِي أَذَلَّنِي بِهِ الرَّبُّ يَوْمَ حُمُوِّ غَضَبِهِ.
\par 13 مِنَ الْعَلاَءِ أَرْسَلَ نَاراً إِلَى عِظَامِي فَسَرَتْ فِيهَا. بَسَطَ شَبَكَةً لِرِجْلَيَّ. رَدَّنِي إِلَى الْوَرَاءِ. جَعَلَنِي خَرِبَةً. الْيَوْمَ كُلَّهُ مَغْمُومَةً.
\par 14 شَدَّ نِيرَ ذُنُوبِي بِيَدِهِ. ضُفِرَتْ. صَعِدَتْ عَلَى عُنُقِي. نَزَعَ قُوَّتِي. دَفَعَنِي السَّيِّدُ إِلَى أَيْدٍ لاَ أَسْتَطِيعُ الْقِيَامَ مِنْهَا.
\par 15 رَذَلَ السَّيِّدُ كُلَّ مُقْتَدِرِيَّ فِي وَسَطِي. دَعَا عَلَيَّ جَمَاعَةً لِحَطْمِ شُبَّانِي. دَاسَ السَّيِّدُ الْعَذْرَاءَ بِنْتَ يَهُوذَا مِعْصَرَةً.
\par 16 عَلَى هَذِهِ أَنَا بَاكِيَةٌ. عَيْنِي عَيْنِي تَسْكُبُ مِيَاهاً لأَنَّهُ قَدِ ابْتَعَدَ عَنِّي الْمُعَزِّي رَادُّ نَفْسِي. صَارَ بَنِيَّ هَالِكِينَ لأَنَّهُ قَدْ تَجَبَّرَ الْعَدُوُّ].
\par 17 بَسَطَتْ صِهْيَوْنُ يَدَيْهَا. لاَ مُعَزِّيَ لَهَا. أَمَرَ الرَّبُّ عَلَى يَعْقُوبَ أَنْ يَكُونَ مُضَايِقُوهُ حَوَالَيْهِ. صَارَتْ أُورُشَلِيمُ نَجِسَةً بَيْنَهُمْ.
\par 18 بَارٌّ هُوَ الرَّبُّ لأَنِّي قَدْ عَصِيتُ أَمْرَهُ. اسْمَعُوا يَا جَمِيعَ الشُّعُوبِ وَانْظُرُوا إِلَى حُزْنِي. عَذَارَايَ وَشُبَّانِي ذَهَبُوا إِلَى السَّبْيِ.
\par 19 نَادَيْتُ مُحِبِّيَّ. هُمْ خَدَعُونِي. كَهَنَتِي وَشُيُوخِي فِي الْمَدِينَةِ مَاتُوا إِذْ طَلَبُوا لِذَوَاتِهِمْ طَعَاماً لِيَرُدُّوا أَنْفُسَهُمْ.
\par 20 انْظُرْ يَا رَبُّ فَإِنِّي فِي ضِيقٍ. أَحْشَائِي غَلَتْ. ارْتَدَّ قَلْبِي فِي بَاطِنِي لأَنِّي قَدْ عَصِيتُ مُتَمَرِّدَةً. فِي الْخَارِجِ يَثْكُلُ السَّيْفُ وَفِي الْبَيْتِ مِثْلُ الْمَوْتِ.
\par 21 سَمِعُوا أَنِّي تَنَهَّدْتُ. لاَ مُعَزِّيَ لِي. كُلُّ أَعْدَائِي سَمِعُوا بِبَلِيَّتِي. فَرِحُوا لأَنَّكَ فَعَلْتَ. تَأْتِي بِالْيَوْمِ الَّذِي نَادَيْتَ بِهِ فَيَصِيرُونَ مِثْلِي.
\par 22 لِيَأْتِ كُلُّ شَرِّهِمْ أَمَامَكَ. وَافْعَلْ بِهِمْ كَمَا فَعَلْتَ بِي مِنْ أَجْلِ كُلِّ ذُنُوبِي لأَنَّ تَنَهُّدَاتِي كَثِيرَةٌ وَقَلْبِي مَغْشِيٌّ عَلَيْهِ.

\chapter{2}

\par 1 كَيْفَ غَطَّى السَّيِّدُ بِغَضَبِهِ ابْنَةَ صِهْيَوْنَ بِالظَّلاَمِ؟ أَلْقَى مِنَ السَّمَاءِ إِلَى الأَرْضِ فَخْرَ إِسْرَائِيلَ وَلَمْ يَذْكُرْ مَوْطِئَ قَدَمَيْهِ فِي يَوْمِ غَضَبِهِ.
\par 2 ابْتَلَعَ السَّيِّدُ وَلَمْ يُشْفِقْ كُلَّ مَسَاكِنِ يَعْقُوبَ. نَقَضَ بِسَخَطِهِ حُصُونَ بِنْتِ يَهُوذَا. أَوْصَلَهَا إِلَى الأَرْضِ. نَجَّسَ الْمَمْلَكَةَ وَرُؤَسَاءَهَا.
\par 3 عَضَبَ بِحُمُوِّ غَضَبِهِ كُلَّ قَرْنٍ لإِسْرَائِيلَ. رَدَّ إِلَى الْوَرَاءِ يَمِينَهُ أَمَامَ الْعَدُوِّ وَاشْتَعَلَ فِي يَعْقُوبَ مِثْلَ نَارٍ مُلْتَهِبَةٍ تَأْكُلُ مَا حَوَالَيْهَا.
\par 4 مَدَّ قَوْسَهُ كَعَدُوٍّ. نَصَبَ يَمِينَهُ كَمُبْغِضٍ وَقَتَلَ كُلَّ مُشْتَهَيَاتِ الْعَيْنِ فِي خِبَاءِ بِنْتِ صِهْيَوْنَ. سَكَبَ كَنَارٍ غَيْظَهُ.
\par 5 صَارَ السَّيِّدُ كَعَدُوٍّ. ابْتَلَعَ إِسْرَائِيلَ. ابْتَلَعَ كُلَّ قُصُورِهِ. أَهْلَكَ حُصُونَهُ وَأَكْثَرَ فِي بِنْتِ يَهُوذَا النَّوْحَ وَالْحُزْنَ.
\par 6 وَنَزَعَ كَمَا مِنْ جَنَّةٍ مَظَلَّتَهُ. أَهْلَكَ مُجْتَمَعَهُ. أَنْسَى الرَّبُّ فِي صِهْيَوْنَ الْمَوْسِمَ وَالسَّبْتَ وَرَذَلَ بِسَخَطِ غَضَبِهِ الْمَلِكَ وَالْكَاهِنَ.
\par 7 كَرِهَ السَّيِّدُ مَذْبَحَهُ. رَذَلَ مَقْدِسَهُ. حَصَرَ فِي يَدِ الْعَدُوِّ أَسْوَارَ قُصُورِهَا. أَطْلَقُوا الصَّوْتَ فِي بَيْتِ الرَّبِّ كَمَا فِي يَوْمِ الْمَوْسِمِ.
\par 8 قَصَدَ الرَّبُّ أَنْ يُهْلِكَ سُورَ بِنْتِ صِهْيَوْنَ. مَدَّ الْمِطْمَارَ. لَمْ يَرْدُدْ يَدَهُ عَنِ الإِهْلاَكِ وَجَعَلَ الْمِتْرَسَةَ وَالسُّورَ يَنُوحَانِ. قَدْ حَزِنَا مَعاً.
\par 9 تَاخَتْ فِي الأَرْضِ أَبْوَابُهَا. أَهْلَكَ وَحَطَّمَ عَوَارِضَهَا. مَلِكُهَا وَرُؤَسَاؤُهَا بَيْنَ الأُمَمِ. لاَ شَرِيعَةَ. أَنْبِيَاؤُهَا أَيْضاً لاَ يَجِدُونَ رُؤْيَا مِنْ قِبَلِ الرَّبِّ.
\par 10 شُيُوخُ بِنْتِ صِهْيَوْنَ يَجْلِسُونَ عَلَى الأَرْضِ سَاكِتِينَ. يَرْفَعُونَ التُّرَابَ عَلَى رُؤُوسِهِمْ. يَتَنَطَّقُونَ بِالْمُسُوحِ. تَحْنِي عَذَارَى أُورُشَلِيمَ رُؤُوسَهُنَّ إِلَى الأَرْضِ.
\par 11 كَلَّتْ مِنَ الدُّمُوعِ عَيْنَايَ. غَلَتْ أَحْشَائِي. انْسَكَبَتْ عَلَى الأَرْضِ كَبِدِي عَلَى سَحْقِ بِنْتِ شَعْبِي لأَجْلِ غَشَيَانِ الأَطْفَالِ وَالرُّضَّعِ فِي سَاحَاتِ الْقَرْيَةِ.
\par 12 يَقُولُونَ لِأُمَّهَاتِهِمْ: «أَيْنَ الْحِنْطَةُ وَالْخَمْرُ؟] إِذْ يُغْشَى عَلَيْهِمْ كَجَرِيحٍ فِي سَاحَاتِ الْمَدِينَةِ إِذْ تُسْكَبُ نَفْسُهُمْ فِي أَحْضَانِ أُمَّهَاتِهِمْ.
\par 13 بِمَاذَا أُنْذِرُكِ بِمَاذَا أُحَذِّرُكِ؟ بِمَاذَا أُشَبِّهُكِ يَا ابْنَةَ أُورُشَلِيمَ؟ بِمَاذَا أُقَايِسُكِ فَأُعَزِّيكِ أَيَّتُهَا الْعَذْرَاءُ بِنْتَ صِهْيَوْنَ؟ لأَنَّ سَحْقَكِ عَظِيمٌ كَالْبَحْرِ. مَنْ يَشْفِيكِ؟
\par 14 أَنْبِيَاؤُكِ رَأُوا لَكِ كَذِباً وَبَاطِلاً وَلَمْ يُعْلِنُوا إِثْمَكِ لِيَرُدُّوا سَبْيَكِ بَلْ رَأُوا لَكِ وَحْياً كَاذِباً وَطَوَائِحَ.
\par 15 يُصَفِّقُ عَلَيْكِ بِالأَيَادِي كُلُّ عَابِرِي الطَّرِيقِ. يَصْفِرُونَ وَيُنْغِضُونَ رُؤُوسَهُمْ عَلَى بِنْتِ أُورُشَلِيمَ قَائِلِينَ: «أَهَذِهِ هِيَ الْمَدِينَةُ الَّتِي يَقُولُونَ إِنَّهَا كَمَالُ الْجَمَالِ بَهْجَةُ كُلِّ الأَرْضِ؟]
\par 16 يَفْتَحُ عَلَيْكِ أَفْوَاهَهُمْ كُلُّ أَعْدَائِكِ. يَصْفِرُونَ وَيُحْرِقُونَ الأَسْنَانَ. يَقُولُونَ: «قَدْ أَهْلَكْنَاهَا. حَقّاً إِنَّ هَذَا الْيَوْمَ الَّذِي رَجَوْنَاهُ. قَدْ وَجَدْنَاهُ! قَدْ رَأَيْنَاهُ].
\par 17 فَعَلَ الرَّبُّ مَا قَصَدَ. تَمَّمَ قَوْلَهُ الَّذِي أَوْعَدَ بِهِ مُنْذُ أَيَّامِ الْقِدَمِ. قَدْ هَدَمَ وَلَمْ يُشْفِقْ وَأَشْمَتَ بِكِ الْعَدُوَّ. نَصَبَ قَرْنَ أَعْدَائِكِ.
\par 18 صَرَخَ قَلْبُهُمْ إِلَى السَّيِّدِ. يَا سُورَ بِنْتِ صِهْيَوْنَ اسْكُبِي الدَّمْعَ كَنَهْرٍ نَهَاراً وَلَيْلاً. لاَ تُعْطِي ذَاتَكِ رَاحَةً. لاَ تَكُفَّ حَدَقَةُ عَيْنِكِ.
\par 19 قُومِي اهْتِفِي فِي اللَّيْلِ فِي أَوَّلِ الْهُزُعِ. اسْكُبِي كَمِيَاهٍ قَلْبَكِ قُبَالَةَ وَجْهِ السَّيِّدِ. ارْفَعِي إِلَيْهِ يَدَيْكِ لأَجْلِ نَفْسِ أَطْفَالِكِ الْمَغْشِيِّ عَلَيْهِمْ مِنَ الْجُوعِ فِي رَأْسِ كُلِّ شَارِعٍ.
\par 20 اُنْظُرْ يَا رَبُّ وَتَطَلَّعْ بِمَنْ فَعَلْتَ هَكَذَا. أَتَأْكُلُ النِّسَاءُ ثَمَرَهُنَّ أَطْفَالَ الْحَضَانَةِ؟ أَيُقْتَلُ فِي مَقْدِسِ السَّيِّدِ الْكَاهِنُ وَالنَّبِيُّ؟
\par 21 اضْطَجَعَتْ عَلَى الأَرْضِ فِي الشَّوَارِعِ الصِّبْيَانُ وَالشُّيُوخُ. عَذَارَايَ وَشُبَّانِي سَقَطُوا بِالسَّيْفِ. قَدْ قَتَلْتَ فِي يَوْمِ غَضَبِكَ. ذَبَحْتَ وَلَمْ تُشْفِقْ.
\par 22 قَدْ دَعَوْتَ كَمَا فِي يَوْمِ مَوْسِمٍ مَخَاوِفِي حَوَالَيَّ فَلَمْ يَكُنْ فِي يَوْمِ غَضَبِ الرَّبِّ نَاجٍ وَلاَ بَاقٍ. الَّذِينَ حَضَنْتُهُمْ وَرَبَّيْتُهُمْ أَفْنَاهُمْ عَدُوِّي.

\chapter{3}

\par 1 أَنَا هُوَ الرَّجُلُ الَّذِي رأَى مَذَلَّةً بِقَضِيبِ سَخَطِهِ.
\par 2 قَادَنِي وَسَيَّرَنِي فِي الظَّلاَمِ وَلاَ نُورَ.
\par 3 حَقّاً إِنَّهُ يَعُودُ وَيَرُدُّ عَلَيَّ يَدَهُ الْيَوْمَ كُلَّهُ.
\par 4 أَبْلَى لَحْمِي وَجِلْدِي. كَسَّرَ عِظَامِي.
\par 5 بَنَى عَلَيَّ وَأَحَاطَنِي بِعَلْقَمٍ وَمَشَقَّةٍ.
\par 6 أَسْكَنَنِي فِي ظُلُمَاتٍ كَمَوْتَى الْقِدَمِ.
\par 7 سَيَّجَ عَلَيَّ فَلاَ أَسْتَطِيعُ الْخُرُوجَ. ثَقَّلَ سِلْسِلَتِي.
\par 8 أَيْضاً حِينَ أَصْرُخُ وَأَسْتَغِيثُ يَصُدُّ صَلاَتِي.
\par 9 سَيَّجَ طُرُقِي بِحِجَارَةٍ مَنْحُوتَةٍ. قَلَبَ سُبُلِي.
\par 10 هُوَ لِي دُبٌّ كَامِنٌ أَسَدٌ فِي مَخَابِئَ.
\par 11 مَيَّلَ طُرُقِي وَمَزَّقَنِي. جَعَلَنِي خَرَاباً.
\par 12 مَدَّ قَوْسَهُ وَنَصَبَنِي كَغَرَضٍ لِلسَّهْمِ.
\par 13 أَدْخَلَ فِي كُلْيَتَيَّ نِبَالَ جُعْبَتِهِ.
\par 14 صِرْتُ ضِحْكَةً لِكُلِّ شَعْبِي وَأُغْنِيَةً لَهُمُ الْيَوْمَ كُلَّهُ.
\par 15 أَشْبَعَنِي مَرَائِرَ وَأَرْوَانِي أَفْسَنْتِيناً
\par 16 وَجَرَشَ بِالْحَصَى أَسْنَانِي. كَبَسَنِي بِالرَّمَادِ.
\par 17 وَقَدْ أَبْعَدْتَ عَنِ السَّلاَمِ نَفْسِي. نَسِيتُ الْخَيْرَ.
\par 18 وَقُلْتُ: بَادَتْ ثِقَتِي وَرَجَائِي مِنَ الرَّبِّ.
\par 19 ذِكْرُ مَذَلَّتِي وَتَيَهَانِي أَفْسَنْتِينٌ وَعَلْقَمٌ.
\par 20 ذِكْراً تَذْكُرُ نَفْسِي وَتَنْحَنِي فِيَّ.
\par 21 أُرَدِّدُ هَذَا فِي قَلْبِي مِنْ أَجْلِ ذَلِكَ أَرْجُو.
\par 22 إِنَّهُ مِنْ إِحْسَانَاتِ الرَّبِّ أَنَّنَا لَمْ نَفْنَ لأَنَّ مَرَاحِمَهُ لاَ تَزُولُ.
\par 23 هِيَ جَدِيدَةٌ فِي كُلِّ صَبَاحٍ. كَثِيرَةٌ أَمَانَتُكَ.
\par 24 نَصِيبِي هُوَ الرَّبُّ قَالَتْ نَفْسِي مِنْ أَجْلِ ذَلِكَ أَرْجُوهُ.
\par 25 طَيِّبٌ هُوَ الرَّبُّ لِلَّذِينَ يَتَرَجُّونَهُ لِلنَّفْسِ الَّتِي تَطْلُبُهُ.
\par 26 جَيِّدٌ أَنْ يَنْتَظِرَ الإِنْسَانُ وَيَتَوَقَّعَ بِسُكُوتٍ خَلاَصَ الرَّبِّ.
\par 27 جَيِّدٌ لِلرَّجُلِ أَنْ يَحْمِلَ النِّيرَ فِي صِبَاهُ.
\par 28 يَجْلِسُ وَحْدَهُ وَيَسْكُتُ لأَنَّهُ قَدْ وَضَعَهُ عَلَيْهِ.
\par 29 يَجْعَلُ فِي التُّرَابِ فَمَهُ لَعَلَّهُ يُوجَدُ رَجَاءٌ.
\par 30 يُعْطِي خَدَّهُ لِضَارِبِهِ. يَشْبَعُ عَاراً.
\par 31 لأَنَّ السَّيِّدَ لاَ يَرْفُضُ إِلَى الأَبَدِ.
\par 32 فَإِنَّهُ وَلَوْ أَحْزَنَ يَرْحَمُ حَسَبَ كَثْرَةِ مَرَاحِمِهِ.
\par 33 لأَنَّهُ لاَ يُذِلُّ مِنْ قَلْبِهِ وَلاَ يُحْزِنُ بَنِي الإِنْسَانِ.
\par 34 أَنْ يَدُوسَ أَحَدٌ تَحْتَ رِجْلَيْهِ كُلَّ أَسْرَى الأَرْضِ
\par 35 أَنْ يُحَرِّفَ حَقَّ الرَّجُلِ أَمَامَ وَجْهِ الْعَلِيِّ
\par 36 أَنْ يَقْلِبَ الإِنْسَانَ فِي دَعْوَاهُ - السَّيِّدُ لاَ يَرَى!
\par 37 مَنْ ذَا الَّذِي يَقُولُ فَيَكُونَ وَالرَّبُّ لَمْ يَأْمُرْ؟
\par 38 مِنْ فَمِ الْعَلِيِّ أَلاَ تَخْرُجُ الشُّرُورُ وَالْخَيْرُ؟
\par 39 لِمَاذَا يَشْتَكِي الإِنْسَانُ الْحَيُّ الرَّجُلُ مِنْ قِصَاصِ خَطَايَاهُ؟
\par 40 لِنَفْحَصْ طُرُقَنَا وَنَمْتَحِنْهَا وَنَرْجِعْ إِلَى الرَّبِّ.
\par 41 لِنَرْفَعْ قُلُوبَنَا وَأَيْدِينَا إِلَى اللَّهِ فِي السَّمَاوَاتِ
\par 42 نَحْنُ أَذْنَبْنَا وَعَصِينَا. أَنْتَ لَمْ تَغْفِرْ.
\par 43 الْتَحَفْتَ بِالْغَضَبِ وَطَرَدْتَنَا. قَتَلْتَ وَلَمْ تُشْفِقْ.
\par 44 الْتَحَفْتَ بِالسَّحَابِ حَتَّى لاَ تَنْفُذَ الصَّلاَةُ.
\par 45 جَعَلْتَنَا وَسَخاً وَكَرْهاً فِي وَسَطِ الشُّعُوبِ.
\par 46 فَتَحَ كُلُّ أَعْدَائِنَا أَفْوَاهَهُمْ عَلَيْنَا.
\par 47 صَارَ عَلَيْنَا خَوْفٌ وَرُعْبٌ هَلاَكٌ وَسَحْقٌ.
\par 48 سَكَبَتْ عَيْنَايَ يَنَابِيعَ مَاءٍ عَلَى سَحْقِ بِنْتِ شَعْبِي.
\par 49 عَيْنِي تَسْكُبُ وَلاَ تَكُفُّ بِلاَ انْقِطَاعٍ
\par 50 حَتَّى يُشْرِفَ وَيَنْظُرَ الرَّبُّ مِنَ السَّمَاءِ.
\par 51 عَيْنِي تُؤَثِّرُ فِي نَفْسِي لأَجْلِ كُلِّ بَنَاتِ مَدِينَتِي.
\par 52 قَدِ اصْطَادَتْنِي أَعْدَائِي كَعُصْفُورٍ بِلاَ سَبَبٍ.
\par 53 قَرَضُوا فِي الْجُبِّ حَيَاتِي وَأَلْقُوا عَلَيَّ حِجَارَةً.
\par 54 طَفَتِ الْمِيَاهُ فَوْقَ رَأْسِي. قُلْتُ: «قَدْ قُرِضْتُ!].
\par 55 دَعَوْتُ بِاسْمِكَ يَا رَبُّ مِنَ الْجُبِّ الأَسْفَلِ.
\par 56 لِصَوْتِي سَمِعْتَ. لاَ تَسْتُرْ أُذُنَكَ عَنْ زَفْرَتِي عَنْ صِيَاحِي.
\par 57 دَنَوْتَ يَوْمَ دَعَوْتُكَ. قُلْتَ: «لاَ تَخَفْ!]
\par 58 خَاصَمْتَ يَا سَيِّدُ خُصُومَاتِ نَفْسِي. فَكَكْتَ حَيَاتِي.
\par 59 رَأَيْتَ يَا رَبُّ ظُلْمِي. أَقِمْ دَعْوَايَ.
\par 60 رَأَيْتَ كُلَّ نَقْمَتِهِمْ كُلَّ أَفْكَارِهِمْ عَلَيَّ.
\par 61 سَمِعْتَ تَعْيِيرَهُمْ يَا رَبُّ كُلَّ أَفْكَارِهِمْ عَلَيَّ.
\par 62 كَلاَمُ مُقَاوِمِيَّ وَمُؤَامَرَتُهُمْ عَلَيَّ الْيَوْمَ كُلَّهُ.
\par 63 اُنْظُرْ إِلَى جُلُوسِهِمْ وَوُقُوفِهِمْ أَنَا أُغْنِيَتُهُمْ!
\par 64 رُدَّ لَهُمْ جَزَاءً يَا رَبُّ حَسَبَ عَمَلِ أَيَادِيهِمْ.
\par 65 أَعْطِهِمْ غَشَاوَةَ قَلْبٍ لَعْنَتَكَ لَهُمْ.
\par 66 اِتْبَعْ بِالْغَضَبِ وَأَهْلِكْهُمْ مِنْ تَحْتِ سَمَاوَاتِ الرَّبِّ.

\chapter{4}

\par 1 كَيْفَ اكْدَرَّ الذَّهَبُ تَغَيَّرَ الإِبْرِيزُ الْجَيِّدُ؟ انْهَالَتْ حِجَارَةُ الْقُدْسِ فِي رَأْسِ كُلِّ شَارِعٍ.
\par 2 بَنُو صِهْيَوْنَ الْكُرَمَاءُ الْمَوْزُونُونَ بِالذَّهَبِ النَّقِيِّ كَيْفَ حُسِبُوا أَبَارِيقَ خَزَفٍ عَمَلَ يَدَيْ فَخَّارِيٍّ؟
\par 3 بَنَاتُ آوَى أَيْضاً أَخْرَجَتْ أَطْبَاءَهَا أَرْضَعَتْ أَجْرَاءَهَا. أَمَّا بِنْتُ شَعْبِي فَجَافِيَةٌ كَالنَّعَامِ فِي الْبَرِّيَّةِ.
\par 4 لَصِقَ لِسَانُ الرَّاضِعِ بِحَنَكِهِ مِنَ الْعَطَشِ. الأَطْفَالُ يَسْأَلُونَ خُبْزاً وَلَيْسَ مَنْ يَكْسِرُهُ لَهُمْ.
\par 5 اَلَّذِينَ كَانُوا يَأْكُلُونَ الْمَآكِلَ الْفَاخِرَةَ قَدْ هَلَكُوا فِي الشَّوَارِعِ. الَّذِينَ كَانُوا يَتَرَبُّونَ عَلَى الْقِرْمِزِ احْتَضَنُوا الْمَزَابِلَ.
\par 6 وَقَدْ صَارَ عِقَابُ بِنْتِ شَعْبِي أَعْظَمَ مِنْ قِصَاصِ خَطِيَّةِ سَدُومَ الَّتِي انْقَلَبَتْ كَأَنَّهُ فِي لَحْظَةٍ وَلَمْ تُلْقَ عَلَيْهَا أَيَادٍ.
\par 7 كَانَ نُذُرُهَا أَنْقَى مِنَ الثَّلْجِ وَأَكْثَرَ بَيَاضاً مِنَ اللَّبَنِ وَأَجْسَامُهُمْ أَشَدَّ حُمْرَةً مِنَ الْمَرْجَانِ. جَرَزُهُمْ كَالْيَاقُوتِ الأَزْرَقِ.
\par 8 صَارَتْ صُورَتُهُمْ أَشَدَّ ظَلاَماً مِنَ السَّوَادِ. لَمْ يُعْرَفُوا فِي الشَّوَارِعِ. لَصِقَ جِلْدُهُمْ بِعَظْمِهِمْ. صَارَ يَابِساً كَالْخَشَبِ.
\par 9 كَانَتْ قَتْلَى السَّيْفِ خَيْراً مِنْ قَتْلَى الْجُوعِ. لأَنَّ هَؤُلاَءِ يَذُوبُونَ مَطْعُونِينَ لِعَدَمِ أَثْمَارِ الْحَقْلِ.
\par 10 أَيَادِي النِّسَاءِ الْحَنَائِنِ طَبَخَتْ أَوْلاَدَهُنَّ. صَارُوا طَعَاماً لَهُنَّ فِي سَحْقِ بِنْتِ شَعْبِي.
\par 11 أَتَمَّ الرَّبُّ غَيْظَهُ. سَكَبَ حُمُوَّ غَضَبِهِ وَأَشْعَلَ نَاراً فِي صِهْيَوْنَ فَأَكَلَتْ أُسُسَهَا.
\par 12 لَمْ تُصَدِّقْ مُلُوكُ الأَرْضِ وَكُلُّ سُكَّانِ الْمَسْكُونَةِ أَنَّ الْعَدُوَّ وَالْمُبْغِضَ يَدْخُلاَنِ أَبْوَابَ أُورُشَلِيمَ.
\par 13 مِنْ أَجْلِ خَطَايَا أَنْبِيَائِهَا وَآثَامِ كَهَنَتِهَا السَّافِكِينَ فِي وَسَطِهَا دَمَ الصِّدِّيقِينَ
\par 14 تَاهُوا كَعُمْيٍ فِي الشَّوَارِعِ وَتَلَطَّخُوا بِالدَّمِ حَتَّى لَمْ يَسْتَطِعْ أَحَدٌ أَنْ يَمَسَّ مَلاَبِسَهُمْ.
\par 15 «حِيدُوا! نَجِسٌ!] يُنَادُونَ إِلَيْهِمْ. «حِيدُوا! حِيدُوا لاَ تَمَسُّوا!]. إِذْ هَرَبُوا تَاهُوا أَيْضاً. قَالُوا بَيْنَ الأُمَمِ إِنَّهُمْ لاَ يَعُودُونَ يَسْكُنُونَ.
\par 16 وَجْهُ الرَّبِّ قَسَمَهُمْ. لاَ يَعُودُ يَنْظُرُ إِلَيْهِمْ. لَمْ يَرْفَعُوا وُجُوهَ الْكَهَنَةِ وَلَمْ يَتَرَأَّفُوا عَلَى الشُّيُوخِ.
\par 17 أَمَّا نَحْنُ فَقَدْ كَلَّتْ أَعْيُنُنَا مِنَ النَّظَرِ إِلَى عَوْنِنَا الْبَاطِلِ. فِي بُرْجِنَا انْتَظَرْنَا أُمَّةً لاَ تُخَلِّصُ.
\par 18 نَصَبُوا فِخَاخاً لِخَطَوَاتِنَا حَتَّى لاَ نَمْشِيَ فِي سَاحَاتِنَا. قَرُبَتْ نِهَايَتُنَا. كَمَلَتْ أَيَّامُنَا لأَنَّ نِهَايَتَنَا قَدْ أَتَتْ.
\par 19 صَارَ طَارِدُونَا أَخَفَّ مِنْ نُسُورِ السَّمَاءِ. عَلَى الْجِبَالِ جَدُّوا فِي أَثَرِنَا. فِي الْبَرِّيَّةِ كَمَنُوا لَنَا.
\par 20 نَفَسُ أُنُوفِنَا مَسِيحُ الرَّبِّ أُخِذَ فِي حُفَرِهِمِ الَّذِي قُلْنَا عَنْهُ فِي ظِلِّهِ نَعِيشُ بَيْنَ الأُمَمِ.
\par 21 اِطْرَبِي وَافْرَحِي يَا بِنْتَ أَدُومَ يَا سَاكِنَةَ عُوصٍ. عَلَيْكِ أَيْضاً تَمُرُّ الْكَأْسُ. تَسْكَرِينَ وَتَتَعَرِّينَ.
\par 22 قَدْ تَمَّ إِثْمُكِ يَا بِنْتَ صِهْيَوْنَ. لاَ يَعُودُ يَسْبِيكِ. سَيُعَاقِبُ إِثْمَكِ يَا بِنْتَ أَدُومَ وَيُعْلِنُ خَطَايَاكِ.

\chapter{5}

\par 1 اُذْكُرْ يَا رَبُّ مَاذَا صَارَ لَنَا. أَشْرِفْ وَانْظُرْ إِلَى عَارِنَا.
\par 2 قَدْ صَارَ مِيرَاثُنَا لِلْغُرَبَاءِ. بُيُوتُنَا لِلأَجَانِبِ.
\par 3 صِرْنَا أَيْتَاماً بِلاَ أَبٍ. أُمَّهَاتُنَا كَأَرَامِلَ.
\par 4 شَرِبْنَا مَاءَنَا بِالْفِضَّةِ. حَطَبُنَا بِالثَّمَنِ يَأْتِي.
\par 5 عَلَى أَعْنَاقِنَا نُضْطَهَدُ. نَتْعَبُ وَلاَ رَاحَةَ لَنَا.
\par 6 أَعْطَيْنَا الْيَدَ لِلْمِصْرِيِّينَ وَالأَشُّورِيِّينَ لِنَشْبَعَ خُبْزاً.
\par 7 آبَاؤُنَا أَخْطَأُوا وَلَيْسُوا بِمَوْجُودِينَ وَنَحْنُ نَحْمِلُ آثَامَهُمْ.
\par 8 عَبِيدٌ حَكَمُوا عَلَيْنَا. لَيْسَ مَنْ يُخَلِّصُ مِنْ أَيْدِيهِمْ.
\par 9 بِأَنْفُسِنَا نَأْتِي بِخُبْزِنَا مِنْ جَرَى سَيْفِ الْبَرِّيَّةِ.
\par 10 جُلُودُنَا اسْوَدَّتْ كَتَنُّورٍ مِنْ جَرَى نِيرَانِ الْجُوعِ.
\par 11 أَذَلُّوا النِّسَاءَ فِي صِهْيَوْنَ الْعَذَارَى فِي مُدُنِ يَهُوذَا.
\par 12 الرُّؤَسَاءُ بِأَيْدِيهِمْ يُعَلَّقُونَ وَلَمْ تُعْتَبَرْ وُجُوهُ الشُّيُوخِ.
\par 13 أَخَذُوا الشُّبَّانَ لِلطَّحْنِ وَالصِّبْيَانَ عَثَرُوا تَحْتَ الْحَطَبِ.
\par 14 كَفَّتِ الشُّيُوخُ عَنِ الْبَابِ وَالشُّبَّانُ عَنْ غِنَائِهِمْ.
\par 15 مَضَى فَرَحُ قَلْبِنَا. صَارَ رَقْصُنَا نَوْحاً.
\par 16 سَقَطَ إِكْلِيلُ رَأْسِنَا. وَيْلٌ لَنَا لأَنَّنَا قَدْ أَخْطَأْنَا.
\par 17 مِنْ أَجْلِ هَذَا حَزِنَ قَلْبُنَا. مِنْ أَجْلِ هَذِهِ أَظْلَمَتْ عُيُونُنَا.
\par 18 مِنْ أَجْلِ جَبَلِ صِهْيَوْنَ الْخَرِبِ. الثَّعَالِبُ مَاشِيَةٌ فِيهِ.
\par 19 أَنْتَ يَا رَبُّ إِلَى الأَبَدِ تَجْلِسُ. كُرْسِيُّكَ إِلَى دَوْرٍ فَدَوْرٍ.
\par 20 لِمَاذَا تَنْسَانَا إِلَى الأَبَدِ وَتَتْرُكُنَا طُولَ الأَيَّامِ؟
\par 21 اُرْدُدْنَا يَا رَبُّ إِلَيْكَ فَنَرْتَدَّ. جَدِّدْ أَيَّامَنَا كَالْقَدِيمِ.
\par 22 هَلْ كُلَّ الرَّفْضِ رَفَضْتَنَا؟ هَلْ غَضِبْتَ عَلَيْنَا جِدّاً؟

\end{document}