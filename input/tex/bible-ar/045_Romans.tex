\begin{document}

\title{رومية}


\chapter{1}

\par 1 بُولُسُ عَبْدٌ لِيَسُوعَ الْمَسِيحِ الْمَدْعُوُّ رَسُولاً الْمُفْرَزُ لإِنْجِيلِ اللهِ
\par 2 الَّذِي سَبَقَ فَوَعَدَ بِهِ بِأَنْبِيَائِهِ فِي الْكُتُبِ الْمُقَدَّسَةِ
\par 3 عَنِ ابْنِهِ. الَّذِي صَارَ مِنْ نَسْلِ دَاوُدَ مِنْ جِهَةِ الْجَسَدِ
\par 4 وَتَعَيَّنَ ابْنَ اللهِ بِقُوَّةٍ مِنْ جِهَةِ رُوحِ الْقَدَاسَةِ بِالْقِيَامَةِ مِنَ الأَمْوَاتِ: يَسُوعَ الْمَسِيحِ رَبِّنَا.
\par 5 الَّذِي بِهِ لأَجْلِ اسْمِهِ قَبِلْنَا نِعْمَةً وَرِسَالَةً لإِطَاعَةِ الإِيمَانِ فِي جَمِيعِ الأُمَمِ
\par 6 الَّذِينَ بَيْنَهُمْ أَنْتُمْ أَيْضاً مَدْعُوُّو يَسُوعَ الْمَسِيحِ.
\par 7 إِلَى جَمِيعِ الْمَوْجُودِينَ فِي رُومِيَةَ أَحِبَّاءَ اللهِ مَدْعُوِّينَ قِدِّيسِينَ: نِعْمَةٌ لَكُمْ وَسَلاَمٌ مِنَ اللهِ أَبِينَا وَالرَّبِّ يَسُوعَ الْمَسِيحِ.
\par 8 أَوَّلاً أَشْكُرُ إِلَهِي بِيَسُوعَ الْمَسِيحِ مِنْ جِهَةِ جَمِيعِكُمْ أَنَّ إِيمَانَكُمْ يُنَادَى بِهِ فِي كُلِّ الْعَالَمِ.
\par 9 فَإِنَّ اللهَ الَّذِي أَعْبُدُهُ بِرُوحِي فِي إِنْجِيلِ ابْنِهِ شَاهِدٌ لِي كَيْفَ بِلاَ انْقِطَاعٍ أَذْكُرُكُمْ
\par 10 مُتَضَرِّعاً دَائِماً فِي صَلَوَاتِي عَسَى الآنَ أَنْ يَتَيَسَّرَ لِي مَرَّةً بِمَشِيئَةِ اللهِ أَنْ آتِيَ إِلَيْكُمْ.
\par 11 لأَنِّي مُشْتَاقٌ أَنْ أَرَاكُمْ لِكَيْ أَمْنَحَكُمْ هِبَةً رُوحِيَّةً لِثَبَاتِكُمْ
\par 12 أَيْ لِنَتَعَزَّى بَيْنَكُمْ بِالإِيمَانِ الَّذِي فِينَا جَمِيعاً إِيمَانِكُمْ وَإِيمَانِي.
\par 13 ثُمَّ لَسْتُ أُرِيدُ أَنْ تَجْهَلُوا أَيُّهَا الإِخْوَةُ أَنَّنِي مِرَاراً كَثِيرَةً قَصَدْتُ أَنْ آتِيَ إِلَيْكُمْ وَمُنِعْتُ حَتَّى الآنَ لِيَكُونَ لِي ثَمَرٌ فِيكُمْ أَيْضاً كَمَا فِي سَائِرِ الأُمَمِ.
\par 14 إِنِّي مَدْيُونٌ لِلْيُونَانِيِّينَ وَالْبَرَابِرَةِ لِلْحُكَمَاءِ وَالْجُهَلاَءِ.
\par 15 فَهَكَذَا مَا هُوَ لِي مُسْتَعَدٌّ لِتَبْشِيرِكُمْ أَنْتُمُ الَّذِينَ فِي رُومِيَةَ أَيْضاً
\par 16 لأَنِّي لَسْتُ أَسْتَحِي بِإِنْجِيلِ الْمَسِيحِ لأَنَّهُ قُوَّةُ اللهِ لِلْخَلاَصِ لِكُلِّ مَنْ يُؤْمِنُ: لِلْيَهُودِيِّ أَوَّلاً ثُمَّ لِلْيُونَانِيِّ.
\par 17 لأَنْ فِيهِ مُعْلَنٌ بِرُّ اللهِ بِإِيمَانٍ لإِيمَانٍ كَمَا هُوَ مَكْتُوبٌ«أَمَّا الْبَارُّ فَبِالإِيمَانِ يَحْيَا».
\par 18 لأَنَّ غَضَبَ اللهِ مُعْلَنٌ مِنَ السَّمَاءِ عَلَى جَمِيعِ فُجُورِ النَّاسِ وَإِثْمِهِمِ الَّذِينَ يَحْجِزُونَ الْحَقَّ بِالإِثْمِ.
\par 19 إِذْ مَعْرِفَةُ اللهِ ظَاهِرَةٌ فِيهِمْ لأَنَّ اللهَ أَظْهَرَهَا لَهُمْ
\par 20 لأَنَّ مُنْذُ خَلْقِ الْعَالَمِ تُرَى أُمُورُهُ غَيْرُ الْمَنْظُورَةِ وَقُدْرَتُهُ السَّرْمَدِيَّةُ وَلاَهُوتُهُ مُدْرَكَةً بِالْمَصْنُوعَاتِ حَتَّى إِنَّهُمْ بِلاَ عُذْرٍ.
\par 21 لأَنَّهُمْ لَمَّا عَرَفُوا اللهَ لَمْ يُمَجِّدُوهُ أَوْ يَشْكُرُوهُ كَإِلَهٍ بَلْ حَمِقُوا فِي أَفْكَارِهِمْ وَأَظْلَمَ قَلْبُهُمُ الْغَبِيُّ.
\par 22 وَبَيْنَمَا هُمْ يَزْعُمُونَ أَنَّهُمْ حُكَمَاءُ صَارُوا جُهَلاَءَ
\par 23 وَأَبْدَلُوا مَجْدَ اللهِ الَّذِي لاَ يَفْنَى بِشِبْهِ صُورَةِ الإِنْسَانِ الَّذِي يَفْنَى وَالطُّيُورِ وَالدَّوَابِّ وَالزَّحَّافَاتِ.
\par 24 لِذَلِكَ أَسْلَمَهُمُ اللهُ أَيْضاً فِي شَهَوَاتِ قُلُوبِهِمْ إِلَى النَّجَاسَةِ لإِهَانَةِ أَجْسَادِهِمْ بَيْنَ ذَوَاتِهِمِ.
\par 25 الَّذِينَ اسْتَبْدَلُوا حَقَّ اللهِ بِالْكَذِبِ وَاتَّقَوْا وَعَبَدُوا الْمَخْلُوقَ دُونَ الْخَالِقِ الَّذِي هُوَ مُبَارَكٌ إِلَى الأَبَدِ. آمِينَ.
\par 26 لِذَلِكَ أَسْلَمَهُمُ اللهُ إِلَى أَهْوَاءِ الْهَوَانِ لأَنَّ إِنَاثَهُمُ اسْتَبْدَلْنَ الاِسْتِعْمَالَ الطَّبِيعِيَّ بِالَّذِي عَلَى خِلاَفِ الطَّبِيعَةِ
\par 27 وَكَذَلِكَ الذُّكُورُ أَيْضاً تَارِكِينَ اسْتِعْمَالَ الأُنْثَى الطَّبِيعِيَّ اشْتَعَلُوا بِشَهْوَتِهِمْ بَعْضِهِمْ لِبَعْضٍ فَاعِلِينَ الْفَحْشَاءَ ذُكُوراً بِذُكُورٍ وَنَائِلِينَ فِي أَنْفُسِهِمْ جَزَاءَ ضَلاَلِهِمِ الْمُحِقَّ.
\par 28 وَكَمَا لَمْ يَسْتَحْسِنُوا أَنْ يُبْقُوا اللهَ فِي مَعْرِفَتِهِمْ أَسْلَمَهُمُ اللهُ إِلَى ذِهْنٍ مَرْفُوضٍ لِيَفْعَلُوا مَا لاَ يَلِيقُ.
\par 29 مَمْلُوئِينَ مِنْ كُلِّ إِثْمٍ وَزِناً وَشَرٍّ وَطَمَعٍ وَخُبْثٍ مَشْحُونِينَ حَسَداً وَقَتْلاً وَخِصَاماً وَمَكْراً وَسُوءاً
\par 30 نَمَّامِينَ مُفْتَرِينَ مُبْغِضِينَ لِلَّهِ ثَالِبِينَ مُتَعَظِّمِينَ مُدَّعِينَ مُبْتَدِعِينَ شُرُوراً غَيْرَ طَائِعِينَ لِلْوَالِدَيْنِ
\par 31 بِلاَ فَهْمٍ وَلاَ عَهْدٍ وَلاَ حُنُوٍّ وَلاَ رِضىً وَلاَ رَحْمَةٍ.
\par 32 الَّذِينَ إِذْ عَرَفُوا حُكْمَ اللهِ أَنَّ الَّذِينَ يَعْمَلُونَ مِثْلَ هَذِهِ يَسْتَوْجِبُونَ الْمَوْتَ لاَ يَفْعَلُونَهَا فَقَطْ بَلْ أَيْضاً يُسَرُّونَ بِالَّذِينَ يَعْمَلُونَ!

\chapter{2}

\par 1 لِذَلِكَ أَنْتَ بِلاَ عُذْرٍ أَيُّهَا الإِنْسَانُ كُلُّ مَنْ يَدِينُ. لأَنَّكَ فِي مَا تَدِينُ غَيْرَكَ تَحْكُمُ عَلَى نَفْسِكَ. لأَنَّكَ أَنْتَ الَّذِي تَدِينُ تَفْعَلُ تِلْكَ الأُمُورَ بِعَيْنِهَا!
\par 2 وَنَحْنُ نَعْلَمُ أَنَّ دَيْنُونَةَ اللهِ هِيَ حَسَبُ الْحَقِّ عَلَى الَّذِينَ يَفْعَلُونَ مِثْلَ هَذِهِ.
\par 3 أَفَتَظُنُّ هَذَا أَيُّهَا الإِنْسَانُ الَّذِي تَدِينُ الَّذِينَ يَفْعَلُونَ مِثْلَ هَذِهِ وَأَنْتَ تَفْعَلُهَا أَنَّكَ تَنْجُو مِنْ دَيْنُونَةِ اللهِ؟
\par 4 أَمْ تَسْتَهِينُ بِغِنَى لُطْفِهِ وَإِمْهَالِهِ وَطُولِ أَنَاتِهِ غَيْرَ عَالِمٍ أَنَّ لُطْفَ اللهِ إِنَّمَا يَقْتَادُكَ إِلَى التَّوْبَةِ؟
\par 5 وَلَكِنَّكَ مِنْ أَجْلِ قَسَاوَتِكَ وَقَلْبِكَ غَيْرِ التَّائِبِ تَذْخَرُ لِنَفْسِكَ غَضَباً فِي يَوْمِ الْغَضَبِ وَاسْتِعْلاَنِ دَيْنُونَةِ اللهِ الْعَادِلَةِ
\par 6 الَّذِي سَيُجَازِي كُلَّ وَاحِدٍ حَسَبَ أَعْمَالِهِ.
\par 7 أَمَّا الَّذِينَ بِصَبْرٍ فِي الْعَمَلِ الصَّالِحِ يَطْلُبُونَ الْمَجْدَ وَالْكَرَامَةَ وَالْبَقَاءَ فَبِالْحَيَاةِ الأَبَدِيَّةِ.
\par 8 وَأَمَّا الَّذِينَ هُمْ مِنْ أَهْلِ التَّحَزُّبِ وَلاَ يُطَاوِعُونَ لِلْحَقِّ بَلْ يُطَاوِعُونَ لِلإِثْمِ فَسَخَطٌ وَغَضَبٌ
\par 9 شِدَّةٌ وَضِيقٌ عَلَى كُلِّ نَفْسِ إِنْسَانٍ يَفْعَلُ الشَّرَّ الْيَهُودِيِّ أَوَّلاً ثُمَّ الْيُونَانِيِّ.
\par 10 وَمَجْدٌ وَكَرَامَةٌ وَسَلاَمٌ لِكُلِّ مَنْ يَفْعَلُ الصَّلاَحَ الْيَهُودِيِّ أَوَّلاً ثُمَّ الْيُونَانِيِّ.
\par 11 لأَنْ لَيْسَ عِنْدَ اللهِ مُحَابَاةٌ.
\par 12 لأَنَّ كُلَّ مَنْ أَخْطَأَ بِدُونِ النَّامُوسِ فَبِدُونِ النَّامُوسِ يَهْلِكُ وَكُلُّ مَنْ أَخْطَأَ فِي النَّامُوسِ فَبِالنَّامُوسِ يُدَانُ.
\par 13 لأَنْ لَيْسَ الَّذِينَ يَسْمَعُونَ النَّامُوسَ هُمْ أَبْرَارٌ عِنْدَ اللهِ بَلِ الَّذِينَ يَعْمَلُونَ بِالنَّامُوسِ هُمْ يُبَرَّرُونَ.
\par 14 لأَنَّهُ الأُمَمُ الَّذِينَ لَيْسَ عِنْدَهُمُ النَّامُوسُ مَتَى فَعَلُوا بِالطَّبِيعَةِ مَا هُوَ فِي النَّامُوسِ فَهَؤُلاَءِ إِذْ لَيْسَ لَهُمُ النَّامُوسُ هُمْ نَامُوسٌ لأَنْفُسِهِمِ
\par 15 الَّذِينَ يُظْهِرُونَ عَمَلَ النَّامُوسِ مَكْتُوباً فِي قُلُوبِهِمْ شَاهِداً أَيْضاً ضَمِيرُهُمْ وَأَفْكَارُهُمْ فِيمَا بَيْنَهَا مُشْتَكِيَةً أَوْ مُحْتَجَّةً
\par 16 فِي الْيَوْمِ الَّذِي فِيهِ يَدِينُ اللهُ سَرَائِرَ النَّاسِ حَسَبَ إِنْجِيلِي بِيَسُوعَ الْمَسِيحِ.
\par 17 هُوَذَا أَنْتَ تُسَمَّى يَهُودِيّاً وَتَتَّكِلُ عَلَى النَّامُوسِ وَتَفْتَخِرُ بِاللَّهِ
\par 18 وَتَعْرِفُ مَشِيئَتَهُ وَتُمَيِّزُ الأُمُورَ الْمُتَخَالِفَةَ مُتَعَلِّماً مِنَ النَّامُوسِ.
\par 19 وَتَثِقُ أَنَّكَ قَائِدٌ لِلْعُمْيَانِ وَنُورٌ لِلَّذِينَ فِي الظُّلْمَةِ
\par 20 وَمُهَذِّبٌ لِلأَغْبِيَاءِ وَمُعَلِّمٌ لِلأَطْفَالِ وَلَكَ صُورَةُ الْعِلْمِ وَالْحَقِّ فِي النَّامُوسِ.
\par 21 فَأَنْتَ إِذاً الَّذِي تُعَلِّمُ غَيْرَكَ أَلَسْتَ تُعَلِّمُ نَفْسَكَ؟ الَّذِي تَكْرِزُ أَنْ لاَ يُسْرَقَ أَتَسْرِقُ؟
\par 22 الَّذِي تَقُولُ أَنْ لاَ يُزْنَى أَتَزْنِي؟ الَّذِي تَسْتَكْرِهُ الأَوْثَانَ أَتَسْرِقُ الْهَيَاكِلَ؟
\par 23 الَّذِي تَفْتَخِرُ بِالنَّامُوسِ أَبِتَعَدِّي النَّامُوسِ تُهِينُ اللهَ؟
\par 24 لأَنَّ اسْمَ اللهِ يُجَدَّفُ عَلَيْهِ بِسَبَبِكُمْ بَيْنَ الأُمَمِ كَمَا هُوَ مَكْتُوبٌ.
\par 25 فَإِنَّ الْخِتَانَ يَنْفَعُ إِنْ عَمِلْتَ بِالنَّامُوسِ. وَلَكِنْ إِنْ كُنْتَ مُتَعَدِّياً النَّامُوسَ فَقَدْ صَارَ خِتَانُكَ غُرْلَةً!
\par 26 إِذاً إِنْ كَانَ الأَغْرَلُ يَحْفَظُ أَحْكَامَ النَّامُوسِ أَفَمَا تُحْسَبُ غُرْلَتُهُ خِتَاناً؟
\par 27 وَتَكُونُ الْغُرْلَةُ الَّتِي مِنَ الطَّبِيعَةِ وَهِيَ تُكَمِّلُ النَّامُوسَ تَدِينُكَ أَنْتَ الَّذِي فِي الْكِتَابِ وَالْخِتَانِ تَتَعَدَّى النَّامُوسَ؟
\par 28 لأَنَّ الْيَهُودِيَّ فِي الظَّاهِرِ لَيْسَ هُوَ يَهُودِيّاً وَلاَ الْخِتَانُ الَّذِي فِي الظَّاهِرِ فِي اللَّحْمِ خِتَاناً
\par 29 بَلِ الْيَهُودِيُّ فِي الْخَفَاءِ هُوَ الْيَهُودِيُّ وَخِتَانُ الْقَلْبِ بِالرُّوحِ لاَ بِالْكِتَابِ هُوَ الْخِتَانُ الَّذِي مَدْحُهُ لَيْسَ مِنَ النَّاسِ بَلْ مِنَ اللهِ.

\chapter{3}

\par 1 إِذاً مَا هُوَ فَضْلُ الْيَهُودِيِّ أَوْ مَا هُوَ نَفْعُ الْخِتَانِ؟
\par 2 كَثِيرٌ عَلَى كُلِّ وَجْهٍ! أَمَّا أَوَّلاً فَلأَنَّهُمُ اسْتُؤْمِنُوا عَلَى أَقْوَالِ اللهِ.
\par 3 فَمَاذَا إِنْ كَانَ قَوْمٌ لَمْ يَكُونُوا أُمَنَاءَ؟ أَفَلَعَلَّ عَدَمَ أَمَانَتِهِمْ يُبْطِلُ أَمَانَةَ اللهِ؟
\par 4 حَاشَا! بَلْ لِيَكُنِ اللهُ صَادِقاً وَكُلُّ إِنْسَانٍ كَاذِباً. كَمَا هُوَ مَكْتُوبٌ: «لِكَيْ تَتَبَرَّرَ فِي كَلاَمِكَ وَتَغْلِبَ مَتَى حُوكِمْتَ».
\par 5 وَلَكِنْ إِنْ كَانَ إِثْمُنَا يُبَيِّنُ بِرَّ اللهِ فَمَاذَا نَقُولُ؟ أَلَعَلَّ اللهَ الَّذِي يَجْلِبُ الْغَضَبَ ظَالِمٌ؟ أَتَكَلَّمُ بِحَسَبِ الإِنْسَانِ.
\par 6 حَاشَا! فَكَيْفَ يَدِينُ اللهُ الْعَالَمَ إِذْ ذَاكَ؟
\par 7 فَإِنَّهُ إِنْ كَانَ صِدْقُ اللهِ قَدِ ازْدَادَ بِكَذِبِي لِمَجْدِهِ فَلِمَاذَا أُدَانُ أَنَا بَعْدُ كَخَاطِئٍ؟
\par 8 أَمَا كَمَا يُفْتَرَى عَلَيْنَا وَكَمَا يَزْعُمُ قَوْمٌ أَنَّنَا نَقُولُ: «لِنَفْعَلِ السَّيِّآتِ لِكَيْ تَأْتِيَ الْخَيْرَاتُ». الَّذِينَ دَيْنُونَتُهُمْ عَادِلَةٌ.
\par 9 فَمَاذَا إِذاً؟ أَنَحْنُ أَفْضَلُ؟ كَلاَّ الْبَتَّةَ! لأَنَّنَا قَدْ شَكَوْنَا أَنَّ الْيَهُودَ وَالْيُونَانِيِّينَ أَجْمَعِينَ تَحْتَ الْخَطِيَّةِ
\par 10 كَمَا هُوَ مَكْتُوبٌ: «أَنَّهُ لَيْسَ بَارٌّ وَلاَ وَاحِدٌ.
\par 11 لَيْسَ مَنْ يَفْهَمُ. لَيْسَ مَنْ يَطْلُبُ اللهَ.
\par 12 الْجَمِيعُ زَاغُوا وَفَسَدُوا مَعاً. لَيْسَ مَنْ يَعْمَلُ صَلاَحاً لَيْسَ وَلاَ وَاحِدٌ.
\par 13 حَنْجَرَتُهُمْ قَبْرٌ مَفْتُوحٌ. بِأَلْسِنَتِهِمْ قَدْ مَكَرُوا. سِمُّ الأَصْلاَلِ تَحْتَ شِفَاهِهِمْ.
\par 14 وَفَمُهُمْ مَمْلُوءٌ لَعْنَةً وَمَرَارَةً.
\par 15 أَرْجُلُهُمْ سَرِيعَةٌ إِلَى سَفْكِ الدَّمِ.
\par 16 فِي طُرُقِهِمِ اغْتِصَابٌ وَسَحْقٌ.
\par 17 وَطَرِيقُ السَّلاَمِ لَمْ يَعْرِفُوهُ.
\par 18 لَيْسَ خَوْفُ اللهِ قُدَّامَ عُيُونِهِمْ».
\par 19 وَنَحْنُ نَعْلَمُ أَنَّ كُلَّ مَا يَقُولُهُ النَّامُوسُ فَهُوَ يُكَلِّمُ بِهِ الَّذِينَ فِي النَّامُوسِ لِكَيْ يَسْتَدَّ كُلُّ فَمٍ وَيَصِيرَ كُلُّ الْعَالَمِ تَحْتَ قِصَاصٍ مِنَ اللهِ.
\par 20 لأَنَّهُ بِأَعْمَالِ النَّامُوسِ كُلُّ ذِي جَسَدٍ لاَ يَتَبَرَّرُ أَمَامَهُ. لأَنَّ بِالنَّامُوسِ مَعْرِفَةَ الْخَطِيَّةِ.
\par 21 وَأَمَّا الآنَ فَقَدْ ظَهَرَ بِرُّ اللهِ بِدُونِ النَّامُوسِ مَشْهُوداً لَهُ مِنَ النَّامُوسِ وَالأَنْبِيَاءِ
\par 22 بِرُّ اللهِ بِالإِيمَانِ بِيَسُوعَ الْمَسِيحِ إِلَى كُلِّ وَعَلَى كُلِّ الَّذِينَ يُؤْمِنُونَ. لأَنَّهُ لاَ فَرْقَ.
\par 23 إِذِ الْجَمِيعُ أَخْطَأُوا وَأَعْوَزَهُمْ مَجْدُ اللهِ
\par 24 مُتَبَرِّرِينَ مَجَّاناً بِنِعْمَتِهِ بِالْفِدَاءِ الَّذِي بِيَسُوعَ الْمَسِيحِ
\par 25 الَّذِي قَدَّمَهُ اللهُ كَفَّارَةً بِالإِيمَانِ بِدَمِهِ لإِظْهَارِ بِرِّهِ مِنْ أَجْلِ الصَّفْحِ عَنِ الْخَطَايَا السَّالِفَةِ بِإِمْهَالِ اللهِ.
\par 26 لإِظْهَارِ بِرِّهِ فِي الزَّمَانِ الْحَاضِرِ لِيَكُونَ بَارّاً وَيُبَرِّرَ مَنْ هُوَ مِنَ الإِيمَانِ بِيَسُوعَ.
\par 27 فَأَيْنَ الافْتِخَارُ؟ قَدِ انْتَفَى! بِأَيِّ نَامُوسٍ؟ أَبِنَامُوسِ الأَعْمَالِ؟ كَلاَّ! بَلْ بِنَامُوسِ الإِيمَانِ.
\par 28 إِذاً نَحْسِبُ أَنَّ الإِنْسَانَ يَتَبَرَّرُ بِالإِيمَانِ بِدُونِ أَعْمَالِ النَّامُوسِ.
\par 29 أَمِ اللهُ لِلْيَهُودِ فَقَطْ؟ أَلَيْسَ لِلأُمَمِ أَيْضاً؟ بَلَى لِلأُمَمِ أَيْضاً؟
\par 30 لأَنَّ اللهَ وَاحِدٌ هُوَ الَّذِي سَيُبَرِّرُ الْخِتَانَ بِالإِيمَانِ وَالْغُرْلَةَ بِالإِيمَانِ.
\par 31 أَفَنُبْطِلُ النَّامُوسَ بِالإِيمَانِ؟ حَاشَا! بَلْ نُثَبِّتُ النَّامُوسَ.

\chapter{4}

\par 1 فَمَاذَا نَقُولُ إِنَّ أَبَانَا إِبْرَاهِيمَ قَدْ وَجَدَ حَسَبَ الْجَسَدِ؟
\par 2 لأَنَّهُ إِنْ كَانَ إِبْرَاهِيمُ قَدْ تَبَرَّرَ بِالأَعْمَالِ فَلَهُ فَخْرٌ - وَلَكِنْ لَيْسَ لَدَى اللهِ.
\par 3 لأَنَّهُ مَاذَا يَقُولُ الْكِتَابُ؟ «فَآمَنَ إِبْرَاهِيمُ بِاللَّهِ فَحُسِبَ لَهُ بِرّاً».
\par 4 أَمَّا الَّذِي يَعْمَلُ فَلاَ تُحْسَبُ لَهُ الأُجْرَةُ عَلَى سَبِيلِ نِعْمَةٍ بَلْ عَلَى سَبِيلِ دَيْنٍ.
\par 5 وَأَمَّا الَّذِي لاَ يَعْمَلُ وَلَكِنْ يُؤْمِنُ بِالَّذِي يُبَرِّرُ الْفَاجِرَ فَإِيمَانُهُ يُحْسَبُ لَهُ بِرّاً.
\par 6 كَمَا يَقُولُ دَاوُدُ أَيْضاً فِي تَطْوِيبِ الإِنْسَانِ الَّذِي يَحْسِبُ لَهُ اللهُ بِرّاً بِدُونِ أَعْمَالٍ:
\par 7 «طُوبَى لِلَّذِينَ غُفِرَتْ آثَامُهُمْ وَسُتِرَتْ خَطَايَاهُمْ.
\par 8 طُوبَى لِلرَّجُلِ الَّذِي لاَ يَحْسِبُ لَهُ الرَّبُّ خَطِيَّةً».
\par 9 أَفَهَذَا التَّطْوِيبُ هُوَ عَلَى الْخِتَانِ فَقَطْ أَمْ عَلَى الْغُرْلَةِ أَيْضاً؟ لأَنَّنَا نَقُولُ إِنَّهُ حُسِبَ لإِبْرَاهِيمَ الإِيمَانُ بِرّاً.
\par 10 فَكَيْفَ حُسِبَ؟ أَوَهُوَ فِي الْخِتَانِ أَمْ فِي الْغُرْلَةِ؟ لَيْسَ فِي الْخِتَانِ بَلْ فِي الْغُرْلَةِ!
\par 11 وَأَخَذَ عَلاَمَةَ الْخِتَانِ خَتْماً لِبِرِّ الإِيمَانِ الَّذِي كَانَ فِي الْغُرْلَةِ لِيَكُونَ أَباً لِجَمِيعِ الَّذِينَ يُؤْمِنُونَ وَهُمْ فِي الْغُرْلَةِ كَيْ يُحْسَبَ لَهُمْ أَيْضاً الْبِرُّ.
\par 12 وَأَباً لِلْخِتَانِ لِلَّذِينَ لَيْسُوا مِنَ الْخِتَانِ فَقَطْ بَلْ أَيْضاً يَسْلُكُونَ فِي خُطُواتِ إِيمَانِ أَبِينَا إِبْرَاهِيمَ الَّذِي كَانَ وَهُوَ فِي الْغُرْلَةِ.
\par 13 فَإِنَّهُ لَيْسَ بِالنَّامُوسِ كَانَ الْوَعْدُ لإِبْرَاهِيمَ أَوْ لِنَسْلِهِ أَنْ يَكُونَ وَارِثاً لِلْعَالَمِ بَلْ بِبِرِّ الإِيمَانِ.
\par 14 لأَنَّهُ إِنْ كَانَ الَّذِينَ مِنَ النَّامُوسِ هُمْ وَرَثَةً فَقَدْ تَعَطَّلَ الإِيمَانُ وَبَطَلَ الْوَعْدُ!
\par 15 لأَنَّ النَّامُوسَ يُنْشِئُ غَضَباً إِذْ حَيْثُ لَيْسَ نَامُوسٌ لَيْسَ أَيْضاً تَعَدٍّ.
\par 16 لِهَذَا هُوَ مِنَ الإِيمَانِ كَيْ يَكُونَ عَلَى سَبِيلِ النِّعْمَةِ لِيَكُونَ الْوَعْدُ وَطِيداً لِجَمِيعِ النَّسْلِ. لَيْسَ لِمَنْ هُوَ مِنَ النَّامُوسِ فَقَطْ بَلْ أَيْضاً لِمَنْ هُوَ مِنْ إِيمَانِ إِبْرَاهِيمَ الَّذِي هُوَ أَبٌ لِجَمِيعِنَا.
\par 17 كَمَا هُوَ مَكْتُوبٌ: «إِنِّي قَدْ جَعَلْتُكَ أَباً لِأُمَمٍ كَثِيرَةٍ». أَمَامَ اللهِ الَّذِي آمَنَ بِهِ الَّذِي يُحْيِي الْمَوْتَى وَيَدْعُو الأَشْيَاءَ غَيْرَ الْمَوْجُودَةِ كَأَنَّهَا مَوْجُودَةٌ.
\par 18 فَهُوَ عَلَى خِلاَفِ الرَّجَاءِ آمَنَ عَلَى الرَّجَاءِ لِكَيْ يَصِيرَ أَباً لِأُمَمٍ كَثِيرَةٍ كَمَا قِيلَ: «هَكَذَا يَكُونُ نَسْلُكَ».
\par 19 وَإِذْ لَمْ يَكُنْ ضَعِيفاً فِي الإِيمَانِ لَمْ يَعْتَبِرْ جَسَدَهُ - وَهُوَ قَدْ صَارَ مُمَاتاً إِذْ كَانَ ابْنَ نَحْوِ مِئَةِ سَنَةٍ - وَلاَ مُمَاتِيَّةَ مُسْتَوْدَعِ سَارَةَ.
\par 20 وَلاَ بِعَدَمِ إِيمَانٍ ارْتَابَ فِي وَعْدِ اللهِ بَلْ تَقَوَّى بِالإِيمَانِ مُعْطِياً مَجْداً لِلَّهِ.
\par 21 وَتَيَقَّنَ أَنَّ مَا وَعَدَ بِهِ هُوَ قَادِرٌ أَنْ يَفْعَلَهُ أَيْضاً.
\par 22 لِذَلِكَ أَيْضاً حُسِبَ لَهُ بِرّاً.
\par 23 وَلَكِنْ لَمْ يُكْتَبْ مِنْ أَجْلِهِ وَحْدَهُ أَنَّهُ حُسِبَ لَهُ
\par 24 بَلْ مِنْ أَجْلِنَا نَحْنُ أَيْضاً الَّذِينَ سَيُحْسَبُ لَنَا الَّذِينَ نُؤْمِنُ بِمَنْ أَقَامَ يَسُوعَ رَبَّنَا مِنَ الأَمْوَاتِ.
\par 25 الَّذِي أُسْلِمَ مِنْ أَجْلِ خَطَايَانَا وَأُقِيمَ لأَجْلِ تَبْرِيرِنَا.

\chapter{5}

\par 1 فَإِذْ قَدْ تَبَرَّرْنَا بِالإِيمَانِ لَنَا سَلاَمٌ مَعَ اللهِ بِرَبِّنَا يَسُوعَ الْمَسِيحِ
\par 2 الَّذِي بِهِ أَيْضاً قَدْ صَارَ لَنَا الدُّخُولُ بِالإِيمَانِ إِلَى هَذِهِ النِّعْمَةِ الَّتِي نَحْنُ فِيهَا مُقِيمُونَ وَنَفْتَخِرُ عَلَى رَجَاءِ مَجْدِ اللهِ.
\par 3 وَلَيْسَ ذَلِكَ فَقَطْ بَلْ نَفْتَخِرُ أَيْضاً فِي الضِّيقَاتِ عَالِمِينَ أَنَّ الضِّيقَ يُنْشِئُ صَبْراً
\par 4 وَالصَّبْرُ تَزْكِيَةً وَالتَّزْكِيَةُ رَجَاءً
\par 5 وَالرَّجَاءُ لاَ يُخْزِي لأَنَّ مَحَبَّةَ اللهِ قَدِ انْسَكَبَتْ فِي قُلُوبِنَا بِالرُّوحِ الْقُدُسِ الْمُعْطَى لَنَا.
\par 6 لأَنَّ الْمَسِيحَ إِذْ كُنَّا بَعْدُ ضُعَفَاءَ مَاتَ فِي الْوَقْتِ الْمُعَيَّنِ لأَجْلِ الْفُجَّارِ.
\par 7 فَإِنَّهُ بِالْجَهْدِ يَمُوتُ أَحَدٌ لأَجْلِ بَارٍّ. رُبَّمَا لأَجْلِ الصَّالِحِ يَجْسُرُ أَحَدٌ أَيْضاً أَنْ يَمُوتَ.
\par 8 وَلَكِنَّ اللهَ بَيَّنَ مَحَبَّتَهُ لَنَا لأَنَّهُ وَنَحْنُ بَعْدُ خُطَاةٌ مَاتَ الْمَسِيحُ لأَجْلِنَا.
\par 9 فَبِالأَوْلَى كَثِيراً وَنَحْنُ مُتَبَرِّرُونَ الآنَ بِدَمِهِ نَخْلُصُ بِهِ مِنَ الْغَضَبِ.
\par 10 لأَنَّهُ إِنْ كُنَّا وَنَحْنُ أَعْدَاءٌ قَدْ صُولِحْنَا مَعَ اللهِ بِمَوْتِ ابْنِهِ فَبِالأَوْلَى كَثِيراً وَنَحْنُ مُصَالَحُونَ نَخْلُصُ بِحَيَاتِهِ.
\par 11 وَلَيْسَ ذَلِكَ فَقَطْ بَلْ نَفْتَخِرُ أَيْضاً بِاللَّهِ بِرَبِّنَا يَسُوعَ الْمَسِيحِ الَّذِي نِلْنَا بِهِ الآنَ الْمُصَالَحَةَ.
\par 12 مِنْ أَجْلِ ذَلِكَ كَأَنَّمَا بِإِنْسَانٍ وَاحِدٍ دَخَلَتِ الْخَطِيَّةُ إِلَى الْعَالَمِ وَبِالْخَطِيَّةِ الْمَوْتُ وَهَكَذَا اجْتَازَ الْمَوْتُ إِلَى جَمِيعِ النَّاسِ إِذْ أَخْطَأَ الْجَمِيعُ.
\par 13 فَإِنَّهُ حَتَّى النَّامُوسِ كَانَتِ الْخَطِيَّةُ فِي الْعَالَمِ. عَلَى أَنَّ الْخَطِيَّةَ لاَ تُحْسَبُ إِنْ لَمْ يَكُنْ نَامُوسٌ.
\par 14 لَكِنْ قَدْ مَلَكَ الْمَوْتُ مِنْ آدَمَ إِلَى مُوسَى وَذَلِكَ عَلَى الَّذِينَ لَمْ يُخْطِئُوا عَلَى شِبْهِ تَعَدِّي آدَمَ الَّذِي هُوَ مِثَالُ الآتِي.
\par 15 وَلَكِنْ لَيْسَ كَالْخَطِيَّةِ هَكَذَا أَيْضاً الْهِبَةُ. لأَنَّهُ إِنْ كَانَ بِخَطِيَّةِ وَاحِدٍ مَاتَ الْكَثِيرُونَ فَبِالأَوْلَى كَثِيراً نِعْمَةُ اللهِ وَالْعَطِيَّةُ بِالنِّعْمَةِ الَّتِي بِالإِنْسَانِ الْوَاحِدِ يَسُوعَ الْمَسِيحِ قَدِ ازْدَادَتْ لِلْكَثِيرِينَ.
\par 16 وَلَيْسَ كَمَا بِوَاحِدٍ قَدْ أَخْطَأَ هَكَذَا الْعَطِيَّةُ. لأَنَّ الْحُكْمَ مِنْ وَاحِدٍ لِلدَّيْنُونَةِ وَأَمَّا الْهِبَةُ فَمِنْ جَرَّى خَطَايَا كَثِيرَةٍ لِلتَّبْرِيرِ.
\par 17 لأَنَّهُ إِنْ كَانَ بِخَطِيَّةِ الْوَاحِدِ قَدْ مَلَكَ الْمَوْتُ بِالْوَاحِدِ فَبِالأَوْلَى كَثِيراً الَّذِينَ يَنَالُونَ فَيْضَ النِّعْمَةِ وَعَطِيَّةَ الْبِرِّ سَيَمْلِكُونَ فِي الْحَيَاةِ بِالْوَاحِدِ يَسُوعَ الْمَسِيحِ.
\par 18 فَإِذاً كَمَا بِخَطِيَّةٍ وَاحِدَةٍ صَارَ الْحُكْمُ إِلَى جَمِيعِ النَّاسِ لِلدَّيْنُونَةِ هَكَذَا بِبِرٍّ وَاحِدٍ صَارَتِ الْهِبَةُ إِلَى جَمِيعِ النَّاسِ لِتَبْرِيرِ الْحَيَاةِ.
\par 19 لأَنَّهُ كَمَا بِمَعْصِيَةِ الإِنْسَانِ الْوَاحِدِ جُعِلَ الْكَثِيرُونَ خُطَاةً هَكَذَا أَيْضاً بِإِطَاعَةِ الْوَاحِدِ سَيُجْعَلُ الْكَثِيرُونَ أَبْرَاراً.
\par 20 وَأَمَّا النَّامُوسُ فَدَخَلَ لِكَيْ تَكْثُرَ الْخَطِيَّةُ. وَلَكِنْ حَيْثُ كَثُرَتِ الْخَطِيَّةُ ازْدَادَتِ النِّعْمَةُ جِدّاً.
\par 21 حَتَّى كَمَا مَلَكَتِ الْخَطِيَّةُ فِي الْمَوْتِ هَكَذَا تَمْلِكُ النِّعْمَةُ بِالْبِرِّ لِلْحَيَاةِ الأَبَدِيَّةِ بِيَسُوعَ الْمَسِيحِ رَبِّنَا.

\chapter{6}

\par 1 فَمَاذَا نَقُولُ؟ أَنَبْقَى فِي الْخَطِيَّةِ لِكَيْ تَكْثُرَ النِّعْمَةُ؟
\par 2 حَاشَا! نَحْنُ الَّذِينَ مُتْنَا عَنِ الْخَطِيَّةِ كَيْفَ نَعِيشُ بَعْدُ فِيهَا؟
\par 3 أَمْ تَجْهَلُونَ أَنَّنَا كُلَّ مَنِ اعْتَمَدَ لِيَسُوعَ الْمَسِيحِ اعْتَمَدْنَا لِمَوْتِهِ
\par 4 فَدُفِنَّا مَعَهُ بِالْمَعْمُودِيَّةِ لِلْمَوْتِ حَتَّى كَمَا أُقِيمَ الْمَسِيحُ مِنَ الأَمْوَاتِ بِمَجْدِ الآبِ هَكَذَا نَسْلُكُ نَحْنُ أَيْضاً فِي جِدَّةِ الْحَيَاةِ.
\par 5 لأَنَّهُ إِنْ كُنَّا قَدْ صِرْنَا مُتَّحِدِينَ مَعَهُ بِشِبْهِ مَوْتِهِ نَصِيرُ أَيْضاً بِقِيَامَتِهِ.
\par 6 عَالِمِينَ هَذَا: أَنَّ إِنْسَانَنَا الْعَتِيقَ قَدْ صُلِبَ مَعَهُ لِيُبْطَلَ جَسَدُ الْخَطِيَّةِ كَيْ لاَ نَعُودَ نُسْتَعْبَدُ أَيْضاً لِلْخَطِيَّةِ.
\par 7 لأَنَّ الَّذِي مَاتَ قَدْ تَبَرَّأَ مِنَ الْخَطِيَّةِ.
\par 8 فَإِنْ كُنَّا قَدْ مُتْنَا مَعَ الْمَسِيحِ نُؤْمِنُ أَنَّنَا سَنَحْيَا أَيْضاً مَعَهُ.
\par 9 عَالِمِينَ أَنَّ الْمَسِيحَ بَعْدَمَا أُقِيمَ مِنَ الأَمْوَاتِ لاَ يَمُوتُ أَيْضاً. لاَ يَسُودُ عَلَيْهِ الْمَوْتُ بَعْدُ.
\par 10 لأَنَّ الْمَوْتَ الَّذِي مَاتَهُ قَدْ مَاتَهُ لِلْخَطِيَّةِ مَرَّةً وَاحِدَةً وَالْحَيَاةُ الَّتِي يَحْيَاهَا فَيَحْيَاهَا لِلَّهِ.
\par 11 كَذَلِكَ أَنْتُمْ أَيْضاً احْسِبُوا أَنْفُسَكُمْ أَمْوَاتاً عَنِ الْخَطِيَّةِ وَلَكِنْ أَحْيَاءً لِلَّهِ بِالْمَسِيحِ يَسُوعَ رَبِّنَا.
\par 12 إِذاً لاَ تَمْلِكَنَّ الْخَطِيَّةُ فِي جَسَدِكُمُ الْمَائِتِ لِكَيْ تُطِيعُوهَا فِي شَهَوَاتِهِ
\par 13 وَلاَ تُقَدِّمُوا أَعْضَاءَكُمْ آلاَتِ إِثْمٍ لِلْخَطِيَّةِ بَلْ قَدِّمُوا ذَوَاتِكُمْ لِلَّهِ كَأَحْيَاءٍ مِنَ الأَمْوَاتِ وَأَعْضَاءَكُمْ آلاَتِ بِرٍّ لِلَّهِ.
\par 14 فَإِنَّ الْخَطِيَّةَ لَنْ تَسُودَكُمْ لأَنَّكُمْ لَسْتُمْ تَحْتَ النَّامُوسِ بَلْ تَحْتَ النِّعْمَةِ.
\par 15 فَمَاذَا إِذاً؟ أَنُخْطِئُ لأَنَّنَا لَسْنَا تَحْتَ النَّامُوسِ بَلْ تَحْتَ النِّعْمَةِ؟ حَاشَا!
\par 16 أَلَسْتُمْ تَعْلَمُونَ أَنَّ الَّذِي تُقَدِّمُونَ ذَوَاتِكُمْ لَهُ عَبِيداً لِلطَّاعَةِ أَنْتُمْ عَبِيدٌ لِلَّذِي تُطِيعُونَهُ إِمَّا لِلْخَطِيَّةِ لِلْمَوْتِ أَوْ لِلطَّاعَةِ لِلْبِرِّ؟
\par 17 فَشُكْراً لِلَّهِ أَنَّكُمْ كُنْتُمْ عَبِيداً لِلْخَطِيَّةِ وَلَكِنَّكُمْ أَطَعْتُمْ مِنَ الْقَلْبِ صُورَةَ التَّعْلِيمِ الَّتِي تَسَلَّمْتُمُوهَا.
\par 18 وَإِذْ أُعْتِقْتُمْ مِنَ الْخَطِيَّةِ صِرْتُمْ عَبِيداً لِلْبِرِّ.
\par 19 أَتَكَلَّمُ إِنْسَانِيّاً مِنْ أَجْلِ ضُعْفِ جَسَدِكُمْ. لأَنَّهُ كَمَا قَدَّمْتُمْ أَعْضَاءَكُمْ عَبِيداً لِلنَّجَاسَةِ وَالإِثْمِ لِلإِثْمِ هَكَذَا الآنَ قَدِّمُوا أَعْضَاءَكُمْ عَبِيداً لِلْبِرِّ لِلْقَدَاسَةِ.
\par 20 لأَنَّكُمْ لَمَّا كُنْتُمْ عَبِيدَ الْخَطِيَّةِ كُنْتُمْ أَحْرَاراً مِنَ الْبِرِّ.
\par 21 فَأَيُّ ثَمَرٍ كَانَ لَكُمْ حِينَئِذٍ مِنَ الأُمُورِ الَّتِي تَسْتَحُونَ بِهَا الآنَ؟ لأَنَّ نِهَايَةَ تِلْكَ الأُمُورِ هِيَ الْمَوْتُ.
\par 22 وَأَمَّا الآنَ إِذْ أُعْتِقْتُمْ مِنَ الْخَطِيَّةِ وَصِرْتُمْ عَبِيداً لِلَّهِ فَلَكُمْ ثَمَرُكُمْ لِلْقَدَاسَةِ وَالنِّهَايَةُ حَيَاةٌ أَبَدِيَّةٌ.
\par 23 لأَنَّ أُجْرَةَ الْخَطِيَّةِ هِيَ مَوْتٌ وَأَمَّا هِبَةُ اللهِ فَهِيَ حَيَاةٌ أَبَدِيَّةٌ بِالْمَسِيحِ يَسُوعَ رَبِّنَا.

\chapter{7}

\par 1 أَمْ تَجْهَلُونَ أَيُّهَا الإِخْوَةُ - لأَنِّي أُكَلِّمُ الْعَارِفِينَ بِالنَّامُوسِ - أَنَّ النَّامُوسَ يَسُودُ عَلَى الإِنْسَانِ مَا دَامَ حَيّاً.
\par 2 فَإِنَّ الْمَرْأَةَ الَّتِي تَحْتَ رَجُلٍ هِيَ مُرْتَبِطَةٌ بِالنَّامُوسِ بِالرَّجُلِ الْحَيِّ. وَلَكِنْ إِنْ مَاتَ الرَّجُلُ فَقَدْ تَحَرَّرَتْ مِنْ نَامُوسِ الرَّجُلِ.
\par 3 فَإِذاً مَا دَامَ الرَّجُلُ حَيّاً تُدْعَى زَانِيَةً إِنْ صَارَتْ لِرَجُلٍ آخَرَ. وَلَكِنْ إِنْ مَاتَ الرَّجُلُ فَهِيَ حُرَّةٌ مِنَ النَّامُوسِ حَتَّى إِنَّهَا لَيْسَتْ زَانِيَةً إِنْ صَارَتْ لِرَجُلٍ آخَرَ.
\par 4 إِذاً يَا إِخْوَتِي أَنْتُمْ أَيْضاً قَدْ مُتُّمْ لِلنَّامُوسِ بِجَسَدِ الْمَسِيحِ لِكَيْ تَصِيرُوا لِآخَرَ لِلَّذِي قَدْ أُقِيمَ مِنَ الأَمْوَاتِ لِنُثْمِرَ لِلَّهِ.
\par 5 لأَنَّهُ لَمَّا كُنَّا فِي الْجَسَدِ كَانَتْ أَهْوَاءُ الْخَطَايَا الَّتِي بِالنَّامُوسِ تَعْمَلُ فِي أَعْضَائِنَا لِكَيْ نُثْمِرَ لِلْمَوْتِ.
\par 6 وَأَمَّا الآنَ فَقَدْ تَحَرَّرْنَا مِنَ النَّامُوسِ إِذْ مَاتَ الَّذِي كُنَّا مُمْسَكِينَ فِيهِ حَتَّى نَعْبُدَ بِجِدَّةِ الرُّوحِ لاَ بِعِتْقِ الْحَرْفِ.
\par 7 فَمَاذَا نَقُولُ؟ هَلِ النَّامُوسُ خَطِيَّةٌ؟ حَاشَا! بَلْ لَمْ أَعْرِفِ الْخَطِيَّةَ إِلاَّ بِالنَّامُوسِ. فَإِنَّنِي لَمْ أَعْرِفِ الشَّهْوَةَ لَوْ لَمْ يَقُلِ النَّامُوسُ «لاَ تَشْتَهِ».
\par 8 وَلَكِنَّ الْخَطِيَّةَ وَهِيَ مُتَّخِذَةٌ فُرْصَةً بِالْوَصِيَّةِ أَنْشَأَتْ فِيَّ كُلَّ شَهْوَةٍ. لأَنْ بِدُونِ النَّامُوسِ الْخَطِيَّةُ مَيِّتَةٌ.
\par 9 أَمَّا أَنَا فَكُنْتُ بِدُونِ النَّامُوسِ عَائِشاً قَبْلاً. وَلَكِنْ لَمَّا جَاءَتِ الْوَصِيَّةُ عَاشَتِ الْخَطِيَّةُ فَمُتُّ أَنَا
\par 10 فَوُجِدَتِ الْوَصِيَّةُ الَّتِي لِلْحَيَاةِ هِيَ نَفْسُهَا لِي لِلْمَوْتِ.
\par 11 لأَنَّ الْخَطِيَّةَ وَهِيَ مُتَّخِذَةٌ فُرْصَةً بِالْوَصِيَّةِ خَدَعَتْنِي بِهَا وَقَتَلَتْنِي.
\par 12 إِذاً النَّامُوسُ مُقَدَّسٌ وَالْوَصِيَّةُ مُقَدَّسَةٌ وَعَادِلَةٌ وَصَالِحَةٌ.
\par 13 فَهَلْ صَارَ لِي الصَّالِحُ مَوْتاً؟ حَاشَا! بَلِ الْخَطِيَّةُ. لِكَيْ تَظْهَرَ خَطِيَّةً مُنْشِئَةً لِي بِالصَّالِحِ مَوْتاً لِكَيْ تَصِيرَ الْخَطِيَّةُ خَاطِئَةً جِدّاً بِالْوَصِيَّةِ.
\par 14 فَإِنَّنَا نَعْلَمُ أَنَّ النَّامُوسَ رُوحِيٌّ وَأَمَّا أَنَا فَجَسَدِيٌّ مَبِيعٌ تَحْتَ الْخَطِيَّةِ.
\par 15 لأَنِّي لَسْتُ أَعْرِفُ مَا أَنَا أَفْعَلُهُ إِذْ لَسْتُ أَفْعَلُ مَا أُرِيدُهُ بَلْ مَا أُبْغِضُهُ فَإِيَّاهُ أَفْعَلُ.
\par 16 فَإِنْ كُنْتُ أَفْعَلُ مَا لَسْتُ أُرِيدُهُ فَإِنِّي أُصَادِقُ النَّامُوسَ أَنَّهُ حَسَنٌ.
\par 17 فَالآنَ لَسْتُ بَعْدُ أَفْعَلُ ذَلِكَ أَنَا بَلِ الْخَطِيَّةُ السَّاكِنَةُ فِيَّ.
\par 18 فَإِنِّي أَعْلَمُ أَنَّهُ لَيْسَ سَاكِنٌ فِيَّ أَيْ فِي جَسَدِي شَيْءٌ صَالِحٌ. لأَنَّ الإِرَادَةَ حَاضِرَةٌ عِنْدِي وَأَمَّا أَنْ أَفْعَلَ الْحُسْنَى فَلَسْتُ أَجِدُ.
\par 19 لأَنِّي لَسْتُ أَفْعَلُ الصَّالِحَ الَّذِي أُرِيدُهُ بَلِ الشَّرَّ الَّذِي لَسْتُ أُرِيدُهُ فَإِيَّاهُ أَفْعَلُ.
\par 20 فَإِنْ كُنْتُ مَا لَسْتُ أُرِيدُهُ إِيَّاهُ أَفْعَلُ فَلَسْتُ بَعْدُ أَفْعَلُهُ أَنَا بَلِ الْخَطِيَّةُ السَّاكِنَةُ فِيَّ.
\par 21 إِذاً أَجِدُ النَّامُوسَ لِي حِينَمَا أُرِيدُ أَنْ أَفْعَلَ الْحُسْنَى أَنَّ الشَّرَّ حَاضِرٌ عِنْدِي.
\par 22 فَإِنِّي أُسَرُّ بِنَامُوسِ اللهِ بِحَسَبِ الإِنْسَانِ الْبَاطِنِ.
\par 23 وَلَكِنِّي أَرَى نَامُوساً آخَرَ فِي أَعْضَائِي يُحَارِبُ نَامُوسَ ذِهْنِي وَيَسْبِينِي إِلَى نَامُوسِ الْخَطِيَّةِ الْكَائِنِ فِي أَعْضَائِي.
\par 24 وَيْحِي أَنَا الإِنْسَانُ الشَّقِيُّ! مَنْ يُنْقِذُنِي مِنْ جَسَدِ هَذَا الْمَوْتِ؟
\par 25 أَشْكُرُ اللهَ بِيَسُوعَ الْمَسِيحِ رَبِّنَا! إِذاً أَنَا نَفْسِي بِذِهْنِي أَخْدِمُ نَامُوسَ اللهِ وَلَكِنْ بِالْجَسَدِ نَامُوسَ الْخَطِيَّةِ.

\chapter{8}

\par 1 إِذاً لاَ شَيْءَ مِنَ الدَّيْنُونَةِ الآنَ عَلَى الَّذِينَ هُمْ فِي الْمَسِيحِ يَسُوعَ السَّالِكِينَ لَيْسَ حَسَبَ الْجَسَدِ بَلْ حَسَبَ الرُّوحِ.
\par 2 لأَنَّ نَامُوسَ رُوحِ الْحَيَاةِ فِي الْمَسِيحِ يَسُوعَ قَدْ أَعْتَقَنِي مِنْ نَامُوسِ الْخَطِيَّةِ وَالْمَوْتِ.
\par 3 لأَنَّهُ مَا كَانَ النَّامُوسُ عَاجِزاً عَنْهُ فِي مَا كَانَ ضَعِيفاً بِالْجَسَدِ فَاللَّهُ إِذْ أَرْسَلَ ابْنَهُ فِي شِبْهِ جَسَدِ الْخَطِيَّةِ وَلأَجْلِ الْخَطِيَّةِ دَانَ الْخَطِيَّةَ فِي الْجَسَدِ
\par 4 لِكَيْ يَتِمَّ حُكْمُ النَّامُوسِ فِينَا نَحْنُ السَّالِكِينَ لَيْسَ حَسَبَ الْجَسَدِ بَلْ حَسَبَ الرُّوحِ.
\par 5 فَإِنَّ الَّذِينَ هُمْ حَسَبَ الْجَسَدِ فَبِمَا لِلْجَسَدِ يَهْتَمُّونَ وَلَكِنَّ الَّذِينَ حَسَبَ الرُّوحِ فَبِمَا لِلرُّوحِ.
\par 6 لأَنَّ اهْتِمَامَ الْجَسَدِ هُوَ مَوْتٌ وَلَكِنَّ اهْتِمَامَ الرُّوحِ هُوَ حَيَاةٌ وَسَلاَمٌ.
\par 7 لأَنَّ اهْتِمَامَ الْجَسَدِ هُوَ عَدَاوَةٌ لِلَّهِ إِذْ لَيْسَ هُوَ خَاضِعاً لِنَامُوسِ اللهِ لأَنَّهُ أَيْضاً لاَ يَسْتَطِيعُ.
\par 8 فَالَّذِينَ هُمْ فِي الْجَسَدِ لاَ يَسْتَطِيعُونَ أَنْ يُرْضُوا اللهَ.
\par 9 وَأَمَّا أَنْتُمْ فَلَسْتُمْ فِي الْجَسَدِ بَلْ فِي الرُّوحِ إِنْ كَانَ رُوحُ اللهِ سَاكِناً فِيكُمْ. وَلَكِنْ إِنْ كَانَ أَحَدٌ لَيْسَ لَهُ رُوحُ الْمَسِيحِ فَذَلِكَ لَيْسَ لَهُ.
\par 10 وَإِنْ كَانَ الْمَسِيحُ فِيكُمْ فَالْجَسَدُ مَيِّتٌ بِسَبَبِ الْخَطِيَّةِ وَأَمَّا الرُّوحُ فَحَيَاةٌ بِسَبَبِ الْبِرِّ.
\par 11 وَإِنْ كَانَ رُوحُ الَّذِي أَقَامَ يَسُوعَ مِنَ الأَمْوَاتِ سَاكِناً فِيكُمْ فَالَّذِي أَقَامَ الْمَسِيحَ مِنَ الأَمْوَاتِ سَيُحْيِي أَجْسَادَكُمُ الْمَائِتَةَ أَيْضاً بِرُوحِهِ السَّاكِنِ فِيكُمْ.
\par 12 فَإِذاً أَيُّهَا الإِخْوَةُ نَحْنُ مَدْيُونُونَ لَيْسَ لِلْجَسَدِ لِنَعِيشَ حَسَبَ الْجَسَدِ.
\par 13 لأَنَّهُ إِنْ عِشْتُمْ حَسَبَ الْجَسَدِ فَسَتَمُوتُونَ وَلَكِنْ إِنْ كُنْتُمْ بِالرُّوحِ تُمِيتُونَ أَعْمَالَ الْجَسَدِ فَسَتَحْيَوْنَ.
\par 14 لأَنَّ كُلَّ الَّذِينَ يَنْقَادُونَ بِرُوحِ اللهِ فَأُولَئِكَ هُمْ أَبْنَاءُ اللهِ.
\par 15 إِذْ لَمْ تَأْخُذُوا رُوحَ الْعُبُودِيَّةِ أَيْضاً لِلْخَوْفِ بَلْ أَخَذْتُمْ رُوحَ التَّبَنِّي الَّذِي بِهِ نَصْرُخُ: «يَا أَبَا الآبُ!».
\par 16 اَلرُّوحُ نَفْسُهُ أَيْضاً يَشْهَدُ لأَرْوَاحِنَا أَنَّنَا أَوْلاَدُ اللهِ.
\par 17 فَإِنْ كُنَّا أَوْلاَداً فَإِنَّنَا وَرَثَةٌ أَيْضاً وَرَثَةُ اللهِ وَوَارِثُونَ مَعَ الْمَسِيحِ. إِنْ كُنَّا نَتَأَلَّمُ مَعَهُ لِكَيْ نَتَمَجَّدَ أَيْضاً مَعَهُ.
\par 18 فَإِنِّي أَحْسِبُ أَنَّ آلاَمَ الزَّمَانِ الْحَاضِرِ لاَ تُقَاسُ بِالْمَجْدِ الْعَتِيدِ أَنْ يُسْتَعْلَنَ فِينَا.
\par 19 لأَنَّ انْتِظَارَ الْخَلِيقَةِ يَتَوَقَّعُ اسْتِعْلاَنَ أَبْنَاءِ اللهِ.
\par 20 إِذْ أُخْضِعَتِ الْخَلِيقَةُ لِلْبُطْلِ - لَيْسَ طَوْعاً بَلْ مِنْ أَجْلِ الَّذِي أَخْضَعَهَا - عَلَى الرَّجَاءِ.
\par 21 لأَنَّ الْخَلِيقَةَ نَفْسَهَا أَيْضاً سَتُعْتَقُ مِنْ عُبُودِيَّةِ الْفَسَادِ إِلَى حُرِّيَّةِ مَجْدِ أَوْلاَدِ اللهِ.
\par 22 فَإِنَّنَا نَعْلَمُ أَنَّ كُلَّ الْخَلِيقَةِ تَئِنُّ وَتَتَمَخَّضُ مَعاً إِلَى الآنَ.
\par 23 وَلَيْسَ هَكَذَا فَقَطْ بَلْ نَحْنُ الَّذِينَ لَنَا بَاكُورَةُ الرُّوحِ نَحْنُ أَنْفُسُنَا أَيْضاً نَئِنُّ فِي أَنْفُسِنَا مُتَوَقِّعِينَ التَّبَنِّيَ فِدَاءَ أَجْسَادِنَا.
\par 24 لأَنَّنَا بِالرَّجَاءِ خَلَصْنَا. وَلَكِنَّ الرَّجَاءَ الْمَنْظُورَ لَيْسَ رَجَاءً لأَنَّ مَا يَنْظُرُهُ أَحَدٌ كَيْفَ يَرْجُوهُ أَيْضاً؟
\par 25 وَلَكِنْ إِنْ كُنَّا نَرْجُو مَا لَسْنَا نَنْظُرُهُ فَإِنَّنَا نَتَوَقَّعُهُ بِالصَّبْرِ.
\par 26 وَكَذَلِكَ الرُّوحُ أَيْضاً يُعِينُ ضَعَفَاتِنَا لأَنَّنَا لَسْنَا نَعْلَمُ مَا نُصَلِّي لأَجْلِهِ كَمَا يَنْبَغِي. وَلَكِنَّ الرُّوحَ نَفْسَهُ يَشْفَعُ فِينَا بِأَنَّاتٍ لاَ يُنْطَقُ بِهَا.
\par 27 وَلَكِنَّ الَّذِي يَفْحَصُ الْقُلُوبَ يَعْلَمُ مَا هُوَ اهْتِمَامُ الرُّوحِ لأَنَّهُ بِحَسَبِ مَشِيئَةِ اللهِ يَشْفَعُ فِي الْقِدِّيسِينَ.
\par 28 وَنَحْنُ نَعْلَمُ أَنَّ كُلَّ الأَشْيَاءِ تَعْمَلُ مَعاً لِلْخَيْرِ لِلَّذِينَ يُحِبُّونَ اللهَ الَّذِينَ هُمْ مَدْعُوُّونَ حَسَبَ قَصْدِهِ.
\par 29 لأَنَّ الَّذِينَ سَبَقَ فَعَرَفَهُمْ سَبَقَ فَعَيَّنَهُمْ لِيَكُونُوا مُشَابِهِينَ صُورَةَ ابْنِهِ لِيَكُونَ هُوَ بِكْراً بَيْنَ إِخْوَةٍ كَثِيرِينَ.
\par 30 وَالَّذِينَ سَبَقَ فَعَيَّنَهُمْ فَهَؤُلاَءِ دَعَاهُمْ أَيْضاً. وَالَّذِينَ دَعَاهُمْ فَهَؤُلاَءِ بَرَّرَهُمْ أَيْضاً. وَالَّذِينَ بَرَّرَهُمْ فَهَؤُلاَءِ مَجَّدَهُمْ أَيْضاً.
\par 31 فَمَاذَا نَقُولُ لِهَذَا؟ إِنْ كَانَ اللهُ مَعَنَا فَمَنْ عَلَيْنَا!
\par 32 اَلَّذِي لَمْ يُشْفِقْ عَلَى ابْنِهِ بَلْ بَذَلَهُ لأَجْلِنَا أَجْمَعِينَ كَيْفَ لاَ يَهَبُنَا أَيْضاً مَعَهُ كُلَّ شَيْءٍ؟
\par 33 مَنْ سَيَشْتَكِي عَلَى مُخْتَارِي اللهِ؟ اللَّهُ هُوَ الَّذِي يُبَرِّرُ!
\par 34 مَنْ هُوَ الَّذِي يَدِينُ؟ الْمَسِيحُ هُوَ الَّذِي مَاتَ بَلْ بِالْحَرِيِّ قَامَ أَيْضاً الَّذِي هُوَ أَيْضاً عَنْ يَمِينِ اللهِ الَّذِي أَيْضاً يَشْفَعُ فِينَا!
\par 35 مَنْ سَيَفْصِلُنَا عَنْ مَحَبَّةِ الْمَسِيحِ؟ أَشِدَّةٌ أَمْ ضَِيْقٌ أَمِ اضْطِهَادٌ أَمْ جُوعٌ أَمْ عُرْيٌ أَمْ خَطَرٌ أَمْ سَيْفٌ؟
\par 36 كَمَا هُوَ مَكْتُوبٌ «إِنَّنَا مِنْ أَجْلِكَ نُمَاتُ كُلَّ النَّهَارِ. قَدْ حُسِبْنَا مِثْلَ غَنَمٍ لِلذَّبْحِ».
\par 37 وَلَكِنَّنَا فِي هَذِهِ جَمِيعِهَا يَعْظُمُ انْتِصَارُنَا بِالَّذِي أَحَبَّنَا.
\par 38 فَإِنِّي مُتَيَقِّنٌ أَنَّهُ لاَ مَوْتَ وَلاَ حَيَاةَ وَلاَ مَلاَئِكَةَ وَلاَ رُؤَسَاءَ وَلاَ قُوَّاتِ وَلاَ أُمُورَ حَاضِرَةً وَلاَ مُسْتَقْبَلَةً
\par 39 وَلاَ عُلْوَ وَلاَ عُمْقَ وَلاَ خَلِيقَةَ أُخْرَى تَقْدِرُ أَنْ تَفْصِلَنَا عَنْ مَحَبَّةِ اللهِ الَّتِي فِي الْمَسِيحِ يَسُوعَ رَبِّنَا.

\chapter{9}

\par 1 أَقُولُ الصِّدْقَ فِي الْمَسِيحِ لاَ أَكْذِبُ وَضَمِيرِي شَاهِدٌ لِي بِالرُّوحِ الْقُدُسِ:
\par 2 إِنَّ لِي حُزْناً عَظِيماً وَوَجَعاً فِي قَلْبِي لاَ يَنْقَطِعُ!
\par 3 فَإِنِّي كُنْتُ أَوَدُّ لَوْ أَكُونُ أَنَا نَفْسِي مَحْرُوماً مِنَ الْمَسِيحِ لأَجْلِ إِخْوَتِي أَنْسِبَائِي حَسَبَ الْجَسَدِ
\par 4 الَّذِينَ هُمْ إِسْرَائِيلِيُّونَ وَلَهُمُ التَّبَنِّي وَالْمَجْدُ وَالْعُهُودُ وَالِاشْتِرَاعُ وَالْعِبَادَةُ وَالْمَوَاعِيدُ
\par 5 وَلَهُمُ الآبَاءُ وَمِنْهُمُ الْمَسِيحُ حَسَبَ الْجَسَدِ الْكَائِنُ عَلَى الْكُلِّ إِلَهاً مُبَارَكاً إِلَى الأَبَدِ. آمِينَ.
\par 6 وَلَكِنْ لَيْسَ هَكَذَا حَتَّى إِنَّ كَلِمَةَ اللهِ قَدْ سَقَطَتْ. لأَنْ لَيْسَ جَمِيعُ الَّذِينَ مِنْ إِسْرَائِيلَ هُمْ إِسْرَائِيلِيُّونَ
\par 7 وَلاَ لأَنَّهُمْ مِنْ نَسْلِ إِبْرَاهِيمَ هُمْ جَمِيعاً أَوْلاَدٌ. بَلْ «بِإِسْحَاقَ يُدْعَى لَكَ نَسْلٌ».
\par 8 أَيْ لَيْسَ أَوْلاَدُ الْجَسَدِ هُمْ أَوْلاَدَ اللهِ بَلْ أَوْلاَدُ الْمَوْعِدِ يُحْسَبُونَ نَسْلاً.
\par 9 لأَنَّ كَلِمَةَ الْمَوْعِدِ هِيَ هَذِهِ: «أَنَا آتِي نَحْوَ هَذَا الْوَقْتِ وَيَكُونُ لِسَارَةَ ابْنٌ».
\par 10 وَلَيْسَ ذَلِكَ فَقَطْ بَلْ رِفْقَةُ أَيْضاً وَهِيَ حُبْلَى مِنْ وَاحِدٍ وَهُوَ إِسْحَاقُ أَبُونَا -
\par 11 لأَنَّهُ وَهُمَا لَمْ يُولَدَا بَعْدُ وَلاَ فَعَلاَ خَيْراً أَوْ شَرّاً لِكَيْ يَثْبُتَ قَصْدُ اللهِ حَسَبَ الِاخْتِيَارِ لَيْسَ مِنَ الأَعْمَالِ بَلْ مِنَ الَّذِي يَدْعُو
\par 12 قِيلَ لَهَا: «إِنَّ الْكَبِيرَ يُسْتَعْبَدُ لِلصَّغِيرِ».
\par 13 كَمَا هُوَ مَكْتُوبٌ: «أَحْبَبْتُ يَعْقُوبَ وَأَبْغَضْتُ عِيسُوَ».
\par 14 فَمَاذَا نَقُولُ؟ أَلَعَلَّ عِنْدَ اللهِ ظُلْماً؟ حَاشَا!
\par 15 لأَنَّهُ يَقُولُ لِمُوسَى: «إِنِّي أَرْحَمُ مَنْ أَرْحَمُ وَأَتَرَاءَفُ عَلَى مَنْ أَتَرَاءَفُ».
\par 16 فَإِذاً لَيْسَ لِمَنْ يَشَاءُ وَلاَ لِمَنْ يَسْعَى بَلْ لِلَّهِ الَّذِي يَرْحَمُ.
\par 17 لأَنَّهُ يَقُولُ الْكِتَابُ لِفِرْعَوْنَ: «إِنِّي لِهَذَا بِعَيْنِهِ أَقَمْتُكَ لِكَيْ أُظْهِرَ فِيكَ قُوَّتِي وَلِكَيْ يُنَادَى بِاسْمِي فِي كُلِّ الأَرْضِ».
\par 18 فَإِذاً هُوَ يَرْحَمُ مَنْ يَشَاءُ وَيُقَسِّي مَنْ يَشَاءُ.
\par 19 فَسَتَقُولُ لِي: «لِمَاذَا يَلُومُ بَعْدُ لأَنْ مَنْ يُقَاوِمُ مَشِيئَتَهُ؟»
\par 20 بَلْ مَنْ أَنْتَ أَيُّهَا الإِنْسَانُ الَّذِي تُجَاوِبُ اللهَ؟ أَلَعَلَّ الْجِبْلَةَ تَقُولُ لِجَابِلِهَا: «لِمَاذَا صَنَعْتَنِي هَكَذَا؟»
\par 21 أَمْ لَيْسَ لِلْخَزَّافِ سُلْطَانٌ عَلَى الطِّينِ أَنْ يَصْنَعَ مِنْ كُتْلَةٍ وَاحِدَةٍ إِنَاءً لِلْكَرَامَةِ وَآخَرَ لِلْهَوَانِ؟
\par 22 فَمَاذَا إِنْ كَانَ اللهُ وَهُوَ يُرِيدُ أَنْ يُظْهِرَ غَضَبَهُ وَيُبَيِّنَ قُوَّتَهُ احْتَمَلَ بِأَنَاةٍ كَثِيرَةٍ آنِيَةَ غَضَبٍ مُهَيَّأَةً لِلْهَلاَكِ -
\par 23 وَلِكَيْ يُبَيِّنَ غِنَى مَجْدِهِ عَلَى آنِيَةِ رَحْمَةٍ قَدْ سَبَقَ فَأَعَدَّهَا لِلْمَجْدِ
\par 24 الَّتِي أَيْضاً دَعَانَا نَحْنُ إِيَّاهَا لَيْسَ مِنَ الْيَهُودِ فَقَطْ بَلْ مِنَ الأُمَمِ أَيْضاً.
\par 25 كَمَا يَقُولُ فِي هُوشَعَ أَيْضاً: «سَأَدْعُو الَّذِي لَيْسَ شَعْبِي شَعْبِي وَالَّتِي لَيْسَتْ مَحْبُوبَةً مَحْبُوبَةً.
\par 26 وَيَكُونُ فِي الْمَوْضِعِ الَّذِي قِيلَ لَهُمْ فِيهِ لَسْتُمْ شَعْبِي أَنَّهُ هُنَاكَ يُدْعَوْنَ أَبْنَاءَ اللهِ الْحَيِّ».
\par 27 وَإِشَعْيَاءُ يَصْرُخُ مِنْ جِهَةِ إِسْرَائِيلَ: «وَإِنْ كَانَ عَدَدُ بَنِي إِسْرَائِيلَ كَرَمْلِ الْبَحْرِ فَالْبَقِيَّةُ سَتَخْلُصُ.
\par 28 لأَنَّهُ مُتَمِّمُ أَمْرٍ وَقَاضٍ بِالْبِرِّ. لأَنَّ الرَّبَّ يَصْنَعُ أَمْراً مَقْضِيّاً بِهِ عَلَى الأَرْضِ».
\par 29 وَكَمَا سَبَقَ إِشَعْيَاءُ فَقَالَ: «لَوْلاَ أَنَّ رَبَّ الْجُنُودِ أَبْقَى لَنَا نَسْلاً لَصِرْنَا مِثْلَ سَدُومَ وَشَابَهْنَا عَمُورَةَ».
\par 30 فَمَاذَا نَقُولُ؟ إِنَّ الأُمَمَ الَّذِينَ لَمْ يَسْعَوْا فِي أَثَرِ الْبِرِّ أَدْرَكُوا الْبِرَّ - الْبِرَّ الَّذِي بِالإِيمَانِ.
\par 31 وَلَكِنَّ إِسْرَائِيلَ وَهُوَ يَسْعَى فِي أَثَرِ نَامُوسِ الْبِرِّ لَمْ يُدْرِكْ نَامُوسَ الْبِرِّ!
\par 32 لِمَاذَا؟ لأَنَّهُ فَعَلَ ذَلِكَ لَيْسَ بِالإِيمَانِ بَلْ كَأَنَّهُ بِأَعْمَالِ النَّامُوسِ. فَإِنَّهُمُ اصْطَدَمُوا بِحَجَرِ الصَّدْمَةِ
\par 33 كَمَا هُوَ مَكْتُوبٌ: «هَا أَنَا أَضَعُ فِي صِهْيَوْنَ حَجَرَ صَدْمَةٍ وَصَخْرَةَ عَثْرَةٍ وَكُلُّ مَنْ يُؤْمِنُ بِهِ لاَ يُخْزَى».

\chapter{10}

\par 1 أَيُّهَا الإِخْوَةُ إِنَّ مَسَرَّةَ قَلْبِي وَطَلِْبَتِي إِلَى اللهِ لأَجْلِ إِسْرَائِيلَ هِيَ لِلْخَلاَصِ.
\par 2 لأَنِّي أَشْهَدُ لَهُمْ أَنَّ لَهُمْ غَيْرَةً لِلَّهِ وَلَكِنْ لَيْسَ حَسَبَ الْمَعْرِفَةِ.
\par 3 لأَنَّهُمْ إِذْ كَانُوا يَجْهَلُونَ بِرَّ اللهِ وَيَطْلُبُونَ أَنْ يُثْبِتُوا بِرَّ أَنْفُسِهِمْ لَمْ يُخْضَعُوا لِبِرِّ اللهِ.
\par 4 لأَنَّ غَايَةَ النَّامُوسِ هِيَ: الْمَسِيحُ لِلْبِرِّ لِكُلِّ مَنْ يُؤْمِنُ.
\par 5 لأَنَّ مُوسَى يَكْتُبُ فِي الْبِرِّ الَّذِي بِالنَّامُوسِ: «إِنَّ الإِنْسَانَ الَّذِي يَفْعَلُهَا سَيَحْيَا بِهَا».
\par 6 وَأَمَّا الْبِرُّ الَّذِي بِالإِيمَانِ فَيَقُولُ هَكَذَا: «لاَ تَقُلْ فِي قَلْبِكَ مَنْ يَصْعَدُ إِلَى السَّمَاءِ؟» (أَيْ لِيُحْدِرَ الْمَسِيحَ)
\par 7 أَوْ «مَنْ يَهْبِطُ إِلَى الْهَاوِيَةِ؟» (أَيْ لِيُصْعِدَ الْمَسِيحَ مِنَ الأَمْوَاتِ)
\par 8 لَكِنْ مَاذَا يَقُولُ؟ «اَلْكَلِمَةُ قَرِيبَةٌ مِنْكَ فِي فَمِكَ وَفِي قَلْبِكَ» (أَيْ كَلِمَةُ الإِيمَانِ الَّتِي نَكْرِزُ بِهَا)
\par 9 لأَنَّكَ إِنِ اعْتَرَفْتَ بِفَمِكَ بِالرَّبِّ يَسُوعَ وَآمَنْتَ بِقَلْبِكَ أَنَّ اللهَ أَقَامَهُ مِنَ الأَمْوَاتِ خَلَصْتَ.
\par 10 لأَنَّ الْقَلْبَ يُؤْمَنُ بِهِ لِلْبِرِّ وَالْفَمَ يُعْتَرَفُ بِهِ لِلْخَلاَصِ.
\par 11 لأَنَّ الْكِتَابَ يَقُولُ: «كُلُّ مَنْ يُؤْمِنُ بِهِ لاَ يُخْزَى».
\par 12 لأَنَّهُ لاَ فَرْقَ بَيْنَ الْيَهُودِيِّ وَالْيُونَانِيِّ لأَنَّ رَبّاً وَاحِداً لِلْجَمِيعِ غَنِيّاً لِجَمِيعِ الَّذِينَ يَدْعُونَ بِهِ.
\par 13 لأَنَّ كُلَّ مَنْ يَدْعُو بِاسْمِ الرَّبِّ يَخْلُصُ.
\par 14 فَكَيْفَ يَدْعُونَ بِمَنْ لَمْ يُؤْمِنُوا بِهِ. وَكَيْفَ يُؤْمِنُونَ بِمَنْ لَمْ يَسْمَعُوا بِهِ؟ وَكَيْفَ يَسْمَعُونَ بِلاَ كَارِزٍ؟
\par 15 وَكَيْفَ يَكْرِزُونَ إِنْ لَمْ يُرْسَلُوا؟ كَمَا هُوَ مَكْتُوبٌ: «مَا أَجْمَلَ أَقْدَامَ الْمُبَشِّرِينَ بِالسَّلاَمِ الْمُبَشِّرِينَ بِالْخَيْرَاتِ».
\par 16 لَكِنْ لَيْسَ الْجَمِيعُ قَدْ أَطَاعُوا الإِنْجِيلَ لأَنَّ إِشَعْيَاءَ يَقُولُ: «يَا رَبُّ مَنْ صَدَّقَ خَبَرَنَا؟»
\par 17 إِذاً الإِيمَانُ بِالْخَبَرِ وَالْخَبَرُ بِكَلِمَةِ اللهِ.
\par 18 لَكِنَّنِي أَقُولُ: أَلَعَلَّهُمْ لَمْ يَسْمَعُوا؟ بَلَى! «إِلَى جَمِيعِ الأَرْضِ خَرَجَ صَوْتُهُمْ وَإِلَى أَقَاصِي الْمَسْكُونَةِ أَقْوَالُهُمْ».
\par 19 لَكِنِّي أَقُولُ: أَلَعَلَّ إِسْرَائِيلَ لَمْ يَعْلَمْ؟ أَوَّلاً مُوسَى يَقُولُ: «أَنَا أُغِيرُكُمْ بِمَا لَيْسَ أُمَّةً. بِأُمَّةٍ غَبِيَّةٍ أُغِيظُكُمْ».
\par 20 ثُمَّ إِشَعْيَاءُ يَتَجَاسَرُ وَيَقُولُ: «وُجِدْتُ مِنَ الَّذِينَ لَمْ يَطْلُبُونِي وَصِرْتُ ظَاهِراً لِلَّذِينَ لَمْ يَسْأَلُوا عَنِّي».
\par 21 أَمَّا مِنْ جِهَةِ إِسْرَائِيلَ فَيَقُولُ: «طُولَ النَّهَارِ بَسَطْتُ يَدَيَّ إِلَى شَعْبٍ مُعَانِدٍ وَمُقَاوِمٍ».

\chapter{11}

\par 1 فَأَقُولُ: أَلَعَلَّ اللهَ رَفَضَ شَعْبَهُ؟ حَاشَا! لأَنِّي أَنَا أَيْضاً إِسْرَائِيلِيٌّ مِنْ نَسْلِ إِبْرَاهِيمَ مِنْ سِبْطِ بِنْيَامِينَ.
\par 2 لَمْ يَرْفُضِ اللهُ شَعْبَهُ الَّذِي سَبَقَ فَعَرَفَهُ. أَمْ لَسْتُمْ تَعْلَمُونَ مَاذَا يَقُولُ الْكِتَابُ فِي إِيلِيَّا؟ كَيْفَ يَتَوَسَّلُ إِلَى اللهِ ضِدَّ إِسْرَائِيلَ قَائِلاً:
\par 3 «يَا رَبُّ قَتَلُوا أَنْبِيَاءَكَ وَهَدَمُوا مَذَابِحَكَ وَبَقِيتُ أَنَا وَحْدِي وَهُمْ يَطْلُبُونَ نَفْسِي».
\par 4 لَكِنْ مَاذَا يَقُولُ لَهُ الْوَحْيُ؟ «أَبْقَيْتُ لِنَفْسِي سَبْعَةَ آلاَفِ رَجُلٍ لَمْ يُحْنُوا رُكْبَةً لِبَعْلٍ».
\par 5 فَكَذَلِكَ فِي الزَّمَانِ الْحَاضِرِ أَيْضاً قَدْ حَصَلَتْ بَقِيَّةٌ حَسَبَ اخْتِيَارِ النِّعْمَةِ.
\par 6 فَإِنْ كَانَ بِالنِّعْمَةِ فَلَيْسَ بَعْدُ بِالأَعْمَالِ وَإِلاَّ فَلَيْسَتِ النِّعْمَةُ بَعْدُ نِعْمَةً. وَإِنْ كَانَ بِالأَعْمَالِ فَلَيْسَ بَعْدُ نِعْمَةً وَإِلاَّ فَالْعَمَلُ لاَ يَكُونُ بَعْدُ عَمَلاً.
\par 7 فَمَاذَا؟ مَا يَطْلُبُهُ إِسْرَائِيلُ ذَلِكَ لَمْ يَنَلْهُ وَلَكِنِ الْمُخْتَارُونَ نَالُوهُ. وَأَمَّا الْبَاقُونَ فَتَقَسَّوْا
\par 8 كَمَا هُوَ مَكْتُوبٌ: «أَعْطَاهُمُ اللهُ رُوحَ سُبَاتٍ وَعُيُوناً حَتَّى لاَ يُبْصِرُوا وَآذَاناً حَتَّى لاَ يَسْمَعُوا إِلَى هَذَا الْيَوْمِ».
\par 9 وَدَاوُدُ يَقُولُ: «لِتَصِرْ مَائِدَتُهُمْ فَخّاً وَقَنَصاً وَعَثْرَةً وَمُجَازَاةً لَهُمْ.
\par 10 لِتُظْلِمْ أَعْيُنُهُمْ كَيْ لاَ يُبْصِرُوا وَلْتَحْنِ ظُهُورَهُمْ فِي كُلِّ حِينٍ».
\par 11 فَأَقُولُ: أَلَعَلَّهُمْ عَثَرُوا لِكَيْ يَسْقُطُوا؟ حَاشَا! بَلْ بِزَلَّتِهِمْ صَارَ الْخَلاَصُ لِلأُمَمِ لإِغَارَتِهِمْ.
\par 12 فَإِنْ كَانَتْ زَلَّتُهُمْ غِنىً لِلْعَالَمِ وَنُقْصَانُهُمْ غِنىً لِلأُمَمِ فَكَمْ بِالْحَرِيِّ مِلْؤُهُمْ؟
\par 13 فَإِنِّي أَقُولُ لَكُمْ أَيُّهَا الأُمَمُ: بِمَا أَنِّي أَنَا رَسُولٌ لِلأُمَمِ أُمَجِّدُ خِدْمَتِي
\par 14 لَعَلِّي أُغِيرُ أَنْسِبَائِي وَأُخَلِّصُ أُنَاساً مِنْهُمْ.
\par 15 لأَنَّهُ إِنْ كَانَ رَفْضُهُمْ هُوَ مُصَالَحَةَ الْعَالَمِ فَمَاذَا يَكُونُ اقْتِبَالُهُمْ إِلاَّ حَيَاةً مِنَ الأَمْوَاتِ؟
\par 16 وَإِنْ كَانَتِ الْبَاكُورَةُ مُقَدَّسَةً فَكَذَلِكَ الْعَجِينُ! وَإِنْ كَانَ الأَصْلُ مُقَدَّساً فَكَذَلِكَ الأَغْصَانُ!
\par 17 فَإِنْ كَانَ قَدْ قُطِعَ بَعْضُ الأَغْصَانِ وَأَنْتَ زَيْتُونَةٌ بَرِّيَّةٌ طُعِّمْتَ فِيهَا فَصِرْتَ شَرِيكاً فِي أَصْلِ الزَّيْتُونَةِ وَدَسَمِهَا
\par 18 فَلاَ تَفْتَخِرْ عَلَى الأَغْصَانِ. وَإِنِ افْتَخَرْتَ فَأَنْتَ لَسْتَ تَحْمِلُ الأَصْلَ بَلِ الأَصْلُ إِيَّاكَ يَحْمِلُ!
\par 19 فَسَتَقُولُ: «قُطِعَتِ الأَغْصَانُ لِأُطَعَّمَ أَنَا».
\par 20 حَسَناً! مِنْ أَجْلِ عَدَمِ الإِيمَانِ قُطِعَتْ وَأَنْتَ بِالإِيمَانِ ثَبَتَّ. لاَ تَسْتَكْبِرْ بَلْ خَفْ!
\par 21 لأَنَّهُ إِنْ كَانَ اللهُ لَمْ يُشْفِقْ عَلَى الأَغْصَانِ الطَّبِيعِيَّةِ فَلَعَلَّهُ لاَ يُشْفِقُ عَلَيْكَ أَيْضاً!
\par 22 فَهُوَذَا لُطْفُ اللهِ وَصَرَامَتُهُ: أَمَّا الصَّرَامَةُ فَعَلَى الَّذِينَ سَقَطُوا وَأَمَّا اللُّطْفُ فَلَكَ إِنْ ثَبَتَّ فِي اللُّطْفِ وَإِلاَّ فَأَنْتَ أَيْضاً سَتُقْطَعُ.
\par 23 وَهُمْ إِنْ لَمْ يَثْبُتُوا فِي عَدَمِ الإِيمَانِ سَيُطَعَّمُونَ. لأَنَّ اللهَ قَادِرٌ أَنْ يُطَعِّمَهُمْ أَيْضاً.
\par 24 لأَنَّهُ إِنْ كُنْتَ أَنْتَ قَدْ قُطِعْتَ مِنَ الزَّيْتُونَةِ الْبَرِّيَّةِ حَسَبَ الطَّبِيعَةِ وَطُعِّمْتَ بِخِلاَفِ الطَّبِيعَةِ فِي زَيْتُونَةٍ جَيِّدَةٍ فَكَمْ بِالْحَرِيِّ يُطَعَّمُ هَؤُلاَءِ الَّذِينَ هُمْ حَسَبَ الطَّبِيعَةِ فِي زَيْتُونَتِهِمِ الْخَاصَّةِ؟
\par 25 فَإِنِّي لَسْتُ أُرِيدُ أَيُّهَا الإِخْوَةُ أَنْ تَجْهَلُوا هَذَا السِّرَّ لِئَلاَّ تَكُونُوا عِنْدَ أَنْفُسِكُمْ حُكَمَاءَ. أَنَّ الْقَسَاوَةَ قَدْ حَصَلَتْ جُزْئِيّاً لإِسْرَائِيلَ إِلَى أَنْ يَدْخُلَ مِلْؤُ الأُمَمِ
\par 26 وَهَكَذَا سَيَخْلُصُ جَمِيعُ إِسْرَائِيلَ. كَمَا هُوَ مَكْتُوبٌ: «سَيَخْرُجُ مِنْ صِهْيَوْنَ الْمُنْقِذُ وَيَرُدُّ الْفُجُورَ عَنْ يَعْقُوبَ.
\par 27 وَهَذَا هُوَ الْعَهْدُ مِنْ قِبَلِي لَهُمْ مَتَى نَزَعْتُ خَطَايَاهُمْ».
\par 28 مِنْ جِهَةِ الإِنْجِيلِ هُمْ أَعْدَاءٌ مِنْ أَجْلِكُمْ وَأَمَّا مِنْ جِهَةِ الِاخْتِيَارِ فَهُمْ أَحِبَّاءُ مِنْ أَجْلِ الآبَاءِ
\par 29 لأَنَّ هِبَاتِ اللهِ وَدَعْوَتَهُ هِيَ بِلاَ نَدَامَةٍ.
\par 30 فَإِنَّهُ كَمَا كُنْتُمْ أَنْتُمْ مَرَّةً لاَ تُطِيعُونَ اللهَ وَلَكِنِ الآنَ رُحِمْتُمْ بِعِصْيَانِ هَؤُلاَءِ
\par 31 هَكَذَا هَؤُلاَءِ أَيْضاً الآنَ لَمْ يُطِيعُوا لِكَيْ يُرْحَمُوا هُمْ أَيْضاً بِرَحْمَتِكُمْ.
\par 32 لأَنَّ اللهَ أَغْلَقَ عَلَى الْجَمِيعِ مَعاً فِي الْعِصْيَانِ لِكَيْ يَرْحَمَ الْجَمِيعَ.
\par 33 يَا لَعُمْقِ غِنَى اللهِ وَحِكْمَتِهِ وَعِلْمِهِ! مَا أَبْعَدَ أَحْكَامَهُ عَنِ الْفَحْصِ وَطُرُقَهُ عَنِ الِاسْتِقْصَاءِ!
\par 34 «لأَنْ مَنْ عَرَفَ فِكْرَ الرَّبِّ أَوْ مَنْ صَارَ لَهُ مُشِيراً؟
\par 35 أَوْ مَنْ سَبَقَ فَأَعْطَاهُ فَيُكَافَأَ؟».
\par 36 لأَنَّ مِنْهُ وَبِهِ وَلَهُ كُلَّ الأَشْيَاءِ. لَهُ الْمَجْدُ إِلَى الأَبَدِ. آمِينَ.

\chapter{12}

\par 1 فَأَطْلُبُ إِلَيْكُمْ أَيُّهَا الإِخْوَةُ بِرَأْفَةِ اللهِ أَنْ تُقَدِّمُوا أَجْسَادَكُمْ ذَبِيحَةً حَيَّةً مُقَدَّسَةً مَرْضِيَّةً عِنْدَ اللهِ عِبَادَتَكُمُ الْعَقْلِيَّةَ.
\par 2 وَلاَ تُشَاكِلُوا هَذَا الدَّهْرَ بَلْ تَغَيَّرُوا عَنْ شَكْلِكُمْ بِتَجْدِيدِ أَذْهَانِكُمْ لِتَخْتَبِرُوا مَا هِيَ إِرَادَةُ اللهِ الصَّالِحَةُ الْمَرْضِيَّةُ الْكَامِلَةُ.
\par 3 فَإِنِّي أَقُولُ بِالنِّعْمَةِ الْمُعْطَاةِ لِي لِكُلِّ مَنْ هُوَ بَيْنَكُمْ: أَنْ لاَ يَرْتَئِيَ فَوْقَ مَا يَنْبَغِي أَنْ يَرْتَئِيَ بَلْ يَرْتَئِيَ إِلَى التَّعَقُّلِ كَمَا قَسَمَ اللهُ لِكُلِّ وَاحِدٍ مِقْدَاراً مِنَ الإِيمَانِ.
\par 4 فَإِنَّهُ كَمَا فِي جَسَدٍ وَاحِدٍ لَنَا أَعْضَاءٌ كَثِيرَةٌ وَلَكِنْ لَيْسَ جَمِيعُ الأَعْضَاءِ لَهَا عَمَلٌ وَاحِدٌ
\par 5 هَكَذَا نَحْنُ الْكَثِيرِينَ: جَسَدٌ وَاحِدٌ فِي الْمَسِيحِ وَأَعْضَاءٌ بَعْضاً لِبَعْضٍ كُلُّ وَاحِدٍ لِلآخَرِ.
\par 6 وَلَكِنْ لَنَا مَوَاهِبُ مُخْتَلِفَةٌ بِحَسَبِ النِّعْمَةِ الْمُعْطَاةِ لَنَا: أَنُبُوَّةٌ فَبِالنِّسْبَةِ إِلَى الإِيمَانِ
\par 7 أَمْ خِدْمَةٌ فَفِي الْخِدْمَةِ أَمِ الْمُعَلِّمُ فَفِي التَّعْلِيمِ
\par 8 أَمِ الْوَاعِظُ فَفِي الْوَعْظِ الْمُعْطِي فَبِسَخَاءٍ الْمُدَبِّرُ فَبِاجْتِهَادٍ الرَّاحِمُ فَبِسُرُورٍ.
\par 9 اَلْمَحَبَّةُ فَلْتَكُنْ بِلاَ رِيَاءٍ. كُونُوا كَارِهِينَ الشَّرَّ مُلْتَصِقِينَ بِالْخَيْرِ
\par 10 وَادِّينَ بَعْضُكُمْ بَعْضاً بِالْمَحَبَّةِ الأَخَوِيَّةِ مُقَدِّمِينَ بَعْضُكُمْ بَعْضاً فِي الْكَرَامَةِ
\par 11 غَيْرَ مُتَكَاسِلِينَ فِي الِاجْتِهَادِ حَارِّينَ فِي الرُّوحِ عَابِدِينَ الرَّبَّ
\par 12 فَرِحِينَ فِي الرَّجَاءِ صَابِرِينَ فِي الضَِّيْقِ مُواظِبِينَ عَلَى الصَّلاَةِ
\par 13 مُشْتَرِكِينَ فِي احْتِيَاجَاتِ الْقِدِّيسِينَ عَاكِفِينَ عَلَى إِضَافَةِ الْغُرَبَاءِ.
\par 14 بَارِكُوا عَلَى الَّذِينَ يَضْطَهِدُونَكُمْ. بَارِكُوا وَلاَ تَلْعَنُوا.
\par 15 فَرَحاً مَعَ الْفَرِحِينَ وَبُكَاءً مَعَ الْبَاكِينَ.
\par 16 مُهْتَمِّينَ بَعْضُكُمْ لِبَعْضٍ اهْتِمَاماً وَاحِداً غَيْرَ مُهْتَمِّينَ بِالأُمُورِ الْعَالِيَةِ بَلْ مُنْقَادِينَ إِلَى الْمُتَّضِعِينَ. لاَ تَكُونُوا حُكَمَاءَ عِنْدَ أَنْفُسِكُمْ.
\par 17 لاَ تُجَازُوا أَحَداً عَنْ شَرٍّ بِشَرٍّ. مُعْتَنِينَ بِأُمُورٍ حَسَنَةٍ قُدَّامَ جَمِيعِ النَّاسِ.
\par 18 إِنْ كَانَ مُمْكِناً فَحَسَبَ طَاقَتِكُمْ سَالِمُوا جَمِيعَ النَّاسِ.
\par 19 لاَ تَنْتَقِمُوا لأَنْفُسِكُمْ أَيُّهَا الأَحِبَّاءُ بَلْ أَعْطُوا مَكَاناً لِلْغَضَبِ لأَنَّهُ مَكْتُوبٌ: «لِيَ النَّقْمَةُ أَنَا أُجَازِي يَقُولُ الرَّبُّ.
\par 20 فَإِنْ جَاعَ عَدُوُّكَ فَأَطْعِمْهُ. وَإِنْ عَطِشَ فَاسْقِهِ. لأَنَّكَ إِنْ فَعَلْتَ هَذَا تَجْمَعْ جَمْرَ نَارٍ عَلَى رَأْسِهِ».
\par 21 لاَ يَغْلِبَنَّكَ الشَّرُّ بَلِ اغْلِبِ الشَّرَّ بِالْخَيْرِ.

\chapter{13}

\par 1 لِتَخْضَعْ كُلُّ نَفْسٍ لِلسَّلاَطِين الْفَائِقَةِ لأَنَّهُ لَيْسَ سُلْطَانٌ إِلاَّ مِنَ اللهِ وَالسَّلاَطِينُ الْكَائِنَةُ هِيَ مُرَتَّبَةٌ مِنَ اللهِ
\par 2 حَتَّى إِنَّ مَنْ يُقَاوِمُ السُّلْطَانَ يُقَاوِمُ تَرْتِيبَ اللهِ وَالْمُقَاوِمُونَ سَيَأْخُذُونَ لأَنْفُسِهِمْ دَيْنُونَةً.
\par 3 فَإِنَّ الْحُكَّامَ لَيْسُوا خَوْفاً لِلأَعْمَالِ الصَّالِحَةِ بَلْ لِلشِّرِّيرَةِ. أَفَتُرِيدُ أَنْ لاَ تَخَافَ السُّلْطَانَ؟ افْعَلِ الصَّلاَحَ فَيَكُونَ لَكَ مَدْحٌ مِنْهُ
\par 4 لأَنَّهُ خَادِمُ اللهِ لِلصَّلاَحِ! وَلَكِنْ إِنْ فَعَلْتَ الشَّرَّ فَخَفْ لأَنَّهُ لاَ يَحْمِلُ السَّيْفَ عَبَثاً إِذْ هُوَ خَادِمُ اللهِ مُنْتَقِمٌ لِلْغَضَبِ مِنَ الَّذِي يَفْعَلُ الشَّرَّ.
\par 5 لِذَلِكَ يَلْزَمُ أَنْ يُخْضَعَ لَهُ لَيْسَ بِسَبَبِ الْغَضَبِ فَقَطْ بَلْ أَيْضاً بِسَبَبِ الضَّمِيرِ.
\par 6 فَإِنَّكُمْ لأَجْلِ هَذَا تُوفُونَ الْجِزْيَةَ أَيْضاً إِذْ هُمْ خُدَّامُ اللهِ مُواظِبُونَ عَلَى ذَلِكَ بِعَيْنِهِ.
\par 7 فَأَعْطُوا الْجَمِيعَ حُقُوقَهُمُ: الْجِزْيَةَ لِمَنْ لَهُ الْجِزْيَةُ. الْجِبَايَةَ لِمَنْ لَهُ الْجِبَايَةُ. وَالْخَوْفَ لِمَنْ لَهُ الْخَوْفُ. وَالإِكْرَامَ لِمَنْ لَهُ الإِكْرَامُ.
\par 8 لاَ تَكُونُوا مَدْيُونِينَ لأَحَدٍ بِشَيْءٍ إِلاَّ بِأَنْ يُحِبَّ بَعْضُكُمْ بَعْضاً لأَنَّ مَنْ أَحَبَّ غَيْرَهُ فَقَدْ أَكْمَلَ النَّامُوسَ.
\par 9 لأَنَّ «لاَ تَزْنِ لاَ تَقْتُلْ لاَ تَسْرِقْ لاَ تَشْهَدْ بِالزُّورِ لاَ تَشْتَهِ» وَإِنْ كَانَتْ وَصِيَّةً أُخْرَى هِيَ مَجْمُوعَةٌ فِي هَذِهِ الْكَلِمَةِ: «أَنْ تُحِبَّ قَرِيبَكَ كَنَفْسِكَ».
\par 10 اَلْمَحَبَّةُ لاَ تَصْنَعُ شَرّاً لِلْقَرِيبِ فَالْمَحَبَّةُ هِيَ تَكْمِيلُ النَّامُوسِ.
\par 11 هَذَا وَإِنَّكُمْ عَارِفُونَ الْوَقْتَ أَنَّهَا الآنَ سَاعَةٌ لِنَسْتَيْقِظَ مِنَ النَّوْمِ فَإِنَّ خَلاَصَنَا الآنَ أَقْرَبُ مِمَّا كَانَ حِينَ آمَنَّا.
\par 12 قَدْ تَنَاهَى اللَّيْلُ وَتَقَارَبَ النَّهَارُ فَلْنَخْلَعْ أَعْمَالَ الظُّلْمَةِ وَنَلْبَسْ أَسْلِحَةَ النُّورِ.
\par 13 لِنَسْلُكْ بِلِيَاقَةٍ كَمَا فِي النَّهَارِ لاَ بِالْبَطَرِ وَالسُّكْرِ لاَ بِالْمَضَاجِعِ وَالْعَهَرِ لاَ بِالْخِصَامِ وَالْحَسَدِ.
\par 14 بَلِ الْبَسُوا الرَّبَّ يَسُوعَ الْمَسِيحَ وَلاَ تَصْنَعُوا تَدْبِيراً لِلْجَسَدِ لأَجْلِ الشَّهَوَاتِ.

\chapter{14}

\par 1 وَمَنْ هُوَ ضَعِيفٌ فِي الإِيمَانِ فَاقْبَلُوهُ لاَ لِمُحَاكَمَةِ الأَفْكَارِ.
\par 2 وَاحِدٌ يُؤْمِنُ أَنْ يَأْكُلَ كُلَّ شَيْءٍ وَأَمَّا الضَّعِيفُ فَيَأْكُلُ بُقُولاً.
\par 3 لاَ يَزْدَرِ مَنْ يَأْكُلُ بِمَنْ لاَ يَأْكُلُ وَلاَ يَدِنْ مَنْ لاَ يَأْكُلُ مَنْ يَأْكُلُ - لأَنَّ اللهَ قَبِلَهُ.
\par 4 مَنْ أَنْتَ الَّذِي تَدِينُ عَبْدَ غَيْرِكَ؟ هُوَ لِمَوْلاَهُ يَثْبُتُ أَوْ يَسْقُطُ. وَلَكِنَّهُ سَيُثَبَّتُ لأَنَّ اللهَ قَادِرٌ أَنْ يُثَبِّتَهُ.
\par 5 وَاحِدٌ يَعْتَبِرُ يَوْماً دُونَ يَوْمٍ وَآخَرُ يَعْتَبِرُ كُلَّ يَوْمٍ - فَلْيَتَيَقَّنْ كُلُّ وَاحِدٍ فِي عَقْلِهِ:
\par 6 الَّذِي يَهْتَمُّ بِالْيَوْمِ فَلِلرَّبِّ يَهْتَمُّ وَالَّذِي لاَ يَهْتَمُّ بِالْيَوْمِ فَلِلرَّبِّ لاَ يَهْتَمُّ. وَالَّذِي يَأْكُلُ فَلِلرَّبِّ يَأْكُلُ لأَنَّهُ يَشْكُرُ اللهَ وَالَّذِي لاَ يَأْكُلُ فَلِلرَّبِّ لاَ يَأْكُلُ وَيَشْكُرُ اللهَ.
\par 7 لأَنْ لَيْسَ أَحَدٌ مِنَّا يَعِيشُ لِذَاتِهِ وَلاَ أَحَدٌ يَمُوتُ لِذَاتِهِ.
\par 8 لأَنَّنَا إِنْ عِشْنَا فَلِلرَّبِّ نَعِيشُ وَإِنْ مُتْنَا فَلِلرَّبِّ نَمُوتُ. فَإِنْ عِشْنَا وَإِنْ مُتْنَا فَلِلرَّبِّ نَحْنُ.
\par 9 لأَنَّهُ لِهَذَا مَاتَ الْمَسِيحُ وَقَامَ وَعَاشَ لِكَيْ يَسُودَ عَلَى الأَحْيَاءِ وَالأَمْوَاتِ.
\par 10 وَأَمَّا أَنْتَ فَلِمَاذَا تَدِينُ أَخَاكَ؟ أَوْ أَنْتَ أَيْضاً لِمَاذَا تَزْدَرِي بِأَخِيكَ؟ لأَنَّنَا جَمِيعاً سَوْفَ نَقِفُ أَمَامَ كُرْسِيِّ الْمَسِيحِ
\par 11 لأَنَّهُ مَكْتُوبٌ: «أَنَا حَيٌّ يَقُولُ الرَّبُّ إِنَّهُ لِي سَتَجْثُو كُلُّ رُكْبَةٍ وَكُلُّ لِسَانٍ سَيَحْمَدُ اللهَ».
\par 12 فَإِذاً كُلُّ وَاحِدٍ مِنَّا سَيُعْطِي عَنْ نَفْسِهِ حِسَاباً لِلَّهِ.
\par 13 فَلاَ نُحَاكِمْ أَيْضاً بَعْضُنَا بَعْضاً بَلْ بِالْحَرِيِّ احْكُمُوا بِهَذَا: أَنْ لاَ يُوضَعَ لِلأَخِ مَصْدَمَةٌ أَوْ مَعْثَرَةٌ.
\par 14 إِنِّي عَالِمٌ وَمُتَيَقِّنٌ فِي الرَّبِّ يَسُوعَ أَنْ لَيْسَ شَيْءٌ نَجِساً بِذَاتِهِ إِلاَّ مَنْ يَحْسِبُ شَيْئاً نَجِساً فَلَهُ هُوَ نَجِسٌ.
\par 15 فَإِنْ كَانَ أَخُوكَ بِسَبَبِ طَعَامِكَ يُحْزَنُ فَلَسْتَ تَسْلُكُ بَعْدُ حَسَبَ الْمَحَبَّةِ. لاَ تُهْلِكْ بِطَعَامِكَ ذَلِكَ الَّذِي مَاتَ الْمَسِيحُ لأَجْلِهِ.
\par 16 فَلاَ يُفْتَرَ عَلَى صَلاَحِكُمْ
\par 17 لأَنْ لَيْسَ مَلَكُوتُ اللهِ أَكْلاً وَشُرْباً بَلْ هُوَ بِرٌّ وَسَلاَمٌ وَفَرَحٌ فِي الرُّوحِ الْقُدُسِ.
\par 18 لأَنَّ مَنْ خَدَمَ الْمَسِيحَ فِي هَذِهِ فَهُوَ مَرْضِيٌّ عِنْدَ اللهِ وَمُزَكًّى عِنْدَ النَّاسِ.
\par 19 فَلْنَعْكُفْ إِذاً عَلَى مَا هُوَ لِلسَّلاَمِ وَمَا هُوَ لِلْبُنْيَانِ بَعْضُنَا لِبَعْضٍ.
\par 20 لاَ تَنْقُضْ لأَجْلِ الطَّعَامِ عَمَلَ اللهِ. كُلُّ الأَشْيَاءِ طَاهِرَةٌ لَكِنَّهُ شَرٌّ لِلإِنْسَانِ الَّذِي يَأْكُلُ بِعَثْرَةٍ.
\par 21 حَسَنٌ أَنْ لاَ تَأْكُلَ لَحْماً وَلاَ تَشْرَبَ خَمْراً وَلاَ شَيْئاً يَصْطَدِمُ بِهِ أَخُوكَ أَوْ يَعْثُرُ أَوْ يَضْعُفُ.
\par 22 أَلَكَ إِيمَانٌ؟ فَلْيَكُنْ لَكَ بِنَفْسِكَ أَمَامَ اللهِ! طُوبَى لِمَنْ لاَ يَدِينُ نَفْسَهُ فِي مَا يَسْتَحْسِنُهُ.
\par 23 وَأَمَّا الَّذِي يَرْتَابُ فَإِنْ أَكَلَ يُدَانُ لأَنَّ ذَلِكَ لَيْسَ مِنَ الإِيمَانِ وَكُلُّ مَا لَيْسَ مِنَ الإِيمَانِ فَهُوَ خَطِيَّةٌ.

\chapter{15}

\par 1 فَيَجِبُ عَلَيْنَا نَحْنُ الأَقْوِيَاءَ أَنْ نَحْتَمِلَ أَضْعَافَ الضُّعَفَاءِ وَلاَ نُرْضِيَ أَنْفُسَنَا.
\par 2 فَلْيُرْضِ كُلُّ وَاحِدٍ مِنَّا قَرِيبَهُ لِلْخَيْرِ لأَجْلِ الْبُنْيَانِ.
\par 3 لأَنَّ الْمَسِيحَ أَيْضاً لَمْ يُرْضِ نَفْسَهُ بَلْ كَمَا هُوَ مَكْتُوبٌ: «تَعْيِيرَاتُ مُعَيِّرِيكَ وَقَعَتْ عَلَيَّ».
\par 4 لأَنَّ كُلَّ مَا سَبَقَ فَكُتِبَ كُتِبَ لأَجْلِ تَعْلِيمِنَا حَتَّى بِالصَّبْرِ وَالتَّعْزِيَةِ بِمَا فِي الْكُتُبِ يَكُونُ لَنَا رَجَاءٌ.
\par 5 وَلْيُعْطِكُمْ إِلَهُ الصَّبْرِ وَالتَّعْزِيَةِ أَنْ تَهْتَمُّوا اهْتِمَاماً وَاحِداً فِيمَا بَيْنَكُمْ بِحَسَبِ الْمَسِيحِ يَسُوعَ
\par 6 لِكَيْ تُمَجِّدُوا اللهَ أَبَا رَبِّنَا يَسُوعَ الْمَسِيحِ بِنَفْسٍ وَاحِدَةٍ وَفَمٍ وَاحِدٍ.
\par 7 لِذَلِكَ اقْبَلُوا بَعْضُكُمْ بَعْضاً كَمَا أَنَّ الْمَسِيحَ أَيْضاً قَبِلَنَا لِمَجْدِ اللهِ.
\par 8 وَأَقُولُ: إِنَّ يَسُوعَ الْمَسِيحَ قَدْ صَارَ خَادِمَ الْخِتَانِ مِنْ أَجْلِ صِدْقِ اللهِ حَتَّى يُثَبِّتَ مَوَاعِيدَ الآبَاءِ.
\par 9 وَأَمَّا الأُمَمُ فَمَجَّدُوا اللهَ مِنْ أَجْلِ الرَّحْمَةِ كَمَا هُوَ مَكْتُوبٌ: «مِنْ أَجْلِ ذَلِكَ سَأَحْمَدُكَ فِي الأُمَمِ وَأُرَتِّلُ لِاسْمِكَ»
\par 10 وَيَقُولُ أَيْضاً: «تَهَلَّلُوا أَيُّهَا الأُمَمُ مَعَ شَعْبِهِ»
\par 11 وَأَيْضاً: «سَبِّحُوا الرَّبَّ يَا جَمِيعَ الأُمَمِ وَامْدَحُوهُ يَا جَمِيعَ الشُّعُوبِ»
\par 12 وَأَيْضاً يَقُولُ إِشَعْيَاءُ: «سَيَكُونُ أَصْلُ يَسَّى وَالْقَائِمُ لِيَسُودَ عَلَى الأُمَمِ. عَلَيْهِ سَيَكُونُ رَجَاءُ الأُمَمِ».
\par 13 وَلْيَمْلَأْكُمْ إِلَهُ الرَّجَاءِ كُلَّ سُرُورٍ وَسَلاَمٍ فِي الإِيمَانِ لِتَزْدَادُوا فِي الرَّجَاءِ بِقُوَّةِ الرُّوحِ الْقُدُسِ.
\par 14 وَأَنَا نَفْسِي أَيْضاً مُتَيَقِّنٌ مِنْ جِهَتِكُمْ يَا إِخْوَتِي أَنَّكُمْ أَنْتُمْ مَشْحُونُونَ صَلاَحاً وَمَمْلُوؤُونَ كُلَّ عِلْمٍ قَادِرُونَ أَنْ يُنْذِرَ بَعْضُكُمْ بَعْضاً.
\par 15 وَلَكِنْ بِأَكْثَرِ جَسَارَةٍ كَتَبْتُ إِلَيْكُمْ جُزْئِيّاً أَيُّهَا الإِخْوَةُ كَمُذَكِّرٍ لَكُمْ بِسَبَبِ النِّعْمَةِ الَّتِي وُهِبَتْ لِي مِنَ اللهِ
\par 16 حَتَّى أَكُونَ خَادِماً لِيَسُوعَ الْمَسِيحِ لأَجْلِ الأُمَمِ مُبَاشِراً لإِنْجِيلِ اللهِ كَكَاهِنٍ لِيَكُونَ قُرْبَانُ الأُمَمِ مَقْبُولاً مُقَدَّساً بِالرُّوحِ الْقُدُسِ.
\par 17 فَلِي افْتِخَارٌ فِي الْمَسِيحِ يَسُوعَ مِنْ جِهَةِ مَا لِلَّهِ.
\par 18 لأَنِّي لاَ أَجْسُرُ أَنْ أَتَكَلَّمَ عَنْ شَيْءٍ مِمَّا لَمْ يَفْعَلْهُ الْمَسِيحُ بِوَاسِطَتِي لأَجْلِ إِطَاعَةِ الأُمَمِ بِالْقَوْلِ وَالْفِعْلِ
\par 19 بِقُوَّةِ آيَاتٍ وَعَجَائِبَ بِقُوَّةِ رُوحِ اللهِ. حَتَّى إِنِّي مِنْ أُورُشَلِيمَ وَمَا حَوْلَهَا إِلَى إِللِّيرِيكُونَ قَدْ أَكْمَلْتُ التَّبْشِيرَ بِإِنْجِيلِ الْمَسِيحِ.
\par 20 وَلَكِنْ كُنْتُ مُحْتَرِصاً أَنْ أُبَشِّرَ هَكَذَا: لَيْسَ حَيْثُ سُمِّيَ الْمَسِيحُ لِئَلاَّ أَبْنِيَ عَلَى أَسَاسٍ لِآخَرَ.
\par 21 بَلْ كَمَا هُوَ مَكْتُوبٌ: «الَّذِينَ لَمْ يُخْبَرُوا بِهِ سَيُبْصِرُونَ وَالَّذِينَ لَمْ يَسْمَعُوا سَيَفْهَمُونَ».
\par 22 لِذَلِكَ كُنْتُ أُعَاقُ الْمِرَارَ الْكَثِيرَةَ عَنِ الْمَجِيءِ إِلَيْكُمْ.
\par 23 وَأَمَّا الآنَ فَإِذْ لَيْسَ لِي مَكَانٌ بَعْدُ فِي هَذِهِ الأَقَالِيمِ وَلِي اشْتِيَاقٌ إِلَى الْمَجِيءِ إِلَيْكُمْ مُنْذُ سِنِينَ كَثِيرَةٍ
\par 24 فَعِنْدَمَا أَذْهَبُ إِلَى اسْبَانِيَا آتِي إِلَيْكُمْ. لأَنِّي أَرْجُو أَنْ أَرَاكُمْ فِي مُرُورِي وَتُشَيِّعُونِي إِلَى هُنَاكَ إِنْ تَمَلَّأْتُ أَوَّلاً مِنْكُمْ جُزْئِيّاً.
\par 25 وَلَكِنِ الآنَ أَنَا ذَاهِبٌ إِلَى أُورُشَلِيمَ لأَخْدِمَ الْقِدِّيسِينَ
\par 26 لأَنَّ أَهْلَ مَكِدُونِيَّةَ وَأَخَائِيَةَ اسْتَحْسَنُوا أَنْ يَصْنَعُوا تَوْزِيعاً لِفُقَرَاءِ الْقِدِّيسِينَ الَّذِينَ فِي أُورُشَلِيمَ.
\par 27 اسْتَحْسَنُوا ذَلِكَ وَإِنَّهُمْ لَهُمْ مَدْيُونُونَ! لأَنَّهُ إِنْ كَانَ الأُمَمُ قَدِ اشْتَرَكُوا فِي رُوحِيَّاتِهِمْ يَجِبُ عَلَيْهِمْ أَنْ يَخْدِمُوهُمْ فِي الْجَسَدِيَّاتِ أَيْضاً.
\par 28 فَمَتَى أَكْمَلْتُ ذَلِكَ وَخَتَمْتُ لَهُمْ هَذَا الثَّمَرَ فَسَأَمْضِي مَارّاً بِكُمْ إِلَى اسْبَانِيَا.
\par 29 وَأَنَا أَعْلَمُ أَنِّي إِذَا جِئْتُ إِلَيْكُمْ سَأَجِيءُ فِي مِلْءِ بَرَكَةِ إِنْجِيلِ الْمَسِيحِ.
\par 30 فَأَطْلُبُ إِلَيْكُمْ أَيُّهَا الإِخْوَةُ بِرَبِّنَا يَسُوعَ الْمَسِيحِ وَبِمَحَبَّةِ الرُّوحِ أَنْ تُجَاهِدُوا مَعِي فِي الصَّلَوَاتِ مِنْ أَجْلِي إِلَى اللهِ
\par 31 لِكَيْ أُنْقَذَ مِنَ الَّذِينَ هُمْ غَيْرُ مُؤْمِنِينَ فِي الْيَهُودِيَّةِ وَلِكَيْ تَكُونَ خِدْمَتِي لأَجْلِ أُورُشَلِيمَ مَقْبُولَةً عِنْدَ الْقِدِّيسِينَ
\par 32 حَتَّى أَجِيءَ إِلَيْكُمْ بِفَرَحٍ بِإِرَادَةِ اللهِ وَأَسْتَرِيحَ مَعَكُمْ.
\par 33 إِلَهُ السَّلاَمِ مَعَكُمْ أَجْمَعِينَ. آمِينَ.

\chapter{16}

\par 1 أُوصِي إِلَيْكُمْ بِأُخْتِنَا فِيبِي الَّتِي هِيَ خَادِمَةُ الْكَنِيسَةِ الَّتِي فِي كَنْخَرِيَا
\par 2 كَيْ تَقْبَلُوهَا فِي الرَّبِّ كَمَا يَحِقُّ لِلْقِدِّيسِينَ وَتَقُومُوا لَهَا فِي أَيِّ شَيْءٍ احْتَاجَتْهُ مِنْكُمْ لأَنَّهَا صَارَتْ مُسَاعِدَةً لِكَثِيرِينَ وَلِي أَنَا أَيْضاً.
\par 3 سَلِّمُوا عَلَى بِرِيسْكِلاَّ وَأَكِيلاَ الْعَامِلَيْنِ مَعِي فِي الْمَسِيحِ يَسُوعَ
\par 4 اللَّذَيْنِ وَضَعَا عُنُقَيْهِمَا مِنْ أَجْلِ حَيَاتِي اللَّذَيْنِ لَسْتُ أَنَا وَحْدِي أَشْكُرُهُمَا بَلْ أَيْضاً جَمِيعُ كَنَائِسِ الأُمَمِ
\par 5 وَعَلَى الْكَنِيسَةِ الَّتِي فِي بَيْتِهِمَا. سَلِّمُوا عَلَى أَبَيْنِتُوسَ حَبِيبِي الَّذِي هُوَ بَاكُورَةُ أَخَائِيَةَ لِلْمَسِيحِ.
\par 6 سَلِّمُوا عَلَى مَرْيَمَ الَّتِي تَعِبَتْ لأَجْلِنَا كَثِيراً.
\par 7 سَلِّمُوا عَلَى أَنْدَرُونِكُوسَ وَيُونِيَاسَ نَسِيبَيَّ الْمَأْسُورَيْنِ مَعِي اللَّذَيْنِ هُمَا مَشْهُورَانِ بَيْنَ الرُّسُلِ وَقَدْ كَانَا فِي الْمَسِيحِ قَبْلِي.
\par 8 سَلِّمُوا عَلَى أَمْبِلِيَاسَ حَبِيبِي فِي الرَّبِّ.
\par 9 سَلِّمُوا عَلَى أُورْبَانُوسَ الْعَامِلِ مَعَنَا فِي الْمَسِيحِ وَعَلَى إِسْتَاخِيسَ حَبِيبِي.
\par 10 سَلِّمُوا عَلَى أَبَلِّسَ الْمُزَكَّى فِي الْمَسِيحِ. سَلِّمُوا عَلَى الَّذِينَ هُمْ مِنْ أَهْلِ أَرِسْتُوبُولُوسَ.
\par 11 سَلِّمُوا عَلَى هِيرُودِيُونَ نَسِيبِي. سَلِّمُوا عَلَى الَّذِينَ هُمْ مِنْ أَهْلِ نَرْكِسُّوسَ الْكَائِنِينَ فِي الرَّبِّ.
\par 12 سَلِّمُوا عَلَى تَرِيفَيْنَا وَتَرِيفُوسَا التَّاعِبَتَيْنِ فِي الرَّبِّ. سَلِّمُوا عَلَى بَرْسِيسَ الْمَحْبُوبَةِ الَّتِي تَعِبَتْ كَثِيراً فِي الرَّبِّ.
\par 13 سَلِّمُوا عَلَى رُوفُسَ الْمُخْتَارِ فِي الرَّبِّ وَعَلَى أُمِّهِ أُمِّي.
\par 14 سَلِّمُوا عَلَى أَسِينْكِرِيتُسَ وَفِلِيغُونَ وَهَرْمَاسَ وَبَتْرُوبَاسَ وَهَرْمِيسَ وَعَلَى الإِخْوَةِ الَّذِينَ مَعَهُمْ.
\par 15 سَلِّمُوا عَلَى فِيلُولُوغُسَ وَجُولِيَا وَنِيرِيُوسَ وَأُخْتِهِ وَأُولُمْبَاسَ وَعَلَى جَمِيعِ الْقِدِّيسِينَ الَّذِينَ مَعَهُمْ.
\par 16 سَلِّمُوا بَعْضُكُمْ عَلَى بَعْضٍ بِقُبْلَةٍ مُقَدَّسَةٍ. كَنَائِسُ الْمَسِيحِ تُسَلِّمُ عَلَيْكُمْ.
\par 17 وَأَطْلُبُ إِلَيْكُمْ أَيُّهَا الإِخْوَةُ أَنْ تُلاَحِظُوا الَّذِينَ يَصْنَعُونَ الشِّقَاقَاتِ وَالْعَثَرَاتِ خِلاَفاً لِلتَّعْلِيمِ الَّذِي تَعَلَّمْتُمُوهُ وَأَعْرِضُوا عَنْهُمْ.
\par 18 لأَنَّ مِثْلَ هَؤُلاَءِ لاَ يَخْدِمُونَ رَبَّنَا يَسُوعَ الْمَسِيحَ بَلْ بُطُونَهُمْ وَبِالْكَلاَمِ الطَّيِّبِ وَالأَقْوَالِ الْحَسَنَةِ يَخْدَعُونَ قُلُوبَ السُّلَمَاءِ.
\par 19 لأَنَّ طَاعَتَكُمْ ذَاعَتْ إِلَى الْجَمِيعِ فَأَفْرَحُ أَنَا بِكُمْ وَأُرِيدُ أَنْ تَكُونُوا حُكَمَاءَ لِلْخَيْرِ وَبُسَطَاءَ لِلشَّرِّ.
\par 20 وَإِلَهُ السَّلاَمِ سَيَسْحَقُ الشَّيْطَانَ تَحْتَ أَرْجُلِكُمْ سَرِيعاً. نِعْمَةُ رَبِّنَا يَسُوعَ الْمَسِيحِ مَعَكُمْ. آمِينَ.
\par 21 يُسَلِّمُ عَلَيْكُمْ تِيمُوثَاوُسُ الْعَامِلُ مَعِي وَلُوكِيُوسُ وَيَاسُونُ وَسُوسِيبَاتْرُسُ أَنْسِبَائِي.
\par 22 أَنَا تَرْتِيُوسُ كَاتِبُ هَذِهِ الرِّسَالَةِ أُسَلِّمُ عَلَيْكُمْ فِي الرَّبِّ.
\par 23 يُسَلِّمُ عَلَيْكُمْ غَايُسُ مُضَيِّفِي وَمُضَيِّفُ الْكَنِيسَةِ كُلِّهَا. يُسَلِّمُ عَلَيْكُمْ أَرَاسْتُسُ خَازِنُ الْمَدِينَةِ وَكَوَارْتُسُ الأَخُ.
\par 24 نِعْمَةُ رَبِّنَا يَسُوعَ الْمَسِيحِ مَعَ جَمِيعِكُمْ. آمِينَ.
\par 25 وَلِلْقَادِرِ أَنْ يُثَبِّتَكُمْ حَسَبَ إِنْجِيلِي وَالْكِرَازَةِ بِيَسُوعَ الْمَسِيحِ حَسَبَ إِعْلاَنِ السِّرِّ الَّذِي كَانَ مَكْتُوماً فِي الأَزْمِنَةِ الأَزَلِيَّةِ
\par 26 وَلَكِنْ ظَهَرَ الآنَ وَأُعْلِمَ بِهِ جَمِيعُ الأُمَمِ بِالْكُتُبِ النَّبَوِيَّةِ حَسَبَ أَمْرِ الإِلَهِ الأَزَلِيِّ لإِطَاعَةِ الإِيمَانِ
\par 27 ِللهِ الْحَكِيمِ وَحْدَهُ بِيَسُوعَ الْمَسِيحِ لَهُ الْمَجْدُ إِلَى الأَبَدِ. آمِينَ.

\end{document}