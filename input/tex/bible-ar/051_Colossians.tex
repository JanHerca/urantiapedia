\begin{document}

\title{كولوسي}


\chapter{1}

\par 1 بُولُسُ، رَسُولُ يَسُوعَ الْمَسِيحِ بِمَشِيئَةِ اللهِ، وَتِيمُوثَاوُسُ الأَخُ،
\par 2 إِلَى الْقِدِّيسِينَ فِي كُولُوسِّي، وَالإِخْوَةِ الْمُؤْمِنِينَ فِي الْمَسِيحِ. نِعْمَةٌ لَكُمْ وَسَلاَمٌ مِنَ اللهِ أبِينَا وَالرَّبِّ يَسُوعَ الْمَسِيحِ.
\par 3 نَشْكُرُ اللهَ وَأَبَا رَبِّنَا يَسُوعَ الْمَسِيحِ كُلَّ حِينٍ، مُصَلِّينَ لأَجْلِكُمْ،
\par 4 إِذْ سَمِعْنَا إيمَانَكُمْ بِالْمَسِيحِ يَسُوعَ، وَمَحَبَّتَكُمْ لِجَمِيعِ الْقِدِّيسِينَ،
\par 5 مِنْ أجْلِ الرَّجَاءِ الْمَوْضُوعِ لَكُمْ فِي السَّمَاوَاتِ الَّذِي سَمِعْتُمْ بِهِ قَبْلاً فِي كَلِمَةِ حَقِّ الإِنْجِيلِ،
\par 6 الَّذِي قَدْ حَضَرَ إلَيْكُمْ كَمَا فِي كُلِّ الْعَالَمِ أيْضاً، وَهُوَ مُثْمِرٌ كَمَا فِيكُمْ أيْضاً مُنْذُ يَوْمَ سَمِعْتُمْ وَعَرَفْتُمْ نِعْمَةَ اللهِ بِالْحَقِيقَةِ.
\par 7 كَمَا تَعَلَّمْتُمْ ايْضاً مِنْ ابَفْرَاسَ الْعَبْدِ الْحَبِيبِ مَعَنَا، الَّذِي هُوَ خَادِمٌ امِينٌ لِلْمَسِيحِ لأَجْلِكُمُ،
\par 8 الَّذِي اخْبَرَنَا ايْضاً بِمَحَبَّتِكُمْ فِي الرُّوحِ.
\par 9 مِنْ أجْلِ ذَلِكَ نَحْنُ أيْضاً، مُنْذُ يَوْمَ سَمِعْنَا، لَمْ نَزَلْ مُصَلِّينَ وَطَالِبِينَ لأَجْلِكُمْ انْ تَمْتَلِئُوا مِنْ مَعْرِفَةِ مَشِيئَتِهِ، فِي كُلِّ حِكْمَةٍ وَفَهْمٍ رُوحِيٍّ
\par 10 لِتَسْلُكُوا كَمَا يَحِقُّ لِلرَّبِّ، فِي كُلِّ رِضىً، مُثْمِرِينَ فِي كُلِّ عَمَلٍ صَالِحٍ، وَنَامِينَ فِي مَعْرِفَةِ اللهِ،
\par 11 مُتَقَوِّينَ بِكُلِّ قُوَّةٍ بِحَسَبِ قُدْرَةِ مَجْدِهِ، لِكُلِّ صَبْرٍ وَطُولِ انَاةٍ بِفَرَحٍ،
\par 12 شَاكِرِينَ الآبَ الَّذِي اهَّلَنَا لِشَرِكَةِ مِيرَاثِ الْقِدِّيسِينَ فِي النُّورِ،
\par 13 الَّذِي انْقَذَنَا مِنْ سُلْطَانِ الظُّلْمَةِ وَنَقَلَنَا الَى مَلَكُوتِ ابْنِ مَحَبَّتِهِ،
\par 14 الَّذِي لَنَا فِيهِ الْفِدَاءُ، بِدَمِهِ غُفْرَانُ الْخَطَايَا،
\par 15 اَلَّذِي هُوَ صُورَةُ اللهِ غَيْرِ الْمَنْظُورِ، بِكْرُ كُلِّ خَلِيقَةٍ.
\par 16 فَإِنَّهُ فِيهِ خُلِقَ الْكُلُّ: مَا فِي السَّمَاوَاتِ وَمَا عَلَى الأَرْضِ، مَا يُرَى وَمَا لاَ يُرَى، سَوَاءٌ كَانَ عُرُوشاً امْ سِيَادَاتٍ امْ رِيَاسَاتٍ امْ سَلاَطِينَ. الْكُلُّ بِهِ وَلَهُ قَدْ خُلِقَ.
\par 17 اَلَّذِي هُوَ قَبْلَ كُلِّ شَيْءٍ، وَفِيهِ يَقُومُ الْكُلُّ
\par 18 وَهُوَ رَأْسُ الْجَسَدِ: الْكَنِيسَةِ. الَّذِي هُوَ الْبَدَاءَةُ، بِكْرٌ مِنَ الأَمْوَاتِ، لِكَيْ يَكُونَ هُوَ مُتَقَدِّماً فِي كُلِّ شَيْءٍ.
\par 19 لأَنَّهُ فِيهِ سُرَّ انْ يَحِلَّ كُلُّ الْمِلْءِ،
\par 20 وَأَنْ يُصَالِحَ بِهِ الْكُلَّ لِنَفْسِهِ، عَامِلاً الصُّلْحَ بِدَمِ صَلِيبِهِ، بِوَاسِطَتِهِ، سَوَاءٌ كَانَ مَا عَلَى الأَرْضِ امْ مَا فِي السَّمَاوَاتِ.
\par 21 وَأَنْتُمُ الَّذِينَ كُنْتُمْ قَبْلاً اجْنَبِيِّينَ وَأَعْدَاءً فِي الْفِكْرِ، فِي الأَعْمَالِ الشِّرِّيرَةِ، قَدْ صَالَحَكُمُ الآنَ
\par 22 فِي جِسْمِ بَشَرِيَّتِهِ بِالْمَوْتِ، لِيُحْضِرَكُمْ قِدِّيسِينَ وَبِلاَ لَوْمٍ وَلاَ شَكْوَى امَامَهُ،
\par 23 إِنْ ثَبَتُّمْ عَلَى الإِيمَانِ، مُتَأَسِّسِينَ وَرَاسِخِينَ وَغَيْرَ مُنْتَقِلِينَ عَنْ رَجَاءِ الإِنْجِيلِ، الَّذِي سَمِعْتُمُوهُ، الْمَكْرُوزِ بِهِ فِي كُلِّ الْخَلِيقَةِ الَّتِي تَحْتَ السَّمَاءِ، الَّذِي صِرْتُ انَا بُولُسَ خَادِماً لَهُ،
\par 24 الَّذِي الآنَ افْرَحُ فِي الاَمِي لأَجْلِكُمْ، وَأُكَمِّلُ نَقَائِصَ شَدَائِدِ الْمَسِيحِ فِي جِسْمِي لأَجْلِ جَسَدِهِ: الَّذِي هُوَ الْكَنِيسَةُ،
\par 25 الَّتِي صِرْتُ انَا خَادِماً لَهَا، حَسَبَ تَدْبِيرِ اللهِ الْمُعْطَى لِي لأَجْلِكُمْ، لِتَتْمِيمِ كَلِمَةِ اللهِ.
\par 26 السِّرِّ الْمَكْتُومِ مُنْذُ الدُّهُورِ وَمُنْذُ الأَجْيَالِ، لَكِنَّهُ الآنَ قَدْ اظْهِرَ لِقِدِّيسِيهِ،
\par 27 الَّذِينَ ارَادَ اللهُ انْ يُعَرِّفَهُمْ مَا هُوَ غِنَى مَجْدِ هَذَا السِّرِّ فِي الأُمَمِ، الَّذِي هُوَ الْمَسِيحُ فِيكُمْ رَجَاءُ الْمَجْدِ.
\par 28 الَّذِي نُنَادِي بِهِ مُنْذِرِينَ كُلَّ انْسَانٍ، وَمُعَلِّمِينَ كُلَّ انْسَانٍ، بِكُلِّ حِكْمَةٍ، لِكَيْ نُحْضِرَ كُلَّ انْسَانٍ كَامِلاً فِي الْمَسِيحِ يَسُوعَ.
\par 29 الأَمْرُ الَّذِي لأَجْلِهِ اتْعَبُ ايْضاً مُجَاهِداً، بِحَسَبِ عَمَلِهِ الَّذِي يَعْمَلُ فِيَّ بِقُوَّةٍ.

\chapter{2}

\par 1 فَإِنِّي ارِيدُ انْ تَعْلَمُوا ايُّ جِهَادٍ لِي لأَجْلِكُمْ، وَلأَجْلِ الَّذِينَ فِي لاَوُدِكِيَّةَ، وَجَمِيعِ الَّذِينَ لَمْ يَرَوْا وَجْهِي فِي الْجَسَدِ،
\par 2 لِكَيْ تَتَعَزَّى قُلُوبُهُمْ مُقْتَرِنَةً فِي الْمَحَبَّةِ لِكُلِّ غِنَى يَقِينِ الْفَهْمِ، لِمَعْرِفَةِ سِرِّ اللهِ الآبِ وَالْمَسِيحِ،
\par 3 الْمُذَّخَرِ فِيهِ جَمِيعُ كُنُوزِ الْحِكْمَةِ وَالْعِلْمِ.
\par 4 وَإِنَّمَا اقُولُ هَذَا لِئَلاَّ يَخْدَعَكُمْ احَدٌ بِكَلاَمٍ مَلِقٍ،
\par 5 فَإِنِّي وَإِنْ كُنْتُ غَائِباً فِي الْجَسَدِ لَكِنِّي مَعَكُمْ فِي الرُّوحِ، فَرِحاً، وَنَاظِراً تَرْتِيبَكُمْ وَمَتَانَةَ ايمَانِكُمْ فِي الْمَسِيحِ.
\par 6 فَكَمَا قَبِلْتُمُ الْمَسِيحَ يَسُوعَ الرَّبَّ اسْلُكُوا فِيهِ،
\par 7 مُتَأَصِّلِينَ وَمَبْنِيِّينَ فِيهِ، وَمُوَطَّدِينَ فِي الإِيمَانِ، كَمَا عُلِّمْتُمْ، مُتَفَاضِلِينَ فِيهِ بِالشُّكْرِ.
\par 8 اُنْظُرُوا انْ لاَ يَكُونَ احَدٌ يَسْبِيكُمْ بِالْفَلْسَفَةِ وَبِغُرُورٍ بَاطِلٍ، حَسَبَ تَقْلِيدِ النَّاسِ، حَسَبَ ارْكَانِ الْعَالَمِ، وَلَيْسَ حَسَبَ الْمَسِيحِ.
\par 9 فَإِنَّهُ فِيهِ يَحِلُّ كُلُّ مِلْءِ اللاَّهُوتِ جَسَدِيّاً.
\par 10 وَأَنْتُمْ مَمْلُوؤُونَ فِيهِ، الَّذِي هُوَ رَأْسُ كُلِّ رِيَاسَةٍ وَسُلْطَانٍ.
\par 11 وَبِهِ ايْضاً خُتِنْتُمْ خِتَاناً غَيْرَ مَصْنُوعٍ بِيَدٍ، بِخَلْعِ جِسْمِ خَطَايَا الْبَشَرِيَّةِ، بِخِتَانِ الْمَسِيحِ.
\par 12 مَدْفُونِينَ مَعَهُ فِي الْمَعْمُودِيَّةِ، الَّتِي فِيهَا اقِمْتُمْ ايْضاً مَعَهُ بِإِيمَانِ عَمَلِ اللهِ، الَّذِي اقَامَهُ مِنَ الأَمْوَاتِ.
\par 13 وَإِذْ كُنْتُمْ امْوَاتاً فِي الْخَطَايَا وَغَلَفِ جَسَدِكُمْ، احْيَاكُمْ مَعَهُ، مُسَامِحاً لَكُمْ بِجَمِيعِ الْخَطَايَا،
\par 14 إِذْ مَحَا الصَّكَّ الَّذِي عَلَيْنَا فِي الْفَرَائِضِ، الَّذِي كَانَ ضِدّاً لَنَا، وَقَدْ رَفَعَهُ مِنَ الْوَسَطِ مُسَمِّراً ايَّاهُ بِالصَّلِيبِ،
\par 15 إِذْ جَرَّدَ الرِّيَاسَاتِ وَالسَّلاَطِينَ اشْهَرَهُمْ جِهَاراً، ظَافِراً بِهِمْ فِيهِ.
\par 16 فَلاَ يَحْكُمْ عَلَيْكُمْ احَدٌ فِي أكْلٍ اوْ شُرْبٍ، اوْ مِنْ جِهَةِ عِيدٍ اوْ هِلاَلٍ اوْ سَبْتٍ،
\par 17 الَّتِي هِيَ ظِلُّ الأُمُورِ الْعَتِيدَةِ، وَأَمَّا الْجَسَدُ فَلِلْمَسِيحِ.
\par 18 لاَ يُخَسِّرْكُمْ احَدٌ الْجِعَالَةَ، رَاغِباً فِي التَّوَاضُعِ وَعِبَادَةِ الْمَلاَئِكَةِ، مُتَدَاخِلاً فِي مَا لَمْ يَنْظُرْهُ، مُنْتَفِخاً بَاطِلاً مِنْ قِبَلِ ذِهْنِهِ الْجَسَدِيِّ،
\par 19 وَغَيْرَ مُتَمَسِّكٍ بِالرَّأْسِ الَّذِي مِنْهُ كُلُّ الْجَسَدِ بِمَفَاصِلَ وَرُبُطٍ، مُتَوَازِراً وَمُقْتَرِناً يَنْمُو نُمُوّاً مِنَ اللهِ.
\par 20 إِذاً انْ كُنْتُمْ قَدْ مُتُّمْ مَعَ الْمَسِيحِ عَنْ ارْكَانِ الْعَالَمِ، فَلِمَاذَا كَأَنَّكُمْ عَائِشُونَ فِي الْعَالَمِ، تُفْرَضُ عَلَيْكُمْ فَرَائِضُ:
\par 21 لاَ تَمَسَّ، وَلاَ تَذُقْ، وَلاَ تَجُسَّ؟
\par 22 الَّتِي هِيَ جَمِيعُهَا لِلْفَنَاءِ فِي الاِسْتِعْمَالِ، حَسَبَ وَصَايَا وَتَعَالِيمِ النَّاسِ،
\par 23 الَّتِي لَهَا حِكَايَةُ حِكْمَةٍ، بِعِبَادَةٍ نَافِلَةٍ، وَتَوَاضُعٍ، وَقَهْرِ الْجَسَدِ، لَيْسَ بِقِيمَةٍ مَا مِنْ جِهَةِ اشْبَاعِ الْبَشَرِيَّةِ.

\chapter{3}

\par 1 فَإِنْ كُنْتُمْ قَدْ قُمْتُمْ مَعَ الْمَسِيحِ فَاطْلُبُوا مَا فَوْقُ، حَيْثُ الْمَسِيحُ جَالِسٌ عَنْ يَمِينِ اللهِ.
\par 2 اهْتَمُّوا بِمَا فَوْقُ لاَ بِمَا عَلَى الأَرْضِ،
\par 3 لأَنَّكُمْ قَدْ مُتُّمْ وَحَيَاتُكُمْ مُسْتَتِرَةٌ مَعَ الْمَسِيحِ فِي اللهِ.
\par 4 مَتَى اظْهِرَ الْمَسِيحُ حَيَاتُنَا، فَحِينَئِذٍ تُظْهَرُونَ انْتُمْ ايْضاً مَعَهُ فِي الْمَجْدِ.
\par 5 فَأَمِيتُوا اعْضَاءَكُمُ الَّتِي عَلَى الأَرْضِ: الزِّنَا، النَّجَاسَةَ، الْهَوَى، الشَّهْوَةَ الرَّدِيَّةَ، الطَّمَعَ الَّذِي هُوَ عِبَادَةُ الأَوْثَانِ،
\par 6 الأُمُورَ الَّتِي مِنْ اجْلِهَا يَأْتِي غَضَبُ اللهِ عَلَى ابْنَاءِ الْمَعْصِيَةِ،
\par 7 الَّذِينَ بَيْنَهُمْ انْتُمْ ايْضاً سَلَكْتُمْ قَبْلاً، حِينَ كُنْتُمْ تَعِيشُونَ فِيهَا.
\par 8 وَأَمَّا الآنَ فَاطْرَحُوا عَنْكُمْ انْتُمْ ايْضاً الْكُلَّ: الْغَضَبَ، السَّخَطَ، الْخُبْثَ، التَّجْدِيفَ، الْكَلاَمَ الْقَبِيحَ مِنْ افْوَاهِكُمْ.
\par 9 لاَ تَكْذِبُوا بَعْضُكُمْ عَلَى بَعْضٍ، اذْ خَلَعْتُمُ الإِنْسَانَ الْعَتِيقَ مَعَ اعْمَالِهِ،
\par 10 وَلَبِسْتُمُ الْجَدِيدَ الَّذِي يَتَجَدَّدُ لِلْمَعْرِفَةِ حَسَبَ صُورَةِ خَالِقِهِ،
\par 11 حَيْثُ لَيْسَ يُونَانِيٌّ وَيَهُودِيٌّ، خِتَانٌ وَغُرْلَةٌ، بَرْبَرِيٌّ سِكِّيثِيٌّ، عَبْدٌ حُرٌّ، بَلِ الْمَسِيحُ الْكُلُّ وَفِي الْكُلِّ.
\par 12 فَالْبَسُوا كَمُخْتَارِي اللهِ الْقِدِّيسِينَ الْمَحْبُوبِينَ احْشَاءَ رَأْفَاتٍ، وَلُطْفاً، وَتَوَاضُعاً، وَوَدَاعَةً، وَطُولَ انَاةٍ،
\par 13 مُحْتَمِلِينَ بَعْضُكُمْ بَعْضاً، وَمُسَامِحِينَ بَعْضُكُمْ بَعْضاً انْ كَانَ لأَحَدٍ عَلَى احَدٍ شَكْوَى. كَمَا غَفَرَ لَكُمُ الْمَسِيحُ هَكَذَا انْتُمْ ايْضاً.
\par 14 وَعَلَى جَمِيعِ هَذِهِ الْبَسُوا الْمَحَبَّةَ الَّتِي هِيَ رِبَاطُ الْكَمَالِ.
\par 15 وَلْيَمْلِكْ فِي قُلُوبِكُمْ سَلاَمُ اللهِ الَّذِي الَيْهِ دُعِيتُمْ فِي جَسَدٍ وَاحِدٍ، وَكُونُوا شَاكِرِينَ.
\par 16 لِتَسْكُنْ فِيكُمْ كَلِمَةُ الْمَسِيحِ بِغِنىً، وَأَنْتُمْ بِكُلِّ حِكْمَةٍ مُعَلِّمُونَ وَمُنْذِرُونَ بَعْضُكُمْ بَعْضاً، بِمَزَامِيرَ وَتَسَابِيحَ وَأَغَانِيَّ رُوحِيَّةٍ، بِنِعْمَةٍ، مُتَرَنِّمِينَ فِي قُلُوبِكُمْ لِلرَّبِّ.
\par 17 وَكُلُّ مَا عَمِلْتُمْ بِقَوْلٍ اوْ فِعْلٍ، فَاعْمَلُوا الْكُلَّ بِاسْمِ الرَّبِّ يَسُوعَ، شَاكِرِينَ اللهَ وَالآبَ بِهِ.
\par 18 أَيَّتُهَا النِّسَاءُ، اخْضَعْنَ لِرِجَالِكُنَّ كَمَا يَلِيقُ فِي الرَّبِّ.
\par 19 أَيُّهَا الرِّجَالُ، احِبُّوا نِسَاءَكُمْ، وَلاَ تَكُونُوا قُسَاةً عَلَيْهِنَّ
\par 20 أَيُّهَا الأَوْلاَدُ، اطِيعُوا وَالِدِيكُمْ فِي كُلِّ شَيْءٍ لأَنَّ هَذَا مَرْضِيٌّ فِي الرَّبِّ.
\par 21 أَيُّهَا الآبَاءُ، لاَ تُغِيظُوا اوْلاَدَكُمْ لِئَلاَّ يَفْشَلُوا.
\par 22 أَيُّهَا الْعَبِيدُ، اطِيعُوا فِي كُلِّ شَيْءٍ سَادَتَكُمْ حَسَبَ الْجَسَدِ، لاَ بِخِدْمَةِ الْعَيْنِ كَمَنْ يُرْضِي النَّاسَ، بَلْ بِبَسَاطَةِ الْقَلْبِ، خَائِفِينَ الرَّبَّ.
\par 23 وَكُلُّ مَا فَعَلْتُمْ فَاعْمَلُوا مِنَ الْقَلْبِ، كَمَا لِلرَّبِّ لَيْسَ لِلنَّاسِ،
\par 24 عَالِمِينَ انَّكُمْ مِنَ الرَّبِّ سَتَأْخُذُونَ جَزَاءَ الْمِيرَاثِ، لأَنَّكُمْ تَخْدِمُونَ الرَّبَّ الْمَسِيحَ.
\par 25 وَأَمَّا الظَّالِمُ فَسَينَالُ مَا ظَلَمَ بِهِ، ولَيْسَ مُحَابَاةٌ.

\chapter{4}

\par 1 أَيُّهَا السَّادَةُ، قَدِّمُوا لِلْعَبِيدِ الْعَدْلَ وَالْمُسَاوَاةَ، عَالِمِينَ انَّ لَكُمْ انْتُمْ ايْضاً سَيِّداً فِي السَّمَاوَاتِ.
\par 2 وَاظِبُوا عَلَى الصَّلاَةِ سَاهِرِينَ فِيهَا بِالشُّكْرِ،
\par 3 مُصَلِّينَ فِي ذَلِكَ لأَجْلِنَا نَحْنُ ايْضاً، لِيَفْتَحَ الرَّبُّ لَنَا بَاباً لِلْكَلاَمِ، لِنَتَكَلَّمَ بِسِرِّ الْمَسِيحِ، الَّذِي مِنْ اجْلِهِ انَا مُوثَقٌ ايْضاً،
\par 4 كَيْ اظْهِرَهُ كَمَا يَجِبُ انْ اتَكَلَّمَ.
\par 5 اُسْلُكُوا بِحِكْمَةٍ مِنْ جِهَةِ الَّذِينَ هُمْ مِنْ خَارِجٍ، مُفْتَدِينَ الْوَقْتَ.
\par 6 لِيَكُنْ كَلاَمُكُمْ كُلَّ حِينٍ بِنِعْمَةٍ، مُصْلَحاً بِمِلْحٍ، لِتَعْلَمُوا كَيْفَ يَجِبُ انْ تُجَاوِبُوا كُلَّ وَاحِدٍ.
\par 7 جَمِيعُ احْوَالِي سَيُعَرِّفُكُمْ بِهَا تِيخِيكُسُ الأَخُ الْحَبِيبُ، وَالْخَادِمُ الأَمِينُ، وَالْعَبْدُ مَعَنَا فِي الرَّبِّ،
\par 8 الَّذِي ارْسَلْتُهُ الَيْكُمْ لِهَذَا عَيْنِهِ، لِيَعْرِفَ احْوَالَكُمْ وَيُعَزِّيَ قُلُوبَكُمْ،
\par 9 مَعَْ انِسِيمُسَ الأَخِ الأَمِينِ الْحَبِيبِ الَّذِي هُوَ مِنْكُمْ. هُمَا سَيُعَرِّفَانِكُمْ بِكُلِّ مَا هَهُنَا.
\par 10 يُسَلِّمُ عَلَيْكُمْ ارِسْتَرْخُسُ الْمَأْسُورُ مَعِي، وَمَرْقُسُ ابْنُ اخْتِ بَرْنَابَا، الَّذِي اخَذْتُمْ لأَجْلِهِ وَصَايَا. انْ اتَى الَيْكُمْ فَاقْبَلُوهُ.
\par 11 وَيَسُوعُ الْمَدْعُوُّ يُسْطُسَ، الَّذِينَ هُمْ مِنَ الْخِتَانِ. هَؤُلاَءِ هُمْ وَحْدَهُمُ الْعَامِلُونَ مَعِي لِمَلَكُوتِ اللهِ، الَّذِينَ صَارُوا لِي تَسْلِيَةً.
\par 12 يُسَلِّمُ عَلَيْكُمْ ابَفْرَاسُ، الَّذِي هُوَ مِنْكُمْ، عَبْدٌ لِلْمَسِيحِ، مُجَاهِدٌ كُلَّ حِينٍ لأَجْلِكُمْ بِالصَّلَوَاتِ، لِكَيْ تَثْبُتُوا كَامِلِينَ وَمُمْتَلِئِينَ فِي كُلِّ مَشِيئَةِ اللهِ.
\par 13 فَإِنِّي اشْهَدُ فِيهِ انَّ لَهُ غَيْرَةً كَثِيرَةً لأَجْلِكُمْ، وَلأَجْلِ الَّذِينَ فِي لاَوُدِكِيَّةَ، وَالَّذِينَ فِي هِيَرَابُولِيسَ.
\par 14 يُسَلِّمُ عَلَيْكُمْ لُوقَا الطَّبِيبُ الْحَبِيبُ، وَدِيمَاسُ.
\par 15 سَلِّمُوا عَلَى الإِخْوَةِ الَّذِينَ فِي لاَوُدِكِيَّةَ، وَعَلَى نِمْفَاسَ وَعَلَى الْكَنِيسَةِ الَّتِي فِي بَيْتِهِ.
\par 16 وَمَتَى قُرِئَتْ عِنْدَكُمْ هَذِهِ الرِّسَالَةُ فَاجْعَلُوهَا تُقْرَأُ ايْضاً فِي كَنِيسَةِ اللاَّوُدِكِيِّينَ، وَالَّتِي مِنْ لاَوُدِكِيَّةَ تَقْرَأُونَهَا انْتُمْ ايْضاً.
\par 17 وَقُولُوا لأَرْخِبُّسَ: «انْظُرْ الَى الْخِدْمَةِ الَّتِي قَبِلْتَهَا فِي الرَّبِّ لِكَيْ تُتَمِّمَهَا».
\par 18 اَلسَّلاَمُ بِيَدِي انَا بُولُسَ. اذْكُرُوا وُثُقِي. النِّعْمَةُ مَعَكُمْ. امِينَ.


\end{document}