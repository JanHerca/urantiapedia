\begin{document}


\title{وصية يعقوب}

\chapter{1}

\par 1 بسم الآب والابن والروح القدس الإله الواحد.

\par 2 نبدأ، بعون الله العلي وعنايته، بكتابة قصة حياة أبينا البطريرك يعقوب، ابن البطريرك إسحاق، في اليوم الثامن والعشرين من شهر مصر

\par 3 لتحفظنا بركة صلاته وتحمينا من إغراءات العدو العنيد. آمين، آمين، آمين!

\par 4 قال: «تعالوا، اسمعوا يا أحبائي وإخوتي الذين يحبون الرب، إلى ما قد أُخذ».

\par 5 ولما اقترب زمان أبينا يعقوب، أبو الآباء، ابن إسحاق، ابن إبراهيم، ودنى أن يختلس من جسده، كان هذا الأمين متقدمًا في السن والمكانة

\par 6 فأرسل الرب إليه ميخائيل رئيس الملائكة، الذي قال له: «يا إسرائيل، يا حبيبي، يا من نسل شريف، اكتب وصيتك التي تكلمت بها وتعليمك لأهل بيتك، وأعطهم عهدًا. واهتم أيضًا بترتيب بيتك، لأنه قد اقترب وقت ذهابك إلى آبائك لتفرح معهم إلى الأبد.»

\par 7 فلما سمع أبونا يعقوب الأمين هذا من الملاك، أجاب وقال، كما كانت عادته كل يوم أن يكلم الملائكة هكذا،

\par 8 «لتكن مشيئة الرب.»

\par 9 وأعلن الله بركة على أبينا يعقوب. وكان ليعقوب مكان منعزل يدخله ليقدم صلواته أمام الرب ليلاً ونهارًا

\par 10 كانت الملائكة تزوره وتحرسه وتقويه في كل شيء

\par 11 باركه الله وكثر شعبه في أرض مصر عندما نزل إلى أرض مصر للقاء ابنه يوسف

\par 12 أصبحت عيناه كليلتين من البكاء، ولكن عندما نزل إلى مصر، رأى بوضوح عندما رأى ابنه

\par 13 فانحنى يعقوب إسرائيل على وجهه إلى الأرض، ثم وقع على عنق ابنه يوسف وقبله، وهو يبكي ويقول: "أستطيع أن أموت الآن يا ابني، لأني رأيت وجهك مرة أخرى في حياتي؛ يا ابني الحبيب."


\chapter{2}


\par 1 استمر يوسف في حكم كل مصر، بينما مكث يعقوب في أرض جاسان سبع عشرة سنة وشاخ جدًا، حتى اكتمل عمره

\par 2 وكان يحفظ جميع الوصايا ويخاف الرب.

\par 3 أصبحت عيناه باهتة، وكانت حياته على وشك الانتهاء لدرجة أنه لم يستطع رؤية شخص واحد بسبب طول عمره وخرفته.

\par 4 ثم رفع عينيه نحو نور إسحق، فخاف واضطرب.

\par 5 فقال له الملاك: لا تخف يا يعقوب، أنا الملاك الذي كان يسير معك ويحرسك منذ طفولتك.

\par 6 وأعلنت أنك ستحصل على بركة أبيك ورفقة أمك.

\par 7 أنا هو الذي معك يا إسرائيل في كل أعمالك وفي كل ما شاهدته.

\par 8 أنقذتك من لابان حين كان يهددك ويطاردك.

\par 9 في ذلك الوقت أعطيتك كل ممتلكاته، وباركتك أنت وزوجاتك وأولادك ومواشيك

\par 10 أنا الذي خلصتك من يد عيسو.

\par 11 أنا الذي رافقتك إلى أرض مصر يا إسرائيل، فأعطي لك شعب عظيم جداً.

\par 12 طوبى لأبيك إبراهيم، لأنه أصبح خليل الله - تعالى - بفضل كرمه ومحبته للغرباء

\par 13 طوبى لأبوك إسحاق الذي ولدك، لأنه كان ذبيحة كاملة مقبولة عند الله

\par 14 «طوبى لك أيضًا يا يعقوب، لأنك رأيت الله وجهًا لوجه.»

\par 15 لقد رأيت ملاك الله -تعالى- ورأيت السلم قائما على الأرض ورأسه في السماء.

\par 16 ثم رأيت الرب جالسًا على قمتها بقوة لا يستطيع أحد وصفها

\par 17 تكلمتَ وقلتَ: هذا بيتُ اللهِ وهذا بابُ السماءِ

\par 18 طوبى لك، لأنك اقتربت من الله وهو قوي بين البشر، فلا تضطرب الآن، يا مختار الله

\par 19 «مبارك أنت يا إسرائيل، ومبارك كل نسلك.»

\par 20 لأنكم جميعاً تُدعون آباءً إلى انقضاء الدهر والأزمنة، أنتم شعب وسلالة عبيد الله.

\par 21 طوبى للأمة التي ستسعى إلى طهارتكم وسترى أعمالكم الصالحة

\par 22 طوبى للرجل الذي سيتذكرك يوم عيدك النبيل

\par 23 طوبى لمن يقوم بأعمال الرحمة تكريمًا لأسمائك المتعددة، فيعطي شخصًا ما كوبًا من الماء ليشربه، أو يأتي بتقدمة إلى الحرم، أو يستقبل غرباء، أو يزور المرضى ويواسي أطفالهم، أو يكسو عريانًا تكريمًا لأسمائك المتعددة

\par 24 «لن يفتقر مثل هذا الشخص إلى أيٍّ من خيرات هذا العالم، ولا إلى الحياة الأبدية في العالم الآتي

\par 25 علاوة على ذلك، كل من ساهم في كتابة قصص حياتكم ومعاناتكم المختلفة على نفقته الخاصة، أو كتبها بيده، أو قرأها بوعي، أو سمعها بإيمان، أو تذكر أعمالكم - هؤلاء الأشخاص ستُغفر خطاياهم وتُغفر تجاوزاتهم، وسيدخلون ملكوت السماوات بسببكم وبسبب ذريتكم

\par 26 «والآن انهض يا يعقوب، لأنك ستُنقل من المشقة وآلام القلب إلى الراحة الأبدية، وستدخل إلى الراحة التي لن تزول، إلى الرحمة والنور الأبدي والفرح الروحي

\par 27 فالآن أبلغ أهل بيتك، والسلام عليكم، لأني ذاهب إلى الذي أرسلني

\chapter{3}

\par 1 فلما قال الملاك هذا الكلام لأبينا يعقوب، صعد منه إلى السماء كما ودّعه يعقوب

\par 2 وكان الذين حول يعقوب يسمعونه وهو يشكر الله ويمجده بالتسبيح.

\par 3 واجتمع حوله جميع أهل بيته، كبيرهم وصغيرهم، يبكون عليه، حزينين حزنًا شديدًا، قائلين: «إنك ذاهب وتتركنا أيتامًا».

\par 4 وكانوا يقولون له: «يا أبانا الحبيب، ماذا نفعل ونحن في أرض غريبة؟»

\par 5 فقال لهم يعقوب: «لا تخافوا. الله نفسه ظهر لي في أعالي بلاد ما بين النهرين وقال لي: أنا إله آبائك. لا تخف، لأني معك إلى الأبد ومع نسلك الذي يأتي من بعدك.»

\par 6 هذه الأرض التي أنت فيها سأعطيها لك ولنسلك من بعدك إلى الأبد

\par 7 ولا تخف من النزول إلى مصر.

\par 8 وأجعل لك شعباً عظيماً، وينمو نسلك ويكثر إلى الأبد.

\par 9 يضع يوسف يده على عينيك، فيكثر شعبك في أرض مصر

\par 10 بعد ذلك سيصلون إلى هذا المكان وسيكونون بلا هموم.

\par 11 "فأُحسِنُ إليهم من أجلكم، ولكنهم سيُهجَّرون من هنا في هذه الأثناء."

\chapter{4}

\par 1 بعد ذلك، حان وقت مغادرة يعقوب إسرائيل لجسده

\par 2 فدعا يوسف وقال له: «إن كنت قد وجدت نعمة، فضع يدك المباركة تحت جنبي، وأقسم أمام الرب أنك ستضع جسدي في قبر آبائي».

\par 3 فقال له يوسف: «أفعل ما تأمرني به يا حبيب الله».

\par 4 فقال ليوسف: أريد أن تحلف لي.

\par 5 فحلف يوسف لأبيه يعقوب أن يحمل جسده إلى قبر أبيه، فقبل يعقوب حلف ابنه.

\par 6 وبعد ذلك وصل هذا التقرير إلى يوسف: "إن أباك قد أصبح مضطربًا".

\par 7 فأخذ ابنيه أفرايم ومنسى وذهب أمام يعقوب أبيه.

\par 8 فقال له يوسف: هؤلاء هم أبنائي الذين أعطاني الله إياهم في أرض مصر ليأتوا بعدي.

\par 9 فقال إسرائيل: «أقربوهم إليّ إلى هنا».

\par 10 لأن عيني إسرائيل كانتا قد ضعفتا من شيخوخته حتى لم يعد يستطيع أن يبصر.

\par 11 فقرب يوسف ابنيه فقبلهما يعقوب.

\par 12 ثم أمر يوسف أفرايم ومنسى أن يسجدا ليعقوب إلى الأرض

\par 13 أخذ يوسف منسى وجعله عن يمين إسرائيل، وأفرايم عن يساره

\par 14 فعاد إسرائيل يديه ووضع يده اليمنى على رأس أفرايم ويده اليسرى على رأس منسى

\par 15 وباركهما وأعادهما إلى أبيهما وقال: «الله الذي كان أبواي إبراهيم وإسحاق يعبدانه بخشوع، الله الذي قوّاني منذ صباي إلى هذا الوقت الذي أنقذني فيه الملاك من كل ضيقاتي، يبارك هذين الغلامين منسى وأفرايم

\par 16 ليكن اسمي عليهما، وأسماء آبائي القديسين إبراهيم وإسحاق

\par 17 وبعد ذلك قال إسرائيل ليوسف: «أنا أموت وأنتم ترجعون إلى أرض آبائكم والله يكون معكم.

\par 18 وقد نلتَ أنت أيضًا نعمةً عظيمةً، أعظم من نعمة إخوتك، لأني أخذتُ هذا السهم بقوسي وسيفي من الأموريين

\chapter{5}

\par 1 ثم أرسل يعقوب إلى جميع بنيه وقال لهم: «اجتمعوا إليّ لأخبركم بكل ما يأتي عليكم وما يصيب كل واحد منكم في الأيام الأخيرة».

\par 2 فاجتمعوا حول إسرائيل من كبيرهم إلى صغيرهم

\par 3 ثم تكلم يعقوب إسرائيل وقال لبنيه: «اسمعوا يا بني يعقوب، اسمعوا لأبيكم إسرائيل، من رأوبين بكري إلى بنيامين».

\par 4 ثم أخبرهم بما سيحدث للأطفال الاثني عشر، داعيًا كل واحد منهم وسبطه باسمه، وباركهم بالبركة السماوية

\par 5 بعد ذلك صمتوا لفترة قصيرة حتى يتمكن من الراحة

\par 6 ففرحت السماوات لأنه استطاع أن يراقب أماكن الراحة.*

\par 7 وإذا بجلادين كثيرين مختلفين في مظاهرهم.

\par 8 كانوا مُهيئين لتعذيب الخطاة، وهم هؤلاء: الزناة والزناة؛ والشهوات الذكورية؛ والأشرار الذين يُهينون السائل المنوي الذي وهبه الله؛ والمنجمون والسحرة؛ والأشرار وعبدة الأصنام الذين يتمسكون بالرجاسات؛ والقذفون الذين يحكمون بلسانين (خادعين).

\par 9 وأما هؤلاء الخطاة جميعًا، فعقوبتهم هي النار التي لا تُطفأ، والظلمة الخارجية حيث البكاء وصرير الأسنان

\par 10 [هنا توجد ثغرة في النص العربي. ففي النص البحيري، يُرفع يعقوب مرة أخرى، هذه المرة إلى السماء، حيث كل شيء نور وفرح

\par 11 يرى إبراهيم وإسحاق ويُرى جميع أفراح المخلَّصين

\par 12 يعود يعقوب إلى الأرض، ويعطي تعليمات لدفنه في أرض آبائه، ويتوفى عن عمر يناهز 147 عامًا

\par 13 ينزل الرب مع الملاكين ميخائيل وجبرائيل ليحملا روح يعقوب إلى السماء

\par 14 أمر يوسف بتحنيط جسد أبيه على الطريقة المصرية

\par 15 يُقضى أربعون يومًا في عملية التحنيط، ويُقضى ثمانين يومًا أخرى في الحداد على البطريرك. ]

\chapter{6}

\par 1 ولما انقضت أيام مناحهم، كان فرعون لا يزال يبكي على يعقوب من أجل محبته ليوسف

\par 2 ثم كلم يوسف عظماء فرعون وقال لهم: "بعد أن وجدت نعمة في أعينكم، أتتكلمون عني إلى فرعون الملك وتقولون له إن يعقوب حلفني أنه عندما يخرج من جسده، أدفن جسده في قبر آبائي في أرض كنعان في ذلك المكان؟"

\par 3 فقال فرعون ليوسف اذهب بسلام وادفن أباك كما حلف لك.

\par 4 "وخذ معك مركبات وخيولاً من خير مملكتي ومن أهل بيتي ما شئت."

\par 5 فسجد يوسف لله أمام فرعون، وخرج من عنده، وقام ليدفن أباه

\par 6 فخرج معه عبيد فرعون وشيوخ مصر وكل بيت يوسف وإخوته وكل إسرائيل

\par 7 صعدوا جميعًا معه إلى المركبات، وسار الوفد كجيش عظيم

\par 8 ونزلوا إلى أرض كنعان إلى ضفة النهر عبر الأردن، وندبوه في ذلك المكان حزنًا شديدًا

\par 9 لقد حافظوا على ذلك الحزن الشديد عليه لمدة سبعة أيام.

\par 10 فلما سمع أهل دان النحيب في أرضهم قالوا: هذا النحيب العظيم هو نحيب المصريين.

\par 11 إلى يومنا هذا [يُطلقون على ذلك المكان اسم "مندب المصريين"].

\par 12 ثم نقل إسرائيل ودُفن في أرض كنعان في القبر الثاني.

\par 13 هذا هو الذي اشتراه إبراهيم بإذن للدفن من عفرون مقابل ممرا

\par 14 بعد ذلك عاد يوسف إلى أرض مصر مع إخوته وجميع حاشية فرعون

\par 15 وعاش يوسف بعد وفاة أبيه سنينًا كثيرة.

\par 16 واستمر في حكم مصر، على الرغم من أن يعقوب مات وترك خلفه مع شعبه.

\chapter{7}

\par 1 هذا ما نقلناه: لقد وصفنا وفاة أبي الآباء، يعقوب إسرائيل، وحزنه عليه، بقدر ما نستطيع، كما هو مكتوب في أسفار الله الروحية، وكما وجدناه في كنز معرفة آبائنا الرسل القديسين الأطهار

\par 2 وإذا كنت ترغب في معرفة تاريخ حياة يعقوب والحصول على معرفة جديدة عنه، فاختر أبًا مذكورًا في العهد القديم

\par 3 موسى هو من كتبها، أول الأنبياء، ومؤلف الشريعة

\par 4 اقرأ منه ونمّ بصيرتك.

\par 5 ستجد هذا وأكثر فيه، مكتوبًا من أجلك

\par 6 وستجد أن الله وملائكته كانوا أصدقاؤهم وهم في أجسادهم، وأن الله ظل يكلمهم مرات عديدة في آيات مختلفة من الكتاب.

\par 7 ويقول أيضاً في آيات كثيرة عن أبينا يعقوب أبو الآباء في الكتاب هكذا: "يا ابني أبارك نسلك كنجوم السماء".

\par 8 وكان أبونا يعقوب يكلم ابنه يوسف ويقول له: ظهر لي إلهي في أرض كنعان في لوز وباركني وقال لي: أباركك وأكثرك وأجعلك شعباً عظيماً.

\par 9 "ويخرجون (إلى الحرب؟) مثل باقي الأمم على هذه الأرض، ويتكاثر نسلك إلى الأبد."

\par 10 هذا ما سمعناه يا إخوتي وأحبائي من آبائنا البطاركة.

\par 11 "ويجب علينا أن نتحمس لأعمالهم، وطهارتهم، وإيمانهم، ومحبتهم للبشرية، وقبولهم للغرباء، حتى نطالب بأن نكون أبنائهم في ملكوت السماوات، حتى يشفعوا لنا أمام الله لكي نخلص من عذاب الجحيم."

\par 12 هؤلاء هم الذين أطلق عليهم العرب لقب الآباء القديسين

\par 13 أرشد يعقوب أبناءه فيما يتعلق بالعقاب، وكان سيسميهم سيف الرب، وهو نهر النار، المُعد بأمواجها ليبتلع الأشرار والنجسين

\par 14 هذه هي الأمور التي شرحها وعلّمها أبو الآباء يعقوب، بقدرته، لجميع أبنائه، لكي يسمعها الحكماء، فيتبعوا البر بمحبة متبادلة، ورحمة ورأفة

\par 15 لأن الرحمة تنقذ الناس من العقوبات، والرحمة تتغلب على كثرة من المظالم

\par 16 حقًا، من يرحم الفقير، فإنه يُقرض الله

\par 17 والآن، يا أبنائي الأحباء، لا تتهاونوا في الصلاة والصيام أبدًا وفي أي وقت، وبحياة الدين ستطردون الشياطين

\par 18 يا ابني العزيز، تجنب طرق العالم الشريرة، وهي الغضب والفساد وجميع الأعمال الشريرة

\par 19 واحذروا الظلم والكفر والخطف.

\par 20 لأن الظالمين لن يرثوا ملكوت الله، ولا الزناة، ولا الملاعين، ولا الذين يرتكبون الفواحش والذين يمارسون الجنس مع الذكور، ولا الشرهون، ولا عبدة الأصنام، ولا الذين ينطقون باللعنات، ولا الذين ينجسون أنفسهم خارج الزواج الطاهر. وآخرون لم نقدمهم أو حتى نذكرهم لن يقتربوا من ملكوت الله.

\par 21 يا أبنائي، كرّموا القديسين، فهم من سيشفعون لكم

\par 22 يا أبنائي، كونوا كرماء مع الغرباء، وستُعطون تمامًا ما أُعطي لإبراهيم العظيم، أبا الآباء، ولأبينا إسحاق ابنه

\par 23 يا أبنائي، افعلوا للفقراء ما يزيد الشفقة عليهم هنا والآن، حتى يعطيكم الله خبز الحياة إلى الأبد في ملكوت الله

\par 24 فمن أعطى فقيرًا خبزًا في هذا العالم، سيعطيه الله نصيبًا من شجرة الحياة

\par 25 ألبسوا الفقير العريان على الأرض، لكي يلبسكم الله ثوب المجد في ملكوت السماوات، فتكونوا أبناء آبائنا القديسين، إبراهيم وإسحاق ويعقوب في السماء إلى الأبد

\par 26 اهتم بقراءة كلمة الله في كتبه هنا أدناه، وتذكر القديسين الذين كتبوا عن حياتهم ومعاناتهم وسجودهم في الصلاة

\par 27 في المستقبل، لن يُمنع من كتابتهم في سفر الحياة في ملكوت السماوات

\par 28 وستُحسب من بين القديسين، أولئك الذين أرضوا الله في حياتهم، وسيفرحون مع الملائكة في أرض الحياة الأبدية

\chapter{8}

\par 1 تُكرّمون ذكرى آبائنا البطاركة في مثل هذا الوقت من كل عام وفي نفس هذا اليوم، وهو الثامن والعشرون من شهر مصر

\par 2 هذا ما وجدناه مكتوبًا في وثائق آبائنا القديسين الذين كانوا مرضيين لله

\par 3 بفضل شفاعتهم وصلاتهم، سننال كل شيء، أي نصيبًا ومكانًا في ملكوت السماوات الذي يخص ربنا وإلهنا وسيدنا ومخلصنا، يسوع المسيح

\par 4 هو الذي نطلب منه أن يغفر لنا أخطائنا وزلاتنا وأن يتجاوز عن سيئاتنا

\par 5 ليكن لطيفًا معنا في يوم دينونته، وليُسمعنا صوتًا مملوءًا بالفرح واللطف والسرور، قائلًا: "تعالوا إليّ يا مباركي أبي، رثوا الملكوت الذي كان لكم من قبل إنشاء العالم."

\par 6 ونسأل الله أن نكون أهلاً لتلقي أسراره الإلهية، التي هي الوسيلة لمغفرة خطايانا

\par 7 ليعيننا على خلاص نفوسنا، وليصرف عنا ضربات العدو الشرير

\par 8 ليُقِفنا عن يمينه في هذا اليوم العظيم والمرعب، بشفاعة سيدة الشفاعات، مصدر الطهارة والكرم والبركات، أم الخلاص؛* وبشفاعة جميع الشهداء والقديسين وصانعي الأعمال المرضية، وكل من أرضى الرب بأعماله الصالحة وإرادته الصالحة

\par 9 آمين، آمين، آمين. والحمد لله دائمًا، إلى الأبد، إلى أبد الآبدين.

\end{document}