\begin{document}

\title{1 يوحنا}


\chapter{1}

\par 1 اَلَّذِي كَانَ مِنَ الْبَدْءِ، الَّذِي سَمِعْنَاهُ، الَّذِي رَأَيْنَاهُ بِعُيُونِنَا، الَّذِي شَاهَدْنَاهُ، وَلَمَسَتْهُ أَيْدِينَا، مِنْ جِهَةِ كَلِمَةِ الْحَيَاةِ.
\par 2 فَإِنَّ الْحَيَاةَ أُظْهِرَتْ، وَقَدْ رَأَيْنَا وَنَشْهَدُ وَنُخْبِرُكُمْ بِالْحَيَاةِ الأَبَدِيَّةِ الَّتِي كَانَتْ عِنْدَ الآبِ وَأُظْهِرَتْ لَنَا.
\par 3 الَّذِي رَأَيْنَاهُ وَسَمِعْنَاهُ نُخْبِرُكُمْ بِهِ، لِكَيْ يَكُونَ لَكُمْ أَيْضاً شَرِكَةٌ مَعَنَا. وَأَمَّا شَرِكَتُنَا نَحْنُ فَهِيَ مَعَ الآبِ وَمَعَ ابْنِهِ يَسُوعَ الْمَسِيحِ.
\par 4 وَنَكْتُبُ إِلَيْكُمْ هَذَا لِكَيْ يَكُونَ فَرَحُكُمْ كَامِلاً.
\par 5 وَهَذَا هُوَ الْخَبَرُ الَّذِي سَمِعْنَاهُ مِنْهُ وَنُخْبِرُكُمْ بِهِ: إِنَّ اللهَ نُورٌ وَلَيْسَ فِيهِ ظُلْمَةٌ الْبَتَّةَ.
\par 6 إِنْ قُلْنَا إِنَّ لَنَا شَرِكَةً مَعَهُ وَسَلَكْنَا فِي الظُّلْمَةِ، نَكْذِبُ وَلَسْنَا نَعْمَلُ الْحَقَّ.
\par 7 وَلَكِنْ إِنْ سَلَكْنَا فِي النُّورِ كَمَا هُوَ فِي النُّورِ، فَلَنَا شَرِكَةٌ بَعْضِنَا مَعَ بَعْضٍ، وَدَمُ يَسُوعَ الْمَسِيحِ ابْنِهِ يُطَهِّرُنَا مِنْ كُلِّ خَطِيَّةٍ.
\par 8 إِنْ قُلْنَا إِنَّهُ لَيْسَ لَنَا خَطِيَّةٌ نُضِلُّ أَنْفُسَنَا وَلَيْسَ الْحَقُّ فِينَا.
\par 9 إِنِ اعْتَرَفْنَا بِخَطَايَانَا فَهُوَ أَمِينٌ وَعَادِلٌ، حَتَّى يَغْفِرَ لَنَا خَطَايَانَا وَيُطَهِّرَنَا مِنْ كُلِّ إِثْمٍ.
\par 10 إِنْ قُلْنَا إِنَّنَا لَمْ نُخْطِئْ نَجْعَلْهُ كَاذِباً، وَكَلِمَتُهُ لَيْسَتْ فِينَا.

\chapter{2}

\par 1 يَا أَوْلاَدِي، أَكْتُبُ إِلَيْكُمْ هَذَا لِكَيْ لاَ تُخْطِئُوا. وَإِنْ أَخْطَأَ أَحَدٌ فَلَنَا شَفِيعٌ عِنْدَ الآبِ، يَسُوعُ الْمَسِيحُ الْبَارُّ.
\par 2 وَهُوَ كَفَّارَةٌ لِخَطَايَانَا. لَيْسَ لِخَطَايَانَا فَقَطْ، بَلْ لِخَطَايَا كُلِّ الْعَالَمِ أَيْضاً.
\par 3 وَبِهَذَا نَعْرِفُ أَنَّنَا قَدْ عَرَفْنَاهُ: إِنْ حَفِظْنَا وَصَايَاهُ.
\par 4 مَنْ قَالَ قَدْ عَرَفْتُهُ وَهُوَ لاَ يَحْفَظُ وَصَايَاهُ، فَهُوَ كَاذِبٌ وَلَيْسَ الْحَقُّ فِيهِ.
\par 5 وَأَمَّا مَنْ حَفِظَ كَلِمَتَهُ، فَحَقّاً فِي هَذَا قَدْ تَكَمَّلَتْ مَحَبَّةُ اللهِ. بِهَذَا نَعْرِفُ أَنَّنَا فِيهِ:
\par 6 مَنْ قَالَ إِنَّهُ ثَابِتٌ فِيهِ، يَنْبَغِي أَنَّهُ كَمَا سَلَكَ ذَاكَ هَكَذَا يَسْلُكُ هُوَ أَيْضاً.
\par 7 أَيُّهَا الإِخْوَةُ، لَسْتُ أَكْتُبُ إِلَيْكُمْ وَصِيَّةً جَدِيدَةً، بَلْ وَصِيَّةً قَدِيمَةً كَانَتْ عِنْدَكُمْ مِنَ الْبَدْءِ. الْوَصِيَّةُ الْقَدِيمَةُ هِيَ الْكَلِمَةُ الَّتِي سَمِعْتُمُوهَا مِنَ الْبَدْءِ.
\par 8 أَيْضاً وَصِيَّةً جَدِيدَةً أَكْتُبُ إِلَيْكُمْ، مَا هُوَ حَقٌّ فِيهِ وَفِيكُمْ، أَنَّ الظُّلْمَةَ قَدْ مَضَتْ، وَالنُّورَ الْحَقِيقِيَّ الآنَ يُضِيءُ.
\par 9 مَنْ قَالَ إِنَّهُ فِي النُّورِ وَهُوَ يُبْغِضُ أَخَاهُ، فَهُوَ إِلَى الآنَ فِي الظُّلْمَةِ.
\par 10 مَنْ يُحِبُّ أَخَاهُ يَثْبُتُ فِي النُّورِ وَلَيْسَ فِيهِ عَثْرَةٌ.
\par 11 وَأَمَّا مَنْ يُبْغِضُ أَخَاهُ فَهُوَ فِي الظُّلْمَةِ، وَفِي الظُّلْمَةِ يَسْلُكُ، وَلاَ يَعْلَمُ أَيْنَ يَمْضِي، لأَنَّ الظُّلْمَةَ أَعْمَتْ عَيْنَيْهِ.
\par 12 أَكْتُبُ إِلَيْكُمْ أَيُّهَا الأَوْلاَدُ لأَنَّهُ قَدْ غُفِرَتْ لَكُمُ الْخَطَايَا مِنْ أَجْلِ اسْمِهِ.
\par 13 أَكْتُبُ إِلَيْكُمْ أَيُّهَا الآبَاءُ لأَنَّكُمْ قَدْ عَرَفْتُمُ الَّذِي مِنَ الْبَدْءِ. أَكْتُبُ إِلَيْكُمْ أَيُّهَا الأَحْدَاثُ لأَنَّكُمْ قَدْ غَلَبْتُمُ الشِّرِّيرَ. أَكْتُبُ إِلَيْكُمْ أَيُّهَا الأَوْلاَدُ لأَنَّكُمْ قَدْ عَرَفْتُمُ الآبَ.
\par 14 كَتَبْتُ إِلَيْكُمْ أَيُّهَا الآبَاءُ لأَنَّكُمْ قَدْ عَرَفْتُمُ الَّذِي مِنَ الْبَدْءِ. كَتَبْتُ إِلَيْكُمْ أَيُّهَا الأَحْدَاثُ لأَنَّكُمْ أَقْوِيَاءُ، وَكَلِمَةُ اللهِ ثَابِتَةٌ فِيكُمْ، وَقَدْ غَلَبْتُمُ الشِّرِّيرَ.
\par 15 لاَ تُحِبُّوا الْعَالَمَ وَلاَ الأَشْيَاءَ الَّتِي فِي الْعَالَمِ. إِنْ أَحَبَّ أَحَدٌ الْعَالَمَ فَلَيْسَتْ فِيهِ مَحَبَّةُ الآبِ.
\par 16 لأَنَّ كُلَّ مَا فِي الْعَالَمِ شَهْوَةَ الْجَسَدِ، وَشَهْوَةَ الْعُيُونِ، وَتَعَظُّمَ الْمَعِيشَةِ، لَيْسَ مِنَ الآبِ بَلْ مِنَ الْعَالَمِ.
\par 17 وَالْعَالَمُ يَمْضِي وَشَهْوَتُهُ، وَأَمَّا الَّذِي يَصْنَعُ مَشِيئَةَ اللهِ فَيَثْبُتُ إِلَى الأَبَدِ.
\par 18 أَيُّهَا الأَوْلاَدُ هِيَ السَّاعَةُ الأَخِيرَةُ. وَكَمَا سَمِعْتُمْ أَنَّ ضِدَّ الْمَسِيحِ يَأْتِي، قَدْ صَارَ الآنَ أَضْدَادٌ لِلْمَسِيحِ كَثِيرُونَ. مِنْ هُنَا نَعْلَمُ أَنَّهَا السَّاعَةُ الأَخِيرَةُ.
\par 19 مِنَّا خَرَجُوا، لَكِنَّهُمْ لَمْ يَكُونُوا مِنَّا، لأَنَّهُمْ لَوْ كَانُوا مِنَّا لَبَقُوا مَعَنَا. لَكِنْ لِيُظْهَرُوا أَنَّهُمْ لَيْسُوا جَمِيعُهُمْ مِنَّا.
\par 20 وَأَمَّا أَنْتُمْ فَلَكُمْ مَسْحَةٌ مِنَ الْقُدُّوسِ وَتَعْلَمُونَ كُلَّ شَيْءٍ.
\par 21 لَمْ أَكْتُبْ إِلَيْكُمْ لأَنَّكُمْ لَسْتُمْ تَعْلَمُونَ الْحَقَّ، بَلْ لأَنَّكُمْ تَعْلَمُونَهُ، وَأَنَّ كُلَّ كَذِبٍ لَيْسَ مِنَ الْحَقِّ.
\par 22 مَنْ هُوَ الْكَذَّابُ، إِلاَّ الَّذِي يُنْكِرُ أَنَّ يَسُوعَ هُوَ الْمَسِيحُ؟ هَذَا هُوَ ضِدُّ الْمَسِيحِ، الَّذِي يُنْكِرُ الآبَ وَالاِبْنَ.
\par 23 كُلُّ مَنْ يُنْكِرُ الاِبْنَ لَيْسَ لَهُ الآبُ أَيْضاً، وَمَنْ يَعْتَرِفُ بِالاِبْنِ فَلَهُ الآبُ أَيْضاً.
\par 24 أَمَّا أَنْتُمْ فَمَا سَمِعْتُمُوهُ مِنَ الْبَدْءِ فَلْيَثْبُتْ إِذاً فِيكُمْ. إِنْ ثَبَتَ فِيكُمْ مَا سَمِعْتُمُوهُ مِنَ الْبَدْءِ، فَأَنْتُمْ أَيْضاً تَثْبُتُونَ فِي الاِبْنِ وَفِي الآبِ.
\par 25 وَهَذَا هُوَ الْوَعْدُ الَّذِي وَعَدَنَا هُوَ بِهِ: الْحَيَاةُ الأَبَدِيَّةُ.
\par 26 كَتَبْتُ إِلَيْكُمْ هَذَا عَنِ الَّذِينَ يُضِلُّونَكُمْ.
\par 27 وَأَمَّا أَنْتُمْ فَالْمَسْحَةُ الَّتِي أَخَذْتُمُوهَا مِنْهُ ثَابِتَةٌ فِيكُمْ، وَلاَ حَاجَةَ بِكُمْ إِلَى أَنْ يُعَلِّمَكُمْ أَحَدٌ، بَلْ كَمَا تُعَلِّمُكُمْ هَذِهِ الْمَسْحَةُ عَيْنُهَا عَنْ كُلِّ شَيْءٍ، وَهِيَ حَقٌّ وَلَيْسَتْ كَذِباً. كَمَا عَلَّمَتْكُمْ تَثْبُتُونَ فِيهِ.
\par 28 وَالآنَ أَيُّهَا الأَوْلاَدُ، اثْبُتُوا فِيهِ، حَتَّى إِذَا أُظْهِرَ يَكُونُ لَنَا ثِقَةٌ، وَلاَ نَخْجَلُ مِنْهُ فِي مَجِيئِهِ.
\par 29 إِنْ عَلِمْتُمْ أَنَّهُ بَارٌّ هُوَ، فَاعْلَمُوا أَنَّ كُلَّ مَنْ يَصْنَعُ الْبِرَّ مَوْلُودٌ مِنْهُ.

\chapter{3}

\par 1 أُنْظُرُوا أَيَّةَ مَحَبَّةٍ أَعْطَانَا الآبُ حَتَّى نُدْعَى أَوْلاَدَ اللهِ! مِنْ أَجْلِ هَذَا لاَ يَعْرِفُنَا الْعَالَمُ، لأَنَّهُ لاَ يَعْرِفُهُ.
\par 2 أَيُّهَا الأَحِبَّاءُ، الآنَ نَحْنُ أَوْلاَدُ اللهِ، وَلَمْ يُظْهَرْ بَعْدُ مَاذَا سَنَكُونُ. وَلَكِنْ نَعْلَمُ أَنَّهُ إِذَا أُظْهِرَ نَكُونُ مِثْلَهُ، لأَنَّنَا سَنَرَاهُ كَمَا هُوَ.
\par 3 وَكُلُّ مَنْ عِنْدَهُ هَذَا الرَّجَاءُ بِهِ، يُطَهِّرُ نَفْسَهُ كَمَا هُوَ طَاهِرٌ.
\par 4 كُلُّ مَنْ يَفْعَلُ الْخَطِيَّةَ يَفْعَلُ التَّعَدِّيَ أَيْضاً. وَالْخَطِيَّةُ هِيَ التَّعَدِّي.
\par 5 وَتَعْلَمُونَ أَنَّ ذَاكَ أُظْهِرَ لِكَيْ يَرْفَعَ خَطَايَانَا، وَلَيْسَ فِيهِ خَطِيَّةٌ.
\par 6 كُلُّ مَنْ يَثْبُتُ فِيهِ لاَ يُخْطِئُ. كُلُّ مَنْ يُخْطِئُ لَمْ يُبْصِرْهُ وَلاَ عَرَفَهُ.
\par 7 أَيُّهَا الأَوْلاَدُ، لاَ يُضِلَّكُمْ أَحَدٌ. مَنْ يَفْعَلُ الْبِرَّ فَهُوَ بَارٌّ، كَمَا أَنَّ ذَاكَ بَارٌّ.
\par 8 مَنْ يَفْعَلُ الْخَطِيَّةَ فَهُوَ مِنْ إِبْلِيسَ، لأَنَّ إِبْلِيسَ مِنَ الْبَدْءِ يُخْطِئُ. لأَجْلِ هَذَا أُظْهِرَ ابْنُ اللهِ لِكَيْ يَنْقُضَ أَعْمَالَ إِبْلِيسَ.
\par 9 كُلُّ مَنْ هُوَ مَوْلُودٌ مِنَ اللهِ لاَ يَفْعَلُ خَطِيَّةً، لأَنَّ زَرْعَهُ يَثْبُتُ فِيهِ، وَلاَ يَسْتَطِيعُ أَنْ يُخْطِئَ لأَنَّهُ مَوْلُودٌ مِنَ اللهِ.
\par 10 بِهَذَا أَوْلاَدُ اللهِ ظَاهِرُونَ وَأَوْلاَدُ إِبْلِيسَ. كُلُّ مَنْ لاَ يَفْعَلُ الْبِرَّ فَلَيْسَ مِنَ اللهِ، وَكَذَا مَنْ لاَ يُحِبُّ أَخَاهُ.
\par 11 لأَنَّ هَذَا هُوَ الْخَبَرُ الَّذِي سَمِعْتُمُوهُ مِنَ الْبَدْءِ: أَنْ يُحِبَّ بَعْضُنَا بَعْضاً -
\par 12 لَيْسَ كَمَا كَانَ قَايِينُ مِنَ الشِّرِّيرِ وَذَبَحَ أَخَاهُ. وَلِمَاذَا ذَبَحَهُ؟ لأَنَّ أَعْمَالَهُ كَانَتْ شِرِّيرَةً، وَأَعْمَالَ أَخِيهِ بَارَّةٌ.
\par 13 لاَ تَتَعَجَّبُوا يَا إِخْوَتِي إِنْ كَانَ الْعَالَمُ يُبْغِضُكُمْ.
\par 14 نَحْنُ نَعْلَمُ أَنَّنَا قَدِ انْتَقَلْنَا مِنَ الْمَوْتِ إِلَى الْحَيَاةِ لأَنَّنَا نُحِبُّ الإِخْوَةَ. مَنْ لاَ يُحِبَّ أَخَاهُ يَبْقَ فِي الْمَوْتِ.
\par 15 كُلُّ مَنْ يُبْغِضُ أَخَاهُ فَهُوَ قَاتِلُ نَفْسٍ، وَأَنْتُمْ تَعْلَمُونَ أَنَّ كُلَّ قَاتِلِ نَفْسٍ لَيْسَ لَهُ حَيَاةٌ أَبَدِيَّةٌ ثَابِتَةٌ فِيهِ.
\par 16 بِهَذَا قَدْ عَرَفْنَا الْمَحَبَّةَ: أَنَّ ذَاكَ وَضَعَ نَفْسَهُ لأَجْلِنَا، فَنَحْنُ يَنْبَغِي لَنَا أَنْ نَضَعَ نُفُوسَنَا لأَجْلِ الإِخْوَةِ.
\par 17 وَأَمَّا مَنْ كَانَ لَهُ مَعِيشَةُ الْعَالَمِ، وَنَظَرَ أَخَاهُ مُحْتَاجاً، وَأَغْلَقَ أَحْشَاءَهُ عَنْهُ، فَكَيْفَ تَثْبُتُ مَحَبَّةُ اللهِ فِيهِ؟
\par 18 يَا أَوْلاَدِي، لاَ نُحِبَّ بِالْكَلاَمِ وَلاَ بِاللِّسَانِ، بَلْ بِالْعَمَلِ وَالْحَقِّ!
\par 19 وَبِهَذَا نَعْرِفُ أَنَّنَا مِنَ الْحَقِّ وَنُسَكِّنُ قُلُوبَنَا قُدَّامَهُ.
\par 20 لأَنَّهُ إِنْ لاَمَتْنَا قُلُوبُنَا فَاللهُ أَعْظَمُ مِنْ قُلُوبِنَا، وَيَعْلَمُ كُلَّ شَيْءٍ.
\par 21 أَيُّهَا الأَحِبَّاءُ، إِنْ لَمْ تَلُمْنَا قُلُوبُنَا فَلَنَا ثِقَةٌ مِنْ نَحْوِ اللهِ.
\par 22 وَمَهْمَا سَأَلْنَا نَنَالُ مِنْهُ، لأَنَّنَا نَحْفَظُ وَصَايَاهُ، وَنَعْمَلُ الأَعْمَالَ الْمَرْضِيَّةَ أَمَامَهُ.
\par 23 وَهَذِهِ هِيَ وَصِيَّتُهُ: أَنْ نُؤْمِنَ بِاسْمِ ابْنِهِ يَسُوعَ الْمَسِيحِ، وَنُحِبَّ بَعْضُنَا بَعْضاً كَمَا أَعْطَانَا وَصِيَّةً.
\par 24 وَمَنْ يَحْفَظْ وَصَايَاهُ يَثْبُتْ فِيهِ وَهُوَ فِيهِ. وَبِهَذَا نَعْرِفُ أَنَّهُ يَثْبُتُ فِينَا: مِنَ الرُّوحِ الَّذِي أَعْطَانَا.

\chapter{4}

\par 1 أَيُّهَا الأَحِبَّاءُ، لاَ تُصَدِّقُوا كُلَّ رُوحٍ، بَلِ امْتَحِنُوا الأَرْوَاحَ: هَلْ هِيَ مِنَ اللهِ؟ لأَنَّ أَنْبِيَاءَ كَذَبَةً كَثِيرِينَ قَدْ خَرَجُوا إِلَى الْعَالَمِ.
\par 2 بِهَذَا تَعْرِفُونَ رُوحَ اللهِ: كُلُّ رُوحٍ يَعْتَرِفُ بِيَسُوعَ الْمَسِيحِ أَنَّهُ قَدْ جَاءَ فِي الْجَسَدِ فَهُوَ مِنَ اللهِ،
\par 3 وَكُلُّ رُوحٍ لاَ يَعْتَرِفُ بِيَسُوعَ الْمَسِيحِ أَنَّهُ قَدْ جَاءَ فِي الْجَسَدِ فَلَيْسَ مِنَ اللهِ. وَهَذَا هُوَ رُوحُ ضِدِّ الْمَسِيحِ الَّذِي سَمِعْتُمْ أَنَّهُ يَأْتِي، وَالآنَ هُوَ فِي الْعَالَمِ.
\par 4 أَنْتُمْ مِنَ اللهِ أَيُّهَا الأَوْلاَدُ، وَقَدْ غَلَبْتُمُوهُمْ لأَنَّ الَّذِي فِيكُمْ أَعْظَمُ مِنَ الَّذِي فِي الْعَالَمِ.
\par 5 هُمْ مِنَ الْعَالَمِ. مِنْ أَجْلِ ذَلِكَ يَتَكَلَّمُونَ مِنَ الْعَالَمِ، وَالْعَالَمُ يَسْمَعُ لَهُمْ.
\par 6 نَحْنُ مِنَ اللهِ. فَمَنْ يَعْرِفُ اللهَ يَسْمَعُ لَنَا، وَمَنْ لَيْسَ مِنَ اللهِ لاَ يَسْمَعُ لَنَا. مِنْ هَذَا نَعْرِفُ رُوحَ الْحَقِّ وَرُوحَ الضَّلاَلِ.
\par 7 أَيُّهَا الأَحِبَّاءُ، لِنُحِبَّ بَعْضُنَا بَعْضاً، لأَنَّ الْمَحَبَّةَ هِيَ مِنَ اللهِ، وَكُلُّ مَنْ يُحِبُّ فَقَدْ وُلِدَ مِنَ اللهِ وَيَعْرِفُ اللهَ.
\par 8 وَمَنْ لاَ يُحِبُّ لَمْ يَعْرِفِ اللهَ، لأَنَّ اللهَ مَحَبَّةٌ.
\par 9 بِهَذَا أُظْهِرَتْ مَحَبَّةُ اللهِ فِينَا: أَنَّ اللهَ قَدْ أَرْسَلَ ابْنَهُ الْوَحِيدَ إِلَى الْعَالَمِ لِكَيْ نَحْيَا بِهِ.
\par 10 فِي هَذَا هِيَ الْمَحَبَّةُ: لَيْسَ أَنَّنَا نَحْنُ أَحْبَبْنَا اللهَ، بَلْ أَنَّهُ هُوَ أَحَبَّنَا، وَأَرْسَلَ ابْنَهُ كَفَّارَةً لِخَطَايَانَا.
\par 11 أَيُّهَا الأَحِبَّاءُ، إِنْ كَانَ اللهُ قَدْ أَحَبَّنَا هَكَذَا، يَنْبَغِي لَنَا أَيْضاً أَنْ يُحِبَّ بَعْضُنَا بَعْضاً.
\par 12 اَللهُ لَمْ يَنْظُرْهُ أَحَدٌ قَطُّ. إِنْ أَحَبَّ بَعْضُنَا بَعْضاً فَاللهُ يَثْبُتُ فِينَا، وَمَحَبَّتُهُ قَدْ تَكَمَّلَتْ فِينَا.
\par 13 بِهَذَا نَعْرِفُ أَنَّنَا نَثْبُتُ فِيهِ وَهُوَ فِينَا: أَنَّهُ قَدْ أَعْطَانَا مِنْ رُوحِهِ.
\par 14 وَنَحْنُ قَدْ نَظَرْنَا وَنَشْهَدُ أَنَّ الآبَ قَدْ أَرْسَلَ الاِبْنَ مُخَلِّصاً لِلْعَالَمِ.
\par 15 مَنِ اعْتَرَفَ أَنَّ يَسُوعَ هُوَ ابْنُ اللهِ، فَاللهُ يَثْبُتُ فِيهِ وَهُوَ فِي اللهِ.
\par 16 وَنَحْنُ قَدْ عَرَفْنَا وَصَدَّقْنَا الْمَحَبَّةَ الَّتِي لِلَّهِ فِينَا. اللهُ مَحَبَّةٌ، وَمَنْ يَثْبُتْ فِي الْمَحَبَّةِ يَثْبُتْ فِي اللهِ وَاللهُ فِيهِ.
\par 17 بِهَذَا تَكَمَّلَتِ الْمَحَبَّةُ فِينَا: أَنْ يَكُونَ لَنَا ثِقَةٌ فِي يَوْمِ الدِّينِ، لأَنَّهُ كَمَا هُوَ فِي هَذَا الْعَالَمِ هَكَذَا نَحْنُ أَيْضاً.
\par 18 لاَ خَوْفَ فِي الْمَحَبَّةِ، بَلِ الْمَحَبَّةُ الْكَامِلَةُ تَطْرَحُ الْخَوْفَ إِلَى خَارِجٍ لأَنَّ الْخَوْفَ لَهُ عَذَابٌ. وَأَمَّا مَنْ خَافَ فَلَمْ يَتَكَمَّلْ فِي الْمَحَبَّةِ.
\par 19 نَحْنُ نُحِبُّهُ لأَنَّهُ هُوَ أَحَبَّنَا أَوَّلاً.
\par 20 إِنْ قَالَ أَحَدٌ: «إِنِّي أُحِبُّ اللهَ» وَأَبْغَضَ أَخَاهُ، فَهُوَ كَاذِبٌ. لأَنَّ مَنْ لاَ يُحِبُّ أَخَاهُ الَّذِي أَبْصَرَهُ، كَيْفَ يَقْدِرُ أَنْ يُحِبَّ اللهَ الَّذِي لَمْ يُبْصِرْهُ؟
\par 21 وَلَنَا هَذِهِ الْوَصِيَّةُ مِنْهُ: أَنَّ مَنْ يُحِبُّ اللهَ يُحِبُّ أَخَاهُ أَيْضاً.

\chapter{5}

\par 1 كُلُّ مَنْ يُؤْمِنُ أَنَّ يَسُوعَ هُوَ الْمَسِيحُ فَقَدْ وُلِدَ مِنَ اللهِ. وَكُلُّ مَنْ يُحِبُّ الْوَالِدَ يُحِبُّ الْمَوْلُودَ مِنْهُ أَيْضاً.
\par 2 بِهَذَا نَعْرِفُ أَنَّنَا نُحِبُّ أَوْلاَدَ اللهِ: إِذَا أَحْبَبْنَا اللهَ وَحَفِظْنَا وَصَايَاهُ.
\par 3 فَإِنَّ هَذِهِ هِيَ مَحَبَّةُ اللهِ: أَنْ نَحْفَظَ وَصَايَاهُ. وَوَصَايَاهُ لَيْسَتْ ثَقِيلَةً،
\par 4 لأَنَّ كُلَّ مَنْ وُلِدَ مِنَ اللهِ يَغْلِبُ الْعَالَمَ. وَهَذِهِ هِيَ الْغَلَبَةُ الَّتِي تَغْلِبُ الْعَالَمَ: إِيمَانُنَا.
\par 5 مَنْ هُوَ الَّذِي يَغْلِبُ الْعَالَمَ، إِلاَّ الَّذِي يُؤْمِنُ أَنَّ يَسُوعَ هُوَ ابْنُ اللهِ؟
\par 6 هَذَا هُوَ الَّذِي أَتَى بِمَاءٍ وَدَمٍ، يَسُوعُ الْمَسِيحُ. لاَ بِالْمَاءِ فَقَطْ، بَلْ بِالْمَاءِ وَالدَّمِ. وَالرُّوحُ هُوَ الَّذِي يَشْهَدُ، لأَنَّ الرُّوحَ هُوَ الْحَقُّ.
\par 7 فَإِنَّ الَّذِينَ يَشْهَدُونَ فِي السَّمَاءِ هُمْ ثَلاَثَةٌ: الآبُ، وَالْكَلِمَةُ، وَالرُّوحُ الْقُدُسُ. وَهَؤُلاَءِ الثَّلاَثَةُ هُمْ وَاحِدٌ.
\par 8 وَالَّذِينَ يَشْهَدُونَ فِي الأَرْضِ هُمْ ثَلاَثَةٌ: الرُّوحُ، وَالْمَاءُ، وَالدَّمُ. وَالثَّلاَثَةُ هُمْ فِي الْوَاحِدِ.
\par 9 إِنْ كُنَّا نَقْبَلُ شَهَادَةَ النَّاسِ فَشَهَادَةُ اللهِ أَعْظَمُ، لأَنَّ هَذِهِ هِيَ شَهَادَةُ اللهِ الَّتِي قَدْ شَهِدَ بِهَا عَنِ ابْنِهِ.
\par 10 مَنْ يُؤْمِنُ بِابْنِ اللهِ فَعِنْدَهُ الشَّهَادَةُ فِي نَفْسِهِ. مَنْ لاَ يُصَدِّقُ اللهَ فَقَدْ جَعَلَهُ كَاذِباً، لأَنَّهُ لَمْ يُؤْمِنْ بِالشَّهَادَةِ الَّتِي قَدْ شَهِدَ بِهَا اللهُ عَنِ ابْنِهِ.
\par 11 وَهَذِهِ هِيَ الشَّهَادَةُ: أَنَّ اللهَ أَعْطَانَا حَيَاةً أَبَدِيَّةً، وَهَذِهِ الْحَيَاةُ هِيَ فِي ابْنِهِ.
\par 12 مَنْ لَهُ الاِبْنُ فَلَهُ الْحَيَاةُ، وَمَنْ لَيْسَ لَهُ ابْنُ اللهِ فَلَيْسَتْ لَهُ الْحَيَاةُ.
\par 13 كَتَبْتُ هَذَا إِلَيْكُمْ أَنْتُمُ الْمُؤْمِنِينَ بِاسْمِ ابْنِ اللهِ لِكَيْ تَعْلَمُوا أَنَّ لَكُمْ حَيَاةً أَبَدِيَّةً، وَلِكَيْ تُؤْمِنُوا بِاسْمِ ابْنِ اللهِ.
\par 14 وَهَذِهِ هِيَ الثِّقَةُ الَّتِي لَنَا عِنْدَهُ: أَنَّهُ إِنْ طَلَبْنَا شَيْئاً حَسَبَ مَشِيئَتِهِ يَسْمَعُ لَنَا.
\par 15 وَإِنْ كُنَّا نَعْلَمُ أَنَّهُ مَهْمَا طَلَبْنَا يَسْمَعُ لَنَا، نَعْلَمُ أَنَّ لَنَا الطِّلْبَاتِ الَّتِي طَلَبْنَاهَا مِنْهُ.
\par 16 إِنْ رَأَى أَحَدٌ أَخَاهُ يُخْطِئُ خَطِيَّةً لَيْسَتْ لِلْمَوْتِ، يَطْلُبُ، فَيُعْطِيهِ حَيَاةً لِلَّذِينَ يُخْطِئُونَ لَيْسَ لِلْمَوْتِ. تُوجَدُ خَطِيَّةٌ لِلْمَوْتِ. لَيْسَ لأَجْلِ هَذِهِ أَقُولُ أَنْ يُطْلَبَ.
\par 17 كُلُّ إِثْمٍ هُوَ خَطِيَّةٌ، وَتُوجَدُ خَطِيَّةٌ لَيْسَتْ لِلْمَوْتِ.
\par 18 نَعْلَمُ أَنَّ كُلَّ مَنْ وُلِدَ مِنَ اللهِ لاَ يُخْطِئُ، بَلِ الْمَوْلُودُ مِنَ اللهِ يَحْفَظُ نَفْسَهُ، وَالشِّرِّيرُ لاَ يَمَسُّهُ.
\par 19 نَعْلَمُ أَنَّنَا نَحْنُ مِنَ اللهِ، وَالْعَالَمَ كُلَّهُ قَدْ وُضِعَ فِي الشِّرِّيرِ.
\par 20 وَنَعْلَمُ أَنَّ ابْنَ اللهِ قَدْ جَاءَ وَأَعْطَانَا بَصِيرَةً لِنَعْرِفَ الْحَقَّ. وَنَحْنُ فِي الْحَقِّ فِي ابْنِهِ يَسُوعَ الْمَسِيحِ. هَذَا هُوَ الإِلَهُ الْحَقُّ وَالْحَيَاةُ الأَبَدِيَّةُ.
\par 21 أَيُّهَا الأَوْلاَدُ احْفَظُوا أَنْفُسَكُمْ مِنَ الأَصْنَامِ. آمِينَ.

\end{document}