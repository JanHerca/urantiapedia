\begin{document}

\title{نحميا}


\chapter{1}

\par 1 كَلاَمُ نَحَمْيَا بْنِ حَكَلْيَا: حَدَثَ فِي شَهْرِ كَسْلُو فِي السَّنَةِ الْعِشْرِينَ بَيْنَمَا كُنْتُ فِي شُوشَنَ الْقَصْرِ
\par 2 أَنَّهُ جَاءَ حَنَانِي وَاحِدٌ مِنْ إِخْوَتِي هُوَ وَرِجَالٌ مِنْ يَهُوذَا فَسَأَلْتُهُمْ عَنِ الْيَهُودِ الَّذِينَ نَجُوا الَّذِينَ بَقُوا مِنَ السَّبْيِ وَعَنْ أُورُشَلِيمَ.
\par 3 فَقَالُوا لِي: [إِنَّ الْبَاقِينَ الَّذِينَ بَقُوا مِنَ السَّبْيِ هُنَاكَ فِي الْبِلاَدِ هُمْ فِي شَرٍّ عَظِيمٍ وَعَارٍ. وَسُورُ أُورُشَلِيمَ مُنْهَدِمٌ وَأَبْوَابُهَا مَحْرُوقَةٌ بِالنَّارِ].
\par 4 فَلَمَّا سَمِعْتُ هَذَا الْكَلاَمَ جَلَسْتُ وَبَكَيْتُ وَنُحْتُ أَيَّاماً وَصُمْتُ وَصَلَّيْتُ أَمَامَ إِلَهِ السَّمَاءِ
\par 5 وَقُلْتُ: [أَيُّهَا الرَّبُّ إِلَهُ السَّمَاءِ الإِلَهُ الْعَظِيمُ الْمَخُوفُ الْحَافِظُ الْعَهْدَ وَالرَّحْمَةَ لِمُحِبِّيهِ وَحَافِظِي وَصَايَاهُ
\par 6 لِتَكُنْ أُذْنُكَ مُصْغِيَةً وَعَيْنَاكَ مَفْتُوحَتَيْنِ لِتَسْمَعَ صَلاَةَ عَبْدِكَ الَّذِي يُصَلِّي إِلَيْكَ الآنَ نَهَاراً وَلَيْلاً لأَجْلِ بَنِي إِسْرَائِيلَ عَبِيدِكَ وَيَعْتَرِفُ بِخَطَايَا بَنِي إِسْرَائِيلَ الَّتِي أَخْطَأْنَا بِهَا إِلَيْكَ. فَإِنِّي أَنَا وَبَيْتُ أَبِي قَدْ أَخْطَأْنَا.
\par 7 لَقَدْ أَفْسَدْنَا أَمَامَكَ وَلَمْ نَحْفَظِ الْوَصَايَا وَالْفَرَائِضَ وَالأَحْكَامَ الَّتِي أَمَرْتَ بِهَا مُوسَى عَبْدَكَ.
\par 8 اذْكُرِ الْكَلاَمَ الَّذِي أَمَرْتَ بِهِ مُوسَى عَبْدَكَ قَائِلاً: إِنْ خُنْتُمْ فَإِنِّي أُفَرِّقُكُمْ فِي الشُّعُوبِ
\par 9 وَإِنْ رَجَعْتُمْ إِلَيَّ وَحَفِظْتُمْ وَصَايَايَ وَعَمِلْتُمُوهَا - إِنْ كَانَ الْمَنْفِيُّونَ مِنْكُمْ فِي أَقْصَاءِ السَّمَاوَاتِ فَمِنْ هُنَاكَ أَجْمَعُهُمْ وَآتِي بِهِمْ إِلَى الْمَكَانِ الَّذِي اخْتَرْتُ لإِسْكَانِ اسْمِي فِيهِ.
\par 10 فَهُمْ عَبِيدُكَ وَشَعْبُكَ الَّذِي افْتَدَيْتَ بِقُوَّتِكَ الْعَظِيمَةِ وَيَدِكَ الشَّدِيدَةِ.
\par 11 يَا سَيِّدُ لِتَكُنْ أُذْنُكَ مُصْغِيَةً إِلَى صَلاَةِ عَبْدِكَ وَصَلاَةِ عَبِيدِكَ الَّذِينَ يُرِيدُونَ مَخَافَةَ اسْمِكَ. وَأَعْطِ النَّجَاحَ الْيَوْمَ لِعَبْدِكَ وَامْنَحْهُ رَحْمَةً أَمَامَ هَذَا الرَّجُلِ]. لأَنِّي كُنْتُ سَاقِياً لِلْمَلِكِ.

\chapter{2}

\par 1 وَفِي شَهْرِ نِيسَانَ فِي السَّنَةِ الْعِشْرِينَ لأَرْتَحْشَسْتَا الْمَلِكِ. كَانَتْ خَمْرٌ أَمَامَهُ فَحَمَلْتُ الْخَمْرَ وَأَعْطَيْتُ الْمَلِكَ. وَلَمْ أَكُنْ قَبْلُ مُكَمَّداً أَمَامَهُ.
\par 2 فَقَالَ لِي الْمَلِكُ: [لِمَاذَا وَجْهُكَ مُكَمَّدٌ وَأَنْتَ غَيْرُ مَرِيضٍ؟ مَا هَذَا إِلاَّ كآبَةَ قَلْبٍ!] فَخِفْتُ كَثِيراً جِدّاً
\par 3 وَقُلْتُ لِلْمَلِكِ: [لِيَحْيَ الْمَلِكُ إِلَى الأَبَدِ. كَيْفَ لاَ يَكْمَدُّ وَجْهِي وَالْمَدِينَةُ بَيْتُ مَقَابِرِ آبَائِي خَرَابٌ وَأَبْوَابُهَا قَدْ أَكَلَتْهَا النَّارُ؟]
\par 4 فَقَالَ لِي الْمَلِكُ: [مَاذَا طَالِبٌ أَنْتَ؟] فَصَلَّيْتُ إِلَى إِلَهِ السَّمَاءِ
\par 5 وَقُلْتُ لِلْمَلِكِ: [إِذَا سُرَّ الْمَلِكُ وَإِذَا أَحْسَنَ عَبْدُكَ أَمَامَكَ تُرْسِلُنِي إِلَى يَهُوذَا إِلَى مَدِينَةِ قُبُورِ آبَائِي فَأَبْنِيهَا].
\par 6 فَقَالَ لِي الْمَلِكُ وَالْمَلِكَةُ جَالِسَةٌ بِجَانِبِهِ: [إِلَى مَتَى يَكُونُ سَفَرُكَ وَمَتَى تَرْجِعُ؟] فَحَسُنَ لَدَى الْمَلِكِ وَأَرْسَلَنِي فَعَيَّنْتُ لَهُ زَمَاناً.
\par 7 وَقُلْتُ لِلْمَلِكِ: [إِنْ حَسُنَ عِنْدَ الْمَلِكِ فَلْتُعْطَ لِي رَسَائِلُ إِلَى وُلاَةِ عَبْرِ النَّهْرِ لِيُجِيزُونِي حَتَّى أَصِلَ إِلَى يَهُوذَا
\par 8 وَرِسَالَةٌ إِلَى آسَافَ حَارِسِ فِرْدَوْسِ الْمَلِكِ لِيُعْطِيَنِي أَخْشَاباً لِسَقْفِ أَبْوَابِ الْقَصْرِ الَّذِي لِلْبَيْتِ وَلِسُورِ الْمَدِينَةِ وَلِلْبَيْتِ الَّذِي أَدْخُلُ إِلَيْهِ]. فَأَعْطَانِي الْمَلِكُ حَسَبَ يَدِ إِلَهِي الصَّالِحَةِ عَلَيَّ.
\par 9 فَأَتَيْتُ إِلَى وُلاَةِ عَبْرِ النَّهْرِ وَأَعْطَيْتُهُمْ رَسَائِلَ الْمَلِكِ. وَأَرْسَلَ مَعِي الْمَلِكُ رُؤَسَاءَ جَيْشٍ وَفُرْسَاناً.
\par 10 وَلَمَّا سَمِعَ سَنْبَلَّطُ الْحُورُونِيُّ وَطُوبِيَّا الْعَبْدُ الْعَمُّونِيُّ سَاءَهُمَا مَسَاءَةً عَظِيمَةً لأَنَّهُ جَاءَ رَجُلٌ يَطْلُبُ خَيْراً لِبَنِي إِسْرَائِيلَ.
\par 11 فَجِئْتُ إِلَى أُورُشَلِيمَ وَكُنْتُ هُنَاكَ ثَلاَثَةَ أَيَّامٍ.
\par 12 ثُمَّ قُمْتُ لَيْلاً أَنَا وَرِجَالٌ قَلِيلُونَ مَعِي. وَلَمْ أُخْبِرْ أَحَداً بِمَا جَعَلَهُ إِلَهِي فِي قَلْبِي لأَعْمَلَهُ فِي أُورُشَلِيمَ. وَلَمْ يَكُنْ مَعِي بَهِيمَةٌ إِلاَّ الْبَهِيمَةُ الَّتِي كُنْتُ رَاكِبَهَا.
\par 13 وَخَرَجْتُ مِنْ بَابِ الْوَادِي لَيْلاً أَمَامَ عَيْنِ التِّنِّينِ إِلَى بَابِ الدِّمْنِ وَصِرْتُ أَتَفَرَّسُ فِي أَسْوَارِ أُورُشَلِيمَ الْمُنْهَدِمَةِ وَأَبْوَابِهَا الَّتِي أَكَلَتْهَا النَّارُ.
\par 14 وَعَبَرْتُ إِلَى بَابِ الْعَيْنِ وَإِلَى بِرْكَةِ الْمَلِكِ وَلَمْ يَكُنْ مَكَانٌ لِعُبُورِ الْبَهِيمَةِ الَّتِي تَحْتِي.
\par 15 فَصَعِدْتُ فِي الْوَادِي لَيْلاً وَكُنْتُ أَتَفَرَّسُ فِي السُّورِ ثُمَّ عُدْتُ فَدَخَلْتُ مِنْ بَابِ الْوَادِي رَاجِعاً.
\par 16 وَلَمْ يَعْرِفِ الْوُلاَةُ إِلَى أَيْنَ ذَهَبْتُ وَلاَ مَا أَنَا عَامِلٌ وَلَمْ أُخْبِرْ إِلَى ذَلِكَ الْوَقْتِ الْيَهُودَ وَالْكَهَنَةَ وَالأَشْرَافَ وَالْوُلاَةَ وَبَاقِي عَامِلِي الْعَمَلِ.
\par 17 ثُمَّ قُلْتُ لَهُمْ: [أَنْتُمْ تَرُونَ الشَّرَّ الَّذِي نَحْنُ فِيهِ كَيْفَ أَنَّ أُورُشَلِيمَ خَرِبَةٌ وَأَبْوَابَهَا قَدْ أُحْرِقَتْ بِالنَّارِ. هَلُمَّ فَنَبْنِيَ سُورَ أُورُشَلِيمَ وَلاَ نَكُونُ بَعْدُ عَاراً].
\par 18 وَأَخْبَرْتُهُمْ عَنْ يَدِ إِلَهِي الصَّالِحَةِ عَلَيَّ وَأَيْضاً عَنْ كَلاَمِ الْمَلِكِ الَّذِي قَالَهُ لِي. فَقَالُوا: [لِنَقُمْ وَلْنَبْنِ]. وَشَدَّدُوا أَيَادِيَهُمْ لِلْخَيْرِ.
\par 19 وَلَمَّا سَمِعَ سَنْبَلَّطُ الْحُورُونِيُّ وَطُوبِيَّا الْعَبْدُ الْعَمُّونِيُّ وَجَشَمٌ الْعَرَبِيُّ هَزَأُوا بِنَا وَاحْتَقَرُونَا وَقَالُوا: [مَا هَذَا الأَمْرُ الَّذِي أَنْتُمْ عَامِلُونَ؟ أَعَلَى الْمَلِكِ تَتَمَرَّدُونَ؟].
\par 20 فَأَجَبْتُهُمْ: [إِنَّ إِلَهَ السَّمَاءِ يُعْطِينَا النَّجَاحَ وَنَحْنُ عَبِيدُهُ نَقُومُ وَنَبْنِي. وَأَمَّا أَنْتُمْ فَلَيْسَ لَكُمْ نَصِيبٌ وَلاَ حَقٌّ وَلاَ ذِكْرٌ فِي أُورُشَلِيمَ].

\chapter{3}

\par 1 وَقَامَ أَلْيَاشِيبُ الْكَاهِنُ الْعَظِيمُ وَإِخْوَتُهُ الْكَهَنَةُ وَبَنُوا بَابَ الضَّأْنِ. هُمْ قَدَّسُوهُ وَأَقَامُوا مَصَارِيعَهُ وَقَدَّسُوهُ إِلَى بُرْجِ الْمِئَةِ إِلَى بُرْجِ حَنَنْئِيلَ.
\par 2 وَبِجَانِبِهِ بَنَى رِجَالُ أَرِيحَا وَبِجَانِبِهِمْ بَنَى زَكُّورُ بْنُ إِمْرِي.
\par 3 وَبَابُ السَّمَكِ بَنَاهُ بَنُو هَسْنَاءَةَ. هُمْ سَقَفُوهُ وَأَوْقَفُوا مَصَارِيعَهُ وَأَقْفَالَهُ وَعَوَارِضَهُ.
\par 4 وَبِجَانِبِهِمْ رَمَّمَ مَرِيمُوثُ بْنُ أُورِيَّا بْنِ هَقُّوصَ. وَبِجَانِبِهِمْ رَمَّمَ مَشُلاَّمُ بْنُ بَرَخْيَا بْنِ مَشِيزَبْئِيلَ. وَبِجَانِبِهِمْ رَمَّمَ صَادُوقُ بْنُ بَعْنَا.
\par 5 وَبِجَانِبِهِمْ رَمَّمَ التَّقُوعِيُّونَ وَأَمَّا عُظَمَاؤُهُمْ فَلَمْ يُدْخِلُوا أَعْنَاقَهُمْ فِي عَمَلِ سَيِّدِهِمْ.
\par 6 وَالْبَابُ الْعَتِيقُ رَمَّمَهُ يُويَادَاعُ بْنُ فَاسِيحَ وَمَشُلاَّمُ بْنُ بَسُودِْيَا. هُمَا سَقَفَاهُ وَأَقَامَا مَصَارِيعَهُ وَأَقْفَالَهُ وَعَوَارِضَهُ.
\par 7 وَبِجَانِبِهِمَا رَمَّمَ مَلَطْيَا الْجِبْعُونِيُّ وَيَادُونُ الْمِيَرُونُوثِيُّ مِنْ أَهْلِ جِبْعُونَ وَالْمِصْفَاةِ إِلَى كُرْسِيِّ وَالِي عَبْرِ النَّهْرِ.
\par 8 وَبِجَانِبِهِمَا رَمَّمَ عُزِّيئِيلُ بْنُ حَرْهَايَا مِنَ الصَّيَّاغِينَ. وَبِجَانِبِهِ رَمَّمَ حَنَنْيَا مِنَ الْعَطَّارِينَ. وَتَرَكُوا أُورُشَلِيمَ إِلَى السُّورِ الْعَرِيضِ.
\par 9 وَبِجَانِبِهِمْ رَمَّمَ رَفَايَا بْنُ حُورٍ رَئِيسُ نِصْفِ دَائِرَةِ أُورُشَلِيمَ.
\par 10 وَبِجَانِبِهِمْ رَمَّمَ يَدَايَا بْنُ حَرُومَافَ وَمُقَابَِلَ بَيْتِهِ. وَبِجَانِبِهِ رَمَّمَ حَطُّوشُ بْنُ حَشَبْنِيَا.
\par 11 قِسْمٌ ثَانٍ رَمَّمَهُ مَلْكِيَّا بْنُ حَارِيمَ وَحَشُّوبُ بْنُ فَحَثَ مُوآبَ وَبُرْجَ التَّنَانِيرِ.
\par 12 وَبِجَانِبِهِ رَمَّمَ شَلُّومُ بْنُ هَلُّوحِيشَ رَئِيسُ نِصْفِ دَائِرَةِ أُورُشَلِيمَ هُوَ وَبَنَاتُهُ.
\par 13 بَابُ الْوَادِي رَمَّمَهُ حَانُونُ وَسُكَّانُ زَانُوحَ هُمْ بَنَوْهُ وَأَقَامُوا مَصَارِيعَهُ وَأَقْفَالَهُ وَعَوَارِضَهُ وَأَلْفَ ذِرَاعٍ عَلَى السُّورِ إِلَى بَابِ الدِّمْنِ.
\par 14 وَبَابُ الدِّمْنِ رَمَّمَهُ مَلْكِيَّا بْنُ رَكَابَ رَئِيسُ دَائِرَةِ بَيْتِ هَكَّارِيمَ. هُوَ بَنَاهُ وَأَقَامَ مَصَارِيعَهُ وَأَقْفَالَهُ وَعَوَارِضَهُ.
\par 15 وَبَابُ الْعَيْنِ رَمَّمَهُ شَلُّونُ بْنُ كَلْحُوزَةَ رَئِيسُ دَائِرَةِ الْمِصْفَاةِ. هُوَ بَنَاهُ وَسَقَفَهُ وَأَقَامَ مَصَارِيعَهُ وَأَقْفَالَهُ وَعَوَارِضَهُ وَسُورَ بِرْكَةِ سِلُوَامٍ عِنْدَ جُنَيْنَةِ الْمَلِكِ إِلَى الدَّرَجِ النَّازِلِ مِنْ مَدِينَةِ دَاوُدَ.
\par 16 وَبَعْدَهُ رَمَّمَ نَحَمْيَا بْنُ عَزْبُوقَ رَئِيسُ نِصْفِ دَائِرَةِ بَيْتِ صُورَ إِلَى مُقَابَِلِ قُبُورِ دَاوُدَ وَإِلَى الْبِرْكَةِ الْمَصْنُوعَةِ وَإِلَى بَيْتِ الْجَبَابِرَةِ.
\par 17 وَبَعْدَهُ رَمَّمَ اللاَّوِيُّونَ رَحُومُ بْنُ بَانِي وَبِجَانِبِهِ رَمَّمَ حَشَبْيَا رَئِيسُ نِصْفِ دَائِرَةِ قَعِيلَةَ فِي قِسْمِهِ.
\par 18 وَبَعْدَهُ رَمَّمَ إِخْوَتُهُمْ بَوَّايُ بْنُ حِينَادَادَ رَئِيسُ نِصْفِ دَائِرَةِ قَعِيلَةَ.
\par 19 وَرَمَّمَ بِجَانِبِهِ عَازَِرُ بْنُ يَشُوعَ رَئِيسُ الْمِصْفَاةِ قِسْماً ثَانِياً مِنْ مُقَابِلِ مَصْعَدِ بَيْتِ السِّلاَحِ عِنْدَ الزَّاوِيَةِ.
\par 20 وَبَعْدَهُ رَمَّمَ بِعَزْمٍ بَارُوخُ بْنُ زَبَّايَ قِسْماً ثَانِياً مِنَ الزَّاوِيَةِ إِلَى مَدْخَلِ بَيْتِ أَلْيَاشِيبَ الْكَاهِنِ الْعَظِيمِ.
\par 21 وَبَعْدَهُ رَمَّمَ مَرِيمُوثُ بْنُ أُورِيَّا بْنِ هَقُّوصَ قِسْماً ثَانِياً مِنْ مَدْخَلِ بَيْتِ أَلْيَاشِيبَ إِلَى نِهَايَةِ بَيْتِ أَلْيَاشِيبَ.
\par 22 وَبَعْدَهُ رَمَّمَ الْكَهَنَةُ أَهْلُ الْغَوْرِ.
\par 23 وَبَعْدَهُمْ رَمَّمَ بِنْيَامِينُ وَحَشُّوبُ مُقَابِلَ بَيْتِهِمَا. وَبَعْدَهُمَا رَمَّمَ عَزَرْيَا بْنُ مَعْسِيَّا بْنِ عَنَنْيَا بِجَانِبِ بَيْتِهِ.
\par 24 وَبَعْدَهُ رَمَّمَ بِنُّويُ بْنُ حِينَادَادَ قِسْماً ثَانِياً مِنْ بَيْتِ عَزَرْيَا إِلَى الزَّاوِيَةِ وَإِلَى الْعَطْفَةِ.
\par 25 وَفَالاَلُ بْنُ أُوزَايَ مِنْ مُقَابَِلِ الزَّاوِيَةِ وَالْبُرْجِ الَّذِي هُوَ خَارِجَ بَيْتِ الْمَلِكِ الأَعْلَى الَّذِي لِدَارِ السِّجْنِ. وَبَعْدَهُ فَدَايَا بْنُ فَرْعُوشَ.
\par 26 وَكَانَ النَّثِينِيمُ سَاكِنِينَ فِي الأَكَمَةِ إِلَى مُقَابِلِ بَابِ الْمَاءِ لِجِهَةِ الشَّرْقِ وَالْبُرْجِ الْخَارِجِيِّ.
\par 27 وَبَعْدَهُمْ رَمَّمَ التَّقُوعِيُّونَ قِسْماً ثَانِياً مِنْ مُقَابِلِ الْبُرْجِ الْكَبِيرِ الْخَارِجِيِّ إِلَى سُورِ الأَكَمَةِ.
\par 28 وَمَا فَوْقَ بَابِ الْخَيْلِ رَمَّمَهُ الْكَهَنَةُ كُلُّ وَاحِدٍ مُقَابِلَ بَيْتِهِ.
\par 29 وَبَعْدَهُمْ رَمَّمَ صَادُوقُ بْنُ إِمِّيرَ مُقَابِلَ بَيْتِهِ. وَبَعْدَهُ رَمَّمَ شَمَعْيَا بْنُ شَكَنْيَا حَارِسُ بَابِ الشَّرْقِ.
\par 30 وَبَعْدَهُ رَمَّمَ حَنَنْيَا بْنُ شَلَمْيَا وَحَانُونُ بْنُ صَالاَفَ السَّادِسُ قِسْماً ثَانِياً. وَبَعْدَهُ رَمَّمَ مَشُلاَّمُ بْنُ بَرَخْيَا مُقَابَِلَ مِخْدَعِهِ.
\par 31 وَبَعْدَهُ رَمَّمَ مَلْكِيَّا ابْنُ الصَّائِغِ إِلَى بَيْتِ النَّثِينِيمِ وَالتُّجَّارِ مُقَابِلَ بَابِ الْعَدِّ إِلَى مَصْعَدِ الْعَطْفَةِ.
\par 32 وَمَا بَيْنَ مَصْعَدِ الْعَطْفَةِ إِلَى بَابِ الضَّأْنِ رَمَّمَهُ الصَّيَّاغُونَ وَالتُّجَّارُ.

\chapter{4}

\par 1 وَلَمَّا سَمِعَ سَنْبَلَّطُ أَنَّنَا آخِذُونَ فِي بِنَاءِ السُّورِ غَضِبَ وَاغْتَاظَ كَثِيراً وَهَزَأَ بِالْيَهُودِ
\par 2 وَقَالَ أَمَامَ إِخْوَتِهِ وَجَيْشِ السَّامِرَةِ: [مَاذَا يَعْمَلُ الْيَهُودُ الضُّعَفَاءُ؟ هَلْ يَتْرُكُونَهُمْ؟ هَلْ يَذْبَحُونَ؟ هَلْ يُكْمِلُونَ فِي يَوْمٍ؟ هَلْ يُحْيُونَ الْحِجَارَةَ مِنْ كُوَمِ التُّرَابِ وَهِيَ مُحْرَقَةٌ؟]
\par 3 وَكَانَ طُوبِيَّا الْعَمُّونِيُّ بِجَانِبِهِ فَقَالَ: [إِنَّ مَا يَبْنُونَهُ إِذَا صَعِدَ ثَعْلَبٌ فَإِنَّهُ يَهْدِمُ حِجَارَةَ حَائِطِهِمِ].
\par 4 اسْمَعْ يَا إِلَهَنَا لأَنَّنَا قَدْ صِرْنَا احْتِقَاراً وَرُدَّ تَعْيِيرَهُمْ عَلَى رُؤُوسِهِمْ وَاجْعَلْهُمْ نَهْباً فِي أَرْضِ السَّبْيِ
\par 5 وَلاَ تَسْتُرْ ذُنُوبَهُمْ وَلاَ تُمْحَ خَطِيَّتُهُمْ مِنْ أَمَامِكَ لأَنَّهُمْ أَغْضَبُوكَ أَمَامَ الْبَانِينَ.
\par 6 فَبَنَيْنَا السُّورَ وَاتَّصَلَ كُلُّ السُّورِ إِلَى نِصْفِهِ وَكَانَ لِلشَّعْبِ قَلْبٌ فِي الْعَمَلِ.
\par 7 وَلَمَّا سَمِعَ سَنْبَلَّطُ وَطُوبِيَّا وَالْعَرَبُ وَالْعَمُّونِيُّونَ وَالأَشْدُودِيُّونَ أَنَّ أَسْوَارَ أُورُشَلِيمَ قَدْ رُمِّمَتْ وَالثُّغَرَ ابْتَدَأَتْ تُسَدُّ غَضِبُوا جِدّاً.
\par 8 وَتَآمَرُوا جَمِيعُهُمْ مَعاً أَنْ يَأْتُوا وَيُحَارِبُوا أُورُشَلِيمَ وَيَعْمَلُوا بِهَا ضَرَراً.
\par 9 فَصَلَّيْنَا إِلَى إِلَهِنَا وَأَقَمْنَا حُرَّاساً ضِدَّهُمْ نَهَاراً وَلَيْلاً بِسَبَبِهِمْ.
\par 10 وَقَالَ يَهُوذَا: [قَدْ ضَعُفَتْ قُوَّةُ الْحَمَّالِينَ وَالتُّرَابُ كَثِيرٌ وَنَحْنُ لاَ نَقْدِرُ أَنْ نَبْنِيَ السُّورَ].
\par 11 وَقَالَ أَعْدَاؤُنَا: [لاَ يَعْلَمُونَ وَلاَ يَرُونَ حَتَّى نَدْخُلَ إِلَى وَسَطِهِمْ وَنَقْتُلَهُمْ وَنُوقِفَ الْعَمَلَ].
\par 12 وَلَمَّا جَاءَ الْيَهُودُ السَّاكِنُونَ بِجَانِبِهِمْ قَالُوا لَنَا عَشَرَ مَرَّاتٍ: [مِنْ جَمِيعِ الأَمَاكِنِ الَّتِي مِنْهَا رَجَعُوا سَيَأْتُونَ عَلَيْنَا].
\par 13 فَأَوْقَفْتُ الشَّعْبَ مِنْ أَسْفَلِ الْمَوْضِعِ وَرَاءَ السُّورِ وَعَلَى الْقِمَمِ أَوْقَفْتُهُمْ حَسَبَ عَشَائِرِهِمْ بِسُيُوفِهِمْ وَرِمَاحِهِمْ وَقِسِيِّهِمْ.
\par 14 وَنَظَرْتُ وَقُمْتُ وَقُلْتُ لِلْعُظَمَاءِ وَالْوُلاَةِ وَلِبَقِيَّةِ الشَّعْبِ: [لاَ تَخَافُوهُمْ بَلِ اذْكُرُوا السَّيِّدَ الْعَظِيمَ الْمَرْهُوبَ وَحَارِبُوا مِنْ أَجْلِ إِخْوَتِكُمْ وَبَنِيكُمْ وَبَنَاتِكُمْ وَنِسَائِكُمْ وَبُيُوتِكُمْ].
\par 15 وَلَمَّا سَمِعَ أَعْدَاؤُنَا أَنَّنَا قَدْ عَرَفْنَا وَأَبْطَلَ اللَّهُ مَشُورَتَهُمْ رَجَعْنَا كُلُّنَا إِلَى السُّورِ كُلُّ وَاحِدٍ إِلَى شُغْلِهِ.
\par 16 وَمِنْ ذَلِكَ الْيَوْمِ كَانَ نِصْفُ غِلْمَانِي يَشْتَغِلُونَ فِي الْعَمَلِ وَنِصْفُهُمْ يُمْسِكُونَ الرِّمَاحَ وَالأَتْرَاسَ وَالْقِسِيَّ وَالدُّرُوعَ. وَالرُّؤَسَاءُ وَرَاءَ كُلِّ بَيْتِ يَهُوذَا.
\par 17 الْبَانُونَ عَلَى السُّورِ بَنُوا وَحَامِلُو الأَحْمَالِ حَمَلُوا. بِالْيَدِ الْوَاحِدَةِ يَعْمَلُونَ الْعَمَلَ وَبِالأُخْرَى يُمْسِكُونَ السِّلاَحَ.
\par 18 وَكَانَ الْبَانُونَ يَبْنُونَ وَسَيْفُ كُلُّ وَاحِدٍ مَرْبُوطٌ عَلَى جَنْبِهِ وَكَانَ النَّافِخُ بِالْبُوقِ بِجَانِبِي.
\par 19 فَقُلْتُ لِلْعُظَمَاءِ وَالْوُلاَةِ وَلِبَقِيَّةِ الشَّعْبِ: [الْعَمَلُ كَثِيرٌ وَمُتَّسِعٌ وَنَحْنُ مُتَفَرِّقُونَ عَلَى السُّورِ وَبَعِيدُونَ بَعْضُنَا عَنْ بَعْضٍ.
\par 20 فَالْمَكَانُ الَّذِي تَسْمَعُونَ مِنْهُ صَوْتَ الْبُوقِ هُنَاكَ تَجْتَمِعُونَ إِلَيْنَا. إِلَهُنَا يُحَارِبُ عَنَّا].
\par 21 فَكُنَّا نَحْنُ نَعْمَلُ الْعَمَلَ وَكَانَ نِصْفُهُمْ يُمْسِكُونَ الرِّمَاحَ مِنْ طُلُوعِ الْفَجْرِ إِلَى ظُهُورِ النُّجُومِ.
\par 22 وَقُلْتُ فِي ذَلِكَ الْوَقْتِ أَيْضاً لِلشَّعْبِ: [لِيَبِتْ كُلُّ وَاحِدٍ مَعَ غُلاَمِهِ فِي وَسَطِ أُورُشَلِيمَ لِيَكُونُوا لَنَا حُرَّاساً فِي اللَّيْلِ وَلِلْعَمَلِ فِي النَّهَارِ].
\par 23 وَلَمْ أَكُنْ أَنَا وَلاَ إِخْوَتِي وَلاَ غِلْمَانِي وَلاَ الْحُرَّاسُ الَّذِينَ وَرَائِي نَخْلَعُ ثِيَابَنَا. كَانَ كُلُّ وَاحِدٍ يَذْهَبُ بِسِلاَحِهِ إِلَى الْمَاءِ.

\chapter{5}

\par 1 وَكَانَ صُرَاخُ الشَّعْبِ وَنِسَائِهِمْ عَظِيماً عَلَى إِخْوَتِهِمِ الْيَهُودِ.
\par 2 وَكَانَ مَنْ يَقُولُ: [نَحْنُ وَبَنُونَا وَبَنَاتُنَا كَثِيرُونَ. دَعْنَا نَأْخُذْ قَمْحاً فَنَأْكُلَ وَنَحْيَا!]
\par 3 وَكَانَ مَنْ يَقُولُ: [حُقُولُنَا وَكُرُومُنَا وَبُيُوتُنَا نَحْنُ رَاهِنُوهَا حَتَّى نَأْخُذَ قَمْحاً فِي الْجُوعِ!]
\par 4 وَكَانَ مَنْ يَقُولُ: [قَدِ اسْتَقْرَضْنَا فِضَّةً لِخَرَاجِ الْمَلِكِ عَلَى حُقُولِنَا وَكُرُومِنَا.
\par 5 وَالآنَ لَحْمُنَا كَلَحْمِ إِخْوَتِنَا وَبَنُونَا كَبَنِيهِمْ وَهَا نَحْنُ نُخْضِعُ بَنِينَا وَبَنَاتِنَا عَبِيداً وَيُوجَدُ مِنْ بَنَاتِنَا مُسْتَعْبَدَاتٌ وَلَيْسَ شَيْءٌ فِي طَاقَةِ يَدِنَا وَحُقُولُنَا وَكُرُومُنَا لِلآخَرِينَ].
\par 6 فَغَضِبْتُ جِدّاً حِينَ سَمِعْتُ صُرَاخَهُمْ وَهَذَا الْكَلاَمَ.
\par 7 فَشَاوَرْتُ قَلْبِي فِيَّ وَبَكَّتُّ الْعُظَمَاءَ وَالْوُلاَةَ وَقُلْتُ لَهُمْ: [إِنَّكُمْ تَأْخُذُونَ الرِّبَا كُلُّ وَاحِدٍ مِنْ أَخِيهِ]. وَأَقَمْتُ عَلَيْهِمْ جَمَاعَةً عَظِيمَةً.
\par 8 وَقُلْتُ لَهُمْ: [نَحْنُ اشْتَرَيْنَا إِخْوَتَنَا الْيَهُودَ الَّذِينَ بِيعُوا لِلأُمَمِ حَسَبَ طَاقَتِنَا. وَأَنْتُمْ أَيْضاً تَبِيعُونَ إِخْوَتَكُمْ فَيُبَاعُونَ لَنَا]. فَسَكَتُوا وَلَمْ يَجِدُوا جَوَاباً.
\par 9 وَقُلْتُ: [لَيْسَ حَسَناً الأَمْرُ الَّذِي تَعْمَلُونَهُ. أَمَا تَسِيرُونَ بِخَوْفِ إِلَهِنَا بِسَبَبِ تَعْيِيرِ الأُمَمِ أَعْدَائِنَا!
\par 10 وَأَنَا أَيْضاً وَإِخْوَتِي وَغِلْمَانِي أَقْرَضْنَاهُمْ فِضَّةً وَقَمْحاً. فَلْنَتْرُكْ هَذَا الرِّبَا.
\par 11 رُدُّوا لَهُمْ هَذَا الْيَوْمَ حُقُولَهُمْ وَكُرُومَهُمْ وَزَيْتُونَهُمْ وَبُيُوتَهُمْ وَالْجُزْءَ مِنْ مِئَةِ الْفِضَّةِ وَالْقَمْحِ وَالْخَمْرِ وَالزَّيْتِ الَّذِي تَأْخُذُونَهُ مِنْهُمْ رِباً].
\par 12 فَقَالُوا: [نَرُدُّ وَلاَ نَطْلُبُ مِنْهُمْ. هَكَذَا نَفْعَلُ كَمَا تَقُولُ]. فَدَعَوْتُ الْكَهَنَةَ وَاسْتَحْلَفْتُهُمْ أَنْ يَعْمَلُوا حَسَبَ هَذَا الْكَلاَمِ.
\par 13 ثُمَّ نَفَضْتُ حِجِْرِي وَقُلْتُ: [هَكَذَا يَنْفُضُ اللَّهُ كُلَّ إِنْسَانٍ لاَ يُقِيمُ هَذَا الْكَلاَمَ مِنْ بَيْتِهِ وَمِنْ تَعَبِهِ وَهَكَذَا يَكُونُ مَنْفُوضاً وَفَارِغاً]. فَقَالَ كُلُّ الْجَمَاعَةِ: [آمِينَ!] وَسَبَّحُوا الرَّبَّ. وَعَمِلَ الشَّعْبُ حَسَبَ هَذَا الْكَلاَمِ.
\par 14 وَأَيْضاً مِنَ الْيَوْمِ الَّذِي أُوصِيتُ فِيهِ أَنْ أَكُونَ وَالِيَهُمْ فِي أَرْضِ يَهُوذَا مِنَ السَّنَةِ الْعِشْرِينَ إِلَى السَّنَةِ الثَّانِيَةِ وَالثَّلاَثِينَ لأَرْتَحْشَسْتَا الْمَلِكِ اثْنَتَيْ عَشَرَةَ سَنَةً لَمْ آكُلْ أَنَا وَلاَ إِخْوَتِي خُبْزَ الْوَالِي.
\par 15 وَلَكِنِ الْوُلاَةُ الأَوَّلُونَ الَّذِينَ قَبْلِي ثَقَّلُوا عَلَى الشَّعْبِ وَأَخَذُوا مِنْهُمْ خُبْزاً وَخَمْراً فَضْلاً عَنْ أَرْبَعِينَ شَاقِلاً مِنَ الْفِضَّةِ حَتَّى إِنَّ غِلْمَانَهُمْ تَسَلَّطُوا عَلَى الشَّعْبِ. وَأَمَّا أَنَا فَلَمْ أَفْعَلْ هَكَذَا مِنْ أَجْلِ خَوْفِ اللَّهِ.
\par 16 وَتَمَسَّكْتُ أَيْضاً بِشُغْلِ هَذَا السُّورِ. وَلَمْ أَشْتَرِ حَقْلاً. وَكَانَ جَمِيعُ غِلْمَانِي مُجْتَمِعِينَ هُنَاكَ عَلَى الْعَمَلِ.
\par 17 وَكَانَ عَلَى مَائِدَتِي مِنَ الْيَهُودِ وَالْوُلاَةِ مِئَةٌ وَخَمْسُونَ رَجُلاً فَضْلاً عَنِ الآتِينَ إِلَيْنَا مِنَ الأُمَمِ الَّذِينَ حَوْلَنَا.
\par 18 وَكَانَ مَا يُعْمَلُ لِيَوْمٍ وَاحِدٍ ثَوْراً وَسِتَّةَ خِرَافٍ مُخْتَارَةٍ. وَكَانَ يُعْمَلُ لِي طُيُورٌ وَفِي كُلِّ عَشَرَةِ أَيَّامٍ كُلُّ نَوْعٍ مِنَ الْخَمْرِ بِكَثْرَةٍ. وَمَعَ هَذَا لَمْ أَطْلُبْ خُبْزَ الْوَالِي لأَنَّ الْعُبُودِيَّةَ كَانَتْ ثَقِيلَةً عَلَى هَذَا الشَّعْبِ.
\par 19 اذْكُرْ لِي يَا إِلَهِي لِلْخَيْرِ كُلَّ مَا عَمِلْتُ لِهَذَا الشَّعْبِ.

\chapter{6}

\par 1 وَلَمَّا سَمِعَ سَنْبَلَّطُ وَطُوبِيَّا وَجَشَمٌ الْعَرَبِيُّ وَبَقِيَّةُ أَعْدَائِنَا أَنِّي قَدْ بَنَيْتُ السُّورَ وَلَمْ تَبْقَ فِيهِ ثُغْرَةٌ (عَلَى أَنِّي لَمْ أَكُنْ إِلَى ذَلِكَ الْوَقْتِ قَدْ أَقَمْتُ مَصَارِيعَ لِلأَبْوَابِ)
\par 2 أَرْسَلَ سَنْبَلَّطُ وَجَشَمٌ إِلَيَّ قَائِلَيْنِ: [هَلُمَّ نَجْتَمِعُ مَعاً فِي الْقُرَى فِي بُقْعَةِ أُونُو]. وَكَانَا يُفَكِّرَانِ أَنْ يَعْمَلاَ بِي شَرّاً.
\par 3 فَأَرْسَلْتُ إِلَيْهِمَا رُسُلاً قَائِلاً: [إِنِّي أَنَا عَامِلٌ عَمَلاً عَظِيماً فَلاَ أَقْدُِرُ أَنْ أَنْزِلَ. لِمَاذَا يَبْطُلُ الْعَمَلُ بَيْنَمَا أَتْرُكُهُ وَأَنْزِلُ إِلَيْكُمَا؟]
\par 4 وَأَرْسَلاَ إِلَيَّ بِمِثْلِ هَذَا الْكَلاَمِ أَرْبَعَ مَرَّاتٍ وَجَاوَبْتُهُمَا بِمِثْلِ هَذَا الْجَوَابِ.
\par 5 فَأَرْسَلَ إِلَيَّ سَنْبَلَّطُ بِمِثْلِ هَذَا الْكَلاَمِ مَرَّةً خَامِسَةً مَعَ غُلاَمِهِ بِرِسَالَةٍ مَنْشُورَةٍ بِيَدِهِ مَكْتُوبٌ فِيهَا:
\par 6 [قَدْ سُمِعَ بَيْنَ الأُمَمِ وَجَشَمٌ يَقُولُ إِنَّكَ أَنْتَ وَالْيَهُودُ تُفَكِّرُونَ أَنْ تَتَمَرَّدُوا لِذَلِكَ أَنْتَ تَبْنِي السُّورَ لِتَكُونَ لَهُمْ مَلِكاً حَسَبَ هَذِهِ الأُمُورِ.
\par 7 وَقَدْ أَقَمْتَ أَيْضاً أَنْبِيَاءَ لِيُنَادُوا بِكَ فِي أُورُشَلِيمَ قَائِلِينَ: فِي يَهُوذَا مَلِكٌ. وَالآنَ يُخْبَرُ الْمَلِكُ بِهَذَا الْكَلاَمِ. فَهَلُمَّ الآنَ نَتَشَاوَرُ مَعاً].
\par 8 فَأَرْسَلْتُ إِلَيْهِ قَائِلاً: [لاَ يَكُونُ مِثْلُ هَذَا الْكَلاَمِ الَّذِي تَقُولُهُ بَلْ إِنَّمَا أَنْتَ مُخْتَلِقُهُ مِنْ قَلْبِكَ].
\par 9 لأَنَّهُمْ كَانُوا جَمِيعاً يُخِيفُونَنَا قَائِلِينَ: [قَدِ ارْتَخَتْ أَيْدِيهِمْ عَنِ الْعَمَلِ فَلاَ يُعْمَلُ]. فَالآنَ يَا إِلَهِي شَدِّدْ يَدَيَّ.
\par 10 وَدَخَلْتُ بَيْتَ شَمَعْيَا بْنِ دَلاَيَا بْنِ مَهِيطَبْئِيلَ وَهُوَ مُغْلَقٌ فَقَالَ: [لِنَجْتَمِعْ إِلَى بَيْتِ اللَّهِ إِلَى وَسَطِ الْهَيْكَلِ وَنُقْفِلْ أَبْوَابَ الْهَيْكَلِ لأَنَّهُمْ يَأْتُونَ لِيَقْتُلُوكَ. فِي اللَّيْلِ يَأْتُونَ لِيَقْتُلُوكَ].
\par 11 فَقُلْتُ: [أَرَجُلٌ مِثْلِي يَهْرُبُ؟ وَمَنْ مِثْلِي يَدْخُلُ الْهَيْكَلَ فَيَحْيَا! لاَ أَدْخُلُ].
\par 12 فَتَحَقَّقْتُ وَهُوَذَا لَمْ يُرْسِلْهُ اللَّهُ لأَنَّهُ تَكَلَّمَ بِالنُّبُوَّةِ عَلَيَّ وَطُوبِيَّا وَسَنْبَلَّطُ قَدِ اسْتَأْجَرَاهُ.
\par 13 لأَجْلِ هَذَا قَدِ اسْتُؤْجِرَ لأَخَافَ وَأَفْعَلَ هَكَذَا وَأُخْطِئَ فَيَكُونَ لَهُمَا خَبَرٌ رَدِيءٌ لِيُعَيِّرَانِي.
\par 14 اذْكُرْ يَا إِلَهِي طُوبِيَّا وَسَنْبَلَّطَ حَسَبَ أَعْمَالِهِمَا هَذِهِ وَنُوعَدْيَةَ النَّبِيَّةَ وَبَاقِيَ الأَنْبِيَاءِ الَّذِينَ يُخِيفُونَنِي.
\par 15 وَكَمِلَ السُّورُ فِي الْخَامِسِ وَالْعِشْرِينَ مِنْ أَيْلُولَ فِي اثْنَيْنِ وَخَمْسِينَ يَوْماً.
\par 16 وَلَمَّا سَمِعَ كُلُّ أَعْدَائِنَا وَرَأَى جَمِيعُ الأُمَمِ الَّذِينَ حَوَالَيْنَا سَقَطُوا كَثِيراً فِي أَعْيُنِ أَنْفُسِهِمْ وَعَلِمُوا أَنَّهُ مِنْ قِبَلِ إِلَهِنَا عُمِلَ هَذَا الْعَمَلُ.
\par 17 وَأَيْضاً فِي تِلْكَ الأَيَّامِ أَكْثَرَ عُظَمَاءُ يَهُوذَا تَوَارُدَ رَسَائِلِهِمْ عَلَى طُوبِيَّا وَمِنْ عِنْدِ طُوبِيَّا أَتَتِ الرَّسَائِلُ إِلَيْهِمْ.
\par 18 لأَنَّ كَثِيرِينَ فِي يَهُوذَا كَانُوا أَصْحَابَ حِلْفٍ لَهُ لأَنَّهُ صِهْرُ شَكَنْيَا بْنِ آرَحَ وَيَهُوحَانَانُ ابْنُهُ أَخَذَ بِنْتَ مَشُلاَّمَ بْنِ بَرَخْيَا.
\par 19 وَكَانُوا أَيْضاً يُخْبِرُونَ أَمَامِي بِحَسَنَاتِهِ وَكَانُوا يُبَلِّغُونَ كَلاَمِي إِلَيْهِ. وَأَرْسَلَ طُوبِيَّا رَسَائِلَ لِيُخَوِّفَنِي.

\chapter{7}

\par 1 وَلَمَّا بُنِيَ السُّورُ وَأَقَمْتُ الْمَصَارِيعَ وَتَرَتَّبَ الْبَوَّابُونَ وَالْمُغَنُّونَ وَاللاَّوِيُّونَ
\par 2 أَقَمْتُ حَنَانِيَ أَخِي وَحَنَنْيَا رَئِيسَ الْقَصْرِ عَلَى أُورُشَلِيمَ لأَنَّهُ كَانَ رَجُلاً أَمِيناً يَخَافُ اللَّهَ أَكْثَرَ مِنْ كَثِيرِينَ.
\par 3 وَقُلْتُ لَهُمَا: [لاَ تُفْتَحْ أَبْوَابُ أُورُشَلِيمَ حَتَّى تَحْمَى الشَّمْسُ. وَمَا دَامُوا وُقُوفاً فَلْيُغْلِقُوا الْمَصَارِيعَ وَيُقْفِلُوهَا. وَأُقِيمَ حِرَاسَاتٌ مِنْ سُكَّانِ أُورُشَلِيمَ كُلُّ وَاحِدٍ عَلَى حِرَاسَتِهِ وَكُلُّ وَاحِدٍ مُقَابَِلَ بَيْتِهِ].
\par 4 وَكَانَتِ الْمَدِينَةُ وَاسِعَةَ الْجَنَابِ وَعَظِيمَةً وَالشَّعْبُ قَلِيلاً فِي وَسَطِهَا وَلَمْ تَكُنِ الْبُيُوتُ قَدْ بُنِيَتْ.
\par 5 فَأَلْهَمَنِي إِلَهِي أَنْ أَجْمَعَ الْعُظَمَاءَ وَالْوُلاَةَ وَالشَّعْبَ لأَجْلِ الاِنْتِسَابِ. فَوَجَدْتُ سِفْرَ انْتِسَابِ الَّذِينَ صَعِدُوا أَوَّلاً وَوَجَدْتُ مَكْتُوباً فِيهِ:
\par 6 هَؤُلاَءِ هُمْ بَنُو الْكُورَةِ الصَّاعِدُونَ مِنْ سَبْيِ الْمَسْبِيِّينَ الَّذِينَ سَبَاهُمْ نَبُوخَذْنَصَّرُ مَلِكُ بَابَِلَ وَرَجَعُوا إِلَى أُورُشَلِيمَ وَيَهُوذَا كُلُّ وَاحِدٍ إِلَى مَدِينَتِهِ.
\par 7 الَّذِينَ جَاءُوا مَعَ زَرُبَّابَِلَ: يَشُوعُ نَحَمْيَا عَزَرْيَا رَعَمْيَا نَحَمَانِي مُرْدَخَايُ بِلْشَانُ مِسْفَارَثُ بِغْوَايُ نَحُومُ وَبَعْنَةُ. عَدَدُ رِجَالِ شَعْبِ إِسْرَائِيلَ.
\par 8 بَنُو فَرْعُوشَ أَلْفَانِ وَمِئَةٌ وَاثْنَانِ وَسَبْعُونَ.
\par 9 بَنُو شَفَطْيَا ثَلاَثُ مِئَةٍ وَاثْنَانِ وَسَبْعُونَ.
\par 10 بَنُو آرَحَ سِتُّ مِئَةٍ وَاثْنَانِ وَخَمْسُونَ.
\par 11 بَنُو فَحَثَ مُوآبَ مِنْ بَنِي يَشُوعَ وَيُوآبَ أَلْفَانِ وَثَمَانُ مِئَةٍ وَثَمَانِيَةَ عَشَرَ.
\par 12 بَنُو عِيلاَمَ أَلْفٌ وَمِئَتَانِ وَأَرْبَعَةٌ وَخَمْسُونَ.
\par 13 بَنُو زَتُّو ثَمَانُ مِئَةٍ وَخَمْسَةٌ وَأَرْبَعُونَ.
\par 14 بَنُو زَكَّايَ سَبْعُ مِئَةٍ وَسِتُّونَ.
\par 15 بَنُو بَنُّويَ سِتُّ مِئَةٍ وَثَمَانِيَةٌ وَأَرْبَعُونَ.
\par 16 بَنُو بَابَايَ سِتُّ مِئَةٍ وَثَمَانِيَةٌ وَعِشْرُونَ.
\par 17 بَنُو عَزْجَدَ أَلْفَانِ وَثَلاَثُ مِئَةٍ وَاثْنَانِ وَعِشْرُونَ.
\par 18 بَنُو أَدُونِيقَامَ سِتُّ مِئَةٍ وَسَبْعَةٌ وَسِتُّونَ.
\par 19 بَنُو بِغْوَايَ أَلْفَانِ وَسَبْعَةٌ وَسِتُّونَ.
\par 20 بَنُو عَادِينَ سِتُّ مِئَةٍ وَخَمْسَةٌ وَخَمْسُونَ.
\par 21 بَنُو أَطِّيرَ لِحَزَقِيَّا ثَمَانِيَةٌ وَتِسْعُونَ.
\par 22 بَنُو حَشُومَ ثَلاَثُ مِئَةٍ وَثَمَانِيَةٌ وَعِشْرُونَ.
\par 23 بَنُو بِيصَايَ ثَلاَثُ مِئَةٍ وَأَرْبَعَةٌ وَعِشْرُونَ.
\par 24 بَنُو حَارِيفَ مِئَةٌ وَاثْنَا عَشَرَ.
\par 25 بَنُو جِبْعُونَ خَمْسَةٌ وَتِسْعُونَ.
\par 26 رِجَالُ بَيْتَِ لَحْمٍَ وَنَطُوفَةَ مِئَةٌ وَثَمَانيَةٌ وَثَمَانُونَ.
\par 27 رِجَالُ عَنَاثُوثَ مِئَةٌ وَثَمَانِيَةٌ وَعِشْرُونَ.
\par 28 رِجَالُ بَيْتِ عَزْمُوتَ اثْنَانِ وَأَرْبَعُونَ.
\par 29 رِجَالُ قَرْيَةِ يَعَارِيمَ كَفِيرَةَ وَبَئِيرُوتَ سَبْعُ مِئَةٍ وَثَلاَثةٌ وَأَرْبَعُونَ.
\par 30 رِجَالُ الرَّامَةِ وَجَبَعٍَ سِتُّ مِئَةٍ وَوَاحِدٌ وَعِشْرُونَ.
\par 31 رِجَالُ مِخْمَاسَ مِئَةٌ وَاثْنَانِ وَعِشْرُونَ.
\par 32 رِجَالُ بَيْتِ إِيلَ وعَايَ مِئَةٌ وَثَلاَثةٌ وَعِشْرُونَ.
\par 33 رِجَالُ نَبُو الأُخْرَى اثْنَانِ وَخَمْسُونَ.
\par 34 بَنُو عِيلاَمَ الآخَرِ أَلْفٌ وَمِئَتَانِ وَأَرْبَعَةٌ وَخَمْسُونَ.
\par 35 بَنُو حَارِيمَ ثَلاَثُ مِئَةٍ وَعِشْرُونَ.
\par 36 بَنُو أَرِيحَا ثَلاَثُ مِئَةٍ وَخَمْسَةٌ وَأَرْبَعُونَ.
\par 37 بَنُو لُودٍَ بَنُو حَادِيدَ وَأُونُو سَبْعُ مِئَةٍ وَوَاحِدٌ وَعِشْرُونَ.
\par 38 بَنُو سَنَاءَةَ ثَلاَثَةُ آلاَفٍ وَتِسْعُ مِئَةٍ وَثَلاَثُونَ.
\par 39 أَمَّا الْكَهَنَةَ فَبَنُو يَدْعِيَا مِنْ بَيْتِ يَشُوعَ تِسْعُ مِئَةٍ وَثَلاَثَةٌ وَسَبْعُونَ.
\par 40 بَنُو إِمِّيرَ أَلْفٌ وَاثْنَانِ وَخَمْسُونَ.
\par 41 بَنُو فَشْحُورَ أَلْفٌ وَمِئَتَانِ وَسَبْعَةٌ وَأَرْبَعُونَ.
\par 42 بَنُو حَارِيمَ أَلْفٌ وَسَبْعَةَ عَشَرَ.
\par 43 أَمَّا اللاَّوِيُّونَ فَبَنُو يَشُوعَ لِقَدْمِيئِيلَ مِنْ بَنِي هُودُويَا أَرْبَعَةٌ وَسَبْعُونَ.
\par 44 اَلْمُغَنُّونَ بَنُو آسَافَ مِئَةٌ وَثَمَانِيَةٌ وَأَرْبَعُونَ.
\par 45 اَلْبَوَّابُونَ بَنُو شَلُّومَ بَنُو أَطِيرَ بَنُو طَلْمُونَ بَنُو عَقُّوبَ بَنُو حَطِيطَا بَنُو شُوبَايَ مِئَةٌ وَثَمَانِيَةٌ وَثَلاَثُونَ.
\par 46 اَلنَّثِينِيمُ بَنُو صِيحَا بَنُو حَسُوفَا بَنُو طَبَاعُوتَ
\par 47 بَنُو قِيرُوسَ بَنُو سِيعَا بَنُو فَادُونَ
\par 48 وَبَنُو لَبَانَةَ وَبَنُو حَجَابَا بَنُو سَلْمَايَ
\par 49 بَنُو حَانَانَ بَنُو جَدِيلَ بَنُو جَاحَرَ
\par 50 بَنُو رَآيَا بَنُو رَصِينَ وَبَنُو نَقُودَا
\par 51 بَنُو جَزَامَ بَنُو عَزَا بَنُو فَاسِيحَ
\par 52 بَنُو بِيسَايَ بَنُو مَعُونِيمَ بَنُو نَفِيشَسِيمَ
\par 53 بَنُو بَقْبُوقَ بَنُو حَقُوفَا بَنُو حَرْحُورَ
\par 54 بَنُو بَصْلِيتَ بَنُو مَحِيدَا بَنُو حَرْشَا
\par 55 بَنُو بَرْقُوسَ بَنُو سِيسَرَا بَنُو تَامَحَ
\par 56 بَنُو نَصِيحَ بَنُو حَطِيفَا.
\par 57 بَنُو عَبِيدِ سُلَيْمَانَ بَنُو سُوطَايَ بَنُو سُوفَرَثَ بَنُو فَرِيدَا
\par 58 بَنُو يَعْلاَ بَنُو دَرْقُونَ بَنُو جَدِّيلَ
\par 59 بَنُو شَفَطْيَا بَنُو حَطِّيلَ بَنُو فُوخَرَةِ الظِّبَاءِ بَنُو آمُونَ.
\par 60 كُلُّ النَّثِينِيمِ وَبَنِي عَبِيدِ سُلَيْمَانَ ثَلاَثُ مِئَةٍ وَاثْنَانِ وَتِسْعُونَ.
\par 61 وَهَؤُلاَءِ هُمُ الَّذِينَ صَعِدُوا مِنْ تَلِّ مِلْحٍ وَتَلِّ حَرْشَا كَرُوبُ وَأَدُونُ وَإِمِّيرُ وَلَمْ يَسْتَطِيعُوا أَنْ يُبَيِّنُوا بُيُوتَ آبَائِهِمْ وَنَسْلَهُمْ هَلْ هُمْ مِنْ إِسْرَائِيلَ:
\par 62 بَنُو دَلاَيَا بَنُو طُوبِيَّا بَنُو نَقُودَا سِتُّ مِئَةٍ وَاثْنَانِ وَأَرْبَعُونَ.
\par 63 وَمِنَ الْكَهَنَةِ: بَنُو حَبَابَا بَنُو هَقُّوصَ بَنُو بَرْزِلاَّيَ الَّذِي أَخَذَ امْرَأَةً مِنْ بَنَاتِ بَرْزِلاَّيَ الْجِلْعَادِيِّ وَتَسَمَّى بِاسْمِهِمْ.
\par 64 هَؤُلاَءِ فَحَصُوا عَنْ كِتَابَةِ أَنْسَابِهِمْ فَلَمْ تُوجَدْ فَرُذِلُوا مِنَ الْكَهَنُوتِ.
\par 65 وَقَالَ لَهُمُ التَّرْشَاثَا أَنْ لاَ يَأْكُلُوا مِنْ قُدْسِ الأَقْدَاسِ حَتَّى يَقُومَ كَاهِنٌ لِلأُورِيمِ وَالتُّمِّيمِ.
\par 66 كُلُّ الْجُمْهُورِ مَعاً أَرْبَعُ رَبَوَاتٍ وَأَلْفَانِ وَثَلاَثُ مِئَةٍ وَسِتُّونَ
\par 67 فَضْلاً عَنْ عَبِيدِهِمْ وَإِمَائِهِمِ الَّذِينَ كَانُوا سَبْعَةَ آلاَفٍ وَثَلاَثَ مِئَةٍ وَسَبْعَةً وَثَلاَثِينَ. وَلَهُمْ مِنَ الْمُغَنِّينَ وَالْمُغَنِّيَاتِ مِئَتَانِ وَخَمْسَةٌ وَأَرْبَعُونَ.
\par 68 وَخَيْلُهُمْ سَبْعُ مِئَةٍ وَسِتَّةٌ وَثَلاَثُونَ وَبِغَالُهُمْ مِئَتَانِ وَخَمْسَةٌ وَأَرْبَعُونَ
\par 69 وَالْجِمَالُ أَرْبَعُ مِئَةٍ وَخَمْسَةٌ وَثَلاَثُونَ وَالْحَمِيرُ سِتَّةُ آلاَفٍ وَسَبْعُ مِئَةٍ وَعِشْرُونَ.
\par 70 وَالْبَعْضُ مِنْ رُؤُوسِ الآبَاءِ أَعْطُوا لِلْعَمَلِ. التَّرْشَاثَا أَعْطَى لِلْخَزِينَةِ أَلْفَ دِرْهَمٍ مِنَ الذَّهَبِ وَخَمْسِينَ مِنْضَحَةً وَخَمْسَ مِئَةٍ وَثَلاَثِينَ قَمِيصاً لِلْكَهَنَةِ.
\par 71 وَالْبَعْضُ مِنْ رُؤُوسِ الآبَاءِ أَعْطُوا لِخَزِينَةِ الْعَمَلِ رَبْوَتَيْنِ مِنَ الذَّهَبِ وَأَلْفَيْنِ وَمِئَتَيْ مَناً مِنَ الْفِضَّةِ.
\par 72 وَمَا أَعْطَاهُ بَقِيَّةُ الشَّعْبِ سِتَّ رَبَوَاتٍ مِنَ الذَّهَبِ وَأَلْفَيْ مَناً مِنَ الْفِضَّةِ وَسَبْعَةً وَسِتِّينَ قَمِيصاً لِلْكَهَنَةِ.
\par 73 وَأَقَامَ الْكَهَنَةُ وَاللاَّوِيُّونَ وَالْبَوَّابُونَ وَالْمُغَنُّونَ وَبَعْضُ الشَّعْبِ وَالنَّثِينِيمُ وَكُلُّ إِسْرَائِيلَ فِي مُدُنِهِمْ.

\chapter{8}

\par 1 وَلَمَّا اسْتُهِلَّ الشَّهْرُ السَّابِعُ وَبَنُو إِسْرَائِيلَ فِي مُدُنِهِمِ اجْتَمَعَ كُلُّ الشَّعْبِ كَرَجُلٍ وَاحِدٍ إِلَى السَّاحَةِ الَّتِي أَمَامَ بَابِ الْمَاءِ وَقَالُوا لِعَزْرَا الْكَاتِبِ أَنْ يَأْتِيَ بِسِفْرِ شَرِيعَةِ مُوسَى الَّتِي أَمَرَ بِهَا الرَّبُّ إِسْرَائِيلَ.
\par 2 فَأَتَى عَزْرَا الْكَاتِبُ بِالشَّرِيعَةِ أَمَامَ الْجَمَاعَةِ مِنَ الرِّجَالِ وَالنِّسَاءِ وَكُلِّ فَاهِمٍ مَا يُسْمَعُ فِي الْيَوْمِ الأَوَّلِ مِنَ الشَّهْرِ السَّابِعِ.
\par 3 وَقَرَأَ فِيهَا أَمَامَ السَّاحَةِ الَّتِي أَمَامَ بَابِ الْمَاءِ مِنَ الصَّبَاحِ إِلَى نِصْفِ النَّهَارِ أَمَامَ الرِّجَالِ وَالنِّسَاءِ وَالْفَاهِمِينَ. وَكَانَتْ آذَانُ كُلِّ الشَّعْبِ نَحْوَ سِفْرِ الشَّرِيعَةِ.
\par 4 وَوَقَفَ عَزْرَا الْكَاتِبُ عَلَى مِنْبَرِ الْخَشَبِ الَّذِي عَمِلُوهُ لِهَذَا الأَمْرِ وَوَقَفَ بِجَانِبِهِ مَتَّثْيَا وَشَمَعُ وَعَنَايَا وَأُورِيَّا وَحِلْقِيَّا وَمَعْسِيَّا عَنْ يَمِينِهِ وَعَنْ يَسَارِهِ فَدَايَا وَمِيشَائِيلُ وَمَلْكِيَّا وَحَاشُومُ وَحَشْبَدَّانَةُ وَزَكَرِيَّا وَمَشُلاَّمُ.
\par 5 وَفَتَحَ عَزْرَا السِّفْرَ أَمَامَ كُلِّ الشَّعْبِ لأَنَّهُ كَانَ فَوْقَ كُلِّ الشَّعْبِ. وَعِنْدَمَا فَتَحَهُ وَقَفَ كُلُّ الشَّعْبِ.
\par 6 وَبَارَكَ عَزْرَا الرَّبَّ الإِلَهَ الْعَظِيمَ. وَأَجَابَ جَمِيعُ الشَّعْبِ: [آمِينَ آمِينَ!] رَافِعِينَ أَيْدِيَهُمْ وَخَرُّوا وَسَجَدُوا لِلرَّبِّ عَلَى وُجُوهِهِمْ إِلَى الأَرْضِ.
\par 7 وَيَشُوعُ وَبَانِي وَشَرَبْيَا وَيَامِينُ وَعَقُّوبُ وَشَبْتَايُ وَهُودِيَّا وَمَعْسِيَّا وَقَلِيطَا وَعَزَرْيَا وَيُوزَابَادُ وَحَنَانُ وَفَلاَيَا وَاللاَّوِيُّونَ أَفْهَمُوا الشَّعْبَ الشَّرِيعَةَ وَالشَّعْبُ فِي أَمَاكِنِهِمْ.
\par 8 وَقَرَأُوا فِي السِّفْرِ فِي شَرِيعَةِ اللَّهِ بِبَيَانٍ وَفَسَّرُوا الْمَعْنَى وَأَفْهَمُوهُمُ الْقِرَاءَةَ.
\par 9 وَنَحَمْيَا (أَيِ التِّرْشَاثَا) وَعَزْرَا الْكَاهِنُ الْكَاتِبُ وَاللاَّوِيُّونَ الْمُفْهِمُونَ الشَّعْبَ قَالُوا لِجَمِيعِ الشَّعْبِ: [هَذَا الْيَوْمُ مُقَدَّسٌ لِلرَّبِّ إِلَهِكُمْ لاَ تَنُوحُوا وَلاَ تَبْكُوا]. لأَنَّ جَمِيعَ الشَّعْبِ بَكُوا حِينَ سَمِعُوا كَلاَمَ الشَّرِيعَةِ.
\par 10 فَقَالَ لَهُمُ: [اذْهَبُوا كُلُوا السَّمِينَ وَاشْرَبُوا الْحُلْوَ وَابْعَثُوا أَنْصِبَةً لِمَنْ لَمْ يُعَدَّ لَهُ لأَنَّ الْيَوْمَ إِنَّمَا هُوَ مُقَدَّسٌ لِسَيِّدِنَا. وَلاَ تَحْزَنُوا لأَنَّ فَرَحَ الرَّبِّ هُوَ قُوَّتُكُمْ].
\par 11 وَكَانَ اللاَّوِيُّونَ يُسَكِّتُونَ كُلَّ الشَّعْبِ قَائِلِينَ: [اسْكُتُوا لأَنَّ الْيَوْمَ مُقَدَّسٌ فَلاَ تَحْزَنُوا].
\par 12 فَذَهَبَ كُلُّ الشَّعْبِ لِيَأْكُلُوا وَيَشْرَبُوا وَيَبْعَثُوا أَنْصِبَةً وَيَعْمَلُوا فَرَحاً عَظِيماً لأَنَّهُمْ فَهِمُوا الْكَلاَمَ الَّذِي عَلَّمُوهُمْ إِيَّاهُ.
\par 13 وَفِي الْيَوْمِ الثَّانِي اجْتَمَعَ رُؤُوسُ آبَاءِ جَمِيعِ الشَّعْبِ وَالْكَهَنَةِ وَاللاَّوِيُّونَ إِلَى عَزْرَا الْكَاتِبِ لِيُفْهِمَهُمْ كَلاَمَ الشَّرِيعَةِ.
\par 14 فَوَجَدُوا مَكْتُوباً فِي الشَّرِيعَةِ الَّتِي أَمَرَ بِهَا الرَّبُّ عَنْ يَدِ مُوسَى أَنَّ بَنِي إِسْرَائِيلَ يَسْكُنُونَ فِي مَظَالَّ فِي الْعِيدِ فِي الشَّهْرِ السَّابِعِ
\par 15 وَأَنْ يُسْمِعُوا وَيُنَادُوا فِي كُلِّ مُدُنِهِمْ وَفِي أُورُشَلِيمَ قَائِلِينَ: [اخْرُجُوا إِلَى الْجَبَلِ وَأْتُوا بِأَغْصَانِ زَيْتُونٍ وَأَغْصَانِ زَيْتُونٍ بَرِّيٍّ وَأَغْصَانِ آسٍ وَأَغْصَانِ نَخْلٍ وَأَغْصَانِ أَشْجَارٍ غَبْيَاءَ لِعَمَلِ مَظَالَّ كَمَا هُوَ مَكْتُوبٌ].
\par 16 فَخَرَجَ الشَّعْبُ وَجَلَبُوا وَعَمِلُوا لأَنْفُسِهِمْ مَظَالَّ كُلُّ وَاحِدٍ عَلَى سَطْحِهِ وَفِي دُورِهِمْ وَدُورِ بَيْتِ اللَّهِ وَفِي سَاحَةِ بَابِ الْمَاءِ وَفِي سَاحَةِ بَابِ أَفْرَايِمَ.
\par 17 وَعَمِلَ كُلُّ الْجَمَاعَةِ الرَّاجِعِينَ مِنَ السَّبْيِ مَظَالَّ وَسَكَنُوا فِي الْمَظَالِّ لأَنَّهُ لَمْ يَعْمَلْ بَنُو إِسْرَائِيلَ هَكَذَا مِنْ أَيَّامِ يَشُوعَ بْنِ نُونٍ إِلَى ذَلِكَ الْيَوْمِ. وَكَانَ فَرَحٌ عَظِيمٌ جِدّاً.
\par 18 وَكَانَ يُقْرَأُ فِي سِفْرِ شَرِيعَةِ اللَّهِ يَوْماً فَيَوْماً مِنَ الْيَوْمِ الأَوَّلِ إِلَى الْيَوْمِ الأَخِيرِ. وَعَمِلُوا عِيداً سَبْعَةَ أَيَّامٍ. وَفِي الْيَوْمِ الثَّامِنِ اعْتِكَافٌ حَسَبَ الْمَرْسُومِ.

\chapter{9}

\par 1 وَفِي الْيَوْمِ الرَّابِعِ وَالْعِشْرِينَ مِنْ هَذَا الشَّهْرِ اجْتَمَعَ بَنُو إِسْرَائِيلَ بِالصَّوْمِ وَعَلَيْهِمْ مُسُوحٌ وَتُرَابٌ.
\par 2 وَانْفَصَلَ نَسْلُ إِسْرَائِيلَ مِنْ جَمِيعِ بَنِي الْغُرَبَاءِ وَوَقَفُوا وَاعْتَرَفُوا بِخَطَايَاهُمْ وَذُنُوبِ آبَائِهِمْ.
\par 3 وَأَقَامُوا فِي مَكَانِهِمْ وَقَرَأُوا فِي سِفْرِ شَرِيعَةِ الرَّبِّ إِلَهِهِمْ رُبْعَ النَّهَارِ وَفِي الرُّبْعِ الآخَرِ كَانُوا يَحْمَدُونَ وَيَسْجُدُونَ لِلرَّبِّ إِلَهِهِمْ.
\par 4 وَوَقَفَ عَلَى دَرَجِ اللاَّوِيِّينَ يَشُوعُ وَبَانِي وَقَدْمِيئِيلُ وَشَبَنْيَا وَبُنِّي وَشَرَبْيَا وَبَانِي وَكَنَانِي وَصَرَخُوا بِصَوْتٍ عَظِيمٍ إِلَى الرَّبِّ إِلَهِهِمْ.
\par 5 وَقَالَ اللاَّوِيُّونَ يَشُوعُ وَقَدْمِيئِيلُ وَبَانِي وَحَشَبْنِيَا وَشَرَبْيَا وَهُودِيَّا وَشَبَنْيَا وَفَتَحْيَا: [ قُومُوا بَارِكُوا الرَّبَّ إِلَهَكُمْ مِنَ الأَزَلِ إِلَى الأَبَدِ وَلْيَتَبَارَكِ اسْمُ جَلاَلِكَ الْمُتَعَالِي عَلَى كُلِّ بَرَكَةٍ وَتَسْبِيحٍ.
\par 6 أَنْتَ هُوَ الرَّبُّ وَحْدَكَ. أَنْتَ صَنَعْتَ السَّمَاوَاتِ وَسَمَاءَ السَّمَاوَاتِ وَكُلَّ جُنْدِهَا وَالأَرْضَ وَكُلَّ مَا عَلَيْهَا وَالْبِحَارَ وَكُلَّ مَا فِيهَا وَأَنْتَ تُحْيِيهَا كُلَّهَا. وَجُنْدُ السَّمَاءِ لَكَ يَسْجُدُ.
\par 7 أَنْتَ هُوَ الرَّبُّ الإِلَهُ الَّذِي اخْتَرْتَ أَبْرَامَ وَأَخْرَجْتَهُ مِنْ أُورِ الْكِلْدَانِيِّينَ وَجَعَلْتَ اسْمَهُ إِبْرَاهِيمَ.
\par 8 وَوَجَدْتَ قَلْبَهُ أَمِيناً أَمَامَكَ وَقَطَعْتَ مَعَهُ الْعَهْدَ أَنْ تُعْطِيَهُ أَرْضَ الْكَنْعَانِيِّينَ وَالْحِثِّيِّينَ وَالأَمُورِيِّينَ وَالْفِرِزِّيِّينَ وَالْيَبُوسِيِّينَ وَالْجِرْجَاشِيِّينَ وَتُعْطِيَهَا لِنَسْلِهِ. وَقَدْ أَنْجَزْتَ وَعْدَكَ لأَنَّكَ صَادِقٌ.
\par 9 وَرَأَيْتَ ذُلَّ آبَائِنَا فِي مِصْرَ وَسَمِعْتَ صُرَاخَهُمْ عِنْدَ بَحْرِ سُوفٍ
\par 10 وَأَظْهَرْتَ آيَاتٍ وَعَجَائِبَ عَلَى فِرْعَوْنَ وَعَلَى جَمِيعِ عَبِيدِهِ وَعَلَى كُلِّ شَعْبِ أَرْضِهِ لأَنَّكَ عَلِمْتَ أَنَّهُمْ بَغُوا عَلَيْهِمْ وَعَمِلْتَ لِنَفْسِكَ اسْماً كَهَذَا الْيَوْمِ.
\par 11 وَفَلَقْتَ الْيَمَّ أَمَامَهُمْ وَعَبَرُوا فِي وَسَطِ الْبَحْرِ عَلَى الْيَابِسَةِ وَطَرَحْتَ مُطَارِدِيهِمْ فِي الأَعْمَاقِ كَحَجَرٍ فِي مِيَاهٍ قَوِيَّةٍ.
\par 12 وَهَدَيْتَهُمْ بِعَمُودِ سَحَابٍ نَهَاراً وَبِعَمُودِ نَارٍ لَيْلاً لِتُضِيءَ لَهُمْ فِي الطَّرِيقِ الَّتِي يَسِيرُونَ فِيهَا.
\par 13 وَنَزَلْتَ عَلَى جَبَلِ سِينَاءَ وَكَلَّمْتَهُمْ مِنَ السَّمَاءِ وَأَعْطَيْتَهُمْ أَحْكَاماً مُسْتَقِيمَةً وَشَرَائِعَ صَادِقَةً فَرَائِضَ وَوَصَايَا صَالِحَةً.
\par 14 وَعَرَّفْتَهُمْ سَبْتَكَ الْمُقَدَّسَ وَأَمَرْتَهُمْ بِوَصَايَا وَفَرَائِضَ وَشَرَائِعَ عَنْ يَدِ مُوسَى عَبْدِكَ.
\par 15 وَأَعْطَيْتَهُمْ خُبْزاً مِنَ السَّمَاءِ لِجُوعِهِمْ وَأَخْرَجْتَ لَهُمْ مَاءً مِنَ الصَّخْرَةِ لِعَطَشِهِمْ وَقُلْتَ لَهُمْ أَنْ يَدْخُلُوا وَيَرِثُوا الأَرْضَ الَّتِي رَفَعْتَ يَدَكَ أَنْ تُعْطِيَهُمْ إِيَّاهَا.
\par 16 [وَلَكِنَّهُمْ بَغُوا هُمْ وَآبَاؤُنَا وَصَلَّبُوا رِقَابَهُمْ وَلَمْ يَسْمَعُوا لِوَصَايَاكَ
\par 17 وَأَبُوا الاِسْتِمَاعَ وَلَمْ يَذْكُرُوا عَجَائِبَكَ الَّتِي صَنَعْتَ مَعَهُمْ وَصَلَّبُوا رِقَابَهُمْ. وَعِنْدَ تَمَرُّدِهِمْ أَقَامُوا رَئِيساً لِيَرْجِعُوا إِلَى عُبُودِيَّتِهِمْ. وَأَنْتَ إِلَهٌ غَفُورٌ وَحَنَّانٌ وَرَحِيمٌ طَوِيلُ الرُّوحِ وَكَثِيرُ الرَّحْمَةِ فَلَمْ تَتْرُكْهُمْ.
\par 18 مَعَ أَنَّهُمْ عَمِلُوا لأَنْفُسِهِمْ عِجْلاً مَسْبُوكاً وَقَالُوا: هَذَا إِلَهُكَ الَّذِي أَخْرَجَكَ مِنْ مِصْرَ وَعَمِلُوا إِهَانَةً عَظِيمَةً
\par 19 أَنْتَ بِرَحْمَتِكَ الْكَثِيرَةِ لَمْ تَتْرُكْهُمْ فِي الْبَرِّيَّةِ وَلَمْ يَزُلْ عَنْهُمْ عَمُودُ السَّحَابِ نَهَاراً لِهِدَايَتِهِمْ فِي الطَّرِيقِ وَلاَ عَمُودُ النَّارِ لَيْلاً لِيُضِيءَ لَهُمْ فِي الطَّرِيقِ الَّتِي يَسِيرُونَ فِيهَا.
\par 20 وَأَعْطَيْتَهُمْ رُوحَكَ الصَّالِحَ لِتَعْلِيمِهِمْ وَلَمْ تَمْنَعْ مَنَّكَ عَنْ أَفْوَاهِهِمْ وَأَعْطَيْتَهُمْ مَاءً لِعَطَشِهِمْ
\par 21 وَعُلْتَهُمْ أَرْبَعِينَ سَنَةً فِي الْبَرِّيَّةِ فَلَمْ يَحْتَاجُوا. لَمْ تَبْلَ ثِيَابُهُمْ وَلَمْ تَتَوَرَّمْ أَرْجُلُهُمْ.
\par 22 وَأَعْطَيْتَهُمْ مَمَالِكَ وَشُعُوباً وَفَرَّقْتَهُمْ إِلَى جِهَاتٍ فَامْتَلَكُوا أَرْضَ سِيحُونَ وَأَرْضَ مَلِكِ حَشْبُونَ وَأَرْضَ عُوجٍ مَلِكِ بَاشَانَ.
\par 23 وَأَكْثَرْتَ بَنِيهِمْ كَنُجُومِ السَّمَاءِ وَأَتَيْتَ بِهِمْ إِلَى الأَرْضِ الَّتِي قُلْتَ لِآبَائِهِمْ أَنْ يَدْخُلُوا وَيَرِثُوهَا.
\par 24 فَدَخَلَ الْبَنُونَ وَوَرِثُوا الأَرْضَ وَأَخْضَعْتَ لَهُمْ سُكَّانَ أَرْضِ الْكَنْعَانِيِّينَ وَدَفَعْتَهُمْ لِيَدِهِمْ مَعَ مُلُوكِهِمْ وَشُعُوبِ الأَرْضِ لِيَعْمَلُوا بِهِمْ حَسَبَ إِرَادَتِهِمْ.
\par 25 وَأَخَذُوا مُدُناً حَصِينَةً وَأَرْضاً سَمِينَةً وَوَرِثُوا بُيُوتاً مَلآنَةً كُلَّ خَيْرٍ وَآبَاراً مَحْفُورَةً وَكُرُوماً وَزَيْتُوناً وَأَشْجَاراً مُثْمِرَةً بِكَثْرَةٍ فَأَكَلُوا وَشَبِعُوا وَسَمِنُوا وَتَلَذَّذُوا بِخَيْرِكَ الْعَظِيمِ.
\par 26 وَعَصُوا وَتَمَرَّدُوا عَلَيْكَ وَطَرَحُوا شَرِيعَتَكَ وَرَاءَ ظُهُورِهِمْ وَقَتَلُوا أَنْبِيَاءَكَ الَّذِينَ أَشْهَدُوا عَلَيْهِمْ لِيَرُدُّوهُمْ إِلَيْكَ وَعَمِلُوا إِهَانَةً عَظِيمَةً.
\par 27 فَدَفَعْتَهُمْ لِيَدِ مُضَايِقِيهِمْ فَضَايَقُوهُمْ. وَفِي وَقْتِ ضِيقِهِمْ صَرَخُوا إِلَيْكَ وَأَنْتَ مِنَ السَّمَاءِ سَمِعْتَ وَحَسَبَ مَرَاحِمِكَ الْكَثِيرَةِ أَعْطَيْتَهُمْ مُخَلِّصِينَ خَلَّصُوهُمْ مِنْ يَدِ مُضَايِقِيهِمْ.
\par 28 وَلَكِنْ لَمَّا اسْتَرَاحُوا رَجَعُوا إِلَى عَمَلِ الشَّرِّ قُدَّامَكَ فَتَرَكْتَهُمْ بِيَدِ أَعْدَائِهِمْ فَتَسَلَّطُوا عَلَيْهِمْ ثُمَّ رَجَعُوا وَصَرَخُوا إِلَيْكَ. وَأَنْتَ مِنَ السَّمَاءِ سَمِعْتَ وَأَنْقَذْتَهُمْ حَسَبَ مَرَاحِمِكَ الْكَثِيرَةِ أَحْيَاناً كَثِيرَةً.
\par 29 وَأَشْهَدْتَ عَلَيْهِمْ لِتَرُدَّهُمْ إِلَى شَرِيعَتِكَ. وَأَمَّا هُمْ فَبَغُوا وَلَمْ يَسْمَعُوا لِوَصَايَاكَ وَأَخْطَأُوا ضِدَّ أَحْكَامِكَ الَّتِي إِذَا عَمِلَهَا إِنْسَانٌ يَحْيَا بِهَا. وَأَعْطُوا كَتِفاً مُعَانِدَةً وَصَلَّبُوا رِقَابَهُمْ وَلَمْ يَسْمَعُوا.
\par 30 فَاحْتَمَلْتَهُمْ سِنِينَ كَثِيرَةً وَأَشْهَدْتَ عَلَيْهِمْ بِرُوحِكَ عَنْ يَدِ أَنْبِيَائِكَ فَلَمْ يُصْغُوا فَدَفَعْتَهُمْ لِيَدِ شُعُوبِ الأَرَاضِي.
\par 31 وَلَكِنْ لأَجْلِ مَرَاحِمِكَ الْكَثِيرَةِ لَمْ تُفْنِهِمْ وَلَمْ تَتْرُكْهُمْ لأَنَّكَ إِلَهٌ حَنَّانٌ وَرَحِيمٌ.
\par 32 [وَالآنَ يَا إِلَهَنَا الإِلَهَ الْعَظِيمَ الْجَبَّارَ الْمَخُوفَ حَافِظَ الْعَهْدِ وَالرَّحْمَةِ لاَ تَصْغُرْ لَدَيْكَ كُلُّ الْمَشَقَّاتِ الَّتِي أَصَابَتْنَا نَحْنُ وَمُلُوكَنَا وَرُؤَسَاءَنَا وَكَهَنَتَنَا وَأَنْبِيَاءَنَا وَآبَاءَنَا وَكُلَّ شَعْبِكَ مِنْ أَيَّامِ مُلُوكِ أَشُّورَ إِلَى هَذَا الْيَوْمِ.
\par 33 وَأَنْتَ بَارٌّ فِي كُلِّ مَا أَتَى عَلَيْنَا لأَنَّكَ عَمِلْتَ بِالْحَقِّ وَنَحْنُ أَذْنَبْنَا.
\par 34 وَمُلُوكُنَا وَرُؤَسَاؤُنَا وَكَهَنَتُنَا وَآبَاؤُنَا لَمْ يَعْمَلُوا شَرِيعَتَكَ وَلاَ أَصْغُوا إِلَى وَصَايَاكَ وَشَهَادَاتِكَ الَّتِي أَشْهَدْتَهَا عَلَيْهِمْ.
\par 35 وَهُمْ لَمْ يَعْبُدُوكَ فِي مَمْلَكَتِهِمْ وَفِي خَيْرِكَ الْكَثِيرِ الَّذِي أَعْطَيْتَهُمْ وَفِي الأَرْضِ الْوَاسِعَةِ السَّمِينَةِ الَّتِي جَعَلْتَهَا أَمَامَهُمْ وَلَمْ يَرْجِعُوا عَنْ أَعْمَالِهِمِ الرَّدِيئَةِ.
\par 36 هَا نَحْنُ الْيَوْمَ عَبِيدٌ وَالأَرْضَ الَّتِي أَعْطَيْتَ لِآبَائِنَا لِيَأْكُلُوا أَثْمَارَهَا وَخَيْرَهَا هَا نَحْنُ عَبِيدٌ فِيهَا
\par 37 وَغَلاَّتُهَا كَثِيرَةٌ لِلْمُلُوكِ الَّذِينَ جَعَلْتَهُمْ عَلَيْنَا لأَجْلِ خَطَايَانَا وَهُمْ يَتَسَلَّطُونَ عَلَى أَجْسَادِنَا وَعَلَى بَهَائِمِنَا حَسَبَ إِرَادَتِهِمْ وَنَحْنُ فِي كَرْبٍ عَظِيمٍ.
\par 38 وَمِنْ أَجْلِ كُلِّ ذَلِكَ نَحْنُ نَقْطَعُ مِيثَاقاً وَنَكْتُبُهُ. وَرُؤَسَاؤُنَا وَلاَوِيُّونَا وَكَهَنَتُنَا يَخْتِمُونَ].

\chapter{10}

\par 1 وَالَّذِينَ خَتَمُوا هُمْ نَحَمْيَا التِّرْشَاثَا ابْنُ حَكَلْيَا وَصِدْقِيَّا
\par 2 وَسَرَايَا وَعَزَرْيَا وَيَرْمِيَا
\par 3 وَفَشْحُورُ وَأَمَرْيَا وَمَلْكِيَّا
\par 4 وَحَطُّوشُ وَشَبَنْيَا وَمَلُّوخُ
\par 5 وَحَارِيمُ وَمَرِيمُوثُ وَعُوبَدْيَا
\par 6 وَدَانِيآلُ وَجِنْثُونُ وَبَارُوخُ
\par 7 وَمَشُلاَّمُ وَأَبِيَّا وَمِيَّامِينُ
\par 8 وَمَعَزْيَا وَبِلْجَايُ وَشَمَعْيَا. هَؤُلاَءِ هُمُ الْكَهَنَةُ.
\par 9 وَاللاَّوِيُّونَ يَشُوعُ بْنُ أَزَنْيَا وَبِنُّويُ مِنْ بَنِي حِينَادَادَ وَقَدْمِيئِيلُ
\par 10 وَإِخْوَتُهُمْ شَبَنْيَا وَهُودِيَّا وَقَلِيطَا وَفَلاَيَا وَحَانَانُ
\par 11 وَمِيخَا وَرَحُوبُ وَحَشَبْيَا
\par 12 وَزَكُّورُ وَشَرَبْيَا وَشَبَنْيَا
\par 13 وَهُودِيَّا وَبَانِي وَبَنِينُو.
\par 14 رُؤُوسُ الشَّعْبِ فَرْعُوشُ وَفَحَثُ مُوآبَ وَعِيلاَمُ وَزَتُّو وَبَانِي
\par 15 وَبُنِّي وَعَزْجَدُ وَبِيبَايُ
\par 16 وَأَدُونِيَّا وَبَغْوَايُ وَعَادِينُ
\par 17 وَآطِيرُ وَحَزَقِيَّا وَعَزُّورُ
\par 18 وَهُودِيَّا وَحَشُومُ وَبِيصَايُ
\par 19 وَحَارِيفُ وَعَنَاثُوثُ وَنِيبَايُ
\par 20 وَمَجْفِيعَاشُ وَمَشُلاَّمُ وَحَزِيرُ
\par 21 وَمَشِيزَبْئِيلُ وَصَادُوقُ وَيَدُّوعُ
\par 22 وَفَلَطْيَا وَحَانَانُ وَعَنَايَا
\par 23 وَهُوشَعُ وَحَنَنْيَا وَحَشُّوبُ
\par 24 وَهَلُوحِيشُ وَفِلْحَا وشُوبِيقُ
\par 25 وَرَحُومُ وَحَشَبْنَا وَمَعْسِيَّا
\par 26 وَأَخِيَا وَحَانَانُ وَعَانَانُ
\par 27 وَمَلُّوخُ وَحَرِيمُ وَبَعْنَةُ.
\par 28 وَبَاقِي الشَّعْبِ وَالْكَهَنَةِ وَاللاَّوِيِّينَ وَالْبَوَّابِينَ وَالْمُغَنِّينَ وَالنَّثِينِيمَ وَكُلِّ الَّذِينَ انْفَصَلُوا مِنْ شُعُوبِ الأَرَاضِي إِلَى شَرِيعَةِ اللَّهِ وَنِسَائِهِمْ وَبَنِيهِمْ وَبَنَاتِهِمْ كُلُِّ أَصْحَابِ الْمَعْرِفَةِ وَالْفَهْمِ
\par 29 لَصِقُوا بِإِخْوَتِهِمْ وَعُظَمَائِهِمْ وَدَخَلُوا فِي قَسَمٍ وَحِلْفٍ أَنْ يَسِيرُوا فِي شَرِيعَةِ اللَّهِ الَّتِي أُعْطِيَتْ عَنْ يَدِ مُوسَى عَبْدِ اللَّهِ وَأَنْ يَحْفَظُوا وَيَعْمَلُوا جَمِيعَ وَصَايَا الرَّبِّ سَيِّدِنَا وَأَحْكَامِهِ وَفَرَائِضِهِ
\par 30 وَأَنْ لاَ نُعْطِيَ بَنَاتِنَا لِشُعُوبِ الأَرْضِ وَلاَ نَأْخُذَ بَنَاتِهِمْ لِبَنِينَا.
\par 31 وَشُعُوبُ الأَرْضِ الَّذِينَ يَأْتُونَ بِالْبَضَائِعِ وَكُلِّ طَعَامِ يَوْمِ السَّبْتِ لِلْبَيْعِ لاَ نَأْخُذُ مِنْهُمْ فِي سَبْتٍ وَلاَ فِي يَوْمٍ مُقَدَّسٍ وَأَنْ نَتْرُكَ السَّنَةَ السَّابِعَةَ وَالْمُطَالَبَةَ بِكُلِّ دَيْنٍ.
\par 32 وَأَقَمْنَا عَلَى أَنْفُسِنَا فَرَائِضَ: أَنْ نَجْعَلَ عَلَى أَنْفُسِنَا ثُلْثَ شَاقِلٍ كُلَّ سَنَةٍ لِخِدْمَةِ بَيْتِ إِلَهِنَا
\par 33 لِخُبْزِ الْوُجُوهِ وَالتَّقْدِمَةِ الدَّائِمَةِ وَالْمُحْرَقَةِ الدَّائِمَةِ وَالسُّبُوتِ وَالأَهِلَّةِ وَالْمَوَاسِمِ وَالأَقْدَاسِ وَذَبَائِحِ الْخَطِيَّةِ لِلتَّكْفِيرِ عَنْ إِسْرَائِيلَ وَلِكُلِّ عَمَلِ بَيْتِ إِلَهِنَا.
\par 34 وَأَلْقَيْنَا قُرَعاً عَلَى قُرْبَانِ الْحَطَبِ بَيْنَ الْكَهَنَةِ وَاللاَّوِيِّينَ وَالشَّعْبِ لإِدْخَالِهِ إِلَى بَيْتِ إِلَهِنَا حَسَبَ بُيُوتِ آبَائِنَا فِي أَوْقَاتٍ مُعَيَّنَةٍ سَنَةً فَسَنَةً لأَجْلِ إِحْرَاقِهِ عَلَى مَذْبَحِ الرَّبِّ إِلَهِنَا كَمَا هُوَ مَكْتُوبٌ فِي الشَّرِيعَةِ
\par 35 وَلإِدْخَالِ بَاكُورَاتِ أَرْضِنَا وَبَاكُورَاتِ ثَمَرِ كُلِّ شَجَرَةٍ سَنَةً فَسَنَةً إِلَى بَيْتِ الرَّبِّ
\par 36 وَأَبْكَارِ بَنِينَا وَبَهَائِمِنَا كَمَا هُوَ مَكْتُوبٌ فِي الشَّرِيعَةِ وَأَبْكَارِ بَقَرِنَا وَغَنَمِنَا لإِحْضَارِهَا إِلَى بَيْتِ إِلَهِنَا إِلَى الْكَهَنَةِ الْخَادِمِينَ فِي بَيْتِ إِلَهِنَا.
\par 37 وَأَنْ نَأْتِيَ بِأَوَائِلِ عَجِينِنَا وَرَفَائِعِنَا وَأَثْمَارِ كُلِّ شَجَرَةٍ مِنَ الْخَمْرِ وَالزَّيْتِ إِلَى الْكَهَنَةِ إِلَى مَخَادِعِ بَيْتِ إِلَهِنَا وَبِعُشْرِ أَرْضِنَا إِلَى اللاَّوِيِّينَ وَاللاَّوِيُّونَ هُمُ الَّذِينَ يُعَشِّرُونَ فِي جَمِيعِ مُدُنِ فَلاَحَتِنَا.
\par 38 وَيَكُونُ الْكَاهِنُ ابْنُ هَارُونَ مَعَ اللاَّوِيِّينَ حِينَ يُعَشِّرُ اللاَّوِيُّونَ وَيُصْعِدُ اللاَّوِيُّونَ عُشْرَ الأَعْشَارِ إِلَى بَيْتِ إِلَهِنَا إِلَى الْمَخَادِعِ إِلَى بَيْتِ الْخَزِينَةِ.
\par 39 لأَنَّ بَنِي إِسْرَائِيلَ وَبَنِي لاَوِي يَأْتُونَ بِرَفِيعَةِ الْقَمْحِ وَالْخَمْرِ وَالزَّيْتِ إِلَى الْمَخَادِعِ وَهُنَاكَ آنِيَةُ الْقُدْسِ وَالْكَهَنَةُ الْخَادِمُونَ وَالْبَوَّابُونَ وَالْمُغَنُّونَ وَلاَ نَتْرُكُ بَيْتَ إِلَهِنَا.

\chapter{11}

\par 1 وَسَكَنَ رُؤَسَاءُ الشَّعْبِ فِي أُورُشَلِيمَ. وَأَلْقَى سَائِرُ الشَّعْبِ قُرَعاً لِيَأْتُوا بِوَاحِدٍ مِنْ عَشَرَةٍ لِلسُّكْنَى فِي أُورُشَلِيمَ مَدِينَةِ الْقُدْسِ وَالتِّسْعَةِ الأَقْسَامِ فِي الْمُدُنِ.
\par 2 وَبَارَكَ الشَّعْبُ جَمِيعَ الْقَوْمِ الَّذِينَ انْتَدَبُوا لِلسُّكْنَى فِي أُورُشَلِيمَ.
\par 3 وَهَؤُلاَءِ هُمْ رُؤُوسُ الْبِلاَدِ الَّذِينَ سَكَنُوا فِي أُورُشَلِيمَ وَفِي مُدُنِ يَهُوذَا (سَكَنَ كُلُّ وَاحِدٍ فِي مُلْكِهِ فِي مُدُنِهِمْ مِنْ إِسْرَائِيلَ الْكَهَنَةُ وَاللاَّوِيُّونَ وَالنَّثِينِيمُ وَبَنُو عَبِيدِ سُلَيْمَانَ).
\par 4 وَسَكَنَ فِي أُورُشَلِيمَ مِنْ بَنِي يَهُوذَا وَمِنْ بَنِي بِنْيَامِينَ. فَمِنْ بَنِي يَهُوذَا عَثَايَا بْنُ عُزِّيَّا بْنِ زَكَرِيَّا بْنِ أَمَرْيَا بْنِ شَفَطْيَا بْنِ مَهْلَلْئِيلَ مِنْ بَنِي فَارَصَ.
\par 5 وَمَعْسِيَّا بْنُ بَارُوخَ بْنِ كَلْحُوزَةَ بْنِ حَزَايَا بْنِ عَدَايَا بْنِ يُويَارِيبَ بْنِ زَكَرِيَّا بْنِ الشِّيلُونِيِّ.
\par 6 جَمِيعُ بَنِي فَارَصَ السَّاكِنِينَ فِي أُورُشَلِيمَ أَرْبَعُ مِئَةٍ وَثَمَانِيَةٌ وَسِتُّونَ مِنْ رِجَالِ الْبَأْسِ.
\par 7 وَهَؤُلاَءِ بَنُو بِنْيَامِينَ سَلُّو بْنُ مَشُلاَّمَ بْنِ يُوعِيدَ بْنِ فَدَايَا بْنِ قُولاَيَا بْنِ مَعْسِيَّا بْنِ إِيثِيئِيلَ بْنِ يَشَعْيَا.
\par 8 وَبَعْدَهُ جَبَّايُ سَلاَّيُ. تِسْعُ مِئَةٍ وَثَمَانِيَةٌ وَعِشْرُونَ.
\par 9 وَكَانَ يُوئِيلُ بْنُ زِكْرِي وَكِيلاً عَلَيْهِمْ وَيَهُوذَا بْنُ هَسْنُوأَةَ ثَانِياً عَلَى الْمَدِينَةِ.
\par 10 مِنَ الْكَهَنَةِ يَدَعْيَا بْنُ يُويَارِيبَ وَيَاكِينُ
\par 11 وَسَرَايَا بْنُ حِلْقِيَّا بْنِ مَشُلاَّمَ بْنِ صَادُوقَ بْنِ مَرَايُوثَ بْنِ أَخِيطُوبَ رَئِيسُ بَيْتِ اللَّهِ.
\par 12 وَإِخْوَتُهُمْ عَامِلُو الْعَمَلِ لِلْبَيْتِ ثَمَانُ مِئَةٍ وَاثْنَانِ وَعِشْرُونَ. وَعَدَايَا بْنُ يَرُوحَامَ بْنِ فَلَلْيَا بْنِ أَمْصِي بْنِ زَكَرِيَّا بْنِ فَشْحُورَ بْنِ مَلْكِيَّا
\par 13 وَإِخْوَتُهُ رُؤُوسُ الآبَاءِ مِئَتَانِ وَاثْنَانِ وَأَرْبَعُونَ. وَعَمْشِسَايُ بْنُ عَزَرْئِيلَ بْنِ أَخْزَايَا بْنِ مَشْلِيمُوثَ بْنِ إِمِّيرَ
\par 14 وَإِخْوَتُهُمْ جَبَابِرَةُ بَأْسٍ مِئَةٌ وَثَمَانِيَةٌ وَعِشْرُونَ. وَالْوَكِيلُ عَلَيْهِمْ زَبْدِيئِيلُ بْنُ هَجْدُولِيمَ.
\par 15 وَمِنَ اللاَّوِيِّينَ شَمَعْيَا بْنُ حَشُّوبَ بْنِ عَزْرِيقَامَ بْنِ حَشَبْيَا بْنِ بُونِّي
\par 16 وَشَبْتَايُ وَيُوزَابَادُ عَلَى الْعَمَلِ الْخَارِجِيِّ لِبَيْتِ اللَّهِ مِنْ رُؤُوسِ اللاَّوِيِّينَ.
\par 17 وَمَتَّنْيَا بْنُ مِيخَا بْنِ زَبْدِي بْنِ آسَافَ رَئِيسُ التَّسْبِيحِ يُحَمِّدُ فِي الصَّلاَةِ وَبَقْبُقْيَا الثَّانِي بَيْنَ إِخْوَتِهِ وَعَبْدَا بْنُ شَمُّوعَ بْنِ جَلاَلَ بْنِ يَدُوثُونَ.
\par 18 جَمِيعُ اللاَّوِيِّينَ فِي الْمَدِينَةِ الْمُقَدَّسَةِ مِئَتَانِ وَثَمَانِيَةٌ وَأَرْبَعُونَ.
\par 19 وَالْبَوَّابُونَ عَقُّوبُ وَطَلْمُونُ وَإِخْوَتُهُمَا حَارِسُو الأَبْوَابِ مِئَةٌ وَاثْنَانِ وَسَبْعُونَ.
\par 20 وَكَانَ سَائِرُ إِسْرَائِيلَ مِنَ الْكَهَنَةِ وَاللاَّوِيِّينَ فِي جَمِيعِ مُدُنِ يَهُوذَا كُلُّ وَاحِدٍ فِي مِيرَاثِهِ.
\par 21 وَأَمَّا النَّثِينِيمُ فَسَكَنُوا فِي الأَكَمَةِ. وَكَانَ صِيحَا وَجِشْفَا عَلَى النَّثِينِيمِ.
\par 22 وَكَانَ وَكِيلَ اللاَّوِيِّينَ فِي أُورُشَلِيمَ عَلَى عَمَلِ بَيْتِ اللَّهِ عُزِّي بْنُ بَانِيَ بْنِ حَشَبْيَا بْنِ مَتَّنْيَا بْنِ مِيخَا مِنْ بَنِي آسَافَ الْمُغَنِّينَ.
\par 23 لأَنَّ وَصِيَّةَ الْمَلِكِ مِنْ جِهَتِهِمْ كَانَتْ أَنَّ لِلْمُرَنِّمِينَ فَرِيضَةً أَمْرَ كُلِّ يَوْمٍ فَيَوْمٍ.
\par 24 وَفَتَحْيَا بْنُ مَشِيزَبْئِيلَ مِنْ بَنِي زَارَحَ بْنِ يَهُوذَا كَانَ تَحْتَ يَدِ الْمَلِكِ فِي كُلِّ أُمُورِ الشَّعْبِ.
\par 25 وَفِي الضِّيَاعِ مَعَ حُقُولِهَا سَكَنَ مِنْ بَنِي يَهُوذَا فِي قَرْيَةِ أَرْبَعَ وَقُرَاهَا وَدِيبُونَ وَقُرَاهَا وَفِي يَقَبْصَِئِيلَ وَضِيَاعِهَا
\par 26 وَفِي يَشُوعَ وَمُولاَدَةَ وَبَيْتِ فَالَِطَ
\par 27 وَفِي حَصَرَ شُوعَالَ وَبِئْرِ سَبْعٍ وَقُرَاهَا
\par 28 وَفِي صِقْلَغَ وَمَكُونَةَ وَقُرَاهَا
\par 29 وَفِي عَيْنِ رِمُّونَ وَصَرْعَةَ وَيِرْمُوثَ
\par 30 وَزَانُوحَ وَعَدُلاَّمَ وَضِيَاعِهِمَا وَلَخِيشَ وَحُقُولِهَا وَعَزِيقَةَ وَقُرَاهَا وَحَلُّوا مِنْ بِئْرِ سَبْعٍ إِلَى وَادِي هِنُّومَ.
\par 31 وَبَنُو بِنْيَامِينَ سَكَنُوا مِنْ جَبَعَ إِلَى مِخْمَاسَ وَعَيَّا وَبَيْتِ إِيلٍ وَقُرَاهَا
\par 32 وَعَنَاثُوثَ وَنُوبٍ وَعَنَنْيَةَ
\par 33 وَحَاصُورَ وَرَامَةَ وَجِتَّايِمَ
\par 34 وَحَادِيدَ وَصَبُوعِيمَ وَنَبَلاَّطَ
\par 35 وَلُودٍ وَأُونُوَ وَادِي الصُّنَّاعِ.
\par 36 وَكَانَ مِنَ اللاَّوِيِّينَ فِرَقٌ فِي يَهُوذَا وَفِي بِنْيَامِينَ.

\chapter{12}

\par 1 وَهَؤُلاَءِ هُمُ الْكَهَنَةُ وَاللاَّوِيُّونَ الَّذِينَ صَعِدُوا مَعَ زَرُبَّابَِلَ بْنِ شَأَلْتِئِيلَ وَيَشُوعَ. سَرَايَا وَيِرْمِيَا وَعَزْرَا
\par 2 وَأَمَرْيَا وَمَلُّوخُ وَحَطُّوشُ
\par 3 وَشَكَنْيَا وَرَحُومُ وَمَرِيمُوثُ
\par 4 وَعِدُّو وَجِنْتُويُ وَأَبِيَّا
\par 5 وَمِيَّامِينُ وَمَعَدْيَا وَبَلْجَةُ
\par 6 وَشَمَعْيَا وَيُويَارِيبُ وَيَدَعْيَا
\par 7 وَسَلُّو وَعَامُوقُ وَحِلْقِيَّا وَيَدَعْيَا. هَؤُلاَءِ هُمْ رُؤُوسُ الْكَهَنَةِ وَإِخْوَتُهُمْ فِي أَيَّامِ يَشُوعَ.
\par 8 وَاللاَّوِيُّونَ يَشُوعُ وَبِنُّويُ وَقَدْمِيئِيلُ وَشَرَبْيَا وَيَهُوذَا وَمَتَّنْيَا الَّذِي عَلَى التَّحْمِيدِ هُوَ وَإِخْوَتُهُ
\par 9 وَبَقْبُقْيَا وَعُنِّي أَخَوَاهُمْ مُقَابَِلَهُمْ فِي الْحِرَاسَاتِ.
\par 10 وَيَشُوعُ وَلَدَ يُويَاقِيمَ وَيُويَاقِيمُ وَلَدَ أَلِيَاشِيبَ وَأَلِيَاشِيبُ وَلَدَ يُويَادَاعَ
\par 11 وَيُويَادَاعُ وَلَدَ يُونَاثَانَ وَيُونَاثَانُ وَلَدَ يَدُّوعَ.
\par 12 وَفِي أَيَّامِ يُويَاقِيمَ كَانَ الْكَهَنَةُ رُؤُوسُ الآبَاءِ لِسَرَايَا مَرَايَا وَلِيرْمِيَا حَنَنْيَا
\par 13 وَلِعَزْرَا مَشُلاَّمُ وَلأَمَرْيَا يَهُوحَانَانُ
\par 14 وَلِمَلِيكُو يُونَاثَانُ وَلِشَبْنِيَا يُوسُفُ
\par 15 وَلِحَرِيمَ عَدْنَا وَلِمَرَايُوثَ حَِلْقَايُ
\par 16 وَلِعِدُّو زَكَرِيَّا وَلِجِنَّثُونَ مَشُلاَّمُ
\par 17 وَلأَبِيَّا زِكْرِي وَلِمِنْيَامِينَ لِمُوعَدْيَا فِلْطَايُ
\par 18 وَلِبِلْجَةَ شَمُّوعُ وَلِشَمَعْيَا يَهُونَاثَانُ
\par 19 وَلِيُويَارِيبَ مَتْنَايُ وَلِيَدَعْيَا عُزِّي
\par 20 وَلِسَلاَّيَ قَلاَّيُ وَلِعَامُوقَ عَابِرُ
\par 21 وَلِحِلْقِيَّا حَشَبْيَا وَلِيَدَعْيَا نَثَنْئِيلُ.
\par 22 وَكَانَ اللاَّوِيُّونَ فِي أَيَّامِ أَلِيَاشِيبَ وَيُويَادَاعَ وَيُوحَانَانَ وَيَدُّوعَ مَكْتُوبِينَ رُؤُوسَ آبَاءٍ وَالْكَهَنَةُ أَيْضاً فِي مُلْكِ دَارِيُوسَ الْفَارِسِيِّ.
\par 23 وَكَانَ بَنُو لاَوِي رُؤُوسُ الآبَاءِ مَكْتُوبِينَ فِي سِفْرِ أَخْبَارِ الأَيَّامِ إِلَى أَيَّامِ يُوحَانَانَ بْنِ أَلْيَاشِيبَ.
\par 24 وَرُؤُوسُ اللاَّوِيِّينَ حَشَبْيَا وَشَرَبْيَا وَيَشُوعُ بْنُ قَدْمِيئِيلَ وَإِخْوَتُهُمْ مُقَابَِلَهُمْ لِلتَّسْبِيحِ وَالتَّحْمِيدِ حَسَبَ وَصِيَّةِ دَاوُدَ رَجُلِ اللَّهِ نَوْبَةً مُقَابَِلَ نَوْبَةٍ.
\par 25 وَكَانَ مَتَّنْيَا وَبَقْبُقْيَا وَعُوبَدْيَا وَمَشُلاَّمُ وَطَلْمُونُ وَعَقُّوبُ بَوَّابِينَ حَارِسِينَ الْحِرَاسَةَ عِنْدَ مَخَازِنِ الأَبْوَابِ.
\par 26 كَانَ هَؤُلاَءِ فِي أَيَّامِ يُويَاقِيمَ بْنِ يَشُوعَ بْنِ يُوصَادَاقَ وَفِي أَيَّامِ نَحَمْيَا الْوَالِي وَعَزْرَا الْكَاهِنِ الْكَاتِبِ.
\par 27 وَعِنْدَ تَدْشِينِ سُورِ أُورُشَلِيمَ طَلَبُوا اللاَّوِيِّينَ مِنْ جَمِيعِ أَمَاكِنِهِمْ لِيَأْتُوا بِهِمْ إِلَى أُورُشَلِيمَ لِكَيْ يُدَشِّنُوا بِفَرَحٍ وَبِحَمْدٍ وَغِنَاءٍ بِالصُّنُوجِ وَالرَّبَابِ وَالْعِيدَانِ.
\par 28 فَاجْتَمَعَ بَنُو الْمُغَنِّينَ مِنَ الدَّائِرَةِ حَوْلَ أُورُشَلِيمَ وَمِنْ ضِيَاعِ النَّطُوفَاتِيِّ
\par 29 وَمِنْ بَيْتِ الْجِلْجَالِ وَمِنْ حُقُولِ جَبَعَ وَعَزْمُوتَ لأَنَّ الْمُغَنِّينَ بَنُوا لأَنْفُسِهِمْ ضِيَاعاً حَوْلَ أُورُشَلِيمَ.
\par 30 وَتَطَهَّرَ الْكَهَنَةُ وَاللاَّوِيُّونَ وَطَهَّرُوا الشَّعْبَ وَالأَبْوَابَ وَالسُّورَ
\par 31 وَأَصْعَدْتُ رُؤَسَاءَ يَهُوذَا عَلَى السُّورِ وَأَقَمْتُ فِرْقَتَيْنِ عَظِيمَتَيْنِ مِنَ الْحَمَّادِينَ وَسَارَتِ الْوَاحِدَةُ يَمِيناً عَلَى السُّورِ نَحْوَ بَابِ الدِّمْنِ
\par 32 وَسَارَ وَرَاءَهُمْ هُوشَعْيَا وَنِصْفُ رُؤَسَاءِ يَهُوذَا
\par 33 وَعَزَرْيَا وَعَزْرَا وَمَشُلاَّمُ
\par 34 وَيَهُوذَا وَبِنْيَامِينُ وَشَمَعْيَا وَيِرْمِيَا
\par 35 وَمِنْ بَنِي الْكَهَنَةِ بِالأَبْوَاقِ زَكَرِيَّا بْنُ يُونَاثَانَ بْنِ شَمَعْيَا بْنِ مَتَّنْيَا بْنِ مِيخَايَا بْنِ زَكُّورَ بْنِ آسَافَ
\par 36 وَإِخْوَتُهُ شَمَعْيَا وَعَزَرْئِيلُ وَمِلَلاَيُ وَجِلَلاَيُ وَمَاعَايُ وَنَثَنْئِيلُ وَيَهُوذَا وَحَنَانِي بِآلاَتِ غِنَاءِ دَاوُدَ رَجُلِ اللَّهِ وَعَزْرَا الْكَاتِبُ أَمَامَهُمْ.
\par 37 وَعِنْدَ بَابِ الْعَيْنِ الَّذِي مُقَابَِلَهُمْ صَعِدُوا عَلَى دَرَجِ مَدِينَةِ دَاوُدَ عِنْدَ مَصْعَدِ السُّورِ فَوْقَ بَيْتِ دَاوُدَ إِلَى بَابِ الْمَاءِ شَرْقاً.
\par 38 وَسَارَتِ الْفِرْقَةُ الثَّانِيَةُ مِنَ الْحَمَّادِينَ مُقَابَِلَهُمْ وَأَنَا وَرَاءَهَا وَنِصْفُ الشَّعْبِ عَلَى السُّورِ مِنْ عِنْدِ بُرْجِ التَّنَانِيرِ إِلَى السُّورِ الْعَرِيضِ
\par 39 وَمِنْ فَوْقِ بَابِ أَفْرَايِمَ وَفَوْقَ الْبَابِ الْعَتِيقِ وَفَوْقَ بَابِ السَّمَكِ وَبُرْجِ حَنَنْئِيلَ وَبُرْجِ الْمِئَةِ إِلَى بَابِ الضَّأْنِ وَوَقَفُوا فِي بَابِ السِّجْنِ.
\par 40 فَوَقَفَ الْفِرْقَتَانِ مِنَ الْحَمَّادِينَ فِي بَيْتِ اللَّهِ وَأَنَا وَنِصْفُ الْوُلاَةِ مَعِي
\par 41 وَالْكَهَنَةُ أَلِيَاقِيمُ وَمَعْسِيَّا وَمِنْيَامِينُ وَمِيخَايَا وَأَلْيُوعِينَايُ وَزَكَرِيَّا وَحَنَنْيَا بِالأَبْوَاقِ
\par 42 وَمَعْسِيَّا وَشَمَعْيَا وَأَلْعَازَارُ وَعُزِّي وَيَهُوحَانَانُ وَمَلْكِيَّا وَعِيلاَمُ وعَازَرُ وَغَنَّى الْمُغَنُّونَ وَيِزْرَحْيَا الْوَكِيلُ.
\par 43 وَذَبَحُوا فِي ذَلِكَ الْيَوْمِ ذَبَائِحَ عَظِيمَةً وَفَرِحُوا لأَنَّ اللَّهَ أَفْرَحَهُمْ فَرَحاً عَظِيماً. وَفَرِحَ الأَوْلاَدُ وَالنِّسَاءُ أَيْضاً وَسُمِعَ فَرَحُ أُورُشَلِيمَ عَنْ بُعْدٍ.
\par 44 وَتَوَكَّلَ فِي ذَلِكَ الْيَوْمِ أُنَاسٌ عَلَى الْمَخَادِعِ لِلْخَزَائِنِ وَالرَّفَائِعِ وَالأَوَائِلِ وَالأَعْشَارِ لِيَجْمَعُوا فِيهَا مِنْ حُقُولِ الْمُدُنِ أَنْصِبَةَ الشَّرِيعَةِ لِلْكَهَنَةِ وَاللاَّوِيِّينَ لأَنَّ يَهُوذَا فَرِحَ بِالْكَهَنَةِ وَاللاَّوِيِّينَ الْوَاقِفِينَ
\par 45 حَارِسِينَ حِرَاسَةَ إِلَهِهِمْ وَحِرَاسَةَ التَّطْهِيرِ. وَكَانَ الْمُغَنُّونَ وَالْبَوَّابُونَ حَسَبَ وَصِيَّةِ دَاوُدَ وَسُلَيْمَانَ ابْنِهِ.
\par 46 لأَنَّهُ فِي أَيَّامِ دَاوُدَ وَآسَافَ مُنْذُ الْقَدِيمِ كَانَ رُؤُوسُ مُغَنِّينَ وَغِنَاءُ تَسْبِيحٍ وَتَحْمِيدٍ لِلَّهِ.
\par 47 وَكَانَ كُلُّ إِسْرَائِيلَ فِي أَيَّامِ زَرُبَّابَِلَ وَأَيَّامِ نَحَمْيَا يُؤَدُّونَ أَنْصِبَةَ الْمُغَنِّينَ وَالْبَوَّابِينَ أَمْرَ كُلِّ يَوْمٍ فِي يَوْمِهِ وَكَانُوا يُقَدِّسُونَ لِلاَّوِيِّينَ وَكَانَ اللاَّوِيُّونَ يُقَدِّسُونَ لِبَنِي هَارُونَ.

\chapter{13}

\par 1 فِي ذَلِكَ الْيَوْمِ قُرِئَ فِي سِفْرِ مُوسَى فِي آذَانِ الشَّعْبِ وَوُجِدَ مَكْتُوباً فِيهِ أَنَّ عَمُّونِيّاً وَمُوآبِيّاً لاَ يَدْخُلُ فِي جَمَاعَةِ اللَّهِ إِلَى الأَبَدِ.
\par 2 لأَنَّهُمْ لَمْ يُلاَقُوا بَنِي إِسْرَائِيلَ بِالْخُبْزِ وَالْمَاءِ بَلِ اسْتَأْجَرُوا عَلَيْهِمْ بَلْعَامَ لِيَلْعَنَهُمْ وَحَوَّلَ إِلَهُنَا اللَّعْنَةَ إِلَى بَرَكَةٍ.
\par 3 وَلَمَّا سَمِعُوا الشَّرِيعَةَ فَرَزُوا كُلَّ اللَّفِيفِ مِنْ إِسْرَائِيلَ.
\par 4 وَقَبْلَ هَذَا كَانَ أَلْيَاشِيبُ الْكَاهِنُ الْمُقَامُ عَلَى مِخْدَعِ بَيْتِ إِلَهِنَا قَرَابَةُ طُوبِيَّا
\par 5 قَدْ هَيَّأَ لَهُ مِخْدَعاً عَظِيماً حَيْثُ كَانُوا سَابِقاً يَضَعُونَ التَّقْدِمَاتِ وَالْبَخُورَ وَالآنِيَةَ وَعُشْرَ الْقَمْحِ وَالْخَمْرِ وَالزَّيْتِ فَرِيضَةَ اللاَّوِيِّينَ وَالْمُغَنِّينَ وَالْبَوَّابِينَ وَرَفِيعَةَ الْكَهَنَةِ.
\par 6 وَفِي كُلِّ هَذَا لَمْ أَكُنْ فِي أُورُشَلِيمَ لأَنِّي فِي السَّنَةِ الاِثْنَتَيْنِ وَالثَّلاَثِينَ لأَرْتَحْشَسْتَا مَلِكِ بَابَِلَ دَخَلْتُ إِلَى الْمَلِكِ وَبَعْدَ أَيَّامٍ اسْتَأْذَنْتُ مِنَ الْمَلِكِ
\par 7 وَأَتَيْتُ إِلَى أُورُشَلِيمَ. وَفَهِمْتُ الشَّرَّ الَّذِي عَمِلَهُ أَلْيَاشِيبُ لأَجْلِ طُوبِيَّا بِعَمَلِهِ لَهُ مِخْدَعاً فِي دِيَارِ بَيْتِ اللَّهِ.
\par 8 وَسَاءَنِي الأَمْرُ جِدّاً وَطَرَحْتُ جَمِيعَ آنِيَةِ بَيْتِ طُوبِيَّا خَارِجَ الْمُِخْدَعِ
\par 9 وَأَمَرْتُ فَطَهَّرُوا الْمَخَادِعَ وَرَدَدْتُ إِلَيْهَا آنِيَةَ بَيْتِ اللَّهِ مَعَ التَّقْدِمَةِ وَالْبَخُورِ.
\par 10 وَعَلِمْتُ أَنَّ أَنْصِبَةَ اللاَّوِيِّينَ لَمْ تُعْطَ بَلْ هَرَبَ اللاَّوِيُّونَ وَالْمُغَنُّونَ عَامِلُو الْعَمَلِ كُلُّ وَاحِدٍ إِلَى حَقْلِهِ.
\par 11 فَخَاصَمْتُ الْوُلاَةَ وَقُلْتُ: [لِمَاذَا تُرِكَ بَيْتُ اللَّهِ؟] فَجَمَعْتُهُمْ وَأَوْقَفْتُهُمْ فِي أَمَاكِنِهِمْ.
\par 12 وَأَتَى كُلُّ يَهُوذَا بِعُشْرِ الْقَمْحِ وَالْخَمْرِ وَالزَّيْتِ إِلَى الْمَخَازِنِ
\par 13 وَأَقَمْتُ خَزَنَةً عَلَى الْخَزَائِنِ: شَلَمْيَا الْكَاهِنَ وَصَادُوقَ الْكَاتِبَ وَفَدَايَا مِنَ اللاَّوِيِّينَ وَبِجَانِبِهِمْ حَانَانَ بْنَُ زَكُّورَ بْنِ مَتَّنْيَا لأَنَّهُمْ حُسِبُوا أُمَنَاءَ وَكَانَ عَلَيْهِمْ أَنْ يَقْسِمُوا عَلَى إِخْوَتِهِمْ.
\par 14 اذْكُرْنِي يَا إِلَهِي مِنْ أَجْلِ هَذَا وَلاَ تَمْحُ حَسَنَاتِي الَّتِي عَمِلْتُهَا نَحْوَ بَيْتِ إِلَهِي وَنَحْوَ شَعَائِرِهِ.
\par 15 فِي تِلْكَ الأَيَّامِ رَأَيْتُ فِي يَهُوذَا قَوْماً يَدُوسُونَ مَعَاصِرَ فِي السَّبْتِ وَيَأْتُونَ بِحُزَمٍ وَيُحَمِّلُونَ حَمِيراً وَأَيْضاً يَدْخُلُونَ أُورُشَلِيمَ فِي يَوْمِ السَّبْتِ بِخَمْرٍ وَعِنَبٍ وَتِينٍ وَكُلِّ مَا يُحْمَلُ فَأَشْهَدْتُ عَلَيْهِمْ يَوْمَ بَيْعِهِمِ الطَّعَامَ.
\par 16 وَالصُّورِيُّونَ السَّاكِنُونَ بِهَا كَانُوا يَأْتُونَ بِسَمَكٍ وَكُلِّ بِضَاعَةٍ وَيَبِيعُونَ فِي السَّبْتِ لِبَنِي يَهُوذَا فِي أُورُشَلِيمَ.
\par 17 فَخَاصَمْتُ عُظَمَاءَ يَهُوذَا وَقُلْتُ لَهُمْ: [مَا هَذَا الأَمْرُ الْقَبِيحُ الَّذِي تَعْمَلُونَهُ وَتُدَنِّسُونَ يَوْمَ السَّبْتِ؟
\par 18 أَلَمْ يَفْعَلْ آبَاؤُكُمْ هَكَذَا فَجَلَبَ إِلَهُنَا عَلَيْنَا كُلَّ هَذَا الشَّرِّ وَعَلَى هَذِهِ الْمَدِينَةِ وَأَنْتُمْ تَزِيدُونَ غَضَباً عَلَى إِسْرَائِيلَ إِذْ تُدَنِّسُونَ السَّبْتَ].
\par 19 وَكَانَ لَمَّا أَظْلَمَتْ أَبْوَابُ أُورُشَلِيمَ قَبْلَ السَّبْتِ أَنِّي أَمَرْتُ بِأَنْ تُغْلَقَ الأَبْوَابُ وَقُلْتُ أَنْ لاَ يَفْتَحُوهَا إِلَى مَا بَعْدَ السَّبْتِ. وَأَقَمْتُ مِنْ غِلْمَانِي عَلَى الأَبْوَابِ حَتَّى لاَ يَدْخُلَ حِمْلٌ فِي يَوْمِ السَّبْتِ.
\par 20 فَبَاتَ التُّجَّارُ وَبَائِعُو كُلِّ بِضَاعَةٍ خَارِجَ أُورُشَلِيمَ مَرَّةً وَاثْنَتَيْنِ.
\par 21 فَأَشْهَدْتُ عَلَيْهِمْ وَقُلْتُ لَهُمْ: [لِمَاذَا أَنْتُمْ بَائِتُونَ بِجَانِبِ السُّورِ؟ إِنْ عُدْتُمْ فَإِنِّي أُلْقِي يَداً عَلَيْكُمْ]. وَمِنْ ذَلِكَ الْوَقْتِ لَمْ يَأْتُوا فِي السَّبْتِ.
\par 22 وَقُلْتُ لِلاَّوِيِّينَ أَنْ يَتَطَهَّرُوا وَيَأْتُوا وَيَحْرُسُوا الأَبْوَابَ لأَجْلِ تَقْدِيسِ يَوْمِ السَّبْتِ. بِهَذَا أَيْضاً اذْكُرْنِي يَا إِلَهِي وَتَرَأَّفْ عَلَيَّ حَسَبَ كَثْرَةِ رَحْمَتِكَ.
\par 23 فِي تِلْكَ الأَيَّامِ أَيْضاً رَأَيْتُ الْيَهُودَ الَّذِينَ سَاكَنُوا نِسَاءً أَشْدُودِيَّاتٍ وَعَمُّونِيَّاتٍ وَمُوآبِيَّاتٍ.
\par 24 وَنِصْفُ كَلاَمِ بَنِيهِمْ بِاللِّسَانِ الأَشْدُودِيِّ وَلَمْ يَكُونُوا يُحْسِنُونَ التَّكَلُّمَ بِاللِّسَانِ الْيَهُودِيِّ بَلْ بِلِسَانِ شَعْبٍ وَشَعْبٍ.
\par 25 فَخَاصَمْتُهُمْ وَلَعَنْتُهُمْ وَضَرَبْتُ مِنْهُمْ أُنَاساً وَنَتَفْتُ شُعُورَهُمْ وَاسْتَحْلَفْتُهُمْ بِاللَّهِ قَائِلاً: [لاَ تُعْطُوا بَنَاتِكُمْ لِبَنِيهِمْ وَلاَ تَأْخُذُوا مِنْ بَنَاتِهِمْ لِبَنِيكُمْ وَلاَ لأَنْفُسِكُمْ.
\par 26 أَلَيْسَ مِنْ أَجْلِ هَؤُلاَءِ أَخْطَأَ سُلَيْمَانُ مَلِكُ إِسْرَائِيلَ وَلَمْ يَكُنْ فِي الأُمَمِ الْكَثِيرَةِ مَلِكٌ مِثْلُهُ وَكَانَ مَحْبُوباً إِلَى إِلَهِهِ فَجَعَلَهُ اللَّهُ مَلِكاً علَى كُلِّ إِسْرَائِيلَ. هُوَ أَيْضاً جَعَلَتْهُ النِّسَاءُ الأَجْنَبِيَّاتُ يُخْطِئُ.
\par 27 فَهَلْ نَسْكُتُ لَكُمْ أَنْ تَعْمَلُوا كُلَّ هَذَا الشَّرِّ الْعَظِيمِ بِالْخِيَانَةِ ضِدَّ إِلَهِنَا بِمُسَاكَنَةِ نِسَاءٍ أَجْنَبِيَّاتٍ؟]
\par 28 وَكَانَ وَاحِدٌ مِنْ بَنِي يُويَادَاعَ بْنِ أَلْيَاشِيبَ الْكَاهِنِ الْعَظِيمِ صِهْراً لِسَنْبَلَّطَ الْحُورُونِيِّ فَطَرَدْتُهُ مِنْ عِنْدِي.
\par 29 اذْكُرْهُمْ يَا إِلَهِي لأَنَّهُمْ نَجَّسُوا الْكَهَنُوتَ وَعَهْدَ الْكَهَنُوتِ وَاللاَّوِيِّينَ.
\par 30 فَطَهَّرْتُهُمْ مِنْ كُلِّ غَرِيبٍ وَأَقَمْتُ حِرَاسَاتِ الْكَهَنَةِ وَاللاَّوِيِّينَ كُلَّ وَاحِدٍ عَلَى عَمَلِهِ
\par 31 وَلأَجْلِ قُرْبَانِ الْحَطَبِ فِي أَزْمِنَةٍ مُعَيَّنَةٍ وَلِلْبَاكُورَاتِ. فَاذْكُرْنِي يَا إِلَهِي بِالْخَيْرِ.

\end{document}