\begin{document}

\title{سيراخ}


\chapter{1}

\par 1 كل الحكمة تأتي من الرب وهي معه إلى الأبد.
\par 2 من يستطيع أن يعد رمل البحر وقطرات المطر وأيام الأبدية؟
\par 3 من يستطيع أن يعرف ارتفاع السماء وعرض الأرض والغمر والحكمة؟
\par 4 لقد خلقت الحكمة قبل كل شيء، والفهم منذ الأزل.
\par 5 كلمة الله العليا هي ينبوع الحكمة، وطرقها وصايا أبدية.
\par 6 لمن أُظهِر أصل الحكمة؟ أو من عرف نصائحها؟
\par 7 [لمن تجلت معرفة الحكمة؟ ومن أدرك خبرتها العظيمة؟]
\par 8 هناك واحد حكيم ومخيف للغاية، وهو الرب الجالس على عرشه.
\par 9 خلقها ورأها وأحصاها وسكبها على جميع أعماله.
\par 10 وهي مع كل جسد حسب عطيته، وقد أعطاها للذين يحبونه.
\par 11 مخافة الرب هي كرامة ومجد وفرح وإكليل ابتهاج.
\par 12 مخافة الرب تفرح القلب وتعطي الفرح والسرور وطول الحياة.
\par 13 من يتقي الرب يكون له خير في آخرته، ويجد نعمة في يوم موته.
\par 14 إن مخافة الرب هي بداية الحكمة، وقد خُلقت مع المؤمنين في البطن.
\par 15 لقد بنت أساسًا أبديًا مع الناس، وستستمر مع نسلهم.
\par 16 مخافة الرب هي ملء الحكمة، وتملأ الإنسان من ثمرها.
\par 17 تملأ كل بيتهم من الشهوات، والغلال من غلتها.
\par 18 إن مخافة الرب هي إكليل الحكمة، وتؤدي إلى ازدهار السلام والصحة الكاملة، وكلاهما عطية من الله، وهي تزيد فرح الذين يحبونه.
\par 19 الحكمة تمطر مهارة ومعرفة وفهمًا، وترفع إلى شرف أولئك الذين يتمسكون بها.
\par 20 جذر الحكمة مخافة الرب، وأغصانها طول الحياة.
\par 21 مخافة الرب تطرد الخطايا، وحيث توجد فإنها تطرد الغضب.
\par 22 الرجل الغاضب لا يتبرر، لأن سلطان غضبه يكون هلاكه.
\par 23 إن الرجل الصبور يتحمل لفترة من الوقت، وبعد ذلك ينبثق له الفرح.
\par 24 ويخفي كلامه إلى حين، وتعلن شفاه كثيرين حكمته.
\par 25 أمثال المعرفة في كنوز الحكمة، أما التقوى فهي رجس عند الخاطئ.
\par 26 إن كنت تريد الحكمة فاحفظ الوصايا، فيعطيك الرب إياها.
\par 27 لأن مخافة الرب هي الحكمة والتأديب، والإيمان والوداعة هما مرضاته.
\par 28 لا تثق بمخافة الرب حين تكون فقيراً، ولا تأت إليه بقلب مزدوج.
\par 29 لا تكن منافقًا أمام الناس، وانتبه لما تقول.
\par 30 لا ترتفع لئلا تسقط وتجلب العار على نفسك، فيكتشف الله أسرارك ويطرحك في وسط الجماعة، لأنك لم تأت بالحق إلى مخافة الرب، بل كان قلبك مملوءًا بالخداع.

\chapter{2}

\par 1 يا ابني، إذا أتيت لخدمة الرب، فأعد نفسك للتجربة.
\par 2 صحح قلبك، واصبر دائمًا، ولا تتعجل في وقت الضيق.
\par 3 تمسك به ولا تبتعد عنه لكي تكثر في آخرتك.
\par 4 مهما حدث لك، خذه بصدر رحب، وكن صبوراً عندما تتحول إلى حالة منخفضة.
\par 5 فإن الذهب يُمتحن في النار، والرجال المقبولون في كور الشدة.
\par 6 آمن به، وسيساعدك، ورتب طريقك بشكل صحيح، وثق به.
\par 7 يا خائفي الرب انتظروا رحمته ولا تحيدوا لئلا تسقطوا.
\par 8 يا أيها الذين تخافون الرب، آمنوا به، فيكون أجركم لا يضيع.
\par 9 يا من تخافون الرب، أرجو الخير والفرح الأبدي والرحمة.
\par 10 انظروا إلى الأجيال القديمة وانظروا هل توكل أحد على الرب فخجل أو ثبت أحد في مخافته فترك أو من احتقر من دعاه؟
\par 11 لأن الرب رحيم ورؤوف، طويل الروح ورحيم جداً، ويغفر الخطايا ويخلص في وقت الضيق.
\par 12 ويل للقلوب الخائفة، والأيدي المرتخية، والخاطئ الذي يسير في طريقين!
\par 13 ويل للذي يضعف قلبه، لأنه لا يؤمن، ولذلك لا يدافع عنه.
\par 14 ويل لكم أيها الذين فقدتم صبركم! وماذا ستفعلون عندما يفتقدكم الرب؟
\par 15 الذين يخافون الرب لا يعصون كلامه والذين يحبونه يحفظون طرقه.
\par 16 الذين يخافون الرب يطلبون ما هو حسن ومقبول عنده، والذين يحبونه يمتلئون بالناموس.
\par 17 "إن الذين يخافون الرب يعدون قلوبهم ويذلون نفوسهم أمامه،
\par 18 قائلين: نقع في يدي الرب لا في أيدي الناس، لأنه على قدر عظمته تكون رحمته.

\chapter{3}

\par 1 اسمعوا لي يا أبنائي كلام أبيكم وافعلوا به لكي تنجووا.
\par 2 لأن الرب أعطى الأب كرامة على الأولاد، وأثبت سلطة الأم على الأبناء.
\par 3 من أكرم أباه فهو يكفر خطاياه.
\par 4 ومن يكرم أمه فهو كمُدَّخِر الكنوز.
\par 5 من يكرم أباه يفرح بأولاده، وعندما يصلي يُسمع له.
\par 6 من يكرم أباه يطول عمره، ومن يطيع الرب يعزّي أمه.
\par 7 من يتقي الرب يكرم أباه ويخدم والديه كسادته.
\par 8 أكرم أباك وأمك في القول والفعل، لكي تأتي عليك البركة منهم.
\par 9 لأن بركة الأب تثبت بيوت الأبناء، ولعنة الأم تقتلع الأساسات.
\par 10 لا تفتخر بهوان أبيك، لأن هوان أبيك ليس فخراً لك.
\par 11 فإن فخر الرجل من كرامة أبيه، والأم في هوانها عار عند الأبناء.
\par 12 يا بني، أعن أباك في شيخوخته، ولا تحزنه ما دام حياً.
\par 13 وإن ضعف فهمه، فاصبر عليه، ولا تحتقره وأنت في كامل قوتك.
\par 14 لأنه لن يُنسى إغاثة أبيك، وعوضاً عن خطاياك يُضاف إليك لبنيانك.
\par 15 في يوم ضيقك تذكر خطاياك وتذوب كما يذوب الجليد في الجو الجميل الدافئ.
\par 16 من يترك أباه فهو كمجدف، ومن يغضب أمه فهو ملعون من الله.
\par 17 يا ابني، قم بأعمالك بوداعة، هكذا تكون محبوباً من قبل من هو مقبول.
\par 18 كلما كنت عظيماً، كلما كنت متواضعاً أكثر، وستجد نعمة أمام الرب.
\par 19 كثيرون هم في مناصب عليا وذوي شهرة، ولكن الأسرار تنكشف للودعاء.
\par 20 لأن قدرة الرب عظيمة، ويُكرَّم من المتواضعين.
\par 21 لا تبحث عن الأشياء التي هي صعبة عليك، ولا تبحث عن الأشياء التي هي فوق قدرتك.
\par 22 ولكن ما أمرت به فاحترمه، لأنه ليس من الضروري أن ترى بعينيك الأمور الخفية.
\par 23 لا تكن فضوليًا في الأمور غير الضرورية، لأنه يُظهر لك أشياء أكثر مما يفهمها الناس.
\par 24 لأن كثيرين قد خدعوا برأيهم الباطل، والظن الشرير قلب حكمهم.
\par 25 بدون عيونك سوف تفتقر إلى النور: فلا تعترف بالمعرفة التي لا تملكها.
\par 26 القلب العنيد يفسد في النهاية، ومن يحب الخطر يهلك فيه.
\par 27 القلب العنيد يكون مثقلا بالأحزان، والشرير يجمع خطيئة فوق خطيئة.
\par 28 في عقاب المتكبرين ليس هناك علاج، لأن غرس الشر قد تأصل فيه.
\par 29 قلب العاقل يفهم المثل، والأذن المصغية هي رغبة الرجل الحكيم.
\par 30 الماء يطفئ النار المشتعلة، والصدقة تكفر الخطايا.
\par 31 ومن يجازي الحسنات يتذكر ما قد يكون في الآخرة، وإذا سقط يجد مأوى.

\chapter{4}

\par 1 يا بني لا تحرم الفقير من رزقه، ولا تجعل عيون المحتاج تنتظر طويلا.
\par 2 لا تحزن نفساً جائعة، ولا تزعج إنساناً في ضيقه.
\par 3 لا تزيدوا قلباً حزيناً، ولا تؤخروا إعطاء المحتاج.
\par 4 لا ترفض دعاء المسكين، ولا تحول وجهك عن الفقير.
\par 5 لا تصرف نظرك عن المسكين، ولا تجعل له سببا ليلعنك.
\par 6 لأنه إذا لعنك في مرارة نفسه، فسوف تُسمع صلاته من قبل صانعه.
\par 7 احصل على حب الجماعة، وانحنِ رأسك لرجل عظيم.
\par 8 لا يحزنك أن تصغي إلى الفقير وتعطيه جواباً ودوداً بوداعة.
\par 9 أنقذ المظلوم من يد الظالم، ولا تضعف عندما تجلس في القضاء.
\par 10 كن كالأب لليتيم وزوجاً لأمهاتهم، هكذا تكون كابن العلي، فيحبك أكثر من أمك.
\par 11 الحكمة ترفع أولادها، وتأسر الذين يطلبونها.
\par 12 من أحبها أحب الحياة، ومن يبكر إليها يمتلئ فرحاً.
\par 13 من تمسك بها يرث مجدا، وحيثما دخلت يبارك الرب.
\par 14 الذين يخدمونها يخدمون القدوس والذين يحبونها يحبهم الرب.
\par 15 من يسمع لها يدين الأمم، ومن يهتم بها يسكن آمنا.
\par 16 إذا التزم رجل بها فإنه يرثها، وترثها نسله.
\par 17 فإنها في البداية سوف تسير معه في طرق ملتوية، وتجلب عليه الخوف والرعب، وتعذبه بتأديبها، حتى تثق في روحه، وتختبره بأحكامها.
\par 18 ثم تعود إليه في الطريق المستقيم، وتعزيه، وتكشف له أسرارها.
\par 19 ولكن إذا أخطأ فإنها ستتخلى عنه وتسلّمه إلى دماره.
\par 20 اغتنم الفرصة، واحذر من الشر، ولا تخجل عندما يتعلق الأمر بنفسك.
\par 21 لأن هناك عارًا يجلب الخطيئة، وهناك عارًا هو مجد ونعمة.
\par 22 لا تقبل أي إنسان ضد نفسك، ولا تدع احترام أي إنسان يجعلك تسقط.
\par 23 ولا تمتنع عن الكلام عند الحاجة إلى فعل الخير، ولا تخف حكمتك في جمالها.
\par 24 لأنه من الكلام تعرف الحكمة، ومن كلام اللسان تعرف التعلم.
\par 25 لا تتكلم ضد الحقيقة بأي حال من الأحوال، ولكن اخجل من خطأ جهلك.
\par 26 لا تخجل من الاعتراف بذنوبك، ولا تعيق مجرى النهر.
\par 27 لا تجعل نفسك تابعا لرجل جاهل، ولا تقبل وجه الأقوياء.
\par 28 جاهد في سبيل الحق حتى الموت، والرب يقاتل عنك.
\par 29 لا تكن متسرعا في لسانك، ولا متراخيا في أعمالك.
\par 30 لا تكن كالأسد في بيتك، ولا كالجبار بين عبيدك.
\par 31 لا تكن يدك ممدودة للأخذ، ومغلقة عندما يجب عليك السداد.

\chapter{5}

\par 1 ضع قلبك على أموالك، ولا تقل: لدي ما يكفي حياتي.
\par 2 لا تتبع عقلك وقوتك للسير في طرق قلبك.
\par 3 ولا تقل من يحاسبني على أعمالي فإن الرب ينتقم لكبريائك.
\par 4 لا تقل أخطأت فأي شر أصابني لأن الرب طويل الأناة فلا يتركك.
\par 5 أما فيما يتعلق بالكفارة فلا تخف من إضافة خطيئة إلى خطيئة.
\par 6 ولا تقل إن رحمته عظيمة، فإنه يرضى عن كثرة خطاياي، لأن الرحمة والغضب يأتيان منه، وسخطه يحل على الخطاة.
\par 7 لا تتأخروا عن الرجوع إلى الرب، ولا تؤجلوا من يوم إلى يوم، لأنه بغتة يخرج غضب الرب، فتهلكوا في أمانكم وتبادوا في يوم الانتقام.
\par 8 لا تضع قلبك على الأموال التي حصلت عليها بغير حق، فإنها لن تنفعك في يوم الكارثة.
\par 9 لا تغربلوا مع كل ريح، ولا تسيروا في كل طريق، لأن هكذا يفعل الخاطئ ذو اللسان المزدوج.
\par 10 كن ثابتًا في فهمك، وليكن كلامك واحدًا.
\par 11 "كن سريعًا في الاستماع، ولتكن حياتك صادقة، وأجب بصبر."
\par 12 إن كان لديك فهم فأجب قريبك، وإلا فضع يدك على فمك.
\par 13 الشرف والعار في الكلام، ولسان الإنسان هو سقوطه.
\par 14 لا تُدعَ هامساً، ولا تكمن بلسانك، لأنه عارا قبيحا على السارق، ودينونة رديئة على اللسان المزدوج.
\par 15 لا تجهل شيئا في أمر كبير أو صغير.

\chapter{6}

\par 1 لا تصير عدواً عوضاً عن الصديق، لأنك بذلك ترث سمعة سيئة، وعاراً، وعاراً، وكذلك الخاطئ ذو اللسان المزدوج.
\par 2 لا تفتخر بمشورة قلبك، لئلا تتمزق نفسك كالثور الضال.
\par 3 فتأكل أوراقك، وتفقد ثمرتك، وتترك نفسك كالشجرة اليابسة.
\par 4 النفس الشريرة تهلك صاحبها وتجعله موضع سخرية أعدائه.
\par 5 إن الكلام الحلو يزيد الأصدقاء، واللسان الجميل يزيد التحيات الطيبة.
\par 6 كن في سلام مع كثيرين، ولكن لا يكن لك إلا مستشار واحد من بين ألف.
\par 7 إذا كنت تريد أن تحصل على صديق، اختبره أولاً ولا تتعجل في منحه الثقة.
\par 8 فمنهم من يكون صديقاً لظروفه الخاصة، ولا يثبت في يوم ضيقك.
\par 9 وهناك صديق، إذا تحول إلى العداوة والنزاع، سيكشف عارك.
\par 10 وأيضاً، فإن بعض الأصدقاء هو رفيقك على المائدة، ولن يستمر في يوم ضيقك.
\par 11 "ولكن في رخائك يكون مثلك، ويكون واثقاً على عبيدك."
\par 12 إذا كنت متواضعًا، فإنه سيكون ضدك، وسيختبئ من وجهك.
\par 13 انعزل عن أعدائك، واحذر من أصدقائك.
\par 14 الصديق الأمين هو دفاع قوي، ومن وجد مثل هذا الصديق فقد وجد كنزًا.
\par 15 لا شيء يعوض الصديق الوفي، وفضله لا يقدر بثمن.
\par 16 الصديق الأمين هو دواء الحياة، ومن يتقي الرب يجده.
\par 17 من يتقي الرب يهدي صداقته إلى الصواب، لأنه كما هو يكون قريبه.
\par 18 يا ابني، اجمع التأديب منذ شبابك، لكي تجد الحكمة إلى شيخوختك.
\par 19 تعال إليها كالحارث والزارع وانتظر ثمرها الصالح، لأنك لن تتعب في عملها كثيراً، بل ستأكل من ثمرها سريعاً.
\par 20 إنها غير سارة للغاية لغير المتعلمين: من ليس لديه فهم لن يبقى معها.
\par 21 فتقع عليه كحجر اختبار عظيم، فيطرحها عنه إلى أن يطول الزمن.
\par 22 لأن الحكمة هي حسب اسمها، وهي ليست ظاهرة للكثيرين.
\par 23 اسمع يا ابني، واقبل نصيحتي، ولا ترفض مشورتي،
\par 24 وأدخل قدميك في قيودها وعنقك في قيودها.
\par 25 انحنِ على كتفك، واحملها، ولا تحزن من قيودها.
\par 26 تعال إليها بكل قلبك، واحفظ طرقها بكل قوتك.
\par 27 ابحث وابحث، وسوف تُعرَف لك. ومتى أمسكتها، فلا تدعها تذهب.
\par 28 لأنك في النهاية سوف تجد راحتها، وهذا سوف يتحول إلى فرحك.
\par 29 فتكون لك قيودها درعاً قوياً، وسلاسلها رداء مجد.
\par 30 لأنها عليها حلية من ذهب وشرائطها دانتيل أرجواني.
\par 31 تلبسها ثوب الشرف، وتضعها عليك إكليل الفرح.
\par 32 يا ابني، إذا أردت فسوف تتعلم، وإذا استخدمت عقلك فسوف تصبح حكيماً.
\par 33 إن أحببت أن تسمع تتلقى الفهم، وإن أسندت أذنك تصير حكيماً.
\par 34 قف في جماعة الشيوخ، والزم الحكيم.
\par 35 "كن مستعدًا لسماع كل حديث صالح، ولا تدع أمثال الفهم تفوتك."
\par 36 وإذا رأيت رجلاً ذا فهم، فاذهب إليه مبكراً، ودع رجلك تطأ درجات بابه.
\par 37 لتكن أفكارك على أحكام الرب، وتأمل في وصاياه دائمًا. فهو يثبت قلبك، ويعطيك الحكمة حسب رغبتك.

\chapter{7}

\par 1 لا تفعل الشر، حتى لا يصيبك الأذى.
\par 2 ابتعد عن الظالم فيبتعد الإثم عنك.
\par 3 يا ابني لا تزرع في عتبات الظلم فلا تحصد سبعة أضعاف.
\par 4 لا تطلبوا من الرب الرئاسة، ولا من الملك كرسي الكرامة.
\par 5 لا تبرّر نفسك أمام الرب، ولا تفتخر بحكمتك أمام الملك.
\par 6 لا تسعَ إلى أن تكون قاضياً، لأنك لا تستطيع أن تزيل الإثم، لئلا تخاف من وجه القوي، حجر عثرة في طريق استقامتك.
\par 7 لا تخطئ إلى جمهور المدينة، لئلا تطرح نفسك بين الشعب.
\par 8 لا تربط خطيئة على خطيئة، لأنه في واحدة لن تتبرأ.
\par 9 لا تقل إن الله سوف ينظر إلى كثرة تقدماتي، وعندما أقدمها إلى الله العلي، فإنه سوف يقبلها.
\par 10 لا تيأس من صلاتك، ولا تقصر في الصدقات.
\par 11 لا تسخر من أحد في مرارة نفسه، لأنه يوجد من يذل ويرفع.
\par 12 لا تخترع كذبة على أخيك، ولا تفعل مثلها مع صديقك.
\par 13 لا ترتكب أي نوع من الكذب، لأن العادة المتبعة في ذلك ليست جيدة.
\par 14 لا تستخدم الكثير من الكلمات في جمع من الشيوخ، ولا تثرثر كثيرًا عندما تصلي.
\par 15 لا تكره العمل الشاق ولا الزراعة التي أمر بها العلي.
\par 16 لا تحسب نفسك ضمن جمهور الخطاة، بل تذكر أن الغضب لن يدوم طويلاً.
\par 17 تواضعوا كثيراً، لأن انتقام الأشرار هو النار والديدان.
\par 18 لا تغير صديقك بالخير أبدًا، ولا الأخ الأمين بذهب أوفير.
\par 19 لا تترك امرأة حكيمة وصالحة، لأن نعمتها أعظم من الذهب.
\par 20 بينما يعمل عبدك بالحق فلا تطلب منه الشر ولا من الأجير الذي يبذل نفسه لك بالكامل.
\par 21 لتحب نفسك العبد الصالح، ولا تحرمه من حريته.
\par 22 هل لديك ماشية؟ فاحتفظ بها، وإذا كانت لمنفعتك، فاحتفظ بها معك.
\par 23 هل لك أولاد فأرشدهم وأخضعهم منذ صغرهم.
\par 24 هل لك بنات؟ اهتم بأجسادهن ولا تظهر لهنّ سرورك.
\par 25 تزوج ابنتك، وتكون قد فعلت أمراً عظيماً. ولكن أعطها لرجل ذي فهم.
\par 26 هل لك زوجة حسب رأيك فلا تتركها، ولكن لا تسلم نفسك لامرأة عادية.
\par 27 أكرم أباك بكل قلبك، ولا تنس أحزان أمك.
\par 28 اذكر أنك ولدت منهم، فكيف تكافئهم على ما فعلوا لك؟
\par 29 اتق الرب بكل نفسك، واحترم كهنته.
\par 30 أحبب الذي خلقك بكل قوتك، ولا تترك خدامه.
\par 31 اتق الرب وأكرم الكاهن وأعطه نصيبه كما أمرتك: البكورة وذبيحة الإثم وتقدمة الكتف وذبيحة التقديس وباكورة الأقداس.
\par 32 ومد يدك إلى الفقير لكي تكتمل بركتك.
\par 33 إن العطية لها نعمة أمام كل حي، ولا تمنعها عن الأموات.
\par 34 لا تفشل في أن تكون مع الباكين، وتحزن مع الحزانى.
\par 35 لا تتأخر عن زيارة المرضى، فهذا سيجعلك محبوبًا.
\par 36 مهما كان ما تأخذه في يدك، تذكر النهاية، ولن تخطئ أبدًا.

\chapter{8}

\par 1 لا تصارع رجلاً قوياً لئلا تقع في يديه.
\par 2 لا تخاصم الغني لئلا يثقل عليك، لأن الذهب أهلك كثيرين وأفسد قلوب الملوك.
\par 3 لا تخاصم رجلاً ثرثار اللسان، ولا تضع حطبًا على ناره.
\par 4 لا تمزح مع الرجل الفظ، لئلا يخجل أسلافك.
\par 5 لا تعاتب أحداً يبتعد عن الخطيئة، بل تذكر أننا جميعاً نستحق العقاب.
\par 6 لا تخجل أحداً في شيخوخته، لأن بعضنا يشيخ أيضاً.
\par 7 لا تفرح بموت أعظم أعدائك، بل تذكر أننا جميعًا نموت.
\par 8 لا تحتقر كلام الحكماء، بل تعرف على أمثالهم، لأنك منهم تتعلم التأديب، وكيف تخدم العظماء بسهولة.
\par 9 لا تفوتك كلمة الشيوخ، لأنهم تعلموا أيضًا من آبائهم، ومنهم تتعلم الفهم والإجابة حسب الحاجة.
\par 10 لا تشعل جمر الخاطئ لئلا تحترق بلهيب ناره.
\par 11 لا تغضب من أمام من أساء إليك، لئلا ينصب لك كمينًا في كلامك.
\par 12 لا تقرض من هو أقوى منك، لأنه إذا أقرضته فاعتبره خاسرا.
\par 13 لا تكن ضامنًا فوق طاقتك، فإذا كنت ضامنًا فاحرص على دفعه.
\par 14 لا تحاكم القاضي، لأنه سيحكم له حسب كرامته.
\par 15 لا تسر في الطريق مع رجل جريء لئلا يزعجك، لأنه سيفعل حسب إرادته، وأنت تهلك معه بسبب جهالة.
\par 16 لا تخاصم رجلاً غاضباً، ولا تدخل معه إلى مكان منعزل. لأن الدم لا قيمة له في عينيه، وحيث لا خلاص يهلكك.
\par 17 لا تتشاور مع الأحمق، فإنه لا يستطيع أن يحفظ المشورة.
\par 18 لا تفعل أمراً سرّياً أمام الغريب، لأنك لا تعلم ماذا سيلد.
\par 19 لا تفتح قلبك لكل إنسان، لئلا يجازيك بالمثل.

\chapter{9}

\par 1 لا تغار على امرأة حضنك، ولا تعلمها درساً سيئاً ضد نفسك.
\par 2 لا تعطي نفسك لامرأة لتضع قدمها على ممتلكاتك.
\par 3 لا تلتقي بزانية لئلا تقع في شركها.
\par 4 لا تستغل صحبة امرأة مغنية كثيراً، حتى لا تتأثر بمحاولاتها.
\par 5 لا تنظر إلى العذراء، حتى لا تقع في غرام الأشياء الثمينة فيها.
\par 6 لا تُسلِّم نفسك للزناة، لئلا تفقد ميراثك.
\par 7 لا تنظر حولك في شوارع المدينة، ولا تتجول في المكان المنعزل منها.
\par 8 صرف نظرك عن المرأة الجميلة، ولا تنظر إلى جمال غيرك، فإن كثيرين قد خدعوا بجمال المرأة، فبهذا يشتعل الحب كالنار.
\par 9 لا تجلس مطلقا مع امرأة رجل آخر، ولا تجلس معها بين ذراعيك، ولا تنفق أموالك معها على الخمر، لئلا يميل قلبك إليها، فتسقط بسبب رغبتك في الهلاك.
\par 10 لا تترك صديقًا قديمًا، لأن الجديد لا يُعادله. الصديق الجديد كالخمر الجديدة، إذا عتقت شربتها بلذة.
\par 11 لا تحسد على مجد الخاطئ، لأنك لا تعلم ما هي نهايته.
\par 12 لا تفرح بما يسعد به الأشرار، بل تذكر أنهم لن يذهبوا إلى قبورهم دون عقاب.
\par 13 "ابتعد عن الرجل الذي له سلطان على القتل، لئلا تشك في خوف الموت، وإذا أتيت إليه فلا تخطئ، لئلا يأخذ حياتك في الحال. تذكر أنك تسير في وسط الفخاخ، وأنك تمشي على أسوار المدينة".
\par 14 بقدر ما تستطيع، حاول أن تخمن من جارك، واستشر الحكماء.
\par 15 لتكن أحاديثك مع الحكماء، وكل كلامك في شريعة العلي.
\par 16 "وليأكل ويشرب معك الصديقون، وليكن افتخارك في مخافة الرب."
\par 17 لأن يد الصانع تُمدح عمله، والحاكم الحكيم للشعب يُمدح من أجل كلامه.
\par 18 الرجل ذو اللسان السيئ يكون خطرا في مدينته، ​​والمتسرع في كلامه يكون مكروهاً.

\chapter{10}

\par 1 القاضي الحكيم يعلم شعبه، وحكم الرجل الحكيم يكون منظما.
\par 2 كما أن قاضي الشعب هو نفسه، كذلك حكامه، وكما أن رئيس المدينة هو نفسه، كذلك جميع سكانها.
\par 3 الملك غير الحكيم يهلك شعبه، ولكن بفطنة أصحاب السلطة تسكن المدينة.
\par 4 إن سلطان الأرض في يد الرب، وفي الوقت المناسب سيقيم عليها من يكون نافعاً.
\par 5 في يد الله سلامة الإنسان وعلى وجه الكاتب يضع كرامته.
\par 6 لا تحمل الكراهية لقريبك بسبب كل خطأ، ولا تفعل أي شيء على الإطلاق من خلال الممارسات الضارة.
\par 7 إن الكبرياء مكروه عند الله وعند الناس، وبهما يرتكب الإنسان الإثم.
\par 8 وبسبب المعاملات غير الصالحة، والأضرار، والثروات المكتسبة بالخداع، تنتقل المملكة من شعب إلى آخر.
\par 9 لماذا تفتخر الأرض والرماد؟ ليس هناك شيء أشر من الإنسان الطماع، لأنه يبيع نفسه، لأنه في حياته يرمي أحشائه.
\par 10 الطبيب يقطع مرضًا طويلًا، ومن هو اليوم ملك غدًا سوف يموت.
\par 11 لأنه عندما يموت الإنسان، فإنه يرث الزواحف والوحوش والديدان.
\par 12 بداية الكبرياء هي الابتعاد عن الله، وانحراف قلبه عن خالقه.
\par 13 لأن الكبرياء هو بدء الخطيئة، والذي فيه يسكب الرجس. لذلك جلب الرب عليهم شرورًا غريبة، وهدمهم تمامًا.
\par 14 لقد هدم الرب عروش الأمراء المتكبرين، وأقام الودعاء مكانهم.
\par 15 اقتلع الرب جذور الأمم المتكبرة، وغرس المتواضعين في مكانهم.
\par 16 فقلب الرب بلاد الأمم ودمرها إلى أساس الأرض.
\par 17 فأخذ بعضهم ودمرهم وأزال ذكرهم من الأرض.
\par 18 لم يُخلق الكبرياء من أجل الرجال، ولا الغضب الشديد من أجل المولودين من امرأة.
\par 19 الذين يخافون الرب هم بذرة آمنة والذين يحبونه هم زرع مكرَّم. الذين لا يحترمون الناموس هم بذرة هوان. الذين يتعدون الوصايا هم بذرة ضالة.
\par 20 بين الإخوة الرئيس هو مكرم، وكذلك الذين يخافون الرب في عينيه.
\par 21 إن مخافة الرب تسبق الحصول على السلطة، ولكن الخشونة والكبرياء هي سبب فقدانها.
\par 22 سواء كان غنيًا أو نبيلًا أو فقيرًا، فإن مجده هو مخافة الرب.
\par 23 لا ينبغي أن نحتقر الفقير الذي له فهم، ولا ينبغي أن نعظم الرجل الخاطئ.
\par 24 يُكرَّم العظماء والقضاة والأمراء، ولكن ليس أحد أعظم من الذي يتقي الرب.
\par 25 "لأجل العبد الحكيم يخدم الأحرار، وصاحب المعرفة لا يتذمر إذا أصلح."
\par 26 لا تكن حكيماً في عملك، ولا تفتخر في وقت ضيقك.
\par 27 من يتعب ويتغنى في كل شيء خير من من يفتخر ويفتقر إلى الخبز.
\par 28 يا ابني، مجّد نفسك بالوداعة، وأعطها شرفًا وفقًا لكرامتها.
\par 29 فمن يبرر من يخطئ إلى نفسه؟ ومن يكرم من يهين نفسه؟
\par 30 الرجل الفقير يُكرَّم لمهارته، والرجل الغني يُكرَّم لغناه.
\par 31 من يكرم في الفقر فكم بالأولى في الغنى؟ ومن لا يكرم في الغنى فكم بالأولى في الفقر؟

\chapter{11}

\par 1 الحكمة ترفع رأس المتضع، وتجلسه بين العظماء.
\par 2 لا تمدح الرجل لجماله، ولا تكره الرجل لمظهره الخارجي.
\par 3 النحلة صغيرة بين الذباب، لكن ثمرها هو أطيب الأشياء.
\par 4 لا تفتخر بلباسك ولباسك، ولا ترتفع في يوم الكرامة، لأن أعمال الرب عجيبة، وأعماله بين الناس خفية.
\par 5 لقد جلس العديد من الملوك على الأرض، وواحد لم يفكر فيه أحد هو الذي ارتدى التاج.
\par 6 لقد أُهين كثير من الرجال الأقوياء إلى حد كبير، وأُسلم الشرفاء إلى أيدي آخرين.
\par 7 لا تلوم قبل أن تفحص الحقيقة. افهم أولاً، ثم وبخ.
\par 8 لا تجيب قبل أن تسمع السبب، ولا تقاطع الرجال في منتصف حديثهم.
\par 9 لا تخاصم في أمر لا يعنيك، ولا تجلس في المحاكمة مع الخطاة.
\par 10 يا بني لا تتدخل في أمور كثيرة، لأنه إذا تدخلت كثيرا لن تكون بريئا، وإذا تابعت فلن تحصل، ولن تنجو بالفرار.
\par 11 هناك من يتعب ويتعب ويستعجل، ولكنه يتأخر كثيرًا.
\par 12 "وهناك أيضًا آخر بطيء، ويحتاج إلى المساعدة، ويفتقر إلى القدرة، ومليء بالفقر؛ ومع ذلك نظرت إليه عين الرب بالخير، وأقامته من حالته المتواضعة،
\par 13 ورفع رأسه من الشقاء، حتى أن كثيرين ممن رأوه كان سلاماً على كل الأرض.
\par 14 الرخاء والشدة، الحياة والموت، الفقر والغنى، تأتي من الرب.
\par 15 الحكمة والمعرفة وفهم الشريعة من الرب. المحبة وطريق الأعمال الصالحة من عنده.
\par 16 لقد بدأ الخطأ والظلام مع الخطاة، وسوف يشيخ الشر مع أولئك الذين يفتخرون به.
\par 17 هبة الرب تبقى مع الأشرار، ورضاه يجلب الرخاء إلى الأبد.
\par 18 هناك من يصبح غنيًا بحذره واقتصاده، وهذا هو نصيبه من مكافأته:
\par 19 بينما يقول: لقد وجدت الراحة، والآن سوف آكل دائمًا من أموالي؛ ومع ذلك فهو لا يعرف أي وقت سيأتي عليه، وأنه يجب عليه أن يترك تلك الأشياء للآخرين ويموت.
\par 20 ثبت في عهدك، وكن على دراية به، وشاخ في عملك.
\par 21 لا تتعجب من أعمال الخطاة، بل ثق بالرب واثبت في تعبك، فإنه هين في عيني الرب غنى الفقير فجأة.
\par 22 بركة الرب في مكافأة الصديق، وفجأة يجعل بركته تزدهر.
\par 23 لا تقل ما الفائدة من خدمتي وما الخير الذي سأحصل عليه بعد ذلك؟
\par 24 مرة أخرى، لا تقل، لدي ما يكفي، وأمتلك أشياء كثيرة، وما هو الشر الذي سأصاب به بعد ذلك؟
\par 25 في يوم الرخاء ينسى الضيق، وفي يوم الضيق لا يكون ذكر الرخاء.
\par 26 فإنه هين على الرب يوم الموت أن يجازي الإنسان حسب طرقه.
\par 27 إن ضيق ساعة ينسي الإنسان لذته، وفي آخرته تنكشف أعماله.
\par 28 لا تدينوا أحداً مباركاً قبل موته، لأن الإنسان يعرف من خلال أبنائه.
\par 29 لا تدخل كل إنسان إلى بيتك، لأن الرجل الغشاش له أوتار كثيرة.
\par 30 مثل الحجل الذي يؤخذ ويحفظ في قفص، كذلك قلب المتكبر، ومثل الجاسوس يراقب سقوطك.
\par 31 فإنه يكمن في الواقع، ويحول الخير إلى شر، ويلقي اللوم عليك في الأمور التي تستحق الثناء.
\par 32 من شرارة نار تشتعل جمراً والرجل الخاطئ يكمن للدم.
\par 33 احذر من الرجل المنافق فإنه يعمل الشر، لئلا يجلب عليك عارا أبديا.
\par 34 إذا أدخلت غريباً إلى بيتك فإنه يزعجك ويطردك من بيتك.

\chapter{12}

\par 1 عندما تريد أن تفعل الخير فاعلم لمن تفعله، وبذلك سوف تشكر على حسناتك.
\par 2 أحسن إلى الرجل التقي فتجد جزاء، وإن لم يكن منه فمن العلي.
\par 3 لا يمكن أن يأتي الخير إلى من هو مشغول دائمًا بالشر، ولا إلى من لا يعطي الصدقات.
\par 4 أعطِ التقيَّ ولا تُساعد الخاطئ.
\par 5 أحسن إلى المتواضع، ولكن لا تعطه للكافر. احتفظ بخبزك ولا تعطه له لئلا يتسلط عليك به، وإلا فإنك ستأخذ ضعف الشر مقابل كل الخير الذي قدمته له.
\par 6 لأن العلي يبغض الخطاة، وينتقم من الأشرار، ويحفظهم من يوم عقابهم العظيم.
\par 7 أعطِ الصالح ولا تساعد الخاطئ.
\par 8 لا يمكن معرفة الصديق في الرخاء، ولا يمكن إخفاء العدو في الشدة.
\par 9 في رخاء الإنسان يحزن الأعداء، ولكن في شدته يبتعد عنه الصديق.
\par 10 لا تثق أبدًا بعدوك، لأنه كما يصدأ الحديد، كذلك شره.
\par 11 وإن تواضع وجلس منحنياً، فاحذر منه واحذره، وسوف تكون معه كما لو كنت قد مسحت مرآة، وسوف تعلم أن صدأه لم يُمحى بالكامل.
\par 12 لا تجعله بجانبك لئلا يقلبك ويقوم في مكانك. ولا يجلس عن يمينك لئلا يحاول أن يجلس على مقعدك، فتذكر كلامي في النهاية وتنخس به.
\par 13 من يشفق على الساحر الذي لدغته الحية، أو على كل من يقترب من الوحوش البرية؟
\par 14 فمن يذهب إلى الخاطئ ويتنجس معه في خطاياه فمن يشفق؟
\par 15 فإنه سوف يبقى معك لفترة من الوقت، ولكن إذا بدأت في السقوط، فإنه لن يتأخر.
\par 16 العدو يتكلم بلطف بشفتيه، لكنه في قلبه يفكر كيف يلقيك في الحفرة. سوف يبكي بعينيه، ولكن إذا وجد فرصة، فلن يكتفي بالدم.
\par 17 إذا جاءتك الشدة، فسوف تجده هناك أولاً؛ ورغم أنه يتظاهر بمساعدتك، إلا أنه سوف يقوضك.
\par 18 فيحرك رأسه ويصفق بيديه ويهمس كثيرا ويغير ملامح وجهه.

\chapter{13}

\par 1 من يمس القار يتنجس به، ومن يخالط المتكبر يكون مثله.
\par 2 لا تثقل نفسك فوق طاقتك وأنت حي، ولا تكن لك شركة مع من هو أقوى منك وأغنى منك، لأنه كيف يتفق القدر والخزف؟ فإنه إن ضرب أحدهما الآخر انكسر.
\par 3 لقد أخطأ الغني، ومع ذلك يهدد. أما الفقير فقد ظُلِم، ويجب عليه أن يتضرع أيضًا.
\par 4 إذا كنت لمصلحته، فإنه سوف يستخدمك، ولكن إذا لم يكن لديك شيء، فإنه سوف يتخلى عنك.
\par 5 إذا كان لديك أي شيء، فإنه سوف يعيش معك، نعم، فإنه سوف يجعلك عارياً، ولن يندم على ذلك.
\par 6 إذا احتاج إليك، فإنه يخدعك، ويبتسم لك، ويضعك في الرجاء، ويتكلم معك بلطف ويقول: ماذا تريد؟
\par 7 "ويخزيك بأطعمته حتى يجففك مرتين أو ثلاث مرات، وأخيراً سيسخر منك بعد ذلك، عندما يراك، سيهجرك ويهز رأسه عليك."
\par 8 احذر أن تخدع وتهبط في فرحك.
\par 9 إذا دُعيت من رجل عظيم فانسحب، وسوف يدعوك أكثر فأكثر.
\par 10 لا تضغط عليه لئلا تتراجع، لا تقف بعيدًا لئلا تنسى.
\par 11 لا تتظاهر بأنك مساوٍ له في الحديث، ولا تصدق كلماته الكثيرة: لأنه بكثرة الحديث سوف يغريك، والابتسامة عليك سوف تكشف أسرارك.
\par 12 ولكنه سيجمع كلماتك بقسوة، ولن يتوانى عن إيذائك ووضعك في السجن.
\par 13 "احذروا واحذروا جيدا، لأنكم تسيرون في خطر الهلاك. وعندما تسمعون هذا، استيقظوا في نومكم."
\par 14 أحب الرب طوال حياتك، وادعه لخلاصك.
\par 15 كل حيوان يحب نظيره، وكل إنسان يحب قريبه.
\par 16 كل جسد يتصرف حسب نوعه، والإنسان يلتصق بمثيله.
\par 17 أية شركة للذئب مع الحمل؟ وكذلك الخاطئ مع البار.
\par 18 ما هو الاتفاق بين الضبع والكلب؟ وما هو السلام بين الغني والفقير؟
\par 19 كما أن الحمار الوحشي فريسة الأسد في البرية، كذلك الأغنياء يأكلون الفقراء.
\par 20 كما أن المتكبرين يكرهون التواضع، كذلك الأغنياء يكرهون الفقراء.
\par 21 الرجل الغني الذي يبدأ في السقوط يدعمه أصدقاؤه، أما الرجل الفقير الذي يبدأ في السقوط فيدفعه أصدقاؤه بعيدًا عنه.
\par 22 عندما يسقط رجل غني، يكون له العديد من المساعدين. يتكلم بأشياء لا ينبغي أن يقال، ومع ذلك يبرره الناس. الرجل الفقير زل، ومع ذلك وبخوه أيضًا. تحدث بحكمة، ومع ذلك لم يكن له مكان.
\par 23 عندما يتكلم الغني يمسك كل إنسان لسانه، وانظروا ماذا يقول، يرفعونه إلى السحاب. ولكن إذا تكلم الفقير يقولون: من هذا الإنسان؟ وإذا تعثر، يساعدونه على إسقاطه.
\par 24 الغنى خير لمن لا خطيئة له، والفقر شر في فم الكافر.
\par 25 قلب الإنسان يغير وجهه، إن كان للخير أو للشر، والقلب الفرحان يجعل وجهه طلقا.
\par 26 الوجه البشوش هو علامة على أن القلب في حالة رخاء، وتعلم الأمثال هو عمل شاق للعقل.

\chapter{14}

\par 1 طوبى للرجل الذي لم يزل فمه ولم يُنخس بكثرة خطاياه.
\par 2 طوبى لمن لم يدانه ضميره، ولم يسقط عن رجائه في الرب.
\par 3 إن الثروة لا تزين البخيل، فماذا يفعل الحسود بالمال؟
\par 4 من يجمع بنهب نفسه يجمع للآخرين، ومن ينفق أمواله ببذخ.
\par 5 من أساء إلى نفسه فإلى من يحسن؟ لا يرضى بماله.
\par 6 لا يوجد أسوأ من الذي يحسد نفسه، وهذا جزاء شره.
\par 7 وإن عمل خيراً فإنه يعمله مكرهاً وأخيراً يعلن شره.
\par 8 الرجل الحسود له عين شريرة، يحول وجهه ويحتقر الناس.
\par 9 عين الرجل الطماع لا تشبع من نصيبه، وإثم الشرير يجفف نفسه.
\par 10 العين الشريرة تحسد على خبزها، وهي بخيل على مائدتها.
\par 11 يا ابني، بحسب طاقتك، افعل الخير لنفسك، وأعط الرب تقدمته الواجبة.
\par 12 تذكر أن الموت لن يتأخر، وأن عهد القبر لم يُكشف لك.
\par 13 افعل الخير لصديقك قبل أن تموت، وحسب قدرتك، مد يدك وأعطه.
\par 14 لا تحرم نفسك من يومك الطيب، ولا تدع جزءًا من رغبتك الطيبة يفوتك.
\par 15 أفلا تترك تعبك لآخر وتقسم تعبك بالقرعة؟
\par 16 أعطِ، وخذ، وقدِّس نفسك؛ لأنه لا بحث عن الأطعمة اللذيذة في القبر.
\par 17 كل جسد يشيخ كالثوب، لأن العهد من البدء هو أنك تموت موتاً.
\par 18 كما أن الأوراق الخضراء في شجرة كثيفة بعضها يسقط وبعضها ينمو، هكذا جيل اللحم والدم، واحد ينتهي، وآخر يولد.
\par 19 كل عمل يفسد ويفنى، وصاحبه يمضي معه.
\par 20 طوبى للرجل الذي يفكر في الخيرات بحكمة، والذي يتفكر في المقدسات بفهمه.
\par 21 من تأمل طرقها في قلبه يفهم أسرارها أيضاً.
\par 22 اتبعها كما تتبعها، واكمن في طرقها.
\par 23 والذي ينبش في نوافذها يسمع أيضاً عند أبوابها.
\par 24 ومن ينزل بالقرب من بيتها فليضع وتداً في حائطها.
\par 25 فينصب خيمته بالقرب منها، ويبيت في منزل حيث تكون الأشياء الجيدة.
\par 26 ويجعل أولاده تحت ظلها، ويبيت تحت أغصانها.
\par 27 بها يستتر من الحر، وفي مجدها يسكن.

\chapter{15}

\par 1 من يتقي الرب يعمل الخير، ومن يعرف الشريعة يقتنيها.
\par 2 وكأم تستقبله، وتستقبله كزوجة من عذراء.
\par 3 وتطعمه خبز الفهم وتسقيه ماء الحكمة.
\par 4 فيعتمد عليها فلا يتزعزع، ويعتمد عليها فلا يخزى.
\par 5 وترفعه فوق جيرانه، وفي وسط الجماعة تفتح فمه.
\par 6 ويجد فرحاً وإكليل سرور، وتورثه اسماً أبدياً.
\par 7 ولكن لا يصل إليها الجاهلون، ولا يراها الخطاة.
\par 8 لأنها بعيدة كل البعد عن الكبرياء، والرجال الكاذبون لا يستطيعون أن يتذكروها.
\par 9 لا يليق التسبيح في فم الخاطئ، لأنه لم يُرسل إليه من الرب.
\par 10 لأنه بالحكمة يُقال التسبيح، والرب ينجحه.
\par 11 لا تقل إني من أجل الرب سقطت، لأنه لا ينبغي لك أن تفعل ما يكرهه.
\par 12 لا تقل إنه أضلني، فإنه ليس له حاجة إلى الرجل الخاطئ.
\par 13 الرب يبغض كل رجس، والذين يخافون الله لا يحبونه.
\par 14 لقد خلق الإنسان بنفسه منذ البدء، وتركه في يد مشورته.
\par 15 إن أردت أن تحفظ الوصايا وتعمل الأمانة المقبولة.
\par 16 لقد وضع أمامك النار والماء، فمدّ يدك إن شئت.
\par 17 قبل الإنسان هناك حياة وموت، وسوف يُعطى ما يريده.
\par 18 لأن حكمة الرب عظيمة وهو شديد القدرة وينظر إلى كل شيء.
\par 19 وعيناه على خائفيه، وهو يعلم كل عمل الإنسان.
\par 20 ولم يأمر أحداً أن يفعل شراً، ولم يعط أحداً رخصة للخطيئة.

\chapter{16}

\par 1 لا ترغب في كثرة الأبناء غير الصالحين، ولا تفرح بالأبناء الأشرار.
\par 2 وإن كثروا فلا تفرح بهم، إن لم تكن مخافة الرب معهم.
\par 3 لا تثق بحياتهم، ولا تنظر إلى كثرة عددهم. لأن واحداً باراً خير من ألف، ومن الأفضل أن تموت بلا أولاد من أن يكون لك من الأشرار.
\par 4 لأنه بواحد ذي فهم تمتلئ المدينة، أما عشيرة الأشرار فتصير خرابا سريعا.
\par 5 أشياء كثيرة مثل هذه رأيتها بعيني، وسمعت أذني أشياء أعظم من هذه.
\par 6 في جماعة الأشرار تشتعل نار، وفي الأمة المتمردة يشتعل الغضب.
\par 7 لم يكن مسالمًا تجاه العمالقة القدامى، الذين سقطوا في قوة حماقتهم.
\par 8 ولم يسلم المكان الذي كان لوط يقيم فيه، بل كرههم بسبب كبريائهم.
\par 9 ولم يشفق على أهل الهلاك الذين انقادوا إلى خطاياهم.
\par 10 ولا الستمائة ألف من المشاة، الذين اجتمعوا في قسوة قلوبهم.
\par 11 وإن كان في الشعب رجل قاس الرقبة، فمن العجيب أن ينجو من العقاب، لأن الرحمة والغضب معه، وهو قادر على الغفران وسكب السخط.
\par 12 كما أن رحمته عظيمة فكذلك توبيخه عظيم، فهو يحكم على الإنسان حسب أعماله.
\par 13 لا ينجو الخاطئ بغنائمه، ولا يخيب صبر التقي.
\par 14 افسحوا المجال لكل عمل رحمة، لأن كل واحد يجد حسب أعماله.
\par 15 فقسّى الرب فرعون حتى لا يعرفه، لكي تُعرَف أعماله العظيمة في العالم.
\par 16 إن رحمته ظاهرة لكل مخلوق، وقد فصل نوره عن الظلمة بالماس.
\par 17 لا تقل إني سأختبئ من الرب، فهل يذكرني أحد من فوق؟ لا أذكر بين هذا العدد من الناس، فما قيمة نفسي بين هذا العدد من الخلائق؟
\par 18 هوذا السماء وسماء السموات والغمر والأرض وكل ما فيها تتزعزع عند افتقاده.
\par 19 فتهتز الجبال وأساسات الأرض رجفة حين ينظر إليها الرب.
\par 20 لا يستطيع قلب أن يفكر في هذه الأمور بشكل لائق، ومن يستطيع أن يتصور طرقه؟
\par 21 إنها عاصفة لا يستطيع أحد أن يراها: لأن معظم أعماله مخفية.
\par 22 فمن يخبر بأعمال بره أو من يصبر عليها لأن عهده بعيد واختبار كل شيء في النهاية.
\par 23 من يفتقر إلى الفهم يفكر في الباطل، والرجل الجاهل الضال يفكر في الحماقات.
\par 24 يا ابني اسمع لي وتعلم المعرفة واحفظ كلامي في قلبك.
\par 25 سأظهر التعليم بالوزن، وأعلن معرفته بالضبط.
\par 26 أعمال الرب معمولة بالدينونة منذ البدء، ومنذ خلقها رتب أجزاءها.
\par 27 "زيّن أعماله إلى الأبد، وفي يده رؤسها إلى دور فدور. لا يتعبون ولا يكلون ولا يكفون عن أعمالهم."
\par 28 لا أحد منهم يعيق الآخر، ولا يعصون كلامه أبدًا.
\par 29 وبعد ذلك نظر الرب إلى الأرض فملأها من بركاته.
\par 30 بكل أنواع الحيوانات غطى وجهها، فتعود إليها أيضًا.

\chapter{17}

\par 1 خلق الرب الإنسان من الأرض، وأعاده إليها.
\par 2 وأعطاهم أيامًا قليلة، ووقتًا قصيرًا، وسلطانًا أيضًا على الأشياء فيه.
\par 3 فأعطاهم قوة من ذواتهم، وصنعهم على صورته،
\par 4 ووضع رعب الإنسان على كل ذي جسد، وأعطاه سلطاناً على البهائم والطيور.
\par 5 وقد تلقوا استخدام العمليات الخمس للرب، وفي المقام السادس أعطاهم الفهم، وفي الخطاب السابع، مترجماً لتأملاتها.
\par 6 وأعطاهم المشورة واللسان والعيون والآذان والقلب ليفهموا.
\par 7 فملأهم بمعرفة الفهم، وأراهم الخير والشر.
\par 8 ووضع نظره على قلوبهم لكي يريهم عظمة أعماله.
\par 9 وأعطاهم المجد في عجائبه إلى الأبد، لكي يخبروا بأعماله بفهم.
\par 10 والمختارون يسبحون اسمه القدوس.
\par 11 وبالإضافة إلى ذلك أعطاهم المعرفة، وشريعة الحياة ميراثًا.
\par 12 وقطع معهم عهداً أبدياً وأظهر لهم أحكامه.
\par 13 فأبصرت عيونهم عظمة مجده، وسمعت آذانهم مجد صوته.
\par 14 وقال لهم: احذروا من كل إثم، وأوصى كل إنسان بخصوص قريبه.
\par 15 طرقهم أمامه دائما، ولا تخفى عن عينيه.
\par 16 كل إنسان منذ حداثته يميل إلى الشر، ولم يستطيعوا أن يجعلوا لأنفسهم قلباً لحمياً حجرياً.
\par 17 "فإنه في تقسيم أمم الأرض كلها جعل رئيسا على كل شعب، وأما إسرائيل فهو نصيب الرب."
\par 18 الذي هو ابنه البكر، يغذيه بالتأديب، ويعطيه نور محبته ولا يتركه.
\par 19 لذلك فإن جميع أعمالهم كالشمس أمامه، وعيناه دائما على طرقهم.
\par 20 لا شيء من أعمالهم الخاطئة يخفى عليه، بل كل خطاياهم أمام الرب.
\par 21 ولكن الرب إذ كان حنوناً وعارفاً بصنعته لم يتركهم ولم يتركهم، بل أشفق عليهم.
\par 22 صدقة الرجل كخاتم عنده، وأعمال الإنسان الصالحة كقرة عينه، وأعط التوبة لبنيه وبناته.
\par 23 ثم يقوم فيجازيهم ويجعل جزاءهم على رؤوسهم.
\par 24 وأما الذين تابوا، فأعطاهم الرجوع، وعزى الذين فشلوا في الصبر.
\par 25 ارجع إلى الرب، واترك خطاياك، واجعل صلاتك أمام وجهه، وأقل من ذنوبك.
\par 26 ارجع إلى العلي وابتعد عن الإثم، فإنه يخرجك من الظلمة إلى نور الصحة، ويبغض الرجس بشدة.
\par 27 فمن يحمد العلي في القبر عوضا عن الأحياء الذين يشكرون؟
\par 28 الشكر يهلك من بين الأموات كما من غير الموجود. الحي والصحيح القلب يسبح الرب.
\par 29 ما أعظم رحمات الرب إلهنا ورأفته على الذين يتوبون إليه.
\par 30 لأنه لا يمكن أن يكون كل شيء في الناس، لأن ابن الإنسان ليس خالداً.
\par 31 ما هو أشد إشراقا من الشمس؟ ولكن نورها يضعف، واللحم والدم يفكران الشر.
\par 32 فهو ينظر إلى قوة ارتفاع السماء، وأن جميع البشر ليسوا سوى تراب ورماد.

\chapter{18}

\par 1 الذي يعيش إلى الأبد هو الذي خلق كل الأشياء بشكل عام.
\par 2 الرب وحده هو البار، وليس هناك آخر غيره،
\par 3 الذي يحكم العالم براحة يده، وكل الأشياء تطيع إرادته. لأنه ملك الجميع، وبقدرته يفصل المقدسات عن الدنيئة فيما بينهم.
\par 4 من أعطاه القدرة على إظهار أعماله؟ ومن سيكتشف أعماله النبيلة؟
\par 5 من يحصي قوة عظمته ومن يخبر أيضا بمراحمه؟
\par 6 وأما عجائب الرب فلا يؤخذ منها شيء، ولا يضاف إليها شيء، ولا يُكتشف أساسها.
\par 7 إذا فعل الإنسان شيئاً فإنه يبدأ، وإذا انتهى فإنه يشك.
\par 8 ما هو الإنسان وماذا يخدم؟ ما هو خيره وما هو شره؟
\par 9 إن عدد أيام الإنسان على الأكثر مائة سنة.
\par 10 كنقطة ماء في البحر، وحصباء في الرمل، هكذا ألف سنة في أيام الأبدية.
\par 11 لذلك فإن الله صبور عليهم، ويفيض عليهم رحمته.
\par 12 فرأى وأدرك أن نهايتهم شريرة، لذلك كثّر رحمته.
\par 13 رحمة الإنسان هي نحو قريبه، وأما رحمة الرب فهي على كل بشر. فهو يوبخ ويربي ويعلم ويرد، مثل الراعي قطيعه.
\par 14 فهو يرحم الذين يتلقون التأديب، والذين يطلبون أحكامه باجتهاد.
\par 15 يا ابني، لا تلطخ أعمالك الصالحة، ولا تستخدم كلمات غير مريحة عندما تعطي أي شيء.
\par 16 ألا يخفف الندى الحر؟ فهل الكلمة خير من الهدية؟
\par 17 أليس الكلام خير من الهدية؟ ولكن كلاهما عند الرجل الفاضل.
\par 18 الأحمق يوبخ بفظاظة، وهدية الحسود تأكل العيون.
\par 19 تعلم قبل أن تتكلم، واستخدم الطب قبل أن تمرض.
\par 20 قبل الحكم اختبر نفسك، وفي يوم الافتقاد تجد رحمة.
\par 21 تواضع قبل أن تمرض، وفي وقت الخطايا أظهر التوبة.
\par 22 لا يمنعك شيء من الوفاء بنذرك في الوقت المحدد، ولا تؤجل تبريرك إلى الموت.
\par 23 قبل أن تصلي، أعدّ نفسك، ولا تكن كمن يجرب الرب.
\par 24 فكر في الغضب الذي سيكون في النهاية، ووقت الانتقام، عندما يحول وجهه.
\par 25 عندما يكون لديك ما يكفي، تذكر وقت الجوع، وعندما تصبح غنيًا، فكر في الفقر والحاجة.
\par 26 ومن الصباح إلى المساء يتغير الوقت، ويتم كل شيء سريعًا أمام الرب.
\par 27 الرجل الحكيم يخاف في كل شيء، وفي يوم الخطيئة يحذر من الإساءة، أما الجاهل فلا يلاحظ الوقت.
\par 28 كل إنسان ذي فهم يعرف الحكمة، فيمدح من وجدها.
\par 29 والذين كانوا فاهمين في الأقوال صاروا هم أنفسهم حكماء، وأنشدوا أمثالاً حسنة.
\par 30 لا تتبع شهواتك، بل امتنع عن شهواتك.
\par 31 إذا أعطيت روحك الرغبات التي ترضيها، فإنها سوف تجعلك أضحوكة لأعدائك الذين يشوهون سمعتك.
\par 32 لا تفرح بالبهجة الكثيرة، ولا تكن مقيدًا بنفقاتها.
\par 33 لا تجعل نفسك متسولاً من خلال التباهي بالاقتراض، عندما لا يكون لديك شيء في محفظتك: لأنك ستنتظر حياتك، وسيتحدث الناس إليك.

\chapter{19}

\par 1 الرجل العامل الذي يسكر لا يغني، والذي يحتقر الأشياء الصغيرة يسقط قليلا قليلا.
\par 2 الخمر والنساء تجعلان ذوي العقول يضلون، ومن يلتصق بالزانيات يتفاخر.
\par 3 العث والديدان سوف يرثونه، والرجل الجريء سوف يؤخذ بعيدا.
\par 4 من يتعجل في الفضل فهو خفيف الخاطر، ومن يخطئ فإنه يسيء إلى نفسه.
\par 5 من يسر بالشر يدان، ومن يقاوم الملذات يكلّل حياته.
\par 6 من يستطيع أن يتحكم في لسانه يعيش بلا نزاع، ومن يكره الثرثرة يكون شره أقل.
\par 7 لا تروي للآخرين ما قيل لك، ولن تحصل على ما هو أسوأ أبدًا.
\par 8 سواء كان الأمر لصديق أو عدو، لا تتحدث عن حياة الآخرين؛ وإذا كنت تستطيع دون إثارة الجدل، فلا تكشفها.
\par 9 لأنه سمعك ولاحظك، وعندما يأتي الوقت سوف يكرهك.
\par 10 إذا سمعت كلمة، فاتركها تموت معك، وكن جريئًا، فلن تمزقك.
\par 11 الأحمق يعمل بالكلمة كالأم التي تلد.
\par 12 كالسهم الذي يغرز في فخذ الرجل، كذلك الكلمة في بطن الأحمق.
\par 13 نبّه صديقك، فقد لا يكون قد فعل ذلك، وإذا كان قد فعل ذلك، فلا يفعله بعد الآن.
\par 14 نبّه صديقك، فقد لا يكون قد قال ذلك، وإذا كان قد قاله، فلا يتكلم به مرة أخرى.
\par 15 نبّه صديقك، فإنه في كثير من الأحيان يكون افتراءً، ولا تصدق كل قصة.
\par 16 هناك من يزل في كلامه ولكن ليس من قلبه، ومن هو الذي لم يزل بلسانه؟
\par 17 أنذر قريبك قبل أن تهدده، ولا تغضب، بل أعط مكانًا لشريعة العلي.
\par 18 إن مخافة الرب هي الخطوة الأولى لقبولنا إياه، والحكمة تحصل على محبته.
\par 19 إن معرفة وصايا الرب هي تعليم الحياة، والذين يعملون ما يرضيه ينالون ثمر شجرة الخلود.
\par 20 مخافة الرب هي كل الحكمة، وفي كل الحكمة العمل بالناموس ومعرفة قدرته.
\par 21 إذا قال عبد لسيده: لا أفعل كما تشاء، فإن فعل بعد ذلك فإنه يغضب الذي يعوله.
\par 22 إن معرفة الشر ليست حكمة، ولا مشورة الخطاة حكمة في أي وقت.
\par 23 هناك شر، وهناك رجس، وهناك أحمق يفتقر إلى الحكمة.
\par 24 من كان فهمه قليلا ويخاف الله فهو أفضل من من كان حكمته كثيرة ويتعدى على شريعة العلي.
\par 25 يوجد دهاء رقيق وهو ظالم ويوجد من يزيغ لإظهار الحق ويوجد حكيم يبرر في الحق.
\par 26 "يوجد رجل شرير يحني رأسه حزينًا، ولكنه في داخله مليء بالخداع،
\par 27 منحط الوجه، كأنه لا يسمع: حيث لا يكون معروفًا، فإنه سوف يفعل بك الشر قبل أن تشعر.
\par 28 وإن منعه نقص القوة عن الخطيئة، فإنه متى وجد الفرصة فإنه يفعل الشر.
\par 29 يمكن معرفة الرجل من مظهره، والفاهم من وجهه، عندما تقابله.
\par 30 لباس الرجل، وضحكه المفرط، ومشيته، كل ذلك يدل على حقيقته.

\chapter{20}

\par 1 يوجد توبيخ غير لائق. أيضا، هناك رجل يمسك لسانه، وهو حكيم.
\par 2 من الأفضل أن توبخ من أن تغضب سراً، ومن يعترف بخطئه ينجو من الأذى.
\par 3 ما أجمل أن تظهر التوبة عند توبيخك، لأنك بذلك تنجو من الخطيئة المتعمدة.
\par 4 كما أن شهوة الخصي هي فض بكارة العذراء، هكذا من ينفذ الحكم بالعنف.
\par 5 يوجد من يصمت فيكون حكيما، ومن كثرة الثرثرة يصير مكروهاً.
\par 6 هناك من يمسك لسانه لأنه ليس لديه ما يجيب به، وهناك من يصمت لأنه يعرف وقته.
\par 7 الرجل الحكيم يمسك لسانه حتى يرى الفرصة، أما الثرثار والأحمق فلا يهتمان بالوقت.
\par 8 من يتكلم كثيرا يبغض، ومن يتسلط على الناس في الكلام يبغض.
\par 9 هناك خاطئ ينجح في الأمور السيئة، وهناك ربح يتحول إلى خسارة.
\par 10 إن هناك هدية لا تنفعك، وهناك هدية جزاؤها مضاعف.
\par 11 هناك تواضع بسبب المجد، وهناك من يرفع رأسه من التواضع.
\par 12 هناك من يشتري الكثير بقليل، فيرده سبعة أضعاف.
\par 13 الرجل الحكيم يجعله محبوبًا بكلامه، أما نعمة الجهال فتفيض.
\par 14 إن هدية الأحمق لا تنفعك حين تملكها، وكذلك الحال بالنسبة للحسود حين يحتاج إليها، لأنه يتطلع إلى الحصول على أشياء كثيرة مقابل شيء واحد.
\par 15 "إنه يعطي قليلاً ويوبخ كثيرًا، يفتح فمه كالمنادي، اليوم يقرض وغدا يطالب، مثل هذا الإنسان مكروه من الله والناس."
\par 16 يقول الجاهل ليس لي أصدقاء، وليس لي شكر على كل أعمالي الصالحة، والذين يأكلون خبزي يتكلمون عليّ شراً.
\par 17 كم من مرة، وكم من الناس سوف يسخرون منه! لأنه لا يعرف حقًا ما هو الشيء الذي يجب أن يمتلكه؛ وهو واحد بالنسبة له كما لو أنه لا يمتلكه.
\par 18 إن الانزلاق على الرصيف خير من الانزلاق باللسان، لذلك فإن سقوط الأشرار سيكون سريعا.
\par 19 ستظل القصة غير المناسبة دائمًا في فم الساذج.
\par 20 إن الحكمة التي خرجت من فم الأحمق تُرفض، لأنه لا يتكلم بها في حينه.
\par 21 هناك من يمنعه العوز من الخطيئة، وعندما يستريح لا يضطرب.
\par 22 هناك من يهلك نفسه بالخجل، ومن يقبل الأشخاص يهلك نفسه.
\par 23 هناك من يخجل من صديقه ويجعله عدوه بلا سبب.
\par 24 إن الكذب وصمة عار في جبين الإنسان، ومع ذلك فهو يبقى في فم الجاهل.
\par 25 اللص خير من الرجل الذي يعتاد الكذب، ولكن كلاهما مصيرهما الهلاك.
\par 26 إن شخصية الكاذب مخزية، وعاره يرافقه إلى الأبد.
\par 27 الرجل الحكيم يرفع نفسه إلى الشرف بكلامه، وصاحب الفهم يرضي العظماء.
\par 28 من يزرع أرضه يكثر ثمرها، ومن يرضي العظماء ينال غفران إثمه.
\par 29 الهدايا والعطايا تعمي عيون الحكيم وتسد فمه حتى لا يستطيع أن يوبخ.
\par 30 الحكمة المخفية والكنز المدخَّن، ما المنفعة فيهما كليهما؟
\par 31 من يخفي جهالته خير من من يخفي حكمته.
\par 32 الصبر الضروري في طلب الرب خير من أن يعيش حياته بلا مرشد.

\chapter{21}

\par 1 يا ابني، هل أخطأت؟ لا تفعل ذلك مرة أخرى، بل اطلب المغفرة عن خطاياك السابقة.
\par 2 اهرب من الخطيئة كما تهرب من وجه الحية، لأنها إذا اقتربت منها فإنها ستعضك: أسنانها كأسنان الأسد، تقتل أرواح البشر.
\par 3 كل الإثم هو مثل سيف ذو حدين، جراحه لا يمكن شفاءها.
\par 4 إن الترويع والإثم يبدد الثروات، وهكذا يصبح بيت المتكبرين خرابا.
\par 5 إن الصلاة من فم الفقير تصل إلى أذني الله، فيأتي حكمه سريعًا.
\par 6 من يكره التوبيخ فهو في طريق الخطاة، وأما من يتقي الرب فيتوب من قلبه.
\par 7 الرجل الفصيح معروف من بعيد ومن قريب، أما الرجل الفصيح فيعرف متى يزل.
\par 8 من يبني بيته بأموال الآخرين يشبه من يجمع حجارة قبره.
\par 9 جماعة الأشرار كشدة ملفوفة وعاقبتهم لهيب نار لإهلاكهم.
\par 10 إن طريق الخطاة واضح بالحجارة، ولكن في نهايته حفرة الجحيم.
\par 11 من يحفظ شريعة الرب يفهمها، وكمال مخافة الرب هو الحكمة.
\par 12 من ليس حكيما لا يتعلم، ولكن هناك حكمة تزيد المرارة.
\par 13 معرفة الرجل الحكيم تزداد كالطوفان، ومشورته كنبع حياة نقي.
\par 14 إن باطنة الأحمق كإناء مكسور، فلا يحتفظ بالمعرفة طوال حياته.
\par 15 إذا سمع الإنسان الحكيم كلمة حكيمة فإنه يمدحها ويزيد عليها، ولكن إذا سمعها أحد غير الفاهم فإنها تغضبه فيلقيها وراء ظهره.
\par 16 كلام الجاهل كحمل في الطريق، وأما النعمة فتوجد في شفتي الحكماء.
\par 17 فيسألون من فم الحكيم في الجماعة، ويتأملون في كلماته في قلوبهم.
\par 18 كالبيت الذي يهدم، هكذا الحكمة عند الجاهل، ومعرفة الجاهل كالكلام الذي لا عقل له.
\par 19 التعليم عند الجهال كالقيود في الرجلين، وكأغلال في اليد اليمنى.
\par 20 يرفع الأحمق صوته من الضحك، أما الرجل الحكيم فلا يكاد يبتسم إلا قليلا.
\par 21 إن العلم عند الرجل الحكيم كزينة من ذهب، وكسوار على يده اليمنى.
\par 22 إن قدم الرجل الأحمق سرعان ما تدخل بيت جاره، أما الرجل ذو الخبرة فيخجل منه.
\par 23 الأحمق ينظر من الباب إلى البيت، أما الذي يرعى جيداً فيقف خارجاً.
\par 24 من وقاحة الرجل أن يستمع عند الباب، أما الرجل الحكيم فسوف يحزن من العار.
\par 25 شفاه المتحدثين تتكلم بما لا يعنيهم، وأما كلام الفهم فيوزن في الميزان.
\par 26 قلب الجهال في أفواههم، وأفواه الحكماء في قلوبهم.
\par 27 عندما يلعن الكافر الشيطان، فإنه يلعن نفسه.
\par 28 النمام ينجس نفسه، وهو مكروه حيثما يسكن.

\chapter{22}

\par 1 يشبه الرجل الكسلان بالحجر القذر، وسوف يصفره الجميع إلى عارهم.
\par 2 يشبه الرجل الكسلان بقذارة المزبلة، كل من يرفعها يصافحه.
\par 3 الرجل الذي يربي تربية سيئة هو عار لأبيه الذي ولده، والبنت الجاهلة تولد له خسارا.
\par 4 البنت الحكيمة ترث زوجها، وأما التي تعيش بلا أمانة فهي ثقل على أبيها.
\par 5 "والشجاعة تهين أباها وزوجها وكلاهما يحتقرانها."
\par 6 إن الحكاية في غير وقتها كالموسيقى في الحزن، ولكن الضربات وتقويم الحكمة لا يكونان خارج وقتهما أبدًا.
\par 7 من يعلم الجاهل فهو كمن يلصق شقفة من الفخار، وكمن يوقظ من نوم عميق.
\par 8 من يحكي قصة لأحمق فهو يحكي لرجل نائم، وعندما يحكي قصته يقول: ما الأمر؟
\par 9 إذا عاش الأطفال بأمانة، وكانوا قادرين على ذلك، فإنهم سيغطون دناءة والديهم.
\par 10 لكن الأطفال، كونهم متكبرين، بسبب الازدراء والافتقار إلى الرعاية، فإنهم يلوثون نبل أقاربهم.
\par 11 ابكوا على الميت لأنه فقد النور، وابكوا على الأحمق لأنه يفتقر إلى الفهم. قللوا من البكاء على الميت لأنه في راحة، ولكن حياة الأحمق أسوأ من الموت.
\par 12 سبعة أيام ينوح الناس على الميت، وأما الأحمق والكافر فيحزنون عليه كل أيام حياته .
\par 13 لا تتكلم كثيرا مع الأحمق، ولا تذهب إلى من لا عقل له. احذر منه لئلا تقع في مشكلة، ولا تتنجس بغبائه. ابتعد عنه، فتجد راحة، ولا تضطرب بالجنون.
\par 14 ما هو أثقل من الرصاص؟ وما اسمه إلا الأحمق؟
\par 15 الرمل والملح وكتلة الحديد أسهل في التحمل من رجل بلا فهم.
\par 16 كما أن الخشب الذي تم تجميعه وربطه معًا في مبنى لا يمكن أن يتفكك بالاهتزاز: كذلك القلب الذي تم تأسيسه عن طريق المشورة لا يخاف في أي وقت.
\par 17 إن القلب الذي يرتكز على فكرة الفهم هو بمثابة طلاء جدار معرض.
\par 18 إن الأعمدة الموضوعة على المرتفعات لا تصمد أمام الريح، وكذلك القلب الخائف في خيال الأحمق لا يستطيع أن يصمد أمام أي خوف.
\par 19 من ينخز العين يُسقِط الدموع، ومن ينخز القلب يُظهر معرفته.
\par 20 من يرمي الطير بحجر فإنه يفرقه، ومن يوبخ صديقه فإنه يهدم الصداقة.
\par 21 وإن سحبت سيفك على صديقك فلا تيأس، فقد يكون هناك عودة [إلى الفضل].
\par 22 إذا فتحت فمك على صديقك فلا تخف، لأنه قد يكون هناك مصالحة، إلا في حالة التوبيخ، أو الكبرياء، أو إفشاء الأسرار، أو الجرح الغادر، لأنه بسبب هذه الأشياء سوف يرحل كل صديق.
\par 23 كن أمينًا لقريبك في فقره، لكي تفرح في رخائه. كن ثابتًا معه في وقت ضيقه، لكي ترث معه في ميراثه. لأنه لا ينبغي دائمًا احتقار الفقير، ولا ينبغي الإعجاب بالغني الأحمق.
\par 24 كما أن بخار ودخان الأتون يذهبان أمام النار، كذلك الاستهزاء أمام الدم.
\par 25 لن أخجل من الدفاع عن صديقي، ولن أخفي نفسي عنه.
\par 26 وإن أصابني شر منه فكل من يسمع يحذر منه.
\par 27 من يضع حارساً لفمي وخاتم الحكمة على شفتي لئلا أسقط بهما بغتة ولا يهلكني لساني؟

\chapter{23}

\par 1 يا رب، أبا وحاكم كل حياتي، لا تتركني لمشوراتهم، ولا تدعني أسقط بسببهم.
\par 2 من يضع سوطاً على أفكاري، وتأديب الحكمة على قلبي؟ حتى لا يشفقوا عليّ بسبب جهلي، ولا يتجاوزوا عن خطاياي.
\par 3 لئلا تكثر جهالي، وتكثر خطاياي إلى هلاكي، وأسقط أمام أعدائي، ويفرح بي عدوي الذي رجاؤه بعيد عن رحمتك.
\par 4 يا رب، أبتاه وإله حياتي، لا تعطيني نظرة فخورة، بل اصرف عن عبيدك دائمًا عقلًا متكبرًا.
\par 5 أبعد عني الآمال الباطلة والشهوات، وسوف تدعم من يرغب في خدمتك دائمًا.
\par 6 لا تدع شهوة البطن وشهوة الجسد تسيطر عليّ، ولا تسلمني عبدك إلى فكر وقح.
\par 7 اسمعوا أيها الأولاد تأديب الفم. من يحفظه لا يؤخذ بشفتيه.
\par 8 ويبقى الخاطئ في جهالته، ويسقط به كل من المتكبر والمتكلم الشرير.
\par 9 لا تعوّد فمك على الحلف، ولا تستخدم نفسك لتسمية القدوس.
\par 10 لأنه كما أن العبد الذي يُضرب باستمرار لا يكون بلا علامة زرقاء، كذلك الذي يحلف ويذكر الله باستمرار لا يكون بلا عيب.
\par 11 الرجل الذي يكثر من الحلف يمتلئ إثماً ولا يبرح الوباء من بيته. إن أخطأ فخطيئته عليه. وإن لم يعترف بخطيئته يرتكب إثماً مضاعفاً. وإن حلف باطلا فلا يكون بريئاً بل بيته يمتلئ شروراً.
\par 12 هناك كلمة مغلفة بالموت: ليسمح الله أن لا توجد في ميراث يعقوب، لأن مثل هذه الأشياء تكون بعيدة عن الأتقياء، ولا يتمرغون في خطاياهم.
\par 13 لا تستخدم فمك في الشتائم، لأن فيها كلمة الخطيئة.
\par 14 اذكر أباك وأمك عند جلوسك بين العظماء. لا تنسَهما، فتُصبح أحمقًا بطبيعتك، وتتمنى لو لم تولد، وتلعن يوم ميلادك.
\par 15 الإنسان الذي يعتاد على الكلام المسيء لن يصلح نفسه كل أيام حياته.
\par 16 نوعان من الرجال يكثرون من الخطيئة، والثالث يجلب الغضب: العقل الساخن كالنار المشتعلة، لا ينطفئ أبدًا حتى يتم استهلاكه: الزاني في جسده البشري لن يتوقف أبدًا حتى يشعل نارًا.
\par 17 كل الخبز حلو للزاني، ولا يتوقف عنه حتى يموت.
\par 18 رجل يفسخ عقد الزواج، ويقول في قلبه: من يراني؟ أنا محاط بالظلام، والجدران تغطيني، ولا أحد يراني، فماذا أخاف؟ لن يتذكر العلي خطاياي.
\par 19 مثل هذا الرجل يخاف فقط من عيون البشر، ولا يعلم أن عيني الرب أكثر إشراقا من الشمس بعشرة آلاف مرة، وهي تنظر إلى جميع طرق البشر، وتنظر إلى الأجزاء الأكثر سرية.
\par 20 لقد كان يعلم كل الأشياء قبل خلقها، وكذلك بعد أن تم خلقها، نظر إليها كلها.
\par 21 هذا الرجل سيعاقب في شوارع المدينة، وأما حيث لا يشتبه فسيؤخذ.
\par 22 وهكذا يكون الأمر أيضا بالنسبة للمرأة التي تترك زوجها وتأتي بوارث من آخر.
\par 23 أولاً، عصت شريعة العلي؛ وثانياً، أخطأت في حق زوجها؛ وثالثاً، زنت وأنجبت أطفالاً من رجل آخر.
\par 24 وتُخرَج إلى الجماعة ويُحاسَب على أولادها.
\par 25 لن يتأصل أولادها ولن يأتي فرعها بثمر.
\par 26 تترك ذكراها ملعونة، ولا يمحى عارها.
\par 27 وأما الباقون فيعلمون أنه ليس هناك شيء أفضل من مخافة الرب، وليس هناك شيء أحلى من الانتباه إلى وصايا الرب.
\par 28 إن اتباع الرب هو مجد عظيم، وأن نقبله هو حياة طويلة.

\chapter{24}

\par 1 الحكمة تفتخر، وتفتخر في وسط شعبها.
\par 2 في جماعة العلي تفتح فمها وتفتخر أمام قوته.
\par 3 خرجت من فم العلي وغطيت الأرض كالسحاب.
\par 4 سكنت في المرتفعات، وعرشي في عمود السحاب.
\par 5 لقد طافت حول السماء وحدي، وسرت في أعماقها.
\par 6 في أمواج البحر، وفي كل الأرض، وفي كل شعب وأمة، حصلت على ممتلكاتي.
\par 7 مع كل هؤلاء طلبت الراحة، وفي ميراث من سأقيم؟
\par 8 "لذلك أعطاني خالق كل الأشياء وصية، والذي خلقني جعل مسكني يستقر، وقال: لتكن مسكنك في يعقوب، وميراثك في إسرائيل."
\par 9 لقد خلقني منذ البدء قبل العالم، ولن أفشل أبدًا.
\par 10 في المسكن المقدس خدمت أمامه، وهكذا ترسخت في صهيون.
\par 11 وكذلك في المدينة المحبوبة أعطاني راحة، وفي أورشليم كان قوتي.
\par 12 وتأصلت في شعب شريف، في نصيب ميراث الرب.
\par 13 ارتفعت كالأرز في لبنان، وكشجرة السرو على جبال حرمون.
\par 14 ارتفعت كالنخلة في عين جدي، وكشجرة الورد في أريحا، كشجرة الزيتون الجميلة في حقل جميل، ونشأت كالدلب على جانب الماء.
\par 15 فأعطيت رائحة طيبة كالقرفة والريحان، وأعطيت رائحة طيبة كأفضل المر، كالجلبانوم، والعقيق، والزعتر العطر، وكرائحة اللبان في المسكن.
\par 16 كشجرة التربنتين مددت أغصاني، وأغصاني هي أغصان الشرف والنعمة.
\par 17 كما أخرجت الكرمة رائحة طيبة، وأزهاري هي ثمرة الشرف والغنى.
\par 18 أنا أم الحب العادل، والخوف، والمعرفة، والأمل المقدس: لذلك، كوني أبدية، أُعطى لجميع أطفالي الذين سُمّوا باسمه.
\par 19 تعالوا إليّ يا جميع الذين يرغبون فيّ، واشبعوا من ثماري.
\par 20 لأن ذكري أحلى من العسل، وميراثي أحلى من شهد العسل.
\par 21 من يأكلني يجوع بعد، ومن يشربني يعطش بعد.
\par 22 من يطيعني فلا يخزى إلى الأبد، والذين يعملون معي لا يخطئون.
\par 23 كل هذه الأشياء هي كتاب عهد الله العلي، الشريعة التي أمر بها موسى ميراثاً لجماعات يعقوب.
\par 24 لا تكلوا عن أن تتشددوا في الرب. لكي يثبتكم، التصقوا به. لأن الرب القدير هو الله وحده، وليس سواه مخلص آخر.
\par 25 "فإنه يملأ كل الأشياء بحكمته، مثل فيشون ومثل دجلة في زمن الثمار الجديدة."
\par 26 ويجعل الفهم يكثر كالفرات وكالأردن في وقت الحصاد.
\par 27 ويجعل عقيدة المعرفة تظهر كالنور، وكجيون في زمن العتيق.
\par 28 لم يكن الرجل الأول يعرفها تمامًا: ولن يتمكن الرجل الأخير من اكتشافها بعد الآن.
\par 29 لأن أفكارها أعظم من البحر، ومشوراتها أعمق من الغمر العظيم.
\par 30 لقد خرجت أيضًا كجدول من نهر، وكقناة إلى حديقة.
\par 31 قلت أسقي خير حديقتي وأسقي جنينتي بكثرة، فإذا جدولي قد صار نهراً، ونهري قد صار بحراً.
\par 32 "سأجعل العقيدة تشرق كالصباح، وسأرسل نورها إلى البعيد."
\par 33 سأستمر في سكب العقيدة كنبوة، وأتركها لكل العصور إلى الأبد.
\par 34 انظروا، إني لم أعمل لنفسي فقط، بل لجميع الذين يطلبون الحكمة.

\chapter{25}

\par 1 في ثلاثة أشياء كنت جميلاً، ووقفت جميلة أمام الله والناس: اتحاد الإخوة، ومحبة القريب، ورجل وامرأة متفقان.
\par 2 ثلاثة أنواع من الرجال تكرههم نفسي، وأستاء من حياتهم: الفقير المتكبر، والغني الكذاب، والزاني الشيخ الفاقد.
\par 3 إذا لم تجمع شيئًا في شبابك، فكيف يمكنك أن تجد شيئًا في شيخوختك؟
\par 4 يا له من أمر جميل أن يحكم على الشيب، وأن يعرف القدماء المشورة!
\par 5 يا له من جمال لحكمة الشيوخ، والفهم والمشورة لرجال الشرف.
\par 6 الخبرة الكثيرة هي تاج الشيوخ، والخوف من الله هو مجدهم.
\par 7 هناك تسعة أشياء حكمت بها في قلبي على أنها سعيدة، والعاشرة سأنطق بها بلساني: الرجل الذي يفرح بأولاده، والذي يعيش ليرى سقوط عدوه.
\par 8 طوبى لمن يسكن مع امرأة عاقلة، ولم يزل لسانه، ولم يخدم من هو أهون منه.
\par 9 طوبى لمن وجد الحكمة ومن تكلم في آذان السامعين.
\par 10 ما أعظم من يجد الحكمة! ولكن ليس من يفوق من يتقي الرب.
\par 11 ولكن محبة الرب تفوق كل الأشياء إلى الاستنارة. فمن يتمسك بها فبماذا يشبه؟
\par 12 إن مخافة الرب هي بداية محبته، والإيمان هو بداية الالتصاق به.
\par 13 أعطني أي داء إلا داء القلب، وأي شر إلا شر المرأة.
\par 14 وكل بلاء إلا البلاء من مبغضي وكل انتقام إلا انتقام الأعداء.
\par 15 ليس رأس فوق رأس الحية، ولا غضب فوق غضب العدو.
\par 16 أفضل أن أسكن مع أسد وتنين، من أن أسكن مع امرأة شريرة.
\par 17 شر المرأة يغير وجهها ويظلم وجهها كالمسح.
\par 18 ويجلس زوجها بين جيرانه، وعندما يسمع ذلك يتنهد تنهيدة مريرة.
\par 19 كل الشرور قليلة مقارنة بشر المرأة: فليقع عليها نصيب الخاطئ.
\par 20 كما أن تسلق طريق رملي بالنسبة لأقدام كبار السن، كذلك الزوجة المليئة بالكلام بالنسبة للرجل الهادئ.
\par 21 لا تتعثر في جمال المرأة، ولا تشتهها من أجل اللذة.
\par 22 المرأة إذا حافظت على زوجها تكون مليئة بالغضب والوقاحة واللوم الكثير.
\par 23 المرأة الشريرة تضعف الشجاعة وتجعل الوجه ثقيلا والقلب مجروحا. المرأة التي لا تعزي زوجها في الضيق تجعل الأيدي ضعيفة والركب مرتخية.
\par 24 من المرأة كانت بداية الخطيئة، ومن خلالها نموت جميعنا.
\par 25 لا تتركوا للماء طريقا، ولا للمرأة الشريرة حرية التبول.
\par 26 وإن لم تسر كما تريد، فاقطعها من جسدك وأعطها كتاب طلاق، ودعها تذهب.

\chapter{26}

\par 1 طوبى للرجل الذي تكون له امرأة فاضلة، فإن عدد أيامه يتضاعف.
\par 2 المرأة الفاضلة تفرح زوجها، فيكمل سني حياته بسلام.
\par 3 المرأة الصالحة هي نصيب صالح، يُعطى من نصيب الذين يخافون الرب.
\par 4 سواء كان الإنسان غنيًا أو فقيرًا، إذا كان قلبه جيدًا نحو الرب، فإنه يفرح دائمًا بوجه مبتهج.
\par 5 ثلاثة أشياء يخافها قلبي، والرابع خشيت منه خوفاً عظيماً: افتراء على مدينة، وتجمع جمع غفير، واتهام كاذب. كل هذه أشر من الموت.
\par 6 ولكن حزن القلب والحزن هما من امرأة تغار على امرأة أخرى، وسوط اللسان الذي يخاطب الجميع.
\par 7 المرأة الشريرة نير متذبذب من أمسكها كأنه أمسك عقرباً.
\par 8 المرأة السكرانة والمتعجرفة تسببان غضبا عظيما ولا تريد أن تخفي عورتها.
\par 9 قد يُعرف زنا المرأة من نظراتها المتغطرسة وجفونها.
\par 10 إذا كانت ابنتك عديمة الحياء، فاحفظها في مكان ضيق، لئلا تسيء إلى نفسها من خلال الحرية الزائدة.
\par 11 احذر من العين الوقحة، ولا تتعجب إذا أخطأت في حقك.
\par 12 فتفتح فمها كالمسافر العطشان حين يجد نبعاً وتشرب من كل ماء بقربها. بجانب كل سياج تجلس وتفتح جعبتها لكل سهم.
\par 13 إن نعمة المرأة تسعد زوجها، وعقلها يقوي عظامه.
\par 14 إن المرأة الصامتة والمحبة هي هبة من الرب، وليس هناك شيء أثمن من العقل المتعلم جيدًا.
\par 15 إن المرأة الخجولة والمخلصة هي نعمة مزدوجة، ولا يمكن تقدير عقلها المتعجرف.
\par 16 كما تشرق الشمس في السماء العالية، كذلك جمال المرأة الصالحة في ترتيب بيتها.
\par 17 كما أن الضوء الصافي على المنارة المقدسة، كذلك جمال الوجه في سن النضج.
\par 18 كما أن الأعمدة الذهبية على قواعد الفضة، كذلك الأقدام الجميلة ذات القلب الثابت.
\par 19 يا ابني، حافظ على زهرة عمرك سليمة، ولا تعطي قوتك للغرباء.
\par 20 عندما تحصل على ممتلكات مثمرة في كل الحقل، ازرعها ببذورك الخاصة، واثقًا بجودة نسلك.
\par 21 لذلك فإن جنسك الذي تتركه سوف يتعظم، لأنه يتمتع بالثقة في نسبه الصالح.
\par 22 تحسب الزانية كالبصاق، وأما المرأة المتزوجة فهي برج من الموت لزوجها.
\par 23 المرأة الشريرة تُعطى نصيباً لرجل شرير، أما المرأة التقية فتُعطى لمن يتقي الرب.
\par 24 المرأة الخائنة تحتقر العار، أما المرأة الصادقة فتحترم زوجها.
\par 25 المرأة التي لا تخجل تحسب كلباً، أما المرأة المستحيية فتتقي الرب.
\par 26 المرأة التي تكرم زوجها تحسب حكيمة عند الجميع، وأما التي تهينه بكبريائها تحسب غير تقيّة عند الجميع.
\par 27 امرأة تصرخ بصوت عال وتوبخ لطرد الأعداء.
\par 28 شيئان يحزنان قلبي والثالث يغيظني: رجل حرب يعاني الفقر، ورجال فهم لا يطيعون، ورجل يرجع عن البر إلى الخطيئة، فإن الرب يعد مثل هذا للسيف.
\par 29 لا ينبغي للتاجر أن يمنع نفسه من ارتكاب الخطأ، ولا ينبغي للبائع أن يتحرر من الخطيئة.

\chapter{27}

\par 1 كثيرون أخطأوا لأجل تافه، ومن يطلب الغنى يصرف عينيه عنه.
\par 2 كما يعلق المسمار بين الحجارة، كذلك تلتصق الخطيئة بين البيع والشراء.
\par 3 إن لم يحفظ الإنسان نفسه في خوف الرب، فسوف يهدم بيته قريبًا.
\par 4 كما أن الإنسان عندما ينخل بالغربال تبقى القذارة، كذلك قذارة الإنسان في كلامه.
\par 5 إن الفرن يختبر آنية الخزاف، وكذلك امتحان الإنسان في تفكيره.
\par 6 إن الثمرة تعلن ما إذا كانت الشجرة قد تم تزيينها، وكذلك نطق الغرور في قلب الإنسان.
\par 7 لا تمدح أحداً قبل أن تسمعه يتكلم، لأن هذا هو اختبار الناس.
\par 8 إن اتبعت البر، فسوف تحصل عليها وتلبسها كثوب طويل مجيد.
\par 9 وترجع الطيور إلى أمثالها، وتعود الحقيقة إلى العاملين بها.
\par 10 كما يكمن الأسد للفريسة، كذلك الخطيئة لفاعلي الإثم.
\par 11 إن كلام الرجل التقي يكون دائما بالحكمة، أما الجاهل فيتغير كالقمر.
\par 12 إذا كنت من غير الحكماء، فاحفظ الوقت، ولكن كن دائمًا من ذوي الفهم.
\par 13 إن كلام السفهاء ممل، ولعبهم هو فجور الخطيئة.
\par 14 إن كلام من يحلف كثيرا يجعل الشعر يقف، ومشاجراتهم تجعل الأذن تصم.
\par 15 إن خصومة المتكبرين هي سفك دماء، وشتائمهم مؤلمة للأذن.
\par 16 من يكشف الأسرار يفقد مصداقيته، ولن يجد صديقًا لعقله أبدًا.
\par 17 أحب صديقك وكن مخلصًا له، ولكن إذا كشفت أسراره فلا تتبعه بعد الآن.
\par 18 فكما أن الإنسان يهلك عدوه، كذلك أنت تفقد محبة قريبك.
\par 19 كما يطلق العصفور من يده، هكذا تطلق جارك ولا تستعيده مرة أخرى.
\par 20 لا تتبعوه بعد الآن، لأنه بعيد جدًا، فهو كالظبي الذي نجا من الفخ.
\par 21 أما الجرح فيمكن أن يُضمد، وبعد الشتائم يمكن أن يكون هناك مصالحة، ولكن من يفشي الأسرار فهو بلا رجاء.
\par 22 من يغمز بعينيه يعمل شراً، ومن يعرفه يبتعد عنه.
\par 23 عندما تكون حاضرا، فإنه سوف يتكلم بلطف، وسوف يعجب بكلماتك: ولكن في النهاية سوف يلوي فمه، ويشتم أقوالك.
\par 24 لقد أبغضت أشياء كثيرة، ولكن ليس مثله، لأن الرب يبغضه.
\par 25 من يلقي حجراً على العلاء يلقيه على رأسه، والضربة الخادعة تصنع جراحاً.
\par 26 من يحفر حفرة يسقط فيها، ومن يضع فخاً يؤخذ فيها.
\par 27 من يصنع الشر فسوف يقع عليه ولا يعلم من أين يأتي.
\par 28 الاستهزاء والتوبيخ من المتكبرين، ولكن الانتقام كالأسد يكمن لهم.
\par 29 "الذين يفرحون بسقوط الصديق يقعون في الفخ، ويأكلهم الضيق قبل أن يموتوا."
\par 30 الحقد والغضب هما رجستان، والإنسان الخاطئ يكون له كلاهما.

\chapter{28}

\par 1 من انتقم ينال الانتقام من الرب، ويحفظ خطاياه.
\par 2 اغفر لقريبك الأذى الذي سببه لك، هكذا تُغفر خطاياك أيضًا عندما تصلي.
\par 3 هل يحمل الإنسان الكراهية تجاه إنسان آخر، ويطلب المغفرة من الرب؟
\par 4 فإنه لا يرحم إنساناً مثله، ويطلب مغفرة خطاياه؟
\par 5 إذا كان من هو مجرد جسد يغذي الكراهية، فمن الذي سيطلب المغفرة عن خطاياه؟
\par 6 اذكر نهايتك، فتبطل العداوة. اذكر الفساد والموت، واثبت في الوصايا.
\par 7 تذكر الوصايا ولا تحمل ضغينة لقريبك. تذكر عهد العلي وتغافل عن الجهل.
\par 8 امتنع عن الخصام فتُقَلِّل خطاياك، لأن الرجل الغاضب يُشعل الخصام،
\par 9 الرجل الخاطئ يقلق الأصدقاء، ويثير الجدل بين المسالمين.
\par 10 كما أن مادة النار تشتعل، وكما أن قوة الإنسان تشتعل، وكما أن غضبه يتزايد بحسب غناه، وكلما قوي المتخاصمون ازدادوا اشتعالاً.
\par 11 الخصام على عجل يشعل النار، والقتال على عجل يسفك الدم.
\par 12 إذا نفخت في شرارة فإنها تحترق، وإذا بصقت عليها فإنها تنطفئ، وكلاهما يخرجان من فمك.
\par 13 لعن النمام وذو اللسانين، فإن مثل هؤلاء أهلكوا كثيرين من أهل السلام.
\par 14 لسان الثرثرة أزعج كثيرين وطردهم من أمة إلى أمة، هدم مدناً حصينة وهدم بيوت العظماء.
\par 15 اللسان الغيّاب يطرد النساء الفاضلات ويحرمهن من أعمالهن.
\par 16 من يصغي إليه لن يجد راحة أبدًا، ولن يسكن بهدوء أبدًا.
\par 17 ضربة السوط تترك علامات في الجسد، وضربة اللسان تحطم العظام.
\par 18 كثيرون سقطوا بحد السيف، ولكن ليس مثل الذين سقطوا باللسان.
\par 19 طوبى لمن نجا من سمها، ولم يجرّد من نيرها، ولم يقيّد بقيودها.
\par 20 لأن نيره نير من حديد وقيوده قيود من نحاس.
\par 21 إن موته ميتة سيئة، والقبر خير منه.
\par 22 لا يكون له سلطان على خائفي الله، ولا يحرقون بلهيبه.
\par 23 من يترك الرب يسقط فيه، فيحترق فيهم ولا ينطفئ، يرسل عليهم كالأسد، ويأكلهم كالنمر.
\par 24 انظر إلى أنك تحيط ممتلكاتك بالشوك، وتربط فضتك وذهبك،
\par 25 ووزن كلماتك بالميزان، واجعل لفمك بابا ومسدسا.
\par 26 احذر أن تنزلق منه، لئلا تسقط في يد من يتربص لك.

\chapter{29}

\par 1 من يرحم يقرض قريبه، ومن يشدد يده يحفظ الوصايا.
\par 2 أقرض جارك في وقت حاجته، ورد له القرض في حينه.
\par 3 احتفظ بكلمتك، وتعامل معه بأمانة، وسوف تجد دائمًا الشيء الذي تحتاجه.
\par 4 كثيرون، عندما أقرضوا شيئًا، حسبوه موجودًا، ووضعوهم في مشكلة من أجل مساعدتهم.
\par 5 حتى يأخذ، فإنه يقبل يد الإنسان، وعلى مال قريبه يتكلم بتواضع، ولكن عندما يجب أن يسدد، فإنه سوف يطيل الوقت، ويرد بكلمات الحزن، ويشكو من الوقت.
\par 6 فإن انتصر، فإنه لن يحصل على النصف بصعوبة، وسوف يعتبر كما لو أنه وجده: وإلا، فقد حرمه من ماله، وقد كسب له عدوًا بلا سبب: يدفع له اللعنات والشتائم؛ وبدلاً من الشرف سوف يدفع له العار.
\par 7 ولذلك رفض كثيرون الإقراض بسبب سوء معاملة الآخرين، خوفًا من أن يتم الاحتيال عليهم.
\par 8 ولكن عليك أن تتحلى بالصبر مع الرجل الفقير، ولا تتأخر في إظهار الرحمة له.
\par 9 أعينوا الفقير من أجل الوصية، ولا ترفضوه بسبب فقره.
\par 10 أضيع مالك من أجل أخيك وصديقك، ولا تدعه يصدأ تحت الحجر فيضيع.
\par 11 ادخر كنزك حسب وصايا العلي، فيعود عليك بربح أكثر من الذهب.
\par 12 احفظ الصدقات في مخازنك، فتنجيك من كل ضيق.
\par 13 فإنه يقاتل عنك أعداءك أفضل من درع عظيم ورمح قوي.
\par 14 الرجل الأمين يضمن قريبه، أما الوقح فيتركه.
\par 15 لا تنسى صداقة من يضمنك، فإنه بذل حياته من أجلك.
\par 16 سوف يفسد الخاطئ حسنات كفيله:
\par 17 ومن كان جاحداً يترك الذي خلصه.
\par 18 لقد أفسد الضمان كثيرين من أصحاب الأموال الطيبة، وزلزلهم كموج البحر. وطرد الأقوياء من بيوتهم، حتى تاهوا بين الأمم الغريبة.
\par 19 الإنسان الشرير الذي يتعدى على وصايا الرب يقع في الكفالة، والذي يقوم بأعمال الآخرين ويتبعها من أجل الربح يقع في الدعاوى.
\par 20 ساعد قريبك حسب قوتك، واحذر أن تقع أنت نفسك في نفس الفخ.
\par 21 إن أهم شيء في الحياة هو الماء والخبز والملابس والبيت لستر العار.
\par 22 إن حياة الرجل الفقير في كوخ متواضع أفضل من طعام شهي في منزل رجل آخر.
\par 23 سواء كان الأمر صغيرًا أو كثيرًا، فاكتفِ بأن لا تسمع تعيير بيتك.
\par 24 فإن الانتقال من بيت إلى بيت هو حياة بائسة، لأنه حيث تكون غريباً لا تجرؤ على فتح فمك.
\par 25 "إنك تستضيف وتحتفل ولا تشكر، فضلاً عن أنك ستسمع كلمات مرة."
\par 26 تعال أيها الغريب، وأعد لي مائدة وأطعمني مما أعددت.
\par 27 أيها الغريب، أفسِح مكانك لرجل شريف، أخي يأتي ليبيت، وأنا أحتاج إلى بيتي.
\par 28 هذه الأمور ثقيلة على الإنسان العاقل: توبيخ البيت وتأنيب المقرض.

\chapter{30}

\par 1 من يحب ابنه يجعله يشعر بالذنب كثيرًا، حتى يفرح به في النهاية.
\par 2 من يؤدب ابنه يفرح به ويفرح به بين معارفه.
\par 3 من يعلّم ابنه يحزن عدوه، ويفرح به أمام أصدقائه.
\par 4 وإن مات أبوه فهو كأنه لم يمت، لأنه ترك خلفه من يشبهه.
\par 5 في حياته رآه ففرح به، وعندما مات لم يحزن.
\par 6 لقد ترك خلفه منتقمًا من أعدائه، ومكافئًا لأصدقائه.
\par 7 من يبالغ في تقدير ابنه فإنه يضمد جراحه، وسوف تضطرب أحشاؤه عند كل صراخ.
\par 8 الحصان غير المدرب يصبح عنيدًا، والطفل الذي يترك لنفسه يصبح عنيدًا.
\par 9 إذا دللت طفلك فإنه سيجعلك خائفاً، وإذا لعبت معه فإنه سيجعلك تشعر بالثقل.
\par 10 لا تضحك معه لئلا تحزن معه، ولئلا تصر أسنانك في النهاية.
\par 11 لا تعطيه الحرية في شبابه، ولا تتغافل عن حماقاته.
\par 12 احني عنقه وهو صغير، واضربيه على جنبيه وهو طفل، لئلا يعاند ويعصيك، فيجلب الحزن على قلبك.
\par 13 أدب ابنك، واجعله يعمل، لئلا يكون سلوكه الفاحش إهانة لك.
\par 14 إن الفقير الصحيح القوي في بنيته خير من غني مريض في جسده.
\par 15 الصحة والجسد السليم فوق كل ذهب، والجسد القوي فوق الثروة اللانهائية.
\par 16 لا غنى فوق صحة الجسد، ولا فرح فوق فرح القلب.
\par 17 الموت أفضل من حياة مرة أو مرض مستمر.
\par 18 إن الأطعمة الشهية التي تُسكب على فم مغلق تشبه قطع اللحم الموضوعة على القبر.
\par 19 ما فائدة التقدمة للصنم؟ لأنه لا يأكل ولا يشم، هكذا من يضطهد من الرب.
\par 20 فينظر بعينيه ويتأوه، كالخصي الذي يعانق العذراء ويتنهد.
\par 21 لا تستسلم لحزنك، ولا تثقل نفسك بأفكارك.
\par 22 فرح القلب حياة الإنسان، وبهجة الإنسان تطيل أيامه.
\par 23 أحبب نفسك، وعزِّ قلبك، وأبعد الحزن عنك، لأن الحزن قتل كثيرين، وليس فيه منفعة.
\par 24 الحسد والغضب يقصران العمر، والحذر يجلب الشيخوخة قبل أوانها.
\par 25 إن القلب البشوش والطيب يهتم بطعامه ونظامه الغذائي.

\chapter{31}

\par 1 إن السهر على الثروات يستهلك الجسد، والاهتمام بها يطرد النوم.
\par 2 إن الحرص على اليقظة لا يسمح للإنسان بالنوم، كما أن المرض الخبيث يقطع النوم،
\par 3 إن الغني يتعب كثيراً في جمع الثروات، وعندما يستريح يمتلئ بأغراضه الفاخرة.
\par 4 الفقير يعمل في فقره، وعندما يتركه يظل محتاجًا.
\par 5 من يحب الذهب لن يتبرر، ومن يتبع الفساد لن ينال منه ما يكفيه.
\par 6 لقد كان الذهب سبباً في هلاك الكثيرين، وكان دمارهم حاضراً.
\par 7 إنها عثرة للذين يذبحون لها، وكل أحمق يؤخذ معها.
\par 8 طوبى للغني الذي وجد بلا عيب ولم يسع وراء الذهب.
\par 9 من هو؟ فنطوبه لأنه صنع عجائب بين شعبه.
\par 10 من ذا الذي جُرِّبَ به ووُجِد كاملاً؟ فليفتخر. من ذا الذي أخطأ ولم يُخطئ؟ أو فعل الشر ولم يفعله؟
\par 11 وتثبت أمواله، وتخبر الجماعة بصدقاته.
\par 12 إذا جلست على مائدة دسمة فلا تكن جشعاً فيها، ولا تقل: يوجد عليها الكثير من اللحم.
\par 13 "اذكر أن العين الشريرة شريرة، وأي شيء خلق أشر من العين؟ لذلك تبكي في كل وقت."
\par 14 لا تمد يدك حيثما تتطلع، ولا تضعها معه في الصحفة.
\par 15 لا تحكم على قريبك من تلقاء نفسك، وكن حكيما في كل شيء.
\par 16 كُلْ كما يليق بالإنسان ما يُقدَّم إليك، ولا تبتلعه لئلا تُبغض.
\par 17 اترك الأمر أولاً من أجل الأخلاق، ولا تكن غير راضٍ حتى لا تُسيء إلى الآخرين.
\par 18 عندما تجلس بين الكثيرين، فلا تمد يدك أولاً.
\par 19 إن القليل جدًا يكفي الرجل الذي يتمتع برعاية جيدة، ولا ينقص ريحه على فراشه.
\par 20 النوم العميق يأتي من الأكل المعتدل: يستيقظ مبكرًا، وعقله معه: لكن ألم السهر، والغثيان، وآلام البطن، تصاحب الرجل الذي لا يشبع.
\par 21 وإذا اضطررت إلى تناول الطعام، فقم، واخرج، وتقيأ، وسوف تحصل على الراحة.
\par 22 يا ابني، اسمع لي ولا تحتقرني، وفي النهاية تجد كما قلت لك: في جميع أعمالك كن سريعاً، فلا يأتيك مرض.
\par 23 من كان كريماً في طعامه، فسوف يتحدث الناس عنه جيداً، وسوف يُصدق تقريره عن حسن تدبيره للمنزل.
\par 24 وأما من يبخل في طعامه فتتذمر عليه كل المدينة، ولا يشك في شهادات بخله.
\par 25 لا تظهر شجاعتك بالخمر، لأن الخمر أهلك كثيرين.
\par 26 إن الفرن يختبر الحد بالغمس، وكذلك الخمر يسقي قلوب المتكبرين بالسكر.
\par 27 إن الخمر كالحياة للإنسان إذا شربه باعتدال، فما هي الحياة إذن للإنسان الذي لا يشرب الخمر؟ لأنه صنع لفرح الناس.
\par 28 إن شرب الخمر في وقته وبكمية مناسبة يجلب سرور القلب وبهجة العقل.
\par 29 ولكن الخمر المسكر بكثرة يثير مرارة النفس، ويثير الخصام والخصام.
\par 30 السُكر يزيد من غضب الأحمق حتى يرتكب الذنب، ويُضعف القوة ويُسبب الجروح.
\par 31 لا توبخ قريبك عند الخمر، ولا تحتقره في فرحه، ولا تقل له كلاما جارحا، ولا تلح عليه بالحث على الشرب.

\chapter{32}

\par 1 إذا أصبحت سيدًا [لحفلة] فلا ترتفع، بل كن بينهم كواحد من الباقين. اعتنِ بهم جيدًا، وهكذا تجلس.
\par 2 وعندما تنتهي من جميع واجباتك، خذ مكانك، حتى تتمكن من الفرح معهم، وتنال الإكليل على تنظيمك الجيد للوليمة.
\par 3 تكلم أيها الشيخ، فهذا يليق بك، ولكن بحكم سليم، ولا تمنع الموسيقى.
\par 4 لا تسكب الكلمات حيث يوجد موسيقي، ولا تظهر الحكمة خارج الوقت.
\par 5 إن حفلة موسيقية في وليمة نبيذ هي بمثابة خاتم من حجر كريم مرصع بالذهب.
\par 6 كخاتم زمرد مرصع في عمل من ذهب، كذلك لحن الموسيقى مع النبيذ اللطيف.
\par 7 تكلم أيها الشاب إذا كنت في حاجة إلى ذلك، ولكن نادرًا ما يحدث ذلك عندما يُطلب منك ذلك مرتين.
\par 8 فليكن كلامك قصيرًا، شاملًا الكثير في كلمات قليلة؛ كن كمن يعرف ولكنه يمسك لسانه.
\par 9 إذا كنت بين الرجال العظماء فلا تجعل نفسك مساوياً لهم، وإذا كان الرجال القدامى في مكانهم فلا تستخدم الكثير من الكلمات.
\par 10 قبل الرعد يذهب البرق، وقبل الرجل الخجول يذهب النعمة.
\par 11 استيقظ مبكرًا، ولا تكن الأخير؛ بل اذهب إلى منزلك دون تأخير.
\par 12 خذ وقتك وافعل ما تريد، ولكن لا تخطئ بالكلام المتكبر.
\par 13 "ولهذا بارك خالقك وملأك خيراته."
\par 14 من يتقي الرب ينال تأديبه، والذين يبادرون إليه يجدون نعمة.
\par 15 من يطلب الناموس يمتلئ منه، وأما المنافق فيعثر به.
\par 16 الذين يخافون الرب يجدون الحق ويشعلون العدل كالنور.
\par 17 الإنسان الخاطئ لا يُوبَّخ، بل يجد عذرًا حسب إرادته.
\par 18 إن الرجل المستشار يكون مراعياً للآخرين، أما الرجل الغريب والمتكبر فلا يخاف حتى لو كان قد استغنى عن المشورة.
\par 19 لا تفعل شيئا دون نصيحة، وعندما تفعله مرة واحدة، لا تتب.
\par 20 لا تذهب في طريق قد تسقط فيه، ولا تتعثر بين الحجارة.
\par 21 لا تكن واثقًا بطريقة واضحة.
\par 22 واحذروا من أبنائكم.
\par 23 في كل عمل صالح، اعتمد على نفسك، لأن هذا هو حفظ الوصايا.
\par 24 من يؤمن بالرب يحفظ الوصية، ومن يتوكل عليه لا يسوءه شيء.

\chapter{33}

\par 1 لا يصيب من يتقي الرب شر، بل في التجربة ينجيه أيضا.
\par 2 الرجل الحكيم لا يكره الناموس، أما المنافق فيه فهو كالسفينة في العاصفة.
\par 3 الرجل العاقل يتوكل على الناموس، والناموس أمين عليه كالوحي.
\par 4 أعد ما ستقوله، وهكذا يُسمع لك. وأعد التعليم، ثم قدم الجواب.
\par 5 قلب الجاهل كعجلة العربة، وأفكاره كشجرة تدور.
\par 6 الحصان هو صديق ساخر، يصهل تحت كل من يجلس عليه.
\par 7 لماذا يتفوق يوم على يوم آخر، بينما ضوء كل يوم في السنة هو من الشمس؟
\par 8 وبمعرفة الرب تميزوا وغير الأوقات والأعياد.
\par 9 بعضها جعل أياماً عظيمة فقدسها، وبعضها جعل أياماً عادية.
\par 10 وكل البشر من التراب، وآدم خلق من التراب.
\par 11 فبمعرفة كثيرة قسمهم الرب، وجعل طرقهم متنوعة.
\par 12 فمنهم من باركه ورفعه، ومنهم من قدسه وقربه، ومنهم من لعنه وأذله وطرده من أماكنه.
\par 13 كما أن الطين في يد الخزاف ليصنعه على هواه، كذلك الإنسان في يد صانعه ليصنعه على مثاله.
\par 14 الخير ضد الشر، والحياة ضد الموت، وكذلك التقي ضد الخاطئ، والخاطئ ضد التقي.
\par 15 فانظروا إلى جميع أعمال العلي، وهناك اثنان واثنان، الواحد ضد الآخر.
\par 16 "استيقظت آخراً كالجامع وراء القطافين، ببركة الرب نفعت، وداست معصرتي كقاطف عنب."
\par 17 فاعلم أني لم أعمل لنفسي فقط، بل لجميع الذين يطلبون التعلم.
\par 18 اسمعوني يا عظماء الشعب، وأصغوا بآذانكم يا رؤساء الجماعة.
\par 19 لا تعطِ ابنك وامرأتك وأخاك وصديقك سلطة عليك وأنت حي، ولا تعطي أموالك لآخر لئلا يندم عليك فتطلب مثلها.
\par 20 ما دمت حيًا وفيك نفس، فلا تسلم نفسك لأحد.
\par 21 فمن الأفضل أن يلجأ إليك أبناؤك، بدلاً من أن تلتزم أنت بمجاملتهم.
\par 22 في كل أعمالك، احتفظ لنفسك بالصدارة، ولا تترك وصمة عار في شرفك.
\par 23 في الوقت الذي ستنتهي فيه أيامك وتنتهي حياتك، قم بتوزيع ميراثك.
\par 24 العلف والعصا والأثقال للحمار، والخبز والتأديب والعمل للعبد.
\par 25 إذا جعلت عبدك يعمل تجد راحة، ولكن إذا تركته بلا عمل يطلب الحرية.
\par 26 إن النير والطوق يحنيان الرقبة، وكذلك العذابات والآلام التي يتعرض لها العبد الشرير.
\par 27 أرسله إلى العمل، لكي لا يكون عاطلاً، لأن الكسل يعلم شرورًا كثيرة.
\par 28 اجعلوه يعمل كما يليق به، وإذا لم يكن مطيعًا، فضعوا عليه قيودًا أثقل.
\par 29 ولكن لا تكن مفرطاً تجاه أحد، ولا تفعل شيئاً بدون حكمة.
\par 30 إن كان لك عبد فليكن لك كنفسك، لأنك اشتريته بثمن.
\par 31 إذا كان لك عبد فعامله كأخ، لأنك تحتاج إليه كما تحتاج إلى نفسك. إذا أسأت إليه فهرب من عندك، فأي طريق تذهب لتطلبه؟

\chapter{34}

\par 1 آمال الرجل عديم الفهم باطلة وكاذبة، والأحلام ترفع الجهال.
\par 2 من يهتم بالأحلام فهو كمن يمسك بالظل ويتبع الريح.
\par 3 إن رؤيا الأحلام هي تشابه الشيء مع الشيء، كما يشبه الوجه وجهاً لوجه.
\par 4 من الشيء النجس ما الذي يمكن تطهيره؟ ومن الشيء الكاذب ما الذي يمكن أن يأتي من الحقيقة؟
\par 5 إن التنبؤات والتنبؤات والأحلام باطلة، والقلب يفكر كقلب امرأة في مخاض.
\par 6 إن لم يرسلهم العلي في زيارتك فلا تضع قلبك عليهم.
\par 7 فإن الأحلام خدعت كثيرين، وفشلت الذين وضعوا ثقتهم فيها.
\par 8 "يكون الناموس كاملا بلا كذب، والحكمة كمال للفم الأمين."
\par 9 الرجل الذي سافر يعرف أشياء كثيرة، والذي لديه الكثير من الخبرة سوف يعلن الحكمة.
\par 10 من ليس لديه خبرة يعرف القليل، ولكن من سافر فهو مليء بالحكمة.
\par 11 عندما سافرت رأيت أشياء كثيرة، وأفهم أكثر مما أستطيع التعبير عنه.
\par 12 لقد كنت في كثير من الأحيان في خطر الموت: ولكنني نجوت بسبب هذه الأشياء.
\par 13 وتحيا روح الذين يخافون الرب، لأن رجاءهم هو في مخلصهم.
\par 14 من يتقي الرب فلا يخاف ولا يرتعب لأنه هو رجاؤه.
\par 15 طوبى لنفس الذي يتقي الرب، إلى من ينظر ومن هو قوته؟
\par 16 لأن عيني الرب على محبيه، هو حصنهم القوي وسندهم القوي، وقاية من الحر، وستار من شمس الظهيرة، وحفظ من العثرة، ومعونة من السقوط.
\par 17 يرفع النفس، وينير العيون، ويعطي الصحة والحياة والبركة.
\par 18 من يذبح من شيء مكتسب ظلماً فقربانه سخيف، وهدايا الظالمين غير مقبولة.
\par 19 لا يسر العلي بتقدمات الأشرار، ولا يرضى عن خطيئته بكثرة الذبائح.
\par 20 من قدم قربانًا من أموال الفقراء فإنه يفعل كمن يقتل الابن أمام أبيه.
\par 21 خبز المسكين هو حياته، ومن يحرمه منه فهو رجل دم.
\par 22 من يأخذ معيشة قريبه يقتله، ومن يسلب أجرة الأجير فهو سافك دم.
\par 23 عندما يبني واحد ويهدم آخر، فما الفائدة التي يحصلان عليها إلا العمل؟
\par 24 عندما يصلي واحد، ويلعن آخر، فصوت من يسمع الرب؟
\par 25 من اغتسل بعد لمس ميت، فإن مسه مرة أخرى، فما فائدة غسله؟
\par 26 هكذا الإنسان الذي يصوم عن خطاياه ثم يتوب فيفعل مثلها، فمن يسمع صلاته؟ وماذا ينفعه تواضعه؟

\chapter{35}

\par 1 من يحفظ الناموس يقدم ذبيحة كافية، ومن يحفظ الوصية يقدم ذبيحة سلامة.
\par 2 من يجازي بالحسنى يقدم دقيقًا جيدًا، ومن يتصدق بالصدقة يذبح الحمد.
\par 3 إن الابتعاد عن الشر هو أمر مرضي أمام الرب، وترك الإثم هو كفارة.
\par 4 لا تظهر فارغا أمام الرب.
\par 5 لأن هذه كلها يجب أن تتم حسب الوصية.
\par 6 تقدمة الصديق تسمن المذبح، ورائحته الطيبة تكون أمام العلي.
\par 7 إن تضحية الرجل الصالح مقبولة، ولا يُنسى ذكراها أبدًا.
\par 8 أعط الرب كرامته بعين صالحة، ولا تقلل من باكورة يديك.
\par 9 أظهر وجهًا مبتهجًا في كل عطاياك، وقدم عشورك بفرح.
\par 10 أعط العلي بقدر ما أغناك، وكما أخذت فأعط بعين سرور.
\par 11 لأن الرب يجازيك ويعطيك سبعة أضعاف.
\par 12 لا تفكروا في الإفساد بالعطايا، فإنه لن يقبل مثل هذه. ولا تتوكلوا على الذبائح الظالمة، لأن الرب هو الديان وليس عنده محاباة للأشخاص.
\par 13 لا يقبل أحداً على فقير، بل يسمع دعاء المظلوم.
\par 14 لا يحتقر دعاء اليتيم ولا الأرملة إذا سكب شكواها.
\par 15 أليست الدموع تسيل على خدود الأرملة؟ أليس صراخها على من يسقطها؟
\par 16 من يخدم الرب يُقبل بالرضا، وصلاته تصل إلى السحاب.
\par 17 صلاة المتواضع تخترق السحاب وحتى تقترب لا يتعزى ولا يزول حتى ينظر العلي ليقضي بالعدل ويجري الحكم.
\par 18 لأن الرب لا يبطئ، ولا يصبر عليهم الجبار، حتى يقطع أحقاء غير الرحماء، ويجازي الأمم بالانتقام، حتى يزيل جمهور المتكبرين، ويكسر صولجان الظالمين.
\par 19 حتى يجازي كل إنسان حسب أعماله، وأعمال البشر حسب تدبيرهم، حتى يحكم في قضية شعبه، ويجعلهم يفرحون برحمته.
\par 20 إن الرحمة في وقت الضيق، كما أن سحاب المطر في وقت القحط.

\chapter{36}

\par 1 ارحمنا يا رب إله الكل وانظر إلينا.
\par 2 وأرسل خوفك على جميع الأمم الذين لا يطلبونك.
\par 3 ارفع يدك على الأمم الغريبة، فيروا قوتك.
\par 4 كما تقدست فينا أمامهم، فكذلك تَعَظَّم فيهم أمامنا.
\par 5 وليعرفوك كما عرفناك أنه لا إله إلا أنت يا الله.
\par 6 أصنع آيات جديدة وعجائب أخرى عجيبة. مجد يدك وذراعك اليمنى لكي يحدثوا بأعمالك العجيبة.
\par 7 أثير السخط واسكب الغضب، أزل الخصم وأهلك العدو.
\par 8 اغتنم الفرصة، وتذكر العهد، ودعهم يخبرون بأعمالك العجيبة.
\par 9 فليحترق الناجي بحرقة النار، وليهلك الذين يظلمون الشعب.
\par 10 أحطم رؤوس حكام الأمم الذين يقولون: ليس غيرنا.
\par 11 اجمع كل أسباط يعقوب معًا، فتورثهم كما في البدء.
\par 12 يا رب ارحم الشعب الذي دعي باسمك، وإسرائيل الذي سميت بكرك.
\par 13 ارحم أورشليم مدينتك المقدسة ومكان راحتك.
\par 14 إملأ صهيون من وحيك الذي لا يوصف، وشعبك من مجدك.
\par 15 "أشهد لمن ملكت من البدء، وأقم الأنبياء الذين كانوا باسمك."
\par 16 كافئ الذين ينتظرونك، وليكن أنبياؤك أمناء.
\par 17 يا رب، اسمع صلاة عبيدك، حسب بركة هارون على شعبك، لكي يعلم جميع سكان الأرض أنك أنت الرب الإله الأزلي.
\par 18 البطن يلتهم كل الأطعمة، ولكن هل يوجد طعام أفضل من آخر.
\par 19 كما أن الحنك يتذوق أصنافاً من لحوم الغزلان، كذلك القلب العاقل يتذوق الأقوال الكاذبة.
\par 20 القلب الملتوي يسبب الثقل، ولكن الرجل ذو الخبرة يكافئه.
\par 21 ستقبل المرأة كل رجل، ولكن هل هناك ابنة أفضل من الأخرى؟
\par 22 جمال المرأة يبهج الوجه، ولا يحب الرجل شيئًا أفضل منها.
\par 23 إذا كان في لسانها اللطف والوداعة والراحة، فهل زوجها ليس كباقي الرجال؟
\par 24 من يحصل على زوجة يبدأ في تملك شيء ما، معينًا مثله، وعمود راحة.
\par 25 حيث لا يوجد سياج، هناك تُنهب الممتلكات، ومن ليس له زوجة يتجول جيئة وذهابا حزينًا.
\par 26 من يثق بلصٍّ مُعيَّنٍ يقفز من مدينةٍ إلى مدينة؟ ومن يثق برجلٍ لا بيتَ له، ويبيت حيثما يأخذه الليل؟

\chapter{37}

\par 1 كل صديق يقول أنا صديقه أيضًا، ولكن يوجد صديق، وهو صديق بالاسم فقط.
\par 2 أليس الحزن حتى الموت عندما يتحول الصديق والرفيق إلى عدو؟
\par 3 يا أيها الخيال الشرير، من أين أتيت لتغطي الأرض بالخداع؟
\par 4 هناك رفيق يفرح بنجاح صديقه، ولكن في وقت الضيق يكون ضده.
\par 5 هناك رفيق يساعد صديقه في البطن، ويحمل الدرع ضد العدو.
\par 6 لا تنس صديقك في عقلك، ولا تغفل عنه في غناك.
\par 7 كل مشير يشيد بالمشورة، ولكن يوجد من ينصح لنفسه.
\par 8 احذر من المستشار واعرف مسبقا ما هي حاجته، فإنه ينصح لنفسه، لئلا يلقي عليك القرعة.
\par 9 ويقول لك طريقك صالح وبعد ذلك يقف في العبر لينظر ماذا يصيبك.
\par 10 لا تتشاور مع من يشك فيك، ولا تخف مشورتك عن الذين يحسدونك.
\par 11 ولا تشاور امرأة تغار منها، ولا جباناً في أمور الحرب، ولا تاجراً في الصرف، ولا مشترياً في البيع، ولا حاسداً شاكراً، ولا عديم الرحمة في اللطف، ولا كسولاً في أي عمل، ولا أجيراً في سنة من العمل، ولا عبداً بطالاً كثير العمل. لا تسمع لهؤلاء في كل أمر مشورة.
\par 12 "ولكن كوني دائمًا مع رجل تقي، تعرفين أنه يحفظ وصايا الرب، الذي يكون رأيه حسب رأيك، وسوف يحزن معك إذا أجهضت."
\par 13 ولتثبت مشورة قلبك، فإنه ليس أحد أكثر أمانة لك منه.
\par 14 فعقل الإنسان قد يخبره أحيانًا بأكثر مما يخبره به سبعة حراس يجلسون في برج عالٍ.
\par 15 وفوق كل هذا صلِّ إلى العلي، لكي يهدي طريقك إلى الحق.
\par 16 فليكن العقل قبل كل عمل، والمشورة قبل كل فعل.
\par 17 الوجه علامة على تغير القلب.
\par 18 تظهر أربعة أنواع من الأشياء: الخير والشر، الحياة والموت: لكن اللسان يسود عليها دائمًا.
\par 19 يوجد واحد حكيم ويعلم كثيرين ولكنه لا ينفع نفسه.
\par 20 يوجد من يتكلم بالحكمة فيُبغض، ويُحرم من كل طعام.
\par 21 لأنه لم تُعطَ له النعمة من الرب، لأنه محروم من كل حكمة.
\par 22 والآخر حكيم في نفسه، وثمار الفهم محمودة في فمه.
\par 23 الرجل الحكيم يعلم شعبه، وثمار فهمه لا تزول.
\par 24 الرجل الحكيم يمتلئ بالبركة، وكل من يراه يحسبه سعيداً.
\par 25 إن أيام حياة الإنسان معدودة، أما أيام إسرائيل فلا تحصى.
\par 26 الرجل الحكيم يرث مجداً بين شعبه، ويكون اسمه إلى الأبد.
\par 27 يا ابني، جرب نفسك في حياتك، وانظر ما هو الشر فيها، ولا تعطيها ذلك.
\par 28 لأنه ليس كل الأشياء مفيدة لجميع الناس، وليس كل نفس ترضى بكل شيء.
\par 29 لا تكن غير راضٍ عن أي شيء لذيذ، ولا تكن جشعًا جدًا بشأن اللحوم:
\par 30 فإن الإفراط في تناول اللحوم يسبب المرض، والإفراط في تناول الطعام يؤدي إلى المغص.
\par 31 بسبب الإفراط في الأكل هلك كثيرون، ولكن من يتنبه يطيل عمره.

\chapter{38}

\par 1 أكرموا الطبيب حسب كرامته من أجل منافعكم التي قد تستفيدونها منه، لأن الرب خلقه.
\par 2 لأنه من العلي يأتي الشفاء، وينال كرامة من الملك.
\par 3 "إن مهارة الطبيب ترفع رأسه، ويكون في نظر العظماء محط إعجاب."
\par 4 لقد خلق الرب الأدوية من الأرض، والحكيم لا يكرهها.
\par 5 ألم يُحلَّ الماء بالخشب لكي تُعرَف فضائله؟
\par 6 وأعطى البشر مهارة لكي يُمجَّدوا في أعماله العجيبة.
\par 7 بمثل هذه يشفي الناس ويزيل آلامهم.
\par 8 من هؤلاء يصنع الصيدلاني دواءً، وليس لأعماله نهاية، ومنه السلام على كل الأرض.
\par 9 يا ابني لا تتهاون في مرضك بل صل إلى الرب فيشفيك.
\par 10 اترك الخطيئة، ورتب يديك، وطهر قلبك من كل شر.
\par 11 أعطوا رائحة سرور، وتذكارًا من السميذ، وقدموا تقدمة سمينة كأنها ليست موجودة.
\par 12 فأعطِ مكانًا للطبيب، فإن الرب خلقه. لا يتركك، لأنك تحتاج إليه.
\par 13 هناك وقت يكون فيه النجاح في أيديهم.
\par 14 لأنهم يصلون أيضًا إلى الرب لكي ينجح ما يقدمونه من راحة وعلاج لإطالة الحياة.
\par 15 من يخطئ أمام خالقه، فليقع في يد الطبيب.
\par 16 يا بني، دع الدموع تسقط على الميت، وابدأ في الندم، كما لو كنت قد عانيت من أذى كبير بنفسك؛ ثم غطِّ جسده حسب العادة، ولا تهمل دفنه.
\par 17 ابكي بمرارة، وأئن بصوت عالٍ، واستخدم الرثاء، كما يستحق، وذلك لمدة يوم أو يومين، لئلا يُقال عنك سوءًا: ثم عزي نفسك على ثقلك.
\par 18 فمن الثقل يأتي الموت، وثقل القلب يكسر القوة.
\par 19 وفي الضيق أيضاً يبقى الحزن، وحياة الفقراء لعنة القلب.
\par 20 لا تأخذ الثقل على قلبك، بل اطرده بعيدًا، وتذكر النهاية الأخيرة.
\par 21 لا تنسَ هذا الأمر، لأنه لا رجوع إلى الوراء: لن تُفيده، بل ستؤذي نفسك.
\par 22 اذكر حكمي، لأنه يكون حكمك كذلك. أمس لي واليوم لك.
\par 23 عندما يستريح الميت، فلتسترح ذكراه، وليكن له عزاء عندما تفارقه روحه.
\par 24 إن حكمة الرجل المتعلم تأتي من فرصة الفراغ، ومن لديه القليل من العمل يصبح حكيماً.
\par 25 كيف يستطيع أن يحصل على الحكمة من يمسك المحراث، ويفتخر بالمنخس، ويسوق الثيران، وينشغل بأعمالها، وحديثه عن الثيران؟
\par 26 فهو يبذل عقله في صنع الأخاديد، ويجتهد في إعطاء العلف للبقر.
\par 27 وهكذا كل نجار وصانع يتعب ليل نهار، والذين يقطعون وينقشون الأختام، ويجتهدون في صنع تنوعات كثيرة، وينغمسون في الصور المزيفة، ويسهرون على إتمام العمل.
\par 28 والحداد أيضاً جالس عند السندان، ينظر إلى العمل الحديدي، وبخار النار يهلك جسده، وهو يحارب حرارة الفرن: صوت المطرقة والسندان لا يزال في أذنيه، وعيناه لا تزالان تنظران إلى نموذج الشيء الذي يصنعه؛ يضع عقله لإكمال عمله، ويراقب لصقله تمامًا.
\par 29 هكذا يفعل الخزاف وهو جالس إلى عمله ويدير دولابه برجليه، وهو دائم الحرص على عمله، ويصنع كل عمله بعدد.
\par 30 "يصنع الطين بذراعه، ويخضع قوته أمام رجليه، ويجتهد في تمريره، ويجتهد في تنظيف الكورون."
\par 31 كل هؤلاء يتوكلون على أيديهم، وكل واحد حكيم في عمله.
\par 32 بدون هذه لا يمكن أن تسكن مدينة ولا يسكنون حيث يشاؤون ولا يصعدون وينزلون.
\par 33 لا يطلبون في مجلس عام، ولا يجلسون في رؤساء الجماعة، ولا يجلسون على كرسي القضاء، ولا يفهمون حكم القضاء، ولا يستطيعون أن ينطقوا بالحق والعدل، ولا يوجدون حيث تقال الأمثال.
\par 34 لكنهم سيحافظون على حالة العالم، وكل رغباتهم في عمل حرفتهم.

\chapter{39}

\par 1 "ولكن من يوجه عقله إلى شريعة العلي، وينشغل بالتأمل فيها، فإنه سوف يبحث عن حكمة القدماء جميعهم، وينشغل بالنبوات.
\par 2 ويحفظ أقوال الرجال المشهورين، وحيث تكون الأمثال الدقيقة، فهو هناك أيضًا.
\par 3 سيبحث عن أسرار الأحكام الخطيرة، ويكون على دراية بالأمثال المظلمة.
\par 4 فيخدم بين العظماء ويظهر أمام الأمراء، ويسافر في بلاد غريبة، لأنه جرب الخير والشر بين الناس.
\par 5 فيُبْكِر قلبه إلى الرب صانعه، ويصلي أمام العلي، ويفتح فمه بالصلاة، ويتضرع من أجل خطاياه.
\par 6 عندما يشاء الرب العظيم يمتلئ بروح الفهم، ويفيض بالحكمة، ويحمد الرب في صلاته.
\par 7 فيُدير مشورته ومعرفته، وفي أسراره يتأمل.
\par 8 ويُخبر بما تعلمه، ويفتخر بشريعة عهد الرب.
\par 9 كثيرون يمدحون فهمه، وما دام العالم قائما فلن يمحى، ذكره لن يزول، واسمه يحيا من جيل إلى جيل.
\par 10 الأمم تخبر بحكمته، والجماعة تخبر بتسبيحه.
\par 11 إن مات يترك اسما أعظم من ألف، وإن عاش يزيده.
\par 12 ولكن لا يزال لدي المزيد لأقوله، مما فكرت فيه؛ لأنني ممتلئ مثل القمر في اكتماله.
\par 13 اسمعوا لي أيها الأطفال القديسون، وأنبتوا كوردة تنمو بجانب جدول الحقل:
\par 14 وأعطوا رائحة طيبة كاللبان، وزهروا كالسوسنة، وأرسلوا رائحة طيبة، وغنوا أغنية التسبيح، وباركوا الرب في كل أعماله.
\par 15 عظموا اسمه، وأظهروا تسبيحه بأغاني شفاهكم وبالقيثارات، وفي تسبيحه تقولون هكذا:
\par 16 كل أعمال الرب صالحة جدا، وكل ما يأمر به سيتم في حينه.
\par 17 ولا يستطيع أحد أن يقول: ما هذا؟ ولماذا؟ لأنه في الوقت المناسب يُطلبون جميعًا. بأمره وقفت المياه ككومة، وبكلمات فمه أحواض المياه.
\par 18 بأمره يتم كل ما يرضيه، ولا أحد يستطيع أن يمنع، عندما يريد أن يخلص.
\par 19 أعمال كل ذي جسد أمامه، ولا شيء يخفى عن عينيه.
\par 20 فهو ينظر من الأزل إلى الأبد، وليس أمامه شيء عجيب.
\par 21 لا يحتاج الإنسان إلى أن يقول: ما هذا؟ ولماذا؟ لأنه خلق كل الأشياء لاستخداماتها.
\par 22 بركاته غطت الأرض اليابسة كالنهر، وسقتها كالسيل.
\par 23 كما حول المياه إلى ملح، كذلك يرث الأمم غضبه.
\par 24 كما أن طرقه واضحة للقديسين، كذلك هي معثرة للأشرار.
\par 25 لأن الخيرات خلقت منذ البدء، والشرارات خلقت للخطاة.
\par 26 الأشياء الرئيسية للاستخدام الكامل لحياة الإنسان هي الماء والنار والحديد والملح ودقيق القمح والعسل والحليب ودم العنب والزيت والملابس.
\par 27 كل هذه الأشياء هي خير للأتقياء، أما بالنسبة للخطاة فهي تتحول إلى شر.
\par 28 هناك أرواح تم إنشاؤها للانتقام، والتي في غضبها تضع ضربات مؤلمة؛ في وقت الدمار فإنها تصب قوتها، وتهدئ غضب من خلقها.
\par 29 النار، والبرد، والمجاعة، والموت، كل هذه خلقت من أجل الانتقام؛
\par 30 أسنان الوحوش، والعقارب، والثعابين، والسيف يعاقب الأشرار بالهلاك.
\par 31 فيفرحون بأمره، ويكونون مستعدين على الأرض عندما تكون هناك حاجة إليهم، وعندما يأتي وقتهم، لا يتعدون كلامه.
\par 32 لذلك قررت منذ البداية وفكرت في هذه الأمور وتركتها مكتوبة.
\par 33 كل أعمال الرب صالحة، وهو يعطي كل شيء في حينه.
\par 34 حتى لا يستطيع أحد أن يقول: هذا أسوأ من ذلك، لأنه في الوقت المناسب سوف يتم تزكيتهم جميعًا.
\par 35 لذلك سبحوا الرب بكل قلوبكم وأفواهكم وباركوا اسم الرب.

\chapter{40}

\par 1 لقد خلق الله لكل إنسان عملاً عظيماً، ووضع على بني آدم نيراً ثقيلاً من يوم خروجهم من بطون أمهاتهم إلى يوم رجوعهم إلى أم الأشياء جميعاً.
\par 2 إن تصوراتهم للأمور الآتية ويوم الموت، [تزعج] أفكارهم، [وتسبب] خوفاً في قلوبهم؛
\par 3 من الجالس على عرش المجد إلى المتواضع في التراب والرماد.
\par 4 من الذي يلبس الأرجوان والتاج إلى الذي يلبس ثوبا من الكتان.
\par 5 الغضب، والحسد، والمتاعب، والاضطراب، والخوف من الموت، والغضب، والنزاع، وفي وقت الراحة على فراشه نومه الليلي، كل ذلك يغير معرفته.
\par 6 إن القليل أو لا شيء هو راحته، وبعد ذلك يكون في نومه، كما في يوم حراسة، مضطربًا في رؤية قلبه، وكأنه نجا من معركة.
\par 7 عندما أصبح كل شيء آمنًا، استيقظ، وتعجب من أن الخوف لم يكن شيئًا.
\par 8 [تحدث مثل هذه الأشياء] لكل ذي جسد، من إنسان إلى حيوان، ويزيد ذلك سبعة أضعاف على الخطاة.
\par 9 الموت، وسفك الدماء، والصراع، والسيف، والكوارث، والمجاعة، والضيق، والآفة؛
\par 10 هذه الأشياء خلقت من أجل الأشرار، ومن أجلهم جاء الطوفان.
\par 11 كل ما هو من الأرض يعود إلى الأرض، وكل ما هو من المياه يعود إلى البحر.
\par 12 كل رشوة وظلم سوف تمحى، ولكن التعامل الصادق سوف يبقى إلى الأبد.
\par 13 وتجف أموال الظالمين كالنهر، وتتلاشى مع الضجيج، مثل الرعد العظيم في المطر.
\par 14 عندما يفتح يده يفرح، وكذلك يبيد الأثمة.
\par 15 إن أبناء الأشرار لا ينبتون أغصاناً كثيرة، بل هم كالجذور النجسة على الصخر الصلب.
\par 16 يجب إزالة الحشائش التي تنمو على كل مياه النهر وضفافه قبل كل عشب.
\par 17 إن الخير كالحديقة المثمرة، والرحمة تدوم إلى الأبد.
\par 18 إن العمل والرضا بما يملكه الإنسان هو حياة حلوة، ولكن من يجد كنزًا فهو فوق كليهما.
\par 19 الأطفال وبناء المدينة يكملان اسم الرجل: ولكن المرأة التي لا تلوم تحسب فوق كليهما.
\par 20 إن الخمر والموسيقى يفرحان القلب، ولكن محبة الحكمة تفوقهما كليهما.
\par 21 المزمار والقيثارة يصنعان لحنًا حلوًا، لكن اللسان اللطيف فوقهما.
\par 22 إن عينك تشتهي الحسنة والجمال، ولكن أكثر من كليهما الذرة وهي خضراء.
\par 23 الصديق والرفيق لا يلتقيان خطأً أبدًا: ولكن فوق كليهما هناك المرأة مع زوجها.
\par 24 الإخوة والمعونة ضد وقت الضيق، ولكن الصدقات تنقذ أكثر منهما كليهما.
\par 25 الذهب والفضة يثبتان القدم، ولكن المشورة أعظم منهما.
\par 26 الغنى والقوة يرفعان القلب، ولكن مخافة الرب فوقهما كليهما، فلا نقص في مخافة الرب، ولا حاجة إلى طلب المساعدة.
\par 27 مخافة الرب جنة مثمرة، وتغطيه فوق كل مجد.
\par 28 يا ابني، لا تعش حياة المتسول، لأن الموت خير من التسول.
\par 29 إن حياة من يعتمد على مائدة رجل آخر لا تحسب له حياة؛ لأنه ينجس نفسه بطعام الآخرين. لكن الرجل الحكيم الذي يرعى جيدًا سوف يحذر من ذلك.
\par 30 إن التسول حلو في فم الوقح، ولكن في بطنه تشتعل نار.

\chapter{41}

\par 1 يا موت، ما أشد مرارة ذكراك على من يعيش في راحة في ممتلكاته، وعلى من ليس لديه ما يزعجه، والذي يتمتع بالرخاء في كل شيء: نعم، على من لا يزال قادرا على تناول الطعام!
\par 2 يا موت، إن حكمك مقبول للمحتاج، ولمن ضعفت قوته، الذي هو الآن في العصر الأخير، وقد انزعج من كل شيء، ولمن يئس وفقد صبره!
\par 3 لا تخف من حكم الموت، تذكر الذين كانوا قبلك والذين يأتون بعدك، لأن هذا هو حكم الرب على كل ذي جسد.
\par 4 ولماذا تخالف رضا العلي؟ ليس هناك تفتيش في القبر، سواء عشت عشر سنوات، أو مئة، أو ألف سنة.
\par 5 إن أولاد الخطاة هم أولاد رجسون، وهم الذين يسكنون مساكن الأشرار.
\par 6 فإن ميراث أبناء الخطاة يهلك، وذريتهم تكون عارًا أبديًا.
\par 7 وسوف يشكو الأبناء من الأب غير التقي، لأنهم سوف يوبخون بسببه.
\par 8 ويل لكم أيها الرجال الأشرار الذين تركتم شريعة الله العلي، لأنه إذا كثرتم يكون ذلك لهلاككم.
\par 9 وإن ولدتم فستولدون للعنة، وإن متم فاللعنة ستكون نصيبكم.
\par 10 كل من هو من الأرض يعود إلى الأرض أيضًا، وهكذا ينتقل الأشرار من اللعنة إلى الهلاك.
\par 11 إن حزن الرجال يتعلق بأجسادهم، ولكن الاسم السيئ للخطاة سوف يُمحى.
\par 12 انظر إلى اسمك، فإنه يبقى لك فوق ألف كنز عظيم من الذهب.
\par 13 إن الحياة الطيبة ليست إلا أياماً معدودة، أما السمعة الطيبة فتدوم إلى الأبد.
\par 14 يا أولادي، احفظوا التأديب في السلام. لأن الحكمة المخفية والكنز الذي لا يُرى، ما المنفعة فيهما كليهما؟
\par 15 الرجل الذي يخفي حماقته خير من الرجل الذي يخفي حكمته.
\par 16 لذلك اخجلوا حسب قولي، لأنه ليس من الجيد أن نحتفظ بكل الخجل، ولا يتم إقراره في كل شيء.
\par 17 اخجل من الزنا أمام الأب والأم، ومن الكذب أمام الأمير والقوي.
\par 18 من جريمة أمام القاضي والحاكم، ومن إثم أمام الجماعة والشعب، ومن ظلم أمام شريكك وصديقك.
\par 19 ومن السرقة فيما يتعلق بالمكان الذي أنت فيه، وفيما يتعلق بحق الله وعهده، ومن الاتكاء بمرفقك على اللحم، ومن الاستهزاء بالعطاء والأخذ.
\par 20 ومن الصمت أمام الذين يحيونك، والنظر إلى الزانية؛
\par 21 وأن تصرف وجهك عن قريبك، أو أن تأخذ نصيبا أو هبة، أو أن تنظر إلى امرأة آخر.
\par 22 أو أن يكون مشغولاً بخادمته، فلا يقترب من فراشها؛ أو أن يلقي كلمات تأنيب أمام الأصدقاء؛ وبعد أن تعطي، لا تأنيب؛
\par 23 أو تكرار ما سمعته، وكشف الأسرار.
\par 24 وبذلك تخجل حقًا وتجد نعمة أمام جميع الناس.

\chapter{42}

\par 1 لا تخجل من هذه الأمور، ولا تقبل أن يخطئ أحد بسببها.
\par 2 من شريعة العلي وعهده، ومن الحكم لتبرير الكافرين.
\par 3 من الحساب مع شركائك والمسافرين؛ أو من هبة ميراث الأصدقاء؛
\par 4 من دقة التوازن والأوزان؛ أو من الحصول على الكثير أو القليل؛
\par 5 ومن بيع التجار غير المبالي، ومن التأديب الكثير للأطفال، ومن جعل جنب العبد الشرير ينزف.
\par 6 الحفظ جيد حيث تكون المرأة الشريرة، والإغلاق حيث تكون الأيدي كثيرة.
\par 7 قم بتسليم كل الأشياء بالعدد والوزن، وسجل كل ما تعطيه أو تستقبله كتابيًا.
\par 8 لا تخجل من إخبار غير الحكماء والحمقى والشيخوخي المتطرف الذي يخاصم الشباب، هكذا تصبح متعلمًا حقًا ومقبولًا من جميع الأحياء.
\par 9 "الأب يسهر على ابنته حين لا يعلم أحد، والاهتمام بها ينزع النوم: حين تكون شابة لئلا تموت زهرة عمرها، وحين تتزوج لئلا تكره.
\par 10 في عذريتها لئلا تتنجس وتحمل في بيت أبيها، وتكون لها زوج لئلا تسيء التصرف، ومتى تزوجت لئلا تكون عاقراً.
\par 11 احذر الابنة عديمة الحياء لئلا تجعلك أضحوكة لأعدائك ومثلا في المدينة وعارا بين الشعب وتخجلك أمام الجمهور.
\par 12 لا تنظر إلى جمال كل إنسان، ولا تجلس بين النساء.
\par 13 فإنه من الثياب يخرج العث، ومن النساء الشر.
\par 14 إن وقاحة الرجل خير من امرأة مهذبة، امرأة تجلب العار والعار، كما أقول.
\par 15 "فأذكر الآن أعمال الرب وأخبر بالأشياء التي رأيتها. في أقوال الرب أعماله."
\par 16 الشمس التي تعطي النور تنظر إلى كل الأشياء، وعملها مملوء من مجد الرب.
\par 17 ولم يعط الرب القديسين سلطانا أن يخبروا بجميع عجائبه التي قررها الرب القدير لكي يثبت كل ما هو لمجده.
\par 18 "فإنه يبحث عن الأعماق والقلوب، ويتأمل في مكائدهم الماكرة. لأن الرب يعلم كل ما يمكن معرفته، وهو ينظر إلى علامات العالم."
\par 19 فهو يخبر بالأمور الماضية والمستقبلية، ويكشف خطوات الأمور الخفية.
\par 20 لا يفلت منه فكر، ولا تخفى عليه كلمة.
\par 21 وقد زين أعمال حكمته الفاضلة، وهو من الأزل إلى الأبد، لا يُزاد عليه شيء، ولا يُنقص منه شيء، ولا يحتاج إلى مشير.
\par 22 يا له من أعمال شهية! حتى أن الإنسان يرى حتى شرارة.
\par 23 كل هذه الأشياء تعيش وتبقى إلى الأبد لجميع الاستخدامات، وهي كلها مطيعة.
\par 24 كل الأشياء مزدوجة بعضها ضد بعض، ولم يخلق شيئًا غير كامل.
\par 25 شيء واحد يثبت الخير أو آخر، ومن يمتلئ من النظر إلى مجده؟

\chapter{43}

\par 1 فخر الارتفاع، والسماء الصافية، وجمال السماء، مع عرضه المجيد؛
\par 2 الشمس عندما تظهر، تعلن عند شروقها عن أداة عجيبة، عمل العلي:
\par 3 في الظهيرة تجفف البلاد، ومن يستطيع أن يتحمل حرارتها الحارقة؟
\par 4 إن الرجل الذي ينفخ في فرن هو في أعمال الحرارة، ولكن الشمس تحرق الجبال ثلاثة أضعاف أكثر من ذلك، وتنفث أبخرة نارية، وترسل أشعة لامعة، وتطفئ العيون.
\par 5 عظيم هو الرب صانعها، والذي حسب أمره يسرع.
\par 6 وجعل القمر أيضًا في موسمه إعلانًا للأزمنة وعلامة للعالم.
\par 7 ومن القمر علامة الأعياد، نور يتناقص في كمالها.
\par 8 يُسمى الشهر باسمها، ويزداد تغيره بشكل عجيب، كونه أداة للجيوش العليا، ويتألق في سماء السماء؛
\par 9 جمال السماء، ومجد النجوم، وزينة تنير في أعالي الرب.
\par 10 بأمر القدوس يقفون في نظامهم ولا يفشلون في حراستهم.
\par 11 انظروا إلى قوس القزح، وسبحوا صانعه، فهو جميل جداً في سطوعه.
\par 12 إنها تحيط بالسماء في دائرة مجيدة، وقد ثنيتها يدا العلي.
\par 13 بأمره يجعل الثلج يتساقط في مكانه، ويرسل بسرعة بروق حكمه.
\par 14 ومن خلال هذا تنفتح الكنوز، وتطير السحب كالطيور.
\par 15 بقوته العظيمة يثبت السحاب ويكسر حبات البرد.
\par 16 عند رؤيته تهتز الجبال، وعند إرادته تهب الرياح الجنوبية.
\par 17 صوت الرعد يهز الأرض، والعاصفة الشمالية والزوبعة، وكما تطير الطيور يبدد الثلج، وسقوطه كبرق الجراد.
\par 18 تعجب العين من حسن بياضه، ويتعجب القلب من هطوله.
\par 19 ويسكب الصقيع مثل الملح على الأرض، فيتجمد، ويستقر على قمة أوتاد حادة.
\par 20 عندما تهب الرياح الشمالية الباردة، وتتجمد المياه إلى جليد، فإنها تبقى على كل تجمع للمياه، وتكسو المياه كدرع.
\par 21 يأكل الجبال ويحرق البرية ويأكل العشب كالنار.
\par 22 العلاج الحاضر للجميع هو الضباب الذي يأتي بسرعة، والندى الذي يأتي بعد الحرارة ينعش.
\par 23 بنصائحه يهدئ الأعماق ويزرع فيها الجزر.
\par 24 إن الذين يبحرون في البحر يخبرون عن خطره، وعندما نسمع ذلك بآذاننا نتعجب منه.
\par 25 ففيه أعمال غريبة وعجيبة، وأنواع مختلفة من الوحوش والحيتان المخلوقة.
\par 26 به تنتهي الأمور بالنجاح، وبكلمته يقوم كل شيء.
\par 27 قد نتحدث كثيرًا، ومع ذلك نفشل: لذلك، باختصار، هو كل شيء.
\par 28 فكيف نستطيع أن نمجّده وهو عظيم فوق جميع أعماله.
\par 29 إن الرب مخيف وعظيم جداً وقدرته عجيبة.
\par 30 عندما تمجدون الرب، ارفعوه بقدر ما تستطيعون، لأنه سوف يتعدى ذلك بكثير. وعندما ترفعونه، ضعوا كل قوتكم، ولا تكلوا، لأنكم لا تستطيعون أن تذهبوا إلى ما يكفي.
\par 31 من رآه فيخبرنا ومن يقدر أن يمجده كما هو؟
\par 32 ولا يزال هناك أشياء أعظم من هذه مخفية، لأننا لم نرَ إلا القليل من أعماله.
\par 33 لأن الرب صنع كل الأشياء وأعطى الأتقياء الحكمة.

\chapter{44}

\par 1 فلنمدح الآن الرجال المشهورين، وآباءنا الذين ولدونا.
\par 2 لقد صنع الرب بهم مجداً عظيماً بقوته العظيمة منذ البدء.
\par 3 مثل الذين حكموا في ممالكهم، رجال معروفون بقوتهم، يعطون المشورة بفهمهم، ويعلنون النبوات:
\par 4 "قادة الشعب بمشورتهم ومعرفتهم بالعلم يليقون بالشعب، وتعليماتهم حكيمة وبليغة.
\par 5 مثل اكتشاف الألحان الموسيقية، وتلاوتها في الكتابة:
\par 6 رجال أغنياء مجهزون بالقدرة، يعيشون بسلام في مساكنهم:
\par 7 كل هؤلاء كانوا مكرمين في أجيالهم، وكانوا فخر عصرهم.
\par 8 ومنهم من تركوا وراءهم اسمًا، لكي يُذكر مدحهم.
\par 9 وهناك من ليس لهم ذكر، وهلكوا كأنهم لم يكونوا، وصاروا كأنهم لم يولدوا، وأبناؤهم من بعدهم.
\par 10 ولكن هؤلاء كانوا رجالاً رحماء، ولم يُنسى برهم.
\par 11 ويكون نسلهم ميراثًا صالحًا إلى الأبد، وأولادهم داخل العهد.
\par 12 زرعهم ثابت، وأولادهم من أجلهم.
\par 13 إن نسلهم يبقى إلى الأبد، ومجدهم لا يمحى.
\par 14 لقد دُفنت أجسادهم بسلام، لكن أسمائهم تبقى إلى الأبد.
\par 15 ويخبر الشعب بحكمتهم، والجماعة تشيد بتسبيحهم.
\par 16 فأرضى الرب أخنوخ، فنُقِلَ، صار مثالاً للتوبة لجميع الأجيال.
\par 17 "ووجد نوح كاملاً وباراً، وفي وقت الغضب أُخذ عوضاً عن العالم، ولذلك تُرك بقية على الأرض حين جاء الطوفان."
\par 18 وقد عقد معه عهدا أبدياً، لكي لا يهلك كل جسد أيضاً بالطوفان.
\par 19 كان إبراهيم أبًا عظيمًا لشعب كثير: لم يكن مثله في المجد.
\par 20 الذي حفظ ناموس العلي وكان في عهد معه. أقام العهد في جسده. ولما امتحن وجد أمينا.
\par 21 لذلك أكد له بقسم أنه سيبارك الأمم في نسله، وأنه سيكثره كتراب الأرض، ويرفع نسله كالنجوم، ويجعلهم يرثون من البحر إلى البحر، ومن النهر إلى أقاصي الأرض.
\par 22 مع إسحق أقام بركة جميع الناس والعهد، وجعلها على رأس يعقوب، وأقره في بركته، وأعطاه ميراثًا، وقسم أنصبته بين الاثني عشر سبطًا.

\chapter{45}

\par 1 وأخرج منه رجلاً رحيماً، وجد نعمة في أعين كل بشر، موسى، محبوب الله والناس، الذي ذكره مبارك.
\par 2 فشبهه بالقديسين المجيديين، وعظمه، حتى خاف منه أعداؤه.
\par 3 فبكلامه أبطل العجائب، ومجده في عيون الملوك، وأعطاه وصية لشعبه، وأراه بعض مجده.
\par 4 وقد قدسه بأمانته ووداعته، واختاره من بين جميع البشر.
\par 5 فأسمعه صوته، وأتى به إلى السحابة المظلمة، وأعطاه أمام وجهه شريعة الحياة والمعرفة، لكي يعلم يعقوب عهوده، وإسرائيل أحكامه.
\par 6 ورفع هارون رجلاً قديسًا مثله، أخاه من سبط لاوي.
\par 7 وقطع معه عهدا أبديا وأعطاه الكهنوت بين الشعب وزينه بحلي جميلة وألبسه ثوب المجد.
\par 8 فألبسه مجداً كاملاً، وقوّاه بثياب فاخرة وسراويل وجبّة وأفود.
\par 9 وأحاط به بالرمان وأجراس من ذهب كثيرة، لكي يكون عند سيره صوت وضوضاء يسمع في الهيكل تذكاراً لأبناء شعبه.
\par 10 مع ثوب مقدس، مع ذهب، وحرير أزرق، وأرجوان، عمل التطريز، مع صدرة القضاء، ومع أوريم وتوميم.
\par 11 من قرمز ملتوي، صنعة صانع ماهر، بحجارة كريمة منقوشة كالخواتم، مرصعة بالذهب، صنعة صائغ، بكتابة منقوشة تذكاراً على عدد أسباط إسرائيل.
\par 12 ووضع على العمامة تاجاً من ذهب، مكتوب عليه القداسة، وزينة الشرف، والعمل الثمين، وشهوات العيون، حسن وجميل.
\par 13 لم يكن هناك مثل هذه الأشياء من قبله، ولم يلبسها أي غريب قط، بل كان يرتديها أبناؤه وأبناء أبنائه فقط إلى الأبد.
\par 14 وتؤكل ذبائحهم كاملة كل يوم مرتين متتاليتين.
\par 15 وقد قدسه موسى ومسحه بزيت مقدس، هذا ما أُعطي له بعهد أبدي ولنسله إلى الأبد، لكي يخدموه ويقوموا بوظيفة الكهنوت ويباركوا الشعب باسمه.
\par 16 فاختاره من بين جميع الناس الأحياء ليقدم ذبائح للرب، بخوراً ورائحة سرور، تذكاراً للمصالحة بين شعبه.
\par 17 وأعطاه وصاياه وسلطانه في أحكام الشريعة لكي يعلم يعقوب الشهادات ويعلم إسرائيل شرائعه.
\par 18 وتآمر عليه الغرباء وشتموه في البرية، رجال داثان وأبيرون وجماعة قورح، بسخط وغيظ.
\par 19 هذا ما رآه الرب، فسرّه، وفي غضبه الشديد أهلكهم. صنع بهم عجائب ليحرقهم بلهيب نار.
\par 20 وأما هارون فأكرمه وأعطاه ميراثا وقسم له باكورة الغلة وخصوصا أنه أعد الخبز بكثرة.
\par 21 لأنهم يأكلون من ذبائح الرب التي أعطاها له ولنسله.
\par 22 ولكن لم يكن له نصيب في أرض الشعب، ولا كان له قسم بين الشعب، لأن الرب هو نصيبه وميراثه.
\par 23 والثالث في المجد هو فينحاس بن العازار لأنه غار في مخافة الرب وقام بشجاعة قلب صالح حين رجع الشعب وأصلح إسرائيل.
\par 24 لذلك عقد معه عهد سلام، حتى يكون رئيساً للمقدس ولشعبه، ويكون له ولنسله كرامة الكهنوت إلى الأبد.
\par 25 حسب العهد الذي قطع مع داود بن يسى من سبط يهوذا أن ميراث الملك يكون لذريته فقط، وكذلك ميراث هارون يكون لنسله أيضاً.
\par 26 ليُعطِكَ اللهُ الحكمةَ في قلبكَ لتحكمَ على شعبِهِ بالعدلِ، حتى لا تُفنى خيراتُهم، بل يدومُ مجدُهم إلى الأبد.

\chapter{46}

\par 1 وكان يسوع ابن مريم شجاعًا في الحروب، وكان خليفة موسى في النبوات، الذي صار عظيمًا حسب اسمه لخلاص مختاري الله، والانتقام من الأعداء الذين قاموا ضدهم، حتى يضع إسرائيل في ميراثهم.
\par 2 ما أعظم المجد الذي ناله حين رفع يديه ومد سيفه على المدن!
\par 3 من وقف أمامه هكذا؟ لأن الرب نفسه جلب إليه أعداءه.
\par 4 ألم ترجع الشمس على يديه ولم يكن يوم كاليومين؟
\par 5 ودعا الرب العلي حين حاصره الأعداء من كل جانب، فاستجاب له الرب العظيم.
\par 6 وبهطول برد شديد القوة جعل الحرب تنزل على الأمم بعنف، وفي نزوله أهلك المقاومين، لكي تعرف الأمم كل قوتهم، لأنه حارب أمام الرب وتبع القدير.
\par 7 وفي أيام موسى أيضاً صنع عمل رحمة هو وكالب بن يفنة، إذ صمدا أمام الجماعة، ومنعا الشعب من الخطيئة، وساكتا الأشرار المتذمرين.
\par 8 ومن بين ستمائة ألف من الشعب المارّين، نجا اثنان لإدخالهم إلى الميراث، إلى الأرض التي تفيض لبنا وعسلا.
\par 9 وأعطى الرب قوة لكالب، فبقي معه إلى شيخوخته، حتى دخل مرتفعات الأرض، ونالها نسله ميراثاً.
\par 10 لكي يرى جميع بني إسرائيل أنه من الجيد اتباع الرب.
\par 11 وأما القضاة، كل واحد باسمه، الذين لم يزن قلبهم ولم يحيدوا عن الرب، فلتكن ذكراهم مباركة.
\par 12 لتخرج عظامهم من أماكنها، وليستمر اسم المكرمين على أبنائهم.
\par 13 صموئيل نبي الرب الحبيب لدى ربه أسس مملكة ومسح رؤساء على شعبه.
\par 14 فبحسب شريعة الرب حكم للجماعة، وأما الرب فقد نظر إلى يعقوب.
\par 15 فبإخلاصه وجد نبيًا حقيقيًا، وبكلامه عرف أنه أمين في الرؤية.
\par 16 ودعا الرب القدير حين حاصره أعداؤه من كل جانب، حين قدم الحمل الرضيع.
\par 17 فأرعد الرب من السماء وأسمع صوته بصوت عظيم.
\par 18 وأباد رؤساء صور وجميع رؤساء الفلسطينيين.
\par 19 وقبل نومه الطويل احتج أمام الرب ومسيحه قائلا: لم آخذ من ممتلكات أحد حتى حذاء واحدا، ولم يشتكي عليه أحد.
\par 20 وبعد موته تنبأ وأظهر للملك نهايته ورفع صوته من الأرض بالنبوة لمحو شر الشعب.

\chapter{47}

\par 1 وبعده قام ناثان ليتنبأ في أيام داود.
\par 2 وكما يُؤخذ الشحم من ذبيحة السلامة، كذلك يُختار داود من بين بني إسرائيل.
\par 3 لقد لعب مع الأسود كما يلعب مع الأطفال، ومع الدببة كما يلعب مع الحملان.
\par 4 ألم يقتل جباراً وهو صغير السن؟ ألم يرفع العار عن الشعب عندما رفع يده بالحجر في المقلاع وحطم فخر جليات؟
\par 5 فإنه دعا الرب العلي فأعطاه قوة في يده اليمنى ليقتل ذلك الجبار وينصب قرن شعبه.
\par 6 فأكرمه الشعب بعشرات الآلاف، وأثنوا عليه ببركات الرب، لأنه أعطاه إكليل المجد.
\par 7 لأنه حطم الأعداء من كل جانب، وأباد الفلسطينيين أعداءه، وكسر قرنهم إلى هذا اليوم.
\par 8 في كل أعماله كان يسبح القدوس العلي بكلمات المجد، وكان يرنم الأغاني بكل قلبه، وأحب صانعه.
\par 9 وأقام مغنين أمام المذبح لكي يغنوا بأصواتهم ألحاناً حلوة ويغنوا تسابيح كل يوم في أغانيهم.
\par 10 فزيّن أعيادهم، ورتب الأوقات إلى النهاية، لكي يسبحوا اسمه القدوس، ويرنم الهيكل من الصباح.
\par 11 رفع الرب خطاياه ورفع قرنه إلى الأبد وأعطاه عهد الملوك وعرش المجد في إسرائيل.
\par 12 وبعده قام ابن حكيم، ومن أجله سكن رحباً.
\par 13 وكان سليمان ملكاً في زمن السلم، وكان مكرماً، لأن الله أهدأ حوله لكي يبني بيتاً باسمه، ويهيئ مقدسه إلى الأبد.
\par 14 كم كنت حكيماً في شبابك، وكطوفان ممتلئ فهماً!
\par 15 لقد غطت روحك كل الأرض، وملأتها بأمثال مظلمة.
\par 16 لقد ذهب اسمك بعيدًا إلى الجزر، ومن أجل سلامك كنت محبوبًا.
\par 17 لقد تعجبت منك البلدان بسبب أغانيك وأمثالك وحكاياتك وتأويلاتك.
\par 18 وباسم الرب الإله الذي يدعى الرب إله إسرائيل جمعت الذهب كالقصدير وكثرت الفضة كالرصاص.
\par 19 لقد أخضعت حقويك للنساء، وبجسدك أُخضعت.
\par 20 لقد لوثت شرفك، ونجستَ نسلك، حتى جلبتَ الغضب على أولادك، وحزنتَ على حماقتك.
\par 21 فانقسمت المملكة، وخرجت من أفرايم مملكة متمردة.
\par 22 ولكن الرب لن يترك رحمته، ولا يهلك شيء من أعماله، ولا يبيد نسل مختاريه، ونسل من يحبه لا ينزعه. لذلك أعطى بقية ليعقوب، وأصلا منه لداود.
\par 23 وهكذا استراح سليمان مع آبائه، وترك من نسله رحبعام، جهل الشعب، الذي أضلّ الشعب بمشورته. وكان يربعام بن نباط أيضًا، الذي جعل إسرائيل تخطئ، وأرشد أفرايم إلى طريق الخطيئة.
\par 24 فكثرت خطاياهم جداً حتى طُردوا من الأرض.
\par 25 فإنهم بحثوا عن كل شر حتى جاءهم الانتقام.

\chapter{48}

\par 1 فقام إيليا النبي كالنار، وكلمته أضاءت كمصباح.
\par 2 فأحضر عليهم مجاعة شديدة، فبغيرته قلل عددهم.
\par 3 فبكلمة الرب أغلق السماء وأنزل نارا ثلاث مرات.
\par 4 يا إيليا، ما أعظم عجائبك! ومن ذا الذي يفتخر مثلك؟
\par 5 الذي أقام ميتاً من الموت، ونفسه من موضع الأموات، بكلمة العلي.
\par 6 الذي أخرج الملوك إلى الهلاك والشرفاء من فراشهم.
\par 7 من سمع توبيخ الرب في سيناء وفي حوريب حكم الانتقام؟
\par 8 الذي مسح الملوك للانتقام، والأنبياء للخلافة بعده:
\par 9 الذي أُخِذَ في عاصفة من نار، وفي مركبة من خيل نارية.
\par 10 الذي أقيم للتوبيخ في أوقاته، لتهدئة غضب دينونة الرب قبل أن يشتعل غضباً، ولإرجاع قلب الأب إلى الابن، ولإرجاع أسباط يعقوب.
\par 11 طوبى للذين رأوك وناموا بمحبتك، لأننا سنحيا حياة.
\par 12 كان إيليا مغطى بالعاصفة، وكان إليشع ممتلئًا من روحه. لم يتحرك أثناء حياته أمام أي رئيس، ولم يستطع أحد أن يستعبده.
\par 13 لم تستطع كلمة أن تغلبه، وبعد موته تنبأ جسده.
\par 14 لقد صنع العجائب في حياته، وعند وفاته كانت أعماله عجيبة.
\par 15 مع كل هذا لم يتوب الشعب ولم يرجعوا عن خطاياهم حتى نهبوا وأخرجوا من أرضهم وتبددوا في كل الأرض. وبقي شعب قليل ورأس في بيت داود.
\par 16 ومنهم من فعل ما يرضي الله، ومنهم من كثرت خطاياه.
\par 17 وحصن حزقيا مدينته وأدخل الماء إلى وسطها وحفر الصخر بالحديد وحفر آباراً للمياه.
\par 18 وفي أيامه صعد سنحاريب وأرسل ربشاقي ورفع يده على صهيون وتباهى كثيرا.
\par 19 ثم ارتجفت قلوبهم وأيديهم، وكانوا في ألم كالنساء في المخاض.
\par 20 فدعوا الرب الرحيم ومدوا إليه أيديهم، فللوقت سمع لهم القدوس من السماء وأنقذهم بخدمة إيليا.
\par 21 فضرب جيش الآشوريين، وملاكه أهلكهم.
\par 22 لأن حزقيا عمل الأمر الذي يرضي الرب، وتقوى في طرق داود أبيه، كما أوصاه أشعيا النبي، الذي كان عظيما وأمينًا في رؤياه.
\par 23 وفي عهده رجعت الشمس إلى الوراء، فأطال عمر الملك.
\par 24 فرأى بروح فاضلة ما سيحدث في النهاية، وعزى الذين حزنوا في صهيون.
\par 25 فأظهر ما يجب أن يحدث إلى الأبد، والأشياء السرية التي سوف تأتي إلى الأبد.

\chapter{49}

\par 1 إن ذكر يوشيا يشبه تركيب العطر الذي يصنعه الصيدلاني: فهو حلو كالعسل في كل أفواه، وكموسيقى في وليمة خمر.
\par 2 وسلك مستقيماً في هداية الشعب، وأزال رجاسات الإثم.
\par 3 ووجه قلبه نحو الرب، وفي زمن الأشرار أسس عبادة الله.
\par 4 وكان الجميع فاسدين ما عدا داود وحزقيا ويوشيا، لأنهم تركوا شريعة العلي، حتى ملوك يهوذا فشلوا.
\par 5 لذلك أعطى قوتهم لآخرين، ومجدهم لأمة غريبة.
\par 6 فأحرقوا المدينة المقدسة المختارة، وأخربوا شوارعها، حسب نبوة إرميا.
\par 7 "فإنهم أساؤوا إليه وهو مع ذلك نبي مقدس في بطن أمه لكي يستأصل ويذل ويهلك ولكي يبني ويغرس أيضاً."
\par 8 وكان حزقيال هو الذي رأى الرؤيا المجيدة التي أُظهِرت له على مركبة الكروبيم.
\par 9 لأنه ذكر الأعداء تحت صورة المطر، وأرشد الذين ذهبوا إلى اليمين.
\par 10 ومن الأنبياء الاثني عشر، فليبارك ذكرهم، ولتزهر عظامهم من مكانها، لأنهم عزوا يعقوب، وأنقذوه بالرجاء الأكيد.
\par 11 كيف نعظم زربابل وهو كخاتم في اليد اليمنى.
\par 12 وكان يسوع بن يهوذا الذي في أيامهم بنى البيت وأقام هيكلاً مقدساً للرب، معداً للمجد الأبدي.
\par 13 ومن بين المختارين نيمياس، الذي له اسم عظيم، الذي أقام لنا الأسوار الساقطة، وأقام الأبواب والمزاليج، وأقام خرابنا أيضًا.
\par 14 ولكن على الأرض لم يُخلق إنسان مثل حنوك، لأنه أُخذ من الأرض.
\par 15 ولم يولد شاب مثل يوسف، حاكماً على إخوته، وسنداً للشعب، وعظامه تعتبر من قبل الرب.
\par 16 وكان سام وشيث يتمتعان بشرف عظيم بين الرجال، وكان آدم أيضًا فوق كل كائن حي في الخليقة.

\chapter{50}

\par 1 سمعان الكاهن الأعظم ابن أونياس الذي جدد البيت في حياته، وفي أيامه حصن الهيكل.
\par 2 وبه بني من الأساس السور المزدوج المرتفع، الحصن العالي حول الهيكل.
\par 3 في أيامه كان خزان الماء، الذي كان محاطًا بالبحر، مغطى بألواح من النحاس.
\par 4 لقد اهتم بالهيكل حتى لا يسقط، وحصن المدينة ضد الحصار:
\par 5 كيف تم تكريمه في وسط الشعب عند خروجه من المقدس!
\par 6 وكان كنجم الصبح في وسط السحاب، وكالقمر في ليلة البدر.
\par 7 كالشمس التي تشرق على هيكل العلي، وقوس قزح يضيء في السحب المضيئة:
\par 8 وكزهر الورد في ربيع السنة، وكزنابق على مجاري المياه، وكأغصان شجرة اللبان في فصل الصيف:
\par 9 كالنار والبخور في المجمرة، وكإناء من ذهب مطروق مرصع بكل أنواع الأحجار الكريمة.
\par 10 وكشجرة زيتونة جميلة تثمر ثمرا، وكشجرة سرو ترتفع إلى السحاب.
\par 11 ولما لبس ثوب الكرامة واكتسى بكمال المجد، وصعد إلى المذبح المقدس، جعل ثوب القداسة مكرما.
\par 12 ولما أخذ الأنصبة من أيدي الكهنة، وقف هو عند موقد المذبح، محيطاً به كالأرز الصغير في لبنان، وكالنخيل أحاطت به من كل جانب.
\par 13 فكان جميع بني هارون في مجدهم، وتقدمات الرب في أيديهم أمام كل جماعة إسرائيل.
\par 14 وأكمل الخدمة على المذبح لتزيين ذبيحة العلي القدير.
\par 15 ومد يده إلى الكأس وسكب من دم العنب، وسكب عند أسفل المذبح رائحة طيبة للملك الأعظم من الجميع.
\par 16 ثم هتف بنو هارون ونفخوا في الأبواق الفضية وأحدثوا صوتا عظيما مسموعاً تذكاراً أمام العلي.
\par 17 فأسرع كل الشعب معًا وسقطوا على وجوههم إلى الأرض ليسجدوا لربهم الله العلي القدير.
\par 18 كما غنى المطربون أيضًا التراتيل بأصواتهم، ومع تنوع كبير في الأصوات تم تقديم لحن عذب.
\par 19 وكان الشعب يتضرعون إلى الرب العلي بالصلاة أمام الرحيم حتى انتهى احتفال الرب وأتموا خدمته.
\par 20 ثم نزل ورفع يديه على كل جماعة بني إسرائيل ليبارك الرب بشفتيه ويهلل باسمه.
\par 21 ثم سجدوا ثانيةً للسجود، لكي ينالوا البركة من العلي.
\par 22 والآن باركوا إله الجميع، الذي يصنع العجائب في كل مكان، والذي يرفع أيامنا من الرحم، ويعاملنا حسب رحمته.
\par 23 ويمنحنا فرح القلب، وأن يكون السلام في أيامنا في إسرائيل إلى الأبد.
\par 24 لكي يثبت رحمته علينا، ويخلصنا في حينه!
\par 25 هناك نوعان من الأمم يكرههما قلبي، والثالث ليس أمة:
\par 26 الجالسون على جبل السامرة والساكنون بين الفلسطينيين والشعب الأحمق الساكن في شكيم.
\par 27 "كتب يسوع بن سيراخ من أورشليم في هذا الكتاب تأديب الفهم والمعرفة، الذي سكب الحكمة من قلبه."
\par 28 طوبى لمن يتدرب على هذه الأمور، ومن يضعها في قلبه يصير حكيماً.
\par 29 لأنه إن فعلها، يكون قويًا على كل شيء، لأن نور الرب يهتدي، وهو الذي يُعطي الحكمة للأتقياء. فليكن اسم الرب مباركًا إلى الأبد. آمين، آمين.

\chapter{51}

\par صلاة يسوع بن سيراخ.

\par 1 أحمدك أيها الرب الملك وأسبحك يا الله مخلصي وأسبح اسمك.
\par 2 لأنك أنت ناصري ومعينتي، وقد حفظت جسدي من الهلاك ومن فخ اللسان الكاذب ومن الشفاه المزيفة، وكنت معيناً لي على مضايقي.
\par 3 وأنقذتني حسب كثرة رحمتك وعظمة اسمك من أسنان الذين يريدون أن يلتهموني ومن أيدي الذين طلبوا نفسي ومن الشدة الكثيرة التي أصابتني.
\par 4 من اختناق النار من كل جانب، ومن وسط النار التي لم أشعلها؛
\par 5 من أعماق بطن الجحيم، ومن اللسان النجس، ومن الكلام الكاذب.
\par 6 بسبب اتهام للملك من لسان ظالم اقتربت روحي من الموت، واقتربت حياتي من الجحيم في الأسفل.
\par 7 لقد أحاطوا بي من كل جانب، ولم يكن هناك رجل ليساعدني. كنت أبحث عن عون الرجال، ولكن لم يكن هناك أحد.
\par 8 ثم فكرت في رحمتك يا رب، وفي أعمالك القديمة، كيف تنقذ الذين ينتظرونك، وتنقذهم من أيدي الأعداء.
\par 9 ثم رفعت تضرعي من الأرض وصليت من أجل النجاة من الموت.
\par 10 دعوت الرب أبا ربي أن لا يتركني في أيام ضيقي وفي زمن المتكبرين حين لا معين.
\par 11 "سأحمد اسمك دائما وأرنم بالحمد، ولذلك سمعت صلاتي."
\par 12 لأنك خلصتني من الهلاك، وأنقذتني من الزمان الرديء. لذلك أحمدك وأسبحك وأبارك اسمك يا رب.
\par 13 عندما كنت صغيراً، أو عندما سافرت إلى الخارج، كنت أرغب في الحكمة علانية في صلاتي.
\par 14 لقد صليت من أجلها أمام الهيكل، وسوف أبحث عنها حتى النهاية.
\par 15 من الزهرة إلى العنب الناضج سر قلبي بها، خطت رجلي في الطريق المستقيم، منذ شبابي طلبتُها.
\par 16 انحنيت لأذني قليلاً، واستقبلتها، وحصلت على الكثير من التعلم.
\par 17 لقد استفدت منه، لذلك سأعطي المجد للذي أعطاني الحكمة.
\par 18 لأني عزمت أن أفعل بعدها، واتبعت الخير بإخلاص، لذلك لن أخزى.
\par 19 صارعتها نفسي، وفي أعمالي كنت دقيقا. مددت يدي إلى السماء من فوق، وبكيت على جهلي بها.
\par 20 وجهت روحي إليها، فوجدتُها نقية. لقد كان قلبي مرتبطًا بها منذ البدء، لذلك لن أتخلى عنها.
\par 21 لقد اضطرب قلبي في البحث عنها، لذلك حصلت على ممتلكات جيدة.
\par 22 أعطاني الرب لساناً لأُجْرِيَهِ، فسأُسَبِّحُهُ بِهِ.
\par 23 اقتربوا مني أيها غير المتعلمين، واسكنوا في بيت المتعلمين.
\par 24 لماذا أنتم بطيئون؟ وماذا تقولون في هذا، وقد عطشت نفوسكم جداً؟
\par 25 فتحت فمي وقلت اشتروها لأنفسكم بلا ثمن.
\par 26 ضع رقبتك تحت النير، ودع روحك تتلقى التعليم: فهي صعبة المنال.
\par 27 انظروا بأعينكم كيف أني لا أتعب إلا قليلا، وقد حصلت على راحة كثيرة.
\par 28 احصل على التعلم بمبلغ كبير من المال، واحصل على الكثير من الذهب منها.
\par 29 لتفرح نفسك برحمته، ولا تخجل من مدحه.
\par 30 اعمل عملك في وقته، وفي وقته سيعطيك مكافأتك.

\end{document}