\begin{document}

\title{أناشيد سليمان}

\chapter{1}

\par 1 الرب على رأسي كالتاج، ولا أستطيع أن أكون بدونه.
\par 2 لقد نسجوا لي إكليلًا من الحق، فجعل أغصانك تُنبت فيّ
\par 3 لأنه ليس كإكليل ذابل لا يُزهر، بل أنت تسكن على رأسي، وقد أزهرت على رأسي
\par 4 ثمارك ناضجة وكاملة، وهي مليئة بخلاصك



\chapter{2}

\par \textit{(لم يتم تحديد أي جزء من هذه القصيدة على الإطلاق.)}

\chapter{3}

\par \textit{اختفت الكلمات الأولى من هذه القصيدة.}

\par 1 . . . أرتدي:
\par 2 وأعضاؤه معه، وعليها أقف، وهو يحبني.
\par 3 لأني لم أكن أعرف أن أحب الرب لو لم يحبني.
\par 4 فمن يستطيع أن يميز المحبة إلا المحبوب؟
\par 5 أحب الحبيب، وروحي تحبه:
\par 6 وحيث يكون راحته فهناك أكون أيضاً.
\par 7 ولن أكون غريبًا، لأنه لا ضغينة عند الرب العلي الرحيم
\par 8 لقد اتحدتُ مع "أنا أركض"، لأن العاشق قد وجد الحبيب،
\par 9 ولأني أحبه، الذي هو الابن، سأصبح ابنًا؛
\par 10 لأن من ينضم إلى الخالد، سيصبح هو نفسه خالدًا؛
\par 11 ومن سرَّه الحيُّ أصبح حيًّا.
\par 12 هذا هو روح الرب الذي لا يكذب، الذي يعلّم بني البشر أن يعرفوا طرقه.
\par 13 كن حكيمًا وفاهمًا ويقظًا. هللويا.

\chapter{4}

هذه القصيدة مهمةٌ بسبب الإشارة التاريخية التي تبدأ بها. قد يشير هذا إلى إغلاق معبد ليونتوبوليس في مصر، والذي يُرجّح أن تاريخ كتابتها يعود إلى حوالي عام ٧٣ ميلاديًا.

\par 1 لا أحد يا إلهي يغير موضع قدسك.
\par 2 ولا يجوز له أن يبدله ويضعه في مكان آخر لأنه لا سلطان له عليه.
\par 3 لأن مقدسك صممته قبل أن تصنع أماكن أخرى.
\par 4 لا يجوز تغيير ما هو أقدم من نفسه بما هو أصغر منه.
\par 5 لقد أعطيت قلبك يا رب للمؤمنين بك، فلن تفشل أبدًا ولن تكون بلا ثمار.
\par 6 فإن ساعة واحدة من إيمانك هي أثمن من كل الأيام والسنين.
\par 7 فمن ذا الذي يلبس نعمتك ويصاب بأذى؟
\par 8 لأن ختمك معروف ومخلوقاتك تعرفه وجندك يمتلكونه ورؤساء الملائكة المختارون يرتدونه.
\par 9 لقد أعطيتنا شركتك، لم يكن الأمر أنك كنت في حاجة إلينا، بل أننا في حاجة إليك.
\par 10 قطّر نديك علينا وافتح ينابيعك الغنية التي تسكب علينا اللبن والعسل.
\par 11 لأنه ليس عندك توبة من أي شيء وعدت به.
\par 12 "وكشفت النهاية أمامك، لأنك ما أعطيت فقد أعطيت بسخاء."
\par 13 لكي لا تجذبهم وتأخذهم مرة أخرى.
\par 14 لأن كل شيء قد كُشِفَ أمامك كإله، ومُرتَّب منذ البدء أمامك: وأنت يا الله، صنعت كل شيء. هللويا

\chapter{5}

\par \textit{ظهرت هذه القصيدة بشكل غريب في خطاب لسالومي في عمل قديم آخر يُسمى بيستيس صوفيا.}

\par 1 أحمدك يا ​​رب لأني أحببتك.
\par 2 أيها العلي، لن تتركني، لأنك أنت رجائي.
\par 3 لقد تلقيت نعمتك مجانًا، وسأعيش بها:
\par 4 سيأتي مضطهديّ ولن يروني.
\par 5 ستسقط سحابة من الظلام على عيونهم، وسيُظلمهم جو من الظلام الدامس
\par 6 ولن يكون لديهم نور يبصرون به: ولن يتمكنوا من التمسك بي.
\par 7 "فلتكن مشورتهم ظلاماً كثيفاً، وليرجع ما خططته بمكر على رؤوسهم."
\par 8 لأنهم فكروا في مؤامرة ولم تنجح:
\par 9 لأن رجائي على الرب فلا أخاف، ولأن الرب هو خلاصي فلا أخاف.
\par 10 وهو كإكليل على رأسي فلا أتزعزع، حتى لو اهتز كل شيء، فأنا ثابت؛
\par 11 وإن هلك كل ما هو مرئي، فلن أموت، لأن الرب معي وأنا معه. هللويا

\chapter{6}

\par \textit{تكشف عالمية القرن الأول بطريقة مثيرة للاهتمام في الآية 10.}

\par 1 كما تتحرك اليد فوق القيثارة، والأوتار تتحدث.
\par 2 هكذا يتكلم روح الرب في أعضائي، وأنا أتكلم بمحبته.
\par 3 فإنه يدمر كل ما هو غريب وكل ما هو مرير.
\par 4 لأنه هكذا كان من البداية وهكذا سيكون إلى النهاية، أن لا يكون شيء عدوًا له، ولا يقف شيء أمامه.
\par 5 لقد زاد الرب من معرفته، وهو غيور على أن تُعرف هذه الأمور التي أعطيت لنا بنعمته.
\par 6 وأعطانا تسبيح اسمه، أرواحنا تسبح روحه القدوس.
\par 7 فخرج جدول وصار نهرًا كبيرًا وواسعًا.
\par 8 لأنه غمر كل شيء وهدمه وجلب الماء إلى الهيكل.
\par 9 ولم يستطع حاصرو بني البشر أن يكبحوها، ولا فنون الذين من عملهم كبح المياه؛
\par 10 لأنها انتشرت على وجه كل الأرض وملأ كل شيء، وكل العطاش على الأرض شربوا منها.
\par 11 وقد ارتوي العطش وأُروِي، لأنه من العلي أُعطي السقاء.
\par 12 طوبى إذًا لخدام ذلك المنبع الذين أُوكِلَ إليهم ماءه:
\par 13 لقد هدأوا الشفاه الجافة، وأثاروا الإرادة التي ضعفت؛
\par 14 والأرواح التي كادت أن تغادر، فقد استعادتها من الموت:
\par 15 والأطراف التي سقطت قاموا بتقويمها وأقاموها.
\par 16 أعطوا قوة لضعفهم ونورًا لأعينهم:
\par 17 لأن الجميع عرفوهم في الرب، وعاشوا على ماء الحياة إلى الأبد. هللويا.



\chapter{7}

\par \textit{مزمور رائع وبسيط ومبهج عن التجسد.}

\par 1 وكما أن دافع الغضب ضد الشر، كذلك دافع الفرح تجاه ما هو جميل، ويجلب ثماره بلا قيود:
\par 2 فرحي هو الرب واندفاعي نحوه: هذا طريقي ممتاز.
\par 3 لأن لي معينًا هو الرب.
\par 4 لقد جعلني أعرف نفسه، دون تذمر، ببساطته: لقد أذل لطفه عظمته
\par 5 لقد أصبح مثلي، حتى أتمكن من استقباله:
\par 6 لقد حُسِبَ مثلي لكي ألبسه.
\par 7 ولم أرتجف حين رأيته، لأنه كان منعمًا عليّ.
\par 8 لقد صار مثل طبيعتي لكي أتعلم منه، ومثل شكلي لكي لا أتراجع عنه.
\par 9 أبو المعرفة هو كلمة المعرفة:
\par 10 الذي خلق الحكمة هو أحكم من أعماله.
\par 11 والذي خلقني ولم أكن يعلم ماذا أفعل حين وُجدت:
\par 12 لذلك أشفق عليّ بنعمته الوفيرة، ومنحني أن أطلب منه وأن آخذ من ذبيحته
\par 13 لأنه هو الذي لا يفنى، ملء الدهور وأبوها
\par 14 أعطاه أن يُرى من قِبَل الذين هم له، لكي يتعرفوا على خالقهم، ولا يظنوا أنهم جاؤوا من ذواتهم
\par 15 لأن المعرفة جعلها سبيلاً، ووسّعها ومدّها، وأوصلها إلى كل كمال
\par 16 وجعل عليها آثار نوره، فسلكت فيها من البداية إلى النهاية
\par 17 لأنه به تم ذلك، وكان يستريح في الابن، ومن أجل خلاصه سيمسك بكل شيء؛
\par 18 ويُعرَف العلي في قديسيه، ليُبشِّرَ أصحاب الترانيم بمجيء الرب؛
\par 19 لكي يخرجوا للقائه، ويغنوا له بفرح وبقيثارة ذات نغمات كثيرة:
\par 20 سيأتي الرائون أمامه، وسيُرَوْنَ أمامه،
\par 21 فيحمدون الرب على رحمته لأنه قريب وناظر.
\par 22 وتُنزع الكراهية من الأرض، وتُغرق مع الغيرة:
\par 23 لأن الجهل قد هُدم، لأن معرفة الرب قد وصلت
\par 24 الذين يصنعون الأغاني سيغنون نعمة الرب العلي.
\par 25 "ويأتون بأغانيهم، ويكون قلبهم مثل النهار، وكجمال الرب الرائع، أغنيتهم ​​اللذيذة."
\par 26 ولا يكون هناك من يتنفس بغير علم، ولا من أخرس:
\par 27 لأنه أعطى خليقته فمًا، ليفتح صوت الفم نحوه، لتسبيحه
\par 28 اعترفوا بقوته، وأظهروا نعمته. هللويا.

\chapter{8}

لاحظ الانتقال المفاجئ من شخصية كاتب المزمور إلى شخصية الرب (الآية ١٠). هذا يُشبه أسلوب المزامير القانونية.

\par 1 افتحوا، افتحوا قلوبكم لفرح الرب:
\par 2 "ولتكثر محبتك من القلب إلى الشفاه،
\par 3 لكي نعطي ثمرًا للرب [ثمرًا] مقدسًا [ثمرًا] ونتكلم بيقظة في نوره.
\par 4 انهضوا وقفوا منتصبين، أيها الذين سقطوا يومًا ما:
\par 5 أخبروا أيها الصمتون أن أفواهكم قد انفتحت.
\par 6 إذن، أيها المحتقرون، ارتفعوا من الآن فصاعدًا، لأن بركم قد ارتفع
\par 7 لأن يمين الرب معك، وهو معينك.
\par 8 وقد أعد لكم السلام قبل أن تكون حربكم.
\par 9 اسمع كلمة الحق، واحصل على معرفة العلي.
\par 10 لم تعرف أجسادكم ما أقوله لكم، ولا قلوبكم ما أعرضه عليكم.
\par 11 احفظوا سري، أيها المحفوظون به:
\par 12 احفظوا إيماني، أيها المحفوظون به
\par 13 وافهموا علمي يا من تعرفونني بالحق.
\par 14 أحبوني بالحب أيها الذين تحبون:
\par 15 لأني لا أحول وجهي عن الذين لي.
\par 16 لأني أعرفهم، وقبل أن يكونوا عرفتهم، وعلى وجوههم ختمت.
\par 17 لقد صوّرت أعضاءهم: هيّأت لهم ثدييّ، لكي يشربوا حليبي المقدس ويعيشوا به
\par 18 استمتعت بها ولا أخجل منها:
\par 19 لأنها صنعتي وقوة أفكاري.
\par 20 فمن ذا الذي ينهض على صنع يدي، أو من هو الذي لا يخضع لها؟
\par 21 أنا أردتُ وصوَّرتُ العقل والقلب، وهما لي، وبيدي اليمنى أضعُ مختاريّ
\par 22 وبري يسير أمامهم، ولا يُنزع عنهم اسمي، لأنه معهم
\par 23 اطلبوا، فتزدادوا، واثبتوا في محبة الرب،
\par 24 "ومع ذلك، أيها الأحباء في الحبيب: أولئك الذين حُفظوا في الذي يحيا:
\par 25 والذين يخلصون في الذي خلص.
\par 26 "وتوجدون بلا فساد في كل العصور لاسم أبيكم. هللويا."

\chapter{9}

\par \textit{لن نعرف أبدًا على وجه اليقين ما إذا كانت الحروب المشار إليها هنا روحية أم حروبًا خارجية فعلية.}

\par 1 افتحوا آذانكم وسأتحدث إليكم. أعطوني أرواحكم لأعطيكم روحي أيضًا،
\par 2 كلمة الرب ومسراته، والفكر المقدس الذي ابتكره بشأن مسيحه
\par 3 لأن في مشيئة الرب خلاصكم، وفكره هو الحياة الأبدية، وعاقبتكم هي الخلود
\par 4 اغتنموا في الله الآب، واستقبلوا فكر العلي.
\par 5 كن قويا وافتدي بنعمته.
\par 6 لأني أعلن لكم السلام، يا قديسيه.
\par 7 لكي لا يسقط أحد من الذين يسمعون في الحرب، ولكي لا يهلك الذين عرفوه، ولكي لا يخزى الذين يأخذون.
\par 8 إكليلٌ خالدٌ إلى الأبد هو الحقيقة. طوبى لمن يضعونه على رؤوسهم.
\par 9 إنه حجر ثمين وقد حدثت حروب من أجل التاج.
\par 10 والبر أخذه وأعطاه لكم.
\par 11 إلبسوا الإكليل في العهد الحقيقي للرب.
\par 12 وسيُكتب في كتابه كل من انتصر.
\par 13 لأن كتابهم هو النصر الذي لك. وهي (النصر) تراك أمامها وتتمنى لك الخلاص. هللويا.

\chapter{10}

\par \textit{قصيدة قصيرة قوية يتحدث فيها المسيح نفسه.}

\par 1 وجه الرب فمي بكلمته، وفتح قلبي بنوره، وأسكن فيّ حياته الخالدة
\par 2 وأعطاني أن أتكلم بثمر سلامه:
\par 3 لتحويل نفوس الراغبين في المجيء إليه، ولأسر أسرى صالحين إلى الحرية.
\par 4 لقد تقوىتُ وقويتُ، وأخذتُ العالم أسيرًا؛
\par 5 فصارت لي تسبيحاً لله العلي ولله أبي.
\par 6 فجُمِعَت الأمم المتشتتة.
\par 7 "وكنت بلا نجاسة بمحبتي لهم، لأنهم اعترفوا بي في الأعالي، وانطبعت آثار النور على قلوبهم."
\par 8 وساروا في حياتي وخلصوا وأصبحوا شعبي إلى الأبد. هللويا

\chapter{11}

\par \textit{رسم جميل للجنة المستعادة ونعمة أولئك الذين عادوا إلى امتيازات آدم الساقط.}

\par 1 "فانشق قلبي وظهرت زهرته، ونبتت فيه النعمة، وأثمر للرب،
\par 2 لأن العلي التصق بقلبي بروحه القدوس، وفحص وجداني نحوه، وملأني من محبته.
\par 3 "وفتحه لي صار خلاصي، فركضت في طريقه بسلامه، في طريق الحق.
\par 4 من البداية وحتى النهاية اكتسبت معرفته:
\par 5 "ولقد أُقِيمتُ على صخرة الحق التي أقامني عليها."
\par 6 "ولمست شفتي مياه ناطقة من ينبوع الرب بكثرة.
\par 7 وشربت وسكرت من الماء الحي الذي لا يموت.
\par 8 ولم يكن سُكرى بلا معرفة، بل تركت الباطل وتوجهت إلى العلي إلهي،
\par 9 وأغنيتُ بفضله، وتركتُ الحماقةَ المنتشرةَ في الأرض، وخلعتُها وألقيتُها عني
\par 10 وجددني الرب بثيابه، وملكني بنوره، ومن فوق أعطاني راحة في عدم فساد
\par 11 وأصبحتُ كالأرض التي تُزهر وتفرح بثمارها:
\par 12 وكان الرب كالشمس المضيئة على وجه الأرض.
\par 13 فأضاء عيني، فاستقبل وجهي الندى، واستمتع أنفي برائحة الرب الطيبة.
\par 14 وحملني إلى جنته، حيث نعيم الرب؛
\par 15 وسجدتُ للربِّ لمجده، وقلتُ: مباركٌ يا ربُّ المغروسون في أرضك، والذين لهم مكانٌ في جنتك
\par 16 وينبتون من ثمار الأشجار. وقد تحولوا من الظلمات إلى النور
\par 17 هوذا جميع عبيدك أمناء، يعملون أعمالاً صالحة، ويبتعدون عن الشر إلى رضاك.
\par 18 وردوا عنهم مرارة الأشجار حين غُرست في أرضك
\par 19 وأصبح كل شيء بمثابة بقايا منك، وذكرى إلى الأبد لأعمالك الأمينة
\par 20 لأن في جنتك متسعًا واسعًا، ولا شيء فيها بلا فائدة؛
\par 21 لكن كل شيء مليء بالثمرة؛ المجد لك يا الله، نعيم الجنة إلى الأبد. هللويا

\chapter{12}

\par \textit{مستوى عالٍ بشكل استثنائي من الفكر الروحي.}

\par 1 لقد ملأني بكلام الحق حتى أتمكن من التحدث به.
\par 2 وكجَريانِ الماءِ يَجري الحقُّ من فمي، وتُظهِرُ شفتاي ثمرَه
\par 3 وقد زادني علمه، لأن فم الرب هو الكلمة الحقة، وباب نوره
\par 4 وقد أعطاها العلي لكلماته، التي هي مُترجمو جماله، ومُرددو تسبيحه، ومُقرّو مشورته، ومُبشرو فكره، ومُؤدّبو عباده
\par 5 لأن سرعة الكلمة لا تُوصف، ومثل تعبيرها كذلك سرعتها وقوتها؛
\par 6 ومساره لا يعرف حدودًا. لا يفشل أبدًا، بل يظل ثابتًا، ولا يعرف نزولًا ولا طريقًا إليه
\par 7 فكما أن عمله كذلك غايته: فهو نور وفجر الفكر؛
\par 8 وبه تتحدث العوالم بعضها إلى بعض؛ وفي الكلمة كان هناك من كان صامتًا؛
\par 9 ومنه انبثقت المحبة والوئام؛ وكانوا يتحدثون فيما بينهم بما لهم؛ وقد اخترقهم الكلمة؛
\par 10 وعرفوا صانعهم، لأنهم كانوا على وفاق، لأن فم العلي كلمهم، وجرى تفسيره من خلاله:
\par 11 لأن مسكن الكلمة هو الإنسان، وحقيقتها هي المحبة.
\par 12 طوبى للذين فهموا كل شيء وعرفوا الرب في حقه. هللويا.

\chapter{13}

\par \textit{قصيدة صغيرة غريبة.}

\par 1 هوذا الرب مرآتنا: افتح أعيننا وانظر إليها فيه: وتعلم طبيعة وجهك:
\par 2 وأخبروا بروحه، وامسحوا القذارة عن وجوهكم، وأحبوا قداسته والبسوا منها.
\par 3 وكونوا بلا دنس في كل الأوقات أمامه. هللويا.

\chapter{14}

\par \textit{هذه القصيدة جميلة الأسلوب مثل سفر المزامير القانوني.}

\par 1 كما أن عيني الابن نحو أبيه، هكذا عيني يا رب نحوك في كل حين.
\par 2 لأن عندك تعزيتي وبهجتي.
\par 3 لا ترد رحمتك عني يا رب، ولا تنزع رحمتك مني.
\par 4 مد لي يا رب يدك اليمنى في كل وقت، وكن لي قائدا إلى النهاية حسب مسرتك.
\par 5 "لتكن لي حسنة أمامك لأجل مجدك ولأجل اسمك.
\par 6 دعني أحفظ من الشر، ولتكن ودائعك يا رب معي، وثمار محبتك.
\par 7 علمني مزامير حقك، فأثمر فيك:
\par 8 وافتح لي قيثارة روحك القدوس، لكي أسبحك بكل نغماتها يا رب
\par 9 وككثرة رحمتك، كذلك تعطيني، وتُسرع في إعطائنا طلباتنا، وأنت قادر على كل احتياجاتنا. هللويا



\chapter{15}

\par \textit{واحدة من أجمل القصائد الغنائية في هذه المجموعة غير العادية.}

\par 1 كما أن الشمس فرحٌ لمن يطلبون فجرها، كذلك فرحي هو الرب
\par 2 لأنه شمسي، وقد رفعتني أشعته، وبددت نوره كل ظلمة عن وجهي
\par 3 فيه اكتسبتُ عيونًا ورأيتُ يومه المقدس:
\par 4 لقد أصبحت أذناي ملكي وسمعت حقيقته.
\par 5 لقد كانت فكرة المعرفة هاجسي، وقد سُررتُ بها
\par 6 لقد تركتُ طريق الضلال، وسلكتُ نحوه، ونلتُ منه الخلاص، دون تذمر
\par 7 وحسب فضله أعطاني وحسب جماله صنعني.
\par 8 لقد لبست عدم الفساد باسمه، وأزلت الفساد بنعمته.
\par 9 "قد هُدِم الموتُ أمامَ وجهي، وأُزيلَ الهاويةُ بكلمتي."
\par 10 وصعدت حياة خالدة في أرض الرب،
\par 11 وقد أُعلن ذلك للمؤمنين به، وأُعطي بلا حدود لجميع الذين يثقون به. هللويا.

\chapter{16}

\par \textit{جمال خلق الله.}

\par 1 كما أن عمل الفلاح هو المحراث، وعمل قائد السفينة هو توجيه السفينة:
\par 2 "كذلك عملي هو مزمور الرب، حرفتي وعملي في تسبيحه.
\par 3 لأن حمام محبته غذّى قلبي، وحتى إلى شفتيّ سكب ثماره.
\par 4 لأن حبي هو الرب ولذلك سأغني له:
\par 5 لأني قد قويت في تسبيحه، وأؤمن به.
\par 6 "أفتح فمي فينطق روحه فيّ بمجد الرب وبهائه، عمل يديه وعمل أصابعه.
\par 7 كثرة رحمته وقوة كلمته.
\par 8 لأن كلمة الرب تبحث عن كل الأشياء، سواء كانت غير مرئية أو تلك التي تكشف فكره.
\par 9 لأن العين ترى أعماله، والأذن تسمع أفكاره.
\par 10 فبسط الأرض وأستقرت المياه في البحر.
\par 11 فَقَاسَ السَّمَاوَاتِ وَثَبَّتَ الْكُتُبَ وَأَسْسَى الْخَلِيقَةَ وَأَقَامَهَا.
\par 12 واستراح من أعماله:
\par 13 وخلق الأشياء تجري في مجاريها، وتعمل أعمالها:
\par 14 وهم لا يعرفون أن يقفوا ولا يتكاسلوا، وجنوده السماويون خاضعون لكلمته
\par 15 كنز النور هو الشمس، وكنز الظلام هو الليل:
\par 16 فخلق الشمس للنهار لتكون مضيئة، وأما الليل فيظلم على وجه الأرض.
\par 17 وتناوبهما الواحد على الآخر يتحدث عن جمال الله:
\par 18 وليس شيء بدون الرب، لأنه كان قبل أن يكون كل شيء.
\par 19 وخُلقت العوالم بكلمته، وبفكر قلبه. المجد والكرامة لاسمه. هللويا

\chapter{17}

\par \textit{تغيير غريب في الشخصية، بالكاد يُدرك حتى العودة منه في الآية الأخيرة.}

\par 1 لقد توجني إلهي: تاجي حي:
\par 2 وقد تبررتُ في ربي، خلاصي الذي لا يفنى هو.
\par 3 لقد تحررت من الغرور، ولم أُدان:
\par 4 لقد قطعت يديها القيود الخانقة، وأخذت وجه وهيئة إنسان جديد، وسرت فيه وخلصت.
\par 5 وقادتني فكرة الحقيقة. وسرت وراءها ولم أتيه:
\par 6 وكل من رآني اندهش، واعتبروني شخصًا غريبًا
\par 7 والذي عرفني ورباني هو العلي في كل كماله. ومجدني بلطفه، ورفع أفكاري إلى علو حقه
\par 8 ومن هناك أعطاني طريق وصاياه، وفتحت الأبواب التي كانت مغلقة،
\par 9 وكسرت قضبان الحديد، لكن حديدي ذاب وذاب أمامي
\par 10 لم يبدُ لي شيء مغلقًا: لأنني كنت باب كل شيء.
\par 11 "وذهبت إلى جميع عبيدي لأحلهم، لكي لا أترك أحداً مربوطاً أو مقيداً.
\par 12 ونشرتُ معرفتي بلا تذمر، وكانت صلاتي في محبتي
\par 13 وزرعتُ ثماري في القلوب، وحولتها إلى نفسي، فنالت بركتي ​​وعاشت
\par 14 واجتمعوا إليّ وخلصوا، لأنهم كانوا لي كأعضائي، وكنتُ رأسهم. المجد لك رأسنا، أيها الرب المسيح. هللويا

\chapter{18}

\par \textit{الرجل الذي خاض تجربة روحية يحمل رسالة.}

\par 1 ارتفع قلبي في محبة العلي واتسع لكي أحمده من أجل اسمه.
\par 2 لقد تقوى أعضائي لكي لا تسقط من قوته.
\par 3 أُزيلت الأمراض من جسدي، وثبت أمام الرب بمشيئته، لأن ملكوته حق.
\par 4 يا رب، من أجل الذين هم ناقصون، لا تنزع كلمتك عني!
\par 5 ولا لأجل أعمالهم تمنعني عن كمالك!
\par 6 لا تدع النور يقهره الظلام، ولا تدع الحقيقة تهرب من الباطل.
\par 7 ستُعينني على النصر، خلاصنا هو يمينك، وستستقبل الرجال من كل حدب وصوب.
\par 8 وتحفظ كل من وقع في الشرور.
\par 9 أنت إلهي. ليس في فمك كذب ولا موت.
\par 10 لأن إرادتك هي الكمال، والباطل لا تعرفه.
\par 11 ولا يعرفك.
\par 12 ولا تعلم خطأ
\par 13 ولا يعرفك.
\par 14 وظهر الجهل كالأعمى، وكزبد البحر،
\par 15 وظنوا ذلك الشيء الباطل أنه شيء عظيم؛
\par 16 "وجاءوا على مثاله وفسدوا، وفهم الذين عرفوا وتأملوا."
\par 17 ولم يكونوا فاسدين في أفكارهم، لأن هؤلاء كانوا في فكر الرب
\par 18 وكانوا يستهزئون بالذين يسلكون في الضلال.
\par 19 وتكلموا بالحق من الروح الذي نفخه العلي فيهم، تسبيح وجمال عظيم لاسمه هللويا.

\chapter{19}

\par \textit{رائعة ولا تنسجم مع القصائد الغنائية الأخرى. الإشارة إلى ولادة عذراء غير مؤلمة جديرة بالملاحظة.}

\par 1 قُدِّم لي كوب من الحليب، وشربته بحلاوة سرور الرب
\par 2 الابن هو الكأس، والذي حُلِبَ هو الآب:
\par 3 وكان الروح القدس يحلبها، لأن ثدييه كانا ممتلئين، وكان من الضروري أن يفرز لبنه بكمية كافية.
\par 4 وفتح الروح القدس صدره ومزج الحليب من ثديي الآب، وأعطى الخليط للعالم دون علمهم
\par 5 والذين يأخذون ملئها هم الذين على يمين الله.
\par 6 فتح الروح رحم العذراء فحبلت وولدت، وأصبحت العذراء أمًا برحمة كثيرة.
\par 7 فتمخضت وولدت ابنًا من غير تعب.
\par 8 ولأنها لم تكن مستعدة بشكل كافٍ، ولم تطلب قابلة (لأنه هو الذي أوجدها) فقد ولدت، كما لو كانت رجلاً، بإرادتها؛
\par 9 وولدته علانيةً، واقتنته بكرامةٍ عظيمة،
\par 10 وأحبوه في قماطه، وحافظوا عليه بلطف، وأظهروه في العظمة. هللويا

\chapter{20}

\par \textit{مزيج من الأخلاق والتصوف؛ من القاعدة الذهبية وشجرة الحياة.}

\par 1 أنا كاهن الرب، وأخدمه خدمة الكهنوت، وأقدم له ذبيحة فكره.
\par 2 لأن فكره ليس كفكر العالم ولا كفكر الجسد، ولا مثل الذين يخدمون جسديين.
\par 3 ذبيحة الرب هي البر ونقاوة القلب والشفتين.
\par 4 "أعطوا أمامه حقيتكم بلا لوم، ولا يظلم قلبك قلباً ولا نفسك نفساً."
\par 5 لا تكتسب غريبًا بثمن فضتك، ولا تحاول أن تأكل قريبك،
\par 6 ولا تنزع عنه غطاء عورته.
\par 7 "ولكن البسوا نعمة الرب بلا حدود، وادخلوا فردوسه، واصنعوا لأنفسكم إكليلاً من شجرته،
\par 8 "ضعه على رأسك وابتهج، واتكئ على راحته، والمجد يسير أمامك،
\par 9 فتنال من لطفه ونعمته، وتزدهر بالحق في مديح قداسته. الحمد والكرامة لاسمه. هللويا.

\chapter{21}

\par \textit{شرح رائع لـ "معاطف الجلد" في الإصحاح الثالث من سفر التكوين.}

\par 1 رفعتُ ذراعيّ إلى العلي، إلى نعمة الرب، لأنه فكّ قيودي عني، ورفعني معيني إلى نعمته وخلاصه
\par 2 وخلعتُ الظلمةَ ولبستُ النورَ،
\par 3 وحصلت روحي على جسد خالي من الحزن والألم والآلام.
\par 4 وكان من المفيد لي بشكل متزايد التفكير في الرب، وشركته في عدم الفساد:
\par 5 وارتفعتُ في نوره، وخدمتُ أمامه،
\par 6 وأصبحتُ قريبًا منه، أسبحه وأعترف له.
\par 7 فاض قلبي ووُجد في فمي، ونهض على شفتي، وزاد بهجة الرب على وجهي، وكذلك تسبيحه. هللويا

\chapter{22}

\par \textit{مثل مزامير داود في ابتهاجها بالحرية.}

\par 1 الذي أنزلني من العلاء، أصعدني أيضًا من الأماكن السفلى.
\par 2 والذي يجمع الأشياء التي بيننا هو الذي طرحني أيضا.
\par 3 الذي شتت أعدائي كان موجودًا من القديم وأعدائي:
\par 4 الذي أعطاني السلطان على القيود حتى أحلها.
\par 5 الذي قلب بيدي التنين ذو الرؤوس السبعة وأقمتني على أصوله لأبيد نسله.
\par 6 كنت هناك وساعدتني، وفي كل مكان كان اسمك حصنًا لي.
\par 7 يمينك دمرت سمومه الشريرة، ويدك هيأت الطريق للذين يؤمنون بك.
\par 8 واختارتهم من القبور، وأفرزتهم من بين الأموات
\par 9 أخذتَ عظامًا ميتة وغَطَّيتَها بالأجساد.
\par 10 لقد كانوا بلا حراك، وأعطيتهم طاقة الحياة.
\par 11 كان طريقك بلا فساد، ووجهك كذلك؛ لقد جلبت عالمك إلى الفساد: لكي يتحلل كل شيء، ثم يتجدد،
\par 12 ويكون أساس كل شيء صخرتك، وعليه بنيت مملكتك، فكانت مسكنًا للقديسين. هللويا

\chapter{23}

\par \textit{يُعد الإشارة إلى الوثيقة المختومة التي أرسلها الله أحد الألغاز العظيمة في المجموعة.}

\par 1 الفرح للقديسين ومن يلبسه إلا هم وحدهم؟
\par 2 النعمة هي للمختارين، ومن ينالها إلا الذين يثقون بها من البدء؟
\par 3 المحبة هي للمختارين، ومن يلبسها إلا الذين امتلكوها من البدء؟
\par 4 "امشوا في معرفة العلي بلا ضجر، إلى فرحه وإلى كمال معرفته."
\par 5 "وفكره كان مثل رسالة، وإرادته نزلت من العلاء، وأُرسلت مثل سهم يُرمى بقوة من القوس.
\par 6 وهرعت أيادٍ كثيرة إلى الرسالة لتنتزعها وتأخذها وتقرأها:
\par 7 فخرج من بين أصابعهم، فارتعبوا منه ومن الخاتم الذي عليه.
\par 8 لأنه لم يكن مأذوناً لهم أن يفكوا ختمه، لأن السلطان الذي على الختم كان أعظم منهم.
\par 9 لكن أولئك الذين رأوها ذهبوا وراء الرسالة ليعرفوا أين ستقع، ومن يقرأها ومن يسمعها
\par 10 لكن عجلةً استقبلتها وتجاوزتها:
\par 11 وكان معه علامة الملكوت والحكومة:
\par 12 وكل ما حاول تحريك العجلة، جزّها وقطعها:
\par 13 فجمع جمهور الأعداء، وأقام الجسور على الأنهار، وعبرت، واقتلعت غابات كثيرة، وجعلت طريقا واسعا.
\par 14 كان الرأس ينزل إلى القدمين، لأن العجلة كانت تنزل إلى القدمين، وما كان علامة عليها
\par 15 كانت الرسالة رسالة أمر، لأنها شملت جميع المقاطعات؛
\par 16 وظهر على رأسه الرأس الذي كشف عنه حتى ابن الحق من الآب الأعظم،
\par 17 ورث وامتلك كل شيء. وضاعت أفكار كثيرين
\par 18 فسارع جميع المرتدين وهربوا. وانقرض الذين اضطهدوا وغضبوا
\par 19 وكانت الرسالة مجلدًا كبيرًا، كُتبت بالكامل بإصبع الله:
\par 20 وكان عليها اسم الآب والابن والروح القدس، ليحكم إلى أبد الآبدين. هللويا

\chapter{24}

\par \textit{يشير ذكر الحمامة إلى إنجيل مفقود توجد إشارات نادرة إليه في الكتابات القديمة.}

\par 1 رفرفت الحمامة فوق المسيح، لأنه كان رأسها، وغنّت فوقه، وسُمع صوتها:
\par 2 فخاف السكان واضطرب الغرباء:
\par 3 فأسقطت الطيور أجنحتها، ومات كل الزواحف في جحورها، وانفتحت الهاوية التي كانت مخفية، وصرخوا إلى الرب كالنساء المخاضات.
\par 4 ولم يُعطَ لهم طعام، لأنه لم يكن لهم.
\par 5 وختموا الهاوية بخاتم الرب، وهلكوا في الفكر من كان موجودًا منذ العصور القديمة.
\par 6 لأنهم كانوا فاسدين منذ البداية، وكانت نهاية فسادهم حياة
\par 7 وهلك كل من كان ناقصًا، إذ لم يكن ممكنًا إعطاؤهم كلمة فيبقون
\par 8 وحطم الرب تصورات كل من لم يكن معه الحق
\par 9 لأن الذين ارتفعوا في قلوبهم كانوا ناقصي الحكمة، ولذلك رُفضوا، لأن الحق لم يكن معهم.
\par 10 لأن الرب أظهر طريقه ونشر نعمته، والذين فهموا عرفوا قداسته. هللويا.

\chapter{25}

\par \textit{العودة مرة أخرى إلى التجربة الشخصية.}

\par 1 لقد نجوت من قيودي، وإليك يا إلهي هربت:
\par 2 لأنك أنت يمين خلاصي ومعينتي.
\par 3 لقد كبحت الذين قاموا عليّ،
\par 4 ولن أراه أيضًا، لأن وجهك كان معي، الذي خلصني بنعمتك.
\par 5 ولكني صرت محتقرا ومرفوضا في عيون كثيرين، وكنت في عيونهم كالرصاص،
\par 6 وكانت لي قوة منك ومعونتك.
\par 7 "وضعت لي سراجا عن يميني وعن يساري، ولن يكون فيّ شيء غير منير."
\par 8 ولبستُ غطاء روحك، ونزعتَ عني لباس جلدي
\par 9 لأن يمينك رفعتني وأزالت عني المرض:
\par 10 فأصبحت قويا في الحق ومقدسا ببرك، وخاف مني جميع أعدائي.
\par 11 وصرت مُعجَبًا باسم الرب، وتبررت بلطفه، وراحته إلى أبد الآبدين. هللويا

\chapter{26}

\par \textit{تسبيح رائع.}

\par 1 سكبت التسبيح للرب، لأني أنا له:
\par 2 وسأغني ترنيمته المقدسة، لأن قلبي معه.
\par 3 لأن قيثارته في يدي، وأغاني راحته لا تصمت.
\par 4 سأصرخ إليه من كل قلبي، سأسبحه وأرفعه بكل أعضائي
\par 5 فمن المشرق وحتى المغرب تسبيحه:
\par 6 ومن الجنوب وحتى الشمال اعتراف به:
\par 7 ومن رأس الجبال إلى أقصى غايتها كماله.
\par 8 من يستطيع أن يكتب مزامير الرب أو من يقرأها؟
\par 9 أو من يقدر أن يدرب نفسه للحياة فتخلص نفسه،
\par 10 ومن يستطيع أن يعتمد على العلي حتى يتكلم بفمه؟
\par 11 من يستطيع تفسير عجائب الرب؟
\par 12 لأن من يستطيع التفسير سيتحلل ويصبح ما يُفسَّر
\par 13 يكفي أن نعرف ونستريح: ففي الراحة يقف المغنون،
\par 14 مثل نهر له ينبوع وافر، ويتدفق لمساعدة أولئك الذين يطلبونه. هللويا.

\chapter{27}

\par \textit{يشكل جسم الإنسان صليبًا عندما يقف الرجل منتصبًا في الصلاة وذراعيه ممدودتان.}

\par 1 مددت يدي وقدست ربي:
\par 2 فإن بسط يدي هو آيته.
\par 3 وممتدي هو الشجرة المستقيمة [أو الصليب].



\chapter{28}

\par \textit{هذه القصيدة جوهرة موسيقية.}

\par 1 كأجنحة الحمام على فراخها، وفم فراخها نحو أفواهها
\par 2 كذلك أجنحة الروح على قلبي.
\par 3 فرح قلبي وابتهج كالطفل الذي يبتهج في بطن أمه.
\par 4 لقد آمنت فاستريحت، لأن الذي آمنت به أمين.
\par 5 لقد باركني بغنى ورأسي معه، ولن يفصلني السيف ولا السيف عنه.
\par 6 لأني مستعد قبل أن يأتي الهلاك، وقد تم تثبيتي على أجنحة الخلود.
\par 7 وأراني آيته: خرج وأعطاني لأشرب، ومنه حياة هي الروح الذي في داخلي، ولا يمكن أن تموت لأنها حية.
\par 8 والذين رأوني تعجبوا مني، لأني كنت مضطهداً، وكانوا يظنون أني ابتلعتُ، لأني كنتُ أبدو لهم كأني من الهالكين.
\par 9 "فصار ظلمي خلاصي، وكنت لهم عقاباً لأنه لم تكن فيّ غيرة."
\par 10 لأني فعلت الخير لكل إنسان كرهوني،
\par 11 وأحاطوا بي كالكلاب المسعورة، التي تهاجم أسيادها بجهل،
\par 12 لأن فكرهم فاسد وفهمهم منحرف.
\par 13 ولكني كنت أحمل الماء في يدي اليمنى، وكنت أتحمل مرارتهم بحلاوتي.
\par 14 ولم أهلك، لأني لم أكن أخاهم، ولم يكن ميلادي مثل ميلادهم
\par 15 وطلبوا موتي فلم يجدوه، لأني كنت أكبر سنًا من ذكراهم
\par 16 وعبثًا هاجموني أنا ومن جاءوا بعدي بلا جزاء:
\par 17 سعوا إلى هدم ذكرى من كان أمامهم.
\par 18 لأن فكر العلي لا يمكن توقعه، وقلبه يفوق كل حكمة. هللويا.

\chapter{29}

\par \textit{يذكرنا مرة أخرى بمزامير داود.}

\par 1 الرب رجائي، عليه لا أخزى.
\par 2 لأنه حسب حمده خلقني، وحسب جوده أعطاني
\par 3 وحسب رحمته رفعني، وحسب بهائه العظيم رفعني
\par 4 وأصعدني من أعماق الهاوية، ومن فم الموت انتشلني
\par 5 وأذللت أعدائي، وهو بررني بنعمته.
\par 6 لأني آمنت بمسيح الرب، فظهر لي أنه هو الرب.
\par 7 فأراه آيته، وأرشدني بنوره، وأعطاني قضيب قدرته
\par 8 لأُخضعَ أفكارَ الشعوب، وقوةَ ذوي القوة لأُذلِّلَهم
\par 9 ليُشنَّ الحربَ بكلمته، ويَظفرَ بالنصرِ بقدرته.
\par 10 "وهزم الرب عدوي بكلمته، فصار كالقش الذي تحمله الريح،
\par 11 وحمدت العلي لأنه رفعني عبده وابن أمته. هللويا.

\chapter{30}

\par \textit{دعوة للعطشانين.}

\par 1 املأوا أنفسكم مياهًا من ينبوع الرب الحي، لأنه مفتوح لكم
\par 2 وتعالوا أيها العطاش، وخذوا مشروبكم واستريحوا عند نبع الرب.
\par 3 فهو جميلٌ ونقيّ، يُريح النفس، وماؤه أطيبُ من العسل.
\par 4 ولا يقارن به عسل النحل.
\par 5 فإنه من شفتي الرب يخرج، ومن قلب الرب اسمه.
\par 6 وجاء ذلك بلا نهاية وبصورة غير مرئية: وحتى تم وضعه في الوسط لم يعرفوه:
\par 7 طوبى للذين شربوا منه ووجدوا فيه راحة. هللويا.

\chapter{31}

\par \textit{أغنية ربما عرفها ماركوس أوريليوس عندما قال: "كن كالصخرة التي تتكسر عليها الأمواج باستمرار."}

\par 1 ذابت الهاوية أمام الرب، وزالت الظلمة بظهوره.
\par 2 لقد ضل الضلال وهلك على يده، ولم يجد الجهل طريقا للمشي، فغمرته حقيقة الرب.
\par 3 ففتح فمه وتكلم بالنعمة والفرح، وتكلم بترنيمة تسبيح جديدة لاسمه.
\par 4 ورفع صوته إلى العلي، وقدم له الأبناء الذين معه
\par 5 فتبرر وجهه، لأن هكذا أعطاه أبوه القدوس
\par 6 اخرجوا أيها المتألمين، واستقبلوا الفرح، وامتلكوا نفوسكم بنعمته، واحصلوا على الحياة الخالدة
\par 7 وجعلوني مديونًا عند قيامي، أنا الذي كنت مديونًا، وقسموا غنيمتي، ولم يكن لهم شيء
\par 8 لكنني صبرت والتزمت الصمت، كما لو أنني لم أتأثر بهم
\par 9 لكنني وقفت ثابتًا كصخرة صلبة تضربها الأمواج وتصمد
\par 10 وتحملت مرارتهم من أجل التواضع:
\par 11 لكي أفدي شعبي وأرثه ولا أنقض وعدي للآباء الذين وعدتهم بخلاص نسلهم. هللويا.

\chapter{32}

\par \textit{الفرح والنور.}

\par 1 للمباركين فرح من قلوبهم، ونور من الساكن فيهم:
\par 2 وكلمات من الحق، الذي كان من تلقاء نفسه: لأنه معزز بقوة العلي المقدسة: وهو غير مضطرب إلى الأبد. هللويا

\chapter{33}

\par \textit{عذراء تقف وتنادي (الآية 5).}

\par 1 ثم ركضت النعمة أيضًا وتركت الفساد، ونزلت فيه لتبيده؛
\par 2 وأباد الهلاك من أمامه، وأباد كل نظامه.
\par 3 ووقف على قمة عالية وألقى صوته من أقصاء الأرض إلى أقصائها:
\par 4 واجتذب إليه جميع الذين أطاعوه، ولم يظهر فيه إنسان شرير.
\par 5 "ولكن كانت هناك عذراء كاملة تنادي وتكرز وتقول:
\par 6 يا بني البشر ارجعوا ويا بنات البشر تعالوا.
\par 7 "اتركوا طرق الفساد واقتربوا مني، فأدخل فيكم وأخرجكم من الهلاك،
\par 8 "ويُعَلِّمُكُمْ حُكَمَاءَ فِي سُبُلِ الْحَقِّ لِكَيْ لاَ تَهْلِكُوا وَلاَ تَهْلِكُوا."
\par 9 اسمعوني فتُخلَّصوا. فإني أُخبركم بنعمة الله، فبواسطتي تُخلَّصون وتُبارَكون.
\par 10 "أنا قاضيكم، والذين لبسوني لن يتضرروا، بل سوف يرثون العالم الجديد الذي لا فساد فيه."
\par 11 "مختاريّ يسلكون فيّ، وأُعرّف طرقي للذين يطلبونني، وأجعلهم يتوكلون على اسمي. هللويا."



\chapter{34}

\par \textit{الشعر الحقيقي - نقي وبسيط.}

\par 1 لا يوجد طريق صعب حيث يوجد قلب بسيط
\par 2 ولا يوجد جرح حيث تكون الأفكار مستقيمة:
\par 3 ولا توجد عاصفة في عمق الفكر المستنير:
\par 4 حيث يكون الإنسان محاطًا بالجمال من كل جانب، فلا يوجد شيء قابل للتقسيم.
\par 5 مثل ما هو أسفل هو ما هو أعلى، لأن كل شيء أعلى، أما ما هو أسفل فهو خيال من لا علم له.
\par 6 لقد أُعلنت النعمة لخلاصك. آمن وعِش وخلص. هللويا.

\chapter{35}

"لا يوجد طفل في المهد يكذب بهدوء أكثر مني: تعال قريبًا، أيها الأبدية."

\par 1 ندى الرب في هدوءٍ قطره عليّ:
\par 2 وأصعد سحابة السلام فوق رأسي، التي كانت تحرسني دائمًا.
\par 3 كان ذلك خلاصًا لي: لقد اهتز كل شيء وخافوا؛
\par 4 وخرج منهم دخان وقضاء، وكنت صامتًا حسب أمر الرب
\par 5 كان بالنسبة لي أكثر من مجرد مأوى، وأكثر من مجرد أساس.
\par 6 "وحملتني أمه كطفل، فأعطاني لبناً ندى الرب."
\par 7 وقد عظمتُ بفضله، واسترحتُ في كماله،
\par 8 وبسطت يدي عند رفع نفسي، فتبررت أمام العلي، وافتديت عنده. هللويا.

\chapter{36}

\par \textit{لم يتفق علماء اللاهوت قط على تفسير لهذه القصيدة المحيرة.}

\par 1 استرحت بروح الرب، ورفعني الروح إلى الأعالي.
\par 2 وأوقفني على قدمي في علو الرب أمام كماله ومجده وأنا أسبحه بتأليف أغانيه.
\par 3 "أخرجني الروح أمام وجه الرب، ورغم أنني كنت ابن الإنسان، فقد سُميت المستنير، ابن الله."
\par 4 بينما كنت أسبح بين الحمدين، وكنت عظيماً بين الأعزاء.
\par 5 لأنه بحسب عظمة العلي هكذا صنعني، وكجدته جددني، ومسحني من كماله.
\par 6 "فصرت واحداً من جيرانه، وانفتح فمي كسحابة الندى،
\par 7 "ففاض قلبي كأنه سيل جارف من البر،
\par 8 وكان وصولي إليه بسلام، وقد ثُبِّتُ بروح حكومته. هللويا.

\chapter{37}

\par \textit{قصيدة بدائية.}

\par 1 مددت يدي إلى ربي، ورفعت صوتي إلى العلي:
\par 2 "وتكلمت بشفتي قلبي فسمعني حين وصل صوتي إليه."
\par 3 وجاءني جوابه، وأعطاني ثمار أعمالي؛
\par 4 وأعطاني الراحة بنعمة الرب. هللويا.



\chapter{38}

\par \textit{وصف جميل لقوة الحقيقة.}

\par 1 صعدتُ إلى نور الحقيقة كما لو كنتُ في عربة:
\par 2 "والحقيقة أخذتني وقادتني، وحملتني عبر الحفر والوديان، ومن بين الصخور والأمواج حفظتني."
\par 3 فأصبحت لي مرفأ الخلاص، ووضعتني على أحضان الحياة الخالدة.
\par 4 لقد ذهب معي وجعلني أرتاح، ولم يسمح لي بالتيه، لأنه كان الحقيقة؛
\par 5 ولم أتعرض لأي خطر، لأنني مشيت معه؛
\par 6 ولم أخطئ في شيء لأني أطعت الحق.
\par 7 لأن الخطأ يهرب منه ولا يقابله، أما الحقيقة فتسير في الطريق الصحيح،
\par 8 كل ما لم أكن أعرفه، فقد أوضح لي كل سموم الخطأ، وآفات الموت التي يظنونها حلاوة:
\par 9 ورأيت مُهلك الهلاك، حين تزين العروس الفاسدة، والعريس المفسد الذي يُفسد.
\par 10 وسألتُ الحقَّ: «من هؤلاء؟» فقال لي: «هذا هو المُضلِّلُ والضَّلالُ:
\par 11 وهم سواء في الحبيب وفي عروسه، ويُضلّون ويُفسدون العالم كله
\par 12 ويدعون كثيرين إلى الوليمة،
\par 13 ويسقونهم من خمر سكرهم، ويسلبون منهم حكمتهم ومعرفتهم، فيجعلونهم بلا عقل؛
\par 14 ثم يتركونهم؛ ثم يتصرفون كالمجانين مفسدين: إذ يرون أنهم بلا قلب، ولا يسعون إليه!
\par 15 وأصبحتُ حكيمًا حتى لا أقع في أيدي المخادع؛ وهنأت نفسي لأن الحقيقة كانت معي،
\par 16 وأُسِّسَتْ وعِشتُ وفُدِيتُ،
\par 17 وعلى يد الرب أسستني لأنه هو أقمني.
\par 18 لأنه هو الذي غرس الأصل وسقاه وأصلحه وباركه، وثمره يكون إلى الأبد
\par 19 ضرب بعمق ونبت وانتشر، وكان ممتلئًا ومتضخمًا؛
\par 20 وتمجد الرب وحده في غرسه وفي فلاحته: برعايته وبركة شفتيه،
\par 21 بغرس يمينه الجميل، وباكتشاف غرسه، وبفكر عقله. هللويا

\chapter{39}

\par \textit{إحدى الإشارات القليلة إلى أحداث في الأناجيل - قصة ربنا وهو يمشي على بحر الجليل.}

\par 1 الأنهار العظيمة هي قدرة الرب:
\par 2 وهي تجرف من يحتقرونه، وتشوش سبلهم:
\par 3 ويجرفون مخاضاتهم، ويلتقطون جثثهم، ويدمرون حياتهم
\par 4 لأنها أسرع من البرق وأكثر إيقاعًا، والذين يعبرونها بإيمان لا يتزعزعون؛
\par 5 ومن يسلك عليها بلا عيب فلا يخاف.
\par 6 لأن العلامة فيها هي الرب، والعلامة هي طريق العابرين باسم الرب.
\par 7 فالبسوا اسم العلي واعرفوه، فتعبروا دون خطر، لأن الأنهار ستكون خاضعة لكم
\par 8 لقد ربط الرب بينهم بكلمته، وسار وعبرهم على قدميه
\par 9 وخطواته ثابتة على الماء، لا تتألم، فهي ثابتة كشجرة راسخة
\par 10 وارتفعت الأمواج من هنا ومن هناك، ولكن خطوات ربنا المسيح ثابتة لا تمحى ولا تتشوه.
\par 11 وقد وضع طريقا للذين عبروا بعده وللذين تمسكوا بطريق الإيمان به وعبدوا اسمه. هللويا.

\chapter{40}

\par \textit{أغنية مديح لا مثيل لها.}

\par 1 كما يقطر العسل من قرص النحل،
\par 2 واللبن يسيل من المرأة التي تحب أولادها؛
\par 3 كذلك رجائي فيك يا إلهي.
\par 4 عندما تتدفق مياه النافورة،
\par 5 فيفيض قلبي تسبيحًا للرب، وشفتاي تسبحانه، ولساني مزاميره.
\par 6 ويفرح وجهي بفرحه، وتبتهج روحي بمحبته، وتشرق روحي فيه
\par 7 والخشوع فيه يثق، والفداء فيه مضمون:
\par 8 وميراثه هو الحياة الخالدة، والذين يشاركون فيه هم بلا فساد. هللويا.

\chapter{41}

\par \textit{نكتشف أن الكاتب قد يكون غير يهودي (الآية 8).}

\par 1 سيُسبِّحه جميع أبناء الرب، وسيجمعون حقيقة إيمانه
\par 2 وسيُعرف أولاده لديه. لذلك سنُرَنِّم في محبته:
\par 3 نعيش في الرب بنعمته، وننال الحياة في مسيحه.
\par 4 لأنه يوم عظيم أشرق علينا وعجيب هو الذي أعطانا من مجده.
\par 5 فلنتحد جميعًا معًا باسم الرب، ولنكرمه في صلاحه،
\par 6 ولتشرق وجوهنا بنوره، ولتتأمل قلوبنا في محبته ليلاً ونهاراً.
\par 7 فلنفرح بفرح الرب.
\par 8 سيُدهش كل من يراني، فأنا من جنس آخر.
\par 9 لأن أبا الحق ذكرني، الذي اقتني من البدء
\par 10 لأن فضله أنجبني، وفكر قلبه:
\par 11 وكلمته معنا في كل طريقنا.
\par 12 المخلص الذي يُحيي أرواحنا ولا يرفضها
\par 13 الرجل الذي تواضع وارتفع بفضل بره،
\par 14 ظهر ابن العلي في كمال أبيه؛
\par 15 وأشرق النور من الكلمة الذي كان قبل الزمان فيه؛
\par 16 المسيح واحد حقًا؛ وكان معروفًا قبل تأسيس العالم،
\par 17 لكي يخلص النفوس إلى الأبد بحقيقة اسمه: ترنيمة جديدة تنبع من أولئك الذين يحبونه. هللويا

\chapter{42}

\par \textit{تُختتم أناشيد سليمان، ابن داود، بالآيات الرائعة التالية.}

\par 1 مددت يدي وتوجهت إلى ربي:
\par 2 فإن بسط يدي هو آيته.
\par 3 توسعي هو الشجرة الممتدة التي أقيمت على طريق البار.
\par 4 وأصبحتُ بلا قيمة عند الذين لم يتمسكوا بي، وسأكون مع الذين يحبونني.
\par 5 "كل مضطهديّ قد ماتوا، وهم الذين طلبوني، والذين كانوا يرجونني، لأني كنت حيّاً."
\par 6 وقمت وأنا معهم، وأتكلم بأفواههم.
\par 7 فإنهم احتقروا الذين اضطهدوهم.
\par 8 ورفعت فوقهم نير محبتي.
\par 9 مثل ذراع العريس على العروس،
\par 10 "هكذا كان نيري على الذين يعرفونني.
\par 11 وكما أن السرير الذي يكون مفروشاً في بيت العريس والعروس،
\par 12 هكذا محبتي للذين يؤمنون بي.
\par 13 ولم أُرفض رغم أنني كنت أعتبر كذلك.
\par 14 لم أهلك، مع أنهم خططوا لي.
\par 15 لقد رآني الهاوية فحزنت.
\par 16 لقد ألقى الموت بي، وكثيرين معي.
\par 17 لقد شعرت بالمرارة والغضب، وهبطت معه إلى أقصى عمقه:
\par 18 وترك قدميه ورأسه، لأنهما لم يستطيعا أن يتحملا وجهي
\par 19 وجعلتُ جماعةً من الأحياء بين أمواته، وتكلمتُ معهم بشفاهٍ حية:
\par 20 لأن كلمتي لن تبطل:
\par 21 فركض نحوي الذين ماتوا، وصرخوا قائلين: يا ابن الله، ارحمنا واصنع معنا كرحمتك،
\par 22 وأخرجنا من قيود الظلمة، وافتح لنا الباب الذي نخرج منه إليك
\par 23 فإننا نرى أن موتنا لم يمسسك.
\par 24 فلنفتدينا أيضًا معك، لأنك أنت فادينا.
\par 25 وسمعت صوتهم، وختمت اسمي على رؤوسهم.
\par 26 لأنهم رجال أحرار وهم لي. هللويا.


\end{document}