\begin{document}

\title{خروج}


\chapter{1}

\par 1 وَهَذِهِ اسْمَاءُ بَنِي اسْرَائِيلَ الَّذِينَ جَاءُوا الَى مِصْرَ. مَعَ يَعْقُوبَ جَاءَ كُلُّ انْسَانٍ وَبَيْتُهُ.
\par 2 رَاوبَيْنُ وَشَمْعُونُ وَلاوِي وَيَهُوذَا
\par 3 وَيَسَّاكَرُ وَزَبُولُونُ وَبِنْيَامِينُ
\par 4 وَدَانُ وَنَفْتَالِي وَجَادُ وَاشِيرُ.
\par 5 وَكَانَتْ جَمِيعُ نُفُوسِ الْخَارِجِينَ مِنْ صُلْبِ يَعْقُوبَ سَبْعِينَ نَفْسا. (وَلَكِنْ يُوسُفُ كَانَ فِي مِصْرَ).
\par 6 وَمَاتَ يُوسُفُ وَكُلُّ اخْوَتِهِ وَجَمِيعُ ذَلِكَ الْجِيلِ.
\par 7 وَامَّا بَنُو اسْرَائِيلَ فَاثْمَرُوا وَتَوَالَدُوا وَنَمُوا وَكَثُرُوا كَثِيرا جِدّا وَامْتَلاتِ الارْضُ مِنْهُمْ.
\par 8 ثُمَّ قَامَ مَلِكٌ جَدِيدٌ عَلَى مِصْرَ لَمْ يَكُنْ يَعْرِفُ يُوسُفَ.
\par 9 فَقَالَ لِشَعْبِهِ: «هُوَذَا بَنُو اسْرَائِيلَ شَعْبٌ اكْثَرُ وَاعْظَمُ مِنَّا.
\par 10 هَلُمَّ نَحْتَالُ لَهُمْ لِئَلا يَنْمُوا فَيَكُونَ اذَا حَدَثَتْ حَرْبٌ انَّهُمْ يَنْضَمُّونَ الَى اعْدَائِنَا وَيُحَارِبُونَنَا وَيَصْعَدُونَ مِنَ الارْضِ».
\par 11 فَجَعَلُوا عَلَيْهِمْ رُؤَسَاءَ تَسْخِيرٍ لِكَيْ يُذِلُّوهُمْ بِاثْقَالِهِمْ فَبَنُوا لِفِرْعَوْنَ مَدِينَتَيْ مَخَازِنَ: فِيثُومَ وَرَعَمْسِيسَ.
\par 12 وَلَكِنْ بِحَسْبِمَا اذَلُّوهُمْ هَكَذَا نَمُوا وَامْتَدُّوا. فَاخْتَشُوا مِنْ بَنِي اسْرَائِيلَ.
\par 13 فَاسْتَعْبَدَ الْمِصْرِيُّونَ بَنِي اسْرَائِيلَ بِعُنْفٍ
\par 14 وَمَرَّرُوا حَيَاتَهُمْ بِعُبُودِيَّةٍ قَاسِيَةٍ فِي الطِّينِ وَاللِّبْنِ وَفِي كُلِّ عَمَلٍ فِي الْحَقْلِ. كُلِّ عَمَلِهِمِ الَّذِي عَمِلُوهُ بِوَاسِطَتِهِمْ عُنْفا.
\par 15 وَكَلَّمَ مَلِكُ مِصْرَ قَابِلَتَيِ الْعِبْرَانِيَّاتِ اللَّتَيْنِ اسْمُ احْدَاهُمَا شِفْرَةُ وَاسْمُ الاخْرَى فُوعَةُ
\par 16 وَقَالَ: «حِينَمَا تُوَلِّدَانِ الْعِبْرَانِيَّاتِ وَتَنْظُرَانِهِنَّ عَلَى الْكَرَاسِيِّ - انْ كَانَ ابْنا فَاقْتُلاهُ وَانْ كَانَ بِنْتا فَتَحْيَا».
\par 17 وَلَكِنَّ الْقَابِلَتَيْنِ خَافَتَا اللهَ وَلَمْ تَفْعَلا كَمَا كَلَّمَهُمَا مَلِكُ مِصْرَ بَلِ اسْتَحْيَتَا الاوْلادَ.
\par 18 فَدَعَا مَلِكُ مِصْرَ الْقَابِلَتَيْنِ وَقَالَ لَهُمَا: «لِمَاذَا فَعَلْتُمَا هَذَا الامْرَ وَاسْتَحْيَيْتُمَا الاوْلادَ؟»
\par 19 فَقَالَتِ الْقَابِلَتَانِ لِفِرْعَوْنَ: «انَّ النِّسَاءَ الْعِبْرَانِيَّاتِ لَسْنَ كَالْمِصْرِيَّاتِ فَانَّهُنَّ قَوِيَّاتٌ يَلِدْنَ قَبْلَ انْ تَاتِيَهُنَّ الْقَابِلَةُ».
\par 20 فَاحْسَنَ اللهُ الَى الْقَابِلَتَيْنِ وَنَمَا الشَّعْبُ وَكَثُرَ جِدّا.
\par 21 وَكَانَ اذْ خَافَتِ الْقَابِلَتَانِ اللهَ انَّهُ صَنَعَ لَهُمَا بُيُوتا.
\par 22 ثُمَّ امَرَ فِرْعَوْنُ جَمِيعَ شَعْبِهِ قَائِلا: «كُلُّ ابْنٍ يُولَدُ تَطْرَحُونَهُ فِي النَّهْرِ لَكِنَّ كُلَّ بِنْتٍ تَسْتَحْيُونَهَا».

\chapter{2}

\par 1 وَذَهَبَ رَجُلٌ مِنْ بَيْتِ لاوِي وَاخَذَ بِنْتَ لاوِي
\par 2 فَحَبِلَتِ الْمَرْاةُ وَوَلَدَتِ ابْنا. وَلَمَّا رَاتْهُ انَّهُ حَسَنٌ خَبَّاتْهُ ثَلاثَةَ اشْهُرٍ.
\par 3 وَلَمَّا لَمْ يُمْكِنْهَا انْ تُخَبِّئَهُ بَعْدُ اخَذَتْ لَهُ سَفَطا مِنَ الْبَرْدِيِّ وَطَلَتْهُ بِالْحُمَرِ وَالزِّفْتِ وَوَضَعَتِ الْوَلَدَ فِيهِ وَوَضَعَتْهُ بَيْنَ الْحَلْفَاءِ عَلَى حَافَةِ النَّهْرِ.
\par 4 وَوَقَفَتْ اخْتُهُ مِنْ بَعِيدٍ لِتَعْرِفَ مَاذَا يُفْعَلُ بِهِ.
\par 5 فَنَزَلَتِ ابْنَةُ فِرْعَوْنَ الَى النَّهْرِ لِتَغْتَسِلَ وَكَانَتْ جَوَارِيهَا مَاشِيَاتٍ عَلَى جَانِبِ النَّهْرِ. فَرَاتِ السَّفَطَ بَيْنَ الْحَلْفَاءِ فَارْسَلَتْ امَتَهَا وَاخَذَتْهُ.
\par 6 وَلَمَّا فَتَحَتْهُ رَاتِ الْوَلَدَ وَاذَا هُوَ صَبِيٌّ يَبْكِي. فَرَقَّتْ لَهُ وَقَالَتْ: «هَذَا مِنْ اوْلادِ الْعِبْرَانِيِّينَ».
\par 7 فَقَالَتْ اخْتُهُ لابْنَةِ فِرْعَوْنَ: «هَلْ اذْهَبُ وَادْعُو لَكِ امْرَاةً مُرْضِعَةً مِنَ الْعِبْرَانِيَّاتِ لِتُرْضِعَ لَكِ الْوَلَدَ؟»
\par 8 فَقَالَتْ لَهَا ابْنَةُ فِرْعَوْنَ: «اذْهَبِي». فَذَهَبَتِ الْفَتَاةُ وَدَعَتْ امَّ الْوَلَدِ.
\par 9 فَقَالَتْ لَهَا ابْنَةُ فِرْعَوْنَ: «اذْهَبِي بِهَذَا الْوَلَدِ وَارْضِعِيهِ لِي وَانَا اعْطِي اجْرَتَكِ». فَاخَذَتِ الْمَرْاةُ الْوَلَدَ وَارْضَعَتْهُ.
\par 10 وَلَمَّا كَبِرَ الْوَلَدُ جَاءَتْ بِهِ الَى ابْنَةِ فِرْعَوْنَ فَصَارَ لَهَا ابْنا وَدَعَتِ اسْمَهُ «مُوسَى» وَقَالَتْ: «انِّي انْتَشَلْتُهُ مِنَ الْمَاءِ».
\par 11 وَحَدَثَ فِي تِلْكَ الايَّامِ لَمَّا كَبِرَ مُوسَى انَّهُ خَرَجَ الَى اخْوَتِهِ لِيَنْظُرَ فِي اثْقَالِهِمْ فَرَاى رَجُلا مِصْرِيّا يَضْرِبُ رَجُلا عِبْرَانِيّا مِنْ اخْوَتِهِ
\par 12 فَالْتَفَتَ الَى هُنَا وَهُنَاكَ وَرَاى انْ لَيْسَ احَدٌ فَقَتَلَ الْمِصْرِيَّ وَطَمَرَهُ فِي الرَّمْلِ.
\par 13 ثُمَّ خَرَجَ فِي الْيَوْمِ الثَّانِي وَاذَا رَجُلانِ عِبْرَانِيَّانِ يَتَخَاصَمَانِ فَقَالَ لِلْمُذْنِبِ: «لِمَاذَا تَضْرِبُ صَاحِبَكَ؟»
\par 14 فَقَالَ: «مَنْ جَعَلَكَ رَئِيسا وَقَاضِيا عَلَيْنَا؟ امُفْتَكِرٌ انْتَ بِقَتْلِي كَمَا قَتَلْتَ الْمِصْرِيَّ؟» فَخَافَ مُوسَى وَقَالَ: «حَقّا قَدْ عُرِفَ الامْرُ!»
\par 15 فَسَمِعَ فِرْعَوْنُ هَذَا الامْرَ فَطَلَبَ انْ يَقْتُلَ مُوسَى. فَهَرَبَ مُوسَى مِنْ وَجْهِ فِرْعَوْنَ وَسَكَنَ فِي ارْضِ مِدْيَانَ وَجَلَسَ عِنْدَ الْبِئْرِ.
\par 16 وَكَانَ لِكَاهِنِ مِدْيَانَ سَبْعُ بَنَاتٍ فَاتَيْنَ وَاسْتَقَيْنَ وَمَلَانَ الاجْرَانَ لِيَسْقِينَ غَنَمَ ابِيهِنَّ.
\par 17 فَاتَى الرُّعَاةُ وَطَرَدُوهُنَّ. فَنَهَضَ مُوسَى وَانْجَدَهُنَّ وَسَقَى غَنَمَهُنَّ.
\par 18 فَلَمَّا اتَيْنَ الَى رَعُوئِيلَ ابِيهِنَّ قَالَ: «مَا بَالُكُنَّ اسْرَعْتُنَّ فِي الْمَجِيءِ الْيَوْمَ؟»
\par 19 فَقُلْنَ: «رَجُلٌ مِصْرِيٌّ انْقَذَنَا مِنْ ايْدِي الرُّعَاةِ وَانَّهُ اسْتَقَى لَنَا ايْضا وَسَقَى الْغَنَمَ».
\par 20 فَقَالَ لِبَنَاتِهِ: «وَايْنَ هُوَ؟ لِمَاذَا تَرَكْتُنَّ الرَّجُلَ؟ ادْعُونَهُ لِيَاكُلَ طَعَاما».
\par 21 فَارْتَضَى مُوسَى انْ يَسْكُنَ مَعَ الرَّجُلِ فَاعْطَى مُوسَى صَفُّورَةَ ابْنَتَهُ.
\par 22 فَوَلَدَتِ ابْنا فَدَعَا اسْمَهُ جَرْشُومَ لانَّهُ قَالَ: «كُنْتُ نَزِيلا فِي ارْضٍ غَرِيبَةٍ».
\par 23 وَحَدَثَ فِي تِلْكَ الايَّامِ الْكَثِيرَةِ انَّ مَلِكَ مِصْرَ مَاتَ. وَتَنَهَّدَ بَنُو اسْرَائِيلَ مِنَ الْعُبُودِيَّةِ وَصَرَخُوا فَصَعِدَ صُرَاخُهُمْ الَى اللهِ مِنْ اجْلِ الْعُبُودِيَّةِ.
\par 24 فَسَمِعَ اللهُ انِينَهُمْ فَتَذَكَّرَ اللهُ مِيثَاقَهُ مَعَ ابْرَاهِيمَ وَاسْحَاقَ وَيَعْقُوبَ.
\par 25 وَنَظَرَ اللهُ بَنِي اسْرَائِيلَ وَعَلِمَ اللهُ.

\chapter{3}

\par 1 وَامَّا مُوسَى فَكَانَ يَرْعَى غَنَمَ يَثْرُونَ حَمِيهِ كَاهِنِ مِدْيَانَ فَسَاقَ الْغَنَمَ الَى وَرَاءِ الْبَرِّيَّةِ وَجَاءَ الَى جَبَلِ اللهِ حُورِيبَ.
\par 2 وَظَهَرَ لَهُ مَلاكُ الرَّبِّ بِلَهِيبِ نَارٍ مِنْ وَسَطِ عُلَّيْقَةٍ فَنَظَرَ وَاذَا الْعُلَّيْقَةُ تَتَوَقَّدُ بِالنَّارِ وَالْعُلَّيْقَةُ لَمْ تَكُنْ تَحْتَرِقُ!
\par 3 فَقَالَ مُوسَى: «امِيلُ الانَ لانْظُرَ هَذَا الْمَنْظَرَ الْعَظِيمَ. لِمَاذَا لا تَحْتَرِقُ الْعُلَّيْقَةُ؟»
\par 4 فَلَمَّا رَاى الرَّبُّ انَّهُ مَالَ لِيَنْظُرَ نَادَاهُ اللهُ مِنْ وَسَطِ الْعُلَّيْقَةِ وَقَالَ: «مُوسَى مُوسَى». فَقَالَ: «هَئَنَذَا».
\par 5 فَقَالَ: «لا تَقْتَرِبْ الَى هَهُنَا. اخْلَعْ حِذَاءَكَ مِنْ رِجْلَيْكَ لانَّ الْمَوْضِعَ الَّذِي انْتَ وَاقِفٌ عَلَيْهِ ارْضٌ مُقَدَّسَةٌ».
\par 6 ثُمَّ قَالَ: «انَا الَهُ ابِيكَ الَهُ ابْرَاهِيمَ وَالَهُ اسْحَاقَ وَالَهُ يَعْقُوبَ». فَغَطَّى مُوسَى وَجْهَهُ لانَّهُ خَافَ انْ يَنْظُرَ الَى اللهِ.
\par 7 فَقَالَ الرَّبُّ: «انِّي قَدْ رَايْتُ مَذَلَّةَ شَعْبِي الَّذِي فِي مِصْرَ وَسَمِعْتُ صُرَاخَهُمْ مِنْ اجْلِ مُسَخِّرِيهِمْ. انِّي عَلِمْتُ اوْجَاعَهُمْ
\par 8 فَنَزَلْتُ لِانْقِذَهُمْ مِنْ ايْدِي الْمِصْرِيِّينَ وَاصْعِدَهُمْ مِنْ تِلْكَ الارْضِ الَى ارْضٍ جَيِّدَةٍ وَوَاسِعَةٍ الَى ارْضٍ تَفِيضُ لَبَنا وَعَسَلا الَى مَكَانِ الْكَنْعَانِيِّينَ وَالْحِثِّيِّينَ وَالامُورِيِّينَ وَالْفِرِزِّيِّينَ وَالْحِوِّيِّينَ وَالْيَبُوسِيِّينَ.
\par 9 وَالانَ هُوَذَا صُرَاخُ بَنِي اسْرَائِيلَ قَدْ اتَى الَيَّ وَرَايْتُ ايْضا الضِّيقَةَ الَّتِي يُضَايِقُهُمْ بِهَا الْمِصْرِيُّونَ
\par 10 فَالانَ هَلُمَّ فَارْسِلُكَ الَى فِرْعَوْنَ وَتُخْرِجُ شَعْبِي بَنِي اسْرَائِيلَ مِنْ مِصْرَ».
\par 11 فَقَالَ مُوسَى لِلَّهِ: «مَنْ انَا حَتَّى اذْهَبَ الَى فِرْعَوْنَ وَحَتَّى اخْرِجَ بَنِي اسْرَائِيلَ مِنْ مِصْرَ؟»
\par 12 فَقَالَ: «انِّي اكُونُ مَعَكَ وَهَذِهِ تَكُونُ لَكَ الْعَلامَةُ انِّي ارْسَلْتُكَ: حِينَمَا تُخْرِجُ الشَّعْبَ مِنْ مِصْرَ تَعْبُدُونَ اللهَ عَلَى هَذَا الْجَبَلِ».
\par 13 فَقَالَ مُوسَى لِلَّهِ: «هَا انَا اتِي الَى بَنِي اسْرَائِيلَ وَاقُولُ لَهُمْ: الَهُ ابَائِكُمْ ارْسَلَنِي الَيْكُمْ. فَاذَا قَالُوا لِي: مَا اسْمُهُ؟ فَمَاذَا اقُولُ لَهُمْ؟»
\par 14 فَقَالَ اللهُ لِمُوسَى: «اهْيَهِ الَّذِي اهْيَهْ». وَقَالَ: «هَكَذَا تَقُولُ لِبَنِي اسْرَائِيلَ: اهْيَهْ ارْسَلَنِي الَيْكُمْ».
\par 15 وَقَالَ اللهُ ايْضا لِمُوسَى: «هَكَذَا تَقُولُ لِبَنِي اسْرَائِيلَ: يَهْوَهْ الَهُ ابَائِكُمْ الَهُ ابْرَاهِيمَ وَالَهُ اسْحَاقَ وَالَهُ يَعْقُوبَ ارْسَلَنِي الَيْكُمْ. هَذَا اسْمِي الَى الابَدِ وَهَذَا ذِكْرِي الَى دَوْرٍ فَدَوْرٍ.
\par 16 اذْهَبْ وَاجْمَعْ شُيُوخَ اسْرَائِيلَ وَقُلْ لَهُمُ: الرَّبُّ الَهُ ابَائِكُمْ الَهُ ابْرَاهِيمَ وَاسْحَاقَ وَيَعْقُوبَ ظَهَرَ لِي قَائِلا: انِّي قَدِ افْتَقَدْتُكُمْ وَمَا صُنِعَ بِكُمْ فِي مِصْرَ.
\par 17 فَقُلْتُ اصْعِدُكُمْ مِنْ مَذَلَّةِ مِصْرَ الَى ارْضِ الْكَنْعَانِيِّينَ وَالْحِثِّيِّينَ وَالامُورِيِّينَ وَالْفِرِزِّيِّينَ وَالْحِوِّيِّينَ وَالْيَبُوسِيِّينَ الَى ارْضٍ تَفِيضُ لَبَنا وَعَسَلا.
\par 18 «فَاذَا سَمِعُوا لِقَوْلِكَ تَدْخُلُ انْتَ وَشُيُوخُ بَنِي اسْرَائِيلَ الَى مَلِكِ مِصْرَ وَتَقُولُونَ لَهُ: الرَّبُّ الَهُ الْعِبْرَانِيِّينَ الْتَقَانَا فَالانَ نَمْضِي سَفَرَ ثَلاثَةِ ايَّامٍ فِي الْبَرِّيَّةِ وَنَذْبَحُ لِلرَّبِّ الَهِنَا.
\par 19 وَلَكِنِّي اعْلَمُ انَّ مَلِكَ مِصْرَ لا يَدَعُكُمْ تَمْضُونَ وَلا بِيَدٍ قَوِيَّةٍ
\par 20 فَامُدُّ يَدِي وَاضْرِبُ مِصْرَ بِكُلِّ عَجَائِبِي الَّتِي اصْنَعُ فِيهَا. وَبَعْدَ ذَلِكَ يُطْلِقُكُمْ.
\par 21 وَاعْطِي نِعْمَةً لِهَذَا الشَّعْبِ فِي عُِيُونِ الْمِصْرِيِّينَ. فَيَكُونُ حِينَمَا تَمْضُونَ انَّكُمْ لا تَمْضُونَ فَارِغِينَ.
\par 22 بَلْ تَطْلُبُ كُلُّ امْرَاةٍ مِنْ جَارَتِهَا وَمِنْ نَزِيلَةِ بَيْتِهَا امْتِعَةَ فِضَّةٍ وَامْتِعَةَ ذَهَبٍ وَثِيَابا وَتَضَعُونَهَا عَلَى بَنِيكُمْ وَبَنَاتِكُمْ. فَتَسْلِبُونَ الْمِصْرِيِّينَ».

\chapter{4}

\par 1 فَاجَابَ مُوسَى: «وَلَكِنْ هَا هُمْ لا يُصَدِّقُونَنِي وَلا يَسْمَعُونَ لِقَوْلِي بَلْ يَقُولُونَ لَمْ يَظْهَرْ لَكَ الرَّبُّ».
\par 2 فَقَالَ لَهُ الرَّبُّ: «مَا هَذِهِ فِي يَدِكَ؟» فَقَالَ: «عَصا».
\par 3 فَقَالَ: «اطْرَحْهَا الَى الارْضِ». فَطَرَحَهَا الَى الارْضِ فَصَارَتْ حَيَّةً فَهَرَبَ مُوسَى مِنْهَا.
\par 4 ثُمَّ قَالَ الرَّبُّ لِمُوسَى: «مُدَّ يَدَكَ وَامْسِكْ بِذَنَبِهَا» (فَمَدَّ يَدَهُ وَامْسَكَ بِهِ فَصَارَتْ عَصا فِي يَدِهِ)
\par 5 «لِكَيْ يُصَدِّقُوا انَّهُ قَدْ ظَهَرَ لَكَ الرَّبُّ الَهُ ابَائِهِمْ الَهُ ابْرَاهِيمَ وَالَهُ اسْحَاقَ وَالَهُ يَعْقُوبَ».
\par 6 ثُمَّ قَالَ لَهُ الرَّبُّ ايْضا: «ادْخِلْ يَدَكَ فِي عُبِّكَ» فَادْخَلَ يَدَهُ فِي عُبِّهِ ثُمَّ اخْرَجَهَا وَاذَا يَدُهُ بَرْصَاءُ مِثْلَ الثَّلْجِ.
\par 7 ثُمَّ قَالَ لَهُ: «رُدَّ يَدَكَ الَى عُبِّكَ» (فَرَدَّ يَدَهُ الَى عُبِّهِ ثُمَّ اخْرَجَهَا مِنْ عُبِّهِ وَاذَا هِيَ قَدْ عَادَتْ مِثْلَ جَسَدِهِ)
\par 8 «فَيَكُونُ اذَا لَمْ يُصَدِّقُوكَ وَلَمْ يَسْمَعُوا لِصَوْتِ الايَةِ الاولَى انَّهُمْ يُصَدِّقُونَ صَوْتَ الايَةِ الاخِيرَةِ.
\par 9 وَيَكُونُ اذَا لَمْ يُصَدِّقُوا هَاتَيْنِ الايَتَيْنِ وَلَمْ يَسْمَعُوا لِقَوْلِكَ انَّكَ تَاخُذُ مِنْ مَاءِ النَّهْرِ وَتَسْكُبُ عَلَى الْيَابِسَةِ فَيَصِيرُ الْمَاءُ الَّذِي تَاخُذُهُ مِنَ النَّهْرِ دَما عَلَى الْيَابِسَةِ».
\par 10 فَقَالَ مُوسَى لِلرَّبِّ: «اسْتَمِعْ ايُّهَا السَّيِّدُ لَسْتُ انَا صَاحِبَ كَلامٍ مُنْذُ امْسِ وَلا اوَّلِ مِنْ امْسِ وَلا مِنْ حِينِ كَلَّمْتَ عَبْدَكَ بَلْ انَا ثَقِيلُ الْفَمِ وَاللِّسَانِ».
\par 11 فَقَالَ لَهُ الرَّبُّ: «مَنْ صَنَعَ لِلْانْسَانِ فَما اوْ مَنْ يَصْنَعُ اخْرَسَ اوْ اصَمَّ اوْ بَصِيرا اوْ اعْمَى؟ امَا هُوَ انَا الرَّبُّ؟
\par 12 فَالانَ اذْهَبْ وَانَا اكُونُ مَعَ فَمِكَ وَاعَلِّمُكَ مَا تَتَكَلَّمُ بِهِ».
\par 13 فَقَالَ: «اسْتَمِعْ ايُّهَا السَّيِّدُ ارْسِلْ بِيَدِ مَنْ تُرْسِلْ».
\par 14 فَحَمِيَ غَضَبُ الرَّبِّ عَلَى مُوسَى وَقَالَ: «الَيْسَ هَارُونُ اللاوِيُّ اخَاكَ؟ انَا اعْلَمُ انَّهُ هُوَ يَتَكَلَّمُ وَايْضا هَا هُوَ خَارِجٌ لاسْتِقْبَالِكَ. فَحِينَمَا يَرَاكَ يَفْرَحُ بِقَلْبِهِ
\par 15 فَتُكَلِّمُهُ وَتَضَعُ الْكَلِمَاتِ فِي فَمِهِ وَانَا اكُونُ مَعَ فَمِكَ وَمَعَ فَمِهِ وَاعْلِمُكُمَا مَاذَا تَصْنَعَانِ.
\par 16 وَهُوَ يُكَلِّمُ الشَّعْبَ عَنْكَ. وَهُوَ يَكُونُ لَكَ فَما وَانْتَ تَكُونُ لَهُ الَها.
\par 17 وَتَاخُذُ فِي يَدِكَ هَذِهِ الْعَصَا الَّتِي تَصْنَعُ بِهَا الايَاتِ».
\par 18 فَمَضَى مُوسَى وَرَجَعَ الَى يَثْرُونَ حَمِيهِ وَقَالَ لَهُ: «انَا اذْهَبُ وَارْجِعُ الَى اخْوَتِي الَّذِينَ فِي مِصْرَ لارَى هَلْ هُمْ بَعْدُ احْيَاءٌ». فَقَالَ يَثْرُونُ لِمُوسَى: «اذْهَبْ بِسَلامٍ».
\par 19 وَقَالَ الرَّبُّ لِمُوسَى فِي مِدْيَانَ: «اذْهَبِ ارْجِعْ الَى مِصْرَ لانَّهُ قَدْ مَاتَ جَمِيعُ الْقَوْمِ الَّذِينَ كَانُوا يَطْلُبُونَ نَفْسَكَ».
\par 20 فَاخَذَ مُوسَى امْرَاتَهُ وَبَنِيهِ وَارْكَبَهُمْ عَلَى الْحَمِيرِ وَرَجَعَ الَى ارْضِ مِصْرَ. وَاخَذَ مُوسَى عَصَا اللهِ فِي يَدِهِ.
\par 21 وَقَالَ الرَّبُّ لِمُوسَى: «عِنْدَمَا تَذْهَبُ لِتَرْجِعَ الَى مِصْرَ انْظُرْ جَمِيعَ الْعَجَائِبِ الَّتِي جَعَلْتُهَا فِي يَدِكَ وَاصْنَعْهَا قُدَّامَ فِرْعَوْنَ. وَلَكِنِّي اشَدِّدُ قَلْبَهُ حَتَّى لا يُطْلِقَ الشَّعْبَ.
\par 22 فَتَقُولُ لِفِرْعَوْنَ: هَكَذَا يَقُولُ الرَّبُّ: اسْرَائِيلُ ابْنِي الْبِكْرُ.
\par 23 فَقُلْتُ لَكَ: اطْلِقِ ابْنِي لِيَعْبُدَنِي فَابَيْتَ انْ تُطْلِقَهُ. هَا انَا اقْتُلُ ابْنَكَ الْبِكْرَ».
\par 24 وَحَدَثَ فِي الطَّرِيقِ فِي الْمَنْزِلِ انَّ الرَّبَّ الْتَقَاهُ وَطَلَبَ انْ يَقْتُلَهُ.
\par 25 فَاخَذَتْ صَفُّورَةُ صَوَّانَةً وَقَطَعَتْ غُرْلَةَ ابْنِهَا وَمَسَّتْ رِجْلَيْهِ. فَقَالَتْ: «انَّكَ عَرِيسُ دَمٍ لِي».
\par 26 فَانْفَكَّ عَنْهُ. حِينَئِذٍ قَالَتْ: «عَرِيسُ دَمٍ مِنْ اجْلِ الْخِتَانِ».
\par 27 وَقَالَ الرَّبُّ لِهَارُونَ: «اذْهَبْ الَى الْبَرِّيَّةِ لاسْتِقْبَالِ مُوسَى». فَذَهَبَ وَالْتَقَاهُ فِي جَبَلِ اللهِ وَقَبَّلَهُ.
\par 28 فَاخْبَرَ مُوسَى هَارُونَ بِجَمِيعِ كَلامِ الرَّبِّ الَّذِي ارْسَلَهُ وَبِكُلِّ الايَاتِ الَّتِي اوْصَاهُ بِهَا.
\par 29 ثُمَّ مَضَى مُوسَى وَهَارُونُ وَجَمَعَا جَمِيعَ شُيُوخِ بَنِي اسْرَائِيلَ.
\par 30 فَتَكَلَّمَ هَارُونُ بِجَمِيعِ الْكَلامِ الَّذِي كَلَّمَ الرَّبُّ مُوسَى بِهِ وَصَنَعَ الايَاتِ امَامَ عُيُونِ الشَّعْبِ.
\par 31 فَامَنَ الشَّعْبُ. وَلَمَّا سَمِعُوا انَّ الرَّبَّ افْتَقَدَ بَنِي اسْرَائِيلَ وَانَّهُ نَظَرَ مَذَلَّتَهُمْ خَرُّوا وَسَجَدُوا.

\chapter{5}

\par 1 وَبَعْدَ ذَلِكَ دَخَلَ مُوسَى وَهَارُونُ وَقَالا لِفِرْعَوْنَ: «هَكَذَا يَقُولُ الرَّبُّ الَهُ اسْرَائِيلَ: اطْلِقْ شَعْبِي لِيُعَيِّدُوا لِي فِي الْبَرِّيَّةِ».
\par 2 فَقَالَ فِرْعَوْنُ: «مَنْ هُوَ الرَّبُّ حَتَّى اسْمَعَ لِقَوْلِهِ فَاطْلِقَ اسْرَائِيلَ؟ لا اعْرِفُ الرَّبَّ وَاسْرَائِيلَ لا اطْلِقُهُ».
\par 3 فَقَالا: «الَهُ الْعِبْرَانِيِّينَ قَدِ الْتَقَانَا فَنَذْهَبُ سَفَرَ ثَلاثَةِ ايَّامٍ فِي الْبَرِّيَّةِ وَنَذْبَحُ لِلرَّبِّ الَهِنَا لِئَلا يُصِيبَنَا بِالْوَبَا اوْ بِالسَّيْفِ».
\par 4 فَقَالَ لَهُمَا مَلِكُ مِصْرَ: «لِمَاذَا يَا مُوسَى وَهَارُونُ تُبَطِّلانِ الشَّعْبَ مِنْ اعْمَالِهِ؟ اذْهَبَا الَى اثْقَالِكُمَا».
\par 5 وَقَالَ فِرْعَوْنُ: «هُوَذَا الانَ شَعْبُ الارْضِ كَثِيرٌ وَانْتُمَا تُرِيحَانِهِمْ مِنْ اثْقَالِهِمْ».
\par 6 فَامَرَ فِرْعَوْنُ فِي ذَلِكَ الْيَوْمِ مُسَخِّرِي الشَّعْبِ وَمُدَبِّرِيهِ قَائِلا:
\par 7 «لا تَعُودُوا تُعْطُونَ الشَّعْبَ تِبْنا لِصُنْعِ اللِّبْنِ كَامْسِ وَاوَّلَ مِنْ امْسِ. لِيَذْهَبُوا هُمْ وَيَجْمَعُوا تِبْنا لانْفُسِهِمْ.
\par 8 وَمِقْدَارَ اللِّبْنِ الَّذِي كَانُوا يَصْنَعُونَهُ امْسِ وَاوَّلَ مِنْ امْسِ تَجْعَلُونَ عَلَيْهِمْ. لا تَنْقُصُوا مِنْهُ فَانَّهُمْ مُتَكَاسِلُونَ لِذَلِكَ يَصْرُخُونَ قَائِلِينَ: نَذْهَبُ وَنَذْبَحُ لالَهِنَا.
\par 9 لِيُثَقَّلِ الْعَمَلُ عَلَى الْقَوْمِ حَتَّى يَشْتَغِلُوا بِهِ وَلا يَلْتَفِتُوا الَى كَلامِ الْكَذِبِ».
\par 10 فَخَرَجَ مُسَخِّرُو الشَّعْبِ وَمُدَبِّرُوهُ وَقَالُوا لِلشَّعْبَ: «هَكَذَا يَقُولُ فِرْعَوْنُ: لَسْتُ اعْطِيكُمْ تِبْنا.
\par 11 اذْهَبُوا انْتُمْ وَخُذُوا لانْفُسِكُمْ تِبْنا مِنْ حَيْثُ تَجِدُونَ. انَّهُ لا يُنْقَصُ مِنْ عَمَلِكُمْ شَيْءٌ».
\par 12 فَتَفَرَّقَ الشَّعْبُ فِي كُلِّ ارْضِ مِصْرَ لِيَجْمَعُوا قَشّا عِوَضا عَنِ التِّبْنِ.
\par 13 وَكَانَ الْمُسَخِّرُونَ يُعَجِّلُونَهُمْ قَائِلِينَ: «كَمِّلُوا اعْمَالَكُمْ امْرَ كُلِّ يَوْمٍ بِيَوْمِهِ كَمَا كَانَ حِينَمَا كَانَ التِّبْنُ».
\par 14 فَضُرِبَ مُدَبِّرُو بَنِي اسْرَائِيلَ الَّذِينَ اقَامَهُمْ عَلَيْهِمْ مُسَخِّرُو فِرْعَوْنَ وَقِيلَ لَهُمْ: «لِمَاذَا لَمْ تُكَمِّلُوا فَرِيضَتَكُمْ مِنْ صُنْعِ اللِّبْنِ امْسِ وَالْيَوْمَ كَالامْسِ وَاوَّلَ مِنْ امْسِ؟»
\par 15 فَاتَى مُدَبِّرُو بَنِي اسْرَائِيلَ وَصَرَخُوا الَى فِرْعَوْنَ قَائِلِينَ: «لِمَاذَا تَفْعَلُ هَكَذَا بِعَبِيدِكَ؟
\par 16 التِّبْنُ لَيْسَ يُعْطَى لِعَبِيدِكَ وَاللِّبْنُ يَقُولُونَ لَنَا اصْنَعُوهُ وَهُوَذَا عَبِيدُكَ مَضْرُوبُونَ وَقَدْ اخْطَا شَعْبُكَ».
\par 17 فَقَالَ: «مُتَكَاسِلُونَ انْتُمْ مُتَكَاسِلُونَ. لِذَلِكَ تَقُولُونَ: نَذْهَبُ وَنَذْبَحُ لِلرَّبِّ.
\par 18 فَالانَ اذْهَبُوا اعْمَلُوا. وَتِبْنٌ لا يُعْطَى لَكُمْ وَمِقْدَارَ اللِّبْنِ تُقَدِّمُونَهُ».
\par 19 فَرَاى مُدَبِّرُو بَنِي اسْرَائِيلَ انْفُسَهُمْ فِي بَلِيَّةٍ اذْ قِيلَ لَهُمْ لا تُنَقِّصُوا مِنْ لِبْنِكُمْ امْرَ كُلِّ يَوْمٍ بِيَوْمِهِ.
\par 20 وَصَادَفُوا مُوسَى وَهَارُونَ وَاقِفَيْنِ لِلِقَائِهِمْ حِينَ خَرَجُوا مِنْ لَدُنْ فِرْعَوْنَ.
\par 21 فَقَالُوا لَهُمَا: «يَنْظُرُ الرَّبُّ الَيْكُمَا وَيَقْضِي لانَّكُمَا انْتَنْتُمَا رَائِحَتَنَا فِي عَيْنَيْ فِرْعَوْنَ وَفِي عُيُونِ عَبِيدِهِ حَتَّى تُعْطِيَا سَيْفا فِي ايْدِيهِمْ لِيَقْتُلُونَا».
\par 22 فَرَجَعَ مُوسَى الَى الرَّبِّ وَقَالَ: «يَا سَيِّدُ لِمَاذَا اسَاتَ الَى هَذَا الشَّعْبِ؟ لِمَاذَا ارْسَلْتَنِي؟
\par 23 فَانَّهُ مُنْذُ دَخَلْتُ الَى فِرْعَوْنَ لاتَكَلَّمَ بِاسْمِكَ اسَاءَ الَى هَذَا الشَّعْبِ. وَانْتَ لَمْ تُخَلِّصْ شَعْبَكَ».

\chapter{6}

\par 1 فَقَالَ الرَّبُّ لِمُوسَى: «الانَ تَنْظُرُ مَا انَا افْعَلُ بِفِرْعَوْنَ. فَانَّهُ بِيَدٍ قَوِيَّةٍ يُطْلِقُهُمْ وَبِيَدٍ قَوِيَّةٍ يَطْرُدُهُمْ مِنْ ارْضِهِ».
\par 2 ثُمَّ قَالَ اللهُ لِمُوسَى: «انَا الرَّبُّ.
\par 3 وَانَا ظَهَرْتُ لابْرَاهِيمَ وَاسْحَاقَ وَيَعْقُوبَ بِانِّي الْالَهُ الْقَادِرُ عَلَى كُلِّ شَيْءٍ. وَامَّا بِاسْمِي «يَهْوَهْ» فَلَمْ اعْرَفْ عِنْدَهُمْ.
\par 4 وَايْضا اقَمْتُ مَعَهُمْ عَهْدِي: انْ اعْطِيَهُمْ ارْضَ كَنْعَانَ ارْضَ غُرْبَتِهِمِ الَّتِي تَغَرَّبُوا فِيهَا.
\par 5 وَانَا ايْضا قَدْ سَمِعْتُ انِينَ بَنِي اسْرَائِيلَ الَّذِينَ يَسْتَعْبِدُهُمُ الْمِصْرِيُّونَ وَتَذَكَّرْتُ عَهْدِي.
\par 6 لِذَلِكَ قُلْ لِبَنِي اسْرَائِيلَ: انَا الرَّبُّ. وَانَا اخْرِجُكُمْ مِنْ تَحْتِ اثْقَالِ الْمِصْرِيِّينَ وَانْقِذُكُمْ مِنْ عُبُودِيَّتِهِمْ وَاخَلِّصُكُمْ بِذِرَاعٍ مَمْدُودَةٍ وَبِاحْكَامٍ عَظِيمَةٍ
\par 7 وَاتَّخِذُكُمْ لِي شَعْبا وَاكُونُ لَكُمْ الَها. فَتَعْلَمُونَ انِّي انَا الرَّبُّ الَهُكُمُ الَّذِي يُخْرِجُكُمْ مِنْ تَحْتِ اثْقَالِ الْمِصْرِيِّينَ.
\par 8 وَادْخِلُكُمْ الَى الارْضِ الَّتِي رَفَعْتُ يَدِي انْ اعْطِيَهَا لابْرَاهِيمَ وَاسْحَاقَ وَيَعْقُوبَ. وَاعْطِيَكُمْ ايَّاهَا مِيرَاثا. انَا الرَّبُّ».
\par 9 فَكَلَّمَ مُوسَى بَنِي اسْرَائِيلَ هَكَذَا وَلَكِنْ لَمْ يَسْمَعُوا لِمُوسَى مِنْ صِغَرِ النَّفْسِ وَمِنَ الْعُبُودِيَّةِ الْقَاسِيَةِ.
\par 10 ثُمَّ قَالَ الرَّبُّ لِمُوسَى:
\par 11 «ادْخُلْ قُلْ لِفِرْعَوْنَ مَلِكِ مِصْرَ انْ يُطْلِقَ بَنِي اسْرَائِيلَ مِنْ ارْضِهِ».
\par 12 فَتَكَلَّمَ مُوسَى امَامَ الرَّبِّ قَائِلا: «هُوَذَا بَنُو اسْرَائِيلَ لَمْ يَسْمَعُوا لِي فَكَيْفَ يَسْمَعُنِي فِرْعَوْنُ وَانَا اغْلَفُ الشَّفَتَيْنِ؟»
\par 13 فَكَلَّمَ الرَّبُّ مُوسَى وَهَارُونَ وَاوْصَى مَعَهُمَا الَى بَنِي اسْرَائِيلَ وَالَى فِرْعَوْنَ مَلِكِ مِصْرَ فِي اخْرَاجِ بَنِي اسْرَائِيلَ مِنْ ارْضِ مِصْرَ.
\par 14 هَؤُلاءِ رُؤَسَاءُ بُيُوتِ ابَائِهِمْ: بَنُو رَاوبَيْنَ بِكْرِ اسْرَائِيلَ: حَنُوكُ وَفَلُّو وَحَصْرُونُ وَكَرْمِي. هَذِهِ عَشَائِرُ رَاوبَيْنَ.
\par 15 وَبَنُو شَمْعُونَ: يَمُوئِيلُ وَيَامِينُ وَاوهَدُ وَيَاكِينُ وَصُوحَرُ وَشَاولُ ابْنُ الْكَنْعَانِيَّةِ. هَذِهِ عَشَائِرُ شَمْعُونَ.
\par 16 وَهَذِهِ اسْمَاءُ بَنِي لاوِي بِحَسَبِ مَوَالِيدِهِمْ: جَرْشُونُ وَقَهَاتُ وَمَرَارِي. وَكَانَتْ سِنُو حَيَاةِ لاوِي مِئَةً وَسَبْعا وَثَلاثِينَ سَنَةً.
\par 17 ابْنَا جَرْشُونَ: لِبْنِي وَشَمْعِي بِحَسَبِ عَشَائِرِهِمَا.
\par 18 وَبَنُو قَهَاتَ: عَمْرَامُ وَيِصْهَارُ وَحَبْرُونُ وَعُزِّيئِيلُ. وَكَانَتْ سِنُو حَيَاةِ قَهَاتَ مِئَةً وَثَلاثا وَثَلاثِينَ سَنَةً.
\par 19 وَابْنَا مَرَارِي: مَحْلِي وَمُوشِي. هَذِهِ عَشَائِرُ اللاوِيِّينَ بِحَسَبِ مَوَالِيدِهِمْ.
\par 20 وَاخَذَ عَمْرَامُ يُوكَابَدَ عَمَّتَهُ زَوْجَةً لَهُ. فَوَلَدَتْ لَهُ هَارُونَ وَمُوسَى. وَكَانَتْ سِنُو حَيَاةِ عَمْرَامَ مِئَةً وَسَبْعا وَثَلاثِينَ سَنَةً.
\par 21 وَبَنُو يِصْهَارَ: قُورَحُ وَنَافَجُ وَذِكْرِي.
\par 22 وَبَنُو عُزِّيئِيلَ: مِيشَائِيلُ وَالْصَافَانُ وَسِتْرِي.
\par 23 وَاخَذَ هَارُونُ الِيشَابَعَ بِنْتَ عَمِّينَادَابَ اخْتَ نَحْشُونَ زَوْجَةً لَهُ فَوَلَدَتْ لَهُ نَادَابَ وَابِيهُوَ وَالِعَازَارَ وَايثَامَارَ.
\par 24 وَبَنُو قُورَحَ اسِّيرُ وَالْقَانَةُ وَابِيَاسَافُ. هَذِهِ عَشَائِرُ الْقُورَحِيِّينَ.
\par 25 وَالِعَازَارُ بْنُ هَارُونَ اخَذَ لِنَفْسِهِ مِنْ بَنَاتِ فُوطِيئِيلَ زَوْجَةً فَوَلَدَتْ لَهُ فِينَحَاسَ. هَؤُلاءِ هُمْ رُؤَسَاءُ ابَاءِ اللاوِيِّينَ بِحَسَبِ عَشَائِرِهِمْ.
\par 26 هَذَانِ هُمَا هَارُونُ وَمُوسَى اللَّذَانِ قَالَ الرَّبُّ لَهُمَا: «اخْرِجَا بَنِي اسْرَائِيلَ مِنْ ارْضِ مِصْرَ» بِحَسَبِ اجْنَادِهِمْ.
\par 27 هُمَا اللَّذَانِ كَلَّمَا فِرْعَوْنَ مَلِكَ مِصْرَ فِي اخْرَاجِ بَنِي اسْرَائِيلَ مِنْ مِصْرَ. هَذَانِ هُمَا مُوسَى وَهَارُونُ.
\par 28 وَكَانَ يَوْمَ كَلَّمَ الرَّبُّ مُوسَى فِي ارْضِ مِصْرَ
\par 29 انَّ الرَّبَّ قَالَ َلهُ: «انَا الرَّبُّ. كَلِّمْ فِرْعَوْنَ مَلِكَ مِصْرَ بِكُلِّ مَا انَا اكَلِّمُكَ بِهِ».
\par 30 فَقَالَ مُوسَى امَامَ الرَّبِّ: «هَا انَا اغْلَفُ الشَّفَتَيْنِ. فَكَيْفَ يَسْمَعُ لِي فِرْعَوْنُ؟»

\chapter{7}

\par 1 فَقَالَ الرَّبُّ لِمُوسَى: «انْظُرْ! انَا جَعَلْتُكَ الَها لِفِرْعَوْنَ. وَهَارُونُ اخُوكَ يَكُونُ نَبِيَّكَ.
\par 2 انْتَ تَتَكَلَّمُ بِكُلِّ مَا امُرُكَ وَهَارُونُ اخُوكَ يُكَلِّمُ فِرْعَوْنَ لِيُطْلِقَ بَنِي اسْرَائِيلَ مِنْ ارْضِهِ.
\par 3 وَلَكِنِّي اقَسِّي قَلْبَ فِرْعَوْنَ وَاكَثِّرُ ايَاتِي وَعَجَائِبِي فِي ارْضِ مِصْرَ.
\par 4 وَلا يَسْمَعُ لَكُمَا فِرْعَوْنُ حَتَّى اجْعَلَ يَدِي عَلَى مِصْرَ فَاخْرِجَ اجْنَادِي شَعْبِي بَنِي اسْرَائِيلَ مِنْ ارْضِ مِصْرَ بِاحْكَامٍ عَظِيمَةٍ.
\par 5 فَيَعْرِفُ الْمِصْرِيُّونَ انِّي انَا الرَّبُّ حِينَمَا امُدُّ يَدِي عَلَى مِصْرَ وَاخْرِجُ بَنِي اسْرَائِيلَ مِنْ بَيْنِهِمْ».
\par 6 فَفَعَلَ مُوسَى وَهَارُونُ كَمَا امَرَهُمَا الرَّبُّ. هَكَذَا فَعَلا.
\par 7 وَكَانَ مُوسَى ابْنَ ثَمَانِينَ سَنَةً وَهَارُونُ ابْنَ ثَلاثٍ وَثَمَانِينَ سَنَةً حِينَ كَلَّمَا فِرْعَوْنَ.
\par 8 وَقَالَ الرَّبُّ لِمُوسَى وَهَارُونَ:
\par 9 «اذَا كَلَّمَكُمَا فِرْعَوْنُ قَائِلا: هَاتِيَا عَجِيبَةً تَقُولُ لِهَارُونَ: خُذْ عَصَاكَ وَاطْرَحْهَا امَامَ فِرْعَوْنَ فَتَصِيرَ ثُعْبَانا».
\par 10 فَدَخَلَ مُوسَى وَهَارُونُ الَى فِرْعَوْنَ وَفَعَلا هَكَذَا كَمَا امَرَ الرَّبُّ. طَرَحَ هَارُونُ عَصَاهُ امَامَ فِرْعَوْنَ وَامَامَ عَبِيدِهِ فَصَارَتْ ثُعْبَانا.
\par 11 فَدَعَا فِرْعَوْنُ ايْضا الْحُكَمَاءَ وَالسَّحَرَةَ فَفَعَلَ عَرَّافُو مِصْرَ ايْضا بِسِحْرِهِمْ كَذَلِكَ.
\par 12 طَرَحُوا كُلُّ وَاحِدٍ عَصَاهُ فَصَارَتِ الْعِصِيُّ ثَعَابِينَ. وَلَكِنْ عَصَا هَارُونَ ابْتَلَعَتْ عِصِيَّهُمْ.
\par 13 فَاشْتَدَّ قَلْبُ فِرْعَوْنَ فَلَمْ يَسْمَعْ لَهُمَا كَمَا تَكَلَّمَ الرَّبُّ.
\par 14 ثُمَّ قَالَ الرَّبُّ لِمُوسَى: «قَلْبُ فِرْعَوْنَ غَلِيظٌ. قَدْ ابَى انْ يُطْلِقَ الشَّعْبَ.
\par 15 اذْهَبْ الَى فِرْعَوْنَ فِي الصَّبَاحِ. انَّهُ يَخْرُجُ الَى الْمَاءِ وَقِفْ لِلِقَائِهِ عَلَى حَافَةِ النَّهْرِ. وَالْعَصَا الَّتِي تَحَوَّلَتْ حَيَّةً تَاخُذُهَا فِي يَدِكَ.
\par 16 وَتَقُولُ لَهُ: الرَّبُّ الَهُ الْعِبْرَانِيِّينَ ارْسَلَنِي الَيْكَ قَائِلا: اطْلِقْ شَعْبِي لِيَعْبُدُونِي فِي الْبَرِّيَّةِ. وَهُوَذَا حَتَّى الانَ لَمْ تَسْمَعْ.
\par 17 هَكَذَا يَقُولُ الرَّبُّ: بِهَذَا تَعْرِفُ انِّي انَا الرَّبُّ: هَا انَا اضْرِبُ بِالْعَصَا الَّتِي فِي يَدِي عَلَى الْمَاءِ الَّذِي فِي النَّهْرِ فَيَتَحَوَّلُ دَما.
\par 18 وَيَمُوتُ السَّمَكُ الَّذِي فِي النَّهْرِ وَيَنْتِنُ النَّهْرُ. فَيَعَافُ الْمِصْرِيُّونَ انْ يَشْرَبُوا مَاءً مِنَ النَّهْرِ».
\par 19 ثُمَّ قَالَ الرَّبُّ لِمُوسَى: «قُلْ لِهَارُونَ: خُذْ عَصَاكَ وَمُدَّ يَدَكَ عَلَى مِيَاهِ الْمِصْرِيِّينَ عَلَى انْهَارِهِمْ وَعَلَى سَوَاقِيهِمْ وَعَلَى اجَامِهِمْ وَعَلَى كُلِّ مُجْتَمَعَاتِ مِيَاهِهِمْ لِتَصِيرَ دَما. فَيَكُونَ دَمٌ فِي كُلِّ ارْضِ مِصْرَ فِي الاخْشَابِ وَفِي الاحْجَارِ».
\par 20 فَفَعَلَ مُوسَى وَهَارُونُ هَكَذَا كَمَا امَرَ الرَّبُّ. رَفَعَ الْعَصَا وَضَرَبَ الْمَاءَ الَّذِي فِي النَّهْرِ امَامَ عَيْنَيْ فِرْعَوْنَ وَامَامَ عُيُونِ عَبِيدِهِ فَتَحَوَّلَ كُلُّ الْمَاءِ الَّذِي فِي النَّهْرِ دَما.
\par 21 وَمَاتَ السَّمَكُ الَّذِي فِي النَّهْرِ وَانْتَنَ النَّهْرُ فَلَمْ يَقْدِرِ الْمِصْرِيُّونَ انْ يَشْرَبُوا مَاءً مِنَ النَّهْرِ. وَكَانَ الدَّمُ فِي كُلِّ ارْضِ مِصْرَ.
\par 22 وَفَعَلَ عَرَّافُو مِصْرَ كَذَلِكَ بِسِحْرِهِمْ. فَاشْتَدَّ قَلْبُ فِرْعَوْنَ فَلَمْ يَسْمَعْ لَهُمَا كَمَا تَكَلَّمَ الرَّبُّ.
\par 23 ثُمَّ انْصَرَفَ فِرْعَوْنُ وَدَخَلَ بَيْتَهُ وَلَمْ يُوَجِّهْ قَلْبَهُ الَى هَذَا ايْضا.
\par 24 وَحَفَرَ جَمِيعُ الْمِصْرِيِّينَ حَوَالَيِ النَّهْرِ لاجْلِ مَاءٍ لِيَشْرَبُوا لانَّهُمْ لَمْ يَقْدِرُوا انْ يَشْرَبُوا مِنْ مَاءِ النَّهْرِ.
\par 25 وَلَمَّا كَمُلَتْ سَبْعَةُ ايَّامٍ بَعْدَ مَا ضَرَبَ الرَّبُّ النَّهْرَ

\chapter{8}

\par 1 قَالَ الرَّبُّ لِمُوسَى: «ادْخُلْ الَى فِرْعَوْنَ وَقُلْ لَهُ: هَكَذَا يَقُولُ الرَّبُّ: اطْلِقْ شَعْبِي لِيَعْبُدُونِي.
\par 2 وَانْ كُنْتَ تَابَى انْ تُطْلِقَهُمْ فَهَا انَا اضْرِبُ جَمِيعَ تُخُومِكَ بِالضَّفَادِعِ.
\par 3 فَيَفِيضُ النَّهْرُ ضَفَادِعَ. فَتَصْعَدُ وَتَدْخُلُ الَى بَيْتِكَ وَالَى مِخْدَعِ فِرَاشِكَ وَعَلَى سَرِيرِكَ وَالَى بُيُوتِ عَبِيدِكَ وَعَلَى شَعْبِكَ وَالَى تَنَانِيرِكَ وَالَى مَعَاجِنِكَ.
\par 4 عَلَيْكَ وَعَلَى شَعْبِكَ وَعَبِيدِكَ تَصْعَدُ الضَّفَادِعُ».
\par 5 فَقَالَ الرَّبُّ لِمُوسَى: «قُلْ لِهَارُونَ: مُدَّ يَدَكَ بِعَصَاكَ عَلَى الانْهَارِ وَالسَّوَاقِي وَالاجَامِ وَاصْعِدِ الضَّفَادِعَ عَلَى ارْضِ مِصْرَ».
\par 6 فَمَدَّ هَارُونُ يَدَهُ عَلَى مِيَاهِ مِصْرَ فَصَعِدَتِ الضَّفَادِعُ وَغَطَّتْ ارْضَ مِصْرَ.
\par 7 وَفَعَلَ كَذَلِكَ الْعَرَّافُونَ بِسِحْرِهِمْ وَاصْعَدُوا الضَّفَادِعَ عَلَى ارْضِ مِصْرَ.
\par 8 فَدَعَا فِرْعَوْنُ مُوسَى وَهَارُونَ وَقَالَ: «صَلِّيَا الَى الرَّبِّ لِيَرْفَعَ الضَّفَادِعَ عَنِّي وَعَنْ شَعْبِي فَاطْلِقَ الشَّعْبَ لِيَذْبَحُوا لِلرَّبِّ».
\par 9 فَقَالَ مُوسَى لِفِرْعَوْنَ: «عَيِّنْ لِي مَتَى اصَلِّي لاجْلِكَ وَلاجْلِ عَبِيدِكَ وَشَعْبِكَ لِقَطْعِ الضَّفَادِعِ عَنْكَ وَعَنْ بُيُوتِكَ. وَلَكِنَّهَا تَبْقَى فِي النَّهْرِ».
\par 10 فَقَالَ: «غَدا». فَقَالَ: «كَقَوْلِكَ». لِكَيْ تَعْرِفَ انْ لَيْسَ مِثْلُ الرَّبِّ الَهِنَا.
\par 11 فَتَرْتَفِعُ الضَّفَادِعُ عَنْكَ وَعَنْ بُيُوتِكَ وَعَبِيدِكَ وَشَعْبِكَ. وَلَكِنَّهَا تَبْقَى فِي النَّهْرِ».
\par 12 ثُمَّ خَرَجَ مُوسَى وَهَارُونُ مِنْ لَدُنْ فِرْعَوْنَ وَصَرَخَ مُوسَى الَى الرَّبِّ مِنْ اجْلِ الضَّفَادِعِ الَّتِي جَعَلَهَا عَلَى فِرْعَوْنَ
\par 13 فَفَعَلَ الرَّبُّ كَقَوْلِ مُوسَى. فَمَاتَتِ الضَّفَادِعُ مِنَ الْبُيُوتِ وَالدُّورِ وَالْحُقُولِ.
\par 14 وَجَمَعُوهَا كُوَما كَثِيرَةً حَتَّى انْتَنَتِ الارْضُ.
\par 15 فَلَمَّا رَاى فِرْعَوْنُ انَّهُ قَدْ حَصَلَ الْفَرَجُ اغْلَظَ قَلْبَهُ وَلَمْ يَسْمَعْ لَهُمَا كَمَا تَكَلَّمَ الرَّبُّ.
\par 16 ثُمَّ قَالَ الرَّبُّ لِمُوسَى: «قُلْ لِهَارُونَ: مُدَّ عَصَاكَ وَاضْرِبْ تُرَابَ الارْضِ لِيَصِيرَ بَعُوضا فِي جَمِيعِ ارْضِ مِصْرَ».
\par 17 فَفَعَلا كَذَلِكَ. مَدَّ هَارُونُ يَدَهُ بِعَصَاهُ وَضَرَبَ تُرَابَ الارْضِ فَصَارَ الْبَعُوضُ عَلَى النَّاسِ وَعَلَى الْبَهَائِمِ. كُلُّ تُرَابِ الارْضِ صَارَ بَعُوضا فِي جَمِيعِ ارْضِ مِصْرَ.
\par 18 وَفَعَلَ كَذَلِكَ الْعَرَّافُونَ بِسِحْرِهِمْ لِيُخْرِجُوا الْبَعُوضَ فَلَمْ يَسْتَطِيعُوا. وَكَانَ الْبَعُوضُ عَلَى النَّاسِ وَعَلَى الْبَهَائِمِ.
\par 19 فَقَالَ الْعَرَّافُونَ لِفِرْعَوْنَ: «هَذَا اصْبِعُ اللهِ». وَلَكِنِ اشْتَدَّ قَلْبُ فِرْعَوْنَ فَلَمْ يَسْمَعْ لَهُمَا كَمَا تَكَلَّمَ الرَّبُّ.
\par 20 ثُمَّ قَالَ الرَّبُّ لِمُوسَى: «بَكِّرْ فِي الصَّبَاحِ وَقِفْ امَامَ فِرْعَوْنَ. انَّهُ يَخْرُجُ الَى الْمَاءِ. وَقُلْ لَهُ: هَكَذَا يَقُولُ الرَّبُّ: اطْلِقْ شَعْبِي لِيَعْبُدُونِي.
\par 21 فَانَّهُ انْ كُنْتَ لا تُطْلِقُ شَعْبِي هَا انَا ارْسِلُ عَلَيْكَ وَعَلَى عَبِيدِكَ وَعَلَى شَعْبِكَ وَعَلَى بُيُوتِكَ الذُّبَّانَ فَتَمْتَلِئُ بُيُوتُ الْمِصْرِيِّينَ ذُبَّانا. وَايْضا الارْضُ الَّتِي هُمْ عَلَيْهَا.
\par 22 وَلَكِنْ امَيِّزُ فِي ذَلِكَ الْيَوْمِ ارْضَ جَاسَانَ حَيْثُ شَعْبِي مُقِيمٌ حَتَّى لا يَكُونُ هُنَاكَ ذُبَّانٌ. لِتَعْلَمَ انِّي انَا الرَّبُّ فِي الارْضِ.
\par 23 وَاجْعَلُ فَرْقا بَيْنَ شَعْبِي وَشَعْبِكَ. غَدا تَكُونُ هَذِهِ الايَةُ».
\par 24 فَفَعَلَ الرَّبُّ هَكَذَا. فَدَخَلَتْ ذُبَّانٌ كَثِيرَةٌ الَى بَيْتِ فِرْعَوْنَ وَبُيُوتِ عَبِيدِهِ. وَفِي كُلِّ ارْضِ مِصْرَ خَرِبَتِ الارْضُ مِنَ الذُّبَّانِ.
\par 25 فَدَعَا فِرْعَوْنُ مُوسَى وَهَارُونَ وَقَالَ: «اذْهَبُوا اذْبَحُوا لالَهِكُمْ فِي هَذِهِ الارْضِ».
\par 26 فَقَالَ مُوسَى: «لا يَصْلُحُ انْ نَفْعَلَ هَكَذَا لانَّنَا انَّمَا نَذْبَحُ رِجْسَ الْمِصْرِيِّينَ لِلرَّبِّ الَهِنَا. انْ ذَبَحْنَا رِجْسَ الْمِصْرِيِّينَ امَامَ عُِيُونِهِمْ افَلا يَرْجُمُونَنَا؟
\par 27 نَذْهَبُ سَفَرَ ثَلاثَةِ ايَّامٍ فِي الْبَرِّيَّةِ وَنَذْبَحُ لِلرَّبِّ الَهِنَا كَمَا يَقُولُ لَنَا».
\par 28 فَقَالَ فِرْعَوْنُ: «انَا اطْلِقُكُمْ لِتَذْبَحُوا لِلرَّبِّ الَهِكُمْ فِي الْبَرِّيَّةِ. وَلَكِنْ لا تَذْهَبُوا بَعِيدا. صَلِّيَا لاجْلِي».
\par 29 فَقَالَ مُوسَى: «هَا انَا اخْرُجُ مِنْ لَدُنْكَ وَاصَلِّي الَى الرَّبِّ فَتَرْتَفِعُ الذُّبَّانُ عَنْ فِرْعَوْنَ وَعَبِيدِهِ وَشَعْبِهِ غَدا. وَلَكِنْ لا يَعُدْ فِرْعَوْنُ يُخَاتِلُ حَتَّى لا يُطْلِقَ الشَّعْبَ لِيَذْبَحَ لِلرَّبِّ».
\par 30 فَخَرَجَ مُوسَى مِنْ لَدُنْ فِرْعَوْنَ وَصَلَّى الَى الرَّبِّ.
\par 31 فَفَعَلَ الرَّبُّ كَقَوْلِ مُوسَى فَارْتَفَعَ الذُّبَّانُ عَنْ فِرْعَوْنَ وَعَبِيدِهِ وَشَعْبِهِ. لَمْ تَبْقَ وَاحِدَةٌ!
\par 32 وَلَكِنْ اغْلَظَ فِرْعَوْنُ قَلْبَهُ هَذِهِ الْمَرَّةَ ايْضا فَلَمْ يُطْلِقِ الشَّعْبَ.

\chapter{9}

\par 1 ثُمَّ قَالَ الرَّبُّ لِمُوسَى: «ادْخُلْ الَى فِرْعَوْنَ وَقُلْ لَهُ: هَكَذَا يَقُولُ الرَّبُّ الَهُ الْعِبْرَانِيِّينَ اطْلِقْ شَعْبِي لِيَعْبُدُونِي.
\par 2 فَانَّهُ انْ كُنْتَ تَابَى انْ تُطْلِقَهُمْ وَكُنْتَ تُمْسِكُهُمْ بَعْدُ
\par 3 فَهَا يَدُ الرَّبِّ تَكُونُ عَلَى مَوَاشِيكَ الَّتِي فِي الْحَقْلِ عَلَى الْخَيْلِ وَالْحَمِيرِ وَالْجِمَالِ وَالْبَقَرِ وَالْغَنَمِ وَبَا ثَقِيلا جِدّا.
\par 4 وَيُمَيِّزُ الرَّبُّ بَيْنَ مَوَاشِي اسْرَائِيلَ وَمَوَاشِي الْمِصْرِيِّينَ. فَلا يَمُوتُ مِنْ كُلِّ مَا لِبَنِي اسْرَائِيلَ شَيْءٌ».
\par 5 وَعَيَّنَ الرَّبُّ وَقْتا قَائِلا: «غَدا يَفْعَلُ الرَّبُّ هَذَا الامْرَ فِي الارْضِ».
\par 6 فَفَعَلَ الرَّبُّ هَذَا الامْرَ فِي الْغَدِ. فَمَاتَتْ جَمِيعُ مَوَاشِي الْمِصْرِيِّينَ. وَامَّا مَوَاشِي بَنِي اسْرَائِيلَ فَلَمْ يَمُتْ مِنْهَا وَاحِدٌ.
\par 7 وَارْسَلَ فِرْعَوْنُ وَاذَا مَوَاشِي اسْرَائِيلَ لَمْ يَمُتْ مِنْهَا وَلا وَاحِدٌ. وَلَكِنْ غَلُظَ قَلْبُ فِرْعَوْنَ فَلَمْ يُطْلِقِ الشَّعْبَ.
\par 8 ثُمَّ قَالَ الرَّبُّ لِمُوسَى وَهَارُونَ: «خُذَا مِلْءَ ايْدِيكُمَا مِنْ رَمَادِ الاتُونِ وَلْيُذَرِّهِ مُوسَى نَحْوَ السَّمَاءِ امَامَ عَيْنَيْ فِرْعَوْنَ
\par 9 لِيَصِيرَ غُبَارا عَلَى كُلِّ ارْضِ مِصْرَ. فَيَصِيرَ عَلَى النَّاسِ وَعَلَى الْبَهَائِمِ دَمَامِلَ طَالِعَةً بِبُثُورٍ فِي كُلِّ ارْضِ مِصْرَ».
\par 10 فَاخَذَا رَمَادَ الاتُونِ وَوَقَفَا امَامَ فِرْعَوْنَ وَذَرَّاهُ مُوسَى نَحْوَ السَّمَاءِ فَصَارَ دَمَامِلَ بُثُورٍ طَالِعَةً فِي النَّاسِ وَفِي الْبَهَائِمِ.
\par 11 وَلَمْ يَسْتَطِعِ الْعَرَّافُونَ انْ يَقِفُوا امَامَ مُوسَى مِنْ اجْلِ الدَّمَامِلِ لانَّ الدَّمَامِلَ كَانَتْ فِي الْعَرَّافِينَ وَفِي كُلِّ الْمِصْرِيِّينَ.
\par 12 وَلَكِنْ شَدَّدَ الرَّبُّ قَلْبَ فِرْعَوْنَ فَلَمْ يَسْمَعْ لَهُمَا كَمَا كَلَّمَ الرَّبُّ مُوسَى.
\par 13 ثُمَّ قَالَ الرَّبُّ لِمُوسَى: «بَكِّرْ فِي الصَّبَاحِ وَقِفْ امَامَ فِرْعَوْنَ وَقُلْ لَهُ: هَكَذَا يَقُولُ الرَّبُّ الَهُ الْعِبْرَانِيِّينَ اطْلِقْ شَعْبِي لِيَعْبُدُونِي.
\par 14 لانِّي هَذِهِ الْمَرَّةَ ارْسِلُ جَمِيعَ ضَرَبَاتِي الَى قَلْبِكَ وَعَلَى عَبِيدِكَ وَشَعْبِكَ لِتَعْرِفَ انْ لَيْسَ مِثْلِي فِي كُلِّ الارْضِ.
\par 15 فَانَّهُ الانَ لَوْ كُنْتُ امُدُّ يَدِي وَاضْرِبُكَ وَشَعْبَكَ بِالْوَبَا لَكُنْتَ تُبَادُ مِنَ الارْضِ.
\par 16 وَلَكِنْ لاجْلِ هَذَا اقَمْتُكَ لِارِيَكَ قُوَّتِي وَلِيُخْبَرَ بِاسْمِي فِي كُلِّ الارْضِ.
\par 17 انْتَ مُعَانِدٌ بَعْدُ لِشَعْبِي حَتَّى لا تُطْلِقَهُ.
\par 18 هَا انَا غَدا مِثْلَ الانَ امْطِرُ بَرَدا عَظِيما جِدّا لَمْ يَكُنْ مِثْلُهُ فِي مِصْرَ مُنْذُ يَوْمِ تَاسِيسِهَا الَى الانَ.
\par 19 فَالانَ ارْسِلِ احْمِ مَوَاشِيَكَ وَكُلَّ مَا لَكَ فِي الْحَقْلِ. جَمِيعُ النَّاسِ وَالْبَهَائِمِ الَّذِينَ يُوجَدُونَ فِي الْحَقْلِ وَلا يُجْمَعُونَ الَى الْبُيُوتِ يَنْزِلُ عَلَيْهِمِ الْبَرَدُ فَيَمُوتُونَ».
\par 20 فَالَّذِي خَافَ كَلِمَةَ الرَّبِّ مِنْ عَبِيدِ فِرْعَوْنَ هَرَبَ بِعَبِيدِهِ وَمَوَاشِيهِ الَى الْبُيُوتِ.
\par 21 وَامَّا الَّذِي لَمْ يُوَجِّهْ قَلْبَهُ الَى كَلِمَةِ الرَّبِّ فَتَرَكَ عَبِيدَهُ وَمَوَاشِيَهُ فِي الْحَقْلِ.
\par 22 ثُمَّ قَالَ الرَّبُّ لِمُوسَى: «مُدَّ يَدَكَ نَحْوَ السَّمَاءِ لِيَكُونَ بَرَدٌ فِي كُلِّ ارْضِ مِصْرَ عَلَى النَّاسِ وَعَلَى الْبَهَائِمِ وَعَلَى كُلِّ عُشْبِ الْحَقْلِ فِي ارْضِ مِصْرَ».
\par 23 فَمَدَّ مُوسَى عَصَاهُ نَحْوَ السَّمَاءِ فَاعْطَى الرَّبُّ رُعُودا وَبَرَدا وَجَرَتْ نَارٌ عَلَى الارْضِ وَامْطَرَ الرَّبُّ بَرَدا عَلَى ارْضِ مِصْرَ.
\par 24 فَكَانَ بَرَدٌ وَنَارٌ مُتَوَاصِلَةٌ فِي وَسَطِ الْبَرَدِ. شَيْءٌ عَظِيمٌ جِدّا لَمْ يَكُنْ مِثْلُهُ فِي كُلِّ ارْضِ مِصْرَ مُنْذُ صَارَتْ امَّةً!
\par 25 فَضَرَبَ الْبَرَدُ فِي كُلِّ ارْضِ مِصْرَ جَمِيعَ مَا فِي الْحَقْلِ مِنَ النَّاسِ وَالْبَهَائِمِ. وَضَرَبَ الْبَرَدُ جَمِيعَ عُشْبِ الْحَقْلِ وَكَسَّرَ جَمِيعَ شَجَرِ الْحَقْلِ
\par 26 الا ارْضَ جَاسَانَ حَيْثُ كَانَ بَنُو اسْرَائِيلَ فَلَمْ يَكُنْ فِيهَا بَرَدٌ.
\par 27 فَارْسَلَ فِرْعَوْنُ وَدَعَا مُوسَى وَهَارُونَ وَقَالَ لَهُمَا: «اخْطَاتُ هَذِهِ الْمَرَّةَ. الرَّبُّ هُوَ الْبَارُّ وَانَا وَشَعْبِي الاشْرَارُ.
\par 28 صَلِّيَا الَى الرَّبِّ وَكَفَى حُدُوثُ رُعُودِ اللهِ وَالْبَرَدُ فَاطْلِقَكُمْ وَلا تَعُودُوا تَلْبَثُونَ».
\par 29 فَقَالَ لَهُ مُوسَى: «عِنْدَ خُرُوجِي مِنَ الْمَدِينَةِ ابْسِطُ يَدَيَّ الَى الرَّبِّ فَتَنْقَطِعُ الرُّعُودُ وَلا يَكُونُ الْبَرَدُ ايْضا لِتَعْرِفَ انَّ لِلرَّبِّ الارْضَ.
\par 30 وَامَّا انْتَ وَعَبِيدُكَ فَانَا اعْلَمُ انَّكُمْ لَمْ تَخْشُوا بَعْدُ مِنَ الرَّبِّ الْالَهِ».
\par 31 فَالْكَتَّانُ وَالشَّعِيرُ ضُرِبَا. لانَّ الشَّعِيرَ كَانَ مُسْبِلا وَالْكَتَّانُ مُبْزِرا.
\par 32 وَامَّا الْحِنْطَةُ وَالْقَطَانِيُّ فَلَمْ تُضْرَبْ لانَّهَا كَانَتْ مُتَاخِّرَةً.
\par 33 فَخَرَجَ مُوسَى مِنَ الْمَدِينَةِ مِنْ لَدُنْ فِرْعَوْنَ وَبَسَطَ يَدَيْهِ الَى الرَّبِّ فَانْقَطَعَتِ الرُّعُودُ وَالْبَرَدُ وَلَمْ يَنْصَبَّ الْمَطَرُ عَلَى الارْضِ.
\par 34 وَلَكِنْ فِرْعَوْنُ لَمَّا رَاى انَّ الْمَطَرَ وَالْبَرَدَ وَالرُّعُودَ انْقَطَعَتْ عَادَ يُخْطِئُ وَاغْلَظَ قَلْبَهُ هُوَ وَعَبِيدُهُ.
\par 35 فَاشْتَدَّ قَلْبُ فِرْعَوْنَ فَلَمْ يُطْلِقْ بَنِي اسْرَائِيلَ كَمَا تَكَلَّمَ الرَّبُّ عَنْ يَدِ مُوسَى.

\chapter{10}

\par 1 ثُمَّ قَالَ الرَّبُّ لِمُوسَى: «ادْخُلْ الَى فِرْعَوْنَ فَانِّي اغْلَظْتُ قَلْبَهُ وَقُلُوبَ عَبِيدِهِ لاصْنَعَ ايَاتِي هَذِهِ بَيْنَهُمْ.
\par 2 وَلِتُخْبِرَ فِي مَسَامِعِ ابْنِكَ وَابْنِ ابْنِكَ بِمَا فَعَلْتُهُ فِي مِصْرَ وَبِايَاتِي الَّتِي صَنَعْتُهَا بَيْنَهُمْ فَتَعْلَمُونَ انِّي انَا الرَّبُّ».
\par 3 فَدَخَلَ مُوسَى وَهَارُونُ الَى فِرْعَوْنَ وَقَالا لَهُ: «هَكَذَا يَقُولُ الرَّبُّ الَهُ الْعِبْرَانِيِّينَ الَى مَتَى تَابَى انْ تَخْضَعَ لِي؟ اطْلِقْ شَعْبِي لِيَعْبُدُونِي.
\par 4 فَانَّهُ انْ كُنْتَ تَابَى انْ تُطْلِقَ شَعْبِي هَا انَا اجِيءُ غَدا بِجَرَادٍ عَلَى تُخُومِكَ
\par 5 فَيُغَطِّي وَجْهَ الارْضِ حَتَّى لا يُسْتَطَاعَ نَظَرُ الارْضِ. وَيَاكُلُ الْفَضْلَةَ السَّالِمَةَ الْبَاقِيَةَ لَكُمْ مِنَ الْبَرَدِ. وَيَاكُلُ جَمِيعَ الشَّجَرِ النَّابِتِ لَكُمْ مِنَ الْحَقْلِ.
\par 6 وَيَمْلَا بُيُوتَكَ وَبُيُوتَ جَمِيعِ عَبِيدِكَ وَبُيُوتَ جَمِيعِ الْمِصْرِيِّينَ الامْرُ الَّذِي لَمْ يَرَهُ ابَاؤُكَ وَلا ابَاءُ ابَائِكَ مُنْذُ يَوْمَ وُجِدُوا عَلَى الارْضِ الَى هَذَا الْيَوْمِ». ثُمَّ تَحَوَّلَ وَخَرَجَ مِنْ لَدُنْ فِرْعَوْنَ.
\par 7 فَقَالَ عَبِيدُ فِرْعَوْنَ لَهُ: «الَى مَتَى يَكُونُ هَذَا لَنَا فَخّا؟ اطْلِقِ الرِّجَالَ لِيَعْبُدُوا الرَّبَّ الَهَهُمْ. الَمْ تَعْلَمْ بَعْدُ انَّ مِصْرَ قَدْ خَرِبَتْ؟»
\par 8 فَرُدَّ مُوسَى وَهَارُونُ الَى فِرْعَوْنَ. فَقَالَ لَهُمَا: «اذْهَبُوا اعْبُدُوا الرَّبَّ الَهَكُمْ. وَلَكِنْ مَنْ وَمَنْ هُمُ الَّذِينَ يَذْهَبُونَ؟»
\par 9 فَقَالَ مُوسَى: «نَذْهَبُ بِفِتْيَانِنَا وَشُيُوخِنَا. نَذْهَبُ بِبَنِينَا وَبَنَاتِنَا بِغَنَمِنَا وَبَقَرِنَا. لانَّ لَنَا عِيدا لِلرَّبِّ».
\par 10 فَقَالَ لَهُمَا: «يَكُونُ الرَّبُّ مَعَكُمْ هَكَذَا كَمَا اطْلِقُكُمْ وَاوْلادَكُمُ. انْظُرُوا انَّ قُدَّامَ وُجُوهِكُمْ شَرّا.
\par 11 لَيْسَ هَكَذَا. اذْهَبُوا انْتُمُ الرِّجَالَ وَاعْبُدُوا الرَّبَّ. لانَّكُمْ لِهَذَا طَالِبُونَ». فَطُرِدَا مِنْ لَدُنْ فِرْعَوْنَ.
\par 12 ثُمَّ قَالَ الرَّبُّ لِمُوسَى: «مُدَّ يَدَكَ عَلَى ارْضِ مِصْرَ لاجْلِ الْجَرَادِ لِيَصْعَدَ عَلَى ارْضِ مِصْرَ وَيَاكُلَ كُلَّ عُشْبِ الارْضِ كُلَّ مَا تَرَكَهُ الْبَرَدُ».
\par 13 فَمَدَّ مُوسَى عَصَاهُ عَلَى ارْضِ مِصْرَ فَجَلَبَ الرَّبُّ عَلَى الارْضِ رِيحا شَرْقِيَّةً كُلَّ ذَلِكَ النَّهَارِ وَكُلَّ اللَّيْلِ. وَلَمَّا كَانَ الصَّبَاحُ حَمَلَتِ الرِّيحُ الشَّرْقِيَّةُ الْجَرَادَ
\par 14 فَصَعِدَ الْجَرَادُ عَلَى كُلِّ ارْضِ مِصْرَ وَحَلَّ فِي جَمِيعِ تُخُومِ مِصْرَ. شَيْءٌ ثَقِيلٌ جِدّا لَمْ يَكُنْ قَبْلَهُ جَرَادٌ هَكَذَا مِثْلَهُ وَلا يَكُونُ بَعْدَهُ كَذَلِكَ
\par 15 وَغَطَّى وَجْهَ كُلِّ الارْضِ حَتَّى اظْلَمَتِ الارْضُ. وَاكَلَ جَمِيعَ عُشْبِ الارْضِ وَجَمِيعَ ثَمَرِ الشَّجَرِ الَّذِي تَرَكَهُ الْبَرَدُ حَتَّى لَمْ يَبْقَ شَيْءٌ اخْضَرُ فِي الشَّجَرِ وَلا فِي عُشْبِ الْحَقْلِ فِي كُلِّ ارْضِ مِصْرَ».
\par 16 فَدَعَا فِرْعَوْنُ مُوسَى وَهَارُونَ مُسْرِعا وَقَالَ: «اخْطَاتُ الَى الرَّبِّ الَهِكُمَا وَالَيْكُمَا.
\par 17 وَالانَ اصْفَحَا عَنْ خَطِيَّتِي هَذِهِ الْمَرَّةَ فَقَطْ وَصَلِّيَا الَى الرَّبِّ الَهِكُمَا لِيَرْفَعَ عَنِّي هَذَا الْمَوْتَ فَقَطْ».
\par 18 فَخَرَجَ مُوسَى مِنْ لَدُنْ فِرْعَوْنَ وَصَلَّى الَى الرَّبِّ.
\par 19 فَرَدَّ الرَّبُّ رِيحا غَرْبِيَّةً شَدِيدَةً جِدّا فَحَمَلَتِ الْجَرَادَ وَطَرَحَتْهُ الَى بَحْرِ سُوفَ. لَمْ تَبْقَ جَرَادَةٌ وَاحِدَةٌ فِي كُلِّ تُخُومِ مِصْرَ.
\par 20 وَلَكِنْ شَدَّدَ الرَّبُّ قَلْبَ فِرْعَوْنَ فَلَمْ يُطْلِقْ بَنِي اسْرَائِيلَ.
\par 21 ثُمَّ قَالَ الرَّبُّ لِمُوسَى: «مُدَّ يَدَكَ نَحْوَ السَّمَاءِ لِيَكُونَ ظَلامٌ عَلَى ارْضِ مِصْرَ حَتَّى يُلْمَسُ الظَّلامُ».
\par 22 فَمَدَّ مُوسَى يَدَهُ نَحْوَ السَّمَاءِ فَكَانَ ظَلامٌ دَامِسٌ فِي كُلِّ ارْضِ مِصْرَ ثَلاثَةَ ايَّامٍ.
\par 23 لَمْ يُبْصِرْ احَدٌ اخَاهُ وَلا قَامَ احَدٌ مِنْ مَكَانِهِ ثَلاثَةَ ايَّامٍ. وَلَكِنْ جَمِيعُ بَنِي اسْرَائِيلَ كَانَ لَهُمْ نُورٌ فِي مَسَاكِنِهِمْ!
\par 24 فَدَعَا فِرْعَوْنُ مُوسَى وَقَالَ: «اذْهَبُوا اعْبُدُوا الرَّبَّ. غَيْرَ انَّ غَنَمَكُمْ وَبَقَرَكُمْ تَبْقَى. اوْلادُكُمْ ايْضا تَذْهَبُ مَعَكُمْ».
\par 25 فَقَالَ مُوسَى: «انْتَ تُعْطِي ايْضا فِي ايْدِينَا ذَبَائِحَ وَمُحْرَقَاتٍ لِنُقَرِّبُهَا لِلرَّبِّ الَهِنَا
\par 26 فَتَذْهَبُ مَوَاشِينَا ايْضا مَعَنَا. لا يَبْقَى ظِلْفٌ. لانَّنَا مِنْهَا نَاخُذُ لِعِبَادَةِ الرَّبِّ الَهِنَا. وَنَحْنُ لا نَعْرِفُ بِمَاذَا نَعْبُدُ الرَّبَّ حَتَّى نَاتِيَ الَى هُنَاكَ».
\par 27 وَلَكِنْ شَدَّدَ الرَّبُّ قَلْبَ فِرْعَوْنَ فَلَمْ يَشَا انْ يُطْلِقَهُمْ.
\par 28 وَقَالَ لَهُ فِرْعَوْنُ: «اذْهَبْ عَنِّي. احْتَرِزْ. لا تَرَ وَجْهِي ايْضا. انَّكَ يَوْمَ تَرَى وَجْهِي تَمُوتُ».
\par 29 فَقَالَ مُوسَى: «نِعِمَّا قُلْتَ! انَا لا اعُودُ ارَى وَجْهَكَ ايْضا».

\chapter{11}

\par 1 ثُمَّ قَالَ الرَّبُّ لِمُوسَى: «ضَرْبَةً وَاحِدَةً ايْضا اجْلِبُ عَلَى فِرْعَوْنَ وَعَلَى مِصْرَ. بَعْدَ ذَلِكَ يُطْلِقُكُمْ مِنْ هُنَا. وَعِنْدَمَا يُطْلِقُكُمْ يَطْرُدُكُمْ طَرْدا مِنْ هُنَا بِالتَّمَامِ.
\par 2 تَكَلَّمْ فِي مَسَامِعِ الشَّعْبِ انْ يَطْلُبَ كُلُّ رَجُلٍ مِنْ صَاحِبِهِ وَكُلُّ امْرَاةٍ مِنْ صَاحِبَتِهَا امْتِعَةَ فِضَّةٍ وَامْتِعَةَ ذَهَبٍ».
\par 3 وَاعْطَى الرَّبُّ نِعْمَةً لِلشَّعْبِ فِي عُيُونِ الْمِصْرِيِّينَ. وَايْضا مُوسَى كَانَ عَظِيما جِدّا فِي ارْضِ مِصْرَ فِي عُِيُونِ عَبِيدِ فِرْعَوْنَ وَعُيُونِ الشَّعْبِ.
\par 4 وَقَالَ مُوسَى: «هَكَذَا يَقُولُ الرَّبُّ انِّي نَحْوَ نِصْفِ اللَّيْلِ اخْرُجُ فِي وَسَطِ مِصْرَ
\par 5 فَيَمُوتُ كُلُّ بِكْرٍ فِي ارْضِ مِصْرَ مِنْ بِكْرِ فِرْعَوْنَ الْجَالِسِ عَلَى كُرْسِيِّهِ الَى بِكْرِ الْجَارِيَةِ الَّتِي خَلْفَ الرَّحَى وَكُلُّ بِكْرِ بَهِيمَةٍ.
\par 6 وَيَكُونُ صُرَاخٌ عَظِيمٌ فِي كُلِّ ارْضِ مِصْرَ لَمْ يَكُنْ مِثْلُهُ وَلا يَكُونُ مِثْلُهُ ايْضا.
\par 7 وَلَكِنْ جَمِيعُ بَنِي اسْرَائِيلَ لا يُسَنِّنُ كَلْبٌ لِسَانَهُ الَيْهِمْ لا الَى النَّاسِ وَلا الَى الْبَهَائِمِ. لِكَيْ تَعْلَمُوا انَّ الرَّبَّ يُمَيِّزُ بَيْنَ الْمِصْرِيِّينَ وَاسْرَائِيلَ.
\par 8 فَيَنْزِلُ الَيَّ جَمِيعُ عَبِيدِكَ هَؤُلاءِ وَيَسْجُدُونَ لِي قَائِلِينَ: اخْرُجْ انْتَ وَجَمِيعُ الشَّعْبِ الَّذِينَ فِي اثَرِكَ. وَبَعْدَ ذَلِكَ اخْرُجُ». ثُمَّ خَرَجَ مِنْ لَدُنْ فِرْعَوْنَ فِي حُمُوِّ الْغَضَبِ.
\par 9 وَقَالَ الرَّبُّ لِمُوسَى: «لا يَسْمَعُ لَكُمَا فِرْعَوْنُ لِتَكْثُرَ عَجَائِبِي فِي ارْضِ مِصْرَ».
\par 10 وَكَانَ مُوسَى وَهَارُونُ يَفْعَلانِ كُلَّ هَذِهِ الْعَجَائِبِ امَامَ فِرْعَوْنَ. وَلَكِنْ شَدَّدَ الرَّبُّ قَلْبَ فِرْعَوْنَ فَلَمْ يُطْلِقْ بَنِي اسْرَائِيلَ مِنْ ارْضِهِ.

\chapter{12}

\par 1 وَقَالَ الرَّبُّ لِمُوسَى وَهَارُونَ فِي ارْضِ مِصْرَ:
\par 2 «هَذَا الشَّهْرُ يَكُونُ لَكُمْ رَاسَ الشُّهُورِ. هُوَ لَكُمْ اوَّلُ شُهُورِ السَّنَةِ.
\par 3 كَلِّمَا كُلَّ جَمَاعَةِ اسْرَائِيلَ قَائِلَيْنِ فِي الْعَاشِرِ مِنْ هَذَا الشَّهْرِ يَاخُذُونَ لَهُمْ كُلُّ وَاحِدٍ شَاةً بِحَسَبِ بُيُوتِ الابَاءِ. شَاةً لِلْبَيْتِ.
\par 4 وَانْ كَانَ الْبَيْتُ صَغِيرا عَنْ انْ يَكُونَ كُفْوا لِشَاةٍ يَاخُذُ هُوَ وَجَارُهُ الْقَرِيبُ مِنْ بَيْتِهِ بِحَسَبِ عَدَدِ النُّفُوسِ. كُلُّ وَاحِدٍ عَلَى حَسَبِ اكْلِهِ تَحْسِبُونَ لِلشَّاةِ.
\par 5 تَكُونُ لَكُمْ شَاةً صَحِيحَةً ذَكَرا ابْنَ سَنَةٍ تَاخُذُونَهُ مِنَ الْخِرْفَانِ اوْ مِنَ الْمَوَاعِزِ.
\par 6 وَيَكُونُ عِنْدَكُمْ تَحْتَ الْحِفْظِ الَى الْيَوْمِ الرَّابِعَ عَشَرَ مِنْ هَذَا الشَّهْرِ. ثُمَّ يَذْبَحُهُ كُلُّ جُمْهُورِ جَمَاعَةِ اسْرَائِيلَ فِي الْعَشِيَّةِ.
\par 7 وَيَاخُذُونَ مِنَ الدَّمِ وَيَجْعَلُونَهُ عَلَى الْقَائِمَتَيْنِ وَالْعَتَبَةِ الْعُلْيَا فِي الْبُيُوتِ الَّتِي يَاكُلُونَهُ فِيهَا.
\par 8 وَيَاكُلُونَ اللَّحْمَ تِلْكَ اللَّيْلَةَ مَشْوِيّا بِالنَّارِ مَعَ فَطِيرٍ. عَلَى اعْشَابٍ مُرَّةٍ يَاكُلُونَهُ.
\par 9 لا تَاكُلُوا مِنْهُ نَيْئا اوْ طَبِيخا مَطْبُوخا بِالْمَاءِ بَلْ مَشْوِيّا بِالنَّارِ. رَاسَهُ مَعَ اكَارِعِهِ وَجَوْفِهِ.
\par 10 وَلا تُبْقُوا مِنْهُ الَى الصَّبَاحِ. وَالْبَاقِي مِنْهُ الَى الصَّبَاحِ تُحْرِقُونَهُ بِالنَّارِ.
\par 11 وَهَكَذَا تَاكُلُونَهُ: احْقَاؤُكُمْ مَشْدُودَةٌ وَاحْذِيَتُكُمْ فِي ارْجُلِكُمْ وَعِصِيُّكُمْ فِي ايْدِيكُمْ. وَتَاكُلُونَهُ بِعَجَلَةٍ. هُوَ فِصْحٌ لِلرَّبِّ.
\par 12 فَانِّي اجْتَازُ فِي ارْضِ مِصْرَ هَذِهِ اللَّيْلَةَ وَاضْرِبُ كُلَّ بِكْرٍ فِي ارْضِ مِصْرَ مِنَ النَّاسِ وَالْبَهَائِمِ. وَاصْنَعُ احْكَاما بِكُلِّ الِهَةِ الْمِصْرِيِّينَ. انَا الرَّبُّ.
\par 13 وَيَكُونُ لَكُمُ الدَّمُ عَلامَةً عَلَى الْبُيُوتِ الَّتِي انْتُمْ فِيهَا فَارَى الدَّمَ وَاعْبُرُ عَنْكُمْ فَلا يَكُونُ عَلَيْكُمْ ضَرْبَةٌ لِلْهَلاكِ حِينَ اضْرِبُ ارْضَ مِصْرَ.
\par 14 وَيَكُونُ لَكُمْ هَذَا الْيَوْمُ تَذْكَارا فَتُعَيِّدُونَهُ عِيدا لِلرَّبِّ. فِي اجْيَالِكُمْ تُعَيِّدُونَهُ فَرِيضَةً ابَدِيَّةً.
\par 15 «سَبْعَةَ ايَّامٍ تَاكُلُونَ فَطِيرا. الْيَوْمَ الاوَّلَ تَعْزِلُونَ الْخَمِيرَ مِنْ بُيُوتِكُمْ فَانَّ كُلَّ مَنْ اكَلَ خَمِيرا مِنَ الْيَوْمِ الاوَّلِ الَى الْيَوْمِ السَّابِعِ تُقْطَعُ تِلْكَ النَّفْسُ مِنْ اسْرَائِيلَ.
\par 16 وَيَكُونُ لَكُمْ فِي الْيَوْمِ الاوَّلِ مَحْفَلٌ مُقَدَّسٌ وَفِي الْيَوْمِ السَّابِعِ مَحْفَلٌ مُقَدَّسٌ. لا يُعْمَلُ فِيهِمَا عَمَلٌ مَا الا مَا تَاكُلُهُ كُلُّ نَفْسٍ فَذَلِكَ وَحْدَهُ يُعْمَلُ مِنْكُمْ.
\par 17 وَتَحْفَظُونَ الْفَطِيرَ لانِّي فِي هَذَا الْيَوْمِ عَيْنِهِ اخْرَجْتُ اجْنَادَكُمْ مِنْ ارْضِ مِصْرَ فَتَحْفَظُونَ هَذَا الْيَوْمَ فِي اجْيَالِكُمْ فَرِيضَةً ابَدِيَّةً.
\par 18 فِي الشَّهْرِ الاوَّلِ فِي الْيَوْمِ الرَّابِعَ عَشَرَ مِنَ الشَّهْرِ مَسَاءً تَاكُلُونَ فَطِيرا الَى الْيَوْمِ الْحَادِي وَالْعِشْرِينَ مِنَ الشَّهْرِ مَسَاءً.
\par 19 سَبْعَةَ ايَّامٍ لا يُوجَدْ خَمِيرٌ فِي بُيُوتِكُمْ. فَانَّ كُلَّ مَنْ اكَلَ مُخْتَمِرا تُقْطَعُ تِلْكَ النَّفْسُ مِنْ جَمَاعَةِ اسْرَائِيلَ الْغَرِيبُ مَعَ مَوْلُودِ الارْضِ.
\par 20 لا تَاكُلُوا شَيْئا مُخْتَمِرا. فِي جَمِيعِ مَسَاكِنِكُمْ تَاكُلُونَ فَطِيرا».
\par 21 فَدَعَا مُوسَى جَمِيعَ شُيُوخِ اسْرَائِيلَ وَقَالَ لَهُمُ: «اسْحَبُوا وَخُذُوا لَكُمْ غَنَما بِحَسَبِ عَشَائِرِكُمْ وَاذْبَحُوا الْفِصْحَ.
\par 22 وَخُذُوا بَاقَةَ زُوفَا وَاغْمِسُوهَا فِي الدَّمِ الَّذِي فِي الطَّسْتِ وَمُسُّوا الْعَتَبَةَ الْعُلْيَا وَالْقَائِمَتَيْنِ بِالدَّمِ الَّذِي فِي الطَّسْتِ. وَانْتُمْ لا يَخْرُجْ احَدٌ مِنْكُمْ مِنْ بَابِ بَيْتِهِ حَتَّى الصَّبَاحِ
\par 23 فَانَّ الرَّبَّ يَجْتَازُ لِيَضْرِبَ الْمِصْرِيِّينَ. فَحِينَ يَرَى الدَّمَ عَلَى الْعَتَبَةِ الْعُلْيَا وَالْقَائِمَتَيْنِ يَعْبُرُ الرَّبُّ عَنِ الْبَابِ وَلا يَدَعُ الْمُهْلِكَ يَدْخُلُ بُيُوتَكُمْ لِيَضْرِبَ.
\par 24 فَتَحْفَظُونَ هَذَا الامْرَ فَرِيضَةً لَكَ وَلاوْلادِكَ الَى الابَدِ.
\par 25 وَيَكُونُ حِينَ تَدْخُلُونَ الارْضَ الَّتِي يُعْطِيكُمُ الرَّبُّ كَمَا تَكَلَّمَ انَّكُمْ تَحْفَظُونَ هَذِهِ الْخِدْمَةَ.
\par 26 وَيَكُونُ حِينَ يَسْالُكُمْ اوْلادُكُمْ: مَا هَذِهِ الْخِدْمَةُ لَكُمْ؟
\par 27 تَقُولُونَ: هِيَ ذَبِيحَةُ فِصْحٍ لِلرَّبِّ الَّذِي عَبَرَ عَنْ بُيُوتِ بَنِي اسْرَائِيلَ فِي مِصْرَ لَمَّا ضَرَبَ الْمِصْرِيِّينَ وَخَلَّصَ بُيُوتَنَا». فَخَرَّ الشَّعْبُ وَسَجَدُوا.
\par 28 وَمَضَى بَنُو اسْرَائِيلَ وَفَعَلُوا كَمَا امَرَ الرَّبُّ مُوسَى وَهَارُونَ. هَكَذَا فَعَلُوا.
\par 29 فَحَدَثَ فِي نِصْفِ اللَّيْلِ انَّ الرَّبَّ ضَرَبَ كُلَّ بِكْرٍ فِي ارْضِ مِصْرَ مِنْ بِكْرِ فِرْعَوْنَ الْجَالِسِ عَلَى كُرْسِيِّهِ الَى بِكْرِ الاسِيرِ الَّذِي فِي السِّجْنِ وَكُلَّ بِكْرِ بَهِيمَةٍ.
\par 30 فَقَامَ فِرْعَوْنُ لَيْلا هُوَ وَكُلُّ عَبِيدِهِ وَجَمِيعُ الْمِصْرِيِّينَ. وَكَانَ صُرَاخٌ عَظِيمٌ فِي مِصْرَ لانَّهُ لَمْ يَكُنْ بَيْتٌ لَيْسَ فِيهِ مَيِّتٌ.
\par 31 فَدَعَا مُوسَى وَهَارُونَ لَيْلا وَقَالَ: «قُومُوا اخْرُجُوا مِنْ بَيْنِ شَعْبِي انْتُمَا وَبَنُو اسْرَائِيلَ جَمِيعا وَاذْهَبُوا اعْبُدُوا الرَّبَّ كَمَا تَكَلَّمْتُمْ.
\par 32 خُذُوا غَنَمَكُمْ ايْضا وَبَقَرَكُمْ كَمَا تَكَلَّمْتُمْ وَاذْهَبُوا. وَبَارِكُونِي ايْضا».
\par 33 وَالَحَّ الْمِصْرِيُّونَ عَلَى الشَّعْبِ لِيُطْلِقُوهُمْ عَاجِلا مِنَ الارْضِ لانَّهُمْ قَالُوا: «جَمِيعُنَا امْوَاتٌ».
\par 34 فَحَمَلَ الشَّعْبُ عَجِينَهُمْ قَبْلَ انْ يَخْتَمِرَ وَمَعَاجِنُهُمْ مَصْرُورَةٌ فِي ثِيَابِهِمْ عَلَى اكْتَافِهِمْ.
\par 35 وَفَعَلَ بَنُو اسْرَائِيلَ بِحَسَبِ قَوْلِ مُوسَى. طَلَبُوا مِنَ الْمِصْرِيِّينَ امْتِعَةَ فِضَّةٍ وَامْتِعَةَ ذَهَبٍ وَثِيَابا.
\par 36 وَاعْطَى الرَّبُّ نِعْمَةً لِلشَّعْبِ فِي عُِيُونِ الْمِصْرِيِّينَ حَتَّى اعَارُوهُمْ. فَسَلَبُوا الْمِصْرِيِّينَ.
\par 37 فَارْتَحَلَ بَنُو اسْرَائِيلَ مِنْ رَعَمْسِيسَ الَى سُكُّوتَ نَحْوَ سِتِّ مِئَةِ الْفِ مَاشٍ مِنَ الرِّجَالِ عَدَا الاوْلادِ.
\par 38 وَصَعِدَ مَعَهُمْ لَفِيفٌ كَثِيرٌ ايْضا مَعَ غَنَمٍ وَبَقَرٍ مَوَاشٍ وَافِرَةٍ جِدّا.
\par 39 وَخَبَزُوا الْعَجِينَ الَّذِي اخْرَجُوهُ مِنْ مِصْرَ خُبْزَ مَلَّةٍ فَطِيرا اذْ كَانَ لَمْ يَخْتَمِرْ. لانَّهُمْ طُرِدُوا مِنْ مِصْرَ وَلَمْ يَقْدِرُوا انْ يَتَاخَّرُوا. فَلَمْ يَصْنَعُوا لانْفُسِهِمْ زَادا.
\par 40 وَامَّا اقَامَةُ بَنِي اسْرَائِيلَ الَّتِي اقَامُوهَا فِي مِصْرَ فَكَانَتْ ارْبَعَ مِئَةٍ وَثَلاثِينَ سَنَةً.
\par 41 وَكَانَ عِنْدَ نِهَايَةِ ارْبَعِ مِئَةٍ وَثَلاثِينَ سَنَةً فِي ذَلِكَ الْيَوْمِ عَيْنِهِ انَّ جَمِيعَ اجْنَادِ الرَّبِّ خَرَجَتْ مِنْ ارْضِ مِصْرَ.
\par 42 هِيَ لَيْلَةٌ تُحْفَظُ لِلرَّبِّ لاخْرَاجِهِ ايَّاهُمْ مِنْ ارْضِ مِصْرَ. هَذِهِ اللَّيْلَةُ هِيَ لِلرَّبِّ. تُحْفَظُ مِنْ جَمِيعِ بَنِي اسْرَائِيلَ فِي اجْيَالِهِمْ.
\par 43 وَقَالَ الرَّبُّ لِمُوسَى وَهَارُونَ: «هَذِهِ فَرِيضَةُ الْفِصْحِ: كُلُّ ابْنِ غَرِيبٍ لا يَاكُلُ مِنْهُ.
\par 44 وَلَكِنْ كُلُّ عَبْدٍ مُبْتَاعٍ بِفِضَّةٍ تَخْتِنُهُ ثُمَّ يَاكُلُ مِنْهُ.
\par 45 النَّزِيلُ وَالاجِيرُ لا يَاكُلانِ مِنْهُ.
\par 46 فِي بَيْتٍ وَاحِدٍ يُؤْكَلُ. لا تُخْرِجْ مِنَ اللَّحْمِ مِنَ الْبَيْتِ الَى خَارِجٍ وَعَظْما لا تَكْسِرُوا مِنْهُ.
\par 47 كُلُّ جَمَاعَةِ اسْرَائِيلَ يَصْنَعُونَهُ.
\par 48 وَاذَا نَزَلَ عِنْدَكَ نَزِيلٌ وَصَنَعَ فِصْحا لِلرَّبِّ فَلْيُخْتَنْ مِنْهُ كُلُّ ذَكَرٍ ثُمَّ يَتَقَدَّمُ لِيَصْنَعَهُ فَيَكُونُ كَمَوْلُودِ الارْضِ. وَامَّا كُلُّ اغْلَفَ فَلا يَاكُلُ مِنْهُ.
\par 49 تَكُونُ شَرِيعَةٌ وَاحِدَةٌ لِمَوْلُودِ الارْضِ وَلِلنَّزِيلِ النَّازِلِ بَيْنَكُمْ».
\par 50 فَفَعَلَ جَمِيعُ بَنِي اسْرَائِيلَ كَمَا امَرَ الرَّبُّ مُوسَى وَهَارُونَ. هَكَذَا فَعَلُوا.
\par 51 وَكَانَ فِي ذَلِكَ الْيَوْمِ عَيْنِهِ انَّ الرَّبَّ اخْرَجَ بَنِي اسْرَائِيلَ مِنْ ارْضِ مِصْرَ بِحَسَبِ اجْنَادِهِمْ.

\chapter{13}

\par 1 وَقَالَ الرَّبُّ لِمُوسَى:
\par 2 «قَدِّسْ لِي كُلَّ بِكْرٍ كُلَّ فَاتِحِ رَحِمٍ مِنْ بَنِي اسْرَائِيلَ مِنَ النَّاسِ وَمِنَ الْبَهَائِمِ. انَّهُ لِي».
\par 3 وَقَالَ مُوسَى لِلشَّعْبِ: «اذْكُرُوا هَذَا الْيَوْمَ الَّذِي فِيهِ خَرَجْتُمْ مِنْ مِصْرَ مِنْ بَيْتِ الْعُبُودِيَّةِ فَانَّهُ بِيَدٍ قَوِيَّةٍ اخْرَجَكُمُ الرَّبُّ مِنْ هُنَا. وَلا يُؤْكَلُ خَمِيرٌ.
\par 4 الْيَوْمَ انْتُمْ خَارِجُونَ فِي شَهْرِ ابِيبَ.
\par 5 وَيَكُونُ مَتَى ادْخَلَكَ الرَّبُّ ارْضَ الْكَنْعَانِيِّينَ وَالْحِثِّيِّينَ وَالامُورِيِّينَ وَالْحِوِّيِّينَ وَالْيَبُوسِيِّينَ الَّتِي حَلَفَ لِابَائِكَ انْ يُعْطِيَكَ ارْضا تَفِيضُ لَبَنا وَعَسَلا انَّكَ تَصْنَعُ هَذِهِ الْخِدْمَةَ فِي هَذَا الشَّهْرِ.
\par 6 سَبْعَةَ ايَّامٍ تَاكُلُ فَطِيرا وَفِي الْيَوْمِ السَّابِعِ عِيدٌ لِلرَّبِّ.
\par 7 فَطِيرٌ يُؤْكَلُ السَّبْعَةَ الايَّامِ وَلا يُرَى عِنْدَكَ مُخْتَمِرٌ وَلا يُرَى عِنْدَكَ خَمِيرٌ فِي جَمِيعِ تُخُومِكَ.
\par 8 «وَتُخْبِرُ ابْنَكَ فِي ذَلِكَ الْيَوْمِ قَائِلا: مِنْ اجْلِ مَا صَنَعَ الَيَّ الرَّبُّ حِينَ اخْرَجَنِي مِنْ مِصْرَ.
\par 9 وَيَكُونُ لَكَ عَلامَةً عَلَى يَدِكَ وَتَذْكَارا بَيْنَ عَيْنَيْكَ لِتَكُونَ شَرِيعَةُ الرَّبِّ فِي فَمِكَ. لانَّهُ بِيَدٍ قَوِيَّةٍ اخْرَجَكَ الرَّبُّ مِنْ مِصْرَ.
\par 10 فَتَحْفَظُ هَذِهِ الْفَرِيضَةَ فِي وَقْتِهَا مِنْ سَنَةٍ الَى سَنَةٍ.
\par 11 «وَيَكُونُ مَتَى ادْخَلَكَ الرَّبُّ ارْضَ الْكَنْعَانِيِّينَ كَمَا حَلَفَ لَكَ وَلِابَائِكَ وَاعْطَاكَ ايَّاهَا
\par 12 انَّكَ تُقَدِّمُ لِلرَّبِّ كُلَّ فَاتِحِ رَحِمٍ وَكُلَّ بِكْرٍ مِنْ نِتَاجِ الْبَهَائِمِ الَّتِي تَكُونُ لَكَ. الذُّكُورُ لِلرَّبِّ.
\par 13 وَلَكِنَّ كُلَّ بِكْرِ حِمَارٍ تَفْدِيهِ بِشَاةٍ. وَانْ لَمْ تَفْدِهِ فَتَكْسِرُ عُنُقَهُ. وَكُلُّ بِكْرِ انْسَانٍ مِنْ اوْلادِكَ تَفْدِيهِ.
\par 14 «وَيَكُونُ مَتَى سَالَكَ ابْنُكَ غَدا: مَا هَذَا؟ تَقُولُ لَهُ: بِيَدٍ قَوِيَّةٍ اخْرَجَنَا الرَّبُّ مِنْ مِصْرَ مِنْ بَيْتِ الْعُبُودِيَّةِ.
\par 15 وَكَانَ لَمَّا تَقَسَّى فِرْعَوْنُ عَنْ اطْلاقِنَا انَّ الرَّبَّ قَتَلَ كُلَّ بِكْرٍ فِي ارْضِ مِصْرَ مِنْ بِكْرِ النَّاسِ الَى بِكْرِ الْبَهَائِمِ. لِذَلِكَ انَا اذْبَحُ لِلرَّبِّ الذُّكُورَ مِنْ كُلِّ فَاتِحِ رَحِمٍ وَافْدِي كُلَّ بِكْرٍ مِنْ اوْلادِي.
\par 16 فَيَكُونُ عَلامَةً عَلَى يَدِكَ وَعِصَابَةً بَيْنَ عَيْنَيْكَ. لانَّهُ بِيَدٍ قَوِيَّةٍ اخْرَجَنَا الرَّبُّ مِنْ مِصْرَ».
\par 17 وَكَانَ لَمَّا اطْلَقَ فِرْعَوْنُ الشَّعْبَ انَّ اللهَ لَمْ يَهْدِهِمْ فِي طَرِيقِ ارْضِ الْفَلَسْطِينِيِّينَ مَعَ انَّهَا قَرِيبَةٌ لانَّ اللهَ قَالَ: «لِئَلا يَنْدَمَ الشَّعْبُ اذَا رَاوا حَرْبا وَيَرْجِعُوا الَى مِصْرَ».
\par 18 فَادَارَ اللهُ الشَّعْبَ فِي طَرِيقِ بَرِّيَّةِ بَحْرِ سُوفٍ. وَصَعِدَ بَنُو اسْرَائِيلَ مُتَجَهِّزِينَ مِنْ ارْضِ مِصْرَ.
\par 19 وَاخَذَ مُوسَى عِظَامَ يُوسُفَ مَعَهُ لانَّهُ كَانَ قَدِ اسْتَحْلَفَ بَنِي اسْرَائِيلَ بِحَلْفٍ قَائِلا: «انَّ اللهَ سَيَفْتَقِدُكُمْ فَتُصْعِدُونَ عِظَامِي مِنْ هُنَا مَعَكُمْ»
\par 20 وَارْتَحَلُوا مِنْ سُكُّوتَ وَنَزَلُوا فِي ايثَامَ فِي طَرَفِ الْبَرِّيَّةِ.
\par 21 وَكَانَ الرَّبُّ يَسِيرُ امَامَهُمْ نَهَارا فِي عَمُودِ سَحَابٍ لِيَهْدِيَهُمْ فِي الطَّرِيقِ وَلَيْلا فِي عَمُودِ نَارٍ لِيُضِيءَ لَهُمْ - لِكَيْ يَمْشُوا نَهَارا وَلَيْلا.
\par 22 لَمْ يَبْرَحْ عَمُودُ السَّحَابِ نَهَارا وَعَمُودُ النَّارِ لَيْلا مِنْ امَامِ الشَّعْبِ.

\chapter{14}

\par 1 وَقَالَ الرَّبُ لِمُوسَى:
\par 2 «كَلِّمْ بَنِي اسْرَائِيلَ انْ يَرْجِعُوا وَيَنْزِلُوا امَامَ فَمِ الْحِيرُوثِ بَيْنَ مَجْدَلَ وَالْبَحْرِ امَامَ بَعْلَ صَفُونَ. مُقَابِلَهُ تَنْزِلُونَ عِنْدَ الْبَحْرِ.
\par 3 فَيَقُولُ فِرْعَوْنُ عَنْ بَنِي اسْرَائِيلَ: هُمْ مُرْتَبِكُونَ فِي الارْضِ. قَدِ اسْتَغْلَقَ عَلَيْهِمِ الْقَفْرُ.
\par 4 وَاشَدِّدُ قَلْبَ فِرْعَوْنَ حَتَّى يَسْعَى وَرَاءَهُمْ. فَاتَمَجَّدُ بِفِرْعَوْنَ وَبِجَمِيعِ جَيْشِهِ وَيَعْرِفُ الْمِصْرِيُّونَ انِّي انَا الرَّبُّ». فَفَعَلُوا هَكَذَا.
\par 5 فَلَمَّا اخْبِرَ مَلِكُ مِصْرَ انَّ الشَّعْبَ قَدْ هَرَبَ تَغَيَّرَ قَلْبُ فِرْعَوْنَ وَعَبِيدِهِ عَلَى الشَّعْبِ. فَقَالُوا: «مَاذَا فَعَلْنَا حَتَّى اطْلَقْنَا اسْرَائِيلَ مِنْ خِدْمَتِنَا؟»
\par 6 فَشَدَّ مَرْكَبَتَهُ وَاخَذَ قَوْمَهُ مَعَهُ.
\par 7 وَاخَذَ سِتَّ مِئَةِ مَرْكَبَةٍ مُنْتَخَبَةٍ وَسَائِرَ مَرْكَبَاتِ مِصْرَ وَجُنُودا مَرْكَبِيَّةً عَلَى جَمِيعِهَا.
\par 8 وَشَدَّدَ الرَّبُّ قَلْبَ فِرْعَوْنَ مَلِكِ مِصْرَ حَتَّى سَعَى وَرَاءَ بَنِي اسْرَائِيلَ وَبَنُو اسْرَائِيلَ خَارِجُونَ بِيَدٍ رَفِيعَةٍ.
\par 9 فَسَعَى الْمِصْرِيُّونَ وَرَاءَهُمْ وَادْرَكُوهُمْ. جَمِيعُ خَيْلِ مَرْكَبَاتِ فِرْعَوْنَ وَفُرْسَانِهِ وَجَيْشِهِ وَهُمْ نَازِلُونَ عَِنْدَ الْبَحْرِ عَِنْدَ فَمِ الْحِيرُوثِ امَامَ بَعْلَ صَفُونَ.
\par 10 فَلَمَّا اقْتَرَبَ فِرْعَوْنُ رَفَعَ بَنُو اسْرَائِيلَ عُيُونَهُمْ وَاذَا الْمِصْرِيُّونَ رَاحِلُونَ وَرَاءَهُمْ فَفَزِعُوا جِدّا. وَصَرَخَ بَنُو اسْرَائِيلَ الَى الرَّبِّ
\par 11 وَقَالُوا لِمُوسَى: «هَلْ لانَّهُ لَيْسَتْ قُبُورٌ فِي مِصْرَ اخَذْتَنَا لِنَمُوتَ فِي الْبَرِّيَّةِ؟ مَاذَا صَنَعْتَ بِنَا حَتَّى اخْرَجْتَنَا مِنْ مِصْرَ؟
\par 12 الَيْسَ هَذَا هُوَ الْكَلامُ الَّذِي كَلَّمْنَاكَ بِهِ فِي مِصْرَ قَائِلِينَ: كُفَّ عَنَّا فَنَخْدِمَ الْمِصْرِيِّينَ لانَّهُ خَيْرٌ لَنَا انْ نَخْدِمَ الْمِصْرِيِّينَ مِنْ انْ نَمُوتَ فِي الْبَرِّيَّةِ».
\par 13 فَقَالَ مُوسَى لِلشَّعْبِ: «لا تَخَافُوا. قِفُوا وَانْظُرُوا خَلاصَ الرَّبِّ الَّذِي يَصْنَعُهُ لَكُمُ الْيَوْمَ. فَانَّهُ كَمَا رَايْتُمُ الْمِصْرِيِّينَ الْيَوْمَ لا تَعُودُونَ تَرُونَهُمْ ايْضا الَى الابَدِ.
\par 14 الرَّبُّ يُقَاتِلُ عَنْكُمْ وَانْتُمْ تَصْمُتُونَ».
\par 15 فَقَالَ الرَّبُّ لِمُوسَى: «مَا لَكَ تَصْرُخُ الَيَّ؟ قُلْ لِبَنِي اسْرَائِيلَ انْ يَرْحَلُوا.
\par 16 وَارْفَعْ انْتَ عَصَاكَ وَمُدَّ يَدَكَ عَلَى الْبَحْرِ وَشُقَّهُ فَيَدْخُلَ بَنُو اسْرَائِيلَ فِي وَسَطِ الْبَحْرِ عَلَى الْيَابِسَةِ.
\par 17 وَهَا انَا اشَدِّدُ قُلُوبَ الْمِصْرِيِّينَ حَتَّى يَدْخُلُوا وَرَاءَهُمْ فَاتَمَجَّدُ بِفِرْعَوْنَ وَكُلِّ جَيْشِهِ بِمَرْكَبَاتِهِ وَفُرْسَانِهِ.
\par 18 فَيَعْرِفُ الْمِصْرِيُّونَ انِّي انَا الرَّبُّ حِينَ اتَمَجَّدُ بِفِرْعَوْنَ وَمَرْكَبَاتِهِ وَفُرْسَانِهِ».
\par 19 فَانْتَقَلَ مَلاكُ اللهِ السَّائِرُ امَامَ عَسْكَرِ اسْرَائِيلَ وَسَارَ وَرَاءَهُمْ وَانْتَقَلَ عَمُودُ السَّحَابِ مِنْ امَامِهِمْ وَوَقَفَ وَرَاءَهُمْ.
\par 20 فَدَخَلَ بَيْنَ عَسْكَرِ الْمِصْرِيِّينَ وَعَسْكَرِ اسْرَائِيلَ وَصَارَ السَّحَابُ وَالظَّلامُ وَاضَاءَ اللَّيْلَ. فَلَمْ يَقْتَرِبْ هَذَا الَى ذَاكَ كُلَّ اللَّيْلِ.
\par 21 وَمَدَّ مُوسَى يَدَهُ عَلَى الْبَحْرِ فَاجْرَى الرَّبُّ الْبَحْرَ بِرِيحٍ شَرْقِيَّةٍ شَدِيدَةٍ كُلَّ اللَّيْلِ وَجَعَلَ الْبَحْرَ يَابِسَةً وَانْشَقَّ الْمَاءُ.
\par 22 فَدَخَلَ بَنُو اسْرَائِيلَ فِي وَسَطِ الْبَحْرِ عَلَى الْيَابِسَةِ وَالْمَاءُ سُورٌ لَهُمْ عَنْ يَمِينِهِمْ وَعَنْ يَسَارِهِمْ.
\par 23 وَتَبِعَهُمُ الْمِصْرِيُّونَ وَدَخَلُوا وَرَاءَهُمْ جَمِيعُ خَيْلِ فِرْعَوْنَ وَمَرْكَبَاتِهِ وَفُرْسَانِهِ الَى وَسَطِ الْبَحْرِ.
\par 24 وَكَانَ فِي هَزِيعِ الصُّبْحِ انَّ الرَّبَّ اشْرَفَ عَلَى عَسْكَرِ الْمِصْرِيِّينَ فِي عَمُودِ النَّارِ وَالسَّحَابِ وَازْعَجَ عَسْكَرَ الْمِصْرِيِّينَ
\par 25 وَخَلَعَ بَكَرَ مَرْكَبَاتِهِمْ حَتَّى سَاقُوهَا بِثَقْلَةٍ. فَقَالَ الْمِصْرِيُّونَ: «نَهْرُبُ مِنْ اسْرَائِيلَ لانَّ الرَّبَّ يُقَاتِلُ الْمِصْرِيِّينَ عَنْهُمْ».
\par 26 فَقَالَ الرَّبُّ لِمُوسَى: «مُدَّ يَدَكَ عَلَى الْبَحْرِ لِيَرْجِعَ الْمَاءُ عَلَى الْمِصْرِيِّينَ عَلَى مَرْكَبَاتِهِمْ وَفُرْسَانِهِمْ».
\par 27 فَمَدَّ مُوسَى يَدَهُ عَلَى الْبَحْرِ فَرَجَعَ الْبَحْرُ عِنْدَ اقْبَالِ الصُّبْحِ الَى حَالِهِ الدَّائِمَةِ وَالْمِصْرِيُّونَ هَارِبُونَ الَى لِقَائِهِ. فَدَفَعَ الرَّبُّ الْمِصْرِيِّينَ فِي وَسَطِ الْبَحْرِ.
\par 28 فَرَجَعَ الْمَاءُ وَغَطَّى مَرْكَبَاتِ وَفُرْسَانَ جَمِيعِ جَيْشِ فِرْعَوْنَ الَّذِي دَخَلَ وَرَاءَهُمْ فِي الْبَحْرِ. لَمْ يَبْقَ مِنْهُمْ وَلا وَاحِدٌ.
\par 29 وَامَّا بَنُو اسْرَائِيلَ فَمَشُوا عَلَى الْيَابِسَةِ فِي وَسَطِ الْبَحْرِ وَالْمَاءُ سُورٌ لَهُمْ عَنْ يَمِينِهِمْ وَعَنْ يَسَارِهِمْ.
\par 30 فَخَلَّصَ الرَّبُّ فِي ذَلِكَ الْيَوْمِ اسْرَائِيلَ مِنْ يَدِ الْمِصْرِيِّينَ. وَنَظَرَ اسْرَائِيلُ الْمِصْرِيِّينَ امْوَاتا عَلَى شَاطِئِ الْبَحْرِ.
\par 31 وَرَاى اسْرَائِيلُ الْفَِعْلَ الْعَظِيمَ الَّذِي صَنَعَهُ الرَّبُّ بِالْمِصْرِيِّينَ. فَخَافَ الشَّعْبُ الرَّبَّ وَامَنُوا بِالرَّبِّ وَبِعَبْدِهِ مُوسَى.

\chapter{15}

\par 1 حِينَئِذٍ رَنَّمَ مُوسَى وَبَنُو اسْرَائِيلَ هَذِهِ التَّسْبِيحَةَ لِلرَّبِّ: «ارَنِّمُ لِلرَّبِّ فَانَّهُ قَدْ تَعَظَّمَ. الْفَرَسَ وَرَاكِبَهُ طَرَحَهُمَا فِي الْبَحْرِ.
\par 2 الرَّبُّ قُوَّتِي وَنَشِيدِي وَقَدْ صَارَ خَلاصِي. هَذَا الَهِي فَامَجِّدُهُ الَهُ ابِي فَارَفِّعُهُ.
\par 3 الرَّبُّ رَجُلُ الْحَرْبِ. الرَّبُّ اسْمُهُ.
\par 4 مَرْكَبَاتِ فِرْعَوْنَ وَجَيْشَهُ الْقَاهُمَا فِي الْبَحْرِ فَغَرِقَ افْضَلُ جُنُودِهِ الْمَرْكَبِيَّةِ فِي بَحْرِ سُوفَ
\par 5 تُغَطِّيهِمُ اللُّجَجُ. قَدْ هَبَطُوا فِي الاعْمَاقِ كَحَجَرٍ.
\par 6 يَمِينُكَ يَا رَبُّ مُعْتَزَّةٌ بِالْقُدْرَةِ. يَمِينُكَ يَا رَبُّ تُحَطِّمُ الْعَدُوَّ.
\par 7 وَبِكَثْرَةِ عَظَمَتِكَ تَهْدِمُ مُقَاوِمِيكَ. تُرْسِلُ سَخَطَكَ فَيَاكُلُهُمْ كَالْقَشِّ
\par 8 وَبِرِيحِ انْفِكَ تَرَاكَمَتِ الْمِيَاهُ. انْتَصَبَتِ الْمِيَاهُ الْجَارِيَةُ كَرَابِيَةٍ. تَجَمَّدَتِ اللُّجَجُ فِي قَلْبِ الْبَحْرِ.
\par 9 قَالَ الْعَدُوُّ: اتْبَعُ ادْرِكُ اقَسِّمُ غَنِيمَةً! تَمْتَلِئُ مِنْهُمْ نَفْسِي. اجَرِّدُ سَيْفِي. تُفْنِيهِمْ يَدِي!
\par 10 نَفَخْتَ بِرِيحِكَ فَغَطَّاهُمُ الْبَحْرُ. غَاصُوا كَالرَّصَاصِ فِي مِيَاهٍ غَامِرَةٍ.
\par 11 مَنْ مِثْلُكَ بَيْنَ الالِهَةِ يَا رَبُّ؟ مَنْ مِثْلُكَ مُعْتَزّا فِي الْقَدَاسَةِ مَخُوفا بِالتَّسَابِيحِ صَانِعا عَجَائِبَ؟
\par 12 تَمُدُّ يَمِينَكَ فَتَبْتَلِعُهُمُ الارْضُ.
\par 13 تُرْشِدُ بِرَافَتِكَ الشَّعْبَ الَّذِي فَدَيْتَهُ. تَهْدِيهِ بِقُوَّتِكَ الَى مَسْكَنِ قُدْسِكَ.
\par 14 يَسْمَعُ الشُّعُوبُ فَيَرْتَعِدُونَ. تَاخُذُ الرَّعْدَةُ سُكَّانَ فَلَسْطِينَ.
\par 15 حِينَئِذٍ يَنْدَهِشُ امَرَاءُ ادُومَ. اقْوِيَاءُ مُوابَ تَاخُذُهُمُ الرَّجْفَةُ. يَذُوبُ جَمِيعُ سُكَّانِ كَنْعَانَ.
\par 16 تَقَعُ عَلَيْهِمِ الْهَيْبَةُ وَالرُّعْبُ. بِعَظَمَةِ ذِرَاعِكَ يَصْمُتُونَ كَالْحَجَرِ حَتَّى يَعْبُرَ شَعْبُكَ يَا رَبُّ. حَتَّى يَعْبُرَ الشَّعْبُ الَّذِي اقْتَنَيْتَهُ.
\par 17 تَجِيءُ بِهِمْ وَتَغْرِسُهُمْ فِي جَبَلِ مِيرَاثِكَ الْمَكَانِ الَّذِي صَنَعْتَهُ يَا رَبُّ لِسَكَنِكَ. الْمَقْدِسِ الَّذِي هَيَّاتْهُ يَدَاكَ يَا رَبُّ.
\par 18 الرَّبُّ يَمْلِكُ الَى الدَّهْرِ وَالابَدِ.
\par 19 فَانَّ خَيْلَ فِرْعَوْنَ دَخَلَتْ بِمَرْكَبَاتِهِ وَفُرْسَانِهِ الَى الْبَحْرِ وَرَدَّ الرَّبُّ عَلَيْهِمْ مَاءَ الْبَحْرِ. وَامَّا بَنُو اسْرَائِيلَ فَمَشُوا عَلَى الْيَابِسَةِ فِي وَسَطِ الْبَحْرِ».
\par 20 فَاخَذَتْ مَرْيَمُ النَّبِيَّةُ اخْتُ هَارُونَ الدُّفَّ بِيَدِهَا وَخَرَجَتْ جَمِيعُ النِّسَاءِ وَرَاءَهَا بِدُفُوفٍ وَرَقْصٍ.
\par 21 وَاجَابَتْهُمْ مَرْيَمُ: «رَنِّمُوا لِلرَّبِّ فَانَّهُ قَدْ تَعَظَّمَ! الْفَرَسَ وَرَاكِبَهُ طَرَحَهُمَا فِي الْبَحْرِ!».
\par 22 ثُمَّ ارْتَحَلَ مُوسَى بِاسْرَائِيلَ مِنْ بَحْرِ سُوفَ وَخَرَجُوا الَى بَرِّيَّةِ شُورٍ. فَسَارُوا ثَلاثَةَ ايَّامٍ فِي الْبَرِّيَّةِ وَلَمْ يَجِدُوا مَاءً
\par 23 فَجَاءُوا الَى مَارَّةَ. وَلَمْ يَقْدِرُوا انْ يَشْرَبُوا مَاءً مِنْ مَارَّةَ لانَّهُ مُرٌّ. لِذَلِكَ دُعِيَ اسْمُهَا «مَارَّةَ».
\par 24 فَتَذَمَّرَ الشَّعْبُ عَلَى مُوسَى قَائِلِينَ: «مَاذَا نَشْرَبُ؟»
\par 25 فَصَرَخَ الَى الرَّبِّ. فَارَاهُ الرَّبُّ شَجَرَةً فَطَرَحَهَا فِي الْمَاءِ فَصَارَ الْمَاءُ عَذْبا. هُنَاكَ وَضَعَ لَهُ فَرِيضَةً وَحُكْما وَهُنَاكَ امْتَحَنَهُ.
\par 26 فَقَالَ: «انْ كُنْتَ تَسْمَعُ لِصَوْتِ الرَّبِّ الَهِكَ وَتَصْنَعُ الْحَقَّ فِي عَيْنَيْهِ وَتَصْغَى الَى وَصَايَاهُ وَتَحْفَظُ جَمِيعَ فَرَائِضِهِ فَمَرَضا مَا مِمَّا وَضَعْتُهُ عَلَى الْمِصْرِيِّينَ لا اضَعُ عَلَيْكَ. فَانِّي انَا الرَّبُّ شَافِيكَ».
\par 27 ثُمَّ جَاءُوا الَى ايلِيمَ وَهُنَاكَ اثْنَتَا عَشْرَةَ عَيْنَ مَاءٍ وَسَبْعُونَ نَخْلَةً. فَنَزَلُوا هُنَاكَ عِنْدَ الْمَاءِ.

\chapter{16}

\par 1 ثُمَّ ارْتَحَلُوا مِنْ ايلِيمَ. وَاتَى كُلُّ جَمَاعَةِ بَنِي اسْرَائِيلَ الَى بَرِّيَّةِ سِينٍ (الَّتِي بَيْنَ ايلِيمَ وَسِينَاءَ) فِي الْيَوْمِ الْخَامِسَ عَشَرَ مِنَ الشَّهْرِ الثَّانِي بَعْدَ خُرُوجِهِمْ مِنْ ارْضِ مِصْرَ.
\par 2 فَتَذَمَّرَ كُلُّ جَمَاعَةِ بَنِي اسْرَائِيلَ عَلَى مُوسَى وَهَارُونَ فِي الْبَرِّيَّةِ.
\par 3 وَقَالَ لَهُمَا بَنُو اسْرَائِيلَ: «لَيْتَنَا مُتْنَا بِيَدِ الرَّبِّ فِي ارْضِ مِصْرَ اذْ كُنَّا جَالِسِينَ عِنْدَ قُدُورِ اللَّحْمِ نَاكُلُ خُبْزا لِلشَّبَعِ! فَانَّكُمَا اخْرَجْتُمَانَا الَى هَذَا الْقَفْرِ لِتُمِيتَا كُلَّ هَذَا الْجُمْهُورِ بِالْجُوعِ».
\par 4 فَقَالَ الرَّبُّ لِمُوسَى: «هَا انَا امْطِرُ لَكُمْ خُبْزا مِنَ السَّمَاءِ! فَيَخْرُجُ الشَّعْبُ وَيَلْتَقِطُونَ حَاجَةَ الْيَوْمِ بِيَوْمِهَا. لامْتَحِنَهُمْ ايَسْلُكُونَ فِي نَامُوسِي امْ لا؟
\par 5 وَيَكُونُ فِي الْيَوْمِ السَّادِسِ انَّهُمْ يُهَيِّئُونَ مَا يَجِيئُونَ بِهِ فَيَكُونُ ضِعْفَ مَا يَلْتَقِطُونَهُ يَوْما فَيَوْما».
\par 6 فَقَالَ مُوسَى وَهَارُونُ لِجَمِيعِ بَنِي اسْرَائِيلَ: «فِي الْمَسَاءِ تَعْلَمُونَ انَّ الرَّبَّ اخْرَجَكُمْ مِنْ ارْضِ مِصْرَ.
\par 7 وَفِي الصَّبَاحِ تَرُونَ مَجْدَ الرَّبِّ لاسْتِمَاعِهِ تَذَمُّرَكُمْ عَلَى الرَّبِّ. وَامَّا نَحْنُ فَمَاذَا حَتَّى تَتَذَمَّرُوا عَلَيْنَا؟»
\par 8 وَقَالَ مُوسَى: «ذَلِكَ بِانَّ الرَّبَّ يُعْطِيكُمْ فِي الْمَسَاءِ لَحْما لِتَاكُلُوا وَفِي الصَّبَاحِ خُبْزا لِتَشْبَعُوا لاسْتِمَاعِ الرَّبِّ تَذَمُّرَكُمُ الَّذِي تَتَذَمَّرُونَ عَلَيْهِ. وَامَّا نَحْنُ فَمَاذَا؟ لَيْسَ عَلَيْنَا تَذَمُّرُكُمْ بَلْ عَلَى الرَّبِّ».
\par 9 وَقَالَ مُوسَى لِهَارُونَ: «قُلْ لِكُلِّ جَمَاعَةِ بَنِي اسْرَائِيلَ: اقْتَرِبُوا الَى امَامِ الرَّبِّ لانَّهُ قَدْ سَمِعَ تَذَمُّرَكُمْ».
\par 10 فَحَدَثَ اذْ كَانَ هَارُونُ يُكَلِّمُ كُلَّ جَمَاعَةِ بَنِي اسْرَائِيلَ انَّهُمُ الْتَفَتُوا نَحْوَ الْبَرِّيَّةِ وَاذَا مَجْدُ الرَّبِّ قَدْ ظَهَرَ فِي السَّحَابِ.
\par 11 فَقَالَ الرَّبُّ لِمُوسَى:
\par 12 «سَمِعْتُ تَذَمُّرَ بَنِي اسْرَائِيلَ. قُلْ لَهُمْ: فِي الْعَشِيَّةِ تَاكُلُونَ لَحْما وَفِي الصَّبَاحِ تَشْبَعُونَ خُبْزا وَتَعْلَمُونَ انِّي انَا الرَّبُّ الَهُكُمْ».
\par 13 فَكَانَ فِي الْمَسَاءِ انَّ السَّلْوَى صَعِدَتْ وَغَطَّتِ الْمَحَلَّةَ. وَفِي الصَّبَاحِ كَانَ سَقِيطُ النَّدَى حَوَالَيِ الْمَحَلَّةِ.
\par 14 وَلَمَّا ارْتَفَعَ سَقِيطُ النَّدَى اذَا عَلَى وَجْهِ الْبَرِّيَّةِ شَيْءٌ دَقِيقٌ مِثْلُ قُشُورٍ. دَقِيقٌ كَالْجَلِيدِ عَلَى الارْضِ.
\par 15 فَلَمَّا رَاى بَنُو اسْرَائِيلَ قَالُوا بَعْضُهُمْ لِبَعْضٍ: «مَنْ هُوَ؟» لانَّهُمْ لَمْ يَعْرِفُوا مَا هُوَ. فَقَالَ لَهُمْ مُوسَى: «هُوَ الْخُبْزُ الَّذِي اعْطَاكُمُ الرَّبُّ لِتَاكُلُوا.
\par 16 هَذَا هُوَ الشَّيْءُ الَّذِي امَرَ بِهِ الرَّبُّ. الْتَقِطُوا مِنْهُ كُلُّ وَاحِدٍ عَلَى حَسَبِ اكْلِهِ. عُمِرا لِلرَّاسِ عَلَى عَدَدِ نُفُوسِكُمْ تَاخُذُونَ كُلُّ وَاحِدٍ لِلَّذِينَ فِي خَيْمَتِهِ».
\par 17 فَفَعَلَ بَنُو اسْرَائِيلَ هَكَذَا وَالْتَقَطُوا بَيْنَ مُكْثِرٍ وَمُقَلِّلٍ.
\par 18 وَلَمَّا كَالُوا بِالْعُمِرِ لَمْ يُفْضِلِ الْمُكْثِرُ وَالْمُقَلِّلُ لَمْ يُنْقِصْ. كَانُوا قَدِ الْتَقَطُوا كُلُّ وَاحِدٍ عَلَى حَسَبِ اكْلِهِ.
\par 19 وَقَالَ لَهُمْ مُوسَى: «لا يُبْقِ احَدٌ مِنْهُ الَى الصَّبَاحِ».
\par 20 لَكِنَّهُمْ لَمْ يَسْمَعُوا لِمُوسَى بَلْ ابْقَى مِنْهُ انَاسٌ الَى الصَّبَاحِ فَتَوَلَّدَ فِيهِ دُودٌ وَانْتَنَ. فَسَخَطَ عَلَيْهِمْ مُوسَى.
\par 21 وَكَانُوا يَلْتَقِطُونَهُ صَبَاحا فَصَبَاحا كُلُّ وَاحِدٍ عَلَى حَسَبِ اكْلِهِ. وَاذَا حَمِيَتِ الشَّمْسُ كَانَ يَذُوبُ.
\par 22 ثُمَّ كَانَ فِي الْيَوْمِ السَّادِسِ انَّهُمُ الْتَقَطُوا خُبْزا مُضَاعَفا عُمِرَيْنِ لِلْوَاحِدِ. فَجَاءَ كُلُّ رُؤَسَاءِ الْجَمَاعَةِ وَاخْبَرُوا مُوسَى.
\par 23 فَقَالَ لَهُمْ: «هَذَا مَا قَالَ الرَّبُّ. غَدا عُطْلَةٌ سَبْتٌ مُقَدَّسٌ لِلرَّبِّ. اخْبِزُوا مَا تَخْبِزُونَ وَاطْبُخُوا مَا تَطْبُخُونَ. وَكُلُّ مَا فَضَلَ ضَعُوهُ عِنْدَكُمْ لِيُحْفَظَ الَى الْغَدِ».
\par 24 فَوَضَعُوهُ الَى الْغَدِ كَمَا امَرَ مُوسَى فَلَمْ يُنْتِنْ وَلا صَارَ فِيهِ دُودٌ.
\par 25 فَقَالَ مُوسَى: «كُلُوهُ الْيَوْمَ لانَّ لِلرَّبِّ الْيَوْمَ سَبْتا. الْيَوْمَ لا تَجِدُونَهُ فِي الْحَقْلِ.
\par 26 سِتَّةَ ايَّامٍ تَلْتَقِطُونَهُ وَامَّا الْيَوْمُ السَّابِعُ فَفِيهِ سَبْتٌ. لا يُوجَدُ فِيهِ».
\par 27 وَحَدَثَ فِي الْيَوْمِ السَّابِعِ انَّ بَعْضَ الشَّعْبِ خَرَجُوا لِيَلْتَقِطُوا فَلَمْ يَجِدُوا.
\par 28 فَقَالَ الرَّبُّ لِمُوسَى: «الَى مَتَى تَابُونَ انْ تَحْفَظُوا وَصَايَايَ وَشَرَائِعِي؟
\par 29 انْظُرُوا! انَّ الرَّبَّ اعْطَاكُمُ السَّبْتَ. لِذَلِكَ هُوَ يُعْطِيكُمْ فِي الْيَوْمِ السَّادِسِ خُبْزَ يَوْمَيْنِ. اجْلِسُوا كُلُّ وَاحِدٍ فِي مَكَانِهِ. لا يَخْرُجْ احَدٌ مِنْ مَكَانِهِ فِي الْيَوْمِ السَّابِعِ».
\par 30 فَاسْتَرَاحَ الشَّعْبُ فِي الْيَوْمِ السَّابِعِ.
\par 31 وَدَعَا بَيْتُ اسْرَائِيلَ اسْمَهُ «مَنّا». وَهُوَ كَبِزْرِ الْكُزْبَرَةِ ابْيَضُ وَطَعْمُهُ كَرِقَاقٍ بِعَسَلٍ.
\par 32 وَقَالَ مُوسَى: «هَذَا هُوَ الشَّيْءُ الَّذِي امَرَ بِهِ الرَّبُّ. مِلْءُ الْعُمِرِ مِنْهُ يَكُونُ لِلْحِفْظِ فِي اجْيَالِكُمْ. لِيَرُوا الْخُبْزَ الَّذِي اطْعَمْتُكُمْ فِي الْبَرِّيَّةِ حِينَ اخْرَجْتُكُمْ مِنْ ارْضِ مِصْرَ».
\par 33 وَقَالَ مُوسَى لِهَارُونَ: «خُذْ قِسْطا وَاحِدا وَاجْعَلْ فِيهِ مِلْءَ الْعُمِرِ مَنّا وَضَعْهُ امَامَ الرَّبِّ لِلْحِفْظِ فِي اجْيَالِكُمْ».
\par 34 كَمَا امَرَ الرَّبُّ مُوسَى وَضَعَهُ هَارُونُ امَامَ الشَّهَادَةِ لِلْحِفْظِ.
\par 35 وَاكَلَ بَنُو اسْرَائِيلَ الْمَنَّ ارْبَعِينَ سَنَةً حَتَّى جَاءُوا الَى ارْضٍ عَامِرَةٍ. اكَلُوا الْمَنَّ حَتَّى جَاءُوا الَى طَرَفِ ارْضِ كَنْعَانَ.
\par 36 وَامَّا الْعُمِرُ فَهُوَ عُشْرُ الْايفَةِ.

\chapter{17}

\par 1 ثُمَّ ارْتَحَلَ كُلُّ جَمَاعَةِ بَنِي اسْرَائِيلَ مِنْ بَرِّيَّةِ سِينٍ بِحَسَبِ مَرَاحِلِهِمْ عَلَى مُوجِبِ امْرِ الرَّبِّ وَنَزَلُوا فِي رَفِيدِيمَ. وَلَمْ يَكُنْ مَاءٌ لِيَشْرَبَ الشَّعْبُ.
\par 2 فَخَاصَمَ الشَّعْبُ مُوسَى وَقَالُوا: «اعْطُونَا مَاءً لِنَشْرَبَ!» فَقَالَ لَهُمْ مُوسَى: «لِمَاذَا تُخَاصِمُونَنِي؟ لِمَاذَا تُجَرِّبُونَ الرَّبَّ؟»
\par 3 وَعَطِشَ هُنَاكَ الشَّعْبُ الَى الْمَاءِ وَتَذَمَّرَ الشَّعْبُ عَلَى مُوسَى وَقَالُوا: «لِمَاذَا اصْعَدْتَنَا مِنْ مِصْرَ لِتُمِيتَنَا وَاوْلادَنَا وَمَوَاشِيَنَا بِالْعَطَشِ؟»
\par 4 فَصَرَخَ مُوسَى الَى الرَّبِّ: «مَاذَا افْعَلُ بِهَذَا الشَّعْبِ؟ بَعْدَ قَلِيلٍ يَرْجُمُونَنِي!»
\par 5 فَقَالَ الرَّبُّ لِمُوسَى: «مُرَّ قُدَّامَ الشَّعْبِ وَخُذْ مَعَكَ مِنْ شُيُوخِ اسْرَائِيلَ. وَعَصَاكَ الَّتِي ضَرَبْتَ بِهَا النَّهْرَ خُذْهَا فِي يَدِكَ وَاذْهَبْ.
\par 6 هَا انَا اقِفُ امَامَكَ هُنَاكَ عَلَى الصَّخْرَةِ فِي حُورِيبَ فَتَضْرِبُ الصَّخْرَةَ فَيَخْرُجُ مِنْهَا مَاءٌ لِيَشْرَبَ الشَّعْبُ». فَفَعَلَ مُوسَى هَكَذَا امَامَ عُيُونِ شُيُوخِ اسْرَائِيلَ.
\par 7 وَدَعَا اسْمَ الْمَوْضِعِ «مَسَّةَ وَمَرِيبَةَ» مِنْ اجْلِ مُخَاصَمَةِ بَنِي اسْرَائِيلَ وَمِنْ اجْلِ تَجْرِبَتِهِمْ لِلرَّبِّ قَائِلِينَ: «افِي وَسَطِنَا الرَّبُّ امْ لا؟».
\par 8 وَاتَى عَمَالِيقُ وَحَارَبَ اسْرَائِيلَ فِي رَفِيدِيمَ.
\par 9 فَقَالَ مُوسَى لِيَشُوعَ: «انْتَخِبْ لَنَا رِجَالا وَاخْرُجْ حَارِبْ عَمَالِيقَ. وَغَدا اقِفُ انَا عَلَى رَاسِ التَّلَّةِ وَعَصَا اللهِ فِي يَدِي».
\par 10 فَفَعَلَ يَشُوعُ كَمَا قَالَ لَهُ مُوسَى لِيُحَارِبَ عَمَالِيقَ. وَامَّا مُوسَى وَهَارُونُ وَحُورُ فَصَعِدُوا عَلَى رَاسِ التَّلَّةِ.
\par 11 وَكَانَ اذَا رَفَعَ مُوسَى يَدَهُ انَّ اسْرَائِيلَ يَغْلِبُ وَاذَا خَفَضَ يَدَهُ انَّ عَمَالِيقَ يَغْلِبُ.
\par 12 فَلَمَّا صَارَتْ يَدَا مُوسَى ثَقِيلَتَيْنِ اخَذَا حَجَرا وَوَضَعَاهُ تَحْتَهُ فَجَلَسَ عَلَيْهِ. وَدَعَمَ هَارُونُ وَحُورُ يَدَيْهِ الْوَاحِدُ مِنْ هُنَا وَالاخَرُ مِنْ هُنَاكَ. فَكَانَتْ يَدَاهُ ثَابِتَتَيْنِ الَى غُرُوبِ الشَّمْسِ.
\par 13 فَهَزَمَ يَشُوعُ عَمَالِيقَ وَقَوْمَهُ بِحَدِّ السَّيْفِ.
\par 14 فَقَالَ الرَّبُّ لِمُوسَى: «اكْتُبْ هَذَا تِذْكَارا فِي الْكِتَابِ وَضَعْهُ فِي مَسَامِعِ يَشُوعَ. فَانِّي سَوْفَ امْحُو ذِكْرَ عَمَالِيقَ مِنْ تَحْتِ السَّمَاءِ».
\par 15 فَبَنَى مُوسَى مَذْبَحا وَدَعَا اسْمَهُ «يَهْوَهْ نِسِّي».
\par 16 وَقَالَ: «انَّ الْيَدَ عَلَى كُرْسِيِّ الرَّبِّ. لِلرَّبِّ حَرْبٌ مَعَ عَمَالِيقَ مِنْ دَوْرٍ الَى دَوْرٍ».

\chapter{18}

\par 1 فَسَمِعَ يَثْرُونُ كَاهِنُ مِدْيَانَ حَمُو مُوسَى كُلَّ مَا صَنَعَ اللهُ الَى مُوسَى وَالَى اسْرَائِيلَ شَعْبِهِ: انَّ الرَّبَّ اخْرَجَ اسْرَائِيلَ مِنْ مِصْرَ.
\par 2 فَاخَذَ يَثْرُونُ حَمُو مُوسَى صَفُّورَةَ امْرَاةَ مُوسَى بَعْدَ صَرْفِهَا
\par 3 وَابْنَيْهَا اللَّذَيْنِ اسْمُ احَدِهِمَا جَرْشُومُ (لانَّهُ قَالَ: «كُنْتُ نَزِيلا فِي ارْضٍ غَرِيبَةٍ»).
\par 4 وَاسْمُ الاخَرِ الِيعَازَرُ (لانَّهُ قَالَ: «الَهُ ابِي كَانَ عَوْنِي وَانْقَذَنِي مِنْ سَيْفِ فِرْعَوْنَ»).
\par 5 وَاتَى يَثْرُونُ حَمُو مُوسَى وَابْنَاهُ وَامْرَاتُهُ الَى مُوسَى الَى الْبَرِّيَّةِ حَيْثُ كَانَ نَازِلا عِنْدَ جَبَلِ اللهِ.
\par 6 فَقَالَ لِمُوسَى: «انَا حَمُوكَ يَثْرُونُ اتٍ الَيْكَ وَامْرَاتُكَ وَابْنَاهَا مَعَهَا».
\par 7 فَخَرَجَ مُوسَى لاسْتِقْبَالِ حَمِيهِ وَسَجَدَ وَقَبَّلَهُ. وَسَالَ كُلُّ وَاحِدٍ صَاحِبَهُ عَنْ سَلامَتِهِ. ثُمَّ دَخَلا الَى الْخَيْمَةِ.
\par 8 فَقَصَّ مُوسَى عَلَى حَمِيهِ كُلَّ مَا صَنَعَ الرَّبُّ بِفِرْعَوْنَ وَالْمِصْرِيِّينَ مِنْ اجْلِ اسْرَائِيلَ وَكُلَّ الْمَشَقَّةِ الَّتِي اصَابَتْهُمْ فِي الطَّرِيقِ فَخَلَّصَهُمُ الرَّبُّ.
\par 9 فَفَرِحَ يَثْرُونُ بِجَمِيعِ الْخَيْرِ الَّذِي صَنَعَهُ الَى اسْرَائِيلَ الرَّبُّ الَّذِي انْقَذَهُ مِنْ ايْدِي الْمِصْرِيِّينَ.
\par 10 وَقَالَ يَثْرُونُ: «مُبَارَكٌ الرَّبُّ الَّذِي انْقَذَكُمْ مِنْ ايْدِي الْمِصْرِيِّينَ وَمِنْ يَدِ فِرْعَوْنَ. الَّذِي انْقَذَ الشَّعْبَ مِنْ تَحْتِ ايْدِي الْمِصْرِيِّينَ.
\par 11 الانَ عَلِمْتُ انَّ الرَّبَّ اعْظَمُ مِنْ جَمِيعِ الالِهَةِ لانَّهُ فِي الشَّيْءِ الَّذِي بَغُوا بِهِ كَانَ عَلَيْهِمْ».
\par 12 فَاخَذَ يَثْرُونُ حَمُو مُوسَى مُحْرَقَةً وَذَبَائِحَ لِلَّهِ. وَجَاءَ هَارُونُ وَجَمِيعُ شُيُوخِ اسْرَائِيلَ لِيَاكُلُوا طَعَاما مَعَ حَمِي مُوسَى امَامَ اللهِ.
\par 13 وَحَدَثَ فِي الْغَدِ انَّ مُوسَى جَلَسَ لِيَقْضِيَ لِلشَّعْبِ. فَوَقَفَ الشَّعْبُ عِنْدَ مُوسَى مِنَ الصَّبَاحِ الَى الْمَسَاءِ.
\par 14 فَلَمَّا رَاى حَمُو مُوسَى كُلَّ مَا هُوَ صَانِعٌ لِلشَّعْبِ قَالَ: «مَا هَذَا الامْرُ الَّذِي انْتَ صَانِعٌ لِلشَّعْبِ؟ مَا بَالُكَ جَالِسا وَحْدَكَ وَجَمِيعُ الشَّعْبِ وَاقِفٌ عِنْدَكَ مِنَ الصَّبَاحِ الَى الْمَسَاءِ؟»
\par 15 فَقَالَ مُوسَى لِحَمِيهِ: «انَّ الشَّعْبَ يَاتِي الَيَّ لِيَسْالَ اللهَ.
\par 16 اذَا كَانَ لَهُمْ دَعْوَى يَاتُونَ الَيَّ فَاقْضِي بَيْنَ الرَّجُلِ وَصَاحِبِهِ وَاعَرِّفُهُمْ فَرَائِضَ اللهِ وَشَرَائِعَهُ».
\par 17 فَقَالَ حَمُو مُوسَى لَهُ: «لَيْسَ جَيِّدا الامْرُ الَّذِي انْتَ صَانِعٌ.
\par 18 انَّكَ تَكِلُّ انْتَ وَهَذَا الشَّعْبُ الَّذِي مَعَكَ جَمِيعا لانَّ الامْرَ اعْظَمُ مِنْكَ. لا تَسْتَطِيعُ انْ تَصْنَعَهُ وَحْدَكَ.
\par 19 الانَ اسْمَعْ لِصَوْتِي فَانْصَحَكَ. فَلْيَكُنِ اللهُ مَعَكَ. كُنْ انْتَ لِلشَّعْبِ امَامَ اللهِ وَقَدِّمْ انْتَ الدَّعَاوِيَ الَى اللهِ
\par 20 وَعَلِّمْهُمُ الْفَرَائِضَ وَالشَّرَائِعَ وَعَرِّفْهُمُ الطَّرِيقَ الَّذِي يَسْلُكُونَهُ وَالْعَمَلَ الَّذِي يَعْمَلُونَهُ.
\par 21 وَانْتَ تَنْظُرُ مِنْ جَمِيعِ الشَّعْبِ ذَوِي قُدْرَةٍ خَائِفِينَ اللهَ امَنَاءَ مُبْغِضِينَ الرَّشْوَةَ وَتُقِيمُهُمْ عَلَيْهِمْ رُؤَسَاءَ الُوفٍ وَرُؤَسَاءَ مِئَاتٍ وَرُؤَسَاءَ خَمَاسِينَ وَرُؤَسَاءَ عَشَرَاتٍ
\par 22 فَيَقْضُونَ لِلشَّعْبِ كُلَّ حِينٍ. وَيَكُونُ انَّ كُلَّ الدَّعَاوِي الْكَبِيرَةِ يَجِيئُونَ بِهَا الَيْكَ. وَكُلَّ الدَّعَاوِي الصَّغِيرَةِ يَقْضُونَ هُمْ فِيهَا. وَخَفِّفْ عَنْ نَفْسِكَ فَهُمْ يَحْمِلُونَ مَعَكَ.
\par 23 انْ فَعَلْتَ هَذَا الامْرَ وَاوْصَاكَ اللهُ تَسْتَطِيعُ الْقِيَامَ. وَكُلُّ هَذَا الشَّعْبِ ايْضا يَاتِي الَى مَكَانِهِ بِالسَّلامِ».
\par 24 فَسَمِعَ مُوسَى لِصَوْتِ حَمِيهِ وَفَعَلَ كُلَّ مَا قَالَ.
\par 25 وَاخْتَارَ مُوسَى ذَوِي قُدْرَةٍ مِنْ جَمِيعِ اسْرَائِيلَ وَجَعَلَهُمْ رُؤُوسا عَلَى الشَّعْبِ رُؤَسَاءَ الُوفٍ وَرُؤَسَاءَ مِئَاتٍ وَرُؤَسَاءَ خَمَاسِينَ وَرُؤَسَاءَ عَشَرَاتٍ.
\par 26 فَكَانُوا يَقْضُونَ لِلشَّعْبِ كُلَّ حِينٍ. الدَّعَاوِي الْعَسِرَةُ يَجِيئُونَ بِهَا الَى مُوسَى وَكُلُّ الدَّعَاوِي الصَّغِيرَةِ يَقْضُونَ هُمْ فِيهَا.
\par 27 ثُمَّ صَرَفَ مُوسَى حَمَاهُ فَمَضَى الَى ارْضِهِ.

\chapter{19}

\par 1 فِي الشَّهْرِ الثَّالِثِ بَعْدَ خُرُوجِ بَنِي اسْرَائِيلَ مِنْ ارْضِ مِصْرَ فِي ذَلِكَ الْيَوْمِ جَاءُوا الَى بَرِّيَّةِ سِينَاءَ.
\par 2 ارْتَحَلُوا مِنْ رَفِيدِيمَ وَجَاءُوا الَى بَرِّيَّةِ سِينَاءَ فَنَزَلُوا فِي الْبَرِّيَّةِ. هُنَاكَ نَزَلَ اسْرَائِيلُ مُقَابِلَ الْجَبَلِ.
\par 3 وَامَّا مُوسَى فَصَعِدَ الَى اللهِ. فَنَادَاهُ الرَّبُّ مِنَ الْجَبَلِ: «هَكَذَا تَقُولُ لِبَيْتِ يَعْقُوبَ وَتُخْبِرُ بَنِي اسْرَائِيلَ:
\par 4 انْتُمْ رَايْتُمْ مَا صَنَعْتُ بِالْمِصْرِيِّينَ. وَانَا حَمَلْتُكُمْ عَلَى اجْنِحَةِ النُّسُورِ وَجِئْتُ بِكُمْ الَيَّ.
\par 5 فَالانَ انْ سَمِعْتُمْ لِصَوْتِي وَحَفِظْتُمْ عَهْدِي تَكُونُونَ لِي خَاصَّةً مِنْ بَيْنِ جَمِيعِ الشُّعُوبِ. فَانَّ لِي كُلَّ الارْضِ.
\par 6 وَانْتُمْ تَكُونُونَ لِي مَمْلَكَةَ كَهَنَةٍ وَامَّةً مُقَدَّسَةً. هَذِهِ هِيَ الْكَلِمَاتُ الَّتِي تُكَلِّمُ بِهَا بَنِي اسْرَائِيلَ».
\par 7 فَجَاءَ مُوسَى وَدَعَا شُيُوخَ الشَّعْبِ وَوَضَعَ قُدَّامَهُمْ كُلَّ هَذِهِ الْكَلِمَاتِ الَّتِي اوْصَاهُ بِهَا الرَّبُّ.
\par 8 فَاجَابَ جَمِيعُ الشَّعْبِ مَعا: «كُلُّ مَا تَكَلَّمَ بِهِ الرَّبُّ نَفْعَلُ». فَرَدَّ مُوسَى كَلامَ الشَّعْبِ الَى الرَّبِّ.
\par 9 فَقَالَ الرَّبُّ لِمُوسَى: «هَا انَا اتٍ الَيْكَ فِي ظَلامِ السَّحَابِ لِيَسْمَعَ الشَّعْبُ حِينَمَا اتَكَلَّمُ مَعَكَ فَيُؤْمِنُوا بِكَ ايْضا الَى الابَدِ». وَاخْبَرَ مُوسَى الرَّبَّ بِكَلامِ الشَّعْبِ.
\par 10 فَقَالَ الرَّبُّ لِمُوسَى: «اذْهَبْ الَى الشَّعْبِ وَقَدِّسْهُمُ الْيَوْمَ وَغَدا وَلْيَغْسِلُوا ثِيَابَهُمْ
\par 11 وَيَكُونُوا مُسْتَعِدِّينَ لِلْيَوْمِ الثَّالِثِ. لانَّهُ فِي الْيَوْمِ الثَّالِثِ يَنْزِلُ الرَّبُّ امَامَ عُِيُونِ جَمِيعِ الشَّعْبِ عَلَى جَبَلِ سِينَاءَ.
\par 12 وَتُقِيمُ لِلشَّعْبِ حُدُودا مِنْ كُلِّ نَاحِيَةٍ قَائِلا: احْتَرِزُوا مِنْ انْ تَصْعَدُوا الَى الْجَبَلِ اوْ تَمَسُّوا طَرَفَهُ. كُلُّ مَنْ يَمَسُّ الْجَبَلَ يُقْتَلُ قَتْلا.
\par 13 لا تَمَسُّهُ يَدٌ بَلْ يُرْجَمُ رَجْما اوْ يُرْمَى رَمْيا. بَهِيمَةً كَانَ امْ انْسَانا لا يَعِيشُ. امَّا عِنْدَ صَوْتِ الْبُوقِ فَهُمْ يَصْعَدُونَ الَى الْجَبَلِ».
\par 14 فَانْحَدَرَ مُوسَى مِنَ الْجَبَلِ الَى الشَّعْبِ وَقَدَّسَ الشَّعْبَ وَغَسَلُوا ثِيَابَهُمْ.
\par 15 وَقَالَ لِلشَّعْبِ: «كُونُوا مُسْتَعِدِّينَ لِلْيَوْمِ الثَّالِثِ. لا تَقْرُبُوا امْرَاةً».
\par 16 وَحَدَثَ فِي الْيَوْمِ الثَّالِثِ لَمَّا كَانَ الصَّبَاحُ انَّهُ صَارَتْ رُعُودٌ وَبُرُوقٌ وَسَحَابٌ ثَقِيلٌ عَلَى الْجَبَلِ وَصَوْتُ بُوقٍ شَدِيدٌ جِدّا. فَارْتَعَدَ كُلُّ الشَّعْبِ الَّذِي فِي الْمَحَلَّةِ.
\par 17 وَاخْرَجَ مُوسَى الشَّعْبَ مِنَ الْمَحَلَّةِ لِمُلاقَاةِ اللهِ فَوَقَفُوا فِي اسْفَلِ الْجَبَلِ.
\par 18 وَكَانَ جَبَلُ سِينَاءَ كُلُّهُ يُدَخِّنُ مِنْ اجْلِ انَّ الرَّبَّ نَزَلَ عَلَيْهِ بِالنَّارِ وَصَعِدَ دُخَانُهُ كَدُخَانِ الاتُونِ وَارْتَجَفَ كُلُّ الْجَبَلِ جِدّا.
\par 19 فَكَانَ صَوْتُ الْبُوقِ يَزْدَادُ اشْتِدَادا جِدّا وَمُوسَى يَتَكَلَّمُ وَاللهُ يُجِيبُهُ بِصَوْتٍ.
\par 20 وَنَزَلَ الرَّبُّ عَلَى جَبَلِ سِينَاءَ الَى رَاسِ الْجَبَلِ وَدَعَا اللهُ مُوسَى الَى رَاسِ الْجَبَلِ. فَصَعِدَ مُوسَى.
\par 21 فَقَالَ الرَّبُّ لِمُوسَى: «انْحَدِرْ حَذِّرِ الشَّعْبَ لِئَلا يَقْتَحِمُوا الَى الرَّبِّ لِيَنْظُرُوا فَيَسْقُطَ مِنْهُمْ كَثِيرُونَ.
\par 22 وَلْيَتَقَدَّسْ ايْضا الْكَهَنَةُ الَّذِينَ يَقْتَرِبُونَ الَى الرَّبِّ لِئَلا يَبْطِشَ بِهِمِ الرَّبُّ».
\par 23 فَقَالَ مُوسَى لِلرَّبِّ: «لا يَقْدِرُ الشَّعْبُ انْ يَصْعَدَ الَى جَبَلِ سِينَاءَ لانَّكَ انْتَ حَذَّرْتَنَا قَائِلا: اقِمْ حُدُودا لِلْجَبَلِ وَقَدِّسْهُ».
\par 24 فَقَالَ لَهُ الرَّبُّ: «اذْهَبِ انْحَدِرْ ثُمَّ اصْعَدْ انْتَ وَهَارُونُ مَعَكَ. وَامَّا الْكَهَنَةُ وَالشَّعْبُ فَلا يَقْتَحِمُوا لِيَصْعَدُوا الَى الرَّبِّ لِئَلا يَبْطِشَ بِهِمْ».
\par 25 فَانْحَدَرَ مُوسَى الَى الشَّعْبِ وَقَالَ لَهُمْ.

\chapter{20}

\par 1 ثُمَّ تَكَلَّمَ اللهُ بِجَمِيعِ هَذِهِ الْكَلِمَاتِ:
\par 2 «انَا الرَّبُّ الَهُكَ الَّذِي اخْرَجَكَ مِنْ ارْضِ مِصْرَ مِنْ بَيْتِ الْعُبُودِيَّةِ.
\par 3 لا يَكُنْ لَكَ الِهَةٌ اخْرَى امَامِي.
\par 4 لا تَصْنَعْ لَكَ تِمْثَالا مَنْحُوتا وَلا صُورَةً مَا مِمَّا فِي السَّمَاءِ مِنْ فَوْقُ وَمَا فِي الارْضِ مِنْ تَحْتُ وَمَا فِي الْمَاءِ مِنْ تَحْتِ الارْضِ.
\par 5 لا تَسْجُدْ لَهُنَّ وَلا تَعْبُدْهُنَّ لانِّي انَا الرَّبَّ الَهَكَ الَهٌ غَيُورٌ افْتَقِدُ ذُنُوبَ الابَاءِ فِي الابْنَاءِ فِي الْجِيلِ الثَّالِثِ وَالرَّابِعِ مِنْ مُبْغِضِيَّ
\par 6 وَاصْنَعُ احْسَانا الَى الُوفٍ مِنْ مُحِبِّيَّ وَحَافِظِي وَصَايَايَ.
\par 7 لا تَنْطِقْ بِاسْمِ الرَّبِّ الَهِكَ بَاطِلا لانَّ الرَّبَّ لا يُبْرِئُ مَنْ نَطَقَ بِاسْمِهِ بَاطِلا.
\par 8 اذْكُرْ يَوْمَ السَّبْتِ لِتُقَدِّسَهُ.
\par 9 سِتَّةَ ايَّامٍ تَعْمَلُ وَتَصْنَعُ جَمِيعَ عَمَلِكَ
\par 10 وَامَّا الْيَوْمُ السَّابِعُ فَفِيهِ سَبْتٌ لِلرَّبِّ الَهِكَ. لا تَصْنَعْ عَمَلا مَا انْتَ وَابْنُكَ وَابْنَتُكَ وَعَبْدُكَ وَامَتُكَ وَبَهِيمَتُكَ وَنَزِيلُكَ الَّذِي دَاخِلَ ابْوَابِكَ -
\par 11 لانْ فِي سِتَّةِ ايَّامٍ صَنَعَ الرَّبُّ السَّمَاءَ وَالارْضَ وَالْبَحْرَ وَكُلَّ مَا فِيهَا وَاسْتَرَاحَ فِي الْيَوْمِ السَّابِعِ. لِذَلِكَ بَارَكَ الرَّبُّ يَوْمَ السَّبْتِ وَقَدَّسَهُ.
\par 12 اكْرِمْ ابَاكَ وَامَّكَ لِتَطُولَ ايَّامُكَ عَلَى الارْضِ الَّتِي يُعْطِيكَ الرَّبُّ الَهُكَ.
\par 13 لا تَقْتُلْ.
\par 14 لا تَزْنِ.
\par 15 لا تَسْرِقْ
\par 16 لا تَشْهَدْ عَلَى قَرِيبِكَ شَهَادَةَ زُورٍ.
\par 17 لا تَشْتَهِ بَيْتَ قَرِيبِكَ. لا تَشْتَهِ امْرَاةَ قَرِيبِكَ وَلا عَبْدَهُ وَلا امَتَهُ وَلا ثَوْرَهُ وَلا حِمَارَهُ وَلا شَيْئا مِمَّا لِقَرِيبِكَ».
\par 18 وَكَانَ جَمِيعُ الشَّعْبِ يَرُونَ الرُّعُودَ وَالْبُرُوقَ وَصَوْتَ الْبُوقِ وَالْجَبَلَ يُدَخِّنُ. وَلَمَّا رَاى الشَّعْبُ ارْتَعَدُوا وَوَقَفُوا مِنْ بَعِيدٍ
\par 19 وَقَالُوا لِمُوسَى: «تَكَلَّمْ انْتَ مَعَنَا فَنَسْمَعَ. وَلا يَتَكَلَّمْ مَعَنَا اللهُ لِئَلا نَمُوتَ».
\par 20 فَقَالَ مُوسَى لِلشَّعْبِ: «لا تَخَافُوا. لانَّ اللهَ انَّمَا جَاءَ لِيَمْتَحِنَكُمْ وَلِتَكُونَ مَخَافَتُهُ امَامَ وُجُوهِكُمْ حَتَّى لا تُخْطِئُوا».
\par 21 فَوَقَفَ الشَّعْبُ مِنْ بَعِيدٍ وَامَّا مُوسَى فَاقْتَرَبَ الَى الضَّبَابِ حَيْثُ كَانَ اللهُ.
\par 22 فَقَالَ الرَّبُّ لِمُوسَى: «هَكَذَا تَقُولُ لِبَنِي اسْرَائِيلَ: انْتُمْ رَايْتُمْ انَّنِي مِنَ السَّمَاءِ تَكَلَّمْتُ مَعَكُمْ.
\par 23 لا تَصْنَعُوا مَعِي الِهَةَ فِضَّةٍ وَلا تَصْنَعُوا لَكُمْ الِهَةَ ذَهَبٍ.
\par 24 مَذْبَحا مِنْ تُرَابٍ تَصْنَعُ لِي وَتَذْبَحُ عَلَيْهِ مُحْرَقَاتِكَ وَذَبَائِحَ سَلامَتِكَ غَنَمَكَ وَبَقَرَكَ. فِي كُلِّ الامَاكِنِ الَّتِي فِيهَا اصْنَعُ لاسْمِي ذِكْرا اتِي الَيْكَ وَابَارِكُكَ.
\par 25 وَانْ صَنَعْتَ لِي مَذْبَحا مِنْ حِجَارَةٍ فَلا تَبْنِهِ مِنْهَا مَنْحُوتَةً. اذَا رَفَعْتَ عَلَيْهَا ازْمِيلَكَ تُدَنِّسُهَا.
\par 26 وَلا تَصْعَدْ بِدَرَجٍ الَى مَذْبَحِي كَيْ لا تَنْكَشِفَ عَوْرَتُكَ عَلَيْهِ.

\chapter{21}

\par 1 «وَهَذِهِ هِيَ الاحْكَامُ الَّتِي تَضَعُ امَامَهُمْ:
\par 2 اذَا اشْتَرَيْتَ عَبْدا عِبْرَانِيّا فَسِتَّ سِنِينَ يَخْدِمُ وَفِي السَّابِعَةِ يَخْرُجُ حُرّا مَجَّانا.
\par 3 انْ دَخَلَ وَحْدَهُ فَوَحْدَهُ يَخْرُجُ. انْ كَانَ بَعْلَ امْرَاةٍ تَخْرُجُ امْرَاتُهُ مَعَهُ.
\par 4 انْ اعْطَاهُ سَيِّدُهُ امْرَاةً وَوَلَدَتْ لَهُ بَنِينَ اوْ بَنَاتٍ فَالْمَرْاةُ وَاوْلادُهَا يَكُونُونَ لِسَيِّدِهِ وَهُوَ يَخْرُجُ وَحْدَهُ.
\par 5 وَلَكِنْ انْ قَالَ الْعَبْدُ: احِبُّ سَيِّدِي وَامْرَاتِي وَاوْلادِي. لا اخْرُجُ حُرّا
\par 6 يُقَدِّمُهُ سَيِّدُهُ الَى اللهِ وَيُقَرِّبُهُ الَى الْبَابِ اوْ الَى الْقَائِمَةِ وَيَثْقُبُ سَيِّدُهُ اذْنَهُ بِالْمِثْقَبِ فَيَخْدِمُهُ الَى الابَدِ.
\par 7 وَاذَا بَاعَ رَجُلٌ ابْنَتَهُ امَةً لا تَخْرُجُ كَمَا يَخْرُجُ الْعَبِيدُ.
\par 8 انْ قَبُحَتْ فِي عَيْنَيْ سَيِّدِهَا الَّذِي خَطَبَهَا لِنَفْسِهِ يَدَعُهَا تُفَكُّ. وَلَيْسَ لَهُ سُلْطَانٌ انْ يَبِيعَهَا لِقَوْمٍ اجَانِبَ لِغَدْرِهِ بِهَا.
\par 9 وَانْ خَطَبَهَا لابْنِهِ فَبِحَسَبِ حَقِّ الْبَنَاتِ يَفْعَلُ لَهَا.
\par 10 انِ اتَّخَذَ لِنَفْسِهِ اخْرَى لا يُنَقِّصُ طَعَامَهَا وَكِسْوَتَهَا وَمُعَاشَرَتَهَا.
\par 11 وَانْ لَمْ يَفْعَلْ لَهَا هَذِهِ الثَّلاثَ تَخْرُجُ مَجَّانا بِلا ثَمَنٍ.
\par 12 «مَنْ ضَرَبَ انْسَانا فَمَاتَ يُقْتَلُ قَتْلا.
\par 13 وَلَكِنَّ الَّذِي لَمْ يَتَعَمَّدْ بَلْ اوْقَعَ اللهُ فِي يَدِهِ فَانَا اجْعَلُ لَكَ مَكَانا يَهْرُبُ الَيْهِ.
\par 14 وَاذَا بَغَى انْسَانٌ عَلَى صَاحِبِهِ لِيَقْتُلَهُ بِغَدْرٍ فَمِنْ عِنْدِ مَذْبَحِي تَاخُذُهُ لِلْمَوْتِ.
\par 15 وَمَنْ ضَرَبَ ابَاهُ اوْ امَّهُ يُقْتَلُ قَتْلا.
\par 16 وَمَنْ سَرِقَ انْسَانا وَبَاعَهُ اوْ وُجِدَ فِي يَدِهِ يُقْتَلُ قَتْلا.
\par 17 وَمَنْ شَتَمَ ابَاهُ اوْ امَّهُ يُقْتَلُ قَتْلا.
\par 18 وَاذَا تَخَاصَمَ رَجُلانِ فَضَرَبَ احَدُهُمَا الاخَرَ بِحَجَرٍ اوْ بِلَكْمَةٍ وَلَمْ يُقْتَلْ بَلْ سَقَطَ فِي الْفِرَاشِ
\par 19 فَانْ قَامَ وَتَمَشَّى خَارِجا عَلَى عُكَّازِهِ يَكُونُ الضَّارِبُ بَرِيئا. الا انَّهُ يُعَوِّضُ عُطْلَتَهُ وَيُنْفِقُ عَلَى شِفَائِهِ.
\par 20 وَاذَا ضَرَبَ انْسَانٌ عَبْدَهُ اوْ امَتَهُ بِالْعَصَا فَمَاتَ تَحْتَ يَدِهِ يُنْتَقَمُ مِنْهُ.
\par 21 لَكِنْ انْ بَقِيَ يَوْما اوْ يَوْمَيْنِ لا يُنْتَقَمُ مِنْهُ لانَّهُ مَالُهُ.
\par 22 وَاذَا تَخَاصَمَ رِجَالٌ وَصَدَمُوا امْرَاةً حُبْلَى فَسَقَطَ وَلَدُهَا وَلَمْ تَحْصُلْ اذِيَّةٌ يُغَرَّمُ كَمَا يَضَعُ عَلَيْهِ زَوْجُ الْمَرْاةِ وَيَدْفَعُ عَنْ يَدِ الْقُضَاةِ.
\par 23 وَانْ حَصَلَتْ اذِيَّةٌ تُعْطِي نَفْسا بِنَفْسٍ
\par 24 وَعَيْنا بِعَيْنٍ وَسِنّا بِسِنٍّ وَيَدا بِيَدٍ وَرِجْلا بِرِجْلٍ
\par 25 وَكَيّا بِكَيٍّ وَجُرْحا بِجُرْحٍ وَرَضّا بِرَضٍّ.
\par 26 وَاذَا ضَرَبَ انْسَانٌ عَيْنَ عَبْدِهِ اوْ عَيْنَ امَتِهِ فَاتْلَفَهَا يُطْلِقُهُ حُرّا عِوَضا عَنْ عَيْنِهِ.
\par 27 وَانْ اسْقَطَ سِنَّ عَبْدِهِ اوْ سِنَّ امَتِهِ يُطْلِقُهُ حُرّا عِوَضا عَنْ سِنِّهِ.
\par 28 «وَاذَا نَطَحَ ثَوْرٌ رَجُلا اوِ امْرَاةً فَمَاتَ يُرْجَمُ الثَّوْرُ وَلا يُؤْكَلُ لَحْمُهُ. وَامَّا صَاحِبُ الثَّوْرِ فَيَكُونُ بَرِيئا.
\par 29 وَلَكِنْ انْ كَانَ ثَوْرا نَطَّاحا مِنْ قَبْلُ وَقَدْ اشْهِدَ عَلَى صَاحِبِهِ وَلَمْ يَضْبِطْهُ فَقَتَلَ رَجُلا اوِ امْرَاةً فَالثَّوْرُ يُرْجَمُ وَصَاحِبُهُ ايْضا يُقْتَلُ.
\par 30 انْ وُضِعَتْ عَلَيْهِ فِدْيَةٌ يَدْفَعُ فِدَاءَ نَفْسِهِ كُلُّ مَا يُوضَعُ عَلَيْهِ.
\par 31 اوْ اذَا نَطَحَ ابْنا اوْ نَطَحَ ابْنَةً فَبِحَسَبِ هَذَا الْحُكْمِ يُفْعَلُ بِهِ.
\par 32 انْ نَطَحَ الثَّوْرُ عَبْدا اوْ امَةً يُعْطِي سَيِّدَهُ ثَلاثِينَ شَاقِلَ فِضَّةٍ وَالثَّوْرُ يُرْجَمُ.
\par 33 وَاذَا فَتَحَ انْسَانٌ بِئْرا اوْ حَفَرَ انْسَانٌ بِئْرا وَلَمْ يُغَطِّهِ فَوَقَعَ فِيهِ ثَوْرٌ اوْ حِمَارٌ
\par 34 فَصَاحِبُ الْبِئْرِ يُعَوِّضُ وَيَرُدُّ فِضَّةً لِصَاحِبِهِ وَالْمَيِّتُ يَكُونُ لَهُ.
\par 35 وَاذَا نَطَحَ ثَوْرُ انْسَانٍ ثَوْرَ صَاحِبِهِ فَمَاتَ يَبِيعَانِ الثَّوْرَ الْحَيَّ وَيَقْتَسِمَانِ ثَمَنَهُ. وَالْمَيِّتُ ايْضا يَقْتَسِمَانِهِ.
\par 36 لَكِنْ اذَا عُلِمَ انَّهُ ثَوْرٌ نَطَّاحٌ مِنْ قَبْلُ وَلَمْ يَضْبِطْهُ صَاحِبُهُ يُعَوِّضُ عَنِ الثَّوْرِ بِثَوْرٍ وَالْمَيِّتُ يَكُونُ لَهُ.

\chapter{22}

\par 1 «اذَا سَرِقَ انْسَانٌ ثَوْرا اوْ شَاةً فَذَبَحَهُ اوْ بَاعَهُ يُعَوِّضُ عَنِ الثَّوْرِ بِخَمْسَةِ ثِيرَانٍ وَعَنِ الشَّاةِ بِارْبَعَةٍ مِنَ الْغَنَمِ.
\par 2 انْ وُجِدَ السَّارِقُ وَهُوَ يَنْقُبُ فَضُرِبَ وَمَاتَ فَلَيْسَ لَهُ دَمٌ.
\par 3 وَلَكِنْ انْ اشْرَقَتْ عَلَيْهِ الشَّمْسُ فَلَهُ دَمٌ. انَّهُ يُعَوِّضُ. انْ لَمْ يَكُنْ لَهُ يُبَعْ بِسِرْقَتِهِ.
\par 4 انْ وُجِدَتِ السِّرْقَةُ فِي يَدِهِ حَيَّةً ثَوْرا كَانَتْ امْ حِمَارا امْ شَاةً يُعَوِّضُ بِاثْنَيْنِ.
\par 5 «اذَا رَعَى انْسَانٌ حَقْلا اوْ كَرْما وَسَرَّحَ مَوَاشِيَهُ فَرَعَتْ فِي حَقْلِ غَيْرِهِ فَمِنْ اجْوَدِ حَقْلِهِ وَاجْوَدِ كَرْمِهِ يُعَوِّضُ.
\par 6 اذَا خَرَجَتْ نَارٌ وَاصَابَتْ شَوْكا فَاحْتَرَقَتْ اكْدَاسٌ اوْ زَرْعٌ اوْ حَقْلٌ فَالَّذِي اوْقَدَ الْوَقِيدَ يُعَوِّضُ.
\par 7 اذَا اعْطَى انْسَانٌ صَاحِبَهُ فِضَّةً اوْ امْتِعَةً لِلْحِفْظِ فَسُرِقَتْ مِنْ بَيْتِ الْانْسَانِ فَانْ وُجِدَ السَّارِقُ يُعَوِّضُ بِاثْنَيْنِ.
\par 8 وَانْ لَمْ يُوجَدِ السَّارِقُ يُقَدَّمُ صَاحِبُ الْبَيْتِ الَى اللهِ لِيَحْكُمَ هَلْ لَمْ يَمُدَّ يَدَهُ الَى مُلْكِ صَاحِبِهِ.
\par 9 فِي كُلِّ دَعْوَى جِنَايَةٍ مِنْ جِهَةِ ثَوْرٍ اوْ حِمَارٍ اوْ شَاةٍ اوْ ثَوْبٍ اوْ مَفْقُودٍ مَا يُقَالُ: «انَّ هَذَا هُوَ» تُقَدَّمُ الَى اللهِ دَعْوَاهُمَا. فَالَّذِي يَحْكُمُ اللهُ بِذَنْبِهِ يُعَوِّضُ صَاحِبَهُ بِاثْنَيْنِ.
\par 10 اذَا اعْطَى انْسَانٌ صَاحِبَهُ حِمَارا اوْ ثَوْرا اوْ شَاةً اوْ بَهِيمَةً مَا لِلْحِفْظِ فَمَاتَ اوِ انْكَسَرَ اوْ نُهِبَ وَلَيْسَ نَاظِرٌ
\par 11 فَيَمِينُ الرَّبِّ تَكُونُ بَيْنَهُمَا هَلْ لَمْ يَمُدَّ يَدَهُ الَى مُلْكِ صَاحِبِهِ. فَيَقْبَلُ صَاحِبُهُ. فَلا يُعَوِّضُ.
\par 12 وَانْ سُرِقَ مِنْ عِنْدِهِ يُعَوِّضُ صَاحِبَهُ.
\par 13 انِ افْتُرِسَ يُحْضِرُهُ شَهَادَةً. لا يُعَوِّضُ عَنِ الْمُفْتَرَسِ.
\par 14 وَاذَا اسْتَعَارَ انْسَانٌ مِنْ صَاحِبِهِ شَيْئا فَانْكَسَرَ اوْ مَاتَ وَصَاحِبُهُ لَيْسَ مَعَهُ يُعَوِّضُ.
\par 15 وَانْ كَانَ صَاحِبُهُ مَعَهُ لا يُعَوِّضُ. انْ كَانَ مُسْتَاجَرا اتَى بِاجْرَتِهِ.
\par 16 «وَاذَا رَاوَدَ رَجُلٌ عَذْرَاءَ لَمْ تُخْطَبْ فَاضْطَجَعَ مَعَهَا يَمْهُرُهَا لِنَفْسِهِ زَوْجَةً.
\par 17 انْ ابَى ابُوهَا انْ يُعْطِيَهُ ايَّاهَا يَزِنُ لَهُ فِضَّةً كَمَهْرِ الْعَذَارَى.
\par 18 لا تَدَعْ سَاحِرَةً تَعِيشُ.
\par 19 كُلُّ مَنِ اضْطَجَعَ مَعَ بَهِيمَةٍ يُقْتَلُ قَتْلا.
\par 20 مَنْ ذَبَحَ لِالِهَةٍ غَيْرِ الرَّبِّ وَحْدَهُ يُهْلَكُ.
\par 21 «وَلا تَضْطَهِدِ الْغَرِيبَ وَلا تُضَايِقْهُ لانَّكُمْ كُنْتُمْ غُرَبَاءَ فِي ارْضِ مِصْرَ.
\par 22 لا تُسِئْ الَى ارْمَلَةٍ مَا وَلا يَتِيمٍ.
\par 23 انْ اسَاتَ الَيْهِ فَانِّي انْ صَرَخَ الَيَّ اسْمَعُ صُرَاخَهُ
\par 24 فَيَحْمَى غَضَبِي وَاقْتُلُكُمْ بِالسَّيْفِ فَتَصِيرُ نِسَاؤُكُمْ ارَامِلَ وَاوْلادُكُمْ يَتَامَى.
\par 25 انْ اقْرَضْتَ فِضَّةً لِشَعْبِي الْفَقِيرِ الَّذِي عِنْدَكَ فَلا تَكُنْ لَهُ كَالْمُرَابِي. لا تَضَعُوا عَلَيْهِ رِبا.
\par 26 انِ ارْتَهَنْتَ ثَوْبَ صَاحِبِكَ فَالَى غُرُوبِ الشَّمْسِ تَرُدُّهُ لَهُ
\par 27 لانَّهُ وَحْدَهُ غِطَاؤُهُ. هُوَ ثَوْبُهُ لِجِلْدِهِ. فِي مَاذَا يَنَامُ؟ فَيَكُونُ اذَا صَرَخَ الَيَّ انِّي اسْمَعُ لانِّي رَاوفٌ.
\par 28 «لا تَسُبَّ اللهَ وَلا تَلْعَنْ رَئِيسا فِي شَعْبِكَ.
\par 29 لا تُؤَخِّرْ مِلْءَ بَيْدَرِكَ وَقَطْرَ مِعْصَرَتِكَ وَابْكَارَ بَنِيكَ تُعْطِينِي.
\par 30 كَذَلِكَ تَفْعَلُ بِبَقَرِكَ وَغَنَمِكَ. سَبْعَةَ ايَّامٍ يَكُونُ مَعَ امِّهِ وَفِي الْيَوْمِ الثَّامِنِ تُعْطِينِي ايَّاهُ.
\par 31 وَتَكُونُونَ لِي انَاسا مُقَدَّسِينَ. وَلَحْمَ فَرِيسَةٍ فِي الصَّحْرَاءِ لا تَاكُلُوا. لِلْكِلابِ تَطْرَحُونَهُ.

\chapter{23}

\par 1 «لا تَقْبَلْ خَبَرا كَاذِبا. وَلا تَضَعْ يَدَكَ مَعَ الْمُنَافِقِ لِتَكُونَ شَاهِدَ ظُلْمٍ.
\par 2 لا تَتْبَعِ الْكَثِيرِينَ الَى فَِعْلِ الشَّرِّ وَلا تُجِبْ فِي دَعْوَى مَائِلا وَرَاءَ الْكَثِيرِينَ لِلتَّحْرِيفِ.
\par 3 وَلا تُحَابِ مَعَ الْمِسْكِينِ فِي دَعْوَاهُ.
\par 4 اذَا صَادَفْتَ ثَوْرَ عَدُوِّكَ اوْ حِمَارَهُ شَارِدا تَرُدُّهُ الَيْهِ.
\par 5 اذَا رَايْتَ حِمَارَ مُبْغِضِكَ وَاقِعا تَحْتَ حِمْلِهِ وَعَدَلْتَ عَنْ حَلِّهِ فَلا بُدَّ انْ تَحِلَّ مَعَهُ.
\par 6 لا تُحَرِّفْ حَقَّ فَقِيرِكَ فِي دَعْوَاهُ.
\par 7 ابْتَعِدْ عَنْ كَلامِ الْكَذِبِ وَلا تَقْتُلِ الْبَرِيءَ وَالْبَارَّ لانِّي لا ابَرِّرُ الْمُذْنِبَ.
\par 8 وَلا تَاخُذْ رَشْوَةً لانَّ الرَّشْوَةَ تُعْمِي الْمُبْصِرِينَ وَتُعَوِّجُ كَلامَ الابْرَارِ.
\par 9 وَلا تُضَايِقِ الْغَرِيبَ فَانَّكُمْ عَارِفُونَ نَفْسَ الْغَرِيبِ لانَّكُمْ كُنْتُمْ غُرَبَاءَ فِي ارْضِ مِصْرَ.
\par 10 وَسِتَّ سِنِينَ تَزْرَعُ ارْضَكَ وَتَجْمَعُ غَلَّتَهَا
\par 11 وَامَّا فِي السَّابِعَةِ فَتُرِيحُهَا وَتَتْرُكُهَا لِيَاكُلَ فُقَرَاءُ شَعْبِكَ. وَفَضْلَتُهُمْ تَاكُلُهَا وُحُوشُ الْبَرِّيَّةِ. كَذَلِكَ تَفْعَلُ بِكَرْمِكَ وَزَيْتُونِكَ.
\par 12 سِتَّةَ ايَّامٍ تَعْمَلُ عَمَلَكَ. وَامَّا الْيَوْمُ السَّابِعُ فَفِيهِ تَسْتَرِيحُ لِيَسْتَرِيحَ ثَوْرُكَ وَحِمَارُكَ وَيَتَنَفَّسَ ابْنُ امَتِكَ وَالْغَرِيبُ.
\par 13 وَكُلُّ مَا قُلْتُ لَكُمُ احْتَفِظُوا بِهِ. وَلا تَذْكُرُوا اسْمَ الِهَةٍ اخْرَى وَلا يُسْمَعْ مِنْ فَمِكَ.
\par 14 «ثَلاثَ مَرَّاتٍ تُعَيِّدُ لِي فِي السَّنَةِ.
\par 15 تَحْفَظُ عِيدَ الْفَطِيرِ. تَاكُلُ فَطِيرا سَبْعَةَ ايَّامٍ كَمَا امَرْتُكَ فِي وَقْتِ شَهْرِ ابِيبَ لانَّهُ فِيهِ خَرَجْتَ مِنْ مِصْرَ. وَلا يَظْهَرُوا امَامِي فَارِغِينَ.
\par 16 وَعِيدَ الْحَصَادِ ابْكَارِ غَلاتِكَ الَّتِي تَزْرَعُ فِي الْحَقْلِ. وَعِيدَ الْجَمْعِ فِي نِهَايَةِ السَّنَةِ عِنْدَمَا تَجْمَعُ غَلاتِكَ مِنَ الْحَقْلِ.
\par 17 ثَلاثَ مَرَّاتٍ فِي السَّنَةِ يَظْهَرُ جَمِيعُ ذُكُورِكَ امَامَ السَّيِّدِ الرَّبِّ.
\par 18 لا تَذْبَحْ عَلَى خَمِيرٍ دَمَ ذَبِيحَتِي. وَلا يَبِتْ شَحْمُ عِيدِي الَى الْغَدِ.
\par 19 اوَّلَ ابْكَارِ ارْضِكَ تُحْضِرُهُ الَى بَيْتِ الرَّبِّ الَهِكَ. لا تَطْبُخْ جَدْيا بِلَبَنِ امِّهِ.
\par 20 «هَا انَا مُرْسِلٌ مَلاكا امَامَ وَجْهِكَ لِيَحْفَظَكَ فِي الطَّرِيقِ وَلِيَجِيءَ بِكَ الَى الْمَكَانِ الَّذِي اعْدَدْتُهُ.
\par 21 احْتَرِزْ مِنْهُ وَاسْمَعْ لِصَوْتِهِ وَلا تَتَمَرَّدْ عَلَيْهِ لانَّهُ لا يَصْفَحُ عَنْ ذُنُوبِكُمْ لانَّ اسْمِي فِيهِ.
\par 22 وَلَكِنْ انْ سَمِعْتَ لِصَوْتِهِ وَفَعَلْتَ كُلَّ مَا اتَكَلَّمُ بِهِ اعَادِي اعْدَاءَكَ وَاضَايِقُ مُضَايِقِيكَ.
\par 23 فَانَّ مَلاكِي يَسِيرُ امَامَكَ وَيَجِيءُ بِكَ الَى الامُورِيِّينَ وَالْحِثِّيِّينَ وَالْفِرِزِّيِّينَ وَالْكَنْعَانِيِّينَ وَالْحِوِّيِّينَ وَالْيَبُوسِيِّينَ. فَابِيدُهُمْ.
\par 24 لا تَسْجُدْ لِالِهَتِهِمْ وَلا تَعْبُدْهَا وَلا تَعْمَلْ كَاعْمَالِهِمْ بَلْ تُبِيدُهُمْ وَتَكْسِرُ انْصَابَهُمْ.
\par 25 وَتَعْبُدُونَ الرَّبَّ الَهَكُمْ فَيُبَارِكُ خُبْزَكَ وَمَاءَكَ وَازِيلُ الْمَرَضَ مِنْ بَيْنِكُمْ.
\par 26 لا تَكُونُ مُسْقِطَةٌ وَلا عَاقِرٌ فِي ارْضِكَ. وَاكَمِّلُ عَدَدَ ايَّامِكَ.
\par 27 ارْسِلُ هَيْبَتِي امَامَكَ وَازْعِجُ جَمِيعَ الشُّعُوبِ الَّذِينَ تَاتِي عَلَيْهِمْ وَاعْطِيكَ جَمِيعَ اعْدَائِكَ مُدْبِرِينَ.
\par 28 وَارْسِلُ امَامَكَ الزَّنَابِيرَ فَتَطْرُدُ الْحِوِّيِّينَ وَالْكَنْعَانِيِّينَ وَالْحِثِّيِّينَ مِنْ امَامِكَ.
\par 29 لا اطْرُدُهُمْ مِنْ امَامِكَ فِي سَنَةٍ وَاحِدَةٍ لِئَلا تَصِيرَ الارْضُ خَرِبَةً فَتَكْثُرَ عَلَيْكَ وُحُوشُ الْبَرِّيَّةِ.
\par 30 قَلِيلا قَلِيلا اطْرُدُهُمْ مِنْ امَامِكَ الَى انْ تُثْمِرَ وَتَمْلِكَ الارْضَ.
\par 31 وَاجْعَلُ تُخُومَكَ مِنْ بَحْرِ سُوفٍ الَى بَحْرِ فَلَسْطِينَ وَمِنَ الْبَرِّيَّةِ الَى النَّهْرِ. فَانِّي ادْفَعُ الَى ايْدِيكُمْ سُكَّانَ الارْضِ فَتَطْرُدُهُمْ مِنْ امَامِكَ.
\par 32 لا تَقْطَعْ مَعَهُمْ وَلا مَعَ الِهَتِهِمْ عَهْدا.
\par 33 لا يَسْكُنُوا فِي ارْضِكَ لِئَلا يَجْعَلُوكَ تُخْطِئُ الَيَّ. اذَا عَبَدْتَ الِهَتَهُمْ فَانَّهُ يَكُونُ لَكَ فَخّا».

\chapter{24}

\par 1 وَقَالَ لِمُوسَى: «اصْعَدْ الَى الرَّبِّ انْتَ وَهَارُونُ وَنَادَابُ وَابِيهُو وَسَبْعُونَ مِنْ شُيُوخِ اسْرَائِيلَ وَاسْجُدُوا مِنْ بَعِيدٍ.
\par 2 وَيَقْتَرِبُ مُوسَى وَحْدَهُ الَى الرَّبِّ وَهُمْ لا يَقْتَرِبُونَ. وَامَّا الشَّعْبُ فَلا يَصْعَدْ مَعَهُ».
\par 3 فَجَاءَ مُوسَى وَحَدَّثَ الشَّعْبَ بِجَمِيعِ اقْوَالِ الرَّبِّ وَجَمِيعِ الاحْكَامِ فَاجَابَ جَمِيعُ الشَّعْبِ بِصَوْتٍ وَاحِدٍ: «كُلُّ الاقْوَالِ الَّتِي تَكَلَّمَ بِهَا الرَّبُّ نَفْعَلُ».
\par 4 فَكَتَبَ مُوسَى جَمِيعَ اقْوَالِ الرَّبِّ. وَبَكَّرَ فِي الصَّبَاحِ وَبَنَى مَذْبَحا فِي اسْفَلِ الْجَبَلِ وَاثْنَيْ عَشَرَ عَمُودا لاسْبَاطِ اسْرَائِيلَ الاثْنَيْ عَشَرَ.
\par 5 وَارْسَلَ فِتْيَانَ بَنِي اسْرَائِيلَ فَاصْعَدُوا مُحْرَقَاتٍ وَذَبَحُوا ذَبَائِحَ سَلامَةٍ لِلرَّبِّ مِنَ الثِّيرَانِ.
\par 6 فَاخَذَ مُوسَى نِصْفَ الدَّمِ وَوَضَعَهُ فِي الطُّسُوسِ. وَنِصْفَ الدَّمِ رَشَّهُ عَلَى الْمَذْبَحِ.
\par 7 وَاخَذَ كِتَابَ الْعَهْدِ وَقَرَا فِي مَسَامِعِ الشَّعْبِ. فَقَالُوا: «كُلُّ مَا تَكَلَّمَ بِهِ الرَّبُّ نَفْعَلُ وَنَسْمَعُ لَهُ».
\par 8 وَاخَذَ مُوسَى الدَّمَ وَرَشَّ عَلَى الشَّعْبِ وَقَالَ: «هُوَذَا دَمُ الْعَهْدِ الَّذِي قَطَعَهُ الرَّبُّ مَعَكُمْ عَلَى جَمِيعِ هَذِهِ الاقْوَالِ».
\par 9 ثُمَّ صَعِدَ مُوسَى وَهَارُونُ وَنَادَابُ وَابِيهُو وَسَبْعُونَ مِنْ شُيُوخِ اسْرَائِيلَ
\par 10 وَرَاوا الَهَ اسْرَائِيلَ وَتَحْتَ رِجْلَيْهِ شِبْهُ صَنْعَةٍ مِنَ الْعَقِيقِ الازْرَقِ الشَّفَّافِ وَكَذَاتِ السَّمَاءِ فِي النَّقَاوَةِ.
\par 11 وَلَكِنَّهُ لَمْ يَمُدَّ يَدَهُ الَى اشْرَافِ بَنِي اسْرَائِيلَ. فَرَاوا اللهَ وَاكَلُوا وَشَرِبُوا.
\par 12 وَقَالَ الرَّبُّ لِمُوسَى: «اصْعَدْ الَيَّ الَى الْجَبَلِ وَكُنْ هُنَاكَ فَاعْطِيَكَ لَوْحَيِ الْحِجَارَةِ وَالشَّرِيعَةِ وَالْوَصِيَّةِ الَّتِي كَتَبْتُهَا لِتَعْلِيمِهِمْ».
\par 13 فَقَامَ مُوسَى وَيَشُوعُ خَادِمُهُ. وَصَعِدَ مُوسَى الَى جَبَلِ اللهِ.
\par 14 وَامَّا الشُّيُوخُ فَقَالَ لَهُمُ: «اجْلِسُوا لَنَا هَهُنَا حَتَّى نَرْجِعَ الَيْكُمْ. وَهُوَذَا هَارُونُ وَحُورُ مَعَكُمْ. فَمَنْ كَانَ صَاحِبَ دَعْوَى فَلْيَتَقَدَّمْ الَيْهِمَا».
\par 15 فَصَعِدَ مُوسَى الَى الْجَبَلِ فَغَطَّى السَّحَابُ الْجَبَلَ
\par 16 وَحَلَّ مَجْدُ الرَّبِّ عَلَى جَبَلِ سِينَاءَ وَغَطَّاهُ السَّحَابُ سِتَّةَ ايَّامٍ. وَفِي الْيَوْمِ السَّابِعِ دُعِيَ مُوسَى مِنْ وَسَطِ السَّحَابِ.
\par 17 وَكَانَ مَنْظَرُ مَجْدِ الرَّبِّ كَنَارٍ اكِلَةٍ عَلَى رَاسِ الْجَبَلِ امَامَ عُيُونِ بَنِي اسْرَائِيلَ.
\par 18 وَدَخَلَ مُوسَى فِي وَسَطِ السَّحَابِ وَصَعِدَ الَى الْجَبَلِ. وَكَانَ مُوسَى فِي الْجَبَلِ ارْبَعِينَ نَهَارا وَارْبَعِينَ لَيْلَةً.

\chapter{25}

\par 1 وَقَالَ الرَّبُّ لِمُوسَى:
\par 2 «كَلِّمْ بَنِي اسْرَائِيلَ انْ يَاخُذُوا لِي تَقْدِمَةً. مِنْ كُلِّ مَنْ يَحِثُّهُ قَلْبُهُ تَاخُذُونَ تَقْدِمَتِي.
\par 3 وَهَذِهِ هِيَ التَّقْدِمَةُ الَّتِي تَاخُذُونَهَا مِنْهُمْ: ذَهَبٌ وَفِضَّةٌ وَنُحَاسٌ
\par 4 وَاسْمَانْجُونِيٌّ وَارْجُوَانٌ وَقِرْمِزٌ وَبُوصٌ وَشَعْرُ مِعْزَى
\par 5 وَجُلُودُ كِبَاشٍ مُحَمَّرَةٌ وَجُلُودُ تُخَسٍ وَخَشَبُ سَنْطٍ
\par 6 وَزَيْتٌ لِلْمَنَارَةِ وَاطْيَابٌ لِدُهْنِ الْمَسْحَةِ وَلِلْبَخُورِ الْعَطِرِ
\par 7 وَحِجَارَةُ جَزْعٍ وَحِجَارَةُ تَرْصِيعٍ لِلرِّدَاءِ وَالصُّدْرَةِ.
\par 8 فَيَصْنَعُونَ لِي مَقْدِسا لاسْكُنَ فِي وَسَطِهِمْ.
\par 9 بِحَسَبِ جَمِيعِ مَا انَا ارِيكَ مِنْ مِثَالِ الْمَسْكَنِ وَمِثَالِ جَمِيعِ انِيَتِهِ هَكَذَا تَصْنَعُونَ.
\par 10 «فَيَصْنَعُونَ تَابُوتا مِنْ خَشَبِ السَّنْطِ طُولُهُ ذِرَاعَانِ وَنِصْفٌ وَعَرْضُهُ ذِرَاعٌ وَنِصْفٌ وَارْتِفَاعُهُ ذِرَاعٌ وَنِصْفٌ.
\par 11 وَتُغَشِّيهِ بِذَهَبٍ نَقِيٍّ. مِنْ دَاخِلٍ وَمِنْ خَارِجٍ تُغَشِّيهِ. وَتَصْنَعُ عَلَيْهِ اكْلِيلا مِنْ ذَهَبٍ حَوَالَيْهِ.
\par 12 وَتَسْبِكُ لَهُ ارْبَعَ حَلَقَاتٍ مِنْ ذَهَبٍ وَتَجْعَلُهَا عَلَى قَوَائِمِهِ الارْبَعِ. عَلَى جَانِبِهِ الْوَاحِدِ حَلْقَتَانِ وَعَلَى جَانِبِهِ الثَّانِي حَلْقَتَانِ.
\par 13 وَتَصْنَعُ عَصَوَيْنِ مِنْ خَشَبِ السَّنْطِ وَتُغَشِّيهِمَا بِذَهَبٍ.
\par 14 وَتُدْخِلُ الْعَصَوَيْنِ فِي الْحَلَقَاتِ عَلَى جَانِبَيِ التَّابُوتِ لِيُحْمَلَ التَّابُوتُ بِهِمَا.
\par 15 تَبْقَى الْعَصَوَانِ فِي حَلَقَاتِ التَّابُوتِ. لا تُنْزَعَانِ مِنْهَا.
\par 16 وَتَضَعُ فِي التَّابُوتِ الشَّهَادَةَ الَّتِي اعْطِيكَ.
\par 17 «وَتَصْنَعُ غِطَاءً مِنْ ذَهَبٍ نَقِيٍّ طُولُهُ ذِرَاعَانِ وَنِصْفٌ وَعَرْضُهُ ذِرَاعُ وَنِصْفٌ
\par 18 وَتَصْنَعُ كَرُوبَيْنِ مِنْ ذَهَبٍ. صَنْعَةَ خِرَاطَةٍ تَصْنَعُهُمَا عَلَى طَرَفَيِ الْغِطَاءِ.
\par 19 فَاصْنَعْ كَرُوبا وَاحِدا عَلَى الطَّرَفِ مِنْ هُنَا وَكَرُوبا اخَرَ عَلَى الطَّرَفِ مِنْ هُنَاكَ. مِنَ الْغِطَاءِ تَصْنَعُونَ الْكَرُوبَيْنِ عَلَى طَرَفَيْهِ.
\par 20 وَيَكُونُ الْكَرُوبَانِ بَاسِطَيْنِ اجْنِحَتَهُمَا الَى فَوْقُ مُظَلِّلَيْنِ بِاجْنِحَتِهِمَا عَلَى الْغِطَاءِ وَوَجْهَاهُمَا كُلُّ وَاحِدٍ الَى الاخَرِ. نَحْوَ الْغِطَاءِ يَكُونُ وَجْهَا الْكَرُوبَيْنِ.
\par 21 وَتَجْعَلُ الْغِطَاءَ عَلَى التَّابُوتِ مِنْ فَوْقُ. وَفِي التَّابُوتِ تَضَعُ الشَّهَادَةَ الَّتِي اعْطِيكَ.
\par 22 وَانَا اجْتَمِعُ بِكَ هُنَاكَ وَاتَكَلَّمُ مَعَكَ مِنْ عَلَى الْغِطَاءِ مِنْ بَيْنِ الْكَرُوبَيْنِ اللَّذَيْنِ عَلَى تَابُوتِ الشَّهَادَةِ بِكُلِّ مَا اوصِيكَ بِهِ الَى بَنِي اسْرَائِيلَ.
\par 23 «وَتَصْنَعُ مَائِدَةً مِنْ خَشَبِ السَّنْطِ طُولُهَا ذِرَاعَانِ وَعَرْضُهَا ذِرَاعٌ وَارْتِفَاعُهَا ذِرَاعٌ وَنِصْفٌ.
\par 24 وَتُغَشِّيهَا بِذَهَبٍ نَقِيٍّ. وَتَصْنَعُ لَهَا اكْلِيلا مِنْ ذَهَبٍ حَوَالَيْهَا.
\par 25 وَتَصْنَعُ لَهَا حَاجِبا بِعَرْضِ شِبْرٍ حَوَالَيْهَا. وَتَصْنَعُ لِحَاجِبِهَا اكْلِيلا مِنْ ذَهَبٍ حَوَالَيْهَا.
\par 26 وَتَصْنَعُ لَهَا ارْبَعَ حَلَقَاتٍ مِنْ ذَهَبٍ وَتَجْعَلُ الْحَلَقَاتِ عَلَى الزَّوَايَا الارْبَعِ الَّتِي لِقَوَائِمِهَا الارْبَعِ.
\par 27 عِنْدَ الْحَاجِبِ تَكُونُ الْحَلَقَاتُ بُيُوتا لِعَصَوَيْنِ لِحَمْلِ الْمَائِدَةِ.
\par 28 وَتَصْنَعُ الْعَصَوَيْنِ مِنْ خَشَبِ السَّنْطِ وَتُغَشِّيهِمَا بِذَهَبٍ فَتُحْمَلُ بِهِمَا الْمَائِدَةُ.
\par 29 وَتَصْنَعُ صِحَافَهَا وَصُحُونَهَا وَكَاسَاتِهَا وَجَامَاتِهَا الَّتِي يُسْكَبُ بِهَا. مِنْ ذَهَبٍ نَقِيٍّ تَصْنَعُهَا.
\par 30 وَتَجْعَلُ عَلَى الْمَائِدَةِ خُبْزَ الْوُجُوهِ امَامِي دَائِما.
\par 31 «وَتَصْنَعُ مَنَارَةً مِنْ ذَهَبٍ نَقِيٍّ. عَمَلَ الْخِرَاطَةِ تُصْنَعُ الْمَنَارَةُ قَاعِدَتُهَا وَسَاقُهَا. تَكُونُ كَاسَاتُهَا وَعُجَرُهَا وَازْهَارُهَا مِنْهَا.
\par 32 وَسِتُّ شُعَبٍ خَارِجَةٌ مِنْ جَانِبَيْهَا. مِنْ جَانِبِهَا الْوَاحِدِ ثَلاثُ شُعَبِ مَنَارَةٍ. وَمِنْ جَانِبِهَا الثَّانِي ثَلاثُ شُعَبِ مَنَارَةٍ.
\par 33 فِي الشُّعْبَةِ الْوَاحِدَةِ ثَلاثُ كَاسَاتٍ لَوْزِيَّةٍ بِعُجْرَةٍ وَزَهْرٍ. وَفِي الشُّعْبَةِ الثَّانِيَةِ ثَلاثُ كَاسَاتٍ لَوْزِيَّةٍ بِعُجْرَةٍ وَزَهْرٍ. وَهَكَذَا الَى السِّتِّ الشُّعَبِ الْخَارِجَةِ مِنَ الْمَنَارَةِ.
\par 34 وَفِي الْمَنَارَةِ ارْبَعُ كَاسَاتٍ لَوْزِيَّةٍ بِعُجَرِهَا وَازْهَارِهَا.
\par 35 وَتَحْتَ الشُّعْبَتَيْنِ مِنْهَا عُجْرَةٌ وَتَحْتَ الشُّعْبَتَيْنِ مِنْهَا عُجْرَةٌ وَتَحْتَ الشُّعْبَتَيْنِ مِنْهَا عُجْرَةٌ الَى السِّتِّ الشُّعَبِ الْخَارِجَةِ مِنَ الْمَنَارَةِ.
\par 36 تَكُونُ عُجَرُهَا وَشُعَبُهَا مِنْهَا. جَمِيعُهَا خِرَاطَةٌ وَاحِدَةٌ مِنْ ذَهَبٍ نَقِيٍّ.
\par 37 وَتَصْنَعُ سُرُجَهَا سَبْعَةً. فَتُصْعَدُ سُرُجُهَا لِتُضِيءَ الَى مُقَابِلِهَا.
\par 38 وَمَلاقِطُهَا وَمَنَافِضُهَا مِنْ ذَهَبٍ نَقِيٍّ.
\par 39 مِنْ وَزْنَةِ ذَهَبٍ نَقِيٍّ تُصْنَعُ مَعَ جَمِيعِ هَذِهِ الاوَانِي.
\par 40 وَانْظُرْ فَاصْنَعْهَا عَلَى مِثَالِهَا الَّذِي اظْهِرَ لَكَ فِي الْجَبَلِ.

\chapter{26}

\par 1 «وَامَّا الْمَسْكَنُ فَتَصْنَعُهُ مِنْ عَشَرِ شُقَقِ بُوصٍ مَبْرُومٍ وَاسْمَانْجُونِيٍّ وَارْجُوَانٍ وَقِرْمِزٍ. بِكَرُوبِيمَ صَنْعَةَ حَائِكٍ حَاذِقٍ تَصْنَعُهَا.
\par 2 طُولُ الشُّقَّةِ الْوَاحِدَةِ ثَمَانٍ وَعِشْرُونَ ذِرَاعا وَعَرْضُ الشُّقَّةِ الْوَاحِدَةِ ارْبَعُ اذْرُعٍ. قِيَاسا وَاحِدا لِجَمِيعِ الشُّقَقِ.
\par 3 تَكُونُ خَمْسٌ مِنَ الشُّقَقِ بَعْضُهَا مَوْصُولٌ بِبَعْضٍ وَخَمْسُ شُقَقٍ بَعْضُهَا مَوْصُولٌ بِبَعْضٍ.
\par 4 وَتَصْنَعُ عُرًى مِنْ اسْمَانْجُونِيٍّ عَلَى حَاشِيَةِ الشُّقَّةِ الْوَاحِدَةِ فِي الطَّرَفِ مِنَ الْمُوَصَّلِ الْوَاحِدِ. وَكَذَلِكَ تَصْنَعُ فِي حَاشِيَةِ الشُّقَّةِ الطَّرَفِيَّةِ مِنَ الْمُوَصَّلِ الثَّانِي.
\par 5 خَمْسِينَ عُرْوَةً تَصْنَعُ فِي الشُّقَّةِ الْوَاحِدَةِ وَخَمْسِينَ عُرْوَةً تَصْنَعُ فِي طَرَفِ الشُّقَّةِ الَّذِي فِي الْمُوَصَّلِ الثَّانِي. تَكُونُ الْعُرَى بَعْضُهَا مُقَابِلٌ لِبَعْضٍ.
\par 6 وَتَصْنَعُ خَمْسِينَ شِظَاظا مِنْ ذَهَبٍ. وَتَصِلُ الشُّقَّتَيْنِ بَعْضَهُمَا بِبَعْضٍ بِالاشِظَّةِ. فَيَصِيرُ الْمَسْكَنُ وَاحِدا.
\par 7 «وَتَصْنَعُ شُقَقا مِنْ شَعْرِ مِعْزَى خَيْمَةً عَلَى الْمَسْكَنِ. احْدَى عَشَرَةَ شُقَّةً تَصْنَعُهَا.
\par 8 طُولُ الشُّقَّةِ الْوَاحِدَةِ ثَلاثُونَ ذِرَاعا وَعَرْضُ الشُّقَّةِ الْوَاحِدَةِ ارْبَعُ اذْرُعٍ. قِيَاسا وَاحِدا لِلْاحْدَى عَشَرَةَ شُقَّةً.
\par 9 وَتَصِلُ خَمْسا مِنَ الشُّقَقِ وَحْدَهَا وَسِتّا مِنَ الشُّقَقِ وَحْدَهَا. وَتَثْنِي الشُّقَّةَ السَّادِسَةَ فِي وَجْهِ الْخَيْمَةِ.
\par 10 وَتَصْنَعُ خَمْسِينَ عُرْوَةً عَلَى حَاشِيَةِ الشُّقَّةِ الْوَاحِدَةِ الطَّرَفِيَّةِ مِنَ الْمُوَصَّلِ الْوَاحِدِ وَخَمْسِينَ عُرْوَةً عَلَى حَاشِيَةِ الشُّقَّةِ مِنَ الْمُوَصَّلِ الثَّانِي.
\par 11 وَتَصْنَعُ خَمْسِينَ شِظَاظا مِنْ نُحَاسٍ. وَتُدْخِلُ الاشِظَّةَ فِي الْعُرَى وَتَصِلُ الْخَيْمَةَ فَتَصِيرُ وَاحِدَةً.
\par 12 وَامَّا الْمُدَلَّى الْفَاضِلُ مِنْ شُقَقِ الْخَيْمَةِ نِصْفُ الشُّقَّةِ الْمُوَصَّلَةِ الْفَاضِلُ فَيُدَلَّى عَلَى مُؤَخَّرِ الْمَسْكَنِ.
\par 13 وَالذِّرَاعُ مِنْ هُنَا وَالذِّرَاعُ مِنْ هُنَاكَ مِنَ الْفَاضِلِ فِي طُولِ شُقَقِ الْخَيْمَةِ تَكُونَانِ مُدَلاتَيْنِ عَلَى جَانِبَيِ الْمَسْكَنِ مِنْ هُنَا وَمِنْ هُنَاكَ لِتَغْطِيَتِهِ.
\par 14 وَتَصْنَعُ غِطَاءً لِلْخَيْمَةِ مِنْ جُلُودِ كِبَاشٍ مُحَمَّرَةٍ. وَغِطَاءً مِنْ جُلُودِ تُخَسٍ مِنْ فَوْقُ.
\par 15 «وَتَصْنَعُ الالْوَاحَ لِلْمَسْكَنِ مِنْ خَشَبِ السَّنْطِ قَائِمَةً.
\par 16 طُولُ اللَّوْحِ عَشَرُ اذْرُعٍ وَعَرْضُ اللَّوْحِ الْوَاحِدِ ذِرَاعٌ وَنِصْفٌ.
\par 17 وَلِلَّوْحِ الْوَاحِدِ رِجْلانِ مَقْرُونَةٌ احْدَاهُمَا بِالاخْرَى. هَكَذَا تَصْنَعُ لِجَمِيعِ الْوَاحِ الْمَسْكَنِ.
\par 18 وَتَصْنَعُ الالْوَاحَ لِلْمَسْكَنِ عِشْرِينَ لَوْحا الَى جِهَةِ الْجَنُوبِ نَحْوَ التَّيْمَنِ.
\par 19 وَتَصْنَعُ ارْبَعِينَ قَاعِدَةً مِنْ فِضَّةٍ تَحْتَ الْعِشْرِينَ لَوْحا. تَحْتَ اللَّوْحِ الْوَاحِدِ قَاعِدَتَانِ لِرِجْلَيْهِ وَتَحْتَ اللَّوْحِ الْوَاحِدِ قَاعِدَتَانِ لِرِجْلَيْهِ.
\par 20 وَلِجَانِبِ الْمَسْكَنِ الثَّانِي الَى جِهَةِ الشِّمَالِ عِشْرِينَ لَوْحا.
\par 21 وَارْبَعِينَ قَاعِدَةً لَهَا مِنْ فِضَّةٍ. تَحْتَ اللَّوْحِ الْوَاحِدِ قَاعِدَتَانِ وَتَحْتَ اللَّوْحِ الْوَاحِدِ قَاعِدَتَانِ.
\par 22 وَلِمُؤَخَّرِ الْمَسْكَنِ نَحْوَ الْغَرْبِ تَصْنَعُ سِتَّةَ الْوَاحٍ.
\par 23 وَتَصْنَعُ لَوْحَيْنِ لِزَاوِيَتَيِ الْمَسْكَنِ فِي الْمُؤَخَّرِ
\par 24 وَيَكُونَانِ مُزْدَوِجَيْنِ مِنْ اسْفَلُ. وَعَلَى سَوَاءٍ يَكُونَانِ مُزْدَوِجَيْنِ الَى رَاسِهِ الَى الْحَلَْقَةِ الْوَاحِدَةِ. هَكَذَا يَكُونُ لِكِلَيْهِمَا. يَكُونَانِ لِلزَّاوِيَتَيْنِ.
\par 25 فَتَكُونُ ثَمَانِيَةَ الْوَاحٍ وَقَوَاعِدُهَا مِنْ فِضَّةٍ سِتَّ عَشْرَةَ قَاعِدَةً. تَحْتَ اللَّوْحِ الْوَاحِدِ قَاعِدَتَانِ وَتَحْتَ اللَّوْحِ الْوَاحِدِ قَاعِدَتَانِ.
\par 26 «وَتَصْنَعُ عَوَارِضَ مِنْ خَشَبِ السَّنْطِ خَمْسا لالْوَاحِ جَانِبِ الْمَسْكَنِ الْوَاحِدِ
\par 27 وَخَمْسَ عَوَارِضَ لالْوَاحِ جَانِبِ الْمَسْكَنِ الثَّانِي وَخَمْسَ عَوَارِضَ لالْوَاحِ جَانِبِ الْمَسْكَنِ فِي الْمُؤَخَّرِ نَحْوَ الْغَرْبِ.
\par 28 وَالْعَارِضَةُ الْوُسْطَى فِي وَسَطِ الالْوَاحِ تَنْفُذُ مِنَ الطَّرَفِ الَى الطَّرَفِ.
\par 29 وَتُغَشِّي الالْوَاحَ بِذَهَبٍ. وَتَصْنَعُ حَلَقَاتِهَا مِنْ ذَهَبٍ بُيُوتا لِلْعَوَارِضِ. وَتُغَشِّي الْعَوَارِضَ بِذَهَبٍ.
\par 30 وَتُقِيمُ الْمَسْكَنَ كَرَسْمِهِ الَّذِي اظْهِرَ لَكَ فِي الْجَبَلِ.
\par 31 وَتَصْنَعُ حِجَابا مِنْ اسْمَانْجُونِيٍّ وَارْجُوَانٍ وَقِرْمِزٍ وَبُوصٍ مَبْرُومٍ. صَنْعَةَ حَائِكٍ حَاذِقٍ يَصْنَعُهُ بِكَرُوبِيمَ.
\par 32 وَتَجْعَلُهُ عَلَى ارْبَعَةِ اعْمِدَةٍ مِنْ سَنْطٍ مُغَشَّاةٍ بِذَهَبٍ. رُزَزُهَا مِنْ ذَهَبٍ. عَلَى ارْبَعِ قَوَاعِدَ مِنْ فِضَّةٍ.
\par 33 وَتَجْعَلُ الْحِجَابَ تَحْتَ الاشِظَّةِ. وَتُدْخِلُ الَى هُنَاكَ دَاخِلَ الْحِجَابِ تَابُوتَ الشَّهَادَةِ فَيَفْصِلُ لَكُمُ الْحِجَابُ بَيْنَ الْقُدْسِ وَقُدْسِ الاقْدَاسِ.
\par 34 وَتَجْعَلُ الْغِطَاءَ عَلَى تَابُوتِ الشَّهَادَةِ فِي قُدْسِ الاقْدَاسِ.
\par 35 وَتَضَعُ الْمَائِدَةَ خَارِجَ الْحِجَابِ وَالْمَنَارَةَ مُقَابِلَ الْمَائِدَةِ عَلَى جَانِبِ الْمَسْكَنِ نَحْوَ التَّيْمَنِ. وَتَجْعَلُ الْمَائِدَةَ عَلَى جَانِبِ الشِّمَالِ.
\par 36 «وَتَصْنَعُ سَجْفا لِمَدْخَلِ الْخَيْمَةِ مِنْ اسْمَانْجُونِيٍّ وَارْجُوَانٍ وَقِرْمِزٍ وَبُوصٍ مَبْرُومٍ صَنْعَةَ الطَّرَّازِ.
\par 37 وَتَصْنَعُ لِلسَّجْفِ خَمْسَةَ اعْمِدَةٍ مِنْ سَنْطٍ وَتُغَشِّيهَا بِذَهَبٍ. رُزَزُهَا مِنْ ذَهَبٍ. وَتَسْبِكُ لَهَا خَمْسَ قَوَاعِدَ مِنْ نُحَاسٍ.

\chapter{27}

\par 1 «وَتَصْنَعُ الْمَذْبَحَ مِنْ خَشَبِ السَّنْطِ طُولُهُ خَمْسُ اذْرُعٍ وَعَرْضُهُ خَمْسُ اذْرُعٍ. مُرَبَّعا يَكُونُ الْمَذْبَحُ. وَارْتِفَاعُهُ ثَلاثُ اذْرُعٍ.
\par 2 وَتَصْنَعُ قُرُونَهُ عَلَى زَوَايَاهُ الارْبَعِ. مِنْهُ تَكُونُ قُرُونُهُ. وَتُغَشِّيهِ بِنُحَاسٍ.
\par 3 وَتَصْنَعُ قُدُورَهُ لِرَفْعِ رَمَادِهِ وَرُفُوشَهُ وَمَرَاكِنَهُ وَمَنَاشِلَهُ وَمَجَامِرَهُ. جَمِيعَ انِيَتِهِ تَصْنَعُهَا مِنْ نُحَاسٍ.
\par 4 وَتَصْنَعُ لَهُ شُبَّاكَةً صَنْعَةَ الشَّبَكَةِ مِنْ نُحَاسٍ. وَتَصْنَعُ عَلَى الشَّبَكَةِ ارْبَعَ حَلَقَاتٍ مِنْ نُحَاسٍ عَلَى ارْبَعَةِ اطْرَافِهِ.
\par 5 وَتَجْعَلُهَا تَحْتَ حَاجِبِ الْمَذْبَحِ مِنْ اسْفَلُ. وَتَكُونُ الشَّبَكَةُ الَى نِصْفِ الْمَذْبَحِ.
\par 6 وَتَصْنَعُ عَصَوَيْنِ لِلْمَذْبَحِ عَصَوَيْنِ مِنْ خَشَبِ السَّنْطِ وَتُغَشِّيهِمَا بِنُحَاسٍ.
\par 7 وَتُدْخَلُ عَصَوَاهُ فِي الْحَلَقَاتِ. فَتَكُونُ الْعَصَوَانِ عَلَى جَانِبَيِ الْمَذْبَحِ حِينَمَا يُحْمَلُ.
\par 8 مُجَوَّفا تَصْنَعُهُ مِنْ الْوَاحٍ. كَمَا اظْهِرَ لَكَ فِي الْجَبَلِ هَكَذَا يَصْنَعُونَهُ.
\par 9 «وَتَصْنَعُ دَارَ الْمَسْكَنِ. الَى جِهَةِ الْجَنُوبِ نَحْوَ التَّيْمَنِ لِلدَّارِ اسْتَارٌ مِنْ بُوصٍ مَبْرُومٍ مِئَةُ ذِرَاعٍ طُولا الَى الْجِهَةِ الْوَاحِدَةِ.
\par 10 وَاعْمِدَتُهَا عِشْرُونَ وَقَوَاعِدُهَا عِشْرُونَ مِنْ نُحَاسٍ. رُزَزُ الاعْمِدَةِ وَقُضْبَانُهَا مِنْ فِضَّةٍ.
\par 11 وَكَذَلِكَ الَى جِهَةِ الشِّمَالِ فِي الطُّولِ اسْتَارٌ مِئَةُ ذِرَاعٍ طُولا. وَاعْمِدَتُهَا عِشْرُونَ وَقَوَاعِدُهَا عِشْرُونَ مِنْ نُحَاسٍ. رُزَزُ الاعْمِدَةِ وَقُضْبَانُهَا مِنْ فِضَّةٍ.
\par 12 وَفِي عَرْضِ الدَّارِ الَى جِهَةِ الْغَرْبِ اسْتَارٌ خَمْسُونَ ذِرَاعا. اعْمِدَتُهَا عَشَرَةٌ وَقَوَاعِدُهَا عَشَرٌ.
\par 13 وَعَرْضُ الدَّارِ الَى جِهَةِ الشَّرْقِ نَحْوَ الشُّرُوقِ خَمْسُونَ ذِرَاعا.
\par 14 وَخَمْسَ عَشَرَةَ ذِرَاعا مِنَ الاسْتَارِ لِلْجَانِبِ الْوَاحِدِ. اعْمِدَتُهَا ثَلاثَةٌ وَقَوَاعِدُهَا ثَلاثٌ.
\par 15 وَلِلْجَانِبِ الثَّانِي خَمْسَ عَشَرَةَ ذِرَاعا مِنَ الاسْتَارِ. اعْمِدَتُهَا ثَلاثَةٌ وَقَوَاعِدُهَا ثَلاثٌ.
\par 16 وَلِبَابِ الدَّارِ سَجْفٌ عِشْرُونَ ذِرَاعا مِنْ اسْمَانْجُونِيٍّ وَارْجُوَانٍ وَقِرْمِزٍ وَبُوصٍ مَبْرُومٍ صَنْعَةَ الطَّرَّازِ. اعْمِدَتُهُ ارْبَعَةٌ وَقَوَاعِدُهَا ارْبَعٌ.
\par 17 لِكُلِّ اعْمِدَةِ الدَّارِ حَوَالَيْهَا قُضْبَانٌ مِنْ فِضَّةٍ. رُزَزُهَا مِنْ فِضَّةٍ وَقَوَاعِدُهَا مِنْ نُحَاسٍ.
\par 18 طُولُ الدَّارِ مِئَةُ ذِرَاعٍ وَعَرْضُهَا خَمْسُونَ فَخَمْسُونَ وَارْتِفَاعُهَا خَمْسُ اذْرُعٍ مِنْ بُوصٍ مَبْرُومٍ وَقَوَاعِدُهَا مِنْ نُحَاسٍ.
\par 19 جَمِيعُ اوَانِي الْمَسْكَنِ فِي كُلِّ خِدْمَتِهِ وَجَمِيعُ اوْتَادِهِ وَجَمِيعُ اوْتَادِ الدَّارِ مِنْ نُحَاسٍ.
\par 20 «وَانْتَ تَامُرُ بَنِي اسْرَائِيلَ انْ يُقَدِّمُوا الَيْكَ زَيْتَ زَيْتُونٍ مَرْضُوضٍ نَقِيّا لِلضُّوءِ لاصْعَادِ السُّرُجِ دَائِما.
\par 21 فِي خَيْمَةِ الاجْتِمَاعِ خَارِجَ الْحِجَابِ الَّذِي امَامَ الشَّهَادَةِ يُرَتِّبُهَا هَارُونُ وَبَنُوهُ مِنَ الْمَسَاءِ الَى الصَّبَاحِ امَامَ الرَّبِّ. فَرِيضَةً دَهْرِيَّةً فِي اجْيَالِهِمْ مِنْ بَنِي اسْرَائِيلَ.

\chapter{28}

\par 1 «وَقَرِّبْ الَيْكَ هَارُونَ اخَاكَ وَبَنِيهِ مَعَهُ مِنْ بَيْنِ بَنِي اسْرَائِيلَ لِيَكْهَنَ لِي. هَارُونَ نَادَابَ وَابِيهُوَ الِعَازَارَ وَايثَامَارَ بَنِي هَارُونَ.
\par 2 وَاصْنَعْ ثِيَابا مُقَدَّسَةً لِهَارُونَ اخِيكَ لِلْمَجْدِ وَالْبَهَاءِ.
\par 3 وَتُكَلِّمُ جَمِيعَ حُكَمَاءِ الْقُلُوبِ الَّذِينَ مَلَاتُهُمْ رُوحَ حِكْمَةٍ انْ يَصْنَعُوا ثِيَابَ هَارُونَ لِتَقْدِيسِهِ لِيَكْهَنَ لِي.
\par 4 وَهَذِهِ هِيَ الثِّيَابُ الَّتِي يَصْنَعُونَهَا: صُدْرَةٌ وَرِدَاءٌ وَجُبَّةٌ وَقَمِيصٌ مُخَرَّمٌ وَعِمَامَةٌ وَمِنْطَقَةٌ. فَيَصْنَعُونَ ثِيَابا مُقَدَّسَةً لِهَارُونَ اخِيكَ وَلِبَنِيهِ لِيَكْهَنَ لِي.
\par 5 وَهُمْ يَاخُذُونَ الذَّهَبَ وَالاسْمَانْجُونِيَّ وَالارْجُوَانَ وَالْقِرْمِزَ وَالْبُوصَ.
\par 6 فَيَصْنَعُونَ الرِّدَاءَ مِنْ ذَهَبٍ وَاسْمَانْجُونِيٍّ وَارْجُوَانٍ وَقِرْمِزٍ وَبُوصٍ مَبْرُومٍ صَنْعَةَ حَائِكٍ حَاذِقٍ.
\par 7 يَكُونُ لَهُ كَتِفَانِ مَوْصُولانِ فِي طَرَفَيْهِ لِيَتَّصِلَ.
\par 8 وَزُنَّارُ شَدِّهِ الَّذِي عَلَيْهِ يَكُونُ مِنْهُ كَصَنْعَتِهِ. مِنْ ذَهَبٍ وَاسْمَانْجُونِيٍّ وَقِرْمِزٍ وَبُوصٍ مَبْرُومٍ.
\par 9 وَتَاخُذُ حَجَرَيْ جَزْعٍ وَتَنْقُشُ عَلَيْهِمَا اسْمَاءَ بَنِي اسْرَائِيلَ.
\par 10 سِتَّةً مِنْ اسْمَائِهِمْ عَلَى الْحَجَرِ الْوَاحِدِ وَاسْمَاءَ السِّتَّةِ الْبَاقِينَ عَلَى الْحَجَرِ الثَّانِي حَسَبَ مَوَالِيدِهِمْ.
\par 11 صَنْعَةَ نَقَّاشِ الْحِجَارَةِ نَقْشَ الْخَاتِمِ تَنْقُشُ الْحَجَرَيْنِ عَلَى حَسَبِ اسْمَاءِ بَنِي اسْرَائِيلَ. مُحَاطَيْنِ بِطَوْقَيْنِ مِنْ ذَهَبٍ تَصْنَعُهُمَا.
\par 12 وَتَضَعُ الْحَجَرَيْنِ عَلَى كَتِفَيِ الرِّدَاءِ حَجَرَيْ تِذْكَارٍ لِبَنِي اسْرَائِيلَ. فَيَحْمِلُ هَارُونُ اسْمَاءَهُمْ امَامَ الرَّبِّ عَلَى كَتِفَيْهِ لِلتِّذْكَارِ.
\par 13 وَتَصْنَعُ طَوْقَيْنِ مِنْ ذَهَبٍ.
\par 14 وَسِلْسِلَتَيْنِ مِنْ ذَهَبٍ نَقِيٍّ. مَجْدُولَتَيْنِ تَصْنَعُهُمَا صَنْعَةَ الضَّفْرِ. وَتَجْعَلُ سِلْسِلَتَيِ الضَّفَائِرِ فِي الطَّوْقَيْنِ.
\par 15 «وَتَصْنَعُ صُدْرَةَ قَضَاءٍ - صَنْعَةَ حَائِكٍ حَاذِقٍ كَصَنْعَةِ الرِّدَاءِ تَصْنَعُهَا. مِنْ ذَهَبٍ وَاسْمَانْجُونِيٍّ وَارْجُوَانٍ وَقِرْمِزٍ وَبُوصٍ مَبْرُومٍ تَصْنَعُهَا.
\par 16 تَكُونُ مُرَبَّعَةً مَثْنِيَّةً طُولُهَا شِبْرٌ وَعَرْضُهَا شِبْرٌ.
\par 17 وَتُرَصِّعُ فِيهَا تَرْصِيعَ حَجَرٍ ارْبَعَةَ صُفُوفِ حِجَارَةٍ. صَفُّ عَقِيقٍ احْمَرَ وَيَاقُوتٍ اصْفَرَ وَزُمُرُّدٍ: الصَّفُّ الاوَّلُ.
\par 18 وَالصَّفُّ الثَّانِي: بَهْرَمَانٌ وَيَاقُوتٌ ازْرَقُ وَعَقِيقٌ ابْيَضُ.
\par 19 وَالصَّفُّ الثَّالِثُ: عَيْنُ الْهِرِّ وَيَشْمٌ وَجَمَشْتٌ.
\par 20 وَالصَّفُّ الرَّابِعُ: زَبَرْجَدٌ وَجَزْعٌ وَيَشْبٌ. تَكُونُ مُطَوَّقَةً بِذَهَبٍ فِي تَرْصِيعِهَا.
\par 21 وَتَكُونُ الْحِجَارَةُ عَلَى اسْمَاءِ بَنِي اسْرَائِيلَ اثْنَيْ عَشَرَ عَلَى اسْمَائِهِمْ. كَنَقْشِ الْخَاتَِمِ كُلُّ وَاحِدٍ عَلَى اسْمِهِ تَكُونُ لِلاثْنَيْ عَشَرَ سِبْطا.
\par 22 «وَتَصْنَعُ عَلَى الصُّدْرَةِ سَلاسِلَ مَجْدُولَةً صَنْعَةَ الضَّفْرِ مِنْ ذَهَبٍ نَقِيٍّ.
\par 23 وَتَصْنَعُ عَلَى الصُّدْرَةِ حَلْقَتَيْنِ مِنْ ذَهَبٍ. وَتَجْعَلُ الْحَلْقَتَيْنِ عَلَى طَرَفَيِ الصُّدْرَةِ.
\par 24 وَتَجْعَلُ ضَفِيرَتَيِ الذَّهَبِ فِي الْحَلْقَتَيْنِ عَلَى طَرَفَيِ الصُّدْرَةِ.
\par 25 وَتَجْعَلُ طَرَفَيِ الضَّفِيرَتَيْنِ الاخَرَيْنِ فِي الطَّوْقَيْنِ وَتَجْعَلُهُمَا عَلَى كَتِفَيِ الرِّدَاءِ الَى قُدَّامِهِ.
\par 26 وَتَصْنَعُ حَلْقَتَيْنِ مِنْ ذَهَبٍ وَتَضَعُهُمَا عَلَى طَرَفَيِ الصُّدْرَةِ عَلَى حَاشِيَتِهَا الَّتِي الَى جِهَةِ الرِّدَاءِ مِنْ دَاخِلٍ.
\par 27 وَتَصْنَعُ حَلْقَتَيْنِ مِنْ ذَهَبٍ. وَتَجْعَلُهُمَا عَلَى كَتِفَيِ الرِّدَاءِ مِنْ اسْفَلُ مِنْ قُدَّامِهِ عِنْدَ وَصْلِهِ مِنْ فَوْقِ زُنَّارِ الرِّدَاءِ.
\par 28 وَيَرْبُطُونَ الصُّدْرَةَ بِحَلْقَتَيْهَا الَى حَلْقَتَيِ الرِّدَاءِ بِخَيْطٍ مِنْ اسْمَانْجُونِيٍّ لِتَكُونَ عَلَى زُنَّارِ الرِّدَاءِ. وَلا تُنْزَعُ الصُّدْرَةُ عَنِ الرِّدَاءِ.
\par 29 فَيَحْمِلُ هَارُونُ اسْمَاءَ بَنِي اسْرَائِيلَ فِي صُدْرَةِ الْقَضَاءِ عَلَى قَلْبِهِ عِنْدَ دُخُولِهِ الَى الْقُدْسِ لِلتِّذْكَارِ امَامَ الرَّبِّ دَائِما.
\par 30 وَتَجْعَلُ فِي صُدْرَةِ الْقَضَاءِ الاورِيمَ وَالتُّمِّيمَ لِتَكُونَ عَلَى قَلْبِ هَارُونَ عِنْدَ دُخُولِهِ امَامَ الرَّبِّ. فَيَحْمِلُ هَارُونُ قَضَاءَ بَنِي اسْرَائِيلَ عَلَى قَلْبِهِ امَامَ الرَّبِّ دَائِما.
\par 31 «وَتَصْنَعُ جُبَّةَ الرِّدَاءِ كُلَّهَا مِنْ اسْمَانْجُونِيٍّ
\par 32 وَتَكُونُ فَتْحَةُ رَاسِهَا فِي وَسَطِهَا. وَيَكُونُ لِفَتْحَتِهَا حَاشِيَةٌ حَوَالَيْهَا صَنْعَةَ الْحَائِكِ. كَفَتْحَةِ الدِّرْعِ يَكُونُ لَهَا. لا تُشَقُّ.
\par 33 وَتَصْنَعُ عَلَى اذْيَالِهَا رُمَّانَاتٍ مِنْ اسْمَانْجُونِيٍّ وَارْجُوانٍ وَقِرْمِزٍ. عَلَى اذْيَالِهَا حَوَالَيْهَا. وَجَلاجِلَ مِنْ ذَهَبٍ بَيْنَهَا حَوَالَيْهَا.
\par 34 جُلْجُلَ ذَهَبٍ وَرُمَّانَةً جُلْجُلَ ذَهَبٍ وَرُمَّانَةً عَلَى اذْيَالِ الْجُبَّةِ حَوَالَيْهَا.
\par 35 فَتَكُونُ عَلَى هَارُونَ لِلْخِدْمَةِ لِيُسْمَعَ صَوْتُهَا عِنْدَ دُخُولِهِ الَى الْقُدْسِ امَامَ الرَّبِّ وَعِنْدَ خُرُوجِهِ لِئَلا يَمُوتَ.
\par 36 «وَتَصْنَعُ صَفِيحَةً مِنْ ذَهَبٍ نَقِيٍّ. وَتَنْقُشُ عَلَيْهَا نَقْشَ خَاتِمٍ «قُدْسٌ لِلرَّبِّ».
\par 37 وَتَضَعُهَا عَلَى خَيْطٍ اسْمَانْجُونِيٍّ لِتَكُونَ عَلَى الْعِمَامَةِ. الَى قُدَّامِ الْعِمَامَةِ تَكُونُ
\par 38 فَتَكُونُ عَلَى جِبْهَةِ هَارُونَ. فَيَحْمِلُ هَارُونُ اثْمَ الاقْدَاسِ الَّتِي يُقَدِّسُهَا بَنُو اسْرَائِيلَ جَمِيعِ عَطَايَا اقْدَاسِهِمْ. وَتَكُونُ عَلَى جِبْهَتِهِ دَائِما لِلرِّضَا عَنْهُمْ امَامَ الرَّبِّ.
\par 39 وَتُخَرِّمُ الْقَمِيصَ مِنْ بُوصٍ وَتَصْنَعُ الْعِمَامَةَ مِنْ بُوصٍ وَالْمِنْطَقَةُ تَصْنَعُهَا صَنْعَةَ الطَّرَّازِ.
\par 40 «وَلِبَنِي هَارُونَ تَصْنَعُ اقْمِصَةً وَتَصْنَعُ لَهُمْ مَنَاطِقَ وَتَصْنَعُ لَهُمْ قَلانِسَ لِلْمَجْدِ وَالْبَهَاءِ.
\par 41 وَتُلْبِسُ هَارُونَ اخَاكَ ايَّاهَا وَبَنِيهِ مَعَهُ وَتَمْسَحُهُمْ وَتَمْلَا ايَادِيهِمْ وَتُقَدِّسُهُمْ لِيَكْهَنُوا لِي.
\par 42 وَتَصْنَعُ لَهُمْ سَرَاوِيلَ مِنْ كَتَّانٍ لِسَتْرِ الْعَوْرَةِ. مِنَ الْحَقَوَيْنِ الَى الْفَخْذَيْنِ تَكُونُ.
\par 43 فَتَكُونُ عَلَى هَارُونَ وَبَنِيهِ عِنْدَ دُخُولِهِمْ الَى خَيْمَةِ الاجْتِمَاعِ اوْ عِنْدَ اقْتِرَابِهِمْ الَى الْمَذْبَحِ لِلْخِدْمَةِ فِي الْقُدْسِ لِئَلا يَحْمِلُوا اثْما وَيَمُوتُوا. فَرِيضَةً ابَدِيَّةً لَهُ وَلِنَسْلِهِ مِنْ بَعْدِهِ.

\chapter{29}

\par 1 «وَهَذَا مَا تَصْنَعُهُ لَهُمْ لِتَقْدِيسِهِمْ لِيَكْهَنُوا لِي: خُذْ ثَوْرا وَاحِدا ابْنَ بَقَرٍ وَكَبْشَيْنِ صَحِيحَيْنِ
\par 2 وَخُبْزَ فَطِيرٍ وَاقْرَاصَ فَطِيرٍ مَلْتُوتَةً بِزَيْتٍ وَرِقَاقَ فَطِيرٍ مَدْهُونَةً بِزَيْتٍ. مِنْ دَقِيقِ حِنْطَةٍ تَصْنَعُهَا.
\par 3 وَتَجْعَلُهَا فِي سَلَّةٍ وَاحِدَةٍ وَتُقَدِّمُهَا فِي السَّلَّةِ مَعَ الثَّوْرِ وَالْكَبْشَيْنِ.
\par 4 «وَتُقَدِّمُ هَارُونَ وَبَنِيهِ الَى بَابِ خَيْمَةِ الاجْتِمَاعِ وَتَغْسِلُهُمْ بِمَاءٍ.
\par 5 وَتَاخُذُ الثِّيَابَ وَتُلْبِسُ هَارُونَ الْقَمِيصَ وَجُبَّةَ الرِّدَاءِ وَالرِّدَاءَ وَالصُّدْرَةَ وَتَشُدُّهُ بِزُنَّارِ الرِّدَاءِ
\par 6 وَتَضَعُ الْعِمَامَةَ عَلَى رَاسِهِ وَتَجْعَلُ الْاكْلِيلَ الْمُقَدَّسَ عَلَى الْعِمَامَةِ
\par 7 وَتَاخُذُ دُهْنَ الْمَسْحَةِ وَتَسْكُبُهُ عَلَى رَاسِهِ وَتَمْسَحُهُ.
\par 8 وَتُقَدِّمُ بَنِيهِ وَتُلْبِسُهُمْ اقْمِصَةً.
\par 9 وَتُنَطِّقُهُمْ بِمَنَاطِقَ هَارُونَ وَبَنِيهِ. وَتَشُدُّ لَهُمْ قَلانِسَ. فَيَكُونُ لَهُمْ كَهَنُوتٌ فَرِيضَةً ابَدِيَّةً. وَتَمْلَا يَدَ هَارُونَ وَايْدِيَ بَنِيهِ.
\par 10 «وَتُقَدِّمُ الثَّوْرَ الَى قُدَّامِ خَيْمَةِ الاجْتِمَاعِ فَيَضَعُ هَارُونُ وَبَنُوهُ ايْدِيَهُمْ عَلَى رَاسِ الثَّوْرِ.
\par 11 فَتَذْبَحُ الثَّوْرَ امَامَ الرَّبِّ عِنْدَ بَابِ خَيْمَةِ الاجْتِمَاعِ.
\par 12 وَتَاخُذُ مِنْ دَمِ الثَّوْرِ وَتَجْعَلُهُ عَلَى قُرُونِ الْمَذْبَحِ بِاصْبِعِكَ وَسَائِرَ الدَّمِ تَصُبُّهُ الَى اسْفَلِ الْمَذْبَحِ.
\par 13 وَتَاخُذُ كُلَّ الشَّحْمِ الَّذِي يُغَشِّي الْجَوْفَ وَزِيَادَةَ الْكَبِدِ وَالْكُلْيَتَيْنِ وَالشَّحْمَ الَّذِي عَلَيْهِمَا وَتُوقِدُهَا عَلَى الْمَذْبَحِ.
\par 14 وَامَّا لَحْمُ الثَّوْرِ وَجِلْدُهُ وَفَرْثُهُ فَتَحْرِقُهَا بِنَارٍ خَارِجَ الْمَحَلَّةِ. هُوَ ذَبِيحَةُ خَطِيَّةٍ.
\par 15 «وَتَاخُذُ الْكَبْشَ الْوَاحِدَ فَيَضَعُ هَارُونُ وَبَنُوهُ ايْدِيَهُمْ عَلَى رَاسِ الْكَبْشِ.
\par 16 فَتَذْبَحُ الْكَبْشَ وَتَاخُذُ دَمَهُ وَتَرُشُّهُ عَلَى الْمَذْبَحِ مِنْ كُلِّ نَاحِيَةٍ.
\par 17 وَتَقْطَعُ الْكَبْشَ الَى قِطَعِهِ وَتَغْسِلُ جَوْفَهُ وَاكَارِعَهُ وَتَجْعَلُهَا عَلَى قِطَعِهِ وَعَلَى رَاسِهِ
\par 18 وَتُوقِدُ كُلَّ الْكَبْشِ عَلَى الْمَذْبَحِ. هُوَ مُحْرَقَةٌ لِلرَّبِّ. رَائِحَةُ سُرُورٍ. وَقُودٌ هُوَ لِلرَّبِّ.
\par 19 «وَتَاخُذُ الْكَبْشَ الثَّانِيَ. فَيَضَعُ هَارُونُ وَبَنُوهُ ايْدِيَهُمْ عَلَى رَاسِ الْكَبْشِ.
\par 20 فَتَذْبَحُ الْكَبْشَ وَتَاخُذُ مِنْ دَمِهِ وَتَجْعَلُ عَلَى شَحْمَةِ اذُنِ هَارُونَ وَعَلَى شَحْمِ اذَانِ بَنِيهِ الْيُمْنَى وَعَلَى ابَاهِمِ ايْدِيهِمِ الْيُمْنَى وَعَلَى ابَاهِمِ ارْجُلِهِمِ الْيُمْنَى. وَتَرُشُّ الدَّمَ عَلَى الْمَذْبَحِ مِنْ كُلِّ نَاحِيَةٍ.
\par 21 وَتَاخُذُ مِنَ الدَّمِ الَّذِي عَلَى الْمَذْبَحِ وَمِنْ دُهْنِ الْمَسْحَةِ وَتَنْضِحُ عَلَى هَارُونَ وَثِيَابِهِ وَعَلَى بَنِيهِ وَثِيَابِ بَنِيهِ مَعَهُ فَيَتَقَدَّسُ هُوَ وَثِيَابُهُ وَبَنُوهُ وَثِيَابُ بَنِيهِ مَعَهُ.
\par 22 ثُمَّ تَاخُذُ مِنَ الْكَبْشِ: الشَّحْمَ وَالْالْيَةَ وَالشَّحْمَ الَّذِي يُغَشِّي الْجَوْفَ وَزِيَادَةَ الْكَبِدِ وَالْكُلْيَتَيْنِ وَالشَّحْمَ الَّذِي عَلَيْهِمَا وَالسَّاقَ الْيُمْنَى. فَانَّهُ كَبْشُ مِلْءٍ.
\par 23 وَرَغِيفا وَاحِدا مِنَ الْخُبْزِ وَقُرْصا وَاحِدا مِنَ الْخُبْزِ بِزَيْتٍ وَرُقَاقَةً وَاحِدَةً مِنْ سَلَّةِ الْفَطِيرِ الَّتِي امَامَ الرَّبِّ
\par 24 وَتَضَعُ الْجَمِيعَ فِي يَدَيْ هَارُونَ وَفِي ايْدِي بَنِيهِ وَتُرَدِّدُهَا تَرْدِيدا امَامَ الرَّبِّ.
\par 25 ثُمَّ تَاخُذُهَا مِنْ ايْدِيهِمْ وَتُوقِدُهَا عَلَى الْمَذْبَحِ فَوْقَ الْمُحْرَقَةِ رَائِحَةَ سُرُورٍ امَامَ الرَّبِّ. وَقُودٌ هُوَ لِلرَّبِّ.
\par 26 «ثُمَّ تَاخُذُ الْقَصَّ مِنْ كَبْشِ الْمِلْءِ الَّذِي لِهَارُونَ وَتُرَدِّدُهُ تَرْدِيدا امَامَ الرَّبِّ فَيَكُونُ لَكَ نَصِيبا.
\par 27 وَتُقَدِّسُ قَصَّ التَّرْدِيدِ وَسَاقَ الرَّفِيعَةِ الَّذِي رُدِّدَ وَالَّذِي رُفِعَ مِنْ كَبْشِ الْمِلْءِ مِمَّا لِهَارُونَ وَلِبَنِيهِ
\par 28 فَيَكُونَانِ لِهَارُونَ وَبَنِيهِ فَرِيضَةً ابَدِيَّةً مِنْ بَنِي اسْرَائِيلَ لانَّهُمَا رَفِيعَةٌ. وَيَكُونَانِ رَفِيعَةً مِنْ بَنِي اسْرَائِيلَ مِنْ ذَبَائِحِ سَلامَتِهِمْ رَفِيعَتَهُمْ لِلرَّبِّ.
\par 29 «وَالثِّيَابُ الْمُقَدَّسَةُ الَّتِي لِهَارُونَ تَكُونُ لِبَنِيهِ بَعْدَهُ لِيُمْسَحُوا فِيهَا وَلِتُمْلا فِيهَا ايْدِيهِمْ.
\par 30 سَبْعَةَ ايَّامٍ يَلْبِسُهَا الْكَاهِنُ الَّذِي هُوَ عِوَضٌ عَنْهُ مِنْ بَنِيهِ الَّذِي يَدْخُلُ خَيْمَةَ الاجْتِمَاعِ لِيَخْدِمَ فِي الْقُدْسِ.
\par 31 «وَامَّا كَبْشُ الْمِلْءِ فَتَاخُذُهُ وَتَطْبُخُ لَحْمَهُ فِي مَكَانٍ مُقَدَّسٍ.
\par 32 فَيَاكُلُ هَارُونُ وَبَنُوهُ لَحْمَ الْكَبْشِ وَالْخُبْزَ الَّذِي فِي السَّلَّةِ عِنْدَ بَابِ خَيْمَةِ الاجْتِمَاعِ.
\par 33 يَاكُلُهَا الَّذِينَ كُفِّرَ بِهَا عَنْهُمْ لِمِلْءِ ايْدِيهِمْ لِتَقْدِيسِهِمْ. وَامَّا الاجْنَبِيُّ فَلا يَاكُلُ لانَّهَا مُقَدَّسَةٌ.
\par 34 وَانْ بَقِيَ شَيْءٌ مِنْ لَحْمِ الْمِلْءِ اوْ مِنَ الْخُبْزِ الَى الصَّبَاحِ تُحْرِقُ الْبَاقِيَ بِالنَّارِ. لا يُؤْكَلُ لانَّهُ مُقَدَّسٌ.
\par 35 وَتَصْنَعُ لِهَارُونَ وَبَنِيهِ هَكَذَا بِحَسَبِ كُلِّ مَا امَرْتُكَ. سَبْعَةَ ايَّامٍ تَمْلَا ايْدِيَهِمْ.
\par 36 وَتُقَدِّمُ ثَوْرَ خَطِيَّةٍ كُلَّ يَوْمٍ لاجْلِ الْكَفَّارَةِ. وَتُطَهِّرُ الْمَذْبَحَ بِتَكْفِيرِكَ عَلَيْهِ. وَتَمْسَحُهُ لِتَقْدِيسِهِ.
\par 37 سَبْعَةَ ايَّامٍ تُكَفِّرُ عَلَى الْمَذْبَحِ وَتُقَدِّسُهُ. فَيَكُونُ الْمَذْبَحُ قُدْسَ اقْدَاسٍ. كُلُّ مَا مَسَّ الْمَذْبَحَ يَكُونُ مُقَدَّسا.
\par 38 «وَهَذَا مَا تُقَدِّمُهُ عَلَى الْمَذْبَحِ: خَرُوفَانِ حَوْلِيَّانِ كُلَّ يَوْمٍ دَائِما.
\par 39 الْخَرُوفُ الْوَاحِدُ تُقَدِّمُهُ صَبَاحا وَالْخَرُوفُ الثَّانِي تُقَدِّمُهُ فِي الْعَشِيَّةِ.
\par 40 وَعُشْرٌ مِنْ دَقِيقٍ مَلْتُوتٍ بِرُبْعِ الْهِينِ مِنْ زَيْتِ الرَّضِّ وَسَكِيبٌ رُبْعُ الْهِينِ مِنَ الْخَمْرِ لِلْخَرُوفِ الْوَاحِدِ.
\par 41 وَالْخَرُوفُ الثَّانِي تُقَدِّمُهُ فِي الْعَشِيَّةِ. مِثْلَ تَقْدِمَةِ الصَّبَاحِ وَسَكِيبِهِ تَصْنَعُ لَهُ. رَائِحَةُ سُرُورٍ وَقُودٌ لِلرَّبِّ.
\par 42 مُحْرَقَةٌ دَائِمَةٌ فِي اجْيَالِكُمْ عِنْدَ بَابِ خَيْمَةِ الاجْتِمَاعِ امَامَ الرَّبِّ. حَيْثُ اجْتَمِعُ بِكُمْ لِاكَلِّمَكَ هُنَاكَ.
\par 43 وَاجْتَمِعُ هُنَاكَ بِبَنِي اسْرَائِيلَ فَيُقَدَّسُ بِمَجْدِي.
\par 44 وَاقَدِّسُ خَيْمَةَ الاجْتِمَاعِ وَالْمَذْبَحَ. وَهَارُونُ وَبَنُوهُ اقَدِّسُهُمْ لِيَكْهَنُوا لِي.
\par 45 وَاسْكُنُ فِي وَسْطِ بَنِي اسْرَائِيلَ وَاكُونُ لَهُمْ الَها
\par 46 فَيَعْلَمُونَ انِّي انَا الرَّبُّ الَهُهُمُ الَّذِي اخْرَجَهُمْ مِنْ ارْضِ مِصْرَ لاسْكُنَ فِي وَسْطِهِمْ. انَا الرَّبُّ الَهُهُمْ.

\chapter{30}

\par 1 «وَتَصْنَعُ مَذْبَحا لايقَادِ الْبَخُورِ. مِنْ خَشَبِ السَّنْطِ تَصْنَعُهُ.
\par 2 طُولُهُ ذِرَاعٌ وَعَرْضُهُ ذِرَاعٌ. مُرَبَّعا يَكُونُ. وَارْتِفَاعُهُ ذِرَاعَانِ. مِنْهُ تَكُونُ قُرُونُهُ.
\par 3 وَتُغَشِّيهِ بِذَهَبٍ نَقِيٍّ: سَطْحَهُ وَحِيطَانَهُ حَوَالَيْهِ وَقُرُونَهُ. وَتَصْنَعُ لَهُ اكْلِيلا مِنْ ذَهَبٍ حَوَالَيْهِ.
\par 4 وَتَصْنَعُ لَهُ حَلْقَتَيْنِ مِنْ ذَهَبٍ تَحْتَ اكْلِيلِهِ عَلَى جَانِبَيْهِ. عَلَى الْجَانِبَيْنِ تَصْنَعُهُمَا لِتَكُونَا بَيْتَيْنِ لِعَصَوَيْنِ لِحَمْلِهِ بِهِمَا.
\par 5 وَتَصْنَعُ الْعَصَوَيْنِ مِنْ خَشَبِ السَّنْطِ وَتُغَشِّيهِمَا بِذَهَبٍ.
\par 6 وَتَجْعَلُهُ قُدَّامَ الْحِجَابِ الَّذِي امَامَ تَابُوتِ الشَّهَادَةِ. قُدَّامَ الْغِطَاءِ الَّذِي عَلَى الشَّهَادَةِ حَيْثُ اجْتَمِعُ بِكَ.
\par 7 فَيُوقِدُ عَلَيْهِ هَارُونُ بَخُورا عَطِرا كُلَّ صَبَاحٍ. حِينَ يُصْلِحُ السُّرُجَ يُوقِدُهُ.
\par 8 وَحِينَ يُصْعِدُ هَارُونُ السُّرُجَ فِي الْعَشِيَّةِ يُوقِدُهُ. بَخُورا دَائِما امَامَ الرَّبِّ فِي اجْيَالِكُمْ.
\par 9 لا تُصْعِدُوا عَلَيْهِ بَخُورا غَرِيبا وَلا مُحْرَقَةً اوْ تَقْدِمَةً وَلا تَسْكُبُوا عَلَيْهِ سَكِيبا.
\par 10 وَيَصْنَعُ هَارُونُ كَفَّارَةً عَلَى قُرُونِهِ مَرَّةً فِي السَّنَةِ. مِنْ دَمِ ذَبِيحَةِ الْخَطِيَّةِ الَّتِي لِلْكَفَّارَةِ مَرَّةً فِي السَّنَةِ يَصْنَعُ كَفَّارَةً عَلَيْهِ فِي اجْيَالِكُمْ. قُدْسُ اقْدَاسٍ هُوَ لِلرَّبِّ».
\par 11 وَقَالَ الرَّبُّ لِمُوسَى:
\par 12 «اذَا اخَذْتَ كَمِّيَّةَ بَنِي اسْرَائِيلَ بِحَسَبِ الْمَعْدُودِينَ مِنْهُمْ يُعْطُونَ كُلُّ وَاحِدٍ فِدْيَةَ نَفْسِهِ لِلرَّبِّ عِنْدَمَا تَعُدُّهُمْ لِئَلا يَصِيرَ فِيهِمْ وَبَا عِنْدَمَا تَعُدُّهُمْ.
\par 13 هَذَا مَا يُعْطِيهِ كُلُّ مَنِ اجْتَازَ الَى الْمَعْدُودِينَ: نِصْفُ الشَّاقِلِ بِشَاقِلِ الْقُدْسِ. (الشَّاقِلُ هُوَ عِشْرُونَ جِيرَةً) نِصْفُ الشَّاقِلِ تَقْدِمَةً لِلرَّبِّ.
\par 14 كُلُّ مَنِ اجْتَازَ الَى الْمَعْدُودِينَ مِنِ ابْنِ عِشْرِينَ سَنَةً فَصَاعِدا يُعْطِي تَقْدِمَةً لِلرَّبِّ.
\par 15 الْغَنِيُّ لا يُكْثِرُ وَالْفَقِيرُ لا يُقَلِّلُ عَنْ نِصْفِ الشَّاقِلِ حِينَ تُعْطُونَ تَقْدِمَةَ الرَّبِّ لِلتَّكْفِيرِ عَنْ نُفُوسِكُمْ.
\par 16 وَتَاخُذُ فِضَّةَ الْكَفَّارَةِ مِنْ بَنِي اسْرَائِيلَ وَتَجْعَلُهَا لِخِدْمَةِ خَيْمَةِ الاجْتِمَاعِ. فَتَكُونُ لِبَنِي اسْرَائِيلَ تِذْكَارا امَامَ الرَّبِّ لِلتَّكْفِيرِ عَنْ نُفُوسِكُمْ».
\par 17 وَقَالَ الرَّبُّ لِمُوسَى:
\par 18 «وَتَصْنَعُ مِرْحَضَةً مِنْ نُحَاسٍ وَقَاعِدَتَهَا مِنْ نُحَاسٍ لِلاغْتِسَالِ. وَتَجْعَلُهَا بَيْنَ خَيْمَةِ الاجْتِمَاعِ وَالْمَذْبَحِ وَتَجْعَلُ فِيهَا مَاءً.
\par 19 فَيَغْسِلُ هَارُونُ وَبَنُوهُ ايْدِيَهُمْ وَارْجُلَهُمْ مِنْهَا.
\par 20 عِنْدَ دُخُولِهِمْ الَى خَيْمَةِ الاجْتِمَاعِ يَغْسِلُونَ بِمَاءٍ لِئَلا يَمُوتُوا. اوْ عِنْدَ اقْتِرَابِهِمْ الَى الْمَذْبَحِ لِلْخِدْمَةِ لِيُوقِدُوا وَقُودا لِلرَّبِّ.
\par 21 يَغْسِلُونَ ايْدِيَهُمْ وَارْجُلَهُمْ لِئَلا يَمُوتُوا. وَيَكُونُ لَهُمْ فَرِيضَةً ابَدِيَّةً لَهُ وَلِنَسْلِهِ فِي اجْيَالِهِمْ».
\par 22 وَقَالَ الرَّبُّ لِمُوسَى:
\par 23 «وَانْتَ تَاخُذُ لَكَ افْخَرَ الاطْيَابِ. مُرّا قَاطِرا خَمْسَ مِئَةِ شَاقِلٍ وَقِرْفَةً عَطِرَةً نِصْفَ ذَلِكَ: مِئَتَيْنِ وَخَمْسِينَ وَقَصَبَ الذَّرِيرَةِ مِئَتَيْنِ وَخَمْسِينَ
\par 24 وَسَلِيخَةً خَمْسَ مِئَةٍ بِشَاقِلِ الْقُدْسِ وَمِنْ زَيْتِ الزَّيْتُونِ هِينا.
\par 25 وَتَصْنَعُهُ دُهْنا مُقَدَّسا لِلْمَسْحَةِ. عِطْرَ عِطَارَةٍ صَنْعَةَ الْعَطَّارِ. دُهْنا مُقَدَّسا لِلْمَسْحَةِ يَكُونُ.
\par 26 وَتَمْسَحُ بِهِ خَيْمَةَ الاجْتِمَاعِ وَتَابُوتَ الشَّهَادَةِ
\par 27 وَالْمَائِدَةَ وَكُلَّ انِيَتِهَا وَالْمَنَارَةَ وَانِيَتَهَا وَمَذْبَحَ الْبَخُورِ
\par 28 وَمَذْبَحَ الْمُحْرَقَةِ وَكُلَّ انِيَتِهِ وَالْمِرْحَضَةَ وَقَاعِدَتَهَا -
\par 29 وَتُقَدِّسُهَا فَتَكُونُ قُدْسَ اقْدَاسٍ. كُلُّ مَا مَسَّهَا يَكُونُ مُقَدَّسا.
\par 30 وَتَمْسَحُ هَارُونَ وَبَنِيهِ وَتُقَدِّسُهُمْ لِيَكْهَنُوا لِي.
\par 31 وَتُكَلِّمُ بَنِي اسْرَائِيلَ قَائِلا: يَكُونُ هَذَا لِي دُهْنا مُقَدَّسا لِلْمَسْحَةِ فِي اجْيَالِكُمْ.
\par 32 عَلَى جَسَدِ انْسَانٍ لا يُسْكَبُ. وَعَلَى مَقَادِيرِهِ لا تَصْنَعُوا مِثْلَهُ. مُقَدَّسٌ هُوَ وَيَكُونُ مُقَدَّسا عِنْدَكُمْ.
\par 33 كُلُّ مَنْ رَكَّبَ مِثْلَهُ وَمَنْ جَعَلَ مِنْهُ عَلَى اجْنَبِيٍّ يُقْطَعُ مِنْ شَعْبِهِ».
\par 34 وَقَالَ الرَّبُّ لِمُوسَى: «خُذْ لَكَ اعْطَارا: مَيْعَةً وَاظْفَارا وَقِنَّةً عَطِرَةً وَلُبَانا نَقِيّا - تَكُونُ اجْزَاءً مُتَسَاوِيَةً.
\par 35 فَتَصْنَعُهَا بَخُورا عَطِرا صَنْعَةَ الْعَطَّارِ مُمَلَّحا نَقِيّا مُقَدَّسا.
\par 36 وَتَسْحَقُ مِنْهُ نَاعِما وَتَجْعَلُ مِنْهُ قُدَّامَ الشَّهَادَةِ فِي خَيْمَةِ الاجْتِمَاعِ حَيْثُ اجْتَمِعُ بِكَ. قُدْسَ اقْدَاسٍ يَكُونُ عِنْدَكُمْ.
\par 37 وَالْبَخُورُ الَّذِي تَصْنَعُهُ عَلَى مَقَادِيرِهِ لا تَصْنَعُوا لانْفُسِكُمْ. يَكُونُ عِنْدَكَ مُقَدَّسا لِلرَّبِّ.
\par 38 كُلُّ مَنْ صَنَعَ مِثْلَهُ لِيَشُمَّهُ يُقْطَعُ مِنْ شَعْبِهِ».

\chapter{31}

\par 1 وَقَالَ الرَّبُّ لِمُوسَى:
\par 2 «انْظُرْ! قَدْ دَعَوْتُ بَصَلْئِيلَ بْنَ اورِي بْنَ حُورَ مِنْ سِبْطِ يَهُوذَا بِاسْمِهِ
\par 3 وَمَلَاتُهُ مِنْ رُوحِ اللهِ بِالْحِكْمَةِ وَالْفَهْمِ وَالْمَعْرِفَةِ وَكُلِّ صَنْعَةٍ
\par 4 لاخْتِرَاعِ مُخْتَرَعَاتٍ لِيَعْمَلَ فِي الذَّهَبِ وَالْفِضَّةِ وَالنُّحَاسِ
\par 5 وَنَقْشِ حِجَارَةٍ لِلتَّرْصِيعِ وَنِجَارَةِ الْخَشَبِ. لِيَعْمَلَ فِي كُلِّ صَنْعَةٍ.
\par 6 وَهَا انَا قَدْ جَعَلْتُ مَعَهُ اهُولِيابَ بْنَ اخِيسَامَاكَ مِنْ سِبْطِ دَانَ. وَفِي قَلْبِ كُلِّ حَكِيمِ الْقَلْبِ جَعَلْتُ حِكْمَةً لِيَصْنَعُوا كُلَّ مَا امَرْتُكَ.
\par 7 خَيْمَةَ الاجْتِمَاعِ وَتَابُوتَ الشَّهَادَةِ وَالْغِطَاءَ الَّذِي عَلَيْهِ وَكُلَّ انِيَةِ الْخَيْمَةِ
\par 8 وَالْمَائِدَةَ وَانِيَتَهَا وَالْمَنَارَةَ الطَّاهِرَةَ وَكُلَّ انِيَتِهَا وَمَذْبَحَ الْبَخُورِ
\par 9 وَمَذْبَحَ الْمُحْرَقَةِ وَكُلَّ انِيَتِهِ وَالْمِرْحَضَةَ وَقَاعِدَتَهَا
\par 10 وَالثِّيَابَ الْمَنْسُوجَةَ وَالثِّيَابَ الْمُقَدَّسَةَ لِهَارُونَ الْكَاهِنِ وَثِيَابَ بَنِيهِ لِلْكَهَانَةِ
\par 11 وَدُهْنَ الْمَسْحَةِ وَالْبَخُورَ الْعَطِرَ لِلْقُدْسِ. حَسَبَ كُلِّ مَا امَرْتُكَ بِهِ يَصْنَعُونَ».
\par 12 وَقَالَ الرَّبُّ لِمُوسَى:
\par 13 «وَانْتَ تُكَلِّمُ بَنِي اسْرَائِيلَ قَائِلا: سُبُوتِي تَحْفَظُونَهَا لانَّهُ عَلامَةٌ بَيْنِي وَبَيْنَكُمْ فِي اجْيَالِكُمْ لِتَعْلَمُوا انِّي انَا الرَّبُّ الَّذِي يُقَدِّسُكُمْ
\par 14 فَتَحْفَظُونَ السَّبْتَ لانَّهُ مُقَدَّسٌ لَكُمْ. مَنْ دَنَّسَهُ يُقْتَلُ قَتْلا. انَّ كُلَّ مَنْ صَنَعَ فِيهِ عَمَلا تُقْطَعُ تِلْكَ النَّفْسُ مِنْ بَيْنِ شَعْبِهَا.
\par 15 سِتَّةَ ايَّامٍ يُصْنَعُ عَمَلٌ. وَامَّا الْيَوْمُ السَّابِعُ فَفِيهِ سَبْتُ عُطْلَةٍ مُقَدَّسٌ لِلرَّبِّ. كُلُّ مَنْ صَنَعَ عَمَلا فِي يَوْمِ السَّبْتِ يُقْتَلُ قَتْلا.
\par 16 فَيَحْفَظُ بَنُو اسْرَائِيلَ السَّبْتَ لِيَصْنَعُوا السَّبْتَ فِي اجْيَالِهِمْ عَهْدا ابَدِيّا.
\par 17 هُوَ بَيْنِي وَبَيْنَ بَنِي اسْرَائِيلَ عَلامَةٌ الَى الابَدِ لانَّهُ فِي سِتَّةِ ايَّامٍ صَنَعَ الرَّبُّ السَّمَاءَ وَالارْضَ وَفِي الْيَوْمِ السَّابِعِ اسْتَرَاحَ وَتَنَفَّسَ».
\par 18 ثُمَّ اعْطَى مُوسَى عِنْدَ فَرَاغِهِ مِنَ الْكَلامِ مَعَهُ فِي جَبَلِ سِينَاءَ لَوْحَيِ الشَّهَادَةِ: لَوْحَيْ حَجَرٍ مَكْتُوبَيْنِ بِاصْبِعِ اللهِ.

\chapter{32}

\par 1 وَلَمَّا رَاى الشَّعْبُ انَّ مُوسَى ابْطَا فِي النُّزُولِ مِنَ الْجَبَلِ اجْتَمَعَ الشَّعْبُ عَلَى هَارُونَ وَقَالُوا لَهُ: «قُمِ اصْنَعْ لَنَا الِهَةً تَسِيرُ امَامَنَا لانَّ هَذَا مُوسَى الرَّجُلَ الَّذِي اصْعَدَنَا مِنْ ارْضِ مِصْرَ لا نَعْلَمُ مَاذَا اصَابَهُ».
\par 2 فَقَالَ لَهُمْ هَارُونُ: «انْزِعُوا اقْرَاطَ الذَّهَبِ الَّتِي فِي اذَانِ نِسَائِكُمْ وَبَنِيكُمْ وَبَنَاتِكُمْ وَاتُونِي بِهَا».
\par 3 فَنَزَعَ كُلُّ الشَّعْبِ اقْرَاطَ الذَّهَبِ الَّتِي فِي اذَانِهِمْ وَاتُوا بِهَا الَى هَارُونَ.
\par 4 فَاخَذَ ذَلِكَ مِنْ ايْدِيهِمْ وَصَوَّرَهُ بِالْازْمِيلِ وَصَنَعَهُ عِجْلا مَسْبُوكا. فَقَالُوا: «هَذِهِ الِهَتُكَ يَا اسْرَائِيلُ الَّتِي اصْعَدَتْكَ مِنْ ارْضِ مِصْرَ!»
\par 5 فَلَمَّا نَظَرَ هَارُونُ بَنَى مَذْبَحا امَامَهُ وَنَادَى هَارُونُ وَقَالَ: «غَدا عِيدٌ لِلرَّبِّ».
\par 6 فَبَكَّرُوا فِي الْغَدِ وَاصْعَدُوا مُحْرَقَاتٍ وَقَدَّمُوا ذَبَائِحَ سَلامَةٍ. وَجَلَسَ الشَّعْبُ لِلاكْلِ وَالشُّرْبِ ثُمَّ قَامُوا لِلَّعِبِ.
\par 7 فَقَالَ الرَّبُّ لِمُوسَى: «اذْهَبِ انْزِلْ! لانَّهُ قَدْ فَسَدَ شَعْبُكَ الَّذِي اصْعَدْتَهُ مِنْ ارْضِ مِصْرَ.
\par 8 زَاغُوا سَرِيعا عَنِ الطَّرِيقِ الَّذِي اوْصَيْتُهُمْ بِهِ. صَنَعُوا لَهُمْ عِجْلا مَسْبُوكا وَسَجَدُوا لَهُ وَذَبَحُوا لَهُ وَقَالُوا: هَذِهِ الِهَتُكَ يَا اسْرَائِيلُ الَّتِي اصْعَدَتْكَ مِنْ ارْضِ مِصْرَ».
\par 9 وَقَالَ الرَّبُّ لِمُوسَى: «رَايْتُ هَذَا الشَّعْبَ وَاذَا هُوَ شَعْبٌ صُلْبُ الرَّقَبَةِ.
\par 10 فَالانَ اتْرُكْنِي لِيَحْمَى غَضَبِي عَلَيْهِمْ وَافْنِيَهُمْ فَاصَيِّرَكَ شَعْبا عَظِيما».
\par 11 فَتَضَرَّعَ مُوسَى امَامَ الرَّبِّ الَهِهِ وَقَالَ: «لِمَاذَا يَا رَبُّ يَحْمَى غَضَبُكَ عَلَى شَعْبِكَ الَّذِي اخْرَجْتَهُ مِنْ ارْضِ مِصْرَ بِقُوَّةٍ عَظِيمَةٍ وَيَدٍ شَدِيدَةٍ؟
\par 12 لِمَاذَا يَتَكَلَّمُ الْمِصْرِيُّونَ قَائِلِينَ: اخْرَجَهُمْ بِخُبْثٍ لِيَقْتُلَهُمْ فِي الْجِبَالِ وَيُفْنِيَهُمْ عَنْ وَجْهِ الارْضِ؟ ارْجِعْ عَنْ حُمُوِّ غَضَبِكَ وَانْدَمْ عَلَى الشَّرِّ بِشَعْبِكَ.
\par 13 اذْكُرْ ابْرَاهِيمَ وَاسْحَاقَ وَاسْرَائِيلَ عَبِيدَكَ الَّذِينَ حَلَفْتَ لَهُمْ بِنَفْسِكَ وَقُلْتَ لَهُمْ: اكَثِّرُ نَسْلَكُمْ كَنُجُومِ السَّمَاءِ وَاعْطِي نَسْلَكُمْ كُلَّ هَذِهِ الارْضِ الَّتِي تَكَلَّمْتُ عَنْهَا فَيَمْلِكُونَهَا الَى الابَدِ».
\par 14 فَنَدِمَ الرَّبُّ عَلَى الشَّرِّ الَّذِي قَالَ انَّهُ يَفْعَلُهُ بِشَعْبِهِ.
\par 15 فَانْصَرَفَ مُوسَى وَنَزَلَ مِنَ الْجَبَلِ وَلَوْحَا الشَّهَادَةِ فِي يَدِهِ: لَوْحَانِ مَكْتُوبَانِ عَلَى جَانِبَيْهِمَا. مِنْ هُنَا وَمِنْ هُنَا كَانَا مَكْتُوبَيْنِ.
\par 16 وَاللَّوْحَانِ هُمَا صَنْعَةُ اللهِ وَالْكِتَابَةُ كِتَابَةُ اللهِ مَنْقُوشَةٌ عَلَى اللَّوْحَيْنِ.
\par 17 وَسَمِعَ يَشُوعُ صَوْتَ الشَّعْبِ فِي هُتَافِهِ فَقَالَ لِمُوسَى: «صَوْتُ قِتَالٍ فِي الْمَحَلَّةِ».
\par 18 فَقَالَ: «لَيْسَ صَوْتَ صِيَاحِ النُّصْرَةِ وَلا صَوْتَ صِيَاحِ الْكَسْرَةِ. بَلْ صَوْتَ غِنَاءٍ انَا سَامِعٌ».
\par 19 وَكَانَ عِنْدَمَا اقْتَرَبَ الَى الْمَحَلَّةِ انَّهُ ابْصَرَ الْعِجْلَ وَالرَّقْصَ. فَحَمِيَ غَضَبُ مُوسَى وَطَرَحَ اللَّوْحَيْنِ مِنْ يَدَيْهِ وَكَسَّرَهُمَا فِي اسْفَلِ الْجَبَلِ
\par 20 ثُمَّ اخَذَ الْعِجْلَ الَّذِي صَنَعُوا وَاحْرَقَهُ بِالنَّارِ وَطَحَنَهُ حَتَّى صَارَ نَاعِما وَذَرَّاهُ عَلَى وَجْهِ الْمَاءِ وَسَقَى بَنِي اسْرَائِيلَ.
\par 21 وَقَالَ مُوسَى لِهَارُونَ: «مَاذَا صَنَعَ بِكَ هَذَا الشَّعْبُ حَتَّى جَلَبْتَ عَلَيْهِ خَطِيَّةً عَظِيمَةً؟»
\par 22 فَقَالَ هَارُونُ: «لا يَحْمَ غَضَبُ سَيِّدِي! انْتَ تَعْرِفُ الشَّعْبَ انَّهُ شِرِّيرٌ.
\par 23 فَقَالُوا لِيَ: اصْنَعْ لَنَا الِهَةً تَسِيرُ امَامَنَا. لانَّ هَذَا مُوسَى الرَّجُلَ الَّذِي اصْعَدَنَا مِنْ ارْضِ مِصْرَ لا نَعْلَمُ مَاذَا اصَابَهُ.
\par 24 فَقُلْتُ لَهُمْ: مَنْ لَهُ ذَهَبٌ فَلْيَنْزِعْهُ وَيُعْطِنِي. فَطَرَحْتُهُ فِي النَّارِ فَخَرَجَ هَذَا الْعِجْلُ».
\par 25 وَلَمَّا رَاى مُوسَى الشَّعْبَ انَّهُ مُعَرًّى (لانَّ هَارُونَ كَانَ قَدْ عَرَّاهُ لِلْهُزْءِ بَيْنَ مُقَاوِمِيهِ)
\par 26 وَقَفَ مُوسَى فِي بَابِ الْمَحَلَّةِ وَقَالَ: «مَنْ لِلرَّبِّ فَالَيَّ!» فَاجْتَمَعَ الَيْهِ جَمِيعُ بَنِي لاوِي.
\par 27 فَقَالَ لَهُمْ: «هَكَذَا قَالَ الرَّبُّ الَهُ اسْرَائِيلَ: ضَعُوا كُلُّ وَاحِدٍ سَيْفَهُ عَلَى فَخِْذِهِ وَمُرُّوا وَارْجِعُوا مِنْ بَابٍ الَى بَابٍ فِي الْمَحَلَّةِ وَاقْتُلُوا كُلُّ وَاحِدٍ اخَاهُ وَكُلُّ وَاحِدٍ صَاحِبَهُ وَكُلُّ وَاحِدٍ قَرِيبَهُ».
\par 28 فَفَعَلَ بَنُو لاوِي بِحَسَبِ قَوْلِ مُوسَى. وَوَقَعَ مِنَ الشَّعْبِ فِي ذَلِكَ الْيَوْمِ نَحْوُ ثَلاثَةِ الافِ رَجُلٍ.
\par 29 وَقَالَ مُوسَى: «امْلَاوا ايْدِيَكُمُ الْيَوْمَ لِلرَّبِّ حَتَّى كُلُّ وَاحِدٍ بِابْنِهِ وَبِاخِيهِ فَيُعْطِيَكُمُ الْيَوْمَ بَرَكَةً».
\par 30 وَكَانَ فِي الْغَدِ انَّ مُوسَى قَالَ لِلشَّعْبِ: «انْتُمْ قَدْ اخْطَاتُمْ خَطِيَّةً عَظِيمَةً. فَاصْعَدُ الانَ الَى الرَّبِّ لَعَلِّي اكَفِّرُ خَطِيَّتَكُمْ».
\par 31 فَرَجَعَ مُوسَى الَى الرَّبِّ وَقَالَ: «اهِ قَدْ اخْطَا هَذَا الشَّعْبُ خَطِيَّةً عَظِيمَةً وَصَنَعُوا لانْفُسِهِمْ الِهَةً مِنْ ذَهَبٍ.
\par 32 وَالانَ انْ غَفَرْتَ خَطِيَّتَهُمْ - وَالا فَامْحُنِي مِنْ كِتَابِكَ الَّذِي كَتَبْتَ».
\par 33 فَقَالَ الرَّبُّ لِمُوسَى: «مَنْ اخْطَا الَيَّ امْحُوهُ مِنْ كِتَابِي.
\par 34 وَالانَ اذْهَبِ اهْدِ الشَّعْبَ الَى حَيْثُ كَلَّمْتُكَ. هُوَذَا مَلاكِي يَسِيرُ امَامَكَ. وَلَكِنْ فِي يَوْمِ افْتِقَادِي افْتَقِدُ فِيهِمْ خَطِيَّتَهُمْ».
\par 35 فَضَرَبَ الرَّبُّ الشَّعْبَ لانَّهُمْ صَنَعُوا الْعِجْلَ الَّذِي صَنَعَهُ هَارُونُ.

\chapter{33}

\par 1 وَقَالَ الرَّبُّ لِمُوسَى: «اذْهَبِ اصْعَدْ مِنْ هُنَا انْتَ وَالشَّعْبُ الَّذِي اصْعَدْتَهُ مِنْ ارْضِ مِصْرَ الَى الارْضِ الَّتِي حَلَفْتُ لابْرَاهِيمَ وَاسْحَاقَ وَيَعْقُوبَ قَائِلا: لِنَسْلِكَ اعْطِيهَا.
\par 2 وَانَا ارْسِلُ امَامَكَ مَلاكا وَاطْرُدُ الْكَنْعَانِيِّينَ وَالامُورِيِّينَ وَالْحِثِّيِّينَ وَالْفِرِزِّيِّينَ وَالْحِوِّيِّينَ وَالْيَبُوسِيِّينَ.
\par 3 الَى ارْضٍ تَفِيضُ لَبَنا وَعَسَلا. فَانِّي لا اصْعَدُ فِي وَسَطِكَ لانَّكَ شَعْبٌ صُلْبُ الرَّقَبَةُِ لِئَلا افْنِيَكَ فِي الطَّرِيقِ».
\par 4 فَلَمَّا سَمِعَ الشَّعْبُ هَذَا الْكَلامَ السُّوءَ نَاحُوا وَلَمْ يَضَعْ احَدٌ زِينَتَهُ عَلَيْهِ.
\par 5 وَكَانَ الرَّبُّ قَدْ قَالَ لِمُوسَى: «قُلْ لِبَنِي اسْرَائِيلَ: انْتُمْ شَعْبٌ صُلْبُ الرَّقَبَةُِ. انْ صَعِدْتُ لَحْظَةً وَاحِدَةً فِي وَسَطِكُمْ افْنَيْتُكُمْ. وَلَكِنِ الانَ اخْلَعْ زِينَتَكَ عَنْكَ فَاعْلَمَ مَاذَا اصْنَعُ بِكَ».
\par 6 فَنَزَعَ بَنُو اسْرَائِيلَ زِينَتَهُمْ مِنْ جَبَلِ حُورِيبَ.
\par 7 وَاخَذَ مُوسَى الْخَيْمَةَ وَنَصَبَهَا لَهُ خَارِجَ الْمَحَلَّةِ بَعِيدا عَنِ الْمَحَلَّةِ وَدَعَاهَا «خَيْمَةَ الاجْتِمَاعِ». فَكَانَ كُلُّ مَنْ يَطْلُبُ الرَّبَّ يَخْرُجُ الَى خَيْمَةِ الاجْتِمَاعِ الَّتِي خَارِجَ الْمَحَلَّةِ.
\par 8 وَكَانَ جَمِيعُ الشَّعْبِ اذَا خَرَجَ مُوسَى الَى الْخَيْمَةِ يَقُومُونَ وَيَقِفُونَ كُلُّ وَاحِدٍ فِي بَابِ خَيْمَتِهِ وَيَنْظُرُونَ وَرَاءَ مُوسَى حَتَّى يَدْخُلَ الْخَيْمَةَ.
\par 9 وَكَانَ عَمُودُ السَّحَابِ اذَا دَخَلَ مُوسَى الْخَيْمَةَ يَنْزِلُ وَيَقِفُ عِنْدَ بَابِ الْخَيْمَةِ. وَيَتَكَلَّمُ الرَّبُّ مَعَ مُوسَى
\par 10 فَيَرَى جَمِيعُ الشَّعْبِ عَمُودَ السَّحَابِ وَاقِفا عِنْدَ بَابِ الْخَيْمَةِ. وَيَقُومُ كُلُّ الشَّعْبِ وَيَسْجُدُونَ كُلُّ وَاحِدٍ فِي بَابِ خَيْمَتِهِ.
\par 11 وَيُكَلِّمُ الرَّبُّ مُوسَى وَجْها لِوَجْهٍ كَمَا يُكَلِّمُ الرَّجُلُ صَاحِبَهُ. وَاذَا رَجَعَ مُوسَى الَى الْمَحَلَّةِ كَانَ خَادِمُهُ يَشُوعُ بْنُ نُونَ الْغُلامُ لا يَبْرَحُ مِنْ دَاخِلِ الْخَيْمَةِ.
\par 12 وَقَالَ مُوسَى لِلرَّبِّ: «انْظُرْ! انْتَ قَائِلٌ لِي اصْعِدْ هَذَا الشَّعْبَ وَانْتَ لَمْ تُعَرِّفْنِي مَنْ تُرْسِلُ مَعِي. وَانْتَ قَدْ قُلْتَ: عَرَفْتُكَ بِاسْمِكَ وَوَجَدْتَ ايْضا نِعْمَةً فِي عَيْنَيَّ.
\par 13 فَالانَ انْ كُنْتُ قَدْ وَجَدْتُ نِعْمَةً فِي عَيْنَيْكَ فَعَلِّمْنِي طَرِيقَكَ حَتَّى اعْرِفَكَ لِكَيْ اجِدَ نِعْمَةً فِي عَيْنَيْكَ. وَانْظُرْ انَّ هَذِهِ الامَّةَ شَعْبُكَ».
\par 14 فَقَالَ: «وَجْهِي يَسِيرُ فَارِيحُكَ».
\par 15 فَقَالَ لَهُ: «انْ لَمْ يَسِرْ وَجْهُكَ فَلا تُصْعِدْنَا مِنْ هَهُنَا
\par 16 فَانَّهُ بِمَاذَا يُعْلَمُ انِّي وَجَدْتُ نِعْمَةً فِي عَيْنَيْكَ انَا وَشَعْبُكَ؟ الَيْسَ بِمَسِيرِكَ مَعَنَا؟ فَنَمْتَازَ انَا وَشَعْبُكَ عَنْ جَمِيعِ الشُّعُوبِ الَّذِينَ عَلَى وَجْهِ الارْضِ».
\par 17 فَقَالَ الرَّبُّ لِمُوسَى: «هَذَا الامْرُ ايْضا الَّذِي تَكَلَّمْتَ عَنْهُ افْعَلُهُ لانَّكَ وَجَدْتَ نِعْمَةً فِي عَيْنَيَّ وَعَرَفْتُكَ بِاسْمِكَ».
\par 18 فَقَالَ: «ارِنِي مَجْدَكَ».
\par 19 فَقَالَ: «اجِيزُ كُلَّ جُودَتِي قُدَّامَكَ. وَانَادِي بِاسْمِ الرَّبِّ قُدَّامَكَ. وَاتَرَافُ عَلَى مَنْ اتَرَافُ وَارْحَمُ مَنْ ارْحَمُ».
\par 20 وَقَالَ: «لا تَقْدِرُ انْ تَرَى وَجْهِي لانَّ الْانْسَانَ لا يَرَانِي وَيَعِيشُ».
\par 21 وَقَالَ الرَّبُّ: «هُوَذَا عِنْدِي مَكَانٌ فَتَقِفُ عَلَى الصَّخْرَةِ.
\par 22 وَيَكُونُ مَتَى اجْتَازَ مَجْدِي انِّي اضَعُكَ فِي نُقْرَةٍ مِنَ الصَّخْرَةِ وَاسْتُرُكَ بِيَدِي حَتَّى اجْتَازَ.
\par 23 ثُمَّ ارْفَعُ يَدِي فَتَنْظُرُ وَرَائِي. وَامَّا وَجْهِي فَلا يُرَى».

\chapter{34}

\par 1 ثُمَّ قَالَ الرَّبُّ لِمُوسَى: «انْحَتْ لَكَ لَوْحَيْنِ مِنْ حَجَرٍ مِثْلَ الاوَّلَيْنِ فَاكْتُبَ انَا عَلَى اللَّوْحَيْنِ الْكَلِمَاتِ الَّتِي كَانَتْ عَلَى اللَّوْحَيْنِ الاوَّلَيْنِ اللَّذَيْنِ كَسَرْتَهُمَا.
\par 2 وَكُنْ مُسْتَعِدّا لِلصَّبَاحِ. وَاصْعَدْ فِي الصَّبَاحِ الَى جَبَلِ سِينَاءَ وَقِفْ عِنْدِي هُنَاكَ عَلَى رَاسِ الْجَبَلِ.
\par 3 وَلا يَصْعَدْ احَدٌ مَعَكَ وَايْضا لا يُرَ احَدٌ فِي كُلِّ الْجَبَلِ. الْغَنَمُ ايْضا وَالْبَقَرُ لا تَرْعَ الَى جِهَةِ ذَلِكَ الْجَبَلِ».
\par 4 فَنَحَتَ لَوْحَيْنِ مِنْ حَجَرٍ كَالاوَّلَيْنِ. وَبَكَّرَ مُوسَى فِي الصَّبَاحِ وَصَعِدَ الَى جَبَلِ سِينَاءَ كَمَا امَرَهُ الرَّبُّ وَاخَذَ فِي يَدِهِ لَوْحَيِ الْحَجَرِ.
\par 5 فَنَزَلَ الرَّبُّ فِي السَّحَابِ فَوَقَفَ عِنْدَهُ هُنَاكَ وَنَادَى بِاسْمِ الرَّبِّ.
\par 6 فَاجْتَازَ الرَّبُّ قُدَّامَهُ. وَنَادَى الرَّبُّ: «الرَّبُّ الَهٌ رَحِيمٌ وَرَاوفٌ بَطِيءُ الْغَضَبِ وَكَثِيرُ الْاحْسَانِ وَالْوَفَاءِ.
\par 7 حَافِظُ الْاحْسَانِ الَى الُوفٍ. غَافِرُ الْاثْمِ وَالْمَعْصِيَةِ وَالْخَطِيَّةِ. وَلَكِنَّهُ لَنْ يُبْرِئَ ابْرَاءً. مُفْتَقِدٌ اثْمَ الابَاءِ فِي الابْنَاءِ وَفِي ابْنَاءِ الابْنَاءِ فِي الْجِيلِ الثَّالِثِ وَالرَّابِعِ».
\par 8 فَاسْرَعَ مُوسَى وَخَرَّ الَى الارْضِ وَسَجَدَ.
\par 9 وَقَالَ: «انْ وَجَدْتُ نِعْمَةً فِي عَيْنَيْكَ ايُّهَا السَّيِّدُ فَلْيَسِرِ السَّيِّدُ فِي وَسَطِنَا فَانَّهُ شَعْبٌ صُلْبُ الرَّقَبَةُِ. وَاغْفِرْ اثْمَنَا وَخَطِيَّتَنَا وَاتَّخِذْنَا مُلْكا».
\par 10 فَقَالَ: «هَا انَا قَاطِعٌ عَهْدا. قُدَّامَ جَمِيعِ شَعْبِكَ افْعَلُ عَجَائِبَ لَمْ تُخْلَقْ فِي كُلِّ الارْضِ وَفِي جَمِيعِ الامَمِ فَيَرَى جَمِيعُ الشَّعْبِ الَّذِي انْتَ فِي وَسَطِهِ فِعْلَ الرَّبِّ. انَّ الَّذِي انَا فَاعِلُهُ مَعَكَ رَهِيبٌ.
\par 11 «احْفَظْ مَا انَا مُوصِيكَ الْيَوْمَ. هَا انَا طَارِدٌ مِنْ قُدَّامِكَ الامُورِيِّينَ وَالْكَنْعَانِيِّينَ وَالْحِثِّيِّينَ وَالْفِرِزِّيِّينَ وَالْحِوِّيِّينَ وَالْيَبُوسِيِّينَ.
\par 12 احْتَرِزْ مِنْ انْ تَقْطَعَ عَهْدا مَعَ سُكَّانِ الارْضِ الَّتِي انْتَ اتٍ الَيْهَا لِئَلا يَصِيرُوا فَخّا فِي وَسَطِكَ
\par 13 بَلْ تَهْدِمُونَ مَذَابِحَهُمْ وَتُكَسِّرُونَ انْصَابَهُمْ وَتَقْطَعُونَ سَوَارِيَهُمْ.
\par 14 فَانَّكَ لا تَسْجُدُ لالَهٍ اخَرَ لانَّ الرَّبَّ اسْمُهُ غَيُورٌ. الَهٌ غَيُورٌ هُوَ.
\par 15 احْتَرِزْ مِنْ انْ تَقْطَعَ عَهْدا مَعَ سُكَّانِ الارْضِ فَيَزْنُونَ وَرَاءَ الِهَتِهِمْ وَيَذْبَحُونَ لِالِهَتِهِمْ فَتُدْعَى وَتَاكُلُ مِنْ ذَبِيحَتِهِمْ
\par 16 وَتَاخُذُ مِنْ بَنَاتِهِمْ لِبَنِيكَ فَتَزْنِي بَنَاتُهُمْ وَرَاءَ الِهَتِهِنَّ وَيَجْعَلْنَ بَنِيكَ يَزْنُونَ وَرَاءَ الِهَتِهِنَّ.
\par 17 «لا تَصْنَعْ لِنَفْسِكَ الِهَةً مَسْبُوكَةً.
\par 18 تَحْفَظُ عِيدَ الْفَطِيرِ. سَبْعَةَ ايَّامٍ تَاكُلُ فَطِيرا كَمَا امَرْتُكَ فِي وَقْتِ شَهْرِ ابِيبَ لانَّكَ فِي شَهْرِ ابِيبَ خَرَجْتَ مِنْ مِصْرَ.
\par 19 لِي كُلُّ فَاتِحِ رَحِمٍ وَكُلُّ مَا يُولَدُ ذَكَرا مِنْ مَوَاشِيكَ بِكْرا مِنْ ثَوْرٍ وَشَاةٍ.
\par 20 وَامَّا بِكْرُ الْحِمَارِ فَتَفْدِيهِ بِشَاةٍ. وَانْ لَمْ تَفْدِهِ تَكْسِرُ عُنُقَهُ. كُلُّ بِكْرٍ مِنْ بَنِيكَ تَفْدِيهِ وَلا يَظْهَرُوا امَامِي فَارِغِينَ.
\par 21 سِتَّةَ ايَّامٍ تَعْمَلُ وَامَّا الْيَوْمُ السَّابِعُ فَتَسْتَرِيحُ فِيهِ. فِي الْفِلاحَةِ وَفِي الْحَصَادِ تَسْتَرِيحُ.
\par 22 وَتَصْنَعُ لِنَفْسِكَ عِيدَ الاسَابِيعِ ابْكَارِ حِصَادِ الْحِنْطَةِ. وَعِيدَ الْجَمْعِ فِي اخِرِ السَّنَةِ.
\par 23 ثَلاثَ مَرَّاتٍ فِي السَّنَةِ يَظْهَرُ جَمِيعُ ذُكُورِكَ امَامَ السَّيِّدِ الرَّبِّ الَهِ اسْرَائِيلَ
\par 24 فَانِّي اطْرُدُ الامَمَ مِنْ قُدَّامِكَ وَاوَسِّعُ تُخُومَكَ وَلا يَشْتَهِي احَدٌ ارْضَكَ حِينَ تَصْعَدُ لِتَظْهَرَ امَامَ الرَّبِّ الَهِكَ ثَلاثَ مَرَّاتٍ فِي السَّنَةِ.
\par 25 لا تَذْبَحْ عَلَى خَمِيرٍ دَمَ ذَبِيحَتِي. وَلا تَبِتْ الَى الْغَدِ ذَبِيحَةُ عِيدِ الْفِصْحِ.
\par 26 اوَّلُ ابْكَارِ ارْضِكَ تُحْضِرُهُ الَى بَيْتِ الرَّبِّ الَهِكَ. لا تَطْبُخْ جَدْيا بِلَبَنِ امِّهِ».
\par 27 وَقَالَ الرَّبُّ لِمُوسَى: «اكْتُبْ لِنَفْسِكَ هَذِهِ الْكَلِمَاتِ لانَّنِي بِحَسَبِ هَذِهِ الْكَلِمَاتِ قَطَعْتُ عَهْدا مَعَكَ وَمَعَ اسْرَائِيلَ».
\par 28 وَكَانَ هُنَاكَ عِنْدَ الرَّبِّ ارْبَعِينَ نَهَارا وَارْبَعِينَ لَيْلَةً لَمْ يَاكُلْ خُبْزا وَلَمْ يَشْرَبْ مَاءً. فَكَتَبَ عَلَى اللَّوْحَيْنِ كَلِمَاتِ الْعَهْدِ الْكَلِمَاتِ الْعَشَرَ.
\par 29 وَلَمَّا نَزَلَ مُوسَى مِنْ جَبَلِ سِينَاءَ وَلَوْحَا الشَّهَادَةِ فِي يَدِهِ عِنْدَ نُزُولِهِ مِنَ الْجَبَلِ لَمْ يَكُنْ يَعْلَمُ انَّ جِلْدَ وَجْهِهِ صَارَ يَلْمَعُ مِنْ كَلامِ الرَّبِّ مَعَهُ.
\par 30 فَنَظَرَ هَارُونُ وَجَمِيعُ بَنِي اسْرَائِيلَ مُوسَى وَاذَا جِلْدُ وَجْهِهِ يَلْمَعُ فَخَافُوا انْ يَقْتَرِبُوا الَيْهِ.
\par 31 فَدَعَاهُمْ مُوسَى. فَرَجَعَ الَيْهِ هَارُونُ وَجَمِيعُ الرُّؤَسَاءِ فِي الْجَمَاعَةِ. فَكَلَّمَهُمْ مُوسَى.
\par 32 وَبَعْدَ ذَلِكَ اقْتَرَبَ جَمِيعُ بَنِي اسْرَائِيلَ فَاوْصَاهُمْ بِكُلِّ مَا تَكَلَّمَ بِهِ الرَّبُّ مَعَهُ فِي جَبَلِ سِينَاءَ.
\par 33 وَلَمَّا فَرَغَ مُوسَى مِنَ الْكَلامِ مَعَهُمْ جَعَلَ عَلَى وَجْهِهِ بُرْقُعا.
\par 34 وَكَانَ مُوسَى عِنْدَ دُخُولِهِ امَامَ الرَّبِّ لِيَتَكَلَّمَ مَعَهُ يَنْزِعُ الْبُرْقُعَ حَتَّى يَخْرُجَ. ثُمَّ يَخْرُجُ وَيُكَلِّمُ بَنِي اسْرَائِيلَ بِمَا يُوصَى.
\par 35 فَاذَا رَاى بَنُو اسْرَائِيلَ وَجْهَ مُوسَى انَّ جِلْدَهُ يَلْمَعُ كَانَ مُوسَى يَرُدُّ الْبُرْقُعَ عَلَى وَجْهِهِ حَتَّى يَدْخُلَ لِيَتَكَلَّمَ مَعَهُ.

\chapter{35}

\par 1 وَجَمَعَ مُوسَى كُلَّ جَمَاعَةِ بَنِي اسْرَائِيلَ وَقَالَ لَهُمْ: «هَذِهِ هِيَ الْكَلِمَاتُ الَّتِي امَرَ الرَّبُّ انْ تُصْنَعَ.
\par 2 سِتَّةَ ايَّامٍ يُعْمَلُ عَمَلٌ. وَامَّا الْيَوْمُ السَّابِعُ فَفِيهِ يَكُونُ لَكُمْ سَبْتُ عُطْلَةٍ مُقَدَّسٌ لِلرَّبِّ. كُلُّ مَنْ يَعْمَلُ فِيهِ عَمَلا يُقْتَلُ.
\par 3 لا تُشْعِلُوا نَارا فِي جَمِيعِ مَسَاكِنِكُمْ يَوْمَ السَّبْتِ».
\par 4 وَقَالَ مُوسَى لِكُلَّ جَمَاعَةِ بَنِي اسْرَائِيلَ: «هَذَا هُوَ الشَّيْءُ الَّذِي امَرَ بِهِ الرَّبُّ قَائِلا:
\par 5 خُذُوا مِنْ عِنْدِكُمْ تَقْدِمَةً لِلرَّبِّ. كُلُّ مَنْ قَلْبُهُ سَمُوحٌ فَلْيَاتِ بِتَقْدِمَةِ الرَّبِّ: ذَهَبا وَفِضَّةً وَنُحَاسا
\par 6 وَاسْمَانْجُونِيّا وَارْجُوَانا وَقِرْمِزا وَبُوصا وَشَعْرَ مِعْزًى
\par 7 وَجُلُودَ كِبَاشٍ مُحَمَّرَةً وَجُلُودَ تُخَسٍ وَخَشَبَ سَنْطٍ
\par 8 وَزَيْتا لِلضُّوءِ وَاطْيَابا لِدُهْنِ الْمَسْحَةِ وَلِلْبَخُورِ الْعَطِرِ
\par 9 وَحِجَارَةَ جَزْعٍ وَحِجَارَةَ تَرْصِيعٍ لِلرِّدَاءِ وَالصُّدْرَةِ.
\par 10 وَكُلُّ حَكِيمِ الْقَلْبِ بَيْنَكُمْ فَلْيَاتِ وَيَصْنَعْ كُلَّ مَا امَرَ بِهِ الرَّبُّ.
\par 11 الْمَسْكَنَ وَخَيْمَتَهُ وَغِطَاءَهُ وَاشِظَّتَهُ وَالْوَاحَهُ وَعَوَارِضَهُ وَاعْمِدَتَهُ وَقَوَاعِدَهُ
\par 12 وَالتَّابُوتَ وَعَصَوَيْهِ وَالْغِطَاءَ وَحِجَابَ السَّجْفِ
\par 13 وَالْمَائِدَةَ وَعَصَوَيْهَا وَكُلَّ انِيَتِهَا وَخُبْزَ الْوُجُوهِ
\par 14 وَمَنَارَةَ الضُّوءِ وَانِيَتَهَا وَسُرُجَهَا وَزَيْتَ الضُّوءِ
\par 15 وَمَذْبَحَ الْبَخُورِ وَعَصَوَيْهِ وَدُهْنَ الْمَسْحَةِ وَالْبَخُورَ الْعَطِرَ وَسَجْفَ الْبَابِ لِمَدْخَلِ الْمَسْكَنِ
\par 16 وَمَذْبَحَ الْمُحْرَقَةِ وَشُبَّاكَةَ النُّحَاسِ الَّتِي لَهُ وَعَصَوَيْهِ وَكُلَّ انِيَتِهِ وَالْمِرْحَضَةَ وَقَاعِدَتَهَا
\par 17 وَاسْتَارَ الدَّارِ وَاعْمِدَتَهَا وَقَوَاعِدَهَا وَسَجْفَ بَابِ الدَّارِ
\par 18 وَاوْتَادَ الْمَسْكَنِ وَاوْتَادَ الدَّارِ وَاطْنَابَهَا
\par 19 وَالثِّيَابَ الْمَنْسُوجَةَ لِلْخِدْمَةِ فِي الْمَقْدِسِ وَالثِّيَابَ الْمُقَدَّسَةَ لِهَارُونَ الْكَاهِنِ وَثِيَابَ بَنِيهِ لِلْكَهَانَةِ».
\par 20 فَخَرَجَ كُلُّ جَمَاعَةِ بَنِي اسْرَائِيلَ مِنْ قُدَّامِ مُوسَى
\par 21 ثُمَّ جَاءَ كُلُّ مَنْ انْهَضَهُ قَلْبُهُ وَكُلُّ مَنْ سَمَّحَتْهُ رُوحُهُ بِتَقْدِمَةِ الرَّبِّ لِعَمَلِ خَيْمَةِ الاجْتِمَاعِ وَلِكُلِّ خِدْمَتِهَا وَلِلثِّيَابِ الْمُقَدَّسَةِ.
\par 22 وَجَاءَ الرِّجَالُ مَعَ النِّسَاءِ - كُلُّ سَمُوحِ الْقَلْبِ - بِخَزَائِمَ وَاقْرَاطٍ وَخَوَاتِمَ وَقَلائِدِ كُلِّ مَتَاعٍ مِنَ الذَّهَبِ - وَكُلُّ مَنْ قَدَّمَ تَقْدِمَةَ ذَهَبٍ لِلرَّبِّ
\par 23 وَكُلُّ مَنْ وُجِدَ عِنْدَهُ اسْمَانْجُونِيٌّ وَارْجُوانٌ وَقِرْمِزٌ وَبُوصٌ وَشَعْرُ مِعْزىً وَجُلُودُ كِبَاشٍ مُحَمَّرَةٌ وَجُلُودُ تُخَسٍ جَاءَ بِهَا.
\par 24 كُلُّ مَنْ قَدَّمَ تَقْدِمَةَ فِضَّةٍ وَنُحَاسٍ جَاءَ بِتَقْدِمَةِ الرَّبِّ. وَكُلُّ مَنْ وُجِدَ عِنْدَهُ خَشَبُ سَنْطٍ لِصَنْعَةٍ مَا مِنَ الْعَمَلِ جَاءَ بِهِ.
\par 25 وَكُلُّ النِّسَاءِ الْحَكِيمَاتِ الْقَلْبِ غَزَلْنَ بِايْدِيهِنَّ وَجِئْنَ مِنَ الْغَزْلِ بِالاسْمَانْجُونِيِّ وَالارْجُوَانِ وَالْقِرْمِزِ وَالْبُوصِ.
\par 26 وَكُلُّ النِّسَاءِ اللَّوَاتِي انْهَضَتْهُنَّ قُلُوبُهُنَّ بِالْحِكْمَةِ غَزَلْنَ شَعْرَ الْمِعْزَى.
\par 27 وَالرُّؤَسَاءُ جَاءُوا بِحِجَارَةِ الْجَزْعِ وَحِجَارَةِ التَّرْصِيعِ لِلرِّدَاءِ وَالصُّدْرَةِ
\par 28 وَبِالطِّيبِ وَالزَّيْتِ لِلضُّوءِ وَلِدُهْنِ الْمَسْحَةِ وَلِلْبَخُورِ الْعَطِرِ.
\par 29 بَنُو اسْرَائِيلَ جَمِيعُ الرِّجَالِ وَالنِّسَاءِ الَّذِينَ سَمَّحَتْهُمْ قُلُوبُهُمْ انْ يَاتُوا بِشَيْءٍ لِكُلِّ الْعَمَلِ الَّذِي امَرَ الرَّبُّ انْ يُصْنَعَ عَلَى يَدِ مُوسَى جَاءُوا بِهِ تَبَرُّعا الَى الرَّبِّ.
\par 30 وَقَالَ مُوسَى لِبَنِي اسْرَائِيلَ: «انْظُرُوا! قَدْ دَعَا الرَّبُّ بَصَلْئِيلَ بْنَ اورِي بْنَ حُورَ مِنْ سِبْطِ يَهُوذَا بِاسْمِهِ
\par 31 وَمَلاهُ مِنْ رُوحِ اللهِ بِالْحِكْمَةِ وَالْفَهْمِ وَالْمَعْرِفَةِ وَكُلِّ صَنْعَةٍ
\par 32 وَلاخْتِرَاعِ مُخْتَرَعَاتٍ لِيَعْمَلَ فِي الذَّهَبِ وَالْفِضَّةِ وَالنُّحَاسِ
\par 33 وَنَقْشِ حِجَارَةٍ لِلتَّرْصِيعِ وَنِجَارَةِ الْخَشَبِ لِيَعْمَلَ فِي كُلِّ صَنْعَةٍ مِنَ الْمُخْتَرَعَاتِ.
\par 34 وَجَعَلَ فِي قَلْبِهِ انْ يُعَلِّمَ هُوَ وَاهُولِيابُ بْنَ اخِيسَامَاكَ مِنْ سِبْطِ دَانَ.
\par 35 قَدْ مَلاهُمَا حِكْمَةَ قَلْبٍ لِيَصْنَعَا كُلَّ عَمَلِ النَّقَّاشِ وَالْحَائِكِ الْحَاذِقِ وَالطَّرَّازِ فِي الاسْمَانْجُونِيِّ وَالارْجُوَانِ وَالْقِرْمِزِ وَالْبُوصِ وَكُلَّ عَمَلِ النَّسَّاجِ. صَانِعِي كُلِّ صَنْعَةٍ وَمُخْتَرِعِي الْمُخْتَرَعَاتِ. (36:1) «فَيَعْمَلُ بَصَلْئِيلُ وَاهُولِيابُ وَكُلُّ انْسَانٍ حَكِيمِ الْقَلْبِ قَدْ جَعَلَ فِيهِ الرَّبُّ حِكْمَةً وَفَهْما لِيَعْرِفَ انْ يَصْنَعَ صَنْعَةً مَا مِنْ عَمَلِ الْمَقْدِسِ بِحَسَبِ كُلِّ مَا امَرَ الرَّبُّ».

\chapter{36}

\par 1 فيعمل بصلئيل وأهوليآب وكل انسان حكيم القلب قد جعل فيه الرب
\par 2 فَدَعَا مُوسَى بَصَلْئِيلَ وَاهُولِيابَ وَكُلَّ رَجُلٍ حَكِيمِ الْقَلْبِ قَدْ جَعَلَ الرَّبُّ حِكْمَةً فِي قَلْبِهِ. كُلَّ مَنْ انْهَضَهُ قَلْبُهُ انْ يَتَقَدَّمَ الَى الْعَمَلِ لِيَصْنَعَهُ.
\par 3 فَاخَذُوا مِنْ قُدَّامِ مُوسَى كُلَّ التَّقْدِمَةِ الَّتِي جَاءَ بِهَا بَنُو اسْرَائِيلَ لِصَنْعَةِ عَمَلِ الْمَقْدِسِ لِيَصْنَعُوهُ. وَهُمْ جَاءُوا الَيْهِ ايْضا بِشَيْءٍ تَبَرُّعا كُلَّ صَبَاحٍ.
\par 4 فَجَاءَ كُلُّ الْحُكَمَاءِ الصَّانِعِينَ كُلَّ عَمَلِ الْمَقْدِسِ كُلُّ وَاحِدٍ مِنْ عَمَلِهِ الَّذِي هُمْ يَصْنَعُونَهُ.
\par 5 وَقَالُوا لِمُوسَى: «يَجِيءُ الشَّعْبُ بِكَثِيرٍ فَوْقَ حَاجَةِ الْعَمَلِ لِلصَّنْعَةِ الَّتِي امَرَ الرَّبُّ بِصُنْعِهَا».
\par 6 فَامَرَ مُوسَى انْ يُنْفِذُوا صَوْتا فِي الْمَحَلَّةِ قَائِلِينَ: «لا يَصْنَعْ رَجُلٌ اوِ امْرَاةٌ عَمَلا ايْضا لِتَقْدِمَةِ الْمَقْدِسِ». فَامْتَنَعَ الشَّعْبُ عَنِ الْجَلَبِ.
\par 7 وَالْمَوَادُّ كَانَتْ كِفَايَتَهُمْ لِكُلِّ الْعَمَلِ لِيَصْنَعُوهُ وَاكْثَرَ.
\par 8 فَصَنَعُوا كُلُّ حَكِيمِ قَلْبٍ مِنْ صَانِعِي الْعَمَلِ الْمَسْكَنَ عَشَرَ شُقَقٍ. مِنْ بُوصٍ مَبْرُومٍ وَاسْمَانْجُونِيٍّ وَارْجُوانٍ وَقِرْمِزٍ بِكَرُوبِيمَ صَنْعَةَ حَائِكٍ حَاذِقٍ صَنَعَهَا.
\par 9 طُولُ الشُّقَّةِ الْوَاحِدَةِ ثَمَانٍ وَعِشْرُونَ ذِرَاعا وَعَرْضُ الشُّقَّةِ الْوَاحِدَةِ ارْبَعُ اذْرُعٍ. قِيَاسا وَاحِدا لِجَمِيعِ الشُّقَقِ.
\par 10 وَوَصَلَ خَمْسا مِنَ الشُّقَقِ بَعْضَهَا بِبَعْضٍ. وَوَصَلَ خَمْسا مِنَ الشُّقَقِ بَعْضَهَا بِبَعْضٍ.
\par 11 وَصَنَعَ عُرىً مِنْ اسْمَانْجُونِيٍّ عَلَى حَاشِيَةِ الشُّقَّةِ الْوَاحِدَةِ فِي الطَّرَفِ مِنَ الْمُوَصَّلِ الْوَاحِدِ. كَذَلِكَ صَنَعَ فِي حَاشِيَةِ الشُّقَّةِ الطَّرَفِيَّةِ مِنَ الْمُوَصَّلِ الثَّانِي.
\par 12 خَمْسِينَ عُرْوَةً صَنَعَ فِي الشُّقَّةِ الْوَاحِدَةِ وَخَمْسِينَ عُرْوَةً صَنَعَ فِي طَرَفِ الشُّقَّةِ الَّذِي فِي الْمُوَصَّلِ الثَّانِي. مُقَابِلَةً كَانَتِ الْعُرَى بَعْضُهَا لِبَعْضٍ.
\par 13 وَصَنَعَ خَمْسِينَ شِظَاظا مِنْ ذَهَبٍ وَوَصَلَ الشُّقَّتَيْنِ بَعْضَهُمَا بِبَعْضٍ بِالاشِظَّةِ فَصَارَ الْمَسْكَنُ وَاحِدا.
\par 14 وَصَنَعَ شُقَقا مِنْ شَعْرِ مِعْزًى خَيْمَةً فَوْقَ الْمَسْكَنِ. احْدَى عَشَرَةَ شُقَّةً صَنَعَهَا.
\par 15 طُولُ الشُّقَّةِ الْوَاحِدَةِ ثَلاثُونَ ذِرَاعا وَعَرْضُ الشُّقَّةِ الْوَاحِدَةِ ارْبَعُ اذْرُعٍ. قِيَاسا وَاحِدا لِلْاحْدَى عَشَرَةَ شُقَّةً.
\par 16 وَوَصَلَ خَمْسا مِنَ الشُّقَقِ وَحْدَهَا وَسِتّا مِنَ الشُّقَقِ وَحْدَهَا.
\par 17 وَصَنَعَ خَمْسِينَ عُرْوَةً عَلَى حَاشِيَةِ الشُّقَّةِ الطَّرَفِيَّةِ مِنَ الْمُوَصَّلِ الْوَاحِدِ. وَصَنَعَ خَمْسِينَ عُرْوَةً عَلَى حَاشِيَةِ الشُّقَّةِ الْمُوَصَّلَةِ الثَّانِيَةِ.
\par 18 وَصَنَعَ خَمْسِينَ شِظَاظا مِنْ نُحَاسٍ لِيَصِلَ الْخَيْمَةَ لِتَصِيرَ وَاحِدَةً.
\par 19 وَصَنَعَ غِطَاءً لِلْخَيْمَةِ مِنْ جُلُودِ كِبَاشٍ مُحَمَّرَةً وَغِطَاءً مِنْ جُلُودِ تُخَسٍ مِنْ فَوْقُ.
\par 20 وَصَنَعَ الالْوَاحَ لِلْمَسْكَنِ مِنْ خَشَبِ السَّنْطِ قَائِمَةً
\par 21 طُولُ اللَّوْحِ عَشَرُ اذْرُعٍ وَعَرْضُ اللَّوْحِ الْوَاحِدِ ذِرَاعٌ وَنِصْفٌ.
\par 22 وَلِلَّوْحِ الْوَاحِدِ رِجْلانِ مَقْرُونَةٌ احْدَاهُمَا بِالاخْرَى. هَكَذَا صَنَعَ لِجَمِيعِ الْوَاحِ الْمَسْكَنِ.
\par 23 وَصَنَعَ الالْوَاحَ لِلْمَسْكَنِ عِشْرِينَ لَوْحا الَى جِهَةِ الْجَنُوبِ نَحْوَ التَّيْمَنِ.
\par 24 وَصَنَعَ ارْبَعِينَ قَاعِدَةً مِنْ فِضَّةٍ تَحْتَ الْعِشْرِينَ لَوْحا تَحْتَ اللَّوْحِ الْوَاحِدِ قَاعِدَتَانِ لِرِجْلَيْهِ وَتَحْتَ اللَّوْحِ الْوَاحِدِ قَاعِدَتَانِ لِرِجْلَيْهِ.
\par 25 وَلِجَانِبِ الْمَسْكَنِ الثَّانِي الَى جِهَةِ الشِّمَالِ صَنَعَ عِشْرِينَ لَوْحا
\par 26 وَارْبَعِينَ قَاعِدَةً لَهَا مِنْ فِضَّةٍ. تَحْتَ اللَّوْحِ الْوَاحِدِ قَاعِدَتَانِ وَتَحْتَ اللَّوْحِ الْوَاحِدِ قَاعِدَتَانِ.
\par 27 وَلِمُؤَخَّرِ الْمَسْكَنِ نَحْوَ الْغَرْبِ صَنَعَ سِتَّةَ الْوَاحٍ.
\par 28 وَصَنَعَ لَوْحَيْنِ لِزَاوِيَتَيِ الْمَسْكَنِ فِي الْمُؤَخَّرِ.
\par 29 وَكَانَا مُزْدَوِجَيْنِ مِنْ اسْفَلُ. وَعَلَى سَوَاءٍ كَانَا مُزْدَوِجَيْنِ الَى رَاسِهِ الَى الْحَلْقَةِ الْوَاحِدَةِ. هَكَذَا صَنَعَ لِكِلْتَيْهِمَا لِكِلْتَا الزَّاوِيَتَيْنِ.
\par 30 فَكَانَتْ ثَمَانِيَةَ الْوَاحٍ وَقَوَاعِدُهَا مِنْ فِضَّةٍ سِتَّ عَشَرَةَ قَاعِدَةً. قَاعِدَتَيْنِ قَاعِدَتَيْنِ تَحْتَ اللَّوْحِ الْوَاحِدِ.
\par 31 وَصَنَعَ عَوَارِضَ مِنْ خَشَبِ السَّنْطِ خَمْسا لالْوَاحِ جَانِبِ الْمَسْكَنِ الْوَاحِدِ
\par 32 وَخَمْسَ عَوَارِضَ لالْوَاحِ جَانِبِ الْمَسْكَنِ الثَّانِي وَخَمْسَ عَوَارِضَ لالْوَاحِ الْمَسْكَنِ فِي الْمُؤَخَّرِ نَحْوَ الْغَرْبِ.
\par 33 وَصَنَعَ الْعَارِضَةَ الْوُسْطَى لِتَنْفُذَ فِي وَسَطِ الالْوَاحِ مِنَ الطَّرَفِ الَى الطَّرَفِ.
\par 34 وَغَشَّى الالْوَاحَ بِذَهَبٍ. وَصَنَعَ حَلَقَاتِهَا مِنْ ذَهَبٍ بُيُوتا لِلْعَوَارِضِ وَغَشَّى الْعَوَارِضَ بِذَهَبٍ.
\par 35 وَصَنَعَ الْحِجَابَ مِنْ اسْمَانْجُونِيٍّ وَارْجُوَانٍ وَقِرْمِزٍ وَبُوصٍ مَبْرُومٍ. صَنْعَةَ حَائِكٍ حَاذِقٍ صَنَعَهُ بِكَرُوبِيمَ.
\par 36 وَصَنَعَ لَهُ ارْبَعَةَ اعْمِدَةٍ مِنْ سَنْطٍ وَغَشَّاهَا بِذَهَبٍ. رُزَزُهَا مِنْ ذَهَبٍ. وَسَبَكَ لَهَا ارْبَعَ قَوَاعِدَ مِنْ فِضَّةٍ.
\par 37 وَصَنَعَ سَجْفا لِمَدْخَلِ الْخَيْمَةِ مِنْ اسْمَانْجُونِيٍّ وَارْجُوَانٍ وَقِرْمِزٍ وَبُوصٍ مَبْرُومٍ صَنْعَةَ الطَّرَّازِ.
\par 38 وَاعْمِدَتَهُ خَمْسَةً وَرُزَزَهَا. وَغَشَّى رُؤُوسَهَا وَقُضْبَانَهَا بِذَهَبٍ. وَقَوَاعِدَهَا خَمْسا مِنْ نُحَاسٍ.

\chapter{37}

\par 1 وَصَنَعَ بَصَلْئِيلُ التَّابُوتَ مِنْ خَشَبِ السَّنْطِ طُولُهُ ذِرَاعَانِ وَنِصْفٌ وَعَرْضُهُ ذِرَاعٌ وَنِصْفٌ وَارْتِفَاعُهُ ذِرَاعٌ وَنِصْفٌ.
\par 2 وَغَشَّاهُ بِذَهَبٍ نَقِيٍّ مِنْ دَاخِلٍ وَمِنْ خَارِجٍ. وَصَنَعَ لَهُ اكْلِيلا مِنْ ذَهَبٍ حَوَالَيْهِ.
\par 3 وَسَبَكَ لَهُ ارْبَعَ حَلَقَاتٍ مِنْ ذَهَبٍ عَلَى ارْبَعِ قَوَائِمِهِ. عَلَى جَانِبِهِ الْوَاحِدِ حَلْقَتَانِ وَعَلَى جَانِبِهِ الثَّانِي حَلْقَتَانِ.
\par 4 وَصَنَعَ عَصَوَيْنِ مِنْ خَشَبِ السَّنْطِ وَغَشَّاهُمَا بِذَهَبٍ.
\par 5 وَادْخَلَ الْعَصَوَيْنِ فِي الْحَلَقَاتِ عَلَى جَانِبَيِ التَّابُوتِ لِحَمْلِ التَّابُوتِ.
\par 6 وَصَنَعَ غِطَاءً مِنْ ذَهَبٍ نَقِيٍّ طُولُهُ ذِرَاعَانِ وَنِصْفٌ وَعَرْضُهُ ذِرَاعٌ وَنِصْفٌ.
\par 7 وَصَنَعَ كَرُوبَيْنِ مِنْ ذَهَبٍ صَنْعَةَ الْخِرَاطَةِ صَنَعَهُمَا عَلَى طَرَفَيِ الْغِطَاءِ.
\par 8 كَرُوبا وَاحِدا عَلَى الطَّرَفِ مِنْ هُنَا وَكَرُوبا وَاحِدا عَلَى الطَّرَفِ مِنْ هُنَاكَ. مِنَ الْغِطَاءِ صَنَعَ الْكَرُوبَيْنِ عَلَى طَرَفَيْهِ.
\par 9 وَكَانَ الْكَرُوبَانِ بَاسِطَيْنِ اجْنِحَتَهُمَا الَى فَوْقُ مُظَلِّلَيْنِ بِاجْنِحَتِهِمَا فَوْقَ الْغِطَاءِ وَوَجْهَاهُمَا كُلُّ الْوَاحِدِ الَى الاخَرِ. نَحْوَ الْغِطَاءِ كَانَ وَجْهَا الْكَرُوبَيْنِ.
\par 10 وَصَنَعَ الْمَائِدَةَ مِنْ خَشَبِ السَّنْطِ طُولُهَا ذِرَاعَانِ وَعَرْضُهَا ذِرَاعٌ وَارْتِفَاعُهَا ذِرَاعٌ وَنِصْفٌ.
\par 11 وَغَشَّاهَا بِذَهَبٍ نَقِيٍّ. وَصَنَعَ لَهَا اكْلِيلا مِنْ ذَهَبٍ حَوَالَيْهَا.
\par 12 وَصَنَعَ لَهَا حَاجِبا بِعَرْضِ شِبْرٍ حَوَالَيْهَا. وَصَنَعَ لِحَاجِبِهَا اكْلِيلا مِنْ ذَهَبٍ حَوَالَيْهَا.
\par 13 وَسَبَكَ لَهَا ارْبَعَ حَلَقَاتٍ مِنْ ذَهَبٍ. وَجَعَلَ الْحَلَقَاتِ عَلَى الزَّوَايَا الارْبَعِ الَّتِي لِقَوَائِمِهَا الارْبَعِ.
\par 14 عِنْدَ الْحَاجِبِ كَانَتِ الْحَلَقَاتُ بُيُوتا لِلْعَصَوَيْنِ لِحَمْلِ الْمَائِدَةِ.
\par 15 وَصَنَعَ الْعَصَوَيْنِ مِنْ خَشَبِ السَّنْطِ وَغَشَّاهُمَا بِذَهَبٍ لِحَمْلِ الْمَائِدَةِ.
\par 16 وَصَنَعَ الاوَانِيَ الَّتِي عَلَى الْمَائِدَةِ صِحَافَهَا وَصُحُونَهَا وَجَامَاتِهَا وَكَاسَاتِهَا الَّتِي يُسْكَبُ بِهَا مِنْ ذَهَبٍ نَقِيٍّ.
\par 17 وَصَنَعَ الْمَنَارَةَ مِنْ ذَهَبٍ نَقِيٍّ. صَنْعَةَ الْخِرَاطَةِ صَنَعَ الْمَنَارَةَ قَاعِدَتَهَا وَسَاقَهَا. كَانَتْ كَاسَاتُهَا وَعُجَرُهَا وَازْهَارُهَا مِنْهَا.
\par 18 وَسِتُّ شُعَبٍ خَارِجَةٌ مِنْ جَانِبَيْهَا. مِنْ جَانِبِهَا الْوَاحِدِ ثَلاثُ شُعَبِ مَنَارَةٍ. وَمِنْ جَانِبِهَا الثَّانِي ثَلاثُ شُعَبِ مَنَارَةٍ.
\par 19 فِي الشُّعْبَةِ الْوَاحِدَةِ ثَلاثُ كَاسَاتٍ لَوْزِيَّةٍ بِعُجْرَةٍ وَزَهْرٍ. وَفِي الشُّعْبَةِ الثَّانِيَةِ ثَلاثُ كَاسَاتٍ لَوْزِيَّةٍ بِعُجْرَةٍ وَزَهْرٍ. وَهَكَذَا الَى السِّتِّ الشُّعَبِ الْخَارِجَةِ مِنَ الْمَنَارَةِ.
\par 20 وَفِي الْمَنَارَةِ ارْبَعُ كَاسَاتٍ لَوْزِيَّةٍ بِعُجَرِهَا وَازْهَارِهَا.
\par 21 وَتَحْتَ الشُّعْبَتَيْنِ مِنْهَا عُجْرَةٌ وَتَحْتَ الشُّعْبَتَيْنِ مِنْهَا عُجْرَةٌ وَتَحْتَ الشُّعْبَتَيْنِ مِنْهَا عُجْرَةٌ. الَى السِّتِّ الشُّعَبِ الْخَارِجَةِ مِنْهَا.
\par 22 كَانَتْ عُجَرُهَا وَشُعَبُهَا مِنْهَا. جَمِيعُهَا خِرَاطَةٌ وَاحِدَةٌ مِنْ ذَهَبٍ نَقِيٍّ.
\par 23 وَصَنَعَ سُرُجَهَا سَبْعَةً وَمَلاقِطَهَا وَمَنَافِضَهَا مِنْ ذَهَبٍ نَقِيٍّ.
\par 24 مِنْ وَزْنَةِ ذَهَبٍ نَقِيٍّ صَنَعَهَا وَجَمِيعَ اوَانِيهَا.
\par 25 وَصَنَعَ مَذْبَحَ الْبَخُورِ مِنْ خَشَبِ السَّنْطِ طُولُهُ ذِرَاعٌ وَعَرْضُهُ ذِرَاعٌ. مُرَبَّعا. وَارْتِفَاعُهُ ذِرَاعَانِ. مِنْهُ كَانَتْ قُرُونُهُ.
\par 26 وَغَشَّاهُ بِذَهَبٍ نَقِيٍّ سَطْحَهُ وَحِيطَانَهُ حَوَالَيْهِ وَقُرُونَهُ. وَصَنَعَ لَهُ اكْلِيلا مِنْ ذَهَبٍ حَوَالَيْهِ.
\par 27 وَصَنَعَ لَهُ حَلْقَتَيْنِ مِنْ ذَهَبٍ تَحْتَ اكْلِيلِهِ عَلَى جَانِبَيْهِ عَلَى الْجَانِبَيْنِ بَيْتَيْنِ لِعَصَوَيْنِ لِحَمْلِهِ بِهِمَا.
\par 28 وَصَنَعَ الْعَصَوَيْنِ مِنْ خَشَبِ السَّنْطِ وَغَشَّاهُمَا بِذَهَبٍ.
\par 29 وَصَنَعَ دُهْنَ الْمَسْحَةِ مُقَدَّسا. وَالْبَخُورَ الْعَطِرَ نَقِيّا صَنْعَةَ الْعَطَّارِ.

\chapter{38}

\par 1 وَصَنَعَ مَذْبَحَ الْمُحْرَقَةِ مِنْ خَشَبِ السَّنْطِ. طُولُهُ خَمْسُ اذْرُعٍ وَعَرْضُهُ خَمْسُ اذْرُعٍ. مُرَبَّعا. وَارْتِفَاعُهُ ثَلاثُ اذْرُعٍ.
\par 2 وَصَنَعَ قُرُونَهُ عَلَى زَوَايَاهُ الارْبَعِ. مِنْهُ كَانَتْ قُرُونُهُ. وَغَشَّاهُ بِنُحَاسٍ.
\par 3 وَصَنَعَ جَمِيعَ انِيَةِ الْمَذْبَحِ: الْقُدُورَ وَالرُّفُوشَ وَالْمَرَاكِنَ وَالْمَنَاشِلَ وَالْمَجَامِرَ جَمِيعَ انِيَتِهِ صَنَعَهَا مِنْ نُحَاسٍ.
\par 4 وَصَنَعَ لِلْمَذْبَحِ شُبَّاكَةً صَنْعَةَ الشَّبَكَةِ مِنْ نُحَاسٍ تَحْتَ حَاجِبِهِ مِنْ اسْفَلُ الَى نِصْفِهِ.
\par 5 وَسَكَبَ ارْبَعَ حَلَقَاتٍ فِي الارْبَعَةِ الاطْرَافِ لِشُبَّاكَةِ النُّحَاسِ بُيُوتا لِلْعَصَوَيْنِ.
\par 6 وَصَنَعَ الْعَصَوَيْنِ مِنْ خَشَبِ السَّنْطِ وَغَشَّاهُمَا بِنُحَاسٍ.
\par 7 وَادْخَلَ الْعَصَوَيْنِ فِي الْحَلَقَاتِ عَلَى جَانِبَيِ الْمَذْبَحِ لِحَمْلِهِ بِهِمَا. مُجَوَّفا صَنَعَهُ مِنْ الْوَاحٍ.
\par 8 وَصَنَعَ الْمِرْحَضَةَ مِنْ نُحَاسٍ وَقَاعِدَتَهَا مِنْ نُحَاسٍ. مِنْ مَرَائِي الْمُتَجَنِّدَاتِ اللَّوَاتِي تَجَنَّدْنَ عِنْدَ بَابِ خَيْمَةِ الاجْتِمَاعِ.
\par 9 وَصَنَعَ الدَّارَ. الَى جِهَةِ الْجَنُوبِ نَحْوَ التَّيْمَنِ اسْتَارُ الدَّارِ مِنْ بُوصٍ مَبْرُومٍ مِئَةُ ذِرَاعٍ.
\par 10 اعْمِدَتُهَا عِشْرُونَ وَقَوَاعِدُهَا عِشْرُونَ مِنْ نُحَاسٍ. رُزَزُ الاعْمِدَةِ وَقُضْبَانُهَا مِنْ فِضَّةٍ.
\par 11 وَالَى جِهَةِ الشِّمَالِ مِئَةُ ذِرَاعٍ. اعْمِدَتُهَا عِشْرُونَ وَقَوَاعِدُهَا عِشْرُونَ مِنْ نُحَاسٍ. رُزَزُ الاعْمِدَةِ وَقُضْبَانُهَا مِنْ فِضَّةٍ.
\par 12 وَالَى جِهَةِ الْغَرْبِ اسْتَارٌ خَمْسُونَ ذِرَاعا. اعْمِدَتُهَا عَشَرَةٌ وَقَوَاعِدُهَا عَشَرٌ. رُزَزُ الاعْمِدَةِ وَقُضْبَانُهَا مِنْ فِضَّةٍ.
\par 13 وَالَى جِهَةِ الشَّرْقِ نَحْوَ الشُّرُوقِ خَمْسُونَ ذِرَاعا.
\par 14 لِلْجَانِبِ الْوَاحِدِ اسْتَارٌ خَمْسَ عَشَرَةَ ذِرَاعا. اعْمِدَتُهَا ثَلاثَةٌ وَقَوَاعِدُهَا ثَلاثٌ.
\par 15 وَلِلْجَانِبِ الثَّانِي مِنْ بَابِ الدَّارِ الَى هُنَا وَالَى هُنَا اسْتَارٌ خَمْسَ عَشَرَةَ ذِرَاعا. اعْمِدَتُهَا ثَلاثَةٌ وَقَوَاعِدُهَا ثَلاثٌ.
\par 16 جَمِيعُ اسْتَارِ الدَّارِ حَوَالَيْهَا مِنْ بُوصٍ مَبْرُومٍ
\par 17 وَقَوَاعِدُ الاعْمِدَةِ مِنْ نُحَاسٍ. رُزَزُ الاعْمِدَةِ وَقُضْبَانُهَا مِنْ فِضَّةٍ وَتَغْشِيَةُ رُؤُوسِهَا مِنْ فِضَّةٍ وَجَمِيعُ اعْمِدَةِ الدَّارِ مَوْصُولَةٌ بِقُضْبَانٍ مِنْ فِضَّةٍ.
\par 18 وَسَجْفُ بَابِ الدَّارِ صَنْعَةَ الطَّرَّازِ مِنْ اسْمَانْجُونِيٍّ وَارْجُوَانٍ وَقِرْمِزٍ وَبُوصٍ مَبْرُومٍ. وَطُولُهُ عِشْرُونَ ذِرَاعا وَارْتِفَاعُهُ بِالْعَرْضِ خَمْسُ اذْرُعٍ بِسَوِيَّةِ اسْتَارِ الدَّارِ.
\par 19 وَاعْمِدَتُهَا ارْبَعَةٌ وَقَوَاعِدُهَا ارْبَعٌ مِنْ نُحَاسٍ. رُزَزُهَا مِنْ فِضَّةٍ وَتَغْشِيَةُ رُؤُوسِهَا وَقُضْبَانِهَا مِنْ فِضَّةٍ.
\par 20 وَجَمِيعُ اوْتَادِ الْمَسْكَنِ وَالدَّارِ حَوَالَيْهَا مِنْ نُحَاسٍ.
\par 21 هَذَا هُوَ الْمَحْسُوبُ لِلْمَسْكَنِ مَسْكَنِ الشَّهَادَةِ الَّذِي حُسِبَ بِمُوجَبِ امْرِ مُوسَى بِخِدْمَةِ اللاوِيِّينَ عَلَى يَدِ ايثَامَارَ بْنِ هَارُونَ الْكَاهِنِ.
\par 22 وَبَصَلْئِيلُ بْنُ اورِي بْنِ حُورَ مِنْ سِبْطِ يَهُوذَا صَنَعَ كُلَّ مَا امَرَ بِهِ الرَّبُّ مُوسَى.
\par 23 وَمَعَهُ اهُولِيابُ بْنُ اخِيسَامَاكَ مِنْ سِبْطِ دَانَ نَقَّاشٌ وَمُوَشٍّ وَطَرَّازٌ بِالاسْمَانْجُونِيِّ وَالارْجُوَانِ وَالْقِرْمِزِ وَالْبُوصِ.
\par 24 كُلُّ الذَّهَبِ الْمَصْنُوعِ لِلْعَمَلِ فِي جَمِيعِ عَمَلِ الْمَقْدِسِ وَهُوَ ذَهَبُ التَّقْدِمَةِ: تِسْعٌ وَعِشْرُونَ وَزْنَةً وَسَبْعُ مِئَةِ شَاقِلٍ وَثَلاثُونَ شَاقِلا بِشَاقِلِ الْمَقْدِسِ.
\par 25 وَفِضَّةُ الْمَعْدُودِينَ مِنَ الْجَمَاعَةِ مِئَةُ وَزْنَةٍ وَالْفٌ وَسَبْعُ مِئَةِ شَاقِلٍ وَخَمْسَةٌ وَسَبْعُونَ شَاقِلا بِشَاقِلِ الْمَقْدِسِ.
\par 26 لِلرَّاسِ نِصْفُ الشَّاقِلِ بِشَاقِلِ الْمَقْدِسِ. لِكُلِّ مَنِ اجْتَازَ الَى الْمَعْدُودِينَ مِنِ ابْنِ عِشْرِينَ سَنَةً فَصَاعِدا. لِسِتِّ مِئَةِ الْفٍ وَثَلاثَةِ الافٍ وَخَمْسِ مِئَةٍ وَخَمْسِينَ.
\par 27 وَكَانَتْ مِئَةُ وَزْنَةٍ مِنَ الْفِضَّةِ لِسَبْكِ قَوَاعِدِ الْمَقْدِسِ وَقَوَاعِدِ الْحِجَابِ. مِئَةُ قَاعِدَةٍ لِلْمِئَةِ وَزْنَةٍ. وَزْنَةٌ لِلْقَاعِدَةِ.
\par 28 وَالالْفُ وَالسَّبْعُ مِئَةِ شَاقِلٍ وَالْخَمْسَةُ وَالسَّبْعُونَ شَاقِلا صَنَعَ مِنْهَا رُزَزا لِلاعْمِدَةِ وَغَشَّى رُؤُوسَهَا وَوَصَلَهَا بِقُضْبَانٍ.
\par 29 وَنُحَاسُ التَّقْدِمَةِ سَبْعُونَ وَزْنَةً وَالْفَانِ وَارْبَعُ مِئَةِ شَاقِلٍ.
\par 30 وَمِنْهُ صَنَعَ قَوَاعِدَ بَابِ خَيْمَةِ الاجْتِمَاعِ وَمَذْبَحَ النُّحَاسِ وَشُبَّاكَةَ النُّحَاسِ الَّتِي لَهُ وَجَمِيعَ انِيَةِ الْمَذْبَحِ
\par 31 وَقَوَاعِدَ الدَّارِ حَوَالَيْهَا وَقَوَاعِدَ بَابِ الدَّارِ وَجَمِيعَ اوْتَادِ الْمَسْكَنِ وَجَمِيعَ اوْتَادِ الدَّارِ حَوَالَيْهَا.

\chapter{39}

\par 1 وَمِنَ الاسْمَانْجُونِيِّ وَالارْجُوَانِ وَالْقِرْمِزِ صَنَعُوا ثِيَابا مَنْسُوجَةً لِلْخِدْمَةِ فِي الْمَقْدِسِ وَصَنَعُوا الثِّيَابَ الْمُقَدَّسَةَ الَّتِي لِهَارُونَ. كَمَا امَرَ الرَّبُّ مُوسَى.
\par 2 فَصَنَعَ الرِّدَاءَ مِنْ ذَهَبٍ وَاسْمَانْجُونِيٍّ وَارْجُوَانٍ وَقِرْمِزٍ وَبُوصٍ مَبْرُومٍ.
\par 3 وَمَدُّوا الذَّهَبَ صَفَائِحَ وَقَدُّوهَا خُيُوطا لِيَصْنَعُوهَا فِي وَسَطِ الاسْمَانْجُونِيِّ وَالارْجُوَانِ وَالْقِرْمِزِ وَالْبُوصِ صَنْعَةَ الْمُوَشِّي.
\par 4 وَصَنَعُوا لَهُ كَتِفَيْنِ مَوْصُولَيْنِ. عَلَى طَرَفَيْهِ اتَّصَلَ.
\par 5 وَزُنَّارُ شَدِّهِ الَّذِي عَلَيْهِ كَانَ مِنْهُ كَصَنْعَتِهِ مِنْ ذَهَبٍ وَاسْمَانْجُونِيٍّ وَقِرْمِزٍ وَبُوصٍ مَبْرُومٍ - كَمَا امَرَ الرَّبُّ مُوسَى.
\par 6 وَصَنَعُوا حَجَرَيِ الْجَزْعِ مُحَاطَيْنِ بِطَوْقَيْنِ مِنْ ذَهَبٍ مَنْقُوشَيْنِ نَقْشَ الْخَاتِمِ عَلَى حَسَبِ اسْمَاءِ بَنِي اسْرَائِيلَ.
\par 7 وَوَضَعَهُمَا عَلَى كَتِفَيِ الرِّدَاءِ حَجَرَيْ تِذْكَارٍ لِبَنِي اسْرَائِيلَ - كَمَا امَرَ الرَّبُّ مُوسَى.
\par 8 وَصَنَعَ الصُّدْرَةَ صَنْعَةَ الْمُوَشِّي كَصَنْعَةِ الرِّدَاءِ مِنْ ذَهَبٍ وَاسْمَانْجُونِيٍّ وَارْجُوَانٍ وَقِرْمِزٍ وَبُوصٍ مَبْرُومٍ.
\par 9 كَانَتْ مُرَبَّعَةً. مَثْنِيَّةً صَنَعُوا الصُّدْرَةَ. طُولُهَا شِبْرٌ وَعَرْضُهَا شِبْرٌ مَثْنِيَّةً.
\par 10 وَرَصَّعُوا فِيهَا ارْبَعَةَ صُفُوفِ حِجَارَةٍ. صَفٌّ عَقِيقٌ احْمَرُ وَيَاقُوتٌ اصْفَرُ وَزُمُرُّدٌ. الصَّفُّ الاوَّلُ.
\par 11 وَالصَّفُّ الثَّانِي: بَهْرَمَانُ وَيَاقُوتٌ ازْرَقُ وَعَقِيقٌ ابْيَضُ.
\par 12 وَالصَّفُّ الثَّالِثُ: عَيْنُ الْهِرِّ وَيَشْمٌ وَجَمَشْتُ.
\par 13 وَالصَّفُّ الرَّابِعُ: زَبَرْجَدٌ وَجَزْعٌ وَيَشْبٌ مُحَاطَةٌ بِاطْوَاقٍ مِنْ ذَهَبٍ فِي تَرْصِيعِهَا.
\par 14 وَالْحِجَارَةُ كَانَتْ عَلَى اسْمَاءِ بَنِي اسْرَائِيلَ اثْنَيْ عَشَرَ عَلَى اسْمَائِهِمْ كَنَقْشِ الْخَاتِمِ. كُلُّ وَاحِدٍ عَلَى اسْمِهِ لِلاثْنَيْ عَشَرَ سِبْطا.
\par 15 وَصَنَعُوا عَلَى الصُّدْرَةِ سَلاسِلَ مَجْدُولَةً صَنْعَةَ الضَّفْرِ مِنْ ذَهَبٍ نَقِيٍّ.
\par 16 وَصَنَعُوا طَوْقَيْنِ مِنْ ذَهَبٍ وَحَلْقَتَيْنِ مِنْ ذَهَبٍ وَجَعَلُوا الْحَلْقَتَيْنِ عَلَى طَرَفَيِ الصُّدْرَةِ.
\par 17 وَجَعَلُوا ضَفِيرَتَيِ الذَّهَبِ فِي الْحَلْقَتَيْنِ عَلَى طَرَفَيِ الصُّدْرَةِ.
\par 18 وَطَرَفَا الضَّفِيرَتَيْنِ جَعَلُوهُمَا فِي الطَّوْقَيْنِ. وَجَعَلُوهُمَا عَلَى كَتِفَيِ الرِّدَاءِ الَى قُدَّامِهِ.
\par 19 وَصَنَعُوا حَلْقَتَيْنِ مِنْ ذَهَبٍ وَوَضَعُوهُمَا عَلَى طَرَفَيِ الصُّدْرَةِ. عَلَى حَاشِيَتِهَا الَّتِي الَى جِهَةِ الرِّدَاءِ مِنْ دَاخِلٍ.
\par 20 وَصَنَعُوا حَلْقَتَيْنِ مِنْ ذَهَبٍ وَجَعَلُوهُمَا عَلَى كَتِفَيِ الرِّدَاءِ مِنْ اسْفَلُ مِنْ قُدَّامِهِ عِنْدَ وَصْلِهِ فَوْقَ زُنَّارِ الرِّدَاءِ.
\par 21 وَرَبَطُوا الصُّدْرَةَ بِحَلْقَتَيْهَا الَى حَلْقَتَيِ الرِّدَاءِ بِخَيْطٍ مِنْ اسْمَانْجُونِيٍّ لِيَكُونَ عَلَى زُنَّارِ الرِّدَاءِ. وَلا تُنْزَعُ الصُّدْرَةُ عَنِ الرِّدَاءِ - كَمَا امَرَ الرَّبُّ مُوسَى.
\par 22 وَصَنَعَ جُبَّةَ الرِّدَاءِ صَنْعَةَ النَّسَّاجِ كُلَّهَا مِنْ اسْمَانْجُونِيٍّ.
\par 23 وَفَتْحَةُ الْجُبَّةِ فِي وَسَطِهَا كَفَتْحَةِ الدِّرْعِ. وَلِفَتْحَتِهَا حَاشِيَةٌ حَوَالَيْهَا. لا تَنْشَقُّ.
\par 24 وَصَنَعُوا عَلَى اذْيَالِ الْجُبَّةِ رُمَّانَاتٍ مِنْ اسْمَانْجُونِيٍّ وَارْجُوَانٍ وَقِرْمِزٍ مَبْرُومٍ.
\par 25 وَصَنَعُوا جَلاجِلَ مِنْ ذَهَبٍ نَقِيٍّ. وَجَعَلُوا الْجَلاجِلَ فِي وَسَطِ الرُّمَّانَاتِ عَلَى اذْيَالِ الْجُبَّةِ حَوَالَيْهَا فِي وَسَطِ الرُّمَّانَاتِ.
\par 26 جُلْجُلٌ وَرُمَّانَةٌ. جُلْجُلٌ وَرُمَّانَةٌ. عَلَى اذْيَالِ الْجُبَّةِ حَوَالَيْهَا لِلْخِدْمَةِ - كَمَا امَرَ الرَّبُّ مُوسَى.
\par 27 وَصَنَعُوا الاقْمِصَةَ مِنْ بُوصٍ صَنْعَةَ النَّسَّاجِ لِهَارُونَ وَبَنِيهِ.
\par 28 وَالْعِمَامَةَ مِنْ بُوصٍ. وَعَصَائِبَ الْقَلانِسِ مِنْ بُوصٍ. وَسَرَاوِيلَ الْكَتَّانِ مِنْ بُوصٍ مَبْرُومٍ.
\par 29 وَالْمِنْطَقَةَ مِنْ بُوصٍ مَبْرُومٍ وَاسْمَانْجُونِيٍّ وَارْجُوَانٍ وَقِرْمِزٍ صَنْعَةَ الطَّرَّازِ - كَمَا امَرَ الرَّبُّ مُوسَى.
\par 30 وَصَنَعُوا صَفِيحَةَ الْاكْلِيلِ الْمُقَدَّسِ مِنْ ذَهَبٍ نَقِيٍّ وَكَتَبُوا عَلَيْهَا كِتَابَةَ نَقْشِ الْخَاتِمِ. «قُدْسٌ لِلرَّبِّ».
\par 31 وَجَعَلُوا عَلَيْهَا خَيْطَ اسْمَانْجُونِيٍّ لِتُجْعَلَ عَلَى الْعِمَامَةِ مِنْ فَوْقُ - كَمَا امَرَ الرَّبُّ مُوسَى.
\par 32 فَكَمُلَ كُلُّ عَمَلِ مَسْكَنِ خَيْمَةِ الاجْتِمَاعِ. وَصَنَعَ بَنُو اسْرَائِيلَ بِحَسَبِ كُلِّ مَا امَرَ الرَّبُّ مُوسَى. هَكَذَا صَنَعُوا.
\par 33 وَجَاءُوا الَى مُوسَى بِالْمَسْكَنِ: الْخَيْمَةِ وَجَمِيعِ اوَانِيهَا اشِظَّتِهَا وَالْوَاحِهَا وَعَوَارِضِهَا وَاعْمِدَتِهَا وَقَوَاعِدِهَا
\par 34 وَالْغِطَاءِ مِنْ جُلُودِ الْكِبَاشِ الْمُحَمَّرَةِ وَالْغِطَاءِ مِنْ جُلُودِ التُّخَسِ وَحِجَابِ السَّجْفِ
\par 35 وَتَابُوتِ الشَّهَادَةِ وَعَصَوَيْهِ وَالْغِطَاءِ
\par 36 وَالْمَائِدَةِ وَكُلِّ انِيَتِهَا وَخُبْزِ الْوُجُوهِ
\par 37 وَالْمَنَارَةِ الطَّاهِرَةِ وَسُرُجِهَا: السُّرُجِ لِلتَّرْتِيبِ وَكُلِّ انِيَتِهَا وَالزَّيْتِ لِلضُّوءِ
\par 38 وَمَذْبَحِ الذَّهَبِ وَدُهْنِ الْمَسْحَةِ وَالْبَخُورِ الْعَطِرِ وَالسَّجْفِ لِمَدْخَلِ الْخَيْمَةِ
\par 39 وَمَذْبَحِ النُّحَاسِ وَشُبَّاكَةِ النُّحَاسِ الَّتِي لَهُ وَعَصَوَيْهِ وَكُلِّ انِيَتِهِ وَالْمِرْحَضَةِ وَقَاعِدَتِهَا
\par 40 وَاسْتَارِ الدَّارِ وَاعْمِدَتِهَا وَقَوَاعِدِهَا وَالسَّجْفِ لِبَابِ الدَّارِ وَاطْنَابِهَا وَاوْتَادِهَا وَجَمِيعِ اوَانِي خِدْمَةِ الْمَسْكَنِ لِخَيْمَةِ الاجْتِمَاعِ
\par 41 وَالثِّيَابِ الْمَنْسُوجَةِ لِلْخِدْمَةِ فِي الْمَقْدِسِ وَالثِّيَابِ الْمُقَدَّسَةِ لِهَارُونَ الْكَاهِنِ وَثِيَابِ بَنِيهِ لِلْكَهَانَةِ.
\par 42 بِحَسَبِ كُلِّ مَا امَرَ الرَّبُّ مُوسَى هَكَذَا صَنَعَ بَنُو اسْرَائِيلَ كُلَّ الْعَمَلِ.
\par 43 فَنَظَرَ مُوسَى جَمِيعَ الْعَمَلِ وَاذَا هُمْ قَدْ صَنَعُوهُ. كَمَا امَرَ الرَّبُّ هَكَذَا صَنَعُوا. فَبَارَكَهُمْ مُوسَى.

\chapter{40}

\par 1 وَقَالَ الرَّبُّ لِمُوسَى:
\par 2 «فِي الشَّهْرِ الاوَّلِ فِي الْيَوْمِ الاوَّلِ مِنَ الشَّهْرِ تُقِيمُ مَسْكَنَ خَيْمَةِ الاجْتِمَاعِ
\par 3 وَتَضَعُ فِيهِ تَابُوتَ الشَّهَادَةِ. وَتَسْتُرُ التَّابُوتَ بِالْحِجَابِ.
\par 4 وَتُدْخِلُ الْمَائِدَةَ وَتُرَتِّبُ تَرْتِيبَهَا. وَتُدْخِلُ الْمَنَارَةَ وَتُصْعِدُ سُرُجَهَا.
\par 5 وَتَجْعَلُ مَذْبَحَ الذَّهَبِ لِلْبَخُورِ امَامَ تَابُوتِ الشَّهَادَةِ. وَتَضَعُ سَجْفَ الْبَابِ لِلْمَسْكَنِ.
\par 6 وَتَجْعَلُ مَذْبَحَ الْمُحْرَقَةِ قُدَّامَ بَابِ مَسْكَنِ خَيْمَةِ الاجْتِمَاعِ.
\par 7 وَتَجْعَلُ الْمِرْحَضَةَ بَيْنَ خَيْمَةِ الاجْتِمَاعِ وَالْمَذْبَحِ وَتَجْعَلُ فِيهَا مَاءً.
\par 8 وَتَضَعُ الدَّارَ حَوْلَهُنَّ. وَتَجْعَلُ السَّجْفَ لِبَابِ الدَّارِ.
\par 9 وَتَاخُذُ دُهْنَ الْمَسْحَةِ وَتَمْسَحُ الْمَسْكَنَ وَكُلَّ مَا فِيهِ وَتُقَدِّسُهُ وَكُلَّ انِيَتِهِ لِيَكُونَ مُقَدَّسا.
\par 10 وَتَمْسَحُ مَذْبَحَ الْمُحْرَقَةِ وَكُلَّ انِيَتِهِ وَتُقَدِّسُ الْمَذْبَحَ لِيَكُونَ الْمَذْبَحُ قُدْسَ اقْدَاسٍ.
\par 11 وَتَمْسَحُ الْمِرْحَضَةَ وَقَاعِدَتَهَا وَتُقَدِّسُهَا.
\par 12 وَتُقَدِّمُ هَارُونَ وَبَنِيهِ الَى بَابِ خَيْمَةِ الاجْتِمَاعِ وَتَغْسِلُهُمْ بِمَاءٍ.
\par 13 وَتُلْبِسُ هَارُونَ الثِّيَابَ الْمُقَدَّسَةَ وَتَمْسَحُهُ وَتُقَدِّسُهُ لِيَكْهَنَ لِي.
\par 14 وَتُقَدِّمُ بَنِيهِ وَتُلْبِسُهُمْ اقْمِصَةً.
\par 15 وَتَمْسَحُهُمْ كَمَا مَسَحْتَ ابَاهُمْ لِيَكْهَنُوا لِي. وَيَكُونُ ذَلِكَ لِتَصِيرَ لَهُمْ مَسْحَتُهُمْ كَهَنُوتا ابَدِيّا فِي اجْيَالِهِمْ.
\par 16 فَفَعَلَ مُوسَى بِحَسَبِ كُلِّ مَا امَرَهُ الرَّبُّ. هَكَذَا فَعَلَ.
\par 17 وَكَانَ فِي الشَّهْرِ الاوَّلِ مِنَ السَّنَةِ الثَّانِيَةِ فِي اوَّلِ الشَّهْرِ انَّ الْمَسْكَنَ اقِيمَ.
\par 18 اقَامَ مُوسَى الْمَسْكَنَ وَجَعَلَ قَوَاعِدَهُ وَوَضَعَ الْوَاحَهُ وَجَعَلَ عَوَارِضَهُ وَاقَامَ اعْمِدَتَهُ.
\par 19 وَبَسَطَ الْخَيْمَةَ فَوْقَ الْمَسْكَنِ. وَوَضَعَ غِطَاءَ الْخَيْمَةِ عَلَيْهَا مِنْ فَوْقُ - كَمَا امَرَ الرَّبُّ مُوسَى.
\par 20 وَاخَذَ الشَّهَادَةَ وَجَعَلَهَا فِي التَّابُوتِ. وَوَضَعَ الْعَصَوَيْنِ عَلَى التَّابُوتِ مِنْ فَوْقُ.
\par 21 وَادْخَلَ التَّابُوتَ الَى الْمَسْكَنِ. وَوَضَعَ حِجَابَ السَّجْفِ وَسَتَرَ تَابُوتَ الشَّهَادَةِ - كَمَا امَرَ الرَّبُّ مُوسَى.
\par 22 وَجَعَلَ الْمَائِدَةَ فِي خَيْمَةِ الاجْتِمَاعِ فِي جَانِبِ الْمَسْكَنِ نَحْوَ الشِّمَالِ خَارِجَ الْحِجَابِ.
\par 23 وَرَتَّبَ عَلَيْهَا تَرْتِيبَ الْخُبْزِ امَامَ الرَّبِّ - كَمَا امَرَ الرَّبُّ مُوسَى.
\par 24 وَوَضَعَ الْمَنَارَةَ فِي خَيْمَةِ الاجْتِمَاعِ مُقَابِلَ الْمَائِدَةِ فِي جَانِبِ الْمَسْكَنِ نَحْوَ الْجَنُوبِ.
\par 25 وَاصْعَدَ السُّرُجَ امَامَ الرَّبِّ - كَمَا امَرَ الرَّبُّ مُوسَى.
\par 26 وَوَضَعَ مَذْبَحَ الذَّهَبِ فِي خَيْمَةِ الاجْتِمَاعِ قُدَّامَ الْحِجَابِ
\par 27 وَبَخَّرَ عَلَيْهِ بِبَخُورٍ عَطِرٍ - كَمَا امَرَ الرَّبُّ مُوسَى.
\par 28 وَوَضَعَ سَجْفَ الْبَابِ لِلْمَسْكَنِ.
\par 29 وَوَضَعَ مَذْبَحَ الْمُحْرَقَةِ عِنْدَ بَابِ مَسْكَنِ خَيْمَةِ الاجْتِمَاعِ وَاصْعَدَ عَلَيْهِ الْمُحْرَقَةَ وَالتَّقْدِمَةَ - كَمَا امَرَ الرَّبُّ مُوسَى.
\par 30 وَوَضَعَ الْمِرْحَضَةَ بَيْنَ خَيْمَةِ الاجْتِمَاعِ وَالْمَذْبَحِ. وَجَعَلَ فِيهَا مَاءً لِلاغْتِسَالِ.
\par 31 لِيَغْسِلَ مِنْهَا مُوسَى وَهَارُونُ وَبَنُوهُ ايْدِيَهُمْ وَارْجُلَهُمْ.
\par 32 عِنْدَ دُخُولِهِمْ الَى خَيْمَةِ الاجْتِمَاعِ وَعِنْدَ اقْتِرَابِهِمْ الَى الْمَذْبَحِ يَغْسِلُونَ - كَمَا امَرَ الرَّبُّ مُوسَى.
\par 33 وَاقَامَ الدَّارَ حَوْلَ الْمَسْكَنِ وَالْمَذْبَحِ وَوَضَعَ سَجْفَ بَابِ الدَّارِ. وَاكْمَلَ مُوسَى الْعَمَلَ.
\par 34 ثُمَّ غَطَّتِ السَّحَابَةُ خَيْمَةَ الاجْتِمَاعِ وَمَلا بَهَاءُ الرَّبِّ الْمَسْكَنَ.
\par 35 فَلَمْ يَقْدِرْ مُوسَى انْ يَدْخُلَ خَيْمَةَ الاجْتِمَاعِ لانَّ السَّحَابَةَ حَلَّتْ عَلَيْهَا وَبَهَاءُ الرَّبِّ مَلا الْمَسْكَنَ.
\par 36 وَعِنْدَ ارْتِفَاعِ السَّحَابَةِ عَنِ الْمَسْكَنِ كَانَ بَنُو اسْرَائِيلَ يَرْتَحِلُونَ فِي جَمِيعِ رِحْلاتِهِمْ.
\par 37 وَانْ لَمْ تَرْتَفِعِ السَّحَابَةُ لا يَرْتَحِلُونَ الَى يَوْمِ ارْتِفَاعِهَا
\par 38 لانَّ سَحَابَةَ الرَّبِّ كَانَتْ عَلَى الْمَسْكَنِ نَهَارا. وَكَانَتْ فِيهَا نَارٌ لَيْلا امَامَ عُِيُونِ كُلِّ بَيْتِ اسْرَائِيلَ فِي جَمِيعِ رِحْلاتِهِمْ.

\end{document}