\begin{document}

\title{سوزانا}


\chapter{1}

\par 1 منفصل عن بداية سفر دانيال، لأنه ليس في العبرية، كما هو الحال في قصة بال والتنين.[1] كان يسكن رجل في بابل اسمه يهواقيم.
\par 2 وتزوج امرأة اسمها سوسنة، ابنة حلقيا، امرأة جميلة جدًا، ومتقية للرب
\par 3 وكان والداها أيضًا بارين، وعلما ابنتهما وفقًا لشريعة موسى
\par 4 وكان يواقيم رجلاً غنيًا جدًا، وكانت له حديقة جميلة ملاصقة لمنزله، وكان اليهود يلجأون إليه لأنه كان أشرف من جميع الآخرين
\par 5 في نفس السنة، عُيّن اثنان من شيوخ الشعب قاضيين، كما تكلم الرب عنهما، أن الشر جاء من بابل من قضاة قدماء، بدا أنهم يحكمون الشعب
\par 6 كانوا يقيمون كثيرًا في بيت يواقيم، وكان كل من له دعوى يأتي إليهم
\par 7 ولما انصرف الشعب عند الظهر، دخلت سوزانا إلى حديقة زوجها لتتمشى
\par 8 وكان الشيخان يراها تدخل كل يوم وتتمشى، فاشتعلت شهوتهما عليها
\par 9 وأفسدوا عقولهم، وحولوا أعينهم، لئلا ينظروا إلى السماء، ولا يتذكروا الأحكام العادلة
\par 10 وعلى الرغم من أنهما كانا مجروحين بحبها، إلا أن أحدهما لم يجرؤ على إظهار حزنه للآخر
\par 11 لأنهما استحيا من الإفصاح عن شهوتهما، وأنهما رغبا في أن يكون لهما علاقة بها
\par 12 ومع ذلك، فقد كانوا يراقبونها باهتمام من يوم لآخر لرؤيتها.
\par 13 فقال أحدهما للآخر لنذهب الآن إلى بيتنا، لأن وقت العشاء قد حان.
\par 14 فلما خرجا افترقا، ثم عادا فجاءا إلى نفس المكان، وبعد أن سأل كل منهما الآخر عن السبب، اعترفا بشهوتهما، ثم حددا وقتًا معًا ليجداها وحدها
\par 15 وفي وقت ما، بينما كانوا يراقبون وقتًا مناسبًا، دخلت كما في السابق مع خادمتين فقط، وكانت ترغب في الاغتسال في الحديقة: لأنه كان الجو حارًا
\par 16 ولم يكن هناك أحد سوى الشيخين اللذين اختبأا ويراقبانها
\par 17 ثم قالت لجارياتها: ائتيني بزيت وكرات غسيل، وأغلقن أبواب الحديقة لأغتسل
\par 18 ففعلتا كما أمرتهما، وأغلقتا أبواب الحديقة، وخرجتا هما أيضًا من أبواب الخلاء لتأتيا بما أمرتهما به. ولكنهما لم تريا الشيوخ لأنهما كانتا مختبئتين
\par 19 ولما خرجت الجاريتان، قام الشيخان وركضا إليها قائلين:
\par 20 هوذا أبواب الحديقة مغلقة، ولا يرانا أحد، ونحن مغرمون بكِ، لذا ارتضي لنا وضطجع معنا
\par 21 وإن لم تفعل، فسنشهد عليك أنه كان معك شاب، ولذلك أرسلت فتياتك بعيدًا عنك
\par 22 فتنهدت سوزانا وقالت: إني مكتئبة من كل جهة، لأنه إن فعلت هذا فهو موت لي، وإن لم أفعل فلا أستطيع أن أفلت من أيديكما.
\par 23 خير لي أن أقع في يديك ولا أفعل ذلك من أن أخطئ أمام الرب
\par 24 صرخت سوزان بصوت عظيم، فصرخ الشيخان عليها
\par 25 ثم ركض، وفتح باب الحديقة.
\par 26 فلما سمع خدم البيت الصراخ في الحديقة، اندفعوا إلى باب الخلاء ليروا ما حدث لها.
\par 27 فلما تكلم الشيوخ بأمرهم، خجل العبيد خجلاً شديداً، لأنه لم يُسمع قط خبر مثل هذا عن سوسنة
\par 28 وفي الغد، لما اجتمع الشعب إلى زوجها يواقيم، جاء الشيخان أيضًا مملوئين فكرًا خبيثا على سوسنة ليقتلوها؛
\par 29 وقال أمام الشعب: أرسلوا إلى سوسنة ابنة حلقيا امرأة يوياقيم. فأرسلوا
\par 30 فأتت مع أبيها وأمها وأولادها وجميع عشيرتها
\par 31 وكانت سوزانا امرأة رقيقة جدًا، وجميلة المنظر.
\par 32 فأمر هؤلاء الأشرار أن يكشفوا وجهها (لأنها كانت مغطاة) لكي يمتلئوا من جمالها.
\par 33 لذلك بكى أصدقاؤها وكل من رآها.
\par 34 فقام الشيخان في وسط الشعب ووضعا أيديهما على رأسها.
\par 35 ورفعت نظرها نحو السماء وهي تبكي، لأن قلبها كان متوكلاً على الرب
\par 36 فقال الشيوخ: بينما كنا نتمشى في الحديقة وحدنا، دخلت هذه المرأة ومعها جاريتان، وأغلقت أبواب الحديقة، وأرسلت الجاريتين
\par 37 فجاء إليها شاب كان مختبئًا هناك واضطجع معها
\par 38 ثم نحن الذين وقفنا في زاوية من الحديقة، لما رأينا هذا الشر، ركضنا إليهم
\par 39 ولما رأيناهما معًا، لم نستطع أن نمسك بالرجل لأنه كان أقوى منا، ففتح الباب وقفز للخارج
\par 40 ولكن بعد أن أخذنا هذه المرأة، سألناها من هو الشاب، لكنها لم تُخبرنا. هذه الأشياء نشهد بها
\par 41 فصدقتهما الجماعة على أنهما شيوخ الشعب وقضاتهم، فحكموا عليها بالموت
\par 42 فصرخت سوزانا بصوت عظيم وقالت: أيها الإله الأزلي، العليم بالأسرار، والعارف بكل شيء قبل أن يكون:
\par 43 أنت تعلم أنهم شهدوا عليّ زورًا، وها أنا ذا أموت، بينما لم أفعل قط مثل هذه الأشياء التي اختلقها هؤلاء الرجال ضدي بخبث
\par 44 فسمع الرب صوتها.
\par 45 فلما كانت ستُقتل، بعث الرب روحًا قدسية لشاب اسمه دانيال:
\par 46 الذي صرخ بصوت عالٍ: أنا بريء من دم هذه المرأة
\par 47 فالتفت إليه جميع الشعب وقالوا: ما معنى هذا الكلام الذي تكلمت به؟
\par 48 فقام في وسطهم وقال: أأنتم أغبياء يا بني إسرائيل حتى تحكموا على ابنة إسرائيل بغير فحص ومعرفة الحق؟
\par 49 ارجعوا إلى مكان القضاء، لأنهم شهدوا عليها زورًا
\par 50 فرجع كل الشعب مسرعًا، فقال له الشيوخ: تعال اجلس بيننا وأرنا، فقد أعطاك الله كرامة الشيخ
\par 51 فقال لهم دانيال: أبعدوا هذين الاثنين أحدهما عن الآخر، فأفحصهما
\par 52 فلما فُصِلَ أحدهما عن الآخر، دعا واحدًا منهما، وقال له: يا أيها الذي شيخ في الشر، الآن قد انكشفت خطاياك التي ارتكبتها سابقًا
\par 53 لأنكم أصدرتم أحكامًا باطلة، وأدينتم الأبرياء، وأطلقتم سراح المذنبين، مع أن الرب قال: البريء والبار لا تقتلهما
\par 54 والآن، إن كنت قد رأيتها، فأخبرني، تحت أي شجرة رأيتهم يتناقشون؟ من أجاب: تحت شجرة صنوبر
\par 55 فقال دانيال: حسنًا! لقد كذبتَ على رأسك، لأنه الآن قد تلقى ملاك الله حكم الله بقطعك نصفين
\par 56 فوضعه جانبًا، وأمر بإحضار الآخر، وقال له: يا نسل كنعان، لا يهوذا، لقد خدعك الجمال، وأفسد الشهوة قلبك
\par 57 هكذا فعلتم ببنات إسرائيل، فخافن من مرافقتكن، وأما ابنة يهوذا فلم تحتمل شركم
\par 58 والآن أخبرني، تحت أي شجرة أخذتهم يجتمعون؟ فأجاب: تحت شجرة هولم
\par 59 فقال له دانيال: حسنًا! أنت أيضًا كذبت على رأسك، لأن ملاك الله واقف ومعه السيف ليقطعك إلى نصفين، حتى يهلكك
\par 60 مع ذلك صرخت الجماعة كلها بصوت عظيم، وسبحت الله الذي يخلص المتوكلين عليه
\par 61 وقاموا على الشيخين، لأن دانيال كان قد أثبت عليهما شهادة زور بفمهما
\par 62 وفعلوا بهما حسب شريعة موسى ما كانوا ينوون أن يفعلوه بجارهم بخبث، فقتلوهما. وهكذا نجا دم البريء في ذلك اليوم
\par 63 فسبح خليقية وامرأته الله لأجل ابنتهما سوسنة، ويواقيم زوجها، وجميع ذوي قرابتها، لأنه لم يوجد فيها غش
\par 64 من ذلك اليوم فصاعدًا، كان لدانيال صيت عظيم في نظر الشعب

\end{document}