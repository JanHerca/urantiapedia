\begin{document}

\title{استشهاد إشعياء}

\chapter{1}

\par 1 وحدث في السنة السادسة والعشرين من ملك حزقيا ملك يهوذا أنه

\par 2 ودعا منسى ابنه، وكان وحيده، ودعاه إلى حضرة إشعياء بن آموص النبي، وإلى حضرة يوساب بن إشعياء.

\par 3 [...]

\par 4 [...]

\par 5 [...]

\par 6 [...]

\par 7 وبينما كان (حزقيا) يعطي الأوامر، ويوشاب بن إشعياء واقفا، قال إشعياء لحزقيا الملك، ولكن ليس في حضور منسى فقط قال له: "حي هو الرب الذي لم يُرسل اسمه إلى هذا العالم [وحي هو حبيب ربي]، وحي هو الروح الذي يتكلم فيّ، إن كل هذه الأوامر وهذه الكلمات ستُبطل بواسطة منسى ابنك، ومن خلال يديه سأرحل وسط عذاب جسدي.

\par 8 فيخدم صموئيل ملكيرا منسى ويفعل كل ما أراده، ويكون تابعا لبليعار أكثر من تابعي.

\par 9 ويجعل كثيرين في أورشليم وفي اليهودية يرتدُّون عن الإيمان الحقيقي، ويسكن بليعال في منسى، وعلى يديه أُنشر.

\par 10 ولما سمع حزقيا هذا الكلام بكى بكاءً شديداً ومزق ثيابه ووضع ترابا على رأسه وسقط على وجهه.

\par 11 فقال له إشعياء: قد تمت مشورة صموئيل على منسى، ولن ينفعك شيء.

\par 12 وفي ذلك اليوم عزم حزقيا في قلبه أن يقتل منسى ابنه.

\par 13 فقال إشعياء لحزقيا: [إن الحبيب أبطل مشيئتك،] ولن يتم قصد قلبك، لأنه بهذا الدعوة دعيت [وسأرث ميراث الحبيب]

\chapter{2}

\par 1 وحدث بعد موت حزقيا وملك منسى أنه لم يذكر وصايا حزقيا أبيه بل نسيها، وأقام صموئيل في منسى وتمسك به.

\par 2 فترك منسى عبادة إله أبيه، وعبد الشيطان وملائكته وقواته.

\par 3 فأزاغ بيت أبيه الذي كان أمام حزقيا كلام الحكمة وخدمة الله.

\par 4 فمال منسى قلبه ليعبد بليعار، لأن ملاك الإثم، رئيس هذا العالم، هو بليعار، واسمه متانبوخوس. وفرح في أورشليم بسبب منسى، وقوّى نفسه على الارتداد (إسرائيل) وعلى الإثم الذي انتشر في أورشليم.

\par 5 وكثر السحر والشعوذة والعرافة والفسوق والزنى واضطهاد الصديقين من قبل منسى وبلحيرا وطوبيا الكنعاني ويوحنا العناثوثي وصادوق رئيس الأعمال.

\par 6 وأما بقية الأمور فهي مكتوبة في سفر ملوك يهوذا وإسرائيل.

\par 7 ولما رأى إشعياء ملك آموص الإثم الذي كان يصنع في أورشليم وعبادة الشيطان وفجوره، خرج من أورشليم وأقام في بيت لحم يهوذا.

\par 8 وكان هناك أيضا إثم كثير، فانصرف من بيت لحم وأقام على جبل في مكان قفر.

\par 9 [وبعد ذلك انصرف ميخا النبي، وحننيا الشيخ، ويوئيل وحبقوق، وابنه يوشاب، وكثيرون من المؤمنين الذين آمنوا بالصعود إلى السماء، واستقروا على الجبل.]

\par 10 وكانوا جميعهم يلبسون أثوابًا من الشعر، وكانوا جميعهم أنبياء، ولم يكن معهم شيء، بل كانوا عراة، وكانوا ينوحون حزنًا عظيمًا على ضلال إسرائيل.

\par 11 وكانوا لا يأكلون إلا الأعشاب البرية التي جمعوها من الجبال، فطبخوها وعاشوا عليها مع إشعياء النبي، وأقاموا سنتين على الجبال والتلال.

\par 12 وبعد ذلك، بينما كانوا في البرية، كان في السامرة رجل اسمه بلخيرا، من عشيرة صدقيا بن كنعان، نبي كذاب، وكان مقيمًا في بيت لحم. وكان حزقيا بن كنعان، أخو أبيه، وكان في أيام آخاب ملك إسرائيل معلمًا لأنبياء البعل الأربعمائة، قد ضرب ميخا بن عمادا النبي ووبخه.

\par 13 وكان ميخا قد وبخه آخاب وألقاه في السجن. وكان مع صدقيا النبي. وكانا مع أخزيا بن آخاب الملك في السامرة.

\par 14 وكان إيليا النبي من طبون جلعاد يوبخ أخزيا والسامرة، وتنبأ على أخزيا أنه سيموت على فراش مرضه، وأن السامرة ستُسلم إلى يد لبا نصر لأنه قتل أنبياء الله.

\par 15 ولما سمع الأنبياء الكذبة الذين مع أخزيا بن آخاب ومعلمهم جمارياس الذي من جبل يوئيل

\par 16 - وكان أخا صدقيا - فلما سمعوا أقنعوا أخزيا ملك أغوارون وقتلوا ميخا.

\chapter{3}

\par 1 فعرف بلخيرا ورأى مكان إشعياء والأنبياء الذين معه، لأنه كان ساكنًا في ناحية بيت لحم، وكان من أتباع منسى، وتنبأ كذبًا في أورشليم، وكان كثيرون من أهل أورشليم حلفاء له، وكان سامريًا.

\par 2 وحدث لما جاء علاجر زاجر ملك أشور واستولى على السامرة وسبي التسع أسباط ونصف الأسباط وذهب بهم إلى جبال مادي وأنهار تازون.

\par 3 هذا (بلخيرا) كان شاباً فهرب وجاء إلى أورشليم في أيام حزقيا ملك يهوذا، ولكنه لم يسلك في طرق أبيه السامرية لأنه خاف حزقيا.

\par 4 فوجد في أيام حزقيا يتكلم بكلام الإثم في أورشليم.

\par 5 فاتهمه عبيد حزقيا، فهرب إلى تخوم بيت لحم.

\par 6 "فأقنعوا... واتهم بلخيرا إشعياء والأنبياء الذين كانوا معه قائلاً: "إن إشعياء والذين معه يتنبأون على أورشليم وعلى مدن يهوذا بأنهم سيُخربون وعلى بني يهوذا وبنيامين أيضاً بأنهم سيذهبون إلى السبي، وأيضاً عليك يا سيد الملك بأنك ستذهب (مقيداً) بخطافات وسلاسل من حديد".

\par 7 لكنهم يتنبأون كذباً على إسرائيل ويهوذا.

\par 8 وقد قال إشعياء نفسه: «إني أرى أكثر من موسى النبي».

\par 9 ولكن موسى قال: لا يستطيع الإنسان أن يرى الله ويعيش، وإشعياء قال: رأيت الله وها أنا حي.

\par 10 فاعلم أيها الملك أنه يكذب. وسمى أورشليم سدوم، وزعم أن رؤساء يهوذا وأورشليم هم شعب عمورة. وقدم شكاوى كثيرة على إشعياء والأنبياء أمام منسى.

\par 11 وأما بليعال فسكن في قلب منسى وفي قلب رؤساء يهوذا وبنيامين والخصيان ومشيري الملك.

\par 12 فسرت كلمات بلخيرا في عينيه، فأرسل فقبض على إشعياء.

\chapter{4}

\par \textit{لا يوجد محتوى موجود لهذا الفصل}

\par 1 [...]

\chapter{5}

\par 1 فنشره بمنشار.

\par 2 ولما نشر إشعياء قام بلخيرا واتهمه، وقام جميع الأنبياء الكذبة يضحكون ويفرحون بسبب إشعياء.

\par 3 فقام بلخرا ومعه مكمبكوس أمام إشعياء ساخرين.

\par 4 فقال بلخير لإشعياء: قل: لقد كذبت في كل ما تكلمت به، وكذلك طرق منسى صالحة ومستقيمة.

\par 5 "وإن طرق بلتشرا ورفاقه جيدة أيضًا."

\par 6 وهذا ما قاله له حين ابتدأ يقطع.

\par 7 ولكن إشعياء كان منغمساً في رؤيا الرب، ورغم أن عينيه كانتا مفتوحتين، إلا أنه رآهما.

\par 8 "فقال بلخيرا لإشعياء هكذا: قل ما أقول لك فأحول قلوبهم وأجبر منسى ورؤساء يهوذا والشعب وكل أورشليم على احترامك."

\par 9 فأجاب إشعياء وقال: «حتى الآن أستطيع أن أتكلم أقول: ملعون أنت وكل قوتك وكل بيتك.

\par 10 "لأنك لا تستطيع أن تأخذ مني إلا جلد جسدي."

\par 11 فأمسكوا وقطعوا إشعياء بن آموص بمنشار الخشب.

\par 12 وكان منسى وبلكرة والأنبياء الكذبة والرؤساء والشعب كلهم ​​واقفين ينظرون.

\par 13 "وإلى الأنبياء الذين كانوا معه قال قبل أن يقطع: اذهبوا إلى نواحي صور وصيدا، لأنه من أجلي وحدي مزج الله الكأس."

\par 14 "وعندما نشر إشعياء لم يصرخ ولا بكى، بل كانت شفتاه تتكلمان بالروح القدس حتى نشر اثنين."

\end{document}