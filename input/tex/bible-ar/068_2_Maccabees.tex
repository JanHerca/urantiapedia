\begin{document}

\title{المكابيين الثاني}


\chapter{1}

\par 1 "الإخوة اليهود الذين في أورشليم وفي أرض اليهودية يتمنون للإخوة اليهود الذين في كل أنحاء مصر الصحة والسلام.
\par 2 لينعم الله عليكم، واذكروا عهده الذي قطعه مع إبراهيم وإسحاق ويعقوب، عبيده الأمناء؛
\par 3 ويعطيكم جميعًا قلبًا لخدمته، ولفعل مشيئته، بشجاعة صالحة وعقل راغب؛
\par 4 ويفتح قلوبكم في شريعته ووصاياه، ويرسل إليكم السلام،
\par 5 ويسمع صلواتك، ويكون معك، ولا يتخلى عنك أبدًا في وقت الضيق.
\par 6 والآن نحن هنا نصلي من أجلكم.
\par 7 في الوقت الذي ملك فيه ديمتريوس، في السنة المئة والتاسعة والستين، كتبنا إليكم نحن اليهود في شدة الضيق الذي أصابنا في تلك السنوات، منذ أن تمرد ياسون ورفاقه من الأرض المقدسة والمملكة،
\par 8 وأحرقوا الرواق وسفكوا دمًا بريئًا. ثم صلينا إلى الرب فاستجاب لنا. وقدمنا ​​أيضًا ذبائح ودقيقًا فاخرًا، وأوقدنا السرج، وقدمنا ​​الأرغفة
\par 9 والآن انظروا أن تحفظوا عيد المظال في شهر كسلو
\par 10 في السنة المائة والثامنة والثمانين، أرسل الشعب الذي في أورشليم وفي اليهودية، والمجلس، ويهوذا، تحيةً وسلامًا إلى أرسطوبولس، سيد الملك بطليموس، الذي كان من نسل الكهنة الممسوحين، وإلى اليهود الذين كانوا في مصر
\par 11 بما أن الله قد نجانا من مخاطر عظيمة، فإننا نشكره جزيل الشكر، لأننا كنا في معركة ضد ملك
\par 12 لأنه طرد الذين كانوا يقاتلون داخل المدينة المقدسة.
\par 13 لأنه عندما جاء الزعيم إلى بلاد فارس، والجيش الذي معه والذي بدا أنه لا يقهر، قُتلوا في معبد نانيا بخداع كهنة نانيا.
\par 14 لأن أنطيوخس، كأنه يريد الزواج منها، جاء إلى المكان هو وأصدقاؤه الذين كانوا معه، ليأخذوا نقودًا باسم المهر
\par 15 ولما انطلق كهنة نانيا، ودخل مع فرقة صغيرة محيط الهيكل، أغلقوا الهيكل فور دخول أنطيوخس
\par 16 ففتحوا بابًا سريًا من السطح، وألقوا حجارة كالصواعق، فأسقطوا القائد، وقطعوها إربًا، وضربوا رؤوسهم، وألقوا بها على من كانوا في الخارج
\par 17 تبارك إلهنا في كل شيء، الذي أسلم المنافقين.
\par 18 "لذلك، إذ نحن الآن مزمعون على الاحتفال بعيد تطهير الهيكل في اليوم الخامس والعشرين من شهر كسلو، رأينا من الضروري أن نخبركم بذلك، لكي تحتفلوا به أنتم أيضاً، كعيد المظال والنار التي أعطيت لنا عندما قدم نيمياس ذبيحة، بعد أن بنى الهيكل والمذبح.
\par 19 "فعندما أُدخل آباؤنا إلى بلاد فارس، أخذ الكهنة الذين كانوا أتقياء آنذاك نار المذبح سراً، وأخفوها في مكان جوف حفرة بلا ماء، حيث احتفظوا بها في مكان آمن، حتى أصبح المكان غير معروف لجميع الناس.
\par 20 وبعد سنوات عديدة، عندما شاء الله، أرسل نيمياس من عند ملك فارس، فأرسل من نسل أولئك الكهنة الذين أخفوها في النار: ولكن عندما أخبرونا أنهم لم يجدوا نارًا، بل ماءً كثيفًا؛
\par 21 ثم أمرهم برفعه وإحضاره، وعندما وُضعت الذبائح، أمر نيميا الكهنة برش الحطب والأشياء الموضوعة عليه بالماء
\par 22 ولما تم ذلك، وجاء الوقت الذي أشرقت فيه الشمس، التي كانت مختبئة سابقًا في السحابة، اشتعلت نار عظيمة، حتى تعجب كل إنسان
\par 23 وصلى الكهنة بينما كانت الذبيحة تُستهلك، أقول، كلا الكهنة، وجميع الباقين، فبدأ يوناثان، وأجاب الباقون، كما فعل نيميا
\par 24 وكانت الصلاة على هذا النحو: أيها الرب، الرب الإله، خالق كل الأشياء، أنت المخيف والقوي، والبار، والرحيم، والملك الوحيد والكريم،
\par 25 أنت المعطي الوحيد لكل الأشياء، والعادل الوحيد، والقدير، والأبدي، أنت الذي تنقذ إسرائيل من كل ضيق، واخترت الآباء وقدسهم
\par 26 اقبل الذبيحة عن جميع شعبك إسرائيل، واحفظ نصيبك وقدسه
\par 27 اجمع المشتتين عنا، وأنقذ الذين يخدمون بين الأمم، وانظر إلى المحتقرين والممقوتين، وليعلم الأمم أنك أنت إلهنا
\par 28 عاقبوا من يضطهدنا، ويظلمنا بكبرياء.
\par 29 "أعد شعبك إلى مكانك المقدس، كما تكلم موسى."
\par 30 وكان الكهنة يرنّمون مزامير الشكر.
\par 31 وبعد أن تم أكل الذبيحة، أمر نيمياس بصب الماء المتبقي على الحجارة الكبيرة.
\par 32 عندما تم ذلك، اشتعلت شعلة، لكنها استهلكت بالنور الذي أشرق من المذبح
\par 33 فلما عُرف هذا الأمر، أُخبر ملك فارس أنه في المكان الذي أخفى فيه الكهنة النار، ظهر ماء، وأن نيميا قد طهّر به الذبائح
\par 34 ثم أغلق الملك المكان وقدسه بعد أن فحص الأمر
\par 35 وأخذ الملك عطايا كثيرة، وأنعم بها على من أراد إرضائهم
\par 36 وأطلق نيمياس على هذا الشيء اسم نفثار، وهو ما يعني في جوهره تطهيرًا: لكن كثيرين من الناس يسمونه نافي

\chapter{2}

\par 1 كما ورد في السجلات أن إرميا النبي أمر الذين سُلبوا أن يأخذوا من النار، كما هو مشار إليه:
\par 2 وكيف أن النبي، بعد أن أعطاهم الشريعة، أوصاهم ألا ينسوا وصايا الرب، وأن لا يضلوا في أفكارهم، عندما يرون تماثيل الفضة والذهب مع زينتها.
\par 3 وبمثل هذه الخطب حثهم على ألا يفارق الناموس قلوبهم
\par 4 وجاء في نفس الكتابة أيضًا أن النبي، إذ حذّره الله، أمر المسكن والتابوت بالسير معه، عند خروجه إلى الجبل الذي صعد إليه موسى، ورأى ميراث الله
\par 5 ولما وصل إرميا إلى هناك، وجد كهفًا مجوفًا، وضع فيه المسكن والتابوت ومذبح البخور، وأغلق الباب
\par 6 وجاء قوم من الذين تبعوه ليعلموا الطريق، لكنهم لم يجدوه
\par 7 فلما أدرك جيريمي ذلك، لامَهم قائلًا: «أما ذلك المكان، فلن يكون معروفًا إلى أن يجمع الله شعبه مرة أخرى، ويقبلهم للرحمة».
\par 8 حينئذٍ يُريهم الرب هذه الأمور، ويظهر مجد الرب، والسحابة أيضًا، كما أُظهرت في عهد موسى، وكما عندما طلب سليمان أن يُقدَّس المكان تقديسًا مشرفًا
\par 9 وأُعلن أيضًا أنه، وهو حكيم، قدّم ذبيحة التدشين وإكمال الهيكل
\par 10 وكما صلى موسى إلى الرب، نزلت النار من السماء وأكلت الذبائح، هكذا صلى سليمان أيضًا، فنزلت النار من السماء وأكلت المحرقات
\par 11 فقال موسى: «لأن ذبيحة الخطية لم تكن لتؤكل، فقد أُكلت».
\par 12 فاحتفظ سليمان بتلك الأيام الثمانية.
\par 13 وقد وردت نفس الأمور أيضًا في كتابات نيميا وتفاسيره؛ وكيف أنه أسس مكتبة جمعت فيها أعمال الملوك والأنبياء وداود ورسائل الملوك المتعلقة بالقرابين المقدسة
\par 14 كذلك جمع يهوذا أيضًا كل الأشياء التي فقدناها بسبب الحرب التي خضناها، وهي باقية معنا،
\par 15 لذلك، إذا كنتم بحاجة إليها، فأرسلوا من يأخذها إليكم.
\par 16 وبما أننا مزمعون أن نحتفل بعيد التطهير، فقد كتبنا إليكم، وأنكم ستفعلون حسناً إذا حفظتم تلك الأيام.
\par 17 ونرجو أيضًا أن الله الذي خلص كل شعبه، وأعطاهم جميعًا ميراثًا، ومملكة، وكهنوتًا، وقدسًا،
\par 18 كما وعد في الناموس، سيرحمنا قريبًا، ويجمعنا من كل أرض تحت السماء إلى المكان المقدس. لأنه أنقذنا من ضيقات عظيمة، وطهّر المكان
\par 19 وأما ما يتعلق بيهوذا المكابي وإخوته، وتطهير الهيكل العظيم، وتدشين المذبح،
\par 20 والحروب ضد أنطيوخس إبيفانيس، وابنه يوباتور،
\par 21 "والآيات الواضحة التي جاءت من السماء على الذين تصرفوا بشجاعة لإكرام اليهود، حتى أنهم، وهم قليلون، غلبوا البلاد كلها، وطردوا جموعًا بربرية،
\par 22 واستعادوا الهيكل المشهور في جميع أنحاء العالم، وحرروا المدينة، وأيدوا القوانين التي كانت سارية، وكان الرب كريمًا عليهم بكل لطف
\par 23 أقول إن كل هذه الأمور، وقد ذكرها ياسون القيرواني في خمسة كتب، سنحاول تلخيصها في مجلد واحد
\par 24 بالنظر إلى العدد اللانهائي، والصعوبة التي يجدونها في الرغبة في النظر في روايات القصة، وتنوع الموضوع،
\par 25 لقد حرصنا على أن يستمتع من يقرأ، وأن يرتاح من يرغب في حفظه، وأن يستفيد كل من تصل إليه
\par 26 لذلك، بالنسبة لنا، الذين أخذنا على عاتقنا هذا العمل المؤلم المتمثل في الاختصار، لم يكن الأمر سهلاً، بل كان مسألة عرق ومراقبة؛
\par 27 كما أنه ليس من السهل على من يُعِدّ وليمة، ويسعى إلى منفعة الآخرين: ومع ذلك، من أجل إرضاء الكثيرين، سنتحمل بكل سرور هذه الآلام الكبيرة؛
\par 28 تاركًا للمؤلف التعامل الدقيق مع كل تفصيل، والعمل على اتباع قواعد الاختصار
\par 29 فكما يجب على البنّاء الرئيسي لمنزل جديد أن يهتم بالبناء بأكمله، فإن من يتولى تصميمه وطلائه يجب أن يبحث عن أشياء مناسبة لتزيينه: هكذا أعتقد أن الأمر كذلك معنا
\par 30 إن الوقوف على كل نقطة، ومراجعة الأمور بشكل عام، والفضول في التفاصيل، أمرٌ يخص المؤلف الأول للقصة:
\par 31 ولكن استخدام الإيجاز وتجنب بذل الكثير من الجهد في العمل أمرٌ مُباحٌ لمن يُقدم على الاختصار
\par 32 هنا إذن نبدأ القصة: نضيف إلى ما قيل هذا القدر، وهو أنه من الحماقة وضع مقدمة طويلة، وأن نكون مختصرين في القصة نفسها

\chapter{3}

\par 1 وكانت المدينة المقدسة تنعم بسلام، وكانت الشرائع محفوظة جيدًا، بسبب تقوى أونيا رئيس الكهنة وكراهيته للشر،
\par 2 وحدث أن الملوك أنفسهم كرّموا المكان، وعظموا الهيكل بأفضل هداياهم؛
\par 3 حتى أن سلوقس ملك آسيا تحمل جميع النفقات المتعلقة بخدمة الذبائح من دخله الخاص
\par 4 ولكن سمعان، رجل من سبط بنيامين، كان قد عُيّن حاكمًا على الهيكل، اختلف مع رئيس الكهنة بسبب اضطراب في المدينة
\par 5 ولما لم يستطع التغلب على أونيا، أتى به إلى أبولونيوس بن ثراسياس، الذي كان آنذاك واليًا على بقاع سورية وفينيقيا،
\par 6 وأخبروه أن خزانة أورشليم مليئة بمبالغ لا حصر لها من المال، وأن كثرة ثرواتهم التي لا تخص حساب الذبائح لا تُحصى، وأنه من الممكن وضع كل ذلك في يد الملك
\par 7 ولما جاء أبولونيوس إلى الملك، وأراه المال الذي قيل له عنه، اختار الملك هليودورس أمين خزانته، وأرسله مع أمر بإحضار المال المذكور إليه.
\par 8 فقام هيليودوروس على الفور برحلته؛ بحجة زيارة مدينتي سيلوسيا وفينيقيا، ولكن في الواقع لتحقيق غرض الملك
\par 9 ولما وصل إلى أورشليم، واستقبله رئيس كهنة المدينة بلطف، أخبره بما أُعطي من معلومات عن المال، وذكر له سبب مجيئه، وسأل إن كانت هذه الأمور كذلك بالفعل
\par 10 ثم أخبره رئيس الكهنة أن هناك أموالاً مخصصة لإغاثة الأرامل والأيتام:
\par 11 وأن بعضها كان لهيركانوس بن طوبيا، وهو رجل ذو مكانة عظيمة، وليس كما ضلله سمعان الشرير: وكان مجموعها أربعمائة وزنة من الفضة، ومائتي وزنة من الذهب
\par 12 وأنه من المستحيل تمامًا أن تُرتكب مثل هذه الأخطاء بحق أولئك الذين أساءوا إلى قداسة المكان، وإلى عظمة المعبد وحرمته، المكرم في جميع أنحاء العالم
\par 13 لكن هليودورس، بسبب أمر الملك الصادر له، قال إنه يجب على أي حال إحضارها إلى خزانة الملك
\par 14 ففي اليوم الذي حدده، دخل ليأمر بهذا الأمر، ولذلك سادت حالة من الكرب الشديد في جميع أنحاء المدينة
\par 15 أما الكهنة، وهم ساجدون أمام المذبح بثيابهم الكهنوتية، فقد نادوا إلى السماء على من وضع قانونًا بشأن الأشياء المعطاة له ليحفظها، حتى تُحفظ بأمان لأولئك الذين استودعوها
\par 16 ثم من نظر إلى وجه رئيس الكهنة، لكان ذلك قد جرح قلبه: لأن وجهه وتغير لونه كانا يعلنان عن عذاب عقله الداخلي
\par 17 لأن الرجل كان محاطًا بالخوف والرعب من الجسد، لدرجة أنه كان واضحًا لمن نظروا إليه مدى الحزن الذي كان يشعر به الآن في قلبه
\par 18 ركض آخرون متوافدين خارج منازلهم إلى الدعاء العام، لأن المكان كان على وشك أن يُحتقر
\par 19 وكانت النساء متشبثات بمسوح تحت صدورهن يكثرن في الشوارع، والعذارى المحبوسات يركضن، بعضهن إلى الأبواب، وبعضهن إلى الأسوار، وأخريات ينظرن من النوافذ
\par 20 ورفع الجميع أيديهم نحو السماء، وتضرعوا.
\par 21 ثم كان من المؤسف أن يرى الإنسان سقوط الجموع من كل نوع، وخوف رئيس الكهنة في مثل هذا العذاب.
\par 22 ثم دعوا الرب القدير أن يحفظ الأمانة الموكولة إليهم سالمة ومضمونة
\par 23 ومع ذلك، نفذ هيليودورس ما قُدِّر له.
\par 24 وبينما كان هناك مع حراسه حول الخزانة، تسبب رب الأرواح، ورئيس كل قوة، في ظهور عظيم، حتى أن كل من افترض أن يدخل معه اندهش من قوة الله، وأغمي عليه، وكان خائفًا جدًا.
\par 25 لأنه ظهر لهم حصان يمتطيه فارس رهيب، ويرتدي غطاءً جميلاً للغاية، وركض بشراسة، وضرب هليودورس بقدميه الأماميتين، وبدا أن الراكب على الحصان كان يرتدي طقمًا كاملاً من الذهب
\par 26 ثم ظهر أمامه شابان آخران، عظيما القوة، فائقا الجمال، جميلا الملبس، ووقفا بجانبه من هنا ومن هناك، وجلداه باستمرار، وأعطوه ضربات كثيرة مؤلمة
\par 27 فسقط هليودورس فجأة على الأرض، وأحاطت به ظلمة عظيمة، فحمله الذين كانوا معه ووضعوه على محفة
\par 28 وهكذا، فإن الذي جاء مؤخرًا مع موكب كبير ومع كل حراسه إلى الخزانة المذكورة، اقتادوه، لأنه لم يكن قادرًا على مساعدة نفسه بأسلحته: ومن الواضح أنهم اعترفوا بقوة الله
\par 29 لأنه على يد الله طُرح أرضًا، وأصبح بلا كلام، بلا أي أمل في الحياة
\par 30 لكنهم سبّحوا الرب الذي كرّم مكانه بأعجوبة: لأن الهيكل، الذي كان مليئًا بالخوف والقلق قبل قليل، امتلأ فرحًا وسرورًا عندما ظهر الرب القدير
\par 31 ثم على الفور صلى بعض أصدقاء هليودورس إلى أونياس، أن يدعو الله العلي أن يمنحه حياته، وهو الذي كان على وشك الموت
\par 32 لذلك، شك رئيس الكهنة في أن الملك قد يظن خطأً أن اليهود قد خدعوا هليودورس، فقدّم ذبيحة من أجل صحة الرجل
\par 33 وبينما كان رئيس الكهنة يُقدم الكفارة، ظهر أولئك الشباب بنفس الثوب، ووقفوا بجانب هليودورس، قائلين: «أعطِ أونيا رئيس الكهنة شكرًا عظيمًا، لأن الرب من أجله وهبك الحياة.»
\par 34 وإذ رأيت أنك جُلدت من السماء، فأخبر جميع الناس بقدرة الله العظيمة. وبعد أن نطقوا بهذه الكلمات، لم يظهروا بعد
\par 35 فبعد أن قدم هليودورس ذبيحة للرب، ونذر نذورًا عظيمة لمن أنقذ حياته، وسلم على أونيا، عاد مع جيشه إلى الملك
\par 36 حينئذٍ شهد لجميع الناس بأعمال الله العظيم التي رآها بعينيه
\par 37 وعندما جاء الملك هيليودورس، الذي قد يكون رجلاً مناسبًا لإرساله مرة أخرى إلى أورشليم، قال:
\par 38 إذا كان لديك أي عدو أو خائن، فأرسله إلى هناك، وستستقبله مجلودًا جيدًا، إذا نجا بحياته: لأنه لا شك في وجود قوة خاصة من الله في ذلك المكان
\par 39 لأن الذي يسكن في السماء ينظر إلى ذلك المكان، ويدافع عنه، ويضرب ويهلك الذين يأتون لإيذائه
\par 40 وكانت الأمور المتعلقة بهليودورس وحفظ الخزانة تحدث على هذا النحو.

\chapter{4}

\par 1 هذا سمعان، الذي تحدثنا عنه سابقًا، كان خائنًا للمال ولبلاده، فشهر على أونيا، كما لو كان قد أرعب هليودورس، وكان سببًا في هذه الشرور
\par 2 وهكذا تجرأ على وصفه بالخائن، وهو الذي استحق الخير للمدينة، وخدم أمته، وكان متحمسًا جدًا للقوانين
\par 3 ولكن عندما وصل كرههم إلى حد ارتكاب أحد أفراد فصيل سمعان جرائم قتل،
\par 4 رأى أونيا خطر هذا الخلاف، وأن أبولونيوس، بصفته واليًا على بقاع سورية وفينيقيا، قد غضب، وزاد من حقد سمعان،
\par 5 ذهب إلى الملك، ليس ليُتهم أبناء وطنه، بل ليسعى إلى خير الجميع، العام والخاص:
\par 6 لأنه رأى أنه من المستحيل أن تبقى الدولة هادئة، وأن يترك سمعان حماقته، ما لم ينظر الملك إلى ذلك
\par 7 ولكن بعد وفاة سلوقس، عندما استولى أنطيوخس، المدعو أبيفانس، على المملكة، عمل ياسون أخو أونيا خفيةً ليكون رئيس كهنة،
\par 8 ووعد الملك بالشفاعة بثلاثمائة وستين وزنة من الفضة، وثمانين وزنة من دخل آخر:
\par 9 إلى جانب ذلك، وعد بتخصيص مائة وخمسين آخرين، إذا سمح له بتجهيز مكان للتدريب، ولتدريب الشباب على أزياء الوثنيين، وكتابة اسم الأنطاكيين لأهل أورشليم
\par 10 وعندما منحه الملك الحكم، قام على الفور بجعل أمته تتبع الأسلوب اليوناني
\par 11 وأزال الامتيازات الملكية الممنوحة لليهود على يد يوحنا والد يوبوليموس، الذي ذهب سفيرًا إلى روما للصداقة والمساعدة؛ وألغى الحكومات التي كانت وفقًا للقانون، وأنشأ عادات جديدة تخالف القانون
\par 12 لأنه بنى بسرور مكانًا للتمرين تحت البرج نفسه، وأخضع كبار الشباب لسيطرته، وجعلهم يرتدون قبعة
\par 13 هكذا كانت ذروة الأزياء اليونانية، وتزايد العادات الوثنية، بسبب تدنيس ياسون المفرط، ذلك الشرير، الذي لم يكن رئيس كهنة؛
\par 14 أن الكهنة لم تكن لديهم الشجاعة للخدمة بعد الآن عند المذبح، بل احتقروا الهيكل، وأهملوا الذبائح، وسارعوا إلى المشاركة في البدل غير القانوني في مكان التمرين، بعد أن دعتهم لعبة القرص إلى ذلك؛
\par 15 لا يكتفون بشرف آبائهم، بل يفضلون مجد اليونانيين أكثر من أي شيء آخر
\par 16 بسبب ذلك، حلت بهم مصيبة شديدة: إذ كان عليهم أن يكونوا أعداءهم ومنتقمين لهم، الذين اتبعوا عاداتهم بجدية شديدة، والذين رغبوا في أن يكونوا مثلهم في كل شيء
\par 17 لأنه ليس من الهين فعل الشر ضد قوانين الله، ولكن الوقت التالي سيُظهر هذه الأمور
\par 18 "وعندما كانت اللعبة التي كانت تُلعب كل سنة في صور، وكان الملك حاضراً،
\par 19 أرسل ياسون الجاحد رسلًا خاصين من أورشليم، وكانوا من الأنطاكيين، لحمل ثلاثمائة درهم من الفضة لذبيحة هرقل، والتي رأى حتى حاملوها أنه من المناسب عدم إهدائها للذبيحة، لأنها لم تكن مناسبة، بل الاحتفاظ بها لمصاريف أخرى
\par 20 إذن، هذه الأموال، فيما يتعلق بالمرسل، خُصصت لتضحية هرقل؛ ولكن بسبب حامليها، استُخدمت في صنع السفن الشراعية
\par 21 ولما أُرسل أبولونيوس بن مينيستايوس إلى مصر لتتويج الملك بطليموس فيلوماتور، أدرك أنطيوخس أنه ليس على ما يرام في شؤونه، فاستعد لسلامته، فجاء إلى يافا، ومن هناك إلى أورشليم
\par 22 حيث استقبله ياسون والمدينة استقبالًا مشرفًا، وأُحضر بالمشاعل وسط هتافات عظيمة. وبعد ذلك ذهب مع جيشه إلى فينيقيا
\par 23 بعد ثلاث سنوات، أرسل ياسون مينيلانس، شقيق سمعان المذكور، ليحمل المال إلى الملك، وليذكره ببعض الأمور الضرورية
\par 24 فلما أُحضر إلى حضرة الملك، عظمه لمجد مظهر سلطانه، فأخذ الكهنوت لنفسه، مقدمًا أكثر من ياسون بثلاثمائة وزنة من الفضة
\par 25 لذلك جاء بتفويض الملك، ولم يحمل معه شيئًا يليق بالكهنوت الأعظم، بل كان يحمل غضب طاغية قاسٍ، وغضب وحش متوحش
\par 26 ثم إن ياسون، الذي كان قد ظلم أخاه، إذ ظلمه آخر، اضطر إلى الفرار إلى بلاد العمونيين
\par 27 وهكذا حصل مينيلانس على الإمارة: أما بالنسبة للمال الذي وعد به الملك، فلم يقبله على نحو جيد، مع أن سوستراتيس حاكم القلعة طلبه
\par 28 لأنه كان من نصيبه جمع الجمارك. ولذلك تم استدعاؤهما أمام الملك
\par 29 وترك مينيلان أخاه ليسيماخوس مكانه في الكهنوت، وترك سوستراتوس كراتيس، الذي كان حاكمًا للقبارصة
\par 30 وبينما كانت تلك الأمور جارية، ثار أهل طرسوس ومالوس، لأنهم أعطوا لسرية الملك، التي تدعى أنطيوخس
\par 31 ثم جاء الملك مسرعًا لتهدئة الأمور، تاركًا أندرونيكوس، رجل السلطة، نائبًا له
\par 32 ثم ظن مينيلاوس أنه قد حصل على فرصة مناسبة، فسرق بعض الأواني الذهبية من الهيكل، وأعطى بعضها لأندرونيكوس، وباع بعضها الآخر في صور والمدن المجاورة
\par 33 فلما علم أونيا بيقين وبخه، انصرف إلى مزار في دافني الواقعة بالقرب من أنطاكية
\par 34 "ولذلك، أخذ مينيلاوس أندرونيكوس على انفراد، وصلى إليه أن يقبض على أونياس؛ فأقنعه أندرونيكوس، وجاء إلى أونياس بالخداع، وأعطاه يده اليمنى مع القسم؛ ورغم أن أندرونيكوس كان يشك فيه، إلا أنه أقنعه بالخروج من الحرم: فسجنه على الفور دون مراعاة للعدالة.
\par 35 ولهذا السبب، لم يكن اليهود وحدهم، بل كثيرون أيضًا من الأمم الأخرى، غاضبين بشدة، وحزنوا كثيرًا على القتل الظالم للرجل
\par 36 ولما عاد الملك من المناطق المحيطة بكيليكية، تذمر اليهود الذين في المدينة، وبعض اليونانيين الذين استهزأوا بذلك، لأن أونيا قُتل بلا سبب
\par 37 لذلك حزن أنطيوخس بشدة، وتأثر بالشفقة، وبكى بسبب سلوك الميت الرصين والمتواضع
\par 38 فاشتعل غضبًا، فأخذ على الفور أرجوان أندرونيكوس، ومزق ثيابه، وقاده في جميع أنحاء المدينة إلى ذلك المكان نفسه الذي ارتكب فيه الكفر ضد أونيا، وهناك قتل القاتل الملعون. وهكذا كافأه الرب عقابه كما استحق
\par 39 ولما ارتكب ليسيماخوس العديد من أعمال التدنيس في المدينة بموافقة مينيلاوس، وانتشرت ثمار ذلك في الخارج، اجتمعت الجموع ضد ليسيماخوس، بعد أن سُلبت العديد من الأواني الذهبية
\par 40 عندها ثار عامة الناس، وامتلأوا غضبًا، فسلح ليسيماخوس نحو ثلاثة آلاف رجل، وبدأ أولًا في استخدام العنف؛ وكان زعيمهم أورانوس، وهو رجل متقدم في السن، ولا يقل حماقة
\par 41 ثم رأوا محاولة ليسيماخوس، فأمسك بعضهم بالحجارة، وبعضهم بالهراوات، والبعض الآخر أخذ حفنة من التراب الذي كان قريبًا، وألقوا بها جميعًا على ليسيماخوس، وعلى من هاجموها
\par 42 فجرحوا الكثير منهم، وطرحوا بعضهم أرضًا، وأجبروهم جميعًا على الفرار: أما سارق الكنيسة نفسه، فقد قتلوه بجانب الخزانة
\par 43 لذلك، وُجِّهت تهمة إلى مينيلانز بشأن هذه الأمور
\par 44 ولما جاء الملك إلى صور، أرسل ثلاثة رجال من مجلس الشيوخ ليرافعوا أمامه في القضية:
\par 45 لكن مينيلانس، بعد إدانته، وعد بطليموس ابن دوريمانس بإعطائه الكثير من المال، إذا نجح في تهدئة الملك تجاهه
\par 46 عندها، أخذ بطليموس الملك جانبًا إلى رواق معين، كما لو كان يستنشق الهواء، مما جعله يغير رأيه:
\par 47 لدرجة أنه برأ مينيلان من التهم، الذي كان على الرغم من ذلك سببًا لكل الأذى: وأولئك الرجال المساكين، الذين لو رووا قضيتهم، نعم، أمام السكيثيين، لكانوا سيُحكم عليهم بالبراءة، فقد حكم عليهم بالإعدام
\par 48 وهكذا، فإن الذين تابعوا الأمر من أجل المدينة، ومن أجل الشعب، ومن أجل الأواني المقدسة، سرعان ما عانوا من عقاب ظالم
\par 49 ولذلك فإن أهل صور، الذين تحركوا بكراهية بسبب هذا العمل الشرير، أمروا بدفنهم بشكل مشرف.
\par 50 وهكذا، بسبب جشع أصحاب السلطة، ظل مينيلان في السلطة، وازداد حقده، وكان خائنًا عظيمًا للمواطنين

\chapter{5}

\par 1 في نفس الوقت تقريبًا، أعد أنطيوخس رحلته الثانية إلى مصر:
\par 2 ثم حدث أنه في جميع أنحاء المدينة، لمدة أربعين يومًا تقريبًا، شوهد فرسان يركضون في الهواء، يرتدون ملابس من ذهب، ومسلحين بالرماح، مثل فرقة من الجنود،
\par 3 وجيوش من الفرسان في صفوف، يلتقون ويركضون ضد بعضهم البعض، مع هز الدروع، وكثرة الرماح، وسحب السيوف، ورمي السهام، وتألق الحلي الذهبية، وأطقم الخيول من جميع الأنواع
\par 4 لذلك صلى كل إنسان أن يتحول ذلك الظهور إلى خير.
\par 5 "ولما انتشرت إشاعة كاذبة بأن أنطيوخس قد مات، أخذ ياسون ألف رجل على الأقل، وهاجم المدينة فجأة؛ وعندما أُعيد الذين كانوا على الأسوار، واستولى على المدينة أخيرًا، هرب مينيلاوس إلى القلعة.
\par 6 لكن ياسون قتل مواطنيه بلا رحمة، لم يظن أن معاقبتهم من قومه سيكون يومًا تعيسًا للغاية بالنسبة له؛ بل ظن أنهم كانوا أعداءه، وليسوا من مواطنيه، الذين قهرهم
\par 7 ولكن على الرغم من كل هذا، لم يحصل على الإمارة، بل نال في النهاية العار جزاءً لخيانته، وهرب مرة أخرى إلى بلاد العمونيين
\par 8 في النهاية، كانت عودته غير سعيدة، إذ اتُهم أمام الحارث ملك العرب، فهرب من مدينة إلى أخرى، وطارده جميع الناس، ومكروه كمن خالف القوانين، ومُنكرًا كعدو صريح لبلاده ومواطنيه، وطُرد إلى مصر
\par 9 وهكذا هلك من طرد كثيرين من بلادهم في أرض غريبة، متقاعدًا إلى اللاكديمونيين، ظانًا أنه سيجد هناك العون بفضل عشيرته:
\par 10 والذي طرد كثيرين دون دفن، لم يكن له من ينوح عليه، ولا جنازات مهيبة على الإطلاق، ولا قبر مع آبائه
\par 11 ولما وصل هذا إلى عربة الملك، ظن أن اليهودية قد ثارت. فخرج من مصر بغضب شديد، واستولى على المدينة بقوة السلاح،
\par 12 وأمر رجال حربه ألا يرحموا من يصادفونه، وأن يقتلوا من يصعد على البيوت
\par 13 وهكذا كان هناك قتل للصغار والكبار، وإزهاق للأرواح، وذبح للرجال والنساء والأطفال، وقتل للعذارى والرضع
\par 14 وهُلِكَ في غضون ثلاثة أيام كاملة ثمانون ألفًا، منهم أربعون ألفًا قُتلوا في الصراع؛ ولم يقل عدد الذين بِيعوا عن عدد القتلى
\par 15 ومع ذلك، لم يكتفِ بهذا، بل افترض أنه ذهب إلى أقدس معبد في العالم أجمع؛ وكان مينيلان، الخائن للقوانين ولوطنه، دليله:
\par 16 وأخذ الأواني المقدسة بأيدي نجسة، وبأيدي دنسة هدم الأشياء التي كرسها ملوك آخرون لزيادة مجد المكان وشرفه، وأعطاها
\par 17 وكان أنطيوخس متكبرًا لدرجة أنه لم يخطر بباله أن الرب كان غاضبًا لفترة من الوقت بسبب خطايا سكان المدينة، ولذلك لم تكن عينه على المكان
\par 18 لأنه لو لم يكونوا متورطين سابقًا في خطايا كثيرة، لكان هذا الرجل، بمجرد مجيئه، قد جُلِد على الفور، وأُبعد عن غروره، كما حدث مع هليودورس، الذي أرسله الملك سلوقس لمعاينة الخزانة
\par 19 ومع ذلك، لم يختر الله الشعب من أجل المكان، بل المكان من أجل الشعب
\par 20 ولذلك فإن المكان نفسه، الذي شاركهم في الشدائد التي حلت بالأمة، شارك فيما بعد في النعم المرسلة من الرب: وكما هُجر في غضب القدير، كذلك أيضًا، بعد أن تصالح الرب العظيم، أُقيم بكل مجد
\par 21 فلما أخرج أنطيوخس من الهيكل ألفًا وثمانمائة وزنة، انطلق مسرعًا إلى أنطاكية، عازمًا في كبريائه على أن يجعل الأرض صالحة للملاحة، والبحر قابلًا لعبور المشاة: هكذا كان غرور عقله
\par 22 وترك حكامًا ليُغيظوا الأمة: في أورشليم، فيليب، لأن بلده فريجي، ولأخلاقه أكثر وحشية من الذي أقامه هناك؛
\par 23 وفي غاريزيم، أندرونيكوس؛ وإلى جانبه، مينيلاوس، الذي كان أسوأ من البقية جميعًا، وكان يفرض قسوة على المواطنين، وكان لديه فكر خبيث ضد مواطنيه اليهود
\par 24 أرسل أيضًا ذلك الزعيم البغيض أبولونيوس مع جيش قوامه عشرين ألفًا، وأمره بقتل كل من كان في أفضل سنه، وبيع النساء والشباب
\par 25 الذي جاء إلى أورشليم، متظاهرًا بالسلام، وامتنع حتى يوم السبت المقدس، وعندما أخذ اليهود الذين يحتفلون بهذا اليوم المقدس، أمر رجاله بتسليح أنفسهم
\par 26 فقتل جميع الذين ذهبوا للاحتفال بالسبت، وجاب المدينة بالسلاح وقتل جموعًا كثيرة
\par 27 أما يهوذا المكابي، ومعه تسعة آخرون، أو نحو ذلك، فانسحب إلى البرية، وسكن في الجبال كسائر الوحوش، مع رفاقه، وكانوا يتغذون على الأعشاب باستمرار، لئلا يكونوا شركاء في النجاسة

\chapter{6}

\par 1 بعد ذلك بوقت قصير، أرسل الملك رجلاً شيخًا من أثينا لإجبار اليهود على التخلي عن قوانين آبائهم، وعدم العيش وفقًا لقوانين الله:
\par 2 ودنسوا أيضًا الهيكل الذي في أورشليم، وأطلقوا عليه اسم هيكل جوبيتر الأولمبي، والذي في غاريزيم، هيكل جوبيتر المدافع عن الغرباء، كما رغبوا في أن يسكن في المكان
\par 3 كان نزول هذا الشر مؤلمًا ومؤلمًا للشعب:
\par 4 لأن الهيكل كان مملوءاً بالشغب والاحتفالات من قبل الأمم، الذين كانوا يعبثون مع الزواني، وكان لهم علاقة بالنساء في دائرة الأماكن المقدسة، بالإضافة إلى أنهم أدخلوا أشياء غير شرعية.
\par 5 وامتلأ المذبح أيضًا بأشياء دنسة، مما ينهى عنه الناموس
\par 6 ولم يكن جائزًا للإنسان أن يحفظ أيام السبت أو الصيام القديم، أو أن يدعي أنه يهودي على الإطلاق
\par 7 وفي يوم ميلاد الملك من كل شهر، كانوا يُؤخذون بإكراه مرير ليأكلوا من الذبائح؛ وعندما كان يُحفظ صيام باخوس، كان اليهود يُجبرون على السير في موكب إلى باخوس، حاملين اللبلاب
\par 8 علاوة على ذلك، صدر مرسوم إلى المدن الوثنية المجاورة، بناءً على اقتراح بطليموس، ضد اليهود، لكي يلتزموا بنفس العادات، ويشتركوا في ذبائحهم:
\par 9 ومن لم يلتزم بأخلاق الأمم، فإنه يُقتل. حينها قد يرى الإنسان البؤس الحاضر
\par 10 لأنه أُحضِرت امرأتان قد ختنتا أطفالهما، فطافتا بالمدينة علانية، والأطفال يسلمون ثدييهما، ثم ألقتاهم من على السور
\par 11 وآخرون، الذين ركضوا معًا إلى الكهوف القريبة، لحفظ يوم السبت سرًا، اكتشفهم فيليب، فأحرقوا جميعًا معًا، لأنهم نذروا أنفسهم لشرف هذا اليوم الأقدس
\par 12 الآن أناشد أولئك الذين يقرؤون هذا الكتاب، ألا ييأسوا بسبب هذه المصائب، بل أن يحكموا على هذه العقوبات بأنها ليست للدمار، بل لتأديب أمتنا
\par 13 لأنه دليل على عظيم صلاحه، عندما لا يُعانى الأشرار لفترة طويلة، بل يُعاقبون على الفور
\par 14 لأنه ليس كما هو الحال مع الأمم الأخرى، التي يتغاضى الرب بصبر عن معاقبتها حتى يبلغوا امتلاء خطاياهم، هكذا يتعامل معنا،
\par 15 لئلا يصل إلى ذروة الخطيئة، فينتقم منا فيما بعد
\par 16 ولذلك فهو لا يسحب رحمته عنا أبدًا: ورغم أنه يعاقب بالشدة، إلا أنه لا يتخلى عن شعبه أبدًا
\par 17 ولكن ليكن ما قلناه تحذيرًا لنا. والآن سننتقل إلى بيان الأمر ببضع كلمات
\par 18 أُجبر أليعازار، أحد كبار الكتبة، رجل شيخ حسن المنظر، على فتح فمه وأكل لحم الخنزير
\par 19 لكنه اختار أن يموت مجيدًا على أن يعيش ملطخًا بمثل هذه الرجسة، فبصقها، وجاء من تلقاء نفسه إلى العذاب،
\par 20 كما كان ينبغي أن يأتوا، أولئك الذين عزموا على الوقوف ضد مثل هذه الأمور التي لا يجوز تذوقها من أجل محبة الحياة
\par 21 "ولكن الذين كانوا مسئولين عن ذلك الوليمة الشريرة، بسبب معارفهم القديمة مع الرجل، أخذوه جانباً، وتوسلوا إليه أن يحضر لحماً من قوت يومه، مما يجوز له استخدامه، ويجعله كما لو أنه يأكل من اللحم المأخوذ من الذبيحة التي أمر بها الملك؛
\par 22 لكي ينجو من الموت بفعله هذا، ويحظى بصداقتهم القديمة
\par 23 لكنه بدأ يفكر بحذر، وبما يليق بعمره، وتميز سنواته المتقدمة، وشرف شيبته، وما أتى عليه، وتعليمه الأمين منذ صغره، أو بالأحرى الشريعة المقدسة التي وضعها الله وأعطاها: لذلك أجاب وفقًا لذلك، وأمرهم بإرساله إلى القبر على الفور
\par 24 قال إنه لا يليق بعصرنا أن نتظاهر بأي شكل من الأشكال، حيث قد يعتقد الكثير من الشباب أن إليعازار، وهو في الثامنة والثمانين من عمره، قد ذهب الآن إلى دين غريب؛
\par 25 وهكذا، من خلال نفاقي، ورغبتي في العيش لفترة أطول قليلاً، سينخدعون بي، وسألطخ شيخوختي، وأجعلها بغيضة
\par 26 فلئن كنتُ سأنجو في الوقت الحاضر من عقاب البشر، فلن أفلت من يد القدير، لا حيًا ولا ميتًا
\par 27 لذلك الآن، وأنا أُغير هذه الحياة بشجاعة، سأُظهر لنفسي شخصًا كهذا الذي يتطلبه عمري،
\par 28 واتركوا مثالاً يُحتذى به للشباب ليموتوا طوعاً وبشجاعة من أجل القوانين الشريفة والمقدسة. وبعد أن قال هذه الكلمات، ذهب على الفور إلى العذاب:
\par 29 أولئك الذين قادوه إلى تغيير حسن النية، حملوه قبل ذلك بقليل إلى الكراهية، لأن الخطب المذكورة كانت، كما اعتقدوا، نابعة من عقل يائس
\par 30 ولكن عندما كان على وشك الموت بالجلد، تأوه وقال: إنه واضح للرب، الذي لديه المعرفة المقدسة، أنه بينما كان من الممكن أن أنقذ من الموت، فإنني الآن أتحمل آلامًا شديدة في جسدي بسبب الضرب: ولكن في نفسي، أنا راضٍ تمامًا عن تحمل هذه الأشياء، لأني أخاف منه
\par 31 وهكذا مات هذا الرجل، تاركًا موته مثالًا للشجاعة النبيلة، وذكرى للفضيلة، ليس فقط للشباب، بل لجميع أمته

\chapter{7}

\par 1 وحدث أيضًا أن سبعة إخوة مع أمهم أُخذوا، وأجبرهم الملك، مخالفًا للقانون، على تذوق لحم الخنزير، وعذبوهم بالسياط والجلد
\par 2 لكن واحدًا ممن تكلموا أولًا قال هكذا: ماذا تريد أن تسأل أو تتعلم منا؟ نحن مستعدون للموت، بدلًا من أن نخالف قوانين آبائنا
\par 3 ثم غضب الملك، وأمر بتسخين القدور والمقالي:
\par 4 فحمى في الحال، وأمر بقطع لسان المتكلم الأول، وقطع أطراف جسده، على مرأى من إخوته وأمه
\par 5 "ولما أصيب بهذه الآلام في جميع أعضائه، أمر بإحضاره وهو لا يزال حياً إلى النار وقليه في المقلاة، وعندما تشتت بخار المقلاة لفترة طويلة، حث بعضهم بعضاً مع أمه على الموت بشجاعة، قائلين:
\par 6 ينظر الرب الإله إلينا، وفي الحقيقة يتعزى فينا، كما أعلن موسى في أنشودته التي شهدت على وجوههم، قائلاً: ويتعزى في عبيده
\par 7 فلما مات الأول بعد هذا العدد، أتوا بالثاني ليصنعوا منه خشبة سخرية، ولما نزعوا جلد رأسه مع الشعر، سألوه: أتأكل قبل أن تُعاقب في كل عضو من جسدك؟
\par 8 فأجاب بلغته وقال: لا. ولذلك تلقى أيضًا العذاب التالي بالترتيب، كما تلقى العذاب الأول
\par 9 وعندما كان في آخر نفس، قال: أنت كالغضب تأخذنا من هذه الحياة الحاضرة، لكن ملك العالم سيقيمنا نحن الذين متنا من أجل شريعته إلى الحياة الأبدية
\par 10 بعده كان الثالث مصنوعًا من خشب ساخر: وعندما طُلب منه، أخرج لسانه، وسرعان ما مد يديه بشجاعة
\par 11 وقال بشجاعة: هذه كانت لي من السماء، ومن أجل شرائعه أحتقرها، ومنه أرجو أن أستعيدها مرة أخرى
\par 12 حتى أن الملك والذين معه تعجبوا من شجاعة الشاب، لأنه لم يكن يبالي بالآلام
\par 13 ولما مات هذا أيضًا، عذبوا الرابع وعذبوه على نحو مماثل
\par 14 فلما كاد أن يموت، قال هكذا: إنه حسنٌ أن يُقتل المرء من قِبَل الناس، وينتظر رجاءً من الله ليُقيمه. أما أنت، فلن تكون لك قيامة للحياة
\par 15 بعد ذلك أحضروا الخامس أيضًا، وقطعوه.
\par 16 ثم نظر إلى الملك وقال: أنت متسلط على الناس وأنت فاسد وتفعل ما تريد ولكن لا تظن أن أمتنا متروكة من الله.
\par 17 لكن انتظر قليلًا، وانظر إلى قوته العظيمة، كيف سيعذبك أنت ونسلك
\par 18 وبعده أحضروا السادس، الذي كان على وشك الموت، وقال: لا تضلوا عبثًا، فإننا نتحمل هذه الأمور لأنفسنا، إذ أخطأنا إلى إلهنا، ولذلك تُصنع بنا عجائب
\par 19 ولكن لا تظن، أيها الذي يجاهد ضد الله، أنك ستنجو من العقاب
\par 20 لكن الأم كانت رائعة فوق كل شيء، وجديرة بالذكر المشرف: فعندما رأت أبناءها السبعة يُقتلون في غضون يوم واحد، تحملت الأمر بشجاعة كبيرة، بسبب رجائها في الرب
\par 21 نعم، لقد حثت كل واحدة منهن بلغتها الخاصة، مليئة بأرواح شجاعة؛ وأثارت أفكارها الأنثوية بمعدة رجولية، وقالت لهن:
\par 22 لا أستطيع أن أعرف كيف دخلتم إلى بطني، لأني لم أعطكم نفساً ولا حياة، ولا أنا شكلت أعضاء كل واحد منكم.
\par 23 ولكن لا شك أن خالق العالم، الذي خلق جيل الإنسان، واكتشف بداية كل الأشياء، سيمنحكم أيضًا برحمته نفسًا وحياة مرة أخرى، لأنكم الآن لا تنظرون إلى أنفسكم من أجل قوانينه
\par 24 أما أنطيوخس، فظن نفسه محتقرًا، وشك في أن هذا كلام تأنيب، بينما كان الأصغر لا يزال على قيد الحياة، فلم يكتفِ بنصحه بالكلام، بل أكد له أيضًا بالأيمان أنه سيجعله رجلًا غنيًا وسعيدًا، إذا عدل عن شرائع آبائه؛ وأنه سيتخذه صديقًا له، ويأتمنه على الأمور
\par 25 ولكن عندما لم يستمع الشاب إليه بأي حال من الأحوال، دعا الملك والدته وحثها على أن تنصح الشاب بإنقاذ حياته
\par 26 وبعد أن حثها بكلام كثير، وعدته أنها ستنصح ابنها
\par 27 لكنها انحنت نحوه، ضاحكة من الطاغية القاسي، وتحدثت بلغة بلدها على هذا النحو: يا بني، ارحمني أنا التي حملت بك تسعة أشهر في بطني، وأعطيتك ثلاث سنوات، وغذيتك، وربيتك إلى هذا العمر، وتحملت مشاق التعليم
\par 28 أتوسل إليك يا بني، انظر إلى السماء والأرض وكل ما فيهما، وتأمل أن الله خلقهما من أشياء لم تكن؛ وهكذا خُلق البشر على هذا النحو
\par 29 لا تخف من هذا المعذب، بل بما أنك جدير بإخوتك، فاقبل موتك حتى أقبلك مرة أخرى برحمة مع إخوتك
\par 30 وبينما هي تتكلم بهذا الكلام، قال الشاب: من تنتظرون؟ أنا لا أطيع أمر الملك، بل أطيع وصية الشريعة التي أعطيت لآبائنا عن طريق موسى
\par 31 وأنت، يا من كنتَ مُسبب كل الشرور ضد العبرانيين، لن تفلت من يدي الله
\par 32 لأننا نتألم بسبب خطايانا.
\par 33 وإن غضب الرب الحي علينا قليلًا لتأديبنا وتقويمنا، فإنه سيعود إلى التناغم مع عبيده
\par 34 أما أنت، أيها الإنسان الكافر، ومن بين جميع الأشرار، فلا تتكبر بلا سبب، ولا تنتفخ بآمال غير مؤكدة، رافعًا يدك على عبيد الله
\par 35 لأنكَ لم تنجُ بعدُ من دينونة الله القدير، الذي يرى كل شيء
\par 36 لأن إخوتنا الذين عانوا الآن ألمًا قصيرًا، قد ماتوا تحت عهد الله بالحياة الأبدية: أما أنت، فبحكم الله، ستنال عقابًا عادلًا على كبريائك
\par 37 ولكنني، كإخوتي، أقدم جسدي وحياتي من أجل شرائع آبائنا، وأتوسل إلى الله أن يرحم أمتنا سريعًا؛ وأن تعترف أنت بالعذابات والأوبئة بأنه هو الله وحده؛
\par 38 ولكي يتوقف عني وعن إخوتي غضب الله القدير الذي نزل بعدل على أمتنا.
\par 39 غضب الملك، فأمره بعقوبة أسوأ من البقية، وتأثر بشدة لأنه تعرض للسخرية
\par 40 فمات هذا الرجل بلا دنس، وتوكل كليًا على الرب.
\par 41 وأخيرًا بعد الأبناء توفيت الأم.
\par 42 يكفي الآن أن نتحدث عن الأعياد الوثنية، والعذابات الشديدة

\chapter{8}

\par 1 ثم ذهب يهوذا المكابي والذين معه سرًا إلى المدن، ودعوا أقرباءهم، وأخذوا إليهم كل من بقي على دين اليهود، وجمعوا نحو ستة آلاف رجل
\par 2 فدعوا الرب أن ينظر إلى الشعب المدوس من الجميع، وأن يشفق أيضًا على الهيكل المدنس من قبل الناس الأشرار؛
\par 3 وأنه يرحم المدينة المدمرة، والمستعدة للتسوية مع الأرض، ويسمع الدم الذي يصرخ إليه،
\par 4 واذكر المذبحة الشريرة للأطفال الأبرياء، والتجديفات التي ارتُكبت على اسمه، وأنه سيُظهر كراهيته للأشرار
\par 5 ولما كان المكابيين في صحبته، لم يستطع الوثنيون أن يقاوموه، لأن غضب الرب تحول إلى رحمة
\par 6 لذلك جاء على حين غرة، وأحرق بلدات ومدنًا، واستولى على أكثر الأماكن ملاءمةً، وتغلب على عدد ليس بقليل من أعدائه وهزمهم
\par 7 لكنه استغل الليل بشكل خاص لمثل هذه المحاولات السرية، لدرجة أن ثمرة قداسته انتشرت في كل مكان
\par 8 فلما رأى فيليب أن هذا الرجل ينمو شيئًا فشيئًا، وأن الأمور تزدهر معه أكثر فأكثر، كتب إلى بطليموس، حاكم بقاع سورية وفينيقية، ليقدم المزيد من المساعدة في شؤون الملك
\par 9 ثم اختار على الفور نيكانور ابن باتروكلس، أحد أصدقائه المقربين، وأرسله مع ما لا يقل عن عشرين ألفًا من جميع الأمم تحت إمرته، لاستئصال جيل اليهود بأكمله؛ وانضم إليه أيضًا جورجياس، وهو قائد كان يتمتع بخبرة كبيرة في أمور الحرب
\par 10 فتعهد نيكانور بجني قدر كبير من المال من اليهود الأسرى، يكفي لسداد الجزية البالغة ألفي وزنة، التي كان على الملك أن يدفعها للرومان
\par 11 لذلك أرسل على الفور إلى المدن الواقعة على ساحل البحر، معلنًا بيع اليهود الأسرى، ووعدهم بأن يكون لهم ثمانين جثة مقابل وزنة واحدة، غير متوقع الانتقام الذي سيلحق به من الله القدير
\par 12 ولما وصل خبر مجيء نيكانور إلى يهوذا، وأخبر الذين كانوا معه أن الجيش قد اقترب،
\par 13 أولئك الذين كانوا خائفين، ولم يثقوا بعدالة الله، هربوا، ورحلوا
\par 14 وباع آخرون كل ما تبقى لهم، وتوسلوا إلى الرب أن ينقذهم، الذين باعهم نكانور الشرير قبل أن يجتمعوا معًا:
\par 15 وإن لم يكن من أجل أنفسهم، فمن أجل العهود التي قطعها مع آبائهم، ومن أجل اسمه القدوس والمجيد الذي دُعوا به
\par 16 لذلك جمع المكابي رجاله حتى بلغ عددهم ستة آلاف، وحثهم على ألا يصابوا برعب العدو، ولا أن يخافوا من الجموع الغفيرة من الوثنيين الذين هاجموهم ظلماً؛ بل أن يقاتلوا بشجاعة،
\par 17 وأن يُعرض أمام أعينهم الضرر الذي ألحقوه ظلماً بالمكان المقدس، والمعاملة القاسية للمدينة، التي سخروا منها، وكذلك نزعهم حكم أجدادهم:
\par 18 لأنهم، كما قال، يثقون في أسلحتهم وجرأتهم؛ أما نحن، فثقتنا في القدير الذي يستطيع في أي لحظة أن يهزم كل من يهاجمنا، والعالم أجمع أيضًا
\par 19 علاوة على ذلك، قص عليهم ما وجده أجدادهم من عون، وكيف نجوا، عندما هلك مائة وخمسة وثمانون ألفًا في عهد سنحاريب
\par 20 وأخبرهم عن المعركة التي خاضوها في بابل مع الغلاطيين، وكيف وصلوا إلى المعركة ثمانية آلاف فقط، مع أربعة آلاف مقدوني، وأن المقدونيين لما تاهوا، أهلك الثمانية آلاف مئة وعشرين ألفًا بسبب المساعدة التي تلقوها من السماء، وهكذا نالوا غنيمة عظيمة
\par 21 وهكذا، بعد أن جعلهم جريئين بهذه الكلمات، ومستعدين للموت من أجل القانون والوطن، قسم جيشه إلى أربعة أقسام؛
\par 22 وانضم إلى إخوته، قادة كل فرقة، سمعان ويوسف ويوناثان، فأعطى كل واحد منهم ألفًا وخمسمائة رجل
\par 23 كما عيّن أليعازار لقراءة الكتاب المقدس، وبعد أن أعطاهم هذه الكلمة الرئيسية: معونة الله، قاد بنفسه الفرقة الأولى،
\par 24 وبمساعدة الله القدير، قتلوا أكثر من تسعة آلاف من أعدائهم، وجرحوا وشوهوا معظم جيش نيكانور، وهكذا أجبروا الجميع على الفرار؛
\par 25 وأخذوا أموالهم الذين جاءوا لشرائهم، وطاردوهم بعيدًا. ولكنهم رجعوا بسبب ضيق الوقت
\par 26 لأنه كان اليوم السابق للسبت، ولذلك لم يعودوا يطاردونهم
\par 27 فلما جمعوا أسلحتهم، ونهبوا أعداءهم، انشغلوا بالسبت، مقدمين تسبيحًا وشكرًا عظيمين للرب الذي حفظهم إلى ذلك اليوم، الذي كان بداية الرحمة التي حلت عليهم
\par 28 وبعد السبت، عندما أعطوا جزءًا من الغنيمة للجرحى والأرامل والأيتام، قسموا الباقي بينهم وبين عبيدهم
\par 29 عندما تم ذلك، وقدموا دعاءً مشتركًا، توسلوا إلى الرب الرحيم أن يتصالح مع عبيده إلى الأبد
\par 30 "ومن بين الذين كانوا مع تيموثاوس وبكيديس الذين حاربوهم، قتلوا أكثر من عشرين ألفًا، واستولوا بسهولة على حصون عالية وقوية، وقسموا فيما بينهم غنائم كثيرة أخرى، وجعلوا الجرحى واليتامى والأرامل، بل وحتى الشيوخ أيضًا، متساوين في الغنائم معهم."
\par 31 ولما جمعوا أسلحةهم، وضعها كلها بعناية في أماكن مناسبة، وأتوا ببقية الغنائم إلى أورشليم
\par 32 وقتلوا أيضًا فيلارخيس، ذلك الشخص الشرير، الذي كان مع تيموثاوس، وكان قد أزعج اليهود كثيرًا
\par 33 علاوة على ذلك، في الوقت الذي كانوا يحتفلون فيه بعيد النصر في بلادهم، أحرقوا كاليسثينيس، الذي أشعل النار في البوابات المقدسة، والذي هرب إلى منزل صغير؛ وهكذا نال مكافأة تليق بشره
\par 34 أما بالنسبة لذلك نيكانور الجاحد، الذي أحضر ألف تاجر لشراء اليهود،
\par 35 لقد أُنزِلَ بمعونة الرب على أيدي أولئك الذين لم يُقدِّرهم حق قدره، فخلع ثياب مجده، وسرَّح رفاقه، وجاء كعبد هارب عبر وسط البلاد إلى أنطاكية، وقد لحق به عار عظيم، لأن جيشه قد هلك
\par 36 وهكذا، هو الذي أخذ على عاتقه دفع الجزية للرومان عن طريق أسرى في القدس، أخبر في الخارج أن اليهود لديهم إله يقاتل من أجلهم، وبالتالي لا يمكن أن يتعرضوا للأذى، لأنهم اتبعوا القوانين التي أعطاها لهم

\chapter{9}

\par 1 في ذلك الوقت، خرج أنطيوخس من بلاد فارس مهانًا
\par 2 لأنه دخل المدينة التي تُدعى برسيبوليس، وكان ينوي نهب الهيكل والاستيلاء على المدينة؛ وعندها هرب الجموع للدفاع عن أنفسهم بأسلحتهم؛ وهكذا حدث أن أنطيوخس، بعد أن هرب من السكان، عاد خجلاً
\par 3 ولما وصل إلى إكباتاني، أُخبر بما حدث لنيكانور وتيموثاوس
\par 4 ثم ثار غضبًا وفكر في الانتقام من اليهود للعار الذي لحق به على يد أولئك الذين أجبروه على الفرار. لذلك أمر سائق عربته بالسير بلا توقف، وبإتمام الرحلة، ودينونة الله تتبعه الآن. لأنه كان قد تكلم بفخر بهذا الشكل، أنه سيأتي إلى أورشليم ويجعلها مدفنًا شائعًا لليهود
\par 5 لكن الرب القدير، إله إسرائيل، ضربه بضربة غير قابلة للشفاء وغير مرئية: أو بمجرد أن نطق بهذه الكلمات، أصابه ألم في أمعائه لا علاج له، وعذابات شديدة في أحشائه؛
\par 6 وذلك بحق: لأنه عذب أحشاء رجال آخرين بعذابات كثيرة وغريبة
\par 7 ولكنه لم يتوقف عن التباهي، بل كان لا يزال ممتلئًا بالكبرياء، ينفث نار غضبه على اليهود، ويأمر بالإسراع في السفر. ولكن حدث أنه سقط من مركبته التي حملت بعنف، حتى أن سقوطه الشديد جعل كل أعضاء جسده تتألم بشدة.
\par 8 وهكذا، فإن من ظن سابقًا أنه يستطيع التحكم بأمواج البحر (لأنه كان مغرورًا جدًا لدرجة تفوق قدرة الإنسان) ويزن الجبال العالية بميزان، أُلقي الآن على الأرض، وحُمل في محفة، مُظهرًا للجميع قوة الله الجلية
\par 9 فخرجت الديدان من جسد هذا الرجل الشرير، وبينما كان يعيش في حزن وألم، تساقط لحمه، وأصبحت قذارة رائحته كريهة لجميع جيشه
\par 10 والرجل الذي فكر قليلاً قبل أن يصل إلى نجوم السماء، لا يمكن لأحد أن يتحمل حمله بسبب رائحته الكريهة التي لا تُطاق
\par 11 لذلك، هنا، وقد أصابه البلاء، بدأ يتخلى عن كبريائه العظيم، ويصل إلى معرفة نفسه من خلال سوط الله، وكان ألمه يزداد في كل لحظة
\par 12 وعندما لم يستطع هو نفسه أن يتحمل رائحته، قال هذه الكلمات: من اللائق أن يخضع الإنسان لله، وأنه لا ينبغي للإنسان الفاني أن يظن بفخر أنه إله
\par 13 نذر هذا الشرير أيضًا للرب، الذي لن يرحمه بعد الآن، قائلاً:
\par 14 أن المدينة المقدسة (التي كان ذاهبًا إليها على عجل ليسوِّيها بالأرض، ويجعلها مدفنًا عامًا) سيُطلقها حرة:
\par 15 وأما بالنسبة لليهود، الذين حكم عليهم بأنهم غير مستحقين حتى للدفن، بل لطردهم مع أطفالهم لتلتهمهم الطيور والوحوش البرية، فإنه سيجعلهم جميعًا متساوين مع مواطني أثينا:
\par 16 وسيُزيّن الهيكل المقدس، الذي كان قد خربه من قبل، بهدايا ثمينة، ويُعيد جميع الأواني المقدسة بمزيد منها، ويدفع من دخله الخاص نفقات الذبائح
\par 17 نعم، وأنه هو أيضًا سيصبح يهوديًا، ويجوب العالم كله المأهول، ويعلن قوة الله
\par 18 ولكن مع كل هذا لم تتوقف آلامه، لأن دينونة الله العادلة قد حلت عليه، فيئس من صحته، فكتب إلى اليهود الرسالة المكتوبة أسفله، والتي تحتوي على شكل دعاء، على النحو التالي:
\par 19 أنطيوخس، الملك والحاكم، يتمنى لليهود الصالحين مواطنيه الكثير من الفرح والصحة والازدهار:
\par 20 إذا كنتم أنتم وأطفالكم بخير، وكانت أموركم على ما يرام، فإنني أشكر الله جزيل الشكر، وأملي في الجنة
\par 21 أما أنا، فقد كنت ضعيفًا، وإلا لكنت تذكرت بلطف شرفكم وحسن نيتكم عند عودتكم من بلاد فارس، ولأنني أصبت بمرض خطير، فقد رأيت أنه من الضروري الاهتمام بالسلامة العامة للجميع:
\par 22 لا أشك في صحتي، بل لدي أمل كبير في النجاة من هذا المرض
\par 23 ولكن بالنظر إلى أن والدي نفسه، عندما قاد جيشًا إلى الأراضي المرتفعة، عيّن خليفةً له،
\par 24 حتى لا ينزعج أهل البلاد، الذين يعرفون لمن تُركت هذه الولاية، إذا وقع أي شيء يخالف التوقعات، أو إذا وردت أي أنباء مؤسفة:
\par 25 وإذ أفكر أيضًا في أن الأمراء الذين هم حدود مملكتي وجيرانها ينتظرون الفرص، ويتوقعون ما سيحدث. فقد عينت ابني أنطيوخس ملكًا، الذي كثيرًا ما أوكلته وأوصيت به إلى كثيرين منكم، عندما صعدت إلى المقاطعات المرتفعة؛ والذي كتبت إليه ما يلي:
\par 26 لذلك أدعوكم وأطلب منكم أن تتذكروا النعم التي قدمتها لكم بشكل عام، وبشكل خاص، وأن يظل كل رجل مخلصًا لي ولابني
\par 27 لأني على يقين من أن من يفهم عقلي سيستجيب لرغباتك بلطف ولطف
\par 28 وهكذا، بعد أن عانى القاتل والمجدف معاناة شديدة، كما توسّل إلى رجال آخرين، مات ميتة بائسة في بلد غريب في الجبال
\par 29 فحمل فيلبس، الذي تربى معه، جثته، وخاف هو أيضًا من ابن أنطيوخس، فذهب إلى مصر إلى بطليموس فيلوماتور

\chapter{10}

\par 1 والآن استعاد المكابي وجماعته، بقيادة الرب، الهيكل والمدينة:
\par 2 لكن المذابح التي بناها الوثنيون في الشارع، وكذلك الكنائس، هدموها
\par 3 وبعد أن طهروا الهيكل، صنعوا مذبحًا آخر، وكسروا حجارة وأخرجوا منها نارًا، وقدموا ذبيحة بعد سنتين، وأقاموا بخورًا ومصابيح وخبز الوجوه
\par 4 عندما تم ذلك، سقطوا على الأرض، وتوسلوا إلى الرب ألا يقعوا في مثل هذه المشاكل مرة أخرى؛ ولكن إذا أخطأوا إليه مرة أخرى، فإنه سيؤدبهم هو نفسه بالرحمة، وألا يُسلموا إلى الأمم المجدفة والهمجية
\par 5 وفي اليوم نفسه الذي دنّس فيه الغرباء الهيكل، طُهِّرَ أيضًا في اليوم نفسه، وهو اليوم الخامس والعشرون من الشهر نفسه، وهو كسلو
\par 6 وأقاموا الأيام الثمانية بفرح، كما في عيد المظال، متذكرين أنهم قبل ذلك بقليل أقاموا عيد المظال، إذ كانوا يتجولون في الجبال والأوكار كالوحوش
\par 7 لذلك حملوا أغصانًا وأغصانًا جميلة، ونخيلًا أيضًا، وغنوا المزامير لمن نجح في تطهير مكانه
\par 8 كما رسموا أيضًا بموجب قانون ومرسوم مشتركين، أن تُحفظ تلك الأيام كل عام من قبل أمة اليهود بأكملها
\par 9 وكانت هذه نهاية أنطيوخس، الملقب بإبيفانيوس.
\par 10 والآن سوف نذكر أعمال أنطيوخس إيوباتور، الذي كان ابن هذا الرجل الشرير، ونجمع بإيجاز كوارث الحروب.
\par 11 فلما تولى العرش، جعل ليسياس على شؤون مملكته، وجعله حاكماً رئيسياً على بابل وفينيقيا.
\par 12 لأن بطليموس، الذي كان يُدعى ماكرون، اختار بالأحرى أن يُنصف اليهود على الظلم الذي لحق بهم، وسعى إلى استمرار السلام معهم
\par 13 عند ذلك، اتُهم من قبل أصدقاء الملك أمام إيفاتور، ووُصف بالخيانة في كل كلمة، لأنه غادر قبرص التي أوكلها إليه فيلوماتور، وذهب إلى أنطيوخس أبيفانس، ورأى أنه ليس في مكان مشرف، فشعر بالإحباط لدرجة أنه سمم نفسه ومات
\par 14 ولكن عندما كان جورجياس حاكمًا للحصون، استأجر جنودًا، وأشعل حربًا مستمرة مع اليهود:
\par 15 ومع ذلك، وبعد أن استولى الأدوميون على أكثر المعاقل راحةً في أيديهم، أبقوا اليهود مشغولين، واستقبلوا أولئك الذين نُفيوا من القدس، وشرعوا في تأجيج الحرب
\par 16 ثم توسل الذين كانوا مع المكابي، وتوسلوا إلى الله أن يكون معينهم، فهاجموا حصون الأدوميين بعنف،
\par 17 وهاجموهم بشدة، واستولوا على الحصون، وصدوا كل من حارب على السور، وقتلوا كل من وقع في أيديهم، وقتلوا ما لا يقل عن عشرين ألفًا
\par 18 ولأن بعضًا منهم، لا يقل عددهم عن تسعة آلاف، فروا معًا إلى قلعتين قويتين للغاية، مزودتين بجميع أنواع الأشياء المناسبة لتحمل الحصار،
\par 19 ترك المكابي سمعان ويوسف وزكا أيضًا والذين معه، الذين كانوا كافيين لمحاصرتهم، وانصرف إلى الأماكن التي كانت في أمس الحاجة إلى مساعدته
\par 20 أما الذين كانوا مع سمعان، فقد انساقوا بالطمع، فاقتنعوا بالمال عن طريق قوم من الذين في القلعة، فأخذوا سبعين ألف درهم، وتركوا قومًا يهربون
\par 21 ولما أُخبر المكابي بما حدث، جمع ولاة الشعب، واتهم أولئك الرجال بأنهم باعوا إخوتهم مقابل المال، وأطلقوا أعداءهم ليقاتلوهم
\par 22 فقتل من وُجد خائنًا، واستولى على القلعتين على الفور
\par 23 وبعد أن حقق نجاحًا كبيرًا باستخدام أسلحته في كل ما استولى عليه، قتل في المعقلين أكثر من عشرين ألفًا
\par 24 ثم جاء تيموثاوس، الذي تغلب عليه اليهود من قبل، بعد أن جمع جيشًا أجنبيًا غفيرًا، وخيولًا ليست قليلة من آسيا، كما لو كان سيأخذ اليهودية بقوة السلاح
\par 25 ولما اقترب، التفت الذين كانوا مع المكابي ليصلوا إلى الله، ورشوا التراب على رؤوسهم، وشدوا أحقاءهم بالمسوح،
\par 26 فخرّوا عند أسفل المذبح، وتوسّلوا إليه أن يرحمهم، ويكون عدوًا لأعدائهم، وخصمًا لمُضادّيهم، كما ينصّ الناموس
\par 27 وبعد الصلاة أخذوا أسلحتهم وابتعدوا عن المدينة، وعندما اقتربوا من أعدائهم، توقفوا عند أنفسهم.
\par 28 والآن، وقد أشرقت الشمس حديثًا، انضموا إلى الاثنين معًا؛ فكان لدى أحدهما، إلى جانب فضيلته، ملجأه أيضًا إلى الرب كعربون نجاحه وانتصاره؛ بينما جعل الجانب الآخر غضبه قائدًا لمعركته
\par 29 ولما اشتدت المعركة، ظهر للأعداء من السماء خمسة رجال جميلين على خيول، ولجمهم من ذهب، وقاد اثنان منهم اليهود،
\par 30 وأخذوا المكابي بينهم، وغطوه بالأسلحة من كل جانب، وحفظوه آمنًا، لكنهم أطلقوا السهام والبروق على الأعداء، فإذ أصيبوا بالعمى وامتلأوا بالمتاعب، قُتلوا
\par 31 وقُتل من المشاة عشرون ألفًا وخمسمائة، ومن الفرسان ستمائة
\par 32 أما تيموثاوس نفسه، فقد هرب إلى حصن منيع للغاية، يُدعى جاورا، حيث كان تشيرياس حاكمًا
\par 33 أما الذين كانوا مع المكابي فقد حاصروا الحصن بشجاعة أربعة أيام
\par 34 والذين كانوا في الداخل، واثقين من حصن المكان، جدفوا جدفًا، ونطقوا بكلمات رديئة
\par 35 ومع ذلك، في اليوم الخامس، هاجم عشرون شابًا من جماعة المكابيين، وقد اشتعل غضبهم بسبب التجديفات، السور بشجاعة، وبشجاعة شديدة قتلوا كل من واجهوه
\par 36 وصعد آخرون أيضًا وراءهم، بينما كانوا مشغولين بمن كانوا في الداخل، أحرقوا الأبراج، وأشعلوا النيران فأحرقوا المجدفين أحياءً؛ وفتح آخرون الأبواب، وبعد أن استقبلوا بقية الجيش، استولوا على المدينة،
\par 37 فقتل تيموثاوس الذي كان مختبئًا في حفرة، وكيرياس أخاه، مع أبولوفان
\par 38 ولما تم ذلك، سبحوا الرب بالمزامير والحمد، الذي صنع أشياء عظيمة لإسرائيل، وأعطاهم النصر

\chapter{11}

\par 1 بعد فترة وجيزة، شعر ليسياس، حامي الملك وابن عمه، والذي كان يدير الشؤون أيضًا، باستياء شديد من الأشياء التي تم القيام بها
\par 2 فجمع نحو ثمانين ألفًا مع جميع الفرسان، وهاجم اليهود، ظانًا أنه سيجعل المدينة مسكنا للأمم،
\par 3 ولجني ربح من الهيكل، كما في سائر كنائس الوثنيين، ولبيع رئاسة الكهنوت كل سنة:
\par 4 لم يكن يفكر إطلاقًا في قوة الله، بل كان منتفخًا بعشرات الآلاف من مشاته، وآلاف فرسانه، وثمانين فيلة
\par 5 فجاء إلى اليهودية، وتقدم إلى بيت صور، وهي مدينة حصينة، بعيدة عن أورشليم نحو خمسة غلوات، وحاصرها حصارًا شديدًا
\par 6 ولما سمع الذين كانوا مع المكابي أنه حاصر الحصون، توسلوا هم وكل الشعب إلى الرب بنحيب ودموع أن يرسل ملاكًا صالحًا لإنقاذ إسرائيل
\par 7 فأخذ المكابي بنفسه السلاح أولاً، وحث الآخرين على أن يخاطروا بأنفسهم معه لمساعدة إخوتهم: فخرجوا معًا بروح راغبة.
\par 8 وفيما هم في أورشليم، ظهر لهم راكبًا على فرس بلباس أبيض، وهو يرتدى درعه الذهبي
\par 9 ثم سبحوا الله الرحيم جميعًا، وتشجعوا، لدرجة أنهم كانوا مستعدين ليس فقط للقتال مع البشر، بل ومع أشد الوحوش قسوة، ولاختراق جدران الحديد
\par 10 وهكذا ساروا بسلاحهم، ومعهم معين من السماء، لأن الرب رحمهم
\par 11 فانقضوا على أعدائهم كالأسود، فقتلوا أحد عشر ألفًا من المشاة، وألفًا وستمائة فارس، وهزموا الباقين
\par 12 وكثيرون منهم جرحى فهربوا عراة، وليسياس نفسه هرب مخجلاً، وهكذا نجا
\par 13 الذي، إذ كان رجلاً فاهمًا، ألقى في نفسه ما لحق به من خسارة، وإذ رأى أن العبرانيين لا يمكن التغلب عليهم، لأن الله القدير ساعدهم، أرسل إليهم،
\par 14 وأقنعهم بالموافقة على جميع الشروط المعقولة، ووعد بأنه سيقنع الملك بأنه يجب أن يكون صديقًا لهم
\par 15 فوافق المكابي على كل ما طلبه ليسياس، حرصًا على الصالح العام؛ وكل ما كتبه المكابي إلى ليسياس بشأن اليهود، أعطاه الملك
\par 16 لأنه كانت هناك رسائل مكتوبة إلى اليهود من ليسياس بهذا المعنى: ليسياس يرسل سلامًا إلى شعب اليهود
\par 17 يوحنا وأبشالوم، اللذان أُرسِلا من قِبَلِكُم، سلَّماني العريضة الموقعة، وطلبا تنفيذ ما فيها
\par 18 لذلك، كل ما كان من اللائق إبلاغه للملك، فقد أعلنته، ومنحه ما أمكن
\par 19 وإذا حافظتم على ولائكم للدولة، فسأسعى أيضًا في المستقبل لأن أكون وسيلة لخيركم
\par 20 ولكنني أمرت هذه التفاصيل، وكذلك الآخرين الذين جاءوا مني، بالتواصل معكم
\par 21 وداعًا. السنة المائة والثامنة والأربعون، اليوم الرابع والعشرون من شهر ديوسكورنثيوس
\par 22 واحتوت رسالة الملك على هذه الكلمات: من الملك أنطيوخس إلى أخيه ليسياس سلام:
\par 23 بما أن والدنا قد نُقل إلى الآلهة، فإن إرادتنا هي أن يعيش من في مملكتنا بهدوء، وأن يتمكن كل واحد منهم من الاهتمام بشؤونه الخاصة
\par 24 ونفهم أيضًا أن اليهود لم يوافقوا على أبينا، حتى لا يعتادوا على عادات الأمم، بل فضلوا أن يحتفظوا بأسلوب حياتهم الخاص: ولهذا السبب يطلبون منا أن نسمح لهم بالعيش وفقًا لشرائعهم الخاصة
\par 25 ولذلك فإننا نعتقد أن هذه الأمة سوف تحظى براحة، وقد قررنا أن نعيد لهم هيكلهم، حتى يتمكنوا من العيش وفقًا لعادات آبائهم.
\par 26 لذا، من الجيد أن ترسل إليهم وتمنحهم السلام، حتى عندما يتأكدوا من صحة أفكارنا، قد يشعرون بالراحة، ويواصلون شؤونهم الخاصة بسعادة دائمًا
\par 27 وكانت رسالة الملك إلى أمة اليهود على هذا النحو: يرسل الملك أنطيوخس سلامًا إلى المجلس وسائر اليهود
\par 28 إن كنتم بخير، فلنا رغبتنا، ونحن أيضًا بصحة جيدة.
\par 29 أخبرنا مينيلانس أن رغبتكم كانت في العودة إلى دياركم ومتابعة أعمالكم الخاصة:
\par 30 لذلك، يجب على أولئك الذين سيغادرون أن يتمتعوا بممر آمن حتى اليوم الثلاثين من شهر زانثيكوس مع ضمان
\par 31 ويجب على اليهود استخدام أنواع اللحوم والقوانين الخاصة بهم، كما في السابق؛ ولن يُضايق أي منهم بأي شكل من الأشكال بسبب أشياء فعلها بجهل
\par 32 لقد أرسلتُ أيضًا مينيلانس ليُعزيكم.
\par 33 وداعًا. في السنة المائة والثامنة والأربعين، وفي اليوم الخامس عشر من شهر زانثيكوس.
\par 34 أرسل الرومان إليهم أيضًا رسالة تحتوي على هذه الكلمات: كوينتوس ميميوس وتيتوس مانليوس، سفيرا الرومان، يرسلان تحياتهما إلى شعب اليهود
\par 35 كل ما منحه ليسياس ابن عم الملك، فنحن أيضًا راضون عنه
\par 36 أما فيما يتعلق بالأمور التي رأى إحالتها إلى الملك، فبعد أن تتشاوروا بشأنها، أرسلوا واحدًا على الفور، حتى نتمكن من إعلان ما يناسبكم، لأننا ذاهبون الآن إلى أنطاكية
\par 37 لذلك أرسل بعضًا بسرعة، حتى نعرف ما تفكر فيه.
\par 38 وداعًا. في هذه السنة المائة والثامنة والأربعين، في اليوم الخامس عشر من شهر زانثيكوس.

\chapter{12}

\par 1 ولما أُبرمت هذه العهود، ذهب ليسياس إلى الملك، وكان اليهود في أعمالهم الزراعية
\par 2 لكن من بين حكام عدة أماكن، تيموثاوس، وأبولونيوس بن جينيوس، وهيرونيموس، وديموفون، وبجانبهم نيكانور حاكم قبرص، لم يسمحوا لهم بالهدوء والعيش في سلام
\par 3 وفعل رجال يافا أيضًا فعلًا شريرًا: فقد صلوا إلى اليهود المقيمين بينهم أن يذهبوا مع زوجاتهم وأطفالهم إلى السفن التي أعدوها، كما لو أنهم لم يقصدوا لهم أي أذى
\par 4 الذين قبلوا ذلك وفقًا للمرسوم العام للمدينة، راغبين في العيش بسلام، ولا يشككون في شيء: ولكن عندما خرجوا إلى الأعماق، أغرقوا ما لا يقل عن مئتين منهم
\par 5 ولما سمع يهوذا بهذه القسوة التي وُجِّهت إلى أهل وطنه، أمر الذين كانوا معه أن يُعِدّوا أنفسهم
\par 6 ودعا الله القاضي العادل، فجاء على قتلة إخوته، وأحرق الميناء ليلًا، وأشعل النار في القوارب، وقتل من هرب إلى هناك
\par 7 ولما أغلقت المدينة، تراجع إلى الوراء كأنه يريد أن يعود ليقتلع جميع أهل مدينة يافا.
\par 8 فلما سمع أن أهل يمنيا ينوون أن يفعلوا مثل ذلك باليهود الساكنين بينهم،
\par 9 ثم هاجم اليمنيين ليلًا أيضًا، وأشعل النار في الميناء والأسطول، حتى ظهر ضوء النار في أورشليم على بُعد مئتين وأربعين غلوة
\par 10 ولما انصرفوا من هناك تسعة فيرلنغ في رحلتهم نحو تيموثاوس، هاجمه ما لا يقل عن خمسة آلاف رجل راجل وخمسمائة فارس من العرب
\par 11 عندها نشبت معركة ضارية للغاية؛ لكن جانب يهوذا انتصر بمعونة الله؛ حتى أن بدو العرب، بعد أن هُزموا، توسلوا إلى يهوذا من أجل السلام، ووعدوه بإعطائه ماشية، وبإرضائه بطريقة أخرى
\par 12 ثم إن يهوذا إذ ظن أنهم ينفعون في أمور كثيرة منحهم السلام، فتصافحوا وانصرفوا إلى خيامهم
\par 13 كما سعى لبناء جسر إلى مدينة حصينة، مسيجة بأسوار، ويسكنها شعوب من بلدان مختلفة؛ وكان اسمها كاسبيس
\par 14 لكن الذين كانوا داخلها كانوا يثقون بمتانة الأسوار ووفرة الطعام، حتى أنهم تصرفوا بفظاظة مع الذين كانوا مع يهوذا، فبدأوا يشتمون ويجدفون، وينطقون بكلمات لا ينبغي النطق بها
\par 15 لذلك، قام يهوذا ورفاقه، داعين رب العالم العظيم، الذي هدم أريحا في زمن يشوع بدون كباش أو آلات حرب، بهجوم شرس على الأسوار،
\par 16 واستولوا على المدينة بمشيئة الله، وارتكبوا مذابح لا توصف، حتى إن بحيرة مجاورة لها، عرضها فرسخانان، قد امتلأت، وظهرت وهي تسيل بالدماء
\par 17 ثم انطلقوا من هناك مسافة سبعمائة وخمسين غلوة، وجاءوا إلى خاراكا إلى اليهود الذين يُدعون توبيين
\par 18 وأما تيموثاوس، فلم يجدوه في تلك الأماكن، لأنه قبل أن يرسل شيئًا، انصرف من هناك، تاركًا حامية قوية جدًا في حصن معين
\par 19 لكن دوسيثيوس وسوسيباتر، وهما من قادة المكابي، خرجا وقتلا من تركهم تيموثاوس في الحصن، وكان عددهم أكثر من عشرة آلاف رجل
\par 20 ووزع المكابي جيشه في فرق، وجعلهم على الفرق، وسار لمحاربة تيموثاوس، وكان معه مئة وعشرون ألف راجل، وألفان وخمسمائة فارس
\par 21 ولما علم تيموثاوس بمجيء يهوذا، أرسل النساء والأطفال والأمتعة الأخرى إلى حصن يُدعى كارنيون، لأن المدينة كانت صعبة الحصار، وكان الوصول إليها صعبًا، بسبب ضيق جميع الأماكن
\par 22 "ولكن عندما ظهرت فرقته الأولى، أصيب الأعداء بالخوف والرعب من ظهور الذي يرى كل شيء، فهربوا، واحد يركض في هذا الطريق، وآخر في ذلك الطريق، حتى أنهم أصيبوا مراراً وتكراراً بأذى من رجالهم، وجرحوا بنهايات سيوفهم."
\par 23 وكان يهوذا أيضًا جادًا جدًا في مطاردتهم، فقتل أولئك الأشرار، فقتل منهم نحو ثلاثين ألف رجل
\par 24 علاوة على ذلك، وقع تيموثاوس نفسه في أيدي دوسيثاوس وسوسيباتر، اللذين توسل إليهما بمكر شديد أن يتركاه يرحل بحياته، لأنه كان لديه العديد من آباء اليهود، وإخوة بعضهم، الذين إذا قتلوه، فلا ينبغي اعتبارهم
\par 25 فلما أكد لهم بكلام كثير أنه سيعيدهم دون أذى، حسب الاتفاق، تركوه من أجل إنقاذ إخوتهم
\par 26 ثم سار المكابي إلى كارنيون، وإلى معبد أتارغاتيس، وهناك قتل خمسة وعشرين ألف شخص
\par 27 وبعد أن هزمهم وأهلكهم، نقل يهوذا الجيش نحو عفرون، وهي مدينة حصينة كان يقيم فيها ليسياس، وجمهور كثير من الأمم المختلفة، وكان الشبان الأقوياء يحرسون الأسوار ويدافعون عنها بشدة، وكان فيها أيضًا مؤن كثيرة من الآلات والسهام
\par 28 ولكن عندما دعا يهوذا وجماعته الله القدير، الذي بقوته يكسر قوة أعدائه، استولوا على المدينة، وقتلوا خمسة وعشرين ألفًا ممن كانوا فيها،
\par 29 ومن هناك انطلقوا إلى سكيثوبوليس، التي تقع على بعد ستمائة فرسخ من القدس،
\par 30 ولكن عندما شهد اليهود الذين سكنوا هناك أن السكيثوبوليتيين عاملوهم بمحبة، وتوسلوا إليهم بلطف في وقت محنتهم؛
\par 31 فشكروهم، وطلبوا منهم أن يظلوا ودودين معهم، وهكذا وصلوا إلى أورشليم، وعيد الأسابيع يقترب
\par 32 وبعد العيد المسمى بالعنصرة، خرجوا على جرجياس والي أدوم،
\par 33 الذي خرج بثلاثة آلاف رجل من المشاة وأربعمائة فارس
\par 34 وحدث أنه في قتالهم قُتل عدد قليل من اليهود
\par 35 في ذلك الوقت، كان دوسيثيوس، أحد أفراد فرقة باكينور، وهو رجل قوي، لا يزال على جورجياس، فأمسك بمعطفه وجره بالقوة؛ وعندما أراد أن يقبض على ذلك الرجل الملعون حيًا، هاجمه فارس من تراقيا وضربه على كتفه، فهرب جورجياس إلى ماريسا
\par 36 ولما طال قتال الذين كانوا مع جورجياس، وتعبوا، دعا يهوذا الرب أن يظهر لهم معينًا وقائدًا في المعركة
\par 37 وبدأ بذلك بلغته الخاصة، وغنى المزامير بصوت عالٍ، واندفع على حين غرة على رجال جورجياس، وهزمهم
\par 38 فجمع يهوذا جيوشه وجاء إلى مدينة أدولَّام. ولما كان اليوم السابع تطهروا على العادة وحفظوا السبت في ذلك المكان.
\par 39 وفي اليوم التالي، كما جرت العادة، جاء يهوذا ورفاقه ليأخذوا جثث القتلى، ويدفنوها مع أقاربهم في قبور آبائهم
\par 40 وجدوا تحت معاطف كل من قُتلوا أشياءً مُقدسة لأصنام اليمنيين، وهو أمرٌ محرمٌ على اليهود بالشريعة. فرأى كل رجل أن هذا هو سبب قتلهم
\par 41 فسبح جميع الناس الرب، الديان العادل، الذي كشف ما كان مخفيًا،
\par 42 توجهوا إلى الصلاة، وتوسلوا إليه أن تُمحى الخطيئة التي ارتكبوها تمامًا من الذاكرة. علاوة على ذلك، حث يهوذا النبيل الناس على الامتناع عن الخطيئة، لأنهم رأوا أمام أعينهم ما حدث بسبب خطايا الذين قُتلوا
\par 43 فجمع في الجماعة ألفي درهم من الفضة، وأرسلها إلى أورشليم ليقدم ذبيحة خطية، وقد أحسن صنعًا وأمانةً إذ تذكر القيامة
\par 44 لأنه لو لم يكن يرجو قيامة القتلى، لكان من غير الضروري والعبث أن يصلي من أجل الموتى
\par 45 وأيضًا لأنه أدرك أن هناك نعمة عظيمة مُعدّة لأولئك الذين ماتوا أتقياء، فقد كان فكرًا مقدسًا وصالحًا. وبناءً عليه، أجرى مصالحة عن الموتى، حتى يتحرروا من الخطيئة

\chapter{13}

\par 1 في السنة المائة والتاسعة والأربعين، أُخبِر يهوذا أن أنطيوخس أوباتور قادم بجيش عظيم إلى اليهودية،
\par 2 ومعه ليسياس حاميه وحاكم شؤونه، ولكل منهما قوة يونانية من المشاة، مئة وعشرة آلاف، والفرسان خمسة آلاف وثلاثمائة، والفيلة اثنان وعشرون، وثلاثمائة عربة مسلحة بخطافات
\par 3 انضم مينيلانوس إليهم أيضًا، وشجع أنطيوخس بمكر شديد، ليس من أجل حماية البلاد، ولكن لأنه اعتقد أنه قد عُيّن حاكمًا
\par 4 لكن ملك الملوك حرك عقل أنطيوخس ضد هذا الشرير، وأبلغ ليسياس الملك أن هذا الرجل هو سبب كل الشر، لذلك أمر الملك بإحضاره إلى بيريا وقتله، كما هو متبع في ذلك المكان
\par 5 وكان في ذلك المكان برج ارتفاعه خمسون ذراعًا، مملوء رمادًا، وله أداة مستديرة تتدلى من كل جانب في الرماد
\par 6 وكل من أُدين بانتهاك المقدسات، أو ارتكب أي جريمة خطيرة أخرى، فقد دفعه جميع الناس إلى الموت
\par 7 لقد حدث أن مات هذا الشرير دون أن يُدفن في الأرض، وكان ذلك عادلاً للغاية:
\par 8 لأنه كان قد ارتكب خطايا كثيرة بشأن المذبح، الذي كانت ناره ورماده مقدسين، فقد تلقى موته في الرماد.
\par 9 ثم جاء الملك بعقل همجي ومتغطرس ليفعل باليهود ما هو أسوأ بكثير مما حدث في عهد أبيه
\par 10 ولما أدرك يهوذا ذلك، أمر الجموع أن يدعوا الرب ليلًا ونهارًا، حتى يساعدهم الآن أيضًا إن أتيحت لهم الفرصة، إذ كانوا على وشك أن يُطردوا من شريعتهم، ومن أرضهم، ومن الهيكل المقدس
\par 11 وأنه لن يدع الشعب، الذي لم ينتعش إلا قليلاً، يخضع للأمم المجدفة
\par 12 فلما فعلوا ذلك جميعًا معًا، وتوسلوا إلى الرب الرحيم بالبكاء والصوم والاستلقاء على الأرض ثلاثة أيام، حثهم يهوذا وأمرهم أن يكونوا على استعداد
\par 13 وكان يهوذا منفردًا مع الشيوخ، فعزم، قبل أن يدخل جيش الملك إلى اليهودية ويستولي على المدينة، أن يخرج ويفحص الأمر بالقتال بمعونة الرب
\par 14 لذلك، بعد أن سلم كل شيء إلى خالق العالم، وحث جنوده على القتال بشجاعة، حتى الموت، من أجل القوانين، والمعبد، والمدينة، والبلاد، والبلد، خيّم عند مودين:
\par 15 وبعد أن أعطى كلمة السر لمن حوله، النصر من الله، دخل خيمة الملك ليلاً مع أشجع وأختار الشباب، وقتل في المعسكر نحو أربعة آلاف رجل، ورئيس الفيلة، مع كل من كان عليه
\par 16 وأخيرًا ملأوا المخيم بالخوف والاضطراب، وغادروا بنجاح كبير
\par 17 تم ذلك عند طلوع الفجر، لأن حماية الرب أعانته
\par 18 ولما ذاق الملك شجاعة اليهود، همّ بأخذ الحصون بالسياسة،
\par 19 وسار نحو بيت صور، التي كانت معقلًا حصينًا لليهود، لكنه هُزم، وفشل، وفقد رجاله
\par 20 لأن يهوذا كان قد نقل إلى الذين كانوا فيها الأمور اللازمة
\par 21 لكن رودوكس، الذي كان في جيش اليهود، كشف الأسرار للأعداء، لذلك تم البحث عنه، وعندما أمسكوا به، وضعوه في السجن
\par 22 عاملهم الملك في بيت سوم للمرة الثانية، ومد يده، وأخذ أيديهم، وانصرف، وحارب يهوذا، وانهزم؛
\par 23 سمع أن فيليب، الذي تُرك على رأس شؤون أنطاكية، كان منحنيًا بشدة، ومُرتبكًا، وتوسل إلى اليهود، وخضع، وأقسم على جميع الشروط المتساوية، ووافقهم الرأي، وقدم ذبيحة، وكرّم الهيكل، وعامل المكان بلطف،
\par 24 وتقبلوا حسن نية المكابيين، وجعلوه حاكمًا رئيسيًا من بطليموس إلى الجرهيين؛
\par 25 وصل إلى بطليموس: وكان الناس هناك حزينين على العهود؛ لأنهم اقتحموا المدينة لأنهم يريدون إبطال عهودهم:
\par 26 صعد ليسياس إلى كرسي القضاء، ودافع عن القضية بكل ما أوتي من قوة، وأقنعهم، وهدأهم، وأقنعهم، ثم عاد إلى أنطاكية. وهكذا سارت الأمور في مجيء الملك ومغادرته.

\chapter{14}

\par 1 بعد ثلاث سنوات، أُبلغ يهوذا أن ديمتريوس بن سلوقس، قد دخل من ميناء طرابلس بقوة وأسطول عظيمين،
\par 2 استولى على البلاد، وقتل أنطيوخس، وليسياس حاميه
\par 3 كان هناك رجل يُدعى ألكيمُس، وكان رئيس كهنة، وقد تنجس عمدًا في أوقات اختلاطهم بالأمم، إذ رأى أنه لا يستطيع بأي حال من الأحوال أن يخلص نفسه، ولا أن يكون له أي وصول إلى المذبح المقدس،
\par 4 جاء إلى الملك ديمتريوس في السنة المائة والحادية والخمسين، وقدم له إكليلًا من ذهب، وسعفة، وأيضًا من الأغصان التي تُستخدم رسميًا في الهيكل، وهكذا التزم الصمت في ذلك اليوم
\par 5 ولكن بعد أن سنحت له الفرصة لمواصلة مغامرته الحمقاء، واستدعاه ديمتريوس إلى المشورة، وسأله عن موقف اليهود، وماذا ينوون، أجاب:
\par 6 أولئك اليهود الذين أسماهم الأسيديين، وقائدهم يهوذا المكابي، يغذون الحرب ويثيرون الفتنة، ولن يدعوا الباقين يعيشون في سلام
\par 7 لذلك، أنا، وقد حُرمتُ من شرف أجدادي، أعني رئاسة الكهنوت، فقد جئتُ الآن إلى هنا:
\par 8 أولًا، حقًا لاهتمامي غير المصطنع بالأمور المتعلقة بالملك؛ وثانيًا، حتى في ذلك أقصد خير مواطني بلدي: لأن أمتنا بأكملها في بؤس كبير بسبب التعامل غير الحكيم من هؤلاء المذكورين آنفًا
\par 9 لذلك، أيها الملك، بما أنك تعلم كل هذه الأمور، فاحرص على بلدنا وأمتنا التي تتعرض للضغط من كل جانب، وفقًا للرحمة التي تظهرها للجميع
\par 10 ما دام يهوذا حيًا، فليس من الممكن أن تبقى الدولة هادئة
\par 11 لم يكد يُقال هذا عنه، حتى قام آخرون من أصدقاء الملك، الذين كانوا يحرضون على يهوذا بخبث، بمزيد من الاستهجان لديمتريوس
\par 12 فدعا في الحال نيكانور، الذي كان سيد الفيلة، وجعله واليًا على اليهودية، وأرسله،
\par 13 وأمره بقتل يهوذا، وتفريق من كانوا معه، وجعل ألكيموس رئيس كهنة الهيكل العظيم
\par 14 ثم جاء الوثنيون الذين فروا من اليهودية من وجه يهوذا إلى نكانور قطعانًا، ظانين أن الأذى والبلاء الذي حل باليهود هو خيرهم
\par 15 فلما سمع اليهود بمجيء نيكانور، وأن الوثنيين يهاجمونهم، ألقوا التراب على رؤوسهم، وتضرعوا إلى من ثبت شعبه إلى الأبد، والذي يساعد نصيبه دائمًا بإظهار حضوره
\par 16 فبأمر القائد، انطلقوا من هناك على الفور، ووصلوا إلى مدينة ديساو
\par 17 وكان سمعان أخو يهوذا قد انضم إلى المعركة مع نكانور، ولكنه شعر ببعض الاضطراب بسبب الصمت المفاجئ لأعدائه.
\par 18 ومع ذلك، لما سمع نكانور بشجاعة الذين كانوا مع يهوذا، وبسالتهم في القتال من أجل وطنهم، لم يجرؤ على أن يحاكم الأمر بالسيف
\par 19 لذلك أرسل بوسيدونيوس، وثيودوتوس، ومتاثياس، لعقد السلام
\par 20 وبعد أن تشاوروا مطوّلاً بشأن ذلك، وأطلع القائد الجموع عليه، وظهر أنهم جميعًا على رأي واحد، وافقوا على العهود،
\par 21 وحددوا يومًا للاجتماع معًا على انفراد، وعندما جاء اليوم، ووُضعت كراسي لكل منهما،
\par 22 وضع لوداس رجالاً مسلحين في أماكن مناسبة، خشية أن يقوم الأعداء بخيانة مفاجئة، فعقدوا مؤتمرًا سلميًا
\par 23 وأما نكانور فكان يقيم في أورشليم، ولم يفعل شيئًا مؤذيًا، بل صرف الجموع التي كانت تأتي إليه متجمهرة
\par 24 ولم يشأ أن يغيب يهوذا عن نظره طوعًا، لأنه أحب الرجل من قلبه
\par 25 ودعاه أيضًا أن يتزوج وينجب أطفالًا: فتزوج، وكان هادئًا، وشارك في هذه الحياة
\par 26 لكن ألكيموس، إذ رأى المحبة التي كانت بينهما، ونظر في العهود التي عُقدت، جاء إلى ديمتريوس، وأخبره أن نيكانور لم يكن متعاطفًا مع الدولة؛ لأنه عيّن يهوذا، الخائن لمملكته، خليفةً للملك
\par 27 ثم غضب الملك واستشاط غضبًا باتهامات الرجل الأكثر شرًا، فكتب إلى نكانور، مبينًا أنه مستاء للغاية من العهود، وأمره بإرسال المكابي أسيرًا على وجه السرعة إلى أنطاكية
\par 28 عندما وصل هذا إلى سمع نيكانور، شعر بالحيرة الشديدة في نفسه، وتقبل بشدة أنه يجب عليه إلغاء البنود التي تم الاتفاق عليها، حيث لم يكن الرجل مخطئًا
\par 29 ولكن لأنه لم يكن هناك أي تعامل ضد الملك، فقد راقب وقته لإنجاز هذا الأمر من خلال السياسة
\par 30 ومع ذلك، عندما رأى المكابي أن نيكانور بدأ يتصرف بفظاظة معه، وأنه يعامله بقسوة أكثر مما اعتاد، أدرك أن هذا السلوك الفظ لا يأتي من الخير، فجمع عددًا غير قليل من رجاله، وانسحب من نيكانور
\par 31 أما الآخر، فلما علم أنه مُنع بشكل ملحوظ بسبب سياسة يهوذا، دخل الهيكل العظيم المقدس، وأمر الكهنة الذين كانوا يقدمون ذبائحهم المعتادة أن يسلموه الرجل
\par 32 ولما أقسموا أنهم لا يستطيعون معرفة مكان الرجل الذي يبحث عنه،
\par 33 ومد يده اليمنى نحو الهيكل، وأقسم قائلًا: إن لم تُسلموا لي يهوذا أسيرًا، فسأهدم هيكل الله هذا بالأرض، وسأهدم المذبح، وأقيم لباخوس هيكلًا عظيمًا
\par 34 وبعد هذه الكلمات انصرف. فرفع الكهنة أيديهم نحو السماء، وتضرعوا إلى من كان حاميًا لأمتهم على الدوام، قائلين:
\par 35 أنت، يا رب كل شيء، الذي لا تحتاج إلى شيء، سُررتَ أن يكون هيكل سكناك بيننا:
\par 36 والآن، أيها الرب القدوس، رب كل قداسة، احفظ هذا البيت بلا دنس إلى الأبد، الذي طُهِّر مؤخرًا، وسدّ كل فم إثم
\par 37 ثم وُجِّهَتْ اتهاماتٌ إلى نكانور، وهو رازيس، أحد شيوخ أورشليم، محبٌّ لأبناء وطنه، ورجلٌ حسن السمعة، وكان يُدعى أبًا لليهود من أجل لطفه
\par 38 لأنه في الأزمنة السابقة، حين كانوا لا يختلطون بالأمم، اتُهم باليهودية، فجازف بجسده وحياته بكل جرأة من أجل ديانة اليهود
\par 39 فأرسل نكانور، راغبًا في إعلان الكراهية التي يكنها لليهود، أكثر من خمسمائة رجل حرب ليأخذوه:
\par 40 لأنه ظن أنه بأخذه سيضر اليهود كثيرًا.
\par 41 والآن عندما أراد الجمع أن يستولي على البرج، وكسر الباب الخارجي بعنف، وأمروا بإحضار النار لحرقه، كان مستعدًا للهجوم من كل جانب، فسقط على سيفه؛
\par 42 اختار أن يموت بشجاعة، بدلاً من أن يقع في أيدي الأشرار، وأن يُساء إليه بطريقة لا تليق بمولده النبيل:
\par 43 لكنه أخطأ ضربته بسبب العجلة، والحشد يتدفق أيضًا داخل الأبواب، فركض بجرأة إلى الجدار، وألقى بنفسه بشجاعة بين أكثرهم كثافة
\par 44 لكنهم سرعان ما تراجعوا، وفُتح مكان، فسقط في وسط المكان الفارغ
\par 45 ومع ذلك، وبينما كانت لا تزال فيه نسمة، اشتعل غضبًا، فنهض؛ ورغم أن دمه انسكب كصواعق الماء، وكانت جراحه خطيرة، إلا أنه ركض وسط الحشد؛ ووقف على صخرة شديدة الانحدار،
\par 46 عندما اختفى دمه تمامًا، انتزع أحشائه، وأخذها بكلتا يديه، وألقاها على الحشد، ودعا رب الحياة والروح أن يعيدها إليه، وهكذا مات

\chapter{15}

\par 1 ولكن نكانور، لما سمع أن يهوذا وجماعته في حصون السامرة، عزم على مهاجمتهم يوم السبت دون أي خطر
\par 2 مع ذلك، قال اليهود الذين أُجبروا على الذهاب معه: لا تُهلكوا بهذه القسوة والوحشية، بل أعطوا كرامة لذلك اليوم الذي كرّمه بقداسة من يرى كل شيء فوق سائر الأيام
\par 3 ثم تساءل أكثر الأشرار قلةً في النعمة: هل يوجد في السماء قديرٌ أمر بحفظ يوم السبت؟
\par 4 ولما قالوا: يوجد في السماء رب حي قدير، وهو الذي أمر بحفظ اليوم السابع،
\par 5 فقال الآخر: وأنا أيضًا قوي على الأرض، وآمر بأخذ السلاح وتنفيذ أمر الملك. ولكنه لم ينجح في تنفيذ مشيئته الشريرة.
\par 6 لذلك عزم نكانور، في كبريائه وغطرسته الشديدين، على إقامة نصب تذكاري عام لانتصاره على يهوذا والذين كانوا معه
\par 7 لكن المكابي كان لديه ثقة أكيدة بأن الرب سيساعده:
\par 8 ولذلك حث شعبه على عدم الخوف من مجيء الوثنيين ضدهم، بل أن يتذكروا المساعدة التي تلقوها في الأوقات السابقة من السماء، وأن يتوقعوا الآن النصر والمساعدة التي يجب أن تأتي إليهم من القدير.
\par 9 وهكذا، إذ عزاهم بالناموس والأنبياء، وأعاد إليهم تذكيرهم بالمعارك التي انتصروا فيها سابقًا، جعلهم أكثر بهجة
\par 10 ولما أثار عقولهم، وجه إليهم التهمة، مبينًا لهم كذب الوثنيين ونقض الأيمان
\par 11 وهكذا سلّح كل واحد منهم، ليس بالدروع والرماح، بقدر ما سلّحهم بكلمات مريحة وجيدة: وإلى جانب ذلك، أخبرهم بحلم جدير بالتصديق، كما لو كان كذلك بالفعل، مما أسعدهم كثيرًا
\par 12 وكانت هذه رؤياه: أن أونياس، الذي كان رئيس كهنة، رجلاً فاضلاً صالحاً، مهيباً في الحديث، وديعاً في الهيئة، حسن اللسان أيضاً، متمرّساً منذ صغره في جميع جوانب الفضيلة، رافعاً يديه يصلي لأجل كل جماعة اليهود
\par 13 بعد ذلك، ظهر رجل ذو شعر رمادي، ومجيد للغاية، وكان ذا جلال عجيب وممتاز
\par 14 فأجاب أونيا قائلاً: هذا محب للإخوة، يصلي كثيرًا لأجل الشعب ولأجل المدينة المقدسة، وهو إرميا نبي الله
\par 15 عندئذٍ، مدّ إرميا يده اليمنى وأعطى يهوذا سيفًا من ذهب، وعند إعطائه إياه قال:
\par 16 خذ هذا السيف المقدس، هدية من الله، الذي ستجرح به الأعداء
\par 17 وإذ تعزوا بكلمات يهوذا، التي كانت جيدة جدًا، وقادرة على إلهامهم بالشجاعة، وتشجيع قلوب الشباب، قرروا ألا يخيمون، بل أن ينقضوا عليهم بشجاعة، وأن يختبروا الأمر بشجاعة عن طريق الصراع، لأن المدينة والمقدس والهيكل كانوا في خطر
\par 18 لأن العناية التي كانوا يولونها لزوجاتهم وأطفالهم وإخوتهم وأهلهم كانت أقل ما يُذكر لديهم، لكن الخوف الأكبر والرئيسي كان على الهيكل المقدس
\par 19 حتى الذين كانوا في المدينة لم يُبدون أدنى اهتمام، إذ كانوا قلقين بشأن الصراع الدائر في الخارج
\par 20 والآن، عندما نظر الجميع إلى ما يجب أن تكون عليه المحنة، وكان الأعداء قد اقتربوا بالفعل، وتم تنظيم الجيش، ووضع الوحوش في أماكنها المناسبة، ونصب الفرسان أجنحتهم،
\par 21 ولما رأى المكابي مجيء الجموع، وتنوع الأسلحة، وعنف الوحوش، مد يديه نحو السماء، ودعا الرب صانع العجائب، عالماً أن النصر لا يأتي بالسلاح، بل كما يبدو له جيداً، فإنه يعطيه لمن يستحقه.
\par 22 لذلك قال في صلاته هكذا: يا رب، لقد أرسلت ملاكك في زمن حزقيا ملك يهوذا، وقتلت من جيش سنحاريب مئة وخمسة وثمانين ألفًا
\par 23 لذلك الآن أيضًا، يا رب السماء، أرسل ملاكًا صالحًا أمامنا لإثارة الخوف والرعب في نفوسهم؛
\par 24 وبقوة ذراعك، فليُضرب بالرعب أولئك الذين يأتون على شعبك المقدس للتجديف. وانتهى هكذا
\par 25 ثم تقدم نكانور والذين معه بالأبواق والأغاني
\par 26 لكن يهوذا ورفاقه واجهوا الأعداء بالدعاء والصلاة
\par 27 فقاتلوا بأيديهم، وصلّوا إلى الله بقلوبهم، فقتلوا ما لا يقل عن خمسة وثلاثين ألف رجل: لأنهم بظهور الله فرحوا فرحًا عظيمًا
\par 28 عندما انتهت المعركة، عادوا بفرح، وعلموا أن نيكانور يرقد ميتًا في سرجه
\par 29 ثم أحدثوا هتافًا عظيمًا وصوتًا عاليًا، يسبحون القدير بلغتهم
\par 30 ويهوذا، الذي كان دائمًا المدافع الرئيسي عن المواطنين جسديًا ونفسيًا، والذي استمر في حبه لأبناء وطنه طوال حياته، أمر بقطع رأس نيكانور ويده بكتفه، وإحضارهما إلى أورشليم
\par 31 فلما كان هناك، دعا أبناء أمته، وأقام الكهنة أمام المذبح، ثم أرسل إلى الذين في البرج،
\par 32 وأراهم رأس نيكانور الحقير، ويد ذلك المجدف التي مدها بتباهي كبير على هيكل القدير المقدس
\par 33 وبعد أن قطع لسان ذلك الشرير نيكانور، أمر بإعطائه قطعًا للطيور، وتعليق جزاء جنونه أمام الهيكل
\par 34 فسبح كل إنسان نحو السماء الرب المجيد قائلين: مبارك من حفظ مكانه بلا دنس
\par 35 كما علق رأس نيكانور على البرج، علامة واضحة وجلية للجميع على عون الرب
\par 36 ورسموا جميعًا بمرسوم مشترك ألا يمر ذلك اليوم دون احتفال، بل أن يحتفلوا باليوم الثلاثين من الشهر الثاني عشر، الذي يُسمى في اللغة السريانية أدار، أي اليوم السابق لعيد مردخاي
\par 37 وهكذا كان الحال مع نيكانور: ومنذ ذلك الوقت، سيطر العبرانيون على المدينة. وهنا سأنهي الأمر
\par 38 وإذا كنت قد أحسنت، وكما يليق بالقصة، فهذا ما تمنيته، ولكن إذا كان الأمر ضئيلاً وحقيراً، فهذا ما استطعت بلوغه
\par 39 فكما أن شرب الخمر أو الماء وحدهما ضار، وكما أن الخمر ممزوجًا بالماء لذيذٌ ويُبهج الذوق، فكذلك الكلام المُحكم يُبهج آذان قارئي القصة. وهنا تكون النهاية.

\end{document}