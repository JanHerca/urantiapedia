\begin{document}

\title{ايوب}


\chapter{1}

\par 1 كَانَ رَجُلٌ فِي أَرْضِ عُوصَ اسْمُهُ أَيُّوبُ. وَكَانَ هَذَا الرَّجُلُ كَامِلاً وَمُسْتَقِيماً يَتَّقِي اللهَ وَيَحِيدُ عَنِ الشَّرِّ.
\par 2 وَوُلِدَ لَهُ سَبْعَةُ بَنِينَ وَثَلاَثُ بَنَاتٍ.
\par 3 وَكَانَتْ مَوَاشِيهِ سَبْعَةَ آلاَفٍ مِنَ الْغَنَمِ وَثَلاَثَةَ آلاَفِ جَمَلٍ وَخَمْسَ مِئَةِ زَوْجِ بَقَرٍ وَخَمْسَ مِئَةِ أَتَانٍ وَخَدَمُهُ كَثِيرِينَ جِدّاً. فَكَانَ هَذَا الرَّجُلُ أَعْظَمَ كُلِّ بَنِي الْمَشْرِقِ.
\par 4 وَكَانَ بَنُوهُ يَذْهَبُونَ وَيَعْمَلُونَ وَلِيمَةً فِي بَيْتِ كُلِّ وَاحِدٍ مِنْهُمْ فِي يَوْمِهِ وَيُرْسِلُونَ وَيَسْتَدْعُونَ أَخَوَاتِهِمِ الثَّلاَثَ لِيَأْكُلْنَ وَيَشْرَبْنَ مَعَهُمْ.
\par 5 وَكَانَ لَمَّا دَارَتْ أَيَّامُ الْوَلِيمَةِ أَنَّ أَيُّوبَ أَرْسَلَ فَقَدَّسَهُمْ وَبَكَّرَ فِي الْغَدِ وَأَصْعَدَ مُحْرَقَاتٍ عَلَى عَدَدِهِمْ كُلِّهِمْ لأَنَّ أَيُّوبَ قَالَ: [رُبَّمَا أَخْطَأَ بَنِيَّ وَجَدَّفُوا عَلَى اللهِ فِي قُلُوبِهِمْ]. هَكَذَا كَانَ أَيُّوبُ يَفْعَلُ كُلَّ الاَْيَّامِ.
\par 6 وَكَانَ ذَاتَ يَوْمٍ أَنَّهُ جَاءَ بَنُو اللهِ لِيَمْثُلُوا أَمَامَ الرَّبِّ وَجَاءَ الشَّيْطَانُ أَيْضَاً فِي وَسَطِهِمْ.
\par 7 فَقَالَ الرَّبُّ لِلشَّيْطَانِ: [مِنْ أَيْنَ جِئْتَ؟] فَأَجَابَ الشَّيْطَانُ: [مِنْ الْجَوَلاَنِ فِي الأَرْضِ وَمِنَ التَّمَشِّي فِيهَا].
\par 8 فَقَالَ الرَّبُّ لِلشَّيْطَانِ: [هَلْ جَعَلْتَ قَلْبَكَ عَلَى عَبْدِي أَيُّوبَ؟ لأَنَّهُ لَيْسَ مِثْلُهُ فِي الأَرْضِ. رَجُلٌ كَامِلٌ وَمُسْتَقِيمٌ يَتَّقِي اللهَ وَيَحِيدُ عَنِ الشَّرِّ].
\par 9 فَأَجَابَ الشَّيْطَانُ: [هَلْ مَجَّاناً يَتَّقِي أَيُّوبُ اللهَ؟
\par 10 أَلَيْسَ أَنَّكَ سَيَّجْتَ حَوْلَهُ وَحَوْلَ بَيْتِهِ وَحَوْلَ كُلِّ مَا لَهُ مِنْ كُلِّ نَاحِيَةٍ؟ بَارَكْتَ أَعْمَالَ يَدَيْهِ فَانْتَشَرَتْ مَوَاشِيهِ فِي الأَرْضِ!
\par 11 وَلَكِنِ ابْسِطْ يَدَكَ الآنَ وَمَسَّ كُلَّ مَا لَهُ فَإِنَّهُ فِي وَجْهِكَ يُجَدِّفُ عَلَيْكَ].
\par 12 فَقَالَ الرَّبُّ لِلشَّيْطَانِ: [هُوَذَا كُلُّ مَا لَهُ فِي يَدِكَ وَإِنَّمَا إِلَيهِ لاَ تَمُدَّ يَدَكَ]. ثمَّ خَرَجَ الشَّيْطَانُ مِنْ أَمَامِ وَجْهِ الرَّبِّ.
\par 13 وَكَانَ ذَاتَ يَوْمٍ وَأَبْنَاؤُهُ وَبَنَاتُهُ يَأْكُلُونَ وَيَشْرَبُونَ خَمْراً فِي بَيْتِ أَخِيهِمِ الأَكْبَرِ
\par 14 أَنَّ رَسُولاً جَاءَ إِلَى أَيُّوبَ وَقَالَ: [الْبَقَرُ كَانَتْ تَحْرُثُ وَالأُتُنُ تَرْعَى بِجَانِبِهَا
\par 15 فَسَقَطَ عَلَيْهَا السَّبَئِيُّونَ وَأَخَذُوهَا وَضَرَبُوا الْغِلْمَانَ بِحَدِّ السَّيْفِ وَنَجَوْتُ أَنَا وَحْدِي لأُخْبِرَكَ].
\par 16 وَبَيْنَمَا هُوَ يَتَكَلَّمُ إِذْ جَاءَ آخَرُ وَقَالَ: [نَارُ اللهِ سَقَطَتْ مِنَ السَّمَاءِ فَأَحْرَقَتِ الْغَنَمَ وَالْغِلْمَانَ وَأَكَلَتْهُمْ وَنَجَوْتُ أَنَا وَحْدِي لأُخْبِرَكَ].
\par 17 وَبَيْنَمَا هُوَ يَتَكَلَّمُ إِذْ جَاءَ آخَرُ وَقَالَ: [الْكِلْدَانِيُّونَ عَيَّنُوا ثَلاَثَ فِرَقٍ فَهَجَمُوا عَلَى الْجِمَالِ وَأَخَذُوهَا وَضَرَبُوا الْغِلْمَانَ بِحَدِّ السَّيْفِ وَنَجَوْتُ أَنَا وَحْدِي لأُخْبِرَكَ].
\par 18 وَبَيْنَمَا هُوَ يَتَكَلَّمُ إِذْ جَاءَ آخَرُ وَقَالَ: [بَنُوكَ وَبَنَاتُكَ كَانُوا يَأْكُلُونَ وَيَشْرَبُونَ خَمْراً فِي بَيْتِ أَخِيهِمِ الأَكْبَرِ
\par 19 وَإِذَا رِيحٌ شَدِيدَةٌ جَاءَتْ مِنْ عَبْرِ الْقَفْرِ وَصَدَمَتْ زَوَايَا الْبَيْتِ الأَرْبَعَ فَسَقَطَ عَلَى الْغِلْمَانِ فَمَاتُوا وَنَجَوْتُ أَنَا وَحْدِي لأُخْبِرَكَ].
\par 20 فَقَامَ أَيُّوبُ وَمَزَّقَ جُبَّتَهُ وَجَزَّ شَعْرَ رَأْسِهِ وَخَرَّ عَلَى الأَرْضِ وَسَجَدَ
\par 21 وَقَالَ: [عُرْيَاناً خَرَجْتُ مِنْ بَطْنِ أُمِّي وَعُرْيَاناً أَعُودُ إِلَى هُنَاكَ. الرَّبُّ أَعْطَى وَالرَّبُّ أَخَذَ فَلْيَكُنِ اسْمُ الرَّبِّ مُبَارَكاً].
\par 22 فِي كُلِّ هَذَا لَمْ يُخْطِئْ أَيُّوبُ وَلَمْ يَنْسِبْ لِلَّهِ جَهَالَةً.

\chapter{2}

\par 1 وَكَانَ ذَاتَ يَوْمٍ أَنَّهُ جَاءَ بَنُو اللهِ لِيَمْثُلُوا أَمَامَ الرَّبِّ وَجَاءَ الشَّيْطَانُ أَيْضاً فِي وَسَطِهِمْ لِيَمْثُلَ أَمَامَ الرَّبِّ.
\par 2 فَقَالَ الرَّبُّ لِلشَّيْطَانِ: [مِنْ أَيْنَ جِئْتَ؟] فَأَجَابَ الشَّيْطَانُ: [مِنَ الْجَوَلاَنِ فِي الأَرْضِ وَمِنَ التَّمَشِّي فِيهَا].
\par 3 فَقَالَ الرَّبُّ لِلشَّيْطَانِ: [هَلْ جَعَلْتَ قَلْبَكَ عَلَى عَبْدِي أَيُّوبَ لأَنَّهُ لَيْسَ مِثْلُهُ فِي الأَرْضِ! رَجُلٌ كَامِلٌ وَمُسْتَقِيمٌ يَتَّقِي اللهَ وَيَحِيدُ عَنِ الشَّرِّ. وَإِلَى الآنَ هُوَ مُتَمَسِّكٌ بِكَمَالِهِ وَقَدْ هَيَّجْتَنِي عَلَيْهِ لأَبْتَلِعَهُ بِلاَ سَبَبٍ].
\par 4 فَأَجَابَ الشَّيْطَانُ: [جِلْدٌ بِجِلْدٍ وَكُلُّ مَا لِلإِنْسَانِ يُعْطِيهِ لأَجْلِ نَفْسِهِ.
\par 5 وَلَكِنِ ابْسِطِ الآنَ يَدَكَ وَمَسَّ عَظْمَهُ وَلَحْمَهُ فَإِنَّهُ فِي وَجْهِكَ يُجَدِّفُ عَلَيْكَ].
\par 6 فَقَالَ الرَّبُّ لِلشَّيْطَانِ: [هَا هُوَ فِي يَدِكَ وَلَكِنِ احْفَظْ نَفْسَهُ].
\par 7 فَخَرَجَ الشَّيْطَانُ مِنْ حَضْرَةِ الرَّبِّ وَضَرَبَ أَيُّوبَ بِقُرْحٍ رَدِيءٍ مِنْ بَاطِنِ قَدَمِهِ إِلَى هَامَتِهِ.
\par 8 فَأَخَذَ لِنَفْسِهِ شَقْفَةً لِيَحْتَكَّ بِهَا وَهُوَ جَالِسٌ فِي وَسَطِ الرَّمَادِ.
\par 9 فَقَالَتْ لَهُ امْرَأَتُهُ: [أَنْتَ مُتَمَسِّكٌ بَعْدُ بِكَمَالِكَ! جَدِّفْ عَلَى اللهِ وَمُتْ!]
\par 10 فَقَالَ لَهَا: [تَتَكَلَّمِينَ كَلاَماً كَإِحْدَى الْجَاهِلاَتِ! أَالْخَيْرَ نَقْبَلُ مِنْ عِنْدِ اللهِ وَالشَّرَّ لاَ نَقْبَلُ؟] فِي كُلِّ هَذَا لَمْ يُخْطِئْ أَيُّوبُ بِشَفَتَيْهِ.
\par 11 فَلَمَّا سَمِعَ أَصْحَابُ أَيُّوبَ الثَّلاَثَةُ بِكُلِّ الشَّرِّ الَّذِي أَتَى عَلَيْهِ جَاءُوا كُلُّ وَاحِدٍ مِنْ مَكَانِهِ: أَلِيفَازُ التَّيْمَانِيُّ وَبِلْدَدُ الشُّوحِيُّ وَصُوفَرُ النَّعْمَاتِيُّ وَتَوَاعَدُوا أَنْ يَأْتُوا لِيَرْثُوا لَهُ وَيُعَزُّوهُ.
\par 12 وَرَفَعُوا أَعْيُنَهُمْ مِنْ بَعِيدٍ وَلَمْ يَعْرِفُوهُ فَرَفَعُوا أَصْوَاتَهُمْ وَبَكُوا وَمَزَّقَ كُلُّ وَاحِدٍ جُبَّتَهُ وَذَرُّوا تُرَاباً فَوْقَ رُؤُوسِهِمْ نَحْوَ السَّمَاءِ
\par 13 وَقَعَدُوا مَعَهُ عَلَى الأَرْضِ سَبْعَةَ أَيَّامٍ وَسَبْعَ لَيَالٍ وَلَمْ يُكَلِّمْهُ أَحَدٌ بِكَلِمَةٍ لأَنَّهُمْ رَأُوا أَنَّ كَآبَتَهُ كَانَتْ عَظِيمَةً جِدّاً.

\chapter{3}

\par 1 بَعْدَ هَذَا سَبَّ أَيُّوبُ يَوْمَهُ
\par 2 وَأَخَذَ يَتَكَلَّمُ فَقَالَ:
\par 3 [لَيْتَهُ هَلَكَ الْيَوْمُ الَّذِي وُلِدْتُ فِيهِ وَاللَّيْلُ الَّذِي قَالَ قَدْ حُبِلَ بِرَجُلٍ!
\par 4 لِيَكُنْ ذَلِكَ الْيَوْمُ ظَلاَماً. لاَ يَعْتَنِ بِهِ اللهُ مِنْ فَوْقُ وَلاَ يُشْرِقْ عَلَيْهِ نَهَارٌ.
\par 5 لِيَمْلِكْهُ الظَّلاَمُ وَظِلُّ الْمَوْتِ. لِيَحُلَّ عَلَيْهِ سَحَابٌ. لِتُرْعِبْهُ كَاسِفَاتُ النَّهَارِ.
\par 6 أَمَّا ذَلِكَ اللَّيْلُ فَلْيُمْسِكْهُ الدُّجَى وَلاَ يَفْرَحْ بَيْنَ أَيَّامِ السَّنَةِ وَلاَ يَدْخُلَنَّ فِي عَدَدِ الشُّهُورِ.
\par 7 هُوَذَا ذَلِكَ اللَّيْلُ لِيَكُنْ عَاقِراً! لاَ يُسْمَعْ فِيهِ هُتَافٌ.
\par 8 لِيَلْعَنْهُ لاَعِنُو الْيَوْمِ الْمُسْتَعِدُّونَ لإِيقَاظِ التِّنِّينِ.
\par 9 لِتُظْلِمْ نُجُومُ عِشَائِهِ. لِيَنْتَظِرِ النُّورَ وَلاَ يَكُنْ وَلاَ يَرَ هُدْبَ الصُّبْحِ
\par 10 لأَنَّهُ لَمْ يُغْلِقْ أَبْوَابَ بَطْنِ أُمِّي وَلَمْ يَسْتُرِ الشَّقَاوَةَ عَنْ عَيْنَيَّ.
\par 11 لِمَ لَمْ أَمُتْ مِنَ الرَّحِمِ؟ عِنْدَمَا خَرَجْتُ مِنَ الْبَطْنِ لِمَ لَمْ أُسْلِمِ الرُّوحَ؟
\par 12 لِمَاذَا أَعَانَتْنِي الرُّكَبُ وَلِمَ الثُّدِيُّ حَتَّى أَرْضَعَ؟
\par 13 لأَنِّي قَدْ كُنْتُ الآنَ مُضْطَجِعاً سَاكِناً. حِينَئِذٍ كُنْتُ نِمْتُ مُسْتَرِيحاً
\par 14 مَعَ مُلُوكٍ وَمُشِيرِي الأَرْضِ الَّذِينَ بَنُوا أَهْرَاماً لأَنْفُسِهِمْ
\par 15 أَوْ مَعَ رُؤَسَاءَ لَهُمْ ذَهَبٌ الْمَالِئِينَ بُيُوتَهُمْ فِضَّةً
\par 16 أَوْ كَسِقْطٍ مَطْمُورٍ فَلَمْ أَكُنْ كَأَجِنَّةٍ لَمْ يَرُوا نُوراً.
\par 17 هُنَاكَ يَكُفُّ الْمُنَافِقُونَ عَنِ الشَّغَبِ وَهُنَاكَ يَسْتَرِيحُ الْمُتْعَبُون.
\par 18 الأَسْرَى يَطْمَئِنُّونَ جَمِيعاً. لاَ يَسْمَعُونَ صَوْتَ الْمُسَخِّرِ.
\par 19 الصَّغِيرُ كَمَا الْكَبِيرُ هُنَاكَ وَالْعَبْدُ حُرٌّ مِنْ سَيِّدِهِ.
\par 20 [لِمَ يُعْطَى لِشَقِيٍّ نُورٌ وَحَيَاةٌ لِمُرِّي النَّفْسِ؟
\par 21 الَّذِينَ يَنْتَظِرُونَ الْمَوْتَ وَلَيْسَ هُوَ وَيَحْفُرُونَ عَلَيْهِ أَكْثَرَ مِنَ الْكُنُوزِ
\par 22 الْمَسْرُورِينَ إِلَى أَنْ يَبْتَهِجُوا الْفَرِحِينَ عِنْدَمَا يَجِدُونَ قَبْراً.
\par 23 لِرَجُلٍ قَدْ خَفِيَ عَلَيْهِ طَرِيقُهُ وَقَدْ سَيَّجَ اللهُ حَوْلَهُ.
\par 24 لأَنَّهُ مِثْلَ خُبْزِي يَأْتِي أَنِينِي وَمِثْلَ الْمِيَاهِ تَنْسَكِبُ زَفْرَتِي
\par 25 لأَنِّي ارْتِعَاباً ارْتَعَبْتُ فَأَتَانِي وَالَّذِي فَزِعْتُ مِنْهُ جَاءَ عَلَيَّ.
\par 26 لَمْ أَطْمَئِنَّ وَلَمْ أَسْكُنْ وَلَمْ أَسْتَرِحْ وَقَدْ جَاءَ الْغَضَبُ].

\chapter{4}

\par 1 فَأَجَابَ أَلِيفَازُ التَّيْمَانِيُّ:
\par 2 [إِنِ امْتَحَنَ أَحَدٌ كَلِمَةً مَعَكَ فَهَلْ تَسْتَاءُ؟ وَلَكِنْ مَنْ يَسْتَطِيعُ الاِمْتِنَاعَ عَنِ الْكَلاَمِ!
\par 3 هَا أَنْتَ قَدْ أَرْشَدْتَ كَثِيرِينَ وَشَدَّدْتَ أَيَادِيَ مُرْتَخِيَةً.
\par 4 قَدْ أَقَامَ كَلاَمُكَ الْعَاثِرَ وَثَبَّتَّ الرُّكَبَ الْمُرْتَعِشَةَ.
\par 5 وَالآنَ إِذْ جَاءَ عَلَيْكَ ضَجِرْتَ! إِذْ مَسَّكَ ارْتَعْتَ!
\par 6 أَلَيْسَتْ تَقْوَاكَ هِيَ مُعْتَمَدَكَ وَرَجَاؤُكَ كَمَالَ طُرُقِكَ؟
\par 7 اُذْكُرْ مَنْ هَلَكَ وَهُوَ بَرِيءٌ وَأَيْنَ أُبِيدَ الْمُسْتَقِيمُونَ؟
\par 8 كَمَا قَدْ رَأَيْتَ أَنَّ الْحَارِثِينَ إِثْماً وَالزَّارِعِينَ شَقَاوَةً يَحْصُدُونَهُمَا.
\par 9 بِنَسَمَةِ اللهِ يَبِيدُونَ وَبِرِيحِ أَنْفِهِ يَفْنُونَ.
\par 10 زَمْجَرَةُ الأَسَدِ وَصَوْتُ الزَّئِيرِ وَأَنْيَابُ الأَشْبَالِ تَكَسَّرَتْ.
\par 11 اَللَّيْثُ هَالِكٌ لِعَدَمِ الْفَرِيسَةِ وَأَشْبَالُ اللَّبْوَةِ تَبَدَّدَتْ.
\par 12 ثُمَّ إِلَيَّ تَسَلَّلَتْ كَلِمَةٌ فَقَبِلَتْ أُذُنِي مِنْهَا هَمْساً.
\par 13 فِي الْهَوَاجِسِ مِنْ رُؤَى اللَّيْلِ عِنْدَ وُقُوعِ سُبَاتٍ عَلَى النَّاسِ
\par 14 أَصَابَنِي رُعْبٌ وَرَعْدَةٌ فَرَجَفَتْ كُلُّ عِظَامِي.
\par 15 فَمَرَّتْ رُوحٌ عَلَى وَجْهِي. اقْشَعَرَّ شَعْرُ جَسَدِي.
\par 16 وَقَفَتْ وَلَكِنِّي لَمْ أَعْرِفْ مَنْظَرَهَا. شِبْهٌ قُدَّامَ عَيْنَيَّ. سَمِعْتُ صَوْتاً مُنْخَفِضاً:
\par 17 أَالإِنْسَانُ أَبَرُّ مِنَ اللهِ أَمِ الرَّجُلُ أَطْهَرُ مِنْ خَالِقِهِ؟
\par 18 هُوَذَا عَبِيدُهُ لاَ يَأْتَمِنُهُمْ وَإِلَى مَلاَئِكَتِهِ يَنْسِبُ حَمَاقَةً.
\par 19 فَكَمْ بِالْحَرِيِّ سُكَّانُ بُيُوتٍ مِنْ طِينٍ الَّذِينَ أَسَاسُهُمْ فِي التُّرَابِ وَيُسْحَقُونَ مِثْلَ الْعُثِّ؟
\par 20 بَيْنَ الصَّبَاحِ وَالْمَسَاءِ يُحَطَّمُونَ. بِدُونِ مُنْتَبِهٍ إِلَيْهِمْ إِلَى الأَبَدِ يَبِيدُونَ.
\par 21 أَمَا انْتُزِعَتْ حِبَالُ خِيَامِهِمْ؟ يَمُوتُونَ بِلاَ حِكْمَةٍ.

\chapter{5}

\par 1 [اُدْعُ الآنَ. فَهَلْ لَكَ مِنْ مُجِيبٍ! وَإِلَى أَيِّ الْقِدِّيسِينَ تَلْتَفِتُ؟
\par 2 لأَنَّ الْغَيْظَ يَقْتُلُ الْغَبِيَّ وَالْغَيْرَةَ تُمِيتُ الأَحْمَقَ.
\par 3 إِنِّي رَأَيْتُ الْغَبِيَّ يَتَأَصَّلُ وَبَغْتَةً لَعَنْتُ مَرْبِضَهُ.
\par 4 بَنُوهُ بَعِيدُونَ عَنِ الأَمْنِ وَقَدْ تَحَطَّمُوا فِي الْبَابِ وَلاَ مُنْقِذَ.
\par 5 الَّذِينَ يَأْكُلُ الْجَوْعَانُ حَصِيدَهُمْ وَيَأْخُذُهُ حَتَّى مِنَ الشَّوْكِ وَيَشْتَفُّ الظَّمْآنُ ثَرْوَتَهُمْ.
\par 6 إِنَّ الْبَلِيَّةَ لاَ تَخْرُجُ مِنَ التُّرَابِ وَالشَّقَاوَةَ لاَ تَنْبُتُ مِنَ الأَرْضِ
\par 7 وَلَكِنَّ الإِنْسَانَ مَوْلُودٌ لِلْمَشَقَّةِ كَمَا أَنَّ الْجَوَارِحَ لاِرْتِفَاعِ الْجَنَاحِ.
\par 8 [لَكِنْ كُنْتُ أَطْلُبُ إِلَى اللهِ وَعَلَى اللهِ أَجْعَلُ أَمْرِي.
\par 9 الْفَاعِلِ عَظَائِمَ لاَ تُفْحَصُ وَعَجَائِبَ لاَ تُعَدُّ.
\par 10 الْمُنْزِلِ مَطَراً عَلَى وَجْهِ الأَرْضِ وَالْمُرْسِلِ الْمِيَاهَ عَلَى الْبَرَارِيِّ.
\par 11 الْجَاعِلِ الْمُتَوَاضِعِينَ فِي الْعُلَى فَيَرْتَفِعُ الْمَحْزُونُونَ إِلَى أَمْنٍ.
\par 12 الْمُبْطِلِ أَفْكَارَ الْمُحْتَالِينَ فَلاَ تُجْرِي أَيْدِيهِمْ قَصْداً.
\par 13 الآخِذِ الْحُكَمَاءَ بِحِيلَتِهِمْ فَتَتَهَوَّرُ مَشُورَةُ الْمَاكِرِينَ.
\par 14 فِي النَّهَارِ يَصْدِمُونَ ظَلاَماً وَيَتَلَمَّسُونَ فِي الظَّهِيرَةِ كَمَا فِي اللَّيْلِ.
\par 15 الْمُنَجِّيَ الْبَائِسَ مِنَ السَّيْفِ مِنْ فَمِهِمْ وَمِنْ يَدِ الْقَوِيِّ.
\par 16 فَيَكُونُ لِلذَّلِيلِ رَجَاءٌ وَتَسُدُّ الْخَطِيَّةُ فَاهَا.
\par 17 [هُوَذَا طُوبَى لِرَجُلٍ يُؤَدِّبُهُ اللهُ. فَلاَ تَرْفُضْ تَأْدِيبَ الْقَدِيرِ.
\par 18 لأَنَّهُ هُوَ يَجْرَحُ وَيَعْصِبُ. يَسْحَقُ وَيَدَاهُ تَشْفِيَانِ.
\par 19 فِي سِتِّ شَدَائِدَ يُنَجِّيكَ وَفِي سَبْعٍ لاَ يَمَسُّكَ سُوءٌ.
\par 20 فِي الْجُوعِ يَفْدِيكَ مِنَ الْمَوْتِ وَفِي الْحَرْبِ مِنْ حَدِّ السَّيْفِ.
\par 21 مِنْ سَوْطِ اللِّسَانِ تُخْتَبَأُ فَلاَ تَخَافُ مِنَ الْخَرَابِ إِذَا جَاءَ.
\par 22 تَضْحَكُ عَلَى الْخَرَابِ وَالْمَجَاعَةِ وَلاَ تَخْشَى وُحُوشَ الأَرْضِ.
\par 23 لأَنَّهُ مَعَ حِجَارَةِ الْحَقْلِ عَهْدُكَ وَوُحُوشُ الْبَرِّيَّةِ تُسَالِمُكَ.
\par 24 فَتَعْلَمُ أَنَّ خَيْمَتَكَ آمِنَةٌ وَتَتَعَهَّدُ مَرْبِضَكَ وَلاَ تَفْقِدُ شَيْئاً.
\par 25 وَتَعْلَمُ أَنَّ زَرْعَكَ كَثِيرٌ وَذُرِّيَّتَكَ كَعُشْبِ الأَرْضِ.
\par 26 تَدْخُلُ الْمَدْفَنَ فِي شَيْخُوخَةٍ كَرَفْعِ الْكُدْسِ فِي أَوَانِهِ.
\par 27 هَا إِنَّ ذَا قَدْ بَحَثْنَا عَنْهُ. كَذَا هُوَ. فَاسْمَعْهُ وَاعْلَمْ أَنْتَ لِنَفْسِكَ].

\chapter{6}

\par 1 فَقَالَ أَيُّوبُ:
\par 2 [لَيْتَ كَرْبِي وُزِنَ وَمَصِيبَتِي رُفِعَتْ فِي الْمَوَازِينِ جَمِيعَهَا.
\par 3 لأَنَّهَا الآنَ أَثْقَلُ مِنْ رَمْلِ الْبَحْرِ. مِنْ أَجْلِ ذَلِكَ لَغَا كَلاَمِي.
\par 4 لأَنَّ سِهَامَ الْقَدِيرِ فِيَّ تَشْرَبُ رُوحِي سُمَّهَا. أَهْوَالُ اللهِ مُصْطَفَّةٌ ضِدِّي.
\par 5 هَلْ يَنْهَقُ الْفَرَاءُ عَلَى الْعُشْبِ أَوْ يَخُورُ الثَّوْرُ عَلَى عَلَفِهِ؟
\par 6 هَلْ يُؤْكَلُ الْمَسِيخُ بِلاَ مِلْحٍ أَوْ يُوجَدُ طَعْمٌ فِي مَرَقِ الْبَقْلَةِ؟
\par 7 عَافَتْ نَفْسِي أَنْ تَمَسَّهَا فَصَارَتْ خُبْزِيَ الْكَرِيهِ!
\par 8 [يَا لَيْتَ طِلْبَتِي تَأْتِي وَيُعْطِينِيَ اللهُ رَجَائِي!
\par 9 أَنْ يَرْضَى اللهُ بِأَنْ يَسْحَقَنِي وَيُطْلِقَ يَدَهُ فَيَقْطَعَنِي.
\par 10 فَلاَ تَزَالُ تَعْزِيَتِي وَابْتِهَاجِي فِي عَذَابٍ لاَ يُشْفِقُ أَنِّي لَمْ أَجْحَدْ كَلاَمَ الْقُدُّوسِ.
\par 11 مَا هِيَ قُوَّتِي حَتَّى أَنْتَظِرَ وَمَا هِيَ نِهَايَتِي حَتَّى أُصَبِّرَ نَفْسِي؟
\par 12 هَلْ قُوَّتِي قُوَّةُ الْحِجَارَةِ؟ هَلْ لَحْمِي نُحَاسٌ؟
\par 13 أَلاَ إِنَّهُ لَيْسَتْ فِيَّ مَعُونَتِي وَالْمُسَاعَدَةُ مَطْرُودَةٌ عَنِّي!
\par 14 [حَقُّ الْمَحْزُونِ مَعْرُوفٌ مِنْ صَاحِبِهِ وَإِنْ تَرَكَ خَشْيَةَ الْقَدِيرِ.
\par 15 أَمَّا إِخْوَانِي فَقَدْ غَدَرُوا مِثْلَ الْغَدِيرِ. مِثْلَ سَاقِيَةِ الْوِدْيَانِ يَعْبُرُونَ.
\par 16 الَّتِي هِيَ عَكِرَةٌ مِنَ الْبَرَدِ وَيَخْتَفِي فِيهَا الْجَلِيدُ.
\par 17 إِذَا جَرَتِ انْقَطَعَتْ. إِذَا حَمِيَتْ جَفَّتْ مِنْ مَكَانِهَا.
\par 18 تَحِيدُ الْقَوَافِلُ عَنْ طَرِيقِهَا تَدْخُلُ التِّيهَ فَتَهْلِكُ.
\par 19 نَظَرَتْ قَوَافِلُ تَيْمَاءَ. مَوَاكِبُ سَبَأٍ رَجَوْهَا.
\par 20 خَزُوا فِي مَا كَانُوا مُطْمَئِنِّينَ. جَاءُوا إِلَيْهَا فَخَجِلُوا.
\par 21 فَالآنَ قَدْ صِرْتُمْ مِثْلَهَا. رَأَيْتُمْ ضَرْبَةً فَفَزِعْتُمْ.
\par 22 هَلْ قُلْتُ: أَعْطُونِي شَيْئاً أَوْ مِنْ مَالِكُمُ ارْشُوا مِنْ أَجْلِي
\par 23 أَوْ نَجُّونِي مِنْ يَدِ الْخَصْمِ أَوْ مِنْ يَدِ الْعُتَاةِ افْدُونِي؟
\par 24 عَلِّمُونِي فَأَنَا أَسْكُتُ وَفَهِّمُونِي فِي أَيِّ شَيْءٍ ضَلَلْتُ.
\par 25 مَا أَشَدَّ الْكَلاَمَ الْمُسْتَقِيمَ وَأَمَّا التَّوْبِيخُ مِنْكُمْ فَعَلَى مَاذَا يُبَرْهِنُ؟
\par 26 هَلْ تَحْسِبُونَ أَنْ تُوَبِّخُوا كَلِمَاتٍ وَكَلاَمُ الْيَائِسِ لِلرِّيحِ!
\par 27 بَلْ تُلْقُونَ عَلَى الْيَتِيمِ وَتَحْفُرُونَ حُفْرَةً لِصَاحِبِكُمْ!
\par 28 وَالآنَ تَفَرَّسُوا فِيَّ فَإِنِّي عَلَى وُجُوهِكُمْ لاَ أَكْذِبُ.
\par 29 اِرْجِعُوا. لاَ يَكُونَنَّ ظُلْمٌ. ارْجِعُوا أَيْضاً. فِيهِ حَقِّي.
\par 30 هَلْ فِي لِسَانِي ظُلْمٌ أَمْ حَنَكِي لاَ يُمَيِّزُ فَسَاداً؟

\chapter{7}

\par 1 [أَلَيْسَتْ حَيَاةُ الإِنْسَانِ جِهَاداً عَلَى الأَرْضِ وَكَأَيَّامِ الأَجِيرِ أَيَّامُهُ؟
\par 2 كَمَا يَتَشَوَّقُ الْعَبْدُ إِلَى الظِّلِّ وَكَمَا يَتَرَجَّى الأَجِيرُ أُجْرَتَهُ
\par 3 هَكَذَا تَعَيَّنَ لِي أَشْهُرُ سُوءٍ وَلَيَالِي شَقَاءٍ قُسِمَتْ لِي.
\par 4 إِذَا اضْطَجَعْتُ أَقُولُ مَتَى أَقُومُ. اللَّيْلُ يَطُولُ وَأَشْبَعُ قَلَقاً حَتَّى الصُّبْحِ.
\par 5 لَبِسَ لَحْمِيَ الدُّودُ مَعَ الطِّينِ. جِلْدِي تَشَقَّقَ وَتَقَيَّحَ.
\par 6 أَيَّامِي أَسْرَعُ مِنَ الْمَكُّوكِ وَتَنْتَهِي بِغَيْرِ رَجَاءٍ.
\par 7 [اُذْكُرْ أَنَّ حَيَاتِي إِنَّمَا هِيَ رِيحٌ وَعَيْنِي لاَ تَعُودُ تَرَى خَيْراً.
\par 8 لاَ تَرَانِي عَيْنُ نَاظِرِي. عَيْنَاكَ عَلَيَّ وَلَسْتُ أَنَا!
\par 9 السَّحَابُ يَضْمَحِلُّ وَيَزُولُ. هَكَذَا الَّذِي يَنْزِلُ إِلَى الْهَاوِيَةِ لاَ يَصْعَدُ.
\par 10 لاَ يَرْجِعُ بَعْدُ إِلَى بَيْتِهِ وَلاَ يَعْرِفُهُ مَكَانُهُ بَعْدُ.
\par 11 أَنَا أَيْضاً لاَ أَمْنَعُ فَمِي. أَتَكَلَّمُ بِضِيقِ رُوحِي. أَشْكُو بِمَرَارَةِ نَفْسِي.
\par 12 أَبَحْرٌ أَنَا أَمْ تِنِّينٌ حَتَّى جَعَلْتَ عَلَيَّ حَارِساً؟
\par 13 إِنْ قُلْتُ: فِرَاشِي يُعَزِّينِي مَضْجَعِي يَنْزِعُ كُرْبَتِي
\par 14 تُرِيعُنِي بِالأَحْلاَمِ وَتُرْهِبُنِي بِرُؤًى
\par 15 فَاخْتَارَتْ نَفْسِي الْخَنْقَ وَالْمَوْتَ عَلَى عِظَامِي هَذِهِ.
\par 16 قَدْ ذُبْتُ. لاَ إِلَى الأَبَدِ أَحْيَا. كُفَّ عَنِّي لأَنَّ أَيَّامِي نَفْخَةٌ!
\par 17 مَا هُوَ الإِنْسَانُ حَتَّى تَعْتَبِرَهُ وَحَتَّى تَضَعَ عَلَيْهِ قَلْبَكَ
\par 18 وَتَتَعَهَّدَهُ كُلَّ صَبَاحٍ وَكُلَّ لَحْظَةٍ تَمْتَحِنُهُ!
\par 19 حَتَّى مَتَى لاَ تَلْتَفِتُ عَنِّي وَلاَ تُرْخِينِي رَيْثَمَا أَبْلَعُ رِيقِي؟
\par 20 أَأَخْطَأْتُ؟ مَاذَا أَفْعَلُ لَكَ يَا رَقِيبَ النَّاسِ! لِمَاذَا جَعَلْتَنِي هَدَفاً لَكَ حَتَّى أَكُونَ عَلَى نَفْسِي حِمْلاً!
\par 21 وَلِمَاذَا لاَ تَغْفِرُ ذَنْبِي وَلاَ تُزِيلُ إِثْمِي لأَنِّي الآنَ أَضْطَجِعُ فِي التُّرَابِ؟ تَطْلُبُنِي فَلاَ أَكُونُ!].

\chapter{8}

\par 1 فَأَجَابَ بِلْدَدُ الشُّوحِيُّ:
\par 2 [إِلَى مَتَى تَقُولُ هَذَا وَتَكُونُ أَقْوَالُكَ رِيحاً شَدِيدَةً!
\par 3 هَلِ اللهُ يُعَوِّجُ الْقَضَاءَ أَوِ الْقَدِيرُ يَعْكِسُ الْحَقَّ؟
\par 4 إِذْ أَخْطَأَ إِلَيْهِ بَنُوكَ دَفَعَهُمْ إِلَى يَدِ مَعْصِيَتِهِمْ.
\par 5 فَإِنْ بَكَّرْتَ أَنْتَ إِلَى اللهِ وَتَضَرَّعْتَ إِلَى الْقَدِيرِ -
\par 6 إِنْ كُنْتَ أَنْتَ زَكِيّاً مُسْتَقِيماً فَإِنَّهُ الآنَ يَتَنَبَّهُ لَكَ وَيُسْلِمُ مَسْكَنَ بِرِّكَ.
\par 7 وَإِنْ تَكُنْ أُولاَكَ صَغِيرَةً فَآخِرَتُكَ تَكْثُرُ جِدّاً.
\par 8 [اِسْأَلِ الْقُرُونَ الأُولَى وَتَأَكَّدْ مَبَاحِثَ آبَائِهِمْ.
\par 9 لأَنَّنَا نَحْنُ مِنْ أَمْسٍ وَلاَ نَعْلَمُ لأَنَّ أَيَّامَنَا عَلَى الأَرْضِ ظِلٌّ.
\par 10 فَهَلاَّ يُعْلِمُونَكَ. يَقُولُونَ لَكَ وَمِنْ قُلُوبِهِمْ يُخْرِجُونَ أَقْوَالاً قَائِلِينَ
\par 11 هَلْ يَنْمُو الْبَرْدِيُّ فِي غَيْرِ الْمُسْتَنْقَعِ أَوْ تَنْبُتُ الْحَلْفَاءُ بِلاَ مَاءٍ؟
\par 12 وَهُوَ بَعْدُ فِي نَضَارَتِهِ لَمْ يُقْطَعْ يَيْبَسُ قَبْلَ كُلِّ الْعُشْبِ.
\par 13 هَكَذَا سُبُلُ كُلِّ النَّاسِينَ اللهَ وَرَجَاءُ الْفَاجِرِ يَخِيبُ
\par 14 فَيَنْقَطِعُ اعْتِمَادُهُ وَمُتَّكَلُهُ بَيْتُ الْعَنْكَبُوتِ!
\par 15 يَسْتَنِدُ إِلَى بَيْتِهِ فَلاَ يَثْبُتُ. يَتَمَسَّكُ بِهِ فَلاَ يَقُومُ.
\par 16 هُوَ رَطْبٌ تُجَاهَ الشَّمْسِ وَعَلَى جَنَّتِهِ تَنْبُتُ أَغْصَانُهُ.
\par 17 وَأُصُولُهُ مُشْتَبِكَةٌ فِي الرُّجْمَةِ فَتَرَى مَحَلَّ الْحِجَارَةِ.
\par 18 إِنِ اقْتَلَعَهُ مِنْ مَكَانِهِ يَجْحَدُهُ قَائِلاً: مَا رَأَيْتُكَ.
\par 19 هَذَا هُوَ فَرَحُ طَرِيقِهِ وَمِنَ التُّرَابِ يَنْبُتُ آخَرُ.
\par 20 [هُوَذَا اللهُ لاَ يَرْفُضُ الْكَامِلَ وَلاَ يَأْخُذُ بِيَدِ فَاعِلِي الشَّرِّ.
\par 21 عِنْدَمَا يَمْلَأُ فَمَكَ ضَحِكاً وَشَفَتَيْكَ هُتَافاً
\par 22 يَلْبِسُ مُبْغِضُوكَ خِزْياً. أَمَّا خَيْمَةُ الأَشْرَارِ فَلاَ تَكُونُ].

\chapter{9}

\par 1 فَقَالَ أَيُّوبُ:
\par 2 [صَحِيحٌ. قَدْ عَلِمْتُ أَنَّهُ كَذَا. فَكَيْفَ يَتَبَرَّرُ الإِنْسَانُ عِنْدَ اللهِ؟
\par 3 إِنْ شَاءَ أَنْ يُحَاجَّهُ لاَ يُجِيبُهُ عَنْ وَاحِدٍ مِنْ أَلْفٍ.
\par 4 هُوَ حَكِيمُ الْقَلْبِ وَشَدِيدُ الْقُوَّةِ. مَنْ تَصَلَّبَ عَلَيْهِ فَسَلِمَ؟
\par 5 الْمُزَحْزِحُ الْجِبَالَ وَلاَ تَعْلَمُ. الَّذِي يَقْلِبُهَا فِي غَضَبِهِ
\par 6 الْمُزَعْزِعُ الأَرْضَ مِنْ مَقَرِّهَا فَتَتَزَلْزَلُ أَعْمِدَتُهَا
\par 7 الآمِرُ الشَّمْسَ فَلاَ تُشْرِقُ وَيَخْتِمُ عَلَى النُّجُومِ.
\par 8 الْبَاسِطُ السَّمَاوَاتِ وَحْدَهُ وَالْمَاشِي عَلَى أَعَالِي الْبَحْرِ.
\par 9 صَانِعُ النَّعْشِ وَالْجَبَّارِ وَالثُّرَيَّا وَمَخَادِعِ الْجَنُوبِ.
\par 10 فَاعِلُ عَظَائِمَ لاَ تُفْحَصُ وَعَجَائِبَ لاَ تُعَدُّ.
\par 11 [هُوَذَا يَمُرُّ عَلَيَّ وَلاَ أَرَاهُ وَيَجْتَازُ فَلاَ أَشْعُرُ بِهِ.
\par 12 إِذَا خَطَفَ فَمَنْ يَرُدُّهُ وَمَنْ يَقُولُ لَهُ: مَاذَا تَفْعَلُ؟
\par 13 اللهُ لاَ يَرُدُّ غَضَبَهُ. يَنْحَنِي تَحْتَهُ أَعْوَانُ رَهَبَ.
\par 14 كَمْ بِالأَقَلِّ أَنَا أُجَاوِبُهُ وَأَخْتَارُ كَلاَمِي مَعَهُ.
\par 15 لأَنِّي وَإِنْ تَبَرَّرْتُ لاَ أُجَاوِبُ بَلْ أَسْتَرْحِمُ دَيَّانِي.
\par 16 لَوْ دَعَوْتُ فَاسْتَجَابَ لِي لَمَا آمَنْتُ بِأَنَّهُ سَمِعَ صَوْتِي.
\par 17 ذَاكَ الَّذِي يَسْحَقُنِي بِالْعَاصِفَةِ وَيُكْثِرُ جُرُوحِي بِلاَ سَبَبٍ.
\par 18 لاَ يَدَعُنِي آخُذُ نَفَسِي وَلَكِنْ يُشْبِعُنِي مَرَائِرَ.
\par 19 إِنْ كَانَ مِنْ جِهَةِ قُوَّةِ الْقَوِيِّ يَقُولُ: هَئَنَذَا. وَإِنْ كَانَ مِنْ جِهَةِ الْقَضَاءِ يَقُولُ: مَنْ يُحَاكِمُنِي؟
\par 20 إِنْ تَبَرَّرْتُ يَحْكُمُ عَلَيَّ فَمِي؟ وَإِنْ كُنْتُ كَامِلاً يَسْتَذْنِبُنِي.
\par 21 [كَامِلٌ أَنَا. لاَ أُبَالِي بِنَفْسِي. رَذَلْتُ حَيَاتِي.
\par 22 هِيَ وَاحِدَةٌ. لِذَلِكَ قُلْتُ إِنَّ الْكَامِلَ وَالشِّرِّيرَ هُوَ يُفْنِيهِمَا.
\par 23 إِذَا قَتَلَ السَّوْطُ بَغْتَةً يَسْتَهْزِئُ بِتَجْرِبَةِ الأَبْرِيَاءِ.
\par 24 الأَرْضُ مُسَلَّمَةٌ لِيَدِ الشِّرِّيرِ. يُغَشِّي وُجُوهَ قُضَاتِهَا. وَإِنْ لَمْ يَكُنْ هُوَ فَإِذاً مَنْ؟
\par 25 أَيَّامِي أَسْرَعُ مِنْ عَدَّاءٍ تَفِرُّ وَلاَ تَرَى خَيْراً.
\par 26 تَمُرُّ مَعَ سُفُنِ الْبَرْدِيِّ. كَنَسْرٍ يَنْقَضُّ إِلَى صَيْدِهِ.
\par 27 إِنْ قُلْتُ: أَنْسَى كُرْبَتِي. أُطْلِقُ وَجْهِي وَأَبْتَسِمُ
\par 28 أَخَافُ مِنْ كُلِّ أَوْجَاعِي عَالِماً أَنَّكَ لاَ تُبَرِّئُنِي.
\par 29 أَنَا مُسْتَذْنَبٌ فَلِمَاذَا أَتْعَبُ عَبَثاً؟
\par 30 وَلَوِ اغْتَسَلْتُ فِي الثَّلْجِ وَنَظَّفْتُ يَدَيَّ بِالأَشْنَانِ
\par 31 فَإِنَّكَ فِي النَّقْعِ تَغْمِسُنِي حَتَّى تَكْرَهَنِي ثِيَابِي.
\par 32 لأَنَّهُ لَيْسَ هُوَ إِنْسَاناً مِثْلِي فَأُجَاوِبَهُ فَنَأْتِي جَمِيعاً إِلَى الْمُحَاكَمَةِ.
\par 33 لَيْسَ بَيْنَنَا مُصَالِحٌ يَضَعُ يَدَهُ عَلَى كِلَيْنَا!
\par 34 لِيَرْفَعْ عَنِّي عَصَاهُ وَلاَ يَبْغَتْنِي رُعْبُهُ.
\par 35 إِذاً أَتَكَلَّمُ وَلاَ أَخَافُهُ. لأَنِّي لَسْتُ هَكَذَا عِنْدَ نَفْسِي.

\chapter{10}

\par 1 [قَدْ كَرِهَتْ نَفْسِي حَيَاتِي. أُسَيِّبُ شَكْوَايَ. أَتَكَلَّمُ فِي مَرَارَةِ نَفْسِي
\par 2 قَائِلاً لِلَّهِ: لاَ تَسْتَذْنِبْنِي. فَهِّمْنِي لِمَاذَا تُخَاصِمُنِي!
\par 3 أَحَسَنٌ عِنْدَكَ أَنْ تَظْلِمَ أَنْ تَرْذُلَ عَمَلَ يَدَيْكَ وَتُشْرِقَ عَلَى مَشُورَةِ الأَشْرَارِ؟
\par 4 أَلَكَ عَيْنَا بَشَرٍ أَمْ كَنَظَرِ الإِنْسَانِ تَنْظُرُ؟
\par 5 أَأَيَّامُكَ كَأَيَّامِ الإِنْسَانِ أَمْ سِنُوكَ كَأَيَّامِ الرَّجُلِ
\par 6 حَتَّى تَبْحَثَ عَنْ إِثْمِي وَتُفَتِّشَ عَلَى خَطِيَّتِي؟
\par 7 فِي عِلْمِكَ أَنِّي لَسْتُ مُذْنِباً وَلاَ مُنْقِذَ مِنْ يَدِكَ.
\par 8 [يَدَاكَ كَوَّنَتَانِي وَصَنَعَتَانِي كُلِّي جَمِيعاً. أَفَتَبْتَلِعُنِي؟
\par 9 اُذْكُرْ أَنَّكَ جَبَلْتَنِي كَالطِّينِ. أَفَتُعِيدُنِي إِلَى التُّرَابِ؟
\par 10 أَلَمْ تَصُبَّنِي كَاللَّبَنِ وَخَثَّرْتَنِي كَالْجُبْنِ؟
\par 11 كَسَوْتَنِي جِلْداً وَلَحْماً فَنَسَجْتَنِي بِعِظَامٍ وَعَصَبٍ.
\par 12 مَنَحْتَنِي حَيَاةً وَرَحْمَةً وَحَفِظَتْ عِنَايَتُكَ رُوحِي.
\par 13 لَكِنَّكَ كَتَمْتَ هَذِهِ فِي قَلْبِكَ. عَلِمْتُ أَنَّ هَذَا عِنْدَكَ.
\par 14 إِنْ أَخْطَأْتُ تُلاَحِظُنِي وَلاَ تُبْرِئُنِي مِنْ إِثْمِي.
\par 15 إِنْ أَذْنَبْتُ فَوَيْلٌ لِي. وَإِنْ تَبَرَّرْتُ لاَ أَرْفَعُ رَأْسِي. إِنِّي شَبْعَانُ هَوَاناً وَنَاظِرٌ مَذَلَّتِي.
\par 16 وَإِنِ ارْتَفَعَ رَأْسِي تَصْطَادُنِي كَأَسَدٍ ثُمَّ تَعُودُ وَتَتَجَبَّرُ عَلَيَّ!
\par 17 تُجَدِّدُ شُهُودَكَ تُجَاهِي وَتَزِيدُ غَضَبَكَ عَلَيَّ. مَصَائِبُ وَجَيْشٌ ضِدِّي.
\par 18 [فَلِمَاذَا أَخْرَجْتَنِي مِنَ الرَّحِمِ؟ كُنْتُ قَدْ أَسْلَمْتُ الرُّوحَ وَلَمْ تَرَنِي عَيْنٌ!
\par 19 فَكُنْتُ كَأَنِّي لَمْ أَكُنْ فَأُقَادَ مِنَ الرَّحِمِ إِلَى الْقَبْرِ.
\par 20 أَلَيْسَتْ أَيَّامِي قَلِيلَةً؟ اتْرُكْ! كُفَّ عَنِّي فَأَبْتَسِمُ قَلِيلاً
\par 21 قَبْلَ أَنْ أَذْهَبَ وَلاَ أَعُودَ. إِلَى أَرْضِ ظُلْمَةٍ وَظِلِّ الْمَوْتِ
\par 22 أَرْضِ ظَلاَمٍ مِثْلِ دُجَى ظِلِّ الْمَوْتِ وَبِلاَ تَرْتِيبٍ وَإِشْرَاقُهَا كَالدُّجَى].

\chapter{11}

\par 1 فَأَجَابَ صُوفَرُ النَّعْمَاتِيُّ:
\par 2 [أَكَثْرَةُ الْكَلاَمِ لاَ يُجَاوَبُ أَمْ رَجُلٌ مِهْذَارٌ يَتَبَرَّرُ؟
\par 3 أَصَلَفُكَ يُفْحِمُ النَّاسَ أَمْ تَلْغُو وَلَيْسَ مَنْ يُخْزِيكَ؟
\par 4 إِذْ تَقُولُ: تَعْلِيمِي زَكِيٌّ وَأَنَا بَارٌّ فِي عَيْنَيْكَ.
\par 5 وَلَكِنْ يَا لَيْتَ اللهَ يَتَكَلَّمُ وَيَفْتَحُ شَفَتَيْهِ مَعَكَ
\par 6 وَيُعْلِنُ لَكَ خَفِيَّاتِ الْحِكْمَةِ! إِنَّهَا مُضَاعَفَةُ الْفَهْمِ فَتَعْلَمَ أَنَّ اللهَ يُغَرِّمُكَ بِأَقَلَّ مِنْ إِثْمِكَ.
\par 7 [أَإِلَى عُمْقِ اللهِ تَتَّصِلُ أَمْ إِلَى نِهَايَةِ الْقَدِيرِ تَنْتَهِي؟
\par 8 هُوَ أَعْلَى مِنَ السَّمَاوَاتِ فَمَاذَا عَسَاكَ أَنْ تَفْعَلَ؟ أَعْمَقُ مِنَ الْهَاوِيَةِ فَمَاذَا تَدْرِي؟
\par 9 أَطْوَلُ مِنَ الأَرْضِ طُولُهُ وَأَعْرَضُ مِنَ الْبَحْرِ.
\par 10 إِنْ بَطَشَ أَوْ أَغْلَقَ أَوْ جَمَّعَ فَمَنْ يَرُدُّهُ؟
\par 11 لأَنَّهُ هُوَ يَعْلَمُ أُنَاسَ السُّوءِ وَيُبْصِرُ الإِثْمَ فَهَلْ لاَ يَنْتَبِهُ؟
\par 12 أَمَّا الرَّجُلُ فَفَارِغٌ عَدِيمُ الْفَهْمِ وَكَجَحْشِ الْفَرَا يُولَدُ الإِنْسَانُ.
\par 13 [إِنْ أَعْدَدْتَ أَنْتَ قَلْبَكَ وَبَسَطْتَ إِلَيْهِ يَدَيْكَ.
\par 14 إِنْ أَبْعَدْتَ الإِثْمَ الَّذِي فِي يَدِكَ وَلاَ يَسْكُنُ الظُّلْمُ فِي خَيْمَتِكَ
\par 15 حِينَئِذٍ تَرْفَعُ وَجْهَكَ بِلاَ عَيْبٍ وَتَكُونُ ثَابِتاً وَلاَ تَخَافُ.
\par 16 لأَنَّكَ تَنْسَى الْمَشَقَّةَ. كَمِيَاهٍ عَبَرَتْ تَذْكُرُهَا.
\par 17 وَفَوْقَ الظَّهِيرَةِ يَقُومُ حَظُّكَ. الظَّلاَمُ يَتَحَوَّلُ صَبَاحاً.
\par 18 وَتَطْمَئِنُّ لأَنَّهُ يُوجَدُ رَجَاءٌ. تَتَجَسَّسُ حَوْلَكَ وَتَضْطَجِعُ آمِناً.
\par 19 وَتَرْبِضُ وَلَيْسَ مَنْ يُزْعِجُ وَيَتَضَرَّعُ إِلَى وَجْهِكَ كَثِيرُونَ.
\par 20 أَمَّا عُيُونُ الأَشْرَارِ فَتَتْلَفُ وَمَلْجَأُهُمْ يَبِيدُ وَرَجَاؤُهُمْ تَسْلِيمُ النَّفْسِ].

\chapter{12}

\par 1 فَقَالَ أَيُّوبُ:
\par 2 [صَحِيحٌ إِنَّكُمْ أَنْتُمْ شَعْبٌ وَمَعَكُمْ تَمُوتُ الْحِكْمَةُ!
\par 3 غَيْرَ أَنَّهُ لِي فَهْمٌ مِثْلَكُمْ. لَسْتُ أَنَا دُونَكُمْ. وَمَنْ لَيْسَ عِنْدَهُ مِثْلُ هَذِهِ؟
\par 4 رَجُلاً أُضْحُوكَةٌ لِصَاحِبِهِ صِرْتُ. دَعَا اللهَ فَاسْتَجَابَهُ. أُضْحُوكَةٌ هُوَ الصِّدِّيقُ الْكَامِلُ.
\par 5 لِلْمُبْتَلِي هَوَانٌ فِي أَفْكَارِ الْمُطْمَئِنِّ مُهَيَّأٌ لِمَنْ زَلَّتْ قَدَمُهُ.
\par 6 خِيَامُ الْمُخَرِّبِينَ مُسْتَرِيحَةٌ وَالَّذِينَ يُغِيظُونَ اللهَ مُطْمَئِنُّونَ الَّذِينَ يَأْتُونَ بِإِلَهِهِمْ فِي يَدِهِمْ!
\par 7 [فَاسْأَلِ الْبَهَائِمَ فَتُعَلِّمَكَ وَطُيُورَ السَّمَاءِ فَتُخْبِرَكَ.
\par 8 أَوْ كَلِّمِ الأَرْضَ فَتُعَلِّمَكَ وَيُحَدِّثَكَ سَمَكُ الْبَحْرِ.
\par 9 مَنْ لاَ يَعْلَمُ مِنْ كُلِّ هَؤُلاَءِ أَنَّ يَدَ الرَّبِّ صَنَعَتْ هَذَا!
\par 10 الَّذِي بِيَدِهِ نَفَسُ كُلِّ حَيٍّ وَرُوحُ كُلِّ الْبَشَرِ.
\par 11 أَفَلَيْسَتِ الأُذُنُ تَمْتَحِنُ الأَقْوَالَ كَمَا أَنَّ الْحَنَكَ يَسْتَطْعِمُ طَعَامَهُ؟
\par 12 عِنْدَ الشَّيْبِ حِكْمَةٌ وَطُولُ الأَيَّامِ فَهْمٌ.
\par 13 [عِنْدَهُ الْحِكْمَةُ وَالْقُدْرَةُ. لَهُ الْمَشُورَةُ وَالْفِطْنَةُ.
\par 14 هُوَذَا يَهْدِمُ فَلاَ يُبْنَى. يُغْلِقُ عَلَى إِنْسَانٍ فَلاَ يُفْتَحُ.
\par 15 يَمْنَعُ الْمِيَاهَ فَتَيْبَسُ. يُطْلِقُهَا فَتَقْلِبُ الأَرْضَ.
\par 16 عِنْدَهُ الْعِزُّ وَالْفَهْمُ. لَهُ الْمُضِلُّ وَالْمُضَلُّ.
\par 17 يَذْهَبُ بِالْمُشِيرِينَ أَسْرَى وَيُحَمِّقُ الْقُضَاةَ.
\par 18 يَحُلُّ مَنَاطِقَ الْمُلُوكِ وَيَشُدُّ أَحْقَاءَهُمْ بِوِثَاقٍ.
\par 19 يَذْهَبُ بِالْكَهَنَةِ أَسْرَى وَيَقْلِبُ الأَقْوِيَاءَ.
\par 20 يَقْطَعُ كَلاَمَ الأُمَنَاءِ وَيَنْزِعُ ذَوْقَ الشُّيُوخِ.
\par 21 يُلْقِي هَوَاناً عَلَى الشُّرَفَاءِ وَيُرْخِي مِنْطَقَةَ الأَشِدَّاءِ.
\par 22 يَكْشِفُ الْعَمَائِقَ مِنَ الظَّلاَمِ وَيُخْرِجُ ظِلَّ الْمَوْتِ إِلَى النُّورِ.
\par 23 يُكَثِّرُ الأُمَمَ ثُمَّ يُبِيدُهَا. يُوَسِّعُ لِلأُمَمِ ثُمَّ يُشَتِّتُها.
\par 24 يَنْزِعُ عُقُولَ رُؤَسَاءِ شَعْبِ الأَرْضِ وَيُضِلُّهُمْ فِي تِيهٍ بِلاَ طَرِيقٍ.
\par 25 يَتَلَمَّسُونَ فِي الظَّلاَمِ وَلَيْسَ نُورٌ وَيُرَنِّحُهُمْ مِثْلَ السَّكْرَانِ.

\chapter{13}

\par 1 [هَذَا كُلُّهُ رَأَتْهُ عَيْنِي. سَمِعَتْهُ أُذُنِي وَفَطِنَتْ بِهِ.
\par 2 مَا تَعْرِفُونَهُ عَرَفْتُهُ أَنَا أَيْضاً. لَسْتُ دُونَكُمْ.
\par 3 وَلَكِنِّي أُرِيدُ أَنْ أُكَلِّمَ الْقَدِيرَ وَأَنْ أُحَاكَمَ إِلَى اللهِ.
\par 4 أَمَّا أَنْتُمْ فَمُلَفِّقُو كَذِبٍ. أَطِبَّاءُ بَطَّالُونَ كُلُّكُمْ.
\par 5 لَيْتَكُمْ تَصْمُتُونَ صَمْتاً. يَكُونُ ذَلِكَ لَكُمْ حِكْمَةً.
\par 6 اِسْمَعُوا الآنَ حُجَّتِي وَاصْغُوا إِلَى دَعَاوِي شَفَتَيَّ.
\par 7 أَتَقُولُونَ لأَجْلِ اللهِ ظُلْماً وَتَتَكَلَّمُونَ بِغِشٍّ لأَجْلِهِ؟
\par 8 أَتُحَابُونَ وَجْهَهُ أَمْ عَنِ اللهِ تُخَاصِمُونَ؟
\par 9 أَخَيْرٌ لَكُمْ أَنْ يَفْحَصَكُمْ أَمْ تُخَاتِلُونَهُ كَمَا يُخَاتَلُ الإِنْسَانُ؟
\par 10 تَوْبِيخاً يُوَبِّخُكُمْ إِنْ حَابَيْتُمُ الْوُجُوهَ خِفْيَةً.
\par 11 فَهَلاَّ يُرْهِبُكُمْ جَلاَلُهُ وَيَسْقُطُ عَلَيْكُمْ رُعْبُهُ!
\par 12 خُطَبُكُمْ أَمْثَالُ رَمَادٍ وَحُصُونُكُمْ حُصُونٌ مِنْ طِينٍ!
\par 13 [اُسْكُتُوا عَنِّي فَأَتَكَلَّمَ أَنَا وَلْيُصِبْنِي مَهْمَا أَصَابَ.
\par 14 لِمَاذَا آخُذُ لَحْمِي بِأَسْنَانِي وَأَضَعُ نَفْسِي فِي كَفِّي؟
\par 15 هُوَذَا يَقْتُلُنِي. لاَ أَنْتَظِرُ شَيْئاً. فَقَطْ أُزَكِّي طَرِيقِي قُدَّامَهُ.
\par 16 فَهَذَا يَعُودُ إِلَى خَلاَصِي أَنَّ الْفَاجِرَ لاَ يَأْتِي قُدَّامَهُ.
\par 17 سَمْعاً اسْمَعُوا أَقْوَالِي وَتَصْرِيحِي بِمَسَامِعِكُمْ.
\par 18 هَئَنَذَا قَدْ أَحْسَنْتُ الدَّعْوَى. أَعْلَمُ أَنِّي أَتَبَرَّرُ.
\par 19 مَنْ هُوَ الَّذِي يُخَاصِمُنِي حَتَّى أَصْمُتَ الآنَ وَأُسْلِمَ الرُّوحَ؟
\par 20 إِنَّمَا أَمْرَيْنِ لاَ تَفْعَلْ بِي فَحِينَئِذٍ لاَ أَخْتَفِي مِنْ حَضْرَتِكَ.
\par 21 أَبْعِدْ يَدَيْكَ عَنِّي وَلاَ تَدَعْ هَيْبَتَكَ تُرْعِبُنِي
\par 22 ثُمَّ ادْعُ فَأَنَا أُجِيبُ أَوْ أَتَكَلَّمُ فَتُجَاوِبُنِي.
\par 23 كَمْ لِي مِنَ الآثَامِ وَالْخَطَايَا. أَعْلِمْنِي ذَنْبِي وَخَطِيَّتِي.
\par 24 لِمَاذَا تَحْجُبُ وَجْهَكَ وَتَحْسِبُنِي عَدُوّاً لَكَ؟
\par 25 أَتُرْعِبُ وَرَقَةً مُنْدَفَعَةً وَتُطَارِدُ قَشّاً يَابِساً!
\par 26 لأَنَّكَ كَتَبْتَ عَلَيَّ أُمُوراً مُرَّةً وَوَرَّثْتَنِي آثَامَ صِبَايَ
\par 27 فَجَعَلْتَ رِجْلَيَّ فِي الْمِقْطَرَةِ وَلاَحَظْتَ جَمِيعَ مَسَالِكِي وَعَلَى أُصُولِ رِجْلَيَّ نَبَشْتَ.
\par 28 وَأَنَا كَمُتَسَوِّسٍ يَبْلَى كَثَوْبٍ أَكَلَهُ الْعُثُّ.

\chapter{14}

\par 1 [اَلإِنْسَانُ مَوْلُودُ الْمَرْأَةِ قَلِيلُ الأَيَّامِ وَشَبْعَانُ تَعَباً.
\par 2 يَخْرُجُ كَالزَّهْرِ ثُمَّ يَذْوِي وَيَبْرَحُ كَالظِّلِّ وَلاَ يَقِفُ.
\par 3 فَعَلَى مِثْلِ هَذَا حَدَّقْتَ عَيْنَيْكَ وَإِيَّايَ أَحْضَرْتَ إِلَى الْمُحَاكَمَةِ مَعَكَ.
\par 4 مَنْ يُخْرِجُ الطَّاهِرَ مِنَ النَّجِسِ؟ لاَ أَحَدٌ!
\par 5 إِنْ كَانَتْ أَيَّامُهُ مَحْدُودَةً وَعَدَدُ أَشْهُرِهِ عِنْدَكَ وَقَدْ عَيَّنْتَ أَجَلَهُ فَلاَ يَتَجَاوَزُهُ
\par 6 فَأَقْصِرْ عَنْهُ لِيَسْتَرِيحَ إِلَى أَنْ يُسَرَّ كَالأَجِيرِ بِانْتِهَاءِ يَوْمِهِ.
\par 7 [لأَنَّ لِلشَّجَرَةِ رَجَاءً. إِنْ قُطِعَتْ تُخْلِفْ أَيْضاً وَلاَ تُعْدَمُ أَغْصَانُهَا.
\par 8 وَلَوْ قَدُمَ فِي الأَرْضِ أَصْلُهَا وَمَاتَ فِي التُّرَابِ جِذْعُهَا
\par 9 فَمِنْ رَائِحَةِ الْمَاءِ تُفْرِخُ وَتُنْبِتُ فُرُوعاً كَالْغَرْسِ.
\par 10 أَمَّا الرَّجُلُ فَيَمُوتُ وَيَبْلَى. الإِنْسَانُ يُسْلِمُ الرُّوحَ فَأَيْنَ هُوَ!
\par 11 قَدْ تَنْفَدُ الْمِيَاهُ مِنَ الْبَحْرِ وَالنَّهْرُ يَنْشَفُ وَيَجِفُّ
\par 12 وَالإِنْسَانُ يَضْطَجِعُ وَلاَ يَقُومُ. لاَ يَسْتَيْقِظُونَ حَتَّى لاَ تَبْقَى السَّمَاوَاتُ وَلاَ يَنْتَبِهُونَ مِنْ نَوْمِهِمْ.
\par 13 لَيْتَكَ تُوارِينِي فِي الْهَاوِيَةِ وَتُخْفِينِي إِلَى أَنْ يَنْصَرِفَ غَضَبُكَ وَتُعَيِّنُ لِي أَجَلاً فَتَذْكُرَنِي.
\par 14 إِنْ مَاتَ رَجُلٌ أَفَيَحْيَا؟ كُلَّ أَيَّامِ جِهَادِي أَصْبِرُ إِلَى أَنْ يَأْتِيَ بَدَلِي.
\par 15 تَدْعُو فَأَنَا أُجِيبُكَ. تَشْتَاقُ إِلَى عَمَلِ يَدِكَ.
\par 16 أَمَّا الآنَ فَتُحْصِي خَطَوَاتِي! أَلاَ تُحَافِظُ عَلَى خَطِيَّتِي.
\par 17 مَعْصِيَتِي مَخْتُومٌ عَلَيْهَا فِي صُرَّةٍ وَتُلَفِّقُ عَلَيَّ فَوْقَ إِثْمِي.
\par 18 [إِنَّ الْجَبَلَ السَّاقِطَ يَنْتَثِرُ وَالصَّخْرَ يُزَحْزَحُ مِنْ مَكَانِهِ.
\par 19 الْحِجَارَةُ تَبْلِيهَا الْمِيَاهُ وَتَجْرُفُ سُيُولُهَا تُرَابَ الأَرْضِ. وَكَذَلِكَ أَنْتَ تُبِيدُ رَجَاءَ الإِنْسَانِ.
\par 20 تَتَجَبَّرُ عَلَيْهِ أَبَداً فَيَذْهَبُ. تُشَوِّهُ وَجْهَهُ وَتَطْرُدُهُ.
\par 21 يُكْرَمُ بَنُوهُ وَلاَ يَعْلَمُ أَوْ يَصْغِرُونَ وَلاَ يَفْهَمُ بِهِمْ.
\par 22 إِنَّمَا عَلَى ذَاتِهِ يَتَوَجَّعُ لَحْمُهُ وَعَلَى ذَاتِهَا تَنُوحُ نَفْسُهُ].

\chapter{15}

\par 1 فَأَجَابَ أَلِيفَازُ التَّيْمَانِيُّ:
\par 2 [أَلَعَلَّ الْحَكِيمَ يُجِيبُ عَنْ مَعْرِفَةٍ بَاطِلَةٍ وَيَمْلَأُ بَطْنَهُ مِنْ رِيحٍ شَرْقِيَّةٍ
\par 3 فَيَحْتَجَّ بِكَلاَمٍ لاَ يُفِيدُ وَبِأَحَادِيثَ لاَ يَنْتَفِعُ بِهَا!
\par 4 أَمَّا أَنْتَ فَتُنَافِي الْمَخَافَةَ وَتُنَاقِضُ التَّقْوَى لَدَى اللهِ.
\par 5 لأَنَّ فَمَكَ يُذِيعُ إِثْمَكَ وَتَخْتَارُ لِسَانَ الْمُحْتَالِينَ.
\par 6 إِنَّ فَمَكَ يَسْتَذْنِبُكَ لاَ أَنَا وَشَفَتَاكَ تَشْهَدَانِ عَلَيْكَ.
\par 7 [أَصُوِّرْتَ أَوَّلَ النَّاسِ أَمْ أُبْدِئْتَ قَبْلَ التِّلاَلِ!
\par 8 هَلْ أَصْغَيتَ فِي مَجْلِسِ اللهِ أَوْ قَصَرْتَ الْحِكْمَةَ عَلَى نَفْسِكَ!
\par 9 مَاذَا تَعْرِفُهُ وَلاَ نَعْرِفُهُ نَحْنُ وَمَاذَا تَفْهَمُ وَلَيْسَ هُوَ عِنْدَنَا؟
\par 10 عِنْدَنَا الشَّيْخُ وَالأَشْيَبُ أَكْبَرُ أَيَّاماً مِنْ أَبِيكَ.
\par 11 أَقَلِيلَةٌ عِنْدَكَ تَعْزِيَاتُ اللهِ وَالْكَلاَمُ مَعَكَ بِالرِّفْقِ!
\par 12 [لِمَاذَا يَأْخُذُكَ قَلْبُكَ وَلِمَاذَا تَخْتَلِجُ عَيْنَاكَ
\par 13 حَتَّى تَرُدَّ عَلَى اللهِ وَتُخْرِجَ مِنْ فَمِكَ أَقْوَالاً؟
\par 14 مَنْ هُوَ الإِنْسَانُ حَتَّى يَزْكُو أَوْ مَوْلُودُ الْمَرْأَةِ حَتَّى يَتَبَرَّرَ؟
\par 15 هُوَذَا قِدِّيسُوهُ لاَ يَأْتَمِنُهُمْ وَالسَّمَاوَاتُ غَيْرُ طَاهِرَةٍ بِعَيْنَيْهِ -
\par 16 فَبِالْحَرِيِّ مَكْرُوهٌ وَفَاسِدٌ الإِنْسَانُ الشَّارِبُ الإِثْمَ كَالْمَاءِ!
\par 17 [أُبَيِّنُ لَكَ. اسْمَعْ لِي فَأُحَدِّثَ بِمَا رَأَيْتُهُ.
\par 18 مَا أَخْبَرَ بِهِ حُكَمَاءُ عَنْ آبَائِهِمْ فَلَمْ يَكْتُمُوهُ.
\par 19 الَّذِينَ لَهُمْ وَحْدَهُمْ أُعْطِيَتِ الأَرْضُ وَلَمْ يَعْبُرْ بَيْنَهُمْ غَرِيبٌ.
\par 20 الشِّرِّيرُ هُوَ يَتَلَوَّى كُلَّ أَيَّامِهِ وَكُلَّ عَدَدِ السِّنِينَ الْمَعْدُودَةِ لِلْعَاتِي.
\par 21 صَوْتُ رُعُوبٍ فِي أُذُنَيْهِ. فِي سَاعَةِ سَلاَمٍ يَأْتِيهِ الْمُخَرِّبُ.
\par 22 لاَ يَأْمُلُ الرُّجُوعَ مِنَ الظُّلْمَةِ وَهُوَ مُرْتَقَبٌ لِلسَّيْفِ.
\par 23 تَائِهٌ هُوَ لأَجْلِ الْخُبْزِ حَيْثُمَا يَجِدُهُ وَيَعْلَمُ أَنَّ يَوْمَ الظُّلْمَةِ مُهَيَّأٌ بَيْنَ يَدَيْهِ.
\par 24 يُرْهِبُهُ الضَّرُّ وَالضِّيقُ. يَتَجَبَّرَانِ عَلَيْهِ كَمَلِكٍ مُسْتَعِدٍّ لِلْوَغَى.
\par 25 لأَنَّهُ مَدَّ عَلَى اللهِ يَدَهُ وَعَلَى الْقَدِيرِ تَجَبَّرَ
\par 26 هَاجِماً عَلَيْهِ مُتَصَلِّبُ الْعُنُقِ بِتُرُوسِهِ الْغَلِيظَةِ.
\par 27 لأَنَّهُ قَدْ كَسَا وَجْهَهُ سَمْناً وَرَبَّى شَحْماً عَلَى كُلْيَتَيْهِ
\par 28 فَيَسْكُنُ مُدُناً خَرِبَةً بُيُوتاً غَيْرَ مَسْكُونَةٍ عَتِيدَةً أَنْ تَصِيرَ رُجَماً.
\par 29 لاَ يَسْتَغْنِي وَلاَ تَثْبُتُ ثَرْوَتُهُ وَلاَ يَمْتَدُّ فِي الأَرْضِ مُقْتَنَاهُ.
\par 30 لاَ تَزُولُ عَنْهُ الظُّلْمَةُ. أَغْصَانُهُ تُيَبِّسُهَا السُّمُومُ وَبِنَفْخَةِ فَمِهِ يَزُولُ.
\par 31 لاَ يَتَّكِلْ عَلَى السُّوءِ. يَضِلُّ. لأَنَّ السُّوءَ يَكُونُ أُجْرَتَهُ.
\par 32 قَبْلَ يَوْمِهِ يُتَوَفَّى وَسَعَفُهُ لاَ يَخْضَرُّ.
\par 33 يُسَاقِطُ كَالْكَرْمَةِ حِصْرِمَهُ وَيَنْثُرُ كَالزَّيْتُونِ زَهْرُهُ.
\par 34 لأَنَّ جَمَاعَةَ الْفُجَّارِ عَاقِرٌ وَالنَّارُ تَأْكُلُ خِيَامَ الرَّشْوَةِ.
\par 35 حَبِلَ شَقَاوَةً وَوَلَدَ إِثْماً وَبَطْنُهُ أَنْشَأَ غِشّاً].

\chapter{16}

\par 1 فَقَالَ أَيُّوبُ:
\par 2 [قَدْ سَمِعْتُ كَثِيراً مِثْلَ هَذَا. مُعَزُّونَ مُتْعِبُونَ كُلُّكُمْ!
\par 3 هَلْ مِنْ نِهَايَةٍ لِكَلاَمٍ فَارِغٍ. أَوْ مَاذَا يُهَيِّجُكَ حَتَّى تُجَاوِبَ؟
\par 4 أَنَا أَيْضاً أَسْتَطِيعُ أَنْ أَتَكَلَّمَ مِثْلَكُمْ لَوْ كَانَتْ أَنْفُسُكُمْ مَكَانَ نَفْسِي وَأَنْ أَسْرُدَ عَلَيْكُمْ أَقْوَالاً وَأَهُزَّ رَأْسِي إِلَيْكُمْ.
\par 5 بَلْ كُنْتُ أُشَدِّدُكُمْ بِفَمِي وَتَعْزِيَةُ شَفَتَيَّ تُمْسِكُكُمْ.
\par 6 [إِنْ تَكَلَّمْتُ لَمْ تَمْتَنِعْ كَآبَتِي. وَإِنْ سَكَتُّ فَمَاذَا يَذْهَبُ عَنِّي؟
\par 7 إِنَّهُ الآنَ ضَجَّرَنِي. خَرَّبْتَ كُلَّ جَمَاعَتِي.
\par 8 قَبَضْتَ عَلَيَّ. وُجِدَ شَاهِدٌ. قَامَ عَلَيَّ هُزَالِي يُجَاوِبُ فِي وَجْهِي.
\par 9 غَضَبُهُ افْتَرَسَنِي وَاضْطَهَدَنِي. حَرَّقَ عَلَيَّ أَسْنَانَهُ. عَدُوِّي يُحَدِّدُ عَيْنَيْهِ عَلَيَّ.
\par 10 فَغَرُوا عَلَيَّ أَفْوَاهَهُمْ. لَطَمُونِي عَلَى فَكِّي تَعْيِيراً. تَعَاوَنُوا عَلَيَّ جَمِيعاً.
\par 11 دَفَعَنِيَ اللهُ إِلَى الظَّالِمِ وَفِي أَيْدِي الأَشْرَارِ طَرَحَنِي.
\par 12 كُنْتُ مُسْتَرِيحاً فَزَعْزَعَنِي وَأَمْسَكَ بِقَفَايَ فَحَطَّمَنِي وَنَصَبَنِي لَهُ هَدَفاً.
\par 13 أَحَاطَتْ بِي رُمَاةُ سِهَامِهِ. شَقَّ كُلْيَتَيَّ وَلَمْ يُشْفِقْ. سَفَكَ مَرَارَتِي عَلَى الأَرْضِ.
\par 14 يَقْتَحِمُنِي اقْتِحَاماً عَلَى اقْتِحَامٍ. يَهْجِمُ عَلَيَّ كَجَبَّارٍ.
\par 15 خِطْتُ مِسْحاً عَلَى جِلْدِي وَدَسَسْتُ فِي التُّرَابِ قَرْنِي.
\par 16 اِحْمَرَّ وَجْهِي مِنَ الْبُكَاءِ وَعَلَى هُدْبِي ظِلُّ الْمَوْتِ.
\par 17 مَعَ أَنَّهُ لاَ ظُلْمَ فِي يَدِي وَصَلاَتِي خَالِصَةٌ.
\par 18 [يَا أَرْضُ لاَ تُغَطِّي دَمِي وَلاَ يَكُنْ مَكَانٌ لِصُرَاخِي.
\par 19 أَيْضاً الآنَ هُوَذَا فِي السَّمَاوَاتِ شَهِيدِي وَشَاهِدِي فِي الأَعَالِي.
\par 20 الْمُسْتَهْزِئُونَ بِي هُمْ أَصْحَابِي. لِلَّهِ تَقْطُرُ عَيْنِي
\par 21 لِكَيْ يُحَاكِمَ الإِنْسَانَ عِنْدَ اللهِ كَابْنِ آدَمَ لَدَى صَاحِبِهِ.
\par 22 إِذَا مَضَتْ سِنُونَ قَلِيلَةٌ أَسْلُكُ فِي طَرِيقٍ لاَ أَعُودُ مِنْهَا.

\chapter{17}

\par 1 [رُوحِي تَلِفَتْ. أَيَّامِي انْطَفَأَتْ. إِنَّمَا الْقُبُورُ لِي.
\par 2 [لَوْلاَ الْمُخَاتِلُونَ عِنْدِي وَعَيْنِي تَبِيتُ عَلَى مُشَاجَرَاتِهِمْ.
\par 3 كُنْ ضَامِنِي عِنْدَ نَفْسِكَ. مَنْ هُوَ الَّذِي يُصَفِّقُ يَدِي؟
\par 4 لأَنَّكَ مَنَعْتَ قَلْبَهُمْ عَنِ الْفِطْنَةِ. لأَجْلِ ذَلِكَ لاَ تَرْفَعُهُمُ.
\par 5 الَّذِي يُسَلِّمُ الأَصْحَابَ لِلسَّلْبِ تَتْلَفُ عُيُونُ بَنِيهِ.
\par 6 أَوْقَفَنِي مَثَلاً لِلشُّعُوبِ وَصِرْتُ لِلْبَصْقِ فِي الْوَجْهِ.
\par 7 كَلَّتْ عَيْنِي مِنَ الْحُزْنِ وَأَعْضَائِي كُلُّهَا كَالظِّلِّ.
\par 8 يَتَعَجَّبُ الْمُسْتَقِيمُونَ مِنْ هَذَا وَالْبَرِئُ يَقُومُ عَلَى الْفَاجِرِ.
\par 9 أَمَّا الصِّدِّيقُ فَيَسْتَمْسِكُ بِطَرِيقِهِ وَالطَّاهِرُ الْيَدَيْنِ يَزْدَادُ قُوَّةً.
\par 10 [وَلَكِنِ ارْجِعُوا كُلُّكُمْ وَتَعَالُوْا فَلاَ أَجِدُ فِيكُمْ حَكِيماً.
\par 11 أَيَّامِي قَدْ عَبَرَتْ. مَقَاصِدِي إِرْثُ قَلْبِي قَدِ انْتَزَعَتْ.
\par 12 يَجْعَلُونَ اللَّيْلَ نَهَاراً نُوراً قَرِيباً لِلظُّلْمَةِ.
\par 13 إِذَا رَجَوْتُ الْهَاوِيَةَ بَيْتاً لِي وَفِي الظَّلاَمِ مَهَّدْتُ فِرَاشِي
\par 14 وَقُلْتُ لِلْقَبْرِ: أَنْتَ أَبِي وَلِلدُّودِ: أَنْتَ أُمِّي وَأُخْتِي
\par 15 فَأَيْنَ إِذاً آمَالِي؟ آمَالِي مَنْ يُعَايِنُهَا!
\par 16 تَهْبِطُ إِلَى مَغَالِيقِ الْهَاوِيَةِ إِذْ تَرْتَاحُ مَعاً فِي التُّرَابِ]

\chapter{18}

\par 1 فَأَجَابَ بِلْدَدُ الشُّوحِيُّ:
\par 2 [إِلَى مَتَى تَضَعُونَ أَشْرَاكاً لِلْكَلاَمِ؟ تَعَقَّلُوا وَبَعْدُ نَتَكَلَّمُ.
\par 3 لِمَاذَا حُسِبْنَا كَالْبَهِيمَةِ وَتَنَجَّسْنَا فِي عُيُونِكُمْ؟
\par 4 يَا أَيُّهَا الْمُفْتَرِسُ نَفْسَهُ فِي غَيْظِهِ هَلْ لأَجْلِكَ تُخْلَى الأَرْضُ أَوْ يُزَحْزَحُ الصَّخْرُ مِنْ مَكَانِهِ؟
\par 5 [نَعَمْ! نُورُ الأَشْرَارِ يَنْطَفِئُ وَلاَ يُضِيءُ لَهِيبُ نَارِهِ.
\par 6 النُّورُ يُظْلِمُ فِي خَيْمَتِهِ وَسِرَاجُهُ فَوْقَهُ يَنْطَفِئُ.
\par 7 تَقْصُرُ خَطَوَاتُ قُوَّتِهِ وَتَصْرَعُهُ مَشُورَتُهُ.
\par 8 لأَنَّ رِجْلَيْهِ تَدْفَعَانِهِ فِي الْفَخِّ فَيَمْشِي إِلَى شَبَكَةٍ.
\par 9 يُمْسِكُ الْفَخُّ بِعَقِبِهِ وَتَتَمَكَّنُ مِنْهُ الشَّرَكُ.
\par 10 حَبْلٌ مَطْمُورٌ لَهُ فِي الأَرْضِ وَمِصْيَدَتُهُ فِي السَّبِيلِ.
\par 11 تُرْهِبُهُ أَهْوَالٌ مِنْ حَوْلِهِ وَتَذْعَرُهُ عِنْدَ رِجْلَيْهِ.
\par 12 تَكُونُ قُوَّتُهُ جَائِعَةً وَالْبَوَارُ مُهَيَّأٌ بِجَانِبِهِ.
\par 13 يَأْكُلُ أَعْضَاءَ جَسَدِهِ. يَأْكُلُ أَعْضَاءَهُ بِكْرُ الْمَوْتِ.
\par 14 يَنْقَطِعُ عَنْ خَيْمَتِهِ عَنِ اعْتِمَادِهِ وَيُسَاقُ إِلَى مَلِكِ الأَهْوَالِ.
\par 15 يَسْكُنُ فِي خَيْمَتِهِ مَنْ لَيْسَ لَهُ. يُذَرُّ عَلَى مَرْبِضِهِ كِبْرِيتٌ.
\par 16 مِنْ تَحْتُ تَيْبَسُ أُصُولُهُ وَمِنْ فَوْقُ يُقْطَعُ فَرْعُهُ.
\par 17 ذِكْرُهُ يَبِيدُ مِنَ الأَرْضِ وَلاَ اسْمَ لَهُ عَلَى وَجْهِ الْبَرِّ.
\par 18 يُدْفَعُ مِنَ النُّورِ إِلَى الظُّلْمَةِ وَمِنَ الْمَسْكُونَةِ يُطْرَدُ.
\par 19 لاَ نَسْلَ وَلاَ ذُرِّيَّةَ لَهُ بَيْنَ شَعْبِهِ وَلاَ بَاقٍ فِي مَنَازِلِهِ.
\par 20 يَتَعَجَّبُ مِنْ يَوْمِهِ الْمُتَأَخِّرُونَ وَيَقْشَعِرُّ الأَقْدَمُونَ.
\par 21 إِنَّمَا تِلْكَ مَسَاكِنُ فَاعِلِي الشَّرِّ وَهَذَا مَقَامُ مَنْ لاَ يَعْرِفُ اللهَ].

\chapter{19}

\par 1 فَقَالَ أَيُّوبُ
\par 2 [حَتَّى مَتَى تُعَذِّبُونَ نَفْسِي وَتَسْحَقُونَنِي بِالْكَلاَمِ.
\par 3 هَذِهِ عَشَرَ مَرَّاتٍ أَخْزَيْتُمُونِي. لَمْ تَخْجَلُوا مِنْ أَنْ تُعَنِّفُونِي.
\par 4 وَهَبْنِي ضَلَلْتُ حَقّاً. عَلَيَّ تَسْتَقِرُّ ضَلاَلَتِي!
\par 5 إِنْ كُنْتُمْ بِالْحَقِّ تَسْتَكْبِرُونَ عَلَيَّ فَثَبِّتُوا عَلَيَّ عَارِي.
\par 6 فَاعْلَمُوا إِذاً أَنَّ اللهَ قَدْ عَوَّجَنِي وَلَفَّ عَلَيَّ أُحْبُولَتَهُ.
\par 7 هَا إِنِّي أَصْرُخُ ظُلْماً فَلاَ أُسْتَجَابُ. أَدْعُو وَلَيْسَ حُكْمٌ.
\par 8 قَدْ حَوَّطَ طَرِيقِي فَلاَ أَعْبُرُ وَعَلَى سُبُلِي جَعَلَ ظَلاَماً.
\par 9 أَزَالَ عَنِّي كَرَامَتِي وَنَزَعَ تَاجَ رَأْسِي.
\par 10 هَدَمَنِي مِنْ كُلِّ جِهَةٍ فَذَهَبْتُ وَقَلَعَ مِثْلَ شَجَرَةٍ رَجَائِي
\par 11 وَأَضْرَمَ عَلَيَّ غَضَبَهُ وَحَسِبَنِي كَأَعْدَائِهِ.
\par 12 مَعاً جَاءَتْ غُزَاتُهُ وَأَعَدُّوا عَلَيَّ طَرِيقَهُمْ وَحَلُّوا حَوْلَ خَيْمَتِي.
\par 13 قَدْ أَبْعَدَ عَنِّي إِخْوَتِي. وَمَعَارِفِي زَاغُوا عَنِّي.
\par 14 أَقَارِبِي قَدْ خَذَلُونِي وَالَّذِينَ عَرَفُونِي نَسُونِي.
\par 15 نُزَلاَءُ بَيْتِي وَإِمَائِي يَحْسِبُونَنِي أَجْنَبِيّاً. صِرْتُ فِي أَعْيُنِهِمْ غَرِيباً.
\par 16 عَبْدِي دَعَوْتُ فَلَمْ يُجِبْ. بِفَمِي تَضَرَّعْتُ إِلَيْهِ.
\par 17 نَكْهَتِي مَكْرُوهَةٌ عِنْدَ امْرَأَتِي وَمُنْتِنَةٌ عِنْدَ أَبْنَاءِ أَحْشَائِي.
\par 18 اَلأَوْلاَدُ أَيْضاً قَدْ رَذَلُونِي. إِذَا قُمْتُ يَتَكَلَّمُونَ عَلَيَّ.
\par 19 كَرِهَنِي كُلُّ رِجَالِي وَالَّذِينَ أَحْبَبْتُهُمُ انْقَلَبُوا عَلَيَّ.
\par 20 عَظْمِي قَدْ لَصِقَ بِجِلْدِي وَلَحْمِي وَنَجَوْتُ بِجِلْدِ أَسْنَانِي.
\par 21 تَرَاءَفُوا! تَرَاءَفُوا أَنْتُمْ عَلَيَّ يَا أَصْحَابِي لأَنَّ يَدَ اللهِ قَدْ مَسَّتْنِي.
\par 22 لِمَاذَا تُطَارِدُونَنِي كَمَا اللهُ وَلاَ تَشْبَعُونَ مِنْ لَحْمِي؟
\par 23 [لَيْتَ كَلِمَاتِي الآنَ تُكْتَبُ. يَا لَيْتَهَا رُسِمَتْ فِي سِفْرٍ
\par 24 وَنُقِرَتْ إِلَى الأَبَدِ فِي الصَّخْرِ بِقَلَمِ حَدِيدٍ وَبِرَصَاصٍ.
\par 25 أَمَّا أَنَا فَقَدْ عَلِمْتُ أَنَّ وَلِيِّي حَيٌّ وَالآخِرَ عَلَى الأَرْضِ يَقُومُ
\par 26 وَبَعْدَ أَنْ يُفْنَى جِلْدِي هَذَا وَبِدُونِ جَسَدِي أَرَى اللهَ.
\par 27 الَّذِي أَرَاهُ أَنَا لِنَفْسِي وَعَيْنَايَ تَنْظُرَانِ وَلَيْسَ آخَرُ. إِلَى ذَلِكَ تَتُوقُ كُلْيَتَايَ فِي جَوْفِي.
\par 28 فَإِنَّكُمْ تَقُولُونَ: لِمَاذَا نُطَارِدُهُ؟ وَالْكَلاَمُ الأَصْلِيُّ يُوجَدُ عِنْدِي.
\par 29 خَافُوا عَلَى أَنْفُسِكُمْ مِنَ السَّيْفِ لأَنَّ الْغَيْظَ مِنْ آثَامِ السَّيْفِ. لِكَيْ تَعْلَمُوا مَا هُوَ الْقَضَاءُ].

\chapter{20}

\par 1 فَأَجَابَ صُوفَرُ النَّعْمَاتِيُّ:
\par 2 [مِنْ أَجْلِ ذَلِكَ هَوَاجِسِي تُجِيبُنِي وَلِهَذَا هَيَجَانِي فِيَّ.
\par 3 تَعْيِيرَ تَوْبِيخِي أَسْمَعُ. وَرُوحٌ مِنْ فَهْمِي يُجِيبُنِي.
\par 4 [أَمَا عَلِمْتَ هَذَا مِنَ الْقَدِيمِ مُنْذُ وُضِعَ الإِنْسَانُ عَلَى الأَرْضِ:
\par 5 أَنَّ هُتَافَ الأَشْرَارِ مِنْ قَرِيبٍ وَفَرَحَ الْفَاجِرِ إِلَى لَحْظَةٍ!
\par 6 وَلَوْ بَلَغَ السَّمَاوَاتِ طُولُهُ وَمَسَّ رَأْسُهُ السَّحَابَ
\par 7 كَجُلَّتِهِ إِلَى الأَبَدِ يَبِيدُ. الَّذِينَ رَأُوهُ يَقُولُونَ: أَيْنَ هُوَ؟
\par 8 كَالْحُلْمِ يَطِيرُ فَلاَ يُوجَدُ وَيُطْرَدُ كَطَيْفِ اللَّيْلِ.
\par 9 عَيْنٌ أَبْصَرَتْهُ لاَ تَعُودُ تَرَاهُ وَمَكَانُهُ لَنْ يَرَاهُ بَعْدُ.
\par 10 بَنُوهُ يَتَرَضُّونَ الْفُقَرَاءَ وَيَدَاهُ تَرُدَّانِ ثَرْوَتَهُ.
\par 11 عِظَامُهُ مَلآنَةٌ قُوَّةً وَمَعَهُ فِي التُّرَابِ تَضْطَجِعُ.
\par 12 إِنْ حَلاَ فِي فَمِهِ الشَّرُّ وَأَخْفَاهُ تَحْتَ لِسَانِهِ
\par 13 أَشْفَقَ عَلَيْهِ وَلَمْ يَتْرُكْهُ بَلْ حَبَسَهُ وَسَطَ حَنَكِهِ
\par 14 فَخُبْزُهُ فِي أَمْعَائِهِ يَتَحَوَّلُ! مَرَارَةُ أَصْلاَلٍ فِي بَطْنِهِ.
\par 15 قَدْ بَلَعَ ثَرْوَةً فَيَتَقَيَّأُهَا. اللهُ يَطْرُدُهَا مِنْ بَطْنِهِ.
\par 16 سِمَّ الأَصْلاَلِ يَرْضَعُ. يَقْتُلُهُ لِسَانُ الأَفْعَى.
\par 17 لاَ يَرَى الْجَدَاوِلَ أَنْهَارَ سَوَاقِيَ عَسَلٍ وَلَبَنٍ.
\par 18 يَرُدُّ تَعَبَهُ وَلاَ يَبْلَعُهُ. وَبِمَكْسَبِ تِجَارَتِهِ لاَ يَفْرَحُ.
\par 19 لأَنَّهُ رَضَّضَ الْمَسَاكِينَ وَتَرَكَهُمْ وَاغْتَصَبَ بَيْتاً وَلَمْ يَبْنِهِ.
\par 20 لأَنَّهُ لَمْ يَعْرِفْ فِي بَطْنِهِ قَنَاعَةً لاَ يَنْجُو بِمُشْتَهَاهُ.
\par 21 لَيْسَتْ مِنْ أَكْلِهِ بَقِيَّةٌ لأَجْلِ ذَلِكَ لاَ يَدُومُ خَيْرُهُ.
\par 22 مَعَ مِلْءِ رَغْدِهِ يَتَضَايَقُ. تَأْتِي عَلَيْهِ يَدُ كُلِّ شَقِيٍّ.
\par 23 يَكُونُ عِنْدَمَا يَمْلَأُ بَطْنَهُ أَنَّ اللهَ يُرْسِلُ عَلَيْهِ حُمُوَّ غَضَبِهِ وَيُمْطِرُهُ عَلَيْهِ عِنْدَ طَعَامِهِ.
\par 24 يَفِرُّ مِنْ سِلاَحِ حَدِيدٍ. تَخْرِقُهُ قَوْسُ نُحَاسٍ.
\par 25 جَذَبَهُ فَخَرَجَ مِنْ بَطْنِهِ وَالْبَارِقُ مِنْ مَرَارَتِهِ مَرَقَ. عَلَيْهِ رُعُوبٌ.
\par 26 كُلُّ ظُلْمَةٍ مُخْتَبَأَةٌ لِذَخَائِرِهِ. تَأْكُلُهُ نَارٌ لَمْ تُنْفَخْ. تَرْعَى الْبَقِيَّةَ فِي خَيْمَتِهِ.
\par 27 السَّمَاوَاتُ تُعْلِنُ إِثْمَهُ وَالأَرْضُ تَنْهَضُ عَلَيْهِ.
\par 28 تَزُولُ غَلَّةُ بَيْتِهِ. تُهْرَاقُ فِي يَوْمِ غَضَبِهِ.
\par 29 هَذَا نَصِيبُ الإِنْسَانِ الشِّرِّيرِ مِنْ عِنْدِ اللهِ وَمِيرَاثُ أَمْرِهِ مِنَ الْقَدِيرِ].

\chapter{21}

\par 1 فَقَالَ أَيُّوبُ:
\par 2 [اِسْمَعُوا قَوْلِي سَمْعاً وَلْيَكُنْ هَذَا تَعْزِيَتَكُمْ.
\par 3 اِحْتَمِلُونِي وَأَنَا أَتَكَلَّمُ وَبَعْدَ كَلاَمِي اسْتَهْزِئُوا!
\par 4 أَمَّا أَنَا فَهَلْ شَكْوَايَ مِنْ إِنْسَانٍ. وَإِنْ كَانَتْ فَلِمَاذَا لاَ تَضِيقُ رُوحِي؟
\par 5 تَفَرَّسُوا فِيَّ وَتَعَجَّبُوا وَضَعُوا الْيَدَ عَلَى الْفَمِ.
\par 6 [عِنْدَمَا أَتَذَكَّرُ أَرْتَاعُ وَأَخَذَتْ بَشَرِي رَعْدَةٌ.
\par 7 لِمَاذَا تَحْيَا الأَشْرَارُ وَيَشِيخُونَ نَعَمْ وَيَتَجَبَّرُونَ قُوَّةً؟
\par 8 نَسْلُهُمْ قَائِمٌ أَمَامَهُمْ مَعَهُمْ وَذُرِّيَّتُهُمْ فِي أَعْيُنِهِمْ.
\par 9 بُيُوتُهُمْ آمِنَةٌ مِنَ الْخَوْفِ وَلَيْسَ عَلَيْهِمْ عَصَا اللهِ.
\par 10 ثَوْرُهُمْ يُلْقِحُ وَلاَ يُخْطِئُ. بَقَرَتُهُمْ تُنْتِجُ وَلاَ تُسْقِطُ.
\par 11 يُسْرِحُونَ مِثْلَ الْغَنَمِ رُضَّعَهُمْ وَأَطْفَالُهُمْ تَرْقُصُ.
\par 12 يَحْمِلُونَ الدُّفَّ وَالْعُودَ وَيُطْرِبُونَ بِصَوْتِ الْمِزْمَارِ.
\par 13 يَقْضُونَ أَيَّامَهُمْ بِالْخَيْرِ. فِي لَحْظَةٍ يَهْبِطُونَ إِلَى الْهَاوِيَةِ.
\par 14 فَيَقُولُونَ لِلَّهِ: ابْعُدْ عَنَّا. وَبِمَعْرِفَةِ طُرُقِكَ لاَ نُسَرُّ.
\par 15 مَنْ هُوَ الْقَدِيرُ حَتَّى نَعْبُدَهُ وَمَاذَا نَنْتَفِعُ إِنِ الْتَمَسْنَاهُ!.
\par 16 [هُوَذَا لَيْسَ فِي يَدِهِمْ خَيْرُهُمْ. لِتَبْعُدْ عَنِّي مَشُورَةُ الأَشْرَارِ.
\par 17 كَمْ يَنْطَفِئُ سِرَاجُ الأَشْرَارِ وَيَأْتِي عَلَيْهِمْ بَوَارُهُمْ أَوْ يَقْسِمُ لَهُمْ أَوْجَاعاً فِي غَضَبِهِ
\par 18 أَوْ يَكُونُونَ كَالتِّبْنِ قُدَّامَ الرِّيحِ وَكَالْعُصَافَةِ الَّتِي تَسْرِقُهَا الزَّوْبَعَةُ.
\par 19 اَللهُ يَخْزِنُ إِثْمَهُ لِبَنِيهِ. لِيُجَازِهِ نَفْسَهُ فَيَعْلَمَ.
\par 20 لِتَنْظُرْ عَيْنَاهُ هَلاَكَهُ وَمِنْ حُمَةِ الْقَدِيرِ يَشْرَبْ.
\par 21 فَمَا هِيَ مَسَرَّتُهُ فِي بَيْتِهِ بَعْدَهُ وَقَدْ تَعَيَّنَ عَدَدُ شُهُورِهِ؟
\par 22 [أَاللهُ يُعَلَّمُ مَعْرِفَةً وَهُوَ يَقْضِي عَلَى الْعَالِينَ؟
\par 23 هَذَا يَمُوتُ فِي عَيْنِ كَمَالِهِ. كُلُّهُ مُطْمَئِنٌّ وَسَاكِنٌ.
\par 24 أَحْوَاضُهُ مَلآنَةٌ لَبَناً وَمُخُّ عِظَامِهِ طَرِيٌّ.
\par 25 وَذَلِكَ يَمُوتُ بِنَفْسٍ مُرَّةٍ وَلَمْ يَذُقْ خَيْراً.
\par 26 كِلاَهُمَا يَضْطَجِعَانِ مَعاً فِي التُّرَابِ وَالدُّودُ يَغْشَاهُمَا.
\par 27 [هُوَذَا قَدْ عَلِمْتُ أَفْكَارَكُمْ وَالنِّيَّاتِ الَّتِي بِهَا تَظْلِمُونَنِي.
\par 28 لأَنَّكُمْ تَقُولُونَ: أَيْنَ بَيْتُ الْعَاتِي وَأَيْنَ خَيْمَةُ مَسَاكِنِ الأَشْرَارِ؟
\par 29 أَفَلَمْ تَسْأَلُوا عَابِرِي السَّبِيلِ وَلَمْ تَفْطَنُوا لِدَلاَئِلِهِمْ.
\par 30 إِنَّهُ لِيَوْمِ الْبَوَارِ يُمْسَكُ الشِّرِّيرُ. لِيَوْمِ السَّخَطِ يُقَادُونَ.
\par 31 مَنْ يُعْلِنُ طَرِيقَهُ لِوَجْهِهِ وَمَنْ يُجَازِيهِ عَلَى مَا عَمِلَ؟
\par 32 هُوَ إِلَى الْقُبُورِ يُقَادُ وَعَلَى الْمَدْفَنِ يُسْهَرُ.
\par 33 حُلْوٌ لَهُ طِينُ الْوَادِي. يَزْحَفُ كُلُّ إِنْسَانٍ وَرَاءَهُ وَقُدَّامَهُ مَا لاَ عَدَدَ لَهُ.
\par 34 فَكَيْفَ تُعَزُّونَنِي بَاطِلاً وَأَجْوِبَتُكُمْ بَقِيَتْ خِيَانَةً؟].

\chapter{22}

\par 1 فَأَجَابَ أَلِيفَازُ التَّيْمَانِيُّ:
\par 2 [هَلْ يَنْفَعُ الإِنْسَانُ اللهَ؟ بَلْ يَنْفَعُ نَفْسَهُ الْفَطِنُ!
\par 3 هَلْ مِنْ مَسَرَّةٍ لِلْقَدِيرِ إِذَا تَبَرَّرْتَ أَوْ مِنْ فَائِدَةٍ إِذَا قَوَّمْتَ طُرُقَكَ؟
\par 4 هَلْ عَلَى تَقْوَاكَ يُوَبِّخُكَ أَوْ يَدْخُلُ مَعَكَ فِي الْمُحَاكَمَةِ؟
\par 5 أَلَيْسَ شَرُّكَ عَظِيماً وَآثَامُكَ لاَ نِهَايَةَ لَهَا!
\par 6 لأَنَّكَ ارْتَهَنْتَ أَخَاكَ بِلاَ سَبَبٍ وَسَلَبْتَ ثِيَابَ الْعُرَاةِ.
\par 7 مَاءً لَمْ تَسْقِ الْعَطْشَانَ وَعَنِ الْجَوْعَانِ مَنَعْتَ خُبْزاً.
\par 8 أَمَّا صَاحِبُ الْقُوَّةِ فَلَهُ الأَرْضُ وَالْمُتَرَفِّعُ الْوَجْهِ سَاكِنٌ فِيهَا.
\par 9 الأَرَامِلَ أَرْسَلْتَ خَالِيَاتٍ وَذِرَاعُ الْيَتَامَى انْسَحَقَتْ.
\par 10 لأَجْلِ ذَلِكَ حَوَالَيْكَ فِخَاخٌ وَيُرِيعُكَ رُعْبٌ بَغْتَةً
\par 11 أَوْ ظُلْمَةٌ فَلاَ تَرَى وَفَيْضُ الْمِيَاهِ يُغَطِّيكَ.
\par 12 [هُوَذَا اللهُ فِي عُلُوِّ السَّمَاوَاتِ. وَانْظُرْ رَأْسَ الْكَوَاكِبِ مَا أَعْلاَهُ.
\par 13 فَقُلْتَ: كَيْفَ يَعْلَمُ اللهُ؟ هَلْ مِنْ وَرَاءِ الضَّبَابِ يَقْضِي؟
\par 14 السَّحَابُ سِتْرٌ لَهُ فَلاَ يُرَى وَعَلَى دَائِرَةِ السَّمَاوَاتِ يَتَمَشَّى.
\par 15 هَلْ تَحْفَظُ طَرِيقَ الْقِدَمِ الَّذِي دَاسَهُ رِجَالُ الإِثْمِ
\par 16 الَّذِينَ قُبِضَ عَلَيْهِمْ قَبْلَ الْوَقْتِ؟ الْغَمْرُ انْصَبَّ عَلَى أَسَاسِهِمِ.
\par 17 الْقَائِلِينَ لِلَّهِ: ابْعُدْ عَنَّا. وَمَاذَا يَفْعَلُ الْقَدِيرُ لَهُمْ.
\par 18 وَهُوَ قَدْ مَلَأَ بُيُوتَهُمْ خَيْراً. لِتَبْعُدْ عَنِّي مَشُورَةُ الأَشْرَارِ.
\par 19 الأَبْرَارُ يَنْظُرُونَ وَيَفْرَحُونَ وَالْبَرِيءُ يَسْتَهْزِئُ بِهِمْ قَائِلِينَ:
\par 20 أَلَمْ يُبَدْ مُقَاوِمُونَا وَبَقِيَّتُهُمْ قَدْ أَكَلَتْهَا النَّارُ؟
\par 21 [تَعَرَّفْ بِهِ وَاسْلَمْ. بِذَلِكَ يَأْتِيكَ خَيْرٌ.
\par 22 اقْبَلِ الشَّرِيعَةَ مِنْ فَمِهِ وَضَعْ كَلاَمَهُ فِي قَلْبِكَ.
\par 23 إِنْ رَجَعْتَ إِلَى الْقَدِيرِ تُبْنَى. إِنْ أَبْعَدْتَ ظُلْماً مِنْ خَيْمَتِكَ
\par 24 وَأَلْقَيْتَ التِّبْرَ عَلَى التُّرَابِ وَذَهَبَ أُوفِيرَ بَيْنَ حَصَا الأَوْدِيَةِ.
\par 25 يَكُونُ الْقَدِيرُ تِبْرَكَ وَفِضَّةَ أَتْعَابٍ لَكَ.
\par 26 لأَنَّكَ حِينَئِذٍ تَتَلَذَّذُ بِالْقَدِيرِ وَتَرْفَعُ إِلَى اللهِ وَجْهَكَ.
\par 27 تُصَلِّي لَهُ فَيَسْتَمِعُ لَكَ وَنُذُورُكَ تُوفِيهَا.
\par 28 وَتَجْزِمُ أَمْراً فَيُثَبَّتُ لَكَ وَعَلَى طُرُقِكَ يُضِيءُ نُورٌ.
\par 29 إِذَا وُضِعُوا تَقُولُ: رَفْعٌ. وَيُخَلِّصُ الْمُنْخَفِضَ الْعَيْنَيْنِ.
\par 30 يُنَجِّي غَيْرَ الْبَرِيءِ وَيُنْجَى بِطَهَارَةِ يَدَيْكَ].

\chapter{23}

\par 1 فَقَالَ أَيُّوبُ:
\par 2 [الْيَوْمَ أَيْضاً شَكْوَايَ تَمَرُّدٌ. ضَرْبَتِي أَثْقَلُ مِنْ تَنَهُّدِي.
\par 3 مَنْ يُعْطِينِي أَنْ أَجِدَهُ فَآتِيَ إِلَى كُرْسِيِّهِ!
\par 4 أُحْسِنُ الدَّعْوَى أَمَامَهُ وَأَمْلَأُ فَمِي حُجَجاً.
\par 5 فَأَعْرِفُ الأَقْوَالَ الَّتِي بِهَا يُجِيبُنِي وَأَفْهَمُ مَا يَقُولُهُ لِي.
\par 6 أَبِكَثْرَةِ قُوَّةٍ يُخَاصِمُنِي؟ كَلاَّ! وَلَكِنَّهُ كَانَ يَنْتَبِهُ إِلَيَّ.
\par 7 هُنَالِكَ كَانَ يُحَاجُّهُ الْمُسْتَقِيمُ وَكُنْتُ أَنْجُو إِلَى الأَبَدِ مِنْ قَاضِيَّ.
\par 8 هَئَنَذَا أَذْهَبُ شَرْقاً فَلَيْسَ هُوَ هُنَاكَ وَغَرْباً فَلاَ أَشْعُرُ بِهِ
\par 9 شِمَالاً حَيْثُ عَمَلُهُ فَلاَ أَنْظُرُهُ. يَتَعَطَّفُ الْجَنُوبَ فَلاَ أَرَاهُ.
\par 10 [لأَنَّهُ يَعْرِفُ طَرِيقِي. إِذَا جَرَّبَنِي أَخْرُجُ كَالذَّهَبِ.
\par 11 بِخَطَوَاتِهِ اسْتَمْسَكَتْ رِجْلِي. حَفِظْتُ طَرِيقَهُ وَلَمْ أَحِدْ.
\par 12 مِنْ وَصِيَّةِ شَفَتَيْهِ لَمْ أَبْرَحْ. أَكْثَرَ مِنْ فَرِيضَتِي ذَخَرْتُ كَلاَمَ فَمِهِ.
\par 13 أَمَّا هُوَ فَوَحْدَهُ فَمَنْ يَرُدُّهُ؟ وَنَفْسُهُ تَشْتَهِي فَيَفْعَلُ.
\par 14 لأَنَّهُ يُتَمِّمُ الْمَفْرُوضَ عَلَيَّ وَكَثِيرٌ مِثْلُ هَذِهِ عِنْدَهُ.
\par 15 مِنْ أَجْلِ ذَلِكَ أَرْتَاعُ قُدَّامَهُ. أَتَأَمَّلُ فَأَرْتَعِبُ مِنْهُ.
\par 16 لأَنَّ اللهَ قَدْ أَضْعَفَ قَلْبِي وَالْقَدِيرَ رَوَّعَنِي.
\par 17 لأَنِّي لَمْ أُقْطَعْ قَبْلَ الظَّلاَمِ وَمِنْ وَجْهِي لَمْ يُغَطِّ الدُّجَى.

\chapter{24}

\par 1 [لِمَاذَا إِذْ لَمْ تَخْتَبِئِ الأَزْمِنَةُ مِنَ الْقَدِيرِ لاَ يَرَى عَارِفُوهُ يَوْمَهُ؟
\par 2 يَنْقُلُونَ التُّخُومَ. يَغْتَصِبُونَ قَطِيعاً وَيَرْعَوْنَهُ.
\par 3 يَسْتَاقُونَ حِمَارَ الْيَتَامَى وَيَرْتَهِنُونَ ثَوْرَ الأَرْمَلَةِ.
\par 4 يَصُدُّونَ الْفُقَرَاءَ عَنِ الطَّرِيقِ. مَسَاكِينُ الأَرْضِ يَخْتَبِئُونَ جَمِيعاً.
\par 5 هَا هُمْ كَالْفَرَاءِ فِي الْقَفْرِ يَخْرُجُونَ إِلَى عَمَلِهِمْ يُبَكِّرُونَ لِلطَّعَامِ. الْبَادِيَةُ لَهُمْ خُبْزٌ لأَوْلاَدِهِمْ.
\par 6 فِي الْحَقْلِ يَحْصُدُونَ عَلَفَهُمْ وَيُعَلِّلُونَ كَرْمَ الشِّرِّيرِ.
\par 7 يَبِيتُونَ عُرَاةً بِلاَ لِبْسٍ وَلَيْسَ لَهُمْ كِسْوَةٌ فِي الْبَرْدِ.
\par 8 يَبْتَلُّونَ مِنْ مَطَرِ الْجِبَالِ وَلِعَدَمِ الْمَلْجَإِ يَعْتَنِقُونَ الصَّخْرَ.
\par 9 [يَخْطُفُونَ الْيَتِيمَ عَنِ الثُّدِيِّ وَمِنَ الْمَسَاكِينِ يَرْتَهِنُونَ.
\par 10 عُرَاةً يَذْهَبُونَ بِلاَ لِبْسٍ وَجَائِعِينَ يَحْمِلُونَ حُزَماً.
\par 11 يَعْصُرُونَ الزَّيْتَ دَاخِلَ أَسْوَارِهِمْ. يَدُوسُونَ الْمَعَاصِرَ وَيَعْطَشُونَ.
\par 12 مِنَ الْوَجَعِ أُنَاسٌ يَئِنُّونَ وَنَفْسُ الْجَرْحَى تَسْتَغِيثُ وَاللهُ لاَ يَنْتَبِهُ إِلَى الظُّلْمِ.
\par 13 [أُولَئِكَ يَكُونُونَ بَيْنَ الْمُتَمَرِّدِينَ عَلَى النُّورِ. لاَ يَعْرِفُونَ طُرُقَهُ وَلاَ يَلْبَثُونَ فِي سُبُلِهِ.
\par 14 مَعَ النُّورِ يَقُومُ الْقَاتِلُ. يَقْتُلُ الْمِسْكِينَ وَالْفَقِيرَ وَفِي اللَّيْلِ يَكُونُ كَاللِّصِّ.
\par 15 وَعَيْنُ الزَّانِي تُلاَحِظُ الْعِشَاءَ. يَقُولُ: لاَ تُرَاقِبُنِي عَيْنٌ. فَيَجْعَلُ سِتْراً عَلَى وَجْهِهِ.
\par 16 يَنْقُبُونَ الْبُيُوتَ فِي الظَّلاَمِ. فِي النَّهَارِ يُغْلِقُونَ عَلَى أَنْفُسِهِمْ. لاَ يَعْرِفُونَ النُّورَ.
\par 17 لأَنَّهُ سَوَاءٌ عَلَيْهِمُ الصَّبَاحُ وَظِلُّ الْمَوْتِ. لأَنَّهُمْ يَعْلَمُونَ أَهْوَالَ ظِلِّ الْمَوْتِ.
\par 18 خَفِيفٌ هُوَ عَلَى وَجْهِ الْمِيَاهِ. مَلْعُونٌ نَصِيبُهُمْ فِي الأَرْضِ. لاَ يَتَوَجَّهُ إِلَى طَرِيقِ الْكُرُومِ.
\par 19 الْقَحْطُ وَالْقَيْظُ يَذْهَبَانِ بِمِيَاهِ الثَّلْجِ كَذَا الْهَاوِيَةُ بِالَّذِينَ أَخْطَأُوا.
\par 20 تَنْسَاهُ الرَّحِمُ يَسْتَحْلِيهِ الدُّودُ. لاَ يُذْكَرُ بَعْدُ وَيَنْكَسِرُ الأَثِيمُ كَشَجَرَةٍ.
\par 21 يُسِيءُ إِلَى الْعَاقِرِ الَّتِي لَمْ تَلِدْ وَلاَ يُحْسِنُ إِلَى الأَرْمَلَةِ.
\par 22 يُمْسِكُ الأَعِزَّاءَ بِقُوَّتِهِ. يَقُومُ فَلاَ يَأْمَنُ أَحَدٌ بِحَيَاتِهِ.
\par 23 يُعْطِيهِ طُمَأْنِينَةً فَيَتَوَكَّلُ وَلَكِنْ عَيْنَاهُ عَلَى طُرُقِهِمْ.
\par 24 يَتَرَفَّعُونَ قَلِيلاً ثُمَّ لاَ يَكُونُونَ وَيُحَطُّونَ. كَالْكُلِّ يُجْمَعُونَ وَكَرَأْسِ السُّنْبُلَةِ يُقْطَعُونَ.
\par 25 وَإِنْ لَمْ يَكُنْ كَذَا فَمَنْ يُكَذِّبُنِي وَيَجْعَلُ كَلاَمِي لاَ شَيْئاً؟].

\chapter{25}

\par 1 فَأَجَابَ بِلْدَدُ الشُّوحِيُّ:
\par 2 [السُّلْطَانُ وَالْهَيْبَةُ عِنْدَهُ. هُوَ صَانِعُ السَّلاَمِ فِي أَعَالِيهِ.
\par 3 هَلْ مِنْ عَدَدٍ لِجُنُودِهِ وَعَلَى مَنْ لاَ يُشْرِقُ نُورُهُ؟
\par 4 فَكَيْفَ يَتَبَرَّرُ الإِنْسَانُ عِنْدَ اللهِ وَكَيْفَ يَزْكُو مَوْلُودُ الْمَرْأَةِ؟
\par 5 هُوَذَا نَفْسُ الْقَمَرِ لاَ يُضِيءُ وَالْكَوَاكِبُ غَيْرُ نَقِيَّةٍ فِي عَيْنَيْهِ.
\par 6 فَكَمْ بِالْحَرِيِّ الإِنْسَانُ الرِّمَّةُ وَابْنُ آدَمَ الدُّودُ].

\chapter{26}

\par 1 فَقَالَ أَيُّوبُ:
\par 2 [كَيْفَ أَعَنْتَ مَنْ لاَ قُوَّةَ لَهُ وَخَلَّصْتَ ذِرَاعاً لاَ عِزَّ لَهَا؟
\par 3 كَيْفَ أَشَرْتَ عَلَى مَنْ لاَ حِكْمَةَ لَهُ وَأَظْهَرْتَ الْفَهْمَ بِكَثْرَةٍ؟
\par 4 لِمَنْ أَعْلَنْتَ أَقْوَالاً وَنَسَمَةُ مَنْ خَرَجَتْ مِنْكَ؟
\par 5 [اَلأَرْوَاحُ تَرْتَعِدُ مِنْ تَحْتِ الْمِيَاهِ وَسُكَّانِهَا.
\par 6 الْهَاوِيَةُ عُرْيَانَةٌ قُدَّامَهُ وَالْهَلاَكُ لَيْسَ لَهُ غِطَاءٌ.
\par 7 يَمُدُّ الشَّمَالَ عَلَى الْخَلاَءِ وَيُعَلِّقُ الأَرْضَ عَلَى لاَ شَيْءٍ.
\par 8 يَصُرُّ الْمِيَاهَ فِي سُحُبِهِ فَلاَ يَتَمَزَّقُ الْغَيْمُ تَحْتَهَا.
\par 9 يَحْجِبُ وَجْهَ كُرْسِيِّهِ بَاسِطاً عَلَيْهِ سَحَابَهُ.
\par 10 رَسَمَ حَدّاً عَلَى وَجْهِ الْمِيَاهِ عِنْدَ اتِّصَالِ النُّورِ بِالظُّلْمَةِ.
\par 11 أَعْمِدَةُ السَّمَاوَاتِ تَرْتَعِدُ وَتَرْتَاعُ مِنْ زَجْرِهِ.
\par 12 بِقُوَّتِهِ يُزْعِجُ الْبَحْرَ وَبِفَهْمِهِ يَسْحَقُ رَهَبَ.
\par 13 بِنَفْخَتِهِ السَّمَاوَاتُ مُشْرِقَةٌ وَيَدَاهُ أَبْدَأَتَا الْحَيَّةَ الْهَارِبَةَ.
\par 14 هَا هَذِهِ أَطْرَافُ طُرُقِهِ وَمَا أَخْفَضَ الْكَلاَمَ الَّذِي نَسْمَعُهُ مِنْهُ! وَأَمَّا رَعْدُ جَبَرُوتِهِ فَمَنْ يَفْهَمُ؟].

\chapter{27}

\par 1 وَعَادَ أَيُّوبُ يَنْطِقُ بِمَثَلِهِ فَقَالَ:
\par 2 [حَيٌّ هُوَ اللهُ الَّذِي نَزَعَ حَقِّي وَالْقَدِيرُ الَّذِي أَمَرَّ نَفْسِي
\par 3 إِنَّهُ مَا دَامَتْ نَسَمَتِي فِيَّ وَنَفْخَةُ اللهِ فِي أَنْفِي
\par 4 لَنْ تَتَكَلَّمَ شَفَتَايَ إِثْماً وَلاَ يَلْفِظَ لِسَانِي بِغِشٍّ.
\par 5 حَاشَا لِي أَنْ أُبَرِّرَكُمْ! حَتَّى أُسْلِمَ الرُّوحَ لاَ أَعْزِلُ كَمَالِي عَنِّي.
\par 6 تَمَسَّكْتُ بِبِرِّي وَلاَ أَرْخِيهِ. قَلْبِي لاَ يُعَيِّرُ يَوْماً مِنْ أَيَّامِي.
\par 7 لِيَكُنْ عَدُوِّي كَالشِّرِّيرِ وَمُعَانِدِي كَفَاعِلِ الشَّرِّ.
\par 8 لأَنَّهُ مَا هُوَ رَجَاءُ الْفَاجِرِ عِنْدَمَا يَقْطَعُهُ عِنْدَمَا يَسْلِبُ اللهُ نَفْسَهُ؟
\par 9 أَفَيَسْمَعُ اللهُ صُرَاخَهُ إِذَا جَاءَ عَلَيْهِ ضِيقٌ؟
\par 10 أَمْ يَتَلَذَّذُ بِالْقَدِيرِ؟ هَلْ يَدْعُو اللهَ فِي كُلِّ حِينٍ؟
\par 11 [إِنِّي أُعَلِّمُكُمْ بِيَدِ اللهِ. لاَ أَكْتُمُ مَا هُوَ عِنْدَ الْقَدِيرِ.
\par 12 هَا أَنْتُمْ كُلُّكُمْ قَدْ رَأَيْتُمْ فَلِمَاذَا تَتَبَطَّلُونَ تَبَطُّلاً قَائِلِينَ:
\par 13 هَذَا نَصِيبُ الإِنْسَانِ الشِّرِّيرِ مِنْ عِنْدِ اللهِ وَمِيرَاثُ الْعُتَاةِ الَّذِي يَنَالُونَهُ مِنَ الْقَدِيرِ.
\par 14 إِنْ كَثُرَ بَنُوهُ فَلِلسَّيْفِ وَذُرِّيَّتُهُ لاَ تَشْبَعُ خُبْزاً.
\par 15 بَقِيَّتُهُ تُدْفَنُ بِالْوَبَاءِ وَأَرَامِلُهُ لاَ تَبْكِي.
\par 16 إِنْ كَنَزَ فِضَّةً كَالتُّرَابِ وَأَعَدَّ مَلاَبِسَ كَالطِّينِ
\par 17 فَهُوَ يُعِدُّ وَالْبَارُّ يَلْبِسُهُ وَالْبَرِئُ يَقْسِمُ الْفِضَّةَ.
\par 18 يَبْنِي بَيْتَهُ كَالْعُثِّ أَوْ كَمِظَلَّةٍ صَنَعَهَا الْحَارِسُ.
\par 19 يَضْطَجِعُ غَنِيّاً وَلَكِنَّهُ لاَ يُضَمُّ. يَفْتَحُ عَيْنَيْهِ وَلاَ يَكُونُ.
\par 20 الأَهْوَالُ تُدْرِكُهُ كَالْمِيَاهِ. لَيْلاً تَخْتَطِفُهُ الزَّوْبَعَةُ
\par 21 تَحْمِلُهُ الشَّرْقِيَّةُ فَيَذْهَبُ وَتَجْرُفُهُ مِنْ مَكَانِهِ.
\par 22 يُلْقِي اللهُ عَلَيْهِ وَلاَ يُشْفِقُ. مِنْ يَدِهِ يَهْرُبُ هَرْباً.
\par 23 يَصْفِقُونَ عَلَيْهِ بِأَيْدِيهِمْ وَيَصْفِرُونَ عَلَيْهِ مِنْ مَكَانِهِ.

\chapter{28}

\par 1 [لأَنَّهُ يُوجَدُ لِلْفِضَّةِ مَعْدَنٌ وَمَوْضِعٌ لِلذَّهَبِ حَيْثُ يُمَحِّصُونَهُ.
\par 2 الْحَدِيدُ يُسْتَخْرَجُ مِنَ التُّرَابِ وَالْحَجَرُ يَسْكُبُ نُحَاساً.
\par 3 قَدْ جَعَلَ لِلظُّلْمَةِ نِهَايَةً وَإِلَى كُلِّ طَرَفٍ هُوَ يَفْحَصُ. حَجَرَ الظُّلْمَةِ وَظِلَّ الْمَوْتِ.
\par 4 حَفَرَ مَنْجَماً بَعِيداً عَنِ السُّكَّانِ. بِلاَ مَوْطِئٍ لِلْقَدَمِ. مُتَدَلِّينَ بَعِيدِينَ مِنَ النَّاسِ يَتَدَلْدَلُونَ.
\par 5 أَرْضٌ يَخْرُجُ مِنْهَا الْخُبْزُ أَسْفَلُهَا يَنْقَلِبُ كَمَا بِالنَّارِ.
\par 6 حِجَارَتُهَا هِيَ مَوْضِعُ الْيَاقُوتِ الأَزْرَقِ وَفِيهَا تُرَابُ الذَّهَبِ.
\par 7 سَبِيلٌ لَمْ يَعْرِفْهُ كَاسِرٌ وَلَمْ تُبْصِرْهُ عَيْنُ بَاشِقٍ
\par 8 وَلَمْ تَدُسْهُ أَجْرَاءُ السَّبْعِ وَلَمْ يَسْلُكْهُ الأَسَدُ.
\par 9 إِلَى الصَّوَّانِ يَمُدُّ يَدَهُ. يَقْلِبُ الْجِبَالَ مِنْ أُصُولِهَا.
\par 10 يَنْقُرُ فِي الصُّخُورِ سَرَباً وَعَيْنُهُ تَرَى كُلَّ ثَمِينٍ.
\par 11 يَمْنَعُ رَشْحَ الأَنْهَارِ وَأَبْرَزَ الْخَفِيَّاتِ إِلَى النُّورِ.
\par 12 [أَمَّا الْحِكْمَةُ فَمِنْ أَيْنَ تُوجَدُ وَأَيْنَ هُوَ مَكَانُ الْفَهْمِ؟
\par 13 لاَ يَعْرِفُ الإِنْسَانُ قِيمَتَهَا وَلاَ تُوجَدُ فِي أَرْضِ الأَحْيَاءِ.
\par 14 الْغَمْرُ يَقُولُ: لَيْسَتْ هِيَ فِيَّ وَالْبَحْرُ يَقُولُ: لَيْسَتْ هِيَ عِنْدِي.
\par 15 لاَ يُعْطَى ذَهَبٌ خَالِصٌ بَدَلَهَا وَلاَ تُوزَنُ فِضَّةٌ ثَمَناً لَهَا.
\par 16 لاَ تُوزَنُ بِذَهَبِ أُوفِيرَ أَوْ بِالْجَزْعِ الْكَرِيمِ أَوِ الْيَاقُوتِ الأَزْرَقِ.
\par 17 لاَ يُعَادِلُهَا الذَّهَبُ وَلاَ الزُّجَاجُ وَلاَ تُبْدَلُ بِإِنَاءِ ذَهَبٍ إِبْرِيزٍ.
\par 18 لاَ يُذْكَرُ الْمَرْجَانُ أَوِ الْبَلُّوْرُ وَتَحْصِيلُ الْحِكْمَةِ خَيْرٌ مِنَ اللَّآلِئِ.
\par 19 لاَ يُعَادِلُهَا يَاقُوتُ كُوشَ الأَصْفَرُ وَلاَ تُوزَنُ بِالذَّهَبِ الْخَالِصِ.
\par 20 [فَمِنْ أَيْنَ تَأْتِي الْحِكْمَةُ وَأَيْنَ هُوَ مَكَانُ الْفَهْمِ.
\par 21 إِذْ أُخْفِيَتْ عَنْ عُيُونِ كُلِّ حَيٍّ وَسُتِرَتْ عَنْ طَيْرِ السَّمَاءِ؟
\par 22 اَلْهَلاَكُ وَالْمَوْتُ يَقُولاَنِ: بِآذَانِنَا قَدْ سَمِعْنَا خَبَرَهَا.
\par 23 اَللهُ يَفْهَمُ طَرِيقَهَا وَهُوَ عَالِمٌ بِمَكَانِهَا.
\par 24 لأَنَّهُ هُوَ يَنْظُرُ إِلَى أَقَاصِي الأَرْضِ. تَحْتَ كُلِّ السَّمَاوَاتِ يَرَى.
\par 25 لِيَجْعَلَ لِلرِّيحِ وَزْناً وَيُعَايِرَ الْمِيَاهَ بِمِقْيَاسٍ.
\par 26 لَمَّا جَعَلَ لِلْمَطَرِ فَرِيضَةً وَسَبِيلاً لِلصَّوَاعِقِ
\par 27 حِينَئِذٍ رَآهَا وَأَخْبَرَ بِهَا هَيَّأَهَا وَأَيْضاً بَحَثَ عَنْهَا
\par 28 وَقَالَ لِلإِنْسَانِ: هُوَذَا مَخَافَةُ الرَّبِّ هِيَ الْحِكْمَةُ وَالْحَيَدَانُ عَنِ الشَّرِّ هُوَ الْفَهْمُ].

\chapter{29}

\par 1 وَعَادَ أَيُّوبُ يَنْطِقُ بِمَثَلِهِ فَقَالَ:
\par 2 [يَا لَيْتَنِي كَمَا فِي الشُّهُورِ السَّالِفَةِ وَكَالأَيَّامِ الَّتِي حَفِظَنِي اللهُ فِيهَا
\par 3 حِينَ أَضَاءَ سِرَاجَهُ عَلَى رَأْسِي وَبِنُورِهِ سَلَكْتُ الظُّلْمَةَ.
\par 4 كَمَا كُنْتُ فِي أَيَّامِ خَرِيفِي وَرِضَا اللهِ عَلَى خَيْمَتِي
\par 5 وَالْقَدِيرُ بَعْدُ مَعِي وَحَوْلِي غِلْمَانِي
\par 6 إِذْ غَسَلْتُ خَطَوَاتِي بِاللَّبَنِ وَالصَّخْرُ سَكَبَ لِي جَدَاوِلَ زَيْتٍ.
\par 7 حِينَ كُنْتُ أَخْرُجُ إِلَى الْبَابِ فِي الْقَرْيَةِ وَأُهَيِّئُ فِي السَّاحَةِ مَجْلِسِي.
\par 8 رَآنِي الْغِلْمَانُ فَاخْتَبَأُوا وَالأَشْيَاخُ قَامُوا وَوَقَفُوا.
\par 9 الْعُظَمَاءُ أَمْسَكُوا عَنِ الْكَلاَمِ وَوَضَعُوا أَيْدِيَهُمْ عَلَى أَفْوَاهِهِمْ.
\par 10 صَوْتُ الشُّرَفَاءِ اخْتَفَى وَلَصِقَتْ أَلْسِنَتُهُمْ بِأَحْنَاكِهِمْ.
\par 11 لأَنَّ الأُذُنَ سَمِعَتْ فَطَوَّبَتْنِي وَالْعَيْنَ رَأَتْ فَشَهِدَتْ لِي.
\par 12 لأَنِّي أَنْقَذْتُ الْمِسْكِينَ الْمُسْتَغِيثَ وَالْيَتِيمَ وَلاَ مُعِينَ لَهُ.
\par 13 بَرَكَةُ الْهَالِكِ حَلَّتْ عَلَيَّ وَجَعَلْتُ قَلْبَ الأَرْمَلَةِ يُسَرُّ.
\par 14 لَبِسْتُ الْبِرَّ فَكَسَانِي. كَجُبَّةٍ وَعَمَامَةٍ كَانَ عَدْلِي.
\par 15 كُنْتُ عُيُوناً لِلْعُمْيِ وَأَرْجُلاً لِلْعُرْجِ.
\par 16 أَبٌ أَنَا لِلْفُقَرَاءِ وَدَعْوَى لَمْ أَعْرِفْهَا فَحَصْتُ عَنْهَا.
\par 17 هَشَّمْتُ أَضْرَاسَ الظَّالِمِ وَمِنْ بَيْنِ أَسْنَانِهِ خَطَفْتُ الْفَرِيسَةَ.
\par 18 فَقُلْتُ: إِنِّي فِي وَكْرِي أُسَلِّمُ الرُّوحَ وَمِثْلَ السَّمَنْدَلِ أُكَثِّرُ أَيَّاماً.
\par 19 أَصْلِي كَانَ مُنْبَسِطاً إِلَى الْمِيَاهِ وَالطَّلُّ بَاتَ عَلَى أَغْصَانِي.
\par 20 كَرَامَتِي بَقِيَتْ حَدِيثَةً عِنْدِي وَقَوْسِي تَجَدَّدَتْ فِي يَدِي.
\par 21 لِي سَمِعُوا وَانْتَظَرُوا وَنَصَتُوا عِنْدَ مَشُورَتِي.
\par 22 بَعْدَ كَلاَمِي لَمْ يُثَنُّوا وَقَوْلِي قَطَرَ عَلَيْهِمْ.
\par 23 وَانْتَظَرُونِي مِثْلَ الْمَطَرِ وَفَغَرُوا أَفْوَاهَهُمْ كَمَا لِلْمَطَرِ الْمُتَأَخِّرِ.
\par 24 إِنْ ضَحِكْتُ عَلَيْهِمْ لَمْ يُصَدِّقُوا وَنُورَ وَجْهِي لَمْ يُعَبِّسُوا.
\par 25 كُنْتُ أَخْتَارُ طَرِيقَهُمْ وَأَجْلِسُ رَأْساً وَأَسْكُنُ كَمَلِكٍ فِي جَيْشٍ كَمَنْ يُعَزِّي النَّائِحِينَ.

\chapter{30}

\par 1 [وَأَمَّا الآنَ فَقَدْ ضَحِكَ عَلَيَّ مَنْ يَصْغُرُنِي فِي الأَيَّامِ الَّذِينَ كُنْتُ أَسْتَنْكِفُ مِنْ أَنْ أَجْعَلَ آبَاءَهُمْ مَعَ كِلاَبِ غَنَمِي.
\par 2 قُوَّةُ أَيْدِيهِمْ أَيْضاً مَا هِيَ لِي. فِيهِمْ عَجِزَتِ الشَّيْخُوخَةُ.
\par 3 فِي الْعَوَزِ وَالْمَجَاعَةِ مَهْزُولُونَ يَنْبِشُونَ الْيَابِسَةَ الَّتِي هِيَ مُنْذُ أَمْسِ خَرَابٌ وَخَرِبَةٌ.
\par 4 الَّذِينَ يَقْطِفُونَ الْمَلاَّحَ عِنْدَ الشِّيحِ وَأُصُولُ الرَّتَمِ خُبْزُهُمْ.
\par 5 مِنَ الْوَسَطِ يُطْرَدُونَ. يَصِيحُونَ عَلَيْهِمْ كَمَا عَلَى لِصٍّ.
\par 6 لِلسَّكَنِ فِي أَوْدِيَةٍ مُرْعِبَةٍ وَثُقَبِ التُّرَابِ وَالصُّخُورِ.
\par 7 بَيْنَ الشِّيحِ يَنْهَقُونَ. تَحْتَ الْعَوْسَجِ يَنْكَبُّونَ.
\par 8 أَبْنَاءُ الْحَمَاقَةِ بَلْ أَبْنَاءُ أُنَاسٍ بِلاَ اسْمٍ دُحِرُوا مِنَ الأَرْضِ.
\par 9 [أَمَّا الآنَ فَصِرْتُ أُغْنِيَتَهُمْ وَأَصْبَحْتُ لَهُمْ مَثَلاً!
\par 10 يَكْرَهُونَنِي. يَبْتَعِدُونَ عَنِّي وَأَمَامَ وَجْهِي لَمْ يُمْسِكُوا عَنِ الْبَصْقِ.
\par 11 لأَنَّهُ أَطْلَقَ الْعَنَانَ وَقَهَرَنِي فَنَزَعُوا الزِّمَامَ قُدَّامِي.
\par 12 عَنِ الْيَمِينِ السَّفَلَةُ يَقُومُونَ يُزِيحُونَ رِجْلِي وَيُعِدُّونَ عَلَيَّ طُرُقَهُمْ لِلْبَوَارِ.
\par 13 أَفْسَدُوا سُبُلِي. أَعَانُوا عَلَى سُقُوطِي. لاَ مُسَاعِدَ عَلَيْهِمْ.
\par 14 يَأْتُونَ كَصَدْعٍ عَرِيضٍ. تَحْتَ الْهَدَّةِ يَتَدَحْرَجُونَ.
\par 15 اِنْقَلَبَتْ عَلَيَّ أَهْوَالٌ. طَرَدَتْ كَالرِّيحِ نِعْمَتِي فَعَبَرَتْ كَالسَّحَابِ سَعَادَتِي.
\par 16 [فَالآنَ انْهَالَتْ نَفْسِي عَلَيَّ وَأَخَذَتْنِي أَيَّامُ الْمَذَلَّةِ.
\par 17 اللَّيْلَ يَنْخَرُ عِظَامِي فِيَّ وَعَارِقِيَّ لاَ تَهْجَعُ.
\par 18 بِكَثْرَةِ الشِّدَّةِ تَنَكَّرَ لِبْسِي. مِثْلَ جَيْبِ قَمِيصِي حَزَمَتْنِي.
\par 19 قَدْ طَرَحَنِي فِي الْوَحْلِ فَأَشْبَهْتُ التُّرَابَ وَالرَّمَادَ.
\par 20 إِلَيْكَ أَصْرُخُ فَمَا تَسْتَجِيبُ لِي. أَقُومُ فَمَا تَنْتَبِهُ إِلَيَّ.
\par 21 تَحَوَّلْتَ إِلَى جَافٍ مِنْ نَحْوِي. بِقُدْرَةِ يَدِكَ تَضْطَهِدُنِي.
\par 22 حَمَلْتَنِي أَرْكَبْتَنِي الرِّيحَ وَذَوَّبْتَنِي تَشَوُّهاً.
\par 23 لأَنِّي أَعْلَمُ أَنَّكَ إِلَى الْمَوْتِ تُعِيدُنِي وَإِلَى بَيْتِ مِيعَادِ كُلِّ حَيٍّ.
\par 24 وَلَكِنْ فِي الْخَرَابِ أَلاَ يَمُدُّ يَداً؟ فِي الْبَلِيَّةِ أَلاَ يَسْتَغِيثُ عَلَيْهَا؟
\par 25 [أَلَمْ أَبْكِ لِمَنْ عَسَرَ يَوْمُهُ؟ أَلَمْ تَكْتَئِبْ نَفْسِي عَلَى الْمِسْكِينِ؟
\par 26 حِينَمَا تَرَجَّيْتُ الْخَيْرَ جَاءَ الشَّرُّ وَانْتَظَرْتُ النُّورَ فَجَاءَ الدُّجَى.
\par 27 أَمْعَائِي تَغْلِي وَلاَ تَكُفُّ. تَقَدَّمَتْنِي أَيَّامُ الْمَذَلَّةِ.
\par 28 اِسْوَدَدْتُ لَكِنْ بِلاَ شَمْسٍ. قُمْتُ فِي الْجَمَاعَةِ أَصْرُخُ.
\par 29 صِرْتُ أَخاً لِلذِّئَابِ وَصَاحِباً لِلنَّعَامِ.
\par 30 إِسْوَدَّ جِلْدِي عَلَيَّ وَعِظَامِي احْتَرَقَتْ مِنَ الْحُمَّى فِيَّ.
\par 31 صَارَ عُودِي لِلنَّوْحِ وَمِزْمَارِي لِصَوْتِ الْبَاكِينَ.

\chapter{31}

\par 1 [عَهْداً قَطَعْتُ لِعَيْنَيَّ فَكَيْفَ أَتَطَلَّعُ فِي عَذْرَاءَ!
\par 2 وَمَا هِيَ قِسْمَةُ اللهِ مِنْ فَوْقُ وَنَصِيبُ الْقَدِيرِ مِنَ الأَعَالِي؟
\par 3 أَلَيْسَ الْبَوَارُ لِعَامِلِ الشَّرِّ وَالنُّكْرُ لِفَاعِلِي الإِثْمِ!
\par 4 أَلَيْسَ هُوَ يَنْظُرُ طُرُقِي وَيُحْصِي جَمِيعَ خَطَوَاتِي.
\par 5 إِنْ كُنْتُ قَدْ سَلَكْتُ مَعَ الْكَذِبِ أَوْ أَسْرَعَتْ رِجْلِي إِلَى الْغِشِّ
\par 6 لِيَزِنِّي فِي مِيزَانِ الْحَقِّ فَيَعْرِفَ اللهُ كَمَالِي.
\par 7 إِنْ حَادَتْ خَطَوَاتِي عَنِ الطَّرِيقِ وَذَهَبَ قَلْبِي وَرَاءَ عَيْنَيَّ أَوْ لَصِقَ عَيْبٌ بِكَفِّي
\par 8 أَزْرَعْ وَغَيْرِي يَأْكُلْ وَفُرُوعِي تُسْتَأْصَلْ.
\par 9 [إِنْ غَوِيَ قَلْبِي عَلَى امْرَأَةٍ أَوْ كَمَنْتُ عَلَى بَابِ قَرِيبِي
\par 10 فَلْتَطْحَنِ امْرَأَتِي لِآخَرَ وَلْيَنْحَنِ عَلَيْهَا آخَرُونَ.
\par 11 لأَنَّ هَذِهِ رَذِيلَةٌ وَهِيَ إِثْمٌ يُعْرَضُ لِلْقُضَاةِ.
\par 12 لأَنَّهَا نَارٌ تَأْكُلُ حَتَّى إِلَى الْهَلاَكِ وَتَسْتَأْصِلُ كُلَّ مَحْصُولِي.
\par 13 [إِنْ كُنْتُ رَفَضْتُ حَقَّ عَبْدِي وَأَمَتِي فِي دَعْوَاهُمَا عَلَيَّ
\par 14 فَمَاذَا كُنْتُ أَصْنَعُ حِينَ يَقُومُ اللهُ؟ وَإِذَا افْتَقَدَ فَبِمَاذَا أُجِيبُهُ؟
\par 15 أَوَلَيْسَ صَانِعِي فِي الْبَطْنِ صَانِعَهُ وَقَدْ صَوَّرَنَا وَاحِدٌ فِي الرَّحِمِ؟
\par 16 إِنْ كُنْتُ مَنَعْتُ الْمَسَاكِينَ عَنْ مُرَادِهِمْ أَوْ أَفْنَيْتُ عَيْنَيِ الأَرْمَلَةِ
\par 17 أَوْ أَكَلْتُ لُقْمَتِي وَحْدِي فَمَا أَكَلَ مِنْهَا الْيَتِيمُ!
\par 18 بَلْ مُنْذُ صِبَايَ كَبِرَ عِنْدِي كَأَبٍ وَمِنْ بَطْنِ أُمِّي هَدَيْتُهَا.
\par 19 إِنْ كُنْتُ رَأَيْتُ هَالِكاً لِعَدَمِ اللِّبْسِ أَوْ فَقِيراً بِلاَ كِسْوَةٍ
\par 20 إِنْ لَمْ تُبَارِكْنِي حَقَوَاهُ وَقَدِ اسْتَدْفَأَ بِجَزَّةِ غَنَمِي.
\par 21 إِنْ كُنْتُ قَدْ هَزَزْتُ يَدِي عَلَى الْيَتِيمِ لَمَّا رَأَيْتُ عَوْنِي فِي الْبَابِ
\par 22 فَلْتَسْقُطْ عَضُدِي مِنْ كَتِفِي وَلْتَنْكَسِرْ ذِرَاعِي مِنْ قَصَبَتِهَا
\par 23 لأَنَّ الْبَوَارَ مِنَ اللهِ رُعْبٌ عَلَيَّ وَمِنْ جَلاَلِهِ لَمْ أَسْتَطِعْ.
\par 24 [إِنْ كُنْتُ قَدْ جَعَلْتُ الذَّهَبَ عُمْدَتِي أَوْ قُلْتُ لِلإِبْرِيزِ: أَنْتَ مُتَّكَلِي.
\par 25 إِنْ كُنْتُ قَدْ فَرِحْتُ إِذْ كَثُرَتْ ثَرْوَتِي وَلأَنَّ يَدِي وَجَدَتْ كَثِيراً.
\par 26 إِنْ كُنْتُ قَدْ نَظَرْتُ إِلَى النُّورِ حِينَ ضَاءَ أَوْ إِلَى الْقَمَرِ يَسِيرُ بِالْبَهَاءِ
\par 27 وَغَوِيَ قَلْبِي سِرّاً وَلَثَمَ يَدِي فَمِي
\par 28 فَهَذَا أَيْضاً إِثْمٌ يُعْرَضُ لِلْقُضَاةِ لأَنِّي أَكُونُ قَدْ جَحَدْتُ اللهَ مِنْ فَوْقُ.
\par 29 [إِنْ كُنْتُ قَدْ فَرِحْتُ بِبَلِيَّةِ مُبْغِضِي أَوْ شَمِتُّ حِينَ أَصَابَهُ سُوءٌ.
\par 30 بَلْ لَمْ أَدَعْ حَنَكِي يُخْطِئُ فِي طَلَبِ نَفْسِهِ بِلَعْنَةٍ.
\par 31 إِنْ كَانَ أَهْلُ خَيْمَتِي لَمْ يَقُولُوا: مَنْ يَأْتِي بِأَحَدٍ لَمْ يَشْبَعْ مِنْ طَعَامِهِ؟
\par 32 غَرِيبٌ لَمْ يَبِتْ فِي الْخَارِجِ. فَتَحْتُ لِلْمُسَافِرِ أَبْوَابِي.
\par 33 إِنْ كُنْتُ قَدْ كَتَمْتُ كَالنَّاسِ ذَنْبِي لإِخْفَاءِ إِثْمِي فِي حِضْنِي.
\par 34 إِذْ رَهِبْتُ جُمْهُوراً غَفِيراً وَرَوَّعَتْنِي إِهَانَةُ الْعَشَائِرِ فَكَفَفْتُ وَلَمْ أَخْرُجْ مِنَ الْبَابِ!
\par 35 مَنْ لِي بِمَنْ يَسْمَعُنِي؟ هُوَذَا إِمْضَائِي. لِيُجِبْنِي الْقَدِيرُ. وَمَنْ لِي بِشَكْوَى كَتَبَهَا خَصْمِي
\par 36 فَكُنْتُ أَحْمِلُهَا عَلَى كَتِفِي. كُنْتُ أُعْصِبُهَا تَاجاً لِي.
\par 37 كُنْتُ أُخْبِرُهُ بِعَدَدِ خَطَوَاتِي وَأَدْنُو مِنْهُ كَشَرِيفٍ.
\par 38 إِنْ كَانَتْ أَرْضِي قَدْ صَرَخَتْ عَلَيَّ وَتَبَاكَتْ أَتْلاَمُهَا جَمِيعاً.
\par 39 إِنْ كُنْتُ قَدْ أَكَلْتُ غَلَّتَهَا بِلاَ فِضَّةٍ أَوْ أَطْفَأْتُ أَنْفُسَ أَصْحَابِهَا
\par 40 فَعِوَضَ الْحِنْطَةِ لِيَنْبُتْ شَوْكٌ وَبَدَلَ الشَّعِيرِ زَوَانٌ]. تَمَّتْ أَقْوَالُ أَيُّوبَ.

\chapter{32}

\par 1 فَكَفَّ هَؤُلاَءِ الرِّجَالُ الثَّلاَثَةُ عَنْ مُجَاوَبَةِ أَيُّوبَ لِكَوْنِهِ بَارّاً فِي عَيْنَيْ نَفْسِهِ.
\par 2 فَحَمِيَ غَضَبُ أَلِيهُوَ بْنِ بَرَخْئِيلَ الْبُوزِيِّ مِنْ عَشِيرَةِ رَامٍ. عَلَى أَيُّوبَ حَمِيَ غَضَبُهُ لِأنَّهُ حَسَبَ نَفْسَهُ أَبَرَّ مِنَ اللهِ.
\par 3 وَعَلَى أَصْحَابِهِ الثَّلاَثَةِ حَمِيَ غَضَبُهُ لأَنَّهُمْ لَمْ يَجِدُوا جَوَاباً وَاسْتَذْنَبُوا أَيُّوبَ.
\par 4 وَكَانَ أَلِيهُو قَدْ صَبِرَ عَلَى أَيُّوبَ بِالْكَلاَمِ لأَنَّهُمْ أَكْثَرُ مِنْهُ أَيَّاماً.
\par 5 فَلَمَّا رَأَى أَلِيهُو أَنَّهُ لاَ جَوَابَ فِي أَفْوَاهِ الرِّجَالِ الثَّلاَثَةِ حَمِيَ غَضَبُهُ.
\par 6 فَقَالَ أَلِيهُو بْنُ بَرَخْئِيلَ الْبُوزِيُّ: [أَنَا صَغِيرٌ فِي الأَيَّامِ وَأَنْتُمْ شُيُوخٌ لأَجْلِ ذَلِكَ خِفْتُ وَخَشِيتُ أَنْ أُبْدِيَ لَكُمْ رَأْيِيِ.
\par 7 قُلْتُ: الأَيَّامُ تَتَكَلَّمُ وَكَثْرَةُ السِّنِينَِ تُظْهِرُ حِكْمَةً.
\par 8 وَلَكِنَّ فِي النَّاسِ رُوحاً وَنَسَمَةُ الْقَدِيرِ تُعَقِّلُهُمْ.
\par 9 لَيْسَ الْكَثِيرُو الأَيَّامِ حُكَمَاءَ وَلاَ الشُّيُوخُ يَفْهَمُونَ الْحَقَّ.
\par 10 لِذَلِكَ قُلْتُ اسْمَعُونِي. أَنَا أَيْضاً أُبْدِي رَأْيِيِ.
\par 11 هَئَنَذَا قَدْ صَبِرْتُ لِكَلاَمِكُمْ. أَصْغَيْتُ إِلَى حُجَجِكُمْ حَتَّى فَحَصْتُمُ الأَقْوَالَ.
\par 12 فَتَأَمَّلْتُ فِيكُمْ وَإِذْ لَيْسَ مَنْ حَجَّ أَيُّوبَ وَلاَ جَوَابَ مِنْكُمْ لِكَلاَمِهِ.
\par 13 فَلاَ تَقُولُوا: قَدْ وَجَدْنَا حِكْمَةً. اللهُ يَغْلِبُهُ لاَ الإِنْسَانُ.
\par 14 فَإِنَّهُ لَمْ يُوَجِّهْ إِلَيَّ كَلاَمَهُ وَلاَ أَرُدُّ عَلَيْهِ أَنَا بِكَلاَمِكُمْ.
\par 15 تَحَيَّرُوا. لَمْ يُجِيبُوا بَعْدُ. انْتَزَعَ عَنْهُمُ الْكَلاَمُ.
\par 16 فَانْتَظَرْتُ لأَنَّهُمْ لَمْ يَتَكَلَّمُوا. لأَنَّهُمْ وَقَفُوا لَمْ يُجِيبُوا بَعْدُ.
\par 17 فَأُجِيبُ أَنَا أَيْضاً حِصَّتِي وَأُبْدِي أَنَا أَيْضاً رَأْيِيِ.
\par 18 لأَنِّي مَلآنٌ أَقْوَالاً. رُوحُ بَاطِنِي تُضَايِقُنِي.
\par 19 هُوَذَا بَطْنِي كَخَمْرٍ لَمْ تُفْتَحْ. كَالزِّقَاقِ الْجَدِيدَةِ يَكَادُ يَنْشَقُّ.
\par 20 أَتَكَلَّمُ فَأُفْرَجُ. أَفْتَحُ شَفَتَيَّ وَأُجِيبُ.
\par 21 لاَ أُحَابِيَنَّ وَجْهَ رَجُلٍ وَلاَ أَتَمَلَّقُ إِنْسَاناً.
\par 22 لأَنِّي لاَ أَعْرِفُ التَّمَلُّقُ. لأَنَّهُ عَنْ قَلِيلٍ يَأْخُذُنِي صَانِعِي.

\chapter{33}

\par 1 [وَلَكِنِ اسْمَعِ الآنَ يَا أَيُّوبُ أَقْوَالِي وَاصْغَ إِلَى كُلِّ كَلاَمِي.
\par 2 هَئَنَذَا قَدْ فَتَحْتُ فَمِي. لِسَانِي نَطَقَ فِي حَنَكِي.
\par 3 اِسْتِقَامَةُ قَلْبِي كَلاَمِي وَمَعْرِفَةُ شَفَتَيَّ هُمَا تَنْطِقَانِ بِهَا خَالِصَةً.
\par 4 رُوحُ اللهِ صَنَعَنِي وَنَسَمَةُ الْقَدِيرِ أَحْيَتْنِي.
\par 5 إِنِ اسْتَطَعْتَ فَأَجِبْنِي. أَحْسِنِ الدَّعْوَى أَمَامِي. انْتَصِبْ.
\par 6 هَئَنَذَا حَسَبَ قَوْلِكَ عِوَضاً عَنِ اللهِ. أَنَا أَيْضاً مِنَ الطِّينِ جُبِلْتُ.
\par 7 هُوَذَا هَيْبَتِي لاَ تُرْهِبُكَ وَجَلاَلِي لاَ يَثْقُلُ عَلَيْكَ.
\par 8 [إِنَّكَ قد قُلْتَ في مَسَامِعِي وَصَوْتَ أَقْوَالِكَ سَمِعْتُ.
\par 9 قُلْتَ: أَنَا بَرِيءٌ بِلاَ ذَنْبٍ. زَكِيٌّ أَنَا وَلاَ إِثْمَ لِي.
\par 10 هُوَذَا يَطْلُبُ عَلَيَّ عِلَلَ عَدَاوَةٍ. يَحْسِبُنِي عَدُوّاً لَهُ.
\par 11 وَضَعَ رِجْلَيَّ فِي الْمِقْطَرَةِ. يُرَاقِبُ كُلَّ طُرُقِي.
\par 12 [هَا إِنَّكَ فِي هَذَا لَمْ تُصِبْ. أَنَا أُجِيبُكَ. لأَنَّ اللهَ أَعْظَمُ مِنَ الإِنْسَانِ.
\par 13 لِمَاذَا تُخَاصِمُهُ؟ لأَنَّ كُلَّ أُمُورِهِ لاَ يُجَاوِبُ عَنْهَا.
\par 14 لَكِنَّ اللهَ يَتَكَلَّمُ مَرَّةً وَبِاثْنَتَيْنِ لاَ يُلاَحِظُ الإِنْسَانُ.
\par 15 فِي حُلْمٍ فِي رُؤْيَا اللَّيْلِ عِنْدَ سُقُوطِ سُبَاتٍ عَلَى النَّاسِ فِي النُّعَاسِ عَلَى الْمَضْجَعِ.
\par 16 حِينَئِذٍ يَكْشِفُ آذَانَ النَّاسِ وَيَخْتِمُ عَلَى تَأْدِيبِهِمْ
\par 17 لِيُحَوِّلَ الإِنْسَانَ عَنْ عَمَلِهِ وَيَكْتُمَ الْكِبْرِيَاءَ عَنِ الرَّجُلِ
\par 18 لِيَمْنَعَ نَفْسَهُ عَنِ الْحُفْرَةِ وَحَيَاتَهُ مِنَ الزَّوَالِ بِحَرْبَةِ الْمَوْتِ.
\par 19 أَيْضاً يُؤَدَّبُ بِالْوَجَعِ عَلَى مَضْجَعِهِ وَمُخَاصَمَةُ عِظَامِهِ دَائِمَةٌ
\par 20 فَتَكْرَهُ حَيَاتُهُ خُبْزاً وَنَفْسُهُ الطَّعَامَ الشَّهِيَّ.
\par 21 فَيَبْلَى لَحْمُهُ عَنِ الْعَيَانِ وَتَنْبَرِي عِظَامُهُ فَلاَ تُرَى
\par 22 وَتَقْرُبُ نَفْسُهُ إِلَى الْقَبْرِ وَحَيَاتُهُ إِلَى الْمُمِيتِينَ.
\par 23 إِنْ وُجِدَ عِنْدَهُ مُرْسَلٌ وَسِيطٌ وَاحِدٌ مِنْ أَلْفٍ لِيُعْلِنَ لِلإِنْسَانِ اسْتِقَامَتَهُ
\par 24 يَتَرَأَّفُ عَلَيْهِ وَيَقُولُ: أُطْلِقُهُ عَنِ الْهُبُوطِ إِلَى الْحُفْرَةِ قَدْ وَجَدْتُ فِدْيَةً.
\par 25 يَصِيرُ لَحْمُهُ أَنْضَرَ مِنْ لَحْمِ الصَّبِيِّ وَيَعُودُ إِلَى أَيَّامِ شَبَابِهِ.
\par 26 يُصَلِّي إِلَى اللهِ فَيَرْضَى عَنْهُ وَيُعَايِنُ وَجْهَهُ بِهُتَافٍ فَيَرُدُّ عَلَى الإِنْسَانِ بِرَّهُ.
\par 27 يُغَنِّي بَيْنَ النَّاسِ فَيَقُولُ: قَدْ أَخْطَأْتُ وَعَوَّجْتُ الْمُسْتَقِيمَ وَلَمْ أُجَازَ عَلَيْهِ.
\par 28 فَدَى نَفْسِي مِنَ الْعُبُورِ إِلَى الْحُفْرَةِ فَتَرَى حَيَاتِيَ النُّورَ.
\par 29 [هُوَذَا كُلُّ هَذِهِ يَفْعَلُهَا اللهُ مَرَّتَيْنِ وَثَلاَثاً بِالإِنْسَانِ
\par 30 لِيَرُدَّ نَفْسَهُ مِنَ الْحُفْرَةِ لِيَسْتَنِيرَ بِنُورِ الأَحْيَاءِ.
\par 31 فَاصْغَ يَا أَيُّوبُ وَاسْتَمِعْ لِي. انْصُتْ فَأَنَا أَتَكَلَّمُ.
\par 32 إِنْ كَانَ عِنْدَكَ كَلاَمٌ فَأَجِبْنِي. تَكَلَّمْ. فَإِنِّي أُرِيدُ تَبْرِيرَكَ.
\par 33 وَإِلاَّ فَاسْتَمِعْ أَنْتَ لِي. انْصُتْ فَأُعَلِّمَكَ الْحِكْمَةَ].

\chapter{34}

\par 1 وَقَالَ أَلِيهُو:
\par 2 [اسْمَعُوا أَقْوَالِي أَيُّهَا الْحُكَمَاءُ وَاصْغُوا لِي أَيُّهَا الْعَارِفُونَ.
\par 3 لأَنَّ الأُذُنَ تَمْتَحِنُ الأَقْوَالَ كَمَا أَنَّ الْحَنَكَ يَذُوقُ طَعَاماً.
\par 4 لِنَمْتَحِنْ لأَنْفُسِنَا الْحَقَّ وَنَعْرِفْ بَيْنَ أَنْفُسِنَا مَا هُوَ طَيِّبٌ.
\par 5 [لأَنَّ أَيُّوبَ قَالَ: تَبَرَّرْتُ وَاللهُ نَزَعَ حَقِّي.
\par 6 عِنْدَ مُحَاكَمَتِي أُكَذَّبُ. جُرْحِي عَدِيمُ الشِّفَاءِ مِنْ دُونِ ذَنْبٍ.
\par 7 فَأَيُّ إِنْسَانٍ كَأَيُّوبَ يَشْرَبُ الْهُزْءَ كَالْمَاءِ
\par 8 وَيَسِيرُ مُتَّحِداً مَعَ فَاعِلِي الإِثْمِ وَذَاهِباً مَعَ أَهْلِ الشَّرِّ؟
\par 9 لأَنَّهُ قَالَ: لاَ يَنْتَفِعُ الإِنْسَانُ بِكَوْنِهِ مَرْضِيّاً عِنْدَ اللهِ.
\par 10 [لأَجْلِ ذَلِكَ اسْمَعُوا لِي يَا ذَوِي الأَلْبَابِ. حَاشَا لِلَّهِ مِنَ الشَّرِّ وَلِلْقَدِيرِ مِنَ الظُّلْمِ.
\par 11 لأَنَّهُ يُجَازِي الإِنْسَانَ عَلَى فِعْلِهِ وَيُنِيلُ الرَّجُلَ كَطَرِيقِهِ.
\par 12 فَحَقّاً إِنَّ اللهَ لاَ يَفْعَلُ سُوءاً وَالْقَدِيرَ لاَ يُعَوِّجُ الْقَضَاءَ.
\par 13 مَنْ وَكَّلَهُ بِالأَرْضِ وَمَنْ صَنَعَ الْمَسْكُونَةَ كُلَّهَا؟
\par 14 إِنْ جَعَلَ عَلَيْهِ قَلْبَهُ إِنْ جَمَعَ إِلَى نَفْسِهِ رُوحَهُ وَنَسَمَتَهُ
\par 15 يُسَلِّمُ الرُّوحَ كُلُّ بَشَرٍ جَمِيعاً وَيَعُودُ الإِنْسَانُ إِلَى التُّرَابِ.
\par 16 فَإِنْ كَانَ لَكَ فَهْمٌ فَاسْمَعْ هَذَا وَاصْغَ إِلَى صَوْتِ كَلِمَاتِي.
\par 17 أَلَعَلَّ مَنْ يُبْغِضُ الْحَقَّ يَتَسَلَّطُ أَمِ الْبَارَّ الْكَبِيرَ تَسْتَذْنِبُ.
\par 18 أَيُقَالُ لِلْمَلِكِ: يَا لَئِيمُ وَلِلشُّرَفَاءِ: يَا أَشْرَارُ؟!
\par 19 الَّذِي لاَ يُحَابِي بِوُجُوهِ الرُّؤَسَاءِ وَلاَ يَعْتَبِرُ غَنِيّاً دُونَ فَقِيرٍ. لأَنَّهُمْ جَمِيعَهُمْ عَمَلُ يَدَيْهِ.
\par 20 بَغْتَةً يَمُوتُونَ وَفِي نِصْفِ اللَّيْلِ. يَرْتَجُّ الشَّعْبُ وَيَزُولُونَ وَيُنْزَعُ الأَعِزَّاءُ لاَ بِيَدٍ.
\par 21 لأَنَّ عَيْنَيْهِ عَلَى طُرُقِ الإِنْسَانِ وَهُوَ يَرَى كُلَّ خَطَوَاتِهِ.
\par 22 لاَ ظَلاَمَ وَلاَ ظِلَّ مَوْتٍ حَيْثُ تَخْتَفِي عُمَّالُ الإِثْمِ.
\par 23 لأَنَّهُ لاَ يُلاَحِظُ الإِنْسَانَ زَمَاناً لِلدُّخُولِ فِي الْمُحَاكَمَةِ مَعَ اللهِ.
\par 24 يُحَطِّمُ الأَعِزَّاءَ مِنْ دُونِ فَحْصٍ وَيُقِيمُ آخَرِينَ مَكَانَهُمْ.
\par 25 لَكِنَّهُ يَعْرِفُ أَعْمَالَهُمْ وَيُقَلِّبُهُمْ لَيْلاً فَيَنْسَحِقُونَ.
\par 26 لِكَوْنِهِمْ أَشْرَاراً يَصْفَعُهُمْ فِي مَرْأَى النَّاظِرِينَ.
\par 27 لأَنَّهُمُ انْصَرَفُوا مِنْ وَرَائِهِ وَكُلُّ طُرُقِهِ لَمْ يَتَأَمَّلُوهَا
\par 28 حَتَّى بَلَّغُوا إِلَيْهِ صُرَاخَ الْمِسْكِينِ فَسَمِعَ زَعْقَةَ الْبَائِسِينَ.
\par 29 إِذَا هُوَ سَكَّنَ فَمَنْ يَشْغَبُ؟ وَإِذَا حَجَبَ وَجْهَهُ فَمَنْ يَرَاهُ سَوَاءٌ كَانَ عَلَى أُمَّةٍ أَوْ عَلَى إِنْسَانٍ؟
\par 30 حَتَّى لاَ يَمْلِكَ الْفَاجِرُ وَلاَ يَكُونَ شَرَكاً لِلشَّعْبِ.
\par 31 [وَلَكِنْ هَلْ لِلَّهِ قَالَ: احْتَمَلْتُ. لاَ أَعُودُ أُفْسِدُ.
\par 32 مَا لَمْ أُبْصِرْهُ فَأَرِنِيهِ أَنْتَ. إِنْ كُنْتُ قَدْ فَعَلْتُ إِثْماً فَلاَ أَعُودُ أَفْعَلُهُ؟
\par 33 هَلْ كَرَأْيِكَ يُجَازِيهِ قَائِلاً: لأَنَّكَ رَفَضْتَ فَأَنْتَ تَخْتَارُ لاَ أَنَا. وَبِمَا تَعْرِفُهُ تَكَلَّمْ؟
\par 34 ذَوُو الأَلْبَابِ يَقُولُونَ لِي بَلِ الرَّجُلُ الْحَكِيمُ الَّذِي يَسْمَعُنِي يَقُولُ:
\par 35 إِنَّ أَيُّوبَ يَتَكَلَّمُ بِلاَ مَعْرِفَةٍ وَكَلاَمُهُ لَيْسَ بِتَعَقُّلٍ.
\par 36 فَلَيْتَ أَيُّوبَ كَانَ يُمْتَحَنُ إِلَى الْغَايَةِ مِنْ أَجْلِ أَجْوِبَتِهِ كَأَهْلِ الإِثْمِ.
\par 37 لَكِنَّهُ أَضَافَ إِلَى خَطِيَّتِهِ مَعْصِيَةً. يُصَفِّقُ بَيْنَنَا وَيُكْثِرُ كَلاَمَهُ عَلَى اللهِ].

\chapter{35}

\par 1 وَقَالَ أَلِيهُو:
\par 2 [أَتَحْسِبُ هَذَا حَقّاً؟ قُلْتَ: أَنَا أَبَرُّ مِنَ اللهِ.
\par 3 لأَنَّكَ قُلْتَ: مَاذَا يُفِيدُكَ؟ بِمَاذَا أَنْتَفِعُ أَكْثَرَ مِنْ خَطِيَّتِي؟
\par 4 أَنَا أَرُدُّ عَلَيْكَ كَلاَماً وَعَلَى أَصْحَابِكَ مَعَكَ.
\par 5 اُنْظُرْ إِلَى السَّمَاوَاتِ وَأَبْصِرْ وَلاَحِظِ الْغَمَامَ. إِنَّهَا أَعْلَى مِنْكَ.
\par 6 إِنْ أَخْطَأْتَ فَمَاذَا فَعَلْتَ بِهِ؟ وَإِنْ كَثَّرْتَ مَعَاصِيَكَ فَمَاذَا عَمِلْتَ لَهُ؟
\par 7 إِنْ كُنْتَ بَارّاً فَمَاذَا أَعْطَيْتَهُ أَوْ مَاذَا يَأْخُذُهُ مِنْ يَدِكَ؟
\par 8 لِرَجُلٍ مِثْلِكَ شَرُّكَ وَلاِبْنِ آدَمٍ بِرُّكَ.
\par 9 مِنْ كَثْرَةِ الْمَظَالِمِ يَصْرُخُونَ. يَسْتَغِيثُونَ مِنْ ذِرَاعِ الأَعِزَّاءِ.
\par 10 وَلَمْ يَقُولُوا: أَيْنَ اللهُ صَانِعِي مُؤْتِي الأَغَانِيِّ فِي اللَّيْلِ؟
\par 11 الَّذِي يُعَلِّمُنَا أَكْثَرَ مِنْ وُحُوشِ الأَرْضِ وَيَجْعَلُنَا أَحْكَمَ مِنْ طُيُورِ السَّمَاءِ.
\par 12 ثَمَّ يَصْرُخُونَ مِنْ كِبْرِيَاءِ الأَشْرَارِ وَلاَ يَسْتَجِيبُ.
\par 13 وَلَكِنَّ اللهَ لاَ يَسْمَعُ كَذِباً وَالْقَدِيرُ لاَ يَنْظُرُ إِلَيْهِ.
\par 14 فَإِذَا قُلْتَ إِنَّكَ لَسْتَ تَرَاهُ فَالدَّعْوَى قُدَّامَهُ فَاصْبِرْ لَهُ.
\par 15 وَأَمَّا الآنَ فَلأَنَّ غَضَبَهُ لاَ يُطَالِبُ وَلاَ يُبَالِي بِكَثْرَةِ الزَّلاَّتِ
\par 16 فَغَرَ أَيُّوبُ فَاهُ بِالْبَاطِلِ وَكَبَّرَ الْكَلاَمَ بِلاَ مَعْرِفَةٍ].

\chapter{36}

\par 1 وَعَادَ أَلِيهُو فَقَالَ:
\par 2 [اصْبِرْ عَلَيَّ قَلِيلاً فَأُبْدِيَ لَكَ أَنَّهُ بَعْدُ لأَجْلِ اللهِ كَلاَمٌ.
\par 3 أَحْمِلُ مَعْرِفَتِي مِنْ بَعِيدٍ وَأَنْسِبُ بِرّاً لِصَانِعِي.
\par 4 حَقّاً لاَ يَكْذِبُ كَلاَمِي. صَحِيحُ الْمَعْرِفَةِ عِنْدَكَ.
\par 5 [هُوَذَا اللهُ عَزِيزٌ وَلَكِنَّهُ لاَ يَرْذُلُ أَحَداً. عَزِيزُ قُدْرَةِ الْقَلْبِ.
\par 6 لاَ يُحْيِي الشِّرِّيرَ بَلْ يُجْرِي قَضَاءَ الْبَائِسِينَ.
\par 7 لاَ يُحَوِّلُ عَيْنَيْهِ عَنِ الْبَارِّ بَلْ مَعَ الْمُلُوكِ يُجْلِسُهُمْ عَلَى الْكُرْسِيِّ أَبَداً فَيَرْتَفِعُونَ.
\par 8 إِنْ أُوثِقُوا بِالْقُيُودِ إِنْ أُخِذُوا فِي حِبَالِ الذُّلِّ
\par 9 فَيُظْهِرُ لَهُمْ أَفْعَالَهُمْ وَمَعَاصِيَهُمْ لأَنَّهُمْ تَجَبَّرُوا
\par 10 وَيَفْتَحُ آذَانَهُمْ لِلإِنْذَارِ وَيَأْمُرُ بِأَنْ يَرْجِعُوا عَنِ الإِثْمِ.
\par 11 إِنْ سَمِعُوا وَأَطَاعُوا قَضُوا أَيَّامَهُمْ بِالْخَيْرِ وَسِنِيهِمْ بِالنِّعَمِ.
\par 12 وَإِنْ لَمْ يَسْمَعُوا فَبِحَرْبَةِ الْمَوْتِ يَزُولُونَ وَيَمُوتُونَ بِعَدَمِ الْمَعْرِفَةِ.
\par 13 أَمَّا فُجَّارُ الْقَلْبِ فَيَذْخَرُونَ غَضَباً. لاَ يَسْتَغِيثُونَ إِذَا هُوَ قَيَّدَهُمْ.
\par 14 تَمُوتُ نَفْسُهُمْ فِي الصِّبَا وَحَيَاتُهُمْ بَيْنَ الْمَأْبُونِينَ.
\par 15 يُنَجِّي الْبَائِسَ فِي ذُلِّهِ وَيَفْتَحُ آذَانَهُمْ فِي الضِّيقِ.
\par 16 [وَأَيْضاً يَقُودُكَ مِنْ وَجْهِ الضِّيقِ إِلَى رُحْبٍ لاَ حَصْرَ فِيهِ وَيَمْلَأُ مَؤُونَةَ مَائِدَتِكَ دُهْناً.
\par 17 حُجَّةَ الشِّرِّيرِ أَكْمَلْتَ فَالْحُجَّةُ وَالْقَضَاءُ يُمْسِكَانِكَ.
\par 18 عِنْدَ غَضَبِهِ لَعَلَّهُ يَقُودُكَ بِصَفْقَةٍ. فَكَثْرَةُ الْفِدْيَةِ لاَ تَفُكُّكَ.
\par 19 هَلْ يَعْتَبِرُ غِنَاكَ؟ لاَ التِّبْرَ وَلاَ جَمِيعَ قُوَى الثَّرْوَةِ!
\par 20 لاَ تَشْتَاقُ إِلَى اللَّيْلِ الَّذِي يَرْفَعُ شُعُوباً مِنْ مَوَاضِعِهِمْ.
\par 21 اِحْذَرْ. لاَ تَلْتَفِتْ إِلَى الإِثْمِ لأَنَّكَ اخْتَرْتَ هَذَا عَلَى الذُّلِّ.
\par 22 [هُوَذَا اللهُ يَتَعَالَى بِقُدْرَتِهِ. مَنْ مِثْلُهُ مُعَلِّماً؟
\par 23 مَنْ فَرَضَ عَلَيْهِ طَرِيقَهُ أَوْ مَنْ يَقُولُ لَهُ: قَدْ فَعَلْتَ شَرّاً؟
\par 24 اُذْكُرْ أَنْ تُعَظِّمَ عَمَلَهُ الَّذِي يَتَرَنَّمُ بِهِ النَّاسُ.
\par 25 كُلُّ إِنْسَانٍ يُبْصِرُ بِهِ. النَّاسُ يَنْظُرُونَهُ مِنْ بَعِيدٍ.
\par 26 هُوَذَا اللهُ عَظِيمٌ وَلاَ نَعْرِفُهُ وَعَدَدُ سِنِيهِ لاَ يُفْحَصُ.
\par 27 لأَنَّهُ يَجْذِبُ قْطَرَاتِ الْمَاءِ. تَسُحُّ مَطَراً مِنْ ضَبَابِهَا
\par 28 الَّذِي تَهْطِلُهُ السُّحُبُ وَتَقْطُرُهُ عَلَى أُنَاسٍ كَثِيرِينَ.
\par 29 فَهَلْ يُعَلِّلُ أَحَدٌ عَنْ شَقِّ الْغَيْمِ أَوْ قَصِيفِ مَظَلَّتِهِ؟
\par 30 هُوَذَا بَسَطَ نُورَهُ عَلَى نَفْسِهِ ثُمَّ يَتَغَطَّى بِأُصُولِ الْبَحْرِ.
\par 31 لأَنَّهُ بِهَذِهِ يَدِينُ الشُّعُوبَ وَيَرْزِقُ الْقُوتَ بِكَثْرَةٍ.
\par 32 يُغَطِّي كَفَّيْهِ بِالنُّورِ وَيَأْمُرُهُ عَلَى الْعَدُوِّ.
\par 33 يُخْبِرُ بِهِ رَعْدُهُ الْمَوَاشِيَ أَيْضاً بِصُعُودِهِ.

\chapter{37}

\par 1 [فَلِهَذَا اضْطَرَبَ قَلْبِي وَخَفَقَ مِنْ مَوْضِعِهِ.
\par 2 اسْمَعُوا سَمَاعاً رَعْدَ صَوْتِهِ وَالدَّوِيَّ الْخَارِجَ مِنْ فَمِهِ.
\par 3 تَحْتَ كُلِّ السَّمَاوَاتِ يُطْلِقُهَا كَذَا نُورُهُ إِلَى أَطْرَافِ الأَرْضِ.
\par 4 بَعْدُ يُزَمْجِرُ صَوْتٌ يُرْعِدُ بِصَوْتِ جَلاَلِهِ وَلاَ يُؤَخِّرُهَا إِذْ سُمِعَ صَوْتُهُ.
\par 5 اَللهُ يُرْعِدُ بِصَوْتِهِ عَجَباً. يَصْنَعُ عَظَائِمَ لاَ نُدْرِكُهَا.
\par 6 لأَنَّهُ يَقُولُ لِلثَّلْجِ: اسْقُطْ عَلَى الأَرْضِ. كَذَا لِوَابِلِ الْمَطَرِ وَابِلِ أَمْطَارِ عِزِّهِ.
\par 7 يَخْتِمُ عَلَى يَدِ كُلِّ إِنْسَانٍ لِيَعْلَمَ كُلُّ النَّاسِ خَالِقَهُمْ
\par 8 فَتَدْخُلُ الْحَيَوَانَاتُ الْمَآوِيَ وَتَسْتَقِرُّ فِي أَوْجِرَتِهَا.
\par 9 مِنَ الْجَنُوبِ تَأْتِي الأَعْصَارُ وَمِنَ الشِّمَالِ الْبَرَدُ.
\par 10 مِنْ نَسَمَةِ اللهِ يُجْعَلُ الْجَمَدُ وَتَتَضَيَّقُ سِعَةُ الْمِيَاهِ.
\par 11 أَيْضاً بِرِيٍّ يَطْرَحُ الْغَيْمَ. يُبَدِّدُ سَحَابَ نُورِهِ.
\par 12 فَهِيَ مُدَوَّرَةٌ مُتَقَلِّبَةٌ بِإِدَارَتِهِ لِتَفْعَلَ كُلَّ مَا يَأْمُرُ بِهِ عَلَى وَجْهِ الأَرْضِ الْمَسْكُونَةِ
\par 13 سِوَاءٌ كَانَ لِلتَّأْدِيبِ أَوْ لأَرْضِهِ أَوْ لِلرَّحْمَةِ يُرْسِلُهَا.
\par 14 [اُنْصُتْ إِلَى هَذَا يَا أَيُّوبُ وَقِفْ وَتَأَمَّلْ بِعَجَائِبِ اللهِ.
\par 15 أَتُدْرِكُ انْتِبَاهَ اللهِ إِلَيْهَا أَوْ إِضَاءَةَ نُورِ سَحَابِهِ.
\par 16 أَتُدْرِكُ مُوازَنَةَ السَّحَابِ مُعْجِزَاتِ الْكَامِلِ الْمَعَارِفِ.
\par 17 كَيْفَ تَسْخُنُ ثِيَابُكَ إِذَا سَكَنَتِ الأَرْضُ مِنْ رِيحِ الْجَنُوبِ.
\par 18 هَلْ صَفَّحْتَ مَعَهُ الْجَلَدَ الْمُمَكَّنَ كَالْمِرْآةِ الْمَسْبُوكَةِ؟
\par 19 عَلِّمْنَا مَا نَقُولُ لَهُ. إِنَّنَا لاَ نُحْسِنُ الْكَلاَمَ بِسَبَبِ الظُّلْمَةِ!
\par 20 هَلْ يُقَصُّ عَلَيْهِ كَلاَمِي إِذَا تَكَلَّمْتُ؟ هَلْ يَنْطِقُ الإِنْسَانُ لِكَيْ يَبْتَلِعَ؟
\par 21 وَالآنَ لاَ يُرَى النُّورُ الْبَاهِرُ الَّذِي هُوَ فِي الْجَلَدِ ثُمَّ تَعْبُرُ الرِّيحُ فَتُنَقِّيهِ.
\par 22 مِنَ الشِّمَالِ يَأْتِي ذَهَبٌ. عِنْدَ اللهِ جَلاَلٌ مُرْهِبٌ.
\par 23 الْقَدِيرُ لاَ نُدْرِكُهُ. عَظِيمُ الْقُوَّةِ وَالْحَقِّ وَكَثِيرُ الْبِرِّ. لاَ يُجَاوِبُ.
\par 24 لِذَلِكَ فَلْتَخَفْهُ النَّاسُ. كُلَّ حَكِيمِ الْقَلْبِ لاَ يُرَاعِي].

\chapter{38}

\par 1 فَقَالَ الرَّبُّ لأَيُّوبَ مِنَ الْعَاصِفَةِ:
\par 2 [مَنْ هَذَا الَّذِي يُظْلِمُ الْقَضَاءَ بِكَلاَمٍ بِلاَ مَعْرِفَةٍ؟
\par 3 اُشْدُدِ الآنَ حَقْوَيْكَ كَرَجُلٍ فَإِنِّي أَسْأَلُكَ فَتُعَلِّمُنِي.
\par 4 أَيْنَ كُنْتَ حِينَ أَسَّسْتُ الأَرْضَ؟ أَخْبِرْ إِنْ كَانَ عِنْدَكَ فَهْمٌ.
\par 5 مَنْ وَضَعَ قِيَاسَهَا؟ لأَنَّكَ تَعْلَمُ! أَوْ مَنْ مَدَّ عَلَيْهَا مِطْمَاراً؟
\par 6 عَلَى أَيِّ شَيْءٍ قَرَّتْ قَوَاعِدُهَا أَوْ مَنْ وَضَعَ حَجَرَ زَاوِيَتِهَا
\par 7 عِنْدَمَا تَرَنَّمَتْ كَوَاكِبُ الصُّبْحِ مَعاً وَهَتَفَ جَمِيعُ بَنِي اللهِ؟
\par 8 [وَمَنْ حَجَزَ الْبَحْرَ بِمَصَارِيعَ حِينَ انْدَفَقَ فَخَرَجَ مِنَ الرَّحِمِ.
\par 9 إِذْ جَعَلْتُ السَّحَابَ لِبَاسَهُ وَالضَّبَابَ قِمَاطَهُ
\par 10 وَجَزَمْتُ عَلَيْهِ حَدِّي وَأَقَمْتُ لَهُ مَغَالِيقَ وَمَصَارِيعَ
\par 11 وَقُلْتُ: إِلَى هُنَا تَأْتِي وَلاَ تَتَعَدَّى وَهُنَا تُتْخَمُ كِبْرِيَاءُ لُجَجِكَ؟
\par 12 [هَلْ فِي أَيَّامِكَ أَمَرْتَ الصُّبْحَ؟ هَلْ عَرَّفْتَ الْفَجْرَ مَوْضِعَهُ
\par 13 لِيُمْسِكَ بِأَطْرَافِ الأَرْضِ فَيُنْفَضَ الأَشْرَارُ مِنْهَا؟
\par 14 تَتَحَوَّلُ كَطِينِ الْخَاتِمِ وَتَقِفُ كَأَنَّهَا لاَبِسَةٌ.
\par 15 وَيُمْنَعُ عَنِ الأَشْرَارِ نُورُهُمْ وَتَنْكَسِرُ الذِّرَاعُ الْمُرْتَفِعَةُ.
\par 16 [هَلِ انْتَهَيْتَ إِلَى يَنَابِيعِ الْبَحْرِ أَوْ فِي مَقْصُورَةِ الْغَمْرِ تَمَشَّيْتَ؟
\par 17 هَلِ انْكَشَفَتْ لَكَ أَبْوَابُ الْمَوْتِ أَوْ عَايَنْتَ أَبْوَابَ ظِلِّ الْمَوْتِ؟
\par 18 هَلْ أَدْرَكْتَ عَرْضَ الأَرْضِ؟ أَخْبِرْ إِنْ عَرَفْتَهُ كُلَّهُ!
\par 19 [أَيْنَ الطَّرِيقُ إِلَى حَيْثُ يَسْكُنُ النُّورُ وَالظُّلْمَةُ أَيْنَ مَقَامُهَا
\par 20 حَتَّى تَأْخُذَهَا إِلَى تُخُومِهَا وَتَعْرِفَ سُبُلَ بَيْتِهَا؟
\par 21 تَعْلَمُ لأَنَّكَ حِينَئِذٍ كُنْتَ قَدْ وُلِدْتَ وَعَدَدُ أَيَّامِكَ كَثِيرٌ!
\par 22 [أَدَخَلْتَ إِلَى خَزَائِنِ الثَّلْجِ أَمْ أَبْصَرْتَ مَخَازِنَ الْبَرَدِ
\par 23 الَّتِي أَبْقَيْتَهَا لِوَقْتِ الضَّرِّ لِيَوْمِ الْقِتَالِ وَالْحَرْبِ؟
\par 24 فِي أَيِّ طَرِيقٍ يَتَوَزَّعُ النُّورُ وَتَتَفَرَّقُ الرِّيحُ الشَّرْقِيَّةُ عَلَى الأَرْضِ؟
\par 25 مَنْ فَرَّعَ قَنَوَاتٍ لِلْهَطْلِ وَطَرِيقاً لِلصَّوَاعِقِ
\par 26 لِيَمْطُرَ عَلَى أَرْضٍ حَيْثُ لاَ إِنْسَانَ عَلَى قَفْرٍ لاَ أَحَدَ فِيهِ
\par 27 لِيُرْوِيَ الْبَلْقَعَ وَالْخَلاَءَ وَيُنْبِتَ مَخْرَجَ الْعُشْبِ؟
\par 28 [هَلْ لِلْمَطَرِ أَبٌ وَمَنْ وَلَدَ مَآجِلَ الطَّلِّ؟
\par 29 مِنْ بَطْنِ مَنْ خَرَجَ الْجَلِيدُ؟ صَقِيعُ السَّمَاءِ مَنْ وَلَدَهُ؟
\par 30 كَحَجَرٍ صَارَتِ الْمِيَاهُ. اخْتَبَأَتْ. وَتَلَكَّدَ وَجْهُ الْغَمْرِ.
\par 31 [هَلْ تَرْبِطُ أَنْتَ عُقْدَ الثُّرَيَّا أَوْ تَفُكُّ رُبُطَ الْجَبَّارِ؟
\par 32 أَتُخْرِجُ الْمَنَازِلَ فِي أَوْقَاتِهَا وَتَهْدِي النَّعْشَ مَعَ بَنَاتِهِ؟
\par 33 هَلْ عَرَفْتَ سُنَنَ السَّمَاوَاتِ أَوْ جَعَلْتَ تَسَلُّطَهَا عَلَى الأَرْضِ؟
\par 34 أَتَرْفَعُ صَوْتَكَ إِلَى السُّحُبِ فَيُغَطِّيَكَ فَيْضُ الْمِيَاهِ؟
\par 35 أَتُرْسِلُ الْبُرُوقَ فَتَذْهَبَ وَتَقُولَ لَكَ: هَا نَحْنُ؟
\par 36 مَنْ وَضَعَ فِي الطَّخَاءِ حِكْمَةً أَوْ مَنْ أَظْهَرَ فِي الشُّهُبِ فِطْنَةً؟
\par 37 مَنْ يُحْصِي الْغُيُومَ بِالْحِكْمَةِ وَمَنْ يَسْكُبُ أَزْقَاقَ السَّمَاوَاتِ
\par 38 إِذْ يَنْسَبِكُ التُّرَابُ سَبْكاً وَيَتَلاَصَقُ الطِّينُ؟
\par 39 [أَتَصْطَادُ لِلَّبْوَةِ فَرِيسَةً أَمْ تُشْبِعُ نَفْسَ الأَشْبَالِ
\par 40 حِينَ تَرْبِضُ فِي عَرِينِهَا وَتَكْمُنُ فِي غَابَتِهَا لِلْكُمُونِ؟
\par 41 مَنْ يُهَيِّئُ لِلْغُرَابِ صَيْدَهُ إِذْ تَنْعَبُ فِرَاخُهُ إِلَى اللهِ وَتَتَرَدَّدُ لِعَدَمِ الْقُوتِ؟

\chapter{39}

\par 1 [أَتَعْرِفُ وَقْتَ وَلاَدَةِ وُعُولِ الصُّخُورِ أَوْ تُلاَحِظُ مَخَاضَ الأَيَائِلِ؟
\par 2 أَتَحْسِبُ الشُّهُورَ الَّتِي تُكَمِّلُهَا أَوْ تَعْلَمُ مِيعَادَ وَلاَدَتِهِنَّ؟
\par 3 يَبْرُكْنَ وَيَضَعْنَ أَوْلاَدَهُنَّ. يَدْفَعْنَ أَوْجَاعَهُنَّ.
\par 4 تَبْلُغُ أَوْلاَدُهُنَّ. تَرْبُو فِي الْبَرِّيَّةِ. تَخْرُجُ وَلاَ تَعُودُ إِلَيْهِنَّ.
\par 5 [مَنْ سَرَّحَ الْفَرَاءَ حُرّاً وَمَنْ فَكَّ رُبُطَ حِمَارِ الْوَحْشِ؟
\par 6 الَّذِي جَعَلْتُ الْبَرِّيَّةَ بَيْتَهُ وَالسِّبَاخَ مَسْكَنَهُ.
\par 7 يَضْحَكُ عَلَى جُمْهُورِ الْقَرْيَةِ. لاَ يَسْمَعُ زَجْرَ السَّائِقِ.
\par 8 دَائِرَةُ الْجِبَالِ مَرْعَاهُ وَعَلَى كُلِّ خُضْرَةٍ يُفَتِّشُ.
\par 9 [أَيَرْضَى الثَّوْرُ الْوَحْشِيُّ أَنْ يَخْدِمَكَ أَمْ يَبِيتُ عِنْدَ مِعْلَفِكَ؟
\par 10 أَتَرْبِطُ الثَّوْرَ الْوَحْشِيَّ بِحَبْلٍ إِلَى خَطِّ الْمِحْرَاثِ أَمْ يُمَهِّدُ الأَوْدِيَةَ وَرَاءَكَ؟
\par 11 أَتَثِقُ بِهِ لأَنَّ قُوَّتَهُ عَظِيمَةٌ أَوْ تَتْرُكُ لَهُ تَعَبَكَ؟
\par 12 أَتَأْتَمِنُهُ أَنَّهُ يَأْتِي بِزَرْعِكَ وَيُجْمَعُ إِلَى بَيْدَرِكَ؟
\par 13 [جَنَاحُ النَّعَامَةِ يُرَفْرِفُ. أَفَهُوَ مَنْكِبٌ رَؤُوفٌ أَمْ رِيشٌ؟
\par 14 لأَنَّهَا تَتْرُكُ بَيْضَهَا وَتُحْمِيهِ فِي التُّرَابِ
\par 15 وَتَنْسَى أَنَّ الرِّجْلَ تَضْغُطُهُ أَوْ حَيَوَانَ الْبَرِّ يَدُوسُهُ!
\par 16 تَقْسُو عَلَى أَوْلاَدِهَا كَأَنَّهَا لَيْسَتْ لَهَا. بَاطِلٌ تَعَبُهَا بِلاَ أَسَفٍ.
\par 17 لأَنَّ اللهَ قَدْ أَنْسَاهَا الْحِكْمَةَ وَلَمْ يَقْسِمْ لَهَا فَهْماً.
\par 18 عِنْدَمَا تُحْوِذُ نَفْسَهَا إِلَى الْعَلاَءِ تَضْحَكُ عَلَى الْفَرَسِ وَعَلَى رَاكِبِهِ.
\par 19 [هَلْ أَنْتَ تُعْطِي الْفَرَسَ قُوَّتَهُ وَتَكْسُو عُنُقَهُ عُرْفاً؟
\par 20 أَتُوثِبُهُ كَجَرَادَةٍ؟ نَفْخُ مِنْخَرِهِ مُرْعِبٌ.
\par 21 يَبْحَثُ فِي الْوَادِي وَيَقْفِزُ بِبَأْسٍ. يَخْرُجُ لِلِقَاءِ الأَسْلِحَةِ.
\par 22 يَضْحَكُ عَلَى الْخَوْفِ وَلاَ يَرْتَاعُ وَلاَ يَرْجِعُ عَنِ السَّيْفِ.
\par 23 عَلَيْهِ تَصِلُّ السِّهَامُ وَسِنَانُ الرُّمْحِ وَالْحَرْبَةِ.
\par 24 فِي وَثْبِهِ وَغَضَبِهِ يَلْتَهِمُ الأَرْضَ وَلاَ يُؤْمِنُ أَنَّهُ صَوْتُ الْبُوقِ.
\par 25 عِنْدَ نَفْخِ الْبُوقِ يَقُولُ: هَهْ! وَمِنْ بَعِيدٍ يَسْتَرْوِحُ الْقِتَالَ صِيَاحَ الْقُوَّادِ وَالْهُتَافَ.
\par 26 [أَمِنْ فَهْمِكَ يَسْتَقِلُّ الْعُقَابُ وَيَنْشُرُ جَنَاحَيْهِ نَحْوَ الْجَنُوبِ؟
\par 27 أَوْ بِأَمْرِكَ يُحَلِّقُ النَّسْرُ وَيُعَلِّي وَكْرَهُ؟
\par 28 يَسْكُنُ الصَّخْرَ وَيَبِيتُ عَلَى سِنِّ الصَّخْرِ وَالْمَعْقَلِ.
\par 29 مِنْ هُنَاكَ يَتَحَسَّسُ قُوتَهُ. تُبْصِرُهُ عَيْنَاهُ مِنْ بَعِيدٍ.
\par 30 فِرَاخُهُ تَحْسُو الدَّمَ وَحَيْثُمَا تَكُنِ الْقَتْلَى فَهُنَاكَ هُوَ].

\chapter{40}

\par 1 وَقَالَ الرَّبُّ لأَيُّوبَ:
\par 2 [هَلْ يُخَاصِمُ الْقَدِيرَ مُوَبِّخُهُ أَمِ الْمُحَاجُّ اللهَ يُجَاوِبُهُ؟].
\par 3 فَأَجَابَ أَيُّوبُ الرَّبَّ:
\par 4 [هَا أَنَا حَقِيرٌ فَمَاذَا أُجَاوِبُكَ؟ وَضَعْتُ يَدِي عَلَى فَمِي.
\par 5 مَرَّةً تَكَلَّمْتُ فَلاَ أُجِيبُ وَمَرَّتَيْنِ فَلاَ أَزِيدُ].
\par 6 فَقَالَ الرَّبُّ لأَيُّوبَ مِنَ الْعَاصِفَةِ:
\par 7 [الآنَ شُدَّ حَقْوَيْكَ كَرَجُلٍ. أَسْأَلُكَ فَتُعْلِمُنِي.
\par 8 لَعَلَّكَ تُنَاقِضُ حُكْمِي. تَسْتَذْنِبُنِي لِتَتَبَرَّرَ أَنْتَ!
\par 9 هَلْ لَكَ ذِرَاعٌ كَمَا لِلَّهِ وَبِصَوْتٍ مِثْلِ صَوْتِهِ تُرْعِدُ؟
\par 10 تَزَيَّنِ الآنَ بِالْجَلاَلِ وَالْعِزِّ وَالْبِسِ الْمَجْدَ وَالْبَهَاءَ.
\par 11 فَرِّقْ فَيْضَ غَضَبِكَ وَانْظُرْ كُلَّ مُتَعَظِّمٍ وَاخْفِضْهُ.
\par 12 اُنْظُرْ إِلَى كُلِّ مُتَعَظِّمٍ وَذَلِّلْهُ وَدُسِ الأَشْرَارَ فِي مَكَانِهِمِ.
\par 13 اُطْمُرْهُمْ فِي التُّرَابِ مَعاً وَاحْبِسْ وُجُوهَهُمْ فِي الظَّلاَمِ.
\par 14 فَأَنَا أَيْضاً أَحْمَدُكَ لأَنَّ يَمِينَكَ تُخَلِّصُكَ.
\par 15 [هُوَذَا فَرَسُ الْبَحْرِ الَّذِي صَنَعْتُهُ مَعَكَ. يَأْكُلُ الْعُشْبَ مِثْلَ الْبَقَرِ.
\par 16 هَا هِيَ قُوَّتُهُ فِي مَتْنَيْهِ وَشِدَّتُهُ فِي عَضَلِ بَطْنِهِ.
\par 17 يَخْفِضُ ذَنَبَهُ كَأَرْزَةٍ. عُرُوقُ فَخْذَيْهِ مَضْفُورَةٌ.
\par 18 عِظَامُهُ أَنَابِيبُ نُحَاسٍ وَأَضْلاَعُهُ حَدِيدٌ مُطَرَّقٌ.
\par 19 هُوَ أَوَّلُ أَعْمَالِ اللهِ. الَّذِي صَنَعَهُ أَعْطَاهُ سَيْفَهُ.
\par 20 لأَنَّ الْجِبَالَ تُخْرِجُ لَهُ مَرْعًى وَجَمِيعَ وُحُوشِ الْبَرِّ تَلْعَبُ هُنَاكَ.
\par 21 تَحْتَ السِّدْرَاتِ يَضْطَجِعُ فِي سِتْرِ الْقَصَبِ وَالْغَمِقَةِ.
\par 22 تُظَلِّلُهُ السِّدْرَاتُ بِظِلِّهَا. يُحِيطُ بِهِ صَفْصَافُ السَّوَاقِي.
\par 23 هُوَذَا النَّهْرُ يَفِيضُ فَلاَ يَفِرُّ هُوَ. يَطْمَئِنُّ وَلَوِ انْدَفَقَ الأُرْدُنُّ فِي فَمِهِ.
\par 24 هَلْ يُؤْخَذُ مِنْ أَمَامِهِ؟ هَلْ يُثْقَبُ أَنْفُهُ بِخِزَامَةٍ؟

\chapter{41}

\par 1 [أَتَصْطَادُ التِّمسَاحَ بِشِصٍّ أَوْ تَضْغَطُ لِسَانَهُ بِحَبْلٍ؟
\par 2 أَتَضَعُ أَسَلَةً فِي خَطْمِهِ أَمْ تَثْقُبُ فَكَّهُ بِخِزَامَةٍ؟
\par 3 أَيُكْثِرُ التَّضَرُّعَاتِ إِلَيْكَ أَمْ يَتَكَلَّمُ مَعَكَ بِاللِّينِ؟
\par 4 هَلْ يَقْطَعُ مَعَكَ عَهْداً فَتَتَّخِذَهُ عَبْداً مُؤَبَّداً؟
\par 5 أَتَلْعَبُ مَعَهُ كَالْعُصْفُورِ أَوْ تَرْبِطُهُ لأَجْلِ فَتَيَاتِكَ؟
\par 6 هَلْ تَحْفُرُ جَمَاعَةُ الصَّيَّادِينَ لأَجْلِهِ حُفْرَةً أَوْ يَقْسِمُونَهُ بَيْنَ الْكَنْعَانِيِّينَ؟
\par 7 أَتَمْلَأُ جِلْدَهُ حِرَاباً وَرَأْسَهُ بِإِلاَلِ السَّمَكِ؟
\par 8 ضَعْ يَدَكَ عَلَيْهِ. لاَ تَعُدْ تَذْكُرُ الْقِتَالَ!
\par 9 هُوَذَا الرَّجَاءُ بِهِ كَاذِبٌ. أَلاَ يُكَبُّ أَيْضاً بِرُؤْيَتِهِ.
\par 10 لَيْسَ مِنْ شُجَاعٍ يُوقِظُهُ فَمَنْ يَقِفُ إِذاً بِوَجْهِي؟
\par 11 مَنْ تَقَدَّمَنِي فَأُوفِيَهُ؟ مَا تَحْتَ كُلِّ السَّمَاوَاتِ هُوَ لِي.
\par 12 [لاَ أَسْكُتُ عَنْ أَعْضَائِهِ وَخَبَرِ قُوَّتِهِ وَبَهْجَةِ عُدَّتِهِ.
\par 13 مَنْ يَكْشِفُ وَجْهَ لِبْسِهِ وَمَنْ يَدْنُو مِنْ مَثْنَى لَجَمَتِهِ؟
\par 14 مَنْ يَفْتَحُ مِصْرَاعَيْ فَمِهِ؟ دَائِرَةُ أَسْنَانِهِ مُرْعِبَةٌ.
\par 15 فَخْرُهُ مَجَانُّ مَانِعَةٌ مُحَكَّمَةٌ مَضْغُوطَةٌ بِخَاتِمٍ.
\par 16 الْوَاحِدُ يَمَسُّ الآخَرَ فَالرِّيحُ لاَ تَدْخُلُ بَيْنَهَا.
\par 17 كُلٌّ مِنْهَا مُلْتَصِقٌ بِصَاحِبِهِ مُتَجَمِّدَةً لاَ تَنْفَصِلُ.
\par 18 عِطَاسُهُ يَبْعَثُ نُوراً وَعَيْنَاهُ كَهُدْبِ الصُّبْحِ.
\par 19 مِنْ فَمِهِ تَخْرُجُ مَصَابِيحُ. شَرَارُ نَارٍ تَتَطَايَرُ مِنْهُ.
\par 20 مِنْ مِنْخَرَيْهِ يَخْرُجُ دُخَانٌ كَأَنَّهُ مِنْ قِدْرٍ مَنْفُوخٍ أَوْ مِنْ مِرْجَلٍ.
\par 21 نَفَسُهُ يُشْعِلُ جَمْراً وَلَهِيبٌ يَخْرُجُ مِنْ فَمِهِ.
\par 22 فِي عُنُقِهِ تَبِيتُ الْقُوَّةُ وَأَمَامَهُ يَدُوسُ الْهَوْلُ.
\par 23 مَطَاوِي لَحْمِهِ مُتَلاَصِقَةٌ مَسْبُوكَةٌ عَلَيْهِ لاَ تَتَحَرَّكُ.
\par 24 قَلْبُهُ صُلْبٌ كَالْحَجَرِ وَقَاسٍ كَالرَّحَى.
\par 25 عِنْدَ نُهُوضِهِ تَفْزَعُ الأَقْوِيَاءُ. مِنَ الْمَخَاوِفِ يَتِيهُونَ.
\par 26 سَيْفُ الَّذِي يَلْحَقُهُ لاَ يَقُومُ وَلاَ رُمْحٌ وَلاَ حَرْبَةًٌ وَلاَ دِرْعٌ.
\par 27 يَحْسِبُ الْحَدِيدَ كَالتِّبْنِ وَالنُّحَاسَ كَالْعُودِ النَّخِرِ.
\par 28 لاَ يَسْتَفِزُّهُ نُبْلُ الْقَوْسِ. حِجَارَةُ الْمِقْلاَعِ تَرْجِعُ عَنْهُ كَالْقَشِّ.
\par 29 يَحْسِبُ الْمِطْرَقَةَ كَقَشٍّ وَيَضْحَكُ عَلَى اهْتِزَازِ الرُّمْحِ.
\par 30 تَحْتَهُ قُطَعُ خَزَفٍ حَادَّةٌ. يُمَدِّدُ نَوْرَجاً عَلَى الطِّينِ.
\par 31 يَجْعَلُ الْعُمْقَ يَغْلِي كَالْقِدْرِ وَيَجْعَلُ الْبَحْرَ كَقِدْرِ عِطَارَةٍ.
\par 32 يُضِيءُ السَّبِيلُ وَرَاءَهُ فَيُحْسَبُ اللُّجُّ أَشْيَبَ.
\par 33 لَيْسَ لَهُ فِي الأَرْضِ نَظِيرٌ. صُنِعَ لِعَدَمِ الْخَوْفِ.
\par 34 يُشْرِفُ عَلَى كُلِّ مُتَعَالٍ. هُوَ مَلِكٌ عَلَى كُلِّ بَنِي الْكِبْرِيَاءِ].

\chapter{42}

\par 1 فَأَجَابَ أَيُّوبُ الرَّبَّ:
\par 2 [قَدْ عَلِمْتُ أَنَّكَ تَسْتَطِيعُ كُلَّ شَيْءٍ وَلاَ يَعْسُرُ عَلَيْكَ أَمْرٌ.
\par 3 فَمَنْ ذَا الَّذِي يُخْفِي الْقَضَاءَ بِلاَ مَعْرِفَةٍ! وَلَكِنِّي قَدْ نَطَقْتُ بِمَا لَمْ أَفْهَمْ. بِعَجَائِبَ فَوْقِي لَمْ أَعْرِفْهَا.
\par 4 اِسْمَعِ الآنَ وَأَنَا أَتَكَلَّمُ. أَسْأَلُكَ فَتُعَلِّمُنِي.
\par 5 بِسَمْعِ الأُذُنِ قَدْ سَمِعْتُ عَنْكَ وَالآنَ رَأَتْكَ عَيْنِي.
\par 6 لِذَلِكَ أَرْفُضُ وَأَنْدَمُ فِي التُّرَابِ وَالرَّمَادِ].
\par 7 وَكَانَ بَعْدَمَا تَكَلَّمَ الرَّبُّ مَعَ أَيُّوبَ بِهَذَا الْكَلاَمِ أَنَّ الرَّبَّ قَالَ لأَلِيفَازَ التَّيْمَانِيِّ: [قَدِ احْتَمَى غَضَبِي عَلَيْكَ وَعَلَى كِلاَ صَاحِبَيْكَ لأَنَّكُمْ لَمْ تَقُولُوا فِيَّ الصَّوَابَ كَعَبْدِي أَيُّوبَ.
\par 8 وَالآنَ فَخُذُوا لأَنْفُسِكُمْ سَبْعَةَ ثِيرَانٍ وَسَبْعَةَ كِبَاشٍ وَاذْهَبُوا إِلَى عَبْدِي أَيُّوبَ وَأَصْعِدُوا مُحْرَقَةً لأَجْلِ أَنْفُسِكُمْ وَعَبْدِي أَيُّوبُ يُصَلِّي مِنْ أَجْلِكُمْ لأَنِّي أَرْفَعُ وَجْهَهُ لِئَلاَّ أَصْنَعَ مَعَكُمْ حَسَبَ حَمَاقَتِكُمْ لأَنَّكُمْ لَمْ تَقُولُوا فِيَّ الصَّوَابَ كَعَبْدِي أَيُّوبَ].
\par 9 فَذَهَبَ أَلِيفَازُ التَّيْمَانِيُّ وَبِلْدَدُ الشُّوحِيُّ وَصُوفَرُ النَّعْمَاتِيُّ وَفَعَلُوا كَمَا قَالَ الرَّبُّ لَهُمْ. وَرَفَعَ الرَّبُّ وَجْهَ أَيُّوبَ.
\par 10 وَرَدَّ الرَّبُّ سَبْيَ أَيُّوبَ لَمَّا صَلَّى لأَجْلِ أَصْحَابِهِ وَزَادَ الرَّبُّ عَلَى كُلِّ مَا كَانَ لأَيُّوبَ ضِعْفاً.
\par 11 فَجَاءَ إِلَيْهِ كُلُّ إِخْوَتِهِ وَكُلُّ أَخَوَاتِهِ وَكُلُّ مَعَارِفِهِ مِنْ قَبْلُ وَأَكَلُوا مَعَهُ خُبْزاً فِي بَيْتِهِ وَرَثُوا لَهُ وَعَزُّوهُ عَنْ كُلِّ الشَّرِّ الَّذِي جَلَبَهُ الرَّبُّ عَلَيْهِ وَأَعْطَاهُ كُلٌّ مِنْهُمْ قَسِيطَةً وَاحِدَةً وَكُلُّ وَاحِدٍ قُرْطاً مِنْ ذَهَبٍ.
\par 12 وَبَارَكَ الرَّبُّ آخِرَةَ أَيُّوبَ أَكْثَرَ مِنْ أُولاَهُ. وَكَانَ لَهُ أَرْبَعَةَ عَشَرَ أَلْفاً مِنَ الْغَنَمِ وَسِتَّةُ آلاَفٍ مِنَ الإِبِلِ وَأَلْفُ زَوْجٍ مِنَ الْبَقَرِ وَأَلْفُ أَتَانٍ.
\par 13 وَكَانَ لَهُ سَبْعَةُ بَنِينَ وَثَلاَثُ بَنَاتٍ.
\par 14 وَسَمَّى اسْمَ الأُولَى يَمِيمَةَ وَاسْمَ الثَّانِيَةِ قَصِيعَةَ وَاسْمَ الثَّالِثَةِ قَرْنَ هَفُّوكَ.
\par 15 وَلَمْ تُوجَدْ نِسَاءٌ جَمِيلاَتٌ كَبَنَاتِ أَيُّوبَ فِي كُلِّ الأَرْضِ. وَأَعْطَاهُنَّ أَبُوهُنَّ مِيرَاثاً بَيْنَ إِخْوَتِهِنَّ.
\par 16 وَعَاشَ أَيُّوبُ بَعْدَ هَذَا مِئَةً وَأَرْبَعِينَ سَنَةً وَرَأَى بَنِيهِ وَبَنِي بَنِيهِ إِلَى أَرْبَعَةِ أَجْيَالٍ.
\par 17 ثُمَّ مَاتَ أَيُّوبُ شَيْخاً وَشَبْعَانَ الأَيَّامِ.

\end{document}