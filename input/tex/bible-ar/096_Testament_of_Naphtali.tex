\begin{document}

\title{وصية نفتالي}

\chapter{1}

نفتالي، الابن الثامن ليعقوب وبلهة. العدّاء. درس في علم وظائف الأعضاء.

\par 1 نسخة وصية نفتالي، التي رسمها وقت وفاته في السنة المائة والثلاثين من حياته

\par 2 ولما جُمِعَ بنوه في الشهر السابع، في اليوم الأول من الشهر، وهم لا يزالون أصحاء، صنع لهم وليمة من الطعام والخمر

\par 3 وبعد أن استيقظ في الصباح، قال لهم: أنا أموت، فلم يصدقوه

\par 4 وفيما هو يُمَجِّدُ الرَّبَّ، تَشَدَّدَ وَقَالَ إِنَّهُ سَيَمُوتُ بَعْدَ لَوْمِ الأَمْسِ

\par 5 ثم ابتدأ يقول: اسمعوا يا أبنائي، يا بني نفتالي، اسمعوا كلام أبيكم

\par 6 وُلِدتُ من بلهة، ولأن راحيل احتالت، وأعطت بلهة مكانها ليعقوب، فحبلت وولدتني على ركبتي راحيل، لذلك دعت اسمي نفتالي

\par 7 لأن راحيل أحبتني كثيرًا لأني وُلدت في حجرها، وعندما كنت لا أزال صغيرًا كانت تقبلني وتقول: ليتني أرزق بأخ لك من بطني مثلك

\par 8 لذلك كان يوسف أيضًا مثلي في كل شيء، حسب صلاة راحيل

\par 9 كانت أمي بلهة، ابنة روثاوس، أخي دبوره، مرضعة رفقة، التي ولدت في نفس اليوم مع راحيل

\par 10 وكان روثيوس من نسل إبراهيم، كلدانيًا، تقيًا، حرًا، شريفًا

\par 11 فأسر واشتراه لابان، فأعطاه إيونا جاريته زوجةً، فولدت ابنةً ودعت اسمها زلفة، على اسم القرية التي أسر منها

\par 12 ثم ولدت بلهة قائلة: ابنتي تسرع في طلب الجديد، لأنها حالما ولدت أمسكت بالثدي وأسرعت في الرضاعة منه

\par 13 وكنتُ سريعًا على قدميّ كالغزال، فعيّنني أبي يعقوب لجميع الرسائل، وكغزالٍ أعطاني بركته

\par 14 لأنه كما أن الخزاف يعرف الإناء، وكم يتسع، ويُخرج الطين وفقًا لذلك، كذلك يصنع الرب الجسد على شبه الروح، وحسب قدرة الجسد يغرس الروح

\par 15 ولا ينقص أحدهما عن الآخر ثلث شعرة، لأنه بالوزن والكيل والقياس خُلقت كل الخليقة

\par 16 وكما أن الخزاف يعرف استخدام كل إناء، وما يصلح له، كذلك الرب يعرف الجسد، وإلى أي مدى سيستمر في الصلاح، ومتى يبدأ في الشر

\par 17 لأنه ليس هناك ميل أو فكر لا يعرفه الرب، لأنه خلق كل إنسان على صورته

\par 18 فكما أن قوة الإنسان كذلك في عمله؛ وكما أن عينه كذلك في نومه؛ وكما أن نفسه كذلك في كلامه إما في شريعة الرب أو في شريعة بليعار

\par 19 وكما أن هناك انقسامًا بين النور والظلمة، وبين البصر والسمع، كذلك يوجد أيضًا انقسام بين الرجل والرجل، وبين المرأة والمرأة؛ ولا يقال إن أحدهما يشبه الآخر إما في الوجه أو في العقل.

\par 20 لأن الله خلق كل الأشياء جيدة بترتيبها، الحواس الخمس في الرأس، وربط الرقبة بالرأس، مضيفًا إليه الشعر أيضًا للجمال والمجد، ثم القلب للفهم، والبطن للبراز، والمعدة للطحن، والقصبة الهوائية للتنفس، والكبد للغضب، والمرارة للمرارة، والطحال للضحك، والأعنة للحكمة، وعضلات الحقوين للقوة، والرئتين للاستنشاق، والحقون للقوة، وهكذا دواليك

\par 21 إذن يا أبنائي، لتكن جميع أعمالكم منظمة بنية صالحة في خوف الله، ولا تفعلوا شيئًا غير منظم باستهزاء أو خارج وقته

\par 22 لأنه إن كنت تأمر العين أن تسمع، فلن تستطيع. كذلك وأنتم في الظلمة لا تستطيعون أن تعملوا أعمال النور

\par 23 فلا تكونوا إذن حريصين على إفساد أعمالكم بالطمع أو بكلمات باطل لخداع نفوسكم؛ لأنكم إن صمتم في طهارة قلب، ستفهمون كيف تتمسكون بإرادة الله، وترفضون إرادة بليعار

\par 24 الشمس والقمر والنجوم لا تبدلوا ترتيبها، كذلك أنتم أيضًا لا تبدلوا شريعة الله باضطراب أعمالكم

\par 25 ضلَّ الأمم، وتركوا الرب، وأمروا بنظامهم، وأطاعوا القضبان والحجارة، أرواح الضلال

\par 26 لكنكم لن تكونوا كذلك يا أبنائي، معترفين في السماء، وفي الأرض، وفي البحر، وفي كل المخلوقات، بالرب الذي خلق كل الأشياء، حتى لا تصبحوا مثل سدوم التي غيرت نظام الطبيعة

\par 27 وبالمثل، غيّر المراقبون أيضًا ترتيب طبيعتهم، الذين لعنهم الرب عند الطوفان، والذي بسببه جعل الأرض بلا سكان وبلا ثمر

\par 28 أقول لكم هذا يا أبنائي، لأني قرأت في كتابات أخنوخ أنكم أنتم أيضًا ستبتعدون عن الرب، سائرين حسب كل إثم الأمم، وستفعلون حسب كل شرور سدوم

\par 29 ويجلب الرب عليكم السبي، وهناك تستعبدون لأعدائكم، وتنحني مع كل ضيق وضيق، حتى يفنيكم الرب جميعكم

\par 30 وبعد أن تصغروا وتصيروا قليلين، ترجعون وتعترفون بالرب إلهكم، فيردكم إلى أرضكم حسب رحمته الكثيرة

\par 31 ويكون أنهم بعد دخولهم أرض آبائهم ينسون الرب مرة أخرى ويعودون إلى الكفر

\par 32 ويبددهم الرب على وجه الأرض كلها، حتى تأتي رحمة الرب، رجلاً يصنع البر والرحمة لكل البعيدين والقريبين



\chapter{2}

يدعو إلى حياة منظمة. وتتميّز الآيات ٢٧-٤٠ بحكمتها الأبدية.

\par 1 ففي السنة الأربعين من حياتي، رأيت رؤيا على جبل الزيتون، شرقي أورشليم، أن الشمس والقمر واقفان

\par 2 وإذا إسحاق أبو أبي يقول لنا: اركضوا وأمسكوا بهم، كل واحد على قدر قوته، ومن يمسكهم تكون الشمس والقمر

\par 3 فركضنا كلنا معًا، فأمسك لاوي بالشمس، وسبق يهوذا الآخرين وأمسك بالقمر، فارتفعا كلاهما معهم

\par 4 ولما صار لاوي كالشمس، إذا بشاب أعطاه اثني عشر سعفة نخلة، وكان يهوذا منيرًا كالقمر، وتحت أرجلهم اثنا عشر شعاعًا

\par 5 فركض الاثنان، لاوي ويهوذا، وأمسكاهما.

\par 6 وإذا ثور على الأرض له قرنان عظيمان وعلى ظهره جناحا نسر فأردنا أن نأخذه فلم نستطع.

\par 7 فجاء يوسف وأمسكه وصعد معه إلى العلاء.

\par 8 ونظرت وأنا هناك، وإذا كتابة مقدسة ظهرت لنا قائلة: إن الأشوريين والماديين والفرس والكلدانيين والآراميين سيملكون سبيا أسباط إسرائيل الاثني عشر.

\par 9 وبعد سبعة أيام، رأيتُ أيضًا أبانا يعقوب واقفًا عند بحر يمنيا، وكنا معه

\par 10 وإذا سفينةٌ مبحرةٌ لا بحارةَ لها ولا ربان، وكان مكتوبًا على السفينة: سفينةُ يعقوب

\par 11 وقال لنا أبونا: تعالوا لنركب سفينتنا.

\par 12 ولما صعد إلى السفينة حدثت عاصفة شديدة ونوء ريح شديد، فذهب من عندنا أبونا الذي كان يمسك الدفة.

\par 13 وإذ كنا في عاصفة، كنا نحمل عبر البحر، وكانت السفينة تمتلئ بالماء، وتضربها أمواج عاتية حتى انكسرت

\par 14 فهرب يوسف على متن سفينة صغيرة، وانقسمنا جميعًا على تسعة ألواح، وكان لاوي ويهوذا معًا

\par 15 وتشتتنا جميعًا إلى أقاصي الأرض.

\par 16 ثم تنطق لاوي بالمسح وصلى لأجلنا جميعا إلى الرب.

\par 17 وعندما هدأت العاصفة، وصلت السفينة إلى اليابسة وكأنها في سلام

\par 18 وإذا أبونا قد جاء، ففرحنا جميعاً بنفس واحدة.

\par 19 وقصت هذين الحلمين على أبي، فقال لي: لا بد أن تتم هذه الأمور في وقتها، بعد أن يكون إسرائيل قد تحمل أموراً كثيرة.

\par 20 فقال لي أبي: أومن بالله أن يوسف حي، لأني أرى الرب يحسبه دائمًا معكم

\par 21 فقال باكيًا: آه يا ​​ابني يوسف، أنت حي، وإن لم أراك، ولم ترَ يعقوب الذي ولدك

\par 22 لذلك، جعلني أنا أيضًا أبكي بهذه الكلمات، واشتعل قلبي لأعلن أن يوسف قد بِيعَ، لكنني خفت من إخوتي

\par 23 "وإذا يا أبنائي، لقد أريتكم في الأزمنة الأخيرة كيف سيحدث كل شيء في إسرائيل.

\par 24 فأوصوا أنتم أيضًا أبناءكم أن يتحدوا مع لاوي ويهوذا، لأنه من خلالهم سيقوم خلاص لإسرائيل، وفيهم يتبارك يعقوب

\par 25 لأنه من خلال أسباطهم سيظهر الله ساكنًا بين البشر على الأرض، ليخلص جنس إسرائيل، ويجمع الصالحين من بين الأمم

\par 26 إن فعلتم الخير يا أبنائي، سيبارككم الناس والملائكة، وسيتمجد الله بين الأمم من خلالكم، وسيهرب الشيطان منكم، وستخاف منكم الوحوش، وسيحبكم الرب، وستلتصق بكم الملائكة

\par 27 كما أن الرجل الذي يُربي ولده جيدًا يُحفظ في ذكرى طيبة، كذلك أيضًا للأعمال الصالحة تُحفظ ذكرى طيبة أمام الله

\par 28 وأما من لا يفعل الصلاح، فسيلعنه الملائكة والناس، وسيُهان الله بين الأمم بسببه، وسيجعله إبليس أداة خاصة به، وسيتسلط عليه كل وحش بري، وسيبغضه الرب

\par 29 لأن وصايا الناموس مزدوجة، وبالحكمة ينبغي إتمامها

\par 30 لأن هناك وقتاً لمعانقة الرجل زوجته، ووقتاً للامتناع عنها من أجل صلاته

\par 31 إذن، هناك وصيتان؛ وما لم تُؤدَّا بالترتيب المناسب، فإنهما تجلبان على البشر خطيئة عظيمة

\par 32 وكذلك الحال مع الوصايا الأخرى.

\par 33 فكونوا حكماء في الله يا أبنائي وفهماء فاهمين ترتيب وصاياه وقوانين كل كلمة لكي يحبكم الرب.

\par 34 وبعد أن أوصاهم بمثل هذا الكلام، حثهم على أن ينقلوا عظامه إلى حبرون، وأن يدفنوه مع آبائه

\par 35 وبعد أن أكل وشرب بقلبٍ طَيب، غطى وجهه ومات

\par 36 ففعل بنوه حسب كل ما أوصاهم به نفتالي أبوهم



\end{document}