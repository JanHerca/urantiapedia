\begin{document}

\title{إضافات إلى سفر إستير}

\chapter{1}

\par 1 فقال مردوخا: إن الله هو الذي فعل هذه الأشياء.
\par 2 لأني أتذكر حلمًا رأيته يتعلق بهذه الأمور، ولم يسقط منه شيء
\par 3 نبع صغير تحول إلى نهر، وكان هناك نور، وشمس، وماء كثير. هذا النهر هو أستير، التي تزوجها الملك وجعلها ملكة
\par 4 والتنينان هما أنا وهامان.
\par 5 وكانت الأمم هي التي اجتمعت لتدمير اسم اليهود:
\par 6 وأمتي هي هذه إسرائيل التي صرخت إلى الله فنجت. لأن الرب خلص شعبه، وأنقذنا الرب من كل تلك الشرور، وصنع الله آيات وعجائب عظيمة لم تُصنع بين الأمم
\par 7 لذلك صنع قرعتين، واحدة لشعب الله، وأخرى لجميع الأمم
\par 8 وجاءت هاتان القرعتان في الساعة، والوقت، ويوم الدينونة، أمام الله بين جميع الأمم
\par 9 فذكر الله شعبه وبرر ميراثه.
\par 10 لذلك تكون لهم تلك الأيام في شهر أدار، اليوم الرابع عشر والخامس عشر من ذلك الشهر، باجتماع وفرح وابتهاج أمام الله في أجيال الأبد في شعبه.

\chapter{2}

\par 1 في السنة الرابعة من حكم بطليموس وكليوباترا، أحضر دوسيتاوس، الذي قال إنه كاهن ولاوي، وبطليموس ابنه، رسالة فوريم هذه، التي قالوا إنها هي نفسها، وأن ليسيماخوس بن بطليموس، الذي كان في أورشليم، هو من فسرها
\par 2 وفي السنة الثانية من ملك أرتحشستا الكبير، في اليوم الأول من شهر نيسان، رأى مردخاوس بن يايروس بن شمعي بن كيشاي من سبط بنيامين حلماً.
\par 3 وكان رجلاً يهودياً ساكناً في مدينة شوشن، رجلاً عظيماً، خادماً في بلاط الملك.
\par 4 وكان هو أيضا أحد السبي الذي سباه نبوخذناصر ملك بابل من أورشليم مع يكنيا ملك يهوذا، وهذا حلمه:
\par 5 هوذا صوت ضجيج ورعد وزلازل واضطراب في الأرض.
\par 6 وإذا تنينان عظيمان خرجا مستعدين للقتال، وكان صراخهما عظيما.
\par 7 وعند صراخهم استعدت كل الأمم للقتال، لكي يحاربوا الشعب الصالح.
\par 8 وإذا يوم ظلمة وظلمة وضيق وضيق وضيق واضطراب عظيم على الأرض.
\par 9 وكانت الأمة الصالحة كلها مضطربة، خائفة من شرورها، وكانت على وشك الهلاك.
\par 10 ثم صرخوا إلى الله، فحدث على صراخهم طوفان عظيم، كأنه من نبع صغير، وماء كثير.
\par 11 أشرق النور والشمس، وارتفع المتواضعون، وأكلوا المجيدون
\par 12 عندما استيقظ مردوخا، الذي رأى هذا الحلم وما قرر الله أن يفعله، احتفظ بهذا الحلم في ذهنه، وحتى حلول الليل كان راغبًا بشدة في معرفته

\chapter{3}

\par 1 واستراح مردوخا في الدار مع جاباتا وثارا، خصيي الملك وحارسي القصر.
\par 2 فسمع مكائدهم، وبحث في مقاصدهم، وعلم أنهم على وشك أن يمدوا أيديهم إلى أرتحشستا الملك، فأكد ملكهم
\par 3 ثم فحص الملك الخصيين، وبعد أن اعترفا، خُنقا
\par 4 وسجل الملك هذه الأمور، وكتب عنها مردوخا أيضًا
\par 5 فأمر الملك مردوخا بالخدمة في البلاط، وكافأه على ذلك
\par 6 ولكن هامان بن أمداثوس الأجاجي، الذي كان يتمتع بشرف عظيم لدى الملك، سعى إلى التحرش بمردوخا وشعبه بسبب خصيي الملك

\chapter{4}

\par 1 كانت نسخة الرسائل كالتالي: يكتب الملك العظيم أرتحشستا هذه الأمور إلى الأمراء والولاة الذين تحت قيادته من الهند إلى أثيوبيا في مئة وسبعة وعشرين مقاطعة
\par 2 "بعد ذلك أصبحت سيدًا على العديد من الأمم، وكانت لي السيادة على العالم أجمع، ولم أكن أتعالى بغطرسة سلطتي، بل كنت أحمل نفسي دائمًا بالإنصاف واللطف، وكنت أعتزم أن أجعل رعيتي دائمًا في حياة هادئة، وأن أجعل مملكتي مسالمة، ومفتوحة للمرور إلى أقصى السواحل، لتجديد السلام، وهو ما يرغب فيه جميع البشر.
\par 3 "وعندما سألت مستشاريّ كيف يمكن أن يتم ذلك، قال أمان، الذي كان متفوقًا في الحكمة بيننا، والذي تم اعتماده لحسن نيته الدائمة وإخلاصه الثابت، وكان له شرف المركز الثاني في المملكة،
\par 4 أعلن لنا أنه في كل الأمم في كل أنحاء العالم كان هناك شعب خبيث متفرق، كان لديه قوانين تخالف كل الأمم، وكان دائمًا يحتقر وصايا الملوك، بحيث لا يمكن توحيد ممالكنا، الذي كنا ننوي أن نحققه بشرف.
\par 5 وإذ نرى أننا نفهم أن هذا الشعب وحده هو الذي يعارض جميع البشر باستمرار، ويختلف في الطريقة الغريبة لقوانينه، ويسبب الشر لدولتنا، ويعمل على إحداث كل الأذى الذي يمكنه أن يفعله حتى لا تستقر مملكتنا بقوة:
\par 6 "لذلك أمرنا أن جميع الذين تم الإشارة إليهم كتابة إليكم من قبل أمان، الذي هو مسؤول عن الأمور، والذي هو قريب منا، سيتم تدميرهم جميعًا، مع زوجاتهم وأطفالهم، بالسيف من أعدائهم، دون كل رحمة أو شفقة، في اليوم الرابع عشر من الشهر الثاني عشر من شهر أدار من هذا العام الجاري.
\par 7 حتى يتسنى لأولئك، الذين كانوا في الماضي وما زالوا أيضًا أشرارًا، أن يذهبوا إلى القبر في يوم واحد وبعنف، وهكذا يجعلون شؤوننا مستقرة إلى الأبد، ودون مشاكل
\par 8 ثم فكر مردوخيوس في جميع أعمال الرب، وصلى إليه،
\par 9 قائلين: أيها الرب، الرب، الملك القدير: لأن العالم كله في قبضتك، وإن عيّنت خلاص إسرائيل، فلا أحد يستطيع مقاومتك
\par 10 لأنك أنت صنعت السماء والأرض وكل العجائب التي تحت السماء.
\par 11 أنت رب كل شيء، ولا أحد يستطيع مقاومتك، فأنت الرب
\par 12 أنت تعلم كل شيء، وأنت تعلم يا رب أنني لم أنحني أمام أمان المتكبر ازدراءً ولا كبرياءً ولا رغبةً في المجد
\par 13 لأنه كان بإمكاني أن أكتفي بحسن نية من أجل خلاص إسرائيل لأقبّل باطن قدميه
\par 14 ولكني فعلت هذا لكي لا أُفضّل مجد الإنسان على مجد الله. ولن أعبد سواك يا الله، ولن أفعل ذلك بكبرياء
\par 15 والآن، أيها الرب الإله الملك، اشفق على شعبك، لأن أعينهم علينا ليبيدونا؛ نعم، إنهم يرغبون في تدمير الميراث الذي كان لك منذ البدء
\par 16 لا تحتقر النصيب الذي أخرجته لنفسك من مصر
\par 17 استمع إلى صلاتي، وارحم ميراثك. حوّل حزننا إلى فرح، فنحيا يا رب ونسبح اسمك. ولا تقطع أفواه مسبحيك يا رب
\par 18 فصرخ جميع إسرائيل إلى الرب صرخة شديدة، لأن موتهم كان أمام أعينهم

\chapter{5}

\par 1 والملكة أستير أيضًا، إذ كانت خائفة من الموت، لجأت إلى الرب:
\par 2 وخلعت ثياب مجدها، ولبست ثياب الضيق والحزن، وبدلًا من الأطياب الثمينة غطت رأسها بالرماد والروث، وأذلت جسدها جدًا، وملأت جميع أماكن فرحها بشعرها الممزق
\par 3 وصلّت إلى الرب إله إسرائيل قائلة: يا سيدي، أنت وحدك ملكنا. أغثني أنا البائسة التي ليس لها معين سواك
\par 4 لأن خطري في يدي.
\par 5 منذ صباي، سمعت في سبط عائلتي أنك يا رب أخذت إسرائيل من بين جميع الشعوب، وآباءنا من جميع آبائهم، ميراثًا أبديًا، وقد فعلت كل ما وعدتهم به
\par 6 والآن قد أخطأنا أمامك، لذلك سلمتنا إلى أيدي أعدائنا،
\par 7 لأننا كنا نعبد آلهتهم: يا رب، أنت بار.
\par 8 ولكن لا يقنعهم أننا في أسر مرير، بل ضربوا أيديهم بأصنامهم،
\par 9 أنهم سيبطلون الأمر الذي شرعته بفمك، ويهلكوا ميراثك، ويسدوا أفواه مسبحيك، ويطفئوا مجد بيتك ومذبحك،
\par 10 وافتحوا أفواه الأمم ليُشيدوا بتسابيح الأصنام، وليُعظموا ملكًا بشريًا إلى الأبد
\par 11 يا رب، لا تعطي صولجانك لأولئك الذين ليسوا شيئًا، ولا تدعهم يسخرون من سقوطنا. بل يحولون مكائدهم إلى أنفسهم، ويجعلون من بدأ هذا ضدنا عبرة.
\par 12 أذكر يا رب أن تعرفنا في وقت ضيقنا، وأعطني الشجاعة يا ملك الأمم ورب كل قوة.
\par 13 أعطني كلامًا بليغًا في فمي أمام الأسد: حوّل قلبه إلى كراهية من يحاربنا، لكي تكون هناك نهاية له ولكل من يشبهه في الرأي
\par 14 لكن نجنا بيدك، وأعنّا أنا الخاوي، الذي ليس له معين سواك
\par 15 أنت تعلم كل شيء يا رب، أنت تعلم أنني أكره مجد الأشرار، وأكره مضجع غير المختونين، وكل الأمم
\par 16 أنت تعلم حاجتي، لأني أكره علامة جلالتي التي تكون على رأسي في الأيام التي أظهر فيها، وأكرهها كقطعة قماش الحيض، ولا أرتديها عندما أكون بمفردي
\par 17 وأن أمتك لم تأكل على مائدة هامان، وأنني لم أُقدّر وليمة الملك، ولم أشرب خمر سكب التقدمات
\par 18 ولم تفرح أمتك منذ اليوم الذي أتيت فيه إلى هنا إلا بك، أيها الرب إله إبراهيم
\par 19 يا إلهنا القدير فوق كل شيء، اسمع صوت البائسين، وأنقذنا من أيدي الأشرار، وأنقذني من خوفي

\chapter{6}

\par 1 وفي اليوم الثالث، بعدما فرغت من صلاتها، خلعَت ثياب حدادها ولبست ثياب مجدها
\par 2 وبعد أن تزينت بشكل رائع، وبعد أن دعت الله، الذي هو ناظر كل شيء ومخلصه، أخذت معها جاريتين:
\par 3 وعلى من اتكأت، كأنها تحمل نفسها برشاقة؛
\par 4 وتبعتها الأخرى وهي تحمل ذيلها.
\par 5 وكانت حمراء من كمال جمالها، وكان وجهها بشوشًا ولطيفًا جدًا: لكن قلبها كان يكتئب من الخوف
\par 6 ثم بعد أن عبرت من جميع الأبواب، وقفت أمام الملك، الذي كان جالسًا على عرشه الملكي، مرتديًا جميع ثياب جلالته، وكلها متلألئة بالذهب والأحجار الكريمة؛ وكان مهيبًا جدًا
\par 7 ثم رفع وجهه الذي كان يتألق بالجلال، ونظر إليها بشراسة شديدة: فسقطت الملكة، وشحبت، وأغمي عليها، وانحنت على رأس الفتاة التي كانت أمامها
\par 8 ثم غيّر الله روح الملك إلى لطف، فقفز من عرشه خوفًا، وأخذها بين ذراعيه، حتى عادت إلى وعيها، وعزّاها بكلمات محبة، وقال لها:
\par 9 إستير، ما الأمر؟ أنا أخوكِ، تشجعي:
\par 10 لا تموت، ولو كانت وصيتنا عامة: اقترب.
\par 11 وهكذا رفع صولجانه الذهبي ووضعه على رقبتها،
\par 12 فاحتضنها وقال: تكلمي معي.
\par 13 فقالت له: رأيتك يا سيدي كأنك ملاك الله، فاضطرب قلبي خوفاً من جلالك.
\par 14 لأنك عجيب يا رب، ووجهك ممتلئ نعمة.
\par 15 وبينما هي تتكلم سقطت على الأرض من شدة التعب.
\par 16 فاضطرب الملك، وعزاها جميع عبيده.

\chapter{7}

\par 1 من الملك العظيم أرتحشستا إلى أمراء وحكام المئة والسبعة والعشرين مقاطعة من الهند إلى أثيوبيا، وإلى جميع رعايانا المخلصين، تحية
\par 2 كثيرون، كلما تم تكريمهم بسخاء أمرائهم الكرام، زاد فخرهم،
\par 3 ولا نسعى لإيذاء رعيتنا فقط، ولكن بما أننا لا نستطيع تحمل الوفرة، فلنتخذ إجراءً ضد من يحسنون إليهم أيضًا:
\par 4 ولا تنزعوا الامتنان من بين الناس فحسب، بل أيضًا ترفعوا بكلمات مجيدة من أشخاص فاسقون، لم يكونوا صالحين أبدًا، يظنون أنهم يهربون من عدالة الله، الذي يرى كل شيء ويكره الشر
\par 5 في كثير من الأحيان، يتسبب الكلام العادل لأولئك الذين أُؤتمنوا على إدارة شؤون أصدقائهم في أن يصبح العديد من أصحاب السلطة شركاء في دماء الأبرياء، ويحيط بهم في كوارث لا علاج لها:
\par 6 يخدعون بكذبهم وخداعهم الفاسق براءة الأمراء وصلاحهم
\par 7 الآن، يمكنك أن ترى هذا، كما ذكرنا، ليس من خلال التواريخ القديمة بقدر ما يمكنك ذلك، إذا بحثت عما تم فعله بشكل شرير مؤخرًا من خلال السلوك الفاسد لأولئك الذين تم وضعهم في السلطة بشكل غير جدير
\par 8 ويجب علينا أن نهتم بالوقت القادم، لكي تكون مملكتنا هادئة ومسالمة لجميع الناس،
\par 9 سواء من خلال تغيير أغراضنا، أو من خلال الحكم دائمًا على الأشياء الواضحة بإجراء أكثر مساواة
\par 10 لأن أمان، المقدوني، ابن أمداثا، كان غريبًا عن الدم الفارسي، وبعيدًا كل البعد عن صلاحنا، وكغريب استقبل منا،
\par 11 لقد نال حتى الآن النعمة التي نظهرها تجاه كل أمة، حيث دُعي أبانا، وكان يُكرم باستمرار من كل من يلي الملك
\par 12 لكنه، إذ لم يكن يتحلى بكرامته العظيمة، كاد أن يحرمنا من ملكوتنا وحياتنا:
\par 13 بعد أن سعوا من خلال خداعهم الماكر والمتنوع إلى هلاكنا، وكذلك مردوخا، الذي أنقذ حياتنا، وضمن لنا الخير باستمرار، وكذلك إستير التي لا عيب فيها، المشاركة في مملكتنا، مع أمتهم بأكملها
\par 14 لأنه بهذه الطريقة ظن، إذ وجدنا بلا أصدقاء، أن ينقل مملكة الفرس إلى المقدونيين
\par 15 لكننا نجد أن اليهود، الذين سلّمهم هذا الشرير إلى الهلاك التام، ليسوا أشرارًا، بل يعيشون وفقًا لأكثر القوانين عدلًا:
\par 16 وأن يكونوا أبناء الله العلي العظيم الحي الذي رتب المملكة لنا ولآبائنا بأفضل طريقة.
\par 17 لذلك، يُحسن بكم عدم تنفيذ الرسائل التي أرسلها إليكم أمان بن أماداثا
\par 18 "فإن الذي صنع هذه الأشياء يُصلب على أبواب شوشن مع كل عائلته: الله الذي يحكم كل شيء، ينتقم منه سريعًا حسب استحقاقاته."
\par 19 لذلك، يجب عليكم نشر نسخة هذه الرسالة في جميع الأماكن، حتى يتمكن اليهود من العيش بحرية وفقًا لقوانينهم الخاصة
\par 20 وتُعينونهم، حتى في ذلك اليوم نفسه، وهو الثالث عشر من الشهر الثاني عشر من شهر أذار، يُنتقم منهم الذين يهاجمونهم في وقت بلائهم
\par 21 لأن الله القدير قد حول لهم اليوم الذي كان ينبغي أن يهلك فيه الشعب المختار إلى فرح
\par 22 فاجعلوه من بين أعيادكم المقدسة يومًا مقدسًا مع كل ولائم
\par 23 لكي يكون هناك أمان لنا وللفرس المتضررين الآن وفي الآخرة؛ ولكن لأولئك الذين يتآمرون ضدنا، سيكونون ذكرى للدمار
\par 24 لذلك، فإن كل مدينة أو بلد لا يفعل مثل هذه الأمور، سيُدمر بلا رحمة بالنار والسيف، ولن يصبح فقط غير صالح للبشر، بل سيصبح أيضًا مكروهًا للغاية للوحوش البرية والطيور إلى الأبد

\end{document}