\begin{document}

\title{ملاخي}


\chapter{1}

\par 1 وَحْيُ كَلِمَةِ الرَّبِّ لإِسْرَائِيلَ عَنْ يَدِ مَلاَخِي:
\par 2 [أَحْبَبْتُكُمْ قَالَ الرَّبُّ]. وَقُلْتُمْ: [بِمَا أَحْبَبْتَنَا؟] أَلَيْسَ عِيسُو أَخاً لِيَعْقُوبَ يَقُولُ الرَّبُّ وَأَحْبَبْتُ يَعْقُوبَ
\par 3 وَأَبْغَضْتُ عِيسُوَ وَجَعَلْتُ جِبَالَهُ خَرَاباً وَمِيرَاثَهُ لِذِئَابِ الْبَرِّيَّةِ؟
\par 4 لأَنَّ أَدُومَ قَالَ قَدْ: [هُدِمْنَا فَنَعُودُ وَنَبْنِي الْخِرَبَ]. هَكَذَا قَالَ رَبُّ الْجُنُودِ: [هُمْ يَبْنُونَ وَأَنَا أَهْدِمُ. وَيَدْعُونَهُمْ تُخُومَ الشَّرِّ وَالشَّعْبَ الَّذِي غَضِبَ عَلَيْهِ الرَّبُّ إِلَى الأَبَدِ.
\par 5 فَتَرَى أَعْيُنُكُمْ وَتَقُولُونَ لِيَتَعَظَّمِ الرَّبُّ مِنْ عَُِنْدِ تُخُمِ إِسْرَائِيلَ.
\par 6 [الاِبْنُ يُكْرِمُ أَبَاهُ وَالْعَبْدُ يُكْرِمُ سَيِّدَهُ. فَإِنْ كُنْتُ أَنَا أَباً فَأَيْنَ كَرَامَتِي؟ وَإِنْ كُنْتُ سَيِّداً فَأَيْنَ هَيْبَتِي؟ قَالَ لَكُمْ رَبُّ الْجُنُودِ أَيُّهَا الْكَهَنَةُ الْمُحْتَقِرُونَ اسْمِي. وَتَقُولُونَ: بِمَ احْتَقَرْنَا اسْمَكَ؟
\par 7 تُقَرِّبُونَ خُبْزاً نَجِساً عَلَى مَذْبَحِي. وَتَقُولُونَ: بِمَ نَجَّسْنَاكَ؟ بِقَوْلِكُمْ إِنَّ مَائِدَةَ الرَّبِّ مُحْتَقَرَةٌ.
\par 8 وَإِنْ قَرَّبْتُمُ الأَعْمَى ذَبِيحَةً أَفَلَيْسَ ذَلِكَ شَرّاً؟ وَإِنْ قَرَّبْتُمُ الأَعْرَجَ وَالسَّقِيمَ أَفَلَيْسَ ذَلِكَ شَرّاً؟ قَرِّبْهُ لِوَالِيكَ أَفَيَرْضَى عَلَيْكَ أَوْ يَرْفَعُ وَجْهَكَ؟ قَالَ رَبُّ الْجُنُودِ.
\par 9 وَالآنَ تَرَضُّوا وَجْهَ اللَّهِ فَيَتَرَأَّفَ عَلَيْنَا. هَذِهِ كَانَتْ مِنْ يَدِكُمْ. هَلْ يَرْفَعُ وَجْهَكُمْ؟ قَالَ رَبُّ الْجُنُودِ.
\par 10 [مَنْ فِيكُمْ يُغْلِقُ الْبَابَ بَلْ لاَ تُوقِدُونَ عَلَى مَذْبَحِي مَجَّاناً؟ لَيْسَتْ لِي مَسَّرَةٌ بِكُمْ قَالَ رَبُّ الْجُنُودِ وَلاَ أَقْبَلُ تَقْدِمَةً مِنْ يَدِكُمْ.
\par 11 لأَنَّهُ مِنْ مَشْرِقِ الشَّمْسِ إِلَى مَغْرِبِهَا اسْمِي عَظِيمٌ بَيْنَ الأُمَمِ وَفِي كُلِّ مَكَانٍ يُقَرَّبُ لاِسْمِي بَخُورٌ وَتَقْدِمَةٌ طَاهِرَةٌ لأَنَّ اسْمِي عَظِيمٌ بَيْنَ الأُمَمِ قَالَ رَبُّ الْجُنُودِ.
\par 12 أَمَّا أَنْتُمْ فَمُنَجِّسُوهُ بِقَوْلِكُمْ: إِنَّ مَائِدَةَ الرَّبِّ تَنَجَّسَتْ وَثَمَرَتَهَا مُحْتَقَرٌ طَعَامُهَا.
\par 13 وَقُلْتُمْ: مَا هَذِهِ الْمَشَقَّةُ؟ وَتَأَفَّفْتُمْ عَلَيْهِ قَالَ رَبُّ الْجُنُودِ وَجِئْتُمْ بِالْمُغْتَصَبِ وَالأَعْرَجِ وَالسَّقِيمِ فَأَتَيْتُمْ بِالتَّقْدِمَةِ. فَهَلْ أَقْبَلُهَا مِنْ يَدِكُمْ؟ قَالَ الرَّبُّ.
\par 14 وَمَلْعُونٌ الْمَاكِرُ الَّذِي يُوجَدُ فِي قَطِيعِهِ ذَكَرٌ وَيَنْذُرُ وَيَذْبَحُ لِلسَّيِّدِ عَائِباً. لأَنِّي أَنَا مَلِكٌ عَظِيمٌ قَالَ رَبُّ الْجُنُودِ وَاسْمِي مَهِيبٌ بَيْنَ الأُمَمِ].

\chapter{2}

\par 1 [وَالآنَ إِلَيْكُمْ هَذِهِ الْوَصِيَّةُ أَيُّهَا الْكَهَنَةُ:
\par 2 إِنْ كُنْتُمْ لاَ تَسْمَعُونَ وَلاَ تَجْعَلُونَ فِي الْقَلْبِ لِتُعْطُوا مَجْداً لاِسْمِي قَالَ رَبُّ الْجُنُودِ فَإِنِّي أُرْسِلُ عَلَيْكُمُ اللَّعْنَ. وَأَلْعَنُ بَرَكَاتِكُمْ بَلْ قَدْ لَعَنْتُهَا لأَنَّكُمْ لَسْتُمْ جَاعِلِينَ فِي الْقَلْبِ.
\par 3 هَئَنَذَا أَنْتَهِرُ لَكُمُ الزَّرْعَ وَأَمُدُّ الْفَرْثَ عَلَى وُجُوهِكُمْ فَرْثَ أَعْيَادِكُمْ فَتُنْزَعُونَ مَعَهُ.
\par 4 فَتَعْلَمُونَ أَنِّي أَرْسَلْتُ إِلَيْكُمْ هَذِهِ الْوَصِيَّةَ لِكَوْنِ عَهْدِي مَعَ لاَوِي قَالَ رَبُّ الْجُنُودِ.
\par 5 كَانَ عَهْدِي مَعَهُ لِلْحَيَاةِ وَالسَّلاَمِ وَأَعْطَيْتُهُ إِيَّاهُمَا لِلتَّقْوَى. فَاتَّقَانِي وَمِنِ اسْمِي ارْتَاعَ هُوَ.
\par 6 شَرِيعَةُ الْحَقِّ كَانَتْ فِي فَمِهِ وَإِثْمٌ لَمْ يُوجَدْ فِي شَفَتَيْهِ. سَلَكَ مَعِي فِي السَّلاَمِ وَالاِسْتِقَامَةِ وَأَرْجَعَ كَثِيرِينَ عَنِ الإِثْمِ.
\par 7 لأَنَّ شَفَتَيِ الْكَاهِنِ تَحْفَظَانِ مَعْرِفَةً وَمِنْ فَمِهِ يَطْلُبُونَ الشَّرِيعَةَ لأَنَّهُ رَسُولُ رَبِّ الْجُنُودِ.
\par 8 أَمَّا أَنْتُمْ فَحِدْتُمْ عَنِ الطَّرِيقِ وَأَعْثَرْتُمْ كَثِيرِينَ بِالشَّرِيعَةِ. أَفْسَدْتُمْ عَهْدَ لاَوِي قَالَ رَبُّ الْجُنُودِ.
\par 9 فَأَنَا أَيْضاً صَيَّرْتُكُمْ مُحْتَقَرِينَ وَدَنِيئِينَ عِنْدَ كُلِّ الشَّعْبِ كَمَا أَنَّكُمْ لَمْ تَحْفَظُوا طُرُقِي بَلْ حَابَيْتُمْ فِي الشَّرِيعَةِ].
\par 10 أَلَيْسَ أَبٌ وَاحِدٌ لِكُلِّنَا؟ أَلَيْسَ إِلَهٌ وَاحِدٌ خَلَقَنَا؟ فَلِمَاذَا نَغْدُرُ الرَّجُلُ بِأَخِيهِ لِتَدْنِيسِ عَهْدِ آبَائِنَا؟
\par 11 غَدَرَ يَهُوذَا وَعُمِلَ الرِّجْسُ فِي إِسْرَائِيلَ وَفِي أُورُشَلِيمَ. لأَنَّ يَهُوذَا قَدْ نَجَّسَ قُدْسَ الرَّبِّ الَّذِي أَحَبَّهُ وَتَزَوَّجَ بِنْتَ إِلَهٍ غَرِيبٍ.
\par 12 يَقْطَعُ الرَّبُّ الرَّجُلَ الَّذِي يَفْعَلُ هَذَا السَّاهِرَ وَالْمُجِيبَ مِنْ خِيَامِ يَعْقُوبَ وَمَنْ يُقَرِّبُ تَقْدِمَةً لِرَبِّ الْجُنُودِ.
\par 13 وَقَدْ فَعَلْتُمْ هَذَا ثَانِيَةً مُغَطِّينَ مَذْبَحَ الرَّبِّ بِالدُّمُوعِ بِالْبُكَاءِ وَالصُّرَاخِ فَلاَ تُرَاعَى التَّقْدِمَةُ بَعْدُ وَلاَ يُقْبَلُ الْمُرْضِي مِنْ يَدِكُمْ.
\par 14 فَقُلْتُمْ: [لِمَاذَا؟] مِنْ أَجْلِ أَنَّ الرَّبَّ هُوَ الشَّاهِدُ بَيْنَكَ وَبَيْنَ امْرَأَةِ شَبَابِكَ الَّتِي أَنْتَ غَدَرْتَ بِهَا وَهِيَ قَرِينَتُكَ وَامْرَأَةُ عَهْدِكَ.
\par 15 أَفَلَمْ يَفْعَلْ وَاحِدٌ وَلَهُ بَقِيَّةُ الرُّوحِ؟ وَلِمَاذَا الْوَاحِدُ؟ طَالِباً زَرْعَ اللَّهِ. فَاحْذَرُوا لِرُوحِكُمْ وَلاَ يَغْدُرْ أَحَدٌ بِامْرَأَةِ شَبَابِهِ.
\par 16 [لأَنَّهُ يَكْرَهُ الطَّلاَقَ] قَالَ الرَّبُّ إِلَهُ إِسْرَائِيلَ [وَأَنْ يُغَطِّيَ أَحَدٌ الظُّلْمَ بِثَوْبِهِ] قَالَ رَبُّ الْجُنُودِ. فَاحْذَرُوا لِرُوحِكُمْ لِئَلاَّ تَغْدُرُوا.
\par 17 لَقَدْ أَتْعَبْتُمُ الرَّبَّ بِكَلاَمِكُمْ. وَقُلْتُمْ: [بِمَ أَتْعَبْنَاهُ؟] بِقَوْلِكُمْ: [كُلُّ مَنْ يَفْعَلُ الشَّرَّ فَهُوَ صَالِحٌ فِي عَيْنَيِ الرَّبِّ وَهُوَ يُسَرُّ بِهِمْ]. أَوْ: [أَيْنَ إِلَهُ الْعَدْلِ؟].

\chapter{3}

\par 1 هَئَنَذَا أُرْسِلُ مَلاَكِي فَيُهَيِّئُ الطَّرِيقَ أَمَامِي. وَيَأْتِي بَغْتَةً إِلَى هَيْكَلِهِ السَّيِّدُ الَّذِي تَطْلُبُونَهُ وَمَلاَكُ الْعَهْدِ الَّذِي تُسَرُّونَ بِهِ. هُوَذَا يَأْتِي قَالَ رَبُّ الْجُنُودِ.
\par 2 وَمَنْ يَحْتَمِلُ يَوْمَ مَجِيئِهِ وَمَنْ يَثْبُتُ عِنْدَ ظُهُورِهِ؟ لأَنَّهُ مِثْلُ نَارِ الْمُمَحِّصِ وَمِثْلُ أَشْنَانِ الْقَصَّارِ.
\par 3 فَيَجْلِسُ مُمَحِّصاً وَمُنَقِّياً لِلْفِضَّةِ. فَيُنَقِّي بَنِي لاَوِي وَيُصَفِّيهِمْ كَالذَّهَبِ وَالْفِضَّةِ لِيَكُونُوا مُقَرِّبِينَ لِلرَّبِّ تَقْدِمَةً بِالْبِرِّ.
\par 4 فَتَكُونُ تَقْدِمَةُ يَهُوذَا وَأُورُشَلِيمَ مَرْضِيَّةً لِلرَّبِّ كَمَا فِي أَيَّامِ الْقِدَمِ وَكَمَا فِي السِّنِينَ الْقَدِيمَةِ.
\par 5 وَأَقْتَرِبُ إِلَيْكُمْ لِلْحُكْمِ وَأَكُونُ شَاهِداً سَرِيعاً عَلَى السَّحَرَةِ وَعَلَى الْفَاسِقِينَ وَعَلَى الْحَالِفِينَ زُوراً وَعَلَى السَّالِبِينَ أُجْرَةَ الأَجِيرِ: الأَرْمَلَةِ وَالْيَتِيمِ وَمَنْ يَصُدُّ الْغَرِيبَ وَلاَ يَخْشَانِي قَالَ رَبُّ الْجُنُودِ.
\par 6 لأَنِّي أَنَا الرَّبُّ لاَ أَتَغَيَّرُ فَأَنْتُمْ يَا بَنِي يَعْقُوبَ لَمْ تَفْنُوا.
\par 7 [مِنْ أَيَّامِ آبَائِكُمْ حِدْتُمْ عَنْ فَرَائِضِي وَلَمْ تَحْفَظُوهَا. ارْجِعُوا إِلَيَّ أَرْجِعْ إِلَيْكُمْ قَالَ رَبُّ الْجُنُودِ. فَقُلْتُمْ: بِمَاذَا نَرْجِعُ؟
\par 8 أَيَسْلُبُ الإِنْسَانُ اللَّهَ؟ فَإِنَّكُمْ سَلَبْتُمُونِي. فَقُلْتُمْ: بِمَ سَلَبْنَاكَ؟ فِي الْعُشُورِ وَالتَّقْدِمَةِ.
\par 9 قَدْ لُعِنْتُمْ لَعْناً وَإِيَّايَ أَنْتُمْ سَالِبُونَ هَذِهِ الأُمَّةُ كُلُّهَا.
\par 10 هَاتُوا جَمِيعَ الْعُشُورِ إِلَى الْخَزْنَةِ لِيَكُونَ فِي بَيْتِي طَعَامٌ وَجَرِّبُونِي بِهَذَا قَالَ رَبُّ الْجُنُودِ إِنْ كُنْتُ لاَ أَفْتَحُ لَكُمْ كُوى السَّمَاوَاتِ وَأُفِيضُ عَلَيْكُمْ بَرَكَةً حَتَّى لاَ تُوسَعَ.
\par 11 وَأَنْتَهِرُ مِنْ أَجْلِكُمْ الآكِلَ فَلاَ يُفْسِدُ لَكُمْ ثَمَرَ الأَرْضِ وَلاَ يُعْقَرُ لَكُمُ الْكَرْمُ فِي الْحَقْلِ قَالَ رَبُّ الْجُنُودِ.
\par 12 وَيُطَوِّبُكُمْ كُلُّ الأُمَمِ لأَنَّكُمْ تَكُونُونَ أَرْضَ مَسَرَّةٍ قَالَ رَبُّ الْجُنُودِ.
\par 13 [أَقْوَالُكُمُ اشْتَدَّتْ عَلَيَّ قَالَ الرَّبُّ. وَقُلْتُمْ: مَاذَا قُلْنَا عَلَيْكَ؟
\par 14 قُلْتُمْ: عِبَادَةُ اللَّهِ بَاطِلَةٌ وَمَا الْمَنْفَعَةُ مِنْ أَنَّنَا حَفِظْنَا شَعَائِرَهُ وَأَنَّنَا سَلَكْنَا بِالْحُزْنِ قُدَّامَ رَبِّ الْجُنُودِ؟
\par 15 وَالآنَ نَحْنُ مُطَوِّبُونَ الْمُسْتَكْبِرِينَ وَأَيْضاً فَاعِلُو الشَّرِّ يُبْنَوْنَ. بَلْ جَرَّبُوا اللَّهَ وَنَجُوا].
\par 16 حِينَئِذٍ كَلَّمَ مُتَّقُو الرَّبِّ كُلُّ وَاحِدٍ قَرِيبَهُ وَالرَّبُّ أَصْغَى وَسَمِعَ وَكُتِبَ أَمَامَهُ سِفْرُ تَذْكَرَةٍ لِلَّذِينَ اتَّقُوا الرَّبَّ وَلِلْمُفَكِّرِينَ فِي اسْمِهِ.
\par 17 وَيَكُونُونَ لِي قَالَ رَبُّ الْجُنُودِ فِي الْيَوْمِ الَّذِي أَنَا صَانِعٌ خَاصَّةً وَأُشْفِقُ عَلَيْهِمْ كَمَا يُشْفِقُ الإِنْسَانُ عَلَى ابْنِهِ الَّذِي يَخْدِمُهُ.
\par 18 فَتَعُودُونَ وَتُمَيِّزُونَ بَيْنَ الصِّدِّيقِ وَالشِّرِّيرِ بَيْنَ مَنْ يَعْبُدُ اللَّهَ وَمَنْ لاَ يَعْبُدُهُ.

\chapter{4}

\par 1 [فَهُوَذَا يَأْتِي الْيَوْمُ الْمُتَّقِدُ كَالتَّنُّورِ وَكُلُّ الْمُسْتَكْبِرِينَ وَكُلُّ فَاعِلِي الشَّرِّ يَكُونُونَ قَشّاً وَيُحْرِقُهُمُ الْيَوْمُ الآتِي قَالَ رَبُّ الْجُنُودِ فَلاَ يُبْقِي لَهُمْ أَصْلاً وَلاَ فَرْعاً.
\par 2 [وَلَكُمْ أَيُّهَا الْمُتَّقُونَ اسْمِي تُشْرِقُ شَمْسُ الْبِرِّ وَالشِّفَاءُ فِي أَجْنِحَتِهَا فَتَخْرُجُونَ وَتَنْشَأُونَ كَعُجُولِ الصِّيرَةِ.
\par 3 وَتَدُوسُونَ الأَشْرَارَ لأَنَّهُمْ يَكُونُونَ رَمَاداً تَحْتَ بُطُونِ أَقْدَامِكُمْ يَوْمَ أَفْعَلُ هَذَا قَالَ رَبُّ الْجُنُودِ.
\par 4 [اذْكُرُوا شَرِيعَةَ مُوسَى عَبْدِي الَّتِي أَمَرْتُهُ بِهَا فِي حُورِيبَ عَلَى كُلِّ إِسْرَائِيلَ. الْفَرَائِضَ وَالأَحْكَامَ.
\par 5 [هَئَنَذَا أُرْسِلُ إِلَيْكُمْ إِيلِيَّا النَّبِيَّ قَبْلَ مَجِيءِ يَوْمِ الرَّبِّ الْيَوْمِ الْعَظِيمِ وَالْمَخُوفِ
\par 6 فَيَرُدُّ قَلْبَ الآبَاءِ عَلَى الأَبْنَاءِ وَقَلْبَ الأَبْنَاءِ عَلَى آبَائِهِمْ. لِئَلاَّ آتِيَ وَأَضْرِبَ الأَرْضَ بِلَعْنٍ].

\end{document}