\begin{document}

\title{يعقوب}


\chapter{1}

\par 1 يَعْقُوبُ، عَبْدُ اللَّهِ وَالرَّبِّ يَسُوعَ الْمَسِيحِ، يُهْدِي السَّلاَمَ إِلَى الاِثْنَيْ عَشَرَ سِبْطاً الَّذِينَ فِي الشَّتَاتِ.
\par 2 اِحْسِبُوهُ كُلَّ فَرَحٍ يَا إِخْوَتِي حِينَمَا تَقَعُونَ فِي تَجَارِبَ مُتَنَّوِعَةٍ،
\par 3 عَالِمِينَ أَنَّ امْتِحَانَ إِيمَانِكُمْ يُنْشِئُ صَبْراً.
\par 4 وَأَمَّا الصَّبْرُ فَلْيَكُنْ لَهُ عَمَلٌ تَامٌّ، لِكَيْ تَكُونُوا تَامِّينَ وَكَامِلِينَ غَيْرَ نَاقِصِينَ فِي شَيْءٍ.
\par 5 وَإِنَّمَا إِنْ كَانَ أَحَدُكُمْ تُعْوِزُهُ حِكْمَةٌ فَلْيَطْلُبْ مِنَ اللَّهِ الَّذِي يُعْطِي الْجَمِيعَ بِسَخَاءٍ وَلاَ يُعَيِّرُ، فَسَيُعْطَى لَهُ.
\par 6 وَلَكِنْ لِيَطْلُبْ بِإِيمَانٍ غَيْرَ مُرْتَابٍ الْبَتَّةَ، لأَنَّ الْمُرْتَابَ يُشْبِهُ مَوْجاً مِنَ الْبَحْرِ تَخْبِطُهُ الرِّيحُ وَتَدْفَعُهُ.
\par 7 فَلاَ يَظُنَّ ذَلِكَ الإِنْسَانُ أَنَّهُ يَنَالُ شَيْئاً مِنْ عِنْدِ الرَّبِّ.
\par 8 رَجُلٌ ذُو رَأْيَيْنِ هُوَ مُتَقَلْقِلٌ فِي جَمِيعِ طُرُقِهِ.
\par 9 وَلْيَفْتَخِرِ الأَخُ الْمُتَّضِعُ بِارْتِفَاعِهِ،
\par 10 وَأَمَّا الْغَنِيُّ فَبِاتِّضَاعِهِ، لأَنَّهُ كَزَهْرِ الْعُشْبِ يَزُولُ.
\par 11 لأَنَّ الشَّمْسَ أَشْرَقَتْ بِالْحَرِّ، فَيَبَّسَتِ الْعُشْبَ، فَسَقَطَ زَهْرُهُ وَفَنِيَ جَمَالُ مَنْظَرِهِ. هَكَذَا يَذْبُلُ الْغَنِيُّ أَيْضاً فِي طُرُقِهِ.
\par 12 طُوبَى لِلرَّجُلِ الَّذِي يَحْتَمِلُ التَّجْرِبَةَ، لأَنَّهُ إِذَا تَزَكَّى يَنَالُ «إِكْلِيلَ الْحَيَاةِ» الَّذِي وَعَدَ بِهِ الرَّبُّ لِلَّذِينَ يُحِبُّونَهُ.
\par 13 لاَ يَقُلْ أَحَدٌ إِذَا جُرِّبَ إِنِّي أُجَرَّبُ مِنْ قِبَلِ اللَّهِ، لأَنَّ اللَّهَ غَيْرُ مُجَرَّبٍ بِالشُّرُورِ وَهُوَ لاَ يُجَرِّبُ أَحَداً.
\par 14 وَلَكِنَّ كُلَّ وَاحِدٍ يُجَرَّبُ إِذَا انْجَذَبَ وَانْخَدَعَ مِنْ شَهْوَتِهِ.
\par 15 ثُمَّ الشَّهْوَةُ إِذَا حَبِلَتْ تَلِدُ خَطِيَّةً، وَالْخَطِيَّةُ إِذَا كَمُلَتْ تُنْتِجُ مَوْتاً.
\par 16 لاَ تَضِلُّوا يَا إِخْوَتِي الأَحِبَّاءَ.
\par 17 كُلُّ عَطِيَّةٍ صَالِحَةٍ وَكُلُّ مَوْهِبَةٍ تَامَّةٍ هِيَ مِنْ فَوْقُ، نَازِلَةٌ مِنْ عِنْدِ أَبِي الأَنْوَارِ، الَّذِي لَيْسَ عِنْدَهُ تَغْيِيرٌ وَلاَ ظِلُّ دَوَرَانٍ.
\par 18 شَاءَ فَوَلَدَنَا بِكَلِمَةِ الْحَقِّ لِكَيْ نَكُونَ بَاكُورَةً مِنْ خَلاَئِقِهِ.
\par 19 إِذاً يَا إِخْوَتِي الأَحِبَّاءَ، لِيَكُنْ كُلُّ إِنْسَانٍ مُسْرِعاً فِي الاِسْتِمَاعِ، مُبْطِئاً فِي التَّكَلُّمِ، مُبْطِئاً فِي الْغَضَبِ،
\par 20 لأَنَّ غَضَبَ الإِنْسَانِ لاَ يَصْنَعُ بِرَّ اللَّهِ.
\par 21 لِذَلِكَ اطْرَحُوا كُلَّ نَجَاسَةٍ وَكَثْرَةَ شَرٍّ. فَاقْبَلُوا بِوَدَاعَةٍ الْكَلِمَةَ الْمَغْرُوسَةَ الْقَادِرَةَ أَنْ تُخَلِّصَ نُفُوسَكُمْ.
\par 22 وَلَكِنْ كُونُوا عَامِلِينَ بِالْكَلِمَةِ، لاَ سَامِعِينَ فَقَطْ خَادِعِينَ نُفُوسَكُمْ.
\par 23 لأَنَّهُ إِنْ كَانَ أَحَدٌ سَامِعاً لِلْكَلِمَةِ وَلَيْسَ عَامِلاً، فَذَاكَ يُشْبِهُ رَجُلاً نَاظِراً وَجْهَ خِلْقَتِهِ فِي مِرْآةٍ،
\par 24 فَإِنَّهُ نَظَرَ ذَاتَهُ وَمَضَى، وَلِلْوَقْتِ نَسِيَ مَا هُوَ.
\par 25 وَلَكِنْ مَنِ اطَّلَعَ عَلَى النَّامُوسِ الْكَامِلِ - نَامُوسِ الْحُرِّيَّةِ - وَثَبَتَ، وَصَارَ لَيْسَ سَامِعاً نَاسِياً بَلْ عَامِلاً بِالْكَلِمَةِ، فَهَذَا يَكُونُ مَغْبُوطاً فِي عَمَلِهِ.
\par 26 إِنْ كَانَ أَحَدٌ فِيكُمْ يَظُنُّ أَنَّهُ دَيِّنٌ، وَهُوَ لَيْسَ يُلْجِمُ لِسَانَهُ، بَلْ يَخْدَعُ قَلْبَهُ، فَدِيَانَةُ هَذَا بَاطِلَةٌ.
\par 27 اَلدِّيَانَةُ الطَّاهِرَةُ النَّقِيَّةُ عِنْدَ اللَّهِ الآبِ هِيَ هَذِهِ: افْتِقَادُ الْيَتَامَى وَالأَرَامِلِ فِي ضِيقَتِهِمْ، وَحِفْظُ الإِنْسَانِ نَفْسَهُ بِلاَ دَنَسٍ مِنَ الْعَالَمِ.

\chapter{2}

\par 1 يَا إِخْوَتِي، لاَ يَكُنْ لَكُمْ إِيمَانُ رَبِّنَا يَسُوعَ الْمَسِيحِ، رَبِّ الْمَجْدِ، فِي الْمُحَابَاةِ.
\par 2 فَإِنَّهُ إِنْ دَخَلَ إِلَى مَجْمَعِكُمْ رَجُلٌ بِخَوَاتِمِ ذَهَبٍ فِي لِبَاسٍ بَهِيٍّ، وَدَخَلَ أَيْضاً فَقِيرٌ بِلِبَاسٍ وَسِخٍ،
\par 3 فَنَظَرْتُمْ إِلَى اللاَّبِسِ اللِّبَاسَ الْبَهِيَّ وَقُلْتُمْ لَهُ: «اجْلِسْ أَنْتَ هُنَا حَسَناً». وَقُلْتُمْ لِلْفَقِيرِ: «قِفْ أَنْتَ هُنَاكَ» أَوِ: «اجْلِسْ هُنَا تَحْتَ مَوْطِئِ قَدَمَيَّ»
\par 4 فَهَلْ لاَ تَرْتَابُونَ فِي أَنْفُسِكُمْ، وَتَصِيرُونَ قُضَاةَ أَفْكَارٍ شِرِّيرَةٍ؟
\par 5 اسْمَعُوا يَا إِخْوَتِي الأَحِبَّاءَ، أَمَا اخْتَارَ اللَّهُ فُقَرَاءَ هَذَا الْعَالَمِ أَغْنِيَاءَ فِي الإِيمَانِ، وَوَرَثَةَ الْمَلَكُوتِ الَّذِي وَعَدَ بِهِ الَّذِينَ يُحِبُّونَهُ؟
\par 6 وَأَمَّا أَنْتُمْ فَأَهَنْتُمُ الْفَقِيرَ. أَلَيْسَ الأَغْنِيَاءُ يَتَسَلَّطُونَ عَلَيْكُمْ وَهُمْ يَجُرُّونَكُمْ إِلَى الْمَحَاكِمِ؟
\par 7 أَمَا هُمْ يُجَدِّفُونَ عَلَى الاِسْمِ الْحَسَنِ الَّذِي دُعِيَ بِهِ عَلَيْكُمْ؟
\par 8 فَإِنْ كُنْتُمْ تُكَمِّلُونَ النَّامُوسَ الْمُلُوكِيَّ حَسَبَ الْكِتَابِ «تُحِبُّ قَرِيبَكَ كَنَفْسِكَ». فَحَسَناً تَفْعَلُونَ.
\par 9 وَلَكِنْ إِنْ كُنْتُمْ تُحَابُونَ تَفْعَلُونَ خَطِيَّةً، مُوَبَّخِينَ مِنَ النَّامُوسِ كَمُتَعَدِّينَ.
\par 10 لأَنَّ مَنْ حَفِظَ كُلَّ النَّامُوسِ، وَإِنَّمَا عَثَرَ فِي وَاحِدَةٍ، فَقَدْ صَارَ مُجْرِماً فِي الْكُلِّ.
\par 11 لأَنَّ الَّذِي قَالَ: «لاَ تَزْنِ» قَالَ أَيْضاً: «لاَ تَقْتُلْ». فَإِنْ لَمْ تَزْنِ وَلَكِنْ قَتَلْتَ، فَقَدْ صِرْتَ مُتَعَدِّياً النَّامُوسَ.
\par 12 هَكَذَا تَكَلَّمُوا وَهَكَذَا افْعَلُوا كَعَتِيدِينَ أَنْ تُحَاكَمُوا بِنَامُوسِ الْحُرِّيَّةِ.
\par 13 لأَنَّ الْحُكْمَ هُوَ بِلاَ رَحْمَةٍ لِمَنْ لَمْ يَعْمَلْ رَحْمَةً، وَالرَّحْمَةُ تَفْتَخِرُ عَلَى الْحُكْمِ.
\par 14 مَا الْمَنْفَعَةُ يَا إِخْوَتِي إِنْ قَالَ أَحَدٌ إِنَّ لَهُ إِيمَاناً وَلَكِنْ لَيْسَ لَهُ أَعْمَالٌ؟ هَلْ يَقْدِرُ الإِيمَانُ أَنْ يُخَلِّصَهُ؟
\par 15 إِنْ كَانَ أَخٌ وَأُخْتٌ عُرْيَانَيْنِ وَمُعْتَازَيْنِ لِلْقُوتِ الْيَوْمِيِّ،
\par 16 فَقَالَ لَهُمَا أَحَدُكُمُ: «امْضِيَا بِسَلاَمٍ، اسْتَدْفِئَا وَاشْبَعَا» وَلَكِنْ لَمْ تُعْطُوهُمَا حَاجَاتِ الْجَسَدِ، فَمَا الْمَنْفَعَةُ؟
\par 17 هَكَذَا الإِيمَانُ أَيْضاً، إِنْ لَمْ يَكُنْ لَهُ أَعْمَالٌ، مَيِّتٌ فِي ذَاتِهِ.
\par 18 لَكِنْ يَقُولُ قَائِلٌ: «أَنْتَ لَكَ إِيمَانٌ، وَأَنَا لِي أَعْمَالٌ!» أَرِنِي إِيمَانَكَ بِدُونِ أَعْمَالِكَ، وَأَنَا أُرِيكَ بِأَعْمَالِي إِيمَانِي.
\par 19 أَنْتَ تُؤْمِنُ أَنَّ اللَّهَ وَاحِدٌ. حَسَناً تَفْعَلُ. وَالشَّيَاطِينُ يُؤْمِنُونَ وَيَقْشَعِرُّونَ!
\par 20 وَلَكِنْ هَلْ تُرِيدُ أَنْ تَعْلَمَ أَيُّهَا الإِنْسَانُ الْبَاطِلُ أَنَّ الإِيمَانَ بِدُونِ أَعْمَالٍ مَيِّتٌ؟
\par 21 أَلَمْ يَتَبَرَّرْ إِبْرَاهِيمُ أَبُونَا بِالأَعْمَالِ، إِذْ قَدَّمَ إِسْحَاقَ ابْنَهُ عَلَى الْمَذْبَحِ؟
\par 22 فَتَرَى أَنَّ الإِيمَانَ عَمِلَ مَعَ أَعْمَالِهِ، وَبِالأَعْمَالِ أُكْمِلَ الإِيمَانُ،
\par 23 وَتَمَّ الْكِتَابُ الْقَائِلُ: «فَآمَنَ إِبْرَاهِيمُ بِاللَّهِ فَحُسِبَ لَهُ بِرّاً» وَدُعِيَ خَلِيلَ اللَّهِ.
\par 24 تَرَوْنَ إِذاً أَنَّهُ بِالأَعْمَالِ يَتَبَرَّرُ الإِنْسَانُ، لاَ بِالإِيمَانِ وَحْدَهُ.
\par 25 كَذَلِكَ رَاحَابُ الّزَانِيَةُ أَيْضاً، أَمَا تَبَرَّرَتْ بِالأَعْمَالِ، إِذْ قَبِلَتِ الرُّسُلَ وَأَخْرَجَتْهُمْ فِي طَرِيقٍ آخَرَ؟
\par 26 لأَنَّهُ كَمَا أَنَّ الْجَسَدَ بِدُونَ رُوحٍ مَيِّتٌ، هَكَذَا الإِيمَانُ أَيْضاً بِدُونِ أَعْمَالٍ مَيِّتٌ.

\chapter{3}

\par 1 لاَ تَكُونُوا مُعَلِّمِينَ كَثِيرِينَ يَا إِخْوَتِي، عَالِمِينَ أَنَّنَا نَأْخُذُ دَيْنُونَةً أَعْظَمَ!
\par 2 لأَنَّنَا فِي أَشْيَاءَ كَثِيرَةٍ نَعْثُرُ جَمِيعُنَا. إِنْ كَانَ أَحَدٌ لاَ يَعْثُرُ فِي الْكَلاَمِ فَذَاكَ رَجُلٌ كَامِلٌ، قَادِرٌ أَنْ يُلْجِمَ كُلَّ الْجَسَدِ أَيْضاً.
\par 3 هُوَذَا الْخَيْلُ، نَضَعُ اللُّجُمَ فِي أَفْوَاهِهَا لِكَيْ تُطَاوِعَنَا، فَنُدِيرَ جِسْمَهَا كُلَّهُ.
\par 4 هُوَذَا السُّفُنُ أَيْضاً، وَهِيَ عَظِيمَةٌ بِهَذَا الْمِقْدَارِ، وَتَسُوقُهَا رِيَاحٌ عَاصِفَةٌ، تُدِيرُهَا دَفَّةٌ صَغِيرَةٌ جِدّاً إِلَى حَيْثُمَا شَاءَ قَصْدُ الْمُدِيرِ.
\par 5 هَكَذَا اللِّسَانُ أَيْضاً، هُوَ عُضْوٌ صَغِيرٌ وَيَفْتَخِرُ مُتَعَظِّماً. هُوَذَا نَارٌ قَلِيلَةٌ، أَيَّ وُقُودٍ تُحْرِقُ؟
\par 6 فَاللِّسَانُ نَارٌ! عَالَمُ الإِثْمِ. هَكَذَا جُعِلَ فِي أَعْضَائِنَا اللِّسَانُ، الَّذِي يُدَنِّسُ الْجِسْمَ كُلَّهُ، وَيُضْرِمُ دَائِرَةَ الْكَوْنِ، وَيُضْرَمُ مِنْ جَهَنَّمَ.
\par 7 لأَنَّ كُلَّ طَبْعٍ لِلْوُحُوشِ وَالطُّيُورِ وَالّزَحَّافَاتِ وَالْبَحْرِيَّاتِ يُذَلَّلُ، وَقَدْ تَذَلَّلَ لِلطَّبْعِ الْبَشَرِيِّ.
\par 8 وَأَمَّا اللِّسَانُ فَلاَ يَسْتَطِيعُ أَحَدٌ مِنَ النَّاسِ أَنْ يُذَلِّلَهُ. هُوَ شَرٌّ لاَ يُضْبَطُ، مَمْلُّوٌ سُمّاً مُمِيتاً.
\par 9 بِهِ نُبَارِكُ اللَّهَ الآبَ، وَبِهِ نَلْعَنُ النَّاسَ الَّذِينَ قَدْ تَكَوَّنُوا عَلَى شِبْهِ اللَّهِ.
\par 10 مِنَ الْفَمِ الْوَاحِدِ تَخْرُجُ بَرَكَةٌ وَلَعْنَةٌ! لاَ يَصْلُحُ يَا إِخْوَتِي أَنْ تَكُونَ هَذِهِ الْأُمُورُ هَكَذَا!
\par 11 أَلَعَلَّ يَنْبُوعاً يُنْبِعُ مِنْ نَفْسِ عَيْنٍ وَاحِدَةٍ الْعَذْبَ وَالْمُرَّ؟
\par 12 هَلْ تَقْدِرُ يَا إِخْوَتِي تِينَةٌ أَنْ تَصْنَعَ زَيْتُوناً، أَوْ كَرْمَةٌ تِيناً؟ وَلاَ كَذَلِكَ يَنْبُوعٌ يَصْنَعُ مَاءً مَالِحاً وَعَذْباً!
\par 13 مَنْ هُوَ حَكِيمٌ وَعَالِمٌ بَيْنَكُمْ فَلْيُرِ أَعْمَالَهُ بِالتَّصَرُّفِ الْحَسَنِ فِي وَدَاعَةِ الْحِكْمَةِ.
\par 14 وَلَكِنْ إِنْ كَانَ لَكُمْ غَيْرَةٌ مُرَّةٌ وَتَحَّزُبٌ فِي قُلُوبِكُمْ، فَلاَ تَفْتَخِرُوا وَتَكْذِبُوا عَلَى الْحَقِّ.
\par 15 لَيْسَتْ هَذِهِ الْحِكْمَةُ نَازِلَةً مِنْ فَوْقُ، بَلْ هِيَ أَرْضِيَّةٌ نَفْسَانِيَّةٌ شَيْطَانِيَّةٌ.
\par 16 لأَنَّهُ حَيْثُ الْغَيْرَةُ وَالتَّحَّزُبُ هُنَاكَ التَّشْوِيشُ وَكُلُّ أَمْرٍ رَدِيءٍ.
\par 17 وَأَمَّا الْحِكْمَةُ الَّتِي مِنْ فَوْقُ فَهِيَ أَوَّلاً طَاهِرَةٌ، ثُمَّ مُسَالِمَةٌ، مُتَرَفِّقَةٌ، مُذْعِنَةٌ، مَمْلُوَّةٌ رَحْمَةً وَأَثْمَاراً صَالِحَةً، عَدِيمَةُ الرَّيْبِ وَالرِّيَاءِ.
\par 18 وَثَمَرُ الْبِرِّ يُزْرَعُ فِي السَّلاَمِ مِنَ الَّذِينَ يَفْعَلُونَ السَّلاَمَ.

\chapter{4}

\par 1 رِبَةِ فِي أَعْضَائِكُمْ؟
\par 2 تَشْتَهُونَ وَلَسْتُمْ تَمْتَلِكُونَ. تَقْتُلُونَ وَتَحْسِدُونَ وَلَسْتُمْ تَقْدِرُونَ أَنْ تَنَالُوا. تُخَاصِمُونَ وَتُحَارِبُونَ وَلَسْتُمْ تَمْتَلِكُونَ، لأَنَّكُمْ لاَ تَطْلُبُونَ.
\par 3 تَطْلُبُونَ وَلَسْتُمْ تَأْخُذُونَ، لأَنَّكُمْ تَطْلُبُونَ رَدِيّاً لِكَيْ تُنْفِقُوا فِي لَذَّاتِكُمْ.
\par 4 أَيُّهَا الّزُنَاةُ وَالّزَوَانِي، أَمَا تَعْلَمُونَ أَنَّ مَحَبَّةَ الْعَالَمِ عَدَاوَةٌ لِلَّهِ؟ فَمَنْ أَرَادَ أَنْ يَكُونَ مُحِبّاً لِلْعَالَمِ فَقَدْ صَارَ عَدُّواً لِلَّهِ.
\par 5 أَمْ تَظُنُّونَ أَنَّ الْكِتَابَ يَقُولُ بَاطِلاً: الرُّوحُ الَّذِي حَلَّ فِينَا يَشْتَاقُ إِلَى الْحَسَدِ؟
\par 6 وَلَكِنَّهُ يُعْطِي نِعْمَةً أَعْظَمَ. لِذَلِكَ يَقُولُ: «يُقَاوِمُ اللَّهُ الْمُسْتَكْبِرِينَ، وَأَمَّا الْمُتَوَاضِعُونَ فَيُعْطِيهِمْ نِعْمَةً».
\par 7 فَاخْضَعُوا لِلَّهِ. قَاوِمُوا إِبْلِيسَ فَيَهْرُبَ مِنْكُمْ.
\par 8 اِقْتَرِبُوا إِلَى اللَّهِ فَيَقْتَرِبَ إِلَيْكُمْ. نَقُّوا أَيْدِيَكُمْ أَيُّهَا الْخُطَاةُ، وَطَهِّرُوا قُلُوبَكُمْ يَا ذَوِي الرَّأْيَيْنِ.
\par 9 اكْتَئِبُوا وَنُوحُوا وَابْكُوا. لِيَتَحَوَّلْ ضِحْكُكُمْ إِلَى نَوْحٍ وَفَرَحُكُمْ إِلَى غَمٍّ.
\par 10 اِتَّضِعُوا قُدَّامَ الرَّبِّ فَيَرْفَعَكُمْ.
\par 11 لاَ يَذُمَّ بَعْضُكُمْ بَعْضاً أَيُّهَا الإِخْوَةُ. الَّذِي يَذُمُّ أَخَاهُ وَيَدِينُ أَخَاهُ يَذُمُّ النَّامُوسَ وَيَدِينُ النَّامُوسَ. وَإِنْ كُنْتَ تَدِينُ النَّامُوسَ فَلَسْتَ عَامِلاً بِالنَّامُوسِ، بَلْ دَيَّاناً لَهُ.
\par 12 وَاحِدٌ هُوَ وَاضِعُ النَّامُوسِ، الْقَادِرُ أَنْ يُخَلِّصَ وَيُهْلِكَ. فَمَنْ أَنْتَ يَا مَنْ تَدِينُ غَيْرَكَ؟
\par 13 هَلُمَّ الآنَ أَيُّهَا الْقَائِلُونَ: «نَذْهَبُ الْيَوْمَ أَوْ غَداً إِلَى هَذِهِ الْمَدِينَةِ أَوْ تِلْكَ، وَهُنَاكَ نَصْرِفُ سَنَةً وَاحِدَةً وَنَتَّجِرُ وَنَرْبَحُ».
\par 14 أَنْتُمُ الَّذِينَ لاَ تَعْرِفُونَ أَمْرَ الْغَدِ! لأَنَّهُ مَا هِيَ حَيَاتُكُمْ؟ إِنَّهَا بُخَارٌ، يَظْهَرُ قَلِيلاً ثُمَّ يَضْمَحِلُّ.
\par 15 عِوَضَ أَنْ تَقُولُوا: «إِنْ شَاءَ الرَّبُّ وَعِشْنَا نَفْعَلُ هَذَا أَوْ ذَاكَ».
\par 16 وَأَمَّا الآنَ فَإِنَّكُمْ تَفْتَخِرُونَ فِي تَعَظُّمِكُمْ. كُلُّ افْتِخَارٍ مِثْلُ هَذَا رَدِيءٌ.
\par 17 فَمَنْ يَعْرِفُ أَنْ يَعْمَلَ حَسَناً وَلاَ يَعْمَلُ، فَذَلِكَ خَطِيَّةٌ لَهُ.

\chapter{5}

\par 1 هَلُمَّ الآنَ أَيُّهَا الأَغْنِيَاءُ، ابْكُوا مُوَلْوِلِينَ عَلَى شَقَاوَتِكُمُ الْقَادِمَةِ.
\par 2 غِنَاكُمْ قَدْ تَهَرَّأَ، وَثِيَابُكُمْ قَدْ أَكَلَهَا الْعُثُّ.
\par 3 ذَهَبُكُمْ وَفِضَّتُكُمْ قَدْ صَدِئَا، وَصَدَأُهُمَا يَكُونُ شَهَادَةً عَلَيْكُمْ، وَيَأْكُلُ لُحُومَكُمْ كَنَارٍ! قَدْ كَنَزْتُمْ فِي الأَيَّامِ الأَخِيرَةِ.
\par 4 هُوَذَا أُجْرَةُ الْفَعَلَةِ الَّذِينَ حَصَدُوا حُقُولَكُمُ الْمَبْخُوسَةُ مِنْكُمْ تَصْرُخُ، وَصِيَاحُ الْحَصَّادِينَ قَدْ دَخَلَ إِلَى أُذْنَيْ رَبِّ الْجُنُودِ.
\par 5 قَدْ تَرَفَّهْتُمْ عَلَى الأَرْضِ وَتَنَعَّمْتُمْ وَرَبَّيْتُمْ قُلُوبَكُمْ، كَمَا فِي يَوْمِ الذَّبْحِ.
\par 6 حَكَمْتُمْ عَلَى الْبَارِّ. قَتَلْتُمُوهُ. لاَ يُقَاوِمُكُمْ!
\par 7 فَتَأَنَّوْا أَيُّهَا الإِخْوَةُ إِلَى مَجِيءِ الرَّبِّ. هُوَذَا الْفَلاَّحُ يَنْتَظِرُ ثَمَرَ الأَرْضِ الثَّمِينَ مُتَأَنِّياً عَلَيْهِ حَتَّى يَنَالَ الْمَطَرَ الْمُبَكِّرَ وَالْمُتَأَخِّرَ.
\par 8 فَتَأَنَّوْا أَنْتُمْ وَثَبِّتُوا قُلُوبَكُمْ، لأَنَّ مَجِيءَ الرَّبِّ قَدِ اقْتَرَبَ.
\par 9 لاَ يَئِنَّ بَعْضُكُمْ عَلَى بَعْضٍ أَيُّهَا الإِخْوَةُ لِئَلاَّ تُدَانُوا. هُوَذَا الدَّيَّانُ وَاقِفٌ قُدَّامَ الْبَابِ.
\par 10 خُذُوا يَا إِخْوَتِي مِثَالاً لاِحْتِمَالِ الْمَشَقَّاتِ وَالأَنَاةِ: الأَنْبِيَاءَ الَّذِينَ تَكَلَّمُوا بِاسْمِ الرَّبِّ.
\par 11 هَا نَحْنُ نُطَّوِبُ الصَّابِرِينَ. قَدْ سَمِعْتُمْ بِصَبْرِ أَيُّوبَ وَرَأَيْتُمْ عَاقِبَةَ الرَّبِّ. لأَنَّ الرَّبَّ كَثِيرُ الرَّحْمَةِ وَرَؤُوفٌ.
\par 12 وَلَكِنْ قَبْلَ كُلِّ شَيْءٍ يَا إِخْوَتِي لاَ تَحْلِفُوا لاَ بِالسَّمَاءِ وَلاَ بِالأَرْضِ وَلاَ بِقَسَمٍ آخَرَ. بَلْ لِتَكُنْ نَعَمْكُمْ نَعَمْ وَلاَكُمْ لاَ، لِئَلاَّ تَقَعُوا تَحْتَ دَيْنُونَةٍ.
\par 13 أَعَلَى أَحَدٍ بَيْنَكُمْ مَشَقَّاتٌ؟ فَلْيُصَلِّ. أَمَسْرُورٌ أَحَدٌ؟ فَلْيُرَتِّلْ.
\par 14 أَمَرِيضٌ أَحَدٌ بَيْنَكُمْ؟ فَلْيَدْعُ شُيُوخَ الْكَنِيسَةِ فَيُصَلُّوا عَلَيْهِ وَيَدْهَنُوهُ بِزَيْتٍ بِاسْمِ الرَّبِّ،
\par 15 وَصَلاَةُ الإِيمَانِ تَشْفِي الْمَرِيضَ وَالرَّبُّ يُقِيمُهُ، وَإِنْ كَانَ قَدْ فَعَلَ خَطِيَّةً تُغْفَرُ لَهْ.
\par 16 اِعْتَرِفُوا بَعْضُكُمْ لِبَعْضٍ بِالّزَلاَّتِ، وَصَلُّوا بَعْضُكُمْ لأَجْلِ بَعْضٍ لِكَيْ تُشْفَوْا. طِلْبَةُ الْبَارِّ تَقْتَدِرُ كَثِيراً فِي فِعْلِهَا.
\par 17 كَانَ إِيلِيَّا إِنْسَاناً تَحْتَ الآلاَمِ مِثْلَنَا، وَصَلَّى صَلاَةً أَنْ لاَ تُمْطِرَ، فَلَمْ تُمْطِرْ عَلَى الأَرْضِ ثَلاَثَ سِنِينَ وَسِتَّةَ أَشْهُرٍ.
\par 18 ثُمَّ صَلَّى أَيْضاً فَأَعْطَتِ السَّمَاءُ مَطَراً وَأَخْرَجَتِ الأَرْضُ ثَمَرَهَا.
\par 19 أَيُّهَا الإِخْوَةُ، إِنْ ضَلَّ أَحَدٌ بَيْنَكُمْ عَنِ الْحَقِّ فَرَدَّهُ أَحَدٌ،
\par 20 فَلْيَعْلَمْ أَنَّ مَنْ رَدَّ خَاطِئاً عَنْ ضَلاَلِ طَرِيقِهِ يُخَلِّصُ نَفْساً مِنَ الْمَوْتِ، وَيَسْتُرُ كَثْرَةً مِنَ الْخَطَايَا.


\end{document}