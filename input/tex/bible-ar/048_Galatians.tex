\begin{document}

\title{غلاطية}


\chapter{1}

\par 1 بُولُسُ، رَسُولٌ لاَ مِنَ النَّاسِ وَلاَ بِإِنْسَانٍ، بَلْ بِيَسُوعَ الْمَسِيحِ وَاللهِ الآبِ الَّذِي أَقَامَهُ مِنَ الأَمْوَاتِ،
\par 2 وَجَمِيعُ الإِخْوَةِ الَّذِينَ مَعِي، إِلَى كَنَائِسِ غَلاَطِيَّةَ.
\par 3 نِعْمَةٌ لَكُمْ وَسَلاَمٌ مِنَ اللهِ الآبِ، وَمِنْ رَبِّنَا يَسُوعَ الْمَسِيحِ،
\par 4 الَّذِي بَذَلَ نَفْسَهُ لأَجْلِ خَطَايَانَا، لِيُنْقِذَنَا مِنَ الْعَالَمِ الْحَاضِرِ الشِّرِّيرِ حَسَبَ إِرَادَةِ اللهِ وَأَبِينَا،
\par 5 الَّذِي لَهُ الْمَجْدُ إِلَى أَبَدِ الآبِدِينَ. آمِينَ.
\par 6 إِنِّي أَتَعَجَّبُ أَنَّكُمْ تَنْتَقِلُونَ هَكَذَا سَرِيعاً عَنِ الَّذِي دَعَاكُمْ بِنِعْمَةِ الْمَسِيحِ إِلَى إِنْجِيلٍ آخَرَ.
\par 7 لَيْسَ هُوَ آخَرَ، غَيْرَ أَنَّهُ يُوجَدُ قَوْمٌ يُزْعِجُونَكُمْ وَيُرِيدُونَ أَنْ يُحَوِّلُوا إِنْجِيلَ الْمَسِيحِ.
\par 8 وَلَكِنْ إِنْ بَشَّرْنَاكُمْ نَحْنُ أَوْ مَلاَكٌ مِنَ السَّمَاءِ بِغَيْرِ مَا بَشَّرْنَاكُمْ، فَلْيَكُنْ «أَنَاثِيمَا».
\par 9 كَمَا سَبَقْنَا فَقُلْنَا أَقُولُ الآنَ أَيْضاً: إِنْ كَانَ أَحَدٌ يُبَشِّرُكُمْ بِغَيْرِ مَا قَبِلْتُمْ، فَلْيَكُنْ «أَنَاثِيمَا».
\par 10 أَفَأَسْتَعْطِفُ الآنَ النَّاسَ أَمِ اللهَ؟ أَمْ أَطْلُبُ أَنْ أُرْضِيَ النَّاسَ؟ فَلَوْ كُنْتُ بَعْدُ أُرْضِي النَّاسَ لَمْ أَكُنْ عَبْداً لِلْمَسِيحِ.
\par 11 وَأُعَرِّفُكُمْ أَيُّهَا الإِخْوَةُ الإِنْجِيلَ الَّذِي بَشَّرْتُ بِهِ، أَنَّهُ لَيْسَ بِحَسَبِ إِنْسَانٍ.
\par 12 لأَنِّي لَمْ أَقْبَلْهُ مِنْ عِنْدِ إِنْسَانٍ وَلاَ عُلِّمْتُهُ. بَلْ بِإِعْلاَنِ يَسُوعَ الْمَسِيحِ.
\par 13 فَإِنَّكُمْ سَمِعْتُمْ بِسِيرَتِي قَبْلاً فِي الدِّيَانَةِ الْيَهُودِيَّةِ، أَنِّي كُنْتُ أَضْطَهِدُ كَنِيسَةَ اللهِ بِإِفْرَاطٍ وَأُتْلِفُهَا.
\par 14 وَكُنْتُ أَتَقَدَّمُ فِي الدِّيَانَةِ الْيَهُودِيَّةِ عَلَى كَثِيرِينَ مِنْ أَتْرَابِي فِي جِنْسِي، إِذْ كُنْتُ أَوْفَرَ غَيْرَةً فِي تَقْلِيدَاتِ آبَائِي.
\par 15 وَلَكِنْ لَمَّا سَرَّ اللهَ الَّذِي أَفْرَزَنِي مِنْ بَطْنِ أُمِّي، وَدَعَانِي بِنِعْمَتِهِ
\par 16 أَنْ يُعْلِنَ ابْنَهُ فِيَّ لِأُبَشِّرَ بِهِ بَيْنَ الأُمَمِ، لِلْوَقْتِ لَمْ أَسْتَشِرْ لَحْماً وَدَماً
\par 17 وَلاَ صَعِدْتُ إِلَى أُورُشَلِيمَ إِلَى الرُّسُلِ الَّذِينَ قَبْلِي، بَلِ انْطَلَقْتُ إِلَى الْعَرَبِيَّةِ، ثُمَّ رَجَعْتُ أَيْضاً إِلَى دِمَشْقَ.
\par 18 ثُمَّ بَعْدَ ثَلاَثِ سِنِينَ صَعِدْتُ إِلَى أُورُشَلِيمَ لأَتَعَرَّفَ بِبُطْرُسَ، فَمَكَثْتُ عِنْدَهُ خَمْسَةَ عَشَرَ يَوْماً.
\par 19 وَلَكِنَّنِي لَمْ أَرَ غَيْرَهُ مِنَ الرُّسُلِ إِلاَّ يَعْقُوبَ أَخَا الرَّبِّ.
\par 20 وَالَّذِي أَكْتُبُ بِهِ إِلَيْكُمْ هُوَذَا قُدَّامَ اللهِ أَنِّي لَسْتُ أَكْذِبُ فِيهِ.
\par 21 وَبَعْدَ ذَلِكَ جِئْتُ إِلَى أَقَالِيمِ سُورِيَّةَ وَكِيلِيكِيَّةَ.
\par 22 وَلَكِنَّنِي كُنْتُ غَيْرَ مَعْرُوفٍ بِالْوَجْهِ عِنْدَ كَنَائِسِ الْيَهُودِيَّةِ الَّتِي فِي الْمَسِيحِ.
\par 23 غَيْرَ أَنَّهُمْ كَانُوا يَسْمَعُونَ أَنَّ الَّذِي كَانَ يَضْطَهِدُنَا قَبْلاً، يُبَشِّرُ الآنَ بِالإِيمَانِ الَّذِي كَانَ قَبْلاً يُتْلِفُهُ.
\par 24 فَكَانُوا يُمَجِّدُونَ اللهَ فِيَّ.

\chapter{2}

\par 1 ثُمَّ بَعْدَ أَرْبَعَ عَشْرَةَ سَنَةً صَعِدْتُ أَيْضاً إِلَى أُورُشَلِيمَ مَعَ بَرْنَابَا، آخِذاً مَعِي تِيطُسَ أَيْضاً.
\par 2 وَإِنَّمَا صَعِدْتُ بِمُوجَبِ إِعْلاَنٍ، وَعَرَضْتُ عَلَيْهِمِ الإِنْجِيلَ الَّذِي أَكْرِزُ بِهِ بَيْنَ الأُمَمِ، وَلَكِنْ بِالاِنْفِرَادِ عَلَى الْمُعْتَبَرِينَ، لِئَلاَّ أَكُونَ أَسْعَى أَوْ قَدْ سَعَيْتُ بَاطِلاً.
\par 3 لَكِنْ لَمْ يَضْطَرَّ وَلاَ تِيطُسُ الَّذِي كَانَ مَعِي، وَهُوَ يُونَانِيٌّ، أَنْ يَخْتَتِنَ.
\par 4 وَلَكِنْ بِسَبَبِ الإِخْوَةِ الْكَذَبَةِ الْمُدْخَلِينَ خُفْيَةً، الَّذِينَ دَخَلُوا اخْتِلاَساً لِيَتَجَسَّسُوا حُرِّيَّتَنَا الَّتِي لَنَا فِي الْمَسِيحِ كَيْ يَسْتَعْبِدُونَا -
\par 5 اَلَّذِينَ لَمْ نُذْعِنْ لَهُمْ بِالْخُضُوعِ وَلاَ سَاعَةً، لِيَبْقَى عِنْدَكُمْ حَقُّ الإِنْجِيلِ.
\par 6 وَأَمَّا الْمُعْتَبَرُونَ أَنَّهُمْ شَيْءٌ، مَهْمَا كَانُوا، لاَ فَرْقَ عِنْدِي: اللهُ لاَ يَأْخُذُ بِوَجْهِ إِنْسَانٍ - فَإِنَّ هَؤُلاَءِ الْمُعْتَبَرِينَ لَمْ يُشِيرُوا عَلَيَّ بِشَيْءٍ.
\par 7 بَلْ بِالْعَكْسِ، إِذْ رَأَوْا أَنِّي اؤْتُمِنْتُ عَلَى إِنْجِيلِ الْغُرْلَةِ كَمَا بُطْرُسُ عَلَى إِنْجِيلِ الْخِتَانِ.
\par 8 فَإِنَّ الَّذِي عَمِلَ فِي بُطْرُسَ لِرِسَالَةِ الْخِتَانِ عَمِلَ فِيَّ أَيْضاً لِلأُمَمِ.
\par 9 فَإِذْ عَلِمَ بِالنِّعْمَةِ الْمُعْطَاةِ لِي يَعْقُوبُ وَصَفَا وَيُوحَنَّا، الْمُعْتَبَرُونَ أَنَّهُمْ أَعْمِدَةٌ، أَعْطَوْنِي وَبَرْنَابَا يَمِينَ الشَّرِكَةِ لِنَكُونَ نَحْنُ لِلأُمَمِ وَأَمَّا هُمْ فَلِلْخِتَانِ.
\par 10 غَيْرَ أَنْ نَذْكُرَ الْفُقَرَاءَ. وَهَذَا عَيْنُهُ كُنْتُ اعْتَنَيْتُ أَنْ أَفْعَلَهُ.
\par 11 وَلَكِنْ لَمَّا أَتَى بُطْرُسُ إِلَى أَنْطَاكِيَةَ قَاوَمْتُهُ مُواجَهَةً، لأَنَّهُ كَانَ مَلُوماً.
\par 12 لأَنَّهُ قَبْلَمَا أَتَى قَوْمٌ مِنْ عِنْدِ يَعْقُوبَ كَانَ يَأْكُلُ مَعَ الأُمَمِ، وَلَكِنْ لَمَّا أَتَوْا كَانَ يُؤَخِّرُ وَيُفْرِزُ نَفْسَهُ، خَائِفاً مِنَ الَّذِينَ هُمْ مِنَ الْخِتَانِ.
\par 13 وَرَاءَى مَعَهُ بَاقِي الْيَهُودِ أَيْضاً، حَتَّى إِنَّ بَرْنَابَا أَيْضاً انْقَادَ إِلَى رِيَائِهِمْ!
\par 14 لَكِنْ لَمَّا رَأَيْتُ أَنَّهُمْ لاَ يَسْلُكُونَ بِاسْتِقَامَةٍ حَسَبَ حَقِّ الإِنْجِيلِ، قُلْتُ لِبُطْرُسَ قُدَّامَ الْجَمِيعِ: «إِنْ كُنْتَ وَأَنْتَ يَهُودِيٌّ تَعِيشُ أُمَمِيّاً لاَ يَهُودِيّاً، فَلِمَاذَا تُلْزِمُ الأُمَمَ أَنْ يَتَهَوَّدُوا؟»
\par 15 نَحْنُ بِالطَّبِيعَةِ يَهُودٌ وَلَسْنَا مِنَ الأُمَمِ خُطَاةً،
\par 16 إِذْ نَعْلَمُ أَنَّ الإِنْسَانَ لاَ يَتَبَرَّرُ بِأَعْمَالِ النَّامُوسِ، بَلْ بِإِيمَانِ يَسُوعَ الْمَسِيحِ، آمَنَّا نَحْنُ أَيْضاً بِيَسُوعَ الْمَسِيحِ، لِنَتَبَرَّرَ بِإِيمَانِ يَسُوعَ لاَ بِأَعْمَالِ النَّامُوسِ. لأَنَّهُ بِأَعْمَالِ النَّامُوسِ لاَ يَتَبَرَّرُ جَسَدٌ مَا.
\par 17 فَإِنْ كُنَّا وَنَحْنُ طَالِبُونَ أَنْ نَتَبَرَّرَ فِي الْمَسِيحِ نُوجَدُ نَحْنُ أَنْفُسُنَا أَيْضاً خُطَاةً، أَفَالْمَسِيحُ خَادِمٌ لِلْخَطِيَّةِ؟ حَاشَا!
\par 18 فَإِنِّي إِنْ كُنْتُ أَبْنِي أَيْضاً هَذَا الَّذِي قَدْ هَدَمْتُهُ، فَإِنِّي أُظْهِرُ نَفْسِي مُتَعَدِّياً.
\par 19 لأَنِّي مُتُّ بِالنَّامُوسِ لِلنَّامُوسِ لأَحْيَا لِلَّهِ.
\par 20 مَعَ الْمَسِيحِ صُلِبْتُ، فَأَحْيَا لاَ أَنَا بَلِ الْمَسِيحُ يَحْيَا فِيَّ. فَمَا أَحْيَاهُ الآنَ فِي الْجَسَدِ فَإِنَّمَا أَحْيَاهُ فِي الإِيمَانِ، إِيمَانِ ابْنِ اللهِ، الَّذِي أَحَبَّنِي وَأَسْلَمَ نَفْسَهُ لأَجْلِي.
\par 21 لَسْتُ أُبْطِلُ نِعْمَةَ اللهِ. لأَنَّهُ إِنْ كَانَ بِالنَّامُوسِ بِرٌّ، فَالْمَسِيحُ إِذاً مَاتَ بِلاَ سَبَبٍ.

\chapter{3}

\par 1 أَيُّهَا الْغَلاَطِيُّونَ الأَغْبِيَاءُ، مَنْ رَقَاكُمْ حَتَّى لاَ تُذْعِنُوا لِلْحَقِّ؟ أَنْتُمُ الَّذِينَ أَمَامَ عُيُونِكُمْ قَدْ رُسِمَ يَسُوعُ الْمَسِيحُ بَيْنَكُمْ مَصْلُوباً!
\par 2 أُرِيدُ أَنْ أَتَعَلَّمَ مِنْكُمْ هَذَا فَقَطْ: أَبِأَعْمَالِ النَّامُوسِ أَخَذْتُمُ الرُّوحَ أَمْ بِخَبَرِ الإِيمَانِ؟
\par 3 أَهَكَذَا أَنْتُمْ أَغْبِيَاءُ! أَبَعْدَمَا ابْتَدَأْتُمْ بِالرُّوحِ تُكَمَّلُونَ الآنَ بِالْجَسَدِ؟
\par 4 أَهَذَا الْمِقْدَارَ احْتَمَلْتُمْ عَبَثاً؟ إِنْ كَانَ عَبَثاً!
\par 5 فَالَّذِي يَمْنَحُكُمُ الرُّوحَ، وَيَعْمَلُ قُوَّاتٍ فِيكُمْ، أَبِأَعْمَالِ النَّامُوسِ أَمْ بِخَبَرِ الإِيمَانِ؟
\par 6 كَمَا «آمَنَ إِبْرَاهِيمُ بِاللهِ فَحُسِبَ لَهُ بِرّاً».
\par 7 اعْلَمُوا إِذاً أَنَّ الَّذِينَ هُمْ مِنَ الإِيمَانِ أُولَئِكَ هُمْ بَنُو إِبْرَاهِيمَ.
\par 8 وَالْكِتَابُ إِذْ سَبَقَ فَرَأَى أَنَّ اللهَ بِالإِيمَانِ يُبَرِّرُ الأُمَمَ، سَبَقَ فَبَشَّرَ إِبْرَاهِيمَ أَنْ «فِيكَ تَتَبَارَكُ جَمِيعُ الأُمَمِ».
\par 9 إِذاً الَّذِينَ هُمْ مِنَ الإِيمَانِ يَتَبَارَكُونَ مَعَ إِبْرَاهِيمَ الْمُؤْمِنِ.
\par 10 لأَنَّ جَمِيعَ الَّذِينَ هُمْ مِنْ أَعْمَالِ النَّامُوسِ هُمْ تَحْتَ لَعْنَةٍ، لأَنَّهُ مَكْتُوبٌ «مَلْعُونٌ كُلُّ مَنْ لاَ يَثْبُتُ فِي جَمِيعِ مَا هُوَ مَكْتُوبٌ فِي كِتَابِ النَّامُوسِ لِيَعْمَلَ بِهِ».
\par 11 وَلَكِنْ أَنْ لَيْسَ أَحَدٌ يَتَبَرَّرُ بِالنَّامُوسِ عِنْدَ اللهِ فَظَاهِرٌ، لأَنَّ «الْبَارَّ بِالإِيمَانِ يَحْيَا».
\par 12 وَلَكِنَّ النَّامُوسَ لَيْسَ مِنَ الإِيمَانِ، بَلِ «الإِنْسَانُ الَّذِي يَفْعَلُهَا سَيَحْيَا بِهَا».
\par 13 اَلْمَسِيحُ افْتَدَانَا مِنْ لَعْنَةِ النَّامُوسِ، إِذْ صَارَ لَعْنَةً لأَجْلِنَا، لأَنَّهُ مَكْتُوبٌ: «مَلْعُونٌ كُلُّ مَنْ عُلِّقَ عَلَى خَشَبَةٍ».
\par 14 لِتَصِيرَ بَرَكَةُ إِبْرَاهِيمَ لِلأُمَمِ فِي الْمَسِيحِ يَسُوعَ، لِنَنَالَ بِالإِيمَانِ مَوْعِدَ الرُّوحِ،
\par 15 أَيُّهَا الإِخْوَةُ بِحَسَبِ الإِنْسَانِ أَقُولُ «لَيْسَ أَحَدٌ يُبْطِلُ عَهْداً قَدْ تَمَكَّنَ وَلَوْ مِنْ إِنْسَانٍ، أَوْ يَزِيدُ عَلَيْهِ».
\par 16 وَأَمَّا الْمَوَاعِيدُ فَقِيلَتْ فِي «إِبْرَاهِيمَ وَفِي نَسْلِهِ». لاَ يَقُولُ «وَفِي الأَنْسَالِ» كَأَنَّهُ عَنْ كَثِيرِينَ، بَلْ كَأَنَّهُ عَنْ وَاحِدٍ. وَ«فِي نَسْلِكَ» الَّذِي هُوَ الْمَسِيحُ.
\par 17 وَإِنَّمَا أَقُولُ هَذَا: إِنَّ النَّامُوسَ الَّذِي صَارَ بَعْدَ أَرْبَعِمِئَةٍ وَثَلاَثِينَ سَنَةً، لاَ يَنْسَخُ عَهْداً قَدْ سَبَقَ فَتَمَكَّنَ مِنَ اللهِ نَحْوَ الْمَسِيحِ حَتَّى يُبَطِّلَ الْمَوْعِدَ.
\par 18 لأَنَّهُ إِنْ كَانَتِ الْوِرَاثَةُ مِنَ النَّامُوسِ فَلَمْ تَكُنْ أَيْضاً مِنْ مَوْعِدٍ. وَلَكِنَّ اللهَ وَهَبَهَا لِإِبْرَاهِيمَ بِمَوْعِدٍ.
\par 19 فَلِمَاذَا النَّامُوسُ؟ قَدْ زِيدَ بِسَبَبِ التَّعَدِّيَاتِ، إِلَى أَنْ يَأْتِيَ النَّسْلُ الَّذِي قَدْ وُعِدَ لَهُ، مُرَتَّباً بِمَلاَئِكَةٍ فِي يَدِ وَسِيطٍ.
\par 20 وَأَمَّا الْوَسِيطُ فَلاَ يَكُونُ لِوَاحِدٍ. وَلَكِنَّ اللهَ وَاحِدٌ.
\par 21 فَهَلِ النَّامُوسُ ضِدَّ مَوَاعِيدِ اللهِ؟ حَاشَا! لأَنَّهُ لَوْ أُعْطِيَ نَامُوسٌ قَادِرٌ أَنْ يُحْيِيَ، لَكَانَ بِالْحَقِيقَةِ الْبِرُّ بِالنَّامُوسِ.
\par 22 لَكِنَّ الْكِتَابَ أَغْلَقَ عَلَى الْكُلِّ تَحْتَ الْخَطِيَّةِ، لِيُعْطَى الْمَوْعِدُ مِنْ إِيمَانِ يَسُوعَ الْمَسِيحِ لِلَّذِينَ يُؤْمِنُونَ.
\par 23 وَلَكِنْ قَبْلَمَا جَاءَ الإِيمَانُ كُنَّا مَحْرُوسِينَ تَحْتَ النَّامُوسِ، مُغْلَقاً عَلَيْنَا إِلَى الإِيمَانِ الْعَتِيدِ أَنْ يُعْلَنَ.
\par 24 إِذاً قَدْ كَانَ النَّامُوسُ مُؤَدِّبَنَا إِلَى الْمَسِيحِ، لِكَيْ نَتَبَرَّرَ بِالإِيمَانِ.
\par 25 وَلَكِنْ بَعْدَ مَا جَاءَ الإِيمَانُ لَسْنَا بَعْدُ تَحْتَ مُؤَدِّبٍ.
\par 26 لأَنَّكُمْ جَمِيعاً أَبْنَاءُ اللهِ بِالإِيمَانِ بِالْمَسِيحِ يَسُوعَ.
\par 27 لأَنَّ كُلَّكُمُ الَّذِينَ اعْتَمَدْتُمْ بِالْمَسِيحِ قَدْ لَبِسْتُمُ الْمَسِيحَ.
\par 28 لَيْسَ يَهُودِيٌّ وَلاَ يُونَانِيٌّ. لَيْسَ عَبْدٌ وَلاَ حُرٌّ. لَيْسَ ذَكَرٌ وَأُنْثَى، لأَنَّكُمْ جَمِيعاً وَاحِدٌ فِي الْمَسِيحِ يَسُوعَ.
\par 29 فَإِنْ كُنْتُمْ لِلْمَسِيحِ فَأَنْتُمْ إِذاً نَسْلُ إِبْرَاهِيمَ، وَحَسَبَ الْمَوْعِدِ وَرَثَةٌ.

\chapter{4}

\par 1 وَإِنَّمَا أَقُولُ: مَا دَامَ الْوَارِثُ قَاصِراً لاَ يَفْرِقُ شَيْئاً عَنِ الْعَبْدِ، مَعَ كَوْنِهِ صَاحِبَ الْجَمِيعِ.
\par 2 بَلْ هُوَ تَحْتَ أَوْصِيَاءَ وَوُكَلاَءَ إِلَى الْوَقْتِ الْمُؤَجَّلِ مِنْ أَبِيهِ.
\par 3 هَكَذَا نَحْنُ أَيْضاً: لَمَّا كُنَّا قَاصِرِينَ كُنَّا مُسْتَعْبَدِينَ تَحْتَ أَرْكَانِ الْعَالَمِ.
\par 4 وَلَكِنْ لَمَّا جَاءَ مِلْءُ الزَّمَانِ، أَرْسَلَ اللهُ ابْنَهُ مَوْلُوداً مِنِ امْرَأَةٍ، مَوْلُوداً تَحْتَ النَّامُوسِ،
\par 5 لِيَفْتَدِيَ الَّذِينَ تَحْتَ النَّامُوسِ، لِنَنَالَ التَّبَنِّيَ.
\par 6 ثُمَّ بِمَا أَنَّكُمْ أَبْنَاءٌ، أَرْسَلَ اللهُ رُوحَ ابْنِهِ إِلَى قُلُوبِكُمْ صَارِخاً: «يَا أَبَا الآبُ».
\par 7 إِذاً لَسْتَ بَعْدُ عَبْداً بَلِ ابْناً، وَإِنْ كُنْتَ ابْناً فَوَارِثٌ لِلَّهِ بِالْمَسِيحِ.
\par 8 لَكِنْ حِينَئِذٍ إِذْ كُنْتُمْ لاَ تَعْرِفُونَ اللهَ اسْتُعْبِدْتُمْ لِلَّذِينَ لَيْسُوا بِالطَّبِيعَةِ آلِهَةً.
\par 9 وَأَمَّا الآنَ إِذْ عَرَفْتُمُ اللهَ، بَلْ بِالْحَرِيِّ عُرِفْتُمْ مِنَ اللهِ، فَكَيْفَ تَرْجِعُونَ أَيْضاً إِلَى الأَرْكَانِ الضَّعِيفَةِ الْفَقِيرَةِ الَّتِي تُرِيدُونَ أَنْ تُسْتَعْبَدُوا لَهَا مِنْ جَدِيدٍ؟
\par 10 أَتَحْفَظُونَ أَيَّاماً وَشُهُوراً وَأَوْقَاتاً وَسِنِينَ؟
\par 11 أَخَافُ عَلَيْكُمْ أَنْ أَكُونَ قَدْ تَعِبْتُ فِيكُمْ عَبَثاً!
\par 12 أَتَضَرَّعُ إِلَيْكُمْ أَيُّهَا الإِخْوَةُ، كُونُوا كَمَا أَنَا لأَنِّي أَنَا أَيْضاً كَمَا أَنْتُمْ. لَمْ تَظْلِمُونِي شَيْئاً.
\par 13 وَلَكِنَّكُمْ تَعْلَمُونَ أَنِّي بِضَعْفِ الْجَسَدِ بَشَّرْتُكُمْ فِي الأَوَّلِ.
\par 14 وَتَجْرِبَتِي الَّتِي فِي جَسَدِي لَمْ تَزْدَرُوا بِهَا وَلاَ كَرِهْتُمُوهَا، بَلْ كَمَلاَكٍ مِنَ اللهِ قَبِلْتُمُونِي، كَالْمَسِيحِ يَسُوعَ.
\par 15 فَمَاذَا كَانَ إِذاً تَطْوِيبُكُمْ؟ لأَنِّي أَشْهَدُ لَكُمْ أَنَّهُ لَوْ أَمْكَنَ لَقَلَعْتُمْ عُيُونَكُمْ وَأَعْطَيْتُمُونِي.
\par 16 أَفَقَدْ صِرْتُ إِذاً عَدُوّاً لَكُمْ لأَنِّي أَصْدُقُ لَكُمْ؟
\par 17 يَغَارُونَ لَكُمْ لَيْسَ حَسَناً، بَلْ يُرِيدُونَ أَنْ يَصُدُّوكُمْ لِكَيْ تَغَارُوا لَهُمْ.
\par 18 حَسَنَةٌ هِيَ الْغَيْرَةُ فِي الْحُسْنَى كُلَّ حِينٍ، وَلَيْسَ حِينَ حُضُورِي عِنْدَكُمْ فَقَطْ.
\par 19 يَا أَوْلاَدِي الَّذِينَ أَتَمَخَّضُ بِكُمْ أَيْضاً إِلَى أَنْ يَتَصَوَّرَ الْمَسِيحُ فِيكُمْ.
\par 20 وَلَكِنِّي كُنْتُ أُرِيدُ أَنْ أَكُونَ حَاضِراً عِنْدَكُمُ الآنَ وَأُغَيِّرَ صَوْتِي، لأَنِّي مُتَحَيِّرٌ فِيكُمْ!
\par 21 قُولُوا لِي، أَنْتُمُ الَّذِينَ تُرِيدُونَ أَنْ تَكُونُوا تَحْتَ النَّامُوسِ، أَلَسْتُمْ تَسْمَعُونَ النَّامُوسَ؟
\par 22 فَإِنَّهُ مَكْتُوبٌ أَنَّهُ كَانَ لِإِبْرَاهِيمَ ابْنَانِ، وَاحِدٌ مِنَ الْجَارِيَةِ وَالآخَرُ مِنَ الْحُرَّةِ.
\par 23 لَكِنَّ الَّذِي مِنَ الْجَارِيَةِ وُلِدَ حَسَبَ الْجَسَدِ، وَأَمَّا الَّذِي مِنَ الْحُرَّةِ فَبِالْمَوْعِدِ.
\par 24 وَكُلُّ ذَلِكَ رَمْزٌ، لأَنَّ هَاتَيْنِ هُمَا الْعَهْدَانِ، أَحَدُهُمَا مِنْ جَبَلِ سِينَاءَ الْوَالِدُ لِلْعُبُودِيَّةِ، الَّذِي هُوَ هَاجَرُ.
\par 25 لأَنَّ هَاجَرَ جَبَلُ سِينَاءَ فِي الْعَرَبِيَّةِ. وَلَكِنَّهُ يُقَابِلُ أُورُشَلِيمَ الْحَاضِرَةَ، فَإِنَّهَا مُسْتَعْبَدَةٌ مَعَ بَنِيهَا.
\par 26 وَأَمَّا أُورُشَلِيمُ الْعُلْيَا، الَّتِي هِيَ أُمُّنَا جَمِيعاً، فَهِيَ حُرَّةٌ.
\par 27 لأَنَّهُ مَكْتُوبٌ: «افْرَحِي أَيَّتُهَا الْعَاقِرُ الَّتِي لَمْ تَلِدْ. اهْتِفِي وَاصْرُخِي أَيَّتُهَا الَّتِي لَمْ تَتَمَخَّضْ، فَإِنَّ أَوْلاَدَ الْمُوحِشَةِ أَكْثَرُ مِنَ الَّتِي لَهَا زَوْجٌ».
\par 28 وَأَمَّا نَحْنُ أَيُّهَا الإِخْوَةُ فَنَظِيرُ إِسْحَاقَ، أَوْلاَدُ الْمَوْعِدِ.
\par 29 وَلَكِنْ كَمَا كَانَ حِينَئِذٍ الَّذِي وُلِدَ حَسَبَ الْجَسَدِ يَضْطَهِدُ الَّذِي حَسَبَ الرُّوحِ، هَكَذَا الآنَ أَيْضاً.
\par 30 لَكِنْ مَاذَا يَقُولُ الْكِتَابُ؟ «اطْرُدِ الْجَارِيَةَ وَابْنَهَا، لأَنَّهُ لاَ يَرِثُ ابْنُ الْجَارِيَةِ مَعَ ابْنِ الْحُرَّةِ».
\par 31 إِذاً أَيُّهَا الإِخْوَةُ لَسْنَا أَوْلاَدَ جَارِيَةٍ بَلْ أَوْلاَدُ الْحُرَّةِ.

\chapter{5}

\par 1 فَاثْبُتُوا إِذاً فِي الْحُرِّيَّةِ الَّتِي قَدْ حَرَّرَنَا الْمَسِيحُ بِهَا، وَلاَ تَرْتَبِكُوا أَيْضاً بِنِيرِ عُبُودِيَّةٍ.
\par 2 هَا أَنَا بُولُسُ أَقُولُ لَكُمْ: إِنَّهُ إِنِ اخْتَتَنْتُمْ لاَ يَنْفَعُكُمُ الْمَسِيحُ شَيْئاً!
\par 3 لَكِنْ أَشْهَدُ أَيْضاً لِكُلِّ إِنْسَانٍ مُخْتَتِنٍ أَنَّهُ مُلْتَزِمٌ أَنْ يَعْمَلَ بِكُلِّ النَّامُوسِ.
\par 4 قَدْ تَبَطَّلْتُمْ عَنِ الْمَسِيحِ أَيُّهَا الَّذِينَ تَتَبَرَّرُونَ بِالنَّامُوسِ. سَقَطْتُمْ مِنَ النِّعْمَةِ.
\par 5 فَإِنَّنَا بِالرُّوحِ مِنَ الإِيمَانِ نَتَوَقَّعُ رَجَاءَ بِرٍّ.
\par 6 لأَنَّهُ فِي الْمَسِيحِ يَسُوعَ لاَ الْخِتَانُ يَنْفَعُ شَيْئاً وَلاَ الْغُرْلَةُ، بَلِ الإِيمَانُ الْعَامِلُ بِالْمَحَبَّةِ.
\par 7 كُنْتُمْ تَسْعَوْنَ حَسَناً. فَمَنْ صَدَّكُمْ حَتَّى لاَ تُطَاوِعُوا لِلْحَقِّ؟
\par 8 هَذِهِ الْمُطَاوَعَةُ لَيْسَتْ مِنَ الَّذِي دَعَاكُمْ.
\par 9 خَمِيرَةٌ صَغِيرَةٌ تُخَمِّرُ الْعَجِينَ كُلَّهُ.
\par 10 وَلَكِنَّنِي أَثِقُ بِكُمْ فِي الرَّبِّ أَنَّكُمْ لاَ تَفْتَكِرُونَ شَيْئاً آخَرَ. وَلَكِنَّ الَّذِي يُزْعِجُكُمْ سَيَحْمِلُ الدَّيْنُونَةَ أَيَّ مَنْ كَانَ.
\par 11 وَأَمَّا أَنَا أَيُّهَا الإِخْوَةُ فَإِنْ كُنْتُ بَعْدُ أَكْرِزُ بِالْخِتَانِ فَلِمَاذَا أُضْطَهَدُ بَعْدُ؟ إِذاً عَثْرَةُ الصَّلِيبِ قَدْ بَطَلَتْ.
\par 12 يَا لَيْتَ الَّذِينَ يُقْلِقُونَكُمْ يَقْطَعُونَ أَيْضاً!
\par 13 فَإِنَّكُمْ إِنَّمَا دُعِيتُمْ لِلْحُرِّيَّةِ أَيُّهَا الإِخْوَةُ. غَيْرَ أَنَّهُ لاَ تُصَيِّرُوا الْحُرِّيَّةَ فُرْصَةً لِلْجَسَدِ، بَلْ بِالْمَحَبَّةِ اخْدِمُوا بَعْضُكُمْ بَعْضاً.
\par 14 لأَنَّ كُلَّ النَّامُوسِ فِي كَلِمَةٍ وَاحِدَةٍ يُكْمَلُ: «تُحِبُّ قَرِيبَكَ كَنَفْسِكَ».
\par 15 فَإِذَا كُنْتُمْ تَنْهَشُونَ وَتَأْكُلُونَ بَعْضُكُمْ بَعْضاً، فَانْظُرُوا لِئَلاَّ تُفْنُوا بَعْضُكُمْ بَعْضاً.
\par 16 وَإِنَّمَا أَقُولُ: اسْلُكُوا بِالرُّوحِ فَلاَ تُكَمِّلُوا شَهْوَةَ الْجَسَدِ.
\par 17 لأَنَّ الْجَسَدَ يَشْتَهِي ضِدَّ الرُّوحِ وَالرُّوحُ ضِدَّ الْجَسَدِ، وَهَذَانِ يُقَاوِمُ أَحَدُهُمَا الآخَرَ، حَتَّى تَفْعَلُونَ مَا لاَ تُرِيدُونَ.
\par 18 وَلَكِنْ إِذَا انْقَدْتُمْ بِالرُّوحِ فَلَسْتُمْ تَحْتَ النَّامُوسِ.
\par 19 وَأَعْمَالُ الْجَسَدِ ظَاهِرَةٌ: الَّتِي هِيَ زِنىً عَهَارَةٌ نَجَاسَةٌ دَعَارَةٌ
\par 20 عِبَادَةُ الأَوْثَانِ سِحْرٌ عَدَاوَةٌ خِصَامٌ غَيْرَةٌ سَخَطٌ تَحَزُّبٌ شِقَاقٌ بِدْعَةٌ
\par 21 حَسَدٌ قَتْلٌ سُكْرٌ بَطَرٌ، وَأَمْثَالُ هَذِهِ الَّتِي أَسْبِقُ فَأَقُولُ لَكُمْ عَنْهَا كَمَا سَبَقْتُ فَقُلْتُ أَيْضاً: إِنَّ الَّذِينَ يَفْعَلُونَ مِثْلَ هَذِهِ لاَ يَرِثُونَ مَلَكُوتَ اللهِ.
\par 22 وَأَمَّا ثَمَرُ الرُّوحِ فَهُوَ: مَحَبَّةٌ فَرَحٌ سَلاَمٌ، طُولُ أَنَاةٍ لُطْفٌ صَلاَحٌ، إِيمَانٌ
\par 23 وَدَاعَةٌ تَعَفُّفٌ. ضِدَّ أَمْثَالِ هَذِهِ لَيْسَ نَامُوسٌ.
\par 24 وَلَكِنَّ الَّذِينَ هُمْ لِلْمَسِيحِ قَدْ صَلَبُوا الْجَسَدَ مَعَ الأَهْوَاءِ وَالشَّهَوَاتِ.
\par 25 إِنْ كُنَّا نَعِيشُ بِالرُّوحِ فَلْنَسْلُكْ أَيْضاً بِحَسَبِ الرُّوحِ.
\par 26 لاَ نَكُنْ مُعْجِبِينَ نُغَاضِبُ بَعْضُنَا بَعْضاً، وَنَحْسِدُ بَعْضُنَا بَعْضاً.

\chapter{6}

\par 1 أَيُّهَا الإِخْوَةُ، إِنِ انْسَبَقَ إِنْسَانٌ فَأُخِذَ فِي زَلَّةٍ مَا، فَأَصْلِحُوا أَنْتُمُ الرُّوحَانِيِّينَ مِثْلَ هَذَا بِرُوحِ الْوَدَاعَةِ، نَاظِراً إِلَى نَفْسِكَ لِئَلاَّ تُجَرَّبَ أَنْتَ أَيْضاً.
\par 2 اِحْمِلُوا بَعْضُكُمْ أَثْقَالَ بَعْضٍ وَهَكَذَا تَمِّمُوا نَامُوسَ الْمَسِيحِ.
\par 3 لأَنَّهُ إِنْ ظَنَّ أَحَدٌ أَنَّهُ شَيْءٌ وَهُوَ لَيْسَ شَيْئاً، فَإِنَّهُ يَغِشُّ نَفْسَهُ.
\par 4 وَلَكِنْ لِيَمْتَحِنْ كُلُّ وَاحِدٍ عَمَلَهُ، وَحِينَئِذٍ يَكُونُ لَهُ الْفَخْرُ مِنْ جِهَةِ نَفْسِهِ فَقَطْ، لاَ مِنْ جِهَةِ غَيْرِهِ.
\par 5 لأَنَّ كُلَّ وَاحِدٍ سَيَحْمِلُ حِمْلَ نَفْسِهِ.
\par 6 وَلَكِنْ لِيُشَارِكِ الَّذِي يَتَعَلَّمُ الْكَلِمَةَ الْمُعَلِّمَ فِي جَمِيعِ الْخَيْرَاتِ.
\par 7 لاَ تَضِلُّوا! اللهُ لاَ يُشْمَخُ عَلَيْهِ. فَإِنَّ الَّذِي يَزْرَعُهُ الإِنْسَانُ إِيَّاهُ يَحْصُدُ أَيْضاً.
\par 8 لأَنَّ مَنْ يَزْرَعُ لِجَسَدِهِ فَمِنَ الْجَسَدِ يَحْصُدُ فَسَاداً، وَمَنْ يَزْرَعُ لِلرُّوحِ فَمِنَ الرُّوحِ يَحْصُدُ حَيَاةً أَبَدِيَّةً.
\par 9 فَلاَ نَفْشَلْ فِي عَمَلِ الْخَيْرِ لأَنَّنَا سَنَحْصُدُ فِي وَقْتِهِ إِنْ كُنَّا لاَ نَكِلُّ.
\par 10 فَإِذاً حَسْبَمَا لَنَا فُرْصَةٌ فَلْنَعْمَلِ الْخَيْرَ لِلْجَمِيعِ، وَلاَ سِيَّمَا لأَهْلِ الإِيمَانِ.
\par 11 اُنْظُرُوا، مَا أَكْبَرَ الأَحْرُفَ الَّتِي كَتَبْتُهَا إِلَيْكُمْ بِيَدِي!
\par 12 جَمِيعُ الَّذِينَ يُرِيدُونَ أَنْ يَعْمَلُوا مَنْظَراً حَسَناً فِي الْجَسَدِ، هَؤُلاَءِ يُلْزِمُونَكُمْ أَنْ تَخْتَتِنُوا، لِئَلاَّ يُضْطَهَدُوا لأَجْلِ صَلِيبِ الْمَسِيحِ فَقَطْ.
\par 13 لأَنَّ الَّذِينَ يَخْتَتِنُونَ هُمْ لاَ يَحْفَظُونَ النَّامُوسَ، بَلْ يُرِيدُونَ أَنْ تَخْتَتِنُوا أَنْتُمْ لِكَيْ يَفْتَخِرُوا فِي جَسَدِكُمْ.
\par 14 وَأَمَّا مِنْ جِهَتِي، فَحَاشَا لِي أَنْ أَفْتَخِرَ إِلاَّ بِصَلِيبِ رَبِّنَا يَسُوعَ الْمَسِيحِ، الَّذِي بِهِ قَدْ صُلِبَ الْعَالَمُ لِي وَأَنَا لِلْعَالَمِ.
\par 15 لأَنَّهُ فِي الْمَسِيحِ يَسُوعَ لَيْسَ الْخِتَانُ يَنْفَعُ شَيْئاً وَلاَ الْغُرْلَةُ، بَلِ الْخَلِيقَةُ الْجَدِيدَةُ.
\par 16 فَكُلُّ الَّذِينَ يَسْلُكُونَ بِحَسَبِ هَذَا الْقَانُونِ عَلَيْهِمْ سَلاَمٌ وَرَحْمَةٌ، وَعَلَى إِسْرَائِيلِ اللهِ.
\par 17 فِي مَا بَعْدُ لاَ يَجْلِبُ أَحَدٌ عَلَيَّ أَتْعَاباً، لأَنِّي حَامِلٌ فِي جَسَدِي سِمَاتِ الرَّبِّ يَسُوعَ.
\par 18 نِعْمَةُ رَبِّنَا يَسُوعَ الْمَسِيحِ مَعَ رُوحِكُمْ أَيُّهَا الإِخْوَةُ. آمِينَ.

\end{document}