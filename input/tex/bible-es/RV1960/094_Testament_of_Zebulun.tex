\begin{document}

\title{Testamento de Zabulón}

\chapter{1}

\par \textit{Zabulón, el sexto hijo de Jacob y Lea. El inventor y filántropo. Lo que aprendió como resultado del complot contra José.}

\par 1 LA copia de las palabras que Zabulón ordenó a sus hijos antes de morir en el año ciento catorce de su vida, dos años después de la muerte de José.

\par 2 Y él les dijo: Escuchenme, hijos de Zabulón, escuchen las palabras de su padre.

\par 3 Yo, Zabulón, nací para ser un buen regalo para mis padres.

\par 4 Porque cuando yo nací, mi padre creció muchísimo en ovejas y en vacas, cuando con las varas atravesadas tenía su parte.

\par 5 No soy consciente de haber pecado en todos mis días, salvo en el pensamiento.

\par 6 Todavía no me acuerdo de haber cometido ninguna iniquidad, excepto el pecado de ignorancia que cometí contra José; porque hice pacto con mis hermanos de no contarle a mi padre lo que había sucedido.

\par 7 Pero lloré en secreto muchos días a causa de José, porque tenía miedo de mis hermanos, porque todos habían acordado que si alguien revelaba el secreto, sería asesinado.

\par 8 Pero cuando quisieron matarlo, les rogué con lágrimas que no fueran culpables de este pecado.

\par 9 Porque Simeón y Gad vinieron contra José para matarlo, y él les dijo entre lágrimas: Compadecidme, hermanos míos, tened piedad de las entrañas de nuestro padre Jacob; no pongáis sobre mí vuestras manos para derramar sangre inocente, porque No he pecado contra ti.

\par 10 Y si en verdad he pecado, castigadme, hermanos míos, con disciplina, pero no pongáis vuestra mano sobre mí, por amor a nuestro padre Jacob,

\par 11 Y mientras él hablaba estas palabras, gimiendo mientras lo hacía, yo no podía soportar sus lamentaciones y comencé a llorar, y mi hígado se derramó y toda la sustancia de mis entrañas se soltó.

\par 12 Y lloré con José, y mi corazón latía con fuerza, y las coyunturas de mi cuerpo temblaban y no podía sostenerme en pie.

\par 13 Y cuando José me vio llorando con él y que venían contra él para matarlo, huyó detrás de mí, suplicándoles.

\par 14 Pero mientras tanto Rubén se levantó y dijo: Venid, hermanos míos, no lo matemos, sino arrojémoslo en una de estas cisternas secas que nuestros padres cavaron y no encontraron agua.

\par 15 Por esta razón el Señor prohibió que subiera agua en ellos para preservar a José.

\par 16 Y así lo hicieron, hasta que lo vendieron a los ismaelitas.

\par 17 Porque yo no participé en su precio, hijos míos.

\par 18 Pero Simeón, Gad y otros seis de nuestros hermanos tomaron el precio de José y compraron sandalias para ellos, sus mujeres y sus hijos, diciendo:

\par 19 No comeremos de él, porque es el precio de la sangre de nuestro hermano, pero ciertamente lo pisotearemos, porque él dijo que sería rey sobre nosotros, y así veremos qué será de él. Sueños.

\par 20 Por eso está escrito en la escritura de la ley de Moisés: A cualquiera que no quiera dar descendencia a su hermano, se le quitará la sandalia y le escupirán en la cara.

\par 21 Pero los hermanos de José no querían que su hermano viviera, y el Señor les quitó la sandalia que llevaban contra José su hermano.

\par 22 Porque cuando entraron en Egipto, los siervos de José los soltaron fuera de la puerta, e hicieron reverencia a José a la manera del rey Faraón.

\par 23 Y no sólo le rindieron reverencia, sino que también fueron escupidos y al instante se postraron ante él, y así quedaron avergonzados ante él los egipcios.

\par 24 Porque después de esto los egipcios se enteraron de todos los males que le habían hecho a José.

\par 25 Y cuando lo vendieron, mis hermanos se sentaron a comer y a beber.

\par 26 Pero yo, por compasión a José, no comí, sino que vigilé el hoyo, porque Judá temía que Simeón, Dan y Gad se precipitaran y lo mataran.

\par 27 Pero como vieron que yo no comía, me pusieron a vigilarlo hasta que fue vendido a los ismaelitas.

\par 28 Cuando Rubén llegó y oyó que José había sido vendido mientras estaba fuera, rasgó sus vestidos y, lamentándose, dijo:

\par 29 ¿Cómo veré el rostro de mi padre Jacob? Y tomó el dinero y corrió tras los mercaderes pero como no los encontró, regresó afligido.

\par 30 Pero los mercaderes abandonaron el camino ancho y atravesaron a los trogloditas por un atajo.

\par 31 Pero Rubén se entristeció y no comió nada aquel día.

\par 32 Entonces Dan se acercó a él y le dijo: No llores ni te aflijas; porque hemos encontrado lo que podemos decirle a nuestro padre Jacob.

\par 33 Matemos un cabrito y mojemos en él la túnica de José; y enviémosla a Jacob, diciendo: Sabes, ¿es ésta la túnica de tu hijo?

\par 34 Y así lo hicieron. Porque le quitaron a José su manto cuando lo vendían, y le pusieron ropa de esclavo.

\par 35 Simeón tomó la túnica y no quiso soltarla, porque quería rasgarla con su espada, enojado porque José vivía y no lo había matado.

\par 36 Entonces todos nos levantamos y le dijimos: Si no entregas la túnica, diremos a nuestro padre que tú solo hiciste esta maldad en Israel.

\par 37 Y él se lo dio, y ellos hicieron tal como Dan había dicho.

\chapter{2}

\par \textit{Él insta a la simpatía humana y la comprensión hacia el prójimo.}

\par 1 Y ahora, hijos, os digo que guardéis los mandamientos del Señor, que seáis misericordiosos con vuestro prójimo y que tengáis compasión de todos, no sólo de los hombres, sino también de las bestias.

\par 2 Por todo esto el Señor me bendijo, y cuando todos mis hermanos estaban enfermos, yo salí sano y salvo, porque el Señor conoce los propósitos de cada uno.

\par 3 Tened, pues, compasión en vuestros corazones, hijos míos, porque lo que el hombre hace con su prójimo, así también hará el Señor con él.

\par 4 Porque los hijos de mis hermanos estaban enfermos y morían a causa de José, porque no mostraron misericordia en sus corazones; pero mis hijos se salvaron sin enfermedad, como sabéis.

\par 5 Y cuando estaba en la tierra de Canaán, junto al mar, preparé peces para mi padre Jacob; y cuando muchos se ahogaron en el mar, yo seguí ileso.

\par 6 Yo fui el primero en construir una barca para navegar por el mar, porque el Señor me dio entendimiento y sabiduría para ello.

\par 7 Y bajé el timón detrás de él, y extendí una vela sobre otro palo de madera vertical en el medio.

\par 8 Y navegué por allí a lo largo de la costa, pescando para la casa de mi padre, hasta que llegamos a Egipto.

\par 9 Y por compasión compartí mi pesca con cada extraño.

\par 10 Y si alguno era forastero, o estaba enfermo o era anciano, yo hervía el pescado, lo preparaba bien y se lo ofrecía a todos, según cada uno tenía necesidad, haciéndome duelo y teniendo compasión de él.

\par 11 Por eso también el Señor me saciaba con abundancia de peces cuando pescaba; porque el que comparte con su prójimo, mucho más recibirá del Señor.

\par 12 Durante cinco años pesqué peces y se los di a todos los que vi, y fue suficiente para toda la casa de mi padre.

\par 13 Y en el verano pescaba, y en el invierno cuidaba ovejas con mis hermanos.

\par 14 Ahora os contaré lo que hice.

\par 15 Vi a un hombre afligido por la desnudez en invierno, y tuve compasión de él, y robé en secreto un vestido de la casa de mi padre y se lo di al que estaba en apuros.

\par 16 Por tanto, hijos míos, vosotros, de lo que Dios os concede, mostrad sin vacilación compasión y misericordia a todos los hombres, y dadlo a cada uno con buen corazón.

\par 17 Y si no tenéis con qué dar al que necesita, tened compasión de él con entrañas de misericordia.

\par 18 Sé que mi mano no encontró con qué dar al que necesitaba, y caminé con él llorando durante siete estadios, y mis entrañas anhelaban verlo con compasión.

\par 19 Por tanto, hijos míos, también vosotros tened compasión de todos con misericordia, para que el Señor también tenga compasión y misericordia de vosotros.

\par 20 Porque también en los últimos días Dios enviará su compasión a la tierra, y dondequiera que encuentre entrañas de misericordia, habitará en él.

\par 21 Porque en la medida en que el hombre tiene compasión de su prójimo, en la misma medida la tendrá el Señor sobre él.

\par 22 Y cuando bajamos a Egipto, José no tuvo rencor contra nosotros.

\par 23 Prestando atención a ellos, también vosotros, hijos míos, aprobaos sin malicia y amaos unos a otros; y no toméis en cuenta, cada uno de vosotros, el mal contra su hermano.

\par 24 Porque esto rompe la unidad, divide a todos los parientes, perturba el alma y desgasta el rostro.

\par 25 Observad, pues, las aguas y sabed que cuando corren juntas, arrastran piedras, árboles, tierra y otras cosas.

\par 26 Pero si se dividen en muchos arroyos, la tierra los traga y desaparecen.

\par 27 Así también seréis vosotros si estáis divididos. No estéis, pues, divididos en dos cabezas, porque todo lo que el Señor hizo tiene una sola cabeza, dos hombros, dos manos, dos pies y todos los demás miembros.

\par 28 Porque en las escrituras de mis padres he aprendido que seréis divididos en Israel, seguiréis a dos reyes y haréis toda abominación.

\par 29 Y vuestros enemigos os llevarán cautivos, y seréis maltratados entre los gentiles, con muchas enfermedades y tribulaciones.

\par 30 Y después de esto os acordaréis del Señor y os arrepentiréis, y Él tendrá misericordia de vosotros, porque es misericordioso y compasivo.

\par 31 Y no toma en cuenta el mal contra los hijos de los hombres, porque son carne y están engañados por sus propias malas acciones.

\par 32 Y después de estas cosas os surgirá el Señor mismo, la luz de la justicia, y volveréis a vuestra tierra.

\par 33 Y le veréis en Jerusalén por amor de su nombre.

\par 34 Y otra vez con la maldad de vuestras obras le provocaréis a ira,

\par 35 Y seréis desechados por Él hasta el tiempo de la consumación.

\par 36 Ahora pues, hijos míos, no os lamentéis porque estoy muriendo, ni os abatáis porque estoy llegando a mi fin.

\par 37 Porque resucitaré en medio de vosotros, como un gobernante en medio de sus hijos; y me alegraré en medio de mi tribu, de cuantos guarden la ley de Jehová y los mandamientos de Zabulón su padre.

\par 38 Pero el Señor traerá fuego eterno sobre los impíos y los destruirá de generación en generación.

\par 39 Pero ahora me apresuro a descansar, como también mis padres.

\par 40 Pero tú temerás al Señor nuestro Dios con todas tus fuerzas todos los días de tu vida.

\par 41 Y habiendo dicho estas cosas, se durmió ya en una buena vejez.

\par 42 Y sus hijos lo pusieron en un ataúd de madera. Y después lo llevaron y lo sepultaron en Hebrón, con sus padres.



\end{document}