\begin{document}

\title{Segundo Libro de Enoc}

\chapter{1}

\par \textit{Un relato del mecanismo del mundo que muestra la maquinaria del sol y la luna en funcionamiento. Astronomía y un interesante calendario antiguo. Véase el Capítulo 15-17 y también el 21. Cómo era el mundo antes de la Creación, consulte el Capítulo 24. El Capítulo 26 es especialmente pintoresco. Un relato único de cómo se creó Satanás (Capítulo 29.)}

\par 1 Había un hombre sabio, un gran artífice, y el Señor concibió amor por él y lo recibió, para que contemplara las moradas más altas y fuera testigo ocular del reino sabio, grande, inconcebible e inmutable de Dios Todopoderoso. , de la posición muy maravillosa, gloriosa, brillante y de muchos ojos de los siervos del Señor, y del trono inaccesible del Señor, y de los grados y manifestaciones de las huestes incorpóreas, y de la inefable ministración de la multitud de los elementos. , y de las diversas apariciones y cantos inexpresables de la hueste de querubines y de la luz ilimitada.

\par 2 En aquel tiempo, dijo, cuando cumplí 165 años, engendré a mi hijo Mathusal.

\par 3 Después de esto también viví doscientos años y completé de todos los años de mi vida trescientos sesenta y cinco años.

\par 4 El primer día del primer mes yo estaba solo en mi casa y descansaba en mi lecho y dormía.

\par 5 Y mientras dormía, una gran angustia subió a mi corazón, y mientras dormía lloraba con los ojos, y no podía entender qué era esta angustia ni qué me sucedería.

\par 6 Y se me aparecieron dos hombres tan grandes que nunca había visto semejantes en la tierra; sus rostros brillaban como el sol, sus ojos también eran como una luz ardiente, y de sus labios salía fuego con ropas y cantos de diversas clases de apariencia púrpura, sus alas eran más brillantes que el oro, sus manos más blancas que la nieve.

\par 7 Estaban parados a la cabecera de mi lecho y comenzaron a llamarme por mi nombre.

\par 8 Y me levanté de mi sueño y vi claramente a aquellos dos hombres que estaban delante de mí.

\par 9 Y los saludé y me asaltó el miedo y el aspecto de mi rostro cambió de terror, y aquellos hombres me dijeron:

\par 10 'Ten valor, Enoc, no temas; El Dios eterno nos envió a ti, y ¡he aquí! Subirás hoy con nosotros al cielo, y contarás a tus hijos y a toda tu casa todo lo que harán sin ti en la tierra en tu casa, y nadie te buscará hasta que el Señor te devuelva a ellos.

\par 11 Y me apresuré a obedecerlos y salí de mi casa y me dirigí a las puertas, como me había ordenado, y llamé a mis hijos Mathusal, Regim y Gaidad y les conté todas las maravillas que aquellos hombres habían contado. a mí.

\chapter{2}

\par \textit{La Instrucción. Cómo Enoc instruyó a sus hijos.}

\par 1 Escúchenme, hijos míos, no sé adónde voy ni qué me sucederá; Ahora pues, hijos míos, os digo: No os apartéis de Dios delante de los vanidosos, que no hicieron el cielo ni la tierra, porque éstos perecerán y los que los adoran, y que el Señor confíe vuestros corazones en el temor de a él. Y ahora, hijos míos, nadie piense en buscarme, hasta que el Señor me devuelva a vosotros.

\chapter{3}

\par \textit{De la suposición de Enoc; cómo los ángeles lo llevaron al primer cielo.}

\par 1 Aconteció que cuando Enoc se lo contó a sus hijos, los ángeles lo tomaron en sus alas, lo llevaron al primer cielo y lo pusieron sobre las nubes. Y allí miré, y otra vez miré más alto, y vi el éter, y me colocaron en el primer cielo y me mostraron un Mar muy grande, más grande que el mar terrestre.

\chapter{4}

\par \textit{De los Ángeles que gobiernan las estrellas.}

\par 1 Trajeron ante mí a los ancianos y gobernantes de los órdenes estelares, y me mostraron doscientos ángeles, que gobiernan las estrellas y sus servicios a los cielos, y vuelan con sus alas y rodean a todos los que navegan.

\chapter{5}

\par \textit{De cómo los Ángeles guardan los almacenes de la nieve.}

\par 1 Y aquí miré hacia abajo y vi los tesoros de la nieve, y los ángeles que guardan sus terribles almacenes, y las nubes de donde salen y a las que van.

\chapter{6}

\par \textit{Del rocío y del aceite de oliva, y de diversas flores.}

\par 1 Me mostraron el tesoro del rocío, como el aceite de oliva, y el aspecto de su forma, como el de todas las flores de la tierra; Además, muchos ángeles guardan los tesoros de estas cosas, y cómo se las hace cerrar y abrir.

\chapter{7}

\par \textit{De cómo Enoc fue llevado al segundo cielo.}

\par 1 Y aquellos hombres me tomaron y me llevaron al segundo cielo, y me mostraron tinieblas, mayores que las tinieblas terrenales, y allí vi a los prisioneros colgados, vigilando, esperando el juicio grande e ilimitado, y estos ángeles estaban en tinieblas. mirando, más que las tinieblas terrenales, y haciendo llorar sin cesar a todas horas.

\par 2 Y dije a los hombres que estaban conmigo: «¿Por qué son torturados sin cesar?» Ellos me respondieron: 'Estos son los apóstatas de Dios, que no obedecieron los mandamientos de Dios, sino que consultaron con su propia voluntad y se alejaron con su príncipe, quien también está fijado en el quinto cielo.'

\par 3 Y sentí gran compasión por ellos, y me saludaron y me dijeron: 'Hombre de Dios, ruega por nosotros al Señor'; y les respondí: '¿Quién soy yo, hombre mortal, para orar por los ángeles? ¿Quién sabe adónde voy o qué me sucederá? ¿O quién orará por mí?

\chapter{8}

\par \textit{De la asunción de Enoc al tercer cielo.}

\par 1 Y aquellos hombres me tomaron de allí, me llevaron al tercer cielo y me pusieron allí; y miré hacia abajo, y observé el producto de estos lugares, tal como nunca se ha conocido por su bondad.

\par 2 Y vi todos los árboles de dulces flores y vi sus frutos, que eran fragantes, y todos los alimentos que llevaban burbujeando con fragantes exhalaciones.

\par 3 Y en medio de los árboles el de la vida, en aquel lugar donde reposa el Señor cuando sube al paraíso; y este árbol es de inefable bondad y fragancia, y adornado más que todo lo existente; y por todos lados tiene forma de oro, bermellón y fuego, y lo cubre todo, y tiene producto de todos los frutos.

\par 4 Su raíz está en el jardín del fin de la tierra.

\par 5 Y el paraíso está entre la corruptibilidad y la incorruptibilidad.

\par 6 Y brotan dos manantiales que dan miel y leche, y sus manantiales dan aceite y vino, y se separan en cuatro partes, y dan vueltas con curso tranquilo, y descienden al PARAÍSO DEL EDEN, entre la corruptibilidad y la corrupción. en la corruptibilidad.

\par 7 Y desde allí avanzan a lo largo de la tierra y hacen una revolución en su círculo, al igual que los demás elementos.

\par 8 Y aquí no hay árbol infructuoso y todo lugar es bendito.

\par 9 Y hay trescientos ángeles muy brillantes que guardan el jardín, y con incesantes dulces cantos y voces nunca silenciosas sirven al Señor durante todos los días y horas.

\par 10 Y dije: «Qué dulce es este lugar», y aquellos hombres me dijeron:

\chapter{9}

\par \textit{La demostración a Enoc del lugar de los justos y compasivos.}

\par 1 ESTE lugar, oh Enoc, está preparado para los justos, que soportan toda clase de ofensas por parte de aquellos que exasperan sus almas, que apartan sus ojos de la iniquidad, y hacen juicios justos, y dan pan a los hambrientos, y cubren a los desnudos con ropa, y levantar a los caídos, y ayudar a los huérfanos heridos, y que caminan sin mancha delante del rostro del Señor, y servirle solo a él, y para ellos está preparado este lugar para herencia eterna.

\chapter{10}

\par \textit{Aquí le mostraron a Enoc el terrible lugar y varias torturas.}

\par 1 Y esos dos hombres me llevaron al lado norte, y me mostraron allí un lugar muy terrible, y había toda clase de torturas en ese lugar: oscuridad cruel y oscuridad sin luz, y no hay luz allí, sino oscuridad. el fuego arde constantemente en lo alto, y de un río de fuego brota, y todo ese lugar es fuego por todas partes, y por todas partes hay escarcha y hielo, sed y temblores, mientras que las ataduras son muy crueles, y los ángeles temerosos y despiadados, soportando enojados. armas, torturas despiadadas, y dije:

\par 2 '¡Ay, ay, qué terrible es este lugar!'

\par 3 Y aquellos hombres me dijeron: Este lugar, oh Enoc, está preparado para aquellos que deshonran a Dios, que en la tierra practican el pecado contra la naturaleza, que es la corrupción infantil al estilo sodomítico, la magia, los encantamientos y las hechicerías diabólicas. , y que se jactan de sus malas obras, hurtos, mentiras, calumnias, envidias, rencores, fornicaciones, asesinatos, y que, malditos, roban las almas de los hombres, que viendo a los pobres quitarles sus bienes y enriquecerse, les hacen daño. por bienes ajenos; quien pudiendo saciar el vacío, hizo morir al hambriento; pudiendo vestir, desnudar al desnudo; y que no conoció a su creador, y se postró ante dioses sin alma (es decir, sin vida), que no pueden ver ni oír, dioses vanos, que también construyeron imágenes talladas y se postraron ante obras impuras, para todos estos está preparado este lugar entre ellos, para la herencia eterna.

\chapter{11}

\par \textit{Aquí llevaron a Enoc al cuarto cielo, donde está el curso del sol y la luna.}

\par 1 Aquellos hombres me tomaron y me llevaron al cuarto cielo, y me mostraron todos los pasos sucesivos y todos los rayos de la luz del sol y de la luna.

\par 2 Y medí sus pasos y comparé su luz, y vi que la luz del sol es mayor que la de la luna.

\par 3 Su círculo y las ruedas sobre las que camina siempre, como un viento que pasa con velocidad muy maravillosa, y de día y de noche no tiene descanso. 1

\par 4 Su paso y su regreso van acompañados de cuatro grandes estrellas, y cada estrella tiene debajo de ella mil estrellas, a la derecha de la rueda del sol, y cuatro a la izquierda, cada una con mil estrellas debajo de ella, en total ocho mil. , saliendo con el sol continuamente.

\par 5 Y de día asisten a ella quince millares de ángeles, y de noche mil.

\par 6 Y unos seres de seis alas salen con los ángeles delante de la rueda del sol hacia las llamas de fuego, y cien ángeles encienden el sol y lo encienden.

\par \textit{85:1 Cfr. «Tránsito rápido.»}}

\chapter{12}

\par \textit{De los maravillosos elementos del sol.}

\par 1 Y miré y vi otros elementos voladores del sol, cuyos nombres son Fénix y Calquidri, maravillosos y maravillosos, con pies y colas en forma de león, y cabeza de cocodrilo, su apariencia es púrpura, como el arco iris. ; su tamaño es de novecientas medidas, sus alas son como las de los ángeles, cada uno tiene doce, y atienden y acompañan al sol, trayendo calor y rocío, como les ordena Dios.

\par 2 Así el sol gira y va, y sale bajo el cielo, y su curso pasa bajo la tierra con la luz de sus rayos sin cesar.

\chapter{13}

\par \textit{Los ángeles tomaron a Enoc y lo colocaron en el este, a las puertas del sol.}

\par 1 Aquellos hombres me llevaron hacia el oriente y me pusieron a las puertas del sol, por donde sale el sol según la regla de las estaciones y el ciclo de los meses de todo el año y el número de las horas del día. y noche,

\par 2 Y vi seis puertas abiertas, cada una de las cuales tenía sesenta y un estadios y un cuarto de estadio, y las medí con exactitud, y comprendí que su tamaño era tal, por donde sale el sol y va hacia el hacia el oeste, y se nivela, y sube a lo largo de todos los meses, y vuelve de las seis puertas según la sucesión de las estaciones; así el período de todo el año termina después del regreso de las cuatro estaciones,

\chapter{14}

\par \textit{Se llevaron a Enoc a Occidente.}

\par 1 Y de nuevo aquellos hombres me llevaron a las partes occidentales, y me mostraron seis grandes puertas abiertas correspondientes a las puertas orientales, frente a donde se pone el sol, según el número de los días trescientos sesenta y cinco y un cuarto.

\par 2 Así desciende de nuevo a las puertas occidentales y arrastra su luz, la grandeza de su brillo, bajo la tierra; porque puesto que la corona de su resplandor está en el cielo con el Señor, y guardada por cuatrocientos ángeles, mientras el sol gira sobre su rueda debajo de la tierra, y permanece siete largas horas en la noche, y pasa la mitad de su recorrido bajo la tierra, cuando llega al este a la hora octava de la noche, trae sus luces y la corona de resplandor, y el sol arde más que el fuego.

\chapter{15}

\par \textit{Los elementos del sol, los Fénix y Chalkydri rompieron a cantar.}

\par 1 ENTONCES los elementos del sol, llamados Fénix y Calkydri, rompen a cantar, por eso cada pájaro bate sus alas, regocijándose ante el dador de luz, y rompieron a cantar por orden del Señor.

\par 2 El dador de luz viene para dar brillo al mundo entero, y la guardia de la mañana toma forma, que son los rayos del sol, y el sol de la tierra sale y recibe su brillo para iluminar toda la faz. de la tierra, y me mostraron este cálculo de la marcha del sol.

\par 3 Y las puertas por las que entra, éstas son las grandes puertas del cómputo de las horas del año; por eso el sol es una gran creación, cuyo circuito dura veintiocho años, y comienza de nuevo desde el principio.

\chapter{16}

\par \textit{Tomaron a Enoc y lo colocaron nuevamente en el este en el curso de la luna.}

\par 1 Aquellos hombres me mostraron el otro curso, el de la luna, doce grandes puertas, coronadas de oeste a este, por las que la luna entra y sale fuera de los tiempos habituales.

\par 2 Entra por la primera puerta hacia los lugares occidentales del sol, por las primeras puertas con treinta y un días exactamente, por la segunda puerta con treinta y un días exactamente, por la tercera con treinta días exactamente, por la el cuarto con treinta días exactamente, por el quinto con treinta y un días exactamente, por el sexto con treinta y un días exactamente, por el séptimo con treinta días exactamente, por el octavo con treinta y un días perfectamente, por el noveno con treinta- un día exactamente, el décimo con treinta días perfectamente, el undécimo con treinta y un días exactamente, el duodécimo con veintiocho días exactamente.

\par 3 Y atraviesa las puertas occidentales en el orden y número de las puertas orientales, y cumple los trescientos sesenta y cinco días y cuarto del año solar, mientras que el año lunar tiene trescientos cincuenta y cuatro, y faltan doce días del círculo solar, que son los epactos lunares de todo el año.

\par 4 (Así, también, el gran círculo contiene quinientos treinta y dos años.)

\par 5 El cuarto de día se omite durante tres años, el cuarto lo cumple exactamente.

\par 6 Por eso son sacados fuera del cielo por tres años y no se añaden al número de días, porque cambian el tiempo de los años a dos nuevos meses hacia el cumplimiento, y a otros dos hacia la disminución.

\par 7 Y cuando las puertas occidentales están terminadas, regresa y va hacia el este, hacia las luces, y así recorre los círculos celestiales día y noche, más bajo que todos los círculos, más rápido que los vientos celestiales, los espíritus, los elementos y los ángeles. volador; cada ángel tiene seis alas.

\par 8 Tiene un curso séptuple en diecinueve años.

\chapter{17}

\par \textit{De los cantos de los ángeles, que es imposible describir.}

\par 1 EN medio del cielo vi soldados armados, sirviendo al Señor, con tímpanos y órganos, con voz incesante, con voz dulce, con voz dulce e incesante y con diversos cantos, que es imposible describir, y que asombran. cada mente, tan maravilloso y maravilloso es el canto de aquellos ángeles, y yo me deleitaba escuchándolo.

\chapter{18}

\par \textit{Del traslado de Enoc al quinto cielo.}

\par 1 LOS hombres me llevaron al quinto cielo y me colocaron, y allí vi muchos e innumerables soldados, llamados Grigori, de apariencia humana, y su tamaño era mayor que el de los grandes gigantes y sus rostros se marchitaron, y el silencio de sus bocas perpetuo, y no hubo servicio en el quinto cielo, y dije a los hombres que estaban conmigo:

\par 2 ¿Por qué están estos tan marchitos, sus rostros melancólicos y sus bocas silenciosas, y por qué no hay servicio en este cielo?

\par 3 Y me dijeron: Estos son los Grigori, quienes con su príncipe Satanail rechazaron al Señor de la luz, y después de ellos están los que están retenidos en gran oscuridad en el segundo cielo, y tres de ellos descendieron a la tierra. desde el trono del Señor, hasta el lugar de Ermón, y rompieron sus votos en la loma del monte Ermón 1 y vieron cuán buenas eran las hijas de los hombres, y tomaron mujeres, y contaminaron la tierra con sus obras, que en todos los tiempos de su época hicieron anarquía y mestizaje, y nacieron gigantes y grandes hombres maravillosos y grandes enemistades.

\par 4 Por eso Dios los juzgó con gran juicio, y lloran por sus hermanos y serán castigados en el gran día del Señor.

\par 5 Y dije a los Grigori: «Vi a vuestros hermanos y sus obras y sus grandes tormentos, y oré por ellos, pero el Señor los ha condenado a estar bajo la tierra hasta que el cielo y la tierra terminen para siempre».

\par 6 Y dije: «¿Por qué, hermanos, esperáis y no servís delante del Señor, ni habéis presentado vuestros servicios delante del Señor, para no enfadar completamente a vuestro Señor?»

\par 7 Y ellos escucharon mi advertencia y hablaron a los cuatro rangos en el cielo, y ¡he aquí! Mientras estaba con esos dos hombres, cuatro trompetas sonaron juntas con gran voz, y los Grigori rompieron a cantar a una sola voz, y su voz se elevó ante el Señor de manera lastimera y conmovedora.

\par \textit{87:1 Comparar el segundo libro de Adán y Eva. Cap. XX.}}

\chapter{19}

\par \textit{Del traslado de Enoc al sexto cielo.}

\par 1 Y de allí aquellos hombres me tomaron y me llevaron al sexto cielo, y allí vi siete grupos de ángeles, muy brillantes y muy gloriosos, y sus rostros brillaban más que el sol, brillaban, y no había diferencia en sus rostros, comportamiento o forma de vestir; y éstos dan las órdenes y aprenden el rumbo de las estrellas, la alteración de la luna o la revolución del sol y el buen gobierno del mundo.

\par 2 Y cuando ven la maldad, pronuncian mandamientos e instrucciones, y cantan dulces y fuertes, y todos los cánticos de alabanza.

\par 3 Estos son los arcángeles que están por encima de los ángeles, miden toda la vida en el cielo y en la tierra, y los ángeles que están designados para las estaciones y los años, los ángeles que están sobre los ríos y el mar, y que están sobre los frutos de la tierra. , y los ángeles que están sobre toda hierba, dando alimento a todos, a todo ser viviente, y los ángeles que escriben todas las almas de los hombres, y todas sus obras, y sus vidas delante del rostro del Señor; en medio de ellos hay seis Fénix y seis Querubines y seis de seis alas cantando continuamente a una sola voz, y no es posible describir su canto, y se regocijan delante del Señor en el estrado de sus pies.

\chapter{20}

\par \textit{Por eso llevaron a Enoc al Séptimo Cielo.}

\par 1 Y aquellos dos hombres me elevaron de allí al séptimo cielo, y vi allí una luz muy grande y tropas de fuego de grandes arcángeles, fuerzas incorpóreas y dominios, órdenes y gobiernos, querubines y serafines, tronos y muchos Los de ojos oscuros, nueve regimientos, las estaciones de luz ioanitas, y tuve miedo y comencé a temblar de gran terror, y esos hombres me tomaron, me llevaron tras ellos y me dijeron:

\par 2 «Ten ánimo, Enoc, no temas», y me mostró al Señor desde lejos, sentado en su altísimo trono. Porque ¿qué hay en el décimo cielo, puesto que aquí habita el Señor?

\par 3 En el décimo cielo está Dios, en lengua hebrea se le llama Aravat.

\par 4 Y todas las tropas celestiales vendrían y subirían a los diez escalones según su rango, se postrarían ante el Señor y volverían a sus lugares con alegría y felicidad, cantando canciones en la luz infinita con pequeñas y voces tiernas, sirviéndole gloriosamente.

\chapter{21}

\par \textit{De cómo los ángeles aquí dejaron a Enoc, al final del séptimo Cielo, y se alejaron de él sin ser vistos.}

\par 1 Y los querubines y serafines que están alrededor del trono, los de seis alas y muchos ojos, no se apartan, de pie ante el rostro del Señor, haciendo su voluntad, y cubren todo su trono, cantando con voz suave ante el rostro del Señor: 'Santo, santo, santo, Señor Gobernante de Sabaoth, los cielos y la tierra están llenos de tu gloria.'

\par 2 Cuando vi todas estas cosas, aquellos hombres me dijeron: «Enoc, hasta aquí nos ha sido ordenado viajar contigo», y aquellos hombres se alejaron de mí y entonces ya no los vi.

\par 3 Y me quedé solo al final del séptimo cielo y tuve miedo, caí de bruces y me dije a mí mismo: '¡Ay de mí! ¿Qué me ha sucedido?'

\par 4 Y el Señor envió a uno de sus gloriosos, el arcángel Gabriel, y me dijo: 'Ten valor, Enoc, no temas, levántate ante el rostro del Señor hacia la eternidad, levántate, ven conmigo'.

\par 5 Y yo le respondí y dije para mis adentros: «Señor mío, mi alma se ha apartado de mí, del terror y del temblor», y llamé a los hombres que me habían conducido hasta este lugar, en ellos me apoyé, y con ellos voy ante el rostro del Señor.

\par 6 Y Gabriel me tomó como a una hoja arrastrada por el viento, y me puso ante el rostro del Señor.

\par 7 Y vi el octavo cielo, que en hebreo se llama Muzaloth, el que cambia las estaciones, la sequía y la humedad, y los doce signos del zodíaco, que están sobre el séptimo cielo.

\par 8 Y vi el noveno Cielo, que en hebreo se llama Kuchavim, donde están los hogares celestiales de los doce signos del zodíaco.

\chapter{22}

\par \textit{En el décimo Cielo, el arcángel Miguel llevó a Enoc ante el rostro del Señor.}

\par 1 En el décimo cielo, Aravoth, vi la apariencia del rostro del Señor, como hierro que brilla en el fuego, y sacado, emitiendo chispas, y arde.

\par 2 Así vi el rostro del Señor, pero el rostro del Señor es inefable, maravilloso y muy terrible, y muy, muy terrible.

\par 3 ¿Y quién soy yo para hablar del ser inefable del Señor y de su maravilloso rostro? Y no puedo decir la cantidad de sus muchas instrucciones, y sus diversas voces, el trono del Señor muy grande y no hecho por manos, ni la cantidad de los que estaban a su alrededor, tropas de querubines y serafines, ni sus cantos incesantes, ni su inmutable belleza. ¿Y quién hablará de la inefable grandeza de su gloria?

\par 4 Y me postré y me incliné ante el Señor, y el Señor con sus labios me dijo:

\par 5 'Ten valor, Enoc, no temas, levántate y permanece ante mí en la eternidad.'

\par 6 Y el archistirato Miguel me levantó y me llevó ante el rostro del Señor.

\par 7 Y el Señor dijo a sus siervos que los tentaban: «Dejen que Enoc esté delante de mí por la eternidad», y los gloriosos se postraron ante el Señor y dijeron: «Deja ir a Enoc según tu palabra».

\par 8 Y el Señor dijo a Miguel: «Ve y quítate a Enoc de sus vestiduras terrenales, úngelo con mi dulce ungüento y vístelo con las vestiduras de mi gloria».

\par 9 Y Miguel hizo así, tal como el Señor le había dicho. Me ungió, y me vistió, y la apariencia de aquel ungüento es más que la gran luz, y su ungüento es como dulce rocío, y su olor suave, brillando como el rayo del sol, y yo me miré, y era como uno de sus gloriosos.

\par 10 Y el Señor llamó a uno de sus arcángeles llamado Pravuil, cuyo conocimiento era más rápido en sabiduría que los otros arcángeles, quienes escribieron todas las obras del Señor; y el Señor dijo a Pravuil:

\par 11 «Saca los libros de mis almacenes y una caña de escritura rápida, y dáselo a Enoc, y entrégale los libros escogidos y reconfortantes que tienes en tu mano».



\chapter{23}

\par \textit{De los escritos de Enoc, cómo escribió sus maravillosos viajes y las apariciones celestiales y él mismo escribió trescientos sesenta y seis libros.}

\par 1 Y me contaba todas las obras del cielo, de la tierra y del mar, y de todos los elementos, sus pasajes y salidas, y los truenos, el sol y la luna, las salidas y cambios de las estrellas, las estaciones. , los años, los días y las horas, las levantamientos del viento, el número de los ángeles y la formación de sus cánticos, y todas las cosas humanas, la lengua de todo canto y vida humana, los mandamientos, las instrucciones y las dulces voces. cantos y todo lo que conviene aprender.

\par 2 Y Pravuil me dijo: 'Todo lo que te he dicho, lo hemos escrito. Siéntate y escribe todas las almas de la humanidad, por muchas que nazcan, y los lugares preparados para ellas hasta la eternidad; porque todas las almas están preparadas para la eternidad, antes de la formación del mundo.'

\par 3 Y todo se duplicó durante treinta días y treinta noches, y escribí todo exactamente y escribí trescientos sesenta y seis libros.

\chapter{24}

\par \textit{De los grandes secretos de Dios, que Dios reveló y contó a Enoc, y habló con él cara a cara.}

\par 1 Y el Señor me llamó y me dijo: 'Enoc, siéntate a mi izquierda con Gabriel'.

\par 2 Y me incliné ante el Señor, y el Señor me habló: Enoc, amado, todo lo que ves, todo lo que está terminado, te lo digo incluso desde el principio, todo lo que creé desde el no ser. , y cosas visibles de invisibles.

\par 3 Escucha, Enoc, y acoge estas mis palabras, porque ni a Mis ángeles les he contado mi secreto, ni les he contado su ascenso, ni mi reino sin fin, ni han entendido mi creación, que te digo. hoy.

\par 4 Porque antes de que todas las cosas fueran visibles, yo era el único que andaba en las cosas invisibles, como el sol, de oriente a occidente y de occidente a oriente.

\par 5 Pero incluso el sol tiene paz en sí mismo, mientras que yo no encontré paz, porque estaba creando todas las cosas, y concebí el pensamiento de poner los cimientos y de crear la creación visible.

\chapter{25}

\par \textit{Dios le cuenta a Enoc cómo desde las más bajas tinieblas desciende lo visible y lo invisible.}

\par 1 ORDENÉ desde lo más bajo que las cosas visibles descendieran de las invisibles, y Adoil descendió muy grande, y lo miré, ¡y he aquí! tenía un vientre de gran luz.

\par 2 Y yo le dije: «Deshazte, Adoil, y deja que lo visible salga de ti».

\par 3 Y él se deshizo y salió una gran luz. Y yo estaba en medio de la gran luz, y como nace luz de la luz, surgió una gran edad, y mostró toda la creación que había pensado crear.

\par 4 Y vi que estaba bueno.

\par 5 Y me puse un trono, me senté en él y dije a la luz: «Sube más alto y ponte muy por encima del trono, y sé el fundamento de las cosas más elevadas».

\par 6 Y por encima de la luz no hay nada más, y entonces me incliné y miré desde mi trono.

\chapter{26}

\par \textit{Dios convoca desde lo más bajo por segunda vez para que salga Archas, pesado y muy rojo.}

\par 1 Y llamé por segunda vez al más bajo y le dije: «Que Archas salga con fuerza», y salió con fuerza de lo invisible.

\par 2 Y salió Archas, duro, pesado y muy rojo.

\par 3 Y dije: «Ábrete, Archas, y nazca de ti», y él se deshizo, surgió una era, muy grande y muy oscura, que llevó la creación de todas las cosas inferiores, y vi que estaba bien y le dijo:

\par 4 «Desciende abajo y hazte firme, y sé un fundamento para las cosas inferiores», y sucedió que él descendió y se arregló, y se convirtió en el fundamento de las cosas inferiores, y debajo de las tinieblas allí. no es nada más.



\chapter{27}

\par \textit{De cómo Dios fundó las aguas, las rodeó de luz y estableció en ellas siete islas.}

\par 1 Y ordené que se sacara de la luz y de las tinieblas, y dije: «Sé espeso», y quedó así y lo extendí con la luz, y se hizo agua, y lo extendí sobre las tinieblas, debajo de la luz, y entonces hice firmes las aguas, es decir el sin fondo, y hice fundamento de luz alrededor del agua, y creé siete círculos desde dentro, y la imaginé (es decir, el agua) como cristal mojado y seco, es decir como vidrio, y la circuncesión de las aguas y los demás elementos, y les mostré a cada uno de ellos su camino, y las siete estrellas cada una de ellas en su cielo, que iban así, y vi que era bueno.

\par 2 Y me separé entre la luz y las tinieblas, es decir, en medio del agua de aquí y de allá, y dije a la luz que sería el día, y a las tinieblas que sería el día. noche, y fue la tarde y fue la mañana el primer día.

\chapter{28}

\par \textit{La semana en la que Dios mostró a Enoc toda su sabiduría y poder, a lo largo de los siete días, cómo creó todas las fuerzas celestiales y terrenales y todas las cosas en movimiento, incluso hasta el hombre.}

\par 1 Y luego hice firme el círculo celestial, e hice que el agua inferior que está debajo del cielo se reuniera en un todo, y que el caos se secara, y así quedó.

\par 2 De las olas creé una roca dura y grande, y de la roca amontoné lo seco, y a lo seco lo llamé tierra, y al medio de la tierra lo llamé abismo, es decir, lo sin fondo, lo recogí. el mar en un solo lugar y lo uní con un yugo.

\par 3 Y dije al mar: «He aquí, te doy tus límites eternos y no te liberarás de tus componentes».

\par 4 Así afiancé el firmamento. Este día me llamé el primero creado.

\chapter{29}

\par \textit{Y llegó la tarde, y luego otra vez la mañana, y fue el segundo día. (El lunes es el primer día.) La Esencia ardiente.}

\par 1 Y para todas las tropas celestiales imaginé la imagen y la esencia del fuego, y mis ojos miraron la roca muy dura y firme, y del brillo de mis ojos el relámpago recibió su maravillosa naturaleza, que es a la vez fuego en agua. y agua en el fuego, y el uno no apaga al otro, ni el uno seca al otro, por eso el relámpago es más brillante que el sol, más blando que el agua y más firme que la dura roca.

\par 2 Y de la roca corté un gran fuego, y del fuego creé las órdenes de los diez ejércitos de ángeles incorpóreos, y sus armas son de fuego y sus vestidos una llama ardiente, y ordené que cada uno se pusiera de pie en su orden. Aquí Satanail con sus ángeles fue arrojado desde lo alto.

\par 3 Y uno del orden de los ángeles, habiéndose apartado del orden que estaba bajo su mando, concibió un pensamiento imposible: poner su trono más alto que las nubes sobre la tierra, para poder llegar a ser igual en rango a mi poder.

\par 4 Y lo arrojé desde lo alto con sus ángeles, y él volaba continuamente en el aire sobre el abismo.

\chapter{30}

\par \textit{Y luego creé todos los cielos, y fue el tercer día (martes)}

\par 1 Al tercer día ordené a la tierra que hiciera crecer árboles grandes y fructíferos, colinas y semillas para sembrar, y planté el Paraíso, lo cerqué y puse como guardianes ángeles llameantes armados, y así creé la renovación.

\par 2 Y llegó la tarde y la mañana del cuarto día.

\par 3 (miércoles). Al cuarto día ordené que hubiera grandes luces en los círculos celestiales.

\par 4 En el primer círculo superior puse las estrellas Kruno, en el segundo Afrodita, en el tercero Aris, en el quinto Zeus, en el sexto Ermis, en el séptimo la luna menor y lo adorné con las estrellas menores. .

\par 5 Y en la parte inferior puse el sol para iluminar el día, y la luna y las estrellas para iluminar la noche.

\par 6 El sol debía ir según cada animal (signos del zodíaco), doce, y designé la sucesión de los meses y sus nombres y vidas, sus truenos y sus marcas horarias, cómo debían tener éxito.

\par 7 Y llegó la tarde y la mañana del quinto día.

\par 8 (jueves). El quinto día ordené al mar que produjera peces y aves de muchas variedades, y todos los animales que se arrastran sobre la tierra, que avanzan sobre la tierra sobre cuatro patas y se elevan en el aire, sexo masculino y femenino. , y cada alma respirando el espíritu de vida.

\par 9 Y llegó la tarde y la mañana del día sexto.

\par 10 (viernes). El sexto día ordené a mi sabiduría que creara al hombre a partir de siete consistencias: una, su carne de la tierra; dos, su sangre del rocío; tres, sus ojos del sol; cuatro, sus huesos de piedra; cinco, su inteligencia de la rapidez de los ángeles y de las nubes; seis, sus venas y sus cabellos de la hierba de la tierra; siete, su alma de mi aliento y del viento.

\par 11 Y le di siete naturalezas: a la carne el oído, los ojos a la vista, al alma el olfato, las venas al tacto, la sangre al gusto, los huesos a la resistencia, y a la inteligencia la dulzura.

\par 12 Se me ocurrió un dicho astuto: Creé al hombre de la naturaleza invisible y de la visible, de ambas son su muerte, su vida y su imagen; él conoce la palabra como una cosa creada, pequeña en grandeza y nuevamente grande en pequeñez, y yo Lo puse en la tierra, un segundo ángel, honorable, grande y glorioso, y lo nombré gobernante para gobernar en la tierra y tener mi sabiduría, y no hubo nadie como él en la tierra de todas mis criaturas existentes.

\par 13 Y le puse un nombre, de las cuatro partes que lo componen, del este, del oeste, del sur y del norte, y le puse cuatro estrellas especiales, y le puse por nombre Adán, y le mostré los dos caminos. , la luz y la oscuridad, y le dije:

\par 14 «Esto es bueno y aquello es malo», para que sepa si me ama o me odia, para que quede claro quiénes de su raza me aman.

\par 15 Porque yo he visto su naturaleza, pero él no ha visto su propia naturaleza, por lo que al no ver pecará peor, y dije: 'Después del pecado, ¿qué hay sino la muerte?'

\par 16 Y le puse sueño y se durmió. Y tomé de él una costilla, y le creé mujer, para que la muerte le viniera por su mujer, y tomé su última palabra y la llamé madre, es decir, Eva.

\chapter{31}

\par \textit{Dios le da el paraíso a Adán y le da la orden de ver los cielos abiertos y de ver a los ángeles cantando el cántico de victoria.}

\par 1 ADÁN tiene vida en la tierra, y yo creé un jardín en el Edén, en el oriente, para que observara el pacto y guardara el mandamiento.

\par 2 Le abrí los cielos para que viera a los ángeles cantando el cántico de victoria y la luz tenebrosa.

\par 3 Y él estaba continuamente en el paraíso, y el diablo entendió que quería crear otro mundo, porque Adán era señor en la tierra, para gobernarla y controlarla.

\par 4 El diablo es el espíritu maligno de los lugares inferiores, como fugitivo hizo a Sotona de los cielos como su nombre era Satanail, así se volvió diferente de los ángeles, pero su naturaleza no cambió su inteligencia en cuanto a su entendimiento. de las cosas justas y pecaminosas.

\par 5 Y comprendió su condenación y el pecado que había cometido antes, por eso concibió un pensamiento contra Adán, de tal forma entró y sedujo a Eva, pero no tocó a Adán.

\par 6 Pero maldije la ignorancia, pero lo que antes había bendecido, no lo maldije, no maldije al hombre, ni a la tierra, ni a otras criaturas, sino a los malos frutos del hombre y a sus obras.

\chapter{32}

\par \textit{Después del pecado de Adán, Dios lo envía a la tierra 'de donde te saqué', pero no desea arruinarlo durante todos los años venideros.}

\par 1 Le dije: «Tierra eres, y a la tierra de donde te saqué irás, y no te arruinaré, sino que te enviaré a donde te saqué».

\par 2 ¡Entonces podré volver a llevarte en Mi segunda venida!

\par 3 Y bendije a todas mis criaturas visibles e invisibles. Y Adán estuvo cinco horas y media en el paraíso.

\par 4 Y bendije el séptimo día, que es el sábado, en el que descansó de todas sus obras.

\chapter{33}

\par \textit{Dios le muestra a Enoc la edad de este mundo, su existencia de siete mil años, y el octavo mil es el fin, ni años, ni meses, ni semanas, ni días.}

\par 1 Y también designé el octavo día, para que el octavo día fuera el primero en ser creado después de mi obra, y que los primeros siete giraran en forma de los séptimos mil, y que al comienzo de los octavos mil será un tiempo sin contar, sin fin, sin años ni meses ni semanas ni días ni horas.

\par 2 Y ahora, Enoc, todo lo que te he dicho, todo lo que has entendido, todo lo que has visto de las cosas celestiales, todo lo que has visto en la tierra y todo lo que he escrito en libros con mi gran sabiduría. , todas estas cosas las he ideado y creado desde el fundamento superior hasta el inferior y hasta el fin, y no hay consejero ni heredero de mis creaciones.

\par 3 Yo soy eterno, no hecho de manos y sin cambio.

\par 4 Mi pensamiento es mi consejero, mi sabiduría y mi palabra están hechas, y mis ojos observan todas las cosas cómo están aquí y tiemblan de terror.

\par 5 Si aparto mi rostro, entonces todo será destruido.

\par 6 Y aplica tu mente, Enoc, y conoce a quien te habla, y toma los libros que tú mismo has escrito.

\par 7 Y te daré a Samuel y a Raguil, quienes te guiaron arriba, y los libros, y bajaré a la tierra, y contaré a tus hijos todo lo que te he dicho, y todo lo que has visto, desde el cielo inferior hasta mi trono y todas las tropas.

\par 8 Porque yo creé todas las fuerzas, y no hay ninguna que se me resista o que no se sujete a mí. Porque todos se someten a mi monarquía y trabajan por mi único gobierno.

\par 9 Dadles los libros escritos a mano, y los leerán y me reconocerán como el creador de todas las cosas, y comprenderán que no hay otro Dios fuera de mí.

\par 10 Y que distribuyan los libros de tu escritura, de niños a niños, de generación en generación, de naciones a naciones.

\par 11 Y te daré a ti, Enoc, mi intercesor, el archistratege Miguel, por los escritos de tus padres Adán, Set, Enós, Cainán, Mahaleleel y Jared, tu padre.

\chapter{34}

\par \textit{Dios convence a los idólatras y a los fornicarios sodomitas, y por lo tanto hace caer sobre ellos un diluvio.}

\par 1 Han rechazado mis mandamientos y mi yugo, ha surgido una simiente inútil, que no teme a Dios y no se inclinan ante mí, sino que han comenzado a inclinarse ante dioses vanos, han negado mi unidad y han cargado la toda la tierra con mentiras, ofensas, abominables lujurias, es decir, de unos con otros, y toda clase de otras maldades inmundas, que es repugnante relatar.

\par 2 Por eso haré caer un diluvio sobre la tierra y destruiré a todos los hombres, y toda la tierra se desmoronará en una gran oscuridad.

\chapter{35}

\par \textit{Dios deja a un hombre justo de la tribu de Enoc con toda su casa, quien hizo la voluntad de Dios según su voluntad.}

\par 1 HE AQUÍ, de su descendencia surgirá otra generación mucho después, pero de ellos muchos serán muy insaciables.

\par 2 El que levante a esa generación les revelará los libros de tu letra, de tus padres, a aquellos a quienes debe señalar la custodia del mundo, a los hombres fieles y trabajadores de mi placer, que no reconocer mi nombre en vano.

\par 3 Y se lo contarán a la siguiente generación, y los demás que lo hayan leído serán más glorificados que los primeros.

\chapter{36}

\par \textit{Dios le ordenó a Enoc que viviera en la tierra treinta días, para dar instrucción a sus hijos y a los hijos de sus hijos. Después de treinta días fue llevado nuevamente al cielo.}

\par 1 AHORA, Enoc, te doy el plazo. de treinta días para estar en tu casa, y contarlo a tus hijos y a toda tu casa, para que todos oigan de mi presencia lo que tú les dices, para que lean y entiendan que no hay otro Dios fuera de mí.

\par 2 Y para que siempre guarden mis mandamientos y comiencen a leer y asimilar los libros escritos por ti.

\par 3 Y después de treinta días enviaré mi ángel por ti, y él te tomará de la tierra y de tus hijos para mí.

\chapter{37}

\par 1 Y el Señor llamó a uno de los ángeles mayores, terrible y amenazador, y lo puso junto a mí, de apariencia blanca como la nieve, y sus manos como hielo, que parecían una gran escarcha, y me heló la cara, porque No pude soportar el terror del Señor, como no se puede soportar el fuego de una estufa, ni el calor del sol, ni la escarcha del aire.

\par 2 Y el Señor me dijo: «Enoc, si tu rostro no está congelado aquí, ningún hombre podrá contemplar tu rostro».

\chapter{38}

\par \textit{Mathusal continuó teniendo esperanza y esperando a su padre Enoc en su lecho día y noche.}

\par 1 Y el Señor dijo a aquellos hombres que me llevaron primero: «Dejen que Enoc baje a la tierra con ustedes y espérenlo hasta el día determinado».

\par 2 Y me colocaron de noche en mi lecho.

\par 3 Y Mathusal, esperando mi venida, velando de día y de noche en mi lecho, se llenó de temor cuando oyó mi venida, y le dije: «Que se reúna toda mi casa, y les contaré todo».

\chapter{39}

\par \textit{La lastimera amonestación de Enoc a sus hijos con llanto y gran lamentación, mientras les hablaba.}

\par 1 Oh, hijos míos, amados míos, escuchad la amonestación de vuestro padre, en la medida en que sea conforme a la voluntad del Señor.

\par 2 Se me ha permitido ir a vosotros hoy y anunciaros, no de mis labios, sino de los labios del Señor, todo lo que es, lo que fue, todo lo que es ahora y todo lo que será hasta el día del juicio. .

\par 3 Porque el Señor me ha permitido ir a vosotros; oís, pues, las palabras de mis labios, de un hombre hecho grande para vosotros, pero yo soy uno que ha visto el rostro del Señor, como el hierro hecho brillar por el fuego que envía. salen chispas y quemaduras,

\par 4 Tú miras ahora mis ojos, los ojos de un hombre grande y lleno de significado para ti, pero yo he visto los ojos del Señor, brillando como los rayos del sol y llenando los ojos del hombre de asombro.

\par 5 Hijos míos, ahora veis la diestra de un hombre que os ayuda, pero yo he visto la diestra del Señor llenando el cielo mientras me ayudaba.

\par 6 Tú ves el ámbito de mi obra como el tuyo, pero yo he visto el ámbito ilimitado y perfecto del Señor, que no tiene fin.

\par 7 Oís las palabras de mis labios, como yo oí las palabras del Señor, como un gran trueno que se agita sin cesar en medio de las nubes.

\par 8 Y ahora, hijos míos, oíd los discursos del padre de la tierra: cuán terrible y terrible es presentarse ante la faz del gobernante de la tierra, cuánto más terrible y terrible es presentarse ante la faz del gobernante de la tierra. del gobernante del cielo, el controlador de vivos y muertos, y de las tropas celestiales. ¿Quién puede soportar ese dolor interminable?

\chapter{40}

\par \textit{Enoc amonesta a sus hijos verdaderamente sobre todas las cosas de labios del Señor, cómo vio, oyó y escribió.}

\par 1 Y ahora, hijos míos, yo sé todas las cosas, porque esto viene de los labios del Señor, y esto lo han visto mis ojos, desde el principio hasta el fin.

\par 2 Yo lo sé todo y lo he escrito todo en libros: los cielos y su fin, y su plenitud, y todos los ejércitos y sus marchas.

\par 3 He medido y descrito las estrellas, su gran multitud e incontable.

\par 4 ¿Quién ha visto sus revoluciones y sus entradas? Porque ni siquiera los ángeles ven su número, mientras yo he escrito todos sus nombres.

\par 5 Y medí el círculo del sol, y medí sus rayos, conté las horas, escribí también todo lo que pasa sobre la tierra, escribí lo que se alimenta, y toda semilla sembrada y no sembrada que la tierra produce y todas las plantas, y cada hierba y cada flor, y sus dulces olores, y sus nombres, y las moradas de las nubes, y su composición, y sus alas, y cómo soportan la lluvia y las gotas de lluvia.

\par 6 Y investigué todas las cosas, y escribí el camino del trueno y del relámpago, y ellos me mostraron las llaves y sus guardianes, su origen, el camino por el que van; se deja salir en medida (sc. suavemente) mediante una cadena, no sea que con una cadena pesada y con violencia arroje las nubes furiosas y destruya todas las cosas en la tierra.

\par 7 Escribí los tesoros de la nieve y los almacenes del aire frío y helado, y observé al guarda de las estaciones, que llena las nubes con ellos y no agota los tesoros. .

\par 8 Y escribí los lugares de reposo de los vientos y observé y vi cómo sus portallaves llevan balanzas y medidas; Primero las ponen en una báscula, luego en la otra las pesas y las extienden astutamente por medida sobre toda la tierra, no sea que con su fuerte respiración hagan que la tierra se balancee.

\par 9 Y medí toda la tierra, sus montañas y todas las colinas, campos, árboles, piedras, ríos, y escribí todo lo que existe, la altura desde la tierra hasta el séptimo cielo, y hacia abajo hasta el infierno más bajo, y el lugar del juicio, y el infierno muy grande, abierto y lloroso.

\par 10 Y vi cómo los prisioneros están sufriendo, esperando el juicio ilimitado.

\par 11 Y escribí a todos los que estaban siendo juzgados por el juez, y todos sus juicios (sentencias) y todas sus obras.

\chapter{41}

\par \textit{De cómo Enoc se lamentó del pecado de Adán.}

\par 1 Y vi a todos los antepasados ​​de todos los tiempos con Adán y Eva, y suspiré y rompí a llorar y dije de la ruina de su deshonra:

\par 2 «¡Ay de mí por mi enfermedad y por la de mis padres!», y pensé en mi corazón y dije:

\par 3 '¡Bienaventurado el hombre que no ha nacido o que ha nacido y no peca delante del Señor, ni viene a este lugar, ni trae el yugo de este lugar!

\chapter{42}

\par \textit{De cómo Enoc vio de pie a los poseedores de las llaves y a los guardias de las puertas del infierno.}

\par 1 Vi a los poseedores de las llaves y a los guardias de las puertas del infierno de pie, como grandes serpientes, y sus rostros como lámparas apagadas, y sus ojos de fuego, sus dientes afilados, y vi todas las obras del Señor, cómo son correctas. , mientras que las obras del hombre son unas buenas y otras malas, y en sus obras se conocen los que mienten mal.

\chapter{43}

\par \textit{Enoc muestra a sus hijos cómo midió y escribió los juicios de Dios.}

\par 1 Yo, hijos míos, medí y escribí cada obra, cada medida y cada juicio justo.

\par 2 Así como un año es más honorable que otro, así un hombre es más honorable que otro: unos por las grandes posesiones, otros por la sabiduría del corazón, otros por la inteligencia particular, otros por la astucia, unos por el silencio de labios, otros por la limpieza. , uno por la fuerza, otro por la hermosura, uno por la juventud, otro por el ingenio agudo, uno por la forma del cuerpo, otro por la sensibilidad, que se escuche en todas partes, pero no hay nadie mejor que el que teme a Dios, será más glorioso. en el futuro.

\chapter{44}

\par \textit{Enoc instruye a sus hijos a que no insulten el rostro del hombre, pequeño o grande.}

\par 1 El Señor creó al hombre con sus manos, a semejanza de su rostro, y lo hizo grande y pequeño.

\par 2 Cualquiera que insulte el rostro del gobernante y aborrezca el rostro del Señor, haya despreciado el rostro del Señor; y el que desahogue su ira contra un hombre sin causarle daño, la gran ira del Señor lo derribará; el que escupe en el rostro del hombre con reproche. , será talado en el gran juicio del Señor.

\par 3 Bienaventurado el hombre que no dirige su corazón con malicia contra nadie, y ayuda al herido y al condenado, y levanta al abatido, y hace caridad al necesitado, porque en el día del gran juicio todo peso , cada medida y cada peso serán como en el mercado, es decir, serán colgados en balanzas y puestos en el mercado, y cada uno aprenderá su propia medida, y según su medida recibirá su recompensa.



\chapter{45}

\par \textit{Dios muestra cómo no quiere de los hombres sacrificios ni holocaustos, sino corazones puros y contritos.}

\par 1 Quien se apresure a hacer una ofrenda ante el Señor, el Señor, por su parte, acelerará esa ofrenda concediéndole su trabajo.

\par 2 Pero quien enciende su lámpara ante el Señor y no juzga con sinceridad, el Señor no aumentará su tesoro en el reino de las alturas.

\par 3 Cuando el Señor exige pan, velas, carne (es decir, ganado), o cualquier otro sacrificio, entonces eso no es nada; pero Dios exige corazones puros, y con todo eso sólo prueba el corazón del hombre.

\chapter{46}

\par \textit{De cómo un gobernante terrenal no acepta del hombre regalos abominables e inmundos, cuánto más abomina Dios los regalos inmundos, sino que los despide con ira y no acepta sus regalos.}

\par 1 OÍD, pueblo mío, y escuchad las palabras de mis labios.

\par 2 Si alguien trae regalos a un gobernante terrenal y tiene pensamientos desleales en su corazón, y el gobernante lo sabe, ¿no se enojará con él, no rechazará sus regalos y no lo entregará al juicio?

\par 3 O si uno se hace pasar por bueno ante otro con engaño de lengua, pero tiene maldad en su corazón, ¿no comprenderá el otro la traición de su corazón y será condenado, ya que su falsedad era evidente para todos?

\par 4 Y cuando el Señor envíe una gran luz, habrá juicio para justos e injustos, y allí nadie pasará desapercibido.

\chapter{47}

\par \textit{Enoc instruye a sus hijos de labios de Dios y les entrega la letra de este libro.}

\par 1 Y ahora, hijos míos, reflexionad en vuestros corazones y prestad atención a las palabras de vuestro padre, que todas os han llegado de los labios del Señor.

\par 2 Toma estos libros escritos por tu padre y léelos.

\par 3 Porque los libros son muchos, y en ellos aprenderéis todas las obras del Señor, todo lo que ha sido desde el principio de la creación y lo que será hasta el fin de los tiempos.

\par 4 Y si guardan mi letra, no pecarán contra el Señor; porque no hay otro fuera del Señor, ni en el cielo, ni en la tierra, ni en los lugares más bajos, ni en el único fundamento.

\par 5 El Señor ha puesto los cimientos en lo desconocido y ha extendido los cielos visibles e invisibles; él fijó la tierra sobre las aguas, y creó innumerables criaturas, y quién ha contado el agua y el fundamento de los no fijados, o el polvo de la tierra, o la arena del mar, o las gotas de la lluvia, o la mañana ¿El rocío o el soplo del viento? ¿Quién ha llenado la tierra y el mar y el indisoluble invierno?

\par 6 Corté las estrellas del fuego, adorné el cielo y lo puse en medio de ellas.

\chapter{48}

\par \textit{Del paso del sol a lo largo de los siete círculos.}

\par 1 QUE el sol recorre los siete círculos celestes, que son el nombramiento de ciento ochenta y dos tronos, que se pone en un día corto, y nuevamente ciento ochenta y dos, que se pone en un día corto. un gran día, y tiene dos tronos sobre los cuales descansa, que giran de aquí para allá por encima de los tronos de los meses,

\par 2 Desde el día diecisiete del mes Tsivan desciende hasta el mes Thevan, desde el día diecisiete de Thevan sube.

\par 3 Y así se acerca a la tierra, entonces la tierra está y hace crecer su fruto, y cuando se aleja, entonces la tierra está triste, y los árboles y todos los frutos no tienen florecimiento.

\par 4 Todo esto lo midió con buena medida de horas, y con su sabiduría fijó la medida de lo visible y de lo invisible.

\par 5 De lo invisible hizo visibles todas las cosas, siendo él mismo invisible.

\par 6 Así os daré a conocer, hijos míos, y distribuiré los libros a vuestros hijos, en todas vuestras generaciones y entre las naciones que tendrán el sentido de temer a Dios, que los reciban y lleguen a amar. ellos más que cualquier alimento o dulce terrenal, y los leen y se aplican a ellos.

\par 7 Y a aquellos que no entienden al Señor, que no temen a Dios, que no los aceptan, sino que los rechazan, que no los reciben (cf. los libros), les espera un juicio terrible.

\par 8 Bienaventurado el hombre que llevará su yugo y los arrastrará, porque será liberado el día del gran juicio.

\chapter{49}

\par \textit{Enoc instruye a sus hijos a no jurar ni por el cielo ni por la tierra, y muestra la promesa de Dios, incluso en el vientre de la madre.}

\par 1 Os lo juro, hijos míos, pero no lo juro por ningún juramento, ni por el cielo ni por la tierra, ni por ninguna otra criatura que Dios creó.

\par 2 El Señor dijo: «No hay en mí juramento ni injusticia, sino verdad».

\par 3 Si no hay verdad en los hombres, que juren por las palabras «sí, sí», o «no, no».

\par 4 Y os juro, sí, sí, que no ha habido ningún hombre en el vientre de su madre, que ya antes, para cada uno haya un lugar preparado para el reposo del alma, y ​​una medida fijada para cada cual. Cuánto se pretende que un hombre sea probado en este mundo.

\par 5 Sí, hijos, no os engañéis, porque previamente ha sido preparado un lugar para cada alma humana.

\chapter{50}

\par \textit{De cómo ninguno de los nacidos en la tierra puede permanecer oculto ni su obra permanecer oculta, pero él (sc. Dios) nos pide que seamos mansos, que soportemos ataques e insultos, y que no ofendamos a las viudas y a los huérfanos.}

\par 1 HE puesto por escrito el trabajo de cada hombre y ninguno nacido en la tierra puede permanecer oculto ni sus obras permanecer ocultas.

\par 2 Veo todas las cosas.

\par 3 Ahora pues, hijos míos, pasad con paciencia y mansedumbre el número de vuestros días, para que heredéis la vida sin fin.

\par 4 Soportad por amor del Señor toda herida, toda injuria, toda mala palabra y ataque.

\par 5 Si te sobrevienen malas recompensas, no las devuelvas ni al prójimo ni al enemigo, porque el Señor te las devolverá y será tu vengador en el día del gran juicio, para que aquí no haya venganza entre los hombres.

\par 6 Quien de vosotros gaste oro o plata por su hermano, recibirá abundante tesoro en el mundo venidero.

\par 7 No hagáis daño a las viudas, ni a los huérfanos, ni a los extranjeros, para que no venga sobre vosotros la ira de Dios.

\chapter{51}

\par \textit{Enoc instruye a sus hijos a que no escondan tesoros en la tierra, sino que les ordena que den limosna a los pobres.}

\par 1 EXTIENDE tus manos a los pobres según tus fuerzas.

\par 2 No escondas tu plata en la tierra.

\par 3 Ayuda al hombre fiel en la aflicción, y la aflicción no te encontrará en el momento de tu angustia.

\par 4 Y todo yugo pesado y cruel que os sobrevenga, soportadlo todo por amor del Señor, y así encontraréis vuestra recompensa en el día del juicio.

\par 5 Es bueno entrar por la mañana, por el mediodía y por la tarde en la morada del Señor, para gloria de tu Creador.

\par 6 Porque todo ser que respira le glorifica, y toda criatura visible e invisible le devuelve alabanzas.

\chapter{52}

\par \textit{Dios instruye a sus fieles. cómo deben alabar su nombre.}

\par 1 BIENAVENTURADO el hombre que abre sus labios para alabar al Dios de los ejércitos y alaba al Señor con su corazón.

\par 2 Maldito todo aquel que abre sus labios para despreciar y calumniar a su prójimo, porque desprecia a Dios.

\par 3 Bienaventurado el que abre sus labios bendiciendo y alabando a Dios.

\par 4 Maldito el que abre sus labios para maldecir y insultar, delante del Señor todos los días de su vida.

\par 5 Bienaventurado el que bendice todas las obras del Señor.

\par 6 Maldito el que menosprecia la creación del Señor.

\par 7 Bienaventurado el que mira hacia abajo y levanta al caído.

\par 8 Maldito el que espera y anhela la destrucción de lo que no es suyo.

\par 9 Bienaventurado el que mantiene firmes los cimientos de sus padres desde el principio.

\par 10 Maldito el que pervierte los decretos de sus antepasados.

\par 11 Bienaventurado el que implanta la paz y el amor.

\par 12 Maldito el que perturba a los que aman a su prójimo.

\par 13 Bienaventurado el que habla a todos con lengua y corazón humildes.

\par 14 Maldito el que habla paz con los suyos. lengua, mientras que en su corazón no hay más paz que la espada.

\par 15 Porque todas estas cosas quedarán descubiertas en las balanzas y en los libros el día del gran juicio.

\chapter{53}

\par \textit{(No digamos: 'Nuestro padre está delante de Dios, él estará delante de nosotros en el día del juicio', porque ni el padre puede ayudar al hijo, ni el hijo al padre.)}

\par 1 Y ahora, hijos míos, no digáis: «Nuestro padre está delante de Dios y ora por nuestros pecados», porque no hay ayuda para ningún hombre que haya pecado.

\par 2 Ya ves cómo escribí todas las obras de cada hombre, antes de su creación, todo lo que se hace entre todos los hombres desde siempre, y nadie puede decir o relatar mi escritura, porque el Señor ve todas las imaginaciones del hombre, cómo son vanos, donde yacen en los tesoros del corazón.

\par 3 Ahora bien, hijos míos, fijaos bien en todas las palabras de vuestro padre que os digo, para que no os arrepintáis diciendo: «¿Por qué nuestro padre no nos lo dijo?»

\chapter{54}

\par \textit{Enoc instruye a sus hijos a que también entreguen los libros a otros.}

\par 1 En aquel tiempo, si no entendéis esto, dejad que estos libros que os he dado os sirvan como herencia de vuestra paz.

\par 2 Entrégaselos a todos los que los necesiten e instrúyelos para que vean las grandísimas y maravillosas obras del Señor.



\chapter{55}

\par \textit{Aquí Enoc muestra a sus hijos, diciéndoles entre lágrimas: 'Hijos míos, se ha acercado la hora de subir al cielo; he aquí, los ángeles están delante de mí.'}

\par 1 Hijitos míos, he aquí que el día de mi término y el tiempo se acercan.

\par 2 Porque los ángeles que irán conmigo están delante de mí y me instan a que me aparte de ti; están parados aquí en la tierra, esperando lo que se les ha dicho.

\par 3 Porque mañana subiré al cielo, a la Jerusalén más alta, a mi herencia eterna.

\par 4 Por eso te ordeno que cumplas delante del Señor todos sus beneplácitos.

\chapter{56}

\par \textit{Methosalam le pide a su padre la bendición, para que él (sc. Methosalam) pueda prepararle (sc. Enoc) comida para comer.}

\par 1 METHOSALAM, habiendo respondido a su padre Enoc, dijo: '¿Qué es agradable a tus ojos, padre, que pueda hacer delante de ti, para que puedas bendecir nuestras moradas y a tus hijos, y que tu pueblo sea glorioso a través de ti, y luego que puedas partir así, como dijo el Señor?,

\par 2 Enoc respondió a su hijo Metosalam y le dijo: «Oye, niño, desde el momento en que el Señor me ungió con el ungüento de su gloria, no ha habido alimento en mí, y mi alma no se acuerda de los placeres terrenales, ni tampoco ¡Quiero algo terrenal!

\chapter{57}

\par \textit{Enoc invitó a su hijo Metosalam. para convocar a todos sus hermanos.}

\par 1 Hijo mío Metosalam, llama a todos tus hermanos y a nuestra casa y a los ancianos del pueblo, para que pueda hablar con ellos y partir, como está planeado para mí.'

\par 2 Y Metosalam. se apresuró y convocó a sus hermanos, Regim, Riman, Uchan, Chermion, Gaidad y todos los ancianos del pueblo ante la presencia de su padre Enoc; y los bendijo, y les dijo:

\chapter{58}

\par \textit{Instrucción de Enoc a sus hijos.}

\par 1 Escúchenme, hijos míos, hoy.

\par 2 En aquellos días, cuando el Señor descendió a la tierra por causa de Adán y visitó a todas las criaturas que él mismo creó, después de todas estas creó a Adán, y el Señor llamó a todas las bestias de la tierra, a todos los reptiles. , y todas las aves que vuelan en el aire, y los trajo a todos delante de la faz de nuestro padre Adán.

\par 3 Y Adán dio nombres a todos los seres que viven en la tierra.

\par 4 Y el Señor lo nombró gobernante de todo, y sometió a él todas las cosas bajo sus manos, y las hizo mudas y embotadas para que sean mandadas por el hombre, y estén en sujeción y obediencia a él.

\par 5 Así también el Señor creó a cada hombre señor sobre todos sus bienes.

\par 6 El Señor no juzgará ni una sola alma de bestia por causa del hombre, sino que adjudica las almas de los hombres a sus bestias en este mundo; Para los hombres tienen un lugar especial.

\par 7 Y así como cada alma del hombre es conforme al número, así los animales no perecerán, ni todas las almas de los animales que el Señor creó, hasta el gran juicio, y acusarán al hombre si los alimenta mal.

\chapter{59}

\par \textit{Enoc instruye a sus hijos por qué no deben tocar la carne de res debido a lo que proviene de ella.}

\par 1 Quien contamina el alma de las bestias, contamina su propia alma.

\par 2 Porque el hombre trae animales limpios para ofrecer sacrificios por el pecado y así curar su alma.

\par 3 Y si ofrecen para el sacrificio animales limpios y aves, el hombre tiene cura, cura su alma.

\par 4 Todo te es dado para comer, átalo por los cuatro pies, eso es para hacer buena la curación, él cura su alma.

\par 5 Pero quien mata bestias sin heridas, mata su propia alma y contamina su propia carne.

\par 6 Y el que hace cualquier daño a un animal en secreto, es una mala práctica y contamina su propia alma.

\chapter{60}

\par \textit{El que hace daño al alma del hombre, hace daño a su propia alma, y ​​no hay cura para su carne, ni perdón para siempre. Cómo no conviene matar al hombre ni con armas ni con lengua.}

\par 1 El que mata el alma del hombre, mata su propia alma y mata su propio cuerpo, y no hay cura para él para siempre.

\par 2 El que pone a un hombre en cualquier trampa, él mismo caerá en ella, y no habrá cura para él para siempre.

\par 3 El que pone a un hombre en cualquier recipiente, su retribución no faltará en el gran juicio para siempre.

\par 4 El que obra perversamente o habla mal de alguien, no hará justicia para sí mismo para siempre.

\chapter{61}

\par \textit{Enoc instruye a sus hijos a mantenerse alejados de la injusticia y a menudo extender sus manos a los pobres para darles una parte de sus trabajos.}

\par 1 Y ahora, hijos míos, guardad vuestro corazón de toda injusticia que el Señor aborrece. Así como un hombre pide (sc. algo) a Dios para su propia alma, así haga con toda alma viviente, porque yo sé todas las cosas, cómo en el gran tiempo (sc. por venir) están preparadas muchas moradas para los hombres, buenos por los buenos, y malos por los malos, sin número muchos.

\par 2 Bienaventurados los que entran en las casas buenas, porque en las malas (es decir, casas) no hay paz ni retorno (es decir, de ellas).

\par 3 ¡Oíd, hijos míos, pequeños y grandes! Cuando el hombre pone un buen pensamiento en su corazón, trae regalos de sus trabajos ante el rostro del Señor y sus manos no los hicieron, entonces el Señor apartará su rostro del trabajo de sus manos, y él (sc. el hombre) no podrá encontrar el trabajo de sus manos.

\par 4 Y si sus manos lo lograron, pero su corazón murmura y su corazón no deja de murmurar sin cesar, no tiene ninguna ventaja.

\chapter{62}

\par \textit{De cómo es apropiado traer el don con fe, porque no hay arrepentimiento después de la muerte.}

\par 1 BIENVENIDO el hombre que con paciencia presenta con fe sus dones ante el rostro del Señor, porque encontrará el perdón de los pecados.

\par 2 Pero si se retracta de sus palabras antes de tiempo, no habrá arrepentimiento para él; y si pasa el tiempo y no hace por su propia voluntad lo prometido, no hay arrepentimiento después de la muerte.

\par 3 Porque toda obra que el hombre hace antes de tiempo, es engaño ante los hombres y pecado ante Dios.



\chapter{63}

\par \textit{De cómo no despreciar a los pobres, sino compartir con ellos igualmente, para que no seas murmurado delante de Dios.}

\par 1 CUANDO el hombre vista al desnudo y sacie al hambriento, encontrará recompensa de Dios.

\par 2 Pero si su corazón murmura, comete un doble mal: ruina de sí mismo y de lo que da; y para él no habrá recompensa por ello.

\par 3 Y si su propio corazón está lleno de su comida y su propia carne (es decir, vestida) de su ropa, comete desprecio y pierde toda su resistencia a la pobreza, y no encontrará recompensa por sus buenas obras.

\par 4 Todo hombre orgulloso y magnilocuente es aborrecible para el Señor, y toda palabra mentirosa, revestida de falsedad; será cortado con el filo de la espada de la muerte, y arrojado al fuego, y arderá para siempre.'

\chapter{64}

\par \textit{De cómo el Señor llama a Enoc, y la gente tomó consejo de ir a besarlo al lugar llamado Achuzan.}

\par 1 CUANDO Enoc había hablado estas palabras a sus hijos, todos, lejos y cerca, oyeron cómo el Señor llamaba a Enoc. Tomaron consejo juntos:

\par 2 'Vamos a besar a Enoc' y dos mil hombres se reunieron y llegaron al lugar de Achuzan donde estaba Enoc y sus hijos.

\par 3 Y los ancianos del pueblo, toda la asamblea, se acercaron, se inclinaron y comenzaron a besar a Enoc y le dijeron:

\par 4 'Padre nuestro, Enoc, seas bendito del Señor, el gobernante eterno, y ahora bendice a tus hijos y a todo el pueblo, para que seamos glorificados hoy ante ti.

\par 5 Porque serás glorificado delante del Señor para siempre, ya que el Señor te eligió a ti, antes que a todos los hombres de la tierra, y te designó escritor de toda su creación, visible e invisible, y redentor de los pecados del hombre. y ayudante de tu casa.

\chapter{65}

\par \textit{De la instrucción de Enoc a sus hijos.}

\par 1 Y Enoc respondió a todo su pueblo diciendo: 'Escuchen, hijos míos, antes de que todas las criaturas fueran creadas, el Señor creó las cosas visibles e invisibles.

\par 2 Y durante todo el tiempo que hubo y pasó, comprended que después creó al hombre a semejanza de su propia forma, y ​​puso en él ojos para ver, oídos para oír, corazón para reflexionar y entendimiento con el que para deliberar.

\par 3 Y el Señor vio todas las obras del hombre, creó todas sus criaturas y dividió el tiempo: del tiempo fijó los años, de los años fijó los meses, de los meses fijó los días y de los días designado siete.

\par 4 Y en ellos dispuso las horas, las midió exactamente, para que el hombre pudiera reflexionar sobre el tiempo y contar los años, los meses y las horas, su alternancia, su principio y su fin, y poder contar su propia vida, desde el principio. comenzando hasta la muerte, y reflexiona sobre su pecado y escribe su obra mala y buena; porque ninguna obra está oculta delante del Señor, para que cada uno conozca sus obras y nunca transgreda todos sus mandamientos, y guarde mi letra de generación en generación.

\par 5 Cuando termine toda la creación visible e invisible, tal como la creó el Señor, entonces cada hombre irá al gran juicio, y entonces perecerá todo el tiempo y los años, y de ahí en adelante no habrá meses ni días ni horas. , quedarán pegados y no se contarán.

\par 6 Habrá un eón, y todos los justos que escaparán del gran juicio del Señor, serán reunidos en el gran eón, porque los justos comenzarán el gran eón, y vivirán eternamente, y entonces también habrá entre ellos ni trabajo, ni enfermedad, ni humillación, ni ansiedad, ni necesidad, ni violencia, ni noche, ni oscuridad, sino. buena luz.

\par 7 Y tendrán un gran muro indestructible y un paraíso resplandeciente e incorruptible, porque todo lo corruptible pasará y habrá vida eterna.

\chapter{66}

\par \textit{Enoc instruye a sus hijos y a todos los ancianos del pueblo, cómo deben caminar con terror y temblando delante del Señor, y servirle solo a él y no postrarse ante los ídolos, sino ante Dios, quien creó los cielos y la tierra. y. toda criatura, y a su imagen.}

\par 1 Y ahora, hijos míos, guardad vuestras almas de toda injusticia que el Señor aborrece.

\par 2 Caminad delante de él con terror y temblor y servidle solo a él.

\par 3 Inclínate ante el Dios verdadero, no ante los ídolos mudos, sino inclínate ante su imagen y presenta todas las ofrendas justas ante el rostro del Señor. El Señor odia lo que es injusto.

\par 4 Porque el Señor ve todas las cosas; cuando el hombre piensa en su corazón, entonces aconseja a los intelectos, y cada pensamiento está siempre delante del Señor, quien afirmó la tierra y puso sobre ella todas las criaturas.

\par 5 Si miras al cielo, allí está el Señor; Si piensas en lo profundo del mar y en todo lo que hay debajo de la tierra, el Señor está allí.

\par 6 Porque el Señor creó todas las cosas. No os inclinéis ante las cosas hechas por el hombre, abandonando al Señor de toda la creación, porque ninguna obra puede permanecer oculta ante el rostro del Señor.

\par 7 Caminad, hijos míos, en la paciencia, en la mansedumbre, en la honestidad, en la provocación, en el dolor, en la fe y en la verdad, en la confianza en las promesas, en la enfermedad, en los abusos, en las heridas, en la tentación, en la desnudez, en las privaciones. , amándonos unos a otros, hasta que salgáis de esta época de males, para que os convirtáis en herederos del tiempo sin fin.

\par 8 Bienaventurados los justos que escaparán del gran juicio, porque brillarán siete veces más que el sol, porque en este mundo le han quitado a todo la séptima parte: la luz, las tinieblas, el alimento, el goce, el dolor, el paraíso, tortura, fuego, escarcha y otras cosas; lo puso todo por escrito, para que pudieras leerlo y entenderlo.'

\chapter{67}

\par \textit{El Señor hizo salir tinieblas sobre la tierra y cubrió al pueblo y a Enoc, y fue elevado a lo alto, y la luz vino otra vez en el cielo.}

\par 1 CUANDO Enoc había hablado con el pueblo, el Señor envió oscuridad sobre la tierra, y hubo oscuridad, y cubrió a aquellos hombres que estaban con Enoc, y llevaron a Enoc a lo más alto del cielo, donde está el Señor. ; y lo recibió y lo puso delante de su rostro, y las tinieblas se fueron de la tierra, y volvió la luz.

\par 2 Y el pueblo vio y no entendió cómo habían sido apresados ​​Enoc, y glorificaron a Dios, y encontraron un rollo en el que estaba escrito «el Dios invisible»; y todos se fueron a sus casas.

\chapter{68}

\par 1 ENOC nació el sexto día del mes Tsivan y vivió trescientos sesenta y cinco años.

\par 2 Fue llevado al cielo el primer día del mes Tsivan y permaneció en el cielo sesenta días.

\par 3 Escribió todos estos signos de toda la creación que el Señor creó, y escribió trescientos sesenta y seis libros, los entregó a sus hijos y permaneció en la tierra treinta días, y fue llevado de nuevo al cielo en el sexto día del mes Tsivan, el mismo día y hora en que nació.

\par 4 Así como la naturaleza de cada hombre en esta vida es oscura, también lo son su concepción, nacimiento y salida de esta vida.

\par 5 A qué hora fue concebido, a esa hora nació y a esa hora también murió.

\par 6 Metosalam y sus hermanos, todos los hijos de Enoc, se apresuraron y erigieron un altar en el lugar llamado Achuzan, desde donde Enoc había sido llevado al cielo.

\par 7 Tomaron bueyes para el sacrificio, convocaron a todo el pueblo y sacrificaron el sacrificio delante del Señor.

\par 8 Todo el pueblo, los ancianos del pueblo y toda la asamblea vinieron a la fiesta y llevaron regalos a los hijos de Enoc.

\par 9 Y celebraron un gran banquete, regocijándose y regocijándose durante tres días, alabando a Dios, que les había dado tal señal a través de Enoc, que había hallado favor en él, y que la transmitirían a sus hijos de generación en generación. generación, de edad en edad.

\par 10 Amén.



\end{document}