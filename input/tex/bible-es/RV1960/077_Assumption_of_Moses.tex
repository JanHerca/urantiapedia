\begin{document}

\title{Asunción de Moisés}

\chapter{1}

\par 1 El testamento de Moisés, las cosas que ordenó en el año ciento veinte de su vida,
\par 2 que es el año dos mil quinientos desde la creación del mundo:
\par 3 [Pero según el cálculo oriental, el dos mil setecientos y el cuatrocientos después de la salida de Fenicia],
\par 4 Cuando el pueblo partió después del éxodo que hizo Moisés a Ammán, al otro lado del Jordán,
\par 5 en la profecía que hizo Moisés en el libro de Deuteronomio:
\par 6 y llamó a Josué hijo de Nun, varón aprobado por el Señor,
\par 7 para ser ministro del pueblo y del tabernáculo del testimonio con todas sus cosas santas,
\par 8 y para llevar al pueblo a la tierra dada a sus padres,
\par 9 para que se les diera conforme al pacto y al juramento que Él pronunció en el tabernáculo de dárselo por medio de Josué, diciendo a Josué estas palabras:
\par 10 «(Sé fuerte) y valiente para hacer con tus fuerzas todo lo que se te ha ordenado, para que seas irreprochable ante Dios».
\par 11 Así dice el Señor del mundo.
\par 12 Porque Él creó el mundo para su pueblo.
\par 13 Pero no le agradó manifestar este propósito de la creación desde la fundación del mundo, para que así los gentiles fueran condenados, ni siquiera para su propia humillación, con sus argumentos, convencerse unos a otros.
\par 14 Por eso Él me diseñó e ideó y me preparó desde la fundación del mundo para que fuera mediador de su pacto.
\par 15 Y ahora os declaro que el tiempo de los años de mi vida se ha cumplido y que voy a pasar a dormir con mis padres, incluso en presencia de todo el pueblo.
\par 16 Y recibe esta escritura para que sepas cómo conservar los libros que te entregaré:
\par 17 Y los ordenarás, los ungirás con aceite de cedro y los guardarás en vasijas de barro en el lugar que Él hizo desde el principio de la creación del mundo.
\par 18 para que se invoque su nombre hasta el día del arrepentimiento en la visita con que el Señor los visitará en la consumación de los últimos días.

\chapter{2}

\par 1 Y ahora, por medio de vosotros, irán a la tierra que Él determinó y prometió dar a sus padres,
\par 2 en el cual los bendecirás y les darás individualmente, les confirmarás su herencia en mí y les establecerás el reino, y les nombrarás magistrados locales según la buena voluntad de su Señor en juicio y justicia.
\par 3 Y cinco años después de su entrada en la tierra, serán gobernados por jefes y reyes durante dieciocho años, y durante diecinueve años las diez tribus se separarán.
\par 4 Y las doce tribus descenderán y trasladarán el tabernáculo del testimonio. Entonces el Dios del cielo hará el atrio de su tabernáculo y la torre de su santuario, y serán establecidas las dos tribus santas:
\par 5 pero las diez tribus establecerán para sí reinos según sus propias ordenanzas.
\par 6 Y ofrecerán sacrificios durante veinte años:
\par 7 Y siete atrincherarán los muros, y yo protegeré a nueve, pero cuatro transgredirán el pacto del Señor y profanarán el juramento que el Señor hizo con ellos.
\par 8 Y sacrificarán a sus hijos a dioses extraños, y levantarán ídolos en el santuario para adorarlos.
\par 9 Y en la casa del Señor practicarán la impiedad y grabarán toda forma de bestia y muchas abominaciones.

\chapter{3}

\par 1 En aquellos días vendrá contra ellos un rey del este y su caballería cubrirá su tierra.
\par 2 Y quemará al fuego su colonia junto con el templo santo del Señor, y se llevará todos los vasos sagrados.
\par 3 Y expulsará a todo el pueblo y los llevará a la tierra de su nacimiento, y también llevará consigo a las dos tribus.
\par 4 Entonces las dos tribus convocarán a las diez tribus y, hambrientas y sedientas, marcharán como leonas por las llanuras polvorientas.
\par 5 Y gritarán en voz alta: «Justo y santo es el Señor, porque por cuanto vosotros habéis pecado, también nosotros hemos sido arrastrados con vosotros, junto con nuestros hijos».
\par 6 Entonces las diez tribus se lamentarán al oír los reproches de las dos tribus,
\par 7 y dirán: «¿Qué os hemos hecho, hermanos, no ha venido esta tribulación sobre toda la casa de Israel?»
\par 8 Y todas las tribus se lamentarán, clamando al cielo y diciendo:
\par 9 Dios de Abraham, Dios de Isaac y Dios de Jacob, recuerda el pacto que hiciste con ellos y el juramento que les hiciste por ti mismo, de que nunca faltará su descendencia de la tierra que les diste.»
\par 10 Entonces se acordarán de mí y dirán aquel día, tribu por tribu y cada uno con su prójimo:
\par 11 ¿No es esto lo que nos anunció entonces Moisés en profecías, cuando padeció muchas cosas en Egipto, en el mar Rojo y en el desierto durante cuarenta años?
\par 12 ¿Y ciertamente llamó al cielo y a la tierra por testigos contra nosotros, para que no traspasemos sus mandamientos, en los cuales él fue mediador con nosotros?
\par 13 «He aquí, estas cosas nos han sucedido después de su muerte, según su declaración, como él nos declaró en aquel momento; sí, he aquí, estas cosas han sucedido incluso hasta que fuimos llevados cautivos a la tierra del oriente».
\par 14 Los cuales también estarán en servidumbre durante unos setenta y siete años.

\chapter{4}

\par 1 Entonces entrará uno que está sobre ellos, extenderá sus manos, se arrodillará y orará por ellos diciendo:
\par 2 «Señor de todo, Rey en el trono sublime, que gobiernas el mundo, y quisiste que este pueblo fuera Tu pueblo elegido, entonces (de hecho) quisiste que tú fueras llamado su Dios, según el pacto que hiciste con sus padres».
\par 3 «Y sin embargo, han ido en cautiverio a otra tierra con sus mujeres y sus hijos, y alrededor de las puertas de pueblos extraños y donde hay gran vanidad».
\par 4 «Míralos y ten compasión de ellos, oh Señor del cielo».
\par 5 Entonces Dios se acordará de ellos a causa del pacto que hizo con sus padres y también en aquellos tiempos manifestará su compasión.
\par 6 Y Él pondrá en la mente de un rey tener compasión de ellos, y los enviará a su tierra y a su país.
\par 7 Entonces algunas partes de las tribus subirán y llegarán al lugar designado y volverán a rodear el lugar con muros.
\par 8 Y las dos tribus continuarán en la fe prescrita, tristes y lamentándose porque no podrán ofrecer sacrificios al Señor de sus padres. Y las diez tribus crecerán y se multiplicarán entre los gentiles durante el tiempo de su cautiverio.

\chapter{5}

\par 1 Y cuando se acerquen los tiempos del castigo y surja la venganza por parte de los reyes que comparten su culpa y los castigan,
\par 2 También ellos estarán divididos en cuanto a la verdad.
\par 3 Por eso se ha dicho: «Se apartarán de la justicia y se acercarán a la iniquidad, y contaminarán con impurezas la casa de su culto», y [porque] «se prostituirán con dioses extraños».
\par 4 Porque no seguirán la verdad de Dios, sino que algunos, que no son sacerdotes sino esclavos, hijos de esclavos, contaminarán el altar con los (mismos) presentes que ofrecen al Señor.
\par 5 Y muchos en aquellos tiempos respetarán a las personas deseables y recibirán regalos y pervertirán el juicio [al recibir regalos].
\par 6 Y por esta razón la colonia y los límites de su habitación se llenarán de actos ilícitos e iniquidades: los que se apartan perversamente del Señor serán jueces: estarán dispuestos a juzgar por dinero como cada uno desee.

\chapter{6}

\par 1 Entonces se les levantarán reyes que gobernarán y se llamarán sacerdotes del Dios Altísimo; ciertamente harán iniquidad en el lugar santísimo.
\par 2 Y los sucederá un rey insolente, que no será del linaje de los sacerdotes, un hombre audaz y desvergonzado, que los juzgará como merecen.
\par 3 Y cortará a espada a sus principales y los destruirá en lugares secretos, para que nadie sepa dónde están sus cuerpos.
\par 4 Matará a viejos y a jóvenes, y no perdonará.
\par 5 Entonces el temor de él será amargo para ellos en su tierra.
\par 6 Y ejecutará contra ellos juicios como los egipcios ejecutaron contra ellos, durante treinta y cuatro años, y los castigará.
\par 7 Y engendrará hijos, quienes en su lugar gobernarán por períodos más cortos.
\par 8 A sus regiones vendrán cohortes y un poderoso rey del oeste que las conquistará.
\par 9 y los tomará cautivos, quemará al fuego una parte de su templo y crucificará a algunos alrededor de su colonia.

\chapter{7}

\par 1 Y cuando esto se haga, los tiempos terminarán, en un momento terminará el (segundo) curso, llegarán las cuatro horas.
\par 2 Serán forzados. . . .
\par 3 Y en el tiempo de estos gobernarán hombres destructivos e impíos, diciendo que son justos.
\par 4 Y éstos despertarán el veneno de sus mentes, siendo hombres traidores, complacientes, hipócritas en todos sus propios asuntos y amantes de los banquetes a cualquier hora del día, glotones, golosos.
\par 5 . . .
\par 6 Devoradores de los bienes de los (pobres) diciendo que lo hacen en base a su justicia,
\par 7 pero en realidad para destruirlos, quejosos, engañosos, ocultándose para no ser descubiertos, impíos, llenos de maldad e iniquidad desde el sol hasta el ocaso:
\par 8 diciendo: «Tendremos festines y lujos, comeremos y beberemos, y nos consideraremos príncipes».
\par 9 Y aunque sus manos y sus mentes toquen cosas inmundas, su boca hablará grandes cosas y dirán además:
\par 10 «No me toques, no sea que me contamines en el lugar (donde estoy). .»

\chapter{8}

\par 1 Y vendrá sobre ellos una segunda visita y una ira, como no les ha sucedido desde el principio hasta entonces, en la cual despertará contra ellos al rey de los reyes de la tierra, y al que gobierna con gran poder, que crucificará a los que confiesan su circuncisión:
\par 2 y a los que lo oculten los torturará y los entregará para que sean atados y conducidos a la cárcel.
\par 3 Y sus esposas serán entregadas a los dioses entre los gentiles, y sus hijos pequeños serán operados por los médicos para que les salga el prepucio.
\par 4 Y otros entre ellos serán castigados con torturas, fuego y espada, y serán obligados a llevar en público sus ídolos, contaminados como están como quienes los guardan.
\par 5 Y de la misma manera, aquellos que los torturan los obligarán a entrar en lo más íntimo de su santuario, y los obligarán con aguijones a blasfemar con insolencia la palabra, finalmente, después de estas cosas, las leyes y lo que tenían sobre su altar.

\chapter{9}

\par 1 En aquel día habrá un hombre de la tribu de Leví, cuyo nombre será Taxo, que, teniendo siete hijos, les hablará y les exhortará:
\par 2 «Mirad, hijos míos, que una segunda visita implacable (e) inmunda ha sobrevenido al pueblo, y un castigo despiadado y muy superior al primero».
\par 3 «¿Qué nación o qué región o qué pueblo de aquellos que son impíos hacia el Señor, que han cometido muchas abominaciones, han sufrido calamidades tan grandes como las que nos han sucedido a nosotros?»
\par 4 «Ahora pues, hijos míos, escúchenme; pues observen y sepan que ni los padres ni sus antepasados ​​tentaron a Dios para transgredir sus mandamientos».
\par 5 «Y sabes que esta es nuestra fuerza, y así lo haremos».
\par 6 «Ayunemos por tres días y al cuarto vayamos a una cueva que está en el campo, y muramos antes que transgredir los mandamientos del Señor de Señores, el Dios de nuestros padres. »
\par 7 «Porque si hacemos esto y morimos, nuestra sangre será vengada delante del Señor».

\chapter{10}

\par 1 Y entonces Su reino aparecerá en toda Su creación, y entonces Satanás ya no existirá, y el dolor se irá con él.
\par 2 Entonces se llenarán las manos del ángel que ha sido designado jefe, y al instante los vengará de sus enemigos.
\par 3 Porque el Celestial se levantará de su trono real, y saldrá de su santa morada con indignación e ira a causa de sus hijos.
\par 4 Y la tierra temblará, se estremecerá hasta sus confines, y las altas montañas se abatirán y las colinas se estremecerán y caerán.
\par 5 Y los cuernos del sol se romperán y él se convertirá en tinieblas; Y la luna no dará su luz, y se convertirá toda en sangre. Y el círculo de las estrellas será perturbado.
\par 6 Y el mar se retirará al abismo, y las fuentes de las aguas se acabarán, y los ríos se secarán.
\par 7 Porque se levantará el Altísimo, el Dios eterno, y aparecerá para castigar a los gentiles y destruirá todos sus ídolos.
\par 8 Entonces tú, oh Israel, serás feliz, y te subirás sobre el cuello y las alas del águila, y se acabarán.
\par 9 Y Dios os exaltará y os acercará al cielo de las estrellas, al lugar donde habitan.
\par 10 Y mirarás desde lo alto y verás a tus enemigos en Ge(henna), y los reconocerás y te alegrarás, y darás gracias y confesarás a tu Creador.
\par 11 Y vosotros; Josué (el hijo de) Nun, guarda estas palabras y este libro;
\par 12 Porque desde mi muerte [asunción] hasta Su venida habrá 250 tiempos [= año-semanas = 1750 años].
\par 13 Y este es el curso de los tiempos que seguirán hasta su consumación.
\par 14 Y me iré a dormir con mis padres.
\par 15 Por tanto, Josué, hijo de Nun, sé fuerte y valiente; (porque) Dios te ha elegido para ser ministro en el mismo pacto.

\chapter{11}

\par 1 Y cuando Josué escuchó las palabras de Moisés, que estaban escritas en su escritura todo lo que había dicho antes, rasgó sus vestidos y se arrojó a los pies de Moisés.
\par 2 Y Moisés lo consoló y lloró con él.
\par 3 Y Josué le respondió y dijo:
\par 4 «¿Por qué me consuelas, (mi) señor Moisés? ¿Y cómo podré consolarme de la amarga palabra que has hablado, que ha salido de tu boca, llena de lágrimas y de lamento, al alejarte de este pueblo?
\par 5 «(Pero ahora) ¿qué lugar os recibirá?»
\par 6 «¿O cuál será la señal que señalará (tu) sepulcro?»
\par 7 ¿O quién se atreverá a trasladar tu cuerpo de allí como el de un simple hombre de un lugar a otro?
\par 8 »Porque todos los hombres, cuando mueren, tienen según su edad sus sepulcros en la tierra; pero tu sepulcro es desde el nacimiento hasta el ocaso, y desde el sur hasta los confines del norte: todo el mundo es tu sepulcro».
\par 9 «Mi señor, tú te vas, ¿y quién alimentará a este pueblo?»
\par 10 ¿O quién tendrá compasión de ellos y quién los guiará en el camino?
\par 11 «¿O quién orará por ellos sin perder un solo día, para que yo pueda conducirlos a la tierra de sus antepasados?»
\par 12 «¿Cómo, pues, voy a acoger a este pueblo como a un padre (su) hijo único, o como a una amante su hija, una virgen que se prepara para ser entregada al marido a quien ella reverenciará, mientras ella guarda su persona? del sol y (cuida) que sus pies no queden descalzos para correr por el suelo».
\par 13 «¿(Y cómo) les daré comida y bebida según el placer de su voluntad?»
\par 14 «De ellos habrá 600.000 (hombres), porque estos se han multiplicado hasta este punto a través de tus oraciones, (mi) señor Moisés».
\par 15 «¿Y qué sabiduría o entendimiento tengo yo para juzgar o responder con palabras en la casa (del Señor)?»
\par 16 »Y también los reyes de los amorreos, cuando oyen que los atacamos, creyendo que ya no está entre ellos el espíritu santo que era digno del Señor, múltiple e incomprensible, el señor de la palabra, que era fiel en todo, el principal profeta de Dios en toda la tierra, el maestro más perfecto del mundo, [que ya no está entre ellos], dirá: 'Vamos contra ellos'».
\par 17 »Si el enemigo ha obrado impíamente contra su Señor una sola vez, no tiene abogado que ore por ellos al Señor, como Moisés, el gran mensajero, que a cada hora del día y de la noche tenía sus rodillas pegadas a la tierra, orando y pidiendo ayuda al que gobierna todo el mundo con compasión y justicia, recordándole el pacto de los padres y propiciando al Señor con el juramento».
\par 18 »Porque dirán: 'Él no está con ellos; vayamos, pues, y acabemos con ellos de la faz de la tierra'. ¿Qué será entonces de este pueblo, mi señor Moisés?

\chapter{12}

\par 1 Y cuando Josué hubo terminado estas palabras, se arrojó de nuevo a los pies de Moisés.
\par 2 Entonces Moisés tomó su mano, lo sentó en el asiento delante de él y respondió y le dijo:
\par 3 «Josué, no te desprecies; pero tranquilízate y escucha mis palabras».
\par 4 Dios ha creado a todas las naciones que hay en la tierra y a nosotros, a ellas y a nosotros las ha previsto desde el principio de la creación de la tierra hasta el fin de los tiempos, y nada ha sido descuidado por Él, ni siquiera hasta el fin mínimo, sino que Él ha previsto todas las cosas y ha hecho que todas se realicen».
\par 5 «(Sí) todas las cosas que sucederán en esta tierra, el Señor las ha previsto y, mira, han sido presentadas (a la luz). . .»
\par 6 «(El Señor) me ha designado en su nombre para (orar) por sus pecados e (interceder) por ellos».
\par 7 «Porque no por alguna virtud o fuerza mía, sino por su beneplácito, me han tocado en suerte su compasión y su paciencia».
\par 8 «Porque te digo, Josué, que no por la piedad de este pueblo extirparás las naciones».
\par 9 «Las luces de los cielos y los cimientos de la tierra han sido hechos y aprobados por Dios y están bajo el anillo de su diestra».
\par 10 «Por lo tanto, aquellos que hacen y cumplen los mandamientos de Dios crecerán y serán prosperados:»
\par 11 «pero aquellos que pecan y desprecian los mandamientos quedarán sin las bendiciones antes mencionadas, y serán castigados con muchos tormentos por las naciones».
\par 12 «Pero no está permitido extirparlos y destruirlos por completo».
\par 13 Porque saldrá Dios, que ha previsto todas las cosas para siempre, y ha establecido su pacto y por el juramento que hizo. . .»

\end{document}