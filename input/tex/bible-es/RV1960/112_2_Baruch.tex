\begin{document}

\title{II Baruc}

\chapter{1}

\par \textit{Anuncio de la próxima destrucción de Jerusalén a Baruc}

\par 1 Y aconteció que en el año veinticinco de Jeconías, rey de Judá, vino palabra del Señor a Baruc hijo de Nerías, y le dijo:

\par 2 «¿Has visto todo lo que este pueblo me está haciendo, que los males que estas dos tribus que quedaron han hecho son mayores que los de las diez tribus que fueron llevadas cautivas?

\par 3 Porque las tribus anteriores fueron obligadas por sus reyes a cometer pecado, pero estas dos mismas han obligado y obligado a sus reyes a cometer pecado.

\par 4 Por esta razón, he aquí, traigo mal sobre esta ciudad y sobre sus habitantes, y será quitada de mi presencia por un tiempo, y esparciré a este pueblo entre los gentiles para que hagan bien a los gentiles. Y Mi pueblo será castigado, y llegará el tiempo en que buscarán la prosperidad de sus tiempos.»

\chapter{2}

\par 1 «Porque te he dicho estas cosas para que puedas ordenar a Jeremías y a todos los que son como tú que se retiren de esta ciudad.

\par 2 Porque tus obras son para esta ciudad como columna firme,

\par Y vuestras oraciones como un muro fuerte.»

\chapter{3}

\par 1 Y dije: «Oh Señor, Señor mío, ¿he venido al mundo para ver los males de mi madre? No (así) mi Señor.

\par 2 Si he hallado gracia ante tus ojos, toma primero mi espíritu para ir a mis padres y no ver la destrucción de mi madre.

\par 3 Porque dos cosas me constriñen fuertemente: porque no puedo resistirte, y además mi alma no puede contemplar los males de mi madre.

\par 4 Pero una cosa diré en tu presencia, oh Señor.

\par 5 ¿Qué, pues, habrá después de estas cosas? Porque si destruyes tu ciudad y entregas tu tierra a los que nos odian, ¿cómo volverá a ser recordado el nombre de Israel?

\par 6 ¿O cómo se hablará de tus alabanzas? ¿O a quién se le explicará lo que está en tu ley? ¿O el mundo volverá a su naturaleza de antaño y la época volverá al silencio primitivo? ¿Y será quitada la multitud de las almas, y la naturaleza del hombre no volverá a ser nombrada? ¿Y dónde está todo lo que dijiste acerca de nosotros?

\chapter{4}

\par 1 Y el Señor me dijo:

\par «Esta ciudad será entregada por un tiempo,

\par Y el pueblo será castigado durante un tiempo,

\par Y el mundo no caerá en el olvido.

\par \textit{La Jerusalén celestial}

\par 2 [¿Crees que ésta es aquella ciudad de la cual dije: «En las palmas de mis manos te tengo grabada»?

\par 3 Este edificio que ahora se construye entre vosotros no es el que me ha sido revelado, el que preparé aquí de antemano desde el momento en que decidí construir el Paraíso y mostré a Adán antes de que pecara, pero cuando transgredió el mandamiento, fue quitado de él, como también el Paraíso.

\par 4 Y después de esto se lo mostré a mi siervo Abraham, de noche, entre las porciones de las víctimas.

\par 5 Y también se lo mostré a Moisés en el monte Sinaí, cuando le mostré la figura del tabernáculo y todos sus utensilios.

\par 6 Y ahora, he aquí, está guardado conmigo como el Paraíso.

\par 7 Ve, pues, y haz lo que yo te mando.»]

\chapter{5}

\par \textit{La queja de Baruc y la tranquilidad de Dios}

\par 1 Y respondí y dije:

\par «Así que estoy destinado a llorar por Sión,

\par Porque tus enemigos vendrán a este lugar y contaminarán tu santuario,

\par y llevar vuestra herencia en cautiverio,

\par y hacerse dueños de aquellos a quienes has amado,

\par Y volverán al lugar de sus ídolos,

\par y se jactará delante de ellos:

\par ¿Y qué harás por tu gran nombre?»

\par 2 Y el Señor me dijo:

\par «Mi nombre y Mi gloria son por toda la eternidad;

\par Y mi juicio mantendrá su derecho a su debido tiempo.

\par 3 Y verás con tus ojos

\par que el enemigo no destruirá a Sión,

\par Ni quemarán a Jerusalén,

\par Pero sed por ahora ministros del Juez.

\par 4 Pero id y haced todo lo que os he dicho.»


\par 5 Fui y tomé a Jeremías, a Adu, a Serías, a Jabish, a Gedalías y a todos los hombres honorables del pueblo, y los conduje al valle de Cedrón, y les conté todo lo que había sido dicho.

\par 6 Y alzaron su voz y todos lloraron.

\par 7 Y nos sentamos allí y ayunamos hasta la noche.

\chapter{6}

\par \textit{Invasión de los caldeos y su entrada a la ciudad después de que los vasos sagrados fueron escondidos y los muros de la ciudad derribados por los ángeles}

\par 1 Y aconteció al día siguiente que, ¡he aquí! El ejército de los caldeos rodeó la ciudad, y al caer la tarde, yo, Baruc, dejé al pueblo, salí y me puse junto a la encina.

\par 2 Y me entristecí por Sion y me lamenté por el cautiverio que había caído sobre el pueblo.

\par 3 ¡Y he aquí! De repente, un espíritu fuerte me levantó y me llevó por encima del muro de Jerusalén.

\par 4 Y miré, ¡y he aquí! cuatro ángeles que estaban en las cuatro esquinas de la ciudad, cada uno de ellos con una antorcha de fuego en sus manos.

\par 5 Y otro ángel comenzó a descender del cielo y les dijo: «Guardad vuestras lámparas y no las encendáis hasta que yo os lo diga.

\par 6 Porque soy el primero en enviar una palabra a la tierra y a poner en ella lo que el Señor Altísimo me ha ordenado.»

\par 7 Y lo vi descender al Lugar Santísimo y tomar de allí el velo, el arca santa, el propiciatorio, las dos mesas, las vestiduras sagradas de los sacerdotes y el altar del incienso, y las cuarenta y ocho piedras preciosas con que se adornaba el sacerdote y todos los utensilios santos del tabernáculo.

\par 8 Y habló a la tierra en alta voz:

\par «Tierra, tierra, tierra, oye la palabra del Dios fuerte,

\par y recibe lo que te he prometido,

\par y guárdalos hasta el último tiempo,

\par Para que, cuando te lo ordenen, puedas restaurarlos,

\par Para que los extraños no se apoderen de ellos.

\par 9 Porque viene el tiempo en que también Jerusalén será liberada por un tiempo,

\par Hasta que se diga que será restaurado nuevamente para siempre.»

\par 10 Y la tierra abrió su boca y se los tragó.

\chapter{7}

\par 1 Y después de esto oí a aquel ángel decir a los ángeles que sostenían las lámparas: «Destruid, pues, y derribad su muro hasta sus cimientos, para que el enemigo no se jacte y diga:

\par «Hemos derribado el muro de Sión,

\par Y hemos quemado el lugar del Dios fuerte.»»

\par 2 Y se han apoderado del lugar donde yo estaba antes.

\chapter{8}

\par 1 Los ángeles hicieron lo que él les había ordenado, y cuando rompieron las esquinas de los muros, se oyó una voz desde el interior del templo, después de que el muro había caído, que decía:

\par 2 «Entrad, enemigos,

\par Y venid, adversarios;

\par Porque el que guardaba la casa la ha abandonado.»

\par 3 Y yo, Baruc, partí.

\par 4 Y aconteció después de esto que el ejército de los caldeos entró y se apoderó de la casa y de todo lo que la rodeaba. Y llevaron cautivo al pueblo, y mataron a algunos de ellos, y ataron al rey Sedequías, y lo enviaron al rey de Babilonia.

\chapter{9}

\par \textit{Primer Ayuno de siete días: Baruc permanecerá en medio de las ruinas de Jerusalén y Jeremías acompañará a los exiliados a Babilonia. Endecha de Baruc sobre Jerusalén}

\par 1 Y llegué yo, Baruc, y Jeremías, cuyo corazón estaba limpio de pecados y que no había sido capturado en la toma de la ciudad.

\par 2 Y rasgamos nuestras vestiduras, lloramos, nos lamentamos y ayunamos siete días.

\chapter{10}

\par 1 Y aconteció que después de siete días vino a mí la palabra de Dios, y me dijo:

\par 2 «Dile a Jeremías que vaya y apoye el cautiverio del pueblo en Babilonia. Pero si permanecéis aquí en medio de la desolación de Sión, os mostraré después de estos días »lo que sucederá al final de los días». Y dije a Jeremías como el Señor me había mandado. Y él, en verdad, partió con el pueblo, pero yo, Baruc, volví y me senté ante las puertas del templo, y lamenté sobre Sion la siguiente lamentación y dije:

\par 3 [...]

\par 4 [...]

\par 5 [...]

\par 6 Bienaventurado el que no nació,

\par O el que, habiendo nacido, ha muerto.

\par 7 Pero nosotros, los que vivimos, ¡ay de nosotros!

\par Porque vemos las aflicciones de Sión,

\par Y lo que le ha sucedido a Jerusalén.

\par 8 Llamaré a las sirenas del mar,

\par Y tú, Lilin, vienes del desierto,

\par Y vosotros, Shedim y dragones de los bosques:

\par Despiertad y ceñios vuestros lomos de luto,

\par y lleva conmigo las endechas,

\par Y haz lamentación conmigo.

\par 9 Labradores, no volváis a sembrar;

\par Y, oh tierra, ¿por qué darte los frutos de tu cosecha?

\par Guarda dentro de ti los dulces de tu sustento.

\par 10 Y tú, vid, ¿por qué das más tu vino?

\par Porque desde allí en Sion no se volverá a hacer ofrenda,

\par Tampoco se ofrecerán más primicias.

\par 11 Y vosotros, oh cielos, retened vuestro rocío,

\par Y no abras los tesoros de la lluvia:

\par 12 Y tú, oh sol, retiene la luz de tus rayos.

\par Y tú, oh luna, apaga la multitud de tu luz;

\par ¿Por qué la luz debería volver a surgir?

\par ¿Dónde se oscurece la luz de Sión?

\par 13 Y vosotros, esposos, no entréis,

\par Y que las novias no se adornen con guirnaldas;

\par Y vosotras mujeres, no oréis para poder dar a luz.

\par 14 Porque los estériles se alegrarán sobre todo,

\par Y los que no tienen hijos se alegrarán,

\par Y los que tengan hijos tendrán angustia.

\par 15 ¿Por qué deberían soportar el dolor?

\par ¿Sólo para enterrar en el dolor?

\par 16 ¿O por qué la humanidad debería tener hijos?

\par ¿O por qué debería volver a nombrarse la semilla de su especie?

\par Donde esta madre está desolada,

\par ¿Y sus hijos son llevados en cautiverio?

\par 17 Desde ahora no hables de hermosura,

\par Y no hables de gracia.

\par 18 Además, vosotros, los sacerdotes, tomad las llaves del santuario,

\par y arrojarlos a lo alto del cielo,

\par Y dáselos al Señor y dile:

\par «Guarda tú mismo tu casa,

\par ¡Por lo! somos encontrados falsos mayordomos».

\par 19 Y vosotras, vírgenes; que tejen lino fino

\par y seda con oro de Ofir,

\par Toma con prisa todas (estas) cosas

\par y arrojarlos al fuego,

\par para llevarlos al que los hizo,

\par Y la llama los envía al que los creó,

\par No sea que el enemigo se apodere de ellos.»

\chapter{11}

\par 1 Además, yo, Baruc, digo esto contra ti, Babilonia:

\par «Si hubieras prosperado,

\par Y Sión habitaba en su gloria,

\par Sin embargo, el dolor para nosotros había sido grande

\par Que seáis iguales a Sión.

\par 2 ¡Pero ahora, he aquí! el dolor es infinito,

\par Y el lamento inmensurable,

\par ¡Por lo! eres prosperado

\par Y Sión desolada.

\par 3 ¿Quién juzgará estas cosas?

\par ¿O ante quién nos quejaremos de lo que nos ha sucedido?

\par Oh Señor, ¿cómo lo has soportado?

\par 4 Nuestros padres descansaron sin pena,

\par ¡Y he aquí! los justos duermen en la tierra en tranquilidad;

\par 5 Porque no conocían esta angustia,

\par Todavía no sabían lo que nos había sucedido.

\par 6 ¡Ojalá tuvieras oídos, oh tierra!

\par Y que tenías corazón, oh polvo:

\par para que puedas ir y anunciar en el Seol,

\par Y di a los muertos:

\par 7 «Bienaventurados vosotros más que nosotros los que vivimos.»»

\chapter{12}

\par \textit{OXY = FRAGMENTO GRIEGO DE OXIRHINCO, de Grenfell y Hunt's Oxyrhynchus Papyri, vol. III. 3-7, 1903. Verso.}

\par 1 [Pero diré esto como pienso.] [OXY: Pero diré esto como pienso,]

\par [Y hablaré contra ti, tierra que aún prospera.] [OXY: Y hablaré contra ti, tierra que prospera.]

\par 2 [El mediodía no siempre arde.] [OXY: No siempre arde el mediodía,]

\par [Ni los rayos del sol dan luz constantemente.] [OXY: Ni los rayos del sol dan luz constantemente.]

\par 3 [No esperes que la Tierra tenga la esperanza de que siempre serás próspero y regocijado.] [OXY: Y no esperes regocijarte,]

\par [Y no os envanezcáis ni os jactéis mucho.] [OXY: Ni condenéis mucho.]

\par 4 [Porque seguramente a su debido tiempo se despertará la ira (divina) contra ti.] [OXY: Porque seguramente a su debido tiempo se despertará la ira (divina) contra ti,]

\par [El cual ahora con gran paciencia está retenido como por riendas.] [OXY: Que ahora está retenido por gran paciencia como por riendas.]

\par 5 [Y habiendo dicho estas cosas, ayuné siete días.] [OXY: Y habiendo dicho estas cosas, ayuné siete días.]


\chapter{13}

\par \textit{OXY = FRAGMENTO GRIEGO DE OXIRHINCO, de Grenfell y Hunt's Oxyrhynchus Papyri, vol. III. 3-7, 1903. Verso.}

\par \textit{Segundo ayuno. Revelación sobre el juicio venidero sobre los paganos.}

\par 1 [Y aconteció después de estas cosas, que yo, Baruc, estaba de pie sobre el monte Sión, y ¡he aquí! vino una voz desde lo alto y me dijo:] [OXY: Y aconteció después de estas cosas que yo, Baruc, estaba parado sobre el monte Sión, y he aquí una voz salió desde lo alto y me dijo:]

\par 2 [«Ponte de pie, Baruc, y escucha la palabra del Dios fuerte.»] [OXY: «Ponte de pie, Baruc, y escucha la palabra del Dios fuerte.»]

\par 3 Porque os habéis asombrado de lo que le ha sucedido a Sión, por eso seréis preservados hasta el fin de los tiempos, para que seáis un testimonio.

\par 4 De modo que, si alguna vez esas ciudades prósperas dicen:

\par 5 «¿Por qué el Dios fuerte ha traído sobre nosotros esta retribución?» Diles a ellos, a ti y a aquellos como tú que habrán visto este mal: '(Este es el mal) y la retribución que vendrá sobre ti y sobre tu pueblo en su tiempo (destinado) para que las naciones sean completamente destruidas.

\par 6 Y entonces estarán angustiados.

\par 7 Y si dicen en aquel tiempo:

\par 8 ¿Por cuánto tiempo? les dirás:

\par «Vosotros, los que habéis bebido el vino colado,

\par Bebed también de sus heces,

\par El juicio del Altísimo

\par Que no respeta a las personas.»»

\par 9 Por eso antes no tuvo misericordia de sus propios hijos,

\par sino que los afligió como a sus enemigos, porque pecaron,

\par 10 Entonces fueron castigados

\par Para que sean santificados.

\par 11 [Pero ahora, pueblos y naciones, sois culpables] [OXY: (Vosotros) pueblos y . . .]

\par [Porque siempre habéis hollado la tierra,] [OXY: (Vosotros) habéis hollado la tierra]

\par [Y usó la creación injustamente.] [OXY: Y usó mal las cosas creadas en ella.]

\par 12 [Porque siempre te he beneficiado.] [OXY: Porque siempre fuiste beneficiado]

\par [Y siempre has sido desagradecido por la beneficencia.] [OXY: Pero siempre fuiste desagradecido.]

\chapter{14}

\par \textit{La justicia de los justos no les ha aprovechado ni a ellos ni a su ciudad; Los Juicios de Dios son incomprensibles; el Mundo fue hecho para los Justos, pero ellos pasan y el Mundo permanece (14). Respuesta: El hombre conoce los juicios de Dios y ha pecado voluntariamente. Este mundo es un cansancio para los justos, pero el siguiente es suyo (15), que debe ganarse mediante el carácter, ya sea que el tiempo de un hombre aquí sea largo o corto (16-17). Bien o mal final: la pregunta suprema (18—19).}

\par 1 [Y yo respondí y dije: «¡Mira! me has mostrado el método de los tiempos, y lo que será después de estas cosas, y me has dicho que la retribución de la que has hablado vendrá sobre las naciones.] [OXY: Y yo Respondió y dijo: »He aquí, tú me has mostrado los métodos de los tiempos, y lo que será. Y me has dicho que la retribución de la que hablaste será soportada por las naciones.]

\par 2 [Y ahora sé que son muchos los que pecaron, y vivieron en prosperidad', y se alejaron del mundo, pero en aquellos tiempos quedarán pocas naciones a quienes les dirá aquellas palabras que tú dijo.] [OXY: Y ahora sé que los que han pecado son muchos, y han vivido. . . , y partieron del mundo, pero que pocas naciones quedarán en aquellos tiempos a quienes . . . las palabras (que) dijiste.]

\par 3 [Porque ¿qué ventaja hay en esto, o qué (mal), peor que lo que hemos visto nos sucede, debemos esperar ver?] [OXY: ¿Y qué ventaja (hay) en esto o algo peor (que estos?)]

\par 4 Pero otra vez hablaré en tu presencia:

\par 5 ¿De qué aprovecharon los que tuvieron conocimiento antes que tú y no anduvieron en vanidad como las demás naciones, ni dijeron a los muertos: «Danos vida», sino que siempre te temieron y no se apartaron de tus caminos? ?

\par 6 ¡Y he aquí! Han sido raptados, y por ellos no has tenido misericordia de Sion.

\par 7 Y si otros hacían el mal, a Sion le correspondía ser perdonada por las obras de los que hacían buenas obras, y no ser abrumada por las obras de los que hacían injusticia.

\par 8 Pero ¿quién, oh Señor, mi Señor, comprenderá tu juicio?

\par ¿O quién buscará la profundidad de Tu camino?

\par ¿O quién calculará el peso de tu camino?

\par 9 ¿O quién podrá pensar en tus incomprensibles consejos?

\par ¿O quién de los que nacen ha encontrado alguna vez?

\par ¿El principio o el fin de tu sabiduría?


\par 10 Porque todos hemos sido creados como un soplo.

\par 11 Porque así como el aliento sube involuntariamente y luego muere, así también ocurre con la naturaleza de los hombres, que no se alejan según su propia voluntad y no saben lo que les sucederá al final.

\par 12 Porque los justos esperan con justicia el fin y sin temor se alejan de esta morada, porque tienen consigo un tesoro de obras guardadas en tesoros.

\par 13 Por esta razón también estos abandonan este mundo sin temor, y confiados con alegría esperan recibir el mundo que les has prometido.

\par 14 Pero nosotros, ¡ay de nosotros, que ahora también somos avergonzados y esperamos el mal!

\par 15 Pero tú sabes exactamente lo que has hecho por medio de tus siervos; porque no podemos entender lo bueno como eres tú, nuestro Creador.

\par 16 Pero volveré a hablar en tu presencia, oh SEÑOR, mi Señor.

\par 17 Cuando en la antigüedad no existía el mundo con sus habitantes, tú ideabas y hablabas con una palabra, y al instante las obras de la creación se presentaron ante ti.

\par 18 Y dijiste que harías para tu mundo al hombre como administrador de tus obras, para que se supiera que él no fue hecho para el mundo, sino el mundo para él.

\par 19 Y ahora veo que en cuanto al mundo que fue hecho por causa nuestra, ¡he aquí! permanece; pero nosotros, por causa de quien fue hecho, partimos.»

\chapter{15}

\par 1 Y el Señor respondió y me dijo: «Con razón estás asombrado por la partida del hombre, pero no has juzgado bien los males que acontecen a los que pecan.

\par 2 Y en cuanto a lo que has dicho, que los justos son raptados y los impíos prosperan,

\par 3 Y en cuanto a lo que habéis dicho: «El hombre no conoce vuestro juicio», por eso oíd, y yo os hablaré, y escuchad, y os haré oír mis palabras.

\par 4 [...]

\par 5 El hombre no habría entendido correctamente mi juicio, si no hubiera aceptado la ley y yo le hubiera instruido en el entendimiento.

\par 6 Pero ahora, por haber transgredido intencionadamente, sí, precisamente por el hecho de que lo sabe, será atormentado.

\par 7 Y en cuanto a lo que dijiste acerca de los justos, que por causa de ellos ha llegado este mundo, así también vendrá por causa de ellos el que está por venir.

\par 8 Porque este mundo es para ellos una contienda y un trabajo lleno de dificultades; y lo que está por venir, una corona de gran gloria.»

\chapter{16}

\par 1 Y yo respondí y dije: «¡Oh Señor, Señor mío, he aquí! Los años de este tiempo son pocos y malos, ¿y quién podrá en su poco tiempo adquirir lo que no se puede medir?»

\chapter{17}

\par 1 Y el Señor respondió y me dijo: «Con el Altísimo no se tiene en cuenta el tiempo ni algunos años.

\par 2 ¿De qué le aprovechó a Adán vivir novecientos treinta años y transgredir lo que se le había ordenado? Por tanto, la multitud de tiempo que vivió no le aprovechó, sino que trajo muerte y cortó los años de los que de él nacieron. ¿En qué sufrió pérdida Moisés porque vivió sólo ciento veinte años y, estando sujeto a Aquel que lo formó, trajo la ley a la descendencia de Jacob y encendió una lámpara para la nación de Israel?

\chapter{18}

\par 1 Y respondí y dije: «El que alumbra ha tomado de la luz, y son pocos los que lo han imitado. Pero aquellos a quienes él alumbró, los sacaron de las tinieblas de Adán y no se regocijaron a la luz de la lámpara.»

\chapter{19}

\par 1 Y él respondió y me dijo: «Por lo tanto, en aquel momento les estableció un pacto y dijo:

\par «He aquí, os he puesto delante la vida y la muerte»,

\par Y llamó al cielo y a la tierra por testigos contra ellos.

\par 2 Porque sabía que le quedaba poco tiempo,

\par Pero que el cielo y la tierra perduran para siempre.

\par 3 Pero después de su muerte pecaron y transgredieron,

\par Aunque sabían que tenían la ley que los reprendía,

\par Y la luz en la que nada podía errar,

\par También las esferas que testifican, y Yo.

\par 4 Ahora bien, sobre todo lo que es, soy yo quien juzga, pero no reflexiones en tu alma sobre estas cosas, ni te aflijas por lo que ha sido.

\par 5 Porque ahora lo que se debe considerar es el fin del tiempo, ya sea de los negocios, o de la prosperidad, o de la vergüenza, y no su comienzo.

\par 6 Porque si un hombre prospera en sus comienzos y es humillado en su vejez, olvida toda la prosperidad que tuvo.

\par 7 Y además, si un hombre es avergonzado al principio, y al final prospera, no volverá a recordar su mala súplica.

\par 8 Y escuchad de nuevo: aunque cada uno hubiera prosperado todo ese tiempo, todo el tiempo desde el día en que se decretó la muerte contra los transgresores, y al final fuera destruido, todo habría sido en vano.»

\chapter{20}

\par \textit{Sión ha sido quitada para acelerar el Advenimiento del Juicio}

\par 1 «¡Por tanto, he aquí! llegan los días,

\par Y los tiempos se acelerarán más que los primeros,

\par Y las estaciones se acelerarán más que las pasadas,

\par Y los años pasarán más rápido que el presente (años).

\par 2 Por eso he quitado ahora a Sión,

\par Para que pueda visitar más rápidamente el mundo en su tiempo.

\par 3 Ahora pues, retén en tu corazón todo lo que te mando,

\par y séllalo en lo más recóndito de tu mente.

\par 4 Y entonces os mostraré el juicio de mi poder,

\par nd Mis caminos que son inescrutables.

\par 5 Ve, pues, y santifícate durante siete días, y no comas pan, ni bebas agua, ni hables con nadie.

\par 6 Y luego vendré a ese lugar y me revelaré a vosotros, y os hablaré verdades, y os daré mandamientos sobre el método de los tiempos; porque vienen y no se demoran.

\chapter{21}

\par \textit{Ayuno de siete días: La oración de Baruc: La respuesta de Dios}

\par \textit{La oración de Baruc hijo de Nerías.}

\par 1 Fui allí y me senté en el valle de Cedrón, en una cueva de la tierra, y allí santifiqué mi alma, y ​​no comí pan, ni tuve hambre, ni bebí agua, ni tuve sed, y estuve allí hasta el séptimo día, como él me había mandado.

\par 2 Y después llegué al lugar donde Él había hablado conmigo.

\par 3 Y aconteció que al ponerse el sol, mi alma pensó mucho y comencé a hablar en presencia del Poderoso, y dije:

\par 4 «Oh vosotros que habéis hecho la tierra, oídme, que habéis fijado el firmamento con la palabra, y que habéis afirmado la altura del cielo con el espíritu, que habéis llamado desde el principio del mundo lo que no todavía existen y te obedecen.

\par 5 tú que has dominado el aire con tu cabeza, y has visto lo que ha de ser como lo que estás haciendo.

\par 6 Tú, que gobiernas con gran pensamiento los ejércitos que están delante de ti; también gobiernas con indignación a los innumerables seres santos que tú hiciste desde el principio, de llama y de fuego, que están alrededor de tu trono.

\par 7 Sólo a ti te corresponde el hecho de que debes hacer inmediatamente lo que desees.

\par 8 Quien hace que las gotas de lluvia caigan en número sobre la tierra, y es el único que conoce la consumación de los tiempos antes de que lleguen; tened respeto a mi oración. Para

\par 9 sólo tú puedes sustentar a todos los que existen, a los que han fallecido, a los que serán, a los que pecan y a los que son demasiado justos. Porque sólo tú vives inmortal e indescifrable, y conoces el número de la humanidad. Y si con el tiempo muchos pecaron, otros no pocos fueron justos.»

\par \textit{La depreciación de esta vida por parte de Baruc.}

\par 10 [...]

\par 11 [...]

\par 12 vosotros sabéis dónde guardar el fin de los que han pecado, o la consumación de los que han sido justos.

\par 13 Porque si sólo existiera esta vida, que es de todos los hombres, nada podría ser más amargo que esto.

\par 14 ¿De qué sirve la fuerza si se convierte en enfermedad?

\par O la saciación de alimento se convierte en hambre,

\par O la belleza que se convierte en fealdad.

\par 15 Porque la naturaleza del hombre es siempre cambiante.

\par 16 Pues lo que éramos antes ahora ya no lo somos, y lo que ahora somos no lo seremos después.

\par 17 Porque si no se hubiera preparado para todos la consumación, en vano habría sido su comienzo. Pero de todo lo que viene de ti me informas, y de todo lo que te pregunto, me iluminas.

\par \textit{Baruc ora a Dios para acelerar el Juicio y cumplir Su Promesa}

\par 18 [...]

\par 19 ¿Hasta cuándo permanecerá lo corruptible, y hasta cuándo prosperará el tiempo de los mortales, y hasta cuándo los que transgreden en el mundo serán contaminados con mucha maldad?

\par 20 Manda, pues, con misericordia y cumple todo lo que dijiste que harías, para que tu poder sea notorio a aquellos que piensan que tu paciencia es debilidad.

\par 21 Y demuestra a los que no lo saben, que todo lo que nos ha sucedido a nosotros y a nuestra ciudad hasta ahora ha sido según la paciencia de tu poder, porque por causa de tu nombre nos has llamado pueblo amado.

\par 22 Poned fin, pues, de aquí en adelante a la mortalidad.

\par 23 Y reprende en consecuencia al ángel de la muerte, y que se manifieste tu gloria, y que se conozca el poder de tu hermosura, y que se selle el Seol, para que desde ahora en adelante no reciba a los muertos, y que los tesoros de las almas restauran lo que está encerrado en ellas.

\par 24 Porque ha habido muchos años como aquellos que están asolados desde los días de Abraham, Isaac y Jacob, y de todos los que son como ellos, que duermen en la tierra, por cuya causa dijiste que habías creado el mundo.

\par 25 Ahora, pues, muestra pronto tu gloria y no postergues lo que has prometido.

\par 26 Y cuando terminé las palabras de esta oración, me sentí muy debilitado.

\chapter{22}

\par \textit{La respuesta de Dios a la oración de Baruc. Cumplirá Su Promesa: Tiempo necesario para su Cumplimiento: Las cosas deben ser juzgadas a la Luz de su Consumación (22). Hasta que todas las Almas no nazcan, el Fin no puede llegar (23).}

\par 1 Y aconteció después de estas cosas que he aquí! Se abrieron los cielos, y vi, y me fue dado poder, y se oyó una voz desde lo alto, que me dijo:

\par 2 Baruc, Baruc, ¿por qué estás preocupado?

\par 3 El que va por un camino y no lo termina, o el que parte por mar y no llega al puerto, ¿podrá consolarse?

\par 4 O el que promete dar un regalo a otro, pero no lo cumple, ¿no es robo?

\par 5 ¿O el que siembra la tierra y no recoge el fruto a su debido tiempo, no lo pierde todo?

\par 6 ¿O el que planta una planta, si no crece hasta el momento adecuado, el que la plantó espera recibir fruto de ella?

\par 7 ¿O la mujer que ha concebido y da a luz fuera de tiempo, no matará a su hijo?

\par 8 O el que construye una casa, si no la techa y la termina, ¿será llamada casa? Dime eso primero.

\chapter{23}

\par 1 Yo respondí y dije: «No es así, Señor, Señor mío».

\par 2 Y Él respondió y me dijo: «¿Por qué, pues, te turbas por lo que no sabes, y por qué te inquietas por lo que ignoras?

\par 3 Porque así como vosotros no habéis olvidado a los que ahora están ni a los que han fallecido, así yo me acuerdo de los que están designados para venir.

\par 4 Porque cuando Adán pecó y se decretó la muerte de los que habían de nacer, entonces se contó la multitud de los que habían de nacer, y para ese número se preparó un lugar donde habitarían los vivos y se guardaría a los muertos. Por tanto, antes que se cumpla el número antedicho, la criatura no volverá a vivir [porque Mi espíritu es el creador de la vida], y el Seol recibirá a los muertos.

\par 5 [...]

\par 6 Y de nuevo os es concedido oír lo que vendrá después de estos tiempos.

\par 7 Porque verdaderamente mi redención está cerca y no está tan lejos como antes.»

\chapter{24}

\par \textit{El Juicio venidero}

\par 1 «¡Porque he aquí! Vienen días y se abrirán los libros en los que están escritos los pecados de todos los que han pecado, y nuevamente también los tesoros en los que está reunida la justicia de todos los que han sido justos en la creación.

\par 2 Porque sucederá que en aquel tiempo veréis, y muchos de los que están con vosotros, la paciencia del Altísimo, que ha sido por todas las generaciones, que ha sido paciente para con todos los que nacen, (por igual) los que pecan y (los que) son justos.»

\par 3 Y yo respondí y dije: «¡Pero he aquí! Oh Señor, nadie sabe el número de las cosas que han pasado ni aún de las que están por venir.

\par 4 Porque yo sé lo que nos ha sucedido, pero no sé qué les sucederá a nuestros enemigos, ni cuándo visitarás tus obras.»

\chapter{25}

\par \textit{Signo del Juicio venidero}

\par 1 Y Él respondió y me dijo: «Tú también serás preservado hasta el momento en que llegue la señal que el Altísimo hará para los habitantes de la tierra al final de los días. »

\par 2 Ésta, pues, será la señal.

\par 3 Cuando el estupor se apodere de los habitantes de la tierra, y caigan en muchas tribulaciones, y nuevamente cuando caigan en grandes tormentos. Y sucederá que cuando digan en sus pensamientos a causa de su mucha tribulación: «El Poderoso «Ya no se acuerda de la tierra»; sí, sucederá que cuando abandonen la esperanza, entonces el tiempo despertará.»

\chapter{26}

\par \textit{Los Doce Ayes que han de venir sobre la Tierra: El Mesías y el Reino Mesiánico temporal}

\par 1 Y respondí y dije: «¿La tribulación que ha de durar durará mucho tiempo y la necesidad abarcará muchos años?»

\chapter{27}

\par 1 Y Él respondió y me dijo: «Ese tiempo se divide en doce partes, y cada una de ellas está reservada para lo que le ha sido asignado.

\par 2 En la primera parte habrá comienzo de conmoción.

\par 3 Y en la segunda parte habrá matanzas de los grandes.

\par 4 Y en la tercera parte, la caída de muchos por muerte.

\par 5 Y en la cuarta parte el envío de la espada.

\par 6 Y en la quinta parte, hambre y retención de lluvias.

\par 7 Y en la sexta parte terremotos y terrores.

\par 8 [Queriendo.]

\par 9 Y en la octava parte, multitud de espectros y ataques de los Shedim.

\par 10 Y en la novena parte la caída de fuego.

\par 11 Y en la décima parte rapiña y mucha opresión.

\par 12 Y en la undécima parte la maldad y la fornicación.

\par 13 Y en la duodécima parte, confusión por la mezcla de todas estas cosas.

\par 14 Porque estas partes de aquel tiempo están reservadas, y se mezclarán unas con otras y se ministrarán unos a otros.

\par 15 Porque algunos dejarán fuera algo propio y recibirán de otros, y otros completarán lo suyo y lo ajeno, para que no entiendan los que están sobre la tierra en aquellos días que esto es la consumación de los tiempos.»

\chapter{28}

\par 1 «Sin embargo, quien entienda, será sabio.

\par 2 Porque la medida y el cómputo de ese tiempo son dos partes por semana de siete semanas.»

\par 3 Yo respondí y dije: «Es bueno que un hombre venga y mire, pero es mejor que no venga para no caer».

\par 4 [Pero también diré esto:

\par 5 ¿Despreciará el que es incorruptible las cosas que son corruptibles y todo lo que les sucede a las cosas que son corruptibles, para mirar sólo las que no son corruptibles?]

\par 6 Pero si; Oh Señor, las cosas que me has predicho ciertamente sucederán, así muéstrame esto también si en verdad he hallado gracia ante tus ojos.

\par 7 ¿Es en un lugar o en una parte de la tierra donde estas cosas acontecieron, o toda la tierra las experimentará?»

\chapter{29}

\par 1 Y Él respondió y me dijo: «Lo que suceda entonces (sucederá) a toda la tierra; por lo tanto, todos los que vivan los experimentarán.

\par 2 Porque en ese momento protegeré sólo a aquellos que se encuentren en esos mismos días en esta tierra.

\par 3 Y sucederá que cuando se cumpla todo lo que había de suceder en aquellas partes, el Mesías comenzará a revelarse.

\par 4 Y Behemoth será revelado desde su lugar y Leviatán ascenderá del mar, esos dos grandes monstruos que creé en el quinto día de la creación, y que habré conservado hasta ese momento; y entonces servirán de alimento a todos los que queden.

\par 5 La tierra también producirá su fruto diez mil veces y en cada (?) vid habrá mil pámpanos, y cada pámpano producirá mil racimos, y cada racimo producirá mil uvas, y cada uva producirá un grano de vino.

\par 6 Y los que tienen hambre se alegrarán; además, verán maravillas cada día.

\par 7 Porque de delante de mí saldrán vientos que traerán cada mañana la fragancia de frutos aromáticos, y al final del día nubes que destilarán el rocío de la salud.

\par 8 Y sucederá en ese mismo tiempo que el tesoro del maná descenderá otra vez de lo alto, y comerán de él en aquellos años, porque estos son los que han llegado a la consumación del tiempo.»

\chapter{30}

\par \textit{La Resurrección}

\par 1 Y sucederá que después de estas cosas, cuando se cumpla el tiempo de la venida del Mesías, él regresará en gloria.

\par 2 Entonces todos los que durmieron con la esperanza en Él resucitarán. Y sucederá en aquel tiempo que se abrirán los tesoros en los que está preservado el número de las almas de los justos, y saldrán, y se verá una multitud de almas juntas en una sola asamblea de un solo pensamiento, y los primeros se alegrarán y los últimos no se entristecerán.

\par 3 Porque saben que ha llegado el tiempo de que se dice, que es la consumación de los tiempos.

\par 4 Pero las almas de los impíos, cuando vean todas estas cosas, se consumirán aún más.

\par 5 Porque sabrán que ha llegado su tormento y su perdición.

\chapter{31}

\par \textit{Baruc exhorta al Pueblo a prepararse para males peores}

\par 1 Y aconteció después de estas cosas: que fui al pueblo y les dije: «Reúnanme a todos sus ancianos y les hablaré palabras».

\par 2 Y se reunieron todos en el valle del Cedrón.

\par 3 Respondí y les dije:

\par Escucha, Israel, y te hablaré,

\par Y escucha, oh descendencia de Jacob, y yo te instruiré.

\par 4 No olvides a Sión,

\par Pero recordad la angustia de Jerusalén.

\par 5 ¡Por lo! llegan los días,

\par Cuando todo lo que existe se convierta en presa de la corrupción

\par Y ser como si no hubiera sido.

\chapter{32}

\par 1 «Pero vosotros, si preparáis vuestros corazones para sembrar en ellos los frutos de la ley, ello os protegerá en el tiempo en que el Poderoso sacudirá a toda la creación.

\par 2 [Porque después de un poco de tiempo el edificio de Sión será sacudido para que pueda ser reconstruido. Pero ese edificio no permanecerá, sino que después de un tiempo será nuevamente desarraigado, y permanecerá desolado hasta el momento.

\par 3 [...]

\par 4 Y después debe renovarse en gloria y perfeccionarse para siempre.]

\par 5 Por lo tanto, no debemos angustiarnos tanto por el mal que ya ha llegado, sino por el que está por venir.

\par 6 Porque habrá una prueba mayor que estas dos tribulaciones cuando el Poderoso renueve Su creación.

\par 7 Y ahora no os acerquéis a mí por algunos días, ni me busquéis hasta que llegue a vosotros.»

\par 8 Y aconteció que cuando les hube hablado todas estas palabras, yo, Baruc, me puse en camino, y cuando la gente me vio partir, alzaron la voz y se lamentaron y dijeron:

\par 9 ¿A dónde te alejas de nosotros, Baruc, y nos abandonas como un padre que abandona a sus hijos huérfanos y se aleja de ellos?

\chapter{33}

\par 1 «¿Son estas las órdenes que os dio vuestro compañero el profeta Jeremías, y os dijo: »Mirad a este pueblo hasta que yo vaya y prepare al resto de los hermanos en Babilonia, contra quienes se ha pronunciado la sentencia que ¿Deberían ser llevados en cautiverio»? Y ahora, si tú también nos abandonas, sería bueno que todos muriéramos antes que tú, y luego tú te alejaras de nosotros.»

\chapter{34}

\par \textit{Lamento de Baruc}

\par 1 Y respondí y dije al pueblo: «Lejos esté de mí abandonaros o alejarme de vosotros, sino que sólo iré al Lugar Santísimo para preguntar al Poderoso acerca de vosotros y de Sión, si en algún aspecto debería recibir más iluminación: y después de estas cosas volveré a vosotros.»

\chapter{35}

\par 1 Y yo, Baruc, fui al lugar santo, me senté sobre las ruinas y lloré y dije:

\par 2 «Oh, si mis ojos fueran manantiales,

\par Y mis párpados son una fuente de lágrimas.

\par 3 ¿Cómo me lamentaré por Sión,

\par ¿Y cómo me lamentaré por Jerusalén?

\par 4 Porque en ese lugar donde ahora estoy postrado,

\par En la antigüedad, el sumo sacerdote ofrecía sacrificios santos,

\par Y puso sobre él un incienso de olores fragantes.

\par 5 Pero ahora nuestra gloria se ha convertido en polvo,

\par Y el deseo de nuestra alma en arena.»

\chapter{36}

\par \textit{La visión del bosque, la vid, la fuente y el cedro}

\par 1 Y habiendo dicho estas cosas, me quedé allí dormido y tuve una visión en la noche.

\par 2 ¡Y he aquí! un bosque de árboles plantados en la llanura, y altas y escarpadas montañas rocosas lo rodeaban, y ese bosque ocupaba mucho espacio.

\par 3 ¡Y he aquí! Frente a ella surgía una enredadera, y de debajo brotaba pacíficamente una fuente.

\par 4 Ahora bien, esa fuente llegó al bosque y se (agitó) en grandes olas, y esas olas sumergieron ese bosque, y de repente arrasaron la mayor parte de ese bosque, y derribaron todas las montañas que lo rodeaban.

\par 5 Y la altura del bosque comenzó a disminuir, y la cima de las montañas se hizo más baja, y la fuente prevaleció en gran medida, de modo que de aquel gran bosque no quedó nada más que un solo cedro.

\par 6 También cuando la hubo derribado y destruido y desarraigado la mayor parte de aquel bosque, de modo que no quedó nada de él, ni se pudo reconocer su lugar, entonces aquella vid comenzó a venir con la fuente en paz y gran tranquilidad, y llegó a un lugar que no estaba lejos de aquel cedro, y trajeron allí el cedro que había sido arrojado.

\par 7 ¡Y miré y he aquí! aquella vid abrió su boca y habló y dijo a aquel cedro: ¿No eres tú ese cedro que quedó del bosque de la maldad, y por cuyo medio la maldad persistió, y fue obrada todos esos años, y el bien nunca?

\par 8 Y seguiste conquistando lo que no era tuyo, y hacia lo que era tuyo nunca tuviste compasión, y seguiste extendiendo tu poder sobre los que estaban lejos de ti, y a los que se acercaban a ti, los retuviste. ¡Ayuna en las redes de tu maldad, y siempre te elevaste como alguien que no podía ser desarraigado!

\par 9 Pero ahora tu tiempo se ha acelerado y ha llegado tu hora.

\par 10 Tú también, oh cedro, vete tras el bosque que te precedió, y conviértete en polvo con él, y tus cenizas se mezclarán.

\par 11 Y ahora recuéstate en la angustia y descansa en el tormento hasta que llegue tu último tiempo, en el cual volverás y serás atormentado aún más.

\chapter{37}

\par 1 Y después de esto vi el cedro ardiendo, y la vid creciendo, ella misma y alrededor de ella, la llanura llena de flores eternas. Y en verdad me desperté y me levanté.

\chapter{38}

\par \textit{Interpretación de la Visión}

\par 1 Y oré y dije: «Oh Señor, mi Señor, tú siempre iluminas a los que se guían por el entendimiento.

\par 2 Tu ley es vida, y tu sabiduría es guía recta.

\par 3 Hazme saber, pues, la interpretación de esta visión.

\par 4 Porque tú sabes que mi alma siempre ha andado en tu ley, y desde mis primeros días no me he apartado de tu sabiduría.»

\chapter{39}

\par 1 Y él respondió y me dijo: «Baruc, ésta es la interpretación de la visión que has tenido.

\par 2 Como habéis visto el gran bosque rodeado de montañas altas y escarpadas, esta es la palabra.

\par 3 ¡Mira! Vienen días, y será destruido este reino que una vez destruyó a Sión, y será sometido al que vendrá después de él.

\par 4 Además, también después de un tiempo será destruido, y otro, un tercero, surgirá, y éste también dominará en su tiempo y será destruido.

\par 5 Y después de esto surgirá un cuarto reino, cuyo poder será mucho más duro y malvado que los que existieron antes de él, y gobernará muchas veces como los bosques en la llanura, y se mantendrá firme por tiempos, y se exaltará más que los cedros del Líbano.

\par 6 Y bajo ella se esconderá la verdad, y todos los contaminados por la iniquidad huirán a ella, como huyen las malas bestias y se esconden en el bosque.

\par 7 Y sucederá que cuando se acerque el tiempo de su consumación, que debe caer, entonces se revelará el principado de mi Mesías, que es como la fuente y la vid, y cuando se manifieste, arrancará de raíz a la multitud de su ejército.

\par 8 Y en cuanto a lo que habéis visto, el alto cedro que quedó de aquel bosque, y el hecho de que la vid pronunció con él las palabras que oísteis, ésta es la palabra.»

\chapter{40}

\par 1 El último líder de aquel tiempo quedará con vida, cuando la multitud de sus ejércitos será pasado a espada, y él será atado, y lo llevarán al monte Sión, y mi Mesías lo convencerá de todas sus impiedades, y recogerá y expondrá ante él todas las obras de sus ejércitos.

\par 2 Y después lo matará y protegerá al resto de mi pueblo que se encontrará en el lugar que yo he elegido.

\par 3 Y su principado permanecerá para siempre, hasta que el mundo de corrupción llegue a su fin y hasta que se cumplan los tiempos antes mencionados.

\par 4 Ésta es vuestra visión y ésta es su interpretación.


\chapter{41}

\par \textit{El destino de los apóstatas y de los prosélitos}

\par 1 Respondí y dije: ¿Para quién y para cuántos serán estas cosas? ¿O quién será digno de vivir en aquel tiempo?

\par 2 Porque hablaré delante de vosotros de todo lo que pienso y os preguntaré sobre lo que medito.

\par 3 ¡Para he aquí! Veo a muchos de tu pueblo que se han apartado de tu pacto y arrojan de sí el yugo de tu ley.

\par 4 Pero también he visto a otros que abandonaron su vanidad y huyeron en busca de refugio bajo tus alas.

\par 5 ¿Qué, pues, será de ellos? ¿O cómo los recibirá la última vez?

\par 6 ¿O tal vez se pesará el tiempo de estos y, a medida que la viga se incline, se los juzgará en consecuencia?»

\chapter{42}

\par 1 Y él respondió y me dijo: «También estas cosas te mostraré.

\par 2 En cuanto a lo que dijiste: «¿A quiénes serán estas cosas, y a cuántos (serán)? »—a los que hayan creído les llegará el bien del que antes se habló, y a los que desprecien les llegará lo contrario de estas cosas.

\par 3 Y en cuanto a lo que dijiste acerca de los que se acercaron y de los que se alejaron en la palabra.

\par 4 En cuanto a los que antes estaban sujetos y después se retiraron y se mezclaron con la simiente de los pueblos mezclados, el tiempo de éstos fue el primero y fue tenido por algo elevado.

\par 5 Y en cuanto a aquellos que antes no conocieron, pero después conocieron la vida, y se mezclaron (sólo) con la simiente del pueblo que se había separado, el tiempo de estos (es) el último, y se considera algo exaltado.

\par 6 Y el tiempo se sucederá al tiempo y la estación a la estación, y el uno recibirá del otro, y luego, con vistas a su consumación, todo se comparará según la medida de los tiempos y las horas de las estaciones.

\par 7 Porque la corrupción se llevará a sus propios habitantes, y la vida a sus propios habitantes.

\par 8 Y será llamado el polvo, y se le dirá: «Devuelve lo que no es tuyo, y levanta todo lo que has guardado hasta su tiempo».»


\chapter{43}

\par \textit{Baruc habló de su muerte y le ordenó dar sus últimas órdenes al pueblo}

\par 1 Pero tú, Baruc, dirige tu corazón a lo que te ha sido dicho:

\par Y comprended las cosas que os han sido mostradas;

\par Porque hay muchos consuelos eternos para vosotros.

\par 2 Porque partiréis de este lugar,

\par Y pasaréis de las regiones que ahora veis,

\par Y olvidaréis todo lo corruptible,

\par Y no volveré a recordar las cosas que suceden entre los mortales.

\par 3 Ve, pues, y manda a tu pueblo, y ven a este lugar, y después ayuna siete días, y luego iré a ti y hablaré contigo.

\par \textit{Baruc les cuenta a los Ancianos de su muerte inminente, pero los anima a esperar el Consuelo de Sión}

\chapter{44}

\par 1 Y yo, Baruc, salí de allí y llegué a mi pueblo, y llamé a mi hijo primogénito y a mis amigos [los Gedalías], y a siete de los ancianos del pueblo, y les dije:

\par 2 He aquí, voy a mis padres

\par Según el camino de toda la tierra.

\par 3 Pero no os apartéis del camino de la ley,

\par Pero guarda y amonesta al pueblo que queda,

\par No sea que se aparten de los mandamientos del Poderoso.

\par 4 Porque ya veis que aquel a quien servimos es justo,

\par Y nuestro Creador no hace acepción de personas.

\par 5 Y mirad lo que le ha sucedido a Sión,

\par ¿Y qué le ha sucedido a Jerusalén?

\par 6 Porque así se dará a conocer el juicio del Poderoso,

\par Y sus caminos, que aunque inescrutables, son rectos.

\par 7 Porque si perseveráis y perseveráis en su temor,

\par Y no os olvidéis de su ley,

\par Los tiempos cambiarán sobre ti para siempre.

\par Y veréis el consuelo de Sión.

\par 8 Porque lo que ahora es nada,

\par Pero lo que sucederá es muy grande.

\par Porque todo lo que es corruptible pasará,

\par 9 Y todo lo que muere, se marchará,

\par Y todo el tiempo presente será olvidado,

\par Tampoco habrá memoria del tiempo presente, que está contaminado con males.

\par 10 Porque lo que ahora corre corre hacia la vanidad,

\par Y el que prospera pronto caerá y será humillado.

\par 11 Porque lo que ha de ser será objeto de deseo,

\par Y esperaremos lo que viene después;

\par Porque es un tiempo que no pasa,

\par 12 Y llega la hora que permanecerá para siempre.

\par Y el nuevo mundo (viene) que no corrompe a los que parten hacia su bienaventuranza,

\par Y no tiene misericordia de los que parten al tormento,

\par Y no conduce a la perdición a los que en ella habitan.

\par 13 Porque éstos son los que heredarán el tiempo del que se ha hablado,

\par Y de ellos es la herencia del tiempo prometido.

\par 14 Estos son los que han adquirido tesoros de sabiduría,

\par Y con ellos se encuentran reservas de entendimiento,

\par Y no se han apartado de la misericordia,

\par Y ellos han guardado la verdad de la ley.

\par 15 Porque a ellos les será dado el mundo venidero,

\par Pero la morada de los demás, que son muchos, estará en el fuego.

\chapter{45}

\par 1 «Por tanto, en la medida de lo que puedas, instruye al pueblo, porque ese trabajo es nuestro. Porque si les enseñas, los vivificarás.»

\chapter{46}

\par 1 Entonces mi hijo y los ancianos del pueblo respondieron y me dijeron:

\par «¿Nos ha humillado el Poderoso hasta tal punto?

\par ¿Cómo alejarte de nosotros rápidamente?

\par 2 Y verdaderamente estaremos en tinieblas,

\par Y no habrá luz para el pueblo que quede,

\par 3 ¿Dónde buscaremos la ley?

\par ¿O quién distinguirá por nosotros entre la muerte y la vida?

\par 4 Y les dije: «No puedo resistir el trono del Poderoso;

\par Sin embargo, no le faltará a Israel un hombre sabio

\par Ni hijo de la ley para el linaje de Jacob.

\par 5 Pero sólo preparad vuestros corazones para que obedezcan la ley,

\par Y estad sujetos a los que en el temor son sabios y entendidos;

\par Y preparad vuestras almas para no apartaros de ellas.

\par 6 Porque si hacéis esto, os llegarán buenas nuevas.

\par [De lo que ya os hablé antes; ni caeréis en el tormento del que os he testificado antes.»

\par 7 Pero en cuanto a la palabra de que me iban a llevar, no se la comuniqué ni a ellos ni a mi hijo.]

\chapter{47}

\par 1 Y cuando salí y los despedí, fui allí y les dije: '¡Mirad! Voy a Hebrón, porque allí me ha enviado el Poderoso.

\par 2 Y llegué al lugar donde me habían hablado la palabra, y allí me senté y ayuné siete días.

\chapter{48}

\par \textit{ORACIÓN DE BARUC}

\par 1 Y aconteció que después del séptimo día oré delante del Poderoso y dije

\par 2 «Oh mi Señor, tú convocas el advenimiento de los tiempos,

\par Y están delante de ti;

\par Tú haces pasar el poder de los siglos,

\par Y no os resisten;

\par Tú organizas el método de las estaciones,

\par Y te obedecen.

\par 3 Sólo tú conoces la duración de las generaciones,

\par Y no revelas tus misterios a muchos.

\par 4 Tú haces notoria la multitud del fuego,

\par Y pesas la ligereza del viento.

\par 5 Exploras el límite de las alturas,

\par Y escrutas las profundidades de la oscuridad.

\par 6 Tú cuidas de los que pasan para que sean preservados, y preparas una morada para los que han de existir.

\par 7 Te acuerdas del principio que has hecho,

\par Y no olvides la destrucción que está por suceder.

\par 8 Con gestos de miedo e indignación controlas las llamas,

\par Y se transforman en espíritus,

\par Y con una palabra vivificas lo que no era,

\par Y con gran poder sostienes lo que aún no ha llegado.

\par 9 Tú enseñas las cosas creadas en tu entendimiento,

\par Y haces sabias las esferas para ministrar en sus órdenes.

\par 10 ejércitos innumerables se presentan ante ti

\par Y cumplir sus órdenes en silencio cuando Tu asentimiento.

\par 11 Escucha a tu siervo

\par Y escucha mi petición.

\par 12 Porque en poco tiempo nacemos,

\par Y en poco tiempo volvemos.

\par 13 Pero para ti las horas son como el tiempo,

\par Y los días como generaciones.

\par 14 No te enojes, pues, con el hombre; porque el no es nada

\par 15 Y no toméis en cuenta nuestras obras; ¿Para qué somos?

\par ¡Por lo! por tu don venimos al mundo,

\par Y no nos apartamos por nuestra propia voluntad.

\par 16 Porque no dijimos a nuestros padres: Engendradnos,

\par Ni enviamos al Seol y dijimos: «Recíbenos».

\par 17 ¿Cuál, pues, es nuestra fuerza para soportar tu ira?

\par ¿O qué somos nosotros para soportar tu juicio?

\par 18 Protégenos en tus misericordias,

\par Y en tu misericordia ayúdanos.

\par 19 Mirad a los pequeños que os están sujetos,

\par y salva a todos los que se acercan a ti:

\par Y no destruyáis la esperanza de nuestro pueblo,

\par Y no acortes los tiempos de nuestra ayuda.

\par 20 Porque ésta es la nación que tú has elegido,

\par Y estas son las personas a las que no encuentras igual.

\par 21 Pero ahora hablaré delante de vosotros,

\par Y diré lo que mi corazón piensa.

\par 22 En ti confiamos, porque ¡he aquí! Tu ley está con nosotros,

\par Y sabemos que no caeremos mientras guardemos tus estatutos.

\par 23 [Por siempre somos bienaventurados en todo cuanto no nos mezclamos con los gentiles.]

\par 24 Porque todos somos un pueblo célebre,

\par Quienes han recibido una ley de Uno:

\par Y la ley que está entre nosotros nos ayudará,

\par Y la excelente sabiduría que hay en nosotros nos ayudará.'

\par 25 Y cuando oré y dije estas cosas, quedé muy debilitado.

\par 26 Él respondió y me dijo:

\par «Has orado con sencillez, oh Baruc,

\par Y todas tus palabras han sido escuchadas.

\par 27 Pero mi juicio exige lo suyo

\par Y Mi ley exige sus derechos.

\par 28 Porque de tus palabras te responderé,

\par Y de tu oración te hablaré.

\par 29 Porque esto es lo siguiente: el que se corrompe no lo es en absoluto; Ha cometido iniquidad hasta donde podía hacer algo, y no se ha acordado de Mi bondad ni ha aceptado Mi longanimidad.

\par 30 Por tanto, ciertamente seréis arrebatados, como os dije antes.

\par 31 Porque llegará el tiempo de aflicción; porque vendrá y pasará con rápida vehemencia, y será turbulento viniendo en el calor de la indignación.

\par 32 Y sucederá en aquellos días que todos los habitantes de la tierra se agitarán unos contra otros, porque no saben que mi juicio se acerca.

\par 33 Porque en aquel tiempo no se encontrarán muchos sabios,

\par Y los inteligentes serán sólo unos pocos:

\par Además, incluso aquellos que saben, sobre todo, guardarán silencio.

\par 34 Y habrá muchos rumores y no pocas noticias,

\par Y la realización de fantasmas será manifiesta,

\par Y promete que no pocas serán contadas,

\par Algunos de ellos (resultarán) inactivos,

\par Y algunos de ellos serán confirmados.

\par 35 Y el honor se convertirá en vergüenza,

\par y la fuerza humillada hasta el desprecio,

\par y la probidad destruida,

\par Y la belleza se convertirá en fealdad.

\par 36 Y muchos dirán entre muchos en aquel tiempo:

\par "¿Dónde se ha escondido la multitud de inteligencia,

\par ¿Y adónde se ha trasladado la multitud de la sabiduría?»

\par 37 Y mientras meditan estas cosas,

\par Entonces surgirá la envidia en aquellos que no habían pensado en sí mismos (?)

\par Y la pasión se apoderará del pacífico,

\par Y muchos se enojarán para hacer daño a muchos,

\par Y levantarán ejércitos para derramar sangre,

\par Y al final perecerán junto con ellos.

\par 38 Y sucederá en el mismo tiempo,

\par Que el cambio de tiempos atraerá manifiestamente a todo hombre,

\par Porque en todos esos tiempos se contaminaron

\par Y practicaron la opresión,

\par y cada uno caminaba en sus propias obras,

\par Y no se acordó de la ley del Poderoso.

\par 39 Por tanto, un fuego consumirá sus pensamientos,

\par Y en llamas serán probadas las meditaciones de sus riñones;

\par Porque el Juez vendrá y no tardará.

\par 40 Porque cada uno de los habitantes de la tierra sabía cuándo estaba transgrediendo.

\par Pero no conocieron mi ley por su orgullo.

\par 41 Pero entonces muchos ciertamente llorarán,

\par Sí, más sobre los vivos que sobre los muertos.»

\par 42 Y yo respondí y dije:

\par «Oh Adán, ¿qué has hecho con todos los que nacen de ti?

\par ¿Y qué se le dirá a la primera Eva que escuchó a la serpiente?

\par 43 Porque toda esta multitud se corrompe,

\par Tampoco hay ningún número de aquellos a quienes el fuego devora.

\par 44 Pero otra vez hablaré en tu presencia.

\par 45 Tú, oh SEÑOR, mi Señor, conoces lo que hay en tu criatura.

\par 46 Porque tú ordenaste desde antiguo al polvo que produjera a Adán, y tú sabes cuántos han nacido de él y hasta qué punto pecaron antes de ti los que existieron y no te confesaron como su Creador.

\par 47 Y en cuanto a todo esto, su fin los convencerá, y Tu ley que transgredieron les recompensará en Tu día.»

\par \textit{Fragmento de un discurso de Baruc al pueblo}

\par 48 [«Pero ahora desechemos a los malvados y preguntemos por los justos.

\par 49 Y contaré sus bienaventuranzas

\par Y no guardéis silencio al celebrar su gloria, que les está reservada.

\par 50 Porque ciertamente, como en poco tiempo en este mundo transitorio en el que vivís, habéis soportado mucho trabajo,

\par Así que en ese mundo que no tiene fin, recibiréis una gran luz.»]

\chapter{49}

\par \textit{La naturaleza del cuerpo resucitado: los destinos finales de los justos y los impíos}

\par 1 Sin embargo, te pediré otra vez, oh Poderoso, y te pediré todas las cosas hechas.

\par 2 «¿En qué forma vivirán los que viven en Tus días?

\par ¿O cómo continuará el esplendor de los que (son) después de ese tiempo?

\par 3 ¿Volverán entonces a esta forma del presente,

\par Y vístete de estos miembros fascinantes,

\par que ahora están envueltos en males,

\par Y en el que se consuman los males,

\par ¿O acaso cambiarás estas cosas que han estado en el mundo?

\par ¿Y también el mundo?

\chapter{50}

\par 1 Y él respondió y me dijo:

\par «Oye, Baruc, esta palabra,

\par Y escribe en la memoria de tu corazón todo lo que aprendas.

\par 2 Porque entonces la tierra ciertamente restaurará a los muertos,

\par [Que ahora recibe, para poder conservarlos].

\par No hará ningún cambio en su forma,

\par Pero tal como los recibió, así los restituirá,

\par Y como yo se los entregué, así también él los resucitará.

\par 3 Porque entonces será necesario mostrar a los vivos que los muertos han resucitado y que los que se fueron han regresado.

\par 4 Y sucederá que cuando todos hayan reconocido a los que ahora conocen, entonces el juicio se fortalecerá y vendrán las cosas de las que antes se hablaba.»

\chapter{51}

\par 1 Y sucederá que cuando haya pasado el día señalado, el aspecto de los condenados cambiará y la gloria de los justificados.

\par 2 Porque el aspecto de aquellos que ahora actúan malvadamente será peor de lo que es, ya que sufrirán tormento.

\par 3 También la gloria de aquellos que ahora han sido justificados en mi ley, que han tenido inteligencia en su vida y que han plantado en su corazón la raíz de la sabiduría, entonces su esplendor será glorificado en cambios, y la forma de su rostro se transformará en la luz de su belleza, para que puedan adquirir y recibir el mundo que no muere, que luego les es prometido.

\par 4 Porque sobre todo se lamentarán los que vengan, que rechazaron mi ley y se taparon los oídos para no oír la sabiduría ni recibir entendimiento.

\par 5 Por tanto, cuando vean a aquellos sobre quienes ahora son exaltados, (pero) que luego serán exaltados y glorificados más que ellos, se transformarán respectivamente: los segundos en el esplendor de los ángeles, y los primeros aún más desperdicio de asombro ante las visiones y la contemplación de las formas.

\par 6 Porque primero verán y luego partirán para ser atormentados.

\par 7 Pero aquellos que se han salvado por sus obras,

\par Y para quienes la ley ha sido ahora una esperanza,

\par y entendiendo una expectativa,

\par y la sabiduría una confianza,

\par Aparecerán maravillas en su tiempo.

\par 8 Porque contemplarán el mundo que ahora les es invisible,

\par Y verán el tiempo que ahora les está oculto.

\par 9 Y el tiempo ya no los envejecerá.

\par 10 Porque en las alturas de ese mundo habitarán,

\par Y serán semejantes a los ángeles,

\par y ser igual a las estrellas,

\par Y serán transformados en cualquier forma que deseen,

\par De la belleza a la hermosura,

\par Y de la luz al esplendor de la gloria.

\par 11 Porque ante ellos se extenderá la extensión del Paraíso y se les mostrará la belleza y la majestad de los seres vivientes que están bajo el trono, y de todos los ejércitos de los ángeles, que ahora están retenidos por Mi palabra, para que no aparezcan, y sean retenidos por una orden, para que puedan permanecer en sus lugares hasta que llegue su advenimiento.

\par 12 Además, los justos tendrán más excelencia que los ángeles.

\par 13 Porque los primeros recibirán a los últimos, a los que esperaban, y los últimos a aquellos de quienes oían que habían fallecido.

\par 14 Porque han sido librados de este mundo de tribulación,

\par y entregó el peso de la angustia.

\par 15 ¿Por qué, pues, perdieron los hombres la vida,

\par ¿Y por qué cambiaron sus almas los que estaban en la tierra?

\par 16 Porque entonces escogieron (no) por sí mismos esta vez,

\par Que, más allá del alcance de la angustia, no pudo desaparecer:

\par Pero ellos mismos eligieron ese momento,

\par Cuyos orígenes están llenos de lamentaciones y males,

\par Y negaron el mundo, que no envejece a quienes vienen a él,

\par Y rechazaron el tiempo de gloria,

\par Para que no lleguen al honor que os dije antes.

\chapter{52}

\par 1 Y respondí y dije:

\par «¿Cómo podemos olvidar a aquellos a quienes entonces está reservado el infortunio?

\par 2 ¿Y por qué, pues, volvemos a llorar a los que mueren?

\par ¿O por qué lloramos por los que van al Seol?

\par 3 Que se reserven las lamentaciones para el comienzo del tormento venidero,

\par Y guardemos las lágrimas por la llegada de la destrucción de aquel tiempo.

\par 4 [Pero incluso ante estas cosas hablaré.

\par 5 Y los justos, ¿qué harán ahora?

\par 6 Alegraos del sufrimiento que ahora sufrís:

\par ¿Por qué buscáis la decadencia de vuestros enemigos?

\par 7 Prepara tu alma para lo que te está reservado,

\par Y preparad vuestras almas para la recompensa que os está reservada.»]

\chapter{53}

\par \textit{EL APOCALIPSIS DEL MESÍAS}

\par \textit{La Visión de la Nube con Aguas blancas y negras}


\par 1 Y habiendo dicho estas cosas, me quedé allí dormido, y tuve una visión, ¡y he aquí! (una nube subía de un mar muy grande, y yo seguía mirando hacia ella) y ¡he aquí! estaba lleno de aguas blancas y negras, y había muchos colores en aquellas mismas aguas, y como se veía en su cima como la semejanza de un gran relámpago.

\par 2 Y vi la nube que pasaba rápidamente y cubría toda la tierra.

\par 3 Y aconteció que después de estas cosas aquella nube comenzó a derramar sobre la tierra las aguas que había en ella.

\par 4 Y vi que no había una misma semejanza en las aguas que descendían de él.

\par 5 Porque en el primer principio eran negras y muchas.O un tiempo, y después vi que las aguas se volvían brillantes, pero no eran muchas, y después de estas cosas volví a ver negras (aguas), y después de estas cosas de nuevo brillante, y de nuevo negro y de nuevo brillante.

\par 6 Ahora bien, esto se hizo doce veces, pero los negros siempre eran más numerosos que los brillantes.

\par 7 Y aconteció que al final de la nube, ¡he aquí! llovieron aguas negras, y eran más oscuras que todas aquellas aguas que habían sido antes, y el fuego se mezcló con ellas, y donde esas aguas descendieron, provocaron devastación y destrucción.

\par 8 Y después de esto vi cómo el relámpago que había visto en la cima de la nube, la agarró y la arrojó a la tierra.

\par 9 Ahora bien, el relámpago brilló con tanta intensidad que iluminó toda la tierra y sanó aquellas regiones donde las últimas aguas habían descendido y habían causado devastación.

\par 10 Y se apoderó de toda la tierra y tuvo dominio sobre ella.

\par 11 Y vi después de estas cosas, y ¡he aquí! doce ríos subían del mar, y comenzaron a rodear aquel relámpago y a sujetarse a él.

\par 12 Y a causa de mi miedo me desperté.

\chapter{54}

\par \textit{La oración de Baruc pidiendo una interpretación de la visión: el advenimiento de Ramiel con este propósito}


\par 1 Y rogué al Poderoso, y dije:

\par «Sólo tú, oh Señor, conoces desde siempre las cosas profundas del mundo,

\par Y con tu palabra haces lo que sucederá en su tiempo, y contra las obras de los habitantes de la tierra apresuras el comienzo de los tiempos.

\par Y el final de las estaciones sólo tú lo sabes.

\par 2 (Tú) para quien nada es demasiado difícil,

\par Pero quienes hacen todo fácilmente con un movimiento de cabeza:

\par 3 (Tú) a quien las profundidades llegan como las alturas,

\par Y a cuya palabra sirven los principios de los siglos:

\par 4 (Tú) que revelas a los que te temen lo que está preparado para ellos,

\par Para que de ahora en adelante sean consolados.

\par 5 Tú muestras grandes obras a los que no saben;

\par Rompes el cerco de los ignorantes,

\par Y ilumina lo que está oscuro,

\par y revela lo que está oculto a los puros,

\par [Quienes con fe se han sometido a ti y a tu ley.]

\par 6 Tú le has mostrado a tu siervo esta visión;

\par Revélame también su interpretación.

\par 7 Porque sé que lo que os pedí he recibido respuesta.

\par Y en cuanto a lo que te pedí, me revelaste con qué voz debía alabarte.

\par Y de cuyos miembros debería hacer subir hasta vosotros alabanzas y aleluyas.

\par 8 Porque si mis miembros fueran bocas,

\par Y los cabellos de mi cabeza voces,

\par Aun así no podría darte la recompensa de la alabanza,

\par ni alabaros como conviene,

\par Ni pude contar tus alabanzas,

\par Ni cuentes la gloria de tu hermosura.

\par 9 ¿Qué soy yo entre los hombres?

\par ¿O por qué se me cuenta entre los que son más excelentes que yo?

\par Que he oído todas estas maravillas del Altísimo,

\par ¿Y las innumerables promesas de Aquel que me creó?

\par 10 Bendita sea mi madre entre las que dan a luz,

\par Y alabada entre las mujeres sea la que me dio a luz.

\par 11 Porque no callaré al alabar al Poderoso,

\par Y con voz de alabanza contaré sus maravillas.

\par 12 ¿Quién, oh Dios, hace semejantes a tus maravillas?

\par O que comprenden Tu profundo pensamiento de la vida.

\par 13 Porque con tu consejo gobiernas todas las criaturas que tu diestra ha creado.

\par Y has establecido a tu lado toda fuente de luz,

\par Y has preparado los tesoros de la sabiduría debajo de tu trono.

\par 14 Y justamente perecen los que no aman tu ley,

\par Y a quienes no se hayan sometido a tu poder les aguardará el tormento del juicio.

\par 15 Porque aunque Adán pecó primero

\par y trajo muerte prematura a todos,

\par Sin embargo, de los que nacieron de él

\par Cada uno de ellos se ha preparado para el tormento de su alma venidero,

\par Y nuevamente cada uno de ellos ha elegido para sí las glorias venideras.

\par 16 [Porque ciertamente el que crea recibirá recompensa.

\par 17 Pero ahora vosotros, los malvados que sois ahora, volveos a la perdición, porque pronto seréis visitados, por cuanto antes rechazasteis la comprensión del Altísimo.

\par 18 Porque sus obras no os han enseñado,

\par Tampoco os ha persuadido la habilidad de Su creación, que está en todo momento.]

\par 19 Por tanto, Adán no es causa sino sólo de su propia alma,

\par Pero cada uno de nosotros ha sido el Adán de su propia alma.

\par 20 Pero tú, oh Señor, explícame acerca de las cosas que me has revelado,

\par E infórmame sobre lo que te rogué.

\par 21 Porque en el fin del mundo se tomará venganza de los que han hecho lo malo según su maldad.

\par Y glorificarás a los fieles según su fidelidad.

\par 22 Tú gobiernas a los que están entre los tuyos,

\par Y a los que pecan los borraréis de entre los vuestros.

\chapter{55}

\par 1 Y aconteció que cuando terminé de pronunciar las palabras de esta oración, me senté allí debajo de un árbol para descansar a la sombra de las ramas.

\par 2 Y quedé maravillado y asombrado, y reflexioné en mis pensamientos acerca de la multitud de bienes que los pecadores que están sobre la tierra han rechazado, y acerca del gran tormento que han despreciado, aunque sabían que serían atormentados por causa de el pecado que habían cometido. Y cuando estaba pensando en estas cosas y cosas semejantes, ¡he aquí! Me fue enviado el ángel Ramiel que preside las visiones verdaderas, y me dijo:

\par 3 [...]

\par 4 «¿Por qué te angustia el corazón, Baruc,

\par y ¿por qué te perturba tu pensamiento?

\par 5 Porque si por el informe que habéis oído sólo de sentencia os conmueves,

\par ¿Qué serás cuando lo veas claramente con tus ojos?

\par 6 Y si con la esperanza con que esperáis el día del Poderoso estáis tan abrumados,

\par ¿Qué (serás) cuando llegues a su advenimiento?

\par 7 Y si al escuchar la noticia del castigo de los que han obrado neciamente os sentís tan perturbados,

\par ¿Cuánto más cuando el evento revelará cosas maravillosas?

\par 8 Y si oísteis noticias de los bienes y de los males que se avecinan entonces y os entristecéis,

\par ¿Qué (serás) cuando veas lo que la majestad revelará, Que convencerá a estos y hará que aquellos se regocijen?

\chapter{56}

\par \textit{Interpretación de la Visión. Las Aguas negras y brillantes simbolizan la Historia del Mundo desde Adán hasta el Advenimiento del Mesías.}


\par 1 Sin embargo, como rogaste al Altísimo que te revelara la interpretación de la visión que has tenido, he sido enviado a decírtelo.

\par 2 Y el Poderoso ciertamente os ha dado a conocer los métodos de los tiempos que han pasado, y de aquellos que están destinados a pasar en Su mundo desde el principio de su creación hasta su consumación, de aquellas cosas que (son ) engaño y de los que son en verdad.

\par 3 Porque como viste una gran nube que subió del mar y fue y cubrió la tierra, así es la duración del mundo (= αιων) que hizo el Poderoso cuando tomó consejo para hacer el mundo.

\par 4 Y aconteció que, cuando la palabra salió de su presencia, la duración del mundo nació en un pequeño grado, y se estableció según la multitud de la inteligencia de aquel que lo envió.

\par 5 Y como antes visteis en la cima de la nube las aguas negras que descendieron anteriormente sobre la tierra, ésta es la transgresión con la que transgredió Adán el primer hombre.

\par 6 Porque [desde] cuando transgredió

\par La muerte prematura surgió,

\par El dolor fue nombrado

\par Y se preparó la angustia,

\par Y se creó el dolor,

\par Y el problema se consuma,

\par Y la enfermedad comenzó a establecerse,

\par Y el Seol seguía exigiendo que fuera renovado con sangre,

\par Y se produjo la generación de hijos,

\par Y la pasión de los padres produjo,

\par Y la grandeza de la humanidad fue humillada,

\par Y la bondad languideció.

\par 7 ¿Qué, pues, puede ser más negro o más oscuro que estas cosas?

\par 8 Este es el comienzo de las aguas negras que habéis visto.

\par 9 Y de estas (aguas) negras nuevamente surgió el negro, y se produjo la oscuridad de las tinieblas.

\par 10 Porque se convirtió en un peligro para su propia alma, incluso para los ángeles.

\par 11 Porque además, en el momento de su creación gozaban de libertad.

\par 12 Y algunos de ellos, convirtiéndose en un peligro, descendieron y se mezclaron con las mujeres.

\par 13 Y luego los que lo hacían eran atormentados en cadenas.

\par 14 Pero el resto de la multitud de los ángeles, de los cuales no hay número, se contuvieron.

\par 15 Y los habitantes de la tierra perecieron junto con ellos por las aguas del diluvio.

\par 16 Estas son las primeras aguas negras.

\chapter{57}

\par 1 Y después de estas (aguas) viste aguas brillantes: esta es la fuente de Abraham, también sus generaciones y el advenimiento de su hijo, y del hijo de su hijo, y de sus semejantes.

\par 2 Porque en aquel tiempo se nombró entre ellos la ley no escrita,

\par Y entonces se cumplieron las obras de los mandamientos,

\par Y entonces se generó la creencia en el juicio venidero,

\par Y entonces se construyó la esperanza del mundo que había de renovarse,

\par Y se implantó la promesa de la vida que vendría en el más allá.

\par 3 Estas son las aguas brillantes que habéis visto.

\chapter{58}

\par 1 Y las terceras aguas negras que habéis visto, son la mezcla de todos los pecados que las naciones cometieron después de la muerte de aquellos justos, y la maldad de la tierra de Egipto, en la que hicieron maldades en el servicio con el cual hacían servir a sus hijos.

\par 2 Sin embargo, estos también finalmente perecieron.

\chapter{59}

\par 1 Y las brillantes cuartas aguas que habéis visto son el advenimiento de Moisés, Aarón, Miriam, Josué hijo de Nun, Caleb y todos los que son como ellos.

\par 2 Porque en aquel tiempo la lámpara de la ley eterna brilló sobre todos los que estaban sentados en las tinieblas, la cual anunció a los que creen la promesa de su recompensa, y a los que niegan, el tormento del fuego que les está reservado.

\par 3 Pero también en aquel momento los cielos se sacudieron de su lugar, y los que estaban bajo el trono del Poderoso se perturbaron cuando tomaba a Moisés consigo.

\par 4 Porque le mostró muchas advertencias junto con los principios de la ley y la consumación de los tiempos, como también a ti, y también el modelo de Sión y sus medidas, según el modelo del cual está construido el santuario del tiempo presente debe hacerse.

\par 5 Pero entonces también le mostró las medidas del fuego, las profundidades del abismo, el peso de los vientos y el número de las gotas de lluvia:

\par 6 y la represión de la ira, la multitud de la paciencia y la verdad del juicio:

\par 7 Y la raíz de la sabiduría, y las riquezas del entendimiento, y la fuente del conocimiento:

\par 8 Y la altura del aire, la grandeza del Paraíso, el fin de los siglos y el comienzo del día del juicio:

\par 9 Y el número de las ofrendas y las tierras que aún no han llegado:

\par 10 Y la boca del Gehena, y el lugar de la venganza, y el lugar de la fe, y la región de la esperanza; y la semejanza del tormento futuro, y la multitud de innumerables ángeles, y las huestes ardientes, y el esplendor de los relámpagos, y la voz de los truenos, y las órdenes de los jefes de los ángeles, y los tesoros de la luz, y los cambios de los tiempos, y las investigaciones de la ley.

\par 11 [...]

\par 12 Éstas son las cuartas aguas brillantes que habéis visto.

\chapter{60}

\par 1 Y las quintas aguas negras que habéis visto llover son las obras que realizaron los amorreos, y los hechizos de sus encantamientos que realizaron, y la maldad de sus misterios, y la mezcla de su contaminación.

\par 2 Pero también Israel estaba contaminado por los pecados en los días de los jueces, aunque veían muchas señales provenientes de Aquel que los hizo.

\chapter{61}

\par 1 Y las brillantes sextas aguas que vieron, este es el tiempo en que nacieron David y Salomón.

\par 2 En aquel tiempo se estaba construyendo Sion,

\par y la dedicación del santuario,

\par Y el derramamiento de mucha sangre de las naciones que entonces pecaron,

\par Y muchas ofrendas que se ofrecían entonces en la dedicación del santuario.

\par 3 Y en aquel tiempo reinaba la paz y la tranquilidad,

\par 4 Y en la asamblea se escuchó la sabiduría:

\par Y las riquezas del entendimiento fueron magnificadas en las congregaciones,

\par 5 Y las fiestas santas transcurrieron en bienaventuranza y gran alegría.

\par 6 Y entonces se vio que el juicio de los gobernantes era sin engaño,

\par Y la justicia de los preceptos del Poderoso se cumplió con la verdad.

\par 7 Y la tierra que entonces era amada por el Señor,

\par Y como sus habitantes no pecaron, fue glorificada en todos los países, y la ciudad de Sión reinó entonces en todos los países y regiones.

\par 8 Estas son las brillantes aguas que habéis visto.

\chapter{62}

\par 1 Y las séptimas aguas negras que habéis visto, ésta es la perversión (provocada) por el consejo de Jeroboam, que decidió hacer dos becerros de oro:

\par 2 Y todas las iniquidades que cometieron los reyes que vinieron después de él.

\par 3 Y la maldición de Jezabel y el culto a los ídolos que practicaba Israel en aquel tiempo.

\par 4 Y la retención de lluvias y las hambrunas que se produjeron hasta que las mujeres comieron el fruto de su vientre.

\par 5 Y el tiempo de su cautiverio que sobrevino a las nueve tribus y media, porque estaban en muchos pecados.

\par 6 Y vino Salmaneszar, rey de Asiria, y se los llevó cautivos.

\par 7 Pero en cuanto a los gentiles, sería tedioso decir que siempre cometieron impiedad y maldad, y nunca obraron justicia.

\par 8 Éstas son las séptimas aguas negras que habéis visto.

\chapter{63}

\par 1 Y la brillante octava agua que habéis visto, es la rectitud y la rectitud de Ezequías, rey de Judá, y la gracia (de Dios) que vino sobre él.

\par 2 Porque cuando Senaquerib se inquietó para perecer, y su ira lo inquietó para que pereciera, también por la multitud de las naciones que estaban con él.

\par 3 Cuando el rey Ezequías oyó las cosas que el rey de Asiria estaba tramando, es decir, venir, apoderarse de él y destruir a su pueblo, las dos tribus y media que quedaban, más aún quería derrocar. También Sion: entonces Ezequías confió en sus obras, y tuvo esperanza en su justicia, y habló con el Poderoso y dijo:

\par 4 «¡He aquí, porque he aquí! Senaquerib está preparado para destruirnos, y se jactará y enaltecerá cuando haya destruido a Sión».

\par 5 Y el Poderoso lo escuchó, porque Ezequías era sabio,

\par Y atendió su oración porque era justo.

\par 6 Entonces el Poderoso ordenó a Ramiel, su ángel que habla contigo.

\par 7 Y salí y destruí su multitud, el número de cuyos jefes sólo era ciento ochenta y cinco mil, y cada uno de ellos tenía un número igual (por orden suya).

\par 8 Y en aquel momento quemé sus cuerpos por dentro, pero conservé sus vestidos y sus brazos por fuera, para que pudieran aparecer las maravillas aún más maravillosas del Poderoso, y que así se hablara de su nombre por toda la tierra.

\par 9 Y Sión fue salvada y Jerusalén liberada; también Israel fue librado de la tribulación.

\par 10 Y todos los que estaban en la tierra santa se regocijaron y el nombre del Poderoso fue glorificado de tal manera que se hablaba de él.

\par 11 Estas son las aguas brillantes que habéis visto.

\chapter{64}

\par 1 Y las novenas aguas negras que habéis visto, ésta es toda la maldad que hubo en los días de Manasés hijo de Ezequías.

\par 2 Porque cometió mucha impiedad, mató a los justos, destruyó el juicio, derramó la sangre de los inocentes, profanó violentamente a las mujeres casadas, derribó los altares, destruyó sus ofrendas y expulsó sus sacerdotes para que no ministraran en el santuario.

\par 3 E hizo una imagen con cinco caras: cuatro de ellas miraban a los cuatro vientos, y la quinta en la cima de la imagen como un adversario del celo del Poderoso.

\par 4 Y entonces la ira salió de la presencia del Poderoso con la intención de extirpar a Sion, como también sucedió en vuestros días. Pero también contra las dos tribus y media salió un decreto para que también ellas fueran llevadas cautivas, como ahora habéis visto.

\par 5 Y la impiedad de Manasés aumentó hasta tal punto que hizo desaparecer del santuario la alabanza del Altísimo.

\par 6 [...]

\par 7 Por esta razón Manasés fue llamado en aquel tiempo «el impío», y finalmente su morada estuvo en el fuego.

\par 8 Porque aunque su oración fue escuchada por el Altísimo, al final, cuando fue arrojado en el caballo de bronce y el caballo de bronce se derritió, le sirvió de señal para la hora.

\par 9 Porque no había vivido perfectamente, porque no era digno, sino para saber en adelante quién le atormentaría finalmente.

\par 10 Porque el que puede beneficiar, también puede atormentar.

\chapter{65}

\par 1 Además, Manasés actuó impíamente, pensando que en su tiempo el Poderoso no investigaría estas cosas.

\par 2 Estas son las novenas aguas negras que has visto.

\chapter{66}

\par 1 Y las décimas aguas brillantes que has visto: esta es la pureza de las generaciones de Josías rey de Judá, quien fue el único en ese momento que se sometió al Poderoso con todo su corazón y con toda su alma.

\par 2 Y limpió la tierra de los ídolos, santificó todos los objetos contaminados, devolvió las ofrendas al altar, levantó el cuerno de los santos, exaltó a los justos y honró a todos los sabios de entendimiento. , y devolvió a los sacerdotes a su ministerio, y destruyó y expulsó de la tierra a los magos, encantadores y nigromantes.

\par 3 Y no sólo mató a los impíos que vivían, sino que también sacaron de los sepulcros los huesos de los muertos y los quemaron al fuego.

\par 4 [Y estableció las fiestas y los sábados en su santidad], y a sus contaminados los quemó en el fuego, y a los profetas mentirosos que engañaban al pueblo, a éstos también los quemó en el fuego, y al pueblo que escuchaba Cuando vivían, los arrojó al arroyo Cedrón, y amontonó piedras sobre ellos.

\par 5 Y fue celoso con celo por el Poderoso con toda su alma, y ​​solo él era firme en la ley en ese momento, de modo que no dejó a nadie que fuera incircunciso o que hiciera impiedad en toda la tierra, a todos los días de su vida.

\par 6 Por lo tanto, recibirá una recompensa eterna, y más tarde será glorificado con el Poderoso más que muchos.

\par 7 Porque por él y por los que son como él fueron creadas y preparadas las honorables glorias de las que os habéis hablado antes. Éstas son las aguas brillantes que habéis visto.

\chapter{67}

\par 1 Y las negras undécimas aguas que habéis visto: ésta es la calamidad que ahora azota a Sion.

\par 2 ¿Crees que no hay angustia para los ángeles en presencia del Poderoso?

\par Que Sión estaba tan entregada,

\par ¡Y eso he aquí! los gentiles se jactan en sus corazones,

\par y se reúnen ante sus ídolos y dicen:

\par «Es pisoteada la que muchas veces es pisoteada,

\par ¿Y ha sido reducida a servidumbre quien redujo (a otros)»?

\par 3 ¿Acaso pensáis que en estas cosas se alegra el Altísimo,

\par ¿O que su nombre sea glorificado?

\par 4 [Pero ¿cómo servirá para Su justo juicio?]

\par 5 Pero después de esto los dispersos entre las naciones serán azotados por la tribulación,

\par Y habitarán avergonzados en todo lugar.

\par 6 Porque en cuanto Sión es entregada

\par Y Jerusalén fue arrasada,

\par ¿Prosperarán los ídolos en las ciudades de los gentiles?

\par Y el vapor del humo del incienso de justicia que es por la ley se apaga en Sion,

\par Y en la región de Sión, en todo lugar, ¡he aquí! hay humo de impiedad.

\par 7 Pero se levantará el rey de Babilonia que ya ha destruido a Sión,

\par Y se jactará ante el pueblo,

\par Y hablará grandes cosas en su corazón delante del Altísimo.

\par 8 Pero él también al final caerá. Estas son las aguas negras.

\chapter{68}

\par 1 Y la duodécima agua brillante que has visto: esta es la palabra. Porque después de estas cosas vendrá el tiempo en que tu pueblo caerá en angustia, de modo que todos correrán el riesgo de perecer a una.

\par 2 [...]

\par 3 Sin embargo, ellos serán salvos y sus enemigos caerán ante ellos.

\par 4 Y a su debido tiempo tendrán mucha alegría.

\par 5 Y en aquel tiempo, después de un breve intervalo, Sión será reconstruida de nuevo, y sus ofrendas serán restauradas, y los sacerdotes volverán a su ministerio, y también los gentiles vendrán a glorificarla.

\par 6 Sin embargo, no del todo como al principio.

\par 7 Pero después de esto sucederá que muchas naciones caerán.

\par 8 Estas son las brillantes aguas que habéis visto.

\chapter{69}

\par 1 Porque las últimas aguas que habéis visto, que eran más oscuras que todas las que las precedieron, las que estaban después del número doce, que se reunieron, pertenecen al mundo entero.

\par 2 Porque el Altísimo hizo la división desde el principio, porque sólo Él sabe lo que sucederá.

\par 3 Porque en cuanto a las atrocidades e impiedades que se cometerían ante Él, previó seis clases de ellas.

\par 4 Y de las buenas obras que los justos debían realizar delante de Él, previó seis clases de ellas, además de las que debería realizar en la consumación del siglo.

\par 5 Por su cuenta no hubo aguas negras con negras, ni claras con claras; porque es la consumación.

\chapter{70}

\par 1 Oíd, pues, la interpretación de las últimas aguas negras que vendrán [después de las negras]: ésta es la palabra.

\par 2 ¡Mira! Vienen días, y será cuando haya llegado el tiempo de la era,

\par Y ha llegado la cosecha de sus malas y buenas semillas,

\par Que el Poderoso traerá sobre la tierra, sus habitantes y sus gobernantes

\par Perturbación del espíritu y estupor del corazón.

\par 3 Y se odiarán unos a otros,

\par y provocarnos unos a otros a pelear,

\par Y el mediocre dominará al honorable,

\par Y los humildes serán ensalzados por encima de los famosos.

\par 4 Y los muchos serán entregados en manos de unos pocos,

\par Y los que no eran nada dominarán a los fuertes,

\par Y los pobres tendrán más abundancia que los ricos,

\par Y los impíos se exaltarán por encima de los heroicos.

\par 5 Y los sabios callarán,

\par Y los necios hablarán,

\par Entonces tampoco se confirmará el pensamiento de los hombres,

\par Ni el consejo de los poderosos,

\par Ni se confirmará la esperanza de los que esperan.

\par 6 Y cuando se cumplan las cosas predichas,

\par Entonces caerá la confusión sobre todos los hombres,

\par Y algunos de ellos caerán en la batalla,

\par Y algunos de ellos perecerán en la angustia,

\par 7 Y algunos de ellos serán destruidos por sus propios medios. Entonces los pueblos altísimos que Él había preparado antes,

\par Y vendrán y harán guerra contra los líderes que entonces queden.

\par 8 Y sucederá que quien salga sano y salvo de la guerra morirá en el terremoto,

\par Y el que se salve del terremoto será quemado por el fuego,

\par Y el que salga del fuego, será destruido por el hambre.

\par 9 [Y sucederá que cualquiera de los vencedores y de los vencidos que salga a salvo y escape de todas estas cosas antedichas será entregado en manos de Mi siervo el Mesías.]

\par 10 Porque toda la tierra devorará a sus habitantes.

\chapter{71}

\par 1 Y la Tierra Santa tendrá misericordia de sí misma y protegerá a sus habitantes en ese momento.

\par 2 Esta es la visión que habéis visto, y ésta es la interpretación.

\par 3 Porque he venido a decirte estas cosas, porque tu oración ha sido escuchada con el Altísimo.

\chapter{72}

\par 1 Escucha ahora también acerca del brillante relámpago que vendrá en la consumación después de estas (aguas) negras: esta es la palabra.

\par 2 Después que vengan las señales que antes os habían contado, cuando las naciones se alboroten y llegue el tiempo de mi Mesías, él convocará a todas las naciones, y a algunas de ellas perdonará, y a otras las matará.

\par 3 Por tanto, estas cosas sucederán sobre las naciones que él habrá de salvar.

\par 4 Toda nación que no conozca a Israel y no haya pisoteado a la descendencia de Jacob, ciertamente será salvada.

\par 5 Y esto porque algunos de cada nación se someterán a tu pueblo.

\par 6 Pero todos los que te gobernaron o te conocieron, serán entregados a la espada.

\chapter{73}

\par 1 Y sucederá que cuando Él haya humillado todo lo que hay en el mundo,

\par y se ha sentado en paz para siempre en el trono de su reino,

\par Ese gozo entonces se revelará,

\par Y aparecerá el descanso.

\par 2 Y entonces la curación descenderá en forma de rocío,

\par Y la enfermedad desaparecerá,

\par Y la ansiedad, la angustia y el lamento pasan de entre los hombres,

\par Y la alegría recorre toda la tierra.

\par 3 Y nadie volverá a morir prematuramente,

\par Tampoco sobrevendrá ninguna adversidad de repente.

\par 4 Y juicios, palabras injuriosas, contiendas y venganzas,

\par Y sangre, y pasiones, y envidia, y odio,

\par Y todo lo que sea semejante a esto, cuando sea quitado, entrará en condenación.

\par 5 Porque son precisamente estas cosas las que han llenado este mundo de males,

\par Y a causa de esto la vida del hombre se ha visto muy perturbada.

\par 6 Y las fieras saldrán del bosque y servirán a los hombres.

\par Y los áspides y los dragones saldrán de sus cuevas para someterse a un niño.

\par 7 Y las mujeres ya no tendrán dolor al dar a luz,

\par Ni sufrirán tormento cuando den el fruto del vientre.

\chapter{74}

\par 1 Y sucederá que en aquellos días los segadores no se cansarán,

\par Ni los que construyen se fatigarán con el trabajo;

\par Porque las obras por sí mismas avanzarán rápidamente

\par Junto a quienes los hacen con mucha tranquilidad.

\par 2 Porque en aquel tiempo llegará el fin de lo corruptible,

\par Y el principio de lo que no es corruptible.

\par 3 Por tanto, lo que fue predicho le pertenecerá:

\par Por eso está lejos de los males y cerca de las cosas que no mueren.

\par 4 Este es el relámpago brillante que vino después de las últimas aguas oscuras.

\chapter{75}

\par \textit{Himno de Baruc sobre lo inescrutable de los caminos de Dios y sobre sus misericordias a través de las cuales los fieles alcanzarán una bendita consumación}

\par 1 Y respondí y dije:

\par «¿Quién podrá comprender, oh Señor, tu bondad?

\par Porque es incomprensible.

\par 2 ¿O quién podrá escudriñar tus misericordias,

\par ¿Cuáles son infinitos?

\par 3 ¿O quién podrá comprender tu inteligencia?

\par 4 ¿O quién podrá contar los pensamientos de tu mente?

\par 5 ¿O quién de los que nacen puede esperar llegar a esas cosas?

\par ¿A menos que sea alguien con quien seas misericordioso y benévolo?

\par 6 Porque si ciertamente no tuvisteis compasión del hombre,

\par Los que están bajo tu diestra,

\par No pudieron llegar a esas cosas,

\par Pero aquellos que están en los números nombrados pueden ser llamados.

\par 7 Pero si en verdad nosotros, los que existimos, sabemos a qué venimos,

\par y someternos a aquel que nos sacó de Egipto,

\par Volveremos y recordaremos las cosas que han pasado,

\par Y se regocijarán por lo que ha sucedido.

\par 8 Pero si ahora no sabemos a qué hemos venido,

\par Y no reconocemos el poder de Aquel que nos sacó de Egipto. Volveremos y buscaremos lo que ahora es,

\par y sed entristecidos con dolor por las cosas que os han acontecido.

\chapter{76}

\par \textit{Baruc recibió el encargo de instruir al Pueblo durante cuarenta días y luego mantenerse preparado para su Asunción en el Advenimiento del Mesías}

\par 1 Y Él respondió y me dijo: «[Ya que la revelación de esta visión te ha sido interpretada como lo pediste], escucha la palabra del Altísimo para que sepas lo que te acontecerá después de estas cosas.

\par 2 Porque ciertamente partiréis de esta tierra, pero no para la muerte, sino que seréis preservados hasta la consumación de los tiempos.

\par 3 Sube, pues, a la cima de esa montaña, y pasarán ante ti todas las regiones de esa tierra, y la figura del mundo habitado, y las cimas de las montañas, y las profundidades de los valles, y de las profundidades de los mares, y del número de los ríos, para que veas lo que dejas y adónde vas.

\par 4 Ahora bien, esto sucederá después de cuarenta días. Ve, pues, ahora durante estos días e instruye al pueblo tanto como puedas, para que aprendan a no morir en los últimos tiempos, pero aprendan para vivir en los últimos tiempos.»

\chapter{77}

\par \textit{La amonestación de Baruc al pueblo y sus dos cartas: una a las nueve tribus y media en Asiria y la otra a las dos y media en Babilonia}


\par 1 Y yo, Baruc, fui allí y llegué al pueblo, los reuní desde el mayor hasta el menor y les dije:

\par 2 «Oíd, hijos de Israel, cuántos sois los que quedan de las doce tribus de Israel.

\par 3 Porque el Señor les dio a ustedes y a sus padres una ley más excelente que la de todos los pueblos.

\par 4 Y por cuanto tus hermanos transgredieron los mandamientos del Altísimo,

\par Él trajo venganza sobre vosotros y sobre ellos,

\par Y no perdonó a los primeros,

\par Y a estos últimos también los entregó en cautiverio:

\par Y no dejó rastro de ellos,

\par 5 ¡Pero he aquí! estás aquí conmigo.

\par 6 Por tanto, si diriges bien tus caminos,

\par Tampoco vosotros os partiréis como se marcharon vuestros hermanos,

\par Pero ellos vendrán a ti.

\par 7 Porque es misericordioso aquel a quien adoráis,

\par Y es misericordioso aquel en quien tú esperas,

\par Y Él es veraz, de modo que hará el bien y no el mal.

\par 8 ¿No habéis visto aquí lo que le ha sucedido a Sión?

\par 9 ¿O acaso piensas que el lugar había pecado,

\par ¿Y que por eso fue derrocado?

\par O que la tierra había hecho necedad,

\par ¿Y que por eso fue entregado?

\par 10 ¿Y no sabéis que a causa de vosotros, que pecasteis,

\par Lo que no pecó fue destruido,

\par Y, a causa de aquellos que obraron malvadamente,

\par ¿Aquello que no hizo necedad fue entregado a (sus) enemigos?»

\par 11 Y todo el pueblo respondió y me dijo: «En la medida en que podamos recordar las cosas buenas que el Poderoso nos ha hecho, las recordaremos; y aquellas cosas que no recordamos, Él en su misericordia las sabe.

\par 12 Sin embargo, haz esto con nosotros, tu pueblo: escribe también a nuestros hermanos en Babilonia una carta de doctrina y un rollo de esperanza, para que también puedas confirmarlos antes de que te apartes de nosotros.

\par 13 Porque los pastores de Israel han perecido,

\par Y las lámparas que alumbraban se apagaron,

\par Y las fuentes han negado el chorro de donde solíamos beber.

\par 14 Y quedamos en la oscuridad,

\par Y entre los árboles del bosque,

\par Y la sed del desierto.»

\par 15 Y respondí y les dije

\par Los pastores, las lámparas y las fuentes provienen de la ley:

\par Y aunque nos apartemos, la ley permanece.

\par 16 Así que, si tenéis respeto por la ley,

\par Y se concentran en la sabiduría,

\par No faltará una lámpara,

\par Y un pastor no fallará,

\par Y una fuente no se secará.

\par 17 Sin embargo, como me dijiste, escribiré también a tus hermanos en Babilonia, y enviaré por medio de hombres, y escribiré de la misma manera a las nueve tribus y media, y enviaré por medio de un pájaro.'

\par 18 Y aconteció que el día veintiuno del mes octavo, yo, Baruc, vine y me senté bajo la encina, a la sombra de las ramas, y no había ningún hombre conmigo, sino que yo estaba solo.

\par 19 Y escribí estas dos epístolas: la primera la envié mediante un águila a las nueve tribus y media; y el otro lo envié a los que estaban en Babilonia por medio de tres hombres.

\par 20 Y llamé al águila y le dije estas palabras:

\par 21 «El Altísimo te ha hecho para que seas más alto que todas las aves.

\par 22 Ahora, pues, id y no os detengáis en ningún lugar, ni entréis en un nido, ni os poséis en ningún árbol, hasta que hayáis pasado la anchura de las muchas aguas del río Éufrates y hayais llegado al pueblo que habita allí, y les arrojaste esta epístola.

\par 23 Acordaos además de que, en el momento del diluvio, Noé recibió de una paloma el fruto de la aceituna, cuando la sacó del arca.

\par 24 Y también los cuervos servían a Elías, llevándole comida, tal como se les había ordenado.

\par 25 También Salomón, durante su reinado, mandaba a un pájaro a cualquier lugar a donde quería enviar o buscar algo, y éste le obedecía tal como él se lo ordenaba.

\par 26 Ahora pues, no os canséis, y no os desviéis ni a derecha ni a izquierda, sino huyed y andad por el camino recto, para que guardéis el mando del Poderoso, como os he dicho.»

\chapter{78}

\par \textit{LA EPÍSTOLA DE BARUC HIJO DE NERIAS QUE ESCRIBIÓ A LAS NUEVE TRIBUS Y MEDIA}

\par 1 Estas son las palabras de la carta que Baruc, hijo de Nerías, envió a las nueve tribus y media que estaban al otro lado del río Éufrates, en la que se escribían estas cosas.

\par 2 Así dice Baruc, hijo de Nerías, a los hermanos llevados en cautiverio: «Misericordia y paz». Tengo presente, hermanos míos, el amor de Aquel que nos creó, que nos amó desde antiguo y nunca nos odió, sino que, sobre todo, nos educó.

\par 3 Y en verdad sé que todos nosotros, las doce tribus, estamos unidos por un mismo vínculo, en cuanto nacemos de un solo padre.

\par 4 Por eso he tenido mucho cuidado en dejaros las palabras de esta epístola antes de morir, para que os consoléis de los males que os han sobrevenido, y para que también os entristezcáis por el mal que ha sobrevenido a vuestros hermanos de religion; y otra vez, también, para que puedas justificar su juicio que

\par 5 Él ha decretado contra vosotros que seáis llevados cautivos, porque lo que habéis sufrido es desproporcionado con lo que habéis hecho, para que en los últimos tiempos seáis hallados dignos de vuestros padres.

\par 6 Por lo tanto, si consideráis que ahora habéis sufrido esas cosas por vuestro bien, para que finalmente no seáis condenados y atormentados, entonces recibiréis la esperanza eterna; si sobre todo destruís de vuestro corazón el vano error que os hizo partir de aquí.

\par 7 Porque si así hacéis esto, siempre se acordará de vosotros Él, que siempre prometió por nosotros a los que eran más excelentes que nosotros, que nunca nos olvidará ni nos desamparará, sino que con mucha misericordia nos reunirá de nuevo a los que estaban dispersos.

\chapter{79}

\par 1 Ahora, hermanos míos, aprended primero lo que le sucedió a Sión: cómo Nabucodonosor, rey de Babilonia, subió contra nosotros.

\par 2 Porque hemos pecado contra el que nos hizo, y no hemos guardado los mandamientos que él nos ordenó, pero él no nos ha castigado como merecíamos.

\par 3 Por lo que os sucedió a vosotros, también nosotros sufrimos en gran medida, porque también nos sucedió a nosotros.

\chapter{80}

\par 1 Y ahora, hermanos míos, os hago saber que cuando el enemigo rodeó la ciudad, fueron enviados los ángeles del Altísimo, y derribaron las fortificaciones de la fuerte muralla y destruyeron las firmes esquinas de hierro, que no se pudo erradicar.

\par 2 Sin embargo, escondieron todos los utensilios del santuario, para que el enemigo no se apoderara de ellos.

\par 3 Y cuando hicieron estas cosas, entregaron al enemigo el muro derribado, la casa saqueada, el templo quemado y el pueblo vencido por haber sido entregados, para que el enemigo no se jactara y dijera :

\par 4 «Así por la fuerza hemos podido destruir incluso la casa del Altísimo en la guerra.» A tus hermanos también los ataron y los llevaron a Babilonia, y los hicieron habitar allí.

\par 5 Pero nosotros hemos quedado aquí, siendo muy pocos.

\par 6 Esta es la tribulación sobre la cual os escribí.

\par 7 Porque ciertamente sé que (el consuelo de) los habitantes de Sión os consuela: en la medida en que sabíais que fue prosperado (tu consuelo) fue mayor que la tribulación que soportaste al tener que partir de él.

\chapter{81}

\par 1 Pero en cuanto a la consolación, escucha la palabra.

\par 2 Porque estaba de luto por Sion, y pedí misericordia al Altísimo, y dije:

\par 3 «¿Hasta cuándo nos durarán estas cosas?

\par ¿Y estos males nos sobrevendrán siempre?»

\par 4 Y el Poderoso hizo conforme a la multitud de sus misericordias,

\par Y el Altísimo, según la grandeza de su compasión,

\par Y me reveló la palabra para que pudiera recibir consuelo,

\par Y me mostró visiones para que no volviera a sufrir angustias.

\par Y me hizo conocer el misterio de los tiempos.

\par Y el advenimiento de las horas me mostró.

\chapter{82}

\par 1 Por eso, hermanos míos, os he escrito para que os consoléis de la multitud de vuestras tribulaciones.

\par 2 Porque sabed que nuestro Hacedor ciertamente nos vengará de todos nuestros enemigos, conforme a todo lo que nos han hecho, y que la consumación que el Altísimo hará está muy cerca, y Su misericordia que está por llegar, y la consumación de su juicio no está lejos.

\par 3 ¡Para he aquí! Vemos ahora la multitud de la prosperidad de los gentiles,

\par Aunque actúen impíamente,

\par Pero serán como vapor:

\par 4 Y contemplamos la multitud de su poder,

\par Aunque hagan lo malo,

\par Pero serán como una gota:

\par 5 Y vemos la firmeza de su poder.

\par Aunque resistan cada hora al Poderoso,

\par Pero serán contados como saliva.

\par 6 Y contemplamos la gloria de su grandeza,

\par Aunque no guarden los estatutos del Altísimo,

\par Pero desaparecerán como humo.

\par 7 Y meditamos en la belleza de su gracia,

\par Aunque tienen que ver con contaminaciones,

\par Pero se marchitarán como hierba que se seca.

\par 8 Y consideramos la fuerza de su crueldad,

\par Aunque no recuerdan el final (del mismo),

\par Pero serán quebrados como una ola que pasa.

\par 9 Y observamos la jactancia de su poder,

\par Aunque nieguen la bondad de Dios, que se la dio,

\par Pero pasarán como una nube pasajera.

\chapter{83}

\par 1 Porque el Altísimo ciertamente acelerará sus tiempos,

\par Y ciertamente traerá sus horas.

\par 2 Y ciertamente juzgará a los que están en su mundo,

\par Y visitará en verdad todas las cosas por medio de todas sus obras ocultas.

\par 3 Y ciertamente Él examinará los pensamientos secretos,

\par Y lo que está guardado en las cámaras secretas de todos los miembros del correo. Y los manifestará en presencia de todos con reprensión.

\par 4 Por tanto, nada de lo presente suba a vuestros corazones; antes todo, estemos expectantes, porque lo que se nos ha prometido llegará.

\par 5 Y no miremos ahora los deleites de los gentiles en el presente, sino recordemos lo que se nos ha prometido al final.

\par 6 Porque el fin de los tiempos y de las estaciones y todo lo que sucede con ellos ciertamente pasará junto.

\par 7 Además, la consumación de esta era mostrará el gran poder de su gobernante, cuando todas las cosas lleguen a su juicio.

\par 8 Preparad, pues, vuestro corazón para lo que antes creísteis, no sea que seáis esclavos en ambos mundos y seáis llevados cautivos aquí y atormentados allá.

\par 9 Pues ni lo que existe ahora, ni lo que pasó, ni lo que está por venir, en todas estas cosas, ni el mal es enteramente malo, ni el bien es plenamente bueno.

\par 10 Porque toda la salud de este tiempo se está convirtiendo en enfermedad,

\par 11 Y todo el poder de este tiempo se está convirtiendo en debilidad,

\par Y toda la fuerza de este tiempo se está convirtiendo en impotencia,

\par 12 Y toda energía de la juventud se está convirtiendo en vejez y consumación.

\par Y toda belleza de gracia de este tiempo se está volviendo descolorida y odiosa,

\par 13 Y todo dominio soberbio del presente se está convirtiendo en humillación y vergüenza,

\par 14 Y toda alabanza de la gloria de este tiempo se está convirtiendo en vergüenza del silencio,

\par Y todo vano esplendor e insolencia de este tiempo se está convirtiendo en una ruina silenciosa.

\par 15 Y todo deleite y alegría de este tiempo se está convirtiendo en gusanos y corrupción,

\par 16 Y todo clamor del orgullo de este tiempo se está convirtiendo en polvo y silencio.

\par 17 Y toda posesión de riquezas de este tiempo se convertirá solo en el Seol,

\par 18 Y toda la rapiña de la pasión de este tiempo se está convirtiendo en muerte involuntaria,

\par Y cada pasión de las concupiscencias de este tiempo se está convirtiendo en un juicio de tormento.

\par 19 Y todo artificio y astucia de este tiempo se está convirtiendo en prueba de la verdad,

\par 20 Y toda dulzura de ungüentos de este tiempo se está convirtiendo en juicio y condenación,

\par 21 Y todo amor a la mentira se convierte en vergüenza por la verdad.

\par 22 [Pues ahora que todas estas cosas han sucedido ahora, ¿pensará alguien que no será vengado? Pero la consumación de todas las cosas llegará a la verdad.]

\chapter{84}

\par 1 ¡Mira! Por tanto, os he hecho saber (estas cosas) mientras vivo: porque os he dicho que aprendáis las cosas excelentes; porque el Poderoso me ha mandado que os enseñe; y os presentaré algunos de los mandamientos de su juicio antes de que muera.

\par 2 Acordaos de que en otro tiempo Moisés ciertamente llamó al cielo y a la tierra por testigos contra vosotros y dijo: «Si quebrantáis la ley, seréis dispersados, pero si la guardéis, seréis guardados».

\par 3 Y también os decía otras cosas cuando vosotros, las doce tribus, estabais juntas en el desierto.

\par 4 Y después de su muerte los echaste de ti: por eso te sobrevino lo que había sido predicho.

\par 5 Y ahora Moisés os lo decía antes de que os sucedieran, y ¡he aquí! os han sucedido, porque habéis abandonado la ley.

\par 6 ¡Mira! También os digo, después de haber sufrido, que si obedecéis las cosas que os han sido dichas, recibiréis del Poderoso todo lo que está guardado y reservado para vosotros.

\par 7 Además, que esta epístola sirva de testimonio entre yo y vosotros, para que os acordéis de los mandamientos del Poderoso, y para que también yo tenga defensa delante del que me envió.

\par 8 Y acordaos de la ley, de Sión, de la Tierra Santa, de vuestros hermanos y del pacto de vuestros padres, y no olvidéis las fiestas ni los sábados. Y entrega esta epístola y las tradiciones de la ley a tus hijos después de ti, como también tus padres te las entregaron a ti.

\par 9 [...]

\par 10 Y en todo momento rogad con perseverancia y orad diligentemente con todo vuestro corazón para que el Poderoso se reconcilie con vosotros y no cuente la multitud de vuestros pecados, sino que se acuerde de la rectitud de vuestros padres.

\par 11 Porque si Él no nos juzga según la multitud de sus misericordias, ¡ay de todos los que nacemos!

\chapter{85}

\par 1 Sepa, además, que

\par En tiempos pasados ​​y en las generaciones antiguas, nuestros padres tuvieron ayudantes,

\par Hombres justos y santos profetas:

\par 2 No más, estábamos en nuestra propia tierra

\par [Y nos ayudaron cuando pecamos],

\end{document}