\begin{document}


\title{Testamento de Isaac}

\chapter{1}

\par 1 En el nombre del Padre, del Hijo y del Espíritu Santo, el único Dios.

\par 2 Comenzamos con la ayuda de Dios y por su mediación a celebrar la muerte del patriarca Isaac, hijo del patriarca Abraham, y su ascensión de su

\par 3 cuerpo en este mismo día, que es el veintiocho del mes Misri. ¡Que la bendición de su intercesión esté con nosotros y nos proteja de las tentaciones del enemigo! ¡Amén!

\par 4 [Y el patriarca Isaac escribió su testamento y dirigió sus palabras de instrucción a Jacob su hijo y a todos los que estaban reunidos con él. ]

\par 5 Él dijo: «Oíd, hermanos míos y amados míos, la instrucción de este orador y esta medicina curativa.

\par 6 Porque el camino de Dios es eterno, oíd no sólo con castos oídos corporales, sino también con la profundidad del corazón y con fe verdadera, sin duda alguna, como está escrito: He aquí, habéis oído una palabra firme en cuanto a lo que debería llegar a ser un hombre. Si lo ha oído con un corazón puro, Dios le dará compasión cuando le pida algo»».

\par 7 «Y también está escrito: «De nada sirve pedir a Dios lo que los hombres solicitan en la tierra». Y si Dios nos ha dado señorío en la tierra, ¡cuánto es el provecho del que ha sido firme en la fe en la palabra de Dios, y se ha aferrado sin duda y con recto corazón al conocimiento de los mandamientos de Dios! Dios y las historias de sus santos; porque él será el heredero del reino de Dios».

\par 8 «Porque he aquí, Dios es compasivo y misericordioso, el que en tiempos pasados ​​recibió en sí a ladrones y recaudadores de impuestos debido a la sinceridad de su fe que viene de Dios. Y Dios, además, está con los siglos venideros».

\chapter{2}

\par 1 Aconteció que cuando se acercaba el tiempo para que nuestro padre Isaac, el padre de nuestros padres, dejara este mundo y dejara su cuerpo, el Compasivo y Misericordioso le envió el jefe de los ángeles, Miguel, el que había enviado a su padre Abraham, en la mañana del día veintiocho del mes Misri.

\par 2 El ángel le dijo: «¡Paz a ti, hijo escogido, nuestro padre Isaac!»

\par 3 Era costumbre que los santos ángeles le hablaran todos los días. Entonces se postró y vio que el ángel se parecía a su padre Abraham.

\par 4 Entonces abrió la boca, gritó a gran voz y dijo con alegría y alborozo: «He aquí, he visto tu rostro como si hubiera visto el rostro del Creador misericordioso».

\par 5 Entonces el ángel le dijo: «Oh mi amado Isaac, he sido enviado a ti desde la presencia del Dios viviente para llevarte al cielo para que estés con tu padre Abraham y con todos los santos.

\par 6 Porque Abraham, tu padre, te está esperando; él mismo está a punto de venir por ti, pero ahora está descansando.

\par 7 Te han preparado un trono junto a tu padre Abraham; lo mismo para tu amado hijo Jacob.

\par 8 Y todos vosotros estaréis por encima de todos en el reino de los cielos, en la gloria del Padre, del Hijo y del Espíritu Santo.

\par 9 A vosotros se os confiará este nombre para todas las generaciones futuras: Los Patriarcas. Así seréis padres de todo el mundo, oh anciano fiel, nuestro padre Isaac».

\par 10 Isaac respondió y dijo al ángel: «Realmente estoy asombrado por ti. ¿No eres tú mi padre Abraham?

\par 11 Entonces el ángel le dijo: «Yo no soy tu padre Abraham, sino el que sirve a tu padre Abraham.

\par 12 Ahora pues, alegraos y alegraos; porque no serás herido (con enfermedad)? y no será tomado (en la muerte) con dolor sino con alegría.

\par 13 Alcanzarás bendiciones y reposo para siempre y saldrás de la prisión a la amplitud.

\par 14 También irás a un regocijo que no tiene fin, a una luz y a una dicha que no tienen límite, y a una aclamación y un deleite sin cesar.

\par 15 Ahora, pues, haz tu testamento y ordena tu casa; porque estás a punto de irte al descanso (final).

\par 16 ¡Sin embargo, la bienaventuranza será para el padre que te engendró y para tu descendencia que vendrá después de ti!

\par 17 Cuando nuestro padre Jacob los escuchó hablar así entre sí, comenzó a escucharlos, pero no habló.

\par 18 Entonces nuestro padre Isaac dijo al ángel con paciencia y humildad: «¿Qué haré ahora con la luz de mis ojos, amado mío Jacob?

\par 19 Temo por él a causa de Esaú. Usted, por supuesto, conoce toda la historia».

\par 20 Entonces el ángel le dijo: «Mi amado Isaac, todos los pueblos del mundo, si se reunieran en un solo lugar, no podrían anular tu bendición sobre Jacob; porque, en aquel tiempo que lo bendijisteis, fue bendecido por el Dios supremo, también por el Hijo y el Espíritu Santo, y por vuestro padre Abraham; todos ellos respondieron, diciendo: «Amén».

\par 21 El hierro (¿espada?) no lo asustará, pero será extremadamente fuerte y conquistará el poder.

\par 22 Entonces será padre de muchas naciones, y de él surgirán doce tribus.

\par 23 Entonces Isaac dijo al ángel: «Tú me has informado y me has traído buenas noticias.

\par 24 Pero Jacob no oiga, porque se entristecerá y se turbará; porque nunca he herido su corazón en absoluto».

\par 25 Entonces el ángel del Señor dijo: «Oh, mi amado Isaac, todos los justos que salen de sus cuerpos son bienaventurados y se sienten bienaventurados cuando ven a Dios, el Misericordioso y el Compasivo.

\par 26 Pero ¡ay, ay tres veces del pecador que nace en la tierra, porque tiene muchos dolores!

\par 27 Enseñarás a tus hijos tus caminos y los mandamientos de tu padre, todos los que él te mandó.

\par 28 Y no ocultes estas cosas a Jacob, para que sirvan de recordatorio a las generaciones de su descendencia después de él, para que los fieles las observen y con ellas alcancen la vida eterna, que es para siempre.

\par 29 Pero tendré en cuenta tu preocupación.

\par 30 He aquí, llegué pronto a vosotros con alegría. La paz que el Señor dio, yo os la doy.

\par 31 Y ahora voy presto al que me envió.

\chapter{3}

\par 1 Dicho esto, el ángel se levantó del lecho de nuestro padre Isaac y se alejó de él.

\par 2 Isaac seguía mirándolo y estaba asombrado de lo que había oído y visto.

\par 3 Entonces se comprometió a decir: «No veré la luz hasta que me mandes llamar».

\par 4 Mientras meditaba en esto, Jacob se acercó a la puerta de la cámara de su padre.

\par 5 El ángel ya le había echado sueño para que no pudiera oírlos.

\par 6 Entonces, cuando entró en el lugar de descanso de su padre, dijo: «Padre, ¿con quién hablabas?»

\par 7 Isaac su padre le dijo: Ahora debes escucharme, hijo mío. Se ha enviado un mensaje a tu venerable padre para que te lo quiten, oh hijo mío Jacob.

\par 8 Entonces Jacob abrazó a su padre y lloró, diciendo: «Mis fuerzas se han ido de mí; ¿Me dejarás huérfano, oh padre mío, de modo que hoy seré desdichado?

\par 9 Abrazó de nuevo a nuestro padre Isaac y lo besó; Ambos lloraron hasta cansarse y cansarse.

\par 10 Entonces Jacob dijo: «Padre, me iré contigo y no te abandonaré».

\par 11 Pero Isaac le dijo: «Muchacho, esto no me corresponde a mí hacer, oh hijo mío y amado mío Jacob; pero doy gracias a Dios que tú también has llegado a ser padre y que permanecerás hasta que seas llamado. . .?

\par 12 Como me informó mi padre Abraham, no puedo anular ninguna parte del decreto, que es válido para todos; así sucederá, porque lo que está escrito no será frustrado.

\par 13 Pero Dios sabe, hijo mío, que mi corazón está cansado por causa de ti. Sin embargo, estoy feliz de ir al Señor.

\par 14 Así que, ahora que habéis experimentado el crecimiento en el Espíritu, dejad de vosotros este llanto y lamento.

\par 15 Escucha, muchacho, para que te hable y te haga entender acerca del primer hombre, es decir, nuestro padre Adán, el creado, a quien Dios formó con su mano; así mismo nuestra madre Eva; también Abel y Set y nuestro padre Enoc (¿Enós?) y Mahalaleel, el padre de Matusalé, y Lamec, el padre de Jared, y Enós (¿Enós?), el padre de nuestro padre Noé y sus hijos, Sem, Cam y Jafet; y después de ellos Finees, Kenán, Noé(?), Heber, Reu, Taré, Nacor, mi padre Abraham y Lot, hijo de su hermano.

\par 16 A todos estos se los llevó la muerte, excepto a nuestro padre Enoc, el perfecto que ascendió al cielo.

\par 17 «Y después de esto surgirán doce gigantes.

\par 18 Entonces, de vuestra descendencia, vendrá Jesús, el Mesías, de una virgen llamada María.

\par 19 Y Dios se encarnará en él hasta el cumplimiento de cien años.

\chapter{4}

\par 1 Isaac solía ayunar todos los días y no rompía el ayuno hasta la tarde.

\par 2 Ofrecería sacrificios por sí mismo y por todos los de su casa, para la salvación de sus almas.

\par 3 Se levantaba a media noche para orar y durante el día oraba a Dios. Siguió haciendo esto durante muchos años.

\par 4 También ayunaba durante los tres períodos de cuarenta días, cada vez que llegaba el período de cuarenta días.'

\par 5 Y no comió carne ni bebió vino en toda su vida.

\par 6 Tampoco le gustaba el sabor de la fruta ni dormía en la cama, porque se dedicaba a la oración todos los días y a la súplica a Dios durante toda su vida.

\par 7 Entonces, cuando las multitudes oyeron que un hombre de Dios había aparecido, acudieron a él desde todas las regiones y lugares para escuchar sus instrucciones y recomendaciones vivificantes y para estar seguros de que el Espíritu de Dios hablaba en él.

\par 8 Entonces los grandes que habían acudido a él dijeron: «¿Qué es este poder que descendió sobre ti después de que el brillo de tu vista te abandonó, y cómo has tenido un respiro para ver ahora?»

\par 9 Entonces el fiel anciano sonrió y les dijo: A los que se han presentado les diré que Dios me sanó cuando vio que me había acercado a la puerta de la muerte.

\par 10 Él me concedió este honor en mi vejez para que pudiera ser sacerdote del Señor.

\par 11 Entonces alguien (¿Jacob?) le dijo: «Empieza a hablarme para que me consuele y me aferre a él».

\par 12 Entonces nuestro padre Isaac le dijo: «Si hablas con ira, guárdate de la calumnia y guardate de la vanidad.

\par 13 Mira que no hables solo (con una mujer).

\par 14 Tened cuidado de que de vuestra boca no salga mala palabra.

\par 15 Guarda tu cuerpo para que sea puro, porque es templo del Espíritu Santo que habita en él.

\par 16 Cuida las funciones menores de tu cuerpo, para que sea puro y santificado.

\par 17 Cuídate de no jugar con tu lengua, no sea que de tu boca salga una mala palabra.

\par 18 «Cuídense de extender la mano hacia lo que no poseen.

\par 19 No presentéis ofrenda si no estáis ritualmente limpios; báñate en agua cuando pretendas acercarte al altar.

\par 20 No mezcles tus pensamientos con los pensamientos del mundo mientras estás ante el altar en presencia de Dios.

\par 21 Haz tu ofrenda para ser pacificador entre los hombres.

\par 22 Cuando estés a punto de presentar tu ofrenda a Dios, cuando te hayas acercado al altar, orarás a Dios cien veces sin cesar.

\par 23 «Al principio expresarás esta acción de gracias de la siguiente manera: «Oh Dios, el incomprensible, que no puede ser buscado, el poseedor del poder, la fuente de la pureza, límpiame con tu misericordia, un regalo gratuito tuyo a mi.

\par 24 Porque soy un ser de carne y hueso que huyo hacia vosotros.

\par 25 Sé de mi inmundicia y ciertamente tú me limpiarás, oh Señor.

\par 26 Porque he aquí, mi causa está en tus manos y a ti recurro.

\par 27 Conozco mi pecado; límpiame, Señor, para que pueda entrar en tu presencia con dignidad.

\par 28 Ahora mis delitos son graves; Me he acercado al fuego que arde.

\par 29 Tu misericordia está sobre todas las cosas, para que puedas quitar todas mis transgresiones.

\par 30 Perdóname, también a mí, pecador.

\par 31 Y perdona a todas tus criaturas que tú creaste, pero que no oyeron ni aprendieron de ti.

\par 32 ««Soy como todos los que son a tu imagen. He recurrido a hacer lo que me está prohibido.

\par 33 He venido a ti y soy tu siervo y el hijo pecador de tu nación, pero tú eres el que perdona.

\par 34 Perdóname por la misericordia que viene de ti y escucha mi súplica para que sea digno de estar ante tu santo altar.

\par 35 Que este holocausto os sea aceptable.

\par 36 No me hagas volver a mi ignorancia a causa de mis pecados. Recíbeme como a la oveja descarriada.

\par 37 Que el Dios que proveyó a nuestro padre Adán, a Abel, a Noé y a nuestro padre Abraham, esté contigo, oh Jacob, y también conmigo.

\par 38 Reciban de mí mi ofrenda.»

\par 39 «Si, pues, te acercaste y hiciste esto antes de subir al altar, ofrece tu sacrificio.

\par 40 Pero vosotros tendréis cuidado y estar alerta para no contristar el espíritu del Señor.

\par 41 Porque la obra del sacerdocio no es fácil, ya que a cada sacerdote le corresponde, desde hoy hasta el fin de las generaciones y el fin del mundo, no saciarse bebiendo vino ni se saciará comiendo pan; y que no debería hablar de

\par 42 las preocupaciones del mundo ni escuchar a quien habla de ellas. Pero los sacerdotes deben dedicar todos sus esfuerzos y su vida a la oración, a la vigilancia y a la perseverancia en la piedad, para que cada uno pueda pedir al Señor con éxito.

\par 43 «Además, a todo hombre en la tierra, ya sea desdichado o afortunado, le corresponde guardar los mandamientos apropiados.

\par 44 Los hombres, después de poco tiempo, serán alejados de este mundo y de su intensa ansiedad.

\par 45 Entonces, por causa de la pureza, se dedicarán al servicio santo y angelical.

\par 46 Por sus ofrendas puras y su servicio angelical, serán presentados ante el Señor y sus ángeles.

\par 47 Porque su comportamiento terrenal se reflejará en el cielo, y los ángeles serán sus amigos debido a su perfecta fe y pureza.

\par 48 Grande es su estima ante el Señor, y no hay nadie, ni pequeño ni grande, en quien el Señor no mejore; porque el Señor quiere que cada uno sea sin culpa ni ofensa.

\par 49 «Y ahora, continúa suplicando a Dios con arrepentimiento por tus pecados pasados, y no cometas más pecados.

\par 50 Por tanto, no matéis con la espada, no matéis con la lengua, no fornicéis con vuestro cuerpo y no os quedéis enojados hasta la puesta del sol.

\par 51 No recibas elogios injustificados ni te alegres de la caída de tus enemigos o de tus hermanos.

\par 52 No blasfemes; cuidado con la calumnia.

\par 53 No mires a una mujer con ojos lujuriosos.

\par 54 De estas cosas y de sus semejantes os guardaréis, para que cada uno de vosotros se salve de la ira que se manifestará desde el cielo. »

\chapter{5}

\par 1 Cuando la multitud que los rodeaba oyó esto, gritaron a una, diciendo: «En verdad, todo lo que este venerable ha dicho es digno de atención».

\par 2 Pero él permaneció en silencio, se recogió el manto y se cubrió el rostro.

\par 3 La asamblea y el sacerdote que estaba presente, después de un silencio, dijeron: «Déjenlo descansar un poco».

\par 4 Entonces el ángel de Dios se le acercó y lo llevó al cielo.

\par 5 Allí vio ciertas cosas con miedo.

\par 6 Muchas bestias salvajes(?) estaban al alcance de la mano.

\par 7 Los lados . . . (?) como los hermanos para que no pudieran verse unos a otros.

\par 8 Sus rostros eran como rostros de camellos y algunos parecían rostros de perros.

\par 9 Otros tenían caras de leones, hienas y tigres; y algunos tenían un solo ojo.

\par 10 Isaac dijo: «Miré y vi que se habían puesto de acuerdo sobre una persona y la estaban apurando.

\par 11 Y cuando hicieron una señal a los leones, los que caminaban con él se alejaron de él.

\par 12 Entonces los leones se volvieron contra él, lo desgarraron por la mitad, lo desmembraron, lo masticaron y lo tragaron.

\par 13 Después de esto lo expulsaron de su boca y volvió a su estado original.

\par 14 Y detrás de los leones se acercaron los demás y le hicieron lo mismo.

\par 15 Uno tras otro lo tomarían, y cada uno de ellos lo masticaría, lo tragaría y lo expulsaría, y volvería a su estado original.

\par 16 Entonces dije al ángel: Señor mío, ¿cuál es el pecado que ha cometido este hombre para tener que soportar una carga como ésta?

\par 17 El ángel me dijo: «Esto se debe a que este hombre que ves estuvo enemistado con su prójimo durante cinco horas y murió sin haberse reconciliado con él.

\par 18 Entonces fue entregado a cinco de los verdugos para que lo atormentaran durante un año entero por cada una de las cinco horas que pasó como enemigo de su amigo.

\par 19 Entonces el ángel me dijo: «Oh mi amado Isaac, mira aquí las sesenta miríadas que infligen tortura por cada hora que el hombre permanece hostil hacia su prójimo.

\par 20 Es traído aquí a estas criaturas que lo torturan, cada una de ellas durante una hora hasta que se cumpla un año completo, si no hubiera estado haciendo las paces y arrepintiéndose de su pecado antes de su remoción y separación de su cuerpo.

\par 21 Luego me llevó a un río de fuego. Lo vi palpitar y sus olas se elevaban hasta unos treinta codos; y su sonido era como el de un trueno.

\par 22 Vi muchas almas sumergidas en él hasta una profundidad de unos nueve codos.

\par 23 Los que estaban en aquel río lloraban y gritaban a gran voz y con grandes gemidos.

\par 24 Y ese río tenía sabiduría en su fuego: no dañaría a los justos, sino sólo a los pecadores al quemarlos.

\par 25 Los quemaría a todos a causa del hedor y la repugnancia del olor que rodeaba a los pecadores.

\par 26 ¿Entonces observé el río profundo? cuyo humo había subido ante mí, y vi un grupo de personas en el fondo, gritando, llorando, cada uno de ellos lamentándose.

\par 27 El ángel me dijo: «Mira hacia el fondo para observar a los que ves en lo más profundo. Ellos son los que han cometido el pecado de Sodoma; En verdad, se les debía un castigo drástico. »

\par 28 Entonces vi al supervisor del castigo y él era todo él fuego.

\par 29 Golpeaba a los mirmidones del infierno (sus ayudantes) y les decía: «Matenlos, para que se sepa que Dios existe para siempre».

\par 30 Entonces el ángel me dijo: «Alza tus ojos y mira toda la gama de castigos».

\par 31 Pero dije al ángel: «Mi vista no puede abarcarlos a causa de su gran número; pero deseo entender cuánto tiempo esta gente estará en esta tortura».

\par 32 Él me dijo: «Hasta que el Dios de misericordia se vuelva misericordioso y tenga misericordia de ellos».

\chapter{6}

\par 1 Después de esto, el ángel me llevó al cielo y vi a Abraham.

\par 2 Entonces me postré ante él y él y todos los santos me recibieron amablemente.

\par 3 Entonces se reunieron todos y me honraron a causa de mi padre.

\par 4 Entonces me tomaron de la mano y me llevaron hasta el velo delante del trono del Padre.

\par 5 Entonces me postré ante él y lo adoré con mi padre y todos los santos, mientras pronunciábamos alabanzas y clamábamos en voz alta, diciendo: «¡Santísimo, santísimo, santísimo es el Señor Sebaot! El cielo y la tierra están llenos de tu gloria santificada».

\par 6 Entonces el Señor me dijo desde su altura santa: «A todo aquel que ponga a su hijo el nombre de mi amado Isaac, mi bendición reposará sobre él y estará en su casa para siempre.

\par 7 Excelente es tu venida, oh Abraham, fiel; excelente es vuestro linaje, y excelente es la presencia aquí de este bendito linaje.

\par 8 Ahora pues, todo lo que pidas en el nombre de tu amado hijo Isaac lo tendrás hoy como pacto para siempre.

\par 9 Entonces mi padre Abraham respondió y dijo: «Tuyo es el imperio, oh Señor, gobernante del universo».

\par 10 El Señor desde su altura santa dijo a mi padre Abraham: Todo aquel que llame a su hijo por el nombre de mi amado Isaac o escriba su propio testamento, tendrá una bendición que no tendrá fin, y mi bendición sobre su casa no cesará.

\par 11 O si alguno le da de comer a un pobre el día de la fiesta de mi amado Isaac, yo te lo daré a ti en mi reino.

\par 12 Entonces mi padre Abraham dijo: «Oh Padre, Dios, gobernante del universo, aunque no pueda escribir su testamento o su pacto, deja que tu bendición y tu misericordia lo envuelvan, porque tú eres el misericordioso. »

\par 13 El Señor dijo a Abraham: Deja que alimente con pan al hambriento y yo le daré un lugar en mi reino y estará presente contigo desde el primer momento del banquete milenario.

\par 14 El Dios salvador también dijo a mi padre Abraham: «Y si es tan pobre que no encuentra pan en su casa, entonces que pase una noche entera conmemorando a mi amado Isaac sin dormir y le daré una herencia en mi reino».

\par 15 Mi padre Abraham dijo: «Y si está débil y no puede soportar la vigilia, que tu misericordia y compasión aún lo envuelvan».

\par 16 Entonces el Señor le dijo: «Entonces ofrecerá un poco de incienso en mi nombre en el día en memoria de mi amado Isaac, tu hijo.

\par 17 Y si no sabe leer, que vaya a oír la lectura de alguien que sepa leer.

\par 18 Si no puede hacer ninguna de estas cosas, que entre en su casa, cierre la puerta con llave y rece cien oraciones de arrepentimiento; entonces te lo daré por hijo en mi reino.

\par 19 Pero, además de todo esto, que traiga una ofrenda en el día conmemorativo de mi amado Isaac.

\par 20 Y a todos los que hagan todo lo que he dicho se les concederá la herencia del reino en mi cielo.

\par 21 Y todos los que se esforzaron en escribir sus testamentos, sus convenios y sus historias de vida, y mostraron misericordia aunque sólo fuera (dando) un vaso de agua fría, y creyeron con todo su corazón, con ellos estará mi fuerza y ​​mi Espíritu Santo para la prosperidad de sus asuntos en el mundo.

\par 22 No habrá ningún problema en su partida (de este mundo), les concederé toda la vida en mi reino y estarán presentes desde el primer momento del banquete milenario.

\par 23 ¡La paz sea con vosotros, amados míos, los santos!

\par 24 Cuando terminó todo este discurso, los seres celestiales comenzaron a gritar, diciendo: «¡Santísimo, santísimo, santísimo es el Señor, Sabaoth! El cielo y la tierra están llenos de tu gloria santificada».

\par 25 El Padre que todo lo controla respondió desde este lugar santo y dijo: «Oh Miguel, mi siervo fiel, llama a todos los ángeles y a todos los santos».

\par 26 Luego subió al carro de los serafines, mientras los querubines iban delante [con los ángeles.

\par 27 Y cuando llegaron al lecho de nuestro padre Isaac, nuestro padre Isaac inmediatamente vio el rostro de nuestro Señor, lleno de alegría hacia él.

\par 28 Y gritó: «¡Bien que hayas venido, Señor mío, con tu gran arcángel Miguel! Bien es que hayas venido, Padre mío, con todos los santos.»].?

\par 29 Al decir esto, Jacob se turbó mucho y, acercándose a su padre, lo besó llorando.

\par 30 Entonces nuestro padre Isaac lo levantó y le hizo una señal con los ojos, diciendo: «Cállate, muchacho».

\par 31 Entonces Abraham dijo al Señor: «Oh Señor, acuérdate también de mi (nieto) Jacob».

\par 32 Entonces el Señor le dijo: «Mi poder estará con él, glorificará mi nombre, será dueño de la tierra prometida y el enemigo no podrá dominarlo».

\par 33 Y nuestro padre Isaac dijo: Jacob, hijo amado, cumple el mandato que te he dado hoy para que conserves mi cuerpo.

\par 34 No profanéis la imagen de Dios con vuestra forma de tratarla; porque la imagen del hombre fue hecha a imagen de Dios; y Dios os tratará como corresponde en el momento en que lo encontréis y lo veáis cara a cara.

\par 35 Él es el primero y el último, como dijeron los profetas.

\chapter{7}

\par 1 Cuando Isaac dijo esto, el Señor sacó su alma de su cuerpo y quedó blanca como la nieve; tomó posesión de él y lo llevó consigo en su santo carro y ascendió con él a los cielos, mientras los querubines cantaban alabanzas delante de él, y también sus santos ángeles.

\par 2 El Señor le concedió el reino de los cielos; y todo lo que nuestro padre deseaba de la abundancia de bendiciones de Dios que tenía, incluso el cumplimiento de su pacto para siempre.

\chapter{8}

\par 1 Tal fue la muerte de nuestro padre Abraham y de nuestro padre Isaac, hijo de Abraham, el día veintiocho del mes de Misri, en este mismo día. Este día lo hemos consagrado y designado.

\par 2 Y el día en que nuestro padre Abraham ofreció el sacrificio a Dios, el día veintiocho del mes de Amshir, los cielos y la tierra se llenaron del dulce aroma de su estilo de vida delante del Señor.

\par 3 Y nuestro padre Isaac era como la plata que se quema, se funde, se purifica y se refina en el fuego; así también todos los que procederán de nuestro padre Isaac, padre de padres.

\par 4 El día que Abraham, padre de padres, lo ofreció en sacrificio a Dios, el perfume de su sacrificio subió hasta el velo del velo del que todo lo controla.

\par 5 Bienaventurado todo aquel que muestra misericordia en el día conmemorativo del padre de nuestros padres, nuestro padre Abraham y nuestro padre Isaac, porque cada uno de ellos tendrá una morada en el reino de los cielos, porque nuestro Señor ha hecho con ellos su verdadero pacto para siempre.

\par 6 Y lo guardará para ellos y para los que vendrán después de ellos, diciéndoles: Cualquiera que haya manifestado misericordia en el nombre de mi amado Isaac, he aquí os lo daré en el reino de los cielos y él Estaremos presentes con ellos en el primer momento del banquete milenario para celebrar con ellos en la luz eterna en el reino de nuestro Maestro y nuestro Dios y nuestro Rey y nuestro Salvador, Jesús el Mesías.

\par 7 A él se le deben la gloria, la dignidad, la majestad, el dominio, la reverencia, el honor, la alabanza y la adoración, junto con el Padre misericordioso y el Espíritu Santo, ahora y por siempre. , y por toda la eternidad y por los siglos de los siglos, ¡amén!»

\chapter{9}

\par 1 Se terminaron las exequias de nuestro padre Isaac. Gracias y alabanzas a Dios, siempre, por los siglos y eternamente.



\end{document}