\begin{document}

\title{Carta de Jeremías}

\chapter{1}


\par 1 Copia de la epístola que Jeremías envió a los que el rey de los babilonios había de llevar cautivos a Babilonia, para hacerles saber lo que Dios le había ordenado. A causa de los pecados que habéis cometido ante Dios, seréis llevados cautivos a Babilonia por Nabucodonosor, rey de los babilonios.
\par 2 Cuando lleguéis a Babilonia, permaneceréis allí muchos años y por un largo tiempo, es decir, siete generaciones; y después os sacaré de allí en paz.
\par 3 Ahora veréis en Babilonia dioses de plata, de oro y de madera, llevados sobre hombros, que hacen temer a las naciones.
\par 4 Mirad, pues, que no seáis como los extraños, ni seáis vosotros y entre ellos, cuando veáis la multitud delante y detrás de ellos adorándoles.
\par 5 Pero decid en vuestros corazones: Señor, debemos adorarte.
\par 6 Porque mi ángel está con vosotros, y yo mismo cuido de vuestras almas.
\par 7 En cuanto a su lengua, la pule el artífice, y ellos mismos están dorados y cubiertos de plata; sin embargo, son falsos y no pueden hablar.
\par 8 Y tomando oro, como si fuera el de una virgen a la que le encanta andar alegre, hacen coronas para las cabezas de sus dioses.
\par 9 A veces también los sacerdotes reciben de sus dioses oro y plata y se los dan a sí mismos.
\par 10 Y darán de ello a las rameras comunes, y las adornarán como a hombres con vestidos, siendo dioses de plata, y dioses de oro y de madera.
\par 11 Pero estos dioses no pueden salvarse del óxido y de la polilla, aunque estén cubiertos con vestiduras de púrpura.
\par 12 Se limpian la cara a causa del polvo del templo, cuando hay mucho sobre ellos.
\par 13 Y el que no puede matar al que le ofende, empuña un cetro, como si fuera juez de la patria.
\par 14 También tiene en su mano derecha un puñal y un hacha, pero no puede librarse de la guerra ni de los ladrones.
\par 15 Por lo cual se sabe que no son dioses; por tanto, no les temáis.
\par 16 Porque como un vaso que un hombre usa, no vale nada cuando se rompe; Así también sucede con sus dioses: cuando están instalados en el templo, sus ojos se llenan de polvo a causa de los pies de los que entran.
\par 17 Y así como las puertas están aseguradas por todos lados contra el que ofende al rey, como si fuera condenado a muerte, así también los sacerdotes cierran sus templos con puertas, cerraduras y barrotes, para que sus dioses no sean estropeados con ladrones.
\par 18 Encienden velas, incluso más que para ellos mismos, de las cuales no pueden ver ni una sola.
\par 19 Son como las vigas del templo, pero dicen que tienen el corazón roído por cosas que se arrastran desde la tierra; y cuando los comen a ellos y a sus vestidos, no lo sienten.
\par 20 Sus rostros están ennegrecidos por el humo que sale del templo.
\par 21 Sobre sus cuerpos y cabezas se posan murciélagos, golondrinas, pájaros y también gatos.
\par 22 En esto sabréis que no son dioses; por tanto, no les temáis.
\par 23 A pesar del oro que los rodea para embellecerlos, si no limpian el óxido, no brillarán: porque ni cuando se fundieron ni lo sintieron.
\par 24 Las cosas en las que no hay aliento se compran a precio altísimo.
\par 25 Los llevan sobre hombros y no tienen pies, con lo que declaran a los hombres que no valen nada.
\par 26 También los que les sirven se avergüenzan; porque si caen en tierra en algún momento, no pueden levantarse por sí mismos; ni si uno los levanta, no pueden moverse por sí solos, ni si están encorvados. abajo, ¿pueden enderezarse? Pero ponen regalos delante de ellos como a hombres muertos.
\par 27 En cuanto a las cosas que se les sacrifican, sus sacerdotes las venden y abusan; de la misma manera sus mujeres ponen parte del mismo en sal; pero a los pobres e impotentes no les dan nada.
\par 28 Las mujeres que menstrúan y las que están en el parto comen sus sacrificios; por esto sabréis que no son dioses; no les temáis.
\par 29 ¿Cómo pueden llamarse dioses? porque las mujeres ponen ante los dioses la carne de plata, de oro y de madera.
\par 30 Y los sacerdotes se sientan en sus templos, con sus vestidos rasgados, sus cabezas y sus barbas afeitadas, y sin nada sobre sus cabezas.
\par 31 Rugen y lloran ante sus dioses, como lo hacen los hombres en la fiesta cuando uno está muerto.
\par 32 Los sacerdotes también se quitan las vestiduras y visten a sus mujeres y a sus hijos.
\par 33 Ya sea que se les haga mal o bien, no pueden recompensarlo: no pueden ni poner rey ni derribarlo.
\par 34 De la misma manera, tampoco pueden dar riquezas ni dinero; si alguien les hace un voto y no lo cumple, no lo exigirán.
\par 35 No pueden salvar a nadie de la muerte, ni librar al débil del fuerte.
\par 36 No pueden devolver la vista a un ciego ni ayudar a nadie en su aflicción.
\par 37 No podrán tener misericordia de la viuda ni hacer bien al huérfano.
\par 38 Sus dioses de madera, cubiertos de oro y plata, son como piedras talladas en la montaña: quienes los adoran serán avergonzados.
\par 39 ¿Cómo podría entonces el hombre pensar y decir que son dioses, cuando hasta los mismos caldeos los deshonran?
\par 40 Quienes, si ven a un mudo que no puede hablar, lo traen y le ruegan a Bel que pueda hablar, como si pudiera entender.
\par 41 Sin embargo, ellos mismos no pueden entender esto y los abandonan, porque no tienen conocimiento.
\par 42 También las mujeres, atadas con cuerdas, sentadas en los caminos, queman salvado para perfumar; pero si alguna de ellas, atraída por alguno que pasa, se acuesta con él, reprocha a su compañero no haber sido considerada digna. como ella misma, ni su cordón roto.
\par 43 Todo lo que se hace entre ellos es falso: ¿cómo se puede entonces pensar o decir que son dioses?
\par 44 Están hechos de carpinteros y orfebres: no pueden ser más que lo que los trabajadores quieren que sean.
\par 45 Y los que los hicieron no podrán durar mucho tiempo; ¿Cómo, pues, deberían ser dioses las cosas que están hechas de ellos?
\par 46 Porque dejaron mentiras y afrentas a los que vinieron después.
\par 47 Porque cuando les sobreviene una guerra o una plaga, los sacerdotes consultan entre sí dónde esconderse con ellos.
\par 48 ¿Cómo, pues, los hombres no pueden darse cuenta de que no son dioses, si no pueden salvarse de la guerra ni de la peste?
\par 49 Porque, como son de madera y están recubiertas de plata y oro, en lo sucesivo se sabrá que son falsas.
\par 50 Y a todas las naciones y reyes les parecerá claramente que no son dioses, sino obras de manos de hombres, y que no hay en ellos obra de Dios.
\par 51 ¿Quién, pues, no sabrá que no es un dios?
\par 52 Porque no pueden erigir rey en la tierra ni dar lluvia a los hombres.
\par 53 No pueden juzgar su propia causa ni reparar el mal, porque son como cuervos entre el cielo y la tierra.
\par 54 Entonces, cuando caiga fuego sobre la casa de los dioses, hecha de madera o cubierta de oro o plata, sus sacerdotes huirán y escaparán; pero ellos mismos serán quemados como vigas.
\par 55 Además, no pueden resistir a ningún rey ni a ningún enemigo: ¿cómo se puede entonces pensar o decir que son dioses?
\par 56 Tampoco esos dioses de madera, recubiertos de plata u oro, pueden escapar ni de los ladrones ni de los salteadores.
\par 57 Los fuertes toman el oro, la plata y los vestidos con que se visten y se van sin poder ayudarse a sí mismos.
\par 58 Por lo tanto, es mejor ser un rey que muestra su poder, o un objeto útil en una casa, del cual su dueño pueda usar, que dioses falsos; o ser puerta en una casa, para guardar en ella cosas tales que dioses falsos. o una columna de madera en un palacio, que esos dioses falsos.
\par 59 Porque el sol, la luna y las estrellas, siendo brillantes y enviados a cumplir sus oficios, son obedientes.
\par 60 De la misma manera, el relámpago, cuando estalla, es fácil de ver; y de la misma manera sopla el viento en todos los países.
\par 61 Y cuando Dios ordena a las nubes que recorran el mundo entero, ellas hacen lo que se les ordena.
\par 62 Y el fuego enviado desde arriba para consumir colinas y bosques hace lo que se le ordena, pero estos no se parecen a ellos ni en apariencia ni en poder.
\par 63 Por lo tanto, no se debe suponer ni decir que sean dioses, ya que no pueden ni juzgar las causas ni hacer el bien a los hombres.
\par 64 Sabiendo, pues, que no son dioses, no les temáis,
\par 65 Porque no pueden maldecir ni bendecir a los reyes:
\par 66 Ni pueden hacer señales en los cielos entre los paganos, ni brillar como el sol, ni alumbrar como la luna.
\par 67 Las bestias son mejores que ellos: porque pueden esconderse bajo un manto y ayudarse a sí mismas.
\par 68 Entonces, de ninguna manera nos resulta evidente que sean dioses; por tanto, no les temáis.
\par 69 Porque como un espantapájaros en un huerto de pepinos no guarda nada, así son sus dioses de madera, cubiertos de plata y oro.
\par 70 Y de la misma manera sus dioses de madera, cubiertos de plata y oro, son como un espino blanco en un huerto, en el que se posa todo pájaro; como también a un cadáver, que está al este en la oscuridad.
\par 71 Y sabréis que no son dioses por la brillante púrpura que se pudre sobre ellos; y ellos mismos después serán comidos y serán un oprobio en el país.
\par 72 Mejor es, pues, el justo que no tiene ídolos, porque estará lejos de toda mancha.



\end{document}