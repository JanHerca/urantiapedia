\begin{document}
\chapter{1}

La esposa y las hijas de Jerusalén  
1:1 Cantar de los cantares, el cual es de Salomón. 
1:2 ¡Oh, si él me besara con besos de su boca!  
Porque mejores son tus amores que el vino.  
1:3 A más del olor de tus suaves ungüentos,  
Tu nombre es como ungüento derramado;  
Por eso las doncellas te aman.  
1:4 Atráeme; en pos de ti correremos.  
El rey me ha metido en sus cámaras;  
Nos gozaremos y alegraremos en ti;  
Nos acordaremos de tus amores más que del vino;  
Con razón te aman.  
1:5 Morena soy, oh hijas de Jerusalén, pero codiciable  
Como las tiendas de Cedar,  
Como las cortinas de Salomón.  
1:6 No reparéis en que soy morena,  
Porque el sol me miró.  
Los hijos de mi madre se airaron contra mí;  
Me pusieron a guardar las viñas;  
Y mi viña, que era mía, no guardé.  
1:7 Hazme saber, oh tú a quien ama mi alma,  
Dónde apacientas, dónde sesteas al mediodía;  
Pues ¿por qué había de estar yo como errante  
Junto a los rebaños de tus compañeros?  
1:8 Si tú no lo sabes, oh hermosa entre las mujeres,  
Ve, sigue las huellas del rebaño,  
Y apacienta tus cabritas junto a las cabañas de los pastores. 
La esposa y el esposo  
1:9 A yegua de los carros de Faraón  
Te he comparado, amiga mía.  
1:10 Hermosas son tus mejillas entre los pendientes,  
Tu cuello entre los collares.  
1:11 Zarcillos de oro te haremos,  
Tachonados de plata.  
1:12 Mientras el rey estaba en su reclinatorio,  
Mi nardo dio su olor.  
1:13 Mi amado es para mí un manojito de mirra,  
Que reposa entre mis pechos.  
1:14 Racimo de flores de alheña en las viñas de En-gadi  
Es para mí mi amado.  
1:15 He aquí que tú eres hermosa, amiga mía; 
He aquí eres bella; tus ojos son como palomas.  
1:16 He aquí que tú eres hermoso, amado mío, y dulce;  
Nuestro lecho es de flores.  
1:17 Las vigas de nuestra casa son de cedro,  
Y de ciprés los artesonados.  

\chapter{2}


2:1 Yo soy la rosa de Sarón,  
Y el lirio de los valles.  
2:2 Como el lirio entre los espinos,  
Así es mi amiga entre las doncellas.  
2:3 Como el manzano entre los árboles silvestres,  
Así es mi amado entre los jóvenes;  
Bajo la sombra del deseado me senté,  
Y su fruto fue dulce a mi paladar.  
2:4 Me llevó a la casa del banquete,  
Y su bandera sobre mí fue amor.  
2:5 Sustentadme con pasas, confortadme con manzanas;  
Porque estoy enferma de amor.  
2:6 Su izquierda esté debajo de mi cabeza,  
Y su derecha me abrace.  
2:7 Yo os conjuro, oh doncellas de Jerusalén,  
Por los corzos y por las ciervas del campo,  
Que no despertéis ni hagáis velar al amor,  
Hasta que quiera.  
2:8 ¡La voz de mi amado! He aquí él viene  
Saltando sobre los montes,  
Brincando sobre los collados.  
2:9 Mi amado es semejante al corzo,  
O al cervatillo.  
Helo aquí, está tras nuestra pared,  
Mirando por las ventanas,  
Atisbando por las celosías.  
2:10 Mi amado habló, y me dijo:  
Levántate, oh amiga mía, hermosa mía, y ven.  
2:11 Porque he aquí ha pasado el invierno,  
Se ha mudado, la lluvia se fue;  
2:12 Se han mostrado las flores en la tierra,  
El tiempo de la canción ha venido,  
Y en nuestro país se ha oído la voz de la tórtola.  
2:13 La higuera ha echado sus higos,  
Y las vides en cierne dieron olor;  
Levántate, oh amiga mía, hermosa mía, y ven.  
2:14 Paloma mía, que estás en los agujeros de la peña, en lo escondido de escarpados parajes,  
Muéstrame tu rostro, hazme oír tu voz;  
Porque dulce es la voz tuya, y hermoso tu aspecto.  
2:15 Cazadnos las zorras, las zorras pequeñas, que echan a perder las viñas;  
Porque nuestras viñas están en cierne.  
2:16 Mi amado es mío, y yo suya;  
El apacienta entre lirios.  
2:17 Hasta que apunte el día, y huyan las sombras,  
Vuélvete, amado mío; sé semejante al corzo, o como el cervatillo  
Sobre los montes de Beter. 

\chapter{3}

El ensueño de la esposa  

3:1 Por las noches busqué en mi lecho al que ama mi alma;  
Lo busqué, y no lo hallé.  
3:2 Y dije: Me levantaré ahora, y rodearé por la ciudad;  
Por las calles y por las plazas  
Buscaré al que ama mi alma; 
Lo busqué, y no lo hallé.  
3:3 Me hallaron los guardas que rondan la ciudad,  
Y les dije: ¿Habéis visto al que ama mi alma?  
3:4 Apenas hube pasado de ellos un poco,  
Hallé luego al que ama mi alma;  
Lo así, y no lo dejé,  
Hasta que lo metí en casa de mi madre,  
Y en la cámara de la que me dio a luz.  
3:5 Yo os conjuro, oh doncellas de Jerusalén,  
Por los corzos y por las ciervas del campo,  
Que no despertéis ni hagáis velar al amor,  
Hasta que quiera.  
El cortejo de bodas 
3:6 ¿Quién es ésta que sube del desierto como columna de humo,  
Sahumada de mirra y de incienso  
Y de todo polvo aromático?  
3:7 He aquí es la litera de Salomón;  
Sesenta valientes la rodean,  
De los fuertes de Israel.  
3:8 Todos ellos tienen espadas, diestros en la guerra; 
Cada uno su espada sobre su muslo,  
Por los temores de la noche.  
3:9 El rey Salomón se hizo una carroza  
De madera del Líbano.  
3:10 Hizo sus columnas de plata,  
Su respaldo de oro,  
Su asiento de grana,  
Su interior recamado de amor  
Por las doncellas de Jerusalén.  
3:11 Salid, oh doncellas de Sion, y ved al rey Salomón  
Con la corona con que le coronó su madre en el día de su desposorio,  
Y el día del gozo de su corazón. 

\chapter{4}

El esposo alaba a la esposa  

4:1 He aquí que tú eres hermosa, amiga mía; he aquí que tú eres hermosa;  
Tus ojos entre tus guedejas como de paloma;  
Tus cabellos como manada de cabras  
Que se recuestan en las laderas de Galaad.  
4:2 Tus dientes como manadas de ovejas trasquiladas,  
Que suben del lavadero,  
Todas con crías gemelas,  
Y ninguna entre ellas estéril.  
4:3 Tus labios como hilo de grana,  
Y tu habla hermosa;  
Tus mejillas, como cachos de granada detrás de tu velo.  
4:4 Tu cuello, como la torre de David, edificada para armería;  
Mil escudos están colgados en ella,  
Todos escudos de valientes.  
4:5 Tus dos pechos, como gemelos de gacela,  
Que se apacientan entre lirios.  
4:6 Hasta que apunte el día y huyan las sombras,  
Me iré al monte de la mirra,  
Y al collado del incienso.  
4:7 Toda tú eres hermosa, amiga mía,  
Y en ti no hay mancha.  
4:8 Ven conmigo desde el Líbano, oh esposa mía;  
Ven conmigo desde el Líbano.  
Mira desde la cumbre de Amana,  
Desde la cumbre de Senir y de Hermón,  
Desde las guaridas de los leones,  
Desde los montes de los leopardos.  
4:9 Prendiste mi corazón, hermana, esposa mía;  
Has apresado mi corazón con uno de tus ojos,  
Con una gargantilla de tu cuello.  
4:10 ¡Cuán hermosos son tus amores, hermana, esposa mía!  
¡Cuánto mejores que el vino tus amores,  
Y el olor de tus ungüentos que todas las especias aromáticas! 
4:11 Como panal de miel destilan tus labios, oh esposa;  
Miel y leche hay debajo de tu lengua;  
Y el olor de tus vestidos como el olor del Líbano.  
4:12 Huerto cerrado eres, hermana mía, esposa mía;  
Fuente cerrada, fuente sellada.  
4:13 Tus renuevos son paraíso de granados, con frutos suaves,  
De flores de alheña y nardos;  
4:14 Nardo y azafrán, caña aromática y canela,  
Con todos los árboles de incienso;  
Mirra y áloes, con todas las principales especias aromáticas.  
4:15 Fuente de huertos,  
Pozo de aguas vivas,  
Que corren del Líbano.  
4:16 Levántate, Aquilón, y ven, Austro;  
Soplad en mi huerto, despréndanse sus aromas.  
Venga mi amado a su huerto,  
Y coma de su dulce fruta.  

\chapter{5}


5:1 Yo vine a mi huerto, oh hermana, esposa mía;  
He recogido mi mirra y mis aromas;  
He comido mi panal y mi miel,  
Mi vino y mi leche he bebido.  
Comed, amigos; bebed en abundancia, oh amados.  
El tormento de la separación  
5:2 Yo dormía, pero mi corazón velaba.  
Es la voz de mi amado que llama:  
Abreme, hermana mía, amiga mía, paloma mía, perfecta mía,  
Porque mi cabeza está llena de rocío,  
Mis cabellos de las gotas de la noche.  
5:3 Me he desnudado de mi ropa; ¿cómo me he de vestir?  
He lavado mis pies; ¿cómo los he de ensuciar?  
5:4 Mi amado metió su mano por la ventanilla,  
Y mi corazón se conmovió dentro de mí.  
5:5 Yo me levanté para abrir a mi amado,  
Y mis manos gotearon mirra,  
Y mis dedos mirra, que corría  
Sobre la manecilla del cerrojo.  
5:6 Abrí yo a mi amado;  
Pero mi amado se había ido, había ya pasado;  
Y tras su hablar salió mi alma.  
Lo busqué, y no lo hallé;  
Lo llamé, y no me respondió.  
5:7 Me hallaron los guardas que rondan la ciudad;  
Me golpearon, me hirieron;  
Me quitaron mi manto de encima los guardas de los muros.  
5:8 Yo os conjuro, oh doncellas de Jerusalén, si halláis a mi amado,  
Que le hagáis saber que estoy enferma de amor.  
La esposa alaba al esposo  
5:9 ¿Qué es tu amado más que otro amado,  
Oh la más hermosa de todas las mujeres?  
¿Qué es tu amado más que otro amado,  
Que así nos conjuras?  
5:10 Mi amado es blanco y rubio,  
Señalado entre diez mil.  
5:11 Su cabeza como oro finísimo;  
Sus cabellos crespos, negros como el cuervo.  
5:12 Sus ojos, como palomas junto a los arroyos de las aguas,  
Que se lavan con leche, y a la perfección colocados.  
5:13 Sus mejillas, como una era de especias aromáticas, como fragantes flores;  
Sus labios, como lirios que destilan mirra fragante.  
5:14 Sus manos, como anillos de oro engastados de jacintos;  
Su cuerpo, como claro marfil cubierto de zafiros.  
5:15 Sus piernas, como columnas de mármol fundadas sobre basas de oro fino;  
Su aspecto como el Líbano, escogido como los cedros.  
5:16 Su paladar, dulcísimo, y todo él codiciable.  
Tal es mi amado, tal es mi amigo,  
Oh doncellas de Jerusalén.  

\chapter{6}

Mutuo encanto del esposo y de la esposa  

6:1 ¿A dónde se ha ido tu amado, oh la más hermosa de todas las mujeres?  
¿A dónde se apartó tu amado,  
Y lo buscaremos contigo?  
6:2 Mi amado descendió a su huerto, a las eras de las especias,  
Para apacentar en los huertos, y para recoger los lirios.  
6:3 Yo soy de mi amado, y mi amado es mío;  
El apacienta entre los lirios.  
6:4 Hermosa eres tú, oh amiga mía, como Tirsa;  
De desear, como Jerusalén;  
Imponente como ejércitos en orden.  
6:5 Aparta tus ojos de delante de mí,  
Porque ellos me vencieron.  
Tu cabello es como manada de cabras  
Que se recuestan en las laderas de Galaad.  
6:6 Tus dientes, como manadas de ovejas que suben del lavadero,  
Todas con crías gemelas,  
Y estéril no hay entre ellas.  
6:7 Como cachos de granada son tus mejillas  
Detrás de tu velo.  
6:8 Sesenta son las reinas, y ochenta las concubinas,  
Y las doncellas sin número;  
6:9 Mas una es la paloma mía, la perfecta mía;  
Es la única de su madre,  
La escogida de la que la dio a luz.  
La vieron las doncellas, y la llamaron bienaventurada;  
Las reinas y las concubinas, y la alabaron.  
6:10 ¿Quién es ésta que se muestra como el alba,  
Hermosa como la luna,  
Esclarecida como el sol,  
Imponente como ejércitos en orden?  
6:11 Al huerto de los nogales descendí  
A ver los frutos del valle,  
Y para ver si brotaban las vides,  
Si florecían los granados.  
6:12 Antes que lo supiera, mi alma me puso  
Entre los carros de Aminadab.  
6:13 Vuélvete, vuélvete, oh sulamita;  
Vuélvete, vuélvete, y te miraremos.  
¿Qué veréis en la sulamita?  
Algo como la reunión de dos campamentos.  

\chapter{7}


7:1 ¡Cuán hermosos son tus pies en las sandalias,  
Oh hija de príncipe!  
Los contornos de tus muslos son como joyas,  
Obra de mano de excelente maestro.  
7:2 Tu ombligo como una taza redonda  
Que no le falta bebida.  
Tu vientre como montón de trigo  
Cercado de lirios.  
7:3 Tus dos pechos, como gemelos de gacela.  
7:4 Tu cuello, como torre de marfil;  
Tus ojos, como los estanques de Hesbón junto a la puerta de Bat-rabim;  
Tu nariz, como la torre del Líbano,  
Que mira hacia Damasco.  
7:5 Tu cabeza encima de ti, como el Carmelo;  
Y el cabello de tu cabeza, como la púrpura del rey  
Suspendida en los corredores. 
7:6 ¡Qué hermosa eres, y cuán suave,  
Oh amor deleitoso!  
7:7 Tu estatura es semejante a la palmera,  
Y tus pechos a los racimos.  
7:8 Yo dije: Subiré a la palmera,  
Asiré sus ramas.  
Deja que tus pechos sean como racimos de vid,  
Y el olor de tu boca como de manzanas,  
7:9 Y tu paladar como el buen vino,  
Que se entra a mi amado suavemente,  
Y hace hablar los labios de los viejos.  
7:10 Yo soy de mi amado,  
Y conmigo tiene su contentamiento.  
7:11 Ven, oh amado mío, salgamos al campo,  
Moremos en las aldeas.  
7:12 Levantémonos de mañana a las viñas;  
Veamos si brotan las vides, si están en cierne,  
Si han florecido los granados;  
Allí te daré mis amores.  
7:13 Las mandrágoras han dado olor,  
Y a nuestras puertas hay toda suerte de dulces frutas,  
Nuevas y añejas, que para ti, oh amado mío, he guardado. 

\chapter{8}


8:1 ¡Oh, si tú fueras como un hermano mío  
Que mamó los pechos de mi madre!  
Entonces, hallándote fuera, te besaría,  
Y no me menospreciarían.  
8:2 Yo te llevaría, te metería en casa de mi madre;  
Tú me enseñarías,  
Y yo te haría beber vino  
Adobado del mosto de mis granadas.  
8:3 Su izquierda esté debajo de mi cabeza,  
Y su derecha me abrace.  
8:4 Os conjuro, oh doncellas de Jerusalén,  
Que no despertéis ni hagáis velar al amor,  
Hasta que quiera.  
El poder del amor  
8:5 ¿Quién es ésta que sube del desierto,  
Recostada sobre su amado?  
Debajo de un manzano te desperté;  
Allí tuvo tu madre dolores,  
Allí tuvo dolores la que te dio a luz.  
8:6 Ponme como un sello sobre tu corazón, como una marca sobre tu brazo;  
Porque fuerte es como la muerte el amor;  
Duros como el Seol los celos;  
Sus brasas, brasas de fuego, fuerte llama.  
8:7 Las muchas aguas no podrán apagar el amor,  
Ni lo ahogarán los ríos.  
Si diese el hombre todos los bienes de su casa por este amor,  
De cierto lo menospreciarían.  
8:8 Tenemos una pequeña hermana,  
Que no tiene pechos;  
¿Qué haremos a nuestra hermana  
Cuando de ella se hablare?  
8:9 Si ella es muro,  
Edificaremos sobre él un palacio de plata;  
Si fuere puerta,  
La guarneceremos con tablas de cedro.  
8:10 Yo soy muro, y mis pechos como torres,  
Desde que fui en sus ojos como la que halla paz.  
8:11 Salomón tuvo una viña en Baal-hamón,  
La cual entregó a guardas,  
Cada uno de los cuales debía traer mil monedas de plata por su fruto.  
8:12 Mi viña, que es mía, está delante de mí;  
Las mil serán tuyas, oh Salomón,  
Y doscientas para los que guardan su fruto.  
8:13 Oh, tú que habitas en los huertos,  
Los compañeros escuchan tu voz;  
Házmela oír.  
8:14 Apresúrate, amado mío,  
Y sé semejante al corzo, o al cervatillo,  
Sobre las montañas de los aromas.

\end{document}