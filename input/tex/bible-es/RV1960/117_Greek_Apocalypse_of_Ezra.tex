\begin{document}

\title{Apocalipsis griego de Esdras}

\chapter{1}

\par \textit{Introducción y oración}

\par 1 Aconteció que el día veintidós del mes del año treinta, yo estaba en mi casa y clamé, diciendo al Altísimo: Señor, concédeme gloria para que

\par 2 [...]


\par 3 para que pueda ver tus misterios. Cuando cayó la noche, vino el ángel Miguel, el arcángel, y me dijo: «Profeta Esdras, reserva pan para setenta semanas». Y

\par 4 Ayuné tal como él me dijo. Y vino el archistrategos Rafael y me dio un bastón de estoraque, y ayuné dos veces sesenta semanas, y vi los misterios de Dios.

\par 5 [...]

\par 6 y sus ángeles. Y les dije: «Quiero suplicar a Dios por el pueblo cristiano. Más vale que el hombre no nazca que que entre en el mundo».

\par \textit{Esdras llevada al cielo: su oración pidiendo misericordia}

\par 7 Por tanto, fui llevado al cielo y vi en el primer cielo un gran

\par 8 orden de los ángeles y ellos me llevaron a los juicios. Y escuché una voz que decía

\par 9 a mí: «Ten piedad de nosotros, Esdras, el elegido de Dios». Entonces comencé a decir: «¡Ay de los pecadores, cuando vean al justo (elevado) por encima de los ángeles, y sean

\par 10 por la ardiente Gehena». Y Esdras dijo: Ten piedad de las obras de tus manos,

\par 11 misericordioso y muy compasivo. Condenadme a mí antes que a las almas de los pecadores, porque es mejor castigar a un alma y no llevar al mundo entero a la ruina.

\par 12 destrucción.» Y Dios dijo: «Daré descanso a los justos en el Paraíso y

\par 13 Soy misericordioso». Y Esdras dijo: «Señor, ¿por qué muestras favor a los justos?

\par 14 Porque como el jornalero cumple su tiempo de servicio y se va, y el esclavo sirve nuevamente a sus amos para recibir su salario, así el justo recibe su recompensa en los cielos. Pero ten piedad de los pecadores porque sabemos que

\par 15 [...]

\par 16 tú eres misericordioso». Y Dios dijo: «No tengo manera de ser misericordioso con ellos». Y

\par 17 [...]

\par 18 Esdras dijo: «(Sé misericordioso) porque ellos no pueden soportar tu ira». Y Dios dijo: «(Estoy enojado) porque tales (son los desiertos) de tales (hombres) como estos».

\par 19 Y Dios dijo: «Quiero guardarte como a Pablo y a Juan. Me has dado incorrupto el tesoro inviolable, el tesoro de la virginidad, el muro de los hombres».

\par 20 [...]

\par \textit{La segunda oración de Esdras}

\par 21 Y Esdras dijo: Mejor sería que el hombre no naciera; estaría bien si lo fuera

\par 22 no vivo. Las bestias mudas son mejores que el hombre, porque no tienen

\par 23 castigo. Nos tomaste y nos entregaste al juicio. ¡Ay de los pecadores en el mundo venidero, porque su condenación es interminable y la llama no se apaga!

\chapter{2}

\par \textit{Esdras protesta ante Dios: el pecado de Adán}

\par 1 Mientras le decía esto, se acercaron Miguel y Gabriel y todos los apóstoles y dijeron:

\par 2 «¡Saludos!» Y Esdras dijo: «¡Hombre fiel de Dios! Levántate y ven acá conmigo, oh Señor, al juicio». Y dijo Dios: He aquí, os doy mi

\par 3 [...]

\par 4 [...]

\par 5 pacto, tanto mío como tuyo, para que lo aceptes». Y Esdras dijo: «Nosotros

\par 6 defenderemos nuestro caso en vuestro(s) oído(s)». Y Dios dijo: Pregunta a tu padre Abraham qué clase de hijo demanda a su padre y ven y defiende el caso ante

\par 7 nosotros.» Y Esdras dijo: «Vive el Señor, que nunca dejaré de defender el caso.

\par 8 con vosotros a causa del pueblo cristiano. ¿Dónde están tus antiguas misericordias, oh?

\par 9 ¿Señor? ¿Dónde está tu gran sufrimiento? Y Dios dijo: «Como hice la noche y el día, hice al justo y al pecador, y conviene que te conduzcas como el

\par 10 hombre justo». Y el profeta dijo: «¿Quién hizo a Adán, el protoplasto, el

\par 11 ¿el primero?» Y dijo Dios: «Mis manos inmaculadas, y lo puse en el Paraíso

\par 12 para guardar la región del árbol de la vida». «Ya que el que estableció

\par 13 la desobediencia hizo que este (hombre) pecara». Y el profeta dijo: ¿No estaba él vigilado?

\par 14 por un ángel? ¿Y no fue preservada la vida por los querubines para la era eterna?

\par 15 ¿Y cómo fue engañado aquel que estaba custodiado por ángeles a quienes tú ordenaste?

\par 16 estar presente pase lo que pase? ¡Atiende también a lo que digo! si tuvieras

\par 17 Si no le hubiera dado a Eva, la serpiente nunca la habría engañado. Si salvas a quien quieras, también destruirás a quien quieras».

\par \textit{Esdras protesta ante Dios: los pecados de los hombres}

\par 18 Y el profeta dijo: «Oh Señor mío, pasemos a un segundo juicio».

\par 19 Y Dios dijo: «Arrojo fuego sobre Sodoma y Gomorra». Y el profeta dijo:

\par 20 [...]

\par 21 «Señor, tú traes sobre nosotros lo que merecemos». Y Dios dijo: «Tus pecados exceden

\par 22 mi bondad.» Y el profeta dijo: Acordaos de la Escritura, padre mío, que

\par 23 midió Jerusalén y la reconstruyó. Compadece, Señor, de los pecadores, compadécete de los tuyos.

\par 24 moldea, ten misericordia de tus obras. Entonces Dios se acordó de sus obras y

\par 25 dijo al profeta: «¿Cómo podré tener misericordia de ellos? Les dieron a beber vinagre y hiel y [...] se arrepintieron».

\par \textit{El día del juicio}

\par 26 Y el profeta dijo: «Revelad vuestros querubines y vayamos juntos al juicio,

\par 27 y muéstrame cuál será el carácter del día del juicio. Y Dios dijo: «Tú

\par 28 [...]

\par 29 Te has desviado, Esdras, porque tal es el día del juicio en el que no lloverá.

\par 30 sobre la tierra, porque aquel día habrá juicio misericordioso». Y el profeta dijo: «Nunca dejaré de discutir con vosotros hasta que vea el día del

\par 31 [...]

\par 32 consumación.» (Y Dios dijo:) «Cuenta las estrellas y la arena del mar y si puedes contar esto, también podrás discutir el caso conmigo».

\chapter{3}

\par 1 Y el profeta dijo: «Señor, tú sabes que soy portador de carne humana. ¿Y cómo puede

\par 2 [...]

\par 3 ¿Cuento las estrellas del cielo y la arena del mar?» Y Dios dijo: «Oh mi profeta elegido, nadie sabrá ese gran día y la manifestación que prevalecerá hasta

\par 4 juzgar el mundo. Por ti, profeta mío, te dije el día, pero la hora

\par 5 Te dije que no». Y el profeta dijo: «Señor, dime también los años». Y Dios dijo: «Si veo que la justicia del mundo se ha hecho abundante, seré paciente para con ellos. Si no, extenderé mi mano y agarraré al mundo habitado por sus cuatro confines y los reuniré a todos en el valle de Josafat y exterminaré a la raza humana y el mundo no existirá.

\par 6 [...]

\par 7 más». Y el profeta dijo: «¿Y cómo será glorificada tu diestra?»

\par 8 Y dijo Dios: «Seré glorificado por mis ángeles».

\par \textit{¿Por qué fue creado el hombre?}

\par 9 Y el profeta dijo: «Señor, si este fue tu cálculo, ¿por qué formaste

\par 10 hombre? Dijiste a Abraham nuestro padre: «Ciertamente multiplicaré tu descendencia como las estrellas del cielo y como la arena a la orilla del mar.» ¿Y dónde está tu promesa?

\par \textit{Señales del fin}

\par 11 Y dijo Dios: Primero haré sacudir la caída de los cuadrúpedos y

\par 12 hombres. Y cuando veas que el hermano entrega a la muerte al hermano y a los hijos

\par 13 se levantará contra los padres, y la esposa abandonará a su propio marido; y cuando una nación se levantará contra otra nación en guerra, entonces sabréis que el fin está cerca.

\par 14 Y entonces el hermano no tendrá misericordia del hermano, ni el hombre de su esposa, ni

\par 15 hijos sobre padres, ni amigos sobre amigos, ni esclavo sobre amo. Porque el mismo adversario de los hombres subirá del Tártaro y mostrará muchas cosas.

\par 16 para hombres. ¿Qué te haré, Ezra, y discutirás el caso conmigo?

\chapter{4}

\par \textit{Esdras desciende al Tártaro}

\par 1 Y el profeta dijo: «Señor, nunca dejaré de discutir contigo».

\par 2 Y dijo Dios: Cuenta las flores de la tierra. Si puedes contarlos, también podrás discutir el caso conmigo».

\par 3 [...]

\par 4 Y el profeta dijo: «Señor, no puedo contarlos; llevo carne humana, pero

\par 5 tampoco dejaré de discutir el caso contigo. Deseo, Señor, ver las partes bajas.

\par 6 del Tártaro.» Y Dios dijo: «¡Desciende y mira!» Y me dio michael

\par 7 [...]

\par 8 y Gabriel y otros treinta y cuatro ángeles, y bajé ochenta y cinco escalones y ellos me hicieron descender quinientos escalones.

\par \textit{El castigo de Herodes}

\par 9 Y vi un trono de fuego y a un anciano sentado en él, y su castigo era

\par 10 despiadado. Y dije a los ángeles: «¿Quién es éste y cuál es su pecado?» Y me dijeron,

\par 11 «Este es Herodes, que fue rey por un tiempo, y mandó matar

\par 12 los infantes de dos años o menos.» Y dije: ¡Ay de su alma! Los desobedientes y el abismo

\par 13 Y de nuevo me hicieron bajar treinta escalones. Y vi allí fuegos hirviendo, y un

\par 14 multitud de pecadores en ellos. Y oí sus voces, pero no percibí sus

\par 15 formularios. Y me llevaron a muchos escalones más profundos que no podía contar.

\par 16 Vi allí a unos ancianos con ejes de fuego girando sobre sus orejas.

\par 17 [...]

\par 18 Dije: ¿Quiénes son estos y cuál es su pecado? Y me dijeron: «Estos son

\par 19 los espías.» Y de nuevo me hicieron bajar quinientos escalones más. Y

\par 20 [...]

\par 21 Allí vi el gusano que no dormía y el fuego que consumía a los pecadores. Y me llevaron hasta los cimientos de Apoleia (Destrucción) y allí vi las doce veces

\par 22 golpe del abismo. Y me llevaron hacia el sur y allí vi a un hombre

\par 23 colgaban de sus párpados y los ángeles lo golpeaban. Y le pregunté: «¿Quién

\par 24 ¿Es esto y cuál es su pecado?» Y Miguel el archistrategos me dijo: «Este hombre es incestuoso; habiendo llevado a cabo una pequeña lujuria, a este hombre se le ordenó que lo ahorcaran».

\par \textit{El Anticristo}

\par 25 Y me llevaron hacia el norte y vi allí a un hombre atado con hierro.

\par 26 barras. Y pregunté: «¿Quién es este?» Y me dijo: Éste es el que dice: Yo soy el Hijo de Dios y el que hizo las piedras el pan y el agua el vino.

\par 27 [...]

\par 28 Y el profeta dijo: «Hazme saber qué apariencia tiene y

\par 29 Informaré a la raza humana para que no crean en él. Y él me dijo: «El aspecto de su rostro es como el de un hombre salvaje. Su ojo derecho es como una estrella que se eleva hacia

\par 30 amanece y el otro está inmóvil. Su boca es un codo, sus dientes un palmo.

\par 31 de largo, sus dedos como guadañas, las plantas de sus pies de dos palmos, y en su frente

\par 32 una inscripción «Anticristo». Fue exaltado hasta el cielo, descenderá tan lejos

\par 33 como Hades. Una vez será un niño, otra un anciano». Y el profeta

\par 34 [...]

\par 35 dijo: «Señor, ¿cómo permites que la raza de los hombres se extravíe?» Y Dios dijo: «¡Oye, profeta mío! Se convierte en niño y en anciano y que nadie le crea.

\par 36 él es mi hijo amado. Y después de estas cosas se tocará la trompeta, y los sepulcros serán

\par 37 abierto y los muertos resucitarán incorruptos. Entonces el oponente, habiendo escuchado

\par 38 la terrible amenaza, se esconderá en las tinieblas exteriores. Entonces el cielo y

\par 39 la tierra y el mar perecerán. Entonces quemaré el cielo por ochenta codos

\par 40 y la tierra ochocientos codos. Y el profeta dijo: «¿Y (en) qué

\par 41 ¿pecó el cielo?» Y Dios dijo: «Ya que [...] es el mal». Y el profeta

\par 42 [...]

\par 43 dijo: Señor, ¿en qué pecó la tierra? Y Dios dijo: «Ya que el oponente, habiendo escuchado mi terrible amenaza, se esconderá (en ella), y por eso derretiré la tierra y con ella a los rebeldes de la raza de los hombres».

\chapter{5}

\par \textit{Castigos adicionales}

\par 1 Y el profeta dijo: «Apiádate, Señor, del linaje de los cristianos». Y vi un

\par 2 [...]

\par 3 mujeres suspendidas y cuatro fieras chupaban sus pechos. Y los ángeles me dijeron: «A ella le disgustó dar su leche, pero también arrojó a los niños en el

\par 4 ríos.» Y vi una oscuridad terrible y una noche sin estrellas ni luna. No hay ni joven ni viejo, ni hermano con hermano ni madre con Niño ni

\par 5 [...]

\par 6 esposa con marido. Y lloré y dije: «Oh Señor, Señor, ten misericordia de los pecadores».

\par \textit{Esdras llevado al cielo}

\par 7 Y mientras decía estas cosas vino una nube, se apoderó de mí y me levantó de nuevo.

\par 8 a los cielos. Y vi muchos juicios y lloré amargamente y dije: «Es

\par 9 [...]

\par 10 Mejor sería si el hombre no saliera del vientre de su madre. Los que estaban en el castigo gritaban, diciendo: Desde que llegaste aquí, santo de Dios, hemos

\par 11 han obtenido un ligero respiro». Y el profeta dijo: «Bienaventurados los que lamentan sus propios pecados».

\par \textit{El nacimiento y su finalidad}

\par 12 Y Dios dijo: «¡Escucha a Esdras, amado! Así como un granjero arroja la semilla

\par 13 de grano en la tierra, así el hombre echa su semilla en lugar de la mujer. En el primero (mes) está entero, en el segundo está hinchado, en el tercero le crece pelo, en el cuarto le crecen uñas, en el quinto se vuelve lechoso, en el sexto está listo y vivificado, en el en el séptimo se prepara, [en el octavo...], en el noveno se abren los cerrojos de las puertas de la mujer y nace sana en la tierra».

\par 14 Y el profeta dijo: «Más le valdría al hombre no haber nacido. ¡Ay, oh!

\par 15 [...]

\par 16 raza humana, en ese momento cuando vengas al juicio! Y le dije al Señor,

\par 17 «Señor, ¿por qué creaste al hombre y lo entregaste al juicio?» Y Dios dijo en su exaltada declaración: «No perdonaré a los que transgreden mi

\par 18 pacto». Y el profeta dijo: «Señor, ¿dónde está tu bondad?» Y dijo Dios: Todo lo preparé por causa del hombre y el hombre no guarda mis mandamientos.

\par 19 [...]

\par \textit{Castigos y recompensas}

\par 20 Y el profeta dijo: «Señor, revélame los castigos y el Paraíso».

\par 21 Y los ángeles me llevaron hacia el oriente y vi el árbol de la vida. Y vi allí

\par 22 Enoc y Elías y Moisés y Pedro y Pablo y Lucas y Mateo y todos

\par 23 los justos y los patriarcas.g Y vi allí el castigo del aire y el soplo de los vientos y los depósitos del hielo y el eterno

\par 24 castigos. Y vi allí a un hombre colgado de su cráneo. Y me dijeron,

\par 25 [...]

\par 26 «Este transfirió límites». Y allí vi grandes juicios y dije a

\par 27 al Señor: «Oh Señor, Señor, ¿quién de los hombres, habiendo nacido, no pecó?» Y me llevaron más abajo en el Tártaro y vi a todos los pecadores lamentándose y

\par 28 llanto y luto malvado. Y yo también lloré al ver la raza de los hombres así castigada.

\chapter{6}

\par 1 Entonces Dios me dijo: Esdras, ¿sabes los nombres de los ángeles que están

\par 2 sobre la consumación: Miguel, Gabriel, Uriel, Rafael, Gabutelón, Aker, Arphugitonos, Beburos, Zebuleon?

\par \textit{Ezra lucha por su alma}

\par 3 Entonces vino a mí una voz: «¡Ven aquí, muere, Esdras, amado mío! devolver eso

\par 4 que te ha sido confiado (a ti)». Y el profeta dijo: «¿Y de dónde puede

\par 5 ¿has dado a luz mi alma? Y los ángeles dijeron: «Podemos expulsarlo a través de

\par 6 tu boca». Y el profeta dijo: «Hablé boca a boca con Dios y

\par 7 no saldrá de allí». Y los ángeles dijeron. «Lo sacaremos a luz a través

\par 8 tus fosas nasales». Y el profeta dijo: «Mis narices olían la gloria de Dios».

\par 9 Y los ángeles dijeron: «Podemos sacarlo a través de tus ojos». Y el profeta

\par 10 [...]

\par 11 dijo: «Mis ojos han visto la espalda de Dios». Y los ángeles dijeron: «Podemos traer

\par 12 sácalo a través de tu cabeza». Y los ángeles dijeron: «Podemos sacarlo a través de tus pies». Y el profeta dijo: Caminé con Moisés sobre la montaña,

\par 13 y no saldrá de allí». Y los ángeles dijeron: «Podemos lanzarlo

\par 14 a través de las puntas de las uñas (de los pies)». Y el profeta dijo: «Mis pies entraron en el santuario». Y los ángeles se marcharon sin éxito, diciendo: Señor, nosotros

\par 15 [...]

\par 16 no puede recibir su alma. Luego le dijo a su hijo unigénito: «Desciende, hijo amado mío, con un gran ejército de ángeles, y toma el alma de mi amado Esdras».

\par 17 Porque el Señor, tomando un gran ejército de muchos ángeles, dijo al profeta: «Dame el depósito que te he confiado. La corona está lista para ti».

\par 18 Y el profeta dijo: Señor, si me quitas el alma, ¿a quién le dejarás

\par 19 para abogar por la raza de los hombres? Y Dios dijo: «Tú que eres mortal

\par 20 y terrenales, no me defiendan este caso». Y el profeta dijo: «Nunca

\par 21 deja de suplicar.» Y Dios dijo: «Dad, mientras tanto, lo que os ha sido confiado».

\par 22 (para ti). La corona está lista para ti. Ven aquí, muere, para que puedas alcanzar

\par 23 eso.» Entonces el profeta comenzó a hablar entre lágrimas: «Oh Señor, ¿de qué me sirve que yo

\par 24 ¿Declararé el caso contigo y yo caeré a la tierra? ¡Ay, ay! porque lo haré

\par 25 ser consumido por los gusanos. Lloren por mí, todos santos y piadosos, les suplico mucho y

\par ¡26 am entregado a la muerte! Llorad por mí, todos santos y justos, porque he entrado en la copa del Hades».

\chapter{7}

\par \textit{Alma y cuerpo}

\par 1 Y Dios le dijo: «Oye, Esdras, mi amado. Yo, siendo inmortal, recibí

\par 2 una cruz, probé vinagre y hiel, fui sepultado en una tumba. Y levanté a mis elegidos y llamé a Adán del Hades para que la raza de los hombres

\par 3 [...]. Por tanto, no temáis a la muerte. Porque lo que es de mí, que es el alma, parte para el cielo. Lo que es de la tierra, es decir el cuerpo, parte hacia el

\par 4 tierra de la cual fue tomada». Y el profeta dijo: «¡Ay, ay! ¿Qué debo hacer? ¿Cómo debo actuar? Yo no sé.»

\par \textit{Oración finalb}

\par 5 Y entonces el bienaventurado Esdras comenzó a decir: «¡Oh Dios eterno, Creador de toda la creación, que con un palmo midiste los cielos y en tu mano contenías la tierra!

\par 6 mano que impulsa a los querubines, que llevó al profeta Elías al cielo en

\par 7 un carro de fuego, que sustenta a toda carne, a quien todas las cosas temen y tiemblan

\par 8 frente a tu poder, escúchame, que suplica mucho y da a todos los que

\par 9 copia este libro y consérvalo y recuerda mi nombre y preserva mi memoria plenamente,

\par 10 dales bendición del cielo. Y bendecir todas sus cosas, así como los fines de

\par 11 José.d Y no te acuerdes de sus pecados anteriores en el día de su juicio. Aquellos

\par 12 [...]

\par 13 Los que no crean en este libro serán quemados como Sodoma y Gomorra. Y vino a él una voz que decía: Esdras. Amado mío, concederé a cada uno lo que me pediste».

\par \textit{Muerte y sepultura de Esdras}

\par 14 E inmediatamente entregó su preciosa alma con mucho honor el día dieciocho.

\par 15 del mes de octubre. Y lo sepultaron con incienso y salmos. Su precioso y santo cuerpo proporciona incesantemente fortalecimiento de almas y cuerpos a quienes se acercan a él voluntariamente.

\par 16 Gloria, poder, honra y adoración a aquel a quien conviene, al Padre, al Hijo y al Espíritu Santo, ahora y siempre, y por los siglos de los siglos. Amén.

\end{document}