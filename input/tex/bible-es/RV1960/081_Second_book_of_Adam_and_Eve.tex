\begin{document}

\title{Segundo libro de Adán y Eva}

\chapter{1}

\par \textit{La familia afligida. Caín se casa con Luluwa y se mudan.}

\par 1 CUANDO Luluwa escuchó las palabras de Caín, lloró y fue donde su padre y su madre, y les contó cómo Caín había matado a su hermano Abel.

\par 2 Entonces todos gritaron y alzaron la voz, se abofetearon, se echaron polvo en la cabeza y rasgaron sus vestidos, y salieron y llegaron al lugar donde habían matado a Abel.

\par 3 Y lo encontraron tendido en el suelo, muerto, y las bestias a su alrededor; mientras ellos lloraban y lloraban por este justo. De su cuerpo, a causa de su pureza, salía un olor a especias dulces.

\par 4 Y Adán lo llevó mientras las lágrimas corrían por su rostro; y fue a la Cueva de los Tesoros, donde lo puso y lo arojó con especias dulces y mirra.

\par 5 Y Adán y Eva duraron ciento cuarenta días en su sepultura con gran dolor. Abel tenía quince años y medio, y Caín diecisiete años y medio.

\par 6 En cuanto a Caín, cuando terminó el duelo por su hermano, tomó a su hermana Luluwa y se casó con ella, sin permiso de su padre y de su madre; porque no podían apartarlo de ella a causa de su corazón apesadumbrado.

\par 7 Luego descendió al pie de la montaña, lejos del jardín, cerca del lugar donde había matado a su hermano.

\par 8 Y en aquel lugar había muchos árboles frutales y árboles forestales. Su hermana le dio a luz hijos, que a su vez comenzaron a multiplicarse poco a poco hasta ocupar ese lugar.

\par 9 Pero Adán y Eva no se reunieron después del funeral de Abel durante siete años. Después de esto, sin embargo, Eva concibió; y estando ella encinta, Adán le dijo: «Ven, tomemos una ofrenda y ofrezcámosla a Dios, y pidámosle que nos dé un niño hermoso, en quien podamos encontrar consuelo y con quien podamos unirnos en matrimonio». a la hermana de Abel».

\par 10 Entonces prepararon una ofrenda, la llevaron al altar, la ofrecieron delante del Señor y comenzaron a rogarle que aceptara su ofrenda y les diera una buena descendencia.

\par 11 Y Dios escuchó a Adán y aceptó su ofrenda. Luego adoraron a Adán, Eva y su hija, descendieron a la Cueva de los Tesoros y colocaron en ella una lámpara para arder de noche y de día delante del cuerpo de Abel.

\par 12 Entonces Adán y Eva continuaron ayunando y orando hasta que llegó el momento de que Eva fuera liberada, cuando le dijo a Adán: «Quiero ir a la cueva en la roca, para dar a luz en ella».

\par 13 Y él dijo: «Ve y lleva contigo a tu hija para que te atienda; pero yo permaneceré en esta Cueva de los Tesoros delante del cuerpo de mi hijo Abel».

\par 14 Entonces Eva escuchó a Adán y se fue, ella y su hija. Pero Adán permaneció solo en la Cueva de los Tesoros.

\chapter{2}

\par \textit{A Adán y Eva les nace un tercer hijo.}

\par 1 Y Eva dio a luz un hijo perfectamente hermoso en figura y semblante. Su belleza era como la de su padre Adán, pero más hermosa.

\par 2 Eva se consoló al verlo y permaneció ocho días en la cueva; Luego envió a su hija a Adán para decirle que viniera a ver al niño y le pusiera nombre. Pero la hija permaneció en su lugar junto al cuerpo de su hermano, hasta que Adán regresó. Ella también.

\par 3 Pero cuando Adán llegó y vio la buena apariencia del niño, su belleza y su figura perfecta, se alegró por él y se consoló por Abel. Luego llamó al niño Set, que significa «que Dios ha oído mi oración y me ha librado de mi aflicción». Pero también significa «poder y fuerza».

\par 4 Luego, después de darle nombre al niño, Adán regresó a la Cueva de los Tesoros; y su hija volvió con su madre.

\par 5 Pero Eva permaneció en su cueva hasta que se cumplieron cuarenta días, cuando llegó a Adán y trajo consigo al niño y a su hija.

\par 6 Y llegaron a un río de agua, donde Adán y su hija se lavaron, a causa de su dolor por Abel; pero Eva y el niño se lavaron para purificarse.

\par 7 Entonces regresaron, tomaron una ofrenda, fueron al monte y la ofrecieron por el niño; y Dios aceptó su ofrenda y envió su bendición sobre ellos y sobre su hijo Set; y regresaron a la Cueva de los Tesoros.

\par 8 En cuanto a Adán, no volvió a conocer a su esposa Eva en todos los días de su vida; Tampoco nació más descendencia de ellos; pero sólo esos cinco, Caín, Luluwa, Abel, Aklia y Seth solos.

\par 9 Pero Set creció en estatura y en fuerza; y comenzó a ayunar y orar fervientemente.

\chapter{3}

\par \textit{Satanás aparece como una hermosa mujer tentando a Adán, diciéndole que todavía es un joven. «Pasa tu juventud en alegría y placer». (12) Las diferentes formas que toma Satanás (15).}

\par 1 En cuanto a nuestro padre Adán, al cabo de siete años desde el día en que se separó de su esposa Eva, Satanás le tuvo envidia al verlo así separado de ella; y se esforzó para que volviera a vivir con ella.

\par 2 Entonces Adán se levantó y subió a la cueva de los tesoros; y seguía durmiendo allí noche tras noche. Pero tan pronto como amanecía, todos los días bajaba a la cueva para orar allí y recibir de ella una bendición.

\par 3 Pero al anochecer subió al tejado de la cueva y durmió solo, temiendo que Satanás lo venciera. Y estuvo así apartado treinta y nueve días.

\par 4 Entonces Satanás, aborrecedor de todo bien, cuando vio a Adán solo, ayunando y orando, se le apareció en forma de una mujer hermosa, la cual se presentó ante él en la noche del cuadragésimo día y le dijo: a él:—

\par 5 «Oh Adán, desde el momento en que habitasteis en esta cueva, hemos experimentado de ti una gran paz, y tus oraciones nos han llegado y hemos sido consolados por ti.

\par 6 «Pero ahora, oh Adán, que has subido al techo de la cueva para dormir, hemos tenido dudas acerca de ti, y un gran dolor nos ha sobrevenido por tu separación de Eva. Luego, cuando estás en el techo de esta cueva, tu oración es derramada y tu corazón vaga de un lado a otro.

\par 7 «Pero cuando estabas en la cueva, tu oración era como fuego reunido; descendió hasta nosotros y encontraste descanso.

\par 8 «También me entristecí por tus hijos que están separados de ti; y mi dolor es grande por el asesinato de tu hijo Abel, porque él era justo; y por un hombre justo todos se entristecen.

\par 9 «Pero me alegré por el nacimiento de tu hijo Set, pero al poco tiempo me entristecí mucho por Eva, porque es mi hermana. Porque cuando Dios envió sobre ti un sueño profundo y la sacó de tu costado, A mí también me sacó con ella, pero a ella la levantó poniéndola contigo, mientras que a mí me bajó.

\par 10 «Me regocijé por mi hermana por estar contigo. Pero Dios me había hecho una promesa antes, y me dijo: 'No te entristezcas cuando Adán haya subido al techo de la Cueva de los Tesoros y se haya separado de Eva. su esposa, te enviaré a él, te unirás a él en matrimonio y le darás cinco hijos, como Eva le dio cinco.'

\par 11 «Y ahora, ¡he aquí! La promesa que Dios me hizo se ha cumplido; porque es Él quien me ha enviado a ti para las bodas; porque si te casas conmigo, te daré hijos mejores y mejores que los de Eva.

\par 12 «Además, aún eres un joven; no termines tu juventud en este mundo con tristeza, sino pasa los días de tu juventud en alegría y placer. Porque tus días son pocos y tu prueba, grande. Sé fuerte. Termina tus días en este mundo con regocijo. Yo me complaceré en ti y tú te alegrarás conmigo de esta manera y sin miedo.

\par 13 «Levántate, pues, y cumple el mandato de tu Dios», entonces se acercó a Adán y lo abrazó.

\par 14 Pero cuando Adán vio que ella lo vencería, oró a Dios con todo su corazón para que lo librara de ella.

\par 15 Entonces Dios envió Su Palabra a Adán, diciendo: «Oh Adán, esa figura es la que te prometió la Deidad y la majestad; no está favorablemente dispuesto hacia ti, sino que se muestra a ti en un momento en la forma de una mujer; otras veces, a semejanza de un ángel; otras veces, a semejanza de una serpiente; y otras veces, a semejanza de un dios; pero todo eso lo hace sólo para destruir tu alma.

\par 16 Ahora pues, oh Adán, conociendo tu corazón, te he librado muchas veces de sus manos; para mostrarte que soy un Dios misericordioso; y que deseo tu bien, y que no deseo tu ruina».

\chapter{4}

\par \textit{Adán ve al Diablo en sus verdaderos colores.}

\par 1 ENTONCES Dios ordenó a Satanás que se mostrara a Adán claramente, en su propia y espantosa forma.

\par 2 Pero cuando Adán lo vio, tuvo miedo y tembló al verlo.

\par 3 Y Dios dijo a Adán: «Mira a este diablo y su espantosa mirada, y sabe que él es quien te hizo caer de la claridad a las tinieblas, de la paz y el descanso al trabajo y la miseria.

\par 4 ¡Y mira, oh Adán, a aquel que decía de sí mismo que es Dios! ¿Puede Dios ser negro? ¿Tomaría Dios la forma de una mujer? ¿Hay alguien más fuerte que Dios? ¿Y puede ser vencido?

\par 5 «¡Mira, entonces, oh Adán, y míralo atado en tu presencia, en el aire, sin poder huir! Por tanto te digo que no tengas miedo de él; De ahora en adelante cuídate y guárdate de él en todo lo que te haga».

\par 6 Entonces Dios expulsó de delante de Adán a Satanás, a quien fortaleció y consoló su corazón, diciéndole: «Desciende a la cueva de los tesoros y no te separes de Eva; Sofocaré en ti toda lujuria animal».

\par 7 Desde aquella hora abandonó a Adán y a Eva, y disfrutaron del descanso por mandato de Dios. Pero Dios no hizo semejante a ninguno de los de la simiente de Adán; pero sólo a Adán y Eva.

\par 8 Entonces Adán adoró ante el Señor, por haberlo librado y por haber dejado sus pasiones. Y descendió de encima de la cueva y habitó con Eva como antes.

\par 9 Así terminaron los cuarenta días de su separación de Eva.

\chapter{5}

\par \textit{El diablo pinta un cuadro brillante para que Seth deleite sus pensamientos.}

\par 1 En cuanto a Set, cuando tenía siete años, conocía el bien y el mal, ayunaba y oraba constantemente y pasaba todas las noches implorando a Dios misericordia y perdón.

\par 2 También ayunaba cada día al subir su ofrenda, más que su padre; porque era de bello rostro, semejante a un ángel de Dios. También tenía buen corazón, conservaba las mejores cualidades de su alma: y por eso elevaba su ofrenda todos los días.

\par 3 Y Dios se alegró de su ofrenda; pero también estaba complacido con su pureza. Y continuó así haciendo la voluntad de Dios, y de su padre y de su madre, hasta los siete años.

\par 4 Después de eso, mientras descendía del altar, después de terminar su ofrenda, se le apareció Satanás en forma de un ángel hermoso, resplandeciente de luz; con un bastón de luz en su mano, él mismo ceñido con un cinto de luz.

\par 5 Saludó a Seth con una hermosa sonrisa y comenzó a seducirlo con hermosas palabras, diciéndole: «Oh Seth, ¿por qué moras en esta montaña? Porque es áspero, lleno de piedras y de arena, y de árboles sin buenos frutos; un desierto sin habitaciones y sin ciudades; No hay un buen lugar para vivir. Pero todo es calor, cansancio y problemas».

\par 6 Dijo además: «Pero vivimos en lugares hermosos, en otro mundo que este mundo. Nuestro mundo es uno de luz y nuestra condición es la mejor; nuestras mujeres son más hermosas que cualquier otra; y te deseo, oh Seth, para casarte con una de ellas, porque veo que eres hermoso de ver, y en esta tierra no hay una mujer lo suficientemente buena para ti. Además, todos los que viven en este mundo, son sólo cinco almas.

\par 7 «Pero en nuestro mundo hay muchísimos hombres y muchas doncellas, todos más bellos unos que otros. Deseo, por tanto, sacarte de aquí, para que puedas ver a mis parientes y casarte con quien quieras.

\par 8 «Entonces permanecerás conmigo y estarás en paz; estarás lleno de esplendor y luz, como nosotros.

\par 9 «Permanecerás en nuestro mundo y descansarás de este mundo y de su miseria; nunca más te sentirás débil y cansado; nunca traerás una ofrenda ni pedirás misericordia; porque no cometerás más pecar, ni dejarse llevar por las pasiones.

\par 10 «Y si escuchas lo que te digo, te casarás con una de mis hijas; porque entre nosotros no es pecado hacerlo, ni se considera lujuria animal.

\par 11 «Porque en nuestro mundo no tenemos Dios; pero todos somos dioses; todos somos de luz, celestiales, poderosos, fuertes y gloriosos».

\chapter{6}

\par \textit{La conciencia de Seth lo ayuda. Regresa a Adán y Eva.}

\par 1 CUANDO Set escuchó estas palabras, quedó asombrado e inclinó su corazón a las traicioneras palabras de Satanás, y le dijo: «Dijiste que hay otro mundo creado además de este; ¿Y otras criaturas más hermosas que las criaturas que hay en este mundo?

\par 2 Y Satanás dijo: «Sí; he aquí, me has oído; pero aún los alabaré a ellos y a sus caminos delante de ti».

\par 3 Pero Set le dijo: «Tus palabras me han asombrado y tu hermosa descripción de todo.

\par 4 «Pero hoy no puedo ir contigo; no hasta que haya ido a mi padre Adán y a mi madre Eva, y les haya contado todo lo que tú me has dicho. Entonces, si me dan permiso para ir contigo, iré.

\par 5 Nuevamente Set dijo: «Tengo miedo de hacer cualquier cosa sin el permiso de mi padre y de mi madre, no sea que perezca como mi hermano Caín y como mi padre Adán, que transgredió el mandamiento de Dios. Pero he aquí, tú conoces este lugar; Ven y reúnete conmigo aquí mañana.

\par 6 Cuando Satanás oyó esto, dijo a Set: «Si le cuentas a tu padre Adán lo que te he dicho, no te dejará venir conmigo.

\par 7 Pero escúchenme; no le digas a tu padre ni a tu madre lo que te he dicho; pero ven conmigo hoy, a nuestro mundo; donde verás cosas hermosas y te divertirás allí, y te deleitarás este día entre mis hijos, contemplándolos y llenándote de alegría; y regocijaos cada vez más. Entonces te traeré de regreso a este lugar mañana; pero si prefieres quedarte conmigo, que así sea».

\par 8 Entonces Set respondió: «El espíritu de mi padre y de mi madre pende de mí; y si un día me escondo de ellos, morirán, y Dios me considerará culpable de pecar contra ellos.

\par 9 »Y si no supieran que he venido a este lugar para llevar mi ofrenda, no se separarían de mí ni una hora; Tampoco debo ir a ningún otro lugar, a menos que me dejen. Pero me tratan muy amablemente porque vuelvo rápidamente con ellos».

\par 10 Entonces Satanás le dijo: «¿Qué te sucederá si te escondes de ellos una noche y regresas a ellos al amanecer?»

\par 11 Pero Set, al ver que seguía hablando y que no lo dejaba, corrió, subió al altar, extendió las manos hacia Dios y pidió su liberación.

\par 12 Entonces Dios envió su palabra y maldijo a Satanás, quien huyó de él.

\par 13 Pero Set, que había subido al altar, decía esto en su corazón. «El altar es el lugar de la ofrenda, y Dios está allí; un fuego divino lo consumirá; así Satanás no podrá hacerme daño ni me llevará de allí».

\par 14 Entonces Set bajó del altar y fue donde su padre y su madre, donde se encontró en el camino, deseando oír su voz; porque se había demorado un tiempo.

\par 15 Entonces comenzó a contarles en forma de ángel lo que le había sucedido de parte de Satanás.

\par 16 Pero cuando Adán escuchó su relato, le besó la cara y le advirtió contra aquel ángel, diciéndole que era Satanás quien se le había aparecido así. Entonces Adán tomó a Set y fueron a la Cueva de los Tesoros, y allí se regocijaron.

\par 17 Pero desde aquel día Adán y Eva no se separaron de él, fuera a donde fuera, ya fuera para ofrecer su ofrenda o para cualquier otra cosa.

\par 18 Esta señal le ocurrió a Set cuando tenía nueve años.

\chapter{7}

\par \textit{Seth se casa con Aklia. Adam vive para ver a sus nietos y bisnietos.}

\par 1 Cuando nuestro padre Adán vio que Set era de corazón perfecto, quiso que se casara; no sea que el enemigo se le aparezca otra vez y lo venza.

\par 2 Entonces Adán dijo a su hijo Set: «Deseo, hijo mío, que te cases con tu hermana Aclia, hermana de Abel, para que ella pueda darte hijos que llenen la tierra, según la promesa que Dios nos hizo.

\par 3 «No temas, hijo mío; no hay ninguna vergüenza en ello. Deseo que te cases por miedo a que el enemigo te venza.

\par 4 Set, sin embargo, no quería casarse; pero, obedeciendo a su padre y a su madre, no dijo una palabra.

\par 5 Entonces Adán lo casó con Aklia. Y tenía quince años.

\par 6 Pero cuando tenía veinte años, engendró un hijo, al que llamó Enós; y luego engendró otros hijos además de él.

\par 7 Entonces Enós creció, se casó y engendró a Cainán.

\par 8 Cainán también creció, se casó y engendró a Mahalaleel.

\par 9 Aquellos padres nacieron en vida de Adán y habitaron en la cueva de los tesoros.

\par 10 Y fueron los días de Adán novecientos treinta años, y los de Mahalaleel cien. Pero Mahalaleel, cuando creció, amaba el ayuno, la oración y los trabajos duros, hasta que se acercó el fin de los días de nuestro padre Adán.

\chapter{8}

\par \textit{Las extraordinarias últimas palabras de Adán. Predice el Diluvio. Exhorta a su descendencia al bien. Él revela ciertos misterios de la vida.}

\par 1 Cuando nuestro padre Adán vio que su fin estaba cerca, llamó a su hijo Set, quien vino a él en la cueva de los tesoros y le dijo:

\par 2 «Oh Seth, hijo mío, tráeme a tus hijos y a los hijos de tus hijos, para que pueda derramar mi bendición sobre ellos antes de que muera».

\par 3 Cuando Set escuchó estas palabras de su padre Adán, se alejó de él, derramó un torrente de lágrimas sobre su rostro, reunió a sus hijos y a los hijos de sus hijos y se los llevó a su padre Adán.

\par 4 Pero nuestro padre Adán, cuando los vio a su alrededor, lloró por tener que ser separado de ellos.

\par 5 Y cuando lo vieron llorar, todos lloraron a una y se postraron sobre su rostro, diciendo: «¿Cómo serás separado de nosotros, oh padre nuestro? ¿Y cómo te recibirá la tierra y te ocultará de nuestros ojos? Así se lamentaron mucho y con palabras similares.

\par 6 Entonces nuestro padre Adán los bendijo a todos y, después de haberlos bendecido, dijo a Set:

\par 7 «Oh Seth, hijo mío, tú conoces este mundo, que está lleno de tristeza y de cansancio; y sabes todo lo que nos ha sobrevenido a causa de nuestras pruebas en él. Por tanto, ahora te ordeno en estas palabras: guardar la inocencia, ser puro y justo, y confiar en Dios; y no apoyarte en los discursos de Satanás, ni en las apariciones en las que se mostrará a ti.

\par 8 Pero guarda los mandamientos que te doy hoy; luego dale lo mismo a tu hijo Enós; y que Enós se lo dé a su hijo Cainán; y Cainán a su hijo Mahalaleel; para que este mandamiento permanezca firme en todos tus hijos.

\par 9 «Oh Seth, hijo mío, en el momento en que muera, toma mi cuerpo y envuélvelo con mirra, áloe y casia, y déjame aquí en esta Cueva de los Tesoros en la que están todas estas señales que Dios nos dio. del jardín.

\par 10 «Oh hijo mío, de ahora en adelante vendrá un diluvio que arrasará con todas las criaturas y dejará fuera sólo a ocho almas.

\par 11 «Pero, hijo mío, aquellos a quienes dejará de entre tus hijos en ese momento, saquen mi cuerpo de esta cueva con ellos; y cuando lo hayan tomado, que el mayor de ellos mande a sus hijos que pongan mi cuerpo en un barco hasta que se calme la inundación, y salgan del barco.

\par 12 Entonces tomarán mi cuerpo y lo pondrán en medio de la tierra, poco después de haber sido salvados de las aguas del diluvio.

\par 13 Porque el lugar donde será puesto mi cuerpo es el centro de la tierra; de allí vendrá Dios y salvará a todos nuestros parientes.

\par 14 «Pero ahora, oh Set, hijo mío, ponte a la cabeza de tu pueblo; cuídalos y cuídalos en el temor de Dios; y condúcelos por el buen camino, ordena que ayunen para Dios; y hazles entender que no deben escuchar a Satanás, para que no los destruya.

\par 15 »Entonces, otra vez separa a tus hijos y a los hijos de tus hijos de los hijos de Caín; nunca dejes que se mezclen con ellos, ni se acerquen a ellos ni en sus palabras ni en sus hechos».

\par 16 Entonces Adán dejó que su bendición descendiera sobre Set, sus hijos y todos los hijos de sus hijos.

\par 17 Entonces se volvió hacia su hijo Set y hacia su esposa Eva, y les dijo: «Guardad este oro, este incienso y esta mirra que Dios nos ha dado como señal; porque en los días venideros, Un diluvio cubrirá toda la creación. Pero los que entren en el arca llevarán consigo el oro, el incienso y la mirra junto con mi cuerpo, y pondrán el oro, el incienso y la mirra junto con mi cuerpo. cuerpo en medio de la tierra.

\par 18 «Después de mucho tiempo, la ciudad en la que se encuentran el oro, el incienso y la mirra con mi cuerpo será saqueada. Pero cuando sea saqueada, el oro, el incienso y la mirra serán saqueados. cuidado con el botín que se guarda; y nada de ellos perecerá, hasta que venga la Palabra de Dios hecha hombre; cuando los reyes los tomarán, y le ofrecerán oro en señal de que es Rey; incienso, en señal de ser Dios del cielo y de la tierra, y mirra, en señal de su pasión.

\par 19 También el frío, como señal de que ha vencido a Satanás y a todos nuestros enemigos; el incienso, como señal de que resucitará de entre los muertos y será exaltado sobre las cosas del cielo y de la tierra; y la mirra, en señal de que ha vencido a Satanás y a todos nuestros enemigos. que beberá hiel amarga y sentirá las penas del infierno de parte de Satanás.

\par 20 »Y ahora, oh Set, hijo mío, he aquí que te he revelado los misterios ocultos que Dios me había revelado a mí. Guarda mi mandamiento, para ti y para tu pueblo».

\chapter{9}

\par \textit{La muerte de Adán.}

\par 1 Cuando Adán terminó su mandamiento a Set, sus miembros se aflojaron, sus manos y pies perdieron todo poder, su boca se quedó muda y su lengua dejó de hablar. Cerró los ojos y abandonó el fantasma.

\par 2 Pero cuando sus hijos vieron que estaba muerto, se arrojaron sobre él, hombres y mujeres, viejos y jóvenes, llorando.

\par 3 La muerte de Adán tuvo lugar al cabo de novecientos treinta años de vida en la tierra; el día quince de Barmudeh, después del cómputo de un impacto del sol, a la hora novena.

\par 4 Era viernes, el mismo día en que fue creado y en el que descansó; y la hora en que murió, fue la misma en que salió del jardín.

\par 5 Entonces Set lo vendó bien y lo embalsamó con abundantes especias aromáticas de árboles sagrados y de la Montaña Sagrada; y será puesto su cuerpo en el lado oriental del interior de la cueva, el lado del incienso; y puso delante de él un candelero que seguía encendido.

\par 6 Entonces sus hijos estuvieron ante él llorando y lamentándose por él toda la noche hasta el amanecer.

\par 7 Entonces Set, Enós, su hijo mayor, y Cainán, hijo de Enós, salieron y tomaron buenas ofrendas para presentarlas al Señor, y llegaron al altar sobre el cual Adán ofrecía ofrendas a Dios, cuando las ofrecía.

\par 8 Pero Eva les dijo: «Esperen hasta que primero le hayamos pedido a Dios que acepte nuestra ofrenda y que guarde junto a Él el alma de Adán, su siervo, y la lleve a descansar».

\par 9 Y todos se levantaron y oraron.

\chapter{10}

\par \textit{«Adán fue el primero...»}

\par 1 Y cuando terminaron su oración, vino la Palabra de Dios y los consoló acerca de su padre Adán.

\par 2 Después de esto, ofrecieron sus presentes para ellos y para su padre.

\par 3 Y cuando terminaron su ofrenda, la Palabra de Dios vino a Seth, el mayor entre ellos, diciéndole: «Oh Seth, Set, Set, tres veces. Como estuve con tu padre, así también seré. Estará contigo, hasta el cumplimiento de la promesa que le hice, diciendo tu padre: Enviaré mi palabra y te salvaré a ti y a tu descendencia.

\par 4 Pero en cuanto a tu padre Adán, guarda tú el mandamiento que él te dio; y separa tu descendencia de la de Caín tu hermano.»

\par 5 Y Dios retiró su palabra de Set.

\par 6 Entonces Set, Eva y sus hijos descendieron de la montaña a la Cueva de los Tesoros.

\par 7 Pero Adán fue el primero cuya alma murió en la tierra del Edén, en la Cueva de los Tesoros; porque nadie murió antes que él, sino su hijo Abel, el cual murió asesinado.

\par 8 Entonces todos los hijos de Adán se levantaron y lloraron sobre su padre Adán y le hicieron ofrendas durante ciento cuarenta días.

\chapter{11}

\par \textit{Seth se convierte en líder de la tribu de personas más feliz y justa que jamás haya existido.}

\par 1 DESPUÉS de la muerte de Adán y Eva, Set separó a sus hijos y a los hijos de sus hijos de los hijos de Caín. Caín y su descendencia descendieron y habitaron al oeste, debajo del lugar donde había matado a su hermano Abel.

\par 2 Pero Set y sus hijos habitaron al norte, en la montaña de la Cueva de los Tesoros, para estar cerca de su padre Adán.

\par 3 Y Set el mayor, alto y bueno, de alma excelente y de mente fuerte, estaba a la cabeza de su pueblo; y los cuidó con inocencia, arrepentimiento y mansedumbre, y no permitió que ninguno de ellos descendiera con los hijos de Caín.

\par 4 Pero a causa de su propia pureza, fueron llamados «Hijos de Dios», y estaban con Dios, en lugar de las huestes de ángeles que cayeron; porque continuaron alabando a Dios y cantándole salmos en su cueva, la Cueva de los Tesoros.

\par 5 Entonces Set se paró ante el cuerpo de su padre Adán y de su madre Eva, y oró noche y día, y pidió misericordia para él y sus hijos; y que cuando tuviera alguna dificultad al tratar con un niño, le daría consejo.

\par 6 Pero a Set y a sus hijos no les gustaban los trabajos terrenales, sino que se entregaban a las cosas celestiales; porque no tenían otro pensamiento que alabanzas, doxologías y salmos a Dios.

\par 7 Por eso oían en todo momento las voces de los ángeles que alababan y glorificaban a Dios; desde dentro del huerto, o cuando eran enviados por Dios a hacer un recado, o cuando subían al cielo.

\par 8 Porque Set y sus hijos, debido a su pureza, oyeron y vieron a aquellos ángeles. Por otra parte, el jardín no estaba muy por encima de ellos, sino sólo unos quince codos espirituales.

\par 9 Ahora bien, un codo espiritual equivale a tres codos del hombre, en total cuarenta y cinco codos.

\par 10 Set y sus hijos habitaban en el monte debajo del jardín; no sembraron ni cosecharon; No produjeron alimento para el cuerpo. ni siquiera trigo; pero sólo ofrendas. Comieron de las frutas y de los árboles sabrosos que crecían en el monte donde habitaban.

\par 11 Entonces Set ayunaba cada cuarenta días, al igual que sus hijos mayores. Porque la familia de Seth olía el olor de los árboles del jardín, cuando el viento soplaba en esa dirección.

\par 12 Eran felices, inocentes, sin miedo repentino, no había celos, ni malas acciones, ni odio entre ellos. No había pasión animal; De ninguna boca de ellos salían malas palabras ni maldiciones; ni malos consejos ni fraude. Porque los hombres de aquella época nunca juraban, pero en circunstancias difíciles, cuando los hombres debían jurar, juraban por la sangre de Abel el justo.

\par 13 Pero obligaban a sus hijos y a sus mujeres todos los días en la cueva a ayunar, orar y adorar al Dios Altísimo. Se bendijeron en el cuerpo de su padre Adán y se ungieron con él.

\par 14 Y así hicieron hasta que se acercó el fin de Set.

\chapter{12}

\par \textit{Asuntos familiares de Seth. Su muerte. La jefatura de Enós. Cómo le fue a la rama marginada de la familia de Adán.}

\par 1 ENTONCES Set el justo llamó a su hijo Enós, a Cainán hijo de Enós y a Mahalaleel hijo de Cainán, y les dijo:

\par 2 «Como mi fin se acerca, deseo construir un techo sobre el altar en el que se ofrecen las ofrendas».

\par 3 Ellos, obedeciendo su mandato, salieron todos, desde los mayores hasta los jóvenes, trabajaron duro y construyeron un hermoso techo sobre el altar.

\par 4 Y al hacerlo Set pensó que una bendición caería sobre sus hijos en la montaña; y que debía presentarles una ofrenda antes de su muerte.

\par 5 Cuando terminaron de construir el techo, les ordenó que hicieran ofrendas. Trabajaron diligentemente en ellos y se los llevaron a Set, su padre, quien los tomó y los ofreció sobre el altar; y oró a Dios para que aceptara sus ofrendas, tuviera misericordia de las almas de sus hijos y los guardara de la mano de Satanás.

\par 6 Y Dios aceptó su ofrenda y envió su bendición sobre él y sobre sus hijos. Y entonces Dios le hizo una promesa a Set, diciendo: «Al final de los grandes cinco días y medio, respecto de los cuales te he hecho una promesa a ti y a tu padre, enviaré Mi Palabra y te salvaré a ti y a tu descendencia. »

\par 7 Entonces Seth, sus hijos y los hijos de sus hijos se reunieron, descendieron del altar y fueron a la cueva de los tesoros, donde oraron, se bendijeron en el cuerpo de nuestro padre Adán y se ungieron. ellos mismos con ello.

\par 8 Pero Set permaneció en la cueva de los tesoros algunos días, y luego sufrió sufrimientos hasta la muerte.

\par 9 Entonces vino a él Enós, su hijo primogénito, con Cainán, su hijo, y Mahalaleel, hijo de Cainán, y Jared, hijo de Mahalaleel, y Enoc, hijo de Jared, con sus esposas e hijos, para recibir una bendición de Set.

\par 10 Entonces Set oró por ellos, los bendijo y los conjuró con la sangre de Abel el justo, diciendo: «Os ruego, hijos míos, que no dejéis que ninguno de vosotros descienda de este monte santo y puro.

\par 11 No hagáis compañerismo con los hijos de Caín, el asesino y pecador que mató a su hermano; porque vosotros sabéis, oh hijos míos, que huimos de él, y de todo su pecado con todas nuestras fuerzas, porque mató a su hermano Abel.»

\par 12 Después de haber dicho esto, Set bendijo a Enós, su hijo primogénito, y le ordenó ministrar habitualmente en pureza ante el cuerpo de nuestro padre Adán, todos los días de su vida; luego, también, ir de vez en cuando al altar que Set había construido. Y le ordenó que alimentara a su pueblo con justicia, juicio y pureza todos los días de su vida.

\par 13 Entonces los miembros de Set se debilitaron; sus manos y pies perdieron todo poder; su boca se quedó muda y no podía hablar; y entregó el espíritu y murió al día siguiente de su novecientos doce años; el día veintisiete del mes de Abib; Enoc tenía entonces veinte años.

\par 14 Luego enrollaron cuidadosamente el cuerpo de Set, lo embalsamaron con especias aromáticas y lo pusieron en la cueva de los tesoros, al lado derecho del cuerpo de nuestro padre Adán, y lo lloraron durante cuarenta días. Le ofrecieron regalos, como lo habían hecho con nuestro padre Adán.

\par 15 Después de la muerte de Set, Enós se levantó a la cabeza de su pueblo, a quien alimentó con justicia y juicio, tal como su padre le había ordenado.

\par 16 Pero cuando Enós tenía ochocientos veinte años, Caín tenía una descendencia numerosa; porque se casaban con frecuencia, dados a las concupiscencias animales; hasta que la tierra debajo del monte se llenó de ellos.

\chapter{13}

\par \textit{«Entre los hijos de Caín hubo mucho robo, asesinato y pecado.»}

\par 1 EN aquellos días vivía Lamec el ciego, que era de los hijos de Caín. Tenía un hijo que se llamaba Atún, y los dos tenían mucho ganado.

\par 2 Pero Lamec solía enviarlos a apacentar con un joven pastor que los cuidaba; y quien, al regresar a casa por la tarde, lloró ante su abuelo, y ante su padre Atun y su madre Hazina, y les dijo: «En cuanto a mí, no puedo alimentar a ese ganado solo, no sea que alguien me robe algunos de ellos, o mátame por causa de ellos». Porque entre los hijos de Caín hubo mucho robo, asesinato y pecado.

\par 3 Entonces Lamec se compadeció de él y dijo: «En verdad, si estuviera solo, los hombres de este lugar podrían derrotarlo».

\par 4 Entonces Lamec se levantó, tomó un arco que había usado desde su juventud, antes de quedarse ciego, tomó grandes flechas, piedras lisas y una honda que tenía, y salió al campo con el joven. pastor, y se puso detrás del ganado; mientras el joven pastor cuidaba el ganado. Así hizo Lamec durante muchos días.

\par 5 Mientras tanto, Caín, desde que Dios lo había desechado y lo había maldecido con temblor y terror, no pudo establecerse ni encontrar descanso en ningún lugar; pero vagaba de un lugar a otro.

\par 6 Mientras andaba deambulando, se acercó a las mujeres de Lamec y les preguntó por él. Le dijeron: «Está en el campo con el ganado».

\par 7 Entonces Caín fue a buscarlo; y al entrar en el campo, el joven pastor oyó el ruido que hacía y el ganado que pastaba delante de él.

\par 8 Entonces dijo a Lamec: «Oh mi señor, ¿es eso una fiera o un ladrón?»

\par 9 Y Lamec le dijo: «Hazme entender hacia dónde mira cuando sube».

\par 10 Entonces Lamec tensó su arco, colocó una flecha sobre él y colocó una piedra en la honda. Cuando Caín salió del campo, el pastor le dijo a Lamec: «Dispara, he aquí que viene».

\par 11 Entonces Lamec disparó su flecha a Caín y le dio en el costado. Y Lamec lo hirió con una piedra de su honda, que cayó sobre su rostro y le arrancó ambos ojos; Entonces Caín cayó al instante y murió.

\par 12 Entonces Lamec y el joven pastor se acercaron a él y lo encontraron tendido en el suelo. Y el joven pastor le dijo: «¡Es Caín, nuestro abuelo, a quien tú mataste, oh señor mío!»

\par 13 Entonces Lamec se arrepintió y, lleno de amargura, juntó las manos y golpeó con la palma plana la cabeza del joven, que cayó como muerto. pero Lamec pensó que era una finta; Entonces tomó una piedra, lo hirió y le destrozó la cabeza hasta que murió.

\chapter{14}

\par \textit{El tiempo, como una corriente en constante movimiento, se lleva a otra generación de hombres.}

\par 1 Cuando Enós tenía novecientos años, todos los hijos de Set y de Cainán, y sus primogénitos, con sus esposas e hijos, se reunieron a su alrededor, pidiéndole una bendición.

\par 2 Entonces oró sobre ellos, los bendijo y los conjuró con la sangre de Abel el justo, diciéndoles: «Que ninguno de vuestros hijos descienda de este monte santo, ni tengan comunión con los hijos de Caín el asesino».

\par 3 Entonces Enós llamó a su hijo Cainán y le dijo: «Mira, hijo mío, pon tu corazón en tu pueblo y confírmalo en la justicia y la inocencia; y estarás ministrando delante del cuerpo de nuestro padre Adán, todos los días de tu vida».

\par 4 Después de esto, Enós entró en reposo, cuando tenía novecientos ochenta y cinco años; y Cainán le dio cuerda y lo puso en la Cueva de los Tesoros a la izquierda de su padre Adán; y le hicieron ofrendas, según la costumbre de sus padres.

\chapter{15}

\par \textit{La descendencia de Adán continúa conservando la Cueva de los Tesoros como santuario familiar.}

\par 1 DESPUÉS de la muerte de Enós, Cainán se mantuvo a la cabeza de su pueblo con rectitud e inocencia, tal como su padre le había ordenado; también continuó ministrando ante el cuerpo de Adán, dentro de la Cueva de los Tesoros.

\par 2 Cuando vivió novecientos diez años, le sobrevino sufrimiento y aflicción. Y cuando estaba a punto de entrar en reposo, vinieron a él todos los padres con sus mujeres y sus hijos, y él los bendijo, y los conjuró por la sangre de Abel el justo, diciéndoles: Ninguno de vosotros vaya bajando de esta Montaña Sagrada; y no hagáis compañerismo con los hijos de Caín el asesino».

\par 3 Mahalaleel, su primogénito, recibió este mandamiento de su padre, quien lo bendijo y murió.

\par 4 Entonces Mahalaleel lo embalsamó con especias aromáticas y lo puso en la cueva de los tesoros, con sus padres; y le hicieron ofrendas, según la costumbre de sus padres.



\chapter{16}

\par \textit{La buena rama de la familia todavía tiene miedo de los hijos de Caín.}

\par 1 ENTONCES Mahalaleel estuvo sobre su pueblo, y los alimentó con justicia e inocencia, y los vigiló para asegurarse de que no tuvieran relaciones sexuales con los hijos de Caín.

\par 2 También continuó en la Cueva de los Tesoros orando y ministrando ante el cuerpo de nuestro padre Adán, pidiendo a Dios misericordia para sí y para su pueblo; hasta los ochocientos setenta años, cuando enfermó.

\par 3 Entonces todos sus hijos se reunieron junto a él para verlo y pedirle su bendición para todos ellos antes de que dejara este mundo.

\par 4 Entonces Mahalaleel se levantó y se sentó en su cama, con las lágrimas corriendo por su rostro, y llamó a su hijo mayor, Jared, quien vino a él.

\par 5 Entonces besó su rostro y le dijo: «Oh Jared, hijo mío, te conjuro por Aquel que hizo los cielos y la tierra, para que cuides de tu pueblo y lo alimentes con justicia e inocencia; y no dejar que uno de ellos descienda de este Monte Santo a los hijos de Caín, para que no perezca con ellos.

\par 6 «Escucha, hijo mío, que en el futuro vendrá una gran destrucción sobre esta tierra a causa de ellos; Dios se enojará contra el mundo y los destruirá con aguas.

\par 7 «Pero también sé que tus hijos no te escucharán, que descenderán de esta montaña y se unirán a los hijos de Caín, y que perecerán con ellos.

\par 8 «¡Oh hijo mío! enséñales y cuídalos para que ninguna culpa te afecte por causa de ellos».

\par 9 Mahalaleel dijo además a su hijo Jared: Cuando muera, embalsama mi cuerpo y ponlo en la cueva de los tesoros, junto a los cuerpos de mis padres; entonces quédate junto a mi cuerpo y ora a Dios; y cuídalos, y cumple tu ministerio delante de ellos, hasta que tú mismo entres en el reposo».

\par 10 Entonces Mahalaleel bendijo a todos sus hijos; y luego se acostó en su cama y entró en reposo como sus padres.

\par 11 Pero cuando Jared vio que su padre Mahalaleel había muerto, lloró y se entristeció, y abrazó y besó sus manos y sus pies; y también todos sus hijos.

\par 12 Y sus hijos lo embalsamaron cuidadosamente y lo pusieron junto a los cuerpos de sus padres. Entonces se levantaron y hicieron duelo por él cuarenta días.

\chapter{17}

\par \textit{Jared se vuelve martinete. Es atraído a la tierra de Caín, donde ve muchos lugares voluptuosos. Jared apenas escapa con el corazón limpio.}

\par 1 ENTONCES Jared guardó el mandamiento de su padre y se levantó como un león sobre su pueblo. Los alimentó con rectitud e inocencia y les ordenó que no hicieran nada sin su consejo. Porque tenía miedo de que se fueran a los hijos de Caín.

\par 2 ¿Por qué les dio órdenes repetidamente? y continuó haciéndolo hasta el final del año cuatrocientos ochenta y cinco de su vida.

\par 3 Al cabo de estos años, le llegó esta señal. Mientras Jared estaba de pie como un león ante los cuerpos de sus padres, orando y advirtiendo a su pueblo, Satanás lo envidió y realizó una hermosa aparición, porque Jared no permitía que sus hijos hicieran nada sin su consejo.

\par 4 Entonces se le apareció Satanás con treinta hombres de su ejército, en forma de hombres hermosos; El mismo Satanás era el mayor y más alto entre ellos, con una hermosa barba.

\par 5 Se pararon a la entrada de la cueva y llamaron a Jared desde dentro.

\par 6 Salió a ellos y los encontró como hombres espléndidos, llenos de luz y de gran hermosura. Se maravilló de su belleza y de su aspecto; y pensó dentro de sí si no serían de los hijos de Caín.

\par 7 Y dijo también en su corazón: «Como los hijos de Caín no pueden subir a la altura de esta montaña, y ninguno de ellos es tan hermoso como parecen ser; y entre estos hombres no hay ninguno de mi parentela; deben ser extraños.

\par 8 Entonces Jared y ellos intercambiaron un saludo y él dijo al mayor de ellos: «Oh padre mío, explícame el milagro que hay en ti, y dime quiénes son estos que están contigo; porque me parecen hombres extraños».

\par 9 Entonces el mayor se puso a llorar, y los demás lloraron con él; y dijo a Jared: «Yo soy Adán a quien Dios hizo primero; y este es Abel mi hijo, que fue asesinado por su hermano Caín, en cuyo corazón puso Satanás para asesinarlo.

\par 10 «Entonces éste es mi hijo Set, a quien pedí al Señor, que me lo dio, que me consolara en lugar de Abel.

\par 11 «Entonces éste es mi hijo Enós, hijo de Set, y aquel otro es Cainán, hijo de Enós, y aquel otro es Mahalaleel, hijo de Cainán, tu padre».

\par 12 Pero Jared se quedó maravillado por su apariencia y por lo que le había dicho el mayor.

\par 13 Entonces el mayor le dijo: «No te maravilles, hijo mío; vivimos en la tierra al norte del jardín que Dios creó antes del mundo. No nos dejó vivir allí, sino que nos puso dentro del jardín, debajo del cual ahora moráis.

\par 14 «Pero después de mi transgresión, Él me hizo salir de allí y me dejaron vivir en esta cueva; me sobrevinieron grandes y dolorosos problemas; y cuando mi muerte se acercaba, ordené a mi hijo Seth que cuidara de mí. bien a su pueblo; y este mi mandamiento será pasado de unos a otros, hasta el fin de las generaciones venideras.

\par 15 «Pero, oh Jared, hijo mío, nosotros vivimos en regiones hermosas, mientras que tú vives aquí en la miseria, como me informó este tu padre Mahalaleel, diciéndome que vendrá un gran diluvio y cubrirá toda la tierra.

\par 16 Por eso, hijo mío, temiendo por ti, me levanté, tomé a mis hijos conmigo y vine aquí para visitarte a ti y a tus hijos; pero te encontré llorando en esta cueva y a tus hijos dispersos. sobre esta montaña, en el calor y en la miseria.

\par 17 «Pero, hijo mío, al desviarnos del camino y llegar hasta aquí, encontramos otros hombres debajo de esta montaña, que habitan en una tierra hermosa, llena de árboles y de frutas, y de toda clase de verdor. ; es como un jardín; de modo que cuando los encontramos pensamos que eran vosotros; hasta que tu padre Mahalaleel me dijo que no eran tal cosa.

\par 18 Ahora pues, hijo mío, escucha mi consejo y desciende a ellos, tú y tus hijos. Descansaréis de todo este sufrimiento en el que os encontráis. Pero si no quieres descender a ellos, levántate, toma a tus hijos y ven con nosotros a nuestro jardín; Vivirás en nuestra hermosa tierra y descansarás de todos estos problemas que tú y tus hijos ahora estáis soportando».

\par 19 Pero Jared, al oír este discurso del mayor, se maravilló; y fue de aquí para allá, pero en ese momento no encontró a ninguno de sus hijos.

\par 20 Entonces él respondió y dijo al anciano: «¿Por qué os habéis escondido hasta el día de hoy?»

\par 21 El anciano respondió: Si tu padre no nos lo hubiera dicho, no lo habríamos sabido.

\par 22 Entonces Jared creyó que sus palabras eran ciertas.

\par 23 Entonces el anciano dijo a Jared: «¿Por qué te volviste fulano de tal?» Y él dijo: Buscaba a uno de mis hijos, para contarle de mi ida contigo, y de su venida a aquellos de quienes me has hablado.

\par 24 Cuando el anciano escuchó la intención de Jared, le dijo: Deja ese propósito por ahora, y ven con nosotros; verás nuestro país; Si la tierra en que habitamos te agrada, nosotros y tú regresaremos acá y tomaremos a tu familia con nosotros. Pero si nuestro país no te agrada, volverás a tu propio lugar».

\par 25 Y el mayor instó a Jared a que se fuera antes de que uno de sus hijos viniera a aconsejarle lo contrario.

\par 26 Entonces Jared salió de la cueva y fue con ellos y entre ellos. Y lo consolaron, hasta que llegaron a la cumbre del monte de los hijos de Caín.

\par 27 Entonces el anciano dijo a uno de sus compañeros: «Hemos olvidado algo en la entrada de la cueva, y es la prenda escogida que habíamos traído para vestir a Jared».

\par 28 Entonces dijo a uno de ellos: «Vuelve tú, alguien; y aquí te esperaremos hasta que regreses. Entonces vestiremos a Jared y él será como nosotros, bueno, apuesto y apto para venir con nosotros a nuestro país».

\par 29 Entonces aquel volvió.

\par 30 Pero cuando estaba a poca distancia, el mayor lo llamó y le dijo: «Quédate hasta que yo suba y te hable».

\par 31 Entonces se quedó quieto, y el anciano se acercó a él y le dijo: «Una cosa que olvidamos en la cueva es esto: apagar la lámpara que arde dentro de ella, encima de los cuerpos que están dentro». . Entonces regresa con nosotros, rápido».

\par 32 Éste se fue, y el mayor volvió con sus compañeros y con Jared. Y descendieron del monte, y Jared con ellos; y se quedaron junto a una fuente de agua, cerca de las casas de los hijos de Caín, y esperaron a su compañero hasta que trajo el manto para Jared.

\par 33 Entonces él, que había regresado a la cueva, apagó la lámpara, se acercó a ellos y trajo consigo un fantasma y se lo mostró. Y cuando Jared lo vio, se maravilló de su belleza y gracia, y se regocijó en su corazón al creer que todo era verdad.

\par 34 Pero mientras estaban allí, tres de ellos entraron en las casas de los hijos de Caín y les dijeron: «Tráigannos hoy algo de comida junto a la fuente de agua, para que comamos nosotros y nuestros compañeros».

\par 35 Pero cuando los hijos de Caín los vieron, se maravillaron y pensaron: «Estos son hermosos a la vista, y como nunca antes los habíamos visto». Entonces se levantaron y fueron con ellos a la fuente de agua, para ver a sus compañeros.

\par 36 Los encontraron tan hermosos que gritaron en voz alta acerca de sus lugares para que otros se reunieran y vinieran a contemplar a estos hermosos seres. Luego se reunieron alrededor de ellos hombres y mujeres.

\par 37 Entonces el anciano les dijo: «Somos extranjeros en vuestra tierra; traednos buena comida y bebida, vosotros y vuestras mujeres, para refrescarnos con vosotros».

\par 38 Cuando aquellos hombres oyeron estas palabras del mayor, cada uno de los hijos de Caín trajo a su esposa, y otro trajo a su hija, y así vinieron a ellos muchas mujeres; cada uno dirigiéndose a Jared ya sea por sí mismo o por su esposa; todos iguales.

\par 39 Pero cuando Jared vio lo que hacían, su misma alma se arrancó de ellos; ni probaría ni su comida ni su bebida.

\par 40 El anciano se dio cuenta y se separó de ellos y le dijo: «No estés triste; Soy el gran anciano, como me verás hacer, haz lo mismo tú mismo».

\par 41 Entonces extendió sus manos y tomó a una de las mujeres, y cinco de sus compañeros hicieron lo mismo delante de Jared, para que él hiciera lo mismo que ellos.

\par 42 Pero cuando Jared los vio cometer infamia, lloró y dijo en su mente: Mis padres nunca hicieron algo así.

\par 43 Entonces extendió las manos y oró con corazón ferviente y con mucho llanto, y suplicó a Dios que lo librara de sus manos.

\par 44 Tan pronto como Jared comenzó a orar, el mayor huyó con sus compañeros; porque no podían permanecer en un lugar de oración.

\par 45 Entonces Jared se volvió, pero no pudo verlos, sino que se encontró en medio de los hijos de Caín.

\par 46 Entonces lloró y dijo: «Oh Dios, no me destruyas con esta raza, acerca de la cual mis padres me han advertido; porque ahora, oh Dios mío, pensaba que los que se me aparecieron eran mis padres; pero He descubierto que eran demonios que me sedujeron con esta hermosa aparición, hasta que les creí.

\par 47 Pero ahora te pido, oh Dios, que me liberes de esta raza en la que ahora estoy, como me libraste de aquellos demonios. Envía a tu ángel para que me saque de en medio de ellos, porque Yo mismo no tengo poder para escapar de entre ellos.»

\par 48 Cuando Jared terminó su oración, Dios envió su ángel en medio de ellos, quien tomó a Jared y lo puso en la montaña, le mostró el camino, le dio consejos y luego se alejó de él.

\chapter{18}

\par \textit{Confusión en la Cueva de los Tesoros. Discurso milagroso del difunto Adán.}

\par 1 LOS hijos de Jared tenían la costumbre de visitarlo hora tras hora para recibir su bendición y pedirle consejo para cada cosa que hacían; y cuando tenía una obra que hacer, la hacían por él.

\par 2 Pero esta vez, cuando entraron en la cueva, no encontraron a Jared, sino que encontraron la lámpara apagada, y los cuerpos de los padres tirados por todos lados, y de ellos salían voces por el poder de Dios, que decían: «Satanás en una aparición ha engañado a nuestro hijo, queriendo destruirlo, como destruyó a nuestro hijo Caín».

\par 3 Dijeron también: «Señor Dios del cielo y de la tierra, libra a nuestro hijo de la mano de Satanás, que realizó ante él una aparición grande y falsa». También hablaron de otras cosas, por el poder de Dios.

\par 4 Pero cuando los hijos de Jared oyeron estas voces, temieron y lloraron por su padre; porque no sabían lo que le había sucedido.

\par 5 Y aquel día lloraron por él hasta la puesta del sol.

\par 6 Entonces vino Jared con el rostro triste, afligido de mente y cuerpo, y afligido por haber sido separado de los cuerpos de sus padres.

\par 7 Pero cuando se acercaba a la cueva, sus hijos lo vieron y corrieron hacia la cueva y se colgaron de su cuello, gritando y diciéndole: «Padre, ¿dónde has estado y por qué has estado? ¿Nos dejaste, como no solías hacerlo? Y nuevamente: «Oh padre, cuando desapareciste, la lámpara sobre los cuerpos de nuestros padres se apagó, los cuerpos fueron arrojados y de ellos salieron voces».

\par 8 Cuando Jared oyó esto, se arrepintió y entró en la cueva; y allí encontraron los cuerpos tirados, la lámpara apagada y los padres mismos orando por su liberación de las manos de Satanás.

\par 9 Entonces Jared se arrojó sobre los cadáveres, los abrazó y dijo: «¡Oh padres míos, por vuestra intercesión, que Dios me libre de las manos de Satanás! Y te ruego que le pidas a Dios que me guarde y me aleje de él hasta el día de mi muerte».

\par 10 Entonces todas las voces cesaron, excepto la voz de nuestro padre Adán, quien habló a Jared por el poder de Dios, tal como uno hablaría a su prójimo, diciendo: «Oh Jared, hijo mío, ofrece regalos a Dios por haberte te libró de la mano de Satanás; y cuando traigas esas ofrendas, así sea que las ofrezcas sobre el altar en el cual yo ofrecí. Entonces también, guardate de Satanás, porque muchas veces me engañó con sus apariciones, deseando para destruirme, pero Dios me libró de su mano.

\par 11 «Ordena a tu pueblo que esté en guardia contra él; y nunca dejes de ofrecer regalos a Dios».

\par 12 Entonces también la voz de Adán enmudeció; y Jared y sus hijos se maravillaron de esto. Luego colocaron los cuerpos como estaban primero; y Jared y sus hijos permanecieron orando toda aquella noche, hasta el amanecer.

\par 13 Entonces Jared hizo una ofrenda y la ofreció sobre el altar, tal como Adán le había ordenado. Y mientras subía al altar, oró a Dios pidiendo misericordia y perdón de su pecado respecto a la lámpara que se apagaba.

\par 14 Entonces Dios se apareció a Jared en el altar y lo bendijo a él y a sus hijos, y aceptó sus ofrendas; y ordenó a Jared que tomara del fuego sagrado del altar, y con él encendiera la lámpara que iluminaría el cuerpo de Adán.

\chapter{19}

\par \textit{Los hijos de Jared están descarriados.}

\par 1 ENTONCES Dios le reveló nuevamente la promesa que le había hecho a Adán; Le explicó los 5500 años y le reveló el misterio de Su venida a la tierra.

\par 2 Y dijo Dios a Jared: En cuanto al fuego que tomaste del altar para encender la lámpara, quede contigo para alumbrar los cuerpos, y no salga de la cueva hasta que De allí sale el cuerpo de Adán.

\par 3 Pero, oh Jared, cuida el fuego, que arda brillantemente en la lámpara; ni vuelvas a salir de la cueva, hasta que recibas una orden mediante una visión, y no en una aparición, cuando seas visto por ti.

\par 4 »Entonces ordena otra vez a tu pueblo que no se relacione con los hijos de Caín ni aprenda sus caminos; porque yo soy Dios que no ama el odio ni las obras de iniquidad».

\par 5 Dios también dio a Jared muchos otros mandamientos y lo bendijo. Y luego retiró Su Palabra de él.

\par 6 Entonces Jared se acercó con sus hijos, tomó fuego, bajó a la cueva y encendió la lámpara delante del cuerpo de Adán; y dio mandamientos a su pueblo como Dios le había dicho que hiciera.

\par 7 Esta señal le sucedió a Jared al final de su año cuatrocientos cincuenta; como también muchas otras maravillas, no las registramos. Pero sólo registramos este para abreviar y para no alargar nuestra narración.

\par 8 Y Jared continuó enseñando a sus hijos ochenta años; pero después comenzaron a transgredir los mandamientos que él les había dado y a hacer muchas cosas sin su consejo. Comenzaron a descender uno tras otro de la Montaña Sagrada y a mezclarse con los hijos de Caín en malas comunidades.

\par 9 Ahora bien, la razón por la cual los hijos de Jared descendieron del Monte Santo es ésta, que ahora os revelaremos.



\chapter{20}

\par \textit{Música deslumbrante; Bebida fuerte desatada entre los hijos de Caín. Se visten con ropa colorida. Los hijos de Set miran con ojos anhelantes. Se rebelan contra los sabios consejos; descienden de la montaña al valle de la iniquidad. No pueden volver a subir a la montaña.}

\par 1 DESPUÉS que Caín descendió a la tierra oscura, y sus hijos se multiplicaron en ella, estaba uno de ellos, que se llamaba Genún, hijo de Lamec el ciego que mató a Caín.

\par 2 Pero a este Genun, Satanás entró en él cuando era niño; E hizo muchas trompetas, trompetas, instrumentos de cuerda, címbalos, salterios, liras, arpas y flautas; y tocaba con ellos en todo momento y a cada hora.

\par 3 Y cuando tocaba con ellos, Satanás entraba en ellos, de modo que de entre ellos se oían sonidos hermosos y dulces que embelesaban el corazón.

\par 4 Entonces reunió compañías tras compañías para jugar con ellas; y cuando jugaban, agradaba mucho a los hijos de Caín, que se inflamaban entre sí con el pecado, y ardían como a fuego; mientras que Satanás inflamaba sus corazones unos con otros, y aumentaba la lujuria entre ellos.

\par 5 Satanás también enseñó a Genún a sacar bebida fuerte del grano; y este Genun solía reunir compañías tras compañías en tabernas; y trajeron en sus manos toda clase de frutos y flores; y bebieron juntos.

\par 6 Así multiplicó este Genun el pecado en gran manera; también actuó con orgullo y enseñó a los hijos de Caín a cometer toda clase de las más graves maldades que no conocían; y los sometió a múltiples hechos que antes no conocían.

\par 7 Entonces Satanás, cuando vio que se sometían a Genun y le escuchaban en todo lo que les decía, se alegró mucho y aumentó el entendimiento de Genun, hasta que tomó hierro y con él hizo armas de guerra.

\par 8 Entonces, cuando se emborracharon, aumentó entre ellos el odio y el homicidio; un hombre usó la violencia contra otro para enseñarle el mal, tomando a sus hijos y profanándolos delante de él.

\par 9 Y cuando los hombres vieron que estaban vencidos y vieron a otros que no estaban vencidos, los que estaban vencidos vinieron a Genun, se refugiaron en él y él los hizo sus aliados.

\par 10 Entonces el pecado aumentó mucho entre ellos; hasta que un hombre se casa con su propia hermana, o hija, o madre, y otras; o la hija de la hermana de su padre, de modo que no hubo más distinción de parentesco, y ya no conocieron lo que es iniquidad; sino que hicieron lo malo, y la tierra se contaminó con el pecado, y enojaron al Dios Juez que los había creado.

\par 11 Pero Genun reunió compañías tras compañías que tocaban trompetas y todos los demás instrumentos que ya hemos mencionado, al pie del Monte Santo; y lo hicieron para que lo oyeran los hijos de Set que estaban en el Monte Santo.

\par 12 Pero cuando los hijos de Set oyeron el ruido, se maravillaron y vinieron en grupos y se pararon en la cima de la montaña para mirar a los que estaban abajo; y así hicieron durante todo un año.

\par 13 Cuando, al final de aquel año, Genun vio que poco a poco se los estaban ganando, Satanás entró en él y le enseñó a hacer sustancias para teñir vestidos de diversos diseños, y le hizo comprender cómo teñirse de carmesí y morado y todo eso.

\par 14 Y los hijos de Caín, que habían hecho todo esto y brillaban con hermosura y ropa espléndida, se reunieron al pie de la montaña en esplendor, con cuernos y vestidos lujosos y carreras de caballos, cometiendo toda clase de abominaciones.

\par 15 Mientras tanto, los hijos de Set, que estaban en el Monte Santo, oraron y alabaron a Dios, en lugar de las huestes de ángeles que habían caído; Por eso Dios los había llamado ángeles, porque se regocijaba mucho por ellos.

\par 16 Pero después de esto, ya no guardaron sus mandamientos ni cumplieron la promesa que había hecho a sus padres; pero se relajaron del ayuno y la oración, y del consejo de Jared su padre. Y seguían reuniéndose en la cima del monte, desde la mañana hasta la tarde, para contemplar a los hijos de Caín, y ver lo que hacían, sus hermosos vestidos y adornos.

\par 17 Entonces los hijos de Caín miraron desde abajo y vieron a los hijos de Set, de pie en tropas en la cima de la montaña; y les gritaron que descendieran a ellos.

\par 18 Pero los hijos de Set les dijeron desde arriba: «No conocemos el camino». Entonces Genún, hijo de Lamec, les oyó decir que no conocían el camino, y pensó cómo podría derribarlos.

\par 19 Entonces Satanás se le apareció de noche y le dijo: «No tienen manera de bajar del monte en el que habitan; pero cuando vengan mañana, diles: 'Venid al lado occidental de la montaña; allí encontraréis el camino de un arroyo de agua, que baja al pie del monte, entre dos cerros; baja por ese camino hasta nosotros'».

\par 20 Entonces, cuando ya era de día, Genun tocó los cuernos y tocó los tambores debajo de la montaña, como tenía por costumbre. Los hijos de Set lo oyeron y vinieron como solían hacerlo.

\par 21 Entonces Genún les dijo desde abajo: «Vayan al lado occidental de la montaña, allí encontrarán el camino para bajar».

\par 22 Pero cuando los hijos de Set oyeron estas palabras de él, regresaron a la cueva donde Jared para contarle todo lo que habían oído.

\par 23 Entonces Jared, al oírlo, se entristeció; porque sabía que transgredirían su consejo.

\par 24 Después de esto, se reunieron cien hombres de los hijos de Set y se dijeron entre ellos: «Venid, bajemos a los hijos de Caín, veamos qué hacen y disfrutemos con ellos».

\par 25 Pero cuando Jared oyó esto de los cien hombres, su alma se conmovió y su corazón se entristeció. Entonces se levantó con gran fervor, se puso en medio de ellos y los conjuró por la sangre de Abel el justo: «Ninguno de vosotros descienda de este monte santo y puro, en el cual nuestros padres le ordenaron habitar. .»

\par 26 Pero cuando Jared vio que no recibían sus palabras, les dijo: «Oh, mis buenos, inocentes y santos hijos, sepan que una vez que bajéis de esta montaña santa, Dios no os permitirá regresar otra vez. lo.»

\par 27 Volvió a conjurarlos, diciendo: «Conjuro por la muerte de nuestro padre Adán y por la sangre de Abel, de Set, de Enós, de Cainán y de Mahalaleel, que me escuchen y no vayan. bajando de este santo monte; pues en el momento en que la dejéis, quedaréis privados de vida y de misericordia; y ya no seréis llamados 'hijos de Dios', sino 'hijos del diablo'».

\par 28 Pero ellos no quisieron escuchar sus palabras.

\par 29 Enoc ya era adulto en aquel tiempo y, en su celo por Dios, se levantó y dijo: Oídme, oh hijos de Set, pequeños y grandes, cuando quebrantéis el mandamiento de nuestros padres y vayáis. Bajad de este monte santo, y no volveréis a subir acá jamás».

\par 30 Pero ellos se levantaron contra Enoc y no quisieron escuchar sus palabras, sino que descendieron del Monte Santo.

\par 31 Y cuando miraron a las hijas de Caín, sus hermosas figuras, sus manos y pies teñidos de colores y tatuados con adornos en sus rostros, el fuego del pecado se encendió en ellas.

\par 32 Entonces Satanás los hizo parecer más hermosos ante los hijos de Set, como también hizo que los hijos de Set parecieran más hermosos a los ojos de las hijas de Caín, de modo que las hijas de Caín codiciaron a los hijos de Set como bestias rapaces, y los hijos de Set después de las hijas de Caín, hasta que cometieron con ellos abominación.

\par 33 Pero después de caer en esta contaminación, regresaron por el camino por donde habían venido y trataron de ascender a la Montaña Sagrada. Pero no pudieron, porque las piedras de aquel monte santo eran de fuego que centelleaba ante ellos, por lo que no podían volver a subir.

\par 34 Y Dios se enojó con ellos y se arrepintió de ellos porque habían descendido de la gloria y por eso habían perdido o abandonado su propia pureza o inocencia, y habían caído en la contaminación del pecado.

\par 35 Entonces Dios envió Su Palabra a Jared, diciendo: «Estos tus hijos, a quienes llamaste 'Mis hijos', he aquí, han transgredido Mi mandamiento y han descendido a la morada de la perdición y del pecado. Envía un mensajero a los que quedan, para que no bajen y se pierdan».

\par 36 Entonces Jared lloró delante del Señor y le pidió misericordia y perdón. Pero deseaba que su alma se separara del cuerpo, antes que escuchar estas palabras de Dios sobre el descenso de sus hijos del Monte Santo.

\par 37 Pero él siguió la orden de Dios y les predicó que no descendieran de aquel monte santo ni tuvieran relaciones con los hijos de Caín.

\par 38 Pero ellos no escucharon su mensaje ni quisieron obedecer su consejo.

\chapter{21}

\par \textit{Jared muere en pena por sus hijos que se habían extraviado. Una predicción del Diluvio.}

\par 1 DESPUÉS de esto se reunió otro grupo y fueron a cuidar de sus hermanos; pero ellos perecieron también como ellos. Y así fue, empresa tras empresa, hasta que sólo quedaron unas pocas.

\par 2 Entonces Jared enfermó de tristeza, y su enfermedad fue tal que el día de su muerte se acercaba.

\par 3 Entonces llamó a Enoc, su hijo mayor, y a Matusalén, Enoc, hijo de Matusalén, y a Lamec, hijo de Matusalén, y a Noé, hijo de Lamec.

\par 4 Y cuando llegaron a él, oró sobre ellos y los bendijo, y les dijo: «Ustedes son hijos justos, inocentes; no bajen de este monte santo; porque he aquí, sus hijos y los hijos de sus hijos tienen descendieron de este monte santo y se alejaron de este monte santo por su abominable concupiscencia y transgresión del mandamiento de Dios.

\par 5 «Pero sé por el poder de Dios que Él no os dejará en este monte santo, porque vuestros hijos han transgredido Su mandamiento y el de nuestros padres, que habíamos recibido de ellos.

\par 6 «Pero, hijos míos, Dios os llevará a una tierra extraña, y nunca más volveréis a contemplar con vuestros ojos este jardín y este monte santo.

\par 7 «Por tanto, oh hijos míos, fijad vuestro corazón en vosotros mismos y guardad el mandamiento de Dios que está con vosotros. Y cuando vayáis de este monte santo a una tierra extraña que no conocéis, llévate contigo el cuerpo de nuestro padre Adán, y con él estos tres preciosos regalos y ofrendas, a saber, el oro, el incienso y la mirra; y que estén en el lugar donde yacerá el cuerpo de nuestro padre Adán.

\par 8 «Y a aquel de vosotros que quede, oh hijos míos, vendrá la Palabra de Dios, y cuando salga de esta tierra, llevará consigo el cuerpo de nuestro padre Adán, y lo pondrá en en medio de la tierra, el lugar en el que se realizará la salvación».

\par 9 Entonces Noé le dijo: «¿Quién es el que quedará de nosotros?»

\par 10 Y Jared respondió: «Tú eres el que quedará. Y sacarás el cuerpo de nuestro padre Adán de la cueva y lo colocarás contigo en el arca cuando venga el diluvio.

\par 11 «Y tu hijo Sem, que saldrá de tus lomos, él es quien pondrá el cuerpo de nuestro padre Adán en medio de la tierra, en el lugar de donde vendrá la salvación».

\par 12 Entonces Jared se volvió hacia su hijo Enoc y le dijo: «Tú, hijo mío, permanece en esta cueva y ministra diligentemente ante el cuerpo de nuestro padre Adán todos los días de tu vida; y alimenta a tu pueblo con justicia e inocencia».

\par 13 Y Jared no dijo más. Se le soltaron las manos, se le cerraron los ojos y entró en reposo como sus padres. Su muerte tuvo lugar en el año trescientos sesenta de Noé, y en el año novecientos ochenta y nueve de su propia vida; el día doce de Takhsas un viernes.

\par 14 Pero cuando Jared murió, las lágrimas corrían por su rostro a causa del gran dolor que sentía por los hijos de Set, que habían caído en sus días.

\par 15 Entonces Enoc, Matusalén, Lamec y Noé, estos cuatro, lloraron sobre él; Lo embalsamó cuidadosamente y luego lo puso en la Cueva de los Tesoros. Entonces se levantaron y lo lloraron durante cuarenta días.

\par 16 Y cuando terminaron estos días de luto, Enoc, Matusalén, Lamec y Noé quedaron afligidos de corazón porque su padre se había apartado de ellos y no lo vieron más.

\chapter{22}

\par \textit{Solo quedan tres hombres justos en el mundo. Las malas condiciones de los hombres antes del Diluvio.}

\par 1 PERO Enoc guardó el mandamiento de Jared su padre y continuó ministrando en la cueva.

\par 2 Es a este Enoc a quien le sucedieron muchas maravillas, y quien también escribió un libro célebre; pero es posible que esas maravillas no se cuenten en este lugar.

\par 3 Después de esto, los hijos de Set se extraviaron y cayeron, ellos, sus hijos y sus mujeres. Y cuando Enoc, Matusalén, Lamec y Noé los vieron, sus corazones sufrieron a causa de su caída en la duda llena de incredulidad; y lloraron y pidieron a Dios misericordia para preservarlos y sacarlos de esa generación malvada.

\par 4 Enoc continuó su ministerio ante el Señor trescientos ochenta y cinco años, y al final de ese tiempo se dio cuenta, por la gracia de Dios, de que Dios tenía la intención de quitarlo de la tierra.

\par 5 Entonces dijo a su hijo: «Hijo mío, sé que Dios tiene la intención de traer las aguas del Diluvio sobre la tierra y destruir nuestra creación.

\par 6 »Y vosotros sois los últimos gobernantes de este pueblo en esta montaña; porque sé que no os quedará nadie que engendre hijos en este monte santo; ni ninguno de vosotros se enseñoreará de los hijos de su pueblo; ni quedará de vosotros mucha compañía en este monte».

\par 7 Enoc también les dijo: «Cuidad vuestras almas y manteneos firmes en vuestro temor de Dios y en vuestro servicio a Él, y adoradle con fe recta y servidle en justicia, inocencia y juicio, en arrepentimiento y también en la pureza.»

\par 8 Cuando Enoc hubo terminado sus mandamientos para ellos, Dios lo transportó de esa montaña a la tierra de la vida, a las mansiones de los justos y de los elegidos, la morada del Paraíso de alegría, en la luz que llega hasta el cielo; luz que está fuera de la luz de este mundo; porque es la luz de Dios que llena el mundo entero, pero que ningún lugar puede contener.

\par 9 Así, como Enoc estaba en la luz de Dios, se encontró fuera del alcance de la muerte; hasta que Dios quisiera que muriera.

\par 10 En total, ninguno de nuestros padres ni ninguno de sus hijos permaneció en aquel monte santo, excepto esos tres: Matusalén, Lamec y Noé. Porque todos los demás descendieron del monte y cayeron en pecado con los hijos de Caín. Por eso se les prohibió esa montaña, y no quedó en ella nadie más que esos tres hombres.


\end{document}