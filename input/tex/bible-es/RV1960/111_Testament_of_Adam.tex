\begin{document}

\title{Testamento de Adán}

\chapter{1}

\par \textit{[El texto etíope y una versión árabe fueron publicados por Bezold en Festschrift de Nöldeke, Gieszen, 1906.]}

\par LAS HORAS DEL DÍA.

\par 1 Y además, comprende las horas del día y de la noche, y cómo conviene suplicar a Dios y orarle en cada una de sus estaciones. Porque todo esto me enseñó mi Creador, y me dijo los nombres de todos los animales y bestias salvajes, y de las aves del cielo, y luego Dios me hizo entender el número de las horas del día y de la noche, y Me contó cómo los ángeles alaban a Dios. Comprende, entonces, oh hijo mío, que a la primera hora del día la oración de mis hijos asciende a Dios. Y a la hora segunda tiene lugar la oración y petición de los ángeles. A la hora tercera las aves del cielo lo alaban. Y a la hora cuarta los seres espirituales le adoran. Y a la hora quinta le saludan todas las fieras y animales salvajes. A la hora sexta tiene lugar la petición de los Kîrûbêl (Querubines). Y a la hora séptima todos los ángeles entran en la presencia de Dios y salen de allí, porque a esta hora la oración de todo ser viviente asciende a Dios. A la hora octava, los resplandecientes habitantes del cielo lo alaban. Y a la hora nona los ángeles de Dios que están delante del trono del Altísimo le rinden homenaje. Y a la hora décima el Espíritu Santo cubre las aguas con su sombra, y los demonios huyen y se alejan de las aguas. Y si el Espíritu Santo no cubriera las aguas a esta hora todos los días, nadie podría beber de las aguas, [porque si lo hiciera] su carne (es decir, el cuerpo) sería destruida por los demonios malvados. Y si el sacerdote toma agua en esta hora y mezcla con ella aceite santo, y unge con la mezcla a los enfermos y a los poseídos de espíritus inmundos, quedarán sanados de su enfermedad. Y a la hora undécima tienen lugar las glorificaciones de los justos. Y a la hora duodécima Dios, el Altísimo, recibe las oraciones y peticiones de los hijos de los hombres.

\chapter{2}

\par LAS HORAS DE LA NOCHE.

\par 1 Y a primera hora de la noche los demonios dan gracias y alabanzas al Dios Altísimo, y no hay en ellos ningún mal ni daño para nadie hasta que hayan terminado su servicio de homenaje. Y a la segunda hora de la noche los peces y todo ser que está en las aguas alaban a Dios, y las fieras y las ballenas. Y a la hora tercera el fuego le alaba; ahora está en lo más profundo, y en esa hora nadie puede dirigirse a Él (?). Y a la hora cuarta los Sûrâfêl (Serafines) lo proclaman Santo. Y a la hora quinta le alaban las aguas que están sobre los cielos. Hace mucho tiempo me senté y escuché a los ángeles en esta hora, y [maravillado] cómo clamaban; [su clamor] era como el ruido de una rueda poderosa, y clamaban como las olas del mar con voz de alabanza a Dios. Y a la hora sexta las nubes alabaron a Dios con temor y temblor. Y a la hora séptima la tierra quedó en silencio, y todo ser que estaba sobre ella, y las aguas se adormecieron. Y si en esta hora el sacerdote toma agua y mezcla con ella aceite santo, y unge con ella a los enfermos y a los que de noche no pueden dormir a causa de sus dolores, los enfermos serán sanados, y los que estén despiertos se quedará dormido. A la hora octava la tierra hace crecer hierba y hierbas verdes, y hace que los árboles produzcan hojas y frutos. Y a la hora novena los ángeles realizan su servicio de homenaje a Dios, y la oración de los hijos de los hombres llega a la presencia del Dios Altísimo. Y a la hora décima se abren las puertas del cielo, y Dios oye la oración de los hijos de los creyentes, y les es concedida la petición que piden a Dios; Y al sonido de las alas de los Serafines en ese momento los gallos cantan y alaban a Dios. Y a la hora undécima hay gozo y alegría en toda la tierra, porque el sol entra en el Jardín (es decir, el Paraíso), y su luz sale por todos los confines del mundo e ilumina toda cosa creada. Y a la hora duodécima es propio que mis hijos se pongan de pie ante Dios y le rindan homenaje, porque a esta hora reposa un gran silencio sobre todos los seres celestiales.

\chapter{3}

\par ADÁN PREDICA LA VENIDA DE CRISTO.

\par 1 Ahora, pues, conoce todo esto y escucha mi palabra, y entiende que la Palabra de Dios, el Altísimo, descenderá sobre la tierra, tal como Él me dijo en el momento en que me expulsó de la tierra. Jardín (Paraíso). Porque Él me dijo que Su Palabra en días posteriores se haría hombre de una mujer virgen que se llamaba María, y se escondería en ella, y se vestiría de carne, y nacería como un hombre con gran poder, y habilidad operativa y conocimiento. Nadie le conocerá excepto Él mismo y aquel a quien se manifestó. Y Dios dijo que andaría con los hombres sobre la tierra, y crecería en días y años, y haría señales y prodigios abiertamente, y caminaría sobre el mar como sobre tierra seca, y reprendería abiertamente al mar y a los vientos, y que se sujetaran a Él, y que Él clamara a las olas del mar y ellas le respondieran rápidamente. Y que haría a los ciegos ver, y a los leprosos ser limpiados, y a los sordos oír, y a los mudos hablar, y levantar a los paralíticos, y hacer andar a los cojos, y convertir a muchos del error en el conocimiento de Dios, y debería expulsar a los demonios de los hombres.

\par 2 Y además [de estas cosas] Dios me habló, diciendo: «No te entristezcas, oh Adán, porque quisiste convertirte en un dios y transgrediste mi orden. He aquí, yo te afirmaré, no ahora, sino dentro de algunos días». Y otra vez me habló, diciendo: «Yo soy el Dios que te hizo salir del Jardín del Gozo a la tierra, de la cual brotarán espinos y zarzas, y en ella habitarás. Dobla tu espalda y haz que tus rodillas tiemblen en la vejez, y haré de tu carne pasto para los gusanos. Y después de cinco días y medio tendré compasión de ti, y te haré misericordia en la abundancia de mi compasión y de mi misericordia. Y descenderé a tu casa, y habitaré en tu carne, y por ti me agradaré nacer como un niño [ordinario]. Y por ti me complacerá caminar por la plaza. Y por ti me complacerá ayunar cuarenta días. Y por ti estaré encantado de aceptar el bautismo. Y por ti me complacerá soportar el sufrimiento. Y por ti me placeré colgar del madero de la Cruz. Todas estas cosas [haré] por ti, oh Adán».

\par 3 A él sean la alabanza, la majestad, el dominio, la gloria, la adoración y los himnos, con su Padre y el Espíritu Santo, desde ahora y por los siglos de los siglos. Amén.

\par 4 Además, debes saber, oh hijo mío, Seth, he aquí que vendrá un diluvio y lavará toda la tierra a causa de los hijos de Kâyal (Caín), el asesino, que mató a su hermano por celos, a causa de su hermana. Lûd. Y después del Diluvio y de muchas semanas vendrán los últimos días, y todo se completará, y vendrá su tiempo y el fuego consumirá todo lo que se encuentre delante de Dios, y la tierra será santificada, y el Señor de Señores caminará por ahí. en eso.

\par 5 Y Set escribió este mandamiento y lo selló con su sello, y con el sello de su padre Adán, que se llevó consigo del Jardín (paraíso), y con el sello de Eva su madre.

\end{document}