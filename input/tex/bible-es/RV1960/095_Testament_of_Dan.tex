\begin{document}

\title{Testamento de Dan}

\chapter{1}

\par \textit{El séptimo hijo de Jacob y Bilhah. El celoso. Aconseja contra la ira diciendo que «da una visión peculiar». Esta es una tesis notable sobre la ira.}

\par 1 Copia de las palabras que Dan habló a sus hijos en sus últimos días, en el año ciento veinticinco de su vida.

\par 2 Porque reunió a su familia y dijo: ¡Oíd mis palabras, hijos de Dan! y presta atención a las palabras de tu padre.

\par 3 He demostrado en mi corazón y en toda mi vida que la verdad

el trato justo es bueno y agradable a Dios, y que la mentira y la ira son malas, porque enseñan al hombre toda maldad.

\par 4 Por tanto, hoy os confieso, hijos míos, que en mi corazón he decidido la muerte de José, mi hermano, el hombre verdadero y bueno. .

\par 5 Y me alegré de que lo hubieran vendido, porque su padre lo amaba más que a nosotros.

\par 6 Porque el espíritu de celos y de vanagloria me dijo: Tú también eres su hijo.

\par 7 Y uno de los espíritus de Beliar me inquietó, diciendo: Toma esta espada y mata con ella a José: así te amará tu padre cuando muera.

\par 8 Ahora bien, este es el espíritu de ira que me persuadió a aplastar a José como el leopardo aplasta a un cabrito.

\par 9 Pero el Dios de mis padres no permitió que cayera en mis manos, para que yo lo encontrara solo, lo matara y destruyera una segunda tribu en Israel.

\par 10 Y ahora, hijos míos, he aquí que estoy muriendo, y os digo en verdad que, a menos que os guardéis del espíritu de mentira y de ira, y améis la verdad y la paciencia, pereceréis.

\par 11 Porque la ira es ceguera y no permite ver el rostro de nadie con la verdad.

\par 12 Porque aunque sea padre o madre, se comporta con ellos como enemigos; aunque sea hermano, no lo conoce; aunque sea profeta del Señor, le desobedece; Aunque es justo, no lo mira; aunque es amigo, no lo reconoce.

\par 13 Porque el espíritu de ira lo rodea con la red del engaño, ciega sus ojos, oscurece su mente con la mentira y le da su propia visión.

\par 14 ¿Y con qué envuelve sus ojos? Con odio de corazón, para tener envidia de su hermano.

\par 15 Porque, hijos míos, la ira es algo malo, pues perturba hasta el alma misma.

\par 16 Y hace suyo el cuerpo del hombre enojado, y domina su alma, y ​​concede al cuerpo poder para que pueda cometer toda iniquidad.

\par 17 Y cuando el cuerpo hace todas estas cosas, el alma justifica lo que hace, porque no ve bien.

\par 18 Por tanto, el que está enojado, si es un hombre valiente, tiene tres poderes en su ira: uno, con la ayuda de sus servidores; y un segundo por su riqueza, por la cual persuade y vence injustamente; y en tercer lugar, teniendo su propio poder natural, obra el mal.

\par 19 Y aunque el hombre iracundo sea débil, tiene el doble de poder del que tiene por naturaleza; porque la ira siempre ayuda a los que cometen anarquía.

\par 20 Este espíritu acompaña siempre a la mentira a la diestra de Satanás, para que con crueldad y mentira se realicen sus obras.

\par 21 Por tanto, comprendan que el poder de la ira es vano.

\par 22 Porque primero que nada provoca con la palabra; luego, con las obras fortalece al que está enojado, y con grandes pérdidas perturba su mente, y así despierta con gran ira su alma.

\par 23 Por tanto, cuando alguno. habla contra vosotros, no os enojéis, y si alguno os alaba como a hombres santos, no os envanezcáis; no os mováis ni al deleite ni al disgusto.

\par 24 Porque, en primer lugar, agrada al oído y, por tanto, estimula la mente para percibir los motivos de la provocación; y luego, enojado, piensa que con razón está enojado.

\par 25 Hijos míos, si caéis en pérdida o ruina, no os aflijáis; porque este mismo espíritu hace al hombre desear lo perecedero, para enojarse por la aflicción.

\par 26 Y si sufréis pérdida voluntaria o involuntariamente, no os enojéis; porque de la aflicción surge la ira con la mentira.

\par 27 Además, doble daño es la ira con la mentira; y se ayudan unos a otros para perturbar el corazón; y cuando el alma está continuamente perturbada, el Señor se aparta de ella, y Beliar se enseñorea de ella.



\chapter{2}

\par \textit{Una profecía de los pecados, el cautiverio, las plagas y la restitución final de la nación. Todavía hablan del Edén (ver versículo 18). El versículo 23 es notable a la luz de la profecía.}

\par 1 OBSERVE, pues, hijos míos, los mandamientos del Señor y guarde su ley; apartaos de la ira, y odiad la mentira, para que el Señor habite entre vosotros, y Beliar huya de vosotros.

\par 2 Hablad verdad cada uno con su prójimo. Así no caeréis en ira y confusión; pero estaréis en paz, teniendo al Dios de paz, y ninguna guerra prevalecerá sobre vosotros.

\par 3 Amad al Señor durante toda vuestra vida y los unos a los otros con corazón sincero.

\par 4 Sé que en los postreros días os apartaréis del Señor, y provocaréis a ira a Leví y pelearéis contra Judá; pero no prevaleceréis contra ellos, porque un ángel del Señor los guiará a ambos; porque junto a ellos estará Israel en pie.

\par 5 Y cuando os apartéis del Señor, andaréis en todo mal y haréis las abominaciones de los gentiles, prostituyéndoos tras las mujeres de los impíos, mientras que con toda maldad los espíritus de maldad obran en vosotros.

\par 6 Porque he leído en el libro de Enoc, el justo, que tu príncipe es Satanás, y que todos los espíritus de maldad y de orgullo conspirarán para asistir constantemente a los hijos de Leví, para hacerlos pecar ante el Señor. .

\par 7 Y mis hijos se acercarán a Leví y pecarán con ellos en todo; y los hijos de Judá serán codiciosos, saqueando como leones los bienes ajenos.

\par 8 Por tanto, seréis llevados con ellos al cautiverio, y allí recibiréis todas las plagas de Egipto y todos los males de los gentiles.

\par 9 Y así, cuando volváis al Señor, obtendréis misericordia, y Él os llevará a Su santuario y os dará paz.

\par 10 Y os surgirá de la tribu de Judá y de Leví la salvación del Señor; y hará guerra contra Beliar.

\par 11 Y ejecutaremos una venganza eterna sobre nuestros enemigos; y en cautiverio tomará de Beliar las almas de los santos, y volverá los corazones desobedientes al Señor, y dará a los que lo invocan paz eterna.

\par 12 Y los santos descansarán en el Edén, y en la Nueva Jerusalén se regocijarán los justos, y será para la gloria de Dios por los siglos.

\par 13 Y Jerusalén ya no será asolada, ni Israel será llevado cautivo; porque Jehová estará en medio de ella [viviendo entre los hombres], y el Santo de Israel reinará sobre ella en humildad y en pobreza; y el que cree en Él reinará entre los hombres en verdad.

\par 14 Ahora pues, hijos míos, temed al Señor y guardaos de Satanás y de sus espíritus.

\par 15 Acercaos a Dios y al ángel que intercede por vosotros, porque él es mediador entre Dios y los hombres, y por la paz de Israel se levantará contra el reino del enemigo.

\par 16 Por eso el enemigo está ansioso por destruir a todos los que invocan al Señor.

\par 17 Porque él sabe que el día en que Israel se arrepienta, el reino del enemigo llegará a su fin.

\par 18 Porque el mismo ángel de la paz fortalecerá a Israel para que no caiga en el extremo del mal.

\par 19 Y sucederá que en el tiempo de la anarquía de Israel, el Señor no se apartará de ellos, sino que los transformará en una nación que haga su voluntad, porque ninguno de los ángeles será igual a él.

\par 20 Y su nombre estará en todo lugar de Israel y de las naciones.

\par 21 Por tanto, hijos míos, guardaos de toda obra mala, y desechad la ira y toda mentira, y amad la verdad y la paciencia.

\par 22 Y lo que habéis oído de vuestro padre, impartidlo también a vuestros hijos, para que os reciba el Salvador de los gentiles; porque él es verdadero y paciente, manso y humilde, y enseña con sus obras la ley de Dios.

\par 23 Apártate, pues, de toda injusticia y adhiérete a la justicia de Dios, y tu raza será salva para siempre.

\par 24 Y entiérrame junto a mis padres.

\par 25 Y habiendo dicho estas cosas, las besó y, ya anciano, se durmió.

\par 26 Sus hijos lo sepultaron, y después llevaron sus huesos y los pusieron junto a Abraham, Isaac y Jacob.

\par 27 Sin embargo, Dan les profetizó que se olvidarían de su Dios y serían alejados de la tierra de su herencia, del linaje de Israel y de la familia de su descendencia.



\end{document}