\begin{document}

\title{Tobías}


\chapter{1}

\par 1 El libro de las palabras de Tobit, hijo de Tobiel, hijo de Ananiel, hijo de Aduel, hijo de Gabael, de la descendencia de Asael, de la tribu de Neftalí;
\par 2 Quien en tiempos de Enemessar, rey de los asirios, fue llevado cautivo desde Tisbe, que está a la derecha de aquella ciudad que propiamente se llama Neftalí en Galilea, cerca de Aser.
\par 3 Yo, Tobit, he andado todos los días de mi vida por el camino de la verdad y de la justicia, y he hecho muchas limosnas a mis hermanos y a mi nación, que vino conmigo a Nínive, a la tierra de los asirios.
\par 4 Y cuando yo era joven en mi tierra, en la tierra de Israel, toda la tribu de mi padre Neftalí cayó de la casa de Jerusalén, la cual era escogida entre todas las tribus de Israel, de modo que todas las tribus debían sacrificar allí, donde fue consagrado y construido el templo de la morada del Altísimo para todos los siglos.
\par 5 Todas las tribus que se rebelaron juntas y la casa de mi padre Neftalí sacrificaron a la novilla Baal.
\par 6 Pero yo solo iba muchas veces a Jerusalén a las fiestas, tal como estaba establecido por decreto eterno para todo el pueblo de Israel, recibiendo las primicias y los décimos de lo crecido con lo primero esquilado; y los entregué en el altar a los sacerdotes hijos de Aarón.
\par 7 La primera décima parte de todos los ingresos la di a los hijos de Aarón que servían en Jerusalén; la otra décima parte la vendí y fui a gastarla cada año en Jerusalén.
\par 8 Y la tercera parte se la di a quienes convenía, tal como me había ordenado Débora, la madre de mi padre, porque mi padre me había dejado huérfana.
\par 9 Además, cuando llegué a la edad adulta, me casé con Ana, de mi familia, y de ella engendré a Tobías.
\par 10 Y cuando fuimos llevados cautivos a Nínive, todos mis hermanos y mis parientes comieron del pan de los gentiles.
\par 11 Pero yo me abstuve de comer;
\par 12 Porque me acordé de Dios con todo mi corazón.
\par 13 Y el Altísimo me dio gracia y favor ante Enemessar, de modo que fui su proveedor.
\par 14 Y fui a Media y dejé en custodia a Gabael, hermano de Gabrias, en Ragés, ciudad de Media, diez talentos de plata.
\par 15 Muerto Enemessar, reinó en su lugar Senaquerib su hijo; cuyo patrimonio estaba en problemas, que no pude entrar en Media.
\par 16 Y en tiempos de Enemessar di muchas limosnas a mis hermanos y mi pan a los hambrientos,
\par 17 Y mis vestidos a los desnudos; y si veía a alguno de mi nación muerto o arrojado entre los muros de Nínive, lo sepultaba.
\par 18 Y si el rey Senaquerib había matado a alguno cuando llegó y huyó de Judea, los sepulté en secreto; porque en su ira mató a muchos; pero los cuerpos no fueron encontrados cuando fueron buscados por el rey.
\par 19 Y cuando uno de los ninivitas fue y se quejó de mí ante el rey, los enterré y me escondí; comprendiendo que me buscaban para matarme, me retiré por miedo.
\par 20 Entonces me quitaron todos mis bienes y no quedó nada de mí, aparte de mi esposa Ana y mi hijo Tobías.
\par 21 No habían pasado cincuenta y cinco días antes de que dos de sus hijos lo mataran y huyeran a las montañas de Ararat; y su hijo Sarchedonus reinó en su lugar; quien puso sobre las cuentas de su padre y sobre todos sus negocios a Aquiácaro, hijo de mi hermano Anael.
\par 22 Y Aquiácaro, rogando por mí, regresé a Nínive. Aquiácaro era copero, guardián del sello, mayordomo y mayordomo de las cuentas; y Sarquedono lo nombró junto a él, y era hijo de mi hermano.

\chapter{2}

\par 1 Cuando volví a casa y me fue devuelta mi esposa Ana y mi hijo Tobías, en la fiesta de Pentecostés, que es la fiesta santa de las siete semanas, me prepararon una buena cena, en el cual me senté a comer.
\par 2 Y cuando vi abundancia de comida, dije a mi hijo: Ve y trae a cualquier hombre pobre que encuentres entre nuestros hermanos, que se acuerde del Señor; y he aquí, me quedo por ti.
\par 3 Pero él volvió y dijo: Padre, uno de nuestra nación ha sido estrangulado y arrojado en la plaza.
\par 4 Entonces, antes de haber probado algo de carne, me levanté y lo llevé a una habitación hasta que se puso el sol.
\par 5 Entonces volví, me lavé y comí con pesadez.
\par 6 Acordándonos de la profecía de Amós, que dijo: Vuestras fiestas se convertirán en luto, y toda vuestra alegría en lamentación.
\par 7 Por eso lloré; y cuando se puso el sol, fui, hice un sepulcro y lo sepulté.
\par 8 Pero mis vecinos se burlaban de mí y decían: Este hombre que huyó no teme todavía ser ejecutado por este asunto; y, sin embargo, he aquí que vuelve a enterrar a los muertos.
\par 9 Esa misma noche volví del sepulcro y dormí junto al muro de mi patio, contaminado y con el rostro descubierto.
\par 10 Y no sabía que había gorriones en la pared, y cuando mis ojos estaban abiertos, los gorriones me echaron estiércol tibio en los ojos y se me pusieron blancos los ojos; y fui a los médicos, pero no me ayudaron. : además Achiacharus me alimentó, hasta que entré a Elymais.
\par 11 Y mi esposa Ana tomó trabajos de mujeres para hacer.
\par 12 Y cuando los envió a casa con los dueños, ellos le pagaron el salario y le dieron además un cabrito.
\par 13 Y cuando estaba en mi casa y se puso a llorar, le dije: ¿De dónde es este cabrito? ¿no es robado? entregárselo a sus propietarios; porque no es lícito comer cosa robada.
\par 14 Pero ella me respondió: Fue dado como regalo más que el salario. Pero yo no la creí, sino que le pedí que se lo diera a sus dueños, y quedé avergonzado ante ella. Pero ella me respondió: ¿Dónde están tus limosnas y tus buenas obras? he aquí, tú y todas tus obras son conocidas.

\chapter{3}

\par 1 Entonces yo, afligido, lloré y oré en mi dolor, diciendo:
\par 2 Oh Señor, tú eres justo, y todas tus obras y todos tus caminos son misericordia y verdad, y juzgas verdadera y justamente para siempre.
\par 3 Acuérdate de mí, mírame y no me castigues por mis pecados e ignorancias, ni por los pecados de mis padres, que pecaron antes que ti.
\par 4 Porque no obedecieron tus mandamientos; por eso nos entregaste por botín, al cautiverio y a la muerte, y por proverbio de oprobio para todas las naciones entre las cuales estamos dispersos.
\par 5 Ahora bien, tus juicios son muchos y verdaderos: trátame según mis pecados y los de mis padres, porque no hemos guardado tus mandamientos ni hemos andado en verdad delante de ti.
\par 6 Ahora, pues, trata conmigo como mejor te parezca y ordena que me quiten el espíritu, para que me disuelva y me convierta en tierra; porque me es más provechoso morir que vivir, porque tengo He oído falsas afrentas y he tenido mucha tristeza. Manda, pues, que ahora sea librado de esta angustia, y vaya al lugar eterno; no apartes de mí tu rostro.
\par 7 Aconteció aquel mismo día que en Ecbatane, ciudad de Media, Sara, hija de Ragüel, también fue vituperada por las criadas de su padre;
\par 8 Porque ella había estado casada con siete maridos, a quienes el espíritu maligno Asmodeo había matado antes de acostarse con ella. ¿No sabes, dijeron, que has estrangulado a tus maridos? Ya has tenido siete maridos, y ninguno de ellos te ha puesto nombre.
\par 9 ¿Por qué nos golpeas por ellos? si están muertos, ve tras ellos, y nunca más te veamos ni hijo ni hija.
\par 10 Al oír estas cosas, se entristeció mucho, y pensó que se había estrangulado; y ella dijo: Soy la única hija de mi padre, y si hago esto, le será afrenta, y llevaré su vejez con dolor al sepulcro.
\par 11 Entonces oró hacia la ventana y dijo: Bendito eres, Señor, Dios mío, y tu santo y glorioso nombre es bendito y honorable por los siglos: todas tus obras te alaban por los siglos.
\par 12 Y ahora, oh Señor, pongo mis ojos y mi rostro hacia ti,
\par 13 Y di: Sácame de la tierra, para que no oiga más afrentas.
\par 14 Tú sabes, Señor, que estoy limpio de todo pecado con el hombre,
\par 15 Y que nunca he profanado mi nombre ni el de mi padre en la tierra de mi cautiverio: soy la única hija de mi padre, y él no tiene ningún hijo que sea su heredero, ni ningún pariente cercano, ni cualquier hijo suyo vivo, a quien pueda reservarme por esposa: mis siete maridos ya están muertos; ¿Y por qué debería vivir? pero si no te place que muera, ordena que se me tenga en cuenta y que se compadezca de mí, para que no escuche más reproches.
\par 16 Y las oraciones de ambos fueron escuchadas ante la majestad del gran Dios.
\par 17 Y Rafael fue enviado para curarlos a ambos, es decir, para quitar la blancura de los ojos de Tobit, y para dar a Sara, hija de Ragüel, por esposa a Tobías, hijo de Tobit; y para atar a Asmodeo el espíritu maligno; porque ella era de Tobías por derecho de herencia. En aquel mismo tiempo llegó Tobit a su casa, y entró en su casa, y Sara, hija de Ragüel, bajó de su aposento alto.

\chapter{4}

\par 1 Aquel día Tobit se acordó del dinero que había confiado a Gabael en Ragues de Media,
\par 2 Y dijo para sí: He deseado la muerte; ¿Por qué no llamo a mi hijo Tobías para darle el dinero antes de morir?
\par 3 Y llamándolo, le dijo: Hijo mío, cuando esté muerto, entiérrame; y no menosprecies a tu madre, sino hónrala todos los días de tu vida, y haz lo que le agrade, y no la entristezca.
\par 4 Recuerda, hijo mío, que ella vio muchos peligros para ti cuando estabas en su vientre; y cuando muera, entiérrala junto a mí en una misma tumba.
\par 5 Hijo mío, recuerda al Señor nuestro Dios todos tus días, y no permitas que tu voluntad peque ni transgreda sus mandamientos; obra con rectitud durante toda tu vida y no sigas los caminos de la injusticia.
\par 6 Porque si actúas con sinceridad, tus obras prosperarán para ti y para todos los que viven con justicia.
\par 7 Da limosna de tus bienes; y cuando des limosna, no tengas envidia, ni apartes tu rostro de ningún pobre, y el rostro de Dios no se apartará de ti.
\par 8 Si tienes mucho, da limosna en consecuencia; si tienes poco, no temas dar según lo poco.
\par 9 Porque te haces un buen tesoro para el día de necesidad.
\par 10 Porque el que da limosna libra de la muerte y no sufre la oscuridad.
\par 11 Porque la limosna es buena dádiva para todo aquel que la da, ante los ojos del Altísimo.
\par 12 Guárdate de toda fornicación, hijo mío, y toma principalmente mujer del linaje de tus padres, y no tomes por mujer a mujer extraña, que no sea de la tribu de tu padre; porque somos hijos de los profetas, Noé. , Abraham, Isaac y Jacob: acuérdate, hijo mío, que nuestros padres desde el principio, que todos tomaron mujeres de su propia sangre, y fueron benditos en sus hijos, y su descendencia heredará la tierra.
\par 13 Ahora pues, hijo mío, ama a tus hermanos y no menosprecies en tu corazón a tus hermanos, hijos e hijas de tu pueblo, al no tomar esposa de ellos; porque en la soberbia hay destrucción, en mucha angustia y en la lascivia. Hay decadencia y gran necesidad: porque la lascivia es la madre del hambre.
\par 14 No dejes que el salario de nadie que haya trabajado para ti se quede contigo, sino dáselo de inmediato; porque si sirves a Dios, él también te lo pagará. Sé prudente, hijo mío, en todo lo que hagas. lo haces, y sé prudente en toda tu conversación.
\par 15 No hagas a nadie lo que aborreces: no bebas vino para embriagarte, ni dejes que la embriaguez te acompañe en tu camino.
\par 16 Da de tu pan al hambriento, y de tus vestidos al desnudo; y según tu abundancia, da limosna; y no tenga envidia tu ojo cuando das limosna.
\par 17 Derrama tu pan en el entierro de los justos, pero no des nada a los impíos.
\par 18 Pide consejo a todos los sabios, y no desprecies ningún consejo que sea provechoso.
\par 19 Bendice siempre al Señor tu Dios y desea que tus caminos sean enderezados y que todas tus sendas y tus consejos sean prosperados; porque ninguna nación tiene consejo; pero el Señor mismo da todos los bienes, y humilla a quien quiere, como quiere; Ahora pues, hijo mío, acuérdate de mis mandamientos, y no los quites de tu mente.
\par 20 Y ahora les digo esto que encomendé diez talentos a Gabael hijo de Gabrias en Ragés de Media.
\par 21 Y no temas, hijo mío, que seamos empobrecidos; porque tendrás muchas riquezas, si temes a Dios, te apartas de todo pecado y haces lo que agrada a sus ojos.

\chapter{5}

\par 1 Entonces Tobías respondió y dijo: Padre, haré todo lo que me has mandado.
\par 2 Pero ¿cómo puedo recibir el dinero si no lo conozco?
\par 3 Entonces le dio la carta y le dijo: Busca un hombre que pueda ir contigo mientras yo viva, y le daré un salario; y ve y recibe el dinero.
\par 4 Entonces, cuando fue a buscar a un hombre, encontró a Rafael, que era un ángel.
\par 5 Pero él no lo sabía; y él le dijo: ¿Puedes ir conmigo a Rages? ¿Y conoces bien esos lugares?
\par 6 A quien el ángel dijo: Iré contigo, y conozco bien el camino, porque me he alojado en casa de nuestro hermano Gabael.
\par 7 Entonces Tobías le dijo: Quédate conmigo hasta que se lo diga a mi padre.
\par 8 Entonces le dijo: Ve y no te demores. Entonces entró y dijo a su padre: He aquí, he encontrado uno que irá conmigo. Entonces dijo: Llámalo, para que sepa de qué tribu es, y si es hombre de confianza para ir contigo.
\par 9 Entonces lo llamó, entró y se saludaron.
\par 10 Entonces Tobit le dijo: Hermano, muéstrame de qué tribu y familia eres.
\par 11 A quien dijo: ¿Buscas una tribu, una familia o un jornalero para ir con tu hijo? Entonces Tobit le dijo: Quiero saber, hermano, tu parentela y tu nombre.
\par 12 Entonces dijo: Yo soy Azarías, el hijo de Ananías el grande, y de tus hermanos.
\par 13 Entonces Tobit dijo: De nada, hermano; No te enojes ahora conmigo, porque he investigado para conocer tu tribu y tu familia; porque tú eres mi hermano, de honesta y buena estirpe; porque conozco a Ananías y a Jonathas, hijos de aquel gran Samaías, cuando íbamos juntos a Jerusalén a adorar, y ofrecíamos los primogénitos, y las décimas de los frutos; y no se dejaron seducir por el error de nuestros hermanos: hermano mío, eres de buena estirpe.
\par 14 Pero dime, ¿qué salario te daré? ¿Quieres una dracma al día y lo necesario para mi propio hijo?
\par 15 Además, si volvéis sanos y salvos, añadiré algo a vuestro salario.
\par 16 Y quedaron muy contentos. Entonces dijo a Tobías: Prepárate para el viaje, y que Dios te envíe un buen viaje. Y cuando su hijo hubo preparado todas las cosas para el viaje, su padre dijo: Ve tú con este hombre, y Dios que habita en los cielos prosperará tu viaje, y el ángel de Dios te hará compañía. Entonces salieron ambos, y el perro del joven con ellos.
\par 17 Pero Ana, su madre, lloró y dijo a Tobit: ¿Por qué has despedido a nuestro hijo? ¿No es él el bastón de nuestra mano para entrar y salir delante de nosotros?
\par 18 No seáis codiciosos en añadir dinero a dinero, sino que sea como basura para nuestro hijo.
\par 19 Porque nos basta lo que el Señor nos ha dado para vivir.
\par 20 Entonces Tobit le dijo: No te preocupes, hermana mía; Él volverá sano y salvo, y tus ojos lo verán.
\par 21 Porque el ángel bueno le hará compañía, su viaje será próspero y regresará sano y salvo.
\par 22 Entonces ella dejó de llorar.

\chapter{6}

\par 1 Y mientras seguían su camino, al anochecer llegaron al río Tigris y se alojaron allí.
\par 2 Y cuando el joven bajó a lavarse, un pez saltó del río y quiso devorarlo.
\par 3 Entonces el ángel le dijo: Toma el pez. Y el joven agarró el pez y lo sacó a tierra.
\par 4 A quien el ángel dijo: Abre el pescado, toma el corazón, el hígado y la hiel, y ponlos en lugar seguro.
\par 5 Entonces el joven hizo lo que el ángel le ordenó; y cuando hubieron asado el pescado, lo comieron. Luego ambos siguieron su camino, hasta llegar cerca de Ecbatane.
\par 6 Entonces el joven dijo al ángel: Hermano Azarías, ¿para qué sirven el corazón, el hígado y la gallina del pez?
\par 7 Y él le dijo: En cuanto al corazón y al hígado, si un demonio o un espíritu maligno molesta a alguien, debemos fumarlo delante del hombre o de la mujer, y la parte no se enojará más.
\par 8 En cuanto a la hiel, bueno es ungir al hombre que tiene blancura en los ojos, y sanará.
\par 9 Y cuando llegaron cerca de Ragès,
\par 10 El ángel dijo al joven: Hermano, hoy nos alojaremos en casa de Ragüel, que es tu primo; también tiene una hija única, llamada Sara; Hablaré por ella, para que te sea dada por esposa.
\par 11 Porque a ti te corresponde el derecho de ella, ya que tú sólo eres de su parentela.
\par 12 Y la doncella es hermosa y sabia. Ahora pues, escúchame y hablaré con su padre; y cuando volvamos de Rages celebraremos las bodas: porque sé que Ragüel no puede casarla con otro según la ley de Moisés, sino que será culpable de muerte, porque el derecho de herencia te corresponde más a ti que a cualquier otro. otro.
\par 13 Entonces el joven respondió al ángel: He oído, hermano Azarías, que esta doncella ha sido entregada a siete hombres, los cuales murieron todos en la cámara nupcial.
\par 14 Y ahora soy el único hijo de mi padre, y temo que, si entro a ella, muero como el otro antes, porque la ama un espíritu maligno que no daña al cuerpo, excepto a los que ven a ella; Por lo cual también temo que muera y lleve con dolor a la tumba la vida de mi padre y de mi madre, porque no tienen otro hijo que los sepulte.
\par 15 Entonces el ángel le dijo: ¿No te acuerdas de los preceptos que te dio tu padre de que te casaras con una mujer de tu propio parentesco? Por tanto, escúchame, oh hermano mío; porque ella te será dada por mujer; y no tengas en cuenta al espíritu maligno; porque esta misma noche te será entregada en matrimonio.
\par 16 Y cuando entres en la cámara nupcial, tomarás las cenizas del perfume, pondrás sobre ellas parte del corazón y del hígado del pescado y harás con ello un humo.
\par 17 Y el diablo lo olerá, huirá y nunca más volverá; pero cuando llegues a ella, levántate ambos y ora a Dios, que es misericordioso y tendrá compasión de vosotros, y te salvará: no temas, porque ella te está señalada desde el principio; y la guardarás, y ella irá contigo. Además supongo que ella te dará hijos. Cuando Tobías oyó estas cosas, la amó, y su corazón se unió efectivamente a ella.

\chapter{7}

\par 1 Cuando llegaron a Ecbatane, llegaron a la casa de Ragüel, y Sara les salió al encuentro; y después de saludarse, los introdujo en la casa.
\par 2 Entonces Ragüel dijo a Edna su mujer: ¡Qué parecido es este joven con Tobit, mi primo!
\par 3 Y Ragüel les preguntó: ¿De dónde sois, hermanos? A quienes dijeron: Nosotros somos de los hijos de Neftalí, que están cautivos en Nínive.
\par 4 Entonces les dijo: ¿Conocéis a Tobit, nuestro pariente? Y ellos dijeron: Nosotros le conocemos. Entonces dijo: ¿Se encuentra bien de salud?
\par 5 Y ellos dijeron: Él está vivo y goza de buena salud. Y Tobías dijo: Él es mi padre.
\par 6 Entonces Ragüel se levantó de un salto, lo besó y lloró.
\par 7 Y lo bendijo y le dijo: Tú eres hijo de un hombre honesto y bueno. Pero cuando oyó que Tobit era ciego, se entristeció y lloró.
\par 8 Y también lloraron Edna su esposa y Sara su hija. Además, los agasajaron alegremente; y después de matar un carnero del rebaño, pusieron provisiones de carne en la mesa. Entonces dijo Tobías a Rafael: Hermano Azarías, habla de las cosas que hablaste en el camino, y que se despache este asunto.
\par 9 Entonces le comunicó el asunto a Ragüel, y Ragüel dijo a Tobías: Come, bebe y haz fiesta.
\par 10 Porque es justo que te cases con mi hija; sin embargo, te declararé la verdad.
\par 11 He dado a mi hija en matrimonio a los siete hombres que murieron la noche que entraron a ella; sin embargo, estad alegres por ahora. Pero Tobías dijo: No comeré nada aquí, hasta que estemos de acuerdo y juremos el uno por el otro.
\par 12 Ragüel dijo: Entonces tómala de ahora en adelante según la costumbre, porque tú eres su prima y ella es tuya, y el Dios misericordioso te dé éxito en todo.
\par 13 Entonces llamó a su hija Sara, y ella vino a su padre, y él la tomó de la mano y se la dio por mujer a Tobías, diciéndole: He aquí, tómala según la ley de Moisés y llévala. a tu padre. Y los bendijo;
\par 14 Y llamó a Edna su mujer, tomó papel, escribió un instrumento de pactos y lo selló.
\par 15 Entonces comenzaron a comer.
\par 16 Después Ragüel llamó a su mujer Edna y le dijo: Hermana, prepara otra cámara y tráela allí.
\par 17 Ella hizo lo que él le había ordenado y la llevó allí, y lloró, recibió las lágrimas de su hija y le dijo:
\par 18 Ten consuelo, hija mía; El Señor del cielo y de la tierra te dé alegría en este dolor tuyo: ten buen consuelo, hija mía.

\chapter{8}

\par 1 Y después de cenar, trajeron a Tobías a ella.
\par 2 Y mientras iba, se acordó de las palabras de Rafael, tomó las cenizas de los perfumes, puso encima el corazón y el hígado del pescado y ahumó con ello.
\par 3 Cuando el espíritu maligno lo olió, huyó a los confines de Egipto, y el ángel lo ató.
\par 4 Y después que estuvieron ambos encerrados juntos, Tobías se levantó de la cama y dijo: Hermana, levántate y oremos para que Dios tenga piedad de nosotros.
\par 5 Entonces Tobías comenzó a decir: Bendito eres, oh Dios de nuestros padres, y bendito es tu santo y glorioso nombre por los siglos; que los cielos te bendigan a ti y a todas tus criaturas.
\par 6 Tú creaste a Adán y le diste a Eva su esposa como ayuda y apoyo. De ellos surgió la humanidad; dijiste: No es bueno que el hombre esté solo; hagamosle una ayuda semejante a él.
\par 7 Ahora bien, Señor, no tomo a esta mi hermana por lujuriosa sino con rectitud: ordena, pues, misericordiosamente que podamos envejecer juntas.
\par 8 Y ella dijo con él: Amén.
\par 9 Así que durmieron ambos esa noche. Y Ragüel se levantó y fue e hizo un sepulcro,
\par 10 Diciendo: Temo que él también esté muerto.
\par 11 Pero cuando Ragüel entró en su casa,
\par 12 Dijo a su esposa Edna. Envía una de las criadas y que vea si está vivo; si no, lo sepultaremos sin que nadie lo sepa.
\par 13 Entonces la criada abrió la puerta, entró y los encontró a ambos dormidos.
\par 14 Y salió y les dijo que estaba vivo.
\par 15 Entonces Ragüel alabó a Dios y dijo: Oh Dios, digno eres de ser alabado con toda alabanza pura y santa; Por tanto, que tus santos te alaben con todas tus criaturas; y todos tus ángeles y tus escogidos te alaben por siempre.
\par 16 Tú eres digna de alabanza, porque me has alegrado; y no me ha sucedido lo que sospechaba; pero tú nos has tratado según tu gran misericordia.
\par 17 Alabado seas porque has tenido misericordia de dos hijos unigénitos de sus padres: concédeles misericordia, oh Señor, y termina su vida con salud, con alegría y misericordia.
\par 18 Entonces Ragüel ordenó a sus sirvientes que llenaran la tumba.
\par 19 Y celebró la fiesta de bodas durante catorce días.
\par 20 Porque antes de que se cumplieran los días de las bodas, Ragüel le había dicho mediante juramento que no partiría hasta que hubieran transcurrido los catorce días de las bodas;
\par 21 Entonces tomará la mitad de sus bienes y regresará sano y salvo a su padre. y deberíamos tener el resto cuando yo y mi esposa estemos muertos.

\chapter{9}

\par 1 Entonces Tobías llamó a Rafael y le dijo:
\par 2 Hermano Azarías, toma contigo un siervo y dos camellos, y ve a Rages de Media, a Gabael, y tráeme el dinero y tráelo a la boda.
\par 3 Porque Ragüel ha jurado que no me iré.
\par 4 Pero mi padre cuenta los días; y si me demoro mucho, lo lamentará mucho.
\par 5 Entonces Rafael salió y se alojó en casa de Gabael, y le dio la escritura, quien sacó bolsas cerradas y se las entregó.
\par 6 Y muy de mañana salieron ambos juntos y llegaron a la boda; y Tobías bendijo a su mujer.

\chapter{10}

\par 1 Tobit, su padre, contaba todos los días; y cuando se acababan los días del viaje y no llegaban,
\par 2 Entonces Tobit dijo: ¿Están detenidos? ¿O está muerto Gabael y no hay hombre que le dé el dinero?
\par 3 Por eso se arrepintió mucho.
\par 4 Entonces su mujer le dijo: Mi hijo ha muerto, ya que se queda mucho tiempo; y ella comenzó a llorarle, y dijo:
\par 5 Ahora nada me importa, hijo mío, desde que te dejé ir, la luz de mis ojos.
\par 6 A quien Tobit dijo: Calla, no te preocupes, porque está a salvo.
\par 7 Pero ella dijo: Calla y no me engañes; mi hijo está muerto. Y ella salía todos los días por el camino que ellos iban, y no comía carne durante el día, y no cesaba de llorar noches enteras a su hijo Tobías, hasta que se cumplieron los catorce días de las bodas que Ragüel había jurado que celebraría. pasar ahí. Entonces Tobías dijo a Ragüel: Déjame ir, porque mi padre y mi madre ya no buscan verme.
\par 8 Pero su suegro le dijo: Quédate conmigo, y enviaré a tu padre y le contarán cómo te va.
\par 9 Pero Tobías dijo: No; pero déjame ir con mi padre.
\par 10 Entonces Ragüel se levantó y le dio a Sara su mujer y la mitad de sus bienes, sirvientes, ganado y dinero.
\par 11 Y los bendijo y los despidió, diciendo: El Dios del cielo os dé un buen viaje, hijos míos.
\par 12 Y dijo a su hija: Honra a tu padre y a tu suegra, que ahora son tus padres, para que pueda oír buenas noticias de ti. Y él la besó. Edna también dijo a Tobías: El Señor del cielo te restaure, mi querido hermano, y concédeme ver a tus hijos de mi hija Sara antes de morir, para que me regocije delante del Señor: he aquí, te encomiendo a mi hija de confianza especial; ¿Dónde están? No le implores el mal.

\chapter{11}

\par 1 Después de estas cosas, Tobías se fue, alabando a Dios por haberle dado un viaje próspero, y bendijo a Ragüel y a Edna su esposa, y siguió su camino hasta llegar cerca de Nínive.
\par 2 Entonces Rafael dijo a Tobías: Tú sabes, hermano, cómo dejaste a tu padre.
\par 3 Apresurémonos delante de tu mujer y preparemos la casa.
\par 4 Y toma en tu mano la hiel del pescado. Entonces ellos siguieron su camino y el perro los persiguió.
\par 5 Ahora Anna estaba sentada mirando hacia el camino de su hijo.
\par 6 Y cuando ella lo vio venir, dijo a su padre: He aquí, viene tu hijo y el hombre que iba con él.
\par 7 Entonces dijo Rafael: Sé, Tobías, que tu padre abrirá los ojos.
\par 8 Unge, pues, sus ojos con hiel y, pinchado con ella, se frotará y se desvanecerá su blancura, y te verá.
\par 9 Entonces Ana salió corriendo, se echó sobre el cuello de su hijo y le dijo: «Ya que te he visto, hijo mío, desde ahora me contentaré con morir». Y ambos lloraron.
\par 10 Tobit también salió hacia la puerta y tropezó; pero su hijo corrió hacia él,
\par 11 Y agarró a su padre y le dio un golpe de hiel en los ojos, diciendo: Ten buena esperanza, padre mío.
\par 12 Y cuando le empezaron a escocer los ojos, se los frotó;
\par 13 Y la blancura desapareció de las comisuras de sus ojos; y cuando vio a su hijo, cayó sobre su cuello.
\par 14 Y lloró y dijo: Bendito eres, oh Dios, y bendito es tu nombre por los siglos; y benditos sean todos tus santos ángeles:
\par 15 Porque me azotaste y te apiadaste de mí; he aquí, veo a mi hijo Tobías. Y su hijo fue gozoso y contó a su padre las grandes cosas que le habían sucedido en Media.
\par 16 Entonces Tobit salió al encuentro de su nuera a las puertas de Nínive, gozoso y alabando a Dios; y los que lo vieron irse se maravillaron, porque había recuperado la vista.
\par 17 Pero Tobías dio gracias delante de ellos porque Dios tenía misericordia de él. Y acercándose a Sara su nuera, la bendijo, diciendo: Bienvenida, hija; bendito sea el Dios que te trajo a nosotros, y benditos sean tu padre y tu madre. Y hubo alegría entre todos sus hermanos que estaban en Nínive.
\par 18 Y vinieron Aquiácaro y Nasbas, el hijo de su hermano.
\par 19 Y las bodas de Tobías se celebraron con gran alegría durante siete días.

\chapter{12}

\par 1 Entonces Tobit llamó a su hijo Tobías y le dijo: Hijo mío, cuida que el hombre que fue contigo tenga su salario, y tendrás que darle más.
\par 2 Y Tobías le dijo: Padre, no me hace daño darle la mitad de lo que he traído.
\par 3 Porque él me hizo volver sano y salvo a ti, sanó a mi mujer, me trajo el dinero y también te sanó a ti.
\par 4 Entonces el anciano dijo: A él le corresponde.
\par 5 Entonces llamó al ángel y le dijo: Toma la mitad de todo lo que has traído y vete sano y salvo.
\par 6 Entonces los tomó aparte a ambos y les dijo: Bendecid a Dios, alabadle, engrandecedlo y alabadle por las cosas que os ha hecho delante de todos los vivientes. Bueno es alabar a Dios, y exaltar su nombre, y mostrar con honor las obras de Dios; Por tanto, no os tardéis en alabarle.
\par 7 Es bueno guardar el secreto de un rey, pero es honor revelar las obras de Dios. Haz lo bueno y ningún mal te tocará.
\par 8 Buena es la oración con el ayuno, la limosna y la justicia. Mejor es un poco con justicia que mucho con injusticia. Es mejor dar limosna que acumular oro:
\par 9 Porque la limosna libra de la muerte y limpia todo pecado. Los que practican limosna y justicia serán llenos de vida:
\par 10 Pero los que pecan son enemigos de su propia vida.
\par 11 Seguramente no te ocultaré nada. Porque dije: Es bueno guardar el secreto de un rey, pero es honor revelar las obras de Dios.
\par 12 Ahora, pues, cuando oraste tú y Sara tu nuera, yo llevé el recuerdo de tus oraciones delante del Santo; y cuando enterraste a los muertos, yo también estuve contigo.
\par 13 Y cuando no tardaste en levantarte y dejar tu comida para ir a cubrir a los muertos, tu buena acción no me fue ocultada, sino que yo estaba contigo.
\par 14 Y ahora Dios me ha enviado a curarte a ti y a Sara tu nuera.
\par 15 Yo soy Rafael, uno de los siete santos ángeles que presentan las oraciones de los santos y que entran y salen delante de la gloria del Santo.
\par 16 Entonces ambos se turbaron y cayeron sobre sus rostros, porque tenían miedo.
\par 17 Pero él les dijo: No temáis, porque os irá bien; Alabado sea, pues, Dios.
\par 18 Porque no he venido por ningún favor mío, sino por la voluntad de nuestro Dios; Por tanto, alabadle por siempre.
\par 19 Todos estos días me aparecí a vosotros; pero no comí ni bebí, pero vosotros visteis una visión.
\par 20 Ahora pues, dad gracias a Dios, porque subo al que me envió; pero escribe en un libro todas las cosas que se hacen.
\par 21 Y cuando se levantaron, no lo vieron más.
\par 22 Entonces confesaron las grandes y maravillosas obras de Dios y cómo se les había aparecido el ángel del Señor.

\chapter{13}

\par 1 Entonces Tobit escribió una oración de regocijo y dijo: Bendito sea el Dios que vive por los siglos, y bendito sea su reino.
\par 2 Porque él azota y tiene misericordia; desciende al infierno y hace subir; no hay quien pueda escapar de su mano.
\par 3 Confesadlo delante de los gentiles, hijos de Israel, porque él nos ha esparcido entre ellos.
\par 4 Allí proclamad su grandeza y ensalzadle delante de todos los vivientes, porque él es nuestro Señor y Dios nuestro Padre para siempre.
\par 5 Y él nos azotará por nuestras iniquidades, y nuevamente tendrá misericordia y nos reunirá de entre todas las naciones entre las cuales nos dispersó.
\par 6 Si os volvéis a él con todo vuestro corazón y con toda vuestra mente, y actuáis con rectitud delante de él, entonces él se volverá hacia vosotros y no ocultará de vosotros su rostro. Mira, pues, lo que hará contigo, y confiésalo con toda tu boca, y alaba al Señor del poder, y ensalza al Rey eterno. En la tierra de mi cautiverio lo alabo y declaro su poder y majestad a una nación pecadora. Oh pecadores, convertíos y haced justicia delante de él: ¿quién puede decir si os aceptará y tendrá misericordia de vosotros?
\par 7 Ensalzaré a mi Dios, y mi alma alabará al Rey del cielo y se regocijará en su grandeza.
\par 8 Que todos hablen y que todos lo alaben por su justicia.
\par 9 Oh Jerusalén, ciudad santa, él te azotará por las obras de tus hijos, y volverá a tener misericordia de los hijos de los justos.
\par 10 Alaba al Señor, porque es bueno, y alaba al Rey eterno, para que con alegría se edifique en ti su tabernáculo, y que allí regocije en ti a los cautivos, y ame en ti por siempre aquellos que son miserables.
\par 11 Muchas naciones vendrán desde lejos al nombre del Señor Dios con presentes en las manos, presentes para el Rey del cielo; todas las generaciones te alabarán con gran alegría.
\par 12 Malditos todos los que te aborrecen, y bienaventurados todos los que te aman por siempre.
\par 13 Alegraos y alegraos por los hijos de los justos, porque se reunirán y bendecirán al Señor de los justos.
\par 14 Bienaventurados los que te aman, porque se alegrarán en tu paz; bienaventurados los que se entristecen por todos tus azotes; porque se alegrarán por ti, cuando hayan visto toda tu gloria, y se alegrarán por siempre.
\par 15 Que mi alma bendiga a Dios, el gran Rey.
\par 16 Porque Jerusalén será edificada con zafiros, esmeraldas y piedras preciosas; tus muros, tus torres y tus almenas, con oro puro.
\par 17 Y las calles de Jerusalén serán pavimentadas con berilo, carbunclo y piedras de Ofir.
\par 18 Y todas sus calles dirán: Aleluya; y lo alabarán, diciendo: Bendito sea el Dios que lo exaltó por los siglos.

\chapter{14}

\par 1 Entonces Tobit terminó de alabar a Dios.
\par 2 Tenía cincuenta y ocho años cuando perdió la vista, que le recuperó al cabo de ocho años; y daba limosna, y crecía en el temor del Señor Dios, y lo alababa.
\par 3 Y siendo ya muy anciano, llamó a su hijo y a los hijos de su hijo, y le dijo: Hijo mío, toma a tus hijos; porque he aquí, ya soy viejo y estoy a punto de partir de esta vida.
\par 4 Ve a Media, hijo mío, porque creo firmemente en lo que el profeta Jonás habló de Nínive: que será destruida; y que por un tiempo habrá paz en Media; y que nuestros hermanos yacerán esparcidos en la tierra desde aquella buena tierra; y Jerusalén será desolada, y la casa de Dios que está en ella será quemada, y será desolada por un tiempo;
\par 5 Y que Dios volverá a tener misericordia de ellos y los traerá de nuevo a la tierra donde construirán un templo, pero no como el primero, hasta que se cumpla el tiempo de esa era; y después volverán de todos los lugares de su cautiverio, y edificarán a Jerusalén gloriosamente, y la casa de Dios será edificada en ella para siempre con un edificio glorioso, como los profetas lo han dicho.
\par 6 Y todas las naciones se volverán y temerán verdaderamente al Señor Dios y enterrarán sus ídolos.
\par 7 Así alabarán al Señor todas las naciones, y su pueblo confesará a Dios, y el Señor exaltará a su pueblo; y todos los que aman al Señor Dios en verdad y justicia se alegrarán, teniendo misericordia de nuestros hermanos.
\par 8 Y ahora, hijo mío, sal de Nínive, porque seguramente se cumplirán las cosas que habló el profeta Jonás.
\par 9 Pero tú guarda la ley y los mandamientos, y muéstrate misericordioso y justo, para que te vaya bien.
\par 10 Y entiérrame dignamente, y tu madre conmigo; pero no os detengáis más en Nínive. Acuérdate, hijo mío, de cómo Amán trató a Aquiácaro que lo crió, cómo de la luz lo llevó a las tinieblas, y cómo lo recompensó de nuevo; sin embargo, Aquiácaro se salvó, pero el otro tuvo su recompensa: porque descendió a las tinieblas. Manasés dio limosna y escapó de los lazos de muerte que le habían tendido; pero Amán cayó en el lazo y pereció.
\par 11 Ahora pues, hijo mío, considera lo que hace la limosna y cómo la justicia salva. Habiendo dicho estas cosas, se entregó en la cama, siendo de ciento cincuenta y ocho años; y lo enterró honorablemente.
\par 12 Y muerta su madre Ana, la enterró con su padre. Pero Tobías se fue con su mujer y sus hijos a Ecbatane, a casa de su suegro Ragüel,
\par 13 Donde envejeció con honor, enterró honorablemente a su padre y a su suegra y heredó sus bienes y los de su padre Tobit.
\par 14 Y murió en Ecbatane de Media, cuando tenía ciento veintisiete años.
\par 15 Pero antes de morir se enteró de la destrucción de Nínive, que fue tomada por Nabucodonosor y Asuero; y antes de morir se regocijó por Nínive.

\end{document}