\begin{document}

\title{I Esdras}


\chapter{1}

\par 1 Y Josías celebró la fiesta de la pascua en Jerusalén a su Señor, y ofreció la pascua el día catorce del primer mes;
\par 2 Habiendo dispuesto a los sacerdotes según sus turnos diarios, vestidos con vestiduras largas, en el templo del Señor.
\par 3 Y dijo a los levitas, los santos ministros de Israel, que se santificaran al Señor y pusieran el arca santa del Señor en la casa que el rey Salomón hijo de David había construido.
\par 4 Y dijo: Nunca más llevaréis el arca sobre vuestros hombros. Servid, pues, ahora al Señor vuestro Dios, y ministrad a su pueblo Israel, y preparaos según vuestras familias y parentescos.
\par 5 Según lo prescrito por David, rey de Israel, y según la magnificencia de su hijo Salomón, y estando en el templo según las diferentes dignidades de vuestras familias, los levitas, que ministran en presencia de vuestros hermanos a los hijos de Israel,
\par 6 Ofreced la pascua en orden, preparad los sacrificios para vuestros hermanos y guardad la pascua según el mandamiento del Señor que fue dado a Moisés.
\par 7 Josías dio al pueblo que allí se encontraba treinta mil corderos y cabritos y tres mil becerros; estas cosas fueron entregadas de la ración del rey, tal como había prometido, al pueblo, a los sacerdotes y a los levitas.
\par 8 Y Helquías, Zacarías y Sielo, gobernadores del templo, dieron a los sacerdotes para la Pascua dos mil seiscientas ovejas y trescientos terneros.
\par 9 Y Jeconías, Samaías, su hermano Natanael, Asabías, Oquiel y Joram, capitanes de mil, dieron a los levitas para la pascua cinco mil ovejas y setecientos terneros.
\par 10 Y cuando terminaron estas cosas, los sacerdotes y los levitas, teniendo los panes sin levadura, se pusieron de pie en muy hermoso orden según sus familias,
\par 11 Y según las diferentes dignidades de los padres, delante del pueblo, ofrecieron al Señor, como está escrito en el libro de Moisés; y así lo hicieron por la mañana.
\par 12 Y asaron la pascua al fuego como corresponde; en cuanto a los sacrificios, los medieron en ollas y sartenes de bronce con buen sabor.
\par 13 Y los dispusieron delante de todo el pueblo; y después prepararon para ellos y para los sacerdotes, sus hermanos, los hijos de Aarón.
\par 14 Porque los sacerdotes ofrecieron sebo hasta la noche; y los levitas se prepararon, y los sacerdotes a sus hermanos, los hijos de Aarón.
\par 15 También los santos cantores, los hijos de Asaf, estaban en su orden, según el nombramiento de David: Asaf, Zacarías y Jedutún, que era del séquito del rey.
\par 16 Además, había porteros en cada puerta; A nadie le era lícito apartarse de su servicio ordinario; porque sus hermanos los levitas se preparaban para ellos.
\par 17 Así se cumplieron los sacrificios del Señor aquel día, para celebrar la Pascua.
\par 18 Y ofrecerán sacrificios sobre el altar del Señor, según el mandato del rey Josías.
\par 19 Entonces los hijos de Israel que estaban allí celebraron la Pascua y la fiesta de los panes dulces durante siete días.
\par 20 Y tal pascua no se celebraba en Israel desde los tiempos del profeta Samuel.
\par 21 Ni todos los reyes de Israel celebraron la Pascua como la que celebraron Josías, los sacerdotes, los levitas y los judíos con todo Israel que se encontraba habitando en Jerusalén.
\par 22 Esta Pascua se celebró en el año dieciocho del reinado de Josías.
\par 23 Y Josías fue recto delante de su Señor, con un corazón lleno de piedad.
\par 24 En cuanto a las cosas que sucedieron en su tiempo, están escritas en tiempos pasados, acerca de los que pecaron e hicieron maldad contra el Señor más que todos los pueblos y reinos, y cómo lo entristecieron en gran manera, de modo que las palabras del Señor se levantó contra Israel.
\par 25 Después de todos estos actos de Josías, aconteció que Faraón, rey de Egipto, vino a declarar la guerra en Carchamis, junto al Éufrates, y Josías salió contra él.
\par 26 Pero el rey de Egipto envió a decirle: ¿Qué tengo yo contigo, rey de Judea?
\par 27 No soy un enviado del Señor Dios contra ti; porque mi guerra es contra el Éufrates; y ahora el Señor está conmigo, sí, el Señor está conmigo, apremiándome hacia adelante; apartaos de mí, y no estéis contra el Señor.
\par 28 Sin embargo, Josías no apartó su carro de él, sino que se puso a luchar contra él, sin hacer caso de las palabras del profeta Jeremías dichas por boca del Señor:
\par 29 Pero se enfrentaron a él en la llanura de Magiddo, y los príncipes vinieron contra el rey Josías.
\par 30 Entonces el rey dijo a sus siervos: Sacadme de la batalla; porque soy muy débil. E inmediatamente sus siervos lo sacaron de la batalla.
\par 31 Entonces subió a su segundo carro; y siendo llevado de regreso a Jerusalén murió, y fue sepultado en el sepulcro de su padre.
\par 32 Y en toda Judea hicieron duelo por Josías, y el profeta Jeremías se lamentó por Josías, y los principales hombres y las mujeres hicieron lamentación por él hasta el día de hoy; la nación de Israel.
\par 33 Estas cosas están escritas en el libro de las Historias de los reyes de Judá, y cada uno de los hechos que hizo Josías, y su gloria, y su entendimiento en la ley del Señor, y las cosas que había hecho antes, y las cosas ahora recitadas, están relatadas en el libro de los reyes de Israel y de Judea.
\par 34 Y el pueblo tomó a Joacaz hijo de Josías y lo hizo rey en lugar de Josías su padre, cuando tenía veintitrés años.
\par 35 Y reinó en Judea y en Jerusalén tres meses; y luego el rey de Egipto lo destituyó del reinado en Jerusalén.
\par 36 Y fijó un impuesto sobre la tierra de cien talentos de plata y un talento de oro.
\par 37 El rey de Egipto también nombró a su hermano Joaquín rey de Judea y de Jerusalén.
\par 38 Y ató a Joaquín y a los nobles, pero apresó a su hermano Zaraces y lo sacó de Egipto.
\par 39 Joaquín tenía veinticinco años cuando fue nombrado rey en la tierra de Judea y de Jerusalén; e hizo lo malo ante el Señor.
\par 40 Entonces subió contra él Nabucodonosor, rey de Babilonia, lo ató con una cadena de bronce y lo llevó a Babilonia.
\par 41 Nabucodonosor tomó también algunos de los objetos sagrados del Señor, se los llevó y los puso en su templo en Babilonia.
\par 42 Pero lo que se cuenta de él, y de su inmundicia e impiedad, está escrito en las crónicas de los reyes.
\par 43 Y reinó en su lugar Joaquín su hijo; cuando tenía dieciocho años fue nombrado rey;
\par 44 Y reinó en Jerusalén sólo tres meses y diez días; e hizo lo malo ante el Señor.
\par 45 Al cabo de un año, Nabucodonosor envió y mandó que lo llevaran a Babilonia con los vasos sagrados del Señor;
\par 46 E hizo a Sedequías rey de Judea y de Jerusalén cuando tenía veintiún años; y reinó once años:
\par 47 Y también hizo lo malo ante los ojos del Señor, y no hizo caso de las palabras que le habló el profeta Jeremías de boca del Señor.
\par 48 Y después que el rey Nabucodonosor le hizo jurar por el nombre del Señor, se retractó y se rebeló; y endureciendo su cerviz, su corazón, transgredió las leyes del Señor Dios de Israel.
\par 49 También los gobernadores del pueblo y los sacerdotes hicieron muchas cosas contra las leyes, cometieron todas las profanaciones de todas las naciones y profanaron el templo del Señor, que estaba santificado en Jerusalén.
\par 50 Sin embargo, el Dios de sus padres envió un mensajero para llamarlos de regreso, porque los había salvado a ellos y también a su tabernáculo.
\par 51 Pero se burlaron de sus mensajeros; y he aquí, cuando el Señor les habló, se burlaron de sus profetas:
\par 52 Hasta el punto de que, enojado con su pueblo por su gran impiedad, ordenó a los reyes de los caldeos que subieran contra ellos;
\par 53 Quienes mataron a espada a sus jóvenes, incluso dentro del recinto de su santo templo, y no perdonaron ni a joven ni a doncella, ni a anciano ni a niño, entre ellos; porque él entregó todo en sus manos.
\par 54 Y tomaron todos los objetos sagrados del Señor, grandes y pequeños, junto con los utensilios del arca de Dios y los tesoros del rey, y se los llevaron a Babilonia.
\par 55 En cuanto a la casa del Señor, la quemaron, derribaron los muros de Jerusalén y prendieron fuego a sus torres.
\par 56 Y sus glorias no cesaron hasta que las consumieron y las redujeron a la nada; y al pueblo que no fue muerto a espada lo llevó a Babilonia.
\par 57 Que fueron siervos de él y de sus hijos, hasta que reinaron los persas, para cumplir la palabra del Señor hablada por boca de Jeremías:
\par 58 Hasta que la tierra haya disfrutado de sus sábados, descansará todo el tiempo de su desolación, hasta cumplir setenta años.

\chapter{2}

\par 1 En el primer año de Ciro, rey de los persas, para que se cumpliera la palabra del Señor que había prometido por boca de Jeremías;
\par 2 El Señor levantó el espíritu de Ciro, rey de los persas, y éste hizo proclamar por todo su reino y también por escrito:
\par 3 Diciendo: Así dice Ciro, rey de los persas; El Señor de Israel, el Señor Altísimo, me ha hecho rey del mundo entero,
\par 4 Y me ordenó que le construyera una casa en Jerusalén para los judíos.
\par 5 Por tanto, si alguno de vosotros es de su pueblo, que el Señor, su Señor, esté con él y suba a Jerusalén, que está en Judea, y edifique la casa del Señor de Israel: porque él es el Señor que habita en Jerusalén.
\par 6 Entonces, los que habitan en los alrededores, que le ayuden, digo, sus vecinos, con oro y plata,
\par 7 Con regalos, caballos y ganado, y otras cosas que se han concedido por voto, para el templo del Señor en Jerusalén.
\par 8 Entonces los jefes de las familias de Judea y de la tribu de Benjamín se levantaron; también los sacerdotes, los levitas y todos aquellos a quienes el Señor había movido para subir y edificar una casa al Señor en Jerusalén,
\par 9 Y los que vivían en sus alrededores, y los ayudaron en todo con plata y oro, con caballos y ganado, y con muchísimos obsequios de un gran número de personas cuyas mentes estaban animadas a ello.
\par 10 El rey Ciro sacó también los vasos sagrados que Nabucodonosor se había llevado de Jerusalén y había colocado en su templo de los ídolos.
\par 11 Cuando Ciro, rey de los persas, los sacó, se los entregó a Mitrídates, su tesorero:
\par 12 Y por él fueron entregados a Sanabassar, gobernador de Judea.
\par 13 Y éste era el número de ellos; Mil copas de oro y mil de plata, incensarios de plata veintinueve, treinta copas de oro y dos mil cuatrocientos diez de plata, y otros mil vasos.
\par 14 Todos los utensilios de oro y de plata que se llevaron fueron cinco mil cuatrocientos sesenta y nueve.
\par 15 Estos fueron traídos por Sanabasar, junto con los cautivos, de Babilonia a Jerusalén.
\par 16 Pero en tiempos de Artejerjes, rey de los persas, Belemo, Mitrídates, Tabelio, Ratumo, Beeltetmo, y el secretario Semelio, con otros que estaban bajo su cargo, que habitaban en Samaria y en otros lugares, escribieron a él contra los que habitaban en Judea y Jerusalén estas cartas que siguen;
\par 17 Al rey Artejerjes, nuestro señor, a tus servidores, Ratumo el narrador, y Semelio el escriba, y el resto de su consejo, y los jueces que están en Celosiria y Fenicia.
\par 18 Sepa ahora el señor rey que los judíos que están arriba de ti a nosotros, al entrar en Jerusalén, ciudad rebelde y malvada, construyen plazas de mercado, reparan sus muros y ponen los cimientos del templo.
\par 19 Ahora bien, si esta ciudad y sus murallas se reconstruyen, no sólo se negarán a pagar tributos, sino que también se rebelarán contra los reyes.
\par 20 Y teniendo en cuenta ya las cosas del templo, consideramos conveniente no descuidar tal asunto.
\par 21 sino para hablar con nuestro señor el rey, para que, si te place, se busque en los libros de tus padres:
\par 22 Y encontrarás en las crónicas lo que está escrito acerca de estas cosas, y entenderás que aquella ciudad era rebelde y perturbaba tanto a reyes como a ciudades:
\par 23 Y que los judíos eran rebeldes y siempre provocaban guerras en ellos; por lo cual aun esta ciudad quedó desolada.
\par 24 Por tanto, ahora te declaramos, oh señor rey, que si esta ciudad se reconstruye y sus murallas se levantan de nuevo, de ahora en adelante no tendrás paso a Celosiria ni a Fenicia.
\par 25 Entonces el rey volvió a escribir de esta manera al escritor Rato, a Beeltetmo, al escriba Semelio y a los demás funcionarios y a los habitantes de Samaria, Siria y Fenicia;
\par 26 He leído la carta que me enviasteis; por eso ordené que se investigara con diligencia, y se descubrió que aquella ciudad desde el principio practicaba contra los reyes;
\par 27 Y los hombres que allí estaban se entregaron a la rebelión y a la guerra, y en Jerusalén había reyes poderosos y feroces, que reinaban y exigían tributos en Celosiria y Fenicia.
\par 28 Ahora, pues, he ordenado que se impida a esos hombres construir la ciudad y que se tenga cuidado de que no se haga nada más en ella;
\par 29 Y que esos malvados trabajadores no sigan adelante para enfadar a los reyes,
\par 30 Cuando el rey Artejerjes leyó sus cartas, Ratumo, el escriba Semelio y los demás que estaban bajo su cargo, partiendo apresuradamente hacia Jerusalén con una tropa de jinetes y una multitud de gente en orden de batalla, comenzaron a obstaculizar los constructores; y cesó la construcción del templo en Jerusalén hasta el año segundo del reinado de Darío rey de los persas.

\chapter{3}

\par 1 Cuando Darío reinaba, hizo un gran banquete a todos sus súbditos, a toda su casa y a todos los príncipes de Media y Persia,
\par 2 Y a todos los gobernadores, capitanes y lugartenientes que estaban bajo su mando, desde la India hasta Etiopía, de ciento veinte y siete provincias.
\par 3 Cuando hubieron comido y bebido y se habían saciado, se fueron a casa, entonces el rey Darío entró en su alcoba, se durmió y poco después despertó.
\par 4 Entonces tres jóvenes de la guardia que guardaban el cuerpo del rey hablaron entre sí;
\par 5 Que cada uno de nosotros pronuncie una sentencia: al que venza y cuya sentencia le parezca más sabia que los demás, el rey Darío le dará grandes regalos y grandes cosas en señal de victoria:
\par 6 Como vestirse de púrpura, beber oro y dormir sobre oro, y un carro con frenos de oro, un tocado de lino fino y una cadena alrededor del cuello.
\par 7 Y se sentará junto a Darío a causa de su sabiduría, y se llamará Darío su primo.
\par 8 Y cada uno escribió su sentencia, la selló y la puso debajo de la almohada del rey Darío;
\par 9 Y dijo que cuando el rey se levante, algunos le entregarán los escritos; y de cuyo bando el rey y los tres príncipes de Persia juzgarán que su sentencia es la más sabia, a él se le dará la victoria, como fue designado.
\par 10 El primero escribió: El vino es el más fuerte.
\par 11 El segundo escribió: El rey es el más fuerte.
\par 12 El tercero escribió: Las mujeres son las más fuertes, pero sobre todas las cosas la verdad lleva la victoria.
\par 13 Cuando el rey se levantó, tomaron sus escritos, se los entregaron y él los leyó:
\par 14 Y enviando llamó a todos los príncipes de Persia y de Media, a los gobernadores, a los capitanes, a los lugartenientes y a los oficiales principales;
\par 15 Y lo sentó en el tribunal real; y los escritos fueron leídos delante de ellos.
\par 16 Y él dijo: Llamad a los jóvenes, y ellos pronunciarán sus propias sentencias. Entonces fueron llamados y entraron.
\par 17 Y él les dijo: Explícanos lo que piensas acerca de los escritos. Entonces empezó el primero, que había hablado de la fuerza del vino;
\par 18 Y él dijo así: ¡Oh hombres, qué fuerte es el vino! hace errar a todos los que lo beben:
\par 19 Esto hace que el pensamiento del rey y el del huérfano sean todos uno; del siervo y del libre, del pobre y del rico:
\par 20 También convierte cada pensamiento en alegría y regocijo, de modo que el hombre no se acuerda de la tristeza ni de las deudas.
\par 21 Y enriquece todo corazón, de modo que nadie se acuerda del rey ni del gobernador; y hace hablar todas las cosas por talentos:
\par 22 Y cuando están bebidos, olvidan su amor hacia amigos y hermanos, y poco después desenvainan sus espadas.
\par 23 Pero cuando se quedan sin vino, no se acuerdan de lo que han hecho.
\par 24 Oh hombres, ¿no es el vino el más fuerte que obliga a hacer esto? Y habiendo dicho esto, calló.

\chapter{4}

\par 1 Entonces el segundo, que había hablado de la fuerza del rey, comenzó a decir:
\par 2 Oh hombres, ¿no sobresalen en fuerza los hombres que dominan el mar y la tierra y todas las cosas que hay en ellos?
\par 3 Pero el rey es aún más poderoso, porque es señor de todas estas cosas y tiene dominio sobre ellas; y hacen todo lo que él les manda.
\par 4 Si él les ordena hacer la guerra unos contra otros, lo hacen; si él los envía contra los enemigos, van y derriban murallas y torres de montañas.
\par 5 Matan y son asesinados, y no transgreden el mandamiento del rey: si obtienen la victoria, se lo entregan todo al rey, junto con el botín y todo lo demás.
\par 6 De la misma manera, aquellos que no son soldados ni participan en la guerra, sino que se dedican a la agricultura, cuando han vuelto a cosechar lo que habían sembrado, lo llevan al rey y se obligan unos a otros a pagar tributo al rey.
\par 7 Y, sin embargo, es un solo hombre: si él ordena matar, matan; si él manda perdonar, ellos perdonan;
\par 8 Si él manda herir, golpean; si él manda desolar, desolan; si él manda construir, ellos construyen;
\par 9 Si él ordena talar, talan; si él manda plantar, ellos plantan.
\par 10 Entonces todo su pueblo y sus ejércitos le obedecen; además él se acuesta, come, bebe y descansa.
\par 11 Y éstos vigilan a su alrededor, y nadie puede apartarse y ocuparse de sus asuntos, ni desobedecerle en nada.
\par 12 Oh hombres, ¿cómo no podría ser más poderoso el rey, cuando de esa manera se le obedece? Y se mordió la lengua.
\par 13 Entonces el tercero, que había hablado de las mujeres y de la verdad (este era Zorobabel), comenzó a hablar.
\par 14 Oh hombres, no es el gran rey, ni la multitud de los hombres, ni el vino lo que sobresale; ¿Quién entonces los gobierna o tiene señorío sobre ellos? ¿No son mujeres?
\par 15 Las mujeres han dado a luz al rey y a todo el pueblo que gobierna en el mar y en la tierra.
\par 16 Algunos de ellos vinieron y alimentaron a los que plantaron las viñas de donde procede el vino.
\par 17 Estos también hacen vestidos para hombres; estos traen gloria a los hombres; y sin las mujeres no pueden existir los hombres.
\par 18 Y si los hombres juntan oro y plata o cualquier otra cosa hermosa, ¿no aman a una mujer hermosa en gracia y hermosura?
\par 19 Y dejando pasar todas esas cosas, ¿no se quedan boquiabiertos y, aun con la boca abierta, fijan sus ojos en ella? ¿Y no tienen todos los hombres más deseo de ella que de la plata o del oro, o de cualquier cosa buena?
\par 20 El hombre deja a su padre, que lo crió, y a su propia patria, y se une a su mujer.
\par 21 Se compromete a no pasar su vida con su esposa y no se acuerda ni de padre, ni de madre, ni de patria.
\par 22 En esto también debéis saber que las mujeres tienen dominio sobre vosotros: ¿no trabajáis y os afanáis, y dais y traéis todo a la mujer?
\par 23 Incluso un hombre toma su espada y se va a robar y hurtar, a navegar por el mar y los ríos;
\par 24 Y mira un león y camina en la oscuridad; y cuando ha robado, despojado y despojado, lo trae a su amor.
\par 25 Por eso el hombre ama más a su mujer que al padre o a la madre.
\par 26 Y hay muchos que, perdiendo el juicio por las mujeres, se convierten en sirvientes por amor a ellas.
\par 27 También muchos perecieron, se extraviaron y pecaron por causa de las mujeres.
\par 28 ¿Y ahora no me creéis? ¿No es grande el rey en su poder? ¿No temen todas las regiones tocarlo?
\par 29 Sin embargo, lo vi a él y a Apame, concubina del rey, hija del admirable Bártaco, sentados a la derecha del rey,
\par 30 Y tomando la corona de la cabeza del rey y poniéndola sobre su propia cabeza; también golpeó al rey con su mano izquierda.
\par 31 Y, sin embargo, a pesar de todo esto, el rey se quedó boquiabierto y la miró con la boca abierta: si ella se reía de él, él también se reía; pero si ella se enfadaba con él, el rey estaba dispuesto a halagarla para que ella pudiera reconciliarse con él otra vez.
\par 32 ¡Oh hombres! ¿Cómo es posible que las mujeres no sean fuertes, si hacen esto?
\par 33 Entonces el rey y los príncipes se miraron y él comenzó a hablar la verdad.
\par 34 Oh hombres, ¿no son fuertes las mujeres? Grande es la tierra, alto el cielo, veloz el sol en su carrera, porque rodea los cielos y en un día regresa a su lugar.
\par 35 ¿No es grande el que hace estas cosas? Por tanto, grande es la verdad y más fuerte que todas las cosas.
\par 36 Toda la tierra clama por la verdad y el cielo la bendice; todas las obras se estremecen y tiemblan ante ella, y no hay nada injusto en ella.
\par 37 El vino es malo, el rey es malo, las mujeres son malas, todos los hijos de los hombres son malos y tales son todas sus malas obras; y no hay verdad en ellos; en su injusticia también perecerán.
\par 38 En cuanto a la verdad, ella permanece y es siempre fuerte; vive y vence para siempre.
\par 39 En ella no se aceptan personas ni recompensas; pero ella hace lo que es justo y se abstiene de todo lo injusto y malo; y todos los hombres gustan de sus obras.
\par 40 Ni en su juicio hay injusticia alguna; y ella es la fuerza, el reino, el poder y la majestad de todas las edades. Bendito sea el Dios de la verdad.
\par 41 Y dicho esto guardó silencio. Y entonces todo el pueblo gritó y dijo: Grande es la Verdad, y poderosa sobre todas las cosas.
\par 42 Entonces el rey le dijo: Pide lo que quieras más de lo que está escrito, y te lo daremos, porque eres considerado el más sabio; y tú te sentarás a mi lado, y te llamarán prima mía.
\par 43 Entonces dijo al rey: Acuérdate del voto que hiciste de edificar Jerusalén el día que llegaste a tu reino.
\par 44 Y para enviar todos los utensilios que fueron sacados de Jerusalén y que Ciro había apartado cuando juró destruir Babilonia, y enviarlos de nuevo allí.
\par 45 También has prometido edificar el templo que los edomitas quemaron cuando Judea fue asolada por los caldeos.
\par 46 Y ahora, oh señor rey, esto es lo que pido y lo que deseo de ti, y esta es la liberalidad principesca que procede de ti mismo: deseo, por tanto, que cumplas el voto, cuyo cumplimiento con tu propia boca has prometido al Rey del cielo.
\par 47 Entonces el rey Darío se levantó, lo besó y le escribió cartas a todos los tesoreros, tenientes, capitanes y gobernadores, para que lo llevaran sanos y salvos a él y a todos los que subían con él a construir Jerusalén.
\par 48 También escribió cartas a los lugartenientes que estaban en Celosiria y Fenicia, y a ellos en el Líbano, para que trajeran madera de cedro del Líbano a Jerusalén y que construyeran con él la ciudad.
\par 49 Además, escribió para todos los judíos que salieron de su reino hacia la judería, acerca de su libertad, que ningún oficial, ningún gobernante, ningún teniente ni tesorero debía entrar por la fuerza en sus puertas;
\par 50 y que todo el territorio que posean sea libre y sin tributos; y que los edomitas deberían entregar las aldeas de los judíos que entonces tenían:
\par 51 Sí, que se dieran veinte talentos cada año para la edificación del templo, hasta el tiempo en que fuera edificado;
\par 52 Y otros diez talentos al año, para mantener los holocaustos sobre el altar cada día, ya que tenían mandamiento de ofrecer diecisiete.
\par 53 Y que todos los que salieron de Babilonia para edificar la ciudad tuvieran libertad, así como ellos y su posteridad, y todos los sacerdotes que se fueron.
\par 54 También escribió acerca de los cargos y las vestiduras de los sacerdotes con que ministran;
\par 55 Y lo mismo para las cargas de los levitas, que se les entregarían hasta el día en que se terminara la casa y se reedificara Jerusalén.
\par 56 Y ordenó que se dieran pensiones y salarios a todos los que tuvieran la ciudad.
\par 57 También despidió de Babilonia todos los utensilios que Ciro había apartado; y todo lo que Ciro había mandado, él también mandó que se hiciera, y lo envió a Jerusalén.
\par 58 Cuando este joven salió, alzó su rostro hacia el cielo, hacia Jerusalén, y alabó al Rey del cielo,
\par 59 Y dijo: De ti viene la victoria, de ti viene la sabiduría, y tuya es la gloria, y yo soy tu siervo.
\par 60 Bendito eres tú, que me has dado sabiduría: porque a ti te doy gracias, oh Señor de nuestros padres.
\par 61 Entonces tomó las cartas, salió y vino a Babilonia, y se lo contó a todos sus hermanos.
\par 62 Y alabaron al Dios de sus padres, porque les había dado libertad y libertad.
\par 63 Para subir y edificar a Jerusalén y el templo que lleva su nombre, y festejaron con instrumentos de música y alegría durante siete días.

\chapter{5}

\par 1 Después de esto, los principales hombres de las familias fueron escogidos según sus tribus, para subir con sus mujeres, sus hijos y sus hijas, sus sirvientes y siervas y sus ganados.
\par 2 Y Darío envió con ellos mil jinetes, hasta que los trajeron sanos y salvos a Jerusalén, con tambores y flautas.
\par 3 Y todos sus hermanos jugaban, y él los hizo subir con ellos.
\par 4 Y estos son los nombres de los hombres que subieron según sus familias entre sus tribus, según sus jefes.
\par 5 Los sacerdotes, hijos de Finees hijo de Aarón: Jesús hijo de Josedec, hijo de Saraías, y Joaquín hijo de Zorobabel, hijo de Salatiel, de la casa de David, de la familia de Fares. , de la tribu de Judá;
\par 6 El cual habló sabias sentencias ante Darío, rey de Persia, en el segundo año de su reinado, en el mes de Nisán, que es el primer mes.
\par 7 Estos son los judíos que subieron del cautiverio, donde habitaron como extranjeros, a quienes Nabucodonosor, rey de Babilonia, había llevado a Babilonia.
\par 8 Y regresaron a Jerusalén y a las demás partes de los judíos, cada uno a su propia ciudad, quienes vinieron con Zorobabel, con Jesús, Nehemías, Zacarías, Reesaías, Enenio y Mardoqueo. Beelsarus, Aspharasus, Reelius, Roimus y Baana, sus guías.
\par 9 El número de los de la nación y sus gobernadores, hijos de Foros, dos mil ciento setenta y dos; los hijos de Safat, cuatrocientos setenta y dos:
\par 10 Los hijos de Ares, setecientos cincuenta y seis:
\par 11 Los hijos de Faat Moab, dos mil ochocientos doce:
\par 12 Los hijos de Elam, mil doscientos cincuenta y cuatro; los hijos de Zathul, novecientos cuarenta y cinco; los hijos de Corbe, setecientos cinco; los hijos de Bani, seiscientos cuarenta y ocho;
\par 13 Los hijos de Bebai, seiscientos veinte y tres; los hijos de Sadas, tres mil doscientos veinte y dos.
\par 14 Los hijos de Adonicam, seiscientos sesenta y siete; los hijos de Bagoi, dos mil sesenta y seis; los hijos de Adin, cuatrocientos cincuenta y cuatro;
\par 15 Los hijos de Aterezias, noventa y dos; los hijos de Ceilán y Azetas, sesenta y siete; los hijos de Azuran, cuatrocientos treinta y dos;
\par 16 Los hijos de Ananías, ciento uno; los hijos de Arom, treinta y dos; los hijos de Basa, trescientos veintitrés; los hijos de Azefurit, ciento dos.
\par 17 Los hijos de Metero, tres mil cinco; los hijos de Betlomón, ciento veintitrés;
\par 18 Los de Netofá, cincuenta y cinco; los de Anatot, ciento cincuenta y ocho; los de Betsamos, cuarenta y dos;
\par 19 Los de Quiriatia, veinticinco; los de Cafira y Beroth, setecientos cuarenta y tres; los de Pira, setecientos.
\par 20 Los de Chadías y Ammidoi, cuatrocientos veintidós; los de Cirama y Gabdes, seiscientos veinte y uno.
\par 21 Los de Macalon, ciento veintidós; los de Betolio, cincuenta y dos; los hijos de Nefis, ciento cincuenta y seis;
\par 22 Los hijos de Calamolalus y Onus, setecientos veinticinco; los hijos de Jerechus, doscientos cuarenta y cinco;
\par 23 Los hijos de Anás, tres mil trescientos treinta.
\par 24 Los sacerdotes: los hijos de Jeddu, el hijo de Jesús entre los hijos de Sanasib, novecientos setenta y dos; los hijos de Meruth, mil cincuenta y dos.
\par 25 Los hijos de Fasarón, mil cuarenta y siete; los hijos de Carme, mil diecisiete.
\par 26 Los levitas: los hijos de Jesué, Cadmiel, Banuas y Sudias, setenta y cuatro.
\par 27 Los santos cantores: los hijos de Asaf, ciento veintiocho.
\par 28 Los porteadores: los hijos de Salum, los hijos de Jatal, los hijos de Talmón, los hijos de Dacobi, los hijos de Teta, los hijos de Sami, en total ciento treinta y nueve.
\par 29 Los sirvientes del templo: los hijos de Esaú, los hijos de Asifá, los hijos de Tabaoth, los hijos de Ceras, los hijos de Sud, los hijos de Faleas, los hijos de Labana, los hijos de Graba,
\par 30 Los hijos de Acua, los hijos de Uta, los hijos de Cetab, los hijos de Agaba, los hijos de Subai, los hijos de Anan, los hijos de Cathua, los hijos de Gedur,
\par 31 Los hijos de Airus, los hijos de Daisan, los hijos de Noeba, los hijos de Chaseba, los hijos de Gazera, los hijos de Azia, los hijos de Finees, los hijos de Azare, los hijos de Bastai, los hijos de Asana, los hijos de Meani, los hijos de Nafisi, los hijos de Acub, los hijos de Acifa, los hijos de Asur, los hijos de Faracim, los hijos de Basalot,
\par 32 Los hijos de Meeda, los hijos de Coutha, los hijos de Carea, los hijos de Charcus, los hijos de Aserer, los hijos de Thomoi, los hijos de Nasith, los hijos de Atipha.
\par 33 Los hijos de los siervos de Salomón: los hijos de Azafión, los hijos de Farira, los hijos de Jeeli, los hijos de Lozón, los hijos de Israel, los hijos de Safet,
\par 34 Los hijos de Hagia, los hijos de Faracaret, los hijos de Sabi, los hijos de Sarothie, los hijos de Masías, los hijos de Gar, los hijos de Addus, los hijos de Suba, los hijos de Apherra, los hijos de Barodis, hijos de Sabat, hijos de Allom.
\par 35 Todos los ministros del templo, y los hijos de los siervos de Salomón, eran trescientos setenta y dos.
\par 36 Estos subieron de Termeleth y Thelersas, con Charaathalar al frente y Aalar;
\par 37 Tampoco pudieron declarar sus familias ni su linaje, cómo eran de Israel: los hijos de Ladán, el hijo de Ban, los hijos de Necodán, seiscientos cincuenta y dos.
\par 38 Y de los sacerdotes que usurparon el sacerdocio y no fueron encontrados: los hijos de Obdia, los hijos de Accoz, los hijos de Addus, que se casó con Augia, una de las hijas de Barzelus, y recibió el nombre de su nombre.
\par 39 Y como se buscó en el registro la descripción de la descendencia de estos hombres, y no se encontró, fueron apartados del ejercicio del sacerdocio:
\par 40 Porque Nehemías y Atarías les dijeron que no participarían de las cosas santas hasta que se levantara un sumo sacerdote revestido de doctrina y verdad.
\par 41 Los israelitas de doce años arriba eran en total cuarenta mil, sin sus siervos y siervas dos mil trescientos sesenta.
\par 42 Sus siervos y siervas, siete mil trescientos cuarenta y siete; los cantores y las cantoras, doscientos cuarenta y cinco.
\par 43 Cuatrocientos treinta y cinco camellos, siete mil treinta y seis caballos, doscientas cuarenta y cinco mulas, cinco mil quinientas veinte y cinco bestias usadas para el yugo.
\par 44 Y algunos de los jefes de sus familias, cuando llegaron al templo de Dios que está en Jerusalén, juraron volver a levantar la casa en su lugar, según sus posibilidades,
\par 45 Y dar al santo tesoro de las obras mil libras de oro, cinco mil de plata y cien vestiduras sacerdotales.
\par 46 Y así habitaban los sacerdotes, los levitas y el pueblo en Jerusalén y en el campo, también los cantores y los porteros; y todo Israel en sus aldeas.
\par 47 Pero cuando estaba cerca el mes séptimo, y cuando los hijos de Israel estaban cada uno en su lugar, todos juntos entraron de común acuerdo por la puerta abierta de la primera puerta que está hacia el oriente.
\par 48 Entonces se levantaron Jesús, hijo de Josedec, y sus hermanos los sacerdotes, y Zorobabel, hijo de Salathiel, y sus hermanos, y prepararon el altar del Dios de Israel.
\par 49 Para ofrecer sobre él holocaustos, como está expresamente ordenado en el libro de Moisés, varón de Dios.
\par 50 Y se reunieron con ellos de las otras naciones de la tierra, y erigieron el altar en su propio lugar, porque todas las naciones de la tierra estaban enemistadas con ellos y los oprimieron; y ofrecían sacrificios según el tiempo, y holocaustos al Señor por la mañana y por la tarde.
\par 51 También celebraban la fiesta de las Tiendas, como está prescrito en la ley, y ofrecían sacrificios cada día, como era debido.
\par 52 Y después de esto, las ofrendas continuas y el sacrificio de los sábados, de las lunas nuevas y de todas las fiestas santas.
\par 53 Y todos los que habían hecho algún voto a Dios comenzaron a ofrecer sacrificios a Dios desde el primer día del mes séptimo, aunque el templo del Señor aún no estaba construido.
\par 54 Y dieron a los albañiles y carpinteros dinero, comida y bebida con alegría.
\par 55 También a los de Sidón y de Tiro les dieron carros para que trajeran cedros del Líbano, que serían llevados en flotadores al puerto de Jope, tal como les había ordenado Ciro, rey de los persas.
\par 56 Y en el segundo año y segundo mes después de su llegada al templo de Dios en Jerusalén, comenzaron Zorobabel hijo de Salathiel, y Jesús hijo de Josedec, y sus hermanos, los sacerdotes, los levitas y todos ellos que vinieron a Jerusalén del cautiverio:
\par 57 Y pusieron los cimientos de la casa de Dios el primer día del segundo mes, en el segundo año de su llegada a los judíos y a Jerusalén.
\par 58 Y pusieron a los levitas de veinte años a cargo de las obras del Señor. Entonces se levantaron Jesús, y sus hijos y hermanos, y Cadmiel su hermano, y los hijos de Madiabun, con los hijos de Joda hijo de Eliadun, con sus hijos y hermanos, todos levitas, unánimes encargados del negocio, trabajando para hacer avanzar las obras en la casa de Dios. Entonces los obreros construyeron el templo del Señor.
\par 59 Y los sacerdotes estaban ataviados con sus vestiduras, con instrumentos musicales y trompetas; y los levitas hijos de Asaf tenían címbalos,
\par 60 Cantando canciones de acción de gracias y alabando al Señor, tal como lo había ordenado David, rey de Israel.
\par 61 Y cantaban a gran voz canciones de alabanza al Señor, porque su misericordia y su gloria están por siempre en todo Israel.
\par 62 Y todo el pueblo tocaba las trompetas y gritaba a gran voz, cantando cánticos de acción de gracias al Señor por la edificación de la Casa del Señor.
\par 63 También los sacerdotes, los levitas y los jefes de sus familias, los ancianos que habían visto la casa anterior, llegaron a la construcción de ésta llorando y con gran llanto.
\par 64 Pero muchos, con trompetas y con alegría, gritaban a gran voz:
\par 65 De modo que las trompetas no se oían a causa del llanto del pueblo, pero la multitud sonaba tan maravillosamente, que se oía desde lejos.
\par 66 Por eso, cuando los enemigos de la tribu de Judá y de Benjamín lo oyeron, supieron lo que significaba aquel sonido de trompetas.
\par 67 Y se dieron cuenta de que los cautivos habían edificado el templo al Señor Dios de Israel.
\par 68 Entonces fueron donde Zorobabel, Jesús y los jefes de las familias, y les dijeron: Edificaremos juntamente con vosotros.
\par 69 Porque nosotros también obedecemos a vuestro Señor y le ofrecemos sacrificios desde los días de Azbazaret, el rey de los asirios, que nos trajo aquí.
\par 70 Entonces Zorobabel, Jesús y los jefes de las familias de Israel les dijeron: No nos corresponde a nosotros y a vosotros edificar juntos una casa para el Señor nuestro Dios.
\par 71 Nosotros solos edificaremos para el Señor de Israel, tal como nos ordenó Ciro, rey de los persas.
\par 72 Pero los paganos de la tierra, que pesaban sobre los habitantes de Judea y los mantenían en apuros, obstaculizaron su construcción;
\par 73 Y con sus conspiraciones secretas, con persuasiones y conmociones populares, obstaculizaron la terminación de la construcción durante todo el tiempo que vivió el rey Ciro; así se les impidió construir durante dos años, hasta el reinado de Darío.

\chapter{6}

\par 1 En el año segundo del reinado de Darío Aggeo y Zacarías hijo de Addo, los profetas profetizaron a los judíos en Judería y en Jerusalén en el nombre del Señor Dios de Israel, que estaba sobre ellos.
\par 2 Entonces se levantaron Zorobabel hijo de Salatiel y Jesús hijo de Josedec, y comenzaron a edificar la casa del Señor en Jerusalén, estando con ellos los profetas del Señor y ayudándolos.
\par 3 En aquel momento vino a ellos Sisinnes, gobernador de Siria y de Fenicia, con Satrabuzanes y sus compañeros, y les dijo:
\par 4 ¿Por mandato de quién construís esta casa y este techo, y hacéis todas las demás cosas? ¿Y quiénes son los obreros que hacen estas cosas?
\par 5 Sin embargo, los ancianos de los judíos obtuvieron favor, porque el Señor había visitado a los cautivos;
\par 6 Y no se les impidió construir hasta que Darío le informó acerca de ellos y recibió una respuesta.
\par 7 La copia de las cartas que Sisinnes, gobernador de Siria y Fenicia, y Sathrabuzanes, con sus compañeros, gobernantes de Siria y Fenicia, escribieron y enviaron a Darío; Al rey Darío, saludo:
\par 8 Sepa todo a nuestro señor el rey, que habiendo entrado en la tierra de Judea y entrando en la ciudad de Jerusalén, encontramos en la ciudad de Jerusalén a los ancianos de los judíos que estaban en cautiverio.
\par 9 Edificar una casa para el Señor, grande y nueva, de piedras labradas y costosas, y con la madera ya colocada en las paredes.
\par 10 Y esas obras se hacen con gran rapidez, y la obra avanza prósperamente en sus manos, y se hace con toda gloria y diligencia.
\par 11 Entonces preguntamos a los ancianos, diciendo: ¿Por mandato de quién edificáis esta casa y echáis los cimientos de estas obras?
\par 12 Por eso, para poder darte conocimiento por escrito, les preguntamos quiénes eran los principales hacedores, y les solicitamos por escrito los nombres de sus principales hombres.
\par 13 Entonces nos respondieron: Nosotros somos los siervos del Señor, que hizo los cielos y la tierra.
\par 14 Y esta casa fue construida hace muchos años por un rey grande y fuerte de Israel, y fue terminada.
\par 15 Pero cuando nuestros padres provocaron a ira a Dios y pecaron contra el Señor de Israel que está en los cielos, él los entregó en poder de Nabucodonosor, rey de Babilonia de los caldeos;
\par 16 Los que derribaron la casa, la quemaron y llevaron cautivos al pueblo a Babilonia.
\par 17 Pero el primer año que Ciro reinó sobre la tierra de Babilonia, el rey Ciro escribió que se edificara esta casa.
\par 18 Y los vasos sagrados de oro y de plata que Nabucodonosor había sacado de la casa de Jerusalén y los había puesto en su templo, los que el rey Ciro sacó del templo de Babilonia, y fueron entregado a Zorobabel y al gobernante Sanabassarus,
\par 19 Con orden de llevarse esos mismos vasos y ponerlos en el templo de Jerusalén; y que el templo del Señor se construyera en su lugar.
\par 20 Entonces Sanabasarus vino acá y puso los cimientos de la casa del Señor en Jerusalén; y desde entonces hasta que este todavía es un edificio, aún no está terminado del todo.
\par 21 Ahora pues, si al rey le parece bien, busquemos en los registros del rey Ciro:
\par 22 Y si se descubre que la construcción de la casa del Señor en Jerusalén se ha hecho con el consentimiento del rey Ciro, y si nuestro señor el rey así lo desea, que nos lo comunique.
\par 23 Entonces ordenó al rey Darío que buscara entre los registros en Babilonia; y en Ecbatane, el palacio que está en el país de Media, se encontró un rollo en el que estaban escritas estas cosas.
\par 24 En el primer año del reinado de Ciro, el rey Ciro ordenó que se reconstruyera la casa del Señor en Jerusalén, donde se hacían sacrificios con fuego continuo.
\par 25 cuya altura será de sesenta codos y sesenta codos de ancho, con tres hileras de piedras labradas y una hilera de madera nueva de aquella tierra; y sus gastos serán pagados de la casa del rey Ciro:
\par 26 Y que los objetos sagrados de la casa del Señor, tanto de oro como de plata, que Nabucodonosor sacó de la casa de Jerusalén y trajo a Babilonia, fueran devueltos a la casa de Jerusalén y colocados en el lugar donde estaban antes.
\par 27 Y también ordenó que Sisinnes, el gobernador de Siria y Fenicia, y Satrabuzanes y sus compañeros, y los que fueron nombrados gobernantes en Siria y Fenicia, tuvieran cuidado de no inmiscuirse en el lugar, sino permitir que Zorobabel, el sirviente, del Señor, y gobernador de Judea, y los ancianos de los judíos, para edificar la casa del Señor en aquel lugar.
\par 28 También he ordenado que se reedifique; y que busquen diligentemente ayudar a los que están en cautiverio de los judíos, hasta que la casa del Señor esté terminada:
\par 29 Y del tributo de Celosiria y de Fenice, una parte cuidadosamente para dar a estos hombres para los sacrificios del Señor, es decir, al gobernador Zorobabel, para becerros, carneros y corderos;
\par 30 Y también trigo, sal, vino y aceite, y esto continuamente cada año sin más preguntas, según lo indiquen los sacerdotes que están en Jerusalén, que se gastan cada día.
\par 31 Para que se hagan ofrendas al Dios Altísimo por el rey y por sus hijos, y que oren por sus vidas.
\par 32 Y ordenó que a cualquiera que transgrediera o ignorara cualquier cosa antes dicha o escrita, se le quitara un árbol de su propia casa, se le colgaría en él y se le confiscarían todos sus bienes para el rey.
\par 33 Por tanto, el Señor, cuyo nombre allí se invoca, destruirá por completo a todo rey y nación que extienda su mano para obstaculizar o dañar la casa del Señor en Jerusalén.
\par 34 Yo, el rey Darío, he ordenado que se hagan estas cosas con diligencia.

\chapter{7}

\par 1 Entonces Sisinnes, gobernador de Celosiria y Fenicia, y Satrabuzanes, con sus compañeros, siguiendo las órdenes del rey Darío,
\par 2 Supervisó con mucho cuidado las obras santas, ayudando a los ancianos de los judíos y a los gobernadores del templo.
\par 3 Y así prosperaron las obras santas cuando los profetas Aggeo y Zacarías profetizaron.
\par 4 Y terminaron estas cosas por orden del Señor Dios de Israel, y con el consentimiento de Ciro, Darío y Artejerjes, reyes de Persia.
\par 5 Y así quedó terminada la Casa Santa el día veintitrés del mes de Adar, en el año sexto de Darío, rey de los persas.
\par 6 Y los hijos de Israel, los sacerdotes, los levitas y los demás cautivos que se les añadieron, hicieron conforme a las cosas escritas en el libro de Moisés.
\par 7 Y para la dedicación del templo del Señor ofrecieron cien becerros, doscientos carneros y cuatrocientos corderos;
\par 8 Y doce machos cabríos por el pecado de todo Israel, según el número de los jefes de las tribus de Israel.
\par 9 También los sacerdotes y los levitas estaban vestidos con sus vestiduras según sus familias, al servicio del Señor Dios de Israel, según el libro de Moisés, y los porteros en cada puerta.
\par 10 Y los hijos de Israel que estaban en cautiverio celebraron la Pascua el día catorce del primer mes, después de que los sacerdotes y los levitas fueron santificados.
\par 11 No todos los cautivos fueron santificados juntos, pero los levitas todos fueron santificados juntos.
\par 12 Y ofrecieron la pascua por todos los cautivos, por sus hermanos los sacerdotes y por ellos mismos.
\par 13 Y comieron los hijos de Israel que habían salido del cautiverio, todos los que se habían apartado de las abominaciones del pueblo de la tierra y habían buscado al Señor.
\par 14 Y celebraron la fiesta de los panes sin levadura durante siete días, regocijándose delante del Señor,
\par 15 Porque había vuelto hacia ellos el consejo del rey de Asiria, para fortalecerlos en las obras del Señor Dios de Israel.

\chapter{8}

\par 1 Después de estas cosas, cuando reinaba Artejerjes, rey de los persas, vino Esdras hijo de Saraías, hijo de Ezerías, hijo de Helquías, hijo de Salum,
\par 2 Hijo de Saduc, hijo de Achitob, hijo de Amarias, hijo de Ezias, hijo de Meremot, hijo de Zaraías, hijo de Savias, hijo de Bocas, hijo de Abisum, hijo de Finees, hijo de Eleazar, hijo de Aarón el sumo sacerdote.
\par 3 Este Esdras subió de Babilonia como escriba, muy conocedor de la ley de Moisés, dada por el Dios de Israel.
\par 4 Y el rey le honró, pues halló gracia ante sus ojos en todas sus peticiones.
\par 5 También subieron con él a Jerusalén algunos de los hijos de Israel, los sacerdotes de los levitas, los cantores santos, los porteros y los ministros del templo,
\par 6 En el año séptimo del reinado de Artejerjes, en el mes quinto, este era el séptimo año del rey; porque salieron de Babilonia el primer día del mes primero, y llegaron a Jerusalén, conforme al próspero viaje que les había dado el Señor.
\par 7 Porque Esdras era muy hábil, de modo que no omitió nada de la ley ni de los mandamientos del Señor, sino que enseñó a todo Israel las ordenanzas y los decretos.
\par 8 Ahora bien, la copia del encargo que fue escrita por el rey Artejerjes y que llegó a Esdras, sacerdote y lector de la ley del Señor, es la siguiente:
\par 9 El rey Artejerjes envía saludos a Esdras, sacerdote y lector de la ley del Señor:
\par 10 Habiendo decidido obrar con gracia, he dado orden de que aquellos de la nación de los judíos, y de los sacerdotes y levitas que están dentro de nuestro reino, que estén dispuestos y deseosos, vayan contigo a Jerusalén.
\par 11 Por tanto, todos los que quieran hacerlo, que se vayan contigo, como nos ha parecido bien a mí y a mis siete amigos los consejeros;
\par 12 Para que se ocupen de los asuntos de Judea y de Jerusalén conforme a lo que está en la ley del Señor;
\par 13 Y llevaréis a Jerusalén los regalos que yo y mis amigos hemos prometido para el Señor de Israel, y todo el oro y la plata que se pueda encontrar en el país de Babilonia, al Señor en Jerusalén.
\par 14 También con lo que el pueblo da para el templo del Señor su Dios en Jerusalén, y para recaudar plata y oro para los becerros, carneros y corderos, y sus pertenencias;
\par 15 Para que ofrezcan sacrificios al Señor sobre el altar del Señor su Dios, que está en Jerusalén.
\par 16 Y todo lo que tú y tus hermanos hagáis con la plata y el oro, hacedlo conforme a la voluntad de vuestro Dios.
\par 17 Y los vasos sagrados del Señor que te han sido dados para el uso del templo de tu Dios que está en Jerusalén, los pondrás delante de tu Dios en Jerusalén.
\par 18 Y cualquier otra cosa que recuerdes para el uso del templo de tu Dios, la darás del tesoro del rey.
\par 19 Y yo, el rey Artejerjes, también he ordenado a los guardas de los tesoros en Siria y Fenicia que todo lo que el sacerdote Esdras y lector de la ley del Dios Altísimo envíe a pedir, se lo entreguen rápidamente.
\par 20 Cien talentos de plata, cien cors de trigo, cien piezas de vino y otras cosas en abundancia.
\par 21 Que todo se cumpla conforme a la ley de Dios, diligentemente, para el Dios Altísimo, para que la ira no caiga sobre el reino del rey y de sus hijos.
\par 22 Os mando también que no exigáis ningún impuesto ni ninguna otra imposición a ninguno de los sacerdotes, ni a los levitas, ni a los cantores santos, ni a los porteros, ni a los ministros del templo, ni a ninguno de los que trabajan en este templo. , y que ningún hombre tiene autoridad para imponerles cosa alguna.
\par 23 Y tú, Esdras, según la sabiduría de Dios, establece jueces y magistrados, para que juzguen en toda Siria y Fenicia a todos los que conocen la ley de tu Dios; y a los que no lo saben les enseñarás.
\par 24 Y cualquiera que transgreda la ley de tu Dios y del rey, será castigado diligentemente, ya sea con la muerte u otro castigo, con dinero o con prisión.
\par 25 Entonces dijo el escriba Esdras: Bendito sea el único Señor, Dios de mis padres, que ha puesto estas cosas en el corazón del rey para glorificar su casa que está en Jerusalén.
\par 26 Y me ha honrado ante los ojos del rey, de sus consejeros y de todos sus amigos y nobles.
\par 27 Por eso, con la ayuda del Señor mi Dios, me animé y reuní a los hombres de Israel para que subieran conmigo.
\par 28 Estos son los principales, según sus familias y sus diferentes dignidades, que subieron conmigo desde Babilonia durante el reinado del rey Artejerjes:
\par 29 De los hijos de Finees, Gersón; de los hijos de Itamar, Gamael; de los hijos de David, Leto, hijo de Sequenias;
\par 30 De los hijos de Farez, Zacarías; y con él fueron contados ciento cincuenta hombres:
\par 31 De los hijos de Pahat Moab, Eliaonias hijo de Zaraías, y con él doscientos hombres:
\par 32 De los hijos de Zathoe, Sequenias hijo de Jezelus, y con él trescientos hombres; de los hijos de Adin, Obeth hijo de Jonatán, y con él doscientos cincuenta hombres.
\par 33 De los hijos de Elam, Josías hijo de Gotholías, y con él setenta hombres:
\par 34 De los hijos de Safatías, Zaraías hijo de Miguel, y con él sesenta y diez hombres:
\par 35 De los hijos de Joab, Abadías hijo de Jezelus, y con él doscientos doce hombres:
\par 36 De los hijos de Banid, Asalimot hijo de Josafías, y con él ciento sesenta hombres:
\par 37 De los hijos de Babi, Zacarías hijo de Bebai, y con él veintiocho hombres:
\par 38 De los hijos de Astat, Juan hijo de Acatán, y con él ciento diez hombres:
\par 39 De los últimos hijos de Adonicam, estos son sus nombres: Eliphalet, Jewel y Samaías, y con ellos setenta hombres:
\par 40 De los hijos de Bago, Uthi hijo de Istalcurus, y con él setenta hombres.
\par 41 Y los reuní en el río llamado Theras, donde plantamos nuestras tiendas durante tres días, y luego los inspeccioné.
\par 42 Pero cuando no encontré allí a ninguno de los sacerdotes ni de los levitas,
\par 43 Entonces envié a Eleazar, a Iduel y a Masman,
\par 44 Y Alnatán, Mamaías, Joribas, Natán, Eunatán, Zacarías y Mosollamón, hombres principales y eruditos.
\par 45 Y les ordené que fueran a ver al capitán Saddeo, que estaba en el lugar del tesoro:
\par 46 Y les ordenó que hablaran con Papá, y con sus hermanos, y con los tesoreros de aquel lugar, para que nos enviaran hombres que pudieran ejercer el oficio sacerdotal en la casa del Señor.
\par 47 Y por mano poderosa de nuestro Señor trajeron hasta nosotros hombres hábiles de los hijos de Moli hijo de Leví, hijo de Israel, Asebebia, y sus hijos y sus hermanos, que eran dieciocho.
\par 48 Y Asebia, Annus y Osaías su hermano, de los hijos de Channuneus, y sus hijos, eran veinte hombres.
\par 49 Y de los sirvientes del templo que David había ordenado, y los principales hombres para el servicio de los levitas, es decir, los sirvientes del templo, doscientos veinte, cuyos nombres se mostraban en el catálogo.
\par 50 Y allí hice voto de ayuno a los jóvenes delante de nuestro Señor, para desearle un viaje próspero tanto para nosotros como para los que estaban con nosotros, para nuestros hijos y para el ganado.
\par 51 Porque me daba vergüenza pedir al rey soldados de a pie, de a caballo y conducta para defendernos de nuestros enemigos.
\par 52 Porque habíamos dicho al rey que el poder del Señor nuestro Dios debería estar con los que lo buscan, para apoyarlos en todo.
\par 53 Y de nuevo rogamos a nuestro Señor acerca de estas cosas, y lo encontramos favorable a nosotros.
\par 54 Entonces separé a doce de los principales sacerdotes, Esebrias y Asanias, y a diez hombres de sus hermanos con ellos.
\par 55 Y les pesé el oro, la plata y los objetos sagrados de la casa de nuestro Señor, que el rey, su consejo, los príncipes y todo Israel habían dado.
\par 56 Y después de pesarlo, les entregué seiscientos cincuenta talentos de plata, y vasos de plata de cien talentos, y cien talentos de oro,
\par 57 Y veinte vasos de oro y doce vasos de bronce, de bronce fino, relucientes como el oro.
\par 58 Y les dije: Vosotros sois santos para el Señor, y los vasos son santos, y el oro y la plata son un voto para el Señor, el Señor de nuestros padres.
\par 59 Velad y guardadlos hasta que los entregéis a los jefes de los sacerdotes y a los levitas y a los principales de las familias de Israel en Jerusalén, en las cámaras de la casa de nuestro Dios.
\par 60 Entonces los sacerdotes y los levitas que habían recibido la plata, el oro y los utensilios, los llevaron a Jerusalén, al templo del Señor.
\par 61 Y partimos del río Theras el día doce del mes primero y llegamos a Jerusalén por la mano poderosa de nuestro Señor, que estaba con nosotros; y desde el comienzo de nuestro viaje el Señor nos libró de todo enemigo, y así llegamos a Jerusalén.
\par 62 Y cuando llevábamos allí tres días, el oro y la plata pesados ​​fueron entregados al cuarto día en la casa de nuestro Señor al sacerdote Marmoth hijo de Iri.
\par 63 Y con él estaba Eleazar hijo de Finees, y con ellos estaban Josabad hijo de Jesús y Moeth hijo de Sabban, levitas; todos fueron entregados por número y peso.
\par 64 Y en aquella misma hora fue registrado todo su peso.
\par 65 Además, los que habían salido del cautiverio ofrecieron sacrificios al Señor Dios de Israel: doce becerros por todo Israel, ochenta y dieciséis carneros,
\par 66 Sesenta y doce corderos y machos cabríos para la ofrenda de paz, doce; todos ellos un sacrificio al Señor.
\par 67 Y entregaron las órdenes del rey a los mayordomos del rey y a los gobernadores de Celosiria y Fenicia; y honraron al pueblo y al templo de Dios.
\par 68 Una vez hechas estas cosas, los gobernantes vinieron a mí y me dijeron:
\par 69 La nación de Israel, los príncipes, los sacerdotes y los levitas, no expulsaron de sí a los extranjeros de la tierra, ni a las profanaciones de los gentiles, es decir, de los cananeos, los hititas, los fereseos, los jebuseos y los Moabitas, egipcios y edomitas.
\par 70 Porque tanto ellos como sus hijos se han casado con sus hijas, y la simiente santa se ha mezclado con los extranjeros de la tierra; y desde el principio de este asunto los gobernantes y los grandes hombres han sido partícipes de esta iniquidad.
\par 71 Y cuando oí estas cosas, rasgué mis vestidos y el manto sagrado, me arranqué el pelo de la cabeza y la barba, y me senté triste y muy pesado.
\par 72 Entonces se reunieron conmigo todos los que entonces estaban conmovidos por la palabra del Señor, Dios de Israel, mientras yo lloraba por mi iniquidad; pero me quedé sentado, lleno de tristeza, hasta el sacrificio de la tarde.
\par 73 Luego, levantándome del ayuno, con mis vestidos y el manto sagrado rasgados, doblando las rodillas y extendiendo las manos al Señor,
\par 74 Dije: Señor, estoy confundido y avergonzado delante de ti;
\par 75 Porque nuestros pecados se multiplican sobre nuestras cabezas, y nuestra ignorancia ha llegado hasta el cielo.
\par 76 Porque desde los tiempos de nuestros padres hemos estado y estamos en gran pecado hasta el día de hoy.
\par 77 Y por nuestros pecados y los de nuestros padres, nosotros, nuestros hermanos, nuestros reyes y nuestros sacerdotes fuimos entregados a los reyes de la tierra, a la espada y a la cautividad, y a presa vergonzosa, hasta el día de hoy.
\par 78 Y ahora, oh Señor, en cierta medida nos has mostrado misericordia de parte de ti, de que nos quede raíz y nombre en el lugar de tu santuario;
\par 79 Y para descubrirnos luz en la casa del Señor nuestro Dios, y darnos alimento en el tiempo de nuestra servidumbre.
\par 80 Sí, cuando estábamos en servidumbre, nuestro Señor no nos abandonó; pero nos hizo agraciados ante los reyes de Persia, de modo que nos dieron de comer;
\par 81 y honraron el templo de nuestro Señor y levantaron la desolada Sión, para que nos hayan dado una morada segura en la judería y en Jerusalén.
\par 82 Y ahora, Señor, ¿qué diremos teniendo estas cosas? porque hemos transgredido tus mandamientos que diste por mano de tus siervos los profetas, diciendo:
\par 83 Que la tierra a la que entráis para poseerla en herencia es una tierra contaminada por las impurezas de los extranjeros, y la han llenado con sus impurezas.
\par 84 Por tanto, ahora no uniréis vuestras hijas a sus hijos, ni tomaréis sus hijas a vuestros hijos.
\par 85 Además, nunca buscaréis tener paz con ellos, para ser fuertes y comer las cosas buenas de la tierra, y dejar la herencia de la tierra a vuestros hijos para siempre.
\par 86 Y todo lo que nos ha sucedido nos ha sido causado por nuestras malas obras y nuestros grandes pecados; porque tú, oh Señor, aligeraste nuestros pecados,
\par 87 Y tú nos diste tal raíz, pero nosotros volvemos atrás para transgredir tu ley y mezclarnos con la inmundicia de las naciones de la tierra.
\par 88 ¿No podrías enojarte contra nosotros y destruirnos hasta no dejarnos ni raíz, ni semilla, ni nombre?
\par 89 Tú, Señor de Israel, eres veraz, porque hoy nos quedamos como raíz.
\par 90 He aquí, ahora estamos ante ti en nuestras iniquidades, porque a causa de estas cosas no podemos permanecer más delante de ti.
\par 91 Y mientras Esdras, en su oración, hacía su confesión, llorando y tendido en el suelo delante del templo, se reunió con él desde Jerusalén una gran multitud de hombres, mujeres y niños; porque había un gran llanto entre la multitud.
\par 92 Entonces Jeconías, hijo de Jeelus, uno de los hijos de Israel, gritó y dijo: Oh Esdras, hemos pecado contra el Señor Dios, nos hemos casado con mujeres extrañas de las naciones de la tierra, y ahora todo es así. Israel en lo alto.
\par 93 Juremos al Señor que repudiaremos a todas nuestras mujeres que hayamos tomado de los paganos, con sus hijos,
\par 94 Como tú has decretado y todos los que obedecen la ley del Señor.
\par 95 Levántate y ponte en ejecución, porque a ti te corresponde este asunto, y nosotros estaremos contigo: hazlo con valentía.
\par 96 Entonces Esdras se levantó y juró a los principales sacerdotes y levitas de todo Israel hacer lo siguiente; y por eso juraron.

\chapter{9}

\par 1 Entonces Esdras, saliendo del atrio del templo, fue a la cámara de Joanán hijo de Eliasib,
\par 2 Y permanecieron allí, sin comer carne ni beber agua, lamentándose por las grandes iniquidades de la multitud.
\par 3 Y se hizo un pregón en toda Judería y en Jerusalén a todos los que estaban en cautiverio, que debían reunirse en Jerusalén:
\par 4 Y a cualquiera que no se reuniera allí dentro de dos o tres días, según lo designado por los ancianos que gobernaban, se le tomaría su ganado para el uso del templo, y él mismo sería expulsado de entre los cautivos.
\par 5 Y en tres días se reunieron todos los de la tribu de Judá y de Benjamín en Jerusalén, el día veinte del noveno mes.
\par 6 Y toda la multitud estaba sentada temblando en el amplio atrio del templo a causa del mal tiempo que se presentaba.
\par 7 Entonces Esdras se levantó y les dijo: Habéis transgredido la ley al casaros con mujeres extranjeras, para aumentar así los pecados de Israel.
\par 8 Y ahora, confesando, dad gloria al Señor Dios de nuestros padres,
\par 9 Y haced su voluntad y apartaos de las naciones de la tierra y de las mujeres extrañas.
\par 10 Entonces toda la multitud gritó y dijo en alta voz: Tal como tú has dicho, así haremos.
\par 11 Pero como la gente es tan numerosa y el tiempo es tan malo que no podemos quedarnos afuera, y esto no es un trabajo de uno o dos días, nuestro pecado en estas cosas se ha extendido mucho.
\par 12 Por tanto, que se queden los jefes de la multitud, y que todos los de nuestras casas que tienen esposas extranjeras vengan a la hora señalada.
\par 13 Y con ellos los gobernantes y jueces de cada lugar, hasta que apartemos de nosotros la ira del Señor por este asunto.
\par 14 Entonces Jonatán, hijo de Azael, y Ezequías, hijo de Teocano, se encargaron de este asunto; y Mosollam, Levis y Sabbatheus los ayudaron.
\par 15 Y los que estaban en cautiverio hicieron conforme a todas estas cosas.
\par 16 Y el sacerdote Esdras escogió para sí a los principales hombres de sus familias, todos por nombre; y el primer día del mes décimo se sentaron juntos para examinar el asunto.
\par 17 Así, el primer día del primer mes, su causa, que tenía esposas extrañas, terminó.
\par 18 Y de los sacerdotes que se habían reunido y tenían mujeres extrañas, se encontró:
\par 19 De los hijos de Jesús, hijo de Josedec, y de sus hermanos; Matelas, Eleazar, Joribus y Joadano.
\par 20 Y dieron sus manos para repudiar a sus mujeres y ofrecer carneros para reconciliarse por sus errores.
\par 21 Y de los hijos de Emmer; Ananías, Zabdeo, Eanes, Sameio, Hiereel y Azarías.
\par 22 Y de los hijos de Faisur; Elionas, Masías Israel, Natanael, Ocidelo y Talsas.
\par 23 Y de los levitas; Jozabad, Semis, Colius, llamado Calitas, Patheus, Judas y Jonás.
\par 24 De los santos cantores; Eleazuro, Baco.
\par 25 De los porteadores; Sallumus y Tolbanes.
\par 26 De los de Israel, de los hijos de Foros; Hiermas, Eddías, Melquías, Maelo, Eleazar, Asibias y Baanias.
\par 27 De los hijos de Ela; Matanías, Zacarías, Hierielus, Hieremoth y Aedias.
\par 28 Y de los hijos de Zamot; Eliadas, Elisimus, Othonias, Jarimoth, Sabatus y Sardeus.
\par 29 De los hijos de Babai; Juan, Ananías, Josabad y Amatheis.
\par 30 De los hijos de Mani; Olamus, Mamuchus, Jedeus, Jasubus, Jasael y Hieremoth.
\par 31 Y de los hijos de Addi; Naathus, Moosias, Lacunus, Naidus, Mathanias, Sesthel, Balnuus y Manasseas.
\par 32 Y de los hijos de Anás; Elionas, Aseas, Melquías, Sabbeus y Simón Cosameo.
\par 33 Y de los hijos de Asom; Altaneo, Matías, Baanaia, Elifelet, Manasés y Semei.
\par 34 Y de los hijos de Maani; Jeremías, Momdis, Omaerus, Juel, Mabdai, Pelias, Anos, Carabasion, Enasibus, Mamnitanaimus, Eliasis, Bannus, Eliali, Samis, Selemias, Nathanias; y de los hijos de Ozora; Sesis, Esril, Azaelus, Samatus, Zambis, Josefo.
\par 35 Y de los hijos de Etma; Mazitias, Zabadaias, Edes, Juel, Banaias.
\par 36 Todos estos habían tomado mujeres extrañas y las repudiaron con sus hijos.
\par 37 Y los sacerdotes, los levitas y los israelitas habitaban en Jerusalén y en el campo el primer día del mes séptimo; y los hijos de Israel estaban en sus habitaciones.
\par 38 Y toda la multitud se juntó unánimemente en la plaza amplia del pórtico santo, hacia el oriente.
\par 39 Y dijeron a Esdras, sacerdote y lector, que traería la ley de Moisés, que fue dada por el Señor Dios de Israel.
\par 40 Entonces Esdras, el sumo sacerdote, llevó la ley a toda la multitud, desde el hombre hasta la mujer, y a todos los sacerdotes, para oír la ley el primer día del mes séptimo.
\par 41 Y leyó en el amplio atrio, delante del pórtico santo, desde la mañana hasta el mediodía, delante de hombres y mujeres; y la multitud escuchaba la ley.
\par 42 Y Esdras, sacerdote y lector de la ley, estaba de pie sobre un púlpito de madera hecho para ello.
\par 43 Y junto a él estaban Matatías, Sammo, Ananías, Azarías, Urías, Ezequías y Balasamo, a la derecha:
\par 44 Y a su izquierda estaban Faldayo, Misael, Melquías, Lotasubo y Nabarías.
\par 45 Entonces Esdras llevó el libro de la ley delante de la multitud, porque a la vista de todos estaba sentado en el primer lugar con honor.
\par 46 Y cuando abrió la ley, todos se pusieron de pie. Entonces Esdras bendijo al Señor Dios Altísimo, Dios de los ejércitos, Todopoderoso.
\par 47 Y todo el pueblo respondió: Amén; y alzando las manos, cayeron al suelo y adoraron al Señor.
\par 48 También Jesús, Anus, Sarabias, Adinus, Jacubus, Sabateas, Auteas, Maianeas, Calitas, Asrias, Joazabdo, Ananías y Biatas, los levitas, enseñaban la ley del Señor, haciéndoles entenderla.
\par 49 Entonces Atratates habló con Esdras, el sumo sacerdote y lector, y a los levitas que enseñaban a la multitud, a todos, diciendo:
\par 50 Este día es santo para el Señor; (porque todos lloraron cuando oyeron la ley:)
\par 51 Id, pues, y comed la grosura, y bebed lo dulce, y enviad parte a los que no tienen;
\par 52 Porque este día es santo para el Señor; y no estéis tristes; porque el Señor te honrará.
\par 53 Entonces los levitas anunciaron todas las cosas al pueblo, diciendo: Este día es santo para el Señor; no os entristezcáis.
\par 54 Entonces se fueron, cada uno para comer y beber, y divertirse, y dar parte a los que no tenían, y hacer gran alegría;
\par 55 Porque entendieron las palabras con las que habían sido instruidos y para las cuales habían sido reunidos.

\end{document}