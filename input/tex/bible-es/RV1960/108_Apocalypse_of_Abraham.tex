\begin{document}

\title{Apocalipsis de Abraham}

\chapter{1}

\par El libro del Apocalipsis de Abraham, hijo de Taré, hijo de Nacor, hijo de Serug, hijo de Roog (Reu), hijo de Arfaxad, hijo de Sem, hijo de Noé, hijo de Lamec, hijo de Matusalén, hijo de Enoc, hijo de Jared (Arad).

\par \textit{La conversión de Abraham de la idolatría <br> (Capítulos I.-VIII.).}

\par 1 El día en que planeé los dioses de mi padre Taré y los dioses de su hermano Nacor, cuando buscaba quién es en verdad el Dios Fuerte, yo, Abraham, en el momento en que me tocó la suerte. , cuando cumplí los servicios (los sacrificios) de mi padre Taré a sus dioses de madera y piedra, oro y plata, bronce y hierro; Habiendo entrado en su templo para servir, encontré al dios cuyo nombre era Merumath (que era) tallado en piedra, caído a los pies del dios de hierro Nahon. Y aconteció que cuando vi esto, mi corazón quedó perplejo, y pensé en mi mente que no podría hacerlo volver a su lugar, yo, Abraham, solo, porque estaba pesado, siendo de gran tamaño. piedra grande, y salí y se lo comuniqué a mi padre. Y entró conmigo, y cuando ambos lo movimos (al dios) hacia adelante, para traerlo de regreso a su lugar, su cabeza se cayó de él mientras yo todavía lo sostenía por la cabeza. Y aconteció que cuando mi padre vio que la cabeza de Merumath se le había caído, me dijo: «¡Abraham!» Y dije: «Aquí estoy». Y me dijo: «Tráeme un hacha, de las pequeñas, de la casa». Y se lo traje. Y cortó otro Merumath de otra piedra, sin cabeza, y la cabeza que había sido arrojada desde Merumath la colocó sobre ella, y el resto de Merumath lo hizo añicos.

\chapter{2}

\par 1 Hizo otros cinco dioses y me los dio y me ordenó que los vendiera afuera, en la calle de la ciudad. Y ensillé el asno de mi padre, y las puse encima, y ​​fui hacia la venta a venderlas. ¡Y he aquí! Los comerciantes de Fandana en Siria viajaban con camellos hacia Egipto para comerciar. Y hablé con ellos. Y uno de sus camellos lanzó un gemido, y el asno se asustó y saltó y trastornó a los dioses; y tres de ellos fueron destrozados, y dos se conservaron. Y aconteció que cuando los sirios vieron que yo tenía dioses, me dijeron: «¿Por qué no nos dijiste [que tenías dioses? Entonces los habríamos comprado] antes de que el asno oyera el ruido del camello, y no se habrían perdido. Danos, en cualquier caso, los dioses que quedan, y te daremos el precio adecuado por los dioses rotos, también por los dioses que se han conservado. Porque estaba preocupado en mi corazón sobre cómo podría llevarle a mi padre el precio de compra, y los tres rotos los arrojé al agua del río Gur, que estaba en ese lugar, y se hundieron en las profundidades, y no había nada más de ellos.

\chapter{3}

\par 1 Mientras todavía iba por el camino, mi corazón estaba perplejo dentro de mí y mi mente estaba distraída. Y dije en mi corazón: [«¿Qué mala acción es esta que está haciendo mi padre? ¿No es más bien el dios de sus dioses, ya que ellos surgen a través de sus cinceles y tornos y de su sabiduría, y no es más bien apropiado que adoren a mi padre, ya que son su obra? ¿Qué es este engaño de mi padre en sus obras?] He aquí, Merumath cayó y no pudo levantarse en su propio templo, ni yo solo pude moverlo hasta que vino mi padre, y los dos lo movimos; y como éramos demasiado débiles, se le cayó la cabeza y él (es decir, mi padre) la puso sobre otra piedra de otro dios, que había hecho sin cabeza. Y fueron despedazados del asno los otros cinco dioses, los cuales no podían ni ayudarse a sí mismos, ni herir al asno, porque éste los había despedazado; ni sus pedazos subieron del río». Y dije en mi corazón: «Si esto es así, ¿cómo puede Merumath, el dios de mi padre, teniendo la cabeza de otra piedra, y siendo él mismo hecho de otra piedra, rescatar a un hombre, o escuchar la oración de un hombre y recompensarlo?»

\chapter{4}

\par 1 Y mientras pensaba así, llegué a la casa de mi padre; y habiendo dado de beber al asno y preparado heno para él, traje la plata y la entregué en manos de mi padre Taré. Cuando lo vio se alegró, [y] dijo: «Bendito eres, Abraham, de mis dioses, porque has traído el precio de los dioses, para que mi obra no sea en vano». Y yo respondí y le dije: «¡Oye, oh padre mío, Taré! Bienaventurados tus dioses, porque tú eres su dios, ya que tú los hiciste; porque su bendición es ruina, y su poder es vano. Aquellos que no se ayudaron a sí mismos, ¿cómo, pues, te ayudarán o te bendecirán? He sido amable contigo en este asunto, porque al (usar) mi inteligencia, te he traído el dinero para los dioses quebrantados». Y cuando oyó esto, se enojó conmigo, porque había hablado palabras duras contra sus dioses.

\chapter{5}

\par 1 Yo, sin embargo, pensando en el enojo de mi padre, salí; [Y después que salí, mi padre gritó, diciendo: «¡Abraham!» Y dije: «Aquí estoy». Y él dijo: «Toma y recoge las astillas de la madera con la que hice dioses de madera de pino antes de que vinieras; y prepárame la comida del almuerzo». Y aconteció que cuando recogí las astillas de madera, encontré debajo de ellas un dios pequeño que había estado tendido entre los matorrales a mi izquierda, y en su frente estaba escrito: DIOS BARISAT. Y no le dije a mi padre que había encontrado al dios de madera Barisat debajo de las astillas. Y aconteció que cuando hube puesto las astillas en el fuego, para poder preparar la comida para mi padre, al salir a hacer una pregunta sobre la comida, puse a Barisat ante el fuego encendido, diciendo amenazadoramente que él: «Presta mucha atención, Barisat, [que] el fuego no se apague hasta que yo llegue; pero si se apaga, soplad sobre él para que vuelva a arder». Y salí y cumplí mi propósito. Y al regresar encontré a Barisat caído de espaldas, y sus pies rodeados de fuego y horriblemente quemados. Estallé en un ataque de risa y me dije a mí mismo: «¡En verdad, oh Barisat, tú puedes encender el fuego y cocinar la comida!» Y sucedió que mientras yo hablaba (así) en mi risa, él (es decir, Barisat) fue quemado gradualmente por el fuego y reducido a cenizas. Y llevé la comida a mi padre, y él comió. Y le di vino y leche, y se alegró y bendijo a su dios Merumath. Y yo le dije: «¡Oh padre Taré, no bendigas a tu dios Merumath, y no lo alabe, sino más bien alaba a tu dios Barisat porque, amándote más, se ha arrojado al fuego para cocinar tu comida!» Y me dijo: «¿Y dónde está ahora?» [Y dije:] «Es reducido a cenizas en la violencia del fuego y reducido a polvo». Y dijo: «¡Grande es el poder de Barisat! Yo haré otro hoy y mañana él preparará mi comida».

\chapter{6}

\par 1 Pero cuando yo, Abraham, escuché tales palabras de mi padre, me reí en mi mente y suspiré en el dolor y en la ira de mi alma, y ​​dije: «¿Cómo, entonces, lo que él hizo, fabricado? estatuas... ¿ser ayudante de mi padre? ¿O entonces el cuerpo estará sujeto a su alma, y ​​el alma al espíritu, y el espíritu a la necedad y la ignorancia? Y dije: «Conviene sufrir el mal una vez. Así que dirigiré mi mente a lo puro y abriré mis pensamientos ante él». [Y] respondí y dije: «Oh padre Taré, cualquiera de estos que alabe como a un dios, es un necio en su mente. He aquí, los dioses de tu hermano Ora, que están en el templo santo, son más dignos de honor que [los tuyos. Porque he aquí Zucheus, el dios de tu hermano Oron, es más digno de honor que tu dios Merumath, porque está hecho de oro que es muy apreciado por la gente, y cuando envejezca será remodelado; pero si vuestro dios Merumath es cambiado o quebrantado, no será renovado, porque es piedra; Lo cual también es el caso del dios Joavon [que está con Zuqueus por encima de los demás dioses: ¡cuánto más digno de honor es él que el dios Barisat, que está hecho de madera, mientras que él está forjado de plata! ¡Cómo se le hace, por adaptación del hombre, valioso para la apariencia exterior! Pero tu dios Barisat, mientras aún estaba, antes de haber sido preparado, desarraigado (? sobre la tierra y era grande y maravilloso con la gloria de las ramas y las flores, tú cortaste con el hacha, y por medio de tu arte ha sido hecho dios. Y he aquí, su gordura ya se secó y pereció, ha caído de lo alto a la tierra, ha pasado de la gran condición a la pequeñez, y la apariencia de su rostro se ha desvanecido, y él] El mismo Barisat es quemado por el fuego y reducido a cenizas y ya no existe;» y tú dices: «¡Hoy haré otro que mañana preparará mi comida!» «¡Ha perecido en destrucción total!»

\chapter{7}

\par 1 «He aquí, el fuego es más digno de honor que todas las cosas formadas, porque incluso lo que no está sujeto está sujeto a él, y las cosas que se corrompen fácilmente son burladas por sus llamas. Pero aún más digna de honor es el agua, porque vence al fuego y sacia la tierra. Pero ni siquiera a él lo llamo Dios, porque está sujeto a la tierra bajo la cual se inclina el agua. Pero llamo a la tierra mucho más digna de honor, porque domina la naturaleza (y la plenitud) del agua. Sin embargo, ni siquiera a ella (es decir, a la tierra) la llamo dios, [porque] también ella es secada por el sol, [y] es asignada al hombre para que la labra. [Llamo al sol más digno de honor que la tierra,] porque con sus rayos ilumina el mundo entero y las diferentes atmósferas. [Pero] ni siquiera a él lo llamo dios, porque de noche y por las nubes su curso se oscurece. Tampoco llamo dioses a la luna ni a las estrellas, porque también en su tiempo oscurecen [su] luz por la noche. [Pero] escucha [esto], Taré mi padre; porque te daré a conocer al Dios que lo hizo todo, no a los que consideramos dioses. ¿Quién es entonces Él? ¿O quién es Él?

\par 2 Quien enrojeció los cielos y doró el sol, resplandeció la luna y con ella las estrellas;

\par 3 y secó la tierra en medio de muchas aguas,

\par 4 Y ponte en . . .[y me puso a prueba en la confusión de mis pensamientos

\par 5 «¡Sin embargo, que Dios se revele a nosotros por sí mismo!»

\chapter{8}

\par 1 Y aconteció que mientras hablaba así a mi padre Taré en el patio de mi casa, descendió la voz de un Poderoso del cielo en una nube de fuego, diciendo y clamando: «¡Abraham, Abraham!» Y dije: «Aquí estoy». Y dijo: «Buscas en el entendimiento de tu corazón al Dios de los dioses y al Creador: Yo soy. Sal de tu padre Taré, y sal de casa, para que tú también no seas asesinado en los pecados. de la casa de tu padre». Y salí. Y aconteció que cuando salí, antes que lograra salir delante de la puerta del atrio, se oyó un trueno y lo quemó a él y a su casa y todo lo que había en su casa, hasta abajo. hasta el suelo, cuarenta codos.

\chapter{9}

\par \textit{Abraham recibe el mandato divino de ofrecer un sacrificio después de cuarenta días como preparación para una revelación divina (Capítulo IX.; cf. Gén. xv.).}

\par 1 Entonces vino a mí una voz que hablaba dos veces: «¡Abraham, Abraham!» Y dije: «¡Aquí estoy!» Y dijo: «He aquí, soy yo; No temáis, porque yo soy delante de los mundos, y un Dios poderoso que ha creado la luz del mundo. Soy un escudo sobre ti y soy tu ayuda. Ve, tómame una novilla de tres años, una cabra de tres años, un carnero de tres años, una tórtola y un palomino, y tráeme un sacrificio puro. Y en este sacrificio pondré delante de ti las edades (por venir), y te haré saber lo que está reservado, y verás grandes cosas que no has visto (hasta ahora); porque te ha gustado buscarme, y te he llamado mi Amigo. Pero abstente de toda forma de alimento que sale del fuego, y de beber vino, y de ungirte con aceite, durante cuarenta días», y luego prepárame el sacrificio que te he mandado, en el lugar que te mostraré, en un monte alto, y allí te mostraré las edades que han sido creadas y establecidas, hechas y renovadas, por mi Palabra, y te haré saber lo que sucederá en ellas en los que han hecho mal y (practicado) justicia en la generación de los hombres.

\chapter{10}

\par \textit{Abraham, bajo la dirección del ángel Jaoel, procede al monte Horeb, viaje de cuarenta días, para ofrecer el Sacrificio (Capítulos X.-XII.).}

\par 1 Y aconteció que cuando oí la voz de Aquel que me hablaba tales palabras, miré de aquí para allá y ¡he aquí! No había aliento de hombre, y mi espíritu se espantó, y mi alma huyó de mí, y quedé como una piedra, y caí a la tierra, porque ya no tenía fuerzas para estar en pie sobre la tierra. Y mientras aún estaba acostado con mi rostro en la tierra, oí la voz del Santo hablar: «Ve, Jaoel, y por medio de mi Nombre inefable levántame a aquel hombre, y fortalécelo (para que se recupere) de su temblor». Y vino el ángel que me había enviado, con semejanza de hombre, y tomándome de mi mano derecha, me levantó sobre mis pies y me dijo: «Levántate, [Abraham,] amigo de Dios que te ama; ¡No dejes que el temblor del hombre te apodere de ti! ¡Porque he aquí! He sido enviado a ti para fortalecerte y bendecirte en el nombre de Dios, que te ama, el Creador de lo celestial y lo terrestre. No tengáis miedo y apresuraos hacia Él. Soy llamado Jaoel por Aquel que mueve lo que existe conmigo en la séptima expansión del firmamento, poder en virtud del Nombre inefable que habita en mí. Yo soy el encargado de frenar, según su mandamiento, el ataque amenazador de los seres vivientes de los querubines unos contra otros, y de enseñar a los que lo portan el cántico de la hora séptima de la noche del hombre. Estoy ordenado para contener al Leviatán, porque a mí están sujetos el ataque y la amenaza de cada uno de los reptiles. [Yo soy el que ha recibido el encargo de desatar el Hades, de destruir al que mira fijamente a los muertos.] Yo soy el que recibió el encargo de prender fuego a la casa de tu padre junto con él, porque mostró reverencia a los muertos (ídolos). He sido enviado para bendecirte ahora, y la tierra que el Eterno, a quien has invocado, ha preparado para ti, y por ti he seguido mi camino sobre la tierra. ¡Levántate, Abrahán! Ve sin miedo; alégrate y regocíjate; ¡Y yo estoy contigo! Porque el honor eterno te ha sido preparado por el Eterno. Id, cumplid los sacrificios mandados. ¡Por he aquí! He sido designado para estar contigo y con la generación preparada (para surgir) de ti; y conmigo Miguel te bendice para siempre. ¡Ánimo, vete!

\chapter{11}

\par 1 Y me levanté y vi al que me había agarrado de la mano derecha y me había levantado sobre mis pies; el aspecto de su cuerpo era como el zafiro, y el aspecto de su rostro como crisólito, y el cabello de su cabeza como nieve, y el turbante sobre su cabeza como la apariencia del arco iris, y el vestido de sus vestiduras como púrpura; y en su mano derecha tenía un cetro de oro. Y él me dijo: «¡Abraham!» Y dije: «Aquí estoy, tu siervo». Y él dijo: «No te espante mi mirada, ni mi palabra, para que tu alma no se turbe. Ven conmigo y yo iré contigo, hasta el sacrificio, visible, pero después del sacrificio, invisible para siempre. ¡Ten ánimo y ven!

\chapter{12}

\par 1 Y fuimos los dos juntos durante cuarenta días y cuarenta noches, y no comí pan ni bebí agua, porque mi comida era ver al ángel que estaba conmigo y sus palabras: esa era mi bebida. Y llegamos al monte de Dios, el glorioso Horeb. Y dije al ángel: «¡Cantante del Eterno! ¡Mira! No tengo sacrificio conmigo, ni conozco lugar de altar en el monte: ¿cómo puedo traer un sacrificio?» Y me dijo: «¡Mira a tu alrededor!» Y miré a mi alrededor, ¡y he aquí! Nos seguían todos los animales de sacrificio prescritos: la novilla, la cabra, el carnero, la tórtola y la paloma. Y el ángel me dijo: «¡Abraham!» Dije: «Aquí estoy». Y me dijo: «Todos estos matan, y parten los animales en mitades, uno contra otro, pero las aves no parten; y ('pero') da a los hombres que te mostraré, que están junto a ti, porque estos son el altar sobre la Montaña, para ofrecer un sacrificio al Eterno; pero la tórtola y el palomo dame, porque subiré sobre las alas del pájaro, para mostrarte en el cielo, y en la tierra, y en el mar, y en el abismo, y en el inframundo. , y en el Jardín del Edén, y en sus ríos y en la plenitud del mundo entero y su círculo, contemplarás en (ellos) todos».

\chapter{13}

\par \textit{Abraham realiza el Sacrificio, bajo la guía del Ángel, y se niega a ser desviado de su Propósito por Azazel (Capítulos XIII.-XIV.).}

\par 1 E hice todo según el mandato del ángel y di a los ángeles que habían venido a nosotros los animales divididos, pero el ángel se llevó las aves. Y esperé el sacrificio de la tarde. Y un pájaro inmundo voló sobre los cadáveres y lo ahuyenté. Y el pájaro inmundo me habló y me dijo: «¿Qué haces tú, Abraham, en las alturas santas, donde nadie come ni bebe, ni hay sobre ellos alimento de hombre, sino que todo lo consumen con fuego, y (te) quemará. Deja al hombre que está contigo y huye; porque si subes a las Alturas te acabarán. Y aconteció que cuando vi hablar el pájaro, dije al ángel: '¿Qué es esto, señor mío?' Y él dijo: 'Esto es impiedad, este es Azazel'. Y le dijo: «¡Deshonra sobre ti, Azazel! Porque la suerte de Abraham está en el cielo, pero la tuya en la tierra. Porque has elegido y amado esto para la morada de tu inmundicia, por eso el eterno y poderoso Señor te hizo habitante de la tierra y por ti todo espíritu maligno de mentira, y por ti ira y pruebas para las generaciones de impíos. hombres; porque Dios, el Eterno, Poderoso, no ha permitido que los cuerpos de los justos estén en tu mano, para que así esté asegurada la vida de los justos y la destrucción de los inmundos. Escucha, amigo, aléjate de mí de la vergüenza. Porque no te ha sido permitido ser el tentador de todos los justos. ¡Apartaos de este hombre! No puedes extraviarlo, porque es enemigo tuyo y de los que te siguen y aman lo que quieres. Porque he aquí, la vestidura que antes era tuya en el cielo ha sido apartada para él, y la mortalidad que era suya ha sido transferida a ti».

\chapter{14}

\par 1 El ángel me dijo: [«¡Abraham!» Y dije: «Aquí estoy, tu siervo». Y dijo: «Sabe desde ahora que el Eterno te ha elegido, (Aquel) a quien amas; ten ánimo y usa esta autoridad, en la medida que yo te ordene, contra aquel que calumnia la verdad; ¿No podría yo avergonzar a aquel que ha esparcido por la tierra los secretos del cielo y se ha rebelado contra el Poderoso?] Dile: 'Sé tú el carbón encendido del horno de la tierra; Ve, Azazel, a las partes inaccesibles de la tierra; [porque tu herencia es (será) sobre aquellos que existen contigo naciendo con las estrellas y las nubes, con los hombres cuya porción eres, y (que) a través de tu ser existen; y tu enemistad es justificación. Por esta razón, por tu perdición, desaparece de mí». Y pronuncié las palabras que el ángel me había enseñado. Y él dijo: «¡Abraham!» Y dije: «Aquí estoy, tu siervo».]

\par 2 Y el ángel me dijo: «No le respondas; porque Dios le ha dado poder (lit. voluntad) sobre aquellos que le responden». [Y el ángel me habló por segunda vez y me dijo: «Ahora más bien, por mucho que te hable, no le respondas, para que su voluntad no tenga libre curso en ti, porque el Eterno y Fuerte le ha dado peso y voluntad; No le respondas». Hice lo que me ordenó el ángel;] y por mucho que me hablaba, nada le respondía.

\chapter{15}

\par \textit{Abraham y el Ángel ascienden en las Alas de los Pájaros al Cielo (Capítulos XV.-XVI.).}

\par 1 Y aconteció que cuando se puso el sol, ¡he aquí! un humo como de horno. Y los ángeles que tenían las porciones del sacrificio subieron de lo alto del horno humeante. Y el ángel me tomó con la mano derecha y me puso en el ala derecha de la paloma, y ​​se puso en el ala izquierda de la tórtola, que (las aves) no habían sido degolladas ni divididas. Y me llevó hasta los límites del fuego llameante [y ascendimos como con muchos vientos al cielo que estaba fijo en la superficie. Y vi en el aire, en la altura a la que ascendíamos, una luz fuerte, que era imposible describir, ¡y he aquí! en esta luz, un fuego ferozmente ardiendo para la gente, muchas personas de apariencia masculina, todos (constantemente) cambiando de aspecto y forma, corriendo y siendo transformados, y adorando y llorando con un sonido de palabras que yo no conocía.

\chapter{16}

\par 1 Y dije al ángel: «¿Por qué me has hecho subir aquí ahora, porque ahora no puedo ver, porque ya estoy débil y mi espíritu se aparta de mí?» Y me dijo: «Quédate conmigo; ¡no temáis! Y aquel a quien ves venir directamente hacia nosotros con gran voz de santidad, ese es el Eterno que te ama; pero a Él mismo no puedes verlo). Pero no desmayes tu espíritu [a causa del fuerte clamor], porque yo estoy contigo, fortaleciéndote».

\chapter{17}

\par \textit{Abraham, enseñado por el Ángel, pronuncia el Canto Celestial y ora por la Iluminación (Capítulo XVII.).}

\par 1 Y mientras aún hablaba (y) ¡he aquí! Vino fuego contra nosotros en derredor, y en el fuego había una voz como el estruendo de muchas aguas, como el estruendo del mar en su alboroto. Y el ángel inclinó su cabeza hacia mí y adoró. Y quise caer sobre la tierra, y el lugar alto en el que estábamos, [en un momento se levantó,] pero en otro rodó hacia abajo.

\par 2 Y él dijo: «Sólo adora, Abraham, y canta el cántico que te he enseñado». porque no había tierra donde caer. Y yo sólo adoré y pronuncié el cántico que él me había enseñado. Y él dijo: «Recitad sin cesar». Y yo recité, y él también conmigo recitó la canción:

\par 3 Eterno, poderoso, Santo, El,
\par     Sólo Dios: ¡Supremo!
\par 4 Tú que eres originado por ti mismo, incorruptible, sin mancha,
\par 5 Increado, inmaculado, inmortal,
\par     Autocompleto, autoiluminante;
\par 6 ¡Sin padre, sin madre, inengendrado, exaltado y ardiente!
\par 7 Amante de los hombres, benevolente, generoso,
\par 8 celoso de mí y muy compasivo;
\par 9 Elí, es decir, Dios mío,
\par 10 Sabaot santo, poderoso y eterno,
\par 11 muy glorioso El, El, El, El, Jaoel!
\par 12 ¡Tú eres Aquel a quien mi alma amó!
\par 13 Protector eterno, resplandeciente como el fuego,
\par 14 Cuya voz es como el trueno,
\par 15 Cuya mirada es como el relámpago, que todo lo ve,
\par 16 ¡Quién recibe las oraciones de los que te honran!
\par 17 [Y se aparta de las peticiones de los que avergüenzan con la vergüenza de sus provocaciones,
\par 18 ¡Quien disuelve las confusiones del mundo que surgen de los impíos y los justos en la era corruptible, renovando la era de los justos!]
\par 19 Tú, oh Luz, brillas ante la luz del
\par 20 mañana sobre tus criaturas,
\par 21 [para que se convierta en día sobre la tierra]
\par 22 Y en tus moradas celestiales no hay
\par     necesidad de cualquier otra luz
\par 23 que (el) del indescriptible esplendor del
\par     luces de tu rostro.
\par 24 Acepta mi oración y alégrate de ella,
\par 25 Asimismo también el sacrificio que has preparado
\par     ¡Tú a través de mí que te busqué!
\par 26 Acéptame, muéstrame y enséñame,
\par 27 ¡Y haz saber a tu siervo lo que me has prometido!

\chapter{18}

\par \textit{La visión de Abraham del Trono Divino (Capítulo XVIII.).}

\par 1 Y mientras todavía recitaba la canción, la boca del fuego que estaba en la superficie se elevó hacia lo alto. Y oí una voz como el rugido del mar; ni cesó a causa de la abundancia del fuego. Y mientras el fuego se elevaba, subiendo a las alturas, vi debajo del fuego un trono de fuego, y alrededor de él seres que todo lo veían, cantando la canción, y debajo del trono cuatro seres vivientes de fuego cantando, y su apariencia Era uno, cada uno de ellos con cuatro caras. Y tal era el aspecto de sus rostros, de león, de hombre, de buey, de águila: cuatro cabezas [había sobre sus cuerpos] [de modo que las cuatro criaturas tenían dieciséis caras] y cada una tenía seis alas; desde sus hombros, [y sus costados] y sus lomos. Y con las (dos) alas de sus hombros cubrieron sus rostros, y con las (dos) alas que (surgieron) de sus lomos cubrieron sus pies, mientras que las (dos) alas del medio las extendieron para volar en línea recta. Y cuando terminaron de cantar, se miraban unos a otros y se amenazaban. Y aconteció que cuando el ángel que estaba conmigo vio que se amenazaban unos a otros, me dejó y fue corriendo hacia ellos y apartó el rostro de cada ser viviente del rostro que estaba inmediatamente frente a él, para que no vieran. sus rostros se amenazaban mutuamente. Y les enseñó el cántico de paz que tiene su origen [en el Eterno].

\par 2 Y mientras estaba solo y miraba, vi detrás de los seres vivientes un carro con ruedas de fuego, cada rueda llena de ojos alrededor; y sobre las ruedas había un trono; Lo cual vi, y esto estaba cubierto de fuego, y el fuego lo rodeaba, y ¡he aquí! un fuego indescriptible envolvió una hueste ardiente. Y oí su santa voz como voz de hombre.

\chapter{19}

\par \textit{Dios revela a Abraham los poderes del cielo (Capítulo XIX.).}

\par 1 Y vino a mí una voz de en medio del fuego, que decía: «¡Abraham, Abraham!» Dije: «¡Aquí estoy!» Y dijo: «Considera las expansiones que hay bajo el firmamento en el que (ahora) estás colocado, y mira cómo en ninguna expansión hay otro que Aquel a quien has buscado o que te ha amado». Y mientras aún estaba hablando (y) ¡he aquí! se abrieron las expansiones, y debajo de mí los cielos. Y vi en el séptimo firmamento sobre el cual estaba un fuego extendido, y luz, y rocío, y multitud de ángeles, y un poder de gloria invisible sobre los seres vivientes que veía; pero no vi allí ningún otro ser.

\par 2 Y miré desde la montaña en la que estaba parado hacia el sexto firmamento, y vi allí una multitud de ángeles, de espíritu (puro), sin cuerpo, que cumplían las órdenes de los ángeles de fuego que estaban sobre el octavo firmamento, mientras yo estaba suspendido sobre ellos. Y he aquí, sobre este firmamento no había otros poderes de (ninguna) otra forma, sino sólo ángeles de espíritu (puro), como el poder que vi en el séptimo firmamento. Y ordenó que el sexto firmamento fuera quitado. Y vi allí, en el quinto firmamento, los poderes de las estrellas que ejecutan los mandamientos que les han sido dados, y los elementos de la tierra los obedecieron.

\chapter{20}

\par \textit{La promesa de una semilla (Capítulo XX.).}

\par 1 Y el Eterno Poderoso me dijo: «¡Abraham, Abraham!» Y dije: «Aquí estoy». [Y Él dijo:] «Considera desde arriba las estrellas que están debajo de ti, y cuéntalas [para mí], y hazme saber [a mí] su número». Y dije: «¿Cuándo puedo? Porque no soy más que un hombre [de polvo y cenizas]». Y él me dijo: «Como el número de las estrellas y su poder, (así) haré de tu descendencia una nación y un pueblo, apartado para mí en mi herencia con Azazel».

\par 2 Y dije: «¡Oh Eterno, Poderoso! ¡Que tu siervo hable delante de ti, y no se encienda tu ira contra tu elegido! He aquí, antes de que me condujeses arriba, Azazel arremetió contra mí. ¿Cómo, pues, si él no está ahora delante de Ti, te has constituido con él?

\chapter{21}

\par \textit{Una visión del pecado y del paraíso: el espejo del mundo (Capítulo XXI.).}

\par 1 Y me dijo: «Mira ahora los firmamentos bajo tus pies y comprende la creación anunciada en esta expansión, las criaturas que existen en ella y la era preparada según ella». Y vi debajo [las superficies de los pies, y vi debajo] el sexto cielo y lo que había en él, y luego la tierra y sus frutos, y lo que se movía sobre ella y sus seres animados: y el poder de sus hombres, y la impiedad de sus almas, y sus obras justas [y los comienzos de sus obras], y las regiones inferiores y la perdición en ellas, el Abismo y sus tormentos. Vi allí el mar y sus islas, sus monstruos y sus peces, y Leviatán y su dominio, y su campamento, y sus cuevas, y el mundo que yacía sobre él, y sus movimientos, y las destrucciones del mundo. en su cuenta. Vi allí arroyos y el crecimiento de sus aguas y sus curvas. Y vi allí el Jardín del Edén y sus frutos, la fuente del arroyo que de él brotaba, y sus árboles y sus flores, y a los que se comportaban con justicia. Y vi en él sus alimentos y sus bienaventuranzas. Y vi allí una gran multitud, hombres, mujeres y niños [la mitad de ellos en el lado derecho del cuadro] y la mitad de ellos en el lado izquierdo del cuadro.

\chapter{22}

\par \textit{La Caída del Hombre y su Secuela (Capítulos XXIL-XXV.).}

\par 1 Y dije: «¡Oh Eterno, Poderoso! ¿Qué es esta imagen de las criaturas? Y me dijo: «Esta es mi voluntad con respecto a los que existen en el (divino) mundo-consejo, y me pareció agradable ante mis ojos, y luego les di mandamiento por mi Palabra. Y sucedió que lo que yo había decidido ser, ya estaba planeado de antemano en esta (imagen), y estuvo delante de mí antes de que fuera creado, como has visto».

\par 2 Y dije: «¡Oh Señor, poderoso y eterno! ¿Quiénes son las personas en esta foto de este lado y de aquel? Y me dijo: «Estos que están al lado izquierdo son la multitud de los pueblos que antes existieron y que están destinados después de ti, unos para juicio y restauración, y otros para venganza y destrucción al final del siglo. mundo. Pero estos que están en el lado derecho del cuadro son el pueblo apartado para mí de los pueblos con Azazel. Éstos son los que he ordenado para que nazcan de ti y sean llamados Mi Pueblo».

\chapter{23}

\par 1 «Ahora mira de nuevo en la imagen, quién es quien sedujo a Eva y cuál es el fruto del árbol, [y] sabrás lo que habrá, y cómo será para tu descendencia entre el pueblo en el fin de los días del mundo, y en lo que no puedas entender, te lo haré saber, porque eres agradable a mis ojos, y te diré lo que está guardado en mi corazón».

\par 2 Y miré el cuadro, y mis ojos se dirigieron hacia el lado del Jardín del Edén. Y vi allí un hombre muy grande en altura y temible en ancho, incomparable en aspecto, abrazando a una mujer, que igualmente se aproximaba al aspecto y forma del hombre. Y estaban parados debajo de un árbol (del Jardín del) Edén, y el fruto de este árbol tenía la apariencia de un racimo de uvas de vid, y detrás del árbol estaba de pie como si fuera una serpiente en forma, que tenía manos. y pies como de hombre, y alas sobre sus hombros, seis al lado derecho y seis al izquierdo, y tenían en sus manos las uvas del árbol, y ambos comían de él a los que yo había visto abrazados.

\par 3 Y dije: «¿Quiénes son estos que se abrazan, o quién es el que está entre ellos, o cuál es el fruto que comen, oh Poderoso Eterno?»

\par 4 Y dijo: «Éste es el mundo humano, éste es Adán, y éste es su deseo en la tierra, ésta es Eva; pero el que está entre ellos representa la impiedad, su comienzo (en camino) a la perdición, es decir, Azazel».

\par 5 Y dije: «¡Oh Eterno, Poderoso! ¿Por qué has dado tal poder para destruir la generación de los hombres en sus obras sobre la tierra?

\par 6 Y me dijo: «Aquellos que quieren (hacer) el mal, ¡y cuánto odié a los que lo hacen! —Le di poder sobre ellos y ser amado por ellos».

\par 7 Y yo respondí y dije: «¡Oh Eterno, Poderoso! ¿Por qué has querido hacer que se desee el mal en los corazones de los hombres, estando en verdad enojado por lo que querías, contra el que hace lo que no es provechoso en tu consejo?

\chapter{24}

\par 1 Y Él me dijo: «Estando enojado contra las naciones por tu causa y por la gente de tu familia que será (será) separada después de ti, como ves en la imagen la carga (del destino) que (es impuesto) sobre ellos, y te diré lo que será y cuánto será en los últimos días. Mire ahora todo lo que hay en la imagen».

\par 2 Y miré y vi allí lo que había delante de mí en la creación; Vi a Adán y a Eva existiendo con él, y con ellos al astuto Adversario, y a Caín que actuó ilegalmente a través del Adversario, y al Abel masacrado, (y) la destrucción que trajo y causó sobre él a través del inicuo. Vi allí también la impureza, y a los que la codician, y su contaminación, y sus celos, y el fuego de su corrupción en las partes más bajas de la tierra. Vi allí el robo, y a los que se apresuran tras él, y el arreglo [de su retribución, el juicio del Gran Assize]. Vi allí hombres desnudos, las frentes unos contra otros, y su deshonra, y la pasión que (tenían) unos contra otros, y su retribución. Vi allí el Deseo, y en su mano la cabeza de toda clase de maldad [y su desprecio y su derroche asignados a la perdición].

\chapter{25}

\par 1 Vi allí la figura del ídolo de los celos, que tenía la forma de una pieza de madera como la que mi padre solía hacer, y su estatua era de bronce reluciente; y delante de ella un hombre, y la adoró; y delante de él un altar, y sobre él un niño inmolado en presencia del ídolo.

\par 2 Pero yo le dije: «¿Qué es este ídolo, o qué es el altar, o quiénes son los que son sacrificados, o quién es el sacrificador? ¿O qué es el templo que veo, que es hermoso en arte, y su belleza (se parece) a la gloria que yace debajo de tu trono?»

\par 3 Y Él dijo: «Oye, Abraham. Esto que ves, el Templo, el altar y la belleza, es mi idea del sacerdocio de mi glorioso Nombre, en el que mora cada oración del hombre, y el surgimiento de reyes y profetas, y cualquier sacrificio que ordeno que se me ofrezca. entre mi pueblo que ha de salir de tu generación. Pero la estatua que has visto es mi ira con la que me enoja el pueblo que ha de partir de ti por mí. Pero el hombre a quien viste degollar, ése es el que incita a sacrificios asesinos, de (sic) los cuales son testigo para mí del juicio final, incluso al comienzo de la creación».

\chapter{26}

\par \textit{Por qué se permite el pecado (Capítulo XXVI.).}

\par 1 Y dije: «¡Oh Eterno, Poderoso! ¿Por qué has establecido que así sea y luego proclamas su conocimiento?»

\par 2 Y Él me dijo: «Oye, Abraham; entiende lo que te digo y respóndeme cuando te pregunto. ¿Por qué tu padre Taré no escuchó tu voz, y (por qué) no cesó en la idolatría diabólica hasta que pereció [y] toda su casa con él?

\par 3 Y dije: «¡Oh Eterno, [Poderoso]! (Fue) enteramente porque él no eligió escucharme; pero yo tampoco seguí sus obras».

\par 4 Y Él me dijo: «Oye, Abraham. Así como el consejo de tu padre está en él, y como tu consejo está en ti, así también está listo el consejo de mi voluntad en mí para los días venideros, antes de que tengas conocimiento de éstos, o (puedas) ver con tus ojos lo que sucederá. Hay futuro en ellos. Cómo serán los de tu descendencia, mira en la imagen».

\chapter{27}

\par \textit{Una Visión de Juicio y Salvación (Capítulo XXVII.).}

\par 1 Y miré y vi: ¡he aquí! el cuadro se balanceó y [de él] surgió, en su lado izquierdo, un pueblo pagano, y saquearon a los que estaban en el lado derecho, hombres y mujeres y niños: [a algunos los masacraron,] a otros se los quedaron. ¡Mira! Los vi correr hacia ellos por cuatro entradas, y quemaron el templo con fuego, y saquearon las cosas sagradas que allí había.

\par 2 Y dije: «¡Oh Eterno! ¡Mira! Al pueblo (que surge) de mí, a quien Tú has aceptado, las hordas de paganos lo saquean, y a algunos los matan, mientras que a otros los retienen como a extraños, y el Templo lo han quemado con fuego, y las cosas hermosas que allí hay. robar [y destruir]. ¡Oh Eterno, Poderoso! Si esto es así, ¿por qué has lacerado ahora mi corazón, y por qué debería ser así?

\par 3 Y Él me dijo: «Oye, Abraham. Lo que has visto sucederá a causa de tu descendencia que me enoja a causa de la estatua que viste, y a causa de la matanza humana en la imagen, por el celo en el Templo; y como has visto, así será».

\par 4 Y dije: «¡Oh Eterno, Poderoso! Que pasen ahora las obras de maldad (hechas) en impiedad, pero (muéstrame) más bien aquellos que cumplieron los mandamientos, incluso las obras de su (?) justicia. Porque tú puedes hacer esto».

\par 5 Y Él me dijo: «El tiempo de los justos llega primero a través de la santidad (que fluye) de los reyes y gobernantes justos que al principio creé para que gobernaran entre ellos. Pero de estos salen hombres que cuidan de sus intereses, como te he hecho saber y has visto.

\chapter{28}

\par \textit{¿Cuánto tiempo? (Capítulos XXVIII.-XXIX.).}

\par 1 Y respondí y dije: «¡Oh Poderoso, [Eterno] santificado por Tu poder! Sé favorable a mi petición, [porque para esto me has traído aquí, y muéstramelo]. Así como me has elevado a tu altura, así hazlo saber a mí, tu amado, cuanto te pida: ¿les sucederá por mucho tiempo lo que vi?

\par 2 Y me mostró una multitud de su pueblo, y me dijo: «Por causa de ellos, a través de cuatro problemas, como viste, seré irritado por ellos, y en estos se (cumplirá) mi retribución por sus obras. Pero en la cuarta salida de cien años y una hora de la edad (lo mismo son cien años) habrá desgracia entre los paganos [pero una hora en misericordia y contumacia, como entre los paganos]».

\chapter{29}

\par 1 Y dije: «¿¡Oh Eterno [Poderoso]!? ¿Y cuánto tiempo es una hora de la Era?

\par 2 Y dijo: Doce años he ordenado a esta Era impía que gobierne entre los paganos y en tu descendencia; y hasta el fin de los tiempos será como has visto. Y considera, comprende y mira el cuadro».

\par 3 Y miré y vi a un hombre que salía del lado izquierdo de las naciones; y salieron muchos ejércitos de entre las naciones, hombres, mujeres y niños, y lo adoraron. Y mientras todavía miraba, salieron del lado derecho (muchos), y algunos insultaban a aquel hombre, mientras otros le golpeaban; otros, sin embargo, lo adoraron. [Y] vi cómo éstos lo adoraban, y Azazel corrió y lo adoró, y después de besarle el rostro, se volvió y se paró detrás de él.

\par 4 Y dije: «¡Oh Etemal, Poderoso! ¿Quién es el hombre insultado y golpeado, que es adorado por las naciones con Azazel?

\par 5 Y él respondió y dijo: «¡Oye, Abraham! El hombre a quien viste insultado, golpeado y nuevamente adorado, ¿ese es el alivio? (concedido) por los paganos al pueblo que procede de ti, en los últimos días, en esta hora duodécima de la Era de la impiedad. Pero en el año duodécimo de mi Era final, levantaré de tu generación a este hombre, a quien viste (proceder) de mi pueblo; A éste le seguirán todos, y los que yo llamo se unirán, incluso los que cambien de consejo. Y aquellos a quienes viste emerger del lado izquierdo de la imagen, el significado es: Habrá muchos de los paganos que pondrán sus esperanzas en él; y en cuanto a los que viste de tu descendencia a la derecha, algunos insultándolo y golpeándolo, otros adorándolo, muchos de ellos se escandalizarán de él. Él, sin embargo, está probando a los que le han adorado de tu descendencia, en esa hora duodécima del Fin, con miras a acortar la Era de la impiedad».

\par 6 »Antes de que comience a crecer la Era de los justos, mi juicio vendrá sobre los paganos inicuos a través de la gente de tu descendencia que ha sido separada por mí. En aquellos días traeré sobre todas las criaturas de la tierra diez plagas, mediante desgracias y enfermedades y los suspiros del dolor de sus almas. Así traeré sobre las generaciones de hombres que estén sobre ella a causa de la provocación y la corrupción de sus criaturas, con las cuales me provocan. Y entonces quedarán hombres justos de tu descendencia en el número que Yo mantengo en secreto, corriendo en la gloria de Mi Nombre al lugar preparado de antemano para ellos, que viste devastado en el cuadro; y vivirán y serán establecidos mediante sacrificios y dones de justicia y verdad en la Era de los justos, y se regocijarán en Mí continuamente; y destruirán a los que los destruyeron, e insultarán a los que los insultaron».

\par 7 A quienes los difamaron les escupirán en la cara, despreciados por Mí, mientras ellos (los justos) Me verán llenos de alegría, regocijándose con Mi pueblo y recibiendo a los que regresan a Mí [en arrepentimiento].»

\par 8 «Mira, Abraham, lo que has visto, y escucha lo que has oído, y [toma pleno conocimiento de] lo que has llegado a saber. Ve a tu herencia, ¡y he aquí! Estoy contigo para siempre.»

\chapter{30}

\par \textit{El castigo de los paganos y la recolección de Israel (Capítulos XXX.-XXXI.).}

\par 1 Pero mientras él aún hablaba, me encontré en la tierra. Y dije: «Oh Eterno, [Poderoso], ya no estoy en la gloria en la que estaba (mientras) estaba en lo alto, y lo que mi alma anhelaba entender en mi corazón no lo entiendo».

\par 2 Y Él me dijo: «Te diré lo que deseas en tu corazón, porque has buscado ver las diez plagas que he preparado para las naciones, y que he preparado de antemano para el paso de la hora duodécima. de la tierra. Oye lo que te divulgo, así sucederá: el primero (es) dolor de gran angustia; el segundo, incendio de muchas ciudades; el tercero, destrucción y pestilencia de los animales; el cuarto, hambre del mundo entero y de su pueblo; el quinto, destrucción de sus gobernantes, destrucción por terremoto y espada; el sexto, multiplicación del granizo y de la nieve; el séptimo, las fieras serán su sepulcro; el octavo, el hambre y la pestilencia se alternarán con su destrucción; el noveno, castigo con espada y huida en apuros; el décimo, truenos y voces y terremoto destructivo».

\chapter{31}

\par 1 »Y entonces tocaré la trompeta en el aire y enviaré a mi Elegido, que tiene en él todo mi poder, una medida; y éste convocará a mi pueblo despreciado de las naciones, y quemaré con fuego a los que los han insultado y a los que han gobernado entre ellos en (esta) Era».

\par 2 »Y a los que me han cubierto de burla, los entregaré a la burla del siglo venidero; y los he preparado para que sean alimento para el fuego del Hades y para el vuelo incesante de un lado a otro por el aire en el inframundo debajo de la tierra [el cuerpo lleno de gusanos]. Porque en ellos verán la justicia del Creador, es decir, aquellos que han elegido hacer mi voluntad y aquellos que han guardado abiertamente mis mandamientos, (y) se regocijarán con gozo por la caída de los hombres que aún permanecen. , que han seguido a los ídolos y sus asesinatos. Porque se pudrirán en el cuerpo del malvado gusano Azazel, y serán quemados con el fuego de la lengua de Azazel; porque esperaba que vendrían a mí, y no habrían amado y alabado al (dios) extraño, y no se habrían adherido a aquel para quien no estaban destinados, sino (en cambio) habrían abandonado al Señor fuerte.»

\chapter{32}

\par \textit{Conclusión (Capítulo XXXII.)}

\par 1 «Por tanto, oh Abraham, oye y mira; ¡Mira! tu séptima generación (irá) contigo, y saldrán a tierra extraña, y los esclavizarán, y el mal les suplicará como si fuera una hora de la Era de la impiedad, pero la nación a quien servirán, la haré. juez.»

\end{document}