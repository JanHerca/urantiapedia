\begin{document}


\title{Testamento de Jacob}

\chapter{1}

\par 1 En el nombre del Padre, del Hijo y del Espíritu Santo, el único Dios.

\par 2 Comenzamos, con la ayuda del Dios Altísimo y por su mediación, a escribir la historia de la vida de nuestro padre, el patriarca Jacob, hijo del patriarca Isaac, el día veintiocho del mes de Misri.

\par 3 Que la bendición de su oración nos guarde y nos proteja de las tentaciones del enemigo obstinado. ¡Amén, amén, amén!

\par 4 Él dijo: «Vengan y escuchen, amados míos y hermanos míos que aman al Señor, lo que han recibido».

\par 5 Cuando se acercaba el tiempo de nuestro padre Jacob, padre de padres, hijo de Isaac, hijo de Abraham, y se acercaba para hurtarse de su cuerpo, este fiel era avanzado en años y en distinción.

\par 6 Entonces el Señor le envió a Miguel, el jefe de los ángeles, quien le dijo: «Oh Israel, amado mío, de noble linaje, escribe tu herencia hablada y tus instrucciones para tu casa y dales un pacto; Ocúpate también del buen orden de tu casa, porque se ha acercado el tiempo en que volverás a tus padres para alegrarte con ellos para siempre».

\par 7 Cuando nuestro padre Jacob, el fiel, oyó esto del ángel, respondió y dijo, como tenía por costumbre hablar todos los días de esta manera con los ángeles:

\par 8 «Hágase la voluntad del Señor».

\par 9 Y Dios pronunció bendición sobre nuestro padre Jacob. Jacob tenía un lugar apartado al que entraba para ofrecer sus oraciones ante el Señor de noche y de día.

\par 10 Los ángeles lo visitarían, lo protegerían y lo fortalecerían en todo.

\par 11 Dios lo bendijo y multiplicó su pueblo en la tierra de Egipto cuando descendió a la tierra de Egipto para encontrarse con su hijo José.

\par 12 Sus ojos se habían vuelto apagados por el llanto, pero cuando descendió a Egipto vio claramente al ver a su hijo.

\par 13 Entonces Jacob-Israel se inclinó rostro en tierra, luego se echó sobre el cuello de su hijo José y lo besó, mientras lloraba y decía: «Ahora puedo morir, oh hijo mío, porque he visto tu rostro una vez. más en mi vida; Oh mi amado hijo».


\chapter{2}


\par 1 José continuó gobernando sobre todo Egipto, mientras que Jacob permaneció en la tierra de Gosén diecisiete años y envejeció mucho, de modo que se cumplió su período de vida.

\par 2 Guardó continuamente todos los mandamientos y temió al Señor.

\par 3 Sus ojos se nublaron y su vida estaba tan cerca de terminar que no podía ver a una sola persona debido a su larga vida y senilidad.

\par 4 Entonces alzó sus ojos hacia la luz de Isaac, pero tuvo miedo y se turbó.

\par 5 Entonces el ángel le dijo: «No temas, oh Jacob; Yo soy el ángel que ha estado caminando contigo y protegiéndote desde tu infancia.

\par 6 Te anuncié que recibirías la bendición de tu padre y de Rebeca, tu madre.

\par 7 Yo soy el que está contigo, oh Israel, en todos tus actos y en todo lo que has presenciado.

\par 8 ¿Te salvé de Labán? cuando os ponía en peligro y os perseguía.

\par 9 En aquel tiempo os di todos sus bienes y os bendije a vosotros, a vuestras mujeres, a vuestros hijos y a vuestros rebaños.

\par 10 Yo soy quien os salvó de la mano de Esaú.

\par 11 Yo soy el que te acompañó a la tierra de Egipto, oh Israel, y te fue dada un pueblo muy grande.

\par 12 Bienaventurado Abraham vuestro padre, porque se ha hecho amigo de Dios (¡exaltado sea él!), por su generosidad y su amor a los extraños.

\par 13 Bienaventurado tu padre Isaac, que te engendró, porque fue un sacrificio perfecto, agradable a Dios.

\par 14 Bienaventurado tú también, oh Jacob, porque has visto a Dios cara a cara.

\par 15 Viste al ángel de Dios, ¡exaltado sea!, y viste la escalera firme en el suelo, con su punta en el cielo.

\par 16 Entonces viste al Señor sentado en su cima con un poder que nadie podría describir.

\par 17 Hablaste y dijiste: «Esta es la casa de Dios y esta es la puerta del cielo».

\par 18 Bienaventurado eres, porque te has acercado a Dios y él es fuerte entre los hombres; así que ahora no te turbes, oh elegido de Dios.

\par 19 «Bendito eres, oh Israel, y bendita toda tu descendencia.

\par 20 Porque todos vosotros seréis llamados patriarcas hasta el fin de los tiempos y de las épocas; vosotros sois el pueblo y el linaje de los siervos de Dios.

\par 21 Bendita sea la nación que se esforzará por vuestra pureza y verá vuestras buenas obras.

\par 22 Bendito sea el hombre que se acuerde de ti en el día de tu noble fiesta.

\par 23 Bienaventurado el que hará obras de misericordia en honor de tus distintos nombres, y dará a alguien un vaso de agua para beber, o vendrá con una ofrenda al santuario, o acogerá a extraños, o visitará el enfermos y consolarán a sus hijos, o vestirán a uno desnudo en honor de vuestros diversos nombres.

\par 24 »A tal persona no le faltarán los bienes de este mundo, ni la vida eterna en el mundo venidero.

\par 25 Además, cualquiera que haya hecho escribir por su propia cuenta las historias de vuestras diversas vidas y sufrimientos, o las haya escrito de su propia mano, o las haya leído con seriedad, o las haya escuchado con fe, o se acordarán de vuestras obras; a tales personas se les perdonarán sus pecados y sus transgresiones, e irán por causa de ti y de tu descendencia al reino de los cielos.

\par 26 »Y ahora levántate, Jacob, porque de las dificultades y del dolor de corazón serás trasladado al descanso eterno, y entrarás en el reposo que no pasará, en la misericordia, en la luz eterna y en el gozo espiritual.

\par 27 Así que ahora declarad a los de vuestra casa y la paz sea con vosotros, porque estoy a punto de ir al que me envió.

\chapter{3}

\par 1 Cuando el ángel hizo estas palabras a nuestro padre Jacob, ascendió de él al cielo, mientras Jacob se despedía de él.

\par 2 Los que estaban alrededor de Jacob lo oyeron mientras daba gracias a Dios y lo glorificaba con alabanzas.

\par 3 Y todos los miembros de su casa, grandes y pequeños, se reunieron a su alrededor, llorando sobre él, profundamente afligidos y diciendo: «Te vas y nos dejas huérfanos».

\par 4 Y ellos seguían diciéndole: «Padre nuestro, amado, ¿qué haremos, ya que estamos en tierra extraña?»

\par 5 Entonces Jacob les dijo: «No temáis; Dios mismo se me apareció en la Alta Mesopotamia y me dijo: 'Yo soy el Dios de tus padres; no temas, porque yo estoy contigo para siempre y con tu descendencia que vendrá después de ti.

\par 6 Esta tierra en la que estás te la daré a ti y a tu descendencia después de ti para siempre.

\par 7 Y no temáis descender a Egipto.

\par 8 Haré para ti un gran pueblo y tu descendencia aumentará y se multiplicará para siempre.

\par 9 José pondrá su mano sobre tus ojos y tu pueblo se multiplicará en la tierra de Egipto.

\par 10 Después vendrán a este lugar y estarán desamparados.

\par 11 Por amor de vosotros les haré bien, aunque por el momento serán desplazados de aquí.'»

\chapter{4}

\par 1 Después de esto, había llegado el momento de que Jacob-Israel dejara su cuerpo.

\par 2 Entonces llamó a José y le dijo: «Si en verdad has hallado gracia, coloca tu bendita mano debajo de mi costado y jura ante el Señor que colocarás mi cuerpo en el sepulcro de mis padres».

\par 3 Entonces José le dijo: «Haré exactamente lo que me mandas, oh amado de Dios».

\par 4 Pero él dijo a José: «Quiero que me lo jures».

\par 5 Entonces José juró a Jacob, su padre, que llevaría su cuerpo al sepulcro de sus padres, y Jacob aceptó el juramento de su hijo.

\par 6 Después, José recibió este informe: «Tu padre se ha puesto inquieto».

\par 7 Entonces tomó a sus dos hijos, Efraín y Manasés, y fue delante de su padre Jacob.

\par 8 José le dijo: Estos son mis hijos que Dios me ha dado en la tierra de Egipto para que vengan después de mí.

\par 9 Israel dijo: «Tráelos más cerca de mí aquí».

\par 10 Porque los ojos de Israel se habían oscurecido a causa de su avanzada edad, de modo que no podía ver.

\par 11 Entonces José acercó a sus hijos y Jacob los besó.

\par 12 Entonces José les ordenó a Efraín y a Manasés que se inclinaran hasta el suelo ante Jacob.

\par 13 José tomó a Manasés y lo puso a la derecha de Israel y a Efraín a su izquierda.

\par 14 Pero Israel invirtió sus manos y puso su mano derecha sobre la cabeza de Efraín y su mano izquierda sobre la cabeza de Manasés.

\par 15 Los bendijo y se los devolvió a su padre, diciendo: «Que el Dios bajo cuya autoridad sirvieron con reverencia mis padres Abraham e Isaac, el Dios que me ha fortalecido desde mi juventud hasta el presente, cuando ángel me ha salvado de todas mis aflicciones, que bendiga a estos muchachos, Manasés y Efraín.

\par 16 Que mi nombre esté sobre ellos, y también los nombres de mis santos padres, Abraham e Isaac.

\par 17 Después de esto, Israel dijo a José: «Yo moriré y todos vosotros volveréis a la tierra de vuestros padres, y Dios estará con vosotros.

\par 18 Y tú personalmente has recibido un gran favor, mayor que el de tus hermanos, porque he tomado esta flecha con mi arco y mi espada de los amorreos(?).

\chapter{5}

\par 1 Entonces Jacob envió a buscar a todos sus hijos y les dijo: «Reúnanse a mi alrededor para que pueda informarles de todo lo que les sobrevendrá y de lo que les sucederá a cada uno de ustedes en los últimos días».

\par 2 Entonces se reunieron alrededor de Israel, desde el mayor hasta el menor.

\par 3 Entonces Jacob Israel habló y dijo a sus hijos: «Escuchen, hijos de Jacob, escuchen a su padre Israel, desde Rubén mi primogénito hasta Benjamín».

\par 4 Entonces les contó lo que les sucedería a los doce niños, llamando a cada uno de ellos y a su tribu por su nombre; y los bendijo con la bendición celestial.

\par 5 Después de esto guardaron silencio por un momento para que descansara.

\par 6 Entonces los cielos se alegraron de que pudiera observar los lugares de reposo.*

\par 7 Y he aquí, se acercaron numerosos verdugos de diferentes apariencias.

\par 8 Estaban preparados para atormentar a los pecadores, que son estos: adúlteros, hombres y mujeres; aquellos que codician a los hombres; los viciosos que degradan el semen dado por Dios; los astrólogos y los hechiceros; los malhechores y los adoradores de ídolos que se aferran a abominaciones; y los calumniadores que juzgan (?) con dos lenguas (engañosamente).

\par 9 Y el castigo de todos estos pecadores es el fuego que no se apaga y las tinieblas exteriores, donde hay llanto y crujir de dientes.

\par 10 [Aquí hay una laguna en el texto árabe. En el bohaírico, Jacob es elevado nuevamente, esta vez al cielo, donde todo es luz y alegría.

\par 11 Él ve a Abraham y a Isaac y se le muestran todas las alegrías de los redimidos.

\par 12 Jacob regresa a la tierra, da instrucciones para su entierro en la tierra de sus padres y fallece a la edad de 147 años.

\par 13 El Señor desciende con los ángeles Miguel y Gabriel para llevar el alma de Jacob al cielo.

\par 14 José ordena que el cuerpo de su padre sea embalsamado a la manera egipcia.

\par 15 Se pasan cuarenta días en el proceso de embalsamamiento, y ochenta días más en luto por el patriarca. ]

\chapter{6}

\par 1 Y cuando terminaron los días de luto, Faraón todavía lloraba por Jacob a causa de su respeto por José.

\par 2 Entonces José se dirigió a los nobles de Faraón y les dijo: «Ya que he hallado favor entre ustedes, ¿hablarán en mi nombre al rey Faraón y le dirán que Jacob me hizo jurar que cuando él fuera ¿Y de su cuerpo sepultaría su cuerpo en el sepulcro de mis padres en la tierra de Canaán, en ese mismo lugar?

\par 3 Entonces Faraón dijo a José: «Ve en paz y entierra a tu padre conforme al juramento que te pidió.

\par 4 Y lleva contigo carros y caballos, lo mejor de mi reino y de mi propia casa, como quieras.

\par 5 Entonces José adoró a Dios en presencia de Faraón, se separó de él y se dispuso a sepultar a su padre.

\par 6 Y partieron con él los siervos de Faraón, los ancianos de Egipto, toda la casa de José, sus hermanos y todo Israel.

\par 7 Todos subieron con él a los carros, y el séquito avanzaba como un gran ejército.

\par 8 Descendieron a la tierra de Canaán, a la orilla del río al otro lado del Jordán, y allí lloraron por él con gran dolor.

\par 9 Durante siete días mantuvieron aquel gran dolor por él.

\par 10 Cuando los habitantes de Dan oyeron el duelo en su tierra, dijeron: «Este gran duelo es el de los egipcios».

\par 11 Hasta el día de hoy [llaman a ese lugar «el duelo de los egipcios»].

\par 12 Entonces Israel fue llevado y sepultado en la tierra de Canaán, en el segundo sepulcro.

\par 13 Éste es el que Abraham había comprado a Efrón, frente a Mamré, con autorización para sepultura.

\par 14 Después de esto, José regresó a la tierra de Egipto con sus hermanos y todo el séquito de Faraón.

\par 15 Y José vivió muchos años después de la muerte de su padre.

\par 16 Él siguió gobernando en Egipto, aunque Jacob había muerto y se había quedado con su propio pueblo.

\chapter{7}

\par 1 Esto es lo que hemos transmitido: Hemos descrito la muerte y el duelo por el padre de padres, Jacob Israel, en la medida de nuestras posibilidades; también como está escrito en los libros espirituales de Dios y como lo hemos encontrado en el antiguo tesoro de la ciencia de nuestros padres, los santos y puros apóstoles.

\par 2 Y si quieres conocer la historia de la vida y adquirir nuevos conocimientos sobre el padre de los padres, Jacob, entonces toma un padre que esté atestiguado en el Antiguo Testamento.

\par 3 Moisés es quien la escribió, el primero de los profetas, el autor de la Ley.

\par 4 Léelo e ilumina tus ideas.

\par 5 Encontrarás esto y más en él, escrito para ti.

\par 6 Descubrirás que Dios y sus ángeles fueron sus amigos mientras estaban en sus cuerpos, y que Dios continuó hablándoles muchas veces en varios pasajes del Libro.

\par 7 También en muchos pasajes del Libro dice del patriarca Jacob, padre de padres, lo siguiente: «Hijo mío, bendeciré a tu descendencia como las estrellas del cielo».

\par 8 Y nuestro padre Jacob hablaba a su hijo José y le decía: «Mi Dios se me apareció en la tierra de Canaán, en Luz, y me bendijo y me dijo: 'Te bendeciré, te multiplicaré y te haré un pueblo poderoso.

\par 9 Saldrán (¿a la guerra?) como las demás naciones de esta tierra y vuestra descendencia aumentará para siempre.' »

\par 10 Esto es lo que hemos oído, hermanos míos y amados, de nuestros padres, los patriarcas.

\par 11 Y nos corresponde a nosotros tener celo por sus obras, su pureza, su fe, su amor por la humanidad y su aceptación de los extraños; para que podamos reclamar ser sus hijos en el reino de los cielos, para que intercedan por nosotros ante Dios para que seamos salvos de las torturas del infierno.

\par 12 Estos son los que los árabes designaron como santos padres.

\par 13 Jacob instruyó a sus hijos sobre el castigo, y los llamaría la espada del Señor, que es el río de fuego, preparado con sus olas para tragar a los malhechores y a los impuros.

\par 14 Estas son las cosas cuyo poder el padre de padres, Jacob, expuso y enseñó a todos sus hijos, para que los sabios pudieran escuchar y perseguir la justicia en amor mutuo con misericordia y compasión.

\par 15 Porque la misericordia salva a los hombres de las penas y la misericordia vence multitud de males.

\par 16 En verdad, el que tiene misericordia de los pobres, ése hace un préstamo a Dios.

\par 17 Así que ahora, mis amados hijos, no dejéis de orar y ayunar en ningún momento, y con la vida de la religión ahuyentaréis a los demonios.

\par 18 Oh, querido hijo, evita los malos caminos del mundo, que son la ira, la depravación y todas las malas acciones.

\par 19 Y guardaos de la injusticia, la blasfemia y el secuestro.

\par 20 Porque los injustos no heredarán el reino de Dios, ni los adúlteros, ni los malditos, ni los que cometen ultrajes y tienen relaciones sexuales con varones, ni los glotones, ni los adoradores de ídolos, ni los que pronuncian imprecaciones. , ni los que se contaminan fuera del puro matrimonio; y otros a quienes no hemos presentado ni siquiera mencionado no se acercarán al reino de Dios.

\par 21 Oh hijos míos, honrad a los santos, porque ellos son los que intercederán por vosotros.

\par 22 Hijos míos, sed generosos con los extranjeros y recibiréis exactamente lo mismo que le fue dado al gran Abraham, padre de padres, y a nuestro padre Isaac, su hijo.

\par 23 Oh hijos míos, haced por los pobres lo que aumente la compasión hacia ellos aquí y ahora, para que Dios os dé el pan de vida para siempre en el reino de Dios.

\par 24 Porque al que da pan a un pobre en este mundo, Dios le dará una porción del árbol de la vida.

\par 25 Vestid al pobre que está desnudo en la tierra, para que Dios os vista con ropas de gloria en el reino de los cielos, y seréis hijos de nuestros santos padres, Abraham, Isaac y Jacob en el cielo. para siempre.

\par 26 Ocúpate de la lectura de la palabra de Dios en sus libros aquí abajo, y recuerda a los santos que han escrito sobre sus vidas, sus sufrimientos y sus postraciones en oración.

\par 27 En el futuro nada impedirá que sean inscritos en el libro de la vida en el reino de los cielos.

\par 28 Y vosotros seréis contados entre los santos, los que agradaron a Dios durante su vida y se regocijarán con los ángeles en la tierra de la vida eterna.

\chapter{8}

\par 1 Honrarás la memoria de nuestros padres, los patriarcas, en esta época cada año y en este mismo día, que es el veintiocho del mes de Misri.

\par 2 Esto es lo que hemos encontrado escrito en los documentos antiguos de nuestros padres, los santos que agradaron a Dios.

\par 3 Por su intercesión y oración tendremos todo, es decir, una parte y un lugar en el reino de los cielos, que pertenece a nuestro Señor, nuestro Dios, nuestro Maestro y nuestro Salvador, Jesús el Mesías.

\par 4 A Él es a quien le pedimos que nos perdone por nuestros errores y nuestros errores y que pase por alto nuestras fechorías.

\par 5 Que él sea bondadoso con nosotros en el día de su juicio y escuchemos la voz llena de gozo, bondad y alegría que dice: Venid a mí, benditos de mi Padre, heredad el reino que era vuestro. desde antes de la creación del mundo.»'

\par 6 Y que seamos dignos de recibir sus secretos divinos, que son el medio para el perdón de nuestros pecados.

\par 7 Que él nos ayude a la salvación de nuestras almas y que nos proteja de los golpes del malvado enemigo.

\par 8 Que él nos permita estar a su diestra en el día grande y terrible, por la intercesión de la maestra de las intercesiones, fuente de pureza, de generosidad y de bendiciones, la madre de la salvación;* y por la intercesión de todos los mártires, santos, hacedores de obras agradables y todos los que han agradado al Señor con sus obras piadosas y su buena voluntad.

\par 9 Amén, amén, amén. Y alabado sea Dios siempre, por los siglos, eternamente.

\end{document}