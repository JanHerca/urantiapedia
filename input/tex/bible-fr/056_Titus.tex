\begin{document}

\title{Épître de Paul à Tite}


\chapter{1}

\par 1 Paul, serviteur de Dieu, et apôtre de Jésus Christ pour la foi des élus de Dieu et la connaissance de la vérité qui est selon la piété, -
\par 2 lesquelles reposent sur l'espérance de la vie éternelle, promise dès les plus anciens temps par le Dieu qui ne ment point,
\par 3 et qui a manifesté sa parole en son temps par la prédication qui m'a été confiée d'après l'ordre de Dieu notre Sauveur, -
\par 4 à Tite, mon enfant légitime en notre commune foi: que la grâce et la paix te soient données de la part de Dieu le Père et de Jésus Christ notre Sauveur!
\par 5 Je t'ai laissé en Crète, afin que tu mettes en ordre ce qui reste à régler, et que, selon mes instructions, tu établisses des anciens dans chaque ville,
\par 6 s'il s'y trouve quelque homme irréprochable, mari d'une seul femme, ayant des enfants fidèles, qui ne soient ni accusés de débauche ni rebelles.
\par 7 Car il faut que l'évêque soit irréprochable, comme économe de Dieu; qu'il ne soit ni arrogant, ni colère, ni adonné au vin, ni violent, ni porté à un gain déshonnête;
\par 8 mais qu'il soit hospitalier, ami des gens de bien, modéré, juste, saint, tempérant,
\par 9 attaché à la vraie parole telle qu'elle a été enseignée, afin d'être capable d'exhorter selon la saine doctrine et de réfuter les contradicteurs.
\par 10 Il y a, en effet, surtout parmi les circoncis, beaucoup de gens rebelles, de vains discoureurs et de séducteurs,
\par 11 auxquels il faut fermer la bouche. Ils bouleversent des familles entières, enseignant pour un gain honteux ce qu'on ne doit pas enseigner.
\par 12 L'un d'entre eux, leur propre prophète, a dit: Crétois toujours menteurs, méchantes bêtes, ventres paresseux.
\par 13 Ce témoignage est vrai. C'est pourquoi reprends-les sévèrement, afin qu'ils aient une foi saine,
\par 14 et qu'ils ne s'attachent pas à des fables judaïques et à des commandements d'hommes qui se détournent de la vérité.
\par 15 Tout est pur pour ceux qui sont purs; mais rien n'est pur pour ceux qui sont souillées et incrédules, leur intelligence et leur conscience sont souillés.
\par 16 Ils font profession de connaître Dieu, mais ils le renient par leurs oeuvres, étant abominables, rebelles, et incapables d'aucune bonne oeuvre.

\chapter{2}

\par 1 Pour toi, dis les choses qui sont conformes à la saine doctrine.
\par 2 Dis que les vieillards doivent être sobres, honnêtes, modérés, sains dans la foi, dans la charité, dans la patience.
\par 3 Dis que les femmes âgées doivent aussi avoir l'extérieur qui convient à la sainteté, n'être ni médisantes, ni adonnées au vin; qu'elles doivent donner de bonnes instructions,
\par 4 dans le but d'apprendre aux jeunes femmes à aimer leurs maris et leurs enfants,
\par 5 à être retenues, chastes, occupées aux soins domestiques, bonnes, soumises à leurs maris, afin que la parole de Dieu ne soit pas blasphémée.
\par 6 Exhorte de même les jeunes gens à être modérés,
\par 7 te montrant toi-même à tous égards un modèle de bonnes oeuvres, et donnant un enseignement pur, digne,
\par 8 une parole saine, irréprochable, afin que l'adversaire soit confus, n'ayant aucun mal à dire de nous.
\par 9 Exhorte les serviteurs à être soumis à leurs maîtres, à leur plaire en toutes choses, à n'être point contredisants,
\par 10 à ne rien dérober, mais à montrer toujours une parfaite fidélité, afin de faire honorer en tout la doctrine de Dieu notre Sauveur.
\par 11 Car la grâce de Dieu, source de salut pour tous les hommes, a été manifestée.
\par 12 Elle nous enseigne à renoncer à l'impiété et aux convoitises mondaines, et à vivre dans le siècle présent selon la sagesse, la justice et la piété,
\par 13 en attendant la bienheureuse espérance, et la manifestation de la gloire du grand Dieu et de notre Sauveur Jésus Christ,
\par 14 qui s'est donné lui-même pour nous, afin de nous racheter de toute iniquité, et de se faire un peuple qui lui appartienne, purifié par lui et zélé pour les bonnes oeuvres.
\par 15 Dis ces choses, exhorte, et reprends, avec une pleine autorité. Que personne ne te méprise.

\chapter{3}

\par 1 Rappelle-leur d'être soumis aux magistrats et aux autorités, d'obéir, d'être prêts à toute bonne oeuvre,
\par 2 de ne médire de personne, d'être pacifiques, modérés, pleins de douceur envers tous les hommes.
\par 3 Car nous aussi, nous étions autrefois insensés, désobéissants, égarés, asservis à toute espèce de convoitises et de voluptés, vivant dans la méchanceté et dans l'envie, dignes d'être haïs, et nous haïssant les uns les autres.
\par 4 Mais, lorsque la bonté de Dieu notre Sauveur et son amour pour les hommes ont été manifestés,
\par 5 il nous a sauvés, non à cause des oeuvres de justice que nous aurions faites, mais selon sa miséricorde, par le baptême de la régénération et le renouvellement du Saint Esprit,
\par 6 qu'il a répandu sur nous avec abondance par Jésus Christ notre Sauveur,
\par 7 afin que, justifiés par sa grâce, nous devenions, en espérance, héritiers de la vie éternelle.
\par 8 Cette parole est certaine, et je veux que tu affirmes ces choses, afin que ceux qui ont cru en Dieu s'appliquent à pratiquer de bonnes oeuvres.
\par 9 Voilà ce qui est bon et utile aux hommes. Mais évite les discussions folles, les généalogies, les querelles, les disputes relatives à la loi; car elles sont inutiles et vaines.
\par 10 Éloigne de toi, après un premier et un second avertissement, celui qui provoque des divisions,
\par 11 sachant qu'un homme de cette espèce est perverti, et qu'il pèche, en se condamnant lui-même.
\par 12 Lorsque je t'enverrai Artémas ou Tychique, hâte-toi de venir me rejoindre à Nicopolis; car c'est là que j'ai résolu de passer l'hiver.
\par 13 Aie soin de pourvoir au voyage de Zénas, le docteur de la loi, et d'Apollos, en sorte que rien ne leur manque.
\par 14 Il faut que les nôtres aussi apprennent à pratiquer de bonnes oeuvres pour subvenir aux besoins pressants, afin qu'ils ne soient pas sans produire des fruits.
\par 15 Tous ceux qui sont avec moi te saluent. Salue ceux qui nous aiment dans la foi. Que la grâce soit avec vous tous!


\end{document}