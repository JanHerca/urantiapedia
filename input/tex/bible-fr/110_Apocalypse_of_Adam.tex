\begin{document}

\title{Apocalypse d'Adam}

\chapter{1}

\par 1 La révélation de l'origine d'Adam racontée à son fils Seth

\par 2 La révélation qu'Adam enseigna à son fils Seth l'année sept centième, disant : Écoute mes paroles, mon fils Seth. Quand Dieu m'avait créé de la terre avec Ève, ta mère, j'allais avec elle dans une gloire qu'elle avait vue dans l'Eon d'où nous étions sortis. Elle m'a appris une parole de connaissance du Dieu éternel. Et nous ressemblions aux grands anges, car nous étions plus élevés que le Dieu qui nous avait créés et les puissances avec lui, que nous ne connaissions pas.

\par 3 Alors Dieu, le souverain des Éons et des puissances, nous a divisés dans sa colère. Ensuite, nous sommes devenus deux Eons. Et la gloire dans nos cœurs nous a quittés, moi et ta mère Eve, avec la première connaissance qui respirait en nous. Et la gloire s'est enfuie de nous ; non pas de cet Éon dont nous étions issus, moi et Ève, votre mère. Mais la connaissance est entrée dans la semence des grands Éons. C'est pourquoi je vous ai moi-même appelé du nom de cet homme qui est la semence de la grande génération ou de qui il vient. Après ces jours-là, la connaissance éternelle du Dieu de vérité s'est retirée de moi et de ta mère Eve. Depuis ce temps-là, nous avons appris les choses mortes, comme les hommes. Alors nous avons reconnu le Dieu qui nous avait créés. Car nous n'étions pas étrangers à ses pouvoirs. Et nous l'avons servi dans la peur et l'esclavage. Et après ces événements, nos cœurs sont devenus obscurcis. Maintenant, je dormais dans les pensées de mon cœur.

\par 4 Et je vis devant moi trois hommes dont je ne pouvais pas reconnaître l'image, car ils n'étaient pas issus des puissances du Dieu qui nous avait créés. Ils ont surpassé la gloire et les hommes, me disant : « Lève-toi, Adam, du sommeil de la mort et entends parler de l'Eon et de la postérité de cet homme à qui la vie est venue, qui est venu de toi et d'Ève, ta femme. »

\par 5 Quand j'eus entendu ces paroles de la part des grands hommes qui se tenaient devant moi, alors nous soupirâmes, moi et Ève, dans nos cœurs. Et le Seigneur, le Dieu qui nous avait créés, se tenait devant nous. Il nous dit , « Adam, pourquoi soupiriez-vous tous les deux dans votre cœur ? Ne sais-tu pas que je suis le Dieu qui t'a créé ? Et je t'ai insufflé un esprit de vie comme une âme vivante. » Puis l'obscurité est venue sur nos yeux.

\par 6 Alors Dieu, qui nous a créés, a créé un fils de lui-même et d'Ève, ta mère. J'ai connu un doux désir pour ta mère. Alors la vigueur de notre connaissance éternelle a été détruite en nous, et la faiblesse nous a poursuivis. C'est pourquoi les jours de notre vie sont devenus peu nombreux, car je savais que j'étais tombé sous l'autorité de la mort.

\par 7 Maintenant donc, mon fils Seth, je vais te révéler les choses que m'ont révélées d'abord les hommes que j'ai vus avant moi, après que j'ai accompli les temps de cette génération et que les années de cette génération aient été accomplies.

\par 8 Car les averses de pluie de Dieu le Tout-Puissant seront déversées afin qu'il puisse détruire toute la chair de Dieu le Tout-Puissant, afin qu'il puisse détruire toute chair de la terre au moyen de ce qui les entoure, ainsi que ceux de la postérité des hommes à qui est passée la vie de la connaissance, qui est venue de moi et d'Ève, ta mère. Car ils lui étaient étrangers. Ensuite, les grands anges viendront sur de hautes nuées, qui amèneront ces hommes dans le lieu où l'esprit de vie habite dans la gloire. Alors toute la multitude de la chair sera abandonnée dans les eaux.

\par 9 Alors Dieu se reposera de sa colère. Et il jettera sa puissance sur les eaux, et donnera puissance à ses fils et à leurs femmes au moyen de l'arche avec les animaux, selon ce qu'il voudra, et les oiseaux du ciel, qu'il a appelé et libéré sur la terre. Et Dieu dira à Noé – que les générations appelleront Deucalion – « Voici, je t’ai protégé dans l’arche avec ta femme et tes fils et leurs femmes et leurs animaux et les oiseaux du ciel, que tu as appelés et relâchés sur le terre. C'est pourquoi je vous donnerai la terre, vous et vos fils. De façon royale, vous la gouvernerez, vous et vos fils. Et aucune semence ne viendra de vous parmi les hommes qui ne se tiendront pas en ma présence dans une autre gloire. .»

\par 10 Alors ils deviendront comme la nuée de la grande lumière. Ces hommes viendront qui ont été chassés de la connaissance du grand Éon et des anges. Ils se tiendront devant Noé et les Éons. Et Dieu dira à Noé , « Pourquoi vous êtes-vous écarté de ce que je vous ai dit ? Vous avez créé une autre génération pour que vous méprisiez mon pouvoir. » Alors Noé dira : « Je témoignerai devant votre puissance que la génération de ces hommes n'est pas venue de moi ni de mes fils. »

\par 11 Et il amènera ces hommes dans leur propre pays et leur bâtira une demeure sainte. Et ils seront appelés de ce nom et y demeureront six cents ans dans une connaissance de l'impérissabilité. Et les anges de la grande Lumière habiteront avec eux. Aucune mauvaise action n'habitera dans leur cœur, mais seulement la connaissance du vrai Dieu.

\par 12 Le Noé partagera la terre entière entre ses fils, Cham et Japhet et Sem. Il leur dira « Mes fils, écoutez mes paroles. Voici, j'ai partagé la terre entre vous. Mais servez-le dans la crainte et l'esclavage tous les jours de votre vie. Que votre postérité ne s'éloigne pas de la face de Dieu tout-puissant. Ma postérité sera agréable devant toi et devant ta puissance. Scelle-le par ta main forte avec crainte et commandement, afin que toute la semence qui est sortie de moi ne s'éloigne pas de toi et de Dieu le Tout-Puissant, mais qu'elle serve dans l'humilité et la crainte de sa connaissance.

\par 13 Puis d'autres, issus de la postérité de Cham et de Japhet, viendront, quatre cent mille hommes, et entreront dans un autre pays et séjourneront avec ces hommes qui sont sortis de la grande connaissance éternelle. Car l'ombre de leur puissance protégera ceux qui ont séjourné avec eux de tout mal et de tout désir impur. Alors la postérité de Cham et de Japhet formera douze royaumes, et leur postérité entrera aussi dans le royaume d'un autre peuple, et prenez conseil auprès des grands éons d'impérissabilité. Et ils iront à Sacla, leur Dieu. Ceux qui entrent dans les puissances accusent les grands hommes qui sont dans leur gloire.

\par 14 Ils diront à Sacla : « Quelle est la puissance de ces hommes qui se tenaient devant toi, qui ont été tirés de la postérité de Cham et de Japhet, et qui seront au nombre de quatre cent mille hommes ? Ils ont été reçus dans un autre éon d’où ils étaient issus, et ils ont bouleversé toute la gloire de ta puissance et la domination de ta main. Car la postérité de Noé, par son fils, a fait toute votre volonté, ainsi que tous les pouvoirs dans les éons sur lesquels règne votre puissance, tandis que ni ces hommes ni ceux qui sont des voyageurs dans leur gloire n'ont pas fait votre volonté. Mais eux j’ai détourné toute ta foule ».

\par 15 Alors le Dieu des Éons leur donnera quelques-uns de ceux qui le servent. Ils viendront dans ce pays où seront les grands hommes qui n'ont pas été souillés et qui ne le seront pas par aucun désir. Car leur âme n'est pas sortie d'une main souillée, mais elle est venue d'un grand commandement de l'ange éternel. Alors le feu, le soufre et l'asphalte seront jetés sur ces hommes, et le feu et la brume aveuglante s'étendront sur ces Éons, et les yeux des puissances des éclaireurs seront obscurcis, et les Éons ne les verront pas en ces jours-là. Et les grands des nuages ​​de lumière descendront, et d’autres nuages ​​de lumière descendront sur eux depuis les grands Eons.

\par 16 Abrasax, Sablo et Gamaliel descendront et feront sortir ces hommes du feu et de la colère, et les emmèneront au-dessus des Éons et des Dirigeants des puissances, et les emmèneront là-bas, avec les saints anges et les Éons. Les hommes seront comme ces anges, car ils ne leur sont pas étrangers. Mais ils travaillent dans la semence impérissable.

\par 17 Une fois de plus, pour la troisième fois, l'illuminateur de la connaissance passera en grande gloire, pour laisser quelque chose de la semence de Noé et des fils de Cham et de Japhet, pour se laisser des arbres fruitiers. Et il rachètera leurs âmes du jour de la mort. Car toute la création issue de la terre morte sera sous l'autorité de la mort. Mais ceux qui réfléchissent à la connaissance du Dieu éternel dans leur cœur ne périront pas. Car ils n'ont pas reçu l'esprit de ce royaume seul, mais ils l'ont reçu d'un des anges éternels. L'illuminateur viendra. Et il accomplira des signes et des prodiges pour mépriser les puissances et leur chef.

\par 18 Alors le Dieu des puissances sera troublé, disant : « Quelle est la puissance de cet homme qui est plus élevé que nous ? Alors il suscitera une grande colère contre cet homme. Et la gloire se retirera et habitera dans les maisons saintes qu'elle s'est choisies. Et les puissances ne la verront pas de leurs yeux, et elles ne verront pas non plus l'illuminateur. Alors elles le feront punissez la chair de l'homme sur lequel le Saint-Esprit est descendu.

\par 19 Alors les anges et toutes les générations des puissances utiliseront le nom avec erreur, en demandant : « D'où vient l'erreur ? ou «D'où viennent les paroles trompeuses, que toutes les puissances n'ont pas réussi à découvrir ?»

\par 20 Or, le premier royaume dit de lui.....[]
\par 21 Il a été nourri dans les cieux.
\par 22 Il reçut la gloire de celui-là et la puissance.
\par 23 Il vint dans le sein de sa mère.
\par 24 Et ainsi il arriva à l'eau.
\par 25 Et le second royaume dit de lui qu'il est issu d'un grand prophète. Et un oiseau vint, prit l'enfant qui était né et l'amena sur une haute montagne. Et il fut nourri par l'oiseau du Ciel. Un Ange sortit de là. Il lui dit : « Lève-toi ! Dieu t’a donné gloire »

\par 26 Il a reçu gloire et force.
\par 27 Et ainsi il arriva à l'eau.
\par 28 Le troisième royaume dit de lui qu'il est sorti d'un sein vierge. Il fut chassé de sa ville, lui et sa mère ; il a été amené dans un endroit désert.

\par 29 Il y fut nourri.
\par 30 Et ainsi il arriva à l'eau.
\par 31 Le quatrième royaume dit de lui qu'il est issu d'une vierge... Salomon la cherchait, lui, Phersalo et Sauel et ses armées qui avaient été envoyées. Salomon lui-même envoya son armée de démons à la recherche de la vierge. Et ils ne trouvèrent pas celle qu'ils cherchaient, mais la vierge qui leur avait été donnée. C'est elle qu'ils allèrent chercher. Salomon la prit. La vierge devint enceinte et y donna naissance à l'enfant.

\par 32 Elle le nourrit au bord du désert. Lorsqu'il eut été nourri, il reçut gloire et puissance de la semence dont il était engendré. Et c'est ainsi qu'il arriva à l'eau.

\par 33 Et le cinquième royaume dit de lui qu'il est sorti d'une goutte du Ciel. Il a été jeté dans la mer. L'Abîme l'a reçu, lui a donné naissance et l'a amené au Ciel.
\par 34 Il a reçu gloire et puissance.
\par 35 Et ainsi il arriva à l'eau.
\par 36 Et le sixième royaume dit que [.....] jusqu'à l'Eon qui est en bas, afin de cueillir des fleurs. Elle devint enceinte du désir des fleurs. Elle lui donna naissance à cet endroit. Les anges du jardin fleuri le nourrirent. Il y reçut gloire et puissance. Et c'est ainsi qu'il arriva à l'eau.

\par 37 Et le septième royaume dit de lui qu'il est une goutte. Elle est venue du Ciel sur la terre. Les dragons l'ont fait descendre dans les grottes. Il est devenu un enfant. Un esprit est venu sur lui et l'a amené en haut à l'endroit où le une goutte était sortie. Il y reçut gloire et puissance. Et c'est ainsi qu'il arriva à l'eau.

\par 38 Et le huitième royaume dit de lui qu'une nuée vint sur la terre et enveloppa un rocher. Il en sortit. Les anges qui étaient au-dessus de la nuée le nourrissèrent. Il y reçut gloire et puissance. Et c'est ainsi qu'il arriva à l'eau.

\par 39 Et le neuvième royaume dit de lui que des neuf Muses une se sépara. Elle arriva sur une haute montagne et y passa quelque temps assise, de sorte qu'elle se désira seule pour devenir androgyne. Elle réalisa son désir et devint enceinte de son désir. Il est né. Les anges qui étaient au-dessus du désir l'ont nourri. Il y reçut gloire et puissance.
\par 40 Et ainsi il arriva à l'eau.
\par 41 Le dixième royaume dit de lui que son dieu aimait un nuage de désir. Il engendra dans sa main et jeta une partie de la goutte sur la nuée au-dessus de lui, et il naquit. Il y reçut gloire et puissance.
\par 42 Et ainsi il arriva à l'eau.
\par 43 Le Onzième royaume dit de lui que le père désirait sa propre fille. Elle devint enceinte de son père. Elle jeta [....] le tombeau dans le désert. L'ange l'y a nourri.
\par 44 Et ainsi il arriva à l'eau.

\par 45 Et le douzième royaume dit de lui qu'il est issu de deux enlumineurs. Il y était nourri.
\par 46 Il a reçu gloire et puissance.
\par 47 Et ainsi il arriva à l'eau.

\par 48 Et le treizième royaume dit de lui que chaque naissance de leur chef est une parole. Et cette parole y reçut un mandat. Il a reçu gloire et puissance.
\par 49 Et ainsi il arriva à l'eau.
\par 50 Mais la génération sans roi sur elle dit que Dieu l'a choisi parmi tous les éons. Il a fait naître en lui la connaissance de la vérité intacte. Il a dit : « D’un air étranger, d’un grand Eon, le grand illuminateur est sorti. Et il a fait briller la génération de ces hommes qu’il s’était choisis, afin qu’ils brillent sur tout l’Eon. »

\par 51 Alors la semence, ceux qui recevront son nom sur l'eau et celui d'eux tous, combattront contre la puissance. Et un nuage de ténèbres viendra sur eux.

\par 52 Alors les peuples crieront à haute voix, disant : Bienheureuse est l'âme de ces hommes, car ils ont connu Dieu avec la connaissance de la vérité ! Ils vivront éternellement, parce qu'ils n'ont pas été corrompus par leur désir, comme les anges, et qu'ils n'ont pas non plus accompli les œuvres des puissances, mais qu'ils se sont tenus en sa présence dans une connaissance de Dieu comme une lumière qui sort du feu et du sang. Mais nous avons commis tous les actes des puissances de manière insensée. Nous nous sommes vantés de la transgression de toutes nos œuvres. Nous avons crié contre le Dieu de vérité parce que toutes ses œuvres sont éternelles. Celles-ci sont contre nos esprits. Car maintenant nous savons que nos âmes mourront de mort.

\par 53 Alors une voix leur parvint, disant : « Micheu, Michar et Mnesinous, qui sont au-dessus du saint baptême et de l'eau vive, pourquoi criiez-vous contre le Dieu vivant avec des voix et des langues sans loi sur eux, et sur les âmes ? plein de sang et d'actes ignobles ? Vous êtes plein d'œuvres qui ne sont pas de la vérité, mais vos voies sont pleines de joie et d'allégresse. Après avoir souillé l'eau de la vie, vous l'avez puisée selon la volonté des puissances à qui vous avez été donné pour les servir. Et votre pensée n'est pas comme celle de ces hommes que vous persécutez. Leurs fruits ne se fanent pas. Mais ils seront connus jusqu'aux grands Eons, parce que les paroles qu'ils avaient gardées, du Dieu des Eons, n'étaient pas confiées aux livre, et ils n’étaient pas non plus écrits. Mais des êtres angéliques les amèneront, que toutes les générations d’hommes ne connaîtront pas. Car ils seront sur la haute montagne, sur le rocher de la vérité. C'est pourquoi ils seront appelés « paroles d'impérissabilité et de vérité » pour ceux qui connaissent le Dieu éternel dans la sagesse, la connaissance et l'enseignement des anges pour toujours, car il connaît toutes choses. .»

\par 54 Ce sont les révélations qu'Adam fit connaître à Seth son fils, et son fils les enseigna à sa postérité. C'est la connaissance cachée d'Adam, qu'il a donnée à Seth, qui est le saint baptême de ceux qui connaissent la connaissance éternelle à travers ceux qui sont nés de la parole et les éclaireurs impérissables, issus de la semence sainte : Yessée. Mazareus Yessedekeus, l'eau vive.


\end{document}