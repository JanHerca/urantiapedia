\begin{document}

\title{Testament d'Asher}

\chapter{1}

\par \textit{Aser, le dixième fils de Jacob et de Zilpa. Une explication de la double personnalité. La première histoire de Jekyll et Hyde. Pour une déclaration sur la Loi de Compensation dont Emerson aurait bénéficié, voir le verset 27.}

\par 1 LA copie du Testament à Aser, ce qu'il dit à ses fils dans la cent vingt-cinquième année de sa vie.

\par 2 Car alors qu'il était encore en bonne santé, il leur dit : Écoutez, enfants d'Aser, votre père, et je vous déclarerai tout ce qui est droit aux yeux de l'Éternel.

\par 3 Dieu a donné deux voies aux fils des hommes, et deux inclinations, et deux sortes d'action, et deux manières d'agir, et deux issues.

\par 4 C'est pourquoi toutes choses sont par deux, l'une contre l'autre.

\par 5 Car il y a deux voies du bien et du mal, et avec celles-ci sont les deux inclinations de nos poitrines qui les distinguent.

\par 6 C'est pourquoi si l'âme prend plaisir au bon penchant, toutes ses actions sont dans la justice ; et s'il pèche, il se repent aussitôt.

\par 7 Car, ayant ses pensées axées sur la justice et rejetant la méchanceté, il renverse aussitôt le mal et déracine le péché.

\par 8 Mais s'il penche vers le mauvais penchant, toutes ses actions sont mauvaises, et il chasse le bien, et s'attache au mal, et est gouverné par Beliar ; même si cela produit ce qui est bien, il le pervertit en mal.

\par 9 Car chaque fois qu'il commence à faire le bien, il force l'issue de l'action au mal pour lui, voyant que le trésor de l'inclination est rempli d'un mauvais esprit.

\par 10 Une personne peut donc, par ses paroles, aider le bien pour le mal, mais l'issue de son action conduit au mal.

\par 11 Il y a un homme qui n'a aucune compassion envers celui qui sert son tour dans le mal ; et cette chose a deux aspects, mais le tout est mauvais.

\par 12 Et il y a un homme qui aime celui qui fait le mal, parce qu'il préférerait même mourir dans le mal à cause de lui ; et à ce sujet, il est clair qu’il y a deux aspects, mais que l’ensemble est une mauvaise œuvre.

\par 13 Même s'il a de l'amour, est méchant celui qui cache ce qui est mal à cause de la bonne réputation, mais la fin de l'action tend au mal.

\par 14 Un autre vole, commet l'injustice, pille, escroque, et en même temps il a pitié des pauvres : cela aussi a un double aspect, mais le tout est mauvais.

\par 15 Celui qui fraude son prochain provoque Dieu et jure faussement contre le Très-Haut, et pourtant il a pitié des pauvres ; le Seigneur qui a commandé la loi, il néglige et provoque, et pourtant il réconforte les pauvres.

\par 16 Il souille l'âme et illumine le corps ; il tue beaucoup et a pitié de quelques-uns : cela aussi a un double aspect, mais le tout est mauvais.

\par 17 Un autre commet l'adultère et la fornication, et s'abstient de viandes, et quand il jeûne, il fait le mal, et par la puissance de sa richesse en accable beaucoup ; et malgré sa méchanceté excessive, il exécute les commandements : cela aussi a un double aspect, mais le tout est mauvais.

\par 18 De tels hommes sont des lièvres ; purs, comme ceux qui ont le sabot divisé, mais qui sont impurs en fait.

\par 19 Car Dieu a ainsi déclaré dans les tables des commandements.

\par 20 Mais vous, mes enfants, ne portez pas deux visages comme eux, de bonté et de méchanceté ; mais attachez-vous uniquement au bien, car Dieu y a sa demeure, et les hommes le désirent.

\par 21 Mais fuyez la méchanceté, détruisant le mauvais penchant par vos bonnes œuvres ; car ceux qui ont un double visage ne servent pas Dieu, mais leurs propres convoitises, afin de plaire à Beliar et aux hommes qui leur ressemblent.

\par 22 Car les hommes de bien, même ceux qui ont un seul visage, même s'ils sont considérés par ceux qui ont un double visage face au péché, sont justes devant Dieu.

\par 23 Car beaucoup, en tuant les méchants, font deux œuvres, une bonne et une mauvaise ; mais le tout est bon, parce qu'il a déraciné et détruit ce qui est mal.

\par 24 Un homme hait l'homme miséricordieux et injuste, et l'homme qui commet l'adultère et jeûne : cela aussi a un double aspect, mais toute l'œuvre est bonne, parce qu'il suit l'exemple du Seigneur, en ce sens qu'il n'accepte pas le paraître bon comme le bien véritable.

\par 25 Un autre ne désire pas voir le bon jour avec ceux qui ne le font pas, de peur de souiller son corps et de souiller son âme ; là aussi, c'est double face, mais l'ensemble est bon.

\par 26 Car de tels hommes sont semblables aux cerfs et aux biches, parce qu'ils semblent impurs à la manière des animaux sauvages, mais ils sont tout à fait purs ; parce qu'ils marchent avec zèle pour le Seigneur et s'abstiennent de ce que Dieu hait et interdit par ses commandements, éloignant le mal du bien.

\par 27 Vous voyez, mes enfants, qu'en toutes choses il y en a deux, l'un contre l'autre, et l'un est caché par l'autre : dans la richesse se cache la convoitise, dans la convivialité l'ivresse, dans le rire la douleur, dans la débauche conjugale.

\par 28 La mort succède à la vie, le déshonneur à la gloire, la nuit au jour, et les ténèbres à la lumière ; et toutes choses sont sous le jour, les choses justes sous la vie, les choses injustes sous la mort ; c'est pourquoi aussi la vie éternelle attend la mort.

\par 29 On ne peut pas non plus dire que la vérité est un mensonge, ni que le bien est un tort ; car toute vérité est sous la lumière, comme toutes choses sont sous Dieu.

\par 30 J'ai donc éprouvé toutes ces choses dans ma vie, et je ne me suis pas égaré loin de la vérité du Seigneur, et j'ai sondé les commandements du Très-Haut, marchant de toutes mes forces, avec un visage simple, vers ce qui est bon.

\par 31 Prenez donc garde, vous aussi, mes enfants, aux commandements du Seigneur, en suivant la vérité avec simplicité de visage.

\par 32 Car ceux qui ont un double visage sont coupables d'un double péché ; car ils font tous deux le mal et ils prennent plaisir à ceux qui le font, suivant l'exemple des esprits de tromperie et luttant contre les hommes.

\par 33 Gardez donc, mes enfants, la loi du Seigneur, et ne prêtez pas attention au mal comme au bien ; mais regardez ce qui est vraiment bon, et observez-le dans tous les commandements du Seigneur, en ayant votre conversation et en vous y reposant.

\par 34 Car les dernières fins des hommes montrent leur justice ou leur injustice lorsqu'ils rencontrent les anges du Seigneur et de Satan.

\par 35 Car lorsque l'âme s'en va troublée, elle est tourmentée par le mauvais esprit qu'elle a aussi servi dans ses convoitises et ses mauvaises œuvres.

\par 36 Mais s'il est paisible dans la joie, il rencontre l'ange de paix, et il le conduit à la vie éternelle.

\par 37 Ne devenez pas, mes enfants, comme Sodome, qui a péché contre les anges du Seigneur et qui a péri pour toujours.

\par 38 Car je sais que vous pécherez et que vous serez livrés entre les mains de vos ennemis ; et votre pays sera désolé, et vos lieux saints détruits, et vous serez dispersés aux quatre coins de la terre.

\par 39 Et vous serez réduits à néant dans la dispersion qui disparaîtra comme l'eau.

\par 40 Jusqu'à ce que le Très-Haut visite la terre, venant lui-même comme un homme, avec des hommes mangeant et buvant, et brisant la tête du dragon dans l'eau.

\par 41 Il sauvera Israël et tous les païens, Dieu parlant en la personne de l'homme.

\par 42 C'est pourquoi vous aussi, mes enfants, dites ces choses à vos enfants, afin qu'ils ne lui désobéissent pas.

\par 43 Car je sais que vous serez assurément désobéissants et que vous agirez assurément de manière impie, ne prêtant pas attention à la loi de Dieu, mais aux commandements des hommes, étant corrompus par la méchanceté.

\par 44 Et c'est pourquoi vous serez dispersés comme Gad et Dan mes frères, et vous ne connaîtrez ni vos pays, ni votre tribu, ni votre langue.

\par 45 Mais le Seigneur vous rassemblera dans la foi par sa tendre miséricorde et à cause d'Abraham, d'Isaac et de Jacob.

\par 46 Et après leur avoir dit ces choses, il leur commanda, disant : Enterrez-moi à Hébron.

\par 47 Et il s'endormit et mourut dans une bonne vieillesse.

\par 48 Et ses fils firent ce qu'il leur avait ordonné, et ils le transportèrent à Hébron, et l'enterrèrent avec ses pères.

\end{document}