\begin{document}

\title{L'histoire d'Ahikar}

\chapter{1}

\par \textit{Ahikar, grand vizir d'Assyrie, a 60 femmes mais est destiné à ne pas avoir de fils. Il adopte donc son neveu. Il le remplit de sagesse et de connaissance plus que de pain et d'eau.}

\par 1 L'histoire de Haiqâr le Sage, vizir de Sennachérib le Roi, et de Nadan, fils de la sœur de Haiqâr le Sage.

\par 2 Il y avait un vizir du temps du roi Sennachérib, fils de Sarhadum, roi d'Assyrie et de Ninive, un sage nommé Haiqâr, et il était vizir du roi Sennachérib.

\par 3 Il avait une amende, une fortune et beaucoup de biens, et il était habile, sage, philosophe, en connaissance, en opinion et en gouvernement, et il avait épousé soixante femmes, et avait bâti un château pour chacune d'elles.

\par 4 Mais avec tout cela, il n’eut d’aucune de ces femmes aucun enfant qui pourrait être son héritier.

\par 5 Et il en fut très triste, et un jour il rassembla les astrologues, les savants et les sorciers et leur expliqua sa condition et le problème de sa stérilité.

\par 6 Et ils lui dirent : « Va, sacrifie aux dieux et supplie-les qu'ils te donnent peut-être un garçon. »

\par 7 Et il fit ce qu'ils lui disaient et offrit des sacrifices aux idoles, et les supplia et les implora avec requête et supplication.

\par 8 Et ils ne lui répondirent pas un mot. Et il s'en alla triste et abattu, partant avec une douleur au cœur.

\par 9 Et il revint et implora le Dieu Très-Haut, et crut, le suppliant avec un cœur brûlant, disant : « Ô Dieu Très-Haut, ô Créateur des Cieux et de la terre, ô Créateur de toutes choses créées. !'

\par 10 'Je te supplie de me donner un garçon, afin que je sois consolé par lui, qu'il soit présent à ma bruyère, qu'il me ferme les yeux et qu'il m'enterre.'

\par 11 Alors une voix lui parvint et lui dit : Puisque tu t'es appuyé d'abord sur des images taillées et que tu leur as offert des sacrifices, c'est pour cette raison que tu resteras sans enfants toute ta vie.

\par 12 'Mais prends Nadan le fils de ta sœur, et fais-en ton enfant et apprends-lui ton savoir et ta bonne éducation, et à ta mort il t'enterrera.'

\par 13 Alors il prit Nadan, le fils de sa sœur, qui était un petit nourrisson. Et il le confia à huit nourrices, afin qu'elles l'allaitent et l'élèvent.

\par 14 Et ils l'élevèrent avec une bonne nourriture, une éducation douce, des vêtements de soie, de pourpre et de cramoisi. Et il était assis sur des canapés de soie.

\par 15 Et quand Nadan devint grand et marcha, s'élevant comme un grand cèdre, il lui apprit les bonnes manières, l'écriture, la science et la philosophie.

\par 16 Et après plusieurs jours, le roi Sennachérib regarda Haiqâr et vit qu'il était devenu très vieux, et en plus il lui dit :

\par 17 'Ô mon honoré ami, l'habile, le fidèle, le sage, le gouverneur, mon secrétaire, mon vizir, mon chancelier et directeur ; En vérité, tu es devenu très vieux et alourdi par les années ; et ton départ de ce monde doit être proche.

\par 18 « Dis-moi qui aura une place à mon service après toi. » Et Haiqâr lui dit : « Ô mon seigneur, que ta tête vive pour toujours ! Voilà Nadan, le fils de ma sœur, j'en ai fait mon enfant.

\par 19 'Et je l'ai élevé et je lui ai enseigné ma sagesse et ma connaissance.'

\par 20 Et le roi lui dit : 'Ô Haiqâr ! amène-le devant moi, afin que je le voie, et si je le trouve convenable, mets-le à ta place ; et tu partiras pour te reposer et vivre le reste de ta vie dans un doux repos.

\par 21 Alors Haiqâr alla présenter Nadan le fils de sa sœur. Et il lui rendit hommage et lui souhaita pouvoir et honneur.

\par 22 Et il le regarda et l'admira et se réjouit en lui et dit à Haiqâr : «Est-ce là ton fils, ô Haiqâr ? Je prie pour que Dieu le préserve. Et comme tu m'as servi moi et mon père Sarhadum, que ton garçon me serve et accomplisse mes engagements, mes besoins et mes affaires, afin que je puisse l'honorer et le rendre puissant pour toi.»

\par 23 Et Haiqâr rendit hommage au roi et lui dit : « Que ta tête vive, ô mon seigneur le roi, pour toujours ! Je demande à toi d'être patient avec mon garçon Nadan et de lui pardonner ses erreurs afin qu'il puisse te servir comme il convient.

\par 24 Alors le roi lui jura qu'il ferait de lui le plus grand de ses favoris et le plus puissant de ses amis, et qu'il serait avec lui en tout honneur et respect. Et il lui baisa les mains et lui dit adieu.

\par 25 Et il prit Nadan le fils de sa sœur avec lui, il l'assit dans un salon et se mit à l'instruire nuit et jour jusqu'à ce qu'il l'ait bourré de sagesse et de connaissance plus que de pain et d'eau.



\chapter{2}

\par \textit{Un « Almanach du pauvre Richard » des temps anciens. Préceptes immortels de conduite humaine concernant l'argent, les femmes, la tenue vestimentaire, les affaires, les amis. Des proverbes particulièrement intéressants se trouvent dans les versets 12, 17, 23, 37, 45, 47. Comparez le verset 63 avec certains cynismes d'aujourd'hui.}

\par 1 AINSI il lui enseigna, disant : « Ô mon fils ! écoutez mon discours, suivez mes conseils et souvenez-vous de ce que je dis.

\par 2 «Ô mon fils ! Si tu entends une parole, qu'elle meure dans ton cœur et ne la révèle pas à un autre, de peur qu'elle ne devienne une braise ardente et ne brûle ta langue et ne cause une douleur dans ton corps, et que tu n'attires l'opprobre et que tu sois honteux devant Dieu et homme.»

\par 3 «Ô mon fils ! si tu as entendu un bruit, ne le répands pas ; et si tu as vu quelque chose, ne le dis pas.»

\par 4 «Ô mon fils ! rends ton éloquence facile à celui qui t'écoute, et ne te hâte pas de répondre.»

\par 5 «Ô mon fils ! quand tu as entendu quelque chose, ne le cache pas.»

\par 6 «Ô mon fils ! ne dénouez pas un nœud scellé, ne le dénouez pas, et ne scellez pas un nœud desserré.»

\par 7 «Ô mon fils ! ne convoitez pas la beauté extérieure, car elle décroît et disparaît, mais un souvenir honorable dure pour toujours.»

\par 8 «Ô mon fils ! qu'une femme stupide ne te trompe pas par son discours, de peur que tu ne meures de la plus misérable des morts, et qu'elle ne t'emmêle dans le filet jusqu'à ce que tu sois pris au piège.»

\par 9 «Ô mon fils ! ne désire pas une femme parée de vêtements et de onguents, qui est méprisable et sotte dans son âme. Malheur à toi si tu lui accordes quelque chose qui est à toi, ou si tu lui confies ce qui est entre tes mains et qu'elle t'entraîne au péché, et que Dieu soit en colère contre toi.»

\par 10 «Ô mon fils ! ne soyez pas comme l'amandier, car il produit des feuilles avant tous les arbres et des fruits comestibles après tous, mais soyez comme le mûrier, qui produit des fruits comestibles avant tous les arbres et des feuilles après tous.»

\par 11 «Ô mon fils ! baisse ta tête, adoucis ta voix, sois courtois, marche dans le droit chemin et ne sois pas insensé. Et n'élève pas la voix quand tu ris, car si c'était à haute voix qu'on bâtissait une maison, l'âne bâtirait plusieurs maisons chaque jour ; et si c'était à force de force qu'on conduisait la charrue, la charrue ne serait jamais retirée de sous les épaules des chameaux.»

\par 12 «Ô mon fils ! Mieux vaut enlever des pierres avec un homme sage que boire du vin avec un homme malheureux.»

\par 13 «Ô mon fils ! verse ton vin sur les tombeaux des justes, et ne bois pas avec des gens ignorants et méprisables.»

\par 14 «Ô mon fils ! attache-toi aux sages qui craignent Dieu et qui leur ressemblent, et ne t'approche pas des ignorants, de peur que tu ne deviennes comme lui et que tu n'apprennes ses voies.»

\par 15 «Ô mon fils ! quand tu t'es trouvé un camarade ou un ami, essaye-le, et fais-en ensuite un camarade et un ami ; et ne le louez pas sans épreuve ; et ne gâche pas ton discours avec un homme qui manque de sagesse.»

\par 16 «Ô mon fils ! pendant qu'une chaussure reste à ton pied, marche avec elle sur les épines, et fais un chemin pour ton fils, et pour ta maison et tes enfants, et tends ton navire avant qu'il ne s'en aille sur la mer et ses vagues et qu'il ne coule et qu'il ne puisse pas enregistré.»

\par 17 «Ô mon fils ! si le riche mange un serpent, on dit : « C'est par sa sagesse », et si un pauvre en mange, les gens disent : « C'est à cause de sa faim. »»

\par 18 «Ô mon fils ! il se contente de ton pain quotidien et de tes biens, et ne convoite pas ce qui appartient à autrui.»

\par 19 «Ô mon fils ! ne sois pas le voisin de l'insensé, ne mange pas de pain avec lui, et ne te réjouis pas des calamités de ton prochain. 1 Si ton ennemi te fait du tort, montre-lui de la bonté.»

\par 20 «Ô mon fils ! Un homme qui craint Dieu, crains-le et honore-le.»

\par 21 «Ô mon fils ! l'ignorant tombe et trébuche, et le sage, même s'il trébuche, ne se laisse pas ébranler, et même s'il tombe, il se relève vite, et s'il est malade, il peut prendre soin de sa vie. Mais quant à l’homme ignorant et stupide, il n’existe pas de médicament contre sa maladie.»

\par 22 «Ô mon fils ! Si quelqu'un qui t'est inférieur s'approche de toi, avance à sa rencontre et reste debout. S'il ne peut pas te récompenser, son Seigneur te récompensera à sa place.»

\par 23 «Ô mon fils ! N'épargne pas de battre ton fils, car la raclée de ton fils est comme du fumier dans le jardin, et comme attacher l'ouverture d'une bourse, et comme attacher des bêtes, et comme verrouiller une porte.»

\par 24 «Ô mon fils ! retiens ton fils de la méchanceté et enseigne-lui les bonnes manières avant qu'il ne se rebelle contre toi et ne te fasse mépriser parmi le peuple et que tu baisses la tête dans les rues et les assemblées et que tu sois puni pour la méchanceté de ses mauvaises actions.»

\par 25 «Ô mon fils ! prends-toi un gros bœuf avec un prépuce et un âne gros avec de grandes cornes, et ne te procure pas un bœuf avec de grandes cornes, ne te lie pas d'amitié avec un homme rusé, ni ne prends un esclave querelleur, ni une servante voleuse, pour tout ce que tu leur confies, ils le ruineront.»

\par 26 «Ô mon fils ! que tes parents ne te maudissent pas, et que l'Éternel soit satisfait d'eux ; car il a été dit : « Celui qui méprise son père ou sa mère, qu'il meure de la mort (je veux dire de la mort du péché) ; et celui qui honore ses parents prolongera ses jours et sa vie et verra tout ce qui est bon.»

\par 27 «Ô mon fils ! ne marche pas sur la route sans armes, car tu ne sais pas quand l'ennemi peut te rencontrer, afin que tu sois prêt à l'affronter.»

\par 28 «Ô mon fils ! ne soyez pas comme un arbre nu, sans feuilles, qui ne pousse pas, mais soyez comme un arbre couvert de ses feuilles et de ses branches ; car l'homme qui n'a ni femme ni enfants est déshonoré dans le monde et est haï par eux, comme un arbre sans feuilles et sans fruit.»

\par 29 «Ô mon fils ! soyez comme un arbre fruitier au bord du chemin, dont les fruits sont mangés par tous les passants, et les bêtes du désert se reposent sous son ombre et mangent de ses feuilles.»

\par 30 «Ô mon fils ! chaque brebis qui s'éloigne de son chemin et ses compagnons deviennent de la nourriture pour le loup.»

\par 31 «Ô mon fils ! ne dis pas : « Mon seigneur est un insensé et je suis sage », et ne raconte pas des discours d'ignorance et de folie, de peur d'être méprisé par lui.»

\par 32 «Ô mon fils ! ne soyez pas de ces serviteurs à qui leurs maîtres disent : « Éloignez-vous de nous », mais soyez de ceux à qui ils disent : « Approchez-vous et approchez-vous de nous. »»

\par 33 «Ô mon fils ! ne caresse pas ton esclave en présence de son compagnon, car tu ne sais pas lequel d'entre eux aura le plus de valeur pour toi à la fin.»

\par 34 «Ô mon fils ! n'aie pas peur de ton Seigneur qui t'a créé, de peur qu'il ne te taise.»

\par 35 «Ô mon fils ! rends ton discours beau et adoucis ta langue ; et ne permets pas à ton compagnon de marcher sur ton pied, de peur qu'il ne marche plus tard sur ta poitrine.»

\par 36 'Ô mon fils ! si tu frappes un sage avec une parole de sagesse, elle se cachera dans sa poitrine comme un subtil sentiment de honte ; mais si tu frappes l'ignorant avec un bâton, il ne comprendra ni n'entendra.

\par 37 'Ô mon fils ! Si tu envoies un homme sage pour tes besoins, ne lui donne pas beaucoup d'ordres, car il fera tes affaires comme tu le désires ; et si tu envoies un insensé, ne lui donne pas d'ordres, mais va toi-même et fais tes affaires, car si tu ordonne-lui, il ne fera pas ce que tu désires. S'ils t'envoient pour une affaire, dépêche-toi de l'accomplir au plus vite.

\par 38 «Ô mon fils ! Ne te fais pas l'ennemi d'un homme plus fort que toi, car il prendra ta mesure et se vengera de toi.»

\par 39 «Ô mon fils ! fais l'épreuve de ton fils et de ton serviteur, avant de leur confier tes biens, de peur qu'ils ne s'en emparent ; car celui qui a la main pleine est appelé sage, même s'il est stupide et ignorant, et celui qui a la main vide est appelé pauvre, ignorant, même s'il est le prince des sages.»

\par 40 «Ô mon fils ! J'ai mangé une coloquinte et avalé de l'aloès, et je n'ai rien trouvé de plus amer que la pauvreté et la disette.»

\par 41 «Ô mon fils ! enseigne à ton fils la frugalité et la faim, afin qu'il puisse bien gérer sa maison.»

\par 42 «Ô mon fils ! n'enseigne pas à l'ignorant le langage des sages, car cela lui serait à charge.»

\par 43 «Ô mon fils ! ne montre pas ta condition à ton ami, de peur d'être méprisé par lui.»

\par 44 «Ô mon fils ! la cécité du cœur est plus grave que la cécité des yeux, car la cécité des yeux peut être guidée petit à petit, mais la cécité du cœur n'est pas guidée, et il quitte le chemin droit et va dans un chemin tortueux.»

\par 45 «Ô mon fils ! Mieux vaut la chute d'un homme avec son pied que la chute d'un homme avec sa langue.»

\par 46 «Ô mon fils ! un ami proche vaut mieux qu'un frère plus excellent qui est loin.»

\par 47 «Ô mon fils ! la beauté s'estompe mais l'apprentissage dure, et le monde décroît et devient vain, mais une bonne réputation ne devient ni vaine ni ne décroît.»

\par 48 «Ô mon fils ! pour l'homme qui n'a pas de repos, sa mort vaut mieux que sa vie ; et le son des pleurs vaut mieux que le son du chant ; car la tristesse et les pleurs, si la crainte de Dieu est en eux, valent mieux que le son des chants et de la réjouissance.»

\par 49 «Ô mon enfant ! la cuisse d'une grenouille dans ta main vaut mieux qu'une oie dans la marmite de ton prochain ; et une brebis près de toi vaut mieux qu'un bœuf loin de là ; et un moineau dans ta main vaut mieux que mille moineaux qui volent ; 1 et la pauvreté qui rassemble vaut mieux que la dispersion de beaucoup de provisions ; et un renard vivant vaut mieux qu'un lion mort ; et une livre de laine vaut mieux qu'une livre de richesse, je veux dire d'or et d'argent ; car l'or et l'argent sont cachés et recouverts dans la terre, et on ne les voit pas ; mais la laine reste sur les marchés et on la voit, et c'est une beauté pour celui qui la porte.»

\par 50 «Ô mon fils ! une petite fortune vaut mieux qu'une fortune dispersée.»

\par 51 «Ô mon fils ! un chien vivant vaut mieux qu'un pauvre homme mort.»

\par 52 «Ô mon fils ! Mieux vaut un pauvre qui fait le bien qu'un riche qui est mort dans ses péchés.»

\par 53 «Ô mon fils ! garde une parole dans ton cœur, et ce sera beaucoup pour toi, et prends garde de ne pas révéler le secret de ton ami.»

\par 54 «Ô mon fils ! qu'aucune parole ne sorte de ta bouche avant d'avoir pris conseil avec ton cœur. Et ne te tiens pas entre des personnes qui se disputent, car d'une mauvaise parole naît une querelle, et d'une querelle naît la guerre, et de la guerre naît le combat, et tu seras obligé d'en rendre témoignage ; mais fuyez de là et reposez-vous.»

\par 55 «Ô mon fils ! Ne résiste pas à un homme plus fort que toi, mais procure-toi un esprit patient, de l'endurance et une conduite droite, car il n'y a rien de plus excellent que cela.»

\par 56 «Ô mon fils ! ne déteste pas ton premier ami, car le second pourrait ne pas durer.»

\par 57 «Ô mon fils ! visitez le pauvre dans son affliction, parlez de lui en présence du sultan, et faites votre diligence pour le sauver de la gueule du lion.»

\par 58 «Ô mon fils ! ne te réjouis pas de la mort de ton ennemi, car dans peu de temps tu seras son prochain, et celui qui se moque de toi, respecte et honore et sois en avance avec lui pour le saluer.»

\par 59 «Ô mon fils ! si l'eau s'arrêtait dans le ciel, si un corbeau noir devenait blanc et si la myrrhe devenait douce comme du miel, alors les hommes ignorants et les insensés pourraient comprendre et devenir sages.»

\par 60 «Ô mon fils ! si tu veux être sage, empêche ta langue de mentir, et ta main de voler, et tes yeux de voir le mal ; alors tu seras appelé sage.»

\par 61 «Ô mon fils ! que le sage te frappe avec une verge, mais que l'insensé ne t'oigne pas d'un baume doux. Sois humble dans ta jeunesse et tu seras honoré dans ta vieillesse.»

\par 62 «Ô mon fils ! ne résistez à aucun homme aux jours de sa puissance, ni à un fleuve aux jours de son déluge.»

\par 63 «Ô mon fils ! ne vous précipitez pas dans le mariage d'une femme, car si tout se passe bien, elle dira : « Monseigneur, prends soin de moi » ; et si cela tourne mal, elle s'en prendra à celui qui en est la cause.»

\par 64 «Ô mon fils ! quiconque est élégant dans sa tenue, il l'est également dans son discours ; et celui qui a une apparence mesquine dans son vêtement, il est aussi le même dans son discours.»

\par 65 «Ô mon fils ! si tu as commis un vol, fais-le savoir au sultan et donne-lui une part, afin que tu sois délivré de lui, car autrement tu souffrirais de l'amertume.»

\par 66 «Ô mon fils ! faites-vous l'ami de l'homme dont la main est rassasiée et remplie, et ne faites pas l'ami de l'homme dont la main est fermée et affamée.»

\par 67 « Il y a quatre choses dans lesquelles ni le roi ni son armée ne peuvent être en sécurité : l'oppression du vizir, et le mauvais gouvernement, et la perversion de la volonté, et la tyrannie sur le sujet ; et quatre choses qui ne peuvent être cachées : les prudents, les insensés, les riches et les pauvres.»

\par \textit{Notes de bas de page}

\par \textit{201:1 Cf. Psaumes CXLI. 4.}

\par \textit{203:1 Cf. «Un tien vaut mieux que deux tu l'auras.»}

\par \textit{203:2 Cf. 2 Timothée, IV, 17.}

\chapter{3}

\par \textit{Ahikar se retire de sa participation active aux affaires de l'État. Il remet ses biens à son neveu perfide. Voici l’histoire étonnante de la façon dont un débauché ingrat devient faussaire. Un complot astucieux visant à emmêler Ahikar aboutit à sa condamnation à mort. Apparemment, c'est la fin d'Ahikar.}

\par 1 AINSI parla Haiqâr, et quand il eut fini ces injonctions et proverbes à Nadan, le fils de sa sœur, il s'imagina qu'il les garderait tous, et il ne savait pas qu'au lieu de cela il lui témoignait de la lassitude, du mépris et de la moquerie. .

\par 2 Ensuite Haiqâr resta assis dans sa maison et remit à Nadan tous ses biens, et les esclaves, et les servantes, et les chevaux, et le bétail, et tout ce qu'il possédait et gagnait ; et le pouvoir d'enchérir et d'interdire resta entre les mains de Nadan.

\par 3 Et Haiqâr se reposait dans sa maison, et de temps en temps Haiqâr allait rendre hommage au roi et rentrait chez lui.

\par 4 Or, lorsque Nadan comprit que le pouvoir d'enchérir et d'interdire était entre ses mains, il méprisa la position de Haiqâr et se moqua de lui, et se mit à le blâmer chaque fois qu'il apparaissait, disant : « Mon oncle Haiqâr est dans son il est adorateur, et il ne sait plus rien maintenant.

\par 5 Et il commença à battre les esclaves et les servantes, et à vendre les chevaux et les chameaux et à gaspiller tout ce que possédait son oncle Haiqâr.

\par 6 Et quand Haïqâr vit qu'il n'avait aucune compassion pour ses serviteurs ni pour sa maison, il se leva et le chassa de sa maison, et envoya informer le roi qu'il avait dispersé ses biens et ses provisions.

\par 7 Et le roi se leva et appela Nadan et lui dit : 'Tant que Haiqâr reste en bonne santé, personne ne gouvernera sur ses biens, ni sur sa maison, ni sur ses possessions.'

\par 8 Et la main de Nadan fut ôtée de son oncle Haiqâr et de tous ses biens, et pendant ce temps il n'entra ni ne sortit, et il ne le salua pas non plus.

\par 9 Alors Haiqâr se repentit de son labeur avec Nadan, le fils de sa sœur, et il continua à être très triste.

\par 10 Et Nadan avait un frère cadet nommé Benuzârdân, alors Haiqâr le prit pour lui à la place de Nadan, et l'éleva et l'honora du plus grand honneur. Et il lui remit tout ce qu'il possédait, et le fit gouverneur de sa maison.

\par 11 Or, lorsque Nadan comprit ce qui était arrivé, il fut saisi d'envie et de jalousie, et il commença à se plaindre à tous ceux qui l'interrogeaient et à se moquer de son oncle Haiqâr, en disant : « Mon oncle m'a chassé de sa maison, et il m'a préféré mon frère, mais si le Dieu Très-Haut m'en donne le pouvoir, je lui ferai subir le malheur d'être tué.

\par 12 Et Nadan continuait à méditer sur la pierre d'achoppement qu'il pourrait lui inventer. Et après un moment, Nadan réfléchit à cela et écrivit une lettre à Achish, fils de Shah le Sage, roi de Perse, disant ainsi :

\par 13 'Paix, santé, puissance et honneur de la part de Sennachérib, roi d'Assyrie et de Ninive, et de son vizir et de son secrétaire Haiqâr, à toi, ô grand roi ! Qu'il y ait de l'argent entre toi et moi.

\par 14 'Et quand cette lettre te parviendra, si tu te lèves et vas rapidement dans la plaine de Nisrîn, et en Assyrie, et à Ninive, je te livrerai le royaume sans guerre et sans bataille.'

\par 15 Et il écrivit aussi une autre lettre au nom de Haiqâr au Pharaon roi d'Egypte. « Qu'il y ait la paix entre toi et moi, ô puissant roi !

\par 16 'Si au moment où cette lettre te parvient, tu te lèves et te rends en Assyrie et à Ninive dans la plaine de Nisrîn, je te livrerai le royaume sans guerre et sans combat.'

\par 17 Et l'écriture de Nadan était semblable à l'écriture de son oncle Haiqâr.

\par 18 Puis il plia les deux lettres, et les scella du sceau de son oncle Haiqâr ; ils étaient néanmoins dans le palais du roi.

\par 19 Puis il alla écrire une lettre également du roi à son oncle Haiqâr : 'Paix et santé à mon Vizir, mon Secrétaire, mon Chancelier, Haiqâr.'

\par 20 'Ô Haiqâr, quand cette lettre te parviendra, rassemble tous les soldats qui sont avec toi, et qu'ils soient parfaits en vêtements et en nombre, et amène-les-moi le cinquième jour dans la plaine de Nisrîn.'

\par 21 «Et quand tu me verras là venir vers toi, hâte-toi et fais marcher l'armée contre moi comme un ennemi qui veut combattre avec moi, car j'ai avec moi les ambassadeurs du Pharaon, roi d'Egypte, afin qu'ils voient le force de notre armée et peuvent nous craindre, car ils sont nos ennemis et ils nous haïssent.

\par 22 Puis il scella la lettre et l'envoya à Haiqâr par l'intermédiaire d'un des serviteurs du roi. Et il prit l'autre lettre qu'il avait écrite, la distribua devant le roi, la lui lut et lui montra le sceau.

\par 23 Et lorsque le roi entendit ce qu'il y avait dans la lettre, il fut très perplexe et irrité d'une colère grande et ardente, et dit : « Ah, j'ai montré ma sagesse ! qu'ai-je fait à Haiqâr pour qu'il ait écrit ces lettres à mes ennemis ? Est-ce là ma récompense de sa part pour les bienfaits que je lui ai apportés ?

\par 24 Et Nadan lui dit : Ne sois pas attristé, ô roi ! ne soyez pas en colère, mais allons dans la plaine de Nisrîn et voyons si cette histoire est vraie ou non.

\par 25 Alors Nadan se leva le cinquième jour et prit le roi et les soldats et le vizir, et ils partirent dans le désert vers la plaine de Nisrîn. Et le roi regarda, et voilà ! Haiqâr et l’armée étaient en ordre.

\par 26 Et quand Haiqâr vit que le roi était là, il s'approcha et fit signe à l'armée de se déplacer comme en guerre et de combattre en rangée contre le roi comme cela avait été trouvé dans la lettre, il ne sachant pas quelle fosse Nadan avait creusée pour lui. .

\par 27 Et lorsque le roi vit l'acte de Haiqâr, il fut saisi d'inquiétude, de terreur et de perplexité, et fut irrité d'une grande colère.

\par 28 Et Nadan lui dit : As-tu vu, ô mon seigneur le roi ! qu'a fait ce misérable ? mais ne te mets pas en colère et ne sois pas attristé ni peiné, mais va dans ta maison et assieds-toi sur ton trône, et je t'amènerai Haiqâr lié et enchaîné avec des chaînes, et je chasserai ton ennemi sans peine.'

\par 29 Et le roi retourna à son trône, irrité à propos de Haiqâr, et ne fit rien à son égard. Et Nadan alla vers Haiqâr et lui dit : « W'allah, ô mon oncle ! En vérité, le roi se réjouit en toi d'une grande joie et te remercie d'avoir fait ce qu'il t'a ordonné.

\par 30 «Et maintenant, il m'a envoyé vers toi pour que tu renvoies les soldats à leurs devoirs et que tu viennes toi-même à lui, les mains liées derrière toi et les pieds enchaînés, afin que les ambassadeurs de Pharaon voient cela et que le le roi peut être craint par eux et par leur roi.

\par 31 Alors Haiqâr répondit et dit : 'Entendre, c'est obéir.' Et il se leva aussitôt, se lia les mains derrière lui et lui enchaîna les pieds.

\par 32 Et Nadan le prit et alla avec lui chez le roi. Et quand Haiqâr entra en présence du roi, il lui rendit hommage à terre et souhaita le pouvoir et la vie perpétuelle au roi.

\par 33 Alors le roi dit : 'Ô Haiqâr, mon secrétaire, le gouverneur de mes affaires, mon chancelier, le souverain de mon État, dis-moi quel mal je t'ai fait pour que tu m'as récompensé par cette vilaine action.'

\par 34 Puis ils lui montrèrent les lettres écrites de sa main et portant son sceau. Et quand Haiqâr vit cela, ses membres tremblèrent et sa langue fut immédiatement liée, et il fut incapable de prononcer un mot à cause de la peur ; mais il baissait la tête vers la terre et restait muet.

\par 35 Et quand le roi vit cela, il fut sûr que la chose venait de lui, et il se leva aussitôt et leur ordonna de tuer Haïqâr et de lui frapper le cou avec l'épée hors de la ville.

\par 36 Alors Nadan cria et dit : 'Ô Haiqâr, ô visage noir ! à quoi te sert ta méditation ou ton pouvoir pour accomplir cet acte envers le roi ?

\par 37 Ainsi parle le conteur. Et le nom de l'épéiste était Abu Samîk. Et le roi lui dit : « Ô épéiste ! lève-toi, va, tranche le cou d'Haiqâr à la porte de sa maison, et éloigne sa tête de son corps de cent coudées.'

\par 38 Alors Haiqâr s'agenouilla devant le roi et dit : « Que mon seigneur le roi vive pour toujours ! et si tu désires me tuer, que ton souhait se réalise ; et je sais que je ne suis pas coupable, mais le méchant doit rendre compte de sa méchanceté ; néanmoins, ô mon seigneur le roi ! Je t'en supplie, ainsi que ton amitié, permets à l'épéiste de donner mon corps à mes esclaves, afin qu'ils m'enterrent, et que ton esclave soit ton sacrifice.

\par 39 Le roi se leva et ordonna à l'épéiste d'agir avec lui selon son désir.

\par 40 Et il ordonna aussitôt à ses serviteurs de prendre Haiqâr et l'épéiste et de l'accompagner nus afin qu'ils le tuent.

\par 41 Et quand Haïqâr fut sûr qu'il allait être tué, il envoya vers sa femme et lui dit : « Sors et à ma rencontre, et qu'il y ait avec toi mille jeunes vierges, et habille-les de robes de pourpre et de soie afin qu'ils puissent pleurer sur moi avant ma mort.»

\par 42 'Et prépare une table pour l'épéiste et pour ses serviteurs. Et mélangez beaucoup de vin, afin qu'ils boivent.

\par 43 Et elle fit tout ce qu'il lui commandait. Et elle était très sage, intelligente et prudente. Et elle a uni toute la courtoisie et l'érudition possibles.

\par 44 Et quand l'armée du roi et l'épéiste arrivèrent, ils trouvèrent la table dressée, le vin et les mets luxueux, et ils commencèrent à manger et à boire jusqu'à ce qu'ils soient rassasiés et ivres.

\par 45 Alors Haiqâr prit l'épéiste à l'écart de la compagnie et dit : «Ô Abou Samîk, ne sais-tu pas que lorsque Sarhadum le roi, le père de Sennachérib, voulut te tuer, je t'ai pris et je t'ai caché dans un certain endroit jusqu'à ce que la colère du roi se calme et qu'il te demande ?»

\par 46 'Et quand je t'ai amené devant lui, il s'est réjoui en toi : et maintenant souviens-toi de la bonté que je t'ai fait.'

\par 47 'Et je sais que le roi se repentira de moi et sera irrité d'une grande colère à propos de mon exécution.'

\par 48 'Car je ne suis pas coupable, et ce sera lorsque tu me présenteras devant lui dans son palais, tu rencontreras une grande chance, et tu sauras que Nadan, le fils de ma sœur, m'a trompé et a commis cette mauvaise action pour moi, et le roi se repentira de m'avoir tué ; et maintenant j'ai une cave dans le jardin de ma maison, et personne ne la connaît.

\par 49 'Cache-moi dedans à la connaissance de ma femme. Et j'ai un esclave en prison qui mérite d'être tué.

\par 50 Faites-le sortir, habillez-le de mes vêtements, et ordonnez aux serviteurs, lorsqu'ils sont ivres, de le tuer. Ils ne sauront pas qui ils tuent.

\par 51 Et jetez sa tête à cent coudées de son corps, et donnez son corps à mes esclaves afin qu'ils l'enterrent. Et tu auras amassé avec moi un grand trésor.

\par 52 'Et alors l'épéiste fit ce qu'Haiqâr lui avait ordonné, et il alla vers le roi et lui dit : 'Que ta tête vive pour toujours !'

\par 53 'Alors la femme de Haiqâr lui faisait descendre chaque semaine dans la cachette ce qui lui suffisait, et personne d'autre qu'elle-même ne le savait.'

\par 54 'Et l'histoire fut rapportée, répétée et répandue dans tous les lieux de la façon dont Haiqâr le Sage avait été tué et était mort, et tous les habitants de cette ville le pleurèrent.'

\par 55 'Et ils pleurèrent et dirent : « Hélas pour toi, ô Haiqâr ! et pour ton savoir et ta courtoisie ! Comme c'est triste pour toi et pour ta connaissance ! Où peut-on en trouver un autre comme toi ? et où peut-il y avoir un homme si intelligent, si instruit, si habile à gouverner qu'il te ressemble pour occuper ta place ?

\par 56 'Mais le roi se repentait de Haiqâr, et son repentir ne lui servait à rien.'

\par 57 'Puis il appela Nadan et lui dit : 'Va, emmène tes amis avec toi et fais le deuil et les pleurs pour ton oncle Haiqâr, et pleure-toi sur lui comme c'est la coutume, faisant honneur à sa mémoire.' '

\par 58 'Mais quand Nadan, l'insensé, l'ignorant, le cœur dur, se rendit dans la maison de son oncle, il ne pleura ni ne s'affligea, mais rassembla des gens sans cœur et dissolus et se mit à manger et à boire.' 1

\par 59 'Et Nadan commença à saisir les servantes et les esclaves appartenant à Haiqâr, et les ligota et les tortura et les frappa d'une raclée douloureuse.'

\par 60 'Et il n'a pas respecté la femme de son oncle, celle qui l'avait élevé comme son propre fils, mais voulait qu'elle tombe dans le péché avec lui.'

\par 61 'Mais Haiqâr avait été coupé dans la cachette, et il entendait les pleurs de ses esclaves et de ses voisins, et il louait le Dieu Très-Haut, le Miséricordieux, et rendait grâce, et il priait et suppliait toujours le Dieu Très-Haut.

\par 62 'Et l'épéiste venait de temps en temps à Haiqâr pendant qu'il était au milieu de la cachette : et Haiqâr venait et le suppliait. Et il le réconforta et lui souhaita la délivrance.

\par 63 'Et quand l'histoire fut rapportée dans d'autres pays selon laquelle Haiqâr le Sage avait été tué, tous les rois furent affligés et méprisèrent le roi Sennachérib, et ils se lamentèrent sur Haiqâr le résolveur d'énigmes.'

\par \textit{Notes de bas de page}

\par \textit{207:1 Comparez ce récit des réjouissances de Nadan et de ses coups sur les serviteurs avec Matthieu XXIV. 48-51 et Luc XII. 43-46. Vous verrez que la langue d'Ahikar a coloré une des paraboles de notre Seigneur.}

\chapter{4}

\par \textit{«Les énigmes du Sphinx.» Qu'est-il réellement arrivé à Ahikar. Son retour.}

\par 1 ET lorsque le roi d'Egypte s'est assuré que Haiqâr était tué, il se leva aussitôt et écrivit une lettre au roi Sennachérib, lui rappelant 'la paix et la santé et la puissance et l'honneur que nous souhaitons spécialement toi, mon frère bien-aimé, le roi Sennachérib.

\par 2 « J'ai désiré bâtir un château entre le ciel et la terre, et je veux que tu m'envoies de toi un homme sage et intelligent pour me le bâtir, et pour me répondre à toutes mes questions, et que je pourra avoir les impôts et les droits de douane de l'Assyrie pendant trois ans.

\par 3 Puis il scella la lettre et l'envoya à Sennachérib.

\par 4 Il le prit, le lut et le donna à ses vizirs et aux nobles de son royaume, et ils furent perplexes et honteux, et il fut irrité d'une grande colère, et ne savait pas comment il devait agir.

\par 5 Alors il assembla les vieillards et les savants et les sages et les philosophes, et les devins et les astrologues, et tous ceux qui étaient dans son pays, et leur lut la lettre et leur dit : « Qui d'entre eux tu iras voir Pharaon, roi d'Egypte, et tu lui répondras à ses questions ?

\par 6 Et ils lui dirent : 'Ô notre roi, notre seigneur ! sache qu'il n'y a personne dans ton royaume qui soit au courant de ces questions, à l'exception de Haiqâr, ton vizir et secrétaire.

\par 7 'Mais quant à nous, nous n'avons aucune compétence en cela, à moins que ce ne soit Nadan, le fils de sa sœur, car il lui a enseigné toute sa sagesse, son savoir et sa connaissance. Appelle-le vers toi, peut-être pourra-t-il dénouer ce nœud difficile.

\par 8 Alors le roi appela Nadan et lui dit : «Regarde cette lettre et comprends ce qu'elle contient.» Et quand Nadan l'a lu, il a dit : « Ô mon seigneur ! qui est capable de construire un château entre le ciel et la terre ?

\par 9 Et lorsque le roi entendit le discours de Nadan, il fut affligé d'un chagrin grand et douloureux, et descendit de son trône et s'assit sur les cendres, et se mit à pleurer et à se lamenter sur Haiqâr.

\par 10 Dire : '¡'Ô mon chagrin ! Ô Haiqâr, qui connaissais les secrets et les énigmes ! malheur à moi pour toi, ô Haiqâr ! Ô maître de mon pays et souverain de mon royaume, où trouverai-je ton semblable ? Ô Haiqâr, ô maître de mon pays, vers qui dois-je me tourner pour toi ? malheur à moi pour toi ! comment t'ai-je détruit ! et j'ai écouté le discours d'un garçon stupide et ignorant, sans connaissance, sans religion, sans virilité.

\par 11 'Ah! et encore Ah pour moi ! qui peut te donner juste pour une fois, ou me faire savoir que Haiqâr est vivant ? et je lui donnerais la moitié de mon royaume.

\par 12 'D'où est-ce que cela me vient ? Ah, Haiqâr ! afin que je puisse te voir juste pour une fois, afin que je puisse me rassasier de te regarder et de me réjouir de toi.

\par 13 'Ah! Ô mon chagrin pour toi pour toujours ! Ô Haiqâr, comment t'ai-je tué ! et je n'ai pas tardé dans ton cas jusqu'à ce que j'aie vu la fin de l'affaire.

\par 14 Et le roi continuait à pleurer nuit et jour. Alors, quand l'épéiste vit la colère du roi et son chagrin pour Haiqâr, son cœur s'adoucit envers lui, et il s'approcha de lui et lui dit :

\par 15 'Ô mon seigneur ! ordonne à tes serviteurs de me couper la tête. Alors le roi lui dit : « Malheur à toi, Abou Samîk, quelle est ta faute ?

\par 16 Et l'épéiste lui dit : 'Ô mon maître ! tout esclave qui agit contrairement à la parole de son maître est tué, et j'ai agi contrairement à ton ordre.

\par 17 Alors le roi lui dit. « Malheur à toi, ô Abou Samîk, en quoi as-tu agi contrairement à mon ordre ?

\par 18 Et l'épéiste lui dit : 'Ô mon seigneur ! tu m'as ordonné de tuer Haiqâr, et je savais que tu te repentirais de lui, et qu'il avait été lésé, et je l'ai caché dans un certain endroit, et j'ai tué un de ses esclaves, et il est maintenant en sécurité dans le citerne, et si tu me l'ordonnes, je te l'amènerai.

\par 19 Et le roi lui dit. « Malheur à toi, ô Abou Samîk ! tu t'es moqué de moi et je suis ton seigneur.

\par 20 Et l'épéiste lui dit : « Non, mais par la vie de ta tête, ô mon seigneur ! Haiqâr est sain et sauf et vivant.

\par 21 Et quand le roi entendit cette parole, il se sentit sûr de la chose, et sa tête tourna, et il s'évanouit de joie, et il leur ordonna d'amener Haiqâr.

\par 22 Et il dit à l'épéiste : « Ô fidèle serviteur ! si ton discours est vrai, je voudrais t'enrichir et élever ta dignité au-dessus de celle de tous tes amis.

\par 23 Et l'épéiste s'en alla en se réjouissant jusqu'à ce qu'il arrive à la maison de Haiqâr. Et il ouvrit la porte de la cachette, descendit et trouva Haïqâr assis, louant Dieu et le remerciant.

\par 24 Et il lui cria, disant : 'Ô Haiqâr, j'apporte la plus grande joie, le plus grand bonheur et le plus grand délice !'

\par 25 Et Haiqâr lui dit : 'Quelles sont les nouvelles, ô Abou Samîk ?' Et il lui raconta tout sur Pharaon du début à la fin. Puis il le prit et alla chez le roi.

\par 26 Et quand le roi le regarda, il le vit dans un état de misère, et que ses cheveux étaient longs comme ceux des bêtes sauvages et ses ongles comme les griffes d'un aigle, et que son corps était sale de poussière. , et la couleur de son visage avait changé et s'estompait et était maintenant comme de la cendre.

\par 27 Et quand le roi le vit, il s'affligea de lui et se leva aussitôt, l'embrassa et l'embrassa, pleura sur lui et dit : « Loué soit Dieu ! » qui t'a ramené vers moi.

\par 28 Puis il le consola et le réconforta. Et il ôta sa robe et la mit sur l'épéiste, et lui fut très aimable, et lui donna une grande richesse, et fit se reposer Haiqâr.

\par 29 Alors Haiqâr dit au roi : « Que mon seigneur le roi vive pour toujours ! Telles sont les actions des enfants du monde. J'ai élevé un palmier pour m'appuyer dessus, et il s'est plié de côté et m'a jeté à terre.

\par 30 Mais, ô mon Seigneur ! puisque je suis apparu devant toi, ne t'opprime pas ! Et le roi lui dit : « Béni soit Dieu, qui t'a fait miséricorde, qui a su que tu avais été lésé, qui t'a sauvé et qui t'a délivré de la mort. »

\par 31 « Mais va au bain chaud, rase-toi la tête, coupe-toi les ongles, change de vêtements et amusez-vous pendant quarante jours, afin que vous puissiez vous faire du bien et améliorer votre condition et votre couleur. » de ton visage peut te revenir.

\par 32 Alors le roi ôta sa robe luxueuse et la mit sur Haiqâr, et Haiqâr remercia Dieu et rendit hommage au roi, et partit vers sa demeure heureux et heureux, louant le Dieu Très-Haut.

\par 33 Et les gens de sa maison se réjouirent avec lui, et ses amis et tous ceux qui apprenaient qu'il était vivant se réjouirent aussi.

\chapter{5}

\par \textit{La lettre des « énigmes » est montrée à Ahikar. Les garçons sur les aigles. Le premier tour « en avion ». En route pour l'Egypte. Ahikar, étant un homme de sagesse, a aussi le sens de l'humour. (Verset 27).}

\par 1 ET il fit ce que le roi lui avait ordonné, et se reposa quarante jours.

\par 2 Alors il s'habilla de ses plus beaux habits et partit à cheval vers le roi, avec ses esclaves derrière lui et devant lui, joyeux et ravi.

\par 3 Mais lorsque Nadan, le fils de sa sœur, s'aperçut de ce qui se passait, la peur et la terreur s'emparèrent de lui, et il resta perplexe, ne sachant que faire.

\par 4 Et quand Haiqâr vit cela, il entra en présence du roi et le salua, et il lui rendit le salut, et le fit asseoir à ses côtés, en lui disant : 'Ô mon cher Haiqâr ! regarde ces lettres que le roi d'Egypte nous a envoyées après avoir appris que tu avais été tué.

\par 5 'Ils nous ont provoqués et nous ont vaincus, et beaucoup d'habitants de notre pays ont fui vers l'Egypte par peur des impôts que le roi d'Egypte a envoyé exiger de nous.'

\par 6 Alors Haiqâr prit la lettre, la lut et comprit son contenu.

\par 7 Alors il dit au roi. « Ne vous fâchez pas, ô mon seigneur ! J'irai en Égypte, et je rendrai les réponses à Pharaon, et je lui montrerai cette lettre, et je lui répondrai au sujet des impôts, et je renverrai tous ceux qui ont fui ; et je ferai honte à tes ennemis avec l'aide du Dieu Très-Haut et pour le bonheur de ton royaume.

\par 8 Et quand le roi entendit ce discours de Haiqâr, il se réjouit d'une grande joie, et son cœur s'épanouit et il lui témoigna de la faveur.

\par 9 Et Haiqâr dit au roi : 'Accorde-moi un délai de quarante jours pour que je puisse examiner cette question et la régler.' Et le roi le permit.

\par 10 Et Haiqâr se rendit à sa demeure, et il ordonna aux chasseurs de lui capturer deux jeunes aiglons, et ils les capturèrent et les lui apportèrent : et il ordonna aux tisserands de cordes de tisser pour lui deux câbles de coton, chacun dont deux mille coudées de long, et il fit amener les charpentiers et leur ordonna de fabriquer deux grandes caisses, et ils firent cela.

\par 11 Puis il prit deux petits garçons et passa chaque jour à sacrifier des agneaux et à nourrir les aigles et les garçons, et à faire monter les garçons sur le dos des aigles, et il les lia avec un nœud solide, et attacha le câble à les pieds des aigles, et les laissa s'élever petit à petit chaque jour, jusqu'à une distance de dix coudées, jusqu'à ce qu'ils s'y habituent et y soient instruits ; et ils s'élevèrent sur toute la longueur de la corde jusqu'à atteindre le ciel ; les garçons étant sur le dos. Puis il les a attirés vers lui.

\par 12 Et quand Haiqâr vit que son désir était exaucé, il ordonna aux garçons que lorsqu'ils seraient portés vers le ciel, ils devaient crier, en disant :

\par 13 'Apportez-nous de l'argile et de la pierre, afin que nous puissions bâtir un château pour le roi Pharaon, car nous sommes oisifs.'

\par 14 Et Haiqâr n'a jamais fini de les entraîner et de les exercer jusqu'à ce qu'ils aient atteint le plus haut niveau possible (de compétence).

\par 15 Alors, les laissant, il alla vers le roi et lui dit : « Ô mon seigneur ! le travail est terminé selon ton désir. Lève-toi avec moi pour que je puisse te montrer la merveille.

\par 16 Alors le roi se leva et s'assit avec Haiqâr et alla dans un endroit large et envoya amener les aigles et les garçons, et Haiqâr les attacha et les laissa en l'air sur toute la longueur des cordes, et ils commencèrent à crier comme il le leur avait appris. Puis il les a attirés vers lui et les a mis à leur place.

\par 17 Et le roi et ceux qui étaient avec lui furent dans un grand étonnement : et le roi baisa Haiqâr entre ses yeux et lui dit : « Va en paix, ô mon bien-aimé ! Ô fierté de mon royaume ! en Égypte, réponds aux questions de Pharaon et vaincs-le par la force du Dieu Très-Haut.

\par 18 Alors il lui fit ses adieux, et prit ses troupes et son armée, les jeunes gens et les aigles, et se dirigea vers les demeures de l'Égypte ; et quand il fut arrivé, il se tourna vers le pays du roi.

\par 19 Et lorsque le peuple d'Égypte apprit que Sennachérib avait envoyé un homme de son conseil privé pour parler avec Pharaon et répondre à ses questions, ils apportèrent la nouvelle au roi Pharaon, et il envoya un groupe de ses conseillers privés pour l'amener devant lui.

\par 20 Et il vint et entra en présence de Pharaon, et lui rendit hommage comme il convient de faire aux rois.

\par 21 Et il lui dit : « Ô mon seigneur le roi ! Le roi Sennachérib te salue avec une abondance de paix, de puissance et d'honneur.

\par 22 « Et il m'a envoyé, moi qui suis l'un de ses esclaves, pour que je puisse répondre à tes questions et réaliser tout ton désir ; car tu as envoyé chercher auprès de mon seigneur le roi un homme qui te construira une maison. » château entre le ciel et la terre.

\par 23 'Et moi, avec l'aide du Dieu Très-Haut et ta noble faveur et la puissance de mon seigneur le roi, je te le bâtirai comme tu le désires.'

\par 24 'Mais, ô mon seigneur le roi ! ce que tu y as dit sur les impôts de l'Egypte pendant trois ans - maintenant la stabilité d'un royaume est une stricte justice, et si tu gagnes et que ma main n'a pas l'habileté de te répondre, alors mon seigneur le roi t'enverra les impôts dont tu as parlé.

\par 25 «Et si je t'ai répondu à tes questions, il te restera à envoyer tout ce que tu as mentionné à mon seigneur le roi.»

\par 26 Et lorsque Pharaon entendit ce discours, il s'étonna et fut perplexe devant la liberté de sa langue et la douceur de son discours.

\par 27 Et le roi Pharaon lui dit : 'Ô homme ! quel est ton nom ? Et il dit : « Ton serviteur est Abiqâm, et moi une petite fourmi des fourmis du roi Sennachérib.

\par 28 Et Pharaon lui dit : Ton seigneur n'avait-il personne de plus haute dignité que toi pour m'envoyer une petite fourmi pour me répondre et converser avec moi ?

\par 29 Et Haiqâr lui dit : 'Ô mon seigneur le roi ! Je voudrais au Dieu Très-Haut d'accomplir ce que tu penses, car Dieu est avec les faibles pour confondre les forts.

\par 30 Alors Pharaon ordonna qu'ils préparent une demeure pour Abiqâm et lui fournissent de la nourriture, de la viande et des boissons, et tout ce dont il avait besoin.

\par 31 Et quand cela fut terminé, trois jours après, Pharaon s'habilla de pourpre et de rouge et s'assit sur son trône, et tous ses vizirs et les magnats de son royaume se tenaient debout, les mains croisées, leurs pieds rapprochés et leur tête baissée.

\par 32 Et Pharaon envoya chercher Abiqâm, et quand il lui fut présenté, il se prosterna devant lui et baisa la terre devant lui.

\par 33 Et le roi Pharaon lui dit : 'Ô Abiqâm, à qui suis-je semblable ? et les nobles de mon royaume, à qui ressemblent-ils ?

\par 34 Et Haiqâr lui dit : 'Ô mon seigneur le parent, tu es comme l'idole Bel, et les nobles de ton royaume sont comme ses serviteurs.'

\par 35 Il lui dit : Va et reviens ici demain. Alors Haiqâr partit comme le roi Pharaon le lui avait ordonné.

\par 36 Et le lendemain, Haïqâr se présenta devant Pharaon, se prosterna et se présenta devant le roi. Et Pharaon était vêtu de rouge, et les nobles étaient vêtus de blanc.

\par 37 Et Pharaon lui dit : « Ô Abiqâm, à qui suis-je semblable ? et les nobles de mon royaume, à qui ressemblent-ils ?

\par 38 Et Abiqâm lui dit : 'Ô mon seigneur ! tu es comme le soleil, et tes serviteurs sont comme ses rayons. Et Pharaon lui dit : Va dans ta demeure et viens ici demain.

\par 39 Alors Pharaon ordonna à sa cour de s'habiller d'un blanc pur, et Pharaon était habillé comme eux et s'assit sur son trône, et il leur ordonna d'aller chercher Haiqâr. Et il entra et s'assit devant lui.

\par 40 Et Pharaon lui dit : 'Ô Abiqâm, à qui suis-je semblable ? et mes nobles, à qui ressemblent-ils ?

\par 41 Et Abiqâm lui dit : 'Ô mon seigneur ! tu es comme la lune, et tes nobles sont comme les planètes et les étoiles. Et Pharaon lui dit : « Va, et sois ici demain. »

\par 42 Alors Pharaon ordonna à ses serviteurs de porter des robes de différentes couleurs, et Pharaon porta une robe de velours rouge, et s'assit sur son trône, et leur ordonna d'aller chercher Abiqâm. Et il entra et se prosterna devant lui.

\par 43 Et il dit : 'Ô Abiqâm, à qui suis-je semblable ? et mes armées, à qui ressemblent-elles ? Et il dit : « Ô mon seigneur ! tu es comme le mois d'avril, et tes armées sont comme ses fleurs.

\par 44 Et quand le roi l'entendit, il se réjouit d'une grande joie et dit : 'Ô Abiqâm ! la première fois tu m'as comparé à l'idole Bel, et mes nobles à ses serviteurs.

\par 45 «Et la deuxième fois tu m'as comparé au soleil, et mes nobles aux rayons du soleil.»

\par 46 «Et la troisième fois tu m'as comparé à la lune, et mes nobles aux planètes et aux étoiles.»

\par 47 « Et la quatrième fois tu m'as comparé au mois d'avril, et mes nobles à ses fleurs. Mais maintenant, ô Abiqâm ! Dis-moi, ton seigneur, le roi Sennachérib, à qui ressemble-t-il ? et ses nobles, à qui ressemblent-ils ?

\par 48 Et Haïqâr cria d'une voix forte et dit : « Que je sois loin de faire mention de mon seigneur le roi et de toi assis sur ton trône. Mais lève-toi et je te dirai à qui ressemble mon seigneur le roi et à qui ressemblent ses nobles.

\par 49 Et Pharaon était perplexe devant la liberté de sa langue et son audace dans ses réponses. Alors Pharaon se leva de son trône, se tint devant Haiqâr et lui dit : « Dis-moi maintenant, afin que je sache à qui ressemble ton seigneur le roi, et ses nobles, à qui ils ressemblent.

\par 50 Et Haiqâr lui dit : 'Mon seigneur est le Dieu du ciel, et ses nobles sont les éclairs et le tonnerre, et quand il veut, les vents soufflent et la pluie tombe.'

\par 51 «Et il commande le tonnerre, et il éclaire et pleut, et il retient le soleil, et il ne donne pas sa lumière, et la lune et les étoiles, et elles ne tournent pas.»

\par 52 'Et il commande la tempête, et elle souffle et la pluie tombe et elle piétine Avril et détruit ses fleurs et ses maisons.'

\par 53 Et lorsque Pharaon entendit ce discours, il fut très perplexe et se mit en colère d'une grande colère, et il lui dit : « Ô homme ! dis-moi la vérité et fais-moi savoir qui tu es réellement.

\par 54 Et il lui dit la vérité. 'Je suis Haiqâr le scribe, le plus grand des conseillers privés du roi Sennachérib, et je suis son vizir et le gouverneur de son royaume, et son chancelier.'

\par 55 Et il lui dit : Tu as dit la vérité dans cette parole. Mais nous avons entendu parler de Haiqâr, que le roi Sennachérib l'a tué, et pourtant tu sembles être vivant et en bonne santé.

\par 56 Et Haiqâr lui dit : « Oui, c'est vrai, mais louange à Dieu, qui sait ce qui est caché, car mon seigneur le roi a ordonné de me faire tuer, et il a cru à la parole des hommes débauchés, mais le Le Seigneur m'a délivré, et béni soit celui qui se confie en lui.

\par 57 Et Pharaon dit à Haiqâr : « Va, et demain sois ici, et dis-moi une parole que je n'ai jamais entendue de la part de mes nobles ni des gens de mon royaume et de mon pays.

\chapter{6}

\par \textit{La ruse réussit. Ahikar répond à toutes les questions de Pharaon. Les garçons sur les aigles sont le point culminant de la journée. L'esprit, si rarement trouvé dans les Écritures anciennes, est révélé dans les versets 34-45.}

\par 1 ET Haiqâr se rendit à sa demeure, et écrivit une lettre, disant dedans ceci :

\par 2 De Sennachérib, roi d'Assyrie et Ninive au Pharaon, roi d'Égypte.

\par 3 « La paix soit avec toi, ô mon frère ! et ce que nous te faisons savoir par ceci, c'est qu'un frère a besoin de son frère et des rois les uns des autres, et j'espère de toi que tu me prêteras neuf cents talents d'or, car j'en ai besoin pour l'approvisionnement de certains des soldats, afin que je puisse le dépenser pour eux. Et dans peu de temps, je te l'enverrai.

\par 4 Puis il plia la lettre et la présenta le lendemain à Pharaon.

\par 5 Et quand il vit cela, il fut perplexe et lui dit : « En vérité, je n'ai jamais entendu de personne rien de semblable à ce langage. »

\par 6 Alors Haiqâr lui dit : 'En vérité, c'est une dette que tu as envers mon seigneur le roi.'

\par 7 Et Pharaon accepta cela, disant : 'Ô Haiqâr, c'est comme toi qui es honnête au service des rois.'

\par 8 'Béni soit Dieu qui t'a rendu parfait en sagesse et qui t'a orné de philosophie et de connaissance.'

\par 9 'Et maintenant, ô Haiqâr, il reste ce que nous désirons de toi, que tu construises comme un château entre le ciel et la terre.'

\par 10 Alors Haiqâr dit : « Entendre, c'est obéir. Je te construirai un château selon ton souhait et ton choix ; mais, ô mon seigneur, je nous prépare de la chaux, de la pierre, de l'argile et des ouvriers, et j'ai des constructeurs habiles qui bâtiront pour toi comme tu le désires.

\par 11 Et le roi prépara tout cela pour lui, et ils se dirigèrent vers un vaste lieu ; et Haiqâr et ses garçons s'y rendirent, et il prit avec lui les aigles et les jeunes hommes ; Et le roi et tous ses nobles s'en allèrent et toute la ville se rassembla, pour voir ce que ferait Haiqâr.

\par 12 Alors Haiqâr fit sortir les aigles des caisses, et attacha les jeunes hommes sur le dos, et attacha les cordes aux pieds des aigles, et les laissa aller dans les airs. Et ils s’élevèrent jusqu’à rester entre le ciel et la terre.

\par 13 Et les garçons se mirent à crier, disant : « Apportez des briques, apportez de l'argile, afin que nous puissions bâtir le château du roi, car nous restons les bras croisés !

\par 14 Et la foule était étonnée et perplexe, et elle s'étonnait. Et le roi et ses nobles se posèrent la question.

\par 15 Et Haiqâr et ses serviteurs commencèrent à battre les ouvriers, et ils crièrent vers les troupes du roi, en leur disant : 'Apportez aux ouvriers habiles ce qu'ils veulent et ne les empêchez pas de faire leur travail.'

\par 16 Et le roi lui dit : Tu es fou ; qui peut amener quoi que ce soit à cette distance ?

\par 17 Et Haiqâr lui dit : 'Ô mon seigneur ! comment allons-nous construire un château dans les airs ? et si monseigneur le roi était ici, il aurait bâti plusieurs châteaux en un seul jour.

\par 18 Et Pharaon lui dit : « Va, ô Haiqâr, dans ta demeure et repose-toi, car nous avons renoncé à construire le château, et demain viens vers moi.

\par 19 Alors Haïqâr se rendit à sa demeure et le lendemain il se présenta devant Pharaon. Et Pharaon dit : « Ô Haiqâr, quelles nouvelles y a-t-il du cheval de ton seigneur ? car lorsqu'il hennit dans le pays d'Assyrie et de Ninive, et que nos juments entendent sa voix, elles jettent leurs petits.

\par 20 Et quand Haïqâr entendit ce discours, il alla prendre un chat, l'attacha et commença à la fouetter avec violence jusqu'à ce que les Égyptiens l'entendent, et ils allèrent en parler au roi.

\par 21 Et Pharaon envoya chercher Haiqâr et lui dit : 'Ô Haiqâr, pourquoi fouettes-tu ainsi et bats-tu cette bête muette ?'

\par 22 Et Haiqâr lui dit : Mon seigneur le roi ! En vérité, elle m'a fait une vilaine action et a mérité cette raclée et cette flagellation, car mon seigneur le roi Sennachérib m'avait donné un beau coq, et il avait une voix forte et vraie et connaissait les heures du jour et de la nuit.

\par 23 Et le chat s'est levé cette nuit même, lui a coupé la tête et s'en est allé, et à cause de cet acte, je lui ai infligé cette raclée.

\par 24 Et Pharaon lui dit : 'Ô Haiqâr, je vois à tout cela que tu vieillis et que tu es en enfance, car entre l'Egypte et Ninive il y a soixante-huit parasanges, et comment est-elle allée cette nuit même et coupe la tête de ton sexe et reviens ?

\par 25 Et Haiqâr lui dit : 'Ô mon seigneur ! S'il y avait une telle distance entre l'Egypte et Ninive, comment tes juments pourraient-elles entendre quand le cheval du roi de mon seigneur hennit et jette leurs petits ? et comment la voix du cheval pourrait-elle parvenir jusqu'en Égypte ?

\par 26 Et quand Pharaon entendit cela, il sut que Haiqâr avait répondu à ses questions.

\par 27 Et Pharaon dit : 'Ô Haiqâr, je veux que tu me fasses des cordes avec le sable de la mer.'

\par 28 Et Haiqâr lui dit : 'Ô mon seigneur le roi ! ordonne-leur de m'apporter une corde du trésor pour que j'en fasse une semblable.

\par 29 Alors Haiqâr se dirigea vers l'arrière de la maison et fora des trous dans le rivage accidenté de la mer, et prit dans sa main une poignée de sable, du sable de mer, et quand le soleil se leva et pénétra dans les trous, il étala le sable au soleil jusqu'à ce qu'il devienne comme tissé comme des cordes.

\par 30 Et Haiqâr dit : « Ordonne à tes serviteurs de prendre ces cordes, et chaque fois que tu le voudras, je t'en tisserai comme elles.

\par 31 Et Pharaon dit : 'Ô Haiqâr, nous avons ici une meule et elle a été cassée et je veux que tu la recoudes.'

\par 32 Alors Haiqâr la regarda et trouva une autre pierre.

\par 33 Et il dit à Pharaon : « Ô mon seigneur ! Je suis étrangère : et je n'ai pas d'outil pour coudre.

\par 34 «Mais je veux que tu ordonnes à tes fidèles cordonniers de tailler des poinçons dans cette pierre, afin que je puisse coudre cette meule.»

\par 35 Alors Pharaon et tous ses nobles se mirent à rire. Et il dit : « Béni soit le Dieu Très-Haut, qui t'a donné cet esprit et cette connaissance. »

\par 36 Et lorsque Pharaon vit qu'Haiqâr l'avait vaincu et lui rendit ses réponses, il s'excita aussitôt et ordonna de percevoir pour lui trois années d'impôts et de les amener à Haiqâr.

\par 37 Et il ôta ses robes et les mit sur Haiqâr, ses soldats et ses serviteurs, et lui donna les frais de son voyage.

\par 38 Et il lui dit : Va en paix, ô force de son seigneur et orgueil de ses docteurs ! As-tu un sultan comme toi ? salue mon seigneur le roi Sennachérib et dis-lui comment nous lui avons envoyé des cadeaux, car les rois se contentent de peu.

\par 39 Alors Haiqâr se leva, baisa les mains du roi Pharaon et baisa le sol devant lui, et lui souhaita force et continuité, et abondance dans son trésor, et lui dit : « Ô mon seigneur ! Je désire de toi qu'aucun de nos compatriotes ne reste en Égypte.

\par 40 Et Pharaon se leva et envoya des hérauts pour proclamer dans les rues d'Égypte qu'aucun des habitants d'Assyrie et de Ninive ne resterait au pays d'Égypte, mais qu'ils devraient aller avec Haiqâr.

\par 41 Alors Haiqâr partit et prit congé du roi Pharaon, et partit, cherchant le pays d'Assyrie et de Ninive ; et il avait des trésors et beaucoup de richesses.

\par 42 Et lorsque la nouvelle parvint au roi Sennachérib de l'arrivée d'Haiqâr, il sortit à sa rencontre et se réjouit extrêmement de lui avec une grande joie et l'embrassa et l'embrassa et lui dit : « Bienvenue à la maison : ô parent ! mon frère Haiqâr, la force de mon royaume et la fierté de mon royaume.

\par 43 «Demande ce que tu voudrais de moi, même si tu désires la moitié de mon royaume et de mes biens.»

\par 44 Alors Haiqâr lui dit : « Ô mon seigneur le roi, vis pour toujours ! Faites preuve de faveur, ô mon seigneur le roi ! à Abu Samîk à ma place, car ma vie était entre les mains de Dieu et entre les siennes.

\par 45 Alors le roi Sennachérib dit : « Honneur à toi, ô mon bien-aimé Haiqâr ! Je ferai du rang d'Abou Samîk, l'épéiste, un rang plus élevé que tous mes conseillers privés et mes favoris.

\par 46 Alors le roi commença à lui demander comment il s'était entendu avec Pharaon depuis son arrivée jusqu'à son départ de sa présence, et comment il avait répondu à toutes ses questions, et comment il avait reçu de lui les impôts, et les changements de vêtements et les cadeaux.

\par 47 Et le roi Sennachérib se réjouit d'une grande joie et dit à Haiqâr : « Prends ce que tu désires avoir de ce tribut, car tout est à la portée de ta main. »

\par 48 Et Haiqâr mid : 'Que le roi vive éternellement !' Je ne désire rien d'autre que la sécurité de mon seigneur le roi et le maintien de sa grandeur.

\par 49 'Ô mon seigneur ! que puis-je faire avec la richesse et autres ? mais si tu veux me témoigner ta faveur, donne-moi Nadan, le fils de ma sœur, afin que je puisse le récompenser de ce qu'il m'a fait, et m'accorder son sang et m'en tenir innocent.

\par 50 Et le roi Sennachérib dit : Prends-le, je te le donne. Et Haiqâr prit Nadan, le fils de sa sœur, et lui lia les mains avec des chaînes de fer, et l'emmena dans sa demeure, et lui mit une lourde chaîne aux pieds, et l'attacha avec un nœud serré, et après l'avoir ainsi lié, il le jeta dans une pièce sombre, à côté du lieu de retraite, et nomma Nebu-hal comme sentinelle sur lui pour lui donner chaque jour une miche de pain et un peu d'eau.

\chapter{7}

\par \textit{Les paraboles d'Ahikar dans lesquelles il complète l'éducation de son neveu. Des comparaisons frappantes. Ahikar appelle le garçon des noms pittoresques. Ici se termine l'histoire d'Ahikar.}

\par 1 Et chaque fois qu'Haïqâr entrait ou sortait, il grondait Nadan, le fils de sa sœur, en lui disant sagement :

\par 2 'Ô Nadan, mon garçon ! Je t'ai fait tout ce qui est bon et gentil et tu m'en as récompensé par ce qui est laid et mauvais et par le meurtre.

\par 3 'Ô mon fils ! il est dit dans les proverbes : Celui qui n'écoute pas avec son oreille, on lui fera écouter avec la peau de son cou.

\par 4 Et Nadan dit : « Pour quelle raison es-tu irrité contre moi ?

\par 5 Et Haiqâr lui dit : Parce que je t'ai élevé, je t'ai enseigné, je t'ai donné honneur et respect, je t'ai rendu grand, je t'ai élevé avec la meilleure race et je t'ai fait asseoir à ma place afin que tu sois mon héritier dans le monde, et tu m'as traité en me tuant et tu m'as récompensé par ma ruine.

\par 6 Mais l'Éternel savait que j'avais été lésé, et il m'a sauvé du traitement que tu m'avais fixé, car l'Éternel guérit les cœurs brisés et empêche les envieux et les hautains.

\par 7 Ô mon garçon ! tu as été pour moi comme le scorpion qui, lorsqu'il frappe l'airain, le transperce.

\par 8 Ô mon garçon ! tu es comme la gazelle qui mangeait les racines de la garance, et ça m'ajoute aujourd'hui et demain ils bronzeront ils se cachent dans mes racines.

\par 9 Ô mon garçon ! tu as été celui qui a vu son camarade nu dans le froid de l'hiver ; et il prit de l'eau froide et la versa sur lui.

\par 10 Ô mon garçon ! tu as été envers moi comme un homme qui prend une pierre et la jette au ciel pour lapider son Seigneur avec. Et la pierre n’a pas frappé et n’est pas allée assez haut, mais elle est devenue la cause de la culpabilité et du péché.

\par 11 Ô mon garçon ! si tu m'avais honoré, respecté et écouté mes paroles, tu aurais été mon héritier et tu aurais régné sur mes domaines.

\par 12 Ô mon fils ! sache que si la queue du chien ou du cochon mesurait dix coudées de long, elle n'atteindrait pas la valeur de celle du cheval, même si elle était comme de la soie.

\par 13 Ô mon garçon ! Je pensais que tu aurais été mon héritier à ma mort ; et toi, par envie et par insolence, tu as voulu me tuer. Mais le Seigneur m'a délivré de ta ruse.

\par 14 Ô mon fils ! tu as été pour moi comme un piège tendu sur le fumier, et un moineau est venu et a trouvé le piège tendu. Et le moineau dit au piège : « Que fais-tu ici ? » Le piège a dit : « Je prie Dieu ici. »

\par 15 Et l'alouette lui demanda aussi : « Quel est le morceau de bois que tu tiens ? Le piège dit : «C'est un jeune chêne sur lequel je m'appuie au moment de la prière.»

\par 16 L'alouette dit : «Et qu'est-ce que c'est que cette chose dans ta bouche ?» Le piège disait : «C'est du pain et des vivres que je porte pour tous les affamés et les pauvres qui s'approchent de moi.»

\par 17 L'alouette dit : «Maintenant, puis-je m'avancer et manger, car j'ai faim ?» Et le piège lui dit : « Avance-toi ». Et l'alouette s'est approchée pour qu'elle puisse manger.

\par 18 Mais le piège se leva et saisit l'alouette par le cou.

\par 19 Et l'alouette répondit et dit au piège : « Si c'est ton pain pour celui qui a faim, Dieu n'accepte pas ton aumône et tes bonnes actions.

\par 20 Et si tels sont ton jeûne et tes prières, Dieu n'accepte de toi ni ton jeûne ni ta prière, et Dieu ne perfectionnera pas ce qui est bon à ton sujet.

\par 21 Ô mon garçon ! tu as été pour moi comme un lion qui s'est lié d'amitié avec un âne, et l'âne a marché pendant un certain temps devant le lion ; et un jour, le lion sauta sur l'âne et le mangea.

\par 22 Ô mon garçon ! tu as été pour moi comme un charançon dans le blé, car il ne sert à rien, mais il gâte le blé et le ronge.

\par 23 Ô mon garçon ! tu es comme un homme qui a semé dix mesures de blé, et quand c'était le temps de la moisson, il se levait et le moissonnait, et l'engloutissait, le battait et travaillait dur jusqu'au bout, et il s'est avéré qu'il y en avait dix mesures, et son maître lui dit : « Ô toi, paresseux ! tu n'as pas grandi et tu n'as pas diminué.

\par 24 Ô mon garçon ! tu as été pour moi comme la perdrix qu'on avait jetée dans le filet, et elle n'a pas pu se sauver, mais elle a appelé les perdrix, pour qu'elle les jette avec elle dans le filet.

\par 25 Ô mon fils ! tu as été pour moi comme le chien qui avait froid et qui entrait dans la maison du potier pour se réchauffer.

\par 26 Et quand il fut réchauffé, il se mit à aboyer contre eux, et ils le chassèrent et le frappèrent, afin qu'il ne les mordait pas.

\par 27 Ô mon fils ! tu as été pour moi comme le cochon qui entrait dans le bain chaud avec des gens de qualité, et quand il sortait du bain chaud, il vit un trou sale et il descendit et s'y vautra.

\par 28 Ô mon fils ! tu as été pour moi comme le bouc qui a rejoint ses camarades en route vers le sacrifice, et il n'a pas pu se sauver.

\par 29 Ô mon garçon ! le chien qui n'est pas nourri de sa chasse devient de la nourriture pour les mouches.

\par 30 Ô mon fils ! la main qui ne travaille pas et ne laboure pas, et qui est cupide et rusée, sera retranchée de son épaule.

\par 31 Ô mon fils ! l'œil dans lequel la lumière n'est pas vue, les corbeaux le gratteront et l'arracheront.

\par 32 Ô mon garçon ! tu as été pour moi comme un arbre dont on coupait les branches, et il leur disait : « Si quelque chose de moi n'était pas entre vos mains, en vérité vous ne pourriez pas me couper.

\par 33 Ô mon garçon ! tu es comme le chat à qui on disait : « Arrête de voler jusqu'à ce que nous te fabriquions une chaîne d'or et te nourrissions de sucre et d'amandes. »

\par 34 Et elle dit : 'Je n'oublie pas le métier de mon père et de ma mère.'

\par 35 Ô mon fils ! tu as été comme le serpent chevauchant un buisson d'épines lorsqu'il était au milieu d'une rivière, et un loup les vit et dit : « Méfait sur méfait, et que celui qui est plus méchant qu'eux les dirige tous les deux.

\par 36 Et le serpent dit au loup : Les agneaux, les chèvres et les brebis dont tu as mangé toute ta vie, les rendras-tu à leurs pères et à leurs parents, ou non ?

\par 37 Le loup dit : « Non ». Et le serpent lui dit : « Je pense qu'après moi tu es le pire d'entre nous. »

\par 38 'Ô mon garçon ! Je t'ai nourri de bonne nourriture et tu ne m'as pas nourri de pain sec.

\par 39 'Ô mon garçon ! Je t'ai donné à boire de l'eau sucrée et du bon sirop, et tu ne m'as pas donné à boire de l'eau du puits.

\par 40 'Ô mon garçon ! Je t'ai enseigné et élevé, et tu m'as creusé une cachette et tu m'as caché.

\par 41 'Ô mon garçon ! Je t'ai élevé dans la meilleure éducation et je t'ai dressé comme un grand cèdre ; et tu m'as tordu et plié.

\par 42 'Ô mon garçon ! j'espérais à ton sujet que tu me bâtirais un château fort, afin que j'y sois caché de mes ennemis, et tu es devenu pour moi comme quelqu'un qui est enterré dans les profondeurs de la terre ; mais l'Éternel a eu pitié de moi et m'a délivré de ta ruse.

\par 43 'Ô mon garçon ! Je t'ai souhaité du bien, et tu m'as récompensé par le mal et la haine, et maintenant je voudrais t'arracher les yeux, et te faire de la nourriture pour chiens, et te couper la langue, et t'arracher la tête avec le tranchant de l'épée, et je te récompenserai pour tes actes abominables.

\par 44 Et quand Nadan entendit ce discours de son oncle Haiqâr, il dit : 'Ô mon oncle ! traite-moi selon ta connaissance, et pardonne-moi mes péchés, car qui a péché comme moi, ou qui pardonne comme toi ?

\par 45 'Accepte-moi, ô mon oncle ! Maintenant, je servirai dans ta maison, je panserai tes chevaux, je balayerai les excréments de ton bétail et je paîtrai tes brebis, car je suis le méchant et tu es le juste : moi le coupable et toi le pardonneur. 1

\par 46 Et Haiqâr lui dit : 'Ô mon garçon ! tu es comme l'arbre qui était stérile au bord de l'eau, et son maître voulait le couper, et il lui dit : « Emmène-moi ailleurs, et si je ne porte pas de fruit, coupe-moi. »

\par 47 Et son maître lui dit : Tu n'as pas porté de fruit étant près de l'eau, comment porteras-tu du fruit quand tu es dans un autre lieu ?

\par 48 'Ô mon garçon ! la vieillesse de l'aigle vaut mieux que la jeunesse du corbeau.

\par 49 'Ô mon garçon ! ils dirent au loup : « Éloigne-toi des brebis, de peur que leur poussière ne te fasse du mal. Et le loup dit : « La lie du lait de brebis est bonne pour mes yeux. »

\par 50 'Ô mon garçon ! ils ont fait aller le loup à l'école pour qu'il apprenne à lire et ils lui ont dit : « Dis A, B. » Il a dit : «Agneau et chèvre dans ma cloche»'

\par 51 'Ô mon garçon ! ils ont posé l'âne à table et il est tombé et a commencé à se rouler dans la poussière et l'un d'eux a dit : «Laissez-le se rouler, car c'est sa nature, il ne changera pas.»

\par 52 'Ô mon garçon ! le dicton a été confirmé : « Si tu engendres un garçon, appelle-le ton fils, et si tu élèves un garçon, appelle-le ton esclave. »

\par 53 'Ô mon garçon ! celui qui fait le bien rencontrera le bien ; et celui qui fait le mal rencontrera le mal, car l'Éternel rend à l'homme selon la mesure de son œuvre.

\par 54 'Ô mon garçon ! que te dirai-je de plus que ces paroles ? car le Seigneur sait ce qui est caché et connaît les mystères et les secrets.

\par 55 'Et il te récompensera et jugera entre moi et toi, et te récompensera selon ton mérite.',

\par 56 Et lorsque Nadan entendit ce discours de son oncle Haiqâr, il enfla aussitôt et devint comme une vessie gonflée.

\par 57 Et ses membres enflés, ses jambes, ses pieds et son côté, et il fut déchiré, et son ventre éclata, et ses entrailles furent dispersées, et il périt, et mourut.

\par 58 Et sa dernière fin fut la destruction, et il alla en enfer. Car celui qui creuse une fosse pour son frère y tombera ; et celui qui tend des pièges y sera pris.

\par 59 Voici ce qui s'est passé et (ce que) nous avons découvert à propos du conte de Haiqâr, et louange à Dieu pour toujours. Amen et paix.

\par 60 Cette chronique s'achève avec l'aide de Dieu, qu'Il soit exalté ! Amen, Amen, Amen.

\par \textit{Notes de bas de page}

\par \textit{218:1 Comparez la parabole du fils prodigue dans Luc XV. 19.}


\end{document}