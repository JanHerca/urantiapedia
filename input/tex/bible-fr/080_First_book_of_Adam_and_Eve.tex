\begin{document}

\title{Premier livre d'Adam et Ève}

\chapter{1}

\par \textit{La mer de cristal. Dieu ordonne à Adam, expulsé d'Eden, d'habiter dans la Grotte des Trésors.}

\par 1 Le troisième jour, Dieu planta le jardin à l'est de la terre, à la frontière du monde vers l'est, au-delà duquel, vers le lever du soleil, on ne trouve que de l'eau, qui entoure le monde entier et atteint jusqu'aux frontières du ciel.

\par 2 Et au nord du jardin il y a une mer de hostie, claire et pure au goût, comme rien d'autre ; afin que, grâce à sa clarté, on puisse regarder dans les profondeurs de la terre.

\par 3 Et quand un homme s'y lave, il devient pur de sa pureté, et blanc de sa blancheur, même s'il était sombre.

\par 4 Et Dieu créa cette mer selon son bon plaisir, car il savait ce qui arriverait de l'homme qu'il ferait ; afin qu'après qu'il eut quitté le jardin, à cause de sa transgression, des hommes naissent sur la terre, parmi lesquels mourraient des justes, dont Dieu ressusciterait les âmes au dernier jour ; quand ils devraient retourner à leur chair; devraient se baigner dans l'eau de cette mer, et tous se repentiront de leurs péchés.

\par 5 Mais quand Dieu fit sortir Adam du jardin, Il ne le plaça pas sur la limite nord de celui-ci, de peur qu'il ne s'approche de la mer d'eau, et que lui et Ève ne s'y lavent, et ne soient purifiés de leur péchés, oublient la transgression qu'ils avaient commise, et il ne s'en souvenait plus en pensant à leur châtiment.

\par 6 Puis, encore une fois, quant au côté sud du jardin, Dieu n'a pas plu à laisser Adam y habiter ; car, quand le vent soufflait du nord, il lui apportait, du côté sud, la délicieuse odeur des arbres du jardin.

\par 7 C'est pourquoi Dieu n'a pas placé Adam là, de peur qu'il ne sente la douce odeur de ces arbres, qu'il n'oublie sa transgression et qu'il ne trouve une consolation pour ce qu'il avait fait, qu'il ne se réjouisse de l'odeur des arbres et qu'il ne soit pas purifié de sa transgression. .

\par 8 Encore une fois, parce que Dieu est miséricordieux et d'une grande pitié, et qu'il gouverne toutes choses d'une manière que lui seul connaît, il a fait habiter notre père Adam à la limite occidentale du jardin, car de ce côté la terre est très large. .

\par 9 Et Dieu lui ordonna d'habiter là dans une grotte dans un rocher, la Grotte des Trésors au-dessous du jardin.

\chapter{2}

\par \textit{Adam et Eve s'évanouissent en quittant le Jardin. Dieu envoie Sa parole pour les encourager.}

\par 1 MAIS lorsque notre père Adam et Ève sont sortis du jardin, ils ont foulé le sol avec leurs pieds, ne sachant pas qu'ils marchaient.

\par 2 Et lorsqu'ils arrivèrent à l'ouverture de la porte du jardin, et qu'ils virent la vaste terre s'étendre devant eux, couverte de pierres grandes et petites, et de sable, ils furent effrayés et tremblèrent, et tombèrent la face contre terre, à cause du la peur qui les a envahis; et ils étaient comme morts.

\par 3 Parce que, alors qu'ils avaient été jusqu'ici dans un jardin magnifiquement planté d'arbres de toutes sortes, ils se voyaient maintenant dans un pays étranger, qu'ils ne connaissaient pas et qu'ils n'avaient jamais vu.

\par 4 Et parce qu'à cette époque ils étaient remplis de la grâce d'une nature lumineuse, et qu'ils n'avaient pas le cœur tourné vers les choses terrestres.

\par 5 C'est pourquoi Dieu eut pitié d'eux ; et quand il les vit tombés devant la porte du jardin, il envoya sa parole au père Adam et Ève, et les releva de leur état déchu.

\chapter{3}

\par \textit{Concernant la promesse des cinq grands jours et demi.}

\par 1 DIEU dit à Adam : « J'ai ordonné sur cette terre des jours et des années, et toi et ta postérité y habiterez et y marcherez, jusqu'à ce que les jours et les années soient accomplis ; quand j'enverrai la Parole qui t'a créé et contre laquelle tu as transgressé, la Parole qui t'a fait sortir du jardin et qui t'a relevé quand tu étais tombé.

\par 2 « Oui, la Parole qui te sauvera de nouveau lorsque les cinq jours et demi seront accomplis. »

\par 3 Mais quand Adam entendit ces paroles de Dieu, et des cinq grands jours et demi, il n'en comprit pas le sens.

\par 4 Car Adam pensait qu'il ne lui resterait que cinq jours et demi jusqu'à la fin du monde.

\par 5 Et Adam pleura et pria Dieu de lui expliquer cela.

\par 6 Alors Dieu, dans sa miséricorde envers Adam, qui avait été créé à son image et à sa similitude, lui expliqua que ces durées étaient de 5 000 et 500 ans ; et comment Quelqu'un viendrait alors le sauver, lui et sa postérité.

\par 7 Mais Dieu avait auparavant conclu cette alliance avec notre père Adam, dans les mêmes termes, avant qu'il ne sorte du jardin, alors qu'il était près de l'arbre dont Eve prit le fruit et le lui donna à manger.

\par 8 Dans la mesure où lorsque notre père Adam est sorti du jardin, il est passé près de cet arbre et a vu comment Dieu en avait alors changé l'apparence en une autre forme, et comment il se desséchait.

\par 9 Et tandis qu'Adam s'y dirigeait, il eut peur, trembla et tomba ; mais Dieu, dans sa miséricorde, l'a élevé et a ensuite conclu cette alliance avec lui.

\par 10 Et encore, quand Adam était près de la porte du jardin, et qu'il vit le chérubin avec une épée de feu fulgurant dans sa main, et que le chérubin se mit en colère et fronça les sourcils contre lui, Adam et Ève eurent peur de lui, et pensait qu'il avait l'intention de les mettre à mort. Ils tombèrent donc la face contre terre et tremblèrent de peur.

\par 11 Mais il eut pitié d'eux et leur fit miséricorde ; et se détournant d'eux, il monta au ciel, pria le Seigneur et dit :

\par 12 «Seigneur, tu m'as envoyé veiller à la porte du jardin, avec une épée de feu.»

\par 13 « Mais quand tes serviteurs, Adam et Ève, m'ont vu, ils sont tombés la face contre terre et étaient comme morts. Ô mon Seigneur, que ferons-nous à tes serviteurs ?

\par 14 Alors Dieu eut pitié d'eux, leur fit miséricorde et envoya son ange pour garder le jardin.

\par 15 Et la Parole du Seigneur fut adressée à Adam et Ève, et les ressuscita.

\par 16 Et le Seigneur dit à Adam : « Je t'ai dit qu'au bout de cinq jours et demi, j'enverrai ma Parole et je te sauverai. »

\par 17 « Fortifie donc ton cœur, et demeure dans la Caverne des Trésors, dont je t'ai déjà parlé. »

\par 18 Et quand Adam entendit cette Parole de Dieu, il fut consolé par ce que Dieu lui avait dit. Car Il lui avait dit comment Il le sauverait.

\chapter{4}

\par \textit{Adam déplore le changement des conditions. Adam et Ève entrent dans la Grotte des Trésors.}

\par 1 MAIS Adam et Eve pleurèrent d'être sortis du jardin, leur première demeure.

\par 2 Et en effet, quand Adam regarda sa chair qui était altérée, il pleura amèrement, lui et Ève, sur ce qu'ils avaient fait. Et ils marchèrent et descendirent doucement dans la Grotte des Trésors.

\par 3 Et comme ils y arrivaient, Adam pleura sur lui-même et dit à Ève : « Regarde cette grotte qui doit être notre prison dans ce monde et un lieu de châtiment ! »

\par 4 « Qu'est-ce que c'est comparé au jardin ? Quelle est son étroitesse par rapport à l’espace de l’autre ?

\par 5 « Quel est ce rocher, à côté de ces bosquets ? Qu’est-ce que l’obscurité de cette caverne, comparée à la lumière du jardin ?

\par 6 « Qu'est-ce que ce rebord rocheux surplombant pour nous abriter, comparé à la miséricorde du Seigneur qui nous a couvert de son ombre ? »

\par 7 « Quel est le sol de cette grotte comparé au terrain du jardin ? Cette terre, parsemée de pierres ; et cela, planté de délicieux arbres fruitiers ?

\par 8 Et Adam dit à Ève : « Regarde tes yeux et les miens, qui voyaient auparavant des anges dans le ciel, louant ; et eux aussi, sans cesse.

\par 9 « Mais maintenant nous ne voyons plus comme nous le faisions : nos yeux sont devenus de chair ; ils ne peuvent pas voir de la même manière qu’avant.

\par 10 Adam dit encore à Ève : « Qu'est-ce que notre corps aujourd'hui, comparé à ce qu'il était autrefois, lorsque nous habitions dans le jardin ?

\par 11 Après cela Adam n'aimait pas entrer dans la grotte, sous le rocher en surplomb ; et il n’y serait jamais entré non plus.

\par 12 Mais il s'inclina devant les ordres de Dieu ; et se dit : « Si je n’entre pas dans la grotte, je serai à nouveau un transgresseur. »

\chapter{5}

\par \textit{Dans lequel Eve fait une intercession noble et émouvante, prenant sur elle-même la responsabilité.}

\par 1 ALORS Adam et Ève entrèrent dans la grotte et prièrent, dans leur propre langue, inconnue de nous, mais qu'ils connaissaient bien.

\par 2 Et pendant qu'ils priaient, Adam leva les yeux et vit le rocher et le toit de la grotte qui le couvrait au-dessus de lui, de sorte qu'il ne pouvait voir ni le ciel, ni les créatures de Dieu. Alors il pleura et se frappa lourdement la poitrine, jusqu'à ce qu'il tombe et soit comme mort.

\par 3 Et Ève était assise et pleurait ; car elle le croyait mort.

\par 4 Alors elle se leva, étendit les mains vers Dieu, lui demandant miséricorde et pitié, et dit : « Ô Dieu, pardonne-moi mon péché, le péché que j'ai commis, et ne t'en souviens pas contre moi.

\par 5 « Car moi seul j'ai fait tomber ton serviteur du jardin dans ce domaine perdu ; de la lumière dans ces ténèbres ; et de la demeure de la joie dans cette prison.

\par 6 « Ô Dieu, regarde ton serviteur ainsi déchu, et relève-le de sa mort, afin qu'il pleure et se repente de la transgression qu'il a commise par moi. »

\par 7 « Ne lui enlevez pas son âme pour une fois ; mais laisse-le vivre afin qu'il puisse se tenir debout après la mesure de sa repentance et faire ta volonté, comme avant sa mort.

\par 8 « Mais si tu ne le relèves pas, alors, ô Dieu, enlève mon âme, afin que je sois semblable à lui ; et ne me laisse pas dans ce cachot, seul et seul ; car je ne pouvais pas rester seul dans ce monde, mais seulement avec lui.

\par 9 «Car toi, ô Dieu, tu as provoqué le sommeil sur lui, tu as ôté un os de son côté, et tu as restauré la chair à la place, par ta puissance divine.»

\par 10 « Et tu m'as pris, moi, l'os, et tu as fait de moi une femme, brillante comme lui, avec un cœur, une raison et une parole ; et en chair, comme le sien ; et tu m'as créé à l'image de son visage, par ta miséricorde et ta puissance.

\par 11 «O Seigneur, moi et lui sommes un et Toi, ô Dieu, tu es notre Créateur, Tu es Celui qui nous a créés tous les deux en un seul jour.»

\par 12 « C'est pourquoi, ô Dieu, donne-lui la vie, afin qu'il soit avec moi dans ce pays étranger, pendant que nous y habitons à cause de notre transgression. »

\par 13 « Mais si tu ne veux pas lui donner la vie, alors prends-moi, même moi, comme lui ; afin que nous puissions mourir tous les deux le même jour.

\par 14 Et Ève pleura amèrement et se jeta sur notre père Adam ; de son grand chagrin.

\chapter{6}

\par \textit{L'exhortation de Dieu à Adam et Ève dans laquelle il souligne comment et pourquoi ils ont péché.}

\par 1 MAIS Dieu les regarda ; car ils s'étaient suicidés dans un grand chagrin.

\par 2 Mais Il les relèverait et les réconforterait.

\par 3 Il leur envoya donc Sa Parole ; qu'ils devraient se lever et être relevés immédiatement.

\par 4 Et le Seigneur dit à Adam et Ève : « Vous avez transgressé de votre plein gré, jusqu'à ce que vous sortiez du jardin dans lequel je vous avais placé. »

\par 5 « De votre plein gré, vous avez transgressé par votre désir de divinité, de grandeur et d'un état exalté, tel que le mien ; de sorte que je t'ai privé de la nature lumineuse dans laquelle tu étais alors, et je t'ai fait sortir du jardin vers ce pays rude et plein de troubles.

\par 6 « Si seulement vous n'aviez pas transgressé mon commandement et observé ma loi, et n'aviez-vous pas mangé du fruit de l'arbre près duquel je vous avais dit de ne pas approcher ! Et il y avait des arbres fruitiers dans le jardin meilleurs que celui-là.

\par 7 « Mais le méchant Satan qui n'a pas persisté dans son premier état, ni gardé sa foi ; en qui il n'y avait aucune bonne intention à mon égard, et qui, bien que je l'avais créé, m'a pourtant méprisé et a cherché la Divinité, de sorte que je l'ai précipité du ciel, c'est lui qui a fait paraître l'arbre agréable à vos yeux. , jusqu'à ce que vous en mangiez, en l'écoutant.

\par 8 « Ainsi vous avez transgressé mon commandement, et c'est pourquoi je vous ai fait subir toutes ces peines. »

\par 9 « Car je suis Dieu le Créateur, qui, lorsque j'ai créé mes créatures, n'avait pas l'intention de les détruire. Mais après qu'ils eurent vivement éveillé ma colère, je les châtiai de terribles plaies jusqu'à ce qu'ils se repentent.

\par 10 «Mais si, au contraire, ils continuent à s'endurcir dans leur transgression, ils seront sous une malédiction pour toujours.»

\chapter{7}

\par \textit{Les bêtes sont réconciliées.}

\par 1 QUAND Adam et Ève entendirent ces paroles de Dieu, ils pleurèrent et sanglotèrent encore davantage ; mais ils fortifièrent leur cœur en Dieu, parce qu'ils sentaient maintenant que le Seigneur était pour eux comme un père et une mère ; et c’est précisément pour cette raison qu’ils pleuraient devant lui et lui demandaient miséricorde.

\par 2 Alors Dieu eut pitié d'eux et dit : « Ô Adam, j'ai fait mon alliance avec toi, et je ne m'en détournerai pas ; et je ne te laisserai pas non plus retourner au jardin, jusqu'à ce que mon alliance des cinq grands jours et demi soit accomplie.

\par 3 Alors Adam dit à Dieu : « Ô Seigneur, tu nous as créés et tu nous as rendus aptes à être dans le jardin ; et avant que je transgresse, tu as fait venir à moi toutes les bêtes, pour que je les nomme.

\par 4 « Ta grâce était alors sur moi ; et j'ai nommé chacun selon ta pensée ; et tu les as tous soumis à moi.

\par 5 « Mais maintenant, Seigneur Dieu, que j'ai transgressé ton commandement, toutes les bêtes se lèveront contre moi et me dévoreront, ainsi qu'Ève ta servante ; et il retranchera notre vie de la surface de la terre.

\par 6 «Je te supplie donc, ô Dieu, que, puisque tu nous as fait sortir du jardin et que tu nous as placés dans un pays étranger, tu ne laisses pas les bêtes nous faire du mal.»

\par 7 Lorsque le Seigneur entendit ces paroles d'Adam, il eut pitié de lui et sentit qu'il avait vraiment dit que les bêtes des champs se lèveraient et le dévoreraient ainsi qu'Ève, parce que Lui, le Seigneur, était en colère contre eux deux. à cause de leur transgression.

\par 8 Alors Dieu ordonna aux bêtes, aux oiseaux et à tout ce qui se meut sur la terre, de venir à Adam et de se familiariser avec lui, et de ne pas le troubler ni Ève ; ni encore aucun des bons et des justes parmi leur postérité.

\par 9 Alors les bêtes rendirent hommage à Adam, selon le commandement de Dieu ; sauf le serpent contre lequel Dieu était en colère. Cela n’est pas arrivé à Adam avec les bêtes.

\chapitre{8}

\par \textit{La « nature lumineuse » de l'homme est supprimée.}

\par 1 ALORS Adam pleura et dit : « Ô Dieu, lorsque nous habitions dans le jardin et que nos cœurs étaient élevés, nous avons vu les anges qui chantaient des louanges dans le ciel, mais maintenant nous ne voyons plus comme nous avions l'habitude de faire ; bien plus, lorsque nous sommes entrés dans la grotte, toute la création nous est devenue cachée.

\par 2 Alors Dieu le Seigneur dit à Adam : « Quand tu étais soumis à moi, tu avais en toi une nature lumineuse, et c'est pour cette raison que tu pouvais voir les choses de loin. Mais après ta transgression, ta nature brillante s'est retirée de toi ; et il ne t'était pas laissé de voir les choses de loin, mais seulement de près ; après la capacité de la chair; car c’est brutal.

\par 3 Quand Adam et Ève eurent entendu ces paroles de Dieu, ils s'en allèrent ; le louant et l’adorant avec un cœur triste.

\par 4 Et Dieu cessa de communier avec eux.

\chapitre{9}

\par \textit{Eau de l'Arbre de Vie. Adam et Ève sont sur le point de se noyer.}

\par 1 ALORS Adam et Ève sortirent de la Caverne des Trésors et s'approchèrent de la porte du jardin, et là ils se tinrent pour la regarder et pleurèrent d'en être sortis.

\par 2 Et Adam et Ève sortirent de devant la porte du jardin vers le côté sud de celui-ci, et trouvèrent là l'eau qui arrosait le jardin, depuis la racine de l'Arbre de Vie, et qui se séparait de là en quatre rivières. sur la terre.

\par 3 Alors ils s'approchèrent de cette eau et la regardèrent ; et je vis que c'était l'eau qui sortait de sous la racine de l'Arbre de Vie dans le jardin.

\par 4 Et Adam pleura et se lamenta, et se frappa la poitrine, parce qu'il avait été séparé du jardin ; et dit à Ève :

\par 5 «Pourquoi as-tu amené sur moi, sur toi-même et sur notre postérité, tant de ces fléaux et châtiments ?»

\par 6 Et Ève lui dit : Qu'as-tu vu, pour pleurer et me parler ainsi ?

\par 7 Et il dit à Ève : Ne vois-tu pas cette eau qui était avec nous dans le jardin, qui arrosait les arbres du jardin, et qui coulait de là ?

\par 8 « Et nous, quand nous étions dans le jardin, nous ne nous en souciions pas ; mais depuis que nous sommes arrivés dans ce pays étranger, nous l’aimons et nous l’utilisons pour notre corps.

\par 9 Mais quand Ève entendit ces paroles de sa part, elle pleura ; et à cause de la douleur de leurs pleurs, ils tombèrent dans cette eau ; et y auraient mis fin, pour ne plus jamais revenir et contempler la création ; car lorsqu'ils contemplaient l'œuvre de la création, ils sentaient qu'ils devaient mettre fin à eux-mêmes.

\chapitre{10}

\par \textit{Leurs corps ont besoin d'eau après avoir quitté le Jardin.}

\par 1 ALORS Dieu, miséricordieux et miséricordieux, les regarda ainsi couchés dans l'eau, et proches de la mort, et envoya un ange qui les fit sortir de l'eau et les déposa sur le bord de la mer comme morts.

\par 2 Alors l'ange s'approcha de Dieu, fut le bienvenu et dit : « Ô Dieu, tes créatures ont rendu leur dernier soupir. »

\par 3 Alors Dieu envoya Sa Parole à Adam et Ève, qui les ressuscitèrent de leur mort.

\par 4 Et Adam dit, après avoir été ressuscité : « Ô Dieu, pendant que nous étions dans le jardin, nous n'avions pas besoin de cette eau, ni ne nous en souciions ; mais depuis que nous sommes venus sur cette terre, nous ne pouvons plus nous en passer.

\par 5 Alors Dieu dit à Adam : « Pendant que tu étais sous mon commandement et que tu étais un ange brillant, tu ne connaissais pas cette eau. »

\par 6 « Mais après avoir transgressé mon commandement, tu ne peux plus te passer d'eau pour laver ton corps et le faire grandir ; car il ressemble maintenant à celui des bêtes et manque d’eau.

\par 7 Quand Adam et Ève entendirent ces paroles de Dieu, ils pleurèrent amèrement ; et Adam supplia Dieu de le laisser retourner dans le jardin et de le regarder une seconde fois.

\par 8 Mais Dieu dit à Adam : « Je t'ai fait une promesse ; quand cette promesse sera accomplie, je te ramènerai dans le jardin, toi et ta juste postérité.

\par 9 Et Dieu cessa de communier avec Adam.

\chapitre{11}

\par \textit{Un souvenir des jours glorieux dans le Jardin.}

\par 1 ALORS Adam et Ève se sentirent brûler de soif, de chaleur et de chagrin.

\par 2 Et Adam dit à Ève : « Nous ne boirons pas de cette eau, même si nous mourrions. Ô Ève, lorsque cette eau entrera dans nos entrailles, elle augmentera nos châtiments et ceux de nos enfants, qui viendront après nous.

\par 3 Adam et Ève se retirèrent alors de l'eau et n'en burent pas du tout ; mais il est venu et est entré dans la Grotte des Trésors.

\par 4 Mais une fois dedans, Adam ne pouvait pas voir Ève ; il n'entendait que le bruit qu'elle faisait. Elle ne pouvait pas non plus voir Adam, mais entendait le bruit qu'il faisait.

\par 5 Alors Adam pleura, dans une profonde affliction, et se frappa la poitrine ; et il se leva et dit à Ève : « Où es-tu ?

\par 6 Et elle lui dit : « Voici, je me tiens dans ces ténèbres. »

\par 7 Il lui dit alors : « Souviens-toi de la nature lumineuse dans laquelle nous vivions, pendant que nous demeurions dans le jardin !

\par 8 « Ô Ève ! souvenez-vous de la gloire qui reposait sur nous dans le jardin. Ô Ève ! souviens-toi des arbres qui nous éclipsaient dans le jardin pendant que nous nous déplacions parmi eux.

\par 9 « Ô Ève ! rappelez-vous que lorsque nous étions dans le jardin, nous ne connaissions ni la nuit ni le jour. Pensez à l’Arbre de Vie, d’en bas duquel coulait l’eau, et qui nous éclairait ! Souviens-toi, ô Ève, du jardin et de sa luminosité !

\par 10 «Pensez, oh pensez à ce jardin dans lequel il n'y avait pas d'obscurité, pendant que nous y habitions.»

\par 11 « A peine sommes-nous entrés dans cette Caverne aux Trésors que l'obscurité nous a environné ; jusqu'à ce qu'on ne puisse plus se voir ; et tout le plaisir de cette vie a pris fin.

\chapitre{12}

\par \textit{Comment les ténèbres sont survenues entre Adam et Ève.}

\par 1 ALORS Adam se frappa la poitrine, lui et Ève, et ils pleurèrent toute la nuit jusqu'à l'aube, et ils soupirèrent pendant toute la nuit à Miyazia.

\par 2 Et Adam se frappa et se jeta à terre dans la grotte, à cause d'un chagrin amer et à cause des ténèbres, et il resta là comme mort.

\par 3 Mais Eve entendit le bruit qu'il faisait en tombant sur la terre. Et elle le chercha avec ses mains et le trouva comme un cadavre.

\par 4 Alors elle eut peur, resta muette et resta près de lui.

\par 5 Mais le Seigneur miséricordieux a regardé la mort d'Adam et le silence d'Ève par crainte des ténèbres.

\par 6 Et la Parole de Dieu vint à Adam et le ressuscita de sa mort, et ouvrit la bouche d'Ève pour qu'elle puisse parler.

\par 7 Alors Adam se leva dans la grotte et dit : « Ô Dieu, pourquoi la lumière nous a-t-elle quittés et les ténèbres sont-elles tombées sur nous ? Pourquoi nous laisses-tu dans cette longue obscurité ? Pourquoi nous tourmentes-tu ainsi ?

\par 8 « Et ces ténèbres, ô Seigneur, où étaient-elles avant qu'elles ne nous arrivent ? C’est tel que nous ne pouvons pas nous voir.

\par 9 «Car, tant que nous étions dans le jardin, nous n'avons ni vu ni même su ce que sont les ténèbres. Je n'étais pas caché à Ève, et elle ne m'était pas cachée non plus, jusqu'à présent qu'elle ne puisse me voir ; et aucune obscurité n’est venue sur nous pour nous séparer les uns des autres.

\par 10 « Mais elle et moi étions tous les deux dans une même lumière. Je l'ai vue et elle m'a vu. Pourtant, depuis que nous sommes entrés dans cette grotte, les ténèbres sont tombées sur nous et nous ont séparés, de sorte que je ne la vois pas, et elle ne me voit pas.

\par 11 « Ô Seigneur, vas-tu donc nous tourmenter avec ces ténèbres ?

\chapitre{13}

\par \textit{La chute d'Adam. Pourquoi la nuit et le jour ont été créés.}

\par 1 PUIS, lorsque Dieu, qui est miséricordieux et plein de pitié, entendit la voix d'Adam, il lui dit : —

\par 2 « Ô Adam, tant que le bon ange M'était obéissant, une lumière brillante reposait sur lui et sur ses hôtes. »

\par 3 «Mais quand il a transgressé Mon commandement, je l'ai privé de cette nature lumineuse, et il est devenu sombre.»

\par 4 «Et lorsqu'il était dans les cieux, dans les royaumes de lumière, il ne connaissait rien des ténèbres.»

\par 5 « Mais il a transgressé, et je l'ai fait tomber du ciel sur la terre ; et ce sont ces ténèbres qui l’ont envahi.

\par 6 «Et sur toi, ô Adam, tandis que dans mon jardin et m'obéissant, cette lumière brillante reposait aussi.»

\par 7 «Mais quand j'ai entendu parler de ta transgression, je t'ai privé de cette lumière brillante. Pourtant, par ma miséricorde, je ne t'ai pas transformé en ténèbres, mais j'ai fait de toi ton corps de chair, sur lequel j'ai étendu cette peau, afin qu'elle supporte le froid et la chaleur.

\par 8 « Si j'avais laissé ma colère tomber lourdement sur toi, je t'aurais détruit ; et si je t'avais transformé dans les ténèbres, cela aurait été comme si je t'avais tué.

\par 9 « Mais dans ma miséricorde, je t'ai fait tel que tu es ; Quand tu as transgressé Mon commandement, ô Adam, je t'ai chassé du jardin et je t'ai fait sortir dans ce pays ; et t'a ordonné d'habiter dans cette grotte ; et les ténèbres sont tombées sur toi, comme sur celui qui a transgressé mon commandement.

\par 10 « Ainsi, ô Adam, cette nuit t'a trompé. Cela ne durera pas éternellement ; mais il ne dure que douze heures ; quand ce sera fini, la lumière du jour reviendra.

\par 11 « Ne soupirez donc pas et ne soyez pas ému ; et ne dis pas dans ton cœur que ces ténèbres sont longues et traînent avec lassitude ; et ne dis pas dans ton cœur que je t'en tourmente.

\par 12 « Fortifie ton cœur et n'aie pas peur. Cette obscurité n'est pas une punition. Mais, ô Adam, j'ai créé le jour et j'y ai placé le soleil pour l'éclairer ; afin que toi et tes enfants puissiez faire votre travail.

\par 13 «Car je savais que tu devrais pécher et transgresser, et sortir dans ce pays. Pourtant je ne voudrais pas te forcer, ni être entendu contre toi, ni me taire ; ni te condamner par ta chute ; ni par ta sortie de la lumière dans les ténèbres ; ni encore par ton passage du jardin dans ce pays.

\par 14 « Car je t'ai fait de la lumière ; et j’ai voulu faire sortir de toi des enfants de lumière et comme toi.

\par 15 « Mais tu n'as pas gardé un jour mon commandement ; jusqu'à ce que j'aie terminé la création et béni tout ce qu'elle contient.

\par 16 « Alors je t'ai ordonné concernant l'arbre, de ne pas en manger. Pourtant, je savais que Satan, qui s’était trompé lui-même, te séduirait aussi.

\par 17 « Alors je t'ai fait connaître au moyen de l'arbre de ne pas t'approcher de lui. Et je t'ai dit de ne pas manger de son fruit, ni d'en goûter, ni de t'asseoir dessous, ni de s'y abandonner.

\par 18 « Si je ne t'avais pas parlé, ô Adam, au sujet de l'arbre, et si je t'avais laissé sans commandement, et que tu avais péché, cela aurait été une offense de ma part, de ne t'avoir pas donné de commandement. commande; tu te retournerais et me blâmerais pour cela.

\par 19 «Mais je te l'ai commandé et je t'ai averti, et tu es tombé. Pour que Mes créatures ne puissent pas Me blâmer ; mais la faute en incombe uniquement à eux.

\par 20 « Et, ô Adam, j'ai fait un jour pour toi et pour tes enfants après toi, pour qu'ils y travaillent et y peinent. Et j'ai préparé la nuit pour qu'ils s'y reposent de leur travail ; et que les bêtes des champs sortent la nuit et cherchent leur nourriture.

\par 21 « Mais il reste peu de ténèbres maintenant, ô Adam ; et bientôt la lumière du jour apparaîtra.

\chapitre{14}

\par \textit{La première prophétie de la venue du Christ.}

\par 1 ALORS Adam dit à Dieu : « Ô Seigneur, prends mon âme, et ne me laisse plus voir cette obscurité ; ou emmène-moi dans un endroit où il n'y a pas d'obscurité.

\par 2 Mais Dieu le Seigneur dit à Adam : « En vérité, je te le dis, ces ténèbres passeront loin de toi, chaque jour que j'ai déterminé pour toi, jusqu'à l'accomplissement de mon alliance ; quand je te sauverai et te ramènerai dans le jardin, dans la demeure de lumière que tu désires tant, où il n'y a pas d'obscurité. Je t’y amènerai, dans le royaume des cieux.

\par 3 Dieu dit encore à Adam : « Toute cette misère que tu as dû prendre sur toi à cause de ta transgression ne te libérera pas de la main de Satan et ne te sauvera pas. »

\par 4 « Mais je le ferai. Quand je descendrai du ciel, que je deviendrai chair de ta semence et que je prendrai sur moi l'infirmité dont tu souffres, alors les ténèbres qui sont tombées sur toi dans cette grotte viendront sur moi dans la tombe, quand je serai dans le chair de ta semence.

\par 5 « Et moi, qui suis sans années, je serai soumis au calcul des années, des temps, des mois et des jours, et je serai compté comme l'un des fils des hommes, afin de te sauver. »

\par 6 Et Dieu cessa de communier avec Adam.

\chapitre{15}

\par 1 ALORS Adam et Ève pleurèrent et furent attristés à cause de la parole que Dieu leur avait adressée, selon laquelle ils ne devaient pas retourner au jardin jusqu'à l'accomplissement des jours qui leur avaient été décrétés ; mais surtout parce que Dieu leur avait dit qu'il devait souffrir pour leur salut.

\chapitre{16}

\par \textit{Le premier lever de soleil. Adam et Ève pensent que c'est un feu qui vient les brûler.}

\par 1 APRÈS cela, Adam et Ève ne cessèrent de se tenir dans la grotte, priant et pleurant, jusqu'à ce que le matin se lève sur eux.

\par 2 Et quand ils virent que la lumière leur revenait, ils se retinrent de peur et fortifièrent leur cœur.

\par 3 Alors Adam commença à sortir de la grotte. Et lorsqu'il arriva à l'embouchure, qu'il se leva et tourna sa face vers l'est, qu'il vit le soleil se lever avec des rayons brillants et qu'il sentit sa chaleur sur son corps, il en eut peur et pensa dans son cœur que cette flamme est sortie pour le tourmenter.

\par 4 Il pleura alors, se frappa la poitrine, tomba la face contre terre et fit sa demande, en disant :

\par 5 «O Seigneur, ne me tourmente pas, ne me consume pas, et n'enlève pas ma vie de la terre.»

\par 6 Car il pensait que le soleil était Dieu.

\par 7 Dans la mesure où pendant qu'il était dans le jardin et qu'il entendait la voix de Dieu et le son qu'il faisait dans le jardin, et qu'il le craignait, Adam n'a jamais vu la brillante lumière du soleil, et sa chaleur flamboyante n'a pas non plus touché son corps.

\par 8 C'est pourquoi il avait peur du soleil lorsque ses rayons flamboyants l'atteignaient. Il pensait que Dieu voulait le tourmenter avec cela tous les jours qu'Il avait décrétés pour lui.

\par 9 Car Adam a aussi dit dans ses pensées, comme Dieu ne nous a pas tourmentés de ténèbres, voici, il a fait lever ce soleil et nous tourmenter d'une chaleur brûlante.

\par 10 Mais pendant qu'il pensait ainsi dans son cœur, la Parole de Dieu lui vint et dit :

\par 11 « Ô Adam, lève-toi et lève-toi. Ce soleil n'est pas Dieu ; mais il a été créé pour éclairer le jour, ce dont je t'ai parlé dans la grotte en disant : « que l'aube se lèverait et qu'il y aurait de la lumière le jour ».

\par 12 «Mais je suis Dieu qui t'ai consolé pendant la nuit.»

\par 13 Et Dieu cessa de communier avec Adam.



\chapitre{17}

\par \textit{Le Chapitre du Serpent.}

\par 1 ALORS Adam et Eve sortirent par l'entrée de la grotte, et se dirigèrent vers le jardin.

\par 2 Mais alors qu'ils s'en approchaient, devant la porte occidentale, d'où venait Satan lorsqu'il trompait Adam et Ève, ils trouvèrent le. serpent devenu Satan venant à la porte, léchant tristement la poussière et se tortillant sur sa poitrine à terre, à cause de la malédiction qui lui était tombée de la part de Dieu.

\par 3 Et tandis qu'autrefois le serpent était le plus élevé de tous les animaux, maintenant il fut changé et devint glissant, et le plus méchant de tous, et il se glissa sur sa poitrine et marcha sur son ventre.

\par 4 Et alors qu'elle était la plus belle de toutes les bêtes, elle avait été changée et était devenue la plus laide de toutes. Au lieu de se nourrir de la meilleure nourriture, il se tourna désormais vers la poussière. Au lieu de demeurer, comme autrefois, dans les meilleurs endroits, il vivait désormais dans la poussière.

\par 5 Et, alors qu'elle avait été la plus belle de toutes les bêtes, dont toutes restaient muettes devant sa beauté, elle était maintenant abhorrée d'eux.

\par 6 Et encore, alors qu'il habitait dans une belle demeure, où tous les autres animaux venaient d'ailleurs ; et là où il buvait, ils en buvaient aussi ; maintenant, après qu'il soit devenu venimeux, à cause de la malédiction de Dieu, toutes les bêtes s'enfuirent de sa demeure et ne voulurent pas boire de l'eau qu'elle buvait ; mais il s'enfuit.

\chapitre{18}

\par \textit{Le combat mortel avec le serpent.}

\par 1 QUAND le serpent maudit aperçut Adam et Ève, il enfla la tête, se dressa sur sa queue et, les yeux rouge sang, fit comme s'il voulait les tuer.

\par 2 Il se dirigea droit vers Ève et courut après elle ; tandis qu'Adam, debout, pleurait parce qu'il n'avait pas de bâton dans la main pour frapper le serpent, et ne savait pas comment le mettre à mort.

\par 3 Mais le cœur brûlant pour Ève, Adam s'approcha du serpent et le tint par la queue ; quand il se tourna vers lui et lui dit :

\par 4 « Ô Adam, à cause de toi et d'Ève, je suis glissant, et je vais sur mon ventre. » Puis, à cause de sa grande force, il renversa Adam et Ève et se pressa contre eux, comme s'il voulait les tuer.

\par 5 Mais Dieu envoya un ange qui chassa loin d'eux le serpent et les releva.

\par 6 Alors la Parole de Dieu s'adressa au serpent et lui dit : « Au début, je t'ai rendu désinvolte, et je t'ai fait marcher sur ton ventre ; mais je ne t'ai pas privé de la parole.

\par 7 « Maintenant, cependant, sois muet ; et ne parle plus, toi et ta race ; parce que d’abord c’est par toi que s’est produite la ruine de mes créatures, et maintenant tu veux les tuer.

\par 8 Alors le serpent fut rendu muet et ne parla plus.

\par 9 Et un vent vint souffler du ciel par ordre de Dieu qui enleva le serpent d'Adam et Ève, le jeta sur le bord de la mer, et il atterrit en Inde.

\chapitre{19}

\par \textit{Bêtes soumises à Adam.}

\par 1 MAIS Adam et Eve ont pleuré devant Dieu. Et Adam lui dit :

\par 2 «O Seigneur, quand j'étais dans la grotte, je t'ai dit ceci, mon Seigneur, que les bêtes des champs se lèveraient et me dévoreraient, et retrancheraient ma vie de la terre.»

\par 3 Alors Adam, à cause de ce qui lui était arrivé, se frappa la poitrine et tomba sur la terre comme un cadavre ; alors la Parole de Dieu lui vint, qui le ressuscita et lui dit :

\par 4 « Ô Adam, aucune de ces bêtes ne pourra te faire du mal ; parce que lorsque j'ai fait venir à toi les bêtes et autres êtres mobiles dans la grotte, je n'ai pas laissé le serpent venir avec eux, de peur qu'il ne se lève contre toi et ne te fasse trembler ; et la peur devrait tomber dans vos cœurs.

\par 5 « Car je savais que ce maudit est méchant ; c'est pourquoi je ne le laisserais pas s'approcher de vous avec les autres bêtes.

\par 6 « Mais maintenant, fortifie ton cœur et ne crains rien. Je suis avec toi jusqu'à la fin des jours que je t'ai fixés.

\chapitre{20}

\par \textit{Adam souhaite protéger Ève.}

\par 1 ALORS Adam pleura et dit : « Ô Dieu, emmène-nous ailleurs, afin que le serpent ne revienne pas près de nous et ne se lève pas contre nous. De peur qu'il ne trouve ta servante Eve seule et ne la tue ; car ses yeux sont hideux et mauvais.

\par 2 Mais Dieu dit à Adam et Ève : « Désormais, n'ayez crainte, je ne le laisserai pas s'approcher de vous ; Je l'ai chassé loin de toi, de cette montagne ; je n’y laisserai rien non plus qui puisse vous faire du mal.

\par 3 Alors Adam et Ève adorèrent Dieu, le remercièrent et le louèrent de les avoir délivrés de la mort.

\chapitre{21}

\par \textit{Adam et Ève tentent de se suicider.}

\par 1 ALORS Adam et Eve partirent à la recherche du jardin.

\par 2 Et la chaleur battait comme une flamme sur leurs visages ; et ils transpiraient à cause de la chaleur et pleuraient devant l'Éternel.

\par 3 Mais l'endroit où ils pleuraient était près d'une haute montagne, face à la porte occidentale du jardin.

\par 4 Alors Adam se jeta du haut de cette montagne ; son visage était tordu et sa chair était écorchée ; beaucoup de sang coulait de lui, et il était sur le point de mourir.

\par 5 Pendant ce temps, Ève restait debout sur la montagne, pleurant sur lui, ainsi couchée.

\par 6 Et elle dit : « Je ne veux pas vivre après lui ; car tout ce qu’il s’est fait, c’est par moi.

\par 7 Alors elle se jeta après lui ; et a été déchiré et écorché par des pierres ; et resta couché comme mort.

\par 8 Mais le Dieu miséricordieux, qui regarde ses créatures, regarda Adam et Ève alors qu'ils gisaient morts, et il leur envoya sa parole et les ressuscita.

\par 9 Et il dit à Adam : « Ô Adam, toute cette misère que tu t'es infligée ne servira à rien contre Mon règne, et elle ne modifiera pas non plus l'alliance des 5 500 ans. »

\chapitre{22}

\par \textit{Adam d'humeur chevaleresque.}

\par 1 ALORS Adam dit à Dieu : « Je me flétrit dans la chaleur ; Je m'évanouis à force de marcher et je déteste ce monde. Et je ne sais pas quand tu me sortiras de là pour me reposer.

\par 2 Alors le Seigneur Dieu lui dit : « Ô Adam, cela ne peut pas arriver maintenant, pas avant que tu n'aies fini tes jours. Alors je te ferai sortir de ce misérable pays.

\par 3 Et Adam dit à Dieu : « Pendant que j'étais dans le jardin, je n'ai connu ni chaleur, ni langueur, ni mouvement, ni tremblement, ni peur ; mais maintenant, depuis que je suis arrivé dans ce pays, toutes ces afflictions m’arrivent.

\par 4 Alors Dieu dit à Adam : « Tant que tu as gardé mon commandement, ma lumière et ma grâce reposaient sur toi. Mais quand tu as transgressé mon commandement, le chagrin et la misère t'ont frappé dans ce pays.

\par 5 Et Adam pleura et dit : « Ô Seigneur, ne me retranche pas pour cela, ne me frappe pas de lourdes plaies, et ne me rends pas encore selon mon péché ; Car nous avons, de notre propre volonté, transgressé ton commandement, abandonné ta loi et cherché à devenir des dieux semblables à toi, lorsque Satan, l'ennemi, nous a trompés.

\par 6 Alors Dieu dit encore à Adam : « Parce que tu as supporté la peur et le tremblement dans ce pays, la langueur et la souffrance en marchant et en marchant, en montant sur cette montagne et en en mourant, je prendrai tout cela sur moi afin de sauve-toi.»

\chapitre{23}

\par \textit{Adam et Ève se ceignent et construisent le premier autel jamais construit.}

\par 1 ALORS Adam pleura davantage et dit : « Ô Dieu, aie pitié de moi, jusqu'à prendre sur toi ce que je ferai. »

\par 2 Mais Dieu a pris Sa Parole d'Adam et Ève.

\par 3 Alors Adam et Ève se levèrent ; et Adam dit à Ève : « Ceins-toi, et moi aussi je me ceindrai. » Et elle se ceignit, comme Adam le lui avait dit.

\par 4 Alors Adam et Ève prirent des pierres et les placèrent en forme d'autel ; et ils prirent des feuilles des arbres du jardin, avec lesquelles ils essuyèrent, sur la surface du rocher, le sang qu'ils avaient répandu.

\par 5 Mais ce qui était tombé sur le sable, ils le prirent avec la poussière avec laquelle il était mêlé et l'offrèrent sur l'autel en offrande à Dieu.

\par 6 Alors Adam et Ève se tinrent sous l'autel et pleurèrent, suppliant ainsi Dieu : « Pardonne-nous notre offense 1 et notre péché, et regarde-nous avec ton œil de miséricorde. Car lorsque nous étions dans le jardin, nos louanges et nos hymnes montaient devant toi sans cesse.

\par 7 «Mais lorsque nous sommes arrivés dans ce pays étranger, nous n'avions plus ni louange pure, ni prière juste, ni cœurs intelligents, ni pensées douces, ni conseils justes, ni long discernement, ni sentiments droits, et notre nature lumineuse n'est pas non plus laissée. nous. Mais notre corps est changé par rapport à la similitude dans laquelle il était initialement, lorsque nous avons été créés.

\par 8 «Mais maintenant, regarde notre sang qui est offert sur ces pierres, et accepte-le de nos mains, comme la louange que nous te chantions au début, lorsque nous étions dans le jardin.»

\par 9 Et Adam commença à faire davantage de requêtes à Dieu.

\par \textit{Notes de bas de page}

\par \textit{16:1 ORIGINAL DE LA PRIÈRE DU SEIGNEUR DIT ÊTRE UTILISÉ ENVIRON 150 ANS AVANT NOTRE SEIGNEUR : Notre Père, qui es aux cieux, aie pitié de nous, Seigneur notre Dieu, que ton nom soit sanctifié, et que le que ton souvenir soit glorifié dans les cieux là-haut et sur la terre ici-bas.}

\par \textit{Que ton royaume règne sur nous maintenant et pour toujours. Les saints hommes d’autrefois disaient de remettre et de pardonner à tous les hommes tout ce qu’ils m’ont fait. Et ne nous soumets pas à la tentation, mais délivre-nous du mal ; car à Toi appartient le royaume et Tu régneras dans la gloire pour toujours et à jamais, AMEN.}

\chapitre{24}

\par \textit{Une prophétie vivante sur la vie et la mort du Christ.}

\par 1 DIX le Dieu miséricordieux, bon et amoureux des hommes, regarda Adam et Ève, et leur sang, qu'ils lui avaient offert en offrande ; sans un ordre de sa part pour le faire. Mais Il s’étonnait d’eux ; et accepta leurs offrandes.

\par 2 Et Dieu envoya de sa présence un feu brillant, qui consuma leur offrande.

\par 3 Il sentit la douce odeur de leur offrande et leur fit miséricorde.

\par 4 Alors la Parole de Dieu vint à Adam et lui dit : « Ô Adam, comme tu as versé ton sang, de même je verserai mon propre sang quand je deviendrai chair de ta postérité ; et comme tu es mort, ô Adam, ainsi je mourrai aussi. Et comme tu as bâti un autel, de même je te ferai un autel sur la terre ; et comme tu as offert ton sang dessus, de même j'offrirai mon sang sur un autel sur la terre.

\par 5 « Et comme tu as demandé le pardon par ce sang, ainsi aussi je ferai pardonner les péchés par mon sang, et j'effacerai les transgressions en lui. »

\par 6 « Et maintenant, voici, j'ai accepté ton offrande, ô Adam, mais les jours de l'alliance dans laquelle je t'ai lié ne sont pas accomplis. Quand elles seront accomplies, je te ramènerai dans le jardin.

\par 7 « Maintenant donc, fortifie ton cœur ; et quand le chagrin te surviendra, fais-moi une offrande, et je te serai favorable.

\chapitre{25}

\par \textit{Dieu représenté comme miséricordieux et aimant. L'établissement du culte.}

\par 1 MAIS Dieu savait qu'Adam avait en tête qu'il devait souvent se suicider et lui faire une offrande de son sang.

\par 2 C'est pourquoi il lui dit : « Ô Adam, ne te tue plus comme tu l'as fait, en te jetant du haut de cette montagne. »

\par 3 Mais Adam dit à Dieu : « J'avais dans l'idée de me mettre fin immédiatement, pour avoir transgressé tes commandements et pour être sorti du beau jardin ; et pour la lumière brillante dont tu m'as privé ; et pour les louanges qui sortaient sans cesse de ma bouche, et pour la lumière qui me couvrait.

\par 4 « Pourtant, par ta bonté, ô Dieu, ne m'éloigne pas complètement ; mais sois-moi favorable chaque fois que je mourrai et ramène-moi à la vie.

\par 5 « Et ainsi il sera fait connaître que tu es un Dieu miséricordieux, qui ne veut pas qu'un seul périsse ; qui n'aime pas qu'on tombe ; et qui ne condamne personne cruellement, mal et par destruction totale.

\par 6 Alors Adam resta silencieux.

\par 7 Et la Parole de Dieu vint à lui, et le bénit, et le consola, et fit alliance avec lui, qu'il le sauverait à la fin des jours qui lui étaient fixés.

\par 8 Ceci donc fut la première offrande qu'Adam fit à Dieu ; et c'est ainsi que c'est devenu sa coutume de le faire.

\chapitre{26}

\par \textit{Une belle prophétie de vie et de joie éternelles (v. 15). La tombée de la nuit.}

\par 1 PUIS Adam prit Eve, et ils commencèrent à retourner à la Grotte des Trésors où ils habitaient. Mais lorsqu’ils s’en approchèrent et le virent de loin, une profonde tristesse tomba sur Adam et Ève lorsqu’ils le regardèrent.

\par 2 Alors Adam dit à Ève : « Lorsque nous étions sur la montagne, nous avons été consolés par la Parole de Dieu qui parlait avec nous ; et la lumière qui venait de l’est a brillé sur nous.

\par 3 « Mais maintenant la Parole de Dieu nous est cachée ; et la lumière qui brillait sur nous est tellement changée qu'elle disparaît, et laisse les ténèbres et le chagrin venir sur nous.

\par 4 « Et nous sommes forcés d'entrer dans cette grotte qui est comme une prison, où les ténèbres nous couvrent, de sorte que nous sommes séparés les uns des autres ; et tu ne peux pas me voir, et je ne peux pas non plus te voir.

\par 5 Quand Adam eut dit ces paroles, ils pleurèrent et étendirent les mains devant Dieu ; car ils étaient pleins de tristesse.

\par 6 Et ils supplièrent Dieu de leur apporter le soleil, pour qu'il brille sur eux, afin que les ténèbres ne reviennent pas sur eux, et qu'ils ne reviennent plus sous cette couverture de rocher. Et ils préféraient mourir plutôt que de voir les ténèbres.

\par 7 Alors Dieu regarda Adam et Ève et leur grande tristesse, et tout ce qu'ils avaient fait avec un cœur fervent, à cause de tous les ennuis dans lesquels ils se trouvaient, au lieu de leur bien-être antérieur, et à cause de tous la misère qui les a frappés dans un pays étranger.

\par 8 C'est pourquoi Dieu n'était pas irrité contre eux ; ni impatient avec eux; mais il se montra patient et indulgent envers eux, comme envers les enfants qu'il avait créés.

\par 9 Alors la Parole de Dieu fut adressée à Adam et lui dit : « Adam, quant au soleil, si je le prenais et te l'apportais, les jours, les heures, les années et les mois seraient tous vains, et l’alliance que j’ai conclue avec toi ne sera jamais remplie.

\par 10 «Mais tu serais alors transformé et laissé dans une longue plaie, et aucun salut ne te serait laissé pour toujours.»

\par 11 « Oui, plutôt, patiente et calme ton âme pendant que tu demeures nuit et jour ; jusqu'à ce que les jours soient accomplis et que le temps de mon alliance soit venu.

\par 12 « Alors je viendrai et te sauverai, ô Adam, car je ne souhaite pas que tu sois affligé. »

\par 13 «Et quand je considère toutes les bonnes choses dans lesquelles tu as vécu, et pourquoi tu en es sorti, alors je te ferais volontiers preuve de miséricorde.»

\par 14 « Mais je ne peux pas modifier l'alliance qui est sortie de ma bouche ; sinon je t'aurais ramené dans le jardin.

\par 15 « Quand, cependant, l'alliance sera accomplie, alors je ferai preuve de miséricorde à toi et à ta semence, et je t'emmènerai dans un pays de joie, où il n'y a ni chagrin ni souffrance ; mais une joie et une allégresse constantes, une lumière qui ne faiblit jamais et des louanges qui ne cessent jamais ; et un beau jardin qui ne passera jamais.

\par 16 Et Dieu dit encore à Adam : « Sois patient et entre dans la grotte, car les ténèbres dont tu avais peur ne dureront que douze heures ; et une fois terminé, la lumière se lèvera.

\par 17 Alors, quand Adam entendit ces paroles de Dieu, lui et Ève adorèrent devant Lui, et leurs cœurs furent consolés. Ils retournèrent dans la grotte selon leur habitude, tandis que les larmes coulaient de leurs yeux, le chagrin et les lamentations sortaient de leur cœur, et ils souhaitaient que leur âme quitte leur corps.

\par 18 Et Adam et Ève restèrent debout en prière, jusqu'à ce que les ténèbres de la nuit les surprennent, et Adam fut caché à Ève, et elle à lui.

\par 19 Et ils restèrent debout en prière.

\chapitre{27}

\par \textit{La deuxième tentation d'Adam et Ève. Le diable prend la forme d'une lumière séduisante.}

\par 1 QUAND Satan, le haineux de tout bien, vit comment ils continuaient à prier, et comment Dieu communiait avec eux et les réconfortait, et comment il avait accepté leur offrande, Satan fit une apparition.

\par 2 Il commença par transformer ses hôtes ; dans ses mains il y avait un feu éclatant, et ils étaient dans une grande lumière.

\par 3 Il plaça alors son trône près de l'entrée de la grotte, car il ne pouvait y entrer à cause de leurs prières. Et il répandit la lumière dans la grotte, jusqu'à ce que la grotte brille sur Adam et Ève ; tandis que ses hôtes commençaient à chanter des louanges.

\par 4 Et Satan fit cela, afin que, lorsqu'Adam vit la lumière, il pense en lui-même que c'était une lumière céleste, et que les armées de Satan étaient des anges ; et que Dieu les avait envoyés pour surveiller la grotte et lui donner de la lumière dans les ténèbres.

\par 5 Ainsi, quand Adam sortirait de la grotte et les verrait, et qu'Adam et Ève se prosterneraient devant Satan, alors il vaincra Adam ainsi et l'humilierait une seconde fois devant Dieu.

\par 6 Quand donc Adam et Ève virent la lumière, pensant qu'elle était réelle, ils fortifièrent leur cœur ; pourtant, comme ils tremblaient, Adam dit à Ève :

\par 7 « Regardez cette grande lumière, et ces nombreux chants de louange, et cette armée qui se tient dehors et qui n'entre pas chez nous, ne nous dites pas ce qu'ils disent, ni d'où ils viennent, ni quelle en est la signification. de cette lumière; quels sont ces éloges ; pourquoi ils ont été envoyés ici, et pourquoi ils n’entrent pas.

\par 8 « S'ils venaient de Dieu, ils viendraient vers nous dans la grotte et nous raconteraient leur mission. »

\par 9 Alors Adam se leva et pria Dieu d'un cœur fervent, et dit : -

\par 10 « O Seigneur, y a-t-il dans le monde un autre dieu que Toi, qui a créé les anges et les a remplis de lumière, et les a envoyés pour nous garder, qui viendrait avec eux ?

\par 11 « Mais voici, nous voyons ces armées qui se tiennent à l'entrée de la grotte ; ils sont sous un grand jour ; ils chantent des louanges bruyantes. S'ils appartiennent à un autre dieu que Toi, dis-le-moi ; et s'ils sont envoyés par toi, informe-moi de la raison pour laquelle tu les as envoyés.

\par 12 A peine Adam eut-il dit cela, qu'un ange de Dieu lui apparut dans la grotte, qui lui dit : « Ô Adam, ne crains pas. C'est Satan et ses armées ; il veut vous tromper comme il vous a trompé d'abord. Pour la première fois, il était caché dans le serpent ; mais cette fois, il est venu à vous sous la forme d'un ange de lumière ; afin que, lorsque vous l'adorez, il vous captive, en présence même de Dieu.

\par 13 Alors l'ange quitta Adam, et saisit Satan à l'ouverture de la grotte, et le dépouille de la feinte qu'il avait assumée, et l'amena sous sa propre forme hideuse à Adam et Ève ; qui avaient peur de lui en le voyant.

\par 14 Et l'ange dit à Adam : « Cette forme hideuse lui appartient depuis que Dieu l'a fait tomber du ciel. Il n'aurait pas pu s'approcher de vous ; c’est pourquoi il s’est transformé en ange de lumière.

\par 15 Alors l'ange chassa Satan et ses armées d'Adam et Ève, et leur dit : « N'ayez pas peur ; Dieu qui t’a créé te fortifiera.

\par 16 Et l'ange les quitta.

\par 17 Mais Adam et Ève restèrent debout dans la grotte ; aucune consolation ne leur est venue ; ils étaient divisés dans leurs pensées.

\par 18 Et quand fut le matin, ils prièrent ; puis il sortit à la recherche du jardin. Car leur cœur était tourné vers elle, et ils ne pouvaient trouver aucune consolation de l’avoir quitté.

\chapitre{28}

\par \textit{Le Diable fait semblant de conduire Adam et Ève jusqu'à l'eau pour se baigner.}

\par 1 MAIS quand le rusé Satan les vit qu'ils allaient au jardin, il rassembla son armée et apparut sur une nuée, avec l'intention de les tromper.

\par 2 Mais quand Adam et Ève le virent ainsi dans une vision, ils pensèrent qu'ils étaient des anges de Dieu venus les consoler de leur sortie du jardin, ou les y ramener.

\par 3 Et Adam étendit les mains vers Dieu, le suppliant de lui faire comprendre ce qu'elles étaient.

\par 4 Alors Satan, le ennemi de tout bien, dit à Adam : « Ô Adam, je suis un ange du grand Dieu ; et voici les armées qui m’entourent.

\par 5 « Dieu m'a envoyé, moi et eux, pour te prendre et t'amener à la limite du jardin vers le nord ; au bord de la mer limpide, et baigne-toi toi et Ève dedans, et ramène-toi à ton ancienne joie, afin que tu retournes au jardin.

\par 6 Ces paroles pénétrèrent dans le cœur d'Adam et d'Ève.

\par 7 Pourtant Dieu a caché sa Parole à Adam, et ne lui a pas fait comprendre tout de suite, mais a attendu de voir sa force ; s'il serait vaincu comme Eve l'était lorsqu'elle était dans le jardin, ou s'il l'emporterait.

\par 8 Alors Satan appela Adam et Ève et dit : « Voici, nous allons à la mer d'eau », et ils commencèrent à y aller.

\par 9 Et Adam et Eve les suivirent à une petite distance.

\par 10 Mais lorsqu'ils arrivèrent à la montagne au nord du jardin, une montagne très haute, sans aucune marche jusqu'au sommet, le Diable s'approcha d'Adam et Ève, et les fit monter au sommet en réalité. , et non dans une vision ; souhaitant, comme il l'a fait, les renverser et les tuer, et effacer leur nom de la terre ; afin que cette terre reste à lui seul et à ses hôtes.



\chapitre{29}

\par \textit{Dieu révèle à Adam le dessein du Diable. (v. 4).}

\par 1 MAIS lorsque le Dieu miséricordieux vit que Satan voulait tuer Adam avec ses multiples artifices, et vit qu'Adam était doux et sans fraude, Dieu parla à Satan d'une voix forte et le maudit.

\par 2 Alors lui et ses armées s'enfuirent, et Adam et Ève restèrent debout au sommet de la montagne, d'où ils virent au-dessous d'eux le vaste monde au-dessus duquel ils étaient. Mais ils ne virent aucun membre de l'armée qui se trouvait immédiatement à côté d'eux.

\par 3 Ils pleurèrent, Adam et Ève, devant Dieu, et Lui demandèrent pardon.

\par 4 Alors la Parole de Dieu fut adressée à Adam et lui dit : « Connais et comprends ce Satan, qu'il cherche à te tromper, toi et ta postérité après toi. »

\par 5 Et Adam pleura devant le Seigneur Dieu, et le supplia et le supplia de lui donner quelque chose du jardin, en signe pour lui, de quoi être consolé.

\par 6 Et Dieu regarda la pensée d'Adam, et envoya l'ange Michel jusqu'à la mer qui atteint l'Inde, pour en prendre des verges d'or et les apporter à Adam.

\par 7 C'est ce que Dieu a fait dans sa sagesse, afin que ces verges d'or, étant avec Adam dans la grotte, brillent de lumière dans la nuit autour de lui, et mettent fin à sa peur des ténèbres.

\par 8 Alors l'ange Michel descendit par ordre de Dieu, prit des verges d'or, comme Dieu le lui avait ordonné, et les apporta à Dieu.

\chapitre{30}

\par \textit{Adam reçoit les premiers biens du monde.}

\par 1 APRÈS ces choses, Dieu ordonna à l'ange Gabriel de descendre au jardin, et de dire au chérubin qui le gardait : « Voici, Dieu m'a ordonné d'entrer dans le jardin et d'en prendre de l'encens odorant, et donne-le à Adam.

\par 2 Alors l'ange Gabriel descendit par l'ordre de Dieu au jardin et rapporta au chérubin ce que Dieu lui avait ordonné.

\par 3 Le chérubin dit alors : « Eh bien. » Et Gabriel entra et prit l'encens.

\par 4 Alors Dieu ordonna à son ange Raphaël de descendre au jardin et de parler au chérubin au sujet de la myrrhe pour la donner à Adam.

\par 5 Et l'ange Raphaël descendit et dit au chérubin ce que Dieu lui avait ordonné, et le chérubin dit : « Bien. » Alors Raphaël entra et prit la myrrhe.

\par 6 Les verges d'or provenaient de la mer des Indes, où se trouvent des pierres précieuses. L'encens provenait de la bordure orientale du jardin ; et la myrrhe de la frontière occidentale, d'où l'amertume est venue sur Adam.

\par 7 Et les anges apportèrent ces trois choses à Dieu, près de l'Arbre de Vie, dans le jardin.

\par 8 Alors Dieu dit aux anges : Trempez-les dans la source d'eau ; puis prends-les et asperge de leur eau Adam et Ève, afin qu'ils soient un peu consolés dans leur chagrin, et donne-les à Adam et Ève.

\par 9 Et les anges firent ce que Dieu leur avait ordonné, et ils donnèrent toutes ces choses à Adam et Ève au sommet de la montagne sur laquelle Satan les avait placés, quand il cherchait à en finir avec eux.

\par 10 Et quand Adam vit les verges d'or, l'encens et la myrrhe, il se réjouit et pleura parce qu'il pensait que l'or était un signe du royaume d'où il était venu, que l'encens était un signe de la lumière brillante qui lui avait été enlevé, et que la myrrhe était un signe du chagrin dans lequel il se trouvait.

\chapitre{31}

\par \textit{Ils s'installent plus confortablement dans la Grotte aux Trésors le troisième jour.}

\par 1 APRÈS ces choses, Dieu dit à Adam : « Tu m'as demandé quelque chose du jardin, pour en être consolé, et je t'ai donné ces trois signes pour te consoler ; que tu aies confiance en moi et en mon alliance avec toi.

\par 2 « Car je viendrai et te sauverai ; et les rois m'apporteront, quand je suis en chair, de l'or, de l'encens et de la myrrhe ; l'or comme signe de Mon royaume ; l'encens comme signe de Ma divinité ; et la myrrhe comme signe de ma souffrance et de ma mort.

\par 3 « Mais, ô Adam, mets-les près de toi dans la grotte ; l'or pour qu'il t'éclaire la nuit ; l'encens, pour que tu sentes sa douce saveur ; et la myrrhe, pour te consoler dans ton chagrin.

\par 4 Quand Adam entendit ces paroles de Dieu, il adora devant Lui. Lui et Ève l'adorèrent et lui rendirent grâces parce qu'il les avait traités avec miséricorde.

\par 5 Alors Dieu ordonna aux trois anges, Michel, Gabriel et Raphaël, chacun d'apporter ce qu'il avait apporté et de le donner à Adam. Et ils l’ont fait, un par un.

\par 6 Et Dieu ordonna à Suriyel et Salathiel de porter Adam et Eve, de les faire descendre du sommet de la haute montagne et de les emmener à la Grotte des Trésors.

\par 7 Là, ils déposèrent l'or du côté sud de la grotte, l'encens du côté oriental et la myrrhe du côté occidental. Car l’entrée de la grotte était du côté nord.

\par 8 Les anges réconfortèrent alors Adam et Ève, et s'en allèrent.

\par 9 L'or était de soixante-dix bâtons ; l'encens, douze livres ; et la myrrhe, trois livres.

\par 10 Ceux-ci sont restés par Adam dans la Maison des Trésors ; c'est pourquoi on l'appelait « de dissimulation ». Mais d’autres interprètes disent qu’on l’appelait la « Grotte des trésors », en raison des corps d’hommes justes qui s’y trouvaient.

\par 11 Ces trois choses, Dieu les donna à Adam, le troisième jour après sa sortie du jardin, en signe des trois jours où le Seigneur resterait au cœur de la terre.

\par 12 Et ces trois choses, pendant qu'ils restaient avec Adam dans la grotte, lui donnèrent de la lumière la nuit ; et le jour, ils lui apportaient un peu de soulagement dans son chagrin.

\chapitre{32}

\par \textit{Adam et Ève vont dans l'eau pour prier.}

\par 1 ET Adam et Ève restèrent dans la Caverne des Trésors jusqu'au septième jour ; ils ne mangeaient ni des fruits de la terre, ni ne buvaient d'eau.

\par 2 Et à l'aube du huitième jour, Adam dit à Ève : « Ô Ève, nous avons prié Dieu de nous donner quelque chose du jardin, et il a envoyé ses anges qui nous ont apporté ce que nous avions désiré. »

\par 3 « Mais maintenant, levons-nous, allons à la mer d'eau que nous avons vue d'abord, et restons-y debout, priant pour que Dieu nous soit à nouveau favorable et nous ramène au jardin ; ou donnez-nous quelque chose; ou qu'Il nous donnera du réconfort dans un autre pays que celui dans lequel nous sommes.

\par 4 Alors Adam et Ève sortirent de la grotte, allèrent se tenir au bord de la mer dans laquelle ils s'étaient jetés auparavant, et Adam dit à Ève : —

\par 5 « Viens, descends dans ce lieu, et n'en sors pas avant la fin de trente jours, quand je reviendrai vers toi. Et priez Dieu d’un cœur fervent et d’une voix douce de nous pardonner.

\par 6 «Et j'irai dans un autre endroit, j'y descendrai et je ferai comme toi.»

\par 7 Alors Eve descendit dans l'eau, comme Adam le lui avait ordonné. Adam descendit aussi dans l'eau ; et ils priaient; et supplia le Seigneur de leur pardonner leur offense et de les restaurer dans leur ancien état.

\par 8 Et ils restèrent ainsi en prière, jusqu'à la fin des trente-cinq jours.

\chapitre{33}

\par \textit{Satan promet faussement la « lumière brillante ! »}

\par 1 MAIS Satan, le ennemi de tout bien, les chercha dans la grotte, mais ne les trouva pas, bien qu'il les cherchât diligemment.

\par 2 Mais il les trouva debout dans l'eau en train de prier et pensa en lui-même : « Adam et Ève se tiennent ainsi dans cette eau, implorant Dieu de leur pardonner leur transgression, de les restaurer dans leur ancien état et de les retirer de dessous. ma main.»

\par 3 «Mais je les tromperai afin qu'ils sortent de l'eau et n'accomplissent pas leur vœu.»

\par 4 Alors celui qui haïssait tout bien n'alla pas vers Adam, mais alla vers Ève, et prit la forme d'un ange de Dieu, louant et se réjouissant, et lui dit :

\par 5 « La paix soit avec toi ! Soyez heureux et réjouissez-vous ! Dieu vous est favorable et il m'a envoyé vers Adam. Je lui ai apporté la bonne nouvelle du salut et du fait qu’il est rempli d’une lumière vive comme il l’était au début.

\par 6 «Et Adam, dans sa joie de sa restauration, m'a envoyé vers toi, pour que tu viennes à moi, afin que je te couronne de lumière comme lui.»

\par 7 « Et il me dit : Parle à Ève ; si elle ne vient pas avec toi, parle-lui du signe lorsque nous étions au sommet de la montagne ; comment Dieu a envoyé ses anges qui nous ont pris et nous ont amenés à la Grotte des Trésors ; et je déposai l'or du côté sud ; l'encens, du côté oriental ; et la myrrhe du côté ouest. Maintenant, viens vers lui.

\par 8 Quand Ève entendit ces paroles de sa part, elle se réjouit grandement. Et pensant que l'apparition de Satan était réelle, elle sortit de la mer.

\par 9 Il marchait devant, et elle le suivit jusqu'à ce qu'ils arrivèrent à Adam. Alors Satan se cacha loin d'elle, et elle ne le vit plus.

\par 10 Elle vint alors et se tint devant Adam, qui se tenait au bord de l'eau et se réjouissait du pardon de Dieu.

\par 11 Et comme elle l'appelait, il se retourna, la trouva là et pleura en la voyant, et se frappa la poitrine ; et sous l'amertume de sa douleur, il tomba dans l'eau.

\par 12 Mais Dieu regarda lui et sa misère, et le fait qu'il était sur le point de rendre son dernier soupir. Et la Parole de Dieu vint du ciel, le releva de l'eau et lui dit : « Monte sur la haute berge jusqu'à Ève. » Et lorsqu'il s'approcha d'Ève, il lui dit : « Qui t'a dit «viens ici» ?

\par 13 Alors elle lui raconta le discours de l'ange qui lui était apparu et lui avait donné un signe.

\par 14 Mais Adam fut affligé et lui fit savoir que c'était Satan. Il l'a ensuite emmenée et ils sont tous deux retournés à la grotte.

\par 15 Ces choses leur sont arrivées la deuxième fois qu'ils descendirent à l'eau, sept jours après leur sortie du jardin.

\par 16 Ils jeûnèrent dans l'eau trente-cinq jours ; Cela faisait en tout quarante-deux jours qu'ils avaient quitté le jardin.

\chapitre{34}

\par \textit{Adam rappelle la création d'Ève. Il demande avec éloquence à manger et à boire.}

\par 1 ET le matin du quarante-troisième jour, ils sortirent de la grotte, tristes et en pleurs. Leurs corps étaient maigres et ils étaient desséchés par la faim et la soif, par le jeûne et la prière, et par leur grande tristesse à cause de leur transgression.

\par 2 Et lorsqu'ils furent sortis de la grotte, ils gravirent la montagne à l'ouest du jardin.

\par 3 Là, ils se tenaient debout et priaient et suppliaient Dieu de leur accorder le pardon de leurs péchés.

\par 4 Et après leurs prières, Adam commença à implorer Dieu, en disant : « Ô mon Seigneur, mon Dieu et mon Créateur, tu as ordonné aux quatre éléments d'être rassemblés, et ils ont été rassemblés selon ton ordre. »

\par 5 « Alors tu étendis ta main et tu m'as créé d'un seul élément, celui de la poussière de la terre ; et tu m'as amené dans le jardin à la troisième heure, un vendredi, et tu m'en as informé dans la grotte.

\par 6 « Alors, au début, je ne connaissais ni la nuit ni le jour, car j'avais une nature lumineuse ; et la lumière dans laquelle je vivais ne m’a jamais permis de connaître la nuit et le jour.

\par 7 « Puis encore, ô Seigneur, à la troisième heure où tu m'as créé, tu m'as amené toutes les bêtes, et les lions, et les autruches, et les oiseaux du ciel, et tout ce qui se meut sur la terre, que tu avais créé à la première heure avant moi du vendredi.

\par 8 « Et ta volonté était que je les nomme tous, un par un, avec un nom approprié. Mais tu m'as donné de ta part de l'intelligence et de la connaissance, ainsi qu'un cœur pur et un esprit droit, afin que je puisse leur donner le nom de ta propre pensée concernant leur nom.

\par 9 « Ô Dieu, tu les as rendus obéissants, et tu as ordonné qu'aucun d'eux ne se détache de mon empire, selon ton commandement et selon la domination que tu m'as donnée sur eux. Mais maintenant, ils sont tous séparés de moi.

\par 10 « Alors c'était à la troisième heure du vendredi, à laquelle tu m'as créé, et tu m'as commandé concernant l'arbre, dont je ne devais ni m'approcher ni en manger ; car tu m'as dit dans le jardin : Quand tu en mangeras, tu mourras de mort.

\par 11 «Et si tu m'avais puni comme tu l'as dit, de la mort, je serais mort à l'instant même.»

\par 12 « De plus, quand tu m'as ordonné concernant l'arbre, je ne devais ni m'en approcher ni m'en approcher, Eve n'était pas avec moi ; Tu ne l'avais pas encore créée, tu ne l'avais pas encore retirée de mon côté ; et elle n’avait pas encore entendu cet ordre de ta part.

\par 13 « Puis, à la fin de la troisième heure de ce vendredi, ô Seigneur, tu as provoqué sur moi un sommeil et un sommeil, et je me suis endormi, et j'ai été accablé par le sommeil. »

\par 14 « Alors tu as arraché une côte de mon côté, et tu l'as créée à ma ressemblance et à mon image. Puis je me suis réveillé; et quand je l'ai vue et que j'ai su qui elle était, j'ai dit : « Ceci est l'os de mes os et la chair de ma chair ; désormais, elle sera appelée femme.

\par 15 « C'est par ta bonne volonté, ô Dieu, que tu as apporté sur moi le sommeil et le sommeil, et que tu as immédiatement fait sortir Eve de mon côté, jusqu'à ce qu'elle soit sortie, de sorte que je ne vois pas comment elle a été fait; et je ne pourrais pas non plus témoigner, ô mon Seigneur, à quel point ta bonté et ta gloire sont terribles et grandes.

\par 16 « Et par ta bonne volonté, ô Seigneur, tu nous as créés tous deux avec des corps d'une nature lumineuse, et tu nous as faits deux, un ; et tu nous as donné ta grâce, et tu nous as remplis de louanges du Saint-Esprit ; que nous n'ayons ni faim ni soif, que nous ne sachions pas ce qu'est le chagrin, ni même la faiblesse du cœur ; ni souffrance, ni jeûne, ni lassitude.

\par 17 «Mais maintenant, ô Dieu, depuis que nous avons transgressé ton commandement et violé ta loi, tu nous as fait sortir dans un pays étranger et tu as fait venir sur nous la souffrance, la faiblesse, la faim et la soif.»

\par 18 « Maintenant donc, ô Dieu, nous te prions, donne-nous à manger du jardin, pour apaiser notre faim ; et quelque chose pour étancher notre soif.

\par 19 «Car voici, pendant plusieurs jours, ô Dieu, nous n'avons rien goûté ni bu, et notre chair est desséchée, et nos forces sont gaspillées, et le sommeil a disparu de nos yeux à cause de la faiblesse et des pleurs.»

\par 20 « Alors, ô Dieu, nous n'osons rien cueillir des fruits des arbres, par crainte de toi. Car lorsque nous avons transgressé au début, tu nous as épargnés et tu ne nous as pas fait mourir.

\par 21 «Mais maintenant, pensions-nous dans nos cœurs, si nous mangeons des fruits des arbres, sans l'ordre de Dieu, il nous détruira cette fois et nous effacera de la surface de la terre.»

\par 22 «Et si nous buvons de cette eau, sans l'ordre de Dieu, Il nous anéantira et nous déracinera immédiatement.»

\par 23 «Maintenant donc, ô Dieu, que je suis venu en ce lieu avec Ève, nous te supplions de nous donner des fruits du jardin, afin que nous puissions en être rassasiés.»

\par 24 « Car nous désirons le fruit qui est sur la terre, et tout ce qui nous manque. »

\chapitre{35}

\par \textit{Réponse de Dieu.}

\par 1 ALORS Dieu regarda de nouveau Adam et ses pleurs et gémissements, et la Parole de Dieu vint à lui et lui dit : —

\par 2 « Ô Adam, quand tu étais dans Mon jardin, tu ne savais ni manger ni boire ; ni malaise ni souffrance ; ni maigreur de chair, ni changement ; et le sommeil n'a pas quitté tes yeux. Mais depuis que tu as transgressé et que tu es entré dans ce pays étranger, toutes ces épreuves t’arrivent.

\chapitre{36}

\par \textit{Fig.}

\par 1 ALORS Dieu ordonna au chérubin, qui gardait la porte du jardin, une épée de feu à la main, de prendre du fruit du figuier et de le donner à Adam.

\par 2 Le chérubin obéit à l'ordre du Seigneur Dieu, et entra dans le jardin et apporta deux figues sur deux brindilles, chaque figue pendant à sa feuille ; ils provenaient de deux des arbres parmi lesquels Adam et Ève se cachaient lorsque Dieu allait se promener dans le jardin, et la Parole de Dieu vint à Adam et Ève et leur dit : « Adam, Adam, où es-tu ?

\par 3 Et Adam répondit : « Ô Dieu, me voici. Quand j'ai entendu ton son et ta voix, je me suis caché, parce que je suis nu.

\par 4 Alors le chérubin prit deux figues et les apporta à Adam et Ève. Mais il les leur lança de loin ; car ils ne pouvaient pas s'approcher du chérubin, à cause de leur chair, qui ne pouvait pas s'approcher du feu.

\par 5 Au début, les anges tremblaient en présence d'Adam et avaient peur de lui. Mais maintenant, Adam tremblait devant les anges et avait peur d'eux.

\par 6 Alors Adam s'approcha et prit une figue, et Eve aussi vint à son tour et prit l'autre.

\par 7 Et tandis qu'ils les prenaient dans leurs mains, ils les regardèrent, et savaient qu'ils venaient des arbres parmi lesquels ils s'étaient cachés, les elfes.

\chapitre{37}

\par \textit{Quarante-trois jours de pénitence ne rachètent pas une heure de péché (v. 6).}

\par 1 ALORS Adam dit à Ève : « Ne vois-tu pas ces figues et leurs feuilles, dont nous nous couvrions lorsque nous étions dépouillés de notre nature lumineuse ? Mais maintenant, nous ne savons pas quelle misère et quelle souffrance nous pouvons subir en les mangeant.

\par 2 « Maintenant donc, ô Ève, retenons-nous et n'en mangeons pas, toi et moi ; et demandons à Dieu de nous donner du fruit de l’Arbre de Vie.

\par 3 Ainsi Adam et Ève se retinrent et ne mangèrent pas de ces figues.

\par 4 Mais Adam commença à prier Dieu et à le supplier de lui donner du fruit de l'Arbre de Vie, disant ainsi : « Ô Dieu, lorsque nous avons transgressé Ton commandement à la sixième heure du vendredi, nous avons été dépouillés du Nous avions une nature lumineuse et nous ne sommes pas restés dans le jardin après notre transgression, plus de trois heures.

\par 5 « Mais le soir tu nous en as fait sortir. Ô Dieu, nous avons transgressé contre Toi une heure, et toutes ces épreuves et ces chagrins nous sont arrivés jusqu'à ce jour.

\par 6 « Et ces jours-là, ainsi que celui-ci, le quarante-troisième jour, ne rachètez pas cette heure pendant laquelle nous avons transgressé ! »

\par 7 « Ô Dieu, regarde-nous avec un œil de pitié, et ne nous rends pas selon notre transgression de ton commandement, en présence de toi. »

\par 8 « Ô Dieu, donne-nous du fruit de l'Arbre de Vie, afin que nous puissions en manger, vivre et ne plus voir les souffrances et autres troubles sur cette terre ; car tu es Dieu.

\par 9 « Quand nous avons transgressé ton commandement, tu nous as fait sortir du jardin, et tu as envoyé un chérubin pour garder l'arbre de vie, de peur que nous n'en mangions et ne vivions ; et nous ne savons rien de l’évanouissement après avoir transgressé.

\par 10 « Mais maintenant, Seigneur, voici, nous avons enduré tous ces jours et avons supporté des souffrances. Faites de ces quarante-trois jours l’équivalent de l’heure pendant laquelle nous avons transgressé.

\chapitre{38}

\par \textit{«Quand 5500 ans seront accomplis. . . . »}

\par 1 APRÈS ces choses, la Parole de Dieu vint à Adam et lui dit : —

\par 2 « Ô Adam, quant au fruit de l'Arbre de Vie, que tu demandes, je ne te le donnerai pas maintenant, mais quand les 5 500 ans seront accomplis. Alors je te donnerai du fruit de l'Arbre de Vie, et tu en mangeras et tu vivras éternellement, toi, Ève et ta juste postérité.

\par 3 «Mais ces quarante-trois jours ne peuvent réparer l'heure à laquelle tu as transgressé mon commandement.»

\par 4 « Ô Adam, je t'ai donné à manger du figuier dans lequel tu t'es caché. Allez en manger, toi et Ève.

\par 5 « Je ne refuserai pas ta demande, et je ne décevrai pas non plus ton espérance ; c’est pourquoi, continue à accomplir l’alliance que j’ai conclue avec toi.

\par 6 Et Dieu retira Sa Parole d'Adam.

\chapitre{39}

\par \textit{Adam est prudent, mais trop tard.}

\par 1 ALORS Adam revint vers Ève et lui dit : « Lève-toi, et prends une figue pour toi, et j'en prendrai une autre ; et allons à notre grotte.

\par 2 Alors Adam et Ève prirent chacun une figue et se dirigèrent vers la grotte ; il s’agissait du coucher du soleil ; et leurs pensées leur donnaient envie de manger du fruit.

\par 3 Mais Adam dit à Ève : « J'ai peur de manger de cette figue. Je ne sais pas ce qui peut m’arriver.

\par 4 Alors Adam pleura et se tint en prière devant Dieu, disant : « Satisfaits ma faim, sans que j'aie à manger de cette figue ; car après l'avoir mangé, à quoi cela me servira-t-il ? Et que te désirerai-je et que te demanderai-je, ô Dieu, quand il sera parti ?

\par 5 Et il dit encore : « J'ai peur d'en manger ; car je ne sais pas ce qui m’arrivera à cause de cela.

\chapitre{40}

\par \textit{La première faim humaine.}

\par 1 ALORS la Parole de Dieu vint à Adam et lui dit : « Ô Adam, pourquoi n'avais-tu pas cette crainte, ni ce jeûne, ni ce souci avant cela ? Et pourquoi n’avais-tu pas cette crainte avant de transgresser ?

\par 2 «Mais quand tu es venu habiter dans ce pays étranger, ton corps animal ne pouvait pas être sur terre sans nourriture terrestre, pour le fortifier et restaurer ses pouvoirs.»

\par 3 Et Dieu retira Sa Parole d'Adam.

\chapitre{41}

\par \textit{La première soif humaine.}

\par 1 ALORS Adam prit la figue et la posa sur les verges d'or. Ève prit aussi sa figue et la mit sur le parfum.

\par 2 Et le poids de chaque figue était celui d'une pastèque ; car le fruit du jardin était bien plus gros que le fruit de cette terre.

\par 3 Mais Adam et Ève restèrent debout et jeûnèrent toute la nuit, jusqu'à l'aube du matin.

\par 4 Quand le soleil se leva, ils étaient en prière, et Adam dit à Ève, après qu'ils eurent fini de prier : —

\par 5 « Ô Ève, viens, allons à la lisière du jardin, face au midi ; jusqu'à l'endroit d'où coule la rivière, et elle est divisée en quatre têtes. Là, nous prierons Dieu et lui demanderons de nous donner à boire de l'eau de la vie.

\par 6 « Car Dieu ne nous a pas nourris de l'Arbre de Vie, afin que nous ne vivions pas. Nous lui demanderons donc de nous donner de l’Eau de Vie et d’étancher notre soif avec elle, plutôt qu’avec une boisson à l’eau de ce pays.

\par 7 Quand Ève entendit ces paroles d'Adam, elle acquiesça ; Et ils se levèrent tous deux et arrivèrent à la limite sud du jardin, au bord de la rivière d'eau, à quelque peu de distance du jardin.

\par 8 Et ils se tinrent debout et prièrent devant le Seigneur, et lui demandèrent de les regarder cette fois, de leur pardonner et de leur accorder leur demande.

\par 9 Après cette prière de tous deux, Adam se mit à prier de sa voix devant Dieu et dit :

\par 10 « O Seigneur, quand j'étais dans le jardin et que j'ai vu l'eau qui coulait sous l'Arbre de Vie, mon cœur n'a pas désiré, et mon corps n'a pas non plus eu besoin d'en boire ; je n’avais pas non plus soif, car j’étais vivant ; et au-dessus de ce que je suis maintenant.

\par 11 «De sorte que pour vivre, je n'ai eu besoin d'aucune nourriture de vie, ni de l'eau de vie.»

\par 12 « Mais maintenant, ô Dieu, je suis mort ; ma chair est desséchée par la soif. Donne-moi de l'Eau de Vie afin que je puisse en boire et vivre.

\par 13 «Par ta miséricorde, ô Dieu, sauve-moi de ces fléaux et de ces épreuves, et amène-moi dans un autre pays différent de celui-ci, si tu ne me laisses pas habiter dans ton jardin.»

\chapitre{42}

\par \textit{Une promesse de l'Eau de Vie. La troisième prophétie de la venue du Christ.}

\par 1 ALORS la Parole de Dieu vint à Adam et lui dit : —

\par 2 «Ô Adam, quant à ce que tu dis : 'Amène-moi dans un pays où il y a du repos', ce n'est pas un autre pays que celui-ci, mais c'est le royaume des cieux où seul il y a du repos.»

\par 3 « Mais tu ne peux pas y entrer pour le moment ; mais seulement après que ton jugement soit passé et accompli.

\par 4 « Alors je te ferai monter au royaume des cieux, toi et ta juste postérité ; et je te donnerai, à toi et à eux, le reste que tu demandes à présent.

\par 5 « Et si tu dis : « Donne-moi de l'eau de vie afin que je puisse boire et vivre », ce ne peut pas être aujourd'hui, mais le jour où je descendrai aux enfers, et briserai les portes d'airain, et briser en morceaux les royaumes de fer.

\par 6 « Alors, par miséricorde, je sauverai ton âme et celle des justes, pour leur donner du repos dans mon jardin. Et ce sera quand la fin du monde viendra.

\par 7 « Et encore, quant à l'Eau de Vie que tu cherches, elle ne te sera pas accordée aujourd'hui ; mais le jour où je verserai mon sang sur ta tête au pays du Golgotha.

\par 8 « Car mon sang sera pour toi, en ce temps-là, l'eau de la vie, et non pas pour toi seul, mais pour tous ceux de ta postérité qui croiront en moi ; que ce soit à eux le repos pour toujours.

\par 9 Le Seigneur dit encore à Adam : « Ô Adam, quand tu étais dans le jardin, ces épreuves ne t'arrivèrent pas »

\par 10 «Mais depuis que tu as transgressé mon commandement, toutes ces souffrances t'arrivent.»

\par 11. « Maintenant aussi, ta chair a besoin de nourriture et de boisson ; bois donc de cette eau qui coule près de toi sur la surface de la terre.

\par 12 Alors Dieu retira Sa Parole d'Adam.

\par 13 Et Adam et Eve adorèrent le Seigneur et revinrent du fleuve d'eau à la grotte. Il était midi ; et lorsqu'ils approchèrent de la grotte, ils virent un grand feu près d'elle.

\chapitre{43}

\par \textit{Le Diable tente un incendie criminel.}

\par 1 ALORS Adam et Ève eurent peur et restèrent immobiles. Et Adam dit à Ève : « Quel est ce feu près de notre grotte ? Nous ne faisons rien pour provoquer cet incendie.

\par 2 « Nous n'avons ni pain à y cuire, ni bouillon à y cuire. Quant à cet incendie, nous n’en savons rien de semblable, et nous ne savons pas non plus comment l’appeler.

\par 3 «Mais depuis que Dieu a envoyé le chérubin avec une épée de feu qui brillait et s'éclairait dans sa main, de peur de laquelle nous sommes tombés et étions comme des cadavres, n'avons-nous pas vu de pareilles.»

\par 4 «Mais maintenant, ô Ève, voici, c'est le même feu qui était dans la main du chérubin, que Dieu a envoyé pour garder la grotte dans laquelle nous habitons.»

\par 5 « Ô Ève, c'est parce que Dieu est en colère contre nous et qu'il nous en chassera. »

\par 6 «Ô Ève, nous avons encore transgressé Son commandement dans cette grotte, de sorte qu'Il a envoyé ce feu pour brûler autour d'elle et pour nous empêcher d'y entrer.»

\par 7 « S'il en est vraiment ainsi, ô Ève, où habiterons-nous ? Et où fuirons-nous devant la face du Seigneur ? Puisque, en ce qui concerne le jardin, Il ne nous laisse pas y demeurer, et Il nous a privé de ses biens ; mais Il nous a placés dans cette grotte, dans laquelle nous avons supporté les ténèbres, les épreuves et les difficultés, jusqu'à ce qu'enfin nous y trouvions du réconfort.

\par 8 « Mais maintenant qu'Il nous a fait sortir dans un autre pays, qui sait ce qui peut y arriver ? Et qui sait si les ténèbres de ce pays ne sont peut-être pas bien plus grandes que les ténèbres de ce pays ?

\par 9 « Qui sait ce qui peut arriver dans ce pays de jour ou de nuit ? Et qui sait si ce sera loin ou proche, ô Ève ? Là où il plaira à Dieu de nous mettre, peut-être loin du jardin, ô Ève ! ou où Dieu nous empêchera-t-il de le voir, parce que nous avons transgressé son commandement et parce que nous lui avons fait des demandes à tout moment ?

\par 10 « Ô Ève, si Dieu nous amène dans un pays étranger autre que celui-ci, dans lequel nous trouvons de la consolation, ce sera pour mettre nos âmes à mort et effacer notre nom de la surface de la terre. »

\par 11 « Ô Ève, si nous sommes plus éloignés du jardin et de Dieu, où le retrouverons-nous et lui demanderons-nous de nous donner de l'or, de l'encens, de la myrrhe et des fruits du figuier ?

\par 12 « Où le trouverons-nous, pour nous consoler une seconde fois ? Où le trouverons-nous, afin qu'il pense à nous, en ce qui concerne l'alliance qu'il a conclue en notre faveur ?

\par 13 Alors Adam ne dit plus rien. Et ils regardaient, lui et Ève, vers la grotte et vers le feu qui brûlait autour d'elle.

\par 14 Mais ce feu venait de Satan. Car il avait cueilli des arbres et des herbes sèches, il les avait transportés et amenés à la grotte, et il y avait mis le feu, afin de consumer la grotte et tout ce qu'elle contenait.

\par 15 Afin qu'Adam et Ève soient laissés dans le chagrin, et qu'il coupe leur confiance en Dieu et les oblige à le renier.

\par 16 Mais par la miséricorde de Dieu, il ne pouvait pas brûler la grotte, car Dieu envoya son ange autour de la grotte pour la garder d'un tel feu, jusqu'à ce qu'elle s'éteigne.

\par 17 Et ce feu dura depuis midi jusqu'au point du jour. C'était le quarante-cinquième jour.

\chapitre{44}

\par \textit{Le pouvoir du feu sur l'homme.}

\par 1 POURTANT Adam et Ève étaient debout et regardaient le feu, et incapables de s'approcher de la grotte à cause de leur peur du feu.

\par 2 Et Satan continuait à amener des arbres et à les jeter dans le feu, jusqu'à ce que la flamme s'élève en haut et couvre toute la grotte, pensant, comme il le faisait dans son propre esprit, consumer la grotte avec beaucoup de feu. Mais l'ange du Seigneur le gardait.

\par 3 Et pourtant il ne pouvait pas maudire Satan, ni le blesser par des paroles, parce qu'il n'avait aucune autorité sur lui, et il ne se mettait pas non plus à le faire avec des paroles sortant de sa bouche.

\par 4 C'est pourquoi l'ange le supporta, sans dire un seul mauvais mot jusqu'à ce que vienne la Parole de Dieu qui dit à Satan : « Va d'ici ; Une fois auparavant tu as trompé mes serviteurs, et cette fois tu cherches à les détruire.

\par 5 « Sans ma miséricorde, je t'aurais détruit, toi et tes armées, du dessus de la terre. Mais j’ai eu patience envers toi jusqu’à la fin du monde.

\par 6 Alors Satan s'enfuit devant le Seigneur. Mais le feu continua à brûler autour de la grotte comme un feu de charbon toute la journée ; c'était le quarante-sixième jour qu'Adam et Ève passaient depuis qu'ils étaient sortis du jardin.

\par 7 Et quand Adam et Ève virent que la chaleur du feu s'était quelque peu refroidie, ils se mirent à marcher vers la grotte pour y entrer comme ils avaient l'habitude ; mais ils ne le purent pas, à cause de la chaleur du feu.

\par 8 Alors ils se mirent tous deux à pleurer à cause du feu qui les séparait de la grotte, et qui s'approchait d'eux en brûlant. Et ils avaient peur.

\par 9 Alors Adam dit à Ève : « Vois ce feu dont nous avons une part en nous : qui autrefois nous cédait, mais ne le fait plus, maintenant que nous avons transgressé la limite de la création et changé notre condition, et notre nature est modifiée. Mais le feu n'a pas changé dans sa nature, ni altéré depuis sa création. C'est pourquoi il a maintenant pouvoir sur nous ; et quand nous nous en approchons, cela nous brûle la chair.

\chapitre{45}

\par \textit{Pourquoi Satan n'a pas tenu ses promesses.}

\par 1 ALORS Adam se leva et pria Dieu, disant : « Vois, ce feu a fait une séparation entre nous et la grotte dans laquelle tu nous as commandé d'habiter ; mais maintenant, voici, nous ne pouvons pas y entrer.

\par 2 Alors Dieu entendit Adam et lui envoya Sa Parole qui disait :

\par 3 « Ô Adam, vois ce feu ! comme la flamme et la chaleur de celui-ci sont différentes du jardin des délices et des bonnes choses qu'il contient !

\par 4 « Quand tu étais sous mon contrôle, toutes les créatures se sont soumises à toi ; mais après que tu as transgressé mon commandement, ils se lèvent tous sur toi.

\par 5 Dieu lui dit encore : « Vois, ô Adam, comme Satan t'a exalté ! Il t'a privé de la Divinité et d'un état exalté semblable à Moi, et n'a pas tenu sa parole envers toi ; mais, après tout, il est devenu ton ennemi. C’est lui qui a fait ce feu dans lequel il voulait te brûler, toi et Ève.

\par 6 « Pourquoi, ô Adam, n'a-t-il pas respecté son accord avec toi, pas même un jour ; mais t'a privé de la gloire qui était sur toi, quand tu t'es soumis à son commandement ?

\par 7 « Penses-tu, Adam, qu'il t'a aimé lorsqu'il a conclu cet accord avec toi ? Ou qu'il t'aimait et souhaitait t'élever très haut ?

\par 8 « Mais non, Adam, il n'a pas fait tout cela par amour pour toi ; mais il voulait te faire sortir de la lumière dans les ténèbres, et d'un état exalté vers la dégradation ; de la gloire à l'humiliation ; de la joie au chagrin; et du repos au jeûne et à l’évanouissement.

\par 9 Dieu dit aussi à Adam : « Vois ce feu allumé par Satan autour de ta caverne ; vois cette merveille qui t'entoure ; et sache que cela englobera à la fois toi et ta semence, lorsque tu écouteras son ordre ; qu'il vous frappera de feu ; et que vous descendrez aux enfers après votre mort.

\par 10 « Alors vous verrez brûler son feu, qui brûlera ainsi autour de vous et de votre semence. Il n'y aura pour vous aucune délivrance, sauf à Ma venue ; de la même manière, tu ne peux pas maintenant entrer dans ta grotte, à cause du grand feu qui l'entoure ; ce n’est que lorsque ma Parole viendra qui t’ouvrira un chemin le jour où mon alliance sera accomplie.

\par 11 « Il n'y a aucun moyen pour toi actuellement de venir d'ici pour te reposer, pas jusqu'à ce que vienne Ma Parole, qui est Ma Parole. Alors il te préparera un chemin et tu auras du repos. Alors Dieu a appelé avec Sa Parole ce feu qui brûlait autour de la grotte, pour qu'il se sépare jusqu'à ce qu'Adam l'ait traversé. Ensuite, le feu s'est séparé sur l'ordre de Dieu, et un chemin a été ouvert pour Adam.

\par 12 Et Dieu retira Sa Parole d'Adam.

\chapitre{46}

\par \textit{« Combien de fois je t'ai délivré de sa main. . . »}

\par 1 PUIS Adam et Eve recommencèrent à entrer dans la grotte. Et quand ils arrivèrent au chemin entre le feu, Satan souffla dans le feu comme un tourbillon, et fit sur Adam et Ève un feu de charbon ardent ; de sorte que leurs corps étaient roussis ; et le feu de charbon les brûla.

\par 2 Et du feu brûlant, Adam et Ève crièrent à haute voix et dirent : « Ô Seigneur, sauve-nous ! Ne nous laissons pas consumer et tourmenter par ce feu ardent ; et ne nous exige pas non plus d’avoir transgressé ton commandement.

\par 3 Alors Dieu regarda leurs corps, sur lesquels Satan avait fait brûler le feu, et Dieu envoya son ange qui arrêta le feu brûlant. Mais les blessures restaient sur leurs corps.

\par 4 Et Dieu dit à Adam : « Vois l'amour de Satan pour toi, qui prétendait te donner la Divinité et la grandeur ; et voici, il te brûle par le feu et cherche à te détruire de dessus la terre.

\par 5 « Alors regarde-moi, ô Adam ; Je t'ai créé, et combien de fois t'ai-je délivré de sa main ? Sinon, ne t’aurait-il pas détruit ?

\par 6 Dieu dit encore à Ève : « Qu'est-ce qu'il t'a promis dans le jardin, en disant : 'Au moment où vous mangerez de l'arbre, vos yeux s'ouvriront, et vous deviendrez comme des dieux, connaissant le bien et le mal. .' Mais voilà ! il a brûlé vos corps au feu, et vous a fait goûter le goût du feu, pour le goût du jardin ; et il vous a fait voir la combustion du feu, ses méfaits et son pouvoir sur vous.

\par 7 « Vos yeux ont vu le bien qu'il vous a pris, et en vérité il vous a ouvert les yeux ; et vous avez vu le jardin dans lequel vous étiez avec moi, et vous avez aussi vu le mal qui vous est arrivé de la part de Satan. Mais quant à la Divinité, il ne peut pas vous la donner, ni accomplir ce qu'il vous a dit. Bien plus, il était amer contre toi et contre ta postérité qui viendra après toi.

\par 8 Et Dieu leur retira Sa Parole.

\chapitre{47}

\par \textit{Les intrigues du Diable.}

\par 1 PUIS Adam et Ève entrèrent dans la grotte, tremblants encore du feu qui avait brûlé leurs corps. Alors Adam dit à Ève :

\par 2 « Voici, le feu a brûlé notre chair dans ce monde ; mais que se passera-t-il lorsque nous serons morts et que Satan punira nos âmes ? Notre délivrance n’est-elle pas longue et lointaine, à moins que Dieu ne vienne et n’accomplisse sa promesse par sa miséricorde envers nous ?

\par 3 Alors Adam et Ève entrèrent dans la grotte, se bénissant d'y être revenus une fois de plus. Car ils pensèrent qu'ils n'y entreraient jamais, lorsqu'ils virent le feu qui l'entourait.

\par 4 Mais alors que le soleil se couchait, le feu brûlait toujours et s'approchait d'Adam et Ève dans la grotte, de sorte qu'ils ne pouvaient pas y dormir. Après le coucher du soleil, ils en sortirent. C'était le quarante-septième jour après leur sortie du jardin.

\par 5 Adam et Ève vinrent alors dormir sous le sommet d'une colline près du jardin, comme ils avaient l'habitude de le faire.

\par 6 Et ils se levèrent et prièrent Dieu de leur pardonner leurs péchés, puis s'endormirent sous le sommet de la montagne.

\par 7 Mais Satan, ennemi de tout bien, pensait en lui-même : Alors que Dieu a promis le salut à Adam par alliance, et qu'il le délivrerait de toutes les épreuves qui lui sont arrivées, mais ne me l'a pas promis par alliance, et ne me délivrera pas de mes difficultés ; bien plus, puisqu'Il lui a promis de le faire habiter, lui et sa postérité, dans le royaume dans lequel j'étais autrefois, je tuerai Adam.

\par 8 La terre sera débarrassée de lui ; et sera laissé à moi seul ; afin qu'à sa mort, il ne lui reste plus aucune postérité pour hériter du royaume qui restera mon propre royaume ; Dieu aura alors besoin de moi, et il me rendra là avec mes hôtes.

\chapitre{48}

\par \textit{Cinquième apparition de Satan à Adam et Ève.}

\par 1 APRÈS cela, Satan appela ses armées, qui vinrent toutes à lui et lui dirent :

\par 2 « Ô, notre Seigneur, que feras-tu ?

\par 3 Il leur dit alors : « Vous savez que cet Adam, que Dieu a créé de la poussière, est celui qui a pris notre royaume. Venez, rassemblons-nous et tuons-le ; ou lancez une pierre sur lui et sur Ève, et écrasez-les dessous.

\par 4 Lorsque les armées de Satan entendirent ces paroles, elles arrivèrent à la partie de la montagne où Adam et Ève dormaient.

\par 5 Alors Satan et ses armées prirent un énorme rocher, large et plat, et sans défaut, pensant en lui-même : « S'il y avait un trou dans le rocher, lorsqu'il tombait sur eux, le trou dans le rocher pourrait tomber sur eux. eux, et ainsi ils s’échapperaient et ne mourraient pas.

\par 6 Il dit alors à ses hôtes : « Prenez cette pierre et jetez-la à plat sur eux, afin qu'elle ne roule pas d'eux vers un autre endroit. Et quand vous l’aurez lancé, fuyez et ne tardez pas.

\par 7 Et ils firent ce qu'il leur ordonnait. Mais lorsque le rocher tomba de la montagne sur Adam et Ève, Dieu ordonna qu'il devienne sur eux une sorte de hangar qui ne leur ferait aucun mal. Et il en était ainsi par ordre de Dieu.

\par 8 Mais quand le rocher tomba, toute la terre trembla avec lui, et. a été secoué par la taille du rocher.

\par 9 Et pendant qu'il tremblait et tremblait, Adam et Eve se réveillèrent du sommeil et se trouvèrent sous un rocher comme un hangar. Mais ils ne savaient pas comment c’était ; car lorsqu'ils s'endormirent, ils étaient sous le ciel et non sous un hangar ; et quand ils le virent, ils eurent peur.

\par 10 Alors Adam dit à Ève : « Pourquoi la montagne s'est-elle courbée, et la terre a-t-elle tremblé et secouée à cause de nous ? Et pourquoi ce rocher s’est-il étendu sur nous comme une tente ?

\par 11 « Dieu a-t-il l'intention de nous tourmenter et de nous enfermer dans cette prison ? Ou va-t-Il nous fermer la terre ?

\par 12 « Il est en colère contre nous parce que nous sommes sortis de la grotte sans son ordre ; et parce que nous l'avons fait de notre propre gré, sans le consulter, lorsque nous avons quitté la grotte et sommes arrivés à cet endroit.

\par 13 Alors Ève dit : « Si, en effet, la terre a tremblé à cause de nous, et que ce rocher forme une tente sur nous à cause de notre transgression, alors malheur à nous, ô Adam, car notre châtiment sera long. »

\par 14 «Mais lève-toi et prie Dieu de nous faire savoir à ce sujet et ce qu'est ce rocher qui s'étend sur nous comme une tente.»

\par 15 Alors Adam se leva et pria devant le Seigneur, pour lui faire connaître ce détroit. Et Adam resta ainsi en prière jusqu'au matin.

\chapitre{49}

\par \textit{La première prophétie de la Résurrection.}

\par 1 ALORS la Parole de Dieu vint et dit :—

\par 2 « Ô Adam, qui t'a conseillé, lorsque tu es sorti de la grotte, de venir à cet endroit ?

\par 3 Et Adam dit à Dieu : « O Seigneur, nous sommes venus à cet endroit à cause de la chaleur du feu qui s'est abattu sur nous à l'intérieur de la grotte. »

\par 4 Alors le Seigneur Dieu dit à Adam : « Ô Adam, tu redoutes la chaleur du feu pendant une nuit, mais que se passera-t-il lorsque tu habiteras en enfer ?

\par 5 «Pourtant, ô Adam, ne crains pas, et ne dis pas dans ton cœur que j'ai étendu ce rocher comme un auvent sur toi, pour t'en tourmenter.»

\par 6 «Cela venait de Satan, qui t'avait promis la Divinité et la majesté. C'est lui qui a jeté ce rocher pour te tuer dessous, et Ève avec toi, et pour t'empêcher ainsi de vivre sur la terre.

\par 7 « Mais, par miséricorde envers vous, au moment où ce rocher tombait sur vous, je lui ai ordonné de former un auvent sur vous ; et le rocher sous toi, pour s'abaisser.

\par 8 « Et ce signe, ô Adam, m'arrivera à mon avènement sur terre : Satan relèvera le peuple des Juifs et me fera mourir ; et ils me déposeront sur un rocher, et scelleront sur moi une grosse pierre, et je resterai dans ce rocher trois jours et trois nuits.

\par 9 « Mais le troisième jour, je ressusciterai, et ce sera un salut pour toi, ô Adam, et pour ta postérité, de croire en moi. Mais, ô Adam, je ne te ferai pas sortir de ce rocher avant que trois jours et trois nuits ne soient passés.

\par 10 Et Dieu retira Sa Parole d'Adam.

\par 11 Mais Adam et Ève demeurèrent sous le rocher trois jours et trois nuits, comme Dieu le leur avait dit.

\par 12 Et Dieu leur fit cela parce qu'ils avaient quitté leur grotte et étaient arrivés à ce même endroit sans l'ordre de Dieu.

\par 13 Mais, après trois jours et trois nuits, Dieu ouvrit le rocher et les fit sortir de dessous. Leur chair était desséchée, et leurs yeux et leur cœur étaient troublés par les pleurs et le chagrin.

\chapitre{50}

\par \textit{Adam et Ève cherchent à couvrir leur nudité.}

\par 1 ALORS Adam et Ève sortirent et entrèrent dans la Caverne des Trésors, et ils y restèrent en prière toute la journée, jusqu'au soir.

\par 2 Et cela eut lieu au bout de cinquante jours après qu'ils eurent quitté le jardin.

\par 3 Mais Adam et Ève se relevèrent et prièrent Dieu dans la grotte toute la nuit et lui implorèrent miséricorde.

\par 4 Et quand le jour se leva, Adam dit à Ève : « Viens ! allons faire un peu de travail pour notre corps.

\par 5 Ils sortirent donc de la grotte et arrivèrent à la limite nord du jardin, et ils cherchèrent de quoi se couvrir le corps. Mais ils ne trouvèrent rien et ne savaient pas comment faire le travail. Pourtant, leurs corps étaient tachés et ils restaient sans voix à cause du froid et de la chaleur.

\par 6 Alors Adam se leva et demanda à Dieu de lui montrer de quoi couvrir leurs corps.

\par 7 Alors vint la Parole de Dieu et lui dit : « Ô Adam, prends Ève et viens au bord de la mer, où tu jeûnais auparavant. Vous y trouverez des peaux de moutons dont la chair a été dévorée par les lions et dont les peaux ont été laissées. Prenez-les, confectionnez-vous des vêtements et habillez-vous-en.

\chapitre{51}

\par \textit{«Quelle est sa beauté pour que vous auriez dû le suivre ?»}

\par 1 QUAND Adam entendit ces paroles de Dieu, il prit Ève et s'éloigna de l'extrémité nord du jardin vers le sud de celui-ci, près de la rivière d'eau, où ils jeûnaient autrefois.

\par 2 Mais pendant qu'ils parcouraient le chemin, et avant qu'ils n'atteignent cet endroit, Satan, le méchant, avait entendu la Parole de Dieu communiquant avec Adam concernant sa couverture.

\par 3 Cela l'affligea, et il se précipita vers l'endroit où se trouvaient les peaux de mouton, avec l'intention de les prendre et de les jeter à la mer, ou de les brûler au feu, afin qu'Adam et Ève ne les trouvent pas.

\par 4 Mais alors qu'il était sur le point de les prendre, la Parole de Dieu vint du ciel et le lia par le côté de ces peaux jusqu'à ce qu'Adam et Ève s'approchent de lui. Mais en s'approchant de lui, ils eurent peur de lui et de son aspect hideux.

\par 5 Alors la Parole de Dieu fut adressée à Adam et Ève, et leur dit : « C'est celui qui était caché dans le serpent, et qui vous a trompé, et vous a dépouillé du vêtement de lumière et de gloire dans lequel vous étiez. »

\par 6 « C'est lui qui vous a promis la majesté et la divinité. Où est donc la beauté qui était sur lui ? Où est sa divinité ? Où est sa lumière ? Où est la gloire qui reposait sur lui ?

\par 7 « Maintenant, sa silhouette est hideuse ; il est devenu abominable parmi les anges ; et on l’appelle désormais Satan.

\par 8 « Ô Adam, il a voulu t'enlever ce vêtement terrestre en peau de mouton, et le détruire, et ne pas te laisser en couvrir. »

\par 9 « Quelle est donc sa beauté pour que vous l'ayez suivi ? Et qu’avez-vous gagné à l’écouter ? Voyez ses mauvaises œuvres et puis regardez-Moi ; envers Moi, votre Créateur, et envers les bonnes actions que Je vous fais.

\par 10 «Voyez, je l'ai lié jusqu'à ce que vous veniez le voir et que vous considériez sa faiblesse, afin qu'il ne lui reste plus aucun pouvoir.»

\par 11 Et Dieu le délivra de ses liens.

\chapitre{52}

\par \textit{Adam et Eve cousent la première chemise.}

\par 1 APRÈS cela Adam et Ève ne parlèrent plus, mais pleurèrent devant Dieu à cause de leur création et de leur corps qui nécessitait une couverture terrestre.

\par 2 Alors Adam dit à Ève : « Ô Ève, ceci est la peau des bêtes dont nous serons couverts. Mais quand nous l'aurons revêtu, voici, un signe de mort viendra sur nous, car les propriétaires de ces peaux sont morts et ont dépéri. Ainsi aussi, nous mourrons et passerons.

\par 3 Alors Adam et Ève prirent les peaux, et retournèrent à la Caverne des Trésors ; et quand ils y étaient, ils se levaient et priaient comme ils avaient l'habitude.

\par 4 Et ils réfléchissaient à la façon dont ils pourraient confectionner des vêtements avec ces peaux ; car ils n’avaient aucune compétence pour cela.

\par 5 Alors Dieu leur envoya son ange pour leur montrer comment y parvenir. Et l'ange dit à Adam : « Va et apporte des épines de palmier. » Alors Adam sortit et en apporta, comme l'ange le lui avait ordonné.

\par 6 Alors l'ange se mit devant eux à travailler les peaux, à la manière de celui qui prépare une chemise. Et il prit les épines et les enfonça dans la peau, sous leurs yeux.

\par 7 Alors l'ange se leva de nouveau et pria Dieu que les épines de ces peaux soient cachées, de manière à être pour ainsi dire cousues avec un seul fil.

\par 8 Et il en fut ainsi, par ordre de Dieu ; ils sont devenus des vêtements pour Adam et Ève, et Il les a revêtus de cela.

\par 9 Depuis ce temps-là, la nudité de leurs corps fut cachée à la vue des yeux l'un de l'autre.

\par 10 Et cela arriva à la fin du cinquante et unième jour.

\par 11 Alors, quand les corps d'Adam et d'Ève furent couverts, ils se levèrent et prièrent, et implorèrent la miséricorde du Seigneur et son pardon, et lui rendirent grâces pour ce qu'il avait eu pitié d'eux et pour avoir couvert leur nudité. Et ils ne cessèrent pas de prier toute la nuit.

\par 12 Puis, quand la maman se leva au lever du soleil, ils dirent leurs prières selon leur habitude ; puis il sortit de la grotte.

\par 13 Et Adam dit à Ève : « Puisque nous ne savons pas ce qu'il y a à l'ouest de cette grotte, sortons et voyons-la aujourd'hui. » Puis ils sortirent et se dirigèrent vers la frontière occidentale.



\chapitre{53}

\par \textit{La prophétie des Terres de l'Ouest.}

\par 1 ILS n'étaient pas très loin de la grotte, lorsque Satan s'approcha d'eux, et se cacha entre eux et la grotte, sous la forme de deux lions voraces trois jours sans nourriture, qui s'avançaient vers Adam et Ève, comme pour briser coupez-les en morceaux et dévorez-les.

\par 2 Alors Adam et Ève pleurèrent et prièrent Dieu de les délivrer de leurs pattes.

\par 3 Alors la Parole de Dieu leur parvint et chassa d'eux les lions.

\par 4 Et Dieu dit à Adam : « Ô Adam, que cherches-tu à la frontière occidentale ? Et pourquoi as-tu volontairement quitté la frontière orientale où se trouvait ta demeure ?

\par 5 «Maintenant, retourne dans ta caverne et reste-y, afin que Satan ne te trompe pas et n'exécute pas son dessein sur toi.»

\par 6 « Car dans cette frontière occidentale, ô Adam, sortira de toi une semence qui la reconstituera ; et qui se souilleront par leurs péchés, par leur soumission aux ordres de Satan et par le fait de suivre ses œuvres.

\par 7 « C'est pourquoi je ferai venir sur eux les eaux d'un déluge et je les submergerai tous. Mais je délivrerai ce qui reste des justes parmi eux ; et je les amènerai dans un pays lointain, et le pays dans lequel tu habites maintenant restera désolé et sans aucun habitant.

\par 8 Après que Dieu leur eut ainsi parlé, ils retournèrent à la Caverne des Trésors. Mais leur chair était desséchée, et leurs forces diminuaient à cause du jeûne et de la prière, et à cause du chagrin qu'ils éprouvaient d'avoir transgressé Dieu.

\chapitre{54}

\par \textit{Adam et Eve partent en exploration.}

\par 1 PUIS Adam et Ève se levèrent dans la grotte et prièrent toute la nuit jusqu'à l'aube du matin. Et quand le soleil se leva, ils sortirent tous deux de la grotte ; leurs têtes s'égaraient à cause du poids du chagrin, et ils ne savaient pas où ils allaient.

\par 2 Et ils marchèrent ainsi jusqu'à la limite sud du jardin. Et ils commencèrent à remonter cette frontière jusqu'à arriver à la frontière orientale au-delà de laquelle il n'y avait plus d'espace.

\par 3 Et le chérubin qui gardait le jardin se tenait à la porte occidentale et la gardait contre Adam et Ève, de peur qu'ils n'entrent soudainement dans le jardin. Et le chérubin se retourna, comme pour les faire mourir ; selon le commandement que Dieu lui avait donné.

\par 4 Quand Adam et Ève arrivèrent à la limite orientale du jardin, pensant dans leur cœur que le chérubin ne les regardait pas, alors qu'ils se tenaient près de la porte comme s'ils voulaient entrer, soudain le chérubin apparut avec une épée étincelante de lumière. le feu dans sa main ; et quand il les vit, il sortit pour les tuer. Car il craignait que Dieu ne le détruise s'ils entraient dans le jardin sans son ordre.

\par 5 Et l'épée du chérubin semblait s'enflammer au loin. Mais lorsqu’il l’éleva au-dessus d’Adam et d’Ève, la flamme ne jaillit pas.

\par 6 C'est pourquoi le chérubin pensait que Dieu leur était favorable, et il les ramenait dans le jardin. Et le chérubin se demandait.

\par 7 Il ne pouvait pas monter au Ciel pour s'assurer de l'ordre de Dieu concernant leur entrée dans le jardin ; il resta donc debout à leurs côtés, incapable de s'en séparer ; car il avait peur qu'ils n'entrent dans le jardin sans la permission de Dieu, qui alors le détruirait.

\par 8 Quand Adam et Ève virent le chérubin venir vers eux avec une épée de feu flamboyante à la main, ils tombèrent la face contre terre de peur et furent comme morts.

\par 9 En ce temps-là, les cieux et la terre furent ébranlés ; et d'autres chérubins descendirent du ciel vers le chérubin qui gardait le jardin, et le virent étonné et silencieux.

\par 10 Puis encore, d'autres anges descendirent près du lieu où se trouvaient Adam et Ève. Ils étaient partagés entre joie et tristesse.

\par 11 Ils étaient heureux, parce qu'ils pensaient que Dieu était favorable à Adam, et souhaitaient qu'il retourne au jardin ; et souhaitait lui rendre la joie dont il jouissait autrefois.

\par 12 Mais ils furent attristés à cause d'Adam, parce qu'il était tombé comme un homme mort, lui et Ève ; et ils dirent dans leurs pensées : « Adam n'est pas mort en ce lieu ; mais Dieu l'a fait mourir, parce qu'il était venu en ce lieu et qu'il voulait entrer dans le jardin sans sa permission.



\chapitre{55}

\par \textit{Le conflit de Satan.}

\par 1 ALORS la Parole de Dieu fut adressée à Adam et Ève, et les ressuscita de leur état mort, en leur disant : « Pourquoi êtes-vous montés ici ? Docteur, avez-vous l'intention d'aller dans le jardin d'où je vous ai fait sortir ? cela ne peut pas être le cas aujourd’hui ; mais seulement lorsque l’alliance que j’ai conclue avec toi sera accomplie.

\par 2 Alors Adam, lorsqu'il entendit la Parole de Dieu et le battement des anges qu'il ne voyait pas, mais qu'il entendait seulement le bruit d'eux avec ses oreilles, lui et Ève pleurèrent et dirent aux anges :

\par 3 « Ô Esprits, qui vous attendez à Dieu, regardez-moi, et voyez que je ne peux vous voir ! Car quand j'étais dans mon ancienne nature brillante, alors je pouvais te voir. J'ai chanté des louanges comme vous ; et mon cœur était bien au-dessus de toi.

\par 4 « Mais maintenant que j'ai transgressé, cette nature brillante a disparu de moi, et je suis arrivé à cet état misérable. Et maintenant j’en suis arrivé à ce que je ne peux pas vous voir, et vous ne me servez pas comme vous aviez l’habitude. Car je suis devenu chair animale.

\par 5 « Et maintenant, ô anges de Dieu, demandez à Dieu avec moi de me restaurer là où j'étais autrefois ; pour me délivrer de cette misère et pour ôter de moi la sentence de mort qu'il m'a prononcée, pour avoir transgressé son autorité.

\par 6 Alors, quand les anges entendirent ces paroles, ils furent tous attristés à cause de lui ; et maudit Satan qui avait séduit Adam, jusqu'à ce qu'il revienne du jardin à la misère ; de la vie à la mort ; de la paix aux ennuis ; et de la joie vers un pays étranger.

\par 7 Alors les anges dirent à Adam : « Tu as écouté Satan et tu as abandonné la Parole de Dieu qui t'a créé ; et tu croyais que Satan accomplirait tout ce qu'il t'avait promis.

\par 8 «Mais maintenant, ô Adam, nous allons te faire connaître ce qui nous est arrivé par lui, avant sa chute du ciel.»

\par 9 « Il rassembla ses armées et les trompa, leur promettant de leur donner un grand royaume, une nature divine ; et d'autres promesses qu'il leur a faites.

\par 10 « Ses hôtes croyaient cela. sa parole était vraie, alors ils lui cédèrent et renoncèrent à la gloire de Dieu.

\par 11 « Alors il nous a fait venir selon les ordres dans lesquels nous devions nous placer sous son commandement et écouter sa vaine promesse. Mais nous ne l’avons pas fait et nous n’avons pas suivi ses conseils.

\par 12 «Puis après avoir combattu avec Dieu et lui avoir agi en avant, il rassembla ses armées et nous fit la guerre. Et sans la force de Dieu qui était avec nous, nous n’aurions pas pu l’emporter sur lui pour le précipiter du ciel.

\par 13 « Mais quand il tomba du milieu de nous, il y eut une grande joie dans le ciel, à cause de sa descente du milieu de nous. Car s’il avait continué au ciel, rien, pas même un seul ange, n’y serait resté.

\par 14 « Mais Dieu, dans sa miséricorde, l'a chassé du milieu de nous vers cette terre sombre ; car il était devenu les ténèbres elles-mêmes et un ouvrier d’injustice.

\par 15 « Et il a continué, ô Adam, à te faire la guerre, jusqu'à ce qu'il t'ait séduit et te fasse sortir du jardin, vers ce pays étranger, où toutes ces épreuves t'ont atteint. Et la mort que Dieu lui a apportée, il t'a aussi apporté la mort, ô Adam, parce que tu lui as obéi et que tu as transgressé contre Dieu.

\par 16 Alors les anges se réjouirent et louèrent Dieu, et lui demandèrent de ne pas détruire Adam cette fois, parce qu'il avait cherché à entrer dans le jardin ; mais le supporter jusqu'à l'accomplissement de la promesse ; et pour l'aider dans ce monde jusqu'à ce qu'il soit libéré de la main de Satan.

\chapitre{56}

\par \textit{Un chapitre de réconfort divin.}

\par 1 ALORS la Parole de Dieu vint à Adam et lui dit : —

\par 2 « Ô Adam, regarde ce jardin de joie et cette terre de labeur, et vois les anges qui sont dans le jardin qui en est plein, et vois-toi seul sur cette terre, avec Satan à qui tu as obéi. .»

\par 3 « Pourtant, si tu m'avais soumis, si tu m'avais obéi et si tu avais gardé ma parole, tu serais avec mes anges dans mon jardin. »

\par 4 « Mais quand tu as transgressé et écouté Satan, tu es devenu son hôte parmi ses anges, qui sont pleins de méchanceté ; et tu es venu sur cette terre, qui t'a donné des épines et des chardons.

\par 5 « Ô Adam, demande à celui qui t'a trompé, de te donner la nature divine qu'il t'a promise, ou de te faire un jardin comme je l'avais fait pour toi ; ou pour te remplir de cette même nature lumineuse dont je t'avais rempli.

\par 6 « Demande-lui de te faire un corps semblable à celui que je t'ai fait, ou de te donner un jour de repos comme je t'ai donné ; ou pour créer en toi une âme raisonnable, comme j'ai créé pour toi ; ou pour te déplacer d'ici vers une autre terre que celle que je t'ai donnée. Mais, ô Adam, il n’accomplira même pas une seule des choses qu’il t’a dit.

\par 7 « Reconnais donc ma faveur envers toi et ma miséricorde envers toi, ma créature ; que je ne t'ai pas récompensé pour ta transgression contre moi, mais dans ma pitié pour toi, je t'ai promis qu'à la fin des cinq grands jours et demi, je viendrai te sauver.

\par 8 Alors Dieu dit encore à Adam et Ève : « Lève-toi, descends d'ici, de peur que le chérubin avec une épée de feu à la main ne te détruise. »

\par 9 Mais le cœur d'Adam fut réconforté par les paroles que Dieu lui avait adressées, et il adora devant Lui.

\par 10 Et Dieu ordonna à ses anges d'escorter Adam et Ève jusqu'à la grotte avec joie, au lieu de la peur qui les avait envahis.

\par 11 Alors les anges prirent Adam et Ève et les firent descendre de la montagne près du jardin, avec des chants et des psaumes, jusqu'à ce qu'ils les conduisirent à la grotte. Là, les anges commencèrent à les réconforter et à les fortifier, puis s'éloignèrent d'eux vers le ciel, vers leur Créateur qui les avait envoyés.

\par 12 Mais, après que les anges furent partis d'Adam et Ève, Satan vint, le visage honteux, et se tint à l'entrée de la caverne dans laquelle se trouvaient Adam et Ève. Il appela alors Adam et lui dit : « Ô Adam, viens, laisse-moi te parler. »

\par 13 Alors Adam sortit de la grotte, pensant qu'il était un des anges de Dieu venu lui donner de bons conseils.

\chapitre{57}

\par \textit{«C'est pourquoi je suis tombé. . .}”

\par 1 MAIS quand Adam sortit et vit sa silhouette hideuse, il eut peur de lui et lui dit : « Qui es-tu ?

\par 2 Alors Satan répondit et lui dit : « C'est moi qui me suis caché dans le serpent, et qui ai parlé à Ève et qui l'ai séduite jusqu'à ce qu'elle écoute mon ordre. C'est moi qui l'ai envoyée, par les ruses de ma parole, pour te tromper, jusqu'à ce que toi et elle mangiez du fruit de l'arbre, et que vous vous éloigniez du commandement de Dieu.

\par 3 Mais quand Adam entendit ces paroles de sa part, il lui dit : « Peux-tu me faire un jardin comme Dieu l'a fait pour moi ? Ou peux-tu me revêtir de la même nature lumineuse dont Dieu m’avait revêtu ?

\par 4 « Où est la nature divine que tu as promis de me donner ? Où est ton beau discours que tu nous as d’abord tenu lorsque nous étions dans le jardin ?

\par 5 Alors Satan dit à Adam : « Penses-tu que lorsque j'ai parlé à quelqu'un de quelque chose, je le lui apporterai ou j'accomplirai ma parole ? Ce n’est pas le cas. Car je n’ai jamais pensé moi-même à obtenir ce que j’ai demandé.

\par 6 « C'est pourquoi je suis tombé, et je vous ai fait tomber par ce pour quoi je suis tombé moi-même ; et avec toi aussi, quiconque accepte mon conseil tombe.

\par 7 « Mais maintenant, ô Adam, à cause de ta chute, tu es sous mon règne, et je suis ton roi ; parce que tu m'as écouté et que tu as transgressé contre ton Dieu. Il n'y aura aucune délivrance de mes mains jusqu'au jour promis par ton Dieu.

\par 8 Il dit encore : « Puisque nous ne connaissons pas le jour convenu avec toi par ton Dieu, ni l'heure à laquelle tu seras délivré, c'est pour cette raison que nous multiplierons la guerre et le meurtre contre toi et ta postérité après toi. .»

\par 9 «C'est notre volonté et notre bon plaisir, afin que nous ne laissions aucun des fils des hommes hériter de nos ordres dans le ciel.»

\par 10 « Car quant à notre demeure, ô Adam, elle est dans un feu brûlant ; et nous ne cesserons de faire le mal, ni un jour ni une heure. Et moi, ô Adam, je sèmerai du feu sur toi quand tu entreras dans la grotte pour y habiter.

\par 11 Quand Adam entendit ces paroles, il pleura et se lamenta, et dit à Ève : « Écoute ce qu'il a dit ; qu'il n'accomplira rien de ce qu'il t'a dit dans le jardin. Est-il vraiment devenu notre roi ?

\par 12 «Mais nous demanderons à Dieu, qui nous a créés, de nous délivrer de ses mains.»

\chapitre{58}

\par \textit{« À propos du coucher du soleil le 53ème jour. . . . »}

\par 1 ALORS Adam et Ève étendirent leurs mains vers Dieu, le priant et le suppliant de chasser Satan loin d'eux ; qu'il ne leur fasse aucune violence et ne les force pas à renier Dieu.

\par 2 Alors Dieu leur envoya aussitôt son ange, qui chassa Satan d'eux. Cela arriva vers le coucher du soleil, le cinquante-troisième jour après qu'ils furent sortis du jardin.

\par 3 Alors Adam et Ève entrèrent dans la grotte, se levèrent et tournèrent leur face vers la terre pour prier Dieu.

\par 4 Mais avant qu'ils prient, Adam dit à Ève : « Voici, tu as vu quelles tentations nous sont arrivées dans ce pays. Venez, levons-nous et demandons à Dieu de nous pardonner les péchés que nous avons commis ; et nous ne sortirons qu'en fin de journée, vers le quarantième. Et si nous mourons ici, Il nous sauvera.

\par 5 Alors Adam et Ève se levèrent et s'unirent pour implorer Dieu.

\par 6 Ils restèrent ainsi en prière dans la grotte ; ils n'en sortaient ni de nuit ni de jour, jusqu'à ce que leurs prières sortent de leur bouche comme une flamme de feu.

\chapitre{59}

\par \textit{Huitième apparition de Satan à Adam et Ève.}

\par 1 MAIS Satan, le haineux de tout bien, ne leur a pas permis de mettre fin à leurs prières. Car il appela ses hôtes, et ils vinrent tous. Il leur dit alors : « Puisque Adam et Ève, que nous avons séduits, sont convenus ensemble de prier Dieu nuit et jour et de le supplier de les délivrer, et qu'ils ne sortiront de la grotte qu'à la fin des temps. quarantième jour.

\par 2 «Et puisqu'ils continueront leurs prières comme ils ont tous deux convenu de le faire, afin qu'Il les délivre de nos mains et les rétablisse dans leur état antérieur, voyez ce que nous leur ferons.» Et ses armées lui dirent : « Tu as le pouvoir, ô notre Seigneur, de faire ce que tu veux. »

\par 3 Alors Satan, grand en méchanceté, prit ses armées et entra dans la grotte, la trentième nuit des quarante jours et un ; et il frappa Adam et Eve, jusqu'à ce qu'il les laisse morts.

\par 4 Alors la Parole de Dieu fut adressée à Adam et Ève, qui les relevèrent de leurs souffrances, et Dieu dit à Adam : « Sois fort et n'aie pas peur de celui qui vient de venir à toi. »

\par 5 Mais Adam pleura et dit : « Où étais-tu, ô mon Dieu, pour qu'ils me frappent de tels coups, et que cette souffrance nous arrive ; sur moi et sur Ève, ta servante ?

\par 6 Alors Dieu lui dit : « Ô Adam, vois, il est seigneur et maître de tout ce que tu as, lui qui a dit qu'il te donnerait la divinité. Où est cet amour pour toi ? Et où est le cadeau qu’il a promis ?

\par 7 « Car une fois qu'il lui a plu, ô Adam, de venir à toi, pour te consoler, et pour te fortifier, et pour se réjouir avec toi, et pour envoyer ses armées pour te garder ; parce que tu l'as écouté et que tu as cédé à son conseil ; et tu as transgressé Mon commandement mais tu as suivi son ordre ?

\par 8 Alors Adam pleura devant le Seigneur et dit : « Ô Seigneur, parce que j'ai un peu transgressé, tu m'as cruellement tourmenté en retour, je te demande de me délivrer de ses mains ; ou bien ayez pitié de moi et enlevez mon âme de mon corps maintenant dans ce pays étranger.

\par 9 Alors Dieu dit à Adam : « Si seulement il y avait eu ces soupirs et ces prières auparavant, avant que tu ne transgresses ! Alors tu voudrais être reposé des ennuis dans lesquels tu te trouves actuellement.

\par 10 Mais Dieu eut patience envers Adam, et le laissa ainsi qu'Ève rester dans la grotte jusqu'à ce qu'ils aient accompli les quarante jours.

\par 11 Mais quant à Adam et Ève, leur force et leur chair se desséchèrent à cause du jeûne et de la prière, à cause de la faim et de la soif ; car ils n'avaient goûté ni nourriture ni boisson depuis qu'ils avaient quitté le jardin ; les fonctions de leur corps n'étaient pas non plus encore réglées ; et ils n'avaient plus la force de continuer à prier à cause de la faim, jusqu'à la fin du lendemain, le quarantième. Ils étaient tombés dans la grotte ; pourtant ce qui s'échappait de leurs bouches n'était que des louanges.

\chapitre{60}

\par \textit{Le Diable apparaît comme un vieil homme. Il offre «un lieu de repos».}

\par 1 PUIS, le quatre-vingt-neuvième jour, Satan entra dans la grotte, vêtu d'un vêtement de lumière et ceint d'une ceinture lumineuse.

\par 2 Dans ses mains il y avait un bâton de lumière, et il avait l'air très affreux ; mais son visage était agréable et sa parole était douce,

\par 3 Il s'est ainsi transformé afin de tromper Adam et Ève, et de les faire sortir de la grotte, avant qu'ils n'aient accompli les quarante jours.

\par 4 Car il se disait en lui-même : « Maintenant qu'après qu'ils auront accompli les quarante jours de jeûne et de prière, Dieu les rendra à leur ancien état ; mais s’il ne le faisait pas, il leur serait quand même favorable ; et même s'il n'avait pas pitié d'eux, leur donnerait-il quand même quelque chose du jardin pour les réconforter ? comme déjà deux fois auparavant.

\par 5 Alors Satan s'approcha de la grotte sous cette belle apparence, et dit :

\par 6 « Ô Adam, levez-vous, levez-vous, vous et Ève, et venez avec moi dans un bon pays ; et n'ayez crainte. Je suis chair et os comme toi ; et au début j’étais une créature que Dieu avait créée.

\par 7 «Et c'est ainsi que lorsqu'il m'eut créé, il me plaça dans un jardin au nord, à la frontière du monde.»

\par 8 « Et il me dit : « Demeure ici ! Et j’y suis resté selon sa parole, et je n’ai pas non plus transgressé son commandement.

\par 9 «Puis il a provoqué un sommeil sur moi, et il t'a fait sortir, ô Adam, de mon côté, mais il ne t'a pas fait demeurer près de moi.»

\par 10 «Mais Dieu te prit dans sa main divine et te plaça dans un jardin à l'est.»

\par 11 «Alors j'ai été affligé à cause de toi, car pendant que Dieu t'avait retiré de mon côté, il ne t'avait pas laissé demeurer avec moi.»

\par 12 « Mais Dieu m'a dit : 'Ne t'afflige pas à cause d'Adam, que j'ai fait sortir de ton côté ; aucun mal ne lui arrivera.

\par 13 « 'Pour l'instant, je lui ai fait sortir de son côté une aide pour lui ; et je lui ai donné de la joie en faisant cela.'»

\par 14 Alors Satan dit encore : « Je ne savais pas comment vous vous trouviez dans cette grotte, ni rien de cette épreuve qui vous est arrivée - jusqu'à ce que Dieu me dise : « Voici, Adam a transgressé, celui que j'avais enlevé de ton côté, et Ève aussi, que j'ai retirée de son côté ; et je les ai chassés du jardin ; Je les ai fait habiter dans un pays de tristesse et de misère, parce qu'ils ont transgressé contre moi et ont écouté Satan. Et voici, ils souffrent jusqu'à ce jour, le quatre-vingtième.'»

\par 15 « Alors Dieu me dit : Lève-toi, va vers eux et fais-les venir chez toi, et ne souffre pas que Satan s'approche d'eux et les afflige. Car ils sont maintenant dans une grande misère ; et je reste impuissant à cause de la faim.

\par 16 « Il me dit encore : 'Quand tu les auras pris pour toi, donne-leur à manger du fruit de l'Arbre de Vie, et donne-leur à boire de l'eau de paix ; et habille-les d'un vêtement de lumière, et redonne-leur leur ancien état de grâce, et ne les laisse pas dans le malheur, car ils sont venus de toi. Mais ne vous attristez pas à leur sujet et ne vous repentez pas de ce qui leur est arrivé.'»

\par 17 « Mais quand j'ai entendu cela, j'ai été désolé ; et mon cœur ne pouvait pas le supporter patiemment à cause de toi, ô mon enfant.

\par 18 «Mais, ô Adam, quand j'ai entendu le nom de Satan, j'ai eu peur, et je me suis dit en moi-même : je ne sortirai pas, de peur qu'il ne me piége, comme il l'a fait pour mes enfants, Adam et Ève.»

\par 19 « Et j'ai dit : 'Ô Dieu, quand je vais vers mes enfants, Satan me rencontrera en chemin et me fera la guerre, comme il l'a fait contre eux.' »

\par 20 « Alors Dieu me dit : 'Ne crains pas ; quand tu le trouveras, frappe-le avec le bâton que tu as à la main, et n'aie pas peur de lui, car tu es vieux, et il ne prévaudra pas contre toi.

\par 21 « Alors je dis : 'Ô mon Seigneur, je suis vieux et je ne peux pas partir. Envoie tes anges pour les amener.'»

\par 22 « Mais Dieu m'a dit : 'Les anges, en vérité, ne sont pas comme eux ; et ils ne consentiront pas à venir avec eux. Mais je t'ai choisi, parce qu'ils sont ta postérité et comme toi, et ils écouteront ce que tu dis.

\par 23 « Dieu me dit encore : 'Si tu n'as pas la force de marcher, j'enverrai une nuée pour te porter et te poser à l'entrée de leur grotte ; alors la nuée reviendra et te laissera là.

\par 24 « 'Et s'ils viennent avec toi, j'enverrai une nuée pour te transporter, toi et eux.' »

\par 25 « Alors il commanda à une nuée, et elle m'a mis à nu et m'a amené vers toi ; puis je suis reparti.

\par 26 « Et maintenant, ô mes enfants, Adam et Ève, regardez mes cheveux blancs et ma faiblesse, et mon arrivée de ce lieu lointain. Viens, viens avec moi, dans un lieu de repos.

\par 27 Alors il se mit à pleurer et à sangloter devant Adam et Ève, et ses larmes se répandirent sur la terre comme de l'eau.

\par 28 Et quand Adam et Ève levèrent les yeux et virent sa barbe et entendirent ses douces paroles, leurs cœurs s'adoucirent envers lui ; ils l'écoutèrent, car ils croyaient qu'il était vrai.

\par 29 Et il leur sembla qu'ils étaient réellement sa postérité, lorsqu'ils virent que son visage était comme le leur ; et ils lui faisaient confiance.

\chapitre{61}

\par \textit{Ils commencent à suivre Satan.}

\par 1 PUIS il prit Adam et Ève par la main et commença à les faire sortir de la grotte.

\par 2 Mais quand ils en furent un peu éloignés, Dieu comprit que Satan les avait vaincus et les avait fait sortir avant la fin des quarante jours, pour les emmener dans un endroit éloigné et les détruire.

\par 3 Alors la Parole du Seigneur Dieu vint de nouveau et maudit Satan, et le chassa d'eux.

\par 4 Et Dieu commença à parler à Adam et Ève, leur disant : « Qu'est-ce qui vous a fait sortir de la grotte jusqu'à cet endroit ?

\par 5 Alors Adam dit à Dieu : « As-tu créé un homme avant nous ? Car alors que nous étions dans la grotte, soudain un bon vieillard nous vint à nous et nous dit : « Je suis un messager de Dieu pour vous, pour vous ramener dans un lieu de repos. »

\par 6 « Et nous avons cru, ô Dieu, qu'il était un messager de ta part ; et nous sommes sortis avec lui; et je ne savais pas où nous devions aller avec lui.

\par 7 Alors Dieu dit à Adam : « Vois, c'est le père des arts mauvais, qui t'a fait sortir, toi et Ève, du Jardin des Délices. Et maintenant, en effet, quand il a vu que toi et Ève vous étiez unis pour jeûner et prier, et que vous n'étiez pas sortis de la grotte avant la fin des quarante jours, il a voulu rendre votre dessein vain, rompre votre lien mutuel. ; pour vous couper tout espoir et vous conduire dans un endroit où il pourrait vous détruire.

\par 8 «Parce qu'il ne pouvait rien vous faire, s'il ne se montrait à votre image.»

\par 9 «C'est pourquoi il est venu vers vous avec un visage semblable au vôtre, et il a commencé à vous donner des signes comme s'ils étaient tous vrais.»

\par 10 « Mais moi, par miséricorde et avec la faveur que je vous avais, je ne lui ai pas permis de vous détruire ; mais je l'ai chassé de toi.

\par 11 « Maintenant donc, ô Adam, prends Ève, retourne dans ta grotte et reste-y jusqu'au lendemain du quarantième jour. Et quand vous sortirez, dirigez-vous vers la porte orientale du jardin.

\par 12 Alors Adam et Ève adorèrent Dieu, le louèrent et le bénirent pour la délivrance qui leur était venue de lui. Et ils revinrent vers la grotte. Cela s'est produit le soir du trente-neuvième jour.

\par 13 Alors Adam et Ève se levèrent et avec un grand zèle, prièrent Dieu d'être délivrés de leur manque de force ; car leurs forces les avaient quittés à cause de la faim, de la soif et de la prière. Mais ils veillèrent toute la nuit à prier, jusqu'au matin.

\par 14 Alors Adam dit à Ève : « Lève-toi, allons vers la porte orientale du jardin comme Dieu nous l'a dit. »

\par 15 Et ils disaient leurs prières comme ils avaient l'habitude de le faire chaque jour ; et ils sortirent de la grotte pour s'approcher de la porte orientale du jardin.

\par 16 Alors Adam et Ève se levèrent et prièrent, et supplièrent Dieu de les fortifier et de leur envoyer de quoi satisfaire leur faim.

\par 17 Mais quand ils eurent fini leurs prières, ils restèrent là où ils étaient à cause de leurs forces défaillantes.

\par 18. Alors la Parole de Dieu revint et leur dit : « Ô Adam, levez-vous, allez et apportez ici deux figues. »

\par 19 Alors Adam et Ève se levèrent et allèrent jusqu'à ce qu'ils s'approchent de la grotte.

\chapitre{62}

\par \textit{Deux arbres fruitiers.}

\par 1 MAIS Satan le méchant était envieux, à cause de la consolation que Dieu leur avait donnée.

\par 2 Alors il les en empêcha, entra dans la grotte, prit les deux figues et les enterra hors de la grotte, afin qu'Adam et Ève ne les trouvent pas. Il avait également en tête de les détruire.

\par 3 Mais par la miséricorde de Dieu, dès que ces deux figues furent sur la terre, Dieu déjoua le conseil de Satan à leur sujet ; et il en fit deux arbres fruitiers qui couvraient la grotte. Car Satan les avait enterrés du côté oriental.

\par 4 Alors, quand les deux arbres furent poussés et furent couverts de fruits, Satan fut attristé et pleuré, et dit : « Mieux valait-il laisser ces figues telles qu'elles étaient ; car maintenant voici, ils sont devenus deux arbres fruitiers, dont Adam mangera tous les jours de sa vie. Alors que j’avais en tête, en les enterrant, de les détruire entièrement et de les cacher pour toujours.

\par 5 « Mais Dieu a renversé mon conseil ; et je ne voudrais pas que ce fruit sacré périsse ; et il a clairement exposé mes intentions et a déjoué le conseil que j'avais formé contre ses serviteurs.

\par 6 Alors Satan s'en alla honteux de n'avoir pas réalisé son dessein.

\chapitre{63}

\par \textit{La première joie des arbres.}

\par 1 MAIS Adam et Ève, alors qu'ils s'approchaient de la grotte, virent deux figuiers couverts de fruits et qui éclipsaient la grotte.

\par 2 Alors Adam dit à Ève : « Il me semble que nous nous sommes égarés. Quand ces deux arbres ont-ils poussé ici ? Il me semble que l’ennemi veut nous égarer. Dis-tu qu’il y a sur terre une autre grotte que celle-ci ?

\par 3 « Pourtant, ô Ève, entrons dans la grotte, et trouvons-y les deux figues ; car c'est notre grotte dans laquelle nous étions. Mais si nous n’y trouvons pas les deux figues, alors ce ne peut pas être notre grotte.

\par 4 Ils entrèrent alors dans la grotte, et regardèrent aux quatre coins, mais ne trouvèrent pas les deux figues.

\par 5 Et Adam pleura et dit à Ève : « Sommes-nous donc dans une mauvaise grotte, ô Ève ? Il me semble que ces deux figuiers sont les deux figues qui étaient dans la grotte. Et Ève répondit : « Moi, je ne sais pas. »

\par 6 Alors Adam se leva et pria et dit : « Ô Dieu, tu nous as ordonné de revenir à la grotte, de prendre les deux figues, puis de retourner vers toi. »

\par 7 « Mais maintenant, nous ne les avons pas trouvés. Ô Dieu, les as-tu pris et semé ces deux arbres, ou sommes-nous égarés sur la terre ? ou l'ennemi nous a-t-il trompé ? Si cela est réel, alors, ô Dieu, révèle-nous le secret de ces deux arbres et des deux figuiers.

\par 8 Alors la Parole de Dieu vint à Adam et lui dit : « Ô Adam, quand je t'ai envoyé chercher les figues, Satan t'a précédé dans la grotte, a pris les figues et les a enterrées dehors, à l'est de la grotte. grotte, pensant les détruire ; et ne pas les semer avec de bonnes intentions.

\par 9 « Ce n'est donc pas pour lui que ces arbres ont poussé d'un coup ; mais j'ai eu pitié de toi et je leur ai commandé de grandir. Et ils devinrent deux grands arbres, afin que vous soyez couvert de leurs branches et que vous trouviez du repos ; et que je vous fais voir ma puissance et mes œuvres merveilleuses.

\par 10 « Et aussi, pour vous montrer la méchanceté de Satan et ses mauvaises œuvres, car depuis que vous êtes sortis du jardin, il n'a cessé, non, pas un seul jour, de vous faire du mal. Mais je ne lui ai pas donné pouvoir sur vous.

\par 11 Et Dieu dit : « Désormais, ô Adam, réjouis-toi à cause des arbres, toi et Ève ; et reposez-vous sous eux quand vous vous sentez fatigué. Mais ne mange pas de leurs fruits et ne t’approche pas d’eux.

\par 12 Alors Adam pleura et dit : « Ô Dieu, vas-tu nous tuer encore, ou vas-tu nous chasser de devant ta face et couper notre vie de la surface de la terre ?

\par 13 « Ô Dieu, je t'en supplie, si tu sais qu'il y a dans ces arbres soit la mort, soit quelque autre mal, comme la première fois, déracine-les près de notre caverne et dessèche-les ; et laissez-nous mourir de chaleur, de faim et de soif.

\par 14 « Car nous connaissons tes merveilles, ô Dieu, qu'elles sont grandes, et que par ta puissance tu peux faire sortir une chose d'une autre, sans le vouloir de chacun. Car ta puissance peut transformer les rochers en arbres, et les arbres en rochers.

\chapitre{64}

\par \textit{Adam et Ève participent à la première nourriture terrestre.}

\par 1 ALORS Dieu regarda Adam et sa force d'esprit, son endurance à la faim, à la soif et à la chaleur. Et il changea les deux figuiers en deux figues, comme c'était le cas au début, puis il dit à Adam et à Ève : « Chacun de vous peut prendre une figue. » Et ils les prirent, comme le Seigneur le leur avait ordonné.

\par 2 Et il leur dit : « Allez dans la grotte, mangez des figues et rassasiez votre faim, de peur que vous ne mourriez. »

\par 3 Ainsi, comme Dieu le leur avait ordonné, ils entrèrent dans la grotte, à peu près au moment où le soleil se couchait. Et Adam et Eve se levèrent et prièrent au moment du coucher du soleil.

\par 4 Puis ils s'assirent pour manger les figues ; mais ils ne savaient pas comment les manger ; car ils n'étaient pas habitués à manger de la nourriture terrestre. Ils craignaient aussi que, s'ils mangeaient, leur estomac ne soit chargé et leur chair ne s'épaississe, et que leur cœur ne se mette à aimer la nourriture terrestre.

\par 5 Mais pendant qu'ils étaient ainsi assis, Dieu, par pitié pour eux, leur envoya son ange, de peur qu'ils ne périssent de faim et de soif.

\par 6 Et l'ange dit à Adam et Ève : « Dieu vous dit que vous n'avez pas la force de jeûner jusqu'à la mort ; mangez donc et fortifiez votre corps ; car vous êtes maintenant de la chair animale, qui ne peut subsister sans nourriture ni boisson.

\par 7 Alors Adam et Ève prirent les figues et commencèrent à en manger. Mais Dieu avait mis en eux un mélange de pain savoureux et de sang.

\par 8 Alors l'ange quitta Adam et Ève, qui mangèrent des figues jusqu'à ce qu'ils aient rassasié leur faim. Puis ils ont mis de côté ce qui restait ; mais par la puissance de Dieu, les figues furent pleines comme auparavant, parce que Dieu les bénit.

\par 9 Après cela, Adam et Ève se levèrent et prièrent avec un cœur joyeux et une force renouvelée, et ils louèrent et se réjouirent abondamment toute la nuit. Et c'était la fin du quatre-vingt-troisième jour.

\chapitre{65}

\par \textit{Adam et Eve acquièrent des organes digestifs. Le dernier espoir de retourner au Jardin est anéanti.}

\par 1 ET quand il fit jour, ils se levèrent et prièrent, selon leur coutume, puis sortirent de la grotte.

\par 2 Mais comme ils éprouvaient de grands ennuis à cause de la nourriture qu'ils avaient mangée et à laquelle ils n'étaient pas habitués, ils allaient et venaient dans la grotte en se disant :

\par 3 « Que nous est-il arrivé en mangeant, pour que cette douleur nous arrive ? Malheur à nous, nous allons mourir ! Mieux vaut pour nous mourir que manger ; et d'avoir gardé nos corps purs, plutôt que de les avoir souillés par la nourriture.

\par 4 Alors Adam dit à Ève : « Cette douleur ne nous est pas venue dans le jardin, et nous n'y avons pas non plus mangé une nourriture aussi mauvaise. Penses-tu, ô Ève, que Dieu nous tourmentera par la nourriture qui est en nous, ou que nos entrailles sortiront ; ou que Dieu a l’intention de nous tuer avec cette douleur avant d’avoir rempli sa promesse ?

\par 5 Alors Adam supplia le Seigneur et dit : « O Seigneur, ne périssons pas à cause de la nourriture que nous avons mangée. O Seigneur, ne nous frappe pas ; mais traite-nous selon ta grande miséricorde, et ne nous abandonne pas jusqu'au jour de la promesse que tu nous as faite.

\par 6 Alors Dieu les regarda, et aussitôt les fit manger de la nourriture ; comme jusqu'à ce jour; afin qu'ils ne périssent pas.

\par 7 Alors Adam et Ève revinrent dans la grotte tristes et pleurant à cause du changement de leur nature. Et ils savaient tous deux à partir de ce moment-là qu'ils étaient des êtres modifiés, que leur espoir de retourner au jardin était désormais coupé ; et qu'ils ne pouvaient pas y entrer.

\par 8 C'est pourquoi leurs corps avaient désormais d'étranges fonctions ; et toute chair qui a besoin de nourriture et de boisson pour son existence ne peut pas être dans le jardin.

\par 9 Alors Adam dit à Ève : « Voici, notre espérance est maintenant coupée ; tout comme notre confiance pour entrer dans le jardin. Nous n'appartenons plus aux habitants du jardin ; mais désormais nous sommes terrestres et de la poussière, et des habitants de la terre, nous ne retournerons pas au jardin, jusqu'au jour où Dieu a promis de nous sauver et de nous ramener dans le jardin, comme il l'a promis. nous.»

\par 10 Alors ils prièrent Dieu pour qu'il ait pitié d'eux ; après quoi, leur esprit fut apaisé, leurs cœurs brisés et leur désir refroidi ; et ils étaient comme des étrangers sur terre. Adam et Ève passèrent cette nuit-là dans la grotte, où ils dormirent lourdement à cause de la nourriture qu'ils avaient mangée.

\chapitre{66}

\par \textit{Adam fait son premier jour de travail.}

\par 1 QUAND c'était le matin, le lendemain après avoir mangé de la nourriture, Adam et Ève prièrent dans la grotte, et Adam dit à Ève : « Voici, nous avons demandé de la nourriture à Dieu, et Il l'a donnée. Mais maintenant, demandons-Lui aussi de nous donner à boire de l’eau.

\par 2 Alors ils se levèrent et se dirigèrent vers le bord du ruisseau qui était sur la limite sud du jardin, dans lequel ils s'étaient jetés auparavant. Et ils se tinrent sur la rive et prièrent Dieu de leur ordonner de boire de l'eau.

\par 3 Alors la Parole de Dieu vint à Adam et lui dit : « Ô Adam, ton corps est devenu stupide et a besoin d'eau pour boire. Prenez et buvez, toi et Ève ; rendez grâce et louez.

\par 4 Adam et Ève s'approchèrent alors et en burent, jusqu'à ce que leurs corps se sentent rafraîchis. Après avoir bu, ils louèrent Dieu, puis retournèrent dans leur grotte, selon leur ancienne coutume. Cela s'est produit au bout de quatre-vingt-trois jours.

\par 5 Et le quatre-vingt-quatrième jour, ils prirent deux figues et les suspendirent dans la grotte avec leurs feuilles, pour être pour eux un signe et une bénédiction de Dieu. Et ils les placèrent là jusqu'à ce qu'il leur apparaisse une postérité qui verrait les choses merveilleuses que Dieu leur avait faites.

\par 6 Alors Adam et Ève se tinrent de nouveau à l'extérieur de la grotte et prièrent Dieu de leur montrer de la nourriture avec laquelle nourrir leur corps.

\par 7 Alors la Parole de Dieu vint et lui dit : « Ô Adam, descends à l'ouest de la grotte, jusqu'à un pays au sol sombre, et là tu trouveras de la nourriture.

\par 8 Et Adam écouta la Parole de Dieu, prit Ève, et descendit dans un pays au sol sombre, et y trouva du blé poussant en épi et mûr, et des figues à manger ; et Adam s'en réjouit.

\par 9 Alors la Parole de Dieu revint à Adam et lui dit : « Prends de ce blé et fais-en du pain, pour nourrir ton corps avec. » Et Dieu a donné au cœur d'Adam la sagesse de travailler le maïs jusqu'à ce qu'il devienne du pain.

\par 10 Adam accomplit tout cela, jusqu'à ce qu'il devienne très faible et fatigué. Il retourna ensuite à la grotte ; se réjouissant de ce qu'il avait appris sur ce qu'on fait du blé, jusqu'à ce qu'on en fasse du pain pour son usage.

\chapitre{67}

\par \textit{« Alors Satan commença à égarer Adam et Ève. . . . »}

\par 1 MAIS quand Adam et Ève descendirent au pays de la boue noire et s'approchèrent du blé que Dieu leur avait montré, et le virent mûr et prêt à être récolté, car ils n'avaient pas de faucille pour le récolter, ils se ceignirent. , et commença à arracher le blé, jusqu'à ce que tout soit fini.

\par 2 Puis ils en firent un tas ; Et, évanouis par la chaleur et la soif, ils allèrent sous un arbre ombragé, où la brise les attisa pour dormir.

\par 3 Mais Satan a vu ce qu'Adam et Ève avaient fait. Et il appela ses hôtes et leur dit : « Puisque Dieu a montré à Adam et Ève tout ce qui concerne ce blé, avec lequel fortifier leurs corps, et voici, ils sont venus et en ont fait un tas, et ils s'évanouissent à cause du Les travailleurs sont maintenant endormis. Allons, mettons le feu à ce tas de blé, brûlons-le, et prenons cette bouteille d'eau qui est à côté d'eux, et vidons-la, afin qu'ils ne trouvent rien à boire, et nous tuez-les de faim et de soif.

\par 4 « Alors, quand ils se réveilleront de leur sommeil et chercheront à retourner à la grotte, nous les rencontrerons par le chemin et les égarerons ; pour qu'ils meurent de faim et de soif ; quand ils peuvent, peut-être, renier Dieu, et qu'Il les détruit. Alors allons-nous nous en débarrasser.

\par 5 Alors Satan et ses armées jetèrent du feu sur le blé et le consumèrent.

\par 6 Mais à cause de la chaleur de la flamme, Adam et Ève se réveillèrent de leur sommeil et virent le blé brûler, et le seau d'eau près d'eux se déversait.

\par 7 Alors ils pleurèrent et retournèrent à la grotte.

\par 8 Mais alors qu'ils montaient du bas de la montagne où ils se trouvaient, Satan et ses armées les rencontrèrent sous la forme d'anges, louant Dieu.

\par 9 Alors Satan dit à Adam : « Ô Adam, pourquoi es-tu si tourmenté par la faim et la soif ? Il me semble que Satan a brûlé le blé. Et Adam lui dit : « Oui. »

\par 10 Encore une fois, Satan dit à Adam : « Reviens avec nous ; nous sommes des anges de Dieu. Dieu nous a envoyé vers toi pour te montrer un autre champ de blé, meilleur que celui-là ; et au-delà se trouve une fontaine de bonne eau et de nombreux arbres, où tu habiteras près d'elle et où tu travailleras le champ de maïs dans un but meilleur que celui que Satan a consommé.

\par 11 Adam pensait qu'il était vrai, et que c'étaient des anges qui lui parlaient ; et il revint avec eux.

\par 12. Alors Satan commença à égarer Adam et Ève pendant huit jours, jusqu'à ce qu'ils tombèrent tous deux comme morts, de faim, de soif et de faiblesse. Puis il s'enfuit avec ses hôtes et les quitta.

\chapitre{68}

\par \textit{Comme la destruction et les troubles sont pour Satan quand il est le maître. Adam et Ève établissent la coutume du culte.}

\par 1 ALORS Dieu regarda Adam et Ève, et ce qui leur était arrivé de la part de Satan, et comment il les avait fait périr.

\par 2 Dieu a donc envoyé Sa Parole et a ressuscité Adam et Ève de leur état de mort.

\par 3 Alors Adam, lorsqu'il fut ressuscité, dit : « Ô Dieu, tu nous as brûlé et tu nous as pris le maïs que tu nous avais donné, et tu as vidé le seau d'eau. Et tu as envoyé tes anges, qui nous ont fait sortir du champ de maïs. Veux-Tu nous faire périr ? Si cela vient de Toi, ô Dieu, alors enlève nos âmes ; mais ne nous punissez pas.

\par 4 Alors Dieu dit à Adam : « Je n'ai pas brûlé le blé, je n'ai pas versé l'eau du seau, et je n'ai pas envoyé mes anges pour t'égarer. »

\par 5 « Mais c'est Satan, ton maître qui l'a fait ; celui à qui tu t'es soumis; Mon commandement étant entre-temps mis de côté. C'est lui qui a brûlé le blé, qui a versé l'eau, et qui t'a égaré ; et toutes les promesses qu'il vous a faites ne sont en vérité que feinte, tromperie et mensonge.

\par 6 «Mais maintenant, ô Adam, tu reconnaîtras mes bonnes actions qui t'ont été faites.»

\par 7 Et Dieu dit à ses anges de prendre Adam et Ève, et de les porter jusqu'au champ de blé, qu'ils trouvèrent comme auparavant, avec le seau plein d'eau.

\par 8 Là, ils virent un arbre, et trouvèrent dessus de la manne solide ; et je me demandais la puissance de Dieu. Et les anges leur ordonnèrent de manger de la manne lorsqu'ils avaient faim.

\par 9 Et Dieu adjura Satan par une malédiction de ne plus revenir et de détruire le champ de blé.

\par 10 Alors Adam et Ève prirent du blé, en firent une offrande, le prirent et l'offrèrent sur la montagne, à l'endroit où ils avaient offert leur première offrande de sang.

\par 11 Et ils représentèrent cette offrande sur l'autel qu'ils avaient bâti d'abord. Et ils se levèrent et prièrent et supplièrent le Seigneur en disant : « Ainsi, ô Dieu, lorsque nous étions dans le jardin, nos louanges montaient vers Toi, comme cette offrande ; et notre innocence montait vers toi comme de l'encens. Mais maintenant, ô Dieu, accepte cette offrande de notre part, et ne nous renvoie pas en arrière, privés de ta miséricorde.

\par 12 Alors Dieu dit à Adam et Ève : « Puisque vous avez fait cette offrande et que vous Me l'avez offerte, j'en ferai ma chair, quand je descendrai sur terre pour vous sauver ; et je le ferai offrir continuellement sur un autel, pour pardon et pour miséricorde, à ceux qui y participent dûment.

\par 13 Et Dieu envoya un feu brillant sur l'offrande d'Adam et d'Ève, et la remplit d'éclat, de grâce et de lumière ; et le Saint-Esprit est descendu sur cette oblation.

\par 14 Alors Dieu ordonna à un ange de prendre des pinces à feu, comme une cuillère, et avec elle de prendre une offrande et de l'apporter à Adam et Ève. Et l'ange fit ainsi, comme Dieu le lui avait ordonné, et le leur offrit.

\par 15 Et les âmes d'Adam et d'Ève furent éclairées, et leurs cœurs furent remplis de joie et d'allégresse et des louanges de Dieu.

\par 16 Et Dieu dit à Adam : « Ce sera pour vous une habitude de le faire, lorsque l'affliction et le chagrin vous surprendront. Mais votre délivrance et votre entrée dans le jardin n'auront lieu que lorsque les jours seront accomplis, comme convenu entre vous et Moi ; s'il n'en était pas ainsi, par ma miséricorde et ma pitié pour vous, je vous ramènerais à mon jardin et à ma faveur à cause de l'offrande que vous venez de faire à mon nom.

\par 17 Adam se réjouit de ces paroles qu'il entendit de Dieu ; et lui et Ève adorèrent devant l'autel, devant lequel ils s'inclinèrent, puis retournèrent à la Grotte des Trésors.

\par 18 Et cela eut lieu à la fin du douzième jour après le quatre-vingtième jour, depuis le moment où Adam et Ève sortirent du jardin.

\par 19 Et ils restèrent debout toute la nuit, priant jusqu'au matin ; puis il sortit de la grotte.

\par 20 Alors Adam dit à Ève, avec joie de cœur, à cause de l'offrande qu'ils avaient faite à Dieu et qui avait été acceptée de Lui : « Faisons cela trois fois chaque semaine, le quatrième jour mercredi, le jour de préparation le vendredi et le dimanche sabbat, tous les jours de notre vie.

\par 21 Et comme ils s'accordaient entre eux sur ces paroles, Dieu fut content de leurs pensées, et de la résolution qu'ils avaient prise chacun avec l'autre.

\par 22 Après cela, la Parole de Dieu fut adressée à Adam et dit : « Ô Adam, tu as déterminé d'avance les jours où les souffrances m'atteindront, quand je serai fait chair ; car c'est le quatrième mercredi et le jour de préparation le vendredi.

\par 23 « Mais quant au premier jour, j'y ai créé toutes choses, et j'ai élevé les cieux. Et encore une fois, par Ma résurrection en ce jour, Je créerai de la joie et élèverai très haut ceux qui croient en Moi ; Ô Adam, offre cette oblation, tous les jours de ta vie.

\par 24 Alors Dieu retira Sa Parole d'Adam.

\par 25 Mais Adam continua à offrir ainsi cette oblation, chaque semaine trois fois, jusqu'à la fin des sept semaines. Et le premier jour, qui est le cinquantième, Adam fit une offrande comme il avait l'habitude de le faire, et lui et Ève la prirent et vinrent à l'autel devant Dieu, comme il le leur avait enseigné.



\chapitre{69}

\par \textit{Douzième apparition de Satan à Adam et Ève, pendant qu'Adam priait sur l'offrande sur l'autel ; quand Satan l'a frappé.}

\par 1 ALORS Satan, ennemi de tout bien, envieux d'Adam et de son offrande par laquelle il trouva grâce auprès de Dieu, se hâta et prit une pierre tranchante parmi les pierres de fer tranchantes ; apparut sous la forme d'un homme et alla se tenir près d'Adam et Ève.

\par 2 Adam offrait alors sur l'autel et avait commencé à prier, les mains étendues vers Dieu.

\par 3 Alors Satan se précipita avec la pierre de fer tranchante qu'il avait avec lui, et avec elle transperça Adam du côté droit, lorsque du sang et de l'eau coulèrent, alors Adam tomba sur l'autel comme un cadavre. Et Satan s'enfuit.

\par 4 Alors Eve vint, prit Adam et le plaça sous l'autel. Et elle resta là, pleurant sur lui ; tandis qu'un flot de sang coulait du côté d'Adam sur son offrande.

\par 5 Mais Dieu a regardé la mort d'Adam. Il envoya alors Sa Parole, le releva et lui dit : « Accomplis ton offrande, car en effet, Adam, elle vaut beaucoup, et elle n'a aucun défaut. »

\par 6 Dieu dit plus loin à Adam : « Ainsi m'arrivera-t-il aussi, sur la terre, lorsque je serai transpercé et que le sang coulera du sang et de l'eau de mon côté et coulera sur mon corps, qui est la véritable offrande ; et qui sera offert sur l’autel comme une offrande parfaite.

\par 7 Alors Dieu ordonna à Adam de terminer son offrande, et quand il l'eut terminée, il adora devant Dieu et le loua pour les signes qu'il lui avait montrés.

\par 8 Et Dieu guérit Adam en un jour, qui est la fin des sept semaines ; et c'est le cinquantième jour.

\par 9 Alors Adam et Ève revinrent de la montagne et entrèrent dans la Caverne des Trésors, comme ils avaient l'habitude de le faire. Cela s'est terminé pour Adam et Ève, cent quarante jours depuis leur sortie du jardin.

\par 10 Alors ils se levèrent tous les deux cette nuit-là et prièrent Dieu. Et quand le matin fut venu, ils sortirent et descendirent à l'ouest de la grotte, à l'endroit où se trouvait leur blé, et s'y reposèrent à l'ombre d'un arbre, comme ils avaient l'habitude.

\par 11 Mais quand une multitude de bêtes les entourèrent. C'était l'œuvre de Satan, dans sa méchanceté ; afin de faire la guerre à Adam par le mariage.

\chapitre{70}

\par \textit{Treizième apparition de Satan à Adam et Ève, pour lui faire la guerre, par son mariage avec Ève.}

\par 1 APRÈS cela, Satan, le ennemi de tout bien, prit la forme d'un ange, et avec lui deux autres, de sorte qu'ils ressemblaient aux trois anges qui avaient apporté à Adam de l'or, de l'encens et de la myrrhe.

\par 2 Ils passèrent devant Adam et Ève alors qu'ils étaient sous l'arbre, et saluèrent Adam et Ève avec de belles paroles pleines de ruse.

\par 3 Mais quand Adam et Ève virent leur belle apparence et entendirent leur doux discours, Adam se leva, les accueillit et les amena à Ève, et ils restèrent tous ensemble ; Pendant ce temps, le cœur d'Adam se réjouissait parce qu'il pensait à leur sujet que c'étaient les mêmes anges qui lui avaient apporté de l'or, de l'encens et de la myrrhe.

\par 4 Parce que, lorsqu'ils sont venus vers Adam pour la première fois, la paix et la joie sont venues d'eux, grâce à ce qu'ils lui ont apporté de bons signes ; alors Adam pensa qu'ils étaient venus une seconde fois pour lui donner d'autres signes pour qu'il se réjouisse également. Car il ne savait pas que c'était Satan ; c'est pourquoi il les reçut avec joie et les accompagna.

\par 5 Alors Satan, le plus grand d'entre eux, dit : « Réjouis-toi, ô Adam, et sois dans la joie. Voici, Dieu nous a envoyés vers toi pour te dire quelque chose.

\par 6 Et Adam dit : « Qu'est-ce qu'il y a ? Alors Satan répondit : « C’est une chose légère, mais c’est une parole de Dieu, veux-tu l’entendre de notre part et la mettre en pratique ? Mais si tu n’écoutes pas, nous retournerons à Dieu et lui dirons que tu ne recevras pas sa parole.

\par 7 Et Satan dit encore à Adam : « Ne crains rien, et qu'un tremblement ne te surprenne pas ; tu ne nous connais pas ?

\par 8 Mais Adam dit : « Je ne te connais pas. »

\par 9 Alors Satan lui dit : « Je suis l'ange qui t'a apporté de l'or et qui l'ai emporté dans la grotte ; cet autre est celui qui t'a apporté de l'encens ; et ce troisième, c'est celui qui t'a apporté de la myrrhe lorsque tu étais au sommet de la montagne, et qui t'a porté à la grotte.

\par 10 « Mais quant aux autres anges nos semblables, qui vous ont porté à la grotte, Dieu ne les a pas envoyés avec nous cette fois ; car Il nous a dit : « Vous suffisez. »

\par 11 Ainsi, quand Adam entendit ces paroles, il les crut et dit à ces anges : « Dis la parole de Dieu, afin que je la reçoive. »

\par 12 Et Satan lui dit : « Jure et promets-moi que tu le recevras. »

\par 13 Alors Adam dit : « Je ne sais pas comment jurer et promettre. »

\par 14 Et Satan lui dit : « Tends ta main et mets-la dans la mienne. »

\par 15 Alors Adam tendit la main et la mit entre les mains de Satan ; quand Satan lui dit : « Dis, maintenant, si vrai que Dieu est vivant, rationnel et parlant, qui a élevé les cieux dans l'espace, et a établi la terre sur les eaux, et m'a créé des quatre éléments, et de la poussière de la terre : je ne romprai pas ma promesse, ni ne renoncerai à ma parole.

\par 16 Et Adam jura ainsi.

\par 17 Alors Satan lui dit : Voici, il y a déjà quelque temps que tu es sorti du jardin, et tu ne connais ni la méchanceté ni le mal. Mais maintenant, Dieu te dit : prends Ève qui est sortie de ton côté, et épouse-la, afin qu'elle te donne des enfants, pour te consoler et pour chasser de toi le trouble et le chagrin ; Or, cette affaire n’est pas difficile et elle ne présente aucun scandale pour toi.

\chapitre{71}

\par \textit{Adam est troublé par son mariage avec Eve.}

\par 1 MAIS quand Adam entendit ces paroles de Satan, il fut très attristé, à cause de son serment et de sa promesse, et dit : « Dois-je commettre adultère avec ma chair et mes os, et pécherai-je contre moi-même, pour que Dieu puisse le faire ? me détruire et m'effacer de la surface de la terre ?

\par 2 « Depuis que, au début, j'ai mangé de l'arbre, Il m'a chassé du jardin dans ce pays étranger, et m'a privé de ma nature lumineuse, et m'a amené la mort. Si donc je fais cela, il retranchera ma vie de la terre, et il me jettera en enfer, où il me tourmentera longtemps.

\par 3 « Mais Dieu n'a jamais prononcé les paroles que tu m'as dites ; et vous n'êtes pas les anges de Dieu, ni encore envoyés de Lui. Mais vous êtes des démons, venez à moi sous la fausse apparence des anges. Loin de moi; vous, maudits de Dieu !

\par 4 Alors ces démons s'enfuirent devant Adam. Et lui et Ève se levèrent, retournèrent à la Caverne des Trésors et y entrèrent.

\par 5 Alors Adam dit à Ève : « Si tu as vu ce que j'ai fait, ne le dis pas ; car j’ai péché contre Dieu en jurant par son grand nom, et j’ai mis une autre fois ma main dans celle de Satan. Ève garda donc le silence, comme Adam le lui avait dit.

\par 6 Alors Adam se leva et étendit les mains vers Dieu, le suppliant et le suppliant avec des larmes de lui pardonner ce qu'il avait fait. Et Adam resta ainsi debout et priant quarante jours et quarante nuits. Il n'a ni mangé ni bu jusqu'à ce qu'il tombe sur terre de faim et de soif.

\par 7 Alors Dieu envoya Sa Parole à Adam, qui le releva de là où il gisait, et lui dit : « Ô Adam, pourquoi as-tu juré par mon nom, et pourquoi as-tu conclu un accord avec Satan une autre fois ?

\par 8 Mais Adam pleura et dit : « Ô Dieu, pardonne-moi, car j'ai fait cela involontairement ; croyant qu'ils étaient les anges de Dieu.

\par 9 Et Dieu pardonna à Adam, en lui disant : « Méfie-toi de Satan. »

\par 10 Et Il retira Sa Parole d'Adam.

\par 11 Alors le cœur d'Adam fut consolé ; et il prit Eve, et ils sortirent de la grotte pour préparer de la nourriture pour leurs corps.

\par 12 Mais à partir de ce jour, Adam se débattit mentalement au sujet de son mariage avec Eve ; il avait peur de le faire, de peur que Dieu ne soit en colère contre lui.

\par 13 Alors Adam et Ève allèrent au fleuve d'eau et s'assirent sur la rive, comme font les gens lorsqu'ils s'amusent.

\par 14 Mais Satan était jaloux d'eux ; et les détruirait.

\chapitre{72}

\par \textit{Le cœur d'Adam est en feu.}

\par 1 ALORS Satan et dix de ses armées se sont transformés en jeunes filles, comme aucune autre dans le monde entier, pour la grâce.

\par 2 Ils remontèrent du fleuve en présence d'Adam et d'Ève, et ils dirent entre eux : « Venez, nous allons regarder les visages d'Adam et d'Ève, qui sont des hommes sur la terre. Comme ils sont beaux et comme leur apparence est différente de celle de nos propres visages. Puis ils s'approchèrent d'Adam et d'Ève et les saluèrent ; et je restai à les contempler.

\par 3 Adam et Ève les regardèrent aussi, et s'étonnèrent de leur beauté, et dirent : « Y a-t-il donc, sous nous, un autre monde, avec de si belles créatures comme celles-ci en lui ? »

\par 4 Et ces jeunes filles dirent à Adam et Ève : « Oui, en effet, nous sommes une création abondante. »

\par 5 Alors Adam leur dit : « Mais comment multipliez-vous ?

\par 6 Et ils lui répondirent : « Nous avons des maris qui nous ont épousés, et nous leur donnons des enfants, qui grandissent, et qui à leur tour se marient et se marient, et qui enfantent aussi des enfants ; et ainsi nous augmentons. Et s’il en est ainsi, ô Adam, tu ne nous crois pas, nous te montrerons nos maris et nos enfants.

\par 7 Alors elles crièrent par-dessus le fleuve, comme pour appeler leurs maris et leurs enfants, qui montaient du fleuve, hommes et enfants ; et chacun venait vers sa femme, ses enfants étant avec lui.

\par 8 Mais quand Adam et Ève les virent, ils restèrent muets et s'étonnèrent de leur sort.

\par 9 Alors ils dirent à Adam et Ève : « Vous voyez nos maris et nos enfants, épousez Ève comme nous épousons nos femmes, et vous aurez des enfants comme nous. » C’était un stratagème de Satan pour tromper Adam.

\par 10 Satan pensait aussi en lui-même : « Dieu a d'abord commandé à Adam concernant le fruit de l'arbre, lui disant : « N'en mange pas ; sinon tu mourras de mort. Mais Adam en a mangé, et pourtant Dieu ne l'a pas tué ; Il lui a seulement décrété la mort, les plaies et les épreuves, jusqu'au jour où il sortira de son corps.

\par 11 « Maintenant, si je le trompe pour qu'il fasse cette chose et qu'il épouse Ève sans le commandement de Dieu, Dieu le tuera alors. »

\par 12 C'est pourquoi Satan a opéré cette apparition avant Adam et Ève ; parce qu'il cherchait à le tuer et à le faire disparaître de la surface de la terre.

\par 13 Pendant ce temps, le feu du péché tomba sur Adam, et il songea à commettre un péché. Mais il se retint, craignant que s'il suivait ce conseil de Satan, Dieu ne le mette à mort.

\par 14 Alors Adam et Ève se levèrent et prièrent Dieu, tandis que Satan et ses armées descendaient dans le fleuve, en présence d'Adam et Ève ; pour leur faire voir qu'ils retournaient dans leur propre région.

\par 15 Alors Adam et Ève retournèrent à la Caverne des Trésors, comme ils avaient l'habitude de le faire ; vers l'heure du soir.

\par 16 Et ils se levèrent tous deux et prièrent Dieu cette nuit-là. Adam resta debout en prière, mais ne sachant pas comment prier, à cause des pensées de son cœur concernant Eve, son mariage ; et il continua ainsi jusqu'au matin.

\par 17 Et quand la lumière se leva, Adam dit à Ève : « Lève-toi, allons en bas de la montagne, où ils nous ont apporté de l'or, et interrogeons le Seigneur à ce sujet. »

\par 18 Alors Ève dit : « Qu'est-ce que ça te fait, ô Adam ?

\par 19 Et il lui répondit : « Afin que je demande au Seigneur de m'informer de ton mariage ; car je ne le ferai pas sans son ordre, de peur qu'il ne nous fasse périr, toi et moi. Car ces démons ont enflammé mon cœur, en pensant à ce qu’ils nous ont montré, dans leurs apparitions pécheresses.

\par 20 Alors Ève dit à Adam : « Pourquoi avons-nous besoin d'aller en dessous de la montagne ? Levons-nous plutôt et prions Dieu dans notre grotte pour nous faire savoir si ce conseil est bon ou non.

\par 21 Alors Adam se leva en prière et dit : « Ô Dieu, tu sais que nous avons transgressé contre Toi, et à partir du moment où nous avons transgressé, nous avons été privés de notre nature lumineuse ; et notre corps est devenu brutal, exigeant de la nourriture et de la boisson ; et avec des désirs animaux.

\par 22 « Ordonne-nous, ô Dieu, de ne pas leur céder le passage sans ton ordre, de peur que tu ne nous ramènes à rien. Car si tu ne nous donnes pas cet ordre, nous serons maîtrisés et suivrons les conseils de Satan ; et tu nous feras encore périr.

\par 23 « Sinon, alors ôtez-nous nos âmes ; débarrassons-nous de cette convoitise animale. Et si tu ne nous donnes aucun ordre à ce sujet, alors sépare-toi d'Ève de moi, et de moi d'elle ; et placez-nous loin les uns des autres.

\par 24 « Encore une fois, ô Dieu, quand tu nous auras séparés les uns des autres, les démons nous tromperont par leurs apparitions, détruiront nos cœurs et souilleront nos pensées les uns envers les autres. Mais si ce n’est pas chacun de nous envers l’autre, ce sera en tout cas par leur apparition lorsqu’ils se montreront à nous. » Ici, Adam a terminé sa prière.

\chapitre{73}

\par \textit{Les fiançailles d'Adam et Ève.}

\par 1 ALORS Dieu considéra les paroles d'Adam comme étant vraies, et qu'il pouvait attendre longtemps son ordre, respectant le conseil de Satan.

\par 2 Et Dieu approuva Adam dans ce qu'il avait pensé à ce sujet et dans la prière qu'il avait offerte en sa présence ; et la Parole de Dieu vint à Adam et lui dit : « Ô Adam, si seulement tu avais eu cette prudence au début, avant de quitter sérieusement le jardin pour aller dans ce pays !

\par 3 Après cela, Dieu envoya son ange qui avait apporté de l'or, et l'ange qui avait apporté de l'encens, et l'ange qui avait apporté de la myrrhe à Adam, pour qu'ils l'instruisent de ses noces d'Ève.

\par 4 Alors ces anges dirent à Adam : « Prends l'or et donne-le à Ève comme cadeau de noces, et fiance-la ; puis offrez-lui de l'encens et de la myrrhe en cadeau ; et soyez, toi et elle, une seule chair.

\par 5 Adam écouta les anges, prit l'or et le mit dans le sein d'Ève dans son vêtement ; et il l'emballa avec sa main.

\par 6 Alors les anges ordonnèrent à Adam et Ève de se lever et de prier quarante jours et quarante nuits ; et après cela, qu'Adam revienne vers sa femme ; car alors ce serait un acte pur et sans souillure ; et il devrait avoir des enfants qui se multiplieraient et reconstitueraient la surface de la terre.

\par 7 Alors Adam et Ève reçurent tous deux les paroles des anges ; et les anges les quittèrent.

\par 8 Alors Adam et Ève commencèrent à jeûner et à prier, jusqu'à la fin des quarante jours ; puis ils se rassemblèrent, comme les anges le leur avaient dit. Et depuis le moment où Adam quitta le jardin jusqu'à ce qu'il épousa Ève, il y eut deux cent vingt-trois jours, soit sept mois et treize jours.

\par 9 Ainsi la guerre de Satan contre Adam fut vaincue.



\chapitre{74}

\par \textit{La naissance de Caïn et Luluwa. Pourquoi ils ont reçu ces noms.}

\par 1 ET ils habitaient la terre en travaillant, afin de continuer dans le bien-être de leurs corps ; et ils le furent jusqu'à ce que les neuf mois de la grossesse d'Ève fussent terminés et que le moment où elle devait accoucher approchait.

\par 2 Alors elle dit à Adam : « Cette grotte est un endroit pur à cause des signes qui s'y sont produits depuis que nous avons quitté le jardin ; et nous y prierons à nouveau. Il n’est donc pas convenable que j’y enfante ; réparons-nous plutôt à celui du rocher abritant, que Satan nous a lancé, lorsqu'il a voulu nous tuer avec ; mais cela a été tendu et étendu comme un auvent sur nous par le commandement de Dieu ; et a formé une grotte.

\par 3 Alors Adam emmena Eve dans cette grotte ; et quand vint le moment où elle devait enfanter, elle travailla beaucoup. Adam était également désolé, et son cœur souffrait à cause d'elle ; car elle était proche de la mort ; afin que s'accomplisse la parole de Dieu qui lui a été adressée : « Dans la souffrance tu enfanteras un enfant, et dans la tristesse tu enfanteras ton enfant. »

\par 4 Mais quand Adam vit l'étroitesse dans laquelle se trouvait Ève, il se leva et pria Dieu, et dit : « Ô Seigneur, regarde-moi avec l'œil de ta miséricorde, et sors-la de sa détresse. »

\par 5 Et Dieu regarda sa servante Ève, et la délivra, et elle enfanta son fils premier-né, et avec lui une fille.

\par 6 Alors Adam se réjouit de la délivrance d'Ève, et aussi des enfants qu'elle lui avait donnés. Et Adam servit Ève dans la grotte jusqu'à la fin de huit jours ; quand ils nommèrent le fils Caïn et la fille Luluwa.

\par 7 Le sens de Caïn est « haïr », parce qu'il haïssait sa sœur dans le sein de leur mère ; avant qu’ils n’en sortent. C’est pourquoi Adam l’a nommé Caïn.

\par 8 Mais Luluwa signifie « belle », parce qu'elle était plus belle que sa mère.

\par 9 Alors Adam et Ève attendirent que Caïn et sa sœur aient quarante jours, lorsqu'Adam dit à Ève : « Nous ferons une offrande et l'offrirons en faveur des enfants. »

\par 10 Et Ève dit : « Nous ferons une offrande pour le fils premier-né ; et ensuite nous en ferons un pour la fille.

\chapitre{75}

\par \textit{La famille revisite la Grotte aux Trésors. Naissance d'Abel et d'Aklemia.}

\par 1 ALORS Adam prépara une offrande, et lui et Ève l'offrèrent pour leurs enfants, et l'apportèrent à l'autel qu'ils avaient d'abord construit.

\par 2 Et Adam offrit l'offrande et supplia Dieu d'accepter son offrande.

\par 3 Alors Dieu accepta l'offrande d'Adam et envoya une lumière du ciel qui brillait sur l'offrande. Et Adam et le fils s'approchèrent de l'offrande, mais Ève et la fille ne s'en approchèrent pas.

\par 4 Alors Adam descendit de l'autel, et ils furent joyeux ; et Adam et Ève attendirent que leur fille ait quatre-vingts jours ; alors Adam prépara une offrande et la porta à Ève et aux enfants ; et ils se rendirent à l'autel, où Adam l'offrit, comme il avait l'habitude, demandant au Seigneur d'accepter son offrande.

\par 5 Et le Seigneur accepta l'offrande d'Adam et Ève. Alors Adam, Ève et les enfants s'approchèrent ensemble et descendirent de la montagne en se réjouissant.

\par 6 Mais ils ne retournèrent pas à la grotte dans laquelle ils étaient nés ; mais il vint à la Grotte des Trésors, afin que les enfants en fassent le tour et soient bénis avec les signes apportés du jardin.

\par 7 Mais après avoir reçu ces signes, ils retournèrent à la grotte dans laquelle ils étaient nés.

\par 8 Cependant, avant qu'Ève ait offert l'offrande, Adam l'avait prise et était allé avec elle jusqu'au fleuve d'eau, dans lequel ils se jetèrent d'abord ; et là ils se lavèrent. Adam a lavé son corps et Ève aussi le sien, après les souffrances et les angoisses qui les avaient frappés.

\par 9 Mais Adam et Ève, après s'être lavés dans le fleuve d'eau, retournaient chaque nuit à la Caverne des Trésors, où ils priaient et étaient bénis ; puis je suis retourné à leur grotte où sont nés les enfants

\par 10 Adam et Ève firent de même jusqu'à ce que les enfants aient fini de téter. Puis, une fois sevrés, Adam fit une offrande pour les âmes de ses enfants ; à part les trois fois où il faisait une offrande pour eux, chaque semaine.

\par 11 Lorsque les jours d'allaitement des enfants furent terminés, Ève conçut de nouveau, et lorsque ses jours furent accomplis, elle enfanta un autre fils et une autre fille ; et ils nommèrent le fils Abel et la fille Aklia.

\par 12 Puis au bout de quarante jours, Adam fit une offrande pour le fils, et au bout de quatre-vingts jours il fit une autre offrande pour la fille, et fit par eux, comme il avait fait auparavant par Caïn et sa sœur Luluwa. .

\par 13 Il les conduisit à la Grotte des Trésors, où ils reçurent une bénédiction, puis retourna à la grotte où ils étaient nés. Après leur naissance, Ève cessa de procréer.

\chapitre{76}

\par \textit{Caïn devient jaloux à cause de ses sœurs.}

\par 1 ET les enfants commencèrent à devenir plus forts et à grandir en stature ; mais Caïn avait le cœur dur et régnait sur son jeune frère.

\par 2 Et souvent, lorsque son père faisait une offrande, il restait en arrière et n'allait pas avec eux pour offrir.

\par 3 Mais quant à Abel, il avait un cœur doux et obéissait à son père et à sa mère, qu'il incitait souvent à faire une offrande, parce qu'il l'aimait ; et j'ai beaucoup prié et jeûné.

\par 4 Alors ce signe fut adressé à Abel. Alors qu'il entrait dans la Caverne des Trésors et qu'il vit les bâtons d'or, l'encens et la myrrhe, il s'enquit auprès de ses parents Adam et Ève à leur sujet et leur dit : « Comment êtes-vous arrivés à cela ?

\par 5 Alors Adam lui raconta tout ce qui leur était arrivé. Et Abel ressentait profondément ce que son père lui avait dit.

\par 6 De plus, son père Adam lui parla des œuvres de Dieu et du jardin ; et après cela, il resta derrière son père toute la nuit dans la Grotte des Trésors.

\par 7 Et cette nuit-là, pendant qu'il priait, Satan lui apparut sous la forme d'un homme, qui lui dit : « Tu as souvent poussé ton père à faire une offrande, à jeûner et à prier, c'est pourquoi je tuerai. toi, et te faire périr de ce monde.

\par 8 Mais quant à Abel, il pria Dieu et chassa de lui Satan ; et je n'ai pas cru aux paroles du diable. Alors, quand le jour fut venu, un ange de Dieu lui apparut et lui dit : « Ne raccourcis pas le jeûne, la prière, et n'offre pas d'offrande à ton Dieu. Car voici, le Seigneur a accepté votre prière. N'aie pas peur de la figure qui t'est apparue pendant la nuit et qui t'a maudit jusqu'à la mort. Et l'ange le quitta.

\par 9 Alors, quand le jour fut venu, Abel vint vers Adam et Ève et leur raconta la vision qu'il avait eue. Mais quand ils l'eurent entendu, ils en furent très affligés, mais ne lui en parlèrent pas ; ils l'ont seulement réconforté.

\par 10 Mais quant à Caïn au cœur dur, Satan vint vers lui pendant la nuit, se montra et lui dit : « Puisque Adam et Ève aiment ton frère Abel bien plus qu'ils ne t'aiment, et souhaitent le marier à ton frère Abel. belle sœur, parce qu'ils l'aiment ; mais je veux te marier avec sa sœur malheureuse, parce qu'ils te haïssent ;

\par 11 « Maintenant donc, je te conseille, quand ils feront cela, de tuer ton frère ; alors ta sœur te sera laissée ; et sa sœur sera rejetée.

\par 12 Et Satan le quitta. Mais le méchant restait dans le cœur de Caïn, qui cherchait à plusieurs reprises à tuer son frère.



\chapitre{77}

\par \textit{Caïn, 15 ans, et Abel 12 ans, se séparent.}

\par 1 MAIS quand Adam vit que le frère aîné haïssait le plus jeune, il s'efforça d'adoucir leurs cœurs et dit à Caïn : « Prends, ô mon fils, des fruits de tes semailles, et fais une offrande à Dieu, afin qu'il puisse te pardonner ta méchanceté et ton péché.

\par 2 Il dit aussi à Abel : « Prends de tes semailles, fais-en une offrande et apporte-la à Dieu, afin qu'il pardonne ta méchanceté et ton péché. »

\par 3 Alors Abel écouta la voix de son père, prit de ses semailles, fit une bonne offrande et dit à son père : Adam : Viens avec moi, montre-moi comment l'offrir.

\par 4 Et ils allèrent, Adam et Ève avec lui, et lui montrèrent comment offrir son offrande sur l'autel. Puis après cela, ils se sont levés et ont prié pour que Dieu accepte l'offrande d'Abel.

\par 5 Alors Dieu regarda Abel et accepta son offrande. Et Dieu était plus satisfait d'Abel que de son offrande, à cause de son cœur bon et de son corps pur. Il n’y avait aucune trace de ruse chez lui.

\par 6 Puis ils descendirent de l'autel et se rendirent à la grotte dans laquelle ils habitaient. Mais Abel, en raison de sa joie d'avoir fait son offrande, la répétait trois fois par semaine, à l'exemple de son père Adam.

\par 7 Mais quant à Caïn, il ne prenait aucun plaisir à offrir ; mais après beaucoup de colère de la part de son père, il offrit une fois son cadeau ; et quand il offrait, son oeil était sur l'offrande qu'il faisait, et il prenait la plus petite de ses brebis en offrande, et son oeil était de nouveau sur elle. '

\par 8 C'est pourquoi Dieu n'accepta pas son offrande, parce que son cœur était plein de pensées meurtrières.

\par 9 Et ils vécurent tous ensemble dans la grotte dans laquelle Eve avait enfanté, jusqu'à ce que Caïn eut quinze ans, et Abel douze ans.

\chapitre{78}

\par \textit{La jalousie triomphe de Caïn. Il crée des problèmes dans la famille. Comment le premier meurtre a été planifié.}

\par 1 ALORS Adam dit à Ève : « Voici, les enfants ont grandi ; il faut penser à leur trouver des épouses.

\par 2 Alors Ève répondit : « Comment pouvons-nous le faire ?

\par 3 Alors Adam lui dit : « Nous épouserons la sœur d'Abel avec Caïn, et la sœur de Caïn avec Abel. »

\par 4 Alors Ève dit à Adam : « Je n'aime pas Caïn parce qu'il a le cœur dur ; mais laissez-les attendre jusqu'à ce que nous offrions au Seigneur en leur faveur.

\par 5 Et Adam ne dit rien de plus.

\par 6 Pendant ce temps, Satan s'approcha de Caïn sous la forme d'un homme des champs et lui dit : « Voici, Adam et Ève ont délibéré ensemble au sujet de votre mariage tous les deux ; et ils sont convenus de te marier avec la sœur d'Abel, et avec lui ta sœur.

\par 7 « Mais si je ne t'aimais pas, je ne t'aurais pas dit cette chose. Pourtant, si tu veux suivre mon conseil et m'écouter, je t'apporterai le jour de ton mariage de belles robes, de l'or et de l'argent en abondance, et mes parents t'accompagneront.

\par 8 Alors Caïn dit avec joie : « Où sont tes parents ?

\par 9 Et Satan répondit : « Mes parents sont dans un jardin au nord, où j'avais autrefois l'intention d'amener ton père Adam ; mais il n'a pas accepté mon offre.

\par 10 « Mais toi, si tu veux recevoir mes paroles et si tu viens à moi après tes noces, tu te reposeras de la misère dans laquelle tu es ; et tu te reposeras et tu seras mieux loti que ton père Adam.

\par 11 A ces paroles de Satan, Caïn ouvrit les oreilles et se pencha vers sa parole.

\par 12 Et il ne resta pas dans les champs, mais il alla vers Ève, sa mère, et la frappa, et la maudit, et lui dit : Pourquoi prends-tu ma sœur pour la marier à mon frère ? Est-ce que je suis mort

\par 13 Mais sa mère le calma et l'envoya dans le champ où il avait été.

\par 14 Puis, quand Adam vint, elle lui raconta ce que Caïn avait fait.

\par 15 Mais Adam fut affligé et garda le silence, et ne dit pas un mot.

\par 16 Alors le lendemain, Adam dit à Caïn, son fils : Prends tes brebis, jeunes et bonnes, et offre-les à ton Dieu ; et je parlerai à ton frère, pour faire à son Dieu une offrande de blé.

\par 17 Ils écoutèrent tous deux leur père Adam, et ils prirent leurs offrandes et les offrèrent sur la montagne près de l'autel.

\par 18 Mais Caïn se comporta avec hauteur envers son frère, et le chassa de l'autel, et ne voulut pas le laisser offrir son offrande sur l'autel ; mais il offrit les siens là-dessus, avec un cœur fier, plein de ruse et de fraude.

\par 19 Mais quant à Abel, il dressa des pierres à proximité, et sur celles-ci, il offrit son présent avec un cœur humble et sans fraude.

\par 20 Caïn se tenait alors près de l'autel sur lequel il avait offert son offrande ; et il cria à Dieu d'accepter son offrande ; mais Dieu ne l'a pas accepté de sa part ; aucun feu divin n’est non plus descendu pour consumer son offrande.

\par 21 Mais il restait debout devant l'autel, d'humeur et de colère, regardant vers son frère Abel, pour voir si Dieu accepterait ou non son offrande.

\par 22 Et Abel pria Dieu d'accepter son offrande. Puis un feu divin descendit et consuma son offrande. Et Dieu sentit la douce odeur de son offrande ; parce qu'Abel l'aimait et se réjouissait en lui.

\par 23 Et comme Dieu était très content de lui, il lui envoya un ange de lumière sous la forme d'un homme qui avait participé à son offrande, parce qu'il avait senti la douce odeur de son offrande, et ils réconfortèrent Abel et fortifièrent son cœur.

\par 24 Mais Caïn regardait tout ce qui se passait lors de l'offrande de son frère, et il en fut irrité.

\par 25 Alors il ouvrit la bouche et blasphéma Dieu, parce qu'il n'avait pas accepté son offrande.

\par 26 Mais Dieu dit à Caïn : « Pourquoi ton visage est-il triste ? Sois juste, afin que j'accepte ton offrande. Pas contre» C'est contre moi que tu as murmuré, mais contre toi-même.

\par 27 Et Dieu dit cela à Caïn en guise de réprimande, et parce qu'il avait en horreur lui et son offrande.

\par 28 Et Caïn descendit de l'autel, sa couleur changée et son visage triste, et vint vers son père et sa mère et leur raconta tout ce qui lui était arrivé. Et Adam était très affligé parce que Dieu n'avait pas accepté l'offrande de Caïn.

\par 29 Mais Abel descendit tout joyeux et le cœur joyeux, et il raconta à son père et à sa mère comment Dieu avait accepté son offrande. Et ils s'en réjouirent et l'embrassèrent au visage.

\par 30 Et Abel dit à son père : Parce que Caïn m'a chassé de l'autel et ne m'a pas permis d'offrir mon offrande dessus, je me suis fait un autel et j'ai offert mon offrande dessus.

\par 31 Mais quand Adam entendit cela, il fut très désolé, car c'était l'autel qu'il avait bâti d'abord, et sur lequel il avait offert ses propres dons.

\par 32 Quant à Caïn, il était si maussade et si irrité qu'il alla dans les champs, où Satan vint vers lui et lui dit : Puisque ton frère Abel s'est réfugié auprès de ton père Adam, parce que tu l'as chassé de sur l'autel, ils lui ont baisé le visage, et ils se réjouissent de lui bien plus que de toi.

\par 33 Quand Caïn entendit ces paroles de Satan, il fut rempli de rage ; et il ne l'a laissé savoir à personne. Mais il attendait pour tuer son frère, jusqu'à ce qu'il l'amène dans la grotte et lui dit alors :

\par 34 « Ô frère, le pays est si beau, et il y a des arbres si beaux et si agréables, et charmants à regarder ! Mais frère, tu n'es jamais allé un jour aux champs pour y prendre ton plaisir.

\par 35 « Aujourd'hui, ô mon frère, je souhaite vraiment que tu viennes avec moi dans les champs, pour te divertir et bénir nos champs et nos troupeaux, car tu es juste et je t'aime beaucoup, ô mon frère! mais tu t'es éloigné de moi.

\par 36 Alors Abel consentit à aller aux champs avec son frère Caïn.

\par 37 Mais avant de sortir, Caïn dit à Abel : Attends-moi jusqu'à ce que je cherche un bâton, à cause des bêtes sauvages.

\par 38 Alors Abel attendait dans son innocence. Mais Caïn, l'avant, alla chercher un bâton et sortit.

\par 39 Et ils commencèrent, Caïn et son frère Abel, à marcher dans le chemin ; Caïn lui parle, et le réconforte, pour lui faire tout oublier.

\chapitre{79}

\par \textit{Un plan diabolique est mené à une conclusion tragique. Caïn a peur. « Suis-je le gardien de mon frère ? Les sept punitions. La paix est brisée.}

\par 1 ET ainsi ils continuèrent leur route, jusqu'à ce qu'ils arrivèrent à un endroit isolé, où il n'y avait pas de brebis ; alors Abel dit à Caïn : « Voici, mon frère, nous sommes fatigués de marcher ; car nous ne voyons ni les arbres, ni les fruits, ni la verdure, ni les moutons, ni aucune des choses dont tu m'as parlé. Où sont tes brebis que tu m’as dit de bénir ?

\par 2 Alors Caïn lui dit : « Viens, et bientôt tu verras beaucoup de belles choses. mais va devant moi, jusqu'à ce que j'arrive à toi.

\par 3 Alors Abel s'avança, mais Caïn resta derrière lui.

\par 4 Et Abel marchait dans son innocence, sans fraude ; il ne croyait pas que son frère allait le tuer.

\par 5 Alors Caïn, lorsqu'il s'approcha de lui, le réconforta par ses paroles, marchant un peu derrière lui ; puis il se hâta et le frappa avec son bâton, coup sur coup, jusqu'à ce qu'il soit étourdi.

\par 6 Mais quand Abel tomba à terre, voyant que son frère voulait le tuer, il dit à Caïn : « Ô mon frère, aie pitié de moi. Par les seins que nous avons sucés, ne me frappe pas ! Par le sein qui nous a enfantés et qui nous a mis au monde, ne me frappe pas à mort avec ce bâton ! Si tu veux me tuer, prends une de ces grosses pierres et tue-moi sur-le-champ.

\par 7 Alors Caïn, le meurtrier au cœur dur et cruel, prit une grosse pierre et en frappa son frère sur la tête, jusqu'à ce que sa cervelle suinte et qu'il se fonde dans son sang devant lui.

\par 8 Et Caïn ne se repentit pas de ce qu'il avait fait.

\par 9 Mais la terre, lorsque le sang du juste Abel tomba sur elle, trembla, en buvant son sang, et aurait voulu anéantir Caïn à cause de cela.

\par 10 Et le sang d'Abel cria mystérieusement à Dieu, pour le venger de son meurtrier.

\par 11 Alors Caïn commença aussitôt à creuser la terre pour y coucher son frère ; car il tremblait de la peur qui l'avait saisi, lorsqu'il vit la terre trembler à cause de lui.

\par 12 Il jeta alors son frère dans la fosse qu'il avait faite et le couvrit de poussière. Mais la terre ne voulait pas le recevoir ; mais cela l'a vomi immédiatement.

\par 13 Caïn creusa encore la terre et y cacha son frère ; mais encore une fois la terre le rejeta sur elle-même ; jusqu'à ce que trois fois la terre rejette ainsi sur elle-même le corps d'Abel.

\par 14 La terre boueuse l'a vomi la première fois, parce qu'il n'était pas la première création ; et il le vomit une seconde fois et ne voulut pas le recevoir, parce qu'il était juste et bon, et qu'il fut tué sans raison ; et la terre le rejeta une troisième fois et ne voulut pas le recevoir, afin qu'il reste devant son frère un témoin contre lui.

\par 15 Et ainsi la terre se moqua de Caïn, jusqu'à ce que la Parole de Dieu lui parvienne concernant son frère.

\par 16 Alors Dieu fut irrité et très mécontent de la mort d'Abel ; et il tonna du ciel, et des éclairs passèrent devant lui, et la parole du Seigneur Dieu vint du ciel à Caïn, et lui dit : « Où est Abel, ton frère ?

\par 17 Alors Caïn répondit d'un cœur fier et d'une voix bourrue : « Comment, ô Dieu ? suis-je le gardien de mon frère ?

\par 18 Alors Dieu dit à Caïn : « Maudite soit la terre qui a bu le sang d'Abel, ton frère ; et toi, tremble et tremble ; et ceci sera pour toi un signe que quiconque te trouvera te tuera.

\par 19 Mais Caïn pleura parce que Dieu lui avait dit ces paroles ; et Caïn lui dit : « Ô Dieu, quiconque me trouvera me tuera, et je serai effacé de la surface de la terre. »

\par 20 Alors Dieu dit à Caïn : « Quiconque te trouvera ne te tuera pas ; » parce qu’avant cela, Dieu avait dit à Caïn : « Je renoncerai à sept châtiments pour celui qui tue Caïn. » Car quant à la parole de Dieu à Caïn : « Où est ton frère ? Dieu l'a dit par miséricorde pour lui, pour essayer de le faire se repentir.

\par 21 Car si Caïn s'était repenti à ce moment-là et avait dit : « Ô Dieu, pardonne-moi mon péché et le meurtre de mon frère », Dieu lui aurait alors pardonné son péché.

\par 22 Et quant à Dieu disant à Caïn : « Maudit soit le sol qui a bu le sang de ton frère », c'était aussi la miséricorde de Dieu envers Caïn. Car Dieu ne l'a pas maudit, mais il a maudit le sol ; bien que ce ne soit pas le terrain qui ait tué Abel et commis l'iniquité.

\par 23 Car il convenait que la malédiction tombât sur le meurtrier ; pourtant, avec miséricorde, Dieu a géré ses pensées de manière à ce que personne ne le sache et ne se détourne de Caïn.

\par 24 Et il lui dit : « Où est ton frère ? Ce à quoi il répondit : « Je ne sais pas. » Alors le Créateur lui dit : « Tremble et tremble. »

\par 25 Alors Caïn trembla et fut terrifié ; et par ce signe Dieu fit de lui un exemple devant toute la création, comme le meurtrier de son frère. Dieu a également fait venir sur lui le tremblement et la terreur, afin qu'il puisse voir la paix dans laquelle il se trouvait au début, et voir aussi le tremblement et la terreur qu'il a endurés à la fin ; afin qu'il puisse s'humilier devant Dieu, se repentir de son péché et rechercher la paix dont il jouissait au début.

\par 26 Et dans la parole de Dieu qui dit : « Je renoncerai à sept châtiments contre quiconque tuera Caïn », Dieu ne cherchait pas à tuer Caïn avec l'épée, mais Il cherchait à le faire mourir en jeûnant, en priant et en pleurant par l'épée. règle dure, jusqu'au moment où il fut délivré de son péché.

\par 27 Et les sept châtiments sont les sept générations pendant lesquelles Dieu attendait Caïn pour le meurtre de son frère.

\par 28 Mais quant à Caïn, depuis qu'il avait tué son frère, il ne pouvait trouver de repos nulle part ; mais il revint vers Adam et Ève, tremblants, terrifiés et souillés de sang. . . .


\end{document}