\begin{document}

\title{Questions d'Esdras}

\chapter{1}

\par \textit{Recension A}

\par \textit{Quel est le sort des justes et des pécheurs ?}

\par 1 Esdras le prophète vit l'ange de Dieu et lui posa une question après l'autre.

\par 2 Et l'ange s'approcha de lui et lui dit ce qui se passerait à la consommation. Le prophète interrogea l’ange et dit : « Qu’est-ce que Dieu a préparé pour les justes et les pécheurs ? Et au moment où arrivera le jour de la fin, que deviendra

\par 3 d'entre eux ? Où vont-ils, pour honorer ou pour torturer ? L'ange répondit et dit au prophète : « Une grande joie et une lumière éternelle ont été préparées pour les justes et pour les pécheurs, les ténèbres extérieures et l'éternité ont été préparées.

\par 4 feu. Le prophète dit à l'ange : « Seigneur, qui des vivants n'a pas péché

\par 5 contre Dieu ? Et si tel est le cas, alors bénis soient les bêtes et les oiseaux qui le font

\par 6 n'attendons pas la résurrection et n'attendons pas la fin. Si tu couronnes les justes qui ont enduré toutes les tortures, et les prophètes et les martyrs, lorsqu'ils prenaient des pierres et avec un marteau, se frappaient le visage jusqu'à ce que leur

\par 7 entrailles ont été vues, elles ont été torturées à cause de toi. Aie pitié de nous, pécheurs, qui avons été occupés et saisis par Satan.

\par \textit{Le prophète réprimandé}

\par 8 L'ange répondit et dit : « S'il y a quelqu'un au-dessus de toi, ne parle pas avec

\par 9 plus lui. sinon un grand malheur vous arrivera. Le prophète dit à l'ange :

\par 10 « Seigneur, je voudrais te parler encore un peu, réponds-moi ! Lorsque le jour de la fin arrivera et qu'il prendra l'âme, l'attribuera-t-il à la place du châtiment ou à la place d'honneur jusqu'à la Parousie ? [...] »

\par \textit{Le jour de la fin}

\par 11 L'ange répondit et dit : « N'attendez pas le jour de la fin, mais comme un

\par 12 aigle volant, hâtez-vous de faire de bonnes actions et de miséricorde. Car ce jour est effrayant, urgent,

\par 13 et exigeant. Il ne permet pas de s'occuper des enfants ou des biens. Il vient et arrive d'un coup comme quelqu'un d'impitoyable et d'impartial, il prend un captif à l'improviste, sûrement. Qu’il pleure ou qu’il soit maman, il n’aura aucune pitié.

\par \textit{Les bons et les mauvais anges}

\par 14 « Mais quand arrive le jour de la fin, un bon ange vient à l'âme bonne et un mauvais ange à l'âme mauvaise. Tout comme quelqu'un envoyé par les rois vers les malfaiteurs

\par 15 et les bonnes actions récompensent le bien au bien et le mal au mal, de même qu'un bon ange vient à l'âme bonne et un mauvais à la mauvaise. Pas

\par 16 que l'ange est méchant, mais que les actions (de chacun) (sont mauvaises).i Il prend l'âme, l'amène à l'est ; ils traversent le gel, la neige, les ténèbres, la grêle, la glace, la tempête, les armées de Satan, les ruisseaux, les vents de pluies terribles, les sentiers terribles et stupéfiants, les passages étroits.

\par 17 des défilés et des hautes montagnes. Ô chemin merveilleux, car un pied est derrière le

\par 18 Autre et devant lui se trouvent des fleuves de feu ! Le prophète fut étonné et dit : « Ô, cette voie merveilleuse et terrible ! »

\par \textit{Les sept étapes vers la Divinité}

\par 19 L'ange dit : « De ce côté-là, il y a sept camps et sept degrés jusqu'au

\par 20 Divinité, si je peux faire en sorte que (quelqu'un) la transmette. Parce que les premiers logements sont mauvais et merveilleux ; le second redoutable et indescriptible ; le troisième enfer et le froid glacial ; le quatrième se dispute et se dispute ; dans le cinquième, donc, l'investigation : s'il est juste, il brille, et s'il est pécheur, il est obscurci ; dans le sixième, donc, l'âme du

\par 21 le juste brille comme le soleil ; dans le septième donc, l'ayant amené, je le fais approcher du grand trône de la Divinité, en face du jardin, face à la gloire de Dieu où est la lumière sublime.

\par \textit{Dieu ne peut pas être vu}

\par 22 Le prophète dit à l'ange : « Mon seigneur, quand tu le fais traverser de telles terreurs, des querelles, des guerres, des chaleurs brûlantes, pourquoi ne le fais-tu pas rencontrer la Divinité, plutôt que de le faire rencontrer la Divinité ? approcher uniquement le

\par 23 trône ? L'ange dit au prophète : « Tu es un des hommes insensés et tu

\par 24 pensez selon la nature humaine. Je suis un ange et je sers perpétuellement Dieu, et je n'ai pas vu le visage de Dieu. Comment dites-vous qu'un homme pécheur devrait être amené

\par 25 pour rencontrer la Divinité ? Car la Divinité est effrayante et merveilleuse et qui ose

\par 26 regarder vers la Divinité incréée ? Si un homme regarde, il fondra comme la cire devant la face de Dieu : car la Divinité est ardente et merveilleuse. Car de tels gardiens se tiennent autour du trône de la Divinité.

\par \textit{Ceux qui entourent le trône divin}

\par 27 « Il y a des stations, [...] des creux, des ardents, des porteurs de ceintures, (et) des lanternes.

\par 28 En cet endroit il y a des tonnerres, des tremblements de terre, des querelles, des guerres, une chaleur brûlante, ô feu

\par 29 porteurs, ceux qui grouillent de flammes, (et) les armées ardentes. Autour de lui se trouvent des séraphins incorporels, des chérubins à six ailes : de deux ailes ils se couvrent le visage, et de deux ailes leurs pieds, et volant de deux, ils crient : « Saint, Saint, (Saint) Seigneur de

\par 30 Armées, le ciel et la terre sont remplis de votre gloire. De tels gardiens se tiennent autour du trône de la Divinité.

\par \textit{Libération de l'âme de Satan}

\par 31 Le prophète interrogea l'ange et dit : « Seigneur, que deviendrons-nous, car nous sommes tous pécheurs et saisis entre les mains de Satan ? Maintenant, par quels moyens sommes-nous

\par 32 délivrés ou qui nous fera sortir de ses mains ? L'ange répondit et dit : « Si quelqu'un reste après la mort, père ou mère ou frère ou sœur ou fils ou fille ou tout autre chrétien, et qu'il offre des prières, avec des jeûnes, pendant quarante

\par 33 jours, il y aura un grand repos et une grande miséricorde grâce au sacrifice du Christ. Car Christ a été sacrifié pour nous sur la croix et pendant six siècles il a délivré (notre) âme

\par 34 des mains de Satan. Comment l'âme est délivrée grâce à ce qui est offert avec révérence

\par 35 par un prêtre, s'il accomplit les quarante jours d'une manière qui plaît à Dieu ! Pendant quarante jours, il restera dans l'église, sans aller dans les lieux publics, mais de

\par 36 de temps en temps, ils réciteront les Psaumes de David avec des prières. C’est cela qui nous fait sortir des mains de Satan. Sinon, donnez aux pauvres.

\par \textit{La nature de la prière}

\par 37 « Car vos prières sont ainsi : tout comme un laboureur sort, vient semer, et la pousse sort joyeuse et gracieuse et désire produire de nombreux fruits, et les épines et les mauvaises herbes sortent aussi et l'étouffent et font ne laissez pas de nombreux fruits être

\par 38 assemblé. De même, vous aussi, lorsque vous entrez dans l'église et désirez offrir des prières devant la Divinité, les soucis de ce monde et la tromperie des grandeurs.

\par 39 (la richesse) sort et vous étouffe et ne laissez pas semer de nombreux fruits. Car si votre prière était telle que Moïse pleurait pendant quarante jours et parlait avec la bouche de Dieu


\par 40 à la bouche, de même Élie fut enlevé au ciel sur un char de feu, de même Daniel a également prié dans la tanière du lion… »

\par \textit{(Voir Recension B, 10—14.)}

\chapter{2}

\par \textit{Recension B}

\par 1 Il vit l'ange de Dieu et s'enquit des justes et des pécheurs quand

\par 2 ils sortent de ce monde. L'ange dit : « Pour les justes, il y a de la lumière

\par 3 et le repos, la vie éternelle, mais pour les pécheurs, des tourments sans fin ». Esdras dit : « S'il en est ainsi, alors bénis soient les animaux, les bêtes des champs et les rampants.

\par 4 choses et les oiseaux du ciel qui n'attendent pas la résurrection et le jugement. L'ange dit : « Vous péchez en disant cela, car Dieu a tout fait pour l'amour de l'homme et l'homme pour l'amour de Dieu. Et ces choses dans lesquelles Dieu trouve l'homme, par

\par 5 c'est lui qui a été jugé. Esdras a dit : « Quand vous prendrez les âmes des hommes, où

\par 6 tu les amènes ? L'ange dit : « J'amène les âmes des justes à adorer Dieu et je les établit dans la haute atmosphère, et les âmes des pécheurs sont

\par 7 saisis par les démons emprisonnés dans l’atmosphère. Et Esdras dit :

\par 8 «Et quand l'âme qui est saisie par Satan sera-t-elle délivrée ?» L'ange dit : « Quand l'âme a quelqu'un comme bon mémorial dans ce monde, (celui-ci) on la libère.

\par 9 de Satan par la prière et (des actes de) miséricorde. Ezra a demandé : « Par quels moyens ? L'ange dit : « Par la prière, par (les actes de) miséricorde et par les sacrifices. » (Ezra a dit)

\par 10 « Si l'âme du pécheur n'a pas de bon mémorial qui l'aide, que se passera-t-il ?

\par 11 à lui ? L'ange lui dit : « Un tel homme est entre les mains de Satan jusqu'à ce que le

\par 12 venue du Christ, quand la trompette de Gabriel sonnera. Alors les âmes sont libérées

\par 13 des mains de Satan et descendez de l'atmosphère. Et ils viennent et s'unissent chacun avec son corps qui avait été rendu poussière et dont le son

\par 14 de la trompette avait construit, suscité et renouvelé. Et il l'élève devant le Christ notre Dieu, qui vient juger (ceux qui sont sur) la terre, c'est-à-dire les justes et les méchants, et récompense chacun pour ses actes. Par la pétition de vos prophètes divinement rapportés, ayez pitié des lecteurs de cet écrit.

\end{document}