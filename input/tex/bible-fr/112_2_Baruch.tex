\begin{document}

\title{2 Baruch}

\chapter{1}

\par \textit{Annonce de la destruction prochaine de Jérusalem à Baruch}

\par 1 Et il arriva, la vingt-cinquième année de Jeconiah, roi de Juda, que la parole de l'Éternel fut adressée à Baruc, fils de Nériya, et lui dit :

\par 2 'Avez-vous vu tout ce que ce peuple me fait, que les maux qu'ont commis ces deux tribus restées sont plus grands que (ceux des) dix tribus qui ont été emmenées captives ?

\par 3 Car les premières tribus ont été forcées par leurs rois à commettre le péché, mais ces deux-là ont forcé et contraint leurs rois à commettre le péché.

\par 4 C'est pourquoi voici, je fais venir le mal sur cette ville et sur ses habitants, et elle sera éloignée de moi pour un temps, et je disperserai ce peuple parmi les païens, afin qu'il fasse du bien aux païens. Et mon peuple sera châtié, et le temps viendra où il recherchera la prospérité de son temps.

\chapter{2}

\par 1 Car je vous ai dit ces choses afin que vous disiez à Jérémie et à tous ceux qui vous ressemblent de se retirer de cette ville.

\par 2 Car tes œuvres sont pour cette ville comme un pilier solide,

\par Et vos prières comme un mur solide.'

\chapter{3}

\par 1 Et je dis : 'O Éternel, mon Seigneur, suis-je venu dans le monde dans ce but pour voir les maux de ma mère ? Pas (ainsi) mon Seigneur.

\par 2 Si j'ai trouvé grâce à tes yeux, prends d'abord mon esprit afin que j'aille vers mes pères et que je ne voie pas la destruction de ma mère.

\par 3 Car deux choses me contraignent avec véhémence : car je ne peux pas te résister, et mon âme, de plus, ne peut pas voir les maux de ma mère.

\par 4 Mais je dirai une chose en ta présence, ô Seigneur.

\par 5 Qu'y aura-t-il donc après ces choses ? car si vous détruisez votre ville et livrez votre pays à ceux qui nous haïssent, comment se souviendra-t-on encore du nom d'Israël ?

\par 6 Ou comment parlerait-on de tes louanges ? ou à qui expliquera-t-on ce qui est dans ta loi ? Ou le monde reviendra-t-il à sa nature d’autrefois, et l’époque reviendra-t-elle au silence primitif ? Et la multitude des âmes sera-t-elle enlevée, et la nature de l'homme ne sera-t-elle plus nommée ? Et où est tout ce que vous avez dit à notre sujet ?

\chapter{4}

\par 1 Et le Seigneur me dit :

\par 'Cette ville sera livrée pour un temps,

\par Et le peuple sera châtié pendant un certain temps,

\par Et le monde ne sera pas livré à l'oubli.

\par \textit{La Jérusalem céleste}

\par 2 [Penses-tu que ce soit cette ville dont j'ai dit : « Sur la paume de mes mains je t'ai gravé » ?

\par 3 Ce bâtiment maintenant construit au milieu de vous n'est pas celui qui m'a été révélé, celui qui m'a été préparé d'avance ici depuis le moment où j'ai pris le conseil de faire le Paradis et que j'ai montré à Adam avant qu'il péchait, mais quand il a transgressé le commandement, c'était éloigné de lui, ainsi que le Paradis.

\par 4 Et après ces choses, je le montrai de nuit à mon serviteur Abraham, parmi les portions des victimes.

\par 5 Et encore une fois, je l'ai montré à Moïse sur le mont Sinaï, lorsque j'ai montré la ressemblance du tabernacle et de tous ses ustensiles.

\par 6 Et maintenant, voici, il est conservé auprès de Moi, comme le Paradis.

\par 7 Allez donc et faites ce que je vous commande.']

\chapter{5}

\par \textit{La plainte de Baruch et le réconfort de Dieu}

\par 1 Et je répondis et dis :

\par 'Alors je suis destiné à pleurer Sion,

\par Car tes ennemis viendront ici et pollueront ton sanctuaire,

\par Et conduis ton héritage en captivité,

\par Et se rendent maîtres de ceux que vous avez aimés,

\par Et ils repartiront vers le lieu de leurs idoles,

\par Et il se vantera devant eux :

\par Et que feras-tu pour ton grand nom ?

\par 2 Et le Seigneur me dit :

\par 'Mon nom et ma gloire sont pour toute l'éternité ;

\par Et mon jugement maintiendra son droit en son temps.

\par 3 Et tu verras de tes yeux

\par Pour que l'ennemi ne renverse pas Sion,

\par Ils ne brûleront pas non plus Jérusalem,

\par Mais soyez les ministres du Juge pour le temps.

\par 4 Mais vas-tu et fais tout ce que je t'ai dit.


\par 5 Et je suis allé et j'ai pris Jérémie, et Adu, et Seriah, et Jabish, et Guedalia, et tous les hommes honorables du peuple, et je les ai conduits dans la vallée de Cédron, et je leur ai raconté tout ce qui s'était passé. me dit.

\par 6 Et ils élevèrent la voix, et ils pleurèrent tous.

\par 7 Et nous restâmes là et jeûnâmes jusqu'au soir.

\chapter{6}

\par \textit{Invasion des Chaldéens et leur entrée dans la ville après que les vases sacrés furent cachés et les murs de la ville renversés par les anges}

\par 1 Et il arriva le lendemain que, voilà ! L'armée des Chaldéens entoura la ville, et au moment du soir, moi, Baruch, j'ai quitté le peuple, et je suis sorti et je me suis tenu près du chêne.

\par 2 Et j'étais affligé à cause de Sion, et je me lamentais sur la captivité qui était tombée sur le peuple.

\par 3 Et voilà ! tout à coup, un esprit fort m'enleva et m'emporta par-dessus la muraille de Jérusalem.

\par 4 Et je vis, et voilà ! quatre anges debout aux quatre coins de la ville, chacun tenant dans ses mains une torche de feu.

\par 5 Et un autre ange commença à descendre du ciel. et leur dit : « Tenez vos lampes et ne les allumez pas jusqu'à ce que je vous le dise.

\par 6 Car je suis envoyé d'abord pour dire une parole à la terre, et pour y placer ce que l'Éternel le Très-Haut m'a commandé.

\par 7 Et je le vis descendre dans le Saint des Saints, et en prendre le voile, et l'arche sainte, et le propitiatoire, et les deux tables, et les vêtements sacrés des prêtres, et l'autel des parfums, et les quarante-huit pierres précieuses dont le prêtre était orné, ainsi que tous les ustensiles sacrés du tabernacle.

\par 8 Et il parla à la terre à haute voix :

\par 'Terre, terre, terre, écoutez la parole du Dieu puissant,

\par Et reçois ce que je te confie,

\par Et garde-les jusqu'aux derniers temps,

\par Afin que, lorsque vous serez commandé, vous puissiez les restaurer,

\par Afin que des étrangers ne puissent pas s'en emparer.

\par 9 Car le temps vient où Jérusalem aussi sera délivrée pour un temps,

\par Jusqu'à ce qu'on dise qu'il est de nouveau restauré pour toujours.'

\par 10 Et la terre ouvrit sa bouche et les engloutit.

\chapter{7}

\par 1 Et après ces choses, j'entendis cet ange dire à ces anges qui tenaient les lampes : Détruisez donc et renversez sa muraille jusqu'à ses fondations, de peur que l'ennemi ne se vante et ne dise :

\par « Nous avons renversé la muraille de Sion,

\par Et nous avons brûlé la place du Dieu puissant.

\par 2 Et ils se sont emparés de l'endroit où je me tenais auparavant.

\chapitre{8}

\par 1 Les anges firent ce qu'il leur avait commandé, et après qu'ils eurent démoli les angles des murs, une voix se fit entendre de l'intérieur du temple, après la chute du mur, disant :

\par 2 « Entrez, vous ennemis,

\par Et venez, vous adversaires ;

\par Car celui qui gardait la maison l'a abandonnée.'

\par 3 Et moi, Baruch, je suis parti.

\par 4 Après ces choses, l'armée des Chaldéens entra et s'empara de la maison et de tout ce qui l'entourait. Ils emmenèrent le peuple en captivité, tuèrent quelques-uns d'entre eux, ligotèrent le roi Sédécias et l'envoyèrent au roi de Babylone.

\chapitre{9}

\par \textit{Premier jeûne de sept jours : Baruch restera au milieu des ruines de Jérusalem et Jérémie accompagnera les exilés à Babylone. Le chant funèbre de Baruch sur Jérusalem}

\par 1 Et moi, Baruch, je suis venu, ainsi que Jérémie, dont le cœur avait été trouvé pur de ses péchés, qui n'avait pas été capturé lors de la prise de la ville.

\par 2 Et nous avons déchiré nos vêtements, nous avons pleuré, et nous avons jeûné sept jours.

\chapitre{10}

\par 1 Et il arriva qu'après sept jours, la parole de Dieu vint à moi et me dit :

\par 2 'Dites à Jérémie d'aller soutenir la captivité du peuple à Babylone. Mais restez ici au milieu de la désolation de Sion, et je vous montrerai après ces jours « ce qui arrivera à la fin des jours ». Et je dis à Jérémie ce que le Seigneur me l'avait ordonné. Et lui, en effet, partit avec le peuple, mais moi, Baruch, je revins et m'assis devant les portes du temple, et je me lamentai sur Sion avec la lamentation suivante et dis :

\par 3 [...]

\par 4 [...]

\par 5 [...]

\par 6 Bienheureux celui qui n'est pas né,

\par Ou bien celui qui est né est mort.

\par 7 Mais quant à nous qui vivons, malheur à nous,

\par Parce que nous voyons les afflictions de Sion,

\par Et ce qui est arrivé à Jérusalem.

\par 8 J'appellerai les sirènes de la mer,

\par Et toi Lilin, tu viens du désert,

\par Et vous Shedim et les dragons des forêts :

\par Réveillez-vous et ceignez vos reins pour le deuil,

\par Et prends avec moi les chants funèbres,

\par Et fais des lamentations avec moi.

\par 9 Vous, laboureurs, ne semez plus ;

\par Et, ô terre, pourquoi donnes-tu les fruits de ta récolte ?

\par Gardez en vous les douceurs de votre subsistance.

\par 10 Et toi, vigne, pourquoi donnes-tu encore ton vin ;

\par Car on ne fera plus d'offrande de là, en Sion,

\par Et les prémices ne seront plus offertes.

\par 11 Et vous, ô cieux, retenez votre rosée,

\par Et n'ouvrez pas les trésors de pluie :

\par 12 Et toi, ô soleil, retiens la lumière de tes rayons.

\par Et toi, ô lune, éteins la multitude de ta lumière ;

\par Car pourquoi la lumière devrait-elle se lever à nouveau

\par Où la lumière de Sion est obscurcie ?

\par 13 Et vous, les mariés, n'entrez pas,

\par Et que les mariées ne se parent pas de guirlandes ;

\par Et vous, femmes, ne priez pas pour porter.

\par 14 Car les stériles se réjouiront par-dessus tout,

\par Et ceux qui n'ont pas de fils seront heureux,

\par Et ceux qui ont des fils auront de l'angoisse.

\par 15 Car pourquoi devraient-ils supporter la douleur,

\par Seulement pour enterrer dans le chagrin ?

\par 16 Ou pourquoi, encore une fois, l'humanité devrait-elle avoir des fils ?

\par Ou pourquoi la graine de son espèce devrait-elle encore être nommée,

\par Où cette mère est désolée,

\par Et ses fils sont emmenés en captivité ?

\par 17 Désormais, ne parlez plus de beauté,

\par Et un discours non gracieux.

\par 18 Et vous, prêtres, prenez les clefs du sanctuaire,

\par Et je les jetai dans les hauteurs du ciel,

\par Et donne-les au Seigneur et dis :

\par « Garde toi-même ta maison,

\par Pour voilà ! nous sommes trouvés de faux intendants.

\par 19 Et vous, vous les vierges ; qui tisse du fin lin

\par Et de la soie avec de l'or d'Ophir,

\par Prenez en toute hâte toutes (ces) choses

\par Et je les jetai au feu,

\par Afin qu'il les porte à Celui qui les a faits,

\par Et la flamme les envoie à Celui qui les a créés,

\par De peur que l'ennemi ne s'en empare.

\chapitre{11}

\par 1 Et moi, Baruc, je dis ceci contre toi, Babylone :

\par 'Si tu avais prospéré,

\par Et Sion avait habité dans sa gloire,

\par Pourtant, le chagrin pour nous avait été grand

\par Afin que tu sois égal à Sion.

\par 2 Mais maintenant, voilà ! le chagrin est infini,

\par Et les lamentations sans mesure,

\par Pour voilà ! tu es prospère

\par Et Sion est désolée.

\par 3 Qui sera juge de ces choses ?

\par Ou à qui devrions-nous nous plaindre de ce qui nous est arrivé ?

\par O Seigneur, comment l'as-tu supporté ?

\par 4 Nos pères se reposèrent sans chagrin,

\par Et voilà ! les justes dorment en paix sur la terre ;

\par 5 Car ils ne connaissaient pas cette angoisse,

\par Et ils n'avaient pas encore entendu parler de ce qui nous était arrivé.

\par 6 Si tu avais des oreilles, ô terre,

\par Et que tu avais un cœur, ô poussière :

\par Afin que tu puisses aller annoncer dans le Schéol,

\par Et dis aux morts :

\par 7 « Heureux es-tu plus que nous qui vivons. »

\chapitre{12}

\par \textit{OXY = OXYRHYNCHUS GREEK FRAGMENT, de Grenfell et Hunt's Oxyrhynchus Papyri, vol. iii. 3-7, 1903. Verso.}

\par 1 [Mais je dirai cela comme je pense.] [OXY : Mais je dirai cela comme je pense, ]

\par [Et je parlerai contre toi, ô pays qui prospère toujours.] [OXY : Et je parlerai contre toi, pays qui prospère.]

\par 2 [Le midi ne brûle pas toujours.] [OXY : Le midi ne brûle pas toujours,]

\par [Les rayons du soleil ne donnent pas non plus constamment de lumière.] [OXY : Les rayons du soleil ne donnent pas non plus constamment de lumière.]

\par 3 [N'espérez pas que vous serez toujours prospère et joyeux.] [OXY : Et n'espérez pas vous réjouir,]

\par [Et ne soyez pas trop élevé et vantard.] [OXY : Ne condamnez pas non plus grandement.]

\par 4 [Car assurément en son temps la colère (divine) s'éveillera contre vous.] [OXY : Car assurément en son temps la colère (divine) s'éveillera contre vous,]

\par [Qui est maintenant retenu par la patience comme par des rênes.] [OXY : Qui est maintenant retenu par la patience comme par une rêne.]

\par 5 [Et après avoir dit ces choses, j'ai jeûné sept jours.] [OXY : Et après avoir dit ces choses, j'ai jeûné sept jours.]


\chapitre{13}

\par \textit{OXY = OXYRHYNCHUS GREEK FRAGMENT, de Grenfell et Hunt's Oxyrhynchus Papyri, vol. iii. 3-7, 1903. Verso.}

\par \textit{Deuxième Rapide. Révélation quant au jugement à venir sur les païens.}

\par 1 [Et il arriva après ces choses, que moi, Baruch, je me tenais sur le mont Sion, et voici ! une voix vint d'une hauteur et me dit :] [OXY : Et il arriva après ces choses que moi, Baruch, je me tenais sur le mont Sion, et voici, une voix sortit d'une hauteur et me dit :]

\par 2 ['Lève-toi, Baruch, et écoute la parole du Dieu puissant.'] [OXY : 'Lève-toi, Baruch, et écoute la parole du Dieu puissant.']

\par 3 Parce que tu as été étonné de ce qui est arrivé à Sion, tu seras donc assurément préservé jusqu'à la consommation des temps, afin que tu sois pour témoignage.

\par 4 De sorte que, si jamais ces villes prospères disent :

\par 5 'Pourquoi le Dieu puissant nous a-t-il infligé ce châtiment ?' Dites-leur, à vous et à ceux comme vous qui aurez vu ce mal : « (C'est le mal) et le châtiment qui vient sur vous et sur votre peuple en son temps (destiné) afin que les nations soient complètement frappées.

\par 6 Et alors ils seront dans l'angoisse.

\par 7 Et s'ils disent à ce moment-là :

\par 8 Pour combien de temps ? tu leur diras :

\par « Vous qui avez bu du vin filtré,

\par Buvez-vous aussi de sa lie,

\par Le jugement du Très-Haut

\par Qui n'a aucun respect pour les personnes.

\par 9 C'est pourquoi il n'avait autrefois aucune pitié pour ses propres fils,

\par mais il les affligea comme ses ennemis, parce qu'ils avaient péché,

\par 10 Alors donc furent-ils châtiés

\par Afin qu'ils soient sanctifiés.

\par 11 [Mais maintenant, vous, peuples et nations, vous êtes coupables] [OXY : (Vous) peuples et . . .]

\par [Parce que vous avez toujours foulé la terre,] [OXY : (Vous) avez foulé la terre]

\par [Et a utilisé injustement la création.] [OXY : Et a abusé des choses créées en elle.]

\par 12 [Car je t'ai toujours été bénéfique.] [OXY : Car tu en as toujours bénéficié]

\par [Et tu as toujours été ingrat pour la bienfaisance.] [OXY : Mais tu as toujours été ingrat.]

\chapitre{14}

\par \textit{La justice des justes n'a profité ni à eux ni à leur ville ; Les jugements de Dieu sont incompréhensibles ; le Monde a été fait pour les Justes, pourtant ils passent et le Monde reste (14). Réponse : L'homme connaît les jugements de Dieu et a péché volontairement. Ce monde est une lassitude pour les justes, mais le prochain leur appartient (15), à conquérir par le caractère, que le temps d'un homme ici soit long ou court (16-17). Bonheur ou malheur final : la question suprême (18-19).}

\par 1 [Et j'ai répondu et j'ai dit : 'Lo ! tu m'as montré la méthode des temps et ce qui arrivera après ces choses, et tu m'as dit que le châtiment dont tu as parlé viendrait sur les nations.] [OXY : Et moi répondit et dit : « Voici, tu m'as montré les méthodes des temps et ce qui arrivera. Et vous m'avez dit que le châtiment dont vous avez parlé sera supporté par les nations.]

\par 2 [Et maintenant je sais que ceux qui ont péché sont nombreux, et qu'ils ont vécu dans la prospérité' et sont partis du monde, mais il restera en ces temps-là quelques nations à qui il dira ces paroles que vous a dit.] [OXY : Et maintenant je sais que ceux qui ont péché sont nombreux, et ils ont vécu. . . , et quitté le monde, mais qu'il restera peu de nations en ces temps à qui . . . les mots (que) vous avez dit.]

\par 3 [Car quel avantage y a-t-il à cela, ou quel (mal), pire que ce que nous avons vu nous arriver, devons-nous nous attendre à voir ?] [OXY : Et quel avantage (y a-t-il) à ceci ou quoi de pire que ceux ci?)]

\par 4 Mais je parlerai encore une fois en ta présence :

\par 5 Qu'ont profité ceux qui avaient la connaissance avant toi et qui n'ont pas marché dans la vanité comme le reste des nations, et qui n'ont pas dit aux morts : « Donne-nous la vie », mais qui t'ont toujours craint et qui n'ont pas quitté tes voies ? ?

\par 6 Et voilà ! ils ont été enlevés, et c'est à cause d'eux que vous avez eu pitié de Sion.

\par 7 Et si d'autres faisaient le mal, il était dû à Sion qu'à cause des œuvres de ceux qui ont fait de bonnes œuvres, elle soit pardonnée, et qu'elle ne soit pas accablée à cause des œuvres de ceux qui ont fait l'injustice.

\par 8 Mais qui, ô Éternel, mon Seigneur, comprendra ton jugement,

\par Ou qui saura découvrir la profondeur de Ta voie ?

\par Ou qui pensera au poids de Ton chemin ?

\par 9 Ou qui saura réfléchir Ton conseil incompréhensible ?

\par Ou lequel de ceux qui sont nés a jamais trouvé

\par Le début ou la fin de Ta sagesse ?


\par 10 Car nous avons tous été créés comme un souffle.

\par 11 Car, comme le souffle monte involontairement et meurt de nouveau, ainsi en est-il de la nature des hommes, qui ne s'en vont pas selon leur propre volonté, et qui ne savent pas ce qui leur arrivera à la fin.

\par 12 Car les justes espèrent avec raison la fin, et quittent sans crainte cette demeure, parce qu'ils ont avec vous une réserve d'œuvres conservée dans des trésors.

\par 13 C'est pour cela aussi que ceux-ci quittent ce monde sans crainte, et confiants dans la joie, ils espèrent recevoir le monde que tu leur as promis.

\par 14 Mais quant à nous, malheur à nous qui sommes maintenant honteusement suppliés et qui, en ce temps-là, attendons (seulement) des malheurs.

\par 15 Mais tu sais exactement ce que tu as fait par l'intermédiaire de tes serviteurs ; car nous ne sommes pas capables de comprendre ce qui est bon comme tu es, notre Créateur.

\par 16 Mais je parlerai encore une fois en ta présence, ô Éternel, mon Seigneur.

\par 17 Quand autrefois il n'y avait pas de monde avec ses habitants, vous avez conçu et prononcé une parole, et aussitôt les œuvres de la création se sont présentées devant vous.

\par 18 Et tu as dit que tu ferais pour ton monde un homme comme administrateur de tes œuvres, afin qu'on sache qu'il n'a en aucune façon été créé à cause du monde, mais que le monde a été créé à cause de lui.

\par 19 Et maintenant je vois que quant au monde qui a été fait à cause de nous, voilà ! il demeure; mais nous, pour qui cela a été fait, partons.

\chapitre{15}

\par 1 Et le Seigneur répondit et me dit : 'Tu es à juste titre étonné du départ de l'homme, mais tu n'as pas bien jugé des maux qui arrivent à ceux qui pèchent.

\par 2 Et quant à ce que vous avez dit, à savoir que les justes sont enlevés et que les impies prospèrent,

\par 3 Et quant à ce que tu as dit : « L’homme ne connaît pas ton jugement » – C’est pourquoi écoutez, et je vous parlerai, et écoutez, et je vous ferai entendre mes paroles.

\par 4 [...]

\par 5 L'homme n'aurait pas correctement compris mon jugement, s'il n'avait pas accepté la loi et si je ne lui avais instruit la compréhension.

\par 6 Mais maintenant, parce qu'il a transgressé sciemment, oui, juste pour ce motif qu'il connaît (à ce sujet), il sera tourmenté.

\par 7 Et quant à ce que vous avez dit concernant les justes, que ce monde est venu à cause d'eux, de même ce qui est à venir viendra à cause d'eux.

\par 8 Car ce monde est pour eux un conflit et un travail très pénible ; et ce qui est donc à venir, une couronne d'une grande gloire.

\chapitre{16}

\par 1 Et je répondis et dis : « Ô Éternel, mon Seigneur, voici ! les années de cette époque sont rares et mauvaises, et qui est capable, dans son peu de temps, d'acquérir ce qui est sans mesure ?

\chapitre{17}

\par 1 Et le Seigneur répondit et me dit : « Au Très-Haut, on ne compte ni le temps ni quelques années.

\par 2 Car quel profit a Adam de vivre neuf cent trente ans et de transgresser ce qui lui avait été commandé ? C'est pourquoi la multitude de temps qu'il a vécu ne lui a pas profité, mais a amené la mort et a retranché les années de ceux qui sont nés de lui. Pourquoi Moïse a-t-il subi une perte dans la mesure où il n'a vécu que cent vingt ans et, dans la mesure où il était soumis à Celui qui l'a formé, a apporté la loi à la postérité de Jacob et a allumé une lampe pour la nation d'Israël ?

\chapitre{18}

\par 1 Et je répondis et dis : Celui qui a éclairé a pris de la lumière, et il y en a peu qui l'ont imité. Mais ceux qu'il a éclairés sont nombreux, sortis des ténèbres d'Adam et ne se sont pas réjouis de la lumière de la lampe.

\chapitre{19}

\par 1 Et il répondit et me dit : « C'est pourquoi, en ce temps-là, il leur établit une alliance et dit :

\par «Voici, j'ai placé devant toi la vie et la mort»,

\par Et il prit à témoin contre eux le ciel et la terre.

\par 2 Car il savait que son temps était court,

\par Mais que le ciel et la terre durent toujours.

\par 3 Mais après sa mort, ils péchèrent et transgressèrent,

\par Bien qu'ils savaient que la loi les réprimandait,

\par Et la lumière dans laquelle rien ne peut errer,

\par Aussi les sphères qui témoignent, et Moi.

\par 4 Or, concernant tout ce qui est, c'est Moi qui juge, mais ne prends pas conseil en ton âme sur ces choses, et ne t'afflige pas à cause de ce qui a été.

\par 5 Car maintenant c'est la consommation du temps qu'il faut considérer, soit des affaires, soit de la prospérité, soit de la honte et non son commencement.

\par 6 Parce que si un homme est prospère dans ses débuts et honteusement supplié dans sa vieillesse, il oublie toute la prospérité qu'il a eue.

\par 7 Et encore, si un homme est honteusement supplié au début, et qu'à la fin il réussit, il ne se souvient plus de ses mauvais traitements.

\par 8 Et encore écoutez : même si chacun avait prospéré pendant tout ce temps — tout le temps depuis le jour où la mort a été décrétée contre ceux qui transgressent — et qu'à sa fin il ait été détruit, tout aurait été en vain.'

\chapitre{20}

\par \textit{Sion a été enlevée pour hâter l'avènement du Jugement}

\par 1 'C'est pourquoi voici ! les jours viennent,

\par Et les temps se hâteront plus que les premiers,

\par Et les saisons s'accéléreront plus que celles qui sont passées,

\par Et les années passeront plus vite que les années présentes.

\par 2 C'est pourquoi j'ai maintenant enlevé Sion,

\par Afin que je puisse visiter plus rapidement le monde en sa saison.

\par 3 Maintenant donc, retiens fermement dans ton cœur tout ce que je te commande,

\par et scellez-le dans les recoins de votre esprit.

\par 4 Et alors je te montrerai le jugement de ma puissance,

\par et Mes voies qui sont insondables.

\par 5 Va donc te sanctifier pendant sept jours, et ne mange pas de pain, ne bois pas d'eau, et ne parle à personne.

\par 6 Et ensuite, je viendrai à cet endroit et je me révélerai à vous, et je vous dirai des choses vraies, et je vous donnerai des commandements concernant la méthode des temps ; car ils viennent et ne tardent pas.

\chapitre{21}

\par \textit{Jeûne de sept jours : la prière de Baruch : la réponse de Dieu}

\par \textit{La prière de Baruch, fils de Nériya.}

\par 1 Et j'y suis allé et je me suis assis dans la vallée du Cédron dans une grotte de la terre, et j'y ai sanctifié mon âme, et je n'ai pas mangé de pain, mais je n'ai pas eu faim, et je n'ai pas bu d'eau, mais je n'ai pas soif. , et j'y restai jusqu'au septième jour, comme il me l'avait ordonné.

\par 2 Et ensuite je suis arrivé à l'endroit où il m'avait parlé.

\par 3 Et il arriva qu'au coucher du soleil, mon âme réfléchit beaucoup, et je me mis à parler en présence du Tout-Puissant, et je dis :

\par 4 'Ô vous qui avez fait la terre, écoutez-moi, qui avez fixé le firmament par la parole, et qui avez affermi la hauteur du ciel par l'esprit, qui avez appelé dès le commencement du monde ce qui n'a pas été fait. existent encore, et ils vous obéissent.

\par 5 vous qui avez commandé l'air par votre signe de tête, et qui avez vu les choses qui doivent arriver comme celles que vous faites.

\par 6 Toi qui gouvernes avec grande pensée les armées qui se tiennent devant toi ; tu gouvernes aussi avec indignation les innombrables êtres saints que tu as créés dès le commencement, de flamme et de feu, qui se tiennent autour de ton trône.

\par 7 À vous seul appartient de faire immédiatement tout ce que vous voulez.

\par 8 Qui fait pleuvoir les gouttes de pluie en nombre sur la terre, et seul connaît la consommation des temps avant qu'ils ne viennent ; ayez du respect pour ma prière. Pour

\par 9 Toi seul peux soutenir tous ceux qui sont, et ceux qui sont décédés, et ceux qui doivent être, ceux qui pèchent et ceux qui sont trop justes [comme vivant (et) étant indécis]. Car vous seul vivez immortel et au-delà de toute connaissance, et connaissez le nombre de l'humanité. Et si au fil du temps beaucoup ont péché, d'autres encore ont été justes.'

\par \textit{La dépréciation de cette vie par Baruch.}

\par 10 [...]

\par 11 [...]

\par 12 vous savez où vous conservez la fin de ceux qui ont péché, ou la consommation de ceux qui ont été justes.

\par 13 Car s'il y avait seulement cette vie, qui appartient à tous les hommes, rien ne pourrait être plus amer que celle-là.

\par 14 Car à quoi profite la force qui se transforme en maladie,

\par Ou une plénitude de nourriture qui se transforme en famine,

\par Ou la beauté qui tourne à la laideur.

\par 15 Car la nature de l'homme est toujours changeante.

\par 16 Car ce que nous étions autrefois, nous ne le sommes plus, et ce que nous sommes maintenant, nous ne le resterons plus ensuite.

\par 17 Car si une consommation n'avait pas été préparée pour tous, leur commencement aurait été en vain. Mais sur tout ce qui vient de vous, informez-moi, et sur tout ce que je vous demande, éclairez-moi.

\par \textit{Baruch prie Dieu d'accélérer le jugement et d'accomplir sa promesse}

\par 18 [...]

\par 19 Jusqu'à quand subsistera ce qui est corruptible, et combien de temps le temps des mortels sera-t-il prospère, et jusqu'à quand ceux qui transgressent dans le monde seront-ils souillés par beaucoup de méchanceté ?

\par 20 Commande donc avec miséricorde et accomplis tout ce que tu as dit que tu apporterais, afin que ta puissance soit manifestée à ceux qui pensent que ta longanimité est une faiblesse.

\par 21 Et montre à ceux qui ne le savent pas que tout ce qui nous est arrivé, à nous et à notre ville, jusqu'à présent, s'est déroulé selon la longanimité de ta puissance, parce qu'à cause de ton nom tu nous as appelés peuple bien-aimé.

\par 22 Mettez donc fin désormais à la mortalité.

\par 23 Et réprimande en conséquence l'ange de la mort, et que ta gloire apparaisse, et que la puissance de ta beauté soit connue, et que le Shéol soit scellé afin qu'à partir de ce moment il ne reçoive plus les morts, et que les trésors de les âmes restaurent celles qui sont enfermées en elles.

\par 24 Car il y a eu de nombreuses années semblables à celles qui sont désolées depuis les jours d'Abraham, d'Isaac et de Jacob, et de tous ceux qui leur ressemblent, qui dorment sur la terre, à cause desquels vous avez dit que vous aviez créé le monde.

\par 25 Et maintenant, montre vite ta gloire, et ne diffère pas ce que tu as promis.

\par 26 Et (quand) j'eus achevé les paroles de cette prière, j'étais grandement affaibli.

\chapitre{22}

\par \textit{Réponse de Dieu à la prière de Baruch. Il accomplira Sa Promesse : Temps nécessaire à son accomplissement : Les choses doivent être jugées à la Lumière de leur consommation (22). Tant que toutes les âmes ne sont pas nées, la fin ne peut pas venir (23).}

\par 1 Et il arriva après ces choses que voici ! les cieux s'ouvrirent, et je vis, et un pouvoir me fut donné, et une voix se fit entendre d'en haut, et elle me dit :

\par 2 Baruch, Baruch, pourquoi es-tu troublé ?

\par 3 Celui qui voyage par une route mais ne la termine pas, ou celui qui part par mer mais n'arrive pas au port, peut-il être consolé ?

\par 4 Ou celui qui promet de faire un présent à autrui, mais ne le tient pas, n'est-ce pas un vol ?

\par 5 Ou celui qui sème la terre, mais ne récolte pas ses fruits en sa saison, ne perd-il pas tout ?

\par 6 Ou celui qui plante une plante qui ne pousse pas jusqu'au temps qui lui convient, celui qui l'a plantée espère-t-il en recevoir du fruit ?

\par 7 Ou une femme qui a conçu, si elle accouche prématurément, ne tue-t-elle pas assurément son enfant ?

\par 8 Ou celui qui bâtit une maison, s'il ne la couvre pas et ne la complète pas, peut-on l'appeler une maison ? Dis-le-moi d’abord.

\chapitre{23}

\par 1 Et je répondis et dis : 'Non, ô Éternel, mon Seigneur.'

\par 2 Et Il répondit et me dit : 'Pourquoi donc es-tu troublé à propos de ce que tu ne sais pas, et pourquoi es-tu mal à l'aise à propos de choses que tu ignores ?

\par 3 Car, comme vous n'avez pas oublié les gens qui sont maintenant et ceux qui sont décédés, ainsi je me souviens de ceux qui sont destinés à venir.

\par 4 Parce que quand Adam a péché et que la mort a été décrétée contre ceux qui devaient naître, alors la multitude de ceux qui devaient naître a été dénombrée, et pour ce nombre un lieu a été préparé où les vivants pourraient habiter et les morts pourraient être gardés. Avant donc que le nombre susmentionné ne soit accompli, la créature ne revivra plus [car Mon esprit est le créateur de la vie], et le Shéol recevra les morts.

\par 5 [...]

\par 6 Et encore il vous est donné d'entendre ce qui arrivera après ces temps.

\par 7 Car en vérité ma rédemption est proche et n'est pas aussi lointaine qu'auparavant.

\chapitre{24}

\par \textit{Le jugement à venir}

\par 1 'Car voici ! les jours viennent et les livres seront ouverts dans lesquels sont écrits les péchés de tous ceux qui ont péché, et encore aussi les trésors dans lesquels est rassemblée la justice de tous ceux qui ont été justes dans la création.

\par 2 Car il arrivera qu'en ce temps-là vous verrez, et ceux qui sont nombreux avec vous, la longanimité du Très-Haut, qui a été de toutes générations, qui a été patient envers tous ceux qui sont nés, (aussi bien) ceux qui pèchent que (ceux qui) sont justes.'

\par 3 Et je répondis et dis : 'Mais voici ! O Seigneur, personne ne connaît le nombre de ces choses passées ni encore celles qui sont à venir.

\par 4 Car je sais en effet ce qui nous est arrivé, mais ce qui arrivera à nos ennemis, je ne le sais pas, ni quand vous visiterez vos œuvres.

\chapitre{25}

\par \textit{Signe du Jugement prochain}

\par 1 Et Il répondit et me dit : « Toi aussi tu seras préservé jusqu'à ce moment-là jusqu'à ce signe que le Très-Haut opérera pour les habitants de la terre à la fin des jours. '

\par 2 Ce sera donc le signe.

\par 3 Quand la stupeur saisira les habitants de la terre, et qu'ils tomberont dans de nombreuses tribulations, et encore quand ils tomberont dans de grands tourments. Et cela arrivera quand ils diront dans leurs pensées à cause de leurs nombreuses tribulations : « Le Puissant « On ne se souvient plus de la terre » – oui, il arrivera quand ils abandonneront l’espérance, que le temps s’éveillera alors. .'

\chapitre{26}

\par \textit{Les douze malheurs qui doivent venir sur la terre : le Messie et le Royaume messianique temporaire}

\par 1 Et je répondis et dis : « Cette tribulation qui doit durer durera-t-elle longtemps, et cette tribulation s'étendra-t-elle nécessairement sur de nombreuses années ?

\chapitre{27}

\par 1 Et Il répondit et me dit : « Ce temps est divisé en douze parties, et chacune d'elles est réservée à ce qui lui est assigné.

\par 2 Dans la première partie, il y aura le début des commotions.

\par 3 Et dans la seconde partie (il y aura) des meurtres des grands.

\par 4 Et dans la troisième partie la chute de plusieurs par la mort.

\par 5 Et dans la quatrième partie l'envoi de l'épée.

\par 6 Et dans la cinquième partie la famine et le refus de la pluie.

\par 7 Et dans la sixième partie les tremblements de terre et les terreurs.

\par 8 [Vouloir.]

\par 9 Et dans la huitième partie une multitude de spectres et d'attaques des Shedim.

\par 10 Et dans la neuvième partie la chute du feu.

\par 11 Et dans la dixième partie, la rapine et beaucoup d'oppression.

\par 12 Et dans la onzième partie la méchanceté et l'impudicité.

\par 13 Et dans la douzième partie, confusion due au mélange de toutes ces choses ci-dessus.

\par 14 Car ces parties de ce temps-là sont réservées, et seront mélangées les unes aux autres et s'administreront les unes aux autres.

\par 15 Car certains laisseront de côté une partie des leurs et recevront (à la place) des autres, et certains complèteront leur propre et celui des autres, afin que ceux qui sont sur la terre en ces jours-là ne comprennent pas que ceci est la consommation des temps.

\chapitre{28}

\par 1 'Néanmoins, celui qui comprend sera alors sage.

\par 2 Car la mesure et le calcul de ce temps sont de deux parties par semaine de sept semaines.

\par 3 Et je répondis et dis : « Il est bon qu'un homme vienne et regarde, mais il vaut mieux qu'il ne vienne pas, de peur de tomber.

\par 4 [Mais je dirai aussi ceci :

\par 5 Celui qui est incorruptible méprisera-t-il les choses qui sont corruptibles, et tout ce qui arrive dans le cas de ces choses qui sont corruptibles, afin de ne regarder que ce qui ne est pas corruptible ?]

\par 6 Mais si ; O Seigneur, les choses que tu m'as prédites arriveront assurément, alors montre-le-moi aussi si j'ai trouvé grâce à tes yeux.

\par 7 Est-ce en un seul endroit ou dans une des parties de la terre que ces choses se sont produites, ou la terre entière en fera-t-elle l'expérience ?'

\chapitre{29}

\par 1 Et Il répondit et me dit : « Tout ce qui arrivera alors (arrivera) à toute la terre ; c'est pourquoi tous les vivants en feront l'expérience.

\par 2 Car en ce temps-là, je protégerai seulement ceux qui se trouveront en ces mêmes jours dans ce pays.

\par 3 Et il arrivera quand tout ce qui devait arriver dans ces régions sera accompli, que le Messie commencera alors à être révélé.

\par 4 Et Behemoth sera révélé de sa place et Léviathan montera de la mer, ces deux grands monstres que j'ai créés le cinquième jour de la création, et que j'aurai gardés jusqu'à ce moment-là ; et alors ils serviront de nourriture à tous ceux qui resteront.

\par 5 La terre aussi rendra son fruit dix mille et sur chaque (?) vigne il y aura mille sarments, et chaque sarment produira mille grappes, et chaque grappe produira mille raisins, et chaque raisin produira un cor. du vin.

\par 6 Et ceux qui ont faim se réjouiront ; et aussi, ils verront des merveilles chaque jour.

\par 7 Car des vents sortiront de devant moi pour apporter chaque matin le parfum des fruits aromatiques, et à la fin du jour des nuages ​​distillant la rosée de la santé.

\par 8 Et il arrivera à ce même moment que le trésor de manne descendra de nouveau d'en haut, et ils en mangeront pendant ces années-là, car ce sont eux qui sont parvenus à la consommation des temps. '

\chapitre{30}

\par \textit{La Résurrection}

\par 1 Et il arrivera après ces choses, lorsque le temps de l'avènement du Messie sera accompli, qu'il reviendra dans la gloire.

\par 2 Alors tous ceux qui se sont endormis dans l'espérance de Lui se relèveront. Et il arrivera à ce moment-là que les trésors dans lesquels est conservé le nombre des âmes des justes seront ouverts, et ils sortiront, et une multitude d'âmes seront vues ensemble dans un seul rassemblement d'une seule pensée, et le premier se réjouira et le dernier ne sera pas attristé.

\par 3 Car ils savent que le temps est venu dont on dit que c'est la consommation des temps.

\par 4 Mais les âmes des méchants, lorsqu'ils verront toutes ces choses, dépériront alors encore davantage.

\par 5 Car ils sauront que leur tourment est venu et que leur perdition est arrivée.

\chapitre{31}

\par \textit{Baruch exhorte le peuple à se préparer à des maux pires}

\par 1 Et il arriva après ces choses : que j'allai vers le peuple et leur dis : « Rassemblez-moi tous vos anciens et je leur dirai des paroles. »

\par 2 Et ils se rassemblèrent tous dans la vallée du Cédron.

\par 3 Et je répondis et leur dis :

\par Écoute, Israël, et je te parlerai,

\par Et prête l'oreille, postérité de Jacob, et je t'instruirai.

\par 4 N'oubliez pas Sion,

\par Mais souvenez-vous de l'angoisse de Jérusalem.

\par 5 Pour voilà ! les jours viennent,

\par Quand tout ce qui existe deviendra la proie de la corruption

\par Et faire comme si cela n'avait pas été.

\chapitre{32}

\par 1 'Mais quant à vous, si vous préparez vos cœurs, de manière à y semer les fruits de la loi, elle vous protégera en ce temps où le Puissant doit ébranler toute la création.

\par 2 [Parce que dans peu de temps la construction de Sion sera ébranlée afin qu'elle puisse être reconstruite. Mais ce bâtiment ne subsistera pas, mais sera de nouveau détruit après un certain temps et restera désolé jusqu'au moment venu.

\par 3 [...]

\par 4 Et ensuite il doit être renouvelé dans la gloire et perfectionné pour toujours.]

\par 5 C'est pourquoi nous ne devrions pas être autant affligés du mal qui est arrivé maintenant que de celui qui est encore à venir.

\par 6 Car il y aura une épreuve plus grande que ces deux tribulations lorsque le Tout-Puissant renouvellera sa création.

\par 7 Et maintenant, ne t'approche pas de moi pendant quelques jours, et ne me cherche pas jusqu'à ce que je vienne vers toi.

\par 8 Et il arriva que lorsque je leur eus dit toutes ces paroles, moi, Baruch, je m'en allai, et quand les gens me virent partir, ils élevèrent la voix et se lamentèrent et dirent :

\par 9 Vers où nous quittes-tu, Baruch, et nous abandonnes-tu comme un père qui abandonne ses enfants orphelins et s'éloigne d'eux ?

\chapitre{33}

\par 1 « Sont-ce là les commandements que votre compagnon, Jérémie le prophète, vous a ordonnés, et vous a dit : » Regardez ce peuple jusqu'à ce que j'aille préparer le reste des frères à Babylone contre lesquels a été prononcée la sentence qui ils devraient être emmenés en captivité » ? Et maintenant, si toi aussi tu nous abandonnes, il serait bon que nous mourions tous avant toi, et ensuite que tu te retires de nous.

\chapitre{34}

\par \textit{Lamentation de Baruch}

\par 1 Et je répondis et dis au peuple : Loin de moi l'idée de vous abandonner ou de me retirer de vous, mais j'irai seulement au Saint des Saints pour consulter le Tout-Puissant à votre sujet et à propos de Sion, si à certains égards, je devrais recevoir plus d'illumination : et après ces choses, je reviendrai vers vous.

\chapitre{35}

\par 1 Et moi, Baruch, je suis allé au lieu saint, je me suis assis sur les ruines, j'ai pleuré et j'ai dit :

\par 2 'Oh, si mes yeux étaient des sources,

\par Et mes paupières sont une source de larmes.

\par 3 Car comment pourrais-je me lamenter sur Sion,

\par Et comment pleurerai-je Jérusalem ?

\par 4 Parce qu'à l'endroit où je suis maintenant prosterné,

\par Autrefois, le grand prêtre offrait de saints sacrifices,

\par Et on y plaça un encens aux odeurs odorantes.

\par 5 Mais maintenant notre gloire a été transformée en poussière,

\par Et le désir de notre âme dans le sable.'

\chapitre{36}

\par \textit{La Vision de la Forêt, de la Vigne, de la Fontaine et du Cèdre}

\par 1 Et après avoir dit ces choses, je m'endormis là, et j'ai eu une vision pendant la nuit.

\par 2 Et voilà ! une forêt d'arbres plantée dans la plaine, et des montagnes rocheuses élevées et escarpées l'entouraient, et cette forêt occupait beaucoup d'espace.

\par 3 Et voilà ! en face s'élevait une vigne, et de dessous sortait une fontaine paisible.

\par 4 Or cette fontaine arriva dans la forêt et fut (remuée) en grandes vagues, et ces vagues submergeèrent cette forêt, et soudain elles déracinèrent la plus grande partie de cette forêt, et renversèrent toutes les montagnes qui l'entouraient.

\par 5 Et la hauteur de la forêt commença à s'abaisser, et le sommet des montagnes fut abaissé et cette fontaine prédomina grandement, de sorte qu'elle ne laissa rien de cette grande forêt, sauf un cèdre seulement.

\par 6 Aussi, après qu'il l'eut abattu et qu'il eut détruit et déraciné la plus grande partie de cette forêt, de sorte qu'il n'en resta plus rien et que sa place ne pouvait être reconnue, alors cette vigne commença à venir avec la fontaine en paix et grande tranquillité, et il arriva à un endroit qui n'était pas loin de ce cèdre, et ils y apportèrent le cèdre qui avait été jeté.

\par 7 Et j'ai vu et voilà ! cette vigne ouvrit la bouche et parla et dit à ce cèdre : N'es-tu pas ce cèdre qui est resté de la forêt de la méchanceté, et par le moyen duquel la méchanceté a persisté et s'est produite pendant toutes ces années, et la bonté jamais.

\par 8 Et tu as continué à conquérir ce qui n'était pas à toi, et envers ce qui était à toi tu n'as jamais montré de compassion, et tu as continué à étendre ton pouvoir sur ceux qui étaient loin de toi, et sur ceux qui s'approchaient de toi, tu as tenu rapide dans les labeurs de ta méchanceté, et tu t'es toujours élevé comme quelqu'un qui ne pouvait pas être déraciné !

\par 9 Mais maintenant votre temps est passé et votre heure est venue.

\par 10 Pars donc toi aussi, ô cèdre, après la forêt qui s'est éloignée avant toi, et qui est devenue poussière avec elle, et que tes cendres se mélangent ensemble.

\par 11 Et maintenant reposez-vous dans l'angoisse et reposez-vous dans le tourment jusqu'à ce que vienne votre dernier temps, dans lequel vous reviendrez et serez tourmenté encore davantage.

\chapitre{37}

\par 1 Et après ces choses, je vis ce cèdre brûlant, et la vigne poussant, elle-même et tout autour, la plaine pleine de fleurs qui ne se fanent pas. Et je me suis effectivement réveillé et je me suis levé.

\chapitre{38}

\par \textit{Interprétation de la Vision}

\par 1 Et j'ai prié et j'ai dit : 'Ô Éternel, mon Seigneur, tu éclaires toujours ceux qui sont conduits par l'intelligence.'

\par 2 Ta loi est la vie, et ta sagesse est une bonne direction.

\par 3 Faites-moi donc connaître l'interprétation de cette vision.

\par 4 Car tu sais que mon âme a toujours marché dans ta loi, et que depuis mes (premiers) jours je ne me suis pas écarté de ta sagesse.'

\chapitre{39}

\par 1 Et Il répondit et me dit : 'Baruch, ceci est l'interprétation de la vision que tu as vue.

\par 2 Comme vous avez vu la grande forêt qu'entouraient des montagnes élevées et escarpées, voici le mot.

\par 3 Voyez ! les jours viennent où ce royaume sera détruit, celui qui détruisit autrefois Sion, et il sera soumis à celui qui viendra après lui.

\par 4 De plus, celui-là aussi sera détruit après un certain temps, et un autre, un troisième, surgira, et celui-là aussi dominera pour son temps, et sera détruit.

\par 5 Et après ces choses, un quatrième royaume s'élèvera, dont la puissance sera dure et mauvaise bien au-delà de celles qui l'ont précédé, et il régnera plusieurs fois comme les forêts de la plaine, et il tiendra bon pendant des temps, et s'exaltera plus que les cèdres du Liban.

\par 6 Et par elle la vérité sera cachée, et tous ceux qui sont pollués par l'iniquité y fuiront, comme les mauvaises bêtes fuient et se rampent dans la forêt.

\par 7 Et il arrivera que lorsque le temps de sa consommation où il devrait tomber sera approché, alors le principat de mon Messie sera révélé, qui est comme la fontaine et la vigne, et quand il sera révélé, il sera déraciné. la multitude de son hôte.

\par 8 Et quant à ce que vous avez vu, le cèdre élevé qui restait de cette forêt, et le fait que la vigne prononçait avec elle les paroles que vous avez entendues, voici la parole.

\chapitre{40}

\par 1 Le dernier chef de ce temps sera laissé en vie, lorsque la multitude de ses armées seront passées au fil de l'épée, et il sera lié, et ils l'emmèneront sur le mont Sion, et mon Messie le convaincra de toutes ses impiétés, et il rassemblera et mettra devant lui toutes les œuvres de ses hôtes.

\par 2 Et ensuite il le fera mourir, et protégera le reste de mon peuple qui se trouvera dans le lieu que j'ai choisi.

\par 3 Et son principat subsistera pour toujours, jusqu'à ce que le monde de corruption soit terminé et jusqu'à ce que les temps susdits soient accomplis.

\par 4 Ceci est votre vision, et ceci est son interprétation.


\chapitre{41}

\par \textit{La Destinée des Apostats et des Prosélytes}

\par 1 Et je répondis et dis : « Pour qui et pour combien de personnes ces choses seront-elles ? ou qui sera digne de vivre à cette époque-là ?

\par 2 Car je dirai devant vous tout ce que je pense, et je vous demanderai ce que je médite.

\par 3 Pour voilà ! Je vois beaucoup de ton peuple qui se sont retirés de ton alliance et qui ont rejeté le joug de ta loi.

\par 4 Mais j'en ai vu d'autres encore qui ont abandonné leur vanité et ont fui pour se réfugier sous tes ailes.

\par 5 Que leur arrivera donc ? ou comment les recevra-t-on la dernière fois ?

\par 6 Ou peut-être que leur temps sera assurément pesé, et que, à mesure que le faisceau s'incline, ils seront jugés en conséquence ?

\chapitre{42}

\par 1 Et il répondit et me dit : 'Ces choses aussi, je vous les montrerai.

\par 2 Quant à ce que tu as dit : « À qui ces choses seront-elles, et à combien (seront-elles) ? » Ceux qui ont cru auront le bien dont il a été parlé auparavant, et ceux qui méprisent auront le contraire de ces choses.

\par 3 Et quant à ce que vous avez dit concernant ceux qui se sont approchés et ceux qui se sont retirés, cela dans la parole.

\par 4 Quant à ceux qui étaient auparavant sujets, puis se retirèrent et se mêlèrent à la postérité des peuples mêlés, le temps de ceux-ci était le premier et était considéré comme quelque chose d'exalté.

\par 5 Et quant à ceux qui auparavant ne connaissaient pas, mais qui ont ensuite connu la vie, et qui se sont mêlés (uniquement) à la postérité du peuple qui s'était séparé, le temps de ceux-ci (est) celui-ci, et est considéré comme quelque chose d'exalté.

\par 6 Et les temps succèderont aux temps et aux saisons, et l'un recevra de l'autre, et alors, en vue de la consommation, tout sera comparé selon la mesure des temps et des heures des saisons.

\par 7 Car la corruption prendra ceux qui lui appartiennent, et la vie ceux qui lui appartiennent.

\par 8 Et la poussière sera appelée, et on lui dira : « Rendez ce qui n'est pas à vous, et relevez tout ce que vous avez gardé jusqu'à son temps ».''


\chapitre{43}

\par \textit{Baruch a raconté sa mort et a demandé de donner ses derniers ordres au peuple}

\par 1 Mais toi, Baruch, dirige ton cœur vers ce qui t'a été dit,

\par Et comprenez ce qui vous a été montré ;

\par Car il y a pour vous de nombreuses consolations éternelles.

\par 2 Car vous partirez de ce lieu,

\par Et vous quitterez les régions que vous voyez maintenant,

\par Et tu oublieras tout ce qui est corruptible,

\par Et je ne me souviendrai plus de ce qui arrive parmi les mortels.

\par 3 Va donc et commande à ton peuple, et viens à cet endroit, et après je jeûne sept jours, et ensuite je viendrai vers toi et je te parlerai.

\par \textit{Baruch annonce aux Anciens sa mort imminente, mais les encourage à s'attendre à la Consolation de Sion}

\chapitre{44}

\par 1 Et moi, Baruc, je partais de là et je suis arrivé vers mon peuple, et j'ai appelé mon fils premier-né et [les Guedalia] mes amis, et sept des anciens du peuple, et je leur ai dit :

\par 2 Voici, je vais vers mes pères

\par Selon la voie de toute la terre.

\par 3 Mais ne vous éloignez pas du chemin de la loi,

\par Mais gardez et avertissez le peuple qui reste,

\par De peur qu'ils ne s'éloignent des commandements du Tout-Puissant.

\par 4 Car vous voyez que Celui que nous servons est juste,

\par Et notre Créateur ne fait acception de personnes.

\par 5 Et voyez ce qui est arrivé à Sion,

\par Et qu'est-il arrivé à Jérusalem?

\par 6 Car le jugement du Tout-Puissant sera (ainsi) fait connaître,

\par Et ses voies, qui, bien qu'inconnues, sont justes.

\par 7 Car si vous endurez et persévérez dans sa crainte,

\par Et n'oubliez pas sa loi,

\par Les temps changeront pour vous pour de bon.

\par Et vous verrez la consolation de Sion.

\par 8 Parce que tout ce qui est maintenant n'est rien,

\par Mais ce qui arrivera est très grand.

\par Car tout ce qui est corruptible passera,

\par 9 Et tout ce qui meurt partira,

\par Et tout le temps présent sera oublié,

\par Il n'y aura pas non plus de souvenir du temps présent, qui est souillé par les maux.

\par 10 Car ce qui court maintenant court à la vanité,

\par Et celui qui prospère tombera bientôt et sera humilié.

\par 11 Car ce qui doit être sera l'objet du désir,

\par Et nous espérons ce qui viendra après ;

\par Car c'est un temps qui ne passe pas,

\par 12 Et vient l'heure qui demeure éternellement.

\par Et le monde nouveau (vient) qui ne tourne pas à la corruption ceux qui s'en vont vers sa béatitude,

\par Et n'a aucune pitié pour ceux qui partent pour le tourment,

\par Et ne conduit pas à la perdition ceux qui y vivent.

\par 13 Car ce sont eux qui hériteront du temps dont il a été parlé,

\par Et à eux est l'héritage du temps promis.

\par 14 Ce sont eux qui se sont acquis des trésors de sagesse,

\par Et avec eux se trouvent des réserves de compréhension,

\par Et ils ne se sont pas retirés de la miséricorde,

\par Et ils ont préservé la vérité de la loi.

\par 15 Car le monde à venir leur sera donné,

\par Mais la demeure des autres, qui sont nombreux, sera dans le feu.

\chapitre{45}

\par 1 « Instruisez donc le peuple autant que vous le pouvez, car ce travail est à nous. Car si vous les enseignez, vous les vivifierez.

\chapitre{46}

\par 1 Et mon fils et les anciens du peuple répondirent et me dirent :

\par 'Le Tout-Puissant nous a-t-il humiliés à un tel degré

\par Quant à vous éloigner de nous rapidement ?

\par 2 Et en vérité nous serons dans les ténèbres,

\par Et il n'y aura pas de lumière pour le peuple qui reste,

\par 3 Car où chercherons-nous encore la loi,

\par Ou qui fera pour nous la distinction entre la mort et la vie ?

\par 4 Et je leur dis : Je ne peux résister au trône du Tout-Puissant ;

\par Néanmoins, il ne manquera pas à Israël d'un homme sage

\par Ni un fils de la loi pour la race de Jacob.

\par 5 Mais préparez seulement vos cœurs, afin que vous obéissiez à la loi,

\par Et soyez soumis à ceux qui, dans la crainte, sont sages et intelligents ;

\par Et préparez vos âmes afin que vous ne puissiez pas vous en éloigner.

\par 6 Car si vous faites ces choses, de bonnes nouvelles vous parviendront.

\par [Dont je vous ai déjà parlé ; et vous ne tomberez pas dans le tourment dont je vous ai déjà témoigné.

\par 7 Mais quant à la parole selon laquelle je devais être emmené, je ne l'ai pas fait savoir à eux ni à mon fils.]

\chapitre{47}

\par 1 Et après être sorti et les avoir congédiés, j'y suis allé et leur ai dit : « Voici ! Je vais à Hébron : c'est là que le Tout-Puissant m'a envoyé.

\par 2 Et je suis arrivé à l'endroit où la parole m'avait été annoncée, et je me suis assis là, et j'ai jeûné sept jours.

\chapitre{48}

\par \textit{PRIÈRE DE BARUCH}

\par 1 Et il arriva après le septième jour que j'ai prié devant le Tout-Puissant et j'ai dit

\par 2 'Ô mon Seigneur, tu appelles l'avènement des temps,

\par Et ils se tiennent devant toi ;

\par Tu fais passer la puissance des siècles,

\par Et ils ne vous résistent pas ;

\par Vous arrangez la méthode des saisons,

\par Et ils vous obéissent.

\par 3 Toi seul connais la durée des générations,

\par Et tu ne révèles pas tes mystères à beaucoup.

\par 4 Tu fais connaître la multitude du feu,

\par Et tu pèses la légèreté du vent.

\par 5 Vous explorez la limite des hauteurs,

\par Et tu scrutes les profondeurs des ténèbres.

\par 6 Vous prenez soin du nombre de ceux qui passent, afin qu'ils soient conservés, et vous préparez une demeure à ceux qui doivent naître.

\par 7 Tu te souviens du commencement que tu as fait,

\par Et la destruction qui doit avoir lieu, Tu ne l'oublies pas.

\par 8 Avec des hochements de tête de peur et d'indignation, tu commandes aux flammes,

\par Et ils se changent en esprits,

\par Et d'une parole tu vivifies ce qui n'était pas,

\par Et avec une grande puissance vous détenez ce qui n'est pas encore venu.

\par 9 Tu instruis les choses créées dans ta compréhension,

\par Et vous rendez sages les sphères afin de servir dans leurs ordres.

\par 10 Armées innombrables se tiennent devant vous

\par Et exécutez leurs ordres tranquillement à votre signe de tête.

\par 11 Écoute ton serviteur

\par Et prête l'oreille à ma requête.

\par 12 Car dans peu de temps naissons-nous,

\par Et dans peu de temps nous reviendrons.

\par 13 Mais pour toi les heures sont comme un temps,

\par Et les jours comme générations.

\par 14 Ne vous irritez donc pas contre l'homme ; car il n'est rien

\par 15 Et ne tenez pas compte de nos œuvres ; Pour quoi sommes-nous ?

\par Pour voilà ! c'est par ton don que nous venons au monde,

\par Et nous ne partons pas de notre propre volonté.

\par 16 Car nous n'avons pas dit à nos parents : Engendez-nous,

\par Nous n'avons pas non plus envoyé au shéol pour dire : « Recevez-nous ».

\par 17 Quelle est donc notre force pour supporter ta colère,

\par Ou que sommes-nous pour endurer ton jugement ?

\par 18 Protège-nous dans Tes compassions,

\par Et dans Ta miséricorde, aide-nous.

\par 19 Voici les petits qui vous sont soumis,

\par Et sauve tous ceux qui s'approchent de toi :

\par Et ne détruis pas l'espérance de notre peuple,

\par Et n'écourtez pas les délais de notre aide.

\par 20 Car c'est ici la nation que tu as choisie,

\par Et ce sont là des gens auxquels vous ne trouvez pas d'égal.

\par 21 Mais je parlerai maintenant devant toi,

\par Et je dirai ce que pense mon cœur.

\par 22 En toi nous avons confiance, car voilà ! Ta loi est avec nous,

\par Et nous savons que nous ne tomberons pas tant que nous garderons tes statuts.

\par 23 [Nous sommes toujours bénis en tout cas de ne pas nous mêler aux Gentils.]

\par 24 Car nous sommes tous un seul peuple célèbre,

\par Qui ont reçu une loi d'Un :

\par Et la loi qui est parmi nous nous aidera,

\par Et la sagesse incomparable qui est en nous nous aidera.

\par 25 Et quand j'eus prié et dit ces choses, j'étais très affaibli.

\par 26 Et il répondit et me dit :

\par 'Tu as prié simplement, ô Baruch,

\par Et toutes tes paroles ont été entendues.

\par 27 Mais mon jugement exige le sien

\par Et ma loi exige ses droits.

\par 28 Car c'est à partir de tes paroles que je te répondrai,

\par Et c'est à partir de votre prière que je vous parlerai.

\par 29 Car voici ceci : celui qui est corrompu ne l'est pas du tout ; il a à la fois commis l'iniquité autant qu'il pouvait faire quelque chose, et ne s'est pas souvenu de ma bonté ni n'a accepté ma longanimité.

\par 30 C'est pourquoi vous serez sûrement enlevés, comme je vous l'ai déjà dit.

\par 31 Car il viendra un temps qui apportera l'affliction ; car cela viendra et passera avec une véhémence rapide, et ce sera turbulent venant dans la chaleur de l'indignation.

\par 32 Et il arrivera en ces jours-là que tous les habitants de la terre seront émus les uns contre les autres, parce qu'ils ne savent pas que mon jugement est proche.

\par 33 Car on ne trouvera pas beaucoup de sages en ce temps-là,

\par Et les intelligents ne seront que quelques-uns :

\par De plus, même ceux qui savent se taisent avant tout.

\par 34 Et il y aura beaucoup de rumeurs et de nouvelles, pas rares,

\par Et l'action des fantasmes sera manifeste,

\par Et promet que pas mal d'entre eux soient racontés,

\par Certains d'entre eux (se révéleront) oisifs,

\par Et certains d'entre eux seront confirmés.

\par 35 Et l'honneur sera changé en honte,

\par Et la force humiliée jusqu'au mépris,

\par Et la probité détruite,

\par Et la beauté deviendra laideur.

\par 36 Et plusieurs diront à plusieurs en ce temps-là :

\par « Où s'est cachée la multitude de l'intelligence,

\par Et où s’est-elle éloignée de la multitude de la sagesse ?

\par 37 Et pendant qu'ils méditent ces choses,

\par Alors l'envie naîtra chez ceux qui n'avaient rien pensé d'eux-mêmes (?)

\par Et la passion saisira celui qui est paisible,

\par Et beaucoup seront excités par la colère pour faire du mal à beaucoup,

\par Et ils lèveront des armées pour verser le sang,

\par Et à la fin ils périront avec eux.

\par 38 Et il arrivera en même temps,

\par Afin qu'un changement de temps plaise manifestement à tout homme,

\par Parce que pendant toutes ces périodes, ils se sont pollués

\par Et ils pratiquèrent l'oppression,

\par Et chacun marchait dans ses propres œuvres,

\par Et il ne s'est pas souvenu de la loi du Tout-Puissant.

\par 39 C'est pourquoi un feu consumera leurs pensées,

\par Et c'est dans la flamme que seront éprouvées les méditations de leurs reins ;

\par Car le juge viendra et ne tardera pas.

\par 40 Parce que chacun des habitants de la terre savait quand il transgressait.

\par Mais ils ne connaissaient pas ma Loi à cause de leur orgueil.

\par 41 Mais alors beaucoup pleureront assurément,

\par Oui, sur les vivants plus que sur les morts.

\par 42 Et je répondis et dis :

\par 'Ô Adam, qu'as-tu fait à tous ceux qui sont nés de toi ?

\par Et que dira-t-on à la première Ève qui écouta le serpent ?

\par 43 Car toute cette multitude va à la corruption,

\par Il n'y a pas non plus de dénombrement de ceux que le feu dévore.

\par 44 Mais je parlerai encore une fois en ta présence.

\par 45 Toi, Éternel, mon Seigneur, tu sais ce qu'il y a dans ta créature.

\par 46 Car tu as autrefois ordonné à la poussière d'engendrer Adam, et tu connais le nombre de ceux qui sont nés de lui, et jusqu'à quel point ils ont péché avant toi, qui ont existé et ne t'ont pas confessé comme leur Créateur.

\par 47 Et pour tout cela, leur fin les convaincra, et ta loi qu'ils ont transgressée leur rendra justice en ton jour.

\par \textit{Fragment d'un discours de Baruch au peuple}

\par 48 ['Mais maintenant, renvoyons les méchants et interrogeons les justes.

\par 49 Et je raconterai leur béatitude

\par Et ne vous taisez pas en célébrant leur gloire qui leur est réservée.

\par 50 Car assurément, comme en peu de temps, dans ce monde transitoire où vous vivez, vous avez enduré beaucoup de travail,

\par Ainsi, dans ce monde sans fin, vous recevrez une grande lumière.']

\chapitre{49}

\par \textit{La nature du corps de résurrection : les destinées finales des justes et des méchants}

\par 1 Néanmoins, je te demanderai encore, ô Puissant, oui, je demanderai toutes choses.

\par 2 « Sous quelle forme vivront ceux qui vivent à ton époque ?

\par Ou comment la splendeur de ceux qui (sont) après ce temps continuera-t-elle ?

\par 3 Vont-ils alors reprendre cette forme du présent,

\par Et revêtez ces membres encombrants,

\par Qui sont maintenant impliqués dans les maux,

\par Et dans lequel les maux sont consommés,

\par Ou vas-tu peut-être changer ces choses qui ont été dans le monde

\par Comme aussi le monde ?

\chapitre{50}

\par 1 Et il répondit et me dit :

\par 'Écoute, Baruch, ce mot,

\par Et écris dans le souvenir de ton cœur tout ce que tu apprendras.

\par 2 Car la terre restaurera alors assurément les morts,

\par [Qu'il reçoit maintenant, afin de les conserver].

\par Cela ne fera aucun changement dans leur forme,

\par Mais comme il les a reçus, il les restituera également,

\par Et comme je les lui ai livrés, ainsi aussi il les ressuscitera.

\par 3 Car alors il faudra montrer aux vivants que les morts sont revenus à la vie, et que ceux qui étaient partis sont revenus (encore).

\par 4 Et il arrivera que lorsqu'ils auront chacun reconnu ceux qu'ils connaissent maintenant, alors le jugement deviendra fort, et ces choses dont il a été parlé auparavant arriveront.

\chapitre{51}

\par 1 Et il arrivera que, lorsque ce jour fixé sera passé, alors l'aspect de ceux qui sont condamnés sera ensuite changé, et la gloire de ceux qui sont justifiés.

\par 2 Car l'aspect de ceux qui agissent mal maintenant deviendra pire qu'il ne l'est, à mesure qu'ils subiront le tourment.

\par 3 Et la gloire de ceux qui ont été maintenant justifiés par ma loi, qui ont eu de l'intelligence dans leur vie et qui ont planté dans leur cœur la racine de la sagesse, alors leur splendeur sera glorifiée dans les changements, et la forme de leur visage sera transformée en la lumière de leur beauté, afin qu'ils puissent acquérir et recevoir le monde qui ne meurt pas, qui leur est alors promis.

\par 4 Car c'est surtout à cause de cela que ceux qui viendront se lamenteront d'avoir rejeté ma loi et d'avoir bouché leurs oreilles pour ne pas entendre la sagesse ni recevoir l'intelligence.

\par 5 Quand donc ils verront ceux sur lesquels ils sont maintenant exaltés, (mais) qui seront alors exaltés et glorifiés plus qu'eux, ils seront respectivement transformés, les seconds en la splendeur des anges, et les premiers seront encore plus s'évanouir d'émerveillement face aux visions et à la contemplation des formes.

\par 6 Car ils verront d'abord, puis partiront pour être tourmentés.

\par 7 Mais ceux qui ont été sauvés par leurs œuvres,

\par Et pour qui la loi est maintenant un espoir,

\par Et comprendre une attente,

\par Et la sagesse une confiance,

\par Des merveilles apparaîtront-elles en leur temps.

\par 8 Car ils verront le monde qui leur est maintenant invisible,

\par Et ils verront le temps qui leur est maintenant caché :

\par 9 Et le temps ne les vieillira plus.

\par 10 Car ils habiteront dans les hauteurs de ce monde,

\par Et ils seront rendus semblables aux anges,

\par Et soyez égal aux étoiles,

\par Et ils seront transformés en toute forme qu'ils désireront,

\par De la beauté à la beauté,

\par Et de la lumière à la splendeur de la gloire.

\par 11 Car les étendues du paradis s'étendront devant eux, et là leur sera montrée la beauté de la majesté des créatures vivantes qui sont sous le trône, et de toutes les armées des anges, qui sont maintenant retenues. par Ma parole, de peur qu'ils n'apparaissent et] ne soient retenus par un ordre, afin qu'ils puissent rester à leur place jusqu'à ce que leur avènement vienne.

\par 12 De plus, il y aura alors une excellence chez les justes, surpassant celle des anges.

\par 13 Car les premiers recevront les derniers, ceux qu'ils attendaient, et les derniers ceux dont ils apprenaient qu'ils étaient décédés.

\par 14 Car ils ont été délivrés de ce monde de tribulation,

\par Et j'ai déposé le fardeau de l'angoisse.

\par 15 Pourquoi donc les hommes ont-ils perdu la vie,

\par Et contre quoi ceux qui étaient sur la terre ont-ils échangé leur âme ?

\par 16 Car alors ils ont choisi (pas) pour eux-mêmes cette fois-ci,

\par Qui, hors de portée de l'angoisse, ne pouvait pas passer :

\par Mais ils ont choisi eux-mêmes ce moment-là,

\par Dont les issues sont pleines de lamentations et de maux,

\par Et ils ont nié le monde qui ne vieillit pas ceux qui y viennent,

\par Et ils rejetèrent le temps de gloire,

\par Afin qu'ils n'obtiennent pas l'honneur dont je vous ai déjà parlé.

\chapitre{52}

\par 1 Et je répondis et dis :

\par 'Comment oublier ceux à qui le malheur est alors réservé ?

\par 2 Et pourquoi pleurons-nous encore une fois ceux qui meurent ?

\par Ou pourquoi pleurons-nous ceux qui partent au schéol ?

\par 3 Que les lamentations soient réservées au début de ce tourment à venir,

\par Et que des larmes soient réservées pour l'avènement de la destruction de ce temps-là.

\par 4 [Mais même face à ces choses, je parlerai.

\par 5 Et quant aux justes, que feront-ils maintenant ?

\par 6 Réjouissez-vous des souffrances que vous souffrez maintenant :

\par Car pourquoi attendez-vous le déclin de vos ennemis ?

\par 7 Préparez votre âme à ce qui vous est réservé,

\par Et préparez vos âmes à la récompense qui vous est réservée.']

\chapitre{53}

\par \textit{L'APOCALYPSE DU MESSIE}

\par \textit{La Vision du Nuage aux Eaux noires et blanches}


\par 1 Et après avoir dit ces choses, je m'endormis là, et j'ai eu une vision, et voilà ! un nuage montait d'une très grande mer, et je le regardais continuellement) et voilà ! elle était pleine d'eaux blanches et noires, et il y avait beaucoup de couleurs dans ces mêmes eaux, et comme on voyait à son sommet l'image d'un grand éclair.

\par 2 Et je vis le nuage passer rapidement en courses rapides, et il couvrit toute la terre.

\par 3 Et il arriva après ces choses que ce nuage commença à déverser sur la terre les eaux qui y étaient.

\par 4 Et je vis qu'il n'y avait pas une seule et même ressemblance dans les eaux qui en descendaient.

\par 5 Car au premier commencement elles étaient noires et nombreuses (Ou un temps, et ensuite j'ai vu que les eaux devenaient claires, mais elles n'étaient pas nombreuses, et après ces choses encore j'ai vu des (eaux) noires, et après ces choses encore une fois brillant, et encore noir et encore brillant.

\par 6 Or cela se faisait douze fois, mais les noirs étaient toujours plus nombreux que les brillants.

\par 7 Et il arriva au bout du nuage, que voici ! il pleuvait des eaux noires, et elles étaient plus sombres que toutes ces eaux qui étaient auparavant, et du feu s'y mêlait, et là où ces eaux descendaient, elles provoquaient la dévastation et la destruction.

\par 8 Et après ces choses, je vis comment l'éclair que j'avais vu au sommet du nuage, le saisit et le jeta à terre.

\par 9 Or, cet éclair brillait extrêmement, au point d'éclairer toute la terre, et il guérissait les régions où les dernières eaux étaient descendues et avaient fait des ravages.

\par 10 Et il s'empara de toute la terre et la domina.

\par 11 Et j'ai vu après ces choses, et voici ! douze fleuves sortaient de la mer, et ils commencèrent à entourer cet éclair et à y devenir soumis.

\par 12 Et à cause de ma peur, je me suis réveillé.

\chapitre{54}

\par \textit{Prière de Baruch pour une interprétation de la vision : l'avènement de Ramiel dans ce but}


\par 1 Et j'ai supplié le Tout-Puissant, et j'ai dit :

\par «Toi seul, Seigneur, connais depuis longtemps les choses profondes du monde,

\par Et tu réalises par ta parole les choses qui arrivent en leurs temps, et contre les œuvres des habitants de la terre tu hâtes le commencement des temps,

\par Et la fin des saisons, toi seul la connais.

\par 2 (Toi) pour qui rien n'est trop dur,

\par Mais qui fait tout facilement par un clin d'œil :

\par 3 (Toi) à qui les profondeurs viennent comme les hauteurs,

\par Et dont la parole est servie par les commencements des siècles :

\par 4 (Toi) qui révèles à ceux qui te craignent ce qui leur est préparé,

\par Afin qu'ils soient désormais consolés.

\par 5 Tu fais de grandes choses à ceux qui ne savent pas ;

\par Tu brises la clôture des ignorants,

\par Et illumine ce qui est sombre,

\par Et révèle ce qui est caché aux purs,

\par [Qui dans la foi se sont soumis à toi et à ta loi.]

\par 6 Tu as montré cette vision à ton serviteur ;

\par Révèle-moi aussi son interprétation.

\par 7 Car je sais qu'en ce qui concerne les choses pour lesquelles je vous ai demandé, j'ai reçu une réponse,

\par Et quant à ce que je t'ai demandé, tu m'as révélé de quelle voix je devais te louer,

\par Et de quels membres je ferais monter vers vous des louanges et des alléluias.

\par 8 Car si mes membres étaient des bouches,

\par Et les cheveux de ma tête voix,

\par Même ainsi, je ne pouvais pas te donner la récompense de la louange,

\par Ni te louer comme il convient,

\par Je ne pourrais pas non plus raconter ta louange,

\par Ne raconte pas non plus la gloire de ta beauté.

\par 9 Car que suis-je parmi les hommes,

\par Ou pourquoi suis-je compté parmi ceux qui sont plus excellents que moi,

\par Que j'ai entendu toutes ces merveilles venant du Très-Haut,

\par Et d'innombrables promesses de Celui qui m'a créé ?

\par 10 Bénie soit ma mère parmi celles qui enfantent,

\par Et louée parmi les femmes soit celle qui m'a enfanté.

\par 11 Car je ne me tairais pas en louant le Tout-Puissant,

\par Et avec une voix de louange, je raconterai ses merveilles.

\par 12 Car qui fait comme tes merveilles, ô Dieu,

\par Ou qui comprends Ta pensée profonde de la vie.

\par 13 Car par ton conseil tu gouvernes toutes les créatures que ta main droite a créées.

\par Et tu as établi toute fontaine de lumière à côté de toi,

\par Et tu as préparé les trésors de la sagesse sous ton trône.

\par 14 Et c'est à juste titre que périssent ceux qui n'ont pas aimé ta loi,

\par Et le tourment du jugement attendra ceux qui ne se seront pas soumis à ta puissance.

\par 15 Car bien qu'Adam ait péché le premier

\par Et a amené une mort prématurée sur tous,

\par Pourtant, de ceux qui sont nés de lui

\par Chacun d'eux a préparé le tourment de son âme à venir,

\par Et encore une fois, chacun d'eux a choisi pour lui-même les gloires à venir.

\par 16 [Car assurément, celui qui croit recevra une récompense.

\par 17 Mais maintenant, vous qui êtes méchants maintenant, tournez-vous vers la perdition, car vous serez bientôt visités, en ce sens que vous aviez autrefois rejeté l'intelligence du Très-Haut.

\par 18 Car ses œuvres ne vous ont pas enseigné,

\par L'habileté de Sa création, qui vous a toujours persuadé, n'est pas non plus valable.]

\par 19 Adam n'est donc la cause que de sa propre âme,

\par Mais chacun de nous a été l'Adam de sa propre âme.

\par 20 Mais toi, Seigneur, explique-moi ce que tu m'as révélé,

\par Et informe-moi de ce que je t'ai demandé.

\par 21 Car à la consommation du monde, la vengeance sera prise sur ceux qui ont commis le mal selon leur méchanceté,

\par Et tu glorifieras les fidèles selon leur fidélité.

\par 22 Car tu gouvernes ceux qui sont parmi les tiens,

\par Et ceux qui pèchent, vous les effacez du milieu des vôtres.

\chapitre{55}

\par 1 Et il arriva, lorsque j'eus fini de prononcer les paroles de cette prière, que je m'assis là sous un arbre, pour me reposer à l'ombre des branches.

\par 2 Et je m'étonnais et j'étais étonné, et je méditais dans mes pensées sur la multitude de bontés que les pécheurs qui sont sur la terre ont rejetées, et sur le grand tourment qu'ils ont méprisé, bien qu'ils savaient qu'ils devraient être tourmentés à cause de le péché qu'ils avaient commis. Et quand je réfléchissais à ces choses et autres, voilà ! l'ange Ramiel qui préside aux vraies visions m'a été envoyé, et il m'a dit :

\par 3 [...]

\par 4 'Pourquoi ton cœur te trouble-t-il, Baruch,

\par et pourquoi ta pensée te dérange-t-elle ?

\par 5 Car si, à cause de la nouvelle que vous avez seulement entendue du jugement, vous êtes ainsi ému,

\par Que (serez-vous) quand vous le verrez manifestement de vos yeux ?

\par 6 Et si, dans l'attente avec laquelle vous attendez le jour du Tout-Puissant, vous êtes tellement submergé,

\par Que (serez-vous) quand vous arriverez à son avènement ?

\par 7 Et si, à la parole de l'annonce du tourment de ceux qui ont fait des folies, vous êtes si complètement bouleversés,

\par Combien plus lorsque l'événement révélera des choses merveilleuses ?

\par 8 Et si vous avez entendu parler des bonnes et des mauvaises choses qui vont alors arriver et que vous en êtes attristés,

\par Que (serez-vous) quand vous verrez ce que la majesté révélera, qui convaincra ceux-ci et réjouira ceux-là.

\chapitre{56}

\par \textit{Interprétation de la Vision. Les eaux noires et brillantes symbolisent l'histoire du monde depuis Adam jusqu'à l'avènement du Messie.}


\par 1 Néanmoins, parce que vous avez supplié le Très-Haut de vous révéler l'interprétation de la vision que vous avez eue, j'ai été envoyé pour vous l'annoncer.

\par 2 Et le Tout-Puissant vous a assurément fait connaître les méthodes des temps passés et de ceux qui sont destinés à se passer dans son monde depuis le commencement de sa création jusqu'à sa consommation, de ces choses qui (sont ) tromperie et de ceux qui (sont) vrais.

\par 3 Car comme vous avez vu un grand nuage qui montait de la mer et allait couvrir la terre, telle est la durée du monde (= αιων) que le Tout-Puissant a faite lorsqu'il a pris conseil de créer le monde.

\par 4 Et il arriva, lorsque la parole fut sortie de sa présence, que la durée du monde fut créée dans une petite mesure, et fut établie selon la multitude de l'intelligence de Celui qui l'avait envoyé.

\par 5 Et comme vous avez vu précédemment au sommet du nuage les eaux noires qui descendaient auparavant sur la terre, c'est la transgression par laquelle Adam le premier homme a transgressé.

\par 6 Car [depuis] quand il a transgressé

\par La mort prématurée est apparue,

\par Grief a été nommé

\par Et l'angoisse fut préparée,

\par Et la douleur fut créée,

\par Et les ennuis consommés,

\par Et la maladie commença à s'établir,

\par Et le Shéol exigeait qu'il soit renouvelé par le sang,

\par Et la procréation des enfants fut réalisée,

\par Et la passion des parents produite,

\par Et la grandeur de l'humanité fut humiliée,

\par Et la bonté languissait.

\par 7 Qu'est-ce qui peut donc être plus noir ou plus sombre que ces choses ?

\par 8 C'est le début des eaux noires que vous avez vues.

\par 9 Et de ces eaux noires naquirent encore du noir, et les ténèbres des ténèbres furent produites.

\par 10 Car il est devenu un danger pour son âme, même pour les anges

\par 11 Car d'ailleurs, au temps où il fut créé, ils jouissaient de la liberté.

\par 12 Et étant devenu un danger, quelques-uns d'entre eux descendirent et se mêlèrent aux femmes.

\par 13 Et puis ceux qui faisaient cela étaient tourmentés et enchaînés.

\par 14 Mais le reste de la multitude des anges, dont il n'y a (aucun) nombre, se retint.

\par 15 Et ceux qui habitaient sur la terre périrent ensemble (avec eux) à cause des eaux du déluge.

\par 16 Ce sont les premières eaux noires.

\chapitre{57}

\par 1 Et après ces (eaux) vous avez vu des eaux claires : c'est la source d'Abraham, ainsi que ses générations et l'avènement de son fils, et du fils de son fils, et de ceux qui leur ressemblent.

\par 2 Parce qu'à cette époque-là, la loi non écrite était nommée parmi eux,

\par Et les œuvres des commandements furent alors accomplies,

\par Et la croyance au jugement à venir fut alors engendrée,

\par Et l'espérance du monde qui devait être renouvelé fut alors construite,

\par Et la promesse de la vie qui devait venir dans l'au-delà fut implantée.

\par 3 Ce sont les eaux claires que vous avez vues.

\chapitre{58}

\par 1 Et les troisièmes eaux noires que vous avez vues, ce sont le mélange de tous les péchés que les nations ont commis ensuite après la mort de ces hommes justes, et la méchanceté du pays d'Égypte, dans lequel ils ont commis le mal dans le service. avec quoi ils ont fait servir leurs fils.

\par 2 Néanmoins, ceux-là aussi périrent enfin.

\chapitre{59}

\par 1 Et les quatrièmes eaux brillantes que vous avez vues sont l'avènement de Moïse, d'Aaron, de Miriam et de Josué, fils de Noun et de Caleb, et de tous ceux qui leur ressemblent.

\par 2 Car en ce temps-là la lampe de la loi éternelle brillait sur tous ceux qui étaient assis dans les ténèbres, annonçant à ceux qui croient la promesse de leur récompense, et à ceux qui nient, le tourment du feu qui leur est réservé.

\par 3 Mais aussi les cieux à ce moment-là furent ébranlés de leur place, et ceux qui étaient sous le trône du Tout-Puissant furent troublés, lorsqu'il emmenait Moïse avec lui.

\par 4 Car il lui a montré de nombreux avertissements ainsi que les principes de la loi et la consommation des temps, comme aussi à vous, et également le modèle de Sion et ses mesures, selon le modèle duquel le sanctuaire du temps présent a été être fabriqué.

\par 5 Mais alors aussi Il lui montra les mesures du feu, et aussi la profondeur de l'abîme, et le poids des vents, et le nombre des gouttes de pluie :

\par 6 Et la répression de la colère, et la multitude de la longanimité, et la vérité du jugement :

\par 7 Et la racine de la sagesse, et les richesses de l'intelligence, et la source de la connaissance :

\par 8 Et la hauteur de l'air, et la grandeur du paradis, et la consommation des siècles, et le commencement du jour du jugement :

\par 9 Et le nombre des offrandes, et les terres qui ne sont pas encore venues :

\par 10 Et la bouche de la Géhenne, et le lieu de la vengeance, et le lieu de la foi, et la région de l'espérance ; et la ressemblance du tourment futur, et la multitude d'anges innombrables, et les armées flamboyantes, et la splendeur de les éclairs, et la voix des tonnerres, et les ordres des chefs des anges, et les trésors de lumière, et les changements des temps, et les investigations de la loi.

\par 11 [...]

\par 12 Ce sont les quatrièmes eaux brillantes que vous avez vues.

\chapitre{60}

\par 1 Et les cinquièmes eaux noires que vous avez vues pleuvoir sont les œuvres que les Amoréens faisaient, et les sortilèges de leurs incantations qu'ils faisaient, et la méchanceté de leurs mystères, et le mélange de leur pollution.

\par 2 Mais même Israël fut alors pollué par les péchés aux jours des juges, bien qu'ils voyaient de nombreux signes qui venaient de Celui qui les avait créés.

\chapitre{61}

\par 1 Et la sixième eau brillante qui a vu à travers, c'est le temps dans lequel David et Salomon sont nés.

\par 2 Et il y avait à cette époque la construction de Sion,

\par Et la dédicace du sanctuaire,

\par Et l'effusion de beaucoup de sang des nations qui péchèrent alors,

\par Et de nombreuses offrandes qui furent alors offertes lors de la dédicace du sanctuaire.

\par 3 Et la paix et la tranquillité existaient à cette époque,

\par 4 Et la sagesse fut entendue dans l'assemblée :

\par Et les richesses de l'intelligence furent magnifiées dans les assemblées,

\par 5 Et les saintes fêtes s'accomplissaient dans la bénédiction et dans une grande joie.

\par 6 Et on vit alors que le jugement des chefs était sans fraude,

\par Et la justice des préceptes du Tout-Puissant s'est accomplie avec vérité.

\par 7 Et le pays [qui] était alors aimé de l'Éternel,

\par Et parce que ses habitants n'avaient pas péché, elle fut glorifiée au-delà de tous les pays, et la ville de Sion régna alors sur tous les pays et toutes les régions.

\par 8 Ce sont les eaux claires que vous avez vues.

\chapitre{62}

\par 1 Et la septième eau noire que vous avez vue, c'est la perversion (provoquée) par le conseil de Jéroboam, qui a pris conseil de faire deux veaux d'or :

\par 2 Et toutes les iniquités commises iniquement par les rois qui furent après lui.

\par 3 Et la malédiction de Jézabel et le culte des idoles qu'Israël pratiquait à cette époque.

\par 4 Et le refus de la pluie, et les famines qui survenaient jusqu'à ce que les femmes mangent le fruit de leurs entrailles.

\par 5 Et le temps de leur captivité est arrivé aux neuf tribus et demie, parce qu'ils étaient dans de nombreux péchés.

\par 6 Et Salmanezsar, roi d'Assyrie, vint et les emmena captifs.

\par 7 Mais en ce qui concerne les Gentils, il était fastidieux de dire comment ils pratiquaient toujours l'impiété et la méchanceté, et n'accomplissaient jamais la justice.

\par 8 Ce sont les septièmes eaux noires que vous avez vues.

\chapitre{63}

\par 1 Et la huitième eau brillante que vous avez vue, c'est la rectitude et la droiture d'Ézéchias, roi de Juda, et la grâce (de Dieu) qui est descendue sur lui.

\par 2 Car quand Sennachérib était excité pour qu'il périsse, et que sa colère le troublait afin qu'il périsse ainsi, à cause aussi de la multitude des nations qui étaient avec lui.

\par 3 Quand, de plus, le roi Ézéchias apprit ce que le roi d'Assyrie projetait, c'est-à-dire venir le saisir et détruire son peuple, les deux tribus et demie qui restaient : bien plus, il souhaita renverser Sion aussi : alors Ezéchias eut confiance en ses œuvres et eut de l'espérance en sa justice, et il parla avec le Tout-Puissant et dit :

\par 4 « Voici, car voici ! Sennachérib est prêt à nous détruire, et il se vantera et s’élèvera lorsqu’il aura détruit Sion.

\par 5 Et le Tout-Puissant l'écouta, car Ezéchias était sage,

\par Et il eut du respect pour sa prière, parce qu'il était juste.

\par 6 Et là-dessus le Tout-Puissant commanda à Ramiel, son ange qui vous parle.

\par 7 Et je sortis et détruisis leur multitude, dont le nombre de chefs seulement était de cent quatre-vingt-cinq mille, et chacun d'eux avait un nombre égal (à ses ordres).

\par 8 Et en ce temps-là, je brûlai leurs corps à l'intérieur, mais je conservai leurs vêtements et leurs armes à l'extérieur, afin que les actions encore plus merveilleuses du Tout-Puissant puissent apparaître, et qu'ainsi son nom puisse être prononcé dans toute la terre entière. .

\par 9 Et Sion fut sauvée et Jérusalem délivrée : Israël aussi fut libéré de la tribulation.

\par 10 Et tous ceux qui étaient dans la terre sainte se réjouirent, et le nom du Tout-Puissant fut glorifié au point qu'on en parla.

\par 11 Ce sont les eaux claires que vous avez vues.

\chapitre{64}

\par 1 Et les neuvièmes eaux noires que vous avez vues, c'est toute la méchanceté qui était du temps de Manassé, fils d'Ézéchias.

\par 2 Car il a commis beaucoup d'impiété, et il a tué les justes, et il a arraché le jugement, et il a versé le sang des innocents, et il a épousé des femmes, il a violemment pollué, et il a renversé les autels, et il a détruit leurs offrandes, et il a chassé leurs prêtres, de peur qu'ils ne fassent le service dans le sanctuaire.

\par 3 Et il fit une image avec cinq visages : quatre d'entre eux regardaient aux quatre vents, et le cinquième au sommet de l'image comme un adversaire du zèle du Tout-Puissant.

\par 4 Et alors la colère sortit de la présence du Tout-Puissant dans l'intention de déraciner Sion, comme cela arriva aussi de votre temps. Mais aussi contre les deux tribus et demie fut publié un décret selon lequel elles seraient également emmenées captives, comme vous l'avez vu maintenant.

\par 5 Et l'impiété de Manassé augmenta à tel point qu'elle ôta du sanctuaire la louange du Très-Haut.

\par 6 [...]

\par 7 C'est pour cette raison que Manassé était alors appelé « l'impie », et finalement sa demeure fut dans le feu.

\par 8 Car bien que sa prière ait été exaucée par le Très-Haut, quand finalement, lorsqu'il fut jeté dans le cheval d'airain et que le cheval d'airain fut fondu, cela lui servit de signe pour l'heure.

\par 9 Car il n'avait pas vécu parfaitement, car il n'en était pas digne, mais pour savoir désormais par qui finalement il serait tourmenté.

\par 10 Car celui qui peut profiter est aussi capable de tourmenter.

\chapitre{65}

\par 1 C'est ainsi que Manassé a agi impiement, et a pensé qu'en son temps le Tout-Puissant ne s'enquérait pas de ces choses.

\par 2 Ce sont les neuvièmes eaux noires que vous avez vues.

\chapitre{66}

\par 1 Et la dixième eau brillante que tu as vue : c'est la pureté des générations de Josias, roi de Juda, qui était le seul à l'époque qui se soumit au Puissant de tout son cœur et de toute son âme. .

\par 2 Et il purifia le pays des idoles, et sanctifia tous les vases qui avaient été pollués, et remit les offrandes à l'autel, et leva la corne du saint, et exalta les justes, et honora tous ceux qui étaient sages en intelligence. , et ramena les prêtres à leur ministère, et détruisit et expulsa du pays les magiciens, les enchanteurs et les nécromanciens.

\par 3 Et non seulement il tua les impies qui étaient vivants, mais ils retirèrent aussi des sépulcres les ossements des morts et les brûlèrent au feu.

\par 4 [Et il établit les fêtes et les sabbats dans leur sainteté], et il brûla dans le feu ceux qui étaient impurs, et les prophètes menteurs qui séduisaient le peuple, ceux-là aussi il les brûla dans le feu, ainsi que le peuple qui écoutait. Quand ils étaient vivants, il les jeta dans le ruisseau du Cédron et il leur jeta des pierres.

\par 5 Et il était zélé de zèle pour le Puissant de toute son âme, et lui seul était ferme dans la loi en ce temps-là, de sorte qu'il ne laissait aucun incirconcis ou qui commettait l'impiété dans tout le pays, tout le monde. jours de sa vie.

\par 6 C'est pourquoi il recevra une récompense éternelle, et il sera glorifié avec le Tout-Puissant plus que beaucoup plus tard.

\par 7 Car c'est à cause de lui et à cause de ceux qui lui ressemblent que les gloires honorables dont on vous a parlé auparavant ont été créées et préparées. Ce sont les eaux brillantes que vous avez vues.

\chapitre{67}

\par 1 Et les onzièmes eaux noires que vous avez vues : c'est le malheur qui s'abat maintenant sur Sion.

\par 2 Pensez-vous qu'il n'y ait aucune angoisse pour les anges en présence du Tout-Puissant,

\par Que Sion a été ainsi livrée,

\par Et ça, voilà ! les païens se vantent dans leur cœur,

\par Et rassemblez-vous devant leurs idoles et dites :

\par « Celle qui a souvent été foulée aux pieds est foulée aux pieds,

\par Et elle a été réduite en servitude, celle qui a réduit (les autres) » ?

\par 3 Pensez-vous que le Très-Haut se réjouisse de ces choses,

\par Ou que Son nom soit glorifié ?

\par 4 [Mais comment cela servira-t-il à Son juste jugement ?]

\par 5 Mais après ces choses, ceux qui sont dispersés parmi les païens seront saisis par la tribulation,

\par Et ils habiteront dans la honte partout.

\par 6 Car autant que Sion est livrée

\par Et Jérusalem fut dévastée,

\par Les idoles prospéreront-elles dans les villes des païens,

\par Et la vapeur de la fumée de l'encens de la justice qui vient de la loi s'éteint dans Sion,

\par Et dans la région de Sion, partout, voici ! il y a la fumée de l'impiété.

\par 7 Mais se lèvera le roi de Babylone, qui a maintenant détruit Sion,

\par Et il se glorifiera auprès du peuple,

\par Et il dira de grandes choses dans son cœur, en présence du Très-Haut.

\par 8 Mais lui aussi tombera enfin. Ce sont les eaux noires.

\chapitre{68}

\par 1 Et la douzième eau lumineuse que vous avez vue : telle est la parole. Car après ces choses viendra un temps où votre peuple tombera dans la détresse, de sorte qu'il courra tous le risque de périr ensemble.

\par 2 [...]

\par 3 Néanmoins, ils seront sauvés, et leurs ennemis tomberont devant eux.

\par 4 Et ils auront, au moment voulu, beaucoup de joie.

\par 5 Et à ce moment-là, après un petit intervalle, Sion sera de nouveau reconstruite, et ses offrandes seront de nouveau rétablies, et les prêtres reviendront à leur ministère, et aussi les Gentils viendront la glorifier.

\par 6 Néanmoins, pas complètement comme au début.

\par 7 Mais il arrivera après ces choses qu'il y aura la chute de nombreuses nations.

\par 8 Ce sont les eaux claires que vous avez vues.

\chapitre{69}

\par 1 Car les dernières eaux que vous avez vues, qui étaient plus sombres que toutes celles qui étaient avant elles, celles qui étaient après le douzième nombre, qui ont été rassemblées, appartiennent au monde entier.

\par 2 Car le Très-Haut a fait la division dès le commencement, parce que Lui seul sait ce qui arrivera.

\par 3 Car quant aux énormités et aux impiétés qui devaient se produire devant lui, il en a prévu six sortes.

\par 4 Et parmi les bonnes œuvres des justes qui devaient être accomplies devant Lui, Il en a prévu six sortes, en plus de celles qu'Il devrait accomplir à la consommation des siècles.

\par 5 Pour lui, il n'y avait pas d'eaux noires avec des eaux noires, ni claires avec des eaux claires ; car c'est la consommation.

\chapitre{70}

\par 1 Écoutez donc l'interprétation des dernières eaux noires qui doivent venir [après les noires] : c'est là la parole.

\par 2 Voyez ! les jours viennent, et ils arriveront quand le temps du siècle sera mûr,

\par Et la moisson de ses mauvaises et bonnes graines est arrivée,

\par Que le Tout-Puissant fera venir sur la terre, sur ses habitants et sur ses dirigeants

\par Perturbation de l'esprit et stupeur du cœur.

\par 3 Et ils se haïront les uns les autres,

\par Et nous nous incitons à nous battre,

\par Et le médiocre règnera sur l'honorable,

\par Et ceux qui sont de bas rang seront exaltés au-dessus des célèbres.

\par 4 Et la plupart seront livrés entre les mains d'un petit nombre,

\par Et ceux qui n'étaient rien régneront sur les forts,

\par Et les pauvres auront une abondance plus grande que les riches,

\par Et les impies s'élèveront au-dessus des héroïques.

\par 5 Et les sages se tairont,

\par Et les insensés parleront,

\par La pensée des hommes ne sera alors pas non plus confirmée,

\par Ni le conseil des puissants,

\par L'espérance de ceux qui espèrent ne sera pas non plus confirmée.

\par 6 Et quand les choses qui étaient prédites furent arrivées,

\par Alors la confusion tombera sur tous les hommes,

\par Et certains d'entre eux tomberont au combat,

\par Et certains d'entre eux périront dans l'angoisse,

\par 7 Et certains d'entre eux seront détruits par les leurs. Puis les peuples Très-Hauts qu'Il a préparés d'avance,

\par Et ils viendront faire la guerre aux chefs qui resteront alors.

\par 8 Et il arrivera que quiconque sortira sain et sauf de la guerre mourra dans le tremblement de terre,

\par Et quiconque échappera au tremblement de terre sera brûlé par le feu,

\par Et quiconque échappera au feu sera détruit par la famine.

\par 9 [Et il arrivera que quiconque parmi les vainqueurs et les vaincus sera sauvé et échappera à toutes ces choses susmentionnées sera livré entre les mains de mon serviteur le Messie.]

\par 10 Car toute la terre dévorera ses habitants.

\chapitre{71}

\par 1 Et la terre sainte aura pitié d'elle-même, et elle protégera ses habitants en ce temps-là.

\par 2 Ceci est la vision que vous avez vue, et ceci en est l'interprétation.

\par 3 Car je suis venu vous annoncer ces choses, parce que votre prière a été entendue par le Très-Haut.

\chapitre{72}

\par 1 Écoutez maintenant aussi concernant l'éclair brillant qui doit arriver à la consommation après ces (eaux) noires : c'est la parole.

\par 2 Après que les signes dont on vous a parlé auparavant soient arrivés, lorsque les nations deviendront turbulentes et que le temps de mon Messie sera venu, il convoquera toutes les nations, et il en épargnera certaines, et quelques-unes d'entre elles. il les tuera.

\par 3 Ces choses arriveront donc aux nations qui doivent être épargnées par lui.

\par 4 Toute nation qui ne connaît pas Israël et qui n'a pas foulé aux pieds la postérité de Jacob sera en effet épargnée.

\par 5 Et cela parce que quelques-uns de chaque nation seront soumis à votre peuple.

\par 6 Mais tous ceux qui vous ont dominé ou qui vous ont connu seront livrés à l'épée.

\chapitre{73}

\par 1 Et il arrivera, quand il aura abaissé tout ce qui est dans le monde,

\par Et il s'est assis en paix pour toujours sur le trône de son royaume,

\par Cette joie sera alors révélée,

\par Et le repos apparaîtra.

\par 2 Et alors la guérison descendra dans la rosée,

\par Et la maladie se retirera,

\par Et l'inquiétude, l'angoisse et les lamentations passent parmi les hommes,

\par Et la joie parcourt toute la terre.

\par 3 Et personne ne mourra plus prématurément,

\par Et aucune adversité ne surviendra soudainement.

\par 4 Et les jugements, et les injures, et les querelles, et les vengeances,

\par Et le sang, et les passions, et l'envie, et la haine,

\par Et tout ce qui ressemble à cela sera condamné lorsqu'il sera enlevé.

\par 5 Car ce sont précisément ces choses qui ont rempli ce monde de maux,

\par Et à cause de cela, la vie de l'homme a été grandement troublée.

\par 6 Et les bêtes sauvages viendront de la forêt et serviront les hommes

\par Et les aspics et les dragons sortiront de leurs trous pour se soumettre à un petit enfant.

\par 7 Et les femmes n'éprouveront plus de douleur lorsqu'elles enfanteront,

\par Ils ne souffriront pas non plus de tourment lorsqu'ils donneront le fruit de leurs entrailles.

\chapitre{74}

\par 1 Et il arrivera en ces jours-là que les moissonneurs ne se lasseront pas,

\par Que ceux qui bâtissent ne soient pas fatigués par le travail ;

\par Car les travaux avanceront d'eux-mêmes rapidement

\par Ensemble avec ceux qui les font en toute tranquillité.

\par 2 Car ce temps est la consommation de ce qui est corruptible,

\par Et le commencement de ce qui n'est pas corruptible.

\par 3 C'est pourquoi les choses qui ont été prédites lui appartiendront :

\par C'est pourquoi elle est loin des maux et proche de ces choses qui ne meurent pas.

\par 4 C'est l'éclair brillant qui est venu après les dernières eaux sombres.

\chapitre{75}

\par \textit{Hymne de Baruch sur le caractère insondable des voies de Dieu et sur ses miséricordes par lesquelles les fidèles atteindront une plénitude parfaite}

\par 1 Et je répondis et dis :

\par 'Qui peut comprendre, ô Seigneur, Ta bonté ?

\par Car c'est incompréhensible.

\par 2 Ou qui peut sonder vos compassions,

\par Lesquels sont infinis ?

\par 3 Ou qui peut comprendre Ton intelligence ?

\par 4 Ou qui est capable de raconter les pensées de ton esprit ?

\par 5 Ou lequel de ceux qui sont nés peut espérer parvenir à ces choses,

\par A moins que ce soit quelqu'un envers qui vous êtes miséricordieux et gracieux ?

\par 6 Car, si assurément vous n'aviez pas compassion de l'homme,

\par Ceux qui sont sous Ta droite,

\par Ils ne pouvaient pas arriver à ces choses,

\par Mais ceux qui sont dans les numéros nommés peuvent être appelés.

\par 7 Mais si, en effet, nous qui existons, savons pourquoi nous sommes venus,

\par Et soumettons-nous à Celui qui nous a fait sortir d'Egypte,

\par Nous reviendrons et nous nous souviendrons de ces choses qui se sont passées,

\par Et je me réjouirai de ce qui a été.

\par 8 Mais si maintenant nous ne savons pas pourquoi nous sommes venus,

\par Et ne reconnaissez pas le principe de Celui qui nous a fait monter d'Egypte. Nous reviendrons et chercherons ce qui a été maintenant,

\par Et sois attristé par la douleur à cause de ce qui est arrivé.'

\chapitre{76}

\par \textit{Baruch a ordonné d'instruire le peuple pendant quarante jours et ensuite de se tenir prêt pour son Assomption lors de l'avènement du Messie}

\par 1 Et Il répondit et me dit : '[Dans la mesure où la révélation de cette vision t'a été interprétée comme tu l'as demandé], écoute la parole du Très-Haut afin que tu saches ce qui t'arrivera après ces choses.

\par 2 Car vous quitterez sûrement cette terre, néanmoins non pas jusqu'à la mort, mais vous serez préservés jusqu'à la consommation des temps.

\par 3 Montez donc au sommet de cette montagne, et là passeront devant vous toutes les régions de ce pays, et la figure du monde habité, et le(s) sommet(s) des montagnes, et la(les) profondeur(s) des vallées, et des profondeurs des mers, et du nombre des fleuves, afin que vous voyiez ce que vous partez et où vous allez.

\par 4 Or, cela arrivera après quarante jours. Allez maintenant, pendant ces jours-ci, et instruisez le peuple autant que vous le pouvez, afin qu'il apprenne à ne pas mourir à la dernière fois, mais qu'il apprenne afin de vivre à la dernière fois.

\chapitre{77}

\par \textit{L'exhortation de Baruch au peuple et l'écriture de deux lettres, l'une aux neuf tribus et demie d'Assyrie et l'autre aux deux tribus et demie de Babylone}


\par 1 Et moi, Baruch, j'y suis allé et je suis venu vers le peuple, je les ai rassemblés, du plus grand au plus petit, et je leur ai dit :

\par 2 'Écoutez, vous, enfants d'Israël, voyez combien vous êtes qui restent des douze tribus d'Israël.

\par 3 Car à vous et à vos pères, l'Éternel a donné une loi plus excellente qu'à tous les peuples.

\par 4 Et parce que vos frères ont transgressé les commandements du Très-Haut,

\par Il a exercé sa vengeance sur vous et sur eux,

\par Et Il n'a pas épargné les premiers,

\par Et ce dernier aussi, Il le livra en captivité :

\par Et Il n’en a laissé aucun résidu,

\par 5 Mais voici ! tu es ici avec moi.

\par 6 Si donc vous dirigez vos voies correctement,

\par Vous non plus ne partirez pas comme vos frères sont partis,

\par Mais ils viendront à vous.

\par 7 Car celui que vous adorez est miséricordieux,

\par Et Celui en qui vous espérez est miséricordieux,

\par Et Il est vrai, de sorte qu'Il fait le bien et non le mal.

\par 8 N'as-tu pas vu ici ce qui est arrivé à Sion ?

\par 9 Ou pensez-vous par hasard que le lieu a péché,

\par Et que c'est pour cette raison qu'il a été renversé ?

\par Ou que le pays avait commis des folies,

\par Et que c'est pour cela qu'il a été livré ?

\par 10 Et ne sais-tu pas qu'à cause de toi qui as péché,

\par Ce qui n'a pas péché a été renversé,

\par Et, à cause de ceux qui ont fait le mal,

\par Celui qui n'a pas commis de folie a été livré à (ses) ennemis ?'

\par 11 Et tout le peuple répondit et me dit : « Dans la mesure où nous pouvons nous souvenir des bonnes choses que le Tout-Puissant nous a faites, nous les rappelons ; et ces choses dont nous ne nous souvenons pas, il les connaît dans sa miséricorde.

\par 12 Néanmoins, faites ceci pour nous, votre peuple : écrivez aussi à nos frères à Babylone une épître de doctrine et un livre d'espérance, afin que vous puissiez aussi les confirmer avant de vous retirer de nous.

\par 13 Car les bergers d'Israël ont péri,

\par Et les lampes qui éclairaient s'éteignirent,

\par Et les sources ont retenu le ruisseau où nous buvions.

\par 14 Et nous sommes laissés dans les ténèbres,

\par Et au milieu des arbres de la forêt,

\par Et la soif du désert.

\par 15 Et je répondis et leur dis

\par Les bergers, les lampes et les fontaines viennent de la loi :

\par Et même si nous partons, la loi demeure.

\par 16 Si donc vous respectez la loi,

\par Et nous sommes attentifs à la sagesse,

\par Une lampe ne manquera pas,

\par Et un berger ne faillira pas,

\par Et une fontaine ne tarit pas.

\par 17 Néanmoins, comme tu me l'as dit, j'écrirai aussi à tes frères à Babylone, et j'enverrai par l'intermédiaire d'hommes, et j'écrirai de la même manière aux neuf tribus et demie, et j'enverrai par l'intermédiaire d'hommes. un oiseau.'

\par 18 Et il arriva, le vingtième jour du huitième mois, que moi, Baruch, je suis venu m'asseoir sous le chêne, à l'ombre des branches, et personne n'était avec moi, mais j'étais seul.

\par 19 Et j'ai écrit ces deux épîtres : l'une que j'ai envoyée par un aigle aux neuf tribus et demie ; et j'envoyai l'autre à ceux qui étaient à Babylone, au moyen de trois hommes.

\par 20 Et j'appelai l'aigle et je lui dis ces paroles :

\par 21 « Le Très-Haut t'a fait pour que tu sois plus haut que tous les oiseaux.

\par 22 Et maintenant, va et ne t'attarde pas dans un endroit, n'entre pas dans un nid, et ne t'installe sur aucun arbre, jusqu'à ce que tu aies parcouru la largeur des nombreuses eaux du fleuve Euphrate et que tu sois allé vers les gens qui habitent. là, et je leur jetai cette épître.

\par 23 Rappelez-vous encore qu'au temps du déluge, Noé reçut d'une colombe le fruit de l'olivier, lorsqu'il l'envoya hors de l'arche.

\par 24 Oui, les corbeaux aussi servaient Élie, lui apportant de la nourriture, comme on leur avait ordonné.

\par 25 Salomon aussi, au temps de son royaume, partout où il voulait envoyer ou chercher quelque chose, ordonnait à un oiseau (d'y aller), et il lui obéissait comme il le lui ordonnait.

\par 26 Et maintenant, que cela ne vous fatigue pas, et ne vous détournez ni à droite ni à gauche, mais fuyez et allez par un chemin direct, afin que vous conserviez le commandement du Tout-Puissant, comme je vous l'ai dit.

\chapitre{78}

\par \textit{L'ÉPISTRE DE BARUCH, FILS DE NÉRIAH, QU'IL A ÉCRIT AUX NEUF TRIBUS ET DEMI}

\par 1 Ce sont les paroles de cette épître que Baruc, fils de Nérija, envoya aux neuf tribus et demie qui étaient de l'autre côté du fleuve Euphrate, dans laquelle ces choses étaient écrites.

\par 2 Ainsi parle Baruc, fils de Nérija, aux frères emmenés en captivité : « Miséricorde et paix. » Je garde à l'esprit, mes frères, l'amour de Celui qui nous a créés, qui nous a aimés dès les temps anciens, et qui ne nous a jamais haïs, mais qui nous a surtout éduqués.

\par 3 Et en vérité je sais que voici, nous tous, les douze tribus, sommes liés par un seul lien, dans la mesure où nous sommes nés d'un seul père.

\par 4 C'est pourquoi j'ai pris d'autant plus soin de vous laisser les paroles de cette épître avant de mourir, afin que vous soyez consolés des maux qui vous sont arrivés, et que vous soyez aussi attristés du mal qui est arrivé à votre frères; et encore une fois, afin que vous puissiez justifier son jugement que

\par 5 Il a décrété contre vous que vous soyez emmenés captifs, car ce que vous avez souffert est disproportionné à ce que vous avez fait, afin qu'aux derniers temps vous soyez trouvés dignes de vos pères.

\par 6 Par conséquent, si vous considérez que vous avez maintenant souffert ces choses pour votre bien, afin que vous ne soyez pas finalement condamné et tourmenté, alors vous recevrez une espérance éternelle ; si surtout vous détruisez de votre cœur la vaine erreur, à cause de laquelle vous êtes partis d'ici.

\par 7 Car si vous faites ainsi ces choses, il se souviendra continuellement de vous, lui qui a toujours promis en notre faveur à ceux qui étaient plus excellents que nous, qu'il ne nous oublierait ni ne nous abandonnerait jamais, mais qu'il se rassemblerait de nouveau avec beaucoup de miséricorde. ceux qui ont été dispersés.

\chapitre{79}

\par 1 Maintenant, mes frères, apprenez d'abord ce qui est arrivé à Sion : comment Nabuchodonosor, roi de Babylone, s'est opposé à nous.

\par 2 Car nous avons péché contre Celui qui nous a créés, et nous n'avons pas gardé les commandements qu'il nous a prescrits, et pourtant il ne nous a pas châtiés comme nous le méritions.

\par 3 Pour ce qui vous est arrivé, nous souffrons aussi à un degré prééminent, car cela nous est également arrivé.

\chapitre{80}

\par 1 Et maintenant, mes frères, je vous fais savoir que lorsque l'ennemi eut encerclé la ville, les anges du Très-Haut furent envoyés, et ils renversèrent les fortifications de la muraille forte, et ils détruisirent les solides angles de fer, qui n'a pas pu être extirpé.

\par 2 Néanmoins, ils cachèrent tous les ustensiles du sanctuaire, de peur que l'ennemi ne s'en emparât.

\par 3 Et après avoir fait ces choses, ils livrèrent à l'ennemi le mur renversé, et la maison pillée, et le temple incendié, et le peuple qui était vaincu parce qu'il avait été livré, de peur que l'ennemi ne se vante et ne dise :

\par 4 'Ainsi, par la force, nous avons pu dévaster même la maison du Très-Haut dans la guerre.' Vos frères aussi ont été liés et emmenés à Babylone, et ils y ont fait demeurer.

\par 5 Mais nous sommes restés ici, étant très peu nombreux.

\par 6 C'est la tribulation au sujet de laquelle je vous ai écrit.

\par 7 Car assurément je sais que (la consolation) des habitants de Sion vous console : dans la mesure où vous saviez qu'elle prospérait, (votre consolation) était plus grande que la tribulation que vous avez endurée pour devoir vous en éloigner.

\chapitre{81}

\par 1 Mais concernant la consolation, écoutez la parole.

\par 2 Car j'étais en deuil au sujet de Sion, et j'ai imploré la miséricorde du Très-Haut, et j'ai dit :

\par 3 « Combien de temps ces choses dureront-elles pour nous ?

\par Et ces maux viendront-ils toujours sur nous ?

\par 4 Et le Tout-Puissant a agi selon la multitude de ses miséricordes,

\par Et le Très-Haut selon la grandeur de sa compassion,

\par Et Il m'a révélé la parole, afin que je reçoive une consolation,

\par Et Il m'a montré des visions pour que je n'endure plus l'angoisse,

\par Et Il m'a fait connaître le mystère des temps.

\par Et l'avènement des heures qu'il m'a montré.

\chapitre{82}

\par 1 C'est pourquoi, mes frères, je vous ai écrit, afin que vous vous consoliez de la multitude de vos tribulations.

\par 2 Car sachez que notre Créateur nous vengera assurément de tous nos ennemis, selon tout ce qu'ils nous ont fait, et que la consommation que fera le Très-Haut est très proche, et sa miséricorde qui vient, et la consommation de son jugement n’est en aucun cas loin.

\par 3 Pour voilà ! nous voyons maintenant la multitude de la prospérité des Gentils,

\par Bien qu'ils agissent impiement,

\par Mais ils seront comme une vapeur :

\par 4 Et nous contemplons la multitude de leur puissance,

\par Bien qu'ils fassent le mal,

\par Mais ils seront rendus semblables à une goutte :

\par 5 Et nous voyons la fermeté de leur puissance.

\par Bien qu'ils résistent au Tout-Puissant à chaque heure,

\par Mais ils seront considérés comme des crachats.

\par 6 Et nous considérons la gloire de leur grandeur,

\par Bien qu'ils n'observent pas les statuts du Très-Haut,

\par Mais ils passeront comme une fumée.

\par 7 Et nous méditons sur la beauté de leur grâce,

\par Bien qu'ils aient à voir avec des pollutions,

\par Mais comme l'herbe qui sèche, ils se faneront.

\par 8 Et nous considérons la force de leur cruauté,

\par Bien qu'ils ne se souviennent pas de la fin,

\par Mais comme une vague qui passe, ils seront brisés.

\par 9 Et nous remarquons la vantardise de leur puissance,

\par Bien qu'ils nient la bienfaisance de Dieu, qui la leur a donnée,

\par Mais ils passeront comme une nuée qui passe.

\chapitre{83}

\par 1 Car le Très-Haut hâtera assurément ses temps,

\par Et il fera certainement venir ses heures.

\par 2 Et il jugera assurément ceux qui sont dans son monde,

\par Et visitera en vérité toutes choses au moyen de toutes leurs œuvres cachées.

\par 3 Et il examinera assurément les pensées secrètes,

\par Et ce qui est déposé dans les chambres secrètes de tous les membres du courrier. Et il (les) manifestera devant tous avec réprobation.

\par 4 Qu'aucune de ces choses présentes ne monte donc dans vos cœurs, mais surtout attendons-nous, car ce qui nous est promis arrivera.

\par 5 Et ne regardons pas maintenant aux délices des Gentils dans le présent, mais rappelons-nous ce qui nous a été promis à la fin.

\par 6 Car les fins des temps et des saisons et tout ce qui les accompagne passeront assurément ensemble.

\par 7 De plus, la consommation du siècle montrera alors la grande puissance de son dirigeant, lorsque tout sera jugé.

\par 8 Préparez donc vos cœurs à ce en quoi vous croyiez avant, de peur que vous ne deveniez esclaves dans les deux mondes, de sorte que vous ne soyez emmenés captifs ici et que vous soyez tourmentés là.

\par 9 Car ce qui existe maintenant ou ce qui est passé, ou ce qui est à venir, dans toutes ces choses, ni le mal n'est pleinement mauvais, ni encore le bien pleinement bon.

\par 10 Car toutes les santés de ce temps se transforment en maladies,

\par 11 Et toute la puissance de ce temps se transforme en faiblesse,

\par Et toute la force de ce temps se transforme en impuissance,

\par 12 Et toute énergie de la jeunesse se transforme en vieillesse et en plénitude.

\par Et toute beauté de grâce de cette époque devient fanée et odieuse,

\par 13 Et toute domination orgueilleuse du présent se transforme en humiliation et en honte,

\par 14 Et toute louange de la gloire de ce temps se transforme en honte du silence,

\par Et toute vaine splendeur et insolence de ce temps se transforme en ruine muette.

\par 15 Et tous les délices et toutes les joies de ce temps se transforment en vers et en corruption,

\par 16 Et chaque clameur de l'orgueil de ce temps se transforme en poussière et en silence.

\par 17 Et toute possession de richesses de ce temps sera transformée en schéol seul,

\par 18 Et toute la rapine des passions de ce temps se transforme en mort involontaire,

\par Et chaque passion des convoitises de ce temps se transforme en un jugement de tourment.

\par 19 Et tout artifice et ruse de ce temps se transforme en preuve de la vérité,

\par 20 Et toute douceur des onguents de ce temps se transforme en jugement et en condamnation,

\par 21 Et tout amour du mensonge se transforme en mépris par la vérité.

\par 22 [Puisque donc toutes ces choses se font maintenant, est-ce que quelqu'un pense qu'ils ne seront pas vengés ? Mais la consommation de toutes choses arrivera à la vérité.]

\chapitre{84}

\par 1 Voyez ! Je vous ai donc fait connaître (ces choses) pendant que je vis : car je (cela) vous ai dit que vous devriez apprendre les choses qui sont excellentes ; car le Tout-Puissant m'a ordonné de vous instruire : et je mettrai devant vous quelques-uns des commandements de son jugement avant de mourir.

\par 2 Rappelez-vous qu'autrefois Moïse a certainement pris le ciel et la terre à témoin contre vous et a dit : 'Si vous transgressez la loi, vous serez dispersés, mais si vous l'observez, vous serez gardé.'

\par 3 Et il vous disait encore d'autres choses lorsque vous, les douze tribus, étiez ensemble dans le désert.

\par 4 Et après sa mort, vous les avez chassés loin de vous : c'est pourquoi ce qui avait été prédit vous est arrivé.

\par 5 Et maintenant Moïse vous le disait avant qu'ils ne vous arrivent, et voici ! ils vous sont arrivés, car vous avez abandonné la loi.

\par 6 Lo! Je vous dis aussi qu'après avoir souffert, si vous obéissez à ce qui vous a été dit, vous recevrez du Tout-Puissant tout ce qui vous a été réservé et réservé.

\par 7 De plus, que cette épître serve de témoignage entre moi et vous, afin que vous vous souveniez des commandements du Tout-Puissant, et qu'il y ait aussi pour moi une défense devant Celui qui m'a envoyé.

\par 8 Et souvenez-vous de la loi et de Sion, et de la terre sainte et de vos frères, et de l'alliance de vos pères, et n'oubliez pas les fêtes et les sabbats. Et remettez cette épître et les traditions de la loi à vos fils après vous, comme vos pères vous les ont transmis.

\par 9 [...]

\par 10 Et en tout temps, priez avec persévérance et priez diligemment de tout votre cœur afin que le Tout-Puissant se réconcilie avec vous et qu'il ne compte pas la multitude de vos péchés, mais qu'il se souvienne de la droiture de vos pères.

\par 11 Car s'il ne nous juge pas selon la multitude de ses miséricordes, malheur à nous tous qui sommes nés.

\chapitre{85}

\par 1 [Sachez d'ailleurs que

\par Autrefois et dans les générations anciennes, nos pères avaient des assistants,

\par Hommes justes et saints prophètes :

\par 2 Non plus, nous étions dans notre propre pays

\par [Et ils nous ont aidés quand nous avons péché],

\end{document}