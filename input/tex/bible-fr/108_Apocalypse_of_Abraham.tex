\begin{document}

\title{Apocalypse d'Abraham}

\chapter{1}

\par Le Livre de l'Apocalypse d'Abraham, fils de Terah, fils de Nahor, fils de Serug, fils de Roog (Reu), fils d'Arphaxad, fils de Sem, fils de Noé, fils de Lamec, fils de Mathusalem, fils d'Enoch, fils de Jared (Arad).

\par \textit{La conversion d'Abraham de l'idolâtrie <br> (Chapitres I.-VIII.).}

\par 1 Le jour où j'ai projeté les dieux de mon père Térach et les dieux de Nahor son frère, quand je cherchais qui est en vérité le Dieu Puissant, moi, Abraham, au moment où il m'est tombé dessus , quand j'ai accompli les services (les sacrifices) de mon père Térach envers ses dieux de bois et de pierre, d'or et d'argent, d'airain et de fer ; étant entré dans leur temple pour le service, j'ai trouvé le dieu dont le nom était Merumath (qui était) taillé dans la pierre, tombé en avant aux pieds du dieu de fer Nahon. Et il arriva que, quand je vis cela, mon cœur fut perplexe, et je réfléchis dans mon esprit que je ne pourrais pas le ramener à sa place, moi, Abraham, seul, parce qu'il était lourd, étant d'un grosse pierre, et je sortis et je le fis connaître à mon père. Et il entra avec moi, et comme nous le faisions tous deux avancer (le dieu) pour le ramener à sa place, sa tête tomba de lui tandis que je le tenais encore par la tête. Et il arriva que lorsque mon père vit que la tête de Merumath était tombée de lui, il me dit : « Abraham ! Et j’ai dit : « Me voici. » Et il m’a dit : « Apportez-moi une hache, une des plus petites, de la maison. » Et je le lui ai apporté. Et il a taillé un autre Merumath dans une autre pierre, sans tête, et il a placé dessus la tête qui avait été renversée de Merumath, et il a brisé le reste de Merumath.

\chapter{2}

\par 1 Et il fit cinq autres dieux, et me les donna [et] m'ordonna de les vendre dehors, dans la rue de la ville. Et je sellai les ânes de mon père, je les plaçai dessus, et je me dirigeai vers l'auberge pour les vendre. Et voilà ! les marchands de Fandana en Syrie voyageaient avec des chameaux en direction de l'Égypte pour faire du commerce. Et j'ai parlé avec eux. Et l'un de leurs chameaux poussa un gémissement, et l'âne, effrayé, s'élança et bouleversa les dieux et trois d'entre eux furent brisés, et deux furent conservés. Et il arriva que, lorsque les Syriens virent que j'avais des dieux, ils me dirent : « Pourquoi ne nous as-tu pas dit [que tu avais des dieux ? Alors nous les aurions achetés] avant que l'âne n'entende le bruit du chameau, et ils n'auraient pas été perdus. Donnez-nous au moins les dieux qui restent, et nous vous donnerons le juste prix pour les dieux brisés, ainsi que pour les dieux qui ont été conservés. Car je me demandais dans mon cœur comment je pourrais apporter à mon père le prix d'achat, et les trois cassés, je les jetai dans l'eau de la rivière Gur, qui était à cet endroit, et ils s'enfoncèrent dans les profondeurs, et il n'y en avait plus.

\chapter{3}

\par 1 Alors que j'étais encore en chemin, mon cœur était perplexe au-dedans de moi, et mon esprit était distrait. Et j’ai dit dans mon cœur : [« Quelle mauvaise action est en train de faire mon père ? N'est-il pas plutôt le dieu de ses dieux, puisqu'ils naissent grâce à ses ciseaux, à ses tours et à sa sagesse, et n'est-il pas plutôt approprié qu'ils adorent mon père, puisqu'ils sont son œuvre ? Quelle est cette illusion de mon père dans ses œuvres ?] Voici, Merumath est tombé et n'a pas pu se relever dans son propre temple, et je ne pouvais pas non plus, par moi-même, le déplacer jusqu'à ce que mon père vienne, et que nous le déplacions tous les deux ; et comme nous étions ainsi trop faibles, sa tête tomba de lui, et il (c'est-à-dire mon père) la posa sur une autre pierre d'un autre dieu, qu'il avait faite sans tête. Et les cinq autres dieux furent mis en pièces depuis l'âne, qui ne purent ni s'aider eux-mêmes, ni blesser l'âne, parce qu'il les avait brisés en morceaux ; et leurs fragments brisés ne sont pas sortis de la rivière. Et je dis dans mon cœur : « S'il en est ainsi, comment Merumath, le dieu de mon père, ayant la tête d'une autre pierre et lui-même étant fait d'une autre pierre, peut-il sauver un homme, ou entendre la prière d'un homme et le récompenser ?

\chapter{4}

\par 1 Et pendant que je réfléchissais ainsi, j'arrivais à la maison de mon père ; et après avoir abreuvé l'âne et mis du foin pour lui, j'apportai l'argent et le remis entre les mains de mon père Terah. Quand il le vit, il fut heureux et dit : « Tu es béni, Abraham, de mes dieux, parce que tu as apporté le prix des dieux, afin que mon œuvre ne soit pas vaine. » Et je répondis et lui dis : « Écoute, ô mon père, Térach ! Bienheureux soient tes dieux, car tu es leur dieu, depuis que tu les as créés ; car leur bénédiction est la ruine, et leur pouvoir est vain. Ceux qui ne se sont pas aidés eux-mêmes, comment pourront-ils donc t'aider ou bénir ? J'ai été gentil avec toi dans cette affaire, car en utilisant mon intelligence, je t'ai apporté l'argent pour les dieux brisés. Et quand il entendit cette parole, il se mit en colère contre moi, parce que j'avais prononcé des paroles dures contre ses dieux.

\chapter{5}

\par 1 Cependant, après avoir réfléchi à la colère de mon père, je sortis ; [et après que je sois sorti, mon père s'écria en disant : « Abraham ! Et j’ai dit : « Me voici. » Et il dit : « Prends et ramasse les éclats du bois avec lequel j'ai fait des dieux en bois de pin avant ton arrivée ; et prépare-moi la nourriture du repas de midi. Et il arriva que, lorsque je ramassai les éclats de bois, je trouvai sous eux un petit dieu qui gisait parmi les broussailles à ma gauche, et sur son front était écrit : DIEU BARISAT. Et je n'ai pas informé mon père que j'avais trouvé le dieu en bois Barisat sous les copeaux. Et il arriva que, lorsque j'eus mis les éclats dans le feu, afin de pouvoir préparer à manger pour mon père, en sortant pour poser une question concernant la nourriture, je plaçai Barisat devant le feu allumé, en disant en menaçant de lui : « Fais bien attention, Barisat, [à ce que] le feu ne s'éteigne pas jusqu'à ce que je vienne ; Mais si elle meurt, soufflez dessus, afin qu'elle brûle de nouveau. Et je suis sorti et j’ai accompli mon objectif. Et en revenant, je trouvai Barisat tombé à la renverse, les pieds entourés de feu et horriblement brûlés. J’éclatai de rire et je me dis : « En vérité, ô Barisat, tu sais allumer le feu et cuisiner ! » Et il arriva que pendant que je parlais (ainsi) dans mon rire, lui (c'est-à-dire Barisat) fut progressivement brûlé par le feu et réduit en cendres. Et j'ai apporté la nourriture à mon père, et il a mangé. Et je lui ai donné du vin et du lait, et il s'est réjoui et a béni son dieu Merumath. Et je lui dis : « Ô père Terah, ne bénis pas ton dieu Merumath et ne le loue pas, mais loue plutôt ton dieu Barisat parce que, t'aimant davantage, il s'est jeté dans le feu pour cuire ta nourriture ! Et il m'a dit : «Et où est-il maintenant ?» [Et j’ai dit :] « Il est réduit en cendres dans la violence du feu et est réduit en poussière. » Et il dit : « Grande est la puissance de Barisat ! J'en ferai un autre aujourd'hui, et demain il préparera ma nourriture.

\chapter{6}

\par 1 Cependant, quand moi, Abraham, j'ai entendu de telles paroles de la part de mon père, j'ai ri dans mon esprit et j'ai soupiré dans le chagrin et dans la colère de mon âme, et j'ai dit : « Comment donc ce qui est fait par lui, fabriqué statues, serais-tu l'aide de mon père ? Ou bien le corps sera-t-il alors soumis à son âme, et l'âme à l'esprit, et l'esprit à la folie et à l'ignorance ! » Et j’ai dit : « Il convient une fois d’endurer le mal. Je dirigerai donc mon esprit vers ce qui est pur et je lui exposerai mes pensées. [Et] je répondis et dis : « Ô père Terah, quel que soit celui d’entre eux que tu loues comme un dieu, tu es insensé dans ton esprit. Voici, les dieux de ton frère Ora, qui se tiennent dans le saint temple, sont plus dignes d'honneur que ceux des tiens. Car voici Zucheus, le dieu de ton frère Oron, est plus digne d'honneur que ton dieu Merumath, parce qu'il est fait d'or, qui est très apprécié des gens, et quand il vieillira, il sera remodelé ; mais si votre dieu Merumath est changé ou brisé, il ne se renouvellera pas, car il est une pierre ; il en est de même du dieu Joavon [qui se tient avec Zucheus au-dessus des autres dieux : combien plus digne d'honneur est-il que le dieu Barisat, qui est fait de bois, alors qu'il est forgé d'argent ! Comment, grâce à l'adaptation de l'homme, est-il rendu précieux pour l'apparence extérieure ! Mais ton dieu Barisat, alors qu'il était encore, avant d'avoir été préparé, enraciné (? sur la terre et qu'il était grand et merveilleux avec la gloire des branches et des fleurs, tu l'as taillé avec la hache, et au moyen de ton art il a été transformé en dieu. Et voici, sa graisse est déjà desséchée et a péri, il est tombé de la hauteur jusqu'au sol, il est passé de la grande condition à la petitesse, et l'apparence de son visage a disparu, et il] Barisat lui-même est brûlé par le feu et réduit en cendres et n'est plus. » Et tu dis : « Aujourd'hui, j'en ferai un autre qui demain préparera ma nourriture ! » « Il a péri jusqu'à la destruction totale. »

\chapter{7}

\par 1 « Voici, le feu est plus digne d'honneur que toutes choses créées, car même ce qui n'y est pas soumis lui est soumis, et les choses facilement périssables sont ridiculisées par ses flammes. Mais l’eau est encore plus digne d’honneur, parce qu’elle vainc le feu et rassasie la terre. Mais même cela, je ne l'appelle pas Dieu, parce qu'il est soumis à la terre sous laquelle l'eau s'incline. Mais j’appelle la terre bien plus digne d’honneur, parce qu’elle domine la nature (et la plénitude) de l’eau. Mais même celle-ci (c'est-à-dire la terre), je ne l'appelle pas dieu, [parce que] elle aussi est desséchée par le soleil, [et] est distribuée à l'homme pour qu'elle soit labourée. [J'appelle le soleil plus digne d'honneur que la terre,] parce qu'il éclaire de ses rayons le monde entier et les différentes atmosphères. [Mais] même cela, je ne l'appelle pas dieu, parce que la nuit et par les nuages, sa course est obscurcie. Je n'appelle pas non plus la lune ou les étoiles comme des dieux, car elles obscurcissent aussi leur lumière la nuit en leur saison. [Mais] écoute [ceci], Terah mon père ; car je te ferai connaître le Dieu qui a tout fait, et non ceux que nous considérons comme des dieux. Qui est-il alors ? ou qu'est-ce qu'Il est ?

\par 2 Qui a cramoisi les cieux, et rendu le soleil doré, et la lune brillante, et avec elle les étoiles ;

\par 3 Et il a asséché la terre au milieu de grandes eaux,

\par 4 Et t'installe. . .[et m'a testé dans la confusion de mes pensées

\par 5 « Mais que Dieu se révèle à nous par lui-même !

\chapter{8}

\par 1 Et il arriva que pendant que je parlais ainsi à mon père Térah dans la cour de ma maison, la voix d'un Puissant descendit du ciel dans un éclat de nuée de feu, disant et criant : « Abraham, Abraham ! » Et j’ai dit : « Me voici. » Et Il dit : « Tu cherches dans l’intelligence de ton cœur le Dieu des Dieux et le Créateur : Je le suis : sors de chez ton père Térah, et sors de la maison, afin que toi aussi tu ne sois pas tué dans les péchés de la maison de ton père. Et je suis sorti. Et il arriva que lorsque je sortis, avant que je parviens à sortir devant la porte du tribunal, il y eut un bruit de tonnerre qui le brûla, ainsi que sa maison et tout ce qui se trouvait dans sa maison jusqu'au sol, quarante coudées.

\chapter{9}

\par \textit{Abraham reçoit un ordre divin d'offrir un sacrifice après quarante jours en guise de préparation à une révélation divine (Chapitre IX.; cf. Gen. XV.).}

\par 1 Alors une voix me parvint, parlant deux fois : « Abraham, Abraham ! Et j’ai dit : « Me voici ! » Et Il dit : « Voici, c’est moi ; n'ayez crainte, car je suis devant les mondes et un Dieu puissant qui a créé la lumière du monde. Je suis ton bouclier et je suis ton aide. Allez, prenez-moi une génisse de trois ans, une chèvre de trois ans, un bélier de trois ans, une tourterelle et un pigeon, et apportez-moi un pur sacrifice. Et dans ce sacrifice, je présenterai devant toi les siècles (à venir), et je te ferai connaître ce qui est réservé, et tu verras de grandes choses que tu n'as pas vues (jusqu'ici) ; parce que tu as aimé me rechercher, et je t'ai nommé mon ami. Mais abstiens-toi de toute forme de nourriture qui sort du feu, et de boire du vin, et de t'oindre d'huile, pendant quarante jours », et alors expose-moi le sacrifice que je t'ai commandé, dans le lieu que je te montrerai, sur une haute montagne, et là je te montrerai les âges qui ont été créés et établis, faits et renouvelés, par ma Parole, et je te ferai connaître ce qui s'y passera ceux qui ont fait le mal et (pratiqué) la justice dans la génération des hommes.

\chapter{10}

\par \textit{Abraham, sous la direction de l'Ange Jaoel, se rend au Mont Horeb, un voyage de quarante jours, pour offrir le sacrifice (Chapitres X.-XII.).}

\par 1 Et il arriva que, lorsque j'entendis la voix de Celui qui me disait de telles paroles, (et) je regardai ici et là et voilà ! il n'y avait plus de souffle d'homme, et mon esprit fut effrayé, et mon âme s'enfuit loin de moi, et je devins comme une pierre, et je tombai sur la terre, car je n'avais plus la force de me tenir debout sur la terre. Et tandis que j'étais encore couché, la face contre terre, j'entendis la voix du Saint qui parlait : « Va, Jaoel, et au moyen de mon Nom ineffable, relève-moi cet homme-là et fortifie-le (afin qu'il se rétablisse) de son tremblement. Et l'ange qu'il m'avait envoyé, sous la forme d'un homme, vint, me saisit par la main droite, me releva et me dit : « Lève-toi, [Abraham,] Ami de Dieu qui t'aime; que le tremblement de l'homme ne te saisisse pas ! Car, voilà ! Je suis envoyé vers toi pour te fortifier et te bénir au nom de Dieu, qui t'aime, le Créateur du céleste et du terrestre. Soyez sans crainte et hâtez-vous vers Lui. Je suis appelé Jaoel par Celui qui déplace ce qui existe avec moi sur la septième étendue du firmament, une puissance en vertu du Nom ineffable qui demeure en moi. Je suis celui qui a été donné pour réprimer, selon son commandement, l'attaque menaçante des êtres vivants des Chérubins les uns contre les autres, et pour enseigner à ceux qui le portent le chant de la septième heure de la nuit de l'homme. Je suis ordonné de retenir le Léviathan, car c'est à moi que sont soumises les attaques et les menaces de chaque reptile. [Je suis celui qui a été chargé de libérer l'Hadès, de détruire celui qui regarde les morts.] Je suis celui qui a été chargé d'incendier la maison de ton père avec lui, parce qu'il montrait du respect pour les morts (idoles). J'ai été envoyé pour te bénir maintenant, et le pays que l'Éternel, que tu as invoqué, a préparé pour toi, et c'est pour toi que j'ai parcouru mon chemin sur la terre. Lève-toi, Abraham ! Partez sans crainte ; soyez vraiment heureux et réjouissez-vous; et je suis avec toi ! Car l’honneur éternel t’a été préparé par l’Éternel. Allez, accomplissez les sacrifices commandés. Pour voilà ! J'ai été désigné pour être avec toi et avec la génération préparée (à naître) de toi ; et avec moi Michel te bénit pour toujours. Bon courage, partez !

\chapter{11}

\par 1 Et je me levai et vis celui qui m'avait saisi par ma main droite et m'avait mis sur mes pieds : et l'apparence de son corps était comme un saphir, et l'aspect de son visage comme de la chrysolite, et les cheveux de son la tête comme la neige, et le turban sur sa tête comme l'apparence d'un arc-en-ciel, et les vêtements de ses vêtements comme la pourpre ; et il avait dans sa main droite un sceptre d'or. Et il m'a dit : « Abraham ! Et je dis : « Me voici, ton serviteur. » Et il dit : « Que mon regard ne t’effraie pas, ni mon discours, afin que ton âme ne soit pas troublée. Viens avec moi et j'irai avec toi, jusqu'au sacrifice, visible, mais après le sacrifice, invisible à jamais. Soyez de bonne humeur et venez !

\chapter{12}

\par 1 Et nous sommes allés tous les deux ensemble, quarante jours et quarante nuits, et je n'ai pas mangé de pain ni bu d'eau, parce que ma nourriture était de voir l'ange qui était avec moi, et sa parole - c'était ma boisson . Et nous arrivâmes à la montagne de Dieu, le glorieux Horeb. Et j'ai dit à l'ange : « Chanteur de l'Éternel ! Lo! Je n’ai aucun sacrifice avec moi et je ne connais pas non plus l’emplacement d’un autel sur la montagne : comment puis-je apporter un sacrifice ? Et il m'a dit : «Regarde autour de toi !» Et j'ai regardé autour de moi, et voilà ! Nous étions suivis de tous les animaux sacrificiels prescrits : la génisse, la chèvre, le bélier, la tourterelle et le pigeon. Et l’ange m’a dit : « Abraham ! » J'ai dit : « Me voici. » Et il me dit : « Tous ceux-là, tu les massacres, et tu divises les animaux en deux, les uns contre les autres, mais les oiseaux ne se séparent pas ; et ('mais') donne aux hommes que je te montrerai, qui se tiennent près de toi, car ceux-ci sont l'autel sur la montagne, pour offrir un sacrifice à l'Éternel ; mais donne-moi la tourterelle et le pigeon, car je monterai sur les ailes de l'oiseau, pour te montrer au ciel, sur la terre, dans la mer, dans l'abîme et dans les enfers. , et dans le jardin d'Eden, et dans ses fleuves et dans la plénitude du monde entier et son cercle, tu contempleras tous.

\chapter{13}

\par \textit{Abraham accomplit le sacrifice, sous la direction de l'Ange, et refuse de se laisser détourner de son dessein par Azazel (Chapitres XIII.-XIV.).}

\par 1 Et j'ai tout fait selon le commandement de l'ange, et j'ai donné aux anges qui étaient venus vers nous les animaux divisés, mais l'ange a pris les oiseaux. Et j'ai attendu le sacrifice du soir. Et un oiseau impur vola sur les cadavres, et je le chassa. Et l'oiseau impur me parla et dit : « Que fais-tu, Abraham, sur les hauteurs saintes, où personne ne mange ni ne boit, et où il n'y a pas de nourriture humaine sur eux, mais qui consument tout par le feu, et (va) te brûler. Abandonne l'homme qui est avec toi, et fuis ; car si tu montes vers les hauteurs, ils te tueront. Et il arriva que lorsque je vis l'oiseau parler, je dis à l'ange : « Qu'est-ce que cela, mon seigneur ? Et il dit : «C'est de l'impiété, c'est Azazel.» Et il lui dit : « Honte à toi, Azazel ! Car le sort d'Abraham est dans les cieux, mais le tien est sur la terre. Parce que tu as choisi et aimé ceci pour la demeure de ton impureté, c'est pourquoi le Seigneur éternel et puissant a fait de toi un habitant sur la terre et à travers toi tout mauvais esprit de mensonge, et à travers toi la colère et les épreuves pour les générations d'impies. Hommes; car Dieu, l'Éternel et le Puissant, n'a pas permis que les corps des justes soient entre tes mains, afin qu'ainsi la vie des justes et la destruction des impurs soient assurées. Écoutez, mon ami, n'ayez plus honte de ma part. Car il ne t'a pas été donné de jouer le tentateur à l'égard de tous les justes. Éloignez-vous de cet homme ! Tu ne peux pas l'égarer, car il est un ennemi de toi et de ceux qui te suivent et aiment ce que tu veux. Car voici, le vêtement qui était à toi autrefois dans le ciel lui a été réservé, et la mortalité qui était la sienne t'a été transférée.

\chapter{14}

\par 1 L'ange m'a dit : [« Abraham ! Et je dis : « Me voici, ton serviteur. » Et il dit : « Sache désormais que l'Éternel t'a choisi, Celui que tu aimes ; prends bon courage et utilise cette autorité, autant que je te l'ordonne, contre celui qui calomnie la vérité ; ne devrais-je pas pouvoir faire honte à celui qui a dispersé sur la terre les secrets du ciel et s'est rebellé contre le Tout-Puissant ?] Dis-lui : « Sois toi le charbon ardent de la fournaise de la terre ; va, Azazel, dans les parties inaccessibles de la terre ; [car ton héritage doit (être) sur ceux qui existent avec toi étant né avec les étoiles et les nuages, avec les hommes dont tu es la part, et (qui) par ton être existent ; et ton inimitié est une justification. C'est pourquoi, par ta perdition, disparais de moi. Et je prononçai les paroles que l'ange m'avait apprises. Et il dit : « Abraham ! » Et j’ai dit : « Me voici, ton serviteur. »]

\par 2 Et l'ange me dit : « Ne lui réponds pas ; car Dieu lui a donné le pouvoir (lit. volonté) sur ceux qui lui répondent. [Et l'ange me parla une seconde fois et dit : « Maintenant plutôt, peu importe ce qu'il te dit, ne lui réponds pas, afin que sa volonté ne puisse pas libre cours en toi, parce que l'Éternel et le Puissant lui a donné du poids et du poids volonté; ne lui réponds pas. J'ai fait ce qui m'a été commandé par l'ange;] et quoi qu'il m'ait dit, je ne lui ai rien répondu du tout.

\chapter{15}

\par \textit{Abraham et l'Ange montent au ciel sur les ailes des oiseaux (Chapitres XV.-XVI.).}

\par 1 Et cela arriva lorsque le soleil se coucha, et voilà ! une fumée comme celle d'une fournaise. Et les anges qui avaient les parts du sacrifice montèrent du haut de la fournaise fumante. Et l'Ange me prit de la main droite et me plaça sur l'aile droite du pigeon, et se plaça sur l'aile gauche de la tourterelle, qui (les oiseaux) n'avait été ni abattue ni divisée. Et il m'a porté jusqu'aux bords du feu flamboyant [et nous sommes montés comme avec de nombreux vents vers le ciel qui était fixé à la surface. Et j'ai vu dans l'air, sur la hauteur vers laquelle nous sommes montés, une forte lumière qu'il était impossible de décrire, et voilà ! dans cette lumière, un feu brûlant pour les gens, beaucoup de gens d'apparence masculine, tous (constamment) changeant d'aspect et de forme, courant et se transformant, et adorant et criant avec un son de mots que je ne connaissais pas.

\chapter{16}

\par 1 Et je dis à l'Ange : « Pourquoi m'as-tu amené ici maintenant, parce que je ne peux plus voir maintenant, car je suis déjà devenu faible et mon esprit se retire de moi ? Et il me dit : « Reste près de moi ; n'ayez crainte ! Et Celui que tu vois venir droit vers nous avec une grande voix de sainteté, c'est l'Éternel qui t'aime ; mais tu ne peux pas voir Lui-même). Mais que ton esprit ne faiblit pas [à cause des grands cris], car je suis avec toi, je te fortifie.

\chapter{17}

\par \textit{Abraham, instruit par l'Ange, prononce le Chant Céleste et prie pour l'Illumination (Chapitre XVII.).}

\par 1 Et pendant qu'il parlait encore (et) voilà ! le feu vint contre nous tout autour, et une voix résonnait dans le feu comme la voix des grandes eaux, comme le bruit de la mer dans son tumulte. Et l'ange pencha la tête vers moi et adora. Et je désirais tomber sur la terre, et le haut lieu sur lequel nous nous tenions, [tantôt se redressait,] tantôt roulait vers le bas.

\par 2 Et il dit : « Adore seulement, Abraham, et prononce le chant que je t'ai enseigné ; » parce qu'il n'y avait pas de terre sur laquelle tomber. Et j'ai seulement adoré et j'ai prononcé le chant qu'il m'avait appris. Et il dit : « Récitez sans cesse. » Et j'ai récité, et lui aussi avec moi a récité la chanson :

\par 3 Éternel, puissant, saint, El,
\par     Dieu seul – Suprême !
\par 4 Toi qui es originaire de toi-même, incorruptible, sans tache,
\par 5 Incréé, immaculé, immortel,
\par     Auto-complet, auto-éclairant ;
\par 6 Sans père, sans mère, inengendré, Exalté, ardent !
\par 7 Amoureux des hommes, bienveillant, généreux,
\par 8 jaloux de moi et très compatissant ;
\par 9 Eli, c'est-à-dire mon Dieu—
\par 10 Éternel, puissant et saint Sabaoth,
\par 11 très glorieux El, El, El, El, Jaoel !
\par 12 Tu es Celui que mon âme a aimé !
\par 13 Protecteur éternel, brillant comme le feu,
\par 14 Dont la voix est comme le tonnerre,
\par 15 Dont le regard est comme l'éclair, qui voit tout,
\par 16 Qui reçoit les prières de ceux qui t'honorent !
\par 17 [Et se détourne des demandes de ceux qui sont embarrassés par l'embarras de leurs provocations,
\par 18 Qui dissout les confusions du monde qui surgissent des impies et des justes dans l'ère corruptible, renouvelant l'ère des justes !]
\par 19 Toi, ô Lumière, tu brilles devant la lumière du
\par 20 matin sur tes créatures,
\par 21 [afin qu'il devienne jour sur la terre,]
\par 22 Et dans tes demeures célestes, il n'y a pas
\par     besoin de toute autre lumière
\par 23 que (cela) de la splendeur indescriptible du
\par     lumières de ton visage.
\par 24 Acceptez ma prière [et soyez-en satisfait],
\par 25 de même aussi le sacrifice que tu as préparé
\par     Toi par moi qui t'ai cherché !
\par 26 Accepte-moi favorablement, montre-moi et enseigne-moi,
\par 27 Et fais connaître à ton serviteur ce que tu me as promis !

\chapter{18}

\par \textit{Vision d'Abraham du Trône Divin (Chapitre XVIII.).}

\par 1 Et tandis que je récitais encore le chant, la bouche du feu qui était à la surface s'élevait en haut. Et j'entendis une voix comme le rugissement de la mer ; il ne cessa pas non plus à cause de la riche abondance du feu. Et tandis que le feu s'élevait, montant vers la hauteur, je vis sous le feu un trône de feu et, tout autour, des êtres qui voyaient tout, récitant le chant, et sous le trône quatre êtres vivants enflammés chantant, et leur apparition il y en avait un, chacun d'eux avec quatre visages. Et telle était l'apparence de leurs visages, d'un lion, d'un homme, d'un bœuf, d'un aigle : quatre têtes [étaient sur leurs corps] [de sorte que les quatre créatures avaient seize faces] et chacune avait six ailes ; de leurs épaules, [et de leurs côtés] et de leurs reins. Et avec les (deux) ailes de leurs épaules, ils se couvraient le visage, et avec les (deux) ailes qui sortaient de leurs reins, ils se couvraient les pieds, tandis que les (deux) ailes du milieu étaient déployées pour voler droit vers l'avant. Et quand ils eurent fini de chanter, ils se regardèrent et se menacèrent. Et il arriva que lorsque l'ange qui était avec moi vit qu'ils se menaçaient l'un l'autre, il me quitta et courut vers eux et détourna la face de chaque être vivant de celle qui lui faisait face immédiatement, afin qu'ils ne voient pas leurs visages se menaçaient. Et il leur enseigna le chant de paix qui a son origine [dans l'Éternel].

\par 2 Et comme j'étais seul et que je regardais, je vis derrière les êtres vivants un char avec des roues enflammées, chaque roue étant pleine d'yeux tout autour ; et au-dessus des roues il y avait un trône ; ce que j'ai vu, et celui-ci était couvert de feu, et le feu l'entourait tout autour, et voilà ! un feu indescriptible entourait une armée ardente. Et j'entendis sa voix sainte comme la voix d'un homme.

\chapter{19}

\par \textit{Dieu révèle à Abraham les puissances du Ciel (Chapitre XIX.).}

\par 1 Et une voix me vint du milieu du feu, disant : « Abraham, Abraham ! J'ai dit : « Me voici ! » Et Il dit : « Considére les étendues qui sont sous le firmament sur lequel tu es (maintenant) placé, et vois comment sur aucune étendue il n’y a d’autre que Celui que tu as cherché ou qui t’a aimé. » Et pendant qu’Il ​​parlait encore (et) voilà ! les étendues s'ouvrirent, et sous moi les cieux. Et je vis sur le septième firmament sur lequel je me tenais un feu largement étendu, de la lumière et de la rosée, et une multitude d'anges, et une puissance de gloire invisible sur les créatures vivantes que je voyais ; mais je n'y ai vu aucun autre être.

\par 2 Et j'ai regardé depuis la montagne dans laquelle je me tenais [vers le bas] jusqu'au sixième firmament, et j'ai vu là une multitude d'anges, d'esprit (pur), sans corps, qui exécutaient les commandements des anges de feu qui étaient sur le huitième firmament, alors que j'étais suspendu au-dessus d'eux. Et voici, sur ce firmament, il n'y avait aucune autre puissance d'une autre forme, mais seulement des anges d'esprit (pur), comme la puissance que j'ai vue au septième firmament. Et Il ordonna que le sixième firmament soit supprimé. Et j'ai vu là, sur le cinquième firmament, les puissances des étoiles qui exécutent les commandements qui leur étaient imposés, et les éléments de la terre leur obéissaient.

\chapter{20}

\par \textit{La promesse d'une graine (Chapitre XX.).}

\par 1 Et l'Éternel Puissant me dit : « Abraham, Abraham ! Et j’ai dit : « Me voici. » [Et Il dit :] « Considérez d’en haut les étoiles qui sont au-dessous de toi, et comptez-les [pour moi], et faites-moi connaître leur nombre. » Et j’ai dit : « Quand puis-je ? Car je ne suis qu’un homme [de poussière et de cendre].» Et il me dit : « Comme le nombre des étoiles et leur puissance, je ferai de ta postérité une nation et un peuple, qui me seront réservés dans mon héritage avec Azazel. »

\par 2 Et je dis : « Ô Éternel, Puissant ! Que ton serviteur parle devant toi, et que ta colère ne s'allume pas contre ton élu ! Voici, avant que tu me fasses monter, Azazel s'est indigné contre moi. Comment donc, alors qu’il n’est pas devant toi, t’es-tu constitué avec lui ?

\chapter{21}

\par \textit{Une vision du péché et du paradis : le miroir du monde (Chapitre XXI.).}

\par 1 Et Il me dit : « Maintenant, regarde sous tes pieds les firmaments et comprends la création annoncée dans cette étendue, les créatures qui y existent et l'ère préparée selon elle. » Et je vis sous [la surface des pieds, et je vis sous] le sixième ciel et ce qu'il contenait, puis la terre et ses fruits, et ce qui se mouvait sur elle et ses êtres animés ; et la puissance de ses hommes, et l'impiété de leurs âmes, et leurs bonnes actions [et les débuts de leurs œuvres], et les régions inférieures et la perdition qui s'y trouve, l'abîme et ses tourments. J'y vis la mer et ses îles, et ses monstres et ses poissons, et Léviathan et sa domination, et son terrain de camping, et ses grottes, et le monde qui reposait sur lui, et ses mouvements, et les destructions du monde sur son compte. J'y ai vu des ruisseaux et la montée de leurs eaux, et leurs détours. Et j'y ai vu le jardin d'Eden et ses fruits, la source du ruisseau qui en sortait, et ses arbres et leurs fleurs, et ceux qui se comportaient bien. Et j'y ai vu leurs aliments et leur bénédiction. Et j'y ai vu une grande multitude d'hommes, de femmes et d'enfants [dont la moitié à droite de l'image] et l'autre moitié à gauche de l'image.

\chapter{22}

\par \textit{La Chute de l'Homme et ses suites (Chapitres XXIL-XXV.).}

\par 1 Et je dis : « Ô Éternel, Puissant ! Quelle est cette image des créatures ? Et Il me dit : « Telle est ma volonté à l'égard de ceux qui existent dans le conseil du monde (divin), et cela m'a semblé agréable à mes yeux, puis ensuite je leur ai donné des commandements par ma Parole. Et il arriva que tout ce que j'avais décidé d'être était déjà prévu dans cette (image), et cela se tenait devant moi avant d'être créé, comme tu l'as vu.

\par 2 Et je dis : « Ô Seigneur, puissant et éternel ! Qui sont les personnes sur cette photo, d’un côté et de l’autre ? Et Il me dit : « Ceux qui sont du côté gauche sont la multitude des peuples qui ont existé autrefois et qui sont destinés après toi, les uns au jugement et à la restauration, et d'autres pour la vengeance et la destruction à la fin du monde. Mais ceux qui sont du côté droit de l’image, ce sont les gens qui m’ont été mis à part parmi les peuples avec Azazel. Ce sont eux que j’ai ordonné de naître de toi et d’être appelés mon peuple.

\chapter{23}

\par 1 «Maintenant, regarde encore dans l'image, qui est celui qui a séduit Ève et quel est le fruit de l'arbre, [et] tu sauras ce qu'il y aura, et ce qui arrivera à ta postérité parmi le peuple à l'époque fin des jours du monde, et ce que tu ne peux pas comprendre, je te le ferai connaître, car tu es agréable à mes yeux, et je te dirai ce qui est gardé dans mon cœur.

\par 2 Et j'ai regardé le tableau, et mes yeux se sont tournés vers le côté du jardin d'Eden. Et j'y vis un homme très grand en taille et effrayant en largeur, d'aspect incomparable, embrassant une femme, qui se rapprochait également de l'aspect et de la forme de l'homme. Et ils se tenaient sous un arbre du (jardin d') Éden, et le fruit de cet arbre ressemblait à l'apparence d'une grappe de raisin de la vigne, et derrière l'arbre se tenait comme s'il s'agissait d'un serpent, ayant des mains et des pieds comme ceux d'un homme, et des ailes sur les épaules, six à droite et six à gauche, et ils tenaient les raisins de l'arbre dans leurs mains, et tous deux le mangeaient, ceux que j'avais vu embrasser.

\par 3 Et je dis : « Qui sont ceux-là qui s'embrassent mutuellement, ou qui est celui qui est entre eux, ou quel est le fruit dont ils mangent, ô Puissant Éternel ?

\par 4 Et Il dit : « Ceci est le monde humain, ceci est Adam, et tel est leur désir sur la terre, ceci est Ève ; mais celui qui est entre eux représente l'impiété, leur commencement (sur le chemin) vers la perdition, même Azazel.

\par 5 Et je dis : « Ô Éternel, Puissant ! Pourquoi as-Tu donné un tel pouvoir pour détruire la génération des hommes dans leurs œuvres sur la terre ?

\par 6 Et Il me dit : « Ceux qui veulent (faire) le mal — et combien je le haïssais chez ceux qui le font ! Je lui ai donné du pouvoir sur eux et d'être aimé d'eux.

\par 7 Et je répondis et dis : « Ô Éternel, Puissant ! Pourquoi as-tu voulu faire en sorte que le mal soit désiré dans le cœur des hommes, puisque tu es en effet irrité à cause de ce que tu as voulu, contre celui qui fait ce qui n'est pas rentable dans ton conseil ?

\chapter{24}

\par 1 Et Il me dit : « Etant irrité contre les nations à cause de toi et à cause des gens de ta famille qui doivent (être) séparés après toi, comme tu vois sur l'image le fardeau (du destin) cela (qui leur est imposé) - et je te dirai ce qui arrivera et combien il y aura dans les derniers jours. Regardez maintenant tout ce qui est sur la photo.

\par 2 Et j'ai regardé et j'ai vu là ce qui était devant moi dans la création ; J'ai vu Adam et Ève existant avec lui, et avec eux l'Adversaire rusé, et Caïn qui a agi illégalement à travers l'Adversaire, et Abel massacré, (et) la destruction apportée et causée sur lui à travers l'Adversaire. J'y ai vu aussi l'impureté, et ceux qui la convoitent, et sa pollution, et leur jalousie, et le feu de leur corruption dans les parties les plus basses de la terre. J'y ai vu le Vol, et ceux qui s'y précipitent, et l'arrangement [de leur châtiment, le jugement de la Grande Assise]. J'y ai vu des hommes nus, les fronts les uns contre les autres, et leur disgrâce, et leur passion qu'ils avaient les uns contre les autres, et leur châtiment. J'y ai vu le Désir, et dans sa main la tête de toute sorte d'anarchie [et son mépris et ses déchets voués à la perdition].

\chapter{25}

\par 1 J'y vis l'image de l'idole de la jalousie, ayant l'apparence d'une boiserie telle que mon père avait l'habitude de la faire, et sa statue était d'airain étincelant ; et devant lui un homme, et il l'adorait ; et devant lui un autel, et dessus un garçon tué en présence de l'idole.

\par 2 Mais je lui dis : « Quelle est cette idole, ou qu'est-ce que l'autel, ou qui sont ceux qui sont sacrifiés, ou qui est celui qui sacrifie ? Ou quel est le Temple que je vois qui soit beau en art, et sa beauté (ressemblant à) la gloire qui repose sous ton trône ?

\par 3 Et Il dit : « Écoute, Abraham. Ce que tu vois, le Temple, l'autel et la beauté, est mon idée du sacerdoce de mon glorieux Nom, dans lequel réside chaque prière de l'homme, et la montée des rois et des prophètes, et tout sacrifice que j'ordonne de m'offrir parmi mon peuple qui sortira de ta génération. Mais la statue que tu as vue est ma colère, avec laquelle m'irritent les gens qui doivent partir de toi pour moi. Mais l'homme que tu as vu égorger, c'est celui qui incite aux sacrifices meurtriers, dont (sic) sont pour moi un témoin du jugement dernier, même au commencement de la création.

\chapter{26}

\par \textit{Pourquoi le péché est permis (Chapitre XXVI.).}

\par 1 Et je dis : « Ô Éternel, Puissant ! Pourquoi as-tu établi qu’il en était ainsi, et ensuite en a-t-il proclamé la connaissance ?

\par 2 Et Il me dit : « Écoute, Abraham ; comprends ce que je te dis, et réponds-moi quand je t'interroge. Pourquoi ton père Térah n'a-t-il pas écouté ta voix, et (pourquoi) n'a-t-il pas renoncé à l'idolâtrie diabolique jusqu'à ce qu'il périsse [et] toute sa maison avec lui ?

\par 3 Et je dis : « Ô Éternel, [Puissant] ! (C'était) entièrement parce qu'il n'avait pas choisi de m'écouter ; mais moi non plus, je n’ai pas suivi ses œuvres.

\par 4 Et Il [me dit] : « Écoute, Abraham. De même que le conseil de ton père est en lui, et comme ton conseil est en toi, de même le conseil de ma volonté est en moi prêt pour les jours à venir, avant que tu en aies connaissance, ou que tu ne puisses voir de tes yeux ce que est l'avenir en eux. Comment seront ceux de ta postérité, regarde sur l’image.

\chapter{27}

\par \textit{Une vision du jugement et du salut (Chapitre XXVII.).}

\par 1 Et j'ai regardé et j'ai vu : voilà ! le tableau vacilla et [de lui] sortit, sur son côté gauche, un peuple païen, et ils pillèrent ceux qui étaient sur le côté droit, hommes, femmes et enfants : [ils égorgeèrent certains,] ils en retinrent d'autres avec eux. Lo! Je les ai vus courir vers eux par quatre entrées, et ils ont incendié le Temple, et ils ont pillé les choses saintes qui s'y trouvaient.

\par 2 Et je dis : « Ô Éternel ! Lo! les gens (qui naissent) de Moi, que Tu as acceptés, les hordes de païens pillent, et certains sont tués, tandis que d'autres sont retenus comme des étrangers, et ils ont brûlé le Temple au feu, et les belles choses qui s'y trouvent volez [et détruisez]. Ô Éternel, Puissant ! S’il en est ainsi, pourquoi m’as-tu maintenant lacéré le cœur, et pourquoi devrait-il en être ainsi ?

\par 3 Et Il me dit : « Écoute, Abraham. Ce que tu as vu arrivera à cause de ta postérité qui m'a irrité à cause de la statue que tu as vue, et à cause du massacre humain dans le tableau, à cause du zèle dans le Temple ; et comme tu l’as vu, il en sera ainsi.

\par 4 Et je dis : « Ô Éternel, Puissant ! Que les œuvres du mal (effectuées) dans l'impiété passent maintenant, mais (montrez-moi) plutôt ceux qui ont accompli les commandements, même les œuvres de sa (?) justice. Car tu peux faire cela.

\par 5 Et Il me dit : « Le temps des justes les rencontre d'abord grâce à la sainteté (qui découle) des rois et des dirigeants justes que j'ai d'abord créés pour que de tels hommes régnent parmi eux. Mais de là sortent des hommes soucieux de leurs intérêts, comme je te l'ai fait savoir et comme tu l'as vu.

\chapter{28}

\par \textit{Combien de temps ? (Chapitres XXVIII.-XXIX.).}

\par 1 Et je répondis et dis : « Ô Puissant, [Éternel] sanctifié par Ta puissance ! Soyez favorable à ma requête, [car c'est pour cela que tu m'as fait monter ici — et montre-le-moi]. De même que tu m'as élevé à ta hauteur, fais-le-moi savoir, ton bien-aimé, autant que je le demande : si ce que j'ai vu leur arrivera longtemps ?

\par 2 Et Il me montra une multitude de Son peuple, et me dit : « À cause d'eux, à travers quatre questions, comme tu l'as vu, je serai irrité par eux, et dans celles-ci mon châtiment pour leurs actes sera (accompli) . Mais dans la quatrième période de cent ans et une heure de l’âge – cela vaut cent ans – il y aura du malheur parmi les païens [mais une heure dans la miséricorde et le mépris, comme parmi les païens].”

\chapter{29}

\par 1 Et j'ai dit : « Ô Éternel [Puissant] !? Et combien de temps dure une heure de l’Âge ?

\par 2 Et Il dit : « J'ai ordonné à cet âge impie de régner sur les païens et sur ta postérité pendant douze ans ; et jusqu'à la fin des temps, il en sera comme tu l'as vu. Et compte, comprends et regarde le tableau.

\par 3 Et j'ai [regardé et] j'ai vu un homme sortir du côté gauche des païens ; et là sortirent des hommes, des femmes et des enfants, du côté des païens, une multitude d'armées, et ils l'adorèrent. Et pendant que je regardais encore, du côté droit (beaucoup) sont sortis, et certains ont insulté cet homme, tandis que d'autres l'ont frappé ; d’autres, cependant, l’adoraient. [Et] j'ai vu comment ceux-ci l'adoraient, et Azazel courut et l'adora, et après lui avoir baisé le visage, il se tourna et se tint derrière lui.

\par 4 Et je dis : « Ô Etemal, Puissant ! Quel est l’homme insulté et battu, qui est adoré par les païens avec Azazel ?

\par 5 Et Il répondit et dit : « Écoute, Abraham ! L'homme que tu as vu insulté et battu et à nouveau adoré, quel est le soulagement ? (accordé) par les païens au peuple qui procède de toi, dans les derniers jours, en cette douzième heure de l'ère de l'impiété. Mais la douzième année de mon âge final, j'établirai cet homme de ta génération, que tu as vu (issu) de mon peuple ; celui-ci tous le suivront, et ceux que j'appellerai se joindront à eux, (même) ceux qui changent de conseil. Et ceux que tu as vu émerger du côté gauche de l'image, la signification est la suivante : Il y en aura beaucoup parmi les païens qui placeront leurs espoirs en lui ; et quant à ceux que tu as vus de ta postérité du côté droit, les uns l'insultant et le frappant, les autres l'adorant, beaucoup d'entre eux seront offensés contre lui. Cependant, il teste ceux de ta postérité qui l'ont adoré, à cette douzième heure de la Fin, en vue d'abréger l'ère de l'impiété.

\par 6 « Avant que l'ère des justes ne commence à croître, mon jugement s'abattra sur les païens sans loi à travers le peuple de ta postérité qui a été séparé pour moi. En ces jours-là, je ferai venir sur toutes les créatures de la terre dix fléaux, à cause du malheur, de la maladie et des soupirs de chagrin de leur âme. J'infligerai tant de choses aux générations d'hommes qui s'y trouvent à cause de la provocation et de la corruption de ses créatures, par lesquelles elles me provoquent. Et alors il restera des hommes justes de ta postérité dans le nombre que j'ai gardé secret, se hâtant dans la gloire de mon Nom vers le lieu préparé d'avance pour eux, que tu as vu dévasté sur le tableau ; et ils vivront et seront affermis par des sacrifices et des dons de justice et de vérité dans l'ère des justes, et se réjouiront continuellement en Moi ; et ils détruiront ceux qui les ont détruits, et ils insulteront ceux qui les ont insultés. »

\par 7 « Et ceux qui les ont diffamés, ils cracheront au visage, méprisés par moi, tandis qu'ils (les justes) me verront plein de joie, se réjouissant avec mon peuple et recevant ceux qui reviennent à moi [dans la repentance] .»

\par 8 «Vois, Abraham, ce que tu as vu, et écoute ce que tu as entendu, et [prends pleinement connaissance de] ce que tu as appris. Va vers ton héritage, Et voilà ! Je suis avec toi pour toujours.

\chapter{30}

\par \textit{Le châtiment des païens et le rassemblement d'Israël (Chapitres XXX.-XXXI.).}

\par 1 Mais pendant qu'il parlait encore, je me trouvai sur la terre. Et je dis : « Ô Éternel, [Puissant], je ne suis plus dans la gloire dans laquelle j’étais (lorsque) là-haut, et ce que mon âme aspirait à comprendre dans mon cœur, je ne le comprends pas. »

\par 2 Et Il me dit : « Ce que tu désires dans ton cœur, je te le dirai, parce que tu as cherché à voir les dix plaies que j'ai préparées pour les païens et que j'ai préparées d'avance au passage de la douzième heure de la terre. Écoute ce que je te divulgue, ainsi cela arrivera : la première (est) une douleur de grande détresse ; la seconde, l'incendie de nombreuses villes ; le troisième, destruction et peste des animaux ; le quatrième, la faim du monde entier et de ses habitants ; le cinquième, par la destruction de ses dirigeants, la destruction par le tremblement de terre et l'épée ; le sixième, multiplication de la grêle et de la neige ; le septième, les bêtes sauvages seront leur tombeau ; le huitième, la faim et la peste alterneront avec leur destruction ; le neuvième, châtiment par l'épée et fuite en détresse ; le dixième, tonnerre, voix et tremblement de terre destructeur.

\chapter{31}

\par 1 « Et alors je sonnerai de la trompette dans les airs, et j'enverrai mon Élu, ayant en lui toute ma puissance, une seule mesure ; et celui-ci appellera mon peuple méprisé du milieu des nations, et je brûlerai par le feu ceux qui l'ont insulté et qui ont régné parmi eux dans (ce) siècle.

\par 2 « Et je livrerai à l'injure du siècle à venir ceux qui m'ont couvert de moquerie ; et je les ai préparés pour être la nourriture du feu de l'Hadès et pour un vol incessant dans les airs dans le monde souterrain, sous la terre [le corps rempli de vers]. Car sur eux ils verront la justice du Créateur, c'est-à-dire ceux qui ont choisi de faire ma volonté et ceux qui ont ouvertement gardé mes commandements, (et) ils se réjouiront avec joie de la chute des hommes qui restent encore. , qui ont suivi les idoles et leurs meurtres. Car ils se pourriront dans le corps du mauvais ver Azazel, et seront brûlés par le feu de la langue d'Azazel ; car j'espérais qu'ils viendraient à moi, et qu'ils n'auraient pas aimé et loué l'étranger (dieu), et qu'ils n'auraient pas adhéré à celui pour lequel ils n'étaient pas attribués, mais (au lieu de cela) ils auraient abandonné le puissant Seigneur.

\chapter{32}

\par \textit{Conclusion (Chapitre XXXII.)}

\par 1 « C’est pourquoi écoute, ô Abraham, et vois ; voila ! ta septième génération (ira) avec toi, et ils sortiront dans un pays étranger, et ils les asserviront et feront du mal - supplie-les comme si c'était une heure de l'âge de l'impiété mais la nation qu'ils serviront, je le ferai juge.»

\end{document}