\begin{document}

\title{Hebrews}


\chapter{1}

\par 1 Après avoir autrefois, à plusieurs reprises et de plusieurs manières, parlé à nos pères par les prophètes,
\par 2 Dieu, dans ces derniers temps, nous a parlé par le Fils, qu'il a établi héritier de toutes choses, par lequel il a aussi créé le monde,
\par 3 et qui, étant le reflet de sa gloire et l'empreinte de sa personne, et soutenant toutes choses par sa parole puissante, a fait la purification des péchés et s'est assis à la droite de la majesté divine dans les lieux très hauts,
\par 4 devenu d'autant supérieur aux anges qu'il a hérité d'un nom plus excellent que le leur.
\par 5 Car auquel des anges Dieu a-t-il jamais dit: Tu es mon Fils, Je t'ai engendré aujourd'hui? Et encore: Je serai pour lui un père, et il sera pour moi un fils?
\par 6 Et lorsqu'il introduit de nouveau dans le monde le premier-né, il dit: Que tous les anges de Dieu l'adorent!
\par 7 De plus, il dit des anges: Celui qui fait de ses anges des vents, Et de ses serviteurs une flamme de feu.
\par 8 Mais il a dit au Fils: Ton trône, ô Dieu est éternel; Le sceptre de ton règne est un sceptre d'équité;
\par 9 Tu as aimé la justice, et tu as haï l'iniquité; C'est pourquoi, ô Dieu, ton Dieu t'a oint D'une huile de joie au-dessus de tes égaux.
\par 10 Et encore: Toi, Seigneur, tu as au commencement fondé la terre, Et les cieux sont l'ouvrage de tes mains;
\par 11 Ils périront, mais tu subsistes; Ils vieilliront tous comme un vêtement,
\par 12 Tu les rouleras comme un manteau et ils seront changés; Mais toi, tu restes le même, Et tes années ne finiront point.
\par 13 Et auquel des anges a-t-il jamais dit: Assieds-toi à ma droite, jusqu'à ce que je fasse de tes ennemis ton marchepied?
\par 14 Ne sont-ils pas tous des esprits au service de Dieu, envoyés pour exercer un ministère en faveur de ceux qui doivent hériter du salut?

\chapter{2}

\par 1 C'est pourquoi nous devons d'autant plus nous attacher aux choses que nous avons entendues, de peur que nous ne soyons emportés loin d'elles.
\par 2 Car, si la parole annoncée par des anges a eu son effet, et si toute transgression et toute désobéissance a reçu une juste rétribution,
\par 3 comment échapperons-nous en négligeant un si grand salut, qui, annoncé d'abord par le Seigneur, nous a été confirmé par ceux qui l'ont entendu,
\par 4 Dieu appuyant leur témoignage par des signes, des prodiges, et divers miracles, et par les dons du Saint Esprit distribués selon sa volonté.
\par 5 En effet, ce n'est pas à des anges que Dieu a soumis le monde à venir dont nous parlons.
\par 6 Or quelqu'un a rendu quelque part ce témoignage: Qu'est-ce que l'homme, pour que tu te souviennes de lui, Ou le fils de l'homme, pour que tu prennes soin de lui?
\par 7 Tu l'as abaissé pour un peu de temps au-dessous des anges, Tu l'as couronné de gloire et d'honneur,
\par 8 Tu as mis toutes choses sous ses pieds. En effet, en lui soumettant toutes choses, Dieu n'a rien laissé qui ne lui fût soumis. Cependant, nous ne voyons pas encore maintenant que toutes choses lui soient soumises.
\par 9 Mais celui qui a été abaissé pour un peu de temps au-dessous des anges, Jésus, nous le voyons couronné de gloire et d'honneur à cause de la mort qu'il a soufferte, afin que, par la grâce de Dieu, il souffrît la mort pour tous.
\par 10 Il convenait, en effet, que celui pour qui et par qui sont toutes choses, et qui voulait conduire à la gloire beaucoup de fils, élevât à la perfection par les souffrances le Prince de leur salut.
\par 11 Car celui qui sanctifie et ceux qui sont sanctifiés sont tous issus d'un seul. C'est pourquoi il n'a pas honte de les appeler frères,
\par 12 lorsqu'il dit: J'annoncerai ton nom à mes frères, Je te célébrerai au milieu de l'assemblée.
\par 13 Et encore: Je me confierai en toi. Et encore: Me voici, moi et les enfants que Dieu m'a donnés.
\par 14 Ainsi donc, puisque les enfants participent au sang et à la chair, il y a également participé lui-même, afin que, par la mort, il anéantît celui qui a la puissance de la mort, c'est à dire le diable,
\par 15 et qu'il délivrât tous ceux qui, par crainte de la mort, étaient toute leur vie retenus dans la servitude.
\par 16 Car assurément ce n'est pas à des anges qu'il vient en aide, mais c'est à la postérité d'Abraham.
\par 17 En conséquence, il a dû être rendu semblable en toutes choses à ses frères, afin qu'il fût un souverain sacrificateur miséricordieux et fidèle dans le service de Dieu, pour faire l'expiation des péchés du peuple;
\par 18 car, ayant été tenté lui-même dans ce qu'il a souffert, il peut secourir ceux qui sont tentés.

\chapter{3}

\par 1 C'est pourquoi, frères saints, qui avez part à la vocation céleste, considérez l'apôtre et le souverain sacrificateur de la foi que nous professons,
\par 2 Jésus, qui a été fidèle à celui qui l'a établi, comme le fut Moïse dans toute sa maison.
\par 3 Car il a été jugé digne d'une gloire d'autant supérieure à celle de Moïse que celui qui a construit une maison a plus d'honneur que la maison même.
\par 4 Chaque maison est construite par quelqu'un, mais celui qui a construit toutes choses, c'est Dieu.
\par 5 Pour Moïse, il a été fidèle dans toute la maison de Dieu, comme serviteur, pour rendre témoignage de ce qui devait être annoncé;
\par 6 mais Christ l'est comme Fils sur sa maison; et sa maison, c'est nous, pourvu que nous retenions jusqu'à la fin la ferme confiance et l'espérance dont nous nous glorifions.
\par 7 C'est pourquoi, selon ce que dit le Saint Esprit: Aujourd'hui, si vous entendez sa voix,
\par 8 N'endurcissez pas vos coeurs, comme lors de la révolte, Le jour de la tentation dans le désert,
\par 9 Où vos pères me tentèrent, Pour m'éprouver, et ils virent mes oeuvres Pendant quarante ans.
\par 10 Aussi je fus irrité contre cette génération, et je dis: Ils ont toujours un coeur qui s'égare. Ils n'ont pas connu mes voies.
\par 11 Je jurai donc dans ma colère: Ils n'entreront pas dans mon repos!
\par 12 Prenez garde, frère, que quelqu'un de vous n'ait un coeur mauvais et incrédule, au point de se détourner du Dieu vivant.
\par 13 Mais exhortez-vous les uns les autres chaque jour, aussi longtemps qu'on peut dire: Aujourd'hui! afin qu'aucun de vous ne s'endurcisse par la séduction du péché.
\par 14 Car nous sommes devenus participants de Christ, pourvu que nous retenions fermement jusqu'à la fin l'assurance que nous avions au commencement,
\par 15 pendant qu'il est dit: Aujourd'hui, si vous entendez sa voix, N'endurcissez pas vos coeurs, comme lors de la révolte.
\par 16 Qui furent, en effet, ceux qui se révoltèrent après l'avoir entendue, sinon tous ceux qui étaient sortis d'Égypte sous la conduite de Moïse?
\par 17 Et contre qui Dieu fut-il irrité pendant quarante ans, sinon contre ceux qui péchaient, et dont les cadavres tombèrent dans le désert?
\par 18 Et à qui jura-t-il qu'ils n'entreraient pas dans son repos, sinon à ceux qui avaient désobéi?
\par 19 Aussi voyons-nous qu'ils ne purent y entrer à cause de leur incrédulité.

\chapter{4}

\par 1 Craignons donc, tandis que la promesse d'entrer dans son repos subsiste encore, qu'aucun de vous ne paraisse être venu trop tard.
\par 2 Car cette bonne nouvelle nous a été annoncée aussi bien qu'à eux; mais la parole qui leur fut annoncée ne leur servit de rien, parce qu'elle ne trouva pas de la foi chez ceux qui l'entendirent.
\par 3 Pour nous qui avons cru, nous entrons dans le repos, selon qu'il dit: Je jurai dans ma colère: Ils n'entreront pas dans mon repos! Il dit cela, quoique ses oeuvres eussent été achevées depuis la création du monde.
\par 4 Car il a parlé quelque part ainsi du septième jour: Et Dieu se reposa de toutes ses oeuvres le septième jour.
\par 5 Et ici encore: Ils n'entreront pas dans mon repos!
\par 6 Or, puisqu'il est encore réservé à quelques-uns d'y entrer, et que ceux à qui d'abord la promesse a été faite n'y sont pas entrés à cause de leur désobéissance,
\par 7 Dieu fixe de nouveau un jour-aujourd'hui-en disant dans David si longtemps après, comme il est dit plus haut: Aujourd'hui, si vous entendez sa voix, N'endurcissez pas vos coeurs.
\par 8 Car, si Josué leur eût donné le repos, il ne parlerait pas après cela d'un autre jour.
\par 9 Il y a donc un repos de sabbat réservé au peuple de Dieu.
\par 10 Car celui qui entre dans le repos de Dieu se repose de ses oeuvres, comme Dieu s'est reposé des siennes.
\par 11 Efforçons-nous donc d'entrer dans ce repos, afin que personne ne tombe en donnant le même exemple de désobéissance.
\par 12 Car la parole de Dieu est vivante et efficace, plus tranchante qu'une épée quelconque à deux tranchants, pénétrante jusqu'à partager âme et esprit, jointures et moelles; elle juge les sentiments et les pensées du coeur.
\par 13 Nulle créature n'est cachée devant lui, mais tout est à nu et à découvert aux yeux de celui à qui nous devons rendre compte.
\par 14 Ainsi, puisque nous avons un grand souverain sacrificateur qui a traversé les cieux, Jésus, le Fils de Dieu, demeurons fermes dans la foi que nous professons.
\par 15 Car nous n'avons pas un souverain sacrificateur qui ne puisse compatir à nos faiblesses; au contraire, il a été tenté comme nous en toutes choses, sans commettre de péché.
\par 16 Approchons-nous donc avec assurance du trône de la grâce afin d'obtenir miséricorde et de trouver grâce, pour être secourus dans nos besoins.

\chapter{5}

\par 1 En effet, tout souverain sacrificateur pris du milieu des hommes est établi pour les hommes dans le service de Dieu, afin de présenter des offrandes et des sacrifice pour les péchés.
\par 2 Il peut être indulgent pour les ignorants et les égarés, puisque la faiblesse est aussi son partage.
\par 3 Et c'est à cause de cette faiblesse qu'il doit offrir des sacrifices pour ses propres péchés, comme pour ceux du peuple.
\par 4 Nul ne s'attribue cette dignité, s'il n'est appelé de Dieu, comme le fut Aaron.
\par 5 Et Christ ne s'est pas non plus attribué la gloire de devenir souverain sacrificateur, mais il la tient de celui qui lui a dit: Tu es mon Fils, Je t'ai engendré aujourd'hui!
\par 6 Comme il dit encore ailleurs: Tu es sacrificateur pour toujours, Selon l'ordre de Melchisédek.
\par 7 C'est lui qui, dans les jours de sa chair, ayant présenté avec de grands cris et avec larmes des prières et des supplications à celui qui pouvait le sauver de la mort, et ayant été exaucé à cause de sa piété,
\par 8 a appris, bien qu'il fût Fils, l'obéissance par les choses qu'il a souffertes,
\par 9 et qui, après avoir été élevé à la perfection, est devenu pour tous ceux qui lui obéissent l'auteur d'un salut éternel,
\par 10 Dieu l'ayant déclaré souverain sacrificateur selon l'ordre de Melchisédek.
\par 11 Nous avons beaucoup à dire là-dessus, et des choses difficiles à expliquer, parce que vous êtes devenus lents à comprendre.
\par 12 Vous, en effet, qui depuis longtemps devriez être des maîtres, vous avez encore besoin qu'on vous enseigne les premiers rudiments des oracles de Dieu, vous en êtes venus à avoir besoin de lait et non d'une nourriture solide.
\par 13 Or, quiconque en est au lait n'a pas l'expérience de la parole de justice; car il est un enfant.
\par 14 Mais la nourriture solide est pour les hommes faits, pour ceux dont le jugement est exercé par l'usage à discerner ce qui est bien et ce qui est mal.

\chapter{6}

\par 1 C'est pourquoi, laissant les éléments de la parole de Christ, tendons à ce qui est parfait, sans poser de nouveau le fondement du renoncement aux oeuvres mortes,
\par 2 de la foi en Dieu, de la doctrine des baptêmes, de l'imposition des mains, de la résurrection des morts, et du jugement éternel.
\par 3 C'est ce que nous ferons, si Dieu le permet.
\par 4 Car il est impossible que ceux qui ont été une fois éclairés, qui ont goûté le don céleste, qui ont eu part au Saint Esprit,
\par 5 qui ont goûté la bonne parole de Dieu et les puissances du siècle à venir,
\par 6 et qui sont tombés, soient encore renouvelés et amenés à la repentance, puisqu'ils crucifient pour leur part le Fils de Dieu et l'exposent à l'ignominie.
\par 7 Lorsqu'une terre est abreuvée par la pluie qui tombe souvent sur elle, et qu'elle produit une herbe utile à ceux pour qui elle est cultivée, elle participe à la bénédiction de Dieu;
\par 8 mais, si elle produit des épines et des chardons, elle est réprouvée et près d'être maudite, et on finit par y mettre le feu.
\par 9 Quoique nous parlions ainsi, bien-aimés, nous attendons, pour ce qui vous concerne, des choses meilleures et favorables au salut.
\par 10 Car Dieu n'est pas injuste, pour oublier votre travail et l'amour que vous avez montré pour son nom, ayant rendu et rendant encore des services aux saints.
\par 11 Nous désirons que chacun de vous montre le même zèle pour conserver jusqu'à la fin une pleine espérance,
\par 12 en sorte que vous ne vous relâchiez point, et que voue imitiez ceux qui, par la foi et la persévérance, héritent des promesses.
\par 13 Lorsque Dieu fit la promesse à Abraham, ne pouvant jurer par un plus grand que lui, il jura par lui-même, et dit:
\par 14 Certainement je te bénirai et je multiplierai ta postérité.
\par 15 Et c'est ainsi qu'Abraham, ayant persévéré, obtint l'effet de la promesse.
\par 16 Or les hommes jurent par celui qui est plus grand qu'eux, et le serment est une garantie qui met fin à toutes leurs différends.
\par 17 C'est pourquoi Dieu, voulant montrer avec plus d'évidence aux héritiers de la promesse l'immutabilité de sa résolution, intervint par un serment,
\par 18 afin que, par deux choses immuables, dans lesquelles il est impossible que Dieu mente, nous trouvions un puissant encouragement, nous dont le seul refuge a été de saisir l'espérance qui nous était proposée.
\par 19 Cette espérance, nous la possédons comme une ancre de l'âme, sûre et solide; elle pénètre au delà du voile,
\par 20 là où Jésus est entré pour nous comme précurseur, ayant été fait souverain sacrificateur pour toujours, selon l'ordre de Melchisédek.

\chapter{7}

\par 1 En effet, ce Melchisédek, roi de Salem, sacrificateur du Dieu Très Haut, -qui alla au-devant d'Abraham lorsqu'il revenait de la défaite des rois, qui le bénit,
\par 2 et à qui Abraham donna la dîme de tout, -qui est d'abord roi de justice, d'après la signification de son nom, ensuite roi de Salem, c'est-à-dire roi de paix, -
\par 3 qui est sans père, sans mère, sans généalogie, qui n'a ni commencement de jours ni fin de vie, -mais qui est rendu semblable au Fils de Dieu, -ce Melchisédek demeure sacrificateur à perpétuité.
\par 4 Considérez combien est grand celui auquel le patriarche Abraham donna la dîme du butin.
\par 5 Ceux des fils de Lévi qui exercent le sacerdoce ont, d'après la loi, l'ordre de lever la dîme sur le peuple, c'est-à-dire, sur leurs frères, qui cependant sont issus des reins d'Abraham;
\par 6 et lui, qui ne tirait pas d'eux son origine, il leva la dîme sur Abraham, et il bénit celui qui avait les promesses.
\par 7 Or c'est sans contredit l'inférieur qui est béni par le supérieur.
\par 8 Et ici, ceux qui perçoivent la dîme sont des hommes mortels; mais là, c'est celui dont il est attesté qu'il est vivant.
\par 9 De plus, Lévi, qui perçoit la dîme, l'a payée, pour ainsi dire, par Abraham;
\par 10 car il était encore dans les reins de son père, lorsque Melchisédek alla au-devant d'Abraham.
\par 11 Si donc la perfection avait été possible par le sacerdoce Lévitique, -car c'est sur ce sacerdoce que repose la loi donnée au peuple, -qu'était-il encore besoin qu'il parût un autre sacrificateur selon l'ordre de Melchisédek, et non selon l'ordre d'Aaron?
\par 12 Car, le sacerdoce étant changé, nécessairement aussi il y a un changement de loi.
\par 13 En effet, celui de qui ces choses sont dites appartient à une autre tribu, dont aucun membre n'a fait le service de l'autel;
\par 14 car il est notoire que notre Seigneur est sorti de Juda, tribu dont Moïse n'a rien dit pour ce qui concerne le sacerdoce.
\par 15 Cela devient plus évident encore, quand il paraît un autre sacrificateur à la ressemblance de Melchisédek,
\par 16 institué, non d'après la loi d'une ordonnance charnelle, mais selon la puissance d'une vie impérissable;
\par 17 car ce témoignage lui est rendu: Tu es sacrificateur pour toujours Selon l'ordre de Melchisédek.
\par 18 Il y a ainsi abolition d'une ordonnance antérieure, à cause de son impuissance et de son inutilité,
\par 19 -car la loi n'a rien amené à la perfection, -et introduction d'une meilleure espérance, par laquelle nous nous approchons de Dieu.
\par 20 Et, comme cela n'a pas eu lieu sans serment,
\par 21 -car, tandis que les Lévites sont devenus sacrificateurs sans serment, Jésus l'est devenu avec serment par celui qui lui a dit: Le Seigneur a juré, et il ne se repentira pas: Tu es sacrificateur pour toujours, Selon l'ordre de Melchisédek. -
\par 22 Jésus est par cela même le garant d'une alliance plus excellente.
\par 23 De plus, il y a eu des sacrificateurs en grand nombre, parce que la mort les empêchait d'être permanents.
\par 24 Mais lui, parce qu'il demeure éternellement, possède un sacerdoce qui n'est pas transmissible.
\par 25 C'est aussi pour cela qu'il peut sauver parfaitement ceux qui s'approchent de Dieu par lui, étant toujours vivant pour intercéder en leur faveur.
\par 26 Il nous convenait, en effet, d'avoir un souverain sacrificateur comme lui, saint, innocent, sans tache, séparé des pécheurs, et plus élevé que les cieux,
\par 27 qui n'a pas besoin, comme les souverains sacrificateurs, d'offrir chaque jour des sacrifices, d'abord pour ses propres péchés, ensuite pour ceux du peuple, -car ceci, il l'a fait une fois pour toutes en s'offrant lui-même.
\par 28 En effet, la loi établit souverains sacrificateurs des hommes sujets à la faiblesse; mais la parole du serment qui a été fait après la loi établit le Fils, qui est parfait pour l'éternité.

\chapter{8}

\par 1 Le point capital de ce qui vient d'être dit, c'est que nous avons un tel souverain sacrificateur, qui s'est assis à la droite du trône de la majesté divine dans les cieux,
\par 2 comme ministre du sanctuaire et du véritable tabernacle, qui a été dressé par le Seigneur et non par un homme.
\par 3 Tout souverain sacrificateur est établi pour présenter des offrandes et des sacrifices; d'où il est nécessaire que celui-ci ait aussi quelque chose a présenter.
\par 4 S'il était sur la terre, il ne serait pas même sacrificateur, puisque là sont ceux qui présentent des offrandes selon la loi
\par 5 (lesquels célèbrent un culte, image et ombre des choses célestes, selon que Moïse en fut divinement averti lorsqu'il allait construire le tabernacle: Aie soin, lui fut-il dit, de faire tout d'après le modèle qui t'a été montré sur la montagne).
\par 6 Mais maintenant il a obtenu un ministère d'autant supérieur qu'il est le médiateur d'une alliance plus excellente, qui a été établie sur de meilleures promesses.
\par 7 En effet, si la première alliance avait été sans défaut, il n'aurait pas été question de la remplacer par une seconde.
\par 8 Car c'est avec l'expression d'un blâme que le Seigneur dit à Israël: Voici, les jours viennent, dit le Seigneur, Où je ferai avec la maison d'Israël et la maison de Juda Une alliance nouvelle,
\par 9 Non comme l'alliance que je traitai avec leurs pères, Le jour où je les saisis par la main Pour les faire sortir du pays d'Égypte; Car ils n'ont pas persévéré dans mon alliance, Et moi aussi je ne me suis pas soucié d'eux, dit le Seigneur.
\par 10 Mais voici l'alliance que je ferai avec la maison d'Israël, Après ces jours-là, dit le Seigneur: Je mettrai mes lois dans leur esprit, Je les écrirai dans leur coeur; Et je serai leur Dieu, Et ils seront mon peuple.
\par 11 Aucun n'enseignera plus son concitoyen, Ni aucun son frère, en disant: Connais le Seigneur! Car tous me connaîtront, Depuis le plus petit jusqu'au plus grand d'entre eux;
\par 12 Parce que je pardonnerai leurs iniquités, Et que je ne me souviendrai plus de leurs péchés.
\par 13 En disant: une alliance nouvelle, il a déclaré la première ancienne; or, ce qui est ancien, ce qui a vieilli, est près de disparaître.

\chapter{9}

\par 1 La première alliance avait aussi des ordonnances relatives au culte, et le sanctuaire terrestre.
\par 2 Un tabernacle fut, en effet, construit. Dans la partie antérieure, appelée le lieu saint, étaient le chandelier, la table, et les pains de proposition.
\par 3 Derrière le second voile se trouvait la partie du tabernacle appelée le saint des saints,
\par 4 renfermant l'autel d'or pour les parfums, et l'arche de l'alliance, entièrement recouverte d'or. Il y avait dans l'arche un vase d'or contenant la manne, la verge d'Aaron, qui avait fleuri, et les tables de l'alliance.
\par 5 Au-dessus de l'arche étaient les chérubins de la gloire, couvrant de leur ombre le propitiatoire. Ce n'est pas le moment de parler en détail là-dessus.
\par 6 Or, ces choses étant ainsi disposées, les sacrificateurs qui font le service entrent en tout temps dans la première partie du tabernacle;
\par 7 et dans la seconde le souverain sacrificateur seul entre une fois par an, non sans y porter du sang qu'il offre pour lui-même et pour les péchés du peuple.
\par 8 Le Saint Esprit montrait par là que le chemin du lieu très saint n'était pas encore ouvert, tant que le premier tabernacle subsistait.
\par 9 C'est une figure pour le temps actuel, où l'on présente des offrandes et des sacrifices qui ne peuvent rendre parfait sous le rapport de la conscience celui qui rend ce culte,
\par 10 et qui, avec les aliments, les boissons et les divers ablutions, étaient des ordonnances charnelles imposées seulement jusqu'à une époque de réformation.
\par 11 Mais Christ est venu comme souverain sacrificateur des biens à venir; il a traversé le tabernacle plus grand et plus parfait, qui n'est pas construit de main d'homme, c'est-à-dire, qui n'est pas de cette création;
\par 12 et il est entré une fois pour toutes dans le lieu très saint, non avec le sang des boucs et des veaux, mais avec son propre sang, ayant obtenu une rédemption éternelle.
\par 13 Car si le sang des taureaux et des boucs, et la cendre d'une vache, répandue sur ceux qui sont souillés, sanctifient et procurent la pureté de la chair,
\par 14 combien plus le sang de Christ, qui, par un esprit éternel, s'est offert lui-même sans tache à Dieu, purifiera-t-il votre conscience des oeuvres mortes, afin que vous serviez le Dieu vivant!
\par 15 Et c'est pour cela qu'il est le médiateur d'une nouvelle alliance, afin que, la mort étant intervenue pour le rachat des transgressions commises sous la première alliance, ceux qui ont été appelés reçoivent l'héritage éternel qui leur a été promis.
\par 16 Car là où il y a un testament, il est nécessaire que la mort du testateur soit constatée.
\par 17 Un testament, en effet, n'est valable qu'en cas de mort, puisqu'il n'a aucune force tant que le testateur vit.
\par 18 Voilà pourquoi c'est avec du sang que même la première alliance fut inaugurée.
\par 19 Moïse, après avoir prononcé devant tout le peuple tous les commandements de la loi, prit le sang des veaux et des boucs, avec de l'eau, de la laine écarlate, et de l'hysope; et il fit l'aspersion sur le livre lui-même et sur tout le peuple, en disant:
\par 20 Ceci est le sang de l'alliance que Dieu a ordonnée pour vous.
\par 21 Il fit pareillement l'aspersion avec le sang sur le tabernacle et sur tous les ustensiles du culte.
\par 22 Et presque tout, d'après la loi, est purifié avec du sang, et sans effusion de sang il n'y a pas de pardon.
\par 23 Il était donc nécessaire, puisque les images des choses qui sont dans les cieux devaient être purifiées de cette manière, que les choses célestes elles-mêmes le fussent par des sacrifices plus excellents que ceux-là.
\par 24 Car Christ n'est pas entré dans un sanctuaire fait de main d'homme, en imitation du véritable, mais il est entré dans le ciel même, afin de comparaître maintenant pour nous devant la face de Dieu.
\par 25 Et ce n'est pas pour s'offrir lui-même plusieurs fois qu'il y est entré, comme le souverain sacrificateur entre chaque année dans le sanctuaire avec du sang étranger;
\par 26 autrement, il aurait fallu qu'il eût souffert plusieurs fois depuis la création du monde, tandis que maintenant, à la fin des siècles, il a paru une seul fois pour abolir le péché par son sacrifice.
\par 27 Et comme il est réservé aux hommes de mourir une seul fois, après quoi vient le jugement,
\par 28 de même Christ, qui s'est offert une seul fois pour porter les péchés de plusieurs, apparaîtra sans péché une seconde fois à ceux qui l'attendent pour leur salut.

\chapter{10}

\par 1 En effet, la loi, qui possède une ombre des biens à venir, et non l'exacte représentation des choses, ne peut jamais, par les mêmes sacrifices qu'on offre perpétuellement chaque année, amener les assistants à la perfection.
\par 2 Autrement, n'aurait-on pas cessé de les offrir, parce que ceux qui rendent ce culte, étant une fois purifiés, n'auraient plus eu aucune conscience de leurs péchés?
\par 3 Mais le souvenir des péchés est renouvelé chaque année par ces sacrifices;
\par 4 car il est impossible que le sang des taureaux et des boucs ôte les péchés.
\par 5 C'est pourquoi Christ, entrant dans le monde, dit: Tu n'as voulu ni sacrifice ni offrande, Mais tu m'as formé un corps;
\par 6 Tu n'as agréé ni holocaustes ni sacrifices pour le péché.
\par 7 Alors j'ai dit: Voici, je viens (Dans le rouleau du livre il est question de moi) Pour faire, ô Dieu, ta volonté.
\par 8 Après avoir dit d'abord: Tu n'as voulu et tu n'as agréé ni sacrifices ni offrandes, Ni holocaustes ni sacrifices pour le péché (ce qu'on offre selon la loi),
\par 9 il dit ensuite: Voici, je viens Pour faire ta volonté. Il abolit ainsi la première chose pour établir la seconde.
\par 10 C'est en vertu de cette volonté que nous sommes sanctifiés, par l'offrande du corps de Jésus Christ, une fois pour toutes.
\par 11 Et tandis que tout sacrificateur fait chaque jour le service et offre souvent les mêmes sacrifices, qui ne peuvent jamais ôter les péchés,
\par 12 lui, après avoir offert un seul sacrifice pour les péchés, s'est assis pour toujours à la droite de Dieu,
\par 13 attendant désormais que ses ennemis soient devenus son marchepied.
\par 14 Car, par une seule offrande, il a amené à la perfection pour toujours ceux qui sont sanctifiés.
\par 15 C'est ce que le Saint Esprit nous atteste aussi; car, après avoir dit:
\par 16 Voici l'alliance que je ferai avec eux, Après ces jours-là, dit le Seigneur: Je mettrai mes lois dans leurs coeurs, Et je les écrirai dans leur esprit, il ajoute:
\par 17 Et je ne me souviendrai plus de leurs péchés ni de leurs iniquités.
\par 18 Or, là où il y a pardon des péchés, il n'y a plus d'offrande pour le péché.
\par 19 Ainsi donc, frères, puisque nous avons, au moyen du sang de Jésus, une libre entrée dans le sanctuaire
\par 20 par la route nouvelle et vivante qu'il a inaugurée pour nous au travers du voile, c'est-à-dire, de sa chair,
\par 21 et puisque nous avons un souverain sacrificateur établi sur la maison de Dieu,
\par 22 approchons-nous avec un coeur sincère, dans la plénitude de la foi, les coeurs purifiés d'une mauvaise conscience, et le corps lavé d'une eau pure.
\par 23 Retenons fermement la profession de notre espérance, car celui qui a fait la promesse est fidèle.
\par 24 Veillons les uns sur les autres, pour nous exciter à la charité et aux bonnes oeuvres.
\par 25 N'abandonnons pas notre assemblée, comme c'est la coutume de quelques-uns; mais exhortons-nous réciproquement, et cela d'autant plus que vous voyez s'approcher le jour.
\par 26 Car, si nous péchons volontairement après avoir reçu la connaissance de la vérité, il ne reste plus de sacrifice pour les péchés,
\par 27 mais une attente terrible du jugement et l'ardeur d'un feu qui dévorera les rebelles.
\par 28 Celui qui a violé la loi de Moïse meurt sans miséricorde, sur la déposition de deux ou de trois témoins;
\par 29 de quel pire châtiment pensez-vous que sera jugé digne celui qui aura foulé aux pieds le Fils de Dieu, qui aura tenu pour profane le sang de l'alliance, par lequel il a été sanctifié, et qui aura outragé l'Esprit de la grâce?
\par 30 Car nous connaissons celui qui a dit: A moi la vengeance, à moi la rétribution! et encore: Le Seigneur jugera son peuple.
\par 31 C'est une chose terrible que de tomber entre les mains du Dieu vivant.
\par 32 Souvenez-vous de ces premiers jours, où, après avoir été éclairés, vous avez soutenu un grand combat au milieu des souffrances,
\par 33 d'une part, exposés comme en spectacle aux opprobres et aux tribulations, et de l'autre, vous associant à ceux dont la position était la même.
\par 34 En effet, vous avez eu de la compassion pour les prisonniers, et vous avez accepté avec joie l'enlèvement de vos biens, sachant que vous avez des biens meilleurs et qui durent toujours.
\par 35 N'abandonnez donc pas votre assurance, à laquelle est attachée une grande rémunération.
\par 36 Car vous avez besoin de persévérance, afin qu'après avoir accompli la volonté de Dieu, vous obteniez ce qui vous est promis.
\par 37 Encore un peu, un peu de temps: celui qui doit venir viendra, et il ne tardera pas.
\par 38 Et mon juste vivra par la foi; mais, s'il se retire, mon âme ne prend pas plaisir en lui.
\par 39 Nous, nous ne sommes pas de ceux qui se retirent pour se perdre, mais de ceux qui ont la foi pour sauver leur âme.

\chapter{11}

\par 1 Or la foi est une ferme assurance des choses qu'on espère, une démonstration de celles qu'on ne voit pas.
\par 2 Pour l'avoir possédée, les anciens ont obtenu un témoignage favorable.
\par 3 C'est par la foi que nous reconnaissons que le monde a été formé par la parole de Dieu, en sorte que ce qu'on voit n'a pas été fait de choses visibles.
\par 4 C'est par la foi qu'Abel offrit à Dieu un sacrifice plus excellent que celui de Caïn; c'est par elle qu'il fut déclaré juste, Dieu approuvant ses offrandes; et c'est par elle qu'il parle encore, quoique mort.
\par 5 C'est par la foi qu'Énoch fut enlevé pour qu'il ne vît point la mort, et qu'il ne parut plus parce Dieu l'avait enlevé; car, avant son enlèvement, il avait reçu le témoignage qu'il était agréable à Dieu.
\par 6 Or sans la foi il est impossible de lui être agréable; car il faut que celui qui s'approche de Dieu croie que Dieu existe, et qu'il est le rémunérateur de ceux qui le cherchent.
\par 7 C'est par la foi que Noé, divinement averti des choses qu'on ne voyait pas encore, et saisi d'une crainte respectueuse, construisit une arche pour sauver sa famille; c'est par elle qu'il condamna le monde, et devint héritier de la justice qui s'obtient par la foi.
\par 8 C'est par la foi qu'Abraham, lors de sa vocation, obéit et partit pour un lieu qu'il devait recevoir en héritage, et qu'il partit sans savoir où il allait.
\par 9 C'est par la foi qu'il vint s'établir dans la terre promise comme dans une terre étrangère, habitant sous des tentes, ainsi qu'Isaac et Jacob, les cohéritiers de la même promesse.
\par 10 Car il attendait la cité qui a de solides fondements, celle dont Dieu est l'architecte et le constructeur.
\par 11 C'est par la foi que Sara elle-même, malgré son âge avancé, fut rendue capable d'avoir une postérité, parce qu'elle crut à la fidélité de celui qui avait fait la promesse.
\par 12 C'est pourquoi d'un seul homme, déjà usé de corps, naquit une postérité nombreuse comme les étoiles du ciel, comme le sable qui est sur le bord de la mer et qu'on ne peut compter.
\par 13 C'est dans la foi qu'ils sont tous morts, sans avoir obtenu les choses promises; mais ils les ont vues et saluées de loin, reconnaissant qu'ils étaient étrangers et voyageurs sur la terre.
\par 14 Ceux qui parlent ainsi montrent qu'ils cherchent une patrie.
\par 15 S'ils avaient eu en vue celle d'où ils étaient sortis, ils auraient eu le temps d'y retourner.
\par 16 Mais maintenant ils en désirent une meilleure, c'est-à-dire une céleste. C'est pourquoi Dieu n'a pas honte d'être appelé leur Dieu, car il leur a préparé une cité.
\par 17 C'est par la foi qu'Abraham offrit Isaac, lorsqu'il fut mis à l'épreuve, et qu'il offrit son fils unique, lui qui avait reçu les promesses,
\par 18 et à qui il avait été dit: En Isaac sera nommée pour toi une postérité.
\par 19 Il pensait que Dieu est puissant, même pour ressusciter les morts; aussi le recouvra-t-il par une sorte de résurrection.
\par 20 C'est par la foi qu'Isaac bénit Jacob et Ésaü, en vue des choses à venir.
\par 21 C'est par la foi que Jacob mourant bénit chacun des fils de Joseph, et qu'il adora, appuyé sur l'extrémité de son bâton.
\par 22 C'est par la foi que Joseph mourant fit mention de la sortie des fils d'Israël, et qu'il donna des ordres au sujet de ses os.
\par 23 C'est par la foi que Moïse, à sa naissance, fut caché pendant trois mois par ses parents, parce qu'ils virent que l'enfant était beau, et qu'ils ne craignirent pas l'ordre du roi.
\par 24 C'est par la foi que Moïse, devenu grand, refusa d'être appelé fils de la fille de Pharaon,
\par 25 aimant mieux être maltraité avec le peuple de Dieu que d'avoir pour un temps la jouissance du péché,
\par 26 regardant l'opprobre de Christ comme une richesse plus grande que les trésors de l'Égypte, car il avait les yeux fixés sur la rémunération.
\par 27 C'est par la foi qu'il quitta l'Égypte, sans être effrayé de la colère du roi; car il se montra ferme, comme voyant celui qui est invisible.
\par 28 C'est par la foi qu'il fit la Pâque et l'aspersion du sang, afin que l'exterminateur ne touchât pas aux premiers-nés des Israélites.
\par 29 C'est par la foi qu'ils traversèrent la mer Rouge comme un lieu sec, tandis que les Égyptiens qui en firent la tentative furent engloutis.
\par 30 C'est par la foi que les murailles de Jéricho tombèrent, après qu'on en eut fait le tour pendant sept jours.
\par 31 C'est par la foi que Rahab la prostituée ne périt pas avec les rebelles, parce qu'elle avait reçu les espions avec bienveillance.
\par 32 Et que dirai-je encore? Car le temps me manquerait pour parler de Gédéon, de Barak, de Samson, de Jephthé, de David, de Samuel, et des prophètes,
\par 33 qui, par la foi, vainquirent des royaumes, exercèrent la justice, obtinrent des promesses, fermèrent la gueule des lions,
\par 34 éteignirent la puissance du feu, échappèrent au tranchant de l'épée, guérirent de leurs maladies, furent vaillants à la guerre, mirent en fuite des armées étrangères.
\par 35 Des femmes recouvrèrent leurs morts par la résurrection; d'autres furent livrés aux tourments, et n'acceptèrent point de délivrance, afin d'obtenir une meilleure résurrection;
\par 36 d'autres subirent les moqueries et le fouet, les chaînes et la prison;
\par 37 ils furent lapidés, sciés, torturés, ils moururent tués par l'épée, ils allèrent çà et là vêtus de peaux de brebis et de peaux de chèvres, dénués de tout, persécutés, maltraités,
\par 38 eux dont le monde n'était pas digne, errants dans les déserts et les montagnes, dans les cavernes et les antres de la terre.
\par 39 Tous ceux-là, à la foi desquels il a été rendu témoignage, n'ont pas obtenu ce qui leur était promis,
\par 40 Dieu ayant en vue quelque chose de meilleur pour nous, afin qu'ils ne parvinssent pas sans nous à la perfection.

\chapter{12}

\par 1 Nous donc aussi, puisque nous sommes environnés d'une si grande nuée de témoins, rejetons tout fardeau, et le péché qui nous enveloppe si facilement, et courons avec persévérance dans la carrière qui nous est ouverte,
\par 2 ayant les regards sur Jésus, le chef et le consommateur de la foi, qui, en vue de la joie qui lui était réservée, a souffert la croix, méprisé l'ignominie, et s'est assis à la droite du trône de Dieu.
\par 3 Considérez, en effet, celui qui a supporté contre sa personne une telle opposition de la part des pécheurs, afin que vous ne vous lassiez point, l'âme découragée.
\par 4 Vous n'avez pas encore résisté jusqu'au sang, en luttant contre le péché.
\par 5 Et vous avez oubliez l'exhortation qui vous est adressée comme à des fils: Mon fils, ne méprise pas le châtiment du Seigneur, Et ne perds pas courage lorsqu'il te reprend;
\par 6 Car le Seigneur châtie celui qu'il aime, Et il frappe de la verge tous ceux qu'il reconnaît pour ses fils.
\par 7 Supportez le châtiment: c'est comme des fils que Dieu vous traite; car quel est le fils qu'un père ne châtie pas?
\par 8 Mais si vous êtes exempts du châtiment auquel tous ont part, vous êtes donc des enfants illégitimes, et non des fils.
\par 9 D'ailleurs, puisque nos pères selon la chair nous ont châtiés, et que nous les avons respectés, ne devons nous pas à bien plus forte raison nous soumettre au Père des esprits, pour avoir la vie?
\par 10 Nos pères nous châtiaient pour peu de jours, comme ils le trouvaient bon; mais Dieu nous châtie pour notre bien, afin que nous participions à sa sainteté.
\par 11 Il est vrai que tout châtiment semble d'abord un sujet de tristesse, et non de joie; mais il produit plus tard pour ceux qui ont été ainsi exercés un fruit paisible de justice.
\par 12 Fortifiez donc vos mains languissantes Et vos genoux affaiblis;
\par 13 et suivez avec vos pieds des voies droites, afin que ce qui est boiteux ne dévie pas, mais plutôt se raffermisse.
\par 14 Recherchez la paix avec tous, et la sanctification, sans laquelle personne ne verra le Seigneur.
\par 15 Veillez à ce que nul ne se prive de la grâce de Dieu; à ce qu'aucune racine d'amertume, poussant des rejetons, ne produise du trouble, et que plusieurs n'en soient infectés;
\par 16 à ce qu'il n'y ait ni impudique, ni profane comme Ésaü, qui pour un mets vendit son droit d'aînesse.
\par 17 Vous savez que, plus tard, voulant obtenir la bénédiction, il fut rejeté, quoiqu'il la sollicitât avec larmes; car son repentir ne put avoir aucun effet.
\par 18 Vous ne vous êtes pas approchés d'une montagne qu'on pouvait toucher et qui était embrasée par le feu, ni de la nuée, ni des ténèbres, ni de la tempête,
\par 19 ni du retentissement de la trompette, ni du bruit des paroles, tel que ceux qui l'entendirent demandèrent qu'il ne leur en fût adressé aucune de plus,
\par 20 car ils ne supportaient pas cette déclaration: Si même une bête touche la montagne, elle sera lapidée.
\par 21 Et ce spectacle était si terrible que Moïse dit: Je suis épouvanté et tout tremblant!
\par 22 Mais vous vous êtes approchés de la montagne de Sion, de la cité du Dieu vivant, la Jérusalem céleste, des myriades qui forment le choeur des anges,
\par 23 de l'assemblé des premiers-nés inscrits dans les cieux, du juge qui est le Dieu de tous, des esprits des justes parvenus à la perfection,
\par 24 de Jésus qui est le médiateur de la nouvelle alliance, et du sang de l'aspersion qui parle mieux que celui d'Abel.
\par 25 Gardez-vous de refuser d'entendre celui qui parle; car si ceux-là n'ont pas échappé qui refusèrent d'entendre celui qui publiait les oracles sur la terre, combien moins échapperons-nous, si nous nous détournons de celui qui parle du haut des cieux,
\par 26 lui, dont la voix alors ébranla la terre, et qui maintenant a fait cette promesse: Une fois encore j'ébranlerai non seulement la terre, mais aussi le ciel.
\par 27 Ces mots: Une fois encore, indiquent le changement des choses ébranlées, comme étant faites pour un temps, afin que les choses inébranlables subsistent.
\par 28 C'est pourquoi, recevant un royaume inébranlable, montrons notre reconnaissance en rendant à Dieu un culte qui lui soit agréable,
\par 29 avec piété et avec crainte, car notre Dieu est aussi un feu dévorant.

\chapter{13}

\par 1 Persévérez dans l'amour fraternel.
\par 2 N'oubliez pas l'hospitalité; car, en l'exerçant, quelques-uns ont logé des anges, sans le savoir.
\par 3 Souvenez-vous des prisonniers, comme si vous étiez aussi prisonniers; de ceux qui sont maltraités, comme étant aussi vous-mêmes dans un corps.
\par 4 Que le mariage soit honoré de tous, et le lit conjugal exempt de souillure, car Dieu jugera les impudiques et les adultères.
\par 5 Ne vous livrez pas à l'amour de l'argent; contentez-vous de ce que vous avez; car Dieu lui-même a dit: Je ne te délaisserai point, et je ne t'abandonnerai point.
\par 6 C'est donc avec assurance que nous pouvons dire: Le Seigneur est mon aide, je ne craindrai rien; Que peut me faire un homme?
\par 7 Souvenez-vous de vos conducteurs qui vous ont annoncé la parole de Dieu; considérez quelle a été la fin de leur vie, et imitez leur foi.
\par 8 Jésus Christ est le même hier, aujourd'hui, et éternellement.
\par 9 Ne vous laissez pas entraîner par des doctrines diverses et étrangères; car il est bon que le coeur soit affermi par la grâce, et non par des aliments qui n'ont servi de rien à ceux qui s'y sont attachés.
\par 10 Nous avons un autel dont ceux qui font le service au tabernacle n'ont pas le pouvoir de manger.
\par 11 Le corps des animaux, dont le sang est porté dans le sanctuaire par le souverain sacrificateur pour le péché, sont brûlés hors du camp.
\par 12 C'est pour cela que Jésus aussi, afin de sanctifier le peuple par son propre sang, a souffert hors de la porte.
\par 13 Sortons donc pour aller à lui, hors du camp, en portant son opprobre.
\par 14 Car nous n'avons point ici-bas de cité permanente, mais nous cherchons celle qui est à venir.
\par 15 Par lui, offrons sans cesse à Dieu un sacrifice de louange, c'est-a-dire le fruit de lèvres qui confessent son nom.
\par 16 Et n'oubliez pas la bienfaisance et la libéralité, car c'est à de tels sacrifices que Dieu prend plaisir.
\par 17 Obéissez à vos conducteurs et ayez pour eux de la déférence, car ils veillent sur vos âmes comme devant en rendre compte; qu'il en soit ainsi, afin qu'ils le fassent avec joie, et non en gémissant, ce qui vous ne serait d'aucun avantage.
\par 18 Priez pour nous; car nous croyons avoir une bonne conscience, voulant en toutes choses nous bien conduire.
\par 19 C'est avec instance que je vous demande de le faire, afin que je vous sois rendu plus tôt.
\par 20 Que le Dieu de paix, qui a ramené d'entre les morts le grand pasteur des brebis, par le sang d'une alliance éternelle, notre Seigneur Jésus,
\par 21 vous rende capables de toute bonne oeuvre pour l'accomplissement de sa volonté, et fasse en vous ce qui lui est agréable, par Jésus Christ, auquel soit la gloire aux siècles des siècles! Amen!
\par 22 Je vous prie, frères, de supporter ces paroles d'exhortation, car je vous ai écrit brièvement.
\par 23 Sachez que notre frère Timothée a été relâché; s'il vient bientôt, j'irai vous voir avec lui.
\par 24 Saluez tous vos conducteurs, et tous les saints. Ceux d'Italie vous saluent.
\par 25 Que la grâce soit avec vous tous! Amen!


\end{document}