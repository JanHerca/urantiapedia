\begin{document}

\title{Bel et le Dragon}

\chapter{1}

\par 1 Et le roi Astyages fut rassemblé auprès de ses pères, et Cyrus de Perse reçut son royaume.
\par 2 Et Daniel s'entretenait avec le roi, et était honoré plus que tous ses amis.
\par 3 Or les Babyloniens avaient une idole appelée Bel, et on dépensait pour elle chaque jour douze grandes mesures de fine farine, quarante brebis et six vases de vin.
\par 4 Et le roi l'adorait et allait chaque jour l'adorer; mais Daniel adorait son propre Dieu. Et le roi lui dit : Pourquoi n'adores-tu pas Bel ?
\par 5 Qui répondit et dit : Parce que je ne dois pas adorer des idoles faites de main d'homme, mais le Dieu vivant, qui a créé le ciel et la terre, et qui a souveraineté sur toute chair.
\par 6 Alors le roi lui dit : Ne penses-tu pas que Bel soit un Dieu vivant ? ne vois-tu pas combien il mange et boit chaque jour ?
\par 7 Alors Daniel sourit et dit : Ô roi, ne te trompe pas ; car ceci n'est que de l'argile au dedans, et de l'airain au dehors, et il n'a jamais mangé ni bu quoi que ce soit.
\par 8 Alors le roi fut en colère, et appela ses prêtres, et leur dit : Si vous ne me dites pas qui est celui qui dévore ces dépenses, vous mourrez.
\par 9 Mais si vous pouvez me certifier que Bel les dévore, alors Daniel mourra, car il a blasphémé contre Bel. Et Daniel dit au roi : Qu'il se fasse selon ta parole.
\par 10 Or, les prêtres de Bel étaient soixante-dix, sans compter leurs femmes et leurs enfants. Et le roi entra avec Daniel dans le temple de Bel.
\par 11 Alors les prêtres de Bel dirent : Voici, nous sortons ; mais toi, ô roi, mets la viande, prépare le vin, ferme bien la porte et scelle-la avec ton propre sceau ;
\par 12 Et demain, quand tu entreras, si tu ne trouves pas que Bel a tout mangé, nous mourrons; ou bien Daniel, qui parle faussement contre nous.
\par 13 Et ils n'y prêtèrent guère attention, car ils avaient fait sous la table une entrée secrète, par laquelle ils entraient continuellement et consommaient ces choses.
\par 14 Et quand ils furent sortis, le roi déposa des plats devant Bel. Daniel avait ordonné à ses serviteurs d'apporter des cendres, et ils les répandirent dans tout le temple, en présence du roi seul. Puis ils sortirent, fermèrent la porte et la scellèrent avec le sceau du roi, et s'en allèrent ainsi.
\par 15 Or, pendant la nuit, les prêtres venaient avec leurs femmes et leurs enfants, comme ils avaient l'habitude de le faire, et ils mangeaient et buvaient tout.
\par 16 Le roi se leva de bonne heure, et Daniel avec lui.
\par 17 Et le roi dit : Daniel, les sceaux sont-ils intacts ? Et il dit : Oui, ô roi, qu'ils soient sains et saufs.
\par 18 Et aussitôt qu'il eut ouvert la porte, le roi regarda la table et s'écria d'une voix forte : Tu es grand, ô Bel, et tu n'as aucune tromperie du tout.
\par 19 Alors Daniel se mit à rire, et retint le roi de ne pas entrer, et dit : Voici maintenant le trottoir, et remarquez bien à qui sont ces pas.
\par 20 Et le roi dit : Je vois les pas des hommes, des femmes et des enfants. Et puis le roi était en colère,
\par 21 Et il prit les prêtres avec leurs femmes et leurs enfants, qui lui montrèrent les portes secrètes par où ils entraient, et consommaient ce qui était sur la table.
\par 22 C'est pourquoi le roi les tua et livra Bel au pouvoir de Daniel, qui le détruisit ainsi que son temple.
\par 23 Et en ce même lieu il y avait un grand dragon, que les Babyloniens adoraient.
\par 24 Et le roi dit à Daniel : Veux-tu aussi dire que ceci est d'airain ? voici, il vit, il mange et boit ; tu ne peux pas dire qu'il n'est pas un dieu vivant : adore-le donc.
\par 25 Alors Daniel dit au roi : J'adorerai l'Éternel, mon Dieu, car il est le Dieu vivant.
\par 26 Mais donne-moi la permission, ô roi, et je tuerai ce dragon sans épée ni bâton. Le roi dit : je te donne la permission.
\par 27 Alors Daniel prit de la poix, de la graisse et des poils, et les fit bouillir ensemble, et en fit des morceaux : il mit cela dans la gueule du dragon, et ainsi le dragon éclata en morceaux. Et Daniel dit : Voici, ce sont les les dieux que vous adorez.
\par 28 Quand les Babyloniens apprirent cela, ils furent très indignés et conspirèrent contre le roi, disant : Le roi est devenu Juif, et il a détruit Bel, il a tué le dragon et a mis à mort les prêtres.
\par 29 Alors ils s'approchèrent du roi et lui dirent : Délivre-nous Daniel, sinon nous te détruirons, toi et ta maison.
\par 30 Et le roi, voyant qu'ils le pressaient fort, étant contraints, il leur livra Daniel.
\par 31 Qui l'a jeté dans la fosse aux lions : où il est resté six jours.
\par 32 Et dans la fosse il y avait sept lions, et ils leur donnaient chaque jour deux cadavres et deux moutons : qui alors ne leur étaient pas donnés, afin qu'ils dévorent Daniel.
\par 33 Or, il y avait dans la communauté juive un prophète appelé Habbacuc, qui avait préparé des plats et qui avait rompu du pain dans un bol, et qui allait aux champs pour l'apporter aux moissonneurs.
\par 34 Mais l'ange du Seigneur dit à Habbacuc : Va, apporte le dîner que tu as à Babylone à Daniel, qui est dans la fosse aux lions.
\par 35 Et Habbacuc dit : Seigneur, je n'ai jamais vu Babylone ; je ne sais pas non plus où se trouve la tanière.
\par 36 Alors l'ange du Seigneur le prit par la couronne, et le mit à nu par les cheveux de sa tête, et, par la véhémence de son esprit, le plaça à Babylone, au-dessus de la tanière.
\par 37 Et Habbacuc s'écria, disant : Ô Daniel, Daniel, prends le dîner que Dieu t'a envoyé.
\par 38 Et Daniel dit : Tu te souviens de moi, ô Dieu, et tu n'as pas abandonné ceux qui te cherchent et qui t'aiment.
\par 39 Alors Daniel se leva et mangea ; et aussitôt l'ange du Seigneur remit Habbacuc à sa place.
\par 40 Le septième jour, le roi alla pleurer Daniel ; et lorsqu'il arriva à la caverne, il regarda à l'intérieur, et voici, Daniel était assis.
\par 41 Alors le roi s'écria d'une voix forte, disant : Grand est le Seigneur Dieu de Daniel, et il n'y en a pas d'autre que toi.
\par 42 Et il le tira dehors, et jeta dans la fosse ceux qui étaient la cause de sa destruction ; et ils furent dévorés en un instant devant sa face.

\end{document}