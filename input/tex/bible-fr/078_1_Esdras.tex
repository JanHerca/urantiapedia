\begin{document}

\title{1 Esdras}


\chapter{1}

\par 1 Et Josias célébra la fête de la Pâque à Jérusalem pour son Seigneur, et offrit la Pâque le quatorzième jour du premier mois ;
\par 2 Ayant placé les prêtres selon leurs cours quotidiens, étant vêtus de longs vêtements, dans le temple du Seigneur.
\par 3 Et il dit aux Lévites, les saints ministres d'Israël, de se consacrer à l'Éternel, et de placer l'arche sainte de l'Éternel dans la maison que le roi Salomon, fils de David, avait bâtie.
\par 4 Et il dit : Vous ne porterez plus l'arche sur vos épaules ; maintenant donc, servez l'Éternel, votre Dieu, et servez son peuple Israël, et préparez-vous selon vos familles et vos tribus,
\par 5 Selon l'ordonnance de David, roi d'Israël, et selon la magnificence de Salomon, son fils, et en vous tenant dans le temple, selon la dignité des familles, vous les Lévites, qui faites le service en présence de vos frères les enfants d'Israël,
\par 6 Offrez la Pâque dans l'ordre, préparez les sacrifices pour vos frères, et observez la Pâque selon le commandement de l'Éternel qui a été donné à Moïse.
\par 7 Et au peuple qui s'y trouvait, Josias donna trente mille agneaux et chevreaux, et trois mille veaux : ces choses furent données de la pension du roi, comme il l'avait promis, au peuple, aux prêtres et aux Lévites. .
\par 8 Et Helkias, Zacharie et Syélus, les gouverneurs du temple, donnèrent aux prêtres, pour la Pâque, deux mille six cents brebis et trois cents veaux.
\par 9 Et Jeconias, et Samaias, et Nathanaël son frère, et Assabias, et Ochiel, et Joram, chefs de milliers, donnèrent aux Lévites pour la Pâque cinq mille brebis et sept cents veaux.
\par 10 Et lorsque ces choses furent accomplies, les prêtres et les Lévites, ayant les pains sans levain, se présentèrent en ordre très convenable selon leurs familles,
\par 11 Et selon les différentes dignités des pères, devant le peuple, pour offrir au Seigneur, comme il est écrit dans le livre de Moïse : et ils firent ainsi le matin.
\par 12 Et ils rôtissaient la Pâque au feu, comme il convient ; quant aux sacrifices, ils les entouraient dans des casseroles et des poêles d'airain d'une bonne saveur,
\par 13 Et ils les placèrent devant tout le peuple ; puis ils préparèrent pour eux-mêmes et pour les prêtres, leurs frères, les fils d'Aaron.
\par 14 Car les prêtres offraient la graisse jusqu'à la nuit ; et les Lévites préparaient pour eux et les prêtres leurs frères, les fils d'Aaron.
\par 15 Les saints chanteurs, fils d'Asaph, étaient dans leur ordre, selon l'ordre de David, à savoir Asaph, Zacharie et Jeduthun, qui était de la suite du roi.
\par 16 Et il y avait des portiers à chaque porte ; il n'était permis à personne de s'éloigner de son service ordinaire : car leurs frères, les Lévites, les préparaient.
\par 17 Ainsi furent accomplies en ce jour-là les choses qui appartenaient aux sacrifices de l'Éternel, afin qu'ils puissent célébrer la Pâque,
\par 18 Et offrez des sacrifices sur l'autel de l'Éternel, selon l'ordre du roi Josias.
\par 19 Les enfants d'Israël qui étaient présents célébrèrent alors la Pâque et la fête des pains sucrés pendant sept jours.
\par 20 Et une telle Pâque n'a pas été célébrée en Israël depuis l'époque du prophète Samuel.
\par 21 Oui, tous les rois d'Israël n'ont pas célébré une Pâque telle que Josias, les prêtres, les Lévites et les Juifs, avec tout Israël qui habitait à Jérusalem.
\par 22 La dix-huitième année du règne de Josias, cette Pâque fut célébrée.
\par 23 Et les œuvres ou Josias étaient debout devant son Seigneur avec un cœur plein de piété.
\par 24 Quant aux choses qui arrivaient de son temps, elles ont été écrites dans les temps anciens, concernant ceux qui ont péché et ont agi méchamment contre le Seigneur entre tous les peuples et tous les royaumes, et comment ils l'ont extrêmement attristé, de sorte que les paroles du Seigneur s'est soulevé contre Israël.
\par 25 Après tous ces actes de Josias, il arriva que Pharaon, roi d'Égypte, vint faire la guerre à Carchamis, sur l'Euphrate, et Josias sortit contre lui.
\par 26 Mais le roi d'Égypte lui envoya dire : Qu'ai-je affaire à toi, ô roi de Judée ?
\par 27 Je ne suis pas envoyé par le Seigneur Dieu contre toi ; car ma guerre est sur l'Euphrate ; et maintenant le Seigneur est avec moi, oui, le Seigneur est avec moi et me précipite en avant : éloigne-toi de moi et ne sois pas contre le Seigneur.
\par 28 Mais Josias ne détourna pas son char, mais entreprit de combattre contre lui, sans tenir compte des paroles du prophète Jérémie prononcées par la bouche du Seigneur :
\par 29 Mais ils combattirent avec lui dans la plaine de Magiddo, et les princes vinrent contre le roi Josias.
\par 30 Alors le roi dit à ses serviteurs : Emmenez-moi hors de la bataille ; car je suis très faible. Et aussitôt ses serviteurs l'emmenèrent hors du combat.
\par 31 Puis il monta sur son deuxième char ; et ramené à Jérusalem, il mourut et fut enterré dans le sépulcre de son père.
\par 32 Dans tout le judaïsme, on pleura Josias ; Jérémie, le prophète, pleura Josias, et les principaux hommes et les femmes le pleurèrent jusqu'à ce jour ; et ce fut une ordonnance qui devait être observée continuellement dans toute la nation d'Israël.
\par 33 Ces choses sont écrites dans le livre des récits des rois de Juda. Tous les actes de Josias, sa gloire, son intelligence de la loi de l'Éternel, les choses qu'il avait faites auparavant et celles qui sont racontées maintenant, sont rapportés dans le livre des rois d'Israël et de Judée.
\par 34 Et le peuple prit Joachaz, fils de Josias, et l'établit roi à la place de Josias, son père, quand il avait vingt-trois ans.
\par 35 Et il régna trois mois en Judée et à Jérusalem ; puis le roi d'Égypte le déposa du règne à Jérusalem.
\par 36 Et il imposa un impôt sur le pays de cent talents d'argent et d'un talent d'or.
\par 37 Le roi d'Égypte fit aussi du roi Joacim, son frère, roi de Judée et de Jérusalem.
\par 38 Et il lia Joacim et les nobles ; mais il apprécia Zaracès, son frère, et le fit sortir d'Égypte.
\par 39 Joacim avait vingt-cinq ans lorsqu'il fut établi roi du pays de Judée et de Jérusalem ; et il fit le mal devant l'Éternel.
\par 40 C'est pourquoi Nabuchodonosor, roi de Babylone, monta contre lui, le lia avec une chaîne d'airain, et le transporta à Babylone.
\par 41 Nabuchodonosor prit aussi des vases sacrés de l'Éternel, les emporta et les plaça dans son propre temple à Babylone.
\par 42 Mais ce qui est rapporté de lui, ainsi que de son impureté et de son impiété, est écrit dans les chroniques des rois.
\par 43 Et Joacim, son fils, régna à sa place : il fut établi roi à l'âge de dix-huit ans ;
\par 44 Et il ne régna que trois mois et dix jours à Jérusalem ; et il fit le mal devant l'Éternel.
\par 45 Au bout d'un an, Nabuchodonosor l'envoya et le fit amener à Babylone avec les vases sacrés de l'Éternel ;
\par 46 Et il établit Sédéchias roi de Judée et de Jérusalem, quand il avait vingt et un ans ; et il régna onze ans :
\par 47 Et il fit aussi ce qui déplaît au Seigneur, et ne se soucia pas des paroles qui lui avaient été dites par le prophète Jérémie de la bouche du Seigneur.
\par 48 Et après que le roi Nabuchodonosor lui eut fait jurer par le nom de l'Éternel, il se renonça et se révolta ; et endurcissant son cou et son cœur, il transgressa les lois du Seigneur, le Dieu d'Israël.
\par 49 Les gouverneurs du peuple et les prêtres aussi firent beaucoup de choses contre les lois, et passèrent toutes les souillures de toutes les nations, et profanèrent le temple de l'Éternel, qui était sanctifié à Jérusalem.
\par 50 Mais le Dieu de leurs pères envoya son messager pour les rappeler, parce qu'il les avait épargnés, ainsi que son tabernacle.
\par 51 Mais ils se moquaient de ses messagers ; et voyez, quand le Seigneur leur parlait, ils se moquaient de ses prophètes :
\par 52 Au point que, irrité contre son peuple à cause de sa grande impiété, il ordonna aux rois des Chaldéens de monter contre eux ;
\par 53 Qui ont tué leurs jeunes gens avec l'épée, même dans l'enceinte de leur saint temple, et qui n'ont épargné ni jeune homme ni servante, ni vieillard ni enfant parmi eux ; car il a tout livré entre leurs mains.
\par 54 Et ils prirent tous les ustensiles saints de l'Éternel, depuis les grands jusqu'aux petits, avec les ustensiles de l'arche de Dieu et les trésors du roi, et les emportèrent à Babylone.
\par 55 Quant à la maison de l'Éternel, ils l'incendièrent, ils démolirent les murs de Jérusalem et mirent le feu à ses tours.
\par 56 Et quant à ses choses glorieuses, elles ne cessèrent jamais jusqu'à ce qu'elles les aient toutes consumées et réduites à néant ; et il emporta à Babylone le peuple qui n'avait pas été tué par l'épée.
\par 57 Qui lui fut serviteur, lui et ses enfants, jusqu'au règne des Perses, pour accomplir la parole de l'Éternel prononcée par la bouche de Jérémie :
\par 58 Jusqu'à ce que le pays ait joui de ses sabbats, elle se reposera pendant tout le temps de sa désolation, jusqu'à la fin de soixante-dix ans.

\chapter{2}

\par 1 La première année de Cyrus, roi des Perses, afin que s'accomplisse la parole de l'Éternel, qu'il avait promise par la bouche de Jérémie ;
\par 2 L'Éternel suscita l'esprit de Cyrus, roi des Perses, et il fit une proclamation dans tout son royaume, et aussi en écrivant :
\par 3 disant : Ainsi parle Cyrus, roi des Perses : L'Éternel d'Israël, le Seigneur très-haut, m'a établi roi du monde entier,
\par 4 Et il m'a ordonné de lui bâtir une maison à Jérusalem, dans la communauté juive.
\par 5 Si quelqu'un d'entre vous est de son peuple, que le Seigneur, son Seigneur, soit avec lui, et qu'il monte à Jérusalem en Judée, pour bâtir la maison du Seigneur d'Israël, car c'est le Seigneur qui habite à Jérusalem.
\par 6 Quiconque habite donc dans les environs, que ceux, dis-je, ses voisins, l'aident avec de l'or et de l'argent,
\par 7 Avec des offrandes, des chevaux, du bétail, et d'autres choses qui ont été prévues par vœu, pour le temple de l'Éternel à Jérusalem.
\par 8 Alors les chefs des familles de Judée et de la tribu de Benjamin se levèrent ; les prêtres aussi, les Lévites, et tous ceux que l'Éternel avait décidé de monter et de bâtir une maison à l'Éternel à Jérusalem,
\par 9 Et ceux qui habitaient autour d'eux et les aidaient en toutes choses avec de l'argent et de l'or, des chevaux et du bétail, et avec de très nombreux cadeaux gratuits d'un grand nombre dont les esprits étaient éveillés à cela.
\par 10 Le roi Cyrus fit aussi sortir les vases sacrés que Nabuchodonosor avait emportés de Jérusalem et qu'il avait placés dans son temple d'idoles.
\par 11 Cyrus, roi des Perses, les ayant fait sortir, les livra à Mithridate, son trésorier.
\par 12 Et c'est par lui qu'ils furent livrés à Sanabassar, gouverneur de la Judée.
\par 13 Et tel était leur nombre ; Mille coupes d'or et mille d'argent, vingt-neuf brasiers d'argent, trente coupes d'or et deux mille quatre cent dix d'argent, et mille autres ustensiles.
\par 14 Tous les ustensiles d'or et d'argent qui furent emportés furent donc au nombre de cinq mille quatre cent soixante-neuf.
\par 15 Ceux-ci furent ramenés par Sanabassar, avec ceux des captifs, de Babylone à Jérusalem.
\par 16 Mais au temps d'Artexerxès, roi des Perses, Belemus, Mithridate, Tabellius, Rathumus, Beeltethmus, Semellius le secrétaire, et d'autres qui étaient en commission avec eux, demeurant à Samarie et ailleurs, écrivirent à lui contre ceux qui habitaient en Judée et à Jérusalem, ces lettres suivent ;
\par 17 Au roi Artexerxès, notre seigneur, à tes serviteurs, Rathumus, le conteur, et Semellius, le scribe, et le reste de leur conseil, et les juges qui sont en Célosyrie et en Phénice.
\par 18 Que le seigneur roi sache que les Juifs qui sont montés de chez vous vers nous, étant entrés dans Jérusalem, ville rebelle et méchante, bâtissent les places, réparent les murailles et posent les fondements du temple.
\par 19 Or, si cette ville et ses murs sont reconstruits, non seulement ils refuseront de payer le tribut, mais encore ils se rebelleront contre les rois.
\par 20 Et puisque les choses concernant le temple sont maintenant en main, nous pensons qu'il convient de ne pas négliger une telle question,
\par 21 Mais parler à notre seigneur le roi, afin que, si tu veux, cela soit recherché dans les livres de tes pères :
\par 22 Et tu trouveras dans les chroniques ce qui est écrit concernant ces choses, et tu comprendras que cette ville était rebelle, troublant les rois et les villes.
\par 23 Et que les Juifs étaient rebelles, et y soulevaient toujours des guerres ; c'est pour cette raison que même cette ville est devenue désolée.
\par 24 C'est pourquoi maintenant nous te déclarons, ô seigneur le roi, que si cette ville est rebâtie et que ses murs sont relevés, tu n'auras désormais plus de passage en Célosyrie et en Phénicie.
\par 25 Alors le roi écrivit de nouveau à Rathumus, le conteur, à Beeltethmus, à Semellius, le scribe, et à tous ceux qui étaient en poste, et aux habitants de Samarie, de Syrie et de Phénice, de cette manière :
\par 26 J'ai lu l'épître que vous m'avez envoyée ; c'est pourquoi j'ai ordonné de faire des recherches diligentes, et il a été constaté que cette ville pratiquait dès le début contre les rois ;
\par 27 Et les hommes qui y étaient étaient livrés à la rébellion et à la guerre; et il y avait à Jérusalem des rois puissants et féroces, qui régnaient et exigeaient des tributs en Célosyrie et en Phénicie.
\par 28 Maintenant, j'ai ordonné d'empêcher ces hommes de rebâtir la ville, et de prendre garde à ce qu'il n'y ait plus rien à y faire ;
\par 29 Et pour que ces méchants ouvriers n'aillent pas plus loin, au grand dam des rois,
\par 30 Alors le roi Artexerxès ayant lu ses lettres, Rathumus, le scribe Semellius et les autres qui étaient en commission avec eux, se dirigèrent en toute hâte vers Jérusalem avec une troupe de cavaliers et une multitude de gens en bataille, et commencèrent à empêcher les constructeurs ; et la construction du temple à Jérusalem cessa jusqu'à la deuxième année du règne de Darius, roi des Perses.

\chapter{3}

\par 1 Or, lorsque Darius régnait, il fit une grande fête à tous ses sujets, et à toute sa maison, et à tous les princes de Médie et de Perse,
\par 2 Et à tous les gouverneurs, capitaines et lieutenants qui étaient sous lui, depuis l'Inde jusqu'à l'Éthiopie, de cent vingt-sept provinces.
\par 3 Et lorsqu'ils eurent mangé et bu, et qu'étant rassasiés ils furent rentrés chez eux, le roi Darius entra dans sa chambre, s'endormit et se réveilla peu après.
\par 4 Alors trois jeunes gens, qui étaient de la garde qui gardait le corps du roi, se parlèrent l'un à l'autre ;
\par 5 Que chacun de nous prononce une sentence : à celui qui vaincra, et dont la sentence paraîtra plus sage que les autres, le roi Darius lui offrira de grands présents et de grandes choses en signe de victoire :
\par 6 Comme pour être vêtu de pourpre, pour boire dans l'or, et pour dormir sur l'or, et un char avec des brides d'or, et un bandeau de fin lin, et une chaîne autour du cou :
\par 7 Et il s'assiéra à côté de Darius à cause de sa sagesse, et il sera appelé Darius son cousin.
\par 8 Et alors chacun écrivit sa phrase, la scella et la déposa sous le roi Darius son oreiller ;
\par 9 Et il dit que, quand le roi sera ressuscité, certains lui donneront les écrits ; et du côté duquel le roi et les trois princes de Perse jugeront que sa sentence est la plus sage, c'est à lui que sera donnée la victoire, comme il a été fixé.
\par 10 Le premier a écrit : Le vin est le plus fort.
\par 11 Le second écrit : Le roi est le plus fort.
\par 12 Le troisième écrit : Les femmes sont les plus fortes ; mais par-dessus tout, la vérité remporte la victoire.
\par 13 Or, lorsque le roi se leva, ils prirent leurs écrits et les lui remirent, et ainsi il les lut :
\par 14 Et il envoya appeler tous les princes de Perse et de Médie, et les gouverneurs, et les capitaines, et les lieutenants, et les officiers en chef ;
\par 15 Et il s'assit sur le siège royal du jugement ; et les écrits furent lus devant eux.
\par 16 Et il dit : Appelez les jeunes gens, et ils prononceront leurs propres jugements. Alors ils furent appelés et arrivèrent.
\par 17 Et il leur dit : Déclarez-nous ce que vous pensez des écrits. Alors commença le premier, qui avait parlé de la force du vin ;
\par 18 Et il dit ainsi : Ô vous les hommes, comme le vin est extrêmement fort ! cela fait égarer tous les hommes qui en boivent :
\par 19 L'esprit du roi et celui de l'orphelin de père ne font qu'un ; de l'esclave et de l'homme libre, du pauvre et du riche :
\par 20 Cela change aussi toute pensée en joie et en gaieté, de sorte qu'un homme ne se souvient ni du chagrin ni de la dette.
\par 21 Et cela enrichit chaque cœur, de sorte qu'un homme ne se souvient ni du roi ni du gouverneur ; et cela fait dire toutes choses par talents :
\par 22 Et lorsqu'ils sont dans leurs coupes, ils oublient leur amour envers leurs amis et leurs frères, et peu après tirent leurs épées :
\par 23 Mais quand ils sont à cause du vin, ils ne se souviennent pas de ce qu'ils ont fait.
\par 24 Ô vous les hommes, le vin n'est-il pas le plus fort qui force à faire ainsi ? Et après avoir ainsi parlé, il se tut.

\chapter{4}

\par 1 Alors le second, qui avait parlé de la force du roi, commença à dire :
\par 2 Ô vous les hommes, les hommes n'excellent-ils pas en force pour dominer la mer et la terre et tout ce qui s'y trouve ?
\par 3 Mais pourtant le roi est plus puissant : car il est le maître de toutes ces choses et il les domine ; et tout ce qu'il leur commande, ils le font.
\par 4 S'il leur ordonne de faire la guerre les uns aux autres, ils le font ; s'il les envoie contre les ennemis, ils s'en vont et détruisent les murs et les tours des montagnes.
\par 5 Ils tuent et sont tués, et ne transgressent pas le commandement du roi : s'ils obtiennent la victoire, ils apportent tout au roi, ainsi que le butin, comme tout le reste.
\par 6 De même, ceux qui ne sont pas soldats et qui n'ont rien à faire à la guerre, mais qui se livrent à la corvée, lorsqu'ils ont récolté ce qu'ils avaient semé, l'apportent au roi, et s'obligent les uns les autres à lui payer un tribut.
\par 7 Et pourtant il n'est qu'un seul homme : s'il ordonne de tuer, ils tuent ; s'il commande d'épargner, ils épargnent ;
\par 8 S'il commande de frapper, ils frappent ; s'il commande de désoler, ils désolent ; s’il commande de bâtir, ils bâtissent ;
\par 9 S'il ordonne de couper, ils coupent ; s'il ordonne de planter, ils plantent.
\par 10 Ainsi tout son peuple et ses armées lui obéissent ; il se couche, il mange et boit, et il se repose.
\par 11 Et ceux-ci veillent autour de lui, et personne ne peut s'éloigner pour vaquer à ses affaires, ni lui désobéir en quoi que ce soit.
\par 12 Ô vous les hommes, comment le roi ne serait-il pas le plus puissant, quand on lui obéit de telle manière ? Et il a tenu sa langue.
\par 13 Alors le troisième, qui avait parlé des femmes et de la vérité, (c'était Zorobabel) commença à parler.
\par 14 Ô vous les hommes, ce n'est pas le grand roi, ni la multitude des hommes, ni le vin qui excelle ; Qui donc les gouverne, ou a-t-il la seigneurie sur eux ? ce ne sont pas des femmes ?
\par 15 Les femmes ont porté le roi et tout le peuple qui dirigeait sur mer et sur terre.
\par 16 Même ceux d'entre eux sont venus, et ils ont nourri ceux qui ont planté les vignes d'où vient le vin.
\par 17 Ceux-ci confectionnent aussi des vêtements pour hommes ; ceux-ci apportent la gloire aux hommes ; et sans les femmes, les hommes ne peuvent exister.
\par 18 Oui, et si les hommes ont amassé de l'or et de l'argent, ou toute autre bonne chose, n'aiment-ils pas une femme qui est belle en grâce et en beauté ?
\par 19 Et laissant aller toutes ces choses, ils ne restent pas bouche bée, et même la bouche ouverte ne fixent pas leurs yeux sur elle ; et tous les hommes ne la désirent-ils pas plus que l'argent, l'or, ou toute autre bonne chose ?
\par 20 Un homme quitte son propre père qui l'a élevé et son propre pays, et s'attache à sa femme.
\par 21 Il ne s'attache pas à passer sa vie avec sa femme et ne se souvient ni de son père, ni de sa mère, ni de son pays.
\par 22 Par ceci aussi, vous devez savoir que les femmes ont domination sur vous : ne travaillez-vous pas et ne peinez-vous pas, et ne donnez-vous pas et n'apportez-vous pas tout à la femme ?
\par 23 Oui, un homme prend son épée, et s'en va pour voler et dérober, pour naviguer sur la mer et sur les fleuves ;
\par 24 Et il regarde un lion, et il marche dans les ténèbres ; et quand il a volé, gâté et volé, il l'apporte à son amour.
\par 25 C'est pourquoi l'homme aime sa femme plus que son père ou sa mère.
\par 26 Oui, il y en a beaucoup qui sont devenus fous à cause des femmes et sont devenus serviteurs à cause d'elles.
\par 27 Beaucoup aussi ont péri, se sont égarés et ont péché à cause des femmes.
\par 28 Et maintenant, vous ne me croyez pas ? le roi n'est-il pas grand en sa puissance ? toutes les régions ne craignent-elles pas de le toucher ?
\par 29 Pourtant je le vis, ainsi que Apame, la concubine du roi, fille de l'admirable Bartacus, assis à la droite du roi,
\par 30 Et prenant la couronne de la tête du roi, et la mettant sur sa propre tête ; elle frappa aussi le roi avec sa main gauche.
\par 31 Et pourtant, malgré tout cela, le roi restait bouche bée et la regardait la bouche ouverte : si elle se moquait de lui, il riait aussi ; mais si elle prenait quelque déplaisir contre lui, le roi se plaisait à la flatter, afin qu'elle puisse se réconcilier. à lui encore.
\par 32 Ô vous les hommes, comment se fait-il que les femmes ne soient pas fortes, puisqu'elles agissent ainsi ?
\par 33 Alors le roi et les princes se regardèrent ; alors il se mit à parler de la vérité.
\par 34 Ô vous les hommes, les femmes ne sont-elles pas fortes ? La terre est grande, les cieux sont hauts, le soleil est rapide dans sa course, car il entoure les cieux tout autour et reprend sa course vers son lieu en un jour.
\par 35 N'est-il pas grand celui qui fait ces choses ? c'est pourquoi la vérité est grande et plus forte que toutes choses.
\par 36 Toute la terre crie sur la vérité, et le ciel la bénit : toutes les œuvres tremblent et tremblent devant elle, et en elle il n'y a rien d'injuste.
\par 37 Le vin est méchant, le roi est méchant, les femmes sont méchantes, tous les enfants des hommes sont méchants, et telles sont toutes leurs mauvaises actions ; et il n'y a aucune vérité en eux ; dans leur injustice aussi, ils périront.
\par 38 Quant à la vérité, elle dure et est toujours forte ; il vit et conquiert pour toujours.
\par 39 Chez elle, il n'y a pas d'acceptation de personnes ni de récompenses ; mais elle fait les choses qui sont justes et s'abstient de toutes choses injustes et mauvaises ; et tous les hommes aiment ses œuvres.
\par 40 Il n'y a pas non plus d'injustice dans son jugement ; et elle est la force, le royaume, la puissance et la majesté de tous les âges. Béni soit le Dieu de vérité.
\par 41 Et sur ce, il se tut. Et tout le peuple cria alors et dit : Grande est la vérité et puissante au-dessus de toutes choses.
\par 42 Alors le roi lui dit : Demande ce que tu veux de plus que ce qui est prescrit dans l'écriture, et nous te le donnerons, parce que tu es trouvé le plus sage ; et tu seras assis à côté de moi, et tu seras appelé mon cousin.
\par 43 Alors il dit au roi : Souviens-toi du vœu que tu as fait de bâtir Jérusalem, le jour où tu viendras dans ton royaume,
\par 44 Et de renvoyer tous les vaisseaux emportés hors de Jérusalem, que Cyrus avait mis à part lorsqu'il avait juré de détruire Babylone et de les y renvoyer.
\par 45 Tu as aussi juré de reconstruire le temple que les Édomites ont brûlé lorsque la Judée fut dévastée par les Chaldéens.
\par 46 Et maintenant, ô seigneur le roi, voici ce que j'exige et ce que je désire de toi, et ceci est la libéralité princière qui procède de toi-même : je désire donc que tu accomplisses le vœu dont l'accomplissement avec ton tu as voué ta propre bouche au Roi des cieux.
\par 47 Alors le roi Darius se leva, l'embrassa et écrivit pour lui des lettres à tous les trésoriers, lieutenants, capitaines et gouverneurs, afin qu'ils le transportent en toute sécurité, lui et tous ceux qui montent avec lui vers bâtissez Jérusalem.
\par 48 Il écrivit aussi des lettres aux lieutenants qui étaient en Célosyrie et en Phénicie, et à ceux du Liban, pour qu'ils apportent du bois de cèdre du Liban à Jérusalem, et qu'ils construisent la ville avec lui.
\par 49 De plus, il écrivit pour tous les Juifs qui sortaient de son royaume pour se rendre dans la communauté juive, concernant leur liberté, qu'aucun officier, aucun chef, aucun lieutenant, ni trésorier ne devait entrer de force dans leurs portes ;
\par 50 Et que tout le pays qu'ils possèdent soit libre et sans tribut ; et que les Edomites devaient céder les villages des Juifs qu'ils détenaient alors :
\par 51 Oui, qu'il y ait chaque année vingt talents pour la construction du temple, jusqu'au moment où il serait construit ;
\par 52 Et dix autres talents par an, pour maintenir chaque jour les holocaustes sur l'autel, comme ils avaient l'ordre d'en offrir dix-sept.
\par 53 Et que tous ceux qui sont partis de Babylone pour bâtir la ville auraient la liberté, ainsi qu'eux et leur postérité, et tous les prêtres qui s'en sont allés.
\par 54 Il écrivit aussi sur les charges et les vêtements des prêtres dans lesquels ils exercent leur ministère ;
\par 55 Et de même pour les charges des Lévites, qui leur seraient confiées jusqu'au jour où la maison serait achevée et où Jérusalem serait bâtie.
\par 56 Et il ordonna de donner à tous ceux qui tenaient la ville des pensions et des salaires.
\par 57 Il renvoya aussi de Babylone tous les ustensiles que Cyrus avait mis à part ; et tout ce que Cyrus avait ordonné, il le fit aussi et l'envoya à Jérusalem.
\par 58 Or, quand ce jeune homme fut sorti, il leva sa face vers le ciel, vers Jérusalem, et loua le Roi des cieux,
\par 59 Et il dit : De toi vient la victoire, de toi vient la sagesse, et à toi est la gloire, et je suis ton serviteur.
\par 60 Bienheureux sois-tu, qui m'as donné la sagesse ; car je te rends grâce, Seigneur de nos pères.
\par 61 Et ainsi il prit les lettres, et sortit, et vint à Babylone, et il le raconta à tous ses frères.
\par 62 Et ils louèrent le Dieu de leurs pères, parce qu'il leur avait donné la liberté et la liberté
\par 63 Pour monter et rebâtir Jérusalem et le temple qui porte son nom ; et ils se régalèrent avec des instruments de musique et de joie pendant sept jours.

\chapter{5}

\par 1 Après cela, les principaux hommes des familles furent choisis selon leurs tribus, pour monter avec leurs femmes, leurs fils et leurs filles, avec leurs serviteurs et servantes, et leur bétail.
\par 2 Et Darius envoya avec eux mille cavaliers, jusqu'à ce qu'ils les eussent ramenés sains et saufs à Jérusalem, et avec des [instruments] de musique, des tabrets et des flûtes.
\par 3 Et tous leurs frères jouaient, et il les fit monter avec eux.
\par 4 Et voici les noms des hommes qui montaient, selon leurs familles, selon leurs tribus, selon leurs différents chefs.
\par 5 Les prêtres, fils de Phinées, fils d'Aaron : Jésus, fils de Josedec, fils de Saraias, et Joacim, fils de Zorobabel, fils de Salathiel, de la maison de David, de la famille de Phares. , de la tribu de Juda ;
\par 6 Qui prononça de sages paroles devant Darius, roi de Perse, la deuxième année de son règne, au mois de Nisan, qui est le premier mois.
\par 7 Et ce sont ceux des Juifs qui sont revenus de la captivité, où ils demeuraient comme étrangers, et que Nabuchodonosor, roi de Babylone, avait emmenés à Babylone.
\par 8 Et ils retournèrent à Jérusalem et dans les autres parties de la communauté juive, chacun dans sa ville, et ils vinrent avec Zorobabel, avec Jésus, Néhémie, Zacharie, Résaïas, Enenius et Mardochée. Beelsarus, Aspharasus, Reelius, Roimus et Baana, leurs guides.
\par 9 Le nombre des membres de la nation et de leurs gouverneurs, fils de Phoros, deux mille cent soixante-douze ; les fils de Saphat, quatre cent soixante-douze :
\par 10 Les fils d'Arès, sept cent cinquante-six :
\par 11 Les fils de Phaath Moab, deux mille huit cent douze :
\par 12 Les fils d'Elam, mille deux cent cinquante-quatre : les fils de Zathul, neuf cent quarante-cinq ; les fils de Corbe, sept cent cinq ; les fils de Bani, six cent quarante-huit :
\par 13 Les fils de Bebaï, six cent vingt-trois : les fils de Sadas, trois mille deux cent vingt-deux :
\par 14 Les fils d'Adonikam, six cent soixante-sept : les fils de Bagoi, deux mille soixante-six ; les fils d'Adin, quatre cent cinquante-quatre :
\par 15 Les fils d'Aterezias, quatre-vingt-douze : les fils de Ceilan et d'Azetas, soixante-sept ; les fils d'Azuran, quatre cent trente-deux :
\par 16 Les fils d'Ananias, cent un : les fils d'Arom, trente-deux ; et les fils de Bassa, trois cent vingt-trois ; les fils d'Azephurith, cent deux :
\par 17 Les fils de Meterus, trois mille cinq ; les fils de Bethlomon, cent vingt-trois :
\par 18 Ceux de Netophah, cinquante-cinq ; ceux d'Anathoth, cent cinquante-huit ; ceux de Bethsamos, quarante-deux ;
\par 19 Ceux de Kiriathiarius, vingt-cinq : ceux de Caphira et de Beroth, sept cent quarante-trois ; ceux de Pira, sept cents ;
\par 20 Ceux de Chadias et d'Ammidoi, quatre cent vingt-deux ; ceux de Cirama et de Gabdes, six cent vingt et un ;
\par 21 Ceux de Macalon, cent vingt-deux : ceux de Betolius, cinquante-deux ; les fils de Néphis, cent cinquante-six :
\par 22 Les fils de Calamolalus et d'Onus, sept cent vingt-cinq ; les fils de Jéréchus, deux cent quarante-cinq :
\par 23 Les fils d'Anne, trois mille trois cent trente.
\par 24 Les prêtres : les fils de Jeddu, fils de Jésus, parmi les fils de Sanassib, neuf cent soixante-douze ; les fils de Meruth, mille cinquante-deux :
\par 25 Les fils de Phassaron, mille quarante-sept ; les fils de Carme, mille dix-sept.
\par 26 Les Lévites : les fils de Jessué, Cadmiel, Banuas et Sudias, soixante-quatorze.
\par 27 Les saints chanteurs : les fils d'Asaph, cent vingt-huit.
\par 28 Les portiers : les fils de Salum, les fils de Jatal, les fils de Talmon, les fils de Dacobi, les fils de Teta, les fils de Sami, en tout cent trente-neuf.
\par 29 Les serviteurs du temple : les fils d'Esaü, les fils d'Asipha, les fils de Tabaoth, les fils de Ceras, les fils de Sud, les fils de Phaleas, les fils de Labana, les fils de Graba,
\par 30 Les fils d'Acua, les fils d'Uta, les fils de Cetab, les fils d'Agaba, les fils de Subaï, les fils d'Anan, les fils de Cathua, les fils de Geddur,
\par 31 Les fils d'Airus, les fils de Daisan, les fils de Noéba, les fils de Chaseba, les fils de Gazera, les fils d'Azia, les fils de Phinées, les fils d'Azaré, les fils de Bastaï, les fils d'Asana, les fils de Meani, les fils de Naphisi, les fils d'Acub, les fils d'Acipha, les fils d'Assur, les fils de Pharacim, les fils de Basaloth,
\par 32 Les fils de Meeda, les fils de Coutha, les fils de Charea, les fils de Charcus, les fils d'Aserer, les fils de Thomoi, les fils de Nasith, les fils d'Atipha.
\par 33 Les fils des serviteurs de Salomon : les fils d'Azaphion, les fils de Pharira, les fils de Jeeli, les fils de Lozon, les fils d'Israël, les fils de Sapheth,
\par 34 Les fils de Hagia, les fils de Pharacareth, les fils de Sabi, les fils de Sarothie, les fils de Masias, les fils de Gar, les fils d'Addus, les fils de Suba, les fils d'Apherra, les fils de Barodis, les fils de Sabat, les fils d'Allom.
\par 35 Tous les ministres du temple et les fils des serviteurs de Salomon étaient trois cent soixante-douze.
\par 36 Ceux-ci arrivèrent de Thermeleth et de Thelersas, Charaathalar les conduisant, et Aalar ;
\par 37 Ils ne purent pas non plus montrer leurs familles, ni leur race, comment ils étaient d'Israël : les fils de Ladan, le fils de Ban, les fils de Necodan, six cent cinquante-deux.
\par 38 Parmi les prêtres qui usurpèrent les fonctions du sacerdoce et qui ne furent pas retrouvés, il y avait les fils d'Obdia, les fils d'Accoz, les fils d'Addus, qui avait épousé Augia, l'une des filles de Barzelus, et qui portait le même nom que lui.
\par 39 Et comme la description de la parenté de ces hommes fut recherchée dans le registre, et ne fut pas trouvée, ils furent retirés de l'exercice de l'office du sacerdoce :
\par 40 Car Néhémie et Atharias leur dirent de ne pas participer aux choses saintes, jusqu'à ce qu'un grand prêtre se lève, revêtu de doctrine et de vérité.
\par 41 Ainsi, pour Israël, âgés de douze ans et au-dessus, ils étaient tous au nombre de quarante mille, sans compter les serviteurs et les servantes deux mille trois cent soixante.
\par 42 Leurs serviteurs et servantes étaient au nombre de sept mille trois cent quarante-sept : les chanteurs et les chanteuses, deux cent quarante-cinq :
\par 43 Quatre cent trente-cinq chameaux, sept mille trente-six chevaux, deux cent quarante-cinq mulets, cinq mille cinq cent vingt-cinq bêtes sous le joug.
\par 44 Et certains des chefs de leurs familles, lorsqu'ils arrivèrent au temple de Dieu qui est à Jérusalem, jurèrent de rétablir la maison à son propre endroit, selon leurs possibilités,
\par 45 Et de verser dans le trésor sacré des œuvres mille livres d'or, cinq mille livres d'argent et cent vêtements sacerdotaux.
\par 46 Ainsi habitaient les prêtres, les Lévites et le peuple à Jérusalem et dans les campagnes, les chanteurs et les portiers ; et tout Israël dans ses villages.
\par 47 Mais lorsque le septième mois fut proche, et que les enfants d'Israël furent chacun à sa place, ils se rassemblèrent tous d'un commun accord dans l'espace ouvert de la première porte qui est vers l'orient.
\par 48 Alors Jésus, fils de Josedec, et ses frères les prêtres, et Zorobabel, fils de Salathiel, et ses frères se levèrent, et préparèrent l'autel du Dieu d'Israël,
\par 49 Pour y offrir des holocaustes, comme il est expressément ordonné dans le livre de Moïse, l'homme de Dieu.
\par 50 Et des autres nations du pays se rassemblèrent auprès d'eux, et ils érigèrent l'autel sur sa propre place, parce que toutes les nations du pays étaient inimitiées contre eux et les opprimaient ; et ils offraient des sacrifices selon le temps, et des holocaustes à l'Éternel, matin et soir.
\par 51 Ils célébraient aussi la fête des tabernacles, comme il est ordonné dans la loi, et offraient chaque jour des sacrifices, comme il convient.
\par 52 Et après cela, les oblations continuelles, et le sacrifice des sabbats, et des nouvelles lunes, et de toutes les saintes fêtes.
\par 53 Et tous ceux qui avaient fait un vœu à Dieu commencèrent à offrir des sacrifices à Dieu dès le premier jour du septième mois, bien que le temple de l'Éternel ne fût pas encore construit.
\par 54 Et ils donnèrent aux maçons et aux charpentiers de l'argent, de la nourriture et des boissons, avec gaieté.
\par 55 Ils donnèrent aussi aux habitants de Sidon et à Tyr des chariots pour amener du Liban des cèdres, qui devaient être transportés sur des chars jusqu'au port de Joppé, selon l'ordre que leur avait ordonné Cyrus, roi des Perses.
\par 56 La seconde année et le second mois après son arrivée, Zorobabel, fils de Salathiel, et Jésus, fils de Josédec, avec leurs frères, les prêtres, les Lévites et tous ceux qui étaient revenus de la captivité à Jérusalem, se mirent à construire le temple de Dieu à Jérusalem :
\par 57 Et ils posèrent les fondations de la maison de Dieu le premier jour du deuxième mois, la deuxième année après leur arrivée dans les Juifs et à Jérusalem.
\par 58 Et ils confièrent aux Lévites, dès l'âge de vingt ans, la responsabilité des œuvres de l'Éternel. Alors Jésus se leva, avec ses fils et ses frères, et Cadmiel son frère, et les fils de Madiabun, avec les fils de Joda, fils d'Eliadun, avec leurs fils et frères, tous Lévites, qui étaient d'un commun accord pour diriger l'affaire, travaillant à faire avancer les œuvres dans la maison de Dieu. Les ouvriers bâtirent donc le temple du Seigneur.
\par 59 Et les prêtres se tenaient debout, vêtus de leurs vêtements avec des instruments de musique et des trompettes ; et les Lévites, fils d'Asaph, avaient des cymbales,
\par 60 Chantant des chants de remerciement et louant l'Éternel, comme l'avait ordonné David, roi d'Israël.
\par 61 Et ils chantèrent à haute voix des chants à la louange de l'Éternel, parce que sa miséricorde et sa gloire sont à jamais dans tout Israël.
\par 62 Et tout le peuple sonnait des trompettes et criait à haute voix, chantant des chants de remerciement à l'Éternel pour le relèvement de la maison de l'Éternel.
\par 63 Parmi les prêtres et les Lévites, et parmi les chefs de leurs familles, les anciens qui avaient vu l'ancienne maison arrivèrent à la construction de celle-ci avec des pleurs et de grands cris.
\par 64 Mais beaucoup, avec des trompettes et de la joie, criaient à haute voix :
\par 65 De sorte que les trompettes ne pouvaient pas être entendues à cause des pleurs du peuple; cependant la multitude sonnait merveilleusement, de sorte qu'on l'entendait de loin.
\par 66 C'est pourquoi, lorsque les ennemis de la tribu de Juda et de Benjamin l'entendirent, ils comprirent ce que signifiait ce bruit des trompettes.
\par 67 Et ils s'aperçurent que ceux qui étaient captifs bâtissaient le temple à l'Éternel, le Dieu d'Israël.
\par 68 Alors ils allèrent trouver Zorobabel et Jésus, et vers les chefs de famille, et leur dirent : Nous bâtirons avec vous.
\par 69 Car nous aussi, comme vous, obéissons à votre Seigneur et lui offrons des sacrifices depuis l'époque d'Azbazareth, roi des Assyriens, qui nous a amenés ici.
\par 70 Alors Zorobabel, Jésus et les chefs des familles d'Israël leur dirent : Ce n'est pas à nous et à vous de bâtir ensemble une maison au Seigneur notre Dieu.
\par 71 Nous seuls bâtirons à l'Éternel d'Israël, comme nous l'a ordonné Cyrus, roi des Perses.
\par 72 Mais les païens du pays qui pesaient sur les habitants de la Judée et les tenaient droits, empêchaient leur construction ;
\par 73 Et par leurs complots secrets, et par les persuasion et les troubles populaires, ils empêchèrent l'achèvement de la construction pendant tout le temps que vécut le roi Cyrus ; ils furent donc empêchés de construire pendant l'espace de deux ans, jusqu'au règne de Darius.

\chapter{6}

\par 1 La deuxième année du règne de Darius Aggeus et de Zacharie, fils d'Addo, les prophètes, prophétisèrent aux Juifs de Juda et de Jérusalem au nom du Seigneur, le Dieu d'Israël, qui était sur eux.
\par 2 Alors Zorobabel, fils de Salatiel, et Jésus, fils de Josedec, se levèrent et commencèrent à bâtir la maison de l'Éternel à Jérusalem, les prophètes de l'Éternel étant avec eux et les aidant.
\par 3 Au même moment, Sisinnes, gouverneur de Syrie et de Phénice, vint vers eux, avec Sathrabuzanes et ses compagnons, et leur dit :
\par 4 Par qui bâtissez-vous cette maison et ce toit, et faites-vous tout le reste ? et qui sont les ouvriers qui font ces choses ?
\par 5 Néanmoins les anciens des Juifs obtinrent faveur, parce que l'Éternel avait visité les captifs ;
\par 6 Et ils ne furent pas empêchés de construire, jusqu'au moment où une signification fut donnée à Darius à leur sujet, et une réponse reçue.
\par 7 La copie des lettres que Sisinnes, gouverneur de Syrie et de Phénice, et Sathrabuzanes, avec leurs compagnons, dirigeants de Syrie et de Phénice, écrivirent et envoyèrent à Darius ; Au roi Darius, salut :
\par 8 Que tout soit connu du roi, notre seigneur, que, étant arrivés dans le pays de Judée, et entrés dans la ville de Jérusalem, nous avons trouvé dans la ville de Jérusalem les anciens des Juifs qui étaient en captivité.
\par 9 Construisez une maison à l'Éternel, grande et neuve, avec des pierres de taille et de luxe, et du bois déjà posé sur les murs.
\par 10 Et ces travaux sont exécutés avec une grande rapidité, et l'ouvrage se poursuit avec succès entre leurs mains, et il est fait en toute gloire et diligence.
\par 11 Alors nous avons interrogé ces anciens, en disant : Par quel commandement bâtissez-vous cette maison et posez-vous les fondations de ces ouvrages ?
\par 12 C'est pourquoi, afin que nous puissions te donner la connaissance par écrit, nous leur avons demandé qui étaient les principaux exécutants, et nous leur avons demandé les noms écrits de leurs principaux hommes.
\par 13 Alors ils nous donnèrent cette réponse : Nous sommes les serviteurs du Seigneur qui a fait le ciel et la terre.
\par 14 Et quant à cette maison, elle fut bâtie il y a de nombreuses années par un roi d'Israël, grand et fort, et elle fut achevée.
\par 15 Mais lorsque nos pères ont irrité Dieu et ont péché contre l'Éternel d'Israël qui est dans les cieux, il les a livrés au pouvoir de Nabuchodonosor, roi de Babylone, des Chaldéens ;
\par 16 Qui a démoli la maison, et l'a incendiée, et a emmené le peuple captif à Babylone.
\par 17 Mais la première année où le roi Cyrus régna sur le pays de Babylone, le roi Cyrus écrivit de reconstruire cette maison.
\par 18 Et les ustensiles sacrés d'or et d'argent que Nabuchodonosor avait emportés hors de la maison de Jérusalem et les avait placés dans son propre temple, ceux que le roi Cyrus fit sortir du temple de Babylone, et ils furent livré à Zorobabel et à Sanabassarus, le souverain,
\par 19 Avec ordre d'emporter les mêmes ustensiles et de les mettre dans le temple de Jérusalem ; et que le temple du Seigneur soit construit à sa place.
\par 20 Alors le même Sanabassarus, étant venu ici, posa les fondations de la maison de l'Éternel à Jérusalem ; et à partir de ce moment-là, étant encore un bâtiment, il n'est pas encore complètement terminé.
\par 21 Maintenant donc, si le roi le trouve bon, qu'on fasse des recherches dans les archives du roi Cyrus.
\par 22 Et s'il s'avère que la construction de la maison de l'Éternel à Jérusalem a été faite avec le consentement du roi Cyrus, et si notre roi, notre seigneur, le souhaite, qu'il nous en fasse part.
\par 23 Alors le roi Darius ordonna de chercher parmi les archives de Babylone. Et ainsi, à Ecbatane, le palais qui est dans le pays de Médie, on trouva un rouleau dans lequel ces choses étaient enregistrées.
\par 24 La première année du règne de Cyrus, le roi Cyrus ordonna qu'on rebâtisse la maison de l'Éternel à Jérusalem, où l'on sacrifie avec un feu continu.
\par 25 dont la hauteur sera de soixante coudées et la largeur de soixante coudées, avec trois rangées de pierres de taille et une rangée de bois neuf de ce pays ; et ses dépenses seront prélevées sur la maison du roi Cyrus :
\par 26 Et que les ustensiles sacrés de la maison de l'Éternel, tant en or qu'en argent, que Nabuchodonosor avait retirés de la maison de Jérusalem et apportés à Babylone, seraient restitués à la maison de Jérusalem et placés dans le l'endroit où ils se trouvaient auparavant.
\par 27 Et il ordonna également à Sisinnes, gouverneur de Syrie et de Phénice, et à Sathrabuzanes, et à leurs compagnons, et à ceux qui étaient nommés dirigeants en Syrie et en Phénice, de prendre garde à ne pas se mêler du lieu, mais de laisser Zorobabel, le serviteur du Seigneur, et gouverneur de la Judée, et les anciens des Juifs, pour bâtir la maison du Seigneur en ce lieu.
\par 28 J'ai aussi ordonné de le reconstruire entièrement; et qu'ils cherchent avec diligence à aider ceux qui sont captifs des Juifs, jusqu'à ce que la maison du Seigneur soit achevée :
\par 29 Et sur le tribut de Célosyrie et de Phénicie, une part devait être donnée soigneusement à ces hommes pour les sacrifices de l'Éternel, c'est-à-dire à Zorobabel, le gouverneur, pour des taureaux, des béliers et des agneaux ;
\par 30 Et aussi du blé, du sel, du vin et de l'huile, et cela continuellement chaque année, sans autre question, selon que les prêtres qui sont à Jérusalem signifieront qu'ils seront dépensés quotidiennement.
\par 31 Afin que des offrandes soient faites au Dieu Très-Haut pour le roi et pour ses enfants, et qu'ils prient pour leur vie.
\par 32 Et il ordonna que quiconque transgresserait, oui, ou ferait peu de cas de quelque chose dit ou écrit ci-dessus, qu'un arbre soit retiré de sa propre maison, et qu'il y soit pendu, et que tous ses biens soient saisis pour le roi.
\par 33 Que l'Éternel donc, dont le nom est invoqué ici, détruise entièrement tout roi et toute nation qui étendrait la main pour gêner ou nuire à la maison de l'Éternel à Jérusalem.
\par 34 Moi, le roi Darius, j'ai ordonné que ces choses soient faites avec diligence.

\chapter{7}

\par 1 Alors Sisinnes, gouverneur de la Célosyrie et de la Phénice, et Sathrabuzanes, avec leurs compagnons, suivant les commandements du roi Darius,
\par 2 A supervisé très soigneusement les œuvres saintes, aidant les anciens des Juifs et les gouverneurs du temple.
\par 3 C'est ainsi que les œuvres saintes prospérèrent lorsque les prophètes Aggée et Zacharie prophétisèrent.
\par 4 Et ils achevèrent ces choses sur l'ordre de l'Éternel, le Dieu d'Israël, et avec le consentement de Cyrus, Darius et Artexerxès, rois de Perse.
\par 5 Ainsi fut achevée la maison sainte le vingt-troisième jour du mois d'Adar, la sixième année de Darius, roi des Perses.
\par 6 Et les enfants d'Israël, les prêtres, les Lévites et les autres captifs qui leur avaient été ajoutés, firent selon les choses écrites dans le livre de Moïse.
\par 7 Et pour la dédicace du temple de l'Éternel, ils offrirent cent taureaux, deux cents béliers, quatre cents agneaux ;
\par 8 Et douze boucs pour le péché de tout Israël, selon le nombre des chefs des tribus d'Israël.
\par 9 Les sacrificateurs et les Lévites se tenaient également vêtus de leurs vêtements, selon leurs familles, au service de l'Éternel, le Dieu d'Israël, selon le livre de Moïse, et les portiers à chaque porte.
\par 10 Et les enfants d'Israël qui étaient en captivité célébrèrent la Pâque le quatorzième jour du premier mois, après quoi les prêtres et les Lévites furent sanctifiés.
\par 11 Ceux qui étaient captifs n'étaient pas tous sanctifiés ensemble, mais les Lévites étaient tous sanctifiés ensemble.
\par 12 Et ils offrirent ainsi la Pâque pour tous les captifs, pour leurs frères les prêtres et pour eux-mêmes.
\par 13 Et les enfants d'Israël qui sortaient de captivité mangèrent, même tous ceux qui s'étaient éloignés des abominations des gens du pays et cherchaient l'Éternel.
\par 14 Et ils célébrèrent la fête des pains sans levain pendant sept jours, se réjouissant devant l'Éternel,
\par 15 C'est pourquoi il avait tourné vers eux le conseil du roi d'Assyrie, afin de renforcer leurs mains dans les œuvres de l'Éternel, le Dieu d'Israël.

\chapter{8}

\par 1 Et après ces choses, sous le règne d'Artexerxès, roi des Perses, vint Esdras, fils de Saraïas, fils d'Ézérias, fils d'Helchias, fils de Salum,
\par 2 Le fils de Sadduc, le fils d'Achitob, le fils d'Amarias, le fils d'Ezias, le fils de Merémoth, le fils de Zaraias, le fils de Savias, le fils de Boccas, le fils d'Abisum, le fils de Phinées, fils d'Éléazar, fils d'Aaron, le grand prêtre.
\par 3 Cet Esdras monta de Babylone comme scribe, connaissant parfaitement la loi de Moïse, donnée par le Dieu d'Israël.
\par 4 Et le roi lui fit honneur, car il trouva grâce à ses yeux dans toutes ses demandes.
\par 5 Certains des enfants d'Israël, le prêtre des Lévites, les saints chanteurs, les portiers et les ministres du temple montèrent avec lui à Jérusalem,
\par 6 La septième année du règne d'Artexerxès, le cinquième mois, fut la septième année du roi ; car ils quittèrent Babylone le premier jour du premier mois, et arrivèrent à Jérusalem, selon le bon voyage que l'Éternel leur avait donné.
\par 7 Car Esdras avait un très grand talent, de sorte qu'il n'a rien omis de la loi et des commandements de l'Éternel, mais il a enseigné à tout Israël les ordonnances et les jugements.
\par 8 Or, la copie de la commission, qui a été écrite par le roi Artexerxès, et qui est parvenue à Esdras, le prêtre et lecteur de la loi de l'Éternel, est celle-ci qui suit :
\par 9 Le roi Artexerxès à Esdras, le prêtre et lecteur de la loi du Seigneur, envoie son salut :
\par 10 Ayant décidé d'agir avec grâce, j'ai donné l'ordre que ceux de la nation des Juifs, et des prêtres et des Lévites se trouvant dans notre royaume, qui le voudraient et le désirent, t'accompagnent à Jérusalem.
\par 11 C'est pourquoi tous ceux qui le désirent partiront avec toi, comme cela a semblé bon à moi et à mes sept amis les conseillers ;
\par 12 Afin qu'ils considèrent les affaires de la Judée et de Jérusalem conformément à ce qui est dans la loi du Seigneur ;
\par 13 Et portez à Jérusalem les dons à l'Éternel d'Israël, que moi et mes amis avons juré, et tout l'or et l'argent qu'on peut trouver dans le pays de Babylone, à l'Éternel à Jérusalem,
\par 14 Avec ce qui est donné également par le peuple pour le temple de l'Éternel, son Dieu, à Jérusalem, et afin que l'argent et l'or soient collectés pour les taureaux, les béliers et les agneaux, et tout ce qui s'y rapporte ;
\par 15 afin qu'ils offrent des sacrifices à l'Éternel sur l'autel de l'Éternel, leur Dieu, qui est à Jérusalem.
\par 16 Et tout ce que toi et tes frères ferez avec l'argent et l'or, faites-le selon la volonté de votre Dieu.
\par 17 Et les ustensiles saints de l'Éternel, qui t'ont été donnés pour l'usage du temple de ton Dieu, qui est à Jérusalem, tu les placeras devant ton Dieu à Jérusalem.
\par 18 Et tout ce dont tu te souviendras pour l'usage du temple de ton Dieu, tu le donneras sur le trésor du roi.
\par 19 Et moi, le roi Artexerxès, j'ai aussi ordonné aux gardiens des trésors en Syrie et en Phénicie, que tout ce qu'Esdras, le prêtre et lecteur de la loi du Dieu Très-Haut, enverrait chercher, ils le lui donneraient au plus vite,
\par 20 Au total cent talents d'argent, ainsi que du blé jusqu'à cent cors, et cent pièces de vin, et d'autres choses en abondance.
\par 21 Que toutes choses soient accomplies selon la loi de Dieu, avec diligence, pour le Dieu Très-Haut, afin que la colère ne vienne pas sur le royaume du roi et de ses fils.
\par 22 Je vous commande également de n'exiger aucun impôt, ni aucune autre imposition, de la part des prêtres, ou des Lévites, ou des chanteurs sacrés, ou des portiers, ou des ministres du temple, ou de quiconque ayant des activités dans ce temple. , et qu'aucun homme n'a le pouvoir de leur imposer quoi que ce soit.
\par 23 Et toi, Esdras, selon la sagesse de Dieu, établis des juges et des juges, afin qu'ils jugent dans toute la Syrie et en Phénicie tous ceux qui connaissent la loi de ton Dieu ; et tu enseigneras à ceux qui ne le savent pas.
\par 24 Et quiconque transgressera la loi de ton Dieu et du roi sera puni avec diligence, que ce soit par la mort ou tout autre châtiment, par une peine d'argent ou par l'emprisonnement.
\par 25 Alors Esdras, le scribe, dit : Béni soit le seul Seigneur, le Dieu de mes pères, qui a mis ces choses dans le cœur du roi, pour glorifier sa maison qui est à Jérusalem !
\par 26 Et il m'a honoré aux yeux du roi et de ses conseillers, et de tous ses amis et nobles.
\par 27 C'est pourquoi j'ai été encouragé par l'aide de l'Éternel, mon Dieu, et j'ai rassemblé des hommes d'Israël pour monter avec moi.
\par 28 Et voici les chefs, selon leurs familles et selon diverses dignités, qui sont montés avec moi de Babylone sous le règne du roi Artexerxès :
\par 29 Des fils de Phinées, Gerson, des fils d'Ithamar, Gamael, des fils de David, Lettus, fils de Séchenias.
\par 30 Des fils de Pharez, Zacharie ; et avec lui on comptait cent cinquante hommes :
\par 31 Des fils de Pahath Moab, Eliaonias, fils de Zaraias, et avec lui deux cents hommes :
\par 32 Des fils de Zathoé, Séchenias, fils de Jézelus, et avec lui trois cents hommes ; des fils d'Adin, Obeth, fils de Jonathan, et avec lui deux cent cinquante hommes :
\par 33 Des fils d'Élam, Josias, fils de Gotholias, et avec lui soixante-dix hommes :
\par 34 Des fils de Saphatias, Zaraias, fils de Michael, et avec lui soixante-dix hommes :
\par 35 Des fils de Joab, Abadias, fils de Jézelus, et avec lui deux cent douze hommes :
\par 36 Des fils de Banid, Assalimoth, fils de Josaphias, et avec lui cent soixante hommes :
\par 37 Des fils de Babi, Zacharie, fils de Bebai, et avec lui vingt-huit hommes :
\par 38 Des fils d'Astath, Johannes, fils d'Acatan, et avec lui cent dix hommes :
\par 39 Des fils d'Adonikam, le dernier, et voici leurs noms, Eliphalet, Jewel et Samaias, et avec eux soixante-dix hommes.
\par 40 Des fils de Bago, Uthi, fils d'Istalcurus, et avec lui soixante-dix hommes.
\par 41 Et je les rassemblai près du fleuve appelé Théras, où nous dressâmes nos tentes pendant trois jours ; puis je les examinai.
\par 42 Mais comme je n'y avais trouvé aucun prêtre ni Lévite,
\par 43 Alors j'envoyai vers Éléazar, et Iduel, et Masman,
\par 44 Et Alnathan, et Mamaias, et Joribas, et Nathan, Eunatan, Zacharie et Mosollamon, hommes principaux et instruits.
\par 45 Et je leur ordonnai d'aller trouver Saddeus, le capitaine, qui était à la place du trésor.
\par 46 Et il leur ordonna de parler à Daddeus, à ses frères et aux trésoriers de ce lieu, de nous envoyer des hommes capables d'accomplir l'office des prêtres dans la maison de l'Éternel.
\par 47 Et par la main puissante de notre Seigneur, ils nous amenèrent des hommes habiles parmi les fils de Moli, fils de Lévi, fils d'Israël, Asebebia, et ses fils et ses frères, qui étaient dix-huit.
\par 48 Et Asebia, Annus et Osaias, son frère, des fils de Channuneus, et leurs fils, étaient vingt hommes.
\par 49 Et parmi les serviteurs du temple que David avait ordonnés, et les principaux hommes pour le service des Lévites, à savoir, les serviteurs du temple, deux cent vingt, dont la liste des noms était montrée.
\par 50 Et là, j'ai juré de jeûner aux jeunes gens devant notre Seigneur, pour lui souhaiter un voyage prospère, tant pour nous que pour ceux qui étaient avec nous, pour nos enfants et pour le bétail.
\par 51 Car j'avais honte de demander au roi des fantassins, des cavaliers et une conduite à tenir pour nous protéger contre nos adversaires.
\par 52 Car nous avions dit au roi que la puissance du Seigneur notre Dieu devrait être avec ceux qui le cherchent, pour les soutenir de toutes les manières.
\par 53 Et nous avons de nouveau supplié notre Seigneur concernant ces choses, et nous l'avons trouvé favorable.
\par 54 Puis je séparai douze des chefs des prêtres, Esebrias et Assanias, et dix hommes de leurs frères avec eux.
\par 55 Et je pesai l'or et l'argent et les ustensiles sacrés de la maison de notre Seigneur, que le roi et son conseil, les princes et tout Israël avaient donnés.
\par 56 Et après que je l'eus pesé, je leur livrai six cent cinquante talents d'argent, et des ustensiles d'argent de cent talents, et cent talents d'or,
\par 57 Et vingt vases d'or, et douze vases d'airain, même d'airain fin, brillant comme de l'or.
\par 58 Et je leur dis : Vous êtes tous deux saints pour le Seigneur, et les ustensiles sont saints, et l'or et l'argent sont un vœu pour le Seigneur, le Seigneur de nos pères.
\par 59 Gardez-les et gardez-les jusqu'à ce que vous les livriez aux chefs des prêtres et des Lévites, et aux chefs des familles d'Israël, à Jérusalem, dans les chambres de la maison de notre Dieu.
\par 60 Alors les prêtres et les Lévites, qui avaient reçu l'argent, l'or et les ustensiles, les amenèrent à Jérusalem, dans le temple de l'Éternel.
\par 61 Et nous quittions le fleuve Théras le douzième jour du premier mois, et arrivâmes à Jérusalem par la main puissante de notre Seigneur, qui était avec nous ; et dès le début de notre voyage, le Seigneur nous délivra de tout ennemi, et ainsi nous sommes arrivés à Jérusalem.
\par 62 Et après que nous y étions restés trois jours, l'or et l'argent qui avaient été pesés furent livrés dans la maison de notre Seigneur le quatrième jour à Marmoth le sacrificateur, fils d'Iri.
\par 63 Et avec lui était Éléazar, fils de Phinées, et avec eux Josabad, fils de Jésus, et Moeth, fils de Sabban, Lévites : le tout leur fut livré en nombre et en poids.
\par 64 Et tout leur poids fut écrit à la même heure.
\par 65 Et ceux qui étaient sortis de captivité offraient en sacrifice à l'Éternel, le Dieu d'Israël, douze taureaux pour tout Israël, quatre-vingt-seize béliers,
\par 66 Soixante-douze agneaux et boucs en offrande de paix, douze ; tous un sacrifice au Seigneur.
\par 67 Et ils remirent les commandements du roi aux intendants du roi et aux gouverneurs de Célosyrie et de Phénice ; et ils honorèrent le peuple et le temple de Dieu.
\par 68 Or, lorsque ces choses furent accomplies, les chefs vinrent vers moi et dirent :
\par 69 La nation d'Israël, les princes, les prêtres et les Lévites, n'ont pas expulsé d'eux les étrangers du pays, ni les souillures des païens, à savoir des Cananéens, des Hittites, des Phérésites, des Jébusiens et des Moabites, Égyptiens et Édomites.
\par 70 Car eux et leurs fils se sont mariés avec leurs filles, et la postérité sainte s'est mêlée aux étrangers du pays ; et depuis le début de cette affaire, les dirigeants et les grands hommes ont participé à cette iniquité.
\par 71 Et aussitôt que j'eus entendu ces choses, je déchirai mes vêtements et le vêtement sacré, et j'arrachai les cheveux de ma tête et de ma barbe, et je m'assis triste et très lourd.
\par 72 Alors tous ceux qui étaient alors émus par la parole de l'Éternel, le Dieu d'Israël, se rassemblèrent auprès de moi, tandis que je pleurais pour l'iniquité ; mais je restais assis tranquillement accablé jusqu'au sacrifice du soir.
\par 73 Puis, me levant du jeûne, mes vêtements et le vêtement sacré déchirés, et fléchissant les genoux, et étendant mes mains vers l'Éternel,
\par 74 J'ai dit : Seigneur, je suis confus et honteux devant ta face ;
\par 75 Car nos péchés sont multipliés au-dessus de nos têtes, et nos ignorances ont atteint jusqu'au ciel.
\par 76 Car depuis le temps de nos pères, nous avons été et sommes dans un grand péché, jusqu'à ce jour.
\par 77 Et à cause de nos péchés et de ceux de nos pères, nous, nos frères, nos rois et nos prêtres, avons été livrés aux rois de la terre, à l'épée, à la captivité et en proie dans l'ignominie, jusqu'à ce jour.
\par 78 Et maintenant, dans une certaine mesure, ta miséricorde nous a été accordée, ô Seigneur, en ce qu'il nous soit laissé une racine et un nom à l'endroit de ton sanctuaire ;
\par 79 Et pour nous découvrir une lumière dans la maison du Seigneur notre Dieu, et pour nous donner de la nourriture pendant le temps de notre servitude.
\par 80 Oui, lorsque nous étions en esclavage, nous n'avons pas été abandonnés de notre Seigneur ; mais il nous a fait grâce devant les rois de Perse, afin qu'ils nous donnent à manger ;
\par 81 Oui, et ils ont honoré le temple de notre Seigneur, et ont relevé la Sion désolée, afin qu'ils nous aient donné une demeure sûre dans la communauté juive et à Jérusalem.
\par 82 Et maintenant, Seigneur, que dirons-nous, ayant ces choses ? car nous avons transgressé tes commandements que tu as donnés par la main de tes serviteurs les prophètes, en disant :
\par 83 Que le pays que vous entrez en possession en héritage est un pays pollué par les souillures des étrangers du pays, et ils l'ont rempli de leurs impuretés.
\par 84 C'est pourquoi vous ne joindrez plus vos filles à leurs fils, et vous ne prendrez pas non plus leurs filles à vos fils.
\par 85 De plus, vous ne chercherez jamais à avoir la paix avec eux, afin d'être forts et de manger les bonnes choses du pays, et de laisser l'héritage du pays à vos enfants pour toujours.
\par 86 Et tout ce qui nous arrive nous est arrivé à cause de nos mauvaises œuvres et de nos grands péchés ; car toi, Seigneur, tu as léger nos péchés,
\par 87 Et tu nous as donné une telle racine, mais nous sommes retournés de nouveau pour transgresser ta loi et nous mêler à l'impureté des nations du pays.
\par 88 Ne pourrais-tu pas être en colère contre nous pour nous détruire, jusqu'à ce que tu ne nous aies laissé ni racine, ni graine, ni nom ?
\par 89 Ô Seigneur d'Israël, tu es vrai; car il nous reste aujourd'hui une racine.
\par 90 Voici, nous sommes maintenant devant toi dans nos iniquités, car nous ne pouvons plus tenir plus longtemps à cause de ces choses devant toi.
\par 91 Et tandis qu'Esdras, dans sa prière, faisait sa confession, pleurant et se couchant à plat ventre devant le temple, une très grande multitude d'hommes, de femmes et d'enfants se rassemblèrent auprès de lui de Jérusalem ; car il y avait de grands pleurs parmi la multitude. .
\par 92 Alors Jéchonias, fils de Jeelus, l'un des fils d'Israël, appela et dit : Ô Esdras, nous avons péché contre l'Éternel Dieu, nous avons épousé des femmes étrangères des nations du pays, et maintenant tout est fini. Israël en altitude.
\par 93 Faisons serment au Seigneur de renvoyer toutes nos femmes que nous avons prises chez les païens, avec leurs enfants,
\par 94 Comme tu l'as décrété, et autant que ceux qui obéissent à la loi du Seigneur.
\par 95 Lève-toi et mets à exécution ; car cette affaire t'appartient, et nous serons avec toi : fais vaillamment.
\par 96 Esdras se leva, et il fit prêter serment aux chefs des prêtres et aux Lévites de tout Israël d'accomplir ces choses ; et ils le jurèrent.

\chapter{9}

\par 1 Alors Esdras, sortant du parvis du temple, se dirigea vers la chambre de Joanan, fils d'Eliasib,
\par 2 Et il resta là, et ne mangea ni viande ni eau, pleurant les grandes iniquités de la multitude.
\par 3 Et il y eut une proclamation dans tous les Juifs et à Jérusalem, à tous les captifs, qu'ils seraient rassemblés à Jérusalem :
\par 4 Et que quiconque ne s'y réunirait pas dans les deux ou trois jours, selon ce que les anciens qui régnaient avaient ordonné, que son bétail soit saisi pour l'usage du temple, et qu'il soit lui-même chassé des captifs.
\par 5 Et en trois jours, tous les membres de la tribu de Juda et de Benjamin furent rassemblés à Jérusalem, le vingtième jour du neuvième mois.
\par 6 Et toute la multitude était assise, tremblante, dans la vaste cour du temple, à cause du mauvais temps actuel.
\par 7 Alors Esdras se leva et leur dit : Vous avez transgressé la loi en épousant des femmes étrangères, pour accroître ainsi les péchés d'Israël.
\par 8 Et maintenant, en vous confessant, rendez gloire au Seigneur, le Dieu de nos pères,
\par 9 Et faites sa volonté, et séparez-vous des païens du pays et des femmes étrangères.
\par 10 Alors toute la multitude s'écria, et dit d'une voix forte : Comme tu l'as dit, ainsi nous ferons.
\par 11 Mais comme le peuple est nombreux et que le temps est mauvais, de sorte que nous ne pouvons pas rester dehors, et que ce n'est pas un travail d'un jour ou deux, voyant que notre péché dans ces choses est étendu :
\par 12 C'est pourquoi que les chefs de la multitude restent, et que tous ceux de nos habitations qui ont des femmes étrangères viennent au temps fixé,
\par 13 Et avec eux les chefs et les juges de tous lieux, jusqu'à ce que nous détournions de nous la colère de l'Éternel à ce sujet.
\par 14 Alors Jonathan, fils d'Azaël, et Ezéchias, fils de Théocan, s'emparèrent de cette affaire ; et Mosollam, Lévis et Sabbathée les aidèrent.
\par 15 Et ceux qui étaient captifs firent selon toutes ces choses.
\par 16 Et Esdras, le sacrificateur, lui choisit les principaux de leurs familles, tous nommément ; et le premier jour du dixième mois, ils s'assirent ensemble pour examiner l'affaire.
\par 17 Ainsi leur cause qui retenait les femmes étrangères prit fin le premier jour du premier mois.
\par 18 Et parmi les prêtres qui s'assemblaient et qui avaient des femmes étrangères, on trouva :
\par 19 Des fils de Jésus, fils de Josedec, et de ses frères ; Matthelas et Eleazar, et Joribus et Joadanus.
\par 20 Et ils donnèrent leurs mains pour répudier leurs femmes et pour offrir des béliers en guise de réconciliation pour leurs erreurs.
\par 21 Et des fils d'Emmer; Ananias, Zabdeus, Eanes, Sameius, Hiereel et Azarias.
\par 22 Et des fils de Phaisur; Elionas, Massias Israël, Nathanaël, Ocidèle et Talsas.
\par 23 Et des Lévites; Jozabad, Semis, Colius, appelé Calitas, Pathée, Judas et Jonas.
\par 24 Des saints chanteurs ; Éléazure, Bacchurus.
\par 25 Des porteurs; Sallumus et Tolbanes.
\par 26 Parmi eux d'Israël, des fils de Phoros ; Hiermas, Eddias, Melchias, Maelus, Eléazar, Asibias et Baanias.
\par 27 Des fils d'Ela; Matthanias, Zacharie, Hiérielus, Hiérémoth et Aedias.
\par 28 Et des fils de Zamoth ; Eliadas, Elisimus, Othonias, Jarimoth, Sabatus et Sardeus.
\par 29 Des fils de Babaï ; Johannes, Ananias, Josabad et Amatheis.
\par 30 Des fils de Mani; Olamus, Mamuchus, Jedeus, Jasubus, Jasael et Hieremoth.
\par 31 Et des fils d'Addi; Naathus, et Moosias, Lacunus, et Naidus, et Mathanias, et Sesthel, Balnuus et Manasseas.
\par 32 Et des fils d'Anne; Elionas et Aseas, et Melchias, et Sabbeus, et Simon Chosameus.
\par 33 Et des fils d'Asom; Altaneus, Matthias, Baanaia, Eliphalet, Manassé et Sémei.
\par 34 Et des fils de Maani; Jeremias, Momdis, Omaerus, Juel, Mabdai, et Pelias, et Anos, Carabasion, et Enasibus, et Mamnitanaimus, Eliasis, Bannus, Eliali, Samis, Selemias, Nathanias : et des fils d'Ozora ; Sesis, Esril, Azaelus, Samatus, Zambis, Josèphe.
\par 35 Et des fils d'Ethma; Mazitias, Zabadaias, Edes, Juel, Banaias.
\par 36 Tous ceux-là avaient pris des femmes étrangères, et ils les répudièrent avec leurs enfants.
\par 37 Et les prêtres et les Lévites, et ceux d'Israël, habitaient à Jérusalem et dans la campagne, le premier jour du septième mois ; ainsi les enfants d'Israël étaient dans leurs habitations.
\par 38 Et toute la multitude se rassembla d'un commun accord dans la large place du porche saint, vers l'est :
\par 39 Et ils dirent à Esdras, le prêtre et lecteur, qu'il apporterait la loi de Moïse, qui a été donnée par l'Éternel, le Dieu d'Israël.
\par 40 Esdras, le grand prêtre, apporta la loi à toute la foule, depuis les hommes jusqu'aux femmes, et à tous les prêtres, pour entendre la loi le premier jour du septième mois.
\par 41 Et il lisait dans la grande cour, devant le porche saint, du matin jusqu'à midi, devant hommes et femmes ; et la multitude prêta attention à la loi.
\par 42 Et Esdras, le prêtre et lecteur de la loi, se tenait debout sur une chaire de bois faite à cet effet.
\par 43 Et Mattathias, Sammus, Ananias, Azarias, Urias, Ezecias, Balasamus, se tenaient à sa droite, à côté de lui.
\par 44 Et à sa gauche se tenaient Phaldaius, Misaël, Melchias, Lothasubus et Nabarias.
\par 45 Alors Esdras prit le livre de la loi devant la foule, car il occupait la première place devant tous.
\par 46 Et quand il ouvrit la loi, ils se redressèrent tous. Esdras bénit donc le Seigneur Dieu Très-Haut, le Dieu des armées, Tout-Puissant.
\par 47 Et tout le peuple répondit : Amen ; et levant les mains, ils tombèrent à terre et adorèrent l'Éternel.
\par 48 Jésus aussi, Anus, Sarabias, Adinus, Jacubus, Sabateas, Auteas, Maianeas, et Calitas, Asrias, et Joazabdus, et Ananias, Biatas, les Lévites, enseignèrent la loi du Seigneur, leur faisant comprendre également.
\par 49 Attharate parla à Esdras, chef des prêtres et des lecteurs, et aux lévites qui enseignaient la foule, à tous, en disant ,
\par 50 Ce jour est saint pour le Seigneur ; (car ils ont tous pleuré en entendant la loi :)
\par 51 Allez donc, mangez du gras, buvez du sucré, et envoyez-en une part à ceux qui n'ont rien ;
\par 52 Car ce jour est saint pour l'Éternel ; et ne vous affligez pas ; car le Seigneur vous honorera.
\par 53 Les Lévites publièrent donc toutes choses au peuple, en disant : Ce jour est saint pour l'Éternel ; ne sois pas triste.
\par 54 Alors ils s'en allèrent, chacun pour manger et boire, et se réjouir, et pour donner sa part à ceux qui n'avaient rien, et pour se réjouir ;
\par 55 Parce qu'ils comprenaient les paroles dans lesquelles ils étaient instruits et pour lesquelles ils avaient été assemblés.

\end{document}