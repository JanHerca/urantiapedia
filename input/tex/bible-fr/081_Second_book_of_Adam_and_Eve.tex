\begin{document}

\title{Deuxième livre d'Adam et Ève}

\chapter{1}

\par \textit{La famille en deuil. Caïn épouse Luluwa et ils s'éloignent.}

\par 1 QUAND Luluwa entendit les paroles de Caïn, elle pleura et alla trouver son père et sa mère et leur raconta comment Caïn avait tué son frère Abel.

\par 2 Alors ils crièrent tous à haute voix et élevèrent la voix, et se frappèrent le visage, et jetèrent de la poussière sur leur tête, et déchirèrent leurs vêtements, puis sortirent et arrivèrent à l'endroit où Abel avait été tué.

\par 3 Et ils le trouvèrent étendu sur la terre, tué, et des bêtes autour de lui ; tandis qu'ils pleuraient et pleuraient à cause de celui-ci. De son corps, en raison de sa pureté, sortait une odeur d'épices douces.

\par 4 Et Adam le porta, ses larmes coulant sur son visage ; et il se rendit à la Caverne des Trésors, où il le déposa et l'enveloppa d'épices douces et de myrrhe.

\par 5 Et Adam et Ève continuèrent à l'enterrer dans une grande douleur pendant cent quarante jours. Abel avait quinze ans et demi et Caïn dix-sept ans et demi.

\par 6 Quant à Caïn, lorsque le deuil de son frère fut terminé, il prit sa sœur Luluwa et l'épousa, sans la permission de son père et de sa mère ; car ils ne pouvaient pas l'éloigner d'elle, à cause de leur cœur lourd.

\par 7 Il descendit ensuite au bas de la montagne, loin du jardin, près de l'endroit où il avait tué son frère.

\par 8 Et à cet endroit il y avait beaucoup d'arbres fruitiers et d'arbres forestiers. Sa sœur lui donna des enfants qui, à leur tour, commencèrent à se multiplier peu à peu jusqu'à occuper cette place.

\par 9 Mais quant à Adam et Ève, ils ne se réunirent pas après les funérailles d'Abel, pendant sept ans. Après cela, Ève conçut ; et tandis qu'elle était enceinte, Adam lui dit : « Viens, prenons une offrande et l'offrons à Dieu, et demandons-lui de nous donner un bel enfant, en qui nous pouvons trouver du réconfort et que nous pouvons unir dans le mariage. à la sœur d'Abel.

\par 10 Alors ils préparèrent une offrande et la portèrent à l'autel, et l'offrèrent devant l'Éternel, et commencèrent à le supplier d'accepter leur offrande et de leur donner une bonne postérité.

\par 11 Et Dieu entendit Adam et accepta son offrande. Ensuite, ils adorèrent Adam, Ève et leur fille, et descendirent à la Caverne des Trésors et y placèrent une lampe pour la brûler nuit et jour devant le corps d'Abel.

\par 12 Alors Adam et Ève continuèrent à jeûner et à prier jusqu'à ce que le moment où Ève soit délivrée soit venu, lorsqu'elle dit à Adam : « Je souhaite aller à la grotte du rocher pour y enfanter. »

\par 13 Et il dit : « Va prendre avec toi ta fille pour te servir ; mais je resterai dans cette Caverne aux Trésors devant le corps de mon fils Abel. »

\par 14 Alors Ève écouta Adam et s'en alla, elle et sa fille. Mais Adam resta seul dans la Grotte aux Trésors.

\chapter{2}

\par \textit{Un troisième fils naît d'Adam et Ève.}

\par 1 ET Ève enfanta un fils parfaitement beau de figure et de visage. Sa beauté était semblable à celle de son père Adam, mais en plus belle.

\par 2 Alors Eve fut consolé quand elle le vit, et resta huit jours dans la grotte ; puis elle envoya sa fille vers Adam pour lui dire de venir voir l'enfant et de lui donner un nom. Mais la fille resta à sa place près du corps de son frère, jusqu'au retour d'Adam. Elle aussi.

\par 3 Mais quand Adam vint et vit la beauté de l'enfant, sa beauté et sa silhouette parfaite, il se réjouit à son sujet et fut consolé pour Abel. Puis il nomma l'enfant Seth, ce qui signifie « que Dieu a entendu ma prière et m'a délivré de mon affliction ». Mais cela signifie aussi « puissance et force ».

\par 4 Puis, après qu'Adam eut nommé l'enfant, il retourna à la Caverne des Trésors ; et sa fille retourna chez sa mère.

\par 5 Mais Ève resta dans sa caverne jusqu'à ce que quarante jours se soient écoulés, lorsqu'elle vint vers Adam et amena avec elle l'enfant et sa fille.

\par 6 Et ils arrivèrent à une rivière d'eau, où Adam et sa fille se lavaient, à cause de leur chagrin pour Abel ; mais Eve et le bébé se sont lavés pour se purifier.

\par 7 Alors ils revinrent et prirent une offrande, et allèrent à la montagne et l'offrèrent pour l'enfant ; et Dieu accepta leur offrande et envoya Sa bénédiction sur eux ainsi que sur leur fils Seth ; et ils revinrent à la Grotte des Trésors.

\par 8 Quant à Adam, il ne connut plus sa femme Ève, tous les jours de sa vie ; et aucune descendance n'en est née ; mais seulement ces cinq-là, Caïn, Luluwa, Abel, Aklia et Seth seuls.

\par 9 Mais Seth grandit en stature et en force ; et commença à jeûner et à prier avec ferveur.

\chapter{3}

\par \textit{Satan apparaît comme une belle femme tentant Adam, lui disant qu'il est encore un jeune. « Passe ta jeunesse dans la joie et le plaisir. » (12) Les différentes formes que prend Satan (15).}

\par 1 Quant à notre père Adam, au bout de sept ans à compter du jour où il avait été séparé de sa femme Ève, Satan l'enviait, lorsqu'il le voyait ainsi séparé d'elle ; et s'efforça de le faire vivre à nouveau avec elle.

\par 2 Alors Adam se leva et monta au-dessus de la Caverne des Trésors ; et j'ai continué à y dormir nuit après nuit. Mais dès qu'il faisait jour, chaque jour, il descendait à la grotte pour y prier et en recevoir une bénédiction.

\par 3 Mais le soir étant venu, il monta sur le toit de la grotte, où il dormit seul, craignant que Satan ne l'emportât. Et il resta ainsi séparé trente-neuf jours.

\par 4 Alors Satan, le ennemi de tout bien, lorsqu'il vit Adam ainsi seul, jeûnant et priant, lui apparut sous la forme d'une belle femme, qui vint se tenir devant lui dans la nuit du quarantième jour, et dit : à lui :--

\par 5 «Ô Adam, depuis le temps que tu habites dans cette grotte, nous avons connu une grande paix de ta part, et tes prières nous sont parvenues, et nous avons été consolés à ton sujet.

\par 6 «Mais maintenant, ô Adam, que tu es monté sur le toit de la grotte pour dormir, nous avons eu des doutes à ton sujet, et une grande tristesse nous est venue à cause de ta séparation d'avec Ève. Puis encore, quand tu es sur le toit de cette grotte, ta prière se déverse et ton cœur erre d'un côté à l'autre.

\par 7 «Mais quand tu étais dans la grotte, ta prière était comme un feu rassemblé; elle descendit vers nous, et tu trouvas du repos.

\par 8 «Alors j'ai aussi été affligé à cause de tes enfants qui sont séparés de toi; et ma tristesse est grande à cause du meurtre de ton fils Abel; car il était juste; et à cause d'un homme juste, chacun sera attristé.

\par 9 «Mais je me suis réjoui de la naissance de ton fils Seth; mais peu de temps après, je me suis beaucoup affligé d'Ève, parce qu'elle est ma sœur. Car lorsque Dieu a envoyé un profond sommeil sur toi et l'a retirée de ton côté, Il m'a fait sortir aussi avec elle, mais il l'a élevée en la plaçant auprès de toi, tandis qu'il m'a abaissé.

\par 10 « Je me suis réjoui pour ma sœur parce qu'elle était avec toi. Mais Dieu m'avait déjà fait une promesse et m'avait dit : «Ne t'afflige pas ; quand Adam sera monté sur le toit de la caverne des trésors et qu'il sera séparé d'Eve, sa femme, je t'enverrai vers lui, tu t'uniras à lui par les liens du mariage, et tu lui donneras cinq enfants, comme Eve lui en a donné cinq.»

\par 11 « Et maintenant, voici ! La promesse que Dieu m'a faite s'accomplit ; car c'est Lui qui m'a envoyé vers toi pour les noces ; parce que si tu m'épouses, je te donnerai des enfants plus beaux et meilleurs que ceux d'Ève.

\par 12 «Alors encore, tu n'es encore qu'un jeune; ne termine pas ta jeunesse dans ce monde dans le chagrin; mais passe les jours de ta jeunesse dans la joie et le plaisir. Car tes jours sont peu nombreux et ton épreuve est grande. Sois fort. » Finis tes jours dans ce monde dans la réjouissance. Je prendrai plaisir en toi, et tu te réjouiras ainsi avec moi et sans crainte.

\par 13 «Lève-toi donc et accomplis le commandement de ton Dieu», elle s'approcha alors d'Adam et l'embrassa.

\par 14 Mais quand Adam vit qu'il serait vaincu par elle, il pria Dieu d'un cœur fervent de le délivrer d'elle.

\par 15 Alors Dieu envoya Sa Parole à Adam, disant : « Ô Adam, cette figure est celle qui t'a promis la Divinité et la majesté ; il n'est pas favorablement disposé envers toi ; mais se montre à toi à un moment donné sous la forme d'une femme; un autre moment, sous la forme d'un ange; à d'autres occasions, sous la forme d'un serpent; et à un autre moment, sous la forme d'un dieu; mais il ne fait tout cela que pour détruire ton âme.

\par 16 « Maintenant donc, ô Adam, connaissant ton cœur, je t'ai souvent délivré de ses mains ; afin de te montrer que je suis un Dieu miséricordieux ; et que je souhaite ton bien, et que je ne souhaite pas ta ruine.

\chapter{4}

\par \textit{Adam voit le Diable sous ses vraies couleurs.}

\par 1 ALORS Dieu ordonna à Satan de se montrer à Adam clairement, sous sa propre forme hideuse.

\par 2 Mais quand Adam le vit, il eut peur et trembla à sa vue.

\par 3 Et Dieu dit à Adam : « Regarde ce diable et son regard hideux, et sache que c'est lui qui t'a fait tomber de la clarté dans les ténèbres, de la paix et du repos au labeur et à la misère.

\par 4 Et regarde, ô Adam, celui qui dit de lui-même qu'il est Dieu ! Dieu peut-il être noir ? Dieu prendrait-il la forme d'une femme ? Existe-t-il quelqu'un de plus fort que Dieu ? Et peut-Il être maîtrisé ?

\par 5 « Regarde donc, ô Adam, et vois-le lié en ta présence, dans les airs, incapable de s'enfuir ! C'est pourquoi, je te le dis, n'aie pas peur de lui ; désormais prends soin de toi et prends garde à lui, dans tout ce qu'il te fera.

\par 6 Alors Dieu chassa Satan de devant Adam, qu'il fortifia et dont il réconforta le cœur, en lui disant : « Descends à la caverne des trésors, et ne te sépare pas d'Ève ; Je réprimerai en vous toute convoitise animale.

\par 7 À partir de cette heure, il quitta Adam et Ève, et ils jouirent du repos par le commandement de Dieu. Mais Dieu n’a fait pareil pour aucun des descendants d’Adam ; mais seulement à Adam et Eve.

\par 8 Alors Adam se prosterna devant le Seigneur, pour l'avoir délivré et pour avoir déposé ses passions. Et il descendit du haut de la grotte, et demeura avec Eve comme autrefois.

\par 9 Ceci termina les quarante jours de sa séparation d'avec Ève.

\chapter{5}

\par \textit{Le diable dresse un tableau brillant pour que Seth puisse se régaler de ses pensées.}

\par 1 Quant à Seth, quand il avait sept ans, il connaissait le bien et le mal, et il était constant dans le jeûne et la prière, et passait toutes ses nuits à implorer Dieu pour sa miséricorde et son pardon.

\par 2 Il jeûnait aussi chaque jour lorsqu'il présentait son offrande, plus que son père ; car il avait un beau visage, comme un ange de Dieu. Il avait aussi un bon cœur, conservait les plus belles qualités de son âme : et c'est pour cette raison qu'il faisait chaque jour son offrande.

\par 3 Et Dieu fut satisfait de son offrande ; mais il était également satisfait de sa pureté. Et il continua ainsi à faire la volonté de Dieu, de son père et de sa mère, jusqu'à l'âge de sept ans.

\par 4 Après cela, alors qu'il descendait de l'autel, ayant terminé son offrande, Satan lui apparut sous la forme d'un bel ange, brillant de lumière ; avec un bâton de lumière à la main, lui-même ceint d'une ceinture de lumière.

\par 5 Il salua Seth avec un beau sourire, et commença à le séduire avec de belles paroles, lui disant : « Ô Seth, pourquoi demeures-tu dans cette montagne ? Car il est rude, plein de pierres et de sable, et d'arbres sans bons fruits ; un désert sans habitations et sans villes ; ce n’est pas un bon endroit où habiter. Mais tout n’est que chaleur, lassitude et ennuis.

\par 6 Il dit plus loin : «Mais nous habitons dans des endroits magnifiques, dans un autre monde que cette terre. Notre monde est un monde de lumière et notre condition est des meilleures ; nos femmes sont plus belles que toutes les autres ; et je te souhaite, ô Seth, pour épouser l'une d'elles, car je vois que tu es belle à regarder, et que dans ce pays il n'y a pas une seule femme assez bonne pour toi. D'ailleurs, tous ceux qui vivent dans ce monde ne sont que cinq âmes.

\par 7 «Mais dans notre monde, il y a de très nombreux hommes et de nombreuses jeunes filles, toutes plus belles les unes que les autres. Je souhaite donc t'éloigner d'ici, afin que tu puisses voir mes parents et te marier avec qui tu voudras.

\par 8 «Tu demeureras alors près de moi et tu seras en paix; tu seras rempli de splendeur et de lumière, comme nous.

\par 9 «Tu resteras dans notre monde et tu te reposeras de ce monde et de sa misère ; tu ne te sentiras plus jamais faible et fatigué ; tu ne présenteras plus d'offrande et tu ne demanderas plus de pitié, car tu ne commettras plus de péché et tu ne seras plus influencé par les passions.

\par 10 «Et si tu écoutes ce que je dis, tu épouseras une de mes filles; car chez nous, ce n'est pas un péché de faire cela; cela n'est pas non plus considéré comme une convoitise animale.

\par 11 « Car dans notre monde, nous n'avons pas de Dieu ; mais nous sommes tous des dieux ; nous sommes tous de lumière, célestes, puissants, forts et glorieux.

\chapter{6}

\par \textit{La conscience de Seth l'aide. Il retourne vers Adam et Eve.}

\par 1 QUAND Seth entendit ces paroles, il fut étonné, et inclina son cœur vers le discours perfide de Satan, et lui dit : « As-tu dit qu'il y a un autre monde créé que celui-ci ; et d'autres créatures plus belles que celles de ce monde ?

\par 2 Et Satan dit : « Oui ; voici, tu m'as entendu; mais je les louerai encore, ainsi que leurs voies, devant toi.

\par 3 Mais Seth lui dit : « Ton discours m'a étonné, et ta belle description de tout cela.

\par 4 « Pourtant, je ne peux pas aller avec toi aujourd'hui ; pas avant d'être allé voir mon père Adam et ma mère Ève et de leur avoir raconté tout ce que tu m'as dit. Alors, s’ils me permettent de partir avec toi, je viendrai.

\par 5 Seth dit encore : « J'ai peur de faire quoi que ce soit sans la permission de mon père et de ma mère, de peur de périr comme mon frère Caïn et comme mon père Adam, qui a transgressé le commandement de Dieu. Mais voici, tu connais cet endroit ; venez me retrouver ici demain.

\par 6 Quand Satan entendit cela, il dit à Seth : « Si tu racontes à ton père Adam ce que je t'ai dit, il ne te laissera pas venir avec moi.

\par 7 Mais écoutez-moi ; ne dis pas à ton père et à ta mère ce que je t'ai dit ; mais venez avec moi aujourd'hui, dans notre monde ; où tu verras de belles choses et où tu t'amuseras là-bas, et où tu te réjouiras de ce jour parmi mes enfants, en les contemplant et en te rassasiant de joie ; et réjouissez-vous toujours davantage. Alors je te ramènerai à cet endroit demain ; mais si tu préfères rester avec moi, qu'il en soit ainsi.

\par 8 Alors Seth répondit : « L'esprit de mon père et de ma mère est suspendu sur moi ; et si je me cache d'eux un jour, ils mourront, et Dieu me tiendra coupable d'avoir péché contre eux.

\par 9 « Et s'ils ne savaient que je suis venu en ce lieu pour y apporter mon offrande, ils ne seraient pas séparés de moi une seule heure ; je ne devrais pas non plus aller ailleurs, à moins qu’ils ne me le permettent. Mais ils me traitent avec beaucoup de gentillesse, car je reviens vers eux rapidement.

\par 10 Alors Satan lui dit : « Que t'arrivera-t-il si tu te caches d'eux une nuit et si tu reviens vers eux au point du jour ?

\par 11 Mais Seth, voyant qu'il continuait à parler et qu'il ne voulait pas le quitter, courut et monta à l'autel, étendit les mains vers Dieu et chercha la délivrance de lui.

\par 12 Alors Dieu envoya Sa Parole et maudit Satan, qui s'enfuyait loin de Lui.

\par 13 Mais quant à Seth, il était monté à l'autel, disant ainsi dans son cœur. « L'autel est le lieu d'offrande, et Dieu est là ; un feu divin le consumera ; Ainsi, Satan ne pourra pas me faire de mal et ne m’emmènera pas là-bas.

\par 14 Alors Seth descendit de l'autel et alla vers son père et sa mère, où il se trouva sur le chemin, désireux d'entendre sa voix ; car il avait tardé un moment.

\par 15 Il commença alors à leur raconter ce qui lui était arrivé de la part de Satan, sous la forme d'un ange.

\par 16 Mais quand Adam entendit son récit, il lui baisa le visage et le mit en garde contre cet ange, lui disant que c'était Satan qui lui était ainsi apparu. Alors Adam prit Seth, et ils se rendirent à la Grotte des Trésors et s'y réjouirent.

\par 17 Mais à partir de ce jour, Adam et Ève ne se séparèrent plus de lui, où qu'il aille, que ce soit pour son offrande ou pour toute autre chose.

\par 18 Ce signe est arrivé à Seth, quand il avait neuf ans.

\chapter{7}

\par \textit{Seth épouse Aklia. Adam vit pour voir ses petits-enfants et ses arrière-petits-enfants.}

\par 1 QUAND notre père Adam vit que Seth était d'un cœur parfait, il souhaita qu'il se marie ; de peur que l'ennemi ne lui apparaisse une autre fois et ne le vainque.

\par 2 Alors Adam dit à son fils Seth : « Je souhaite, ô mon fils, que tu épouses ta sœur Aklia, la sœur d'Abel, afin qu'elle te donne des enfants qui rempliront la terre, selon la promesse que Dieu nous a faite.

\par 3 « N'aie pas peur, ô mon fils ; il n’y a aucune honte là-dedans. Je souhaite que tu te maries, de peur que l'ennemi ne te submerge.

\par 4 Seth, cependant, ne souhaitait pas se marier ; mais par obéissance à son père et à sa mère, il ne dit pas un mot.

\par 5 Adam le maria donc à Aklia. Et il avait quinze ans.

\par 6 Mais à l'âge de vingt ans, il engendra un fils, qu'il appela Enos ; puis il engendra d'autres enfants que lui.

\par 7 Alors Enos grandit, se maria et engendra Caïnan.

\par 8 Caïnan grandit également, se maria et engendra Mahalaleel.

\par 9 Ces pères sont nés du vivant d'Adam et ont habité près de la Grotte des Trésors.

\par 10 Alors les jours d'Adam furent neuf cent trente ans, et ceux de Mahalaleel cent. Mais Mahalaleel, lorsqu'il fut grand, aimait le jeûne, la prière et les travaux pénibles, jusqu'à ce que la fin des jours de notre père Adam approchait.

\chapter{8}

\par \textit{Les derniers mots remarquables d'Adam. Il prédit le Déluge. Il exhorte sa progéniture au bien. Il révèle certains mystères de la vie.}

\par 1 QUAND notre père Adam vit que sa fin était proche, il appela son fils Seth, qui vint vers lui dans la Caverne des Trésors, et il lui dit :

\par 2 « Ô Seth, mon fils, amène-moi tes enfants et les enfants de tes enfants, afin que je puisse répandre ma bénédiction sur eux avant de mourir. »

\par 3 Lorsque Seth entendit ces paroles de son père Adam, il s'éloigna de lui, versa un flot de larmes sur son visage, et rassembla ses enfants et les enfants de ses enfants, et les amena à son père Adam.

\par 4 Mais quand notre père Adam les vit autour de lui, il pleura d'avoir dû être séparé d'eux.

\par 5 Et quand ils le virent pleurer, ils pleurèrent tous ensemble et tombèrent sur sa face en disant : « Comment seras-tu séparé de nous, ô notre père ? Et comment la terre te recevra-t-elle et te cachera-t-elle à nos yeux ? C'est ainsi qu'ils se lamentèrent beaucoup, et en termes semblables.

\par 6 Alors notre père Adam les bénit tous, et dit à Seth, après les avoir bénis : --

\par 7 « Ô Seth, mon fils, tu sais que ce monde est plein de tristesse et de lassitude ; et tu sais tout ce qui nous est arrivé à cause de nos épreuves. C'est pourquoi je te commande maintenant dans ces mots : garder l'innocence, être pur et juste, et avoir confiance en Dieu, et ne pas s'appuyer sur les discours de Satan, ni sur les apparitions dans lesquelles il se montrera à toi.

\par 8 Mais garde les commandements que je te donne aujourd'hui ; puis donne-le à ton fils Enos ; et qu'Enos le donne à son fils Caïnan ; et Caïnan à son fils Mahalaleel ; afin que ce commandement demeure ferme parmi tous tes enfants.

\par 9 «Ô Seth, mon fils, dès que je serai mort, prenez mon corps, enroulez-le de myrrhe, d'aloès et de casse, et laissez-moi ici, dans cette caverne aux trésors, où se trouvent tous ces gages que Dieu nous a donnés depuis le jardin.

\par 10 « Ô mon fils, désormais un déluge viendra et submergera toutes les créatures et n'abandonnera que huit âmes.

\par 11 «Mais, ô mon fils, que ceux qu'il laissera dehors parmi tes enfants à ce moment-là, emmènent mon corps avec eux hors de cette grotte; et quand ils l'auront emporté avec eux, que le plus âgé d'entre eux ordonne ses enfants de déposer mon corps dans un bateau jusqu'à ce que l'inondation soit apaisée, et qu'ils sortent du bateau.

\par 12 Alors ils prendront mon corps et le déposeront au milieu de la terre, peu de temps après avoir été sauvés des eaux du déluge.

\par 13 « Car le lieu où mon corps sera déposé est le milieu de la terre ; Dieu viendra de là et sauvera toute notre parenté.

\par 14 «Mais maintenant, ô Seth, mon fils, place-toi à la tête de ton peuple; prends soin d'eux et veille sur eux dans la crainte de Dieu; et conduis-les dans le bon chemin, commande-leur de jeûner à Dieu; et faites-leur comprendre qu’ils ne doivent pas écouter Satan, de peur qu’il ne les détruise.

\par 15 « Alors encore, sépare tes enfants et les enfants de tes enfants des enfants de Caïn ; ne les laissez jamais se mêler à eux, ni les approcher ni dans leurs paroles ni dans leurs actes.

\par 16 Alors Adam fit descendre sa bénédiction sur Seth, et sur ses enfants, et sur tous les enfants de ses enfants.

\par 17 Puis il se tourna vers son fils Seth et vers Ève, sa femme, et leur dit : Conservez cet or, cet encens et cette myrrhe, que Dieu nous a donnés en signe, car, dans les jours qui viennent, un déluge submergera toute la création. Ceux qui entreront dans l'arche prendront avec eux l'or, l'encens et la myrrhe, ainsi que mon corps, et ils déposeront l'or, l'encens et la myrrhe, ainsi que mon corps, au milieu de la terre.

\par 18 «Puis, après longtemps, la ville dans laquelle l'or, l'encens et la myrrhe se trouvent avec mon corps, sera pillée. Mais quand elle sera pillée, l'or, l'encens et la myrrhe seront pris en charge avec le butin qui est gardé ; et rien d’eux ne périra, jusqu’à ce que la Parole de Dieu faite homme vienne, quand les rois les prendront et lui offriront de l’or en signe de sa qualité de roi ; de l’encens, en signe de sa qualité de Dieu du ciel et de la terre, et la myrrhe en signe de sa passion.

\par 19 « Du froid aussi, comme signe de sa victoire sur Satan et sur tous nos ennemis ; de l'encens comme signe qu'il ressuscitera d'entre les morts et qu'il sera élevé au-dessus des choses dans les cieux et de la terre ; et de la myrrhe, en signe qu'il boira du fiel amer et qu'il ressentira les douleurs de l'enfer causées par Satan.

\par 20 «Et maintenant, ô Seth, mon fils, voici, je t'ai révélé les mystères cachés que Dieu m'avait révélés. Garde mon commandement, pour toi et pour ton peuple. »

\chapter{9}

\par \textit{La mort d'Adam.}

\par 1 QUAND Adam eut terminé son commandement à Seth, ses membres furent relâchés, ses mains et ses pieds perdirent tout pouvoir, sa bouche devint muette et sa langue cessa complètement de parler. Il ferma les yeux et rendit l'âme.

\par 2 Mais quand ses enfants virent qu'il était mort, ils se jetèrent sur lui, hommes et femmes, vieux et jeunes, en pleurant.

\par 3 La mort d'Adam eut lieu au bout de neuf cent trente années qu'il vécut sur la terre ; le quinzième jour de Barmudeh, après le calcul d'un épacte du soleil, à la neuvième heure.

\par 4 C'était un vendredi, le jour même où il fut créé, et où il se reposa ; et l'heure à laquelle il mourut fut la même que celle à laquelle il sortit du jardin.

\par 5 Alors Seth le pansa bien et l'embauma avec beaucoup d'aromates doux, provenant d'arbres sacrés et de la Montagne Sainte ; et que son corps soit déposé du côté oriental de l'intérieur de la grotte, du côté de l'encens ; et il plaça devant lui un chandelier qui restait allumé.

\par 6 Alors ses enfants se tenaient devant lui, pleurant et se lamentant sur lui toute la nuit jusqu'au point du jour.

\par 7 Alors Seth et son grand-fils Enos, et Caïnan, le fils d'Enos, sortirent et prirent de bonnes offrandes à présenter au Seigneur, et ils arrivèrent à l'autel sur lequel Adam offrit des cadeaux à Dieu, lorsqu'il en offrit.

\par 8 Mais Ève leur dit : « Attendez que nous ayons d'abord demandé à Dieu d'accepter notre offrande, de garder près de lui l'âme d'Adam, son serviteur, et de la prendre au repos. »

\par 9 Et ils se levèrent tous et prièrent.

\chapter{10}

\par \textit{«Adam fut le premier...»}

\par 1 ET lorsqu'ils eurent terminé leur prière, la Parole de Dieu vint les réconforter concernant leur père Adam.

\par 2 Après cela, ils offrirent leurs cadeaux pour eux et pour leur père.

\par 3 Lorsqu'ils eurent achevé leur offrande, la parole de Dieu fut adressée à Seth, le plus âgé d'entre eux, et lui dit : "Seth, Seth, Seth, trois fois. Comme j'ai été avec ton père, je serai avec toi, jusqu'à l'accomplissement de la promesse que j'ai faite à ton père, en disant : J'enverrai ma parole et je te sauverai, toi et ta postérité.

\par 4 Mais quant à ton père Adam, garde le commandement qu'il t'a donné ; et sépare ta postérité de celle de Caïn, ton frère.

\par 5 Et Dieu retira Sa Parole de Seth.

\par 6 Alors Seth, Eve et leurs enfants descendirent de la montagne à la Caverne des Trésors.

\par 7 Mais Adam fut le premier dont l'âme mourut au pays d'Eden, dans la Caverne des Trésors ; car personne n'est mort avant lui, si ce n'est son fils Abel, qui est mort assassiné.

\par 8 Alors tous les enfants d'Adam se levèrent et pleurèrent sur leur père Adam et lui firent des offrandes pendant cent quarante jours.

\chapter{11}

\par \textit{Seth devient le chef de la tribu la plus heureuse et la plus juste qui ait jamais vécu.}

\par 1 APRÈS la mort d'Adam et d'Ève, Seth sépara ses enfants, et les enfants de ses enfants, des enfants de Caïn. Caïn et sa postérité descendirent et habitèrent vers l'ouest, en contrebas de l'endroit où il avait tué son frère Abel.

\par 2 Mais Seth et ses enfants habitaient vers le nord, sur la montagne de la Caverne des Trésors, afin d'être près de leur père Adam.

\par 3 Et Seth l'aîné, grand et bon, avec une belle âme et un esprit fort, se tenait à la tête de son peuple ; et il les soignait avec innocence, pénitence et douceur, et ne permettait à aucun d'eux de descendre vers les enfants de Caïn.

\par 4 Mais à cause de leur propre pureté, ils furent appelés « Enfants de Dieu », et ils étaient avec Dieu, au lieu des armées d'anges qui tombèrent ; car ils continuaient à louer Dieu et à lui chanter des psaumes dans leur grotte, la grotte des trésors.

\par 5 Alors Seth se tint devant le corps de son père Adam et de sa mère Ève, et pria nuit et jour, et demanda miséricorde envers lui et ses enfants ; et que lorsqu'il avait des difficultés à s'occuper d'un enfant, il lui donnait des conseils.

\par 6 Mais Seth et ses enfants n'aimaient pas les travaux terrestres, mais se livraient aux choses célestes ; car ils n'avaient d'autre pensée que des louanges, des doxologies et des psaumes à Dieu.

\par 7 C'est pourquoi ils entendaient toujours les voix des anges louant et glorifiant Dieu ; de l'intérieur du jardin, ou lorsqu'ils étaient envoyés par Dieu en mission, ou lorsqu'ils montaient au ciel.

\par 8 Car Seth et ses enfants, à cause de leur propre pureté, ont entendu et vu ces anges. Là encore, le jardin n’était pas loin au-dessus d’eux, mais seulement à une quinzaine de coudées spirituelles.

\par 9 Or, une coudée spirituelle correspond à trois coudées d'homme, soit en tout quarante-cinq coudées.

\par 10 Seth et ses enfants habitaient sur la montagne au-dessous du jardin ; ils ne semaient ni ne moissonnaient ; ils ne produisaient pas de nourriture pour le corps, pas même du blé, mais seulement des offrandes. Ils mangèrent des fruits et des arbres savoureux qui croissaient sur la montagne où ils habitaient.

\par 11 Alors Seth jeûnait souvent tous les quarante jours, ainsi que ses enfants aînés. Car la famille de Seth sentait l'odeur des arbres du jardin, quand le vent soufflait dans cette direction.

\par 12 Ils étaient heureux, innocents, sans crainte soudaine, il n'y avait entre eux ni jalousie, ni mauvaise action, ni haine. Il n'y avait pas de passion animale ; d'aucune bouche parmi eux ne sortaient ni paroles grossières ni malédictions ; ni mauvais conseil ni fraude. Car les hommes de cette époque ne juraient jamais, mais dans les circonstances difficiles, quand il fallait jurer, ils juraient par le sang d'Abel le juste.

\par 13 Mais ils contraignaient chaque jour leurs enfants et leurs femmes dans la grotte à jeûner et à prier et à adorer le Dieu Très-Haut. Ils se bénirent dans le corps de leur père Adam et s'en oignèrent.

\par 14 Et ils le firent jusqu'à ce que la fin de Seth approchait.

\chapter{12}

\par \textit{Les affaires familiales de Seth. Sa mort. La direction d'Enos. Comment s'est comportée la branche exclue de la famille d'Adam.}

\par 1 Alors Seth, le juste, appela son fils Énos, et Caïnan, fils d'Énos, et Mahalaleel, fils de Caïnan, et leur dit :

\par 2 « Comme ma fin est proche, je souhaite construire un toit sur l'autel sur lequel les cadeaux sont offerts. »

\par 3 Ils obéirent à son commandement et sortirent tous, vieux et jeunes, et y travaillèrent dur, et bâtirent un beau toit sur l'autel.

\par 4 Et Seth pensait, en agissant ainsi, qu'une bénédiction viendrait sur ses enfants sur la montagne ; et qu'il devrait présenter une offrande pour eux avant sa mort.

\par 5 Puis, lorsque la construction du toit fut achevée, il leur ordonna de faire des offrandes. Ils y travaillèrent avec diligence et les apportèrent à Seth, leur père, qui les prit et les offrit sur l'autel ; et a prié Dieu d'accepter leurs offrandes, d'avoir pitié des âmes de ses enfants et de les garder de la main de Satan.

\par 6 Et Dieu accepta son offrande et envoya sa bénédiction sur lui et sur ses enfants. Et alors Dieu fit une promesse à Seth, disant : « À la fin des cinq grands jours et demi au sujet desquels je t'ai fait une promesse, à toi et à ton père, j'enverrai ma Parole et je te sauverai, toi et ta postérité. »

\par 7 Alors Seth et ses enfants, et les enfants de ses enfants, se réunirent et descendirent de l'autel et se rendirent à la Caverne des Trésors - où ils prièrent et se bénirent dans le corps de notre père Adam, et oignèrent eux-mêmes avec.

\par 8 Mais Seth resta quelques jours dans la Caverne des Trésors, et ensuite il souffrit des souffrances jusqu'à la mort.

\par 9 Alors Enos, son fils premier-né, vint vers lui, avec Caïnan, son fils, et Mahalaleel, le fils de Caïnan, et Jared, le fils de Mahalaleel, et Enoch, le fils de Jared, avec leurs femmes et leurs enfants pour recevoir un bénédiction de Seth.

\par 10 Alors Seth pria pour eux, les bénit et les adjura par le sang d'Abel le juste, en disant : Je vous supplie, mes enfants, de ne laisser aucun de vous descendre de cette montagne sainte et pure.

\par 11 N'ayez aucune communion avec les enfants de Caïn, le meurtrier et le pécheur, qui a tué son frère ; car vous savez, ô mes enfants, que nous fuyons loin de lui et de tous ses péchés de toutes nos forces, parce qu'il a tué son frère Abel.

\par 12 Après avoir dit cela, Seth bénit Enos, son fils premier-né, et lui commanda habituellement de servir dans la pureté devant le corps de notre père Adam, tous les jours de sa vie ; puis aussi, de temps en temps, pour aller à l'autel qu'il avait bâti. Et il lui ordonna de nourrir son peuple avec justice, jugement et pureté, tous les jours de sa vie.

\par 13 Alors les membres de Seth furent détachés ; ses mains et ses pieds perdirent tout pouvoir ; sa bouche est devenue muette et incapable de parler ; et il rendit l'âme et mourut le lendemain de sa neuf cent douzième année ; le vingt-septième jour du mois d'Abib ; Hénoc avait alors vingt ans.

\par 14 Puis ils enroulèrent soigneusement le corps de Seth, et l'embaumèrent avec des épices douces, et le déposèrent dans la Caverne des Trésors, du côté droit du corps de notre père Adam, et ils le pleurèrent quarante jours. Ils lui offrirent des cadeaux, comme ils l'avaient fait pour notre père Adam.

\par 15 Après la mort de Seth, Enos se leva à la tête de son peuple, qu'il nourrit dans la justice et le jugement, comme son père le lui avait ordonné.

\par 16 Mais lorsqu'Énos eut huit cent vingt ans, Caïn avait une nombreuse descendance ; car ils se mariaient fréquemment, étant livrés aux convoitises animales ; jusqu'à ce que le pays au-dessous de la montagne en soit rempli.

\chapter{13}

\par \textit{«Parmi les enfants de Caïn, il y eut beaucoup de vols, de meurtres et de péchés.»}

\par 1 En ces jours-là vivait Lémec l'aveugle, qui était des fils de Caïn. Il avait un fils nommé Atun, et ils possédaient tous deux beaucoup de bétail.

\par 2 Mais Lamek avait l'habitude de les envoyer paître chez un jeune berger qui les gardait et qui, le soir, en rentrant, pleurait devant son grand-père, devant son père Atun et sa mère Hazina, et leur disait : "Moi, je ne peux pas faire paître ces bêtes tout seul, de peur qu'on ne m'en vole quelques-unes ou qu'on ne me tue à cause d'elles." En effet, parmi les enfants de Caïn, il y avait beaucoup de vols, de meurtres et de péchés.

\par 3 Alors Lémec eut pitié de lui, et il dit : En vérité, s'il était seul, il pourrait être vaincu par les hommes de ce lieu.

\par 4 Lamek se leva, prit un arc qu'il avait gardé depuis sa jeunesse, avant de devenir aveugle, de grandes flèches, des pierres lisses et une fronde qu'il avait, et il alla dans les champs avec le jeune berger, se plaçant derrière le bétail, tandis que le jeune berger surveillait le bétail. Lémec travailla ainsi pendant plusieurs jours.

\par 5 Cependant Caïn, depuis que Dieu l'avait rejeté et l'avait maudit de tremblement et de terreur, ne pouvait ni s'installer ni trouver de repos en un seul endroit ; mais il errait d'un endroit à l'autre.

\par 6 Au cours de ses pérégrinations, il vint vers les femmes de Lémec et les interrogea sur lui. Ils lui dirent : « Il est dans les champs avec le bétail. »

\par 7 Alors Caïn partit pour le chercher ; Et comme il entrait dans les champs, le jeune berger entendit le bruit qu'il faisait, et le bétail qui se rassemblait devant lui.

\par 8 Alors il dit à Lémec : « Ô mon seigneur, est-ce une bête sauvage ou un voleur ?

\par 9 Et Lémec lui dit : Fais-moi comprendre de quel côté il regarde quand il monte.

\par 10 Alors Lémec tendit son arc, y plaça une flèche et mit une pierre dans la fronde, et quand Caïn sortit de la campagne, le berger dit à Lémec : « Tire, voici, il arrive. »

\par 11 Alors Lémec tira sur Caïn avec sa flèche et le frappa au côté. Et Lémec le frappa avec une pierre de sa fronde, qui tomba sur son visage, et lui arracha les deux yeux ; puis Caïn tomba aussitôt et mourut.

\par 12 Alors Lémec et le jeune berger s'approchèrent de lui et le trouvèrent étendu par terre. Et le jeune berger lui dit : « C'est Caïn, notre grand-père, que tu as tué, ô mon seigneur ! »

\par 13 Alors Lémec en fut désolé, et dans l'amertume de son regret, il joignit les mains et frappa de sa paume plate la tête du jeune, qui tomba comme s'il était mort ; mais Lémec pensait que c'était une feinte ; alors il prit une pierre, le frappa et lui brisa la tête jusqu'à ce qu'il meure.

\chapter{14}

\par \textit{Le temps, tel un courant incessant, emporte une autre génération d'hommes.}

\par 1 QUAND Énos eut neuf cents ans, tous les enfants de Seth et de Caïnan, et ses premiers-nés, avec leurs femmes et leurs enfants, se rassemblèrent autour de lui, lui demandant une bénédiction.

\par 2 Il pria alors sur eux et les bénit, et les adjura par le sang d'Abel le juste en leur disant : « Qu'aucun de vos enfants ne descende de cette montagne sainte, et qu'ils n'aient aucune communion avec les enfants de Caïn le meurtrier.

\par 3 Alors Énos appela son fils Caïnan et lui dit : Vois, ô mon fils, et mets ton cœur sur ton peuple, et affermis-le dans la justice et l'innocence ; et reste debout devant le corps de notre père Adam, tous les jours de ta vie.

\par 4 Après cela, Énos entra en repos, âgé de neuf cent quatre-vingt-cinq ans ; et Caïnan l'enroula et le déposa dans la Caverne des Trésors à la gauche de son père Adam ; et il lui fit des offrandes, selon la coutume de ses pères.

\chapter{15}

\par \textit{La progéniture d'Adam continue de garder la Grotte aux Trésors comme sanctuaire familial.}

\par 1 APRÈS la mort d'Enos, Caïnan se tenait à la tête de son peuple dans la justice et l'innocence, comme son père le lui avait ordonné ; il a également continué à exercer son ministère devant le corps d'Adam, à l'intérieur de la Grotte des Trésors.

\par 2 Puis, lorsqu'il eut vécu neuf cent dix ans, la souffrance et l'affliction le surprirent. Et comme il était sur le point d'entrer dans le repos, tous les pères avec leurs femmes et leurs enfants vinrent vers lui, et il les bénit et les adjura par le sang d'Abel le juste, en leur disant : « Qu'aucun de vous ne descende de cette montagne sainte, et qu'il n'ait pas de communion avec les fils de Caïn, le meurtrier.»

\par 3 Mahalaleel, son fils aîné, reçut ce commandement de son père, qui le bénit et mourut.

\par 4 Alors Mahalaleel l'embauma avec des épices douces et le déposa dans la Caverne des Trésors, avec ses pères ; et ils lui firent des offrandes, selon la coutume de leurs pères.



\chapter{16}

\par \textit{La bonne branche de la famille a encore peur des enfants de Caïn.}

\par 1 ALORS Mahalaleel se tenait sur son peuple, et le nourrissait dans la justice et l'innocence, et les surveillait pour s'assurer qu'ils n'avaient aucun rapport sexuel avec les enfants de Caïn.

\par 2 Il a également continué dans la Grotte des Trésors à prier et à servir devant le corps de notre père Adam, demandant à Dieu sa miséricorde pour lui-même et pour son peuple ; jusqu'à l'âge de huit cent soixante-dix ans, lorsqu'il tomba malade.

\par 3 Alors tous ses enfants se rassemblèrent auprès de lui, pour le voir et lui demander sa bénédiction sur eux tous, avant qu'il ne quitte ce monde.

\par 4 Alors Mahalaleel se leva et s'assit sur son lit, les larmes coulant sur son visage, et il appela son fils aîné Jared, qui vint vers lui.

\par 5 Il lui baisa alors le visage et lui dit : « Ô Jared, mon fils, je t'adjure par Celui qui a fait le ciel et la terre, de veiller sur ton peuple et de le nourrir dans la justice et l'innocence ; et non pour que l'un d'eux descende de cette montagne sainte vers les enfants de Caïn, de peur qu'il ne périsse avec eux.

\par 6 « Écoute, ô mon fils, désormais il y aura une grande destruction sur cette terre à cause d'eux ; Dieu sera en colère contre le monde et le détruira avec des eaux.

\par 7 «Mais je sais aussi que tes enfants ne t'écouteront pas, et qu'ils descendront de cette montagne et auront des relations sexuelles avec les enfants de Caïn, et qu'ils périront avec eux.

\par 8 « Ô mon fils ! enseigne-leur et veille sur eux, afin qu'aucune culpabilité ne t'attache à cause d'eux.

\par 9 Mahalaleel dit en outre à son fils Jared : « Quand je mourrai, embaume mon corps et dépose-le dans la Caverne des Trésors, près des corps de mes pères ; alors tiens-toi près de mon corps et prie Dieu ; prends soin d'eux et accomplis ton ministère devant eux, jusqu'à ce que tu entres toi-même dans le repos.

\par 10 Mahalaleel bénit alors tous ses enfants ; puis il se coucha sur son lit et entra dans le repos comme ses pères.

\par 11 Mais quand Jared vit que son père Mahalaleel était mort, il pleura et se lamenta, et embrassa et baisa ses mains et ses pieds ; et tous ses enfants aussi.

\par 12 Et ses enfants l'embaumèrent soigneusement et le déposèrent près des corps de ses pères. Alors ils se levèrent et le pleurèrent quarante jours.

\chapter{17}

\par \textit{Jared devient Martinet. Il est attiré au pays de Caïn où il voit de nombreux spectacles voluptueux. Jared s'en sort de justesse avec le cœur pur.}

\par 1 ALORS Jared observa le commandement de son père et se leva comme un lion sur son peuple. Il les a nourris de justice et d’innocence et leur a ordonné de ne rien faire sans son conseil. Car il avait peur à leur sujet, de peur qu'ils n'aillent vers les enfants de Caïn.

\par 2 Pourquoi leur a-t-il donné des ordres à plusieurs reprises ; et il continua à le faire jusqu'à la fin de la quatre cent quatre-vingt-cinquième année de sa vie.

\par 3 À la fin de ces années dites, ce signe lui vint. Alors que Jared se tenait comme un lion devant les corps de ses pères, priant et avertissant son peuple, Satan l'enviait et provoqua une belle apparition, car Jared ne laissait pas ses enfants faire quoi que ce soit sans son conseil.

\par 4 Satan lui apparut alors avec trente hommes de ses armées, sous la forme de beaux hommes ; Satan lui-même étant le plus âgé et le plus grand d'entre eux, avec une belle barbe.

\par 5 Ils se tinrent à l'entrée de la grotte et appelèrent Jared de l'intérieur.

\par 6 Il sortit vers eux et les trouva comme des hommes beaux, pleins de lumière et d'une grande beauté. Il s'étonnait de leur beauté et de leur apparence ; et il se demanda s'ils ne pourraient pas être des enfants de Caïn.

\par 7 Il dit aussi dans son cœur : « Comme les enfants de Caïn ne peuvent pas monter à la hauteur de cette montagne, et qu'aucun d'eux n'est aussi beau qu'ils le paraissent ; et parmi ces hommes, il n’y a aucun de mes parents ; ce doivent être des étrangers.

\par 8 Alors Jared et eux échangèrent un salut et il dit à l'aîné d'entre eux : « Ô mon père, explique-moi la merveille qui est en toi, et dis-moi qui sont ceux qui sont avec toi ; car ils me semblent des hommes étrangers.

\par 9 Alors l'aîné se mit à pleurer, et les autres pleurèrent avec lui ; et il dit à Jared : « Je suis Adam, que Dieu a créé le premier ; et voici mon fils Abel, qui a été tué par son frère Caïn, dans le cœur duquel Satan a mis pour l'assassiner.

\par 10 «Alors voici mon fils Seth, que j'ai demandé au Seigneur, qui me l'a donné, de me consoler à la place d'Abel.

\par 11 «Alors celui-ci est mon fils Enos, fils de Seth, et cet autre est Caïnan, fils d'Enos, et cet autre est Mahalaleel, fils de Caïnan, ton père.»

\par 12 Mais Jared resta étonné de leur apparition et du discours que l'aîné lui avait adressé.

\par 13 Alors l'aîné lui dit : « Ne t'étonne pas, ô mon fils ; nous vivons dans le pays au nord du jardin, que Dieu a créé avant le monde. Il ne nous a pas permis d'y vivre, mais il nous a placés à l'intérieur du jardin, au-dessous duquel vous habitez maintenant.

\par 14 «Mais après cela j'ai transgressé, Il m'a fait sortir de là, et j'ai été laissé habiter dans cette grotte; de ​​grands et douloureux problèmes m'ont survenu; et quand ma mort approchait, j'ai ordonné à mon fils Seth de s'occuper bien son peuple ; et ceci est mon commandement qui doit être transmis de l'un à l'autre jusqu'à la fin des générations à venir.

\par 15 «Mais, ô Jared, mon fils, nous vivons dans de belles régions, tandis que toi tu vis ici dans la misère, comme me l'a informé ton père Mahalaleel, me disant qu'un grand déluge viendra et submergera la terre entière.

\par 16 « C'est pourquoi, mon fils, craignant pour toi, je me suis levée, j'ai pris mes enfants avec moi, et je suis venue ici pour que nous te rendions visite, à toi et à tes enfants ; mais je t'ai trouvée debout dans cette caverne, pleurant, et tes enfants dispersés sur cette montagne, dans la chaleur et la misère.

\par 17 « Mais, ô mon fils, comme nous nous sommes égarés et sommes arrivés jusqu'ici, nous avons trouvé d'autres hommes au-dessous de cette montagne ; qui habitent un beau pays, plein d'arbres et de fruits, et de toutes sortes de verdure. ; c'est comme un jardin ; de sorte que lorsque nous les avons trouvés, nous avons pensé que c'était vous ; jusqu'à ce que ton père Mahalaleel me dise qu'ils n'étaient pas de telles choses.

\par 18 « Maintenant donc, ô mon fils, écoute mon conseil, et descends vers eux, toi et tes enfants. Vous vous reposerez de toutes ces souffrances dans lesquelles vous vous trouvez. Mais si tu ne veux pas descendre vers eux, lève-toi, prends tes enfants et viens avec nous dans notre jardin ; vous vivrez dans notre beau pays, et vous vous reposerez de tous ces troubles que vous et vos enfants portez actuellement.

\par 19 Mais Jared, lorsqu'il entendit ce discours de l'aîné, se demanda : et alla çà et là, mais à ce moment il ne trouva aucun de ses enfants.

\par 20 Alors il répondit et dit à l'ancien : « Pourquoi vous êtes-vous cachés jusqu'à ce jour ?

\par 21 Et l'aîné répondit : « Si ton père ne nous l'avait pas dit, nous ne l'aurions pas su. »

\par 22 Alors Jared a cru que ses paroles étaient vraies.

\par 23 Alors cet ancien dit à Jared : « Pourquoi t'es-tu retourné, untel et untel ? Et il dit : « Je cherchais un de mes enfants pour lui parler de mon départ avec toi et de leur descente vers ceux dont tu m'as parlé. »

\par 24 Lorsque l'aîné entendit l'intention de Jared, il lui dit : « Laisse de côté ce projet pour le moment, et viens avec nous ; tu verras notre pays; si le pays dans lequel nous habitons te plaît, nous et toi reviendrons ici et emmènerons votre famille avec nous. Mais si notre pays ne te plaît pas, tu reviendras chez toi.

\par 25 Et l'aîné pressa Jared de partir avant qu'un de ses enfants ne vienne lui conseiller autrement.

\par 26 Jared sortit donc de la grotte et alla avec eux et parmi eux. Et ils le consolèrent jusqu'à ce qu'ils arrivèrent au sommet de la montagne des fils de Caïn.

\par 27 Alors l'aîné dit à l'un de ses compagnons : « Nous avons oublié quelque chose près de l'entrée de la grotte, et c'est le vêtement choisi que nous avions apporté pour vêtir Jared. »

\par 28 Il dit alors à l'un d'eux : « Retourne, toi, quelqu'un ; et nous t'attendrons ici jusqu'à ce que tu reviennes. Alors nous habillerons Jared et il sera comme nous, bon, beau et apte à venir avec nous dans notre pays.

\par 29 Puis celui-là est reparti.

\par 30 Mais alors qu'il était à peu de distance, l'ancien l'appela et lui dit : « Attends jusqu'à ce que je monte et te parle. »

\par 31 Alors il s'arrêta, et l'ancien s'approcha de lui et lui dit : « Une chose que nous avons oubliée dans la grotte, c'est celle-ci : éteindre la lampe qui brûle à l'intérieur, au-dessus des corps qui s'y trouvent. . Alors revenez vers nous, vite.

\par 32 Celui-là partit, et l'aîné revint vers ses camarades et vers Jared. Et ils descendirent de la montagne, et Jared avec eux ; Et ils restèrent près d'une fontaine d'eau, près des maisons des enfants de Caïn, et attendirent leur compagnon jusqu'à ce qu'il apporte le vêtement pour Jared.

\par 33 Celui donc qui retourna à la grotte, éteignit la lampe, vint vers eux, amena un fantôme avec lui et le leur montra. Et quand Jared l'a vu, il s'est étonné de sa beauté et de sa grâce, et s'est réjoui dans son cœur en croyant que tout était vrai.

\par 34 Mais pendant qu'ils étaient là, trois d'entre eux entrèrent dans les maisons des fils de Caïn, et leur dirent : Apportez-nous aujourd'hui de la nourriture près de la source d'eau, pour que nous et nos compagnons puissions manger.

\par 35 Mais quand les fils de Caïn les virent, ils s'étonnèrent d'eux et pensèrent : « Ils sont beaux à regarder, et comme nous n'en avons jamais vu auparavant. » Ils se levèrent donc et les accompagnèrent à la fontaine d'eau pour voir leurs compagnons.

\par 36 Ils les trouvèrent si beaux, qu'ils crièrent à haute voix sur leurs places pour que d'autres se rassemblent et viennent regarder ces beaux êtres. Puis ils rassemblèrent autour d’eux hommes et femmes.

\par 37 Alors l'ancien leur dit : Nous sommes étrangers dans votre pays, apportez-nous de la bonne nourriture et buvez, vous et vos femmes, pour nous rafraîchir avec vous.

\par 38 Quand ces hommes entendirent ces paroles de l'aîné, chacun des fils de Caïn amena sa femme, et un autre amena sa fille, et ainsi, beaucoup de femmes vinrent vers eux ; chacun s'adressant à Jared soit pour lui-même, soit pour sa femme ; tous pareils.

\par 39 Mais quand Jared vit ce qu'ils faisaient, son âme même s'arracha d'eux ; il ne goûterait pas non plus leur nourriture ou leur boisson.

\par 40 L'aîné vit une allusion alors qu'il s'arrachait à eux et lui dit : « Ne sois pas triste ; Je suis le grand aîné, comme tu me verras faire, fais-le toi-même.

\par 41 Puis il étendit les mains et prit une des femmes, et cinq de ses compagnons firent de même devant Jared, afin qu'il fasse comme eux.

\par 42 Mais quand Jared les vit commettre l'infamie, il pleura et dit dans son esprit : Mes pères n'ont jamais fait pareille.

\par 43 Alors il étendit les mains et pria avec un cœur fervent et avec beaucoup de larmes, et supplia Dieu de le délivrer de leurs mains.

\par 44 A peine Jared commença-t-il à prier que l'aîné s'enfuit avec ses compagnons ; car ils ne pouvaient pas demeurer dans un lieu de prière.

\par 45 Alors Jared se retourna mais ne pouvait pas les voir, mais se trouva debout au milieu des enfants de Caïn.

\par 46 Il pleura alors et dit : « Ô Dieu, ne me détruis pas avec cette race, au sujet de laquelle mes pères m'ont prévenu ; car maintenant, ô mon Seigneur Dieu, je pensais que ceux qui m'étaient apparus étaient mes pères ; mais J'ai découvert que c'étaient des démons, qui m'ont séduit par cette belle apparition, jusqu'à ce que je les croie.

\par 47 «Mais maintenant je te demande, ô Dieu, de me délivrer de cette race, parmi laquelle je demeure maintenant, comme tu m'as délivré de ces démons. Envoie ton ange pour me tirer du milieu d'eux; car Je n'ai pas moi-même le pouvoir de m'échapper du milieu d'eux.

\par 48 Quand Jared eut terminé sa prière, Dieu envoya son ange au milieu d'eux, qui prit Jared et le plaça sur la montagne, lui montra le chemin, lui donna des conseils, puis s'éloigna de lui.

\chapter{18}

\par \textit{Confusion dans la Grotte aux Trésors. Discours miraculeux d'Adam mort.}

\par 1 LES enfants de Jared avaient l'habitude de lui rendre visite heure après heure, pour recevoir sa bénédiction et lui demander conseil pour tout ce qu'ils faisaient ; et quand il avait une œuvre à faire, ils la faisaient pour lui.

\par 2 Mais cette fois, quand ils entrèrent dans la grotte, ils ne trouvèrent pas Jared, mais ils trouvèrent la lampe éteinte et les corps des pères jetés partout, et des voix sortaient d'eux par la puissance de Dieu, qui disaient : « Satan, dans une apparition, a trompé notre fils, voulant le détruire, comme il a détruit notre fils Caïn.»

\par 3 Ils dirent aussi : « Seigneur Dieu du ciel et de la terre, délivre notre fils de la main de Satan, qui a fait devant lui une grande et fausse apparition. » Ils parlèrent aussi d'autres choses, par la puissance de Dieu.

\par 4 Mais quand les enfants de Jared entendirent ces voix, ils furent saisis de crainte et pleurèrent leur père ; car ils ne savaient pas ce qui lui était arrivé.

\par 5 Et ils le pleurèrent ce jour-là jusqu'au coucher du soleil.

\par 6 Alors vint Jared avec un visage triste, misérable d'esprit et de corps, et triste d'avoir été séparé des corps de ses pères.

\par 7 Mais comme il s'approchait de la grotte, ses enfants l'aperçurent, et se précipitèrent vers la grotte, et se pendirent à son cou, criant et lui disant : « Ô père, où étais-tu, et pourquoi as-tu été ? nous a quittés, comme tu n’avais pas l’habitude de le faire ? Et encore : « Ô père, quand tu as disparu, la lampe au-dessus des corps de nos pères s'est éteinte, les corps ont été jetés et des voix sont sorties d'eux. »

\par 8 Quand Jared entendit cela, il fut désolé et entra dans la grotte ; et là on trouva les corps jetés, la lampe éteinte, et les pères eux-mêmes priant pour qu'il soit délivré de la main de Satan.

\par 9 Alors Jared tomba sur les corps et les embrassa, et dit : « Ô mes pères, par votre intercession, que Dieu me délivre de la main de Satan ! Et je vous prie de demander à Dieu de me garder et de m'éloigner de lui jusqu'au jour de ma mort.

\par 10 Alors toutes les voix cessèrent, sauf la voix de notre père Adam, qui parlait à Jared par la puissance de Dieu, tout comme on parlerait à son prochain, en disant : « Ô Jared, mon fils, offre des cadeaux à Dieu pour avoir délivré de la main de Satan ; et quand tu apporteras ces offrandes, qu'il en soit ainsi que tu les offres sur l'autel sur lequel j'ai offert. Ensuite aussi, méfie-toi de Satan, car il m'a trompé à maintes reprises par ses apparitions, souhaitant pour me détruire, mais Dieu m'a délivré de sa main.

\par 11 « Ordonne à ton peuple de se tenir en garde contre lui ; et ne cessez jamais d’offrir des cadeaux à Dieu.

\par 12 Alors la voix d'Adam se tut aussi ; et Jared et ses enfants s'en étonnèrent. Ensuite, ils déposèrent les corps tels qu'ils étaient d'abord ; et Jared et ses enfants prièrent toute la nuit, jusqu'à l'aube.

\par 13 Alors Jared fit une offrande et l'offrit sur l'autel, comme Adam le lui avait ordonné. Et tandis qu'il montait à l'autel, il pria Dieu pour qu'il lui accorde miséricorde et lui pardonne son péché concernant l'extinction de la lampe.

\par 14 Alors Dieu apparut à Jared sur l'autel et le bénit ainsi que ses enfants, et accepta leurs offrandes ; et ordonna à Jared de prendre le feu sacré de l'autel et d'allumer avec lui la lampe qui éclairait le corps d'Adam.

\chapter{19}

\par \textit{Les enfants de Jared sont égarés.}

\par 1 ALORS Dieu lui révéla de nouveau la promesse qu'Il avait faite à Adam ; Il lui expliqua les 5 500 ans et lui révéla le mystère de sa venue sur terre.

\par 2 Et Dieu dit à Jared : « Quant au feu que tu as pris de l'autel pour allumer la lampe avec lui, laisse-le demeurer avec toi pour éclairer les corps ; et qu'il ne sorte pas de la grotte, jusqu'à ce que le corps d'Adam en sort.

\par 3 Mais, ô Jared, prends soin du feu, afin qu'il brûle brillamment dans la lampe ; ne sors pas non plus de la grotte, jusqu'à ce que tu reçoives un ordre par une vision, et non par une apparition, lorsque tu le verras.

\par 4 « Alors commande encore à ton peuple de ne pas avoir de relations sexuelles avec les enfants de Caïn, et de ne pas apprendre leurs voies ; car je suis Dieu qui n’aime ni la haine ni les œuvres d’iniquité.

\par 5 Dieu donna aussi beaucoup d'autres commandements à Jared et le bénit. Et puis il lui a retiré Sa Parole.

\par 6 Alors Jared s'approcha avec ses enfants, prit du feu, descendit à la grotte et alluma la lampe devant le corps d'Adam ; et il donna à son peuple des commandements comme Dieu lui avait dit de le faire.

\par 7 Ce signe arriva à Jared à la fin de sa quatre cent cinquantième année ; comme bien d’autres merveilles, nous n’enregistrons pas. Mais nous n’enregistrons que celui-ci par souci de brièveté et pour ne pas allonger notre récit.

\par 8 Et Jared continua à instruire ses enfants pendant quatre-vingts ans ; mais après cela, ils commencèrent à transgresser les commandements qu'il leur avait donnés et à faire beaucoup de choses sans son conseil. Ils commencèrent à descendre de la Montagne Sainte l'un après l'autre et à se mêler aux enfants de Caïn, dans des communions immondes.

\par 9 Or, la raison pour laquelle les enfants de Jared descendirent de la Montagne Sainte, c'est celle-là que nous allons maintenant vous révéler.



\chapter{20}

\par \textit{Musique ravissante ; boisson forte répandue parmi les fils de Caïn. Ils portent des vêtements colorés. Les enfants de Seth regardent avec des yeux impatients. Ils se révoltent contre les sages conseils ; ils descendent de la montagne dans la vallée de l'iniquité. Ils ne peuvent plus gravir la montagne.}

\par 1 APRÈS que Caïn fut descendu au pays de la terre sombre, et que ses enfants s'y furent multipliés, il y en avait un, nommé Genun, fils de Lémec l'aveugle qui tua Caïn.

\par 2 Mais quant à ce Genun, Satan est entré en lui dans son enfance ; et il fabriqua diverses trompettes et cors, et instruments à cordes, cymbales et psaltères, et lyres, et harpes, et flûtes ; et il jouait dessus à tout moment et à toute heure.

\par 3 Et quand il jouait sur eux, Satan entra en eux, de sorte que parmi eux se faisaient entendre des sons beaux et doux, qui ravissaient le cœur.

\par 4 Puis il rassembla compagnies sur compagnies pour jouer sur elles ; et quand ils jouaient, cela plaisait bien aux enfants de Caïn, qui s'enflammaient entre eux de péché et brûlaient comme par le feu ; tandis que Satan enflammait leurs cœurs les uns avec les autres et augmentait la convoitise parmi eux.

\par 5 Satan enseigna aussi à Genun à faire sortir des boissons fortes de la terre ; et ce Genun avait l'habitude de rassembler des compagnies sur des compagnies dans des débits de boissons ; et apportèrent entre leurs mains toutes sortes de fruits et de fleurs ; et ils burent ensemble.

\par 6 Ainsi ce Genun multiplia excessivement le péché ; il a également agi avec orgueil et a enseigné aux enfants de Caïn à commettre toutes sortes de méchancetés les plus grossières, qu'ils ne connaissaient pas ; et les soumettaient à de multiples actes qu'ils ne connaissaient pas auparavant.

\par 7 Alors Satan, quand il vit qu'ils cédaient à Genun et l'écoutaient dans tout ce qu'il leur disait, se réjouit grandement, élargit l'intelligence de Genun, jusqu'à ce qu'il prenne du fer et avec lui fabriqua des armes de guerre.

\par 8 Alors, quand ils étaient ivres, la haine et le meurtre augmentaient parmi eux ; un homme usait de violence contre un autre pour lui enseigner le mal en prenant ses enfants et en les souillés devant lui.

\par 9 Et quand les hommes virent qu'ils étaient vaincus, et qu'ils en virent d'autres qui ne l'étaient pas, ceux qui étaient vaincus vinrent à Genun, se réfugièrent auprès de lui, et il en fit ses confédérés.

\par 10 Alors le péché augmenta considérablement parmi eux ; jusqu'à ce qu'un homme épouse sa propre sœur, ou sa fille, ou sa mère, et d'autres ; ou la fille de la sœur de son père, de sorte qu'il n'y avait plus de distinction de parenté, et qu'ils ne savaient plus ce qu'est l'iniquité ; mais ils ont fait le mal, et la terre a été souillée par le péché, et ils ont irrité Dieu le Juge, qui les avait créés.

\par 11 Mais Genun rassembla des troupes sur des troupes qui jouaient du cor et de tous les autres instruments dont nous avons déjà parlé, au pied de la Montagne Sainte ; et ils le firent afin que les enfants de Seth qui étaient sur la Montagne Sainte l'entendent.

\par 12 Mais quand les enfants de Seth entendirent le bruit, ils furent dans l'étonnement, et vinrent par groupes, et se tinrent au sommet de la montagne pour regarder ceux d'en bas ; et ils firent ainsi pendant une année entière.

\par 13 Quand, à la fin de cette année-là, Genun vit qu'ils se gagnaient peu à peu à lui, Satan entra en lui et lui apprit à faire des teintures pour des vêtements de divers motifs, et lui fit comprendre comment teindre le cramoisi et le violet et ainsi de suite.

\par 14 Et les fils de Caïn, qui ont fait tout cela et qui brillaient par leur beauté et leurs vêtements somptueux, se sont rassemblés au pied de la montagne avec splendeur, avec des cornes et des vêtements somptueux, et des courses de chevaux, commettant toutes sortes d'abominations.

\par 15 Pendant ce temps, les enfants de Seth, qui étaient sur la Montagne Sainte, priaient et louaient Dieu, à la place des armées d'anges tombées ; c'est pourquoi Dieu les avait appelés « anges », parce qu'il se réjouissait grandement à leur sujet.

\par 16 Mais après cela, ils ne gardèrent plus son commandement, ni ne tenèrent plus à la promesse qu'il avait faite à leurs pères ; mais ils se détendirent de leur jeûne et de leurs prières, ainsi que des conseils de Jared, leur père. Et ils se rassemblaient continuellement au sommet de la montagne, pour contempler les enfants de Caïn, du matin au soir, et ce qu'ils faisaient, leurs beaux vêtements et leurs ornements.

\par 17 Alors les enfants de Caïn levèrent les yeux d'en bas, et virent les enfants de Seth, debout en troupes au sommet de la montagne ; et ils les appelèrent à descendre vers eux.

\par 18 Mais les enfants de Seth leur dirent d'en haut : Nous ne connaissons pas le chemin. Alors Genun, fils de Lémec, les entendit dire qu'ils ne connaissaient pas le chemin, et il réfléchit comment il pourrait les faire tomber.

\par 19 Alors Satan lui apparut pendant la nuit, disant : « Il n'y a aucun moyen pour eux de descendre de la montagne sur laquelle ils habitent ; mais quand ils viendront demain, dites-leur : Venez au versant occidental de la montagne ; là vous trouverez le chemin d'un ruisseau d'eau qui descend jusqu'au pied de la montagne, entre deux collines ; descends par ici vers nous.

\par 20 Et quand le jour fut venu, Genun sonna du cor et frappa des tambours au pied de la montagne, comme il avait l'habitude de le faire. Les enfants de Seth l'entendirent et vinrent comme ils avaient l'habitude de le faire.

\par 21 Alors Genun leur dit d'en bas : « Allez vers le côté occidental de la montagne, vous y trouverez le chemin pour descendre. »

\par 22 Mais lorsque les enfants de Seth entendirent ces paroles de sa part, ils retournèrent dans la grotte vers Jared pour lui raconter tout ce qu'ils avaient entendu.

\par 23 Alors, quand Jared l'entendit, il fut attristé ; car il savait qu'ils transgresseraient son conseil.

\par 24 Après cela, cent hommes des enfants de Seth se rassemblèrent et dirent entre eux : « Venez, descendons vers les enfants de Caïn, voyons ce qu'ils font, et amusons-nous avec eux. »

\par 25 Mais quand Jared entendit cela de la part des cent hommes, son âme même fut émue et son cœur fut attristé. Il se leva alors avec une grande ferveur, se tint au milieu d'eux et les adjura par le sang d'Abel le juste : « Qu'aucun de vous ne descende de cette montagne sainte et pure, dans laquelle nos pères lui ont ordonné d'habiter.»

\par 26 Mais quand Jared vit qu'ils n'avaient pas reçu ses paroles, il leur dit : « Ô mes bons, innocents et saints enfants, sachez que lorsque vous descendez de cette montagne sainte, Dieu ne vous permettra pas d'y revenir. à cela.

\par 27 Il les adjura de nouveau, disant : « Je vous adjure, par la mort de notre père Adam, et par le sang d'Abel, de Seth, d'Enos, de Caïnan et de Mahalaleel, de m'écouter et de ne pas descendre de cette montagne sainte ; car, dès que vous la quitterez, vous serez privés de la vie et de la miséricorde. ; et vous ne serez plus appelés « enfants de Dieu », mais « enfants du diable ».

\par 28 Mais ils n'écoutèrent pas ses paroles.

\par 29 Enoch, à cette époque, était déjà grand, et dans son zèle pour Dieu, il se leva et dit : « Hear me, O ye sons of Seth, small and great—when ye transgress the commandment of our fathers, and go down from this holy mountain—ye shall not come up hither again for ever.»

\par 30 Mais ils se soulevèrent contre Enoch, et ne voulurent pas écouter ses paroles, mais descendirent de la Montagne Sainte.

\par 31 Et quand ils regardèrent les filles de Caïn, leurs belles figures, leurs mains et leurs pieds teints de couleur et tatoués d'ornements sur leurs visages, le feu du péché s'alluma en elles.

\par 32 Alors Satan les rendit très beaux devant les fils de Seth, comme il fit aussi paraître les fils de Seth parmi les plus beaux aux yeux des filles de Caïn, de sorte que les filles de Caïn convoitèrent les fils de Seth comme des bêtes voraces, et les fils de Seth après les filles de Caïn, jusqu'à ce qu'ils commettent des abominations avec eux.

\par 33 Mais après être ainsi tombés dans cette souillure, ils revinrent par le chemin par lequel ils étaient venus et essayèrent de gravir la Montagne Sainte. Mais ils ne le pouvaient pas, parce que les pierres de cette montagne sainte étaient en feu devant eux, à cause de laquelle ils ne pouvaient plus remonter.

\par 34 Et Dieu fut en colère contre eux, et se repentit d'eux parce qu'ils étaient descendus de la gloire, et qu'ils avaient ainsi perdu ou abandonné leur propre pureté ou innocence, et étaient tombés dans la souillure du péché.

\par 35 Alors Dieu envoya Sa Parole à Jared, disant : « Ces tes enfants, que tu as appelés « Mes enfants », voici, ils ont transgressé Mon commandement et sont descendus dans la demeure de la perdition et du péché. Envoie un messager à ceux qui restent, afin qu'ils ne descendent pas et ne se perdent pas.

\par 36 Alors Jared pleura devant le Seigneur et lui demanda miséricorde et pardon. Mais il souhaitait que son âme quitte son corps, plutôt que d'entendre ces paroles de Dieu concernant la descente de ses enfants de la Montagne Sainte.

\par 37 Mais il suivit l'ordre de Dieu et leur prêcha de ne pas descendre de cette montagne sainte et de ne pas avoir de relations sexuelles avec les enfants de Caïn.

\par 38 Mais ils n'écoutèrent pas son message et n'obéirent pas à ses conseils.

\chapter{21}

\par \textit{Jared meurt dans le chagrin pour ses fils qui s'étaient égarés. Une prédiction du Déluge.}

\par 1 APRÈS cela, une autre troupe se rassembla et ils allèrent s'occuper de leurs frères ; mais ils périrent aussi bien qu'eux. Et ainsi de suite, entreprise après entreprise, jusqu’à ce qu’il n’en reste plus que quelques-uns.

\par 2 Alors Jared tomba malade de chagrin, et sa maladie était telle que le jour de sa mort approchait.

\par 3 Puis il appela Hénoc son fils aîné, et Mathusalem, fils d'Hénoc, et Lamec, fils de Mathusalem, et Noé, fils de Lémec.

\par 4 Et quand ils furent venus vers lui, il pria pour eux et les bénit, et leur dit : « Vous êtes des fils justes et innocents ; ne descendez pas de cette montagne sainte ; car voici, vos enfants et les enfants de vos enfants ont sont descendus de cette montagne sainte et se sont éloignés de cette montagne sainte, à cause de leur abominable convoitise et de leur transgression du commandement de Dieu.

\par 5 «Mais je sais, par la puissance de Dieu, qu'il ne vous laissera pas sur cette montagne sainte, parce que vos enfants ont transgressé son commandement et celui de nos pères, que nous avions reçu d'eux.

\par 6 «Mais, ô mes fils, Dieu vous emmènera dans un pays étranger, et vous ne reviendrez plus jamais voir de vos yeux ce jardin et cette montagne sainte.

\par 7 « C'est pourquoi, ô mes fils, concentrez-vous sur vous-mêmes et gardez le commandement de Dieu qui est avec vous. Et lorsque vous quitterez cette montagne sainte pour aller dans un pays étranger que vous ne connaissez pas, emportez avec vous le corps de notre père Adam, et avec lui ces trois précieux dons et offrandes, à savoir l'or, l'encens et la myrrhe, et qu'ils soient à l'endroit où reposera le corps de notre père Adam.

\par 8 «Et à celui d'entre vous qui restera, ô mes fils, la Parole de Dieu viendra, et quand il sortira de ce pays, il prendra avec lui le corps de notre père Adam, et le déposera dans le milieu de la terre, le lieu où le salut sera opéré. »

\par 9 Alors Noé lui dit : « Qui est celui d'entre nous qui restera ? »

\par 10 Et Jared répondit : «Tu es celui qui restera. Et tu prendras le corps de notre père Adam de la grotte, et tu le placeras avec toi dans l'arche quand le déluge viendra.

\par 11 «Et ton fils Sem, qui sortira de tes reins, c'est lui qui déposera le corps de notre père Adam au milieu de la terre, dans le lieu d'où viendra le salut.»

\par 12 Alors Jared se tourna vers son fils Enoch et lui dit : « Toi, mon fils, demeure dans cette grotte, et sers diligemment devant le corps de notre père Adam tous les jours de ta vie ; et nourris ton peuple dans la justice et l’innocence.

\par 13 Et Jared n'a rien dit de plus. Ses mains furent relâchées, ses yeux fermés et il entra dans le repos comme ses pères. Sa mort eut lieu la trois cent soixantième année de Noé et la neuf cent quatre-vingt-neuvième année de sa propre vie ; le 12 Takhsas un vendredi.

\par 14 Mais lorsque Jared mourut, des larmes coulèrent sur son visage à cause de sa grande tristesse à cause des enfants de Seth, tombés de son vivant.

\par 15 Alors Hénoch, Mathusalem, Lémec et Noé, ces quatre-là, pleurèrent sur lui ; l'embauma soigneusement, puis le déposa dans la Grotte des Trésors. Puis ils se levèrent et le pleurèrent quarante jours.

\par 16 Et lorsque ces jours de deuil furent terminés, Hénoc, Mathusalem, Lémec et Noé demeurèrent dans une tristesse de cœur, parce que leur père les avait quittés, et qu'ils ne le revoyaient plus.

\chapter{22}

\par \textit{Il ne reste que trois hommes justes dans le monde. Les mauvaises conditions des hommes avant le déluge.}

\par 1 MAIS Enoch a gardé le commandement de Jared son père, et a continué à exercer son ministère dans la grotte.

\par 2 C'est cet Enoch à qui bien des prodiges sont arrivés, et qui a aussi écrit un livre célèbre ; mais ces merveilles ne peuvent pas être racontées ici.

\par 3 Puis après cela, les enfants de Seth s'égarèrent et tombèrent, eux, leurs enfants et leurs femmes. Et quand Enoch, Mathusalem, Lamec et Noé les virent, leurs cœurs souffrèrent à cause de leur chute dans le doute plein d'incrédulité ; et ils pleurèrent et cherchèrent la miséricorde de Dieu, pour les préserver et les faire sortir de cette génération méchante.

\par 4 Enoch a continué son ministère devant le Seigneur trois cent quatre-vingt-cinq ans, et à la fin de ce temps, il a pris conscience, par la grâce de Dieu, que Dieu avait l'intention de l'enlever de la terre.

\par 5 Il dit alors à son fils : « Ô mon fils, je sais que Dieu a l'intention d'apporter les eaux du Déluge sur la terre et de détruire notre création.

\par 6 « Et vous êtes les derniers dirigeants de ce peuple sur cette montagne ; car je sais qu'il ne vous en restera pas un pour engendrer des enfants sur cette montagne sainte ; et aucun de vous ne dominera sur les enfants de son peuple ; et il ne restera de vous aucune grande compagnie sur cette montagne.

\par 7 Hénoc leur dit aussi : « Veillez sur vos âmes, et tenez ferme par votre crainte de Dieu et par votre service, et adorez-le avec une foi droite, et servez-le dans la justice, l'innocence et le jugement, dans la repentance et aussi dans la pureté.

\par 8 Quand Hénoc leur eut achevé ses commandements, Dieu le transporta de cette montagne au pays de la vie, aux demeures des justes et des élus, la demeure du Paradis de joie, dans la lumière qui monte jusqu'au ciel ; la lumière qui est en dehors de la lumière de ce monde ; car c'est la lumière de Dieu qui remplit le monde entier, mais qu'aucun lieu ne peut contenir.

\par 9 Ainsi, parce qu'Hénoc était dans la lumière de Dieu, il se trouva hors de portée de la mort ; jusqu'à ce que Dieu veuille qu'il meure.

\par 10 Au total, aucun de nos pères ni aucun de leurs enfants ne sont restés sur cette montagne sainte, sauf ces trois-là, Mathusalem, Lémec et Noé. Car tous les autres descendirent de la montagne et tombèrent dans le péché avec les enfants de Caïn. C'est pourquoi cette montagne leur fut interdite, et il n'y resta que ces trois hommes.


\end{document}