\begin{document}

\title{5 Maccabées}


\chapter{1}

\par \textit{La tentative d'Héliodore sur le trésor. Il fut ordonné par les rois grecs}

\par 1 Il a été ordonné par les rois des Gentils grecs que de grosses sommes d'argent seraient envoyées chaque année dans la ville sainte et devraient être remises aux prêtres, afin qu'ils puissent l'ajouter au trésor de la maison de Dieu. , comme argent pour les receveurs d'aumônes [orphelins] et pour les veuves.

\par 2 Or Séleucus était roi en Macédoine ; il avait un ami, l'un de ses capitaines, appelé Héliodore. Cet homme fut envoyé pour piller le trésor et prendre tout l'argent qui s'y trouvait.

\par 3 Lorsque cela se fit entendre au dehors, cela provoqua une grande tristesse parmi les citoyens ; et ils craignaient qu'Héliodore n'allât plus loin ;

\par 4 car ils n'avaient pas assez de pouvoir pour l'empêcher d'exécuter ses ordres.

\par 5 C'est pourquoi ils ont tous couru vers Dieu pour obtenir de l'aide, et ont ordonné un jeûne général, et ont supplié avec humilité, fléchissant les genoux et de grands lamentations ;

\par 6 se revêtirent de sacs et se roulèrent dans la cendre, avec Onias, le grand prêtre, et les autres princes, et les anciens, même le peuple, les femmes et les enfants.

\par 7 Et le lendemain, Héliodore entra dans la maison de Dieu avec une suite de disciples ; Il entra dans la maison avec ses fantassins, lui-même à cheval, et cherchait de l'argent.

\par 8 Mais le Dieu grand et bon envoya contre lui une voix forte et terrible ; et il vit un homme armé d'armes de guerre, monté sur un grand cheval, et s'avançant contre lui :

\par 9 C'est pourquoi il fut saisi de frayeur et de tremblement ; et cet homme s'approcha de lui, le fit tomber de sa selle, et le frappa violemment à terre.

\par 10 De sorte qu'étant extrêmement frappé de terreur et effrayé, il devint muet.

\par 11 Mais quand ses serviteurs virent ce qui lui était arrivé, et ne purent apercevoir personne qui lui avait fait ces choses, ils le transportèrent en toute hâte dans sa propre maison :

\par 12 et il resta plusieurs jours, sans parler ni prendre de nourriture.

\par 13 C'est pourquoi les principaux de ses amis allèrent trouver le prêtre Onias, le suppliant de s'apaiser à son égard et de supplier le Dieu grand et bon de ne pas le punir.

\par 14 Ce qu'a fait Onias ; et Héliodore fut guéri de sa maladie.

\par 15 Et il vit dans une vision la personne qu'il avait vue dans le sanctuaire, lui ordonnant d'aller trouver le prêtre Onias, de le saluer et de lui rendre les honneurs convenables ; lui disant que le grand et bon Dieu avait entendu ses prières et l'avait guéri à la demande d'Onias.

\par 16 Héliodore courut donc vers le prêtre Onias, qu'il salua en tombant. et lui donna de l'argent de toutes sortes !, lui demandant de l'ajouter à celui qui était dans le trésor.

\par 17 Puis il partit de Jérusalem pour le pays de Macédoine, et raconta au roi Séleucus ce qui lui était arrivé ; suppliant qu'il ne le forcerait pas à devenir son représentant à Jérusalem.

\par 18 C'est pourquoi le roi s'étonnait des choses que lui avait racontées Héliodore ; et lui ordonna de les publier dans le monde.

\par 19 Et il veilla à ce que ses hommes fussent retirés et renvoyés de Jérusalem, augmentant les présents qu'il y envoyait chaque année, à cause de ce qui était arrivé à Héliodore.

\par 20 Et les rois ajoutèrent davantage à l'argent qu'ils ordonnaient de donner aux prêtres, afin qu'il soit dépensé pour les orphelins et les veuves ; aussi à ce qui devait être dépensé pour les sacrifices.

\chapter{2}

\par \textit{L'histoire de la traduction des vingt-quatre livres de la langue hébraïque vers la langue grecque, pour Ptolémée, roi d'Égypte.}

\par 1 Il y avait un homme de Macédoine nommé Ptolémée, doué de connaissance et d'intelligence ; que, comme il demeurait en Égypte, les Égyptiens l'établirent roi du pays d'Égypte.

\par 2 C'est pourquoi lui, possédé du désir de rechercher diverses connaissances, rassembla de tous côtés tous les livres des sages.

\par 3 Et étant impatient d'obtenir « les vingt-quatre livres », il écrivit au grand prêtre de Jérusalem, de lui envoyer soixante-dix anciens parmi ceux qui étaient les plus habiles dans ces livres ; et il envoya au prêtre une lettre avec un présent.

\par 4 Ainsi, lorsque la lettre du roi parvint au prêtre, il choisit soixante-dix hommes instruits et les envoya avec un homme nommé Éléazar, excellent en religion, en science et en savoir, qui partit en Égypte.

\par 5 Et lorsque leur approche fut signalée au roi, il ordonna que soixante-dix logements soient préparés et que les hommes y soient reçus.

\par 6 Il ordonna également qu'un secrétaire soit nommé pour chacun, qui notait l'interprétation de ces livres en caractère et en langue grecque.

\par 7 Il défendit également à l'un d'entre eux d'avoir des relations avec aucun de ses semblables ; de peur qu’ils ne s’entendent pour apporter quelque changement à ces livres.

\par 8 Les secrétaires retirèrent donc de chacun d'eux la traduction des « vingt-quatre livres ».

\par 9 Et quand les traductions furent terminées, Éléazar les apporta au roi ; et les compara ensemble en sa présence : sur quelle comparaison, ils se trouvèrent d'accord.

\par 10 Ce à quoi le roi fut extrêmement heureux et ordonna qu'une grosse somme d'argent soit partagée entre le groupe. Mais Eleazar Hanase (fi) le récompensa par une généreuse récompense.

\par 11 Ce jour-là aussi, il relâcha tous les captifs qui se trouvaient en Égypte, de la tribu de Juda et de Benjamin, afin qu'ils puissent retourner dans leur pays, la Syrie.

\par 12 Leur nombre était d'environ cent trente mille.

\par 13 Et il ordonna qu'on leur distribuât de l'argent, de sorte que plusieurs deniers reviendraient à la part de chacun ; qui, les recevant, s'en allèrent dans leur propre pays.

\par 14 Puis il ordonna de faire une grande table de l'or le plus pur, qui soit assez grande pour contenir une représentation de tout le pays d'Égypte et une image du Nil, depuis le commencement de son cours jusqu'à la fin de il en Egypte, avec ses diverses divisions à travers le pays, et comment il lave tout le pays.

\par 15 Il ordonna également que la table soit garnie de nombreuses pierres précieuses.

\par 16 Et cette table fut faite ; et sa sculpture fut achevée, et elle fut sertie de pierres précieuses ; et elle fut emportée dans la ville de Jérusalem, comme présent à la magnifique maison.

\par 17 Et, arrivé sain et sauf, il fut placé dans la maison, selon l'ordre du roi. Et en vérité, les hommes ne sont jamais considérés comme pareils, pour la beauté des tableaux et l'excellence du travail.

\chapter{3}

\par \textit{L'histoire des Juifs. Une relation de ce qui est arrivé aux Juifs sous le roi Antiochus ; et quelles batailles ont eu lieu entre eux et ses capitaines ; et jusqu'où il est finalement allé.}

\par 1 Il y avait un homme des rois de Macédoine, appelé Antiochus ; parmi les actes de qui se trouvait celui-ci :

\par 2 que lorsque Ptolémée, le roi d'Égypte susmentionné, fut mort, il alla avec ses armées attaquer le second Ptolémée. Et après avoir vaincu et tué Ptolémée, il a conquis son pays ? l'Égypte et en prit possession.

\par 3 De là, à mesure que ses affaires gagnaient en force, il soumit une grande partie de la terre ; le roi de Perse et d'autres lui obéissant.

\par 4 C'est pourquoi son cœur s'est élevé ; et, enflé d'orgueil, il a ordonné que des images soient faites à son image ; que les hommes devraient les adorer, à sa glorification et à son honneur.

\par 5 Et quand ceux-ci furent faits, il envoya des messagers dans toutes les régions de son empire, ordonnant qu'ils soient adorés et adorés. Les nations acquiescèrent à ces commandements, craignant et redoutant sa tyrannie.

\par 6 Il y avait alors en Judée trois hommes, les pires de tous les mortels ; et chacun d’eux avait, pour ainsi dire, un lien avec le même genre de vice. Le nom de l’un de ces trois était Ménélas ; du second, Siméon ; du troisième, Alcimus.

\par 7 Et vers ce temps-là apparurent certaines images que les citoyens de Jérusalem contemplèrent dans les airs pendant quarante jours : c'étaient des apparitions d'hommes montés sur des chevaux enflammés, combattant les uns contre les autres.

\par 8 Ces hommes impies se rendirent donc chez Antiochus pour obtenir de lui quelque autorisation, afin de pouvoir commettre facilement tout ce qu'ils voudraient, de la prostitution et du pillage des biens des hommes ; et en bref, pourrait régner sur les autres et les maintenir soumis. Et ils lui dirent :

\par 9 « Ô roi, des cavaliers enflammés sont apparus dernièrement dans les airs au-dessus de Jérusalem, se battant les uns contre les autres ; et à cause de cela les Hébreux se sont réjouis, disant : « cela présageait la mort du roi Antiochus ».

\par 10 Croyant ces paroles, le roi, rempli de colère, se dirigea vers Jérusalem dans les plus brefs délais ; et tomba sur la nation sans être du tout prévenue de son approche.

\par 11 Et ses hommes attaquèrent les habitants, et les frappèrent avec l'épée, faisant un très grand massacre ; Ils en blessèrent également beaucoup, et ils en conduisirent une grande multitude en captivité.

\par 12 Mais quelques-uns, s'enfuyant, s'enfuirent dans les montagnes et les bois, où ils demeurèrent longtemps, se nourrissant d'herbes.

\par 13 Après cela, Antiochus résolut de quitter le pays.

\par 14 Mais le mal qu'il avait fait à la nation ne lui suffisait pas : mais il laissa pour substitut un homme nommé Félix, en lui enjoignant de contraindre les Juifs à adorer son image et à manger de la chair de porc.

\par 15 Ce que fit Félix, en envoyant le peuple pour qu'il obéisse au roi dans les choses qu'il lui avait commandées.

\par 16 Mais ils refusèrent de faire les choses auxquelles ils étaient appelés ; c'est pourquoi il en tua une grande multitude ; préserver ces méchants et leur famille, et élever leur dignité.

\chapter{4}

\par \textit{L'histoire de la mort d'Éléazar le prêtre}

\par 1 Ensuite fut saisi Éléazar, qui était allé avec les médecins chez Ptolémée », et était alors un homme très âgé, âgé de quatre-vingt-dix ans ; et il fut placé devant Félix ;

\par 2 qui lui dit : Éléazar, tu es vraiment un homme sage et prudent ; et en effet je t'aime depuis de nombreuses années, et c'est pourquoi je ne devrais pas souhaiter « ta mort » :

\par 3 obéissez donc au roi, adorez son image, mangez de ses sacrifices et partez en toute sécurité.

\par 4 À qui Éléazar répondit : « Je ne suis pas sur le point d’abandonner mon obéissance à Dieu pour obéir au roi. »

\par 5 Et Félix, s'approchant, lui dit tout bas : « Prends soin de faire venir quelqu'un pour t'apporter de la chair de tes propres offrandes, qui sont placées sur ma table :

\par 6 et mange-en une partie en présence du peuple, afin qu'ils sachent que tu as obéi au roi ; et tu sauveras ta vie, sans qu'il soit fait de mal à ta religion.

\par 7 Éléazar lui répondit : « Je n'obéis à Dieu sous aucun prétexte, mais je supporterai plutôt ta violence. Car étant donné que je suis un vieil homme de quatre-vingt-dix ans, mes os sont maintenant affaiblis et mon corps a dépéri.

\par 8 Si donc je supporte avec un esprit courageux ces tourments devant lesquels même les jeunes hommes les plus courageux reculent de peur ; mon peuple et les jeunes de ma nation m'imiteront courageusement et diront : ;

\par 9 'Comment se fait-il que nous ne puissions pas supporter les douleurs qu'a endurées celui qui est inférieur à nous en force et moins substantiel en chair et en os ?'

\par 10 ce qui vaudra mieux pour moi que de les tromper par une feinte obéissance au roi :

\par 11 car ils diront alors : « Si ce vieil homme décrépide, aussi sage et prudent soit-il, s'accroche à la vie et est accablé par la douleur des choses passagères, abdiquant sa religion ; en vérité, ce qui lui était permis nous sera permis, car c'est un vieil homme et un sage, et que nous devons suivre.

\par 12 C'est pourquoi j'aimerais plutôt mourir, leur laissant une constance dans la religion et une patience contre la tyrannie ; que de vivre ; après avoir affaibli leur constance à obéir à leur Seigneur et à suivre ses commandements ; afin que grâce à moi « ils soient rendus heureux et non malheureux ».

\par 13 Or, lorsque Félix eut entendu la détermination d'Éléazar, il fut violemment enragé contre lui et ordonna de le torturer de diverses manières : de sorte qu'il entra dans la lutte mortelle la plus désespérée et dit :

\par 14 « Tu, ô Dieu, sais que j'aurais pu me délivrer des ennuis dans lesquels je suis tombé, en obéissant à un autre plutôt qu'à toi.

\par 15 Cependant, je ne l'ai pas fait ; mais j'ai préféré vous obéir, et j'ai considéré comme légère toute la violence qui m'était offerte, pour la constance dans votre obéissance.

\par 16 Et maintenant, je pense peu aux choses qui me sont arrivées selon ton bon plaisir, et je les supporte du mieux que je peux.

\par 17 Je te prie donc d'accepter cela de ma part, et de me faire mourir avant que je devienne plus faible en endurance.

\par 18 Et Dieu entendit ses prières ; et aussitôt il mourut.

\par 19 Mais il laissa son peuple dévoué au culte de son Dieu, et doté d'une solide force d'âme, de persévérance dans la religion, et de patience pour résister aux épreuves qui les attendaient.

\chapter{5}

\par \textit{L'histoire de la mort des sept frères.}

\par 1 Après cela, sept frères furent arrêtés, ainsi que leur mère ; et ils furent envoyés au roi; car il n'était pas encore loin de Jérusalem.

\par 2 Et lorsqu'ils eurent été portés au roi, l'un d'eux fut amené devant lui ; à qui il a ordonné de renoncer à sa religion :

\par 3 mais il refusant lui dit : « Si tu penses nous enseigner la vérité pour la première fois, il n'en est pas ainsi :

\par 4 car la vérité est celle que nous avons apprise de nos pères, et par laquelle nous nous sommes engagés à embrasser le culte de Dieu seul et à observer constamment la loi ; et nous ne nous en éloignerons en aucun cas.

\par 5 Et le roi Antiechus, irrité par ces paroles, ordonna qu'on apporte une poêle à frire en fer et qu'on la mette sur le feu.

\par 6 Puis il ordonna que la langue du jeune homme soit coupée, que ses mains et ses pieds soient coupés, et que la peau de sa tête soit écorchée et placée dans la poêle. Et ils le firent ainsi. .

\par 7 Puis il ordonna d'apporter un grand chaudron d'airain et de le placer sur le feu, dans lequel le reste de son corps fut jeté.

\par 8 Et comme l'homme était sur le point de mourir, il ordonna qu'on ôte de lui le feu, afin qu'il puisse être torturé plus longtemps, dans l'intention par ces actes d'effrayer sa mère et ses frères.

\par 9 Mais en effet, par cela, il leur donna un surcroît de courage et de force, pour maintenir leur religion avec constance, et pour supporter tous ces tourments que la tyrannie pouvait leur infliger.

\par 10 Le premier étant mort, on lui amena le second, auquel quelques-uns des serviteurs dirent : Obéissez aux ordres que le roi vous donnera, de peur que vous ne périssiez comme votre frère a péri.

\par 11 Mais il répondit : « Je ne suis pas plus faible en esprit que mon frère, ni en retard dans ma foi. Apportez votre feu et votre épée ; et ne diminue en rien ce que tu as fait à mon frère. Et ils lui firent ce qu'on avait fait à son frère.

\par 12 Et il appela le roi, et lui dit : « Écoute, ô monstre de cruauté envers les hommes, et sache que tu ne gagnes rien de nous sauf nos corps ; mais tu n’obtiens en aucun cas nos âmes ; et ceux-ci iront bientôt vers leur Créateur,

\par 13 qu'il restituera à leurs corps, lorsqu'il ressuscitera> les morts de sa nation et les tués de son peuple.

\par 14 Et le troisième fut sorti ; qui fit signe de la main et dit au roi : « Pourquoi nous fais-tu peur, ô ennemi ?

\par 15 sachez que cela nous est envoyé du ciel, que nous subissons aussi comme tels, rendant grâces à Dieu, et de Lui nous espérons notre récompense.

\par 16 Et le roi et ceux qui se tenaient près de lui admiraient le courage du jeune homme, la fermeté de son esprit et son beau discours. Puis il donna des ordres et il fut tué.

\par 17 Et on fit sortir le quatrième, qui dit : « Pour la religion de Dieu, nous mettons nos vies en vente, et nous les louons, afin de lui demander un paiement, le jour où vous n'aurez aucune excuse au jugement, et je ne pourrai pas supporter vos tortures.

\par 18 Le roi ordonna, et il fut mis à mort.

\par 19 Et le cinquième était. amené, qui lui a dit; « Ne pense pas en toi-même que Dieu nous a abandonnés à cause des choses qu'il nous a envoyées.

\par 20 Mais en vérité sa volonté est de nous montrer l'honneur et l'amour par ces choses ; et il nous vengera de toi et de ta postérité.

\par 21 Et le roi commanda, et il fut tué.

\par 22 Et on fit sortir le sixième, qui dit : « J'avoue en effet mes offenses à Dieu, mais je crois qu'elles me seront pardonnées par sa mort.

\par 23 Mais vous vous êtes maintenant opposés à Dieu en tuant ceux qui embrassent sa religion ; et sûrement il vous rendra selon vos œuvres et vous déracinera de sa terre. Et il donna des ordres pour lui, et il fut tué.

\par 24 Et on fit sortir le septième, qui était un garçon.

\par 25 Alors sa mère se leva, intrépide et impassible, et regarda les cadavres de ses enfants :

\par 26 Et elle dit : Mes fils, je ne sais pas comment j'ai conçu chacun de vous, quand je l'ai conçu. Je n’avais pas non plus le pouvoir de lui donner le souffle ; ou de le faire sortir à la lumière de ce monde ; ou de lui conférer du courage et de la compréhension :

\par 27 mais en effet, le Dieu grand et bon lui-même l'a formé selon sa propre volonté, et lui a donné une forme selon son bon plaisir :

\par 28 et l'a mis au monde par sa puissance ; lui assignant une durée de vie, de bonnes règles et une dispensation de religion, comme bon lui semble.

\par 29 Mais vous avez maintenant vendu à Dieu vos corps qu'il a lui-même formés et vos âmes qu'il a créées, et vous avez acquiescé à ses jugements qu'il a décrétés.

\par 30 C'est pourquoi vous êtes heureux dans les choses que vous avez heureusement obtenues ; et vous serez bénis pour les choses dans lesquelles vous « avez été victorieux ».

\par 31 Or, Antiochus avait cru, en la voyant se lever, qu'elle avait fait cela parce qu'elle était envahie par la crainte pour son enfant ; et il pensait entièrement qu'elle était sur le point de lui enjoindre l'obéissance au roi, afin qu'il ne périsse pas comme ses frères avaient péri.

\par 32 Mais après avoir entendu ses paroles, il eut honte et rougit, et ordonna qu'on lui amène le garçon ; afin qu'il puisse l'exhorter et le persuader d'aimer la vie et de le dissuader de la mort :

\par 33 de peur que tous ceux-là ne paraissent s'opposer à son autorité, et que beaucoup d'autres ne suivent leur exemple.

\par 34 C'est pourquoi, lorsqu'on lui fut amené, il l'exhorta par des discours, et lui promit des richesses, et lui jura qu'il le ferait vice-roi de lui-même.

\par 35 Mais comme le garçon n'était pas du tout ému par ses paroles et n'y prêtait aucune attention ; le roi se tourna vers sa mère et lui dit :

\par 36 « Heureuse femme, plains ton fils, que tu as seul survécu ; et exhortez-le à se conformer à mes ordres et à échapper aux souffrances qui sont arrivées à ses frères.

\par 37 Et elle dit : Amenez-le ici, afin que je l'exhorte selon les paroles de Dieu.

\par 38 Et lorsqu'on le lui eut amené, elle s'écarta de la foule ; puis elle l'embrassa, et se moqua de ce que lui avait dit Antiochus.

\par 39 puis il lui dit : Mon fils, viens maintenant, obéis-moi, car je t'ai enfanté, je t'ai allaité, je t'ai élevé et je t'ai enseigné la religion divine.

\par 40 « Regardez maintenant vers le ciel, et la terre, et l'eau, et le feu ; et comprenez que le seul vrai Dieu lui-même les a créés ; et formé un homme de chair et de sang, qui vit peu de temps, puis mourra.

\par 41 C'est pourquoi craignez le vrai Dieu, qui ne meurt pas, et obéissez à l'Être véritable,

\par 42 qui ne change pas ses promesses ; et ne craignez pas ce simple géant ; et mourez pour la religion de Dieu, comme vos frères sont morts.

\par 43 Car si tu pouvais voir, mon fils, leur honorable demeure, et la lumière de leur habitation, et à quelle gloire ils ont atteint, tu ne supporterais pas de ne pas les suivre :

\par 44 et en vérité j'espère aussi que le Dieu grand et bon me préparera, et que je te suivrai de près.

\par 45 Alors le garçon dit : « Sachez que j'obéis bien à Dieu et que je n'obéirai pas aux commandements d'Antiochus : c'est pourquoi, ne tardez pas à me laisser suivre mes frères ; ne m'empêche pas de partir vers le lieu où ils sont allés.

\par 46 Alors il dit au roi : « Malheur à toi de la part de Dieu ! où fuiras-tu loin de lui ? où chercheras-tu un refuge ? ou à qui imploreras-tu l'aide, afin qu'il ne se venge pas de toi ?

\par 47 En vérité tu nous as fait du bien, quand tu avais projeté de nous faire du mal : tu as fait du mal à ton âme, et tu l'as détruite, alors que tu pensais lui faire du bien.

\par 48 Maintenant nous sommes en route vers une vie que la mort ne suivra jamais ; et habitera dans une lumière que les ténèbres ne pourront jamais éloigner.

\par 49 Mais votre demeure sera dans les régions infernales, avec des châtiments exquis de la part de Dieu.

\par 50 Et j'espère que la colère de Dieu se retirera de son peuple, à cause de ce que nous avons souffert pour lui.

\par 51 mais qu'Il vous tourmentera dans ce monde et vous amènera à une mort misérable ; et qu'après vous partirez dans des tourments éternels.

\par 52 Et Antiochus était en colère, voyant que le garçon s'opposait à son autorité ; c'est pourquoi il ordonna qu'il soit torturé encore plus que ses frères. Et cela fut fait, et il mourut.

\par 53 Mais leur mère implora Dieu et le supplia de suivre ses fils ; et aussitôt elle mourut.

\par 54 Alors Antiochus partit pour son pays, la Macédoine, et il écrivit à Félix et aux autres gouverneurs de Syrie de tuer tous les Juifs, sauf ceux qui embrasseraient sa religion.

\par 55 Et ses serviteurs obéirent à son ordre, mettant à mort une multitude d'hommes.

\chapter{6}

\par \textit{L'histoire de Mattathias le grand prêtre, fils de Jochanan, qui est le fils d'Hesmaï le prêtre}

\par 1 Un homme nommé Mattathias, fils de Jochanan, s'enfuit vers l'une des montagnes fortifiées. Et les hommes qui étaient dispersés s'enfuirent vers lui ; et certains se cachèrent dans des lieux retirés.

\par 2 Mais après qu'Antiochus fut parti plus loin du pays, Mattathias envoya secrètement son fils Judas dans les villes de Juda ;

\par 3 pour les certifier de sa propre santé et de celle de son peuple, et désirer que tous ceux qui étaient inspirés de courage, de magnanimité et de zèle pour la religion, pour leurs femmes et leurs enfants, viennent à lui.

\par 4 Et certains des ordres supérieurs du peuple, qui étaient restés en arrière, sortirent vers lui ; et, lorsqu'ils furent arrivés vers lui, leur dit :

\par 5 « Il ne nous reste plus que la prière à Dieu, la confiance en Lui, et le combat contre nos ennemis, si peut-être Dieu nous accorde l'assistance et la victoire sur eux. »

\par 6 Et le peuple accepta l'opinion de Mattathias, et il a agi selon elle.

\par 7 Et cela fut raconté à Félix ; et il marcha contre eux avec une grande armée.

\par 8 Et pendant qu'il était en route, on lui apprit qu'environ un millier de Juifs, hommes et femmes mélangés, étaient rassemblés et demeuraient dans une certaine grotte, afin de pouvoir conserver leur propre façon de culte.

\par 9 Et il se tourna vers eux avec une partie de ses troupes, et envoya les chefs de ses hommes avec le reste de l'armée contre Mattathias.

\par 10 Or Félix demanda à ceux qui étaient dans la grotte de sortir vers lui et de consentir à entrer dans sa religion ; mais ils ont refusé.

\par 11 Sur quoi il menaça de mettre de la fumée dessous ; et ils ont enduré cela, et ne sont pas sortis vers lui ; et il mit de la fumée sous eux, et ils moururent tous.

\par 12 Et lorsque les généraux de son armée marchaient contre Mattathias, et s'approchaient de lui, il était prêt pour le combat ;

\par 13 un des généraux, de sang noble, alla vers lui, lui proposant d'obéir au roi, et de ne pas s'opposer à son autorité ; afin que lui-même vive, ainsi que ceux qui étaient avec lui, et qu'il ne périsse pas.

\par 14 À qui il dit : « J’obéis en effet à Dieu le vrai roi ; mais obéissez à votre roi et faites tout ce qui vous semble bon. » Et il cessa de parler.

\par 15 Et ils commencèrent à lui tendre des pièges.

\par 16 Et voici qu'un homme, l'un des plus mauvais des Juifs qui étaient avec eux, arriva et les incita à marcher contre lui et à préparer la guerre.

\par 17 Et Mattathias se précipita sur lui avec son épée nue, et coupa la tête du Juif ; puis il frappa le général à qui le Juif parlait, et le tua aussi.

\par 18 Mais les compagnons de Mattathias, voyant ce qu'il avait fait, se précipitèrent vers lui ; et ils firent irruption dans le camp des ennemis, en tuèrent un grand nombre et les mirent en fuite ; ensuite ils poursuivirent les fuyards, jusqu'à ce qu'ils les tuèrent tous.

\par 19 Après cela, Mattathias sonna de la trompette et proclama une expédition contre Félix. Et lui et ses compagnons entrèrent dans le pays de Juda, et prirent possession d'un grand nombre de leurs villes.

\par 20 Et le Dieu Très-Haut leur donna du repos par ses mains contre les généraux d'Antiochus ; et ils retournèrent à l'observance de leur propre religion, et les bandes de leurs ennemis se retirèrent devant eux.

\chapter{7}

\par \textit{Le récit de la mort de Mattathias et des actes de Judas, son fils après lui.}

\par 1 Or Mattathias devint infirme. Et lorsqu'il fut sur le point de mourir, il appela ses fils, qui avaient cinq ans, et leur dit :

\par 2 « Je sais avec certitude que de très nombreuses et grandes guerres seront allumées dans le pays de Juda, à cause [ou, en raison] des affaires pour lesquelles le Dieu grand et bon nous a incités à faire la guerre. contre nos ennemis.

\par 3 Mais je vous ordonne de craindre Dieu, de vous confier en lui, et d'être zélé pour la loi, et le sanctuaire, et aussi pour le peuple ;

\par 4 et préparez-vous à faire la guerre à ses ennemis ; et ne craignez pas la mort, car, sans aucun doute, cela est décrété à tous les hommes.

\par 5 De sorte que, si Dieu vous rend victorieux, vous avez immédiatement obtenu ce que vous désiriez ; mais si vous tombez, ce n'est pas une perte pour vous à ses yeux.

\par 6 Et Mattathias mourut et fut enterré ; et ses fils firent ce qu'il leur avait commandé. Et ils convinrent de faire de leur frère Judas leur chef.

\par 7 Or Judas, leur frère, était de tous le meilleur en conseil et le plus courageux en force.

\par 8 Et une armée fut envoyée contre eux par Félix », sous la direction d'un homme appelé Seron', que Judas et sa compagnie mirent en fuite, et il en tua un grand nombre.

\par 9 Et la renommée de Judas se répandit et grandit considérablement aux oreilles des hommes ; et toutes les nations qui l'entouraient le craignaient extrêmement.

\par 10 Et on raconta au roi Antiochus ce qu'avaient fait Mattathias et son fils Judas.

\par 11 La nouvelle en parvint aussi au roi des Perses ; de sorte qu'il a joué faux avec Antiochus, s'éloignant de son amitié, suivant l'exemple de Judas.

\par 12 Ce qui inquiéta beaucoup Antiochus, il appela un de ses officiers de maison, nommé Lysias, un homme vaillant et courageux, et lui dit :

\par 13 J'ai maintenant décidé d'aller au pays de Perse pour faire la guerre ; et je souhaite laisser derrière moi mon fils à ma place ; et de prendre avec moi la moitié de mon armée, et de laisser le reste à mon fils :

\par 14 et voici, je t'ai donné la gouvernance de mon fils, et la gouvernance des hommes que je lui laisse.

\par 15 Et en vérité, vous savez ce que Mattathias et Judas ont fait à mes amis et à mes sujets.

\par 16 C'est pourquoi, envoie quelqu'un pour conduire une armée puissante dans le pays de Juda ; et ordonne-lui d'attaquer le pays de Juda avec l'épée, de les déraciner, de démolir leurs habitations et d'en détruire toute trace.

\par 17 Alors Antiochus partit pour le pays de Perse.

\par 18 Mais Lysias prépara trois généraux vaillants et vaillants, habiles à la guerre ; dont l'un a été nommé. Ptolémée, un deuxième Nicanor et le troisième Gorgias.

\par 19 Et il envoya avec eux quarante mille soldats d'élite et sept mille cavaliers. Il leur chargea également d'amener avec eux une armée de Syriens et de Philistins ; et leur ordonna d'extirper entièrement les Juifs.

\par 20 Et ils marchèrent, emportant avec eux une multitude de marchands, afin de leur vendre les captifs qu'ils allaient prendre parmi les Juifs.

\par 21 Mais la nouvelle parvint à Judas, fils de Mattathias ; et il se rendit à la maison du grand et bon Dieu ;

\par 22 et il rassembla ses hommes, et leur enjoignit un jeûne, des supplications et des prières au Dieu grand et bon ; et leur demanda de le supplier pour la victoire contre leurs ennemis ; quelle chose ils ont fait.

\par 23 Après cela, Judas rassembla ses hommes et établit un chef pour chaque millier, et de même pour chaque cent, et pour chaque cinquante, et pour chaque dix.

\par 24 Puis il ordonna qu'on proclame à la trompette dans toute son armée que quiconque aurait peur et que quiconque Dieu ordonnerait d'être renvoyé de l'armée devrait rentrer chez lui.

\par 25 Et un grand nombre revint ; et il restait avec eux sept mille hommes vaillants et vaillants, habiles à la guerre et habitués à la guerre ; et aucun d’eux n’avait jamais fui ; et ils marchaient contre leurs ennemis.

\par 26 Mais quand ils furent approchés d'eux, Judas pria son Seigneur, le suppliant de détourner de lui la méchanceté de son ennemi ; et qu'il l'assisterait et le rendrait victorieux.

\par 27 Alors il ordonna aux prêtres de sonner des trompettes, ce qu'ils firent. Et tous ses hommes invoquèrent Dieu et se précipitèrent sur l'armée de Nicanor.

\par 28 Et Dieu leur donna la victoire sur eux, et ils le mirent en fuite, lui et ses hommes, tuant neuf mille hommes, et les autres furent dispersés.

\par 29 Et Judas et sa troupe retournèrent au camp de Nicanor et en firent du butin ; il pilla une grande partie des biens des marchands et les envoya pour être partagés entre les malades.

\par 30 Cette bataille eut lieu le sixième jour de la semaine ; c'est pourquoi Judas et ses hommes restèrent au même endroit jusqu'à ce que le jour du sabbat soit passé.

\par 31 Puis ils marchèrent contre Ptolémée et Gorgias, qu'ils trouvèrent et battirent, et remportèrent la victoire sur eux, tuant vingt mille de leurs soldats.

\par 32 Et Ptolémée et Gorgias s'enfuirent ; que Judas et sa compagnie poursuivaient ; mais il ne put les rattraper, car ils s'enfermèrent dans une ville aux deux idoles et s'y fortifièrent avec le reste de leur armée.

\par 33 Et Judas attaqua Félix ; et il fut mis en fuite devant lui. Et Judas le poursuivit. Celui-ci, arrivant dans une maison voisine, y entra et ferma les portes, car c'était une maison forte.

\par 34 Et Judas commanda, et il y mit le feu ; et la maison fut incendiée, et Félix y fut brûlé. Judas se vengea donc de Hléazar et des autres que Feelix avait mis à mort.

\par 35 Ensuite, le peuple revint vers les tués et prit leur butin et leurs armures ; mais ils envoyèrent la meilleure des proies en Terre Sainte.

\par 36 Mais Nicanor partit sous un déguisement inconnu, et revint vers Lysias, et lui raconta tout ce qui lui était arrivé, lui et sa compagnie.

\chapter{8}

\par \textit{Le récit du retour d'Antiochus et de son entrée dans le pays de Juda, et de la maladie qui lui tomba dessus et dont il mourut au cours de son voyage.}

\par 1 Mais Antiochus revint du pays de Perse en fuite, avec son armée dissoute.

\par 2 Et quand il eut appris ce qui était arrivé à son armée que Lysias avait envoyée et à tous ses hommes, il sortit avec une grande armée, marchant vers le pays de Juda.

\par 3 Or, alors qu'il était arrivé au milieu de son voyage, Dieu frappa ses troupes avec les armes les plus puissantes :

\par 4 mais cela ne pouvait pas l'empêcher de son voyage ; mais il persista, prononçant toutes sortes d'insolences contre Dieu, et disant que personne ne pouvait le détourner, ni l'empêcher de poursuivre ses desseins déterminés.

\par 5 C'est pourquoi le Dieu grand et bon le frappa aussi d'ulcères qui attaquèrent tout son corps ;

\par 6 mais il était plus rempli de colère et enflammé d'un désir ardent d'obtenir ce qu'il avait décidé et de mettre sa résolution à exécution.

\par 7 Or, il y avait dans son armée un très grand nombre d'éléphants. Il arriva que l'un d'eux s'enfuit et poussa un beuglement ; sur quoi les chevaux qui tiraient le lit sur lequel Antiochus était couché s'enfuirent et le jetèrent dehors.

\par 8 Et comme il était gros et corpulent, ses membres étaient meurtris et certaines de ses articulations étaient disloquées.

\par 9 Et la mauvaise odeur de ses ulcères, qui dégageait déjà une odeur fétide, était tellement accrue, que ni lui-même ne pouvait la supporter plus longtemps, ni ceux qui l'approchaient.

\par 10 Et quand il tombait, ses serviteurs le relevaient et le portaient sur leurs épaules ; mais comme l'odeur nauséabonde devenait plus forte, ils le jetèrent à terre et s'en allèrent au loin.

\par 11 C'est pourquoi, apercevant les maux qui l'entouraient, il crut avec certitude que tout ce châtiment lui était venu du grand et bon Dieu ; en raison de l'injure et de la tyrannie dont il avait usé envers les Hébreux, et de l'injuste effusion de leur sang.

\par 12 Dans la crainte, il se tourna vers Dieu et, confessant ses péchés, dit : « Ô Dieu, en vérité je mérite les choses que tu m’as envoyées ; et tu es vraiment juste dans tes jugements ;

\par 13 Tu humilies celui qui est élevé, et tu abaisses celui qui est enflé ; mais à toi appartient la grandeur, la magnificence, la majesté et la prouesse.

\par 14 En vérité, je l'avoue, j'ai opprimé le peuple, et j'ai agi et décrété tyranniquement contre lui.

\par 15 Pardonne, je te prie, ô Dieu, cette erreur de ma part ; et efface mon péché, et accorde-moi ma santé ; et mon soin sera de remplir le trésor de ta maison d'or et d'argent.

\par 16 et pour parsemer le sol de la maison de ton sanctuaire de vêtements de pourpre ; et être circoncis; et proclamer dans tout mon royaume que tu es le seul vrai Dieu, sans aucun associé, et qu'il n'y a pas d'autre Dieu que toi.

\par 17 Mais Dieu n'exauça pas ses prières et n'accepta pas sa supplication ; mais ses angoisses augmentèrent tellement sur lui qu'il vidait ses entrailles ; et ses ulcères augmentèrent au point que sa chair tomba de son corps.

\par 18 Puis il mourut et fut enterré à sa place. Et à sa place régna son fils, nommé Eupator.

\chapter{9}

\par \textit{L'histoire des huit jours de dédicace}

\par 1 Lorsque Judas eut mis en fuite Ptolémée, Nicanor et Gorgias, et qu'il eut tué leurs hommes ; lui-même et ses troupes retournèrent dans le pays » de la sainte maison.

\par 2 Et il ordonna de détruire tous les autels qu'Antiochus avait ordonné de construire :

\par 3 et il enleva toutes les idoles qui étaient dans le sanctuaire ; et ils bâtirent un nouvel autel, et il ordonna d'offrir des sacrifices dessus.

\par 4 Ils prièrent aussi le Dieu grand et bon, afin qu'il fasse jaillir le feu sacré qui resterait sur l'autel :

\par 5 et du feu sortit de quelques pierres de l'autel, et brûla le bois et les sacrifices ; et de là le feu continua sur l'autel jusqu'au troisième enlèvement en captivité.

\par 6 Et puis ils célébrèrent la fête du nouvel autel pendant huit jours, à partir du vingt-cinquième jour du mois de Casleu.

\par 7 Et puis ils déposèrent du pain sur la table de la maison de Dieu, et allumèrent les lampes du chandelier.

\par 8 Et chacun de ces huit jours, ils se rassemblaient pour la prière et la louange ; et en outre, ils en faisaient une ordonnance pour chaque année à venir.

\chapter{10}

\par \textit{L'histoire des batailles de Judas avec Gorgias et Ptolémée}

\par 1 Or, après les jours de consécration, Judas marcha vers le pays des Idumzans, vers la montagne de Sarah, car Gorgias y demeurait.

\par 2 Et Gorgias sortit contre lui avec une grande armée, et il y eut entre eux de rudes combats ; Et là tombèrent vingt mille hommes de Gorgias.

\par 3 Et Gorgias s'enfuit vers Ptolémée dans le pays de l'Occident (car Antiochus l'avait nommé gouverneur de ce pays, et c'est là qu'il séjournait), et lui raconta ce qui lui était arrivé.

\par 4 Sur quoi Ptolémée sortit avec une armée composée de cent vingt mille hommes de Macédoine et de l'Orient.

\par 5 Et il continua jusqu'à ce qu'il atteigne le pays de Giares, c'est-à-dire Galaad, et les régions adjacentes ; et il tua un grand nombre de Juifs.

\par 6 Ils écrivirent donc à Judas, lui racontant ce qui leur était arrivé, le priant de venir vaincre Ptolémée et de le chasser d'eux.

\par 7 Et leur lettre lui parvint en même temps qu'une lettre lui parvenait également des habitants de la montagne de Galilée, l'informant de la façon dont les Macédoniens ; ceux qui étaient à Tyr et à Sidon s'étaient alors unis contre eux et les avaient attaqués, en tuant plusieurs.

\par 8 Or, lorsque Judas eut lu les deux lettres, il rassembla ses hommes, leur montra le contenu des lettres, et fixa un jeûne et une supplication.

\par 9 Après cela, il ordonna à son frère Siméon de prendre avec lui trois mille hommes juifs et de marcher en toute hâte vers la montagne de Galilée, « et de réprimer les Macédoniens qui s'y trouvaient.

\par 10 Et Siméon s'en alla. Mais Judas s'empressa de rencontrer Ptolémée.

\par 11 Et Siméon attaqua les Macédoniens à l'improviste, et tua d'eux huit mille hommes, et donna du repos aux Galilzans.

\par 12 Mais Judas marcha jusqu'à ce qu'il atteigne Gorgias et Ptolémée ; Les pressant et les assiégeant ; les deux armées se rencontrèrent, et des combats très violents eurent lieu entre elles.

\par 13 Car Ptolémée était à la tête d'un corps d'hommes nombreux, vaillants et vaillants. Mais Judas était accompagné d'un tout petit groupe :

\par 14 Cependant, comme le peuple qui était avec lui était composé des troupes les plus vaillantes et les plus fortes, il résista fermement, et la bataille entre eux dura longtemps et devint très douloureuse.

\par 15 C'est pourquoi Judas invoqua le Dieu grand et bon et invoqua son aide.

\par 16 Et il raconta qu'il avait vu cinq jeunes cavaliers, dont trois combattaient contre l'armée de Ptolémée, et deux se tenaient près de lui.

\par 17 Et lorsqu'il les regardait attentivement, ils lui semblaient être des anges de Dieu.

\par 18 C'est pourquoi son cœur fut consolé, ainsi que le cœur de ses compagnons ; et faisant de fréquents assauts contre l'ennemi, ils le mirent en fuite et en tuèrent une grande multitude.

\par 19 Et le nombre de ceux qui furent tués dans l'armée de Ptolémée, depuis le début de cette bataille jusqu'à la fin, fut de vingt mille cinq cents.

\par 20 Après ces choses, Ptolémée et ses hommes s'enfuirent vers le bord de la mer ; tandis que Judas les poursuivait et en tuait autant qu'il en attrapait.

\par 21 Mais Ptolémée s'enfuit à Gaza et y resta ; et les hommes de Chalisam vinrent vers lui.

\par 22 Et Judas marcha contre eux ; et quand il les trouva, il les battit : et les hommes de Ptolémée furent dispersés, mais lui-même s'enfuit à Gaza et s'y fortifia.

\par 23 Et les hommes de Judas poursuivirent le corps volant et en tuèrent un grand nombre. Et Judas et les hommes qui étaient avec lui marchèrent droit vers Gaza, et il dressa son camp et l'assiégea.

\par 24 Et les hommes de Judas revinrent vers lui ; Et ceux qui restaient des forces de Ptolémée montèrent sur la fortification et injurièrent Judas avec beaucoup d'insultes.

\par 25 Et le combat entre eux et les troupes de Judas dura cinq jours. Mais le cinquième jour venu, le peuple continua à jeter des injures sur Judas et à insulter sa religion :

\par 26 Sur quoi vingt hommes de Judas se mirent en colère ; qui, prenant des boucliers dans leur main gauche et des épées dans la droite, et ayant avec eux un homme portant une échelle qu'ils avaient faite, marchèrent jusqu'à ce qu'ils arrivèrent au mur.

\par 27 et dix-huit d'entre eux se levèrent et jetèrent des traits sur ceux qui étaient sur le mur ; et deux, se précipitant vers le mur, soulevèrent l'échelle et montèrent par là.

\par 28 Mais certains de ceux qui étaient là, s'apercevant qu'ils étaient montés et que leurs compagnons les avaient suivis, et qu'ils étaient aussi descendus du mur dans la ville, descendirent du mur après eux. Les hommes de Judas les battirent, les tuant. un grand nombre de leurs ennemis.

\par 29 Mais l'armée de Judas se pressa jusqu'à la porte de la ville ; Et les vingt commencèrent à courir vers la porte pour l'ouvrir ; mais ils en furent chassés avec la plus grande violence ; c'est pourquoi ils crièrent à grands cris.

\par 30 Judas et ses hommes savaient donc qu'ils s'étaient approchés de la porte ; et la bataille devint acharnée, tant à l'extérieur qu'à l'intérieur de la porte.

\par 31 Et Judas et ses hommes attaquèrent la porte par le feu, et elle tomba ; Et le peuple périt, et les hommes qui avaient injurié Judas furent pris, et il ordonna de les faire sortir et de les brûler.

\par 32 Et il ordonna que la ville soit entièrement frappée par l'épée ; et le massacre y dura deux jours, puis il fut consumé par le feu.

\par 33 Mais Ptolémée s'enfuit ; et aucune nouvelle de lui n'a été entendue à ce moment-là ; parce qu'il avait changé de vêtements et s'était caché dans l'une des fosses, et qu'on n'avait aucun compte de lui.

\par 34 Mais ses deux frères furent pris et amenés à Judas ; et il ordonna de les décapiter.

\par 35 Après cela, il entra dans le pays du sanctuaire, avec un butin en abondance ; et lui et sa compagnie y firent des prières, rendant grâce à Dieu pour les bienfaits qu'ils avaient reçus.


\chapter{11}

\par \textit{La relation du batile entre Judas et Lysias, général d'Eupator, après la mort du roi Antiochus}

\par 1 Le nom d'Antiochus, dont nous avons parlé plus haut, était Épiphane ; mais le nom de son fils qui régna après lui était Eupator, qui s'appelait aussi Antiochus.

\par 2 Et lorsque les batailles de Judas avec ces généraux eurent eu lieu, ils » écrivirent à ce sujet à Eupator ; qui envoya avec Lysias, fils de son cousin, une grande armée, composée de quatre-vingt mille cavaliers et quatre-vingts éléphants.

\par 3 Ils arrivèrent à une ville appelée Bethner, dressèrent leur camp autour d'elle et l'assiégèrent, parce que c'était une grande ville et qu'il y avait beaucoup de monde.

\par 4 Et Lysias leva des machines de guerre autour d'elle, et commença à assiéger les habitants :

\par 5 Ce qui fut annoncé à Judas, il sortit lui-même et sa compagnie vers des montagnes fortifiées ;

\par 6 et ils demeurèrent là ; de peur que s'ils restaient dans une ville, Lysias ne viendrait l'assiéger et ne les maîtriserait.

\par 7 Judas rassembla donc sa troupe et résolut de marcher avec eux vers le camp de Lysias, après qu'ils seraient allés à la maison de Dieu et y auraient offert des sacrifices ;

\par 8 suppliant le Dieu grand et bon de détourner d'eux la méchanceté de leurs ennemis et de leur accorder la victoire sur eux : ce qu'ils firent.

\par 9 Après cela, ils marchèrent du territoire de la maison sainte vers Bethner. Car ils avaient prévu de tomber sur l’armée soudainement et de la vaincre sans combat.

\par 10 On raconte qu'entre le ciel et la terre apparut à Judas un personnage monté sur un cheval de feu et tenant à la main une grande lance avec laquelle il frappa l'armée des païens.

\par 11 De sorte que ce qu'ils avaient vu leur donna du courage et de l'esprit supplémentaires. Ils se hâtèrent, chargeèrent l'armée et tuèrent un grand nombre de ses hommes.

\par 12 C'est pourquoi l'armée ennemie fut troublée et plongée dans la plus grande confusion, et toute elle se mit en fuite dans la confusion.

\par 13 Et l'épée de Judas et. sa compagnie les pressait fortement ; et il en tua onze mille fantassins et seize cents cavaliers.

\par 14 Liysias fut également poursuivi avec sa compagnie dans un lieu éloigné, où il resta en sécurité.

\par 15 Et il envoya à Judas, lui demandant de se soumettre au roi, en conservant sa propre religion et celle de son peuple :

\par 16 à qui Judas consentit dans cette affaire, jusqu'à ce qu'une parole puisse être écrite au roi, et qu'une réponse de son accord puisse être reçue.

\par 17 Et Judas écrivit à propos de cette affaire : Lysias écrivit aussi au roi, l'informant de ce qui était arrivé et des preuves qu'il avait eues de la force et de la bravoure de la nation juive ;

\par 18 et que la continuation des guerres avec eux exterminerait ses hommes, comme ceux-ci avaient été exterminés : il lui dit aussi leur accord, et le sien attendant de recevoir une lettre pour dire ce qu'il devait faire.

\par 19 A quoi le roi répondit qu'il lui semblait juste de faire la paix avec la nation des Juifs, en ôtant cette pierre d'achoppement concernant l'exercice de leur religion : car cela même les avait incités aux révoltes et à la révolte. les attaques lancées contre ses prédécesseurs.

\par 20 Il lui ordonna aussi de conclure avec eux un traité de paix et d'obéissance ; afin qu'aucun obstacle ne soit mis sur leur chemin en matière de religion.

\par 21 Il écrivit aussi à Judas et à tous les Juifs qui étaient dans le pays de Juda, à cet effet ; et cette paix dura entre eux pendant un certain temps.


\chapter{12}

\par \textit{Récit du début de la puissance des Romains et de l'agrandissement de leur empire.}

\par 1 A cette même époque dont nous avons parlé, les affaires des Romains commencèrent à être exaltées, afin que le Dieu grand et bon accomplisse ce qu'avait prédit le prophète Daniel (à qui soit la paix) concernant le quatrième. Empire.

\par 2 Il y avait aussi à cette époque un certain roi le plus généreux d'Afrique, dont le nom était Annibal. Et le siège royal de son empire était Carthage. Il résolut de prendre possession du royaume des Romains :

\par 3 C'est pourquoi ils se sont unis pour s'opposer à lui ; et les guerres se multiplièrent entre eux, de sorte qu'ils en combattirent dix-huit ? des combats en l’espace de dix ans ; et ils ne purent le chasser de leur pays, à cause de son armée et de son peuple innombrables.

\par 4 Ils décidèrent donc de lever une grande force choisie parmi leurs troupes et armées les plus vaillantes, et d'attaquer Annibal en guerre, et de persévérer jusqu'à ce qu'ils détournent d'eux ses forces.

\par 5 Ce qu'ils firent en vérité : et ils placèrent à la tête de leurs armées deux hommes des plus illustres ; le nom de l'un était Aimilius, et celui de l'autre Varro.

\par 6 Qui rencontrant Annibal s'engagea avec lui ; et quatre-vingt-dix mille hommes de leur armée furent tués ; et quarante mille hommes de l'armée d'Annibal furent tués.

\par 7 Mais Varron s'enfuit dans une certaine ville très grande et forte appelée Vénusia : Annibal ne le poursuivit pas ; mais il marcha vers Rome pour la prendre et y rester.

\par 8 Il resta donc huit jours devant elle et commença à bâtir des maisons en face d'elle ;

\par 9 que, lorsque les citoyens virent, ils délibérèrent sur la conclusion d'une paix et d'un traité avec lui, et sur la reddition du pays.

\par 10 Mais il y avait parmi eux un jeune homme nommé Scipion (car les Romains à cette époque n'avaient pas de roi, et toute l'administration de leurs affaires était confiée à trois cent vingt hommes, présidés par un personnage qui était appelé senior ou aîné.)

\par 11 Scipion vient donc vers eux et les persuade de ne pas se fier à Annibal ni de se soumettre à lui. A quoi ils répondirent qu'ils ne lui faisaient pas confiance, mais qu'ils étaient incapables de lui résister.

\par 12 À qui il dit : le pays d'Afrique est totalement dépourvu de soldats, parce qu'ils sont tous ici avec Annibal : donnez-moi donc une troupe d'hommes choisis, afin que j'aille en Afrique :

\par 13 et j'y accomplirai de tels exploits, que lorsque la nouvelle lui parviendra, peut-être qu'il vous quittera, et vous serez libéré de lui et serez en paix ; et après avoir récupéré et renforcé vos ressources, si s'il se prépare à revenir, vous pourrez vous opposer à lui.

\par 14 Et le conseil de Scipion leur parut juste ; et ils lui confièrent trente mille de leurs hommes les plus vaillants.

\par 15 Et il se rendit en Afrique. Asdrubal, frère d'Annibal, vint à sa rencontre et combattit contre lui. Scipion le battit, lui coupa la tête, la prit avec le reste de la proie et revint à Rome.

\par 16 Et montant sur le rempart, il appela Annibal et dit : Comment pourrez-vous vaincre notre pays, si vous ne pouvez pas m'expulser de votre propre pays, où je suis allé : j'ai tu l'as détruit, tu as tué ton frère et tu lui as emporté sa tête.

\par 17 Puis il lui lança la tête. Celui-ci, amené à Annibal et reconnu par lui, grandit en fureur et en colère contre le peuple, et jura qu'il ne partirait pas avant d'avoir pris Rome.

\par 18 Mais les citoyens, pour le soustraire à eux et le tenir en échec, décidèrent de renvoyer Scipion assiéger et attaquer Carthage.

\par 19 Et Scipion revint avec son armée en Afrique ; et ils établirent leur camp autour de Carthage, et l'assiégèrent d'un siège très actif.

\par 20 C'est pourquoi les habitants écrivirent à Annibal, disant : Vous convoitez un pays étranger, dont vous ne savez pas si vous pourrez ou non conquérir ; mais il est venu dans votre pays quelqu'un qui cherche à en prendre possession. .

\par 21 C'est pourquoi, si tu tardes à venir, nous lui céderons le pays, et nous abandonnerons ta famille, tous tes biens et tes trésors ; afin que nous et nos biens restions indemnes.

\par 22 Or, lorsque cette lettre lui fut apportée, il quitta Rome ; et se hâta jusqu'à ce qu'il vienne en Afrique :

\par 23 Scipion s'avança et le rencontra, et livra trois fois contre lui un combat très acharné, et cinquante mille de ses hommes furent tués.

\par 24 Mais Annibal, mis en fuite, se retira dans le pays d'Égypte ; que Scipion poursuivit, le fit prisonnier et retourna en Afrique.

\par 25 Et quand il était là, Annibal dédaignait d'être vu des Africains ; c'est pourquoi il prit du poison et mourut.

\par 26 Et Scipion gagna le pays d'Afrique, et s'empara de tous les biens, serviteurs et trésors d'Annibal.

\par 27 C'est ainsi que la renommée des Romains s'est amplifiée, et leur puissance a commencé à s'accroître à partir de ce moment-là.

\chapter{13}

\par \textit{Récit de la lettre des Romains à Judas et du traité qui eut lieu entre eux.}

\par 1 « Depuis l'ancien et trois cent vingt gouverneurs, jusqu'à Judas, général de l'armée, et aux Juifs.

\par 2 La santé soit avec vous. Nous avons déjà entendu parler de vos victoires, de votre courage et de votre endurance à la guerre ; dont nous nous réjouissons. Nous avons également compris que vous aviez conclu un accord avec Antiochus.

\par 3 Nous vous écrivons afin que vous soyez amis avec nous, et non avec les Grecs qui vous ont fait du mal : en outre, nous avons l'intention d'aller à Antioche et de faire la guerre à ses habitants :

\par 4 C'est pourquoi hâte-toi de nous faire savoir avec qui tu es en inimitié et avec qui tu as une ligue d'amitié ; afin que nous puissions agir en conséquence.

\par 5 LA COPIE DU TRAITÉ. « Ceci est le traité conclu par l'ancien et trois cent vingt gouverneurs avec Judas, général de l'armée, et les Juifs ; afin qu'ils soient joints aux Romains, et que les Romains et les Juifs puissent être à jamais d'un même avis dans les guerres et les victoires.

\par 6 Or, si la guerre devait éclater contre les Romains, Judas et son peuple les secourraient, sans apporter d'aide aux ennemis des Romains, ni par des provisions ni par aucune sorte d'armes.

\par 7 Et lorsque la guerre éclatera contre les Juifs, les Romains les aideront de tout leur pouvoir, sans apporter aucune aide à leurs ennemis.

\par 8 Et comme les Juifs sont liés aux Romains, de même les Romains le sont aux Juifs, sans aucune augmentation ni diminution.

\par 9 Et Judas et son peuple acceptèrent cela ; Le traité dura longtemps entre eux et les Romains.

\chapter{14}

\par \textit{Un récit du batile qui eut lieu entre Judas, Ptolémée et Gorgias.}

\par 1 Après cela, Ptolémée rassembla cent vingt mille hommes et mille cavaliers, et ils se dirigèrent vers Judas. Et Judas le rencontra avec dix mille hommes et le mit en déroute, et beaucoup de hommes de Ptolémée furent tués.

\par 2 Et il supplia Judas, et le supplia humblement de le laisser échapper avec sa vie ; et jura qu'il ne lui ferait plus jamais la guerre et qu'il ferait preuve de bonté envers les Juifs qui étaient dans tous ses pays.

\par 3 Et Judas eut compassion de lui, et le laissa partir ; et Ptolémée respecta son serment.

\par 4 Mais Gorgias ayant rassemblé trois mille hommes de la montagne de Sarah (c'est-à-dire de l'Idumée) et quatre cents cavaliers, rencontra Judas et tua le chef de son armée et certains de ses hommes.

\par 5 Alors Judas et ses hommes s'avancèrent vers eux ; Gorgias fut mis en fuite, et la plus grande partie de son armée fut tuée ou s'enfuit. On le chercha, et on n'entendit aucune nouvelle de lui. mais on rapporte qu'il tomba dans la bataille.

\chapter{15}

\par \textit{Récit de la dissolution du traité qu'Antiochus avait conclu avec Judas, et de sa marche (avec Lysias, fils de son cousin) avec une grande armée, et de ses guerres.}

\par 1 Mais quand on apprit à Antiochus Eupator que les affaires de Judas s'étaient renforcées et quelles victoires il avait remportées, il fut très en colère ;

\par 2 et rompit le traité qu'il avait fait avec Judas, et rassembla une grande armée, dans laquelle se trouvaient vingt-deux éléphants :

\par 3 Et il marcha avec Lysias, fils de son cousin, dans le pays de Juda, se dirigeant vers la ville de Beth-Ner ?, devant laquelle il dressa son camp et l'assiégea.

\par 4 Or, lorsque cela fut rapporté à Judas, lui et tous les anciens des enfants d'Israël se réunirent et prièrent le Dieu grand et bon, offrant de nombreux sacrifices ;

\par 5 Ce qui étant terminé, Judas partit avec les chefs de ses forces, et entra dans le camp de nuit, et fit une attaque soudaine contre lui, et tua des ennemis quatre mille hommes et un des éléphants ; et il revint à son propre camp jusqu'à l'aube du jour devrait commencer à se lever.

\par 6 Alors chaque armée se retira, et la bataille devint féroce entre elles.

\par 7 Et Judas aperçut un des éléphants avec des harnais d'or, et il crut que le roi était assis sur lui. Il appela donc ses hommes et leur dit : Lequel d'entre vous sortira et tuera cet éléphant ?

\par 8 Et un jeune homme, l'un de ses serviteurs, appelé Éléazar, sortit et se précipita sur la ligne ennemie, tuant à droite et à gauche, de sorte que les hommes se détournèrent hors de sa vue ;

\par 9 et il avança jusqu'à ce qu'il atteigne même l'éléphant ; et se glissant sous lui, il lui ouvrit le ventre ; et l'éléphant tomba sur lui, et il mourut. Le roi s'en apercevant, ordonna de sonner la retraite ; et c'était fait.

\par 10 Et le nombre des hommes de rang supérieur tués ce jour-là dans la bataille était de huit cents hommes, sans compter ceux des hommes ordinaires qui furent tués et ceux qui avaient été tués pendant la nuit.

\par 11 Alors on rapporta au roi qu'un certain homme de ses amis, nommé Philippe, s'était révolté contre lui, et que Démétrius, fils de Séleucus, était sorti de Rome avec une grande armée de Romains, dans l'intention de retirer le royaume de Rome. sa main.

\par 12 Sur quoi, très effrayé, il envoya à Judas pour lui demander de faire la paix entre eux : ce à quoi Judas consentit ; Antiochus et Lysias, fils de son cousin, lui jurèrent de ne plus lui faire la guerre.

\par 13 Et le roi présenta une grosse somme d'argent et la donna à Judas en cadeau à la maison de Dieu.

\par 14 Le roi ordonna également d'arrêter Ménélas, l'un des trois méchants hommes qui avaient semé le mal chez les Juifs du temps d'Antiochus, son père ; et il ordonna qu'on le transporte jusqu'à une haute tour, et qu'on le jette de là tête baissée ; ce qui a été fait.

\par 15 Car le roi voulait ainsi satisfaire les Juifs, puisque cet homme était l'un de leurs principaux ennemis et qu'il en avait tué un grand nombre.

\chapter{16}

\par \textit{L'histoire de l'arrivée à Antioche de Démétrius, fils de Séleucus, et de sa défaite contre Eupator.}

\par 1 Après cela, le roi Eupator entra en Macédoine, puis revint à Antioche.

\par 2 Lequel Démétrius attaqua avec une armée de Romains, le battit et le tua avec Lysias, le fils de son cousin ; et il régna à Antioche.

\par 3 Mais c'est vers lui qu'allait Alcimus, le chef de ces trois-là ? des hommes méchants ; qui, venant en sa présence, se prosterna devant lui, pleura avec véhémence et dit :

\par 4 « Ô roi, Judas et sa compagnie ont tué un grand nombre d'entre nous ; car, ayant abandonné leur religion, nous avons embrassé la religion du roi. C'est pourquoi, ô roi, aide-nous contre eux et venge-nous d'eux.

\par 5 Alors il fit venir les Juifs vers lui et l'irrita ; leur suggérant des choses qui pourraient provoquer Démétrius et l'irriter pour équiper une armée pour vaincre Judas.

\par 6 Le roi, attentif, envoya un général nommé Nicanor, avec une grande armée et une abondante provision d'armes de guerre.

\par 7 Et lorsque Nicanor fut arrivé en Terre Sainte, il envoya des messagers à Judas pour qu'ils viennent vers lui ; et n'a pas révélé qu'il était venu pour conquérir la nation,

\par 8 mais il déclara qu'il était venu seulement à cause de la paix qui avait été faite entre lui et la nation, et qu'eux aussi étaient soumis à l'obéissance aux Romains.

\par 9 Et Judas sortit vers lui avec un certain nombre de ses hommes, qui étaient dotés de force et de courage ; et il leur ordonna de ne pas s'éloigner de lui, de peur que Démétrius ne lui tende un piège.

\par 10 Lorsqu'il rencontra Démétrius, il le salua ; Et, ayant placé un siège pour chacun d'eux, ils s'assirent, et Démétrius conversa avec lui à sa guise. Ensuite, chacun d'eux entra dans une tente que les troupes lui avaient dressée.

\par 11 Et Nicanor et Judas partirent pour la Ville Sainte, et y demeurèrent ensemble ; et une solide amitié grandit entre eux.

\par 12 Celui-ci, ayant été informé d'Alcimus, alla trouver Démétrius et l'irrita contre Judas, et le persuada d'écrire et d'ordonner à Nicanor de lui envoyer Judas enchaîné.

\par 13 Mais la nouvelle de cela parvint à Judas, et il sortit de nuit de la ville, et partit pour Sébaste, et envoya ses compagnons pour venir vers lui.

\par 14 Et quand ils furent arrivés, il sonna de la trompette et leur ordonna de se préparer à attaquer Nicanor.

\par 15 Mais Nicanor cherchait Judas avec beaucoup de diligence, et ne pouvait rien savoir de lui.

\par 16 C'est pourquoi il se rendit à la maison de Dieu, demandant aux prêtres de le lui livrer, afin qu'il l'envoyât lié et enchaîné au roi ; mais ils jurèrent qu'il n'était pas entré dans la maison de Dieu.

\par 17 Sur quoi il les insulta ainsi que la maison de Dieu, et parla insolemment du temple, et menaça de le démolir jusqu'aux fondations ; et il partit en colère. Il prit également soin de fouiller toutes les maisons de la Ville Sainte.

\par 18 De même, il envoya ses hommes dans la maison d'un certain homme excellent, qui avait été capturé du temps d'Antiochus et soumis à de terribles tortures ; mais après la mort d'Antiochus, les Juifs augmentèrent son autorité et l'honorèrent grandement.

\par 19 Et lorsque les messagers de Nicanor vinrent vers lui, il craignit de subir le même traitement que celui qu'il avait reçu d'Antiochus ; c'est pourquoi il s'imposa les mains.

\par 20 Quand cela fut raconté à Judas, il fut très désolé et très affligé ; et il envoya dire à Nicanor : « Ne me cherchez pas dans la ville, car je n’y suis pas : sortez donc vers moi, afin que nous nous rencontrions soit dans les plaines, soit dans les montagnes, selon votre choix. »

\par 21 Et Nicanor sortit vers lui, et Judas le rencontra avec ces paroles : « Ô Dieu, c'est Toi qui as exterminé l'armée du roi Sennachérib ; et il était en effet plus grand que cet homme, en renommée, en empire et en multitude de son armée :

\par 22 Et tu as délivré de lui Ezéchias, roi de Juda, alors qu'il s'était confié en toi et t'avait prié : délivre-nous, je te prie, ô Dieu, de sa méchanceté, et rends-nous victorieux sur lui.

\par 23 Alors il se prépara au combat et s'avança vers Nicanor, en disant : Prends soin de toi, c'est à toi que je viens.

\par 24 Et Nicanor tourna le dos et s'enfuit ; et Judas, le poursuivant, le frappa aux épaules, qu'il fendit ; et ses hommes furent mis en fuite.

\par 25 Et il en tomba ce jour-là trente mille ; et les habitants des villes sortirent et les tuèrent, de sorte qu'ils n'en laissèrent aucun.

\par 26 Et ils décrétèrent que ce jour serait chaque année un jour de remerciement au Dieu grand et bon, et un jour de joie, de festin et de boisson. [Le deuxième livre de la traduction des Hébreux est ainsi terminé.]

\chapter{17}

\par \textit{Récit de la mort de Judas}

\par 1 Mais à peu près à la même époque de l'année, Bacchidès sortit avec trente mille des plus vaillants Macédoniens ;

\par 2 Et il rencontra Judas, sans qu'il en soit informé, alors qu'il était dans une certaine ville appelée Lalis, avec trois mille hommes.

\par 3 C'est pourquoi la plupart de ceux qui étaient avec lui s'enfuirent ; et il resta avec lui huit cents hommes, ainsi que ses frères Siméon et Jonathan.

\par 4 Mais ceux qui restèrent avec Judas étaient les plus forts et les plus courageux, et qui avaient déjà enduré beaucoup de choses dans les nombreuses batailles qu'il avait livrées.

\par 5 Et Judas et sa troupe sortirent à la rencontre de Bacchidès et de son armée.

\par 6 Et Bacchidès divisa son armée, plaçant quinze mille à la droite de Judas et de sa compagnie, et quinze mille à leur gauche.

\par 7 Alors chaque partie cria contre Judas et sa troupe. Celui-ci, les regardant attentivement, s'aperçut que les troupes ennemies les plus fortes et les plus vaillantes étaient à droite, et s'aperçut que Bacchidès lui-même était parmi elles.

\par 8 Judas divisa également son groupe, prit avec lui les plus vaillants d'entre eux et donna le reste à ses frères. Puis il chargea ceux de droite et tua avec sa compagnie environ deux mille hommes.

\par 9 Alors, apercevant Bacchidès, il tourna ses yeux et ses pas vers lui, et tua tous les hommes les plus courageux qui l'entouraient.

\par 10 Et lui, avec sa troupe, soutint les multitudes qui se pressaient contre lui, faisant tomber à terre la plupart d'entre eux, et il s'approcha de Bacchidès.

\par 11 Celui-ci, lorsque Bacchidès vit venir vers lui comme un lion, brandissant dans sa main une grande épée tachée de sang, fut extrêmement effrayé par lui, et trembla et s'enfuit hors de sa vue.

\par 12 Et Judas et sa troupe le poursuivirent, et ils tuèrent son peuple par l'épée, de sorte qu'ils mirent à mort la plupart de ces quinze mille; et Bacchidès s'enfuit jusqu'à Asdod.

\par 13 Et les quinze mille qui étaient à la gauche de Judas le suivirent et attaquèrent Judas, vers lequel étaient arrivés à ce moment-là ses frères et ceux qui étaient avec eux, très fatigués.

\par 14 Et ces quinze mille se précipitèrent sur eux, et une très grande bataille eut lieu entre eux et Judas ; et il tomba des deux côtés un certain nombre de tués, parmi lesquels se trouvait Judas.

\par 15 Que ses frères portèrent et enterrèrent près du sépulcre de Mattathias, son père, [Dieu leur ait pitié] ; et les enfants d'Israël le pleurèrent pendant de nombreuses années.

\par 16 La durée de son gouvernement fut de sept ans, et Jonathan, son frère, lui succéda dans le gouvernement.

\chapter{18}

\par \textit{L'histoire de Jonathan, fils de Mattathias}

\par 1 Et Jonathan succéda à son frère, et il partit pour le Jourdain avec un petit nombre d'hommes ; Quand Bacchidès en eut connaissance, il marcha vers lui avec une grande armée.

\par 2 Et lorsque Jonathan l'aperçut, ses hommes traversèrent le Jourdain à la nage ; Bacchidès et son armée les suivirent et les encerclèrent.

\par 3 Mais Jonathan se précipita sur Bacchidès ; Et comme les hommes cédaient la place à Jonathan, lui et sa troupe sortirent du milieu d'eux, et s'en allèrent à Beer-Sheva.

\par 4 et son frère Siméon le rejoignirent, et ils demeurèrent là ; Ils réparèrent ce qui était tombé dans les fortifications et s'y fortifièrent.

\par 5 Mais Bacchidès marcha vers eux et les assiégea. Jonathan, son frère et ceux qui étaient avec eux sortirent vers lui de nuit, tuèrent un grand nombre de son armée et brûlèrent les béliers et les machines de l'armée. guerre;

\par 6 et son armée fut dispersée, et Bacchidès s'enfuit dans le désert. Et Jonathan et Siméon, et les hommes qui étaient avec lui, le poursuivirent et le prirent.

\par 7 Celui-ci, voyant Jonathan, comprit que sa mort était proche ; c'est pourquoi il proclama la paix avec Jonathan, et jura qu'il ne lui ferait plus jamais la guerre, et qu'en outre il rendrait tous les captifs qu'il avait. avait pris de l'armée de Judas.

\par 8 Et Jonathan lui tendit la main et le quitta; et après cela il n'y eut plus de guerre entre eux. Et peu de temps après, Jonathan mourut, et son frère Siméon lui succéda.


\chapter{19}

\par \textit{L'histoire de Siméon, fils de Mattathias}

\par 1 Alors Siméon, fils de Mattathias, succéda au gouvernement ; et il rassembla tous ceux qui restaient de l'armée de Judas :

\par 2 et ses affaires prospérèrent, et il soumit tous ceux qui avaient exercé des hostilités contre les Juifs après la mort de son frère Judas ; et il se comporta bien envers son peuple, et les affaires de son pays furent correctement réglées.

\par 3 Pourquoi An tiochus ? L'attaqua, ainsi que Démétrius, fils de Séleucus ; et envoya une grande armée contre lui :

\par 4 À cette rencontre, Siméon et ses deux fils sortirent ; Il divisa son armée en deux parties, dont il garda l'une avec lui, et donna l'autre à ses fils.

\par 5 Alors lui et ceux qui étaient avec lui allèrent à l'armée ; et il envoya ses deux fils et leurs partisans par un autre chemin, et désigna avec eux d'attaquer l'armée à un moment donné.

\par 6 Après cela, il rencontra l'armée d'Antiochus, et l'attaqua, et commença à l'emporter. Et ses deux fils arrivèrent alors que la bataille était maintenant commencée, et le combat devint acharné, et ils contournèrent l'arrière de l'armée. armée.

\par 7 Et l'armée d'Antiochus, placée entre deux armées, fut coupée en morceaux, et aucun d'entre eux ne s'échappa ; et Antiochus ne revint plus combattre avec Siméon.

\par 8 Et la paix et la tranquillité demeurèrent parmi les Juifs pendant toute la vie de Siméon. Et la durée de son gouvernement était de deux ans.

\par 9 Alors Ptolémée, son gendre, se précipita sur lui et le tua lors d'une certaine fête à laquelle il assistait. Et il s'empara de sa femme et de ses deux fils. Et le fils de Siméon, nommé Hyrcan, fut mis à la place de son père.

\par [Ici se termine l'histoire telle qu'elle est donnée dans les deux livres habituellement attachés à nos Bibles.]

\chapter{20}

\par \textit{L'histoire d'Hyrcan, fils de Siméon}

\par 1 Or Siméon, de son vivant, avait nommé Jochanan, son fils, comme capitaine ; Après avoir rassemblé de très nombreuses troupes, il l'envoya vaincre un certain homme qui s'était avancé contre lui et qui s'appelait Hyrcan.

\par 2 Or, c'était un homme d'une grande renommée, puissant en force et d'une ancienne souveraineté.

\par 3 Que Jonathan rencontra et vainquit : c'est pourquoi Siméon nomma son fils Jochanan Hyrcanus ; à cause de la mort d'Hyrcan et de sa victoire sur lui.

\par 4 Mais cet Hyrcan ayant appris que Ptolémée avait tué son père, il eut peur de Ptolémée et s'enfuit à Gaza ; et Ptolémée le poursuivit avec de nombreux partisans.

\par 5 Mais les citoyens de Gaza aidèrent Hyrcan, fermèrent les portes de leur ville et empêchèrent Ptolémée d'atteindre Hyrcan.

\par 6 Et Ptolémée revint et partit pour Dagon, ayant avec lui la mère d'Hyrcan et ses deux frères. château fortement fortifié. Or, Dagon possédait à cette époque un château fortement fortifié.

\par 7 Mais Hyrcan se rendit à la Sainte Maison, offrit des sacrifices et succéda à son père. Il rassembla une grande armée et partit attaquer Ptolémée. C'est pourquoi Ptolémée ferma la porte de Dagon à lui et à ses compagnons, et s'y fortifia.

\par 8 Et Hyrcan l'assiégea, et fit un bélier de fer pour battre la muraille et l'ouvrir. Et la bataille entre eux dura longtemps,

\par 9 et Hyrcan l'emporta contre Ptolémée, et s'approcha du château, et faillit s'en emparer.

\par 10 Ptolémée voyant cela, ordonna de faire sortir la mère d'Hyrcan et ses deux frères sur la muraille, et de les torturer très sévèrement. ce qui leur a été fait.

\par 11 Mais Hyrcan, voyant cela, s'arrêta ; et craignant qu'ils ne soient mis à mort, ils renoncèrent au combat.

\par 12 À qui sa mère appela et dit : « Mon fils, ne te laisse pas émouvoir par l'amour et la piété filiale envers moi et tes frères, de préférence à ton père :

\par 13 et ne soyez pas affaibli, à cause de notre captivité, dans votre désir de le venger ; mais exigez satisfaction pour les droits de votre père et des miens, autant que vous le pouvez.

\par 14 Mais ce que vous craignez pour nous de la part de ce tyran, il nous le fera nécessairement dans tous les cas : c'est pourquoi continuez votre siège sans relâche.

\par 15 Hyrcan, ayant entendu les paroles de sa mère, insista pour le siège. C'est pourquoi Ptolémée augmenta les tourments de sa mère et de ses frères ; et jura qu'il les jetterait tête baissée hors du château, chaque fois qu'Hyrcan s'approcherait du mur.

\par 16 C'est pourquoi Hyrcan craignait qu'il ne soit la cause de leur mort ; et il retourna à son camp, continuant toujours « le siège de Ptolémée ».

\par 17 Or, il arriva que la fête des tabernacles était proche'; C'est pourquoi Hyrcan se rendit dans la ville de la Sainte Maison, afin d'assister à la fête, à la solennité et aux sacrifices.

\par 18 Et lorsque Ptolémée apprit qu'il était parti pour la Ville Sainte et qu'il y était détenu, il s'empara de la mère d'Hyrcan et de ses frères, et les tua ; et il s'enfuit dans un endroit où Hyrcan ne pouvait pas venir.



\chapter{21}

\par \textit{L'histoire de la montée d'Antiochus dans la ville de la Sainte Maison, pour combattre Hyrcan.}

\par 1 Or, quand Antiochus apprit que Siméon était mort, il rassembla une armée et marcha jusqu'à ce qu'il atteigne la ville de la Sainte Maison :

\par 2 et il campa autour d'elle et l'assiégea, avec l'intention de la prendre par la force ; mais il ne le put pas, à cause de la hauteur et de la force des murs, et de la multitude des guerriers qui s'y trouvaient.

\par 3 Mais par la volonté de Dieu, il ne put la gagner ; car il s'était replié du côté nord de la ville, et y avait bâti cent trente tours en face du mur ;

\par 4 et il avait fait monter des hommes sur eux pour combattre ceux qui tenteraient de monter sur les murs de la ville.

\par 5 Il chargea aussi des hommes de creuser la terre en un certain endroit, jusqu'à ce qu'ils arrivent aux fondations du mur ; trouvant qu'elle était en bois, ils la brûlèrent au feu, et une très grande partie du mur tomba. .

\par 6 Et les hommes d'Hyrcan leur résistèrent et les empêchèrent d'entrer, gardant la partie en ruine ;

\par 7 Hyrcan sortit avec la plupart de ses combattants contre l'armée d'Antiochus et les battit avec un grand massacre.

\par 8 Et Antiochus et ses hommes furent mis en déroute ; Hyrcan et ses troupes poursuivirent jusqu'à ce qu'ils les aient chassés de la ville.

\par 9 Puis, revenant aux tours qu'Antiochus avait bâties, ils les détruisirent ; et demeura dans la ville et autour d'elle.

\par 10 Mais Antiochus campait dans un certain lieu, éloigné d'environ deux stades de la ville de la maison de Dieu.

\par 11 Et à l'approche de la fête des tabernacles, Hyrcan lui envoya des ambassadeurs pour négocier une trêve jusqu'à ce que la solennité soit passée ; qu'il lui a accordé; et il envoya des victimes, de l'or et de l'argent », à la maison de Dieu.

\par 12 Et Hyrcan ordonna aux prêtres de recevoir ce qu'Antiochus avait envoyé ; et ils l'ont fait.

\par 13 Hyrcan et les prêtres, voyant la révérence d'Antiochus envers le temple de Dieu, lui envoyèrent des ambassadeurs pour traiter de paix.

\par 14 Ce à quoi Antiochus accepta ; Il partit pour Jérusalem. Hyrcan le rencontra et ils entrèrent ensemble dans la ville.

\par 15 Et Hyrécan fit un festin pour Antiochus et ses princes ; et ils mangèrent et burent ensemble et il lui fit présent de trois cents talents d'or :

\par 16 Et chacun d'eux s'entendit avec son compagnon sur la paix et le secours, et Antiochus partit dans son propre pays.

\par 17 Mais on raconte qu'Hyrcan ouvrit le trésor, qui avait été fait par quelques rois des fils de David, [à qui soit la paix,] et il en sortit une grande somme d'argent, et en laissa autant dans le renvoyant à son ancien état de secret.

\par 18 Puis il reconstruisit et répara la partie du mur qui était tombée ; et il veillait soigneusement au confort et à l'avantage de son troupeau, et se comportait honnêtement envers eux.

\par 19 Or, lorsqu'Antiochus fut arrivé dans son pays, il résolut d'aller combattre contre le roi de Perse, car il s'était révolté dès le temps du premier Antiochus.

\par 20 Et il envoya des ambassadeurs à Hyrcan, pour qu'il aille vers lui ; Hyrcan l'accompagna et partit pour le pays de Perse.

\par 21 Et une armée des Perses vint à sa rencontre et combattit contre lui ; qu'Antiochus, mettant en fuite, vainquit et passa au fil de l'épée.

\par 22 Puis il resta à l'endroit où il se trouvait et érigea un édifice magnifique, afin que ce soit son mémorial dans leur pays.

\par 23 Et après quelque temps il partit à la rencontre du roi des Perses ; et Hyrcan resta en arrière, à cause du sabbat, qui suivit immédiatement la Pentecôte.

\par 24 Et le roi de Perse et Antiochus se rencontrèrent ; et de très grandes batailles eurent lieu entre eux, au cours desquelles Antiochus et une grande partie de son armée furent tués.

\par 25 Et lorsque la nouvelle fut rapportée à Hyricanus, il marcha vers le pays de Syrie et, pendant son voyage, assiégea Halepus.

\par 26 et les citoyens se rendirent à lui, lui rendant hommage ; et il les quitta, revint dans la ville sainte et y resta quelques jours.

\par 27 Puis il partit pour le pays de Samarie, et combattit Neapolis ; mais les citoyens l'empêchèrent d'y entrer.

\par 28 Et il détruisit tous les bâtiments qu'ils avaient sur la montagne de Jezabel et le temple ; ce qui fut fait deux cents ans après que Sanballat le Samaritain l'eut construit. Il tua aussi les prêtres qui étaient à Sébaste.

\par 29 Et il marcha dans le pays d'Iduméza, c'est-à-dire les montagnes de Sarah, et ils se rendirent à lui : à qui il stipula qu'ils devaient se circoncire et adopter la religion de la Torah (ou la loi mosaïque).

\par 30 Et ils furent d'accord avec lui, et furent circoncis, et devinrent Juifs, et furent confirmés dans cette pratique jusqu'à la destruction de la seconde maison.

\par 31 Et Hyréanus | est allé vers toutes les nations environnantes; et bon, tous se sont soumis à lui, et en même temps ont conclu un accord de paix et d'obéissance.

\par 32 Il envoya également des ambassadeurs aux Romains, pour leur parler du renouvellement de la ligue qui les séparait.

\par 33 Quand donc ses ambassadeurs furent venus vers les Romains, ils les honorèrent ; et leur a nommé un siège de dignité; et ils prêtèrent attention à l'ambassade pour laquelle ils étaient venus ; et expédia leurs affaires et répondit à sa lettre.

\chapter{22}

\textit{La copie de la lettre des Romains à Hyrcan}

\par 1 « Depuis l'ancien et ses trois cent vingt gouverneurs, jusqu'à Hyrcan, roi de Juda, santé.

\par 2 Votre lettre nous est déjà parvenue, à la lecture de laquelle nous nous sommes réjouis ; et nous avons interrogé vos ambassadeurs sur l'état de vos affaires.

\par 3 Nous avons aussi reconnu leur place de dignité dans la science, la discipline morale et les vertus ; et nous les avons honorés et les avons fait asseoir en présence de notre aîné :

\par 4 qui a pris soin de régler toutes leurs affaires, en ordonnant que toutes les villes qu'Antiochus avait conquises de force vous soient restituées ;

\par 5 et que tout obstacle à l'exercice de votre religion soit supprimé ; et que tout ce qu'Antiochus avait décrété contre vous serait annulé.

\par 6 Il a aussi ordonné que toutes les villes qu'il avait prises vous restent fidèles ; il a également donné des ordres par lettre à toutes ses provinces, que vos ambassadeurs soient traités avec respect et honneur.

\par 7 Il vous a envoyé avec eux un ambassadeur nommé Cynzeus, porteur d'une lettre ; à qui il a également confié une ambassade, afin qu'il puisse traiter avec vous en personne.

\par 8 C'est pourquoi, lorsque cette épître des Romains fut parvenue à Hyrcan, il commença à être appelé roi, alors qu'on l'appelait autrefois grand prêtre : et ainsi les dignités royales et sacerdotales furent réunies en lui.

\par 9 Et il fut le premier qui fut appelé roi parmi les chefs des Juifs, au temps de la seconde maison.


\chapter{23}

\par \textit{L'histoire des guerres d'Hyrcan avec les Samaritains}

\par 1 Hyrcan marcha vers Sébaste et assiégea longtemps les Samaritains qui s'y trouvaient ; jusqu'à ce qu'il les réduise à une telle situation, qu'ils furent obligés de se nourrir de toutes sortes de carcasses mortes.

\par 2 Néanmoins ils supportèrent cela patiemment, craignant son épée et se confiant aux Macédoniens et aux Égyptiens, dont ils avaient imploré l'aide.

\par 3 Entre-temps arrive le grand jeûne, auquel Hyrcan doit être présent dans la Sainte Maison pour offrir ce jour-là des sacrifices.

\par 4 C'est pourquoi il substitua ses deux fils, Antigone et Aristobule, comme chefs de l'armée ; leur laissant l'ordre d'assiéger les Samaritains et de les réduire aux extrémités.

\par 5 De même, il ordonna à l'armée d'obéir à ses fils et d'exécuter leurs ordres ; et il partit pour la ville de la Sainte Maison.

\par 6 Antiochus le Macédonien marcha au secours des habitants de Sébaste ; La nouvelle en fut portée aux deux fils d'Hyrcan.

\par 7 qui, ayant substitué un général pour conduire le siège de Sébaste, alla à la rencontre d'Antiochus ; qu'ils rencontrèrent et mirent en déroute, et retournèrent à Sébaste.

\par 8 Lythras, fils de la reine Cléopâtre, sortit également d'Egypte pour aider les Samaritains.

\par 9 Lorsque la nouvelle fut rapportée à Hyrcan, il alla à sa rencontre, la solennité étant passée. Lorsqu'il le rencontra, il le frappa avec acharnement et tua un grand nombre de ses hommes.

\par 10 et Lythras fut mis en fuite ; Les Égyptiens ne voulurent pas non plus porter secours aux Samaritains après ce retour.

\par 11 Et le roi Hyrcanys revint à Sébaste, et l'attaqua violemment, jusqu'à ce qu'il la prenne avec l'épée, et tua ceux de ses citoyens qui restaient, et la détruisit entièrement, et démolit ses murs.

\chapter{24}

\par \textit{L'histoire de Lythras, fils de Cléopâtre, et de sa marche contre sa mère en Égypte.}

\par 1 Lythras, fils de Clegpatra, devenu fort en biens et en hommes, se révolta contre Cléopâtre, sa mère ; les principaux hommes du royaume étaient ses complices.

\par 2 C'est pourquoi Cléopâtre, ayant fait venir deux Juifs, dont l'un s'appelait Chelcias et l'autre Hananias, les plaça à la tête des princes d'Égypte qui restaient à ses côtés, et les fit tous deux généraux de l'armée égyptienne.

\par 3 Or, ils géraient bien toutes les affaires du peuple et dirigeaient les affaires de l'empire avec sagesse. Cléopâtre les envoya combattre Lythras ;

\par 4 Celui-ci partit pour lui faire la guerre, et le mit en déroute, mettant ses hommes en fuite ; et il s'enfuit à Chypre, et y resta, avec quelques-uns qui lui adhèrent.

\chapter{25}

\par \textit{Un récit des sectes juives à cette époque.}

\par 1 A cette époque, il y avait trois sectes parmi les Juifs. L'un des Pharisiens, c'est-à-dire les « séparés » ou religieux ;

\par 2 dont la règle était de maintenir tout ce qui était contenu dans la loi, selon les exposés de leurs ancêtres.

\par 3 La seconde, celle des Sadducéens ; et ce sont les disciples d'un certain homme des médecins, nommé Sadoc ;

\par 4 dont la règle était de maintenir selon les choses trouvées dans le texte de la loi, et dont il y a une démonstration dans l'Écriture elle-même ; mais pas ce qui n'existe pas dans le texte, ni qui n'en est prouvé.

\par 5 La troisième secte était celle des Hasdanim, ou ceux qui étudiaient les vertus : mais l'auteur de ce livre n'a pas fait mention de leur règle, et nous ne le savons pas non plus sauf dans la mesure où elle est découverte par leur nom :

\par 6 car ils s'appliquaient à des pratiques qui se rapprochaient des vertus les plus éminentes ; à savoir, choisir parmi ces deux autres règles celle qui était la plus sûre en croyance, la plus sûre et la plus gardée.

\par 7 Hyrcan était d'abord un des Pharisiens ; ensuite il passa chez les Sadducéens ;

\par 8 parce que l'un des Pharisiens lui avait dit : Il ne t'est pas permis d'être grand prêtre, parce que ta mère était captive avant de t'enfanter, du temps d'Antiochus ; mais il ne convient pas que le fils d'un captif soit grand prêtre.

\par 9 Et cette conversation eut lieu en présence des chefs des pharisiens ; ce qui fut la cause de son passage au règne des Sadducéens. .

\par 10 Or les Sadducéens étaient en inimitié avec les Pharisiens ; C'est pourquoi ils entretinrent des différends entre eux, et ils l'emportèrent jusqu'à tuer un grand nombre de pharisiens.

\par 11 Et le trouble atteignit une telle ampleur que des guerres et de nombreux maux durent parmi eux pendant un très long temps.

\chapter{26}

\par \textit{Le récit de la mort d'Hyrcan et de l'époque de son règne}

\par 1 Hyrcan eut trois fils, Antigone, Aristobule et Alexandre.

\par 2 Et Hyrcan aimait Antigone et Aristobule ; mais Alexandre lui était odieux.

\par 3 Et un jour il vit en songe que parmi ses fils, Alexandre régnerait après sa mort ; et cela l'inquiétait.

\par 4 Et il n'a pas jugé bon, de son vivant, d'établir aucun des fils qu'il aimait, à cause de sa vision ;

\par 5 ni de nommer Alexandre roi, parce qu'il ne l'aimait pas. C'est pourquoi il a ajourné l'affaire ; qu'après sa mort, cela puisse prendre cette tournure qui devrait plaire au grand et bon Dieu.

\par 6 Or, les Juifs avaient été, du temps de son père et de ses oncles, unis dans l'affection envers eux ; et prompts à leur obéir, en raison de la soumission de leurs ennemis et des excellents exploits qu'ils ont accomplis.

\par 7 Ils restèrent également unis par l'affection envers Hyrcan ; jusqu'à ce qu'il commette le massacre des Pharisiens, l'extermination des Juifs et les guerres civiles à cause de la religion.

\par 8 De là naquirent des inimitiés perpétuelles, des maux incessants et de nombreux meurtres. C’est la raison pour laquelle beaucoup détestaient Hyrcan.

\par 9 Or, la durée de son règne était de trente et un ? ans, et il est mort.

\chapter{27}

\par \textit{L'histoire d'Aristobule, fils d'Hyrcan}

\par 1 Hyrcannus étant mort, son fils Aristobule lui succéda sur le trône ; qui a fait preuve de hauteur, de fierté et de puissance ; et il plaça sur sa tête une grande couronne, au mépris de la couronne du sacerdoce sacré.

\par 2 Or, il avait une tendresse pour son frère Antigone, qu'il préférait à tous ses amis ; mais il gardait en prison son frère Sess, ainsi que sa mère, à cause de son amour pour Alexandre.

\par 3 Et il envoya son frère Antigone, qui combattit contre lui et le vainquit, avec tous ses complices et ses troupes, qu'il mit en fuite, et retourna dans la ville de la Sainte Maison. Cela arriva pendant qu'Aristobule était malade.

\par 4 Alors qu'Antigone se rendait à la ville, on lui rapporta la maladie de son frère ; qui, entrant dans la ville, se rendit à la maison de Dieu, pour rendre grâce pour la miséricorde manifestée dans sa délivrance de l'ennemi, et pour implorer le Dieu grand et bon de redonner la santé à son frère.

\par 5 C'est pourquoi certains de ceux qui étaient adversaires et ennemis d'Antigone se rendirent chez Aristobule et lui dirent :

\par 6 En effet, la nouvelle de ta maladie a été portée à ton frère, et voici, il arrive avec ses partisans, armés ; et il est maintenant allé dans le sanctuaire pour se faire des amis, afin qu'il puisse venir soudainement sur vous et vous tuer.

\par 7 Et le roi Aristobule craignait de prendre quelque mesure hâtive contre son frère concernant ce qui lui avait été dit, jusqu'à ce qu'il connaisse l'exactitude de la nouvelle.

\par 8 C'est pourquoi il ordonna à tous ses serviteurs de se poster armés dans un certain endroit, d'où quiconque venait à son palais ne pouvait se détourner.

\par 9 Il ordonna également qu'il soit proclamé publiquement que personne portant des armes, de quelque sorte que ce soit, ne devrait entrer dans la cour auprès du roi sans se cacher.

\par 10 Après cela, il envoya parler à Antigone, lui ordonnant de venir vers lui ; sur quoi Antigone ôta ses armes pour obéir au roi.

\par 11 Entre-temps, un messager de la femme de son frère Aristobule (qui le haïssait) lui dit :

\par 12 Le roi te dit : « J'ai maintenant entendu parler de la beauté de ta tenue lorsque tu es entré dans la ville, et j'ai désireux de te voir ainsi habillé ; c'est pourquoi viens à moi sous cette forme, afin que je sois heureux de te voir.

\par 13 Et Antigone ne doutait pas que ce message vienne du roi, comme l'avait rapporté le messager ;

\par 14 et qu'il ne voulait pas le mettre sur le même pied que les autres quant au dépôt des armes : et il alla vers lui de cette manière et s'habilla.

\par 15 Et lorsqu'il fut arrivé au lieu où le roi Aristobule avait ordonné à ses hommes de se poster, avec ordre de tuer quiconque y viendrait armé ;

\par 16 et quand les hommes le virent portant ses armes, ils se précipitèrent sur lui et le tuèrent sur-le-champ ; et son sang coulait à cet endroit sur le pavé de marbre.

\par 17 Et le cri des hommes grandissait, et leurs pleurs et leurs lamentations s'amplifiaient, affligés de la mort d'Antigone, à cause de sa beauté, et de l'élégance de son discours et de ses exploits.

\par 18 Alors le roi, entendant le bruit des hommes, s'enquit à ce sujet ; et découvrit qu'Antigonus avait été tué ;

\par 19 ce qui lui causa le plus grand chagrin, soit à cause de l'affection qu'il lui portait, soit parce qu'il ne méritait pas ce sort : et il s'aperçut qu'un piège avait été tendu à son frère :

\par 20 Et il cria à haute voix et pleura abondamment ; et se frappait la poitrine sans cesse ; de sorte que quelques vaisseaux sanguins de sa poitrine éclatèrent et que le sang coula de sa bouche.

\par 21 Mais ses serviteurs et les principaux de ses amis vinrent vers lui, le consolèrent, l'apaisèrent et l'apaisèrent, afin de l'empêcher de faire cette action ;

\par 22 craignant qu'il ne meure, car il était faible et expirait presque sous ce qu'il avait déjà fait.

\par 23 Et ils prirent un bassin d'or, pour recevoir le sang qui jaillissait de sa bouche ;

\par 24 Et ils envoyèrent le bassin, avec le sang, par l'un des serviteurs d'un médecin, afin qu'il le voie et conseille ce qu'il fallait faire pour lui.

\par 25 Et quand il arriva au lieu où Antigone avait été tué, et où son sang avait coulé, le page glissa et tomba ; et il versa le sang du roi qui était dans le bassin sur le sang de son frère assassiné.

\par 26 Et le page revint avec le bassin, et raconta aux courtisans ce qui était arrivé ; qui l'a maltraité et injurié; tandis qu'il se justifiait et jurait qu'il n'avait pas fait cela intentionnellement ou volontairement.

\par 27 Mais quand le roi les entendit se disputer, il le demanda. qu'on leur dise ce qu'ils disaient : et ils gardèrent le silence ; mais comme il les menaçait, ils le lui dirent.

\par 28 Qui dit alors : « Loué soit le juste juge, qui a versé le sang de l'oppresseur sur le sang de l'opprimé. »

\par 29 Alors il gémit et expira aussitôt. Et la durée de son règne fut d'une année complète.

\par 30 Et tout son troupeau le pleurait; car il était noble, victorieux et libéral : et son frère Alexandre nous régnait à sa place.

\chapter{28}

\par \textit{Le récit d'Alexandre, fils d'Hyrcan}

\par 1 Après la mort d'Aristobule, son frère Alexandre fut libéré de ses chaînes ; et étant sorti de prison, il accéda au trône.

\par 2 Or, le gouverneur de la ville d'Acché (qui est Ptolémaïs) s'était révolté ; et il avait envoyé des messagers à Lythras, fils de Cléopâtre, lui demandant de l'aider et de le prendre sous sa protection ;

\par 3 mais il refusa longtemps, craignant que ne se reproduisent les choses qu'il avait souffert auparavant de la part d'Hyrcan.

\par 4 Mais le messager lui donna du courage grâce aux secours promis par le seigneur de Tyr, de Sidon et d'autres. Lythras marcha avec trente mille hommes.

\par 5 et le rapport en fut porté à Alexandre, qui le précéda à Ptolémaïs et l'attaqua ; Les citoyens de Ptolémaïs lui fermèrent la porte au nez et s'efforcèrent de l'empêcher d'entrer.

\par 6 C'est pourquoi Alexandre les mit à l'étroit et continua de les assiéger ; jusqu'à ce qu'il fut informé de la marche de Lythras ; puis il se retira devant eux, Lythras et ses troupes étant à portée de main.

\par 7 Il y avait parmi les citoyens de Ptolémaïs un vieillard d'une autorité reconnue, qui persuada les citoyens de ne pas permettre à Lythras d'entrer dans leur ville, ni de lui obéir, car il était d'une autre religion.

\par 8 Il leur dit encore : La soumission à Alexandre, qui est de la même religion, vous sera bien plus avantageuse à tous égards que la soumission à Lythras ; et il ne cessa pas jusqu'à ce qu'ils acceptent ses sentiments.

\par 9 Et ils empêchèrent Lythras d'entrer dans Ptolémaïs, refusant de se soumettre à lui. Et Lythras était perplexe dans ses affaires, et il ne se demandait pas ce qui était le mieux pour lui de faire.

\par 10 Et cela fut rapporté au roi de Sidon, et il lui envoya des messagers pour qu'il l'aide dans la guerre contre Alexandre ; que soit ils pourraient le vaincre, soit simuler certaines de ses villes, et ainsi le punir ;

\par 11 et ainsi Lythras pourrait retourner dans son propre pays, après avoir accompli des actes qui pourraient le rendre redoutable ; ce qui, en vérité, lui serait plus avantageux que de revenir sans avoir réalisé son dessein.

\par 12 Et cela fut raconté à Alexandre ; qui envoya à Lythras une honorable ambassade avec un présent très précieux, et lui proposa de ne pas aider le roi de Sidon.

\par 13 Et Lythras accepta le présent d'Alexandre, accédant à sa demande.

\par 14 Mais Alexandre marcha vers Sidon et combattit son souverain ; et Dieu le rendit victorieux sur lui, et il tua un grand nombre de ses hommes ; et l'ayant mis en fuite, il prit possession de son pays.

\par 15 Après cela, Alexandre envoya des messagers à Cléopâtre, pour qu'elle vienne avec un PC militaire. contre Lythras, son fils ; et qu'il marcherait aussi avec son armée contre lui et le lui livrerait prisonnier.

\par 16 Quand Lythras l'eut découvert, il s'en alla dans la montagne de Galilée, tua un grand nombre d'habitants et emporta dix mille captifs ; un grand nombre de ses hommes furent également tués.

\par 17 De là il marcha jusqu'à arriver au Jourdain, et il campa là ; afin que ses hommes et ses chevaux puissent se reposer, et ensuite qu'il puisse marcher vers Jérusalem pour combattre avec Alexandre.

\par 18 Cela fut dit à Alexandre ; qui marcha contre lui avec cinquante mille hommes, dont six mille avaient des boucliers d'airain ; et on dit que chacun d'eux pouvait résister à n'importe quel nombre d'hommes.

\par 19 Et il l'attaqua au Jourdain, et là il combattit contre lui ; mais il n'obtint pas la victoire, parce qu'il avait confiance en ses hommes et qu'il avait placé sa confiance dans leur nombre.

\par 20 Mais il y avait avec Lythras des hommes très habiles dans les combats et dans la constitution des armées ; qui lui conseilla de diviser ses forces en deux parties, de sorte que l'une puisse être avec Lythras et sa compagnie prêtes au combat, et l'autre partie avec un autre capitaine de leur compagnie.

\par 21 Et il combattit même jusqu'à midi, et un grand nombre de ses hommes furent tués.

\par 22 Et son ami s'avança, avec le reste de l'armée qui était avec lui, dont la force était encore entière, contre Alexandre et ses hommes, qui étaient alors accablés de fatigue.

\par 23 et il les traita comme il lui plaisait, et il en tua de grandes multitudes ; Alexandre et les hommes qui étaient restés avec lui s'enfuirent dans la ville de la Sainte Maison.

\par 24 Lythras partit aussi vers le soir pour se rendre dans une certaine ville voisine ; et par hasard des femmes juives avec leurs enfants le rencontrèrent ;

\par 25 Et il ordonna de tuer quelques-uns des enfants et de préparer leur chair, prétendant qu'il y en avait dans son armée qui se nourrissaient de chair humaine ; projetant par ces actes de frapper les habitants du pays d'une crainte de ses troupes.

\par 26 Après cela vint Cléopâtre ; qu'Alexandre rencontra, et lui raconta ce que Lythras avait fait à son armée, et lui proposa de l'accompagner à sa recherche.

\par 27 Ce qui étant dit à Lythras, il s'enfuit vers un endroit où était la station de ses navires ; montant à bord duquel il retourna à Chypre ; et Cléopâtre retourna en Egypte.

\par 28 Mais à la fin de l'année, Alexandre marcha contre Gaza ; parce que son chef s'était révolté contre lui et avait envoyé un certain roi des Arabes nommé Hartasi pour l'assister ; qui y consentit et marcha vers Gaza :

\par 29 cela fut dit à Alexandre ; qui, laissant quelques-uns de ses hommes devant Gaza, marcha contre Hartas, l'engagea et le mit en fuite.

\par 30 Puis il revint à Gaza, et, étant endolori, il le prit au bout d'un an.

\par 31 Mais la cause de sa prise était le frère de ce chef ; qui s'étant précipité sur lui, le tua.

\par 32 Comme les citoyens cherchaient à le tuer, il rassembla ses amis, se rendit à la porte de la ville et s'adressa à Alexandre, le suppliant qu'après avoir assuré sa vie et celle de ses amis, il entrerait dans la ville ;

\par 33 Ce qu'Alexandre avait promis, il entra dans Gaza, tua ses habitants, renversa le temple qui s'y trouvait et brûla l'idole dorée qui était dans le temple.

\par 34 Après quoi il partit pour la ville de la Sainte Maison, et y célébra la fête des tabernacles.

\par 35 Et lorsque la fête fut passée, il se prépara contre Hartas, qu'il rencontra, et tua un grand nombre de ses hommes :

\par 36 Les affaires de Hartas étaient très difficiles et paralysées, et il craignait sa propre extinction complète. C'est pourquoi, réclamant la vie d'Alexandre, il lui rendit obéissance et lui rendit hommage.

\par 37 Et Alexandre le quitta, et marcha contre Hémath et Tyr, et les prit ; et après avoir reçu le tribut des habitants, il retourna dans la ville de la Sainte Maison.

\chapter{29}

\par \textit{Un récit des batailles qui ont eu lieu entre les Pharisiens et les Sadducéens.}

\par 1 Ensuite des malheurs surgirent entre les Pharisiens et les Sadducéens, et se poursuivirent pendant six ans.

\par 2 Et Alexandre secourut les sadducéens contre les pharisiens, dont cinquante mille furent tués en six ans.

\par 3 C'est pourquoi entre ces deux sectes, l'état de choses fut réduit à une destruction totale, et leur inimitié fut complètement confirmée.

\par 4 Alors Alexandre, ayant envoyé chercher les anciens de chaque secte, leur parla gentiment et leur conseilla une réconciliation.

\par 5 Mais ils lui répondirent : « En vérité, à notre avis, tu mérites la mort, à cause de l'abondance de sang innocent que tu as versé ; qu'il n'y ait donc entre nous que l'épée. »

\par 6 Après cela, ils commencèrent à manifester ouvertement leur inimitié, en envoyant des messagers à Démétrius, roi de Macédoine, pour qu'il vienne vers eux avec une armée ;

\par 7 promettant qu'ils l'aideraient contre Alexandre et son parti, et réduiraient les Hébreux à la soumission aux Macédoniens. Et Démétrius marcha vers eux avec une grande armée.

\par 8 Ce qui fut aussi raconté à Alexandre ; qui envoya quelqu'un pour embaucher six mille Macédoniens, qu'il joignit à ses propres forces et avança contre Démétrius.

\par 9 Beaucoup aussi de Juifs, de pharisiens, passèrent chez Démétrius.

\par 10 Et Démétrius envoya secrètement des gens vers les Macédoniens qui étaient avec Alexandre, pour les détourner de lui ; mais ils ne l'écoutèrent pas.

\par 11 Alexandre envoya aussi secrètement des hommes vers les Juifs qui étaient avec Démétrius, pour les ramener à son côté ; mais ceux-ci non plus ne firent pas ce qu'il voulait qu'ils fassent.

\par 12 Et Alexandre et Démétrius se rencontrèrent et livrèrent bataille ; dans lequel tous les hommes d'Alexandre tombèrent, et il s'enfuit seul dans le pays de Juda.

\par 13 Mais quand ses hommes l'entendirent, ils murmurèrent qu'il s'était enfui sain et sauf et qu'il avait découvert l'endroit où il se trouvait ;

\par 14 Là, environ six mille hommes, parmi les plus vaillants des fils d'Israël, s'assemblèrent autour de lui ; et beaucoup de ceux qui s'étaient révoltés contre Démétrius se joignirent à lui.

\par 15 Ensuite, les hommes affluèrent vers lui de toutes parts ; et il revint pour livrer bataille à Démétrius avec une force nombreuse, et le mit en fuite ; et Démétrius retourna dans son propre pays.

\par 16 Et Alexandre marcha contre lui jusqu'à Antioche, et l'assiégea trois ans ; et quand Démétrius sortit pour combattre, Alexandre le vainquit et le tua :

\par 17 Et il quitta la ville, et retourna à Jérusalem vers ses citoyens ; qui l'a magnifié, l'honorant et le louant pour avoir vaincu ses ennemis.

\par 18 Et les Juifs convinrent de se soumettre à lui, et son cœur fut en paix ; et il envoya ses armées contre tous ses ennemis, qu'il mit en fuite, et remporta la victoire sur eux.

\par 19 Il prit également possession des montagnes de Sara, et du pays d'Ammon, et de Moab, et du pays des Philistins, et de toutes les parties qui étaient aux mains des Arabes qui combattaient avec lui, jusqu'aux limites. du désert.

\par 20 Et les affaires de son royaume étaient bien réglées ; et il mit son peuple et son pays en sécurité.

\chapter{30}

\par \textit{Le récit de la mort d'Alexandre, fils d'Hyrcan}

\par 1 Ensuite, le roi Alexandre fut atteint d'une fièvre quarte pendant trois années entières.

\par 2 Mais lorsque le gouverneur d'une ville nommée Ragaba se révolta contre lui, il y conduisit une armée puissante, emmenant avec lui sa femme et sa famille, et assiégea la ville.

\par 3 Mais quand il fut sur le point d'être pris, sa maladie s'aggrava et ses forces déclinèrent ; et sa femme, qui s'appelait Alexandra, perdit tout espoir de guérison :

\par 4 qui s'approchant de lui dit : « Vous savez maintenant quelles différences il y a entre vous et les pharisiens : et vos deux fils sont des petits garçons, et je suis une femme, et nous ne pourrons pas leur résister : quel conseil donc me donnez-vous, ainsi qu'à eux ? »

\par 5 Il lui dit : « Mon conseil est que tu persévères contre la ville jusqu'à ce qu'elle soit prise, ce qui sera bientôt.

\par 6 Et quand elle sera conquise, établissez son gouvernement comme les autres villes ont été établies.

\par 7 Mais envers tous ces gens, faites semblant que je suis malade ; et quoi que vous fassiez, faites comme si vous le faisiez à ma suggestion ; et révèle ma mort à ces serviteurs sur lesquels tu peux compter.

\par 8 Et quand vous aurez terminé ces choses, allez dans la ville de la Sainte Maison, après avoir préalablement séché et embaumé mon corps avec des aromates ; et remplis de parfums l'endroit où je repose, afin qu'aucune odeur désagréable ne s'échappe de moi.

\par 9 Et quand les affaires du pays seront réglées, pars de là, et roule-moi dans une abondance de parfums, et emmène-moi dans le palais, comme si j'étais malade.

\par 10 Et quand j'y serai, fais venir les principaux hommes des Pharisiens ; et quand ils viendront, honorez-les et dites-leur de bonnes paroles :

\par 11 Alors dis : Alexandre est déjà mort, et voici, je te le livre, fais-en ce qui te semblera bon, et désormais je te traiterai comme tu voudras.

\par 12 Car si vous faites cela, je sais bien qu'ils ne feront rien à moi et à vous, sinon ce qui est bon ; et le peuple les suivra, et vos affaires seront parfaitement réglées après ma mort, et vous régnerez en toute sécurité jusqu'à ce que vos deux fils soient grands.

\par 13 Après cela, Alexandre mourut ; et sa femme a caché sa mort ; et quand la ville fut prise, elle retourna à Jérusalem ; et ayant fait appeler les principaux des pharisiens, elle leur parla comme Alexandre le lui avait conseillé.

\par 14 A quoi ils répondirent qu'Alexandre avait été leur roi, et qu'ils avaient été son peuple ; et ils lui parlèrent avec toute l'affection, et promirent de la mettre à la tête de leur gouvernement.

\par 15 Alors ils sortirent et rassemblèrent des hommes ; et prenant le corps d'Alexandre, ils l'emportèrent magnifiquement jusqu'à son enterrement : et ils envoyèrent chercher des hommes pour nommer Alexandra reine ; avec l'accord duquel elle a été ainsi nommée.

\par 16 Et les années du règne d'Alexandre furent de vingt-sept ans.

\chapter{31}

\par \textit{L'histoire de la reine Alexandra}

\par 1 Or, pendant qu'Alexandra régnait, elle appela auprès d'elle les chefs des Pharisiens et leur ordonna d'écrire à tous ceux de leur secte qui s'étaient enfuis en Égypte et ailleurs, aux jours d'Hyrcan et d'Alexandre, qu'ils devrait retourner au pays de Juda.

\par 2 Et elle leur montra son inclination favorable envers eux, et ne s'opposa pas à leurs rites, ni n'interdit leurs cérémonies, comme Alexandre et Hyrcan le leur avaient défendu.

\par 3 Elle a également libéré tous ceux qui étaient détenus en prison.

\par 4 Et ils se rassemblèrent de tous côtés ; et les Sadducéens s'abstinrent de leur offrir toute violence.

\par 5 Et leurs affaires étaient bien réglées, et leur situation s'améliorait grâce à la résolution des querelles.

\par 6 Mais quand Hyrcan et Aristobule, les deux fils d'Alexandre, furent grands, la reine fit Hyrcan grand prêtre, car il était doux, doux et honnête.

\par 7 Mais elle nomma Aristobule général de l'armée, car il était vaillant, courageux et plein d'entrain ; et elle lui donna aussi l'armée des Sadducéens : mais elle ne jugea pas convenable de le nommer roi, car il était encore un garçon.

\par 8 Elle envoya en outre tous ceux qui payaient tribut à Alexandre, et prit les fils de leurs rois, qu'elle retenait près d'elle comme otages ; et ils continuèrent sans interruption à lui obéir, lui rendant hommage chaque année.

\par 9 Et elle marchait honnêtement avec son peuple, rendant la justice et ordonnant à son peuple de faire de même. C'est pourquoi il y eut une paix durable entre les parties et elle gagna leur bonne volonté.

\chapter{32}

\par \textit{Un récit des choses qui furent faites aux Sadducéens par les Pharisiens au temps d'Alexandra.}

\par 1 Il y avait parmi les sadducéens un homme important, promu par Alexandre, nommé Diogène, qui l'avait autrefois incité à tuer huit cents hommes pharisiens.

\par 2 C'est pourquoi les chefs des Pharisiens viennent vers Alexandra et lui rappellent ce qu'avait fait Diogène, lui demandant la permission de le tuer ; et, l'ayant reçu, ils tuèrent avec lui de nombreux Sadducéens.

\par 3 Ce que les Sadducéens, prenant très à cœur, allèrent trouver Aristobule ; et l'emmenant avec eux, il alla vers la reine et lui dit :

\par 4 « Vous savez quelles choses terribles et lourdes nous avons endurées, et les nombreuses guerres et batailles que nous avons livrées, en faveur d'Alexandre et de son père Hyrcan.

\par 5 C'est pourquoi il n'était pas convenable de fouler aux pieds nos droits, de lever sur nous la main de nos ennemis et d'abaisser notre dignité ;

\par 6 car une affaire de ce genre ne sera pas cachée à Hartas et aux autres de vos ennemis ; qui ont expérimenté notre courage et n'ont pas pu nous résister, et leur cœur a été rempli de peur de nous.

\par 7 Quand donc ils comprendront ce que vous nous avez fait, ils s'imagineront que nos cœurs forment des plans contre vous ; et quand ils seront certifiés, ayez confiance qu'ils vous joueront un mauvais tour.

\par 8 Nous ne supporterons pas non plus d'être tués par les pharisiens, comme des moutons.

\par 9 C'est pourquoi, ou retiens de nous leur méchanceté, ou permets-nous de sortir de la ville dans certaines des villes de Juda.

\par 10 Et elle leur dit : Faites ceci, afin que leur ennui à votre égard soit évité.

\par 11 Et les Sadducéens sortirent de la ville ; et leurs chefs partirent avec les hommes de guerre qui leur étaient attachés ; et ils allèrent avec leur bétail vers celles des villes de Juda qu'ils avaient choisies, et y demeurèrent ;

\par 12 et se joignirent à eux ceux qui étaient dévoués à la vertu, (c'est-à-dire les Hasdanim.)

\chapter{33}

\par \textit{Le récit de la mort d'Alexandra}

\par 1 Après ces choses, Alexandra tomba dans une maladie dont elle mourut.

\par 2 Et comme sa guérison était presque désespérée, son fils Aristobule sortit de nuit de Jérusalem, accompagné de son serviteur.

\par 3 et il partit pour Gabatha ?, chez un certain chef parmi les Sadducéens, un de ses amis ;

\par 4 et l'emmenant avec lui, il se rendit dans les villes où habitaient les sadducéens ; il leur révéla son dessein et les exhorta à sortir avec lui, à être ses alliés dans la guerre contre son frère et les pharisiens, et à le nommer roi.

\par 5 À qui ils consentirent, ils jouèrent ouvertement faux avec Alexandra, rassemblant des hommes de pied pour rejoindre Aristobule.

\par 6 Lorsque la nouvelle de ces choses parvint à Hyrcan, fils d'Alexandra, le grand prêtre, et aux anciens des pharisiens, ils allèrent trouver Alexandra, malade comme elle, et lui racontèrent l'affaire ;

\par 7 lui faisant sentir la grande crainte qu'ils avaient pour elle et pour son fils Hyrcan, de la part d'Aristobule et de ceux qui étaient avec lui.

\par 8 À qui elle répondit ; « Je suis vraiment près de la mort, de sorte qu'il est plus approprié et plus rentable pour moi de m'occuper de mes propres affaires ; que puis-je donc faire, étant ainsi placé ?

\par 9 Mais mes hommes, mes biens et mes armes sont avec vous et entre vos mains ; ordonnez donc les affaires comme bon vous semble, implorant l'aide de Dieu dans vos affaires et lui demandant la délivrance. Puis elle est morte.

\par 10 Le montant de son âge était de soixante-treize ans ; et le temps de son règne neuf ans,

\chapter{34}

\par \textit{Le récit de l'attaque d'Aristobule contre son frère Hyrcan, après la mort d'Alexandra.}

\par 1 Lorsqu'Aristobule quitta Jérusalem, au temps d'Alexandra, il laissa sa femme et ses enfants à Jérusalem.

\par 2 Mais lorsque la nouvelle de son départ parvint à Alexandra, elle les enferma dans une certaine maison, les plaçant sous garde.

\par 3 Mais quand Alexandra fut morte, Hyrcan les appela, et se comporta avec eux avec bonté et prit soin d'eux ; afin qu'ils puissent le délivrer de son frère, s'il parvenait à le vaincre.

\par 4 Alors Aristobule mena une grande armée jusqu'au Jourdain ; et Hyrcan sortit contre lui avec une armée de pharisiens.

\par 5 Et lorsque les deux armées se rencontrèrent, un grand nombre de soldats d'Hyrcan furent tués ; Hyrcan et le reste de son armée prirent la fuite.

\par 6 Aristobule et ses troupes, qu'ils poursuivaient, tuèrent tous ceux qu'ils capturèrent, à l'exception de ceux qui se rendirent.

\par 7 Alors Hyrcan se retira dans la Ville sainte ; où arrivèrent aussi Aristobule et son armée ; Il l'entoura de toutes parts avec ses tentes, et tenta par stratagème de détruire la fortification.

\par 8 Et les anciens de Juda et les anciens des prêtres sortirent vers lui et lui défendirent de faire ce qu'il avait projeté ; lui demandant d'écarter de son esprit tout sentiment hostile qu'il avait envers son frère : proposition à laquelle il consentit.

\par 9 Alors il fut convenu entre eux qu'Aristobule serait roi de Juda, et qu'Hyrcan serait grand prêtre dans la maison de Dieu, et à côté de la dignité royale.

\par 10 Aristobule accepta ces conditions, entra dans la ville et eut un entretien avec son frère dans la maison de Dieu. et ils prêtèrent ensemble serment de ratifier les termes dont les anciens s'étaient mutuellement mis d'accord.

\par 11 Ainsi Aristobule fut nommé roi, et Hyrcanus fut placé après lui.

\par 12 Et les hommes étaient en paix, et les affaires de ces deux frères étaient correctement réglées, et l'état de leur peuple et de leur pays devint un état de tranquillité.

\chapter{35}

\par \textit{Le récit d'Antipater (c'est-à-dire du roi Hérode) et des séditions et des batailles qu'il provoqua entre Hyrcan et Aristobule.}

\par 1 Il y avait un homme d'entre les Juifs, des fils de certains de ceux qui montèrent de Babylone avec Esdras le prêtre, nommé Antipater.

\par 2 Et il était sage, prudent, perspicace, courageux et noble, de bonne humeur, bon et courtois ; également riche et possédant de nombreuses maisons, biens et troupeaux.

\par 3 Cet homme, le roi Alexandre, l'avait nommé gouverneur du pays des Idumzans, d'où il avait pris une femme ; dont il eut quatre fils, à savoir Phaselus, Hérode, qui régna sur Juda, Phéroras et Josèphe.

\par 4 Ensuite, ayant quitté les montagnes de Sara, c'est-à-dire le pays des Idumézans, au temps d'Alexandre, il habita dans la ville de la Sainte Maison :

\par 5 Et Hyrcan l'aimait et était très enclin à lui ; c'est pourquoi Aristobule cherchait à le tuer ; ce qu’il n’a cependant pas accompli.

\par 6 Antipater avait donc une peur excessive d'Aristobule et, pour cette raison, commença à comploter secrètement contre le royaume d'Aristobule.

\par 7 Il se rendit donc chez les principaux du royaume, et ayant obtenu d'eux un gage de secret concernant les choses qu'il allait communiquer,

\par 8 Il commença à leur parler de la vie infâme d'Aristobule, de sa tyrannie, de son impiété, du sang versé qu'il avait causé et de son usurpation du trône ; dont son frère aîné était plus digne.

\par 9 Puis il leur ordonna de se méfier du Dieu grand et bon, à moins qu'ils n'enlèvent la main dirigeante du tyran et ne rétablissent ce qui était dû à leur souverain légitime.

\par 10 Il ne resta pas non plus un seul des principaux hommes qu'il n'abusa et ne penchant à se soumettre à Hyrcan, les détournant de leur obéissance à Aristobule, Hyrcan n'en sachant rien.

\par 11 mais Antipater attribué ? tout cela, ne voulant pas le lui dire avant d'avoir établi la chose.

\par 12 C'est pourquoi, après avoir entièrement réglé cette affaire avec le peuple, il alla vers Hyrcan et lui dit :

\par 13 En vérité, ton frère a très peur de toi, parce qu'il voit que son domaine ne sera en aucune façon assuré pendant que tu es en vie ; c'est pourquoi il cherche une occasion de vous tuer et ne permettra pas que vous viviez.

\par 14 Mais Hyrcan ne lui accorda pas foi, à cause de la bonté et de la sincérité de son cœur. C'est pourquoi Antipater lui répétait sans cesse ce discours.

\par 15 Il donna également de grosses sommes d'argent aux personnes en qui Hyrcan avait confiance, et convint avec elles qu'elles lui diraient des choses semblables à celles d'Antipater ;

\par 16 en prenant garde seulement qu'il ne s'imagine pas qu'ils savaient qu'Antipater lui avait parlé de ce sujet.

\par 17 Hyrcan crut donc à leurs paroles ; et fut amené à élaborer un plan par lequel il pourrait être délivré de son frère.

\par 18 Quand donc Antipater lui reparla de l'affaire, il l'informa que le gc. la vérité de ses paroles lui était maintenant manifeste, et il savait qu'il l'avait bien conseillé ; et il demanda son conseil dans cette affaire.

\par 19 Et Antipater lui conseilla d'aller hors de la ville vers quelqu'un en qui il pourrait se confier et qui pourrait l'aider et l'assister.

\par 20 Et Antipater se rendit chez Hartam, et convint avec lui de recevoir Hyrcan comme hôte à son arrivée, car il craignait plutôt de demeurer avec son frère.

\par 21 Ce dont Hartam se réjouit, entra dans le projet et convint avec Antipater qu'en aucun cas il ne livrerait Hyrcan et Antipater à leurs ennemis, et qu'il les assisterait et les protégerait.

\par 22 Et il revint à Jérusalem, et fit connaître à Hyrcan ce qu'il avait fait, et comment il s'était entendu avec Hartam concernant leur départ vers lui.

\par 23 C'est pourquoi tous deux sortirent de la ville de nuit, et se rendirent à Hartam, et restèrent quelque temps avec lui.

\par 24 Alors Antipater commença à persuader Hartam de diriger une armée avec Hyrcan pour réduire et capturer son frère Aristobule.

\par 25 Mais Hartam refusa de poursuivre ce projet, craignant de ne pas avoir la force de résister à Aristobule.

\par 26 Mais Antipater ne cessait de lui montrer que l'affaire avec Aristobule était facile, et de l'y pousser par des arguments sur le trésor à gagner, et par la grandeur de gloire qu'il acquerrait, et la mémoire qu'il aurait. laisse derrière lui :

\par 27 jusqu'à ce qu'il consente à marcher ; mais à condition qu'Hyrcan lui rendrait toutes les villes et villages qui lui appartenaient, son père Alexandre l'avait emporté.

\par 28 Hyrcan acceptant et complétant le traité, Hartam marcha (et Hyrcan avec lui) avec cinquante mille cavaliers et fantassins, se dirigeant vers le pays de Juda : contre lequel Aristobule sortit et les engagea.

\par 29 Et lorsque le combat fut devenu acharné, une grande partie de l'armée d'Aristobule se dirigea vers Hyrcan.

\par 30 Aristobule, s'en apercevant, sonna la retraite et revint à son camp, craignant que toute son armée ne se dérobât peu à peu devant l'ennemi, et qu'ainsi il ne fût lui-même fait prisonnier.

\par 31 Mais la nuit étant venue, Aristobule quitta seul le camp et se rendit à la Ville Sainte.

\par 32 Et lorsque, au point du jour, son départ fut connu de l'armée, la plupart d'entre eux se joignirent à Hyrcan, et le reste se dispersa et partit.

\par 33 Mais Hyrcan, Hartam et Antipater se dirigèrent droit vers la ville de la Sainte Maison, emportant avec eux une grande armée ;

\par 34 et ils trouvèrent Aristobule déjà préparé pour le siège ; car il avait fermé les portes de la ville et placé des hommes sur les remparts pour les défendre.

\par 35 Et Hyrcan et Hartam campèrent avec leurs troupes contre la ville et l'assiégèrent.


\chapter{36}

\par \textit{L'histoire de Gneus, général de l'armée des Romains.}

\par 1 Il arriva que Gneus, général de l'armée des Romains, partit combattre l'Arménien Tyrcan.

\par 2 car les citoyens de Damas, Hamès et Halepum, et le reste de la Syrie qui appartient aux Arméniens, s'étaient récemment rebellés contre les Romains :

\par 3 C'est pourquoi Gneus avait envoyé Scaurus à Damas et dans ses territoires pour en prendre possession ; ce qui fut raconté à Aristobule et à Hyrcan.

\par 4 C'est pourquoi Aristobule envoya des ambassadeurs à Scaurus et beaucoup d'argent, le priant de venir vers lui avec une armée et de l'aider contre Hyrcan.

\par 5 Hyrcan lui envoya aussi des ambassadeurs pour lui demander son aide contre Aristobule ; mais il ne lui a pas envoyé de cadeau.

\par 6 Mais Scaurus refusa d'aller vers l'un ou l'autre ; mais il écrivit à Hartam, lui ordonnant de se retirer avec son armée de la ville de la Sainte Maison, et lui défendit de prêter secours à Hyrcan contre son frère ;

\par 7 et il le menaça d'entrer dans son pays avec une armée de Romains et de Syriens, s'il n'obéissait pas.

\par 8 Or, lorsque cette lettre fut parvenue à Hartam, il se retira aussitôt de la ville :

\par 9 Hyrcan se retira également ; Aristobule les poursuivit avec un certain nombre de ses troupes, les rattrapa et les engagea. Un grand nombre d'Arabes furent tués dans cette bataille, ainsi qu'un très grand nombre de Juifs. Et Aristobule retourna dans la Ville Sainte.

\par 10 Entre-temps, Gneus arriva à Damas ; à qui Aristobule envoya, par la main d'un homme nommé Nicomède, un jardin et une vigne : d'or, pesant au total cinq cents talents, avec un présent des plus riches ; et le supplia de l'aider contre Hyrcan.

\par 11 Hyrcan envoya également Antipater à Pompée, avec la même demande.

\par 12 Et Pompée (qui est Gneus) était enclin à aider Aristobule.

\par 13 Ce qu'Antipater, voyant, chercha une occasion de parler seul avec Pompée, et lui dit :

\par 14 « En vérité, le présent que vous avez reçu d'Aristobule n'a pas besoin de lui être restitué, même si vous ne deviez pas l'aider ;

\par 15 Pourtant, Hyrcan vous offre deux fois plus : et Aristobule ne pourra pas vous soumettre les Juifs, mais Hyrcan le fera.

\par 16 Et Pompée supposait qu'il en était ainsi comme l'avait dit Antipater ; et se réjouissait de penser qu'il pourrait amener les Juifs sous sa domination.

\par 17 C'est pourquoi il dit à Antipater : soutiendra ton ami contre Aristobule ; bien que je puisse prétendre l'aider contre vous, afin qu'il se confie à moi.

\par 18 Car je suis sûr que dès qu'il découvrira que j'aide son frère contre lui, il jouera faux avec tous ses hommes, et prendra soin de lui-même, et ses affaires dureront beaucoup plus longtemps. retardé.

\par 19 Mais je l'enverrai chercher, et j'irai avec lui dans la Ville Sainte, et j'agirai alors de manière à ce que ton ami obtienne son droit ; mais à cette condition qu’il nous paiera un tribut annuel.

\par 20 LE MESSAGER D'ARISTOBULE. Après cela, ayant fait appeler Nicomède, il lui dit : « Allez chez votre maître et dites-lui que j'ai consenti à sa demande ; et porte-lui ma lettre, et dis-lui qu'il doit venir me voir en toute hâte, car je l'attends.

\par 21 Et il écrivit une lettre à Aristobule, dont voici une copie :

\par 22 « Depuis Gneus, général de l'armée des Romains, jusqu'au roi Aristobule, héritier du trône et du grand sacerdoce, la santé soit pour vous.

\par 23 Ton jardin et ta vigne d'or sont arrivés ; et je les ai reçus et je les ai envoyés « à l'ancien et aux gouverneurs ; qu'ils ont accepté « et qu'ils ont placé dans le temple ! à Rome, je vous rends grâce.

\par 24 Ils ont écrit que je t'assisterais et que je t'établirais roi des Juifs.

\par 25 Si donc vous jugez bon de « venir à moi en toute hâte, afin que je monte avec vous à la Ville Sainte et accomplisse vos souhaits, je le ferai ».

\par 26 Et Nicomède partit vers Aristobule avec la lettre de Gneus. Et Antipater, retournant à Hyrcan, lui fit part de la promesse de Gneus, et lui conseilla d'aller à Damas.

\par 27 Hyrcan se rendit donc à Damas ; Aristobule s'y rendit aussi ; et ils se réunirent à Damas dans la salle d'audience de Pompée, c'est-à-dire Gneus ; et Antipater et les anciens des Juifs dirent à Gneus :

\par 28 « Sachez, très illustre général, que ceci. Aristobule a agi faussement envers nous et a usurpé par l'épée le royaume de son frère Hyrcan, qui en est plus digne que lui, puisqu'il est le frère aîné et qu'il mène une vie meilleure et plus correcte.

\par 29 Et il ne lui a pas suffi d'opprimer son frère, mais il a opprimé toutes les nations qui nous entourent ; versant leur sang et pillant injustement leurs biens, et entretenant des inimitiés entre nous et eux, une chose que nous abhorrons.

\par 30 Alors se levèrent mille vieillards, attestant la vérité de ses paroles.

\par 31 Et Aristobule dit : « En vérité, mon frère est meilleur que moi ; mais je n'ai pas cherché le trône, jusqu'à ce que je m'aperçoive que tous ceux qui avaient été soumis à notre père Alexandre nous traitaient faussement après sa mort, connaissant l'incapacité de mon frère.

\par 32 Et en y réfléchissant, je compris que c'était mon devoir d'entreprendre la souveraineté, en ce sens que j'étais meilleur que lui en matière de guerre, et par là j'étais mieux placé pour conserver la monarchie.

\par 33 et je suis allé en guerre contre tous ceux qui nous traitaient faussement, et je les ai réduits à l'obéissance : et tel était le commandement de notre père avant sa mort.

\par 34 Et il fit venir des témoins qui attestèrent la vérité de ses paroles.

\par 35 Après ces choses, Pompée quitta la ville de Damas pour se rendre à la Sainte Maison.

\par 36 Mais Antipater envoya en particulier auprès des habitants des villes conquises par Aristobule, les incitant à se plaindre à Gneus, et leur exposant la tyrannie qu'il avait exercée sur eux ; quelle chose ils ont fait.

\par 37 Et Gneus lui ordonna de leur écrire un témoignage de leur liberté, et de dire qu'il ne les dérangerait plus en aucune manière ; ce qu'il fit réellement, et les nations furent libérées de leur obéissance aux Juifs.

\par 38 Mais quand Aristobule vit ce que Gneus lui avait fait, lui et ses hommes quittèrent de nuit l'armée de Gneus sans l'en informer, et se dirigèrent vers la ville de la Sainte Maison.

\par 39 Et Gneus le suivit jusqu'à ce qu'il arrivât à la ville de la Sainte Maison, autour de laquelle il campait.

\par 40 Mais voyant la hauteur des murs et la force de ses bâtiments, et la multitude d'hommes qui s'y trouvaient, et les montagnes qui l'entouraient, il comprit que la flatterie et la ruse seraient plus utiles contre Aristobule que actes de provocation :

\par 41 C'est pourquoi il lui envoya des ambassadeurs pour qu'il vienne vers lui, en lui promettant un sauf-conduit. Et Aristobule sortit vers lui ; que Gneus reçut avec bonté, sans dire un mot de ses actes passés. Après cela, Aristobule dit à Gneus :

\par 42 « Je voudrais que tu m'aides contre mon frère, sans donner à mes ennemis aucun pouvoir sur moi ; et pour cela, vous aurez tout ce que vous voudrez.

\par 43 Gneus répondit : « Si tu le souhaites, apporte-moi l'argent et les pierres précieuses qui se trouvent dans le « temple, et je te mettrai en possession de ce que tu veux. » Et Aristobule lui dit :

\par 44 «C'est ce que je ferai sans aucun doute.» Et Gneus envoya un capitaine nommé Gabinius avec un grand nombre d'hommes, pour recevoir tout l'or et les joyaux qui se trouvaient dans le temple.

\par 45 Mais les citoyens et les prêtres refusèrent de permettre cela. C'est pourquoi ils résistèrent à Gabinius, tuèrent beaucoup de ses hommes et de ses amis, et le chassèrent de la ville.

\par 46 Sur quoi Gneus, en colère contre Aristobule, le jeta en prison. »

\par 47 Puis il marcha avec son armée pour se frayer un chemin dans la ville et y entrer. Mais un grand nombre de citoyens qui sortaient l'en empêchèrent, en tuant un grand nombre de ses hommes.

\par 48 Et en vérité, le nombre, l'esprit et la bravoure de la nation qu'il avait vu l'effrayaient ; De sorte que, alarmé par ces événements, il avait résolu de se retirer d'eux, si des querelles malveillantes n'avaient pas éclaté dans la ville entre les amis d'Aristobule et les amis d'Hyrcan.

\par 49 Car certains d'entre eux voulaient ouvrir les portes à Pompée, mais d'autres y étaient opposés. C’est pourquoi ils en sont venus aux mains à ce sujet ; et comme cet état de choses augmentait plutôt que diminuait, la guerre continua.

\par 50 Ce que Pompée remarqua, entoura avec son armée la porte de la ville ; et comme quelques gens du peuple lui ouvraient un guichet, il entra et prit possession du palais du roi ; mais il ne put gagner le temple, parce que les prêtres avaient fermé les portes et sécurisé les abords par des hommes armés.

\par 51 Il envoya contre eux des hommes pour les attaquer de toutes parts, et ils les mirent en fuite. Et ses amis, arrivant au temple, montèrent sur le mur, y descendirent et en ouvrirent les portes, après avoir tué une multitude de prêtres.

\par 52 Alors Gneus vint et y entra, et admira grandement la beauté et la magnificence qu'il voyait, et fut étonné quand il vit ses richesses et les pierres précieuses qui y étaient.

\par 53 et il s'est abstenu d'en retirer quoi que ce soit ; et il ordonna aux prêtres de nettoyer la maison des tués et d'offrir des sacrifices selon les cérémonies de leur pays.

\chapter{37}

\par \textit{Le récit de la nomination d'Hyrcan, fils d'Alexandre, comme roi des Juifs, et du retour à Rome du général de l'armée romaine.}

\par 1 Après avoir arrangé ces choses, Pompée nomma Hyrcan pour roi ; et emporta son frère Aristobule enchaîné :

\par 2 il ordonna également que les Juifs n'aient aucune domination sur les nations qui avaient été soumises par leurs rois avant son arrivée ;

\par 3 et il exigea un tribut de la ville de la Sainte Maison ; et il fit alliance avec Hyrcan de recevoir chaque année l'inauguration des Romains.

\par 4 Et il partit, emmenant avec lui Aristobule, et deux de ses fils et ses filles ; et il lui restait un fils, nommé Alexandre, que Pompée ne put saisir, parce qu'il s'était enfui.

\par 5 Ainsi Pompée plaça dans sa chambre, dans la ville de la Sainte Maison, Hyrcan et Antipater, et avec eux son propre collègue Scaurus.

\chapter{38}

\par \textit{L'histoire d'Alexandre, fils d'Aristobule}

\par 1 Lorsque Pompée partit pour Rome, Hyrcan et Antipater marchèrent contre les Arabes pour les soumettre aux Romains.

\par 2 Ce à quoi les Arabes se soumirent, confiants dans leur intimité avec Antipater, et prêtant une grande attention à ses conseils ; par quels actes Antipater avait pour but de réconcilier les Romains avec lui.

\par 3 C'est pourquoi Alexandre, fils d'Aristoulus, aperçut l'expédition d'Hyrcan, d'Antipater et de Scaurus contre les Arabes, et qu'ils s'étaient éloignés très loin de la ville sainte ;

\par 4 il a voyagé jusqu'à ce qu'il soit arrivé là-bas ; et, entrant dans le palais, il en sortit de l'argent pour réparer les murs de la ville que Pompée avait détruits.

\par 5 Et il leva une armée et arrangea toutes les affaires qu'il voulait, avant qu'Hyrcan et son groupe ne retournent à la ville de la Sainte Maison. Et quand ils revinrent,

\par 6 Il sortit à leur rencontre, les engagea et les mit en fuite.


\chapter{39}

\par \textit{L'histoire de Gabinius et d'Alexandre, fils d'Aristobule.}

\par 1 Gabinius était parti de Rome pour s'établir dans le pays de Syrie et en prendre soin ;

\par 2 et on lui raconta ce qu'Alexandre, fils d'Aristobule, avait fait, en reconstruisant ce que Pompée avait démoli, en s'opposant à son successeur et en tuant ses amis.

\par 3 C'est pourquoi il alla droit jusqu'à ce qu'il atteigne Jérusalem ; Hyrcan et ses compagnons le rejoignirent.

\par 4 Contre lequel Alexandre sortit avec dix mille fantassins et quinze cents chevaux, et les rencontra :

\par 5 et ils le mirent en déroute, et tuèrent un certain nombre de ses amis ; et il s'enfuit dans une certaine ville du pays de Juda, appelée Alexandrium, dans laquelle il se fortifia avec sa compagnie.

\par 6 Hyrcan, Gabinius et leurs troupes marchèrent contre lui et l'assiégèrent.

\par 7 Et Alexandre sortit contre eux, les engagea et tua un grand nombre de leurs hommes.

\par 8 Et Marc, appelé Antoine, marcha contre lui et le força de nouveau à s'enfuir à Alexandrie.

\par 9 Et la mère d'Alexandre sortit vers Gabinius, dénigrant sa colère et le suppliant d'accorder la vie à son fils Alexandre.

\par 10 à qui Gabinius consentit sur ce point ; et Alexandre sortit vers lui ; et Gabinius le fit mourir ; et il jugea approprié de diviser les territoires de Juda en cinq parties.

\par 11 L'un est le pays de Jérusalem et ses environs ; et Hyrcan fut nommé gouverneur de cette partie. Une autre partie est Gadira et les lieux qui s'y rapportent.

\par 12 Le troisième est Jéricho et les plaines. Le quatrième est Hamath, au pays de Juda. Et le cinquième est Sephoris.

\par 13 Par ces moyens, il avait l'intention d'éloigner les guerres et les séditions du pays de Juda ; mais ils ne furent en aucun cas supprimés.

\chapter{40}

\par \textit2{L'histoire du combat d'Aristobule et de son fils Antigone de Rome, et de leur retour au pays de Juda : aussi, un récit de la mort d'Aristobule}

\par 1 Alors Aristobule conçut des projets jusqu'à ce qu'il réussisse à s'enfuir de Rome avec son fils Antigone et à arriver dans la ville de Juda.

\par 2 Et quand Aristobule se montrait en public, une grande multitude d'hommes se pressaient autour de lui ; Parmi eux, il en choisit huit mille, marcha contre Gabinius et l'engagea ; et il y eut un très grand nombre de tués par l'armée romaine :

\par 3 Il tomba aussi sept mille hommes parmi ses propres hommes, mais mille réfugièrent ; et l'armée ennemie le poursuivit ; mais lui et ceux qui lui restaient ne cessèrent de résister jusqu'à la destruction totale de ses hommes ;

\par 4 et il ne restait plus que lui seul ; Il combattit avec acharnement jusqu'à ce qu'il tombe, accablé par ses blessures, et fut pris et conduit à Gabinius. qui a ordonné qu'on le soigne jusqu'à sa guérison.

\par 5 Puis il l'envoya enchaîné à Rome.

\par [Et il resta enfermé en prison jusqu'au règne de César ; qui l'a fait sortir de prison et l'a comblé de cadeaux et de faveurs ;

\par 6 et lui donnant deux généraux et douze mille hommes, il l'envoya dans le pays de Juda, [BC 49.] pour détacher les Juifs du parti de Pompée, et les amener à obéir à César : car Pompée était alors gouverneur. du pays d'Egypte.

\par 7 Et le bruit d'Aristobule et de son groupe parvint à Hyrcan ; qui eut très peur et écrivit à Antipater pour lui détourner son pouvoir par ses artifices habituels.

\par 8 Antipater envoya donc quelques-uns des principaux de Jérusalem, donnant à l'un d'eux du poison, et le chargeant d'en administrer astucieusement à Aristobule.

\par 9 Et ils le rencontrèrent dans le pays de Syrie, comme s'ils étaient ses ambassadeurs de la ville sainte ; et il les reçut avec joie, et ils mangèrent et burent avec lui.

\par 10 Et ces hommes complotèrent jusqu'à ce qu'ils lui donnèrent le poison ; et il mourut, et fut enterré au pays de Syrie.

\par 11 La durée de son règne, jusqu'à ce qu'il soit fait prisonnier pour la première fois, fut de trois ans et demi ; et c'était un homme de courage, de poids et d'excellente disposition. ]

\par 12 Or Gabinius avait écrit au sénat pour renvoyer ses deux fils chez leur mère, comme elle l'avait demandé ; ce qu'ils ont fait.

\par 13 Mais il arriva que, lorsque Pompée s'éloigna très loin de Jérusalem, ils rompirent leur engagement d'obéissance aux Romains :

\par 14 C'est pourquoi Gabinius partit contre eux, les rencontra, les vainquit et les réduisit de nouveau à la soumission aux Romains.

\par 15 Entre-temps, le pays d'Égypte se révolta contre Ptolémée et le chassa de sa ville royale, refusant de payer tribut aux Romains.

\par 16 Sur quoi Ptolémée écrivit à Gabinius qu'il viendrait l'aider contre les Égyptiens, afin de les soumettre de nouveau aux Romains.

\par 17 Et Gabinius sortit du pays de Syrie et écrivit à Hyrcan de le rencontrer avec une armée, afin qu'ils puissent aller vers Ptolémée.

\par 18 Et Antipater se rendit avec une grande armée vers Gabinius, et le rencontra à Damas, le félicitant de la victoire qu'il avait remportée sur les Perses.

\par 19 Et Gabinius lui ordonna de se précipiter vers Ptolémée, ce qu'il fit, et combattit les Egyptiens, et en tua un très grand nombre.

\par 20 Ensuite Gabinius monta, remplaça Ptolémée sur son trône, revint dans la Ville sainte, renouvela la souveraineté d'Hyrcan et revint à Rome.


\chapter{41}

\par \t ext{L'histoire de Crassus}

\par 1 Quand Gabinius fut revenu à Rome, les Perses trompèrent les Romains ;

\par 2 Et Crassus marcha avec une grande armée en Syrie, et arriva à Jérusalem, exigeant des prêtres qu'ils lui remettaient tout l'argent qu'il y avait dans la maison de Dieu.

\par 3 À qui ils ont répondu, comment cela vous sera-t-il licite, alors que Pompée, Gabinius et d'autres l'ont jugé illégal ? Mais il répondit : je dois le faire de toute façon.

\par 4 Et Éléazar, le prêtre, lui dit : Jure-moi que tu ne porteras la main sur rien de ce qui lui appartient, et je te donnerai trois cents mines d'or.

\par 5 Et il lui jura qu'il ne prendrait rien du trésor de la maison de Dieu, s'il lui remettait ce qu'il avait mentionné.

\par 6 Et Éléazar lui donna une barre d'or ouvré, dont la partie supérieure avait été insérée dans le mur du trésor du temple, sur laquelle étaient placés chaque année les vieux voiles de la maison, de nouveaux leur étant remplacés. .

\par 7 Et le lingot pesait trois cents mines d'or, et il était recouvert des voiles qui s'étaient accumulés au cours de longues années, et n'étaient connus de personne sauf d'Éléazar.

\par 8 Crassus alors, ayant reçu cette barre, manqua sa parole, revenant sur l'accord conclu avec Éléazar ; Il prit tous les trésors du temple et pilla tout l'argent qui s'y trouvait, à hauteur de deux mille talents.

\par 9 Car cet argent avait été accumulé jusqu'à ce moment-là lors de la construction du temple, sur le butin des rois de Juda et sur leurs offrandes, et aussi sur les présents que les rois des nations avaient envoyés ;

\par 10 et ils furent multipliés et accrus au fil des années ; tout ce qu'il a pris.

\par 11 Alors l'infâme Crassus s'en alla avec l'argent et son armée au pays des Perses ; et ils le vainquirent lui et son armée au combat, les tuant en un seul jour :

\par 12 et l'armée perse prit pour butin tout ce qui était dans le camp de Crassus.

\par 13 Après cet exploit, ils marchèrent dans le pays de Syrie, qu'ils conquirent et se détachèrent de sa soumission aux Romains.

\par 14 Ce que les Romains ayant appris, envoya un général renommé, nommé Cassius, avec une grande armée : qui, arrivant dans le pays de Syrie, chassa ceux des Perses qui s'y trouvaient.

\par 15 Puis, se dirigeant vers la Ville sainte, il délivra Hyrcan de la guerre que les Juifs lui livraient et réconcilia les partis.

\par 16 Ensuite, passant l'Euphrate», il combattit les Perses et les ramena à leur soumission aux Romains :

\par 17 il réduisit aussi à la soumission les vingt-deux rois ! que Pompée avait maîtrisé ; et il réduisit sous l'obéissance aux Romains tout ce qui se trouvait dans les pays de l'Orient.

\chapter{42}

\par \textit{L'histoire de César, roi des Romains}

\par 1 On rapporte qu'il y avait à Rome une certaine femme enceinte, qui, étant sur le point d'accoucher, et déchirée par les plus violentes douleurs de l'accouchement, mourut :

\par 2 Mais comme l'enfant était en mouvement, le ventre de la mère s'ouvrit, et de là il naquit, vécut et grandit, et fut nommé Julius, parce qu'il était né au cinquième mois ; et on l'appelait César,

\par 3 parce que le ventre de sa mère, d'où il avait été extrait, a été éventré. (Lat. césa.)

\par 4 Mais lorsque l'ancien de Rome envoya Pompée en Orient, il envoya également César en Occident, pour soumettre certaines nations qui s'étaient révoltées contre les Romains.

\par 5 Et César s'en alla, les vainquit, et les réduisit à l'obéissance aux Romains, et revint à Rome avec une grande gloire.

\par 6 et sa renommée grandit, et ses affaires devinrent très renommées, et un orgueil excessif s'empara de lui ; c'est pourquoi il demanda aux Romains de le nommer roi.

\par 7 Mais l'ancien et les gouverneurs lui répondirent : En vérité, nos pères ont prêté serment du temps du roi Tarquin, qui avait pris de force la femme d'un autre homme, qui s'était emparé d'elle-même pour qu'il ne jouisse pas d'elle,

\par 8 — qu'ils ne donneraient le titre de roi à aucun de ceux qui seraient placés à la tête de leurs affaires ; C'est à cause de quel serment, dirent-ils, que nous ne pouvons pas vous satisfaire sur ce point.

\par 9 C'est pourquoi il suscita des séditions et livra de furieux combats à Rome, tuant de nombreuses personnes, jusqu'à ce qu'il s'empare du trône des Romains et s'intitule roi, mettant un diadème sur sa tête.

\par 10 Désormais, ils furent appelés rois des Romains, à cause de leur royaume ; ils furent aussi appelés Césars.

\par 11 Lorsque donc Pompée apprit la nouvelle de César, et qu'il avait tué les trois cent vingt gouverneurs, il rassembla ses armées et marcha en Cappadoce.

\par 12 Et César, allant à sa rencontre, l'attaqua, le vainquit et le tua, et s'empara de tout le territoire des Romains.

\par 13 Après cela, César se rendit dans la province de Syrie ; que Mithridate l'Arménien rencontra avec son armée, l'assurant qu'il était venu avec des desseins pacifiques et qu'il était prêt à attaquer tous les ennemis qu'il commanderait.

\par 14 César lui ordonna de partir en Egypte ; et Mithridate marcha jusqu'à ce qu'il atteigne Ascalon.

\par 15 Hyrcan craignait beaucoup César, car on savait sa soumission à Pompée, que César avait tué.

\par 16 C'est pourquoi il envoya en toute hâte Antipater avec une armée vaillante pour aider Mithridate ; et Antipater marcha vers lui et l'aida contre une certaine des villes d'Égypte, et ils s'en emparèrent.

\par 17 Mais en partant de là, ils trouvèrent une armée de Juifs qui habitaient en Égypte, qui se tenaient à l'entrée, pour empêcher Mithridate d'entrer en Égypte.

\par 18 Et Antipater leur présenta une lettre d'Hyrcan, leur ordonnant de renoncer et de ne pas s'opposer à Mithridate, l'ami de César. Et ils se sont abstenus.

\par 19 Mais les autres marchèrent jusqu'à arriver à la ville du roi alors régnant ; qui sortit vers eux avec toutes les armées des Égyptiens, et quand ils combattirent avec lui, il les vainquit et les mit en déroute ;

\par 20 et Mithridate lui tourna le dos et s'enfuit ; que, lorsqu'« il fut encerclé par les troupes égyptiennes, Antipater sauva de la mort :

\par 21 Et Antipater et ses hommes ne cessèrent de résister dans la bataille aux Égyptiens, qu'il mit en déroute et conquit, et conquit tout le pays d'Égypte.

\par 22 Et Mithridate écrivit à César, lui montrant ce qu'Antipater avait fait, et quelles batailles il avait endurées, et quelles blessures il avait reçues ;

\par 23 et que la victoire du pays ne devait pas lui être attribuée mais à Antipater, et. qu'il avait réduit les Egyptiens à l'obéissance à César.

\par 24 Et lorsque César eut lu la lettre de Mithridate, il félicita Antipater pour ses exploits, et résolut de l'avancer et de l'exalter.

\par 25 Après ces actes, Mithridate et Antipater se rendirent chez César, qui était alors à Damas ; Il obtint de César tout ce qu'il voulait, et il lui promit tout ce qu'il désirait.

\chapter{43}

\par \textit{Le récit de la venue d'Antigone, fils d'Aristobule, chez César, se plaignant d'Antipater qui avait causé la mort de son père.}

\par 1 Mais Antigone, fils d'Aristobule, vint trouver César et lui raconta l'expédition d'Aristobule, son père, pour attaquer Pompée, et combien il lui était obéissant et obséquieux.

\par 2 Puis il lui dit qu'Hyrcan et Antipater avaient secrètement envoyé un homme vers son père pour le détruire par le poison, dans l'intention (dit-il) d'aider Pompée contre vos amis.

\par 3 César envoya donc voir Antipater et l'interrogea à ce sujet ; à qui Antipater a répondu ;

\par 4 « Certainement, j'ai obéi à Pompée, parce qu'alors il était le dirigeant et il m'a conféré des avantages ; mais je n'ai pas combattu les Egyptiens à cause de Pompée, qui est déjà mort ;

\par 5 Je n'ai pas non plus éprouvé de difficultés pour les vaincre et les réduire à l'obéissance à Pompée ; mais j’ai fait cela par devoir envers César, et pour réduire Lins à lui obéir.

\par 6 Alors Antipater découvrit sa tête et ses mains, et dit : « Ces blessures, qui sont sur ma tête et sur mon corps, témoignent que mon affection et mon obéissance à César sont plus grandes que mon affection et mon obéissance à Pompée ;

\par 7 Car je ne me suis pas exposé, du temps de Pompée, aux choses auxquelles je me suis exposé du temps du roi César.

\par 8 Et César lui dit : « La paix soit avec toi et avec tous tes amis, ô le plus courageux des Juifs ; car tu as vraiment montré cette force d'âme, cette magnanimité, cette obéissance et cette affection envers nous. »

\par 9 Dès lors César grandit en affection envers Antipater, et le plaça au-dessus de tous ses amis, et le promut général de ses armées, et l'emmena avec lui dans le pays des Perses.

\par 10 et il vit à sa bravoure et à ses exploits réussis, qu'il excitait en lui de plus en plus un désir et une affection pour lui :

\par 11 enfin il le ramena au pays de Juda, couvert d'honneurs et couronné d'une position d'autorité.

\par 12 Et César marcha vers Rome, après avoir réglé les affaires d'Hyrcan ; qui a construit les murs de la ville sainte et s'est conduit envers le peuple de la manière la plus excellente :

\par 13 car c'était un homme bon, doté de vertus, d'une vie irréprochable, mais son incapacité dans les guerres était connue de tous les hommes.


\chapter{44}

\par \textit{Le récit de l'ambassade d'Hyrcan auprès de César, demandant le renouvellement du traité entre eux ; et de la copie du traité qu'Hyrcan lui a envoyé.}

\par 1 Hyrcan envoya donc des ambassadeurs à César, avec une lettre concernant le renouvellement du traité qui était entre lui et les Romains.

\par 2 Et lorsque les ambassadeurs d'Hyrcan vinrent auprès de César, il leur ordonna de s'asseoir en sa présence ; honneur qu'il n'avait conféré à aucun des ambassadeurs des rois qui venaient chez lui.

\par 3 Il les traita avec bienveillance, en accélérant leurs affaires, et ordonna qu'on réponde à la lettre d'Hyrcan. à qui il écrivit également le traité, dont ce qui suit est une copie.

\par 4 « Depuis César, roi des rois, jusqu'aux princes des Romains qui sont à Tyr et à Sidon, la paix soit avec vous.

\par 5 Je vous fais savoir qu'une lettre d'Hyrcan, fils d'Alexandre, tous deux rois des Juifs, m'a été apportée ;

\par 6 dont je me suis réjoui de l'arrivée de celui-ci, en raison de la bonne volonté constante que lui et son peuple déclarent avoir envers moi et envers la nation romaine.

\par 7 Et en vérité, j'ai prouvé par ceci la vérité de ses paroles ; qu'il envoya autrefois Antipater, capitaine des Juifs, et leur cavalerie, avec mon ami Mithridate, que les troupes égyptiennes attaquèrent ;

\par 8 et il sauva Mithridate de la mort, après nous avoir conquis le pays d'Égypte et réduit les Égyptiens à l'obéissance aux Romains ; il marcha aussi avec moi dans le pays des Perses, en tant que volontaire.

\par 9 Et c'est pourquoi j'ordonne que tous les habitants du littoral, depuis Gaza jusqu'à Sidon, paieront chaque année tous les tributs qu'ils nous doivent à la maison du grand Dieu qui est à Jérusalem ;

\par 10 sauf les citoyens de Sidon ; et que ceux-ci lui payent, selon le montant de leur tribut, vingt mille cinq cent cinquante vibes de blé chaque année.

\par 11 J'ordonne aussi que Laodicée et ses possessions, et tout ce qui était entre les mains des rois de Juda, jusqu'au bord de l'Euphrate ;

\par 12 et tous les lieux que les Asmonéens ont conquis en passant le Jourdain, seront restitués à Hyrcan, fils d'Alexandre, roi de Juda.

\par 13 Car toutes ces choses, ses pères les avaient conquises par leur épée, mais Pompée les avait injustement ravies au temps d'Aristobule.

\par 14 et dès maintenant et pour l'avenir, qu'ils appartiennent à Hyrcan et aux rois de Juda qui lui succéderont.

\par 15 Et ce traité est pour moi, et pour chacun des rois de Rome mes successeurs : quiconque donc le rompra en tout ou en partie, que Dieu le détruise par l'épée, et que sa maison et son gouvernement soient rendus désolé et être abattu!

\par 16 Et quand vous aurez lu ceci, mon épître, écrivez-la en lettres gravées sur des tables d'airain, dans la langue des Romains et en leurs caractères, et dans la langue des Grecs et en leurs caractères :

\par 17 et placez les tables dans les parties bien en vue des temples qui sont à Tyr et à Sidon ; afin que chacun puisse les voir et comprendre ce que j'ai désigné pour «Hyrcan et les Juifs».

\chapter{45}

\par \textit{L'histoire de la mort de César}

\par 1 Il y avait avec César deux amis de Pompée ; dont l'un s'appelait Cassius, et l'autre Brutus ; qui a préparé un complot pour tuer César.

\par 2 Dans quel but se sont-ils cachés dans le temple ? à Rome, qu'il s'était réservé pour prier.

\par 3 C'est pourquoi, lorsqu'il arriva, insouciant, en sécurité et sans se soucier de lui-même, ils se précipitèrent sur lui et le tuèrent.

\par 4 Et Cassius prit possession du trône, rassembla une grande armée et la transporta au-delà de la mer ; craignant le parti de César s'il devait continuer à résider à Rome.

\par 5 Et il entra dans le pays d'Asie, et le dévasta ; de là il entra dans le pays de Juda :

\par 6 et Antipater voulut l'attaquer ; mais voyant que ses forces n'étaient pas à la hauteur de la tâche, il fit la paix avec lui.

\par 7 Et Cassius imposa un tribut de sept cents talents d'or sur le pays de Juda ; Antipater se porta garant de l'argent ;

\par 8 Il chargea son fils Hérode de l'élever dans le pays de Juda et de le porter à Cassius. Celui-ci, le recevant, marcha vers le pays de Macédoine, et y resta par crainte des Romains.

\chapter{46}

\par \textit{L'histoire de la mort d'Antipater}

\par 1 Or les princes de Juda avaient délibéré pour tuer Antipater ; et à cet effet avait se| Il lui confia secrètement un homme appelé Malkiah.

\par 2 Et Malkiah fit la tentative, mais son exécution fut longtemps retardée.

\par 3 Et le bruit en parvint à Antipater, qui cherchait Malkiah pour le tuer :

\par 4 mais Malkija se disculpa aux yeux d'Antipater des choses dont il lui avait été accusé ; et lui jura que cette nouvelle était sans fondement ; et Antipater le crut, écartant de lui tout soupçon.

\par 5 Mais Malkija, ayant donné une grosse somme d'argent à l'échanson d'Hyrcan, convint avec lui de donner du poison à Antipater, pendant qu'il était sur le lit du festin, en présence du roi.

\par 6 Et l'échanson fit cela, et le roi Antipater> mourut ce même jour; et cela n'était ni le dessein ni la connaissance du roi. Et quand Antipater fut mort, Hyrcan le remplaça par Malkiah.

\chapter{47}

\par \textit{L'histoire de la mort de Malchiah}

\par 1 Or, quand Hérode, fils d'Antipater, apprit que Malkiah avait causé la mort de son père, il songea à se précipiter ouvertement sur Malkiah ; mais son frère l'en empêcha, lui conseillant de l'enlever par stratagème.

\par 2 Et Hérode alla trouver Cassius et lui raconta ce que Malkiah avait fait. Ce à quoi l'autre répondit, quand je serai parti à Tyr, et qu'Hyrcan sera avec moi, et avec lui Malkiah, alors foncez sur lui et tuez-le.

\par 3 Cassius étant parti pour Tyr, et Hyrcan étant allé le rejoindre, emmenant Malkija avec lui ; et ils se tenaient ensemble en présence de Cassius, à une certaine fête à laquelle Cassius les avait invités avec tous ses amis :

\par 4 (maintenant Cassius avait donné l'ordre à ses serviteurs de faire tout ce qu'Hérode leur ordonnerait :)

\par 5 Hérode se tenait également avec son frère parmi les compagnons d'Hyrcan, et Hérode convint avec quelques serviteurs de tuer Malkiah, lorsqu'un signal serait donné par un clin d'œil.

\par 6 Hyrcan ayant donc mangé et bu avec ses amis, ils s'endormirent l'après-midi.

\par 7 Et quand ils furent réveillés du sommeil, Hyrcan ordonna qu'on lui prépare un lit en plein air, devant l'entrée de la salle du festin dans laquelle ils avaient dormi :

\par 8 Et lui-même s'assit, et ordonna à Malkija de s'asseoir avec lui ; il ordonna aussi à Hérode et à son frère de s'asseoir :

\par 9 et les serviteurs de Cassius se tenaient près d'Hyrcan ; à qui Hérode fit un clin d'œil à Malkiah, et ils se précipitèrent aussitôt sur lui et le tuèrent :

\par 10 Hyrcan fut très effrayé et tomba dans un évanouissement.

\par 11 Mais lorsque les serviteurs de Cassius se furent retirés et que Malkiah, tué, fut emmené, Hyrcan revint à lui-même et demanda à Hérode la cause de la mort de Malkiah.

\par 12 Et Hérode répondit : « Je suis totalement ignorant et je ne connais pas non plus la « cause de la chose ». Et Hyrcan se tut et n’en demanda plus jamais davantage.

\par 13 Et Cassius marcha en Macédoine, à la rencontre d'Octave, fils du frère de César, et d'Antoine, général de son armée ; car ils étaient partis de Rome avec une grande armée à la recherche de Cassius.

\chapter{48}

\par \textit{L'histoire d'Octave (c'est Auguste, fils du frère de César) et d'Antoine, général de son armée, et de la mort de Cassius.}

\par 1 Quand Octave fut entré en Macédoine, Cassius sortit à sa rencontre et s'engagea contre lui ; et Cassius fut mis en fuite ;

\par 2 qu'Octave poursuivait, entièrement vaincu et tué : et Octave gagna le royaume à la place de son oncle César ; et il était aussi surnommé César, d'après le nom de son oncle.

\par 3 Or, lorsque Hyrcan apprit la mort de Cassius, il envoya des ambassadeurs avec des présents, de l'argent et des bijoux à Auguste et à Antoine.

\par 4 et il lui écrivit, demandant le renouvellement du traité qui avait été conclu avec César ;

\par 5 et qu'il ordonnerait que tous les captifs de Juda qui étaient dans son royaume, et ceux qui avaient été faits captifs du temps de Cassius, soient libérés ;

\par 6 et qu'il permettrait à tous les Juifs qui étaient dans le pays des Grecs et dans le pays d'Asie de retourner dans le pays de Juda,

\par 7 sans exiger aucune rançon, ni rachat, ni aucun obstacle mis sur le chemin par qui que ce soit.

\par 8 Ainsi, lorsque les ambassadeurs d'Hyrcan vinrent chez Auguste, avec leur lettre et leurs présents, il honora les ambassadeurs,

\par 9 et il accepta les présents, et accéda à tout ce qu'Hyrcan avait demandé ; lui écrivant une lettre dont voici la copie.

\par 10 « Depuis Auguste, roi des rois, et Antoine son collègue, jusqu'à Hyrcan, roi de Juda ; La santé soit avec vous.

\par 11 Votre lettre nous est déjà parvenue, ce dont nous nous sommes réjouis ; et nous avons envoyé ce que vous vouliez, concernant le renouvellement du traité et l'écriture, à toutes nos provinces, qui s'étendent depuis le pays des Indes jusqu'à l'océan occidental.

\par 12 Mais ce qui nous a retardé de vous écrire plus tôt au sujet du renouvellement du traité, c'est notre occupation à soumettre Cassius, ce sale tyran ;

\par 13 qui, agissant méchamment envers César,

\par 14 C'est pourquoi nous avons lutté contre lui de toutes nos forces, jusqu'à ce que le Dieu grand et bon nous ait rendus victorieux et l'ait fait tomber entre nos mains ;

\par 15 que nous avons mis à mort. Nous avons aussi tué Brutus, son collègue ; et nous avons délivré de sa main le pays d'Asie, après qu'il l'avait dévasté et exterminé ses habitants.

\par 16 Il n'a non plus respecté aucun engagement; ni honorer aucun temple ; ni rendre justice aux opprimés ; ni plaindre un Juif, ou tout autre de nos sujets :

\par 17 mais avec ses disciples, il fit méchamment beaucoup de mal à tous les hommes par l'oppression et la tyrannie :

\par 18 C'est pourquoi Dieu a retourné leur méchanceté contre leurs propres têtes, les livrant avec ceux qui étaient alliés avec eux.

\par 19 Réjouissez-vous donc maintenant, ô roi Hyrcan, et autres Juifs, habitants de la région sainte et prêtres qui sont dans le temple de Jérusalem :

\par 20 et qu'ils acceptent le présent que nous avons envoyé au temple le plus glorieux, et qu'ils prient pour Auguste à jamais.

\par 21 Nous avons aussi écrit à toutes nos provinces, qu'il ne reste dans aucune d'elles aucun Juif, ni serviteur ni servante, mais que tous soient relâchés sans prix et sans rançon.

\par 22 et que personne ne les empêcherait de retourner dans le pays de Juda ; et cela par ordre d'Auguste, et également d'Antoine, son collègue.

\par 23 Et il écrivit à ses amis qui sont à Tyr, à Sidon et ailleurs, de restituer tout ce qu'ils avaient pris du pays de Juda au temps de cet immonde Cassius :

\par 24 et de traiter les Juifs en paix, de ne rien opposer à eux, et de faire pour eux tout ce que César avait décrété dans son traité avec eux.

\par 25 Or Antoine resta dans le pays de Syrie ; et Cléopâtre, reine d'Egypte, vint vers lui, qu'il prit pour femme.

\par 26 C'était une femme sage, habile dans les arts magiques et les propriétés des choses : de sorte qu'elle le séduisit et s'empara de son cœur à un point tel qu'il ne pouvait rien lui refuser.

\par 27 À la même époque, cent hommes du chef des Juifs allèrent vers Antoine et se plaignirent d'Hérode et de son frère Phaselus, fils d'Antipater, en disant :

\par 28 Ils ont maintenant tout acquis à Hyrcan, et il ne lui reste plus rien du royaume, si ce n'est le nom ; et la dissimulation de cette affaire est une preuve de la captivité de leur seigneur.

\par 29 Mais Antoine ayant demandé à Hyrcan la vérité sur ce qu'ils lui avaient dit, Hyrcan déclara qu'ils parlaient faussement ; débarrassant Hérode et son frère de ce dont ils s'étaient chargés.

\par 30 Et Antoine s'en réjouit ; car il était très enclin à eux et les aimait.

\par 31 Et d'autres personnes, à une autre époque, se plaignirent auprès de lui d'Hérode et de son frère, lorsqu'il était à Tyr :

\par 32 mais non seulement il refusa d'entendre leurs paroles, mais il fit mourir quelques-uns d'entre eux et jeta les autres en prison ;

\par 33 et il rehaussa la dignité d'Hérode et de son frère, leur rendant des services, et les renvoya à Jérusalem avec un grand honneur. Mais Antoine lui-même ; Il entra dans le pays des Perses, les vainquit, les soumit et revint à Rome.

\chapter{49}

\par \textit{L'histoire d'Antigone, fils d'Aristobule, et de sa capédition contre son oncle Hyrcan, et du secours obtenu du roi des Perses.}

\par 1 Quand Auguste et Antoine furent arrivés à Rome, Antigone alla trouver le roi des Perses et lui promit mille talents d'or monnayé, et huit cents vierges des filles de Juda et de ses princes, belles et sages ;

\par 2 s'il envoyait avec lui un général menant une grande armée contre Jérusalem, et lui ordonnait de le faire roi de Juda, et qu'il faisait prisonnier son oncle Hyrcan, et qu'il tue Hérode et son frère.

\par 3 Le roi, ayant consenti, envoya avec lui un général avec une grande armée :

\par 4 et ils marchèrent jusqu'à ce qu'ils arrivèrent au pays de Syrie ; et ils tuèrent un ami d'Antoine et certains Romains qui habitaient là.

\par 5 De là ils marchèrent contre la ville sainte ; professant la sécurité et la paix, et qu'Antigone était seulement venu prier dans le sanctuaire, puis retournerait chez ses propres amis.

\par 6 Et ils entrèrent dans la ville ; Une fois arrivés, ils commencèrent à commettre des actes criminels, et commencèrent à tuer des hommes et à piller la ville, selon les ordres que leur avait donnés le roi de Perse.

\par 7 Hérode et ses hommes coururent en avant pour défendre le palais d'Hyrcan ; mais il envoya son frère et lui ordonna de garder le chemin qui mène des murs au palais.

\par 8 Et après s'être emparé de chaque position, il choisit quelques-uns de ses hommes et marcha contre les Perses qui étaient dans la ville ;

\par 9 et son frère le suivit avec un certain nombre de ses hommes ; et ils tuèrent la plupart des Perses qui étaient dans la ville, mais le reste s'enfuit hors de la ville.

\par 10 Et lorsque le général des Perses vit que les choses ne lui venaient pas à l'esprit, il envoya des messagers à Hérode et à son frère, pour traiter de paix ;

\par 11 leur annonçant que maintenant il était convaincu de leur valeur et de leur bravoure, qu'ils devaient être préférés à Antigone ; et c'est pour cette raison qu'il persuaderait ses troupes d'aider Hyrcan et eux plutôt qu'Antigonus :

\par 12 et il confirma ce vœu par les serments les plus solennels, de sorte qu'Hyrcan et Phaselus le crurent, mais non Hérode.

\par 13 Hyrcan et Phaselus, s'étant rendus chez le général des Perses, lui signifièrent qu'ils comptaient sur lui ; et il leur conseilla d'aller chez son collègue qui était à Damas ; et ils sont partis.

\par 14 Et lorsqu'ils arrivèrent chez lui, il les reçut honorablement, et fit montre de les tenir en haute estime, et les traita avec courtoisie ; bien qu'il eût secrètement donné l'ordre de les faire prisonniers.

\par 15 Et quelques-uns des principaux hommes du pays, venant vers eux, leur parlèrent de ce projet même ; leur conseillant de fuir, avec la promesse de les aider à s'échapper.

\par 16 Mais ils n'avaient pas confiance en ces hommes, craignant que ce ne soit un complot contre eux ; c'est pourquoi ils sont restés.

\par 17 Et la nuit étant venue, ils furent saisis. Phaselus s'empara en effet de lui-même ; mais Hyrcan fut enchaîné, et, sur ordre du général des Perses, on lui coupa l'oreille, afin qu'il ne puisse plus jamais être grand prêtre ;

\par 18 et il l'envoya à Hérak, vers le roi des Perses ; à qui, lorsqu'il arriva, le roi ordonna que ses chaînes soient coupées et lui témoigna de la bonté ;

\par 19 et il resta à Hérak chargé d'honneurs, jusqu'à ce qu'Hérode le réclamât au roi des Perses ; et lorsqu'il fut renvoyé vers Hérode, il lui arriva les mêmes choses qui lui étaient arrivées.

\par 20 Après cela, le général monta avec Antigone dans la Ville Sainte ; et on rapporta à Hérode ce qui avait été fait. Hyrcan et Phaselus :

\par 21 C'est pourquoi, prenant sa mère Cypris, sa femme Mariamne, fille d'Aristobule, et sa mère Alexandra, il les envoya avec des chevaux et de nombreux bagages chez Joseph, son frère, sur la montagne de Sarah.

\par 22 mais lui-même, avec une armée de mille hommes, marchait lentement et attendait ceux des Perses qui tenteraient de le poursuivre.

\par 23 Et le général des Perses le poursuivit avec son armée ; qu'Hérode a attaqué, vaincu et mis en fuite.

\par 24 Après cela, les troupes d'Antigone le poursuivirent également et combattirent avec acharnement contre lui ; il les frappa et en tua un grand nombre.

\par 25 Puis il marcha vers les montagnes de Sara ; et trouva son frère Josèphe, à qui il ordonna de sécuriser les familles dans un endroit sûr, et de leur fournir tout ce qui leur était nécessaire :

\par 26 et il leur donna beaucoup d'argent, afin qu'en cas de besoin, ils puissent s'acheter des provisions.

\par 27 Et ayant laissé ses hommes avec son frère Josèphe, il se rendit lui-même avec quelques compagnons en Égypte, afin de prendre un bateau et de se rendre au pays des Romains.

\par 28 Cléopâtre le reçut avec courtoisie et lui demanda de prendre le commandement de ses armées et la direction de toutes ses affaires ; à qui il fit savoir qu'il lui fallait absolument se rendre à Rome.

\par 29 Et elle lui donna de l'argent et des navires. Et il alla jusqu'à Rome, et demeura avec Antoine, et lui raconta ce qu'Antigone avait fait et ce qu'il avait commis contre Hyrcan et son frère, avec l'aide du roi des Pays-Bas. Perses :

\par 30 Et Antoine chevaucha avec lui chez Auguste et au Sénat, et leur dit la même chose.

\chapter{50}

\par \textit{L'histoire d'Hérode lorsque les Romains le nommèrent roi des Juifs, et son départ de Rome avec une armée pour lutter contre la Sainte Maison.}

\par 1 Avugustus et le sénat, informés de ce qu'Antigone avait fait, désignèrent d'un commun accord Hérode comme roi sur les Juifs ;

\par 2 lui ordonnant de mettre un diadème d'or sur sa tête et de monter à cheval, et qu'on proclamerait par les trompettes avant lui : « Hérode est roi des Juifs « et de la ville sainte de Jérusalem » : ce qui fut fait.

\par 3 Et retournant vers Auguste, il chevaucha, avec Auguste et Antoine ; et ils se rendirent chez Antoine, qui avait invité le sénat et tous les citoyens de Rome à un banquet qu'il avait préparé.

\par 4 Et ils mangèrent et burent, et se réjouirent d'une grande joie à propos d'Hérode, faisant avec lui un traité gravé sur des tables d'airain ; et on le plaça dans les temples.

\par 5 Et ils inscrivirent ce jour comme le premier du règne d'Hérode, et à partir de ce moment-là, il fut pris pour un zéra, par lequel les temps sont comptés.

\par 6 Après ces choses, Antoine et Hérode partirent par mer avec une armée nombreuse et abondante ; et lorsqu'ils arrivèrent à Antioche, ils divisèrent leurs forces :

\par 7 Et Antoine en prit une partie, et la conduisit dans le pays des Perses qui est Hérak » et les parties adjacentes ; et Hérode, prenant une autre partie, alla droit jusqu'à ce qu'il atteigne Ptolémaïs.

\par 8 Ainsi Antigone, apprenant qu'Antoine avait fait une expédition dans le pays des Perses et qu'Hérode était arrivé à Ptolémaïs, sortit de la Sainte Maison vers la montagne de Sarah, pour prendre Josèphe, le frère d'Hérode, et ceux qui étaient avec lui.

\par 9 Qu'il a assailli et assiégé; et ayant coupé un canal, il intercepta l'eau qui coulait jusqu'à eux ; de sorte que la soif régna parmi eux, et leurs affaires furent réduites à de grandes difficultés.

\par 10 C'est pourquoi Josèphe résolut de fuir ; et les familles avaient délibéré de se rendre à Antigone, si Josèphe devait fuir.

\par 11 Mais Dieu leur envoya une pluie abondante, qui remplit toutes leurs citernes et leurs vases : c'est pourquoi leurs cœurs furent encouragés et leur condition s'améliora ;

\par 12 et Josèphe continua à repousser Antoine et ses hommes de la place forte, et ces derniers ne purent obtenir aucun avantage sur lui.

\par 13 Mais Hérode marcha droit vers la montagne de Sara, pour ramener à Jéru son frère, ses familles et les hommes qui étaient avec lui. Salem.

\par 14 Et il trouva Antigone assiégeant son frère ; sur qui il a lancé une attaque soudaine ; Josèphe et ses hommes sortirent vers eux, et la plus grande partie de l'armée d'Antigone fut détruite, et il s'enfuit à Jérusalem.

\par 15 Qu'Hérode poursuivit avec une grande armée de Juifs, qui étaient venus vers lui de toutes parts, lorsqu'ils s'aperçurent qu'il était revenu ; et il était bien pourvu en secours, de sorte qu'il avait moins besoin de l'armée des Romains.

\par 16 Lorsqu'Hérode fut donc arrivé à la Ville Sainte, Antigone lui ferma les portes au nez ; et combattu contre lui; et ils envoyèrent beaucoup d'argent aux chefs de l'armée des Romains, leur demandant de ne pas aider Hérode : ce qu'ils firent pour lui.

\par 17 C'est pourquoi la guerre dura longtemps entre Antigone et Hérode, aucun d'eux ne l'emportant sur son semblable [c'est-à-dire l'antagoniste].

\chapter{51}

\par \textit{L'histoire de la magnanimité de certains hommes d'Hérode et de leur bravoure.}

\par 1 Or, au temps d'Antigone, les voleurs et ceux qui convoitaient le bien d'autrui se multipliaient ;

\par 2 se rendant dans certaines grottes des montagnes, auxquelles il n'y avait d'accès que pour un homme à la fois, à travers certains endroits aménagés à cet effet par eux et connus d'eux seuls :

\par 3 et même si d'autres les connaissaient, ils ne pouvaient pas monter à la grotte ; parce qu'il y avait toujours à la bouche un homme qui, avec très peu de peine, pouvait facilement repousser une personne qui grimpait.

\par 4 Et maintenant, certains de ces hommes s'étaient retrouvés seuls dans cette grotte en abondance d'armes, de provisions, de boissons, et de toutes ces choses dont ils avaient besoin ;

\par 5 avec tout le butin qu'ils avaient gagné en attaquant ceux qu'ils rencontraient, et ce qu'ils avaient pris à tort ou à raison.

\par 6 Lorsqu'Hérode eut donc appris leur procédure et constata que leurs affaires risquaient de causer des retards » ; aussi que les hommes ne pouvaient pas actuellement monter jusqu'à eux par des échelles, ni en fait grimper de quelque manière que ce soit :

\par 7 il se servait de grands coffres en bois assemblés et assemblés, et les remplissait d'hommes (ajoutant de la nourriture et de l'eau) portant de très longues lances crochues :

\par 8 et il fit descendre ces coffres du sommet des montagnes, au milieu desquelles se trouvaient les grottes, jusqu'à ce qu'ils soient placés en face de leurs bouches :

\par 9 Et quand ils étaient en face d'eux, il voulait que ses hommes les attaquent au corps à corps avec des épées, et qu'ils les traînent de loin avec ces lances.

\par 10 Et les coffres furent faits et remplis d'hommes.

\par 11 Et quand quelques-unes d'entre elles furent descendues, et se trouvèrent en face des bouches de ces grottes, aucune information n'ayant été donnée aux personnes qui y habitaient ; un des hommes qui se trouvaient dans les coffres se précipita dans les grottes, suivi de ses compagnons ;

\par 12 Ils tuèrent les voleurs qui s'y trouvaient et leurs partisans, et les jetèrent dans les vallées en contrebas. tous les hommes qu'Hérode avait envoyés, imitant ces paroles.

\par 13 Et dans cet exploit, leur courage, leur bravoure et leur audace étaient si remarquables, qu'on n'en a jamais vu de pareil : et ils ont complètement exterminé les voleurs de toutes ces régions.

\chapter{52}

\par \textit{Récit du retour d'Antoine du pays des Perses après avoir tué le roi des Perses, et de sa rencontre avec Hérode.}

\par 1 Alors Antoine, après avoir quitté Hérode », marcha d'Antioche dans le pays des Perses, et combattit le roi des Perses, le vainquit, le tua et gagna son pays ;

\par 2 et ayant réduit les Perses à l'obéissance aux Romains, il se détourna vers l'Euphrate.

\par 3 Et quand sa renommée fut racontée à Hérode, il partit pour le féliciter de sa victoire ; et de lui demander de l'accompagner dans le Pays Saint.

\par 4 Et il trouva une très grande multitude rassemblée, souhaitant s'approcher d'Antoine ; auquel de nombreux groupes d'Arabes s'étaient opposés, l'empêchant de se présenter en présence d'Antoine.

\par 5 Et Hérode marcha contre les Arabes et les tua, ouvrant un passage à tous ceux qui voulaient s'approcher d'Antoine.

\par 6 Et cela fut rapporté à Antoine, avant qu'Hérode arrivât ; sur quoi il lui envoya un diadème d'or et un grand nombre de chevaux.

\par 7 Mais quand Hérode arriva, Antoine le reçut avec courtoisie, le louant pour ses exploits contre les Arabes : et il lui attacha Sosius, le général de son armée, avec une grande force, lui ordonnant de l'accompagner dans la ville des Maison Sainte :

\par 8 lui donnant aussi des lettres pour tout le pays de Syrie, depuis Damas jusqu'à l'Euphrate, et depuis l'Euphrate jusqu'au pays d'Arménie ;

\par 9 leur disant : Auguste, roi des rois, et Antoine son collègue, et le Sénat romain, ont maintenant établi Hérode comme roi sur les Juifs ; et ils désirent que vous conduisiez tous « vos hommes de guerre avec Hérode pour l'aider : si « c'est pourquoi vous agissez contrairement à cela, vous devez nous faire la guerre ».

\par 10 Alors Antoine marcha vers le bord de la mer, et de là en Egypte ; mais Hérode et Sosius avec son armée commandaient les forces de Syrie.

\par 11 Mais quand Hérode approchait de Damas, il découvrit que « « son frère Josèphe était sorti de la Sainte Maison avec une armée de Romains, pour assiéger Jéricho et couper son blé :

\par 12 contre lequel s'avança Pappus, général des forces d'Antigone, et en tua trente mille, après avoir tué le frère d'Hérode Josèphe.

\par 13 Et lorsque sa tête fut présentée à Antigone, Phéroras, son frère, l'acheta cinq cents talents et l'enterra dans le sépulcre de ses pères.

\par 14 et il apprit aussi qu'Antigone et Pappus marchaient contre lui avec une grande armée.

\par 15 Ce qu'Hérode, ayant bien compris, résolut de se jeter sur Antigone et de l'écraser à l'improviste :

\par 16 Il convint avec Sosius qu'il prendrait douze mille Romains et vingt mille Juifs et marcherait contre Antigone, mais que les autres suivraient lentement ses traces avec le reste de l'armée.

\par 17 Et Hérode marcha en corps avec ses troupes, et rencontra Antigone dans les montagnes de la Galilée ; et ils combattirent contre lui depuis midi jusqu'à la nuit.

\par 18 Alors l'armée fut dispersée ; Hérode et quelques-uns de ses hommes passèrent la nuit dans une certaine maison, et la maison tomba sur eux ; mais ils échappèrent tous à la ruine avec leur vie, sans qu'aucun os d'eux ne soit brisé.

\par 19 Peu de temps après, Hérode se hâta de combattre Antigone, et il y eut entre eux une très grande bataille, et Antigone s'enfuit dans la Sainte Maison ; Pappus, quant à lui, résistait vaillamment et poursuivait le combat, car il était plein d'entrain et très courageux.

\par 20 Et la plus grande partie de l'armée d'Antigone fut tuée ce jour-là ; Pappus fut également tué, et Phéroras coupa la tête, et ils la portèrent à Hérode, qui ordonna de l'enterrer.

\par 21 Alors, comme il ne restait plus de l'armée d'Antigone que des prisonniers ou des fugitifs, Hérode ordonna à ses hommes de se reposer, de manger et de boire.

\par 22 Mais lui-même se rendit à un certain bain qui se trouvait dans la ville voisine, et y entra sans arme.

\par 23 Or, se cachaient dans le bain trois hommes forts et vaillants, tenant à la main des épées nues ; qui, le voyant entrer dans le bain et sans armes, se hâtèrent d'en sortir l'un après l'autre, avoir peur de lui; et ainsi il s'est échappé.

\par 24 Après cela vint Sosius ; et ils marchèrent ensemble vers la ville de la Sainte Maison, qu'ils entourèrent d'une tranchée ; et de violents combats eurent lieu entre eux et Antigone :

\par 25 et un grand nombre d'hommes de Sosius furent tués, Antigone les vainquant fréquemment ; mais il ne put les mettre en fuite, à cause de leur fermeté et de leur endurance à supporter ses assauts.

\par 26 Alors Hérode prévalut contre Antigone ; Antigone s'enfuit, et, entrant dans la ville, il ferma les portes à Hérode, et Hérode l'assiégea longtemps.

\par 27 Mais une certaine nuit, les gardes de la porte s'endormirent. Certains des hommes d'Hérode l'ayant découvert, vingt d'entre eux coururent, prirent des échelles et les placèrent contre le mur, et, en montant, tuèrent les gardes.

\par 28 Et Hérode et ses hommes se précipitèrent vers la porte de la ville qui était en face d'eux, la firent irruption et entrèrent dans la ville.

\par 29 Les Romains s'en emparèrent et commencèrent à massacrer les citoyens ; Hérode, troublé, dit à Sosius : « Si tu détruis tout mon peuple, sur qui m'établiras-tu roi ?

\par 30 et Sosius ordonna qu'on proclame que l'épée serait. séjourné; et personne n’a été tué après la proclamation.

\par 31 Mais les capitaines de Sosius, avides de proie, coururent pour piller la maison de Dieu ; mais Hérode, debout à la porte, tenant une épée nue à la main, les en empêcha ; et il envoya Sosius pour retenir ses hommes, leur promettant de l'argent.

\par 32 Et Sosius ordonna qu'on fasse proclamer à ses hommes de s'abstenir de tout pillage, et ils s'abstenirent. Et ils cherchèrent Antigone et le trouvèrent, et Antigone fut fait prisonnier.

\par 33 Après cela, Sosius se rendit en Égypte chez son collègue Antoine, portant avec lui Antigone enchaîné.

\par 34 Mais Hérode envoya à Antoine un présent très important et très beau, le priant de tuer Antigone ; et Antoine le tua ; et c'était la troisième année du règne d'Hérode, qui était aussi la troisième année d'Antigone.

\chapter{53}

\par \textit{L'histoire d'Hérode après la mort d'Antigone}

\par 1 Lorsque Hérode fut certifié de la mort d'Antigone, il se considérait sûr que personne de la famille royale d'Asmonzan ne lui disputerait :

\par 2 c'est pourquoi il s'employa à « faire progresser les dignités, en faveurs et en promotions, de ceux qui étaient bien enclins à lui et obéissaient à sa volonté ».

\par 3 Il s'efforça également de détruire ces personnes, ainsi que leurs familles, et de piller leur bétail et leurs biens, qui s'étaient opposés à lui, fournissant de l'aide contre lui.

\par 4 Et il opprima les gens, leur enlevant leurs biens, et dépouillant tous ceux qui avaient renoncé à l'obéissance aux Juifs ; il tua ceux qui lui résistaient et pilla leurs biens.

\par 5 Et il conclut un accord avec tous ceux qui lui obéiraient, qu'ils lui paieraient de l'argent.

\par 6 Il plaça également des gardes aux portes de la Sainte Maison, qui pouvaient fouiller ceux qui sortaient, et prendre tout l'or ou l'argent qu'ils trouveraient sur quelqu'un, et le lui apporter.

\par 7 Il ordonna également de fouiller les cercueils des morts ; et tout argent que quelqu'un pourrait tenter d'obtenir par stratagème, il doit le prendre.

\par 8 Et il amassa autant d'argent qu'aucun des rois de la seconde maison n'en avait amassé.

\chapter{54}

\par \textit{L'histoire d'Hyrcan, fils d'Alexandre, oncle d'Antigone, et de son retour à Jérusalem à la demande d'Hérode, et de la mort à laquelle il le fit mourir.}

\par 1 Hyrcan, après que le roi des Perses l'eut mis en liberté, resta à Hérakin, dans une condition très respectable et un grand honneur :

\par 2 C'est pourquoi Hérode craignait que quelque chose n'incitât le roi des Perses à le nommer roi et à l'envoyer dans le pays de Juda.

\par 3 C'est pourquoi, voulant se rassurer, il trama des complots pour cette affaire ; et envoya au roi des Perses un très gros présent et une lettre ;

\par 4 dans lequel il fait mention des déserts d'Hyrcan et des bonnes actions envers lui ; et comment il était allé à Rome à cause de ce que lui avait fait Antigone, le fils de son frère ;

\par 5 et qu'étant maintenant parvenu au trône, et ses affaires étant en ordre, il voulait le récompenser d'une manière convenable pour les bienfaits qu'il lui avait conférés.

\par 6 Le roi des Perses envoya donc un messager à Hyrcan, pour lui dire : « Si vous souhaitez retourner au « pays de Juda, retournez :

\par 7 mais je vous préviens de « méfiez-vous d'Hérode ; et je vous informe distinctement qu'il ne cherche pas à ce que vous vous fassiez du bien, mais que son dessein est de se mettre en sécurité, car il ne reste plus personne qu'il craint, sauf vous. C'est pourquoi, prenez garde à lui. avec la plus grande diligence, et ne vous laissez pas entraîner dans un piège.

\par 8 Les Juifs de Babylone vinrent aussi vers lui et lui dirent des paroles pareilles. On lui dit encore :

\par 9 « Tu es maintenant un vieil homme, et tu n'es pas apte à exercer la charge de grand prêtre, à cause de la souillure que ton neveu t'a infligée :

\par 10 mais Hérode est un homme mauvais et un homme qui verse le sang ; et il ne vous rappelle que parce qu'il vous craint ; et tu ne manques de rien parmi nous, et tu es avec nous dans la position où tu devrais être.

\par 11 Et ta famille y est dans le meilleur état ; c'est pourquoi reste avec nous, et n'aide pas ton ennemi contre toi-même.

\par 12 Mais Hyrcan n'acquiesça pas à leurs paroles ; ni écouté les conseils de celui qui le conseillait bien.

\par 13 Et il partit et voyagea jusqu'à ce qu'il arrivât dans la Ville Sainte, à cause du très grand désir qu'il avait envers la maison de Dieu, sa famille et son pays.

\par 14 Et lorsqu'il fut approché de la ville, Hérode le rencontra, faisant preuve d'un tel honneur et d'une telle magnificence, qu'Hyrcan fut trompé et se confia en lui.

\par 15 Et Hérode, dans l'assemblée publique et devant ses propres amis, l'appelait «Père», mais néanmoins il ne cessa pas de méditer des complots dans son cœur, uniquement pour qu'ils ne lui soient pas imputés.

\par 16 C'est pourquoi Alexandra et Mariamne, sa fille, se rendent chez Hyrcan, lui faisant craindre Hérode et lui conseillant de prendre soin de lui-même ;

\par 17 mais il ne s'occupa pas non plus d'eux, bien qu'ils lui répétaient cela encore et encore, lui conseillant de fuir vers l'un des rois des Arabes :

\par 18 mais il ne s'occupa pas de toutes ces choses, jusqu'à ce qu'ils l'y poussent par des avertissements et des alarmes répétés.

\par 19 Alors donc, il écrivit à ce roi d'Arabie : et ayant fait venir un certain homme (dont Hérode avait tué le frère, et avait confisqué ses biens et lui avait rendu de nombreux maux), il lui dit qu'il souhaitait lui confier un certain secret, l'adjurant de ne pas le dire. à qui que ce soit;

\par 20 et lui donnant de l'argent et la lettre au roi des Arabes, il lui communiqua ce qu'il demandait dans la lettre.

\par 21 Ainsi le messager, après avoir reçu la lettre, pensa qu'il obtiendrait un poste élevé auprès d'Hérode, et qu'il éloignerait de lui le mal qu'il craignait continuellement de sa part, s'il communiquait l'affaire à Hérode ;

\par 22 et que cela lui serait plus profitable que de garder le secret d'Hyrcan : puisque dans l'autre cas il n'était pas en sécurité, et sûr que la chose ne serait pas révélée à Hérode à un moment ou à un autre, et ainsi être la cause de sa destruction.

\par 23 Il porta donc la lettre à Hérode, et lui dévoila toute l'affaire. Celui-ci lui dit : Portez la lettre telle qu'elle est au roi des Arabes, et rapportez-moi sa réponse, afin que je sache. il:

\par 24 Dis-moi aussi où seront les hommes que le roi des Arabes enverra, pour qu'Hyrcan reparte avec eux.

\par 25 Le messager s'en alla donc et porta la lettre d'Hyrcan au roi des Arabes. qui s'est réjoui et a envoyé certains de ses hommes;

\par 26 leur ordonnant de se rendre dans un lieu proche de la ville sainte, et d'y attendre qu'Hyrcan vienne à eux ; puis d'assister Hyrcan jusqu'à ce qu'ils l'amènent en sa présence.

\par 27 Il écrivit également à Hyrcan une réponse à sa lettre, et l'envoya par le messager.

\par 28 Les hommes se rendirent donc avec le messager au lieu désigné, et y attendirent. Mais le messager porta la lettre à Hérode, qui en apprit le contenu. Il lui indiqua aussi le lieu des hommes auxquels Hérode envoya des personnes pour les amener. eux.

\par 29 Ensuite, ayant envoyé chercher soixante-dix vieillards parmi les anciens des Juifs, et ayant aussi fait venir Hyrcan ; Quand il fut arrivé, il lui dit : Y a-t-il un échange de lettres entre toi et le roi des Arabes ?

\par 30 Et Hyrcan dit : Non. Alors il lui dit : L'as-tu envoyé pour fuir vers lui ? et il a dit : Non.

\par 31 Et Hérode ordonna à son messager d'avancer, ainsi qu'aux Arabes et aux chevaux ; il apporta également la réponse à sa lettre, et elle fut lue.

\par 32 Puis il ordonna de couper la tête d'Hyrcan ; et sa tête fut arrachée, et personne n'osa dire un mot en sa faveur.

\par 33 Or, Hyrcan avait délivré Hérode de la mort qui lui avait été justement accordée dans l'assemblée du jugement, en ordonnant que l'assemblée soit reportée au lendemain, et en renvoyant Hérode la nuit même.

\par 34 D'où il était destiné à devenir son meurtrier, quels que soient les services rendus à lui et à son père.

\par 35 Hyrcan fut mis à mort à l'âge de quatre-vingts ans, et il régna quarante ans. Et il n'y eut aucun roi de la race d'Asmonzan qui ait une conduite plus louable et une manière de vivre plus honorable.

\chapter{55}

\par \textit{L'histoire d'Aristobule, fils d'Hyrcan}

\par 1 Aristobule, fils d'Hyrcan, était d'une telle beauté de forme, d'une figure et d'un esprit si exquis, qu'on ne connaissait pas son égal.

\par 2 Sa sœur Mariamne, la femme d'Hérode, lui ressemblait aussi en beauté ; et Hérode lui était merveilleusement attaché.

\par 3 Mais Hérode était réticent à nommer Aristobule grand prêtre à la place de son père ; de peur que les Juifs, attachés à lui par leur affection pour son père, ne le fassent plus tard roi.

\par 4 C'est pourquoi il établit comme grand prêtre quelqu'un du nombre des prêtres ordinaires, qui n'était pas de la famille des Asmonéens.

\par 5 Ce dont Alexandra, la mère d'Aristobule, mécontente, écrivit à Cléopâtre ; demandant une lettre d'Antoine à Hérode, lui demandant de destituer le prêtre qu'il avait élevé et de nommer son fils Aristobule grand prêtre à sa place.

\par 6 Et Cléopâtre accorda cela ; et demanda à Antoine d'écrire une lettre à Hérode à ce sujet, et de l'envoyer par quelque chef de ses serviteurs.

\par 7 Ainsi Antoine écrivit une lettre et l'envoya. par son serviteur Gellius ; et Gellius venant trouver Hérode, lui remit la lettre d'Antoine.

\par 8 Mais Hérode s'abstint de faire ce qu'Antoine avait écrit par ordre, affirmant que ce n'était pas l'usage parmi les Juifs de destituer un prêtre de son poste.

\par 9 Or il arriva que Gellius vit Aristobule, et fut grandement frappé de la beauté de sa forme et de la perfection de son port, qu'il vit.

\par 10 C'est pourquoi il peignit un tableau à son image, et l'envoya à Antoine, en écrivant sous le tableau à cet effet ; qu'aucun homme n'avait engendré Aristobule, mais qu'un ange cohabitant avec Alexandra l'avait engendré sur elle.

\par 11 C'est pourquoi, lorsque le tableau parvint à Antoine, celui-ci fut saisi du désir le plus véhément de voir Aristobule.

\par 12 Et il écrivit une lettre à Hérode, lui rappelant comment il l'avait établi roi et comment il l'avait secouru contre ses ennemis, racontant ses bontés envers lui :

\par 13 ajoutant une demande, qu'il lui enverrait Aristobule ; et il l'a menacé dans cette affaire pour les paroles qu'il lui avait renvoyées.

\par 14 Mais quand l'épître d'Antoine fut apportée à Hérode, il refusa d'envoyer Aristobule, sachant ce qu'Antoine avait prévu ; et c'est pour cette raison qu'il dédaigna de le faire : et il déposa en toute hâte le grand prêtre qu'il avait nommé, établissant Aristobule à sa place.

\par 15 Et il écrivit alors à Antoine, l'informant qu'il avait déjà exécuté ce qu'il lui avait écrit auparavant, au sujet de la mise d'Aristobule à la place de son père, avant que sa dernière lettre n'arrive :

\par 16 quelle affaire il avait retardée à ce moment-là, parce qu'il était nécessaire d'en débattre avec les prêtres et les Juifs après un intervalle de quelques jours, car la chose était inhabituelle ; mais cela s'étant passé selon son désir, il l'avait immédiatement nommé.

\par 17 Mais maintenant qu'il était nommé, il ne lui était plus permis de sortir de Jérusalem ; comme il n'était pas roi, mais prêtre attaché au service du temple :

\par 18 et chaque fois qu'il voulait le contraindre à sortir, les Juifs refusaient et ne le permettaient pas, même s'il tuait la plupart d'entre eux.

\par 19 C'est pourquoi, lorsque la lettre d'Hérode parvint à Antoine, celui-ci ne demanda pas Aristobule ; et Aristobule fut nommé grand prêtre.

\par 20 Puis vint la fête des tabernacles ; Et les hommes, assemblés devant la maison de Dieu, virent Aristobule vêtu des robes sacerdotales, debout près de l'autel, et ils l'entendirent les bénir :

\par 21 et il plut tellement aux hommes, qu'ils lui montrèrent leur affection d'une manière très marquée.

\par 22 Ce dont Hérode, étant pleinement informé, fut très attristé ; et il craignit que, lorsque le parti d'Aristobule se renforcerait, il ne lui demande le royaume, si sa vie devait se prolonger. C'est pourquoi il commença à comploter sa mort.

\par 23 Or, après la fête des tabernacles, les rois avaient l'habitude de se rendre dans certaines résidences de plaisir à Jéricho que les anciens rois avaient faites.

\par 24 et il y a de nombreux jardins contigus les uns aux autres, dans lesquels se trouvaient des étangs à poissons larges et profonds, vers lesquels ils avaient conduit des ruisseaux d'eau, et avaient érigé de beaux bâtiments dans ces jardins ; ils avaient aussi bâti à Jéricho de beaux palais et de beaux palais. édifices.

\par 25 Or, l'auteur du livre raconte que les baumiers poussaient en abondance à Jéricho ; et qu'ils n'ont été trouvés nulle part ailleurs que là-bas ; et que de nombreux rois les avaient transportés de là dans leur propre pays, mais aucun ne grandit, sauf ceux qui furent transportés en Egypte ;

\par 26 et qu'ils n'échouèrent à Jéricho qu'après la destruction de la seconde Maison ; mais ensuite ils se sont desséchés et n'ont plus jamais repoussé.

\par 27 Hérode partit donc à Jéricho en quête de plaisir, et Aristobule le suivit.

\par 28 Et lorsqu'ils arrivèrent à Jéricho, Hérode ordonna à quelques-uns de ses serviteurs de descendre dans les étangs à poissons et de jouer comme c'était l'habitude ; et que si Aristobule descendait vers eux, ils joueraient avec lui pendant quelque temps, puis le noyer.

\par 29 Mais Hérode était assis dans une salle de festin qu'il s'était préparée pour lui-même. Hérode fit appeler Aristobule et le fit asseoir à ses côtés. Le chef de ses serviteurs et de ses amis était également assis en sa présence.

\par 30 et il ordonna qu'on apporte de la nourriture et des boissons ; et ils mangèrent et burent ; et les serviteurs se précipitèrent vers les eaux, selon la coutume, et s'amusèrent.

\par 31 Et Aristobule désirait grandement descendre avec eux dans l'eau, le vin les maîtrisant maintenant, et il demanda la permission à Hérode de le faire. Celui-ci répondit :

\par 32 Cela ne convient ni à toi, ni à personne comme toi. Et lorsqu'il était pressé, il le réprimandait et le lui défendait. Mais Aristobule, lui réitérant sa demande, lui dit : Fais ce qu'il te plaît.

\par 33 Et alors Hérode, se levant, se rendit dans un certain palais pour y dormir.

\par 34 Et Aristobule descendit aux eaux et joua longtemps avec les serviteurs. Ceux-ci, voyant qu'étant maintenant fatigué et las, il voulait remonter, le retinrent sous l'eau, le tuèrent et le portèrent. mort.

\par 35 Et il y eut un grand tumulte du peuple, et des cris, et une lamentation fut élevée.

\par 36 Et Hérode accourut et sortit pour voir ce qui était arrivé. Celui-ci, voyant Aristobule mort, le lamenta et pleura sur lui très tendrement avec un flot de larmes très véhémentes.

\par 37 Puis il ordonna qu'on le transporte dans la ville sainte, et il l'accompagna jusqu'à ce qu'il vienne dans la ville, et il força le peuple à assister à ses funérailles, et il n'y eut aucun point du plus grand honneur qu'il omet de lui payer. .

\par 38 Et il mourut quand il avait seize ans, et son sacerdoce ne dura que quelques jours.

\par 39 C'est pour cette raison que l'inimitié grandit entre sa mère Alexandra et sa fille Mariamne, la femme d'Hérode, et la mère et la sœur d'Hérode.

\par 40 Et les exécrations et les injures dont Mariamne les accablait étaient connues ; et bien que ceux-ci soient parvenus à Hérode, il ne lui a pas interdit ni réprimandé, à cause de sa grande affection pour elle :

\par 41 il craignait aussi qu'elle ne s'imaginât qu'il était bien disposé envers les autres : c'est pourquoi ces liaisons durèrent longtemps entre ces femmes.

\par 42 Et la sœur d'Hérode, qui était douée de la plus grande méchanceté et d'un artifice consommé, commença à comploter contre Mariamne :

\par 43 mais Mariamne était religieuse, droite, modeste et vertueuse : mais elle était un peu teintée de hauteur, d'orgueil et de haine envers son mari.

\chapter{56}

\par \textit{L'histoire d'Antoine, et de son expédition contre Auguste, et du secours qu'il demanda à Hérode. Et un récit du tremblement de terre qui s'est produit dans le pays de Juda, et de la bataille qui a eu lieu entre eux et les Arabes.}

\par 1 Cléopâtre, reine d'Egypte, était la femme d'Antoine ; et elle découvrit elle-même de telles méthodes de parure et de peinture, par lesquelles les femmes ont coutume de séduire les hommes, comme aucune autre femme au monde ne l'avait découvert :

\par 2 de sorte que, bien qu'elle fût une femme avancée en âge, elle paraissait comme une petite fille célibataire, et encore plus délicate et plus blonde.

\par 3 Antoine trouva aussi en elle ces méthodes de beauté et ces moyens de créer du plaisir, qu'il n'avait jamais trouvés chez le grand nombre de femmes dont il avait joui. C'est pourquoi elle s'empara si complètement du cœur d'Antoine, qu'il n'y resta plus de place pour l'affection envers autrui.

\par 4 Elle le persuada donc de déconcerter certains rois soumis aux Romains, d'après ses propres considérations privées ; et il lui obéit en cela, mettant à mort certains rois sur sa demande ; et il en laissa quelques-uns vivants sur ses ordres, faisant d'eux ses serviteurs et ses esclaves.

\par 5 Et cela fut dit à Auguste ; qui lui écrivit, abominant une telle conduite et désirant qu'il ne se rende plus coupable d'une telle conduite.

\par 6 Et Antoine raconta à Cléopâtre ce qu'Auguste lui avait écrit ; et elle lui conseilla de se révolter contre Auguste, et lui montra que la chose était très facile.

\par 7 À l'opinion de qui il souscrivait, il a ouvertement joué faux avec Auguste ; et il rassembla une armée et des provisions, afin qu'il puisse se rendre par mer à Antioche, et de là marcher par terre à la rencontre d'Auguste partout où il pourrait le trouver.

\par 8 Il envoya aussi chercher Hérode, pour qu'il l'accompagne. Et Hérode se rendit vers lui avec une armée très puissante et des provisions très complètes.

\par 9 Et quand il fut venu vers lui, Antoine lui dit : La bonne raison nous conseille de faire une expédition contre les Arabes et d'engager un combat avec eux : car nous ne sommes en aucun cas sûrs qu'ils ne feront pas une incursion contre les Juifs et le pays d'Égypte, dès que nous aurons tourné le dos. .

\par 10 Et Antoine partit par mer ; mais Hérode fit une incursion chez les Arabes ; et Cléopâtre envoya un général nommé Athénio avec une grande armée, pour aider Hérode à soumettre les Arabes.

\par 11 et elle lui ordonna de placer Hérode et ses hommes au premier rang, et de s'entendre avec le roi des Arabes, pour qu'ils enfermeraient ensemble Hérode et mettraient en pièces ses hommes.

\par 12 À cela, elle était conduite par le désir de prendre possession de tout ce que valait Hérode :

\par 13 Alexandra lui avait aussi demandé, quelque temps auparavant, d'inciter Antoine à mettre à mort Hérode ; ce qu'elle avait effectivement fait, mais Antoine refusa de commettre cet acte.

\par 14 A cela s'ajoutait le fait que Cléopâtre avait autrefois désiré Hérode, et avait à un moment donné désiré avoir des relations sexuelles avec lui ; mais il se retint, car il était chaste. Et c'étaient là les causes qui l'avaient poussée à cette ligne de conduite.

\par 15 Alors Athénio, venant vers Hérode, selon l'ordre de Cléopâtre, envoya se mettre d'accord avec le roi des Arabes, afin qu'il l'entoure.

\par 16 Et quand Hérode et ses Arabes se rencontrèrent et se rencontrèrent, Athénio et ses hommes attaquèrent Hérode, qui fut intercepté entre les deux armées, et la bataille devint féroce contre lui, tant devant que derrière.

\par 17 Mais Hérode, voyant ce qui était arrivé, rassembla ses hommes et combattit avec la plus grande vigueur jusqu'à ce qu'ils soient hors de portée des deux armées, après le plus grand effort ; et il rentra dans la Sainte Maison.

\par 18 Et il se produisit un grand tremblement de terre dans le pays de Juda, tel qu'il n'y en avait pas eu depuis le temps du roi Harbah, dans lequel un grand nombre d'hommes et de bêtes furent détruits.

\par 19 Et cela effraya beaucoup Hérode, lui causa une grande frayeur et lui fit perdre le moral. Il prit donc conseil avec les anciens de Juda pour conclure un accord avec toutes les nations qui l'entouraient ; concevoir la paix et la tranquillité, ainsi que la suppression des guerres et des effusions de sang.

\par 20 Il envoya également des ambassadeurs sur ces questions auprès des nations environnantes, qui embrassèrent toutes la paix à laquelle il les avait invités, à l'exception du roi des Arabes ;

\par 21 qui ordonna de mettre à mort les ambassadeurs qu'Hérode lui avait envoyés ; car il supposait qu'Hérode avait fait cela parce que ses hommes avaient été détruits dans le tremblement de terre, et par conséquent, étant affaibli, il s'était tourné vers la paix.

\par 22 C'est pourquoi il résolut d'entrer en guerre contre Hérode ; et ayant rassemblé une armée nombreuse et bien fournie, il marcha contre lui.

\par 23 Et cela fut raconté à Hérode : et il était très contrarié, pour deux raisons : l'une, à cause du massacre de ses ambassadeurs, acte qu'aucun des rois n'avait commis jusqu'alors ; un autre, parce qu'il avait osé l'attaquer, imaginant sa faiblesse et son manque de troupes.

\par 24 Mais il voulait lui montrer qu'il en était autrement, afin que tous ceux à qui il avait envoyé des ambassadeurs pour traiter de paix sachent qu'il n'avait pas fait cela par crainte ou par faiblesse, mais par désir de cela. ce qui était gentil et bon ; que personne n'oserait faire d'attentats contre les Juifs, ni imaginer dans son esprit qu'ils étaient faibles.

\par 25 En outre, il voulait se venger du roi des Arabes à cause de ses ambassadeurs : c'est pour cela qu'il résolut de marcher en toute hâte contre lui.

\par 26 C'est pourquoi il rassembla des troupes du pays de Juda et leur dit : « Vous êtes au courant du « massacre de nos ambassadeurs perpétré par » cet Arabe ; un acte qu’aucun roi n’a jusqu’à présent « commis :

\par 27 car il pense que nous avons été affaiblis et sommes devenus impuissants ; et il a osé nous provoquer, et pense qu'il obtiendra tous ses désirs sur nous : et il ne cessera pas non plus de nous faire continuellement la guerre.

\par 28 C'est pourquoi vous devez lutter contre les difficultés, afin de montrer votre bravoure et de « soumettre vos ennemis et d'emporter leur butin » :

\par 29 bien que la fortune puisse se montrer tantôt favorable, tantôt défavorable, selon l'usage et les vicissitudes habituelles de ce monde.

\par 30 En vérité, vous devez immédiatement entreprendre une expédition, pour vous venger de ces oppresseurs, et réprimer l'audace de tous ceux qui vous tiennent en peu d'estime.

\par 31 Mais si vous dites : ce tremblement de terre nous a découragés et a détruit un grand nombre d'entre nous ; vous savez bien qu'elle n'a détruit aucun des combattants, mais quelques autres.

\par 32 Nous ne devrions pas non plus penser qu'il est déraisonnable du tout qu'il ait détruit les pires de notre nation, mais ait laissé les meilleurs survivre. Il ne fait aucun doute également que cela a amélioré votre moral et vos sentiments intérieurs.

\par 33 Mais le devoir de celui que Dieu a sauvé de la destruction et préservé de la ruine, exige qu'il lui obéisse et qu'il fasse ce qui est bon et juste.

\par 34 Et en vérité, aucune obéissance n'est plus honorable ni plus glorieuse que de chercher réparation pour l'opprimé auprès de l'oppresseur ; et pour soumettre les ennemis de Dieu, de sa religion et de sa nation, en aidant ceux qui lui montrent obéissance et attention.

\par 35 Vous ne savez pas non plus ce qui est arrivé récemment à ces Arabes, lorsqu'ils nous avaient entourés d'Athénio ; et comment le Dieu grand et bon nous a aidés contre eux et nous en a délivrés.

\par 36 Craignez donc Dieu, suivant votre ancienne coutume et la louable coutume de vos ancêtres ; et préparez-vous contre cet ennemi avant qu'il ne se prépare contre vous, et soyez en avance avec lui avant qu'il ne vous anticipe : et Dieu vous fournira aide et secours contre votre ennemi.

\par 37 Ainsi, après que les hommes eurent entendu le discours d'Hérode, ils répondirent qu'ils étaient prêts à entreprendre l'expédition et qu'ils ne tarderaient pas.

\par 38 Et il rendit grâce à Dieu et à eux pour cela, et ordonna d'offrir de nombreux sacrifices ; il ordonna aussi de lever une armée ; et une grande multitude se rassembla de la tribu de Juda et de Benjamin.

\par 39 Et Hérode, marchant contre le roi des Arabes, le rencontra ; et la bataille devint féroce entre eux, cinq mille Arabes étant tués.

\par 40 Il y eut encore une bataille, et quatre mille Arabes furent tués. C'est pourquoi les Arabes retournèrent à leur camp et y restèrent ; et Hérode ne pouvait rien faire contre eux, car la place était fortifiée ; mais il resta avec son armée, les assiégeant au même endroit, et ne leur permettant pas de sortir.

\par 41 Et ils restèrent cinq jours dans cet état ; et une soif très violente les saisit ; ils envoyèrent donc des ambassadeurs à Hérode avec un présent des plus précieux, demandant une trêve et la liberté de puiser de l'eau à boire : mais il ne les écouta pas, mais continua dans la même furieuse hostilité.

\par 42 Les Arabes dirent donc : Sortons contre cette nation ; car il vaut mieux vaincre ou mourir que de périr de soif.

\par 43 Et ils sortirent contre eux ; et le parti d'Hérode les vainquit et en tua neuf mille ; Hérode et ses hommes poursuivirent les Arabes alors qu'ils s'enfuyaient, en tuant un grand nombre ; et il assiégea leurs villes et les prit.

\par 44 C'est pourquoi ils demandèrent pour leur vie, promettant l'obéissance ; ce à quoi il accepta, se retira d'eux et retourna dans la Sainte Maison.

\par 45 Or, les Arabes mentionnés dans ce livre sont les Arabes qui habitaient depuis le pays de Sara jusqu'à Hégiaz et les régions adjacentes ; et ils étaient d'une grande renommée et d'un grand nombre.


\chapter{57}

\par \textit{L'histoire de la bataille d'Antoine contre Auguste, de la mort d'Antoine et du départ d'Hérode vers Auguste.}

\par 1 Lorsqu'Antoine sortit d'Égypte pour entrer dans le pays des Romains et rencontra Auguste, il y eut entre eux des batailles très sévères, dans lesquelles la victoire fut du côté d'Auguste, et Antoine tomba au combat ;

\par 2 et Auguste prit possession de son camp et de tout ce qui s'y trouvait. Après cela, il se rendit à Rhodes, afin de s'y embarquer pour passer en Égypte.

\par 3 Et la nouvelle fut rapportée à Hérode, et il fut très inquiet de la mort d'Antoine ; et il craignait extrêmement Auguste ; et il résolut d'aller vers lui, de le saluer et de le féliciter.

\par 4 C'est pourquoi il envoya sa mère et sa sœur avec son frère dans une forteresse qu'il avait sur la montagne de Sarah ; il envoya aussi sa femme Mariamne et sa mère Alexandra à Alexandrie, sous la garde de Josèphe, un Tyrien ; l'adjurant de tuer sa femme et sa mère, dès que sa mort lui serait signalée.

\par 5 Après cela, il se rendit chez Auguste avec un présent très précieux. Auguste avait déjà décidé de faire mourir Hérode ;

\par 6 parce qu'il avait été l'ami et le partisan d'Antoine, et parce qu'il avait auparavant délibéré de marcher avec Antoine pour l'attaquer.

\par 7 Lorsque donc l'arrivée d'Hérode fut annoncée à Auguste, il l'ordonna de se présenter devant lui, dans l'habit royal qu'il portait ; sauf le diadème, pour lequel il avait ordonné qu'on le retire de sa tête.

\par 8 Celui-ci, étant en sa présence, après avoir déposé son diadème, comme Auguste l'avait ordonné, dit :

\par 9 « Ô roi, peut-être à cause de mon amour envers Antoine tu as été si violemment irrité contre moi, que tu as ôté le diadème de ma tête ;

\par 10 ou était-ce dû à une autre cause ? Car, « si vous m'en voulez à cause de mon adhésion à Antoine, en vérité, dis-je, je lui ai adhéré parce qu'il méritait bien de moi, et j'ai mis sur ma tête ce diadème que vous avez ôté.

\par 11 Et en effet, il avait demandé mon secours contre vous, ce que je lui ai accordé ; même « comme il m'a aussi souvent apporté son aide :

\par 12 Mais ce n'était pas mon sort d'assister à la bataille qu'il a livrée contre vous, et je n'ai pas non plus tiré mon épée contre vous, ni combattu ; la cause en était que j'étais occupé à soumettre les Arabes.

\par 13 Mais je n'ai jamais manqué de lui fournir le secours en hommes, en armes et en provisions, comme l'exigeaient son amitié et ses bonnes actions envers moi. Et en vérité, je suis désolé de l'avoir quitté ; de peur que les hommes ne croient que j’ai abandonné mon ami alors qu’il avait besoin de mon aide.

\par 14 Certes, si j'avais été avec lui, je l'aurais secouru de toutes mes forces ; et il l'aurait encouragé s'il avait eu peur, et l'aurait fortifié s'il avait été affaibli, et l'aurait relevé s'il était tombé, jusqu'à ce que Dieu ait gouverné les choses à sa guise.

\par 15 Et cela m'aurait vraiment été moins pénible que de croire que j'avais laissé tomber un homme qui avait imploré mon secours, et qu'ainsi il arriverait que mon amitié serait peu estimée.

\par 16 À mon avis, il est en effet tombé à cause de sa propre mauvaise politique, en cédant à cette enchanteresse Cléopâtre ; que je lui avais conseillé de tuer, et ainsi de lui ôter sa méchanceté ; mais il n'y consentit pas.

\par 17 Mais maintenant, si vous avez ôté de ma tête le diadème, certainement vous ne m'enlèverez pas mon intelligence et mon courage ; et quoi que je sois, je serai l’ami de mes amis et l’ennemi de mes ennemis.

\par 18 Auguste lui répondit : « En effet, nous avons vaincu Antoine par nos troupes ; mais nous vous maîtriserons en vous attirant à nous ; et veillerai, par nos bons offices envers vous, à ce que votre affection pour nous soit doublée, parce que vous en êtes digne.

\par 19 Et comme Antoine a trompé le conseil de Cléopâtre, pour la même raison il s'est comporté avec ingratitude envers nous ; nous revenons pour nos bienfaits, pour nos maux, et pour nos faveurs, pour la rébellion.

\par 20 Mais nous nous réjouissons de la guerre que vous avez menée contre les Arabes, qui sont nos ennemis ; car quiconque est votre ennemi est aussi le nôtre ; et quiconque vous rend obéissance, nous le rendra également.

\par 21 Alors Auguste ordonna qu'on mette le diadème d'or sur la tête d'Hérode, et qu'on lui ajoute autant de provinces qu'il en avait déjà.

\par 22 Et Hérode accompagna Auguste en Egypte ; et tout ce qu'Antoine avait destiné à Cléopâtre lui fut rendu. Auguste partit pour Rome ; mais Hérode retourna dans la Ville sainte.

\chapter{58}

\par \textit{L'histoire du meurtre qu'Hérode a commis sur sa femme Mariamne.}

\par 1 Or Josèphe, le mari de la sœur d'Hérode, avait révélé à Mariamne qu'Hérode lui avait ordonné de la mettre à mort, elle et sa mère, dès qu'il périrait lui-même en montant vers Auguste.

\par 2 Et elle avait déjà de l'aversion pour Hérode, depuis le temps où il tua son père et son frère ; et à cela s'ajouta un peu de haine, lorsqu'elle fut informée des ordres qu'il avait donnés contre elle.

\par 3 C'est pourquoi, quand Hérode arriva hors d'Egypte, il la trouva totalement envahie par la haine envers lui : ce qui, très troublé, essaya de la réconcilier avec lui par tous les moyens possibles.

\par 4 Mais sa sœur vint un certain jour, après quelques disputes qui avaient eu lieu entre elle et Mariamne, et lui dit : Certainement Joseph, mon mari, est parti avec Mariamne.

\par 5 Mais Hérode ne prêta aucune attention à ses paroles, sachant combien Mariamne était pure et chaste.

\par 6 Après cela, Hérode alla voir Mariamne la nuit qui suivit ce jour-là, et se comporta avec elle avec bonté et affection, lui racontant son amour pour elle, disant beaucoup de choses à ce sujet :

\par 7 à qui elle dit : « As-tu déjà vu un homme en aimer un autre et lui ordonner de le faire mourir ? et est-il un haineux à moins qu'il ne montre de telles preuves ?

\par 8 Alors Hérode s'aperçut que Josèphe avait découvert à Mariamne le secret qu'il lui avait confié ; et croyait qu'il ne l'aurait pas fait, si elle ne s'était livrée à lui :

\par 9 et il crut ce que sa sœur lui avait dit à ce sujet ; et aussitôt il quitta Mariamne, il la haït et la détesta.

\par 10 Ce que sa sœur apprenant, alla trouver l'échanson, lui donna de l'argent, lui apporta du poison et dit : Apportez ceci au roi, et dites-lui : Mariamne, la femme du roi, m'a donné ce poison et cet argent, en ordonnant qu'on les mélange à la boisson du roi.

\par 11 C'est ce que fit l'échanson. Et le roi, voyant le poison, ne douta pas de la vérité de la chose : sur quoi il donna l'ordre de décapiter immédiatement Josèphe, son beau-frère ; et il ordonne également que Mariamne soit enchaînée jusqu'à ce que les soixante-dix anciens soient présents et qu'ils lui rendent une sentence appropriée.

\par 12 La sœur d'Hérode craignit donc que ce qu'elle avait fait ne soit découvert et qu'elle ne périsse elle-même si Mariamne était libérée. Elle lui dit donc : Ô roi, si tu remets à demain la mort de Mariamne, tu ne pourra pas du tout l’effectuer :

\par 13 car dès qu'on saura que vous voulez la tuer, toute la maison de son père viendra, et tous leurs serviteurs et voisins, et interviendront ; et vous ne pourrez obtenir sa mort qu'après de grands tumultes.

\par 14 Et Hérode dit : Fais ce qui te semble le mieux.

\par 15 Et la sœur d'Hérode envoya en toute hâte un homme pour amener Mariamne au lieu de massacre, attaquant ses servantes et d'autres femmes pour l'insulter et lui faire des reproches de toutes sortes d'indécences :

\par 16 mais elle ne répondit à aucune d'elles, ni ne bougea même la tête le moins du monde : ni sa couleur ne fut changée par tous ces traitements, ni aucune crainte ou confusion ne parut en elle, ni sa démarche ne fut altérée ;

\par 17 mais, avec sa manière habituelle, elle se dirigea vers le lieu où elle avait été conduite pour être tuée ; et, pliant les genoux, elle tendit volontairement le cou :

\par 18 et quitta cette vie, réputée pour sa religion et sa chasteté, marquée par « aucun crime, marquée sans culpabilité ; mais elle n'était pas tout à fait exempte de hauteur, selon l'habitude de sa famille.

\par 19 Et ce n'est pas la moindre cause qui fut l'attention et l'affection obséquieuses d'Hérode envers elle, en raison de l'élégance de sa forme ; d'où elle ne soupçonnait aucun changement de sa part à son égard.

\par 20 Or Hérode avait engendré de ses deux fils, Alexandre et Aristobule ; qui, lorsque leur mère fut tuée, vivaient à Rome ; car il les avait envoyés là pour apprendre la littérature et la langue des Romains.

\par 21 Ensuite, Hérode se repentit d'avoir tué sa femme ; et il fut tellement affligé de la mort de cette femme qu'il contracta une maladie dont il faillit mourir.

\par 22 Mariamne étant morte, sa mère Alexandra projeta de faire mourir Hérode ; ce qui, à sa connaissance, il la fit disparaître.

\chapter{59}

\par \textit{L'histoire de la venue des deux fils d'Hérode, Alexandre et Ari ère dès qu'ils apprirent que leur mère avait été mise à mort par Hérode.}

\par 1 Lorsque la nouvelle fut annoncée à Alexandre et à Aristobule du meurtre commis sur leur mère par Hérode, ils furent saisis d'un chagrin excessif ;

\par 2 et quittèrent Rome, ils arrivèrent dans la Ville Sainte, sans rendre aucun respect à leur père Hérode, comme ils avaient coutume de le faire autrefois, à cause de la haine qu'ils avaient envers lui à cause de la mort de leur mère.

\par 3 Or Alexandre avait épousé la fille du roi Archélaüs, et Aristobule avait épousé la fille de la sœur d'Hérode.

\par 4 C'est pourquoi, quand Hérode s'aperçut qu'ils ne lui rendaient aucun respect, il vit qu'il était haï d'eux, et il les évita : et cela n'échappa pas à l'observation des jeunes gens et de sa famille.

\par 5 Le roi Hérode avait épousé avant Mariamne une femme nommée Dosithée, de laquelle il avait un fils nommé Antipater.

\par 6 Comme Hérode fut donc rassuré sur ses deux fils, comme nous l'avons dit plus haut, il amena sa femme Dosithée dans son palais, et attacha à lui son fils Antipater, lui confiant toutes ses affaires ; et il le nomma par testament son successeur.

\par 7 Et qu'Antipater persécutait ses frères Alexandre et Aristobule, dans le dessein de se procurer la paix pendant que son père vivait, afin qu'après sa mort il n'ait plus de rival.

\par 8 C'est pourquoi il dit à son père : En vérité, mes frères recherchent un héritage à cause de la famille de leur mère, parce qu'elle est plus noble que la famille de ma mère ; et c'est pourquoi ils ont plus de droits que moi à la fortune dont le roi m'a jugé digne :

\par 9 C'est pour cette raison qu'ils s'efforcent de vous faire mourir, et moi aussi ils me tueront peu de temps après.

\par 10 Et il répétait cela fréquemment à Hérode, en lui envoyant aussi secrètement des personnes pour lui insinuer des choses qui pourraient produire en lui une plus grande haine à leur égard.

\par 11 Entre-temps, Hérode se rend à Rome chez Auguste, emmenant avec lui son fils Alexandre. « Et lorsqu'il fut arrivé en présence d'Auguste, Hérode se plaignit auprès de lui de son fils, lui demandant de le reprendre.

\par 12 Mais Alexandre dit : « En effet, je ne nie pas mon angoisse à cause du meurtre de ma mère sans aucune faute ; car même les bêtes brutes elles-mêmes montrent de l'affection à leurs mères bien mieux que les hommes et les aiment davantage :

\par 13 mais je nie absolument tout dessein de parricide, et je m'en innocente devant Dieu : car j'ai les mêmes sentiments envers mon père qu'envers ma mère :

\par 14 Je ne suis pas non plus du genre d'homme à m'attirer la culpabilité d'un crime envers mon parent, et plus particulièrement des tourments éternels.

\par 15 Alexandre pleura alors avec des pleurs amers et très véhéments ; Auguste eut pitié de lui, et tous les chefs romains qui se tenaient là pleurèrent aussi.

\par 16 Alors Auguste demanda à Hérode de ramener ses fils dans sa bonté et son intimité d'antan ; et il pria Alexandre de baiser les pieds de son père, qui le fit. Il ordonna également à Hérode de l'embrasser et de l'embrasser, et Hérode lui obéit.

\par 17 Ensuite Auguste commanda un présent magnifique pour Hérode, et on le lui apporta ; et après avoir passé quelques jours avec lui, Hérode retourna à la Sainte Maison ; et appelant les anciens de Juda, il dit :

\par 18 « Sachez qu'Antipater est mon fils aîné et mon premier-né, mais sa mère est d'une famille ignoble ; mais la mère d'Alexandre et d'Aristobule, mes fils, est de la famille des grands prêtres et des rois.

\par 19 De plus, Dieu a élargi mon royaume et a étendu ma puissance ; et donc : il me semble bon de nommer ces trois fils à une autorité égale ; de sorte qu'Antipater n'aura aucun commandement sur ses frères, et ses frères n'auront aucun commandement sur lui.

\par 20 Obéissez donc à tous trois, ô assemblée d'hommes, et ne vous mêlez pas de ce sur quoi leurs esprits peuvent s'entendre ; ni proposer quoi que ce soit qui puisse produire des erreurs et des désaccords entre eux.

\par 21 Et ne bois pas avec eux, et ne parle pas trop avec eux. Car à partir de là il arrivera que l’un d’eux vous exposera sans précaution les desseins qu’il a contre son frère :

\par 22 sur quoi, afin que vous puissiez les concilier, suivra votre accord avec chacun d'eux, selon ce qui lui semblera bon ; et vous les mènerez à la destruction, et vous serez détruits aussi.

\par 23 C'est en effet votre rôle, mes fils, d'être obéissants à Dieu et à moi ; afin que vous viviez longtemps et que vos affaires prospèrent. Peu de temps après, il les embrassa et les baisa, et ordonna au peuple de se retirer.

\par 24 Mais ce qu'Hérode fit n'aboutit pas à un résultat heureux, et les cœurs de ses fils ne furent pas d'accord. Car Antipater voulait que tout soit remis entre ses mains, comme son père l'avait autrefois désigné ; et il ne semblait pas du tout juste à ses frères qu'il soit considéré comme leur égal.

\par 25 Or Antipater était doté de persévérance et de toute mauvaise amitié feinte ; mais ce n'était pas le cas de ses deux frères : Antipater plaça donc des espions sur ses frères, qui devaient lui apporter des nouvelles d'eux : il en plaça également d'autres qui devaient porter de faux rapports à leur sujet à Pilate ».

\par 26 Mais quand Antipater était en présence du roi, et qu'il entendait quelqu'un raconter de telles choses sur ses frères, il repoussa l'accusation de leur part, déclarant que les auteurs n'étaient pas dignes de crédit, et suppliant le roi de ne pas croire les rapports.

\par 27 Ce qu'Antipater fit, afin de n'inspirer au roi aucun doute ni soupçon : sur lui-même.

\par 28 De là le roi ne doutait pas qu'il était bien enclin envers le sien. frères, et ne leur souhaitait aucun mal.

\par 29 Quand Antipater l'apprit, il se pencha sur son oncle Phéroras et sa tante (car ils étaient en inimitié avec ses frères à cause de leur mère), offrant à Phéroras un présent des plus précieux, le priant d'en informer le roi. qu'Alexandre et Aristobule avaient élaboré un plan pour assassiner le roi.

\par 30 (Or Hérode était favorable à Phéroras, son frère, et il s'occupait de tout ce qu'il disait, car il lui payait chaque année une grosse somme sur les provinces qu'il gouvernait sur les rives de l'Euphrate.)

\par 31 Et c'est ce que fit Phéroras. Ensuite Antipater alla trouver Hérode et lui dit : «Ô roi, en SS, mes frères ont préparé un complot pour me détruire.»

\par 32 Antipater donna en outre de l'argent aux trois eunuques du roi, afin qu'ils disent : Alexandre nous a donné de l'argent pour qu'il puisse faire un mauvais usage de nous et pour que nous puissions te tuer. Et comme nous avions reculé, il menaçait nous avec la mort.

\par 33 Et le roi fut irrité contre Alexandre, et ordonna qu'on le mette enchaîné ; et il saisit et fit torturer tous les serviteurs d'Alexandre, jusqu'à ce qu'ils avouent ce qu'ils savaient du complot d'Alexandre pour le tuer.

\par 34 Et beaucoup d'entre eux, bien qu'ils soient morts sous la torture, n'ont jamais dit de mensonge sur Alexandre ; mais certains d'entre eux, ne pouvant supporter la violence du tourment, ont inventé des mensonges dans le désir de se libérer ;

\par 35 affirmant qu'Alexandre et Aristobule avaient projeté d'attaquer le roi, de le tuer et de fuir à Rome ; et ayant reçu une armée d'Auguste, pour marcher contre la Sainte Maison, tuer leur frère Antipater et s'emparer du trône de Judza.

\par 36 Le roi ordonna qu'on arrête Aristobule et qu'on l'enchaîne. Il fut lié et placé avec son frère.

\par 37 Mais lorsque la nouvelle d'Alexandre fut rapportée à son beau-père Archélaüs, celui-ci se rendit chez Hérode, feignant d'être en grande fureur contre Alexandre :


\par 38 comme si, en entendant parler du projet de parricide, il était venu exprès pour voir si sa fille, la femme d'Alexandre, était au courant de l'affaire, et ne le lui avait pas révélé, afin qu'il la mette en danger. à mort : mais que, si elle n'était au courant de rien de pareil, il pourrait la séparer d'Alexandre et l'emmener dans sa propre maison.

\par 39 Or, cet Archélaüs était un homme prudent, sage et éloquent. Et quand Hérode eut entendu ses paroles et fut satisfait de sa prudence et de son honnêteté, il reprit merveilleusement possession de son cœur ; et il se confiait en lui, et comptait sur lui sans la moindre hésitation.

\par 40 Archélaos, trouvant donc l'inclination d'Hérode pour lui, après une longue intimité, lui dit un jour qu'ils s'étaient retirés ensemble :

\par 41 « En vérité, ô roi, en réfléchissant à tes affaires, j'ai découvert que tu es maintenant dans un âge avancé et que tu as beaucoup besoin de repos d'esprit et de réconfort auprès de tes fils ; alors qu'au contraire vous en avez tiré du chagrin et de l'inquiétude.

\par 42 De plus, j'ai pensé à vos deux fils, et je ne trouve pas que vous ayez manqué de bien les mériter ; car vous les avez promus et établis rois, et vous n'avez rien laissé de côté qui pourrait les pousser méchamment à organiser votre mort, et ils n'ont aucune raison de se lancer dans cette affaire.

\par 43 Mais peut-être que cela vient d'une personne malveillante, qui veut du mal contre vous et contre eux, ou qui, par envie ou par inimitié, vous a poussé à les détester.

\par 44 Si donc il a acquis de l'influence sur toi, qui es un vieil homme doué de connaissance, d'instruction et d'expérience, te changeant de la douceur paternelle en cruauté et en fureur contre tes enfants ;

\par 45 combien plus facile aurait-il pu agir sur eux, qui sont jeunes, inexpérimentés, sans surveillance, et sans connaissance des hommes et de leurs ruses, de sorte qu'il ait obtenu d'eux ce qu'il souhaitait dans cette affaire.

\par 46 Considérez donc vos affaires, ô roi ; et n'écoutez pas les paroles des informateurs, et ne faites rien à la hâte contre vos enfants ; et demandez qui est celui qui a comploté le mal contre vous et contre eux.

\par 47 Et le roi lui répondit : «En effet, la situation est telle que vous l'avez mentionnée : j'aurais aimé savoir qui les a poussés à faire cela.» Archélaüs répondit : « Voici ton frère Phéroyas. » Le roi répondit : « C'est peut-être vrai. »

\par 48 Après cela, le roi changea considérablement dans sa conduite envers Phéroras ; et Phéroras, s'en apercevant, eut peur de lui ; et venant vers Archelaus, lui dit :

\par 49 « Je perçois combien le roi est changé envers moi ; c'est pourquoi je vous supplie de réconcilier son esprit avec moi, en supprimant les sentiments qu'il chérit dans son cœur contre moi.

\par 50 À qui Archelaus répondit : « Je le ferai en effet, si vous promettez de révéler au roi la vérité sur les complots que vous avez formés contre Alexandre et Aristobule. » Et il y consentit.

\par 51 Et après quelques jours, Archélaüs dit au roi : « Ô roi, en vérité, les parents d'un homme sont pour lui comme ses propres membres ; et comme il est bon pour un homme, si l'un de ses membres est atteint d'une maladie qui lui arrive, de le guérir par des médicaments, même s'il peut le faire. lui causer de la douleur ;

\par 52 et il n'est pas bon de le retrancher, de peur que la douleur ne s'accentue, que le corps ne s'affaiblisse et que les membres ne tombent en panne ; et ainsi, à cause de la perte de ce membre, il devrait ressentir le besoin de nombreuses commodités :

\par 53 mais qu'il endure les douleurs du traitement médical, afin que le membre aille mieux et soit guéri, et que son corps retrouve sa perfection et sa force d'antan.

\par 54 Ainsi est-il convenable qu'un homme, toutes les fois qu'un de ses parents est altéré à son égard, pour quelque cause abominable que ce soit, se réconcilie avec lui-même ;

\par 55 l'attirer à la civilité et à l'amitié, admettre ses excuses et rejeter les accusations portées contre lui ; et qu'il ne le mette pas à mort précipitamment, ni ne l'éloigne trop longtemps de sa présence.

\par 56 Car les parents d'un homme sont ses soutiens et ses assistants, et en eux consistent son honneur et sa gloire ; et grâce à eux, il obtient ce qu'il ne pourrait pas obtenir autrement.

\par 57 Phéroras est véritablement le frère du roi, et le fils de son père et de sa mère ; et il avoue sa faute, suppliant le roi de l'épargner et de chasser de son esprit son erreur. Et le roi répondit : « C'est ce que je ferai. »

\par 58 Et il ordonna à Phéroras de venir devant lui ; qui, lorsqu'il était en présence, lui dit : « J'ai péché maintenant aux yeux du grand et bon Dieu et du roi, en concevant des méfaits et des plans qui pourraient nuire aux affaires du roi et de ses fils, par des mensonges menteurs.

\par 59 Mais ce qui m'a poussé à agir ainsi, c'est que le roi m'a enlevé une certaine femme, ma concubine, et l'a séparée de moi.

\par 60 Le roi dit à Archélaos : « J'ai maintenant pardonné à Phéroras, comme tu me l'as demandé ; car je trouve que tu as guéri le mal qui était dans nos affaires par tes méthodes apaisantes, comme un médecin ingénieux guérit les corruptions d'un homme. corps malade.

\par 61 C'est pourquoi je te supplie de pardonner à Alexandre, en réconciliant ta fille avec son mari ; car considère-la comme ma fille, puisque je sais qu'elle est plus prudente que lui, et qu'elle le détourne de beaucoup de choses par sa prudence et ses exhortations.

\par 62 C'est pourquoi je te prie de ne pas les séparer et de ne pas le détruire, car il est d'accord avec elle et obtient de nombreux avantages grâce à sa direction.

\par 63 Mais Archélaüs répondit : « Ma fille est la servante du roi ; mais mon âme l'a récemment détesté à cause de son mauvais dessein. Que le roi me permette donc de le séparer de ma fille, que le roi pourra unir à qui de ses serviteurs il voudra.

\par 64 A qui le roi répondit : « N'allez pas au-delà de ma demande ; et laisse ta fille rester avec lui, et ne me contredis pas. Et Archélaüs dit : « Je le ferai sûrement ; et je ne contredirai pas le roi dans tout ce qu’il m’ordonnera.

\par 65 Peu de temps après, Hérode ordonne qu'Alexandre et Aristobule soient libérés de leurs chaînes et se présentent devant lui : qui, lorsqu'ils étaient en sa présence, se prosternaient devant lui, avouant leurs fautes, s'excusant et implorant pardon et le pardon.

\par 66 Et il leur ordonna de se lever, et les faisant s'approcher de lui, il les baisa et leur ordonna de rentrer chez eux et de revenir le lendemain. Et ils vinrent manger et boire avec lui, et il les réinstalla dans une place plus honorable.

\par 67 Et il donna à Archélaüs soixante-dix talents et un lit d'or, en enjoignant également à tous les chefs de ses amis d'offrir des cadeaux de valeur à Archélaüs : et ils le firent.

\par 68 Ceci étant accompli, Archélaüs quitta la ville de la Sainte Maison pour se rendre dans son propre pays ; qu'Hérode accompagna, et enfin, après l'avoir quitté, il retourna à la Sainte Maison.

\par 69 Néanmoins, Antipater ne cessa pas de comploter contre ses frères, afin de les rendre odieux.

\par 70 Or, il arriva qu'un certain homme vint chez Hérode, avec de beaux et précieux objets, avec lesquels on gagne habituellement les rois ;

\par 71 il les présenta au roi, qui les lui prit et les lui rendit ; et l'homme obtint une place très élevée dans ses affections, et ayant été pris dans sa suite, il jouit de sa confiance : cet homme s'appelait Eurycles.

\par 72 Antipater, s'apercevant que cet homme avait entièrement attiré la faveur de son père, lui offrit de l'argent, le priant d'insinuer adroitement à Hérode et de soutenir que ses deux fils Alexandre et Aristobule projetaient de l'assassiner. ce que l'homme lui avait promis de faire.

\par 73 Peu de temps après, il se rendit chez Alexandre et devint intime et familier avec lui à tel point qu'on savait qu'il était dans son amitié, et il fut fait savoir au roi qu'il était intime avec lui.

\par 74 Après cela, il s'éloigna avec le roi et lui dit : « Certes, tu as ce droit sur moi, ô roi, que rien ne doit m'empêcher de te donner de bons conseils ; et en vérité j'ai une affaire que le roi doit savoir et que je dois te dévoiler. »

\par 75 Le roi lui dit : Qu'as-tu ? L'homme lui répondit : « J'ai entendu Alexandre dire : « En vérité, Dieu a différé la vengeance sur mon père pour la mort de ma mère, de mon grand-père et de mes parents, sans aucun crime, afin que cela se produise par ma main. » J'espère que je m'en vengerai sur lui.

\par 76 Et maintenant il s'est entendu avec quelques chefs pour vous attaquer, et il a voulu m'impliquer dans les plans qu'il avait formés : mais j'ai tenu cela pour un crime, à cause des actes de bonté du roi envers moi, et sa libéralité.

\par 77 Mais mon intention est de bien le réprimander et de lui rapporter cela, car il a des yeux et de l'intelligence.

\par 78 Et quand le roi eut entendu ces paroles, il ne les mit nullement au néant, mais se mit promptement à s'enquérir de leur vérité :

\par 79 mais il ne trouva rien sur quoi s'appuyer, si ce n'est une lettre contrefaite au nom d'Alexandre et d'Aristobule adressée au gouverneur d'une certaine ville.

\par 80 Et il y avait dans la lettre : « Nous voulons tuer notre père et fuir vers toi ; préparez-nous donc un endroit où nous pourrons rester jusqu'à ce que le peuple se rassemble autour de nous et que nos affaires soient réglées.

\par 81 Et cela fut en effet confirmé au roi et parut probable ; c'est pourquoi il saisit le gouverneur de cette ville et le fit tourmenter, afin qu'il puisse avouer ce qui était inséré dans cette lettre.

\par 82 Ce que cet homme a nié, se disculpant de l'accusation : et rien n'a été prouvé contre eux dans cette affaire, ni dans aucune autre chose que l'informateur leur avait imputée.

\par 83 Mais Hérode ordonna qu'ils soient saisis et liés avec des chaînes et des chaînes. Puis il se rendit à Tyr et de Tyr à Césarée, les emportant avec lui enchaînés.

\par 84 Et tous les capitaines et tous les soldats les plaignirent ; mais personne n'intercéda pour eux auprès du roi, de peur qu'il n'admît que ce que l'informateur avait affirmé était vrai de lui-même.

\par 85 Or, il y avait dans l'armée un certain vieux guerrier qui avait un fils au service d'Alexandre. Lorsque le vieillard vit donc la misérable condition des deux fils d'Hérode, il eut une merveilleuse pitié pour leur fortune et s'écria d'une voix aussi forte qu'il le pouvait : « La pitié est partie ; la bonté et la piété ont disparu ; la vérité est retirée du monde.

\par 86 Alors il dit au roi : « Ô toi, impitoyable envers tes enfants, ennemi de tes amis et ami de tes ennemis, qui reçois les paroles des informateurs et des personnes qui ne te souhaitent aucun bien !

\par 87 Et les ennemis d'Alexandre et d'Aristobule accoururent vers lui, le réprimandèrent et dirent au roi : « Ô roi, ce n'est pas l'amour envers toi et envers tes fils qui a poussé cet homme à parler ainsi ;

\par 88 mais il a voulu bavarder la haine qu'il vous portait dans son cœur, et dire du mal de vos conseils et de votre administration, comme étant un conseiller fidèle.

\par 89 Et en effet, certains observateurs nous ont informé de lui, qu'il avait déjà fait alliance avec le barbier du roi de le tuer avec le rasoir pendant qu'il le rasait.

\par 90 Et le roi ordonna de saisir le vieillard, son fils et le barbier ; et le vieil homme et le barbier seraient flagellés avec des verges jusqu'à ce qu'ils se confessent. Et ils furent battus à coups de verges très cruellement et soumis à diverses sortes de tortures ; mais ils n'avouèrent rien de ce qu'ils n'avaient pas fait.

\par 91 Quand donc le fils du vieillard vit la triste condition de son père et l'état où il était arrivé, il le plaignit et pensa qu'il serait libéré s'il confessait lui-même ce qui lui avait été confié. son père, après avoir reçu du roi une promesse pour sa vie.

\par 92 C'est pourquoi il dit au roi : « Ô roi, donne-moi la sécurité de mon père et de moi, afin que je puisse te dire ce que tu cherches. » Et le roi dit : « Vous pouvez avoir ceci. »

\par 93 A qui il dit : «Alexandre était déjà d'accord avec mon père pour qu'il te tue ; mais mon père était d'accord avec le barbier, comme on te l'a dit.»

\par 94 Alors le roi ordonna de tuer ce vieil homme et son fils, ainsi que le barbier. Il ordonna également que ses fils Alexandre et Aristobule soient emmenés à Sébaste, où ils furent tués et fixés sur un gibet ; et ils furent pris, tués et fixés sur un gibet.

\par 95 Or Alexandre laissa deux fils qui lui survécurent, savoir Tyrcan et Alexandre, de la fille du roi Archélaüs ; et Aristobule laissa trois fils, savoir Aristobule, Agrippa et Hérode.

\par 96 Mais l'histoire d'Antipater, fils d'Hérode, a déjà été décrite dans nos récits précédents.

\end{document}