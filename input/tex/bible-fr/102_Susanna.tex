\begin{document}

\title{Susanne}


\chapter{1}

\par 1 Mis à part du début de Daniel, car ce n'est pas en hébreu, comme ni le récit de Bel et du Dragon.[1] Il y avait à Babylone un homme appelé Joacim :
\par 2 Et il prit une femme, nommée Suzanne, fille de Chelcias, une femme très belle et qui craignait l'Éternel.
\par 3 Ses parents aussi étaient justes et enseignaient à leur fille selon la loi de Moïse.
\par 4 Or Joacim était un homme très riche, et il avait un beau jardin attenant à sa maison ; et c'est vers lui que venaient les Juifs ; parce qu'il était plus honorable que tous les autres.
\par 5 La même année, deux anciens du peuple furent nommés juges, comme l'Éternel l'avait dit, selon lequel la méchanceté venait de Babylone, de la part des anciens juges qui semblaient gouverner le peuple.
\par 6 Ceux-ci avaient beaucoup de choses dans la maison de Joacim ; et tous ceux qui avaient des litiges venaient vers eux.
\par 7 Or, quand les gens partaient à midi, Suzanne alla se promener dans le jardin de son mari.
\par 8 Et les deux anciens la voyaient entrer et se promener chaque jour ; de sorte que leur convoitise s'enflamma envers elle.
\par 9 Et ils ont perverti leur propre esprit, et ont détourné leurs yeux, afin de ne pas regarder vers le ciel, ni se souvenir des justes jugements.
\par 10 Et bien qu'ils fussent tous deux blessés par son amour, l'un n'osait pourtant pas montrer à l'autre sa douleur.
\par 11 Car ils avaient honte de déclarer leur convoitise, qu'ils désiraient avoir affaire à elle.
\par 12 Pourtant, ils veillaient diligemment de jour en jour pour la voir.
\par 13 Et l'un dit à l'autre : Rentrons maintenant à la maison, car c'est l'heure du dîner.
\par 14 Et quand ils furent sortis, ils se séparèrent l'un de l'autre, et revenant sur eux, ils arrivèrent au même endroit ; et après s'être demandé mutuellement la cause, ils reconnurent leur convoitise : puis ils fixèrent un moment tous deux ensemble, où ils pourraient la trouver seule.
\par 15 Et comme ils veillaient à un moment propice, elle entra comme auparavant avec deux servantes seulement, et elle désirait se laver dans le jardin, car il faisait chaud.
\par 16 Et il n'y avait personne là, sauf les deux anciens, qui s'étaient cachés et qui la surveillaient.
\par 17 Alors elle dit à ses servantes : Apportez-moi de l'huile et des boules de lessive, et fermez les portes du jardin, afin que je me lave.
\par 18 Et ils firent ce qu'elle leur avait ordonné, et fermèrent les portes du jardin, et sortirent eux-mêmes par des portes secrètes pour aller chercher ce qu'elle leur avait commandé ; mais ils ne virent pas les anciens, parce qu'ils étaient cachés.
\par 19 Lorsque les servantes furent sorties, les deux anciens se levèrent et coururent vers elle, en disant :
\par 20 Voici, les portes du jardin sont fermées, et personne ne peut nous voir, et nous sommes amoureux de toi ; consentez-nous donc et couchez avec nous.
\par 21 Si tu ne le veux pas, nous témoignerons contre toi qu'un jeune homme était avec toi, et c'est pourquoi tu as renvoyé tes servantes loin de toi.
\par 22 Alors Suzanne soupira et dit : Je suis à l'étroit de tous côtés ; car si je fais cela, c'est la mort pour moi ; et si je ne le fais pas, je ne peux pas échapper à tes mains.
\par 23 Il vaut mieux pour moi tomber entre vos mains et ne pas le faire, plutôt que de pécher devant le Seigneur.
\par 24 Sur ce, Suzanne cria d'une voix forte, et les deux anciens crièrent contre elle.
\par 25 Alors celui-ci courut et ouvrit la porte du jardin.
\par 26 Ainsi, lorsque les domestiques de la maison entendirent le cri dans le jardin, ils se précipitèrent par la porte privée, pour voir ce qui lui était fait.
\par 27 Mais lorsque les anciens eurent exposé leur affaire, les serviteurs furent très honteux, car on n'a jamais fait pareil bruit au sujet de Suzanne.
\par 28 Et il arriva le lendemain que le peuple était rassemblé auprès de son mari Joacim, que les deux anciens vinrent aussi, pleins d'imagination malicieuse, contre Suzanne pour la faire mourir ;
\par 29 Et il dit devant le peuple : Faites venir Suzanne, fille de Chelcias, femme de Joacim. Et donc ils ont envoyé.
\par 30 Elle vint donc avec son père et sa mère, ses enfants et toute sa parenté.
\par 31 Or Suzanne était une femme très délicate et belle à voir.
\par 32 Et ces méchants hommes ordonnèrent de découvrir son visage, (car elle était couverte) afin qu'ils puissent être remplis de sa beauté.
\par 33 C'est pourquoi ses amis et tous ceux qui la voyaient pleuraient.
\par 34 Alors les deux anciens se levèrent au milieu du peuple et posèrent leurs mains sur sa tête.
\par 35 Et elle leva les yeux vers le ciel en pleurant, car son cœur se confiait au Seigneur.
\par 36 Et les anciens dirent : Tandis que nous nous promenions seuls dans le jardin, cette femme entra avec deux servantes, ferma les portes du jardin et renvoya les servantes.
\par 37 Alors un jeune homme, qui était caché là, vint vers elle et coucha avec elle.
\par 38 Alors nous qui étions dans un coin du jardin, voyant cette méchanceté, nous avons couru vers eux.
\par 39 Et quand nous les avons vus ensemble, nous n'avons pas pu retenir l'homme, car il était plus fort que nous, il a ouvert la porte et a bondi dehors.
\par 40 Mais après avoir pris cette femme, nous avons demandé qui était ce jeune homme, mais elle n'a pas voulu nous le dire : nous témoignons de ces choses.
\par 41 Alors l'assemblée les crut comme étant les anciens et les juges du peuple : ils la condamnèrent donc à mort.
\par 42 Alors Suzanne s'écria d'une voix forte et dit : Ô Dieu éternel, qui connais les secrets et qui connais toutes choses avant qu'elles soient !
\par 43 Tu sais qu'ils ont porté un faux témoignage contre moi, et voici, je dois mourir ; alors que je n’ai jamais fait ce que ces hommes ont inventé avec méchanceté contre moi.
\par 44 Et le Seigneur entendit sa voix.
\par 45 C'est pourquoi, lorsqu'elle fut conduite à la mort, le Seigneur suscita l'esprit saint d'un jeune garçon nommé Daniel :
\par 46 Qui a crié d'une voix forte, je suis pur du sang de cette femme.
\par 47 Alors tout le peuple se tourna vers lui et dit : Que signifient ces paroles que tu as prononcées ?
\par 48 Alors, debout au milieu d'eux, il dit : Êtes-vous tellement insensés, vous, fils d'Israël, que sans examen ni connaissance de la vérité vous avez condamné une fille d'Israël ?
\par 49 Retournez encore au lieu du jugement, car ils ont porté un faux témoignage contre elle.
\par 50 Alors tout le peuple se retourna en toute hâte, et les anciens lui dirent : Viens, assieds-toi parmi nous, et montre-le-nous, puisque Dieu t'a donné l'honneur d'un ancien.
\par 51 Alors Daniel leur dit : Éloignez ces deux-là l'un de l'autre, et je les examinerai.
\par 52 Alors, quand ils furent séparés l'un de l'autre, il appela l'un d'eux et lui dit : Ô toi qui as vieilli dans la méchanceté, maintenant tes péchés que tu as commis autrefois sont révélés.
\par 53 Car tu as prononcé un faux jugement, tu as condamné l'innocent et tu as laissé libre le coupable ; bien que l'Éternel dise : Tu ne tueras pas les innocents et les justes.
\par 54 Maintenant donc, si tu l'as vue, dis-moi : Sous quel arbre les as-tu vus se réunir ? Qui répondit : Sous un arbre à bâtons.
\par 55 Et Daniel dit : Très bien ; tu as menti contre ta propre tête ; car même maintenant, l'ange de Dieu a reçu la sentence de Dieu de te couper en deux.
\par 56 Alors il l'écarta, et ordonna d'amener l'autre, et lui dit : Ô postérité de Chanaan, et non de Juda, la beauté t'a trompé, et la convoitise a perverti ton cœur.
\par 57 Ainsi avez-vous traité les filles d'Israël, et elles vous ont accompagné avec crainte; mais la fille de Juda n'a pas supporté votre méchanceté.
\par 58 Et maintenant, dis-moi : sous quel arbre les as-tu réunis ? Qui répondit : Sous un arbre vert.
\par 59 Alors Daniel lui dit : Eh bien ; tu as aussi menti contre ta propre tête, car l'ange de Dieu attend avec l'épée pour te couper en deux, afin de te détruire.
\par 60 Sur ce, toute l'assemblée cria à haute voix et loua Dieu, qui sauve ceux qui se confient en lui.
\par 61 Et ils se soulevèrent contre les deux anciens, car Daniel les avait convaincus de faux témoignage par leur propre bouche.
\par 62 Et selon la loi de Moïse, ils leur firent ce qu'ils voulaient faire mal à leur prochain : et ils les mirent à mort. Ainsi le sang innocent fut sauvé le jour même.
\par 63 C'est pourquoi Chelcias et sa femme louèrent Dieu pour leur fille Suzanne, ainsi que pour Joacim, son mari, et toute leur famille, parce qu'on ne trouvait chez elle aucune malhonnêteté.
\par 64 Depuis ce jour, Daniel jouit d'une grande réputation aux yeux du peuple.

\end{document}