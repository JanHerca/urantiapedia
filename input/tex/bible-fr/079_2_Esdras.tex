\begin{document}

\title{2 Esdras}

\chapter{1}

\par 1 Le deuxième livre du prophète Esdras, fils de Saraïas, fils d'Azarias, fils de Helchias, fils de Sadamias, fils de Sadoc, fils d'Achitob,
\par 2 Le fils d'Achias, le fils de Phinées, le fils d'Héli, le fils d'Amarias, le fils d'Aziei, le fils de Marimoth, le fils de Et il parla à Borith, fils d'Abisei, le fils de Phinées, fils d'Éléazar,
\par 3 Fils d'Aaron, de la tribu de Lévi ; qui était captif au pays des Mèdes, sous le règne d'Artexerxès, roi des Perses.
\par 4 Et la parole du Seigneur me fut adressée, disant :
\par 5 Va, et montre à mon peuple ses péchés, et à ses enfants la méchanceté qu'ils ont commise contre moi ; afin qu'ils puissent dire aux enfants de leurs enfants :
\par 6 Parce que les péchés de leurs pères se sont accrus en eux, car ils m'ont oublié et se sont offerts à des dieux étrangers.
\par 7 Ne suis-je pas celui qui les a fait sortir du pays d'Égypte, de la maison de servitude ? mais ils m'ont irrité et ont méprisé mes conseils.
\par 8 Arrache donc les cheveux de ta tête, et jette sur eux tout le mal, car ils n'ont pas obéi à ma loi, mais c'est un peuple rebelle.
\par 9 Jusqu'à quand les abstiendrai-je, à qui j'ai fait tant de bien ?
\par 10 J'ai détruit beaucoup de rois à cause d'eux ; J'ai frappé Pharaon, ses serviteurs et toute sa puissance.
\par 11 J'ai détruit toutes les nations devant eux, et à l'est j'ai dispersé les habitants de deux provinces, Tyrus et Sidon, et j'ai tué tous leurs ennemis.
\par 12 Parle-leur donc, en disant : Ainsi parle l'Éternel :
\par 13 Je t'ai fait traverser la mer et, au début, je t'ai donné un passage large et sûr ; Je vous ai donné Moïse pour chef et Aaron pour prêtre.
\par 14 Je vous ai éclairé dans une colonne de feu, et j'ai fait de grands prodiges parmi vous ; mais vous m'avez oublié, dit le Seigneur.
\par 15 Ainsi parle l'Eternel tout-puissant : Les cailles étaient pour vous un signe ; Je vous ai donné des tentes pour votre sauvegarde : néanmoins vous y avez murmuré :
\par 16 Et vous n'avez pas triomphé en mon nom pour la destruction de vos ennemis, mais vous murmurez encore aujourd'hui.
\par 17 Où sont les bienfaits que j'ai faits pour vous ? Quand vous aviez faim et soif dans le désert, ne m'avez-vous pas crié :
\par 18 En disant : Pourquoi nous as-tu amenés dans ce désert pour nous tuer ? il valait mieux pour nous servir les Égyptiens que de mourir dans ce désert.
\par 19 Alors j'ai eu pitié de vos deuils, et je vous ai donné de la manne à manger ; vous avez donc mangé du pain des anges.
\par 20 Quand vous aviez soif, n'ai-je pas brisé le rocher, et des eaux ont coulé à votre faim ? pour la chaleur, je t'ai couvert de feuilles d'arbres.
\par 21 Je vous ai partagé un pays fertile, j'ai chassé devant vous les Cananéens, les Phérézites et les Philistins : que ferai-je encore pour vous ? dit le Seigneur.
\par 22 Ainsi parle l'Éternel tout-puissant : Lorsque vous étiez dans le désert, dans le fleuve des Amoréens, ayant soif et blasphémant mon nom,
\par 23 Je ne vous ai pas donné de feu pour vos blasphèmes, mais j'ai jeté un arbre dans l'eau et j'ai rendu le fleuve doux.
\par 24 Que te ferai-je, ô Jacob ? toi, Juda, tu ne m'obéiras pas : je me tournerai vers d'autres nations, et à celles-là je donnerai mon nom, afin qu'elles gardent mes statuts.
\par 25 Puisque vous m'avez abandonné, je vous abandonnerai aussi ; quand vous désirez que je vous fasse grâce, je n'aurai aucune pitié envers vous.
\par 26 Chaque fois que vous m'invoquerez, je ne vous écouterai pas ; car vous avez souillé vos mains avec du sang, et vos pieds sont prompts à commettre un homicide.
\par 27 Ce n'est pas vous qui m'avez abandonné, mais vous-mêmes, dit l'Éternel.
\par 28 Ainsi parle l'Eternel tout-puissant : Ne vous ai-je pas prié comme un père pour ses fils, comme une mère pour ses filles, et une nourrice pour ses petits enfants,
\par 29 Afin que vous soyez mon peuple, et que je sois votre Dieu ; que vous seriez mes enfants et que je serais votre père ?
\par 30 Je vous ai rassemblés, comme une poule rassemble ses poules sous ses ailes ; mais maintenant, que vous ferai-je ? Je te chasserai de ma face.
\par 31 Lorsque vous m'offrirez des offrandes, je détournerai ma face de vous ; car j'ai abandonné vos fêtes solennelles, vos nouvelles lunes et vos circoncisions.
\par 32 Je vous ai envoyé mes serviteurs, les prophètes, que vous avez pris et tués, et que vous avez déchirés leurs corps, et dont je redemanderai le sang de vos mains, dit l'Éternel.
\par 33 Ainsi parle l'Éternel tout-puissant : Ta maison est désolée, je te chasserai comme le vent rend le chaume.
\par 34 Et vos enfants ne produiront pas de fruits ; car ils ont méprisé mon commandement et ont fait ce qui est mal à mes yeux.
\par 35 Je donnerai vos maisons à un peuple qui viendra ; qui, n'ayant pas encore entendu parler de moi, me croira ; à qui je n'ai montré aucun signe, ils feront pourtant ce que je leur ai commandé.
\par 36 Ils n'ont pas vu de prophètes, mais ils se souviendront de leurs péchés et les reconnaîtront.
\par 37 Je prends à témoin la grâce du peuple à venir, dont les petits se réjouissent dans la joie ; et bien qu'ils ne m'aient pas vu des yeux corporels, cependant en esprit ils croient ce que je dis.
\par 38 Et maintenant, frère, vois quelle gloire ; et je vois les gens qui viennent de l'est :
\par 39 À qui je donnerai pour chefs Abraham, Isaac et Jacob, Oséas, Amos et Michéas, Joël, Abdias et Jonas,
\par 40 Nahum, et Abacuc, Sophonias, Aggée, Zacharie et Malachie, qui est aussi appelé ange du Seigneur.

\chapter{2}

\par 1 Ainsi parle l'Éternel : J'ai libéré ce peuple de la servitude, et je lui ai donné mes commandements par les serviteurs les prophètes ; qu'ils n'ont pas voulu entendre, mais qui ont méprisé mes conseils.
\par 2 La mère qui les a enfantés leur dit : Allez, enfants ; car je suis veuve et abandonnée.
\par 3 Je t'ai élevé dans la joie ; mais je vous ai perdu avec tristesse et tristesse ; car vous avez péché devant l'Éternel, votre Dieu, et vous avez fait ce qui est mal devant lui.
\par 4 Mais que dois-je vous faire maintenant ? Je suis veuve et abandonnée : partez, ô mes enfants, et demandez miséricorde au Seigneur.
\par 5 Quant à moi, ô père, je te prends à témoin sur la mère de ces enfants, qui n'a pas respecté mon alliance,
\par 6 Afin que tu les conduises à la confusion et à la dépouille de leur mère, afin qu'il n'y ait pas de postérité d'eux.
\par 7 Qu'ils soient dispersés parmi les païens, que leurs noms soient rayés de la terre, car ils ont méprisé mon alliance.
\par 8 Malheur à toi, Assur, toi qui caches l'injuste en toi ! Ô méchants gens, souvenez-vous de ce que j'ai fait à Sodome et à Gomorrhe ;
\par 9 Dont le pays est recouvert de mottes de poix et de monceaux de cendre : ainsi je ferai aussi à ceux qui ne m'écoutent pas, dit l'Éternel tout-puissant.
\par 10 Ainsi parle l'Éternel à Esdras : Dites à mon peuple que je lui donnerai le royaume de Jérusalem, que j'aurais donné à Israël.
\par 11 Je prendrai aussi leur gloire, et je leur donnerai les tabernacles éternels que je leur avais préparés.
\par 12 Ils auront l'arbre de vie comme un parfum de douce odeur ; ils ne travailleront ni ne se fatigueront.
\par 13 Allez, et vous recevrez : priez pendant quelques jours, afin qu'ils soient abrégés : le royaume est déjà préparé pour vous : veillez.
\par 14 Prenez à témoin le ciel et la terre ; car j'ai brisé le mal et créé le bien; car je vis, dit le Seigneur.
\par 15 Mère, embrasse tes enfants, et élève-les dans la joie, affermis leurs pieds comme un pilier : car je t'ai choisie, dit le Seigneur.
\par 16 Et ceux qui sont morts, je les ressusciterai de leurs lieux, et je les ferai sortir des tombeaux, car j'ai connu mon nom en Israël.
\par 17 Ne crains rien, mère des enfants, car je t'ai choisie, dit l'Éternel.
\par 18 Pour ton aide, j'enverrai mes serviteurs Ésaü et Jérémie, selon le conseil desquels j'ai sanctifié et préparé pour toi douze arbres chargés de fruits divers,
\par 19 Et autant de fontaines coulant de lait et de miel, et sept montagnes puissantes, sur lesquelles poussent des roses et des lys, par lesquels je comblerai de joie tes enfants.
\par 20 Faites justice à la veuve, jugez pour l'orphelin, donnez aux pauvres, défendez l'orphelin, habillez celui qui est nu,
\par 21 Guéris ceux qui sont brisés et faibles, ne te moque pas du boiteux, défends les estropiés, et que l'aveugle vienne à la vue de ma clarté.
\par 22 Garde les vieux et les jeunes dans tes murs.
\par 23 Partout où tu trouves des morts, prends-les et enterre-les, et je te donnerai la première place à ma résurrection.
\par 24 Demeure tranquille, ô mon peuple, et repose-toi, car ta tranquillité est toujours venue.
\par 25 Nourris tes enfants, ô bonne nourrice ; stabiliser leurs pieds.
\par 26 Quant aux serviteurs que je t'ai donnés, aucun d'eux ne périra ; car je les exigerai du nombre de toi.
\par 27 Ne vous lassez pas ; car quand le jour de détresse et de lourdeur viendra, les autres pleureront et seront tristes, mais toi vous serez joyeux et dans l'abondance.
\par 28 Les païens t'envieront, mais ils ne pourront rien faire contre toi, dit l'Éternel.
\par 29 Mes mains te couvriront, afin que tes enfants ne voient pas l'enfer.
\par 30 Sois joyeuse, ô mère, avec tes enfants ; car je te délivrerai, dit l'Éternel.
\par 31 Souvenez-vous de vos enfants qui dorment, car je les ferai sortir des extrémités de la terre et je leur ferai miséricorde ; car je suis miséricordieux, dit le Seigneur tout-puissant.
\par 32 Embrasse tes enfants jusqu'à ce que je vienne leur faire miséricorde ; car mes puits débordent, et ma grâce ne faillira pas.
\par 33 Moi, Esdras, j'ai reçu l'ordre de l'Éternel sur la montagne d'Oreb, de me rendre en Israël; mais quand je suis arrivé vers eux, ils m'ont méprisé et ont méprisé le commandement du Seigneur.
\par 34 Et c'est pourquoi je vous le dis : ô païens, qui écoutez et comprenez, cherchez votre berger, il vous donnera le repos éternel ; car il est proche, celui qui viendra à la fin du monde.
\par 35 Soyez prêts à recevoir la récompense du royaume, car la lumière éternelle brillera sur vous pour toujours.
\par 36 Fuyez l'ombre de ce monde, recevez la joie de votre gloire : je témoigne ouvertement de mon Sauveur.
\par 37 Recevez le don qui vous est fait, et réjouissez-vous, en rendant grâce à celui qui vous a conduit au royaume des cieux.
\par 38 Lève-toi et tiens-toi debout, voici le nombre de ceux qui seront scellés à la fête du Seigneur ;
\par 39 Qui sont sortis de l'ombre du monde et ont reçu les vêtements glorieux du Seigneur.
\par 40 Prends ton numéro, ô Sion, et enferme ceux d'entre toi qui sont vêtus de blanc et qui ont accompli la loi du Seigneur.
\par 41 Le nombre de tes enfants que tu désirais est comblé : implore la puissance du Seigneur, afin que ton peuple, appelé dès le commencement, soit sanctifié.
\par 42 Moi, Esdras, je vis sur la montagne de Sion un grand peuple que je ne pouvais pas compter, et ils louaient tous le Seigneur par des chants.
\par 43 Et au milieu d'eux il y avait un jeune homme de haute stature, plus grand que tous les autres, et sur chacune de leurs têtes il posait des couronnes, et était plus élevé ; ce qui m'a beaucoup émerveillé.
\par 44 J'interrogeai donc l'ange et je dis : Monsieur, qu'est-ce que c'est ?
\par 45 Il répondit et me dit : Ce sont ceux-là qui ont dépouillé les vêtements mortels, et revêtu l'immortel, et qui ont confessé le nom de Dieu ; maintenant ils sont couronnés et reçoivent des palmes.
\par 46 Alors je dis à l'ange : Quel est ce jeune homme qui les couronne et leur donne des palmes dans leurs mains ?
\par 47 Alors il répondit et me dit : C'est le Fils de Dieu qu'ils ont confessé dans le monde. Alors j’ai commencé à féliciter chaleureusement ceux qui défendaient avec tant de fermeté le nom du Seigneur.
\par 48 Alors l'ange me dit : Va, et raconte à mon peuple quelles sortes de choses et quelles grandes merveilles tu as vues de l'Éternel, ton Dieu.

\chapter{3}

\par 1 La trentième année après la ruine de la ville, j'étais à Babylone, et j'étais troublé sur mon lit, et mes pensées montaient dans mon cœur :
\par 2 Car j'ai vu la désolation de Sion et la richesse de ceux qui habitaient à Babylone.
\par 3 Et mon esprit fut très ému, de sorte que je me mis à dire au Très-Haut des paroles pleines de crainte, et je dis :
\par 4 O Seigneur, qui gouvernes, tu as dit au commencement, quand tu plantais la terre, et cela toi seul, et que tu commandais au peuple :
\par 5 Et tu as donné à Adam un corps sans âme, qui était l'ouvrage de tes mains, et tu lui as insufflé le souffle de vie, et il a été rendu vivant devant toi.
\par 6 Et tu le conduises au paradis, que ta main droite avait planté, avant que la terre ne s'avance.
\par 7 Et tu lui as donné le commandement d'aimer ta voie, qu'il a transgressée, et aussitôt tu as fixé la mort en lui et dans ses générations, d'où sont issues des nations, des tribus, des peuples et des tribus innombrables.
\par 8 Et chaque peuple marchait selon sa propre volonté, et faisait des choses merveilleuses devant toi, et méprisait tes commandements.
\par 9 Et de nouveau, au fil du temps, tu as fait venir le déluge sur ceux qui habitaient dans le monde, et tu les as détruits.
\par 10 Et il arriva chez chacun d'eux que, comme la mort fut pour Adam, ainsi le déluge fut pour eux.
\par 11 Mais tu as laissé l'un d'eux, à savoir Noé et sa maison, d'où venaient tous les justes.
\par 12 Et il arriva que lorsque les habitants de la terre commencèrent à se multiplier, et qu'ils leur eurent donné beaucoup d'enfants et formèrent un grand peuple, ils recommencèrent à être plus impies que les premiers.
\par 13 Or, alors qu'ils vivaient si méchamment devant toi, tu t'es choisi parmi eux un homme nommé Abraham.
\par 14 Celui que tu as aimé, et c'est à lui seul que tu as montré ta volonté :
\par 15 Et tu as conclu une alliance éternelle avec lui, en lui promettant que tu n'abandonnerais jamais sa postérité.
\par 16 Et tu lui as donné Isaac, et à Isaac aussi tu as donné Jacob et Ésaü. Quant à Jacob, tu l'as choisi pour toi et tu l'as mis auprès d'Ésaü; et ainsi Jacob est devenu une grande multitude.
\par 17 Et il arriva que, lorsque tu conduisit sa postérité hors d'Egypte, tu la fit monter sur le mont Sinaï.
\par 18 Et en courbant les cieux, tu as consolidé la terre, tu as ébranlé le monde entier, tu as fait trembler les profondeurs et tu as troublé les hommes de ce siècle.
\par 19 Et ta gloire passait par quatre portes, du feu, du tremblement de terre, du vent et du froid ; afin que tu donnes la loi à la postérité de Jacob, et le diligence à la génération d'Israël.
\par 20 Mais tu ne leur as pas ôté un cœur méchant, afin que ta loi produise en eux du fruit.
\par 21 Car le premier Adam, qui avait un cœur méchant, a transgressé et a été vaincu ; et ainsi soient tous ceux qui sont nés de lui.
\par 22 Ainsi l'infirmité est devenue permanente; et la loi (aussi) dans le cœur du peuple avec la malignité de la racine ; de sorte que les bons s'en allèrent et que les méchants restèrent immobiles.
\par 23 Ainsi les temps passèrent, et les années furent terminées ; alors tu t'es suscité un serviteur, appelé David.
\par 24 À qui tu as ordonné de bâtir une ville à ton nom, et de t'y offrir de l'encens et des oblations.
\par 25 Après de nombreuses années que cela fut fait, alors ceux qui habitaient la ville t'abandonnèrent,
\par 26 Et ils firent en toutes choses comme Adam et toutes ses générations avaient fait ; car eux aussi avaient un cœur méchant :
\par 27 Et ainsi tu as livré ta ville entre les mains de tes ennemis.
\par 28 Leurs actions sont-elles donc meilleures que celles qui habitent Babylone, pour qu'ils dominent donc Sion ?
\par 29 Car lorsque j'y suis arrivé, et que j'ai vu un nombre incalculable d'impiétés, mon âme a vu beaucoup de méchants en cette trentième année, de sorte que mon cœur m'a défailli.
\par 30 Car j'ai vu comment tu leur permets de pécher, et que tu as épargné les méchants, et que tu as détruit ton peuple, et que tu as préservé tes ennemis, et que tu ne l'as pas signifié.
\par 31 Je ne me souviens pas comment on peut laisser cette voie : ceux de Babylone sont-ils donc meilleurs que ceux de Sion ?
\par 32 Ou y a-t-il un autre peuple qui te connaisse en dehors d'Israël ? ou quelle génération a autant cru à tes alliances que Jacob ?
\par 33 Et pourtant leur récompense n'apparaît pas, et leur travail n'a aucun fruit ; car j'ai parcouru les païens ici et là, et je vois qu'ils abondent en richesses et ne pensent pas à tes commandements.
\par 34 Pese donc maintenant notre méchanceté dans la balance, et celle aussi des habitants du monde ; et ton nom ne se trouvera nulle part ailleurs qu'en Israël.
\par 35 Ou quand est-ce que les habitants de la terre n'ont pas péché à tes yeux ? ou quel peuple a ainsi gardé tes commandements ?
\par 36 Tu découvriras qu'Israël, nommément, a gardé tes préceptes ; mais pas les païens.

\chapter{4}

\par 1 Et l'ange qui m'a été envoyé, dont le nom était Uriel, m'a répondu :
\par 2 Et il dit : Ton cœur est allé trop loin dans ce monde, et penses-tu comprendre la voie du Très-Haut ?
\par 3 Alors je dis : Oui, mon seigneur. Et il me répondit, et dit : Je suis envoyé pour te montrer trois voies et pour te présenter trois similitudes :
\par 4 Si tu peux m'en déclarer un, je te montrerai aussi le chemin que tu désires voir, et je te montrerai d'où vient le cœur méchant.
\par 5 Et j'ai dit : Parlez, mon seigneur. Alors il me dit : Va, pèse-moi le poids du feu, ou mesure-moi le souffle du vent, ou rappelle-moi le jour qui est passé.
\par 6 Alors je répondis et dis : Quel homme est capable de faire cela, pour que tu me demandes de telles choses ?
\par 7 Et il me dit : Si je te demandais combien il y a de grandes habitations au milieu de la mer, ou combien y a-t-il de sources au commencement de l'abîme, ou combien de sources il y a au-dessus du firmament, ou quelles sont les sources sorties de paradis :
\par 8 Peut-être me dirais-tu : Je ne suis jamais descendu dans l'abîme, ni encore en enfer, et je ne suis jamais non plus monté au ciel.
\par 9 Mais maintenant je ne t'ai interrogé que sur le feu et le vent, et sur le jour que tu as traversé, et sur les choses dont tu ne peux pas être séparé, et pourtant tu ne peux pas me répondre.
\par 10 Il me dit en outre : Tu ne peux pas connaître tes propres choses et celles qui ont grandi avec toi ;
\par 11 Comment ton vaisseau pourrait-il alors comprendre la voie du Très-Haut, et, le monde étant maintenant extérieurement corrompu, comprendre la corruption qui est évidente à mes yeux ?
\par 12 Alors je lui dis : Il vaudrait mieux que nous ne le soyons pas du tout, plutôt que de vivre encore dans la méchanceté et de souffrir sans savoir pourquoi.
\par 13 Il me répondit et dit : Je suis allé dans une forêt dans une plaine, et les arbres se sont concertés,
\par 14 Et il dit : Venez, allons faire la guerre à la mer, afin qu'elle s'éloigne devant nous, et que nous puissions nous créer davantage de bois.
\par 15 Les flots de la mer prirent également conseil et dirent : Venez, montons et soumettons les bois de la plaine, afin que là aussi nous puissions faire de nous un autre pays.
\par 16 La pensée du bois était vaine, car le feu venait et le consumait.
\par 17 La pensée des crues de la mer fut également vaine, car le sable se dressa et les arrêta.
\par 18 Si tu étais maintenant juge entre ces deux-là, qui commencerais-tu à justifier ? ou qui condamnerais-tu ?
\par 19 Je répondis et dis : En vérité, c'est une pensée insensée qu'ils ont tous deux imaginée, car le sol est donné aux bois, et la mer aussi a sa place pour supporter ses inondations.
\par 20 Alors il me répondit et dit : Tu as rendu un jugement juste, mais pourquoi ne te juges-tu pas aussi toi-même ?
\par 21 Car, comme le sol est livré au bois, et la mer à ses flots, de même ceux qui habitent sur la terre ne peuvent comprendre que ce qui est sur la terre, et ceux qui habitent au-dessus des cieux ne peuvent comprendre que ce qui est au-dessus de la hauteur des cieux.
\par 22 Alors je répondis et dis : Je te prie, Seigneur, laisse-moi comprendre.
\par 23 Car je n'étais pas curieux des choses élevées, mais de celles qui passent chaque jour près de nous, à savoir pourquoi Israël est livré en opprobre aux païens, et pour quelle raison le peuple que tu as aimé est livrée aux nations impies, et pourquoi la loi de nos ancêtres est réduite à néant, et les alliances écrites sont sans effet,
\par 24 Et nous disparaissons du monde comme des sauterelles, et notre vie est étonnement et crainte, et nous ne sommes pas dignes d'obtenir miséricorde.
\par 25 Que fera-t-il alors à son nom par lequel nous sommes appelés ? de ces choses ai-je demandé.
\par 26 Alors il me répondit et dit : Plus tu cherches, plus tu t'étonneras ; car le monde s'empresse de passer,
\par 27 Et il ne peut pas comprendre les choses qui sont promises aux justes dans les temps à venir : car ce monde est plein d'injustice et d'infirmités.
\par 28 Mais quant aux choses que tu me demandes, je te le dirai ; car le mal a été semé, mais sa destruction n'est pas encore venue.
\par 29 Si donc ce qui a été semé n'est pas renversé, et si le lieu où le mal a été semé ne disparaît pas, alors ce qui est semé de bien ne peut pas venir.
\par 30 Car le grain de mauvaise semence a été semé dans le cœur d'Adam dès le commencement, et combien d'impiété a-t-il suscité jusqu'à présent ? et combien produira-t-il encore jusqu'à ce que vienne le temps du battage ?
\par 31 Réfléchis maintenant par toi-même, combien de fruits de méchanceté le grain de la mauvaise semence a produit.
\par 32 Et quand les épis seront coupés, qui sont innombrables, quelle superficie rempliront-ils ?
\par 33 Alors je répondis et dis : Comment et quand ces choses arriveront-elles ? pourquoi nos années sont-elles rares et mauvaises ?
\par 34 Et il me répondit, disant : Ne te hâte pas au-dessus du Très-Haut ; car ta hâte est vaine d'être au-dessus de lui, car tu as largement dépassé.
\par 35 Les âmes des justes ne posaient-elles pas aussi ces questions dans leurs chambres, en disant : Jusqu'à quand espérerai-je de cette façon ? quand viendra le fruit de l’aire de notre récompense ?
\par 36 Et à ces choses Uriel, l'archange, leur répondit et dit : Même lorsque le nombre de graines est rempli en vous, car il a pesé le monde dans la balance.
\par 37 Avec mesure il a mesuré les temps ; et il a compté les temps par nombre ; et il ne les bouge ni ne les remue jusqu'à ce que ladite mesure soit accomplie.
\par 38 Alors je répondis et dis : Seigneur, qui gouvernes, nous sommes tous pleins d'impiété.
\par 39 Et c'est peut-être à cause de nous que les aires des justes ne sont pas remplies, à cause des péchés des habitants de la terre.
\par 40 Alors il me répondit et dit : Va vers une femme enceinte, et demande-lui, quand elle aura accompli ses neuf mois, si son ventre peut garder encore en elle l'enfantement.
\par 41 Alors je dis : Non, Seigneur, cela ne peut pas être le cas. Et il me dit : Dans le tombeau, les chambres des âmes sont comme le ventre d'une femme.
\par 42 Car, comme une femme qui enfante se hâte d'échapper à la nécessité de l'enfantement, de même ces lieux se hâtent de livrer ce qui leur est confié.
\par 43 Dès le commencement, regarde, ce que tu désires voir, cela te sera montré.
\par 44 Alors je répondis et dis : Si j'ai trouvé grâce à tes yeux, et si cela est possible, et si je suis donc convenable,
\par 45 Montrez-moi donc s'il y aura plus à venir que ce qui est passé, ou s'il y aura plus de passé que ce qui est à venir.
\par 46 Ce qui est passé, je le sais, mais ce qui est à venir, je l'ignore.
\par 47 Et il me dit : Lève-toi du côté droit, et je t'exposerai la similitude.
\par 48 Alors je me tenais là, et je vis, et voici, un four brûlant passait devant moi; et il arriva que lorsque la flamme disparut, je regardai, et voici, la fumée resta silencieuse.
\par 49 Après cela, une nuée d'eau passa devant moi, et fit tomber beaucoup de pluie avec une tempête ; et quand la pluie orageuse fut passée, les gouttes restèrent immobiles.
\par 50 Alors il me dit : Réfléchis avec toi-même ; comme la pluie est plus grande que les gouttes, et comme le feu est plus grand que la fumée ; mais les gouttes et la fumée restent en arrière : ainsi la quantité qui est passée n'a plus dépassé.
\par 51 Alors j'ai prié et j'ai dit : Puis-je vivre, penses-tu, jusqu'à ce moment-là ? ou que se passera-t-il en ces jours-là ?
\par 52 Il me répondit et dit : Quant aux signes que tu me demandes, je peux t'en parler en partie ; mais quant à ta vie, je ne suis pas envoyé pour te le montrer ; car je ne le sais pas.

\chapter{5}

\par 1 Néanmoins, à mesure que les signes arrivent, voici, les jours viendront où ceux qui habitent sur la terre seront pris en grand nombre, et le chemin de la vérité sera caché, et le pays sera stérile de foi.
\par 2 Mais l'iniquité augmentera au-delà de ce que tu vois maintenant, ou de ce que tu as entendu autrefois.
\par 3 Et le pays que tu vois maintenant avoir des racines, tu le verras subitement dévasté.
\par 4 Mais si le Très-Haut t'accorde la vie, tu verras après la troisième trompette que le soleil brillera de nouveau pendant la nuit, et la lune trois fois pendant le jour.
\par 5 Et le sang coulera du bois, et la pierre donnera sa voix, et le peuple sera troublé.
\par 6 Et même celui qui habite sur la terre sera dominé par ceux qu'ils n'attendent pas, et les oiseaux s'envoleront ensemble :
\par 7 Et la mer de Sodomie rejettera les poissons et fera pendant la nuit un bruit que beaucoup n'ont pas connu, mais ils en entendront tous la voix.
\par 8 Il y aura aussi une confusion en plusieurs endroits, et le feu se rallumera souvent, et les bêtes sauvages changeront de place, et les femmes menstruées enfanteront des monstres.
\par 9 Et l'eau salée se trouvera dans l'eau douce, et tous les amis se détruiront les uns les autres ; alors l'esprit se cachera, et l'intelligence se retirera dans sa chambre secrète,
\par 10 Et ils seront recherchés par beaucoup, et pourtant ils ne seront pas trouvés : alors l'injustice et l'incontinence se multiplieront sur la terre.
\par 11 Un pays aussi interrogera un autre et dira : La justice qui rend un homme juste est-elle passée par toi ? Et il dira : Non.
\par 12 En même temps, les hommes espèreront, mais n'obtiendront rien : ils travailleront, mais leurs voies ne prospéreront pas.
\par 13 Pour te montrer de tels signes, j'ai la permission ; et si tu pries encore, et pleures comme maintenant, et jeûnes même plusieurs jours, tu entendras des choses encore plus grandes.
\par 14 Alors je me suis réveillé, et une peur extrême a parcouru tout mon corps, et mon esprit a été troublé, au point qu'il s'est évanoui.
\par 15 Alors l'ange qui était venu me parler m'a retenu, m'a consolé et m'a mis debout.
\par 16 Et la deuxième nuit, Salathiel, le chef du peuple, vint vers moi et me dit : Où étais-tu ? et pourquoi ton visage est-il si lourd ?
\par 17 Ne sais-tu pas qu'Israël t'est confié dans le pays de sa captivité ?
\par 18 Lève-toi donc, mange du pain, et ne nous abandonne pas, comme le berger qui laisse son troupeau entre les mains de loups cruels.
\par 19 Alors je lui dis : Éloigne-toi de moi et ne t'approche pas de moi. Et il entendit ce que je disais et s'éloigna de moi.
\par 20 Et ainsi je jeûnai sept jours, en deuil et en pleurant, comme me l'avait ordonné l'ange Uriel.
\par 21 Et après sept jours, les pensées de mon cœur me furent de nouveau très pénibles,
\par 22 Et mon âme recouvra l'esprit de compréhension, et je recommençai à parler avec le Très-Haut,
\par 23 Et il dit : Ô Seigneur, qui gouvernes, de tous les bois de la terre et de tous ses arbres, tu t'as choisi une seule vigne :
\par 24 Et de toutes les terres du monde entier tu t'es choisi un seul noyau, et de toutes ses fleurs un lis :
\par 25 Et de tous les abîmes de la mer tu t'es rempli d'un seul fleuve, et de toutes les villes bâties tu t'es sanctifiée à Sion.
\par 26 Et de tous les oiseaux qui ont été créés tu t'as nommé une colombe ; et de tous les bovins qui ont été créés tu t'as pourvu d'une seule brebis :
\par 27 Et parmi toutes les multitudes du peuple tu t'es acquis un seul peuple ; et à ce peuple que tu as aimé, tu as donné une loi qui est approuvée par tous.
\par 28 Et maintenant, ô Seigneur, pourquoi as-tu livré ce seul peuple à plusieurs ? et sur une seule racine tu en as préparé d'autres, et pourquoi as-tu dispersé ton seul peuple parmi plusieurs ?
\par 29 Et ceux qui ont contredit tes promesses et qui n'ont pas cru à tes alliances les ont foulés aux pieds.
\par 30 Si tu haïssais tant ton peuple, tu devrais le punir de tes propres mains.
\par 31 Après que j'eus prononcé ces paroles, l'ange qui était venu vers moi la nuit précédente m'a été envoyé,
\par 32 Et il me dit : Écoute-moi, et je t'instruirai ; écoute ce que je dis, et je t'en dirai davantage.
\par 33 Et je dis : Continuez à parler, mon Seigneur. Alors il me dit : Tu as l'esprit très troublé à cause d'Israël ; aimes-tu ce peuple plus que celui qui l'a fait ?
\par 34 Et j'ai dit : Non, Seigneur ; mais j'ai parlé d'une grande douleur ; car mes reins me font souffrir à chaque heure, tandis que je travaille à comprendre la voie du Très-Haut et à rechercher une partie de son jugement.
\par 35 Et il me dit : Tu ne peux pas. Et j'ai dit : Pourquoi, Seigneur ? Où suis-je donc né ? ou pourquoi le ventre de ma mère n'était-il pas alors mon tombeau, afin que je n'aie pas vu le travail de Jacob et le labeur fastidieux de la race d'Israël ?
\par 36 Et il me dit : Compte-moi les choses qui ne sont pas encore arrivées, rassemble-moi les scories qui sont dispersées, rends-moi vertes les fleurs qui sont fanées,
\par 37 Ouvre-moi les lieux qui sont fermés, et fais-moi sortir les vents qui y sont enfermés, montre-moi l'image d'une voix : et alors je t'annoncerai ce que tu t'efforces de connaître.
\par 38 Et je dis : Seigneur, qui gouvernes, qui connaît ces choses, sinon celui qui n'a pas sa demeure avec les hommes ?
\par 39 Quant à moi, je suis insensé : comment pourrais-je donc parler de ces choses dont tu me demandes ?
\par 40 Alors il me dit : De même que tu ne peux faire aucune de ces choses dont j'ai parlé, de même tu ne peux pas découvrir mon jugement, ni finalement l'amour que j'ai promis à mon peuple.
\par 41 Et je dis : Voici, Seigneur, tu es encore proche de ceux qui seront réservés jusqu'à la fin ; et que feront ceux qui ont été avant moi, ou nous qui sommes maintenant, ou ceux qui viendront après nous ?
\par 42 Et il me dit : Je comparerai mon jugement à un anneau ; comme il n'y a pas de lenteur dans le dernier, de même il n'y a pas de rapidité dans le premier.
\par 43 Alors je répondis et dis : Ne pourrais-tu pas faire immédiatement ceux qui ont été faits et qui existent maintenant, et ceux qui doivent venir ? afin que tu puisses montrer ton jugement plus tôt ?
\par 44 Alors il me répondit et dit : La créature ne doit pas se précipiter au-dessus de celui qui l'a créé ; le monde ne peut pas non plus retenir immédiatement ceux qui y seront créés.
\par 45 Et je dis : Comme tu l'as dit à ton serviteur, toi qui donnes la vie à tous, tu as donné la vie aussitôt à la créature que tu as créée, et la créature l'a enfantée : de même, elle pourrait maintenant aussi enfanter ceux qui sont maintenant présents immédiatement.
\par 46 Et il me dit : Interroge le ventre d'une femme, et dis-lui : Si tu enfantes des enfants, pourquoi ne le fais-tu pas ensemble, mais l'un après l'autre ? priez-la donc de donner naissance à dix enfants à la fois.
\par 47 Et j'ai dit : Elle ne peut pas : mais doit le faire en s'éloignant du temps.
\par 48 Alors il me dit : De même, j'ai donné le sein de la terre à ceux qui y seront semés en leur temps.
\par 49 Car, comme un jeune enfant ne peut pas enfanter ce qui appartient aux vieillards, ainsi j'ai disposé le monde que j'ai créé.
\par 50 Et je demandai, et je dis : Puisque tu m'as maintenant indiqué le chemin, je vais parler devant toi ; car notre mère, dont tu m'as dit qu'elle était jeune, approche maintenant de l'âge.
\par 51 Il me répondit et dit : Interroge une femme qui enfante, et elle te le dira.
\par 52 Dis-lui : Pourquoi ceux que tu as enfantés maintenant sont-ils semblables à ceux qui étaient auparavant, mais de moindre taille ?
\par 53 Et elle te répondra : Ceux qui naissent dans la force de la jeunesse sont d'une certaine manière, et ceux qui naissent dans la vieillesse, lorsque l'utérus manque, sont d'une autre manière.
\par 54 Considérez donc aussi que vous êtes de moindre taille que ceux qui étaient avant vous.
\par 55 Et ainsi ceux qui vous suivent sont moins nombreux que vous, comme les créatures qui commencent maintenant à être vieilles et qui ont perdu la force de la jeunesse.
\par 56 Alors je dis : Seigneur, je te prie, si j'ai trouvé grâce à tes yeux, montre à ton serviteur par qui tu visites ta créature.

\chapter{6}

\par 1 Et il me dit : Au commencement, quand la terre fut créée, avant que les frontières du monde existaient, ou que les vents soufflaient,
\par 2 Avant qu'il ne tonne et qu'il ne s'éclaire, ou que les fondations du paradis ne soient posées,
\par 3 Avant que les belles fleurs fussent vues, ou que les puissances mobiles fussent établies, avant que la multitude innombrable des anges ne fût rassemblée,
\par 4 Jamais les hauteurs de l'air ne s'élevèrent avant que les mesures du firmament ne fussent nommées, ni jamais les cheminées de Sion furent chaudes,
\par 5 Et avant que les années présentes aient été recherchées, et que les inventions de ceux qui sont maintenant péchés aient été transformées, avant d'être scellées, ceux qui ont rassemblé la foi pour un trésor :
\par 6 Alors j'ai réfléchi à ces choses, et elles ont toutes été faites par moi seul et par personne d'autre : par moi aussi elles seront terminées et par personne d'autre.
\par 7 Alors je répondis et dis : Quelle sera la séparation des temps ? ou quand sera la fin du premier, et le début de celui qui suit ?
\par 8 Et il me dit : Depuis Abraham jusqu'à Isaac, lorsque Jacob et Ésaü naquirent de lui, la main de Jacob tenait d'abord le talon d'Ésaü.
\par 9 Car Ésaü est la fin du monde, et Jacob est le commencement de celui qui suit.
\par 10 La main de l'homme est entre le talon et la main : autre question, Esdras, ne la pose pas.
\par 11 Je répondis alors et dis : Ô Seigneur, qui gouvernes, si j'ai trouvé grâce à tes yeux,
\par 12 Je te prie, montre à ton serviteur la fin de tes gages, dont tu m'as montré une partie la nuit dernière.
\par 13 Alors il répondit et me dit : Lève-toi, et entends une voix puissante.
\par 14 Et ce sera comme un grand mouvement ; mais le lieu où tu te tiens ne sera pas déplacé.
\par 15 Et c'est pourquoi, quand elle parle, n'ayez pas peur ; car la parole est de la fin, et le fondement de la terre est compris.
\par 16 Et pourquoi ? parce que le discours de ces choses tremble et est ému, car il sait que la fin de ces choses doit être changée.
\par 17 Et il arriva que, après avoir entendu cela, je me levai et j'écoutai, et voici, il y eut une voix qui parla, et son bruit était comme le bruit de grandes eaux.
\par 18 Et il dit : Voici, les jours viennent où je commencerai à approcher et à visiter ceux qui habitent sur la terre,
\par 19 Et il commencera à les inquisitionner pour savoir quels sont ceux qui ont blessé injustement par leur injustice, et quand l'affliction de Sion sera accomplie ;
\par 20 Et quand le monde, qui commencera à disparaître, sera terminé, alors je montrerai ces signes : les livres seront ouverts devant le firmament, et ils verront tous ensemble :
\par 21 Et les enfants d'un an parleront avec leurs voix, les femmes enceintes enfanteront des enfants prématurés de trois ou quatre mois, et ils vivront et seront ressuscités.
\par 22 Et soudain les lieux ensemencés paraîtront non ensemencés, les greniers pleins seront soudain trouvés vides.
\par 23 Et la trompette donnera un son qui, lorsque chacun l'entendra, sera soudainement effrayé.
\par 24 En ce temps-là, les amis se battront les uns contre les autres comme des ennemis, et la terre sera dans la crainte avec ceux qui l'habitent, les sources des fontaines s'arrêteront, et dans trois heures elles ne couleront plus.
\par 25 Quiconque restera de tout ce que je t'ai dit échappera et verra mon salut et la fin de ton monde.
\par 26 Et les hommes qui seront reçus le verront, ceux qui n'ont pas goûté la mort dès leur naissance ; et le cœur des habitants sera changé et tourné vers une autre signification.
\par 27 Car le mal sera éteint, et la tromperie sera éteinte.
\par 28 Quant à la foi, elle fleurira, la corruption sera vaincue, et la vérité, qui est restée si longtemps sans fruit, sera déclarée.
\par 29 Et pendant qu'il me parlait, voici, je regardais peu à peu celui devant lequel je me tenais.
\par 30 Et il me dit ces paroles : Je viens te montrer l'heure de la nuit à venir.
\par 31 Si tu pries encore davantage et jeûnes encore sept jours, je te dirai chaque jour des choses plus grandes que celles que j'ai entendues.
\par 32 Car ta voix est entendue devant le Très-Haut ; car le Puissant a vu ta justice, il a vu aussi ta chasteté, que tu as depuis ta jeunesse.
\par 33 C'est pourquoi il m'a envoyé pour te montrer toutes ces choses et pour te dire : Rassure-toi et ne crains rien.
\par 34 Et ne te hâte pas, avec les temps passés, de penser à des choses vaines, afin de ne pas te hâter des derniers temps.
\par 35 Et il arriva après cela que je pleurai de nouveau, et je jeûnai sept jours de la même manière, afin d'accomplir les trois semaines qu'il m'avait dit.
\par 36 Et la huitième nuit, mon cœur fut de nouveau tourmenté au dedans de moi, et je me mis à parler devant le Très-Haut.
\par 37 Car mon esprit était très enflammé, et mon âme était dans la détresse.
\par 38 Et je dis : Seigneur, tu as parlé dès le commencement de la création, même le premier jour, et tu as dit ainsi : Que le ciel et la terre soient faits ; et ta parole était une œuvre parfaite.
\par 39 Et alors l'esprit était, et les ténèbres et le silence étaient de tous côtés ; le son de la voix humaine n’était pas encore formé.
\par 40 Alors tu as ordonné qu'une belle lumière sorte de tes trésors, afin que ton œuvre paraisse.
\par 41 Le deuxième jour, tu as créé l'esprit du firmament, et tu lui as ordonné de se séparer et de faire une division entre les eaux, afin qu'une partie puisse monter et que l'autre reste en dessous.
\par 42 Le troisième jour, tu as ordonné que les eaux soient rassemblées dans la septième partie de la terre ; tu as mis à sec six parcelles, et tu les as gardées, afin que quelques-unes d'entre elles, plantées par Dieu et cultivées, puissent te servir.
\par 43 Car aussitôt que ta parole fut sortie, l'ouvrage fut fait.
\par 44 Car aussitôt il y eut des fruits grands et innombrables, et des plaisirs gustatifs nombreux et divers, et des fleurs d'une couleur immuable, et des odeurs d'une odeur merveilleuse : et cela se fit le troisième jour.
\par 45 Le quatrième jour, tu as ordonné que le soleil brille, que la lune éclaire et que les étoiles soient en ordre.
\par 46 Et il leur donna la mission de rendre le service à l'homme, ce qui devait être fait.
\par 47 Le cinquième jour, tu as dit à la septième partie, là où les eaux étaient recueillies, qu'elle donnerait naissance à des êtres vivants, des oiseaux et des poissons : et ainsi il arriva.
\par 48 Car l'eau muette et sans vie a donné naissance à des êtres vivants sur l'ordre de Dieu, afin que tous puissent louer tes merveilles.
\par 49 Alors tu as ordonné deux êtres vivants, l'un que tu as appelé Enoch, et l'autre Léviathan ;
\par 50 Et tu as séparé l'un de l'autre, car la septième partie, c'est-à-dire là où l'eau était rassemblée, ne pouvait pas les retenir tous deux.
\par 51 Tu as donné à Hénoc une partie qui était desséchée le troisième jour, pour qu'il habite dans la même partie, où se trouvent mille collines.
\par 52 Mais tu as donné à Léviathan la septième partie, à savoir l'humide ; et tu l'as gardé pour qu'il soit dévoré de qui tu veux et quand.
\par 53 Le sixième jour, tu as ordonné à la terre de produire devant toi des bêtes, du bétail et des reptiles.
\par 54 Et après ceux-ci aussi Adam, que tu as établi seigneur de toutes tes créatures : de lui nous sommes tous issus, et aussi le peuple que tu as choisi.
\par 55 Tout cela, je l'ai dit devant toi, Seigneur, parce que tu as créé le monde pour nous.
\par 56 Quant aux autres peuples, qui sont aussi issus d'Adam, tu as dit qu'ils ne sont rien, mais qu'ils sont semblables à de la salive ; et tu as comparé leur abondance à une goutte qui tombe d'un vase.
\par 57 Et maintenant, Seigneur, voici, ces païens, qui ont toujours été réputés pour rien, ont commencé à être nos seigneurs et à nous dévorer.
\par 58 Mais nous, ton peuple, que tu as appelé ton premier-né, ton unique engendré et ton fervent amant, sommes livrés entre leurs mains.
\par 59 Si le monde est maintenant créé pour nous, pourquoi ne possédons-nous pas un héritage avec le monde ? combien de temps cela va-t-il durer ?

\chapter{7}

\par 1 Et quand j'eus fini de prononcer ces paroles, l'ange qui m'avait été envoyé les nuits précédentes me fut envoyé :
\par 2 Et il me dit : Lève-toi, Esdras, et écoute les paroles que je suis venu te dire.
\par 3 Et j'ai dit : Parle, mon Dieu. Alors il me dit : La mer est située dans un endroit large, afin qu'elle soit profonde et grande.
\par 4 Mais disons que l'entrée était étroite et comme une rivière ;
\par 5 Qui donc pourrait aller dans la mer pour la contempler et la gouverner ? s'il ne passait pas par le étroit, comment pourrait-il entrer dans le large ?
\par 6 Il y a aussi autre chose ; Une ville est bâtie et située sur un vaste champ, et elle est pleine de toutes bonnes choses :
\par 7 Son entrée est étroite, et elle est placée dans un endroit dangereux pour tomber, comme s'il y avait un feu à droite, et à gauche une eau profonde.
\par 8 Et un seul chemin entre eux deux, même entre le feu et l'eau, si petit qu'il ne pouvait y avoir qu'un seul homme à la fois.
\par 9 Si cette ville était maintenant donnée en héritage à un homme, s'il ne surmonte jamais le danger qui l'attend, comment recevra-t-il cet héritage ?
\par 10 Et j'ai dit : Il en est ainsi, Seigneur. Alors il me dit : De même est la part d'Israël.
\par 11 Parce que c'est pour eux que j'ai créé le monde ; et quand Adam a transgressé mes statuts, alors il a été décrété que maintenant c'était fait.
\par 12 Alors les entrées de ce monde furent rendues étroites, pleines de tristesse et de travail : elles sont peu nombreuses et mauvaises, pleines de périls et très douloureuses.
\par 13 Car les entrées du monde ancien étaient larges et sûres, et portaient des fruits immortels.
\par 14 Si donc ceux qui vivent ne s'efforcent pas d'entrer dans ces choses étroites et vaines, ils ne pourront jamais recevoir celles qui leur sont réservées.
\par 15 Maintenant donc, pourquoi t'inquiètes-tu, puisque tu n'es qu'un homme corruptible ? et pourquoi es-tu ému, alors que tu n'es que mortel ?
\par 16 Pourquoi n'as-tu pas pensé à ce qui est à venir, plutôt qu'à ce qui est présent ?
\par 17 Alors je répondis et dis : Ô Seigneur, qui gouvernes, tu as ordonné dans ta loi que les justes hériteraient de ces choses, mais que les impies périraient.
\par 18 Mais les justes souffriront des choses difficiles et espèrent de grandes choses ; car ceux qui ont fait le mal ont souffert des choses difficiles, et pourtant ils ne verront pas les choses larges.
\par 19 Et il me dit : Il n’y a pas de juge au-dessus de Dieu, et aucun n’a une intelligence au-dessus du Très-Haut.
\par 20 Car nombreux sont ceux qui périssent dans cette vie, parce qu'ils méprisent la loi de Dieu qui leur est présentée.
\par 21 Car Dieu a donné des commandements stricts à ceux qui sont venus, ce qu'ils doivent faire pour vivre, comme ils sont venus, et ce qu'ils doivent observer pour éviter le châtiment.
\par 22 Mais ils ne lui obéirent pas ; mais il parlait contre lui et imaginait des choses vaines ;
\par 23 Et ils se sont trompés par leurs mauvaises actions ; et il a dit du Très-Haut qu'il ne l'est pas ; et il ne connaissait pas ses voies :
\par 24 Mais ils ont méprisé sa loi et renié ses alliances ; ils n'ont pas été fidèles à ses statuts et n'ont pas accompli ses œuvres.
\par 25 Et c'est pourquoi, Esdras, car le vide sont les choses vides, et pour le plein sont les choses pleines.
\par 26 Voici, le temps viendra où ces signes que je t'ai annoncés s'accompliront, et l'épouse apparaîtra, et on verra sortir celle qui est maintenant retirée de la terre.
\par 27 Et quiconque est délivré des maux susdits verra mes merveilles.
\par 28 Car mon fils Jésus se révélera avec ceux qui sont avec lui, et ceux qui resteront se réjouiront dans quatre cents ans.
\par 29 Après ces années, mon fils Christ mourra, ainsi que tous les hommes qui ont la vie.
\par 30 Et le monde sera ramené à l'ancien silence pendant sept jours, comme lors des jugements précédents, afin que personne ne reste.
\par 31 Et après sept jours, le monde qui ne se réveille pas encore se relèvera, et celui qui est corrompu mourra.
\par 32 Et la terre restaurera ceux qui dorment en elle, et ainsi la poussière ceux qui habitent dans le silence, et les lieux secrets délivreront les âmes qui leur étaient confiées.
\par 33 Et le Très-Haut apparaîtra sur le siège du jugement, et la misère passera, et les longues souffrances prendront fin.
\par 34 Mais seul le jugement subsistera, la vérité subsistera et la foi deviendra forte :
\par 35 Et l'œuvre suivra, et la récompense sera annoncée, et les bonnes actions seront fortes, et les mauvaises actions ne subiront aucune règle.
\par 36 Alors je dis : Abraham a prié d'abord pour les Sodomites, et Moïse pour les pères qui ont péché dans le désert :
\par 37 Et Jésus après lui pour Israël au temps d'Acan :
\par 38 Et Samuel et David pour la destruction, et Salomon pour ceux qui devaient venir au sanctuaire :
\par 39 Et Hélias pour ceux qui ont reçu de la pluie ; et pour les morts, afin qu'il vive :
\par 40 Et Ezéchias pour le peuple du temps de Sennachérib : et plusieurs pour plusieurs.
\par 41 Ainsi donc maintenant, puisque la corruption s'est développée, et la méchanceté s'est accrue, et que les justes ont prié pour les impies : pourquoi n'en serait-il pas ainsi maintenant aussi ?
\par 42 Il me répondit et dit : Cette vie présente n'est pas la fin où demeure beaucoup de gloire ; c'est pourquoi ils ont prié pour les faibles.
\par 43 Mais le jour de jugement sera la fin de ce temps et le commencement de l'immortalité à venir, où la corruption est passée,
\par 44 L'intempérance est terminée, l'infidélité est retranchée, la justice grandit et la vérité surgit.
\par 45 Alors personne ne pourra sauver celui qui est détruit, ni opprimer celui qui a remporté la victoire.
\par 46 Je répondis alors et dis : Ceci est ma première et dernière parole, qu'il valait mieux ne pas avoir donné la terre à Adam, ou bien, lorsqu'elle lui fut donnée, l'empêcher de pécher.
\par 47 Car quel profit y a-t-il pour les hommes maintenant, dans le temps présent, à vivre dans la lourdeur, et après la mort à attendre le châtiment ?
\par 48 Ô toi Adam, qu'as-tu fait ? car même si c'est toi qui as péché, tu n'es pas tombé seul, mais nous tous qui sommes issus de toi.
\par 49 Car à quoi nous sert-il, si l'on nous promet un temps immortel, alors que nous avons fait les œuvres qui amènent la mort ?
\par 50 Et qu'on nous promet une espérance éternelle, alors que nous-mêmes, étant les plus méchants, sommes rendus vains ?
\par 51 Et qu'on nous réserve des habitations saines et sûres, alors que nous avons vécu méchamment ?
\par 52 Et que la gloire du Très-Haut est gardée pour défendre ceux qui ont mené une vie prudente, alors que nous avons marché dans les voies les plus mauvaises de toutes ?
\par 53 Et qu'il soit montré un paradis dont le fruit dure à jamais, où sont la sécurité et la médecine, puisque nous n'y entrerons pas ?
\par 54 (Car nous avons marché dans des endroits désagréables.)
\par 55 Et que les visages de ceux qui ont eu recours à l'abstinence brilleront au-dessus des étoiles, tandis que nos visages seront plus noirs que les ténèbres ?
\par 56 Car, pendant que nous vivions et commettions l'iniquité, nous ne pensions pas que nous devrions commencer à en souffrir après la mort.
\par 57 Alors il me répondit et dit : Telle est la condition du combat que l'homme né sur la terre combattra ;
\par 58 S'il est vaincu, il souffrira comme tu l'as dit ; mais s'il remporte la victoire, il recevra ce que je dis.
\par 59 Car c'est ici la vie dont Moïse parlait au peuple pendant qu'il vivait, disant : Choisissez-vous la vie, afin que vous viviez.
\par 60 Mais ils ne l'ont pas cru, ni les prophètes après lui, ni moi qui leur ai parlé,
\par 61 Afin qu'il n'y ait pas dans leur destruction une telle lourdeur, qu'il y aura de la joie pour ceux qui sont persuadés de se sauver.
\par 62 Je répondis alors et dis : Je sais, Seigneur, que le Très-Haut est appelé miséricordieux, en ce sens qu'il a pitié de ceux qui ne sont pas encore venus au monde,
\par 63 Et contre ceux aussi qui se tournent vers sa loi ;
\par 64 Et qu'il est patient, et qu'il souffre longtemps ceux qui ont péché, comme ses créatures ;
\par 65 Et qu'il est généreux, car il est prêt à donner là où il en a besoin ;
\par 66 Et qu'il est d'une grande miséricorde, car il multiplie de plus en plus de miséricordes envers ceux qui sont présents et ceux qui sont passés, et aussi envers ceux qui sont à venir.
\par 67 Car s'il ne multipliait pas ses miséricordes, le monde ne subsisterait pas avec ceux qui en héritent.
\par 68 Et il pardonne ; car s'il ne le faisait pas par bonté, afin que ceux qui ont commis des iniquités en soient soulagés, la dix millième partie des hommes ne resterait pas en vie.
\par 69 Et étant juge, s'il ne pardonne pas à ceux qui sont guéris par sa parole et n'éteint pas la multitude de contestations,
\par 70 Il devrait en rester très peu, par hasard, dans une multitude innombrable.

\chapitre{8}

\par 1 Et il me répondit, disant : Le Très-Haut a fait ce monde pour beaucoup, mais le monde à venir pour quelques-uns.
\par 2 Je vais te raconter une similitude, Esdras ; Comme lorsque tu interrogeras la terre, elle te dira qu'elle donne beaucoup de moisissure dont sont faits les vases de terre, mais peu de poussière dont sort l'or : ainsi est le cours de ce monde présent.
\par 3 Il y en aura beaucoup qui seront créés, mais peu seront sauvés.
\par 4 Alors je répondis et dis : Avale donc, ô mon âme intelligente, et dévore la sagesse.
\par 5 Car tu as accepté de prêter l'oreille, et tu es disposé à prophétiser ; car tu n'as plus d'espace que pour vivre seulement.
\par 6 O Seigneur, si tu ne permets pas à ton serviteur que nous puissions prier devant toi, et que tu nous donnes de la semence à notre cœur et de la culture à notre intelligence, afin qu'il en vienne du fruit ; Comment vivra tout homme corrompu, qui porte la place d’un homme ?
\par 7 Car tu es seul, et nous sommes tous un ouvrage de tes mains, comme tu l'as dit.
\par 8 Car lorsque le corps est maintenant façonné dans le sein de la mère, et que tu lui donnes des membres, ta créature est préservée dans le feu et l'eau, et ton ouvrage dure neuf mois ta créature qui est créée en elle.
\par 9 Mais ce qui garde et ce qui est gardé sera tous deux préservé ; et le moment venu, l'utérus préservé livrera les choses qui ont grandi en lui.
\par 10 Car tu as ordonné que l'on donne du lait, qui est le fruit des seins, des parties du corps, c'est-à-dire des seins,
\par 11 Afin que la chose façonnée soit nourrie pendant un certain temps, jusqu'à ce que tu la disposes à ta miséricorde.
\par 12 Tu l'as élevé par ta justice, tu l'as nourri dans ta loi, et tu l'as réformé par ton jugement.
\par 13 Et tu la mortifieras comme ta créature, et tu la vivras comme ton œuvre.
\par 14 Si donc tu détruis celui qui a été façonné avec tant de travail, il est facile d'être ordonné par ton commandement, afin que la chose qui a été faite soit préservée.
\par 15 Maintenant donc, Seigneur, je vais parler ; touchant l’homme en général, tu le sais mieux ; mais toucher ton peuple, pour le bien de qui je suis désolé ;
\par 16 Et pour ton héritage, pour la cause duquel je pleure ; et pour Israël, pour qui je suis lourd ; et pour Jacob, à cause de qui je suis troublé ;
\par 17 C'est pourquoi je commencerai à prier devant toi pour moi et pour eux, car je vois la chute de nous qui habitons dans le pays.
\par 18 Mais j'ai entendu la rapidité du juge qui doit venir.
\par 19 C'est pourquoi écoute ma voix, et comprends mes paroles, et je parlerai devant toi. C'est le début des paroles d'Esdras, avant qu'il ne soit enlevé : et je dis :
\par 20 O Seigneur, toi qui habites dans l'éternité, qui vois d'en haut les choses dans le ciel et dans les airs ;
\par 21 Dont le trône est inestimable ; dont la gloire ne peut être comprise; devant lequel se tiennent les armées des anges en tremblant,
\par 22 Dont le service est familier avec le vent et le feu ; dont la parole est vraie et les paroles constantes ; dont le commandement est fort et l'ordonnance effrayante ;
\par 23 Dont le regard assèche les abîmes, et l'indignation fait fondre les montagnes ; dont témoigne la vérité :
\par 24 Écoute la prière de ton serviteur, et prête l'oreille à la supplication de ta créature.
\par 25 Car tant que je vivrai, je parlerai, et tant que j'aurai de l'intelligence, je répondrai.
\par 26 Ne regarde pas les péchés de ton peuple ; mais sur ceux qui te servent en vérité.
\par 27 Ne considère pas les mauvaises inventions des païens, mais le désir de ceux qui gardent tes témoignages dans les afflictions.
\par 28 Ne pense pas à ceux qui ont marché en feignant devant toi ; mais souviens-toi de ceux qui, selon ta volonté, ont connu ta peur.
\par 29 Que tu ne veuilles pas détruire ceux qui ont vécu comme des bêtes ; mais de regarder ceux qui ont clairement enseigné ta loi.
\par 30 Ne t'indigne pas de ceux qui sont considérés comme pires que les bêtes ; mais aime ceux qui mettent toujours leur confiance en ta justice et en ta gloire.
\par 31 Car nous et nos pères languissons de telles maladies; mais à cause de nous, pécheurs, tu seras appelé miséricordieux.
\par 32 Car si tu désires avoir pitié de nous, tu seras appelé miséricordieux, c'est-à-dire envers nous qui n'avons pas d'œuvres de justice.
\par 33 Car les justes, qui ont en réserve beaucoup de bonnes œuvres, recevront une récompense de leurs propres actions.
\par 34 Car qu'est-ce que l'homme, pour que tu sois mécontent de lui ? ou qu'est-ce qu'une génération corruptible, pour que tu sois si amer à son égard ?
\par 35 Car en vérité, parmi ceux qui sont nés, il n'y a aucun homme qui n'ait agi méchamment ; et parmi les fidèles, il n’y en a aucun qui n’ait fait quelque chose de mal.
\par 36 Car c'est en cela, Seigneur, que ta justice et ta bonté seront manifestées, si tu es miséricordieux envers ceux qui n'ont pas la confiance dans les bonnes œuvres.
\par 37 Alors il me répondit et dit : Tu as bien dit certaines choses, et cela sera selon tes paroles.
\par 38 Car en effet, je ne penserai pas au tempérament de ceux qui ont péché avant la mort, avant le jugement, avant la destruction :
\par 39 Mais je me réjouirai du tempérament des justes, et je me souviendrai aussi de leur pèlerinage, du salut et de la récompense qu'ils auront.
\par 40 Comme je l'ai dit maintenant, ainsi cela arrivera.
\par 41 Car, de même que le laboureur sème beaucoup de semence sur la terre et plante beaucoup d'arbres, et que ce qui est semé de bon en sa saison ne pousse pas, et tout ce qui est planté ne prend pas non plus racine : ainsi en est-il de ceux qui sont semés dans le monde ; ils ne seront pas tous sauvés.
\par 42 Je répondis alors et dis : Si j'ai trouvé grâce, laissez-moi parler.
\par 43 Comme la semence du laboureur périt, si elle ne pousse pas et ne reçoit pas ta pluie au temps convenable ; ou s'il pleut trop et que cela le corrompt :
\par 44 Ainsi périt aussi l'homme, qui a été formé de tes mains, et qui est appelé ta propre image, parce que tu es semblable à celui pour qui tu as tout fait, et que tu l'as comparé à la semence du laboureur.
\par 45 Ne sois pas en colère contre nous, mais épargne ton peuple et aie pitié de ton propre héritage, car tu es miséricordieux envers ta créature.
\par 46 Alors il me répondit et dit : Les choses présentes sont pour le présent, et les choses à venir pour celles qui seront à venir.
\par 47 Car tu es loin d'être capable d'aimer ma créature plus que moi ; mais je me suis souvent approché de toi et d'elle, mais jamais des injustes.
\par 48 En cela aussi tu es merveilleux devant le Très-Haut :
\par 49 En ce sens que tu t'es humilié comme il te convenait, et que tu ne t'es pas jugé digne d'être beaucoup glorifié parmi les justes.
\par 50 Car beaucoup de grandes misères seront faites à ceux qui habiteront dans les derniers temps dans le monde, parce qu'ils ont marché avec un grand orgueil.
\par 51 Mais comprends-toi par toi-même, et recherche la gloire de ceux qui te ressemblent.
\par 52 Car le paradis vous est ouvert, l'arbre de vie est planté, le temps à venir est préparé, l'abondance est préparée, une ville est construite et le repos est permis, oui, la bonté et la sagesse parfaites.
\par 53 La racine du mal est scellée loin de vous, la faiblesse et la mite vous sont cachées, et la corruption s'enfuit en enfer pour être oubliée :
\par 54 Les chagrins sont passés, et à la fin est montré le trésor de l'immortalité.
\par 55 Et c'est pourquoi ne pose plus de questions sur la multitude de ceux qui périssent.
\par 56 Car, lorsqu'ils eurent pris la liberté, ils méprisèrent le Très-Haut, méprisèrent sa loi et abandonnèrent ses voies.
\par 57 Et ils ont foulé aux pieds ses justes,
\par 58 Et ils disaient dans leur cœur qu'il n'y avait pas de Dieu ; oui, et cela sachant qu'ils doivent mourir.
\par 59 Car, de même que les choses mentionnées ci-dessus vous recevront, de même la soif et la souffrance leur sont préparées : car ce n'était pas sa volonté que les hommes échouent.
\par 60 Mais ceux qui ont été créés ont souillé le nom de celui qui les a créés, et ont été ingrats envers celui qui leur a préparé la vie.
\par 61 C'est pourquoi mon jugement est maintenant proche.
\par 62 Je n'ai pas montré ces choses à tous, mais à toi et à quelques-uns comme toi. Alors j'ai répondu et j'ai dit :
\par 63 Voici, Seigneur, tu m'as montré maintenant la multitude des merveilles que tu commenceras à faire dans les derniers temps ; mais à quelle époque tu ne me l'as pas montrée.

\chapitre{9}

\par 1 Il me répondit alors et dit : Mesure soigneusement le temps en lui-même ; et quand tu verras passer une partie des signes que je t'ai annoncés auparavant,
\par 2 Alors tu comprendras que c'est le même moment où le Très-Haut commencera à visiter le monde qu'il a créé.
\par 3 C'est pourquoi, quand on verra des tremblements de terre et des tumultes parmi les peuples du monde :
\par 4 Alors tu comprendras bien que le Très-Haut a dit ces choses dès les jours qui étaient avant toi, dès le commencement.
\par 5 Car, de même que tout ce qui est fait dans le monde a un commencement et une fin, et que la fin est manifeste :
\par 6 De même, les temps du Très-Haut commencent clairement par des merveilles et des œuvres puissantes, et se terminent par des effets et des signes.
\par 7 Et quiconque sera sauvé et pourra échapper par ses œuvres et par la foi par laquelle vous avez cru,
\par 8 Sera préservé desdits périls, et verra mon salut dans mon pays et à l'intérieur de mes frontières : car je les ai sanctifiés pour moi dès le commencement.
\par 9 Alors ceux qui auront abusé de mes voies seront dans un cas pitoyable, et ceux qui les auront rejetés de manière méprisante demeureront dans les tourments.
\par 10 Car ceux qui dans leur vie ont reçu des bienfaits et ne m'ont pas connu ;
\par 11 Et ceux qui ont détesté ma loi, alors qu'ils avaient encore la liberté, et, alors que le lieu de la repentance leur était encore ouvert, ne l'ont pas comprise, mais l'ont méprisée ;
\par 12 Celui-là doit le connaître après la mort par la douleur.
\par 13 Ne sois donc pas curieux de savoir comment et quand les impies seront punis ; mais cherche comment les justes seront sauvés, à qui appartient le monde et pour qui le monde a été créé.
\par 14 Alors je répondis et dis :
\par 15 J'ai déjà dit, et je le dis maintenant, et je le dirai encore plus tard, qu'il y a beaucoup plus de ceux qui périssent que de ceux qui seront sauvés.
\par 16 Comme une vague est plus grande qu'une goutte.
\par 17 Et il me répondit, disant : Tel est le champ, telle est la semence ; telles que soient les fleurs, telles sont aussi les couleurs ; tel qu'est l'ouvrier, tel est aussi l'ouvrage ; et comme le laboureur est lui-même, ainsi est son travail, car c'était le temps du monde.
\par 18 Et maintenant, quand j'ai préparé le monde, qui n'était pas encore fait, même pour qu'ils y habitent celui qui vit maintenant, personne n'a parlé contre moi.
\par 19 Car alors chacun obéissait; mais maintenant les mœurs de ceux qui sont créés dans ce monde qui est créé sont corrompus par une semence perpétuelle, et par une loi insondable se débarrassent d'eux-mêmes.
\par 20 J'ai donc considéré le monde, et voici, il y avait un péril à cause des artifices qui y étaient introduits.
\par 21 Et je l'ai vu, et je l'ai grandement épargné, et j'ai gardé le raisin de la grappe et la plante d'un grand peuple.
\par 22 Que périsse donc la multitude, qui est née en vain ; et que mon raisin et ma plante soient conservés ; car avec beaucoup de travail je l'ai rendu parfait.
\par 23 Néanmoins, si tu cesses encore sept jours, (sans que tu jeûnes pendant ces jours,
\par 24 Mais allez dans un champ de fleurs, où aucune maison n'est bâtie, et mangez seulement les fleurs des champs ; ne goûtez pas de chair, ne buvez pas de vin, mais mangez uniquement des fleurs ;)
\par 25 Et prie continuellement le Très-Haut, alors je viendrai et je te parlerai.
\par 26 Je m'en allai dans le champ appelé Ardath, comme il me l'avait ordonné ; je m'y assis parmi les fleurs, et je mangeai des herbes des champs, dont la chair me rassasia.
\par 27 Sept jours plus tard, j'étais assis sur l'herbe, et mon cœur était tourmenté au dedans de moi, comme auparavant :
\par 28 Et j'ouvris la bouche, et je me mis à parler devant le Très-Haut, et je dis :
\par 29 O Seigneur, toi qui t'es montré à nous, tu as été montré à nos pères dans le désert, dans un lieu où personne ne marche, dans un lieu stérile, lorsqu'ils sont sortis d'Egypte.
\par 30 Et tu disais : Écoute-moi, ô Israël ! et écoute mes paroles, postérité de Jacob.
\par 31 Car voici, je sème en vous ma loi, et elle portera du fruit en vous, et vous y serez honorés pour toujours.
\par 32 Mais nos pères, qui ont reçu la loi, ne l'ont pas observée et n'ont pas observé tes ordonnances ; et bien que le fruit de ta loi n'ait pas péri, il ne pouvait pas non plus, car il était à toi ;
\par 33 Mais ceux qui l'ont reçu ont péri, parce qu'ils n'ont pas gardé ce qui avait été semé en eux.
\par 34 Et voici, c'est une coutume, lorsque la terre a reçu de la semence, ou la mer un navire, ou tout récipient de la viande ou de la boisson, que, étant péri, ce dans lequel il a été semé ou jeté,
\par 35 Même ce qui a été semé, ou jeté, ou reçu, périt et ne reste pas avec nous ; mais il n'est pas arrivé ainsi à nous.
\par 36 Car nous qui avons reçu la loi périssons par le péché, et notre cœur aussi qui l'a reçue
\par 37 Malgré cela, la loi ne périt pas, mais demeure dans sa force.
\par 38 Lorsque j'eus dit ces choses en mon coeur, je regardai en arrière de mes yeux, et je vis une femme à ma droite ; et voici, elle était dans le deuil, elle pleurait à haute voix, et son coeur était très attristé ; ses vêtements étaient déchirés, et elle avait de la cendre sur la tête.
\par 39 Alors j'ai laissé aller mes pensées dans lesquelles j'étais, et je me suis tourné vers elle,
\par 40 Et il lui dit : Pourquoi pleures-tu ? pourquoi es-tu si attristé dans ton esprit ?
\par 41 Et elle me dit : Monsieur, laissez-moi tranquille, afin que je puisse me pleurer et ajouter à mon chagrin, car je suis très tourmenté dans mon esprit et très abattu.
\par 42 Et je lui dis : Qu'as-tu ? dites-moi.
\par 43 Elle me dit : Moi, ta servante, je suis stérile et je n'ai pas eu d'enfant, bien que j'aie eu un mari pendant trente ans.
\par 44 Et pendant ces trente années, je n'ai rien fait d'autre, jour et nuit, et à chaque heure, que d'adresser ma prière au Très-Haut.
\par 45 Après trente ans, Dieu m'a entendu, ta servante, a vu ma misère, a considéré mon malheur et m'a donné un fils ; et j'ai été très content de lui, ainsi que mon mari et tous mes voisins ; et nous avons donné beaucoup honneur au Tout-Puissant.
\par 46 Et je l'ai nourri avec un grand travail.
\par 47 Et quand il fut grand, et qu'il arriva au moment où il devait avoir une femme, j'organisai un festin.

\chapitre{10}

\par 1 Et il arriva que lorsque mon fils entra dans sa chambre nuptiale, il tomba et mourut.
\par 2 Alors nous avons tous éteint les lumières, et tous mes voisins se sont levés pour me consoler : alors je me suis reposé jusqu'au deuxième jour dans la nuit.
\par 3 Et il arriva, quand ils eurent tous fini de me consoler, que jusqu'à la fin je pus me taire ; puis je me suis levé de nuit, je me suis enfui et je suis venu ici dans ce champ, comme tu le vois.
\par 4 Et maintenant je me propose de ne pas retourner dans la ville, mais d'y rester, et de ne pas manger ni boire, mais de pleurer et de jeûner continuellement jusqu'à ma mort.
\par 5 Alors je quittai les méditations où j'étais, et je lui parlai avec colère, disant :
\par 6 Toi, femme insensée entre toutes, ne vois-tu pas notre deuil, et que nous arrive-t-il ?
\par 7 Comment Sion notre mère est pleine de toute lourdeur, et très humiliée, portant un deuil très douloureux ?
\par 8 Et maintenant, puisque nous sommes tous en deuil et tristes, car nous sommes tous dans le poids, es-tu attristé pour un seul fils ?
\par 9 Car interroge la terre, et elle te dira que c'est elle qui doit pleurer la chute de tant de gens qui poussent sur elle.
\par 10 Car d'elle sont tous sortis d'abord, et d'elle tous les autres sortiront, et voici, ils marchent presque tous vers la perdition, et une multitude d'entre eux sont entièrement déracinés.
\par 11 Qui donc devrait faire plus de deuil qu'elle, qui a perdu une si grande multitude ? et pas toi, qui es désolé sinon pour un ?
\par 12 Mais si tu me dis : Mes lamentations ne sont pas comme celles de la terre, parce que j'ai perdu le fruit de mes entrailles, que j'ai enfanté avec douleurs et que j'ai porté avec douleurs ;
\par 13 Mais il n'en est pas ainsi pour la terre : car la multitude qui y était présente, selon le cours de la terre, s'en est allée comme elle était venue :
\par 14 Alors je te dis : Comme tu as enfanté avec le travail ; de même la terre aussi a donné son fruit, c'est-à-dire l'homme, depuis le commencement à celui qui l'a créée.
\par 15 Maintenant donc, garde ta tristesse pour toi, et supporte avec bon courage ce qui t'arrive.
\par 16 Car si tu reconnais la détermination de Dieu d'être juste, tu recevras ton fils à temps, et tu seras loué parmi les femmes.
\par 17 Va donc dans la ville vers ton mari.
\par 18 Et elle me dit : Je ne ferai pas cela : je n'entrerai pas dans la ville, mais ici je mourrai.
\par 19 Alors je me mis à lui parler davantage et je dis :
\par 20 Ne le faites pas, mais laissez-vous conseiller par moi ; Car combien d'épreuves Sion a-t-elle à supporter ? Consolez-vous à cause de la douleur de Jérusalem.
\par 21 Car tu vois que notre sanctuaire est dévasté, notre autel détruit, notre temple détruit ;
\par 22 Notre psaltérion est posé à terre, notre chant est réduit au silence, notre réjouissance est terminée, la lumière de notre chandelier s'éteint, l'arche de notre alliance est gâtée, nos choses saintes sont souillées, et le le nom qui nous est invoqué est presque profané : nos enfants sont honteux, nos prêtres sont brûlés, nos Lévites sont allés en captivité, nos vierges sont souillées et nos femmes violées ; nos justes sont emportés, nos petits sont détruits, nos jeunes hommes sont réduits en esclavage et nos hommes forts sont devenus faibles ;
\par 23 Et ce qui est le plus grand de tous, le sceau de Sion a maintenant perdu son honneur ; car elle est livrée entre les mains de ceux qui nous haïssent.
\par 24 Et donc, secoue-toi de ton grand fardeau, et mets de côté la multitude de chagrins, afin que le Puissant te fasse à nouveau miséricorde, et que le Très-Haut te donne du repos et de l'aisance dans ton travail.
\par 25 Et pendant que je parlais avec elle, voici, tout d'un coup, son visage s'éclaira extrêmement, et son visage brillait, de sorte que j'eus peur d'elle et je réfléchis à ce que cela pouvait être.
\par 26 Et voici, tout à coup, elle poussa un grand cri très effrayant, de sorte que la terre trembla au bruit de la femme.
\par 27 Et je regardai, et voici, la femme ne m'apparut plus, mais une ville était bâtie, et une grande place se montrait depuis les fondations. Alors j'eus peur, et je criai d'une voix forte, et je dis : ,
\par 28 Où est Uriel, l'ange, qui est venu vers moi le premier ? car il m'a fait tomber dans de nombreuses transes, et ma fin s'est transformée en corruption, et ma prière en réprimande.
\par 29 Et pendant que je disais ces paroles, voici, il s'approcha de moi et me regarda.
\par 30 Et voici, j'étais couché comme un mort, et mon intelligence m'a été enlevée ; et il m'a pris par la main droite, et m'a consolé, et m'a mis sur mes pieds, et m'a dit :
\par 31 Qu'as-tu ? et pourquoi es-tu si inquiet ? et pourquoi ton entendement et les pensées de ton cœur sont-ils troublés ?
\par 32 Et je dis : Parce que tu m'as abandonné, et pourtant j'ai fait selon tes paroles, et je suis allé dans les champs, et voici, j'ai vu, et je vois encore, ce que je ne puis exprimer.
\par 33 Et il me dit : Lève-toi vaillamment, et je te conseillerai.
\par 34 Alors je dis : Parle en moi, mon seigneur ; seulement ne m'abandonne pas, de peur que je ne meure frustré de mon espérance.
\par 35 Car j'ai vu que je ne savais pas, et j'ai entendu dire que je ne sais pas.
\par 36 Ou bien mes sens sont-ils trompés, ou mon âme est-elle dans un rêve ?
\par 37 Maintenant donc, je te prie, de montrer cette vision à ton serviteur.
\par 38 Il me répondit alors et dit : Écoute-moi, je t'informerai et je te dirai pourquoi tu as peur ; car le Très-Haut te révélera beaucoup de choses secrètes.
\par 39 Il a vu que ta voie est droite : c'est pourquoi tu t'affliges continuellement pour ton peuple et tu fais de grandes lamentations sur Sion.
\par 40 Voici donc le sens de la vision que tu as eue récemment :
\par 41 Tu as vu une femme en deuil, et tu as commencé à la consoler :
\par 42 Mais maintenant tu ne vois plus l'image de la femme, mais une ville bâtie t'est apparue.
\par 43 Et tandis qu'elle t'a annoncé la mort de son fils, voici la solution :
\par 44 Cette femme que tu as vue est Sion ; et tandis qu'elle t'a dit, celle que tu vois comme une ville bâtie,
\par 45 Or, je le dis, elle t'a dit qu'elle était stérile depuis trente ans : ce sont les trente années pendant lesquelles aucune offrande n'a été faite en elle.
\par 46 Mais après trente ans, Salomon rebâtit la ville et offrit des offrandes ; puis il enfanta un fils à la stérile.
\par 47 Et tandis qu'elle t'a dit qu'elle le nourrissait de travail : c'était la demeure à Jérusalem.
\par 48 Mais tandis qu'elle t'a dit : Que mon fils, entrant dans sa chambre nuptiale, est tombé malade et est mort : voilà la destruction qui est arrivée à Jérusalem.
\par 49 Et voici, tu as vu son image, et parce qu'elle pleurait son fils, tu as commencé à la consoler ; et parmi ces choses qui sont arrivées, celles-ci doivent t'être révélées.
\par 50 Car maintenant le Très-Haut voit que tu es sincèrement attristé et que tu souffres de tout ton cœur pour elle, ainsi il t'a montré l'éclat de sa gloire et la beauté de sa beauté.
\par 51 Et c'est pourquoi je t'ai ordonné de rester dans un champ où aucune maison n'était bâtie :
\par 52 Car je savais que le Très-Haut te le montrerait.
\par 53 C'est pourquoi je t'ai ordonné d'aller dans un champ où il n'y avait aucune fondation d'aucun bâtiment.
\par 54 Car au lieu où le Très-Haut commence à montrer sa ville, aucun édifice humain ne peut tenir debout.
\par 55 Et donc ne crains rien, que ton cœur ne soit pas effrayé, mais entre, et vois la beauté et la grandeur de l'édifice, autant que tes yeux peuvent le voir.
\par 56 Et alors tu entendras autant que tes oreilles peuvent comprendre.
\par 57 Car tu es béni plus que beaucoup d'autres, et tu es appelé auprès du Très-Haut ; et ils sont peu nombreux aussi.
\par 58 Mais demain soir tu resteras ici ;
\par 59 Et ainsi le Très-Haut te montrera des visions des choses élevées que le Très-Haut fera à ceux qui habitent sur la terre dans les derniers jours. J'ai donc dormi cette nuit-là et une autre, comme il me l'avait ordonné.

\chapitre{11}

\par 1 Alors j'ai eu un rêve, et voici, un aigle montait de la mer, qui avait douze ailes emplumées et trois têtes.
\par 2 Et je vis, et voici, elle étendit ses ailes sur toute la terre, et tous les vents de l'air soufflèrent sur elle, et se rassemblèrent.
\par 3 Et je vis, et de ses plumes poussèrent d'autres plumes contraires ; et ils devinrent de petites plumes et petits.
\par 4 Mais ses têtes étaient au repos : la tête au milieu était plus grande que l'autre, mais elle reposait sur le reste.
\par 5 Et je vis, et voici, l'aigle volait avec ses plumes, et régnait sur la terre et sur ceux qui l'habitaient.
\par 6 Et je vis que toutes choses sous le ciel lui étaient soumises, et que personne ne parlait contre elle, non, pas une seule créature sur la terre.
\par 7 Et je vis, et voici, l'aigle se leva sur ses serres, et parla à ses plumes, disant :
\par 8 Ne veillez pas tous à la fois : dormez chacun à sa place, et veillez par ordre :
\par 9 Mais que les têtes soient conservées pour la fin.
\par 10 Et je vis, et voici, la voix ne sortait pas de sa tête, mais du milieu de son corps.
\par 11 Et je comptais ses plumes contraires, et voici, il y en avait huit.
\par 12 Et je regardai, et voici, du côté droit se levait une seule plume, et régnait sur toute la terre ;
\par 13 Après son règne, il arriva qu'il s'éteignit et que son emplacement disparut. Le suivant se leva, régna et eut beaucoup de plaisir ;
\par 14 Et il arriva que lorsqu'il régna, sa fin vint aussi, comme la première, de sorte qu'il n'apparut plus.
\par 15 Alors une voix lui parvint et dit :
\par 16 Écoute, toi qui as si longtemps régné sur la terre : je te dis ceci, avant que tu ne commences plus à apparaître,
\par 17 Personne après toi n'atteindra ton temps, ni la moitié de celui-ci.
\par 18 Alors le troisième se leva, et régna comme l'autre auparavant, et ne parut plus non plus.
\par 19 Ainsi en fut-il avec tout le reste, l'un après l'autre, de sorte que chacun régna, puis ne réapparut plus.
\par 20 Alors je vis, et voici, au fil du temps, les plumes qui suivirent se dressèrent sur le côté droit, afin de pouvoir aussi régner ; et certains d'entre eux régnèrent, mais au bout d'un moment ils ne parurent plus :
\par 21 Car certains d'entre eux ont été établis, mais n'ont pas gouverné.
\par 22 Après cela, je regardai, et voici, les douze plumes n'apparurent plus, ni les deux petites plumes :
\par 23 Et il n'y avait plus sur le corps de l'aigle, mais trois têtes qui reposaient et six petites ailes.
\par 24 Alors je vis aussi que deux petites plumes se séparaient des six et restaient sous la tête qui était du côté droit : car les quatre restaient à leur place.
\par 25 Et je vis, et voici, les plumes qui étaient sous l'aile pensèrent se dresser et avoir la règle.
\par 26 Et je vis, et voici, il y en avait un installé, mais bientôt il n'apparut plus.
\par 27 Et le second fut parti plus tôt que le premier.
\par 28 Et je vis, et voici, les deux qui restèrent pensèrent aussi en eux-mêmes régner :
\par 29 Et comme ils pensaient ainsi, voici, une des têtes qui étaient au repos se réveilla, à savoir celle qui était au milieu ; car cela était plus grand que les deux autres têtes.
\par 30 Et puis j'ai vu que les deux autres têtes étaient jointes à elle.
\par 31 Et voici, la tête se tourna avec ceux qui étaient avec elle, et dévorèrent les deux plumes sous l'aile qui auraient régné.
\par 32 Mais ce chef fit trembler toute la terre, et y régna sur tous ceux qui habitaient sur la terre avec une grande oppression ; et elle avait la gouvernance du monde plus que toutes les ailes qui l'avaient été.
\par 33 Et après cela, je vis, et voici, la tête qui était au milieu n'apparut soudain plus, comme les ailes.
\par 34 Mais il restait les deux têtes, qui régnaient également de la même manière sur la terre et sur ceux qui l'habitaient.
\par 35 Et je vis, et voici, la tête du côté droit dévora celle qui était du côté gauche.
\par 36 Alors j'entendis une voix qui me disait : Regarde devant toi, et considère ce que tu vois.
\par 37 Et je vis, et voici, comme si c'était un lion rugissant chassé du bois ; et je vis qu'il envoyait une voix d'homme à l'aigle, et dit :
\par 38 Écoute, je te parlerai, et le Très-Haut te dira :
\par 39 N'es-tu pas le reste des quatre bêtes que j'ai faites régner sur mon monde, afin que la fin de leurs temps vienne par elles ?
\par 40 Et le quatrième vint, et vainquit toutes les bêtes du passé, et il eut pouvoir sur le monde avec une grande frayeur, et sur toute l'étendue de la terre avec une grande et méchante oppression ; et il demeura si longtemps sur la terre dans la tromperie.
\par 41 Car tu n'as pas jugé la terre avec vérité.
\par 42 Car tu as humilié les doux, tu as blessé les paisibles, tu as aimé les menteurs, tu as détruit les habitations de ceux qui produisaient du fruit, et tu as renversé les murs de ceux qui ne te faisaient aucun mal.
\par 43 C'est pourquoi ton iniquité s'élève jusqu'au plus haut, et ton orgueil jusqu'au Tout-Puissant.
\par 44 Le Très-Haut a aussi regardé les temps orgueilleux, et voici, ils sont terminés, et ses abominations sont accomplies.
\par 45 Et c'est pourquoi n'apparais plus, toi, aigle, ni tes ailes horribles, ni tes plumes méchantes, ni tes têtes malveillantes, ni tes griffes nuisibles, ni tout ton corps vain :
\par 46 Afin que toute la terre soit rafraîchie et revienne, délivrée de ta violence, et qu'elle puisse espérer le jugement et la miséricorde de celui qui l'a créée.

\chapitre{12}

\par 1 Et il arriva que pendant que le lion disait ces paroles à l'aigle, je vis :
\par 2 Et voici, la tête qui restait et les quatre ailes n'apparurent plus, et les deux s'y rendirent et s'établirent pour régner, et leur royaume était petit et rempli de tumulte.
\par 3 Je regardai, et voici qu'ils n'apparurent plus, et tout le corps de l'aigle fut brûlé, de sorte que la terre fut saisie d'une grande frayeur. Alors je me réveillai du trouble et de la transe de mon esprit, et de la grande frayeur, et je dis à mon esprit ,
\par 4 Voici, c'est ainsi que tu m'as fait, en recherchant les voies du Très-Haut.
\par 5 Voici, pourtant je suis fatigué dans mon esprit, et très faible dans mon esprit ; et il y a peu de force en moi, à cause de la grande peur dont j'ai été affligé cette nuit.
\par 6 C'est pourquoi je prierai maintenant le Très-Haut de me consoler jusqu'à la fin.
\par 7 Et je dis : Seigneur qui gouvernes, si j'ai trouvé grâce devant toi, et si je suis justifié avec toi devant beaucoup d'autres, et si ma prière s'élève effectivement devant ta face ;
\par 8 Console-moi donc, et montre-moi à ton serviteur l'interprétation et la différence claire de cette vision effrayante, afin que tu puisses parfaitement consoler mon âme.
\par 9 Car tu m'as jugé digne de me montrer les derniers temps.
\par 10 Et il me dit : Voici l'interprétation de la vision :
\par 11 L'aigle que tu as vu monter de la mer est le royaume qui a été vu dans la vision de ton frère Daniel.
\par 12 Mais cela ne lui a pas été expliqué, c'est pourquoi maintenant je te le déclare.
\par 13 Voici, les jours viendront où un royaume s'élèvera sur la terre, et il sera redouté plus que tous les royaumes qui l'ont précédé.
\par 14 Là-bas régneront douze rois, l'un après l'autre :
\par 15 Dont le second commencera à régner, et aura plus de temps que n'importe lequel des douze.
\par 16 Et c'est ce que signifient les douze ailes que tu as vues.
\par 17 Quant à la voix que tu as entendu parler, et que tu as vu sortir non pas des têtes mais du milieu du corps, voici l'interprétation :
\par 18 Après le temps de ce royaume, de grandes difficultés surgiront, et il sera en danger d'échouer ; néanmoins il ne tombera pas alors, mais sera restauré à son commencement.
\par 19 Et tandis que tu as vu les huit petits sous les plumes collés à ses ailes, voici l'interprétation :
\par 20 En lui s'élèveront huit rois, dont les temps seront courts et leurs années rapides.
\par 21 Et deux d'entre eux périront, à l'approche du milieu des temps ; quatre seront gardés jusqu'à ce que leur fin commence à approcher ; mais deux seront gardés jusqu'à la fin.
\par 22 Et tandis que tu as vu trois têtes posées, voici l'interprétation :
\par 23 Dans ses derniers jours, le Très-Haut suscitera trois royaumes et y renouvellera beaucoup de choses, et ils domineront la terre,
\par 24 Et parmi ceux qui y habitaient, avec une grande oppression, plus que tous ceux qui étaient avant eux : c'est pourquoi on les appelle têtes d'aigle.
\par 25 Car ce sont eux qui accompliront sa méchanceté et qui achèveront sa fin dernière.
\par 26 Et tandis que tu as vu que la grosse tête n'apparaissait plus, cela signifie que l'un d'eux mourra sur son lit, et cependant dans la douleur.
\par 27 Car les deux qui resteront seront tués par l'épée.
\par 28 Car l'épée de l'un dévorera l'autre, mais à la fin il tombera lui-même par l'épée.
\par 29 Et considérant que tu as vu deux plumes sous les ailes passant au-dessus de la tête qui est du côté droit ;
\par 30 Cela signifie que ce sont ceux-là que le Très-Haut a gardés jusqu'à leur fin : c'est le petit royaume et plein de troubles, comme tu l'as vu.
\par 31 Et le lion, que tu as vu sortir du bois, rugir et parler à l'aigle, et la réprimander à cause de son injustice, avec toutes les paroles que tu as entendues ;
\par 32 Celui-ci est l'oint que le Très-Haut a gardé pour eux et pour leur méchanceté jusqu'à la fin : il les reprendra et leur reprochera leur cruauté.
\par 33 Car il les présentera vivants devant lui lors du jugement, et les réprimandera et les corrigera.
\par 34 Car il délivrera avec miséricorde le reste de mon peuple, ceux qui étaient pressés sur mes frontières, et il les rendra joyeux jusqu'à l'arrivée du jour du jugement dont je t'ai parlé dès le commencement.
\par 35 Ceci est le rêve que tu as vu, et voici les interprétations.
\par 36 Toi seul es digne de connaître ce secret du Très-Haut.
\par 37 C'est pourquoi, écris dans un livre toutes ces choses que tu as vues, et cache-les.
\par 38 Et enseigne-les aux sages du peuple, dont tu sais que le cœur peut comprendre et garder ces secrets.
\par 39 Mais attends ici encore sept jours, afin qu'on te montre tout ce qu'il plaira au Très-Haut de te l'annoncer. Et sur ce, il poursuivit son chemin.
\par 40 Et il arriva que, quand tout le peuple vit que les sept jours étaient passés et que je ne revenais plus dans la ville, ils les rassemblèrent tous, depuis le plus petit jusqu'au plus grand, et vinrent vers moi et dirent : ,
\par 41 Que t'avons-nous offensé ? et quel mal avons-nous fait contre toi, pour que tu nous abandonnes et que tu sois assis ici, en ce lieu ?
\par 42 Car de tous les prophètes, tu es le seul qui nous reste, comme une grappe de vendange, et comme une bougie dans un lieu obscur, et comme un port ou un navire préservé de la tempête.
\par 43 Les maux qui nous arrivent ne sont-ils pas suffisants ?
\par 44 Si tu nous abandonnes, combien cela aurait-il été mieux pour nous, si nous aussi avions été brûlés au milieu de Sion ?
\par 45 Car nous ne valons pas mieux que ceux qui sont morts là. Et ils pleurèrent à haute voix. Alors je leur répondis et dis :
\par 46 Sois rassuré, ô Israël ! et ne sois pas lourde, maison de Jacob !
\par 47 Car le Très-Haut vous a rappelé, et le Puissant ne vous a pas oublié dans la tentation.
\par 48 Quant à moi, je ne vous ai pas abandonné, et je ne me suis pas éloigné de vous ; mais je suis venu en ce lieu pour prier pour la désolation de Sion, et pour implorer miséricorde pour la misère de votre sanctuaire.
\par 49 Et maintenant, chacun rentre chez soi, et après ces jours je viendrai vers vous.
\par 50 Le peuple entra donc dans la ville, comme je le lui avais ordonné :
\par 51 Mais je restai sept jours dans les champs, comme l'ange me l'avait ordonné ; et je ne mangeais en ce temps-là que des fleurs des champs, et je mangeais des herbes.

\chapitre{13}

\par 1 Et il arriva qu'au bout de sept jours, je fis un songe pendant la nuit :
\par 2 Et voici, un vent s'éleva de la mer, qui remua toutes ses vagues.
\par 3 Et je vis, et voici, cet homme devenait fort comme les milliers des cieux ; et quand il tourna son visage pour regarder, toutes les choses qui se voyaient sous lui tremblaient.
\par 4 Et chaque fois que la voix sortait de sa bouche, tous ceux qui entendaient sa voix brûlaient, comme la terre tremble lorsqu'elle sent le feu.
\par 5 Et après cela, je regardai, et voici, une multitude d'hommes, sans nombre, étaient rassemblés des quatre vents du ciel, pour soumettre l'homme qui sortait de la mer.
\par 6 Mais je vis, et voici, il s'était taillé une grande montagne et s'envolait dessus.
\par 7 Mais j'aurais vu la région ou le lieu où la colline était taillée, et je ne le pouvais pas.
\par 8 Et après cela, je vis, et voici, tous ceux qui étaient rassemblés pour le soumettre eurent très peur, et pourtant ils osèrent se battre.
\par 9 Et voici, comme il voyait la violence de la multitude qui arrivait, il ne leva ni la main, ni ne tint l'épée, ni aucun instrument de guerre :
\par 10 Mais seulement j'ai vu qu'il faisait sortir de sa bouche comme une explosion de feu, et de ses lèvres un souffle enflammé, et de sa langue il jetait des étincelles et des tempêtes.
\par 11 Et ils étaient tous mélangés; le souffle du feu, le souffle enflammé et la grande tempête ; et je tombai avec violence sur la multitude qui était prête à combattre, et les brûlai tous, de sorte que, soudain, d'une multitude innombrable, rien ne fut perçu, mais seulement de la poussière et une odeur de fumée. Quand j'ai vu cela, j'ai eu peur. .
\par 12 Ensuite, je vis le même homme descendre de la montagne et appeler à lui une autre multitude paisible.
\par 13 Et beaucoup de monde vint vers lui, dont les uns se réjouissaient, les autres étaient désolés, et certains étaient liés, et d'autres encore apportèrent de ceux qui étaient offerts. Alors je fus malade d'une grande peur, et je me réveillai, et dit,
\par 14 Tu as fait ces merveilles à ton serviteur dès le commencement, et tu m'as estimé digne que tu reçoives ma prière :
\par 15 Montre-moi maintenant encore l'interprétation de ce rêve.
\par 16 Car tel que je le conçois dans ma compréhension, malheur à ceux qui seront laissés en ces jours-là et bien plus malheur à ceux qui ne seront pas laissés pour compte !
\par 17 Car ceux qui n'étaient pas restés étaient dans le fardeau.
\par 18 Maintenant, je comprends les choses qui sont réservées dans les derniers jours, ce qui leur arrivera, ainsi qu'à ceux qui resteront derrière.
\par 19 C'est pourquoi ils se trouvent confrontés à de grands périls et à de nombreuses nécessités, comme le déclarent ces rêves.
\par 20 Pourtant, il est plus facile pour celui qui est en danger d'entrer dans ces choses que de disparaître comme une nuée hors du monde, et de ne pas voir les choses qui arriveront dans les derniers jours. Et il me répondit et dit :
\par 21 Je te montrerai l'interprétation de la vision, et je t'ouvrirai ce que tu as demandé.
\par 22 Puisque tu as parlé de ceux qui sont restés, voici l'interprétation :
\par 23 Celui qui endurera le péril en ce temps-là s'est gardé lui-même : ceux qui tombent dans le danger sont ceux qui ont des œuvres et de la foi envers le Tout-Puissant.
\par 24 Sachez donc que ceux qui restent sont plus bénis que ceux qui sont morts.
\par 25 Voici le sens de la vision : Alors que tu as vu un homme monter du milieu de la mer :
\par 26 C'est lui que Dieu le plus haut a réservé une grande saison, qui délivrera de lui-même sa créature, et il ordonnera à ceux qui resteront en arrière.
\par 27 Et tandis que tu as vu que de sa bouche sortait comme un souffle de vent, de feu et de tempête ;
\par 28 Et qu'il ne possédait ni épée, ni aucun instrument de guerre, mais que sa précipitation détruisit toute la multitude qui était venue pour le soumettre ; voici l'interprétation :
\par 29 Voici, les jours viennent où le Très-Haut commencera à délivrer ceux qui sont sur la terre.
\par 30 Et il viendra à la grande surprise des habitants de la terre.
\par 31 Et chacun entreprendra de combattre un autre, une ville contre une autre, un lieu contre un autre, un peuple contre un autre, et un royaume contre un autre.
\par 32 Et le temps viendra où ces choses arriveront, et où arriveront les signes que je t'ai montrés auparavant, et alors sera déclaré mon Fils, que tu as vu comme un homme montant.
\par 33 Et quand tout le peuple entendra sa voix, chacun dans son pays abandonnera la bataille qu'il a l'un contre l'autre.
\par 34 Et une multitude innombrable se rassemblera, comme tu les as vus, disposés à venir et à le vaincre par le combat.
\par 35 Mais il se tiendra au sommet de la montagne de Sion.
\par 36 Et Sion viendra, et sera montrée à tous, étant préparée et bâtie, comme tu as vu la colline taillée sans mains.
\par 37 Et mon Fils réprimandera les mauvaises inventions de ces nations qui, à cause de leur mauvaise vie, sont tombées dans la tempête ;
\par 38 Et il leur présentera leurs mauvaises pensées et les tourments avec lesquels ils commenceront à être tourmentés, qui sont semblables à une flamme ; et il les détruira sans travail par la loi qui est semblable à moi.
\par 39 Et considérant que tu as vu qu'il rassemblait auprès de lui une autre multitude paisible ;
\par 40 Ce sont là les dix tribus qui furent emmenées captives hors de leur propre pays au temps du roi Osée, que Salmanasar, roi d'Assyrie, emmena captifs, et les transporta sur les eaux, et ainsi ils entrèrent dans une autre terre.
\par 41 Mais ils prirent entre eux le conseil de quitter la multitude des païens et de s'en aller dans un pays plus éloigné, où jamais l'humanité n'a habité,
\par 42 Afin qu'ils y observent leurs statuts, qu'ils n'ont jamais observés dans leur propre pays.
\par 43 Et ils entrèrent dans l'Euphrate par les passages étroits du fleuve.
\par 44 Car le Très-Haut leur montra alors des signes et arrêta le déluge jusqu'à ce qu'ils soient passés.
\par 45 Car à travers ce pays il y avait un long chemin à parcourir, à savoir un an et demi ; et cette même région s'appelle Arsareth.
\par 46 Alors ils demeurèrent là jusqu'à la dernière fois ; et maintenant, quand ils commenceront à venir,
\par 47 Le Très-Haut retiendra les sources du ruisseau, afin qu'elles puissent passer. C'est pourquoi tu as vu la multitude en paix.
\par 48 Mais ceux qui restent de ton peuple sont ceux qui se trouvent à l'intérieur de mes frontières.
\par 49 Maintenant, quand il détruira la multitude des nations rassemblées, il défendra son peuple qui reste.
\par 50 Et alors il leur montrera de grands prodiges.
\par 51 Alors je dis : Seigneur qui gouvernes, montre-moi ceci : Pourquoi ai-je vu l'homme monter du milieu de la mer ?
\par 52 Et il me dit : De même que tu ne peux ni chercher ni connaître ce qui se trouve dans les profondeurs de la mer, de même personne sur la terre ne peut voir mon Fils, ni ceux qui sont avec lui, si ce n'est dans le la journée.
\par 53 Ceci est l'interprétation du rêve que tu as vu, et par lequel tu es ici seulement éclairé.
\par 54 Car tu as abandonné ta propre voie, et tu as appliqué ton zèle à ma loi, et tu l'as recherchée.
\par 55 Tu as ordonné ta vie avec sagesse, et tu as appelé ta mère à l'intelligence.
\par 56 C'est pourquoi je t'ai montré les trésors du Très-Haut. Dans trois jours, je te dirai d'autres choses, et je te déclarerai des choses puissantes et merveilleuses.
\par 57 Puis je sortis dans les champs, louant et remerciant grandement le Très-Haut à cause des prodiges qu'il avait accomplis dans le temps ;
\par 58 Et parce qu'il gouverne ces choses, et tout ce qui arrive en leur temps, et je suis resté là trois jours.

\chapitre{14}

\par 1 Et il arriva le troisième jour, je m'assis sous un chêne, et voici, une voix sortit d'un buisson devant moi, et dit : Esdras, Esdras.
\par 2 Et je dis : Me voici, Seigneur Et je me levai sur mes pieds.
\par 3 Alors il me dit : Dans le buisson, je me suis manifesté manifestement à Moïse, et j'ai parlé avec lui, lorsque mon peuple servait en Égypte :
\par 4 Et je l'envoyai et conduisis mon peuple hors d'Egypte, et je le fis monter sur la montagne où je l'ai retenu près de moi pendant une longue saison,
\par 5 Et il lui raconta beaucoup de choses merveilleuses, et lui montra les secrets des temps et de la fin ; et lui commanda, disant :
\par 6 Tu déclareras ces paroles, et tu les cacheras.
\par 7 Et maintenant je te le dis :
\par 8 Que tu gardes dans ton cœur les signes que j'ai montrés, et les songes que tu as vus, et les interprétations que tu as entendues :
\par 9 Car tu seras éloigné de tous, et désormais tu resteras avec mon Fils et avec ceux qui te ressemblent, jusqu'à la fin des temps.
\par 10 Car le monde a perdu sa jeunesse, et les temps commencent à vieillir.
\par 11 Car le monde est divisé en douze parties, et les dix parties ont déjà disparu, ainsi que la moitié d'un dixième :
\par 12 Et il reste ce qui est après la moitié du dixième.
\par 13 Maintenant donc mets de l'ordre dans ta maison, et reprends ton peuple, console ceux d'entre eux qui sont en difficulté, et maintenant renonce à la corruption,
\par 14 Abandonne les pensées mortelles, rejette les fardeaux de l'homme, dépouille-toi maintenant de la nature faible,
\par 15 Et mets de côté les pensées qui te sont les plus lourdes, et hâte-toi de fuir ces temps.
\par 16 Car des maux encore plus grands que ceux que tu as vu arriver se produiront désormais.
\par 17 Car voyez combien le monde s'affaiblira à travers les âges, et plus les maux augmenteront sur ceux qui l'habitent.
\par 18 Car le temps est loin, et la location est proche; car maintenant la vision que tu as vue est à venir.
\par 19 Alors je répondis devant toi, et je dis :
\par 20 Voici, Seigneur, j'irai, comme tu me l'as ordonné, et je reprendrai le peuple qui est présent. Mais ceux qui naîtront ensuite, qui les avertira ? ainsi le monde est plongé dans les ténèbres, et ceux qui y habitent sont sans lumière.
\par 21 Car ta loi est brûlée, c'est pourquoi personne ne sait ce que tu as fait, ni l'œuvre qui va commencer.
\par 22 Mais si j'ai trouvé grâce devant toi, envoie le Saint-Esprit en moi, et j'écrirai tout ce qui s'est fait dans le monde depuis le commencement, qui a été écrit dans ta loi, afin que les hommes trouvent ton chemin, et afin que ceux qui vivront dans les derniers jours puissent vivre.
\par 23 Et il me répondit, disant : Va, rassemble le peuple, et dis-leur qu'ils ne te cherchent pas pendant quarante jours.
\par 24 Mais regarde, prépare-toi beaucoup de buis, et prends avec toi Sarea, Dabria, Selemia, Ecanus et Asiel, ces cinq qui sont prêts à écrire rapidement ;
\par 25 Et viens ici, et j'allumerai dans ton cœur une bougie d'intelligence qui ne s'éteindra pas jusqu'à ce que les choses que tu vas commencer à écrire soient accomplies.
\par 26 Et quand tu auras fait, tu publieras certaines choses, et tu montreras certaines choses en secret aux sages : demain à cette heure tu commenceras à écrire.
\par 27 Alors je sortis, comme il l'avait ordonné, et je rassemblai tout le peuple, et je dis :
\par 28 Écoute ces paroles, ô Israël.
\par 29 Nos pères, au commencement, étaient étrangers en Égypte, d'où ils furent délivrés.
\par 30 Et ils ont reçu la loi de la vie, qu'ils n'ont pas observée, et que vous aussi avez transgressée après eux.
\par 31 Alors le pays, le pays de Sion, fut partagé entre vous par tirage au sort ; mais vos pères et vous-mêmes avez commis l'injustice et n'avez pas observé les voies que le Très-Haut vous a prescrites.
\par 32 Et comme il est un juste juge, il vous a ôté à temps ce qu'il vous avait donné.
\par 33 Et maintenant vous êtes ici, et vos frères parmi vous.
\par 34 C'est pourquoi, si vous soumettez votre propre intelligence et réformez vos cœurs, vous serez gardés en vie et après la mort vous obtiendrez miséricorde.
\par 35 Car après la mort viendra le jugement, quand nous revivrons ; et alors les noms des justes seront manifestés, et les œuvres des impies seront publiées.
\par 36 Que personne donc ne vienne à moi maintenant, et ne me cherche pendant quarante jours.
\par 37 J'ai donc pris les cinq hommes, comme il me l'avait ordonné, et nous sommes allés dans les champs et nous y sommes restés.
\par 38 Et le lendemain, voici, une voix m'appela, disant : Esdras, ouvre la bouche, et bois ce que je te donne à boire.
\par 39 Alors j'ouvris la bouche, et voici, il m'approcha d'une coupe pleine, qui était comme pleine d'eau, mais dont la couleur était comme celle du feu.
\par 40 Et j'en pris, et je bus ; et quand j'en eus bu, mon cœur exprima l'intelligence, et la sagesse grandit dans ma poitrine, car mon esprit fortifia ma mémoire.
\par 41 Et ma bouche s'ouvrit et ne se ferma plus.
\par 42 Le Très-Haut donna l'intelligence aux cinq hommes, et ils écrivirent les merveilleuses visions de la nuit qu'on racontait et qu'ils ne connaissaient pas ; et ils restèrent assis quarante jours, et ils écrivirent pendant le jour, et la nuit ils mangèrent du pain.
\par 43 Quant à moi. Je parlais le jour, et je ne me taisais pas la nuit.
\par 44 En quarante jours, ils écrivirent deux cent quatre livres.
\par 45 Et il arriva, lorsque les quarante jours furent remplis, que le Très-Haut parla, disant : La première que tu as écrite, publie-la publiquement, afin que les dignes et les indignes la lisent.
\par 46 Mais garde les soixante-dix derniers, afin de les livrer seulement à ceux qui sont sages parmi le peuple.
\par 47 Car en eux se trouve la source de l'intelligence, la source de la sagesse et le courant de la connaissance.
\par 48 Et je l'ai fait.

\chapitre{15}

\par 1 Voici, dis aux oreilles de mon peuple les paroles de prophétie que je mettrai dans ta bouche, dit l'Éternel :
\par 2 Et faites-les écrire sur du papier : car ils sont fidèles et vrais.
\par 3 Ne crains pas les imaginations contre toi, que l'incrédulité de ceux qui parlent contre toi ne te trouble pas.
\par 4 Car tous les infidèles mourront dans leur infidélité.
\par 5 Voici, dit l'Éternel, je ferai venir des plaies sur le monde ; l'épée, la famine, la mort et la destruction.
\par 6 Car la méchanceté a pollué à l'extrême toute la terre, et leurs œuvres pernicieuses se sont accomplies.
\par 7 C'est pourquoi dit l'Éternel :
\par 8 Je ne tiendrai plus ma langue concernant leurs méchancetés, qu'ils commettent de manière profanatrice, et je ne les souffrirai plus dans les choses dans lesquelles ils s'exercent méchamment. Voici, le sang innocent et juste crie vers moi, et les âmes des justes se plaignent continuellement.
\par 9 Et c'est pourquoi, dit l'Éternel, je les vengerai sûrement, et je recevrai pour moi tout le sang innocent d'entre eux.
\par 10 Voici, mon peuple est conduit comme un troupeau à l'abattoir : je ne permettrai plus qu'il habite au pays d'Égypte.
\par 11 Mais je les ferai venir à main forte et à bras étendu, et je frapperai l'Égypte de plaies, comme auparavant, et je détruirai tout son pays.
\par 12 L'Égypte sera en deuil, et ses fondations seront frappées de la plaie et du châtiment que Dieu fera venir sur elle.
\par 13 Ceux qui cultivent la terre seront en deuil, car leurs graines disparaîtront à cause du souffle et de la grêle, et avec une constellation effrayante.
\par 14 Malheur au monde et à ceux qui l'habitent !
\par 15 Car l'épée et leur destruction approchent, et un peuple se lèvera et combattra un autre, l'épée à la main.
\par 16 Car il y aura des séditions parmi les hommes et des invasions les unes des autres ; ils ne considéreront ni leurs rois ni leurs princes, et le cours de leurs actions dépendra de leur pouvoir.
\par 17 Un homme désirera entrer dans une ville et ne le pourra pas.
\par 18 Car à cause de leur orgueil, les villes seront troublées, les maisons seront détruites, et les hommes seront effrayés.
\par 19 Un homme n'aura pas pitié de son prochain, mais il détruira ses maisons par l'épée et dévastera ses biens, à cause du manque de pain et à cause de grandes tribulations.
\par 20 Voici, dit Dieu, j'appellerai tous les rois de la terre pour me révérer, qui sont du lever du soleil, du midi, de l'orient et du Liban ; se retourner les uns contre les autres et rendre justice aux choses qu'ils leur ont faites.
\par 21 Ce qu'ils font encore aujourd'hui envers mes élus, je le ferai aussi, et je le récompenserai dans leur sein. Ainsi parle le Seigneur Dieu :
\par 22 Ma main droite n'épargnera pas les pécheurs, et mon épée ne cessera pas sur ceux qui ont versé le sang innocent sur la terre.
\par 23 Le feu est sorti de sa colère, et a consumé les fondements de la terre et les pécheurs, comme la paille qui est allumée.
\par 24 Malheur à ceux qui pèchent et qui ne gardent pas mes commandements ! dit le Seigneur.
\par 25 Je ne les épargnerai pas : partez, enfants, loin de la puissance, ne souillez pas mon sanctuaire.
\par 26 Car l'Éternel connaît tous ceux qui pèchent contre lui, et c'est pourquoi il les livre à la mort et à la destruction.
\par 27 Car maintenant les plaies sont tombées sur toute la terre et vous y resterez ; car Dieu ne vous délivrera pas, parce que vous avez péché contre lui.
\par 28 Voici une vision horrible, et son apparence venant de l'orient :
\par 29 Où les nations des dragons d'Arabie sortiront avec de nombreux chars, et la multitude d'entre eux sera emportée comme le vent sur la terre, afin que tous ceux qui les entendent aient peur et tremblent.
\par 30 Et les Carmaniens, furieux, sortiront comme les sangliers des bois, et ils viendront avec une grande puissance, se battront contre eux, et ravageront une partie du pays des Assyriens.
\par 31 Et alors les dragons auront le dessus, se souvenant de leur nature ; et s'ils se retournent, conspirant ensemble avec une grande puissance pour les persécuter,
\par 32 Alors ceux-ci seront troublés, saignés, et garderont le silence par leur puissance, et s'enfuiront.
\par 33 Et depuis le pays des Assyriens, l'ennemi les assiégera et en consumera quelques-uns, et dans leur armée il y aura de la crainte et de l'effroi, et des querelles entre leurs rois.
\par 34 Voici les nuages ​​qui viennent de l'est et du nord jusqu'au sud, et ils sont très horribles à voir, pleins de colère et de tempête.
\par 35 Ils se frapperont les uns les autres, et ils feront tomber une grande multitude d'étoiles sur la terre, même leur propre étoile ; et le sang coulera de l'épée jusqu'au ventre,
\par 36 Et du fumier d'hommes dans le troupeau de chameaux.
\par 37 Et il y aura une grande frayeur et un grand tremblement sur la terre ; et ceux qui verront la colère seront effrayés, et le tremblement les saisira.
\par 38 Et alors de grandes tempêtes viendront du midi et du nord, et une autre partie de l'ouest.
\par 39 Et des vents violents s'élèveront de l'est et l'ouvriront ; et la nuée qu'il a soulevée dans sa colère, et l'étoile agitée pour faire peur vers le vent d'est et d'ouest, seront détruites.
\par 40 Les nuages ​​grands et puissants s'enfleront de colère, ainsi que l'étoile, pour effrayer toute la terre et ceux qui y habitent ; et ils répandront sur tout lieu élevé et éminent une étoile horrible,
\par 41 Feu, grêle, épées volantes, et eaux abondantes, afin que tous les champs soient remplis, et que tous les fleuves soient remplis de grandes eaux en abondance.
\par 42 Et ils démoliront les villes et les murailles, les montagnes et les collines, les arbres des bois, l'herbe des prairies et leur blé.
\par 43 Et ils iront avec fermeté à Babylone, et lui feront peur.
\par 44 Ils viendront vers elle et l'assiégeront, ils déverseront sur elle l'étoile et toute la colère ; alors la poussière et la fumée monteront jusqu'au ciel, et tous ceux qui seront autour d'elle la pleureront.
\par 45 Et ceux qui resteront sous elle rendront service à ceux qui lui ont fait peur.
\par 46 Et toi, Asie, qui as part à l'espérance de Babylone et qui es la gloire de sa personne :
\par 47 Malheur à toi, misérable, parce que tu t'es rendu semblable à elle ; et tu as paré tes filles de prostitution, afin qu'elles puissent plaire et se glorifier de tes amants, qui ont toujours désiré se livrer à la prostitution avec toi.
\par 48 Tu as suivi celle qui est haïe dans toutes ses œuvres et inventions : c'est pourquoi dit Dieu :
\par 49 J'enverrai sur toi des plaies; le veuvage, la pauvreté, la famine, l'épée et la peste, pour ravager tes maisons par la destruction et la mort.
\par 50 Et la gloire de ta puissance sera séchée comme une fleur, la chaleur envoyée sur toi s'élèvera.
\par 51 Tu seras affaiblie comme une pauvre femme par des coups, et comme une femme châtiée par des blessures, afin que les puissants et les amants ne puissent te recevoir.
\par 52 Aurais-je agi ainsi contre toi avec jalousie, dit l'Éternel,
\par 53 Si tu n'avais pas toujours tué mes élus, en exaltant le coup de tes mains, et en disant sur leurs morts, quand tu étais ivre :
\par 54 Montre-moi la beauté de ton visage ?
\par 55 La récompense de ta prostitution sera dans ton sein, c'est pourquoi tu recevras la récompense.
\par 56 Comme tu as fait à mes élus, dit l'Éternel, ainsi Dieu te fera et te livrera au malheur.
\par 57 Tes enfants mourront de faim, et tu tomberas sous l'épée ; tes villes seront détruites, et toutes les tiennes périront par l'épée dans les champs.
\par 58 Ceux qui sont dans les montagnes mourront de faim, mangeront leur propre chair et boiront leur propre sang, à cause de la faim de pain et de la soif d'eau.
\par 59 Toi, comme malheureux, tu traverseras la mer et tu recevras de nouveau des fléaux.
\par 60 Et dans le passage ils se précipiteront sur la ville oisive, et détruiront une partie de ton pays, et consumeront une partie de ta gloire, et retourneront à Babylone qui a été détruite.
\par 61 Et tu seras abattu par eux comme du chaume, et ils seront pour toi comme du feu ;
\par 62 Et ils te consumeront, toi et tes villes, ton pays et tes montagnes ; tous tes bois et tes arbres fruitiers seront brûlés au feu.
\par 63 Ils emmèneront tes enfants en captivité, et regarde, ce que tu as, ils le gâteront et gâcheront la beauté de ton visage.

\chapitre{16}

\par 1 Malheur à toi, Babylone et Asie ! malheur à toi, Égypte et Syrie !
\par 2 Ceignez-vous de toiles de jute et de poils, pleurez vos enfants et soyez désolés ; car ta destruction est proche.
\par 3 Une épée est envoyée contre vous, et qui peut la repousser ?
\par 4 Un feu est envoyé parmi vous, et qui peut l'éteindre ?
\par 5 Des plaies vous sont envoyées, et quel est celui qui peut les chasser ?
\par 6 Quelqu'un peut-il chasser un lion affamé dans la forêt ? ou peut-on éteindre le feu dans le chaume, quand il a commencé à brûler ?
\par 7 Peut-on retourner la flèche tirée par un archer fort ?
\par 8 Le Seigneur puissant envoie les plaies et qui est celui qui peut les chasser ?
\par 9 Un feu sortira de sa colère, et qui est celui qui pourra l'éteindre ?
\par 10 Il lancera des éclairs, et qui ne craindrait ? il tonnera, et qui n'aura pas peur ?
\par 11 Le Seigneur menacera, et qui ne sera pas complètement réduit en poudre en sa présence ?
\par 12 La terre et ses fondements tremblent ; la mer s'élève avec des vagues de l'abîme, et ses vagues sont agitées, ainsi que ses poissons, devant l'Éternel et devant la gloire de sa puissance :
\par 13 Car sa main droite qui tend l'arc est forte, ses flèches qu'il lance sont pointues et ne manqueront pas lorsqu'elles commenceront à être lancées jusqu'aux extrémités du monde.
\par 14 Voici, les plaies sont envoyées, et ne reviendront plus, jusqu'à ce qu'elles arrivent sur la terre.
\par 15 Le feu s'allume et ne s'éteindra pas jusqu'à ce qu'il consume les fondements de la terre.
\par 16 Comme une flèche lancée par un puissant archer ne revient pas en arrière : de même les plaies qui seront envoyées sur la terre ne reviendront plus.
\par 17 Malheur à moi ! Pauvre de moi! qui me délivrera en ces jours-là ?
\par 18 Le début des douleurs et des grands deuils ; le début de la famine et de la grande mort ; le début des guerres, et les puissances auront peur ; le début des maux ! que ferai-je quand ces maux arriveront ?
\par 19 Voici, la famine et la peste, la tribulation et l'angoisse sont envoyées comme des fléaux pour se corriger.
\par 20 Mais malgré toutes ces choses, ils ne se détourneront pas de leur méchanceté et ne se souviendront pas toujours des fléaux.
\par 21 Voici, les vivres seront si bon marché sur la terre qu'ils se croiront en bonne santé, et même alors les maux grandiront sur la terre, l'épée, la famine et une grande confusion.
\par 22 Car beaucoup d'entre eux qui habitent sur la terre périront de famine ; et l'autre, qui échappera à la faim, sera détruit par l'épée.
\par 23 Et les morts seront jetés comme du fumier, et il n'y aura personne pour les consoler ; car la terre sera dévastée, et les villes seront détruites.
\par 24 Il ne restera plus personne pour cultiver la terre et pour la semer.
\par 25 Les arbres donneront des fruits, et qui les récoltera ?
\par 26 Les raisins mûriront, et qui les foulera ? car tous les lieux seront déserts d'hommes :
\par 27 De sorte qu'un homme désirera en voir un autre et entendre sa voix.
\par 28 Car il en restera dix de la ville, et deux des champs, qui se cacheront dans les bosquets touffus et dans les fentes des rochers.
\par 29 Comme dans un verger d'oliviers, sur chaque arbre il reste trois ou quatre olives ;
\par 30 Ou comme lorsqu'une vigne est vendangée, il en reste quelques grappes qui cherchent diligemment à travers la vigne :
\par 31 De même, en ces jours-là, il en restera trois ou quatre de ceux qui fouilleront leurs maisons avec l'épée.
\par 32 Et la terre sera dévastée, et ses champs vieilliront, et ses voies et tous ses sentiers seront pleins d'épines, parce que personne n'y passera.
\par 33 Les vierges seront dans le deuil, n'ayant pas d'époux ; les femmes pleureront, n'ayant pas de mari ; leurs filles seront dans le deuil, sans aide.
\par 34 Dans les guerres, leurs fiancés seront détruits, et leurs maris périront de famine.
\par 35 Écoutez maintenant ces choses et comprenez-les, vous, serviteurs du Seigneur.
\par 36 Voici, la parole du Seigneur, recevez-la : ne croyez pas les dieux dont le Seigneur a parlé.
\par 37 Voici, les plaies approchent et ne se relâchent pas.
\par 38 Comme lorsqu'une femme enceinte, au neuvième mois, enfante son fils, deux ou trois heures après sa naissance, de grandes douleurs enveloppent son ventre, qui, lorsque l'enfant sort, ne se relâche pas un instant :
\par 39 De même, les plaies ne tarderont pas à s'abattre sur la terre, et le monde sera en deuil, et les tristesses l'atteindront de toutes parts.
\par 40 Ô mon peuple, écoutez ma parole : préparez-vous à votre combat, et dans ces maux soyez comme des pèlerins sur la terre.
\par 41 Celui qui vend, qu'il soit comme celui qui s'enfuit, et celui qui achète, comme celui qui perdra.
\par 42 Celui qui occupe une marchandise comme celui qui n'en tire aucun profit, et celui qui bâtit comme celui qui n'y habite pas :
\par 43 Celui qui sème comme s'il ne moissonnait pas, ainsi celui qui plante la vigne comme celui qui ne vendange pas les raisins :
\par 44 Ceux qui se marient, comme ceux qui n'auront pas d'enfants ; et ceux qui ne se marient pas, comme les veufs.
\par 45 Et c'est pourquoi ceux qui travaillent travaillent en vain :
\par 46 Car des étrangers récolteront leurs fruits, pilleront leurs biens, renverseront leurs maisons et emmèneront leurs enfants en captivité, car c'est dans la captivité et dans la famine qu'ils auront des enfants.
\par 47 Et ceux qui pillent leurs marchandises, décorent davantage leurs villes, leurs maisons, leurs biens et leurs propres personnes.
\par 48 Plus je serai en colère contre eux à cause de leur péché, dit l'Éternel.
\par 49 Comme une putain envie une femme honnête et vertueuse :
\par 50 Ainsi la justice détestera l'iniquité quand elle se pare, et l'accusera en face quand viendra celui qui défendra celui qui recherche diligemment tout péché sur la terre.
\par 51 Et c'est pourquoi ne soyez pas semblables à lui, ni à ses œuvres.
\par 52 Car encore peu, et l'iniquité sera ôtée de la terre, et la justice régnera parmi vous.
\par 53 Que le pécheur ne dise pas qu'il n'a pas péché ; car Dieu brûlera des charbons ardents sur sa tête, qui dit devant le Seigneur Dieu et sa gloire : Je n'ai pas péché.
\par 54 Voici, l'Éternel connaît toutes les œuvres des hommes, leurs imaginations, leurs pensées et leurs cœurs :
\par 55 Qui n'a prononcé que la parole : Que la terre soit faite ; et il fut fait : Que le ciel soit fait ; et il a été créé.
\par 56 Dans sa parole ont été faites les étoiles, et il en connaît le nombre.
\par 57 Il sonde l'abîme et ses trésors ; il a mesuré la mer et ce qu'elle contient.
\par 58 Il a fermé la mer au milieu des eaux, et par sa parole il a suspendu la terre aux eaux.
\par 59 Il étend les cieux comme une voûte ; il l'a fondé sur les eaux.
\par 60 Dans le désert, il a fait des sources d'eau et des étangs sur le sommet des montagnes, afin que les flots descendent des hauts rochers pour arroser la terre.
\par 61 Il a créé l'homme, et a mis son cœur au milieu du corps, et lui a donné le souffle, la vie et l'intelligence.
\par 62 Oui, et l'Esprit du Dieu Tout-Puissant, qui a fait toutes choses et qui sonde toutes choses cachées dans les secrets de la terre,
\par 63 Assurément, il connaît vos inventions et ce que vous pensez dans votre cœur, même ceux qui pèchent et veulent cacher leur péché.
\par 64 C'est pourquoi l'Eternel a soigneusement examiné toutes vos œuvres, et il vous fera tous honte.
\par 65 Et lorsque vos péchés seront manifestés, vous aurez honte devant les hommes, et vos propres péchés seront vos accusateurs en ce jour-là.
\par 66 Que ferez-vous ? ou comment cacherez-vous vos péchés devant Dieu et ses anges ?
\par 67 Voici, Dieu lui-même est le juge, craignez-le : abandonnez vos péchés et oubliez vos iniquités, pour ne plus vous en mêler pour toujours : ainsi Dieu vous fera sortir et vous délivrera de toute détresse.
\par 68 Car voici, la colère ardente d'une grande multitude s'enflamme contre vous, et ils enlèveront certains d'entre vous et vous nourriront, pendant votre oisiveté, de choses offertes aux idoles.
\par 69 Et ceux qui y consentent seront pris en dérision et dans l'opprobre, et foulés aux pieds.
\par 70 Car il y aura partout, et dans les villes voisines, une grande insurrection contre ceux qui craignent l'Éternel.
\par 71 Ils seront comme des fous, n'épargnant personne, mais pillant et détruisant néanmoins ceux qui craignent l'Éternel.
\par 72 Car ils gaspilleront et emporteront leurs biens, et les chasseront de leurs maisons.
\par 73 Alors on connaîtra qui sont mes élus ; et ils seront éprouvés comme l'or dans le feu.
\par 74 Écoutez, ô vous mes bien-aimés, dit l'Éternel : voici, les jours de détresse sont proches, mais je vous en délivrerai.
\par 75 N'ayez pas peur et ne doutez pas ; car Dieu est ton guide,
\par 76 Et le guide de ceux qui gardent mes commandements et mes préceptes, dit le Seigneur Dieu : Que vos péchés ne vous alourdissent pas, et que vos iniquités ne se soulèvent pas.
\par 77 Malheur à ceux qui sont liés par leurs péchés et couverts de leurs iniquités comme un champ est couvert de buissons et son chemin couvert d'épines, que personne ne peut traverser !
\par 78 On le laisse nu et on le jette au feu pour y être consumé.

\end{document}