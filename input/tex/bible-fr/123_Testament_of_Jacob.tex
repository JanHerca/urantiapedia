\begin{document}


\title{Testament de Jacob}

\chapter{1}

\par 1 Au nom du Père, du Fils et du Saint-Esprit, le Dieu unique.

\par 2 Nous commençons, avec l'aide de Dieu Très-Haut et par sa médiation, à écrire l'histoire de la vie de notre père, le patriarche Jacob, fils du patriarche Isaac, le vingt-huitième jour du mois de Misri.

\par 3 Que la bénédiction de sa prière nous garde et nous protège des tentations de l'ennemi obstiné. Amen, amen, amen !

\par 4 Il dit : « Venez et écoutez, mes bien-aimés et mes frères qui aiment le Seigneur, ce qui a été reçu. »

\par 5 Or, lorsque le temps de notre père Jacob, père des pères, fils d'Isaac, fils d'Abraham, approchait et approchait pour qu'il puisse s'éloigner de son corps, ce fidèle était avancé en années et en distinction.

\par 6 Alors le Seigneur lui envoya Michel, le chef des anges, qui lui dit : « Ô Israël, mon bien-aimé, de noble lignée, écris ton héritage annoncé et ton instruction pour ta maison et donne-leur une alliance ; Occupez-vous aussi du bon ordre de votre maison, car le moment est proche pour vous d'aller vers vos pères et de vous réjouir avec eux pour toujours.

\par 7 Ainsi, lorsque notre père Jacob, le fidèle, entendit cela de la part de l'ange, il répondit et dit, comme il avait l'habitude de parler chaque jour de cette manière avec les anges :

\par 8 « Que la volonté du Seigneur soit faite. »

\par 9 Et Dieu prononça une bénédiction sur notre père Jacob. Jacob avait un endroit isolé dans lequel il entrait pour offrir ses prières devant le Seigneur la nuit et le jour.

\par 10 Les anges le visitaient, le gardaient et le fortifiaient en toutes choses.

\par 11 Dieu le bénit et multiplia son peuple au pays d'Egypte au moment où il descendit au pays d'Egypte à la rencontre de son fils Joseph.

\par 12 Ses yeux étaient devenus ternes à force de pleurer, mais lorsqu'il descendit en Égypte, il vit clairement en voyant son fils.

\par 13 Alors Jacob-Israël s'inclina la face contre terre, puis tomba sur le cou de son fils Joseph et l'embrassa, tout en pleurant et en disant : « Je peux mourir maintenant, ô mon fils, parce que j'ai vu ton visage une fois. plus dans ma vie ; Ô mon fils bien-aimé.


\chapter{2}


\par 1 Joseph continua à régner sur toute l'Égypte, tandis que Jacob resta dix-sept ans dans le pays de Goshen et devint très vieux, de sorte que sa durée de vie était achevée.

\par 2 Il gardait continuellement tous les commandements et craignait le Seigneur.

\par 3 Ses yeux s'obscurcirent et sa vie était si proche de la fin qu'il ne pouvait voir une seule personne à cause de sa longue vie et de sa sénilité.

\par 4 Alors il leva les yeux vers la lumière d'Isaac, mais il eut peur et fut troublé.

\par 5 Alors l'ange lui dit : « Ne crains pas, ô Jacob ; Je suis l'ange qui marche avec toi et te garde depuis ton enfance.

\par 6 J'ai annoncé que tu recevrais la bénédiction de ton père et de Rébecca, ta mère.

\par 7 Je suis celui qui est avec toi, ô Israël, dans tous tes actes et dans tout ce dont tu as été témoin.

\par 8 Je t'ai sauvé de Laban ? quand il vous mettait en danger et vous poursuivait.

\par 9 En ce temps-là, je vous ai donné tous ses biens et je vous ai bénis, ainsi que vos femmes, vos enfants et vos troupeaux.

\par 10 « Je suis celui qui t'ai sauvé de la main d'Ésaü.

\par 11 C'est moi qui t'ai accompagné jusqu'au pays d'Égypte, ô Israël, et un très grand peuple t'a été donné.

\par 12 Bienheureux ton père Abraham, car il est devenu l'ami de Dieu - qu'il (Dieu) soit exalté ! - à cause de sa générosité et de son amour des étrangers.

\par 13 Bienheureux est ton père Isaac qui t'a engendré, car il était un sacrifice parfait, agréable à Dieu.

\par 14 « Bienheureux es-tu aussi, ô Jacob, car tu as vu Dieu face à face.

\par 15 Vous avez vu l'ange de Dieu — qu'il soit exalté ! — et vous avez vu l'échelle qui se tenait fermement sur le sol et dont le sommet était dans les cieux.

\par 16 Alors vous avez vu le Seigneur assis au sommet avec une puissance que personne ne pouvait décrire.

\par 17 Tu as pris la parole et tu as dit : « Ceci est la maison de Dieu et ceci est la porte du ciel. »

\par 18 Bienheureux es-tu, car tu t'es approché de Dieu et il est fort parmi les hommes, alors maintenant ne sois pas troublé, ô élu de Dieu.

\par 19 « Béni sois-tu, ô Israël, et bénie soit toute ta postérité.

\par 20 Car vous serez tous appelés « les patriarches » jusqu'à la fin des temps et des époques ; vous êtes le peuple et la lignée des serviteurs de Dieu.

\par 21 Béni soit la nation qui luttera pour ta pureté et verra tes bonnes œuvres.

\par 22 Béni soit l'homme qui se souviendra de toi le jour de ta noble fête.

\par 23 Bienheureux soit celui qui accomplira des actes de miséricorde en l'honneur de vos différents noms, et donnera à quelqu'un un verre d'eau à boire, ou viendra avec une offrande au sanctuaire, ou accueillera des étrangers, ou visitera le sanctuaire. malades et consoleront leurs enfants, ou vêtiront un enfant nu en l'honneur de vos différents noms.

\par 24 « Un tel homme ne manquera d'aucun des biens de ce monde, ni de la vie éternelle dans le monde à venir.

\par 25 De plus, quiconque aura fait écrire à ses frais le récit de vos diverses vies et souffrances, ou les aura écrits de sa propre main, ou les aura lu sobrement, ou les entendra avec foi, ou se souviendront de vos actes : de telles personnes verront leurs péchés et leurs offenses pardonnés, et ils iront à cause de vous et de votre progéniture dans le royaume des cieux.

\par 26 « Et maintenant, lève-toi, Jacob, car tu seras transféré des difficultés et des douleurs du cœur au repos éternel, et tu entreras dans le repos qui ne passera pas, dans la miséricorde, la lumière éternelle et la joie spirituelle.

\par 27 Maintenant donc, fais ta déclaration à ta maison, et que la paix soit sur toi, car je vais aller vers celui qui m'a envoyé.

\chapter{3}

\par 1 Ainsi, lorsque l'ange eut fait cette déclaration à notre père Jacob, il monta de lui au ciel, tandis que Jacob lui faisait ses adieux.

\par 2 Ceux qui étaient autour de Jacob l'entendirent tandis qu'il remerciait Dieu et le glorifiait de louanges.

\par 3 Et tous les membres de sa maison, grands et petits, se rassemblaient autour de lui, pleurant sur lui, profondément attristés et disant : « Vous vous en allez et nous laissez orphelins. »

\par 4 Et ils lui répétaient : « Ô notre père bien-aimé, que ferons-nous, car nous sommes dans un pays étranger ?

\par 5 Alors Jacob leur dit : Ne craignez pas ; Dieu lui-même m'est apparu en Haute Mésopotamie et m'a dit : « Je suis le Dieu de vos pères ; ne crains pas, car je suis avec toi pour toujours et avec ta postérité qui viendra après toi.

\par 6 Ce pays dans lequel tu es, je vais le donner à toi et à ta postérité après toi pour toujours.

\par 7 Et n'ayez pas peur de descendre en Égypte.

\par 8 Je ferai pour toi un grand peuple et ta descendance augmentera et se multipliera pour toujours.

\par 9 Joseph mettra sa main sur tes yeux et ton peuple se multipliera au pays d'Egypte.

\par 10 Ensuite, ils reviendront à cet endroit et seront sans souci.

\par 11 Je leur ferai du bien à cause de toi, même si pour le moment ils seront déplacés d'ici.'»

\chapter{4}

\par 1 Après cela, le moment pour Jacob-Israël de quitter son corps était arrivé.

\par 2 Alors il appela Joseph et lui dit : « Si tu as trouvé grâce, place ta main bénie sous mon côté et jure devant l'Éternel que tu placeras mon corps dans le tombeau de mes pères. »

\par 3 Alors Joseph lui dit : Je ferai exactement ce que tu me commanderas, ô bien-aimé de Dieu.

\par 4 Mais il dit à Joseph : « Je veux que tu me le jures. »

\par 5 Joseph jura donc à Jacob, son père, de porter son corps au tombeau de ses pères, et Jacob accepta le serment de son fils.

\par 6 Par la suite, cette nouvelle parvint à Joseph : « Ton père est devenu inquiet. »

\par 7 Il prit donc ses deux fils, Éphraïm et Manassé, et se rendit devant son père Jacob.

\par 8 Joseph lui dit : « Ce sont mes fils que Dieu m'a donnés au pays d'Égypte pour me suivre. »

\par 9 Israël dit : « Rapprochez-les de moi ici. »

\par 10 Car les yeux d'Israël étaient devenus obscurs à cause de son âge avancé, au point qu'il ne pouvait plus voir.

\par 11 Alors Joseph rapprocha ses fils et Jacob les embrassa.

\par 12 Alors Joseph leur ordonna, à savoir Éphraïm et Manassé, de se prosterner jusqu'à terre devant Jacob.

\par 13 Joseph prit Manassé et le plaça à la droite d'Israël et Ephraïm à sa gauche.

\par 14 Mais Israël renversa ses mains et laissa sa main droite reposer sur la tête d'Éphraïm et sa main gauche sur la tête de Manassé.

\par 15 Il les bénit et les rendit à leur père et dit : « Que le Dieu sous l'autorité duquel mes pères, Abraham et Isaac, aient servi avec révérence, le Dieu qui m'a fortifié depuis ma jeunesse jusqu'à ce jour où le L'ange m'a sauvé de toutes mes afflictions, puisse-t-il bénir ces garçons, Manassé et Éphraïm.

\par 16 Que mon nom soit sur eux, ainsi que les noms de mes saints pères, Abraham et Isaac.

\par 17 Après cela, Israël dit à Joseph : « Je mourrai, et vous retournerez tous au pays de vos pères et Dieu sera avec vous.

\par 18 Et vous avez personnellement reçu une grande faveur, plus grande que celle de vos frères, car j'ai pris cette flèche avec mon arc et mon épée des Amoréens (?).

\chapter{5}

\par 1 Alors Jacob fit appeler tous ses enfants et leur dit : « Rassemblez-vous autour de moi afin que je vous informe de tout ce qui vous arrivera et de ce qui arrivera à chacun de vous dans les derniers jours. »

\par 2 Ils se rassemblèrent donc autour d'Israël, depuis l'aîné jusqu'au plus jeune d'entre eux.

\par 3 Alors Jacob-Israël prit la parole et dit à ses fils : « Écoutez, ô fils de Jacob, écoutez votre père Israël, depuis Ruben, mon premier-né, jusqu'à Benjamin. »

\par 4 Alors il leur raconta ce qui arriverait aux douze enfants, appelant chacun d'eux et sa tribu par leur nom ; et il les bénit de la bénédiction céleste.

\par 5 Après cela, ils restèrent silencieux pendant un court moment afin qu'il puisse se reposer.

\par 6 Alors les cieux se réjouirent qu'il puisse observer les lieux de repos.*

\par 7 Et voici, de nombreux bourreaux s'approchaient, différents par leurs aspects.

\par 8 Ils étaient prêts à tourmenter les pécheurs, qui sont ceux-ci : les adultères, hommes et femmes ; ceux qui convoitent les mâles ; les vicieux qui dégradent le sperme donné par Dieu ; les astrologues et les sorciers ; les malfaiteurs et les adorateurs d'idoles qui s'accrochent aux abominations ; et les calomniateurs qui jugent (?) avec deux langues (trompeusement).

\par 9 Et quant à tous ces pécheurs, leur châtiment est le feu qui ne s'éteint pas et les ténèbres extérieures où il y a des pleurs et des grincements de dents.

\par 10 [Ici il y a une lacune dans le texte arabe. Dans le Bohairic, Jacob est de nouveau élevé, cette fois au ciel, où tout est lumière et joie.

\par 11 Il voit Abraham et Isaac et se voit montrer toutes les joies des rachetés.

\par 12 Jacob revient sur terre, donne des instructions pour son enterrement au pays de ses pères, et décède à l'âge de 147 ans.

\par 13 Le Seigneur descend avec les anges Michel et Gabriel pour porter l'âme de Jacob au ciel.

\par 14 Joseph ordonne que le corps de son père soit embaumé à la manière égyptienne.

\par 15 Quarante jours sont consacrés à l'embaumement, et quatre-vingts jours supplémentaires sont consacrés au deuil du patriarche. ]

\chapter{6}

\par 1 Et lorsque les jours de leur deuil furent terminés, Pharaon pleurait encore sur Jacob à cause de sa considération pour Joseph.

\par 2 Alors Joseph s'adressa aux nobles de Pharaon et leur dit : Puisque j'ai trouvé grâce auprès de vous, parlerez en ma faveur au roi Pharaon, et lui direz que Jacob m'a fait jurer que lorsqu'il partira, de son corps, j'enterrerais son corps dans le tombeau de mes pères au pays de Canaan, à cet endroit même ?

\par 3 Alors Pharaon dit à Joseph : « Va en paix et enterre ton père conformément au serment qu'il t'a demandé.

\par 4 Et prends avec toi des chars et des chevaux, le meilleur de mon royaume et de ma propre maison, comme tu le désires.

\par 5 Alors Joseph adora Dieu en présence de Pharaon, sortit de lui et partit pour enterrer son père.

\par 6 Et là partirent avec lui les esclaves de Pharaon, les anciens de l'Égypte, toute la maison de Joseph, ses frères et tout Israël.

\par 7 Ils montèrent tous avec lui dans les chars, et la suite avançait comme une grande armée.

\par 8 Ils descendirent au pays de Canaan jusqu'au bord du fleuve, de l'autre côté du Jourdain, et ils le pleurèrent dans ce lieu avec une très grande douleur.

\par 9 Pendant sept jours, ils maintinrent sur lui cette grande douleur.

\par 10 Ainsi, lorsque les habitants de Dan entendirent parler du deuil dans leur pays, ils dirent : Ce grand deuil est celui des Égyptiens.

\par 11 Jusqu'à ce jour [ils appellent ce lieu « le deuil des Egyptiens » ].

\par 12 Alors Israël fut emmené et fut enterré au pays de Canaan dans le deuxième tombeau.

\par 13 C'est celui qu'Abraham avait acheté avec autorisation pour les enterrements à Ephron en face de Mamré.

\par 14 Après cela, Joseph retourna au pays d'Égypte avec ses frères et toute la suite de Pharaon.

\par 15 Et Joseph vécut plusieurs années après la mort de son père.

\par 16 Il continua à régner sur l'Égypte, bien que Jacob soit mort et ait été laissé avec son propre peuple.

\chapter{7}

\par 1 Voici ce que nous avons transmis : Nous avons décrit la disparition et le deuil du père des pères, Jacob-Israël, dans la mesure de nos possibilités ; aussi tel qu'il est écrit dans les livres spirituels de Dieu et tel que nous l'avons trouvé dans l'ancien trésor de connaissance de nos pères, les saints et purs apôtres.

\par 2 Et si vous souhaitez connaître l'histoire de la vie et acquérir de nouvelles connaissances sur le père des pères, Jacob, alors prenez un père qui est attesté dans l'Ancien Testament.

\par 3 Moïse est celui qui l'a écrit, le premier des prophètes, l'auteur de la Loi.

\par 4 Lisez-le et éclairez vos idées.

\par 5 Vous y trouverez ceci et bien plus encore, écrit pour vous.

\par 6 Vous découvrirez que Dieu et ses anges étaient leurs amis lorsqu'ils étaient dans leur corps, et que Dieu n'arrêtait pas de leur parler à plusieurs reprises dans divers passages du Livre.

\par 7 Aussi il dit dans de nombreux passages à propos du patriarche Jacob, le père des pères, dans le Livre, ainsi : « Mon fils, je bénirai ta descendance comme les étoiles des cieux. »

\par 8 Et notre père Jacob parlait à son fils Joseph et lui disait : « Mon Dieu m'est apparu au pays de Canaan à Luz et m'a béni et m'a dit : 'Je te bénirai, je te multiplierai et te rendrai un peuple puissant.

\par 9 Ils sortiront (à la guerre ?) comme les autres nations de cette terre et ta descendance augmentera pour toujours. »

\par 10 C'est ce que nous avons entendu, ô mes frères et mes bien-aimés, de la part de nos pères, les patriarches.

\par 11 Et il nous incombe d'avoir du zèle pour leurs actes, leur pureté, leur foi, leur amour de l'humanité et leur acceptation des étrangers ; afin que nous puissions prétendre être leurs fils dans le royaume des cieux, afin qu'ils intercèdent pour nous devant Dieu afin que nous soyons sauvés des tourments de l'enfer.

\par 12 Ce sont ceux-là que les Arabes ont désignés comme les saints pères.

\par 13 Jacob instruisit ses fils en ce qui concerne le châtiment, et il les appelait l'épée du Seigneur, qui est le fleuve de feu, préparé avec ses vagues pour engloutir les malfaiteurs et les impurs.

\par 14 Ce sont là les choses dont le père des pères, Jacob, a exposé et enseigné la puissance à tous ses fils, afin que les sages entendent et recherchent la justice dans un amour mutuel avec miséricorde et compassion.

\par 15 Car la miséricorde sauve les hommes des châtiments et la miséricorde triomphe d'une multitude d'injustices.

\par 16 En vérité, celui qui fait miséricorde aux pauvres, celui-là fait un prêt à Dieu.

\par 17 Maintenant donc, mes fils bien-aimés, ne vous relâchez jamais de la prière et du jeûne, et par la vie de la religion vous chasserez les démons.

\par 18 Ô mon fils bien-aimé, évite les mauvaises voies du monde, qui sont la colère, la dépravation et toutes les actions vicieuses.

\par 19 Et méfiez-vous de l'injustice, du blasphème et de l'enlèvement.

\par 20 Car les injustes n'hériteront pas du royaume de Dieu, ni les adultères, ni les maudits, ni ceux qui commettent des outrages et ont des relations sexuelles avec des hommes, ni les gourmands, ni les adorateurs d'idoles, ni ceux qui prononcent des imprécations. , ni ceux qui se polluent en dehors du pur mariage ; et d'autres que nous n'avons pas présentés ni même mentionnés ne s'approcheront pas du royaume de Dieu.

\par 21 Ô mes fils, honorez les saints, car ce sont eux qui intercèderont pour vous.

\par 22 Ô mes fils, soyez généreux envers les étrangers et vous recevrez exactement ce qui a été donné au grand Abraham, le père des pères, et à notre père Isaac, son fils.

\par 23 Ô mes fils, faites pour les pauvres ce qui augmentera la compassion pour eux ici et maintenant, afin que Dieu vous donne le pain de vie pour toujours dans le royaume de Dieu.

\par 24 Car à celui qui a donné du pain à un pauvre dans ce monde, Dieu donnera une part de l'arbre de vie.

\par 25 Habillez le pauvre qui est nu sur la terre, afin que Dieu vous revête du vêtement de gloire dans le royaume des cieux, et vous serez les fils de nos saints pères, Abraham, Isaac et Jacob dans les cieux. pour toujours.

\par 26 Soyez préoccupé par la lecture de la parole de Dieu dans ses livres ici-bas, et souvenez-vous des saints qui ont écrit leur vie, leurs souffrances et leurs prosternations dans la prière.

\par 27 À l'avenir, il ne sera pas empêché qu'ils soient inscrits dans le livre de vie dans le royaume des cieux.

\par 28 Et vous serez comptés parmi les saints, ceux qui ont plu à Dieu dans leur vie et qui vous réjouiront avec les anges au pays de la vie éternelle.

\chapter{8}

\par 1 Vous honorerez la mémoire de nos pères, les patriarches, à cette époque chaque année et ce même jour, qui est le vingt-huitième du mois de Misri.

\par 2 C'est ce que nous avons trouvé écrit dans les anciens documents de nos pères, les saints qui plaisaient à Dieu.

\par 3 Grâce à leur intercession et à leur prière, nous aurons toutes choses, à savoir une part et une place dans le royaume des cieux qui appartient à notre Seigneur et notre Dieu et à notre Maître et notre Sauveur, Jésus le Messie.

\par 4 C'est lui à qui nous demandons de nous pardonner nos fautes et nos fautes et de passer outre nos méfaits.

\par 5 Qu'il soit bon envers nous le jour de son jugement et que nous entendions la voix pleine de joie, de bonté et d'allégresse, disant : « Venez à moi, ô bienheureux de mon Père, héritez du royaume qui était le vôtre. d'avant la création du monde.

\par 6 Et puissions-nous être dignes de recevoir ses divins secrets, qui sont les moyens du pardon de nos péchés.

\par 7 Qu'il nous aide à sauver nos âmes, et qu'il nous détourne des coups du méchant ennemi.

\par 8 Qu'il nous laisse se tenir à sa droite, au grand et terrible jour, pour l'intercession de la maîtresse des intercessions, source de pureté, de générosité et de bénédictions, la mère du salut ;* et pour l'intercession de tous les martyrs, saints, faiseurs d'œuvres agréables et tous ceux qui ont plu au Seigneur par leurs œuvres pieuses et leur bonne volonté.

\par 9 Amen, amen, amen. Et louange à Dieu toujours, pour toujours, éternellement.

\end{document}