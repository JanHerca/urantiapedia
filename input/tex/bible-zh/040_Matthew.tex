\begin{document}

\title{马太福音}


\chapter{1}

\par 1 亚伯拉罕的後裔,大卫的子孙(後裔,子孙:原文作儿子;下同)耶稣基督的家谱:
\par 2 亚伯拉罕生以撒;以撒生雅各;雅各生犹大和他的弟兄;
\par 3 犹大从他玛氏生法勒斯和谢拉;法勒斯生希斯仑;希斯仑生亚兰;
\par 4 亚兰生亚米拿达;亚米拿达生拿顺;拿顺生撒门;
\par 5 撒门从喇合氏生波阿斯;波阿斯从路得氏生俄备得;俄备得生耶西;
\par 6 耶西生大卫王。大卫从乌利亚的妻子生所罗门;
\par 7 所罗门生罗波安;罗波安生亚比雅;亚比雅生亚撒;
\par 8 亚撒生约沙法;约沙法生约兰;约兰生乌西亚;
\par 9 乌西亚生约坦;约坦生亚哈斯;亚哈斯生希西家;
\par 10 希西家生玛拿西;玛拿西生亚们;亚们生约西亚;
\par 11 百姓被迁到巴比伦的时候,约西亚生耶哥尼雅和他的弟兄。
\par 12 迁到巴比伦之後,耶哥尼雅生撒拉铁;撒拉铁生所罗巴伯;
\par 13 所罗巴伯生亚比玉;亚比玉生以利亚敬;以利亚敬生亚所;
\par 14 亚所生撒督;撒督生亚金;亚金生以律;
\par 15 以律生以利亚撒;以利亚撒生马但;马但生雅各;
\par 16 雅各生约瑟,就是马利亚的丈夫。那称为基督的耶稣是从马利亚生的。
\par 17 这样,从亚伯拉罕到大卫共有十四代;从大卫到迁至巴比伦的时候也有十四代;从迁至巴比伦的时候到基督又有十四代。
\par 18 耶稣基督降生的事记在下面:他母亲马利亚已经许配了约瑟,还没有迎娶,马利亚就从圣灵怀了孕。
\par 19 他丈夫约瑟是个义人,不愿意明明的羞辱他,想要暗暗的把他休了。
\par 20 正思念这事的时候,有主的使者向他梦中显现,说:「大卫的子孙约瑟,不要怕!只管娶过你的妻子马利亚来,因他所怀的孕是从圣灵来的。
\par 21 他将要生一个儿子,你要给他起名叫耶稣,因他要将自己的百姓从罪恶里救出来。」
\par 22 这一切的事成就是要应验主藉先知所说的话,
\par 23 说:「必有童女怀孕生子;人要称他的名为以马内利。」(以马内利翻出来就是「神与我们同在」。)
\par 24 约瑟醒了,起来,就遵著主使者的吩咐把妻子娶过来;
\par 25 只是没有和他同房,等他生了儿子(有古卷:等他生了头胎的儿子),就给他起名叫耶稣。

\chapter{2}

\par 1 当希律王的时候,耶稣生在犹太的伯利恒。有几个博士从东方来到耶路撒冷,说:
\par 2 「那生下来作犹太人之王的在那里?我们在东方看见他的星,特来拜他。」
\par 3 希律王听见了,就心里不安;耶路撒冷合城的人也都不安。
\par 4 他就召齐了祭司长和民间的文士,问他们说:「基督当生在何处?」
\par 5 他们回答说:「在犹太的伯利恒。因为有先知记著,说:
\par 6 犹大地的伯利恒啊,你在犹大诸城中并不是最小的;因为将来有一位君王要从你那里出来,牧养我以色列民。」
\par 7 当下,希律暗暗的召了博士来,细问那星是什麽时候出现的,
\par 8 就差他们往伯利恒去,说:「你们去仔细寻访那小孩子,寻到了,就来报信,我也好去拜他。」
\par 9 他们听见王的话就去了。在东方所看见的那星忽然在他们前头行,直行到小孩子的地方,就在上头停住了。
\par 10 他们看见那星,就大大的欢喜;
\par 11 进了房子,看见小孩子和他母亲马利亚,就俯伏拜那小孩子,揭开宝盒,拿黄金、乳香、没药为礼物献给他。
\par 12 博士因为在梦中被主指示不要回去见希律,就从别的路回本地去了。
\par 13 他们去後,有主的使者向约瑟梦中显现,说:「起来!带著小孩子同他母亲逃往埃及,住在那里,等我吩咐你;因为希律必寻找小孩子,要除灭他。」
\par 14 约瑟就起来,夜间带著小孩子和他母亲往埃及去,
\par 15 住在那里,直到希律死了。这是要应验主藉先知所说的话,说:「我从埃及召出我的儿子来。」
\par 16 希律见自己被博士愚弄,就大大发怒,差人将伯利恒城里并四境所有的男孩,照著他向博士仔细查问的时候,凡两岁以里的,都杀尽了。
\par 17 这就应了先知耶利米的话,说:
\par 18 在拉玛听见号 大哭的声音,是拉结哭他儿女,不肯受安慰,因为他们都不在了。
\par 19 希律死了以後,有主的使者在埃及向约瑟梦中显现,说:
\par 20 「起来!带著小孩子和他母亲往以色列地去,因为要害小孩子性命的人已经死了。」
\par 21 约瑟就起来,把小孩子和他母亲带到以色列地去;
\par 22 只因听见亚基老接著他父亲希律作了犹太王,就怕往那里去,又在梦中被主指示,便往加利利境内去了,
\par 23 到了一座城,名叫拿撒勒,就住在那里。这是要应验先知所说,他将称为拿撒勒人的话了。

\chapter{3}

\par 1 那时,有施洗的约翰出来,在犹太的旷野传道,说:
\par 2 「天国近了,你们应当悔改!」
\par 3 这人就是先知以赛亚所说的。他说:「在旷野有人声喊著说:预备主的道,修直他的路!」
\par 4 这约翰身穿骆驼毛的衣服,腰束皮带,吃的是蝗虫、野蜜。
\par 5 那时,耶路撒冷和犹太全地,并约但河一带地方的人,都出去到约翰那里,
\par 6 承认他们的罪,在约但河里受他的洗。
\par 7 约翰看见许多法利赛人和撒都该人也来受洗,就对他们说:「毒蛇的种类!谁指示你们逃避将来的忿怒呢?
\par 8 你们要结出果子来,与悔改的心相称。
\par 9 不要自己心里说:『有亚伯拉罕为我们的祖宗。』我告诉你们,神能从这些石头中给亚伯拉罕兴起子孙来。
\par 10 现在斧子已经放在树根上,凡不结好果子的树就砍下来,丢在火里。
\par 11 我是用水给你们施洗,叫你们悔改。但那在我以後来的,能力比我更大,我就是给他提鞋也不配。他要用圣灵与火给你们施洗。
\par 12 他手里拿著簸箕,要扬净他的场,把麦子收在仓里,把糠用不灭的火烧尽了。」
\par 13 当下耶稣从加利利来到约但河,见了约翰,要受他的洗。
\par 14 约翰想要拦住他,说:「我当受你的洗,你反倒上我这里来吗?」
\par 15 耶稣回答说:「你暂且许我,因为我们理当这样尽诸般的义(或作:礼)。」於是约翰许了他。
\par 16 耶稣受了洗,随即从水里上来。天忽然为他开了,他就看见神的灵彷佛鸽子降下,落在他身上。
\par 17 从天上有声音说:「这是我的爱子,我所喜悦的。」

\chapter{4}

\par 1 当时,耶稣被圣灵引到旷野,受魔鬼的试探。
\par 2 他禁食四十昼夜,後来就饿了。
\par 3 那试探人的进前来,对他说:「你若是神的儿子,可以吩咐这些石头变成食物。」
\par 4 耶稣却回答说:「经上记著说:人活著,不是单靠食物,乃是靠神口里所出的一切话。」
\par 5 魔鬼就带他进了圣城,叫他站在殿顶(顶原文作翅)上,
\par 6 对他说:「你若是神的儿子,可以跳下去,因为经上记著:主要为你吩咐他的使者用手托著你,免得你的脚碰在石头上。」
\par 7 耶稣对他说:「经上又记著说:『不可试探主你的神。』」
\par 8 魔鬼又带他上了一座最高的山,将世上的万国与万国的荣华都指给他看,
\par 9 对他说:「你若俯伏拜我,我就把这一切都赐给你。」
\par 10 耶稣说:「撒但(撒但就是抵挡的意思,乃魔鬼的别名),退去吧!因为经上记著说:当拜主你的神,单要事奉他。」
\par 11 於是,魔鬼离了耶稣,有天使来伺候他。
\par 12 耶稣听见约翰下了监,就退到加利利去;
\par 13 後又离开拿撒勒,往迦百农去,就住在那里。那地方靠海,在西布伦和拿弗他利的边界上。
\par 14 这是要应验先知以赛亚的话,
\par 15 说:西布伦地,拿弗他利地,就是沿海的路,约但河外,外邦人的加利利地,
\par 16 那坐在黑暗里的百姓看见了大光;坐在死荫之地的人有光发现照著他们。
\par 17 从那时候,耶稣就传起道来,说:「天国近了,你们应当悔改!」
\par 18 耶稣在加利利海边行走,看见弟兄二人,就是那称呼彼得的西门和他兄弟安得烈,在海里撒网;他们本是打鱼的。
\par 19 耶稣对他们说:「来跟从我,我要叫你们得人如得鱼一样。」
\par 20 他们就立刻舍了网,跟从了他。
\par 21 从那里往前走,又看见弟兄二人,就是西庇太的儿子雅各和他兄弟约翰,同他们的父亲西庇太在船上补网,耶稣就招呼他们,
\par 22 他们立刻舍了船,别了父亲,跟从了耶稣。
\par 23 耶稣走遍加利利,在各会堂里教训人,传天国的福音,医治百姓各样的病症。
\par 24 他的名声就传遍了利亚。那里的人把一切害病的,就是害各样疾病、各样疼痛的和被鬼附的、癫痫的、瘫痪的,都带了来,耶稣就治好了他们。
\par 25 当下,有许多人从加利利、低加波利、耶路撒冷、犹太、约但河外来跟著他。

\chapter{5}

\par 1 耶稣看见这许多的人,就上了山,既已坐下,门徒到他跟前来,
\par 2 他就开口教训他们,说:
\par 3 虚心的人有福了!因为天国是他们的。
\par 4 哀恸的人有福了!因为他们必得安慰。
\par 5 温柔的人有福了!因为他们必承受地土。
\par 6 饥渴慕义的人有福了!因为他们必得饱足。
\par 7 怜恤人的人有福了!因为他们必蒙怜恤。
\par 8 清心的人有福了!因为他们必得见神。
\par 9 使人和睦的人有福了!因为他们必称为神的儿子。
\par 10 为义受逼迫的人有福了!因为天国是他们的。
\par 11 「人若因我辱骂你们,逼迫你们,捏造各样坏话毁谤你们,你们就有福了!
\par 12 应当欢喜快乐,因为你们在天上的赏赐是大的。在你们以前的先知,人也是这样逼迫他们。」
\par 13 「你们是世上的盐。盐若失了味,怎能叫他再咸呢?以後无用,不过丢在外面,被人践踏了。
\par 14 你们是世上的光。城造在山上是不能隐藏的。
\par 15 人点灯,不放在斗底下,是放在灯台上,就照亮一家的人。
\par 16 你们的光也当这样照在人前,叫他们看见你们的好行为,便将荣耀归给你们在天上的父。」
\par 17 「莫想我来要废掉律法和先知。我来不是要废掉,乃是要成全。
\par 18 我实在告诉你们,就是到天地都废去了,律法的一点一画也不能废去,都要成全。
\par 19 所以,无论何人废掉这诫命中最小的一条,又教训人这样做,他在天国要称为最小的。但无论何人遵行这诫命,又教训人遵行,他在天国要称为大的。
\par 20 我告诉你们,你们的义若不胜於文士和法利赛人的义,断不能进天国。」
\par 21 「你们听见有吩咐古人的话,说:『不可杀人』;又说:『凡杀人的难免受审判。』
\par 22 只是我告诉你们:凡(有古卷在凡字下加:无缘无故地)向弟兄动怒的,难免受审判;凡骂弟兄是拉加的,难免公会的审断;凡骂弟兄是魔利的,难免地狱的火。
\par 23 所以,你在祭坛上献礼物的时候,若想起弟兄向你怀怨,
\par 24 就把礼物留在坛前,先去同弟兄和好,然後来献礼物。
\par 25 你同告你的对头还在路上,就赶紧与他和息,恐怕他把你送给审判官,审判官交付衙役,你就下在监里了。
\par 26 我实在告诉你,若有一文钱没有还清,你断不能从那里出来。」
\par 27 「你们听见有话说:『不可奸淫。』
\par 28 只是我告诉你们,凡看见妇女就动淫念的,这人心里已经与他犯奸淫了。
\par 29 若是你的右眼叫你跌倒,就剜出来丢掉,宁可失去百体中的一体,不叫全身丢在地狱里。
\par 30 若是右手叫你跌倒,就砍下来丢掉,宁可失去百体中的一体,不叫全身下入地狱。」
\par 31 「又有话说:『人若休妻,就当给他休书。』
\par 32 只是我告诉你们,凡休妻的,若不是为淫乱的缘故,就是叫他作淫妇了;人若娶这被休的妇人,也是犯奸淫了。」
\par 33 「你们又听见有吩咐古人的话,说:『不可背誓,所起的誓总要向主谨守。』
\par 34 只是我告诉你们,什麽誓都不可起。不可指著天起誓,因为天是神的座位;
\par 35 不可指著地起誓,因为地是他的脚凳;也不可指著耶路撒冷起誓,因为耶路撒冷是大君的京城;
\par 36 又不可指著你的头起誓,因为你不能使一根头发变黑变白了。
\par 37 你们的话,是,就说是;不是,就说不是;若再多说,就是出於那恶者(或作就是从恶里出来的)。」
\par 38 「你们听见有话说:『以眼还眼,以牙还牙。』
\par 39 只是我告诉你们,不要与恶人作对。有人打你的右脸,连左脸也转过来由他打;
\par 40 有人想要告你,要拿你的里衣,连外衣也由他拿去;
\par 41 有人强逼你走一里路,你就同他走二里;
\par 42 有求你的,就给他;有向你借贷的,不可推辞。」
\par 43 「你们听见有话说:『当爱你的邻舍,恨你的仇敌。』
\par 44 只是我告诉你们,要爱你们的仇敌,为那逼迫你们的祷告。
\par 45 这样就可以作你们天父的儿子;因为他叫日头照好人,也照歹人;降雨给义人,也给不义的人。
\par 46 你们若单爱那爱你们的人,有什麽赏赐呢?就是税吏不也是这样行吗?
\par 47 你们若单请你弟兄的安,比人有什麽长处呢?就是外邦人不也是这样行吗?
\par 48 所以,你们要完全,像你们的天父完全一样。」

\chapter{6}

\par 1 「你们要小心,不可将善事行在人的面前,故意叫他们看见,若是这样,就不能得你们天父的赏赐了。
\par 2 所以,你施舍的时候,不可在你前面吹号,像那假冒为善的人在会堂里和街道上所行的,故意要得人的荣耀。我实在告诉你们,他们已经得了他们的赏赐。
\par 3 你施舍的时候,不要叫左手知道右手所做的,
\par 4 要叫你施舍的事行在暗中。你父在暗中察看,必然报答你(有古卷:必在明处报答你)。」
\par 5 「你们祷告的时候,不可像那假冒为善的人,爱站在会堂里和十字路口上祷告,故意叫人看见。我实在告诉你们,他们已经得了他们的赏赐。
\par 6 你祷告的时候,要进你的内屋,关上门,祷告你在暗中的父;你父在暗中察看,必然报答你。
\par 7 你们祷告,不可像外邦人,用许多重复话,他们以为话多了必蒙垂听。
\par 8 你们不可效法他们;因为你们没有祈求以先,你们所需用的,你们的父早已知道了。
\par 9 所以,你们祷告要这样说:我们在天上的父:愿人都尊你的名为圣。
\par 10 愿你的国降临;愿你的旨意行在地上,如同行在天上。
\par 11 我们日用的饮食,今日赐给我们。
\par 12 免我们的债,如同我们免了人的债。
\par 13 不叫我们遇见试探;救我们脱离凶恶(或作:脱离恶者)。因为国度、权柄、荣耀,全是你的,直到永远。阿们(有古卷没有因为至阿们等字)!』
\par 14 你们饶恕人的过犯,你们的天父也必饶恕你们的过犯;
\par 15 你们不饶恕人的过犯,你们的天父也必不饶恕你们的过犯。」
\par 16 「你们禁食的时候,不可像那假冒为善的人,脸上带著愁容;因为他们把脸弄得难看,故意叫人看出他们是禁食。我实在告诉你们,他们已经得了他们的赏赐。
\par 17 你禁食的时候,要梳头洗脸,
\par 18 不叫人看出你禁食来,只叫你暗中的父看见;你父在暗中察看,必然报答你。」
\par 19 「不要为自己积 财宝在地上;地上有虫子咬,能锈坏,也有贼挖窟窿来偷。
\par 20 只要积 财宝在天上;天上没有虫子咬,不能锈坏,也没有贼挖窟窿来偷。
\par 21 因为你的财宝在那里,你的心也在那里。」
\par 22 「眼睛就是身上的灯。你的眼睛若了亮,全身就光明;
\par 23 你的眼睛若昏花,全身就黑暗。你里头的光若黑暗了,那黑暗是何等大呢!」
\par 24 「一个人不能事奉两个主;不是恶这个,爱那个,就是重这个,轻那个。你们不能又事奉神,又事奉玛门(玛门:财利的意思)。」
\par 25 「所以我告诉你们,不要为生命忧虑吃什麽,喝什麽;为身体忧虑穿什麽。生命不胜於饮食吗?身体不胜於衣裳吗?
\par 26 你们看那天上的飞鸟,也不种,也不收,也不积蓄在仓里,你们的天父尚且养活他。你们不比飞鸟贵重得多吗?
\par 27 你们那一个能用思虑使寿数多加一刻呢(或作:使身量多加一肘呢)?
\par 28 何必为衣裳忧虑呢?你想野地里的百合花怎麽长起来;他也不劳苦,也不纺线。
\par 29 然而我告诉你们,就是所罗门极荣华的时候,他所穿戴的,还不如这花一朵呢!
\par 30 你们这小信的人哪!野地里的草今天还在,明天就丢在炉里,神还给他这样的妆饰,何况你们呢!
\par 31 所以,不要忧虑说:吃什麽?喝什麽?穿什麽?
\par 32 这都是外邦人所求的,你们需用的这一切东西,你们的天父是知道的。
\par 33 你们要先求他的国和他的义,这些东西都要加给你们了。
\par 34 所以,不要为明天忧虑,因为明天自有明天的忧虑;一天的难处一天当就够了。」

\chapter{7}

\par 1 「你们不要论断人,免得你们被论断。
\par 2 因为你们怎样论断人,也必怎样被论断;你们用什麽量器量给人,也必用什麽量器量给你们。
\par 3 为什麽看见你弟兄眼中有刺,却不想自己眼中有梁木呢?
\par 4 你自己眼中有梁木,怎能对你弟兄说:『容我去掉你眼中的刺』呢?
\par 5 你这假冒为善的人!先去掉自己眼中的梁木,然後才能看得清楚,去掉你弟兄眼中的刺。
\par 6 不要把圣物给狗,也不要把你们的珍珠丢在猪前,恐怕他践踏了珍珠,转过来咬你们。」
\par 7 「你们祈求,就给你们;寻找,就寻见;叩门,就给你们开门。
\par 8 因为凡祈求的,就得著;寻找的,就寻见;叩门的,就给他开门。
\par 9 你们中间谁有儿子求饼,反给他石头呢?
\par 10 求鱼,反给他蛇呢?
\par 11 你们虽然不好,尚且知道拿好东西给儿女,何况你们在天上的父,岂不更把好东西给求他的人吗?
\par 12 所以,无论何事,你们愿意人怎样待你们,你们也要怎样待人,因为这就是律法和先知的道理。」
\par 13 「你们要进窄门。因为引到灭亡,那门是宽的,路是大的,进去的人也多;
\par 14 引到永生,那门是窄的,路是小的,找著的人也少。」
\par 15 「你们要防备假先知。他们到你们这里来,外面披著羊皮,里面却是残暴的狼。
\par 16 凭著他们的果子,就可以认出他们来。荆棘上岂能摘葡萄呢?蒺藜里岂能摘无花果呢?
\par 17 这样,凡好树都结好果子,惟独坏树结坏果子。
\par 18 好树不能结坏果子;坏树不能结好果子。
\par 19 凡不结好果子的树就砍下来,丢在火里。
\par 20 所以,凭著他们的果子就可以认出他们来。」
\par 21 「凡称呼我『主啊,主啊』的人不能都进天国;惟独遵行我天父旨意的人才能进去。
\par 22 当那日必有许多人对我说:『主啊,主啊,我们不是奉你的名传道,奉你的名赶鬼,奉你的名行许多异能吗?』
\par 23 我就明明的告诉他们说:『我从来不认识你们,你们这些作恶的人,离开我去吧!』」
\par 24 「所以,凡听见我这话就去行的,好比一个聪明人,把房子盖在磐石上;
\par 25 雨淋,水冲,风吹,撞著那房子,房子总不倒塌,因为根基立在磐石上。
\par 26 凡听见我这话不去行的,好比一个无知的人,把房子盖在沙土上;
\par 27 雨淋,水冲,风吹,撞著那房子,房子就倒塌了,并且倒塌得很大。」
\par 28 耶稣讲完了这些话,众人都希奇他的教训;
\par 29 因为他教训他们,正像有权柄的人,不像他们的文士。

\chapter{8}

\par 1 耶稣下了山,有许多人跟著他。
\par 2 有一个长大麻疯的来拜他,说:「主若肯,必能叫我洁净了。」
\par 3 耶稣伸手摸他,说:「我肯,你洁净了吧!」他的大麻疯立刻就洁净了。
\par 4 耶稣对他说:「你切不可告诉人,只要去把身体给祭司察看,献上摩西所吩咐的礼物,对众人作证据。」
\par 5 耶稣进了迦百农,有一个百夫长进前来,求他说:
\par 6 「主啊,我的仆人害瘫痪病,躺在家里,甚是疼苦。」
\par 7 耶稣说:「我去医治他。」
\par 8 百夫长回答说:「主啊,你到我舍下,我不敢当;只要你说一句话,我的仆人就必好了。
\par 9 因为我在人的权下,也有兵在我以下;对这个说:『去!』他就去;对那个说:『来!』他就来;对我的仆人说:『你做这事!』他就去做。」
\par 10 耶稣听见就希奇,对跟从的人说:「我实在告诉你们,这麽大的信心,就是在以色列中,我也没有遇见过。
\par 11 我又告诉你们,从东从西,将有许多人来,在天国里与亚伯拉罕、以撒、雅各一同坐席;
\par 12 惟有本国的子民竟被赶到外边黑暗里去,在那里必要哀哭切齿了。」
\par 13 耶稣对百夫长说:「你回去吧!照你的信心,给你成全了。」那时,他的仆人就好了。
\par 14 耶稣到了彼得家里,见彼得的岳母害热病躺著。
\par 15 耶稣把他的手一摸,热就退了;他就起来服事耶稣。
\par 16 到了晚上,有人带著许多被鬼附的来到耶稣跟前,他只用一句话就把鬼都赶出去,并且治好了一切有病的人。
\par 17 这是要应验先知以赛亚的话,说:他代替我们的软弱,担当我们的疾病。
\par 18 耶稣见许多人围著他,就吩咐渡到那边去。
\par 19 有一个文士来,对他说:「夫子,你无论往那里去,我要跟从你。」
\par 20 耶稣说:「狐狸有洞,天空的飞鸟有窝,人子却没有枕头的地方。」
\par 21 又有一个门徒对耶稣说:「主啊,容我先回去埋葬我的父亲。」
\par 22 耶稣说:「任凭死人埋葬他们的死人;你跟从我吧!」
\par 23 耶稣上了船,门徒跟著他。
\par 24 海里忽然起了暴风,甚至船被波浪掩盖;耶稣却睡著了。
\par 25 门徒来叫醒了他,说:「主啊,救我们,我们丧命啦!」
\par 26 耶稣说:「你们这小信的人哪,为什麽胆怯呢?」於是起来,斥责风和海,风和海就大大的平静了。
\par 27 众人希奇,说:「这是怎样的人?连风和海也听从他了!」
\par 28 耶稣既渡到那边去,来到加大拉人的地方,就有两个被鬼附的人从坟茔里出来迎著他,极其凶猛,甚至没有人能从那条路上经过。
\par 29 他们喊著说:「神的儿子,我们与你有什麽相干?时候还没有到,你就上这里来叫我们受苦吗?」
\par 30 离他们很远,有一大群猪吃食。
\par 31 鬼就央求耶稣,说:「若把我们赶出去,就打发我们进入猪群吧!」
\par 32 耶稣说:「去吧!」鬼就出来,进入猪群。全群忽然闯下山崖,投在海里淹死了。
\par 33 放猪的就逃跑进城,将这一切事和被鬼附的人所遭遇的都告诉人。
\par 34 合城的人都出来迎见耶稣,既见了就央求他离开他们的境界。

\chapter{9}

\par 1 耶稣上了船,渡过海,来到自己的城里。
\par 2 有人用褥子抬著一个瘫子到耶稣跟前来。耶稣见他们的信心,就对瘫子说:「小子,放心吧!你的罪赦了。」
\par 3 有几个文士心里说:「这个人说僭妄的话了。」
\par 4 耶稣知道他们的心意,就说:「你们为什麽心里怀著恶念呢?
\par 5 或说:『你的罪赦了』,或说:『你起来行走』,那一样容易呢?
\par 6 但要叫你们知道,人子在地上有赦罪的权柄」;就对瘫子说:「起来!拿你的褥子回家去吧。」
\par 7 那人就起来,回家去了。
\par 8 众人看见都惊奇,就归荣耀与神,因为他将这样的权柄赐给人。
\par 9 耶稣从那里往前走,看见一个人名叫马太,坐在税关上,就对他说:「你跟从我来。」他就起来跟从了耶稣。
\par 10 耶稣在屋里坐席的时候,有好些税吏和罪人来,与耶稣和他的门徒一同坐席。
\par 11 法利赛人看见,就对耶稣的门徒说:「你们的先生为什麽和税吏并罪人一同吃饭呢?」
\par 12 耶稣听见,就说:「康健的人用不著医生,有病的人才用得著。
\par 13 经上说:『我喜爱怜恤,不喜爱祭祀。』这句话的意思,你们且去揣摩。我来本不是召义人,乃是召罪人。」
\par 14 那时,约翰的门徒来见耶稣,说:「我们和法利赛人常常禁食,你的门徒倒不禁食,这是为什麽呢?」
\par 15 耶稣对他们说:「新郎和陪伴之人同在的时候,陪伴之人岂能哀恸呢?但日子将到,新郎要离开他们,那时候他们就要禁食。
\par 16 没有人把新布补在旧衣服上;因为所补上的反带坏了那衣服,破的就更大了。
\par 17 也没有人把新酒装在旧皮袋里;若是这样,皮袋就裂开,酒漏出来,连皮袋也坏了。惟独把新酒装在新皮袋里,两样就都保全了。」
\par 18 耶稣说这话的时候,有一个管会堂的来拜他,说:「我女儿刚才死了,求你去按手在他身上,他就必活了。」
\par 19 耶稣便起来跟著他去;门徒也跟了去。
\par 20 有一个女人,患了十二年的血漏,来到耶稣背後,摸他的衣裳 子;
\par 21 因为他心里说:「我只摸他的衣裳,就必痊愈。」
\par 22 耶稣转过来,看见他,就说:「女儿,放心!你的信救了你。」从那时候,女人就痊愈了。
\par 23 耶稣到了管会堂的家里,看见有吹手,又有许多人乱嚷,
\par 24 就说:「退去吧!这闺女不是死了,是睡著了。」他们就嗤笑他。
\par 25 众人既被撵出,耶稣就进去,拉著闺女的手,闺女便起来了。
\par 26 於是这风声传遍了那地方。
\par 27 耶稣从那里往前走,有两个瞎子跟著他,喊叫说:「大卫的子孙,可怜我们吧!」
\par 28 耶稣进了房子,瞎子就来到他跟前。耶稣说:「你们信我能做这事吗?」他们说:「主啊,我们信。」
\par 29 耶稣就摸他们的眼睛,说:「照著你们的信给你们成全了吧。」
\par 30 他们的眼睛就开了。耶稣切切的嘱咐他们说:「你们要小心,不可叫人知道。」
\par 31 他们出去,竟把他的名声传遍了那地方。
\par 32 他们出去的时候,有人将鬼所附的一个哑巴带到耶稣跟前来。
\par 33 鬼被赶出去,哑巴就说出话来。众人都希奇,说:「在以色列中,从来没有见过这样的事。」
\par 34 法利赛人却说:「他是靠著鬼王赶鬼。」
\par 35 耶稣走遍各城各乡,在会堂里教训人,宣讲天国的福音,又医治各样的病症。
\par 36 他看见许多的人,就怜悯他们;因为他们困苦流离,如同羊没有牧人一般。
\par 37 於是对门徒说:「要收的庄稼多,做工的人少。
\par 38 所以,你们当求庄稼的主打发工人出去收他的庄稼。」

\chapter{10}

\par 1 耶稣叫了十二个门徒来,给他们权柄,能赶逐污鬼,并医治各样的病症。
\par 2 这十二使徒的名:头一个叫西门(又称彼得),还有他兄弟安得烈,西庇太的儿子雅各和雅各的兄弟约翰,
\par 3 腓力和巴多罗买,多马和税吏马太,亚勒腓的儿子雅各,和达太,
\par 4 奋锐党的西门,还有卖耶稣的加略人犹大。
\par 5 耶稣差这十二个人去,吩咐他们说:「外邦人的路,你们不要走;撒玛利亚人的城,你们不要进;
\par 6 宁可往以色列家迷失的羊那里去。
\par 7 随走随传,说『天国近了!』
\par 8 医治病人,叫死人复活,叫长大麻疯的洁净,把鬼赶出去。你们白白的得来,也要白白的舍去。
\par 9 腰袋里不要带金银铜钱。
\par 10 行路不要带口袋;不要带两件褂子,也不要带鞋和 杖;因为工人得饮食是应当的。
\par 11 你们无论进哪一城,哪一村,要打听那里谁是好人,就住在他家,直住到走的时候。
\par 12 进他家里去,要请他的安。
\par 13 那家若配得平安,你们所求的平安就必临到那家;若不配得,你们所求的平安仍归你们。
\par 14 凡不接待你们、不听你们话的人,你们离开那家,或是那城的时候,就把脚上的尘土跺下去。
\par 15 我实在告诉你们,当审判的日子,所多玛和蛾摩拉所受的,比那城还容易受呢!」
\par 16 「我差你们去,如同羊进入狼群;所以你们要灵巧像蛇,驯良像鸽子。
\par 17 你们要防备人;因为他们要把你们交给公会,也要在会堂里鞭打你们,
\par 18 并且你们要为我的缘故被送到诸侯君王面前,对他们和外邦人作见证。
\par 19 你们被交的时候,不要思虑怎样说话,或说什麽话。到那时候,必赐给你们当说的话;
\par 20 因为不是你们自己说的,乃是你们父的灵在你们里头说的。
\par 21 弟兄要把弟兄,父亲要把儿子,送到死地;儿女要与父母为敌,害死他们;
\par 22 并且你们要为我的名被众人恨恶。惟有忍耐到底的必然得救。
\par 23 有人在这城里逼迫你们,就逃到那城里去。我实在告诉你们,以色列的城邑,你们还没有走遍,人子就到了。
\par 24 学生不能高过先生;仆人不能高过主人。
\par 25 学生和先生一样,仆人和主人一样,也就罢了。人既骂家主是别西卜(别西卜:是鬼王的名),何况他的家人呢?」
\par 26 「所以,不要怕他们;因为掩盖的事没有不露出来的,隐藏的事没有不被人知道的。
\par 27 我在暗中告诉你们的,你们要在明处说出来;你们耳中所听的,要在房上宣扬出来。
\par 28 那杀身体,不能杀灵魂的,不要怕他们;惟有能把身体和灵魂都灭在地狱里的,正要怕他。
\par 29 两个麻雀不是卖一分银子吗?若是你们的父不许,一个也不能掉在地上;
\par 30 就是你们的头发也都被数过了。
\par 31 所以,不要惧怕,你们比许多麻雀还贵重!」
\par 32 「凡在人面前认我的,我在我天上的父面前也必认他;
\par 33 凡在人面前不认我的,我在我天上的父面前也必不认他。」
\par 34 「你们不要想我来是叫地上太平;我来并不是叫地上太平,乃是叫地上动刀兵。
\par 35 因为我来是叫人与父亲生疏,女儿与母亲生疏,媳妇与婆婆生疏。
\par 36 人的仇敌就是自己家里的人。
\par 37 「爱父母过於爱我的,不配作我的门徒;爱儿女过於爱我的,不配作我的门徒;
\par 38 不背著他的十字架跟从我的,也不配作我的门徒。
\par 39 得著生命的,将要失丧生命;为我失丧生命的,将要得著生命。」
\par 40 「人接待你们就是接待我;接待我就是接待那差我来的。
\par 41 人因为先知的名接待先知,必得先知所得的赏赐;人因为义人的名接待义人,必得义人所得的赏赐。
\par 42 无论何人,因为门徒的名,只把一杯凉水给这小子里的一个喝,我实在告诉你们,这人不能不得赏赐。」

\chapter{11}

\par 1 耶稣吩咐完了十二个门徒,就离开那里,往各城去传道,教训人。
\par 2 约翰在监里听见基督所做的事,就打发两个门徒去,
\par 3 问他说:「那将要来的是你吗?还是我们等候别人呢?」
\par 4 耶稣回答说:「你们去,把所听见,所看见的事告诉约翰。
\par 5 就是瞎子看见,瘸子行走,长大麻疯的洁净,聋子听见,死人复活,穷人有福音传给他们。
\par 6 凡不因我跌倒的就有福了!」
\par 7 他们走的时候,耶稣就对众人讲论约翰说:「你们从前出到旷野是要看什麽呢?要看风吹动的芦苇吗?
\par 8 你们出去到底是要看什麽?要看穿细软衣服的人吗?那穿细软衣服的人是在王宫里。
\par 9 你们出去究竟是为什麽?是要看先知吗?我告诉你们,是的,他比先知大多了。
\par 10 经上记著说:『我要差遣我的使者在你前面预备道路』,所说的就是这个人。
\par 11 我实在告诉你们,凡妇人所生的,没有一个兴起来大过施洗约翰的;然而天国里最小的比他还大。
\par 12 从施洗约翰的时候到如今,天国是努力进入的,努力的人就得著了。
\par 13 因为众先知和律法说预言,到约翰为止。
\par 14 你们若肯领受,这人就是那应当来的以利亚。
\par 15 有耳可听的,就应当听!
\par 16 我可用什麽比这世代呢?好像孩童坐在街市上招呼同伴,说:
\par 17 我们向你们吹笛,你们不跳舞;我们向你们举哀,你们不捶胸。
\par 18 约翰来了,也不吃也不喝,人就说他是被鬼附著的;
\par 19 人子来了,也吃也喝,人又说他是贪食好酒的人,是税吏和罪人的朋友。但智慧之子总以智慧为是(有古卷:但智慧在行为上就显为是)。」
\par 20 耶稣在诸城中行了许多异能,那些城的人终不悔改,就在那时候责备他们,说:
\par 21 「哥拉汛哪,你有祸了!伯赛大啊,你有祸了!因为在你们中间所行的异能,若行在推罗、西顿,他们早已披麻蒙灰悔改了。
\par 22 但我告诉你们,当审判的日子,推罗、西顿所受的,比你们还容易受呢!
\par 23 迦百农啊,你已经升到天上(或作:你将要升到天上吗),将来必坠落阴间;因为在你那里所行的异能,若行在所多玛,他还可以存到今日。
\par 24 但我告诉你们,当审判的日子,所多玛所受的,比你还容易受呢!」
\par 25 那时,耶稣说:「父啊,天地的主,我感谢你!因为你将这些事向聪明通达人就藏起来,向婴孩就显出来。
\par 26 父啊,是的,因为你的美意本是如此。
\par 27 一切所有的,都是我父交付我的;除了父,没有人知道子;除了子和子所愿意指示的,没有人知道父。
\par 28 凡劳苦担重担的人可以到我这里来,我就使你们得安息。
\par 29 我心里柔和谦卑,你们当负我的轭,学我的样式;这样,你们心里就必得享安息。
\par 30 因为我的轭是容易的,我的担子是轻省的。」

\chapter{12}

\par 1 那时,耶稣在安息日从麦地经过。他的门徒饿了,就掐起麦穗来吃。
\par 2 法利赛人看见,就对耶稣说:「看哪,你的门徒做安息日不可做的事了!」
\par 3 耶稣对他们说:「经上记著大卫和跟从他的人饥饿之时所做的事,你们没有念过吗?
\par 4 他怎麽进了神的殿,吃了陈设饼,这饼不是他和跟从他的人可以吃得,惟独祭司才可以吃。
\par 5 再者,律法上所记的,当安息日,祭司在殿里犯了安息日还是没有罪,你们没有念过吗?
\par 6 但我告诉你们,在这里有一人比殿更大。
\par 7 『我喜爱怜恤,不喜爱祭祀。』你们若明白这话的意思,就不将无罪的当作有罪的了。
\par 8 因为人子是安息日的主。」
\par 9 耶稣离开那地方,进了一个会堂。
\par 10 那里有一个人枯乾了一只手。有人问耶稣说:「安息日治病可以不可以?」意思是要控告他。
\par 11 耶稣说:「你们中间谁有一只羊,当安息日掉在坑里,不把他抓住,拉上来呢?
\par 12 人比羊何等贵重呢!所以,在安息日做善事是可以的。」
\par 13 於是对那人说:「伸出手来!」他把手一伸,手就复了原,和那只手一样。
\par 14 法利赛人出去,商议怎样可以除灭耶稣。
\par 15 耶稣知道了,就离开那里,有许多人跟著他。他把其中有病的人都治好了;
\par 16 又嘱咐他们,不要给他传名。
\par 17 这是要应验先知以赛亚的话,说:
\par 18 看哪!我的仆人,我所拣选,所亲爱,心里所喜悦的,我要将我的灵赐给他;他必将公理传给外邦。
\par 19 他不争竞,不喧嚷;街上也没有人听见他的声音。
\par 20 压伤的芦苇,他不折断;将残的灯火,他不吹灭;等他施行公理,叫公理得胜。
\par 21 外邦人都要仰望他的名。
\par 22 当下,有人将一个被鬼附著、又瞎又哑的人带到耶稣那里,耶稣就医治他,甚至那哑巴又能说话,又能看见。
\par 23 众人都惊奇,说:「这不是大卫的子孙吗?」
\par 24 但法利赛人听见,就说:「这个人赶鬼,无非是靠著鬼王别西卜啊。」
\par 25 耶稣知道他们的意念,就对他们说:「凡一国自相分争,就成为荒场;一城一家自相分争,必站立不住;
\par 26 若撒但赶逐撒但,就是自相分争,他的国怎能站得住呢?
\par 27 我若靠著别西卜赶鬼,你们的子弟赶鬼又靠著谁呢?这样,他们就要断定你们的是非。
\par 28 我若靠著神的灵赶鬼,这就是神的国临到你们了。
\par 29 人怎能进壮士家里,抢夺他的家具呢?除非先捆住那壮士,才可以抢夺他的家财。
\par 30 不与我相合的,就是敌我的;不同我收聚的,就是分散的。」
\par 31 所以我告诉你们:「人一切的罪和亵渎的话都可得赦免,惟独亵渎圣灵,总不得赦免。
\par 32 凡说话干犯人子的,还可得赦免;惟独说话干犯圣灵的,今世来世总不得赦免。」
\par 33 「你们或以为树好,果子也好;树坏,果子也坏;因为看果子就可以知道树。
\par 34 毒蛇的种类!你们既是恶人,怎能说出好话来呢?因为心里所充满的,口里就说出来。
\par 35 善人从他心里所存的善就发出善来;恶人从他心里所存的恶就发出恶来。
\par 36 我又告诉你们,凡人所说的闲话,当审判的日子,必要句句供出来;
\par 37 因为要凭你的话定你为义,也要凭你的话定你有罪。」
\par 38 当时,有几个文士和法利赛人对耶稣说:「夫子,我们愿意你显个神迹给我们看。」
\par 39 耶稣回答说:「一个邪恶淫乱的世代求看神迹,除了先知约拿的神迹以外,再没有神迹给他们看。
\par 40 约拿三日三夜在大鱼肚腹中,人子也要这样三日三夜在地里头。
\par 41 当审判的时候,尼尼微人要起来定这世代的罪,因为尼尼微人听了约拿所传的就悔改了。看哪,在这里有一人比约拿更大!
\par 42 当审判的时候,南方的女王要起来定这世代的罪,因为他从地极而来,要听所罗门的智慧话。看哪!在这里有一人比所罗门更大。」
\par 43 「污鬼离了人身,就在无水之地过来过去,寻求安歇之处,却寻不著。
\par 44 於是说:『我要回到我所出来的屋里去。』到了,就看见里面空闲,打扫乾净,修饰好了,
\par 45 便去另带了七个比自己更恶的鬼来,都进去住在那里。那人末後的景况比先前更不好了。这邪恶的世代也要如此。」
\par 46 耶稣还对众人说话的时候,不料他母亲和他弟兄站在外边,要与他说话。
\par 47 有人告诉他说:「看哪,你母亲和你弟兄站在外边,要与你说话。」
\par 48 他却回答那人说:「谁是我的母亲?谁是我的弟兄?」
\par 49 就伸手指著门徒,说:「看哪,我的母亲,我的弟兄。
\par 50 凡遵行我天父旨意的人,就是我的弟兄姐妹和母亲了。」

\chapter{13}

\par 1 当那一天,耶稣从房子里出来,坐在海边。
\par 2 有许多人到他那里聚集,他只得上船坐下,众人都站在岸上。
\par 3 他用比喻对他们讲许多道理,说:「有一个撒种的出去撒种;
\par 4 撒的时候,有落在路旁的,飞鸟来吃尽了;
\par 5 有落在土浅石头地上的,土既不深,发苗最快,
\par 6 日头出来一晒,因为没有根,就枯乾了;
\par 7 有落在荆棘里的,荆棘长起来,把他挤住了;
\par 8 又有落在好土里的,就结实,有一百倍的,有六十倍的,有三十倍的。
\par 9 有耳可听的,就应当听!」
\par 10 门徒进前来,问耶稣说:「对众人讲话,为什麽用比喻呢?」
\par 11 耶稣回答说:「因为天国的奥秘只叫你们知道,不叫他们知道。
\par 12 凡有的,还要加给他,叫他有余;凡没有的,连他所有的,也要夺去。
\par 13 所以我用比喻对他们讲,是因他们看也看不见,听也听不见,也不明白。
\par 14 在他们身上,正应了以赛亚的预言,说:你们听是要听见,却不明白;看是要看见,却不晓得;
\par 15 因为这百姓油蒙了心,耳朵发沉,眼睛闭著,恐怕眼睛看见,耳朵听见,心里明白,回转过来,我就医治他们。
\par 16 「但你们的眼睛是有福的,因为看见了;你们的耳朵也是有福的,因为听见了。
\par 17 我实在告诉你们,从前有许多先知和义人要看你们所看的,却没有看见,要听你们所听的,却没有听见。」
\par 18 「所以,你们当听这撒种的比喻。
\par 19 凡听见天国道理不明白的,那恶者就来,把所撒在他心里的夺了去;这就是撒在路旁的了。
\par 20 撒在石头地上的,就是人听了道,当下欢喜领受,
\par 21 只因心里没有根,不过是暂时的,及至为道遭了患难,或是受了逼迫,立刻就跌倒了。
\par 22 撒在荆棘里的,就是人听了道,後来有世上的思虑、钱财的迷惑把道挤住了,不能结实。
\par 23 撒在好地上的,就是人听道明白了,後来结实,有一百倍的,有六十倍的,有三十倍的。」
\par 24 耶稣又设个比喻对他们说:「天国好像人撒好种在田里,
\par 25 及至人睡觉的时候,有仇敌来,将稗子撒在麦子里就走了。
\par 26 到长苗吐穗的时候,稗子也显出来。
\par 27 田主的仆人来告诉他说:『主啊,你不是撒好种在田里吗?从那里来的稗子呢?』
\par 28 主人说:『这是仇敌做的。』仆人说:『你要我们去薅出来吗?』
\par 29 主人说:『不必,恐怕薅稗子,连麦子也拔出来。
\par 30 容这两样一齐长,等著收割。当收割的时候,我要对收割的人说,先将稗子薅出来,捆成捆,留著烧;惟有麦子要收在仓里。』」
\par 31 他又设个比喻对他们说:「天国好像一粒芥菜种,有人拿去种在田里。
\par 32 这原是百种里最小的,等到长起来,却比各样的菜都大,且成了树,天上的飞鸟来宿在他的枝上。」
\par 33 他又对他们讲个比喻说:「天国好像面酵,有妇人拿来,藏在三斗面里,直等全团都发起来。」
\par 34 这都是耶稣用比喻对众人说的话;若不用比喻,就不对他们说什麽。
\par 35 这是要应验先知的话,说:我要开口用比喻,把创世以来所隐藏的事发明出来。
\par 36 当下,耶稣离开众人,进了房子。他的门徒进前来,说:「请把田间稗子的比喻讲给我们听。」
\par 37 他回答说:「那撒好种的就是人子;
\par 38 田地就是世界;好种就是天国之子;稗子就是那恶者之子;
\par 39 撒稗子的仇敌就是魔鬼;收割的时候就是世界的末了;收割的人就是天使。
\par 40 将稗子薅出来用火焚烧,世界的末了也要如此。
\par 41 人子要差遣使者,把一切叫人跌倒的和作恶的,从他国里挑出来,
\par 42 丢在火炉里;在那里必要哀哭切齿了。
\par 43 那时,义人在他们父的国里,要发出光来,像太阳一样。有耳可听的,就应当听!」
\par 44 「天国好像宝贝藏在地里,人遇见了就把他藏起来,欢欢喜喜的去变卖一切所有的,买这块地。
\par 45 天国又好像买卖人寻找好珠子,
\par 46 遇见一颗重价的珠子,就去变卖他一切所有的,买了这颗珠子。
\par 47 天国又好像网撒在海里,聚拢各样水族,
\par 48 网既满了,人就拉上岸来,坐下,拣好的收在器具里,将不好的丢弃了。
\par 49 世界的末了也要这样。天使要出来,从义人中把恶人分别出来,
\par 50 丢在火炉里;在那里必要哀哭切齿了。」
\par 51 耶稣说:「这一切的话你们都明白了吗?」他们说:「我们明白了。」
\par 52 他说:「凡文士受教作天国的门徒,就像一个家主从他库里拿出新旧的东西来。」
\par 53 耶稣说完了这些比喻,就离开那里,
\par 54 来到自己的家乡,在会堂里教训人,甚至他们都希奇,说:「这人从那里有这等智慧和异能呢?
\par 55 这不是木匠的儿子吗?他母亲不是叫马利亚吗?他弟兄们不是叫雅各、约西(有古卷:约瑟)、西门、犹大吗?
\par 56 他妹妹们不是都在我们这里吗?这人从那里有这一切的事呢?」
\par 57 他们就厌弃他(厌弃他:原文作因他跌倒)。耶稣对他们说:「大凡先知,除了本地本家之外,没有不被人尊敬的。」
\par 58 耶稣因为他们不信,就在那里不多行异能了。

\chapter{14}

\par 1 那时,分封的王希律听见耶稣的名声,
\par 2 就对臣仆说:「这是施洗的约翰从死里复活,所以这些异能从他里面发出来。」
\par 3 起先,希律为他兄弟腓力的妻子希罗底的缘故,把约翰拿住,锁在监里。
\par 4 因为约翰曾对他说:「你娶这妇人是不合理的。」
\par 5 希律就想要杀他,只是怕百姓,因为他们以约翰为先知。
\par 6 到了希律的生日,希罗底的女儿在众人面前跳舞,使希律欢喜。
\par 7 希律就起誓,应许随他所求的给他。
\par 8 女儿被母亲所使,就说:「请把施洗约翰的头放在盘子里,拿来给我。」
\par 9 王便忧愁,但因他所起的誓,又因同席的人,就吩咐给他;
\par 10 於是打发人去,在监里斩了约翰,
\par 11 把头放在盘子里,拿来给了女子;女子拿去给他母亲。
\par 12 约翰的门徒来,把尸首领去埋葬了,就去告诉耶稣。
\par 13 耶稣听见了,就上船从那里独自退到野地里去。众人听见,就从各城里步行跟随他。
\par 14 耶稣出来,见有许多的人,就怜悯他们,治好了他们的病人。
\par 15 天将晚的时候,门徒进前来,说:「这是野地,时候已经过了,请叫众人散开,他们好往村子里去,自己买吃的。」
\par 16 耶稣说:「不用他们去,你们给他们吃吧!」
\par 17 门徒说:「我们这里只有五个饼,两条鱼。」
\par 18 耶稣说:「拿过来给我。」
\par 19 於是吩咐众人坐在草地上,就拿著这五个饼,两条鱼,望著天祝福,擘开饼,递给门徒,门徒又递给众人。
\par 20 他们都吃,并且吃饱了;把剩下的零碎收拾起来,装满了十二个篮子。
\par 21 吃的人,除了妇女孩子,约有五千。
\par 22 耶稣随即催门徒上船,先渡到那边去,等他叫众人散开。
\par 23 散了众人以後,他就独自上山去祷告。到了晚上,只有他一人在那里。
\par 24 那时船在海中,因风不顺,被浪摇撼。
\par 25 夜里四更天,耶稣在海面上走,往门徒那里去。
\par 26 门徒看见他在海面上走,就惊慌了,说:「是个鬼怪!」便害怕,喊叫起来。
\par 27 耶稣连忙对他们说:「你们放心!是我,不要怕!」
\par 28 彼得说:「主,如果是你,请叫我从水面上走到你那里去。」
\par 29 耶稣说:「你来吧。」彼得就从船上下去,在水面上走,要到耶稣那里去;
\par 30 只因见风甚大,就害怕,将要沉下去,便喊著说:「主啊,救我!」
\par 31 耶稣赶紧伸手拉住他,说:「你这小信的人哪,为什麽疑惑呢?」
\par 32 他们上了船,风就住了。
\par 33 在船上的人都拜他,说:「你真是神的儿子了。」
\par 34 他们过了海,来到革尼撒勒地方。
\par 35 那里的人一认出是耶稣,就打发人到周围地方去,把所有的病人带到他那里,
\par 36 只求耶稣准他们摸他的衣裳 子;摸著的人就都好了。

\chapter{15}

\par 1 那时,有法利赛人和文士从耶路撒冷来见耶稣,说:
\par 2 「你的门徒为什麽犯古人的遗传呢?因为吃饭的时候,他们不洗手。」
\par 3 耶稣回答说:「你们为什麽因著你们的遗传犯神的诫命呢?
\par 4 神说:『当孝敬父母』;又说:『咒骂父母的,必治死他。』
\par 5 你们倒说:『无论何人对父母说:我所当奉给你的已经作了供献,
\par 6 他就可以不孝敬父母。』这就是你们藉著遗传,废了神的诫命。
\par 7 假冒为善的人哪,以赛亚指著你们说的预言是不错的。他说:
\par 8 这百姓用嘴唇尊敬我,心却远离我;
\par 9 他们将人的吩咐当作道理教导人,所以拜我也是枉然。」
\par 10 耶稣就叫了众人来,对他们说:「你们要听,也要明白。
\par 11 入口的不能污秽人,出口的乃能污秽人。」
\par 12 当时,门徒进前来对他说:「法利赛人听见这话,不服(原文作跌倒),你知道吗?」
\par 13 耶稣回答说:「凡栽种的物,若不是我天父栽种的,必要拔出来。
\par 14 任凭他们吧!他们是瞎眼领路的;若是瞎子领瞎子,两个人都要掉在坑里。」
\par 15 彼得对耶稣说:「请将这比喻讲给我们听。」
\par 16 耶稣说:「你们到如今还不明白吗?
\par 17 岂不知凡入口的,是运到肚子里,又落在茅厕里吗?
\par 18 惟独出口的,是从心里发出来的,这才污秽人。
\par 19 因为从心里发出来的,有恶念、凶杀、奸淫、苟合、偷盗、妄证、谤 。
\par 20 这都是污秽人的;至於不洗手吃饭,那却不污秽人。」
\par 21 耶稣离开那里,退到推罗、西顿的境内去。
\par 22 有一个迦南妇人,从那地方出来,喊著说:「主啊,大卫的子孙,可怜我!我女儿被鬼附得甚苦。」
\par 23 耶稣却一言不答。门徒进前来,求他说:「这妇人在我们後头喊叫,请打发他走吧。」
\par 24 耶稣说:「我奉差遣不过是到以色列家迷失的羊那里去。」
\par 25 那妇人来拜他,说:「主啊,帮助我!」
\par 26 他回答说:「不好拿儿女的饼丢给狗吃。」
\par 27 妇人说:「主啊,不错;但是狗也吃他主人桌子上掉下来的碎渣儿。」
\par 28 耶稣说:「妇人,你的信心是大的!照你所要的,给你成全了吧。」从那时候,他女儿就好了。
\par 29 耶稣离开那地方,来到靠近加利利的海边,就上山坐下。
\par 30 有许多人到他那里,带著瘸子、瞎子、哑巴、有残疾的,和好些别的病人,都放在他脚前;他就治好了他们。
\par 31 甚至众人都希奇;因为看见哑巴说话,残疾的痊愈,瘸子行走,瞎子看见,他们就归荣耀给以色列的神。
\par 32 耶稣叫门徒来,说:「我怜悯这众人,因为他们同我在这里已经三天,也没有吃的了。我不愿意叫他们饿著回去,恐怕在路上困乏。」
\par 33 门徒说:「我们在这野地,那里有这麽多的饼叫这许多人吃饱呢?」
\par 34 耶稣说:「你们有多少饼?」他们说:「有七个,还有几条小鱼。」
\par 35 他就吩咐众人坐在地上,
\par 36 拿著这七个饼和几条鱼,祝谢了,擘开,递给门徒;门徒又递给众人。
\par 37 众人都吃,并且吃饱了,收拾剩下的零碎,装满了七个筐子。
\par 38 吃的人,除了妇女孩子,共有四千。
\par 39 耶稣叫众人散去,就上船,来到马加丹的境界。

\chapter{16}

\par 1 法利赛人和撒都该人来试探耶稣,请他从天上显个神迹给他们看。
\par 2 耶稣回答说:「晚上天发红,你们就说:『天必要晴。』
\par 3 早晨天发红,又发黑,你们就说:『今日必有风雨。」你们知道分辨天上的气色,倒不能分辨这时候的神迹。
\par 4 一个邪恶淫乱的世代求神迹,除了约拿的神迹以外,再没有神迹给他看。」耶稣就离开他们去了。
\par 5 门徒渡到那边去,忘了带饼。
\par 6 耶稣对他们说:「你们要谨慎,防备法利赛人和撒都该人的酵。」
\par 7 门徒彼此议论说:「这是因为我们没有带饼吧。」
\par 8 耶稣看出来,就说:「你们这小信的人,为什麽因为没有饼彼此议论呢?
\par 9 你们还不明白吗?不记得那五个饼分给五千人、又收拾了多少篮子的零碎吗?
\par 10 也不记得那七个饼分给四千人、又收拾了多少筐子的零碎吗?
\par 11 我对你们说:『要防备法利赛人和撒都该人的酵』,这话不是指著饼说的,你们怎麽不明白呢?」
\par 12 门徒这才晓得他说的不是叫他们防备饼的酵,乃是防备法利赛人和撒都该人的教训。
\par 13 耶稣到了该撒利亚腓立比的境内,就问门徒说:「人说我(有古卷没有我字),人子是谁?」
\par 14 他们说:「有人说是施洗的约翰;有人说是以利亚;又有人说是耶利米或是先知里的一位。」
\par 15 耶稣说:「你们说我是谁?」
\par 16 西门彼得回答说:「你是基督,是永生神的儿子。」
\par 17 耶稣对他说:「西门巴约拿,你是有福的!因为这不是属血肉的指示你的,乃是我在天上的父指示的。
\par 18 我还告诉你,你是彼得,我要把我的教会建造在这磐石上;阴间的权柄(权柄:原文作门),不能胜过他。
\par 19 我要把天国的钥匙给你,凡你在地上所捆绑的,在天上也要捆绑;凡你在地上所释放的,在天上也要释放。」
\par 20 当下,耶稣嘱咐门徒,不可对人说他是基督。
\par 21 从此,耶稣才指示门徒,他必须上耶路撒冷去,受长老、祭司长、文士许多的苦,并且被杀,第三日复活。
\par 22 彼得就拉著他,劝他说:「主啊,万不可如此!这事必不临到你身上。」
\par 23 耶稣转过来,对彼得说:「撒但,退我後边去吧!你是绊我脚的;因为你不体贴神的意思,只体贴人的意思。」
\par 24 於是耶稣对门徒说:「若有人要跟从我,就当舍己,背起他的十字架来跟从我。
\par 25 因为,凡要救自己生命(生命:译灵魂;下同)的,必丧掉生命;凡为我丧掉生命的,必得著生命。
\par 26 人若赚得全世界,赔上自己的生命,有什麽益处呢?人还能拿什麽换生命呢?
\par 27 人子要在他父的荣耀里同著众使者降临;那时候,他要照各人的行为报应各人。
\par 28 我实在告诉你们,站在这里的,有人在没尝死味以前必看见人子降临在他的国里。」

\chapter{17}

\par 1 过了六天,耶稣带著彼得、雅各,和雅各的兄弟约翰,暗暗的上了高山,
\par 2 就在他们面前变了形像,脸面明亮如日头,衣裳洁白如光。
\par 3 忽然,有摩西、以利亚向他们显现,同耶稣说话。
\par 4 彼得对耶稣说:「主啊,我们在这里真好!你若愿意,我就在这里搭三座棚,一座为你,一座为摩西,一座为以利亚。」
\par 5 说话之间,忽然有一朵光明的云彩遮盖他们,且有声音从云彩里出来,说:「这是我的爱子,我所喜悦的。你们要听他!」
\par 6 门徒听见,就俯伏在地,极其害怕。
\par 7 耶稣进前来,摸他们,说:「起来,不要害怕!」
\par 8 他们举目不见一人,只见耶稣在那里。
\par 9 下山的时候,耶稣吩咐他们说:「人子还没有从死里复活,你们不要将所看见的告诉人。」
\par 10 门徒问耶稣说:「文士为什麽说以利亚必须先来?」
\par 11 耶稣回答说:「以利亚固然先来,并要复兴万事;
\par 12 只是我告诉你们,以利亚已经来了,人却不认识他,竟任意待他。人子也将要这样受他们的害。」
\par 13 门徒这才明白耶稣所说的是指著施洗的约翰。
\par 14 耶稣和门徒到了众人那里,有一个人来见耶稣,跪下,
\par 15 说:「主啊,怜悯我的儿子。他害癫痫的病很苦,屡次跌在火里,屡次跌在水里。
\par 16 我带他到你门徒那里,他们却不能医治他。」
\par 17 耶稣说:「嗳!这又不信又悖谬的世代啊,我在你们这里要到几时呢?我忍耐你们要到几时呢?把他带到我这里来吧!」
\par 18 耶稣斥责那鬼,鬼就出来;从此孩子就痊愈了。
\par 19 门徒暗暗的到耶稣跟前,说:「我们为什麽不能赶出那鬼呢?」
\par 20 耶稣说:「是因你们的信心小。我实在告诉你们,你们若有信心,像一粒芥菜种,就是对这座山说:『你从这边挪到那边。』他也必挪去;并且你们没有一件不能做的事了。
\par 21 至於这一类的鬼,若不祷告、禁食,他就不出来(或作:不能赶他出来)。」
\par 22 他们还住在加利利的时候,耶稣对门徒说:「人子将要被交在人手里。
\par 23 他们要杀害他,第三日他要复活。」门徒就大大的忧愁。
\par 24 到了迦百农,有收丁税的人来见彼得,说:「你们的先生不纳丁税(丁税约有半块钱)吗?」
\par 25 彼得说:「纳。」他进了屋子,耶稣先向他说:「西门,你的意思如何?世上的君王向谁徵收关税、丁税?是向自己的儿子呢?是向外人呢?」
\par 26 彼得说:「是向外人。」耶稣说:「既然如此,儿子就可以免税了。
\par 27 但恐怕触犯(触犯:原文作绊倒)他们,你且往海边去钓鱼,把先钓上来的鱼拿起来,开了他的口,必得一块钱,可以拿去给他们,作你我的税银。」

\chapter{18}

\par 1 当时,门徒进前来,问耶稣说:「天国里谁是最大的?」
\par 2 耶稣便叫一个小孩子来,使他站在他们当中,
\par 3 说:「我实在告诉你们,你们若不回转,变成小孩子的样式,断不得进天国。
\par 4 所以,凡自己谦卑像这小孩子的,他在天国里就是最大的。
\par 5 凡为我的名接待一个像这小孩子的,就是接待我。」
\par 6 「凡使这信我的一个小子跌倒的,倒不如把大磨石拴在这人的颈项上,沉在深海里。
\par 7 这世界有祸了,因为将人绊倒;绊倒人的事是免不了的,但那绊倒人的有祸了!
\par 8 倘若你一只手,或是一只脚,叫你跌倒,就砍下来丢掉。你缺一只手,或是一只脚,进入永生,强如有两手两脚被丢在永火里。
\par 9 倘若你一只眼叫你跌倒,就把他剜出来丢掉。你只有一只眼进入永生,强如有两只眼被丢在地狱的火里。」
\par 10 「你们要小心,不可轻看这小子里的一个;我告诉你们,他们的使者在天上,常见我天父的面。(有古卷在此有
\par 11 人子来,为要拯救失丧的人。)
\par 12 一个人若有一百只羊,一只走迷了路,你们的意思如何?他岂不撇下这九十九只,往山里去找那只迷路的羊吗?
\par 13 若是找著了,我实在告诉你们,他为这一只羊欢喜,比为那没有迷路的九十九只欢喜还大呢!
\par 14 你们在天上的父也是这样,不愿意这小子里失丧一个。」
\par 15 「倘若你的弟兄得罪你,你就去,趁著只有他和你在一处的时候,指出他的错来。他若听你,你便得了你的弟兄;
\par 16 他若不听,你就另外带一两个人同去,要凭两三个人的口作见证,句句都可定准。
\par 17 若是不听他们,就告诉教会;若是不听教会,就看他像外邦人和税吏一样。
\par 18 我实在告诉你们,凡你们在地上所捆绑的,在天上也要捆绑;凡你们在地上所释放的,在天上也要释放。
\par 19 我又告诉你们,若是你们中间有两个人在地上同心合意的求什麽事,我在天上的父必为他们成全。
\par 20 因为无论在那里,有两三个人奉我的名聚会,那里就有我在他们中间。」
\par 21 那时,彼得进前来,对耶稣说:「主啊,我弟兄得罪我,我当饶恕他几次呢?到七次可以吗?」
\par 22 耶稣说:「我对你说,不是到七次,乃是到七十个七次。
\par 23 天国好像一个王要和他仆人算账。
\par 24 才算的时候,有人带了一个欠一千万银子的来。
\par 25 因为他没有什麽偿还之物,主人吩咐把他和他妻子儿女,并一切所有的都卖了偿还。
\par 26 那仆人就俯伏拜他,说:『主啊,宽容我,将来我都要还清。』
\par 27 那仆人的主人就动了慈心,把他释放了,并且免了他的债。
\par 28 「那仆人出来,遇见他的一个同伴欠他十两银子,便揪著他,掐住他的喉咙,说:『你把所欠的还我!』
\par 29 他的同伴就俯伏央求他,说:『宽容我吧,将来我必还清。』
\par 30 他不肯,竟去把他下在监里,等他还了所欠的债。
\par 31 众同伴看见他所做的事就甚忧愁,去把这事都告诉了主人。
\par 32 於是主人叫了他来,对他说:『你这恶奴才!你央求我,我就把你所欠的都免了,
\par 33 你不应当怜恤你的同伴,像我怜恤你吗?』
\par 34 主人就大怒,把他交给掌刑的,等他还清了所欠的债。
\par 35 你们各人若不从心里饶恕你的弟兄,我天父也要这样待你们了。」

\chapter{19}

\par 1 耶稣说完了这些话,就离开加利利,来到犹太的境界约但河外。
\par 2 有许多人跟著他,他就在那里把他们的病人治好了。
\par 3 有法利赛人来试探耶稣,说:「人无论什麽缘故都可以休妻吗?」
\par 4 耶稣回答说:「那起初造人的,是造男造女,
\par 5 并且说:『因此,人要离开父母,与妻子连合,二人成为一体。』这经你们没有念过吗?
\par 6 既然如此,夫妻不再是两个人,乃是一体的了。所以,神配合的,人不可分开。」
\par 7 法利赛人说:「这样,摩西为什麽吩咐给妻子休书,就可以休他呢?」
\par 8 耶稣说:「摩西因为你们的心硬,所以许你们休妻,但起初并不是这样。
\par 9 我告诉你们,凡休妻另娶的,若不是为淫乱的缘故,就是犯奸淫了;有人娶那被休的妇人,也是犯奸淫了。」
\par 10 门徒对耶稣说:「人和妻子既是这样,倒不如不娶。」
\par 11 耶稣说:「这话不是人都能领受的,惟独赐给谁,谁才能领受。
\par 12 因为有生来是阉人,也有被人阉的,并有为天国的缘故自阉的。这话谁能领受就可以领受。」
\par 13 那时,有人带著小孩子来见耶稣,要耶稣给他们按手祷告,门徒就责备那些人。
\par 14 耶稣说:「让小孩子到我这里来,不要禁止他们;因为在天国的,正是这样的人。」
\par 15 耶稣给他们按手,就离开那地方去了。
\par 16 有一个人来见耶稣,说:「夫子(有古卷:良善的夫子),我该做什麽善事才能得永生?」
\par 17 耶稣对他说:「你为什麽以善事问我呢?只有一位是善的(有古卷:你为什麽称我是良善的?除了神以外,没有一个良善的)。你若要进入永生,就当遵守诫命。」
\par 18 他说:「什麽诫命?」耶稣说:「就是不可杀人;不可奸淫;不可偷盗;不可作假见证;
\par 19 当孝敬父母,又当爱人如己。」
\par 20 那少年人说:「这一切我都遵守了,还缺少什麽呢?」
\par 21 耶稣说:「你若愿意作完全人,可去变卖你所有的,分给穷人,就必有财宝在天上;你还要来跟从我。」
\par 22 那少年人听见这话,就忧忧愁愁的走了,因为他的产业很多。
\par 23 耶稣对门徒说:「我实在告诉你们,财主进天国是难的。
\par 24 我又告诉你们,骆驼穿过针的眼,比财主进神的国还容易呢!」
\par 25 门徒听见这话,就希奇得很,说:「这样谁能得救呢?」
\par 26 耶稣看著他们,说:「在人这是不能的,在神凡事都能。」
\par 27 彼得就对他说:「看哪,我们已经撇下所有的跟从你,将来我们要得什麽呢?」
\par 28 耶稣说:「我实在告诉你们,你们这跟从我的人,到复兴的时候,人子坐在他荣耀的宝座上,你们也要坐在十二个宝座上,审判以色列十二个支派。
\par 29 凡为我的名撇下房屋,或是弟兄、姐妹、父亲、母亲、(有古卷加:妻子)儿女、田地的,必要得著百倍,并且承受永生。
\par 30 然而,有许多在前的,将要在後;在後的,将要在前。」

\chapter{20}

\par 1 「因为天国好像家主清早去雇人进他的葡萄园做工,
\par 2 和工人讲定一天一钱银子,就打发他们进葡萄园去。
\par 3 约在巳初出去,看见市上还有闲站的人,
\par 4 就对他们说:『你们也进葡萄园去,所当给的,我必给你们。』他们也进去了。
\par 5 约在午正和申初又出去,也是这样行。
\par 6 约在酉初出去,看见还有人站在那里,就问他们说:『你们为什麽整天在这里闲站呢?』
\par 7 他们说:『因为没有人雇我们。』他说:『你们也进葡萄园去。』
\par 8 到了晚上,园主对管事的说:『叫工人都来,给他们工钱,从後来的起,到先来的为止。
\par 9 约在酉初雇的人来了,各人得了一钱银子。
\par 10 及至那先雇的来了,他们以为必要多得;谁知也是各得一钱。
\par 11 他们得了,就埋怨家主说:
\par 12 『我们整天劳苦受热,那後来的只做了一小时,你竟叫他们和我们一样吗?』
\par 13 家主回答其中的一人说:『朋友,我不亏负你,你与我讲定的不是一钱银子吗?
\par 14 拿你的走吧!我给那後来的和给你一样,这是我愿意的。
\par 15 我的东西难道不可随我的意思用吗?因为我作好人,你就红了眼吗?』
\par 16 这样,那在後的,将要在前;在前的,将要在後了。(有古卷加:因为被召的人多,选上的人少。)」
\par 17 耶稣上耶路撒冷去的时候,在路上把十二个门徒带到一边,对他们说:
\par 18 「看哪,我们上耶路撒冷去,人子要被交给祭司长和文士。他们要定他死罪,
\par 19 又交给外邦人,将他戏弄,鞭打,钉在十字架上;第三日他要复活。」
\par 20 那时,西庇太儿子的母亲同他两个儿子上前来拜耶稣,求他一件事。
\par 21 耶稣说:「你要什麽呢?」他说:「愿你叫我这两个儿子在你国里,一个坐在你右边,一个坐在你左边。」
\par 22 耶稣回答说:「你们不知道所求的是什麽;我将要喝的杯,你们能喝吗?」他们说:「我们能。」
\par 23 耶稣说:「我所喝的杯,你们必要喝;只是坐在我的左右,不是我可以赐的,乃是我父为谁预备的,就赐给谁。」
\par 24 那十个门徒听见,就恼怒他们弟兄二人。
\par 25 耶稣叫了他们来,说:「你们知道外邦人有君王为主治理他们,有大臣操权管束他们。
\par 26 只是在你们中间,不可这样;你们中间谁愿为大,就必作你们的用人;
\par 27 谁愿为首,就必作你们的仆人。
\par 28 正如人子来,不是要受人的服事,乃是要服事人,并且要舍命,作多人的赎价。」
\par 29 他们出耶利哥的时候,有极多的人跟随他。
\par 30 有两个瞎子坐在路旁,听说是耶稣经过,就喊著说:「主啊,大卫的子孙,可怜我们吧!」
\par 31 众人责备他们,不许他们作声;他们却越发喊著说:「主啊,大卫的子孙,可怜我们吧!」
\par 32 耶稣就站住,叫他们来,说:「要我为你们做什麽?」
\par 33 他们说:「主啊,要我们的眼睛能看见!」
\par 34 耶稣就动了慈心,把他们的眼睛一摸,他们立刻看见,就跟从了耶稣。

\chapter{21}

\par 1 耶稣和门徒将近耶路撒冷,到了伯法其,在橄榄山那里。
\par 2 耶稣就打发两个门徒,对他们说:「你们往对面村子里去,必看见一匹驴拴在那里,还有驴驹同在一处;你们解开,牵到我这里来。
\par 3 若有人对你们说什麽,你们就说:『主要用他。』那人必立时让你们牵来。
\par 4 这事成就是要应验先知的话,说:
\par 5 要对锡安的居民(原文作女子)说:看哪,你的王来到你这里,是温柔的,又骑著驴,就是骑著驴驹子。」
\par 6 门徒就照耶稣所吩咐的去行,
\par 7 牵了驴和驴驹来,把自己的衣服搭在上面,耶稣就骑上。
\par 8 众人多半把衣服铺在路上;还有人砍下树枝来铺在路上。
\par 9 前行後随的众人喊著说:和散那(原有求救的意思,在此是称颂的话)归於大卫的子孙!奉主名来的是应当称颂的!高高在上和散那!」
\par 10 耶稣既进了耶路撒冷,合城都惊动了,说:「这是谁?」
\par 11 众人说:「这是加利利拿撒勒的先知耶稣。」
\par 12 耶稣进了神的殿,赶出殿里一切做买卖的人,推倒兑换银钱之人的桌子,和卖鸽子之人的凳子;
\par 13 对他们说:「经上记著说:我的殿必称为祷告的殿,你们倒使他成为贼窝了。」
\par 14 在殿里有瞎子、瘸子到耶稣跟前,他就治好了他们。
\par 15 祭司长和文士看见耶稣所行的奇事,又见小孩子在殿里喊著说:「和散那归於大卫的子孙!」就甚恼怒,
\par 16 对他说:「这些人所说的,你听见了吗?」耶稣说:「是的。经上说『你从婴孩和吃奶的口中完全了赞美』的话,你们没有念过吗?」
\par 17 於是离开他们,出城到伯大尼去,在那里住宿。
\par 18 早晨回城的时候,他饿了,
\par 19 看见路旁有一棵无花果树,就走到跟前,在树上找不著什麽,不过有叶子,就对树说:「从今以後,你永不结果子。」那无花果树就立刻枯乾了。
\par 20 门徒看见了,便希奇说:「无花果树怎麽立刻枯乾了呢?」
\par 21 耶稣回答说:「我实在告诉你们,你们若有信心,不疑惑,不但能行无花果树上所行的事,就是对这座山说:『你挪开此地,投在海里!』也必成就。
\par 22 你们祷告,无论求什麽,只要信,就必得著。」
\par 23 耶稣进了殿,正教训人的时候,祭司长和民间的长老来问他说:「你仗著什麽权柄做这些事?给你这权柄的是谁呢?」
\par 24 耶稣回答说:「我也要问你们一句话,你们若告诉我,我就告诉你们我仗著什麽权柄做这些事。
\par 25 约翰的洗礼是从那里来的?是从天上来的?是从人间来的呢?」他们彼此商议说:「我们若说『从天上来,』他必对我们说:『这样,你们为什麽不信他呢?』
\par 26 若说『从人间来』,我们又怕百姓,因为他们都以约翰为先知。」
\par 27 於是回答耶稣说:「我们不知道。」耶稣说:「我也不告诉你们我仗著什麽权柄做这些事。」
\par 28 又说:「一个人有两个儿子。他来对大儿子说:『我儿,你今天到葡萄园里去做工。』
\par 29 他回答说:『我不去』,以後自己懊悔,就去了。
\par 30 又来对小儿子也是这样说。他回答说:『父啊,我去』,他却不去。
\par 31 你们想,这两个儿子是那一个遵行父命呢?」他们说:「大儿子。」耶稣说:「我实在告诉你们,税吏和娼妓倒比你们先进神的国。
\par 32 因为约翰遵著义路到你们这里来,你们却不信他;税吏和娼妓倒信他。你们看见了,後来还是不懊悔去信他。」
\par 33 「你们再听一个比喻:有个家主栽了一个葡萄园,周围圈上篱笆,里面挖了一个压酒池,盖了一座楼,租给园户,就往外国去了。
\par 34 收果子的时候近了,就打发仆人到园户那里去收果子。
\par 35 园户拿住仆人,打了一个,杀了一个,用石头打死一个。
\par 36 主人又打发别的仆人去,比先前更多;园户还是照样待他们。
\par 37 後来打发他的儿子到他们那里去,意思说:『他们必尊敬我的儿子。』
\par 38 不料,园户看见他儿子,就彼此说:(这是承受产业的。来吧,我们杀他,占他的产业!』
\par 39 他们就拿住他,推出葡萄园外,杀了。
\par 40 园主来的时候要怎样处治这些园户呢?」
\par 41 他们说:「要下毒手除灭那些恶人,将葡萄园另租给那按著时候交果子的园户。」
\par 42 耶稣说:「经上写著:匠人所弃的石头已作了房角的头块石头。这是主所做的,在我们眼中看为希奇。这经你们没有念过吗?
\par 43 所以我告诉你们,神的国必从你们夺去,赐给那能结果子的百姓。
\par 44 谁掉在这石头上,必要跌碎;这石头掉在谁的身上,就要把谁砸得稀烂。」
\par 45 祭司长和法利赛人听见他的比喻,就看出他是指著他们说的。
\par 46 他们想要捉拿他,只是怕众人,因为众人以他为先知。

\chapter{22}

\par 1 耶稣又用比喻对他们说:
\par 2 「天国好比一个王为他儿子摆设娶亲的筵席,
\par 3 就打发仆人去,请那些被召的人来赴席,他们却不肯来。
\par 4 王又打发别的仆人,说:『你们告诉那被召的人,我的筵席已经预备好了,牛和肥畜已经宰了,各样都齐备,请你们来赴席。』
\par 5 那些人不理就走了;一个到自己田里去;一个做买卖去;
\par 6 其余的拿住仆人,凌辱他们,把他们杀了。
\par 7 王就大怒,发兵除灭那些凶手,烧毁他们的城。
\par 8 於是对仆人说:『喜筵已经齐备,只是所召的人不配。
\par 9 所以你们要往岔路口上去,凡遇见的,都召来赴席。』
\par 10 那些仆人就出去,到大路上,凡遇见的,不论善恶都召聚了来,筵席上就坐满了客。
\par 11 王进来观看宾客,见那里有一个没有穿礼服的,
\par 12 就对他说:『朋友,你到这里来怎麽不穿礼服呢?』那人无言可答。
\par 13 於是王对使唤的人说:『捆起他的手脚来,把他丢在外边的黑暗里;在那里必要哀哭切齿了。』
\par 14 因为被召的人多,选上的人少。」
\par 15 当时,法利赛人出去商议,怎样就著耶稣的话陷害他,
\par 16 就打发他们的门徒同希律党的人去见耶稣,说:「夫子,我们知道你是诚实人,并且诚诚实实传神的道,什麽人你都不徇情面,因为你不看人的外貌。
\par 17 请告诉我们,你的意见如何?纳税给该撒可以不可以?」
\par 18 耶稣看出他们的恶意,就说:「假冒为善的人哪,为什麽试探我?
\par 19 拿一个上税的钱给我看!」他们就拿一个银钱来给他。
\par 20 耶稣说:「这像和这号是谁的?」
\par 21 他们说:「是该撒的。」耶稣说:「这样,该撒的物当归给该撒;神的物当归给神。」
\par 22 他们听见就希奇,离开他走了。
\par 23 撒都该人常说没有复活的事。那天,他们来问耶稣说:
\par 24 「夫子,摩西说:『人若死了,没有孩子,他兄弟当娶他的妻,为哥哥生子立後。』
\par 25 从前,在我们这里有弟兄七人,第一个娶了妻,死了,没有孩子,撇下妻子给兄弟。
\par 26 第二、第三、直到第七个,都是如此。
\par 27 末後,妇人也死了。
\par 28 这样,当复活的时候,他是七个人中那一个的妻子呢?因为他们都娶过他。」
\par 29 耶稣回答说:「你们错了;因为不明白圣经,也不晓得神的大能。
\par 30 当复活的时候,人也不娶也不嫁,乃像天上的使者一样。
\par 31 论到死人复活,神在经上向你们所说的,你们没有念过吗?
\par 32 他说:『我是亚伯拉罕的神,以撒的神,雅各的神。』神不是死人的神,乃是活人的神。」
\par 33 众人听见这话,就希奇他的教训。
\par 34 法利赛人听见耶稣堵住了撒都该人的口,他们就聚集。
\par 35 内中有一个人是律法师,要试探耶稣,就问他说:
\par 36 「夫子,律法上的诫命,那一条是最大的呢?」
\par 37 耶稣对他说:「你要尽心、尽性、尽意爱主你的神。
\par 38 这是诫命中的第一,且是最大的。
\par 39 其次也相仿,就是要爱人如己。
\par 40 这两条诫命是律法和先知一切道理的总纲。」
\par 41 法利赛人聚集的时候,耶稣问他们说:
\par 42 「论到基督,你们的意见如何?他是谁的子孙呢?」他们回答说:「是大卫的子孙。」
\par 43 耶稣说:「这样,大卫被圣灵感动,怎麽还称他为主,说:
\par 44 主对我主说:你坐在我的右边,等我把你仇敌放在你的脚下。
\par 45 大卫既称他为主,他怎麽又是大卫的子孙呢?」
\par 46 他们没有一个人能回答一言。从那日以後,也没有人敢再问他什麽。

\chapter{23}

\par 1 那时,耶稣对众人和门徒讲论,
\par 2 说:「文士和法利赛人坐在摩西的位上,
\par 3 凡他们所吩咐你们的,你们都要谨守遵行;但不要效法他们的行为;因为他们能说,不能行。
\par 4 他们把难担的重担捆起来,搁在人的肩上,但自己一个指头也不肯动。
\par 5 他们一切所做的事都是要叫人看见,所以将佩戴的经文做宽了,衣裳的 子做长了,
\par 6 喜爱筵席上的首座,会堂里的高位,
\par 7 又喜爱人在街市上问他安,称呼他拉比(拉比就是夫子)。
\par 8 但你们不要受拉比的称呼,因为只有一位是你们的夫子;你们都是弟兄。
\par 9 也不要称呼地上的人为父,因为只有一位是你们的父,就是在天上的父。
\par 10 也不要受师尊的称呼,因为只有一位是你们的师尊,就是基督。
\par 11 你们中间谁为大,谁就要作你们的用人。
\par 12 凡自高的,必降为卑;自卑的,必升为高。
\par 13 「你们这假冒为善的文士和法利赛人有祸了!因为你们正当人前,把天国的门关了,自己不进去,正要进去的人,你们也不容他们进去。(有古卷在此有
\par 14 你们这假冒为善的文士和法利赛人有祸了!因为你们侵吞寡妇的家产,假意做很长的祷告,所以要受更重的刑罚。)
\par 15 「你们这假冒为善的文士和法利赛人有祸了!因为你们走遍洋海陆地,勾引一个人入教,既入了教,却使他作地狱之子,比你们还加倍。
\par 16 「你们这瞎眼领路的有祸了!你们说:『凡指著殿起誓的,这算不得什麽;只是凡指著殿中金子起誓的,他就该谨守。』
\par 17 你们这无知瞎眼的人哪,什麽是大的?是金子呢?还是叫金子成圣的殿呢?
\par 18 你们又说:『凡指著坛起誓的,这算不得什麽;只是凡指著坛上礼物起誓的,他就该谨守。』
\par 19 你们这瞎眼的人哪,什麽是大的?是礼物呢?还是叫礼物成圣的坛呢?
\par 20 所以,人指著坛起誓,就是指著坛和坛上一切所有的起誓;
\par 21 人指著殿起誓,就是指著殿和那住在殿里的起誓;
\par 22 人指著天起誓,就是指著神的宝座和那坐在上面的起誓。
\par 23 「你们这假冒为善的文士和法利赛人有祸了!因为你们将薄荷、茴香、芹菜,献上十分之一,那律法上更重的事,就是公义、怜悯、信实,反倒不行了。这更重的是你们当行的;那也是不可不行的。
\par 24 你们这瞎眼领路的,蠓虫你们就滤出来,骆驼你们倒吞下去。
\par 25 「你们这假冒为善的文士和法利赛人有祸了!因为你们洗净杯盘的外面,里面却盛满了勒索和放荡。
\par 26 你这瞎眼的法利赛人,先洗净杯盘的里面,好叫外面也乾净了。
\par 27 「你们这假冒为善的文士和法利赛人有祸了!因为你们好像粉饰的坟墓,外面好看,里面却装满了死人的骨头和一切的污秽。
\par 28 你们也是如此,在人前,外面显出公义来,里面却装满了假善和不法的事。
\par 29 「你们这假冒为善的文士和法利赛人有祸了!因为你们建造先知的坟,修饰义人的墓,说:
\par 30 『若是我们在我们祖宗的时候,必不和他们同流先知的血。』
\par 31 这就是你们自己证明是杀害先知者的子孙了。
\par 32 你们去充满你们祖宗的恶贯吧!
\par 33 你们这些蛇类、毒蛇之种啊,怎能逃脱地狱的刑罚呢?
\par 34 所以我差遣先知和智慧人并文士到你们这里来,有的你们要杀害,要钉十字架;有的你们要在会堂里鞭打,从这城追逼到那城,
\par 35 叫世上所流义人的血都归到你们身上,从义人亚伯的血起,直到你们在殿和坛中间所杀的巴拉加的儿子撒迦利亚的血为止。
\par 36 我实在告诉你们,这一切的罪都要归到这世代了。」
\par 37 「耶路撒冷啊,耶路撒冷啊,你常杀害先知,又用石头打死那奉差遣到你这里来的人。我多次愿意聚集你的儿女,好像母鸡把小鸡聚集在翅膀底下,只是你们不愿意。
\par 38 看哪,你们的家成为荒场留给你们。
\par 39 我告诉你们,从今以後,你们不得再见我,直等到你们说:『奉主名来的是应当称颂的。』」

\chapter{24}

\par 1 耶稣出了圣殿,正走的时候,门徒进前来,把殿宇指给他看。
\par 2 耶稣对他们说:「你们不是看见这殿宇吗?我实在告诉你们,将来在这里没有一块石头留在石头上,不被拆毁了。」
\par 3 耶稣在橄榄山上坐著,门徒暗暗的来说:「请告诉我们,什麽时候有这些事?你降临和世界的末了有什麽预兆呢?」
\par 4 耶稣回答说:「你们要谨慎,免得有人迷惑你们。
\par 5 因为将来有好些人冒我的名来,说:『我是基督』,并且要迷惑许多人。
\par 6 你们也要听见打仗和打仗的风声,总不要惊慌;因为这些事是必须有的,只是末期还没有到。
\par 7 民要攻打民,国要攻打国;多处必有饥荒、地震。
\par 8 这都是灾难(灾难:原文作生产之难)的起头。
\par 9 那时,人要把你们陷在患难里,也要杀害你们;你们又要为我的名被万民恨恶。
\par 10 那时,必有许多人跌倒,也要彼此陷害,彼此恨恶;
\par 11 且有好些假先知起来,迷惑多人。
\par 12 只因不法的事增多,许多人的爱心才渐渐冷淡了。
\par 13 惟有忍耐到底的,必然得救。
\par 14 这天国的福音要传遍天下,对万民作见证,然後末期才来到。」
\par 15 「你们看见先知但以理所说的『那行毁坏可憎的』站在圣地(读这经的人须要会意)。
\par 16 那时,在犹太的,应当逃到山上;
\par 17 在房上的,不要下来拿家里的东西;
\par 18 在田里的,也不要回去取衣裳。
\par 19 当那些日子,怀孕的和奶孩子的有祸了。
\par 20 你们应当祈求,叫你们逃走的时候,不遇见冬天或是安息日。
\par 21 因为那时必有大灾难,从世界的起头直到如今,没有这样的灾难,後来也必没有。
\par 22 若不减少那日子,凡有血气的总没有一个得救的;只是为选民,那日子必减少了。
\par 23 那时,若有人对你们说:『基督在这里』,或说:『基督在那里』,你们不要信!
\par 24 因为假基督、假先知将要起来,显大神迹、大奇事,倘若能行,连选民也就迷惑了。
\par 25 看哪,我预先告诉你们了。
\par 26 若有人对你们说:『看哪,基督在旷野里』,你们不要出去!或说:『看哪,基督在内屋中』,你们不要信!
\par 27 闪电从东边发出,直照到西边。人子降临,也要这样。
\par 28 尸首在那里,鹰也必聚在那里。」
\par 29 「那些日子的灾难一过去,日头就变黑了,月亮也不放光,众星要从天上坠落,天势都要震动。
\par 30 那时,人子的兆头要显在天上,地上的万族都要哀哭。他们要看见人子,有能力,有大荣耀,驾著天上的云降临。
\par 31 他要差遣使者,用号筒的大声,将他的选民,从四方(方:原文作风),从天这边到天那边,都招聚了来。」
\par 32 「你们可以从无花果树学个比方:当树枝发嫩长叶的时候,你们就知道夏天近了。
\par 33 这样,你们看见这一切的事,也该知道人子近了,正在门口了。
\par 34 我实在告诉你们,这世代还没有过去,这些事都要成就。
\par 35 天地要废去,我的话却不能废去。」
\par 36 「那日子,那时辰,没有人知道,连天上的使者也不知道,子也不知道,惟独父知道。
\par 37 挪亚的日子怎样,人子降临也要怎样。
\par 38 当洪水以前的日子,人照常吃喝嫁娶,直到挪亚进方舟的那日;
\par 39 不知不觉洪水来了,把他们全都冲去。人子降临也要这样。
\par 40 那时,两个人在田里,取去一个,撇下一个。
\par 41 两个女人推磨,取去一个,撇下一个。
\par 42 所以,你们要警醒,因为不知道你们的主是那一天来到。
\par 43 家主若知道几更天有贼来,就必警醒,不容人挖透房屋;这是你们所知道的。
\par 44 所以,你们也要预备,因为你们想不到的时候,人子就来了。」
\par 45 「谁是忠心有见识的仆人,为主人所派,管理家里的人,按时分粮给他们呢?
\par 46 主人来到,看见他这样行,那仆人就有福了。
\par 47 我实在告诉你们,主人要派他管理一切所有的。
\par 48 倘若那恶仆心里说:『我的主人必来得迟』,
\par 49 就动手打他的同伴,又和酒醉的人一同吃喝。
\par 50 在想不到的日子,不知道的时辰,那仆人的主人要来,
\par 51 重重的处治他(或作:把他腰斩了),定他和假冒为善的人同罪;在那里必要哀哭切齿了。」

\chapter{25}

\par 1 「那时,天国好比十个童女拿著灯出去迎接新郎。
\par 2 其中有五个是愚拙的,五个是聪明的。
\par 3 愚拙的拿著灯,却不预备油;
\par 4 聪明的拿著灯,又预备油在器皿里。
\par 5 新郎迟延的时候,他们都打盹,睡著了。
\par 6 半夜有人喊著说:『新郎来了,你们出来迎接他!』
\par 7 那些童女就都起来收拾灯。
\par 8 愚拙的对聪明的说:『请分点油给我们,因为我们的灯要灭了。』
\par 9 聪明的回答说:『恐怕不够你我用的;不如你们自己到卖油的那里去买吧。』
\par 10 他们去买的时候,新郎到了。那预备好了的,同他进去坐席,门就关了。
\par 11 其余的童女随後也来了,说:『主啊,主啊,给我们开门!』
\par 12 他却回答说:『我实在告诉你们,我不认识你们。』
\par 13 所以,你们要警醒;因为那日子,那时辰,你们不知道。」
\par 14 「天国又好比一个人要往外国去,就叫了仆人来,把他的家业交给他们,
\par 15 按著各人的才干给他们银子:一个给了五千,一个给了二千,一个给了一千,就往外国去了。
\par 16 那领五千的随即拿去做买卖,另外赚了五千。
\par 17 那领二千的也照样另赚了二千。
\par 18 但那领一千的去掘开地,把主人的银子埋藏了。
\par 19 过了许久,那些仆人的主人来了,和他们算账。
\par 20 那领五千银子的又带著那另外的五千来,说:『主啊,你交给我五千银子。请看,我又赚了五千。』
\par 21 主人说:『好,你这又良善又忠心的仆人,你在不多的事上有忠心,我要把许多事派你管理;可以进来享受你主人的快乐。』
\par 22 那领二千的也来,说:『主啊,你交给我二千银子。请看,我又赚了二千。』
\par 23 主人说:『好,你这又良善又忠心的仆人,你在不多的事上有忠心,我要把许多事派你管理;可以进来享受你主人的快乐。』
\par 24 那领一千的也来,说:『主啊,我知道你是忍心的人,没有种的地方要收割,没有散的地方要聚敛,
\par 25 我就害怕,去把你的一千银子埋藏在地里。请看,你的原银子在这里。』
\par 26 主人回答说:『你这又恶又懒的仆人,你既知道我没有种的地方要收割,没有散的地方要聚敛,
\par 27 就当把我的银子放给兑换银钱的人,到我来的时候,可以连本带利收回。
\par 28 夺过他这一千来,给那有一万的。
\par 29 因为凡有的,还要加给他,叫他有余;没有的,连他所有的也要夺过来。
\par 30 把这无用的仆人丢在外面黑暗里;在那里必要哀哭切齿了。』」
\par 31 「当人子在他荣耀里,同著众天使降临的时候,要坐在他荣耀的宝座上。
\par 32 万民都要聚集在他面前。他要把他们分别出来,好像牧羊的分别绵羊山羊一般,
\par 33 把绵羊安置在右边,山羊在左边。
\par 34 於是王要向那右边的说:『你们这蒙我父赐福的,可来承受那创世以来为你们所预备的国;
\par 35 因为我饿了,你们给我吃,渴了,你们给我喝;我作客旅,你们留我住;
\par 36 我赤身露体,你们给我穿;我病了、你们看顾我;我在监里,你们来看我。』
\par 37 义人就回答说:『主啊,我们什麽时候见你饿了,给你吃,渴了,给你喝?
\par 38 什麽时候见你作客旅,留你住,或是赤身露体,给你穿?
\par 39 又什麽时候见你病了,或是在监里,来看你呢?』
\par 40 王要回答说:『我实在告诉你们,这些事你们既做在我这弟兄中一个最小的身上,就是做在我身上了。』
\par 41 王又要向那左边的说:『你们这被咒诅的人,离开我!进入那为魔鬼和他的使者所预备的永火里去!
\par 42 因为我饿了,你们不给我吃,渴了,你们不给我喝;
\par 43 我作客旅,你们不留我住;我赤身露体,你们不给我穿;我病了,我在监里,你们不来看顾我。』
\par 44 他们也要回答说:『主啊,我们什麽时候见你饿了,或渴了,或作客旅,或赤身露体,或病了,或在监里,不伺候你呢?』
\par 45 王要回答说:『我实在告诉你们,这些事你们既不做在我这弟兄中一个最小的身上,就是不做在我身上了。』
\par 46 这些人要往永刑里去;那些义人要往永生里去。」

\chapter{26}

\par 1 耶稣说完了这一切的话,就对门徒说:
\par 2 「你们知道,过两天是逾越节,人子将要被交给人,钉在十字架上。」
\par 3 那时,祭司长和民间的长老聚集在大祭司称为该亚法的院里。
\par 4 大家商议要用诡计拿住耶稣,杀他,
\par 5 只是说:「当节的日子不可,恐怕民间生乱。」
\par 6 耶稣在伯大尼长大麻疯的西门家里,
\par 7 有一个女人拿著一玉瓶极贵的香膏来,趁耶稣坐席的时候,浇在他的头上。
\par 8 门徒看见就很不喜悦,说:「何用这样的枉费呢!
\par 9 这香膏可以卖许多钱, 济穷人。」
\par 10 耶稣看出他们的意思,就说:「为什麽难为这女人呢?他在我身上做的是一件美事。
\par 11 因为常有穷人和你们同在;只是你们不常有我。
\par 12 他将这香膏浇在我身上是为我安葬做的。
\par 13 我实在告诉你们,普天之下,无论在什麽地方传这福音,也要述说这女人所行的,作个纪念。」
\par 14 当下,十二门徒里有一个称为加略人犹大的,去见祭司长,说:
\par 15 「我把他交给你们,你们愿意给我多少钱?」他们就给了他三十块钱。
\par 16 从那时候,他就找机会要把耶稣交给他们。
\par 17 除酵节的第一天,门徒来问耶稣说:「你吃逾越节的筵席,要我们在那里给你预备?」
\par 18 耶稣说:「你们进城去,到某人那里,对他说:『夫子说:我的时候快到了,我与门徒要在你家里守逾越节。』」
\par 19 门徒遵著耶稣所吩咐的就去预备了逾越节的筵席。
\par 20 到了晚上,耶稣和十二个门徒坐席。
\par 21 正吃的时候,耶稣说:「我实在告诉你们,你们中间有一个人要卖我了。」
\par 22 他们就甚忧愁,一个一个的问他说:「主,是我吗?」
\par 23 耶稣回答说:「同我蘸手在盘子里的,就是他要卖我。
\par 24 人子必要去世,正如经上指著他所写的;但卖人子的人有祸了!那人不生在世上倒好。」
\par 25 卖耶稣的犹大问他说:「拉比,是我吗?」耶稣说:「你说的是。」
\par 26 他们吃的时候,耶稣拿起饼来,祝福,就擘开,递给门徒,说:「你们拿著吃,这是我的身体」;
\par 27 又拿起杯来,祝谢了,递给他们,说:「你们都喝这个;
\par 28 因为这是我立约的血,为多人流出来,使罪得赦。
\par 29 但我告诉你们,从今以後,我不再喝这葡萄汁,直到我在我父的国里同你们喝新的那日子。」
\par 30 他们唱了诗,就出来往橄榄山去。
\par 31 那时,耶稣对他们说:「今夜,你们为我的缘故都要跌倒。因为经上记著说:我要击打牧人,羊就分散了。
\par 32 但我复活以後,要在你们以先往加利利去。」
\par 33 彼得说:「众人虽然为你的缘故跌倒,我却永不跌倒。」
\par 34 耶稣说:「我实在告诉你,今夜鸡叫以先,你要三次不认我。」
\par 35 彼得说:「我就是必须和你同死,也总不能不认你。」众门徒都是这样说。
\par 36 耶稣同门徒来到一个地方,名叫客西马尼,就对他们说:「你们坐在这里,等我到那边去祷告。」
\par 37 於是带著彼得和西庇太的两个儿子同去,就忧愁起来,极其难过,
\par 38 便对他们说:「我心里甚是忧伤,几乎要死;你们在这里等候,和我一同警醒。」
\par 39 他就稍往前走,俯伏在地,祷告说:「我父啊,倘若可行,求你叫这杯离开我。然而,不要照我的意思,只要照你的意思。」
\par 40 来到门徒那里,见他们睡著了,就对彼得说:「怎麽样?你们不能同我警醒片时吗?
\par 41 总要警醒祷告,免得入了迷惑。你们心灵固然愿意,肉体却软弱了。」
\par 42 第二次又去祷告说:「我父啊,这杯若不能离开我,必要我喝,就愿你的意旨成全。」
\par 43 又来,见他们睡著了,因为他们的眼睛困倦。
\par 44 耶稣又离开他们去了。第三次祷告,说的话还是与先前一样。
\par 45 於是来到门徒那里,对他们说:「现在你们仍然睡觉安歇吧(吧:或作吗?)!时候到了,人子被卖在罪人手里了。
\par 46 起来!我们走吧。看哪,卖我的人近了。」
\par 47 说话之间,那十二个门徒里的犹大来了,并有许多人带著刀棒,从祭司长和民间的长老那里与他同来。
\par 48 那卖耶稣的给了他们一个暗号,说:「我与谁亲嘴,谁就是他。你们可以拿住他。」
\par 49 犹大随即到耶稣跟前,说:「请拉比安」,就与他亲嘴。
\par 50 耶稣对他说:「朋友,你来要做的事,就做吧。」於是那些人上前,下手拿住耶稣。
\par 51 有跟随耶稣的一个人伸手拔出刀来,将大祭司的仆人砍了一刀,削掉了他一个耳朵。
\par 52 耶稣对他说:「收刀入鞘吧!凡动刀的,必死在刀下。
\par 53 你想,我不能求我父现在为我差遣十二营多天使来吗?
\par 54 若是这样,经上所说,事情必须如此的话怎麽应验呢?」
\par 55 当时,耶稣对众人说:「你们带著刀棒出来拿我,如同拿强盗吗?我天天坐在殿里教训人,你们并没有拿我。
\par 56 但这一切的事成就了,为要应验先知书上的话。」当下,门徒都离开他逃走了。
\par 57 拿耶稣的人把他带到大祭司该亚法那里去;文士和长老已经在那里聚会。
\par 58 彼得远远的跟著耶稣,直到大祭司的院子,进到里面,就和差役同坐,要看这事到底怎样。
\par 59 祭司长和全公会寻找假见证控告耶稣,要治死他。
\par 60 虽有好些人来作假见证,总得不著实据,末後有两个人前来,说:
\par 61 「这个人曾说:『我能拆毁神的殿,三日内又建造起来。』」
\par 62 大祭司就站起来,对耶稣说:「你什麽都不回答吗?这些人作见证告你的是什麽呢?」
\par 63 耶稣却不言语。大祭司对他说:「我指著永生神叫你起誓告诉我们,你是神的儿子基督不是?」
\par 64 耶稣对他说:「你说的是。然而,我告诉你们,後来你们要看见人子坐在那权能者的右边,驾著天上的云降临。」
\par 65 大祭司就撕开衣服,说:「他说了僭妄的话,我们何必再用见证人呢?这僭妄的话,现在你们都听见了。
\par 66 你们的意见如何?」他们回答说:「他是该死的。」
\par 67 他们就吐唾沫在他脸上,用拳头打他;也有用手掌打他的,说:
\par 68 「基督啊!你是先知,告诉我们打你的是谁?」
\par 69 彼得在外面院子里坐著,有一个使女前来,说:「你素来也是同那加利利人耶稣一夥的。」
\par 70 彼得在众人面前却不承认,说:「我不知道你说的是什麽!」
\par 71 既出去,到了门口,又有一个使女看见他,就对那里的人说:「这个人也是同拿撒勒人耶稣一夥的。」
\par 72 彼得又不承认,并且起誓说:「我不认得那个人。」
\par 73 过了不多的时候,旁边站著的人前来,对彼得说:「你真是他们一党的,你的口音把你露出来了。」
\par 74 彼得就发咒起誓的说:「我不认得那个人。」立时,鸡就叫了。
\par 75 彼得想起耶稣所说的话:「鸡叫以先,你要三次不认我。」他就出去痛哭。

\chapter{27}

\par 1 到了早晨,众祭司长和民间的长老大家商议要治死耶稣,
\par 2 就把他捆绑,解去,交给巡抚彼拉多。
\par 3 这时候,卖耶稣的犹大看见耶稣已经定了罪,就後悔,把那三十块钱拿回来给祭司长和长老,说:
\par 4 「我卖了无辜之人的血是有罪了。」他们说:「那与我们有什麽相干?你自己承当吧!」
\par 5 犹大就把那银钱丢在殿里,出去吊死了。
\par 6 祭司长拾起银钱来,说:「这是血价,不可放在库里。」
\par 7 他们商议,就用那银钱买了窑户的一块田,为要埋葬外乡人。
\par 8 所以那块田直到今日还叫做「血田」。
\par 9 这就应验了先知耶利米的话,说:「他们用那三十块钱,就是被估定之人的价钱,是以色列人中所估定的,
\par 10 买了窑户的一块田;这是照著主所吩咐我的。」
\par 11 耶稣站在巡抚面前;巡抚问他说:「你是犹太人的王吗?」耶稣说:「你说的是。」
\par 12 他被祭司长和长老控告的时候,什麽都不回答。
\par 13 彼拉多就对他说:「他们作见证告你这麽多的事,你没有听见吗?」
\par 14 耶稣仍不回答,连一句话也不说,以致巡抚甚觉希奇。
\par 15 巡抚有一个常例,每逢这节期,随众人所要的释放一个囚犯给他们。
\par 16 当时有一个出名的囚犯叫巴拉巴。
\par 17 众人聚集的时候,彼拉多就对他们说:「你们要我释放那一个给你们?是巴拉巴呢?是称为基督的耶稣呢?』
\par 18 巡抚原知道他们是因为嫉妒才把他解了来。
\par 19 正坐堂的时候,他的夫人打发人来说:「这义人的事,你一点不可管,因为我今天在梦中为他受了许多的苦。」
\par 20 祭司长和长老挑唆众人,求释放巴拉巴,除灭耶稣。
\par 21 巡抚对众人说:「这两个人,你们要我释放那一个给你们呢?」他们说:「巴拉巴。」
\par 22 彼拉多说:「这样,那称为基督的耶稣我怎麽办他呢?」他们都说:「把他钉十字架!」
\par 23 巡抚说:「为什麽呢?他做了什麽恶事呢?」他们便极力的喊著说:「把他钉十字架!」
\par 24 彼拉多见说也无济於事,反要生乱,就拿水在众人面前洗手,说:「流这义人的血,罪不在我,你们承当吧。」
\par 25 众人都回答说:「他的血归到我们和我们的子孙身上。」
\par 26 於是彼拉多释放巴拉巴给他们,把耶稣鞭打了,交给人钉十字架。
\par 27 巡抚的兵就把耶稣带进衙门,叫全营的兵都聚集在他那里。
\par 28 他们给他脱了衣服,穿上一件朱红色袍子,
\par 29 用荆棘编做冠冕,戴在他头上,拿一根苇子放在他右手里,跪在他面前,戏弄他,说:「恭喜,犹太人的王啊!」
\par 30 又吐唾沫在他脸上,拿苇子打他的头。
\par 31 戏弄完了,就给他脱了袍子,仍穿上他自己的衣服,带他出去,要钉十字架。
\par 32 他们出来的时候,遇见一个古利奈人,名叫西门,就勉强他同去,好背著耶稣的十字架。
\par 33 到了一个地方名叫各各他,意思就是「髑髅地」。
\par 34 兵丁拿苦胆调和的酒给耶稣喝。他尝了,就不肯喝。
\par 35 他们既将他钉在十字架上,就拈阄分他的衣服,
\par 36 又坐在那里看守他。
\par 37 在他头以上安一个牌子,写著他的罪状,说:「这是犹太人的王耶稣。」
\par 38 当时,有两个强盗和他同钉十字架,一个在右边,一个在左边。
\par 39 从那里经过的人讥诮他,摇著头,说:
\par 40 「你这拆毁圣殿、三日又建造起来的,可以救自己吧!你如果是神的儿子,就从十字架上下来吧!」
\par 41 祭司长和文士并长老也是这样戏弄他,说:
\par 42 「他救了别人,不能救自己。他是以色列的王,现在可以从十字架上下来,我们就信他。
\par 43 他倚靠神,神若喜悦他,现在可以救他;因为他曾说:『我是神的儿子。』」
\par 44 那和他同钉的强盗也是这样的讥诮他。
\par 45 从午正到申初,遍地都黑暗了。
\par 46 约在申初,耶稣大声喊著说:「以利!以利!拉马撒巴各大尼?」就是说:「我的神!我的神!为什麽离弃我?」
\par 47 站在那里的人,有的听见就说:「这个人呼叫以利亚呢!」
\par 48 内中有一个人赶紧跑去,拿海绒蘸满了醋,绑在苇子上,送给他喝。
\par 49 其余的人说:「且等著,看以利亚来救他不来。」
\par 50 耶稣又大声喊叫,气就断了。
\par 51 忽然,殿里的幔子从上到下裂为两半,地也震动,磐石也崩裂,
\par 52 坟墓也开了,已睡圣徒的身体多有起来的。
\par 53 到耶稣复活以後,他们从坟墓里出来,进了圣城,向许多人显现。
\par 54 百夫长和一同看守耶稣的人看见地震并所经历的事,就极其害怕,说:「这真是神的儿子了!」
\par 55 有好些妇女在那里,远远的观看;他们是从加利利跟随耶稣来服事他的。
\par 56 内中有抹大拉的马利亚,又有雅各和约西的母亲马利亚,并有西庇太两个儿子的母亲。
\par 57 到了晚上,有一个财主,名叫约瑟,是亚利马太来的,他也是耶稣的门徒。
\par 58 这人去见彼拉多,求耶稣的身体;彼拉多就吩咐给他。
\par 59 约瑟取了身体,用乾净细麻布裹好,
\par 60 安放在自己的新坟墓里,就是他凿在磐石里的。他又把大石头辊到墓门口,就去了。
\par 61 有抹大拉的马利亚和那个马利亚在那里,对著坟墓坐著。
\par 62 次日,就是预备日的第二天,祭司长和法利赛人聚集来见彼拉多,说:
\par 63 「大人,我们记得那诱惑人的还活著的时候曾说:『三日後我要复活。』
\par 64 因此,请吩咐人将坟墓把守妥当,直到第三日,恐怕他的门徒来,把他偷了去,就告诉百姓说:『他从死里复活了。』这样,那後来的迷惑比先前的更利害了!」
\par 65 彼拉多说:「你们有看守的兵,去吧!尽你们所能的把守妥当。」
\par 66 他们就带著看守的兵同去,封了石头,将坟墓把守妥当。

\chapter{28}

\par 1 安息日将尽,七日的头一日,天快亮的时候,抹大拉的马利亚和那个马利亚来看坟墓。
\par 2 忽然,地大震动;因为有主的使者从天上下来,把石头辊开,坐在上面。
\par 3 他的像貌如同闪电,衣服洁白如雪。
\par 4 看守的人就因他吓得浑身乱战,甚至和死人一样。
\par 5 天使对妇女说:「不要害怕!我知道你们是寻找那钉十字架的耶稣。
\par 6 他不在这里,照他所说的,已经复活了。你们来看安放主的地方。
\par 7 快去告诉他的门徒,说他从死里复活了,并且在你们以先往加利利去,在那里你们要见他。看哪,我已经告诉你们了。」
\par 8 妇女们就急忙离开坟墓,又害怕,又大大的欢喜,跑去要报给他的门徒。
\par 9 忽然,耶稣遇见他们,说:「愿你们平安!」他们就上前抱住他的脚拜他。
\par 10 耶稣对他们说:「不要害怕!你们去告诉我的弟兄,叫他们往加利利去,在那里必见我。」
\par 11 他们去的时候,看守的兵有几个进城去,将所经历的事都报给祭司长。
\par 12 祭司长和长老聚集商议,就拿许多银钱给兵丁,说:
\par 13 「你们要这样说:『夜间我们睡觉的时候,他的门徒来,把他偷去了。』
\par 14 倘若这话被巡抚听见,有我们劝他,保你们无事。」
\par 15 兵丁受了银钱,就照所嘱咐他们的去行。这话就传说在犹太人中间,直到今日。
\par 16 十一个门徒往加利利去,到了耶稣约定的山上。
\par 17 他们见了耶稣就拜他,然而还有人疑惑。
\par 18 耶稣进前来,对他们说:「天上地下所有的权柄都赐给我了。
\par 19 所以,你们要去,使万民作我的门徒,奉父、子、圣灵的名给他们施洗(或作:给他们施洗,归於父、子、圣灵的名)。
\par 20 凡我所吩咐你们的,都教训他们遵守,我就常与你们同在,直到世界的末了。」



\end{document}