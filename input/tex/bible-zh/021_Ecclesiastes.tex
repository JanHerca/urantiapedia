\begin{document}

\title{传道书}


\chapter{1}

\par 1 在耶路撒冷作王、大卫的儿子、传道者的言语。
\par 2 传道者说:虚空的虚空,虚空的虚空,凡事都是虚空。
\par 3 人一切的劳碌,就是他在日光之下的劳碌,有什麽益处呢?
\par 4 一代过去,一代又来,地却永远长存。
\par 5 日头出来,日头落下,急归所出之地。
\par 6 风往南刮,又向北转,不住的旋转,而且返回转行原道。
\par 7 江河都往海里流,海却不满;江河从何处流,仍归还何处。
\par 8 万事令人厌烦(或作:万物满有困乏),人不能说尽。眼看,看不饱;耳听,听不足。
\par 9 已有的事後必再有;已行的事後必再行。日光之下并无新事。
\par 10 岂有一件事人能指著说这是新的?那知,在我们以前的世代早已有了。
\par 11 已过的世代,无人记念;将来的世代,後来的人也不记念。
\par 12 我传道者在耶路撒冷作过以色列的王。
\par 13 我专心用智慧寻求、查究天下所做的一切事,乃知神叫世人所经练的是极重的劳苦。
\par 14 我见日光之下所做的一切事,都是虚空,都是捕风。
\par 15 弯曲的,不能变直;缺少的,不能足数。
\par 16 我心里议论说:我得了大智慧,胜过我以前在耶路撒冷的众人,而且我心中多经历智慧和知识的事。
\par 17 我又专心察明智慧、狂妄,和愚昧,乃知这也是捕风。
\par 18 因为多有智慧,就多有愁烦;加增知识的,就加增忧伤。

\chapter{2}

\par 1 我心里说:「来吧,我以喜乐试试你,你好享福!」谁知,这也是虚空。
\par 2 我指嬉笑说:「这是狂妄。」论喜乐说:「有何功效呢?」
\par 3 我心里察究,如何用酒使我肉体舒畅,我心却仍以智慧引导我;又如何持住愚昧,等我看明世人,在天下一生当行何事为美。
\par 4 我为自己动大工程,建造房屋,栽种葡萄园,
\par 5 修造园囿,在其中栽种各样果木树;
\par 6 挖造水池,用以浇灌嫩小的树木。
\par 7 我买了仆婢,也有生在家中的仆婢;又有许多牛群羊群,胜过以前在耶路撒冷众人所有的。
\par 8 我又为自己积蓄金银和君王的财宝,并各省的财宝;又得唱歌的男女和世人所喜爱的物,并许多的妃嫔。
\par 9 这样,我就日见昌盛,胜过以前在耶路撒冷的众人。我的智慧仍然存留。
\par 10 凡我眼所求的,我没有留下不给他的;我心所乐的,我没有禁止不享受的;因我的心为我一切所劳碌的快乐,这就是我从劳碌中所得的分。
\par 11 後来,我察看我手所经营的一切事和我劳碌所成的功。谁知都是虚空,都是捕风;在日光之下毫无益处。
\par 12 我转念观看智慧、狂妄,和愚昧。在王以後而来的人还能做什麽呢?也不过行早先所行的就是了。
\par 13 我便看出智慧胜过愚昧,如同光明胜过黑暗。
\par 14 智慧人的眼目光明(光明:原文作在他头上),愚昧人在黑暗里行。我却看明有一件事,这两等人都必遇见。
\par 15 我就心里说:「愚昧人所遇见的,我也必遇见,我为何更有智慧呢?」我心里说,这也是虚空。
\par 16 智慧人和愚昧人一样,永远无人记念,因为日後都被忘记;可叹智慧人死亡,与愚昧人无异。
\par 17 我所以恨恶生命;因为在日光之下所行的事我都以为烦恼,都是虚空,都是捕风。
\par 18 我恨恶一切的劳碌,就是我在日光之下的劳碌,因为我得来的必留给我以後的人。
\par 19 那人是智慧是愚昧,谁能知道?他竟要管理我劳碌所得的,就是我在日光之下用智慧所得的。这也是虚空。
\par 20 故此,我转想我在日光之下所劳碌的一切工作,心便绝望。
\par 21 因为有人用智慧、知识、灵巧所劳碌得来的,却要留给未曾劳碌的人为分。这也是虚空,也是大患。
\par 22 人在日光之下劳碌累心,在他一切的劳碌上得著什麽呢?
\par 23 因为他日日忧虑,他的劳苦成为愁烦,连夜间心也不安。这也是虚空。
\par 24 人莫强如吃喝,且在劳碌中享福,我看这也是出於神的手。
\par 25 论到吃用、享福,谁能胜过我呢?
\par 26 神喜悦谁,就给谁智慧、知识,和喜乐;惟有罪人,神使他劳苦,叫他将所收聚的、所堆积的归给神所喜悦的人。这也是虚空,也是捕风。

\chapter{3}

\par 1 凡事都有定期,天下万务都有定时。
\par 2 生有时,死有时;栽种有时,拔出所栽种的也有时;
\par 3 杀戮有时,医治有时;拆毁有时,建造有时;
\par 4 哭有时,笑有时;哀恸有时,跳舞有时;
\par 5 抛掷石头有时,堆聚石头有时;怀抱有时,不怀抱有时;
\par 6 寻找有时,失落有时;保守有时,舍弃有时;
\par 7 撕裂有时,缝补有时;静默有时,言语有时;
\par 8 喜爱有时,恨恶有时;争战有时,和好有时。
\par 9 这样看来,做事的人在他的劳碌上有什麽益处呢?
\par 10 我见神叫世人劳苦,使他们在其中受经练。
\par 11 神造万物,各按其时成为美好,又将永生(原文作永远)安置在世人心里。然而神从始至终的作为,人不能参透。
\par 12 我知道世人,莫强如终身喜乐行善;
\par 13 并且人人吃喝,在他一切劳碌中享福,这也是神的恩赐。
\par 14 我知道神一切所做的都必永存;无所增添,无所减少。神这样行,是要人在他面前存敬畏的心。
\par 15 现今的事早先就有了,将来的事早已也有了,并且神使已过的事重新再来(或作:并且神再寻回已过的事)。
\par 16 我又见日光之下,在审判之处有奸恶,在公义之处也有奸恶。
\par 17 我心里说,神必审判义人和恶人;因为在那里,各样事务,一切工作,都有定时。
\par 18 我心里说,这乃为世人的缘故,是神要试验他们,使他们觉得自己不过像兽一样。
\par 19 因为世人遭遇的,兽也遭遇,所遭遇的都是一样:这个怎样死,那个也怎样死,气息都是一样。人不能强於兽,都是虚空。
\par 20 都归一处,都是出於尘土,也都归於尘土。
\par 21 谁知道人的灵是往上升,兽的魂是下入地呢?
\par 22 故此,我见人莫强如在他经营的事上喜乐,因为这是他的分。他身後的事谁能使他回来得见呢?

\chapter{4}

\par 1 我又转念,见日光之下所行的一切欺压。看哪,受欺压的流泪,且无人安慰;欺压他们的有势力,也无人安慰他们。
\par 2 因此,我赞叹那早已死的死人,胜过那还活著的活人。
\par 3 并且我以为那未曾生的,就是未见过日光之下恶事的,比这两等人更强。
\par 4 我又见人为一切的劳碌和各样灵巧的工作就被邻舍嫉妒。这也是虚空,也是捕风。
\par 5 愚昧人抱著手,吃自己的肉。
\par 6 满了一把,得享安静,强如满了两把,劳碌捕风。
\par 7 我又转念,见日光之下有一件虚空的事:
\par 8 有人孤单无二,无子无兄,竟劳碌不息,眼目也不以钱财为足。他说:「我劳劳碌碌,刻苦自己,不享福乐,到底是为谁呢?」这也是虚空,是极重的劳苦。
\par 9 两个人总比一个人好,因为二人劳碌同得美好的果效。
\par 10 若是跌倒,这人可以扶起他的同伴;若是孤身跌倒,没有别人扶起他来,这人就有祸了。
\par 11 再者,二人同睡就都暖和,一人独睡怎能暖和呢?
\par 12 有人攻胜孤身一人,若有二人便能敌挡他;三股合成的绳子不容易折断。
\par 13 贫穷而有智慧的少年人胜过年老不肯纳谏的愚昧王。
\par 14 这人是从监牢中出来作王,在他国中,生来原是贫穷的。
\par 15 我见日光之下一切行动的活人都随从那第二位,就是起来代替老王的少年人。
\par 16 他所治理的众人就是他的百姓,多得无数;在他後来的人尚且不喜悦他。这真是虚空,也是捕风。

\chapter{5}

\par 1 你到神的殿要谨慎脚步;因为近前听,胜过愚昧人献祭(或作:胜过献愚昧人的祭),他们本不知道所做的是恶。
\par 2 你在神面前不可冒失开口,也不可心急发言;因为神在天上,你在地下,所以你的言语要寡少。
\par 3 事务多,就令人做梦;言语多,就显出愚昧。
\par 4 你向神许愿,偿还不可迟延,因他不喜悦愚昧人,所以你许的愿应当偿还。
\par 5 你许愿不还,不如不许。
\par 6 不可任你的口使肉体犯罪,也不可在祭司(原文作使者)面前说是错许了。为何使神因你的声音发怒,败坏你手所做的呢?
\par 7 多梦和多言,其中多有虚幻,你只要敬畏神。
\par 8 你若在一省之中见穷人受欺压,并夺去公义公平的事,不要因此诧异;因有一位高过居高位的鉴察,在他们以上还有更高的。
\par 9 况且地的益处归众人,就是君王也受田地的供应。
\par 10 贪爱银子的,不因得银子知足;贪爱丰富的,也不因得利益知足。这也是虚空。
\par 11 货物增添,吃的人也增添,物主得什麽益处呢?不过眼看而已!
\par 12 劳碌的人不拘吃多吃少,睡得香甜;富足人的丰满却不容他睡觉。
\par 13 我见日光之下有一宗大祸患,就是财主积存资财,反害自己。
\par 14 因遭遇祸患,这些资财就消灭;那人若生了儿子,手里也一无所有。
\par 15 他怎样从母胎赤身而来,也必照样赤身而去;他所劳碌得来的,手中分毫不能带去。
\par 16 他来的情形怎样,他去的情形也怎样。这也是一宗大祸患。他为风劳碌有什麽益处呢?
\par 17 并且他终身在黑暗中吃喝,多有烦恼,又有病患呕气。
\par 18 我所见为善为美的,就是人在神赐他一生的日子吃喝,享受日光之下劳碌得来的好处,因为这是他的分。
\par 19 神赐人资财丰富,使他能以吃用,能取自己的分,在他劳碌中喜乐,这乃是神的恩赐。
\par 20 他不多思念自己一生的年日,因为神应他的心使他喜乐。

\chapter{6}

\par 1 我见日光之下有一宗祸患重压在人身上,
\par 2 就是人蒙神赐他资财、丰富、尊荣,以致他心里所愿的一样都不缺,只是神使他不能吃用,反有外人来吃用。这是虚空,也是祸患。
\par 3 人若生一百个儿子,活许多岁数,以致他的年日甚多,心里却不得满享福乐,又不得埋葬;据我说,那不到期而落的胎比他倒好。
\par 4 因为虚虚而来,暗暗而去,名字被黑暗遮蔽,
\par 5 并且没有见过天日,也毫无知觉;这胎,比那人倒享安息。
\par 6 那人虽然活千年,再活千年,却不享福,众人岂不都归一个地方去吗?
\par 7 人的劳碌都为口腹,心里却不知足。
\par 8 这样看来,智慧人比愚昧人有什麽长处呢?穷人在众人面前知道如何行,有什麽长处呢?
\par 9 眼睛所看的比心里妄想的倒好。这也是虚空,也是捕风。
\par 10 先前所有的,早已起了名,并知道何为人,他也不能与那比自己力大的相争。
\par 11 加增虚浮的事既多,这与人有什麽益处呢?
\par 12 人一生虚度的日子,就如影儿经过,谁知道什麽与他有益呢?谁能告诉他身後在日光之下有什麽事呢?

\chapter{7}

\par 1 名誉强如美好的膏油;人死的日子胜过人生的日子。
\par 2 往遭丧的家去,强如往宴乐的家去;因为死是众人的结局,活人也必将这事放在心上。
\par 3 忧愁强如喜笑;因为面带愁容,终必使心喜乐。
\par 4 智慧人的心在遭丧之家;愚昧人的心在快乐之家。
\par 5 听智慧人的责备,强如听愚昧人的歌唱。
\par 6 愚昧人的笑声,好像锅下烧荆棘的爆声;这也是虚空。
\par 7 勒索使智慧人变为愚妄;贿赂能败坏人的慧心。
\par 8 事情的终局强如事情的起头;存心忍耐的,胜过居心骄傲的。
\par 9 你不要心里急躁恼怒,因为恼怒存在愚昧人的怀中。
\par 10 不要说:先前的日子强过如今的日子,是什麽缘故呢?你这样问,不是出於智慧。
\par 11 智慧和产业并好,而且见天日的人得智慧更为有益。
\par 12 因为智慧护庇人,好像银钱护庇人一样。惟独智慧能保全智慧人的生命。这就是知识的益处。
\par 13 你要察看神的作为;因神使为曲的,谁能变为直呢?
\par 14 遇亨通的日子你当喜乐;遭患难的日子你当思想;因为神使这两样并列,为的是叫人查不出身後有什麽事。
\par 15 有义人行义,反致灭亡;有恶人行恶,倒享长寿。这都是我在虚度之日中所见过的。
\par 16 不要行义过分,也不要过於自逞智慧,何必自取败亡呢?
\par 17 不要行恶过分,也不要为人愚昧,何必不到期而死呢?
\par 18 你持守这个为美,那个也不要松手;因为敬畏神的人,必从这两样出来。
\par 19 智慧使有智慧的人比城中十个官长更有能力。
\par 20 时常行善而不犯罪的义人,世上实在没有。
\par 21 人所说的一切话,你不要放在心上,恐怕听见你的仆人咒诅你。
\par 22 因为你心里知道,自己也曾屡次咒诅别人。
\par 23 我曾用智慧试验这一切事;我说,要得智慧,智慧却离我远。
\par 24 万事之理,离我甚远,而且最深,谁能测透呢?
\par 25 我转念,一心要知道,要考察,要寻求智慧和万事的理由;又要知道邪恶为愚昧,愚昧为狂妄。
\par 26 我得知有等妇人比死还苦:他的心是网罗,手是锁链。凡蒙神喜悦的人必能躲避他;有罪的人却被他缠住了。
\par 27 传道者说:「看哪,一千男子中,我找到一个正直人,但众女子中,没有找到一个。
\par 28 我将这事一一比较,要寻求其理,我心仍要寻找,却未曾找到。
\par 29 我所找到的只有一件,就是神造人原是正直,但他们寻出许多巧计。

\chapter{8}

\par 1 谁如智慧人呢?谁知道事情的解释呢?人的智慧使他的脸发光,并使他脸上的暴气改变。
\par 2 我劝你遵守王的命令;既指神起誓,理当如此。
\par 3 不要急躁离开王的面前,不要固执行恶,因为他凡事都随自己的心意而行。
\par 4 王的话本有权力,谁敢问他说「你做什麽」呢?
\par 5 凡遵守命令的,必不经历祸患;智慧人的心能辨明时候和定理(原文作审判;下节同)。
\par 6 各样事务成就都有时候和定理,因为人的苦难重压在他身上。
\par 7 他不知道将来的事,因为将来如何,谁能告诉他呢?
\par 8 无人有权力掌管生命,将生命留住;也无人有权力掌管死期;这场争战,无人能免;邪恶也不能救那好行邪恶的人。
\par 9 这一切我都见过,也专心查考日光之下所做的一切事。有时这人管辖那人,令人受害。
\par 10 我见恶人埋葬,归入坟墓;又见行正直事的离开圣地,在城中被人忘记。这也是虚空。
\par 11 因为断定罪名不立刻施刑,所以世人满心作恶。
\par 12 罪人虽然作恶百次,倒享长久的年日;然而我准知道,敬畏神的,就是在他面前敬畏的人,终久必得福乐。
\par 13 恶人却不得福乐,也不得长久的年日;这年日好像影儿,因他不敬畏神。
\par 14 世上有一件虚空的事,就是义人所遭遇的,反照恶人所行的;又有恶人所遭遇的,反照义人所行的。我说,这也是虚空。
\par 15 我就称赞快乐,原来人在日光之下,莫强如吃喝快乐;因为他在日光之下,神赐他一生的年日,要从劳碌中,时常享受所得的。
\par 16 我专心求智慧,要看世上所做的事。(有昼夜不睡觉不合眼的。)
\par 17 我就看明神一切的作为,知道人查不出日光之下所做的事;任凭他费多少力寻查,都查不出来,就是智慧人虽想知道,也是查不出来。

\chapter{9}

\par 1 我将这一切事放在心上,详细考究,就知道义人和智慧人,并他们的作为都在神手中;或是爱,或是恨,都在他们的前面,人不能知道。
\par 2 凡临到众人的事都是一样:义人和恶人都遭遇一样的事;好人,洁净人和不洁净人,献祭的与不献祭的,也是一样。好人如何,罪人也如何;起誓的如何,怕起誓的也如何。
\par 3 在日光之下所行的一切事上有一件祸患,就是众人所遭遇的都是一样,并且世人的心充满了恶;活著的时候心里狂妄,後来就归死人那里去了。
\par 4 与一切活人相连的,那人还有指望,因为活著的狗比死了的狮子更强。
\par 5 活著的人知道必死;死了的人毫无所知,也不再得赏赐;他们的名无人记念。
\par 6 他们的爱,他们的恨,他们的嫉妒,早都消灭了。在日光之下所行的一切事上,他们永不再有分了。
\par 7 你只管去欢欢喜喜吃你的饭,心中快乐喝你的酒,因为神已经悦纳你的作为。
\par 8 你的衣服当时常洁白,你头上也不要缺少膏油。
\par 9 在你一生虚空的年日,就是神赐你在日光之下虚空的年日,当同你所爱的妻,快活度日,因为那是你生前在日光之下劳碌的事上所得的分。
\par 10 凡你手所当做的事要尽力去做;因为在你所必去的阴间没有工作,没有谋算,没有知识,也没有智慧。
\par 11 我又转念:见日光之下,快跑的未必能赢;力战的未必得胜;智慧的未必得粮食;明哲的未必得资财;灵巧的未必得喜悦。所临到众人的是在乎当时的机会。
\par 12 原来人也不知道自己的定期。鱼被恶网圈住,鸟被网罗捉住,祸患忽然临到的时候,世人陷在其中也是如此。
\par 13 我见日光之下有一样智慧,据我看乃是广大,
\par 14 就是有一小城,其中的人数稀少,有大君王来攻击,修筑营垒,将城围困。
\par 15 城中有一个贫穷的智慧人,他用智慧救了那城,却没有人记念那穷人。
\par 16 我就说,智慧胜过勇力;然而那贫穷人的智慧被人藐视,他的话也无人听从。
\par 17 宁可在安静之中听智慧人的言语,不听掌管愚昧人的喊声。
\par 18 智慧胜过打仗的兵器;但一个罪人能败坏许多善事。

\chapter{10}

\par 1 死苍蝇使做香的膏油发出臭气;这样,一点愚昧也能败坏智慧和尊荣。
\par 2 智慧人的心居右;愚昧人的心居左。
\par 3 并且愚昧人行路显出无知,对众人说,他是愚昧人。
\par 4 掌权者的心若向你发怒,不要离开你的本位,因为柔和能免大过。
\par 5 我见日光之下有一件祸患,似乎出於掌权的错误,
\par 6 就是愚昧人立在高位;富足人坐在低位。
\par 7 我见过仆人骑马,王子像仆人在地上步行。
\par 8 挖陷坑的,自己必掉在其中;拆墙垣的,必为蛇所咬。
\par 9 凿开(或作:挪移)石头的,必受损伤;劈开木头的,必遭危险。
\par 10 铁器钝了,若不将刃磨快,就必多费气力;但得智慧指教,便有益处。
\par 11 未行法术以先,蛇若咬人,後行法术也是无益。
\par 12 智慧人的口说出恩言;愚昧人的嘴吞灭自己。
\par 13 他口中的言语起头是愚昧;他话的末尾是奸恶的狂妄。
\par 14 愚昧人多有言语,人却不知将来有什麽事;他身後的事谁能告诉他呢?
\par 15 凡愚昧人,他的劳碌使自己困乏,因为连进城的路,他也不知道。
\par 16 邦国啊,你的王若是孩童,你的群臣早晨宴乐,你就有祸了!
\par 17 邦国啊,你的王若是贵胄之子,你的群臣按时吃喝,为要补力,不为酒醉,你就有福了!
\par 18 因人懒惰,房顶塌下;因人手懒,房屋滴漏。
\par 19 设摆筵席是为喜笑。酒能使人快活;钱能叫万事应心。
\par 20 你不可咒诅君王,也不可心怀此念;在你卧房也不可咒诅富户。因为空中的鸟必传扬这声音,有翅膀的也必述说这事。

\chapter{11}

\par 1 当将你的粮食撒在水面,因为日久必能得著。
\par 2 你要分给七人,或分给八人,因为你不知道将来有什麽灾祸临到地上。
\par 3 云若满了雨,就必倾倒在地上。树若向南倒,或向北倒,树倒在何处,就存在何处。
\par 4 看风的,必不撒种;望云的,必不收割。
\par 5 风从何道来,骨头在怀孕妇人的胎中如何长成,你尚且不得知道;这样,行万事之神的作为,你更不得知道。
\par 6 早晨要撒你的种,晚上也不要歇你的手,因为你不知道哪一样发旺;或是早撒的,或是晚撒的,或是两样都好。
\par 7 光本是佳美的,眼见日光也是可悦的。
\par 8 人活多年,就当快乐多年;然而也当想到黑暗的日子。因为这日子必多,所要来的都是虚空。
\par 9 少年人哪,你在幼年时当快乐。在幼年的日子,使你的心欢畅,行你心所愿行的,看你眼所爱看的;却要知道,为这一切的事,神必审问你。
\par 10 所以,你当从心中除掉愁烦,从肉体克去邪恶;因为一生的开端和幼年之时,都是虚空的。

\chapter{12}

\par 1 你趁著年幼、衰败的日子尚未来到,就是你所说,我毫无喜乐的那些年日未曾临近之先,当记念造你的主。
\par 2 不要等到日头、光明、月亮、星宿变为黑暗,雨後云彩反回,
\par 3 看守房屋的发颤,有力的屈身,推磨的稀少就止息,从窗户往外看的都昏暗;
\par 4 街门关闭,推磨的响声微小,雀鸟一叫,人就起来,唱歌的女子也都衰微。
\par 5 人怕高处,路上有惊慌,杏树开花,蚱蜢成为重担,人所愿的也都废掉;因为人归他永远的家,吊丧的在街上往来。
\par 6 银链折断,金罐破裂,瓶子在泉旁损坏,水轮在井口破烂,
\par 7 尘土仍归於地,灵仍归於赐灵的神。
\par 8 传道者说:「虚空的虚空,凡事都是虚空。」
\par 9 再者,传道者因有智慧,仍将知识教训众人;又默想,又考查,又陈说许多箴言。
\par 10 传道者专心寻求可喜悦的言语,是凭正直写的诚实话。
\par 11 智慧人的言语好像刺棍;会中之师的言语又像钉稳的钉子,都是一个牧者所赐的。
\par 12 我儿,还有一层,你当受劝戒:著书多,没有穷尽;读书多,身体疲倦。
\par 13 这些事都已听见了,总意就是:敬畏神,谨守他的诫命,这是人所当尽的本分(或作:这是众人的本分)。
\par 14 因为人所做的事,连一切隐藏的事,无论是善是恶,神都必审问。


\end{document}