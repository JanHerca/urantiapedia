\begin{document}

\title{以斯帖记}


\chapter{1}

\par 1 亚哈随鲁作王,从印度直到古实,统管一百二十七省。
\par 2 亚哈随鲁王在书珊城的宫登基;
\par 3 在位第三年,为他一切首领臣仆设摆筵席,有波斯和玛代的权贵,就是各省的贵胄与首领,在他面前。
\par 4 他把他荣耀之国的丰富和他美好威严的尊贵给他们看了许多日,就是一百八十日。
\par 5 这日子满了,又为所有住书珊城的大小人民在御园的院子里设摆筵席七日。
\par 6 有白色、绿色、蓝色的帐子,用细麻绳、紫色绳从银环内系在白玉石柱上;有金银的床榻摆在红、白、黄、黑玉石铺的石地上。
\par 7 用金器皿赐酒,器皿各有不同。御酒甚多,足显王的厚意。
\par 8 喝酒有例,不准勉强人,因王吩咐宫里的一切臣宰,让人各随己意。
\par 9 王后瓦实提在亚哈随鲁王的宫内也为妇女设摆筵席。
\par 10 第七日,亚哈随鲁王饮酒,心中快乐,就吩咐在他面前侍立的七个太监米户幔、比斯他、哈波拿、比革他、亚拔他、西达、甲迦,
\par 11 请王后瓦实提头戴王后的冠冕到王面前,使各等臣民看他的美貌,因为他容貌甚美。
\par 12 王后瓦实提却不肯遵太监所传的王命而来,所以王甚发怒,心如火烧。
\par 13 那时,在王左右常见王面、国中坐高位的,有波斯和玛代的七个大臣,就是甲示拿、示达、押玛他、他施斯、米力、玛西拿、米母干,都是达时务的明哲人。
\par 14 按王的常规,办事必先询问知例明法的人。王问他们说:
\par 15 「王后瓦实提不遵太监所传的王命,照例应当怎样办理呢?」
\par 16 米母干在王和众首领面前回答说:「王后瓦实提这事,不但得罪王,并且有害於王各省的臣民;
\par 17 因为王后这事必传到众妇人的耳中,说:『亚哈随鲁王吩咐王后瓦实提到王面前,他却不来』,他们就藐视自己的丈夫。
\par 18 今日波斯和玛代的众夫人听见王后这事,必向王的大臣照样行;从此必大开藐视和忿怒之端。
\par 19 王若以为美,就降旨写在波斯和玛代人的例中,永不更改,不准瓦实提再到王面前,将他王后的位分赐给比他还好的人。
\par 20 所降的旨意传遍通国(国度本来广大),所有的妇人,无论丈夫贵贱都必尊敬他。」
\par 21 王和众首领都以米母干的话为美,王就照这话去行,
\par 22 发诏书,用各省的文字、各族的方言通知各省,使为丈夫的在家中作主,各说本地的方言。

\chapter{2}

\par 1 这事以後,亚哈随鲁王的忿怒止息,就想念瓦实提和他所行的,并怎样降旨办他。
\par 2 於是王的侍臣对王说:「不如为王寻找美貌的处女。
\par 3 王可以派官在国中的各省招聚美貌的处女到书珊城(或作:宫)的女院,交给掌管女子的太监希该,给他们当用的香品。
\par 4 王所喜爱的女子可以立为王后,代替瓦实提。」王以这事为美,就如此行。
\par 5 书珊城有一个犹大人,名叫末底改,是便雅悯人基士的曾孙,示每的孙子,睚珥的儿子。
\par 6 从前巴比伦王尼布甲尼撒将犹大王耶哥尼雅(又名约雅斤)和百姓从耶路撒冷掳去,末底改也在其内。
\par 7 末底改抚养他叔叔的女儿哈大沙(後名以斯帖),因为他没有父母。这女子又容貌俊美;他父母死了,末底改就收他为自己的女儿。
\par 8 王的谕旨传出,就招聚许多女子到书珊城,交给掌管女子的希该;以斯帖也送入王宫,交付希该。
\par 9 希该喜悦以斯帖,就恩待他,急忙给他需用的香品和他所当得的分,又派所当得的七个宫女服事他,使他和他的宫女搬入女院上好的房屋。
\par 10 以斯帖未曾将籍贯宗族告诉人,因为末底改嘱咐他不可叫人知道。
\par 11 末底改天天在女院前边行走,要知道以斯帖平安不平安,并後事如何。
\par 12 众女子照例先洁净身体十二个月:六个月用没药油,六个月用香料和洁身之物。满了日期,然後挨次进去见亚哈随鲁王。
\par 13 女子进去见王是这样:从女院到王宫的时候,凡他所要的都必给他。
\par 14 晚上进去,次日回到女子第二院,交给掌管妃嫔的太监沙甲;除非王喜爱他,再提名召他,就不再进去见王。
\par 15 末底改叔叔亚比孩的女儿,就是末底改收为自己女儿的以斯帖,按次序当进去见王的时候,除了掌管女子的太监希该所派定给他的,他别无所求。凡看见以斯帖的都喜悦他。
\par 16 亚哈随鲁王第七年十月,就是提别月,以斯帖被引入宫见王。
\par 17 王爱以斯帖过於爱众女,他在王眼前蒙宠爱比众处女更甚。王就把王后的冠冕戴在他头上,立他为王后,代替瓦实提。
\par 18 王因以斯帖的缘故给众首领和臣仆设摆大筵席,又豁免各省的租税,并照王的厚意大颁赏赐。
\par 19 第二次招聚处女的时候,末底改坐在朝门。
\par 20 以斯帖照著末底改所嘱咐的,还没有将籍贯宗族告诉人;因为以斯帖遵末底改的命,如抚养他的时候一样。
\par 21 当那时候,末底改坐在朝门,王的太监中有两个守门的,辟探和提列,恼恨亚哈随鲁王,想要下手害他。
\par 22 末底改知道了,就告诉王后以斯帖。以斯帖奉末底改的名,报告於王;
\par 23 究察这事,果然是实,就把二人挂在木头上,将这事在王面前写於历史上。

\chapter{3}

\par 1 这事以後,亚哈随鲁王抬举亚甲族哈米大他的儿子哈曼,使他高升,叫他的爵位超过与他同事的一切臣宰。
\par 2 在朝门的一切臣仆都跪拜哈曼,因为王如此吩咐;惟独末底改不跪不拜。
\par 3 在朝门的臣仆问末底改说:「你为何违背王的命令呢?」
\par 4 他们天天劝他,他还是不听,他们就告诉哈曼,要看末底改的事站得住站不住,因他已经告诉他们自己是犹大人。
\par 5 哈曼见末底改不跪不拜,他就怒气填胸。
\par 6 他们已将末底改的本族告诉哈曼;他以为下手害末底改一人是小事,就要灭绝亚哈随鲁王通国所有的犹大人,就是末底改的本族。
\par 7 亚哈随鲁王十二年正月,就是尼散月,人在哈曼面前,按日日月月掣普珥,就是掣签,要定何月何日为吉,择定了十二月,就是亚达月。
\par 8 哈曼对亚哈随鲁王说:「有一种民散居在王国各省的民中;他们的律例与万民的律例不同,也不守王的律例,所以容留他们与王无益。
\par 9 王若以为美,请下旨意灭绝他们;我就捐一万他连得银子交给掌管国帑的人,纳入王的府库。」
\par 10 於是王从自己手上摘下戒指给犹大人的仇敌亚甲族哈米大他的儿子哈曼。
\par 11 王对哈曼说:「这银子仍赐给你,这民也交给你,你可以随意待他们。」
\par 12 正月十三日,就召了王的书记来,照著哈曼一切所吩咐的,用各省的文字、各族的方言,奉亚哈随鲁王的名写旨意,传与总督和各省的省长,并各族的首领;又用王的戒指盖印,
\par 13 交给驿卒传到王的各省,吩咐将犹大人,无论老少妇女孩子,在一日之间,十二月,就是亚达月十三日,全然剪除,杀戮灭绝,并夺他们的财为掠物。
\par 14 抄录这旨意,颁行各省,宣告各族,使他们预备等候那日。
\par 15 驿卒奉王命急忙起行,旨意也传遍书珊城。王同哈曼坐下饮酒,书珊城的民却都慌乱。

\chapter{4}

\par 1 末底改知道所做的这一切事,就撕裂衣服,穿麻衣,蒙灰尘,在城中行走,痛哭哀号。
\par 2 到了朝门前停住脚步,因为穿麻衣的不可进朝门。
\par 3 王的谕旨所到的各省各处,犹大人大大悲哀,禁食哭泣哀号,穿麻衣躺在灰中的甚多。
\par 4 王后以斯帖的宫女和太监来把这事告诉以斯帖,他甚是忧愁,就送衣服给末底改穿,要他脱下麻衣,他却不受。
\par 5 以斯帖就把王所派伺候他的一个太监,名叫哈他革召来,吩咐他去见末底改,要知道这是什麽事,是什麽缘故。
\par 6 於是哈他革出到朝门前的宽阔处见末底改。
\par 7 末底改将自己所遇的事,并哈曼为灭绝犹大人应许捐入王库的银数都告诉了他;
\par 8 又将所抄写传遍书珊城要灭绝犹大人的旨意交给哈他革,要给以斯帖看,又要给他说明,并嘱咐他进去见王,为本族的人在王面前恳切祈求。
\par 9 哈他革回来,将末底改的话告诉以斯帖;
\par 10 以斯帖就吩咐哈他革去见末底改,说:
\par 11 「王的一切臣仆和各省的人民都知道有一个定例:若不蒙召,擅入内院见王的,无论男女必被治死;除非王向他伸出金杖,不得存活。现在我没有蒙召进去见王已经三十日了。」
\par 12 人就把以斯帖这话告诉末底改。
\par 13 末底改托人回覆以斯帖说:「你莫想在王宫里强过一切犹大人,得免这祸。
\par 14 此时你若闭口不言,犹大人必从别处得解脱,蒙拯救;你和你父家必至灭亡。焉知你得了王后的位分不是为现今的机会吗?」
\par 15 以斯帖就吩咐人回报末底改说:
\par 16 「你当去招聚书珊城所有的犹大人,为我禁食三昼三夜,不吃不喝;我和我的宫女也要这样禁食。然後我违例进去见王,我若死就死吧!」
\par 17 於是末底改照以斯帖一切所吩咐的去行。

\chapter{5}

\par 1 第三日,以斯帖穿上朝服,进王宫的内院,对殿站立。王在殿里坐在宝座上,对著殿门。
\par 2 王见王后以斯帖站在院内,就施恩於他,向他伸出手中的金杖;以斯帖便向前摸杖头。
\par 3 王对他说:「王后以斯帖啊,你要什麽?你求什麽,就是国的一半也必赐给你。」
\par 4 以斯帖说:「王若以为美,就请王带著哈曼今日赴我所预备的筵席。」
\par 5 王说:「叫哈曼速速照以斯帖的话去行。」於是王带著哈曼赴以斯帖所预备的筵席。
\par 6 在酒席筵前,王又问以斯帖说:「你要什麽,我必赐给你;你求什麽,就是国的一半也必为你成就。」
\par 7 以斯帖回答说:「我有所要,我有所求。
\par 8 我若在王眼前蒙恩,王若愿意赐我所要的,准我所求的,就请王带著哈曼再赴我所要预备的筵席。明日我必照王所问的说明。」
\par 9 那日哈曼心中快乐,欢欢喜喜地出来;但见末底改在朝门不站起来,连身也不动,就满心恼怒末底改。
\par 10 哈曼暂且忍耐回家,叫人请他朋友和他妻子细利斯来。
\par 11 哈曼将他富厚的荣耀、众多的儿女,和王抬举他使他超乎首领臣仆之上,都述说给他们听。
\par 12 哈曼又说:「王后以斯帖预备筵席,除了我之外不许别人随王赴席。明日王后又请我随王赴席;
\par 13 只是我见犹大人末底改坐在朝门,虽有这一切荣耀,也与我无益。」
\par 14 他的妻细利斯和他一切的朋友对他说:「不如立一个五丈高的木架,明早求王将末底改挂在其上,然後你可以欢欢喜喜地随王赴席。」哈曼以这话为美,就叫人做了木架。

\chapter{6}

\par 1 那夜王睡不著觉,就吩咐人取历史来,念给他听。
\par 2 正遇见书上写著说:王的太监中有两个守门的,辟探和提列,想要下手害亚哈随鲁王,末底改将这事告诉王后。
\par 3 王说:「末底改行了这事,赐他什麽尊荣爵位没有?」伺候王的臣仆回答说:「没有赐他什麽。」
\par 4 王说:「谁在院子里?」(那时哈曼正进王宫的外院,要求王将末底改挂在他所预备的木架上。)
\par 5 臣仆说:「哈曼站在院内。」王说:「叫他进来。」
\par 6 哈曼就进去。王问他说:「王所喜悦尊荣的人,当如何待他呢?」哈曼心里说:「王所喜悦尊荣的,不是我是谁呢?」
\par 7 哈曼就回答说:「王所喜悦尊荣的人,
\par 8 当将王常穿的朝服和戴冠的御马,
\par 9 都交给王极尊贵的一个大臣,命他将衣服给王所喜悦尊荣的人穿上,使他骑上马,走遍城里的街市,在他面前宣告说:王所喜悦尊荣的人,就如此待他。」
\par 10 王对哈曼说:「你速速将这衣服和马,照你所说的,向坐在朝门的犹大人末底改去行。凡你所说的,一样不可缺。」
\par 11 於是哈曼将朝服给末底改穿上,使他骑上马,走遍城里的街市,在他面前宣告说:「王所喜悦尊荣的人,就如此待他。」
\par 12 末底改仍回到朝门,哈曼却忧忧闷闷地蒙著头,急忙回家去了,
\par 13 将所遇的一切事详细说给他的妻细利斯和他的众朋友听。他的智慧人和他的妻细利斯对他说:「你在末底改面前始而败落,他如果是犹大人,你必不能胜他,终必在他面前败落。」
\par 14 他们还与哈曼说话的时候,王的太监来催哈曼快去赴以斯帖所预备的筵席。

\chapter{7}

\par 1 王带著哈曼来赴王后以斯帖的筵席。
\par 2 这第二次在酒席筵前,王又问以斯帖说:「王后以斯帖啊,你要什麽,我必赐给你;你求什麽,就是国的一半也必为你成就。」
\par 3 王后以斯帖回答说:「我若在王眼前蒙恩,王若以为美,我所愿的,是愿王将我的性命赐给我;我所求的,是求王将我的本族赐给我。
\par 4 因我和我的本族被卖了,要剪除杀戮灭绝我们。我们若被卖为奴为婢,我也闭口不言;但王的损失,敌人万不能补足。」
\par 5 亚哈随鲁王问王后以斯帖说:「擅敢起意如此行的是谁?这人在那里呢?」
\par 6 以斯帖说:「仇人敌人就是这恶人哈曼!」哈曼在王和王后面前就甚惊惶。
\par 7 王便大怒,起来离开酒席往御园去了。哈曼见王定意要加罪与他,就起来,求王后以斯帖救命。
\par 8 王从御园回到酒席之处,见哈曼伏在以斯帖所靠的榻上;王说:「他竟敢在宫内、在我面前凌辱王后吗?」这话一出王口,人就蒙了哈曼的脸。
\par 9 伺候王的一个太监名叫哈波拿,说:「哈曼为那救王有功的末底改做了五丈高的木架,现今立在哈曼家里。」王说:「把哈曼挂在其上。」
\par 10 於是人将哈曼挂在他为末底改所预备的木架上。」王的忿怒这才止息。

\chapter{8}

\par 1 当日,亚哈随鲁王把犹大人仇敌哈曼的家产赐给王后以斯帖。末底改也来到王面前,因为以斯帖已经告诉王,末底改是他的亲属。
\par 2 王摘下自己的戒指,就是从哈曼追回的,给了末底改。以斯帖派末底改管理哈曼的家产。
\par 3 以斯帖又俯伏在王脚前,流泪哀告,求他除掉亚甲族哈曼害犹大人的恶谋。
\par 4 王向以斯帖伸出金杖;以斯帖就起来,站在王前,
\par 5 说:「亚甲族哈米大他的儿子哈曼设谋传旨,要杀灭在王各省的犹大人。现今王若愿意,我若在王眼前蒙恩,王若以为美,若喜悦我,请王另下旨意,废除哈曼所传的那旨意。
\par 6 我何忍见我本族的人受害?何忍见我同宗的人被灭呢?」
\par 7 亚哈随鲁王对王后以斯帖和犹大人末底改说:「因哈曼要下手害犹大人,我已将他的家产赐给以斯帖,人也将哈曼挂在木架上。
\par 8 现在你们可以随意奉王的名写谕旨给犹大人,用王的戒指盖印;因为奉王名所写、用王戒指盖印的谕旨,人都不能废除。」
\par 9 三月,就是西弯月二十三日,将王的书记召来,按著末底改所吩咐的,用各省的文字、各族的方言,并犹大人的文字方言写谕旨,传给那从印度直到古实一百二十七省的犹大人和总督省长首领。
\par 10 末底改奉亚哈随鲁王的名写谕旨,用王的戒指盖印,交给骑御马圈快马的驿卒,传到各处。
\par 11 谕旨中,王准各省各城的犹大人在一日之间,十二月,就是亚达月十三日,聚集保护性命,
\par 12 剪除杀戮灭绝那要攻击犹大人的一切仇敌和他们的妻子儿女,夺取他们的财为掠物。
\par 13 抄录这谕旨,颁行各省,宣告各族,使犹大人预备等候那日,在仇敌身上报仇。
\par 14 於是骑快马的驿卒被王命催促,急忙起行;谕旨也传遍书珊城。
\par 15 末底改穿著蓝色白色的朝服,头戴大金冠冕,又穿紫色细麻布的外袍,从王面前出来;书珊城的人民都欢呼快乐。
\par 16 犹大人有光荣,欢喜快乐而得尊贵。
\par 17 王的谕旨所到的各省各城,犹大人都欢喜快乐,设摆筵宴,以那日为吉日。那国的人民,有许多因惧怕犹大人,就入了犹大籍。

\chapter{9}

\par 1 十二月,乃亚达月十三日,王的谕旨将要举行,就是犹大人的仇敌盼望辖制他们的日子,犹大人反倒辖制恨他们的人。
\par 2 犹大人在亚哈随鲁王各省的城里聚集,下手击杀那要害他们的人。无人能敌挡他们,因为各族都惧怕他们。
\par 3 各省的首领、总督、省长,和办理王事的人,因惧怕末底改,就都帮助犹大人。
\par 4 末底改在朝中为大,名声传遍各省,日渐昌盛。
\par 5 犹大人用刀击杀一切仇敌,任意杀灭恨他们的人。
\par 6 在书珊城,犹大人杀灭了五百人;
\par 7 又杀巴珊大他、达分、亚斯帕他、
\par 8 破拉他、亚大利雅、亚利大他、
\par 9 帕玛斯他、亚利赛、亚利代、瓦耶撒他;
\par 10 这十人都是哈米大他的孙子、犹大人仇敌哈曼的儿子。犹大人却没有下手夺取财物。
\par 11 当日,将书珊城被杀的人数呈在王前。
\par 12 王对王后以斯帖说:「犹大人在书珊城杀灭了五百人,又杀了哈曼的十个儿子,在王的各省不知如何呢?现在你要什麽,我必赐给你;你还求什麽,也必为你成就。」
\par 13 以斯帖说:「王若以为美,求你准书珊的犹大人,明日也照今日的旨意行,并将哈曼十个儿子的尸首挂在木架上。」
\par 14 王便允准如此行。旨意传在书珊,人就把哈曼十个儿子的尸首挂起来了。
\par 15 亚达月十四日书珊的犹大人又聚集在书珊,杀了三百人,却没有下手夺取财物。
\par 16 在王各省其余的犹大人也都聚集保护性命,杀了恨他们的人七万五千,却没有下手夺取财物。这样,就脱离仇敌,得享平安。
\par 17 亚达月十三日,行了这事;十四日安息,以这日为设筵欢乐的日子。
\par 18 但书珊的犹大人,这十三日、十四日聚集杀戮仇敌;十五日安息,以这日为设筵欢乐的日子。
\par 19 所以住无城墙乡村的犹大人,如今都以亚达月十四日为设筵欢乐的吉日,彼此馈送礼物。
\par 20 末底改记录这事,写信与亚哈随鲁王各省远近所有的犹大人,
\par 21 嘱咐他们每年守亚达月十四、十五两日,
\par 22 以这月的两日为犹大人脱离仇敌得平安、转忧为喜、转悲为乐的吉日。在这两日设筵欢乐,彼此馈送礼物, 济穷人。
\par 23 於是,犹大人按著末底改所写与他们的信,应承照初次所守的守为永例;
\par 24 是因犹大人的仇敌亚甲族哈米大他的儿子哈曼设谋杀害犹大人,掣普珥,就是掣签,为要杀尽灭绝他们;
\par 25 这事报告於王,王便降旨使哈曼谋害犹大人的恶事归到他自己的头上,并吩咐把他和他的众子都挂在木架上。
\par 26 照著普珥的名字,犹大人就称这两日为「普珥日」。他们因这信上的话,又因所看见所遇见的事,
\par 27 就应承自己与後裔,并归附他们的人,每年按时必守这两日,永远不废。
\par 28 各省各城、家家户户、世世代代纪念遵守这两日,使这「普珥日」在犹大人中不可废掉,在他们後裔中也不可忘记。
\par 29 亚比孩的女儿王后以斯帖和犹大人末底改以全权写第二封信,坚嘱犹大人守这「普珥日」,
\par 30 用和平诚实话写信给亚哈随鲁王国中一百二十七省所有的犹大人,
\par 31 劝他们按时守这「普珥日」,禁食呼求,是照犹大人末底改和王后以斯帖所嘱咐的,也照犹大人为自己与後裔所应承的。
\par 32 以斯帖命定守「普珥日」,这事也记录在书上。

\chapter{10}

\par 1 亚哈随鲁王使旱地和海岛的人民都进贡。
\par 2 他以权柄能力所行的,并他抬举末底改使他高升的事,岂不都写在玛代和波斯王的历史上吗?
\par 3 犹大人末底改作亚哈随鲁王的宰相,在犹大人中为大,得他众弟兄的喜悦,为本族的人求好处,向他们说和平的话。


\end{document}