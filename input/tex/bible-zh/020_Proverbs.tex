\begin{document}

\title{箴言}


\chapter{1}

\par 1 以色列王大卫儿子所罗门的箴言:
\par 2 要使人晓得智慧和训诲,分辨通达的言语,
\par 3 使人处事领受智慧、仁义、公平、正直的训诲,
\par 4 使愚人灵明,使少年人有知识和谋略,
\par 5 使智慧人听见,增长学问,使聪明人得著智谋,
\par 6 使人明白箴言和譬喻,懂得智慧人的言词和谜语。
\par 7 敬畏耶和华是知识的开端;愚妄人藐视智慧和训诲。
\par 8 我儿,要听你父亲的训诲,不可离弃你母亲的法则(或作:指教);
\par 9 因为这要作你头上的华冠,你项上的金链。
\par 10 我儿,恶人若引诱你,你不可随从。
\par 11 他们若说:你与我们同去,我们要埋伏流人之血,要蹲伏害无罪之人;
\par 12 我们好像阴间,把他们活活吞下;他们如同下坑的人,被我们囫囵吞了;
\par 13 我们必得各样宝物,将所掳来的,装满房屋;
\par 14 你与我们大家同分,我们共用一个囊袋;
\par 15 我儿,不要与他们同行一道,禁止你脚走他们的路。
\par 16 因为,他们的脚奔跑行恶;他们急速流人的血,
\par 17 好像飞鸟,网罗设在眼前仍不躲避。
\par 18 这些人埋伏,是为自流己血;蹲伏,是为自害己命。
\par 19 凡贪恋财利的,所行之路都是如此;这贪恋之心乃夺去得财者之命。
\par 20 智慧在街市上呼喊,在宽阔处发声,
\par 21 在热闹街头喊叫,在城门口,在城中发出言语,
\par 22 说:你们愚昧人喜爱愚昧,亵慢人喜欢亵慢,愚顽人恨恶知识,要到几时呢?
\par 23 你们当因我的责备回转;我要将我的灵浇灌你们,将我的话指示你们。
\par 24 我呼唤,你们不肯听从;我伸手,无人理会;
\par 25 反轻弃我一切的劝戒,不肯受我的责备。
\par 26 你们遭灾难,我就发笑;惊恐临到你们,我必嗤笑。
\par 27 惊恐临到你们,好像狂风;灾难来到,如同暴风;急难痛苦临到你们身上。
\par 28 那时,你们必呼求我,我却不答应,恳切的寻找我,却寻不见。
\par 29 因为,你们恨恶知识,不喜爱敬畏耶和华,
\par 30 不听我的劝戒,藐视我一切的责备,
\par 31 所以必吃自结的果子,充满自设的计谋。
\par 32 愚昧人背道,必杀己身;愚顽人安逸,必害己命。
\par 33 惟有听从我的,必安然居住,得享安静,不怕灾祸。

\chapter{2}

\par 1 我儿,你若领受我的言语,存记我的命令,
\par 2 侧耳听智慧,专心求聪明,
\par 3 呼求明哲,扬声求聪明,
\par 4 寻找他,如寻找银子,搜求他,如搜求隐藏的珍宝,
\par 5 你就明白敬畏耶和华,得以认识神。
\par 6 因为,耶和华赐人智慧;知识和聪明都由他口而出。
\par 7 他给正直人存留真智慧,给行为纯正的人作盾牌,
\par 8 为要保守公平人的路,护庇虔敬人的道。
\par 9 你也必明白仁义、公平、正直、一切的善道。
\par 10 智慧必入你心;你的灵要以知识为美。
\par 11 谋略必护卫你;聪明必保守你,
\par 12 要救你脱离恶道(或作:恶人的道),脱离说乖谬话的人。
\par 13 那等人舍弃正直的路,行走黑暗的道,
\par 14 欢喜作恶,喜爱恶人的乖僻,
\par 15 在他们的道中弯曲,在他们的路上偏僻。
\par 16 智慧要救你脱离淫妇,就是那油嘴滑舌的外女。
\par 17 他离弃幼年的配偶,忘了神的盟约。
\par 18 他的家陷入死地;他的路偏向阴间。
\par 19 凡到他那里去的,不得转回,也得不著生命的路。
\par 20 智慧必使你行善人的道,守义人的路。
\par 21 正直人必在世上居住;完全人必在地上存留。
\par 22 惟有恶人必然剪除;奸诈的,必然拔出。

\chapter{3}

\par 1 我儿,不要忘记我的法则(或作:指教);你心要谨守我的诫命;
\par 2 因为他必将长久的日子,生命的年数与平安,加给你。
\par 3 不可使慈爱、诚实离开你,要系在你颈项上,刻在你心版上。
\par 4 这样,你必在神和世人眼前蒙恩宠,有聪明。
\par 5 你要专心仰赖耶和华,不可倚靠自己的聪明,
\par 6 在你一切所行的事上都要认定他,他必指引你的路。
\par 7 不要自以为有智慧;要敬畏耶和华,远离恶事。
\par 8 这便医治你的肚脐,滋润你的百骨。
\par 9 你要以财物和一切初熟的土产尊荣耶和华。
\par 10 这样,你的仓房必充满有余;你的酒 有新酒盈溢。
\par 11 我儿,你不可轻看耶和华的管教(或作:惩治),也不可厌烦他的责备;
\par 12 因为耶和华所爱的,他必责备,正如父亲责备所喜爱的儿子。
\par 13 得智慧,得聪明的,这人便为有福。
\par 14 因为得智慧胜过得银子,其利益强如精金,
\par 15 比珍珠(或作:红宝石)宝贵;你一切所喜爱的,都不足与比较。
\par 16 他右手有长寿,左手有富贵。
\par 17 他的道是安乐;他的路全是平安。
\par 18 他与持守他的作生命树;持定他的,俱各有福。
\par 19 耶和华以智慧立地,以聪明定天,
\par 20 以知识使深渊裂开,使天空滴下甘露。
\par 21 我儿,要谨守真智慧和谋略,不可使他离开你的眼目。
\par 22 这样,他必作你的生命,颈项的美饰。
\par 23 你就坦然行路,不至碰脚。
\par 24 你躺下,必不惧怕;你躺卧,睡得香甜。
\par 25 忽然来的惊恐,不要害怕;恶人遭毁灭,也不要恐惧。
\par 26 因为耶和华是你所倚靠的;他必保守你的脚不陷入网罗。
\par 27 你手若有行善的力量,不可推辞,就当向那应得的人施行。
\par 28 你那里若有现成的,不可对邻舍说:去吧,明天再来,我必给你。
\par 29 你的邻舍既在你附近安居,你不可设计害他。
\par 30 人未曾加害与你,不可无故与他相争。
\par 31 不可嫉妒强暴的人,也不可选择他所行的路。
\par 32 因为,乖僻人为耶和华所憎恶;正直人为他所亲密。
\par 33 耶和华咒诅恶人的家庭,赐福与义人的居所。
\par 34 他讥诮那好讥诮的人,赐恩给谦卑的人。
\par 35 智慧人必承受尊荣;愚昧人高升也成为羞辱。

\chapter{4}

\par 1 众子啊,要听父亲的教训,留心得知聪明。
\par 2 因我所给你们的是好教训;不可离弃我的法则(或作:指教)。
\par 3 我在父亲面前为孝子,在母亲眼中为独一的娇儿。
\par 4 父亲教训我说:你心要存记我的言语,遵守我的命令,便得存活。
\par 5 要得智慧,要得聪明,不可忘记,也不可偏离我口中的言语。
\par 6 不可离弃智慧,智慧就护卫你;要爱他,他就保守你。
\par 7 智慧为首;所以,要得智慧。在你一切所得之内必得聪明(或作:用你一切所得的去换聪明)。
\par 8 高举智慧,他就使你高升;怀抱智慧,他就使你尊荣。
\par 9 他必将华冠加在你头上,把荣冕交给你。
\par 10 我儿,你要听受我的言语,就必延年益寿。
\par 11 我已指教你走智慧的道,引导你行正直的路。
\par 12 你行走,脚步必不至狭窄;你奔跑,也不至跌倒。
\par 13 要持定训诲,不可放松;必当谨守,因为他是你的生命。
\par 14 不可行恶人的路;不要走坏人的道。
\par 15 要躲避,不可经过;要转身而去。
\par 16 这等人若不行恶,不得睡觉;不使人跌倒,睡卧不安;
\par 17 因为他们以奸恶吃饼,以强暴喝酒。
\par 18 但义人的路好像黎明的光,越照越明,直到日午。
\par 19 恶人的道好像幽暗,自己不知因什麽跌倒。
\par 20 我儿,要留心听我的言词,侧耳听我的话语,
\par 21 都不可离你的眼目,要存记在你心中。
\par 22 因为得著他的,就得了生命,又得了医全体的良药。
\par 23 你要保守你心,胜过保守一切(或作:你要切切保守你心),因为一生的果效是由心发出。
\par 24 你要除掉邪僻的口,弃绝乖谬的嘴。
\par 25 你的眼目要向前正看;你的眼睛(原文作皮)当向前直观。
\par 26 要修平你脚下的路,坚定你一切的道。
\par 27 不可偏向左右;要使你的脚离开邪恶。

\chapter{5}

\par 1 我儿,要留心我智慧的话语,侧耳听我聪明的言词,
\par 2 为要使你谨守谋略,嘴唇保存知识。
\par 3 因为淫妇的嘴滴下蜂蜜;他的口比油更滑,
\par 4 至终却苦似茵 ,快如两刃的刀。
\par 5 他的脚下入死地;他脚步踏住阴间,
\par 6 以致他找不著生命平坦的道。他的路变迁不定,自己还不知道。
\par 7 众子啊,现在要听从我;不可离弃我口中的话。
\par 8 你所行的道要离他远,不可就近他的房门,
\par 9 恐怕将你的尊荣给别人,将你的岁月给残忍的人;
\par 10 恐怕外人满得你的力量,你劳碌得来的,归入外人的家;
\par 11 终久,你皮肉和身体消毁,你就悲叹,
\par 12 说:我怎麽恨恶训诲,心中藐视责备,
\par 13 也不听从我师傅的话,又不侧耳听那教训我的人?
\par 14 我在圣会里,几乎落在诸般恶中。
\par 15 你要喝自己池中的水,饮自己井里的活水。
\par 16 你的泉源岂可涨溢在外?你的河水岂可流在街上?
\par 17 惟独归你一人,不可与外人同用。
\par 18 要使你的泉源蒙福;要喜悦你幼年所娶的妻。
\par 19 他如可爱的 鹿,可喜的母鹿;愿他的胸怀使你时时知足,他的爱情使你常常恋慕。
\par 20 我儿,你为何恋慕淫妇?为何抱外女的胸怀?
\par 21 因为,人所行的道都在耶和华眼前;他也修平人一切的路。
\par 22 恶人必被自己的罪孽捉住;他必被自己的罪恶如绳索缠绕。
\par 23 他因不受训诲就必死亡;又因愚昧过甚,必走差了路。

\chapter{6}

\par 1 我儿,你若为朋友作保,替外人击掌,
\par 2 你就被口中的话语缠住,被嘴里的言语捉住。
\par 3 我儿,你既落在朋友手中,就当这样行才可救自己:你要自卑,去恳求你的朋友。
\par 4 不要容你的眼睛睡觉;不要容你的眼皮打盹。
\par 5 要救自己,如鹿脱离猎户的手,如鸟脱离捕鸟人的手。
\par 6 懒惰人哪,你去察看蚂蚁的动作就可得智慧。
\par 7 蚂蚁没有元帅,没有官长,没有君王,
\par 8 尚且在夏天预备食物,在收割时聚敛粮食。
\par 9 懒惰人哪,你要睡到几时呢?你何时睡醒呢?
\par 10 再睡片时,打盹片时,抱著手躺卧片时,
\par 11 你的贫穷就必如强盗速来,你的缺乏彷佛拿兵器的人来到。
\par 12 无赖的恶徒,行动就用乖僻的口,
\par 13 用眼传神,用脚示意,用指点划,
\par 14 心中乖僻,常设恶谋,布散分争。
\par 15 所以,灾难必忽然临到他身;他必顷刻败坏,无法可治。
\par 16 耶和华所恨恶的有六样,连他心所憎恶的共有七样:
\par 17 就是高傲的眼,撒谎的舌,流无辜人血的手,
\par 18 图谋恶计的心,飞跑行恶的脚,
\par 19 吐谎言的假见证,并弟兄中布散分争的人。
\par 20 我儿,要谨守你父亲的诫命;不可离弃你母亲的法则(或作:指教),
\par 21 要常系在你心上,挂在你项上。
\par 22 你行走,他必引导你;你躺卧,他必保守你;你睡醒,他必与你谈论。
\par 23 因为诫命是灯,法则(或作:指教)是光,训诲的责备是生命的道,
\par 24 能保你远离恶妇,远离外女谄媚的舌头。
\par 25 你心中不要恋慕他的美色,也不要被他眼皮勾引。
\par 26 因为,妓女能使人只剩一块饼;淫妇猎取人宝贵的生命。
\par 27 人若怀里搋火,衣服岂能不烧呢?
\par 28 人若在火炭上走,脚岂能不烫呢?
\par 29 亲近邻舍之妻的,也是如此;凡挨近他的,不免受罚。
\par 30 贼因饥饿偷窃充饥,人不藐视他,
\par 31 若被找著,他必赔还七倍,必将家中所有的尽都偿还。
\par 32 与妇人行淫的,便是无知;行这事的,必丧掉生命。
\par 33 他必受伤损,必被凌辱;他的羞耻不得涂抹。
\par 34 因为人的嫉恨成了烈怒,报仇的时候决不留情。
\par 35 什麽赎价,他都不顾;你虽送许多礼物,他也不肯干休。

\chapter{7}

\par 1 我儿,你要遵守我的言语,将我的命令存记在心。
\par 2 遵守我的命令就得存活;保守我的法则(或作:指教),好像保守眼中的瞳人,
\par 3 系在你指头上,刻在你心版上。
\par 4 对智慧说:你是我的姊妹,称呼聪明为你的亲人,
\par 5 他就保你远离淫妇,远离说谄媚话的外女。
\par 6 我曾在我房屋的窗户内,从我窗棂之间往外观看:
\par 7 见愚蒙人内,少年人中,分明有一个无知的少年人,
\par 8 从街上经过,走近淫妇的巷口,直往通他家的路去,
\par 9 在黄昏,或晚上,或半夜,或黑暗之中。
\par 10 看哪,有一个妇人来迎接他,是妓女的打扮,有诡诈的心思。
\par 11 这妇人喧嚷,不守约束,在家里停不住脚,
\par 12 有时在街市上,有时在宽阔处,或在各巷口蹲伏,
\par 13 拉住那少年人,与他亲嘴,脸无羞耻对他说:
\par 14 平安祭在我这里,今日才还了我所许的愿。
\par 15 因此,我出来迎接你,恳切求见你的面,恰巧遇见了你。
\par 16 我已经用绣花毯子和埃及线织的花纹布铺了我的床。
\par 17 我又用没药、沉香、桂皮薰了我的榻。
\par 18 你来,我们可以饱享爱情,直到早晨;我们可以彼此亲爱欢乐。
\par 19 因为我丈夫不在家,出门行远路;
\par 20 他手拿银囊,必到月望才回家。
\par 21 淫妇用许多巧言诱他随从,用谄媚的嘴逼他同行。
\par 22 少年人立刻跟随他,好像牛往宰杀之地,又像愚昧人带锁链去受刑罚,
\par 23 直等箭穿他的肝,如同雀鸟急入网罗,却不知是自丧己命。
\par 24 众子啊,现在要听从我,留心听我口中的话。
\par 25 你的心不可偏向淫妇的道,不要入他的迷途。
\par 26 因为,被他伤害仆倒的不少;被他杀戮的而且甚多。
\par 27 他的家是在阴间之路,下到死亡之宫。

\chapter{8}

\par 1 智慧岂不呼叫?聪明岂不发声?
\par 2 他在道旁高处的顶上,在十字路口站立,
\par 3 在城门旁,在城门口,在城门洞,大声说:
\par 4 众人哪,我呼叫你们,我向世人发声。
\par 5 说:愚蒙人哪,你们要会悟灵明;愚昧人哪,你们当心里明白。
\par 6 你们当听,因我要说极美的话;我张嘴要论正直的事。
\par 7 我的口要发出真理;我的嘴憎恶邪恶。
\par 8 我口中的言语都是公义,并无弯曲乖僻。
\par 9 有聪明的,以为明显,得知识的,以为正直。
\par 10 你们当受我的教训,不受白银;宁得知识,胜过黄金。
\par 11 因为智慧比珍珠(或作:红宝石)更美;一切可喜爱的都不足与比较。
\par 12 我智慧以灵明为居所,又寻得知识和谋略。
\par 13 敬畏耶和华在乎恨恶邪恶;那骄傲、狂妄,并恶道,以及乖谬的口,都为我所恨恶。
\par 14 我有谋略和真知识;我乃聪明,我有能力。
\par 15 帝王藉我坐国位;君王藉我定公平。
\par 16 王子和首领,世上一切的审判官,都是藉我掌权。
\par 17 爱我的,我也爱他;恳切寻求我的,必寻得见。
\par 18 丰富尊荣在我;恒久的财并公义也在我。
\par 19 我的果实胜过黄金,强如精金;我的出产超乎高银。
\par 20 我在公义的道上走,在公平的路中行,
\par 21 使爱我的,承受货财,并充满他们的府库。
\par 22 在耶和华造化的起头,在太初创造万物之先,就有了我。
\par 23 从亘古,从太初,未有世界以前,我已被立。
\par 24 没有深渊,没有大水的泉源,我已生出。
\par 25 大山未曾奠定,小山未有之先,我已生出。
\par 26 耶和华还没有创造大地和田野,并世上的土质,我已生出。
\par 27 他立高天,我在那里;他在渊面的周围,划出圆圈。
\par 28 上使穹苍坚硬,下使渊源稳固,
\par 29 为沧海定出界限,使水不越过他的命令,立定大地的根基。
\par 30 那时,我在他那里为工师,日日为他所喜爱,常常在他面前踊跃,
\par 31 踊跃在他为人预备可住之地,也喜悦住在世人之间。
\par 32 众子啊,现在要听从我,因为谨守我道的,便为有福。
\par 33 要听教训就得智慧,不可弃绝。
\par 34 听从我、日日在我门口仰望、在我门框旁边等候的,那人便为有福。
\par 35 因为寻得我的,就寻得生命,也必蒙耶和华的恩惠。
\par 36 得罪我的,却害了自己的性命;恨恶我的,都喜爱死亡。

\chapter{9}

\par 1 智慧建造房屋,凿成七根柱子,
\par 2 宰杀牲畜,调和旨酒,设摆筵席;
\par 3 打发使女出去,自己在城中至高处呼叫,
\par 4 说:谁是愚蒙人,可以转到这里来!又对那无知的人说:
\par 5 你们来,吃我的饼,喝我调和的酒。
\par 6 你们愚蒙人,要舍弃愚蒙,就得存活,并要走光明的道。
\par 7 指斥亵慢人的,必受辱骂;责备恶人的,必被玷污。
\par 8 不要责备亵慢人,恐怕他恨你;要责备智慧人,他必爱你。
\par 9 教导智慧人,他就越发有智慧;指示义人,他就增长学问。
\par 10 敬畏耶和华是智慧的开端;认识至圣者便是聪明。
\par 11 你藉著我,日子必增多,年岁也必加添。
\par 12 你若有智慧,是与自己有益;你若亵慢,就必独自担当。
\par 13 愚昧的妇人喧嚷;他是愚蒙,一无所知。
\par 14 他坐在自己的家门口,坐在城中高处的座位上,
\par 15 呼叫过路的,就是直行其道的人,
\par 16 说:谁是愚蒙人,可以转到这里来!又对那无知的人说:
\par 17 偷来的水是甜的,暗吃的饼是好的。
\par 18 人却不知有阴魂在他那里;他的客在阴间的深处。

\chapter{10}

\par 1 所罗门的箴言:智慧之子使父亲欢乐;愚昧之子叫母亲担忧。
\par 2 不义之财毫无益处;惟有公义能救人脱离死亡。
\par 3 耶和华不使义人受饥饿;恶人所欲的,他必推开。
\par 4 手懒的,要受贫穷;手勤的,却要富足。
\par 5 夏天聚敛的,是智慧之子;收割时沉睡的,是贻羞之子。
\par 6 福祉临到义人的头;强暴蒙蔽恶人的口。
\par 7 义人的纪念被称赞;恶人的名字必朽烂。
\par 8 心中智慧的,必受命令;口里愚妄的,必致倾倒。
\par 9 行正直路的,步步安稳;走弯曲道的,必致败露。
\par 10 以眼传神的,使人忧患;口里愚妄的,必致倾倒。
\par 11 义人的口是生命的泉源;强暴蒙蔽恶人的口。
\par 12 恨能挑启争端;爱能遮掩一切过错。
\par 13 明哲人嘴里有智慧;无知人背上受刑杖。
\par 14 智慧人积存知识;愚妄人的口速致败坏。
\par 15 富户的财物是他的坚城;穷人的贫乏是他的败坏。
\par 16 义人的勤劳致生;恶人的进项致死(死:原文作罪)。
\par 17 谨守训诲的,乃在生命的道上;违弃责备的,便失迷了路。
\par 18 隐藏怨恨的,有说谎的嘴;口出谗谤的,是愚妄的人。
\par 19 多言多语难免有过;禁止嘴唇是有智慧。
\par 20 义人的舌乃似高银;恶人的心所值无几。
\par 21 义人的口教养多人;愚昧人因无知而死亡。
\par 22 耶和华所赐的福使人富足,并不加上忧虑。
\par 23 愚妄人以行恶为戏耍;明哲人却以智慧为乐。
\par 24 恶人所怕的,必临到他;义人所愿的,必蒙应允。
\par 25 暴风一过,恶人归於无有;义人的根基却是永久。
\par 26 懒惰人叫差他的人如醋倒牙,如烟薰目。
\par 27 敬畏耶和华使人日子加多;但恶人的年岁必被减少。
\par 28 义人的盼望必得喜乐;恶人的指望必至灭没。
\par 29 耶和华的道是正直人的保障,却成了作孽人的败坏。
\par 30 义人永不挪移;恶人不得住在地上。
\par 31 义人的口滋生智慧;乖谬的舌必被割断。
\par 32 义人的嘴能令人喜悦;恶人的口说乖谬的话。

\chapter{11}

\par 1 诡诈的天平为耶和华所憎恶;公平的法码为他所喜悦。
\par 2 骄傲来,羞耻也来;谦逊人却有智慧。
\par 3 正直人的纯正必引导自己;奸诈人的乖僻必毁灭自己。
\par 4 发怒的日子资财无益;惟有公义能救人脱离死亡。
\par 5 完全人的义必指引他的路;但恶人必因自己的恶跌倒。
\par 6 正直人的义必拯救自己;奸诈人必陷在自己的罪孽中。
\par 7 恶人一死,他的指望必灭绝;罪人的盼望也必灭没。
\par 8 义人得脱离患难,有恶人来代替他。
\par 9 不虔敬的人用口败坏邻舍;义人却因知识得救。
\par 10 义人享福,合城喜乐;恶人灭亡,人都欢呼。
\par 11 城因正直人祝福便高举,却因邪恶人的口就倾覆。
\par 12 藐视邻舍的,毫无智慧;明哲人却静默不言。
\par 13 往来传舌的,泄漏密事;心中诚实的,遮隐事情。
\par 14 无智谋,民就败落;谋士多,人便安居。
\par 15 为外人作保的,必受亏损;恨恶击掌的,却得安稳。
\par 16 恩德的妇女得尊荣;强暴的男子得资财。
\par 17 仁慈的人善待自己;残忍的人扰害己身。
\par 18 恶人经营,得虚浮的工价;撒义种的,得实在的果效。
\par 19 恒心为义的,必得生命;追求邪恶的,必致死亡。
\par 20 心中乖僻的,为耶和华所憎恶;行事完全的,为他所喜悦。
\par 21 恶人虽然连手,必不免受罚;义人的後裔必得拯救。
\par 22 妇女美貌而无见识,如同金环带在猪鼻上。
\par 23 义人的心愿尽得好处;恶人的指望致干忿怒。
\par 24 有施散的,却更增添;有吝惜过度的,反致穷乏。
\par 25 好施舍的,必得丰裕;滋润人的,必得滋润。
\par 26 屯粮不卖的,民必咒诅他;情愿出卖的,人必为他祝福。
\par 27 恳切求善的,就求得恩惠;惟独求恶的,恶必临到他身。
\par 28 倚仗自己财物的,必跌倒;义人必发旺,如青叶。
\par 29 扰害己家的,必承受清风;愚昧人必作慧心人的仆人。
\par 30 义人所结的果子就是生命树;有智慧的,必能得人。
\par 31 看哪,义人在世尚且受报,何况恶人和罪人呢?

\chapter{12}

\par 1 喜爱管教的,就是喜爱知识;恨恶责备的,却是畜类。
\par 2 善人必蒙耶和华的恩惠;设诡计的人,耶和华必定他的罪。
\par 3 人靠恶行不能坚立;义人的根必不动摇。
\par 4 才德的妇人是丈夫的冠冕;贻羞的妇人如同朽烂在他丈夫的骨中。
\par 5 义人的思念是公平;恶人的计谋是诡诈。
\par 6 恶人的言论是埋伏流人的血;正直人的口必拯救人。
\par 7 恶人倾覆,归於无有;义人的家必站得住。
\par 8 人必按自己的智慧被称赞;心中乖谬的,必被藐视。
\par 9 被人轻贱,却有仆人,强如自尊,缺少食物。
\par 10 义人顾惜他牲畜的命;恶人的怜悯也是残忍。
\par 11 耕种自己田地的,必得饱食;追随虚浮的,却是无知。
\par 12 恶人想得坏人的网罗;义人的根得以结实。
\par 13 恶人嘴中的过错是自己的网罗;但义人必脱离患难。
\par 14 人因口所结的果子,必饱得美福;人手所做的,必为自己的报应。
\par 15 愚妄人所行的,在自己眼中看为正直;惟智慧人肯听人的劝教。
\par 16 愚妄人的恼怒立时显露;通达人能忍辱藏羞。
\par 17 说出真话的,显明公义;作假见证的,显出诡诈。
\par 18 说话浮躁的,如刀刺人;智慧人的舌头却为医人的良药。
\par 19 口吐真言,永远坚立;舌说谎话,只存片时。
\par 20 图谋恶事的,心存诡诈;劝人和睦的,便得喜乐。
\par 21 义人不遭灾害;恶人满受祸患。
\par 22 说谎言的嘴为耶和华所憎恶;行事诚实的,为他所喜悦。
\par 23 通达人隐藏知识;愚昧人的心彰显愚昧。
\par 24 殷勤人的手必掌权;懒惰的人必服苦。
\par 25 人心忧虑,屈而不伸;一句良言,使心欢乐。
\par 26 义人引导他的邻舍;恶人的道叫人失迷。
\par 27 懒惰的人不烤打猎所得的;殷勤的人却得宝贵的财物。
\par 28 在公义的道上有生命;其路之中并无死亡。

\chapter{13}

\par 1 智慧子听父亲的教训;亵慢人不听责备。
\par 2 人因口所结的果子,必享美福;奸诈人必遭强暴。
\par 3 谨守口的,得保生命;大张嘴的,必致败亡。
\par 4 懒惰人羡慕,却无所得;殷勤人必得丰裕。
\par 5 义人恨恶谎言;恶人有臭名,且致惭愧。
\par 6 行为正直的,有公义保守;犯罪的,被邪恶倾覆。
\par 7 假作富足的,却一无所有;装作穷乏的,却广有财物。
\par 8 人的资财是他生命的赎价;穷乏人却听不见威吓的话。
\par 9 义人的光明亮(原文作欢喜);恶人的灯要熄灭。
\par 10 骄傲只启争竞;听劝言的,却有智慧。
\par 11 不劳而得之财必然消耗;勤劳积蓄的,必见加增。
\par 12 所盼望的迟延未得,令人心忧;所愿意的临到,却是生命树。
\par 13 藐视训言的,自取灭亡;敬畏诫命的,必得善报。
\par 14 智慧人的法则(或作:指教)是生命的泉源,可以使人离开死亡的网罗。
\par 15 美好的聪明使人蒙恩;奸诈人的道路崎岖难行。
\par 16 凡通达人都凭知识行事;愚昧人张扬自己的愚昧。
\par 17 奸恶的使者必陷在祸患里;忠信的使臣乃医人的良药。
\par 18 弃绝管教的,必致贫受辱;领受责备的,必得尊荣。
\par 19 所欲的成就,心觉甘甜;远离恶事,为愚昧人所憎恶。
\par 20 与智慧人同行的,必得智慧;和愚昧人作伴的,必受亏损。
\par 21 祸患追赶罪人;义人必得善报。
\par 22 善人给子孙遗留产业;罪人为义人积存资财。
\par 23 穷人耕种多得粮食,但因不义,有消灭的。
\par 24 不忍用杖打儿子的,是恨恶他;疼爱儿子的,随时管教。
\par 25 义人吃得饱足;恶人肚腹缺粮。

\chapter{14}

\par 1 智慧妇人建立家室;愚妄妇人亲手拆毁。
\par 2 行动正直的,敬畏耶和华;行事乖僻的,却藐视他。
\par 3 愚妄人口中骄傲,如杖责打己身;智慧人的嘴必保守自己。
\par 4 家里无牛,槽头乾净;土产加多乃凭牛力。
\par 5 诚实见证人不说谎话;假见证人吐出谎言。
\par 6 亵慢人寻智慧,却寻不著;聪明人易得知识。
\par 7 到愚昧人面前,不见他嘴中有知识。
\par 8 通达人的智慧在乎明白己道;愚昧人的愚妄乃是诡诈(或作:自叹)。
\par 9 愚妄人犯罪,以为戏耍(或作:赎愆祭愚弄愚妄人);正直人互相喜悦。
\par 10 心中的苦楚,自己知道;心里的喜乐,外人无干。
\par 11 奸恶人的房屋必倾倒;正直人的帐棚必兴盛。
\par 12 有一条路,人以为正,至终成为死亡之路。
\par 13 人在喜笑中,心也忧愁;快乐至极就生愁苦。
\par 14 心中背道的,必满得自己的结果;善人必从自己的行为得以知足。
\par 15 愚蒙人是话都信;通达人步步谨慎。
\par 16 智慧人惧怕,就远离恶事;愚妄人却狂傲自恃。
\par 17 轻易发怒的,行事愚妄;设立诡计的,被人恨恶。
\par 18 愚蒙人得愚昧为产业;通达人得知识为冠冕。
\par 19 坏人俯伏在善人面前;恶人俯伏在义人门口。
\par 20 贫穷人连邻舍也恨他;富足人朋友最多。
\par 21 藐视邻舍的,这人有罪;怜悯贫穷的,这人有福。
\par 22 谋恶的,岂非走入迷途吗?谋善的,必得慈爱和诚实。
\par 23 诸般勤劳都有益处;嘴上多言乃致穷乏。
\par 24 智慧人的财为自己的冠冕;愚妄人的愚昧终是愚昧。
\par 25 作真见证的,救人性命;吐出谎言的,施行诡诈。
\par 26 敬畏耶和华的,大有倚靠;他的儿女也有避难所。
\par 27 敬畏耶和华就是生命的泉源,可以使人离开死亡的网罗。
\par 28 帝王荣耀在乎民多;君王衰败在乎民少。
\par 29 不轻易发怒的,大有聪明;性情暴躁的,大显愚妄。
\par 30 心中安静是肉体的生命;嫉妒是骨中的朽烂。
\par 31 欺压贫寒的,是辱没造他的主;怜悯穷乏的,乃是尊敬主。
\par 32 恶人在所行的恶上必被推倒;义人临死,有所投靠。
\par 33 智慧存在聪明人心中;愚昧人心里所存的,显而易见。
\par 34 公义使邦国高举;罪恶是人民的羞辱。
\par 35 智慧的臣子蒙王恩惠;贻羞的仆人遭其震怒。

\chapter{15}

\par 1 回答柔和,使怒消退;言语暴戾,触动怒气。
\par 2 智慧人的舌善发知识;愚昧人的口吐出愚昧。
\par 3 耶和华的眼目无处不在;恶人善人,他都鉴察。
\par 4 温良的舌是生命树;乖谬的嘴使人心碎。
\par 5 愚妄人藐视父亲的管教;领受责备的,得著见识。
\par 6 义人家中多有财宝;恶人得利反受扰害。
\par 7 智慧人的嘴播扬知识;愚昧人的心并不如此。
\par 8 恶人献祭,为耶和华所憎恶;正直人祈祷,为他所喜悦。
\par 9 恶人的道路,为耶和华所憎恶;追求公义的,为他所喜爱。
\par 10 舍弃正路的,必受严刑;恨恶责备的,必致死亡。
\par 11 阴间和灭亡尚在耶和华眼前,何况世人的心呢?
\par 12 亵慢人不爱受责备;他也不就近智慧人。
\par 13 心中喜乐,面带笑容;心里忧愁,灵被损伤。
\par 14 聪明人心求知识;愚昧人口吃愚昧。
\par 15 困苦人的日子都是愁苦;心中欢畅的,常享丰筵。
\par 16 少有财宝,敬畏耶和华,强如多有财宝,烦乱不安。
\par 17 吃素菜,彼此相爱,强如吃肥牛,彼此相恨。
\par 18 暴怒的人挑启争端;忍怒的人止息分争。
\par 19 懒惰人的道像荆棘的篱笆;正直人的路是平坦的大道。
\par 20 智慧子使父亲喜乐;愚昧人藐视母亲。
\par 21 无知的人以愚妄为乐;聪明的人按正直而行。
\par 22 不先商议,所谋无效;谋士众多,所谋乃成。
\par 23 口善应对,自觉喜乐;话合其时,何等美好。
\par 24 智慧人从生命的道上升,使他远离在下的阴间。
\par 25 耶和华必拆毁骄傲人的家,却要立定寡妇的地界。
\par 26 恶谋为耶和华所憎恶;良言乃为纯净。
\par 27 贪恋财利的,扰害己家;恨恶贿赂的,必得存活。
\par 28 义人的心,思量如何回答;恶人的口吐出恶言。
\par 29 耶和华远离恶人,却听义人的祷告。
\par 30 眼有光,使心喜乐;好信息,使骨滋润。
\par 31 听从生命责备的,必常在智慧人中。
\par 32 弃绝管教的,轻看自己的生命;听从责备的,却得智慧。
\par 33 敬畏耶和华是智慧的训诲;尊荣以前,必有谦卑。

\chapter{16}

\par 1 心中的谋算在乎人;舌头的应对由於耶和华。
\par 2 人一切所行的,在自己眼中看为清洁;惟有耶和华衡量人心。
\par 3 你所做的,要交托耶和华,你所谋的,就必成立。
\par 4 耶和华所造的,各适其用;就是恶人也为祸患的日子所造。
\par 5 凡心里骄傲的,为耶和华所憎恶;虽然连手,他必不免受罚。
\par 6 因怜悯诚实,罪孽得赎;敬畏耶和华的,远离恶事。
\par 7 人所行的,若蒙耶和华喜悦,耶和华也使他的仇敌与他和好。
\par 8 多有财利,行事不义,不如少有财利,行事公义。
\par 9 人心筹算自己的道路;惟耶和华指引他的脚步。
\par 10 王的嘴中有神语,审判之时,他的口必不差错。
\par 11 公道的天平和秤都属耶和华;囊中一切法码都为他所定。
\par 12 作恶,为王所憎恶,因国位是靠公义坚立。
\par 13 公义的嘴为王所喜悦;说正直话的,为王所喜爱。
\par 14 王的震怒如杀人的使者;但智慧人能止息王怒。
\par 15 王的脸光使人有生命;王的恩典好像春云时雨。
\par 16 得智慧胜似得金子;选聪明强如选银子。
\par 17 正直人的道是远离恶事;谨守己路的,是保全性命。
\par 18 骄傲在败坏以先;狂心在跌倒之前。
\par 19 心里谦卑与穷乏人来往,强如将掳物与骄傲人同分。
\par 20 谨守训言的,必得好处;倚靠耶和华的,便为有福。
\par 21 心中有智慧,必称为通达人;嘴中的甜言,加增人的学问。
\par 22 人有智慧就有生命的泉源;愚昧人必被愚昧惩治。
\par 23 智慧人的心教训他的口,又使他的嘴增长学问。
\par 24 良言如同蜂房,使心觉甘甜,使骨得医治。
\par 25 有一条路,人以为正,至终成为死亡之路。
\par 26 劳力人的胃口使他劳力,因为他的口腹催逼他。
\par 27 匪徒图谋奸恶,嘴上彷佛有烧焦的火。
\par 28 乖僻人播散分争;传舌的,离间密友。
\par 29 强暴人诱惑邻舍,领他走不善之道。
\par 30 眼目紧合的,图谋乖僻;嘴唇紧闭的,成就邪恶。
\par 31 白发是荣耀的冠冕,在公义的道上必能得著。
\par 32 不轻易发怒的,胜过勇士;治服己心的,强如取城。
\par 33 签放在怀里,定事由耶和华。

\chapter{17}

\par 1 设筵满屋,大家相争,不如有块乾饼,大家相安。
\par 2 仆人办事聪明,必管辖贻羞之子,又在众子中同分产业。
\par 3 鼎为炼银,炉为炼金;惟有耶和华熬炼人心。
\par 4 行恶的,留心听奸诈之言;说谎的,侧耳听邪恶之语。
\par 5 戏笑穷人的,是辱没造他的主;幸灾乐祸的,必不免受罚。
\par 6 子孙为老人的冠冕;父亲是儿女的荣耀。
\par 7 愚顽人说美言本不相宜,何况君王说谎话呢?
\par 8 贿赂在馈送的人眼中看为宝玉,随处运动都得顺利。
\par 9 遮掩人过的,寻求人爱;屡次挑错的,离间密友。
\par 10 一句责备话深入聪明人的心,强如责打愚昧人一百下。
\par 11 恶人只寻背叛,所以必有严厉的使者奉差攻击他。
\par 12 宁可遇见丢崽子的母熊,不可遇见正行愚妄的愚昧人。
\par 13 以恶报善的,祸患必不离他的家。
\par 14 分争的起头如水放开,所以,在争闹之先必当止息争竞。
\par 15 定恶人为义的,定义人为恶的,这都为耶和华所憎恶。
\par 16 愚昧人既无聪明,为何手拿价银买智慧呢?
\par 17 朋友乃时常亲爱,弟兄为患难而生。
\par 18 在邻舍面前击掌作保乃是无知的人。
\par 19 喜爱争竞的,是喜爱过犯;高立家门的,乃自取败坏。
\par 20 心存邪僻的,寻不著好处;舌弄是非的,陷在祸患中。
\par 21 生愚昧子的,必自愁苦;愚顽人的父毫无喜乐。
\par 22 喜乐的心乃是良药;忧伤的灵使骨枯乾。
\par 23 恶人暗中受贿赂,为要颠倒判断。
\par 24 明哲人眼前有智慧;愚昧人眼望地极。
\par 25 愚昧子使父亲愁烦,使母亲忧苦。
\par 26 刑罚义人为不善;责打君子为不义。
\par 27 寡少言语的,有知识;性情温良的,有聪明。
\par 28 愚昧人若静默不言也可算为智慧;闭口不说也可算为聪明。

\chapter{18}

\par 1 与众寡合的,独自寻求心愿,并恼恨一切真智慧。
\par 2 愚昧人不喜爱明哲,只喜爱显露心意。
\par 3 恶人来,藐视随来;羞耻到,辱骂同到。
\par 4 人口中的言语如同深水;智慧的泉源好像涌流的河水。
\par 5 瞻徇恶人的情面,偏断义人的案件,都为不善。
\par 6 愚昧人张嘴启争端,开口招鞭打。
\par 7 愚昧人的口自取败坏;他的嘴是他生命的网罗。
\par 8 传舌人的言语如同美食,深入人的心腹。
\par 9 做工懈怠的,与浪费人为弟兄。
\par 10 耶和华的名是坚固台;义人奔入便得安稳。
\par 11 富足人的财物是他的坚城,在他心想,犹如高墙。
\par 12 败坏之先,人心骄傲;尊荣以前,必有谦卑。
\par 13 未曾听完先回答的,便是他的愚昧和羞辱。
\par 14 人有疾病,心能忍耐;心灵忧伤,谁能承当呢?
\par 15 聪明人的心得知识;智慧人的耳求知识。
\par 16 人的礼物为他开路,引他到高位的人面前。
\par 17 先诉情由的,似乎有理;但邻舍来到,就察出实情。
\par 18 掣签能止息争竞,也能解散强胜的人。
\par 19 弟兄结怨,劝他和好,比取坚固城还难;这样的争竞如同坚寨的门闩。
\par 20 人口中所结的果子,必充满肚腹;他嘴所出的,必使他饱足。
\par 21 生死在舌头的权下,喜爱他的,必吃他所结的果子。
\par 22 得著贤妻的,是得著好处,也是蒙了耶和华的恩惠。
\par 23 贫穷人说哀求的话;富足人用威吓的话回答。
\par 24 滥交朋友的,自取败坏;但有一朋友比弟兄更亲密。

\chapter{19}

\par 1 行为纯正的贫穷人胜过乖谬愚妄的富足人。
\par 2 心无知识的,乃为不善;脚步急快的,难免犯罪。
\par 3 人的愚昧倾败他的道;他的心也抱怨耶和华。
\par 4 财物使朋友增多;但穷人朋友远离。
\par 5 作假见证的,必不免受罚;吐出谎言的,终不能逃脱。
\par 6 好施散的,有多人求他的恩情;爱送礼的,人都为他的朋友。
\par 7 贫穷人,弟兄都恨他;何况他的朋友,更远离他!他用言语追随,他们却走了。
\par 8 得著智慧的,爱惜生命;保守聪明的,必得好处。
\par 9 作假见证的,不免受罚;吐出谎言的,也必灭亡。
\par 10 愚昧人宴乐度日是不合宜的;何况仆人管辖王子呢?
\par 11 人有见识就不轻易发怒;宽恕人的过失便是自己的荣耀。
\par 12 王的忿怒好像狮子吼叫;他的恩典却如草上的甘露。
\par 13 愚昧的儿子是父亲的祸患;妻子的争吵如雨连连滴漏。
\par 14 房屋钱财是祖宗所遗留的;惟有贤慧的妻是耶和华所赐的。
\par 15 懒惰使人沉睡;懈怠的人必受饥饿。
\par 16 谨守诫命的,保全生命;轻忽己路的,必致死亡。
\par 17 怜悯贫穷的,就是借给耶和华;他的善行,耶和华必偿还。
\par 18 趁有指望,管教你的儿子;你的心不可任他死亡。
\par 19 暴怒的人必受刑罚;你若救他,必须再救。
\par 20 你要听劝教,受训诲,使你终久有智慧。
\par 21 人心多有计谋;惟有耶和华的筹算才能立定。
\par 22 施行仁慈的,令人爱慕;穷人强如说谎言的。
\par 23 敬畏耶和华的,得著生命;他必恒久知足,不遭祸患。
\par 24 懒惰人放手在盘子里,就是向口撤回,他也不肯。
\par 25 鞭打亵慢人,愚蒙人必长见识;责备明哲人,他就明白知识。
\par 26 虐待父亲、撵出母亲的,是贻羞致辱之子。
\par 27 我儿,不可听了教训而又偏离知识的言语。
\par 28 匪徒作见证戏笑公平;恶人的口吞下罪孽。
\par 29 刑罚是为亵慢人预备的;鞭打是为愚昧人的背预备的。

\chapter{20}

\par 1 酒能使人亵慢,浓酒使人喧嚷;凡因酒错误的,就无智慧。
\par 2 王的威吓如同狮子吼叫;惹动他怒的,是自害己命。
\par 3 远离分争是人的尊荣;愚妄人都爱争闹。
\par 4 懒惰人因冬寒不肯耕种,到收割的时候,他必讨饭而无所得。
\par 5 人心怀藏谋略,好像深水,惟明哲人才能汲引出来。
\par 6 人多述说自己的仁慈,但忠信人谁能遇著呢?
\par 7 行为纯正的义人,他的子孙是有福的!
\par 8 王坐在审判的位上,以眼目驱散诸恶。
\par 9 谁能说,我洁净了我的心,我脱净了我的罪?
\par 10 两样的法码,两样的升斗,都为耶和华所憎恶。
\par 11 孩童的动作是清洁,是正直,都显明他的本性。
\par 12 能听的耳,能看的眼,都是耶和华所造的。
\par 13 不要贪睡,免致贫穷;眼要睁开,你就吃饱。
\par 14 买物的说:不好,不好;及至买去,他便自夸。
\par 15 有金子和许多珍珠(或作:红宝石),惟有知识的嘴乃为贵重的珍宝。
\par 16 谁为生人作保,就拿谁的衣服;谁为外人作保,谁就要承当。
\par 17 以虚谎而得的食物,人觉甘甜;但後来,他的口必充满尘沙。
\par 18 计谋都凭筹算立定;打仗要凭智谋。
\par 19 往来传舌的,泄漏密事;大张嘴的,不可与他结交。
\par 20 咒骂父母的,他的灯必灭,变为漆黑的黑暗。
\par 21 起初速得的产业,终久却不为福。
\par 22 你不要说,我要以恶报恶;要等候耶和华,他必拯救你。
\par 23 两样的法码为耶和华所憎恶;诡诈的天平也为不善。
\par 24 人的脚步为耶和华所定;人岂能明白自己的路呢?
\par 25 人冒失说,这是圣物,许愿之後才查问,就是自陷网罗。
\par 26 智慧的王簸散恶人,用碌碡滚轧他们。
\par 27 人的灵是耶和华的灯,鉴察人的心腹。
\par 28 王因仁慈和诚实得以保全他的国位,也因仁慈立稳。
\par 29 强壮乃少年人的荣耀;白发为老年人的尊荣。
\par 30 鞭伤除净人的罪恶;责打能入人的心腹。

\chapter{21}

\par 1 王的心在耶和华手中,好像陇沟的水随意流转。
\par 2 人所行的,在自己眼中都看为正;惟有耶和华衡量人心。
\par 3 行仁义公平比献祭更蒙耶和华悦纳。
\par 4 恶人发达(发达:原文作灯),眼高心傲,这乃是罪。
\par 5 殷勤筹划的,足致丰裕;行事急躁的,都必缺乏。
\par 6 用诡诈之舌求财的,就是自己取死;所得之财乃是吹来吹去的浮云。
\par 7 恶人的强暴必将自己扫除,因他们不肯按公平行事。
\par 8 负罪之人的路甚是弯曲;至於清洁的人,他所行的乃是正直。
\par 9 宁可住在房顶的角上,不在宽阔的房屋与争吵的妇人同住。
\par 10 恶人的心乐人受祸;他眼并不怜恤邻舍。
\par 11 亵慢的人受刑罚,愚蒙的人就得智慧;智慧人受训诲,便得知识。
\par 12 义人思想恶人的家,知道恶人倾倒,必至灭亡。
\par 13 塞耳不听穷人哀求的,他将来呼吁也不蒙应允。
\par 14 暗中送的礼物挽回怒气;怀中搋的贿赂止息暴怒。
\par 15 秉公行义使义人喜乐,使作孽的人败坏。
\par 16 迷离通达道路的,必住在阴魂的会中。
\par 17 爱宴乐的,必致穷乏;好酒,爱膏油的,必不富足。
\par 18 恶人作了义人的赎价;奸诈人代替正直人。
\par 19 宁可住在旷野,不与争吵使气的妇人同住。
\par 20 智慧人家中积蓄宝物膏油;愚昧人随得来随吞下。
\par 21 追求公义仁慈的,就寻得生命、公义,和尊荣。
\par 22 智慧人爬上勇士的城墙,倾覆他所倚靠的坚垒。
\par 23 谨守口与舌的,就保守自己免受灾难。
\par 24 心骄气傲的人名叫亵慢;他行事狂妄,都出於骄傲。
\par 25 懒惰人的心愿将他杀害,因为他手不肯做工。
\par 26 有终日贪得无餍的;义人施舍而不吝惜。
\par 27 恶人的祭物是可憎的;何况他存恶意来献呢?
\par 28 作假见证的必灭亡;惟有听真情而言的,其言长存。
\par 29 恶人脸无羞耻;正直人行事坚定。
\par 30 没有人能以智慧、聪明、谋略敌挡耶和华。
\par 31 马是为打仗之日预备的;得胜乃在乎耶和华。

\chapter{22}

\par 1 美名胜过大财;恩宠强如金银。
\par 2 富户穷人在世相遇,都为耶和华所造。
\par 3 通达人见祸藏躲;愚蒙人前往受害。
\par 4 敬畏耶和华心存谦卑,就得富有、尊荣、生命为赏赐。
\par 5 乖僻人的路上有荆棘和网罗;保守自己生命的,必要远离。
\par 6 教养孩童,使他走当行的道,就是到老他也不偏离。
\par 7 富户管辖穷人;欠债的是债主的仆人。
\par 8 撒罪孽的,必收灾祸;他逞怒的杖也必废掉。
\par 9 眼目慈善的,就必蒙福,因他将食物分给穷人。
\par 10 赶出亵慢人,争端就消除;分争和羞辱也必止息。
\par 11 喜爱清心的人因他嘴上的恩言,王必与他为友。
\par 12 耶和华的眼目眷顾聪明人,却倾败奸诈人的言语。
\par 13 懒惰人说:外头有狮子;我在街上就必被杀。
\par 14 淫妇的口为深坑;耶和华所憎恶的,必陷在其中。
\par 15 愚蒙迷住孩童的心,用管教的杖可以远远赶除。
\par 16 欺压贫穷为要利己的,并送礼与富户的,都必缺乏。
\par 17 你须侧耳听受智慧人的言语,留心领会我的知识。
\par 18 你若心中存记,嘴上咬定,这便为美。
\par 19 我今日以此特特指教你,为要使你倚靠耶和华。
\par 20 谋略和知识的美事,我岂没有写给你吗?
\par 21 要使你知道真言的实理,你好将真言回覆那打发你来的人。
\par 22 贫穷人,你不可因他贫穷就抢夺他的物,也不可在城门口欺压困苦人;
\par 23 因耶和华必为他辨屈;抢夺他的,耶和华必夺取那人的命。
\par 24 好生气的人,不可与他结交;暴怒的人,不可与他来往;
\par 25 恐怕你效法他的行为,自己就陷在网罗里。
\par 26 不要与人击掌,不要为欠债的作保。
\par 27 你若没有什麽偿还,何必使人夺去你睡卧的床呢?
\par 28 你先祖所立的地界,你不可挪移。
\par 29 你看见办事殷勤的人吗?他必站在君王面前,必不站在下贱人面前。

\chapter{23}

\par 1 你若与官长坐席,要留意在你面前的是谁。
\par 2 你若是贪食的,就当拿刀放在喉咙上。
\par 3 不可贪恋他的美食,因为是哄人的食物。
\par 4 不要劳碌求富,休仗自己的聪明。
\par 5 你岂要定睛在虚无的钱财上吗?因钱财必长翅膀,如鹰向天飞去。
\par 6 不要吃恶眼人的饭,也不要贪他的美味;
\par 7 因为他心怎样思量,他为人就是怎样。他虽对你说,请吃,请喝,他的心却与你相背。
\par 8 你所吃的那点食物必吐出来;你所说的甘美言语也必落空。
\par 9 你不要说话给愚昧人听,因他必藐视你智慧的言语。
\par 10 不可挪移古时的地界,也不可侵入孤儿的田地;
\par 11 因他们的救赎主大有能力,他必向你为他们辨屈。
\par 12 你要留心领受训诲,侧耳听从知识的言语。
\par 13 不可不管教孩童;你用杖打他,他必不至於死。
\par 14 你要用杖打他,就可以救他的灵魂免下阴间。
\par 15 我儿,你心若存智慧,我的心也甚欢喜。
\par 16 你的嘴若说正直话,我的心肠也必快乐。
\par 17 你心中不要嫉妒罪人,只要终日敬畏耶和华;
\par 18 因为至终必有善报,你的指望也不至断绝。
\par 19 我儿,你当听,当存智慧,好在正道上引导你的心。
\par 20 好饮酒的,好吃肉的,不要与他们来往;
\par 21 因为好酒贪食的,必至贫穷;好睡觉的,必穿破烂衣服。
\par 22 你要听从生你的父亲;你母亲老了,也不可藐视他。
\par 23 你当买真理;就是智慧、训诲,和聪明也都不可卖。
\par 24 义人的父亲必大得快乐;人生智慧的儿子,必因他欢喜。
\par 25 你要使父母欢喜,使生你的快乐。
\par 26 我儿,要将你的心归我;你的眼目也要喜悦我的道路。
\par 27 妓女是深坑;外女是窄阱。
\par 28 他埋伏好像强盗;他使人中多有奸诈的。
\par 29 谁有祸患?谁有忧愁?谁有争斗?谁有哀叹(或作:怨言)?谁无故受伤?谁眼目红赤?
\par 30 就是那流连饮酒、常去寻找调和酒的人。
\par 31 酒发红,在杯中闪烁,你不可观看,虽然下咽舒畅,
\par 32 终久是咬你如蛇,刺你如毒蛇。
\par 33 你眼必看见异怪的事(或作:淫妇);你心必发出乖谬的话。
\par 34 你必像躺在海中,或像卧在桅杆上。
\par 35 你必说:人打我,我却未受伤;人鞭打我,我竟不觉得。我几时清醒,我仍去寻酒。

\chapter{24}

\par 1 你不要嫉妒恶人,也不要起意与他们相处;
\par 2 因为,他们的心图谋强暴,他们的口谈论奸恶。
\par 3 房屋因智慧建造,又因聪明立稳;
\par 4 其中因知识充满各样美好宝贵的财物。
\par 5 智慧人大有能力;有知识的人力上加力。
\par 6 你去打仗,要凭智谋;谋士众多,人便得胜。
\par 7 智慧极高,非愚昧人所能及,所以在城门内不敢开口。
\par 8 设计作恶的,必称为奸人。
\par 9 愚妄人的思念乃是罪恶;亵慢者为人所憎恶。
\par 10 你在患难之日若胆怯,你的力量就微小。
\par 11 人被拉到死地,你要解救;人将被杀,你须拦阻。
\par 12 你若说:这事我未曾知道,那衡量人心的岂不明白吗?保守你命的岂不知道吗?他岂不按各人所行的报应各人吗?
\par 13 我儿,你要吃蜜,因为是好的;吃蜂房下滴的蜜便觉甘甜。
\par 14 你心得了智慧,也必觉得如此。你若找著,至终必有善报;你的指望也不至断绝。
\par 15 你这恶人,不要埋伏攻击义人的家;不要毁坏他安居之所。
\par 16 因为,义人虽七次跌倒,仍必兴起;恶人却被祸患倾倒。
\par 17 你仇敌跌倒,你不要欢喜;他倾倒,你心不要快乐;
\par 18 恐怕耶和华看见就不喜悦,将怒气从仇敌身上转过来。
\par 19 不要为作恶的心怀不平,也不要嫉妒恶人;
\par 20 因为,恶人终不得善报;恶人的灯也必熄灭。
\par 21 我儿,你要敬畏耶和华与君王,不要与反覆无常的人结交,
\par 22 因为他们的灾难必忽然而起。耶和华与君王所施行的毁灭,谁能知道呢?
\par 23 以下也是智慧人的箴言:审判时看人情面是不好的。
\par 24 对恶人说:你是义人的,这人万民必咒诅,列邦必憎恶。
\par 25 责备恶人的,必得喜悦;美好的福也必临到他。
\par 26 应对正直的,犹如与人亲嘴。
\par 27 你要在外头预备工料,在田间办理整齐,然後建造房屋。
\par 28 不可无故作见证陷害邻舍,也不可用嘴欺骗人。
\par 29 不可说:人怎样待我,我也怎样待他;我必照他所行的报复他。
\par 30 我经过懒惰人的田地、无知人的葡萄园,
\par 31 荆棘长满了地皮,刺草遮盖了田面,石墙也坍塌了。
\par 32 我看见就留心思想;我看著就领了训诲。
\par 33 再睡片时,打盹片时,抱著手躺卧片时,
\par 34 你的贫穷就必如强盗速来,你的缺乏彷佛拿兵器的人来到。

\chapter{25}

\par 1 以下也是所罗门的箴言,是犹大王希西家的人所誊录的。
\par 2 将事隐秘乃神的荣耀;将事察清乃君王的荣耀。
\par 3 天之高,地之厚,君王之心也测不透。
\par 4 除去银子的渣滓就有银子出来,银匠能以做器皿。
\par 5 除去王面前的恶人,国位就靠公义坚立。
\par 6 不要在王面前妄自尊大;不要在大人的位上站立。
\par 7 宁可有人说:请你上来,强如在你觐见的王子面前叫你退下。
\par 8 不要冒失出去与人争竞,免得至终被他羞辱,你就不知道怎样行了。
\par 9 你与邻舍争讼,要与他一人辩论,不可泄漏人的密事,
\par 10 恐怕听见的人骂你,你的臭名就难以脱离。
\par 11 一句话说得合宜,就如金苹果在银网子里。
\par 12 智慧人的劝戒,在顺从的人耳中,好像金耳环和精金的妆饰。
\par 13 忠信的使者叫差他的人心里舒畅,就如在收割时有冰雪的凉气。
\par 14 空夸赠送礼物的,好像无雨的风云。
\par 15 恒常忍耐可以劝动君王;柔和的舌头能折断骨头。
\par 16 你得了蜜吗?只可吃够而已,恐怕你过饱就呕吐出来。
\par 17 你的脚要少进邻舍的家,恐怕他厌烦你,恨恶你。
\par 18 作假见证陷害邻舍的,就是大槌,是利刀,是快箭。
\par 19 患难时倚靠不忠诚的人,好像破坏的牙,错骨缝的脚。
\par 20 对伤心的人唱歌,就如冷天脱衣服,又如硷上倒醋。
\par 21 你的仇敌若饿了,就给他饭吃;若渴了,就给他水喝;
\par 22 因为,你这样行就是把炭火堆在他的头上;耶和华也必赏赐你。
\par 23 北风生雨,谗谤人的舌头也生怒容。
\par 24 宁可住在房顶的角上,不在宽阔的房屋与争吵的妇人同住。
\par 25 有好消息从远方来,就如拿凉水给口渴的人喝。
\par 26 义人在恶人面前退缩,好像 浑之泉,弄浊之井。
\par 27 吃蜜过多是不好的;考究自己的荣耀也是可厌的。
\par 28 人不制伏自己的心,好像毁坏的城邑没有墙垣。

\chapter{26}

\par 1 夏天落雪,收割时下雨,都不相宜;愚昧人得尊荣也是如此。
\par 2 麻雀往来,燕子翻飞;这样,无故的咒诅也必不临到。
\par 3 鞭子是为打马,辔头是为勒驴;刑杖是为打愚昧人的背。
\par 4 不要照愚昧人的愚妄话回答他,恐怕你与他一样。
\par 5 要照愚昧人的愚妄话回答他,免得他自以为有智慧。
\par 6 藉愚昧人手寄信的,是砍断自己的脚,自受(原文作:喝)损害。
\par 7 瘸子的脚空存无用;箴言在愚昧人的口中也是如此。
\par 8 将尊荣给愚昧人的,好像人把石子包在机弦里。
\par 9 箴言在愚昧人的口中,好像荆棘刺入醉汉的手。
\par 10 雇愚昧人的,与雇过路人的,就像射伤众人的弓箭手。
\par 11 愚昧人行愚妄事,行了又行,就如狗转过来吃他所吐的。
\par 12 你见自以为有智慧的人吗?愚昧人比他更有指望。
\par 13 懒惰人说:道上有猛狮,街上有壮狮。
\par 14 门在枢纽转动,懒惰人在床上也是如此。
\par 15 懒惰人放手在盘子里,就是向口撤回也以为劳乏。
\par 16 懒惰人看自己比七个善於应对的人更有智慧。
\par 17 过路被事激动,管理不干己的争竞,好像人揪住狗耳。
\par 18 人欺凌邻舍,却说:我岂不是戏耍吗?
\par 19 他就像疯狂的人抛掷火把、利箭,与杀人的兵器(原文作死亡)。
\par 20 火缺了柴就必熄灭;无人传舌,争竞便止息。
\par 21 好争竞的人煽惑争端,就如余火加炭,火上加柴一样。
\par 22 传舌人的言语,如同美食,深入人的心腹。
\par 23 火热的嘴,奸恶的心,好像银渣包的瓦器。
\par 24 怨恨人的,用嘴粉饰,心里却藏著诡诈;
\par 25 他用甜言蜜语,你不可信他,因为他心中有七样可憎恶的。
\par 26 他虽用诡诈遮掩自己的怨恨,他的邪恶必在会中显露。
\par 27 挖陷坑的,自己必掉在其中;滚石头的,石头必反滚在他身上。
\par 28 虚谎的舌恨他所压伤的人;谄媚的口败坏人的事。

\chapter{27}

\par 1 不要为明日自夸,因为一日要生何事,你尚且不能知道。
\par 2 要别人夸奖你,不可用口自夸;等外人称赞你,不可用嘴自称。
\par 3 石头重,沙土沉,愚妄人的恼怒比这两样更重。
\par 4 忿怒为残忍,怒气为狂澜,惟有嫉妒,谁能敌得住呢?
\par 5 当面的责备强如背地的爱情。
\par 6 朋友加的伤痕出於忠诚;仇敌连连亲嘴却是多余。
\par 7 人吃饱了,厌恶蜂房的蜜;人饥饿了,一切苦物都觉甘甜。
\par 8 人离本处飘流,好像雀鸟离窝游飞。
\par 9 膏油与香料使人心喜悦;朋友诚实的劝教也是如此甘美。
\par 10 你的朋友和父亲的朋友,你都不可离弃。你遭难的日子,不要上弟兄的家去;相近的邻舍强如远方的弟兄。
\par 11 我儿,你要作智慧人,好叫我的心欢喜,使我可以回答那讥诮我的人。
\par 12 通达人见祸藏躲;愚蒙人前往受害。
\par 13 谁为生人作保,就拿谁的衣服;谁为外女作保,谁就承当。
\par 14 清晨起来,大声给朋友祝福的,就算是咒诅他。
\par 15 大雨之日连连滴漏,和争吵的妇人一样;
\par 16 想拦阻他的,便是拦阻风,也是右手抓油。
\par 17 铁磨铁,磨出刃来;朋友相感(原文作磨朋友的脸)也是如此。
\par 18 看守无花果树的,必吃树上的果子;敬奉主人的,必得尊荣。
\par 19 水中照脸,彼此相符;人与人,心也相对。
\par 20 阴间和灭亡永不满足;人的眼目也是如此。
\par 21 鼎为炼银,炉为炼金,人的称赞也试炼人。
\par 22 你虽用杵将愚妄人与打碎的麦子一同捣在臼中,他的愚妄还是离不了他。
\par 23 你要详细知道你羊群的景况,留心料理你的牛群;
\par 24 因为资财不能永有,冠冕岂能存到万代?
\par 25 乾草割去,嫩草发现,山上的菜蔬也被收敛。
\par 26 羊羔之毛是为你作衣服;山羊是为作田地的价值,
\par 27 并有母山羊奶够你吃,也够你的家眷吃,且够养你的婢女。

\chapter{28}

\par 1 恶人虽无人追赶也逃跑;义人却胆壮像狮子。
\par 2 邦国因有罪过,君王就多更换;因有聪明知识的人,国必长存。
\par 3 穷人欺压贫民,好像暴雨冲没粮食。
\par 4 违弃律法的,夸奖恶人;遵守律法的,却与恶人相争。
\par 5 坏人不明白公义;惟有寻求耶和华的,无不明白。
\par 6 行为纯正的穷乏人胜过行事乖僻的富足人。
\par 7 谨守律法的,是智慧之子;与贪食人作伴的,却羞辱其父。
\par 8 人以厚利加增财物,是给那怜悯穷人者积蓄的。
\par 9 转耳不听律法的,他的祈祷也为可憎。
\par 10 诱惑正直人行恶道的,必掉在自己的坑里;惟有完全人必承受福分。
\par 11 富足人自以为有智慧,但聪明的贫穷人能将他查透。
\par 12 义人得志,有大荣耀;恶人兴起,人就躲藏。
\par 13 遮掩自己罪过的,必不亨通;承认离弃罪过的,必蒙怜恤。
\par 14 常存敬畏的,便为有福;心存刚硬的,必陷在祸患里。
\par 15 暴虐的君王辖制贫民,好像吼叫的狮子、觅食的熊。
\par 16 无知的君多行暴虐;以贪财为可恨的,必年长日久。
\par 17 背负流人血之罪的,必往坑里奔跑,谁也不可拦阻他。
\par 18 行动正直的,必蒙拯救;行事弯曲的,立时跌倒。
\par 19 耕种自己田地的,必得饱食;追随虚浮的,足受穷乏。
\par 20 诚实人必多得福;想要急速发财的,不免受罚。
\par 21 看人的情面乃为不好;人因一块饼枉法也为不好。
\par 22 人有恶眼想要急速发财,却不知穷乏必临到他身。
\par 23 责备人的,後来蒙人喜悦,多於那用舌头谄媚人的。
\par 24 偷窃父母的,说:这不是罪,此人就是与强盗同类。
\par 25 心中贪婪的,挑起争端;倚靠耶和华的,必得丰裕。
\par 26 心中自是的,便是愚昧人;凭智慧行事的,必蒙拯救。
\par 27 济贫穷的,不至缺乏;佯为不见的,必多受咒诅。
\par 28 恶人兴起,人就躲藏;恶人败亡,义人增多。

\chapter{29}

\par 1 人屡次受责罚,仍然硬著颈项;他必顷刻败坏,无法可治。
\par 2 义人增多,民就喜乐;恶人掌权,民就叹息。
\par 3 爱慕智慧的,使父亲喜乐;与妓女结交的,却浪费钱财。
\par 4 王藉公平,使国坚定;索要贿赂,使国倾败。
\par 5 谄媚邻舍的,就是设网罗绊他的脚。
\par 6 恶人犯罪,自陷网罗;惟独义人欢呼喜乐。
\par 7 义人知道查明穷人的案;恶人没有聪明,就不得而知。
\par 8 亵慢人煽惑通城;智慧人止息众怒。
\par 9 智慧人与愚妄人相争,或怒或笑,总不能使他止息。
\par 10 好流人血的,恨恶完全人,索取正直人的性命。
\par 11 愚妄人怒气全发;智慧人忍气含怒。
\par 12 君王若听谎言,他一切臣仆都是奸恶。
\par 13 贫穷人、强暴人在世相遇;他们的眼目都蒙耶和华光照。
\par 14 君王凭诚实判断穷人;他的国位必永远坚立。
\par 15 杖打和责备能加增智慧;放纵的儿子使母亲羞愧。
\par 16 恶人加多,过犯也加多,义人必看见他们跌倒。
\par 17 管教你的儿子,他就使你得安息,也必使你心里喜乐。
\par 18 没有异象(或作:默示),民就放肆;惟遵守律法的,便为有福。
\par 19 只用言语,仆人不肯受管教;他虽然明白,也不留意。
\par 20 你见言语急躁的人吗?愚昧人比他更有指望。
\par 21 人将仆人从小娇养,这仆人终久必成了他的儿子。
\par 22 好气的人挑启争端;暴怒的人多多犯罪。
\par 23 人的高傲必使他卑下;心里谦逊的,必得尊荣。
\par 24 人与盗贼分赃,是恨恶自己的性命;他听见叫人发誓的声音,却不言语。
\par 25 惧怕人的,陷入网罗;惟有倚靠耶和华的,必得安稳。
\par 26 求王恩的人多;定人事乃在耶和华。
\par 27 为非作歹的,被义人憎嫌;行事正直的,被恶人憎恶。

\chapter{30}

\par 1 雅基的儿子亚古珥的言语就是真言。这人对以铁和乌甲说:
\par 2 我比众人更蠢笨,也没有人的聪明。
\par 3 我没有学好智慧,也不认识至圣者。
\par 4 谁升天又降下来?谁聚风在掌握中?谁包水在衣服里?谁立定地的四极?他名叫什麽?他儿子名叫什麽?你知道吗?
\par 5 神的言语句句都是炼净的;投靠他的,他便作他们的盾牌。
\par 6 他的言语,你不可加添,恐怕他责备你,你就显为说谎言的。
\par 7 我求你两件事,在我未死之先,不要不赐给我:
\par 8 求你使虚假和谎言远离我;使我也不贫穷也不富足;赐给我需用的饮食,
\par 9 恐怕我饱足不认你,说:耶和华是谁呢?又恐怕我贫穷就偷窃,以致亵渎我神的名。
\par 10 你不要向主人谗谤仆人,恐怕他咒诅你,你便算为有罪。
\par 11 有一宗人(宗:原文作代;下同),咒诅父亲,不给母亲祝福。
\par 12 有一宗人,自以为清洁,却没有洗去自己的污秽。
\par 13 有一宗人,眼目何其高傲,眼皮也是高举。
\par 14 有一宗人,牙如剑,齿如刀,要吞灭地上的困苦人和世间的穷乏人。
\par 15 蚂蟥有两个女儿,常说:给呀,给呀!有三样不知足的,连不说「够的」共有四样:
\par 16 就是阴间和石胎,浸水不足的地,并火。
\par 17 戏笑父亲、藐视而不听从母亲的,他的眼睛必为谷中的乌鸦啄出来,为鹰雏所吃。
\par 18 我所测不透的奇妙有三样,连我所不知道的共有四样:
\par 19 就是鹰在空中飞的道;蛇在磐石上爬的道;船在海中行的道;男与女交合的道。
\par 20 淫妇的道也是这样:他吃了,把嘴一擦就说:我没有行恶。
\par 21 使地震动的有三样,连地担不起的共有四样:
\par 22 就是仆人作王;愚顽人吃饱;
\par 23 丑恶的女子出嫁;婢女接续主母。
\par 24 地上有四样小物,却甚聪明:
\par 25 蚂蚁是无力之类,却在夏天预备粮食。
\par 26 沙番是软弱之类,却在磐石中造房。
\par 27 蝗虫没有君王,却分队而出。
\par 28 守宫用爪抓墙,却住在王宫。
\par 29 步行威武的有三样,连行走威武的共有四样:
\par 30 就是狮子,乃百兽中最为猛烈、无所躲避的,
\par 31 猎狗,公山羊,和无人能敌的君王。
\par 32 你若行事愚顽,自高自傲,或是怀了恶念,就当用手 口。
\par 33 摇牛奶必成奶油;扭鼻子必出血。照样,激动怒气必起争端。

\chapter{31}

\par 1 利慕伊勒王的言语,是他母亲教训他的真言。
\par 2 我的儿啊,我腹中生的儿啊,我许愿得的儿啊!我当怎样教训你呢?
\par 3 不要将你的精力给妇女;也不要有败坏君王的行为。
\par 4 利慕伊勒啊,君王喝酒,君王喝酒不相宜;王子说浓酒在那里也不相宜;
\par 5 恐怕喝了就忘记律例,颠倒一切困苦人的是非。
\par 6 可以把浓酒给将亡的人喝,把清酒给苦心的人喝,
\par 7 让他喝了,就忘记他的贫穷,不再记念他的苦楚。
\par 8 你当为哑巴(或作:不能自辩的)开口,为一切孤独的伸冤。
\par 9 你当开口按公义判断,为困苦和穷乏的辨屈。
\par 10 才德的妇人谁能得著呢?他的价值远胜过珍珠。
\par 11 他丈夫心里倚靠他,必不缺少利益;
\par 12 他一生使丈夫有益无损。
\par 13 他寻找羊 和麻,甘心用手做工。
\par 14 他好像商船从远方运粮来,
\par 15 未到黎明他就起来,把食物分给家中的人,将当做的工分派婢女。
\par 16 他想得田地就买来;用手所得之利栽种葡萄园。
\par 17 他以能力束腰,使膀臂有力。
\par 18 他觉得所经营的有利;他的灯终夜不灭。
\par 19 他手拿捻线竿,手把纺线车。
\par 20 他张手 济困苦人,伸手帮补穷乏人。
\par 21 他不因下雪为家里的人担心,因为全家都穿著朱红衣服。
\par 22 他为自己制作绣花毯子;他的衣服是细麻和紫色布做的。
\par 23 他丈夫在城门口与本地的长老同坐,为众人所认识。
\par 24 他做细麻布衣裳出卖,又将腰带卖与商家。
\par 25 能力和威仪是他的衣服;他想到日後的景况就喜笑。
\par 26 他开口就发智慧;他舌上有仁慈的法则。
\par 27 他观察家务,并不吃闲饭。
\par 28 他的儿女起来称他有福;他的丈夫也称赞他,
\par 29 说:才德的女子很多,惟独你超过一切。
\par 30 艳丽是虚假的,美容是虚浮的;惟敬畏耶和华的妇女必得称赞。
\par 31 愿他享受操作所得的;愿他的工作在城门口荣耀他。


\end{document}