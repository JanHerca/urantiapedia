\begin{document}

\title{出埃及}


\chapter{1}

\par 1 以色列的众子,各带家眷,和雅各一同来到埃及。他们的名字记在下面。
\par 2 有流便、西缅、利未、犹大、
\par 3 以萨迦、西布伦、便雅悯、
\par 4 但、拿弗他利、迦得、亚设。
\par 5 凡从雅各而生的,共有七十人。约瑟已经在埃及。
\par 6 约瑟和他的弟兄,并那一代的人,都死了。
\par 7 以色列人生养众多,并且繁茂,极其强盛,满了那地。
\par 8 有不认识约瑟的新王起来,治理埃及,
\par 9 对他的百姓说:「看哪,这以色列民比我们还多,又比我们强盛。
\par 10 来吧,我们不如用巧计待他们,恐怕他们多起来,日後若遇什麽争战的事,就连合我们的仇敌攻击我们,离开这地去了。」
\par 11 於是埃及人派督工的辖制他们,加重担苦害他们。他们为法老建造两座积货城,就是比东和兰塞。
\par 12 只是越发苦害他们,他们越发多起来,越发蔓延;埃及人就因以色列人愁烦。
\par 13 埃及人严严地使以色列人做工,
\par 14 使他们因做苦工觉得命苦;无论是和泥,是做砖,是做田间各样的工,在一切的工上都严严地待他们。
\par 15 有希伯来的两个收生婆,一名施弗拉,一名普阿;埃及王对他们说:
\par 16 「你们为希伯来妇人收生,看他们临盆的时候,若是男孩,就把他杀了;若是女孩,就留他存活。」
\par 17 但是收生婆敬畏神,不照埃及王的吩咐行,竟存留男孩的性命。
\par 18 埃及王召了收生婆来,说:「你们为什麽做这事,存留男孩的性命呢?」
\par 19 收生婆对法老说:「因为希伯来妇人与埃及妇人不同;希伯来妇人本是健壮的(原文作活泼的),收生婆还没有到,他们已经生产了。」
\par 20 神厚待收生婆。以色列人多起来,极其强盛。
\par 21 收生婆因为敬畏神,神便叫他们成立家室。
\par 22 法老吩咐他的众民说:「以色列人所生的男孩,你们都要丢在河里;一切的女孩,你们要存留他的性命。」

\chapter{2}

\par 1 有一个利未家的人娶了一个利未女子为妻。
\par 2 那女人怀孕,生一个儿子,见他俊美,就藏了他三个月,
\par 3 後来不能再藏,就取了一个蒲草箱,抹上石漆和石油,将孩子放在里头,把箱子搁在河边的芦荻中。
\par 4 孩子的姊姊远远站著,要知道他究竟怎麽样。
\par 5 法老的女儿来到河边洗澡,他的使女们在河边行走。他看见箱子在芦荻中,就打发一个婢女拿来。
\par 6 他打开箱子,看见那孩子。孩子哭了,他就可怜他,说:「这是希伯来人的一个孩子。」
\par 7 孩子的姊姊对法老的女儿说:「我去在希伯来妇人中叫一个奶妈来,为你奶这孩子,可以不可以?」
\par 8 法老的女儿说:「可以。」童女就去叫了孩子的母亲来。
\par 9 法老的女儿对他说:「你把这孩子抱去,为我奶他,我必给你工价。」妇人就抱了孩子去奶他。
\par 10 孩子渐长,妇人把他带到法老的女儿那里,就作了他的儿子。他给孩子起名叫摩西,意思说:「因我把他从水里拉出来。」
\par 11 後来,摩西长大,他出去到他弟兄那里,看他们的重担,见一个埃及人打希伯来人的一个弟兄。
\par 12 他左右观看,见没有人,就把埃及人打死了,藏在沙土里。
\par 13 第二天他出去,见有两个希伯来人争斗,就对那欺负人的说:「你为什麽打你同族的人呢?」
\par 14 那人说:「谁立你作我们的首领和审判官呢?难道你要杀我,像杀那埃及人吗?」摩西便惧怕,说:「这事必是被人知道了。」
\par 15 法老听见这事,就想杀摩西,但摩西躲避法老,逃往米甸地居住。
\par 16 一日,他在井旁坐下。米甸的祭司有七个女儿;他们来打水,打满了槽,要饮父亲的群羊。
\par 17 有牧羊的人来,把他们赶走了,摩西却起来帮助他们,又饮了他们的群羊。
\par 18 他们来到父亲流珥那里;他说:「今日你们为何来得这麽快呢?」
\par 19 他们说:「有一个埃及人救我们脱离牧羊人的手,并且为我们打水饮了群羊。」
\par 20 他对女儿们说:「那个人在那里?你们为什麽撇下他呢?你们去请他来吃饭。」
\par 21 摩西甘心和那人同住;那人把他的女儿西坡拉给摩西为妻。
\par 22 西坡拉生了一个儿子,摩西给他起名叫革舜,意思说:「因我在外邦作了寄居的。」
\par 23 过了多年,埃及王死了。以色列人因做苦工,就叹息哀求,他们的哀声达於神。
\par 24 神听见他们的哀声,就记念他与亚伯拉罕、以撒、雅各所立的约。
\par 25 神看顾以色列人,也知道他们的苦情。

\chapter{3}

\par 1 摩西牧养他岳父米甸祭司叶忒罗的羊群;一日领羊群往野外去,到了神的山,就是何烈山。
\par 2 耶和华的使者从荆棘里火焰中向摩西显现。摩西观看,不料,荆棘被火烧著,却没有烧毁。
\par 3 摩西说:「我要过去看这大异象,这荆棘为何没有烧坏呢?」
\par 4 耶和华 神见他过去要看,就从荆棘里呼叫说:「摩西!摩西!」他说:「我在这里。」
\par 5 神说:「不要近前来。当把你脚上的鞋脱下来,因为你所站之地是圣地」;
\par 6 又说:「我是你父亲的神,是亚伯拉罕的神,以撒的神,雅各的神。」摩西蒙上脸,因为怕看神。
\par 7 耶和华说:「我的百姓在埃及所受的困苦,我实在看见了;他们因受督工的辖制所发的哀声,我也听见了。我原知道他们的痛苦,
\par 8 我下来是要救他们脱离埃及人的手,领他们出了那地,到美好、宽阔、流奶与蜜之地,就是到迦南人、赫人、亚摩利人、比利洗人、希未人、耶布斯人之地。
\par 9 现在以色列人的哀声达到我耳中,我也看见埃及人怎样欺压他们。
\par 10 故此,我要打发你去见法老,使你可以将我的百姓以色列人从埃及领出来。」
\par 11 摩西对神说:「我是什麽人,竟能去见法老,将以色列人从埃及领出来呢?」
\par 12 神说:「我必与你同在。你将百姓从埃及领出来之後,你们必在这山上事奉我;这就是我打发你去的证据。」
\par 13 摩西对神说:「我到以色列人那里,对他们说:『你们祖宗的神打发我到你们这里来。』他们若问我说:『他叫什麽名字?』我要对他们说什麽呢?」
\par 14 神对摩西说:「我是自有永有的」;又说:「你要对以色列人这样说:『那自有的打发我到你们这里来。』」
\par 15 神又对摩西说:「你要对以色列人这样说:『耶和华你们祖宗的神,就是亚伯拉罕的神,以撒的神,雅各的神,打发我到你们这里来。』耶和华是我的名,直到永远;这也是我的纪念,直到万代。
\par 16 你去招聚以色列的长老,对他们说:『耶和华你们祖宗的神,就是亚伯拉罕的神,以撒的神,雅各的神,向我显现,说:我实在眷顾了你们,我也看见埃及人怎样待你们。
\par 17 我也说:要将你们从埃及的困苦中领出来,往迦南人、赫人、亚摩利人、比利洗人、希未人、耶布斯人的地去,就是到流奶与蜜之地。』
\par 18 他们必听你的话。你和以色列的长老要去见埃及王,对他说:『耶和华希伯来人的神遇见了我们,现在求你容我们往旷野去,走三天的路程,为要祭祀耶和华我们的神。』
\par 19 我知道虽用大能的手,埃及王也不容你们去。
\par 20 我必伸手在埃及中间施行我一切的奇事,攻击那地,然後他才容你们去。
\par 21 我必叫你们在埃及人眼前蒙恩,你们去的时候就不至於空手而去。
\par 22 但各妇女必向他的邻舍,并居住在他家里的女人,要金器银器和衣裳,好给你们的儿女穿戴。这样你们就把埃及人的财物夺去了。」

\chapter{4}

\par 1 摩西回答说:「他们必不信我,也不听我的话,必说:『耶和华并没有向你显现。』」
\par 2 耶和华对摩西说:「你手里是什麽?」他说:「是杖。」
\par 3 耶和华说:「丢在地上。」他一丢下去,就变作蛇;摩西便跑开。
\par 4 耶和华对摩西说:「伸出手来,拿住他的尾巴,他必在你手中仍变为杖;
\par 5 如此好叫他们信耶和华他们祖宗的神,就是亚伯拉罕的神,以撒的神,雅各的神,是向你显现了。」
\par 6 耶和华又对他说:「把手放在怀里。」他就把手放在怀里,及至抽出来,不料,手长了大麻疯,有雪那样白。
\par 7 耶和华说:「再把手放在怀里。」他就再把手放在怀里,及至从怀里抽出来,不料,手已经复原,与周身的肉一样;
\par 8 又说:「倘或他们不听你的话,也不信头一个神迹,他们必信第二个神迹。
\par 9 这两个神迹若都不信,也不听你的话,你就从河里取些水,倒在旱地上,你从河里取的水必在旱地上变作血。」
\par 10 摩西对耶和华说:「主啊,我素日不是能言的人,就是从你对仆人说话以後,也是这样。我本是拙口笨舌的。」
\par 11 耶和华对他说:「谁造人的口呢?谁使人口哑、耳聋、目明、眼瞎呢?岂不是我耶和华吗?
\par 12 现在去吧,我必赐你口才,指教你所当说的话。」
\par 13 摩西说:「主啊,你愿意打发谁,就打发谁去吧!」
\par 14 耶和华向摩西发怒说:「不是有你的哥哥利未人亚伦吗?我知道他是能言的;现在他出来迎接你,他一见你,心里就欢喜。
\par 15 你要将当说的话传给他;我也要赐你和他口才,又要指教你们所当行的事。
\par 16 他要替你对百姓说话;你要以他当作口,他要以你当作神。
\par 17 你手里要拿这杖,好行神迹。」
\par 18 於是,摩西回到他岳父叶忒罗那里,对他说:「求你容我回去见我在埃及的弟兄,看他们还在不在。」叶忒罗对摩西说:「你可以平平安安的去吧!」
\par 19 耶和华在米甸对摩西说:「你要回埃及去,因为寻索你命的人都死了。」
\par 20 摩西就带著妻子和两个儿子,叫他们骑上驴,回埃及地去。摩西手里拿著神的杖。
\par 21 耶和华对摩西说:「你回到埃及的时候,要留意将我指示你的一切奇事行在法老面前。但我要使(或作:任凭;下同)他的心刚硬,他必不容百姓去。
\par 22 你要对法老说:『耶和华这样说:以色列是我的儿子,我的长子。
\par 23 我对你说过:容我的儿子去,好事奉我。你还是不肯容他去。看哪,我要杀你的长子。』」
\par 24 摩西在路上住宿的地方,耶和华遇见他,想要杀他。
\par 25 西坡拉就拿一块火石,割下他儿子的阳皮,丢在摩西脚前,说:「你真是我的血郎了。」
\par 26 这样,耶和华才放了他。西坡拉说:「你因割礼就是血郎了。」
\par 27 耶和华对亚伦说:「你往旷野去迎接摩西。」他就去,在神的山遇见摩西,和他亲嘴。
\par 28 摩西将耶和华打发他所说的言语和嘱咐他所行的神迹都告诉了亚伦。
\par 29 摩西、亚伦就去招聚以色列的众长老。
\par 30 亚伦将耶和华对摩西所说的一切话述说了一遍,又在百姓眼前行了那些神迹,
\par 31 百姓就信了。以色列人听见耶和华眷顾他们,鉴察他们的困苦,就低头下拜。

\chapter{5}

\par 1 後来摩西、亚伦去对法老说:「耶和华以色列的神这样说:『容我的百姓去,在旷野向我守节。』」
\par 2 法老说:「耶和华是谁,使我听他的话,容以色列人去呢?我不认识耶和华,也不容以色列人去!」
\par 3 他们说:「希伯来人的神遇见了我们。求你容我们往旷野去,走三天的路程,祭祀耶和华我们的神,免得他用瘟疫、刀兵攻击我们。」
\par 4 埃及王对他们说:「摩西、亚伦!你们为什麽叫百姓旷工呢?你们去担你们的担子吧!」
\par 5 又说:「看哪,这地的以色列人如今众多,你们竟叫他们歇下担子!」
\par 6 当天,法老吩咐督工的和官长说:
\par 7 「你们不可照常把草给百姓做砖,叫他们自己去捡草。
\par 8 他们素常做砖的数目,你们仍旧向他们要,一点不可减少;因为他们是懒惰的,所以呼求说:『容我们去祭祀我们的神。』
\par 9 你们要把更重的工夫加在这些人身上,叫他们劳碌,不听虚谎的言语。」
\par 10 督工的和官长出来对百姓说:「法老这样说:『我不给你们草。
\par 11 你们自己在那里能找草,就往那里去找吧!但你们的工一点不可减少。』」
\par 12 於是百姓散在埃及遍地,捡碎 当作草。
\par 13 督工的催著说:「你们一天当完一天的工,与先前有草一样。」
\par 14 法老督工的,责打他所派以色列人的官长,说:「你们昨天今天为什麽没有照向来的数目做砖、完你们的工作呢?」
\par 15 以色列人的官长就来哀求法老说:「为什麽这样待你的仆人?
\par 16 督工的不把草给仆人,并且对我们说:『做砖吧!』看哪,你仆人挨了打,其实是你百姓的错。」
\par 17 但法老说:「你们是懒惰的!你们是懒惰的!所以说:『容我们去祭祀耶和华。』
\par 18 现在你们去做工吧!草是不给你们的,砖却要如数交纳。」
\par 19 以色列人的官长听说「你们每天做砖的工作一点不可减少」,就知道是遭遇祸患了。
\par 20 他们离了法老出来,正遇见摩西、亚伦站在对面,
\par 21 就向他们说:「愿耶和华鉴察你们,施行判断;因你们使我们在法老和他臣仆面前有了臭名,把刀递在他们手中杀我们。」
\par 22 摩西回到耶和华那里,说:「主啊,你为什麽苦待这百姓呢?为什麽打发我去呢?
\par 23 自从我去见法老,奉你的名说话,他就苦待这百姓,你一点也没有拯救他们。」

\chapter{6}

\par 1 耶和华对摩西说:「现在你必看见我向法老所行的事,使他因我大能的手容以色列人去,且把他们赶出他的地。」
\par 2 神晓谕摩西说:「我是耶和华。
\par 3 我从前向亚伯拉罕、以撒、雅各显现为全能的神;至於我名耶和华,他们未曾知道。
\par 4 我与他们坚定所立的约,要把他们寄居的迦南地赐给他们。
\par 5 我也听见以色列人被埃及人苦待的哀声,我也记念我的约。
\par 6 所以你要对以色列人说:『我是耶和华;我要用伸出来的膀臂重重的刑罚埃及人,救赎你们脱离他们的重担,不做他们的苦工。
\par 7 我要以你们为我的百姓,我也要作你们的神。你们要知道我是耶和华你们的神,是救你们脱离埃及人之重担的。
\par 8 我起誓应许给亚伯拉罕、以撒、雅各的那地,我要把你们领进去,将那地赐给你们为业。我是耶和华。』」
\par 9 摩西将这话告诉以色列人,只是他们因苦工愁烦,不肯听他的话。
\par 10 耶和华晓谕摩西说:
\par 11 「你进去对埃及王法老说,要容以色列人出他的地。」
\par 12 摩西在耶和华面前说:「以色列人尚且不听我的话,法老怎肯听我这拙口笨舌的人呢?」
\par 13 耶和华吩咐摩西、亚伦往以色列人和埃及王法老那里去,把以色列人从埃及地领出来。
\par 14 以色列人家长的名字记在下面。以色列长子流便的儿子是哈诺、法路、希斯仑、迦米;这是流便的各家。
\par 15 西缅的儿子是耶母利、雅悯、阿辖、雅斤、琐辖,和迦南女子的儿子扫罗;这是西缅的各家。
\par 16 利未众子的名字按著他们的後代记在下面:就是革顺、哥辖、米拉利。利未一生的岁数是一百三十七岁。
\par 17 革顺的儿子按著家室是立尼、示每。
\par 18 哥辖的儿子是暗兰、以斯哈、希伯伦、乌薛。哥辖一生的岁数是一百三十三岁。
\par 19 米拉利的儿子是抹利和母示;这是利未的家,都按著他们的後代。
\par 20 暗兰娶了他父亲的妹妹约基别为妻,他给他生了亚伦和摩西。暗兰一生的岁数是一百三十七岁。
\par 21 以斯哈的儿子是可拉、尼斐、细基利。
\par 22 乌薛的儿子是米沙利、以利撒反、西提利。
\par 23 亚伦娶了亚米拿达的女儿,拿顺的妹妹,以利沙巴为妻,他给他生了拿答、亚比户、以利亚撒、以他玛。
\par 24 可拉的儿子是亚惜、以利加拿、亚比亚撒;这是可拉的各家。
\par 25 亚伦的儿子以利亚撒娶了普铁的一个女儿为妻,他给他生了非尼哈。这是利未人的家长,都按著他们的家。
\par 26 耶和华说:「将以色列人按著他们的军队从埃及地领出来。」这是对那亚伦、摩西说的。
\par 27 对埃及王法老说要将以色列人从埃及领出来的,就是这摩西、亚伦。
\par 28 当耶和华在埃及地对摩西说话的日子,
\par 29 他向摩西说:「我是耶和华;我对你说的一切话,你都要告诉埃及王法老。」
\par 30 摩西在耶和华面前说:「看哪,我是拙口笨舌的人,法老怎肯听我呢?」

\chapter{7}

\par 1 耶和华对摩西说:「我使你在法老面前代替神,你的哥哥亚伦是替你说话的。
\par 2 凡我所吩咐你的,你都要说。你的哥哥亚伦要对法老说,容以色列人出他的地。
\par 3 我要使法老的心刚硬,也要在埃及地多行神迹奇事。
\par 4 但法老必不听你们;我要伸手重重的刑罚埃及,将我的军队以色列民从埃及地领出来。
\par 5 我伸手攻击埃及,将以色列人从他们中间领出来的时候,埃及人就要知道我是耶和华。」
\par 6 摩西、亚伦这样行;耶和华怎样吩咐他们,他们就照样行了。
\par 7 摩西、亚伦与法老说话的时候,摩西八十岁,亚伦八十三岁。
\par 8 耶和华晓谕摩西、亚伦说:
\par 9 「法老若对你们说:『你们行件奇事吧!』你就吩咐亚伦说:『把杖丢在法老面前,使杖变作蛇。』」
\par 10 摩西、亚伦进去见法老,就照耶和华所吩咐的行。亚伦把杖丢在法老和臣仆面前,杖就变作蛇。
\par 11 於是法老召了博士和术士来;他们是埃及行法术的,也用邪术照样而行。
\par 12 他们各人丢下自己的杖,杖就变作蛇;但亚伦的杖吞了他们的杖。
\par 13 法老心里刚硬,不肯听从摩西、亚伦,正如耶和华所说的。
\par 14 耶和华对摩西说:「法老心里固执,不肯容百姓去。
\par 15 明日早晨,他出来往水边去,你要往河边迎接他,手里要拿著那变过蛇的杖,
\par 16 对他说:『耶和华希伯来人的神打发我来见你,说:容我的百姓去,好在旷野事奉我。到如今你还是不听。
\par 17 耶和华这样说:我要用我手里的杖击打河中的水,水就变作血;因此,你必知道我是耶和华。
\par 18 河里的鱼必死,河也要腥臭,埃及人就要厌恶吃这河里的水。』」
\par 19 耶和华晓谕摩西说:「你对亚伦说:『把你的杖伸在埃及所有的水以上,就是在他们的江、河、池、塘以上,叫水都变作血。在埃及遍地,无论在木器中,石器中,都必有血。』」
\par 20 摩西、亚伦就照耶和华所吩咐的行。亚伦在法老和臣仆眼前举杖击打河里的水,河里的水都变作血了。
\par 21 河里的鱼死了,河也腥臭了,埃及人就不能吃这河里的水;埃及遍地都有了血。
\par 22 埃及行法术的,也用邪术照样而行。法老心里刚硬,不肯听摩西、亚伦,正如耶和华所说的。
\par 23 法老转身进宫,也不把这事放在心上。
\par 24 埃及人都在河的两边挖地,要得水喝,因为他们不能喝这河里的水。
\par 25 耶和华击打河以後满了七天。

\chapter{8}

\par 1 耶和华吩咐摩西说:「你进去见法老,对他说:『耶和华这样说:容我的百姓去,好事奉我。
\par 2 你若不肯容他们去,我必使青蛙糟蹋你的四境。
\par 3 河里要滋生青蛙;这青蛙要上来进你的宫殿和你的卧房,上你的床榻,进你臣仆的房屋,上你百姓的身上,进你的炉灶和你的抟面盆,
\par 4 又要上你和你百姓并你众臣仆的身上。』」
\par 5 耶和华晓谕摩西说:「你对亚伦说:『把你的杖伸在江、河、池以上,使青蛙到埃及地上来。』」
\par 6 亚伦便伸杖在埃及的诸水以上,青蛙就上来,遮满了埃及地。
\par 7 行法术的也用他们的邪术照样而行,叫青蛙上了埃及地。
\par 8 法老召了摩西、亚伦来,说:「请你们求耶和华使这青蛙离开我和我的民,我就容百姓去祭祀耶和华。」
\par 9 摩西对法老说:「任凭你吧,我要何时为你和你的臣仆并你的百姓祈求,除灭青蛙离开你和你的宫殿只留在河里呢?」
\par 10 他说:「明天。」摩西说:「可以照你的话吧,好叫你知道没有像耶和华我们神的。
\par 11 青蛙要离开你和你的宫殿,并你的臣仆与你的百姓,只留在河里。」
\par 12 於是摩西、亚伦离开法老出去。摩西为扰害法老的青蛙呼求耶和华。
\par 13 耶和华就照摩西的话行。凡在房里、院中、田间的青蛙都死了。
\par 14 众人把青蛙聚拢成堆,遍地就都腥臭。
\par 15 但法老见灾祸松缓,就硬著心,不肯听他们,正如耶和华所说的。
\par 16 耶和华吩咐摩西说:「你对亚伦说:『伸出你的杖击打地上的尘土,使尘土在埃及遍地变作虱子(或作:虼蚤;下同)。』」
\par 17 他们就这样行。亚伦伸杖击打地上的尘土,就在人身上和牲畜身上有了虱子;埃及遍地的尘土都变成虱子了。
\par 18 行法术的也用邪术要生出虱子来,却是不能。於是在人身上和牲畜身上都有了虱子。
\par 19 行法术的就对法老说:「这是神的手段。」法老心里刚硬,不肯听摩西、亚伦,正如耶和华所说的。
\par 20 耶和华对摩西说:「你清早起来,法老来到水边,你站在他面前,对他说:『耶和华这样说:容我的百姓去,好事奉我。
\par 21 你若不容我的百姓去,我要叫成群的苍蝇到你和你臣仆并你百姓的身上,进你的房屋,并且埃及人的房屋和他们所住的地都要满了成群的苍蝇。
\par 22 当那日,我必分别我百姓所住的歌珊地,使那里没有成群的苍蝇,好叫你知道我是天下的耶和华。
\par 23 我要将我的百姓和你的百姓分别出来。明天必有这神迹。』」
\par 24 耶和华就这样行。苍蝇成了大群,进入法老的宫殿,和他臣仆的房屋;埃及遍地就因这成群的苍蝇败坏了。
\par 25 法老召了摩西、亚伦来,说:「你们去,在这地祭祀你们的神吧!」
\par 26 摩西说:「这样行本不相宜,因为我们要把埃及人所厌恶的祭祀耶和华我们的神;若把埃及人所厌恶的在他们眼前献为祭,他们岂不拿石头打死我们吗?
\par 27 我们要往旷野去,走三天的路程,照著耶和华我们神所要吩咐我们的祭祀他。」
\par 28 法老说:「我容你们去,在旷野祭祀耶和华你们的神;只是不要走得很远。求你们为我祈祷。」
\par 29 摩西说:「我要出去求耶和华,使成群的苍蝇明天离开法老和法老的臣仆并法老的百姓;法老却不可再行诡诈,不容百姓去祭祀耶和华。」
\par 30 於是摩西离开法老去求耶和华。
\par 31 耶和华就照摩西的话行,叫成群的苍蝇离开法老和他的臣仆并他的百姓,一个也没有留下。
\par 32 这一次法老又硬著心,不容百姓去。

\chapter{9}

\par 1 耶和华吩咐摩西说:「你进去见法老,对他说:『耶和华希伯来人的神这样说:容我的百姓去,好事奉我。
\par 2 你若不肯容他们去,仍旧强留他们,
\par 3 耶和华的手加在你田间的牲畜上,就是在马、驴、骆驼、牛群、羊群上,必有重重的瘟疫。
\par 4 耶和华要分别以色列的牲畜和埃及的牲畜,凡属以色列人的,一样都不死。』」
\par 5 耶和华就定了时候,说:「明天耶和华必在此地行这事。」
\par 6 第二天,耶和华就行这事。埃及的牲畜几乎都死了,只是以色列人的牲畜,一个都没有死。
\par 7 法老打发人去看,谁知以色列人的牲畜连一个都没有死。法老的心却是固执,不容百姓去。
\par 8 耶和华吩咐摩西、亚伦说:「你们取几捧炉灰,摩西要在法老面前向天扬起来。
\par 9 这灰要在埃及全地变作尘土,在人身上和牲畜身上成了起泡的疮。」
\par 10 摩西、亚伦取了炉灰,站在法老面前。摩西向天扬起来,就在人身上和牲畜身上成了起泡的疮。
\par 11 行法术的在摩西面前站立不住,因为在他们身上和一切埃及人身上都有这疮。
\par 12 耶和华使法老的心刚硬,不听他们,正如耶和华对摩西所说的。
\par 13 耶和华对摩西说:「你清早起来,站在法老面前,对他说:『耶和华希伯来人的神这样说:容我的百姓去,好事奉我。
\par 14 因为这一次我要叫一切的灾殃临到你和你臣仆并你百姓的身上,叫你知道在普天下没有像我的。
\par 15 我若伸手用瘟疫攻击你和你的百姓,你早就从地上除灭了。
\par 16 其实,我叫你存立,是特要向你显我的大能,并要使我的名传遍天下。
\par 17 你还向我的百姓自高,不容他们去吗?
\par 18 到明天约在这时候,我必叫重大的冰雹降下,自从埃及开国以来,没有这样的冰雹。
\par 19 现在你要打发人把你的牲畜和你田间一切所有的催进来;凡在田间不收回家的,无论是人是牲畜,冰雹必降在他们身上,他们就必死。』」
\par 20 法老的臣仆中,惧怕耶和华这话的,便叫他的奴仆和牲畜跑进家来。
\par 21 但那不把耶和华这话放在心上的,就将他的奴仆和牲畜留在田里。
\par 22 耶和华对摩西说:「你向天伸杖,使埃及遍地的人身上和牲畜身上,并田间各样菜蔬上,都有冰雹。」
\par 23 摩西向天伸杖,耶和华就打雷下雹,有火闪到地上;耶和华下雹在埃及地上。
\par 24 那时,雹与火搀杂,甚是厉害,自从埃及成国以来,遍地没有这样的。
\par 25 在埃及遍地,雹击打了田间所有的人和牲畜,并一切的菜蔬,又打坏田间一切的树木。
\par 26 惟独以色列人所住的歌珊地没有冰雹。
\par 27 法老打发人召摩西、亚伦来,对他们说:「这一次我犯了罪了。耶和华是公义的;我和我的百姓是邪恶的。
\par 28 这雷轰和冰雹已经够了。请你们求耶和华,我就容你们去,不再留住你们。」
\par 29 摩西对他说:「我一出城,就要向耶和华举手祷告;雷必止住,也不再有冰雹,叫你知道全地都是属耶和华的。
\par 30 至於你和你的臣仆,我知道你们还是不惧怕耶和华 神。」
\par 31 (那时,麻和大麦被雹击打;因为大麦已经吐穗,麻也开了花。
\par 32 只是小麦和粗麦没有被击打,因为还没有长成。)
\par 33 摩西离了法老出城,向耶和华举手祷告;雷和雹就止住,雨也不再浇在地上了。
\par 34 法老见雨和雹与雷止住,就越发犯罪;他和他的臣仆都硬著心。
\par 35 法老的心刚硬,不容以色列人去,正如耶和华藉著摩西所说的。

\chapter{10}

\par 1 耶和华对摩西说:「你进去见法老。我使他和他臣仆的心刚硬,为要在他们中间显我这些神迹,
\par 2 并要叫你将我向埃及人所做的事,和在他们中间所行的神迹,传於你儿子和你孙子的耳中,好叫你们知道我是耶和华。」
\par 3 摩西、亚伦就进去见法老,对他说:「耶和华希伯来人的神这样说:『你在我面前不肯自卑要到几时呢?容我的百姓去,好事奉我。
\par 4 你若不肯容我的百姓去,明天我要使蝗虫进入你的境内,
\par 5 遮满地面,甚至看不见地,并且吃那冰雹所剩的和田间所长的一切树木。
\par 6 你的宫殿和你众臣仆的房屋,并一切埃及人的房屋,都要被蝗虫占满了;自从你祖宗和你祖宗的祖宗在世以来,直到今日,没有见过这样的灾。』」摩西就转身离开法老出去。
\par 7 法老的臣仆对法老说:「这人为我们的网罗要到几时呢?容这些人去事奉耶和华他们的神吧!埃及已经败坏了,你还不知道吗?」
\par 8 於是摩西、亚伦被召回来见法老;法老对他们说:「你们去事奉耶和华你们的神;但那要去的是谁呢?」
\par 9 摩西说:「我们要和我们老的少的、儿子女儿同去,且把羊群牛群一同带去,因为我们务要向耶和华守节。」
\par 10 法老对他们说:「我容你们和你们妇人孩子去的时候,耶和华与你们同在吧!你们要谨慎;因为有祸在你们眼前(或作:你们存著恶意),
\par 11 不可都去!你们这壮年人去事奉耶和华吧,因为这是你们所求的。」於是把他们从法老面前撵出去。
\par 12 耶和华对摩西说:「你向埃及地伸杖,使蝗虫到埃及地上来,吃地上一切的菜蔬,就是冰雹所剩的。」
\par 13 摩西就向埃及地伸杖,那一昼一夜,耶和华使东风刮在埃及地上;到了早晨,东风把蝗虫刮了来。
\par 14 蝗虫上来,落在埃及的四境,甚是厉害;以前没有这样的,以後也必没有。
\par 15 因为这蝗虫遮满地面,甚至地都黑暗了,又吃地上一切的菜蔬和冰雹所剩树上的果子。埃及遍地,无论是树木,是田间的菜蔬,连一点青的也没有留下。
\par 16 於是法老急忙召了摩西、亚伦来,说:「我得罪耶和华你们的神,又得罪了你们。
\par 17 现在求你,只这一次,饶恕我的罪,求耶和华你们的神使我脱离这一次的死亡。」
\par 18 摩西就离开法老去求耶和华。
\par 19 耶和华转了极大的西风,把蝗虫刮起,吹入红海;在埃及的四境连一个也没有留下。
\par 20 但耶和华使法老的心刚硬,不容以色列人去。
\par 21 耶和华对摩西说:「你向天伸杖,使埃及地黑暗;这黑暗似乎摸得著。」
\par 22 摩西向天伸杖,埃及遍地就乌黑了三天。
\par 23 三天之久,人不能相见,谁也不敢起来离开本处;惟有以色列人家中都有亮光。
\par 24 法老就召摩西来,说:「你们去事奉耶和华;只是你们的羊群牛群要留下;你们的妇人孩子可以和你们同去。」
\par 25 摩西说:「你总要把祭物和燔祭牲交给我们,使我们可以祭祀耶和华我们的神。
\par 26 我们的牲畜也要带去,连一蹄也不留下;因为我们要从其中取出来,事奉耶和华我们的神。我们未到那里,还不知道用什麽事奉耶和华。」
\par 27 但耶和华使法老的心刚硬,不肯容他们去。
\par 28 法老对摩西说:「你离开我去吧,你要小心,不要再见我的面!因为你见我面的那日你就必死!」
\par 29 摩西说:「你说得好!我必不再见你的面了。」

\chapter{11}

\par 1 耶和华对摩西说:「我再使一样的灾殃临到法老和埃及,然後他必容你们离开这地。他容你们去的时候,总要催逼你们都从这地出去。
\par 2 你要传於百姓的耳中,叫他们男女各人向邻舍要金器银器。」
\par 3 耶和华叫百姓在埃及人眼前蒙恩,并且摩西在埃及地、法老臣仆,和百姓的眼中看为极大。
\par 4 摩西说:「耶和华这样说:『约到半夜,我必出去巡行埃及遍地。
\par 5 凡在埃及地,从坐宝座的法老直到磨子後的婢女所有的长子,以及一切头生的牲畜,都必死。
\par 6 埃及遍地必有大哀号;从前没有这样的,後来也必没有。
\par 7 至於以色列中,无论是人是牲畜,连狗也不敢向他们摇舌,好叫你们知道耶和华是将埃及人和以色列人分别出来。』
\par 8 你这一切臣仆都要俯伏来见我,说:『求你和跟从你的百姓都出去』,然後我要出去。」於是,摩西气忿忿地离开法老,出去了。
\par 9 耶和华对摩西说:「法老必不听你们,使我的奇事在埃及地多起来。」
\par 10 摩西、亚伦在法老面前行了这一切奇事;耶和华使法老的心刚硬,不容以色列人出离他的地。

\chapter{12}

\par 1 耶和华在埃及地晓谕摩西、亚伦说:
\par 2 「你们要以本月为正月,为一年之首。
\par 3 你们吩咐以色列全会众说:本月初十日,各人要按著父家取羊羔,一家一只。
\par 4 若是一家的人太少,吃不了一只羊羔,本人就要和他隔壁的邻舍共取一只。你们预备羊羔,要按著人数和饭量计算。
\par 5 要无残疾、一岁的公羊羔,你们或从绵羊里取,或从山羊里取,都可以。
\par 6 要留到本月十四日,在黄昏的时候,以色列全会众把羊羔宰了。
\par 7 各家要取点血,涂在吃羊羔的房屋左右的门框上和门楣上。
\par 8 当夜要吃羊羔的肉;用火烤了,与无酵饼和苦菜同吃。
\par 9 不可吃生的,断不可吃水煮的,要带著头、腿、五脏,用火烤了吃。
\par 10 不可剩下一点留到早晨;若留到早晨,要用火烧了。
\par 11 你们吃羊羔当腰间束带,脚上穿鞋,手中拿杖,赶紧地吃;这是耶和华的逾越节。
\par 12 因为那夜我要巡行埃及地,把埃及地一切头生的,无论是人是牲畜,都击杀了,又要败坏埃及一切的神。我是耶和华。
\par 13 这血要在你们所住的房屋上作记号;我一见这血,就越过你们去。我击杀埃及地头生的时候,灾殃必不临到你们身上灭你们。」
\par 14 「你们要记念这日,守为耶和华的节,作为你们世世代代永远的定例。
\par 15 你们要吃无酵饼七日。头一日要把酵从你们各家中除去;因为从头一日起,到第七日为止,凡吃有酵之饼的,必从以色列中剪除。
\par 16 头一日你们当有圣会,第七日也当有圣会。这两日之内,除了预备各人所要吃的以外,无论何工都不可做。
\par 17 你们要守无酵节,因为我正当这日把你们的军队从埃及地领出来。所以,你们要守这日,作为世世代代永远的定例。
\par 18 从正月十四日晚上,直到二十一日晚上,你们要吃无酵饼。
\par 19 在你们各家中,七日之内不可有酵;因为凡吃有酵之物的,无论是寄居的,是本地的,必从以色列的会中剪除。
\par 20 有酵的物,你们都不可吃;在你们一切住处要吃无酵饼。」
\par 21 於是,摩西召了以色列的众长老来,对他们说:「你们要按著家口取出羊羔,把这逾越节的羊羔宰了。
\par 22 拿一把牛膝草,蘸盆里的血,打在门楣上和左右的门框上。你们谁也不可出自己的房门,直到早晨。
\par 23 因为耶和华要巡行击杀埃及人,他看见血在门楣上和左右的门框上,就必越过那门,不容灭命的进你们的房屋,击杀你们。
\par 24 这例,你们要守著,作为你们和你们子孙永远的定例。
\par 25 日後,你们到了耶和华按著所应许赐给你们的那地,就要守这礼。
\par 26 你们的儿女问你们说:『行这礼是什麽意思?』
\par 27 你们就说:『这是献给耶和华逾越节的祭。当以色列人在埃及的时候,他击杀埃及人,越过以色列人的房屋,救了我们各家。』」於是百姓低头下拜。
\par 28 耶和华怎样吩咐摩西、亚伦,以色列人就怎样行。
\par 29 到了半夜,耶和华把埃及地所有的长子,就是从坐宝座的法老,直到被掳囚在监里之人的长子,以及一切头生的牲畜,尽都杀了。
\par 30 法老和一切臣仆,并埃及众人,夜间都起来了。在埃及有大哀号,无一家不死一个人的。
\par 31 夜间,法老召了摩西、亚伦来,说:「起来!连你们带以色列人,从我民中出去,依你们所说的,去事奉耶和华吧!
\par 32 也依你们所说的,连羊群牛群带著走吧!并要为我祝福。」
\par 33 埃及人催促百姓,打发他们快快出离那地,因为埃及人说:「我们都要死了。」
\par 34 百姓就拿著没有酵的生面,把抟面盆包在衣服中,扛在肩头上。
\par 35 以色列人照著摩西的话行,向埃及人要金器、银器,和衣裳。
\par 36 耶和华叫百姓在埃及人眼前蒙恩,以致埃及人给他们所要的。他们就把埃及人的财物夺去了。
\par 37 以色列人从兰塞起行,往疏割去;除了妇人孩子,步行的男人约有六十万。
\par 38 又有许多闲杂人,并有羊群牛群,和他们一同上去。
\par 39 他们用埃及带出来的生面烤成无酵饼。这生面原没有发起;因为他们被催逼离开埃及,不能耽延,也没有为自己预备什麽食物。
\par 40 以色列人住在埃及共有四百三十年。
\par 41 正满了四百三十年的那一天,耶和华的军队都从埃及地出来了。
\par 42 这夜是耶和华的夜;因耶和华领他们出了埃及地,所以当向耶和华谨守,是以色列众人世世代代该谨守的。
\par 43 耶和华对摩西、亚伦说:「逾越节的例是这样:外邦人都不可吃这羊羔。
\par 44 但各人用银子买的奴仆,既受了割礼就可以吃。
\par 45 寄居的和雇工人都不可吃。
\par 46 应当在一个房子里吃;不可把一点肉从房子里带到外头去。羊羔的骨头一根也不可折断。
\par 47 以色列全会众都要守这礼。
\par 48 若有外人寄居在你们中间,愿向耶和华守逾越节,他所有的男子务要受割礼,然後才容他前来遵守,他也就像本地人一样;但未受割礼的,都不可吃这羊羔。
\par 49 本地人和寄居在你们中间的外人同归一例。」
\par 50 耶和华怎样吩咐摩西、亚伦,以色列众人就怎样行了。
\par 51 正当那日,耶和华将以色列人按著他们的军队,从埃及地领出来。

\chapter{13}

\par 1 耶和华晓谕摩西说:
\par 2 「以色列中凡头生的,无论是人是牲畜,都是我的,要分别为圣归我。」
\par 3 摩西对百姓说:「你们要记念从埃及为奴之家出来的这日,因为耶和华用大能的手将你们从这地方领出来。有酵的饼都不可吃。
\par 4 亚笔月间的这日是你们出来的日子。
\par 5 将来耶和华领你进迦南人、赫人、亚摩利人、希未人、耶布斯人之地,就是他向你的祖宗起誓应许给你那流奶与蜜之地,那时你要在这月间守这礼。
\par 6 你要吃无酵饼七日,到第七日要向耶和华守节。
\par 7 这七日之久,要吃无酵饼;在你四境之内不可见有酵的饼,也不可见发酵的物。
\par 8 当那日,你要告诉你的儿子说:『这是因耶和华在我出埃及的时候为我所行的事。
\par 9 这要在你手上作记号,在你额上作纪念,使耶和华的律法常在你口中,因为耶和华曾用大能的手将你从埃及领出来。
\par 10 所以你每年要按著日期守这例。』」
\par 11 「将来,耶和华照他向你和你祖宗所起的誓将你领进迦南人之地,把这地赐给你,
\par 12 那时你要将一切头生的,并牲畜中头生的,归给耶和华;公的都要属耶和华。
\par 13 凡头生的驴,你要用羊羔代赎;若不代赎,就要打折他的颈项。凡你儿子中头生的都要赎出来。
\par 14 日後,你的儿子问你说:『这是什麽意思?』你就说:『耶和华用大能的手将我们从埃及为奴之家领出来。
\par 15 那时法老几乎不容我们去,耶和华就把埃及地所有头生的,无论是人是牲畜,都杀了。因此,我把一切头生的公牲畜献给耶和华为祭,但将头生的儿子都赎出来。
\par 16 这要在你手上作记号,在你额上作经文,因为耶和华用大能的手将我们从埃及领出来。』」
\par 17 法老容百姓去的时候,非利士地的道路虽近,神却不领他们从那里走;因为神说:「恐怕百姓遇见打仗後悔,就回埃及去。」
\par 18 所以神领百姓绕道而行,走红海旷野的路。以色列人出埃及地,都带著兵器上去。
\par 19 摩西把约瑟的骸骨一同带去;因为约瑟曾叫以色列人严严地起誓,对他们说:「神必眷顾你们,你们要把我的骸骨从这里一同带上去。」
\par 20 他们从疏割起行,在旷野边的以倘安营。
\par 21 日间,耶和华在云柱中领他们的路;夜间,在火柱中光照他们,使他们日夜都可以行走。
\par 22 日间云柱,夜间火柱,总不离开百姓的面前。

\chapter{14}

\par 1 耶和华晓谕摩西说:
\par 2 「你吩咐以色列人转回,安营在比哈希录前,密夺和海的中间,对著巴力洗分,靠近海边安营。
\par 3 法老必说:『以色列人在地中绕迷了,旷野把他们困住了。』
\par 4 我要使法老的心刚硬,他要追赶他们,我便在法老和他全军身上得荣耀;埃及人就知道我是耶和华。」於是以色列人这样行了。
\par 5 有人告诉埃及王说:「百姓逃跑。」法老和他的臣仆就向百姓变心,说:「我们容以色列人去,不再服事我们,这做的是什麽事呢?」
\par 6 法老就预备他的车辆,带领军兵同去,
\par 7 并带著六百辆特选的车和埃及所有的车,每辆都有车兵长。
\par 8 耶和华使埃及王法老的心刚硬,他就追赶以色列人,因为以色列人是昂然无惧地出埃及。
\par 9 埃及人追赶他们,法老一切的马匹、车辆、马兵,与军兵就在海边上,靠近比哈希录,对著巴力洗分,在他们安营的地方追上了。
\par 10 法老临近的时候,以色列人举目看见埃及人赶来,就甚惧怕,向耶和华哀求。
\par 11 他们对摩西说:「难道在埃及没有坟地,你把我们带来死在旷野吗?你为什麽这样待我们,将我们从埃及领出来呢?
\par 12 我们在埃及岂没有对你说过,不要搅扰我们,容我们服事埃及人吗?因为服事埃及人比死在旷野还好。」
\par 13 摩西对百姓说:「不要惧怕,只管站住!看耶和华今天向你们所要施行的救恩。因为,你们今天所看见的埃及人必永远不再看见了。
\par 14 耶和华必为你们争战;你们只管静默,不要作声。」
\par 15 耶和华对摩西说:「你为什麽向我哀求呢?你吩咐以色列人往前走。
\par 16 你举手向海伸杖,把水分开。以色列人要下海中走乾地。
\par 17 我要使埃及人的心刚硬,他们就跟著下去。我要在法老和他的全军、车辆、马兵上得荣耀。
\par 18 我在法老和他的车辆、马兵上得荣耀的时候,埃及人就知道我是耶和华了。」
\par 19 在以色列营前行走神的使者,转到他们後边去;云柱也从他们前边转到他们後边立住。
\par 20 在埃及营和以色列营中间有云柱,一边黑暗,一边发光,终夜两下不得相近。
\par 21 摩西向海伸杖,耶和华便用大东风,使海水一夜退去,水便分开,海就成了乾地。
\par 22 以色列人下海中走乾地,水在他们的左右作了墙垣。
\par 23 埃及人追赶他们,法老一切的马匹、车辆,和马兵都跟著下到海中。
\par 24 到了晨更的时候,耶和华从云火柱中向埃及的军兵观看,使埃及的军兵混乱了;
\par 25 又使他们的车轮脱落,难以行走,以致埃及人说:「我们从以色列人面前逃跑吧!因耶和华为他们攻击我们了。」
\par 26 耶和华对摩西说:「你向海伸杖,叫水仍合在埃及人并他们的车辆、马兵身上。」
\par 27 摩西就向海伸杖,到了天一亮,海水仍旧复原。埃及人避水逃跑的时候,耶和华把他们推翻在海中,
\par 28 水就回流,淹没了车辆和马兵。那些跟著以色列人下海法老的全军,连一个也没有剩下。
\par 29 以色列人却在海中走乾地;水在他们的左右作了墙垣。
\par 30 当日,耶和华这样拯救以色列人脱离埃及人的手,以色列人看见埃及人的死尸都在海边了。
\par 31 以色列人看见耶和华向埃及人所行的大事,就敬畏耶和华,又信服他和他的仆人摩西。

\chapter{15}

\par 1 那时,摩西和以色列人向耶和华唱歌说:我要向耶和华歌唱,因他大大战胜,将马和骑马的投在海中。
\par 2 耶和华是我的力量,我的诗歌,也成了我的拯救。这是我的神,我要赞美他,是我父亲的神,我要尊崇他。
\par 3 耶和华是战士;他的名是耶和华。
\par 4 法老的车辆、军兵,耶和华已抛在海中;他特选的军长都沉於红海。
\par 5 深水淹没他们;他们如同石头坠到深处。
\par 6 耶和华啊,你的右手施展能力,显出荣耀;耶和华啊,你的右手摔碎仇敌。
\par 7 你大发威严,推翻那些起来攻击你的;你发出烈怒如火,烧灭他们像烧碎一样。
\par 8 你发鼻中的气,水便聚起成堆,大水直立如垒,海中的深水凝结。
\par 9 仇敌说:我要追赶,我要追上;我要分掳物,我要在他们身上称我的心愿。我要拔出刀来,亲手杀灭他们。
\par 10 你叫风一吹,海就把他们淹没;他们如铅沉在大水之中。
\par 11 耶和华啊,众神之中,谁能像你?谁能像你至圣至荣,可颂可畏,施行奇事?
\par 12 你伸出右手,地便吞灭他们。
\par 13 你凭慈爱领了你所赎的百姓;你凭能力引他们到了你的圣所。
\par 14 外邦人听见就发颤;疼痛抓住非利士的居民。
\par 15 那时,以东的族长惊惶,摩押的英雄被战兢抓住,迦南的居民心都消化了。
\par 16 惊骇恐惧临到他们。耶和华啊,因你膀臂的大能,他们如石头寂然不动,等候你的百姓过去,等候你所赎的百姓过去。
\par 17 你要将他们领进去,栽於你产业的山上,耶和华啊,就是你为自己所造的住处;主啊,就是你手所建立的圣所。
\par 18 耶和华必作王,直到永永远远!
\par 19 法老的马匹、车辆,和马兵下到海中,耶和华使海水回流,淹没他们;惟有以色列人在海中走乾地。
\par 20 亚伦的姊姊,女先知米利暗,手里拿著鼓;众妇女也跟他出去拿鼓跳舞。
\par 21 米利暗应声说:你们要歌颂耶和华,因他大大战胜,将马和骑马的投在海中。
\par 22 摩西领以色列人从红海往前行,到了书珥的旷野,在旷野走了三天,找不著水。
\par 23 到了玛拉,不能喝那里的水;因为水苦,所以那地名叫玛拉。
\par 24 百姓就向摩西发怨言,说:「我们喝什麽呢?」
\par 25 摩西呼求耶和华,耶和华指示他一棵树。他把树丢在水里,水就变甜了。耶和华在那里为他们定了律例、典章,在那里试验他们;
\par 26 又说:「你若留意听耶和华你神的话,又行我眼中看为正的事,留心听我的诫命,守我一切的律例,我就不将所加与埃及人的疾病加在你身上,因为我耶和华是医治你的。」
\par 27 他们到了以琳,在那里有十二股水泉,七十棵棕树;他们就在那里的水边安营。

\chapter{16}

\par 1 以色列全会众从以琳起行,在出埃及後第二个月十五日到了以琳和西乃中间、汛的旷野。
\par 2 以色列全会众在旷野向摩西、亚伦发怨言,
\par 3 说:「巴不得我们早死在埃及地、耶和华的手下;那时我们坐在肉锅旁边,吃得饱足。你们将我们领出来,到这旷野,是要叫这全会众都饿死啊!」
\par 4 耶和华对摩西说:「我要将粮食从天降给你们。百姓可以出去,每天收每天的分,我好试验他们遵不遵我的法度。
\par 5 到第六天,他们要把所收进来的预备好了,比每天所收的多一倍。」
\par 6 摩西、亚伦对以色列众人说:「到了晚上,你们要知道是耶和华将你们从埃及地领出来的。
\par 7 早晨,你们要看见耶和华的荣耀,因为耶和华听见你们向他所发的怨言了。我们算什麽,你们竟向我们发怨言呢?」
\par 8 摩西又说:「耶和华晚上必给你们肉吃,早晨必给你们食物得饱;因为你们向耶和华发的怨言,他都听见了。我们算什麽,你们的怨言不是向我们发的,乃是向耶和华发的。」
\par 9 摩西对亚伦说:「你告诉以色列全会众说:『你们就近耶和华面前,因为他已经听见你们的怨言了。』」
\par 10 亚伦正对以色列全会众说话的时候,他们向旷野观看,不料,耶和华的荣光在云中显现。
\par 11 耶和华晓谕摩西说:
\par 12 「我已经听见以色列人的怨言。你告诉他们说:『到黄昏的时候,你们要吃肉,早晨必有食物得饱,你们就知道我是耶和华你们的神。』」
\par 13 到了晚上,有鹌鹑飞来,遮满了营;早晨在营四围的地上有露水。
\par 14 露水上升之後,不料,野地面上有如白霜的小圆物。
\par 15 以色列人看见,不知道是什麽,就彼此对问说:「这是什麽呢?」摩西对他们说:「这就是耶和华给你们吃的食物。
\par 16 耶和华所吩咐的是这样:你们要按著各人的饭量,为帐棚里的人,按著人数收起来,各拿一俄梅珥。」
\par 17 以色列人就这样行;有多收的,有少收的。
\par 18 及至用俄梅珥量一量,多收的也没有余,少收的也没有缺;各人按著自己的饭量收取。
\par 19 摩西对他们说:「所收的,不许什麽人留到早晨。」
\par 20 然而他们不听摩西的话,内中有留到早晨的,就生虫变臭了;摩西便向他们发怒。
\par 21 他们每日早晨,按著各人的饭量收取,日头一发热,就消化了。
\par 22 到第六天,他们收了双倍的食物,每人两俄梅珥。会众的官长来告诉摩西;
\par 23 摩西对他们说:「耶和华这样说:『明天是圣安息日,是向耶和华守的圣安息日。你们要烤的就烤了,要煮的就煮了,所剩下的都留到早晨。』」
\par 24 他们就照摩西的吩咐留到早晨,也不臭,里头也没有虫子。
\par 25 摩西说:「你们今天吃这个吧!因为今天是向耶和华守的安息日;你们在田野必找不著了。
\par 26 六天可以收取,第七天乃是安息日,那一天必没有了。」
\par 27 第七天,百姓中有人出去收,什麽也找不著。
\par 28 耶和华对摩西说:「你们不肯守我的诫命和律法,要到几时呢?
\par 29 你们看!耶和华既将安息日赐给你们,所以第六天他赐给你们两天的食物,第七天各人要住在自己的地方,不许什麽人出去。」
\par 30 於是百姓第七天安息了。
\par 31 这食物,以色列家叫吗哪;样子像芫荽子,颜色是白的,滋味如同搀蜜的薄饼。
\par 32 摩西说:「耶和华所吩咐的是这样:『要将一满俄梅珥(俄梅珥就是伊法十分之一)吗哪留到世世代代,使後人可以看见我当日将你们领出埃及地,在旷野所给你们吃的食物。』」
\par 33 摩西对亚伦说:「你拿一个罐子,盛一满俄梅珥吗哪,存在耶和华面前,要留到世世代代。」
\par 34 耶和华怎麽吩咐摩西,亚伦就怎麽行,把吗哪放在法柜前存留。
\par 35 以色列人吃吗哪共四十年,直到进了有人居住之地,就是迦南的境界。

\chapter{17}

\par 1 以色列全会众都遵耶和华的吩咐,按著站口从汛的旷野往前行,在利非订安营。百姓没有水喝,
\par 2 所以与摩西争闹,说:「给我们水喝吧!」摩西对他们说:「你们为什麽与我争闹?为什麽试探耶和华呢?」
\par 3 百姓在那里甚渴,要喝水,就向摩西发怨言,说:「你为什麽将我们从埃及领出来,使我们和我们的儿女并牲畜都渴死呢?」
\par 4 摩西就呼求耶和华说:「我向这百姓怎样行呢?他们几乎要拿石头打死我。」
\par 5 耶和华对摩西说:「你手里拿著你先前击打河水的杖,带领以色列的几个长老,从百姓面前走过去。
\par 6 我必在何烈的磐石那里,站在你面前。你要击打磐石,从磐石里必有水流出来,使百姓可以喝。」摩西就在以色列的长老眼前这样行了。
\par 7 他给那地方起名叫玛撒(就是试探的意思),又叫米利巴(就是争闹的意思);因以色列人争闹,又因他们试探耶和华,说:「耶和华是在我们中间不是?」
\par 8 那时,亚玛力人来在利非订,和以色列人争战。
\par 9 摩西对约书亚说:「你为我们选出人来,出去和亚玛力人争战。明天我手里要拿著神的杖,站在山顶上。」
\par 10 於是约书亚照著摩西对他所说的话行,和亚玛力人争战。摩西、亚伦,与户珥都上了山顶。
\par 11 摩西何时举手,以色列人就得胜,何时垂手,亚玛力人就得胜。
\par 12 但摩西的手发沉,他们就搬石头来,放在他以下,他就坐在上面。亚伦与户珥扶著他的手,一个在这边,一个在那边,他的手就稳住,直到日落的时候。
\par 13 约书亚用刀杀了亚玛力王和他的百姓。
\par 14 耶和华对摩西说:「我要将亚玛力的名号从天下全然涂抹了;你要将这话写在书上作纪念,又念给约书亚听。」
\par 15 摩西筑了一座坛,起名叫「耶和华尼西」(就是耶和华是我旌旗的意思),
\par 16 又说:「耶和华已经起了誓,必世世代代和亚玛力人争战。」

\chapter{18}

\par 1 摩西的岳父,米甸祭司叶忒罗,听见神为摩西和神的百姓以色列所行的一切事,就是耶和华将以色列从埃及领出来的事,
\par 2 便带著摩西的妻子西坡拉,就是摩西从前打发回去的,
\par 3 又带著西坡拉的两个儿子,一个名叫革舜,因为摩西说:「我在外邦作了寄居的」;
\par 4 一个名叫以利以谢,因为他说:「我父亲的神帮助了我,救我脱离法老的刀。」
\par 5 摩西的岳父叶忒罗带著摩西的妻子和两个儿子来到神的山,就是摩西在旷野安营的地方。
\par 6 他对摩西说:「我是你岳父叶忒罗,带著你的妻子和两个儿子来到你这里。」
\par 7 摩西迎接他的岳父,向他下拜,与他亲嘴,彼此问安,都进了帐棚。
\par 8 摩西将耶和华为以色列的缘故向法老和埃及人所行的一切事,以及路上所遭遇的一切艰难,并耶和华怎样搭救他们,都述说与他岳父听。
\par 9 叶忒罗因耶和华待以色列的一切好处,就是拯救他们脱离埃及人的手,便甚欢喜。
\par 10 叶忒罗说:「耶和华是应当称颂的;他救了你们脱离埃及人和法老的手,将这百姓从埃及人的手下救出来。
\par 11 我现今在埃及人向这百姓发狂傲的事上得知,耶和华比万神都大。」
\par 12 摩西的岳父叶忒罗把燔祭和平安祭献给神。亚伦和以色列的众长老都来了,与摩西的岳父在神面前吃饭。
\par 13 第二天,摩西坐著审判百姓,百姓从早到晚都站在摩西的左右。
\par 14 摩西的岳父看见他向百姓所做的一切事,就说:「你向百姓做的是什麽事呢?你为什麽独自坐著,众百姓从早到晚都站在你的左右呢?」
\par 15 摩西对岳父说:「这是因百姓到我这里来求问神。
\par 16 他们有事的时候就到我这里来,我便在两造之间施行审判;我又叫他们知道神的律例和法度。」
\par 17 摩西的岳父说:「你这做的不好。
\par 18 你和这些百姓必都疲惫;因为这事太重,你独自一人办理不了。
\par 19 现在你要听我的话。我为你出个主意,愿神与你同在。你要替百姓到神面前,将案件奏告神;
\par 20 又要将律例和法度教训他们,指示他们当行的道,当做的事;
\par 21 并要从百姓中拣选有才能的人,就是敬畏神、诚实无妄、恨不义之财的人,派他们作千夫长、百夫长、五十夫长、十夫长,管理百姓,
\par 22 叫他们随时审判百姓,大事都要呈到你这里,小事他们自己可以审判。这样,你就轻省些,他们也可以同当此任。
\par 23 你若这样行,神也这样吩咐你,你就能受得住,这百姓也都平平安安归回他们的住处。」
\par 24 於是,摩西听从他岳父的话,按著他所说的去行。
\par 25 摩西从以色列人中拣选了有才能的人,立他们为百姓的首领,作千夫长、百夫长、五十夫长、十夫长。
\par 26 他们随时审判百姓,有难断的案件就呈到摩西那里,但各样小事他们自己审判。
\par 27 此後,摩西让他的岳父去,他就往本地去了。

\chapter{19}

\par 1 以色列人出埃及地以後,满了三个月的那一天,就来到西乃的旷野。
\par 2 他们离了利非订,来到西乃的旷野,就在那里的山下安营。
\par 3 摩西到神那里,耶和华从山上呼唤他说:「你要这样告诉雅各家,晓谕以色列人说:
\par 4 『我向埃及人所行的事,你们都看见了,且看见我如鹰将你们背在翅膀上,带来归我。
\par 5 如今你们若实在听从我的话,遵守我的约,就要在万民中作属我的子民,因为全地都是我的。
\par 6 你们要归我作祭司的国度,为圣洁的国民。』这些话你要告诉以色列人。」
\par 7 摩西去召了民间的长老来,将耶和华所吩咐他的话都在他们面前陈明。
\par 8 百姓都同声回答说:「凡耶和华所说的,我们都要遵行。」摩西就将百姓的话回覆耶和华。
\par 9 耶和华对摩西说:「我要在密云中临到你那里,叫百姓在我与你说话的时候可以听见,也可以永远信你了。」於是,摩西将百姓的话奏告耶和华。
\par 10 耶和华又对摩西说:「你往百姓那里去,叫他们今天明天自洁,又叫他们洗衣服。
\par 11 到第三天要预备好了,因为第三天耶和华要在众百姓眼前降临在西乃山上。
\par 12 你要在山的四围给百姓定界限,说:『你们当谨慎,不可上山去,也不可摸山的边界;凡摸这山的,必要治死他。
\par 13 不可用手摸他,必用石头打死,或用箭射透;无论是人是牲畜,都不得活。到角声拖长的时候,他们才可到山根来。』」
\par 14 摩西下山往百姓那里去,叫他们自洁,他们就洗衣服。
\par 15 他对百姓说:「到第三天要预备好了。不可亲近女人。」
\par 16 到了第三天早晨,在山上有雷轰、闪电,和密云,并且角声甚大,营中的百姓尽都发颤。
\par 17 摩西率领百姓出营迎接神,都站在山下。
\par 18 西乃全山冒烟,因为耶和华在火中降於山上。山的烟气上腾,如烧窑一般,遍山大大的震动。
\par 19 角声渐渐地高而又高,摩西就说话,神有声音答应他。
\par 20 耶和华降临在西乃山顶上,耶和华召摩西上山顶,摩西就上去。
\par 21 耶和华对摩西说:「你下去嘱咐百姓,不可闯过来到我面前观看,恐怕他们有多人死亡;
\par 22 又叫亲近我的祭司自洁,恐怕我忽然出来击杀他们。」
\par 23 摩西对耶和华说:「百姓不能上西乃山,因为你已经嘱咐我们说:『要在山的四围定界限,叫山成圣。』」
\par 24 耶和华对他说:「下去吧,你要和亚伦一同上来;只是祭司和百姓不可闯过来上到我面前,恐怕我忽然出来击杀他们。」
\par 25 於是摩西下到百姓那里告诉他们。

\chapter{20}

\par 1 神吩咐这一切的话说:
\par 2 「我是耶和华你的神,曾将你从埃及地为奴之家领出来。
\par 3 「除了我以外,你不可有别的神。
\par 4 「不可为自己雕刻偶像,也不可做什麽形像彷佛上天、下地,和地底下、水中的百物。
\par 5 不可跪拜那些像,也不可事奉他,因为我耶和华你的神是忌邪的神。恨我的,我必追讨他的罪,自父及子,直到三四代;
\par 6 爱我、守我诫命的,我必向他们发慈爱,直到千代。
\par 7 「不可妄称耶和华你神的名;因为妄称耶和华名的,耶和华必不以他为无罪。
\par 8 「当记念安息日,守为圣日。
\par 9 六日要劳碌做你一切的工,
\par 10 但第七日是向耶和华你神当守的安息日。这一日你和你的儿女、仆婢、牲畜,并你城里寄居的客旅,无论何工都不可做;
\par 11 因为六日之内,耶和华造天、地、海,和其中的万物,第七日便安息,所以耶和华赐福与安息日,定为圣日。
\par 12 「当孝敬父母,使你的日子在耶和华你神所赐你的地上得以长久。
\par 13 「不可杀人。
\par 14 「不可奸淫。
\par 15 「不可偷盗。
\par 16 「不可作假见证陷害人。
\par 17 「不可贪恋人的房屋;也不可贪恋人的妻子、仆婢、牛驴,并他一切所有的。」
\par 18 众百姓见雷轰、闪电、角声、山上冒烟,就都发颤,远远的站立,
\par 19 对摩西说:「求你和我们说话,我们必听;不要神和我们说话,恐怕我们死亡。」
\par 20 摩西对百姓说:「不要惧怕;因为神降临是要试验你们,叫你们时常敬畏他,不至犯罪。」
\par 21 於是百姓远远的站立;摩西就挨近神所在的幽暗之中。
\par 22 耶和华对摩西说:「你要向以色列人这样说:『你们自己看见我从天上和你们说话了。
\par 23 你们不可做什麽神像与我相配,不可为自己做金银的神像。
\par 24 你要为我筑土坛,在上面以牛羊献为燔祭和平安祭。凡记下我名的地方,我必到那里赐福给你。
\par 25 你若为我筑一座石坛,不可用凿成的石头,因你在上头一动家具,就把坛污秽了。
\par 26 你上我的坛,不可用台阶,免得露出你的下体来。』」

\chapter{21}

\par 1 「你在百姓面前所要立的典章是这样:
\par 2 你若买希伯来人作奴仆,他必服事你六年;第七年他可以自由,白白地出去。
\par 3 他若孤身来就可以孤身去;他若有妻,他的妻就可以同他出去。
\par 4 他主人若给他妻子,妻子给他生了儿子或女儿,妻子和儿女要归主人,他要独自出去。
\par 5 倘或奴仆明说:『我爱我的主人和我的妻子儿女,不愿意自由出去。』
\par 6 他的主人就要带他到审判官(或作:神;下同)那里,又要带他到门前,靠近门框,用锥子穿他的耳朵,他就永远服事主人。
\par 7 「人若卖女儿作婢女,婢女不可像男仆那样出去。
\par 8 主人选定他归自己,若不喜欢他,就要许他赎身;主人既然用诡诈待他,就没有权柄卖给外邦人。
\par 9 主人若选定他给自己的儿子,就当待他如同女儿。
\par 10 若另娶一个,那女子的吃食、衣服,并好合的事,仍不可减少。
\par 11 若不向他行这三样,他就可以不用钱赎,白白地出去。」
\par 12 「打人以致打死的,必要把他治死。
\par 13 人若不是埋伏著杀人,乃是神交在他手中,我就设下一个地方,他可以往那里逃跑。
\par 14 人若任意用诡计杀了他的邻舍,就是逃到我的坛那里,也当捉去把他治死。
\par 15 「打父母的,必要把他治死。
\par 16 「拐带人口,或是把人卖了,或是留在他手下,必要把他治死。
\par 17 「咒骂父母的,必要把他治死。
\par 18 「人若彼此相争,这个用石头或是拳头打那个,尚且不至於死,不过躺卧在床,
\par 19 若再能起来扶杖而出,那打他的可算无罪;但要将他耽误的工夫用钱赔补,并要将他全然医好。
\par 20 「人若用棍子打奴仆或婢女,立时死在他的手下,他必要受刑。
\par 21 若过一两天才死,就可以不受刑,因为是用钱买的。
\par 22 「人若彼此争斗,伤害有孕的妇人,甚至坠胎,随後却无别害,那伤害他的,总要按妇人的丈夫所要的,照审判官所断的,受罚。
\par 23 若有别害,就要以命偿命,
\par 24 以眼还眼,以牙还牙,以手还手,以脚还脚,
\par 25 以烙还烙,以伤还伤,以打还打。
\par 26 「人若打坏了他奴仆或是婢女的一只眼,就要因他的眼放他去得以自由。
\par 27 若打掉了他奴仆或是婢女的一个牙,就要因他的牙放他去得以自由。」
\par 28 「牛若触死男人或是女人,总要用石头打死那牛,却不可吃他的肉;牛的主人可算无罪。
\par 29 倘若那牛素来是触人的,有人报告了牛主,他竟不把牛拴著,以致把男人或是女人触死,就要用石头打死那牛,牛主也必治死。
\par 30 若罚他赎命的价银,他必照所罚的赎他的命。
\par 31 牛无论触了人的儿子或是女儿,必照这例办理。
\par 32 牛若触了奴仆或是婢女,必将银子三十舍客勒给他们的主人,也要用石头把牛打死。
\par 33 「人若敞著井口,或挖井不遮盖,有牛或驴掉在里头,
\par 34 井主要拿钱赔还本主人,死牲畜要归自己。
\par 35 「这人的牛若伤了那人的牛,以至於死,他们要卖了活牛,平分价值,也要平分死牛。
\par 36 人若知道这牛素来是触人的,主人竟不把牛拴著,他必要以牛还牛,死牛要归自己。」

\chapter{22}

\par 1 「人若偷牛或羊,无论是宰了,是卖了,他就要以五牛赔一牛,四羊赔一羊。
\par 2 人若遇见贼挖窟窿,把贼打了,以至於死,就不能为他有流血的罪。
\par 3 若太阳已经出来,就为他有流血的罪。贼若被拿,总要赔还。若他一无所有,就要被卖,顶他所偷的物。
\par 4 若他所偷的,或牛,或驴,或羊,仍在他手下存活,他就要加倍赔还。
\par 5 「人若在田间或在葡萄园里放牲畜,任凭牲畜上别人的田里去吃,就必拿自己田间上好的和葡萄园上好的赔还。
\par 6 「若点火焚烧荆棘,以致将别人堆积的禾捆,站著的禾稼,或是田园,都烧尽了,那点火的必要赔还。
\par 7 「人若将银钱或家具交付邻舍看守,这物从那人的家被偷去,若把贼找到了,贼要加倍赔还;
\par 8 若找不到贼,那家主必就近审判官,要看看他拿了原主的物件没有。
\par 9 「两个人的案件,无论是为什麽过犯,或是为牛,为驴,为羊,为衣裳,或是为什麽失掉之物,有一人说:『这是我的』,两造就要将案件禀告审判官,审判官定谁有罪,谁就要加倍赔还。
\par 10 「人若将驴,或牛,或羊,或别的牲畜,交付邻舍看守,牲畜或死,或受伤,或被赶去,无人看见,
\par 11 那看守的人要凭著耶和华起誓,手里未曾拿邻舍的物,本主就要罢休,看守的人不必赔还。
\par 12 牲畜若从看守的那里被偷去,他就要赔还本主;
\par 13 若被野兽撕碎,看守的要带来当作证据,所撕的不必赔还。
\par 14 「人若向邻舍借什麽,所借的或受伤,或死,本主没有同在一处,借的人总要赔还;
\par 15 若本主同在一处,他就不必赔还;若是雇的,也不必赔还,本是为雇价来的。」
\par 16 「人若引诱没有受聘的处女,与他行淫,他总要交出聘礼,娶他为妻。
\par 17 若女子的父亲决不肯将女子给他,他就要按处女的聘礼,交出钱来。
\par 18 「行邪术的女人,不可容他存活。
\par 19 「凡与兽淫合的,总要把他治死。
\par 20 「祭祀别神,不单单祭祀耶和华的,那人必要灭绝。
\par 21 「不可亏负寄居的,也不可欺压他,因为你们在埃及地也作过寄居的。
\par 22 不可苦待寡妇和孤儿;
\par 23 若是苦待他们一点,他们向我一哀求,我总要听他们的哀声,
\par 24 并要发烈怒,用刀杀你们,使你们的妻子为寡妇,儿女为孤儿。
\par 25 「我民中有贫穷人与你同住,你若借钱给他,不可如放债的向他取利。
\par 26 你即或拿邻舍的衣服作当头,必在日落以先归还他;
\par 27 因他只有这一件当盖头,是他盖身的衣服,若是没有,他拿什麽睡觉呢?他哀求我,我就应允,因为我是有恩惠的。
\par 28 「不可毁谤神;也不可毁谤你百姓的官长。
\par 29 「你要从你庄稼中的谷和酒 中滴出来的酒拿来献上,不可迟延。「你要将头生的儿子归给我。
\par 30 你牛羊头生的,也要这样;七天当跟著母,第八天要归给我。
\par 31 「你要在我面前为圣洁的人。因此,田间被野兽撕裂牲畜的肉,你们不可吃,要丢给狗吃。」

\chapter{23}

\par 1 「不可随夥布散谣言;不可与恶人连手妄作见证。
\par 2 不可随众行恶;不可在争讼的事上随众偏行,作见证屈枉正直;
\par 3 也不可在争讼的事上偏护穷人。
\par 4 「若遇见你仇敌的牛或驴失迷了路,总要牵回来交给他。
\par 5 若看见恨你人的驴压卧在重驮之下,不可走开,务要和驴主一同抬开重驮。
\par 6 「不可在穷人争讼的事上屈枉正直。
\par 7 当远离虚假的事。不可杀无辜和有义的人,因我必不以恶人为义。
\par 8 不可受贿赂;因为贿赂能叫明眼人变瞎了,又能颠倒义人的话。
\par 9 「不可欺压寄居的;因为你们在埃及地作过寄居的,知道寄居的心。」
\par 10 「六年你要耕种田地,收藏土产,
\par 11 只是第七年要叫地歇息,不耕不种,使你民中的穷人有吃的;他们所剩下的,野兽可以吃。你的葡萄园和橄榄园也要照样办理。
\par 12 「六日你要做工,第七日要安息,使牛、驴可以歇息,并使你婢女的儿子和寄居的都可以舒畅。
\par 13 「凡我对你们说的话,你们要谨守。别神的名,你不可提,也不可从你口中传说。」
\par 14 「一年三次,你要向我守节。
\par 15 你要守除酵节,照我所吩咐你的,在亚笔月内所定的日期,吃无酵饼七天。谁也不可空手朝见我,因为你是这月出了埃及。
\par 16 又要守收割节,所收的是你田间所种、劳碌得来初熟之物。并在年底收藏,要守收藏节。
\par 17 一切的男丁要一年三次朝见主耶和华。
\par 18 「不可将我祭牲的血和有酵的饼一同献上;也不可将我节上祭牲的脂油留到早晨。
\par 19 「地里首先初熟之物要送到耶和华你神的殿。「不可用山羊羔母的奶煮山羊羔。」
\par 20 「看哪,我差遣使者在你前面,在路上保护你,领你到我所预备的地方去。
\par 21 他是奉我名来的;你们要在他面前谨慎,听从他的话,不可惹(或作:违背)他,因为他必不赦免你们的过犯。
\par 22 「你若实在听从他的话,照著我一切所说的去行,我就向你的仇敌作仇敌,向你的敌人作敌人。
\par 23 「我的使者要在你前面行,领你到亚摩利人、赫人、比利洗人、迦南人、希未人、耶布斯人那里去,我必将他们剪除。
\par 24 你不可跪拜他们的神,不可事奉他,也不可效法他们的行为,却要把神像尽行拆毁,打碎他们的柱像。
\par 25 你们要事奉耶和华你们的神,他必赐福与你的粮与你的水,也必从你们中间除去疾病。
\par 26 你境内必没有坠胎的,不生产的。我要使你满了你年日的数目。
\par 27 凡你所到的地方,我要使那里的众民在你面前惊骇,扰乱,又要使你一切仇敌转背逃跑。
\par 28 我要打发黄蜂飞在你前面,把希未人、迦南人、赫人撵出去。
\par 29 我不在一年之内将他们从你面前撵出去,恐怕地成为荒凉,野地的兽多起来害你。
\par 30 我要渐渐地将他们从你面前撵出去,等到你的人数加多,承受那地为业。
\par 31 我要定你的境界,从红海直到非利士海,又从旷野直到大河。我要将那地的居民交在你手中,你要将他们从你面前撵出去。
\par 32 不可和他们并他们的神立约。
\par 33 他们不可住在你的地上,恐怕他们使你得罪我。你若事奉他们的神,这必成为你的网罗。」

\chapter{24}

\par 1 耶和华对摩西说:「你和亚伦、拿答、亚比户,并以色列长老中的七十人,都要上到我这里来,远远的下拜。
\par 2 惟独你可以亲近耶和华;他们却不可亲近;百姓也不可和你一同上来。」
\par 3 摩西下山,将耶和华的命令典章都述说与百姓听。众百姓齐声说:「耶和华所吩咐的,我们都必遵行。」
\par 4 摩西将耶和华的命令都写上。清早起来,在山下筑一座坛,按以色列十二支派立十二根柱子,
\par 5 又打发以色列人中的少年人去献燔祭,又向耶和华献牛为平安祭。
\par 6 摩西将血一半盛在盆中,一半洒在坛上;
\par 7 又将约书念给百姓听。他们说:「耶和华所吩咐的,我们都必遵行。」
\par 8 摩西将血洒在百姓身上,说:「你看!这是立约的血,是耶和华按这一切话与你们立约的凭据。」
\par 9 摩西、亚伦、拿答、亚比户,并以色列长老中的七十人,都上了山。
\par 10 他们看见以色列的神,他脚下彷佛有平铺的蓝宝石,如同天色明净。
\par 11 他的手不加害在以色列的尊者身上。他们观看神;他们又吃又喝。
\par 12 耶和华对摩西说:「你上山到我这里来,住在这里,我要将石版并我所写的律法和诫命赐给你,使你可以教训百姓。」
\par 13 摩西和他的帮手约书亚起来,上了神的山。
\par 14 摩西对长老说:「你们在这里等著,等到我们再回来,有亚伦、户珥与你们同在。凡有争讼的,都可以就近他们去。」
\par 15 摩西上山,有云彩把山遮盖。
\par 16 耶和华的荣耀停於西乃山;云彩遮盖山六天,第七天他从云中召摩西。
\par 17 耶和华的荣耀在山顶上,在以色列人眼前,形状如烈火。
\par 18 摩西进入云中上山,在山上四十昼夜。

\chapter{25}

\par 1 耶和华晓谕摩西说:
\par 2 「你告诉以色列人当为我送礼物来;凡甘心乐意的,你们就可以收下归我。
\par 3 所要收的礼物:就是金、银、铜,
\par 4 蓝色、紫色、朱红色线,细麻,山羊毛,
\par 5 染红的公羊皮,海狗皮,皂荚木,
\par 6 点灯的油并做膏油和香的香料,
\par 7 红玛瑙与别样的宝石,可以镶嵌在以弗得和胸牌上。
\par 8 又当为我造圣所,使我可以住在他们中间。
\par 9 制造帐幕和其中的一切器具都要照我所指示你的样式。」
\par 10 「要用皂荚木做一柜,长二肘半,宽一肘半,高一肘半。
\par 11 要里外包上精金,四围镶上金牙边。
\par 12 也要铸四个金环,安在柜的四脚上;这边两环,那边两环。
\par 13 要用皂荚木做两根杠,用金包裹。
\par 14 要把杠穿在柜旁的环内,以便抬柜。
\par 15 这杠要常在柜的环内,不可抽出来。
\par 16 必将我所要赐给你的法版放在柜里。
\par 17 要用精金做施恩座(施恩:或作蔽罪;下同),长二肘半,宽一肘半。
\par 18 要用金子锤出两个基路伯来,安在施恩座的两头。
\par 19 这头做一个基路伯,那头做一个基路伯,二基路伯要接连一块,在施恩座的两头。
\par 20 二基路伯要高张翅膀,遮掩施恩座。基路伯要脸对脸,朝著施恩座。
\par 21 要将施恩座安在柜的上边,又将我所要赐给你的法版放在柜里。
\par 22 我要在那里与你相会,又要从法柜施恩座上二基路伯中间,和你说我所要吩咐你传给以色列人的一切事。」
\par 23 「要用皂荚木做一张桌子,长二肘,宽一肘,高一肘半。
\par 24 要包上精金,四围镶上金牙边。
\par 25 桌子的四围各做一掌宽的横梁,横梁上镶著金牙边。
\par 26 要做四个金环,安在桌子的四角上,就是桌子四脚上的四角。
\par 27 安环子的地方要挨近横梁,可以穿杠抬桌子。
\par 28 要用皂荚木做两根杠,用金包裹,以便抬桌子。
\par 29 要做桌子上的盘子、调羹,并奠酒的爵和瓶;这都要用精金制作。
\par 30 又要在桌子上,在我面前,常摆陈设饼。」
\par 31 「要用精金做一个灯台。灯台的座和干与杯、球、花,都要接连一块锤出来。
\par 32 灯台两旁要杈出六个枝子:这旁三个,那旁三个。
\par 33 这旁每枝上有三个杯,形状像杏花,有球,有花;那旁每枝上也有三个杯,形状像杏花,有球,有花。从灯台杈出来的六个枝子都是如此。
\par 34 灯台上有四个杯,形状像杏花,有球,有花。
\par 35 灯台每两个枝子以下有球与枝子接连一块。灯台出的六个枝子都是如此。
\par 36 球和枝子要接连一块,都是一块精金锤出来的。
\par 37 要做灯台的七个灯盏。祭司要点这灯,使灯光对照。
\par 38 灯台的蜡剪和蜡花盘也是要精金的。
\par 39 做灯台和这一切的器具要用精金一他连得。
\par 40 要谨慎做这些物件,都要照著在山上指示你的样式。」

\chapter{26}

\par 1 「你要用十幅幔子做帐幕。这些幔子要用捻的细麻和蓝色、紫色、朱红色线制造,并用巧匠的手工绣上基路伯。
\par 2 每幅幔子要长二十八肘,宽四肘,幔子都要一样的尺寸。
\par 3 这五幅幔子要幅幅相连;那五幅幔子也要幅幅相连。
\par 4 在这相连的幔子末幅边上要做蓝色的钮扣;在那相连的幔子末幅边上也要照样做。
\par 5 要在这相连的幔子上做五十个钮扣;在那相连的幔子上也做五十个钮扣,都要两两相对。
\par 6 又要做五十个金钩,用钩使幔子相连,这才成了一个帐幕。
\par 7 「你要用山羊毛织十一幅幔子,作为帐幕以上的罩棚。
\par 8 每幅幔子要长三十肘,宽四肘;十一幅幔子都要一样的尺寸。
\par 9 要把五幅幔子连成一幅,又把六幅幔子连成一幅,这第六幅幔子要在罩棚的前面摺上去。
\par 10 在这相连的幔子末幅边上要做五十个钮扣;在那相连的幔子末幅边上也做五十个钮扣。
\par 11 又要做五十个铜钩,钩在钮扣中,使罩棚连成一个。
\par 12 罩棚的幔子所余那垂下来的半幅幔子,要垂在帐幕的後头。
\par 13 罩棚的幔子所余长的,这边一肘,那边一肘,要垂在帐幕的两旁,遮盖帐幕。
\par 14 又要用染红的公羊皮做罩棚的盖;再用海狗皮做一层罩棚上的顶盖。
\par 15 「你要用皂荚木做帐幕的竖板。
\par 16 每块要长十肘,宽一肘半;
\par 17 每块必有两榫相对。帐幕一切的板都要这样做。
\par 18 帐幕的南面要做板二十块。
\par 19 在这二十块板底下要做四十个带卯的银座,两卯接这块板上的两榫,两卯接那块板上的两榫。
\par 20 帐幕第二面,就是北面,也要做板二十块
\par 21 和带卯的银座四十个;这板底下有两卯,那板底下也有两卯。
\par 22 帐幕的後面,就是西面,要做板六块。
\par 23 帐幕後面的拐角要做板两块。
\par 24 板的下半截要双的,上半截要整的,直顶到第一个环子;两块都要这样做两个拐角。
\par 25 必有八块板和十六个带卯的银座;这板底下有两卯,那板底下也有两卯。
\par 26 「你要用皂荚木作闩:为帐幕这面的板作五闩,
\par 27 为帐幕那面的板做五闩,又为帐幕後面的板做五闩。
\par 28 板腰间的中闩要从这一头通到那一头。
\par 29 板要用金子包裹,又要做板上的金环套闩;闩也要用金子包裹。
\par 30 要照著在山上指示你的样式立起帐幕。
\par 31 「你要用蓝色、紫色、朱红色线,和捻的细麻织幔子,以巧匠的手工绣上基路伯。
\par 32 要把幔子挂在四根包金的皂荚木柱子上,柱子上当有金钩,柱子安在四个带卯的银座上。
\par 33 要使幔子垂在钩子下,把法柜抬进幔子内;这幔子要将圣所和至圣所隔开。
\par 34 又要把施恩座安在至圣所内的法柜上,
\par 35 把桌子安在幔子外帐幕的北面;把灯台安在帐幕的南面,彼此相对。
\par 36 「你要拿蓝色、紫色、朱红色线,和捻的细麻,用绣花的手工织帐幕的门帘。
\par 37 要用皂荚木为帘子做五根柱子,用金子包裹。柱子上当有金钩;又要为柱子用铜铸造五个带卯的座。」

\chapter{27}

\par 1 「你要用皂荚木做坛。这坛要四方的,长五肘,宽五肘,高三肘。
\par 2 要在坛的四拐角上做四个角,与坛接连一块,用铜把坛包裹。
\par 3 要做盆,收去坛上的灰,又做铲子、盘子、肉锸子、火鼎;坛上一切的器具都用铜做。
\par 4 要为坛做一个铜网,在网的四角上做四个铜环,
\par 5 把网安在坛四面的围腰板以下,使网从下达到坛的半腰。
\par 6 又要用皂荚木为坛做杠,用铜包裹。
\par 7 这杠要穿在坛两旁的环子内,用以抬坛。
\par 8 要用板做坛,坛是空的,都照著在山上指示你的样式做。」
\par 9 「你要做帐幕的院子。院子的南面要用捻的细麻做帷子,长一百肘。
\par 10 帷子的柱子要二十根,带卯的铜座二十个。柱子上的钩子和杆子都要用银子做。
\par 11 北面也当有帷子,长一百肘,帷子的柱子二十根,带卯的铜座二十个。柱子上的钩子和杆子都要用银子做。
\par 12 院子的西面当有帷子,宽五十肘,帷子的柱子十根,带卯的座十个。
\par 13 院子的东面要宽五十肘。
\par 14 门这边的帷子要十五肘,帷子的柱子三根,带卯的座三个。
\par 15 门那边的帷子也要十五肘,帷子的柱子三根,带卯的座三个。
\par 16 院子的门当有帘子,长二十肘,要拿蓝色、紫色、朱红色线,和捻的细麻,用绣花的手工织成,柱子四根,带卯的座四个。
\par 17 院子四围一切的柱子都要用银杆连络,柱子上的钩子要用银做,带卯的座要用铜做。
\par 18 院子要长一百肘,宽五十肘,高五肘,帷子要用捻的细麻做,带卯的座要用铜做。
\par 19 帐幕各样用处的器具,并帐幕一切的橛子,和院子里一切的橛子,都要用铜做。」
\par 20 「你要吩咐以色列人,把那为点灯捣成的清橄榄油拿来给你,使灯常常点著。
\par 21 在会幕中法柜前的幔外,亚伦和他的儿子,从晚上到早晨,要在耶和华面前经理这灯。这要作以色列人世世代代永远的定例。」

\chapter{28}

\par 1 「你要从以色列人中,使你的哥哥亚伦和他的儿子拿答、亚比户、以利亚撒、以他玛一同就近你,给我供祭司的职分。
\par 2 你要给你哥哥亚伦做圣衣为荣耀,为华美。
\par 3 又要吩咐一切心中有智慧的,就是我用智慧的灵所充满的,给亚伦做衣服,使他分别为圣,可以给我供祭司的职分。
\par 4 所要做的就是胸牌、以弗得、外袍、杂色的内袍、冠冕、腰带,使你哥哥亚伦和他儿子穿这圣服,可以给我供祭司的职分。
\par 5 要用金线和蓝色、紫色、朱红色线,并细麻去做。
\par 6 「他们要拿金线和蓝色、紫色、朱红色线,并捻的细麻,用巧匠的手工做以弗得。
\par 7 以弗得当有两条肩带,接上两头,使他相连。
\par 8 其上巧工织的带子,要和以弗得一样的做法,用以束上,与以弗得接连一块,要用金线和蓝色、紫色、朱红色线,并捻的细麻做成。
\par 9 要取两块红玛瑙,在上面刻以色列儿子的名字:
\par 10 六个名字在这块宝石上,六个名字在那块宝石上,都照他们生来的次序。
\par 11 要用刻宝石的手工,彷佛刻图书,按著以色列儿子的名字,刻这两块宝石,要镶在金槽上。
\par 12 要将这两块宝石安在以弗得的两条肩带上,为以色列人做纪念石。亚伦要在两肩上担他们的名字,在耶和华面前作为纪念。
\par 13 要用金子做二槽,
\par 14 又拿精金,用拧工彷佛拧绳子,做两条链子,把这拧成的链子搭在二槽上。」
\par 15 「你要用巧匠的手工做一个决断的胸牌。要和以弗得一样的做法:用金线和蓝色、紫色、朱红色线,并捻的细麻做成。
\par 16 这胸牌要四方的,叠为两层,长一虎口,宽一虎口。
\par 17 要在上面镶宝石四行:第一行是红宝石、红璧玺、红玉;
\par 18 第二行是绿宝石、蓝宝石、金钢石;
\par 19 第三行是紫玛瑙、白玛瑙、紫晶;
\par 20 第四行是水苍玉、红玛瑙、碧玉。这都要镶在金槽中。
\par 21 这些宝石都要按著以色列十二个儿子的名字,彷佛刻图书,刻十二个支派的名字。
\par 22 要在胸牌上用精金拧成如绳的链子。
\par 23 在胸牌上也要做两个金环,安在胸牌的两头。
\par 24 要把那两条拧成的金链子,穿过胸牌两头的环子。
\par 25 又要把链子的那两头接在两槽上,安在以弗得前面肩带上。
\par 26 要做两个金环,安在胸牌的两头,在以弗得里面的边上。
\par 27 又要做两个金环,安在以弗得前面两条肩带的下边,挨近相接之处,在以弗得巧工织的带子以上。
\par 28 要用蓝细带子把胸牌的环子与以弗得的环子系住,使胸牌贴在以弗得巧工织的带子上,不可与以弗得离缝。
\par 29 亚伦进圣所的时候,要将决断胸牌,就是刻著以色列儿子名字的,带在胸前,在耶和华面前常作纪念。
\par 30 又要将乌陵和土明放在决断的胸牌里;亚伦进到耶和华面前的时候,要带在胸前,在耶和华面前常将以色列人的决断牌带在胸前。」
\par 31 「你要做以弗得的外袍,颜色全是蓝的。
\par 32 袍上要为头留一领口,口的周围织出领边来,彷佛铠甲的领口,免得破裂。
\par 33 袍子周围底边上要用蓝色、紫色、朱红色线做石榴。在袍子周围的石榴中间要有金铃铛:
\par 34 一个金铃铛一个石榴,一个金铃铛一个石榴,在袍子周围的底边上。
\par 35 亚伦供职的时候要穿这袍子。他进圣所到耶和华面前,以及出来的时候,袍上的响声必被听见,使他不至於死亡。
\par 36 「你要用精金做一面牌,在上面按刻图书之法刻著『归耶和华为圣』。
\par 37 要用一条蓝细带子将牌系在冠冕的前面。
\par 38 这牌必在亚伦的额上,亚伦要担当干犯圣物条例的罪孽;这圣物是以色列人在一切的圣礼物上所分别为圣的。这牌要常在他的额上,使他们可以在耶和华面前蒙悦纳。
\par 39 要用杂色细麻线织内袍,用细麻布做冠冕,又用绣花的手工做腰带。
\par 40 「你要为亚伦的儿子做内袍、腰带、裹头巾,为荣耀,为华美。
\par 41 要把这些给你的哥哥亚伦和他的儿子穿戴,又要膏他们,将他们分别为圣,好给我供祭司的职分。
\par 42 要给他们做细麻布裤子,遮掩下体;裤子当从腰达到大腿。
\par 43 亚伦和他儿子进入会幕,或就近坛,在圣所供职的时候必穿上,免得担罪而死。这要为亚伦和他的後裔作永远的定例。」

\chapter{29}

\par 1 「你使亚伦和他儿子成圣,给我供祭司的职分,要如此行:取一只公牛犊,两只无残疾的公绵羊,
\par 2 无酵饼和调油的无酵饼,与抹油的无酵薄饼;这都要用细麦面做成。
\par 3 这饼要装在一个筐子里,连筐子带来,又把公牛和两只公绵羊牵来。
\par 4 要使亚伦和他儿子到会幕门口来,用水洗身。
\par 5 要给亚伦穿上内袍和以弗得的外袍,并以弗得,又带上胸牌,束上以弗得巧工织的带子。
\par 6 把冠冕戴在他头上,将圣冠加在冠冕上,
\par 7 就把膏油倒在他头上膏他。
\par 8 要叫他的儿子来,给他们穿上内袍。
\par 9 给亚伦和他儿子束上腰带,包上裹头巾,他们就凭永远的定例得了祭司的职任。又要将亚伦和他儿子分别为圣。
\par 10 「你要把公牛牵到会幕前,亚伦和他儿子要按手在公牛的头上。
\par 11 你要在耶和华面前,在会幕门口,宰这公牛。
\par 12 要取些公牛的血,用指头抹在坛的四角上,把血都倒在坛脚那里。
\par 13 要把一切盖脏的脂油与肝上的网子,并两个腰子和腰子上的脂油,都烧在坛上。
\par 14 只是公牛的皮、肉、粪都要用火烧在营外。这牛是赎罪祭。
\par 15 「你要牵一只公绵羊来,亚伦和他儿子要按手在这羊的头上。
\par 16 要宰这羊,把血洒在坛的周围。
\par 17 要把羊切成块子,洗净五脏和腿,连块子带头,都放在一处。
\par 18 要把全羊烧在坛上,是给耶和华献的燔祭,是献给耶和华为馨香的火祭。」
\par 19 「你要将那一只公绵羊牵来,亚伦和他儿子要按手在羊的头上。
\par 20 你要宰这羊,取点血抹在亚伦的右耳垂上和他儿子的右耳垂上,又抹在他们右手的大拇指上和右脚的大拇指上;并要把血洒在坛的四围。
\par 21 你要取点膏油和坛上的血,弹在亚伦和他的衣服上,并他儿子和他儿子的衣服上,他们和他们的衣服就一同成圣。
\par 22 「你要取这羊的脂油和肥尾巴,并盖脏的脂油与肝上的网子,两个腰子和腰子上的脂油并右腿(这是承接圣职所献的羊)。
\par 23 再从耶和华面前装无酵饼的筐子中取一个饼,一个调油的饼和一个薄饼,
\par 24 都放在亚伦的手上和他儿子的手上,作为摇祭,在耶和华面前摇一摇。
\par 25 要从他们手中接过来,烧在耶和华面前坛上的燔祭上,是献给耶和华为馨香的火祭。
\par 26 「你要取亚伦承接圣职所献公羊的胸,作为摇祭,在耶和华面前摇一摇,这就可以作你的分。
\par 27 那摇祭的胸和举祭的腿,就是承接圣职所摇的、所举的,是归亚伦和他儿子的。这些你都要成为圣,
\par 28 作亚伦和他子孙从以色列人中永远所得的分,因为是举祭。这要从以色列人的平安祭中,作为献给耶和华的举祭。
\par 29 「亚伦的圣衣要留给他的子孙,可以穿著受膏,又穿著承接圣职。
\par 30 他的子孙接续他当祭司的,每逢进会幕在圣所供职的时候,要穿七天。
\par 31 「你要将承接圣职所献公羊的肉煮在圣处。
\par 32 亚伦和他儿子要在会幕门口吃这羊的肉和筐内的饼。
\par 33 他们吃那些赎罪之物,好承接圣职,使他们成圣;只是外人不可吃,因为这是圣物。
\par 34 那承接圣职所献的肉或饼,若有一点留到早晨,就要用火烧了,不可吃这物,因为是圣物。
\par 35 「你要这样照我一切所吩咐的,向亚伦和他儿子行承接圣职的礼七天。
\par 36 每天要献公牛一只为赎罪祭。你洁净坛的时候,坛就洁净了;且要用膏抹坛,使坛成圣。
\par 37 要洁净坛七天,使坛成圣,坛就成为至圣。凡挨著坛的都成为圣。」
\par 38 「你每天所要献在坛上的就是两只一岁的羊羔;
\par 39 早晨要献这一只,黄昏的时候要献那一只。
\par 40 和这一只羊羔同献的,要用细面伊法十分之一与捣成的油一欣四分之一调和,又用酒一欣四分之一作为奠祭。
\par 41 那一只羊羔要在黄昏的时候献上,照著早晨的素祭和奠祭的礼办理,作为献给耶和华馨香的火祭。
\par 42 这要在耶和华面前、会幕门口,作你们世世代代常献的燔祭。我要在那里与你们相会,和你们说话。
\par 43 我要在那里与以色列人相会,会幕就要因我的荣耀成为圣。
\par 44 我要使会幕和坛成圣,也要使亚伦和他的儿子成圣,给我供祭司的职分。
\par 45 我要住在以色列人中间,作他们的神。
\par 46 他们必知道我是耶和华他们的神,是将他们从埃及地领出来的,为要住在他们中间。我是耶和华他们的神。」

\chapter{30}

\par 1 「你要用皂荚木做一座烧香的坛。
\par 2 这坛要四方的,长一肘,宽一肘,高二肘;坛的四角要与坛接连一块。
\par 3 要用精金把坛的上面与坛的四围,并坛的四角,包裹;又要在坛的四围镶上金牙边。
\par 4 要做两个金环安在牙子边以下,在坛的两旁,两根横撑上,作为穿杠的用处,以便抬坛。
\par 5 要用皂荚木做杠,用金包裹。
\par 6 要把坛放在法柜前的幔子外,对著法柜上的施恩座,就是我要与你相会的地方。
\par 7 亚伦在坛上要烧馨香料做的香;每早晨他收拾灯的时候,要烧这香。
\par 8 黄昏点灯的时候,他要在耶和华面前烧这香,作为世世代代常烧的香。
\par 9 在这坛上不可奉上异样的香,不可献燔祭、素祭,也不可浇上奠祭。
\par 10 亚伦一年一次要在坛的角上行赎罪之礼。他一年一次要用赎罪祭牲的血在坛上行赎罪之礼,作为世世代代的定例。这坛在耶和华面前为至圣。」
\par 11 耶和华晓谕摩西说:
\par 12 「你要按以色列人被数的,计算总数,你数的时候,他们各人要为自己的生命把赎价奉给耶和华,免得数的时候在他们中间有灾殃。
\par 13 凡过去归那些被数之人的,每人要按圣所的平,拿银子半舍客勒;这半舍客勒是奉给耶和华的礼物(一舍客勒是二十季拉)。
\par 14 凡过去归那些被数的人,从二十岁以外的,要将这礼物奉给耶和华。
\par 15 他们为赎生命将礼物奉给耶和华,富足的不可多出,贫穷的也不可少出,各人要出半舍客勒。
\par 16 你要从以色列人收这赎罪银,作为会幕的使用,可以在耶和华面前为以色列人作纪念,赎生命。」
\par 17 耶和华晓谕摩西说:
\par 18 「你要用铜做洗濯盆和盆座,以便洗濯。要将盆放在会幕和坛的中间,在盆里盛水。
\par 19 亚伦和他的儿子要在这盆里洗手洗脚。
\par 20 他们进会幕,或是就近坛前供职给耶和华献火祭的时候,必用水洗濯,免得死亡。
\par 21 他们洗手洗脚就免得死亡。这要作亚伦和他後裔世世代代永远的定例。」
\par 22 耶和华晓谕摩西说:
\par 23 「你要取上品的香料,就是流质的没药五百舍客勒,香肉桂一半,就是二百五十舍客勒,菖蒲二百五十舍客勒,
\par 24 桂皮五百舍客勒,都按著圣所的平,又取橄榄油一欣,
\par 25 按做香之法调和做成圣膏油。
\par 26 要用这膏油抹会幕和法柜,
\par 27 桌子与桌子的一切器具,灯台和灯台的器具,并香坛、
\par 28 燔祭坛,和坛的一切器具,洗濯盆和盆座。
\par 29 要使这些物成为圣,好成为至圣;凡挨著的都成为圣。
\par 30 要膏亚伦和他的儿子,使他们成为圣,可以给我供祭司的职分。
\par 31 你要对以色列人说:『这油,我要世世代代以为圣膏油。
\par 32 不可倒在别人的身上,也不可按这调和之法做与此相似的。这膏油是圣的,你们也要以为圣。
\par 33 凡调和与此相似的,或将这膏膏在别人身上的,这人要从民中剪除。』」
\par 34 耶和华吩咐摩西说:「你要取馨香的香料,就是拿他弗、施喜列、喜利比拿;这馨香的香料和净乳香各样要一般大的分量。
\par 35 你要用这些加上盐,按做香之法做成清净圣洁的香。
\par 36 这香要取点捣得极细,放在会幕内、法柜前,我要在那里与你相会。你们要以这香为至圣。
\par 37 你们不可按这调和之法为自己做香;要以这香为圣,归耶和华。
\par 38 凡做香和这香一样,为要闻香味的,这人要从民中剪除。」

\chapter{31}

\par 1 耶和华晓谕摩西说:
\par 2 「看哪,犹大支派中,户珥的孙子、乌利的儿子比撒列,我已经提他的名召他。
\par 3 我也以我的灵充满了他,使他有智慧,有聪明,有知识,能做各样的工,
\par 4 能想出巧工,用金、银、铜制造各物,
\par 5 又能刻宝石,可以镶嵌,能雕刻木头,能做各样的工。
\par 6 我分派但支派中、亚希撒抹的儿子亚何利亚伯与他同工。凡心里有智慧的,我更使他们有智慧,能做我一切所吩咐的,
\par 7 就是会幕和法柜,并其上的施恩座,与会幕中一切的器具,
\par 8 桌子和桌子的器具,精金的灯台和灯台的一切器具并香坛,
\par 9 燔祭坛和坛的一切器具,并洗濯盆与盆座,
\par 10 精工做的礼服,和祭司亚伦并他儿子用以供祭司职分的圣衣,
\par 11 膏油和为圣所用馨香的香料。他们都要照我一切所吩咐的去做。」
\par 12 耶和华晓谕摩西说:
\par 13 「你要吩咐以色列人说:『你们务要守我的安息日;因为这是你我之间世世代代的证据,使你们知道我耶和华是叫你们成为圣的。
\par 14 所以你们要守安息日,以为圣日。凡干犯这日的,必要把他治死;凡在这日做工的,必从民中剪除。
\par 15 六日要做工,但第七日是安息圣日,是向耶和华守为圣的。凡在安息日做工的,必要把他治死。』
\par 16 故此,以色列人要世世代代守安息日为永远的约。
\par 17 这是我和以色列人永远的证据;因为六日之内耶和华造天地,第七日便安息舒畅。」
\par 18 耶和华在西乃山和摩西说完了话,就把两块法版交给他,是神用指头写的石版。

\chapter{32}

\par 1 百姓见摩西迟延不下山,就大家聚集到亚伦那里,对他说:「起来!为我们做神像,可以在我们前面引路;因为领我们出埃及地的那个摩西,我们不知道他遭了什麽事。」
\par 2 亚伦对他们说:「你们去摘下你们妻子、儿女耳上的金环,拿来给我。」
\par 3 百姓就都摘下他们耳上的金环,拿来给亚伦。
\par 4 亚伦从他们手里接过来,铸了一只牛犊,用雕刻的器具做成。他们就说:「以色列啊,这是领你出埃及地的神。」
\par 5 亚伦看见,就在牛犊面前筑坛,且宣告说:「明日要向耶和华守节。」
\par 6 次日清早,百姓起来献燔祭和平安祭,就坐下吃喝,起来玩耍。
\par 7 耶和华吩咐摩西说:「下去吧,因为你的百姓,就是你从埃及地领出来的,已经败坏了。
\par 8 他们快快偏离了我所吩咐的道,为自己铸了一只牛犊,向他下拜献祭,说:『以色列啊,这就是领你出埃及地的神。』」
\par 9 耶和华对摩西说:「我看这百姓真是硬著颈项的百姓。
\par 10 你且由著我,我要向他们发烈怒,将他们灭绝,使你的後裔成为大国。」
\par 11 摩西便恳求耶和华他的神说:「耶和华啊,你为什麽向你的百姓发烈怒呢?这百姓是你用大力和大能的手从埃及地领出来的。
\par 12 为什麽使埃及人议论说『他领他们出去,是要降祸与他们,把他们杀在山中,将他们从地上除灭』?求你转意,不发你的烈怒,後悔,不降祸与你的百姓。
\par 13 求你记念你的仆人亚伯拉罕、以撒、以色列。你曾指著自己起誓说:『我必使你们的後裔像天上的星那样多,并且我所应许的这全地,必给你们的後裔,他们要永远承受为业。』」
\par 14 於是耶和华後悔,不把所说的祸降与他的百姓。
\par 15 摩西转身下山,手里拿著两块法版。这版是两面写的,这面那面都有字,
\par 16 是神的工作,字是神写的,刻在版上。
\par 17 约书亚一听见百姓呼喊的声音,就对摩西说:「在营里有争战的声音。」
\par 18 摩西说:「这不是人打胜仗的声音,也不是人打败仗的声音,我所听见的乃是人歌唱的声音。」
\par 19 摩西挨近营前就看见牛犊,又看见人跳舞,便发烈怒,把两块版扔在山下摔碎了,
\par 20 又将他们所铸的牛犊用火焚烧,磨得粉碎,撒在水面上,叫以色列人喝。
\par 21 摩西对亚伦说:「这百姓向你做了什麽?你竟使他们陷在大罪里!」
\par 22 亚伦说:「求我主不要发烈怒。这百姓专於作恶,是你知道的。
\par 23 他们对我说:『你为我们做神像,可以在我们前面引路;因为领我们出埃及地的那个摩西,我们不知道他遭了什麽事。』
\par 24 我对他们说:『凡有金环的可以摘下来』,他们就给了我。我把金环扔在火中,这牛犊便出来了。」
\par 25 摩西见百姓放肆(亚伦纵容他们,使他们在仇敌中间被讥刺),
\par 26 就站在营门中,说:「凡属耶和华的,都要到我这里来!」於是利未的子孙都到他那里聚集。
\par 27 他对他们说:「耶和华以色列的神这样说:『你们各人把刀跨在腰间,在营中往来,从这门到那门,各人杀他的弟兄与同伴并邻舍。』」
\par 28 利未的子孙照摩西的话行了。那一天百姓中被杀的约有三千。
\par 29 摩西说:「今天你们要自洁,归耶和华为圣,各人攻击他的儿子和弟兄,使耶和华赐福与你们。」
\par 30 到了第二天,摩西对百姓说:「你们犯了大罪。我如今要上耶和华那里去,或者可以为你们赎罪。」
\par 31 摩西回到耶和华那里,说:「唉!这百姓犯了大罪,为自己做了金像。
\par 32 倘或你肯赦免他们的罪,······不然,求你从你所写的册上涂抹我的名。」
\par 33 耶和华对摩西说:「谁得罪我,我就从我的册上涂抹谁的名。
\par 34 现在你去领这百姓,往我所告诉你的地方去,我的使者必在你前面引路;只是到我追讨的日子,我必追讨他们的罪。」
\par 35 耶和华杀百姓的缘故是因他们同亚伦做了牛犊。

\chapter{33}

\par 1 耶和华吩咐摩西说:「我曾起誓应许亚伯拉罕、以撒、雅各说:『要将迦南地赐给你的後裔。』现在你和你从埃及地所领出来的百姓,要从这里往那地去。
\par 2 我要差遣使者在你前面,撵出迦南人、亚摩利人、赫人、比利洗人、希未人、耶布斯人,
\par 3 领你到那流奶与蜜之地。我自己不同你们上去;因为你们是硬著颈项的百姓,恐怕我在路上把你们灭绝。」
\par 4 百姓听见这凶信就悲哀,也没有人佩戴妆饰。
\par 5 耶和华对摩西说:「你告诉以色列人说:『耶和华说:你们是硬著颈项的百姓,我若一霎时临到你们中间,必灭绝你们。现在你们要把身上的妆饰摘下来,使我可以知道怎样待你们。』」
\par 6 以色列人从住何烈山以後,就把身上的妆饰摘得乾净。
\par 7 摩西素常将帐棚支搭在营外,离营却远,他称这帐棚为会幕。凡求问耶和华的,就到营外的会幕那里去。
\par 8 当摩西出营到会幕去的时候,百姓就都起来,各人站在自己帐棚的门口,望著摩西,直等到他进了会幕。
\par 9 摩西进会幕的时候,云柱降下来,立在会幕的门前,耶和华便与摩西说话。
\par 10 众百姓看见云柱立在会幕门前,就都起来,各人在自己帐棚的门口下拜。
\par 11 耶和华与摩西面对面说话,好像人与朋友说话一般。摩西转到营里去,惟有他的帮手,一个少年人嫩的儿子约书亚不离开会幕。
\par 12 摩西对耶和华说:「你吩咐我说:『将这百姓领上去』,却没有叫我知道你要打发谁与我同去,只说:『我按你的名认识你,你在我眼前也蒙了恩。』
\par 13 我如今若在你眼前蒙恩,求你将你的道指示我,使我可以认识你,好在你眼前蒙恩。求你想到这民是你的民。」
\par 14 耶和华说:「我必亲自和你同去,使你得安息。」
\par 15 摩西说:「你若不亲自和我同去,就不要把我们从这里领上去。
\par 16 人在何事上得以知道我和你的百姓在你眼前蒙恩呢?岂不是因你与我们同去、使我和你的百姓与地上的万民有分别吗?」
\par 17 耶和华对摩西说:「你这所求的我也要行;因为你在我眼前蒙了恩,并且我按你的名认识你。」
\par 18 摩西说:「求你显出你的荣耀给我看。」
\par 19 耶和华说:「我要显我一切的恩慈,在你面前经过,宣告我的名。我要恩待谁就恩待谁;要怜悯谁就怜悯谁」;
\par 20 又说:「你不能看见我的面,因为人见我的面不能存活。」
\par 21 耶和华说:「看哪,在我这里有地方,你要站在磐石上。
\par 22 我的荣耀经过的时候,我必将你放在磐石穴中,用我的手遮掩你,等我过去,
\par 23 然後我要将我的手收回,你就得见我的背,却不得见我的面。」

\chapter{34}

\par 1 耶和华吩咐摩西说:「你要凿出两块石版,和先前你摔碎的那版一样;其上的字我要写在这版上。
\par 2 明日早晨,你要预备好了,上西乃山,在山顶上站在我面前。
\par 3 谁也不可和你一同上去,遍山都不可有人,在山根也不可叫羊群牛群吃草。」
\par 4 摩西就凿出两块石版,和先前的一样。清晨起来,照耶和华所吩咐的上西乃山去,手里拿著两块石版。
\par 5 耶和华在云中降临,和摩西一同站在那里,宣告耶和华的名。
\par 6 耶和华在他面前宣告说:「耶和华,耶和华,是有怜悯有恩典的神,不轻易发怒,并有丰盛的慈爱和诚实,
\par 7 为千万人存留慈爱,赦免罪孽、过犯,和罪恶,万不以有罪的为无罪,必追讨他的罪,自父及子,直到三、四代。」
\par 8 摩西急忙伏地下拜,
\par 9 说:「主啊,我若在你眼前蒙恩,求你在我们中间同行,因为这是硬著颈项的百姓。又求你赦免我们的罪孽和罪恶,以我们为你的产业。」
\par 10 耶和华说:「我要立约,要在百姓面前行奇妙的事,是在遍地万国中所未曾行的。在你四围的外邦人就要看见耶和华的作为,因我向你所行的是可畏惧的事。
\par 11 「我今天所吩咐你的,你要谨守。我要从你面前撵出亚摩利人、迦南人、赫人、比利洗人、希未人、耶布斯人。
\par 12 你要谨慎,不可与你所去那地的居民立约,恐怕成为你们中间的网罗;
\par 13 却要拆毁他们的祭坛,打碎他们的柱像,砍下他们的木偶。
\par 14 不可敬拜别神;因为耶和华是忌邪的神,名为忌邪者。
\par 15 只怕你与那地的居民立约,百姓随从他们的神,就行邪淫,祭祀他们的神,有人叫你,你便吃他的祭物,
\par 16 又为你的儿子娶他们的女儿为妻,他们的女儿随从他们的神,就行邪淫,使你的儿子也随从他们的神行邪淫。
\par 17 「不可为自己铸造神像。
\par 18 「你要守除酵节,照我所吩咐你的,在亚笔月内所定的日期吃无酵饼七天,因为你是这亚笔月内出了埃及。
\par 19 凡头生的都是我的;一切牲畜头生的,无论是牛是羊,公的都是我的。
\par 20 头生的驴要用羊羔代赎,若不代赎就要打折他的颈项。凡头生的儿子都要赎出来。谁也不可空手朝见我。」
\par 21 「你六日要做工,第七日要安息,虽在耕种收割的时候也要安息。
\par 22 在收割初熟麦子的时候要守七七节;又在年底要守收藏节。
\par 23 你们一切男丁要一年三次朝见主耶和华以色列的神。
\par 24 我要从你面前赶出外邦人,扩张你的境界。你一年三次上去朝见耶和华你神的时候,必没有人贪慕你的地土。」
\par 25 「你不可将我祭物的血和有酵的饼一同献上。逾越节的祭物也不可留到早晨。
\par 26 地里首先初熟之物要送到耶和华你神的殿。不可用山羊羔母的奶煮山羊羔。」
\par 27 耶和华吩咐摩西说:「你要将这些话写上,因为我是按这话与你和以色列人立约。」
\par 28 摩西在耶和华那里四十昼夜,也不吃饭也不喝水。耶和华将这约的话,就是十条诫,写在两块版上。
\par 29 摩西手里拿著两块法版下西乃山的时候,不知道自己的面皮因耶和华和他说话就发了光。
\par 30 亚伦和以色列众人看见摩西的面皮发光就怕挨近他。
\par 31 摩西叫他们来;於是亚伦和会众的官长都到他那里去,摩西就与他们说话。
\par 32 随後以色列众人都近前来,他就把耶和华在西乃山与他所说的一切话都吩咐他们。
\par 33 摩西与他们说完了话就用帕子蒙上脸。
\par 34 但摩西进到耶和华面前与他说话就揭去帕子,及至出来的时候便将耶和华所吩咐的告诉以色列人。
\par 35 以色列人看见摩西的面皮发光。摩西又用帕子蒙上脸,等到他进去与耶和华说话就揭去帕子。

\chapter{35}

\par 1 摩西招聚以色列全会众,对他们说:「这是耶和华所吩咐的话,叫你们照著行:
\par 2 六日要做工,第七日乃为圣日,当向耶和华守为安息圣日。凡这日之内做工的,必把他治死。
\par 3 当安息日,不可在你们一切的住处生火。」
\par 4 摩西对以色列全会众说:「耶和华所吩咐的是这样:
\par 5 你们中间要拿礼物献给耶和华,凡乐意献的可以拿耶和华的礼物来,就是金、银、铜,
\par 6 蓝色、紫色、朱红色线,细麻,山羊毛,
\par 7 染红的公羊皮,海狗皮,皂荚木,
\par 8 点灯的油,并做膏油和香的香料,
\par 9 红玛瑙与别样的宝石,可以镶嵌在以弗得和胸牌上。」
\par 10 「你们中间凡心里有智慧的都要来做耶和华一切所吩咐的:
\par 11 就是帐幕和帐幕的罩棚,并帐幕的盖、钩子、板、闩、柱子、带卯的座,
\par 12 柜和柜的杠,施恩座和遮掩柜的幔子,
\par 13 桌子和桌子的杠与桌子的一切器具,并陈设饼,
\par 14 灯台和灯台的器具,灯盏并点灯的油,
\par 15 香坛和坛的杠,膏油和馨香的香料,并帐幕门口的帘子,
\par 16 燔祭坛和坛的铜网,坛的杠并坛的一切器具,洗濯盆和盆座,
\par 17 院子的帷子和帷子的柱子,带卯的座和院子的门帘,
\par 18 帐幕的橛子并院子的橛子,和这两处的绳子,
\par 19 精工做的礼服和祭司亚伦并他儿子在圣所用以供祭司职分的圣衣。」
\par 20 以色列全会众从摩西面前退去。
\par 21 凡心里受感和甘心乐意的都拿耶和华的礼物来,用以做会幕和其中一切的使用,又用以做圣衣。
\par 22 凡心里乐意献礼物的,连男带女,各将金器,就是胸前针、耳环(或作:鼻环)、打印的戒指,和手钏带来献给耶和华。
\par 23 凡有蓝色、紫色、朱红色线,细麻,山羊毛,染红的公羊皮,海狗皮的,都拿了来;
\par 24 凡献银子和铜给耶和华为礼物的都拿了来;凡有皂荚木可做什麽使用的也拿了来。
\par 25 凡心中有智慧的妇女亲手纺线,把所纺的蓝色、紫色、朱红色线,和细麻都拿了来。
\par 26 凡有智慧、心里受感的妇女就纺山羊毛。
\par 27 众官长把红玛瑙和别样的宝石,可以镶嵌在以弗得与胸牌上的,都拿了来;
\par 28 又拿香料做香,拿油点灯,做膏油。
\par 29 以色列人,无论男女,凡甘心乐意献礼物给耶和华的,都将礼物拿来,做耶和华藉摩西所吩咐的一切工。
\par 30 摩西对以色列人说:「犹大支派中,户珥的孙子、乌利的儿子比撒列,耶和华已经提他的名召他,
\par 31 又以神的灵充满了他,使他有智慧、聪明、知识,能做各样的工,
\par 32 能想出巧工,用金、银、铜制造各物,
\par 33 又能刻宝石,可以镶嵌,能雕刻木头,能做各样的巧工。
\par 34 耶和华又使他,和但支派中亚希撒抹的儿子亚何利亚伯,心里灵明,能教导人。
\par 35 耶和华使他们的心满有智慧,能做各样的工,无论是雕刻的工,巧匠的工,用蓝色、紫色、朱红色线,和细麻、绣花的工,并机匠的工,他们都能做,也能想出奇巧的工。

\chapter{36}

\par 1 比撒列和亚何利亚伯,并一切心里有智慧的,就是蒙耶和华赐智慧聪明、叫他知道做圣所各样使用之工的,都要照耶和华所吩咐的做工。」
\par 2 凡耶和华赐他心里有智慧、而且受感前来做这工的,摩西把他们和比撒列并亚何利亚伯一同召来。
\par 3 这些人就从摩西收了以色列人为做圣所并圣所使用之工所拿来的礼物。百姓每早晨还把甘心献的礼物拿来。
\par 4 凡做圣所一切工的智慧人各都离开他所做的工,
\par 5 来对摩西说:「百姓为耶和华吩咐使用之工所拿来的,富富有余。」
\par 6 摩西传命,他们就在全营中宣告说:「无论男女,不必再为圣所拿什麽礼物来。」这样才拦住百姓不再拿礼物来。
\par 7 因为他们所有的材料够做一切当做的物,而且有余。
\par 8 他们中间,凡心里有智慧做工的,用十幅幔子做帐幕。这幔子是比撒列用捻的细麻和蓝色、紫色、朱红色线制造的,并用巧匠的手工绣上基路伯。
\par 9 每幅幔子长二十八肘,宽四肘,都是一样的尺寸。
\par 10 他使这五幅幔子幅幅相连,又使那五幅幔子幅幅相连;
\par 11 在这相连的幔子末幅边上做蓝色的钮扣,在那相连的幔子末幅边上也照样做;
\par 12 在这相连的幔子上做五十个钮扣,在那相连的幔子上也做五十个钮扣,都是两两相对;
\par 13 又做五十个金钩,使幔子相连。这才成了一个帐幕。
\par 14 他用山羊毛织十一幅幔子,作为帐幕以上的罩棚。
\par 15 每幅幔子长三十肘,宽四肘;十一幅幔子都是一样的尺寸。
\par 16 他把五幅幔子连成一幅,又把六幅幔子连成一幅;
\par 17 在这相连的幔子末幅边上做五十个钮扣,在那相连的幔子末幅边上也做五十个钮扣;
\par 18 又做五十个铜钩,使罩棚连成一个;
\par 19 并用染红的公羊皮做罩棚的盖,再用海狗皮做一层罩棚上的顶盖。
\par 20 他用皂荚木做帐幕的竖板。
\par 21 每块长十肘,宽一肘半;
\par 22 每块有两榫相对。帐幕一切的板都是这样做。
\par 23 帐幕的南面做板二十块。
\par 24 在这二十块板底下又做四十个带卯的银座:两卯接这块板上的两榫,两卯接那块板上的两榫。
\par 25 帐幕的第二面,就是北面,也做板二十块
\par 26 和带卯的银座四十个:这板底下有两卯,那板底下也有两卯。
\par 27 帐幕的後面,就是西面,做板六块。
\par 28 帐幕後面的拐角做板两块。
\par 29 板的下半截是双的,上半截是整的,直到第一个环子;在帐幕的两个拐角上都是这样做。
\par 30 有八块板和十六个带卯的银座,每块板底下有两卯。
\par 31 他用皂荚木做闩:为帐幕这面的板做五闩,
\par 32 为帐幕那面的板做五闩,又为帐幕後面的板做五闩,
\par 33 使板腰间的中闩从这一头通到那一头。
\par 34 用金子将板包裹,又做板上的金环套闩;闩也用金子包裹。
\par 35 他用蓝色、紫色、朱红色线,和捻的细麻织幔子,以巧匠的手工绣上基路伯。
\par 36 为幔子做四根皂荚木柱子,用金包裹,柱子上有金钩,又为柱子铸了四个带卯的银座。
\par 37 拿蓝色、紫色、朱红色线,和捻的细麻,用绣花的手工织帐幕的门帘;
\par 38 又为帘子做五根柱子和柱子上的钩子,用金子把柱顶和柱子上的杆子包裹。柱子有五个带卯的座,是铜的。

\chapter{37}

\par 1 比撒列用皂荚木做柜,长二肘半,宽一肘半,高一肘半。
\par 2 里外包上精金,四围镶上金牙边,
\par 3 又铸四个金环,安在柜的四脚上:这边两环,那边两环。
\par 4 用皂荚木做两根杠,用金包裹。
\par 5 把杠穿在柜旁的环内,以便抬柜。
\par 6 用精金做施恩座,长二肘半,宽一肘半。
\par 7 用金子锤出两个基路伯来,安在施恩座的两头,
\par 8 这头做一个基路伯,那头做一个基路伯,二基路伯接连一块,在施恩座的两头。
\par 9 二基路伯高张翅膀,遮掩施恩座;基路伯是脸对脸,朝著施恩座。
\par 10 他用皂荚木做一张桌子,长二肘,宽一肘,高一肘半,
\par 11 又包上精金,四围镶上金牙边。
\par 12 桌子的四围各做一掌宽的横梁,横梁上镶著金牙边,
\par 13 又铸了四个金环,安在桌子四脚的四角上。
\par 14 安环子的地方是挨近横梁,可以穿杠抬桌子。
\par 15 他用皂荚木做两根杠,用金包裹,以便抬桌子;
\par 16 又用精金做桌子上的器皿,就是盘子、调羹,并奠酒的瓶和爵。
\par 17 他用精金做一个灯台;这灯台的座和干,与杯、球、花,都是接连一块锤出来的。
\par 18 灯台两旁杈出六个枝子:这旁三个,那旁三个。
\par 19 这旁每枝上有三个杯,形状像杏花,有球有花;那旁每枝上也有三个杯,形状像杏花,有球有花。从灯台杈出来的六个枝子都是如此。
\par 20 灯台上有四个杯,形状像杏花,有球有花。
\par 21 灯台每两个枝子以下有球,与枝子接连一块;灯台杈出的六个枝子都是如此。
\par 22 球和枝子是接连一块,都是一块精金锤出来的。
\par 23 用精金做灯台的七个灯盏,并灯台的蜡剪和蜡花盘。
\par 24 他用精金一他连得做灯台和灯台的一切器具。
\par 25 他用皂荚木做香坛,是四方的,长一肘,宽一肘,高二肘,坛的四角与坛接连一块;
\par 26 又用精金把坛的上面与坛的四面并坛的四角包裹,又在坛的四围镶上金牙边。
\par 27 做两个金环,安在牙子边以下,在坛的两旁、两根横撑上,作为穿杠的用处,以便抬坛。
\par 28 用皂荚木做杠,用金包裹。
\par 29 又按做香之法做圣膏油和馨香料的净香。

\chapter{38}

\par 1 他用皂荚木做燔祭坛,是四方的,长五肘,宽五肘,高三肘,
\par 2 在坛的四拐角上做四个角,与坛接连一块,用铜把坛包裹。
\par 3 他做坛上的盆、铲子、盘子、肉锸子、火鼎;这一切器具都是用铜做的。
\par 4 又为坛做一个铜网,安在坛四面的围腰板以下,从下达到坛的半腰。
\par 5 为铜网的四角铸四个环子,作为穿杠的用处。
\par 6 用皂荚木做杠,用铜包裹,
\par 7 把杠穿在坛两旁的环子内,用以抬坛,并用板做坛;坛是空的。
\par 8 他用铜做洗濯盆和盆座,是用会幕门前伺候的妇人之镜子做的。
\par 9 他做帐幕的院子。院子的南面用捻的细麻做帷子,宽一百肘。
\par 10 帷子的柱子二十根,带卯的铜座二十个;柱子上的钩子和杆子都是用银子做的。
\par 11 北面也有帷子,宽一百肘。帷子的柱子二十根,带卯的铜座二十个;柱子上的钩子和杆子都是用银子做的。
\par 12 院子的西面有帷子,宽五十肘。帷子的柱子十根,带卯的座十个;柱子的钩子和杆子都是用银子做的。
\par 13 院子的东面,宽五十肘。
\par 14 门这边的帷子十五肘,那边也是一样。帷子的柱子三根,带卯的座三个。
\par 15 在门的左右各有帷子十五肘,帷子的柱子三根,带卯的座三个。
\par 16 院子四面的帷子都是用捻的细麻做的。
\par 17 柱子带卯的座是铜的,柱子上的钩子和杆子是银的,柱顶是用银子包的。院子一切的柱子都是用银杆连络的。
\par 18 院子的门帘是以绣花的手工,用蓝色、紫色、朱红色线,和捻的细麻织的,宽二十肘,高五肘,与院子的帷子相配。
\par 19 帷子的柱子四根,带卯的铜座四个;柱子上的钩子和杆子是银的;柱顶是用银子包的。
\par 20 帐幕一切的橛子和院子四围的橛子都是铜的。
\par 21 这是法柜的帐幕中利未人所用物件的总数,是照摩西的吩咐,经祭司亚伦的儿子以他玛的手数点的。
\par 22 凡耶和华所吩咐摩西的都是犹大支派户珥的孙子、乌利的儿子比撒列做的。
\par 23 与他同工的有但支派中亚希撒抹的儿子亚何利亚伯;他是雕刻匠,又是巧匠,又能用蓝色、紫色、朱红色线,和细麻绣花。
\par 24 为圣所一切工作使用所献的金子,按圣所的平,有二十九他连得并七百三十舍客勒。
\par 25 会中被数的人所出的银子,按圣所的平,有一百他连得并一千七百七十五舍客勒。
\par 26 凡过去归那些被数之人的,从二十岁以外,有六十万零三千五百五十人。按圣所的平,每人出银半舍客勒,就是一比加。
\par 27 用那一百他连得银子铸造圣所带卯的座和幔子柱子带卯的座;一百他连得共一百带卯的座,每带卯的座用一他连得。
\par 28 用那一千七百七十五舍客勒银子做柱子上的钩子,包裹柱顶并柱子上的杆子。
\par 29 所献的铜有七十他连得并二千四百舍客勒。
\par 30 用这铜做会幕门带卯的座和铜坛,并坛上的铜网和坛的一切器具,
\par 31 并院子四围带卯的座和院门带卯的座,与帐幕一切的橛子和院子四围所有的橛子。

\chapter{39}

\par 1 比撒列用蓝色、紫色、朱红色线做精致的衣服,在圣所用以供职,又为亚伦做圣衣,是照耶和华所吩咐摩西的。
\par 2 他用金线和蓝色、紫色、朱红色线,并捻的细麻做以弗得;
\par 3 把金子锤成薄片,剪出线来,与蓝色、紫色、朱红色线,用巧匠的手工一同绣上。
\par 4 又为以弗得做两条相连的肩带,接连在以弗得的两头。
\par 5 其上巧工织的带子和以弗得一样的做法,用以束上,与以弗得接连一块,是用金线和蓝色、紫色、朱红色线,并捻的细麻做的,是照耶和华所吩咐摩西的。
\par 6 又琢出两块红玛瑙,镶在金槽上,彷佛刻图书,按著以色列儿子的名字雕刻;
\par 7 将这两块宝石安在以弗得的两条肩带上,为以色列人做纪念石,是照耶和华所吩咐摩西的。
\par 8 他用巧匠的手工做胸牌,和以弗得一样的做法,用金线与蓝色、紫色、朱红色线,并捻的细麻做的。
\par 9 胸牌是四方的,叠为两层;这两层长一虎口,宽一虎口,
\par 10 上面镶著宝石四行:第一行是红宝石、红璧玺、红玉;
\par 11 第二行是绿宝石、蓝宝石、金钢石;
\par 12 第三行是紫玛瑙、白玛瑙、紫晶;
\par 13 第四行是水苍玉、红玛瑙、碧玉。这都镶在金槽中。
\par 14 这些宝石都是按著以色列十二个儿子的名字,彷佛刻图书,刻十二个支派的名字。
\par 15 在胸牌上,用精金拧成如绳子的链子。
\par 16 又做两个金槽和两个金环,安在胸牌的两头。
\par 17 把那两条拧成的金链子穿过胸牌两头的环子,
\par 18 又把链子的那两头接在两槽上,安在以弗得前面肩带上。
\par 19 做两个金环,安在胸牌的两头,在以弗得里面的边上,
\par 20 又做两个金环,安在以弗得前面两条肩带的下边,挨近相接之处,在以弗得巧工织的带子以上。
\par 21 用一条蓝细带子把胸牌的环子和以弗得的环子系住,使胸牌贴在以弗得巧工织的带子上,不可与以弗得离缝,是照耶和华所吩咐摩西的。
\par 22 他用织工做以弗得的外袍,颜色全是蓝的。
\par 23 袍上留一领口,口的周围织出领边来,彷佛铠甲的领口,免得破裂。
\par 24 在袍子底边上,用蓝色、紫色、朱红色线,并捻的细麻做石榴,
\par 25 又用精金做铃铛,把铃铛钉在袍子周围底边上的石榴中间:
\par 26 一个铃铛一个石榴,一个铃铛一个石榴,在袍子周围底边上用以供职,是照耶和华所吩咐摩西的。
\par 27 他用织成的细麻布为亚伦和他的儿子做内袍,
\par 28 并用细麻布做冠冕和华美的裹头巾,用捻的细麻布做裤子,
\par 29 又用蓝色、紫色、朱红色线,并捻的细麻,以绣花的手工做腰带,是照耶和华所吩咐摩西的。
\par 30 他用精金做圣冠上的牌,在上面按刻图书之法,刻著「归耶和华为圣」。
\par 31 又用一条蓝细带子将牌系在冠冕上,是照耶和华所吩咐摩西的。
\par 32 帐幕,就是会幕,一切的工就这样做完了。凡耶和华所吩咐摩西的,以色列人都照样做了。
\par 33 他们送到摩西那里。帐幕和帐幕的一切器具,就是钩子、板、闩、柱子、带卯的座,
\par 34 染红公羊皮的盖、海狗皮的顶盖,和遮掩柜的幔子,
\par 35 法柜和柜的杠并施恩座,
\par 36 桌子和桌子的一切器具并陈设饼,
\par 37 精金的灯台和摆列的灯盏,与灯台的一切器具,并点灯的油,
\par 38 金坛、膏油、馨香的香料、会幕的门帘,
\par 39 铜坛和坛上的铜网,坛的杠并坛的一切器具,洗濯盆和盆座,
\par 40 院子的帷子和柱子,并带卯的座,院子的门帘、绳子、橛子,并帐幕和会幕中一切使用的器具,
\par 41 精工做的礼服,和祭司亚伦并他儿子在圣所用以供祭司职分的圣衣。
\par 42 这一切工作都是以色列人照耶和华所吩咐摩西做的。
\par 43 耶和华怎样吩咐的,他们就怎样做了。摩西看见一切的工都做成了,就给他们祝福。

\chapter{40}

\par 1 耶和华晓谕摩西说:
\par 2 「正月初一日,你要立起帐幕,
\par 3 把法柜安放在里面,用幔子将柜遮掩。
\par 4 把桌子搬进去,摆设上面的物。把灯台搬进去,点其上的灯。
\par 5 把烧香的金坛安在法柜前,挂上帐幕的门帘。
\par 6 把燔祭坛安在帐幕门前。
\par 7 把洗濯盆安在会幕和坛的中间,在盆里盛水。
\par 8 又在四围立院帷,把院子的门帘挂上。
\par 9 用膏油把帐幕和其中所有的都抹上,使帐幕和一切器具成圣,就都成圣。
\par 10 又要抹燔祭坛和一切器具,使坛成圣,就都成为至圣。
\par 11 要抹洗濯盆和盆座,使盆成圣。
\par 12 要使亚伦和他儿子到会幕门口来,用水洗身。
\par 13 要给亚伦穿上圣衣,又膏他,使他成圣,可以给我供祭司的职分;
\par 14 又要使他儿子来,给他们穿上内袍。
\par 15 怎样膏他们的父亲,也要照样膏他们,使他们给我供祭司的职分。他们世世代代凡受膏的,就永远当祭司的职任。」
\par 16 摩西这样行,都是照耶和华所吩咐他的。
\par 17 第二年正月初一日,帐幕就立起来。
\par 18 摩西立起帐幕,安上带卯的座,立上板,穿上闩,立起柱子。
\par 19 在帐幕以上搭罩棚,把罩棚的顶盖盖在其上,是照耶和华所吩咐他的。
\par 20 又把法版放在柜里,把杠穿在柜的两旁,把施恩座安在柜上。
\par 21 把柜抬进帐幕,挂上遮掩柜的幔子,把法柜遮掩了,是照耶和华所吩咐他的。
\par 22 又把桌子安在会幕内,在帐幕北边,在幔子外。
\par 23 在桌子上将饼陈设在耶和华面前,是照耶和华所吩咐他的。
\par 24 又把灯台安在会幕内,在帐幕南边,与桌子相对,
\par 25 在耶和华面前点灯,是照耶和华所吩咐他的。
\par 26 把金坛安在会幕内的幔子前,
\par 27 在坛上烧了馨香料做的香,是照耶和华所吩咐他的。
\par 28 又挂上帐幕的门帘。
\par 29 在会幕的帐幕门前,安设燔祭坛,把燔祭和素祭献在其上,是照耶和华所吩咐他的。
\par 30 把洗濯盆安在会幕和坛的中间,盆中盛水,以便洗濯。
\par 31 摩西和亚伦并亚伦的儿子在这盆里洗手洗脚。
\par 32 他们进会幕或就近坛的时候,便都洗濯,是照耶和华所吩咐他的。
\par 33 在帐幕和坛的四围立了院帷,把院子的门帘挂上。这样,摩西就完了工。
\par 34 当时,云彩遮盖会幕,耶和华的荣光就充满了帐幕。
\par 35 摩西不能进会幕;因为云彩停在其上,并且耶和华的荣光充满了帐幕。
\par 36 每逢云彩从帐幕收上去,以色列人就起程前往;
\par 37 云彩若不收上去,他们就不起程,直等到云彩收上去。
\par 38 日间,耶和华的云彩是在帐幕以上;夜间,云中有火,在以色列全家的眼前。在他们所行的路上都是这样。


\end{document}