\begin{document}

\title{希伯来书}


\chapter{1}

\par 1 神既在古时藉著众先知多次多方的晓谕列祖,
\par 2 就在这末世藉著他儿子晓谕我们;又早已立他为承受万有的,也曾藉著他创造诸世界。
\par 3 他是神荣耀所发的光辉,是神本体的真像,常用他权能的命令托住万有。他洗净了人的罪,就坐在高天至大者的右边。
\par 4 他所承受的名,既比天使的名更尊贵,就远超过天使。
\par 5 所有的天使,神从来对那一个说,你是我的儿子,我今日生你?又指著那一个说:我要作他的父,他要作我的子?
\par 6 再者,神使长子到世上来的时候(或作:神再使长子到世上来的时候),就说:神的使者都要拜他。
\par 7 论到使者,又说:神以风为使者,以火焰为仆役;
\par 8 论到子却说:神啊,你的宝座是永永远远的;你的国权是正直的。
\par 9 你喜爱公义,恨恶罪恶;所以神,就是你的神,用喜乐油膏你,胜过膏你的同伴;
\par 10 又说:主啊,你起初立了地的根基;天也是你手所造的。
\par 11 天地都要灭没,你却要长存。天地都要像衣服渐渐旧了;
\par 12 你要将天地卷起来,像一件外衣,天地就都改变了。惟有你永不改变;你的年数没有穷尽。
\par 13 所有的天使,神从来对那一个说:你坐在我的右边,等我使你仇敌作你的脚凳?
\par 14 天使岂不都是服役的灵、奉差遣为那将要承受救恩的人效力吗?

\chapter{2}

\par 1 所以,我们当越发郑重所听见的道理,恐怕我们随流失去。
\par 2 那藉著天使所传的话既是确定的;凡干犯悖逆的都受了该受的报应。
\par 3 我们若忽略这麽大的救恩,怎能逃罪呢?这救恩起先是主亲自讲的,後来是听见的人给我们证实了。
\par 4 神又按自己的旨意,用神迹、奇事和百般的异能,并圣灵的恩赐,同他们作见证。
\par 5 我们所说将来的世界,神原没有交给天使管辖。
\par 6 但有人在经上某处证明说:人算什麽,你竟顾念他?世人算什麽,你竟眷顾他?
\par 7 你叫他比天使微小一点(或作:你叫他暂时比天使小),赐他荣耀尊贵为冠冕,并将你手所造的都派他管理,
\par 8 叫万物都服在他的脚下。既叫万物都服他,就没有剩下一样不服他的。只是如今我们还不见万物都服他。
\par 9 惟独见那成为比天使小一点的耶稣(或作:惟独见耶稣暂时比天使小);因为受死的苦,就得了尊贵荣耀为冠冕,叫他因著神的恩,为人人尝了死味。
\par 10 原来那为万物所属为万物所本的,要领许多的儿子进荣耀里去,使救他们的元帅,因受苦难得以完全,本是合宜的。
\par 11 因那使人成圣的和那些得以成圣的,都是出於一。所以,他称他们为弟兄也不以为耻,
\par 12 说:我要将你的名传与我的弟兄,在会中我要颂扬你;
\par 13 又说:我要倚赖他;又说:看哪,我与神所给我的儿女。
\par 14 儿女既同有血肉之体,他也照样亲自成了血肉之体,特要藉著死败坏那掌死权的,就是魔鬼,
\par 15 并要释放那些一生因怕死而为奴仆的人。
\par 16 他并不救拔天使,乃是救拔亚伯拉罕的後裔。
\par 17 所以,他凡事该与他的弟兄相同,为要在神的事上成为慈悲忠信的大祭司,为百姓的罪献上挽回祭。
\par 18 他自己既然被试探而受苦,就能搭救被试探的人。

\chapter{3}

\par 1 同蒙天召的圣洁弟兄啊,你们应当思想我们所认为使者、为大祭司的耶稣。
\par 2 他为那设立他的尽忠,如同摩西在神的全家尽忠一样。
\par 3 他比摩西算是更配多得荣耀,好像建造房屋的比房屋更尊荣;
\par 4 因为房屋都必有人建造,但建造万物的就是神。
\par 5 摩西为仆人,在神的全家诚然尽忠,为要证明将来必传说的事。
\par 6 但基督为儿子,治理神的家;我们若将可夸的盼望和胆量坚持到底,便是他的家了。
\par 7 圣灵有话说:你们今日若听他的话,
\par 8 就不可硬著心,像在旷野惹他发怒、试探他的时候一样。
\par 9 在那里,你们的祖宗试我探我,并且观看我的作为有四十年之久。
\par 10 所以,我厌烦那世代的人,说:他们心里常常迷糊,竟不晓得我的作为!
\par 11 我就在怒中起誓说;他们断不可进入我的安息。
\par 12 弟兄们,你们要谨慎,免得你们中间或有人存著不信的恶心,把永生神离弃了。
\par 13 总要趁著还有今日,天天彼此相劝,免得你们中间有人被罪迷惑,心里就刚硬了。
\par 14 我们若将起初确实的信心坚持到底,就在基督里有分了。
\par 15 经上说:「你们今日若听他的话,就不可硬著心,像惹他发怒的日子一样。」
\par 16 那时,听见他话惹他发怒的是谁呢?岂不是跟著摩西从埃及出来的众人吗?
\par 17 神四十年之久,又厌烦谁呢?岂不是那些犯罪、尸首倒在旷野的人吗?
\par 18 又向谁起誓,不容他们进入他的安息呢;岂不是向那些不信从的人吗?
\par 19 这样看来,他们不能进入安息是因为不信的缘故了。

\chapter{4}

\par 1 我们既蒙留下,有进入他安息的应许,就当畏惧,免得我们(原文作你们)中间或有人似乎是赶不上了。
\par 2 因为有福音传给我们,像传给他们一样;只是所听见的道与他们无益,因为他们没有信心与所听见的道调和。
\par 3 但我们已经相信的人得以进入那安息,正如神所说:「我在怒中起誓说:『他们断不可进入我的安息!』」其实造物之工,从创世以来已经成全了。
\par 4 论到第七日,有一处说,「到第七日神就歇了他一切的工。」
\par 5 又有一处说:「他们断不可进入我的安息!」
\par 6 既有必进安息的人,那先前听见福音的,因为不信从,不得进去。
\par 7 所以过了多年,就在大卫的书上,又限定一日,如以上所引的说:「你们今日若听他的话,就不可硬著心。」
\par 8 若是约书亚已叫他们享了安息,後来神就不再提别的日子了。
\par 9 这样看来,必另有一安息日的安息为神的子民存留。
\par 10 因为那进入安息的,乃是歇了自己的工,正如神歇了他的工一样。
\par 11 所以,我们务必竭力进入那安息,免得有人学那不信从的样子跌倒了。
\par 12 神的道是活泼的,是有功效的,比一切两刃的剑更快,甚至魂与灵,骨节与骨髓,都能刺入、剖开,连心中的思念和主意都能辨明。
\par 13 并且被造的没有一样在他面前不显然的;原来万物在那与我们有关系的主眼前,都是赤露敞开的。
\par 14 我们既然有一位已经升入高天尊荣的大祭司,就是神的儿子耶稣,便当持定所承认的道。
\par 15 因我们的大祭司并非不能体恤我们的软弱。他也曾凡事受过试探,与我们一样,只是他没有犯罪。
\par 16 所以,我们只管坦然无惧的来到施恩的宝座前,为要得怜恤,蒙恩惠,作随时的帮助。

\chapter{5}

\par 1 凡从人间挑选的大祭司,是奉派替人办理属神的事,为要献上礼物和赎罪祭(或作:要为罪献上礼物和祭物)。
\par 2 他能体谅那愚蒙的和失迷的人,因为他自己也是被软弱所困。
\par 3 故此,他理当为百姓和自己献祭赎罪。
\par 4 这大祭司的尊荣,没有人自取。惟要蒙神所召,像亚伦一样。
\par 5 如此,基督也不是自取荣耀作大祭司,乃是在乎向他说「你是我的儿子,我今日生你」的那一位;
\par 6 就如经上又有一处说:「你是照著麦基洗德的等次永远为祭司。」
\par 7 基督在肉体的时候,既大声哀哭,流泪祷告,恳求那能救他免死的主,就因他的虔诚蒙了应允。
\par 8 他虽然为儿子,还是因所受的苦难学了顺从。
\par 9 他既得以完全,就为凡顺从他的人成了永远得救的根源、
\par 10 并蒙神照著麦基洗德的等次称他为大祭司。
\par 11 论到麦基洗德,我们有好些话,并且难以解明,因为你们听不进去。
\par 12 看你们学习的工夫,本该作师傅,谁知还得有人将神圣言小学的开端另教导你们,并且成了那必须吃奶,不能吃乾粮的人。
\par 13 凡只能吃奶的都不熟练仁义的道理,因为他是婴孩;
\par 14 惟独长大成人的才能吃乾粮;他们的心窍习练得通达,就能分辨好歹了。

\chapter{6}

\par 1 所以,我们应当离开基督道理的开端,竭力进到完全的地步,不必再立根基,就如那懊悔死行,信靠神、
\par 2 各样洗礼、按手之礼、死人复活,以及永远审判各等教训。
\par 3 神若许我们,我们必如此行。
\par 4 论到那些已经蒙了光照、尝过天恩的滋味、又於圣灵有分,
\par 5 并尝过神善道的滋味、觉悟来世权能的人,
\par 6 若是离弃道理,就不能叫他们从新懊悔了。因为他们把神的儿子重钉十字架,明明的羞辱他。
\par 7 就如一块田地,吃过屡次下的雨水,生长菜蔬,合乎耕种的人用,就从神得福;
\par 8 若长荆棘和蒺藜,必被废弃,近於咒诅,结局就是焚烧。
\par 9 亲爱的弟兄们,我们虽是这样说,却深信你们的行为强过这些,而且近乎得救。
\par 10 因为神并非不公义,竟忘记你们所做的工和你们为他名所显的爱心,就是先前伺候圣徒,如今还是伺候。
\par 11 我们愿你们各人都显出这样的殷勤,使你们有满足的指望,一直到底。
\par 12 并且不懈怠,总要效法那些凭信心和忍耐承受应许的人。
\par 13 当初神应许亚伯拉罕的时候,因为没有比自己更大可以指著起誓的,就指著自己起誓,说:
\par 14 「论福,我必赐大福给你;论子孙,我必叫你的子孙多起来。」
\par 15 这样,亚伯拉罕既恒久忍耐,就得了所应许的。
\par 16 人都是指著比自己大的起誓,并且以起誓为实据,了结各样的争论。
\par 17 照样,神愿意为那承受应许的人格外显明他的旨意是不更改的,就起誓为证。
\par 18 藉这两件不更改的事,神决不能说谎,好叫我们这逃往避难所、持定摆在我们前头指望的人可以大得勉励。
\par 19 我们有这指望,如同灵魂的锚,又坚固又牢靠,且通入幔内。
\par 20 作先锋的耶稣,既照著麦基洗德的等次成了永远的大祭司,就为我们进入幔内。

\chapter{7}

\par 1 这麦基洗德就是撒冷王,又是至高神的祭司,本是长远为祭司的。他当亚伯拉罕杀败诸王回来的时候,就迎接他,给他祝福。
\par 2 亚伯拉罕也将自己所得来的,取十分之一给他。他头一个名翻出来就是仁义王,他又名撒冷王,就是平安王的意思。
\par 3 他无父,无母,无族谱,无生之始,无命之终,乃是与神的儿子相似。
\par 4 你们想一想,先祖亚伯拉罕将自己所掳来上等之物取十分之一给他,这人是何等尊贵呢!
\par 5 那得祭司职任的利未子孙,领命照例向百姓取十分之一,这百姓是自己的弟兄,虽是从亚伯拉罕身(原文作腰)中生的,还是照例取十分之一;
\par 6 独有麦基洗德,不与他们同谱,倒收纳亚伯拉罕的十分之一,为那蒙应许的亚伯拉罕祝福。
\par 7 从来位分大的给位分小的祝福,这是驳不倒的理。
\par 8 在这里收十分之一的都是必死的人;但在那里收十分之一的,有为他作见证的说,他是活的;
\par 9 并且可说那受十分之一的利未,也是藉著亚伯拉罕纳了十分之一。
\par 10 因为麦基洗德迎接亚伯拉罕的时候,利未已经在他先祖的身(原文作腰)中。
\par 11 从前百姓在利未人祭司职任以下受律法,倘若藉这职任能得完全,又何用另外兴起一位祭司,照麦基洗德的等次,不照亚伦的等次呢?
\par 12 祭司的职任既已更改,律法也必须更改。
\par 13 因为这话所指的人本属别的支派,那支派里从来没有一人伺候祭坛。
\par 14 我们的主分明是从犹大出来的;但这支派,摩西并没有提到祭司。
\par 15 倘若照麦基洗德的样式,另外兴起一位祭司来,我的话更是显而易见的了。
\par 16 他成为祭司,并不是照属肉体的条例,乃是照无穷(原文作不能毁坏)之生命的大能。
\par 17 因为有给他作见证的说:「你是照著麦基洗德的等次永远为祭司。」
\par 18 先前的条例,因软弱无益,所以废掉了,
\par 19 (律法原来一无所成)就引进了更美的指望;靠这指望,我们便可以进到神面前。
\par 20 再者,耶稣为祭司,并不是不起誓立的。
\par 21 至於那些祭司,原不是起誓立的,只有耶稣是起誓立的;因为那立他的对他说:「主起了誓,决不後悔,你是永远为祭司。」
\par 22 既是起誓立的,耶稣就作了更美之约的中保。
\par 23 那些成为祭司的,数目本来多,是因为有死阻隔,不能长久。
\par 24 这位既是永远常存的,他祭司的职任就长久不更换。
\par 25 凡靠著他进到神面前的人,他都能拯救到底;因为他是长远活著,替他们祈求。
\par 26 像这样圣洁、无邪恶、无玷污、远离罪人、高过诸天的大祭司,原是与我们合宜的。
\par 27 他不像那些大祭司,每日必须先为自己的罪,後为百姓的罪献祭;因为他只一次将自己献上,就把这事成全了。
\par 28 律法本是立软弱的人为大祭司;但在律法以後起誓的话,是立儿子为大祭司,乃是成全到永远的。

\chapter{8}

\par 1 我们所讲的事,其中第一要紧的,就是我们有这样的大祭司,已经坐在天上至大者宝座的右边,
\par 2 在圣所,就是真帐幕里,作执事;这帐幕是主所支的,不是人所支的。
\par 3 凡大祭司都是为献礼物和祭物设立的,所以这位大祭司也必须有所献的。
\par 4 他若在地上,必不得为祭司,因为已经有照律法献礼物的祭司。
\par 5 他们供奉的事本是天上事的形状和影像,正如摩西将要造帐幕的时候,蒙神警戒他,说:『你要谨慎,作各样的物件都要照著在山上指示你的样式。』
\par 6 如今耶稣所得的职任是更美的,正如他作更美之约的中保;这约原是凭更美之应许立的。
\par 7 那前约若没有瑕疵,就无处寻求後约了。
\par 8 所以主指著他的百姓说(或作:所以主指前约的缺欠说):日子将到,我要与以色列家和犹大家另立新约,
\par 9 不像我拉著他们祖宗的手,领他们出埃及的时候,与他们所立的约。因为他们不恒心守我的约,我也不理他们。这是主说的。
\par 10 主又说:那些日子以後,我与以色列家所立的约乃是这样:我要将我的律法放在他们里面,写在他们心上;我要作他们的神;他们要作我的子民。
\par 11 他们不用各人教导自己的乡邻和自己的弟兄,说:你该认识主;因为他们从最小的到至大的,都必认识我。
\par 12 我要宽恕他们的不义,不再记念他们的罪愆。
\par 13 既说新约。就以前约为旧了;但那渐旧渐衰的,就必快归无有了。

\chapter{9}

\par 1 原来前约有礼拜的条例和属世界的圣幕。
\par 2 因为有预备的帐幕,头一层叫作圣所,里面有灯台、桌子,和陈设饼。
\par 3 第二幔子後又有一层帐幕,叫作至圣所,
\par 4 有金香炉(炉:或作坛),有包金的约柜,柜里有盛吗哪的金罐和亚伦发过芽的杖,并两块约版;
\par 5 柜上面有荣耀基路伯的影罩著施恩座(施恩:原文作蔽罪)。这几件我现在不能一一细说。
\par 6 这些物件既如此预备齐了,众祭司就常进头一层帐幕,行拜神的礼。
\par 7 至於第二层帐幕,惟有大祭司一年一次独自进去,没有不带著血为自己和百姓的过错献上。
\par 8 圣灵用此指明,头一层帐幕仍存的时候,进入至圣所的路还未显明。
\par 9 那头一层帐幕作现今的一个表样,所献的礼物和祭物,就著良心说,都不能叫礼拜的人得以完全。
\par 10 这些事,连那饮食和诸般洗濯的规矩,都不过是属肉体的条例,命定到振兴的时候为止。
\par 11 但现在基督已经来到,作了将来美事的大祭司,经过那更大更全备的帐幕,不是人手所造、也不是属乎这世界的;
\par 12 并且不用山羊和牛犊的血,乃用自己的血,只一次进入圣所,成了永远赎罪的事。
\par 13 若山羊和公牛的血,并母牛犊的灰,洒在不洁的人身上,尚且叫人成圣,身体洁净,
\par 14 何况基督藉著永远的灵,将自己无瑕无疵献给神,他的血岂不更能洗净你们的心(原文作良心),除去你们的死行,使你们事奉那永生神吗?
\par 15 为此,他作了新约的中保,既然受死赎了人在前约之时所犯的罪过,便叫蒙召之人得著所应许永远的产业。
\par 16 凡有遗命必须等到留遗命(遗命:原文与约字同)的人死了;
\par 17 因为人死了,遗命才有效力,若留遗命的尚在,那遗命还有用处吗?
\par 18 所以,前约也不是不用血立的;
\par 19 因为摩西当日照著律法将各样诫命传给众百姓,就拿朱红色绒和牛膝草,把牛犊山羊的血和水洒在书上,又洒在众百姓身上,说:
\par 20 「这血就是神与你们立约的凭据。」
\par 21 他又照样把血洒在帐幕和各样器皿上。
\par 22 按著律法,凡物差不多都是用血洁净的;若不流血,罪就不得赦免了。
\par 23 照著天上样式作的物件必须用这些祭物去洁净;但那天上的本物自然当用更美的祭物去洁净。
\par 24 因为基督并不是进了人手所造的圣所(这不过是真圣所的影像),乃是进了天堂,如今为我们显在神面前;
\par 25 也不是多次将自己献上,像那大祭司每年带著牛羊的血(牛羊的血:原文作不是自己的血)进入圣所,
\par 26 如果这样,他从创世以来,就必多次受苦了。但如今在这末世显现一次,把自己献为祭,好除掉罪。
\par 27 按著定命,人人都有一死,死後且有审判。
\par 28 像这样,基督既然一次被献,担当了多人的罪,将来要向那等候他的人第二次显现,并与罪无关,乃是为拯救他们。

\chapter{10}

\par 1 律法既是将来美事的影儿,不是本物的真像,总不能藉著每年常献一样的祭物叫那近前来的人得以完全。
\par 2 若不然,献祭的事岂不早已止住了吗?因为礼拜的人,良心既被洁净,就不再觉得有罪了。
\par 3 但这些祭物是叫人每年想起罪来;
\par 4 因为公牛和山羊的血,断不能除罪。
\par 5 所以基督到世上来的时候,就说:神啊,祭物和礼物是你不愿意的;你曾给我预备了身体。
\par 6 燔祭和赎罪祭是你不喜欢的。
\par 7 那时我说:神啊,我来了,为要照你的旨意行;我的事在经卷上已经记载了。
\par 8 以上说:「祭物和礼物,燔祭和赎罪祭,是你不愿意的,也是你不喜欢的(这都是按著律法献的)」;
\par 9 後又说:「我来了为要照你的旨意行」;可见他是除去在先的,为要立定在後的。
\par 10 我们凭这旨意,靠耶稣基督,只一次献上他的身体,就得以成圣。
\par 11 凡祭司天天站著事奉神,屡次献上一样的祭物,这祭物永不能除罪。
\par 12 但基督献了一次永远的赎罪祭,就在神的右边坐下了。
\par 13 从此,等候他仇敌成了他的脚凳。
\par 14 因为他一次献祭,便叫那得以成圣的人永远完全。
\par 15 圣灵也对我们作见证;因为他既已说过:
\par 16 主说:那些日子以後,我与他们所立的约乃是这样:我要将我的律法写在他们心上,又要放在他们的里面。
\par 17 以後就说:我不再记念他们的罪愆和他们的过犯。
\par 18 这些罪过既已赦免,就不用再为罪献祭了。
\par 19 弟兄们,我们既因耶稣的血得以坦然进入至圣所,
\par 20 是藉著他给我们开了一条又新又活的路,从幔子经过,这幔子就是他的身体。
\par 21 又有一位大祭司治理神的家!
\par 22 并我们心中天良的亏欠已经洒去,身体用清水洗净了,就当存著诚心和充足的信心来到神面前;
\par 23 也要坚守我们所承认的指望,不至摇动,因为那应许我们的是信实的。
\par 24 又要彼此相顾,激发爱心,勉励行善。
\par 25 你们不可停止聚会,好像那些停止惯了的人,倒要彼此劝勉,既知道(原文作看见)那日子临近,就更当如此。
\par 26 因为我们得知真道以後,若故意犯罪,赎罪的祭就再没有了;
\par 27 惟有战惧等候审判和那烧灭众敌人的烈火。
\par 28 人干犯摩西的律法,凭两三个见证人,尚且不得怜恤而死,
\par 29 何况人践踏神的儿子,将那使他成圣之约的血当作平常,又亵慢施恩的圣灵,你们想,他要受的刑罚该怎样加重呢!
\par 30 因为我们知道谁说:「伸冤在我,我必报应」;又说:「主要审判他的百姓。」
\par 31 落在永生神的手里,真是可怕的!
\par 32 你们要追念往日,蒙了光照以後所忍受大争战的各样苦难:
\par 33 一面被毁谤,遭患难,成了戏景,叫众人观看;一面陪伴那些受这样苦难的人。
\par 34 因为你们体恤了那些被捆锁的人,并且你们的家业被人抢去,也甘心忍受,知道自己有更美长存的家业。
\par 35 所以,你们不可丢弃勇敢的心;存这样的心必得大赏赐。
\par 36 你们必须忍耐,使你们行完了神的旨意,就可以得著所应许的。
\par 37 因为还有一点点时候,那要来的就来,并不迟延;
\par 38 只是义人(有古卷:我的义人)必因信得生。他若退後,我心里就不喜欢他。
\par 39 我们却不是退後入沉沦的那等人,乃是有信心以致灵魂得救的人。

\chapter{11}

\par 1 信就是所望之事的实底,是未见之事的确据。
\par 2 古人在这信上得了美好的证据。
\par 3 我们因著信,就知道诸世界是藉神话造成的;这样,所看见的,并不是从显然之物造出来的。
\par 4 亚伯因著信,献祭与神,比该隐所献的更美,因此便得了称义的见证,就是神指他礼物作的见证。他虽然死了,却因这信,仍旧说话。
\par 5 以诺因著信,被接去,不至於见死,人也找不著他,因为神已经把他接去了;只是他被接去以先,已经得了神喜悦他的明证。
\par 6 人非有信,就不能得神的喜悦;因为到神面前来的人必须信有神,且信他赏赐那寻求他的人。
\par 7 挪亚因著信,既蒙神指示他未见的事,动了敬畏的心,预备了一只方舟,使他全家得救。因此就定了那世代的罪,自己也承受了那从信而来的义。
\par 8 亚伯拉罕因著信,蒙召的时候就遵命出去,往将来要得为业的地方去;出去的时候,还不知往那里去。
\par 9 他因著信,就在所应许之地作客,好像在异地居住帐棚,与那同蒙一个应许的以撒、雅各一样。
\par 10 因为他等候那座有根基的城,就是神所经营所建造的。
\par 11 因著信,连撒拉自己,虽然过了生育的岁数,还能怀孕,因他以为那应许他的是可信的。
\par 12 所以从一个彷佛已死的人就生出子孙,如同天上的星那样众多,海边的沙那样无数。
\par 13 这些人都是存著信心死的,并没有得著所应许的;却从远处望见,且欢喜迎接,又承认自己在世上是客旅,是寄居的。
\par 14 说这样话的人是表明自己要找一个家乡。
\par 15 他们若想念所离开的家乡,还有可以回去的机会。
\par 16 他们却羡慕一个更美的家乡,就是在天上的。所以神被称为他们的神,并不以为耻,因为他已经给他们预备了一座城。
\par 17 亚伯拉罕因著信,被试验的时候,就把以撒献上;这便是那欢喜领受应许的,将自己独生的儿子献上。
\par 18 论到这儿子,曾有话说:「从以撒生的才要称为你的後裔。」
\par 19 他以为神还能叫人从死里复活;他也彷佛从死中得回他的儿子来。
\par 20 以撒因著信,就指著将来的事给雅各、以扫祝福。
\par 21 雅各因著信,临死的时候,给约瑟的两个儿子各自祝福,扶著杖头敬拜神。
\par 22 约瑟因著信,临终的时候,提到以色列族将来要出埃及,并为自己的骸骨留下遗命。
\par 23 摩西生下来,他的父母见他是个俊美的孩子,就因著信,把他藏了三个月,并不怕王命。
\par 24 摩西因著信,长大了就不肯称为法老女儿之子。
\par 25 他宁可和神的百姓同受苦害,也不愿暂时享受罪中之乐。
\par 26 他看为基督受的凌辱比埃及的财物更宝贵,因他想望所要得的赏赐。
\par 27 他因著信,就离开埃及,不怕王怒;因为他恒心忍耐,如同看见那不能看见的主。
\par 28 他因著信,就守(或作:立)逾越节,行洒血的礼,免得那灭长子的临近以色列人。
\par 29 他们因著信,过红海如行乾地;埃及人试著要过去,就被吞灭了。
\par 30 以色列人因著信,围绕耶利哥城七日,城墙就倒塌了。
\par 31 妓女喇合因著信,曾和和平平的接待探子,就不与那些不顺从的人一同灭亡。
\par 32 我又何必再说呢?若要一一细说,基甸、巴拉、参孙、耶弗他、大卫、撒母耳,和众先知的事,时候就不够了。
\par 33 他们因著信,制伏了敌国,行了公义,得了应许,堵了狮子的口,
\par 34 灭了烈火的猛势,脱了刀剑的锋刃;软弱变为刚强,争战显出勇敢,打退外邦的全军。
\par 35 有妇人得自己的死人复活。又有人忍受严刑,不肯苟且得释放(原文作赎),为要得著更美的复活。
\par 36 又有人忍受戏弄、鞭打、捆锁、监禁、各等的磨炼,
\par 37 被石头打死,被锯锯死,受试探,被刀杀,披著绵羊山羊的皮各处奔跑,受穷乏、患难、苦害,
\par 38 在旷野、山岭、山洞、地穴,飘流无定,本是世界不配有的人。
\par 39 这些人都是因信得了美好的证据,却仍未得著所应许的;
\par 40 因为神给我们预备了更美的事,叫他们若不与我们同得,就不能完全。

\chapter{12}

\par 1 我们既有这许多的见证人,如同云彩围著我们,就当放下各样的重担,脱去容易缠累我们的罪,存心忍耐,奔那摆在我们前头的路程,
\par 2 仰望为我们信心创始成终的耶稣(或作:仰望那将真道创始成终的耶稣)。他因那摆在前面的喜乐,就轻看羞辱,忍受了十字架的苦难,便坐在神宝座的右边。
\par 3 那忍受罪人这样顶撞的,你们要思想,免得疲倦灰心。
\par 4 你们与罪恶相争,还没有抵挡到流血的地步。
\par 5 你们又忘了那劝你们如同劝儿子的话,说:我儿,你不可轻看主的管教,被他责备的时候也不可灰心;
\par 6 因为主所爱的,他必管教,又鞭打凡所收纳的儿子。
\par 7 你们所忍受的,是神管教你们,待你们如同待儿子。焉有儿子不被父亲管教的呢?
\par 8 管教原是众子所共受的,你们若不受管教,就是私子,不是儿子了。
\par 9 再者,我们曾有生身的父管教我们,我们尚且敬重他,何况万灵的父,我们岂不更当顺服他得生吗?
\par 10 生身的父都是暂随己意管教我们;惟有万灵的父管教我们,是要我们得益处,使我们在他的圣洁上有分。
\par 11 凡管教的事,当时不觉得快乐,反觉得愁苦;後来却为那经练过的人结出平安的果子,就是义。
\par 12 所以,你们要把下垂的手、发酸的腿、挺起来;
\par 13 也要为自己的脚,把道路修直了,使瘸子不至歪脚(或作:差路),反得痊愈。
\par 14 你们要追求与众人和睦,并要追求圣洁;非圣洁没有人能见主。
\par 15 又要谨慎,恐怕有人失了神的恩;恐怕有毒根生出来扰乱你们,因此叫众人沾染污秽;
\par 16 恐怕有淫乱的,有贪恋世俗如以扫的,他因一点食物把自己长子的名分卖了。
\par 17 後来想要承受父所祝的福,竟被弃绝,虽然号哭切求,却得不著门路,使他父亲的心意回转。这是你们知道的。
\par 18 你们原不是来到那能摸的山;此山有火焰、密云、黑暗、暴风、
\par 19 角声与说话的声音。那些听见这声音的,都求不要再向他们说话;
\par 20 因为他们当不起所命他们的话,说:『靠近这山的,即便是走兽,也要用石头打死。』
\par 21 所见的极其可怕,甚至摩西说:『我甚是恐惧战兢。』
\par 22 你们乃是来到锡安山,永生神的城邑,就是天上的耶路撒冷。那里有千万的天使,
\par 23 有名录在天上诸长子之会所共聚的总会,有审判众人的神和被成全之义人的灵魂,
\par 24 并新约的中保耶稣,以及所洒的血;这血所说的比亚伯的血所说的更美。
\par 25 你们总要谨慎,不可弃绝那向你们说话的。因为,那些弃绝在地上警戒他们的尚且不能逃罪,何况我们违背那从天上警戒我们的呢?
\par 26 当时他的声音震动了地,但如今他应许说:『再一次我不单要震动地,还要震动天。』
\par 27 这再一次的话,是指明被震动的,就是受造之物都要挪去,使那不被震动的常存。
\par 28 所以我们既得了不能震动的国,就当感恩,照神所喜悦的,用虔诚、敬畏的心事奉神。
\par 29 因为我们的神乃是烈火。

\chapter{13}

\par 1 你们务要常存弟兄相爱的心。
\par 2 不可忘记用爱心接待客旅;因为曾有接待客旅的,不知不觉就接待了天使。
\par 3 你们要记念被捆绑的人,好像与他们同受捆绑;,也要记念遭苦害的人,想到自己也在肉身之内。
\par 4 婚姻,人人都当尊重,床也不可污秽;因为苟合行淫的人,神必要审判。
\par 5 你们存心不可贪爱钱财,要以自己所有的为足;因为主曾说:『我总不撇下你,也不丢弃你。』
\par 6 所以我们可以放胆说:主是帮助我的,我必不惧怕;人能把我怎麽样呢?
\par 7 从前引导你们、传神之道给你们的人,你们要想念他们,效法他们的信心,留心看他们为人的结局。
\par 8 耶稣基督,昨日、今日、一直到永远、是一样的。
\par 9 你们不要被那诸般怪异的教训勾引了去;因为人心靠恩得坚固才是好的,并不是靠饮食。那在饮食上专心的从来没有得著益处。
\par 10 我们有一祭坛,上面的祭物是那些在帐幕中供职的人不可同吃的。
\par 11 原来牲畜的血被大祭司带入圣所作赎罪祭;牲畜的身子被烧在营外。
\par 12 所以,耶稣要用自己的血叫百姓成圣,也就在城门外受苦。
\par 13 这样,我们也当出到营外,就了他去,忍受他所受的凌辱。
\par 14 我们在这里本没有常存的城,乃是寻求那将来的城。
\par 15 我们应当靠著耶稣,常常以颂赞为祭献给神,这就是那承认主名之人嘴唇的果子。
\par 16 只是不可忘记行善和捐输的事;因为这样的祭,是神所喜悦的。
\par 17 你们要依从那些引导你们的,且要顺服;因他们为你们的灵魂时刻警醒,好像那将来交账的人。你们要使他们交的时候有快乐,不至忧愁;若忧愁就与你们无益了。
\par 18 请你们为我们祷告,因我们自觉良心无亏,愿意凡事按正道而行。
\par 19 我更求你们为我祷告,使我快些回到你们那里去。
\par 20 但愿赐平安的神,就是那凭永约之血、使群羊的大牧人我主耶稣从死里复活的神,
\par 21 在各样善事上成全你们,叫你们遵行他的旨意;又藉著耶稣基督在你们心里行他所喜悦的事。愿荣耀归给他,直到永永远远。阿们!
\par 22 弟兄们,我略略写信给你们,望你们听我劝勉的话。
\par 23 你们该知道,我们的兄弟提摩太已经释放了;他若快来,我必同他去见你们。
\par 24 请你们问引导你们的诸位和众圣徒安。从义大利来的人也问你们安。
\par 25 愿恩惠常与你们众人同在。阿们!


\end{document}