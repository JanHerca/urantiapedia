\begin{document}

\title{雅各书}


\chapter{1}

\par 1 作神和主耶稣基督仆人的雅各请散住十二个支派之人的安。
\par 2 我的弟兄们,你们落在百般试炼中,都要以为大喜乐;
\par 3 因为知道你们的信心经过试验,就生忍耐。
\par 4 但忍耐也当成功,使你们成全、完备,毫无缺欠。
\par 5 你们中间若有缺少智慧的,应当求那厚赐与众人、也不斥责人的神,主就必赐给他。
\par 6 只要凭著信心求,一点不疑惑;因为那疑惑的人,就像海中的波浪,被风吹动翻腾。
\par 7 这样的人不要想从主那里得什麽。
\par 8 心怀二意的人,在他一切所行的路上都没有定见。
\par 9 卑微的弟兄升高,就该喜乐;
\par 10 富足的降卑,也该如此;因为他必要过去,如同草上的花一样。
\par 11 太阳出来,热风刮起,草就枯乾,花也凋谢,美容就消没了;那富足的人,在他所行的事上也要这样衰残。
\par 12 忍受试探的人是有福的,因为他经过试验以後,必得生命的冠冕,这是主应许给那些爱他之人的。
\par 13 人被试探,不可说:「我是被神试探」;因为神不能被恶试探,他也不试探人。
\par 14 但各人被试探,乃是被自己的私欲牵引诱惑的。
\par 15 私欲既怀了胎,就生出罪来;罪既长成,就生出死来。
\par 16 我亲爱的弟兄们,不要看错了。
\par 17 各样美善的恩赐和各样全备的赏赐都是从上头来的,从众光之父那里降下来的;在他并没有改变,也没有转动的影儿。
\par 18 他按自己的旨意,用真道生了我们,叫我们在他所造的万物中好像初熟的果子。
\par 19 我亲爱的弟兄们,这是你们所知道的,但你们各人要快快的听,慢慢的说,慢慢的动怒,
\par 20 因为人的怒气并不成就神的义。
\par 21 所以你们要脱去一切的污秽和盈余的邪恶,存温柔的心领受那所栽种的道,就是能救你们灵魂的道。
\par 22 只是你们要行道,不要单单听道,自己欺哄自己。
\par 23 因为听道而不行道的,就像人对著镜子看自己本来的面目,
\par 24 看见,走後,随即忘了他的相貌如何。
\par 25 惟有详细察看那全备,使人自由之律法的,并且时常如此,这人既不是听了就忘,乃是实在行出来,就在他所行的事上必然得福。
\par 26 若有人自以为虔诚,却不勒住他的舌头,反欺哄自己的心,这人的虔诚是虚的。
\par 27 在神我们的父面前,那清洁没有玷污的虔诚,就是看顾在患难中的孤儿寡妇,并且保守自己不沾染世俗。

\chapter{2}

\par 1 我的弟兄们,你们信奉我们荣耀的主耶稣基督,便不可按著外貌待人。
\par 2 若有一个人带著金戒指,穿著华美衣服,进你们的会堂去;又有一个穷人穿著肮脏衣服也进去;
\par 3 你们就看重那穿华美衣服的人,说:「请坐在这好位上」;又对那穷人说:「你站在那里」,或「坐在我脚凳下边。」
\par 4 这岂不是你们偏心待人,用恶意断定人吗?
\par 5 我亲爱的弟兄们,请听,神岂不是拣选了世上的贫穷人,叫他们在信上富足,并承受他所应许给那些爱他之人的国吗?
\par 6 你们反倒羞辱贫穷人。那富足人岂不是欺压你们,拉你们到公堂去吗?
\par 7 他们不是亵渎你们所敬奉(所敬奉:或作被称)的尊名吗?
\par 8 经上记著说:「要爱人如己。」你们若全守这至尊的律法,才是好的。
\par 9 但你们若按外貌待人,便是犯罪,被律法定为犯法的。
\par 10 因为凡遵守全律法的,只在一条上跌倒,他就是犯了众条。
\par 11 原来那说「不可奸淫」的,也说「不可杀人」;你就是不奸淫,却杀人,仍是成了犯律法的。
\par 12 你们既然要按使人自由的律法受审判,就该照这律法说话行事。
\par 13 因为那不怜悯人的,也要受无怜悯的审判;怜悯原是向审判夸胜。
\par 14 我的弟兄们,若有人说自己有信心,却没有行为,有什麽益处呢?这信心能救他吗?
\par 15 若是弟兄或是姐妹,赤身露体,又缺了日用的饮食;
\par 16 你们中间有人对他们说:「平平安安的去吧!愿你们穿得暖,吃得饱」;却不给他们身体所需用的,这有什麽益处呢?
\par 17 这样,信心若没有行为就是死的。
\par 18 必有人说:「你有信心,我有行为;你将你没有行为的信心指给我看,我便藉著我的行为,将我的信心指给你看。」
\par 19 你信神只有一位,你信的不错;鬼魔也信,却是战惊。
\par 20 虚浮的人哪,你愿意知道没有行为的信心是死的吗?
\par 21 我们的祖宗亚伯拉罕把他儿子以撒献在坛上,岂不是因行为称义吗?
\par 22 可见信心是与他的行为并行,而且信心因著行为才得成全。
\par 23 这就应验经上所说:「亚伯拉罕信神,这就算为他的义。」他又得称为神的朋友。
\par 24 这样看来,人称义是因著行为,不是单因著信。
\par 25 妓女喇合接待使者,又放他们从别的路上出去,不也是一样因行为称义吗?
\par 26 身体没有灵魂是死的,信心没有行为也是死的。

\chapter{3}

\par 1 我的弟兄们,不要多人作师傅,因为晓得我们要受更重的判断。
\par 2 原来我们在许多事上都有过失;若有人在话语上没有过失,他就是完全人,也能勒住自己的全身。
\par 3 我们若把嚼环放在马嘴里,叫他顺服,就能调动他的全身。
\par 4 看哪,船只虽然甚大,又被大风催逼,只用小小的舵,就随著掌舵的意思转动。
\par 5 这样,舌头在百体里也是最小的,却能说大话。看哪,最小的火能点著最大的树林。
\par 6 舌头就是火,在我们百体中,舌头是个罪恶的世界,能污秽全身,也能把生命的轮子点起来,并且是从地狱里点著的。
\par 7 各类的走兽,飞禽,昆虫,水族,本来都可以制伏,也已经被人制伏了;
\par 8 惟独舌头没有人能制伏,是不止息的恶物,满了害死人的毒气。
\par 9 我们用舌头颂赞那为主、为父的,又用舌头咒诅那照著神形像被造的人;
\par 10 颂赞和咒诅从一个口里出来!我的弟兄们,这是不应当的!
\par 11 泉源从一个眼里能发出甜苦两样的水吗?
\par 12 我的弟兄们,无花果树能生橄榄吗?葡萄树能结无花果吗?咸水里也不能发出甜水来。
\par 13 你们中间谁是有智慧有见识的呢?他就当在智慧的温柔上显出他的善行来。
\par 14 你们心里若怀著苦毒的嫉妒和分争,就不可自夸,也不可说谎话抵挡真道。
\par 15 这样的智慧不是从上头来的,乃是属地的,属情欲的,属鬼魔的。
\par 16 在何处有嫉妒、分争,就在何处有扰乱和各样的坏事。
\par 17 惟独从上头来的智慧,先是清洁,後是和平,温良柔顺,满有怜悯,多结善果,没有偏见,没有假冒。
\par 18 并且使人和平的,是用和平所栽种的义果。

\chapter{4}

\par 1 你们中间的争战斗殴是从那里来的呢?不是从你们百体中战斗之私欲来的吗?
\par 2 你们贪恋,还是得不著;你们杀害嫉妒,又斗殴争战,也不能得。你们得不著,是因为你们不求。
\par 3 你们求也得不著,是因为你们妄求,要浪费在你们的宴乐中。
\par 4 你们这些淫乱的人(原文作淫妇)哪,岂不知与世俗为友就是与神为敌吗?所以凡想要与世俗为友的,就是与神为敌了。
\par 5 你们想经上所说是徒然的吗?神所赐、住在我们里面的灵,是恋爱至於嫉妒吗?
\par 6 但他赐更多的恩典,所以经上说:神阻挡骄傲的人,赐恩给谦卑的人。
\par 7 故此你们要顺服神。务要抵挡魔鬼,魔鬼就必离开你们逃跑了。
\par 8 你们亲近神,神就必亲近你们。有罪的人哪,要洁净你们的手!心怀二意的人哪,要清洁你们的心!
\par 9 你们要愁苦、悲哀、哭泣,将喜笑变作悲哀,欢乐变作愁闷。
\par 10 务要在主面前自卑,主就必叫你们升高。
\par 11 弟兄们,你们不可彼此批评。人若批评弟兄,论断弟兄,就是批评律法,论断律法。你若论断律法,就不是遵行律法,乃是判断人的。
\par 12 设立律法和判断人的,只有一位,就是那能救人也能灭人的。你是谁,竟敢论断别人呢?
\par 13 !你们有话说:「今天明天我们要往某城里去,在那里住一年,作买卖得利。」
\par 14 其实明天如何,你们还不知道。你们的生命是什麽呢?你们原来是一片云雾,出现少时就不见了。
\par 15 你们只当说:「主若愿意,我们就可以活著,也可以做这事,或做那事。」
\par 16 现今你们竟以张狂夸口;凡这样夸口都是恶的。
\par 17 人若知道行善,却不去行,这就是他的罪了。

\chapter{5}

\par 1 !你们这些富足人哪,应当哭泣、号 ,因为将有苦难临到你们身上。
\par 2 你们的财物坏了,衣服也被虫子咬了。
\par 3 你们的金银都长了锈;那锈要证明你们的不是,又要吃你们的肉,如同火烧。你们在这末世只知积 钱财。
\par 4 工人给你们收割庄稼,你们亏欠他们的工钱,这工钱有声音呼叫,并且那收割之人的冤声已经入了万军之主的耳了。
\par 5 你们在世上享美福,好宴乐,当宰杀的日子竟娇养你们的心。
\par 6 你们定了义人的罪,把他杀害,他也不抵挡你们。
\par 7 弟兄们哪,你们要忍耐,直到主来。看哪,农夫忍耐等候地里宝贵的出产,直到得了秋雨春雨。
\par 8 你们也当忍耐,坚固你们的心;因为主来的日子近了。
\par 9 弟兄们,你们不要彼此埋怨,免得受审判。看哪,审判的主站在门前了。
\par 10 弟兄们,你们要把那先前奉主名说话的众先知当作能受苦能忍耐的榜样。
\par 11 那先前忍耐的人,我们称他们是有福的。你们听见过约伯的忍耐,也知道主给他的结局,明显主是满心怜悯,大有慈悲。
\par 12 我的弟兄们,最要紧的是不可起誓;不可指著天起誓,也不可指著地起誓,无论何誓都不可起。你们说话,是,就说是;不是,就说不是,免得你们落在审判之下。
\par 13 你们中间有受苦的呢,他就该祷告;有喜乐的呢,他就该歌颂。
\par 14 你们中间有病了的呢,他就该请教会的长老来;他们可以奉主的名用油抹他,为他祷告。
\par 15 出於信心的祈祷要救那病人,主必叫他起来;他若犯了罪,也必蒙赦免。
\par 16 所以你们要彼此认罪,互相代求,使你们可以得医治。义人祈祷所发的力量是大有功效的。
\par 17 以利亚与我们是一样性情的人,他恳切祷告,求不要下雨,雨就三年零六个月不下在地上。
\par 18 他又祷告,天就降下雨来,地也生出土产。
\par 19 我的弟兄们,你们中间若有失迷真道的,有人使他回转,
\par 20 这人该知道:叫一个罪人从迷路上转回便是救一个灵魂不死,并且遮盖许多的罪。


\end{document}