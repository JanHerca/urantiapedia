\begin{document}

\title{尼希米记}


\chapter{1}

\par 1 哈迦利亚的儿子尼希米的言语如下:亚达薛西王二十年基斯流月,我在书珊城的宫中。
\par 2 那时,有我一个弟兄哈拿尼,同著几个人从犹大来。我问他们那些被掳归回、剩下逃脱的犹大人和耶路撒冷的光景。
\par 3 他们对我说:「那些被掳归回剩下的人在犹大省遭大难,受凌辱;并且耶路撒冷的城墙拆毁,城门被火焚烧。」
\par 4 我听见这话,就坐下哭泣,悲哀几日,在天上的神面前禁食祈祷,说:
\par 5 「耶和华天上的神,大而可畏的神啊,你向爱你、守你诫命的人守约施慈爱。
\par 6 愿你睁眼看,侧耳听,你仆人昼夜在你面前为你众仆人以色列民的祈祷,承认我们以色列人向你所犯的罪;我与我父家都有罪了。
\par 7 我们向你所行的甚是邪恶,没有遵守你藉著仆人摩西所吩咐的诫命、律例、典章。
\par 8 求你记念所吩咐你仆人摩西的话,说:『你们若犯罪,我就把你们分散在万民中;
\par 9 但你们若归向我,谨守遵行我的诫命,你们被赶散的人虽在天涯,我也必从那里将他们招聚回来,带到我所选择立为我名的居所。』
\par 10 这都是你的仆人、你的百姓,就是你用大力和大能的手所救赎的。
\par 11 主啊,求你侧耳听你仆人的祈祷,和喜爱敬畏你名众仆人的祈祷,使你仆人现今亨通,在王面前蒙恩。」我是作王酒政的。

\chapter{2}

\par 1 亚达薛西王二十年尼散月,在王面前摆酒,我拿起酒来奉给王。我素来在王面前没有愁容。
\par 2 王对我说:「你既没有病,为什麽面带愁容呢?这不是别的,必是你心中愁烦。」於是我甚惧怕。
\par 3 我对王说:「愿王万岁!我列祖坟墓所在的那城荒凉,城门被火焚烧,我岂能面无愁容吗?」
\par 4 王问我说:「你要求什麽?」於是我默祷天上的神。
\par 5 我对王说:「仆人若在王眼前蒙恩,王若喜欢,求王差遣我往犹大,到我列祖坟墓所在的那城去,我好重新建造。」
\par 6 那时王后坐在王的旁边。王问我说:「你去要多少日子?几时回来?」我就定了日期。於是王喜欢差遣我去。
\par 7 我又对王说:「王若喜欢,求王赐我诏书,通知大河西的省长准我经过,直到犹大;
\par 8 又赐诏书,通知管理王园林的亚萨,使他给我木料,做属殿营楼之门的横梁和城墙,与我自己房屋使用的。」王就允准我,因我神施恩的手帮助我。
\par 9 王派了军长和马兵护送我。我到了河西的省长那里,将王的诏书交给他们。
\par 10 和伦人参巴拉,并为奴的亚扪人多比雅,听见有人来为以色列人求好处,就甚恼怒。
\par 11 我到了耶路撒冷,在那里住了三日。
\par 12 我夜间起来,有几个人也一同起来;但神使我心里要为耶路撒冷做什麽事,我并没有告诉人。除了我骑的牲口以外,也没有别的牲口在我那里。
\par 13 当夜我出了谷门,往野狗井去(野狗:或作龙),到了粪厂门,察看耶路撒冷的城墙,见城墙拆毁,城门被火焚烧。
\par 14 我又往前,到了泉门和王池,但所骑的牲口没有地方过去。
\par 15 於是夜间沿溪而上,察看城墙,又转身进入谷门,就回来了。
\par 16 我往那里去,我做什麽事,官长都不知道。我还没有告诉犹大平民、祭司、贵胄、官长,和其余做工的人。
\par 17 以後,我对他们说:「我们所遭的难,耶路撒冷怎样荒凉,城门被火焚烧,你们都看见了。来吧,我们重建耶路撒冷的城墙,免得再受凌辱!」
\par 18 我告诉他们我神施恩的手怎样帮助我,并王对我所说的话。他们就说:「我们起来建造吧!」於是他们奋勇做这善工。
\par 19 但和伦人参巴拉,并为奴的亚扪人多比雅和亚拉伯人基善听见就嗤笑我们,藐视我们,说:「你们做什麽呢?要背叛王吗?」
\par 20 我回答他们说:「天上的神必使我们亨通。我们作他仆人的,要起来建造;你们却在耶路撒冷无分、无权、无纪念。」

\chapter{3}

\par 1 那时,大祭司以利亚实和他的弟兄众祭司起来建立羊门,分别为圣,安立门扇,又筑城墙到哈米亚楼,直到哈楠业楼,分别为圣。
\par 2 其次是耶利哥人建造。其次是音利的儿子撒刻建造。
\par 3 哈西拿的子孙建立鱼门,架横梁、安门扇,和闩锁。
\par 4 其次是哈哥斯的孙子、乌利亚的儿子米利末修造。其次是米示萨别的孙子、比利迦的儿子米书兰修造。其次是巴拿的儿子撒督修造。
\par 5 其次是提哥亚人修造;但是他们的贵胄不用肩(原文作颈项)担他们主的工作。
\par 6 巴西亚的儿子耶何耶大与比所玳的儿子米书兰修造古门,架横梁,安门扇和闩锁。
\par 7 其次是基遍人米拉提,米伦人雅顿与基遍人,并属河西总督所管的米斯巴人修造。
\par 8 其次是银匠哈海雅的儿子乌薛修造。其次是做香的哈拿尼雅修造。这些人修坚耶路撒冷,直到宽墙。
\par 9 其次是管理耶路撒冷一半、户珥的儿子利法雅修造。
\par 10 其次是哈路抹的儿子耶大雅对著自己的房屋修造。其次是哈沙尼的儿子哈突修造。
\par 11 哈琳的儿子玛基雅和巴哈摩押的儿子哈述修造一段,并修造炉楼。
\par 12 其次是管理耶路撒冷那一半、哈罗黑的儿子沙龙和他的女儿们修造。
\par 13 哈嫩和撒挪亚的居民修造谷门,立门,安门扇和闩锁,又建筑城墙一千肘,直到粪厂门。
\par 14 管理伯哈基琳、利甲的儿子玛基雅修造粪厂门,立门,安门扇和闩锁。
\par 15 管理米斯巴、各荷西的儿子沙仑修造泉门,立门,盖门顶,安门扇和闩锁,又修造靠近王园西罗亚池的墙垣,直到那从大卫城下来的台阶。
\par 16 其次是管理伯夙一半、押卜的儿子尼希米修造,直到大卫坟地的对面,又到挖成的池子,并勇士的房屋。
\par 17 其次是利未人巴尼的儿子利宏修造。其次是管理基伊拉一半、哈沙比雅为他所管的本境修造。
\par 18 其次是利未人弟兄中管理基伊拉那一半、希拿达的儿子巴瓦伊修造。
\par 19 其次是管理米斯巴、耶书亚的儿子以谢修造一段,对著武库的上坡、城墙转弯之处。
\par 20 其次是萨拜的儿子巴录竭力修造一段,从城墙转弯,直到大祭司以利亚实的府门。
\par 21 其次是哈哥斯的孙子、乌利亚的儿子米利末修造一段,从以利亚实的府门,直到以利亚实府的尽头。
\par 22 其次是住平原的祭司修造。
\par 23 其次是便雅悯与哈述对著自己的房屋修造。其次是亚难尼的孙子、玛西雅的儿子亚撒利雅在靠近自己的房屋修造。
\par 24 其次是希拿达的儿子宾内修造一段,从亚撒利雅的房屋直到城墙转弯,又到城角。
\par 25 乌赛的儿子巴拉修造对著城墙的转弯和王上宫凸出来的城楼,靠近护卫院的那一段。其次是巴录的儿子毗大雅修造。
\par 26 (尼提宁住在俄斐勒,直到朝东水门的对面和凸出来的城楼。)
\par 27 其次是提哥亚人又修一段,对著那凸出来的大楼,直到俄斐勒的墙。
\par 28 从马门往上,众祭司各对自己的房屋修造。
\par 29 其次是音麦的儿子撒督对著自己的房屋修造。其次是守东门、示迦尼的儿子示玛雅修造。
\par 30 其次是示利米雅的儿子哈拿尼雅和萨拉的第六子哈嫩又修一段。其次是比利迦的儿子米书兰对著自己的房屋修造。
\par 31 其次是银匠玛基雅修造到尼提宁和商人的房屋,对著哈米弗甲门,直到城的角楼。
\par 32 银匠与商人在城的角楼和羊门中间修造。

\chapter{4}

\par 1 参巴拉听见我们修造城墙就发怒,大大恼恨,嗤笑犹大人,
\par 2 对他弟兄和撒玛利亚的军兵说:「这些软弱的犹大人做什麽呢?要保护自己吗?要献祭吗?要一日成功吗?要从土堆里拿出火烧的石头再立墙吗?」
\par 3 亚扪人多比雅站在旁边,说:「他们所修造的石墙,就是狐狸上去也必 倒。」
\par 4 我们的神啊,求你垂听,因为我们被藐视。求你使他们的毁谤归於他们的头上,使他们在掳到之地作为掠物。
\par 5 不要遮掩他们的罪孽,不要使他们的罪恶从你面前涂抹,因为他们在修造的人眼前惹动你的怒气。
\par 6 这样,我们修造城墙,城墙就都连络,高至一半,因为百姓专心做工。
\par 7 参巴拉、多比雅、亚拉伯人、亚扪人、亚实突人听见修造耶路撒冷城墙,著手进行堵塞破裂的地方,就甚发怒。
\par 8 大家同谋要来攻击耶路撒冷,使城内扰乱。
\par 9 然而,我们祷告我们的神,又因他们的缘故,就派人看守,昼夜防备。
\par 10 犹大人说:「灰土尚多,扛抬的人力气已经衰败,所以我们不能建造城墙。」
\par 11 我们的敌人且说:「趁他们不知不见,我们进入他们中间,杀他们,使工作止住。」
\par 12 那靠近敌人居住的犹大人十次从各处来见我们,说:「你们必要回到我们那里。」
\par 13 所以我使百姓各按宗族拿刀、拿枪、拿弓站在城墙後边低洼的空处。
\par 14 我察看了,就起来对贵胄、官长,和其余的人说:「不要怕他们!当记念主是大而可畏的。你们要为弟兄、儿女、妻子、家产争战。」
\par 15 仇敌听见我们知道他们的心意,见神也破坏他们的计谋,就不来了。我们都回到城墙那里,各做各的工。
\par 16 从那日起,我的仆人一半做工,一半拿枪、拿盾牌、拿弓、穿(或作:拿)铠甲,官长都站在犹大众人的後边。
\par 17 修造城墙的,扛抬材料的,都一手做工一手拿兵器。
\par 18 修造的人都腰间佩刀修造,吹角的人在我旁边。
\par 19 我对贵胄、官长,和其余的人说:「这工程浩大,我们在城墙上相离甚远;
\par 20 你们听见角声在那里,就聚集到我们那里去。我们的神必为我们争战。」
\par 21 於是,我们做工,一半拿兵器,从天亮直到星宿出现的时候。
\par 22 那时,我又对百姓说:「各人和他的仆人当在耶路撒冷住宿,好在夜间保守我们,白昼做工。」
\par 23 这样,我和弟兄仆人,并跟从我的护兵都不脱衣服,出去打水也带兵器。

\chapter{5}

\par 1 百姓和他们的妻大大呼号,埋怨他们的弟兄犹大人。
\par 2 有的说:「我们和儿女人口众多,要去得粮食度命」;
\par 3 有的说:「我们典了田地、葡萄园、房屋,要得粮食充饥」;
\par 4 有的说:「我们已经指著田地、葡萄园,借了钱给王纳税。
\par 5 我们的身体与我们弟兄的身体一样;我们的儿女与他们的儿女一般。现在我们将要使儿女作人的仆婢,我们的女儿已有为婢的;我们并无力拯救,因为我们的田地、葡萄园已经归了别人。」
\par 6 我听见他们呼号说这些话,便甚发怒。
\par 7 我心里筹划,就斥责贵胄和官长说:「你们各人向弟兄取利!」於是我招聚大会攻击他们。
\par 8 我对他们说:「我们尽力赎回我们弟兄,就是卖与外邦的犹大人;你们还要卖弟兄,使我们赎回来吗?」他们就静默不语,无话可答。
\par 9 我又说:「你们所行的不善!你们行事不当敬畏我们的神吗?不然,难免我们的仇敌外邦人毁谤我们。
\par 10 我和我的弟兄与仆人也将银钱粮食借给百姓;我们大家都当免去利息。
\par 11 如今我劝你们将他们的田地、葡萄园、橄榄园、房屋,并向他们所取的银钱、粮食、新酒,和油,百分之一的利息都归还他们。」
\par 12 众人说:「我们必归还,不再向他们索要,必照你的话行。」我就召了祭司来,叫众人起誓,必照著所应许的而行。
\par 13 我也抖著胸前的衣襟,说:「凡不成就这应许的,愿神照样抖他离开家产和他劳碌得来的,直到抖空了。」会众都说:「阿们!」又赞美耶和华。百姓就照著所应许的去行。
\par 14 自从我奉派作犹大地的省长,就是从亚达薛西王二十年直到三十二年,共十二年之久,我与我弟兄都没有吃省长的俸禄。
\par 15 在我以前的省长加重百姓的担子,每日索要粮食和酒,并银子四十舍客勒,就是他们的仆人也辖制百姓;但我因敬畏神不这样行。
\par 16 并且我恒心修造城墙,并没有置买田地;我的仆人也都聚集在那里做工。
\par 17 除了从四围外邦中来的犹大人以外,有犹大平民和官长一百五十人在我席上吃饭。
\par 18 每日预备一只公牛,六只肥羊,又预备些飞禽;每十日一次,多预备各样的酒。虽然如此,我并不要省长的俸禄,因为百姓服役甚重。
\par 19 我的神啊,求你记念我为这百姓所行的一切事,施恩与我。

\chapter{6}

\par 1 参巴拉、多比雅、亚拉伯人基善和我们其余的仇敌听见我已经修完了城墙,其中没有破裂之处(那时我还没有安门扇),
\par 2 参巴拉和基善就打发人来见我,说:「请你来,我们在阿挪平原的一个村庄相会。」他们却想害我。
\par 3 於是我差遣人去见他们,说:「我现在办理大工,不能下去。焉能停工下去见你们呢?」
\par 4 他们这样四次打发人来见我,我都如此回答他们。
\par 5 参巴拉第五次打发仆人来见我,手里拿著未封的信,
\par 6 信上写著说:「外邦人中有风声,迦施慕(就是基善,见二章十九节)也说,你和犹大人谋反,修造城墙,你要作他们的王;
\par 7 你又派先知在耶路撒冷指著你宣讲,说在犹大有王。现在这话必传与王知;所以请你来,与我们彼此商议。」
\par 8 我就差遣人去见他,说:「你所说的这事,一概没有,是你心里捏造的。」
\par 9 他们都要使我们惧怕,意思说,他们的手必软弱,以致工作不能成就。神啊,求你坚固我的手。
\par 10 我到了米希大别的孙子、第来雅的儿子示玛雅家里;那时,他闭门不出。他说:「我们不如在神的殿里会面,将殿门关锁;因为他们要来杀你,就是夜里来杀你。」
\par 11 我说:「像我这样的人岂要逃跑呢?像我这样的人岂能进入殿里保全生命呢?我不进去!」
\par 12 我看明神没有差遣他,是他自己说这话攻击我,是多比雅和参巴拉贿买了他。
\par 13 贿买他的缘故,是要叫我惧怕,依从他犯罪,他们好传扬恶言毁谤我。
\par 14 我的神啊,多比雅、参巴拉、女先知挪亚底,和其余的先知要叫我惧怕,求你记念他们所行的这些事。
\par 15 以禄月二十五日,城墙修完了,共修了五十二天。
\par 16 我们一切仇敌、四围的外邦人听见了便惧怕,愁眉不展;因为见这工作完成是出乎我们的神。
\par 17 在那些日子,犹大的贵胄屡次寄信与多比雅,多比雅也来信与他们。
\par 18 在犹大有许多人与多比雅结盟;因他是亚拉的儿子,示迦尼的女婿,并且他的儿子约哈难娶了比利迦儿子米书兰的女儿为妻。
\par 19 他们常在我面前说多比雅的善行,也将我的话传与他。多比雅又常寄信来,要叫我惧怕。

\chapter{7}

\par 1 城墙修完,我安了门扇,守门的、歌唱的,和利未人都已派定。
\par 2 我就派我的弟兄哈拿尼和营楼的宰官哈拿尼雅管理耶路撒冷;因为哈拿尼雅是忠信的,又敬畏神过於众人。
\par 3 我吩咐他们说:「等到太阳上升才可开耶路撒冷的城门;人尚看守的时候就要关门上闩;也当派耶路撒冷的居民各按班次看守自己房屋对面之处。」
\par 4 城是广大,其中的民却稀少,房屋还没有建造。
\par 5 我的神感动我心,招聚贵胄、官长,和百姓,要照家谱计算。我找著第一次上来之人的家谱,其上写著:
\par 6 巴比伦王尼布甲尼撒从前掳去犹大省的人,现在他们的子孙从被掳到之地回耶路撒冷和犹大,各归本城。
\par 7 他们是同著所罗巴伯、耶书亚、尼希米、亚撒利雅、拉米、拿哈玛尼、末底改、必珊、米斯毗列、比革瓦伊、尼宏、巴拿回来的。
\par 8 以色列人民的数目记在下面:巴录的子孙二千一百七十二名;
\par 9 示法提雅的子孙三百七十二名;
\par 10 亚拉的子孙六百五十二名;
\par 11 巴哈摩押的後裔,就是耶书亚和约押的子孙二千八百一十八名;
\par 12 以拦的子孙一千二百五十四名;
\par 13 萨土的子孙八百四十五名;
\par 14 萨改的子孙七百六十名;
\par 15 宾内的子孙六百四十八名;
\par 16 比拜的子孙六百二十八名;
\par 17 押甲的子孙二千三百二十二名;
\par 18 亚多尼干的子孙六百六十七名;
\par 19 比革瓦伊的子孙二千零六十七名;
\par 20 亚丁的子孙六百五十五名;
\par 21 亚特的後裔,就是希西家的子孙九十八名;
\par 22 哈顺的子孙三百二十八名;
\par 23 比赛的子孙三百二十四名;
\par 24 哈拉的子孙一百一十二名;
\par 25 基遍人九十五名;
\par 26 伯利恒人和尼陀法人共一百八十八名;
\par 27 亚拿突人一百二十八名;
\par 28 伯亚斯玛弗人四十二名;
\par 29 基列耶琳人、基非拉人、比录人共七百四十三名;
\par 30 拉玛人和迦巴人共六百二十一名;
\par 31 默玛人一百二十二名;
\par 32 伯特利人和艾人共一百二十三名;
\par 33 别的尼波人五十二名;
\par 34 别的以拦子孙一千二百五十四名;
\par 35 哈琳的子孙三百二十名;
\par 36 耶利哥人三百四十五名;
\par 37 罗德人、哈第人、阿挪人共七百二十一名;
\par 38 西拿人三千九百三十名。
\par 39 祭司:耶书亚家,耶大雅的子孙九百七十三名;
\par 40 音麦的子孙一千零五十二名;
\par 41 巴施户珥的子孙一千二百四十七名;
\par 42 哈琳的子孙一千零一十七名。
\par 43 利未人:何达威的後裔,就是耶书亚和甲篾的子孙七十四名。
\par 44 歌唱的:亚萨的子孙一百四十八名。
\par 45 守门的:沙龙的子孙、亚特的子孙、达们的子孙、亚谷的子孙、哈底大的子孙、朔拜的子孙,共一百三十八名。
\par 46 尼提宁(就是殿役):西哈的子孙、哈苏巴的子孙、答巴俄的子孙、
\par 47 基绿的子孙、西亚的子孙、巴顿的子孙、
\par 48 利巴拿的子孙、哈迦巴的子孙、萨买的子孙、
\par 49 哈难的子孙、吉德的子孙、迦哈的子孙、
\par 50 利亚雅的子孙、利汛的子孙、尼哥大的子孙、
\par 51 迦散的子孙、乌撒的子孙、巴西亚的子孙、
\par 52 比赛的子孙、米乌宁的子孙、尼普心的子孙、
\par 53 巴卜的子孙、哈古巴的子孙、哈忽的子孙、
\par 54 巴洗律的子孙、米希大的子孙、哈沙的子孙、
\par 55 巴柯的子孙、西西拉的子孙、答玛的子孙、
\par 56 尼细亚的子孙、哈提法的子孙。
\par 57 所罗门仆人的後裔,就是琐太的子孙、琐斐列的子孙、比路大的子孙、
\par 58 雅拉的子孙、达昆的子孙、吉德的子孙、
\par 59 示法提雅的子孙、哈替的子孙、玻黑列哈斯巴音的子孙、亚们的子孙。
\par 60 尼提宁和所罗门仆人的後裔共三百九十二名。
\par 61 从特米拉、特哈萨、基绿、亚顿、音麦上来的,不能指明他们的宗族谱系是以色列人不是;
\par 62 他们是第莱雅的子孙、多比雅的子孙、尼哥大的子孙,共六百四十二名。
\par 63 祭司中,哈巴雅的子孙、哈哥斯的子孙、巴西莱的子孙;因为他们的先祖娶了基列人巴西莱的女儿为妻,所以起名叫巴西莱。
\par 64 这三家的人在族谱之中寻查自己的谱系,却寻不著,因此算为不洁,不准供祭司的职任。
\par 65 省长对他们说:「不可吃至圣的物,直到有用乌陵和土明决疑的祭司兴起来。」
\par 66 会众共有四万二千三百六十名。
\par 67 此外,还有他们的仆婢七千三百三十七名,又有歌唱的男女二百四十五名。
\par 68 他们有马七百三十六匹,骡子二百四十五匹,
\par 69 骆驼四百三十五只,驴六千七百二十匹。
\par 70 有些族长为工程捐助。省长捐入库中的金子一千达利克,碗五十个,祭司的礼服五百三十件。
\par 71 又有族长捐入工程库的金子二万达利克,银子二千二百弥拿。
\par 72 其余百姓所捐的金子二万达利克,银子二千弥拿,祭司的礼服六十七件。
\par 73 於是祭司、利未人、守门的、歌唱的、民中的一些人、尼提宁,并以色列众人,各住在自己的城里。

\chapter{8}

\par 1 到了七月,以色列人住在自己的城里。那时,他们如同一人聚集在水门前的宽阔处,请文士以斯拉将耶和华藉摩西传给以色列人的律法书带来。
\par 2 七月初一日,祭司以斯拉将律法书带到听了能明白的男女会众面前。
\par 3 在水门前的宽阔处,从清早到晌午,在众男女、一切听了能明白的人面前读这律法书。众民侧耳而听。
\par 4 文士以斯拉站在为这事特备的木台上。玛他提雅、示玛、亚奈雅、乌利亚、希勒家,和玛西雅站在他的右边;毗大雅、米沙利、玛基雅、哈顺、哈拔大拿、撒迦利亚,和米书兰站在他的左边。
\par 5 以斯拉站在众民以上,在众民眼前展开这书。他一展开,众民就都站起来。
\par 6 以斯拉称颂耶和华至大的神;众民都举手应声说:「阿们!阿们!」就低头,面伏於地,敬拜耶和华。
\par 7 耶书亚、巴尼、示利比、雅悯、亚谷、沙比太、荷第雅、玛西雅、基利他、亚撒利雅、约撒拔、哈难、毗莱雅,和利未人使百姓明白律法;百姓都站在自己的地方。
\par 8 他们清清楚楚地念神的律法书,讲明意思,使百姓明白所念的。
\par 9 省长尼希米和作祭司的文士以斯拉,并教训百姓的利未人,对众民说:「今日是耶和华你们神的圣日,不要悲哀哭泣。」这是因为众民听见律法书上的话都哭了;
\par 10 又对他们说:「你们去吃肥美的,喝甘甜的,有不能预备的就分给他,因为今日是我们主的圣日。你们不要忧愁,因靠耶和华而得的喜乐是你们的力量。」
\par 11 於是利未人使众民静默,说:「今日是圣日;不要作声,也不要忧愁。」
\par 12 众民都去吃喝,也分给人,大大快乐,因为他们明白所教训他们的话。
\par 13 次日,众民的族长、祭司,和利未人都聚集到文士以斯拉那里,要留心听律法上的话。
\par 14 他们见律法上写著,耶和华藉摩西吩咐以色列人要在七月节住棚,
\par 15 并要在各城和耶路撒冷宣传报告说:「你们当上山,将橄榄树、野橄榄树、番石榴树、棕树,和各样茂密树的枝子取来,照著所写的搭棚。」
\par 16 於是百姓出去,取了树枝来,各人在自己的房顶上,或院内,或神殿的院内,或水门的宽阔处,或以法莲门的宽阔处搭棚。
\par 17 从掳到之地归回的全会众就搭棚,住在棚里。从嫩的儿子约书亚的时候直到这日,以色列人没有这样行。於是众人大大喜乐。
\par 18 从头一天直到末一天,以斯拉每日念神的律法书。众人守节七日,第八日照例有严肃会。

\chapter{9}

\par 1 这月二十四日,以色列人聚集禁食,身穿麻衣,头蒙灰尘。
\par 2 以色列人(人:原文作种类)就与一切外邦人离绝,站著承认自己的罪恶和列祖的罪孽。
\par 3 那日的四分之一站在自己的地方念耶和华他们神的律法书,又四分之一认罪,敬拜耶和华他们的神。
\par 4 耶书亚、巴尼、甲篾、示巴尼、布尼、示利比、巴尼、基拿尼站在利未人的台上,大声哀求耶和华他们的神。
\par 5 利未人耶书亚、甲篾、巴尼、哈沙尼、示利比、荷第雅、示巴尼、毗他希雅说:「你们要站起来称颂耶和华你们的神,永世无尽。耶和华啊,你荣耀之名是应当称颂的!超乎一切称颂和赞美。」
\par 6 「你,惟独你是耶和华!你造了天和天上的天,并天上的万象,地和地上的万物,海和海中所有的;这一切都是你所保存的。天军也都敬拜你。
\par 7 你是耶和华 神,曾拣选亚伯兰,领他出迦勒底的吾珥,给他改名叫亚伯拉罕。
\par 8 你见他在你面前心里诚实,就与他立约,应许把迦南人、赫人、亚摩利人、比利洗人、耶布斯人、革迦撒人之地赐给他的後裔,且应验了你的话,因为你是公义的。
\par 9 「你曾看见我们列祖在埃及所受的困苦,垂听他们在红海边的哀求,
\par 10 就施行神迹奇事在法老和他一切臣仆,并他国中的众民身上。你也得了名声,正如今日一样,因为你知道他们向我们列祖行事狂傲。
\par 11 你又在我们列祖面前把海分开,使他们在海中行走乾地,将追赶他们的人抛在深海,如石头抛在大水中;
\par 12 并且白昼用云柱引导他们,黑夜用火柱照亮他们当行的路。
\par 13 你也降临在西乃山,从天上与他们说话,赐给他们正直的典章、真实的律法、美好的条例与诫命,
\par 14 又使他们知道你的安息圣日,并藉你仆人摩西传给他们诫命、条例、律法。
\par 15 从天上赐下粮食充他们的饥,从磐石使水流出解他们的渴,又吩咐他们进去得你起誓应许赐给他们的地。
\par 16 「但我们的列祖行事狂傲,硬著颈项不听从你的诫命;
\par 17 不肯顺从,也不记念你在他们中间所行的奇事,竟硬著颈项,居心背逆,自立首领,要回他们为奴之地。但你是乐意饶恕人,有恩典,有怜悯,不轻易发怒,有丰盛慈爱的神,并不丢弃他们。
\par 18 他们虽然铸了一只牛犊,彼此说『这是领你出埃及的神』,因而大大惹动你的怒气;
\par 19 你还是大施怜悯,在旷野不丢弃他们。白昼,云柱不离开他们,仍引导他们行路;黑夜,火柱也不离开他们,仍照亮他们当行的路。
\par 20 你也赐下你良善的灵教训他们;未尝不赐吗哪使他们 口,并赐水解他们的渴。
\par 21 在旷野四十年,你养育他们,他们就一无所缺:衣服没有穿破,脚也没有肿。
\par 22 并且你将列国之地照分赐给他们,他们就得了西宏之地、希实本王之地,和巴珊王噩之地。
\par 23 你也使他们的子孙多如天上的星,带他们到你所应许他们列祖进入得为业之地。
\par 24 这样,他们进去得了那地,你在他们面前制伏那地的居民,就是迦南人;将迦南人和其君王,并那地的居民,都交在他们手里,让他们任意而待。
\par 25 他们得了坚固的城邑、肥美的地土、充满各样美物的房屋、凿成的水井、葡萄园、橄榄园,并许多果木树。他们就吃而得饱,身体肥胖,因你的大恩,心中快乐。
\par 26 「然而,他们不顺从,竟背叛你,将你的律法丢在背後,杀害那劝他们归向你的众先知,大大惹动你的怒气。
\par 27 所以你将他们交在敌人的手中,磨难他们。他们遭难的时候哀求你,你就从天上垂听,照你的大怜悯赐给他们拯救者,救他们脱离敌人的手。
\par 28 但他们得平安之後,又在你面前行恶,所以你丢弃他们在仇敌的手中,使仇敌辖制他们。然而他们转回哀求你,你仍从天上垂听,屡次照你的怜悯拯救他们,
\par 29 又警戒他们,要使他们归服你的律法。他们却行事狂傲,不听从你的诫命,干犯你的典章(人若遵行就必因此活著),扭转肩头,硬著颈项,不肯听从。
\par 30 但你多年宽容他们,又用你的灵藉众先知劝戒他们,他们仍不听从,所以你将他们交在列国之民的手中。
\par 31 然而你大发怜悯,不全然灭绝他们,也不丢弃他们;因为你是有恩典、有怜悯的神。
\par 32 「我们的神啊,你是至大、至能、至可畏、守约施慈爱的神。我们的君王、首领、祭司、先知、列祖,和你的众民,从亚述列王的时候直到今日所遭遇的苦难,现在求你不要以为小。
\par 33 在一切临到我们的事上,你却是公义的;因你所行的是诚实,我们所做的是邪恶。
\par 34 我们的君王、首领、祭司、列祖都不遵守你的律法,不听从你的诫命和你警戒他们的话。
\par 35 他们在本国里沾你大恩的时候,在你所赐给他们这广大肥美之地上不事奉你,也不转离他们的恶行。
\par 36 我们现今作了奴仆;至於你所赐给我们列祖享受其上的土产,并美物之地,看哪,我们在这地上作了奴仆!
\par 37 这地许多出产归了列王,就是你因我们的罪所派辖制我们的。他们任意辖制我们的身体和牲畜,我们遭了大难。」
\par 38 因这一切的事,我们立确实的约,写在册上。我们的首领、利未人,和祭司都签了名。

\chapter{10}

\par 1 签名的是:哈迦利亚的儿子省长尼希米,和西底家;
\par 2 祭司:西莱雅、亚撒利雅、耶利米、
\par 3 巴施户珥、亚玛利雅、玛基雅、
\par 4 哈突、示巴尼、玛鹿、
\par 5 哈琳、米利末、俄巴底亚、
\par 6 但以理、近顿、巴录、
\par 7 米书兰、亚比雅、米雅民、
\par 8 玛西亚、璧该、示玛雅;
\par 9 又有利未人,就是亚散尼的儿子耶书亚、希拿达的子孙宾内、甲篾;
\par 10 还有他们的弟兄示巴尼、荷第雅、基利他、毗莱雅、哈难、
\par 11 米迦、利合、哈沙比雅、
\par 12 撒刻、示利比、示巴尼、
\par 13 荷第雅、巴尼、比尼努;
\par 14 又有民的首领,就是巴录、巴哈摩押、以拦、萨土、巴尼、
\par 15 布尼、押甲、比拜、
\par 16 亚多尼雅、比革瓦伊、亚丁、
\par 17 亚特、希西家、押朔、
\par 18 荷第雅、哈顺、比赛、
\par 19 哈拉、亚拿突、尼拜、
\par 20 抹比押、米书兰、希悉、
\par 21 米示萨别、撒督、押杜亚、
\par 22 毗拉提、哈难、亚奈雅、
\par 23 何细亚、哈拿尼雅、哈述、
\par 24 哈罗黑、毗利哈、朔百、
\par 25 利宏、哈沙拿、玛西雅、
\par 26 亚希雅、哈难、亚难、
\par 27 玛鹿、哈琳、巴拿。
\par 28 其余的民、祭司、利未人、守门的、歌唱的、尼提宁,和一切离绝邻邦居民归服神律法的,并他们的妻子、儿女,凡有知识能明白的,
\par 29 都随从他们贵胄的弟兄,发咒起誓,必遵行神藉他仆人摩西所传的律法,谨守遵行耶和华我们主的一切诫命、典章、律例;
\par 30 并不将我们的女儿嫁给这地的居民,也不为我们的儿子娶他们的女儿。
\par 31 这地的居民若在安息日,或什麽圣日,带了货物或粮食来卖给我们,我们必不买。每逢第七年必不耕种,凡欠我们债的必不追讨。
\par 32 我们又为自己定例,每年各人捐银一舍客勒三分之一,为我们神殿的使用,
\par 33 就是为陈设饼、常献的素祭,和燔祭,安息日、月朔、节期所献的与圣物,并以色列人的赎罪祭,以及我们神殿里一切的费用。
\par 34 我们的祭司、利未人,和百姓都掣签,看每年是哪一族按定期将献祭的柴奉到我们神的殿里,照著律法上所写的,烧在耶和华我们神的坛上。
\par 35 又定每年将我们地上初熟的土产和各样树上初熟的果子都奉到耶和华的殿里。
\par 36 又照律法上所写的,将我们头胎的儿子和首生的牛羊都奉到我们神的殿,交给我们神殿里供职的祭司;
\par 37 并将初熟之麦子所磨的面和举祭、各样树上初熟的果子、新酒与油奉给祭司,收在我们神殿的库房里,把我们地上所产的十分之一奉给利未人,因利未人在我们一切城邑的土产中当取十分之一。
\par 38 利未人取十分之一的时候,亚伦的子孙中,当有一个祭司与利未人同在。利未人也当从十分之一中取十分之一,奉到我们神殿的屋子里,收在库房中。
\par 39 以色列人和利未人要将五谷、新酒,和油为举祭,奉到收存圣所器皿的屋子里,就是供职的祭司、守门的、歌唱的所住的屋子。这样,我们就不离弃我们神的殿。

\chapter{11}

\par 1 百姓的首领住在耶路撒冷。其余的百姓掣签,每十人中使一人来住在圣城耶路撒冷,那九人住在别的城邑。
\par 2 凡甘心乐意住在耶路撒冷的,百姓都为他们祝福。
\par 3 以色列人、祭司、利未人、尼提宁,和所罗门仆人的後裔都住在犹大城邑,各在自己的地业中。本省的首领住在耶路撒冷的记在下面:
\par 4 其中有些犹大人和便雅悯人。犹大人中有法勒斯的子孙、乌西雅的儿子亚他雅。乌西雅是撒迦利雅的儿子;撒迦利雅是亚玛利雅的儿子;亚玛利雅是示法提雅的儿子;示法提雅是玛勒列的儿子。
\par 5 又有巴录的儿子玛西雅。巴录是谷何西的儿子;谷何西是哈赛雅的儿子;哈赛雅是亚大雅的儿子;亚大雅是约雅立的儿子;约雅立是撒迦利雅的儿子;撒迦利雅是示罗尼的儿子。
\par 6 住在耶路撒冷、法勒斯的子孙共四百六十八名,都是勇士。
\par 7 便雅悯人中有米书兰的儿子撒路。米书兰是约叶的儿子;约叶是毗大雅的儿子;毗大雅是哥赖雅的儿子;哥赖雅是玛西雅的儿子;玛西雅是以铁的儿子;以铁是耶筛亚的儿子。
\par 8 其次有迦拜、撒来的子孙,共九百二十八名。
\par 9 细基利的儿子约珥是他们的长官。哈西努亚的儿子犹大是耶路撒冷的副官。
\par 10 祭司中有雅斤,又有约雅立的儿子耶大雅;
\par 11 还有管理神殿的西莱雅。西莱雅是希勒家的儿子;希勒家是米书兰的儿子;米书兰是撒督的儿子;撒督是米拉约的儿子;米拉约是亚希突的儿子。
\par 12 还有他们的弟兄在殿里供职的,共八百二十二名;又有耶罗罕的儿子亚大雅。耶罗罕是毗拉利的儿子;毗拉利是暗洗的儿子;暗洗是撒迦利亚的儿子;撒迦利亚是巴施户珥的儿子;巴施户珥是玛基雅的儿子。
\par 13 还有他的弟兄作族长的,二百四十二名;又有亚萨列的儿子亚玛帅。亚萨列是亚哈赛的儿子;亚哈赛是米实利末的儿子;米实利末是音麦的儿子。
\par 14 还有他们弟兄、大能的勇士共一百二十八名。哈基多琳的儿子撒巴第业是他们的长官。
\par 15 利未人中有哈述的儿子示玛雅。哈述是押利甘的儿子;押利甘是哈沙比雅的儿子;哈沙比雅是布尼的儿子。
\par 16 又有利未人的族长沙比太和约撒拔管理神殿的外事。
\par 17 祈祷的时候,为称谢领首的是米迦的儿子玛他尼。米迦是撒底的儿子;撒底是亚萨的儿子;又有玛他尼弟兄中的八布迦为副。还有沙母亚的儿子押大。沙母亚是加拉的儿子;加拉是耶杜顿的儿子。
\par 18 在圣城的利未人共二百八十四名。
\par 19 守门的是亚谷和达们,并守门的弟兄,共一百七十二名。
\par 20 其余的以色列人、祭司、利未人都住在犹大的一切城邑,各在自己的地业中。
\par 21 尼提宁却住在俄斐勒;西哈和基斯帕管理他们。
\par 22 在耶路撒冷、利未人的长官,管理神殿事务的是歌唱者亚萨的子孙、巴尼的儿子乌西。巴尼是哈沙比雅的儿子;哈沙比雅是玛他尼的儿子;玛他尼是米迦的儿子。
\par 23 王为歌唱的出命令,每日供给他们必有一定之粮。
\par 24 犹大儿子谢拉的子孙、米示萨别的儿子毗他希雅辅助王办理犹大民的事。
\par 25 至於村庄和属村庄的田地,有犹大人住在基列亚巴和属基列亚巴的乡村;底本和属底本的乡村;叶甲薛和属叶甲薛的村庄;
\par 26 耶书亚、摩拉大、伯帕列、
\par 27 哈萨书亚、别是巴,和属别是巴的乡村;
\par 28 洗革拉、米哥拿,和属米哥拿的乡村;
\par 29 音临门、琐拉、耶末、
\par 30 撒挪亚、亚杜兰,和属这两处的村庄;拉吉和属拉吉的田地;亚西加和属亚西加的乡村。他们所住的地方是从别是巴直到欣嫩谷。
\par 31 便雅悯人从迦巴起,住在密抹、亚雅、伯特利和属伯特利的乡村。
\par 32 亚拿突、挪伯、亚难雅、
\par 33 夏琐、拉玛、基他音、
\par 34 哈叠、洗编、尼八拉、
\par 35 罗德、阿挪、匠人之谷。
\par 36 利未人中有几班曾住在犹大地归於便雅悯的。

\chapter{12}

\par 1 同著撒拉铁的儿子所罗巴伯和耶书亚回来的祭司与利未人记在下面:祭司是西莱雅、耶利米、以斯拉、
\par 2 亚玛利雅、玛鹿、哈突、
\par 3 示迦尼、利宏、米利末、
\par 4 易多、近顿、亚比雅、
\par 5 米雅民、玛底雅、璧迦、
\par 6 示玛雅、约雅立、耶大雅、
\par 7 撒路、亚木、希勒家、耶大雅。这些人在耶书亚的时候作祭司和他们弟兄的首领。
\par 8 利未人是耶书亚、宾内、甲篾、示利比、犹大、玛他尼。这玛他尼和他的弟兄管理称谢的事。
\par 9 他们的弟兄八布迦和乌尼照自己的班次与他们相对。
\par 10 耶书亚生约雅金;约雅金生以利亚实;以利亚实生耶何耶大;
\par 11 耶何耶大生约拿单;约拿单生押杜亚。
\par 12 在约雅金的时候,祭司作族长的西莱雅族(或作:班;本段同)有米拉雅;耶利米族有哈拿尼雅;
\par 13 以斯拉族有米书兰;亚玛利雅族有约哈难;
\par 14 米利古族有约拿单;示巴尼族有约瑟;
\par 15 哈琳族有押拿;米拉约族有希勒恺;
\par 16 易多族有撒迦利亚;近顿族有米书兰;
\par 17 亚比雅族有细基利;米拿民族某;摩亚底族有毗勒太;
\par 18 璧迦族有沙母亚;示玛雅族有约拿单;
\par 19 约雅立族有玛特乃;耶大雅族有乌西;
\par 20 撒来族有加莱;亚木族有希伯;
\par 21 希勒家族有哈沙比雅;耶大雅族有拿坦业。
\par 22 至於利未人,当以利亚实、耶何耶大、约哈难、押杜亚的时候,他们的族长记在册上。波斯王大利乌在位的时候,作族长的祭司也记在册上。
\par 23 利未人作族长的记在历史上,直到以利亚实的儿子约哈难的时候。
\par 24 利未人的族长是哈沙比雅、示利比、甲篾的儿子耶书亚,与他们弟兄的班次相对,照著神人大卫的命令一班一班地赞美称谢。
\par 25 玛他尼、八布迦、俄巴底亚、米书兰、达们、亚谷是守门的,就是在库房那里守门。
\par 26 这都是在约撒达的孙子、耶书亚的儿子约雅金和省长尼希米,并祭司文士以斯拉的时候,有职任的。
\par 27 耶路撒冷城墙告成的时候,众民就把各处的利未人招到耶路撒冷,要称谢、歌唱、敲钹、鼓瑟、弹琴,欢欢喜喜地行告成之礼。
\par 28 歌唱的人从耶路撒冷的周围和尼陀法的村庄与伯吉甲,
\par 29 又从迦巴和押玛弗的田地聚集,因为歌唱的人在耶路撒冷四围为自己立了村庄。
\par 30 祭司和利未人就洁净自己,也洁净百姓和城门,并城墙。
\par 31 我带犹大的首领上城,使称谢的人分为两大队,排列而行:第一队在城上往右边向粪厂门行走,
\par 32 在他们後头的有何沙雅与犹大首领的一半,
\par 33 又有亚撒利雅、以斯拉、米书兰、
\par 34 犹大、便雅悯、示玛雅、耶利米。
\par 35 还有些吹号之祭司的子孙,约拿单的儿子撒迦利亚。约拿单是示玛雅的儿子;示玛雅是玛他尼的儿子;玛他尼是米该亚的儿子;米该亚是撒刻的儿子;撒刻是亚萨的儿子;
\par 36 又有撒迦利亚的弟兄示玛雅、亚撒利、米拉莱、基拉莱、玛艾、拿坦业、犹大、哈拿尼,都拿著神人大卫的乐器,文士以斯拉引领他们。
\par 37 他们经过泉门往前,从大卫城的台阶随地势而上,在大卫宫殿以上,直行到朝东的水门。
\par 38 第二队称谢的人要与那一队相迎而行。我和民的一半跟随他们,在城墙上过了炉楼,直到宽墙;
\par 39 又过了以法莲门、古门、鱼门、哈楠业楼、哈米亚楼,直到羊门,就在护卫门站住。
\par 40 於是,这两队称谢的人连我和官长的一半,站在神的殿里。
\par 41 还有祭司以利亚金、玛西雅、米拿民、米该雅、以利约乃、撒迦利亚、哈楠尼亚吹号;
\par 42 又有玛西雅、示玛雅、以利亚撒、乌西、约哈难、玛基雅、以拦,和以谢奏乐。歌唱的就大声歌唱,伊斯拉希雅管理他们。
\par 43 那日,众人献大祭而欢乐;因为神使他们大大欢乐,连妇女带孩童也都欢乐,甚至耶路撒冷中的欢声听到远处。
\par 44 当日,派人管理库房,将举祭、初熟之物和所取的十分之一,就是按各城田地,照律法所定归给祭司和利未人的分,都收在里头。犹大人因祭司和利未人供职,就欢乐了。
\par 45 祭司利未人遵守神所吩咐的,并守洁净的礼。歌唱的、守门的,照著大卫和他儿子所罗门的命令也如此行。
\par 46 古时,在大卫和亚萨的日子,有歌唱的伶长,并有赞美称谢神的诗歌。
\par 47 当所罗巴伯和尼希米的时候,以色列众人将歌唱的、守门的,每日所当得的分供给他们,又给利未人当得的分;利未人又给亚伦的子孙当得的分。

\chapter{13}

\par 1 当日,人念摩西的律法书给百姓听,遇见书上写著说,亚扪人和摩押人永不可入神的会;
\par 2 因为他们没有拿食物和水来迎接以色列人,且雇了巴兰咒诅他们,但我们的神使那咒诅变为祝福。
\par 3 以色列民听见这律法,就与一切闲杂人绝交
\par 4 先是蒙派管理我们神殿中库房的祭司以利亚实与多比雅结亲,
\par 5 便为他预备一间大屋子,就是从前收存素祭、乳香、器皿,和照命令供给利未人、歌唱的、守门的五谷、新酒,和油的十分之一,并归祭司举祭的屋子。
\par 6 那时我不在耶路撒冷;因为巴比伦王亚达薛西三十二年,我回到王那里。过了多日,我向王告假。
\par 7 我来到耶路撒冷,就知道以利亚实为多比雅在神殿的院内预备屋子的那件恶事。
\par 8 我甚恼怒,就把多比雅的一切家具从屋里都抛出去,
\par 9 吩咐人洁净这屋子,遂将神殿的器皿和素祭、乳香又搬进去。
\par 10 我见利未人所当得的分无人供给他们,甚至供职的利未人与歌唱的俱各奔回自己的田地去了。
\par 11 我就斥责官长说:「为何离弃神的殿呢?」我便招聚利未人,使他们照旧供职。
\par 12 犹大众人就把五谷、新酒,和油的十分之一送入库房。
\par 13 我派祭司示利米雅、文士撒督,和利未人毗大雅作库官管理库房;副官是哈难。哈难是撒刻的儿子;撒刻是玛他尼的儿子。这些人都是忠信的,他们的职分是将所供给的分给他们的弟兄。
\par 14 我的神啊,求你因这事记念我,不要涂抹我为神的殿与其中的礼节所行的善。
\par 15 那些日子,我在犹大见有人在安息日 酒(原文作踹酒 ),搬运禾捆驮在驴上,又把酒、葡萄、无花果,和各样的担子在安息日担入耶路撒冷,我就在他们卖食物的那日警戒他们。
\par 16 又有推罗人住在耶路撒冷;他们把鱼和各样货物运进来,在安息日卖给犹大人。
\par 17 我就斥责犹大的贵胄说:「你们怎麽行这恶事犯了安息日呢?
\par 18 从前你们列祖岂不是这样行,以致我们神使一切灾祸临到我们和这城吗?现在你们还犯安息日,使忿怒越发临到以色列!」
\par 19 在安息日的前一日,耶路撒冷城门有黑影的时候,我就吩咐人将门关锁,不过安息日不准开放。我又派我几个仆人管理城门,免得有人在安息日担什麽担子进城。
\par 20 於是商人和贩卖各样货物的,一两次住宿在耶路撒冷城外。
\par 21 我就警戒他们说:「你们为何在城外住宿呢?若再这样,我必下手拿办你们。」从此以後,他们在安息日不再来了。
\par 22 我吩咐利未人洁净自己,来守城门,使安息日为圣。我的神啊,求你因这事记念我,照你的大慈爱怜恤我。
\par 23 那些日子,我也见犹大人娶了亚实突、亚扪、摩押的女子为妻。
\par 24 他们的儿女说话,一半是亚实突的话,不会说犹大的话,所说的是照著各族的方言。
\par 25 我就斥责他们,咒诅他们,打了他们几个人,拔下他们的头发,叫他们指著神起誓,必不将自己的女儿嫁给外邦人的儿子,也不为自己和儿子娶他们的女儿。
\par 26 我又说:「以色列王所罗门不是在这样的事上犯罪吗?在多国中并没有一王像他,且蒙他神所爱,神立他作以色列全国的王;然而连他也被外邦女子引诱犯罪。
\par 27 如此,我岂听你们行这大恶,娶外邦女子干犯我们的神呢?」
\par 28 大祭司以利亚实的孙子、耶何耶大的一个儿子是和伦人参巴拉的女婿,我就从我这里把他赶出去。
\par 29 我的神啊,求你记念他们的罪;因为他们玷污了祭司的职任,违背你与祭司利未人所立的约。
\par 30 这样,我洁净他们,使他们离绝一切外邦人,派定祭司和利未人的班次,使他们各尽其职。
\par 31 我又派百姓按定期献柴和初熟的土产。我的神啊,求你记念我,施恩与我。



\end{document}