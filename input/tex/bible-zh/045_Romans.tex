\begin{document}

\title{罗马书}


\chapter{1}

\par 1 耶稣基督的仆人保罗,奉召为使徒,特派传神的福音。
\par 2 这福音是神从前藉众先知在圣经上所应许的,
\par 3 论到他儿子我主耶稣基督。按肉体说,是从大卫後裔生的;
\par 4 按圣善的灵说,因从死里复活,以大能显明是神的儿子。
\par 5 我们从他受了恩惠并使徒的职分,在万国之中叫人为他的名信服真道;
\par 6 其中也有你们这蒙召属耶稣基督的人。
\par 7 我写信给你们在罗马、为神所爱、奉召作圣徒的众人。愿恩惠、平安从我们的父神并主耶稣基督归与你们!
\par 8 第一,我靠著耶稣基督,为你们众人感谢我的神,因你们的信德传遍了天下。
\par 9 我在他儿子福音上,用心灵所事奉的神,可以见证我怎样不住的提到你们;
\par 10 在祷告之间常常恳求,或者照神的旨意,终能得平坦的道路往你们那里去。
\par 11 因为我切切的想见你们,要把些属灵的恩赐分给你们,使你们可以坚固。
\par 12 这样,我在你们中间,因你与我彼此的信心,就可以同得安慰。
\par 13 弟兄们,我不愿意你们不知道,我屡次定意往你们那里去,要在你们中间得些果子,如同在其余的外邦人中一样;只是到如今仍有阻隔。
\par 14 无论是希利尼人、化外人、聪明人、愚拙人,我都欠他们的债,
\par 15 所以情愿尽我的力量,将福音也传给你们在罗马的人。
\par 16 我不以福音为耻;这福音本是神的大能,要救一切相信的,先是犹太人,後是希利尼人。
\par 17 因为神的义正在这福音上显明出来;这义是本於信,以致於信。如经上所记:「义人必因信得生。」
\par 18 原来,神的忿怒从天上显明在一切不虔不义的人身上,就是那些行不义阻挡真理的人。
\par 19 神的事情,人所能知道的,原显明在人心里,因为神已经给他们显明。
\par 20 自从造天地以来,神的永能和神性是明明可知的,虽是眼不能见,但藉著所造之物就可以晓得,叫人无可推诿。
\par 21 因为,他们虽然知道神,却不当作神荣耀他,也不感谢他。他们的思念变为虚妄,无知的心就昏暗了。
\par 22 自称为聪明,反成了愚拙,
\par 23 将不能朽坏之神的荣耀变为偶像,彷佛必朽坏的人和飞禽、走兽、昆虫的样式。
\par 24 所以,神任凭他们逞著心里的情欲行污秽的事,以致彼此玷辱自己的身体。
\par 25 他们将神的真实变为虚谎,去敬拜事奉受造之物,不敬奉那造物的主;主乃是可称颂的,直到永远。阿们!
\par 26 因此,神任凭他们放纵可羞耻的情欲。他们的女人把顺性的用处变为逆性的用处;
\par 27 男人也是如此,弃了女人顺性的用处,欲火攻心,彼此贪恋,男和男行可羞耻的事,就在自己身上受这妄为当得的报应。
\par 28 他们既然故意不认识神,神就任凭他们存邪僻的心,行那些不合理的事;
\par 29 装满了各样不义、邪恶、贪婪、恶毒(或作:阴毒),满心是嫉妒、凶杀、争竞、诡诈、毒恨;
\par 30 又是谗毁的、背後说人的、怨恨神的(或作:被神所憎恶的)、侮慢人的、狂傲的、自夸的、捏造恶事的、违背父母的。
\par 31 无知的,背约的,无亲情的,不怜悯人的。
\par 32 他们虽知道神判定行这样事的人是当死的,然而他们不但自己去行,还喜欢别人去行。

\chapter{2}

\par 1 你这论断人的,无论你是谁,也无可推诿。你在什麽事上论断人,就在什麽事上定自己的罪;因你这论断人的,自己所行却和别人一样。
\par 2 我们知道这样行的人,神必照真理审判他。
\par 3 你这人哪,你论断行这样事的人,自己所行的却和别人一样,你以为能逃脱神的审判吗?
\par 4 还是你藐视他丰富的恩慈、宽容、忍耐,不晓得他的恩慈是领你悔改呢?
\par 5 你竟任著你刚硬不悔改的心,为自己积蓄忿怒,以致神震怒,显出他公义审判的日子来到。
\par 6 他必照各人的行为报应各人。
\par 7 凡恒心行善、寻求荣耀、尊贵和不能朽坏之福的,就以永生报应他们;
\par 8 惟有结党、不顺从真理、反顺从不义的,就以忿怒、恼恨报应他们。
\par 9 将患难、困苦加给一切作恶的人,先是犹太人,後是希利尼人;
\par 10 却将荣耀、尊贵、平安加给一切行善的人,先是犹太人,後是希利尼人。
\par 11 因为神不偏待人。
\par 12 凡没有律法犯了罪的,也必不按律法灭亡;凡在律法以下犯了罪的,也必按律法受审判。
\par 13 (原来在神面前,不是听律法的为义,乃是行律法的称义。
\par 14 没有律法的外邦人若顺著本性行律法上的事,他们虽然没有律法,自己就是自己的律法。
\par 15 这是显出律法的功用刻在他们心里,他们是非之心同作见证,并且他们的思念互相较量,或以为是,或以为非。)
\par 16 就在神藉耶稣基督审判人隐秘事的日子,照著我的福音所言。
\par 17 你称为犹太人,又倚靠律法,且指著神夸口;
\par 18 既从律法中受了教训,就晓得神的旨意,也能分别是非(或作:也喜爱那美好的事);
\par 19 又深信自己是给瞎子领路的,是黑暗中人的光,
\par 20 是蠢笨人的师傅,是小孩子的先生,在律法上有知识和真理的模范。
\par 21 你既是教导别人,还不教导自己吗?你讲说人不可偷窃,自己还偷窃吗?
\par 22 你说人不可奸淫,自己还奸淫吗?你厌恶偶像,自己还偷窃庙中之物吗?
\par 23 你指著律法夸口,自己倒犯律法,玷辱神麽?
\par 24 神的名在外邦人中,因你们受了亵渎,正如经上所记的。
\par 25 你若是行律法的,割礼固然於你有益;若是犯律法的,你的割礼就算不得割礼。
\par 26 所以那未受割礼的,若遵守律法的条例,他虽然未受割礼,岂不算是有割礼吗?
\par 27 而且那本来未受割礼的,若能全守律法,岂不是要审判你这有仪文和割礼竟犯律法的人吗?
\par 28 因为外面作犹太人的,不是真犹太人;外面肉身的割礼,也不是真割礼。
\par 29 惟有里面作的,才是真犹太人;真割礼也是心里的,在乎灵,不在乎仪文。这人的称赞不是从人来的,乃是从神来的。

\chapter{3}

\par 1 这样说来,犹太人有什麽长处?割礼有什麽益处呢?
\par 2 凡事大有好处:第一是神的圣言交托他们。
\par 3 即便有不信的,这有何妨呢?难道他们的不信就废掉神的信吗?
\par 4 断乎不能!不如说,神是真实的,人都是虚谎的。如经上所记:你责备人的时候,显为公义;被人议论的时候,可以得胜。
\par 5 我且照著人的常话说,我们的不义若显出神的义来,我们可以怎麽说呢?神降怒,是他不义吗?
\par 6 断乎不是!若是这样,神怎能审判世界呢?
\par 7 若神的真实,因我的虚谎越发显出他的荣耀,为什麽我还受审判,好像罪人呢?
\par 8 为什麽不说,我们可以作恶以成善呢?这是毁谤我们的人说我们有这话。这等人定罪是该当的。
\par 9 这却怎麽样呢?我们比他们强吗?决不是的!因我们已经证明:犹太人和希利尼人都在罪恶之下。
\par 10 就如经上所记:没有义人,连一个也没有。
\par 11 没有明白的;没有寻求神的;
\par 12 都是偏离正路,一同变为无用。没有行善的,连一个也没有。
\par 13 他们的喉咙是敞开的坟墓;他们用舌头弄诡诈,嘴唇里有虺蛇的毒气,
\par 14 满口是咒骂苦毒。
\par 15 杀人流血,他们的脚飞跑,
\par 16 所经过的路便行残害暴虐的事。
\par 17 平安的路,他们未曾知道;
\par 18 他们眼中不怕神。
\par 19 我们晓得律法上的话都是对律法以下之人说的,好塞住各人的口,叫普世的人都伏在神审判之下。
\par 20 所以凡有血气的,没有一个因行律法能在神面前称义,因为律法本是叫人知罪。
\par 21 但如今,神的义在律法以外已经显明出来,有律法和先知为证:
\par 22 就是神的义,因信耶稣基督加给一切相信的人,并没有分别。
\par 23 因为世人都犯了罪,亏缺了神的荣耀;
\par 24 如今却蒙神的恩典,因基督耶稣的救赎,就白白的称义。
\par 25 神设立耶稣作挽回祭,是凭著耶稣的血,藉著人的信,要显明神的义;因为他用忍耐的心宽容人先时所犯的罪,
\par 26 好在今时显明他的义,使人知道他自己为义,也称信耶稣的人为义。
\par 27 既是这样,那里能夸口呢?没有可夸的了。用何法没有的呢?是用立功之法吗?不是,乃用信主之法。
\par 28 所以(有古卷:因为)我们看定了:人称义是因著信,不在乎遵行律法。
\par 29 难道神只作犹太人的神吗?不也是作外邦人的神吗?是的,也作外邦人的神。
\par 30 神既是一位,他就要因信称那受割礼的为义,也要因信称那未受割礼的为义。
\par 31 这样,我们因信废了律法吗?断乎不是!更是坚固律法。

\chapter{4}

\par 1 如此说来,我们的祖宗亚伯拉罕凭著肉体得了什麽呢?
\par 2 倘若亚伯拉罕是因行为称义,就有可夸的;只是在神面前并无可夸。
\par 3 经上说什麽呢?说:「亚伯拉罕信神,这就算为他的义。」
\par 4 做工的得工价,不算恩典,乃是该得的;
\par 5 惟有不做工的,只信称罪人为义的神,他的信就算为义。
\par 6 正如大卫称那在行为以外蒙神算为义的人是有福的。
\par 7 他说:得赦免其过、遮盖其罪的,这人是有福的。
\par 8 主不算为有罪的,这人是有福的。
\par 9 如此看来,这福是单加给那受割礼的人吗?不也是加给那未受割礼的人吗?因我们所说,亚伯拉罕的信,就算为他的义,
\par 10 是怎麽算的呢?是在他受割礼的时候呢?是在他未受割礼的时候呢?不是在受割礼的时候,乃是在未受割礼的时候。
\par 11 并且他受了割礼的记号,作他未受割礼的时候因信称义的印证,叫他作一切未受割礼而信之人的父,使他们也算为义;
\par 12 又作受割礼之人的父,就是那些不但受割礼,并且按我们的祖宗亚伯拉罕未受割礼而信之踪迹去行的人。
\par 13 因为神应许亚伯拉罕和他後裔,必得承受世界,不是因律法,乃是因信而得的义。
\par 14 若是属乎律法的人才得为後嗣,信就归於虚空,应许也就废弃了。
\par 15 因为律法是惹动忿怒的(或作:叫人受刑的);那里没有律法,那里就没有过犯。
\par 16 所以人得为後嗣是本乎信,因此就属乎恩,叫应许定然归给一切後裔;不但归给那属乎律法的,也归给那效法亚伯拉罕之信的。
\par 17 亚伯拉罕所信的,是那叫死人复活、使无变为有的神,他在主面前作我们世人的父。如经上所记:「我已经立你作多国的父。」
\par 18 他在无可指望的时候,因信仍有指望,就得以作多国的父,正如先前所说,「你的後裔将要如此。」
\par 19 他将近百岁的时候,虽然想到自己的身体如同已死,撒拉的生育已经断绝,他的信心还是不软弱;
\par 20 并且仰望神的应许,总没有因不信心里起疑惑,反倒因信心里得坚固,将荣耀归给神,
\par 21 且满心相信神所应许的必能做成。
\par 22 所以,这就算为他的义。
\par 23 「算为他义」的这句话不是单为他写的,
\par 24 也是为我们将来得算为义之人写的,就是我们这信神使我们的主耶稣从死里复活的人。
\par 25 耶稣被交给人,是为我们的过犯;复活,是为叫我们称义(或作:耶稣是为我们的过犯交付了,是为我们称义复活了)。

\chapter{5}

\par 1 我们既因信称义,就藉著我们的主耶稣基督得与神相和。
\par 2 我们又藉著他,因信得进入现在所站的这恩典中,并且欢欢喜喜盼望神的荣耀。
\par 3 不但如此,就是在患难中也是欢欢喜喜的;因为知道患难生忍耐,
\par 4 忍耐生老练,老练生盼望;
\par 5 盼望不至於羞耻,因为所赐给我们的圣灵将神的爱浇灌在我们心里。
\par 6 因我们还软弱的时候,基督就按所定的日期为罪人死。
\par 7 为义人死,是少有的;为仁人死、或者有敢做的。
\par 8 惟有基督在我们还作罪人的时候为我们死,神的爱就在此向我们显明了。
\par 9 现在我们既靠著他的血称义,就更要藉著他免去神的忿怒。
\par 10 因为我们作仇敌的时候,且藉著神儿子的死,得与神和好;既已和好,就更要因他的生得救了。
\par 11 不但如此,我们既藉著我主耶稣基督得与神和好,也就藉著他以神为乐。
\par 12 这就如罪是从一人入了世界,死又是从罪来的;於是死就临到众人,因为众人都犯了罪。
\par 13 没有律法之先,罪已经在世上;但没有律法,罪也不算罪。
\par 14 然而从亚当到摩西,死就作了王,连那些不与亚当犯一样罪过的,也在他的权下。亚当乃是那以後要来之人的预像。
\par 15 只是过犯不如恩赐,若因一人的过犯,众人都死了,何况神的恩典,与那因耶稣基督一人恩典中的赏赐,岂不更加倍的临到众人吗?
\par 16 因一人犯罪就定罪,也不如恩赐,原来审判是由一人而定罪,恩赐乃是由许多过犯而称义。
\par 17 若因一人的过犯,死就因这一人作了王,何况那些受洪恩又蒙所赐之义的,岂不更要因耶稣基督一人在生命中作王吗?
\par 18 如此说来,因一次的过犯,众人都被定罪;照样,因一次的义行,众人也就被称义得生命了。
\par 19 因一人的悖逆,众人成为罪人;照样,因一人的顺从,众人也成为义了。
\par 20 律法本是外添的,叫过犯显多;只是罪在那里显多,恩典就更显多了。
\par 21 就如罪作王叫人死;照样,恩典也藉著义作王,叫人因我们的主耶稣基督得永生。

\chapter{6}

\par 1 这样,怎麽说呢?我们可以仍在罪中、叫恩典显多吗?
\par 2 断乎不可!我们在罪上死了的人岂可仍在罪中活著呢?
\par 3 岂不知我们这受洗归入基督耶稣的人是受洗归入他的死吗?
\par 4 所以,我们藉著洗礼归入死,和他一同埋葬,原是叫我们一举一动有新生的样式,像基督藉著父的荣耀从死里复活一样。
\par 5 我们若在他死的形状上与他联合,也要在他复活的形状上与他联合;
\par 6 因为知道我们的旧人和他同钉十字架,使罪身灭绝,叫我们不再作罪的奴仆;
\par 7 因为已死的人是脱离了罪。
\par 8 我们若是与基督同死,就信必与他同活。
\par 9 因为知道基督既从死里复活,就不再死,死也不再作他的主了。
\par 10 他死是向罪死了,只有一次;他活是向神活著。
\par 11 这样,你们向罪也当看自己是死的;向神在基督耶稣里,却当看自己是活的。
\par 12 所以,不要容罪在你们必死的身上作王,使你们顺从身子的私欲。
\par 13 也不要将你们的肢体献给罪作不义的器具;倒要像从死里复活的人,将自己献给神,并将肢体作义的器具献给神。
\par 14 罪必不能作你们的主,因你们不在律法之下,乃在恩典之下。
\par 15 这却怎麽样呢?我们在恩典之下,不在律法之下,就可以犯罪吗?断乎不可!
\par 16 岂不晓得你们献上自己作奴仆,顺从谁,就作谁的奴仆吗?或作罪的奴仆,以至於死;或作顺命的奴仆,以至成义。
\par 17 感谢神!因为你们从前虽然作罪的奴仆,现今却从心里顺服了所传给你们道理的模范。
\par 18 你们既从罪里得了释放,就作了义的奴仆。
\par 19 我因你们肉体的软弱,就照人的常话对你们说。你们从前怎样将肢体献给不洁不法作奴仆,以至於不法;现今也要照样将肢体献给义作奴仆,以至於成圣。
\par 20 因为你们作罪之奴仆的时候,就不被义约束了。
\par 21 你们现今所看为羞耻的事,当日有什麽果子呢?那些事的结局就是死。
\par 22 但现今,你们既从罪里得了释放,作了神的奴仆,就有成圣的果子,那结局就是永生。
\par 23 因为罪的工价乃是死;惟有神的恩赐,在我们的主基督耶稣里,乃是永生。

\chapter{7}

\par 1 弟兄们,我现在对明白律法的人说,你们岂不晓得律法管人是在活著的时候吗?
\par 2 就如女人有了丈夫,丈夫还活著,就被律法约束;丈夫若死了,就脱离了丈夫的律法。
\par 3 所以丈夫活著,他若归於别人,便叫淫妇;丈夫若死了,他就脱离了丈夫的律法,虽然归於别人,也不是淫妇。
\par 4 我的弟兄们,这样说来,你们藉著基督的身体,在律法上也是死了,叫你们归於别人,就是归於那从死里复活的,叫我们结果子给神。
\par 5 因为我们属肉体的时候,那因律法而生的恶欲就在我们肢体中发动,以致结成死亡的果子。
\par 6 但我们既然在捆我们的律法上死了,现今就脱离了律法,叫我们服事主,要按著心灵(心灵:或作圣灵)的新样,不按著仪文的旧样。
\par 7 这样,我们可说什麽呢?律法是罪吗?断乎不是!只是非因律法,我就不知何为罪。非律法说「不可起贪心」,我就不知何为贪心。
\par 8 然而罪趁著机会,就藉著诫命叫诸般的贪心在我里头发动;因为没有律法,罪是死的。
\par 9 我以前没有律法是活著的;但是诫命来到,罪又活了,我就死了。
\par 10 那本来叫人活的诫命,反倒叫我死;
\par 11 因为罪趁著机会,就藉著诫命引诱我,并且杀了我。
\par 12 这样看来,律法是圣洁的,诫命也是圣洁、公义、良善的。
\par 13 既然如此,那良善的是叫我死吗?断乎不是!叫我死的乃是罪。但罪藉著那良善的叫我死,就显出真是罪,叫罪因著诫命更显出是恶极了。
\par 14 我们原晓得律法是属乎灵的,但我是属乎肉体的,是已经卖给罪了。
\par 15 因为我所做的,我自己不明白;我所愿意的,我并不做;我所恨恶的,我倒去做。
\par 16 若我所做的,是我所不愿意的,我就应承律法是善的。
\par 17 既是这样,就不是我做的,乃是住在我里头的罪做的。
\par 18 我也知道在我里头,就是我肉体之中,没有良善。因为,立志为善由得我,只是行出来由不得我。
\par 19 故此,我所愿意的善,我反不做;我所不愿意的恶,我倒去做。
\par 20 若我去做所不愿意做的,就不是我做的,乃是住在我里头的罪做的。
\par 21 我觉得有个律,就是我愿意为善的时候,便有恶与我同在。
\par 22 因为按著我里面的意思(原文作人),我是喜欢神的律;
\par 23 但我觉得肢体中另有个律和我心中的律交战,把我掳去,叫我附从那肢体中犯罪的律。
\par 24 我真是苦啊!谁能救我脱离这取死的身体呢?
\par 25 感谢神,靠著我们的主耶稣基督就能脱离了。这样看来,我以内心顺服神的律,我肉体却顺服罪的律了。

\chapter{8}

\par 1 如今,那些在基督耶稣里的就不定罪了。
\par 2 因为赐生命圣灵的律,在基督耶稣里释放了我,使我脱离罪和死的律了。
\par 3 律法既因肉体软弱,有所不能行的,神就差遣自己的儿子,成为罪身的形状,作了赎罪祭,在肉体中定了罪案,
\par 4 使律法的义成就在我们这不随从肉体、只随从圣灵的人身上。
\par 5 因为随从肉体的人体贴肉体的事,随从圣灵的人体贴圣灵的事。
\par 6 体贴肉体的,就是死;体贴圣灵的,乃是生命、平安。
\par 7 原来体贴肉体的,就是与神为仇;因为不服神的律法,也是不能服,
\par 8 而且属肉体的人不能得神的喜欢。
\par 9 如果神的灵住在你们心里,你们就不属肉体,乃属圣灵了。人若没有基督的灵,就不是属基督的。
\par 10 基督若在你们心里,身体就因罪而死,心灵却因义而活。
\par 11 然而,叫耶稣从死里复活者的灵若住在你们心里,那叫基督耶稣从死里复活的,也必藉著住在你们心里的圣灵,使你们必死的身体又活过来。
\par 12 弟兄们,这样看来,我们并不是欠肉体的债去顺从肉体活著。
\par 13 你们若顺从肉体活著,必要死;若靠著圣灵治死身体的恶行,必要活著。
\par 14 因为凡被神的灵引导的,都是神的儿子。
\par 15 你们所受的,不是奴仆的心,仍旧害怕;所受的,乃是儿子的心,因此我们呼叫:「阿爸!父!」
\par 16 圣灵与我们的心同证我们是神的儿女;
\par 17 既是儿女,便是後嗣,就是神的後嗣,和基督同作後嗣。如果我们和他一同受苦,也必和他一同得荣耀。
\par 18 我想,现在的苦楚若比起将来要显於我们的荣耀就不足介意了。
\par 19 受造之物切望等候神的众子显出来。
\par 20 因为受造之物服在虚空之下,不是自己愿意,乃是因那叫他如此的。
\par 21 但受造之物仍然指望脱离败坏的辖制,得享(享:原文作入)神儿女自由的荣耀。
\par 22 我们知道一切受造之物一同叹息、劳苦,直到如今。
\par 23 不但如此,就是我们这有圣灵初结果子的,也是自己心里叹息,等候得著儿子的名分,乃是我们的身体得赎。
\par 24 我们得救是在乎盼望;只是所见的盼望不是盼望,谁还盼望他所见的呢(有古卷:人所著见的何必再盼望呢)?
\par 25 但我们若盼望那所不见的,就必忍耐等候。
\par 26 况且我们的软弱有圣灵帮助,我们本不晓得当怎样祷告,只是圣灵亲自用说不出来的叹息替我们祷告。
\par 27 鉴察人心的,晓得圣灵的意思,因为圣灵照著神的旨意替圣徒祈求。
\par 28 我们晓得万事都互相效力,叫爱神的人得益处,就是按他旨意被召的人。
\par 29 因为他预先所知道的人,就预先定下效法他儿子的模样,使他儿子在许多弟兄中作长子。
\par 30 预先所定下的人又召他们来;所召来的人又称他们为义;所称为义的人又叫他们得荣耀。
\par 31 既是这样,还有什麽说的呢?神若帮助我们,谁能敌挡我们呢?
\par 32 神既不爱惜自己的儿子,为我们众人舍了,岂不也把万物和他一同白白的赐给我们吗?
\par 33 谁能控告神所拣选的人呢?有神称他们为义了(或作:是称他们为义的神吗)。
\par 34 谁能定他们的罪呢?有基督耶稣已经死了,而且从死里复活,现今在神的右边,也替我们祈求(有基督云云或作是已经死了,而且从死里复活,现今在神的右边,也替我们祈求的基督耶稣吗)
\par 35 谁能使我们与基督的爱隔绝呢?难道是患难吗?是困苦吗?是逼迫吗?是饥饿吗?是赤身露体吗?是危险吗?是刀剑吗?
\par 36 如经上所记:我们为你的缘故终日被杀;人看我们如将宰的羊。
\par 37 然而,靠著爱我们的主,在这一切的事上已经得胜有余了。
\par 38 因为我深信无论是死,是生,是天使,是掌权的,是有能的,是现在的事,是将来的事,
\par 39 是高处的,是低处的,是别的受造之物,都不能叫我们与神的爱隔绝;这爱是在我们的主基督耶稣里的。

\chapter{9}

\par 1 我在基督里说真话,并不谎言,有我良心被圣灵感动,给我作见证;
\par 2 我是大有忧愁,心里时常伤痛;
\par 3 为我弟兄,我骨肉之亲,就是自己被咒诅,与基督分离,我也愿意。
\par 4 他们是以色列人;那儿子的名分、荣耀、诸约、律法、礼仪、应许都是他们的。
\par 5 列祖就是他们的祖宗,按肉体说,基督也是从他们出来的,他是在万有之上,永远可称颂的神。阿们!
\par 6 这不是说神的话落了空。因为从以色列生的不都是以色列人,
\par 7 也不因为是亚伯拉罕的後裔就都作他的儿女;惟独「从以撒生的才要称为你的後裔。」
\par 8 这就是说,肉身所生的儿女不是神的儿女,惟独那应许的儿女才算是後裔。
\par 9 因为所应许的话是这样说:「到明年这时候我要来,撒拉必生一个儿子。」
\par 10 不但如此,还有利百加,既从一个人,就是从我们的祖宗以撒怀了孕,
\par 11 (双子还没有生下来,善恶还没有做出来,只因要显明神拣选人的旨意,不在乎人的行为,乃在乎召人的主。)
\par 12 神就对利百加说:「将来大的要服事小的。」
\par 13 正如经上所记:雅各是我所爱的;以扫是我所恶的。
\par 14 这样,我们可说什麽呢?难道神有什麽不公平吗?断乎没有!
\par 15 因他对摩西说:我要怜悯谁就怜悯谁,要恩待谁就恩待谁。
\par 16 据此看来,这不在乎那定意的,也不在乎那奔跑的,只在乎发怜悯的神。
\par 17 因为经上有话向法老说:「我将你兴起来,特要在你身上彰显我的权能,并要使我的名传遍天下。」
\par 18 如此看来,神要怜悯谁就怜悯谁,要叫谁刚硬就叫谁刚硬。
\par 19 这样,你必对我说:「他为什麽还指责人呢?有谁抗拒他的旨意呢?」
\par 20 你这个人哪,你是谁,竟敢向神强嘴呢?受造之物岂能对造他的说:「你为什麽这样造我呢?
\par 21 窑匠难道没有权柄从一团泥里拿一块作成贵重的器皿,又拿一块作成卑贱的器皿吗?
\par 22 倘若神要显明他的忿怒,彰显他的权能,就多多忍耐宽容那可怒预备遭毁灭的器皿,
\par 23 又要将他丰盛的荣耀彰显在那蒙怜悯早预备得荣耀的器皿上。
\par 24 这器皿就是我们被神所召的,不但是从犹太人中,也是从外邦人中。这有什麽不可呢?
\par 25 就像神在何西阿书上说:那本来不是我子民的,我要称为「我的子民」;本来不是蒙爱的,我要称为「蒙爱的」。
\par 26 从前在什麽地方对他们说:你们不是我的子民,将来就在那里称他们为「永生神的儿子」。
\par 27 以赛亚指著以色列人喊著说:「以色列人虽多如海沙,得救的不过是剩下的余数;
\par 28 因为主要在世上施行他的话,叫他的话都成全,速速的完结。」
\par 29 又如以赛亚先前说过:若不是万军之主给我们存留余种,我们早已像所多玛,蛾摩拉的样子了。
\par 30 这样,我们可说什麽呢?那本来不追求义的外邦人反得了义,就是因信而得的义。
\par 31 但以色列人追求律法的义,反得不著律法的义。
\par 32 这是什麽缘故呢?是因为他们不凭著信心求,只凭著行为求,他们正跌在那绊脚石上。
\par 33 就如经上所记:我在锡安放一块绊脚的石头,跌人的磐石;信靠他的人必不至於羞愧。

\chapter{10}

\par 1 弟兄们,我心里所愿的,向神所求的,是要以色列人得救。
\par 2 我可以证明他们向神有热心,但不是按著真知识;
\par 3 因为不知道神的义,想要立自己的义,就不服神的义了。
\par 4 律法的总结就是基督,使凡信他的都得著义。
\par 5 摩西写著说:「人若行那出於律法的义,就必因此活著。」
\par 6 惟有出於信心的义如此说:「你不要心里说:谁要升到天上去呢?(就是要领下基督来;)
\par 7 谁要下到阴间去呢?(就是要领基督从死里上来。)」
\par 8 他到底怎麽说呢?他说:这道离你不远,正在你口里,在你心里。(就是我们所传信主的道。)
\par 9 你若口里认耶稣为主,心里信神叫他从死里复活,就必得救。
\par 10 因为人心里相信,就可以称义;口里承认,就可以得救。
\par 11 经上说:「凡信他的人必不至於羞愧。」
\par 12 犹太人和希利尼人并没有分别,因为众人同有一位主;他也厚待一切求告他的人。
\par 13 因为「凡求告主名的,就必得救。」
\par 14 然而,人未曾信他,怎能求他呢?未曾听见他,怎能信他呢?没有传道的,怎能听见呢?
\par 15 若没有奉差遣,怎能传道呢?如经上所记:「报福音、传喜信的人,他们的脚踪何等佳美。」
\par 16 只是人没有都听从福音,因为以赛亚说:「主啊,我们所传的有谁信呢?」
\par 17 可见信道是从听道来的,听道是从基督的话来的。
\par 18 但我说,人没有听见吗?诚然听见了。他们的声音传遍天下;他们的言语传到地极。
\par 19 我再说,以色列人不知道吗?先有摩西说:我要用那不成子民的,惹动你们的愤恨;我要用那无知的民触动你们的怒气。
\par 20 又有以赛亚放胆说:没有寻找我的,我叫他们遇见;没有访问我的,我向他们显现。
\par 21 至於以色列人,他说:「我整天伸手招呼那悖逆顶嘴的百姓。」

\chapter{11}

\par 1 我且说,神弃绝了他的百姓吗?断乎没有!因为我也是以色列人,亚伯拉罕的後裔,属便雅悯支派的。
\par 2 神并没有弃绝他预先所知道的百姓。你们岂不晓得经上论到以利亚是怎麽说的呢?他在神面前怎样控告以色列人说:
\par 3 「主啊,他们杀了你的先知,拆了你的祭坛,只剩下我一个人,他们还要寻索我的命。」
\par 4 神的回话是怎麽说的呢?他说:「我为自己留下七千人,是未曾向巴力屈膝的。」
\par 5 如今也是这样,照著拣选的恩典,还有所留的余数。
\par 6 既是出於恩典,就不在乎行为;不然,恩典就不是恩典了。
\par 7 这是怎麽样呢?以色列人所求的,他们没有得著,惟有蒙拣选的人得著了;其余的就成了顽梗不化的。
\par 8 如经上所记:神给他们昏迷的心,眼睛不能看见,耳朵不能听见,直到今日。
\par 9 大卫也说:愿他们的筵席变为网罗,变为机槛,变为绊脚石,作他们的报应。
\par 10 愿他们的眼睛昏蒙,不得看见;愿你时常弯下他们的腰。
\par 11 我且说,他们失脚是要他们跌倒吗?断乎不是!反倒因他们的过失,救恩便临到外邦人,要激动他们发愤。
\par 12 若他们的过失,为天下的富足,他们的缺乏,为外邦人的富足;何况他们的丰满呢?
\par 13 我对你们外邦人说这话;因我是外邦人的使徒,所以敬重(原文作荣耀)我的职分,
\par 14 或者可以激动我骨肉之亲发愤,好救他们一些人。
\par 15 若他们被丢弃,天下就得与神和好;他们被收纳,岂不是死而复生吗?
\par 16 所献的新面若是圣洁,全团也就圣洁了;树根若是圣洁,树枝也就圣洁了。
\par 17 若有几根枝子被折下来,你这野橄榄得接在其中,一同得著橄榄根的肥汁,
\par 18 你就不可向旧枝子夸口;若是夸口,当知道不是你托著根,乃是根托著你。
\par 19 你若说,那枝子被折下来是特为叫我接上。
\par 20 不错!他们因为不信,所以被折下来;你因为信,所以立得住;你不可自高,反要惧怕。
\par 21 神既不爱惜原来的枝子,也必不爱惜你。
\par 22 可见神的恩慈和严厉,向那跌倒的人是严厉的,向你是有恩慈的;只要你长久在他的恩慈里,不然,你也要被砍下来。
\par 23 而且他们若不是长久不信,仍要被接上,因为神能够把他们从新接上。
\par 24 你是从那天生的野橄榄上砍下来的,尚且逆著性得接在好橄榄上,何况这本树的枝子,要接在本树上呢!
\par 25 弟兄们,我不愿意你们不知道这奥秘(恐怕你们自以为聪明),就是以色列人有几分是硬心的,等到外邦人的数目添满了,
\par 26 於是以色列全家都要得救。如经上所记:必有一位救主从锡安出来,要消除雅各家的一切罪恶;
\par 27 又说:我除去他们罪的时候,这就是我与他们所立的约。
\par 28 就著福音说,他们为你们的缘故是仇敌;就著拣选说,他们为列祖的缘故是蒙爱的。
\par 29 因为神的恩赐和选召是没有後悔的。
\par 30 你们从前不顺服神,如今因他们的不顺服,你们倒蒙了怜恤。
\par 31 这样,他们也是不顺服,叫他们因著施给你们的怜恤,现在也就蒙怜恤。
\par 32 因为神将众人都圈在不顺服之中,特意要怜恤众人。
\par 33 深哉,神丰富的智慧和知识!他的判断何其难测!他的踪迹何其难寻!
\par 34 谁知道主的心?谁作过他的谋士呢?
\par 35 谁是先给了他,使他後来偿还呢?
\par 36 因为万有都是本於他,倚靠他,归於他。愿荣耀归给他,直到永远。阿们!

\chapter{12}

\par 1 所以弟兄们,我以神的慈悲劝你们,将身体献上,当作活祭,是圣洁的,是神所喜悦的;你们如此事奉乃是理所当然的。
\par 2 不要效法这个世界,只要心意更新而变化,叫你们察验何为神的善良、纯全、可喜悦的旨意。
\par 3 我凭著所赐我的恩对你们各人说:不要看自己过於所当看的,要照著神所分给各人信心的大小,看得合乎中道。
\par 4 正如我们一个身子上有好些肢体,肢体也不都是一样的用处。
\par 5 我们这许多人,在基督里成为一身,互相联络作肢体,也是如此。
\par 6 按我们所得的恩赐,各有不同。或说预言,就当照著信心的程度说预言,
\par 7 或作执事,就当专一执事;或作教导的,就当专一教导;
\par 8 或作劝化的,就当专一劝化;施舍的,就当诚实;治理的,就当殷勤;怜悯人的,就当甘心。
\par 9 爱人不可虚假;恶要厌恶,善要亲近。
\par 10 爱弟兄,要彼此亲热;恭敬人,要彼此推让。
\par 11 殷勤不可懒惰。要心里火热,常常服事主。
\par 12 在指望中要喜乐,在患难中要忍耐,祷告要恒切。
\par 13 圣徒缺乏要帮补;客要一味的款待。
\par 14 逼迫你们的,要给他们祝福;只要祝福,不可咒诅。
\par 15 与喜乐的人要同乐;与哀哭的人要同哭。
\par 16 要彼此同心;不要志气高大,倒要俯就卑微的人(人:或作事);不要自以为聪明。
\par 17 不要以恶报恶;众人以为美的事要留心去做。
\par 18 若是能行,总要尽力与众人和睦。
\par 19 亲爱的弟兄,不要自己伸冤,宁可让步,听凭主怒(或作:让人发怒);因为经上记著:「主说:『伸冤在我;我必报应。』」
\par 20 所以,「你的仇敌若饿了,就给他吃,若渴了,就给他喝;因为你这样行就是把炭火堆在他的头上。」
\par 21 你不可为恶所胜,反要以善胜恶。

\chapter{13}

\par 1 在上有权柄的,人人当顺服他,因为没有权柄不是出於神的。凡掌权的都是神所命的。
\par 2 所以,抗拒掌权的就是抗拒神的命;抗拒的必自取刑罚。
\par 3 作官的原不是叫行善的惧怕,乃是叫作恶的惧怕。你愿意不惧怕掌权的吗?你只要行善,就可得他的称赞;
\par 4 因为他是神的用人,是与你有益的。你若作恶,却当惧怕;因为他不是空空的佩剑,他是神的用人,是伸冤的,刑罚那作恶的。
\par 5 所以你们必须顺服,不但是因为刑罚,也是因为良心。
\par 6 你们纳粮,也为这个缘故;因他们是神的差役,常常特管这事。
\par 7 凡人所当得的,就给他。当得粮的,给他纳粮;当得税的,给他上税;当惧怕的,惧怕他;当恭敬的,恭敬他。
\par 8 凡事都不可亏欠人,惟有彼此相爱要常以为亏欠;因为爱人的,就完全了律法。
\par 9 像那不可奸淫,不可杀人,不可偷盗,不可贪婪,或有别的诫命,都包在爱人如己这一句话之内了。
\par 10 爱是不加害与人的,所以爱就完全了律法。
\par 11 再者,你们晓得现今就是该趁早睡醒的时候;因为我们得救,现今比初信的时候更近了。
\par 12 黑夜已深,白昼将近;我们就当脱去暗昧的行为,带上光明的兵器。
\par 13 行事为人要端正,好像行在白昼。不可荒宴醉酒,不可好色邪荡,不可争竞嫉妒;
\par 14 总要披戴主耶稣基督,不要为肉体安排,去放纵私欲。

\chapter{14}

\par 1 信心软弱的,你们要接纳,但不要辩论所疑惑的事。
\par 2 有人信百物都可吃;但那软弱的,只吃蔬菜。
\par 3 吃的人不可轻看不吃的人;不吃的人不可论断吃的人;因为神已经收纳他了。
\par 4 你是谁,竟论断别人的仆人呢?他或站住,或跌倒,自有他的主人在;而且他也必要站住,因为主能使他站住。
\par 5 有人看这日比那日强;有人看日日都是一样。只是各人心里要意见坚定。
\par 6 守日的人是为主守的;吃的人是为主吃的,因他感谢神;不吃的人是为主不吃的,也感谢神。
\par 7 我们没有一个人为自己活,也没有一个人为自己死。
\par 8 我们若活著,是为主而活;若死了,是为主而死。所以,我们或活或死总是主的人。
\par 9 因此基督死了又活了,为要作死人并活人的主。
\par 10 你这个人,为什麽论断弟兄呢?又为什麽轻看弟兄呢?因我们都要站在神的台前。
\par 11 经上写著:主说:我凭著我的永生起誓:万膝必向我跪拜;万口必向我承认。
\par 12 这样看来,我们各人必要将自己的事在神面前说明。
\par 13 所以,我们不可再彼此论断,宁可定意谁也不给弟兄放下绊脚跌人之物。
\par 14 我凭著主耶稣确知深信,凡物本来没有不洁净的;惟独人以为不洁净的,在他就不洁净了。
\par 15 你若因食物叫弟兄忧愁,就不是按著爱人的道理行。基督已经替他死,你不可因你的食物叫他败坏。
\par 16 不可叫你的善被人毁谤;
\par 17 因为神的国不在乎吃喝,只在乎公义、和平,并圣灵中的喜乐。
\par 18 在这几样上服事基督的,就为神所喜悦,又为人所称许。
\par 19 所以,我们务要追求和睦的事与彼此建立德行的事。
\par 20 不可因食物毁坏神的工程。凡物固然洁净,但有人因食物叫人跌倒,就是他的罪了。
\par 21 无论是吃肉,是喝酒,是什麽别的事,叫弟兄跌倒,一概不做才好。
\par 22 你有信心,就当在神面前守著。人在自己以为可行的事上能不自责,就有福了。
\par 23 若有疑心而吃的,就必有罪,因为他吃不是出於信心。凡不出於信心的都是罪。

\chapter{15}

\par 1 我们坚固的人应该担代不坚固人的软弱,不求自己的喜悦。
\par 2 我们各人务要叫邻舍喜悦,使他得益处,建立德行。
\par 3 因为基督也不求自己的喜悦,如经上所记:「辱骂你人的辱骂都落在我身上。」
\par 4 从前所写的圣经都是为教训我们写的,叫我们因圣经所生的忍耐和安慰可以得著盼望。
\par 5 但愿赐忍耐安慰的神叫你们彼此同心,效法基督耶稣,
\par 6 一心一口荣耀神我们主耶稣基督的父!
\par 7 所以,你们要彼此接纳,如同基督接纳你们一样,使荣耀归与神。
\par 8 我说,基督是为神真理作了受割礼人的执事,要证实所应许列祖的话,
\par 9 并叫外邦人因他的怜悯荣耀神。如经上所记:因此,我要在外邦中称赞你,歌颂你的名;
\par 10 又说:你们外邦人当与主的百姓一同欢乐;
\par 11 又说:外邦啊,你们当赞美主!万民哪,你们都当颂赞他!
\par 12 又有以赛亚说:将来有耶西的根,就是那兴起来要治理外邦的;外邦人要仰望他。
\par 13 但愿使人有盼望的神,因信将诸般的喜乐、平安充满你们的心,使你们藉著圣灵的能力大有盼望。
\par 14 弟兄们,我自己也深信你们是满有良善,充足了诸般的知识,也能彼此劝戒。
\par 15 但我稍微放胆写信给你们,是要提醒你们的记性,特因神所给我的恩典,
\par 16 使我为外邦人作基督耶稣的仆役,作神福音的祭司,叫所献上的外邦人,因著圣灵成为圣洁,可蒙悦纳。
\par 17 所以论到神的事,我在基督耶稣里有可夸的。
\par 18 除了基督藉我做的那些事,我什麽都不敢提,只提他藉我言语作为,用神迹奇事的能力,并圣灵的能力,使外邦人顺服;
\par 19 甚至我从耶路撒冷,直转到以利哩古,到处传了基督的福音。
\par 20 我立了志向,不在基督的名被称过的地方传福音,免得建造在别人的根基上。
\par 21 就如经上所记:未曾闻知他信息的,将要看见;未曾听过的,将要明白。
\par 22 我因多次被拦阻,总不得到你们那里去。
\par 23 但如今,在这里再没有可传的地方,而且这好几年,我切心想望到士班雅去的时候,可以到你们那里,
\par 24 盼望从你们那里经过,得见你们,先与你们彼此交往,心里稍微满足,然後蒙你们送行。
\par 25 但现在,我往耶路撒冷去供给圣徒。
\par 26 因为马其顿和亚该亚人乐意凑出捐项给耶路撒冷圣徒中的穷人。
\par 27 这固然是他们乐意的,其实也算是所欠的债;因外邦人既然在他们属灵的好处上有分,就当把养身之物供给他们。
\par 28 等我办完了这事,把这善果向他们交付明白,我就要路过你们那里,往士班雅去。
\par 29 我也晓得去的时候,必带著基督丰盛的恩典而去。
\par 30 弟兄们,我藉著我们主耶稣基督,又藉著圣灵的爱,劝你们与我一同竭力,为我祈求神,
\par 31 叫我脱离在犹太不顺从的人,也叫我为耶路撒冷所办的捐项可蒙圣徒悦纳,
\par 32 并叫我顺著神的旨意,欢欢喜喜的到你们那里,与你们同得安息。
\par 33 愿赐平安的神常和你们众人同在。阿们!

\chapter{16}

\par 1 我对你们举荐我们的姊妹非比,他是坚革哩教会中的女执事。
\par 2 请你们为主接待他,合乎圣徒的体统。他在何事上要你们帮助,你们就帮助他,因他素来帮助许多人,也帮助了我。
\par 3 问百基拉和亚居拉安。他们在基督耶稣里与我同工,
\par 4 也为我的命将自己的颈项置之度外。不但我感谢他们,就是外邦的众教会也感谢他们。
\par 5 又问在他们家中的教会安。问我所亲爱的以拜尼土安;他在亚西亚是归基督初结的果子。
\par 6 又问马利亚安;他为你们多受劳苦。
\par 7 又问我亲属与我一同坐监的安多尼古和犹尼亚安;他们在使徒中是有名望的,也是比我先在基督里。
\par 8 又问我在主里面所亲爱的暗伯利安。
\par 9 又问在基督里与我们同工的耳巴奴,并我所亲爱的士大古安。
\par 10 又问在基督里经过试验的亚比利安。问亚利多布家里的人安。
\par 11 又问我亲属希罗天安。问拿其数家在主里的人安。
\par 12 又问为主劳苦的土非拿氏和土富撒氏安。问可亲爱为主多受劳苦的彼息氏安。
\par 13 又问在主蒙拣选的鲁孚和他母亲安;他的母亲就是我的母亲。
\par 14 又问亚逊其土、弗勒干、黑米、八罗巴、黑马,并与他们在一处的弟兄们安。
\par 15 又问非罗罗古和犹利亚,尼利亚和他姊妹,同阿林巴并与他们在一处的众圣徒安。
\par 16 你们亲嘴问安,彼此务要圣洁。基督的众教会都问你们安。
\par 17 弟兄们,那些离间你们、叫你们跌倒、背乎所学之道的人,我劝你们要留意躲避他们。
\par 18 因为这样的人不服事我们的主基督,只服事自己的肚腹,用花言巧语诱惑那些老实人的心。
\par 19 你们的顺服已经传於众人,所以我为你们欢喜;但我愿意你们在善上聪明,在恶上愚拙。
\par 20 赐平安的神快要将撒但践踏在你们脚下。愿我主耶稣基督的恩常和你们同在!
\par 21 与我同工的提摩太,和我的亲属路求、耶孙、所西巴德,问你们安。
\par 22 我这代笔写信的德丢,在主里面问你们安。
\par 23 那接待我、也接待全教会的该犹问你们安。
\par 24 城内管银库的以拉都,和兄弟括土问你们安。
\par 25 惟有神能照我所传的福音和所讲的耶稣基督,并照永古隐藏不言的奥秘,坚固你们的心。
\par 26 这奥秘如今显明出来,而且按著永生神的命,藉众先知的书指示万国的民,使他们信服真道。
\par 27 愿荣耀因耶稣基督归与独一全智的神,直到永远。阿们!


\end{document}