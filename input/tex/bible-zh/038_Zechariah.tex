\begin{document}

\title{撒迦利亚书}


\chapter{1}

\par 1 大利乌王第二年八月,耶和华的话临到易多的孙子、比利家的儿子先知撒迦利亚,说:
\par 2 「耶和华曾向你们列祖大大发怒。
\par 3 所以你要对以色列人说,万军之耶和华如此说:你们要转向我,我就转向你们。这是万军之耶和华说的。
\par 4 不要效法你们列祖。从前的先知呼叫他们说,万军之耶和华如此说:『你们要回头离开你们的恶道恶行。』他们却不听,也不顺从我。这是耶和华说的。
\par 5 你们的列祖在那里呢?那些先知能永远存活吗?
\par 6 只是我的言语和律例,就是所吩咐我仆人众先知的,岂不临到你们列祖吗?他们就回头,说:『万军之耶和华定意按我们的行动作为向我们怎样行,他已照样行了。』」
\par 7 大利乌第二年十一月,就是细罢特月二十四日,耶和华的话临到易多的孙子、比利家的儿子先知撒迦利亚,说:
\par 8 「我夜间观看,见一人骑著红马,站在洼地番石榴树中间。在他身後又有红马、黄马,和白马。」
\par 9 我对与我说话的天使说:「主啊,这是什麽意思?」他说:「我要指示你这是什麽意思。」
\par 10 那站在番石榴树中间的人说:「这是奉耶和华差遣在遍地走来走去的。」
\par 11 那些骑马的对站在番石榴树中间耶和华的使者说:「我们已在遍地走来走去,见全地都安息平静。」
\par 12 於是,耶和华的使者说:「万军之耶和华啊,你恼恨耶路撒冷和犹大的城邑已经七十年,你不施怜悯要到几时呢?」
\par 13 耶和华就用美善的安慰话回答那与我说话的天使。
\par 14 与我说话的天使对我说:「你要宣告说,万军之耶和华如此说:我为耶路撒冷为锡安,心里极其火热。
\par 15 我甚恼怒那安逸的列国,因我从前稍微恼怒我民,他们就加害过分。
\par 16 所以耶和华如此说:现今我回到耶路撒冷,仍施怜悯,我的殿必重建在其中,准绳必拉在耶路撒冷之上。这是万军之耶和华说的。
\par 17 你要再宣告说,万军之耶和华如此说:我的城邑必再丰盛发达。耶和华必再安慰锡安,拣选耶路撒冷。」
\par 18 我举目观看,见有四角。
\par 19 我就问与我说话的天使说:「这是什麽意思?」他回答说:「这是打散犹大、以色列,和耶路撒冷的角。」
\par 20 耶和华又指四个匠人给我看。
\par 21 我说:「他们来做什麽呢?」他说:「这是打散犹大的角,使人不敢抬头;但这些匠人来威吓列国,打掉他们的角,就是举起打散犹大地的角。」

\chapter{2}

\par 1 我又举目观看,见一人手拿准绳。
\par 2 我说:「你往那里去?」他对我说:「要去量耶路撒冷,看有多宽多长。」
\par 3 与我说话的天使去的时候,又有一位天使迎著他来,
\par 4 对他说:「你跑去告诉那少年人说,耶路撒冷必有人居住,好像无城墙的乡村,因为人民和牲畜甚多。
\par 5 耶和华说:我要作耶路撒冷四围的火城,并要作其中的荣耀。」
\par 6 耶和华说:「我从前分散你们在天的四方(原文作犹如天的四风),现在你们要从北方之地逃回。这是耶和华说的。
\par 7 与巴比伦人同住的锡安民哪,应当逃脱。
\par 8 万军之耶和华说,在显出荣耀之後,差遣我去惩罚那掳掠你们的列国,摸你们的就是摸他眼中的瞳人。
\par 9 看哪,我(或作:他)要向他们抡手,他们就必作服事他们之人的掳物,你们便知道万军之耶和华差遣我了。
\par 10 锡安城啊,应当欢乐歌唱,因为我来要住在你中间。这是耶和华说的。」
\par 11 那时,必有许多国归附耶和华,作他(原文作我)的子民。他(原文作我)要住在你中间,你就知道万军之耶和华差遣我到你那里去了。
\par 12 耶和华必收回犹大作他圣地的分,也必再拣选耶路撒冷。
\par 13 凡有血气的都当在耶和华面前静默无声;因为他兴起,从圣所出来了。

\chapter{3}

\par 1 天使(原文作他)又指给我看:大祭司约书亚站在耶和华的使者面前;撒但也站在约书亚的右边,与他作对。
\par 2 耶和华向撒但说:「撒但哪,耶和华责备你!就是拣选耶路撒冷的耶和华责备你!这不是从火中抽出来的一根柴吗?」
\par 3 约书亚穿著污秽的衣服站在使者面前。
\par 4 使者吩咐站在面前的说:「你们要脱去他污秽的衣服」;又对约书亚说:「我使你脱离罪孽,要给你穿上华美的衣服。」
\par 5 我说:「要将洁净的冠冕戴在他头上。」他们就把洁净的冠冕戴在他头上,给他穿上华美的衣服,耶和华的使者在旁边站立。
\par 6 耶和华的使者告诫约书亚说:
\par 7 「万军之耶和华如此说:你若遵行我的道,谨守我的命令,你就可以管理我的家,看守我的院宇;我也要使你在这些站立的人中间来往。
\par 8 大祭司约书亚啊,你和坐在你面前的同伴都当听。(他们是作预兆的。)我必使我仆人大卫的苗裔发出。
\par 9 看哪,我在约书亚面前所立的石头,在一块石头上有七眼。万军之耶和华说:我要亲自雕刻这石头,并要在一日之间除掉这地的罪孽。
\par 10 当那日,你们各人要请邻舍坐在葡萄树和无花果树下。这是万军之耶和华说的。」

\chapter{4}

\par 1 那与我说话的天使又来叫醒我,好像人睡觉被唤醒一样。
\par 2 他问我说:「你看见了什麽?」我说:「我看见了一个纯金的灯台,顶上有盏灯,灯台上有七盏灯,每盏有七个管子。
\par 3 旁边有两棵橄榄树,一棵在灯盏的右边,一棵在灯盏的左边。」
\par 4 我问与我说话的天使说:「主啊,这是什麽意思?」
\par 5 与我说话的天使回答我说:「你不知道这是什麽意思吗?」我说:「主啊,我不知道。」
\par 6 他对我说:「这是耶和华指示所罗巴伯的。万军之耶和华说:不是倚靠势力,不是倚靠才能,乃是倚靠我的灵方能成事。
\par 7 大山哪,你算什麽呢?在所罗巴伯面前,你必成为平地。他必搬出一块石头,安在殿顶上。人且大声欢呼说:『愿恩惠恩惠归与这殿(殿:或作石)!』」
\par 8 耶和华的话又临到我说:
\par 9 「所罗巴伯的手立了这殿的根基,他的手也必完成这工,你就知道万军之耶和华差遣我到你们这里来了。
\par 10 谁藐视这日的事为小呢?这七眼乃是耶和华的眼睛,遍察全地,见所罗巴伯手拿线铊就欢喜。」
\par 11 我又问天使说:「这灯台左右的两棵橄榄树是什麽意思?」
\par 12 我二次问他说:「这两根橄榄枝在两个流出金色油的金嘴旁边是什麽意思?」
\par 13 他对我说:「你不知道这是什麽意思吗?」我说:「主啊,我不知道。」
\par 14 他说:「这是两个受膏者站在普天下主的旁边。」

\chapter{5}

\par 1 我又举目观看,见有一飞行的书卷。
\par 2 他问我说:「你看见什麽?」我回答说:「我看见一飞行的书卷,长二十肘,宽十肘。」
\par 3 他对我说:「这是发出行在遍地上的咒诅。凡偷窃的必按卷上这面的话除灭;凡起假誓的必按卷上那面的话除灭。
\par 4 万军之耶和华说:我必使这书卷出去,进入偷窃人的家和指我名起假誓人的家,必常在他家里,连房屋带木石都毁灭了。」
\par 5 与我说话的天使出来,对我说:「你要举目观看,见所出来的是什麽?」
\par 6 我说:「这是什麽呢?」他说:「这出来的是量器。」他又说:「这是恶人在遍地的形状。」
\par 7 (我见有一片圆铅被举起来。)这坐在量器中的是个妇人。
\par 8 天使说:「这是罪恶。」他就把妇人扔在量器中,将那片圆铅扔在量器的口上。
\par 9 我又举目观看,见有两个妇人出来,在他们翅膀中有风,飞得甚快,翅膀如同鹳鸟的翅膀。他们将量器抬起来,悬在天地中间。
\par 10 我问与我说话的天使说:「他们要将量器抬到那里去呢?」
\par 11 他对我说:「要往示拿地去,为他盖造房屋;等房屋齐备,就把他安置在自己的地方。」

\chapter{6}

\par 1 我又举目观看,见有四辆车从两山中间出来;那山是铜山。
\par 2 第一辆车套著红马,第二辆车套著黑马。
\par 3 第三辆车套著白马,第四辆车套著有斑点的壮马。
\par 4 我就问与我说话的天使说:「主啊,这是什麽意思?」
\par 5 天使回答我说:「这是天的四风,是从普天下的主面前出来的。」
\par 6 套著黑马的车往北方去,白马跟随在後;有斑点的马往南方去。
\par 7 壮马出来,要在遍地走来走去。天使说:「你们只管在遍地走来走去。」他们就照样行了。
\par 8 他又呼叫我说:「看哪,往北方去的已在北方安慰我的心。」
\par 9 耶和华的话临到我说:
\par 10 「你要从被掳之人中取黑玳、多比雅、耶大雅的金银。这三人是从巴比伦来到西番雅的儿子约西亚的家里。当日你要进他的家,
\par 11 取这金银做冠冕,戴在约撒答的儿子大祭司约书亚的头上,
\par 12 对他说,万军之耶和华如此说:看哪,那名称为大卫苗裔的,他要在本处长起来,并要建造耶和华的殿。
\par 13 他要建造耶和华的殿,并担负尊荣,坐在位上掌王权;又必在位上作祭司,使两职之间筹定和平。
\par 14 这冠冕要归希连(就是黑玳)、多比雅、耶大雅,和西番雅的儿子贤(就是约西亚),放在耶和华的殿里为记念。」
\par 15 远方的人也要来建造耶和华的殿,你们就知道万军之耶和华差遣我到你们这里来。你们若留意听从耶和华你们神的话,这事必然成就。

\chapter{7}

\par 1 大利乌王第四年九月,就是基斯流月初四日,耶和华的话临到撒迦利亚。
\par 2 那时伯特利人已经打发沙利亚和利坚米勒,并跟从他们的人,去恳求耶和华的恩,
\par 3 并问万军之耶和华殿中的祭司和先知说:「我历年以来,在五月间哭泣斋戒,现在还当这样行吗?」
\par 4 万军之耶和华的话就临到我说:
\par 5 「你要宣告国内的众民和祭司,说:『你们这七十年,在五月、七月禁食悲哀,岂是丝毫向我禁食吗?
\par 6 你们吃喝,不是为自己吃,为自己喝吗?
\par 7 当耶路撒冷和四围的城邑有居民,正兴盛,南地高原有人居住的时候,耶和华藉从前的先知所宣告的话,你们不当听吗?』」
\par 8 耶和华的话又临到撒迦利亚说:
\par 9 「万军之耶和华曾对你们的列祖如此说:『要按至理判断,各人以慈爱怜悯弟兄。
\par 10 不可欺压寡妇、孤儿、寄居的,和贫穷人。谁都不可心里谋害弟兄。』」
\par 11 他们却不肯听从,扭转肩头,塞耳不听,
\par 12 使心硬如金钢石,不听律法和万军之耶和华用灵藉从前的先知所说的话。故此,万军之耶和华大发烈怒。
\par 13 万军之耶和华说:「我曾呼唤他们,他们不听;将来他们呼求我,我也不听!
\par 14 我必以旋风吹散他们到素不认识的万国中。这样,他们的地就荒凉,甚至无人来往经过,因为他们使美好之地荒凉了。」

\chapter{8}

\par 1 万军之耶和华的话临到我说:
\par 2 「万军之耶和华如此说:我为锡安心里极其火热,我为他火热,向他的仇敌发烈怒。
\par 3 耶和华如此说:我现在回到锡安,要住在耶路撒冷中。耶路撒冷必称为诚实的城,万军之耶和华的山必称为圣山。
\par 4 万军之耶和华如此说:将来必有年老的男女坐在耶路撒冷街上,因为年纪老迈就手拿 杖。
\par 5 城中街上必满有男孩女孩玩耍。
\par 6 万军之耶和华如此说:到那日,这事在余剩的民眼中看为希奇,在我眼中也看为希奇吗?这是万军之耶和华说的。
\par 7 万军之耶和华如此说:我要从东方从西方救回我的民。
\par 8 我要领他们来,使他们住在耶路撒冷中。他们要作我的子民,我要作他们的神,都凭诚实和公义。」
\par 9 万军之耶和华如此说:「当建造万军之耶和华的殿,立根基之日的先知所说的话,现在你们听见,应当手里强壮。
\par 10 那日以先,人得不著雇价,牲畜也是如此;且因敌人的缘故,出入之人不得平安,乃因我使众人互相攻击。
\par 11 但如今,我待这余剩的民必不像从前。这是万军之耶和华说的。
\par 12 因为他们必平安撒种,葡萄树必结果子,地土必有出产,天也必降甘露。我要使这余剩的民享受这一切的福。
\par 13 犹大家和以色列家啊,你们从前在列国中怎样成为可咒诅的;照样,我要拯救你们,使人称你们为有福的(或作:使你们叫人得福)。你们不要惧怕,手要强壮。」
\par 14 万军之耶和华如此说:「你们列祖惹我发怒的时候,我怎样定意降祸,并不後悔。
\par 15 现在我照样定意施恩与耶路撒冷和犹大家,你们不要惧怕。
\par 16 你们所当行的是这样:各人与邻舍说话诚实,在城门口按至理判断,使人和睦。
\par 17 谁都不可心里谋害邻舍,也不可喜爱起假誓,因为这些事都为我所恨恶。这是耶和华说的。」
\par 18 万军之耶和华的话临到我说:
\par 19 「万军之耶和华如此说:四月、五月禁食的日子,七月、十月禁食的日子,必变为犹大家欢喜快乐的日子和欢乐的节期;所以你们要喜爱诚实与和平。」
\par 20 万军之耶和华如此说:「将来必有列国的人和多城的居民来到。
\par 21 这城的居民必到那城,说:『我们要快去恳求耶和华的恩,寻求万军之耶和华;我也要去。』
\par 22 必有列邦的人和强国的民来到耶路撒冷寻求万军之耶和华,恳求耶和华的恩。
\par 23 万军之耶和华如此说:在那些日子,必有十个人从列国诸族(原文作方言)中出来,拉住一个犹大人的衣襟,说:『我们要与你们同去,因为我们听见神与你们同在了。』」

\chapter{9}

\par 1 耶和华的默示应验在哈得拉地大马色,(世人和以色列各支派的眼目都仰望耶和华,)
\par 2 和靠近的哈马,并推罗、西顿;因为这二城的人大有智慧。
\par 3 推罗为自己修筑保障,积蓄银子如尘沙,堆起精金如街上的泥土。
\par 4 主必赶出他,打败他海上的权利;他必被火烧灭。
\par 5 亚实基伦看见必惧怕;迦萨看见甚痛苦;以革伦因失了盼望蒙羞。迦萨必不再有君王;亚实基伦也不再有居民。
\par 6 私生子(或作:外族人)必住在亚实突;我必除灭非利士人的骄傲。
\par 7 我必除去他口中带血之肉和牙齿内可憎之物。他必作为余剩的人归与我们的神,必在犹大像族长;以革伦人必如耶布斯人。
\par 8 我必在我家的四围安营,使敌军不得任意往来,暴虐的人也不再经过,因为我亲眼看顾我的家。
\par 9 锡安的民哪,应当大大喜乐;耶路撒冷的民哪,应当欢呼。看哪,你的王来到你这里!他是公义的,并且施行拯救,谦谦和和地骑著驴,就是骑著驴的驹子。
\par 10 我必除灭以法莲的战车和耶路撒冷的战马;争战的弓也必除灭。他必向列国讲和平;他的权柄必从这海管到那海,从大河管到地极。
\par 11 锡安哪,我因与你立约的血,将你中间被掳而囚的人从无水的坑中释放出来。
\par 12 你们被囚而有指望的人都要转回保障。我今日说明,我必加倍赐福给你们。
\par 13 我拿犹大作上弦的弓;我拿以法莲为张弓的箭。锡安哪,我要激发你的众子,攻击希利尼(原文作雅完)的众子,使你如勇士的刀。
\par 14 耶和华必显现在他们以上;他的箭必射出像闪电。主耶和华必吹角,乘南方的旋风而行。
\par 15 万军之耶和华必保护他们;他们必吞灭仇敌,践踏弹石。他们必喝血呐喊,犹如饮酒;他们必像盛满血的碗,又像坛的四角满了血。
\par 16 当那日,耶和华他们的神必看他的民如群羊,拯救他们;因为他们必像冠冕上的宝石,高举在他的地以上(或作:在他的地上发光辉)。
\par 17 他的恩慈何等大!他的荣美何其盛!五谷健壮少男;新酒培养处女。

\chapter{10}

\par 1 当春雨的时候,你们要向发闪电的耶和华求雨。他必为众人降下甘霖,使田园生长菜蔬。
\par 2 因为,家神所言的是虚空;卜士所见的是虚假;做梦者所说的是假梦。他们白白地安慰人,所以众人如羊流离,因无牧人就受苦。
\par 3 我的怒气向牧人发作;我必惩罚公山羊;因我万军之耶和华眷顾自己的羊群,就是犹大家,必使他们如骏马在阵上。
\par 4 房角石、钉子、争战的弓,和一切掌权的都从他而出。
\par 5 他们必如勇士在阵上将仇敌践踏在街上的泥土中。他们必争战,因为耶和华与他们同在;骑马的也必羞愧。
\par 6 我要坚固犹大家,拯救约瑟家,要领他们归回。我要怜恤他们;他们必像未曾弃绝的一样,都因我是耶和华他们的神,我必应允他们的祷告。
\par 7 以法莲人必如勇士;他们心中畅快如同喝酒;他们的儿女必看见而快活;他们的心必因耶和华喜乐。
\par 8 我要发嘶声,聚集他们,因我已经救赎他们。他们的人数必加增,如从前加增一样。
\par 9 我虽然(或作:必)播散他们在列国中,他们必在远方记念我。他们与儿女都必存活,且得归回。
\par 10 我必再领他们出埃及地,招聚他们出亚述,领他们到基列和利巴嫩;这地尚且不够他们居住。
\par 11 耶和华必经过苦海,击打海浪,使尼罗河的深处都枯乾。亚述的骄傲必致卑微;埃及的权柄必然灭没。
\par 12 我必使他们倚靠我,得以坚固;一举一动必奉我的名。这是耶和华说的。

\chapter{11}

\par 1 利巴嫩哪,开开你的门,任火烧灭你的香柏树。
\par 2 松树啊,应当哀号;因为香柏树倾倒,佳美的树毁坏。巴珊的橡树啊,应当哀号,因为茂盛的树林已经倒了。
\par 3 听啊,有牧人哀号的声音,因他们荣华的草场毁坏了。有少壮狮子咆哮的声音,因约但河旁的丛林荒废了。
\par 4 耶和华我的神如此说:「你撒迦利亚要牧养这将宰的群羊。
\par 5 买他们的宰了他们,以自己为无罪;卖他们的说:『耶和华是应当称颂的,因我成为富足。』牧养他们的并不怜恤他们。
\par 6 耶和华说:『我不再怜恤这地的居民,必将这民交给各人的邻舍和他们王的手中。他们必毁灭这地,我也不救这民脱离他们的手。』
\par 7 於是,我牧养这将宰的群羊,就是群中最困苦的羊。我拿著两根杖,一根我称为「荣美」,一根我称为「联索」。这样,我牧养了群羊。
\par 8 一月之内,我除灭三个牧人,因为我的心厌烦他们;他们的心也憎嫌我。
\par 9 我就说:「我不牧养你们。要死的,由他死;要丧亡的,由他丧亡;余剩的,由他们彼此相食。」
\par 10 我折断那称为「荣美」的杖,表明我废弃与万民所立的约。
\par 11 当日就废弃了。这样,那些仰望我的困苦羊就知道所说的是耶和华的话。
\par 12 我对他们说:「你们若以为美,就给我工价。不然,就罢了!」於是他们给了三十块钱作为我的工价。
\par 13 耶和华吩咐我说:「要把众人所估定美好的价值丢给窑户。」我便将这三十块钱,在耶和华的殿中丢给窑户了。
\par 14 我又折断称为「联索」的那根杖,表明我废弃犹大与以色列弟兄的情谊。
\par 15 耶和华又吩咐我说:「你再取愚昧人所用的器具,
\par 16 因我要在这地兴起一个牧人。他不看顾丧亡的,不寻找分散的,不医治受伤的,也不牧养强壮的;却要吃肥羊的肉,撕裂他的蹄子。
\par 17 无用的牧人丢弃羊群有祸了!刀必临到他的膀臂和右眼上。他的膀臂必全然枯乾;他的右眼也必昏暗失明。」

\chapter{12}

\par 1 耶和华论以色列的默示。铺张诸天、建立地基、造人里面之灵的耶和华说:
\par 2 「我必使耶路撒冷被围困的时候,向四围列国的民成为令人昏醉的杯;这默示也论到犹大(或作:犹大也是如此)。
\par 3 那日,我必使耶路撒冷向聚集攻击他的万民当作一块重石头;凡举起的必受重伤。
\par 4 耶和华说:到那日,我必使一切马匹惊惶,使骑马的颠狂。我必看顾犹大家,使列国的一切马匹瞎眼。
\par 5 犹大的族长必心里说:『耶路撒冷的居民倚靠万军之耶和华他们的神,就作我们的能力。』
\par 6 「那日,我必使犹大的族长如火盆在木柴中,又如火把在禾捆里;他们必左右烧灭四围列国的民。耶路撒冷人必仍住本处,就是耶路撒冷。
\par 7 「耶和华必先拯救犹大的帐棚,免得大卫家的荣耀和耶路撒冷居民的荣耀胜过犹大。
\par 8 那日,耶和华必保护耶路撒冷的居民。他们中间软弱的必如大卫;大卫的家必如神,如行在他们前面之耶和华的使者。
\par 9 那日,我必定意灭绝来攻击耶路撒冷各国的民。
\par 10 「我必将那施恩叫人恳求的灵,浇灌大卫家和耶路撒冷的居民。他们必仰望我(或作:他;本节同),就是他们所扎的;必为我悲哀,如丧独生子,又为我愁苦,如丧长子。
\par 11 那日,耶路撒冷必有大大的悲哀,如米吉多平原之哈达临门的悲哀。
\par 12 境内一家一家地都必悲哀。大卫家,男的独在一处,女的独在一处。拿单家,男的独在一处,女的独在一处。
\par 13 利未家,男的独在一处,女的独在一处。示每家,男的独在一处,女的独在一处。
\par 14 其余的各家,男的独在一处,女的独在一处。

\chapter{13}

\par 1 「那日,必给大卫家和耶路撒冷的居民开一个泉源,洗除罪恶与污秽。」
\par 2 万军之耶和华说:「那日,我必从地上除灭偶像的名,不再被人记念;也必使这地不再有假先知与污秽的灵。
\par 3 若再有人说预言,生他的父母必对他说:『你不得存活,因为你托耶和华的名说假预言。』生他的父母在他说预言的时候,要将他刺透。
\par 4 那日,凡作先知说预言的必因他所论的异象羞愧,不再穿毛衣哄骗人。
\par 5 他必说:『我不是先知,我是耕地的;我从幼年作人的奴仆。』
\par 6 必有人问他说:『你两臂中间是什麽伤呢?』他必回答说:『这是我在亲友家中所受的伤。』」
\par 7 万军之耶和华说:刀剑哪,应当兴起,攻击我的牧人和我的同伴。击打牧人,羊就分散;我必反手加在微小者的身上。
\par 8 耶和华说:这全地的人,三分之二必剪除而死,三分之一仍必存留。
\par 9 我要使这三分之一经火,熬炼他们,如熬炼银子;试炼他们,如试炼金子。他们必求告我的名,我必应允他们。我要说:这是我的子民。他们也要说:耶和华是我们的神。

\chapter{14}

\par 1 耶和华的日子临近,你的财物必被抢掠,在你中间分散。
\par 2 因为我必聚集万国与耶路撒冷争战,城必被攻取,房屋被抢夺,妇女被玷污,城中的民一半被掳去;剩下的民仍在城中,不至剪除。
\par 3 那时,耶和华必出去与那些国争战,好像从前争战一样。
\par 4 那日,他的脚必站在耶路撒冷前面朝东的橄榄山上。这山必从中间分裂,自东至西成为极大的谷。山的一半向北挪移,一半向南挪移。
\par 5 你们要从我山的谷中逃跑,因为山谷必延到亚萨。你们逃跑,必如犹大王乌西雅年间的人逃避大地震一样。耶和华我的神必降临,有一切圣者同来。
\par 6 那日,必没有光,三光必退缩。
\par 7 那日,必是耶和华所知道的,不是白昼,也不是黑夜,到了晚上才有光明。
\par 8 那日,必有活水从耶路撒冷出来,一半往东海流,一半往西海流;冬夏都是如此。
\par 9 耶和华必作全地的王。那日耶和华必为独一无二的,他的名也是独一无二的。
\par 10 全地,从迦巴直到耶路撒冷南方的临门,要变为亚拉巴。耶路撒冷必仍居高位,就是从便雅悯门到第一门之处,又到角门,并从哈楠业楼,直到王的酒 。
\par 11 人必住在其中,不再有咒诅。耶路撒冷人必安然居住。
\par 12 耶和华用灾殃攻击那与耶路撒冷争战的列国人,必是这样:他们两脚站立的时候,肉必消没,眼在眶中乾瘪,吞在口中溃烂。
\par 13 那日,耶和华必使他们大大扰乱。他们各人彼此揪住,举手攻击。
\par 14 犹大也必在耶路撒冷争战。那时四围各国的财物,就是许多金银衣服,必被收聚。
\par 15 那临到马匹、骡子、骆驼、驴,和营中一切牲畜的灾殃是与那灾殃一般。
\par 16 所有来攻击耶路撒冷列国中剩下的人,必年年上来敬拜大君王万军之耶和华,并守住棚节。
\par 17 地上万族中,凡不上耶路撒冷敬拜大君王万军之耶和华的,必无雨降在他们的地上。
\par 18 埃及族若不上来,雨也不降在他们的地上;凡不上来守住棚节的列国人,耶和华也必用这灾攻击他们。
\par 19 这就是埃及的刑罚和那不上来守住棚节之列国的刑罚。
\par 20 当那日,马的铃铛上必有「归耶和华为圣」的这句话。耶和华殿内的锅必如祭坛前的碗一样。
\par 21 凡耶路撒冷和犹大的锅都必归万军之耶和华为圣。凡献祭的都必来取这锅,煮肉在其中。当那日,在万军之耶和华的殿中必不再有迦南人。


\end{document}