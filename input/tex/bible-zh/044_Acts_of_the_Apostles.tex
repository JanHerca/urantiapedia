\begin{document}

\title{使徒行传}


\chapter{1}

\par 1 提阿非罗啊,我已经作了前书,论到耶稣开头一切所行所教训的,
\par 2 直到他藉著圣灵吩咐所拣选的使徒,以後被接上升的日子为止。
\par 3 他受害之後,用许多的凭据将自己活活的显给使徒看,四十天之久向他们显现,讲说神国的事。
\par 4 耶稣和他们聚集的时候,嘱咐他们说:「不要离开耶路撒冷,要等候父所应许的,就是你们听见我说过的。
\par 5 约翰是用水施洗,但不多几日,你们要受圣灵的洗。」
\par 6 他们聚集的时候,问耶稣说:「主啊,你复兴以色列国就在这时候吗?」
\par 7 耶稣对他们说:「父凭著自己的权柄所定的时候、日期,不是你们可以知道的。
\par 8 但圣灵降临在你们身上,你们就必得著能力,并要在耶路撒冷、犹太全地,和撒玛利亚,直到地极,作我的见证。」
\par 9 说了这话,他们正看的时候,他就被取上升,有一朵云彩把他接去,便看不见他了。
\par 10 当他往上去,他们定睛望天的时候,忽然有两个人身穿白衣,站在旁边,说:
\par 11 「加利利人哪,你们为什麽站著望天呢?这离开你们被接升天的耶稣,你们见他怎样往天上去,他还要怎样来。」
\par 12 有一座山,名叫橄榄山,离耶路撒冷不远,约有安息日可走的路程。当下,门徒从那里回耶路撒冷去,
\par 13 进了城,就上了所住的一间楼房;在那里有彼得、约翰、雅各、安得烈、腓力、多马、巴多罗买、马太、亚勒腓的儿子雅各、奋锐党的西门,和雅各的儿子(或作:兄弟)犹大。
\par 14 这些人同著几个妇人和耶稣的母亲马利亚,并耶稣的弟兄,都同心合意的恒切祷告。
\par 15 那时,有许多人聚会,约有一百二十名,彼得就在弟兄中间站起来,说:
\par 16 「弟兄们!圣灵藉大卫的口,在圣经上预言领人捉拿耶稣的犹大,这话是必须应验的。
\par 17 他本来列在我们数中,并且在使徒的职任上得了一分。
\par 18 这人用他作恶的工价买了一块田,以後身子仆倒,肚腹崩裂,肠子都流出来。
\par 19 住在耶路撒冷的众人都知道这事,所以按著他们那里的话给那块田起名叫亚革大马,就是「血田」的意思。
\par 20 因为诗篇上写著,说:愿他的住处变为荒场,无人在内居住;又说:愿别人得他的职分。
\par 21 所以,主耶稣在我们中间始终出入的时候,
\par 22 就是从约翰施洗起,直到主离开我们被接上升的日子为止,必须从那常与我们作伴的人中立一位与我们同作耶稣复活的见证。」
\par 23 於是选举两个人,就是那叫做巴撒巴,又称呼犹士都的约瑟,和马提亚。
\par 24 众人就祷告说:「主阿,你知道万人的心,求你从这两个人中,
\par 25 指明你所拣选的是谁,叫他得这使徒的位分。这位分犹大已经丢弃,往自己的地方去了。」
\par 26 於是众人为他们摇签,摇出马提亚来;他就和十一个使徒同列。

\chapter{2}

\par 1 五旬节到了,门徒都聚集在一处。
\par 2 忽然,从天上有响声下来,好像一阵大风吹过,充满了他们所坐的屋子,
\par 3 又有舌头如火焰显现出来,分开落在他们各人头上。
\par 4 他们就都被圣灵充满,按著圣灵所赐的口才说起别国的话来。
\par 5 那时,有虔诚的犹太人从天下各国来,住在耶路撒冷。
\par 6 这声音一响,众人都来聚集,各人听见门徒用众人的乡谈说话,就甚纳闷;
\par 7 都惊讶希奇说:「看哪,这说话的不都是加利利人吗?
\par 8 我们各人,怎麽听见他们说我们生来所用的乡谈呢?
\par 9 我们帕提亚人、玛代人、以拦人,和住在米所波大米、犹太、加帕多家、本都、亚西亚、
\par 10 弗吕家、旁非利亚、埃及的人,并靠近古利奈的吕彼亚一带地方的人,从罗马来的客旅中,或是犹太人,或是进犹太教的人,
\par 11 革哩底和亚拉伯人,都听见他们用我们的乡谈,讲说神的大作为。」
\par 12 众人就都惊讶猜疑,彼此说:「这是什麽意思呢?」
\par 13 还有人讥诮说:他们无非是新酒灌满了。
\par 14 彼得和十一个使徒站起,高声说:「犹太人和一切住在耶路撒冷的人哪,这件事你们当知道,也当侧耳听我的话。
\par 15 你们想这些人是醉了;其实不是醉了,因为时候刚到巳初。
\par 16 这正是先知约珥所说的:
\par 17 神说:在末後的日子,我要将我的灵浇灌凡有血气的。你们的儿女要说预言;你们的少年人要见异象;老年人要做异梦。
\par 18 在那些日子,我要将我的灵浇灌我的仆人和使女,他们就要说预言。
\par 19 在天上、我要显出奇事;在地下、我要显出神迹;有血,有火,有烟雾。
\par 20 日头要变为黑暗,月亮要变为血;这都在主大而明显的日子未到以前。
\par 21 到那时候,凡求告主名的,就必得救。
\par 22 「以色列人哪,请听我的话:神藉著拿撒勒人耶稣在你们中间施行异能、奇事神迹,将他证明出来,这是你们自己知道的。
\par 23 他既按著神的定旨先见被交与人,你们就藉著无法之人的手,把他钉在十字架上,杀了。
\par 24 神却将死的痛苦解释了,叫他复活,因为他原不能被死拘禁。
\par 25 大卫指著他说:我看见主常在我眼前;他在我右边,叫我不至於摇动。
\par 26 所以,我心里欢喜,我的灵(原文作舌)快乐;并且我的肉身要安居在指望中。
\par 27 因你必不将我的灵魂撇在阴间,也不叫你的圣者见朽坏。
\par 28 你已将生命的道路指示我,必叫我因见你的面(或作:叫我在你面前)得著满足的快乐。
\par 29 「弟兄们!先祖大卫的事,我可以明明的对你们说:他死了,也葬埋了,并且他的坟墓直到今日还在我们这里。
\par 30 大卫既是先知,又晓得神曾向他起誓,要从他的後裔中立一位坐在他的宝座上,
\par 31 就预先看明这事,讲论基督复活说:他的灵魂不撇在阴间;他的肉身也不见朽坏。
\par 32 这耶稣,神已经叫他复活了,我们都为这事作见证。
\par 33 他既被神的右手高举(或作:他既高举在神的右边),又从父受了所应许的圣灵,就把你们所看见所听见的,浇灌下来。
\par 34 大卫并没有升到天上,但自己说:主对我主说:你坐在我的右边,
\par 35 等我使你仇敌作你的脚凳。
\par 36 「故此,以色列全家当确实的知道,你们钉在十字架上的这位耶稣,神已经立他为主,为基督了。」
\par 37 众人听见这话,觉得扎心,就对彼得和其余的使徒说:「弟兄们,我们当怎样行?」
\par 38 彼得说:「你们各人要悔改,奉耶稣基督的名受洗,叫你们的罪得赦,就必领受所赐的圣灵;
\par 39 因为这应许是给你们和你们的儿女,并一切在远方的人,就是主我们神所召来的。」
\par 40 彼得还用许多话作见证,劝勉他们说:「你们当救自己脱离这弯曲的世代。」
\par 41 於是领受他话的人就受了洗。那一天,门徒约添了三千人,
\par 42 都恒心遵守使徒的教训,彼此交接,擘饼,祈祷。
\par 43 众人都惧怕;使徒又行了许多奇事神迹。
\par 44 信的人都在一处,凡物公用;
\par 45 并且卖了田产,家业,照各人所需用的分给各人。
\par 46 他们天天同心合意恒切的在殿里,且在家中擘饼,存著欢喜、诚实的心用饭,
\par 47 赞美神,得众民的喜爱。主将得救的人天天加给他们。

\chapter{3}

\par 1 申初祷告的时候,彼得、约翰上圣殿去。
\par 2 有一个人,生来是瘸腿的,天天被人抬来,放在殿的一个门口(那门名叫美门),要求进殿的人 济。
\par 3 他看见彼得、约翰将要进殿,就求他们 济。
\par 4 彼得约翰定睛看他;彼得说:「你看我们!」
\par 5 那人就留意看他们,指望得著什麽。
\par 6 彼得说:「金银我都没有,只把我所有的给你:我奉拿撒勒人耶稣基督的名,叫你起来行走!」
\par 7 於是拉著他的右手,扶他起来;他的脚和踝子骨立刻健壮了,
\par 8 就跳起来,站著,又行走,同他们进了殿,走著,跳著,赞美神。
\par 9 百姓都看见他行走,赞美神;
\par 10 认得他是那素常坐在殿的美门口求 济的,就因他所遇著的事满心希奇、惊讶。
\par 11 那人正在称为所罗门的廊下,拉著彼得、约翰;众百姓一齐跑到他们那里,很觉希奇。
\par 12 彼得看见,就对百姓说:「以色列人哪,为什麽把这事当作希奇呢?为什麽定睛看我们,以为我们凭自己的能力和虔诚使这人行走呢?
\par 13 亚伯拉罕、以撒、雅各的神,就是我们列祖的神,已经荣耀了他的仆人(或作:儿子)耶稣;你们却把他交付彼拉多。彼拉多定意要释放他,你们竟在彼拉多面前弃绝了他。
\par 14 你们弃绝了那圣洁公义者,反求著释放一个凶手给你们。
\par 15 你们杀了那生命的主,神却叫他从死里复活了;我们都是为这事作见证。
\par 16 我们因信他的名,他的名便叫你们所看见所认识的这人健壮了;正是他所赐的信心,叫这人在你们众人面前全然好了。
\par 17 弟兄们,我晓得你们作这事是出於不知,你们的官长也是如此。
\par 18 但神曾藉众先知的口,预言基督将要受害,就这样应验了。
\par 19 所以,你们当悔改归正,使你们的罪得以涂抹,这样,那安舒的日子就必从主面前来到;
\par 20 主也必差遣所预定给你们的基督(耶稣)降临。
\par 21 天必留他,等到万物复兴的时候,就是神从创世以来、藉著圣先知的口所说的。
\par 22 摩西曾说:『主神要从你们弟兄中间给你们兴起一位先知像我,凡他向你们所说的,你们都要听从。
\par 23 凡不听从那先知的,必要从民中全然灭绝。』
\par 24 从撒母耳以来的众先知,凡说预言的,也都说到这些日子。
\par 25 你们是先知的子孙,也承受神与你们祖宗所立的约,就是对亚伯拉罕说:『地上万族都要因你的後裔得福。』
\par 26 神既兴起他的仆人,(或作:儿子),就先差他到你们这里来,赐福给你们,叫你们各人回转,离开罪恶。」

\chapter{4}

\par 1 使徒对百姓说话的时候,祭司们和守殿官,并撒都该人忽然来了。
\par 2 因他们教训百姓,本著耶稣,传说死人复活,就很烦恼,
\par 3 於是下手拿住他们;因为天已经晚了,就把他们押到第二天。
\par 4 但听道之人有许多信的,男丁数目约有五千。
\par 5 第二天,官府、长老,和文士在耶路撒冷聚会,
\par 6 又有大祭司亚那和该亚法、约翰、亚力山大,并大祭司的亲族都在那里,
\par 7 叫使徒站在当中,就问他们说:「你们用什麽能力,奉谁的名做这事呢?」
\par 8 那时彼得被圣灵充满,对他们说:
\par 9 「治民的官府和长老啊,倘若今日因为在残疾人身上所行的善事查问我们他是怎麽得了痊愈,
\par 10 你们众人和以色列百姓都当知道,站在你们面前的这人得痊愈是因你们所钉十字架、神叫他从死里复活的拿撒勒人耶稣基督的名。
\par 11 他是你们匠人所弃的石头,已成了房角的头块石头。
\par 12 除他以外,别无拯救;因为在天下人间,没有赐下别的名,我们可以靠著得救。」
\par 13 他们见彼得、约翰的胆量,又看出他们原是没有学问的小民,就希奇,认明他们是跟过耶稣的;
\par 14 又看见那治好了的人和他们一同站著,就无话可驳。
\par 15 於是吩咐他们从公会出去,就彼此商议说:
\par 16 「我们当怎样办这两个人呢?因为他们诚然行了一件明显的神迹,凡住耶路撒冷的人都知道,我们也不能说没有。
\par 17 惟恐这事越发传扬在民间,我们必须恐吓他们,叫他们不再奉这名对人讲论。」
\par 18 於是叫了他们来,禁止他们总不可奉耶稣的名讲论教训人。
\par 19 彼得、约翰说:「听从你们,不听从神,这在神面前合理不合理,你们自己酌量吧!
\par 20 我们所看见所听见的,不能不说。」
\par 21 官长为百姓的缘故,想不出法子刑罚他们,又恐吓一番,把他们释放了。这是因众人为所行的奇事都归荣耀与神。
\par 22 原来藉著神迹医好的那人有四十多岁了。
\par 23 二人既被释放,就到会友那里去,把祭司长和长老所说的话都告诉他们。
\par 24 他们听见了,就同心合意的高声向神说:「主啊!你是造天、地、海,和其中万物的,
\par 25 你曾藉著圣灵,托你仆人我们祖宗大卫的口,说:外邦为什麽争闹?万民为什麽谋算虚妄的事?
\par 26 世上的君王一齐起来,臣宰也聚集,要敌挡主,并主的受膏者(或作:基督)。
\par 27 希律和本丢彼拉多,外邦人和以色列民,果然在这城里聚集,要攻打你所膏的圣仆(仆:或作子)耶稣,
\par 28 成就你手和你意旨所预定必有的事。
\par 29 他们恐吓我们,现在求主鉴察,一面叫你仆人大放胆量讲你的道,
\par 30 一面伸出你的手来医治疾病,并且使神迹奇事因著你圣仆(仆:或作子)耶稣的名行出来。」
\par 31 祷告完了,聚会的地方震动,他们就都被圣灵充满,放胆讲论神的道。
\par 32 那许多信的人都是一心一意的,没有一人说他的东西有一样是自己的,都是大家公用。
\par 33 使徒大有能力,见证主耶稣复活;众人也都蒙大恩。
\par 34 内中也没有一个缺乏的,因为人人将田产房屋都卖了,把所卖的价银拿来,放在使徒脚前,
\par 35 照各人所需用的,分给各人。
\par 36 有一个利未人,生在居比路,名叫约瑟,使徒称他为巴拿巴(巴拿巴翻出来就是劝慰子)。
\par 37 他有田地,也卖了,把价银拿来,放在使徒脚前。

\chapter{5}

\par 1 有一个人,名叫亚拿尼亚,同他的妻子撒非喇卖了田产,
\par 2 把价银私自留下几分,他的妻子也知道,其余的几分拿来放在使徒脚前。
\par 3 彼得说:「亚拿尼亚!为什麽撒但充满了你的心,叫你欺哄圣灵,把田地的价银私自留下几分呢?
\par 4 田地还没有卖,不是你自己的吗?既卖了,价银不是你作主吗?你怎麽心里起这意念呢?你不是欺哄人,是欺哄神了。」
\par 5 亚拿尼亚听见这话,就仆倒,断了气;听见的人都甚惧怕。
\par 6 有些少年人起来,把他包裹,抬出去埋葬了。
\par 7 约过了三小时,他的妻子进来,还不知道这事。
\par 8 彼得对他说:「你告诉我,你们卖田地的价银就是这些吗?」他说:「就是这些。」
\par 9 彼得说:「你们为什麽同心试探主的灵呢?埋葬你丈夫之人的脚已到门口,他们也要把你抬出去。」
\par 10 妇人立刻仆倒在彼得脚前,断了气。那些少年人进来,见他已经死了,就抬出去,埋在他丈夫旁边。
\par 11 全教会和听见这事的人都甚惧怕。
\par 12 主藉使徒的手在民间行了许多神迹奇事;他们(或作:信的人)都同心合意的在所罗门的廊下。
\par 13 其余的人没有一个敢贴近他们百姓却尊重他们。
\par 14 信而归主的人越发增添,连男带女很多。
\par 15 甚至有人将病人抬到街上,放在床上或褥子上,指望彼得过来的时候,或者得他的影儿照在什麽人身上。
\par 16 还有许多人带著病人和被污鬼缠磨的,从耶路撒冷四围的城邑来,全都得了医治。
\par 17 大祭司和他的一切同人,就是撒都该教门的人,都起来,满心忌恨,
\par 18 就下手拿住使徒,收在外监。
\par 19 但主的使者夜间开了监门,领他们出来,
\par 20 说:「你们去站在殿里,把这生命的道都讲给百姓听。」
\par 21 使徒听了这话,天将亮的时候就进殿里去教训人。大祭司和他的同人来了,叫齐公会的人,和以色列族的众长老,就差人到监里去,要把使徒提出来。
\par 22 但差役到了,不见他们在监里,就回来禀报说:
\par 23 「我们看见监牢关得极妥当,看守的人也站在门外;及至开了门,里面一个人都不见。」
\par 24 守殿官和祭司长听见这话,心里犯难,不知这事将来如何。
\par 25 有一个人来禀报说:「你们收在监里的人,现在站在殿里教训百姓。」
\par 26 於是守殿官和差役去带使徒来,并没有用强暴,因为怕百姓用石头打他们。
\par 27 带到了,便叫使徒站在公会前;大祭司问他们说:
\par 28 「我们不是严严的禁止你们,不可奉这名教训人吗?你们倒把你们的道理充满了耶路撒冷,想要叫这人的血归到我们身上!」
\par 29 彼得和众使徒回答说:「顺从神,不顺从人,是应当的。
\par 30 你们挂在木头上杀害的耶稣,我们祖宗的神已经叫他复活。
\par 31 神且用右手将他高举(或作:他就是神高举在自己的右边),叫他作君王,作救主,将悔改的心和赦罪的恩赐给以色列人。
\par 32 我们为这事作见证;神赐给顺从之人的圣灵也为这事作见证。」
\par 33 公会的人听见就极其恼怒,想要杀他们。
\par 34 但有一个法利赛人,名叫迦玛列,是众百姓所敬重的教法师,在公会中站起来,吩咐人把使徒暂且带到外面去,
\par 35 就对众人说:「以色列人哪,论到这些人,你们应当小心怎样办理。
\par 36 从前丢大起来,自夸为大;附从他的人约有四百,他被杀後,附从他的全都散了,归於无有。
\par 37 此後,报名上册的时候,又有加利利的犹大起来,引诱些百姓跟从他;他也灭亡,附从他的人也都四散了。
\par 38 现在,我劝你们不要管这些人,任凭他们吧!他们所谋的、所行的,若是出於人,必要败坏;
\par 39 若是出於神,你们就不能败坏他们,恐怕你们倒是攻击神了。」
\par 40 公会的人听从了他,便叫使徒来,把他们打了,又吩咐他们不可奉耶稣的名讲道,就把他们释放了。
\par 41 他们离开公会,心里欢喜,因被算是配为这名受辱。
\par 42 他们就每日在殿里、在家里、不住的教训人,传耶稣是基督。

\chapter{6}

\par 1 那时,门徒增多,有说希利尼话的犹太人向希伯来人发怨言,因为在天天的供给上忽略了他们的寡妇。
\par 2 十二使徒叫众门徒来,对他们说:「我们撇下神的道去管理饭食,原是不合宜的。
\par 3 所以弟兄们,当从你们中间选出七个有好名声、被圣灵充满、智慧充足的人,我们就派他们管理这事。
\par 4 但我们要专心以祈祷、传道为事。」
\par 5 大众都喜悦这话,就拣选了司提反,乃是大有信心、圣灵充满的人,又拣选腓利、伯罗哥罗、尼迦挪、提门、巴米拿,并进犹太教的安提阿人尼哥拉,
\par 6 叫他们站在使徒面前。使徒祷告了,就按手在他们头上。
\par 7 神的道兴旺起来;在耶路撒冷门徒数目加增的甚多,也有许多祭司信从了这道。
\par 8 司提反满得恩惠、能力,在民间行了大奇事和神迹。
\par 9 当时有称利百地拿会堂的几个人,并有古利奈、亚力山大、基利家、亚西亚、各处会堂的几个人,都起来和司提反辩论。
\par 10 司提反是以智慧和圣灵说话,众人敌挡不住,
\par 11 就买出人来说:「我们听见他说谤 摩西和神的话。」
\par 12 他们又耸动了百姓、长老,并文士,就忽然来捉拿他,把他带到公会去,
\par 13 设下假见证,说:「这个人说话,不住的糟践圣所和律法。
\par 14 我们曾听见他说:这拿撒勒人耶稣要毁坏此地,也要改变摩西所交给我们的规条。」
\par 15 在公会里坐著的人都定睛看他,见他的面貌,好像天使的面貌。

\chapter{7}

\par 1 大祭司就说:『这些事果然有吗?』
\par 2 司提反说:「诸位父兄请听!当日我们的祖宗亚伯拉罕在米所波大米还未住哈兰的时候,荣耀的神向他显现,
\par 3 对他说:『你要离开本地和亲族,往我所要指示你的地方去。』
\par 4 他就离开迦勒底人之地,住在哈兰。他父亲死了以後,神使他从那里搬到你们现在所住之地。
\par 5 在这地方,神并没有给他产业,连立足之地也没有给他;但应许要将这地赐给他和他的後裔为业;那时他还没有儿子。
\par 6 神说:『他的後裔必寄居外邦,那里的人要叫他们作奴仆,苦待他们四百年。』
\par 7 神又说:『使他们作奴仆的那国,我要惩罚。以後他们要出来,在这地方事奉我。』
\par 8 神又赐他割礼的约。於是亚伯拉罕生了以撒,第八日给他行了割礼。以撒生雅各,雅各生十二位先祖。
\par 9 先祖嫉妒约瑟,把他卖到埃及去;神却与他同在,
\par 10 救他脱离一切苦难,又使他在埃及王法老面前得恩典,有智慧。法老就派他作埃及国的宰相兼管全家。
\par 11 後来埃及和迦南全地遭遇饥荒,大受艰难,我们的祖宗就绝了粮。
\par 12 雅各听见在埃及有粮,就打发我们的祖宗初次往那里去。
\par 13 第二次约瑟与弟兄们相认,他的亲族也被法老知道了。
\par 14 约瑟就打发弟兄请父亲雅各和全家七十五个人都来。
\par 15 於是雅各下了埃及,後来他和我们的祖宗都死在那里;
\par 16 又被带到示剑,葬於亚伯拉罕在示剑用银子从哈抹子孙买来的坟墓里。
\par 17 「及至神应许亚伯拉罕的日期将到,以色列民在埃及兴盛众多,
\par 18 直到有不晓得约瑟的新王兴起。
\par 19 他用诡计待我们的宗族,苦害我们的祖宗,叫他们丢弃婴孩,使婴孩不能存活。
\par 20 那时,摩西生下来,俊美非凡,在他父亲家里抚养了三个月。
\par 21 他被丢弃的时候,法老的女儿拾了去,养为自己的儿子。
\par 22 摩西学了埃及人一切的学问,说话行事都有才能。
\par 23 「他将到四十岁,心中起意去看望他的弟兄以色列人;
\par 24 到了那里,见他们一个人受冤屈,就护庇他,为那受欺压的人报仇,打死了那埃及人。
\par 25 他以为弟兄必明白神是藉他的手搭救他们;他们却不明白。
\par 26 第二天,遇见两个以色列人争斗,就劝他们和睦,说:『你们二位是弟兄,为什麽彼此欺负呢?』
\par 27 那欺负邻舍的把他推开,说:『谁立你作我们的首领和审判官呢?
\par 28 难道你要杀我,像昨天杀那埃及人麽?』
\par 29 摩西听见这话就逃走了,寄居於米甸;在那里生了两个儿子。
\par 30 「过了四十年,在西乃山的旷野,有一位天使从荆棘火焰中向摩西显现。
\par 31 摩西见了那异象,便觉希奇,正进前观看的时候,有主的声音说:
\par 32 『我是你列祖的神,就是亚伯拉罕的神,以撒的神,雅各的神。』摩西战战兢兢,不敢观看。
\par 33 主对他说:『把你脚上的鞋脱下来;因为你所站之地是圣地。
\par 34 我的百姓在埃及所受的困苦,我实在看见了,他们悲叹的声音,我也听见了。我下来要救他们。你来!我要差你往埃及去。』
\par 35 这摩西就是百姓弃绝说『谁立你作我们的首领和审判官』的;神却藉那在荆棘中显现之使者的手差派他作首领、作救赎的。
\par 36 这人领百姓出来,在埃及,在红海、在旷野,四十年间行了奇事神迹。
\par 37 那曾对以色列人说『神要从你们弟兄中间给你们兴起一位先知像我』的,就是这位摩西。
\par 38 这人曾在旷野会中和西乃山上,与那对他说话的天使同在,又与我们的祖宗同在,并且领受活泼的圣言传给我们。
\par 39 我们的祖宗不肯听从,反弃绝他,心里归向埃及,
\par 40 对亚伦说:『你且为我们造些神像,在我们前面引路;因为领我们出埃及地的那个摩西,我们不知道他遭了什麽事。』
\par 41 那时,他们造了一个牛犊,又拿祭物献给那像,欢喜自己手中的工作。
\par 42 神就转脸不顾,任凭他们事奉天上的日月星辰,正如先知书上所写的说:以色列家啊,你们四十年间在旷野,岂是将牺牲和祭物献给我吗?
\par 43 你们抬著摩洛的帐幕和理番神的星,就是你们所造为要敬拜的像。因此,我要把你们迁到巴比伦外去。
\par 44 「我们的祖宗在旷野,有法柜的帐幕,是神吩咐摩西叫他照所看见的样式做的。
\par 45 这帐幕,我们的祖宗相继承受。当神在他们面前赶出外邦人去的时候,他们同约书亚把帐幕搬进承受为业之地,直存到大卫的日子。
\par 46 大卫在神面前蒙恩,祈求为雅各的神预备居所;
\par 47 却是所罗门为神造成殿宇。
\par 48 其实,至高者并不住人手所造的,就如先知所言:
\par 49 主说:天是我的座位,地是我的脚凳;你们要为我造何等的殿宇?那里是我安息的地方呢?
\par 50 这一切不都是我手所造的吗?
\par 51 「你们这硬著颈项、心与耳未受割礼的人,常时抗拒圣灵!你们的祖宗怎样,你们也怎样。
\par 52 那一个先知不是你们祖宗逼迫呢?他们也把预先传说那义者要来的人杀了;如今你们又把那义者卖了,杀了。
\par 53 你们受了天使所传的律法,竟不遵守。」
\par 54 众人听见这话就极其恼怒,向司提反咬牙切齿。
\par 55 但司提反被圣灵充满,定睛望天,看见神的荣耀,又看见耶稣站在神的右边,
\par 56 就说:「我看见天开了,人子站在神的右边。」
\par 57 众人大声喊叫, 著耳朵,齐心拥上前去,
\par 58 把他推到城外,用石头打他。作见证的人把衣裳放在一个少年人名叫扫罗的脚前。
\par 59 他们正用石头打的时候,司提反呼吁主说:「求主耶稣接收我的灵魂!」
\par 60 又跪下大声喊著说:「主啊,不要将这罪归於他们!」说了这话,就睡了。扫罗也喜悦他被害。

\chapter{8}

\par 1 从这日起,耶路撒冷的教会大遭逼迫,除了使徒以外,门徒都分散在犹太和撒玛利亚各处。
\par 2 有虔诚的人把司提反埋葬了,为他捶胸大哭。
\par 3 扫罗却残害教会,进各人的家,拉著男女下在监里。
\par 4 那些分散的人往各处去传道。
\par 5 腓利下撒玛利亚城去,宣讲基督。
\par 6 众人听见了,又看见腓利所行的神迹,就同心合意的听从他的话。
\par 7 因为有许多人被污鬼附著,那些鬼大声呼叫,从他们身上出来;还有许多瘫痪的,瘸腿的,都得了医治。
\par 8 在那城里,就大有欢喜。
\par 9 有一个人,名叫西门,向来在那城里行邪术,妄自尊大,使撒玛利亚的百姓惊奇;
\par 10 无论大小都听从他,说:「这人就是那称为神的大能者。」
\par 11 他们听从他,因他久用邪术,使他们惊奇。
\par 12 及至他们信了腓利所传神国的福音和耶稣基督的名,连男带女就受了洗。
\par 13 西门自己也信了;既受了洗,就常与腓利在一处,看见他所行的神迹和大异能,就甚惊奇。
\par 14 使徒在耶路撒冷听见撒玛利亚人领受了神的道,就打发彼得、约翰往他们那里去。
\par 15 两个人到了,就为他们祷告,要叫他们受圣灵。
\par 16 因为圣灵还没有降在他们一个人身上,他们只奉主耶稣的名受了洗。
\par 17 於是使徒按手在他们头上,他们就受了圣灵。
\par 18 西门看见使徒按手,便有圣灵赐下,就拿钱给使徒,
\par 19 说:「把这权柄也给我,叫我手按著谁,谁就可以受圣灵。」
\par 20 彼得说:「你的银子和你一同灭亡吧!因你想神的恩赐是可以用钱买的。
\par 21 你在这道上无分无关;因为在神面前,你的心不正。
\par 22 你当懊悔你这罪恶,祈求主,或者你心里的意念可得赦免。
\par 23 我看出你正在苦胆之中,被罪恶捆绑。」
\par 24 西门说:愿你们为我求主,叫你们所说的,没有一样临到我身上。
\par 25 使徒既证明主道,而且传讲,就回耶路撒冷去,一路在撒玛利亚好些村庄传扬福音。
\par 26 有主的一个使者对腓利说:「起来!向南走,往那从耶路撒冷下迦萨的路上去。」那路是旷野。
\par 27 腓利就起身去了,不料,有一个埃提阿伯(就是古实,见以赛亚十八章一节)人,是个有大权的太监,在埃提阿伯女王干大基的手下总管银库,他上耶路撒冷礼拜去了。
\par 28 现在回来,在车上坐著,念先知以赛亚的书。
\par 29 圣灵对腓利说:「你去!贴近那车走。」
\par 30 腓利就跑到太监那里,听见他念先知以赛亚的书,便问他说:「你所念的,你明白吗?」
\par 31 他说:「没有人指教我,怎能明白呢?」於是请腓利上车,与他同坐。
\par 32 他所念的那段经,说:他像羊被牵到宰杀之地,又像羊羔在剪毛的人手下无声;他也是这样不开口。
\par 33 他卑微的时候,人不按公义审判他(原文作他的审判被夺去);谁能述说他的世代,因为他的生命从地上夺去。』
\par 34 太监对腓利说:「请问,先知说这话是指著谁?是指著自己呢?是指著别人呢?」
\par 35 腓利就开口从这经上起,对他传讲耶稣。
\par 36 二人正往前走,到了有水的地方,太监说:「看哪,这里有水,我受洗有什麽妨碍呢?」(有古卷在此有
\par 37 腓利说:「你若是一心相信,就可以。」他回答说:「我信耶稣基督是神的儿子。」)
\par 38 於是吩咐车站住,腓利和太监二人同下水里去,腓利就给他施洗。
\par 39 从水里上来,主的灵把腓利提了去,太监也不再见他了,就欢欢喜喜的走路。
\par 40 後来有人在亚锁都遇见腓利;他走遍那地方,在各城宣传福音,直到该撒利亚。

\chapter{9}

\par 1 扫罗仍然向主的门徒口吐威吓凶杀的话,去见大祭司,
\par 2 求文书给大马色的各会堂,若是找著信奉这道的人,无论男女,都准他捆绑带到耶路撒冷。
\par 3 扫罗行路,将到大马色,忽然从天上发光,四面照著他;
\par 4 他就仆倒在地,听见有声音对他说:「扫罗!扫罗!你为什麽逼迫我?」
\par 5 他说:「主啊!你是谁?」主说:「我就是你所逼迫的耶稣。
\par 6 起来!进城去,你所当作的事,必有人告诉你。」
\par 7 同行的人站在那里,说不出话来,听见声音,却看不见人。
\par 8 扫罗从地上起来,睁开眼睛,竟不能看见什麽。有人拉他的手,领他进了大马色;
\par 9 三日不能看见,也不吃也不喝。
\par 10 当下,在大马色有一个门徒,名叫亚拿尼亚。主在异象中对他说:「亚拿尼亚。」他说:「主,我在这里。」
\par 11 主对他说:「起来!往直街去,在犹大的家里,访问一个大数人,名叫扫罗。他正祷告,
\par 12 又看见了一个人,名叫亚拿尼亚,进来按手在他身上,叫他能看见。」
\par 13 亚拿尼亚回答说:「主啊,我听见许多人说:这人怎样在耶路撒冷多多苦害你的圣徒,
\par 14 并且他在这里有从祭司长得来的权柄捆绑一切求告你名的人。」
\par 15 主对亚拿尼亚说:「你只管去!他是我所拣选的器皿,要在外邦人和君王,并以色列人面前宣扬我的名。
\par 16 我也要指示他,为我的名必须受许多的苦难。」
\par 17 亚拿尼亚就去了,进入那家,把手按在扫罗身上,说:「兄弟扫罗,在你来的路上,向你显现的主,就是耶稣,打发我来,叫你能看见,又被圣灵充满。」
\par 18 扫罗的眼睛上,好像有鳞立刻掉下来,他就能看见。於是起来受了洗;
\par 19 吃过饭就健壮了。扫罗和大马色的门徒同住了些日子,
\par 20 就在各会堂里宣传耶稣,说他是神的儿子。
\par 21 凡听见的人都惊奇,说:「在耶路撒冷残害求告这名的,不是这人吗?并且他到这里来,特要捆绑他们,带到祭司长那里。」
\par 22 但扫罗越发有能力,驳倒住大马色的犹太人,证明耶稣是基督。
\par 23 过了好些日子,犹太人商议要杀扫罗,
\par 24 但他们的计谋被扫罗知道了。他们又昼夜在城门守候,要杀他。
\par 25 他的门徒就在夜间用筐子把他从城墙上缒下去。
\par 26 扫罗到了耶路撒冷,想与门徒结交,他们却都怕他,不信他是门徒。
\par 27 惟有巴拿巴接待他,领去见使徒,把他在路上怎麽看见主,主怎麽向他说话,他在大马色怎麽奉耶稣的名放胆传道,都述说出来。
\par 28 於是扫罗在耶路撒冷和门徒出入来往,
\par 29 奉主的名放胆传道,并与说希利尼话的犹太人讲论辩驳;他们却想法子要杀他。
\par 30 弟兄们知道了就送他下该撒利亚,打发他往大数去。
\par 31 那时,犹太加利利、撒玛利亚各处的教会都得平安,被建立;凡事敬畏主,蒙圣灵的安慰,人数就增多了。
\par 32 彼得周流四方的时候,也到了居住吕大的圣徒那里;
\par 33 遇见一个人,名叫以尼雅,得了瘫痪,在褥子上躺卧八年。
\par 34 彼得对他说:「以尼雅,耶稣基督医好你了;起来!收拾你的褥子。」他就立刻起来了。
\par 35 凡住吕大和沙仑的人都看见了他,就归服主。
\par 36 在约帕有一个女徒,名叫大比大,翻希利尼话就是多加(就是羚羊的意思);他广行善事,多施 济。
\par 37 当时,他患病而死,有人把他洗了,停在楼上。
\par 38 吕大原与约帕相近;门徒听见彼得在那里,就打发两个人去见他,央求他说:「快到我们那里去,不要耽延。」
\par 39 彼得就起身和他们同去;到了,便有人领他上楼。众寡妇都站在彼得旁边哭,拿多加与他们同在时所做的里衣外衣给他看。
\par 40 彼得叫他们都出去,就跪下祷告,转身对著死人说:「大比大,起来!」他就睁开眼睛,见了彼得,便坐起来。
\par 41 彼得伸手扶他起来,叫众圣徒和寡妇进去,把多加活活的交给他们。
\par 42 这事传遍了约帕,就有许多人信了主。
\par 43 此後,彼得在约帕一个硝皮匠西门的家里住了多日。

\chapter{10}

\par 1 在该撒利亚有一个人,名叫哥尼流,是「义大利营」的百夫长。
\par 2 他是个虔诚人,他和全家都敬畏神,多多 济百姓,常常祷告神。
\par 3 有一天,约在申初,他在异象中明明看见神的一个使者进去,到他那里,说:「哥尼流。」
\par 4 哥尼流定睛看他,惊怕说:「主啊,什麽事呢?」天使说:「你的祷告和你的 济达到神面前,已蒙记念了。
\par 5 现在你当打发人往约帕去,请那称呼彼得的西门来。
\par 6 他住在海边一个硝皮匠西门的家里,房子在海边上。」
\par 7 向他说话的天使去後,哥尼流叫了两个家人和常伺候他的一个虔诚兵来,
\par 8 把这事都述说给他们听,就打发他们往约帕去。
\par 9 第二天,他们行路将近那城。彼得约在午正,上房顶去祷告,
\par 10 觉得饿了,想要吃。那家的人正预备饭的时候,彼得魂游象外,
\par 11 看见天开了,有一物降下,好像一块大布,系著四角,缒在地上,
\par 12 里面有地上各样四足的走兽和昆虫,并天上的飞鸟;
\par 13 又有声音向他说:「彼得,起来,宰了吃!」
\par 14 彼得却说:「主啊,这是不可的!凡俗物和不洁净的物,我从来没有吃过。」
\par 15 第二次有声音向他说:「神所洁净的,你不可当作俗物。」
\par 16 这样一连三次,那物随即收回天上去了。
\par 17 彼得心里正在猜疑之间,不知所看见的异象是什麽意思。哥尼流所差来的人已经访问到西门的家,站在门外,
\par 18 喊著问:「有称呼彼得的西门住在这里没有?」
\par 19 彼得还思想那异象的时候,圣灵向他说:「有三个人来找你。」
\par 20 起来,下去,和他们同往,不要疑惑,因为是我差他们来的。」
\par 21 於是彼得下去见那些人,说:「我就是你们所找的人。你们来是为什麽缘故?」
\par 22 他们说:「百夫长哥尼流是个义人,敬畏神,为犹太通国所称赞。他蒙一位圣天使指示,叫他请你到他家里去,听你的话。」
\par 23 彼得就请他们进去,住了一宿。次日,起身和他们同去,还有约帕的几个弟兄同著他去;
\par 24 又次日,他们进入该撒利亚,哥尼流已经请了他的亲属密友等候他们。
\par 25 彼得一进去,哥尼流就迎接他,俯伏在他脚前拜他。
\par 26 彼得却拉他,说:「你起来,我也是人。」
\par 27 彼得和他说著话进去,见有好些人在那里聚集,
\par 28 就对他们说:「你们知道,犹太人和别国的人亲近来往本是不合例的,但神已经指示我,无论什麽人都不可看作俗而不洁净的。
\par 29 所以我被请的时候,就不推辞而来。现在请问:你们叫我来有什麽意思呢?」
\par 30 哥尼流说:「前四天,这个时候,我在家中守著申初的祷告,忽然有一个人穿著光明的衣裳,站在我面前,
\par 31 说:『哥尼流,你的祷告已蒙垂听,你的 济达到神面前已蒙记念了。
\par 32 你当打发人往约帕去,请那称呼彼得的西门来,他住在海边一个硝皮匠西门的家里。』
\par 33 所以我立时打发人去请你。你来了很好;现今我们都在神面前,要听主所吩咐你的一切话。」
\par 34 彼得就开口说:「我真看出神是不偏待人。
\par 35 原来,各国中那敬畏主、行义的人都为主所悦纳。
\par 36 神藉著耶稣基督(他是万有的主)传和平的福音,将这道赐给以色列人。
\par 37 这话在约翰宣传洗礼以後,从加利利起,传遍了犹太。
\par 38 神怎样以圣灵和能力膏拿撒勒人耶稣,这都是你们知道的。他周流四方,行善事,医好凡被魔鬼压制的人,因为神与他同在。
\par 39 他在犹太人之地,并耶路撒冷所行的一切事,有我们作见证。他们竟把他挂在木头上杀了。
\par 40 第三日,神叫他复活,显现出来;
\par 41 不是显现给众人看,乃是显现给神预先所拣选为他作见证的人看,就是我们这些在他从死里复活以後和他同吃同喝的人。
\par 42 他吩咐我们传道给众人,证明他是神所立定的,要作审判活人、死人的主。
\par 43 众先知也为他作见证说:『凡信他的人必因他的名得蒙赦罪。』」
\par 44 彼得还说这话的时候,圣灵降在一切听道的人身上。
\par 45 那些奉割礼、和彼得同来的信徒,见圣灵的恩赐也浇在外邦人身上,就都希奇;
\par 46 因听见他们说方言,称赞神为大。
\par 47 於是彼得说:「这些人既受了圣灵,与我们一样,谁能禁止用水给他们施洗呢?」
\par 48 就吩咐奉耶稣基督的名给他们施洗。他们又请彼得住了几天。

\chapter{11}

\par 1 使徒和在犹太的众弟兄听说外邦人也领受了神的道。
\par 2 及至彼得上了耶路撒冷,那些奉割礼的门徒和他争辩说:
\par 3 「你进入未受割礼之人的家和他们一同吃饭了。」
\par 4 彼得就开口把这事挨次给他们讲解说:
\par 5 「我在约帕城里祷告的时候,魂游象外,看见异象,有一物降下,好像一块大布,系著四角,从天缒下,直来到我跟前。」
\par 6 我定睛观看,见内中有地上四足的牲畜和野兽、昆虫,并天上的飞鸟。
\par 7 我且听见有声音向我说:『彼得,起来,宰了吃!』
\par 8 我说:『主啊,这是不可的!凡俗而不洁净的物从来没有入过我的口。』
\par 9 第二次,有声音从天上说:神所洁净的,你不可当作俗物。
\par 10 这样一连三次,就都收回天上去了。
\par 11 正当那时,有三个人站在我们所住的房门前,是从该撒利亚差来见我的。
\par 12 圣灵吩咐我和他们同去,不要疑惑。(或作:不要分别等类)。同著我去的,还有这六位弟兄;我们都进了那人的家,
\par 13 那人就告诉我们,他如何看见一位天使,站在他屋里,说:你打发人往约帕去,请那称呼彼得的西门来;』
\par 14 他有话告诉你,可以叫你和你的全家得救。
\par 15 我一开讲,圣灵便降在他们身上,正像当初降在我们身上一样。
\par 16 我就想起主的话说:『约翰是用水施洗,但你们要受圣灵的洗。』
\par 17 神既然给他们恩赐,像在我们信主耶稣基督的时候给了我们一样;我是谁,能拦阻神呢!」
\par 18 众人听见这话,就不言语了,只归荣耀与神,说:「这样看来,神也赐恩给外邦人,叫他们悔改得生命了。」
\par 19 那些因司提反的事遭患难四散的门徒直走到腓尼基和居比路,并安提阿;他们不向别人讲道,只向犹太人讲。
\par 20 但内中有居比路和古利奈人,他们到了安提阿也向希利尼人传讲主耶稣(有古卷:也向说希利尼话的犹太人传讲主耶稣)。
\par 21 主与他们同在,信而归主的人就很多了。
\par 22 这风声传到耶路撒冷教会人的耳中,他们就打发巴拿巴出去,走到安提阿为止。
\par 23 他到了那里,看见神所赐的恩就欢喜,劝勉众人,立定心志,恒久靠主。
\par 24 这巴拿巴原是个好人,被圣灵充满,大有信心。於是有许多人归服了主。
\par 25 他又往大数去找扫罗,
\par 26 找著了,就带他到安提阿去。他们足有一年的工夫和教会一同聚集,教训了许多人。门徒称为「基督徒」是从安提阿起首。
\par 27 当那些日子,有几位先知从耶路撒冷下到安提阿。
\par 28 内中有一位,名叫亚迦布,站起来,藉著圣灵指明天下将有大饥荒(这事到革老丢年间果然有了。)
\par 29 於是门徒定意照各人的力量捐钱,送去供给住在犹太的弟兄。
\par 30 他们就这样行,把捐项托巴拿巴和扫罗送到众长老那里。

\chapter{12}

\par 1 那时,希律王下手苦害教会中几个人,
\par 2 用刀杀了约翰的哥哥雅各。
\par 3 他见犹太人喜欢这事,又去捉拿彼得。那时正是除酵的日子。
\par 4 希律拿了彼得,收在监里,交付四班兵丁看守,每班四个人,意思要在逾越节後把他提出来,当著百姓办他。
\par 5 於是彼得被囚在监里;教会却为他切切的祷告神。
\par 6 希律将要提他出来的前一夜,彼得被两条铁链锁著,睡在两个兵丁当中;看守的人也在门外看守。
\par 7 忽然,有主的一个使者站在旁边,屋里有光照耀,天使拍彼得的肋旁,拍醒了他,说:「快快起来!」那铁链就从他手上脱落下来。
\par 8 天使对他说:「束上带子,穿上鞋。」他就那样做。天使又说:「披上外衣,跟著我来。」
\par 9 彼得就出来跟著他,不知道天使所做是真的,只当见了异象。
\par 10 过了第一层第二层监牢,就来到临街的铁门,那门自己开了。他们出来,走过一条街,天使便离开他去了。
\par 11 彼得醒悟过来,说:「我现在真知道主差遣他的使者,救我脱离希律的手和犹太百姓一切所盼望的。」
\par 12 想了一想,就往那称呼马可的约翰、他母亲马利亚家去,在那里有好些人聚集祷告。
\par 13 彼得敲外门,有一个使女,名叫罗大,出来探听,
\par 14 听见是彼得的声音,就欢喜的顾不得开门,跑进去告诉众人说:「彼得站在门外。」
\par 15 他们说:「你是疯了!」使女极力的说:「真是他!」他们说:「必是他的天使!」
\par 16 彼得不住的敲门。他们开了门,看见他,就甚惊奇。
\par 17 彼得摆手,不要他们作声,就告诉他们主怎样领他出监;又说:「你们把这事告诉雅各和众弟兄。」於是出去,往别处去了。
\par 18 到了天亮,兵丁扰乱得很,不知道彼得往那里去了。
\par 19 希律找他,找不著,就审问看守的人,吩咐把他们拉去杀了。後来希律离开犹太,下该撒利亚去,住在那里。
\par 20 希律恼怒推罗、西顿的人。他们那一带地方是从王的地土得粮,因此就托了王的内侍臣伯拉斯都的情,一心来求和。
\par 21 希律在所定的日子,穿上朝服,坐在位上,对他们讲论一番。
\par 22 百姓喊著说:「这是神的声音,不是人的声音。」
\par 23 希律不归荣耀给神,所以主的使者立刻罚他,他被虫所咬,气就绝了。
\par 24 神的道日见兴旺,越发广传。
\par 25 巴拿巴和扫罗办完了他们供给的事,就从耶路撒冷回来,带著称呼马可的约翰同去。

\chapter{13}

\par 1 在安提阿的教会中,有几位先知和教师,就是巴拿巴和称呼尼结的西面、古利奈人路求,与分封之王希律同养的马念,并扫罗。
\par 2 他们事奉主、禁食的时候,圣灵说:「要为我分派巴拿巴和扫罗,去做我召他们所做的工。」
\par 3 於是禁食祷告,按手在他们头上,就打发他们去了。
\par 4 他们既被圣灵差遣,就下到西流基,从那里坐船往居比路去。
\par 5 到了撒拉米,就在犹太人各会堂里传讲神的道,也有约翰作他们的帮手。
\par 6 经过全岛,直到帕弗,在那里遇见一个有法术,假充先知的犹太人,名叫巴耶稣。
\par 7 这人常和方伯士求保罗同在。士求保罗是个通达人,他请了巴拿巴和扫罗来,要听神的道。
\par 8 只是那行法术的以吕马(这名翻出来就是行法术的意思)敌挡使徒,要叫方伯不信真道。
\par 9 扫罗又名保罗,被圣灵充满,定睛看他,
\par 10 说:「你这充满各样诡诈奸恶,魔鬼的儿子,众善的仇敌,你混乱主的正道还不止住吗?
\par 11 现在主的手加在你身上,你要瞎眼,暂且不见日光。」他的眼睛立刻昏蒙黑暗,四下里求人拉著手领他。
\par 12 方伯看见所做的事,很希奇主的道,就信了。
\par 13 保罗和他的同人从帕弗开船,来到旁非利亚的别加,约翰就离开他们,回耶路撒冷去。
\par 14 他们离了别加往前行,来到彼西底的安提阿,在安息日进会堂坐下。
\par 15 读完了律法和先知的书,管会堂的,叫人过去,对他们说:「二位兄台,若有什麽劝勉众人的话,请说。」
\par 16 保罗就站起来,举手,说:「以色列人和一切敬畏神的人,请听。
\par 17 这以色列民的神拣选了我们的祖宗,当民寄居埃及的时候抬举他们,用大能的手领他们出来;
\par 18 又在旷野容忍(或作:抚养)他们,约有四十年。
\par 19 既灭了迦南地七族的人,就把那地分给他们为业;
\par 20 此後给他们设立士师,约有四百五十年,直到先知撒母耳的时候。
\par 21 後来他们求一个王,神就将便雅悯支派中基士的儿子扫罗,给他们作王四十年。
\par 22 既废了扫罗,就选立大卫作他们的王,又为他作见证说:『我寻得耶西的儿子大卫,他是合我心意的人,凡事要遵行我的旨意。』
\par 23 从这人的後裔中,神已经照著所应许的,为以色列人立了一位救主,就是耶稣。
\par 24 在他没有出来以先,约翰向以色列众民宣讲悔改的洗礼。
\par 25 约翰将行尽他的程途说:『你们以为我是谁?我不是基督;只是有一位在我以後来的,我解他脚上的鞋带也是不配的。』
\par 26 弟兄们,亚伯拉罕的子孙和你们中间敬畏神的人哪,这救世的道是传给我们的。
\par 27 耶路撒冷居住的人和他们的官长,因为不认识基督,也不明白每安息日所读众先知的书,就把基督定了死罪,正应了先知的预言;
\par 28 虽然查不出他有当死的罪来,还是求彼拉多杀他;
\par 29 既成就了经上指著他所记的一切话,就把他从木头上取下来,放在坟墓里。
\par 30 神却叫他从死里复活。
\par 31 那从加利利同他上耶路撒冷的人多日看见他,这些人如今在民间是他的见证。
\par 32 我们也报好信息给你们,就是那应许祖宗的话,
\par 33 神已经向我们这作儿女的应验,叫耶稣复活了。正如诗篇第二篇上记著说:你是我的儿子,我今日生你。
\par 34 论到神叫他从死里复活,不再归於朽坏,就这样说:我必将所应许大卫那圣洁、可靠的恩典赐给你们。
\par 35 又有一篇上说:你必不叫你的圣者见朽坏。
\par 36 「大卫在世的时候遵行了神的旨意,就睡了(或作:大卫按神的旨意服事了他那一世的人,就睡了),归到他祖宗那里,已见朽坏;
\par 37 惟独神所复活的,他并未见朽坏。
\par 38 所以,弟兄们,你们当晓得:赦罪的道是由这人传给你们的。
\par 39 你们靠摩西的律法,在一切不得称义的事上信靠这人,就都得称义了。
\par 40 所以,你们务要小心,免得先知书上所说的临到你们。
\par 41 主说:你们这轻慢的人要观看,要惊奇,要灭亡;因为在你们的时候,我行一件事,虽有人告诉你们,你们总是不信。
\par 42 他们出会堂的时候,众人请他们到下安息日再讲这话给他们听。
\par 43 散会以後,犹太人和敬虔进犹太教的人多有跟从保罗、巴拿巴的。二人对他们讲道,劝他们务要恒久在神的恩中。
\par 44 到下安息日,合城的人几乎都来聚集,要听神的道。
\par 45 但犹太人看见人这样多,就满心嫉妒,硬驳保罗所说的话,并且毁谤。
\par 46 保罗和巴拿巴放胆说:「神的道先讲给你们原是应当的;只因你们弃绝这道,断定自己不配得永生,我们就转向外邦人去。
\par 47 因为主曾这样吩咐我们说:我已经立你作外邦人的光,叫你施行救恩,直到地极。」
\par 48 外邦人听见这话,就欢喜了,赞美神的道;凡预定得永生的人都信了。
\par 49 於是主的道传遍了那一带地方。
\par 50 但犹太人挑唆虔敬、尊贵的妇女和城内有名望的人,逼迫保罗、巴拿巴,将他们赶出境外。
\par 51 二人对著众人跺下脚上的尘土,就往以哥念去了。
\par 52 门徒满心喜乐,又被圣灵充满。

\chapter{14}

\par 1 二人在以哥念同进犹太人的会堂,在那里讲的,叫犹太人和希利尼人信的很多。
\par 2 但那不顺从的犹太人耸动外邦人,叫他们心里恼恨弟兄。
\par 3 二人在那里住了多日,倚靠主放胆讲道;主藉他们的手施行神迹奇事,证明他的恩道。
\par 4 城里的众人就分了党,有附从犹太人的,有附从使徒的。
\par 5 那时,外邦人和犹太人,并他们的官长,一齐拥上来,要凌辱使徒,用石头打他们。
\par 6 使徒知道了,就逃往吕高尼的路司得、特庇两个城和周围地方去,
\par 7 在那里传福音À
\par 8 路司得城里坐著一个两脚无力的人,生来是瘸腿的,从来没有走过。
\par 9 他听保罗讲道,保罗定睛看他,见他有信心,可得痊愈,
\par 10 就大声说:「你起来,两脚站直!」那人就跳起来,而且行走。
\par 11 众人看见保罗所做的事,就用吕高尼的话大声说:「有神藉著人形降临在我们中间了。」
\par 12 於是称巴拿巴为丢斯,称保罗为希耳米,因为他说话领首。
\par 13 有城外丢斯庙的祭司牵著牛,拿著花圈,来到门前,要同众人向使徒献祭。
\par 14 巴拿巴、保罗二使徒听见,就撕开衣裳,跳进众人中间,喊著说:
\par 15 「诸君,为什麽做这事呢?我们也是人,性情和你们一样。我们传福音给你们,是叫你们离弃这些虚妄,归向那创造天、地、海、和其中万物的永生神。
\par 16 他在从前的世代,任凭万国各行其道;
\par 17 然而为自己未尝不显出证据来,就如常施恩惠,从天降雨,赏赐丰年,叫你们饮食饱足,满心喜乐。」
\par 18 二人说了这些话,仅仅的拦住众人不献祭与他们。
\par 19 但有些犹太人从安提阿和以哥念来,挑唆众人,就用石头打保罗,以为他是死了,便拖到城外。
\par 20 门徒正围著他,他就起来,走进城去。第二天,同巴拿巴往特庇去,
\par 21 对那城里的人传了福音,使好些人作门徒,就回路司得、以哥念、安提阿去,
\par 22 坚固门徒的心,劝他们恒守所信的道;又说:「我们进入神的国,必须经历许多艰难。」
\par 23 二人在各教会中选立了长老,又禁食祷告,就把他们交托所信的主。
\par 24 二人经过彼西底,来到旁非利亚。
\par 25 在别加讲了道,就下亚大利去,
\par 26 从那里坐船,往安提阿去。当初,他们被众人所托、蒙神之恩,要办现在所做之工,就是在这地方。
\par 27 到了那里,聚集了会众,就述说神藉他们所行的一切事,并神怎样为外邦人开了信道的门。
\par 28 二人就在那里同门徒住了多日。

\chapter{15}

\par 1 有几个人从犹太下来,教训弟兄们说:「你们若不按摩西的规条受割礼,不能得救。」
\par 2 保罗、巴拿巴与他们大大的分争辩论;众门徒就定规,叫保罗、巴拿巴和本会中几个人,为所辩论的,上耶路撒冷去见使徒和长老。
\par 3 於是教会送他们起行。他们经过腓尼基、撒玛利亚,随处传说外邦人归主的事,叫众弟兄都甚欢喜。
\par 4 到了耶路撒冷,教会和使徒并长老都接待他们,他们就述说神同他们所行的一切事。
\par 5 惟有几个信徒、是法利赛教门的人,起来说:「必须给外邦人行割礼,吩咐他们遵守摩西的律法。」
\par 6 使徒和长老聚会商议这事;
\par 7 辩论已经多了,彼得就起来,说:「诸位弟兄,你们知道神早已在你们中间拣选了我,叫外邦人从我口中得听福音之道,而且相信。
\par 8 知道人心的神也为他们作了见证,赐圣灵给他们,正如给我们一样;
\par 9 又藉著信洁净了他们的心,并不分他们我们。
\par 10 现在为什麽试探神,要把我们祖宗和我们所不能负的轭放在门徒的颈项上呢?
\par 11 我们得救乃是因主耶稣的恩,和他们一样,这是我们所信的。」
\par 12 众人都默默无声,听巴拿巴和保罗述说神藉他们在外邦人中所行的神迹奇事。
\par 13 他们住了声,雅各就说:「诸位弟兄,请听我的话。
\par 14 方才西门述说神当初怎样眷顾外邦人,从他们中间选取百姓归於自己的名下;
\par 15 众先知的话也与这意思相合。
\par 16 正如经上所写的:此後,我要回来,重新修造大卫倒塌的帐幕,把那破坏的重新修造建立起来,
\par 17 叫余剩的人,就是凡称为我名下的外邦人,都寻求主。
\par 18 这话是从创世以来,显明这事的主说的。
\par 19 「所以据我的意见,不可难为那归服神的外邦人;
\par 20 只要写信,吩咐他们禁戒偶像的污秽和奸淫,并勒死的牲畜和血。
\par 21 因为从古以来,摩西的书在各城有人传讲,每逢安息日,在会堂里诵读。」
\par 22 那时,使徒和长老并全教会定意从他们中间拣选人,差他们和保罗、巴拿巴同往安提阿去;所拣选的就是称呼巴撒巴的犹大和西拉。这两个人在弟兄中是作首领的。
\par 23 於是写信交付他们,内中说:「使徒和作长老的弟兄们问安提阿、利亚、基利家外邦众弟兄的安。
\par 24 我们听说,有几个人从我们这里出去,用言语搅扰你们,惑乱你们的心。(有古卷加:你们必须受割礼,守摩西的律法。)其实我们并没有吩咐他们。
\par 25 所以,我们同心定意,拣选几个人,差他们同我们所亲爱的巴拿巴和保罗往你们那里去。
\par 26 这二人是为我主耶稣基督的名不顾性命的。
\par 27 我们就差了犹大和西拉,他们也要亲口诉说这些事。
\par 28 因为圣灵和我们定意不将别的重担放在你们身上;惟有几件事是不可少的,
\par 29 就是禁戒祭偶像的物和血,并勒死的牲畜和奸淫。这几件你们若能自己禁戒不犯就好了。愿你们平安!」
\par 30 他们既奉了差遣,就下安提阿去,聚集众人,交付书信。
\par 31 众人念了,因为信上安慰的话就欢喜了。
\par 32 犹大和西拉也是先知,就用许多话劝勉弟兄,坚固他们。
\par 33 住了些日子,弟兄们打发他们平平安安的回到差遣他们的人那里去。(有古卷加:)
\par 34 (惟有西拉定意仍住在那里。)
\par 35 但保罗和巴拿巴仍住在安提阿,和许多别人一同教训人,传主的道。
\par 36 过了些日子,保罗对巴拿巴说:「我们可以回到从前宣传主道的各城,看望弟兄们景况如何。」
\par 37 巴拿巴有意要带称呼马可的约翰同去;
\par 38 但保罗因为马可从前在旁非利亚离开他们,不和他们同去做工,就以为不可带他去。
\par 39 於是二人起了争论,甚至彼此分开。巴拿巴带著马可,坐船往居比路去;
\par 40 保罗拣选了西拉,也出去,蒙弟兄们把他交於主的恩中。
\par 41 他就走遍利亚、基利家,坚固众教会。

\chapter{16}

\par 1 保罗来到特庇,又到路司得。在那里有一个门徒,名叫提摩太,是信主之犹太妇人的儿子,他父亲却是希利尼人。
\par 2 路司得和以哥念的弟兄都称赞他。
\par 3 保罗要带他同去,只因那些地方的犹太人都知道他父亲是希利尼人,就给他行了割礼。
\par 4 他们经过各城,把耶路撒冷使徒和长老所定的条规交给门徒遵守。
\par 5 於是众教会信心越发坚固,人数天天加增。
\par 6 圣灵既然禁止他们在亚西亚讲道,他们就经过弗吕家、加拉太一带地方。
\par 7 到了每西亚的边界,他们想要往庇推尼去,耶稣的灵却不许。
\par 8 他们就越过每西亚,下到特罗亚去。
\par 9 在夜间有异象现与保罗。有一个马其顿人站著求他说:「请你过到马其顿来帮助我们。」
\par 10 保罗既看见这异象,我们随即想要往马其顿去,以为神召我们传福音给那里的人听。
\par 11 於是从特罗亚开船,一直行到撒摩特喇,第二天到了尼亚波利。
\par 12 从那里来到腓立比,就是马其顿这一方的头一个城,也是罗马的驻防城。我们在这城里住了几天。
\par 13 当安息日,我们出城门,到了河边,知道那里有一个祷告的地方,我们就坐下对那聚会的妇女讲道。
\par 14 有一个卖紫色布疋的妇人,名叫吕底亚,是推雅推喇城的人,素来敬拜神。他听见了,主就开导他的心,叫他留心听保罗所讲的话。
\par 15 他和他一家既领了洗,便求我们说:「你们若以为我是真信主的(或作:你们若以为我是忠心事主的),请到我家里来住。於是强留我们。
\par 16 後来,我们往那祷告的地方去。有一个使女迎著面来,他被巫鬼所附,用法术,叫他主人们大得财利。
\par 17 他跟随保罗和我们,喊著说:「这些人是至高神的仆人,对你们传说救人的道。」
\par 18 他一连多日这样喊叫,保罗就心中厌烦,转身对那鬼说:「我奉耶稣基督的名,吩咐你从他身上出来!」那鬼当时就出来了。
\par 19 使女的主人们见得利的指望没有了,便揪住保罗和西拉,拉他们到市上去见首领;
\par 20 又带到官长面前说:「这些人原是犹太人,竟骚扰我们的城,
\par 21 传我们罗马人所不可受不可行的规矩。」
\par 22 众人就一同起来攻击他们。官长吩咐剥了他们的衣裳,用棍打;
\par 23 打了许多棍,便将他们下在监里,嘱咐禁卒严紧看守。
\par 24 禁卒领了这样的命,就把他们下在内监里,两脚上了木狗。
\par 25 约在半夜,保罗和西拉祷告,唱诗赞美神,众囚犯也侧耳而听。
\par 26 忽然,地大震动,甚至监牢的地基都摇动了,监门立刻全开,众囚犯的锁链也都松开了。
\par 27 禁卒一醒,看见监门全开,以为囚犯已经逃走,就拔刀要自杀。
\par 28 保罗大声呼叫说:「不要伤害自己!我们都在这里。」
\par 29 禁卒叫人拿灯来,就跳进去,战战兢兢的俯伏在保罗、西拉面前;
\par 30 又领他们出来,说:「二位先生,我当怎样行才可以得救?」
\par 31 他们说:「当信主耶稣,你和你一家都必得救。」
\par 32 他们就把主的道讲给他和他全家的人听。
\par 33 当夜,就在那时候,禁卒把他们带去,洗他们的伤;他和属乎他的人立时都受了洗。
\par 34 於是禁卒领他们上自己家里去,给他们摆上饭。他和全家,因为信了神,都很喜乐。
\par 35 到了天亮,官长打发差役来,说:「释放那两个人吧。」
\par 36 禁卒就把这话告诉保罗说:「官长打发人来叫释放你们,如今可以出监,平平安安的去吧。」
\par 37 保罗却说:「我们是罗马人,并没有定罪,他们就在众人面前打了我们,又把我们下在监里,现在要私下撵我们出去吗?这是不行的。叫他们自己来领我们出去吧?」
\par 38 差役把这话回禀官长。官长听见他们是罗马人,就害怕了,
\par 39 於是来劝他们,领他们出来,请他们离开那城。
\par 40 二人出了监,往吕底亚家里去;见了弟兄们,劝慰他们一番,就走了。

\chapter{17}

\par 1 保罗和西拉经过暗妃波里、亚波罗尼亚,来到帖撒罗尼迦,在那里有犹太人的会堂。
\par 2 保罗照他素常的规矩进去,一连三个安息日,本著圣经与他们辩论,
\par 3 讲解陈明基督必须受害,从死里复活;又说:「我所传与你们的这位耶稣就是基督。」
\par 4 他们中间有些人听了劝,就附从保罗和西拉,并有许多虔敬的希利尼人,尊贵的妇女也不少。
\par 5 但那不信的犹太人心里嫉妒,招聚了些市井匪类,搭夥成群,耸动合城的人闯进耶孙的家,要将保罗、西拉带到百姓那里。
\par 6 找不著他们,就把耶孙和几个弟兄拉到地方官那里,喊叫说:「那搅乱天下的也到这里来了,
\par 7 耶孙收留他们。这些人都违背该撒的命令,说另有一个王耶稣。」
\par 8 众人和地方官听见这话,就惊慌了;
\par 9 於是取了耶孙和其余之人的保状,就释放了他们。
\par 10 弟兄们随即在夜间打发保罗和西拉往庇哩亚去。二人到了,就进入犹太人的会堂。
\par 11 这地方的人贤於帖撒罗尼迦的人,甘心领受这道,天天考查圣经,要晓得这道是与不是。
\par 12 所以他们中间多有相信的,又有希利尼尊贵的妇女,男子也不少。
\par 13 但帖撒罗尼迦的犹太人知道保罗又在庇哩亚传神的道,也就往那里去,耸动搅扰众人。
\par 14 当时弟兄们便打发保罗往海边去,西拉和提摩太仍住在庇哩亚。
\par 15 送保罗的人带他到了雅典,既领了保罗的命,叫西拉和提摩太速速到他这里来,就回去了。
\par 16 保罗在雅典等候他们的时候,看见满城都是偶像,就心里著急;
\par 17 於是在会堂里与犹太人和虔敬的人,并每日在市上所遇见的人,辩论。
\par 18 还有以彼古罗和斯多亚两门的学士,与他争论。有的说:「这胡言乱语的要说什麽?」有的说:「他似乎是传说外邦鬼神的。」这话是因保罗传讲耶稣与复活的道。
\par 19 他们就把他带到亚略巴古,说:「你所讲的这新道,我们也可以知道吗?」
\par 20 因为你有些奇怪的事传到我们耳中,我们愿意知道这些事是什麽意思。」
\par 21 (雅典人和住在那里的客人都不顾别的事,只将新闻说说听听。)
\par 22 保罗站在亚略巴古当中,说:「众位雅典人哪,我看你们凡事很敬畏鬼神。
\par 23 我游行的时候,观看你们所敬拜的,遇见一座坛,上面写著『未识之神』。你们所不认识而敬拜的,我现在告诉你们。
\par 24 创造宇宙和其中万物的神,既是天地的主,就不住人手所造的殿,
\par 25 也不用人手服事,好像缺少什麽;自己倒将生命、气息、万物,赐给万人。
\par 26 他从一本(本:有古卷是血脉)造出万族的人,住在全地上,并且预先定准他们的年限和所住的疆界,
\par 27 要叫他们寻求神,或者可以揣摩而得,其实他离我们各人不远;
\par 28 我们生活、动作、存留,都在乎他。就如你们作诗的,有人说:『我们也是他所生的。』
\par 29 我们既是神所生的,就不当以为神的神性像人用手艺、心思所雕刻的金、银、石。
\par 30 世人蒙昧无知的时候,神并不监察,如今却吩咐各处的人都要悔改。
\par 31 因为他已经定了日子,要藉著他所设立的人按公义审判天下,并且叫他从死里复活,给万人作可信的凭据。」
\par 32 众人听见从死里复活的话,就有讥诮他的;又有人说:「我们再听你讲这个吧!」
\par 33 於是保罗从他们当中出去了。
\par 34 但有几个人贴近他,信了主,其中有亚略巴古的官丢尼修,并一个妇人,名叫大马哩,还有别人一同信从。

\chapter{18}

\par 1 这事以後,保罗离了雅典,来到哥林多。
\par 2 遇见一个犹太人,名叫亚居拉,他生在本都;因为革老丢犹太人都离开罗马,新近带著妻百基拉,从义大利来。保罗就投奔了他们。
\par 3 他们本是制造帐棚为业。保罗因与他们同业,就和他们同住做工。
\par 4 每逢安息日,保罗在会堂里辩论,劝化犹太人和希利尼人。
\par 5 西拉和提摩太从马其顿来的时候,保罗为道迫切,向犹太人证明耶稣是基督。
\par 6 他们既抗拒、毁谤,保罗就抖著衣裳,说:「你们的罪 原文作血 归到你们自己头上,与我无干(原文作我却乾净)。从今以後,我要往外邦人那里去。」
\par 7 於是离开那里,到了一个人的家中;这人名叫提多犹士都,是敬拜神的,他的家靠近会堂。
\par 8 管会堂的基利司布和全家都信了主,还有许多哥林多人听了,就相信受洗。
\par 9 夜间,主在异象中对保罗说:「不要怕,只管讲,不要闭口,
\par 10 有我与你同在,必没有人下手害你,因为在这城里我有许多的百姓。」
\par 11 保罗在那里住了一年零六个月,将神的道教训他们。
\par 12 到迦流作亚该亚方伯的时候,犹太人同心起来攻击保罗,拉他到公堂,
\par 13 说:「这个人劝人不按著律法敬拜神。」
\par 14 保罗刚要开口,迦流就对犹太人说:「你们这些犹太人!如果是为冤枉或奸恶的事,我理当耐性听你们。
\par 15 但所争论的,若是关乎言语、名目,和你们的律法,你们自己去办吧!这样的事我不愿意审问」;
\par 16 就把他们撵出公堂。
\par 17 众人便揪住管会堂的所提尼,在堂前打他。这些事迦流都不管。
\par 18 保罗又住了多日,就辞别了弟兄,坐船往利亚去;百基拉、亚居拉和他同去。他因为许过愿,就在坚革哩剪了头发。
\par 19 到了以弗所,保罗就把他们留在那里,自己进了会堂,和犹太人辩论。
\par 20 众人请他多住些日子,他却不允,
\par 21 就辞别他们,说:「神若许我,我还要回到你们这里」;於是开船离了以弗所。
\par 22 在该撒利亚下了船,就上耶路撒冷去问教会安,随後下安提阿去。
\par 23 住了些日子,又离开那里,挨次经过加拉太和弗吕家地方,坚固众门徒。
\par 24 有一个犹太人,名叫亚波罗,来到以弗所。他生在亚力山大,是有学问(或作:口才)的,最能讲解圣经。
\par 25 这人已经在主的道上受了教训,心里火热,将耶稣的事详细讲论教训人;只是他单晓得约翰的洗礼。
\par 26 他在会堂里放胆讲道;百基拉,亚居拉听见,就接他来,将神的道给他讲解更加详细。
\par 27 他想要往亚该亚去,弟兄们就勉励他,并写信请门徒接待他(或作:弟兄们就写信劝门徒接待他)。他到了那里,多帮助那蒙恩信主的人,
\par 28 在众人面前极有能力、驳倒犹太人,引圣经证明耶稣是基督。

\chapter{19}

\par 1 亚波罗在哥林多的时候,保罗经过了上边一带地方,就来到以弗所;在那里遇见几个门徒,
\par 2 问他们说:「你们信的时候受了圣灵没有?」他们回答说:「没有,也未曾听见有圣灵赐下来。」
\par 3 保罗说:「这样,你们受的是什麽洗呢?」他们说:「是约翰的洗。」
\par 4 保罗说:「约翰所行的是悔改的洗,告诉百姓当信那在他以後要来的,就是耶稣。」
\par 5 他们听见这话,就奉主耶稣的名受洗。
\par 6 保罗按手在他们头上,圣灵便降在他们身上,他们就说方言,又说预言(或作:又讲道)
\par 7 一共约有十二个人。
\par 8 保罗进会堂,放胆讲道,一连三个月,辩论神国的事,劝化众人。
\par 9 後来,有些人心里刚硬不信,在众人面前毁谤这道,保罗就离开他们,也叫门徒与他们分离,便在推喇奴的学房天天辩论。
\par 10 这样有两年之久,叫一切住在亚西亚的,无论是犹太人,是希利尼人,都听见主的道。
\par 11 神藉保罗的手行了些非常的奇事;
\par 12 甚至有人从保罗身上拿手巾或围裙放在病人身上,病就退了,恶鬼也出去了。
\par 13 那时,有几个游行各处、念咒赶鬼的犹太人,向那被恶鬼附的人擅自称主耶稣的名,说:「我奉保罗所传的耶稣敕令你们出来!」
\par 14 做这事的,有犹太祭司长士基瓦的七个儿子。
\par 15 恶鬼回答他们说:「耶稣我认识,保罗我也知道。你们却是谁呢?」
\par 16 恶鬼所附的人就跳在他们身上,胜了其中二人,制伏他们,叫他们赤著身子受了伤,从那房子里逃出去了。
\par 17 凡住在以弗所的,无论是犹太人,是希利尼人,都知道这事,也都惧怕;主耶稣的名从此就尊大了。
\par 18 那已经信的,多有人来承认诉说自己所行的事。
\par 19 平素行邪术的,也有许多人把书拿来,堆积在众人面前焚烧。他们算计书价,便知道共合五万块钱。
\par 20 主的道大大兴旺,而且得胜,就是这样。
\par 21 这些事完了,保罗心里定意经过了马其顿、亚该亚,就往耶路撒冷去;又说:「我到了那里以後,也必须往罗马去看看。」
\par 22 於是从帮助他的人中打发提摩太、以拉都二人往马其顿去,自己暂时等在亚西亚。
\par 23 那时,因为这道起的扰乱不小。
\par 24 有一个银匠,名叫底米丢,是制造亚底米神银龛的,他使这样手艺人生意发达。
\par 25 他聚集他们和同行的工人,说:「众位,你们知道我们是倚靠这生意发财。
\par 26 这保罗不但在以弗所,也几乎在亚西亚全地,引诱迷惑许多人,说:『人手所做的,不是神。』这是你们所看见所听见的。
\par 27 这样,不独我们这事业被人藐视,就是大女神亚底米的庙也要被人轻忽,连亚西亚全地和普天下所敬拜的大女神之威荣也要消灭了。」
\par 28 众人听见,就怒气填胸,喊著说:「大哉,以弗所人的亚底米啊!」
\par 29 满城都轰动起来。众人拿住与保罗同行的马其顿人该犹和亚里达古,齐心拥进戏园里去。
\par 30 保罗想要进去,到百姓那里,门徒却不许他去。
\par 31 还有亚西亚几位首领,是保罗的朋友,打发人来劝他,不要冒险到戏园里去。
\par 32 聚集的人纷纷乱乱,有喊叫这个的,有喊叫那个的,大半不知道是为什麽聚集。
\par 33 有人把亚力山大从众人中带出来,犹太人推他往前,亚力山大就摆手,要向百姓分诉;
\par 34 只因他们认出他是犹太人,就大家同声喊著说:「大哉!以弗所人的亚底米啊。」如此约有两小时。
\par 35 那城里的书记安抚了众人,就说:「以弗所人哪,谁不知道以弗所人的城是看守大亚底米的庙和从丢斯那里落下来的像呢?
\par 36 这事既是驳不倒的,你们就当安静,不可造次。
\par 37 你们把这些人带来,他们并没有偷窃庙中之物,也没有谤 我们的女神。
\par 38 若是底米丢和他同行的人有控告人的事,自有放告的日子(或作:自有公堂),也有方伯可以彼此对告。
\par 39 你们若问别的事,就可以照常例聚集断定。
\par 40 今日的扰乱本是无缘无故,我们难免被查问。论到这样聚众,我们也说不出所以然来。」
\par 41 说了这话,便叫众人散去。

\chapter{20}

\par 1 乱定之後,保罗请门徒来,劝勉他们,就辞别起行,往马其顿去。
\par 2 走遍了那一带地方,用许多话劝勉门徒。(或作:众人),然後来到希利尼。
\par 3 在那里住了三个月,将要坐船往利亚去,犹太人设计要害他,他就定意从马其顿回去。
\par 4 同他到亚西亚去的,有庇哩亚人毕罗斯的儿子所巴特,帖撒罗尼迦人亚里达古和西公都,还有特庇人该犹,并提摩太,又有亚西亚人推基古和特罗非摩。
\par 5 这些人先走,在特罗亚等候我们。
\par 6 过了除酵的日子,我们从腓立比开船,五天到了特罗亚,和他们相会,在那里住了七天。
\par 7 七日的第一日,我们聚会擘饼的时候,保罗因为要次日起行,就与他们讲论,直讲到半夜。
\par 8 我们聚会的那座楼上,有好些灯烛。
\par 9 有一个少年人,名叫犹推古,坐在窗台上,困倦沉睡。保罗讲了多时,少年人睡熟了,就从三层楼上掉下去;扶起他来,已经死了。
\par 10 保罗下去,伏在他身上,抱著他,说:「你们不要发慌,他的灵魂还在身上。」
\par 11 保罗又上去,擘饼,吃了,谈论许久,直到天亮,这才走了。
\par 12 有人把那童子活活的领来,得的安慰不少。
\par 13 我们先上船,开往亚朔去,意思要在那里接保罗;因为他是这样安排的,他自己打算要步行。
\par 14 他既在亚朔与我们相会,我们就接他上船,来到米推利尼。
\par 15 从那里开船,次日到了基阿的对面;又次日,在撒摩靠岸;又次日,来到米利都。
\par 16 乃因保罗早已定意越过以弗所,免得在亚西亚耽延,他急忙前走,巴不得赶五旬节能到耶路撒冷。
\par 17 保罗从米利都打发人往以弗所去,请教会的长老来。
\par 18 他们来了,保罗就说:「你们知道,自从我到亚西亚的日子以来,在你们中间始终为人如何,
\par 19 服事主,凡事谦卑,眼中流泪,又因犹太人的谋害,经历试炼。
\par 20 你们也知道,凡与你们有益的,我没有一样避讳不说的,或在众人面前,或在各人家里,我都教导你们;
\par 21 又对犹太人和希利尼人证明当向神悔改,信靠我主耶稣基督。
\par 22 现在我往耶路撒冷去,心甚迫切(原文作心被捆绑),不知道在那里要遇见什麽事;
\par 23 但知道圣灵在各城里向我指证,说有捆锁与患难等待我。
\par 24 我却不以性命为念,也不看为宝贵,只要行完我的路程,成就我从主耶稣所领受的职事,证明神恩惠的福音。
\par 25 「我素常在你们中间来往,传讲神国的道;如今我晓得,你们以後都不得再见我的面了。
\par 26 所以我今日向你们证明,你们中间无论何人死亡,罪不在我身上(原文作我於众人的血是洁净的)。
\par 27 因为神的旨意,我并没有一样避讳不传给你们的。
\par 28 圣灵立你们作全群的监督,你们就当为自己谨慎,也为全群谨慎,牧养神的教会,就是他用自己血所买来的(或作:救赎的)。
\par 29 我知道,我去之後必有凶暴的豺狼进入你们中间,不爱惜羊群。
\par 30 就是你们中间,也必有人起来说悖谬的话,要引诱门徒跟从他们。
\par 31 所以你们应当警醒,记念我三年之久昼夜不住的流泪、劝戒你们各人。
\par 32 如今我把你们交托神和他恩惠的道;这道能建立你们,叫你们和一切成圣的人同得基业。
\par 33 我未曾贪图一个人的金、银、衣服。
\par 34 我这两只手常供给我和同人的需用,这是你们自己知道的。
\par 35 我凡事给你们作榜样,叫你们知道应当这样劳苦,扶助软弱的人,又当记念主耶稣的话,说:『施比受更为有福。』」
\par 36 保罗说完了这话,就跪下同众人祷告。
\par 37 众人痛哭,抱著保罗的颈项,和他亲嘴。
\par 38 叫他们最伤心的,就是他说:「以後不能再见我的面」那句话,於是送他上船去了。

\chapter{21}

\par 1 我们离别了众人,就开船一直行到哥士。第二天到了罗底,从那里到帕大喇,
\par 2 遇见一只船要往腓尼基去,就上船起行。
\par 3 望见居比路,就从南边行过,往利亚去,我们就在推罗上岸,因为船要在那里卸货。
\par 4 找著了门徒,就在那里住了七天。他们被圣灵感动,对保罗说:「不要上耶路撒冷去。」
\par 5 过了这几天,我们就起身前行。他们众人同妻子儿女,送我们到城外,我们都跪在岸上祷告,彼此辞别。
\par 6 我们上了船,他们就回家去了。
\par 7 我们从推罗行尽了水路,来到多利买,就问那里的弟兄安,和他们同住了一天。
\par 8 第二天,我们离开那里,来到该撒利亚,就进了传福音的腓利家里,和他同住。他是那七个执事里的一个。
\par 9 他有四个女儿,都是处女,是说预言的。
\par 10 我们在那里多住了几天,有一个先知,名叫亚迦布,从犹太下来,
\par 11 到了我们这里,就拿保罗的腰带捆上自己的手脚,说:「圣灵说:犹太人在耶路撒冷,要如此捆绑这腰带的主人,把他交在外邦人手里。」
\par 12 我们和那本地的人听见这话,都苦劝保罗不要上耶路撒冷去。
\par 13 保罗说:「你们为什麽这样痛哭,使我心碎呢?我为主耶稣的名,不但被人捆绑,就是死在耶路撒冷也是愿意的。」
\par 14 保罗既不听劝,我们便住了口,只说:「愿主的旨意成就,」便了。
\par 15 过了几日,我们收拾行李上耶路撒冷去。
\par 16 有该撒利亚的几个门徒和我们同去,带我们到一个久为(久为:或作老)门徒的家里,叫我们与他同住;他名叫拿孙,是居比路人。
\par 17 到了耶路撒冷,弟兄们欢欢喜喜的接待我们。
\par 18 第二天,保罗同我们去见雅各;长老们也都在那里。
\par 19 保罗问了他们安,便将神用他传教,在外邦人中间所行之事,一一的述说了。
\par 20 他们听见,就归荣耀与神,对保罗说:「兄台,你看犹太人中信主的有多少万,并且都为律法热心。
\par 21 他们听见人说:你教训一切在外邦的犹太人离弃摩西,对他们说:不要给孩子行割礼,也不要遵行条规。
\par 22 众人必听见你来了,这可怎麽办呢?
\par 23 你就照著我们的话行吧?我们这里有四个人,都有愿在身。
\par 24 你带他们去,与他们一同行洁净的礼,替他们拿出规费,叫他们得以剃头。这样,众人就可知道,先前所听见你的事都是虚的;并可知道,你自己为人,循规蹈矩,遵行律法。
\par 25 至於信主的外邦人,我们已经写信拟定,叫他们谨忌那祭偶像之物,和血,并勒死的牲畜,与奸淫。」
\par 26 於是保罗带著那四个人,第二天与他们一同行了洁净的礼,进了殿,报明洁净的日期满足,只等祭司为他们各人献祭。
\par 27 那七日将完,从亚西亚来的犹太人看见保罗在殿里,就耸动了众人,下手拿他,
\par 28 喊叫说:「以色列人来帮助,这就是在各处教训众人糟践我们百姓和律法,并这地方的。他又带著希利尼人进殿,污秽了这圣地。」
\par 29 (这话是因他们曾看见以弗所人特罗非摩同保罗在城里,以为保罗在带他进了殿。)
\par 30 合城都震动,百姓一齐跑来,拿住保罗,拉他出殿,殿门立刻都关了。
\par 31 他们正想要杀他,有人报信给营里的千夫长说:「耶路撒冷合城都乱了。」
\par 32 千夫长立时带著兵丁和几个百夫长,跑下去到他们那里。他们见了千夫长和兵丁,就止住不打保罗。
\par 33 於是千夫长上前拿住他,吩咐用两条铁链捆锁;又问他是什麽人,做的是什麽事。
\par 34 众人有喊叫这个的,有喊叫那个的;千夫长因为这样乱嚷,得不著实情,就吩咐人将保罗带进营楼去。
\par 35 到了台阶上,众人挤得凶猛,兵丁只得将保罗抬起来。
\par 36 众人跟在後面,喊著说:「除掉他!」
\par 37 将要带他进营楼,保罗对千夫长说:「我对你说句话可以不可以?」他说「:你懂得希利尼话吗?」
\par 38 你莫非是从前作乱、带领四千凶徒往旷野去的那埃及人吗?
\par 39 保罗说:「我本是犹太人,生在基利家的大数,并不是无名小城的人。求你准我对百姓说话。」
\par 40 千夫长准了。保罗就站在台阶上,向百姓摆手,他们都静默无声,保罗便用希伯来话对他们说:

\chapter{22}

\par 1 「诸位父兄请听,我现在对你们分诉。」
\par 2 众人听他说的是希伯来话,就更加安静了。
\par 3 保罗说:「我原是犹太人,生在基利家的大数,长在这城里,在迦玛列门下,按著我们祖宗严紧的律法受教,热心事奉神,像你们众人今日一样。
\par 4 我也曾逼迫奉这道的人,直到死地,无论男女都锁拿下监。
\par 5 这是大祭司和众长老都可以给我作见证的。我又领了他们达与弟兄的书信,往大马色去,要把在那里奉这道的人锁拿,带到耶路撒冷受刑。
\par 6 我将到大马色,正走的时候,约在晌午,忽然从天上发大光,四面照著我。
\par 7 我就仆倒在地,听见有声音对我说:『扫罗!扫罗!你为什麽逼迫我?』
\par 8 我回答说:『主啊,你是谁?』他说:『我就是你所逼迫的拿撒勒人耶稣。』
\par 9 与我同行的人看见了那光,却没有听明那位对我说话的声音。
\par 10 我说:『主啊,我当做什麽?』主说:『起来,进大马色去,在那里,要将所派你做的一切事告诉你。』
\par 11 我因那光的荣耀不能看见,同行的人就拉著我手进了大马色。
\par 12 那里有一个人,名叫亚拿尼亚,按著律法是虔诚人,为一切住在那里的犹太人所称赞。
\par 13 他来见我,站在旁边,对我说:『兄弟扫罗,你可以看见。』我当时往上一看,就看见了他。
\par 14 他又说:『我们祖宗的神拣选了你,叫你明白他的旨意,又得见那义者,听他口中所出的声音。
\par 15 因为你要将所看见的,所听见的,对著万人为他作见证。
\par 16 现在你为什麽耽延呢?起来,求告他的名受洗,洗去你的罪。』」
\par 17 「後来,我回到耶路撒冷,在殿里祷告的时候,魂游象外,
\par 18 看见主向我说:『你赶紧的离开耶路撒冷,不可迟延;因你为我作的见证,这里的人必不领受。』
\par 19 我就说:『主啊,他们知道我从前把信你的人收在监里,又在各会堂里鞭打他们。
\par 20 并且你的见证人司提反被害流血的时候,我也站在旁边欢喜;又看守害死他之人的衣裳。』
\par 21 主向我说:『你去吧!我要差你远远的往外邦人那里去。』」
\par 22 众人听他说到这句话,就高声说:「这样的人,从世上除掉他吧!他是不当活著的。」
\par 23 众人喧嚷,摔掉衣裳,把尘土向空中扬起来。
\par 24 千夫长就吩咐将保罗带进营楼去,叫人用鞭子拷问他,要知道他们向他这样喧嚷是为什麽缘故。
\par 25 刚用皮条捆上,保罗对旁边站著的百夫长说:「人是罗马人,又没有定罪,你们就鞭打他,有这个例吗?」
\par 26 百夫长听见这话,就去见千夫长,告诉他说:「你要作什麽?这人是罗马人。」
\par 27 千夫长就来问保罗说:「你告诉我,你是罗马人吗?」保罗说:「是。」
\par 28 千夫长说:「我用许多银子才入了罗马的民籍。」保罗说:「我生来就是。」
\par 29 於是那些要拷问保罗的人就离开他去了。千夫长既知道他是罗马人,又因为捆绑了他,也害怕了。
\par 30 第二天,千夫长为要知道犹太人控告保罗的实情,便解开他,吩咐祭司长和全公会的人都聚集,将保罗带下来,叫他站在他们面前。

\chapter{23}

\par 1 保罗定睛看著公会的人,说:「弟兄们,我在神面前行事为人都是凭著良心,直到今日。」
\par 2 大祭司亚拿尼亚就吩咐旁边站著的人打他的嘴。
\par 3 保罗对他说:「你这粉饰的墙,神要打你!你坐堂为的是按律法审问我,你竟违背律法,吩咐人打我吗?」
\par 4 站在旁边的人说:「你辱骂神的大祭司吗?」
\par 5 保罗说:「弟兄们,我不晓得他是大祭司;经上记著说:『不可毁谤你百姓的官长。』」
\par 6 保罗看出大众一半是撒都该人,一半是法利赛人,就在公会中大声说:「弟兄们,我是法利赛人,也是法利赛人的子孙。我现在受审问,是为盼望死人复活。」
\par 7 说了这话,法利赛人和撒都该人就争论起来,会众分为两党。
\par 8 因为撒都该人说,没有复活,也没有天使和鬼魂;法利赛人却说,两样都有。
\par 9 於是大大的喧嚷起来。有几个法利赛党的文士站起来争辩说:「我们看不出这人有什麽恶处,倘若有鬼魂或是天使对他说过话,怎麽样呢?」
\par 10 那时大起争吵,千夫长恐怕保罗被他们扯碎了,就吩咐兵丁下去,把他从众人当中抢出来,带进营楼去。
\par 11 当夜,主站在保罗旁边,说:「放心吧!你怎样在耶路撒冷为我作见证,也必怎样在罗马为我作见证。
\par 12 到了天亮,犹太人同谋起誓,说:「若不先杀保罗就不吃不喝。」
\par 13 这样同心起誓的有四十多人。
\par 14 他们来见祭司长和长老,说:「我们已经起了一个大誓,若不先杀保罗就不吃什麽。
\par 15 现在你们和公会要知会千夫长,叫他带下保罗到你们这里来,假作要详细察考他的事;我们已经预备好了,不等他来到跟前就杀他。」
\par 16 保罗的外甥听见他们设下埋伏,就来到营楼里告诉保罗。
\par 17 保罗请一个百夫长来,说:「你领这少年人去见千夫长,他有事告诉他。」
\par 18 於是把他领去见千夫长,说:「被囚的保罗请我到他那里,求我领这少年人来见你;他有事告诉你。」
\par 19 千夫长就拉著他的手,走到一旁,私下问他说:「你有什麽事告诉我呢?」
\par 20 他说:「犹太人已经约定,要求你明天带下保罗到公会里去,假作要详细查问他的事。
\par 21 你切不要随从他们;因为他们有四十多人埋伏,已经起誓说:若不先杀保罗就不吃不喝。现在预备好了,只等你应允。」
\par 22 於是千夫长打发少年人走,嘱咐他说:「不要告诉人你将这事报给我了。」
\par 23 千夫长便叫了两个百夫长来,说:「预备步兵二百,马兵七十,长枪手二百,今夜亥初往该撒利亚去;
\par 24 也要预备牲口叫保罗骑上,护送到巡抚腓力斯那里去。」
\par 25 千夫长又写了文书,
\par 26 大略说:「革老丢吕西亚,请巡抚腓力斯大人安。
\par 27 这人被犹太人拿住,将要杀害,我得知他是罗马人,就带兵丁下去救他出来。
\par 28 因要知道他们告他的缘故,我就带他下到他们的公会去,
\par 29 便查知他被告是因他们律法的辩论,并没有什麽该死该绑的罪名。
\par 30 後来有人把要害他的计谋告诉我,我就立时解他到你那里去,又吩咐告他的人在你面前告他。(有古卷加:愿你平安!)」
\par 31 於是,兵丁照所吩咐他们的,将保罗夜里带到安提帕底。
\par 32 第二天,让马兵护送,他们就回营楼去。
\par 33 马兵来到该撒利亚,把文书呈给巡抚,便叫保罗站在他面前。
\par 34 巡抚看了文书,问保罗是哪省的人,既晓得他是基利家人,
\par 35 就说:「等告你的人来到,我要细听你的事」;便吩咐人把他看守在希律的衙门里。

\chapter{24}

\par 1 过了五天,大祭司亚拿尼亚同几个长老,和一个辩士帖土罗下来,向巡抚控告保罗。
\par 2 保罗被提了来,帖土罗就告他说:
\par 3 「腓力斯大人,我们因你得以大享太平,并且这一国的弊病,因著你的先见得以更正了;我们随时随地满心感谢不尽。
\par 4 惟恐多说,你嫌烦絮,只求你宽容听我们说几句话。
\par 5 我们看这个人,如同瘟疫一般,是鼓动普天下众犹太人生乱的,又是拿撒勒教党里的一个头目,
\par 6 连圣殿他也想要污秽;我们把他捉住了。(有古卷在此有:要按我们的律法审问,
\par 7 不料千夫长吕西亚前来,甚是强横,从我们手中把他夺去,吩咐告他的人到你这里来。)
\par 8 你自己究问他,就可以知道我们告他的一切事了。」
\par 9 众犹太人也随著告他说:「事情诚然是这样。」
\par 10 巡抚点头叫保罗说话。他就说:「我知道你在这国里断事多年,所以我乐意为自己分诉。
\par 11 你查问就可以知道,从我上耶路撒冷礼拜到今日不过有十二天。
\par 12 他们并没有看见我在殿里,或是在会堂里,或是在城里,和人辩论,耸动众人。
\par 13 他们现在所告我的事并不能对你证实了。
\par 14 但有一件事,我向你承认,就是他们所称为异端的道,我正按著那道事奉我祖宗的神,又信合乎律法的和先知书上一切所记载的,
\par 15 并且靠著神,盼望死人,无论善恶,都要复活,就是他们自己也有这个盼望。
\par 16 我因此自己勉励,对神对人,常存无亏的良心。
\par 17 过了几年,我带著 济本国的捐项和供献的物上去。
\par 18 正献的时候,他们看见我在殿里已经洁净了,并没有聚众,也没有吵嚷,惟有几个从亚西亚来的犹太人。
\par 19 他们若有告我的事,就应当到你面前来告我。
\par 20 即或不然,这些人若看出我站在公会前,有妄为的地方,他们自己也可以说明。
\par 21 纵然有,也不过一句话,就是我站在他们中间大声说:『我今日在你们面前受审,是为死人复活的道理。』」
\par 22 腓力斯本是详细晓得这道,就支吾他们说:「且等千夫长吕西亚下来,我要审断你们的事。」
\par 23 於是吩咐百夫长看守保罗,并且宽待他,也不拦阻他的亲友来供给他。
\par 24 过了几天,腓力斯和他夫人犹太的女子土西拉一同来到,就叫了保罗来,听他讲论信基督耶稣的道。
\par 25 保罗讲论公义、节制,和将来的审判。腓力斯甚觉恐惧,说:「你暂且去吧,等我得便再叫你来。」
\par 26 腓力斯又指望保罗送他银钱,所以屡次叫他来,和他谈论。
\par 27 过了两年,波求非斯都接了腓力斯的任;腓力斯要讨犹太人的喜欢,就留保罗在监里。

\chapter{25}

\par 1 非斯都到了任,过了三天,就从该撒利亚上耶路撒冷去。
\par 2 祭司长和犹太人的首领向他控告保罗,
\par 3 又央告他,求他的情,将保罗提到耶路撒冷来,他们要在路上埋伏杀害他。
\par 4 非斯都却回答说:「保罗押在该撒利亚,我自己快要往那里去」;
\par 5 又说:「你们中间有权势的人与我一同下去,那人若有什麽不是,就可以告他。」
\par 6 非斯都在他们那里住了不过十天八天,就下该撒利亚去;第二天坐堂,吩咐将保罗提上来。
\par 7 保罗来了,那些从耶路撒冷下来的犹太人周围站著,将许多重大的事控告他,都是不能证实的。
\par 8 保罗分诉说:「无论犹太人的律法,或是圣殿,或是该撒,我都没有干犯。」
\par 9 但非斯都要讨犹太人的喜欢,就问保罗说:「你愿意上耶路撒冷去,在那里听我审断这事吗?」
\par 10 保罗说:「我站在该撒的堂前,这就是我应当受审的地方。我向犹太人并没有行过什麽不义的事,这也是你明明知道的。
\par 11 我若行了不义的事,犯了什麽该死的罪,就是死,我也不辞。他们所告我的事若都不实,就没有人可以把我交给他们。我要上告於该撒。」
\par 12 非斯都和议会商量了,就说:「你既上告於该撒,可以往该撒那里去。」
\par 13 过了些日子,亚基帕王和百尼基氏来到该撒利亚,问非斯都安。
\par 14 在那里住了多日,非斯都将保罗的事告诉王,说:「这里有一个人,是腓力斯留在监里的。
\par 15 我在耶路撒冷的时候,祭司长和犹太的长老将他的事禀报了我,求我定他的罪。
\par 16 我对他们说,无论什麽人,被告还没有和原告对质,未得机会分诉所告他的事,就先定他的罪,这不是罗马人的条例。
\par 17 及至他们都来到这里,我就不耽延,第二天便坐堂,吩咐把那人提上来。
\par 18 告他的人站著告他;所告的,并没有我所逆料的那等恶事。
\par 19 不过是有几样辩论,为他们自己敬鬼神的事,又为一个人名叫耶稣,是已经死了,保罗却说他是活著的。
\par 20 这些事当怎样究问,我心里作难,所以问他说:『你愿意上耶路撒冷去,在那里为这些事听审吗?』
\par 21 但保罗求我留下他,要听皇上审断,我就吩咐把他留下,等我解他到该撒那里去。」
\par 22 亚基帕对非斯都说:「我自己也愿听这人辩论。」非斯都说:「明天你可以听。」
\par 23 第二天,亚基帕和百尼基大张威势而来,同著众千夫长和城里的尊贵人进了公厅。非斯都吩咐一声,就有人将保罗带进来。
\par 24 非斯都说:「亚基帕王和在这里的诸位啊,你们看这人,就是一切犹太人,在耶路撒冷和这里,曾向我恳求、呼叫说:『不可容他再活著。』」
\par 25 但我查明他没有犯什麽该死的罪,并且他自己上告於皇帝,所以我定意把他解去。
\par 26 论到这人,我没有确实的事可以奏明主上。因此,我带他到你们面前,也特意带他到你亚基帕王面前,为要在查问之後有所陈奏。
\par 27 据我看来,解送囚犯,不指明他的罪案是不合理的。」

\chapter{26}

\par 1 亚基帕对保罗说:「准你为自己辩明。」於是保罗伸手分诉,说:
\par 2 「亚基帕王啊,犹太人所告我的一切事,今日得在你面前分诉,实为万幸;
\par 3 更可幸的,是你熟悉犹太人的规矩和他们的辩论;所以求你耐心听我。
\par 4 我从起初在本国的民中,并在耶路撒冷,自幼为人如何,犹太人都知道。
\par 5 他们若肯作见证就晓得,我从起初是按著我们教中最严紧的教门作了法利赛人。
\par 6 现在我站在这里受审,是因为指望神向我们祖宗所应许的;
\par 7 这应许,我们十二个支派,昼夜切切的事奉神,都指望得著。王啊,我被犹太人控告,就是因这指望。
\par 8 神叫死人复活,你们为什麽看作不可信的呢?
\par 9 从前我自己以为应当多方攻击拿撒勒人耶稣的名,
\par 10 我在耶路撒冷也曾这样行了。既从祭司长得了权柄,我就把许多圣徒囚在监里。他们被杀,我也出名定案。
\par 11 在各会堂,我屡次用刑强逼他们说亵渎的话,又分外恼恨他们,甚至追逼他们,直到外邦的城邑。」
\par 12 「那时,我领了祭司长的权柄和命令,往大马色去。
\par 13 王啊,我在路上,晌午的时候,看见从天发光,比日头还亮,四面照著我并与我同行的人。
\par 14 我们都仆倒在地,我就听见有声音用希伯来话向我说:『扫罗!扫罗!为什麽逼迫我?你用脚踢刺是难的!』
\par 15 我说:『主啊,你是谁?』主说:『我就是你所逼迫的耶稣。
\par 16 你起来站著,我特意向你显现,要派你作执事,作见证,将你所看见的事和我将要指示你的事证明出来。;
\par 17 我也要救你脱离百姓和外邦人的手。
\par 18 我差你到他们那里去,要叫他们的眼睛得开,从黑暗中归向光明,从撒但权下归向神;又因信我,得蒙赦罪,和一切成圣的人同得基业。』」
\par 19 「亚基帕王啊,我故此没有违背那从天上来的异象;
\par 20 先在大马色,後在耶路撒冷和犹太全地,以及外邦,劝勉他们应当悔改归向神,行事与悔改的心相称。
\par 21 因此,犹太人在殿里拿住我,想要杀我。
\par 22 然而我蒙神的帮助,直到今日还站得住,对著尊贵、卑贱、老幼作见证;所讲的并不外乎众先知和摩西所说将来必成的事,
\par 23 就是基督必须受害,并且因从死里复活,要首先把光明的道传给百姓和外邦人。」
\par 24 保罗这样分诉,非斯都大声说:「保罗,你癫狂了吧。你的学问太大,反叫你癫狂了!」
\par 25 保罗说:「非斯都大人,我不是癫狂,我说的乃是真实明白话。
\par 26 王也晓得这些事,所以我向王放胆直言,我深信这些事没有一件向王隐藏的,因都不是在背地里做的。
\par 27 亚基帕王啊,你信先知吗?我知道你是信的。」
\par 28 亚基帕对保罗说:「你想少微一劝,便叫我作基督徒啊(或作:你这样劝我,几乎叫我作基督徒了)!」
\par 29 保罗说:「无论是少劝是多劝,我向神所求的,不但你一个人,就是今天一切听我的,都要像我一样,只是不要像我有这些锁链。」
\par 30 於是,王和巡抚并百尼基与同坐的人都起来,
\par 31 退到里面,彼此谈论说:「这人并没有犯什麽该死该绑的罪。」
\par 32 亚基帕又对非斯都说:「这人若没有上告於该撒,就可以释放了。」

\chapter{27}

\par 1 非斯都既然定规了,叫我们坐船往义大利去,便将保罗和别的囚犯交给御营里的一个百夫长,名叫犹流。
\par 2 有一只亚大米田的船,要沿著亚西亚一带地方的海边走,我们就上了那船开行;有马其顿的帖撒罗尼迦人亚里达古和我们同去。
\par 3 第二天,到了西顿;犹流宽待保罗,准他往朋友那里去,受他们的照应。
\par 4 从那里又开船,因为风不顺,就贴著居比路背风岸行去。
\par 5 过了基利家、旁非利亚前面的海,就到了吕家的每拉。
\par 6 在那里,百夫长遇见一只亚力山大的船,要往义大利去,便叫我们上了那船。
\par 7 一连多日,船行得慢,仅仅来到革尼土的对面。因为被风拦阻,就贴著革哩底背风岸,从撒摩尼对面行过。
\par 8 我们沿岸行走,仅仅来到一个地方,名叫佳澳;离那里不远,有拉西亚城。
\par 9 走的日子多了,已经过了禁食的节期,行船又危险,保罗就劝众人说:
\par 10 「众位,我看这次行船,不但货物和船要受伤损,大遭破坏,连我们的性命也难保。」
\par 11 但百夫长信从掌船的和船主,不信从保罗所说的。
\par 12 且因在这海口过冬不便,船上的人就多半说:不如开船离开这地方,或者能到非尼基过冬。非尼基是革哩底的一个海口,一面朝东北,一面朝东南。
\par 13 这时,微微起了南风,他们以为得意,就起了锚,贴近革哩底行去。
\par 14 不多几时,狂风从岛上扑下来;那风名叫「友拉革罗。」
\par 15 船被风抓住,敌不住风,我们就任风刮去。
\par 16 贴著一个小岛的背风岸奔行,那岛名叫高大,在那里仅仅收住了小船。
\par 17 既然把小船拉上来,就用缆索捆绑船底,又恐怕在赛耳底沙滩上搁了浅,就落下篷来,任船飘去。
\par 18 我们被风浪逼得甚急,第二天众人就把货物抛在海里。
\par 19 到第三天,他们又亲手把船上的器具抛弃了。
\par 20 太阳和星辰多日不显露,又有狂风大浪催逼,我们得救的指望就都绝了。
\par 21 众人多日没有吃什麽,保罗就出来站在他们中间,说:「众位,你们本该听我的话,不离开革哩底,免得遭这样的伤损破坏。
\par 22 现在我还劝你们放心,你们的性命一个也不失丧,惟独失丧这船。
\par 23 因我所属所事奉的神,他的使者昨夜站在我旁边,说:
\par 24 『保罗,不要害怕,你必定站在该撒面前,并且与你同船的人,神都赐给你了。』
\par 25 所以众位可以放心,我信神他怎样对我说:事情也要怎样成就。
\par 26 只是我们必要撞在一个岛上。」
\par 27 到了第十四天夜间,船在亚底亚海飘来飘去。约到半夜,水手以为渐近旱地,
\par 28 就探深浅,探得有十二丈;稍往前行,又探深浅,探得有九丈。
\par 29 恐怕撞在石头上,就从船尾抛下四个锚,盼望天亮。
\par 30 水手想要逃出船去,把小船放在海里,假作要从船头抛锚的样子。
\par 31 保罗对百夫长和兵丁说:「这些人若不等在船上,你们必不能得救。」
\par 32 於是兵丁砍断小船的绳子,由他飘去。
\par 33 天渐亮的时候,保罗劝众人都吃饭,说:「你们悬望忍饿不吃什麽,已经十四天了。
\par 34 所以我劝你们吃饭,这是关乎你们救命的事;因为你们各人连一根头发也不至於损坏。」
\par 35 保罗说了这话,就拿著饼,在众人面前祝谢了神,擘开吃。
\par 36 於是他们都放下心,也就吃了。
\par 37 我们在船上的共有二百七十六个人。
\par 38 他们吃饱了,就把船上的麦子抛在海里,为要叫船轻一点。
\par 39 到了天亮,他们不认识那地方,但见一个海湾,有岸可登,就商议能把船拢进去不能。
\par 40 於是砍断缆索,弃锚在海里;同时也松开舵绳,拉起头篷,顺著风向岸行去。
\par 41 但遇著两水夹流的地方,就把船搁了浅;船头胶住不动,船尾被浪的猛力冲坏。
\par 42 兵丁的意思要把囚犯杀了,恐怕有 水脱逃的。
\par 43 但百夫长要救保罗,不准他们任意而行,就吩咐会 水的,跳下水去先上岸;
\par 44 其余的人可以用板子或船上的零碎东西上岸。这样,众人都得了救,上了岸。

\chapter{28}

\par 1 我们既已得救,才知道那岛名叫米利大。
\par 2 土人看待我们,有非常的情分;因为当时下雨,天气又冷,就生火接待我们众人。
\par 3 那时,保罗拾起一捆柴,放在火上,有一条毒蛇,因为热了出来,咬住他的手。
\par 4 土人看见那毒蛇悬在他手上,就彼此说:「这人必是个凶手,虽然从海里救上来,天理还不容他活著。」
\par 5 保罗竟把那毒蛇甩在火里,并没有受伤。
\par 6 土人想他必要肿起来,或是忽然仆倒死了;看了多时,见他无害,就转念,说:「他是个神。」
\par 7 离那地方不远,有田产是岛长部百流的;他接纳我们,尽情款待三日。
\par 8 当时,部百流的父亲患热病和痢疾躺著。保罗进去,为他祷告,按手在他身上,治好了他。
\par 9 从此,岛上其余的病人也来,得了医治。
\par 10 他们又多方的尊敬我们;到了开船的时候,也把我们所需用的送到船上。
\par 11 过了三个月,我们上了亚力山大的船往前行;这船以「丢斯双子」为记,是在那海岛过了冬的。
\par 12 到了叙拉古,我们停泊三日;
\par 13 又从那里绕行,来到利基翁。过了一天,起了南风,第二天就来到部丢利。
\par 14 在那里遇见弟兄们,请我们与他们同住了七天。这样,我们来到罗马。
\par 15 那里的弟兄们一听见我们的信息就出来,到亚比乌市和三馆地方迎接我们。保罗见了他们,就感谢神,放心壮胆。
\par 16 进了罗马城,(有古卷加:百夫长把众囚犯交给御营的统领,惟有)保罗蒙准和一个看守他的兵另住在一处。
\par 17 过了三天,保罗请犹太人的首领来。他们来了,就对他们说:「弟兄们,我虽没有作什麽事干犯本国的百姓和我们祖宗的规条,却被锁绑,从耶路撒冷解在罗马人的手里。
\par 18 他们审问了我,就愿意释放我;因为在我身上,并没有该死的罪。
\par 19 无奈犹太人不服,我不得已,只好上告於该撒,并非有什麽事要控告我本国的百姓。
\par 20 因此,我请你们来见面说话,我原为以色列人所指望的,被这链子捆锁。」
\par 21 他们说:「我们并没有接著从犹太来论你的信,也没有弟兄到这里来报给我们说你有什麽不好处。
\par 22 但我们愿意听你的意见如何;因为这教门,我们晓得是到处被毁谤的。」
\par 23 他们和保罗约定了日子,就有许多人到他的寓处来。保罗从早到晚,对他们讲论这事,证明神国的道,引摩西的律法和先知的书,以耶稣的事劝勉他们。
\par 24 他所说的话,有信的,有不信的。
\par 25 他们彼此不合,就散了;未散以先,保罗说了一句话,说:「圣灵藉先知以赛亚向你们祖宗所说的话是不错的。
\par 26 他说:你去告诉这百姓说:你们听是要听见,却不明白;看是要看见,却不晓得;
\par 27 因为这百姓油蒙了心,耳朵发沉,眼睛闭著;恐怕眼睛看见,耳朵听见,心里明白,回转过来,我就医治他们。
\par 28 所以你们当知道,神这救恩,如今传给外邦人,他们也必听受。」(有古卷在此有:
\par 29 保罗说了这话,犹太人议论纷纷的就走了。)
\par 30 保罗在自己所租的房子里住了足足两年。凡来见他的人,他全都接待,
\par 31 放胆传讲神国的道,将主耶稣基督的事教导人,并没有人禁止。


\end{document}