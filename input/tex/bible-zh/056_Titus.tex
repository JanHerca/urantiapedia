\begin{document}

\title{提多书}


\chapter{1}

\par 1 神的仆人,耶稣基督的使徒保罗,凭著神选民的信心与敬虔真理的知识,
\par 2 盼望那无谎言的神在万古之先所应许的永生,
\par 3 到了日期,藉著传扬的工夫把他的道显明了;这传扬的责任是按著神我们救主的命令交托了我。
\par 4 现在写信给提多,就是照著我们共信之道作我真儿子的。愿恩惠、平安从父神和我们的救主基督耶稣归与你。
\par 5 我从前留你在革哩底,是要你将那没有办完的事都办整齐了,又照我所吩咐你的,在各城设立长老。
\par 6 若有无可指责的人,只作一个妇人的丈夫,儿女也是信主的,没有人告他们是放荡不服约束的,就可以设立。
\par 7 监督既是神的管家,必须无可指责,不任性,不暴躁,不因酒滋事,不打人,不贪无义之财。
\par 8 乐意接待远人,好善,庄重,公平,圣洁自持。
\par 9 坚守所教真实的道理,就能将纯正的教训劝化人,又能把争辩的人驳倒了。
\par 10 因为有许多人不服约束,说虚空话欺哄人;那奉割礼的更是这样。
\par 11 这些人的口总要堵住。他们因贪不义之财,将不该教导的教导人,败坏人的全家。
\par 12 有革哩底人中的一个本地先知说:「革哩底人常说谎话,乃是恶兽,又馋又懒。」
\par 13 这个见证是真的。所以,你要严严的责备他们,使他们在真道上纯全无疵,
\par 14 不听犹太人荒渺的言语和离弃真道之人的诫命。
\par 15 在洁净的人,凡物都洁净;在污秽不信的人,什麽都不洁净,连心地和天良也都污秽了。
\par 16 他们说是认识神,行事却和他相背,本是可憎恶的,是悖逆的,在各样善事上是可废弃的。

\chapter{2}

\par 1 但你所讲的总要合乎那纯正的道理。
\par 2 劝老年人要有节制、端庄、自守,在信心、爱心、忍耐上都要纯全无疵。
\par 3 又劝老年妇人,举止行动要恭敬,不说谗言,不给酒作奴仆,用善道教训人,
\par 4 好指教少年妇人,爱丈夫,爱儿女,
\par 5 谨守,贞洁,料理家务,待人有恩,顺服自己的丈夫,免得神的道理被毁谤。
\par 6 又劝少年人要谨守。
\par 7 你自己凡事要显出善行的榜样,在教训上要正直、端庄,
\par 8 言语纯全,无可指责,叫那反对的人,既无处可说我们的不是,便自觉羞愧。
\par 9 劝仆人要顺服自己的主人,凡事讨他的喜欢,不可顶撞他,
\par 10 不可私拿东西,要显为忠诚,以致凡事尊荣我们救主神的道。
\par 11 因为神救众人的恩典已经显明出来,
\par 12 教训我们除去不敬虔的心和世俗的情欲,在今世自守、公义、敬虔度日,
\par 13 等候所盼望的福,并等候至大的神和(或作无和字)我们救主耶稣基督的荣耀显现。
\par 14 他为我们舍了自己,要赎我们脱离一切罪恶,又洁净我们,特作自己的子民,热心为善。
\par 15 这些事你要讲明,劝戒人,用各等权柄责备人,不可叫人轻看你。

\chapter{3}

\par 1 你要提醒众人,叫他们顺服作官的、掌权的,遵他的命,预备行各样的善事。
\par 2 不要毁谤,不要争竞,总要和平,向众人大显温柔。
\par 3 我们从前也是无知、悖逆、受迷惑、服事各样私欲和宴乐,常存恶毒(或作阴毒)嫉妒的心,是可恨的,又是彼此相恨。
\par 4 但到了神我们救主的恩慈和他向人所施的慈爱显明的时候,
\par 5 他便救了我们,并不是因我们自己所行的义,乃是照他的怜悯,藉著重生的洗和圣灵的更新。
\par 6 圣灵就是神藉著耶稣基督我们救主厚厚浇灌在我们身上的,
\par 7 好叫我们因他的恩得称为义,可以凭著永生的盼望成为後嗣。(或作可以凭著盼望承受永生)。
\par 8 这话是可信的。我也愿你把这些事切切实实的讲明,使那些已信神的人留心做正经事业(或作留心行善)。这都是美事,并且与人有益。
\par 9 要远避无知的辩论和家谱的空谈,以及分争,并因律法而起的争竞,因为这都是虚妄无益的。
\par 10 分门结党的人,警戒过一两次,就要弃绝他。
\par 11 因为知道这等人已经背道,犯了罪,自己明知不是,还是去做。
\par 12 我打发亚提马或是推基古到你那里去的时候,你要赶紧往尼哥波立去见我,因为我已经定意在那里过冬。
\par 13 你要赶紧给律师西纳和亚波罗送行,叫他们没有缺乏。
\par 14 并且我们的人要学习正经事业(或作要学习行善),预备所需用的,免得不结果子。
\par 15 同我在一处的人都问你安。请代问那些因有信心爱我们的人安。愿恩惠常与你们众人同在。


\end{document}