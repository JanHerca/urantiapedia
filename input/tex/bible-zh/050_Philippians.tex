\begin{document}

\title{腓立比书}


\chapter{1}

\par 1 基督耶稣的仆人保罗和提摩太写信给凡住腓立比、在基督耶稣里的众圣徒,和诸位监督,诸位执事。
\par 2 愿恩惠、平安从神我们的父并主耶稣基督归与你们!
\par 3 我每逢想念你们,就感谢我的神;
\par 4 每逢为你们众人祈求的时候,常是欢欢喜喜的祈求。
\par 5 因为从头一天直到如今,你们是同心合意的兴旺福音。
\par 6 我深信那在你们心里动了善工的,必成全这工,直到耶稣基督的日子。
\par 7 我为你们众人有这样的意念,原是应当的;因你们常在我心里,无论我是在捆锁之中,是辩明证实福音的时候,你们都与我一同得恩。
\par 8 我体会基督耶稣的心肠,切切的想念你们众人;这是神可以给我作见证的。
\par 9 我所祷告的,就是要你们的爱心在知识和各样见识上多而又多,
\par 10 使你们能分别是非(或作:喜爱那美好的事),作诚实无过的人,直到基督的日子;
\par 11 并靠著耶稣基督结满了仁义的果子,叫荣耀称赞归与神。
\par 12 弟兄们,我愿意你们知道,我所遭遇的事更是叫福音兴旺,
\par 13 以致我受的捆锁在御营全军和其余的人中,已经显明是为基督的缘故。
\par 14 并且那在主里的弟兄多半因我受的捆锁就笃信不疑,越发放胆传神的道,无所惧怕。
\par 15 有的传基督是出於嫉妒分争,也有的是出於好意。
\par 16 这一等是出於爱心,知道我是为辩明福音设立的;
\par 17 那一等传基督是出於结党,并不诚实,意思要加增我捆锁的苦楚。
\par 18 这有何妨呢?或是假意,或是真心,无论怎样,基督究竟被传开了。为此,我就欢喜,并且还要欢喜;
\par 19 因为我知道,这事藉著你们的祈祷和耶稣基督之灵的帮助,终必叫我得救。
\par 20 照著我所切慕、所盼望的,没有一事叫我羞愧。只要凡事放胆,无论是生是死,总叫基督在我身上照常显大。
\par 21 因我活著就是基督,我死了就有益处。
\par 22 但我在肉身活著,若成就我工夫的果子,我就不知道该挑选什麽。
\par 23 我正在两难之间,情愿离世与基督同在,因为这是好得无比的。
\par 24 然而,我在肉身活著,为你们更是要紧的。
\par 25 我既然这样深信,就知道仍要住在世间,且与你们众人同住,使你们在所信的道上又长进又喜乐,
\par 26 叫你们在基督耶稣里的欢乐,因我再到你们那里去,就越发加增。
\par 27 只要你们行事为人与基督的福音相称,叫我或来见你们,或不在你们那里,可以听见你们的景况,知道你们同有一个心志,站立得稳,为所信的福音齐心努力。
\par 28 凡事不怕敌人的惊吓,这是证明他们沉沦,你们得救都是出於神。
\par 29 因为你们蒙恩,不但得以信服基督,并要为他受苦。
\par 30 你们的争战,就与你们在我身上从前所看见、现在所听见的一样。

\chapter{2}

\par 1 所以,在基督里若有什麽劝勉,爱心有什麽安慰,圣灵有什麽交通,心中有什麽慈悲怜悯,
\par 2 你们就要意念相同,爱心相同,有一样的心思,有一样的意念,使我的喜乐可以满足。
\par 3 凡事不可结党,不可贪图虚浮的荣耀;只要存心谦卑,各人看别人比自己强。
\par 4 各人不要单顾自己的事,也要顾别人的事。
\par 5 你们当以基督耶稣的心为心:
\par 6 他本有神的形像,不以自己与神同等为强夺的;
\par 7 反倒虚己,取了奴仆的形像,成为人的样式;
\par 8 既有人的样子,就自己卑微,存心顺服,以至於死,且死在十字架上。
\par 9 所以,神将他升为至高,又赐给他那超乎万名之上的名,
\par 10 叫一切在天上的、地上的,和地底下的,因耶稣的名无不屈膝,
\par 11 无不口称「耶稣基督为主」,使荣耀归与父神。
\par 12 这样看来,我亲爱的弟兄,你们既是常顺服的,不但我在你们那里,就是我如今不在你们那里,更是顺服的,就当恐惧战兢做成你们得救的工夫。
\par 13 因为你们立志行事都是神在你们心里运行,为要成就他的美意。
\par 14 凡所行的,都不要发怨言,起争论,
\par 15 使你们无可指摘,诚实无伪,在这弯曲悖谬的世代作神无瑕疵的儿女。你们显在这世代中,好像明光照耀,
\par 16 将生命的道表明出来,叫我在基督的日子好夸我没有空跑,也没有徒劳。
\par 17 我以你们的信心为供献的祭物,我若被浇奠在其上,也是喜乐,并且与你们众人一同喜乐。
\par 18 你们也要照样喜乐,并且与我一同喜乐。
\par 19 我靠主耶稣指望快打发提摩太去见你们,叫我知道你们的事,心里就得著安慰。
\par 20 因为我没有别人与我同心,实在挂念你们的事。
\par 21 别人都求自己的事,并不求耶稣基督的事。
\par 22 但你们知道提摩太的明证;他兴旺福音,与我同劳,待我像儿子待父亲一样。
\par 23 所以,我一看出我的事要怎样了结,就盼望立刻打发他去;
\par 24 但我靠著主自信我也必快去。
\par 25 然而,我想必须打发以巴弗提到你们那里去。他是我的兄弟,与我一同做工,一同当兵,是你们所差遣的,也是供给我需用的。
\par 26 他很想念你们众人,并且极其难过,因为你们听见他病了。
\par 27 他实在是病了,几乎要死;然而神怜恤他,不但怜恤他,也怜恤我,免得我忧上加忧。
\par 28 所以我越发急速打发他去,叫你们再见他,就可以喜乐,我也可以少些忧愁。
\par 29 故此,你们要在主里欢欢乐乐的接待他,而且要尊重这样的人;
\par 30 因他为做基督的工夫,几乎至死,不顾性命,要补足你们供给我的不及之处。

\chapter{3}

\par 1 弟兄们,我还有话说,你们要靠主喜乐。我把这话再写给你们,於我并不为难,於你们却是妥当。
\par 2 应当防备犬类,防备作恶的,防备妄自行割的。
\par 3 因为真受割礼的,乃是我们这以神的灵敬拜、在基督耶稣里夸口、不靠著肉体的。
\par 4 其实,我也可以靠肉体;若是别人想他可以靠肉体,我更可以靠著了。
\par 5 我第八天受割礼;我是以色列族、便雅悯支派的人,是希伯来人所生的希伯来人。就律法说,我是法利赛人;
\par 6 就热心说,我是逼迫教会的;就律法上的义说,我是无可指摘的。
\par 7 只是我先前以为与我有益的,我现在因基督都当作有损的。
\par 8 不但如此,我也将万事当作有损的,因我以认识我主基督耶稣为至宝。我为他已经丢弃万事,看作粪土,为要得著基督;
\par 9 并且得以在他里面,不是有自己因律法而得的义,乃是有信基督的义,就是因信神而来的义,
\par 10 使我认识基督,晓得他复活的大能,并且晓得和他一同受苦,效法他的死,
\par 11 或者我也得以从死里复活。
\par 12 这不是说我已经得著了,已经完全了;我乃是竭力追求,或者可以得著基督耶稣所以得著我的(所以得著我的:或作所要我得的)。
\par 13 弟兄们,我不是以为自己已经得著了;我只有一件事,就是忘记背後,努力面前的,
\par 14 向著标竿直跑,要得神在基督耶稣里从上面召我来得的奖赏。
\par 15 所以我们中间,凡是完全人总要存这样的心;若在什麽事上存别样的心,神也必以此指示你们。
\par 16 然而,我们到了什麽地步,就当照著什麽地步行。
\par 17 弟兄们,你们要一同效法我,也当留意看那些照我们榜样行的人。
\par 18 因为有许多人行事是基督十字架的仇敌。我屡次告诉你们,现在又流泪的告诉你们:
\par 19 他们的结局就是沉沦;他们的神就是自己的肚腹。他们以自己的羞辱为荣耀,专以地上的事为念。
\par 20 我们却是天上的国民,并且等候救主,就是主耶稣基督从天上降临。
\par 21 他要按著那能叫万有归服自己的大能,将我们这卑贱的身体改变形状,和他自己荣耀的身体相似。

\chapter{4}

\par 1 我所亲爱、所想念的弟兄们,你们就是我的喜乐,我的冠冕。我亲爱的弟兄,你们应当靠主站立得稳。
\par 2 我劝友阿爹和循都基,要在主里同心。
\par 3 我也求你这真实同负一轭的,帮助这两个女人,因为他们在福音上曾与我一同劳苦;还有革利免,并其余和我一同做工的,他们的名字都在生命册上。
\par 4 你们要靠主常常喜乐。我再说,你们要喜乐。
\par 5 当叫众人知道你们谦让的心。主已经近了。
\par 6 应当一无挂虑,只要凡事藉著祷告、祈求,和感谢,将你们所要的告诉神。
\par 7 神所赐、出人意外的平安必在基督耶稣里保守你们的心怀意念。
\par 8 弟兄们,我还有未尽的话:凡是真实的、可敬的、公义的、清洁的、可爱的、有美名的,若有什麽德行,若有什麽称赞,这些事你们都要思念。
\par 9 你们在我身上所学习的,所领受的,所听见的,所看见的,这些事你们都要去行,赐平安的神就必与你们同在。
\par 10 我靠主大大的喜乐,因为你们思念我的心如今又发生;你们向来就思念我,只是没有机会。
\par 11 我并不是因缺乏说这话;我无论在什麽景况都可以知足,这是我已经学会了。
\par 12 我知道怎样处卑贱,也知道怎样处丰富;或饱足,或饥饿;或有余,或缺乏,随事随在,我都得了秘诀。
\par 13 我靠著那加给我力量的,凡事都能做。
\par 14 然而,你们和我同受患难原是美事。
\par 15 腓立比人哪,你们也知道我初传福音离了马其顿的时候,论到授受的事,除了你们以外,并没有别的教会供给我。
\par 16 就是我在帖撒罗尼迦,你们也一次两次的打发人供给我的需用。
\par 17 我并不求什麽馈送,所求的就是你们的果子渐渐增多,归在你们的账上。
\par 18 但我样样都有,并且有余。我已经充足,因我从以巴弗提受了你们的馈送,当作极美的香气,为神所收纳、所喜悦的祭物。
\par 19 我的神必照他荣耀的丰富,在基督耶稣里,使你们一切所需用的都充足。
\par 20 愿荣耀归给我们的父神,直到永永远远。阿们!
\par 21 请问在基督耶稣里的各位圣徒安。在我这里的众弟兄都问你们安。
\par 22 众圣徒都问你们安。在该撒家里的人特特的问你们安。
\par 23 愿主耶稣基督的恩常在你们心里!


\end{document}