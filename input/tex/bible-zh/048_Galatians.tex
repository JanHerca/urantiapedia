\begin{document}

\title{加拉太书}


\chapter{1}

\par 1 作使徒的保罗(不是由於人,也不是藉著人,乃是藉著耶稣基督,与叫他从死里复活的父神)
\par 2 和一切与我同在的众弟兄,写信给加拉太的各教会。
\par 3 愿恩惠、平安从父神与我们的主耶稣基督归与你们!
\par 4 基督照我们父神的旨意,为我们的罪舍己,要救我们脱离这罪恶的世代。
\par 5 但愿荣耀归於神,直到永永远远。阿们!
\par 6 我希奇你们这麽快离开那藉著基督之恩召你们的,去从别的福音。
\par 7 那并不是福音,不过有些人搅扰你们,要把基督的福音更改了。
\par 8 但无论是我们,是天上来的使者,若传福音给你们,与我们所传给你们的不同,他就应当被咒诅。
\par 9 我们已经说了,现在又说,若有人传福音给你们,与你们所领受的不同,他就应当被咒诅。
\par 10 我现在是要得人的心呢?还是要得神的心呢?我岂是讨人的喜欢吗?若仍旧讨人的喜欢,我就不是基督的仆人了。
\par 11 弟兄们,我告诉你们,我素来所传的福音不是出於人的意思。
\par 12 因为我不是从人领受的,也不是人教导我的,乃是从耶稣基督启示来的。
\par 13 你们听见我从前在犹太教中所行的事,怎样极力逼迫残害神的教会。
\par 14 我又在犹太教中,比我本国许多同岁的人更有长进,为我祖宗的遗传更加热心。
\par 15 然而,那把我从母腹里分别出来、又施恩召我的神,
\par 16 既然乐意将他儿子启示在我心里,叫我把他传在外邦人中,我就没有与属血气的人商量,
\par 17 也没有上耶路撒冷去见那些比我先作使徒的,惟独往亚拉伯去,後又回到大马色。
\par 18 过了三年,才上耶路撒冷去见矶法,和他同住了十五天。
\par 19 至於别的使徒,除了主的兄弟雅各,我都没有看见。
\par 20 我写给你们的不是谎话,这是我在神面前说的。
\par 21 以後我到了利亚和基利家境内。
\par 22 那时,犹太信基督的各教会都没有见过我的面。
\par 23 不过听说那从前逼迫我们的,现在传扬他原先所残害的真道。
\par 24 他们就为我的缘故,归荣耀给神。

\chapter{2}

\par 1 过了十四年,我同巴拿巴又上耶路撒冷去,并带著提多同去。
\par 2 我是奉启示上去的,把我在外邦人中所传的福音,对弟兄们陈说;却是背地里对那有名望之人说的,惟恐我现在,或是从前,徒然奔跑。
\par 3 但与我同去的提多,虽是希利尼人,也没有勉强他受割礼;
\par 4 因为有偷著引进来的假弟兄,私下窥探我们在基督耶稣里的自由,要叫我们作奴仆。
\par 5 我们就是一刻的工夫也没有容让顺服他们,为要叫福音的真理仍存在你们中间。
\par 6 至於那些有名望的,不论他是何等人,都与我无干。神不以外貌取人。那些有名望的,并没有加增我什麽,
\par 7 反倒看见了主托我传福音给那未受割礼的人,正如托彼得传福音给那受割礼的人。
\par 8 (那感动彼得、叫他为受割礼之人作使徒的,也感动我,叫我为外邦人作使徒;)
\par 9 又知道所赐给我的恩典,那称为教会柱石的雅各、矶法、约翰,就向我和巴拿巴用右手行相交之礼,叫我们往外邦人那里去,他们往受割礼的人那里去。
\par 10 只是愿意我们记念穷人;这也是我本来热心去行的。
\par 11 後来,矶法到了安提阿;因他有可责之处,我就当面抵挡他。
\par 12 从雅各那里来的人未到以先,他和外邦人一同吃饭,及至他们来到,他因怕奉割礼的人,就退去与外邦人隔开了。
\par 13 其余的犹太人也都随著他装假,甚至连巴拿巴也随夥装假。
\par 14 但我一看见他们行的不正,与福音的真理不合,就在众人面前对矶法说:「你既是犹太人,若随外邦人行事,不随犹太人行事,怎麽还勉强外邦人随犹太人呢?」
\par 15 我们这生来的犹太人,不是外邦的罪人;
\par 16 既知道人称义不是因行律法,乃是因信耶稣基督,连我们也信了基督耶稣,使我们因信基督称义,不因行律法称义;因为凡有血气的,没有一人因行律法称义。
\par 17 我们若求在基督里称义,却仍旧是罪人,难道基督是叫人犯罪的吗?断乎不是!
\par 18 我素来所拆毁的,若重新建造,这就证明自己是犯罪的人。
\par 19 我因律法,就向律法死了,叫我可以向神活著。
\par 20 我已经与基督同钉十字架,现在活著的不再是我,乃是基督在我里面活著;并且我如今在肉身活著,是因信神的儿子而活;他是爱我,为我舍己。
\par 21 我不废掉神的恩;义若是藉著律法得的,基督就是徒然死了。

\chapter{3}

\par 1 无知的加拉太人哪,耶稣基督钉十字架,已经活画在你们眼前,谁又迷惑了你们呢?
\par 2 我只要问你们这一件:你们受了圣灵,是因行律法呢?是因听信福音呢?
\par 3 你们既靠圣灵入门,如今还靠肉身成全吗?你们是这样的无知吗?
\par 4 你们受苦如此之多,都是徒然的吗?难道果真是徒然的吗?
\par 5 那赐给你们圣灵,又在你们中间行异能的,是因你们行律法呢?是因你们听信福音呢?
\par 6 正如「亚伯拉罕信神,这就算为他的义」。
\par 7 所以,你们要知道:那以信为本的人,就是亚伯拉罕的子孙。
\par 8 并且圣经既然预先看明,神要叫外邦人因信称义,就早已传福音给亚伯拉罕,说:「万国都必因你得福。」
\par 9 可见那以信为本的人和有信心的亚伯拉罕一同得福。
\par 10 凡以行律法为本的,都是被咒诅的;因为经上记著:「凡不常照律法书上所记一切之事去行的,就被咒诅。」
\par 11 没有一个人靠著律法在神面前称义,这是明显的;因为经上说,「义人必因信得生。」
\par 12 律法原不本乎信,只说:「行这些事的,就必因此活著。」
\par 13 基督既为我们受(原文作成)了咒诅,就赎出我们脱离律法的咒诅;因为经上记著:「凡挂在木头上都是被咒诅的。」
\par 14 这便叫亚伯拉罕的福,因基督耶稣可以临到外邦人,使我们因信得著所应许的圣灵。
\par 15 弟兄们,我且照著人的常话说:虽然是人的文约,若已经立定了,就没有能废弃或加增的。
\par 16 所应许的原是向亚伯拉罕和他子孙说的。神并不是说「众子孙」,指著许多人,乃是说「你那一个子孙」,指著一个人,就是基督。
\par 17 我是这麽说,神预先所立的约,不能被那四百三十年以後的律法废掉,叫应许归於虚空。
\par 18 因为承受产业,若本乎律法,就不本乎应许;但神是凭著应许把产业赐给亚伯拉罕。
\par 19 这样说来,律法是为什麽有的呢?原是为过犯添上的,等候那蒙应许的子孙来到,并且是藉天使经中保之手设立的。
\par 20 但中保本不是为一面作的;神却是一位。
\par 21 这样,律法是与神的应许反对吗?断乎不是!若曾传一个能叫人得生的律法,义就诚然本乎律法了。
\par 22 但圣经把众人都圈在罪里,使所应许的福因信耶稣基督,归给那信的人。
\par 23 但这因信得救的理还未来以先,我们被看守在律法之下,直圈到那将来的真道显明出来。
\par 24 这样,律法是我们训蒙的师傅,引我们到基督那里,使我们因信称义。
\par 25 但这因信得救的理既然来到,我们从此就不在师傅的手下了。
\par 26 所以,你们因信基督耶稣都是神的儿子。
\par 27 你们受洗归入基督的都是披戴基督了。
\par 28 并不分犹太人、希利尼人,自主的、为奴的,或男或女,因为你们在基督耶稣里都成为一了。
\par 29 你们既属乎基督,就是亚伯拉罕的後裔,是照著应许承受产业的了。

\chapter{4}

\par 1 我说那承受产业的,虽然是全业的主人,但为孩童的时候却与奴仆毫无分别,
\par 2 乃在师傅和管家的手下,直等他父亲预定的时候来到。
\par 3 我们为孩童的时候,受管於世俗小学之下,也是如此。
\par 4 及至时候满足,神就差遣他的儿子,为女子所生,且生在律法以下,
\par 5 要把律法以下的人赎出来,叫我们得著儿子的名分。
\par 6 你们既为儿子,神就差他儿子的灵进入你们(原文作我们)的心,呼叫:「阿爸!父!」
\par 7 可见,从此以後,你不是奴仆,乃是儿子了;既是儿子,就靠著神为後嗣。
\par 8 但从前你们不认识神的时候,是给那些本来不是神的作奴仆;
\par 9 现在你们既然认识神,更可说是被神所认识的,怎麽还要归回那懦弱无用的小学,情愿再给他作奴仆呢?
\par 10 你们谨守日子、月分、节期、年分。
\par 11 我为你们害怕、惟恐我在你们身上是枉费了工夫。
\par 12 弟兄们,我劝你们要像我一样,因为我也像你们一样。你们一点没有亏负我。
\par 13 你们知道我头一次传福音给你们,是因为身体有疾病。
\par 14 你们为我身体的缘故受试炼,没有轻看我,也没有厌弃我,反倒接待我,如同神的使者,如同基督耶稣。
\par 15 你们当日所夸的福气在那里呢?那时你们若能行,就是把自己的眼睛剜出来给我,也都情愿。这是我可以给你们作见证的。
\par 16 如今我将真理告诉你们,就成了你们的仇敌吗?
\par 17 那些人热心待你们,却不是好意,是要离间你们(原文作把你们关在外面),叫你们热心待他们。
\par 18 在善事上,常用热心待人原是好的,却不单我与你们同在的时候才这样。
\par 19 我小子啊,我为你们再受生产之苦,直等到基督成形在你们心里。
\par 20 我巴不得现今在你们那里,改换口气,因我为你们心里作难。
\par 21 你们这愿意在律法以下的人,请告诉我,你们岂没有听见律法吗?
\par 22 因为律法上记著,亚伯拉罕有两个儿子,一个是使女生的,一个是自主之妇人生的。
\par 23 然而,那使女所生的是按著血气生的;那自主之妇人所生的是凭著应许生的。
\par 24 这都是比方:那两个妇人就是两约。一约是出於西乃山,生子为奴,乃是夏甲。
\par 25 这夏甲二字是指著亚拉伯的西乃山,与现在的耶路撒冷同类,因耶路撒冷和他的儿女都是为奴的。
\par 26 但那在上的耶路撒冷是自主的,他是我们的母。
\par 27 因为经上记著:不怀孕、不生养的,你要欢乐;未曾经过产难的,你要高声欢呼;因为没有丈夫的,比有丈夫的儿女更多。
\par 28 弟兄们,我们是凭著应许作儿女,如同以撒一样。
\par 29 当时,那按著血气生的逼迫了那按著圣灵生的,现在也是这样。
\par 30 然而经上是怎麽说的呢?是说:「把使女和他儿子赶出去!因为使女的儿子不可与自主妇人的儿子一同承受产业。」
\par 31 弟兄们,这样看来,我们不是使女的儿女,乃是自主妇人的儿女了。

\chapter{5}

\par 1 基督释放了我们,叫我们得以自由。所以要站立得稳,不要再被奴仆的轭挟制。
\par 2 我保罗告诉你们,若受割礼,基督就与你们无益了。
\par 3 我再指著凡受割礼的人确实的说,他是欠著行全律法的债。
\par 4 你们这要靠律法称义的,是与基督隔绝,从恩典中坠落了。
\par 5 我们靠著圣灵,凭著信心,等候所盼望的义。
\par 6 原来在基督耶稣里,受割礼不受割礼全无功效,惟独使人生发仁爱的信心才有功效。
\par 7 你们向来跑得好,有谁拦阻你们,叫你们不顺从真理呢?
\par 8 这样的劝导不是出於那召你们的。
\par 9 一点面酵能使全团都发起来。
\par 10 我在主里很信你们必不怀别样的心;但搅扰你们的,无论是谁,必担当他的罪名。
\par 11 弟兄们,我若仍旧传割礼,为什麽还受逼迫呢?若是这样,那十字架讨厌的地方就没有了。
\par 12 恨不得那搅乱你们的人把自己割绝了。
\par 13 弟兄们,你们蒙召是要得自由,只是不可将你们的自由当作放纵情欲的机会,总要用爱心互相服事。
\par 14 因为全律法都包在「爱人如己」这一句话之内了。
\par 15 你们要谨慎,若相咬相吞,只怕要彼此消灭了。
\par 16 我说,你们当顺著圣灵而行,就不放纵肉体的情欲了。
\par 17 因为情欲和圣灵相争,圣灵和情欲相争,这两个是彼此相敌,使你们不能做所愿意做的。
\par 18 但你们若被圣灵引导,就不在律法以下。
\par 19 情欲的事都是显而易见的,就如奸淫、污秽、邪荡、
\par 20 拜偶像、邪术、仇恨、争竞、忌恨、恼怒、结党、分争、异端、
\par 21 嫉妒(有古卷加:凶杀二字)、醉酒、荒宴等类。我从前告诉你们,现在又告诉你们,行这样事的人必不能承受神的国。
\par 22 圣灵所结的果子,就是仁爱、喜乐、和平、忍耐、恩慈、良善、信实、
\par 23 温柔、节制。这样的事没有律法禁止。
\par 24 凡属基督耶稣的人,是已经把肉体连肉体的邪情私欲同钉在十字架上了。
\par 25 我们若是靠圣灵得生,就当靠圣灵行事。
\par 26 不要贪图虚名,彼此惹气,互相嫉妒。

\chapter{6}

\par 1 弟兄们,若有人偶然被过犯所胜,你们属灵的人就当用温柔的心把他挽回过来;又当自己小心,恐怕也被引诱。
\par 2 你们各人的重担要互相担当,如此,就完全了基督的律法。
\par 3 人若无有,自己还以为有,就是自欺了。
\par 4 各人应当察验自己的行为;这样,他所夸的就专在自己,不在别人了,
\par 5 因为各人必担当自己的担子。
\par 6 在道理上受教的,当把一切需用的供给施教的人。
\par 7 不要自欺,神是轻慢不得的。人种的是什麽,收的也是什麽。
\par 8 顺著情欲撒种的,必从情欲收败坏;顺著圣灵撒种的,必从圣灵收永生。
\par 9 我们行善,不可丧志;若不灰心,到了时候就要收成。
\par 10 所以,有了机会就当向众人行善,向信徒一家的人更当这样。
\par 11 请看我亲手写给你们的字是何等的大呢!
\par 12 凡希图外貌体面的人都勉强你们受割礼,无非是怕自己为基督的十字架受逼迫。
\par 13 他们那些受割礼的,连自己也不守律法;他们愿意你们受割礼,不过要藉著你们的肉体夸口。
\par 14 但我断不以别的夸口,只夸我们主耶稣基督的十字架;因这十字架,就我而论,世界已经钉在十字架上;就世界而论,我已经钉在十字架上。
\par 15 受割礼不受割礼都无关紧要,要紧的就是作新造的人。
\par 16 凡照此理而行的,愿平安、怜悯加给他们,和神的以色列民。
\par 17 从今以後,人都不要搅扰我,因为我身上带著耶稣的印记。
\par 18 弟兄们,愿我主耶稣基督的恩常在你们心里。阿们!


\end{document}