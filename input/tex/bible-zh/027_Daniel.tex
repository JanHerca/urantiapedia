\begin{document}

\title{但以理书}


\chapter{1}

\par 1 犹大王约雅敬在位第三年,巴比伦王尼布甲尼撒来到耶路撒冷,将城围困。
\par 2 主将犹大王约雅敬,并神殿中器皿的几分交付他手。他就把这器皿带到示拿地,收入他神的庙里,放在他神的库中。
\par 3 王吩咐太监长亚施毗拿,从以色列人的宗室和贵胄中带进几个人来,
\par 4 就是年少没有残疾、相貌俊美、通达各样学问、知识聪明俱备、足能侍立在王宫里的,要教他们迦勒底的文字言语。
\par 5 王派定将自己所用的膳和所饮的酒,每日赐他们一分,养他们三年。满了三年,好叫他们在王面前侍立。
\par 6 他们中间有犹大族的人:但以理、哈拿尼雅、米沙利、亚撒利雅。
\par 7 太监长给他们起名:称但以理为伯提沙撒,称哈拿尼雅为沙得拉,称米沙利为米煞,称亚撒利雅为亚伯尼歌。
\par 8 但以理却立志不以王的膳和王所饮的酒玷污自己,所以求太监长容他不玷污自己。
\par 9 神使但以理在太监长眼前蒙恩惠,受怜悯。
\par 10 太监长对但以理说:「我惧怕我主我王,他已经派定你们的饮食;倘若他见你们的面貌比你们同岁的少年人肌瘦,怎麽好呢?这样,你们就使我的头在王那里难保。」
\par 11 但以理对太监长所派管理但以理、哈拿尼雅、米沙利、亚撒利雅的委办说:
\par 12 「求你试试仆人们十天,给我们素菜吃,白水喝,
\par 13 然後看看我们的面貌和用王膳那少年人的面貌,就照你所看的待仆人吧!」
\par 14 委办便允准他们这件事,试看他们十天。
\par 15 过了十天,见他们的面貌比用王膳的一切少年人更加俊美肥胖。
\par 16 於是委办撤去派他们用的膳,饮的酒,给他们素菜吃。
\par 17 这四个少年人,神在各样文字学问(学问:原文作智慧)上赐给他们聪明知识;但以理又明白各样的异象和梦兆。
\par 18 尼布甲尼撒王预定带进少年人来的日期满了,太监长就把他们带到王面前。
\par 19 王与他们谈论,见少年人中无一人能比但以理、哈拿尼雅、米沙利、亚撒利雅,所以留他们在王面前侍立。
\par 20 王考问他们一切事,就见他们的智慧聪明比通国的术士和用法术的胜过十倍。
\par 21 到古列王元年,但以理还在。

\chapter{2}

\par 1 尼布甲尼撒在位第二年,他做了梦,心里烦乱,不能睡觉。
\par 2 王吩咐人将术士、用法术的、行邪术的,和迦勒底人召来,要他们将王的梦告诉王,他们就来站在王前。
\par 3 王对他们说:「我做了一梦,心里烦乱,要知道这是什麽梦。」
\par 4 迦勒底人用亚兰的言语对王说:「愿王万岁!请将那梦告诉仆人,仆人就可以讲解。」
\par 5 王回答迦勒底人说:「梦我已经忘了(或作:我已定命;八节同),你们若不将梦和梦的讲解告诉我,就必被凌迟,你们的房屋必成为粪堆;
\par 6 你们若将梦和梦的讲解告诉我,就必从我这里得赠品和赏赐,并大尊荣。现在你们要将梦和梦的讲解告诉我。」
\par 7 他们第二次对王说:「请王将梦告诉仆人,仆人就可以讲解。」
\par 8 王回答说:「我准知道你们是故意迟延,因为你们知道那梦我已经忘了。
\par 9 你们若不将梦告诉我,只有一法待你们;因为你们预备了谎言乱语向我说,要等候时势改变。现在你们要将梦告诉我,因我知道你们能将梦的讲解告诉我。」
\par 10 迦勒底人在王面前回答说:「世上没有人能将王所问的事说出来;因为没有君王、大臣、掌权的向术士,或用法术的,或迦勒底人问过这样的事。
\par 11 王所问的事甚难。除了不与世人同居的神明,没有人在王面前能说出来。」
\par 12 因此,王气忿忿地大发烈怒,吩咐灭绝巴比伦所有的哲士。
\par 13 於是命令发出,哲士将要见杀,人就寻找但以理和他的同伴,要杀他们。
\par 14 王的护卫长亚略出来,要杀巴比伦的哲士,但以理就用婉言回答他,
\par 15 向王的护卫长亚略说:「王的命令为何这样紧急呢?」亚略就将情节告诉但以理。
\par 16 但以理遂进去求王宽限,就可以将梦的讲解告诉王。
\par 17 但以理回到他的居所,将这事告诉他的同伴哈拿尼雅、米沙利、亚撒利雅,
\par 18 要他们祈求天上的神施怜悯,将这奥秘的事指明,免得但以理和他的同伴与巴比伦其余的哲士一同灭亡。
\par 19 这奥秘的事就在夜间异象中给但以理显明,但以理便称颂天上的神。
\par 20 但以理说:「神的名是应当称颂的!从亘古直到永远,因为智慧能力都属乎他。
\par 21 他改变时候、日期,废王,立王,将智慧赐与智慧人,将知识赐与聪明人。
\par 22 他显明深奥隐秘的事,知道暗中所有的,光明也与他同居。
\par 23 我列祖的神啊,我感谢你,赞美你,因你将智慧才能赐给我,允准我们所求的,把王的事给我们指明。」
\par 24 於是,但以理进去见亚略,就是王所派灭绝巴比伦哲士的,对他说:「不要灭绝巴比伦的哲士,求你领我到王面前,我要将梦的讲解告诉王。」
\par 25 亚略就急忙将但以理领到王面前,对王说:「我在被掳的犹大人中遇见一人,他能将梦的讲解告诉王。」
\par 26 王问称为伯提沙撒的但以理说:「你能将我所做的梦和梦的讲解告诉我吗?」
\par 27 但以理在王面前回答说:「王所问的那奥秘事,哲士、用法术的、术士、观兆的都不能告诉王;
\par 28 只有一位在天上的神能显明奥秘的事。他已将日後必有的事指示尼布甲尼撒王。你的梦和你在床上脑中的异象是这样:
\par 29 王啊,你在床上想到後来的事,那显明奥秘事的主把将来必有的事指示你。
\par 30 至於那奥秘的事显明给我,并非因我的智慧胜过一切活人,乃为使王知道梦的讲解和心里的思念。
\par 31 「王啊,你梦见一个大像,这像甚高,极其光耀,站在你面前,形状甚是可怕。
\par 32 这像的头是精金的,胸膛和膀臂是银的,肚腹和腰是铜的,
\par 33 腿是铁的,脚是半铁半泥的。
\par 34 你观看,见有一块非人手凿出来的石头打在这像半铁半泥的脚上,把脚砸碎;
\par 35 於是金、银、铜、铁、泥都一同砸得粉碎,成如夏天禾场上的糠秕,被风吹散,无处可寻。打碎这像的石头变成一座大山,充满天下。
\par 36 「这就是那梦。我们在王面前要讲解那梦。
\par 37 王啊,你是诸王之王。天上的神已将国度、权柄、能力、尊荣都赐给你。
\par 38 凡世人所住之地的走兽,并天空的飞鸟,他都交付你手,使你掌管这一切。你就是那金头。
\par 39 在你以後必另兴一国,不及於你;又有第三国,就是铜的,必掌管天下。
\par 40 第四国,必坚壮如铁,铁能打碎克制百物,又能压碎一切,那国也必打碎压制列国。
\par 41 你既见像的脚和脚指头,一半是窑匠的泥,一半是铁,那国将来也必分开。你既见铁与泥搀杂,那国也必有铁的力量。
\par 42 那脚指头,既是半铁半泥,那国也必半强半弱。
\par 43 你既见铁与泥搀杂,那国民也必与各种人搀杂,却不能彼此相合,正如铁与泥不能相合一样。
\par 44 当那列王在位的时候,天上的神必另立一国,永不败坏,也不归别国的人,却要打碎灭绝那一切国,这国必存到永远。
\par 45 你既看见非人手凿出来的一块石头从山而出,打碎金、银、铜、铁、泥,那就是至大的神把後来必有的事给王指明。这梦准是这样,这讲解也是确实的。」
\par 46 当时,尼布甲尼撒王俯伏在地,向但以理下拜,并且吩咐人给他奉上供物和香品。
\par 47 王对但以理说:「你既能显明这奥秘的事,你们的神诚然是万神之神、万王之主,又是显明奥秘事的。」
\par 48 於是王高抬但以理,赏赐他许多上等礼物,派他管理巴比伦全省,又立他为总理,掌管巴比伦的一切哲士。
\par 49 但以理求王,王就派沙得拉、米煞、亚伯尼歌管理巴比伦省的事务,只是但以理常在朝中侍立。

\chapter{3}

\par 1 尼布甲尼撒王造了一个金像,高六十肘,宽六肘,立在巴比伦省杜拉平原。
\par 2 尼布甲尼撒王差人将总督、钦差、巡抚、臬司、藩司、谋士、法官,和各省的官员都召了来,为尼布甲尼撒王所立的像行开光之礼。
\par 3 於是总督、钦差、巡抚、臬司、藩司、谋士、法官,和各省的官员都聚集了来,要为尼布甲尼撒王所立的像行开光之礼,就站在尼布甲尼撒所立的像前。
\par 4 那时传令的大声呼叫说:「各方、各国、各族(原文作舌:下同)的人哪,有令传与你们:
\par 5 你们一听见角、笛、琵琶、琴、瑟、笙,和各样乐器的声音,就当俯伏敬拜尼布甲尼撒王所立的金像。
\par 6 凡不俯伏敬拜的,必立时扔在烈火的窑中。」
\par 7 因此各方、各国、各族的人民一听见角、笛、琵琶、琴、瑟,和各样乐器的声音,就都俯伏敬拜尼布甲尼撒王所立的金像。
\par 8 那时,有几个迦勒底人进前来控告犹大人。
\par 9 他们对尼布甲尼撒王说:「愿王万岁!
\par 10 王啊,你曾降旨说,凡听见角、笛、琵琶、琴、瑟、笙,和各样乐器声音的都当俯伏敬拜金像。
\par 11 凡不俯伏敬拜的,必扔在烈火的窑中。
\par 12 现在有几个犹大人,就是王所派管理巴比伦省事务的沙得拉、米煞、亚伯尼歌;王啊,这些人不理你,不事奉你的神,也不敬拜你所立的金像。」
\par 13 当时,尼布甲尼撒冲冲大怒,吩咐人把沙得拉、米煞、亚伯尼歌带过来,他们就把那些人带到王面前。
\par 14 尼布甲尼撒问他们说:「沙得拉、米煞、亚伯尼歌,你们不事奉我的神,也不敬拜我所立的金像,是故意的吗?
\par 15 你们再听见角、笛、琵琶、琴、瑟、笙,和各样乐器的声音,若俯伏敬拜我所造的像,却还可以;若不敬拜,必立时扔在烈火的窑中,有何神能救你们脱离我手呢?」
\par 16 沙得拉、米煞、亚伯尼歌对王说:「尼布甲尼撒啊,这件事我们不必回答你;
\par 17 即便如此,我们所事奉的神能将我们从烈火的窑中救出来。王啊,他也必救我们脱离你的手;
\par 18 即或不然,王啊,你当知道我们决不事奉你的神,也不敬拜你所立的金像。」
\par 19 当时,尼布甲尼撒怒气填胸,向沙得拉、米煞、亚伯尼歌变了脸色,吩咐人把窑烧热,比寻常更加七倍;
\par 20 又吩咐他军中的几个壮士,将沙得拉、米煞、亚伯尼歌捆起来,扔在烈火的窑中。
\par 21 这三人穿著裤子、内袍、外衣,和别的衣服,被捆起来扔在烈火的窑中。
\par 22 因为王命紧急,窑又甚热,那抬沙得拉、米煞、亚伯尼歌的人都被火焰烧死。
\par 23 沙得拉、米煞、亚伯尼歌这三个人都被捆著落在烈火的窑中。
\par 24 那时,尼布甲尼撒王惊奇,急忙起来,对谋士说:「我捆起来扔在火里的不是三个人吗?」他们回答王说:「王啊,是。」
\par 25 王说:「看哪,我见有四个人,并没有捆绑,在火中游行,也没有受伤;那第四个的相貌好像神子。」
\par 26 於是,尼布甲尼撒就近烈火窑门,说:「至高神的仆人沙得拉、米煞、亚伯尼歌出来,上这里来吧!」沙得拉、米煞、亚伯尼歌就从火中出来了。
\par 27 那些总督、钦差、巡抚,和王的谋士一同聚集看这三个人,见火无力伤他们的身体,头发也没有烧焦,衣裳也没有变色,并没有火燎的气味。
\par 28 尼布甲尼撒说:「沙得拉、米煞、亚伯尼歌的神是应当称颂的!他差遣使者救护倚靠他的仆人,他们不遵王命,舍去己身,在他们神以外不肯事奉敬拜别神。
\par 29 现在我降旨,无论何方、何国、何族的人,谤 沙得拉、米煞、亚伯尼歌之神的,必被凌迟,他的房屋必成粪堆,因为没有别神能这样施行拯救。」
\par 30 那时王在巴比伦省,高升了沙得拉、米煞、亚伯尼歌。

\chapter{4}

\par 1 尼布甲尼撒王晓谕住在全地各方、各国、各族的人说:「愿你们大享平安!
\par 2 我乐意将至高的神向我所行的神迹奇事宣扬出来。
\par 3 他的神迹何其大!他的奇事何其盛!他的国是永远的;他的权柄存到万代!
\par 4 「我尼布甲尼撒安居在宫中,平顺在殿内。
\par 5 我做了一梦,使我惧怕。我在床上的思念,并脑中的异象,使我惊惶。
\par 6 所以我降旨召巴比伦的一切哲士到我面前,叫他们把梦的讲解告诉我。
\par 7 於是那些术士、用法术的、迦勒底人、观兆的都进来,我将那梦告诉了他们,他们却不能把梦的讲解告诉我。
\par 8 末後那照我神的名,称为伯提沙撒的但以理来到我面前,他里头有圣神的灵,我将梦告诉他说:
\par 9 『术士的领袖伯提沙撒啊,因我知道你里头有圣神的灵,什麽奥秘的事都不能使你为难。现在要把我梦中所见的异象和梦的讲解告诉我。』
\par 10 「我在床上脑中的异象是这样:我看见地当中有一棵树,极其高大。
\par 11 那树渐长,而且坚固,高得顶天,从地极都能看见,
\par 12 叶子华美,果子甚多,可作众生的食物;田野的走兽卧在荫下,天空的飞鸟宿在枝上;凡有血气的都从这树得食。
\par 13 「我在床上脑中的异象,见有一位守望的圣者从天而降。
\par 14 大声呼叫说:『伐倒这树!砍下枝子!摇掉叶子!抛散果子!使走兽离开树下,飞鸟躲开树枝。
\par 15 树却要留在地内,用铁圈和铜圈箍住,在田野的青草中让天露滴湿,使他与地上的兽一同吃草,
\par 16 使他的心改变,不如人心;给他一个兽心,使他经过七期(期:或作年;本章同)。
\par 17 这是守望者所发的命,圣者所出的令,好叫世人知道至高者在人的国中掌权,要将国赐与谁就赐与谁,或立极卑微的人执掌国权。』
\par 18 「这是我尼布甲尼撒王所做的梦。伯提沙撒啊,你要说明这梦的讲解;因为我国中的一切哲士都不能将梦的讲解告诉我,惟独你能,因你里头有圣神的灵。」
\par 19 於是称为伯提沙撒的但以理惊讶片时,心意惊惶。王说:「伯提沙撒啊,不要因梦和梦的讲解惊惶。」伯提沙撒回答说:「我主啊,愿这梦归与恨恶你的人,讲解归与你的敌人。
\par 20 你所见的树渐长,而且坚固,高得顶天,从地极都能看见;
\par 21 叶子华美,果子甚多,可作众生的食物;田野的走兽住在其下;天空的飞鸟宿在枝上。
\par 22 「王啊,这渐长又坚固的树就是你。你的威势渐长及天,你的权柄管到地极。
\par 23 王既看见一位守望的圣者从天而降,说:『将这树砍伐毁坏,树却要留在地内,用铁圈和铜圈箍住;在田野的青草中,让天露滴湿,使他与地上的兽一同吃草,直到经过七期。』
\par 24 「王啊,讲解就是这样:临到我主我王的事是出於至高者的命。
\par 25 你必被赶出离开世人,与野地的兽同居,吃草如牛,被天露滴湿,且要经过七期。等你知道至高者在人的国中掌权,要将国赐与谁就赐与谁。
\par 26 守望者既吩咐存留树,等你知道诸天掌权,以後你的国必定归你。
\par 27 王啊,求你悦纳我的谏言,以施行公义断绝罪过,以怜悯穷人除掉罪孽,或者你的平安可以延长。」
\par 28 这事都临到尼布甲尼撒王。
\par 29 过了十二个月,他游行在巴比伦王宫里(原文作上)。
\par 30 他说:「这大巴比伦不是我用大能大力建为京都,要显我威严的荣耀吗?」
\par 31 这话在王口中尚未说完,有声音从天降下,说:「尼布甲尼撒王啊,有话对你说,你的国位离开你了。
\par 32 你必被赶出离开世人,与野地的兽同居,吃草如牛,且要经过七期。等你知道至高者在人的国中掌权,要将国赐与谁就赐与谁。」
\par 33 当时这话就应验在尼布甲尼撒的身上,他被赶出离开世人,吃草如牛,身被天露滴湿,头发长长,好像鹰毛;指甲长长,如同鸟爪。
\par 34 日子满足,我尼布甲尼撒举目望天,我的聪明复归於我,我便称颂至高者,赞美尊敬活到永远的神。他的权柄是永有的;他的国存到万代。
\par 35 世上所有的居民都算为虚无;在天上的万军和世上的居民中,他都凭自己的意旨行事。无人能拦住他手,或问他说,你做什麽呢?
\par 36 那时,我的聪明复归於我,为我国的荣耀、威严,和光耀也都复归於我;并且我的谋士和大臣也来朝见我。我又得坚立在国位上,至大的权柄加增於我。
\par 37 现在我尼布甲尼撒赞美、尊崇、恭敬天上的王;因为他所做的全都诚实,他所行的也都公平。那行动骄傲的,他能降为卑。

\chapter{5}

\par 1 伯沙撒王为他的一千大臣设摆盛筵,与这一千人对面饮酒。
\par 2 伯沙撒欢饮之间,吩咐人将他父(或作:祖;下同)尼布甲尼撒从耶路撒冷殿中所掠的金银器皿拿来,王与大臣、皇后、妃嫔好用这器皿饮酒。
\par 3 於是他们把耶路撒冷神殿库房中所掠的金器皿拿来,王和大臣、皇后、妃嫔就用这器皿饮酒。
\par 4 他们饮酒,赞美金、银、铜、铁、木、石所造的神。
\par 5 当时,忽有人的指头显出,在王宫与灯台相对的粉墙上写字。王看见写字的指头
\par 6 就变了脸色,心意惊惶,腰骨好像脱节,双膝彼此相碰,
\par 7 大声吩咐将用法术的和迦勒底人并观兆的领进来,对巴比伦的哲士说,谁能读这文字,把讲解告诉我,他必身穿紫袍,项带金链,在我国中位列第三。
\par 8 於是王的一切哲士都进来,却不能读那文字,也不能把讲解告诉王。
\par 9 伯沙撒王就甚惊惶,脸色改变,他的大臣也都惊奇。
\par 10 太后(或作:皇后;下同)因王和他大臣所说的话,就进入宴宫,说:「愿王万岁!你心意不要惊惶,脸面不要变色。
\par 11 在你国中有一人,他里头有圣神的灵,你父在世的日子,这人心中光明,又有聪明智慧,好像神的智慧。你父尼布甲尼撒王,就是王的父,立他为术士、用法术的,和迦勒底人,并观兆的领袖。
\par 12 在他里头有美好的灵性,又有知识聪明,能圆梦,释谜语,解疑惑。这人名叫但以理,尼布甲尼撒王又称他为伯提沙撒,现在可以召他来,他必解明这意思。」
\par 13 但以理就被领到王前。王问但以理说:「你是被掳之犹大人中的但以理吗?就是我父王从犹大掳来的吗?
\par 14 我听说你里头有神的灵,心中光明,又有聪明和美好的智慧。
\par 15 现在哲士和用法术的都领到我面前,为叫他们读这文字,把讲解告诉我,无奈他们都不能把讲解说出来。
\par 16 我听说你善於讲解,能解疑惑;现在你若能读这文字,把讲解告诉我,就必身穿紫袍,项戴金链,在我国中位列第三。」
\par 17 但以理在王面前回答说:「你的赠品可以归你自己,你的赏赐可以归给别人;我却要为王读这文字,把讲解告诉王。
\par 18 王啊,至高的神曾将国位、大权、荣耀、威严赐与你父尼布甲尼撒;
\par 19 因神所赐他的大权,各方、各国、各族的人都在他面前战兢恐惧。他可以随意生杀,随意升降。
\par 20 但他心高气傲,灵也刚愎,甚至行事狂傲,就被革去王位,夺去荣耀。
\par 21 他被赶出离开世人,他的心变如兽心,与野驴同居,吃草如牛,身被天露滴湿,等他知道至高的神在人的国中掌权,凭自己的意旨立人治国。
\par 22 伯沙撒啊,你是他的儿子(或作:孙子),你虽知道这一切,你心仍不自卑,
\par 23 竟向天上的主自高,使人将他殿中的器皿拿到你面前,你和大臣、皇后、妃嫔用这器皿饮酒。你又赞美那不能看、不能听、无知无识、金、银、铜、铁、木、石所造的神,却没有将荣耀归与那手中有你气息,管理你一切行动的神。
\par 24 因此从神那里显出指头来写这文字。
\par 25 「所写的文字是:『弥尼,弥尼,提客勒,乌法珥新。』
\par 26 讲解是这样:弥尼,就是神已经数算你国的年日到此完毕。
\par 27 提客勒,就是你被称在天平里,显出你的亏欠。
\par 28 毗勒斯(与乌法珥新同义),就是你的国分裂,归与玛代人和波斯人。」
\par 29 伯沙撒下令,人就把紫袍给但以理穿上,把金链给他戴在颈项上,又传令使他在国中位列第三。
\par 30 当夜,迦勒底王伯沙撒被杀。
\par 31 玛代人大利乌年六十二岁,取了迦勒底国。

\chapter{6}

\par 1 大利乌随心所愿,立一百二十个总督,治理通国。
\par 2 又在他们以上立总长三人(但以理在其中),使总督在他们三人面前回覆事务,免得王受亏损。
\par 3 因这但以理有美好的灵性,所以显然超乎其余的总长和总督,王又想立他治理通国。
\par 4 那时,总长和总督寻找但以理误国的把柄,为要参他;只是找不著他的错误过失,因他忠心办事,毫无错误过失。
\par 5 那些人便说:「我们要找参这但以理的把柄,除非在他神的律法中就寻不著。」
\par 6 於是,总长和总督纷纷聚集来见王,说:「愿大利乌王万岁!
\par 7 国中的总长、钦差、总督、谋士,和巡抚彼此商议,要立一条坚定的禁令(或作:求王下旨要立一条云云),三十日内,不拘何人,若在王以外,或向神或向人求什麽,就必扔在狮子坑中。
\par 8 王啊,现在求你立这禁令,加盖玉玺,使禁令决不更改;照玛代和波斯人的例是不可更改的。」
\par 9 於是大利乌王立这禁令,加盖玉玺。
\par 10 但以理知道这禁令盖了玉玺,就到自己家里(他楼上的窗户开向耶路撒冷),一日三次,双膝跪在他神面前,祷告感谢,与素常一样。
\par 11 那些人就纷纷聚集,见但以理在他神面前祈祷恳求。
\par 12 他们便进到王前,提王的禁令,说:「王啊,三十日内不拘何人,若在王以外,或向神或向人求什麽,必被扔在狮子坑中。王不是在这禁令上盖了玉玺吗?」王回答说:「实有这事,照玛代和波斯人的例是不可更改的。」
\par 13 他们对王说:「王啊,那被掳之犹大人中的但以理不理你,也不遵你盖了玉玺的禁令,他竟一日三次祈祷。」
\par 14 王听见这话,就甚愁烦,一心要救但以理,筹划解救他,直到日落的时候。
\par 15 那些人就纷纷聚集来见王,说:「王啊,当知道玛代人和波斯人有例,凡王所立的禁令和律例都不可更改。」
\par 16 王下令,人就把但以理带来,扔在狮子坑中。王对但以理说:「你所常事奉的神,他必救你。」
\par 17 有人搬石头放在坑口,王用自己的玺和大臣的印,封闭那坑,使惩办但以理的事毫无更改。
\par 18 王回宫,终夜禁食,无人拿乐器到他面前,并且睡不著觉。
\par 19 次日黎明,王就起来,急忙往狮子坑那里去。
\par 20 临近坑边,哀声呼叫但以理,对但以理说:「永生神的仆人但以理啊,你所常事奉的神能救你脱离狮子吗?」
\par 21 但以理对王说:「愿王万岁!
\par 22 我的神差遣使者,封住狮子的口,叫狮子不伤我;因我在神面前无辜,我在王面前也没有行过亏损的事。」
\par 23 王就甚喜乐,吩咐人将但以理从坑里系上来。於是但以理从坑里被系上来,身上毫无伤损,因为信靠他的神。
\par 24 王下令,人就把那些控告但以理的人,连他们的妻子儿女都带来,扔在狮子坑中。他们还没有到坑底,狮子就抓住(原文作胜了)他们,咬碎他们的骨头。
\par 25 那时,大利乌王传旨,晓谕住在全地各方、各国、各族的人说:「愿你们大享平安!
\par 26 现在我降旨晓谕我所统辖的全国人民,要在但以理的神面前,战兢恐惧。因为他是永远长存的活神,他的国永不败坏;他的权柄永存无极!
\par 27 他护庇人,搭救人,在天上地下施行神迹奇事,救了但以理脱离狮子的口。」
\par 28 如此,这但以理,当大利乌王在位的时候和波斯王古列在位的时候,大享亨通。

\chapter{7}

\par 1 巴比伦王伯沙撒元年,但以理在床上做梦,见了脑中的异象,就记录这梦,述说其中的大意。
\par 2 但以理说:我夜里见异象,看见天的四风陡起,刮在大海之上。
\par 3 有四个大兽从海中上来,形状各有不同:
\par 4 头一个像狮子,有鹰的翅膀;我正观看的时候,兽的翅膀被拔去,兽从地上得立起来,用两脚站立,像人一样,又得了人心。
\par 5 又有一兽如熊,就是第二兽,旁跨而坐,口齿内衔著三根肋骨。有吩咐这兽的说:「起来吞吃多肉。」
\par 6 此後我观看,又有一兽如豹,背上有鸟的四个翅膀;这兽有四个头,又得了权柄。
\par 7 其後我在夜间的异象中观看,见第四兽甚是可怕,极其强壮,大有力量,有大铁牙,吞吃嚼碎,所剩下的用脚践踏。这兽与前三兽大不相同,头有十角。
\par 8 我正观看这些角,见其中又长起一个小角;先前的角中有三角在这角前,连根被他拔出来。这角有眼,像人的眼,有口说夸大的话。
\par 9 我观看,见有宝座设立,上头坐著亘古常在者。他的衣服洁白如雪,头发如纯净的羊毛。宝座乃火焰,其轮乃烈火。
\par 10 从他面前有火,像河发出;事奉他的有千千,在他面前侍立的有万万;他坐著要行审判,案卷都展开了。
\par 11 那时我观看,见那兽因小角说夸大话的声音被杀,身体损坏,扔在火中焚烧。
\par 12 其余的兽,权柄都被夺去,生命却仍存留,直到所定的时候和日期。
\par 13 我在夜间的异象中观看,见有一位像人子的,驾著天云而来,被领到亘古常在者面前,
\par 14 得了权柄、荣耀、国度,使各方、各国、各族的人都事奉他。他的权柄是永远的,不能废去;他的国必不败坏。
\par 15 至於我但以理,我的灵在我里面愁烦,我脑中的异象使我惊惶。
\par 16 我就近一位侍立者,问他这一切的真情。他就告诉我,将那事的讲解给我说明:
\par 17 这四个大兽就是四王将要在世上兴起。
\par 18 然而,至高者的圣民,必要得国享受,直到永永远远。
\par 19 那时我愿知道第四兽的真情,他为何与那三兽的真情大不相同,甚是可怕,有铁牙铜爪,吞吃嚼碎,所剩下的用脚践踏;
\par 20 头有十角和那另长的一角,在这角前有三角被他打落。这角有眼,有说夸大话的口,形状强横,过於他的同类。
\par 21 我观看,见这角与圣民争战,胜了他们。
\par 22 直到亘古常在者来给至高者的圣民伸冤,圣民得国的时候就到了。
\par 23 那侍立者这样说:第四兽就是世上必有的第四国,与一切国大不相同,必吞吃全地,并且践踏嚼碎。
\par 24 至於那十角,就是从这国中必兴起的十王,後来又兴起一王,与先前的不同;他必制伏三王。
\par 25 他必向至高者说夸大的话,必折磨至高者的圣民,必想改变节期和律法。圣民必交付他手一载、二载、半载。
\par 26 然而,审判者必坐著行审判;他的权柄必被夺去,毁坏,灭绝,一直到底。
\par 27 国度、权柄,和天下诸国的大权必赐给至高者的圣民。他的国是永远的;一切掌权的都必事奉他,顺从他。
\par 28 那事至此完毕。至於我但以理,心中甚是惊惶,脸色也改变了,却将那事存记在心。

\chapter{8}

\par 1 伯沙撒王在位第三年,有异象现与我但以理,是在先前所见的异象之後。
\par 2 我见了异象的时候,我以为在以拦省书珊城(或作:宫)中;我见异象又如在乌莱河边。
\par 3 我举目观看,见有双角的公绵羊站在河边,两角都高。这角高过那角,更高的是後长的。
\par 4 我见那公绵羊往西、往北、往南抵触。兽在他面前都站立不住,也没有能救护脱离他手的;但他任意而行,自高自大。
\par 5 我正思想的时候,见有一只公山羊从西而来,遍行全地,脚不沾尘。这山羊两眼当中有一非常的角。
\par 6 他往我所看见、站在河边有双角的公绵羊那里去,大发忿怒,向他直闯。
\par 7 我见公山羊就近公绵羊,向他发烈怒,抵触他,折断他的两角。绵羊在他面前站立不住;他将绵羊触倒在地,用脚践踏,没有能救绵羊脱离他手的。
\par 8 这山羊极其自高自大,正强盛的时候,那大角折断了,又在角根上向天的四方(原文作风)长出四个非常的角来。
\par 9 四角之中有一角长出一个小角,向南、向东、向荣美之地,渐渐成为强大。
\par 10 他渐渐强大,高及天象,将些天象和星宿抛落在地,用脚践踏。
\par 11 并且他自高自大,以为高及天象之君;除掉常献给君的燔祭,毁坏君的圣所。
\par 12 因罪过的缘故,有军旅和常献的燔祭交付他。他将真理抛在地上,任意而行,无不顺利。
\par 13 我听见有一位圣者说话,又有一位圣者问那说话的圣者说:「这除掉常献的燔祭和施行毁坏的罪过,将圣所与军旅(或作:以色列的军)践踏的异象,要到几时才应验呢?」
\par 14 他对我说:「到二千三百日,圣所就必洁净。」
\par 15 我但以理见了这异象,愿意明白其中的意思。忽有一位形状像人的站在我面前。
\par 16 我又听见乌莱河两岸中有人声呼叫说:「加百列啊,要使此人明白这异象。」
\par 17 他便来到我所站的地方。他一来,我就惊慌俯伏在地;他对我说:「人子啊,你要明白,因为这是关乎末後的异象。」
\par 18 他与我说话的时候,我面伏在地沉睡;他就摸我,扶我站起来,
\par 19 说:「我要指示你恼怒临完必有的事,因为这是关乎末後的定期。
\par 20 你所看见双角的公绵羊,就是玛代和波斯王。
\par 21 那公山羊就是希利尼王(希利尼:原文作雅完;下同);两眼当中的大角就是头一王。
\par 22 至於那折断了的角,在其根上又长出四角,这四角就是四国,必从这国里兴起来,只是权势都不及他。
\par 23 这四国末时,犯法的人罪恶满盈,必有一王兴起,面貌凶恶,能用双关的诈语。
\par 24 他的权柄必大,却不是因自己的能力;他必行非常的毁灭,事情顺利,任意而行;又必毁灭有能力的和圣民。
\par 25 他用权术成就手中的诡计,心里自高自大,在人坦然无备的时候,毁灭多人;又要站起来攻击万君之君,至终却非因人手而灭亡。
\par 26 所说二千三百日的异象是真的,但你要将这异象封住,因为关乎後来许多的日子。」
\par 27 於是我但以理昏迷不醒,病了数日,然後起来办理王的事务。我因这异象惊奇,却无人能明白其中的意思。

\chapter{9}

\par 1 玛代族亚哈随鲁的儿子大利乌立为迦勒底国的王元年,
\par 2 就是他在位第一年,我但以理从书上得知耶和华的话临到先知耶利米,论耶路撒冷荒凉的年数,七十年为满。
\par 3 我便禁食,披麻蒙灰,定意向主神祈祷恳求。
\par 4 我向耶和华我的神祈祷、认罪,说:「主啊,大而可畏的神,向爱主、守主诫命的人守约施慈爱。
\par 5 我们犯罪作孽,行恶叛逆,偏离你的诫命典章,
\par 6 没有听从你仆人众先知奉你名向我们君王、首领、列祖,和国中一切百姓所说的话。
\par 7 主啊,你是公义的,我们是脸上蒙羞的;因我们犹大人和耶路撒冷的居民,并以色列众人,或在近处,或在远处,被你赶到各国的人,都得罪了你,正如今日一样。
\par 8 主啊,我们和我们的君王、首领、列祖因得罪了你,就都脸上蒙羞。
\par 9 主我们的神是怜悯饶恕人的,我们却违背了他,
\par 10 也没有听从耶和华我们神的话,没有遵行他藉仆人众先知向我们所陈明的律法。
\par 11 以色列众人都犯了你的律法,偏行,不听从你的话;因此,在你仆人摩西律法上所写的咒诅和誓言都倾在我们身上,因我们得罪了神。
\par 12 他使大灾祸临到我们,成就了警戒我们和审判我们官长的话;原来在普天之下未曾行过像在耶路撒冷所行的。
\par 13 这一切灾祸临到我们身上是照摩西律法上所写的,我们却没有求耶和华我们神的恩典,使我们回头离开罪孽,明白你的真理。
\par 14 所以耶和华留意使这灾祸临到我们身上,因为耶和华我们的神在他所行的事上都是公义;我们并没有听从他的话。
\par 15 主我们的神啊,你曾用大能的手领你的子民出埃及地,使自己得了名,正如今日一样。我们犯了罪,作了恶。
\par 16 主啊,求你按你的大仁大义,使你的怒气和忿怒转离你的城耶路撒冷,就是你的圣山。耶路撒冷和你的子民,因我们的罪恶和我们列祖的罪孽被四围的人羞辱。
\par 17 我们的神啊,现在求你垂听仆人的祈祷恳求,为自己使脸光照你荒凉的圣所。
\par 18 我的神啊,求你侧耳而听,睁眼而看,眷顾我们荒凉之地和称为你名下的城。我们在你面前恳求,原不是因自己的义,乃因你的大怜悯。
\par 19 求主垂听,求主赦免,求主应允而行,为你自己不要迟延。我的神啊,因这城和这民都是称为你名下的。」
\par 20 我说话,祷告,承认我的罪和本国之民以色列的罪,为我神的圣山,在耶和华我神面前恳求。
\par 21 我正祷告的时候,先前在异象中所见的那位加百列,奉命迅速飞来,约在献晚祭的时候,按手在我身上。
\par 22 他指教我说:「但以理啊,现在我出来要使你有智慧,有聪明。
\par 23 你初恳求的时候,就发出命令,我来告诉你,因你大蒙眷爱;所以你要思想明白这以下的事和异象。
\par 24 「为你本国之民和你圣城,已经定了七十个七。要止住罪过,除净罪恶,赎尽罪孽,引进(或作:彰显)永义,封住异象和预言,并膏至圣者(者:或作所)。
\par 25 你当知道,当明白,从出令重新建造耶路撒冷,直到有受膏君的时候,必有七个七和六十二个七。正在艰难的时候,耶路撒冷城连街带濠都必重新建造。
\par 26 过了六十二个七,那(或作:有)受膏者必被剪除,一无所有;必有一王的民来毁灭这城和圣所,至终必如洪水冲没。必有争战,一直到底,荒凉的事已经定了。
\par 27 一七之内,他必与许多人坚定盟约;一七之半,他必使祭祀与供献止息。那行毁坏可憎的(或作:使地荒凉的)如飞而来,并且有忿怒倾在那行毁坏的身上(或作:倾在那荒凉之地),直到所定的结局。」

\chapter{10}

\par 1 波斯王古列第三年,有事显给称为伯提沙撒的但以理。这事是真的,是指著大争战;但以理通达这事,明白这异象。
\par 2 当那时,我但以理悲伤了三个七日。
\par 3 美味我没有吃,酒肉没有入我的口,也没有用油抹我的身,直到满了三个七日。
\par 4 正月二十四日,我在希底结大河边,
\par 5 举目观看,见有一人身穿细麻衣,腰束乌法精金带。
\par 6 他身体如水苍玉,面貌如闪电,眼目如火把,手和脚如光明的铜,说话的声音如大众的声音。
\par 7 这异象惟有我但以理一人看见,同著我的人没有看见。他们却大大战兢,逃跑隐藏,
\par 8 只剩下我一人。我见了这大异象便浑身无力,面貌失色,毫无气力。
\par 9 我却听见他说话的声音,一听见就面伏在地沉睡了。
\par 10 忽然,有一手按在我身上,使我用膝和手掌支持微起。
\par 11 他对我说:「大蒙眷爱的但以理啊,要明白我与你所说的话,只管站起来,因为我现在奉差遣来到你这里。」他对我说这话,我便战战兢兢地立起来。
\par 12 他就说:「但以理啊,不要惧怕!因为从你第一日专心求明白将来的事,又在你神面前刻苦己心,你的言语已蒙应允;我是因你的言语而来。
\par 13 但波斯国的魔君拦阻我二十一日。忽然有大君(就是天使长;二十一节同)中的一位米迦勒来帮助我,我就停留在波斯诸王那里。
\par 14 现在我来,要使你明白本国之民日後必遭遇的事,因为这异象关乎後来许多的日子。」
\par 15 他向我这样说,我就脸面朝地,哑口无声。
\par 16 不料,有一位像人的,摸我的嘴唇,我便开口向那站在我面前的说:「我主啊,因见这异象,我大大愁苦,毫无气力。
\par 17 我主的仆人怎能与我主说话呢?我一见异象就浑身无力,毫无气息。」
\par 18 有一位形状像人的又摸我,使我有力量。
\par 19 他说:「大蒙眷爱的人哪,不要惧怕,愿你平安!你总要坚强。」他一向我说话,我便觉得有力量,说:「我主请说,因你使我有力量。」
\par 20 他就说:「你知道我为何来见你吗?现在我要回去与波斯的魔君争战,我去後,希利尼(原文作雅完)的魔君必来。
\par 21 但我要将那录在真确书上的事告诉你。除了你们的大君米迦勒之外,没有帮助我抵挡这两魔君的。」

\chapter{11}

\par 1 又说:「当玛代王大利乌元年,我曾起来扶助米迦勒,使他坚强。
\par 2 现在我将真事指示你:「波斯还有三王兴起,第四王必富足远胜诸王。他因富足成为强盛,就必激动大众攻击希利尼国。
\par 3 必有一个勇敢的王兴起,执掌大权,随意而行。
\par 4 他兴起的时候,他的国必破裂,向天的四方(方:原文作风)分开,却不归他的後裔,治国的权势也都不及他;因为他的国必被拔出,归与他後裔之外的人。
\par 5 「南方的王必强盛,他将帅中必有一个比他更强盛,执掌权柄,他的权柄甚大。
\par 6 过些年後,他们必互相连合,南方王的女儿必就了北方王来立约;但这女子帮助之力存立不住,王和他所倚靠之力也不能存立。这女子和引导他来的,并生他的,以及当时扶助他的,都必交与死地。
\par 7 但这女子的本家(原文作根)必另生一子(子:原文作枝)继续王位,他必率领军队进入北方王的保障,攻击他们,而且得胜;
\par 8 并将他们的神像和铸成的偶像,与金银的宝器掠到埃及去。数年之内,他不去攻击北方的王。
\par 9 北方的王(原文作他)必入南方王的国,却要仍回本地。
\par 10 「北方王(原文作他)的二子必动干戈,招聚许多军兵。这军兵前去,如洪水泛滥,又必再去争战,直到南方王的保障。
\par 11 南方王必发烈怒,出来与北方王争战,摆列大军;北方王的军兵必交付他手。
\par 12 他的众军高傲,他的心也必自高;他虽使数万人仆倒,却不得常胜。
\par 13 「北方王必回来摆列大军,比先前的更多。满了所定的年数,他必率领大军,带极多的军装来。
\par 14 那时,必有许多人起来攻击南方王,并且你本国的强暴人必兴起,要应验那异象,他们却要败亡。
\par 15 北方王必来筑垒攻取坚固城;南方的军兵必站立不住,就是选择的精兵(原文作民)也无力站住。
\par 16 来攻击他的,必任意而行,无人在北方王(原文作他)面前站立得住。他必站在那荣美之地,用手施行毁灭。
\par 17 「他必定意用全国之力而来,立公正的约,照约而行,将自己的女儿给南方王为妻,想要败坏他(或作:埃及),这计却不得成就,与自己毫无益处。
\par 18 其後他必转回夺取了许多海岛。但有一大帅,除掉他令人受的羞辱,并且使这羞辱归他本身。
\par 19 他就必转向本地的保障,却要绊跌仆倒,归於无有。
\par 20 「那时,必有一人兴起接续他为王,使横征暴敛的人通行国中的荣美地。这王不多日就必灭亡,却不因忿怒,也不因争战。」
\par 21 「必有一个卑鄙的人兴起接续为王,人未曾将国的尊荣给他,他却趁人坦然无备的时候,用谄媚的话得国。
\par 22 必有无数的军兵势如洪水,在他面前冲没败坏;同盟的君也必如此。
\par 23 与那君结盟之後,他必行诡诈,因为他必上来以微小的军(原文作民)成为强盛。
\par 24 趁人坦然无备的时候,他必来到国中极肥美之地,行他列祖和他列祖之祖所未曾行的,将掳物、掠物,和财宝散给众人,又要设计攻打保障,然而这都是暂时的。
\par 25 「他必奋勇向前,率领大军攻击南方王;南方王也必以极大极强的军兵与他争战,却站立不住,因为有人设计谋害南方王。
\par 26 吃王膳的,必败坏他;他的军队必被冲没,而且被杀的甚多。
\par 27 至於这二王,他们心怀恶计,同席说谎,计谋却不成就;因为到了定期,事就了结。
\par 28 北方王(原文作他)必带许多财宝回往本国,他的心反对圣约,任意而行,回到本地。
\par 29 「到了定期,他必返回,来到南方。後一次却不如前一次,
\par 30 因为基提战船必来攻击他,他就丧胆而回,又要恼恨圣约,任意而行;他必回来联络背弃圣约的人。
\par 31 他必兴兵,这兵必亵渎圣地,就是保障,除掉常献的燔祭,设立那行毁坏可憎的。
\par 32 作恶违背圣约的人,他必用巧言勾引;惟独认识神的子民必刚强行事。
\par 33 民间的智慧人必训诲多人;然而他们多日必倒在刀下,或被火烧,或被掳掠抢夺。
\par 34 他们仆倒的时候,稍得扶助,却有许多人用谄媚的话亲近他们。
\par 35 智慧人中有些仆倒的,为要熬炼其余的人,使他们清净洁白,直到末了;因为到了定期,事就了结。
\par 36 「王必任意而行,自高自大,超过所有的神,又用奇异的话攻击万神之神。他必行事亨通,直到主的忿怒完毕,因为所定的事必然成就。
\par 37 他必不顾他列祖的神,也不顾妇女所羡慕的神,无论何神他都不顾;因为他必自大,高过一切。
\par 38 他倒要敬拜保障的神,用金、银、宝石和可爱之物敬奉他列祖所不认识的神。
\par 39 他必靠外邦神的帮助,攻破最坚固的保障。凡承认他的,他必将荣耀加给他们,使他们管辖许多人,又为贿赂分地与他们。
\par 40 「到末了,南方王要与他交战。北方王必用战车、马兵,和许多战船,势如暴风来攻击他,也必进入列国,如洪水泛滥。
\par 41 又必进入那荣美之地,有许多国就被倾覆,但以东人、摩押人,和一大半亚扪人必脱离他的手。
\par 42 他必伸手攻击列国;埃及地也不得脱离。
\par 43 他必把持埃及的金银财宝和各样的宝物。吕彼亚人和古实人都必跟从他。
\par 44 但从东方和北方必有消息扰乱他,他就大发烈怒出去,要将多人杀灭净尽。
\par 45 他必在海和荣美的圣山中间设立他如宫殿的帐幕;然而到了他的结局,必无人能帮助他。」

\chapter{12}

\par 1 「那时,保佑你本国之民的天使长(原文作大君)米迦勒必站起来,并且有大艰难,从有国以来直到此时,没有这样的。你本国的民中,凡名录在册上的,必得拯救。
\par 2 睡在尘埃中的,必有多人复醒。其中有得永生的,有受羞辱永远被憎恶的。
\par 3 智慧人必发光如同天上的光;那使多人归义的,必发光如星,直到永永远远。
\par 4 但以理啊,你要隐藏这话,封闭这书,直到末时。必有多人来往奔跑(或作:切心研究),知识就必增长。」
\par 5 我但以理观看,见另有两个人站立:一个在河这边,一个在河那边。
\par 6 有一个问那站在河水以上、穿细麻衣的说:「这奇异的事到几时才应验呢?」
\par 7 我听见那站在河水以上、穿细麻衣的,向天举起左右手,指著活到永远的主起誓说:「要到一载、二载、半载,打破圣民权力的时候,这一切事就都应验了。」
\par 8 我听见这话,却不明白,就说:「我主啊,这些事的结局是怎样呢?」
\par 9 他说:「但以理啊,你只管去;因为这话已经隐藏封闭,直到末时。
\par 10 必有许多人使自己清净洁白,且被熬炼;但恶人仍必行恶,一切恶人都不明白,惟独智慧人能明白。
\par 11 从除掉常献的燔祭,并设立那行毁坏可憎之物的时候,必有一千二百九十日。
\par 12 等到一千三百三十五日的,那人便为有福。
\par 13 「你且去等候结局,因为你必安歇。到了末期,你必起来,享受你的福分。」


\end{document}