\begin{document}

\title{利未记}


\chapter{1}

\par 1 耶和华从会幕中呼叫摩西,对他说:
\par 2 「你晓谕以色列人说:你们中间若有人献供物给耶和华,要从牛群羊群中献牲畜为供物。
\par 3 「他的供物若以牛为燔祭,就要在会幕门口献一只没有残疾的公牛,可以在耶和华面前蒙悦纳。
\par 4 他要按手在燔祭牲的头上,燔祭便蒙悦纳,为他赎罪。
\par 5 他要在耶和华面前宰公牛;亚伦子孙作祭司的,要奉上血,把血洒在会幕门口、坛的周围。
\par 6 那人要剥去燔祭牲的皮,把燔祭牲切成块子。
\par 7 祭司亚伦的子孙要把火放在坛上,把柴摆在火上。
\par 8 亚伦子孙作祭司的,要把肉块和头并脂油摆在坛上火的柴上。
\par 9 但燔祭的脏腑与腿要用水洗。祭司就要把一切全烧在坛上,当作燔祭,献与耶和华为馨香的火祭。
\par 10 「人的供物若以绵羊或山羊为燔祭,就要献上没有残疾的公羊。
\par 11 要把羊宰於坛的北边,在耶和华面前;亚伦子孙作祭司的,要把羊血洒在坛的周围。
\par 12 要把燔祭牲切成块子,连头和脂油,祭司就要摆在坛上火的柴上;
\par 13 但脏腑与腿要用水洗,祭司就要全然奉献,烧在坛上。这是燔祭,是献与耶和华为馨香的火祭。
\par 14 「人奉给耶和华的供物,若以鸟为燔祭,就要献斑鸠或是雏鸽为供物。
\par 15 祭司要把鸟拿到坛前,揪下头来,把鸟烧在坛上;鸟的血要流在坛的旁边;
\par 16 又要把鸟的嗉子和脏物(脏物:或作翎毛)除掉,丢在坛的东边倒灰的地方。
\par 17 要拿著鸟的两个翅膀,把鸟撕开,只是不可撕断;祭司要在坛上、在火的柴上焚烧。这是燔祭,是献与耶和华为馨香的火祭。」

\chapter{2}

\par 1 「若有人献素祭为供物给耶和华,要用细面浇上油,加上乳香,
\par 2 带到亚伦子孙作祭司的那里;祭司就要从细面中取出一把来,并取些油和所有的乳香,然後要把所取的这些作为纪念,烧在坛上,是献与耶和华为馨香的火祭。
\par 3 素祭所剩的要归给亚伦和他的子孙;这是献与耶和华的火祭中为至圣的。
\par 4 「若用炉中烤的物为素祭,就要用调油的无酵细面饼,或是抹油的无酵薄饼。
\par 5 若用铁鏊上做的物为素祭,就要用调油的无酵细面,
\par 6 分成块子,浇上油;这是素祭。
\par 7 若用煎盘做的物为素祭,就要用油与细面做成。
\par 8 要把这些东西做的素祭带到耶和华面前,并奉给祭司,带到坛前。
\par 9 祭司要从素祭中取出作为纪念的,烧在坛上,是献与耶和华为馨香的火祭。
\par 10 素祭所剩的要归给亚伦和他的子孙。这是献与耶和华的火祭中为至圣的。
\par 11 「凡献给耶和华的素祭都不可有酵;因为你们不可烧一点酵、一点蜜当作火祭献给耶和华。
\par 12 这些物要献给耶和华作为初熟的供物,只是不可在坛上献为馨香的祭。
\par 13 凡献为素祭的供物都要用盐调和,在素祭上不可缺了你神立约的盐。一切的供物都要配盐而献。
\par 14 若向耶和华献初熟之物为素祭,要献上烘了的禾穗子,就是轧了的新穗子,当作初熟之物的素祭。
\par 15 并要抹上油,加上乳香;这是素祭。
\par 16 祭司要把其中作为纪念的,就是一些轧了的禾穗子和一些油,并所有的乳香,都焚烧,是向耶和华献的火祭。」

\chapter{3}

\par 1 「人献供物为平安祭(平安:或作酬恩;下同),若是从牛群中献,无论是公的是母的,必用没有残疾的献在耶和华面前。
\par 2 他要按手在供物的头上,宰於会幕门口。亚伦子孙作祭司的,要把血洒在坛的周围。
\par 3 从平安祭中,将火祭献给耶和华,也要把盖脏的脂油和脏上所有的脂油,
\par 4 并两个腰子和腰子上的脂油,就是靠腰两旁的脂油,与肝上的网子和腰子,一概取下。
\par 5 亚伦的子孙要把这些烧在坛的燔祭上,就是在火的柴上,是献与耶和华为馨香的火祭。
\par 6 「人向耶和华献供物为平安祭,若是从羊群中献,无论是公的是母的,必用没有残疾的。
\par 7 若献一只羊羔为供物,必在耶和华面前献上,
\par 8 并要按手在供物的头上,宰於会幕前。亚伦的子孙要把血洒在坛的周围。
\par 9 从平安祭中,将火祭献给耶和华,其中的脂油和整肥尾巴都要在靠近脊骨处取下,并要把盖脏的脂油和脏上所有的脂油,
\par 10 两个腰子和腰子上的脂油,就是靠腰两旁的脂油,并肝上的网子和腰子,一概取下。
\par 11 祭司要在坛上焚烧,是献给耶和华为食物的火祭。
\par 12 「人的供物若是山羊,必在耶和华面前献上。
\par 13 要按手在山羊头上,宰於会幕前。亚伦的子孙要把血洒在坛的周围,
\par 14 又把盖脏的脂油和脏上所有的脂油,两个腰子和腰子上的脂油,
\par 15 就是靠腰两旁的脂油,并肝上的网子和腰子,一概取下,献给耶和华为火祭。
\par 16 祭司要在坛上焚烧,作为馨香火祭的食物。脂油都是耶和华的。
\par 17 在你们一切的住处,脂油和血都不可吃;这要成为你们世世代代永远的定例。」

\chapter{4}

\par 1 耶和华对摩西说:
\par 2 「你晓谕以色列人说:若有人在耶和华所吩咐不可行的什麽事上误犯了一件,
\par 3 或是受膏的祭司犯罪,使百姓陷在罪里,就当为他所犯的罪把没有残疾的公牛犊献给耶和华为赎罪祭。
\par 4 他要牵公牛到会幕门口,在耶和华面前按手在牛的头上,把牛宰於耶和华面前。
\par 5 受膏的祭司要取些公牛的血带到会幕,
\par 6 把指头蘸於血中,在耶和华面前对著圣所的幔子弹血七次,
\par 7 又要把些血抹在会幕内、耶和华面前香坛的四角上,再把公牛所有的血倒在会幕门口、燔祭坛的脚那里。
\par 8 要把赎罪祭公牛所有的脂油,乃是盖脏的脂油和脏上所有的脂油,
\par 9 并两个腰子和腰子上的脂油,就是靠腰两旁的脂油,与肝上的网子和腰子,一概取下,
\par 10 与平安祭公牛上所取的一样;祭司要把这些烧在燔祭的坛上。
\par 11 公牛的皮和所有的肉,并头、腿、脏、腑、粪,
\par 12 就是全公牛,要搬到营外洁净之地、倒灰之所,用火烧在柴上。
\par 13 「以色列全会众若行了耶和华所吩咐不可行的什麽事,误犯了罪,是隐而未现、会众看不出来的,
\par 14 会众一知道所犯的罪就要献一只公牛犊为赎罪祭,牵到会幕前。
\par 15 会中的长老就要在耶和华面前按手在牛的头上,将牛在耶和华面前宰了。
\par 16 受膏的祭司要取些公牛的血带到会幕,
\par 17 把指头蘸於血中,在耶和华面前对著幔子弹血七次,
\par 18 又要把些血抹在会幕内、耶和华面前坛的四角上,再把所有的血倒在会幕门口、燔祭坛的脚那里。
\par 19 把牛所有的脂油都取下,烧在坛上;
\par 20 收拾这牛,与那赎罪祭的牛一样。祭司要为他们赎罪,他们必蒙赦免。
\par 21 他要把牛搬到营外烧了,像烧头一个牛一样;这是会众的赎罪祭。
\par 22 「官长若行了耶和华他神所吩咐不可行的什麽事,误犯了罪,
\par 23 所犯的罪自己知道了,就要牵一只没有残疾的公山羊为供物,
\par 24 按手在羊的头上,宰於耶和华面前、宰燔祭牲的地方;这是赎罪祭。
\par 25 祭司要用指头蘸些赎罪祭牲的血,抹在燔祭坛的四角上,把血倒在燔祭坛的脚那里。
\par 26 所有的脂油,祭司都要烧在坛上,正如平安祭的脂油一样。至於他的罪,祭司要为他赎了,他必蒙赦免。
\par 27 「民中若有人行了耶和华所吩咐不可行的什麽事,误犯了罪,
\par 28 所犯的罪自己知道了,就要为所犯的罪牵一只没有残疾的母山羊为供物,
\par 29 按手在赎罪祭牲的头上,在那宰燔祭牲的地方宰了。
\par 30 祭司要用指头蘸些羊的血,抹在燔祭坛的四角上,所有的血都要倒在坛的脚那里,
\par 31 又要把羊所有的脂油都取下,正如取平安祭牲的脂油一样。祭司要在坛上焚烧,在耶和华面前作为馨香的祭,为他赎罪,他必蒙赦免。
\par 32 「人若牵一只绵羊羔为赎罪祭的供物,必要牵一只没有残疾的母羊,
\par 33 按手在赎罪祭牲的头上,在那宰燔祭牲的地方宰了作赎罪祭。
\par 34 祭司要用指头蘸些赎罪祭牲的血,抹在燔祭坛的四角上,所有的血都要倒在坛的脚那里,
\par 35 又要把所有的脂油都取下,正如取平安祭羊羔的脂油一样。祭司要按献给耶和华火祭的条例,烧在坛上。至於所犯的罪,祭司要为他赎了,他必蒙赦免。」

\chapter{5}

\par 1 「若有人听见发誓的声音(或作:若有人听见叫人发誓的声音),他本是见证,却不把所看见的、所知道的说出来,这就是罪;他要担当他的罪孽。
\par 2 或是有人摸了不洁的物,无论是不洁的死兽,是不洁的死畜,是不洁的死虫,他却不知道,因此成了不洁,就有了罪。
\par 3 或是他摸了别人的污秽,无论是染了什麽污秽,他却不知道,一知道了就有了罪。
\par 4 或是有人嘴里冒失发誓,要行恶,要行善,无论人在什麽事上冒失发誓,他却不知道,一知道了就要在这其中的一件上有了罪。
\par 5 他有了罪的时候,就要承认所犯的罪,
\par 6 并要因所犯的罪,把他的赎愆祭牲,就是羊群中的母羊,或是一只羊羔,或是一只山羊牵到耶和华面前为赎罪祭。至於他的罪,祭司要为他赎了。
\par 7 「他的力量若不够献一只羊羔,就要因所犯的罪,把两只斑鸠或是两只雏鸽带到耶和华面前为赎愆祭:一只作赎罪祭,一只作燔祭。
\par 8 把这些带到祭司那里,祭司就要先把那赎罪祭献上,从鸟的颈项上揪下头来,只是不可把鸟撕断,
\par 9 也把些赎罪祭牲的血弹在坛的旁边,剩下的血要流在坛的脚那里;这是赎罪祭。
\par 10 他要照例献第二只为燔祭。至於他所犯的罪,祭司要为他赎了,他必蒙赦免。
\par 11 「他的力量若不够献两只斑鸠或是两只雏鸽,就要因所犯的罪带供物来,就是细面伊法十分之一为赎罪祭;不可加上油,也不可加上乳香,因为是赎罪祭。
\par 12 他要把供物带到祭司那里,祭司要取出自己的一把来作为纪念,按献给耶和华火祭的条例烧在坛上;这是赎罪祭。
\par 13 至於他在这几件事中所犯的罪,祭司要为他赎了,他必蒙赦免。剩下的面都归与祭司,和素祭一样。」
\par 14 耶和华晓谕摩西说:
\par 15 「人若在耶和华的圣物上误犯了罪,有了过犯,就要照你所估的,按圣所的舍客勒拿银子,将赎愆祭牲就是羊群中一只没有残疾的公绵羊牵到耶和华面前为赎愆祭;
\par 16 并且他因在圣物上的差错要偿还,另外加五分之一,都给祭司。祭司要用赎愆祭的公绵羊为他赎罪,他必蒙赦免。
\par 17 「若有人犯罪,行了耶和华所吩咐不可行的什麽事,他虽然不知道,还是有了罪,就要担当他的罪孽;
\par 18 也要照你所估定的价,从羊群中牵一只没有残疾的公绵羊来,给祭司作赎愆祭。至於他误行的那错事,祭司要为他赎罪,他必蒙赦免。
\par 19 这是赎愆祭,因他在耶和华面前实在有了罪。」

\chapter{6}

\par 1 耶和华晓谕摩西说:
\par 2 「若有人犯罪,干犯耶和华,在邻舍交付他的物上,或是在交易上行了诡诈,或是抢夺人的财物,或是欺压邻舍,
\par 3 或是在捡了遗失的物上行了诡诈,说谎起誓,在这一切的事上犯了什麽罪;
\par 4 他既犯了罪,有了过犯,就要归还他所抢夺的,或是因欺压所得的,或是人交付他的,或是人遗失他所捡的物,
\par 5 或是他因什麽物起了假誓,就要如数归还,另外加上五分之一,在查出他有罪的日子要交还本主。
\par 6 也要照你所估定的价,把赎愆祭牲,就是羊群中一只没有残疾的公绵羊牵到耶和华面前,给祭司为赎愆祭。
\par 7 祭司要在耶和华面前为他赎罪;他无论行了什麽事,使他有了罪,都必蒙赦免。」
\par 8 耶和华晓谕摩西说:
\par 9 「你要吩咐亚伦和他的子孙说,燔祭的条例乃是这样:燔祭要放在坛的柴上,从晚上到天亮,坛上的火要常常烧著。
\par 10 祭司要穿上细麻布衣服,又要把细麻布裤子穿在身上,把坛上所烧的燔祭灰收起来,倒在坛的旁边;
\par 11 随後要脱去这衣服,穿上别的衣服,把灰拿到营外洁净之处。
\par 12 坛上的火要在其上常常烧著,不可熄灭。祭司要每日早晨在上面烧柴,并要把燔祭摆在坛上,在其上烧平安祭牲的脂油。
\par 13 在坛上必有常常烧著的火,不可熄灭。」
\par 14 「素祭的条例乃是这样:亚伦的子孙要在坛前把这祭献在耶和华面前。
\par 15 祭司要从其中,就是从素祭的细面中取出自己的一把,又要取些油和素祭上所有的乳香,烧在坛上,奉给耶和华为馨香素祭的纪念。
\par 16 所剩下的,亚伦和他子孙要吃,必在圣处不带酵而吃,要在会幕的院子里吃。
\par 17 烤的时候不可搀酵。这是从所献给我的火祭中赐给他们的分,是至圣的,和赎罪祭并赎愆祭一样。
\par 18 凡献给耶和华的火祭,亚伦子孙中的男丁都要吃这一分,直到万代,作他们永得的分。摸这些祭物的,都要成为圣。」
\par 19 耶和华晓谕摩西说:
\par 20 「当亚伦受膏的日子,他和他子孙所要献给耶和华的供物,就是细面伊法十分之一,为常献的素祭:早晨一半,晚上一半。
\par 21 要在铁鏊上用油调和做成,调匀了,你就拿进来;烤好了分成块子,献给耶和华为馨香的素祭。
\par 22 亚伦的子孙中,接续他为受膏的祭司,要把这素祭献上,要全烧给耶和华。这是永远的定例。
\par 23 祭司的素祭都要烧了,却不可吃。」
\par 24 耶和华晓谕摩西说:
\par 25 「你对亚伦和他的子孙说,赎罪祭的条例乃是这样:要在耶和华面前、宰燔祭牲的地方宰赎罪祭牲;这是至圣的。
\par 26 为赎罪献这祭的祭司要吃,要在圣处,就是在会幕的院子里吃。
\par 27 凡摸这祭肉的要成为圣;这祭牲的血若弹在什麽衣服上,所弹的那一件要在圣处洗净。
\par 28 惟有煮祭物的瓦器要打碎;若是煮在铜器里,这铜器要擦磨,在水中涮净。
\par 29 凡祭司中的男丁都可以吃;这是至圣的。
\par 30 凡赎罪祭,若将血带进会幕在圣所赎罪,那肉都不可吃,必用火焚烧。」

\chapter{7}

\par 1 「赎愆祭的条例乃是如此:这祭是至圣的。
\par 2 人在那里宰燔祭牲,也要在那里宰赎愆祭牲;其血,祭司要洒在坛的周围。
\par 3 又要将肥尾巴和盖脏的脂油,
\par 4 两个腰子和腰子上的脂油,就是靠腰两旁的脂油,并肝上的网子和腰子,一概取下。
\par 5 祭司要在坛上焚烧,为献给耶和华的火祭,是赎愆祭。
\par 6 祭司中的男丁都可以吃这祭物;要在圣处吃,是至圣的。
\par 7 赎罪祭怎样,赎愆祭也是怎样,两个祭是一个条例。献赎愆祭赎罪的祭司要得这祭物。
\par 8 献燔祭的祭司,无论为谁奉献,要亲自得他所献那燔祭牲的皮。
\par 9 凡在炉中烤的素祭和煎盘中做的,并铁鏊上做的,都要归那献祭的祭司。
\par 10 凡素祭,无论是油调和的是乾的,都要归亚伦的子孙,大家均分。」
\par 11 「人献与耶和华平安祭的条例乃是这样:
\par 12 他若为感谢献上,就要用调油的无酵饼和抹油的无酵薄饼,并用油调匀细面做的饼,与感谢祭一同献上。
\par 13 要用有酵的饼和为感谢献的平安祭,与供物一同献上。
\par 14 从各样的供物中,他要把一个饼献给耶和华为举祭,是要归给洒平安祭牲血的祭司。
\par 15 为感谢献平安祭牲的肉,要在献的日子吃,一点不可留到早晨。
\par 16 若所献的是为还愿,或是甘心献的,必在献祭的日子吃,所剩下的第二天也可以吃。
\par 17 但所剩下的祭肉,到第三天要用火焚烧;
\par 18 第三天若吃了平安祭的肉,这祭必不蒙悦纳,人所献的也不算为祭,反为可憎嫌的,吃这祭肉的,就必担当他的罪孽。
\par 19 「挨了污秽物的肉就不可吃,要用火焚烧。至於平安祭的肉,凡洁净的人都要吃;
\par 20 只是献与耶和华平安祭的肉,人若不洁净而吃了,这人必从民中剪除。
\par 21 有人摸了什麽不洁净的物,或是人的不洁净,或是不洁净的牲畜,或是不洁可憎之物,吃了献与耶和华平安祭的肉,这人必从民中剪除。」
\par 22 耶和华对摩西说:
\par 23 「你晓谕以色列人说:牛的脂油、绵羊的脂油、山羊的脂油,你们都不可吃。
\par 24 自死的和被野兽撕裂的,那脂油可以做别的使用,只是你们万不可吃。
\par 25 无论何人吃了献给耶和华当火祭牲畜的脂油,那人必从民中剪除。
\par 26 在你们一切的住处,无论是雀鸟的血是野兽的血,你们都不可吃。
\par 27 无论是谁吃血,那人必从民中剪除。」
\par 28 耶和华对摩西说:
\par 29 「你晓谕以色列人说:献平安祭给耶和华的,要从平安祭中取些来奉给耶和华。
\par 30 他亲手献给耶和华的火祭,就是脂油和胸,要带来,好把胸在耶和华面前作摇祭,摇一摇。
\par 31 祭司要把脂油在坛上焚烧,但胸要归亚伦和他的子孙。
\par 32 你们要从平安祭中把右腿作举祭,奉给祭司。
\par 33 亚伦子孙中,献平安祭牲血和脂油的,要得这右腿为分;
\par 34 因为我从以色列人的平安祭中,取了这摇的胸和举的腿给祭司亚伦和他子孙,作他们从以色列人中所永得的分。」
\par 35 这是从耶和华火祭中,作亚伦受膏的分和他子孙受膏的分,正在摩西(原文作他)叫他们前来给耶和华供祭司职分的日子,
\par 36 就是在摩西(原文作他)膏他们的日子,耶和华吩咐以色列人给他们的。这是他们世世代代永得的分。
\par 37 这就是燔祭、素祭、赎罪祭、赎愆祭,和平安祭的条例,并承接圣职的礼,
\par 38 都是耶和华在西乃山所吩咐摩西的,就是他在西乃旷野吩咐以色列人献供物给耶和华之日所说的。

\chapter{8}

\par 1 耶和华晓谕摩西说:
\par 2 「你将亚伦和他儿子一同带来,并将圣衣、膏油,与赎罪祭的一只公牛、两只公绵羊、一筐无酵饼都带来,
\par 3 又招聚会众到会幕门口。」
\par 4 摩西就照耶和华所吩咐的行了;於是会众聚集在会幕门口。
\par 5 摩西告诉会众说:「这就是耶和华所吩咐当行的事。」
\par 6 摩西带了亚伦和他儿子来,用水洗了他们。
\par 7 给亚伦穿上内袍,束上腰带,穿上外袍,又加上以弗得,用其上巧工织的带子把以弗得系在他身上,
\par 8 又给他戴上胸牌,把乌陵和土明放在胸牌内,
\par 9 把冠冕戴在他头上,在冠冕的前面钉上金牌,就是圣冠,都是照耶和华所吩咐摩西的。
\par 10 摩西用膏油抹帐幕和其中所有的,使他成圣;
\par 11 又用膏油在坛上弹了七次,又抹了坛和坛的一切器皿,并洗濯盆和盆座,使他成圣;
\par 12 又把膏油倒在亚伦的头上膏他,使他成圣。
\par 13 摩西带了亚伦的儿子来,给他们穿上内袍,束上腰带,包上裹头巾,都是照耶和华所吩咐摩西的。
\par 14 他牵了赎罪祭的公牛来,亚伦和他儿子按手在赎罪祭公牛的头上,
\par 15 就宰了公牛。摩西用指头蘸血,抹在坛上四角的周围,使坛洁净,把血倒在坛的脚那里,使坛成圣,坛就洁净了;
\par 16 又取脏上所有的脂油和肝上的网子,并两个腰子与腰子上的脂油,都烧在坛上;
\par 17 惟有公牛,连皮带肉并粪,用火烧在营外,都是照耶和华所吩咐摩西的。
\par 18 他奉上燔祭的公绵羊;亚伦和他儿子按手在羊的头上,
\par 19 就宰了公羊。摩西把血洒在坛的周围,
\par 20 把羊切成块子,把头和肉块并脂油都烧了。
\par 21 用水洗了脏腑和腿,就把全羊烧在坛上为馨香的燔祭,是献给耶和华的火祭,都是照耶和华所吩咐摩西的。
\par 22 他又奉上第二只公绵羊,就是承接圣职之礼的羊;亚伦和他儿子按手在羊的头上,
\par 23 就宰了羊。摩西把些血抹在亚伦的右耳垂上和右手的大拇指上,并右脚的大拇指上,
\par 24 又带了亚伦的儿子来,把些血抹在他们的右耳垂上和右手的大拇指上,并右脚的大拇指上,又把血洒在坛的周围。
\par 25 取脂油和肥尾巴,并脏上一切的脂油与肝上的网子,两个腰子和腰子上的脂油,并右腿,
\par 26 再从耶和华面前、盛无酵饼的筐子里取出一个无酵饼,一个油饼,一个薄饼,都放在脂油和右腿上,
\par 27 把这一切放在亚伦的手上和他儿子的手上作摇祭,在耶和华面前摇一摇。
\par 28 摩西从他们的手上拿下来,烧在坛上的燔祭上,都是为承接圣职献给耶和华馨香的火祭。
\par 29 摩西拿羊的胸作为摇祭,在耶和华面前摇一摇,是承接圣职之礼,归摩西的分,都是照耶和华所吩咐摩西的。
\par 30 摩西取点膏油和坛上的血,弹在亚伦和他的衣服上,并他儿子和他儿子的衣服上,使他和他们的衣服一同成圣。
\par 31 摩西对亚伦和他儿子说:「把肉煮在会幕门口,在那里吃,又吃承接圣职筐子里的饼,按我所吩咐的说(或作:按所吩咐我的说):『这是亚伦和他儿子要吃的。』
\par 32 剩下的肉和饼,你们要用火焚烧。
\par 33 你们七天不可出会幕的门,等到你们承接圣职的日子满了,因为主叫你们七天承接圣职。
\par 34 像今天所行的都是耶和华吩咐行的,为你们赎罪。
\par 35 七天你们要昼夜住在会幕门口,遵守耶和华的吩咐,免得你们死亡,因为所吩咐我的就是这样。」
\par 36 於是亚伦和他儿子行了耶和华藉著摩西所吩咐的一切事。

\chapter{9}

\par 1 到了第八天,摩西召了亚伦和他儿子,并以色列的众长老来,
\par 2 对亚伦说:「你当取牛群中的一只公牛犊作赎罪祭,一只公绵羊作燔祭,都要没有残疾的,献在耶和华面前。
\par 3 你也要对以色列人说:『你们当取一只公山羊作赎罪祭,又取一只牛犊和一只绵羊羔,都要一岁、没有残疾的,作燔祭,
\par 4 又取一只公牛,一只公绵羊作平安祭,献在耶和华面前,并取调油的素祭,因为今天耶和华要向你们显现。』
\par 5 於是他们把摩西所吩咐的,带到会幕前;全会众都近前来,站在耶和华面前。
\par 6 摩西说:「这是耶和华吩咐你们所当行的;耶和华的荣光就要向你们显现。」
\par 7 摩西对亚伦说:「你就近坛前,献你的赎罪祭和燔祭,为自己与百姓赎罪,又献上百姓的供物,为他们赎罪,都照耶和华所吩咐的。」
\par 8 於是,亚伦就近坛前,宰了为自己作赎罪祭的牛犊。
\par 9 亚伦的儿子把血奉给他,他就把指头蘸在血中,抹在坛的四角上,又把血倒在坛脚那里。
\par 10 惟有赎罪祭的脂油和腰子,并肝上取的网子,都烧在坛上,是照耶和华所吩咐摩西的;
\par 11 又用火将肉和皮烧在营外。
\par 12 亚伦宰了燔祭牲,他儿子把血递给他,他就洒在坛的周围,
\par 13 又把燔祭一块一块地、连头递给他,他都烧在坛上;
\par 14 又洗了脏腑和腿,烧在坛上的燔祭上。
\par 15 他奉上百姓的供物,把那给百姓作赎罪祭的公山羊宰了,为罪献上,和先献的一样;
\par 16 也奉上燔祭,照例而献。
\par 17 他又奉上素祭,从其中取一满把,烧在坛上;这是在早晨的燔祭以外。
\par 18 亚伦宰了那给百姓作平安祭的公牛和公绵羊。他儿子把血递给他,他就洒在坛的周围;
\par 19 又把公牛和公绵羊的脂油、肥尾巴,并盖脏的脂油与腰子,和肝上的网子,都递给他;
\par 20 把脂油放在胸上,他就把脂油烧在坛上。
\par 21 胸和右腿,亚伦当作摇祭,在耶和华面前摇一摇,都是照摩西所吩咐的。
\par 22 亚伦向百姓举手,为他们祝福。他献了赎罪祭、燔祭、平安祭就下来了。
\par 23 摩西、亚伦进入会幕,又出来为百姓祝福,耶和华的荣光就向众民显现。
\par 24 有火从耶和华面前出来,在坛上烧尽燔祭和脂油;众民一见,就都欢呼,俯伏在地。

\chapter{10}

\par 1 亚伦的儿子拿答、亚比户各拿自己的香炉,盛上火,加上香,在耶和华面前献上凡火,是耶和华没有吩咐他们的,
\par 2 就有火从耶和华面前出来,把他们烧灭,他们就死在耶和华面前。
\par 3 於是摩西对亚伦说:「这就是耶和华所说:『我在亲近我的人中要显为圣;在众民面前,我要得荣耀。』」亚伦就默默不言。
\par 4 摩西召了亚伦叔父乌薛的儿子米沙利、以利撒反来,对他们说:「上前来,把你们的亲属从圣所前抬到营外。」
\par 5 於是二人上前来,把他们穿著袍子抬到营外,是照摩西所吩咐的。
\par 6 摩西对亚伦和他儿子以利亚撒、以他玛说:「不可蓬头散发,也不可撕裂衣裳,免得你们死亡,又免得耶和华向会众发怒;只要你们的弟兄以色列全家为耶和华所发的火哀哭。
\par 7 你们也不可出会幕的门,恐怕你们死亡,因为耶和华的膏油在你们的身上。」他们就照摩西的话行了。
\par 8 耶和华晓谕亚伦说:
\par 9 「你和你儿子进会幕的时候,清酒、浓酒都不可喝,免得你们死亡;这要作你们世世代代永远的定例。
\par 10 使你们可以将圣的、俗的,洁净的、不洁净的,分别出来;
\par 11 又使你们可以将耶和华藉摩西晓谕以色列人的一切律例教训他们。」
\par 12 摩西对亚伦和他剩下的儿子以利亚撒、以他玛说:「你们献给耶和华火祭中所剩的素祭,要在坛旁不带酵而吃,因为是至圣的。
\par 13 你们要在圣处吃;因为在献给耶和华的火祭中,这是你的分和你儿子的分;所吩咐我的本是这样。
\par 14 所摇的胸,所举的腿,你们要在洁净地方吃。你和你的儿女都要同吃;因为这些是从以色列人平安祭中给你,当你的分和你儿子的分。
\par 15 所举的腿,所摇的胸,他们要与火祭的脂油一同带来当摇祭,在耶和华面前摇一摇;这要归你和你儿子,当作永得的分,都是照耶和华所吩咐的。」
\par 16 当下摩西急切地寻找作赎罪祭的公山羊,谁知已经焚烧了,便向亚伦剩下的儿子以利亚撒、以他玛发怒,说:
\par 17 「这赎罪祭既是至圣的,主又给了你们,为要你们担当会众的罪孽,在耶和华面前为他们赎罪,你们为何没有在圣所吃呢?
\par 18 看哪,这祭牲的血并没有拿到圣所里去,你们本当照我所吩咐的,在圣所里吃这祭肉。」
\par 19 亚伦对摩西说:「今天他们在耶和华面前献上赎罪祭和燔祭,我又遇见这样的灾,若今天吃了赎罪祭,耶和华岂能看为美呢?」
\par 20 摩西听见这话,便以为美。

\chapter{11}

\par 1 耶和华对摩西、亚伦说:
\par 2 「你们晓谕以色列人说,在地上一切走兽中可吃的乃是这些:
\par 3 凡蹄分两瓣、倒嚼的走兽,你们都可以吃。
\par 4 但那倒嚼或分蹄之中不可吃的乃是:骆驼因为倒嚼不分蹄,就与你们不洁净;
\par 5 沙番因为倒嚼不分蹄,就与你们不洁净;
\par 6 兔子因为倒嚼不分蹄,就与你们不洁净;
\par 7 猪因为蹄分两瓣,却不倒嚼,就与你们不洁净。
\par 8 这些兽的肉,你们不可吃;死的,你们不可摸,都与你们不洁净。
\par 9 「水中可吃的乃是这些:凡在水里、海里、河里、有翅有鳞的,都可以吃。
\par 10 凡在海里、河里,并一切水里游动的活物,无翅无鳞的,你们都当以为可憎。
\par 11 这些无翅无鳞、以为可憎的,你们不可吃他的肉;死的也当以为可憎。
\par 12 凡水里无翅无鳞的,你们都当以为可憎。
\par 13 「雀鸟中你们当以为可憎、不可吃的乃是: 、狗头 、红头 、
\par 14 鹞鹰、小鹰与其类;
\par 15 乌鸦与其类i
\par 16 鸵鸟、夜鹰、鱼鹰、鹰与其类;
\par 17 鸟、鸬鹚、猫头鹰、
\par 18 角鸱、鹈鹕、秃 、
\par 19 鹳、鹭鸶与其类;戴 与蝙蝠。
\par 20 「凡有翅膀用四足爬行的物,你们都当以为可憎。
\par 21 只是有翅膀用四足爬行的物中,有足有腿,在地上蹦跳的,你们还可以吃。
\par 22 其中有蝗虫、蚂蚱、蟋蟀与其类;蚱蜢与其类;这些你们都可以吃。
\par 23 但是有翅膀有四足的爬物,你们都当以为可憎。
\par 24 「这些都能使你们不洁净。凡摸了死的,必不洁净到晚上。
\par 25 凡拿了死的,必不洁净到晚上,并要洗衣服。
\par 26 凡走兽分蹄不成两瓣、也不倒嚼的,是与你们不洁净;凡摸了的就不洁净。
\par 27 凡四足的走兽,用掌行走的,是与你们不洁净;摸其尸的,必不洁净到晚上。
\par 28 拿其尸的,必不洁净到晚上,并要洗衣服。这些是与你们不洁净的。
\par 29 「地上爬物与你们不洁净的乃是这些:鼬鼠、 鼠、蜥蜴与其类;
\par 30 壁虎、龙子、守宫、蛇医、 蜓。
\par 31 这些爬物都是与你们不洁净的。在他死了以後,凡摸了的,必不洁净到晚上。
\par 32 其中死了的,掉在什麽东西上,这东西就不洁净,无论是木器、衣服、皮子、口袋,不拘是做什麽工用的器皿,须要放在水中,必不洁净到晚上,到晚上才洁净了。
\par 33 若有死了掉在瓦器里的,其中不拘有什麽,就不洁净,你们要把这瓦器打破了。
\par 34 其中一切可吃的食物,沾水的就不洁净,并且那样器皿中一切可喝的,也必不洁净。
\par 35 其中已死的,若有一点掉在什麽物件上,那物件就不洁净,不拘是炉子,是锅台,就要打碎,都不洁净,也必与你们不洁净。
\par 36 但是泉源或是聚水的池子仍是洁净;惟挨了那死的,就不洁净。
\par 37 若是死的,有一点掉在要种的子粒上,子粒仍是洁净;
\par 38 若水已经浇在子粒上,那死的有一点掉在上头,这子粒就与你们不洁净。
\par 39 「你们可吃的走兽若是死了,有人摸他,必不洁净到晚上;
\par 40 有人吃那死了的走兽,必不洁净到晚上,并要洗衣服;拿了死走兽的,必不洁净到晚上,并要洗衣服。
\par 41 「凡地上的爬物是可憎的,都不可吃。
\par 42 凡用肚子行走的和用四足行走的,或是有许多足的,就是一切爬在地上的,你们都不可吃,因为是可憎的。
\par 43 你们不可因什麽爬物使自己成为可憎的,也不可因这些使自己不洁净,以致染了污秽。
\par 44 我是耶和华你们的神;所以你们要成为圣洁,因为我是圣洁的。你们也不可在地上的爬物污秽自己。
\par 45 我是把你们从埃及地领出来的耶和华,要作你们的神;所以你们要圣洁,因为我是圣洁的。」
\par 46 这是走兽、飞鸟,和水中游动的活物,并地上爬物的条例。
\par 47 要把洁净的和不洁净的,可吃的与不可吃的活物,都分别出来。

\chapter{12}

\par 1 耶和华对摩西说:
\par 2 「你晓谕以色列人说:若有妇人怀孕生男孩,他就不洁净七天,像在月经污秽的日子不洁净一样。
\par 3 第八天,要给婴孩行割礼。
\par 4 妇人在产血不洁之中,要家居三十三天。他洁净的日子未满,不可摸圣物,也不可进入圣所。
\par 5 他若生女孩,就不洁净两个七天,像污秽的时候一样,要在产血不洁之中,家居六十六天。
\par 6 「满了洁净的日子,无论是为男孩是为女孩,他要把一岁的羊羔为燔祭,一只雏鸽或是一只斑鸠为赎罪祭,带到会幕门口交给祭司。
\par 7 祭司要献在耶和华面前,为他赎罪,他的血源就洁净了。这条例是为生育的妇人,无论是生男生女。
\par 8 他的力量若不够献一只羊羔,他就要取两只斑鸠或是两只雏鸽,一只为燔祭,一只为赎罪祭。祭司要为他赎罪,他就洁净了。」

\chapter{13}

\par 1 耶和华晓谕摩西、亚伦说:
\par 2 「人的肉皮上若长了疖子,或长了癣,或长了火斑,在他肉皮上成了大麻疯的灾病,就要将他带到祭司亚伦或亚伦作祭司的一个子孙面前。
\par 3 祭司要察看肉皮上的灾病,若灾病处的毛已经变白,灾病的现象深於肉上的皮,这便是大麻疯的灾病。祭司要察看他,定他为不洁净。
\par 4 若火斑在他肉皮上是白的,现象不深於皮,其上的毛也没有变白,祭司就要将有灾病的人关锁七天。
\par 5 第七天,祭司要察看他,若看灾病止住了,没有在皮上发散,祭司还要将他关锁七天。
\par 6 第七天,祭司要再察看他,若灾病发暗,而且没有在皮上发散,祭司要定他为洁净,原来是癣;那人就要洗衣服,得为洁净。
\par 7 但他为得洁净,将身体给祭司察看以後,癣若在皮上发散开了,他要再将身体给祭司察看。
\par 8 祭司要察看,癣若在皮上发散,就要定他为不洁净,是大麻疯。
\par 9 「人有了大麻疯的灾病,就要将他带到祭司面前。
\par 10 祭司要察看,皮上若长了白疖,使毛变白,在长白疖之处有了红瘀肉,
\par 11 这是肉皮上的旧大麻疯,祭司要定他为不洁净,不用将他关锁,因为他是不洁净了。
\par 12 大麻疯若在皮上四外发散,长满了患灾病人的皮,据祭司察看,从头到脚无处不有,
\par 13 祭司就要察看,全身的肉若长满了大麻疯,就要定那患灾病的为洁净;全身都变为白,他乃洁净了。
\par 14 但红肉几时显在他的身上就几时不洁净。
\par 15 祭司一看那红肉就定他为不洁净。红肉本是不洁净,是大麻疯。
\par 16 红肉若复原,又变白了,他就要来见祭司。
\par 17 祭司要察看,灾病处若变白了,祭司就要定那患灾病的为洁净,他乃洁净了。
\par 18 「人若在皮肉上长疮,却治好了,
\par 19 在长疮之处又起了白疖,或是白中带红的火斑,就要给祭司察看。
\par 20 祭司要察看,若现象洼於皮,其上的毛也变白了,就要定他为不洁净,是大麻疯的灾病发在疮中。
\par 21 祭司若察看,其上没有白毛,也没有洼於皮,乃是发暗,就要将他关锁七天。
\par 22 若在皮上发散开了,祭司就要定他为不洁净,是灾病。
\par 23 火斑若在原处止住,没有发散,便是疮的痕迹,祭司就要定他为洁净。
\par 24 「人的皮肉上若起了火毒,火毒的瘀肉成了火斑,或是白中带红的,或是全白的,
\par 25 祭司就要察看,火斑中的毛若变白了,现象又深於皮,是大麻疯在火毒中发出,就要定他为不洁净,是大麻疯的灾病。
\par 26 但是祭司察看,在火斑中若没有白毛,也没有洼於皮,乃是发暗,就要将他关锁七天。
\par 27 到第七天,祭司要察看他,火斑若在皮上发散开了,就要定他为不洁净,是大麻疯的灾病。
\par 28 火斑若在原处止住,没有在皮上发散,乃是发暗,是起的火毒,祭司要定他为洁净,不过是火毒的痕迹。
\par 29 「无论男女,若在头上有灾病,或是男人胡须上有灾病,
\par 30 祭司就要察看;这灾病现象若深於皮,其间有细黄毛,就要定他为不洁净,这是头疥,是头上或是胡须上的大麻疯。
\par 31 祭司若察看头疥的灾病,现象不深於皮,其间也没有黑毛,就要将长头疥灾病的关锁七天。
\par 32 第七天,祭司要察看灾病,若头疥没有发散,其间也没有黄毛,头疥的现象不深於皮,
\par 33 那人就要剃去须发,但他不可剃头疥之处。祭司要将那长头疥的,再关锁七天。
\par 34 第七天,祭司要察看头疥,头疥若没有在皮上发散,现象也不深於皮,就要定他为洁净,他要洗衣服,便成为洁净。
\par 35 但他得洁净以後,头疥若在皮上发散开了,
\par 36 祭司就要察看他。头疥若在皮上发散,就不必找那黄毛,他是不洁净了。
\par 37 祭司若看头疥已经止住,其间也长了黑毛,头疥已然痊愈,那人是洁净了,就要定他为洁净。
\par 38 「无论男女,皮肉上若起了火斑,就是白火斑,
\par 39 祭司就要察看,他们肉皮上的火斑若白中带黑,这是皮上发出的白癣,那人是洁净了。
\par 40 「人头上的发若掉了,他不过是头秃,还是洁净。
\par 41 他顶前若掉了头发,他不过是顶门秃,还是洁净。
\par 42 头秃处或是顶门秃处若有白中带红的灾病,这就是大麻疯发在他头秃处或是顶门秃处,
\par 43 祭司就要察看,他起的那灾病若在头秃处或是顶门秃处有白中带红的,像肉皮上大麻疯的现象,
\par 44 那人就是长大麻疯,不洁净的,祭司总要定他为不洁净,他的灾病是在头上。
\par 45 「身上有长大麻疯灾病的,他的衣服要撕裂,也要蓬头散发,蒙著上唇,喊叫说:『不洁净了!不洁净了!』
\par 46 灾病在他身上的日子,他便是不洁净;他既是不洁净,就要独居营外。」
\par 47 「染了大麻疯灾病的衣服,无论是羊毛衣服、是麻布衣服,
\par 48 无论是在经上、在纬上,是麻布的、是羊毛的,是在皮子上,或在皮子做的什麽物件上,
\par 49 或在衣服上、皮子上,经上、纬上,或在皮子做的什麽物件上,这灾病若是发绿,或是发红,是大麻疯的灾病,要给祭司察看。
\par 50 祭司就要察看那灾病,把染了灾病的物件关锁七天。
\par 51 第七天,他要察看那灾病,灾病或在衣服上,经上、纬上,皮子上,若发散,这皮子无论当作何用,这灾病是蚕食的大麻疯,都是不洁净了。
\par 52 那染了灾病的衣服,或是经上、纬上,羊毛上,麻衣上,或是皮子做的什麽物件上,他都要焚烧;因为这是蚕食的大麻疯,必在火中焚烧。
\par 53 「祭司要察看,若灾病在衣服上,经上、纬上,或是皮子做的什麽物件上,没有发散,
\par 54 祭司就要吩咐他们,把染了灾病的物件洗了,再关锁七天。
\par 55 洗过以後,祭司要察看,那物件若没有变色,灾病也没有消散,那物件就不洁净,是透重的灾病,无论正面反面,都要在火中焚烧。
\par 56 洗过以後,祭司要察看,若见那灾病发暗,他就要把那灾病从衣服上,皮子上,经上、纬上,都撕去。
\par 57 若仍现在衣服上,或是经上、纬上、皮子做的什麽物件上,这就是灾病又发了,必用火焚烧那染灾病的物件。
\par 58 所洗的衣服,或是经,或是纬,或是皮子做的什麽物件,若灾病离开了,要再洗,就洁净了。」
\par 59 这就是大麻疯灾病的条例,无论是在羊毛衣服上,麻布衣服上,经上、纬上,和皮子做的什麽物件上,可以定为洁净或是不洁净。

\chapter{14}

\par 1 耶和华晓谕摩西说:
\par 2 「长大麻疯得洁净的日子,其例乃是这样:要带他去见祭司;
\par 3 祭司要出到营外察看,若见他的大麻疯痊愈了,
\par 4 就要吩咐人为那求洁净的拿两只洁净的活鸟和香柏木、朱红色线,并牛膝草来。
\par 5 祭司要吩咐用瓦器盛活水,把一只鸟宰在上面。
\par 6 至於那只活鸟,祭司要把他和香柏木、朱红色线并牛膝草一同蘸於宰在活水上的鸟血中,
\par 7 用以在那长大麻疯求洁净的人身上洒七次,就定他为洁净,又把活鸟放在田野里。
\par 8 求洁净的人当洗衣服,剃去毛发,用水洗澡,就洁净了;然後可以进营,只是要在自己的帐棚外居住七天。
\par 9 第七天,再把头上所有的头发与胡须、眉毛,并全身的毛,都剃了;又要洗衣服,用水洗身,就洁净了。
\par 10 「第八天,他要取两只没有残疾的公羊羔和一只没有残疾、一岁的母羊羔,又要把调油的细面伊法十分之三为素祭,并油一罗革,一同取来。
\par 11 行洁净之礼的祭司要将那求洁净的人和这些东西安置在会幕门口、耶和华面前。
\par 12 祭司要取一只公羊羔献为赎愆祭,和那一罗革油一同作摇祭,在耶和华面前摇一摇;
\par 13 把公羊羔宰於圣地,就是宰赎罪祭牲和燔祭牲之地。赎愆祭要归祭司,与赎罪祭一样,是至圣的。
\par 14 祭司要取些赎愆祭牲的血,抹在求洁净人的右耳垂上和右手的大拇指上,并右脚的大拇指上。
\par 15 祭司要从那一罗革油中取些倒在自己的左手掌里,
\par 16 把右手的一个指头蘸在左手的油里,在耶和华面前用指头弹七次。
\par 17 将手里所剩的油抹在那求洁净人的右耳垂上和右手的大拇指上,并右脚的大拇指上,就是抹在赎愆祭牲的血上。
\par 18 祭司手里所剩的油要抹在那求洁净人的头上,在耶和华面前为他赎罪。
\par 19 祭司要献赎罪祭,为那本不洁净、求洁净的人赎罪;然後要宰燔祭牲,
\par 20 把燔祭和素祭献在坛上,为他赎罪,他就洁净了。
\par 21 「他若贫穷不能预备够数,就要取一只公羊羔作赎愆祭,可以摇一摇,为他赎罪;也要把调油的细面伊法十分之一为素祭,和油一罗革一同取来;
\par 22 又照他的力量取两只斑鸠或是两只雏鸽,一只作赎罪祭,一只作燔祭。
\par 23 第八天,要为洁净,把这些带到会幕门口、耶和华面前,交给祭司。
\par 24 祭司要把赎愆祭的羊羔和那一罗革油一同作摇祭,在耶和华面前摇一摇。
\par 25 要宰了赎愆祭的羊羔,取些赎愆祭牲的血,抹在那求洁净人的右耳垂上和右手的大拇指上,并右脚的大拇指上。
\par 26 祭司要把些油倒在自己的左手掌里,
\par 27 把左手里的油,在耶和华面前,用右手的一个指头弹七次,
\par 28 又把手里的油抹些在那求洁净人的右耳垂上和右手的大拇指上,并右脚的大拇指上,就是抹赎愆祭之血的原处。
\par 29 祭司手里所剩的油要抹在那求洁净人的头上,在耶和华面前为他赎罪。
\par 30 那人又要照他的力量献上一只斑鸠或是一只雏鸽,
\par 31 就是他所能办的,一只为赎罪祭,一只为燔祭,与素祭一同献上;祭司要在耶和华面前为他赎罪。
\par 32 这是那有大麻疯灾病的人、不能将关乎得洁净之物预备够数的条例。」
\par 33 耶和华晓谕摩西、亚伦说:
\par 34 「你们到了我赐给你们为业的迦南地,我若使你们所得为业之地的房屋中有大麻疯的灾病,
\par 35 房主就要去告诉祭司说:『据我看,房屋中似乎有灾病。』
\par 36 祭司还没有进去察看灾病以前,就要吩咐人把房子腾空,免得房子里所有的都成了不洁净;然後祭司要进去察看房子。
\par 37 他要察看那灾病,灾病若在房子的墙上有发绿或发红的凹斑纹,现象洼於墙,
\par 38 祭司就要出到房门外,把房子封锁七天。
\par 39 第七天,祭司要再去察看,灾病若在房子的墙上发散,
\par 40 就要吩咐人把那有灾病的石头挖出来,扔在城外不洁净之处;
\par 41 也要叫人刮房内的四围,所刮掉的灰泥要倒在城外不洁净之处;
\par 42 又要用别的石头代替那挖出来的石头,要另用灰泥墁房子。
\par 43 「他挖出石头,刮了房子,墁了以後,灾病若在房子里又发现,
\par 44 祭司就要进去察看,灾病若在房子里发散,这就是房内蚕食的大麻疯,是不洁净。
\par 45 他就要拆毁房子,把石头、木头、灰泥都搬到城外不洁净之处。
\par 46 在房子封锁的时候,进去的人必不洁净到晚上;
\par 47 在房子里躺著的必洗衣服;在房子里吃饭的也必洗衣服。
\par 48 「房子墁了以後,祭司若进去察看,见灾病在房内没有发散,就要定房子为洁净,因为灾病已经消除。
\par 49 要为洁净房子取两只鸟和香柏木、朱红色线并牛膝草,
\par 50 用瓦器盛活水,把一只鸟宰在上面,
\par 51 把香柏木、牛膝草、朱红色线,并那活鸟,都蘸在被宰的鸟血中与活水中,用以洒房子七次。
\par 52 要用鸟血、活水、活鸟、香柏木、牛膝草,并朱红色线,洁净那房子。
\par 53 但要把活鸟放在城外田野里。这样洁净房子(原文作为房子赎罪),房子就洁净了。」
\par 54 这是为各类大麻疯的灾病和头疥,
\par 55 并衣服与房子的大麻疯,
\par 56 以及疖子、癣、火斑所立的条例,
\par 57 指明何时为洁净,何时为不洁净。这是大麻疯的条例。

\chapter{15}

\par 1 耶和华对摩西、亚伦说:
\par 2 「你们晓谕以色列人说:人若身患漏症,他因这漏症就不洁净了。
\par 3 他患漏症,无论是下流的,是止住的,都是不洁净。
\par 4 他所躺的床都为不洁净,所坐的物也为不洁净。
\par 5 凡摸那床的,必不洁净到晚上,并要洗衣服,用水洗澡。
\par 6 那坐患漏症人所坐之物的,必不洁净到晚上,并要洗衣服,用水洗澡。
\par 7 那摸患漏症人身体的,必不洁净到晚上,并要洗衣服,用水洗澡。
\par 8 若患漏症人吐在洁净的人身上,那人必不洁净到晚上,并要洗衣服,用水洗澡。
\par 9 患漏症人所骑的鞍子也为不洁净。
\par 10 凡摸了他身下之物的,必不洁净到晚上;拿了那物的,必不洁净到晚上,并要洗衣服,用水洗澡。
\par 11 患漏症的人没有用水涮手,无论摸了谁,谁必不洁净到晚上,并要洗衣服,用水洗澡。
\par 12 患漏症人所摸的瓦器就必打破;所摸的一切木器也必用水涮洗。
\par 13 「患漏症的人痊愈了,就要为洁净自己计算七天,也必洗衣服,用活水洗身,就洁净了。
\par 14 第八天,要取两只斑鸠或是两只雏鸽,来到会幕门口、耶和华面前,把鸟交给祭司。
\par 15 祭司要献上一只为赎罪祭,一只为燔祭;因那人患的漏症,祭司要在耶和华面前为他赎罪。
\par 16 「人若梦遗,他必不洁净到晚上,并要用水洗全身。
\par 17 无论是衣服是皮子,被精所染,必不洁净到晚上,并要用水洗。
\par 18 若男女交合,两个人必不洁净到晚上,并要用水洗澡。
\par 19 「女人行经,必污秽七天;凡摸他的,必不洁净到晚上。
\par 20 女人在污秽之中,凡他所躺的物件都为不洁净,所坐的物件也都不洁净。
\par 21 凡摸他床的,必不洁净到晚上,并要洗衣服,用水洗澡。
\par 22 凡摸他所坐什麽物件的,必不洁净到晚上,并要洗衣服,用水洗澡。
\par 23 在女人的床上,或在他坐的物上,若有别的物件,人一摸了,必不洁净到晚上。
\par 24 男人若与那女人同房,染了他的污秽,就要七天不洁净;所躺的床也为不洁净。
\par 25 「女人若在经期以外患多日的血漏,或是经期过长,有了漏症,他就因这漏症不洁净,与他在经期不洁净一样。
\par 26 他在患漏症的日子所躺的床、所坐的物都要看为不洁净,与他月经的时候一样。
\par 27 凡摸这些物件的,就为不洁净,必不洁净到晚上,并要洗衣服,用水洗澡。
\par 28 女人的漏症若好了,就要计算七天,然後才为洁净。
\par 29 第八天,要取两只斑鸠或是两只雏鸽,带到会幕门口给祭司。
\par 30 祭司要献一只为赎罪祭,一只为燔祭;因那人血漏不洁,祭司要在耶和华面前为他赎罪。
\par 31 「你们要这样使以色列人与他们的污秽隔绝,免得他们玷污我的帐幕,就因自己的污秽死亡。」
\par 32 这是患漏症和梦遗而不洁净的,
\par 33 并有月经病的和患漏症的,无论男女,并人与不洁净女人同房的条例。

\chapter{16}

\par 1 亚伦的两个儿子近到耶和华面前死了。死了之後,耶和华晓谕摩西说:
\par 2 「要告诉你哥哥亚伦,不可随时进圣所的幔子内、到柜上的施恩座前,免得他死亡,因为我要从云中显现在施恩座上。
\par 3 亚伦进圣所,要带一只公牛犊为赎罪祭,一只公绵羊为燔祭。
\par 4 要穿上细麻布圣内袍,把细麻布裤子穿在身上,腰束细麻布带子,头戴细麻布冠冕;这都是圣服。他要用水洗身,然後穿戴。
\par 5 要从以色列会众取两只公山羊为赎罪祭,一只公绵羊为燔祭。
\par 6 「亚伦要把赎罪祭的公牛奉上,为自己和本家赎罪;
\par 7 也要把两只公山羊安置在会幕门口、耶和华面前,
\par 8 为那两只羊拈阄,一阄归与耶和华,一阄归与阿撒泻勒。
\par 9 亚伦要把那拈阄归与耶和华的羊献为赎罪祭,
\par 10 但那拈阄归与阿撒泻勒的羊要活著安置在耶和华面前,用以赎罪,打发人送到旷野去,归与阿撒泻勒。
\par 11 「亚伦要把赎罪祭的公牛牵来宰了,为自己和本家赎罪;
\par 12 拿香炉,从耶和华面前的坛上盛满火炭,又拿一捧捣细的香料,都带入幔子内,
\par 13 在耶和华面前,把香放在火上,使香的烟云遮掩法柜上的施恩座,免得他死亡;
\par 14 也要取些公牛的血,用指头弹在施恩座的东面,又在施恩座的前面弹血七次。
\par 15 「随後他要宰那为百姓作赎罪祭的公山羊,把羊的血带入幔子内,弹在施恩座的上面和前面,好像弹公牛的血一样。
\par 16 他因以色列人诸般的污秽、过犯,就是他们一切的罪愆,当这样在圣所行赎罪之礼,并因会幕在他们污秽之中,也要照样而行。
\par 17 他进圣所赎罪的时候,会幕里不可有人,直等到他为自己和本家并以色列全会众赎了罪出来。
\par 18 他出来,要到耶和华面前的坛那里,在坛上行赎罪之礼,又要取些公牛的血和公山羊的血,抹在坛上四角的周围;
\par 19 也要用指头把血弹在坛上七次,洁净了坛,从坛上除掉以色列人诸般的污秽,使坛成圣。」
\par 20 「亚伦为圣所和会幕并坛献完了赎罪祭,就要把那只活著的公山羊奉上。
\par 21 两手按在羊头上,承认以色列人诸般的罪孽过犯,就是他们一切的罪愆,把这罪都归在羊的头上,藉著所派之人的手,送到旷野去。
\par 22 要把这羊放在旷野,这羊要担当他们一切的罪孽,带到无人之地。
\par 23 「亚伦要进会幕,把他进圣所时所穿的细麻布衣服脱下,放在那里,
\par 24 又要在圣处用水洗身,穿上衣服,出来,把自己的燔祭和百姓的燔祭献上,为自己和百姓赎罪。
\par 25 赎罪祭牲的脂油要在坛上焚烧。
\par 26 那放羊归与阿撒泻勒的人要洗衣服,用水洗身,然後进营。
\par 27 作赎罪祭的公牛和公山羊的血既带入圣所赎罪,这牛羊就要搬到营外,将皮、肉、粪用火焚烧。
\par 28 焚烧的人要洗衣服,用水洗身,然後进营。」
\par 29 「每逢七月初十日,你们要刻苦己心,无论是本地人,是寄居在你们中间的外人,什麽工都不可做;这要作你们永远的定例。
\par 30 因在这日要为你们赎罪,使你们洁净。你们要在耶和华面前得以洁净,脱尽一切的罪愆。
\par 31 这日你们要守为圣安息日,要刻苦己心;这为永远的定例。
\par 32 那受膏、接续他父亲承接圣职的祭司要穿上细麻布的圣衣,行赎罪之礼。
\par 33 他要在至圣所和会幕与坛行赎罪之礼,并要为众祭司和会众的百姓赎罪。
\par 34 这要作你们永远的定例,就是因以色列人一切的罪,要一年一次为他们赎罪。」於是,亚伦照耶和华所吩咐摩西的行了。

\chapter{17}

\par 1 耶和华对摩西说:
\par 2 「你晓谕亚伦和他儿子并以色列众人说,耶和华所吩咐的乃是这样:
\par 3 凡以色列家中的人宰公牛,或是绵羊羔,或是山羊,不拘宰於营内营外,
\par 4 若未曾牵到会幕门口、耶和华的帐幕前献给耶和华为供物,流血的罪必归到那人身上。他流了血,要从民中剪除。
\par 5 这是为要使以色列人把他们在田野里所献的祭带到会幕门口、耶和华面前,交给祭司,献与耶和华为平安祭。
\par 6 祭司要把血洒在会幕门口、耶和华的坛上,把脂油焚烧,献给耶和华为馨香的祭。
\par 7 他们不可再献祭给他们行邪淫所随从的鬼魔(原文作公山羊);这要作他们世世代代永远的定例。
\par 8 「你要晓谕他们说:凡以色列家中的人,或是寄居在他们中间的外人,献燔祭或是平安祭,
\par 9 若不带到会幕门口献给耶和华,那人必从民中剪除。
\par 10 「凡以色列家中的人,或是寄居在他们中间的外人,若吃什麽血,我必向那吃血的人变脸,把他从民中剪除。
\par 11 因为活物的生命是在血中。我把这血赐给你们,可以在坛上为你们的生命赎罪;因血里有生命,所以能赎罪。
\par 12 因此,我对以色列人说:你们都不可吃血;寄居在你们中间的外人也不可吃血。
\par 13 凡以色列人,或是寄居在他们中间的外人,若打猎得了可吃的禽兽,必放出他的血来,用土掩盖。
\par 14 「论到一切活物的生命,就在血中。所以我对以色列人说:无论什麽活物的血,你们都不可吃,因为一切活物的血就是他的生命。凡吃了血的,必被剪除。
\par 15 凡吃自死的,或是被野兽撕裂的,无论是本地人,是寄居的,必不洁净到晚上,都要洗衣服,用水洗身,到了晚上才为洁净。
\par 16 但他若不洗衣服,也不洗身,就必担当他的罪孽。」

\chapter{18}

\par 1 耶和华对摩西说:
\par 2 「你晓谕以色列人说:我是耶和华你们的神。
\par 3 你们从前住的埃及地,那里人的行为,你们不可效法,我要领你们到的迦南地,那里人的行为也不可效法,也不可照他们的恶俗行。
\par 4 你们要遵我的典章,守我的律例,按此而行。我是耶和华你们的神。
\par 5 所以,你们要守我的律例典章;人若遵行,就必因此活著。我是耶和华。
\par 6 「你们都不可露骨肉之亲的下体,亲近他们。我是耶和华。
\par 7 不可露你母亲的下体,羞辱了你父亲。他是你的母亲,不可露他的下体。
\par 8 不可露你继母的下体;这本是你父亲的下体。
\par 9 你的姊妹,不拘是异母同父的,是异父同母的,无论是生在家生在外的,都不可露他们的下体。
\par 10 不可露你孙女或是外孙女的下体,露了他们的下体就是露了自己的下体。
\par 11 你继母从你父亲生的女儿本是你的妹妹,不可露他的下体。
\par 12 不可露你姑母的下体;他是你父亲的骨肉之亲。
\par 13 不可露你姨母的下体;他是你母亲的骨肉之亲。
\par 14 不可亲近你伯叔之妻,羞辱了你伯叔;他是你的伯叔母。
\par 15 不可露你儿妇的下体;他是你儿子的妻,不可露他的下体。
\par 16 不可露你弟兄妻子的下体;这本是你弟兄的下体。
\par 17 不可露了妇人的下体,又露他女儿的下体,也不可娶他孙女或是外孙女,露他们的下体;他们是骨肉之亲,这本是大恶。
\par 18 你妻还在的时候,不可另娶他的姊妹作对头,露他的下体。
\par 19 「女人行经不洁净的时候,不可露他的下体,与他亲近。
\par 20 不可与邻舍的妻行淫,玷污自己。
\par 21 不可使你的儿女经火归与摩洛,也不可亵渎你神的名。我是耶和华。
\par 22 不可与男人苟合,像与女人一样;这本是可憎恶的。
\par 23 不可与兽淫合,玷污自己。女人也不可站在兽前,与他淫合;这本是逆性的事。
\par 24 「在这一切的事上,你们都不可玷污自己;因为我在你们面前所逐出的列邦,在这一切的事上玷污了自己;
\par 25 连地也玷污了,所以我追讨那地的罪孽,那地也吐出他的居民。
\par 26 故此,你们要守我的律例典章。这一切可憎恶的事,无论是本地人,是寄居在你们中间的外人,都不可行,
\par 27 (在你们以先居住那地的人行了这一切可憎恶的事,地就玷污了,)
\par 28 免得你们玷污那地的时候,地就把你们吐出,像吐出在你们以先的国民一样。
\par 29 无论什麽人,行了其中可憎的一件事,必从民中剪除。
\par 30 所以,你们要守我所吩咐的,免得你们随从那些可憎的恶俗,就是在你们以先的人所常行的,以致玷污了自己。我是耶和华你们的神。」

\chapter{19}

\par 1 耶和华对摩西说:
\par 2 「你晓谕以色列全会众说:你们要圣洁,因为我耶和华你们的神是圣洁的。
\par 3 你们各人都当孝敬父母,也要守我的安息日。我是耶和华你们的神。
\par 4 你们不可偏向虚无的神,也不可为自己铸造神像。我是耶和华你们的神。
\par 5 「你们献平安祭给耶和华的时候,要献得可蒙悦纳。
\par 6 这祭物要在献的那一天和第二天吃,若有剩到第三天的,就必用火焚烧。
\par 7 第三天若再吃,这就为可憎恶的,必不蒙悦纳。
\par 8 凡吃的人必担当他的罪孽;因为他亵渎了耶和华的圣物,那人必从民中剪除。
\par 9 「在你们的地收割庄稼,不可割尽田角,也不可拾取所遗落的。
\par 10 不可摘尽葡萄园的果子,也不可拾取葡萄园所掉的果子;要留给穷人和寄居的。我是耶和华你们的神。
\par 11 「你们不可偷盗,不可欺骗,也不可彼此说谎。
\par 12 不可指著我的名起假誓,亵渎你神的名。我是耶和华。
\par 13 「不可欺压你的邻舍,也不可抢夺他的物。雇工人的工价,不可在你那里过夜,留到早晨。
\par 14 不可咒骂聋子,也不可将绊脚石放在瞎子面前,只要敬畏你的神。我是耶和华。
\par 15 「你们施行审判,不可行不义;不可偏护穷人,也不可重看有势力的人,只要按著公义审判你的邻舍。
\par 16 不可在民中往来搬弄是非,也不可与邻舍为敌,置之於死(原文作流他的血)。我是耶和华。
\par 17 「不可心里恨你的弟兄;总要指摘你的邻舍,免得因他担罪。
\par 18 不可报仇,也不可埋怨你本国的子民,却要爱人如己。我是耶和华。
\par 19 「你们要守我的律例。不可叫你的牲畜与异类配合;不可用两样搀杂的种种你的地,也不可用两样搀杂的料做衣服穿在身上。
\par 20 「婢女许配了丈夫,还没有被赎、得释放,人若与他行淫,二人要受刑罚,却不把他们治死,因为婢女还没有得自由。
\par 21 那人要把赎愆祭,就是一只公绵羊牵到会幕门口、耶和华面前。
\par 22 祭司要用赎愆祭的羊在耶和华面前赎他所犯的罪,他的罪就必蒙赦免。
\par 23 「你们到了迦南地,栽种各样结果子的树木,就要以所结的果子如未受割礼的一样。三年之久,你们要以这些果子,如未受割礼的,是不可吃的。
\par 24 但第四年所结的果子全要成为圣,用以赞美耶和华。
\par 25 第五年,你们要吃那树上的果子,好叫树给你们结果子更多。我是耶和华你们的神。
\par 26 「你们不可吃带血的物;不可用法术,也不可观兆。
\par 27 头的周围(或作:两鬓)不可剃,胡须的周围也不可损坏。
\par 28 不可为死人用刀划身,也不可在身上刺花纹。我是耶和华。
\par 29 「不可辱没你的女儿,使他为娼妓,恐怕地上的人专向淫乱,地就满了大恶。
\par 30 你们要守我的安息日,敬我的圣所。我是耶和华。
\par 31 「不可偏向那些交鬼的和行巫术的;不可求问他们,以致被他们玷污了。我是耶和华你们的神。
\par 32 「在白发的人面前,你要站起来;也要尊敬老人,又要敬畏你的神。我是耶和华。
\par 33 「若有外人在你们国中和你同居,就不可欺负他。
\par 34 和你们同居的外人,你们要看他如本地人一样,并要爱他如己,因为你们在埃及地也作过寄居的。我是耶和华你们的神。
\par 35 「你们施行审判,不可行不义;在尺、秤、升、斗上也是如此。
\par 36 要用公道天平、公道法码、公道升斗、公道秤。我是耶和华你们的神,曾把你们从埃及地领出来的。
\par 37 你们要谨守遵行我一切的律例典章。我是耶和华。」

\chapter{20}

\par 1 耶和华对摩西说:
\par 2 「你还要晓谕以色列人说:凡以色列人,或是在以色列中寄居的外人,把自己的儿女献给摩洛的,总要治死他;本地人要用石头把他打死。
\par 3 我也要向那人变脸,把他从民中剪除;因为他把儿女献给摩洛,玷污我的圣所,亵渎我的圣名。
\par 4 那人把儿女献给摩洛,本地人若佯为不见,不把他治死,
\par 5 我就要向这人和他的家变脸,把他和一切随他与摩洛行邪淫的人都从民中剪除。
\par 6 「人偏向交鬼的和行巫术的,随他们行邪淫,我要向那人变脸,把他从民中剪除。
\par 7 所以你们要自洁成圣,因为我是耶和华你们的神。
\par 8 你们要谨守遵行我的律例;我是叫你们成圣的耶和华。
\par 9 凡咒骂父母的,总要治死他;他咒骂了父母,他的罪(原文作血;本章同)要归到他身上。
\par 10 「与邻舍之妻行淫的,奸夫淫妇都必治死。
\par 11 与继母行淫的,就是羞辱了他父亲,总要把他们二人治死,罪要归到他们身上。
\par 12 与儿妇同房的,总要把他们二人治死;他们行了逆伦的事,罪要归到他们身上。
\par 13 人若与男人苟合,像与女人一样,他们二人行了可憎的事,总要把他们治死,罪要归到他们身上。
\par 14 人若娶妻,并娶其母,便是大恶,要把这三人用火焚烧,使你们中间免去大恶。
\par 15 人若与兽淫合,总要治死他,也要杀那兽。
\par 16 女人若与兽亲近,与他淫合,你要杀那女人和那兽,总要把他们治死,罪要归到他们身上。
\par 17 「人若娶他的姊妹,无论是异母同父的,是异父同母的,彼此见了下体,这是可耻的事;他们必在本民的眼前被剪除。他露了姊妹的下体,必担当自己的罪孽。
\par 18 妇人有月经,若与他同房,露了他的下体,就是露了妇人的血源,妇人也露了自己的血源,二人必从民中剪除。
\par 19 不可露姨母或是姑母的下体,这是露了骨肉之亲的下体;二人必担当自己的罪孽。
\par 20 人若与伯叔之妻同房,就羞辱了他的伯叔;二人要担当自己的罪,必无子女而死。
\par 21 人若娶弟兄之妻,这本是污秽的事,羞辱了他的弟兄;二人必无子女。
\par 22 「所以,你们要谨守遵行我一切的律例典章,免得我领你们去住的那地把你们吐出。
\par 23 我在你们面前所逐出的国民,你们不可随从他们的风俗;因为他们行了这一切的事,所以我厌恶他们。
\par 24 但我对你们说过,你们要承受他们的地,就是我要赐给你们为业、流奶与蜜之地。我是耶和华你们的神,使你们与万民有分别的。
\par 25 所以,你们要把洁净和不洁净的禽兽分别出来;不可因我给你们分为不洁净的禽兽,或是滋生在地上的活物,使自己成为可憎恶的。
\par 26 你们要归我为圣,因为我耶和华是圣的,并叫你们与万民有分别,使你们作我的民。
\par 27 「无论男女,是交鬼的或行巫术的,总要治死他们。人必用石头把他们打死,罪要归到他们身上。」

\chapter{21}

\par 1 耶和华对摩西说:「你告诉亚伦子孙作祭司的说:祭司不可为民中的死人沾染自己,
\par 2 除非为他骨肉之亲的父母、儿女、弟兄,
\par 3 和未曾出嫁、作处女的姊妹,才可以沾染自己。
\par 4 祭司既在民中为首,就不可从俗沾染自己。
\par 5 不可使头光秃;不可剃除胡须的周围,也不可用刀划身。
\par 6 要归神为圣,不可亵渎神的名;因为耶和华的火祭,就是神的食物,是他们献的,所以他们要成为圣。
\par 7 不可娶妓女或被污的女人为妻,也不可娶被休的妇人为妻,因为祭司是归神为圣。
\par 8 所以你要使他成圣,因为他奉献你神的食物;你要以他为圣,因为我使你们成圣的耶和华是圣的。
\par 9 祭司的女儿若行淫辱没自己,就辱没了父亲,必用火将他焚烧。
\par 10 「在弟兄中作大祭司、头上倒了膏油、又承接圣职,穿了圣衣的,不可蓬头散发,也不可撕裂衣服。
\par 11 不可挨近死尸,也不可为父母沾染自己。
\par 12 不可出圣所,也不可亵渎神的圣所,因为神膏油的冠冕在他头上。我是耶和华。
\par 13 他要娶处女为妻。
\par 14 寡妇或是被休的妇人,或是被污为妓的女人,都不可娶;只可娶本民中的处女为妻。
\par 15 不可在民中辱没他的儿女,因为我是叫他成圣的耶和华。」
\par 16 耶和华对摩西说:
\par 17 「你告诉亚伦说:你世世代代的後裔,凡有残疾的,都不可近前来献他神的食物。
\par 18 因为凡有残疾的,无论是瞎眼的、瘸腿的、塌鼻子的、肢体有余的、
\par 19 折脚折手的、
\par 20 驼背的、矮矬的、眼睛有毛病的、长癣的、长疥的,或是损坏肾子的,都不可近前来。
\par 21 祭司亚伦的後裔,凡有残疾的,都不可近前来,将火祭献给耶和华。他有残疾,不可近前来献神的食物。
\par 22 神的食物,无论是圣的,至圣的,他都可以吃。
\par 23 但不可进到幔子前,也不可就近坛前;因为他有残疾,免得亵渎我的圣所。我是叫他成圣的耶和华。」
\par 24 於是,摩西晓谕亚伦和亚伦的子孙,并以色列众人。

\chapter{22}

\par 1 耶和华对摩西说:
\par 2 「你吩咐亚伦和他子孙说:要远离以色列人所分别为圣、归给我的圣物,免得亵渎我的圣名。我是耶和华。
\par 3 你要对他们说:你们世世代代的後裔,凡身上有污秽、亲近以色列人所分别为圣、归耶和华圣物的,那人必在我面前剪除。我是耶和华。
\par 4 亚伦的後裔,凡长大麻疯的,或是有漏症的,不可吃圣物,直等他洁净了。无论谁摸那因死尸不洁净的物(或作:人),或是遗精的人,
\par 5 或是摸什麽使他不洁净的爬物,或是摸那使他不洁净的人(不拘那人有什麽不洁净),
\par 6 摸了这些人、物的,必不洁净到晚上;若不用水洗身,就不可吃圣物。
\par 7 日落的时候,他就洁净了,然後可以吃圣物,因为这是他的食物。
\par 8 自死的或是被野兽撕裂的,他不可吃,因此污秽自己。我是耶和华。
\par 9 所以他们要守我所吩咐的,免得轻忽了,因此担罪而死。我是叫他们成圣的耶和华。
\par 10 「凡外人不可吃圣物;寄居在祭司家的,或是雇工人,都不可吃圣物;
\par 11 倘若祭司买人,是他的钱买的,那人就可以吃圣物;生在他家的人也可以吃。
\par 12 祭司的女儿若嫁外人,就不可吃举祭的圣物。
\par 13 但祭司的女儿若是寡妇,或是被休的,没有孩子,又归回父家,与他青年一样,就可以吃他父亲的食物;只是外人不可吃。
\par 14 若有人误吃了圣物,要照圣物的原数加上五分之一交给祭司。
\par 15 祭司不可亵渎以色列人所献给耶和华的圣物,
\par 16 免得他们在吃圣物上自取罪孽,因为我是叫他们成圣的耶和华。」
\par 17 耶和华对摩西说:
\par 18 「你晓谕亚伦和他子孙,并以色列众人说:以色列家中的人,或在以色列中寄居的,凡献供物,无论是所许的愿,是甘心献的,就是献给耶和华作燔祭的,
\par 19 要将没有残疾的公牛,或是绵羊,或是山羊献上,如此方蒙悦纳。
\par 20 凡有残疾的,你们不可献上,因为这不蒙悦纳。
\par 21 凡从牛群或是羊群中,将平安祭献给耶和华,为要还特许的愿,或是作甘心献的,所献的必纯全无残疾的才蒙悦纳。
\par 22 瞎眼的、折伤的、残废的、有瘤子的、长癣的、长疥的都不可献给耶和华,也不可在坛上作为火祭献给耶和华。
\par 23 无论是公牛是绵羊羔,若肢体有余的,或是缺少的,只可作甘心祭献上;用以还愿,却不蒙悦纳。
\par 24 肾子损伤的,或是压碎的,或是破裂的,或是骟了的,不可献给耶和华,在你们的地上也不可这样行。
\par 25 这类的物,你们从外人的手,一样也不可接受作你们神的食物献上;因为这些都有损坏,有残疾,不蒙悦纳。」
\par 26 耶和华晓谕摩西说:
\par 27 「才生的公牛,或是绵羊或是山羊,七天当跟著母;从第八天以後,可以当供物蒙悦纳,作为耶和华的火祭。
\par 28 无论是母牛是母羊,不可同日宰母和子。
\par 29 你们献感谢祭给耶和华,要献得可蒙悦纳。
\par 30 要当天吃,一点不可留到早晨。我是耶和华。
\par 31 「你们要谨守遵行我的诫命。我是耶和华。
\par 32 你们不可亵渎我的圣名;我在以色列人中,却要被尊为圣。我是叫你们成圣的耶和华,
\par 33 把你们从埃及地领出来,作你们的神。我是耶和华。」

\chapter{23}

\par 1 耶和华对摩西说:
\par 2 「你晓谕以色列人说:耶和华的节期,你们要宣告为圣会的节期。
\par 3 六日要做工,第七日是圣安息日,当有圣会;你们什麽工都不可做。这是在你们一切的住处向耶和华守的安息日。
\par 4 耶和华的节期,就是你们到了日期要宣告为圣会的,乃是这些。」
\par 5 「正月十四日,黄昏的时候,是耶和华的逾越节。
\par 6 这月十五日是向耶和华守的无酵节;你们要吃无酵饼七日。
\par 7 第一日当有圣会,什麽劳碌的工都不可做;
\par 8 要将火祭献给耶和华七日。第七日是圣会,什麽劳碌的工都不可做。」
\par 9 耶和华对摩西说:
\par 10 「你晓谕以色列人说:你们到了我赐给你们的地,收割庄稼的时候,要将初熟的庄稼一捆带给祭司。
\par 11 他要把这一捆在耶和华面前摇一摇,使你们得蒙悦纳。祭司要在安息日的次日把这捆摇一摇。
\par 12 摇这捆的日子,你们要把一岁、没有残疾的公绵羊羔献给耶和华为燔祭。
\par 13 同献的素祭,就是调油的细面伊法十分之二,作为馨香的火祭,献给耶和华。同献的奠祭,要酒一欣四分之一。
\par 14 无论是饼,是烘的子粒,是新穗子,你们都不可吃,直等到把你们献给神的供物带来的那一天才可以吃。这在你们一切的住处作为世世代代永远的定例。」
\par 15 「你们要从安息日的次日,献禾捆为摇祭的那日算起,要满了七个安息日。
\par 16 到第七个安息日的次日,共计五十天,又要将新素祭献给耶和华。
\par 17 要从你们的住处取出细面伊法十分之二,加酵,烤成两个摇祭的饼,当作初熟之物献给耶和华。
\par 18 又要将一岁、没有残疾的羊羔七只、公牛犊一只、公绵羊两只,和饼一同奉上。这些与同献的素祭和奠祭要作为燔祭献给耶和华,就是作馨香的火祭献给耶和华。
\par 19 你们要献一只公山羊为赎罪祭,两只一岁的公绵羊羔为平安祭。
\par 20 祭司要把这些和初熟麦子做的饼一同作摇祭,在耶和华面前摇一摇;这是献与耶和华为圣物归给祭司的。
\par 21 当这日,你们要宣告圣会;什麽劳碌的工都不可做。这在你们一切的住处作为世世代代永远的定例。
\par 22 「在你们的地收割庄稼,不可割尽田角,也不可拾取所遗落的;要留给穷人和寄居的。我是耶和华你们的神。」
\par 23 耶和华对摩西说:
\par 24 「你晓谕以色列人说:七月初一,你们要守为圣安息日,要吹角作纪念,当有圣会。
\par 25 什麽劳碌的工都不可做;要将火祭献给耶和华。」
\par 26 耶和华晓谕摩西说:
\par 27 「七月初十是赎罪日;你们要守为圣会,并要刻苦己心,也要将火祭献给耶和华。
\par 28 当这日,什麽工都不可做;因为是赎罪日,要在耶和华你们的神面前赎罪。
\par 29 当这日,凡不刻苦己心的,必从民中剪除。
\par 30 凡这日做什麽工的,我必将他从民中除灭。
\par 31 你们什麽工都不可做。这在你们一切的住处作为世世代代永远的定例。
\par 32 你们要守这日为圣安息日,并要刻苦己心。从这月初九日晚上到次日晚上,要守为安息日。」
\par 33 耶和华对摩西说:
\par 34 「你晓谕以色列人说:这七月十五日是住棚节,要在耶和华面前守这节七日。
\par 35 第一日当有圣会,什麽劳碌的工都不可做。
\par 36 七日内要将火祭献给耶和华。第八日当守圣会,要将火祭献给耶和华。这是严肃会,什麽劳碌的工都不可做。
\par 37 「这是耶和华的节期,就是你们要宣告为圣会的节期;要将火祭、燔祭、素祭、祭物,并奠祭,各归各日,献给耶和华。
\par 38 这是在耶和华的安息日以外,又在你们的供物和所许的愿,并甘心献给耶和华的以外。
\par 39 「你们收藏了地的出产,就从七月十五日起,要守耶和华的节七日。第一日为圣安息;第八日也为圣安息。
\par 40 第一日要拿美好树上的果子和棕树上的枝子,与茂密树的枝条并河旁的柳枝,在耶和华你们的神面前欢乐七日。
\par 41 每年七月间,要向耶和华守这节七日。这为你们世世代代永远的定例。
\par 42 你们要住在棚里七日;凡以色列家的人都要住在棚里,
\par 43 好叫你们世世代代知道,我领以色列人出埃及地的时候曾使他们住在棚里。我是耶和华你们的神。」
\par 44 於是,摩西将耶和华的节期传给以色列人。

\chapter{24}

\par 1 耶和华晓谕摩西说:
\par 2 「要吩咐以色列人,把那为点灯捣成的清橄榄油拿来给你,使灯常常点著。
\par 3 在会幕中法柜的幔子外,亚伦从晚上到早晨必在耶和华面前经理这灯。这要作你们世世代代永远的定例。
\par 4 他要在耶和华面前常收拾精金灯台上的灯。」
\par 5 「你要取细面,烤成十二个饼,每饼用面伊法十分之二。
\par 6 要把饼摆列两行(或作:摞;下同),每行六个,在耶和华面前精金的桌子上;
\par 7 又要把净乳香放在每行饼上,作为纪念,就是作为火祭献给耶和华。
\par 8 每安息日要常摆在耶和华面前;这为以色列人作永远的约。
\par 9 这饼是要给亚伦和他子孙的;他们要在圣处吃,为永远的定例,因为在献给耶和华的火祭中是至圣的。」
\par 10 有一个以色列妇人的儿子,他父亲是埃及人,一日闲游在以色列人中。这以色列妇人的儿子和一个以色列人在营里争斗。
\par 11 这以色列妇人的儿子亵渎了圣名,并且咒诅,就有人把他送到摩西那里。(他母亲名叫示罗密,是但支派底伯利的女儿。)
\par 12 他们把那人收在监里,要得耶和华所指示的话。
\par 13 耶和华晓谕摩西说:
\par 14 「把那咒诅圣名的人带到营外。叫听见的人都放手在他头上;全会众就要用石头打死他。
\par 15 你要晓谕以色列人说:凡咒诅神的,必担当他的罪。
\par 16 那亵渎耶和华名的,必被治死;全会众总要用石头打死他。不管是寄居的是本地人,他亵渎耶和华名的时候,必被治死。
\par 17 打死人的,必被治死;
\par 18 打死牲畜的,必赔上牲畜,以命偿命。
\par 19 人若使他邻舍的身体有残疾,他怎样行,也要照样向他行:
\par 20 以伤还伤,以眼还眼,以牙还牙。他怎样叫人的身体有残疾,也要照样向他行。
\par 21 打死牲畜的,必赔上牲畜;打死人的,必被治死。
\par 22 不管是寄居的是本地人,同归一例。我是耶和华你们的神。」
\par 23 於是,摩西晓谕以色列人,他们就把那咒诅圣名的人带到营外,用石头打死。以色列人就照耶和华所吩咐摩西的行了。

\chapter{25}

\par 1 耶和华在西乃山对摩西说:
\par 2 「你晓谕以色列人说:你们到了我所赐你们那地的时候,地就要向耶和华守安息。
\par 3 六年要耕种田地,也要修理葡萄园,收藏地的出产。
\par 4 第七年,地要守圣安息,就是向耶和华守的安息,不可耕种田地,也不可修理葡萄园。
\par 5 遗落自长的庄稼不可收割;没有修理的葡萄树也不可摘取葡萄。这年,地要守圣安息。
\par 6 地在安息年所出的,要给你和你的仆人、婢女、雇工人,并寄居的外人当食物。
\par 7 这年的土产也要给你的牲畜和你地上的走兽当食物。」
\par 8 「你要计算七个安息年,就是七七年。这便为你成了七个安息年,共是四十九年。
\par 9 当年七月初十日,你要大发角声;这日就是赎罪日,要在遍地发出角声。
\par 10 第五十年,你们要当作圣年,在遍地给一切的居民宣告自由。这年必为你们的禧年,各人要归自己的产业,各归本家。
\par 11 第五十年要作为你们的禧年。这年不可耕种;地中自长的,不可收割;没有修理的葡萄树也不可摘取葡萄。
\par 12 因为这是禧年,你们要当作圣年,吃地中自出的土产。
\par 13 「这禧年,你们各人要归自己的地业。
\par 14 你若卖什麽给邻舍,或是从邻舍的手中买什麽,彼此不可亏负。
\par 15 你要按禧年以後的年数向邻舍买;他也要按年数的收成卖给你。
\par 16 年岁若多,要照数加添价值;年岁若少,要照数减去价值,因为他照收成的数目卖给你。
\par 17 你们彼此不可亏负,只要敬畏你们的神,因为我是耶和华你们的神。」
\par 18 「我的律例,你们要遵行,我的典章,你们要谨守,就可以在那地上安然居住。
\par 19 地必出土产,你们就要吃饱,在那地上安然居住。
\par 20 你们若说:『这第七年我们不耕种,也不收藏土产,吃什麽呢?』
\par 21 我必在第六年将我所命的福赐给你们,地便生三年的土产。
\par 22 第八年,你们要耕种,也要吃陈粮,等到第九年出产收来的时候,你们还吃陈粮。」
\par 23 「地不可永卖,因为地是我的;你们在我面前是客旅,是寄居的。
\par 24 在你们所得为业的全地,也要准人将地赎回。
\par 25 你的弟兄(弟兄是指本国人说;下同)若渐渐穷乏,卖了几分地业,他至近的亲属就要来把弟兄所卖的赎回。
\par 26 若没有能给他赎回的,他自己渐渐富足,能够赎回,
\par 27 就要算出卖地的年数,把余剩年数的价值还那买主,自己便归回自己的地业。
\par 28 倘若不能为自己得回所卖的,仍要存在买主的手里直到禧年;到了禧年,地业要出买主的手,自己便归回自己的地业。
\par 29 「人若卖城内的住宅,卖了以後,一年之内可以赎回;在一整年,必有赎回的权柄。
\par 30 若在一整年之内不赎回,这城内的房屋就定准永归买主,世世代代为业;在禧年也不得出买主的手。
\par 31 但房屋在无城墙的村庄里,要看如乡下的田地一样,可以赎回;到了禧年,都要出买主的手。
\par 32 然而利未人所得为业的城邑,其中的房屋,利未人可以随时赎回。
\par 33 若是一个利未人不将所卖的房屋赎回,是在所得为业的城内,到了禧年就要出买主的手,因为利未人城邑的房屋是他们在以色列人中的产业。
\par 34 只是他们各城郊野之地不可卖,因为是他们永远的产业。」
\par 35 「你的弟兄在你那里若渐渐贫穷,手中缺乏,你就要帮补他,使他与你同住,像外人和寄居的一样。
\par 36 不可向他取利,也不可向他多要;只要敬畏你的神,使你的弟兄与你同住。
\par 37 你借钱给他,不可向他取利;借粮给他,也不可向他多要。
\par 38 我是耶和华你们的神,曾领你们从埃及地出来,为要把迦南地赐给你们,要作你们的神。」
\par 39 「你的弟兄若在你那里渐渐穷乏,将自己卖给你,不可叫他像奴仆服事你。
\par 40 他要在你那里像雇工人和寄居的一样,要服事你直到禧年。
\par 41 到了禧年,他和他儿女要离开你,一同出去归回本家,到他祖宗的地业那里去。
\par 42 因为他们是我的仆人,是我从埃及地领出来的,不可卖为奴仆。
\par 43 不可严严地辖管他,只要敬畏你的神。
\par 44 至於你的奴仆、婢女,可以从你四围的国中买。
\par 45 并且那寄居在你们中间的外人和他们的家属,在你们地上所生的,你们也可以从其中买人;他们要作你们的产业。
\par 46 你们要将他们遗留给你们的子孙为产业,要永远从他们中间拣出奴仆;只是你们的弟兄以色列人,你们不可严严地辖管。
\par 47 「住在你那里的外人,或是寄居的,若渐渐富足,你的弟兄却渐渐穷乏,将自己卖给那外人,或是寄居的,或是外人的宗族,
\par 48 卖了以後,可以将他赎回。无论是他的弟兄,
\par 49 或伯叔、伯叔的儿子,本家的近支,都可以赎他。他自己若渐渐富足,也可以自赎。
\par 50 他要和买主计算,从卖自己的那年起,算到禧年;所卖的价值照著年数多少,好像工人每年的工价。
\par 51 若缺少的年数多,就要按著年数从买价中偿还他的赎价。
\par 52 若到禧年只缺少几年,就要按著年数和买主计算,偿还他的赎价。
\par 53 他和买主同住,要像每年雇的工人,买主不可严严地辖管他。
\par 54 他若不这样被赎,到了禧年,要和他的儿女一同出去。
\par 55 因为以色列人都是我的仆人,是我从埃及地领出来的。我是耶和华你们的神。」

\chapter{26}

\par 1 「你们不可做什麽虚无的神像,不可立雕刻的偶像或是柱像,也不可在你们的地上安什麽錾成的石像,向他跪拜,因为我是耶和华你们的神。
\par 2 你们要守我的安息日,敬我的圣所。我是耶和华。
\par 3 「你们若遵行我的律例,谨守我的诫命,
\par 4 我就给你们降下时雨,叫地生出土产,田野的树木结果子。
\par 5 你们打粮食要打到摘葡萄的时候,摘葡萄要摘到撒种的时候;并且要吃得饱足,在你们的地上安然居住。
\par 6 我要赐平安在你们的地上;你们躺卧,无人惊吓。我要叫恶兽从你们的地上息灭;刀剑也必不经过你们的地。
\par 7 你们要追赶仇敌,他们必倒在你们刀下。
\par 8 你们五个人要追赶一百人,一百人要追赶一万人;仇敌必倒在你们刀下。
\par 9 我要眷顾你们,使你们生养众多,也要与你们坚定所立的约。
\par 10 你们要吃陈粮,又因新粮挪开陈粮。
\par 11 我要在你们中间立我的帐幕;我的心也不厌恶你们。
\par 12 我要在你们中间行走;我要作你们的神,你们要作我的子民。
\par 13 我是耶和华你们的神,曾将你们从埃及地领出来,使你们不作埃及人的奴仆;我也折断你们所负的轭,叫你们挺身而走。」
\par 14 「你们若不听从我,不遵行我的诫命,
\par 15 厌弃我的律例,厌恶我的典章,不遵行我一切的诫命,背弃我的约,
\par 16 我待你们就要这样:我必命定惊惶,叫眼目乾瘪、精神消耗的痨病热病辖制你们。你们也要白白地撒种,因为仇敌要吃你们所种的。
\par 17 我要向你们变脸,你们就要败在仇敌面前。恨恶你们的,必辖管你们;无人追赶,你们却要逃跑。
\par 18 你们因这些事若还不听从我,我就要为你们的罪加七倍惩罚你们。
\par 19 我必断绝你们因势力而有的骄傲,又要使覆你们的天如铁,载你们的地如铜。
\par 20 你们要白白地劳力;因为你们的地不出土产,其上的树木也不结果子。
\par 21 「你们行事若与我反对,不肯听从我,我就要按你们的罪加七倍降灾与你们。
\par 22 我也要打发野地的走兽到你们中间,抢吃你们的儿女,吞灭你们的牲畜,使你们的人数减少,道路荒凉。
\par 23 「你们因这些事若仍不改正归我,行事与我反对,
\par 24 我就要行事与你们反对,因你们的罪击打你们七次。
\par 25 我又要使刀剑临到你们,报复你们背约的仇;聚集你们在各城内,降瘟疫在你们中间,也必将你们交在仇敌的手中。
\par 26 我要折断你们的杖,就是断绝你们的粮。那时,必有十个女人在一个炉子给你们烤饼,按分量秤给你们;你们要吃,也吃不饱。
\par 27 「你们因这一切的事若不听从我,却行事与我反对,
\par 28 我就要发烈怒,行事与你们反对,又因你们的罪惩罚你们七次。
\par 29 并且你们要吃儿子的肉,也要吃女儿的肉。
\par 30 我又要毁坏你们的邱坛,砍下你们的日像,把你们的尸首扔在你们偶像的身上;我的心也必厌恶你们。
\par 31 我要使你们的城邑变为荒凉,使你们的众圣所成为荒场;我也不闻你们馨香的香气。
\par 32 我要使地成为荒场,住在其上的仇敌就因此诧异。
\par 33 我要把你们散在列邦中;我也要拔刀追赶你们。你们的地要成为荒场;你们的城邑要变为荒凉。
\par 34 「你们在仇敌之地居住的时候,你们的地荒凉,要享受众安息;正在那时候,地要歇息,享受安息。
\par 35 地多时为荒场,就要多时歇息;地这样歇息,是你们住在其上的安息年所不能得的。
\par 36 至於你们剩下的人,我要使他们在仇敌之地心惊胆怯。叶子被风吹的响声,要追赶他们;他们要逃避,像人逃避刀剑,无人追赶,却要跌倒。
\par 37 无人追赶,他们要彼此撞跌,像在刀剑之前。你们在仇敌面前也必站立不住。
\par 38 你们要在列邦中灭亡;仇敌之地要吞吃你们。
\par 39 你们剩下的人必因自己的罪孽和祖宗的罪孽在仇敌之地消灭。
\par 40 「他们要承认自己的罪和他们祖宗的罪,就是干犯我的那罪,并且承认自己行事与我反对,
\par 41 我所以行事与他们反对,把他们带到仇敌之地。那时,他们未受割礼的心若谦卑了,他们也服了罪孽的刑罚,
\par 42 我就要记念我与雅各所立的约,与以撒所立的约,与亚伯拉罕所立的约,并要记念这地。
\par 43 他们离开这地,地在荒废无人的时候就要享受安息。并且他们要服罪孽的刑罚;因为他们厌弃了我的典章,心中厌恶了我的律例。
\par 44 虽是这样,他们在仇敌之地,我却不厌弃他们,也不厌恶他们,将他们尽行灭绝,也不背弃我与他们所立的约,因为我是耶和华他们的神。
\par 45 我却要为他们的缘故记念我与他们先祖所立的约。他们的先祖是我在列邦人眼前、从埃及地领出来的,为要作他们的神。我是耶和华。」
\par 46 这些律例、典章,和法度是耶和华与以色列人在西乃山藉著摩西立的。

\chapter{27}

\par 1 耶和华对摩西说:
\par 2 「你晓谕以色列人说:人还特许的愿,被许的人要按你所估的价值归给耶和华。
\par 3 你估定的,从二十岁到六十岁的男人,要按圣所的平,估定价银五十舍客勒。
\par 4 若是女人,你要估定三十舍客勒。
\par 5 若是从五岁到二十岁,男子你要估定二十舍客勒,女子估定十舍客勒。
\par 6 若是从一月到五岁,男子你要估定五舍客勒,女子估定三舍客勒。
\par 7 若是从六十岁以上,男人你要估定十五舍客勒,女人估定十舍客勒。
\par 8 他若贫穷,不能照你所估定的价,就要把他带到祭司面前,祭司要按许愿人的力量估定他的价。
\par 9 「所许的若是牲畜,就是人献给耶和华为供物的,凡这一类献给耶和华的,都要成为圣。
\par 10 人不可改换,也不可更换,或是好的换坏的,或是坏的换好的。若以牲畜更换牲畜,所许的与所换的都要成为圣。
\par 11 若牲畜不洁净,是不可献给耶和华为供物的,就要把牲畜安置在祭司面前。
\par 12 祭司就要估定价值;牲畜是好是坏,祭司怎样估定,就要以怎样为是。
\par 13 他若一定要赎回,就要在你所估定的价值以外加上五分之一。
\par 14 「人将房屋分别为圣,归给耶和华,祭司就要估定价值。房屋是好是坏,祭司怎样估定,就要以怎样为定。
\par 15 将房屋分别为圣的人,若要赎回房屋,就必在你所估定的价值以外加上五分之一,房屋仍旧归他。
\par 16 「人若将承受为业的几分地分别为圣,归给耶和华,你要按这地撒种多少估定价值,若撒大麦一贺梅珥,要估价五十舍客勒。
\par 17 他若从禧年将地分别为圣,就要以你所估定的价为定。
\par 18 倘若他在禧年以後将地分别为圣,祭司就要按著未到禧年所剩的年数推算价值,也要从你所估的减去价值。
\par 19 将地分别为圣的人若定要把地赎回,他便要在你所估的价值以外加上五分之一,地就准定归他。
\par 20 他若不赎回那地,或是将地卖给别人,就再不能赎了。
\par 21 但到了禧年,那地从买主手下出来的时候,就要归耶和华为圣,和永献的地一样,要归祭司为业。
\par 22 他若将所买的一块地,不是承受为业的,分别为圣归给耶和华,
\par 23 祭司就要将你所估的价值给他推算到禧年。当日,他要以你所估的价银为圣,归给耶和华。
\par 24 到了禧年,那地要归卖主,就是那承受为业的原主。
\par 25 凡你所估定的价银都要按著圣所的平:二十季拉为一舍客勒。
\par 26 「惟独牲畜中头生的,无论是牛是羊,既归耶和华,谁也不可再分别为圣,因为这是耶和华的。
\par 27 若是不洁净的牲畜生的,就要按你所估定的价值加上五分之一赎回;若不赎回,就要按你所估定的价值卖了。
\par 28 「但一切永献的,就是人从他所有永献给耶和华的,无论是人,是牲畜,是他承受为业的地,都不可卖,也不可赎。凡永献的是归给耶和华为至圣。
\par 29 凡从人中当灭的都不可赎,必被治死。」
\par 30 「地上所有的,无论是地上的种子是树上的果子,十分之一是耶和华的,是归给耶和华为圣的。
\par 31 人若要赎这十分之一的什麽物,就要加上五分之一。
\par 32 凡牛群羊群中,一切从杖下经过的,每第十只要归给耶和华为圣。
\par 33 不可问是好是坏,也不可更换;若定要更换,所更换的与本来的牲畜都要成为圣,不可赎回。」
\par 34 这就是耶和华在西乃山为以色列人所吩咐摩西的命令。


\end{document}