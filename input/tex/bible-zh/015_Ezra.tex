\begin{document}

\title{以斯拉记}


\chapter{1}

\par 1 波斯王古列元年,耶和华为要应验藉耶利米口所说的话,就激动波斯王古列的心,使他下诏通告全国说:
\par 2 「波斯王古列如此说:『耶和华天上的神已将天下万国赐给我,又嘱咐我在犹大的耶路撒冷为他建造殿宇。
\par 3 在你们中间凡作他子民的,可以上犹大的耶路撒冷,在耶路撒冷重建耶和华以色列神的殿(只有他是神)。愿神与这人同在。
\par 4 凡剩下的人,无论寄居何处,那地的人要用金银、财物、牲畜帮助他,另外也要为耶路撒冷神的殿甘心献上礼物。』」
\par 5 於是,犹大和便雅悯的族长、祭司、利未人,就是一切被神激动他心的人,都起来要上耶路撒冷去建造耶和华的殿。
\par 6 他们四围的人就拿银器、金子、财物、牲畜、珍宝帮助他们(原文作坚固他们的手),另外还有甘心献的礼物。
\par 7 古列王也将耶和华殿的器皿拿出来,这器皿是尼布甲尼撒从耶路撒冷掠来、放在自己神之庙中的。
\par 8 波斯王古列派库官米提利达将这器皿拿出来,按数交给犹大的首领设巴萨。
\par 9 器皿的数目记在下面:金盘三十个,银盘一千个,刀二十九把,
\par 10 金碗三十个,银碗之次的四百一十个,别样的器皿一千件。
\par 11 金银器皿共有五千四百件。被掳的人从巴比伦上耶路撒冷的时候,设巴萨将这一切都带上来。

\chapter{2}

\par 1 巴比伦王尼布甲尼撒从前掳到巴比伦之犹大省的人,现在他们的子孙从被掳到之地回耶路撒冷和犹大,各归本城。
\par 2 他们是同著所罗巴伯、耶书亚、尼希米、西莱雅、利来雅、末底改、必珊、米斯拔、比革瓦伊、利宏、巴拿回来的。
\par 3 以色列人民的数目记在下面:巴录的子孙二千一百七十二名;
\par 4 示法提雅的子孙三百七十二名;
\par 5 亚拉的子孙七百七十五名;
\par 6 巴哈摩押的後裔,就是耶书亚和约押的子孙二千八百一十二名;
\par 7 以拦的子孙一千二百五十四名;
\par 8 萨土的子孙九百四十五名;
\par 9 萨改的子孙七百六十名;
\par 10 巴尼的子孙六百四十二名;
\par 11 比拜的子孙六百二十三名;
\par 12 押甲的子孙一千二百二十二名;
\par 13 亚多尼干的子孙六百六十六名;
\par 14 比革瓦伊的子孙二千零五十六名;
\par 15 亚丁的子孙四百五十四名;
\par 16 亚特的後裔,就是希西家的子孙九十八名;
\par 17 比赛的子孙三百二十三名;
\par 18 约拉的子孙一百一十二名;
\par 19 哈顺的子孙二百二十三名;
\par 20 吉罢珥人九十五名;
\par 21 伯利恒人一百二十三名;
\par 22 尼陀法人五十六名;
\par 23 亚拿突人一百二十八名;
\par 24 亚斯玛弗人四十二名;
\par 25 基列耶琳人、基非拉人、比录人共七百四十三名;
\par 26 拉玛人、迦巴人共六百二十一名;
\par 27 默玛人一百二十二名;
\par 28 伯特利人、艾人共二百二十三名;
\par 29 尼波人五十二名;
\par 30 末必人一百五十六名;
\par 31 别的以拦子孙一千二百五十四名;
\par 32 哈琳的子孙三百二十名;
\par 33 罗德人、哈第人、阿挪人共七百二十五名;
\par 34 耶利哥人三百四十五名;
\par 35 西拿人三千六百三十名。
\par 36 祭司:耶书亚家耶大雅的子孙九百七十三名;
\par 37 音麦的子孙一千零五十二名;
\par 38 巴施户珥的子孙一千二百四十七名;
\par 39 哈琳的子孙一千零一十七名。
\par 40 利未人:何达威雅的後裔,就是耶书亚和甲篾的子孙七十四名。
\par 41 歌唱的:亚萨的子孙一百二十八名。
\par 42 守门的:沙龙的子孙、亚特的子孙、达们的子孙、亚谷的子孙、哈底大的子孙、朔拜的子孙,共一百三十九名。
\par 43 尼提宁(就是殿役):西哈的子孙、哈苏巴的子孙、答巴俄的子孙、
\par 44 基绿的子孙、西亚的子孙、巴顿的子孙、
\par 45 利巴拿的子孙、哈迦巴的子孙、亚谷的子孙、
\par 46 哈甲的子孙、萨买的子孙、哈难的子孙、
\par 47 吉德的子孙、迦哈的子孙、利亚雅的子孙、
\par 48 利汛的子孙、尼哥大的子孙、迦散的子孙、
\par 49 乌撒的子孙、巴西亚的子孙、比赛的子孙、
\par 50 押拿的子孙、米乌宁的子孙、尼普心的子孙、
\par 51 巴卜的子孙、哈古巴的子孙、哈忽的子孙、
\par 52 巴洗律的子孙、米希大的子孙、哈沙的子孙、
\par 53 巴柯的子孙、西西拉的子孙、答玛的子孙、
\par 54 尼细亚的子孙、哈提法的子孙。
\par 55 所罗门仆人的後裔,就是琐太的子孙、琐斐列的子孙、比路大的子孙、
\par 56 雅拉的子孙、达昆的子孙、吉德的子孙、
\par 57 示法提雅的子孙、哈替的子孙、玻黑列哈斯巴音的子孙、亚米的子孙。
\par 58 尼提宁和所罗门仆人的後裔共三百九十二名。
\par 59 从特米拉、特哈萨、基绿、押但、音麦上来的,不能指明他们的宗族谱系是以色列人不是;
\par 60 他们是第来雅的子孙、多比雅的子孙、尼哥大的子孙,共六百五十二名。
\par 61 祭司中,哈巴雅的子孙、哈哥斯的子孙、巴西莱的子孙;因为他们的先祖娶了基列人巴西莱的女儿为妻,所以起名叫巴西莱。
\par 62 这三家的人在族谱之中寻查自己的谱系,却寻不著,因此算为不洁,不准供祭司的职任。
\par 63 省长对他们说:「不可吃至圣的物,直到有用乌陵和土明决疑的祭司兴起来。」
\par 64 会众共有四万二千三百六十名。
\par 65 此外,还有他们的仆婢七千三百三十七名,又有歌唱的男女二百名。
\par 66 他们有马七百三十六匹,骡子二百四十五匹,
\par 67 骆驼四百三十五只,驴六千七百二十匹。
\par 68 有些族长到了耶路撒冷耶和华殿的地方,便为神的殿甘心献上礼物,要重新建造。
\par 69 他们量力捐入工程库的金子六万一千达利克,银子五千弥拿,并祭司的礼服一百件。
\par 70 於是祭司、利未人、民中的一些人、歌唱的、守门的、尼提宁,并以色列众人,各住在自己的城里。

\chapter{3}

\par 1 到了七月,以色列人住在各城;那时他们如同一人,聚集在耶路撒冷。
\par 2 约萨达的儿子耶书亚和他的弟兄众祭司,并撒拉铁的儿子所罗巴伯与他的弟兄,都起来建筑以色列神的坛,要照神人摩西律法书上所写的,在坛上献燔祭。
\par 3 他们在原有的根基上筑坛,因惧怕邻国的民,又在其上向耶和华早晚献燔祭,
\par 4 又照律法书上所写的守住棚节,按数照例献每日所当献的燔祭;
\par 5 其後献常献的燔祭,并在月朔与耶和华的一切圣节献祭,又向耶和华献各人的甘心祭。
\par 6 从七月初一日起,他们就向耶和华献燔祭。但耶和华殿的根基尚未立定。
\par 7 他们又将银子给石匠、木匠,把粮食、酒、油给西顿人、推罗人,使他们将香柏树从利巴嫩运到海里,浮海运到约帕,是照波斯王古列所允准的。
\par 8 百姓到了耶路撒冷神殿的地方。第二年二月,撒拉铁的儿子所罗巴伯,约萨达的儿子耶书亚和其余的弟兄,就是祭司、利未人,并一切被掳归回耶路撒冷的人,都兴工建造;又派利未人,从二十岁以外的,督理建造耶和华殿的工作。
\par 9 於是犹大(在二章四十节作何达威雅)的後裔,就是耶书亚和他的子孙与弟兄,甲篾和他的子孙,利未人希拿达的子孙与弟兄,都一同起来,督理那在神殿做工的人。
\par 10 匠人立耶和华殿根基的时候,祭司皆穿礼服吹号,亚萨的子孙利未人敲钹,照以色列王大卫所定的例,都站著赞美耶和华。
\par 11 他们彼此唱和,赞美称谢耶和华说:他本为善,他向以色列人永发慈爱。他们赞美耶和华的时候,众民大声呼喊,因耶和华殿的根基已经立定。
\par 12 然而有许多祭司、利未人、族长,就是见过旧殿的老年人,现在亲眼看见立这殿的根基,便大声哭号,也有许多人大声欢呼,
\par 13 甚至百姓不能分辨欢呼的声音和哭号的声音;因为众人大声呼喊,声音听到远处。

\chapter{4}

\par 1 犹大和便雅悯的敌人听说被掳归回的人为耶和华以色列的神建造殿宇,
\par 2 就去见所罗巴伯和以色列的族长,对他们说:「请容我们与你们一同建造;因为我们寻求你们的神,与你们一样。自从亚述王以撒哈顿带我们上这地以来,我们常祭祀神。」
\par 3 但所罗巴伯、耶书亚,和其余以色列的族长对他们说:「我们建造神的殿与你们无干,我们自己为耶和华以色列的神协力建造,是照波斯王古列所吩咐的。」
\par 4 那地的民,就在犹大人建造的时候,使他们的手发软,扰乱他们;
\par 5 从波斯王古列年间,直到波斯王大利乌登基的时候,贿买谋士,要败坏他们的谋算。
\par 6 在亚哈随鲁才登基的时候,上本控告犹大和耶路撒冷的居民。
\par 7 亚达薛西年间,比施兰、米特利达、他别,和他们的同党上本奏告波斯王亚达薛西。本章是用亚兰文字,亚兰方言。
\par 8 省长利宏、书记伸帅要控告耶路撒冷人,也上本奏告亚达薛西王。
\par 9 省长利宏、书记伸帅,和同党的底拿人、亚法萨提迦人、他毗拉人、亚法撒人、亚基卫人、巴比伦人、书珊迦人、底亥人、以拦人,
\par 10 和尊大的亚斯那巴所迁移、安置在撒玛利亚城,并大河西一带地方的人等,
\par 11 上奏亚达薛西王说:「河西的臣民云云:
\par 12 王该知道,从王那里上到我们这里的犹大人,已经到耶路撒冷重建这反叛恶劣的城,筑立根基,建造城墙。
\par 13 如今王该知道,他们若建造这城,城墙完毕就不再与王进贡,交课,纳税,终久王必受亏损。
\par 14 我们既食御盐,不忍见王吃亏,因此奏告於王。
\par 15 请王考察先王的实录,必在其上查知这城是反叛的城,与列王和各省有害;自古以来,其中常有悖逆的事,因此这城曾被拆毁。
\par 16 我们谨奏王知,这城若再建造,城墙完毕,河西之地王就无分了。」
\par 17 那时王谕覆省长利宏、书记伸帅,和他们的同党,就是住撒玛利亚并河西一带地方的人,说:「愿你们平安云云。
\par 18 你们所上的本,已经明读在我面前。
\par 19 我已命人考查,得知此城古来果然背叛列王,其中常有反叛悖逆的事。
\par 20 从前耶路撒冷也有大君王统管河西全地,人就给他们进贡,交课,纳税。
\par 21 现在你们要出告示命这些人停工,使这城不得建造,等我降旨。
\par 22 你们当谨慎,不可迟延,为何容害加重,使王受亏损呢?」
\par 23 亚达薛西王的上谕读在利宏和书记伸帅,并他们的同党面前,他们就急忙往耶路撒冷去见犹大人,用势力强迫他们停工。
\par 24 於是,在耶路撒冷神殿的工程就停止了,直停到波斯王大利乌第二年。

\chapter{5}

\par 1 那时,先知哈该和易多的孙子撒迦利亚奉以色列神的名向犹大和耶路撒冷的犹大人说劝勉的话。
\par 2 於是撒拉铁的儿子所罗巴伯和约萨达的儿子耶书亚都起来动手建造耶路撒冷神的殿,有神的先知在那里帮助他们。
\par 3 当时河西的总督达乃和示他波斯乃,并他们的同党来问说:「谁降旨让你们建造这殿,修成这墙呢?」
\par 4 我们便告诉他们建造这殿的人叫什麽名字。
\par 5 神的眼目看顾犹大的长老,以致总督等没有叫他们停工,直到这事奏告大利乌,得著他的回谕。
\par 6 河西的总督达乃和示他波斯乃,并他们的同党,就是住河西的亚法萨迦人,上本奏告大利乌王。
\par 7 本上写著说:「愿大利乌王诸事平安。
\par 8 王该知道,我们往犹大省去,到了至大神的殿,这殿是用大石建造的。梁木插入墙内,工作甚速,他们手下亨通。
\par 9 我们就问那些长老说:『谁降旨让你们建造这殿,修成这墙呢?』
\par 10 又问他们的名字,要记录他们首领的名字,奏告於王。
\par 11 他们回答说:『我们是天地之神的仆人,重建前多年所建造的殿,就是以色列的一位大君王建造修成的。
\par 12 只因我们列祖惹天上的神发怒,神把他们交在迦勒底人巴比伦王尼布甲尼撒的手中,他就拆毁这殿,又将百姓掳到巴比伦。
\par 13 然而巴比伦王古列元年,他降旨允准建造神的这殿。
\par 14 神殿中的金、银器皿,就是尼布甲尼撒从耶路撒冷的殿中掠去带到巴比伦庙里的,古列王从巴比伦庙里取出来,交给派为省长的,名叫设巴萨,
\par 15 对他说可以将这些器皿带去,放在耶路撒冷的殿中,在原处建造神的殿。
\par 16 於是这设巴萨来建立耶路撒冷神殿的根基。这殿从那时直到如今尚未造成。』
\par 17 现在王若以为美,请察巴比伦王的府库,看古列王降旨允准在耶路撒冷建造神的殿没有,王的心意如何?请降旨晓谕我们。」

\chapter{6}

\par 1 於是大利乌王降旨,要寻察典籍库内,就是在巴比伦藏宝物之处;
\par 2 在玛代省亚马他城的宫内寻得一卷,其中记著说:
\par 3 「古列王元年,他降旨论到耶路撒冷神的殿,要建造这殿为献祭之处,坚立殿的根基。殿高六十肘,,宽六十肘,
\par 4 用三层大石头,一层新木头,经费要出於王库;
\par 5 并且神殿的金银器皿,就是尼布甲尼撒从耶路撒冷的殿中掠到巴比伦的,要归还带到耶路撒冷的殿中,各按原处放在神的殿里。」
\par 6 「现在河西的总督达乃和示他波斯乃,并你们的同党,就是住河西的亚法萨迦人,你们当远离他们。
\par 7 不要拦阻神殿的工作,任凭犹大人的省长和犹大人的长老在原处建造神的这殿。
\par 8 我又降旨,吩咐你们向犹大人的长老为建造神的殿当怎样行,就是从河西的款项中,急速拨取贡银作他们的经费,免得耽误工作。
\par 9 他们与天上的神献燔祭所需用的公牛犊、公绵羊、绵羊羔,并所用的麦子、盐、酒、油,都要照耶路撒冷祭司的话,每日供给他们,不得有误;
\par 10 好叫他们献馨香的祭给天上的神,又为王和王众子的寿命祈祷。
\par 11 我再降旨,无论谁更改这命令,必从他房屋中拆出一根梁来,把他举起,悬在其上,又使他的房屋成为粪堆。
\par 12 若有王和民伸手更改这命令,拆毁这殿,愿那使耶路撒冷的殿作为他名居所的神将他们灭绝。我大利乌降这旨意,当速速遵行。」
\par 13 於是,河西总督达乃和示他波斯乃,并他们的同党,因大利乌王所发的命令,就急速遵行。
\par 14 犹大长老因先知哈该和易多的孙子撒迦利亚所说劝勉的话就建造这殿,凡事亨通。他们遵著以色列神的命令和波斯王古列、大利乌、亚达薛西的旨意,建造完毕。
\par 15 大利乌王第六年,亚达月初三日,这殿修成了。
\par 16 以色列的祭司和利未人,并其余被掳归回的人都欢欢喜喜地行奉献神殿的礼。
\par 17 行奉献神殿的礼就献公牛一百只,公绵羊二百只,绵羊羔四百只,又照以色列支派的数目献公山羊十二只,为以色列众人作赎罪祭;
\par 18 且派祭司和利未人按著班次在耶路撒冷事奉神,是照摩西律法书上所写的。
\par 19 正月十四日,被掳归回的人守逾越节。
\par 20 原来,祭司和利未人一同自洁,无一人不洁净。利未人为被掳归回的众人和他们的弟兄众祭司,并为自己宰逾越节的羊羔。
\par 21 从掳到之地归回的以色列人和一切除掉所染外邦人污秽、归附他们、要寻求耶和华以色列神的人都吃这羊羔,
\par 22 欢欢喜喜地守除酵节七日;因为耶和华使他们欢喜,又使亚述王的心转向他们,坚固他们的手,作以色列神殿的工程。

\chapter{7}

\par 1 这事以後,波斯王亚达薛西年间,有个以斯拉,他是西莱雅的儿子,西莱雅是亚撒利雅的儿子,亚撒利雅是希勒家的儿子,
\par 2 希勒家是沙龙的儿子,沙龙是撒督的儿子,撒督是亚希突的儿子,
\par 3 亚希突是亚玛利雅的儿子,亚玛利雅是亚撒利雅的儿子,亚撒利雅是米拉约的儿子,
\par 4 米拉约是西拉希雅的儿子,西拉希雅是乌西的儿子,乌西是布基的儿子,
\par 5 布基是亚比书的儿子,亚比书是非尼哈的儿子,非尼哈是以利亚撒的儿子,以利亚撒是大祭司亚伦的儿子。
\par 6 这以斯拉从巴比伦上来,他是敏捷的文士,通达耶和华以色列神所赐摩西的律法书。王允准他一切所求的,是因耶和华他神的手帮助他。
\par 7 亚达薛西王第七年,以色列人、祭司、利未人、歌唱的、守门的、尼提宁,有上耶路撒冷的。
\par 8 王第七年五月,以斯拉到了耶路撒冷。
\par 9 正月初一日,他从巴比伦起程;因他神施恩的手帮助他,五月初一日就到了耶路撒冷。
\par 10 以斯拉定志考究遵行耶和华的律法,又将律例典章教训以色列人。
\par 11 祭司以斯拉是通达耶和华诫命和赐以色列之律例的文士。亚达薛西王赐给他们谕旨,上面写著说:
\par 12 「诸王之王亚达薛西,达於祭司以斯拉通达天上神律法大德的文士,云云。
\par 13 住在我国中的以色列人、祭司、利未人,凡甘心上耶路撒冷去的,我降旨准他们与你同去。
\par 14 王与七个谋士既然差你去,照你手中神的律法书察问犹大和耶路撒冷的景况;
\par 15 又带金银,就是王和谋士甘心献给住耶路撒冷、以色列神的,
\par 16 并带你在巴比伦全省所得的金银,和百姓、祭司乐意献给耶路撒冷他们神殿的礼物。
\par 17 所以你当用这金银,急速买公牛、公绵羊、绵羊羔,和同献的素祭奠祭之物,献在耶路撒冷你们神殿的坛上。
\par 18 剩下的金银,你和你的弟兄看著怎样好,就怎样用,总要遵著你们神的旨意。
\par 19 所交给你神殿中使用的器皿,你要交在耶路撒冷神面前。
\par 20 你神殿里若再有需用的经费,你可以从王的府库里支取。
\par 21 「我亚达薛西王又降旨与河西的一切库官,说:『通达天上神律法的文士祭司以斯拉,无论向你们要什麽,你们要速速地备办,
\par 22 就是银子直到一百他连得,麦子一百柯珥,酒一百罢特,油一百罢特,盐不计其数,也要给他。
\par 23 凡天上之神所吩咐的,当为天上神的殿详细办理。为何使忿怒临到王和王众子的国呢?
\par 24 我又晓谕你们,至於祭司、利未人、歌唱的、守门的,和尼提宁,并在神殿当差的人,不可叫他们进贡,交课,纳税。』
\par 25 「以斯拉啊,要照著你神赐你的智慧,将所有明白你神律法的人立为士师、审判官,治理河西的百姓,使他们教训一切不明白神律法的人。
\par 26 凡不遵行你神律法和王命令的人就当速速定他的罪,或治死,或充军,或抄家,或囚禁。」
\par 27 以斯拉说:「耶和华我们列祖的神是应当称颂的!因他使王起这心意修饰耶路撒冷耶和华的殿,
\par 28 又在王和谋士,并大能的军长面前施恩於我。因耶和华我神的手帮助我,我就得以坚强,从以色列中招聚首领,与我一同上来。」

\chapter{8}

\par 1 当亚达薛西王年间,同我从巴比伦上来的人,他们的族长和他们的家谱记在下面:
\par 2 属非尼哈的子孙有革顺;属以他玛的子孙有但以理;属大卫的子孙有哈突;
\par 3 属巴录的後裔,就是示迦尼的子孙有撒迦利亚,同著他,按家谱计算,男丁一百五十人;
\par 4 属巴哈摩押的子孙有西拉希雅的儿子以利约乃,同著他有男丁二百;
\par 5 属示迦尼的子孙有雅哈悉的儿子,同著他有男丁三百;
\par 6 属亚丁的子孙有约拿单的儿子以别,同著他有男丁五十;
\par 7 属以拦的子孙有亚他利雅的儿子耶筛亚,同著他有男丁七十;
\par 8 属示法提雅的子孙有米迦勒的儿子西巴第雅,同著他有男丁八十;
\par 9 属约押的子孙有耶歇的儿子俄巴底亚,同著他有男丁二百一十八;
\par 10 属示罗密的子孙有约细斐的儿子,同著他有男丁一百六十;
\par 11 属比拜的子孙有比拜的儿子撒迦利亚,同著他有男丁二十八;
\par 12 属押甲的子孙有哈加坦的儿子约哈难,同著他有男丁一百一十;
\par 13 属亚多尼干的子孙,就是末尾的,他们的名字是以利法列、耶利、示玛雅,同著他们有男丁六十;
\par 14 属比革瓦伊的子孙有乌太和撒布,同著他们有男丁七十。
\par 15 我招聚这些人在流入亚哈瓦的河边,我们在那里住了三日。我查看百姓和祭司,见没有利未人在那里,
\par 16 就召首领以利以谢、亚列、示玛雅、以利拿单、雅立、以利拿单、拿单、撒迦利亚、米书兰,又召教习约雅立和以利拿单。
\par 17 我打发他们往迦西斐雅地方去见那里的首领易多,又告诉他们当向易多和他的弟兄尼提宁说什麽话,叫他们为我们神的殿带使用的人来。
\par 18 蒙我们神施恩的手帮助我们,他们在以色列的曾孙、利未的孙子、抹利的後裔中带一个通达人来;还有示利比和他的众子与弟兄共一十八人。
\par 19 又有哈沙比雅,同著他有米拉利的子孙耶筛亚,并他的众子和弟兄共二十人。
\par 20 从前大卫和众首领派尼提宁服事利未人,现在从这尼提宁中也带了二百二十人来,都是按名指定的。
\par 21 那时,我在亚哈瓦河边宣告禁食,为要在我们神面前克苦己心,求他使我们和妇人孩子,并一切所有的,都得平坦的道路。
\par 22 我求王拨步兵马兵帮助我们抵挡路上的仇敌,本以为羞耻;因我曾对王说:「我们神施恩的手必帮助一切寻求他的;但他的能力和忿怒必攻击一切离弃他的。」
\par 23 所以我们禁食祈求我们的神,他就应允了我们。
\par 24 我分派祭司长十二人,就是示利比、哈沙比雅,和他们的弟兄十人,
\par 25 将王和谋士、军长,并在那里的以色列众人为我们神殿所献的金银和器皿,都秤了交给他们。
\par 26 我秤了交在他们手中的银子有六百五十他连得;银器重一百他连得;金子一百他连得;
\par 27 金碗二十个,重一千达利克;上等光铜的器皿两个,宝贵如金。
\par 28 我对他们说:「你们归耶和华为圣,器皿也为圣;金银是甘心献给耶和华你们列祖之神的。
\par 29 你们当警醒看守,直到你们在耶路撒冷耶和华殿的库内,在祭司长和利未族长,并以色列的各族长面前过了秤。」
\par 30 於是,祭司、利未人按著分量接受金银和器皿,要带到耶路撒冷我们神的殿里。
\par 31 正月十二日,我们从亚哈瓦河边起行,要往耶路撒冷去。我们神的手保佑我们,救我们脱离仇敌和路上埋伏之人的手。
\par 32 我们到了耶路撒冷,在那里住了三日。
\par 33 第四日,在我们神的殿里把金银和器皿都秤了,交在祭司乌利亚的儿子米利末的手中。同著他有非尼哈的儿子以利亚撒,还有利未人耶书亚的儿子约撒拔和宾内的儿子挪亚底。
\par 34 当时都点了数目,按著分量写在册上。
\par 35 从掳到之地归回的人向以色列的神献燔祭,就是为以色列众人献公牛十二只,公绵羊九十六只,绵羊羔七十七只,又献公山羊十二只作赎罪祭,这都是向耶和华焚献的。
\par 36 他们将王的谕旨交给王所派的总督与河西的省长,他们就帮助百姓,又供给神殿里所需用的。

\chapter{9}

\par 1 这事做完了,众首领来见我,说:「以色列民和祭司并利未人,没有离绝迦南人、赫人、比利洗人、耶布斯人、亚扪人、摩押人、埃及人、亚摩利人,仍效法这些国的民,行可憎的事。
\par 2 因他们为自己和儿子娶了这些外邦女子为妻,以致圣洁的种类和这些国的民混杂;而且首领和官长在这事上为罪魁。」
\par 3 我一听见这事,就撕裂衣服和外袍,拔了头发和胡须,惊惧忧闷而坐。
\par 4 凡为以色列神言语战兢的,都因这被掳归回之人所犯的罪聚集到我这里来。我就惊惧忧闷而坐,直到献晚祭的时候。
\par 5 献晚祭的时候我起来,心中愁苦,穿著撕裂的衣袍,双膝跪下向耶和华我的神举手,
\par 6 说:「我的神啊,我抱愧蒙羞,不敢向我神仰面;因为我们的罪孽灭顶,我们的罪恶滔天。
\par 7 从我们列祖直到今日,我们的罪恶甚重;因我们的罪孽,我们和君王、祭司都交在外邦列王的手中,杀害、掳掠、抢夺、脸上蒙羞正如今日的光景。
\par 8 现在耶和华我们的神暂且施恩与我们,给我们留些逃脱的人,使我们安稳如钉子钉在他的圣所,我们的神好光照我们的眼目,使我们在受辖制之中稍微复兴。
\par 9 我们是奴仆,然而在受辖制之中,我们的神仍没有丢弃我们,在波斯王眼前向我们施恩,叫我们复兴,能重建我们神的殿,修其毁坏之处,使我们在犹大和耶路撒冷有墙垣。
\par 10 「我们的神啊,既是如此,我们还有什麽话可说呢?因为我们已经离弃你的命令,
\par 11 就是你藉你仆人众先知所吩咐的说:『你们要去得为业之地是污秽之地;因列国之民的污秽和可憎的事,叫全地从这边直到那边满了污秽。
\par 12 所以不可将你们的女儿嫁他们的儿子,也不可为你们的儿子娶他们的女儿,永不可求他们的平安和他们的利益,这样你们就可以强盛,吃这地的美物,并遗留这地给你们的子孙永远为业。』
\par 13 神啊,我们因自己的恶行和大罪,遭遇了这一切的事,并且你刑罚我们轻於我们罪所当得的,又给我们留下这些人。
\par 14 我们岂可再违背你的命令,与这行可憎之事的民结亲呢?若这样行,你岂不向我们发怒,将我们灭绝,以致没有一个剩下逃脱的人吗?
\par 15 耶和华以色列的神啊,因你是公义的,我们这剩下的人才得逃脱,正如今日的光景。看哪,我们在你面前有罪恶,因此无人在你面前站立得住。」

\chapter{10}

\par 1 以斯拉祷告,认罪,哭泣,俯伏在神殿前的时候,有以色列中的男女孩童聚集到以斯拉那里,成了大会,众民无不痛哭。
\par 2 属以拦的子孙、耶歇的儿子示迦尼对以斯拉说:「我们在此地娶了外邦女子为妻,干犯了我们的神,然而以色列人还有指望。
\par 3 现在当与我们的神立约,休这一切的妻,离绝他们所生的,照著我主和那因神命令战兢之人所议定的,按律法而行。
\par 4 你起来,这是你当办的事,我们必帮助你,你当奋勉而行。」
\par 5 以斯拉便起来,使祭司长和利未人,并以色列众人起誓说,必照这话去行;他们就起了誓。
\par 6 以斯拉从神殿前起来,进入以利亚实的儿子约哈难的屋里,到了那里不吃饭,也不喝水;因为被掳归回之人所犯的罪,心里悲伤。
\par 7 他们通告犹大和耶路撒冷被掳归回的人,叫他们在耶路撒冷聚集。
\par 8 凡不遵首领和长老所议定、三日之内不来的,就必抄他的家,使他离开被掳归回之人的会。
\par 9 於是,犹大和便雅悯众人,三日之内都聚集在耶路撒冷。那日正是九月二十日,众人都坐在神殿前的宽阔处;因这事,又因下大雨,就都战兢。
\par 10 祭司以斯拉站起来,对他们说:「你们有罪了;因你们娶了外邦的女子为妻,增添以色列人的罪恶。
\par 11 现在当向耶和华你们列祖的神认罪,遵行他的旨意,离绝这些国的民和外邦的女子。」
\par 12 会众都大声回答说:「我们必照著你的话行,
\par 13 只是百姓众多,又逢大雨的时令,我们不能站在外头,这也不是一两天办完的事,因我们在这事上犯了大罪;
\par 14 不如为全会众派首领办理。凡我们城邑中娶外邦女子为妻的,当按所定的日期,同著本城的长老和士师而来,直到办完这事,神的烈怒就转离我们了。」
\par 15 惟有亚撒黑的儿子约拿单,特瓦的儿子雅哈谢阻挡(或作:总办)这事,并有米书兰和利未人沙比太帮助他们。
\par 16 被掳归回的人如此而行。祭司以斯拉和些族长按著宗族都指名见派;在十月初一日,一同在座查办这事,
\par 17 到正月初一日,才查清娶外邦女子的人数。
\par 18 在祭司中查出娶外邦女子为妻的,就是耶书亚的子孙约萨达的儿子,和他弟兄玛西雅、以利以谢、雅立、基大利;
\par 19 他们便应许必休他们的妻。他们因有罪,就献群中的一只公绵羊赎罪。
\par 20 音麦的子孙中,有哈拿尼、西巴第雅。
\par 21 哈琳的子孙中,有玛西雅、以利雅、示玛雅、耶歇、乌西雅。
\par 22 巴施户珥的子孙中,有以利约乃、玛西雅、以实玛利、拿坦业、约撒拔、以利亚撒。
\par 23 利未人中,有约撒拔、示每、基拉雅(基拉雅就是基利他),还有毗他希雅、犹大、以利以谢。
\par 24 歌唱的人中有以利亚实。守门的人中,有沙龙、提联、乌利。
\par 25 以色列人巴录的子孙中,有拉米、耶西雅、玛基雅、米雅民、以利亚撒、玛基雅、比拿雅。
\par 26 以拦的子孙中,有玛他尼、撒迦利亚、耶歇、押底、耶利末、以利雅。
\par 27 萨土的子孙中,有以利约乃、以利亚实、玛他尼、耶利末、撒拔、亚西撒。
\par 28 比拜的子孙中,有约哈难、哈拿尼雅、萨拜、亚勒。
\par 29 巴尼的子孙中,有米书兰、玛鹿、亚大雅、雅述、示押、耶利末。
\par 30 巴哈摩押的子孙中,有阿底拿、基拉、比拿雅、玛西雅、玛他尼、比撒列、宾内、玛拿西。
\par 31 哈琳的子孙中,有以利以谢、伊示雅、玛基雅、示玛雅、西缅、
\par 32 便雅悯、玛鹿、示玛利雅。
\par 33 哈顺的子孙中,有玛特乃、玛达他、撒拔、以利法列、耶利买、玛拿西、示每。
\par 34 巴尼的子孙中,有玛玳、暗兰、乌益、
\par 35 比拿雅、比底雅、基禄、
\par 36 瓦尼雅、米利末、以利亚实、
\par 37 玛他尼、玛特乃、雅扫、
\par 38 巴尼、宾内、示每、
\par 39 示利米雅、拿单、亚大雅、
\par 40 玛拿底拜、沙赛、沙赖、
\par 41 亚萨利、示利米雅、示玛利雅、
\par 42 沙龙、亚玛利雅、约瑟。
\par 43 尼波的子孙中,有耶利、玛他提雅、撒拔、西比拿、雅玳、约珥、比拿雅。
\par 44 这些人都娶了外邦女子为妻,其中也有生了儿女的。


\end{document}