\begin{document}

\title{历代志 下}


\chapter{1}

\par 1 大卫的儿子所罗门国位坚固;耶和华他的神与他同在,使他甚为尊大。
\par 2 所罗门吩咐以色列众人,就是千夫长、百夫长、审判官、首领与族长都来。
\par 3 所罗门和会众都往基遍的邱坛去,因那里有神的会幕,就是耶和华仆人摩西在旷野所制造的。
\par 4 只是神的约柜,大卫已经从基列耶琳搬到他所预备的地方,因他曾在耶路撒冷为约柜支搭了帐幕,
\par 5 并且户珥的孙子、乌利的儿子比撒列所造的铜坛也在基遍耶和华的会幕前。所罗门和会众都就近坛前。
\par 6 所罗门上到耶和华面前会幕的铜坛那里,献一千牺牲为燔祭。
\par 7 当夜,神向所罗门显现,对他说:「你愿我赐你什麽,你可以求。」
\par 8 所罗门对神说:「你曾向我父大卫大施慈爱,使我接续他作王。
\par 9 耶和华 神啊,现在求你成就向我父大卫所应许的话;因你立我作这民的王,他们如同地上尘沙那样多。
\par 10 求你赐我智慧聪明,我好在这民前出入;不然,谁能判断这众多的民呢?」
\par 11 神对所罗门说:「我已立你作我民的王。你既有这心意,并不求资财、丰富、尊荣,也不求灭绝那恨你之人的性命,又不求大寿数,只求智慧聪明好判断我的民;
\par 12 我必赐你智慧聪明,也必赐你资财、丰富、尊荣。在你以前的列王都没有这样,在你以後也必没有这样的。」
\par 13 於是,所罗门从基遍邱坛会幕前回到耶路撒冷,治理以色列人。
\par 14 所罗门聚集战车马兵,有战车一千四百辆,马兵一万二千名,安置在屯车的城邑和耶路撒冷,就是王那里。
\par 15 王在耶路撒冷使金银多如石头,香柏木多如高原的桑树。
\par 16 所罗门的马是从埃及带来的,是王的商人一群一群按著定价买来的。
\par 17 他们从埃及买来的车,每辆价银七百舍客勒,马每匹一百五十舍客勒。赫人诸王和亚兰诸王所买的车马,也是按这价值经他们手买来的。

\chapter{2}

\par 1 所罗门定意要为耶和华的名建造殿宇,又为自己的国建造宫室。
\par 2 所罗门就挑选七万扛抬的,八万在山上凿石头的,三千六百督工的。
\par 3 所罗门差人去见推罗王希兰,说:「你曾运香柏木与我父大卫建宫居住,求你也这样待我。
\par 4 我要为耶和华我神的名建造殿宇,分别为圣献给他,在他面前焚烧美香,常摆陈设饼,每早晚、安息日、月朔,并耶和华我们神所定的节期献燔祭。这是以色列人永远的定例。
\par 5 我所要建造的殿宇甚大;因为我们的神至大,超乎诸神。
\par 6 天和天上的天,尚且不足他居住的,谁能为他建造殿宇呢?我是谁?能为他建造殿宇吗?不过在他面前烧香而已。
\par 7 现在求你差一个巧匠来,就是善用金、银、铜、铁,和紫色、朱红色、蓝色线,并精於雕刻之工的巧匠,与我父大卫在犹大和耶路撒冷所预备的巧匠一同做工;
\par 8 又求你从利巴嫩运些香柏木、松木、檀香木到我这里来,因我知道你的仆人善於砍伐利巴嫩的树木。我的仆人也必与你的仆人同工。
\par 9 这样,可以给我预备许多的木料,因我要建造的殿宇高大出奇。
\par 10 你的仆人砍伐树木,我必给他们打好了的小麦二万歌珥,大麦二万歌珥,酒二万罢特,油二万罢特。」
\par 11 推罗王希兰写信回答所罗门说:「耶和华因为爱他的子民,所以立你作他们的王」;
\par 12 又说:「创造天地的耶和华以色列的神是应当称颂的!他赐给大卫王一个有智慧的儿子,使他有谋略聪明,可以为耶和华建造殿宇,又为自己的国建造宫室。
\par 13 「现在我打发一个精巧有聪明的人去,他是我父亲希兰所用的,
\par 14 是但支派一个妇人的儿子。他父亲是推罗人,他善用金、银、铜、铁、石、木,和紫色、蓝色、朱红色线与细麻制造各物,并精於雕刻,又能想出各样的巧工。请你派定这人,与你的巧匠和你父我主大卫的巧匠一同做工。
\par 15 我主所说的小麦、大麦、酒、油,愿我主运来给众仆人。
\par 16 我们必照你所需用的,从利巴嫩砍伐树木,扎成筏子,浮海运到约帕;你可以从那里运到耶路撒冷。」
\par 17 所罗门仿照他父大卫数点住在以色列地所有寄居的外邦人,共有十五万三千六百名;
\par 18 使七万人扛抬材料,八万人在山上凿石头,三千六百人督理工作。

\chapter{3}

\par 1 所罗门就在耶路撒冷、耶和华向他父大卫显现的摩利亚山上,就是耶布斯人阿珥楠的禾场上、大卫所指定的地方预备好了,开工建造耶和华的殿。
\par 2 所罗门作王第四年二月初二日开工建造。
\par 3 所罗门建筑神殿的根基,乃是这样:长六十肘,宽二十肘,都按著古时的尺寸。
\par 4 殿前的廊子长二十肘,与殿的宽窄一样,高一百二十肘;里面贴上精金。
\par 5 大殿的墙都用松木板遮蔽,又贴了精金,上面雕刻棕树和链子;
\par 6 又用宝石装饰殿墙,使殿华美;所用的金子都是巴瓦音的金子。
\par 7 又用金子贴殿和殿的栋梁、门槛、墙壁、门扇;墙上雕刻基路伯。
\par 8 又建造至圣所,长二十肘,与殿的宽窄一样,宽也是二十肘;贴上精金,共用金子六百他连得。
\par 9 金钉重五十舍客勒。楼房都贴上金子。
\par 10 在至圣所按造像的法子造两个基路伯,用金子包裹。
\par 11 两个基路伯的翅膀共长二十肘。这基路伯的一个翅膀长五肘,挨著殿这边的墙;那一个翅膀也长五肘,与那基路伯翅膀相接。
\par 12 那基路伯的一个翅膀长五肘,挨著殿那边的墙;那一个翅膀也长五肘,与这基路伯的翅膀相接。
\par 13 两个基路伯张开翅膀,共长二十肘,面向外殿而立。
\par 14 又用蓝色、紫色、朱红色线和细麻织幔子,在其上绣出基路伯来。
\par 15 在殿前造了两根柱子,高三十五肘;每柱顶高五肘。
\par 16 又照圣所内链子的样式做链子,安在柱顶上;又做一百石榴,安在链子上。
\par 17 将两根柱子立在殿前,一根在右边,一根在左边;右边的起名叫雅斤,左边的起名叫波阿斯。

\chapter{4}

\par 1 他又制造一座铜坛,长二十肘,宽二十肘,高十肘;
\par 2 又铸一个铜海,样式是圆的,高五肘,径十肘,围三十肘;
\par 3 海周围有野瓜(野瓜:原文作牛)的样式,每肘十瓜,共有两行,是铸海的时候铸上的;
\par 4 有十二只铜牛驮海:三只向北,三只向西,三只向南,三只向东;海在牛上,牛尾向内;
\par 5 海厚一掌,边如杯边,又如百合花,可容三千罢特;
\par 6 又制造十个盆:五个放在右边,五个放在左边,献燔祭所用之物都洗在其内;但海是为祭司沐浴的。
\par 7 他又照所定的样式造十个金灯台放在殿里:五个在右边,五个在左边;
\par 8 又造十张桌子放在殿里:五张在右边,五张在左边;又造一百个金碗;
\par 9 又建立祭司院和大院,并院门,用铜包裹门扇;
\par 10 将海安在殿门的右边,就是南边。
\par 11 户兰又造了盆、铲、碗。这样,他为所罗门王做完了神殿的工。
\par 12 所造的就是:两根柱子和柱上两个如球的顶,并两个盖柱顶的网子
\par 13 和四百石榴,安在两个网子上(每网两行盖著两个柱上如球的顶)。
\par 14 盆座和其上的盆,
\par 15 海和海下的十二只牛,
\par 16 盆、铲子、肉锸子,与耶和华殿里的一切器皿,都是巧匠户兰用光亮的铜为所罗门王造成的,
\par 17 是在约但平原疏割和撒利但中间藉胶泥铸成的。
\par 18 所罗门制造的这一切甚多,铜的轻重无法可查。
\par 19 所罗门又造神殿里的金坛和陈设饼的桌子,
\par 20 并精金的灯台和灯盏,可以照例点在内殿前。
\par 21 灯台上的花和灯盏,并蜡剪都是金的,且是纯金的;
\par 22 又用精金制造镊子、盘子、调羹、火鼎。至於殿门和至圣所的门扇,并殿的门扇,都是金子妆饰的。

\chapter{5}

\par 1 所罗门做完了耶和华殿的一切工,就把他父大卫分别为圣的金银和器皿都带来,放在神殿的府库里。
\par 2 那时,所罗门将以色列的长老、各支派的首领,并以色列的族长招聚到耶路撒冷,要把耶和华的约柜从大卫城就是锡安运上来。
\par 3 於是以色列众人在七月节前都聚集到王那里。
\par 4 以色列众长老来到,利未人便抬起约柜。
\par 5 祭司利未人将约柜运上来,又将会幕和会幕的一切圣器具都带上来。
\par 6 所罗门王和聚集到他那里的以色列全会众都在约柜前献牛羊为祭,多得不可胜数。
\par 7 祭司将耶和华的约柜抬进内殿,就是至圣所,放在两个基路伯的翅膀底下。
\par 8 基路伯张著翅膀在约柜之上,遮掩约柜和抬柜的杠。
\par 9 这杠甚长,杠头在内殿前可以看见,在殿外却不能看见,直到如今还在那里。
\par 10 约柜里惟有两块石版,就是以色列人出埃及後,耶和华与他们立约的时候,摩西在何烈山所放的。除此以外,并无别物。
\par 11 当时,在那里所有的祭司都已自洁,并不分班供职。
\par 12 他们出圣所的时候,歌唱的利未人亚萨、希幔、耶杜顿,和他们的众子众弟兄都穿细麻布衣服,站在坛的东边,敲钹、鼓瑟、弹琴,同著他们有一百二十个祭司吹号。
\par 13 吹号的、歌唱的都一齐发声,声合为一,赞美感谢耶和华。吹号、敲钹,用各种乐器,扬声赞美耶和华说:耶和华本为善,他的慈爱永远长存!那时,耶和华的殿有云充满,
\par 14 甚至祭司不能站立供职,因为耶和华的荣光充满了神的殿。

\chapter{6}

\par 1 那时,所罗门说:耶和华曾说他必住在幽暗之处。
\par 2 但我已经建造殿宇作你的居所,为你永远的住处。
\par 3 王转脸为以色列会众祝福,以色列会众就都站立。
\par 4 所罗门说:「耶和华以色列的神是应当称颂的!因他亲口向我父大卫所应许的,也亲手成就了。
\par 5 他说:『自从我领我民出埃及地以来,我未曾在以色列众支派中选择一城建造殿宇为我名的居所,也未曾拣选一人作我民以色列的君;
\par 6 但选择耶路撒冷为我名的居所,又拣选大卫治理我民以色列。』」
\par 7 所罗门说:「我父大卫曾立意要为耶和华以色列神的名建殿,
\par 8 耶和华却对我父大卫说:『你立意要为我的名建殿,这意思甚好;
\par 9 只是你不可建殿,惟你所生的儿子必为我名建殿。』
\par 10 「现在耶和华成就了他所应许的话,使我接续我父大卫坐以色列的国位,是照耶和华所说的,又为耶和华以色列神的名建造了殿。
\par 11 我将约柜安置在其中,柜内有耶和华的约,就是他与以色列人所立的约。」
\par 12 所罗门当著以色列会众,站在耶和华的坛前,举起手来,
\par 13 (所罗门曾造一个铜台,长五肘,宽五肘,高三肘,放在院中)就站在台上,当著以色列的会众跪下,向天举手,
\par 14 说:「耶和华以色列的神啊,天上地下没有神可比你的!你向那尽心行在你面前的仆人守约施慈爱;
\par 15 向你仆人我父大卫所应许的话现在应验了。你亲口应许,亲手成就,正如今日一样。
\par 16 耶和华以色列的神啊,你所应许你仆人我父大卫的话说:『你的子孙若谨慎自己的行为,遵守我的律法,像你在我面前所行的一样,就不断人坐以色列的国位。』现在求你应验这话。
\par 17 耶和华以色列的神啊,求你成就向你仆人大卫所应许的话。
\par 18 「神果真与世人同住在地上吗?看哪,天和天上的天尚且不足你居住的,何况我所建的这殿呢?
\par 19 惟求耶和华我的神垂顾仆人的祷告祈求,俯听仆人在你面前的祈祷呼吁。
\par 20 愿你昼夜看顾这殿,就是你应许立为你名的居所;求你垂听仆人向此处祷告的话。
\par 21 你仆人和你民以色列向此处祈祷的时候,求你从天上你的居所垂听,垂听而赦免。
\par 22 「人若得罪邻舍,有人叫他起誓,他来到这殿,在你的坛前起誓,
\par 23 求你从天上垂听,判断你的仆人,定恶人有罪,照他所行的报应在他头上;定义人有理,照他的义赏赐他。
\par 24 「你的民以色列若得罪你,败在仇敌面前,又回心转意承认你的名,在这殿里向你祈求祷告,
\par 25 求你从天上垂听,赦免你民以色列的罪,使他们归回你赐给他们和他们列祖之地。
\par 26 「你的民因得罪你,你惩罚他们,使天闭塞不下雨,他们若向此处祷告,承认你的名,离开他们的罪,
\par 27 求你在天上垂听,赦免你仆人和你民以色列的罪,将当行的善道指教他们,且降雨在你的地,就是你赐给你民为业之地。
\par 28 「国中若有饥荒、瘟疫、旱风、霉烂、蝗虫、蚂蚱,或有仇敌犯境,围困城邑,无论遭遇什麽灾祸疾病,
\par 29 你的民以色列,或是众人,或是一人,自觉灾祸甚苦,向这殿举手,无论祈求什麽,祷告什麽,
\par 30 求你从天上你的居所垂听赦免。你是知道人心的,要照各人所行的待他们(惟有你知道世人的心),
\par 31 使他们在你赐给我们列祖之地上一生一世敬畏你,遵行你的道。
\par 32 「论到不属你民以色列的外邦人,为你的大名和大能的手,并伸出来的膀臂,从远方而来,向这殿祷告,
\par 33 求你从天上你的居所垂听,照著外邦人所祈求的而行,使天下万民都认识你的名,敬畏你,像你的民以色列一样,又使他们知道我建造的这殿是称为你名下的。
\par 34 「你的民若奉你的差遣,无论往何处去与仇敌争战,向你所选择的城与我为你名所建造的殿祷告,
\par 35 求你从天上垂听他们的祷告祈求,使他们得胜。
\par 36 「你的民若得罪你(世上没有不犯罪的人),你向他们发怒,将他们交给仇敌掳到或远或近之地;
\par 37 他们若在掳到之地想起罪来,回心转意,恳求你说:『我们有罪了,我们悖逆了,我们作恶了』;
\par 38 他们若在掳到之地尽心尽性归服你,又向自己的地,就是你赐给他们列祖之地和你所选择的城,并我为你名所建造的殿祷告,
\par 39 求你从天上你的居所垂听你民的祷告祈求,为他们伸冤,赦免他们的过犯。
\par 40 「我的神啊,现在求你睁眼看,侧耳听在此处所献的祷告。
\par 41 耶和华 神啊,求你起来,和你有能力的约柜同入安息之所。耶和华 神啊,愿你的祭司披上救恩;愿你的圣民蒙福欢乐。
\par 42 耶和华 神啊,求你不要厌弃你的受膏者,要记念向你仆人大卫所施的慈爱。」

\chapter{7}

\par 1 所罗门祈祷已毕,就有火从天上降下来,烧尽燔祭和别的祭。耶和华的荣光充满了殿;
\par 2 因耶和华的荣光充满了耶和华殿,所以祭司不能进殿。
\par 3 那火降下、耶和华的荣光在殿上的时候,以色列众人看见,就在铺石地俯伏叩拜,称谢耶和华说:耶和华本为善,他的慈爱永远长存!
\par 4 王和众民在耶和华面前献祭。
\par 5 所罗门王用牛二万二千,羊十二万献祭。这样,王和众民为神的殿行奉献之礼。
\par 6 祭司侍立,各供其职;利未人也拿著耶和华的乐器,就是大卫王造出来、藉利未人颂赞耶和华的。(他的慈爱永远长存!)祭司在众人面前吹号,以色列人都站立。
\par 7 所罗门因他所造的铜坛容不下燔祭、素祭,和脂油,便将耶和华殿前院子当中分别为圣,在那里献燔祭和平安祭牲的脂油。
\par 8 那时所罗门和以色列众人,就是从哈马口直到埃及小河,所有的以色列人都聚集成为大会,守节七日。
\par 9 第八日设立严肃会,行奉献坛的礼七日,守节七日。
\par 10 七月二十三日,王遣散众民;他们因见耶和华向大卫和所罗门与他民以色列所施的恩惠,就都心中喜乐,各归各家去了。
\par 11 所罗门造成了耶和华殿和王宫;在耶和华殿和王宫凡他心中所要做的,都顺顺利利地做成了。
\par 12 夜间耶和华向所罗门显现,对他说:「我已听了你的祷告,也选择这地方作为祭祀我的殿宇。
\par 13 我若使天闭塞不下雨,或使蝗虫吃这地的出产,或使瘟疫流行在我民中,
\par 14 这称为我名下的子民,若是自卑、祷告,寻求我的面,转离他们的恶行,我必从天上垂听,赦免他们的罪,医治他们的地。
\par 15 我必睁眼看、侧耳听在此处所献的祷告。
\par 16 现在我已选择这殿,分别为圣,使我的名永在其中,我的眼、我的心也必常在那里。
\par 17 你若在我面前效法你父大卫所行的,遵行我一切所吩咐你的,谨守我的律例典章,
\par 18 我就必坚固你的国位,正如我与你父大卫所立的约,说:『你的子孙必不断人作以色列的王。』
\par 19 「倘若你们转去丢弃我指示你们的律例诫命,去事奉敬拜别神,
\par 20 我就必将以色列人从我赐给他们的地上拔出根来,并且我为己名所分别为圣的殿也必舍弃不顾,使他在万民中作笑谈,被讥诮。
\par 21 这殿虽然甚高,将来经过的人必惊讶说:『耶和华为何向这地和这殿如此行呢?』
\par 22 人必回答说:『是因此地的人离弃耶和华他们列祖的神,就是领他们出埃及地的神,去亲近别神,敬拜事奉他,所以耶和华使这一切灾祸临到他们。』」

\chapter{8}

\par 1 所罗门建造耶和华殿和王宫,二十年才完毕了。
\par 2 以後所罗门重新修筑希兰送给他的那些城邑,使以色列人住在那里。
\par 3 所罗门往哈马琐巴去,攻取了那地方。
\par 4 所罗门建造旷野里的达莫,又建造哈马所有的积货城,
\par 5 又建造上伯和仑、下伯和仑作为保障,都有墙,有门,有闩;
\par 6 又建造巴拉和所有的积货城,并屯车辆马兵的城,与耶路撒冷、利巴嫩,以及自己治理的全国中所愿意建造的。
\par 7 至於国中所剩下不属以色列人的赫人、亚摩利人、比利洗人、希未人、耶布斯人,
\par 8 就是以色列人未曾灭绝的,所罗门挑取他们的後裔作服苦的奴仆,直到今日。
\par 9 惟有以色列人,所罗门不使他们当奴仆做工,乃是作他的战士、军长的统领、车兵长、马兵长。
\par 10 所罗门王有二百五十督工的,监管工人。
\par 11 所罗门将法老的女儿带出大卫城,上到为他建造的宫里;因所罗门说:「耶和华约柜所到之处都为圣地,所以我的妻不可住在以色列王大卫的宫里。」
\par 12 所罗门在耶和华的坛上,就是在廊子前他所筑的坛上,与耶和华献燔祭;
\par 13 又遵著摩西的吩咐在安息日、月朔,并一年三节,就是除酵节、七七节、住棚节,献每日所当献的祭。
\par 14 所罗门照著他父大卫所定的例,派定祭司的班次,使他们各供己事,又使利未人各尽其职,赞美耶和华,在祭司面前做每日所当做的;又派守门的按著班次看守各门,因为神人大卫是这样吩咐的。
\par 15 王所吩咐众祭司和利未人的,无论是管府库或办别的事,他们都不违背。
\par 16 所罗门建造耶和华的殿,从立根基直到成功的日子,工料俱备。这样,耶和华的殿全然完毕。
\par 17 那时,所罗门往以东地靠海的以旬迦别和以禄去。
\par 18 希兰差遣他的臣仆,将船只和熟悉泛海的仆人送到所罗门那里。他们同著所罗门的仆人到了俄斐,得了四百五十他连得金子,运到所罗门王那里。

\chapter{9}

\par 1 示巴女王听见所罗门的名声,就来到耶路撒冷,要用难解的话试问所罗门;跟随他的人甚多,又有骆驼驮著香料、宝石,和许多金子。他来见了所罗门,就把心里所有的对所罗门都说出来。
\par 2 所罗门将他所问的都答上了,没有一句不明白、不能答的。
\par 3 示巴女王见所罗门的智慧和他所建造的宫室、
\par 4 席上的珍馐美味、群臣分列而坐、仆人两旁侍立,以及他们的衣服装饰、酒政,和酒政的衣服装饰,又见他上耶和华殿的台阶,就诧异得神不守舍,
\par 5 对王说:「我在本国里所听见论到你的事和你的智慧实在是真的!
\par 6 我先不信那些话,及至我来亲眼见了,才知道你的大智慧;人所告诉我的,还不到一半;你的实迹越过我所听见的名声。
\par 7 你的群臣、你的仆人常侍立在你面前听你智慧的话是有福的。
\par 8 耶和华你的神是应当称颂的!他喜悦你,使你坐他的国位,为耶和华你的神作王;因为你的神爱以色列人,要永远坚立他们,所以立你作他们的王,使你秉公行义。」
\par 9 於是示巴女王将一百二十他连得金子和宝石,与极多的香料送给所罗门王;他送给王的香料,以後再没有这样的。
\par 10 希兰的仆人和所罗门的仆人从俄斐运了金子来,也运了檀香木(或作:乌木;下同)和宝石来。
\par 11 王用檀香木为耶和华殿和王宫做台,又为歌唱的人做琴瑟;犹大地从来没有见过这样的。
\par 12 所罗门王按示巴女王所带来的,还他礼物,另外照他一切所要所求的,都送给他。於是女王和他臣仆转回本国去了。
\par 13 所罗门每年所得的金子共有六百六十六他连得,
\par 14 另外还有商人所进的金子,并且亚拉伯诸王与属国的省长都带金银给所罗门。
\par 15 所罗门王用锤出来的金子打成挡牌二百面,每面用金子六百舍客勒;
\par 16 又用锤出来的金子打成盾牌三百面,每面用金子三百舍客勒,都放在利巴嫩林宫里。
\par 17 王用象牙制造一个大宝座,用精金包裹。
\par 18 宝座有六层台阶,又有金脚凳,与宝座相连。宝座两旁有扶手,靠近扶手有两个狮子站立。
\par 19 六层台阶上有十二个狮子站立,每层有两个:左边一个,右边一个;在列国中没有这样做的。
\par 20 所罗门王一切的饮器都是金的,利巴嫩林宫里的一切器皿都是精金的。所罗门年间,银子算不了什麽。
\par 21 因为王的船只与希兰的仆人一同往他施去;他施船只三年一次装载金、银、象牙、猿猴、孔雀回来。
\par 22 所罗门王的财宝与智慧胜过天下的列王。
\par 23 普天下的王都求见所罗门,要听神赐给他智慧的话。
\par 24 他们各带贡物,就是金器、银器、衣服、军械、香料、骡马,每年有一定之例。
\par 25 所罗门有套车的马四千棚,有马兵一万二千,安置在屯车的城邑和耶路撒冷,就是王那里。
\par 26 所罗门统管诸王,从大河到非利士地,直到埃及的边界。
\par 27 王在耶路撒冷使银子多如石头,香柏木多如高原的桑树。
\par 28 有人从埃及和各国为所罗门赶马群来。
\par 29 所罗门其余的事,自始至终,不都写在先知拿单的书上和示罗人亚希雅的预言书上,并先见易多论尼八儿子耶罗波安的默示书上吗?
\par 30 所罗门在耶路撒冷作以色列众人的王共四十年。
\par 31 所罗门与他列祖同睡,葬在他父大卫城里。他儿子罗波安接续他作王。

\chapter{10}

\par 1 罗波安往示剑去,因为以色列人都到了示剑,要立他作王。
\par 2 尼八的儿子耶罗波安先前躲避所罗门王,逃往埃及,住在那里;他听见这事,就从埃及回来。
\par 3 以色列人打发人去请他,他就和以色列众人来见罗波安,对他说:
\par 4 「你父亲使我们负重轭做苦工,现在求你使我们做的苦工负的重轭轻松些,我们就事奉你。」
\par 5 罗波安对他们说:「第三日再来见我吧!」民就去了。
\par 6 罗波安之父所罗门在世的日子,有侍立在他面前的老年人,罗波安王和他们商议,说:「你们给我出个什麽主意,我好回覆这民。」
\par 7 老年人对他说:「王若恩待这民,使他们喜悦,用好话回覆他们,他们就永远作王的仆人。」
\par 8 王却不用老年人给他出的主意,就和那些与他一同长大、在他面前侍立的少年人商议,
\par 9 说:「这民对我说:『你父亲使我们负重轭,求你使我们轻松些』;你们给我出个什麽主意,我好回覆他们。」
\par 10 那同他长大的少年人说:「这民对王说:『你父亲使我们负重轭,求你使我们轻松些』;王要对他们如此说:『我的小拇指比我父亲的腰还粗;
\par 11 我父亲使你们负重轭,我必使你们负更重的轭;我父亲用鞭子责打你们,我要用蝎子鞭责打你们。』」
\par 12 耶罗波安和众百姓遵著罗波安王所说「你们第三日再来见我」的那话,第三日他们果然来了。
\par 13 罗波安王用严厉的话回覆他们,不用老年人所出的主意,
\par 14 照著少年人所出的主意对他们说:「我父亲使你们负重轭,我必使你们负更重的轭;我父亲用鞭子责打你们,我要用蝎子鞭责打你们。」
\par 15 王不肯依从百姓;这事乃出於神,为要应验耶和华藉示罗人亚希雅对尼八儿子耶罗波安所说的话。
\par 16 以色列众民见王不依从他们,就对王说:我们与大卫有什麽分儿呢?与耶西的儿子并没有关涉!以色列人哪,各回各家去吧!大卫家啊,自己顾自己吧!於是,以色列众人都回自己家里去了。
\par 17 惟独住在犹大城邑的以色列人,罗波安仍作他们的王。
\par 18 罗波安王差遣掌管服苦之人的哈多兰往以色列人那里去,以色列人就用石头打死他。罗波安王急忙上车,逃回耶路撒冷去了。
\par 19 这样,以色列人背叛大卫家,直到今日。

\chapter{11}

\par 1 罗波安来到耶路撒冷,招聚犹大家和便雅悯家,共十八万人,都是挑选的战士,要与以色列人争战,好将国夺回再归自己。
\par 2 但耶和华的话临到神人示玛雅说:
\par 3 「你去告诉所罗门的儿子犹大王罗波安和住犹大、便雅悯的以色列众人说,
\par 4 耶和华如此说:『你们不可上去与你们的弟兄争战,各归各家去吧!因为这事出於我。』」众人就听从耶和华的话归回,不去与耶罗波安争战。
\par 5 罗波安住在耶路撒冷,在犹大地修筑城邑,
\par 6 为保障修筑伯利恒、以坦、提哥亚、
\par 7 伯夙、梭哥、亚杜兰、
\par 8 迦特、玛利沙、西弗、
\par 9 亚多莱音、拉吉、亚西加、
\par 10 琐拉、亚雅仑、希伯仑。这都是犹大和便雅悯的坚固城。
\par 11 罗波安又坚固各处的保障,在其中安置军长,又预备下粮食、油、酒。
\par 12 他在各城里预备盾牌和枪,且使城极其坚固。犹大和便雅悯都归了他。
\par 13 以色列全地的祭司和利未人都从四方来归罗波安。
\par 14 利未人撇下他们的郊野和产业,来到犹大与耶路撒冷,是因耶罗波安和他的儿子拒绝他们,不许他们供祭司职分事奉耶和华。
\par 15 耶罗波安为邱坛、为鬼魔(原文作公山羊)、为自己所铸造的牛犊设立祭司。
\par 16 以色列各支派中,凡立定心意寻求耶和华以色列神的,都随从利未人,来到耶路撒冷祭祀耶和华他们列祖的神。
\par 17 这样,就坚固犹大国,使所罗门的儿子罗波安强盛三年,因为他们三年遵行大卫和所罗门的道。
\par 18 罗波安娶大卫儿子耶利摩的女儿玛哈拉为妻,又娶耶西儿子以利押的女儿亚比孩为妻。
\par 19 从他生了几个儿子,就是耶乌施、示玛利雅、撒罕。
\par 20 後来又娶押沙龙的女儿玛迦(十三章二节是乌列的女儿米该雅),从他生了亚比雅、亚太、细撒、示罗密。
\par 21 罗波安娶十八个妻,立六十个妾,生二十八个儿子,六十个女儿;他却爱押沙龙的女儿玛迦,比爱别的妻妾更甚。
\par 22 罗波安立玛迦的儿子亚比雅作太子,在他弟兄中为首,因为想要立他接续作王。
\par 23 罗波安办事精明,使他众子分散在犹大和便雅悯全地各坚固城里,又赐他们许多粮食,为他们多寻妻子。

\chapter{12}

\par 1 罗波安的国坚立,他强盛的时候就离弃耶和华的律法,以色列人也都随从他。
\par 2 罗波安王第五年,埃及王示撒上来攻打耶路撒冷,因为王和民得罪了耶和华。
\par 3 示撒带战车一千二百辆,马兵六万,并且跟从他出埃及的路比人、苏基人,和古实人,多得不可胜数。
\par 4 他攻取了犹大的坚固城,就来到耶路撒冷。
\par 5 那时,犹大的首领因为示撒就聚集在耶路撒冷。有先知示玛雅去见罗波安和众首领,对他们说:「耶和华如此说:『你们离弃了我,所以我使你们落在示撒手里。』」
\par 6 於是王和以色列的众首领都自卑说:「耶和华是公义的。」
\par 7 耶和华见他们自卑,耶和华的话就临到示玛雅说:「他们既自卑,我必不灭绝他们;必使他们略得拯救,我不藉著示撒的手将我的怒气倒在耶路撒冷。
\par 8 然而他们必作示撒的仆人,好叫他们知道,服事我与服事外邦人有何分别。」
\par 9 於是,埃及王示撒上来攻取耶路撒冷,夺了耶和华殿和王宫里的宝物,尽都带走,又夺去所罗门制造的金盾牌。
\par 10 罗波安王制造铜盾牌代替那金盾牌,交给守王宫门的护卫长看守。
\par 11 王每逢进耶和华的殿,护卫兵就拿这盾牌,随後仍将盾牌送回,放在护卫房。
\par 12 王自卑的时候,耶和华的怒气就转消了,不将他灭尽,并且在犹大中间也有善益的事。
\par 13 罗波安王自强,在耶路撒冷作王。他登基的时候年四十一岁,在耶路撒冷,就是耶和华从以色列众支派中所选择立他名的城,作王十七年。罗波安的母亲名叫拿玛,是亚扪人。
\par 14 罗波安行恶,因他不立定心意寻求耶和华。
\par 15 罗波安所行的事,自始至终不都写在先知示玛雅和先见易多的史记上吗?罗波安与耶罗波安时常争战。
\par 16 罗波安与他列祖同睡,葬在大卫城里。他儿子亚比雅接续他作王。

\chapter{13}

\par 1 耶罗波安王十八年,亚比雅登基作犹大王,
\par 2 在耶路撒冷作王三年。他母亲名叫米该亚(又作玛迦),是基比亚人乌列的女儿。亚比雅常与耶罗波安争战。
\par 3 有一次亚比雅率领挑选的兵四十万摆阵,都是勇敢的战士;耶罗波安也挑选大能的勇士八十万,对亚比雅摆阵。
\par 4 亚比雅站在以法莲山地中的洗玛脸山上,说:「耶罗波安和以色列众人哪,要听我说!
\par 5 耶和华以色列的神曾立盐约(盐就是不废坏的意思),将以色列国永远赐给大卫和他的子孙,你们不知道吗?
\par 6 无奈大卫儿子所罗门的臣仆、尼八儿子耶罗波安起来背叛他的主人。
\par 7 有些无赖的匪徒聚集跟从他,逞强攻击所罗门的儿子罗波安;那时罗波安还幼弱,不能抵挡他们。
\par 8 「现在你们有意抗拒大卫子孙手下所治耶和华的国,你们的人也甚多,你们那里又有耶罗波安为你们所造当作神的金牛犊。
\par 9 你们不是驱逐耶和华的祭司亚伦的後裔和利未人吗?不是照著外邦人的恶俗为自己立祭司吗?无论何人牵一只公牛犊、七只公绵羊将自己分别出来,就可作虚无之神的祭司。
\par 10 至於我们,耶和华是我们的神,我们并没有离弃他。我们有事奉耶和华的祭司,都是亚伦的後裔,并有利未人各尽其职,
\par 11 每日早晚向耶和华献燔祭,烧美香,又在精金的桌子上摆陈设饼;又有金灯台和灯盏,每晚点起,因为我们遵守耶和华我们神的命;惟有你们离弃了他。
\par 12 率领我们的是神,我们这里也有神的祭司拿号向你们吹出大声。以色列人哪,不要与耶和华你们列祖的神争战,因你们必不能亨通。」
\par 13 耶罗波安却在犹大人的後头设伏兵。这样,以色列人在犹大人的前头,伏兵在犹大人的後头。
\par 14 犹大人回头观看,见前後都有敌兵,就呼求耶和华,祭司也吹号。
\par 15 於是犹大人呐喊;犹大人呐喊的时候,神就使耶罗波安和以色列众人败在亚比雅与犹大人面前。
\par 16 以色列人在犹大人面前逃跑,神将他们交在犹大人手里。
\par 17 亚比雅和他的军兵大大杀戮以色列人,以色列人仆倒死亡的精兵有五十万。
\par 18 那时,以色列人被制伏了,犹大人得胜,是因倚靠耶和华他们列祖的神。
\par 19 亚比雅追赶耶罗波安,攻取了他的几座城,就是伯特利和属伯特利的镇市,耶沙拿和属耶沙拿的镇市,以法拉音(或作:以弗伦)和属以法拉音的镇市。
\par 20 亚比雅在世的时候,耶罗波安不能再强盛;耶和华攻击他,他就死了。
\par 21 亚比雅却渐渐强盛,娶妻妾十四个,生了二十二个儿子,十六个女儿。
\par 22 亚比雅其余的事和他的言行都写在先知易多的传上。

\chapter{14}

\par 1 亚比雅与他列祖同睡,葬在大卫城里。他儿子亚撒接续他作王。亚撒年间,国中太平十年。
\par 2 亚撒行耶和华他神眼中看为善为正的事,
\par 3 除掉外邦神的坛和邱坛,打碎柱像,砍下木偶,
\par 4 吩咐犹大人寻求耶和华他们列祖的神,遵行他的律法、诫命;
\par 5 又在犹大各城邑除掉邱坛和日像,那时国享太平;
\par 6 又在犹大建造了几座坚固城。国中太平数年,没有战争,因为耶和华赐他平安。
\par 7 他对犹大人说:「我们要建造这些城邑,四围筑墙,盖楼,安门,做闩;地还属我们,是因寻求耶和华我们的神;我们既寻求他,他就赐我们四境平安。」於是建造城邑,诸事亨通。
\par 8 亚撒的军兵,出自犹大拿盾牌拿枪的三十万人;出自便雅悯拿盾牌拉弓的二十八万人。这都是大能的勇士。
\par 9 有古实王谢拉率领军兵一百万,战车三百辆,出来攻击犹大人,到了玛利沙。
\par 10 於是亚撒出去与他迎敌,就在玛利沙的洗法谷彼此摆阵。
\par 11 亚撒呼求耶和华他的神说:「耶和华啊,惟有你能帮助软弱的,胜过强盛的。耶和华我们的神啊,求你帮助我们;因为我们仰赖你,奉你的名来攻击这大军。耶和华啊,你是我们的神,不要容人胜过你。」
\par 12 於是耶和华使古实人败在亚撒和犹大人面前,古实人就逃跑了;
\par 13 亚撒和跟随他的军兵追赶他们,直到基拉耳。古实人被杀的甚多,不能再强盛,因为败在耶和华与他军兵面前。犹大人就夺了许多财物,
\par 14 又打破基拉耳四围的城邑;耶和华使其中的人都甚恐惧。犹大人又将所有的城掳掠一空,因其中的财物甚多,
\par 15 又毁坏了群畜的圈,夺取许多的羊和骆驼,就回耶路撒冷去了。

\chapter{15}

\par 1 神的灵感动俄德的儿子亚撒利雅。
\par 2 他出来迎接亚撒,对他说:「亚撒和犹大、便雅悯众人哪,要听我说:你们若顺从耶和华,耶和华必与你们同在;你们若寻求他,就必寻见;你们若离弃他,他必离弃你们。
\par 3 以色列人不信真神,没有训诲的祭司,也没有律法,已经好久了;
\par 4 但他们在急难的时候归向耶和华以色列的神,寻求他,他就被他们寻见。
\par 5 那时,出入的人不得平安,列国的居民都遭大乱;
\par 6 这国攻击那国,这城攻击那城,互相破坏,因为神用各样灾难扰乱他们。
\par 7 现在你们要刚强,不要手软,因你们所行的必得赏赐。」
\par 8 亚撒听见这话和俄德儿子先知亚撒利雅的预言,就壮起胆来,在犹大、便雅悯全地,并以法莲山地所夺的各城,将可憎之物尽都除掉,又在耶和华殿的廊前重新修筑耶和华的坛;
\par 9 又招聚犹大、便雅悯的众人,并他们中间寄居的以法莲人、玛拿西人、西缅人。有许多以色列人归降亚撒,因见耶和华他的神与他同在。
\par 10 亚撒十五年三月,他们都聚集在耶路撒冷。
\par 11 当日他们从所取的掳物中,将牛七百只、羊七千只献给耶和华。
\par 12 他们就立约,要尽心尽性地寻求耶和华他们列祖的神。
\par 13 凡不寻求耶和华以色列神的,无论大小、男女,必被治死。
\par 14 他们就大声欢呼,吹号吹角,向耶和华起誓。
\par 15 犹大众人为所起的誓欢喜;因他们是尽心起誓,尽意寻求耶和华,耶和华就被他们寻见,且赐他们四境平安。
\par 16 亚撒王贬了他祖母玛迦太后的位,因他造了可憎的偶像亚舍拉。亚撒砍下他的偶像,捣得粉碎,烧在汲沦溪边。
\par 17 只是邱坛还没有从以色列中废去,然而亚撒的心一生诚实。
\par 18 亚撒将他父所分别为圣、与自己所分别为圣的金银和器皿都奉到神的殿里。
\par 19 从这时直到亚撒三十五年,都没有争战的事。

\chapter{16}

\par 1 亚撒三十六年,以色列王巴沙上来攻击犹大,修筑拉玛,不许人从犹大王亚撒那里出入。
\par 2 於是亚撒从耶和华殿和王宫的府库里拿出金银来,送与住大马色的亚兰王便哈达,说:
\par 3 「你父曾与我父立约,我与你也要立约。现在我将金银送给你,求你废掉你与以色列王巴沙所立的约,使他离开我。」
\par 4 便哈达听从亚撒王的话,派军长去攻击以色列的城邑。他们就攻破以云、但、亚伯玛音,和拿弗他利一切的积货城。
\par 5 巴沙听见就停工,不修筑拉玛了。
\par 6 於是亚撒王率领犹大众人,将巴沙修筑拉玛所用的石头、木头都运去,用以修筑迦巴和米斯巴。
\par 7 那时,先见哈拿尼来见犹大王亚撒,对他说:「因你仰赖亚兰王,没有仰赖耶和华你的神,所以亚兰王的军兵脱离了你的手。
\par 8 古实人、路比人的军队不是甚大吗?战车马兵不是极多吗?只因你仰赖耶和华,他便将他们交在你手里。
\par 9 耶和华的眼目遍察全地,要显大能帮助向他心存诚实的人。你这事行得愚昧;此後,你必有争战的事。」
\par 10 亚撒因此恼恨先见,将他囚在监里。那时亚撒也虐待一些人民。
\par 11 亚撒所行的事,自始至终都写在犹大和以色列诸王记上。
\par 12 亚撒作王三十九年,他脚上有病,而且甚重。病的时候没有求耶和华,只求医生。
\par 13 他作王四十一年而死,与他列祖同睡,
\par 14 葬在大卫城自己所凿的坟墓里,放在床上,其床堆满各样馨香的香料,就是按做香的作法调和的香料,又为他烧了许多的物件。

\chapter{17}

\par 1 亚撒的儿子约沙法接续他作王,奋勇自强,防备以色列人,
\par 2 安置军兵在犹大一切坚固城里,又安置防兵在犹大地和他父亚撒所得以法莲的城邑中。
\par 3 耶和华与约沙法同在;因为他行他祖大卫初行的道,不寻求巴力,
\par 4 只寻求他父亲的神,遵行他的诫命,不效法以色列人的行为。
\par 5 所以耶和华坚定他的国,犹大众人给他进贡;约沙法大有尊荣资财。
\par 6 他高兴遵行耶和华的道,并且从犹大除掉一切邱坛和木偶。
\par 7 他作王第三年,就差遣臣子便亥伊勒、俄巴底、撒迦利雅、拿坦业、米该亚往犹大各城去教训百姓。
\par 8 同著他们有利未人示玛雅、尼探雅、西巴第雅、亚撒黑、示米拉末、约拿单、亚多尼雅、多比雅、驼巴多尼雅,又有祭司以利沙玛、约兰同著他们。
\par 9 他们带著耶和华的律法书,走遍犹大各城教训百姓。
\par 10 耶和华使犹大四围的列国都甚恐惧,不敢与约沙法争战。
\par 11 有些非利士人与约沙法送礼物,纳贡银。亚拉伯人也送他公绵羊七千七百只,公山羊七千七百只。
\par 12 约沙法日渐强大,在犹大建造营寨和积货城。
\par 13 他在犹大城邑中有许多工程,又在耶路撒冷有战士,就是大能的勇士。
\par 14 他们的数目,按著宗族,记在下面:犹大族的,千夫长押拿为首率领大能的勇士三十万;
\par 15 其次是,千夫长约哈难率领大能的勇士二十八万;
\par 16 其次是,细基利的儿子亚玛斯雅(他为耶和华牺牲自己)率领大能的勇士二十万。
\par 17 便雅悯族,是大能的勇士以利雅大率领,拿弓箭和盾牌的二十万;
\par 18 其次是,约萨拔率领预备打仗的十八万。
\par 19 这都是伺候王的,还有王在犹大全地坚固城所安置的不在其内。

\chapter{18}

\par 1 约沙法大有尊荣资财,就与亚哈结亲。
\par 2 过了几年,他下到撒玛利亚去见亚哈;亚哈为他和跟从他的人宰了许多牛羊,劝他与自己同去攻取基列的拉末。
\par 3 以色列王亚哈问犹大王约沙法说:「你肯同我去攻取基列的拉末吗?」他回答说:「你我不分彼此,我的民与你的民一样,必与你同去争战。」
\par 4 约沙法对以色列王说:「请你先求问耶和华。」
\par 5 於是以色列王招聚先知四百人,问他们说:「我们上去攻取基列的拉末可以不可以?」他们说:「可以上去,因为神必将那城交在王的手里。」
\par 6 约沙法说:「这里不是还有耶和华的先知,我们可以求问他吗?」
\par 7 以色列王对约沙法说:「还有一个人,是音拉的儿子米该雅。我们可以托他求问耶和华,只是我恨他;因为他指著我所说的预言,不说吉语,常说凶言。」约沙法说:「王不必这样说。」
\par 8 以色列王就召了一个太监来,说:「你快去将音拉的儿子米该雅召来。」
\par 9 以色列王和犹大王约沙法在撒玛利亚城门前的空场上,各穿朝服坐在位上,所有的先知都在他们面前说预言。
\par 10 基拿拿的儿子西底家造了两个铁角,说:「耶和华如此说:『你要用这角抵触亚兰人,直到将他们灭尽。』」
\par 11 所有的先知也都这样预言说:「可以上基列的拉末去,必然得胜,因为耶和华必将那城交在王的手中。」
\par 12 那去召米该雅的使者对米该雅说:「众先知一口同音地都向王说吉言,你不如与他们说一样的话,也说吉言。」
\par 13 米该雅说:「我指著永生的耶和华起誓,我的神说什麽,我就说什麽。」
\par 14 米该雅到王面前,王问他说:「米该雅啊,我们上去攻取基列的拉末可以不可以?」他说:「可以上去,必然得胜,敌人必交在你们手里。」
\par 15 王对他说:「我当嘱咐你几次,你才奉耶和华的名向我说实话呢?」
\par 16 米该雅说:「我看见以色列众民散在山上,如同没有牧人的羊群一般。耶和华说:『这民没有主人,他们可以平平安安的各归各家去。』」
\par 17 以色列王对约沙法说:「我岂没有告诉你,这人指著我所说的预言,不说吉语,单说凶言吗?」
\par 18 米该雅说:「你们要听耶和华的话。我看见耶和华坐在宝座上,天上的万军侍立在他左右。
\par 19 耶和华说:『谁去引诱以色列王亚哈上基列的拉末去阵亡呢?』这个就这样说,那个就那样说。
\par 20 随後,有一个神灵出来,站在耶和华面前说:『我去引诱他。』耶和华问他说:『你用何法呢?』
\par 21 他说:『我去,要在他众先知口中作谎言的灵。』耶和华说:『这样,你必能引诱他,你去如此行吧!』
\par 22 现在耶和华使谎言的灵入了你这些先知的口,并且耶和华已经命定降祸与你。」
\par 23 基拿拿的儿子西底家前来打米该雅的脸,说:「耶和华的灵从那里离开我与你说话呢?」
\par 24 米该雅说:「你进严密的屋子藏躲的那日,就必看见了。」
\par 25 以色列王说:「将米该雅带回,交给邑宰亚们和王的儿子约阿施,说:
\par 26 『王如此说:把这个人下在监里,使他受苦,吃不饱喝不足,等候我平平安安的回来。』」
\par 27 米该雅说:「你若能平安回来,那就是耶和华没有藉我说这话了」;又说:「众民哪,你们都要听!」
\par 28 以色列王和犹大王约沙法上基列的拉末去了。
\par 29 以色列王对约沙法说:「我要改装上阵,你可以仍穿王服。」於是以色列王改装,他们就上阵去了。
\par 30 先是亚兰王吩咐车兵长说:「他们的兵将,无论大小,你们都不可与他们争战,只要与以色列王争战。」
\par 31 车兵长看见约沙法便说,这必是以色列王,就转过去与他争战。约沙法一呼喊,耶和华就帮助他,神又感动他们离开他。
\par 32 车兵长见不是以色列王,就转去不追他了。
\par 33 有一人随便开弓,恰巧射入以色列王的甲缝里。王对赶车的说:「我受了重伤,你转过车来,拉我出阵吧!」
\par 34 那日阵势越战越猛,以色列王勉强站在车上抵挡亚兰人,直到晚上。约在日落的时候,王就死了。

\chapter{19}

\par 1 犹大王约沙法平平安安的回耶路撒冷,到宫里去了。
\par 2 先见哈拿尼的儿子耶户出来迎接约沙法王,对他说:「你岂当帮助恶人,爱那恨恶耶和华的人呢?因此耶和华的忿怒临到你。
\par 3 然而你还有善行,因你从国中除掉木偶,立定心意寻求神。」
\par 4 约沙法住在耶路撒冷,以後又出巡民间,从别是巴直到以法莲山地,引导民归向耶和华他们列祖的神;
\par 5 又在犹大国中遍地的坚固城里设立审判官,
\par 6 对他们说:「你们办事应当谨慎;因为你们判断不是为人,乃是为耶和华。判断的时候,他必与你们同在。
\par 7 现在你们应当敬畏耶和华,谨慎办事;因为耶和华我们的神没有不义,不偏待人,也不受贿赂。」
\par 8 约沙法从利未人和祭司,并以色列族长中派定人,在耶路撒冷为耶和华判断,听民间的争讼,就回耶路撒冷去了。
\par 9 约沙法嘱咐他们说:「你们当敬畏耶和华,忠心诚实办事。
\par 10 住在各城里你们的弟兄,若有争讼的事来到你们这里,或为流血,或犯律法、诫命、律例、典章,你们要警戒他们,免得他们得罪耶和华,以致他的忿怒临到你们和你们的弟兄;这样行,你们就没有罪了。
\par 11 凡属耶和华的事,有大祭司亚玛利雅管理你们;凡属王的事,有犹大支派的族长以实玛利的儿子西巴第雅管理你们;在你们面前有利未人作官长。你们应当壮胆办事,愿耶和华与善人同在。」

\chapter{20}

\par 1 此後,摩押人和亚扪人,又有米乌尼人,一同来攻击约沙法。
\par 2 有人来报告约沙法说:「从海外亚兰(又作以东)那边有大军来攻击你,如今他们在哈洗逊他玛,就是隐基底。」
\par 3 约沙法便惧怕,定意寻求耶和华,在犹大全地宣告禁食。
\par 4 於是犹大人聚会,求耶和华帮助。犹大各城都有人出来寻求耶和华。
\par 5 约沙法就在犹大和耶路撒冷的会中,站在耶和华殿的新院前,
\par 6 说:「耶和华我们列祖的神啊,你不是天上的神吗?你不是万邦万国的主宰吗?在你手中有大能大力,无人能抵挡你。
\par 7 我们的神啊,你不是曾在你民以色列人面前驱逐这地的居民,将这地赐给你朋友亚伯拉罕的後裔永远为业吗?
\par 8 他们住在这地,又为你的名建造圣所,说:
\par 9 『倘有祸患临到我们,或刀兵灾殃,或瘟疫饥荒,我们在急难的时候,站在这殿前向你呼求,你必垂听而拯救,因为你的名在这殿里。』
\par 10 从前以色列人出埃及地的时候,你不容以色列人侵犯亚扪人、摩押人,和西珥山人,以色列人就离开他们,不灭绝他们。
\par 11 看哪,他们怎样报复我们,要来驱逐我们出离你的地,就是你赐给我们为业之地。
\par 12 我们的神啊,你不惩罚他们吗?因为我们无力抵挡这来攻击我们的大军,我们也不知道怎样行,我们的眼目单仰望你。」
\par 13 犹大众人和他们的婴孩、妻子、儿女都站在耶和华面前。
\par 14 那时,耶和华的灵在会中临到利未人亚萨的後裔玛探雅的玄孙,耶利的曾孙,比拿雅的孙子,撒迦利雅的儿子雅哈悉。
\par 15 他说:「犹大众人、耶路撒冷的居民,和约沙法王,你们请听。耶和华对你们如此说:『不要因这大军恐惧惊惶;因为胜败不在乎你们,乃在乎神。
\par 16 明日你们要下去迎敌,他们是从洗斯坡上来,你们必在耶鲁伊勒旷野前的谷口遇见他们。
\par 17 犹大和耶路撒冷人哪,这次你们不要争战,要摆阵站著,看耶和华为你们施行拯救。不要恐惧,也不要惊惶。明日当出去迎敌,因为耶和华与你们同在。』」
\par 18 约沙法就面伏於地,犹大众人和耶路撒冷的居民也俯伏在耶和华面前,叩拜耶和华。
\par 19 哥辖族和可拉族的利未人都起来,用极大的声音赞美耶和华以色列的神。
\par 20 次日清早,众人起来往提哥亚的旷野去。出去的时候,约沙法站著说:「犹大人和耶路撒冷的居民哪,要听我说:信耶和华你们的神就必立稳;信他的先知就必亨通。」
\par 21 约沙法既与民商议了,就设立歌唱的人,颂赞耶和华,使他们穿上圣洁的礼服,走在军前赞美耶和华说:「当称谢耶和华,因他的慈爱永远长存!」
\par 22 众人方唱歌赞美的时候,耶和华就派伏兵击杀那来攻击犹大人的亚扪人、摩押人,和西珥山人,他们就被打败了。
\par 23 因为亚扪人和摩押人起来,击杀住西珥山的人,将他们灭尽;灭尽住西珥山的人之後,他们又彼此自相击杀。
\par 24 犹大人来到旷野的望楼,向那大军观看,见尸横遍地,没有一个逃脱的。
\par 25 约沙法和他的百姓就来收取敌人的财物,在尸首中见了许多财物、珍宝,他们剥脱下来的多得不可携带;因为甚多,直收取了三日。
\par 26 第四日众人聚集在比拉迦谷(就是称颂的意思),在那里称颂耶和华。因此那地方名叫比拉迦谷,直到今日。
\par 27 犹大人和耶路撒冷人都欢欢喜喜地回耶路撒冷,约沙法率领他们;因为耶和华使他们战胜仇敌,就欢喜快乐。
\par 28 他们弹琴、鼓瑟、吹号来到耶路撒冷,进了耶和华的殿。
\par 29 列邦诸国听见耶和华战败以色列的仇敌,就甚惧怕。
\par 30 这样,约沙法的国得享太平,因为神赐他四境平安。
\par 31 约沙法作犹大王,登基的时候年三十五岁,在耶路撒冷作王二十五年。他母亲名叫阿苏巴,乃示利希的女儿。
\par 32 约沙法效法他父亚撒所行的,不偏左右,行耶和华眼中看为正的事。
\par 33 只是邱坛还没有废去,百姓也没有立定心意归向他们列祖的神。
\par 34 约沙法其余的事,自始至终都写在哈拿尼的儿子耶户的书上,也载入以色列诸王记上。
\par 35 此後,犹大王约沙法与以色列王亚哈谢交好;亚哈谢行恶太甚。
\par 36 二王合夥造船要往他施去,遂在以旬迦别造船。
\par 37 那时玛利沙人、多大瓦的儿子以利以谢向约沙法预言说:「因你与亚哈谢交好,耶和华必破坏你所造的。」後来那船果然破坏,不能往他施去了。

\chapter{21}

\par 1 约沙法与他列祖同睡,葬在大卫城他列祖的坟地里。他儿子约兰接续他作王。
\par 2 约兰有几个兄弟,就是约沙法的儿子亚撒利雅、耶歇、撒迦利雅、亚撒利雅、米迦勒、示法提雅。这都是犹大王约沙法的儿子。
\par 3 他们的父亲将许多金银、财宝,和犹大地的坚固城赐给他们;但将国赐给约兰,因为他是长子。
\par 4 约兰兴起坐他父的位,奋勇自强,就用刀杀了他的众兄弟和以色列的几个首领。
\par 5 约兰登基的时候年三十二岁,在耶路撒冷作王八年。
\par 6 他行以色列诸王的道,与亚哈家一样;因他娶了亚哈的女儿为妻,行耶和华眼中看为恶的事。
\par 7 耶和华却因自己与大卫所立的约,不肯灭大卫的家,照他所应许的,永远赐灯光与大卫和他的子孙。
\par 8 约兰年间,以东人背叛犹大,脱离他的权下,自己立王。
\par 9 约兰就率领军长和所有的战车,夜间起来,攻击围困他的以东人和车兵长。
\par 10 这样,以东人背叛犹大,脱离他的权下,直到今日。那时,立拿人也背叛了,因为约兰离弃耶和华他列祖的神。
\par 11 他又在犹大诸山建筑邱坛,使耶路撒冷的居民行邪淫,诱惑犹大人。
\par 12 先知以利亚达信与约兰说:「耶和华你祖大卫的神如此说:『因为你不行你父约沙法和犹大王亚撒的道,
\par 13 乃行以色列诸王的道,使犹大人和耶路撒冷的居民行邪淫,像亚哈家一样,又杀了你父家比你好的诸兄弟。
\par 14 故此,耶和华降大灾与你的百姓和你的妻子、儿女,并你一切所有的。
\par 15 你的肠子必患病,日加沉重,以致你的肠子坠落下来。』」
\par 16 以後,耶和华激动非利士人和靠近古实的亚拉伯人来攻击约兰。
\par 17 他们上来攻击犹大,侵入境内,掳掠了王宫里所有的财货和他的妻子、儿女,除了他小儿子约哈斯(又名亚哈谢)之外,没有留下一个儿子。
\par 18 这些事以後,耶和华使约兰的肠子患不能医治的病。
\par 19 他患此病缠绵日久,过了二年,肠子坠落下来,病重而死。他的民没有为他烧什麽物件,像从前为他列祖所烧的一样。
\par 20 约兰登基的时候年三十二岁,在耶路撒冷作王八年。他去世无人思慕,众人葬他在大卫城,只是不在列王的坟墓里。

\chapter{22}

\par 1 耶路撒冷的居民立约兰的小儿子亚哈谢接续他作王;因为跟随亚拉伯人来攻营的军兵曾杀了亚哈谢的众兄长。这样,犹大王约兰的儿子亚哈谢作了王。
\par 2 亚哈谢登基的时候年四十二岁(列王下八章二十六节是二十二岁),在耶路撒冷作王一年。他母亲名叫亚他利雅,是暗利的孙女。
\par 3 亚哈谢也行亚哈家的道;因为他母亲给他主谋,使他行恶。
\par 4 他行耶和华眼中看为恶的事,像亚哈家一样;因他父亲死後有亚哈家的人给他主谋,以致败坏。
\par 5 他听从亚哈家的计谋,同以色列王亚哈的儿子约兰往基列的拉末去,与亚兰王哈薛争战;亚兰人打伤了约兰。
\par 6 约兰回到耶斯列,医治在拉末与亚兰王哈薛打仗所受的伤,犹大王约兰的儿子亚撒利雅(就是亚哈谢)因为亚哈的儿子约兰病了,就下到耶斯列看望他。
\par 7 亚哈谢去见约兰就被害了,这是出乎神;因为他到了,就同约兰出去攻击宁示的孙子耶户。这耶户是耶和华所膏、使他剪除亚哈家的。
\par 8 耶户讨亚哈家罪的时候,遇见犹大的众首领和亚哈谢的众侄子服事亚哈谢,就把他们都杀了。
\par 9 亚哈谢藏在撒玛利亚,耶户寻找他,众人将他拿住,送到耶户那里,就杀了他,将他葬埋;因他们说,他是那尽心寻求耶和华之约沙法的儿子。这样,亚哈谢的家无力保守国权。
\par 10 亚哈谢的母亲亚他利雅见他儿子死了,就起来剿灭犹大王室。
\par 11 但王的女儿约示巴将亚哈谢的儿子约阿施从那被杀的王子中偷出来,把他和他的乳母都藏在卧房里。约示巴是约兰王的女儿,亚哈谢的妹子,祭司耶何耶大的妻。他收藏约阿施,躲避亚他利雅,免得被杀。
\par 12 约阿施和他们一同藏在神殿里六年;亚他利雅篡了国位。

\chapter{23}

\par 1 第七年,耶何耶大奋勇自强,将百夫长耶罗罕的儿子亚撒利雅,约哈难的儿子以实玛利,俄备得的儿子亚撒利雅,亚大雅的儿子玛西雅,细基利的儿子以利沙法召来,与他们立约。
\par 2 他们走遍犹大,从犹大各城里招聚利未人和以色列的众族长到耶路撒冷来。
\par 3 会众在神殿里与王立约。耶何耶大对他们说:「看哪,王的儿子必当作王,正如耶和华指著大卫子孙所应许的话」;
\par 4 又说:「你们当这样行:祭司和利未人凡安息日进班的,三分之一要把守各门,
\par 5 三分之一要在王宫,三分之一要在基址门;众百姓要在耶和华殿的院内。
\par 6 除了祭司和供职的利未人之外,不准别人进耶和华的殿;惟独他们可以进去,因为他们圣洁。众百姓要遵守耶和华所吩咐的。
\par 7 利未人要手中各拿兵器,四围护卫王;凡擅入殿宇的,必当治死。王出入的时候,你们当跟随他。」
\par 8 利未人和犹大众人都照著祭司耶何耶大一切所吩咐的去行,各带所管安息日进班出班的人来,因为祭司耶何耶大不许他们下班。
\par 9 祭司耶何耶大便将神殿里所藏大卫王的枪、盾牌、挡牌交给百夫长,
\par 10 又分派众民手中各拿兵器,在坛和殿那里,从殿右直到殿左,站在王子的四围;
\par 11 於是领王子出来,给他戴上冠冕,将律法书交给他,立他作王。耶何耶大和众子膏他,众人说:「愿王万岁!」
\par 12 亚他利雅听见民奔走赞美王的声音,就到民那里,进耶和华的殿,
\par 13 看见王站在殿门的柱旁,百夫长和吹号的人侍立在王左右,国民都欢乐吹号,又有歌唱的,用各样的乐器领人歌唱赞美;亚他利雅就撕裂衣服,喊叫说:「反了!反了!」
\par 14 祭司耶何耶大带管辖军兵的百夫长出来,吩咐他们说:「将他赶到班外,凡跟随他的必用刀杀死!」因为祭司说:「不可在耶和华殿里杀他。」
\par 15 众兵就闪开,让他去;他走到王宫的马门,便在那里把他杀了。
\par 16 耶何耶大与众民和王立约,都要作耶和华的民。
\par 17 於是众民都到巴力庙,拆毁了庙,打碎坛和像,又在坛前将巴力的祭司玛坦杀了。
\par 18 耶何耶大派官看守耶和华的殿,是在祭司利未人手下。这祭司利未人是大卫分派在耶和华殿中、照摩西律法上所写的,给耶和华献燔祭,又按大卫所定的例,欢乐歌唱;
\par 19 且设立守门的把守耶和华殿的各门,无论为何事,不洁净的人都不准进去。
\par 20 又率领百夫长和贵胄,与民间的官长,并国中的众民,请王从耶和华殿下来,由上门进入王宫,立王坐在国位上。
\par 21 国民都欢乐,合城都安静。众人已将亚他利雅用刀杀了。

\chapter{24}

\par 1 约阿施登基的时候年七岁,在耶路撒冷作王四十年。他母亲名叫西比亚,是别是巴人。
\par 2 祭司耶何耶大在世的时候,约阿施行耶和华眼中看为正的事。
\par 3 耶何耶大为他娶了两个妻,并且生儿养女。
\par 4 此後,约阿施有意重修耶和华的殿,
\par 5 便召聚众祭司和利未人,吩咐他们说:「你们要往犹大各城去,使以色列众人捐纳银子,每年可以修理你们神的殿;你们要急速办理这事。」只是利未人不急速办理。
\par 6 王召了大祭司耶何耶大来,对他说:「从前耶和华的仆人摩西,为法柜的帐幕与以色列会众所定的捐项,你为何不叫利未人照这例从犹大和耶路撒冷带来作殿的费用呢?」
\par 7 (因为那恶妇亚他利雅的众子曾拆毁神的殿,又用耶和华殿中分别为圣的物供奉巴力。)
\par 8 於是王下令,众人做了一柜,放在耶和华殿的门外,
\par 9 又通告犹大和耶路撒冷的百姓,要将神仆人摩西在旷野所吩咐以色列人的捐项给耶和华送来。
\par 10 众首领和百姓都欢欢喜喜地将银子送来,投入柜中,直到捐完。
\par 11 利未人见银子多了,就把柜抬到王所派的司事面前;王的书记和大祭司的属员来将柜倒空,仍放在原处。日日都是这样,积蓄的银子甚多。
\par 12 王与耶何耶大将银子交给耶和华殿里办事的人,他们就雇了石匠、木匠重修耶和华的殿,又雇了铁匠、铜匠修理耶和华的殿。
\par 13 工人操作,渐渐修成,将神殿修造得与从前一样,而且甚是坚固。
\par 14 工程完了,他们就把其余的银子拿到王与耶何耶大面前,用以制造耶和华殿供奉所用的器皿和调羹,并金银的器皿。耶何耶大在世的时候,众人常在耶和华殿里献燔祭。
\par 15 耶何耶大年纪老迈,日子满足而死。死的时候年一百三十岁,
\par 16 葬在大卫城列王的坟墓里;因为他在以色列人中行善,又事奉神,修理神的殿。
\par 17 耶何耶大死後,犹大的众首领来朝拜王;王就听从他们。
\par 18 他们离弃耶和华他们列祖神的殿,去事奉亚舍拉和偶像;因他们这罪,就有忿怒临到犹大和耶路撒冷。
\par 19 但神仍遣先知到他们那里,引导他们归向耶和华。这先知警戒他们,他们却不肯听。
\par 20 那时,神的灵感动祭司耶何耶大的儿子撒迦利亚,他就站在上面对民说:「神如此说:你们为何干犯耶和华的诫命,以致不得亨通呢?因为你们离弃耶和华,所以他也离弃你们。」
\par 21 众民同心谋害撒迦利亚,就照王的吩咐,在耶和华殿的院内用石头打死他。
\par 22 这样,约阿施王不想念撒迦利亚的父亲耶何耶大向自己所施的恩,杀了他的儿子。撒迦利亚临死的时候说:「愿耶和华鉴察伸冤!」
\par 23 满了一年,亚兰的军兵上来攻击约阿施,来到犹大和耶路撒冷,杀了民中的众首领,将所掠的财货送到大马色王那里。
\par 24 亚兰的军兵虽来了一小队,耶和华却将大队的军兵交在他们手里,是因犹大人离弃耶和华他们列祖的神,所以藉亚兰人惩罚约阿施。
\par 25 亚兰人离开约阿施的时候,他患重病;臣仆背叛他,要报祭司耶何耶大儿子流血之仇,杀他在床上,葬他在大卫城,只是不葬在列王的坟墓里。
\par 26 背叛他的是亚扪妇人示米押的儿子撒拔和摩押妇人示米利的儿子约萨拔。
\par 27 至於他的众子和他所受的警戒,并他重修神殿的事,都写在列王的传上。他儿子亚玛谢接续他作王。

\chapter{25}

\par 1 亚玛谢登基的时候年二十五岁,在耶路撒冷作王二十九年。他母亲名叫约耶但,是耶路撒冷人。
\par 2 亚玛谢行耶和华眼中看为正的事,只是心不专诚。
\par 3 国一坚定,就把杀他父王的臣仆杀了,
\par 4 却没有治死他们的儿子,是照摩西律法书上耶和华所吩咐的说:「不可因子杀父,也不可因父杀子,各人要为本身的罪而死。」
\par 5 亚玛谢招聚犹大人,按著犹大和便雅悯的宗族设立千夫长、百夫长,又数点人数,从二十岁以外,能拿枪拿盾牌出去打仗的精兵共有三十万;
\par 6 又用银子一百他连得,从以色列招募了十万大能的勇士。
\par 7 有一个神人来见亚玛谢,对他说:「王啊,不要使以色列的军兵与你同去,因为耶和华不与以色列人以法莲的後裔同在。
\par 8 你若一定要去,就奋勇争战吧!但神必使你败在敌人面前;因为神能助人得胜,也能使人倾败。」
\par 9 亚玛谢问神人说:「我给了以色列军的那一百他连得银子怎麽样呢?」神人回答说:「耶和华能把更多的赐给你。」
\par 10 於是亚玛谢将那从以法莲来的军兵分别出来,叫他们回家去。故此,他们甚恼怒犹大人,气忿忿地回家去了。
\par 11 亚玛谢壮起胆来,率领他的民到盐谷,杀了西珥人一万。
\par 12 犹大人又生擒了一万带到山崖上,从那里把他们扔下去,以致他们都摔碎了。
\par 13 但亚玛谢所打发回去、不许一同出征的那些军兵攻打犹大各城,从撒玛利亚直到伯和仑,杀了三千人,抢了许多财物。
\par 14 亚玛谢杀了以东人回来,就把西珥的神像带回,立为自己的神,在他面前叩拜烧香。
\par 15 因此,耶和华的怒气向亚玛谢发作,就差一个先知去见他,说:「这些神不能救他的民脱离你的手,你为何寻求他呢?」
\par 16 先知与王说话的时候,王对他说:「谁立你作王的谋士呢?你住口吧!为何找打呢?」先知就止住了,又说:「你行这事,不听从我的劝戒,我知道神定意要灭你。」
\par 17 犹大王亚玛谢与群臣商议,就差遣使者去见耶户的孙子、约哈斯的儿子、以色列王约阿施,说:「你来,我们二人相见於战场。」
\par 18 以色列王约阿施差遣使者去见犹大王亚玛谢,说:「利巴嫩的蒺藜差遣使者去见利巴嫩的香柏树,说:『将你的女儿给我儿子为妻。』後来利巴嫩有一个野兽经过,把蒺藜践踏了。
\par 19 你说:『看哪,我打败了以东人』,你就心高气傲,以致矜夸。你在家里安居就罢了,为何要惹祸使自己和犹大国一同败亡呢?」
\par 20 亚玛谢却不肯听从。这是出乎神,好将他们交在敌人手里,因为他们寻求以东的神。
\par 21 於是以色列王约阿施上来,在犹大的伯示麦与犹大王亚玛谢相见於战场。
\par 22 犹大人败在以色列人面前,各自逃回家里去了。
\par 23 以色列王约阿施在伯示麦擒住约哈斯(就是亚哈谢)的孙子、约阿施的儿子、犹大王亚玛谢,将他带到耶路撒冷,又拆毁耶路撒冷的城墙,从以法莲门直到角门,共四百肘;
\par 24 又将俄别以东所看守神殿里的一切金银和器皿,与王宫里的财宝都拿了去,并带人去为质,就回撒玛利亚去了。
\par 25 以色列王约哈斯的儿子约阿施死後,犹大王约阿施的儿子亚玛谢又活了十五年。
\par 26 亚玛谢其余的事,自始至终不都写在犹大和以色列诸王记上吗?
\par 27 自从亚玛谢离弃耶和华之後,在耶路撒冷有人背叛他,他就逃到拉吉;叛党却打发人到拉吉,将他杀了。
\par 28 人就用马将他的尸首驮回,葬在犹大京城他列祖的坟地里。

\chapter{26}

\par 1 犹大众民立亚玛谢的儿子乌西雅(又名亚撒利雅)接续他父作王,那时他年十六岁。
\par 2 (亚玛谢与他列祖同睡之後,乌西雅收回以禄仍归犹大,又重新修理。)
\par 3 乌西雅登基的时候年十六岁,在耶路撒冷作王五十二年。他母亲名叫耶可利雅,是耶路撒冷人。
\par 4 乌西雅行耶和华眼中看为正的事,效法他父亚玛谢一切所行的;
\par 5 通晓神默示,撒迦利亚在世的时候,乌西雅定意寻求神;他寻求耶和华,神就使他亨通。
\par 6 他出去攻击非利士人,拆毁了迦特城、雅比尼城,和亚实突城;在非利士人中,在亚实突境内,又建筑了些城。
\par 7 神帮助他攻击非利士人和住在姑珥巴力的亚拉伯人,并米乌尼人。
\par 8 亚扪人给乌西雅进贡。他的名声传到埃及,因他甚是强盛。
\par 9 乌西雅在耶路撒冷的角门和谷门,并城墙转弯之处,建筑城楼,且甚坚固;
\par 10 又在旷野与高原和平原,建筑望楼,挖了许多井,因他的牲畜甚多;又在山地和佳美之地,有农夫和修理葡萄园的人,因为他喜悦农事。
\par 11 乌西雅又有军兵,照书记耶利和官长玛西雅所数点的,在王的一个将军哈拿尼雅手下,分队出战。
\par 12 族长、大能勇士的总数共有二千六百人,
\par 13 他们手下的军兵共有三十万七千五百人,都有大能,善於争战,帮助王攻击仇敌。
\par 14 乌西雅为全军预备盾牌、枪、盔、甲、弓,和甩石的机弦,
\par 15 又在耶路撒冷使巧匠做机器,安在城楼和角楼上,用以射箭发石。乌西雅的名声传到远方;因为他得了非常的帮助,甚是强盛。
\par 16 他既强盛,就心高气傲,以致行事邪僻,干犯耶和华他的神,进耶和华的殿,要在香坛上烧香。
\par 17 祭司亚撒利雅率领耶和华勇敢的祭司八十人,跟随他进去。
\par 18 他们就阻挡乌西雅王,对他说:「乌西雅啊,给耶和华烧香不是你的事,乃是亚伦子孙承接圣职祭司的事。你出圣殿吧!因为你犯了罪。你行这事,耶和华 神必不使你得荣耀。」
\par 19 乌西雅就发怒,手拿香炉要烧香。他向祭司发怒的时候,在耶和华殿中香坛旁众祭司面前,额上忽然发出大麻疯。
\par 20 大祭司亚撒利雅和众祭司观看,见他额上发出大麻疯,就催他出殿;他自己也急速出去,因为耶和华降灾与他。
\par 21 乌西雅王长大麻疯直到死日,因此住在别的宫里,与耶和华的殿隔绝。他儿子约坦管理家事,治理国民。
\par 22 乌西雅其余的事,自始至终都是亚摩斯的儿子先知以赛亚所记的。
\par 23 乌西雅与他列祖同睡,葬在王陵的田间他列祖的坟地里;因为人说,他是长大麻疯的。他儿子约坦接续他作王。

\chapter{27}

\par 1 约坦登基的时候年二十五岁,在耶路撒冷作王十六年,他母亲名叫耶路沙,是撒督的女儿。
\par 2 约坦行耶和华眼中看为正的事,效法他父乌西雅一切所行的,只是不入耶和华的殿。百姓还行邪僻的事。
\par 3 约坦建立耶和华殿的上门,在俄斐勒城上多有建造,
\par 4 又在犹大山地建造城邑,在树林中建筑营寨和高楼。
\par 5 约坦与亚扪人的王打仗胜了他们,当年他们进贡银一百他连得,小麦一万歌珥,大麦一万歌珥;第二年、第三年也是这样。
\par 6 约坦在耶和华他神面前行正道,以致日渐强盛。
\par 7 约坦其余的事和一切争战,并他的行为,都写在以色列和犹大列王记上。
\par 8 他登基的时候年二十五岁,在耶路撒冷作王十六年。
\par 9 约坦与他列祖同睡,葬在大卫城里。他儿子亚哈斯接续他作王。

\chapter{28}

\par 1 亚哈斯登基的时候年二十岁,在耶路撒冷作王十六年;不像他祖大卫行耶和华眼中看为正的事,
\par 2 却行以色列诸王的道,又铸造巴力的像,
\par 3 并且在欣嫩子谷烧香,用火焚烧他的儿女,行耶和华在以色列人面前所驱逐的外邦人那可憎的事;
\par 4 并在邱坛上、山冈上、各青翠树下献祭烧香。
\par 5 所以,耶和华他的神将他交在亚兰王手里。亚兰王打败他,掳了他许多的民,带到大马色去。神又将他交在以色列王手里,以色列王向他大行杀戮。
\par 6 利玛利的儿子比加一日杀了犹大人十二万,都是勇士,因为他们离弃了耶和华他们列祖的神。
\par 7 有一个以法莲中的勇士,名叫细基利,杀了王的儿子玛西雅和管理王宫的押斯利甘,并宰相以利加拿。
\par 8 以色列人掳了他们的弟兄,连妇人带儿女共有二十万,又掠了许多的财物,带到撒玛利亚去了。
\par 9 但那里有耶和华的一个先知,名叫俄德,出来迎接往撒玛利亚去的军兵,对他们说:「因为耶和华你们列祖的神恼怒犹大人,所以将他们交在你们手里,你们竟怒气冲天,大行杀戮。
\par 10 如今你们又有意强逼犹大人和耶路撒冷人作你们的奴婢,你们岂不也有得罪耶和华你们神的事吗?
\par 11 现在你们当听我说,要将掳来的弟兄释放回去,因为耶和华向你们已经大发烈怒。」
\par 12 於是,以法莲人的几个族长,就是约哈难的儿子亚撒利雅、米实利末的儿子比利家、沙龙的儿子耶希西家、哈得莱的儿子亚玛撒起来拦挡出兵回来的人,
\par 13 对他们说:「你们不可带进这被掳的人来!你们想要使我们得罪耶和华,加增我们的罪恶过犯?因为我们的罪过甚大,已经有烈怒临到以色列人了。」
\par 14 於是带兵器的人将掳来的人口和掠来的财物都留在众首领和会众的面前。
\par 15 以上提名的那些人就站起,使被掳的人前来;其中有赤身的,就从所掠的财物中拿出衣服和鞋来,给他们穿,又给他们吃喝,用膏抹他们;其中有软弱的,就使他们骑驴,送到棕树城耶利哥他们弟兄那里;随後就回撒玛利亚去了。
\par 16 那时,亚哈斯王差遣人去见亚述诸王,求他们帮助;
\par 17 因为以东人又来攻击犹大,掳掠子民。
\par 18 非利士人也来侵占高原和犹大南方的城邑,取了伯示麦、亚雅仑、基低罗,梭哥和属梭哥的乡村,亭纳和属亭纳的乡村,瑾锁和属瑾锁的乡村,就住在那里。
\par 19 因为以色列王亚哈斯在犹大放肆,大大干犯耶和华,所以耶和华使犹大卑微。
\par 20 亚述王提革拉毗尼色上来,却没有帮助他,反倒欺凌他。
\par 21 亚哈斯从耶和华殿里和王宫中,并首领家内所取的财宝给了亚述王,这也无济於事。
\par 22 这亚哈斯王在急难的时候,越发得罪耶和华。
\par 23 他祭祀攻击他的大马色之神,说:「因为亚兰王的神帮助他们,我也献祭与他,他好帮助我。」但那些神使他和以色列众人败亡了。
\par 24 亚哈斯将神殿里的器皿都聚了来,毁坏了,且封锁耶和华殿的门;在耶路撒冷各处的拐角建筑祭坛,
\par 25 又在犹大各城建立邱坛,与别神烧香,惹动耶和华他列祖神的怒气。
\par 26 亚哈斯其余的事和他的行为,自始至终都写在犹大和以色列诸王记上。
\par 27 亚哈斯与他列祖同睡,葬在耶路撒冷城里,没有送入以色列诸王的坟墓中。他儿子希西家接续他作王。

\chapter{29}

\par 1 希西家登基的时候年二十五岁,在耶路撒冷作王二十九年。他母亲名叫亚比雅,是撒迦利雅的女儿。
\par 2 希西家行耶和华眼中看为正的事,效法他祖大卫一切所行的。
\par 3 元年正月,开了耶和华殿的门,重新修理。
\par 4 他召众祭司和利未人来,聚集在东边的宽阔处,
\par 5 对他们说:「利未人哪,当听我说:现在你们要洁净自己,又洁净耶和华你们列祖神的殿,从圣所中除去污秽之物。
\par 6 我们列祖犯了罪,行耶和华我们神眼中看为恶的事,离弃他,转脸背向他的居所,
\par 7 封锁廊门,吹灭灯火,不在圣所中向以色列神烧香,或献燔祭。
\par 8 因此,耶和华的忿怒临到犹大和耶路撒冷,将其中的人抛来抛去,令人惊骇、嗤笑,正如你们亲眼所见的。
\par 9 所以我们的祖宗倒在刀下,我们的妻子儿女也被掳掠。
\par 10 现在我心中有意与耶和华以色列的神立约,好使他的烈怒转离我们。
\par 11 我的众子啊,现在不要懈怠;因为耶和华拣选你们站在他面前事奉他,与他烧香。」
\par 12 於是,利未人哥辖的子孙、亚玛赛的儿子玛哈,亚撒利雅的儿子约珥;米拉利的子孙、亚伯底的儿子基士,耶哈利勒的儿子亚撒利雅;革顺的子孙、薪玛的儿子约亚,约亚的儿子伊甸;
\par 13 以利撒反的子孙申利和耶利;亚萨的子孙撒迦利雅和玛探雅;
\par 14 希幔的子孙耶歇和示每;耶杜顿的子孙示玛雅和乌薛;
\par 15 起来聚集他们的弟兄,洁净自己,照著王的吩咐、耶和华的命令,进去洁净耶和华的殿。
\par 16 祭司进入耶和华的殿要洁净殿,将殿中所有污秽之物搬到耶和华殿的院内,利未人接去,搬到外头汲沦溪边。
\par 17 从正月初一日洁净起,初八日到了耶和华的殿廊,用八日的工夫洁净耶和华的殿,到正月十六日才洁净完了。
\par 18 於是,他们晋见希西家王,说:「我们已将耶和华的全殿和燔祭坛,并坛的一切器皿、陈设饼的桌子,与桌子的一切器皿都洁净了;
\par 19 并且亚哈斯王在位犯罪的时候所废弃的器皿,我们预备齐全,且洁净了,现今都在耶和华的坛前。」
\par 20 希西家王清早起来,聚集城里的首领都上耶和华的殿;
\par 21 牵了七只公牛,七只公羊,七只羊羔,七只公山羊,要为国、为殿、为犹大人作赎罪祭。王吩咐亚伦的子孙众祭司,献在耶和华的坛上,
\par 22 就宰了公牛,祭司接血洒在坛上,宰了公羊,把血洒在坛上,又宰了羊羔,也把血洒在坛上;
\par 23 把那作赎罪祭的公山羊牵到王和会众面前,他们就按手在其上。
\par 24 祭司宰了羊,将血献在坛上作赎罪祭,为以色列众人赎罪,因为王吩咐将燔祭和赎罪祭为以色列众人献上。
\par 25 王又派利未人在耶和华殿中敲钹,鼓瑟,弹琴,乃照大卫和他先见迦得,并先知拿单所吩咐的,就是耶和华藉先知所吩咐的。
\par 26 利未人拿大卫的乐器,祭司拿号,一同站立。
\par 27 希西家吩咐在坛上献燔祭,燔祭一献,就唱赞美耶和华的歌,用号,并用以色列王大卫的乐器相和。
\par 28 会众都敬拜,歌唱的歌唱,吹号的吹号,如此直到燔祭献完了。
\par 29 献完了祭,王和一切跟随的人都俯伏敬拜。
\par 30 希西家王与众首领又吩咐利未人用大卫和先见亚萨的诗词颂赞耶和华;他们就欢欢喜喜地颂赞耶和华,低头敬拜。
\par 31 希西家说:「你们既然归耶和华为圣,就要前来把祭物和感谢祭奉到耶和华殿里。」会众就把祭物和感谢祭奉来,凡甘心乐意的也将燔祭奉来。
\par 32 会众所奉的燔祭如下:公牛七十只,公羊一百只,羊羔二百只,这都是作燔祭献给耶和华的;
\par 33 又有分别为圣之物,公牛六百只,绵羊三千只。
\par 34 但祭司太少,不能剥尽燔祭牲的皮,所以他们的弟兄利未人帮助他们,直等燔祭的事完了,又等别的祭司自洁了;因为利未人诚心自洁,胜过祭司。
\par 35 燔祭和平安祭牲的脂油,并燔祭同献的奠祭甚多。这样,耶和华殿中的事务俱都齐备了(或作:就整顿了)。
\par 36 这事办的甚速,希西家和众民都喜乐,是因神为众民所预备的。

\chapter{30}

\par 1 希西家差遣人去见以色列和犹大众人,又写信给以法莲和玛拿西人,叫他们到耶路撒冷耶和华的殿,向耶和华以色列的神守逾越节;
\par 2 因为王和众首领,并耶路撒冷全会众已经商议,要在二月内守逾越节。
\par 3 正月(原文作那时)间他们不能守;因为自洁的祭司尚不敷用,百姓也没有聚集在耶路撒冷;
\par 4 王与全会众都以这事为善。
\par 5 於是定了命令,传遍以色列,从别是巴直到但,使他们都来,在耶路撒冷向耶和华以色列的神守逾越节;因为照所写的例,守这节的不多了(或作:因为民许久没有照所写的例守节了)。
\par 6 驿卒就把王和众首领的信,遵著王命传遍以色列和犹大。信内说:「以色列人哪,你们当转向耶和华亚伯拉罕、以撒、以色列的神,好叫他转向你们这脱离亚述王手的余民。
\par 7 你们不要效法你们列祖和你们的弟兄;他们干犯耶和华他们列祖的神,以致耶和华丢弃他们,使他们败亡(或作:令人惊骇),正如你们所见的。
\par 8 现在不要像你们列祖硬著颈项,只要归顺耶和华,进入他的圣所,就是永远成圣的居所;又要事奉耶和华你们的神,好使他的烈怒转离你们。
\par 9 你们若转向耶和华,你们的弟兄和儿女必在掳掠他们的人面前蒙怜恤,得以归回这地,因为耶和华你们的神有恩典、施怜悯。你们若转向他,他必不转脸不顾你们。」
\par 10 驿卒就由这城跑到那城,传遍了以法莲、玛拿西,直到西布伦。那里的人却戏笑他们,讥诮他们。
\par 11 然而亚设、玛拿西、西布伦中也有人自卑,来到耶路撒冷。
\par 12 神也感动犹大人,使他们一心遵行王与众首领凭耶和华之言所发的命令。
\par 13 二月,有许多人在耶路撒冷聚集,成为大会,要守除酵节。
\par 14 他们起来,把耶路撒冷的祭坛和烧香的坛尽都除去,抛在汲沦溪中。
\par 15 二月十四日,宰了逾越节的羊羔。祭司与利未人觉得惭愧,就洁净自己,把燔祭奉到耶和华殿中,
\par 16 遵著神人摩西的律法,照例站在自己的地方;祭司从利未人手里接过血来,洒在坛上。
\par 17 会中有许多人尚未自洁,所以利未人为一切不洁之人宰逾越节的羊羔,使他们在耶和华面前成为圣洁。
\par 18 以法莲、玛拿西、以萨迦、西布伦有许多人尚未自洁,他们却也吃逾越节的羊羔,不合所记录的定例。
\par 19 希西家为他们祷告说:「凡专心寻求神,就是耶和华他列祖之神的,虽不照著圣所洁净之礼自洁,求至善的耶和华也饶恕他。」
\par 20 耶和华垂听希西家的祷告,就饶恕(原文作医治)百姓。
\par 21 在耶路撒冷的以色列人大大喜乐,守除酵节七日。利未人和祭司用响亮的乐器,日日颂赞耶和华。
\par 22 希西家慰劳一切善於事奉耶和华的利未人。於是众人吃节筵七日,又献平安祭,且向耶和华他们列祖的神认罪。
\par 23 全会众商议,要再守节七日;於是欢欢喜喜地又守节七日。
\par 24 犹大王希西家赐给会众公牛一千只,羊七千只为祭物;众首领也赐给会众公牛一千只,羊一万只,并有许多的祭司洁净自己。
\par 25 犹大全会众、祭司、利未人,并那从以色列地来的会众和寄居的人,以及犹大寄居的人,尽都喜乐。
\par 26 这样,在耶路撒冷大有喜乐,自从以色列王大卫儿子所罗门的时候,在耶路撒冷没有这样的喜乐。
\par 27 那时,祭司、利未人起来,为民祝福。他们的声音蒙神垂听,他们的祷告达到天上的圣所。

\chapter{31}

\par 1 这事既都完毕,在那里的以色列众人就到犹大的城邑,打碎柱像,砍断木偶,又在犹大、便雅悯、以法莲、玛拿西遍地将邱坛和祭坛拆毁净尽。於是以色列众人各回各城,各归各地。
\par 2 希西家派定祭司利未人的班次,各按各职献燔祭和平安祭,又在耶和华殿(原文作营)门内事奉,称谢颂赞耶和华。
\par 3 王又从自己的产业中定出分来为燔祭,就是早晚的燔祭和安息日、月朔,并节期的燔祭,都是按耶和华律法上所载的;
\par 4 又吩咐住耶路撒冷的百姓将祭司、利未人所应得的分给他们,使他们专心遵守耶和华的律法。
\par 5 谕旨一出,以色列人就把初熟的五谷、新酒、油、蜜,和田地的出产多多送来,又把各物的十分之一送来的极多。
\par 6 住犹大各城的以色列人和犹大人也将牛羊的十分之一,并分别为圣归耶和华他们神之物,就是十分取一之物,尽都送来,积成堆垒;
\par 7 从三月积起,到七月才完。
\par 8 希西家和众首领来,看见堆垒,就称颂耶和华,又为耶和华的民以色列人祝福。
\par 9 希西家向祭司、利未人查问这堆垒。
\par 10 撒督家的大祭司亚撒利雅回答说:「自从民将供物送到耶和华殿以来,我们不但吃饱,且剩下的甚多;因为耶和华赐福给他的民,所剩下的才这样丰盛。」
\par 11 希西家吩咐在耶和华殿里预备仓房,他们就预备了。
\par 12 他们诚心将供物和十分取一之物,并分别为圣之物,都搬入仓内。利未人歌楠雅掌管这事,他兄弟示每为副管。
\par 13 耶歇、亚撒细雅、拿哈、亚撒黑、耶利末、约撒拔、以列、伊斯玛基雅、玛哈、比拿雅都是督理,在歌楠雅和他兄弟示每的手下,是希西家王和管理神殿的亚撒利雅所派的。
\par 14 守东门的利未人音拿的儿子可利,掌管乐意献与神的礼物,发放献与耶和华的供物和至圣的物。
\par 15 在他手下有伊甸、泯雅泯、耶书亚、示玛雅、亚玛利雅、示迦尼雅,在祭司的各城里供紧要的职任,无论弟兄大小,都按著班次分给他们。
\par 16 按家谱,三岁以外的男丁,凡每日进耶和华殿、按班次供职的,也分给他;
\par 17 又按宗族家谱分给祭司,按班次职任分给二十岁以外的利未人,
\par 18 又按家谱计算,分给他们会中的妻子、儿女;因他们身供要职,自洁成圣。
\par 19 按名派定的人要把应得的分给亚伦子孙,住在各城郊野、祭司所有的男丁和一切载入家谱的利未人。
\par 20 希西家在犹大遍地这样办理,行耶和华他神眼中看为善为正为忠的事。
\par 21 凡他所行的,无论是办神殿的事,是遵律法守诫命,是寻求他的神,都是尽心去行,无不亨通。

\chapter{32}

\par 1 这虔诚的事以後,亚述王西拿基立来侵入犹大,围困一切坚固城,想要攻破占据。
\par 2 希西家见西拿基立来,定意要攻打耶路撒冷,
\par 3 就与首领和勇士商议,塞住城外的泉源;他们就都帮助他。
\par 4 於是有许多人聚集,塞了一切泉源,并通流国中的小河,说:「亚述王来,为何让他得著许多水呢?」
\par 5 希西家力图自强,就修筑所有拆毁的城墙,高与城楼相齐;在城外又筑一城,坚固大卫城的米罗,制造了许多军器、盾牌;
\par 6 设立军长管理百姓,将他们招聚在城门的宽阔处,用话勉励他们,说:
\par 7 「你们当刚强壮胆,不要因亚述王和跟随他的大军恐惧、惊慌;因为与我们同在的,比与他们同在的更大。
\par 8 与他们同在的是肉臂,与我们同在的是耶和华我们的神,他必帮助我们,为我们争战」。百姓就靠犹大王希西家的话,安然无惧了。
\par 9 此後,亚述王西拿基立和他的全军攻打拉吉,就差遣臣仆到耶路撒冷见犹大王希西家和一切在耶路撒冷的犹大人,说:
\par 10 「亚述王西拿基立如此说:『你们倚靠什麽,还在耶路撒冷受困呢?
\par 11 希西家对你们说「耶和华我们的神必救我们脱离亚述王的手」,这不是诱惑你们,使你们受饥渴而死吗?
\par 12 这希西家岂不是废去耶和华的邱坛和祭坛,吩咐犹大与耶路撒冷的人说「你们当在一个坛前敬拜,在其上烧香」吗?
\par 13 我与我列祖向列邦所行的,你们岂不知道吗?列邦的神何尝能救自己的国脱离我手呢?
\par 14 我列祖所灭的国,那些神中谁能救自己的民脱离我手呢?难道你们的神能救你们脱离我手吗?
\par 15 所以你们不要叫希西家这样欺哄诱惑你们,也不要信他;因为没有一国一邦的神能救自己的民脱离我手和我列祖的手,何况你们的神更不能救你们脱离我的手。』」
\par 16 西拿基立的臣仆还有别的话毁谤耶和华 神和他仆人希西家。
\par 17 西拿基立也写信毁谤耶和华以色列的神说:「列邦的神既不能救他的民脱离我手,希西家的神也不能救他的民脱离我手了。」
\par 18 亚述王的臣仆用犹大言语向耶路撒冷城上的民大声呼叫,要惊吓他们,扰乱他们,以便取城。
\par 19 他们论耶路撒冷的神,如同论世上人手所造的神一样。
\par 20 希西家王和亚摩斯的儿子先知以赛亚因此祷告,向天呼求。
\par 21 耶和华就差遣一个使者进入亚述王营中,把所有大能的勇士和官长、将帅尽都灭了。亚述王满面含羞地回到本国,进了他神的庙中,有他亲生的儿子在那里用刀杀了他。
\par 22 这样,耶和华救希西家和耶路撒冷的居民脱离亚述王西拿基立的手,也脱离一切仇敌的手,又赐他们四境平安。
\par 23 有许多人到耶路撒冷,将供物献与耶和华,又将宝物送给犹大王希西家。此後,希西家在列邦人的眼中看为尊大。
\par 24 那时希西家病得要死,就祷告耶和华,耶和华应允他,赐他一个兆头。
\par 25 希西家却没有照他所蒙的恩报答耶和华;因他心里骄傲,所以忿怒要临到他和犹大并耶路撒冷。
\par 26 但希西家和耶路撒冷的居民觉得心里骄傲,就一同自卑,以致耶和华的忿怒在希西家的日子没有临到他们。
\par 27 希西家大有尊荣资财,建造府库,收藏金银、宝石、香料、盾牌,和各样的宝器,
\par 28 又建造仓房,收藏五谷、新酒,和油,又为各类牲畜盖棚立圈;
\par 29 并且建立城邑,还有许多的羊群牛群,因为神赐他极多的财产。
\par 30 这希西家也塞住基训的上源,引水直下,流在大卫城的西边。希西家所行的事尽都亨通。
\par 31 惟有一件事,就是巴比伦王差遣使者来见希西家,访问国中所现的奇事;这件事神离开他,要试验他,好知道他心内如何。
\par 32 希西家其余的事和他的善行都写在亚摩斯的儿子先知以赛亚的默示书上和犹大、以色列的诸王记上。
\par 33 希西家与他列祖同睡,葬在大卫子孙的高陵上。他死的时候,犹大人和耶路撒冷的居民都尊敬他。他儿子玛拿西接续他作王。

\chapter{33}

\par 1 玛拿西登基的时候年十二岁,在耶路撒冷作王五十五年。
\par 2 他行耶和华眼中看为恶的事,效法耶和华在以色列人面前赶出的外邦人那可憎的事,
\par 3 重新建筑他父希西家所拆毁的邱坛,又为巴力筑坛,做木偶,且敬拜事奉天上的万象,
\par 4 在耶和华的殿宇中筑坛,耶和华曾指著这殿说:「我的名必永远在耶路撒冷。」
\par 5 他在耶和华殿的两院中为天上的万象筑坛,
\par 6 并在欣嫩子谷使他的儿女经火,又观兆,用法术,行邪术,立交鬼的和行巫术的,多行耶和华眼中看为恶的事,惹动他的怒气,
\par 7 又在神殿内立雕刻的偶像。神曾对大卫和他儿子所罗门说:「我在以色列各支派中所选择的耶路撒冷和这殿,必立我的名直到永远。
\par 8 以色列人若谨守遵行我藉摩西所吩咐他们的一切法度、律例、典章,我就不再使他们挪移离开我所赐给他们列祖之地。」
\par 9 玛拿西引诱犹大和耶路撒冷的居民,以致他们行恶比耶和华在以色列人面前所灭的列国更甚。
\par 10 耶和华警戒玛拿西和他的百姓,他们却是不听。
\par 11 所以耶和华使亚述王的将帅来攻击他们,用铙钩钩住玛拿西,用铜链锁住他,带到巴比伦去。
\par 12 他在急难的时候,就恳求耶和华他的神,且在他列祖的神面前极其自卑。
\par 13 他祈祷耶和华,耶和华就允准他的祈求,垂听他的祷告,使他归回耶路撒冷,仍坐国位。玛拿西这才知道惟独耶和华是神。
\par 14 此後,玛拿西在大卫城外,从谷内基训西边直到鱼门口,建筑城墙,环绕俄斐勒,这墙筑得甚高;又在犹大各坚固城内设立勇敢的军长;
\par 15 并除掉外邦人的神像与耶和华殿中的偶像,又将他在耶和华殿的山上和耶路撒冷所筑的各坛都拆毁抛在城外;
\par 16 重修耶和华的祭坛,在坛上献平安祭、感谢祭,吩咐犹大人事奉耶和华以色列的神。
\par 17 百姓却仍在邱坛上献祭,只献给耶和华他们的神。
\par 18 玛拿西其余的事和祷告他神的话,并先见奉耶和华以色列神的名警戒他的言语,都写在以色列诸王记上。
\par 19 他的祷告,与神怎样应允他,他未自卑以前的罪愆过犯,并在何处建筑邱坛,设立亚舍拉和雕刻的偶像,都写在何赛的书上。
\par 20 玛拿西与他列祖同睡,葬在自己的宫院里。他儿子亚们接续他作王。
\par 21 亚们登基的时候年二十二岁,在耶路撒冷作王二年。
\par 22 他行耶和华眼中看为恶的事,效法他父玛拿西所行的,祭祀事奉他父玛拿西所雕刻的偶像,
\par 23 不在耶和华面前像他父玛拿西自卑。这亚们所犯的罪越犯越大。
\par 24 他的臣仆背叛,在宫里杀了他。
\par 25 但国民杀了那些背叛亚们王的人,立他儿子约西亚接续他作王。

\chapter{34}

\par 1 约西亚登基的时候年八岁,在耶路撒冷作王三十一年。
\par 2 他行耶和华眼中看为正的事,效法他祖大卫所行的,不偏左右。
\par 3 他作王第八年,尚且年幼,就寻求他祖大卫的神。到了十二年才洁净犹大和耶路撒冷,除掉邱坛、木偶、雕刻的像,和铸造的像。
\par 4 众人在他面前拆毁巴力的坛,砍断坛上高高的日像,又把木偶和雕刻的像,并铸造的像打碎成灰,撒在祭偶像人的坟上,
\par 5 将他们祭司的骸骨烧在坛上,洁净了犹大和耶路撒冷;
\par 6 又在玛拿西、以法莲、西缅、拿弗他利各城,和四围破坏之处,都这样行;
\par 7 又拆毁祭坛,把木偶和雕刻的像打碎成灰,砍断以色列遍地所有的日像,就回耶路撒冷去了。
\par 8 约西亚王十八年,净地净殿之後,就差遣亚萨利雅的儿子沙番、邑宰玛西雅、约哈斯的儿子史官约亚去修理耶和华他神的殿。
\par 9 他们就去见大祭司希勒家,将奉到神殿的银子交给他;这银子是看守殿门的利未人从玛拿西、以法莲,和一切以色列剩下的人,以及犹大、便雅悯众人,并耶路撒冷的居民收来的。
\par 10 又将这银子交给耶和华殿里督工的,转交修理耶和华殿的工匠,
\par 11 就是交给木匠、石匠,买凿成的石头和架木与栋梁,修犹大王所毁坏的殿。
\par 12 这些人办事诚实,督工的是利未人米拉利的子孙雅哈、俄巴底;督催的是哥辖的子孙撒迦利亚、米书兰;还有善於作乐的利未人。
\par 13 他们又监管扛抬的人,督催一切做工的。利未人中也有作书记、作司事、作守门的。
\par 14 他们将奉到耶和华殿的银子运出来的时候,祭司希勒家偶然得了摩西所传耶和华的律法书。
\par 15 希勒家对书记沙番说:「我在耶和华殿里得了律法书。」遂将书递给沙番。
\par 16 沙番把书拿到王那里,回覆王说:「凡交给仆人们办的都办理了。
\par 17 耶和华殿里的银子倒出来,交给督工的和匠人的手里了。」
\par 18 书记沙番又对王说:「祭司希勒家递给我一卷书。」沙番就在王面前读那书。
\par 19 王听见律法上的话,就撕裂衣服,
\par 20 吩咐希勒家与沙番的儿子亚希甘、米迦的儿子亚比顿、书记沙番,和王的臣仆亚撒雅说:
\par 21 「你们去为我、为以色列和犹大剩下的人,以这书上的话求问耶和华;因我们列祖没有遵守耶和华的言语,没有照这书上所记的去行,耶和华的烈怒就倒在我们身上。」
\par 22 於是,希勒家和王所派的众人都去见女先知户勒大。户勒大是掌管礼服沙龙的妻,沙龙是哈斯拉的孙子、特瓦的儿子。户勒大住在耶路撒冷第二区;他们请问於他。
\par 23 他对他们说:「耶和华以色列的神如此说:『你们可以回覆那差遣你们来见我的人说,
\par 24 耶和华如此说:我必照著在犹大王面前所读那书上的一切咒诅,降祸与这地和其上的居民;
\par 25 因为他们离弃我,向别神烧香,用他们手所做的惹我发怒,所以我的忿怒如火倒在这地上,总不息灭。』
\par 26 然而差遣你们来求问耶和华的犹大王,你们要这样回覆他说:『耶和华以色列的神如此说:至於你所听见的话,
\par 27 就是听见我指著这地和其上居民所说的话,你便心里敬服,在我面前自卑,撕裂衣服,向我哭泣,因此我应允了你。这是我耶和华说的。
\par 28 我必使你平平安安的归到坟墓,到你列祖那里,我要降与这地和其上居民的一切灾祸,你也不至亲眼看见。』」他们就回覆王去了。
\par 29 王差遣人招聚犹大和耶路撒冷的众长老来。
\par 30 王和犹大众人,与耶路撒冷的居民,并祭司利未人,以及所有的百姓,无论大小,都一同上到耶和华的殿;王就把殿里所得的约书念给他们听。
\par 31 王站在他的地位上,在耶和华面前立约,要尽心尽性地顺从耶和华,遵守他的诫命、法度、律例,成就这书上所记的约言;
\par 32 又使住耶路撒冷和便雅悯的人都服从这约。於是耶路撒冷的居民都遵行他们列祖之神的约。
\par 33 约西亚从以色列各处将一切可憎之物尽都除掉,使以色列境内的人都事奉耶和华他们的神。约西亚在世的日子,就跟从耶和华他们列祖的神,总不离开。

\chapter{35}

\par 1 约西亚在耶路撒冷向耶和华守逾越节。正月十四日,就宰了逾越节的羊羔。
\par 2 王分派祭司各尽其职,又勉励他们办耶和华殿中的事;
\par 3 又对那归耶和华为圣、教训以色列人的利未人说:「你们将圣约柜安放在以色列王大卫儿子所罗门建造的殿里,不必再用肩扛抬。现在要事奉耶和华你们的神,服事他的民以色列。
\par 4 你们应当按著宗族,照著班次,遵以色列王大卫和他儿子所罗门所写的,自己预备。
\par 5 要按著你们的弟兄,这民宗族的班次,站在圣所,每班中要利未宗族的几个人。
\par 6 要宰逾越节的羊羔,洁净自己,为你们的弟兄预备了,好遵守耶和华藉摩西所吩咐的话。」
\par 7 约西亚从群畜中赐给在那里所有的人民,绵羊羔和山羊羔三万只,牛三千只,作逾越节的祭物;这都是出自王的产业中。
\par 8 约西亚的众首领也乐意将牺牲给百姓和祭司利未人;又有管理神殿的希勒家、撒迦利亚、耶歇将羊羔二千六百只,牛三百只,给祭司作逾越节的祭物。
\par 9 利未人的族长歌楠雅和他两个兄弟示玛雅、拿坦业,与哈沙比雅、耶利、约撒拔将羊羔五千只,牛五百只,给利未人作逾越节的祭物。
\par 10 这样,供献的事齐备了。祭司站在自己的地方,利未人按著班次站立,都是照王所吩咐的。
\par 11 利未人宰了逾越节的羊羔,祭司从他们手里接过血来洒在坛上;利未人剥皮,
\par 12 将燔祭搬来,按著宗族的班次分给众民,好照摩西书上所写的,献给耶和华;献牛也是这样。
\par 13 他们按著常例,用火烤逾越节的羊羔。别的圣物用锅,用釜,用罐煮了,速速地送给众民。
\par 14 然後为自己和祭司预备祭物;因为祭司亚伦的子孙献燔祭和脂油,直到晚上。所以利未人为自己和祭司亚伦的子孙,预备祭物。
\par 15 歌唱的亚萨之子孙,照著大卫、亚萨、希幔,和王的先见耶杜顿所吩咐的,站在自己的地位上。守门的看守各门,不用离开他们的职事,因为他们的弟兄利未人给他们预备祭物。
\par 16 当日,供奉耶和华的事齐备了,就照约西亚王的吩咐守逾越节,献燔祭在耶和华的坛上。
\par 17 当时在耶路撒冷的以色列人守逾越节,又守除酵节七日。
\par 18 自从先知撒母耳以来,在以色列中没有守过这样的逾越节,以色列诸王也没有守过,像约西亚、祭司、利未人、在那里的犹大人,和以色列人,以及耶路撒冷居民所守的逾越节。
\par 19 这逾越节是约西亚作王十八年守的。
\par 20 这事以後,约西亚修完了殿,有埃及王尼哥上来,要攻击靠近伯拉河的迦基米施;约西亚出去抵挡他。
\par 21 他差遣使者来见约西亚,说:「犹大王啊,我与你何干?我今日来不是要攻击你,乃是要攻击与我争战之家,并且神吩咐我速行,你不要干预神的事,免得他毁灭你,因为神是与我同在。」
\par 22 约西亚却不肯转去离开他,改装要与他打仗,不听从神藉尼哥之口所说的话,便来到米吉多平原争战。
\par 23 弓箭手射中约西亚王。王对他的臣仆说:「我受了重伤,你拉我出阵吧!」
\par 24 他的臣仆扶他下了战车,上了次车,送他到耶路撒冷,他就死了,葬在他列祖的坟墓里。犹大人和耶路撒冷人都为他悲哀。
\par 25 耶利米为约西亚作哀歌。所有歌唱的男女也唱哀歌,追悼约西亚,直到今日;而且在以色列中成了定例。这歌载在哀歌书上。
\par 26 约西亚其余的事和他遵著耶和华律法上所记而行的善事,
\par 27 并他自始至终所行的,都写在以色列和犹大列王记上。

\chapter{36}

\par 1 国民立约西亚的儿子约哈斯在耶路撒冷接续他父作王。
\par 2 约哈斯登基的时候年二十三岁,在耶路撒冷作王三个月。
\par 3 埃及王在耶路撒冷废了他,又罚犹大国银子一百他连得,金子一他连得。
\par 4 埃及王尼哥立约哈斯的哥哥以利雅敬作犹大和耶路撒冷的王,改名叫约雅敬,又将约哈斯带到埃及去了。
\par 5 约雅敬登基的时候年二十五岁,在耶路撒冷作王十一年,行耶和华他神眼中看为恶的事。
\par 6 巴比伦王尼布甲尼撒上来攻击他,用铜链锁著他,要将他带到巴比伦去。
\par 7 尼布甲尼撒又将耶和华殿里的器皿带到巴比伦,放在他神的庙里(或作:自己的宫里)。
\par 8 约雅敬其余的事和他所行可憎的事,并他一切的行为,都写在以色列和犹大列王记上。他儿子约雅斤接续他作王。
\par 9 约雅斤登基的时候年八岁(在列王下二十四章八节是十八岁),在耶路撒冷作王三个月零十天,行耶和华眼中看为恶的事。
\par 10 过了一年,尼布甲尼撒差遣人将约雅斤和耶和华殿里各样宝贵的器皿带到巴比伦,就立约雅斤的叔叔(原文作兄)西底家作犹大和耶路撒冷的王。
\par 11 西底家登基的时候年二十一岁,在耶路撒冷作王十一年,
\par 12 行耶和华他神眼中看为恶的事。先知耶利米以耶和华的话劝他,他仍不在耶利米面前自卑。
\par 13 尼布甲尼撒曾使他指著神起誓,他却背叛,强项硬心,不归服耶和华以色列的神。
\par 14 众祭司长和百姓也大大犯罪,效法外邦人一切可憎的事,污秽耶和华在耶路撒冷分别为圣的殿。
\par 15 耶和华他们列祖的神因为爱惜自己的民和他的居所,从早起来差遣使者去警戒他们。
\par 16 他们却嘻笑神的使者,藐视他的言语,讥诮他的先知,以致耶和华的忿怒向他的百姓发作,无法可救。
\par 17 所以,耶和华使迦勒底人的王来攻击他们,在他们圣殿里用刀杀了他们的壮丁,不怜恤他们的少男处女、老人白叟。耶和华将他们都交在迦勒底王手里。
\par 18 迦勒底王将神殿里的大小器皿与耶和华殿里的财宝,并王和众首领的财宝,都带到巴比伦去了。
\par 19 迦勒底人焚烧神的殿,拆毁耶路撒冷的城墙,用火烧了城里的宫殿,毁坏了城里宝贵的器皿。
\par 20 凡脱离刀剑的,迦勒底王都掳到巴比伦去,作他和他子孙的仆婢,直到波斯国兴起来。
\par 21 这就应验耶和华藉耶利米口所说的话:地享受安息;因为地土荒凉便守安息,直满了七十年。
\par 22 波斯王古列元年,耶和华为要应验藉耶利米口所说的话,就激动波斯王古列的心,使他下诏通告全国,说:
\par 23 「波斯王古列如此说:耶和华天上的神已将天下万国赐给我,又嘱咐我在犹大的耶路撒冷为他建造殿宇。你们中间凡作他子民的,可以上去,愿耶和华他的神与他同在。」


\end{document}