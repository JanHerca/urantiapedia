\begin{document}

\title{彼得前书}


\chapter{1}

\par 1 耶稣基督的使徒彼得写信给那分散在本都、加拉太、加帕多家、亚西亚、庇推尼寄居的,
\par 2 就是照父神的先见被拣选、藉著圣灵得成圣洁,以致顺服耶稣基督,又蒙他血所洒的人。愿恩惠、平安多多的加给你们。
\par 3 愿颂赞归与我们主耶稣基督的父神!他曾照自己的大怜悯,藉耶稣基督从死里复活,重生了我们,叫我们有活泼的盼望,
\par 4 可以得著不能朽坏、不能玷污、不能衰残、为你们存留在天上的基业。
\par 5 你们这因信蒙神能力保守的人,必能得著所预备,到末世要显现的救恩。
\par 6 因此,你们是大有喜乐;但如今,在百般的试炼中暂时忧愁,
\par 7 叫你们的信心既被试验,就比那被火试验仍然能坏的金子更显宝贵,可以在耶稣基督显现的时候得著称赞、荣耀、尊贵。
\par 8 你们虽然没有见过他,却是爱他;如今虽不得看见,却因信他就有说不出来、满有荣光的大喜乐;
\par 9 并且得著你们信心的果效,就是灵魂的救恩。
\par 10 论到这救恩,那预先说你们要得恩典的众先知早已详细的寻求考察,
\par 11 就是考察在他们心里基督的灵,预先证明基督受苦难,後来得荣耀,是指著什麽时候,并怎样的时候。
\par 12 他们得了启示,知道他们所传讲(原文作服事)的一切事,不是为自己,乃是为你们。那靠著从天上差来的圣灵传福音给你们的人,现在将这些事报给你们;天使也愿意详细察看这些事。
\par 13 所以要约束你们的心,(原文作束上你们心中的腰)谨慎自守,专心盼望耶稣基督显现的时候所带来给你们的恩。
\par 14 你们既作顺命的儿女,就不要效法从前蒙昧无知的时候那放纵私欲的样子。
\par 15 那召你们的既是圣洁,你们在一切所行的事上也要圣洁。
\par 16 因为经上记著说:「你们要圣洁,因为我是圣洁的。」
\par 17 你们既称那不偏待人,按各人行为审判人的主为父,就当存敬畏的心度你们在世寄居的日子,
\par 18 知道你们得赎,脱去你们祖宗所传流虚妄的行为,不是凭著能坏的金银等物,
\par 19 乃是凭著基督的宝血,如同无瑕疵、无玷污的羔羊之血。
\par 20 基督在创世以前是预先被神知道的,却在这末世才为你们显现。
\par 21 你们也因著他,信那叫他从死里复活、又给他荣耀的神,叫你们的信心和盼望都在於神。
\par 22 你们既因顺从真理,洁净了自己的心,以致爱弟兄没有虚假,就当从心里(从心里:有古卷是从清洁的心)彼此切实相爱。
\par 23 你们蒙了重生,不是由於能坏的种子,乃是由於不能坏的种子,是藉著神活泼常存的道。
\par 24 因为凡有血气的,尽都如草;他的美荣都像草上的花。草必枯乾,花必凋谢;
\par 25 惟有主的道是永存的。所传给你们的福音就是这道。

\chapter{2}

\par 1 所以,你们既除去一切的恶毒、(或作阴毒)诡诈,并假善、嫉妒,和一切毁谤的话,
\par 2 就要爱慕那纯净的灵奶,像才生的婴孩爱慕奶一样,叫你们因此渐长,以致得救。
\par 3 你们若尝过主恩的滋味,就必如此。
\par 4 主乃活石,固然是被人所弃的,却是被神所拣选、所宝贵的。
\par 5 你们来到主面前,也就像活石,被建造成为灵宫,作圣洁的祭司,藉著耶稣基督奉献神所悦纳的灵祭。
\par 6 因为经上说:看哪,我把所拣选、所宝贵的房角石安放在锡安;信靠他的人必不至於羞愧。
\par 7 所以,他在你们信的人就为宝贵,在那不信的人有话说:匠人所弃的石头已作了房角的头块石头。
\par 8 又说:作了绊脚的石头,跌人的磐石。他们既不顺从,就在道理上绊跌;(或作:他们绊跌都因不顺从道理)他们这样绊跌也是预定的。
\par 9 惟有你们是被拣选的族类,是有君尊的祭司,是圣洁的国度,是属神的子民,要叫你们宣扬那召你们出黑暗入奇妙光明者的美德。
\par 10 你们从前算不得子民,现在却作了神的子民;从前未曾蒙怜恤,现在却蒙了怜恤。
\par 11 亲爱的弟兄啊,你们是客旅,是寄居的。我劝你们要禁戒肉体的私欲;这私欲是与灵魂争战的。
\par 12 你们在外邦人中,应当品行端正,叫那些毁谤你们是作恶的,因看见你们的好行为,便在鉴察(或作:眷顾)的日子归荣耀给神。
\par 13 你们为主的缘故,要顺服人的一切制度,或是在上的君王,
\par 14 或是君王所派罚恶赏善的臣宰。
\par 15 因为神的旨意原是要你们行善,可以堵住那糊涂无知人的口。
\par 16 你们虽是自由的,却不可藉著自由遮盖恶毒,(或作:阴毒)总要作神的仆人。
\par 17 务要尊敬众人,亲爱教中的弟兄,敬畏神,尊敬君王。
\par 18 你们作仆人的,凡事要存敬畏的心顺服主人;不但顺服那善良温和的,就是那乖僻的也要顺服。
\par 19 倘若人为叫良心对得住神,就忍受冤屈的苦楚,这是可喜爱的。
\par 20 你们若因犯罪受责打,能忍耐,有什麽可夸的呢?但你们若因行善受苦,能忍耐,这在神看是可喜爱的。
\par 21 你们蒙召原是为此;因基督也为你们受过苦,给你们留下榜样,叫你们跟随他的脚踪行。
\par 22 他并没有犯罪,口里也没有诡诈。
\par 23 他被骂不还口;受害不说威吓的话,只将自己交托那按公义审判人的主。
\par 24 他被挂在木头上,亲身担当了我们的罪,使我们既然在罪上死,就得以在义上活。因他受的鞭伤,你们便得了医治。
\par 25 你们从前好像迷路的羊,如今却归到你们灵魂的牧人监督了。

\chapter{3}

\par 1 你们作妻子的要顺服自己的丈夫;这样,若有不信从道理的丈夫,他们虽然不听道,也可以因妻子的品行被感化过来;
\par 2 这正是因看见你们有贞洁的品行和敬畏的心。
\par 3 你们不要以外面的辫头发,戴金饰,穿美衣为妆饰,
\par 4 只要以里面存著长久温柔,安静的心为妆饰;这在神面前是极宝贵的。
\par 5 因为古时仰赖神的圣洁妇人正是以此为妆饰,顺服自己的丈夫,
\par 6 就如撒拉听从亚伯拉罕,称他为主。你们若行善,不因恐吓而害怕,便是撒拉的女儿了。
\par 7 你们作丈夫的,也要按情理 原文作知识 和妻子同住;因他比你软弱,(比你软弱:原文作是软弱的器皿)与你一同承受生命之恩的,所以要敬重他。这样,便叫你们的祷告没有阻碍。
\par 8 总而言之,你们都要同心,彼此体恤,相爱如弟兄,存慈怜谦卑的心。
\par 9 不以恶报恶,以辱骂还辱骂,倒要祝福;因你们是为此蒙召,好叫你们承受福气。
\par 10 因为经上说:人若爱生命,愿享美福,须要禁止舌头不出恶言,嘴唇不说诡诈的话;
\par 11 也要离恶行善;寻求和睦,一心追赶。
\par 12 因为,主的眼看顾义人;主的耳听他们的祈祷。惟有行恶的人,主向他们变脸。
\par 13 你们若是热心行善,有谁害你们呢?
\par 14 你们就是为义受苦,也是有福的。不要怕人的威吓,也不要惊慌;(的威吓:或作所怕的)
\par 15 只要心里尊主基督为圣。有人问你们心中盼望的缘由,就要常作准备,以温柔、敬畏的心回答各人;
\par 16 存著无亏的良心,叫你们在何事上被毁谤,就在何事上可以叫那诬赖你们在基督里有好品行的人自觉羞愧。
\par 17 神的旨意若是叫你们因行善受苦,总强如因行恶受苦。
\par 18 因基督也曾一次为罪受苦(有古卷:受死),就是义的代替不义的,为要引我们到神面前。按著肉体说,他被治死;按著灵性说,他复活了。
\par 19 他藉这灵曾去传道给那些在监狱里的灵听,
\par 20 就是那从前在挪亚预备方舟、神容忍等待的时候,不信从的人。当时进入方舟,藉著水得救的不多,只有八个人。
\par 21 这水所表明的洗礼,现在藉著耶稣基督复活也拯救你们;这洗礼本不在乎除掉肉体的污秽,只求在神面前有无亏的良心。
\par 22 耶稣已经进入天堂,在神的右边;众天使和有权柄的,并有能力的,都服从了他。

\chapter{4}

\par 1 基督既在肉身受苦,你们也当将这样的心志作为兵器,因为在肉身受过苦的,就已经与罪断绝了。
\par 2 你们存这样的心,从今以後就可以不从人的情欲,只从神的旨意在世度余下的光阴。
\par 3 因为往日随从外邦人的心意行邪淫、恶欲、醉酒、荒宴、群饮,并可恶拜偶像的事,时候已经够了。
\par 4 他们在这些事上,见你们不与他们同奔那放荡无度的路,就以为怪,毁谤你们。
\par 5 他们必在那将要审判活人死人的主面前交账。
\par 6 为此,就是死人也曾有福音传给他们,要叫他们的肉体按著人受审判,他们的灵性却靠神活著。
\par 7 万物的结局近了。所以,你们要谨慎自守,警醒祷告。
\par 8 最要紧的是彼此切实相爱,因为爱能遮掩许多的罪。
\par 9 你们要互相款待,不发怨言。
\par 10 各人要照所得的恩赐彼此服事,作神百般恩赐的好管家。
\par 11 若有讲道的,要按著神的圣言讲;若有服事人的,要按著神所赐的力量服事,叫神在凡事上因耶稣基督得荣耀。原来荣耀、权能都是他的,直到永永远远。阿们!
\par 12 亲爱的弟兄啊,有火炼的试验临到你们,不要以为奇怪(似乎是遭遇非常的事)
\par 13 倒要欢喜;因为你们是与基督一同受苦,使你们在他荣耀显现的时候,也可以欢喜快乐。
\par 14 你们若为基督的名受辱骂,便是有福的;因为神荣耀的灵常住在你们身上。
\par 15 你们中间却不可有人因为杀人、偷窃、作恶、好管闲事而受苦。
\par 16 若为作基督徒受苦,却不要羞耻,倒要因这名归荣耀给神。
\par 17 因为时候到了,审判要从神的家起首。若是先从我们起首,那不信从神福音的人将有何等的结局呢?
\par 18 若是义人仅仅得救,那不虔敬和犯罪的人将有何地可站呢?
\par 19 所以那照神旨意受苦的人要一心为善,将自己灵魂交与那信实的造化之主。

\chapter{5}

\par 1 我这作长老、作基督受苦的见证、同享後来所要显现之荣耀的,劝你们中间与我同作长老的人:
\par 2 务要牧养在你们中间神的群羊,按著神旨意照管他们;不是出於勉强,乃是出於甘心;也不是因为贪财,乃是出於乐意;
\par 3 也不是辖制所托付你们的,乃是作群羊的榜样。
\par 4 到了牧长显现的时候,你们必得那永不衰残的荣耀冠冕。
\par 5 你们年幼的,也要顺服年长的。就是你们众人也都要以谦卑束腰,彼此顺服;因为神阻挡骄傲的人,赐恩给谦卑的人。
\par 6 所以,你们要自卑,服在神大能的手下,到了时候他必叫你们升高。
\par 7 你们要将一切的忧虑卸给神,因为他顾念你们。
\par 8 务要谨守,警醒。因为你们的仇敌魔鬼,如同吼叫的狮子,遍地游行,寻找可吞吃的人。
\par 9 你们要用坚固的信心抵挡他,因为知道你们在世上的众弟兄也是经历这样的苦难。
\par 10 那赐诸般恩典的神曾在基督里召你们,得享他永远的荣耀,等你们暂受苦难之後,必要亲自成全你们,坚固你们,赐力量给你们。
\par 11 愿权能归给他,直到永永远远。阿们!
\par 12 我略略的写了这信,托我所看为忠心的兄弟西拉转交你们,劝勉你们,又证明这恩是神的真恩。你们务要在这恩上站立得住。
\par 13 在巴比伦与你们同蒙拣选的教会问你们安。我儿子马可也问你们安。
\par 14 你们要用爱心彼此亲嘴问安。愿平安归与你们凡在基督里的人!


\end{document}