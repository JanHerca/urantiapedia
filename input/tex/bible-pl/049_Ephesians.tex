\begin{document}

\title{List do Efezjan}


\chapter{1}

\par 1 Pawel, Apostol Jezusa Chrystusa przez wole Boza, swietym, którzy sa w Efezie, i wiernym w Chrystusie Jezusie.
\par 2 Laska wam i pokój niech bedzie od Boga, Ojca naszego, i Pana Jezusa Chrystusa.
\par 3 Blogoslawiony niech bedzie Bóg i Ojciec Pana naszego Jezusa Chrystusa, który nas ublogoslawil wszelkiem blogoslawienstwem duchownem w niebieskich rzeczach w Chrystusie;
\par 4 Jako nas wybral w nim przed zalozeniem swiata, abysmy byli swietymi i nienaganionymi przed oblicznoscia jego w milosci.
\par 5 Który nas przenaznaczyl ku przysposobieniu za synów przez Jezusa Chrystusa dla siebie samego, wedlug upodobania woli swojej,
\par 6 Ku chwale slawnej laski swojej, która nas udarowal w onym umilowanym:
\par 7 W którym mamy odkupienie przez krew jego, to jest odpuszczenie grzechów, wedlug bogactwa laski jego,
\par 8 Która hojnie pokazal przeciwko nam we wszelkiej madrosci i roztropnosci.
\par 9 Oznajmiwszy nam tajemnice woli swojej wedlug upodobania swego, które byl postanowil w samym sobie;
\par 10 Aby w rozrzadzeniu zupelnosci czasów w jedno zgromadzil wszystkie rzeczy w Chrystusie, i te, które sa na niebiesiech, i te, które sa na ziemi;
\par 11 W nim, mówie, w którymesmy i do dzialu przypuszczeni, przenaznaczeni bedac wedlug postanowienia tego, który sprawuje wszystko wedlug rady woli swojej;
\par 12 Abysmy my byli ku chwale slawy jego, którzysmy pierwej nadzieje mieli w Chrystusie,
\par 13 W którym i wy nadzieje macie, uslyszawszy slowo prawdy, to jest Ewangielije zbawienia waszego, przez która tez uwierzywszy, jestescie zapieczetowani Duchem onym Swietym obiecanym,
\par 14 Który jest zadatkiem dziedzictwa naszego na wykupienie nabytej wlasnosci, ku chwale slawy jego.
\par 15 Przetoz i ja uslyszawszy o tej wierze, która macie w Panu Jezusie, i o milosci ku wszystkim swietym,
\par 16 Nie przestaje dziekowac za was, wzmianke czyniac o was w modlitwach moich,
\par 17 Aby Bóg Pana naszego Jezusa Chrystusa, Ojciec on chwaly, dal wam Ducha madrosci i objawienia ku poznaniu samego siebie;
\par 18 Azeby oswiecil oczy mysli waszej, abyscie wiedzieli, która jest nadzieja powolania jego i które jest bogactwo chwaly dziedzictwa jego w swietych;
\par 19 I która jest ona przewyzszajaca wielkosc mocy jego przeciwko nam, którzy wierzymy wedlug skutecznej mocy sily jego,
\par 20 Której dokazal w Chrystusie, gdy go wzbudzil od umarlych i posadzil na prawicy swojej na niebiesiech,
\par 21 Wysoko nad wszystkie ksiestwa i zwierzchnosci, i mocy, i panstwa, i nad wszelkie imie, które sie mianuje, nie tylko w tym wieku, ale i w przyszlym;
\par 22 I wszystko poddal pod nogi jego, a onego dal za glowe nad wszystkim kosciolowi,
\par 23 Który jest cialem jego i pelnoscia tego, który wszystko we wszystkich napelnia.

\chapter{2}

\par 1 I was ozywil, którzyscie byli umarli w upadkach i w grzechach.
\par 2 W którychescie niekiedy chodzili wedlug zwyczaju swiata tego i wedlug ksiazecia, który ma wladze na powietrzu, ducha tego, który teraz jest skuteczny w synach niedowiarstwa.
\par 3 Miedzy którymi i my wszyscy obcowalismy niekiedy w pozadliwosciach ciala naszego, czyniac to, co sie podobalo cialu i myslom, i bylismy z przyrodzenia dziecmi gniewu, jako i drudzy.
\par 4 Lecz Bóg, który jest bogaty w milosierdziu, dla wielkiej milosci swojej, która nas umilowal.
\par 5 I gdysmy byli umarlymi w grzechach, ozywil nas pospolu z Chrystusem, (gdyz laska zbawieni jestescie)
\par 6 I pospolu z nim wzbudzil, i pospolu z nim posadzil na niebiesiech w Chrystusie Jezusie,
\par 7 Aby okazal w przyszlych wiekach ono nader obfite bogactwo laski swojej, z dobrotliwosci swojej przeciwko nam w Chrystusie Jezusie.
\par 8 Albowiem laska jestescie zbawieni przez wiare, i to nie jest z was, dar to Bozy jest;
\par 9 Nie z uczynków, aby sie kto nie chlubil.
\par 10 Albowiem czynem jego jestesmy stworzeni w Chrystusie Jezusie ku uczynkom dobrym, które przedtem Bóg zgotowal, abysmy w nich chodzili.
\par 11 Przetoz pamietajcie, ze wy niekiedy bedac poganami w ciele, którzyscie byli zwani nieobrzezka od onych, których zwano obrzezka w ciele, która sie reka dzieje;
\par 12 Izescie byli naonczas bez Chrystusa, oddaleni od spolecznosci Izraelskiej i obcymi od umów obietnicy, nadziei nie majacy i bez Boga na swiecie.
\par 13 Ale teraz w Chrystusie Jezusie wy, którzyscie niekiedy byli dalekimi, staliscie sie bliskimi przez krew Chrystusowa,
\par 14 Albowiem on jest pokojem naszym, który oboje jednem uczynil i srednia sciane, która byla przegroda, rozwalil;
\par 15 Nieprzyjazn, to jest zakon przykazan, który zalezal w ustawach, skaziwszy przez cialo swoje, aby dwóch stworzyl w samym sobie w jednego nowego czlowieka, czyniac pokój;
\par 16 I pojednal obydwóch w jednem ciele z Bogiem przez krzyz, zgladziwszy nieprzyjazn przezen.
\par 17 I przyszedlszy opowiedzial pokój wam, którzyscie dalekimi i którzyscie bliskimi.
\par 18 Albowiem przezen mamy przystep obie strony w jednym Duchu do Ojca.
\par 19 A przetoz juz wiecej nie jestescie goscmi i przychodniami, ale spólmieszczaninami swietych i domownikami Bozymi.
\par 20 Zbudowani na fundamencie Apostolów i proroków, którego jest gruntownym wegielnym kamieniem sam Jezus Chrystus,
\par 21 Na którym wszystko budowanie wespól spojone rosnie w kosciól swiety w Panu;
\par 22 Na którym tez i wy sie wespól budujecie, abyscie byli mieszkaniem Bozem w Duchu Swietym.

\chapter{3}

\par 1 Dlatego ja Pawel jestem wiezniem Chrystusa Jezusa za was pogan;
\par 2 Jezliscie tylko slyszeli o udzieleniu laski Bozej, która mi jest dana dla was,
\par 3 Iz mi Bóg przez objawienie oznajmil tajemnice, (jakom wam przedtem krótko napisal,
\par 4 Skad czytajac mozecie obaczyc wiadomosc moje w tajemnicy Chrystusowej).
\par 5 Która inszych wieków nie byla znajoma synom ludzkim, jako teraz objawiona jest swietym Apostolom jego i prorokom przez Ducha;
\par 6 To jest, iz poganie sa spóldziedzicami i spólnem cialem, i spóluczestnikami obietnicy jego w Chrystusie przez Ewangielije.
\par 7 Której stalem sie sluga wedlug daru laski Bozej, która mi jest dana wedlug skutku mocy jego.
\par 8 Mnie mówie, najmniejszemu ze wszystkich swietych dana jest ta laska, abym miedzy poganami opowiadal te niedoscigle bogactwa Chrystusowe.
\par 9 A izbym objasnil wszystkim, jaka by byla spolecznosc onej tajemnicy zakrytej od wieków w Bogu, który wszystko stworzyl przez Jezusa Chrystusa;
\par 10 Aby teraz przez zbór wiadoma byla ksiestwom i mocom na niebiesiech nader rozliczna madrosc Boza.
\par 11 Wedlug postanowienia wiecznego, które uczynil w Chrystusie Jezusie, Panu naszym,
\par 12 W którym mamy bezpiecznosc i przystep z ufnoscia przez wiare jego,
\par 13 Przetoz prosze, abyscie nie slabieli dla ucisków moich za was, co jest chwala wasza.
\par 14 Dlatego sklaniam kolana swoje przed Ojcem Pana naszego Jezusa Chrystusa,
\par 15 Z którego sie wszelka rodzina na niebie i na ziemi nazywa;
\par 16 Aby wam dal wedlug bogactwa chwaly swej, zebyscie byli moca utwierdzeni przez Ducha jego w wewnetrznym czlowieku;
\par 17 Aby Chrystus przez wiare mieszkal w sercach waszych;
\par 18 Zebyscie w milosci wkorzenieni i ugruntowani bedac, mogli doscignac ze wszystkimi swietymi, która jest szerokosc i dlugosc, i glebokosc, i wysokosc;
\par 19 I poznac milosc Chrystusowa przewyzszajaca wszelka znajomosc, abyscie napelnieni byli wszelaka zupelnoscia Boza.
\par 20 A temu, który moze nade wszystko uczynic daleko obficiej nizeli prosimy albo myslimy, wedlug onej mocy, która skuteczna jest w nas;
\par 21 Temu niech bedzie chwala w kosciele przez Chrystusa Jezusa po wszystkie czasy na wieki wieków. Amen.

\chapter{4}

\par 1 Prosze was tedy ja wiezien w Panu, abyscie chodzili tak, jako przystoi na powolanie, którem jestescie powolani;
\par 2 Ze wszelaka pokora i cichoscia, i z nieskwapliwoscia, znoszac jedni drugich w milosci,
\par 3 Starajac sie, abyscie zachowali jednosc ducha w zwiazce pokoju.
\par 4 Jedno jest cialo i jeden duch, jako tez jestescie powolani w jednej nadziei powolania waszego.
\par 5 Jeden Pan, jedna wiara, jeden chrzest;
\par 6 Jeden Bóg i Ojciec wszystkich, który jest nade wszystko i po wszystkich, i we wszystkich was.
\par 7 Lecz kazdemu z nas dana jest laska wedlug miary daru Chrystusowego.
\par 8 Dlatego Pismo mówi: Wstapiwszy na wysokosc, wiódl pojmanych wiezni i dal dary ludziom.
\par 9 Ale to, ze wstapil, cóz jest, jedno iz pierwej byl zstapil do najnizszych stron ziemi?
\par 10 A który zstapil, ten jest, który i wstapil wysoko nad wszystkie niebiosa, aby napelnil wszystko.
\par 11 I tenze dal niektóre Apostoly, a niektóre proroki, a drugie ewangielisty, drugie tez pasterze i nauczyciele.
\par 12 Ku spojeniu swietych, ku pracy uslugiwania, ku budowaniu ciala Chrystusowego;
\par 13 A izbysmy sie wszyscy zeszli w jednosc wiary i znajomosci Syna Bozego, w meza doskonalego, w miare zupelnego wieku Chrystusowego,
\par 14 Abysmy wiecej nie byli dziecmi, chwiejacymi sie i unoszacymi sie kazdym wiatrem nauki przez fortel ludzki i przez chytrosc podejscia bledem.
\par 15 Ale szczerymi bedac w milosci, rosnijmy w onego we wszystkiem, który jest glowa, to jest w Chrystusa,
\par 16 Z którego wszystko cialo przystojnie zlozone i spojone we wszystkich stawach, przez które jeden czlonek drugiemu posilku dodaje przez moc, która jest w kazdym czlonku, wedlug miary jego, wzrost cialu nalezacy bierze ku budowaniu samego siebie w milosci.
\par 17 To tedy mówie i oswiadczam sie przez Pana, abyscie juz wiecej nie postepowali, jako insi poganie postepuja, w próznosci umyslu swego;
\par 18 Zacmiony majac rozsadek, bedac oddaleni od zywota Bozego dla nieumiejetnosci, która w nich jest z zatwardzenia serca ich.
\par 19 Którzy zakamieniawszy, udali sie na rozpuste ku popelnianiu wszelakiej nieczystosci z chciwoscia.
\par 20 Lecz wy nie takescie sie nauczyli Chrystusa,
\par 21 Jezliscie go tylko sluchali i o nim wyuczeni byli, jako jest (ta) prawda w Jezusie,
\par 22 To jest, zebyscie zlozyli wedlug pierwszego obcowania starego czlowieka, który sie psuje przez pozadliwosci oszukiwajace;
\par 23 I odnowili sie duchem umyslu waszego;
\par 24 I oblekli sie w onego nowego czlowieka, który wedlug Boga stworzony jest w sprawiedliwosci i w swietobliwosci prawdy.
\par 25 Przetoz zlozywszy klamstwo mówcie prawde, kazdy z bliznim swoim; boscie czlonkami jedni drugich.
\par 26 Gniewajcie sie, a nie grzeszcie; slonce niech nie zachodzi na rozgniewanie wasze.
\par 27 Nie dawajcie miejsca dyjablu.
\par 28 Kto kradl, niech wiecej nie kradnie, ale raczej niech pracuje, robiac rekoma, co jest dobrego, aby mial skad udzielac potrzebujacemu.
\par 29 Zadna mowa plugawa niech z ust waszych nie pochodzi; ale jezli która jest dobra ku potrzebnemu zbudowaniu, aby byla przyjemna sluchajacym.
\par 30 A nie zasmucajcie Ducha Swietego Bozego, którym zapieczetowani jestescie na dzien odkupienia.
\par 31 Wszelka gorzkosc i zapalczywosc, i gniew, i wrzask, i bluznierstwo, niech bedzie odjete od was, ze wszelaka zloscia;
\par 32 A badzcie jedni przeciwko drugim dobrotliwi, milosierni, odpuszczajac sobie, jako wam Bóg w Chrystusie odpuscil.

\chapter{5}

\par 1 Badzciez tedy nasladowcami Bozymi, jako dzieci mile;
\par 2 A chodzcie w milosci, jako i Chrystus umilowal nas i wydal samego siebie na ofiare i na zabicie Bogu ku wdziecznej wonnosci.
\par 3 A wszeteczenstwo i wszelka nieczystosc albo lakomstwo niechaj nie bedzie ani mianowane miedzy wami, jako przystoi na swietych.
\par 4 Takze sprosnosc i blazenskie mowy, i zarty, które nie przystoja, ale raczej dziekowanie.
\par 5 Bo to wiecie, iz zaden wszetecznik, albo nieczysty, albo lakomca, (który jest balwochwalca), nie ma dziedzictwa w królestwie Chrystusowem i Bozem.
\par 6 Niechaj was nikt nie zwodzi próznemi mowami; albowiem dla tych rzeczy przychodzi gniew Bozy na synów upornych;
\par 7 Nie badzciez tedy uczestnikami ich.
\par 8 Albowiemescie byli niekiedy ciemnoscia; alescie teraz swiatloscia w Panu; chodzciez jako dziatki swiatlosci,
\par 9 (Bo owoc Ducha zalezy we wszelakiej dobrotliwosci i w sprawiedliwosci i w prawdzie.)
\par 10 Obierajac to, co by sie podobalo Panu;
\par 11 A nie spólkujcie z uczynkami niepozytecznemi ciemnosci, ale je raczej strofujcie.
\par 12 Albowiem co sie potajemnie od nich dzieje, sromota i mówic.
\par 13 Lecz to wszystko, gdy bywa od swiatlosci strofowane, bywa objawione; albowiem to wszystko, co bywa objawione, jest swiatloscia;
\par 14 Dlatego mówi Pismo: Ocuc sie, który spisz i powstan od umarlych, a oswieci cie Chrystus.
\par 15 Patrzajcie tedy, jakobyscie ostroznie chodzili, nie jako niemadrzy, ale jako madrzy.
\par 16 Czas odkupujac; bo dni zle sa.
\par 17 Przetoz nie badzcie nierozumnymi, ale zrozumiewajacymi, która jest wola Panska.
\par 18 A nie upijajcie sie winem, w którem jest rozpusta; ale badzcie napelnieni duchem,
\par 19 Rozmawiajac z soba przez psalmy i hymny, i piesni duchowne, spiewajac i grajac w sercu swojem Panu,
\par 20 Dzieki czyniac zawsze za wszystko, w imieniu Pana naszego, Jezusa Chrystusa, Bogu i Ojcu.
\par 21 Bedac poddani jedni drugim w bojazni Bozej.
\par 22 Zony! badzcie poddane mezom swoim, jako Panu;
\par 23 Albowiem maz jest glowa zony, jako i Chrystus glowa kosciola; a on jest zbawicielem ciala.
\par 24 Jako tedy kosciól poddany jest Chrystusowi, tak tez zony mezom swoim we wszystkiem.
\par 25 Mezowie! milujcie zony wasze, jako i Chrystus umilowal kosciól i wydal samego siebie za niego,
\par 26 Aby go poswiecil, oczysciwszy omyciem wody przez slowo;
\par 27 Aby go sobie wystawil chwalebnym kosciolem, nie majacym zmazy albo zmarszczku, albo czego takiego, ale izby byl swiety i bez nagany.
\par 28 Tak powinni mezowie milowac zony swoje, jako swoje wlasne ciala; kto miluje zone swoje, samego siebie miluje.
\par 29 Albowiem zaden nigdy ciala swego nie mial w nienawisci, ale je zywi i ogrzewa, jako i Pan kosciól.
\par 30 Gdyzesmy czlonkami ciala jego, z ciala jego i z kosci jego.
\par 31 Dlatego opusci czlowiek ojca swego i matke, i przylaczy sie do zony swojej, i beda dwoje jednem cialem.
\par 32 Tajemnica to wielka jest; lecz ja mówie o Chrystusie i o kosciele.
\par 33 A wszakze i kazdy z was z osobna niechaj miluje zone swoje jako siebie samego, a zona niech sie boi meza swego.

\chapter{6}

\par 1 Dziatki! badzcie posluszne rodzicom waszym w Panu; boc to jest sprawiedliwa.
\par 2 Czcij ojca twego i matke (toc jest pierwsze przykazanie z obietnica).
\par 3 Aby ci sie dobrze dzialo i abys dlugo zyl na ziemi.
\par 4 A wy ojcowie! nie pobudzajcie do gniewu dziatek waszych, ale je wychowujcie w karnosci i w napominaniu Panskiem.
\par 5 Sludzy! posluszni badzcie panom wedlug ciala, z bojaznia i ze drzeniem w prostosci serca waszego, jako Chrystusowi;
\par 6 Nie na oko sluzac, jako ci, którzy sie ludziom podobac chca, ale jako sludzy Chrystusowi, czyniac z duszy wole Boza.
\par 7 Z dobra wola sluzac jako Panu a nie ludziom;
\par 8 Wiedzac, iz kazdy, co by uczynil dobrego, za to odniesie nagrode od Pana, choc niewolnik, choc wolny.
\par 9 A wy panowie! takze sie zachowujcie przeciwko nim, odpuszczajac grozby, wiedzac, ze i wy sami macie Pana w niebiesiech, a wzgledu na osoby u niego nie masz.
\par 10 Na ostatek, bracia moi! zmacniajcie sie w Panu i w sile mocy jego;
\par 11 Obleczcie sie w zupelna zbroje Boza, abyscie mogli stac przeciwko zasadzkom dyjabelskim.
\par 12 Albowiem nie mamy boju przeciwko krwi i cialu, ale przeciwko ksiestwom, przeciwko zwierzchnosciom, przeciwko dzierzawcom swiata ciemnosci wieku tego, przeciwko duchownym zlosciom, które sa wysoko.
\par 13 A przetoz wezmijcie zupelna zbroje Boza, abyscie mogli dac odpór w dzien zly, a wszystko wykonawszy, ostac sie.
\par 14 Stójciez tedy, przepasawszy biodra wasze prawda i obleklszy pancerz sprawiedliwosci.
\par 15 I obuwszy nogi w gotowosc Ewangielii pokoju.
\par 16 A nade wszystko wziawszy tarcze wiary, która byscie mogli wszystkie strzaly ogniste onego zlosnika zagasic.
\par 17 Przylbice tez zbawienia wezmijcie i miecz Ducha, który jest slowo Boze!
\par 18 W kazdej modlitwie i w prosbie modlac sie na kazdy czas w duchu, i okolo tego czujac ze wszelka ustawicznoscia i z prosba za wszystkich swietych,
\par 19 I za mie, aby mi byla dana mowa ku otworzeniu ust moich z bezpieczenstwem, abym oznajmial tajemnice Ewangielii,
\par 20 Dla której poselstwo sprawuje w lancuchu, abym w nim bezpiecznie mówil, jako mi mówic potrzeba.
\par 21 A izbyscie wiedzieli i wy, co sie ze mna dzieje i co czynie, wszystko wam oznajmi Tychykus, brat mily i wierny sluga w Panu,
\par 22 Któregom poslal do was dla tego samego, abyscie wiedzieli, co sie z nami dzieje i aby pocieszyl serca wasze.
\par 23 Pokój niech bedzie braciom i milosc z wiara od Boga Ojca i Pana Jezusa Chrystusa.
\par 24 Laska niech bedzie ze wszystkimi milujacymi Pana naszego, Jezusa Chrystusa ku nieskazitelnosci. Amen.


\end{document}