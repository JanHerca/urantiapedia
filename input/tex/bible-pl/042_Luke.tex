\begin{document}

\title{Łukasza}


\chapter{1}

\par 1 Poniewaz wiele sie ich podjelo sporzadzic historyje o tych sprawach, o których my pewna wiadomosc mamy;
\par 2 Tak jako nam podali ci, którzy od poczatku sami widzieli, i slugami tego slowa byli;
\par 3 Zdalo sie tez i mnie, którym tego wszystkiego z poczatku pilnie doszedl, tobie to porzadnie wypisac, zacny Teofilu!
\par 4 Abys poznal pewnosc tych rzeczy, których cie nauczono.
\par 5 Byl za dni Heroda, króla Judzkiego, kaplan niektóry, imieniem Zacharyjasz, z przemiany Abijaszowej, a zona jego byla z córek Aaronowych, której imie bylo Elzbieta.
\par 6 A byli oboje sprawiedliwymi przed obliczem Bozem, chodzac we wszystkich przykazaniach i usprawiedliwieniach Panskich bez nagany.
\par 7 I nie mieli potomstwa, przeto iz Elzbieta byla nieplodna, a byli oboje podeszlymi w latach swoich.
\par 8 Stalo sie tedy, gdy odprawial urzad kaplanski w porzadku przemiany swojej przed Bogiem.
\par 9 Ze wedlug zwyczaju urzedu kaplanskiego przypadl nan los, aby kadzil, wszedlszy do kosciola Panskiego.
\par 10 A wszystko mnóstwo ludu bylo na dworze, modlac sie w godzine kadzenia.
\par 11 Tedy mu sie pokazal Aniol Panski, stojacy po prawej stronie oltarza, na którym kadzono.
\par 12 I zatrwozyl sie Zacharyjasz ujrzawszy go, a bojazn przypadla nan.
\par 13 I rzekl do niego Aniol: Nie bój sie, Zacharyjaszu! boc jest wysluchana modlitwa twoja, a Elzbieta, zona twoja, urodzi ci syna, i nazwiesz imie jego Jan,
\par 14 Z którego bedziesz mial radosc i wesele, i wiele ich radowac sie beda z narodzenia jego.
\par 15 Albowiem bedzie wielkim przed obliczem Panskiem; wina i napoju mocnego nie bedzie pil, a Duchem Swietym bedzie napelniony zaraz z zywota matki swojej.
\par 16 A wielu z synów Izraelskich obróci ku Panu, Bogu ich.
\par 17 Bo on pójdzie wprzód przed obliczem jego w duchu i w mocy Elijaszowej, aby obrócil serca ojców ku dzieciom, a odporne ku roztropnosci sprawiedliwych, aby sporzadzil Panu lud gotowy.
\par 18 I rzekl Zacharyjasz do Aniola: Po czemze to poznam? bom ja jest stary, a zona moja podeszla w dniach swych.
\par 19 A odpowiadajac Aniol, rzekl mu: Jam jest Gabryjel, który stoje przed obliczem Bozem, a poslanym jest, abym mówil do ciebie, a izbym ci to wesole poselstwo odniósl.
\par 20 A oto oniemiejesz, i nie bedziesz mógl mówic az do onego dnia, którego sie to stanie, dlatego, zes nie uwierzyl slowom moim, które sie wypelnia czasu swego.
\par 21 A lud oczekiwal Zacharyjasza; i dziwowali sie, ze tak dlugo bawil w kosciele.
\par 22 A wyszedlszy nie mógl do nich mówic; i poznali, ze widzenie widzial w kosciele; bo im przez znaki ukazywal, i zostal niemym.
\par 23 I stalo sie, gdy sie wypelnily dni poslugiwania jego, odszedl do domu swego.
\par 24 A po onych dniach poczela Elzbieta, zona jego, i kryla sie przez piec miesiecy, mówiac:
\par 25 Iz mi tak Pan uczynil we dni, w które na mie wejrzal, aby odjal hanbe moje miedzy ludzmi.
\par 26 A w miesiacu szóstym poslany jest Aniol Gabryjel od Boga do miasta Galilejskiego, które zwano Nazaret,
\par 27 Do Panny poslubionej mezowi, któremu imie bylo Józef, z domu Dawidowego, a imie Panny Maryja.
\par 28 A wszedlszy Aniol do niej, rzekl: Badz pozdrowiona, laska udarowana, Pan jest z toba; blogoslawionas ty miedzy niewiastami.
\par 29 Ale ona ujrzawszy go, zatrwozyla sie na slowa jego, i myslala, jakie by to bylo pozdrowienie.
\par 30 I rzekl jej Aniol: Nie bój sie, Maryjo! albowiem znalazlas laske u Boga.
\par 31 A oto poczniesz w zywocie i porodzisz syna, i nazwiesz imie jego Jezus.
\par 32 Ten bedzie wielki, a Synem Najwyzszego bedzie nazwany, i da mu Pan Bóg stolice Dawida, ojca jego;
\par 33 I bedzie królowal nad domem Jakóbowym na wieki, a królestwu jego nie bedzie konca.
\par 34 Zatem Maryja rzekla do Aniola: Jakoz to bedzie, gdyz ja meza nie znam?
\par 35 A odpowiadajac Aniol, rzekl jej: Duch Swiety zstapi na cie, a moc Najwyzszego zacieni cie; przetoz i to, co sie z ciebie swiete narodzi, nazwane bedzie Synem Bozym.
\par 36 A oto Elzbieta, pokrewna twoja, i ona poczela syna w starosci swojej, a ten miesiac jest szósty onej, która nazywano nieplodna.
\par 37 Bo nie bedzie niemozne u Boga zadne slowo.
\par 38 I rzekla Maryja: Oto sluzebnica Panska; niechze mi sie stanie wedlug slowa twego. I odszedl od niej Aniol.
\par 39 Tedy wstawszy Maryja w onych dniach, poszla w górna kraine z kwapieniem do miasta Judzkiego.
\par 40 A wszedlszy w dom Zacharyjaszowy, pozdrowila Elzbiete.
\par 41 I stalo sie, skoro uslyszala Elzbieta pozdrowienie Maryi, skoczylo niemowlatko w zywocie jej, i napelniona jest Elzbieta Duchem Swietym.
\par 42 I zawolala glosem wielkim, i rzekla: Blogoslawionas ty miedzy niewiastami, i blogoslawiony owoc zywota twego!
\par 43 A skadze mi to, iz przyszla matka Pana mego do mnie?
\par 44 Albowiem jako doszedl glos pozdrowienia twego do uszów moich, podskoczylo od radosci niemowlatko w zywocie moim.
\par 45 A blogoslawiona, która uwierzyla: Gdyz sie wykonaja te rzeczy, które jej sa opowiedziane od Pana.
\par 46 Tedy rzekla Maryja: Wielbi dusza moja Pana;
\par 47 I rozradowal sie duch mój w Bogu, zbawicielu moim,
\par 48 Iz wejrzal na unizenie sluzebnicy swojej; albowiem oto odtad blogoslawiona mie zwac beda wszystkie narody.
\par 49 Bo mi uczynil wielkie rzeczy ten, który mocny jest, i swiete imie jego;
\par 50 I którego milosierdzie zostaje od narodu do narodu nad tymi, co sie go boja.
\par 51 Dokazal mocy ramieniem swojem, i rozproszyl pyszne w myslach serca ich.
\par 52 Sciagnal mocarze z stolic ich, a wywyzszyl unizone.
\par 53 Laknace napelnil dobremi rzeczami, a bogacze rozpuscil prózne.
\par 54 Przyjal Izraela, sluge swego, pamietajac na milosierdzie swoje.
\par 55 Jako mówil do ojców naszych, do Abrahama i nasienia jego na wieki.
\par 56 I zostala z nia Maryja jakoby trzy miesiace; potem sie wrócila do domu swego.
\par 57 A Elzbiecie wypelnil sie czas, aby porodzila, i porodzila syna.
\par 58 A uslyszawszy sasiedzi i pokrewni jej, iz Pan z nia uczynil wielkie milosierdzie swoje, radowali sie pospolu z nia.
\par 59 I stalo sie, ze ósmego dnia przyszli, aby obrzezali dzieciatko; i nazwali je imieniem ojca jego, Zacharyjaszem.
\par 60 Ale odpowiadajac matka jego rzekla: Nie tak; lecz nazwany bedzie Janem.
\par 61 I rzekli do niej: Zadnego nie masz w rodzinie twojej, co by go zwano tem imieniem.
\par 62 I skineli na ojca jego, jako by go chcial nazwac.
\par 63 A on kazawszy sobie podac tabliczke, napisal mówiac: Jan jest imie jego. I dziwowali sie wszyscy.
\par 64 A zarazem otworzyly sie usta jego, i jezyk jego, i mówil, wielbiac Boga.
\par 65 I przyszedl strach na wszystkie sasiady ich, i po wszystkiej górnej krainie Judzkiej rozgloszone sa wszystkie te slowa.
\par 66 Tedy wszyscy, którzy o tem slyszeli, skladali to do serca swego, mówiac: Cóz to wzdy za dziecie bedzie? I byla z nim reka Panska.
\par 67 A Zacharyjasz, ojciec jego, napelniony bedac Duchem Swietym, prorokowal mówiac:
\par 68 Blogoslawiony niech bedzie Pan, Bóg Izraelski, iz nawiedzil i sprawil odkupienie ludowi swojemu;
\par 69 I wystawil nam róg zbawienia w domu Dawida, slugi swego,
\par 70 Tak jako mówil przez usta swietych proroków swoich, którzy byli od wieku:
\par 71 Iz im mial dac wybawienie od nieprzyjaciól naszych i z reki wszystkich, którzy nas nienawidzili;
\par 72 Aby uczynil milosierdzie z ojcami naszymi, i wspomnial na przymierze swoje swiete,
\par 73 I na przysiege, która przysiagl Abrahamowi, ojcu naszemu, ze nam to dac mial,
\par 74 Izbysmy mu bez bojazni, z reki nieprzyjaciól naszych bedac wybawieni, sluzyli;
\par 75 W swietobliwosci i w sprawiedliwosci przed obliczem jego, po wszystkie dni zywota naszego.
\par 76 A ty dzieciatko! Prorokiem Najwyzszego nazwane bedziesz; bo pójdziesz wprzód przed obliczem Panskiem, abys gotowal drogi jego,
\par 77 A izbys dal znajomosc zbawienia ludowi jego przez odpuszczenie grzechów ich.
\par 78 Przez wnetrznosci milosierdzia Boga naszego, w których nawiedzil nas Wschód z wysokosci.
\par 79 Aby sie ukazal siedzacym w ciemnosci i w cieniu smierci ku wyprostowaniu nóg naszych na droge pokoju.
\par 80 A ono dzieciatko roslo, i umacnialo sie w duchu, i bylo na pustyniach az do onego dnia, którego sie okazalo przed Izraelem.

\chapter{2}

\par 1 I stalo sie w one dni, ze wyszedl dekret od cesarza Augusta, aby popisano wszystek swiat.
\par 2 A ten popis pierwszy stal sie, gdy Cyreneusz byl starosta Syryjskim.
\par 3 I szli wszyscy, aby popisani byli, kazdy do miasta swego.
\par 4 Wstapil tez i Józef z Galilei z miasta Nazaretu do ziemi Judzkiej, do miasta Dawidowego, które zowia Betlehem, (przeto iz on byl z domu i z familii Dawidowej;)
\par 5 Aby byl popisany z Maryja, poslubiona sobie malzonka, która byla brzemienna.
\par 6 I stalo sie, gdy tam byli, wypelnily sie dni, aby porodzila.
\par 7 I porodzila syna swego pierworodnego; a uwinela go w pieluszki, i polozyla go w zlobie, przeto iz miejsca nie mieli w gospodzie.
\par 8 A byli pasterze w onej krainie w polu nocujacy i straz nocna trzymajacy nad stadem swojem.
\par 9 A oto Aniol Panski stanal podle nich, a chwala Panska zewszad oswiecila je, i bali sie bojaznia wielka.
\par 10 I rzekl do nich Aniol: Nie bójcie sie; bo oto zwiastuje wam radosc wielka, która bedzie wszystkiemu ludowi:
\par 11 Iz sie wam dzis narodzil zbawiciel, który jest Chrystus Pan, w miescie Dawidowem.
\par 12 A to wam bedzie za znak: znajdziecie niemowlatko uwinione w pieluszki, lezace w zlobie.
\par 13 A zaraz z onym Aniolem przybylo mnóstwo wojsk niebieskich, chwalacych Boga i mówiacych:
\par 14 Chwala na wysokosciach Bogu, a na ziemi pokój, w ludziach dobre upodobanie.
\par 15 I stalo sie, gdy odeszli Aniolowie od nich do nieba, ze oniz pasterze rzekli jedni do drugich: Pójdzmyz az do Betlehemu, a ogladajmy te rzecz, która sie stala, która nam Pan oznajmil.
\par 16 A tak spieszac sie, przyszli i znalezli Maryje i Józefa, i ono niemowlatko lezace w zlobie.
\par 17 I ujrzawszy rozslawili to, co im bylo powiedziano o tem dzieciatku.
\par 18 A wszyscy, którzy slyszeli, dziwowali sie temu, co im pasterze powiadali.
\par 19 Lecz Maryja zachowywala wszystkie te slowa, uwazajac je w sercu swojem.
\par 20 I wrócili sie pasterze, wielbiac i chwalac Boga ze wszystkiego, co slyszeli i widzieli, tak jako im bylo powiedziano.
\par 21 A gdy sie wypelnilo osm dni, aby obrzezano ono dzieciatko, tedy imie jego nazwane jest Jezus, którem bylo nazwane od Aniola, pierwej niz sie w zywocie poczelo.
\par 22 Gdy sie tez wypelnily dni oczyszczenia jej wedlug zakonu Mojzeszowego, przyniesli go do Jeruzalemu, aby go stawili Panu,
\par 23 (Tak jako napisano w zakonie Panskim: ze wszelki mezczyzna, otwierajacy zywot, swietym Panu nazwany bedzie.)
\par 24 A zeby oddali ofiare wedlug tego, co powiedziano w zakonie Panskim, pare synogarlic, albo dwoje golabiat.
\par 25 A oto byl czlowiek w Jeruzalemie, któremu imie bylo Symeon; a ten czlowiek byl sprawiedliwy i bogobojny, oczekujacy pociechy Izraelskiej, a Duch Swiety byl nad nim.
\par 26 I obwieszczony byl od Boga przez Ducha Swietego, ze nie mial ogladac smierci, azby pierwej ogladal Chrystusa Panskiego.
\par 27 Ten przyszedl z natchnienia Ducha Swietego do kosciola; a gdy rodzice wnosili dzieciatko, Jezusa, aby uczynili wedlug zwyczaju zakonnego przy nim.
\par 28 Tedy on wziawszy go na rece swoje, chwalil Boga i mówil:
\par 29 Teraz puszczasz sluge twego, Panie! wedlug slowa twego, w pokoju:
\par 30 Gdyz oczy moje ogladaly zbawienie twoje,
\par 31 Któres zgotowal przed obliczem wszystkich ludzi;
\par 32 Swiatlosc ku objawieniu poganom, a chwale ludu twego Izraelskiego.
\par 33 A ojciec i matka jego dziwowali sie temu, co powiadano o nim.
\par 34 I blogoslawil im Symeon, i rzekl do Maryi, matki jego: Oto ten polozony jest na upadek i na powstanie wielu ich w Izraelu, i na znak, przeciw któremu mówic beda.
\par 35 (I twoje wlasna dusze miecz przeniknie,)aby mysli z wielu serc objawione byly.
\par 36 A byla Anna prorokini, córka Fanuelowa, z pokolenia Asser, która byla bardzo podeszla w latach, i zyla siedm lat z mezem od panienstwa swego.
\par 37 A ta byla wdowa, okolo osmdziesiat i czterech lat; która nie wychodzila z kosciola, w postach i w modlitwach sluzac Bogu w nocy i we dnie.
\par 38 Ta tez onejze godziny nadszedlszy, wyznawala Pana, i mówila o nim wszystkim, którzy oczekiwali odkupienia w Jeruzalemie.
\par 39 A tak wykonawszy wszystko wedlug zakonu Panskiego, wrócili sie do Galilei, do miasta swego Nazaretu.
\par 40 A dzieciatko ono roslo, i umacnialo sie w Duchu, pelne bedac madrosci, a laska Boza byla nad niem.
\par 41 A rodzice jego chadzali na kazdy rok do Jeruzalemu na swieto wielkanocne.
\par 42 A gdy juz byl we dwunastym roku, a oni wstepowali do Jeruzalemu wedlug zwyczaju onego swieta;
\par 43 I gdy skonczyli one dni, a juz sie wracali nazad, zostalo dziecie Jezus w Jeruzalemie, a tego nie wiedzial Józef i matka jego.
\par 44 Lecz mniemajac, ze jest w towarzystwie podróznem, uszli dzien drogi, i szukali go miedzy krewnymi i miedzy znajomymi.
\par 45 A gdy go nie znalezli, wrócili sie do Jeruzalemu, szukajac go,
\par 46 I stalo sie po trzech dniach, ze go znalezli siedzacego w kosciele w posrodku doktorów, sluchajacego ich i pytajacego ich.
\par 47 I zdumiewali sie wszyscy, którzy go sluchali, nad rozumem i nad odpowiedziami jego.
\par 48 A ujrzawszy go rodzice, zdumieli sie. I rzekla do niego matka jego: Synu! przeczzes nam to uczynil? Oto ojciec twój i ja z bolescia szukalismy cie.
\par 49 I rzekl do nich: Cóz jest, zescie mie szukali? Izaliscie nie wiedzieli, iz w tych rzeczach, które sa Ojca mego, ja byc musze?
\par 50 Lecz oni nie zrozumieli tego slowa, które im mówil.
\par 51 I zstapil z nimi, i przyszedl do Nazaretu, a byl im poddany. A matka jego zachowywala wszystkie te slowa w sercu swojem.
\par 52 A Jezus pomnazal sie w madrosci, i we wzroscie i w lasce u Boga i u ludzi.

\chapter{3}

\par 1 A roku pietnastego panowania Tyberyjusza Cesarza, gdy Poncki Pilat byl starosta Judzkim, a Herod Tetrarcha Galilejskim, a Filip, brat jego, Tetrarcha Iturejskim i krainy Trachonickiej, a Lizanijasz Tetrarcha Abilenskim,
\par 2 Za najwyzszych kaplanów Annasza i Kaifasza, stalo sie slowo Boze do Jana, Zacharyjaszowego syna, na puszczy.
\par 3 I przyszedl do wszystkiej krainy lezacej okolo Jordanu, kazac chrzest pokuty na odpuszczenie grzechów.
\par 4 Jako napisano w ksiegach proroctw Izajasza proroka, mówiacego: Glos wolajacego na puszczy; gotujcie droge Panska, proste czyncie sciezki jego.
\par 5 Kazdy padól bedzie wypelniony, a kazda góra i pagórek bedzie znizony, i miejsca krzywe wyprostuja sie, a ostre drogi beda gladkiemi;
\par 6 I oglada wszelkie cialo zbawienie Boze.
\par 7 Mówil tedy ludowi, który wychodzil, aby byl ochrzczony od niego: Rodzaju jaszczurczy! któz wam pokazal, zebyscie uciekali przed przyszlym gniewem?
\par 8 Przynosciez tedy owoce godne pokuty, a nie poczynajcie mówic sami w sobie: Ojca mamy Abrahama; albowiem powiadam wam, ze Bóg moze i z tych kamieni wzbudzic dzieci Abrahamowi.
\par 9 A juz siekiera do korzenia drzew przylozona jest; przetoz kazde drzewo, które nie przynosi owocu dobrego, bywa wyciete i w ogien wrzucone.
\par 10 I pytal go lud mówiac: Cóz tedy czynic bedziemy?
\par 11 A on odpowiadajac rzekl im: Kto ma dwie suknie, niechaj udzieli temu, co nie ma; a kto ma pokarm niech takze uczyni.
\par 12 Przyszli tez i celnicy, aby byli chrzczeni, i rzekli do niego: Nauczycielu! a my cóz czynic bedziemy?
\par 13 A on rzekl do nich: Nic wiecej nie wyciagajcie nad to, co wam postanowiono.
\par 14 Pytali go tez i zolnierze, mówiac: A my cóz czynic bedziemy? I rzekl do nich: Nikomu gwaltu nie czyncie, i nikogo nie potwarzajcie, a przestawajcie na zoldzie waszym.
\par 15 A gdy lud oczekiwal, i myslili wszyscy w sercach swych o Janie, jesliby snac on nie byl Chrystusem,
\par 16 Odpowiedzial Jan wszystkim, mówiac: Jac was chrzcze woda; lecz idzie mocniejszy nad mie, któremum nie jest godzien rozwiazac rzemyka u butów jego; ten was chrzcic bedzie Duchem Swietym i ogniem.
\par 17 Którego lopata jest w reku jego, a wyczysci bojewisko swoje, i zgromadzi pszenice do gumna swego, ale plewy spali ogniem nieugaszonym.
\par 18 A tak wiele i innych rzeczy napominajac, odpowiadal ludowi.
\par 19 A Herod Tetrarcha, bedac strofowany od niego dla Herodyjady, zony Filipa, brata jego, i dla wszystkich zlych spraw, które czynil Herod.
\par 20 Przydal i to nade wszystko, iz wsadzil Jana do wiezienia.
\par 21 I stalo sie, gdy byl ochrzczony wszystek lud, i gdy Jezus byl ochrzczony, i modlil sie, ze sie niebo otworzylo;
\par 22 I zstapil nan Duch Swiety w ksztalcie cielesnym jako golebica, i stal sie glos z nieba, mówiac: Tys jest on Syn mój mily; w tobie mi sie upodobalo.
\par 23 A Jezus poczynal byc jakoby w trzydziestu latach, bedac (jako mniemano,)synem Józefa, syna Helego,
\par 24 Syna Matatowego, syna Lewiego, syna Melchyjego, syna Jannego, syna Józefowego,
\par 25 Syna Matatyjaszowego, syna Amosowego, syna Naumowego, syna Eslego, syna Naggiego,
\par 26 Syna Maatowego, syna Mattatyjaszowego, syna Semejego, syna Józefowego, syna Judowego,
\par 27 Syna Joannowego, syna Resowego, syna Zorobabelowego, syna Salatyjelowego, syna Neryjego,
\par 28 Syna Melchyjego, syna Addyjego, syna Kosamowego, syna Elmodamowego, syna Irowego,
\par 29 Syna Jozego, syna Elijezerowego, syna Jorymowego, syna Mattatego, syna Lewiego,
\par 30 Syna Symeonowego, syna Judowego, syna Józefowego, syna Jonanowego, syna Elijakimowego,
\par 31 Syna Meleowego, syna Mainanowego, syna Mattatanowego, syna Natanowego, syna Dawidowego,
\par 32 Syna Jessego, syna Obedowego, syna Boozowego, syna Salmonowego, syna Nasonowego,
\par 33 Syna Aminadabowego, syna Aramowego, syna Esromowego, syna Faresowego, syna Judowego,
\par 34 Syna Jakóbowego, syna Izaakowego, syna Abrahamowego, syna Tarego, syna Nachorowego,
\par 35 Syna Saruchowego, syna Ragawowego, syna Falekowego, syna Heberowego, syna Salego,
\par 36 Syna Kainowego, syna Arfaksadowego, syna Semowego, syna Noego, syna Lamechowego,
\par 37 Syna Matusalemowego, syna Enochowego, syna Jaredowego, syna Malaleelowego, syna Kainanowego,
\par 38 Syna Enosowego, syna Setowego, syna Adamowego, syna Bozego.

\chapter{4}

\par 1 A Jezus pelen bedac Ducha Swietego, wrócil sie od Jordanu, i pedzony jest od Ducha na puszcza.
\par 2 I byl przez czterdziesci dni kuszony od dyjabla, a nie jadl nic przez one dni; ale gdy sie te skonczyly, potem laknal.
\par 3 I rzekl mu dyjabel: Jezlis jest Syn Bozy, rzecz kamieniowi temu, aby sie stal chlebem.
\par 4 Ale Jezus odpowiedzial mu, mówiac: Napisano, iz nie samym chlebem zyc bedzie czlowiek, ale kazdem slowem Bozem.
\par 5 I wwiódl go dyjabel na góre wysoka, i pokazal mu wszystkie królestwa swiata we mgnieniu oka.
\par 6 I rzekl mu dyjabel: Dam ci te wszystke moc i slawe ich; bo mi jest dana, a komu chce, dawam ja.
\par 7 A tak jezli sie uklonisz przede mna, bedzie wszystko twoje.
\par 8 A odpowiadajac Jezus rzekl mu: Pójdz precz ode mnie, szatanie! albowiem napisano: Panu, Bogu twemu, klaniac sie bedziesz, i jemu samemu sluzyc bedziesz.
\par 9 Potem wiódl go do Jeruzalemu, i postawil go na ganku koscielnym, i rzekl mu: Jezlis jest Syn Bozy, spusc sie stad na dól;
\par 10 Albowiem napisano: Ze Aniolom swoim przykazal o tobie, aby cie strzegli,
\par 11 A ze cie na rekach nosic beda, bys snac nie obrazil o kamien nogi twojej.
\par 12 A odpowiadajac Jezus rzekl mu: Powiedziano: Nie bedziesz kusil Pana, Boga twego.
\par 13 A gdy dokonczyl wszystkich pokus dyjabel, odstapil od niego do czasu.
\par 14 I wrócil sie Jezus w mocy onego Ducha do Galilei. I rozeszla sie o nim wiesc po wszystkiej onej okolicznej krainie.
\par 15 A on nauczal w bóznicach ich, i byl slawiony od wszystkich.
\par 16 I przyszedl do Nazaretu, gdzie byl wychowany, i wszedl wedlug zwyczaju swego w dzien sabatu do bóznicy, i wstal, aby czytal.
\par 17 I podano mu ksiegi Izajasza proroka; a otworzywszy ksiegi, znalazl miejsce, gdzie bylo napisano:
\par 18 Duch Panski nade mna; przeto mie pomazal, abym opowiadal Ewangielije ubogim; poslal mie, abym uzdrawial skruszone na sercu, abym zwiastowal pojmanym wyzwolenie, i slepym przejrzenie, i abym wypuscil ucisnione na wolnosc;
\par 19 Abym opowiadal rok Panski przyjemny.
\par 20 A zawarlszy ksiege i oddawszy ja sludze, usiadl; a oczy wszystkich w bóznicy pilnie nan patrzaly.
\par 21 I poczal do nich mówic: Dzisci sie wypelnilo to pismo w uszach waszych.
\par 22 I wszyscy mu dawali swiadectwo, i dziwowali sie wdziecznosci onych slów, które pochodzily z ust jego, i mówili: Izaz ten nie jest syn Józefowy?
\par 23 I rzekl do nich: Pewnie mi rzeczecie one przypowiesc: Lekarzu! ulecz samego siebie! Cosmy slyszeli, zes uczynil w Kapernaum, uczyn i tu w ojczyznie swojej.
\par 24 I rzekl do nich: Zaprawde wam powiadam: Zaden prorok nie jest przyjemnym w ojczyznie swojej.
\par 25 Alec wam w prawdzie powiadam, ze wiele wdów bylo za dni Elijaszowych w ludzie Izraelskim, gdy bylo zamknione niebo przez trzy lata i szesc miesiecy, tak iz byl wielki glód po wszystkiej ziemi;
\par 26 Wszakze do zadnej z nich nie byl poslany Elijasz, tylko do Sarepty, miasta Sydonskiego, do jednej wdowy.
\par 27 I wiele bylo tredowatych za Elizeusza proroka, w ludzie Izraelskim, wszakze zaden z nich nie byl oczyszczony, tylko Naaman, Syryjczyk.
\par 28 Tedy wszyscy w bóznicy, gdy to slyszeli, napelnieni byli gniewem;
\par 29 A wstawszy, wypchneli go precz z miasta, i wywiedli go na wierzch góry, na której miasto ich zbudowane bylo, aby go z niej na dól zrzucili.
\par 30 Ale on przeszedlszy przez posrodek ich, uszedl.
\par 31 I zstapil do Kapernaum, miasta Galilejskiego, a tam je nauczal w sabaty.
\par 32 I zdumiewali sie nad nauka jego; bo byla mocna mowa jego.
\par 33 A w bóznicy byl czlowiek, który mial ducha dyjabla nieczystego, i zawolal glosem wielkim,
\par 34 Mówiac: Ach! Cóz my z toba mamy, Jezusie Nazarenski? Przyszedles, abys nas wytracil; znam cie, ktos jest, zes on Swiety Bozy.
\par 35 I zgromil go Jezus, mówiac: Umilknij, a wynijdz z niego. Tedy dyjabel porzuciwszy go w posrodek, wyszedl z niego, nic mu nie zaszkodziwszy.
\par 36 I przyszedl strach na wszystkie, i rozmawiali miedzy soba, mówiac: Cóz to za slowo, ze z wladza i z moca rozkazuje duchom nieczystym, a wychodza?
\par 37 I rozeszla sie o nim wiesc na wszystkie miejsca okolicznej krainy.
\par 38 A Jezus wstawszy, z bóznicy wszedl w dom Szymona, a swiekra Szymonowa miala goraczke wielka; i prosili go za nia.
\par 39 Tedy on stanawszy nad nia, zgromil goraczke, i opuscila ja; a zarazem wstawszy, poslugiwala im.
\par 40 A gdy slonce zachodzilo, wszyscy, którzy mieli chorujace na rozmaite niemocy, przywodzili je do niego, a on na kazdego z nich rece wlozywszy, uzdrawial je.
\par 41 Ku temu wychodzili i dyjabli z wielu ich, wolajac i mówiac: Tys jest on Chrystus, Syn Bozy; ale on zgromiwszy je, nie dopuszczal im mówic; bo wiedzieli, iz on jest Chrystus.
\par 42 A gdy byl dzien, wyszedlszy, szedl na miejsce puste. A lud go szukal, i przyszli az do niego, i zatrzymywali go, aby nie odchodzil od nich.
\par 43 A on rzekl do nich: I innym miastom musze opowiadac królestwo Boze; bom na to poslany.
\par 44 I kazal w bóznicach Galilejskich.

\chapter{5}

\par 1 I stalo sie, gdy nan lud nalegal, aby sluchal slowa Bozego, ze on stal podle jeziora Gienezaretskiego.
\par 2 I ujrzal dwie lodzi stojace przy jeziorze; ale rybitwi wyszedlszy z nich, plukali sieci.
\par 3 A wstapiwszy w jedna z tych lodzi, która byla Szymonowa, prosil go, aby maluczko odjechal od brzegu; a usiadlszy, uczyl on lud z onej lodzi.
\par 4 A gdy przestal mówic, rzekl do Szymona: Zajedz na glebie, a zapusccie sieci wasze ku lowieniu.
\par 5 A odpowiadajac Szymon, rzekl mu: Mistrzu! przez cala noc robiac, nicesmy nie pojmali, wszakze na slowo twoje zapuszcze siec.
\par 6 A gdy to uczynili, zagarneli ryb mnóstwo wielkie, tak ze sie rwala siec ich.
\par 7 I skineli na towarzysze, którzy byli w drugiej lodzi, aby przybywszy ratowali ich; i przybyli i napelnili obie lodzi, az sie zanurzaly.
\par 8 Co widzac Szymon Piotr, przypadl do kolan Jezusowych, mówiac: Wynijdz ode mnie; bom jest czlowiek grzeszny, Panie!.
\par 9 Albowiem go byl strach ogarnal, i wszystkie, co z nim byli, z onego oblowu ryb, które byli zagarneli.
\par 10 Takze i Jakóba i Jana, syny Zebedeuszowe, którzy byli towarzysze Szymonowi. I rzekl Jezus do Szymona: Nie bój sie; od tego czasu ludzi lowic bedziesz.
\par 11 A oni wyciagnawszy lódz na brzeg, wszystko opusciwszy, poszli za nim.
\par 12 I stalo sie, gdy byl w niektórem miescie, ze oto byl tam maz pelen tradu, który ujrzawszy Jezusa, padl na twarz, i prosil go, mówiac: Panie! jezli chcesz, mozesz mie oczyscic.
\par 13 Tedy wyciagnawszy Jezus reke, dotknal sie go, mówiac: Chce, badz oczyszczony; i zaraz odszedl trad od niego.
\par 14 I przykazal mu, aby tego nikomu nie powiadal: ale rzekl: Idz, a ukaz sie kaplanowi, i ofiaruj za oczyszczenie twoje, tak jako rozkazal Mojzesz, na swiadectwo przeciwko nim.
\par 15 I rozchodzila sie tem wiecej powiesc o nim; i schodzily sie mnóstwa wielkie, aby go sluchaly i uzdrowieni byli od niego od niemocy swoich.
\par 16 Ale on odchodzil na pustynia, i modlil sie.
\par 17 I stalo sie dnia niektórego, ze on nauczal, a siedzieli tez tam i Faryzeuszowie i nauczyciele Zakonu, którzy sie byli zeszli ze wszystkich miasteczek Galilejskich i Judzkich, i z Jeruzalemu; a moc Panska przytomna byla uzdrawianiu ich.
\par 18 A oto mezowie niesli na lozu czlowieka powietrzem ruszonego, i szukali, jakoby go wniesc i postawic przed nim.
\par 19 A gdy nie znalezli, któredy by go wniesli, dla cizby, wstapiwszy na dach, przez posowe spuscili go lozem w posrodek przed Jezusa.
\par 20 Który ujrzawszy wiare ich, rzekl mu: Czlowiecze! odpuszczone sa tobie grzechy twoje.
\par 21 Tedy poczeli myslic nauczeni w Pismie i Faryzeuszowie, mówiac: Któz to jest, co mówi bluznierstwa? Któz moze odpuszczac grzechy, tylko sam Bóg.
\par 22 Ale Jezus poznawszy mysli ich, odpowiadajac rzekl do nich: Cóz myslicie w sercach waszych?
\par 23 Cóz jest latwiejszego, rzec: Odpuszczone sa tobie grzechy twoje, czyli rzec: Wstan a chodz?
\par 24 Ale izbyscie wiedzieli, ze Syn czlowieczy ma moc na ziemi odpuszczac grzechy, (rzekl powietrzem ruszonemu:) Tobie mówie: Wstan, a wziawszy na sie loze swoje, idz do domu twego.
\par 25 A on zarazem wstawszy przed nimi, wziawszy na sie to, na czem lezal, szedl do domu swego, wielbiac Boga.
\par 26 I zdumieli sie wszyscy, i chwalili Boga, i napelnieni byli bojaznia, mówiac: Widzielismy dzis dziwne rzeczy.
\par 27 A potem wyszedl i ujrzal celnika, imieniem Lewiego, siedzacego na cle, i rzekl mu: Pójdz za mna.
\par 28 I opuscil wszystko, a wstawszy, szedl za nim.
\par 29 I sprawil mu lewi uczte wielka w domu swoim; a bylo wielkie zgromadzenie celników i innych, którzy z nim za stolem siedzieli.
\par 30 Tedy szemrali nauczeni w Pismie i Faryzeuszowie, mówiac do uczniów jego: Przecz z celnikami i z grzesznikami jecie i pijecie?
\par 31 A Jezus odpowiadajac, rzekl do nich: Nie potrzebujac zdrowi lekarza, ale ci, co sie zle maja;
\par 32 Nie przyszedlem, wzywac sprawiedliwych, ale grzesznych do pokuty.
\par 33 A oni mu rzekli: Przecz uczniowie Janowi czesto poszcza i modla sie, takze i Faryzejscy, a twoi jedza i pija?
\par 34 A on im rzekl: Izali mozecie uczynic, zeby synowie loznicy malzenskiej poscili, póki z nimi jest oblubieniec?
\par 35 Lecz przyjda dni, gdy oblubieniec odjety bedzie od nich; tedy w one dni poscic beda.
\par 36 Powiedzial im tez podobienstwo: Iz zaden laty z szaty nowej nie przyprawia do szaty wiotchej; bo inaczej to, co jest nowego, drze wiotche, a do wiotchego nie zgadza sie lata z nowego.
\par 37 I nikt nie leje wina nowego w stare statki; bo inaczej wino mlode rozsadzi statki, i samo wyciecze, i statki sie popsuja.
\par 38 Ale mlode wino ma byc wlewane w statki nowe; a tak oboje bywaja zachowane.
\par 39 A nikt, kto sie napil starego, nie zaraz chce mlodego; ale mówi: Lepsze jest stare.

\chapter{6}

\par 1 I stalo sie w drugi sabat, ze szedl Jezus przez zboza; i rwali uczniowie jego klosy, a rekami wycierajac jedli.
\par 2 Ale niektórzy z Faryzeuszów rzekli do nich: Przeczze czynicie to, czego sie nie godzi czynic w sabat?
\par 3 A odpowiadajac Jezus, rzekl do nich: Azascie tego nie czytali, co uczynil Dawid, gdy laknal sam, i ci, którzy z nim byli?
\par 4 Jako wszedl do domu Bozego, a wzial chleby pokladne, i jadl, a dal i tym, którzy z nim byli; których sie nie godzilo jesc, tylko samym kaplanom?
\par 5 I rzekl im: Syn czlowieczy jestci Panem i sabatu.
\par 6 Stalo sie takze i w inszy sabat, ze Jezus wszedl do bóznicy, i nauczal; i byl tam czlowiek, którego reka prawa byla uschla.
\par 7 I podstrzegli go nauczeni w Pismie i Faryzeuszowie, jezliby w sabat uzdrawial, aby znalezli, o coby nan skarzyli.
\par 8 Ale on wiedzial mysli ich, i rzekl czlowiekowi, który mial reke uschla: Wstan a stan posrodku. A on wstawszy, stanal.
\par 9 Rzekl tedy do nich Jezus: Spytam was o jedne rzecz: Godzili sie w sabaty dobrze czynic, czyli zle czynic? Czlowieka zachowac, czyli zatracic?
\par 10 A spojrzawszy w kolo po wszystkich, rzekl onemu czlowiekowi: Wyciagnij reke twoje! a on tak uczynil i przywrócona jest do zdrowia reka jego, jako i druga.
\par 11 Ale oni napelnieni bedac szalenstwem, rozmawiali miedzy soba, coby uczynic mieli Jezusowi.
\par 12 I stalo sie w onez dni, odszedl na góre, aby sie modlil; i byl tam przez noc na modlitwie Bozej.
\par 13 A gdy byl dzien, zwolal uczniów swych i wybral z nich dwanascie, które tez nazwal Apostolami:
\par 14 Szymona, którego tez nazwal Piotrem, i Andrzeja brata jego, Jakóba, i Jana, Filipa, i Bartlomieja;
\par 15 Mateusza, i Tomasza, Jakóba, syna Alfeuszowego, i Szymona, którego zowia Zelotes;
\par 16 Judasza, brata Jakóbowego, i Judasza Iszkarjote, który potem byl zdrajca.
\par 17 A zstapiwszy z nimi stanal na miejscu pola równego, i gormada uczniów jego, i wielkie mnóstwo ludu ze wszystkiej Judzkiej ziemi, i z Jeruzalemu, i z kraju pomorskiego, lezacego przy Tyrze i Sydonie, którzy byli przyszli, aby go sluchali, i byli uzdrowieni od chorób swoich;
\par 18 I ci, którzy byli trapieni od duchów nieczystych, byli uzdrowieni.
\par 19 A wszystek lud szukal, jakoby sie go dotknac; albowiem moc wychodzila z niego, i uzdrawiala wszystkich.
\par 20 A on podnióslszy oczy swoje na uczniów, mówil: Blogoslawieni jestescie wy, ubodzy! bo wasze jest królestwo Boze.
\par 21 Blogoslawieni jestescie, którzy teraz lakniecie; bo bedziecie nasyceni. Blogoslawieni jestescie, którzy teraz placzecie; bo sie smiac bedziecie.
\par 22 Blogoslawieni bedziecie, gdy was ludzie nienawidziec beda, i gdy wylacza, i beda was sromocic, i imie wasze wyrzuca jako zle, dla Syna czlowieczego.
\par 23 Radujcie sie dnia tego i weselcie sie; albowiem oto zaplata wasza jest obfita w niebiesiech; boc tak wlasnie prorokom czynili ojcowie ich.
\par 24 Ale biada wam bogaczom! bo juz macie pocieche wasze.
\par 25 Biada wam, którzyscie nasyceni! albowiem laknac bedziecie. Biada wam, którzy sie teraz smiejecie! bo sie smucic i plakac bedziecie.
\par 26 Biada wam, gdyby dobrze o was mówili wszyscy ludzie; bo tak czynili falszywym prorokom ojcowie ich.
\par 27 Ale wam powiadam, którzy sluchacie: Milujcie nieprzyjacioly wasze, czyncie dobrze tym, którzy was maja w nienawisci.
\par 28 Blogoslawcie tym, którzy was przeklinaja; módlcie sie za tymi, którzy wam zlosc wyrzadzaja.
\par 29 Temu, któryby cie uderzyl w policzek, nastaw mu i drugiego: a temu, którycby bral plaszcz, i sukni nie zabraniaj;
\par 30 I kazdemu, któryby cie prosil, daj, a temu, co twoje bierze, nie upominaj sie.
\par 31 I cobyscie chcieli, aby wam ludzie czynili, tak i wy im czyncie.
\par 32 Albowiem jezli milujecie te, którzy was miluja, jakaz laske macie? albowiem toz i grzesznicy wlasnie czynia.
\par 33 A jezli dobrze czynicie tym, którzy wam dobrze czynia, jakaz laske macie? albowiem toz i grzesznicy wlasnie czynia.
\par 34 A jezli pozyczacie tym, od których sie spodziewacie odebrac, jakaz laske macie? albowiem i grzesznicy grzesznikom pozyczaja, aby zasie tyle odebrali.
\par 35 Owszem milujcie nieprzyjacioly wasze, i czyncie im dobrze, i pozyczajcie, nic sie stad nie spodziewajac, a bedzie wielka zaplata wasza, i bedziecie synami Najwyzszego; albowiem on dobrotliwy jest przeciw niewdziecznym i zlym.
\par 36 Przetoz badzcie milosierni, jako i Ojciec wasz milosierny jest.
\par 37 Nie sadzcie, a nie badziecie sadzeni; nie potepiajcie, a nie bedziecie potepieni, a bedzie wam odpuszczono.
\par 38 Dawajcie, a bedzie wam dano; miare dobra, natloczona, i potrzesiona, i oplywajaca dadza na lono wasze; albowiem taz miara, która mierzycie, bedzie wam zas obmierzono.
\par 39 I powiedzial im podobienstwo: Izali moze slepy slepego prowadzic? azaz nie obadwaj w dól wpadna?
\par 40 Nie jestci uczen nad mistrza swego; lecz doskonaly bedzie kazdy, bedzieli jako mistrz jego.
\par 41 A czemuz widzisz zdzblo w oku brata twego, a balki, która jest w oku twojem, nie baczysz?
\par 42 Albo jakoz mozesz rzec bratu twemu: Bracie! dopusc, iz wyjme zdzblo, które jest w oku twojem, a sam balki, która jest w oku twojem, nie widzisz? Obludniku! wyjmij pierwej balke z oka twego, a tedy przejrzysz, abys wyjal zdzblo, które jest w oku brata twego.
\par 43 Nie jest bowiem drzewo dobre, które przynosi owoc zly; ani jest drzewo zle, które przynosi owoc dobry;
\par 44 Gdyz kazde drzewo z owocu wlasnego poznane bywa; boc nie zbieraja z ciernia figów, ani z glogu zbieraja winnych gron.
\par 45 Czlowiek dobry z dobrego skarbu serca swego wynosi rzeczy dobre, a zly czlowiek ze zlego skarbu serca swego wynosi rzeczy zle; albowiem z obfitosci serca mówia usta jego.
\par 46 Przeczze mie tedy zowiecie Panie, Panie! a nie czynicie tego, co mówie?
\par 47 Kazdy, który przychodzi do mnie, a slucha slów moich, i czyni je, pokaze wam, komu jest podobnym.
\par 48 Podobny jest czlowiekowi dom budujacemu, który kopal i wykopal gleboko, a zalozyl grunt na opoce; a gdy przyszla powódz, otracila sie rzeka o on dom, ale nie mogla go poruszyc; bo byl zalozony na opoce.
\par 49 Ale który slucha, a nie czyni, podobny jest czlowiekowi, który zbudowal dom swój na ziemi bez gruntu; o który sie otracila rzeka, i zarazem upadl, a byl upadek domu onego wielki.

\chapter{7}

\par 1 A gdy dokonczyl wszystkich mów swoich przed onym ludem, wszedl do Kapernaum;
\par 2 A niektórego setnika sluga zle sie majac, juz prawie mial umrzec, którego on sobie bardzo powazal.
\par 3 Ten uslyszawszy o Jezusie, poslal do niego starszych z Zydów, proszac go, aby przyszedlszy uzdrowil sluge jego.
\par 4 A oni przyszedlszy do Jezusa, prosili go z pilnoscia, mówiac: Godzien jest, abys mu to uczynil;
\par 5 Albowiem miluje naród nasz, i on nam bóznice zbudowal.
\par 6 A tak Jezus szedl z nimi. Ale gdy niedaleko byl od domu, poslal do niego on setnik przyjacioly, mówiac mu: Panie! nie zadawaj sobie pracy; bomci nie jest godzien, abys wszedl pod dach mój.
\par 7 Przetoz i samego siebie nie mialem za godnego, abym mial przyjsc do ciebie; ale rzecz slowo, a bedzie uzdrowiony sluga mój.
\par 8 Bomci i ja czlowiek pod moca postanowiony, majacy pod soba zolnierzy, i mówie temu: Idz, a idzie, a drugiemu: Przyjdz, a przychodzi, a sludze mojemu: Czyn to, a czyni.
\par 9 Tedy uslyszawszy to Jezus, zadziwil mu sie, i obróciwszy sie, rzekl do ludu, który za nim szedl: Powiadam wam, zem ani w Izraelu tak wielkiej wiary nie znalazl.
\par 10 A wróciwszy sie do domu ci, którzy byli poslani, znalezli sluge, który sie zle mial, zdrowego.
\par 11 I stalo sie nazajutrz, ze szedl do miasta, które zowia Naim, a szlo z nim uczniów jego wiele i lud wielki.
\par 12 A gdy sie przyblizyl do bramy miejskiej, tedy oto wynoszono umarlego, syna jedynego matki swojej, a ta byla wdowa, a z nia szedl wielki lud miasta onego.
\par 13 Która ujrzawszy Pan uzalil sie jej, i rzekl jej: Nie placz!
\par 14 I przystapiwszy dotknal sie trumny (mar), (a ci, co niesli, staneli) i rzekl: Mlodziencze! tobie mówie, wstan.
\par 15 I usiadl on, który byl umarl, i poczal mówic; i oddal go matce jego.
\par 16 Tedy wszystkich strach zdjal, a wielbili Boga, mówiac: Prorok wielki powstal miedzy nami, a Bóg nawiedzil lud swój.
\par 17 I rozeszla sie o nim ta wiesc po wszystkiej Judzkiej ziemi, i po wszystkiej okolicznej krainie.
\par 18 I oznajmili Janowi uczniowie jego o tem wszystkiem. A Jan wezwawszy dwóch niektórych z uczniów swoich,
\par 19 Poslal je do Jezusa, mówiac: Tyzes jest ten, który ma przyjsc, czyli inszego czekac mamy?
\par 20 A gdy przyszli do niego mezowie oni, rzekli: Jan Chrzciciel poslal nas do ciebie, mówiac: Tyzes jest ten, który ma przyjsc, czyli inszego czekac mamy?
\par 21 A onejze godziny wiele ich uzdrowil od chorób, od niemocy, i od duchów zlych, i wiele slepych wzrokiem darowal.
\par 22 A odpowiadajac Jezus, rzekl im: Szedlszy oznajmijcie Janowi, coscie widzieli i slyszeli, iz slepi widza, chromi chodza, tredowaci biora oczyszczenie, glusi slysza, umarli zmartwychwstaja, a ubogim opowiadana bywa Ewangielija.
\par 23 A blogoslawiony jest, kto by sie nie zgorszyl ze mnie.
\par 24 A gdy odeszli poslowie Janowi, poczal mówic do ludu o Janie: Coscie wyszli na puszcze widziec? Izali trzcine chwiejaca sie od wiatru?
\par 25 Ale coscie wyszli widziec? Izali czlowieka w miekkie szaty obleczonego? Oto ci, którzy w szatach kosztownych i w rozkoszy zyja, sa w domach królewskich.
\par 26 Ale coscie wyszli widziec? Izali proroka? Zaiste powiadam wam, iz wiecej niz proroka.
\par 27 Tenci bowiem jest, o którym napisano: Oto Ja posylam Aniola mego przed obliczem twojem, który zgotuje droge twoje przed toba.
\par 28 Albowiem powiadam wam: Wiekszego proroka z tych, którzy sie z niewiast rodza, nie masz nad Jana Chrzciciela zadnego; lecz kto najmniejszy jest w królestwie Bozem, wiekszy jest, nizeli on.
\par 29 Tedy wszystek lud slyszac to, i celnicy, wielbili Boga, bedac ochrzczeni chrztem Janowym.
\par 30 Ale Faryzeuszowie i zakonnicy pogardzili rada Boza sami przeciwko sobie, nie bedac ochrzczeni od niego.
\par 31 I rzekl Pan: Komuz tedy przypodobam ludzi rodzaju tego, a komu sa podobni?
\par 32 Podobni sa dzieciom, które siedza na rynku, a jedne na drugie wolaja, mówiac: Gralysmy wam na piszczalkach, a nie tancowaliscie; spiewalysmy zalobne piesni, a nie plakaliscie.
\par 33 Albowiem przyszedl Jan Chrzciciel, i chleba nie jedzac i wina nie pijac, a mówicie: Dyjabelstwo ma.
\par 34 Przyszedl Syn czlowieczy jedzac i pijac, a mówicie: Oto czlowiek obzerca i pijanica wina, przyjaciel celników i grzeszników.
\par 35 Ale usprawiedliwiona jest madrosc od wszystkich synów swoich.
\par 36 I prosil go niektóry z Faryzeuszów, aby z nim jadl. A tak wszedlszy w dom Faryzeuszów, usiadl.
\par 37 A oto niewiasta, która byla w miescie grzeszna, dowiedziawszy sie, iz siedzi w domu Faryzeuszowym, przyniosla alabastrowy sloik masci;
\par 38 A stanawszy z tylu u nóg jego, placzac poczela lzami polewac nogi jego, a wlosami glowy swojej ucierala, i calowala nogi jego, i mascia mazala.
\par 39 A widzac to Faryzeusz, który go byl wezwal, rzekl sam w sobie, mówiac: Byc ten byl prorokiem, wiedzialby, która i jaka jest ta niewiasta, co sie go dotyka; bo jest grzesznica.
\par 40 A odpowiadajac Jezus, rzekl do niego: Szymonie! mam ci nieco powiedziec, a on rzekl: Powiedz, Nauczycielu!
\par 41 Mial niektóry lichwiarz dwóch dluzników; jeden dluzen byl piecset groszy, a drugi piecdziesiat.
\par 42 A gdy oni nie mieli czem zaplacic, odpuscil obydwom. Powiedz tedy, któryz z nich bardziej go milowac bedzie?
\par 43 A odpowiadajac Szymon, rzekl: Mniemam, iz ten, któremu wiecej odpuscil. A on mu rzekl: Dobrzes rozsadzil.
\par 44 I obróciwszy sie do niewiasty, rzekl Szymonowi: Widzisz te niewiaste? Wszedlem do domu twego, nie dales wody na nogi moje; ale ta lzami polala nogi moje, i wlosami glowy swej otarla.
\par 45 Nie pocalowales mie, ale ta jako weszla, nie przestala calowac nóg moich.
\par 46 Nie pomazales oliwa glowy mojej, ale ta mascia pomazala nogi moje.
\par 47 Dlaczego, mówie tobie, odpuszczono jej wiele grzechów, gdyz wiele umilowala; a komu malo odpuszczono, malo miluje.
\par 48 A on jej rzekl: Odpuszczone sa tobie grzechy.
\par 49 I poczeli spólsiedzacy mówic miedzy soba: Któz jest ten, który i grzechy odpuszcza?
\par 50 I rzekl do niewiasty: Wiara twoja ciebie zbawila. Idzze w pokoju.

\chapter{8}

\par 1 I stalo sie potem, ze on chodzil po miastach i po miasteczkach kazac i opowiadajac królestwo Boze, a oni dwunastu byli z nim,
\par 2 I niektóre niewiasty, które byl uzdrowil od duchów zlych i od niemocy ich, jako Maryja, która zwano Magdalena, z której bylo siedm dyjablów wyszlo;
\par 3 I Joanna, zona Chuzego, urzednika Herodowego, i Zuzanna, i inszych wiele, które mu sluzyly z majetnosci swoich.
\par 4 A gdy sie schodzil wielki lud, i z róznych miast garneli sie do niego, rzekl przez podobienstwo;
\par 5 Wyszedl rozsiewca, aby rozsiewal nasienie swoje; a gdy on rozsiewal, tedy jedno padlo podle drogi i podeptane jest, a ptaki niebieskie podziobaly je.
\par 6 A drugie padlo na opoke, a gdy wzeszlo, uschlo, przeto iz nie mialo wilgotnosci.
\par 7 A drugie padlo miedzy ciernie; ale ciernie wespól z niem wzrosly, i zadusily je.
\par 8 A drugie padlo na ziemie dobra, a gdy wzeszlo, przynioslo pozytek stokrotny. To mówiac wolal: Kto ma uszy ku sluchaniu, niechaj slucha!
\par 9 I pytali go uczniowie jego, mówiac: Co by to bylo za podobienstwo?
\par 10 A on im rzekl: Wam dano wiedziec tajemnice królestwa Bozego; ale innym w podobienstwach, aby widzac nie widzieli, a slyszac nie rozumieli.
\par 11 A to podobienstwo takie jest: nasienie jest slowo Boze.
\par 12 A którzy podle drogi, ci sa, którzy sluchaja, zatem przychodzi dyjabel, i wybiera slowo z serca ich, aby uwierzywszy, nie byli zbawieni.
\par 13 A którzy na opoce, ci sa, którzy gdy sluchaja, z radoscia slowo przyjmuja, ale ci korzenia nie maja, ci do czasu wierza, a czasu pokusy odstepuja.
\par 14 A które padlo miedzy ciernie, ci sa, którzy sluchaja slowa: ale odszedlszy, od pieczolowania i bogactw, i rozkoszy zywota bywaja zaduszeni, i nie przynosza pozytku.
\par 15 Ale które padlo na ziemie dobra, ci sa, którzy w sercu uprzejmem i dobrem slyszane slowo zachowuja, i owoc przynosza w cierpliwosci.
\par 16 A zaden zapaliwszy swiece, nie nakrywa jej naczyniem, ani jej kladzie pod loze, ale ja stawia na swieczniku, aby ci, którzy wchodza, widzieli swiatlo.
\par 17 Bo nie masz nic tajemnego, co by nie mialo byc objawiono; i nie masz nic skrytego, czego by sie nie dowiedziano, i co by na jaw nie wyszlo.
\par 18 Przetoz patrzcie, jako sluchacie: albowiem kto ma, temu bedzie dane, a kto nie ma, i to, co mniema, ze ma, bedzie odjete od niego.
\par 19 Tedy przyszli do niego matka i bracia jego; ale do niego przystapic nie mogli dla ludu.
\par 20 I dano mu znac, mówiac: Matka twoja i bracia twoi stoja przed domem, chcac cie widziec.
\par 21 A on odpowiadajac, rzekl do nich: Matka moja i bracia moi sa ci, którzy slowa Bozego sluchaja i czynia je.
\par 22 I stalo sie dnia jednego, ze on wstapil w lódz i uczniowie jego, i rzekl do nich: Przeprawmy sie na druga strone jeziora. I puscili sie.
\par 23 A gdy plyneli, usnal. I przypadla nawalnosc wiatru na jezioro, i lódz sie zalewala, tak ze byli w niebezpieczenstwie.
\par 24 A przystapiwszy, obudzili go, mówiac: Mistrzu, mistrzu! giniemy. A on ocknawszy sie, zgromil wiatr i waly wodne, i usmierzyly sie, i stalo sie uciszenie.
\par 25 Tedy im rzekl: Gdziez jest wiara wasza? A bojac sie, dziwowali sie, mówiac jedni do drugich: Któz wzdy jest ten, ze i wiatrom rozkazuje i wodom, a sa mu posluszne?
\par 26 I przewiezli sie do krainy Gadarenczyków, która jest przeciw Galilei.
\par 27 A gdy wstapil na ziemie, zabiezal mu maz niektóry z onego miasta, co mial dyjably od niemalego czasu, a nie oblóczyl sie w szaty, i nie mieszkal w domu, tylko w grobach.
\par 28 Ten ujrzawszy Jezusa, zakrzyknal, i upadl przed nim, a glosem wielkim rzekl: Cóz ja mam z toba, Jezusie, Synu Boga najwyzszego? prosze cie, nie drecz mie.
\par 29 Albowiem rozkazal onemu duchowi nieczystemu, aby wyszedl z onego czlowieka: bo od wielu czasów porywal go; a chociaz go wiazano lancuchami i w petach strzezono, jednak on porwawszy okowy, bywal od dyjabla na pustynie pedzony.
\par 30 I pytal go Jezus, mówiac: Co masz za imie? A on rzekl: Wojsko; albowiem wiele dyjablów wstapilo bylo wen.
\par 31 Tedy go prosili, aby im nie rozkazywal stamtad odejsc w przepasc.
\par 32 A byla tam trzoda wielka swin, która sie pasla na górze, i prosili go, aby im dopuscil wstapic w nie. I dopuscil im.
\par 33 A wyszedlszy dyjabli z onego czlowieka, weszli w swinie; i porwala sie ona trzoda pedem z przykra do jeziora, i utonela.
\par 34 A widzac pasterze, co sie stalo, uciekli; a poszedlszy, oznajmili to w miescie i we wsiach.
\par 35 I wyszli, aby ogladali to, co sie stalo; a przyszedlszy do Jezusa, znalezli czlowieka onego, z którego wyszli dyjabli, obleczonego, przy dobrem baczeniu, siedzacego u nóg Jezusowych, i bali sie.
\par 36 Opowiedzieli im tedy ci, którzy widzieli, jako uzdrowiono tego, który byl opetany.
\par 37 I prosilo go wszystko mnóstwo onej okolicznej krainy Gadarenczyków, aby odszedl od nich; albowiem ich byl wielki strach ogarnal. A on wstapiwszy w lódz, wrócil sie.
\par 38 I prosil go on maz, z którego wyszli dyjabli, aby byl przy nim; ale go Jezus odprawil, mówiac:
\par 39 Wróc sie do domu twego, a opowiadaj, jakoc wielkie rzeczy Bóg uczynil. I odszedl, po wszystkiem miescie opowiadajac, jako mu wielkie rzeczy Jezus uczynil.
\par 40 I stalo sie, gdy sie wrócil Jezus, ze go przyjal lud; albowiem nan wszyscy oczekiwali.
\par 41 A oto przyszedl maz imieniem Jairus, a ten byl przelozonym bóznicy; a przypadlszy do nóg Jezusowych, prosil go, aby wszedl w dom jego.
\par 42 Albowiem mial córke jedyna okolo dwunastu lat, która juz konala. (A gdy on szedl, cisnal go lud.)
\par 43 A niewiasta, która plynienie krwi cierpiala od lat dwunastu, i wynalozyla byla na lekarzy wszystko swoje pozywienie, a nie mogla byc od nikogo uleczona,
\par 44 Przystapiwszy z tylu, dotknela sie podolka szaty jego, a zarazem sie zastanowilo plynienie krwi jej.
\par 45 I rzekl Jezus: Któz jest, co sie mnie dotknal? a gdy sie wszyscy zapierali, rzekl Piotr, i ci, którzy z nim byli: Mistrzu! lud cie cisnie i tloczy, a ty mówisz: Kto sie mnie dotknal?
\par 46 I rzekl Jezus: Dotknal sie mnie ktos, bom poznal, ze moc ode mnie wyszla.
\par 47 A widzac ona niewiasta, ze sie nie utaila, ze drzeniem przystapila i upadla przed nim, i dlaczego sie go dotknela, powiedziala mu przed wszystkim ludem, i jako zaraz uzdrowiona byla.
\par 48 A on jej rzekl: Ufaj, córko! wiara twoja ciebie uzdrowila; idzze w pokoju.
\par 49 A gdy on to jeszcze mówil, przyszedl niektóry od przelozonego bóznicy, powiadajac mu: Iz umarla córka twoja, nie trudz Nauczyciela.
\par 50 Ale Jezus uslyszawszy to, odpowiedzial mu, mówiac: Nie bój sie, tylko wierz, a bedzie uzdrowiona.
\par 51 A wszedlszy w dom, nie dopuscil z soba wnijsc nikomu, tylko Piotrowi, i Jakóbowi, i Janowi, i ojcu i matce onej dzieweczki.
\par 52 A plakali wszyscy, i narzekali nad nia. Ale on rzekl: Nie placzciez! Nie umarlac, ale spi.
\par 53 I nasmiewali sie z niego, wiedzac, iz byla umarla.
\par 54 A on wygnawszy precz wszystkich, i ujawszy ja za reke, zawolal, mówiac: Dzieweczko, wstan!
\par 55 I wrócil sie duch jej; i wstala zaraz; i rozkazal, aby jej jesc dano.
\par 56 I zdumieli sie rodzice jej. A on im zakazal, aby nikomu nie powiadali tego, co sie bylo stalo.

\chapter{9}

\par 1 A zwolawszy Jezus dwunastu uczniów swoich, dal im moc i wladze nad wszystkimi dyjably, i aby uzdrawiali choroby.
\par 2 I rozeslal je, zeby kazali królestwo Boze, i uzdrawiali niemocne.
\par 3 Tedy rzekl do nich: Nie bierzcie nic na droge, ani laski, ani taistry, ani chleba, ani pieniedzy, ani dwóch sukien miejcie.
\par 4 A do któregokolwiek domu wnijdziecie, tamze zostancie, i stamtad wynijdzcie.
\par 5 A którzybykolwiek was nie przyjeli, wychodzac z miasta onego, i proch z nóg waszych otrzasnijcie na swiadectwo przeciwko nim.
\par 6 Wyszedlszy tedy, obchodzili wszystkie miasteczka, opowiadajac Ewangielije, a wszedzie chore uzdrawiajac.
\par 7 I uslyszal Herod, Tetrarcha, o wszystkiem, co sie dzialo od niego, i byl watpliwym dla tego, ze niektórzy powiadali, iz Jan zmartwychwstal.
\par 8 A niektórzy zasie: Iz sie Elijasz ukazal; a drudzy, iz prorok jeden z onych starych zmartwychwstal.
\par 9 Tedy rzekl Herod: Janamci ja scial; któz wzdy ten jest, o którym ja takie rzeczy slysze? i pragnal go widziec.
\par 10 A wróciwszy sie Apostolowie, powiadali mu, cokolwiek czynili. A on wziawszy je z soba, ustapil osobno na miejsce puste przy miescie, które zowia Betsaida.
\par 11 Czego gdy sie lud dowiedzial, szedl za nim; a przyjawszy je, mówil im o królestwie Bozem; a te, którzy uzdrowienia potrzebowali, uzdrawial.
\par 12 A gdy sie dzien poczal sklaniac ku wieczorowi, przystapiwszy oni dwunastu, rzekli mu: Rozpusc ten lud, aby odszedlszy do miasteczek okolicznych, i do wsi, i do gospód, znalezli zywnosc; bosmy tu na miejscu pustem.
\par 13 Ale on rzekl do nich: Dajciez wy im jesc. A oni powiedzieli: Nie mamy wiecej, tylko piec chlebów i dwie ryby, oprócz zebysmy szli, a kupili, na ten wszystek lud zywnosci.
\par 14 Albowiem bylo mezów okolo pieciu tysiecy. I rzekl do uczniów swoich: Rozkazcie im usiasc w kazdym rzedzie po piecdziesiat.
\par 15 I uczynili tak, i usiedli wszyscy.
\par 16 A on wziawszy onych piec chlebów i one dwie ryby, wejrzawszy w niebo, blogoslawil im, i lamal i dawal uczniom, aby kladli przed on lud.
\par 17 I jedli, i nasyceni sa wszyscy; i zebrano, co im zbylo z ulomków, dwanascie koszów.
\par 18 I stalo sie, gdy sie on sam osobno modlil, ze z nim byli uczniowie; i pytal ich mówiac: Kimze mie byc powiadaja ludzie?
\par 19 A oni odpowiadajac rzekli: Janem Chrzcicielem, a drudzy Elijaszem, a drudzy mówia, iz prorok niektóry z onych starych zmartwychwstal.
\par 20 I rzekl im: A wy kim mie byc powiadacie? A odpowiadajac Piotr rzekl: Chrystusem, onym Bozym.
\par 21 Ale on przygroziwszy im, rozkazal, aby tego nikomu nie powiadali,
\par 22 Mówiac: Ze Syn czlowieczy musi wiele cierpiec, a byc odrzuconym od starszych ludu i od przedniejszych kaplanów i od nauczonych w Pismie, i byc zabitym, a trzeciego dnia zmartwychwstac.
\par 23 I mówil do wszystkich: Kto chce za mna isc, niech zaprze samego siebie, a niech bierze krzyz swój na kazdy dzien, i nasladuje mie.
\par 24 Albowiem ktobykolwiek chcial zachowac dusze swoje, straci ja; a ktobykolwiek stracil dusze swa dla mnie, ten ja zachowa.
\par 25 Albowiem cóz pomoze czlowiekowi, chocby wszystek swiat pozyskal, gdyby samego siebie stracil, albo sobie uszkodzil.
\par 26 Albowiem ktokolwiek by sie wstydzil za mie i za slowa moje, za tego sie Syn czlowieczy wstydzic bedzie, gdy przyjdzie w chwale swej i w ojcowskiej i swietych Aniolów.
\par 27 Alec wam powiadam prawdziwie: Sa niektórzy z tych co tu stoja, którzy nie ukusza smierci, az ogladaja królestwo Boze.
\par 28 I stalo sie po tych mowach, jakoby po osmiu dniach, ze wziawszy z soba Piotra i Jana i Jakóba, wstapil na góre, aby sie modlil.
\par 29 A gdy sie modlil, stal sie inakszy ksztalt oblicza jego, i szaty jego staly sie biale i swietne.
\par 30 A oto dwaj mezowie rozmawiali z nim, a ci byli Mojzesz i Elijasz;
\par 31 Którzy pokazawszy sie w slawie, powiadali o jego smierci, która mial podstapic w Jeruzalemie.
\par 32 A Piotr i ci, którzy byli z nim, obciazeni byli snem, a ocuciwszy sie, ujrzeli chwale jego i onych dwóch mezów, którzy z nim stali.
\par 33 I stalo sie, gdy oni odeszli od niego, rzekl Piotr do Jezusa: Mistrzu! dobrze nam tu byc; przetoz uczynmy trzy namioty, tobie jeden i Mojzeszowi jeden i Elijaszowi jeden; nie wiedzac, co mówil.
\par 34 A gdy on to mówil, stal sie oblok, i zacienil je; i bali sie, gdy oni wchodzili w oblok.
\par 35 I stal sie glos z obloku mówiacy: Ten jest Syn mój mily, tego sluchajcie.
\par 36 A gdy sie stal on glos, znaleziony jest sam Jezus. A oni milczeli, i nie powiadali w one dni nikomu nic z tego, co widzieli.
\par 37 I stalo sie nazajutrz, gdy oni zstapili z góry, ze mu lud wielki zabiezal.
\par 38 A oto maz z onego ludu zawolal, mówiac: Nauczycielu! prosze cie, wejrzyj na syna mego; boc jedynego mam.
\par 39 A oto duch zly popada go, a zaraz krzyczy, a on rozdziera go, sliniacego sie, a zaledwie odchodzi od niego, skruszywszy go.
\par 40 I prosilem uczniów twoich, aby go wygnali; ale nie mogli.
\par 41 Tedy Jezus odpowiadajac rzekl: O rodzaju niewierny i przewrotny! dokadze z wami bede, i dokadze was cierpiec bede? Przywiedz tu syna twego.
\par 42 A wtem, gdy on przychodzil, rozdarl go dyjabel i roztargal; ale Jezus zgromil ducha nieczystego i uzdrowil mlodzienca, i oddal go ojcu jego.
\par 43 I zdumieli sie wszyscy nad wielmoznoscia Boza. A gdy sie wszyscy dziwowali wszystkim rzeczom, które czynil Jezus, rzekl do uczniów swoich:
\par 44 Skladajcie wy do uszów waszych slowa te: albowiem Syn czlowieczy ma byc wydany w rece ludzkie.
\par 45 Lecz oni nie rozumieli slowa tego, i bylo zakryte od nich, ze go pojac nie mogli, i nie smieli go pytac o to slowo.
\par 46 I wszczela sie gadka miedzy nimi, kto by z nich byl najwiekszym.
\par 47 A Jezus widzac mysl serca ich, wziawszy dziecie, postawil je podle siebie,
\par 48 I rzekl im: Ktobykolwiek przyjal to dzieciatko w imieniu mojem, mnie przyjmuje; a ktobykolwiek mnie przyjal, przyjmuje onego, który mie poslal: albowiem kto jest najmniejszy miedzy wszystkimi wami, tenci bedzie wielkim.
\par 49 A Jan odpowiadajac, rzekl: Mistrzu! widzielismy niektórego w imieniu twojem dyjably wyganiajacego, i zabranialismy mu, przeto ze za toba z nami nie chodzi.
\par 50 I rzekl do niego Jezus: Nie zabraniajcie mu; bo kto nie jest przeciwko nam, za nami jest.
\par 51 I stalo sie, gdy sie wypelnily dni, aby byl wziety w góre, ze sie on na to udal, aby szedl do Jeruzalemu.
\par 52 Tedy poslal posly przed soba; którzy weszli do miasteczka Samarytanskiego, aby mu nagotowali gospode.
\par 53 Lecz oni go nie przyjeli, przeto ze oblicze jego obrócone bylo do Jeruzalemu.
\par 54 A widzac to uczniowie jego, Jakób i Jan, rzekli: Panie! chceszze, iz rzeczemy, aby ogien zstapil z nieba i spalil je, jako i Elijasz uczynil?
\par 55 Ale Jezus obróciwszy sie, zgromil je, i rzekl: Nie wiecie wy, jakiego jestescie ducha.
\par 56 Albowiem Syn czlowieczy nie przyszedl, zatracac dusz ludzkich, ale zachowac. I szli do inszego miasteczka.
\par 57 I stalo sie, gdy oni szli, ze w drodze rzekl niektóry do niego: Pójde za toba, gdziekolwiek pójdziesz, Panie!
\par 58 A Jezus mu rzekl: Liszki maja jamy, i ptaszki niebieskie gniazda; ale Syn czlowieczy nie ma, gdzie by glowe sklonil.
\par 59 I rzekl do drugiego: Pójdz za mna! Ale on rzekl: Panie! dopusc mi pierwej odejsc i pogrzesc ojca mego.
\par 60 Ale mu Jezus rzekl: Niechaj umarli grzebia umarlych swoich; a ty poszedlszy, opowiadaj królestwo Boze.
\par 61 Rzekl tez i drugi: Pójde za toba, Panie! ale mi pierwej dopusc pozegnac sie z tymi, którzy sa w domu moim.
\par 62 Rzekl do niego Jezus: Zaden, który by przylozyl reke swoje do pluga, a ogladalby sie nazad, nie jest sposobny do królestwa Bozego.

\chapter{10}

\par 1 A potem naznaczyl Pan i drugich siedmdziesiat, i rozeslal je po dwóch przed obliczem swojem do kazdego miasta i miejsca, do którego przyjsc mial.
\par 2 I mówil im: Zniwoc wprawdzie wielkie, ale robotników malo; prosciez tedy Pana zniwa, aby wypchnal robotników na zniwo swoje.
\par 3 Idzciez: Oto ja was posylam jako baranki wposród wilków.
\par 4 Nie nosciez mieszka, ani taistry, ani obuwia, i nikogo w drodze nie pozdrawiajcie;
\par 5 A do któregokolwiek domu wnijdziecie, naprzód mówcie: Pokój temu domowi.
\par 6 A jezliby tam byl który syn pokoju, odpocznie nad nim pokój wasz; a jezliz nie, wróci sie do was.
\par 7 A w tymze domu zostancie, jedzac i pijac to, co maja; albowiem godzien jest robotnik zaplaty swojej; nie przechodzcie sie z domu do domu.
\par 8 A do któregokolwiek miasta weszlibyscie, a przyjeliby was, jedzcie, co przed was poloza;
\par 9 I uzdrawiajcie niemocnych, którzy by w niem byli, a mówcie im: Przyblizylo sie do was królestwo Boze.
\par 10 A do któregobysciekolwiek miasta weszli, a nie przyjeto by was, wyszedlszy na ulice jego, mówcie:
\par 11 I proch, który przylgnal do nas z miasta waszego, otrzasamy na was; wszakze to wiedzcie, ze sie do was przyblizylo królestwo Boze.
\par 12 A mówie wam: Iz Sodomie w on dzien lzej bedzie, niz onemu miastu.
\par 13 Biada tobie, Chorazynie! biada tobie, Betsaido! bo gdyby sie byly w Tyrze i w Sydonie te cuda staly, które sie staly w was, dawno by byly w worze i w popiele siedzac pokutowaly.
\par 14 Dlatego Tyrowi i Sydonowi lzej bedzie na sadzie, nizeli wam.
\par 15 A ty, Kapernaum! któres az do nieba wywyzszone, az do piekla stracone bedziesz.
\par 16 Kto was slucha, mnie slucha: a kto wami gardzi, mna gardzi; a kto mna gardzi, gardzi onym, który mie poslal.
\par 17 A tak wrócili sie oni siedmdziesiat z weselem, mówiac: Panie! i dyjablic sie nam poddawaja w imieniu twojem.
\par 18 Tedy im rzekl: Widzialem szatana, jako blyskawice z nieba spadajacego.
\par 19 Oto wam daje moc, abyscie deptali po wezach i po niedzwiadkach i po wszystkiej mocy nieprzyjacielskiej, a nic wam nie uszkodzi.
\par 20 Wszakze nie radujcie sie z tego, iz sie wam duchy poddawaja; ale raczej radujcie sie, ze imiona wasze napisane sa w niebiesiech.
\par 21 Onejze godziny rozradowal sie Jezus w duchu, i rzekl: Wyslawiam cie, Ojcze, Panie nieba i ziemi! zes te rzeczy zakryl przed madrymi i roztropnymi, a objawiles je niemowlatkom; zaprawde, Ojcze! ze sie tak upodobalo tobie.
\par 22 Wszystkie rzeczy dane mi sa od Ojca mego, a nikt nie zna, kto jest Syn, tylko Ojciec, i kto jest Ojciec, tylko Syn, a komu by chcial Syn objawic.
\par 23 Tedy obróciwszy sie do uczniów, rzekl im z osobna: Blogoslawione oczy, które widza, co wy widzicie.
\par 24 Bo powiadam wam, iz wiele proroków i królów zadali widziec, co wy widzicie, ale nie widzieli; i slyszec, co wy slyszycie, ale nie slyszeli.
\par 25 A oto niektóry zakonnik powstal, kuszac go i mówiac: Nauczycielu! co czyniac odziedzicze zywot wieczny?
\par 26 A on rzekl do niego: W zakonie co napisano, jako czytasz?
\par 27 A on odpowiadajac rzekl: Bedziesz milowal Pana, Boga twego, ze wszystkiego serca twego, i ze wszystkiej duszy twojej, i ze wszystkiej sily twojej, i ze wszystkiej mysli twojej; a blizniego twego, jako samego siebie.
\par 28 I rzekl mu: Dobrzes odpowiedzial; to czyn, a bedziesz zyl.
\par 29 A on chcac samego siebie usprawiedliwic, rzekl do Jezusa: I któz jest mi blizni?
\par 30 Ale Jezus odpowiadajac rzekl: Czlowiek niektóry zstepowal z Jeruzalemu do Jerycha, i wpadl miedzy zbójców, którzy zlupiwszy go i rany mu zadawszy, odeszli, na pól umarlego zostawiwszy.
\par 31 I przydalo sie, ze kaplan niektóry szedl ta droga, a ujrzawszy go, pominal.
\par 32 Takze i Lewita, dostawszy sie na ono miejsce, a przyszedlszy i ujrzawszy go, pominal.
\par 33 Ale Samarytanin niektóry jadac, przyjechal do niego, a ujrzawszy, uzalil sie go.
\par 34 A przystapiwszy zawiazal rany jego, a nalawszy oliwy i wina, i wlozywszy go na bydle swoje, wiódl go do gospody, i mial staranie o nim.
\par 35 A nazajutrz odjezdzajac, wyjal dwa grosze, i dal gospodarzowi, mówiac: Miej o nim staranie, a cokolwiek nadto wylozysz, ja, gdy sie wróce, oddam ci.
\par 36 Któryz tedy z tych trzech zda sie tobie bliznim byc onemu, co byl wpadl miedzy zbójców?
\par 37 A on rzekl: Ten, który uczynil milosierdzie nad nim. Rzekl mu tedy Jezus: Idzze, i ty uczyn takze.
\par 38 I stalo sie, gdy oni szli, ze on wszedl do niektórego miasteczka, a niewiasta niektóra, imieniem Marta, przyjela go do domu swego.
\par 39 A ta miala siostre, która zwano Maryja, która usiadlszy u nóg Jezusowych, sluchala slów jego.
\par 40 Ale Marta roztargniona byla okolo rozmaitej poslugi; która przystapiwszy, rzekla: Panie! i nie dbasz, ze siostra moja mnie sama zostawila, abym poslugiwala? Rzeczze jej, aby mi pomogla.
\par 41 A odpowiadajac Jezus rzekl jej: Marto, Marto! troszczysz sie i klopoczesz sie okolo wielu rzeczy;
\par 42 Alec jednego potrzeba. Lecz Maryja dobra czastke obrala, która od niej odjeta nie bedzie.

\chapter{11}

\par 1 I stalo sie, gdy on byl na niektórem miejscu, modlac sie, ze gdy przestal, rzekl do niego jeden z uczniów jego, Panie! naucz nas modlic sie, tak jako i Jan nauczyl uczniów swoich.
\par 2 I rzekl im: Gdy sie modlicie, mówcie: Ojcze nasz, którys jest w niebiesiech! Swiec sie imie twoje; przyjdz królestwo twoje; badz wola twoja, jako w niebie tak i na ziemi.
\par 3 Chleba naszego powszedniego daj nam na kazdy dzien.
\par 4 I odpusc nam grzechy nasze; bo tez i my odpuszczamy kazdemu winowajcy naszemu. A nie wwódz nas na pokuszenie, ale nas zbaw od zlego.
\par 5 Zatem rzekl do nich: Któz z was miec bedzie przyjaciela, i pójdzie do niego o pólnocy i rzecze mu: Przyjacielu! pozycz mi trzech chlebów;
\par 6 Albowiem przyjaciel mój przyszedl z drogi do mnie, a nie mam, co przed niego polozyc.
\par 7 A on bedac w domu, odpowiedzialby mówiac: Nie uprzykrzaj mi sie; juz sa drzwi zamkniete, a dziatki moje sa ze mna w pokoju; nie moge wstac, abym ci dal.
\par 8 Powiadam wam: Chociazby mu nie dal wstawszy, przeto ze jest przyjacielem jego, wszakze dla niewstydliwego nalegania jego wstawszy, da mu, ile potrzebuje.
\par 9 I jac wam powiadam: Proscie, a bedzie wam dano; szukajcie, a znajdziecie; kolaczcie, a bedzie wam otworzono.
\par 10 Kazdy bowiem, kto prosi, bierze, a kto szuka, znajduje, a temu, co kolacze, bedzie otworzono.
\par 11 A któryz jest z was ojciec, którego gdyby prosil syn o chleb, izali mu da kamien? Albo prosilby o rybe, izali mu zamiast ryby da weza?
\par 12 Albo prosilliby o jaje, izali mu da niedzwiadka?
\par 13 Poniewaz tedy wy, bedac zlymi, umiecie dobre dary dawac dzieciom waszym: jakoz daleko wiecej Ojciec wasz niebieski da Ducha Swietego tym, którzy go on prosza?
\par 14 Tedy wyganial dyjabla, który byl niemy. I stalo sie, gdy wyszedl on dyjabel, przemówil niemy; i dziwowal sie lud.
\par 15 Ale niektórzy z nich mówili: Przez Beelzebuba, ksiazecia dyjabelskiego, wygania dyjably.
\par 16 Drudzy zasie kuszac go, zadali znamienia od niego z nieba.
\par 17 Ale on widzac mysli ich, rzekl im: Kazde królestwo rozdzielone samo przeciwko sobie pustoszeje, a dom na dom upada.
\par 18 A jezlizec i szatan rozdzielony jest przeciwko sobie, jakoz sie ostoi królestwo jego? albowiem powiadacie, iz ja przez Beelzebuba wyganiam dyjably.
\par 19 A jezliz ja przez Beelzebuba wyganiam dyjably, synowie wasi przez kogoz wyganiaja? Przetoz oni beda sedziami waszymi.
\par 20 Ale jezliz ja palcem Bozym wyganiam dyjably, zaistec przyszlo do was królestwo Boze.
\par 21 Gdy mocarz uzbrojony strzeze palacu swego, w pokoju sa majetnosci jego;
\par 22 Ale gdy mocniejszy naden nadszedlszy, zwyciezy go, odejmuje wszystko oreze jego, w którem ufal, a lupy jego rozdaje.
\par 23 Kto nie jest ze mna, przeciwko mnie jest; a kto nie zbiera ze mna, rozprasza.
\par 24 Gdy duch nieczysty wychodzi od czlowieka, przechadza sie po miejscach suchych, szukajac odpocznienia, a nie znalazlszy, mówi: Wróce sie do domu mego, skadem wyszedl.
\par 25 A przyszedlszy znajduje umieciony i ochedozony.
\par 26 Tedy idzie i bierze z soba siedm innych duchów gorszych nizeli sam, a wszedlszy mieszkaja tam, i bywaja rzeczy ostatnie czlowieka onego gorsze, nizeli pierwsze.
\par 27 I stalo sie, gdy on to mówil, ze wynióslszy glos niektóra niewiasta z ludu, rzekla mu: Blogoslawiony zywot, który cie nosil, i piersi, któres ssal!
\par 28 Ale on rzekl: Owszem blogoslawieni sa, którzy sluchaja slowa Bozego i strzega go.
\par 29 A gdy sie lud gromadzil, poczal mówic: Rodzaj ten rodzaj zly jest; znamienia szuka, ale mu znamie nie bedzie dane, tylko ono znamie Jonasza proroka.
\par 30 Albowiem jako Jonasz byl za znamie Niniwczykom, tak bedzie i Syn czlowieczy temu rodzajowi.
\par 31 Królowa z poludnia stanie na sadzie z mezami rodzaju tego, i potepi je; bo przyjechala od konczyn ziemi, aby sluchala madrosci Salomonowej; a oto tu wiecej, nizeli Salomon.
\par 32 Mezowie Niniwiccy stana na sadzie z tym rodzajem i potepia go, przeto ze pokutowali na kazanie Jonaszowe; a oto tu wiecej, nizeli Jonasz.
\par 33 A nikt swiece zapaliwszy, nie stawia jej w skrytosci, ani pod korzec, ale na swiecznik, aby ci, którzy wchodza, swiatlo widzieli.
\par 34 Swieca ciala jest oko; jezliby tedy oko twoje bylo szczere, i cialo twoje wszystko bedzie jasne; a jezliby zle bylo, i cialo twoje ciemne bedzie.
\par 35 Patrzajze tedy, aby swiatlo, które jest w tobie, nie bylo ciemnoscia.
\par 36 Jezli tedy wszystko cialo twoje jasne bedzie, nie majac jakiej czastki zacmionej, bedziec wszystko tak jasne, ze cie jako swieca blaskiem oswieci.
\par 37 A gdy to mówil, prosil go niektóry Faryzeusz, aby jadl obiad u niego; wszedlszy tedy, usiadl za stolem.
\par 38 A widzac to Faryzeusz, dziwowal sie, ze sie nie umyl przed obiadem.
\par 39 I rzekl Pan do niego: Teraz wy, Faryzeuszowie! to, co jest zewnatrz kubka i misy, ochedazacie, ale to, co jest wewnatrz w was, pelne jest drapiestwa i zlosci.
\par 40 Szaleni! izaz ten, który uczynil to, co jest zewnatrz, nie uczynil tez tego, co jest wewnatrz?
\par 41 Wszakze i z tego, co jest wewnatrz, dawajcie jalmuzne, a oto wszystkie rzeczy beda wam czyste.
\par 42 Ale biada wam, Faryzeuszowie! ze dajecie dziesiecine z miety, i z ruty, i z kazdego ziela, lecz opuszczacie sad i milosc Boza: tec rzeczy trzeba czynic, a onych nie opuszczac.
\par 43 Biada wam Faryzeuszowie! ze milujecie pierwsze miejsca w bóznicach i pozdrawiania na rynkach.
\par 44 Biada wam, nauczeni w Pismie i Faryzeuszowie obludni! bo jestescie jako groby, których nie widac, a ludzie, którzy chodza po nich, nie wiedza o nich.
\par 45 A odpowiadajac niektóry z zakonników, rzekl mu: Nauczycielu! to mówiac i nas hanbisz.
\par 46 A on rzekl: I wam zakonnikom biada! albowiem obciazacie ludzi brzemiony nieznosnemi, a sami sie i jednym palcem swoim tych brzemion nie dotykacie.
\par 47 Biada wam! ze budujecie groby proroków, a ojcowie wasi pozabijali je.
\par 48 Zaiste swiadczycie, iz sie kochacie w uczynkach ojców waszych; albowiem oni je pozabijali, a wy budujecie groby ich.
\par 49 Dlategoz tez madrosc Boza rzekla: Posle do nich proroki i Apostoly, a z nich niektóre zabijac i przesladowac beda;
\par 50 Aby szukano od tego rodzaju krwi wszystkich proroków, która wylana jest od zalozenia swiata,
\par 51 Od krwi Abla az do krwi Zacharyjasza, który zginal miedzy oltarzem, i kosciolem; zaiste powiadam wam, beda jej szukac od rodzaju tego.
\par 52 Biada wam zakonnikom! boscie wzieli klucz umiejetnosci; samiscie nie weszli, a tym, którzy wnijsc chcieli, zabranialiscie.
\par 53 A gdy im to mówil, poczeli nan nauczeni w Pismie i Faryzeuszowie bardzo nacierac, i przyczyne mu dawac do mówienia o wielu rzeczach;
\par 54 Czyhajac nan i szukajac, aby co uchwycili z ust jego, zeby go oskarzyli.

\chapter{12}

\par 1 Miedzy tem, gdy sie zgromadzilo wiele tysiecy ludu, tak iz jedni po drugich deptali, poczal mówic do uczniów swoich. Naprzód strzezcie sie kwasu Faryzejskiego, który jest obluda.
\par 2 Boc nie jest nic skrytego, co by objawione byc nie mialo, ani tajemnego czego by sie dowiedziec nie miano.
\par 3 Przetoz, coscie mówili w ciemnosci, na swietle slyszane bedzie, a coscie w ucho szeptali w komorach, obwolane bedzie na dachach.
\par 4 A mówie wam przyjaciolom moim: Nie bójcie sie tych, którzy cialo zabijaja, a potem nie maja co by wiecej uczynili.
\par 5 Ale wam okaze, kogo sie bac macie: Bójcie sie tego, który, gdy zabije, ma moc wrzucic do piekielnego ognia; zaiste powiadam wam, tego sie bójcie.
\par 6 Izali pieciu wróblików nie sprzedaja za dwa pieniazki? Wszakze jeden z nich nie jest w zapamietaniu przed obliczem Bozem.
\par 7 Owszem i wlosy glowy waszej wszystkie sa policzone. Przetoz sie nie bójcie, nad wiele wróblików wy jestescie zacniejsi.
\par 8 A mówie wam: Wszelaki, który by mie wyznal przed ludzmi, i Syn czlowieczy wyzna go przed Anioly Bozymi.
\par 9 Ale kto by sie mie zaprzal przed ludzmi, zapre sie go przed Anioly Bozymi.
\par 10 I kazdemu, kto by mówil slowo przeciwko Synowi czlowieczemu, bedzie mu odpuszczone: ale temu, kto by przeciwko Duchowi Swietemu bluznil, nie bedzie odpuszczone.
\par 11 A gdy was beda wodzic do bóznic, i do przelozonych, i do zwierzchnosci, nie troszczcie sie, jako i co byscie ku obronie odpowiedziec, albo co byscie mówic mieli.
\par 12 Albowiem Duch Swiety nauczy was onejze godziny, co byscie mówic mieli.
\par 13 I rzekl mu niektóry z ludu: Nauczycielu! rzecz bratu memu, aby sie ze mna podzielil dziedzictwem.
\par 14 Ale mu on rzekl: Czlowiecze! któz mie postanowil sedzia albo dzielca miedzy wami?
\par 15 I rzekl do nich: Patrzcie, a strzezcie sie lakomstwa, gdyz nie w tem, ze kto ma obfite majetnosci, zywot jego zalezy.
\par 16 I powiedzial im podobienstwo, mówiac: Niektórego czlowieka bogatego pole obfity urodzaj przynioslo.
\par 17 I rozmyslal sam w sobie, mówiac: Cóz uczynie, gdyz nie mam, gdzie bym zgromadzil urodzaje moje?
\par 18 I rzekl: To uczynie: Rozwale gumna moje, a wieksze pobuduje i zgromadze tam wszystkie urodzaje moje i dobra moje;
\par 19 I rzeke do duszy mojej: Duszo! masz wiele dóbr zlozonych na wiele lat; odpocznijze, jedz, pij, badz dobrej mysli.
\par 20 Ale mu rzekl Bóg: O glupi, tej nocy upomne sie duszy twojej od ciebie, a to, cos nagotowal, czyjez bedzie?
\par 21 Takci jest, który sobie skarbi, a nie jest w Bogu bogaty.
\par 22 I rzekl do uczniów swoich: Dlatego powiadam wam, nie troszczcie sie o zywot wasz, co byscie jedli, ani o cialo, czem byscie sie przyodziewali.
\par 23 Zacniejszy jest zywot, niz pokarm, a cialo, niz odzienie.
\par 24 Przypatrzcie sie krukom, iz nie sieja ani zna, i nie maja spizarni, ani gumna, a wzdy je Bóg zywi; czemzescie wy zacniejsi niz ptacy?
\par 25 I któz z was troskliwie myslac, moze przydac do wzrostu swego lokiec jeden?
\par 26 Poniewaz tedy i najmniejszej rzeczy nie przemozecie, czemuz sie o inne troszczycie?
\par 27 Przypatrzcie sie lilijom, jako rosna, nie pracuja, ani przeda; a powiadam wam, ze ani Salomon we wszystkiej slawie swojej nie byl tak przyodziany, jako jedna z tych.
\par 28 A jezlize trawe, która dzis jest na polu, a jutro bedzie w piec wrzucona, Bóg tak przyodziewa, jakoz daleko wiecej was, o malowierni!
\par 29 Wy tedy nie pytajcie sie, co byscie jesc, albo co byscie pic mieli, ani wysoko latajcie myslami waszemi.
\par 30 Albowiem tego wszystkiego narody swiata szukaja; alec Ojciec wasz wie, ze tego potrzebujecie.
\par 31 Owszem szukajcie królestwa Bozego, a to wszystko bedzie wam przydane.
\par 32 Nie bój sie, o maluczkie stadko! albowiem sie upodobalo Ojcu waszemu, dac wam królestwo.
\par 33 Sprzedawajcie majetnosci wasze, a dawajcie jalmuzne; gotujcie sobie mieszki, które nie wiotszeja, skarb, którego nie ubywa w niebiesiech, gdzie zlodziej przystepu nie ma, ani mól psuje.
\par 34 Bo gdzie jest skarb wasz, tam bedzie i serce wasze.
\par 35 Niech beda przepasane biodra wasze, i swiece zapalone.
\par 36 A wy badzcie podobni ludziom oczekujacym pana swego, azeby sie wrócil z wesela, zeby gdyby przyszedl, a zakolatal, wnet mu otworzyli.
\par 37 Blogoslawieni oni sludzy, których gdy przyjdzie pan, czujacych znajdzie; zaprawde powiadam wam, iz sie przepasze, a posadzi ich za stól, a przechadzajac sie, bedzie im sluzyl.
\par 38 A jezliby przyszedl o wtórej strazy, i o trzeciej strazy przyszedlliby, a tak by ich znalazl, blogoslawieni sa oni sludzy.
\par 39 A to wiedzcie, izby gdyby wiedzial gospodarz, o której godzinie zlodziej ma przyjsc, wzdyby czul, a nie dopuscilby podkopac domu swego.
\par 40 Przetoz i wy badzcie gotowi; bo o tej godzinie, o której sie nie spodziewacie, Syn czlowieczy przyjdzie.
\par 41 I rzekl mu Piotr: Panie! do nasze mówisz to podobienstwo, czyli do wszystkich?
\par 42 A Pan rzekl: Któryz tedy jest wierny szafarz i roztropny, którego Pan postanowi nad czeladzia swoja, aby im na czas wydawal obrok naznaczony?
\par 43 Blogoslawiony jest on sluga, którego gdyby przyszedl pan jego, znajdzie, ze tak czyni;
\par 44 Zaprawde wam powiadam, ze go nad wszystkiemi dobrami swojemi postanowi.
\par 45 Ale jezliby rzekl on sluga w sercu swojem: Odwlacza pan mój z przyjsciem swojem, i poczalby bic slugi i sluzebnice, a jesc, pic i opijac sie;
\par 46 Przyjdzie pan slugi onego dnia, którego sie nie spodzieje, i godziny, której nie wie, i odlaczy go, a czesc jego polozy z niewiernymi.
\par 47 Ten zasie sluga, który by znal wole pana swego, a nie byl gotowym, ani czynil wedlug woli jego, wielce bedzie karany;
\par 48 Ale który nie znal, a czynil rzeczy godne karania, mniej plag odniesie; a od kazdego, komu wiele dano, wiele sie od niego upominac beda: a komu wiele powierzono, wiecej beda chciec od niego.
\par 49 Przyszedlem, abym ogien puscil na ziemie, i czegoz chce, jezli juz gore?
\par 50 Alec mam byc chrztem ochrzczony; a jakom jest scisniony, póki sie to nie wykona.
\par 51 Mniemacie, abym przyszedl, pokój dawac na ziemie? Bynajmniej, powiadam wam, ale rozerwanie.
\par 52 Albowiem od tego czasu bedzie ich piec w domu jednym rozerwanych, trzej przeciwko dwom, a dwaj przeciwko trzem.
\par 53 Powstanie ojciec przeciwko synowi, a syn przeciwko ojcu, matka przeciwko córce, a córka przeciwko matce, swiekra przeciwko synowej swojej, a synowa przeciwko swiekrze swojej.
\par 54 Mówil tez i do ludu: Gdy widzicie oblok wschodzacy od zachodu, zaraz mówicie: Przychodzi gwaltowny deszcz; i tak bywa.
\par 55 A gdy wiatr wiejacy od poludnia, mówicie: Goraco bedzie; i bywa tak.
\par 56 Obludnicy! postawe nieba i ziemi rozeznawac umiecie, a tego czasu jakoz nie poznawacie?
\par 57 Przeczze i sami przez sie nie sadzicie, co jest sprawiedliwego?
\par 58 Gdy tedy idziesz z przeciwnikiem swoim przed urzad, starajze sie w drodze, abys byl wolen, by cie snac nie pociagnal przed sedziego, a sedzia by cie podal ceklarzowi, a ceklarz by cie wrzucil do wiezienia.
\par 59 Powiadam ci: Nie wynijdziesz stamtad, póki bys nie oddal do ostatniego pieniazka.

\chapter{13}

\par 1 A prawie natenczas byli przytomni niektórzy, oznajmujac mu o Galilejczykach, których krew Pilat pomieszal z ofiarami ich.
\par 2 A Jezus odpowiadajac, rzekl im: Mniemacie, ze ci Galilejczycy nad wszystkie inne Galilejczyki grzeszniejszymi byli, iz takowe rzeczy ucierpieli?
\par 3 Bynajmniej, mówie wam: i owszem, jezli nie bedziecie pokutowac, wszyscy takze poginiecie.
\par 4 Albo osmnascie onych, na które upadla wieza w Syloe i pobila je, mniemacie zeby ci winniejszymi byli nad wszystkie ludzie mieszkajace w Jeruzalemie?
\par 5 Bynajmniej, mówie wam: i owszem, jezli pokutowac nie bedziecie, wszyscy takze poginiecie.
\par 6 I powiedzial im to podobienstwo: Czlowiek niektóry mial figowe drzewo wsadzone na winnicy swojej, a przyszedlszy, szukal na niem owocu, ale nie znalazl.
\par 7 Tedy rzekl do winiarza: Oto po trzy lata przychodze, szukajac owocu na tem drzewie figowem, ale nie znajduje. Wytnijze je; bo przeczze te ziemie prózno zastepuje?
\par 8 Ale on odpowiadajac rzekl mu: Panie! zaniechaj go jeszcze i na ten rok, az je okopie i obloze gnojem;
\par 9 Owa snac przyniesie owoc, a jezli nie, potem je wytniesz.
\par 10 I nauczal w jednej bóznicy w sabat.
\par 11 A oto byla tam niewiasta, która miala ducha niemocy osmnascie lat, a byla skurczona, tak iz sie zadna miara nie mogla rozprostowac.
\par 12 Te gdy ujrzal Jezus, zawolal jej do siebie i rzekl: Niewiasto! uwolnionas od niemocy twojej.
\par 13 I wlozyl na nia rece, a zarazem rozprostowala sie i chwalila Boga.
\par 14 Tedy odpowiadajac przelozony nad bóznica, który sie bardzo gniewal, ze Jezus w sabat uzdrawial, rzekl do ludu: Szesc dni jest, w które trzeba robic; w te tedy dni przychodzac, leczcie sie, a nie w dzien sabatu.
\par 15 Ale mu odpowiedzial Pan i rzekl: Obludniku, azaz kazdy z was w sabat nie odwiazuje wolu swego, albo osla swego od zlobu, a nie wiedzie, zeby go napoil?
\par 16 A ta córka Abrahamowa, która byl zwiazal szatan oto juz osmnascie lat, zaz nie miala byc rozwiazana od tej zwiazki w dzien sabatu?
\par 17 A gdy on to mówil, zawstydzili sie wszyscy przeciwnicy jego: ale wszystek lud radowal sie ze wszystkich onych chwalebnych spraw, które sie dzialy od niego.
\par 18 Zatem rzekl Jezus: Komuz podobne jest królestwo Boze, a do czegoz je przypodobam?
\par 19 Podobne jest ziarnu gorczycznemu, które wziawszy czlowiek, wrzucil do ogrodu swego; i roslo i stalo sie drzewem wielkiem, a ptaszki niebieskie czynily sobie gniazda na galeziach jego.
\par 20 I rzekl znowu: Do czegoz przypodobam królestwo Boze?
\par 21 Podobne jest kwasowi, który wziawszy niewiasta, zakryla go we trzy miary maki, azby wszystko skwasnialo.
\par 22 I chodzil po miastach i miasteczkach, nauczajac a idac w droge do Jeruzalemu.
\par 23 I rzekl mu niektóry: Panie! czyli malo tych jest, którzy maja byc zbawieni? A on rzekl do nich:
\par 24 Usilujcie, abyscie weszli przez ciasna brame; albowiem powiadam wam: Wiele ich beda chcieli wnijsc, ale nie beda mogli.
\par 25 Gdy wstanie gospodarz i zamknie drzwi, a poczniecie stac przede drzwiami, i kolatac we drzwi, mówiac: Panie, Panie! otwórz nam, tedy on odpowiadajac rzecze wam: Nie znam was, skad jestescie.
\par 26 Tedy poczniecie mówic: Jadalismy przed toba i pijali, i uczyles na ulicach naszych.
\par 27 A on rzecze: Powiadam wam, nie znam was, skad jestescie; odstapcie ode mnie wszyscy, którzy czynicie nieprawosc.
\par 28 Tam bedzie placz i zgrzytanie zebów, gdy ujrzycie Abrahama, Izaaka, i Jakóba, i wszystkie proroki w królestwie Bozem, a samych siebie precz wyrzuconych.
\par 29 I przyjda drudzy od wschodu i od zachodu, i od pólnocy, i od poludnia, a usiada za stolem w królestwie Bozem.
\par 30 A oto sa ostatni, którzy beda pierwszymi, a sa pierwsi, którzy beda ostatnimi.
\par 31 W onze dzien przystapili niektórzy z Faryzeuszów, mówiac mu: Wynijdz, a idz stad; bo cie Herod chce zabic.
\par 32 I rzekl im: Idzcie, a powiedzcie temu lisowi: Oto wyganiam dyjably, i uzdrawiam dzis i jutro, a trzeciego dnia dokonczenie wezme.
\par 33 Wszakze musze dzis i jutro i pojutrze odprawiac droge: albowiem nie moze byc, aby mial prorok zginac, oprócz w Jeruzalemie.
\par 34 Jeruzalem! Jeruzalem! które zabijasz proroki, a kamionujesz te, którzy do ciebie bywaja poslani; ilekroc chcialem zgromadzic dzieci twoje, tak jako kokosz zgromadza kurczeta swoje pod skrzydla, a nie chcieliscie.
\par 35 Otoz zostanie wam dom wasz pusty. A zaprawdec wam powiadam, ze mie nie ujrzycie, az przyjdzie czas, gdy rzeczecie: Blogoslawiony, który idzie w imieniu Panskiem.

\chapter{14}

\par 1 I stalo sie, gdy wszedl Jezus w dom niektórego przedniejszego Faryzeusza w sabat, aby jadl chleb, ze go oni podstrzegali.
\par 2 A oto czlowiek niektóry opuchly byl przed nim.
\par 3 A odpowiadajac Jezus, rzekl do zakonników, i do Faryzeuszów, mówiac: Godzili sie w sabat uzdrawiac?
\par 4 A oni milczeli. Tedy on ujawszy go, uzdrowil i odprawil.
\par 5 A odpowiadajac rzekl do nich: Któregoz z was osiel albo wól wpadnie w studnie, a nie wnet go wyciagnie w dzien sabatu?
\par 6 I nie mogli mu na to odpowiedziec.
\par 7 Powiedzial tez i wezwanym podobienstwo, (baczac, jako przedniejsze miejsca obierali,)mówiac do nich:
\par 8 Gdybys byl od kogo wezwany na wesele, nie siadajze na przedniejszem miejscu, by snac zacniejszy nad cie nie byl wezwany od niego;
\par 9 A przyszedlszy ten, który ciebie i onego wezwal, rzeklby tobie: Daj temu miejsce: a tedy bys ze wstydem poczal siedziec na posledniem miejscu.
\par 10 Ale gdybys byl wezwany, szedlszy, usiadz na posledniem miejscu; a gdyby przyszedl ten, który cie wezwal, rzeklby tobie: Przyjacielu! posiadz sie wyzej; tedy bedziesz mial czesc przed spólsiedzacymi z toba.
\par 11 Bo wszelki, kto sie wywyzsza, ponizony bedzie, a kto sie poniza, wywyzszony bedzie.
\par 12 Mówil tez i onemu, który go byl wezwal: Gdy sprawujesz obiad albo wieczerze, nie wzywajze przyjaciól twoich, ani braci twoich, ani krewnych twoich, ani sasiadów bogatych, zeby cie snac i oni zasie nie wezwali, a stalaby ci sie nagroda.
\par 13 Ale gdy sprawujesz uczte, wezwijze ubogich, ulomnych, chromych i slepych,
\par 14 A bedziesz blogoslawionym; bo nie maja tobie czem nagrodzic, ale ci bedzie nagrodzono przy zmartwychwstaniu sprawiedliwych.
\par 15 A uslyszawszy to niektóry z spólsiedzacych, rzekl mu: Blogoslawiony, który je chleb w królestwie Bozem.
\par 16 A on mu rzekl: Czlowiek niektóry sprawil wieczerze wielka i zaprosil wielu;
\par 17 I poslal sluge swego w godzine wieczerzy, zeby rzekl zaproszonym: Pójdzcie! bo juz wszystko gotowe.
\par 18 I poczeli sie wszyscy jednostajnie wymawiac. Pierwszy mu rzekl: Kupilem wies, i musze isc, a ogladac ja, prosze cie, miej mie za wymówionego.
\par 19 A drugi rzekl: Kupilem piec jarzm wolów, i ide, abym ich doswiadczyl: prosze cie, miej mie za wymówionego.
\par 20 A drugi rzekl: Zonem pojal, a dlatego przyjsc nie moge.
\par 21 A wróciwszy sie on sluga, oznajmil to panu swemu. Tedy sie gospodarz rozgniewawszy, rzekl sludze swemu: Wynijdz predko na ulice i na drogi miejskie, a ubogie i ulomne i chrome i slepe wprowadz tu.
\par 22 I rzekl sluga: Panie! stalo sie, jakos rozkazal, a jeszcze miejsce jest.
\par 23 I rzekl Pan do slugi: Wynijdz na drogi i miedzy oplotki, a przymus wnijsc, aby byl napelniony dom mój.
\par 24 Albowiem powiadam wam, ze zaden z onych mezów, którzy byli zaproszeni, nie ukusi wieczerzy mojej.
\par 25 I szedl z nim wielki lud; a obróciwszy sie, rzekl do nich:
\par 26 Jezli kto idzie do mnie, a nie ma w nienawisci ojca swego, i matki, i zony, i dzieci, i braci, i sióstr, nawet i duszy swojej, nie moze byc uczniem moim.
\par 27 A ktokolwiek nie niesie krzyza swego, a idzie za mna, nie moze byc uczniem moim.
\par 28 Bo któz z was jest, chcac zbudowac wieze, aby pierwej usiadlszy, nie obrachowal nakladu, mali to, czemby jej dokonczyl?
\par 29 Aby snac, gdyby zalozyl fundament, a dokonczyc nie mógl, wszyscy którzy by to widzieli, nie poczeli sie nasmiewac z niego,
\par 30 Mówiac: Ten czlowiek poczal budowac, a nie mógl dokonczyc.
\par 31 Albo który król jadac na wojne, potykac sie z drugim królem, pierwej usiadlszy, nie radzi sie, móglliby sie w dziesiec tysiecy spotkac z onym, który we dwadziescia tysiecy jedzie przeciwko niemu?
\par 32 A jezli nie, gdy on jeszcze jest daleko od niego, posly wyprawiwszy do niego, prosi o to, co nalezy do pokoju.
\par 33 Takzec i kazdy z was, kto by sie nie wyrzekl wszystkich majetnosci swoich, nie moze byc uczniem moim.
\par 34 Dobrac jest sól; lecz jezli sól zwietrzeje, czemze ja naprawia?
\par 35 Nie przygodzi sie ani do ziemi ani do gnoju, ale ja precz wyrzucaja. Kto ma uszy ku sluchaniu, niechaj slucha.

\chapter{15}

\par 1 I przyblizali sie do niego wszyscy celnicy i grzesznicy, aby go sluchali.
\par 2 I szemrali Faryzeuszowie i nauczeni w Pismie, mówiac: Ten grzeszniki przyjmuje i je z nimi.
\par 3 I powiedzial im to podobienstwo, mówiac:
\par 4 Któryz z was czlowiek, gdyby mial sto owiec, a stracilby jedne z nich, izali nie zostawia onych dziewiecdziesieciu i dziewieciu na puszczy, a nie idzie za ona, która zginela, azby ja znalazl?
\par 5 A znalazlszy kladzie ja na ramiona swoje, radujac sie.
\par 6 A przyszedlszy do domu, zwoluje przyjaciól, i sasiadów, mówiac im: Radujcie sie ze mna; bom znalazl owce, która byla zginela.
\par 7 Powiadam wam, ze taka bedzie radosc w niebie nad jednym grzesznikiem pokutujacym, wiecej niz nad dziewiecdziesiat i dziewieciu sprawiedliwych, którzy nie potrzebuja pokuty.
\par 8 Albo która niewiasta majac dziesiec groszy, jezliby stracila grosz jeden, izali nie zapala swiecy, i nie umiata domu, a nie szuka z pilnoscia, azby znalazla?
\par 9 A znalazlszy, zwoluje przyjaciólek i sasiadek, mówiac: Radujcie sie ze mna; albowiem znalazlam grosz, którym byla stracila.
\par 10 Tak, powiadam wam, bedzie radosc przed Anioly Bozymi nad jednym grzesznikiem pokutujacym.
\par 11 Nadto rzekl: Czlowiek niektóry mial dwóch synów,
\par 12 I rzekl mlodszy z nich ojcu: Ojcze! daj mi dzial majetnosci na mie przypadajacy. I rozdzielil im majetnosc.
\par 13 A po niewielu dniach, zebrawszy wszystko on mlodszy syn, odjechal w daleka kraine, i rozproszyl tam majetnosc swoje, zyjac rozpustnie.
\par 14 A gdy wszystko potracil, stal sie glód wielki w onej krainie, a on poczal niedostatek cierpiec.
\par 15 A tak szedlszy, przystal do jednego mieszczanina onej krainy, który go poslal do folwarku swego, aby pasl swinie.
\par 16 I zadal napelnic brzuch swój mlótem, które jadaly swinie; ale mu nikt nie dawal.
\par 17 Potem przyszedlszy do siebie, rzekl: O jako wiele najemników ojca mego maja dosyc chleba, a ja od glodu gine!
\par 18 Wstawszy tedy, pójde do ojca mego i rzeke mu: Ojcze! zgrzeszylem przeciwko niebu i przed toba.
\par 19 I nie jestem godzien wiecej byc nazywany synem twoim, uczyn mie jako jednego z najemników twoich.
\par 20 Tedy wstawszy, szedl do ojca swego. A gdy on jeszcze byl opodal, ujrzal go ojciec jego, i uzaliwszy sie go, przybiezal, a padlszy na szyje jego, pocalowal go.
\par 21 I rzekl mu syn: Ojcze! zgrzeszylem przeciwko niebu i przed toba, i juzem nie jest godzien, abym byl nazywany synem twoim.
\par 22 Rzekl tedy ojciec do slug swoich: Przyniescie one przednia szate, a obleczcie go, i dajcie pierscien na reke jego, i obuwie na nogi jego.
\par 23 A przywiódlszy ono tluste ciele, zabijcie, a jedzac badzmy weseli.
\par 24 Albowiem ten syn mój umarl byl, a zasie ozyl; zginal byl, i znaleziony jest; i poczeli sie weselic.
\par 25 Ale starszy syn jego byl na polu; a gdy przychodzac przyblizyl sie ku domowi, uslyszal muzyke i tance;
\par 26 A zawolawszy jednego z slug, pytal, co by to bylo.
\par 27 A on mu powiedzial: Brat twój przyszedl, i zabil ojciec twój ono tluste ciele, iz go zdrowego dostal.
\par 28 I rozgniewal sie, a nie chcial wnijsc; ale ojciec jego wyszedlszy prosil go.
\par 29 A on odpowiadajac, rzekl ojcu: Oto przez tak wiele lat sluze tobie, a nigdym nie przestapil przykazania twego; wszakzes mi nigdy nie dal kozlecia, abym sie z przyjacioly moimi weselil.
\par 30 Ale gdy ten syn twój, który pozarl majetnosc twoje z wszetecznicami, przyszedl, zabiles mu ono tluste ciele.
\par 31 A on mu rzekl; Synu! tys zawsze ze mna, a wszystkie dobra moje twoje sa.
\par 32 Lecz trzeba bylo weselic sie i radowac, ze ten brat twój umarl byl, a zasie ozyl, i zginal byl a znaleziony jest.

\chapter{16}

\par 1 Mówil tez i do uczniów swoich: Czlowiek niektóry byl bogaty, który mial szafarza, a ten byl odniesiony do niego, jakoby rozpraszal dobra jego.
\par 2 A zawolawszy go, rzekl mu: Cóz slysze o tobie? Oddaj liczbe z szafarstwa twego; albowiem juz wiecej nie bedziesz mógl szafowac.
\par 3 I rzekl on szafarz sam w sobie: Cóz uczynie, gdyz pan mój odbiera ode mnie szafarstwo? Kopac nie moge, zebrac sie wstydze.
\par 4 Wiem, co uczynie, ze gdy bede zlozony z szafarstwa, przyjma mie do domów swoich.
\par 5 Zawolawszy tedy do siebie kazdego z dluzników pana swego rzekl pierwszemu: Wieles winien panu memu?
\par 6 A on rzekl: Sto barel oliwy. I rzekl mu: Wezmij zapis twój, a siadlszy predko, napisz piecdziesiat.
\par 7 Potem drugiemu rzekl: A tys wiele winien? A on mu rzekl: Sto korcy pszenicy. I rzekl mu: Wezmij zapis twój, a napisz osmdziesiat.
\par 8 I pochwalil pan szafarza niesprawiedliwego, iz roztropnie uczynil; bo synowie tego swiata roztropniejsi sa nad syny swiatlosci w rodzaju swoim.
\par 9 I jac wam powiadam: Czyncie sobie przyjacioly z mammony niesprawiedliwosci, aby gdy ustaniecie, przyjeli was do wiecznych przybytków.
\par 10 Kto wierny jest w malem, i w wielu wiernym jest; a kto w malem niesprawiedliwy, i w wielu niesprawiedliwym jest.
\par 11 Poniewazescie tedy w mammonie niesprawiedliwej wiernymi nie byli, prawdziwego któz wam powierzy?
\par 12 A jezliscie w cudzem wiernymi nie byli, któz wam da, co waszego jest?
\par 13 Zaden sluga nie moze dwom panom sluzyc, gdyz albo jednego bedzie mial w nienawisci, a drugiego bedzie milowal; albo sie jednego trzymac bedzie, a drugim pogardzi; nie mozecie Bogu sluzyc i mammonie.
\par 14 A sluchali tego wszystkiego i Faryzeuszowie, którzy byli lakomi, i nasmiewali sie z niego.
\par 15 I rzekl im: Wy jestescie, którzy sami siebie usprawiedliwiacie przed ludzmi, ale Bóg zna serca wasze; bo co jest u ludzi wynioslego, obrzydliwoscia jest przed Bogiem.
\par 16 Zakon i prorocy az do Jana; a od tego czasu królestwo Boze opowiadane bywa, a kazdy sie do niego gwaltem cisnie.
\par 17 I latwiej jest niebu i ziemi przeminac, nizeli jednej kresce zakonu upasc.
\par 18 Wszelki, który opuszcza zone swoje, a inna pojmuje, cudzolozy; a kto od meza opuszczona pojmuje, cudzolozy.
\par 19 A byl niektóry czlowiek bogaty, który sie oblóczyl w szarlat i w bisior, i uzywal na kazdy dzien hojnie.
\par 20 Byl tez niektóry zebrak, imieniem Lazarz, który lezal u wrót jego owrzodzialy.
\par 21 Pragnac byc nasycony z odrobin, które padaly z stolu bogaczowego; ale i psy przychodzac lizaly wrzody jego.
\par 22 I stalo sie, ze umarl on zebrak, i odniesiony byl od Aniolów na lono Abrahamowe; umarl tez i bogacz, i pogrzebiony jest.
\par 23 A bedac w piekle, podnióslszy oczy swe, gdy byl w mekach, ujrzal Abrahama z daleka, i Lazarza na lonie jego.
\par 24 Tedy bogacz zawolawszy, rzekl: Ojcze Abrahamie! zmiluj sie nade mna, a poslij Lazarza, aby omoczyl koniec palca swego w wodzie, a ochlodzil jezyk mój, bo meki cierpie w tym plomieniu.
\par 25 I rzekl Abraham: Synu! wspomnij, zes ty odebral dobre rzeczy twoje za zywota twego, a Lazarz takze zle; a teraz on ma pocieche, a ty meki cierpisz.
\par 26 A nad to wszystko miedzy nami i wami otchlan wielka jest utwierdzona, aby ci, którzy chca stad przyjsc do was, nie mogli, ani owi stamtad przyjsc do nas.
\par 27 A on rzekl: Prosze cie tedy, ojcze! abys poslal do domu ojca mego:
\par 28 Albowiem mam piec braci, aby im swiadectwo wydal, zeby tez i oni nie przyszli na to miejsce meki.
\par 29 I rzekl mu Abraham: Majac Mojzesza i proroków, niechze ich sluchaja.
\par 30 A on rzekl: Nie, ojcze Abrahamie! ale gdyby kto z umarlych szedl do nich, beda pokutowac.
\par 31 I rzekl mu: Poniewaz Mojzesza i proroków nie sluchaja, tedy, chocby tez kto zmartwychwstal, nie uwierza.

\chapter{17}

\par 1 I rzekl do uczniów: Nie mozna, aby zgorszenia przyjsc nie mialy; ale biada temu, przez którego przychodza!
\par 2 Lepiej by mu bylo, aby mlynski kamien zawieszony byl na szyi jego, i wrzucony byl w morze, nizby jednego z tych malych zgorszyc mial.
\par 3 Miejciez sie na pieczy. A jezliby zgrzeszyl przeciwko tobie brat twój, strofuj go, a jezliby sie upamietal, odpusc mu.
\par 4 A chocby siedmkroc na dzien zgrzeszyl przeciwko tobie, i siedmkroc przez dzien sie do ciebie nawrócil, mówiac: Zal mi tego; odpusc mu.
\par 5 I rzekli Apostolowie Panu: Przymnóz nam wiary.
\par 6 A Pan rzekl: Jezlibyscie mieli wiare jako ziarno gorczyczne, a rzeklibyscie temu drzewu lesnej figi: Wyrwij sie z korzenia, a wsadz sie w morzu, usluchaloby was.
\par 7 I któz z was jest, co by mial sluge orzacego albo pasacego, który gdyby sie wrócil, zaraz by mu rzekl: Pójdz, a usiadz za stól?
\par 8 I owszem, izali mu nie rzecze: Nagotuj, co bym wieczerzal, a przepasawszy sie, sluz mi, az sie najem i napije, a potem i ty jedz i pij?
\par 9 Izali dziekuje sludze onemu, iz uczynil to, co mu bylo rozkazano? Nie zda mi sie.
\par 10 Takze i wy, gdy uczynicie wszystko, co wam rozkazano, mówcie: Sludzy nieuzyteczni jestesmy, bo cosmy byli powinni uczynic, uczynilismy.
\par 11 I stalo sie, gdy szedl do Jeruzalemu, ze szedl posrodkiem Samaryi i Galilei.
\par 12 A gdy wchodzil do niektórego miasteczka, zabiezalo mu dziesiec mezów tredowatych, którzy staneli z daleka.
\par 13 A ci podnióslszy glos swój, rzekli: Jezusie, Nauczycielu! zmiluj sie nad nami.
\par 14 Które on ujrzawszy, rzekl im: Szedlszy okazcie sie kaplanom. I stalo sie, gdy szli, ze oczyszczeni sa.
\par 15 Ale jeden z nich ujrzawszy, ze jest uzdrowiony, wrócil sie, wielkim glosem chwalac Boga;
\par 16 I padl na oblicze swoje u nóg jego, dziekujac mu; a ten byl Samarytanin.
\par 17 A Jezus odpowiadajac, rzekl: Zaz nie dziesiec jest oczyszczonych, a dziewiec kedy?
\par 18 Nie znalezli sie, aby sie wrócili, i dali chwale Bogu, jedno ten cudzoziemiec?
\par 19 I rzekl mu: Wstan, idz, wiara twoja ciebie uzdrowila.
\par 20 A bedac pytany od Faryzeuszów, kiedy przyjdzie królestwo Boze, odpowiedzial im i rzekl: Nie przyjdziec królestwo Boze z postrzezeniem;
\par 21 Ani rzeka: Oto tu, albo oto tam jest: albowiem oto królestwo Boze wewnatrz was jest.
\par 22 I rzekl do uczniów: Przyjda dni, ze bedziecie zadac, abyscie widzieli jeden dzien ze dni Syna czlowieczego, ale nie ogladacie.
\par 23 I rzeka wam: Oto tu, albo oto tam jest; ale nie chodzcie, ani sie za nimi udawajcie.
\par 24 Albowiem jako blyskawica, blyskajac sie od jednej strony, która jest pod niebem, az do drugiej, która jest pod niebem, swieci: tak bedzie i Syn czlowieczy w dzien swój.
\par 25 Ale pierwej musi wiele ucierpiec, i byc odrzuconym od narodu tego.
\par 26 A jako bylo za dni Noego, tak bedzie i za dni Syna czlowieczego.
\par 27 Jedli, pili, zenili sie i za maz wydawali az do onego dnia, którego wszedl Noe do korabia, i przyszedl potop, a wytracil wszystkie.
\par 28 Takze tez jako sie dzialo za dni Lotowych, jedli, pili, kupowali, sprzedawali, szczepili, budowali.
\par 29 Ale dnia tego, gdy wyszedl Lot z Sodomy, spadl jako deszcz ogien z siarka z nieba, i wytracil wszystkie.
\par 30 Takci tez bedzie w on dzien, którego sie Syn czlowieczy objawi.
\par 31 Onegoz dnia, bylliby kto na dachu, a naczynia jego w domu, niech nie zstepuje, aby je pobral; a kto na roli, niech sie takze nie wraca do tego, co jest pozad.
\par 32 Pamietajcie na zone Lotowa.
\par 33 Ktobykolwiek chcial zachowac dusze swoje, straci ja; a kto by ja kolwiek stracil, ozywi ja.
\par 34 Powiadam wam: Onej nocy beda dwaj na jednem lozu; jeden wziety bedzie, a drugi zostawiony.
\par 35 Dwie beda mlec z soba; jedna wzieta bedzie, a druga zostawiona.
\par 36 Dwaj beda na roli; jeden bedzie wziety, a drugi zostawiony.
\par 37 A odpowiadajac rzekli mu: Gdziez Panie? A on im rzekl: Gdzie bedzie scierw, tam sie zgromadza i orly.

\chapter{18}

\par 1 I powiedzial im jeszcze podobienstwo do tego zmierzajace, iz sie zawsze potrzeba modlic, a nie ustawac,
\par 2 Mówiac: Byl niektóry sedzia w jednem miescie, który sie Boga nie bal, i czlowieka sie nie wstydzil.
\par 3 Byla tez wdowa w temze miescie, która przychodzila do niego, mówiac: Pomscij sie krzywdy mojej nad przeciwnikiem moim.
\par 4 Lecz on dlugo nie chcial. Ale potem rzekl sam w sobie: Aczci sie Boga nie boje i czlowieka sie nie wstydze,
\par 5 Wszakze iz mi sie uprzykrza ta wdowa, pomszcze sie krzywdy jej, aby na ostatek przyszedlszy, nie byla mi ciezka.
\par 6 Rzekl tedy Pan: Sluchajciez, co mówi niesprawiedliwy sedzia.
\par 7 A Bóg izali sie nie pomsci krzywdy wybranych swoich, wolajacych do siebie we dnie i w nocy, chociaz im dlugo cierpi?
\par 8 Powiadam wam, iz sie pomsci krzywdy ich w rychle. Ale gdy przyjdzie Syn czlowieczy, izali znajdzie wiare na ziemi?
\par 9 Rzekl tez i do niektórych, którzy ufali sami w sobie, ze byli sprawiedliwymi, a inszych za nic nie mieli, to podobienstwo:
\par 10 Dwoje ludzi wstapilo do kosciola, aby sie modlili, jeden Faryzeusz a drugi celnik.
\par 11 Faryzeusz stanawszy, tak sie sam u siebie modlil: Dziekuje tobie, Boze! zem nie jest jako inni ludzie, drapiezni, niesprawiedliwi, cudzoloznicy, albo jako i ten celnik.
\par 12 Poszcze dwakroc w tydzien; daje dziesiecine ze wszystkiego, co mam.
\par 13 A celnik stojac z daleka, nie chcial podniesc i oczu swych w niebo, ale sie bil w piersi swoje, mówiac: Boze! badz milosciw mnie grzesznemu.
\par 14 Powiadam wam, zec ten odszedl usprawiedliwionym do domu swego, wiecej nizeli on: albowiem kto sie wywyzsza, bedzie ponizony, a kto sie poniza, bedzie wywyzszony.
\par 15 Przynoszono tez do niego i niemowlatka, aby sie ich dotykal; co gdy widzieli uczniowie, gromili je.
\par 16 Ale Jezus zwolawszy ich, rzekl: Dopusccie dziatkom przychodzic do mnie, a nie zabraniajcie im; albowiem takowych jest królestwo Boze.
\par 17 Zaprawde powiadam wam: Ktobykolwiek nie przyjal królestwa Bozego jako dzieciatko, nie wnijdzie do niego.
\par 18 I pytal go niektóry ksiaze, mówiac: Nauczycielu dobry! co czyniac odziedzicze zywot wieczny?
\par 19 I rzekl mu Jezus: Przecz mie zowiesz dobrym? Nikt nie jest dobry, tylko jeden, to jest Bóg.
\par 20 Umiesz przykazania? Nie cudzolóz, nie zabijaj, nie kradnij, nie swiadcz falszywie, czcij ojca twego i matke twoje.
\par 21 A on rzekl: Tegom wszystkiego przestrzegal od mlodosci mojej.
\par 22 Co uslyszawszy Jezus, rzekl mu: Jednego ci jeszcze nie dostaje; wszystko, co masz, sprzedaj, a rozdaj ubogim, a bedziesz mial skarb w niebie; a przyszedlszy nasladuj mie.
\par 23 A on uslyszawszy to, bardzo sie zasmucil; bo byl nader bogaty.
\par 24 A gdy go Jezus ujrzal bardzo zasmuconego, rzekl: Jakoz trudno ci, co maja pieniadze, wnijda do królestwa Bozego!
\par 25 Albowiem latwiej jest wielbladowi przejsc przez ucho igielne, niz bogatemu wnijsc do królestwa Bozego.
\par 26 Tedy rzekli ci, którzy to slyszeli: I któz moze byc zbawiony?
\par 27 Ale on rzekl: Co jest niemozebne u ludzi, mozebne jest u Boga.
\par 28 I rzekl Piotr: Otosmy my opuscili wszystko, a poszlismy za toba.
\par 29 Tedy im on rzekl: Zaprawde powiadam wam, iz nie masz nikogo, co by opuscil dom, albo rodziców, albo braci, albo zone, albo dzieci dla królestwa Bozego,
\par 30 Aby nie wzial daleko wiecej w tym czasie, a w przyszlym wieku zywota wiecznego.
\par 31 A wziawszy z soba onych dwunastu, rzekl im: Oto wstepujemy do Jeruzalemu, a wypelni sie wszystko, co napisano przez proroki o Synu czlowieczym.
\par 32 Bo bedzie wydany poganom, i bedzie nasmiewany, i zelzony, i uplwany:
\par 33 A ubiczowawszy zabija go; ale dnia trzeciego zmartwychwstanie.
\par 34 Lecz oni z tego nic nie zrozumieli, i bylo to slowo zakryte przed nimi, i nie wiedzieli, co mówiono.
\par 35 I stalo sie, gdy sie on przyblizal do Jerycha, slepy niektóry siedzial podle drogi, zebrzac.
\par 36 A uslyszawszy lud przechodzacy, pytal, co by to bylo?
\par 37 I powiedziano mu, iz Jezus Nazarenski tedy idzie.
\par 38 I zawolal, mówiac: Jezusie, Synu Dawidowy! zmiluj sie nade mna.
\par 39 Lecz ci, co szli wprzód, gromili go, aby milczal. Ale on tem wiecej wolal: Synu Dawidowy! zmiluj sie nade mna.
\par 40 Zastanowiwszy sie tedy Jezus, kazal go przywiesc do siebie; a gdy sie przyblizyl, pytal go, mówiac:
\par 41 Co chcesz, abym ci uczynil? A on rzekl: Panie! abym przejrzal.
\par 42 A Jezus mu rzekl: Przejrzyj, wiara twoja ciebie uzdrowila.
\par 43 I zarazem przejrzal, i szedl za nim, wielbiac Boga. Co wszystek lud widzac, dal chwale Bogu.

\chapter{19}

\par 1 A Jezus wszedlszy, szedl przez Jerycho.
\par 2 A oto maz, którego zwano imieniem Zacheusz, który byl przelozony nad celnikami, a ten byl bogaty.
\par 3 I zadal widziec Jezusa, co by zacz byl; lecz nie mógl przed ludem, bo byl malego wzrostu.
\par 4 A biezawszy naprzód, wstapil na drzewo lesnej figi, aby go ujrzal; bo tamtedy isc mial.
\par 5 A gdy przyszedl na ono miejsce, spojrzawszy Jezus w góre, ujrzal go, i rzekl do niego: Zacheuszu! zstap predko na dól, albowiem dzis musze zostac w domu twoim.
\par 6 I zstapil predko i przyjal go z radoscia.
\par 7 A widzac to wszyscy, szemrali, mówiac: U czlowieka grzesznego gospoda stanal.
\par 8 A stanawszy Zacheusz, rzekl do Pana: Oto polowe majetnosci moich dam ubogim, Panie! a jezlizem kogo w czem podszedl, oddam w czwórnasób.
\par 9 I rzekl mu Jezus: Dzis sie stalo zbawienie domowi temu, dlatego ze i on jest synem Abrahamowym.
\par 10 Bo przyszedl Syn czlowieczy, aby szukal i zachowal, co bylo zginelo.
\par 11 Tedy gdy oni sluchali, mówiac dalej powiedzial im podobienstwo, dlatego ze byl blisko od Jeruzalemu, a iz oni mniemali, ze sie wnet królestwo Boze objawic mialo.
\par 12 Rzekl tedy: Niektóry czlowiek rodu zacnego jechal w daleka kraine, aby sobie wzial królestwo, i zasie sie wrócil.
\par 13 A zawolawszy dziesieciu slug swoich, dal im dziesiec grzywien i rzekl do nich: Handlujcie, az przyjade.
\par 14 Lecz mieszczanie jego mieli go w nienawisci, i wyprawili za nim poselstwo, mówiac: Nie chcemy, aby ten królowal nad nami.
\par 15 I stalo sie, gdy sie wrócil wziawszy królestwo, ze rozkazal do siebie zawolac slug onych, którym byl dal pieniadze, aby sie dowiedzial, co który handlujac zyskal.
\par 16 Tedy przyszedl pierwszy, mówiac: Panie! grzywna twoja dziesiec grzywien urobila.
\par 17 I rzekl mu: Dobrze, slugo dobry! izes byl nad malem wiernym, miejze wladze nad dziesiecioma miastami.
\par 18 Przyszedl i drugi, mówiac: Panie! grzywna twoja piec grzywien urobila.
\par 19 Rzekl i temu: I ty badz nad piecioma miastami.
\par 20 A inszy przyszedl, mówiac: Panie! oto grzywna twoja, któram mial schowana w chustce;
\par 21 Bom sie ciebie bal, zes jest czlowiek srogi; bierzesz, czegos nie polozyl, a zniesz, czegos nie sial.
\par 22 Tedy mu rzekl: Z ust twoich sadze cie, zly slugo! Wiedziales, zem ja jest czlowiek srogi, który biore, czegom nie polozyl, a zne, czegom nie sial.
\par 23 Przeczzes tedy nie dal srebra mego do lichwiarzy? a ja przyszedlszy, wzialbym je byl z lichwa.
\par 24 I rzekl tym, którzy tuz stali: Wezmijcie od niego te grzywne, a dajcie temu, który ma dziesiec grzywien.
\par 25 I rzekli mu: Panie! mac dziesiec grzywien.
\par 26 Zaprawde powiadam wam, iz wszelkiemu, który ma, bedzie dane, a od tego, który nie ma, i to, co ma, bedzie od niego odjete.
\par 27 Ale i nieprzyjacioly moje, którzy nie chcieli, abym królowal nad nimi, przywiedzcie tu, a pobijcie przede mna.
\par 28 A to powiedziawszy, szedl wprzód, wstepujac do Jeruzalemu.
\par 29 I stalo sie, gdy sie przyblizyl do Betfagie i Betanii, ku górze, która zowia oliwna, poslal dwóch z uczniów swoich,
\par 30 Mówiac: Idzcie do miasteczka, które jest przeciwko wam, do którego wszedlszy, znajdziecie osle uwiazane, na którem zaden czlowiek nigdy nie siedzial; odwiazawszy je, przywiedzcie:
\par 31 A jezliby was kto spytal, przecz je odwiazujecie? tak mu powiecie: Przeto, ze go Pan potrzebuje.
\par 32 Odszedlszy tedy ci, którzy byli poslani, znalezli, jako im byl powiedzial.
\par 33 A gdy oni odwiazywali ono osle, rzekli panowie jego do nich: Przecz odwiazujecie osle?
\par 34 A oni powiedzieli: Pan go potrzebuje.
\par 35 I przywiedli je do Jezusa, a wlozywszy szaty swoje na ono osle, wsadzili Jezusa na nie.
\par 36 A gdy on jechal, slali szaty swoje na drodze.
\par 37 A gdy sie juz przyblizal tam, gdzie sie spuszczaja z góry oliwnej, poczelo wszystko mnóstwo uczniów radujac sie chwalic Boga glosem wielkim ze wszystkich cudów, które widzieli,
\par 38 Mówiac: Blogoslawiony król, który idzie w imieniu Panskiem; pokój na niebie, a chwala na wysokosciach.
\par 39 Ale niektórzy z Faryzeuszów z onego ludu rzekli do niego: Nauczycielu! zgrom ucznie twoje.
\par 40 A on odpowiadajac, rzekl im: Powiadam wam, jezliby ci milczeli, wnet kamienie wolac beda.
\par 41 A gdy sie przyblizyl, ujrzawszy miasto, plakal nad niem, mówiac:
\par 42 O gdybys poznalo i ty, a zwlaszcza w ten to dzien twój, co jest ku pokojowi twemu! lecz to teraz zakryte od oczów twoich.
\par 43 Albowiem przyjda na cie dni, gdy cie otocza nieprzyjaciele twoi walem, i oblega cie, i scisna cie zewszad;
\par 44 I zrównaja cie z ziemia, i dzieci twoje w tobie, a nie zostawia w tobie kamienia na kamieniu, dlatego zes nie poznalo czasu nawiedzenia twego.
\par 45 A wszedlszy do kosciola, poczal wyganiac te, którzy w nim sprzedawali i kupowali.
\par 46 Mówiac im: Napisano: Dom mój dom modlitwy jest, a wyscie go uczynili jaskinia zbójców.
\par 47 I uczyl na kazdy dzien w kosciele; lecz przedniejsi kaplani i nauczeni w Pismie, i przedniejsi z ludu szukali go stracic;
\par 48 Ale nie znalezli, co by mu uczynili; albowiem wszystek lud zawieszal sie na nim, sluchajac go.

\chapter{20}

\par 1 I stalo sie z onych dni dnia jednego, gdy uczyl lud w kosciele i kazal Ewangelije, ze nadeszli przedniejsi kaplani i nauczeni w Pismie z starszymi,
\par 2 I rzekli do niego, mówiac: Powiedz nam, która moca to czynisz, albo kto jest ten, coc dal te moc?
\par 3 A on odpowiadajac, rzekl do nich: Spytam i ja was o jedne rzecz, a powiedzcie mi:
\par 4 Chrzest Jana bylli z nieba, czyli z ludzi?
\par 5 A oni myslili sami w sobie, mówiac: Jezli powiemy, z nieba, rzecze: Czemuzescie mu tedy nie wierzyli?
\par 6 Jezliz zasie rzeczemy, z ludzi, wszystek lud ukamionuje nas, poniewaz za pewne maja, ze Jan jest prorokiem.
\par 7 I odpowiedzieli, ze nie wiedza, skad by byl.
\par 8 A Jezus im rzekl: I ja wam nie powiem, która moca to czynie.
\par 9 I poczal do ludu mówic to podobienstwo: Czlowiek niektóry nasadzil winnice, i najal ja winiarzom, i odjechal precz na czas niemaly.
\par 10 A czasu swego poslal sluge do onych winiarzy, aby mu dali z pozytku onej winnicy; ale oni winiarze ubiwszy go, odeslali próznego.
\par 11 I poslal zasie drugiego sluge; ale oni i tego ubiwszy i zelzywszy, odeslali próznego.
\par 12 I poslal zasie trzeciego; ale oni i tego zraniwszy, wyrzucili precz.
\par 13 A tak rzekl Pan onej winnicy: Cóz uczynie? posle syna mego milego, snac gdy tego ujrza, zawstydza sie.
\par 14 Ale winiarze ujrzawszy go, rzekli miedzy soba, mówiac: Tenci jest dziedzic; pójdzcie zabijmy go, aby nasze bylo dziedzictwo.
\par 15 I wypchnawszy go precz z winnicy, zabili. Cóz im tedy uczyni Pan onej winnicy?
\par 16 Przyjdzie, a potraci one winiarze, a winnice odda innym. Co oni uslyszawszy, rzekli: Nie daj tego Boze!
\par 17 Lecz on spojrzawszy na nie, rzekl: Cóz tedy jest ono, co napisano: Kamien, który odrzucili budujacy, ten sie stal glowa wegielna?
\par 18 Wszelki, który upadnie na ten kamien, roztraci sie, a na kogo by upadl, zetrze go.
\par 19 I starali sie przedniejsi kaplani i nauczeni w Pismie, jakoby nan rece wrzucili onejze godziny, ale sie ludu bali; albowiem poznali, iz przeciwko nim wyrzekl to podobienstwo.
\par 20 A tak podstrzegajac go, poslali szpiegi, którzy zmyslali, jakoby byli sprawiedliwymi, aby go podchwycili w mowie jego, a potem aby go podali zwierzchnosci i w moc staroscina.
\par 21 I pytali go, mówiac: Nauczycielu! wiemy, ze dobrze mówisz i uczysz, ani przyjmujesz osób; ale drogi Bozej w prawdzie uczysz.
\par 22 Godzili sie nam dac czynsz cesarzowi, czyli nie?
\par 23 Ale on obaczywszy chytrosc ich, rzekl do nich: Czemuz mie kusicie?
\par 24 Ukazcie mi grosz; czyj ma obraz i napis? A odpowiadajac rzekli: Cesarza.
\par 25 Zatem on im rzekl: Oddawajciez tedy, co jest cesarskiego, cesarzowi, a co jest Bozego, Bogu.
\par 26 I nie mogli go podchwycic w mowie jego przed ludem, a zadziwiwszy sie odpowiedzi jego, umilkneli.
\par 27 A przyszedlszy niektórzy z Saduceuszów, (którzy przecza i mówia, iz nie masz zmartwychwstania), pytali go,
\par 28 Mówiac: Nauczycielu! Mojzesz nam napisal: Jezliby komu brat umarl, majac zone, a umarlby bez dziatek, aby brat jego pojal jego zone, a wzbudzil nasienie bratu swemu.
\par 29 Bylo tedy siedm braci, z których pierwszy pojawszy zone, umarl bez dziatek.
\par 30 I pojal wtóry one zone, a umarl i ten bez dziatek.
\par 31 Potem ja pojal i trzeci, takze i oni wszyscy siedmiu, a nie zostawiwszy dziatek, pomarli.
\par 32 Po wszystkich tez umarla i ona niewiasta.
\par 33 Przetoz przy zmartwychwstaniu, któregoz z nich ona bedzie zona, poniewaz siedmiu ich mialo ja za zone?
\par 34 Tedy odpowiadajac, rzekl im Jezus: Synowie tego wieku zenia sie i za maz wydaja.
\par 35 Ale ci, którzy godni sa, aby dostapili onego wieku, i powstana od umarlych, ani sie zenic, ani za maz dawac beda.
\par 36 Albowiem umierac wiecej nie beda mogli; bo beda równi Aniolom, bedac synami Bozymi, gdyz sa synami zmartwychwstania.
\par 37 A iz umarli zmartwychwstana, i Mojzesz pokazal przy onym krzaku, gdy zowie Pana Boga Bogiem Abrahamowym i Bogiem Izaakowym i Bogiem Jakóbowym.
\par 38 A Bógci nie jest Bogiem umarlych, ale zywych; bo jemu wszyscy zyja.
\par 39 Tedy odpowiadajac niektórzy z nauczonych w Pismie, rzekli: Nauczycielu! dobrzes powiedzial.
\par 40 I nie smieli go wiecej o nic pytac.
\par 41 I rzekl do nich: Jakoz powiadaja, ze Chrystus jest synem Dawidowym?
\par 42 A sam Dawid mówi w ksiegach Psalmów: Rzekl Pan Panu memu: Siadz po prawicy mojej
\par 43 Az poloze nieprzyjacioly twoje podnózkiem nóg twoich.
\par 44 Poniewaz go tedy Dawid nazywa Panem, i jakoz jest synem jego?
\par 45 A gdy sluchal wszystek lud, rzekl uczniom swoim:
\par 46 Strzezcie sie nauczonych w Pismie, którzy chca chodzic w szatach dlugich, i miluja pozdrawiania na rynkach i pierwsze stolki w bóznicach, i pierwsze miejsca na wieczerzach;
\par 47 Którzy pozeraja domy wdów, a to pod pokrywka dlugich modlitw: cic odniosa ciezszy sad.

\chapter{21}

\par 1 A spojrzawszy ujrzal bogacze rzucajace dary swoje do skarbnicy.
\par 2 Ujrzal tez i niektóra wdowe ubozuchna, wrzucajaca tamze dwa drobne pieniazki.
\par 3 I rzekl: Prawdziwiec wam powiadam, zec ta uboga wdowa wiecej niz ci wszyscy wrzucila.
\par 4 Ci bowiem wszyscy z tego, co im zbywalo, wrzucili do darów Bozych, ale ta z niedostatku swego wszystke zywnosc, która miala, wrzucila.
\par 5 A gdy niektórzy mówili o kosciele, iz byl pieknym kamieniem i upominkami ozdobiony, rzekl:
\par 6 Z tego, co widzicie, przyjda dni, w które nie bedzie zostawiony kamien na kamieniu, który by nie byl rozwalony.
\par 7 I pytali go, mówiac: Nauczycielu! kiedyz to bedzie? a co za znak, gdy sie to bedzie mialo dziac?
\par 8 A on rzekl: Patrzcie, abyscie nie byli zwiedzeni; boc wiele ich przyjdzie w imieniu mojem, mówiac: Jam jest Chrystus, a czas sie przyblizyl; nie udawajciez sie tedy za nimi.
\par 9 A gdy uslyszycie o wojnach i rozruchach, nie lekajcie sie; albowiem musi to byc pierwej, alec jeszcze nie tu jest koniec.
\par 10 Tedy im mówil: Powstanie naród przeciwko narodowi, i królestwo przeciwko królestwu;
\par 11 I beda miejscami wielkie trzesienia ziemi, i glody i mory, takze strachy i znaki wielkie z nieba beda.
\par 12 Ale przed tem wszystkiem wrzuca na was rece swoje, i beda was przesladowac, podawajac do bóznic i do wiezienia, wodzac przed króle i przed starosty dla imienia mego.
\par 13 A to was spotka na swiadectwo.
\par 14 Przetoz zlózcie to do serc waszych, abyscie przed czasem nie myslili, jako byscie odpowiadac mieli.
\par 15 Albowiem ja wam dam usta i madrosc, której nie beda mogli odeprzec, ani sie sprzeciwic wszyscy przeciwnicy wasi.
\par 16 A bedziecie tez wydani od rodziców i od braci i od krewnych i od przyjaciól, i zabija niektóre z was;
\par 17 Bedziecie w nienawisci u wszystkich dla imienia mego.
\par 18 Ale ani wlos z glowy waszej nie zginie.
\par 19 W cierpliwosci waszej posiadajcie dusze wasze.
\par 20 A gdy ujrzycie Jeruzalem od wojsk otoczone, tedy wiedzcie, zec sie przyblizylo spustoszenie jego.
\par 21 Tedy ci, co sa w Judzkiej ziemi, niech uciekaja na góry, a ci, co sa w posrodku jej, niech wychodza, a ci, co sa w polach, niechaj nie wchodza do niej.
\par 22 Albowiem te dni sa pomsty, aby sie wypelnilo wszystko, co napisane.
\par 23 Ale biada brzemiennym i piersiami karmiacym w owe dni! albowiem bedzie ucisk wielki w tej ziemi i gniew Bozy nad tym ludem.
\par 24 I polegna od ostrza miecza, i zapedzeni beda w niewole miedzy wszystkie narody, i bedzie Jeruzalem deptane od pogan, az sie wypelnia czasy pogan.
\par 25 Tedy beda znaki na sloncu i na ksiezycu i na gwiazdach, a na ziemi ucisnienie narodów z rozpacza, gdy zaszumi morze i waly;
\par 26 Tak, iz ludzie dretwiec beda przed strachem i oczekiwaniem tych rzeczy, które przyjda na wszystek swiat; albowiem mocy niebieskie porusza sie.
\par 27 A tedy ujrza Syna czlowieczego, przychodzacego w obloku z moca i chwala wielka.
\par 28 A gdy sie to pocznie dziac, spogladajciez a podnoscie glowy wasze, przeto iz sie przybliza odkupienie wasze.
\par 29 I powiedzial im podobienstwo: Spojrzyjcie na figowe drzewo i na wszystkie drzewa;
\par 30 Gdy sie juz pukaja, widzac to sami to uznawacie, ze juz blisko jest lato.
\par 31 Takze i wy, gdy ujrzycie, iz sie to dzieje, wiedzcie, ze blisko jest królestwo Boze.
\par 32 Zaprawde powiadam wam, zec nie przeminie ten wiek, azby sie to wszystko stalo.
\par 33 Niebo i ziemia przemina, ale slowa moje nie przemina.
\par 34 A strzezcie sie, aby snac nie byly obciazone serca wasze obzarstwem i opilstwem i pieczolowaniem o ten zywot, a nagle by na was przyszedl ten dzien.
\par 35 Albowiem jako sidlo przypadnie na wszystkie, którzy mieszkaja na obliczu wszystkiej ziemi.
\par 36 Przetoz czujcie, modlac sie na kazdy czas, abyscie byli godni ujsc tego wszystkiego, co sie dziac ma, i stanac przed Synem czlowieczym.
\par 37 I nauczal we dnie w kosciele; ale w nocy wychodzac, przebywal na górze, która zowia oliwna.
\par 38 A wszystek lud rano sie schodzil do niego, aby go sluchal w kosciele.

\chapter{22}

\par 1 A przyblizalo sie swieto przasników, które zowia wielkanoca.
\par 2 I szukali przedniejsi kaplani i nauczeni w Pismie, jakoby go zabili; ale sie bali ludu.
\par 3 I wstapil szatan w Judasza, którego zwano Iszkaryjotem, który byl z liczby dwunastu.
\par 4 Ten tedy odszedlszy, zmówil sie z przedniejszymi kaplanami, i z przelozonymi kosciola, jakoby go im wydal.
\par 5 I uradowali sie, i umówili sie z nim, ze mu chca dac pieniadze.
\par 6 I obiecal, i szukal sposobnego czasu, aby go im wydal bez rozruchu.
\par 7 Tedy przyszedl dzien przasników, którego mial baranek byc zabity.
\par 8 I poslal Piotra i Jana, mówiac: Poszedlszy nagotujcie nam baranka, abysmy jedli.
\par 9 Ale oni mu rzekli: Gdziez chcesz, abysmy go nagotowali?
\par 10 A on rzekl do nich: Oto gdy do miasta wchodzic bedziecie, spotka sie z wami czlowiek, niosacy dzban wody; idzciez za nim do domu, do którego wnijdzie,
\par 11 A rzeczcie gospodarzowi domu onego: Kazal ci powiedziec nauczyciel: Gdzie jest gospoda, kedy bym jadl baranka z uczniami moimi?
\par 12 A on wam ukaze sale wielka uslana, tamze nagotujcie.
\par 13 Tedy odszedlszy znalezli, jako im byl powiedzial, i nagotowali baranka.
\par 14 A gdy przyszla godzina, usiadl za stól, i dwanascie apostolów z nim.
\par 15 I rzekl do nich: Zadajac zadalem tego baranka jesc z wami, pierwej nizbym cierpial.
\par 16 Boc wam powiadam, ze go wiecej jesc nie bede, az sie wypelni w królestwie Bozem.
\par 17 A wziawszy kielich i podziekowawszy, rzekl: Wezmijcie to, a podzielcie miedzy sie.
\par 18 Albowiem powiadam wam, ze nie bede pil z rodzaju winnej macicy, az przyjdzie królestwo Boze.
\par 19 A wziawszy chleb i podziekowawszy, lamal i dal im, mówiac: To jest cialo moje, które sie za was daje; to czyncie na pamiatke moje.
\par 20 Takze i kielich, gdy bylo po wieczerzy, mówiac: Ten kielich jest nowy testament we krwi mojej, która sie za was wylewa.
\par 21 Ale oto reka tego, co mie wydaje, ze mna jest za stolem.
\par 22 Synci zaiste czlowieczy idzie, tak jako jest postanowione; ale biada czlowiekowi temu, który go wydaje!
\par 23 Tedy sie oni poczeli miedzy soba pytac o tem, kto by wzdy z nich byl, co by to uczynic mial.
\par 24 A wszczal sie tez spór miedzy nimi o tem, kto by sie z nich zdal byc wiekszy.
\par 25 Ale on im rzekl: Królowie narodów panuja nad nimi, a którzy nad nimi moc maja, dobrodziejami nazywani bywaja.
\par 26 Lecz wy nie tak: owszem kto najwiekszy jest miedzy wami, niech bedzie jako najmniejszy, a kto jest wodzem, bedzie jako ten, co sluzy.
\par 27 Bo któryz wiekszy jest? Ten, co siedzi, czyli ten, co sluzy? Izali nie ten, co siedzi? Alem ja jest w posrodku was jako ten, co sluzy.
\par 28 A wy jestescie, którzyscie wytrwali przy mnie w pokusach moich.
\par 29 I jac wam sporzadzam, jako mi sporzadzil Ojciec mój, królestwo,
\par 30 Abyscie jedli i pili za stolem moim w królestwie mojem, i siedzieli na stolicach, sadzac dwanascie pokolen Izraelskich.
\par 31 I rzekl Pan: Szymonie, Szymonie! oto szatan wyprosil was, aby was odwiewal jako pszenice,
\par 32 Alem ja prosil za toba, aby nie ustala wiara twoja; a ty niekiedy nawróciwszy sie, utwierdzaj braci twoich.
\par 33 A on mu rzekl: Panie! gotówem z toba isc i do wiezienia i na smierc.
\par 34 A on rzekl: Powiadam ci, Piotrze! nie zapieje dzis kur, az sie pierwej trzykroc zaprzesz, ze mie nie znasz.
\par 35 I rzekl im: Gdym was posylal bez mieszka, i bez taistry, i bez butów, izali wam czego nie dostawalo? A oni rzekli: Niczego.
\par 36 Tedy im rzekl: Ale teraz kto ma mieszek, niech go wezmie, takze i taistre; a kto nie ma miecza, niech sprzeda suknie swoje, a kupi miecz.
\par 37 Albowiem powiadam wam, iz sie jeszcze musi to, co napisano, wypelnic na mnie, mianowicie: I z zloczyncami policzony jest; boc te rzeczy, które swiadcza o mnie, koniec biora.
\par 38 Ale oni rzekli: Panie! oto tu dwa miecze. A on im rzekl: Dosyc jest.
\par 39 I wyszedlszy szedl wedlug zwyczaju na góre Oliwna, a szli za nim i uczniowie jego.
\par 40 A gdy przyszedl na miejsce, rzekl im: Módlcie sie, abyscie nie weszli w pokuszenie.
\par 41 A sam oddalil sie od nich, jakoby na cisnienie kamieniem, a kleknawszy na kolana, modlil sie,
\par 42 Mówiac: Ojcze! jezli chcesz, przenies ten kielich ode mnie; wszakze nie moja wola, lecz twoja niech sie stanie.
\par 43 I ukazal mu sie Aniol z nieba, posilajacy go.
\par 44 Ale bedac w boju, gorliwiej sie modlil, a byl pot jego jako krople krwi sciekajace na ziemie.
\par 45 A wstawszy od modlitwy, przyszedl do uczniów, i znalazl je spiace od smutku.
\par 46 I rzekl im: Cóz spicie? wstancie, a módlcie sie, byscie nie weszli w pokuszenie.
\par 47 A gdy on jeszcze mówil, oto zgraja i ten, którego zwano Judaszem, jeden ze dwunastu, szedl przed nimi, i przyblizyl sie do Jezusa, aby go pocalowal.
\par 48 A Jezus mu rzekl: Judaszu! pocalowaniem wydajesz Syna czlowieczego?
\par 49 A widzac ci, którzy przy nim byli, co sie dziac mialo, rzekli mu: Panie! mamyli bic mieczem?
\par 50 I uderzyl jeden z nich sluge najwyzszego kaplana, i ucial mu ucho prawe.
\par 51 Ale Jezus odpowiadajac, rzekl: Zaniechajcie az póty; a dotknawszy sie ucha jego, uzdrowil go.
\par 52 I rzekl Jezus do onych, którzy byli przyszli przeciwko niemu, do przedniejszych kaplanów i przelozonych swiatyni, i do starszych: Wyszliscie jako na zbójce z mieczami i z kijami.
\par 53 Gdym na kazdy dzien bywal z wami w kosciele, nie sciagneliscie rak na mie; ale tac jest ona godzina wasza i moc ciemnosci.
\par 54 Pojmawszy go tedy, prowadzili go i przyprowadzili go w dom najwyzszego kaplana, a Piotr szedl za nim z daleka.
\par 55 A gdy oni rozniecili ogien w posrodku dworu i wespól usiedli, usiadl i Piotr miedzy nimi.
\par 56 A ujrzawszy go niektóra dziewka u ognia siedzacego, i pilnie mu sie przypatrzywszy, rzekla: I ten z nim byl.
\par 57 A on sie go zaprzal, mówiac: Niewiasto! Nie znam go.
\par 58 A po malej chwili ujrzawszy go drugi, rzekl: I tys jest z nich; ale Piotr rzekl: Czlowiecze! nie jestem.
\par 59 A gdy wyszla jakoby jedna godzina, ktos inszy twierdzil, mówiac: Prawdziwie i ten z nim byl; bo tez jest Galilejczyk.
\par 60 A Piotr rzekl: Czlowiecze! nie wiem, co mówisz; a zatem zaraz, gdy on jeszcze mówil, kur zapial.
\par 61 A Pan obróciwszy sie, spojrzal na Piotra. I wspomnial Piotr na slowo Panskie, jako mu byl powiedzial: Iz pierwej niz kur zapieje, trzykroc sie mnie zaprzesz.
\par 62 A Piotr wyszedlszy precz, gorzko plakal.
\par 63 Lecz mezowie, którzy wespól trzymali Jezusa, nasmiewali sie z niego, bijac go;
\par 64 A zakrywszy go, bili twarz jego i pytali go, mówiac: Prorokuj, kto jest, co cie uderzyl.
\par 65 I wiele innych rzeczy bluzniac mówili przeciwko niemu.
\par 66 A gdy byl dzien, zeszli sie starsi z ludu i najwyzsi kaplani i nauczeni w Pismie, a przywiedli go do rady swojej.
\par 67 Mówiac: Jezlis ty jest Chrystus, powiedz nam? I rzekl im: Chocbym wam powiedzial, nie uwierzycie.
\par 68 A jezlibym tez o co pytal, nie odpowiecie mi, ani mie wypuscicie.
\par 69 Od tego czasu bedzie Syn czlowieczy siedzial na prawicy mocy Bozej.
\par 70 I rzekli wszyscy: Tys tedy jest on syn Bozy? A on rzekl do nich: Wy powiadacie, zem ja jest.
\par 71 A oni rzekli: Cóz jeszcze potrzebujemy swiadectwa? Wszakiesmy sami slyszeli z ust jego.

\chapter{23}

\par 1 Tedy powstawszy wszystko mnóstwo ich, wiedli go do Pilata.
\par 2 I poczeli nan skarzyc, mówiac: Tegosmy znalezli, ze odwraca lud i zakazuje dani dawac cesarzowi, powiadajac: Ze on jest Chrystusem królem.
\par 3 I pytal go Pilat, mówiac: Tyzes jest on król zydowski? A on mu odpowiadajac rzekl: Ty powiadasz.
\par 4 I rzekl Pilat do przedniejszych kaplanów i do ludu: Zadnej winy nie znajduje w tym czlowieku.
\par 5 Lecz sie oni bardziej silili, mówiac: Iz wzrusza lud, uczac po wszystkiej Judzkiej ziemi, poczawszy od Galilei az dotad.
\par 6 Tedy Pilat uslyszawszy o Galilei, pytal, jezliby byl czlowiekiem Galilejskim?
\par 7 A gdy sie dowiedzial, iz byl z panstwa Herodowego, odeslal go do Heroda, który tez w Jeruzalemie byl w one dni.
\par 8 A Herod ujrzawszy Jezusa, uradowal sie bardzo; bo go z dawna pragnal widziec, dlatego, iz wiele o nim slyszal, i spodziewal sie, iz mial ujrzec jaki cud od niego uczyniony.
\par 9 I pytal go wiela slów; ale mu on nic nie odpowiadal.
\par 10 A przedniejsi kaplani i nauczeni w Pismie stali, poteznie skarzac nan.
\par 11 Ale wzgardziwszy nim Herod z zolnierstwem swem i nasmiawszy sie z niego, oblekl go w szate biala i odeslal go zas do Pilata.
\par 12 I stali sie sobie przyjaciolmi Pilat z Herodem onegoz to dnia; bo sobie byli przedtem nieprzyjaciolmi.
\par 13 A Pilat zwolawszy przedniejszych kaplanów i przelozonych, i ludu,
\par 14 Rzekl do nich: Oddaliscie mi tego czlowieka, jakoby lud odwracal: a oto ja przed wami pytajac go, zadnej winy nie znalazlem w tym czlowieku z tego, co nan skarzycie;
\par 15 Ale ani Herod, bom was odeslal do niego, a oto nic mu sie godnego smierci nie stalo;
\par 16 Przetoz skarawszy wypuszcze go.
\par 17 A musial im Pilat wypuszczac jednego na swieto.
\par 18 Tedy zawolalo spolem wszystko mnóstwo, mówiac: Strac tego a wypusc nam Barabbasza!
\par 19 Który byl dla niejakiego rozruchu w miescie uczynionego, i dla mezobójstwa wsadzony do wiezienia.
\par 20 Tedy Pilat znowu mówil, chcac wypuscic Jezusa.
\par 21 Ale oni przecie wolali, mówiac: Ukrzyzuj, ukrzyzuj go!
\par 22 A on po trzecie rzekl do nich: I cóz wzdy ten zlego uczynil? Zadnej winy smierci nie znalazlem w nim; przetoz skarawszy wypuszcze go.
\par 23 A oni przecie nalegali glosy wielkimi, zadajac, aby byl ukrzyzowany; i zmacnialy sie glosy ich i przedniejszych kaplanów.
\par 24 A tak Pilat przysadzil, aby sie dosyc stalo zadnosci ich.
\par 25 I wypuscil im onego, który byl dla rozruchu i mezobójstwa wsadzony do wiezienia, o którego prosili; ale Jezusa podal na wole ich.
\par 26 Gdy go tedy wiedli, uchwyciwszy Szymona niektórego Cyrenejczyka, idacego z pola, wlozyli nan krzyz, aby go niósl za Jezusem.
\par 27 I szlo za nim wielkie mnóstwo ludu i niewiast, które go plakaly i narzekaly.
\par 28 Ale Jezus obróciwszy sie do nich, rzekl: Córki Jeruzalemskie! nie placzcie nade mna, ale raczej same nad soba placzcie i nad dziatkami waszemi.
\par 29 Albowiemci oto ida dni, których beda mówic: Blogoslawione nieplodne i zywoty, które nie rodzily, i piersi, które nie karmily.
\par 30 Tedy poczna mówic górom: Padnijcie na nas! a pagórkom: Przykryjcie nas!
\par 31 Albowiem poniewaz sie to na zielonem drzewie dzieje, a cóz bedzie na suchem?
\par 32 Wiedzieni tez byli i inni dwaj zloczyncy, aby wespól z nim straceni byli.
\par 33 A gdy przyszli na miejsce, które zowia trupich glów, tam go ukrzyzowali, i onych zloczynców, jednego po prawicy, a drugiego po lewicy.
\par 34 Tedy Jezus rzekl: Ojcze! odpusc im: boc nie wiedza, co czynia. A rozdzieliwszy szaty jego, los o nie miotali.
\par 35 I stal lud, przypatrujac sie, a nasmiewali sie z niego i przelozeni z nimi, mówiac: Inszych ratowal, niechze ratuje samego siebie, jezlize on jest Chrystus, on wybrany Bozy.
\par 36 Nasmiewali sie tez z niego i zolnierze, przystepujac, a ocet mu podawajac,
\par 37 I mówiac: Jezlis ty jest król zydowski, ratujze samego siebie.
\par 38 A byl tez i napis napisany nad nim literami Greckiemi i Lacinskiemi i Zydowskiemi: Tenci jest on król zydowski.
\par 39 A jeden z onych zloczynców, którzy z nim wisieli, uragal mu, mówiac: Jezlizes ty jest Chrystus, ratujze siebie i nas.
\par 40 A odpowiadajac drugi, gromil go mówiac: I ty sie Boga nie boisz, chociazes jest w temze skazaniu?
\par 41 A myc zaiste sprawiedliwie; (bo godna zaplate za uczynki nasze bierzemy;) ale ten nic zlego nie uczynil.
\par 42 I rzekl do Jezusa: Panie! pomnij na mnie, gdy przyjdziesz do królestwa twego.
\par 43 A Jezus mu rzekl: Zaprawde powiadam tobie, dzis ze mna bedziesz w raju.
\par 44 A bylo okolo szóstej godziny, i stala sie ciemnosc po wszystkiej ziemi az do godziny dziewiatej.
\par 45 I zacmilo sie slonce, a zaslona koscielna rozerwala sie w pól.
\par 46 A Jezus zawolawszy glosem wielkim, rzekl: Ojcze! w rece twoje polecam ducha mojego; a to rzeklszy, skonal.
\par 47 A widzac setnik, co sie dzialo, chwalil Boga, mówiac: Zaprawde czlowiek to byl sprawiedliwy.
\par 48 Takze i wszystek lud, który sie byl zszedl na to dziwowisko, widzac, co sie dzialo, bijac sie w piersi swoje, wracal sie.
\par 49 A znajomi jego wszyscy z daleka stali, i niewiasty, które za nim byly przyszly z Galilei, przypatrujac sie temu.
\par 50 A oto maz, imieniem Józef, który byl senatorem, maz dobry i sprawiedliwy,
\par 51 Który byl nie zezwolil na rade i na uczynek ich, z Arymatyi, miasta Judzkiego, który tez oczekiwal królestwa Bozego;
\par 52 Ten przyszedlszy do Pilata, prosil o cialo Jezusowe.
\par 53 I zdjawszy je, obwinal je przescieradlem a polozyl je w grobie w opoce wykowanym, w którym jeszcze nikt nigdy nie byl polozony.
\par 54 A byl dzien przygotowania, i sabat nastawal.
\par 55 Poszedlszy tez za nim i niewiasty, które byly z nim przyszly z Galilei, ogladaly grób, i jako bylo polozone cialo jego.
\par 56 A wróciwszy sie, nagotowaly wonnych rzeczy i masci; ale w sabat odpoczely wedlug przykazania.

\chapter{24}

\par 1 A pierwszego dnia po sabacie bardzo rano przyszly do grobu, niosac rzeczy wonne, które byly nagotowaly i niektóre inne z niemi;
\par 2 I znalazly kamien odwalony od grobu.
\par 3 A wszedlszy w grób, nie znalazly ciala Pana Jezusowego.
\par 4 I stalo sie, gdy sie dlatego zatrwozyly, ze oto dwaj mezowie staneli przy nich w szatach swietnych.
\par 5 A gdy sie one baly i schylily twarz swoje ku ziemi, rzekli do nich: Cóz szukacie zyjacego miedzy umarlymi?
\par 6 Nie maszci go tu, ale wstal: wspomnijcie, jako wam powiadal, gdy jeszcze byl w Galilei,
\par 7 Mówiac: Iz Syn czlowieczy musi byc wydany w rece ludzi grzesznych, i byc ukrzyzowany, a trzeciego dnia zmartwychwstac.
\par 8 I wspomnialy na slowa jego.
\par 9 A wróciwszy sie od grobu, oznajmily to wszystko onym jedenastu i innym wszystkim.
\par 10 A byla Maryja Magdalena i Joanna, i Maryja, matka Jakóbowa, i inne z niemi, które to powiadaly Apostolom.
\par 11 Ale sie im zdaly jako plotki slowa ich, i nie wierzyli im.
\par 12 Tedy Piotr wstawszy, biezal do grobu, a nachyliwszy sie, ujrzal same tylko przescieradla lezace, i odszedl, dziwujac sie sam u siebie temu, co sie stalo.
\par 13 A oto dwaj z nich tegoz dnia szli do miasteczka, które bylo na szescdziesiat stajan od Jeruzalemu, które zwano Emaus.
\par 14 A ci rozmawiali z soba o tem wszystkiem, co sie bylo stalo.
\par 15 I stalo sie, gdy oni rozmawiali i wespól sie pytali, ze i Jezus przyblizywszy sie, szedl z nimi.
\par 16 Ale oczy ich byly zatrzymane, aby go nie poznali.
\par 17 I rzekl do nich: Cóz to za rozmowy, które macie miedzy soba idac, a jestescie smutni?
\par 18 A odpowiadajac mu jeden, któremu bylo imie Kleofas, rzekl mu: Tys sam przychodniem w Jeruzalemie, a nie wiesz, co sie w niem w tych dniach stalo?
\par 19 I rzekl im: Cóz? A oni mu rzekli: O Jezusie Nazarenskim, który byl maz prorok, mocny w uczynku i w mowie przed Bogiem i wszystkim ludem;
\par 20 A jako go wydali przedniejsi kaplani i przelozeni nasi, aby byl skazany na smierc; i ukrzyzowali go.
\par 21 A mysmy sie spodziewali, iz on mial odkupic Izraela; ale teraz temu wszystkiemu dzis jest trzeci dzien, jako sie to stalo.
\par 22 Lecz i niewiasty niektóre z naszych przestraszyly nas, które raniuczko byly u grobu;
\par 23 A nie znalazlszy ciala jego, przyszly powiadajac, iz widzenie Anielskie widzialy, którzy powiadaja, iz on zyje.
\par 24 I chodzili niektórzy z naszych do grobu, i tak znalezli, jako i niewiasty powiadaly; ale samego nie widzieli.
\par 25 Tedy on rzekl do nich: O glupi, a leniwego serca ku wierzeniu temu wszystkiemu, co powiedzieli prorocy!
\par 26 Azaz nie musial Chrystus tego cierpiec i wnijsc do chwaly swojej?
\par 27 A poczawszy od Mojzesza i od wszystkich proroków, wykladal im wszystkie one Pisma, które o nim napisane byly.
\par 28 I przyblizyl sie ku miasteczku, do którego szli, a on pokazywal, jakoby mial dalej isc.
\par 29 Ale go oni przymusili, mówiac: Zostan z nami, boc sie ma ku wieczorowi, i juz sie dzien nachylil. I wszedl, aby zostal z nimi.
\par 30 I stalo sie, gdy on siedzial z nimi za stolem, wziawszy chleb, blogoslawil, a lamiac podawal im.
\par 31 I otworzyly sie oczy ich, i poznali go; ale on zniknal z oczu ich.
\par 32 I mówili miedzy soba: Izali serce nasze nie palalo w nas, gdy z nami w drodze mówil, i gdy nam Pisma otwieral?
\par 33 A wstawszy onejze godziny, wrócili sie do Jeruzalemu, i znalezli zgromadzonych onych jedenascie, i tych, którzy z nimi byli,
\par 34 Powiadajacych: Iz wstal Pan prawdziwie, i ukazal sie Szymonowi.
\par 35 A oni tez powiedzieli, co sie stalo w drodze, i jako go poznali w lamaniu chleba.
\par 36 A gdy oni to mówili, stanal sam Jezus w posrodku nich, i rzekl im: Pokój wam!
\par 37 A oni przeleknawszy sie i przestraszeni bedac, mniemali, iz ducha widzieli.
\par 38 I rzekl im: Czemuscie sie zatrwozyli, i czemu mysli wstepuja do serc waszych?
\par 39 Ogladajcie rece moje i nogi moje, zemci ja jest on; dotykajcie sie mnie, a obaczcie; bo duch nie ma ciala ani kosci, jako widzicie, ze ja mam.
\par 40 A to rzeklszy, pokazal im rece i nogi.
\par 41 Lecz gdy oni jeszcze nie wierzyli od radosci, ale sie dziwowali, rzekl im: Macie tu co jesc?
\par 42 A oni mu podali sztuke ryby pieczonej i plastr miodu.
\par 43 A on wziawszy, jadl przed nimi.
\par 44 I rzekl do nich: Tec sa slowa, którem mówil do was, bedac jeszcze z wami, iz sie musi wypelnic wszystko, co napisano w zakonie Mojzeszowym i w prorokach, i w psalmach o mnie.
\par 45 Tedy im otworzyl zmysl, zeby rozumieli Pisma.
\par 46 I rzekl im: Takci napisano, i tak musial Chrystus cierpiec, i trzeciego dnia zmartwychwstac;
\par 47 I aby byla kazana w imieniu jego pokuta i odpuszczenie grzechów miedzy wszystkimi narody, poczawszy od Jeruzalemu.
\par 48 A wy jestescie swiadkami tego.
\par 49 A oto ja posle na was obietnice Ojca mego, a wy zostancie w miescie Jeruzalemie, dokad nie bedziecie przyobleczeni moca z wysokosci.
\par 50 I wywiódl je precz az do Betanii, a podnióslszy rece swoje blogoslawil im.
\par 51 I stalo sie, gdy im blogoslawil, rozstal sie z nimi, i byl niesiony w góre do nieba.
\par 52 A oni pokloniwszy mu sie, wrócili sie do Jeruzalemu z radoscia wielka.
\par 53 I byli zawsze w kosciele, chwalac i blogoslawiac Boga. Amen.


\end{document}