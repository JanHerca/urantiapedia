\begin{document}

\title{2 Królewska}


\chapter{1}

\par 1 I odstapil Moab od Izraela po smierci Achabowej.
\par 2 A Ochozyjasz spadl przez krate sali swej, która mial w Samaryi, i rozniemógl sie. I wyprawil posly, mówiac im: Idzcie, poradzcie sie Beelzebuba, boga Akkaronskiego, jezeli powstane z tej choroby.
\par 3 Ale Aniol Panski rzekl do Elijasza Tesbity: Wstan, idz przeciwko poslom króla Samaryi, i mów do nich: Izali niemasz Boga w Izraelu, iz sie idziecie radzic Beelzebuba, boga Akkaronskiego?
\par 4 Przetoz tak mówi Pan: Z loza, na któres sie polozyl, nie wstaniesz, ale pewnie umrzesz. I odszedl Elijasz.
\par 5 A gdy sie poslowie wrócili do niego, rzekl do nich: Czemuzescie sie wrócili?
\par 6 Odpowiedzieli mu: Maz niektóry zaszedl nam droge, i mówil do nas: Idzcie, wróccie sie do króla, który was poslal, i rzeczcie mu: Tak mówi Pan: Izaliz niemasz Boga w Izraelu, ze sie posylasz radzic Beelzebuba, boga Akkaronskiego? Przetoz z loza na któres sie polozyl, nie wstaniesz, ale pewnie umrzesz.
\par 7 I rzekl do nich: Cóz za osoba byla tego meza, który wam zaszedl droge, i mówil do was te slowa?
\par 8 I opowiedzieli mu: Maz kosmaty, a pasem skórzanym przepasany na biodrach swych. I rzekl: Elijasz Tesbita jest.
\par 9 Przetoz poslal do niego piecdziesiatnika z piecdziesiecioma jego, który poszedl do niego, (a oto siedzial na wierzchu góry,)i rzekl mu: Mezu Bozy, król rozkazal, abys zstapil.
\par 10 A odpowiadajac Elijasz, rzekl piecdziesiatnikowi: Jezlizem jest maz Bozy, niech ogien zstapi z nieba, a pozre ciebie i piecdziesieciu twoich. Zstapil tedy ogien z nieba, i pozarl go i piecdziesieciu jego.
\par 11 Znowu poslal do niego piecdziesiatnika drugiego z piecdziesiecioma jego, który mówil do niego, i rzekl: Mezu Bozy, tak mówi król: Rychlo zstap.
\par 12 I odpowiedzial Elijasz, a rzekl mu: Jezlim jest maz Bozy, niech zstapi ogien z nieba, a pozre ciebie i piecdziesiat twoich. Tedy zstapil ogien Bozy z nieba, i pozarl go i piecdziesieciu jego.
\par 13 Tedy jeszcze poslal piecdziesiatnika trzeciego z piecdziesiecioma jego. Przetoz poszedl piecdziesiatnik on trzeci, a przyszedlszy pokleknal na kolana swoje przed Elijaszem, a proszac go pokornie, mówil do niego: Mezu Bozy, prosze niech bedzie droga dusza moja, i dusza tych slug twoich piecdziesieciu w oczach twoich:
\par 14 Oto zstapil ogien z nieba, i pozarl dwóch piecdziesiatników pierwszych z piecdziesieciu ich; ale teraz niech bedzie droga dusza moja w oczach twoich.
\par 15 I rzekl Aniol Panski do Elijasza: Zstap z nim, nie bój sie twarzy jego. Który wstawszy poszedl z nim do króla.
\par 16 I rzekl mu: Tak mówi Pan: Przeto, zes wyprawil posly radzic sie Beelzebuba, boga Akkaronskiego, jakoby Boga nie bylo w Izraelu, abys sie pytal slowa jego, dlatego z loza, na któremes sie polozyl, nie wstaniesz, ale pewnie umrzesz.
\par 17 A tak umarl wedlug slowa Panskiego, które mówil Elijasz. I królowal Joram miasto niego, roku wtórego Jorama, syna Jozafatowego, króla Judzkiego; albowiem on nie mial syna.
\par 18 A inne sprawy Ochozyjaszowe, które czynil, azaz nie sa napisane w kronikach o królach Izraelskich?

\chapter{2}

\par 1 I stalo sie, gdy mial wziac Pan Elijasza w wichrze do nieba, ze wyszedl Elijasz z Elizeuszem z Galgal.
\par 2 I rzekl Elijasz do Elizeusza: Prosze siedz tu; bo mie Pan poslal az do Betel. I rzekl Elizeusz: Jako zywy Pan, i jako zywa dusza twoja, ze sie ciebie nie puszcze. I przyszli do Betel.
\par 3 Tedy wyszli synowie proroccy, którzy byli w Betel, do Elizeusza, i rzekli do niego: Wieszze, iz dzis Pan wezmie od ciebie pana twego? A on rzekl: Wiemci; milczcie tylko.
\par 4 Znowu rzekl mu Elijasz: Elizeuszu, prosze siedz tu; bo mie Pan poslal do Jerycha. A on odpowiedzial: Jako zywy Pan, i jako zywa dusza twoja, ze sie ciebie nie puszcze. A tak przyszli do Jerycha.
\par 5 Tedy przystapiwszy synowie proroccy, którzy byli w Jerychu, do Elizeusza, rzekli do niego: Wieszze, ze dzis Pan wezmie pana twego od ciebie? A on rzekl: Wiemci; milczcie.
\par 6 Jeszcze mu rzekl Elijasz: Prosze siedz tu; bo mie Pan poslal do Jordanu. Który odpowiedzial: Jako zywy Pan, jako zywa i dusza twoja, ze sie ciebie nie puszcze.
\par 7 I szli obadwaj. A piecdziesiat mezów synów prorockich szli, i staneli naprzeciwko z daleko; ale oni obaj staneli nad Jordanem.
\par 8 A wziawszy Elijasz plaszcz swój, zwinal go, a uderzyl nim wody, i rozdzielily sie tam i sam, tak iz przeszli obaj po suszy.
\par 9 A gdy przeszli, rzekl Elijasz do Elizeusza: Zadaj, czego chcesz, abym ci uczynil pierwej niz bede wziety od ciebie. Tedy rzekl Elizeusz: Prosze niech bedzie dwójnasobny duch twój we mnie;
\par 10 Ale mu on odpowiedzial: Trudnejs rzeczy pozadal; wszakze ujrzyszli mie, gdy bede wziety od ciebie, tak ci sie stanie; ale jezli nie ujrzysz, nie staniec sie.
\par 11 I stalo sie, gdy oni przecie szli rozmawiajac, oto wóz ognisty, i konie ogniste rozlaczyly obydwóch. I wstapil Elijasz w wichrze do nieba.
\par 12 Co Elizeusz widzac, wolal: Ojcze mój, ojcze mój! Wozie Izraelski i jazdo jego. I nie widzial go wiecej. A pochwyciwszy szaty swe rozdarl je na dwie czesci.
\par 13 I podniósl plaszcz Elijaszowy, który byl spadl z niego, a wróciwszy sie, stanal nad brzegiem Jordanu. A tak wziawszy plaszcz Elijaszowy, który byl spadl z niego, uderzyl nim wody, mówiac: Gdziez jest Pan, Bóg Elijaszowy?
\par 14 A tak i on uderzyl nim wody, a rozdzielily sie tam i sam, i przeszedl Elizeusz.
\par 15 Co widzac synowie proroccy, którzy byli w Jerycho, stojac na przeciwko, rzekli: Odpoczal duch Elijaszowy nad Elizeuszem; a wyszedlszy przeciwko niemu poklonili mu sie az do ziemi.
\par 16 I rzekli do niego: Oto teraz jest przy slugach twych piecdziesiat mezów mocnych. Prosze niech ida, a niech szukaja Pana twego; by go snac nie zaniósl Duch Panski, a nie porzucil go na której górze, albo w której dolinie. Ale im on rzekl: Nie posylajcie.
\par 17 A gdy nan nalegali az do uprzykrzenia, rzekl: Poslijciez. A tak poslali onych piecdziesiat mezów, którzy szukajac przez trzy dni nie znalezli go.
\par 18 A gdy sie wrócili do niego, (a on mieszkal w Jerycho,)rzekl do nich: Azazem wam nie mówil: Nie chodzcie?
\par 19 Rzekli tez mezowie onego miasta do Elizeusza: Wej, oto mieszkanie miasta tego jest dobre, jako panie mój widzisz; ale wody zle i ziemia nieplodna.
\par 20 Tedy rzekl: Przyniescie mi banke nowa, a wlózcie w nie soli. I przyniesli mu.
\par 21 A poszedlszy do zródla wód, wrzucil tam soli, i rzekl: Tak mówi Pan: Uzdrowilem te wody; nie bedzie wiecej stamtad smierci, ani nieplodnosci.
\par 22 A tak uzdrowione sa one wody az do dnia tego, wedlug slowa Elizeuszowego, które byl powiedzial.
\par 23 Potem szedl stamtad do Betel. A gdy szedl droga, dzieci male wyszly z miasta, i nasmiewaly sie z niego, i mówily mu: Idzze lysy, idzze lysy!
\par 24 Który obejrzawszy sie, ujrzal je, i zlorzeczyl im w imieniu Panskiem. Przetoz wyszedlszy dwie niedzwiedzice z lasu, rozdrapaly z nich czterdziesci i dwoje dzieci.
\par 25 I szedl stamtad na góre Karmel, a z onad zasie wrócil sie do Samaryi.

\chapter{3}

\par 1 A Joram, syn Achaba, poczal królowac nad Izraelem w Samaryi roku osmnastego Jozafata, króla Judzkiego, a królowal dwanascie lat.
\par 2 I czynil zle przed oczyma Panskiemi, acz nie tak jako ojciec jego, i jako matka jego. Albowiem wyrzucil slupy Baalowe, których byl naczynil ojciec jego.
\par 3 Wszakze w grzechach Jeroboama, syna Nabatowego, który przywiódl do grzechu Izraela, trwal a nie odstapil od nich.
\par 4 A Meza, król Moabski, mial dosyc bydla, a dawal królowi Izraelskiemu sto tysiecy jagniat, i sto tysiecy baranów z welna ich.
\par 5 I stalo sie, gdy umarl Achab, ze odstapil król Moabski od króla Izraelskiego.
\par 6 Wyciagnal tedy król Joram dnia onego z Samaryi, i obliczyl wszystkiego Izraela.
\par 7 A wyszedlszy poslal do Jozafata, króla Judzkiego, mówiac: Król Moabski odstapil odemnie; pociagnieszze zemna przeciw Moabowi na wojne? I odpowiedzial: Pociagne. Jakom ja, tak ty; jako lud mój, tak lud twój; jako konie moje, tak konie twoje.
\par 8 Zatem rzekl: Któraz droga pociagniemy? I odpowiedzial: Droga puszczy Edomskiej.
\par 9 A tak wyciagnal król Izraelski i król Judzki, i król Edomski. A gdy krazyli droga siedm dni, nie dostalo wody wojsku, ani bydlu, które szlo z nimi.
\par 10 I rzekl król Izraelski: Ach, ach! Albowiem wezwal Pan tych trzech królów, aby je podal w rece Moabskie.
\par 11 Ale Jozafat rzekl: Niemaszze tu proroka Panskiego, zebysmy sie poradzili Pana przezen? I odpowiedzial jeden z slug króla Izraelskiego, a rzekl: Jest tu Elizeusz, syn Safata, który nalewal wody na rece Eliaszowe.
\par 12 Tedy rzekl Jozafat: U tegoc jest slowo Panskie. I szli do niego król Izraelski, i Jozafat, i król Edomski.
\par 13 I rzekl Elizeusz do króla Izraelskiego: Co mnie i tobie? Idz do proroków ojca twego, i do proroków matki twej. I rzekl mu król Izraelski: Nie mów tak; bo Pan powolal tych trzech królów, aby je podal w rece Moabowe.
\par 14 I odpowiedzial Elizeusz: Jako zywy Pan zastepów, przed którego obliczem stoje, ze gdybym sie nie ogladal na Jozafata, króla Judzkiego, nie dbalbym na cie, anibym na cie wejrzal.
\par 15 Przetoz teraz przywiedzcie mi na harfie grajacego. A gdy on gracz gral, byla nad nim reka Panska.
\par 16 I rzekl: Tak mówi Pan: Poczyncie w tym potoku geste doly.
\par 17 Albowiem tak mówi Pan: Nie ujrzycie wiatru, ani ujrzycie deszczu, wszakze ten potok bedzie pelen wody, tak, ze pic bedziecie wy, i stada wasze, i bydla wasze.
\par 18 Alec to jeszcze mala w oczach Panskich; albowiem da i Moabity w rece wasze.
\par 19 I poburzycie wszystkie miasta obronne, i kazde miasto wyborne, a wszystkie drzewa dobre podrabicie, i wszystkie zródla wód zatkacie, i kazda role dobra kamieniem zawalicie.
\par 20 I stalo sie rano, kiedy ofiarowana bywa ofiara sniedna, oto wody przychodzily droga Edomska, i napelnila sie ziemia wodami.
\par 21 Tedy wszyscy Moabitowie uslyszawszy, ze ciagna królowie walczyc przeciwko nim, zwolali wszystkie, którzy tak starzy byli, ze pas przypasac mogli, i wyzej; a staneli na granicach.
\par 22 A wstawszy rano, gdy slonce weszlo nad temi wodami, ujrzeli Moabczycy naprzeciw sobie wody czerwone jako krew,
\par 23 I rzekli: Krew jest; pewnie sie pobili królowie, i pobici sa jeden od drugiego. A tak teraz do lupu, o Moabczycy!
\par 24 A gdy przyszli do obozu Izraelskiego, powstali Izraelczycy, i porazili Moabczyki, którzy uciekali przed nimi, a oni je bili, i porazili Moabczyki;
\par 25 I miasta poburzyli, i na kazde pole wyborne rzucil kazdy kamien swój, i zarzucili je, i wszystkie zródla wód pozatykali, i wszystkie drzewa dobre podrabali, tak, ze tylko zaniechali kamienia w murze Kichareset. A obleglszy je ci, co byli z procami dobywali go.
\par 26 Tedy widzac król Moabski, ze przemagalo przeciw niemu wojsko, wzial z soba siedm set mezów dobywajacych miecza, zeby sie przebil przez wojsko króla Edomskiego, ale nie mogli.
\par 27 Przetoz pojmawszy syna jego pierworodnego, który mial królowac miasto niego, ofiarowal go calopaleniem na murze. I stalo sie zagniewanie wielkie, przeciw Izraelowi, i odciagneli od niego, a wrócili sie do ziemi swej.

\chapter{4}

\par 1 A niewiasta jedna z zon synów prorockich wolala do Elizeusza, mówiac: Sluga twój, maz mój, umarl. A ty wiesz, iz sluga twój bal sie Pana. A teraz przyszedl pozyczalnik, aby sobie wzial dwóch synów moich za niewolniki.
\par 2 Do której rzekl Elizeusz: Cóz ci mam uczynic? Powiedz, mi co masz w domu? A ona odpowiedziala: Nie ma sluzebnica twoja nic wiecej w domu, jedno banke oliwy.
\par 3 I rzekl: Idzze, napozyczaj sobie naczynia z inad u wszystkich sasiadek twoich, naczynia próznego nie malo.
\par 4 A wszedlszy zamknij drzwi za soba i za synami twymi, a nalej we wszystkie te naczynia, a które bedzie pelne, rozkaz odstawic.
\par 5 A tak odszedlszy od niego, zamknela drzwi za soba i za synami swymi.(Oni przynosili do niej, a ona nalewala.)
\par 6 I stalo sie, gdy napelnila one naczynia, rzekla do syna swego: Przynies mi jeszcze naczynie. A on jej odpowiedzial: Niemasz wiecej naczynia. I zastanowila sie oliwa.
\par 7 Potem ona przyszedlszy, oznajmila to mezowi Bozemu, który do niej rzekl: Idzze sprzedaj te oliwe, a oddaj pozyczalnikowi twemu, a ty i synowie twoi zywcie sie ostatkiem.
\par 8 Stalo sie potem czasu niektórego, iz szedl Elizeusz przez Sunem, gdzie byla niewiasta zacna, która go zatrzymywala, aby jadl chleb; a tak ile kroc tamtedy chodzil, wstepowal do niej, aby jadl chleb.
\par 9 Bo rzekla byla do meza swego: Oto teraz wiem, ze ten maz Bozy swiety jest, który tedy przechodzi czesto.
\par 10 Prosze, uczynmy gmaszek maly, a postawmy mu tam lózko i stól, i krzeslo i lichtarz, ze kiedy przyjdzie do nas, skloni sie tam.
\par 11 A tak dnia jednego, gdy tam przyszedl, sklonil sie do onego gmaszku, i odpoczal tam.
\par 12 I rzekl do Giezego, slugi swego: Zawolaj tej Sunamitki. I zawolal jej, a stanela przed nim.
\par 13 Tedy mu rzekl: Powiedz jej: Oto pieczolujesz a starasz sie o wszystki nasze potrzeby; cóz chcesz, abym ci uczynil? Maszze jaka potrzebe u króla, albo u hetmana wojska? A ona rzekla: W posrodku ludu mego mieszkam.
\par 14 A on rzekl: Cóz wzdy mam uczynic dla niej? I odpowiedzial Giezy: Oto syna nie ma, a maz jej stary.
\par 15 Przetoz rzekl: Zawolajze jej. I zawolal jej, a ona stala u drzwi.
\par 16 I rzekl: O tym czasie po roku bedziesz piastowala syna. A ona rzekla: Nie omylajze, panie mój, mezu Bozy, nie omylaj sluzebnicy twojej.
\par 17 A tak poczela niewiasta, i porodzila syna o onymze czasie po roku, jako jej byl powiedzial Elizeusz.
\par 18 I podroslo dziecie. I stalo sie dnia niektórego, ze wyszedlszy do ojca swego, do zenców,
\par 19 Rzeklo do ojca swego: Glowa moja! Glowa moja! A on rzekl sludze: Zanies go do matki jego.
\par 20 Który wziawszy go, zaniósl go do matki jego; i siedzial na lonie jej az do poludnia i umarl.
\par 21 Tedy ona szedlszy polozyla go na lózku meza Bozego, a zamknawszy drzwi, wyszla.
\par 22 Potem przyzwala meza swego, i rzekla: Prosze cie, poslij ze mna jednego z slug, i jedne oslice, ze pobieze az do meza Bozego, i wróce sie zas.
\par 23 Który rzekl: Po cóz chcesz jechac do niego? Dzis nie masz nowiu miesiaca, ani sabatu. Ale ona rzekla: Daj pokój.
\par 24 A tak osiodlawszy oslice, rzekla do slugi swego: Poganiaj, a jedz, i nie mieszkaj dla mnie w drodze, chyba zebym ci rozkazala.
\par 25 Jechala tedy, i przyjechala do meza Bozego na góre Karmel. A gdy ja ujrzal maz Bozy z daleka, rzekl do Giezego slugi swego: Oto ona Sunamitka.
\par 26 Przetoz wynijdz przeciwko niej, a rzecz jej: A zdrowas dobrze? zdrów i maz twój? zdrów i syn?
\par 27 A ona rzekla: Zdrowi dobrze. A gdy przyszla do meza Bozego na góre, uchwycila sie nóg jego; i przystapil Giezy, aby ja odepchnal. Ale maz Bozy rzekl: Zaniechaj jej, boc w gorzkosci jest dusza jej, a Pan zatail przedemna i nie oznajmil mi.
\par 28 A ona rzekla: Azazem pana mego prosila o syna? Izalim nie mówila: Nie omylaj mie?
\par 29 Tedy on rzekl do Giezego: Przepasz biodra twe, a wezmij laske moje w reke twa, a idz; jezli kogo spotkasz, nie pozdrawiaj go; a jezli by cie kto pozdrowil, nie odpowiadaj mu; i polóz laske moje na oblicze dzieciecia.
\par 30 A matka dzieciecia onego rzekla: Jako zywy Pan, i jako zywa dusza twoja, ze sie ciebie nie puszcze. A tak wstawszy szedl za nia.
\par 31 A Giezy uprzedzil je, i polozyl laske na oblicze dzieciecia; lecz nie bylo glosu, ani czucia. Przetoz sie wrócil przeciwko niemu i oznajmil mu, mówiac: Nie ocucilo sie dziecie.
\par 32 Tedy wszedl Elizeusz do domu, a oto dziecie umarle lezalo na lózku jego.
\par 33 A gdy tam wszedl, zamknal drzwi przed onymi obydwoma, i modlil sie Panu.
\par 34 Potem wstapiwszy na loze, polozyl sie na dzieciatko, przylozywszy usta swe do ust jego, a oczy swe do oczów jego, i rece swe do rak jego, i rozpostarl sie na niem, tak iz sie zagrzalo cialo dzieciece.
\par 35 A odwróciwszy sie od niego, przechadzal sie po domu tam i sam; potem wstapil, a polozyl sie na niem. Tedy kichalo dziecie az do siódmego razu, i otworzylo dziecie oczy swoje.
\par 36 Tedy zawolal Giezego, i rzekl: Zawolaj tej Sunamitki. I zawolal jej, i przyszla do niego, i rzekl: Wezmij syna twego,
\par 37 Która wszedlszy, upadla u nóg jego, i klaniala sie az do ziemi, a wziawszy syna swego, wyszla.
\par 38 Potem wrócil sie Elizeusz do Galgal, a glód byl w onej ziemi, i synowie proroccy mieszkali przy nim. Tedy rzekl do slugi swego: Przystaw garniec wielki, a uwarz kasze synom prorockim.
\par 39 Przetoz wyszedl jeden na pole, aby zbieral ziola, i znalazl macice polna, a nazbieral z niej owoców polnych pelen plaszcz swój, a przyszedlszy nakrajal ich w garniec kaszy; bo tego nie znali.
\par 40 I wylali mezom onym, aby jedli. A gdy jedli one kasze zawolali, i rzekli: Smierc w garncu, mezu Bozy! I nie mogli jesc.
\par 41 I rzekl: Przyniescie sami maki; a wsypawszy ja w garniec rzekl: Nalej ludowi. I jedli, i nie bylo nic wiecej zlego w garncu.
\par 42 Wtem maz przyszedl z Baalsalisa, a przyniósl mezowi Bozemu chleby, z pierwocin zbóz, dwadziescia chlebów jeczmiennych, i klosów pelnych swiezych nie wykruszonych, i rzekl: Daj ludowi, aby jedli.
\par 43 Ale odpowiedzial sluga jego: Cóz to mam dac przed sto mezów? I rzekl: Daj ludowi, aby jedli; albowiem tak mówi Pan: Beda jedli, i zbedzie.
\par 44 A tak polozyl przed nie; i jedli, a zbylo wedlug slowa Panskiego.

\chapter{5}

\par 1 A Naaman, hetman wojska króla Syryjskiego, byl maz wielki u pana swego, i osoba zacna. Albowiem przezen dal byl Pan wybawienie Syryjczykom; a ten maz byl duzy w sile, ale tredowaty.
\par 2 A z Syryi wyszla byla swawolna kupa, która pojmala z ziemi Izraelskiej dzieweczke nie wielka, a ta sluzyla zonie Naamanowej.
\par 3 Która rzekla do pani swej: O gdyby sie pan mój dostal do proroka, który jest w Samaryi! pewnieby go uzdrowil od tradu jego.
\par 4 Wszedl tedy Naaman, i oznajmil to panu swemu, mówiac: Tak a tak mówila dzieweczka, która jest z ziemi Izraelskiej.
\par 5 Na co odpowiedzial król Syryjski: Idz, wypraw sie, a posle list do króla Izraelskiego. A tak jechal, wziawszy z soba dziesiec talentów srebra, i szesc tysiecy zlotych, i dziesiecioro szat odmiennych.
\par 6 I przyniósl list do króla Izraelskiego w te slowa: Jako cie predko dojdzie ten list, wiedz, zem poslal do ciebie Naamana, sluge mego, abys go uzdrowil od tradu jego.
\par 7 A gdy przeczytal król Izraelski list, rozdarl odzienie swoje, mówiac: Azazem ja jest Bóg, zebym mógl umorzyc i ozywic, iz ten do mnie sle, abym uzdrowil meza tego od tradu jego? Uwazcie prosze, a obaczcie, ze szuka przyczyny na mie.
\par 8 Co gdy uslyszal Elizeusz, maz Bozy, iz rozdarl król Izraelski szaty swe, poslal do króla, mówiac: Przeczzes rozdarl szaty swe? Niech przyjdzie do mnie, a dowie sie, ze jest prorok w Izraelu.
\par 9 A tak przyjechal Naaman z konmi swemi, i z wozem swym, i stanal u drzwi domu Elizeuszowego.
\par 10 I wyslal do niego Elizeusz posla, mówiac: Idz, a omyj sie siedm kroc w Jordanie, a przywrócic sie zdrowie ciala twego, i bedziesz oczyszczony.
\par 11 Tedy rozgniewawszy sie Naaman, bral sie w droge, mówiac: Otom myslal sam u siebie, iz pewnie wynijdzie, a stanawszy przy mnie, wzywac bedzie imienia Pana, Boga swego, podnióslszy reke swoje nad miejscem tradu, uzdrowi tredowatego.
\par 12 Azaz nie lepsze sa rzeki Abana i Farfar w Damaszku nad wszystkie wody Izraelskie? izalibym sie niemógl w nich omyc, abym sie oczyscil! A tak obróciwszy sie, odjezdzal z gniewem.
\par 13 Ale przystapiwszy sludzy jego, mówili do niego, i rzekli: Ojcze mój, gdybyc byl co wielkiego ten prorok rozkazal, azazbys nie mial tego uczynic? Jako daleko wiecej, gdy rzekl: Omyj sie, a bedziesz czystym?
\par 14 Przetoz szedlszy omyl sie w Jordanie siedm kroc wedlug slowa meza Bozego; i stalo sie cialo jego, jako cialo dzieciecia malego, i oczyszczony jest.
\par 15 Potem sie wrócil do meza Bozego, on i wszystek poczet jego, a przyszedlszy stanal przed nim, i rzekl: Otom teraz poznal, ze nie masz Boga na wszystkiej ziemi, tylko w Izraelu; przetoz wezmij prosze te upominki od slugi twego.
\par 16 A on rzekl: Jako zywy Pan, przed którego obliczem stoje, ze nic nie wezme; a choc go przymuszal, aby wzial, przecie nie chcial.
\par 17 I rzekl Naaman: A nie chcesz? niechze bedzie dane prosze sludze twemu brzemie ziemi na dwa muly; boc nie bedzie wiecej sprawowal sluga twój calopalenia i innych ofiar bogom cudzym, jedno Panu.
\par 18 Wszakze w tej mierze niech odpusci Pan sludze swemu, gdy wchodzi pan mój do kosciola Remmon, aby sie tam klanial, a wesprze sie na rece mojej, ze sie i ja klaniam w kosciele Remmon. Takowe moje klanianie w kosciele Remmon prosze niech odpusci Pan sludze twemu w tej mierze.
\par 19 I rzekl mu: Idz w pokoju. A gdy odjechal od niego, jakoby na mile drogi,
\par 20 Rzekl Giezy, sluga Elizeusza, meza Bozego: Oto nie dopuscil pan mój temu Naamanowi Syryjskiemu, aby dal z reki swej, co byl przywiózl; jako zywy Pan, ze pobieze za nim, a wezme co od niego.
\par 21 A tak biezal Giezy za Naamanem. Którego ujrzawszy Naaman biezacego za soba, skoczyl z wozu przeciw niemu, i rzekl: Dobrzez sie wszystko dzieje?
\par 22 Któremu odpowiedzial: Dobrze. Pan mój poslal mie, abym ci powiedzial: Oto dopiero teraz przyszli do mnie dwaj mlodziency z góry Efraim z synów prorockich; dajze im prosze talent srebra, i dwie odmienne szaty.
\par 23 Tedy rzekl Naaman; Radniej wezmij dwa talenty. I przymusil go, i zawiazal dwa talenty srebra we dwa worki, i dwie odmienne szaty, i wlozyl na dwóch slug swoich, którzy niesli przed nim.
\par 24 A przyszedlszy na pagórek, wzial to z reki ich, i zlozyl w niektórym domu, a meze one odprawil, i odeszli.
\par 25 Potem przyszedlszy stanal przed panem swym. I rzekl do niego Elizeusz: Skadze Giezy? A on odpowiedzial: Nie chodzil nigdzie sluga twój.
\par 26 Ale mu on rzekl: Azaz serce moje nie chodzilo z toba, kiedy sie obrócil on maz z wozu swego przeciwko tobie? azaz czas byl do brania srebra, i do brania szat, i oliwnic, i winnic, i bydla, i wolów, i slug, i sluzebnic?
\par 27 Przetoz trad Naamanowy przylgnie do ciebie, i do nasienia twego na wieki. I wyszedl od twarzy jego tredowaty, jako snieg.

\chapter{6}

\par 1 I rzekli synowie proroccy do Elizeusza: Oto miejsce, na którem mieszkamy przed toba, ciasne jest dla nas.
\par 2 Niech idziemy prosze az do Jordanu, a wezmiemy stamtad kazdy po jednem drzewie, i zbudujemy tam sobie miejsce ku mieszkaniu. Którym on rzekl: Idzcie.
\par 3 I rzekl jeden z nich: Pójdz prosze i ty z slugami twoimi. A on rzekl: I ja pójde; i szedl z nimi.
\par 4 A przyszedlszy do Jordanu, rabali drzewo.
\par 5 I stalo sie, gdy jeden z nich obalal drzewo, ze mu siekiera wpadla w wode; i zawolal, mówiac: Ach, ach, panie mój! i tac byla pozyczona.
\par 6 Rzekl tedy maz Bozy: Gdziez upadla? i ukazal mu miejsce. A on uciawszy drewno, wrzucil tam, i sprawil, ze wyplynela ona siekiera.
\par 7 I rzekl: Wezmij ja sobie; który sciagnawszy reke swa, wzial ja.
\par 8 A gdy król Syryjski walczyl z Izraelem, i naradzal sie z slugami swoimi, mówiac: Na tem a na tem miejscu polozy sie wojsko moje;
\par 9 Tedy poslal maz Bozy do króla Izraelskiego, mówiac: Strzez sie, abys nie przechodzil przez ono miejsce; bo tam Syryjczycy sa na zasadzce.
\par 10 Przetoz poslal król Izraelski na ono miejsce, o którem mu byl powiedzial maz Bozy, i przestrzegl go, aby sie go chronil, nie raz ani dwa.
\par 11 A tak zatrwozylo sie serce króla Syryjskiego, dla tego. Przetoz zwolawszy slug swoich, rzekl do nich: Czemuz mi nie powiecie, kto wzdy z was donosi to królowi Izraelskiemu?
\par 12 I rzekl jeden z slug jego: Nie tak, królu, panie mój; ale Elizeusz prorok, który jest w Izraelu, oznajmuje królowi Izraelskiemu slowa, które mówisz w tajemnym pokoju twoim.
\par 13 A on rzekl: Idzcie, a dowiedzcie sie, gdzie jest, abym poslal i pojmal go. I powiedziano mu, mówiac: Oto jest w Dotanie.
\par 14 Przetoz poslal tam konie i wozy z wielkiem wojskiem, którzy przyciagnawszy w nocy, oblegli miasto.
\par 15 Tedy wstawszy rano sluga meza Bozego, wyszedl, a oto wojsko otoczylo miasto, i konie, i wozy. I rzekl sluga jego do niego: Ach panie mój! Cóz mamy czynic?
\par 16 A on odpowiedzial: Nie bój sie; bo wiecej ich z nami, niz z nimi.
\par 17 Modlil sie tedy Elizeusz, i rzekl: O Panie, otwórz prosze oczy jego, zeby widzial. I otworzyl Pan oczy slugi onego, i ujrzal, a oto góra pelna koni, i wozy ogniste okolo Elizeusza.
\par 18 A gdy Syryjczycy szli do niego, modlil sie Elizeusz Panu, mówiac: Prosze, zaraz ten lud slepota. I zarazil je Pan slepota wedlug slowa Elizeuszowego.
\par 19 Wtem rzekl do nich Elizeusz: Nie tac to droga, ani to miasto. Pójdzcie za mna, a zawiode was do meza, którego szukacie. I przywiódl je do Samaryi,
\par 20 A gdy weszli do Samaryi, rzekl Elizeusz: O Panie, otwórz oczy tych, aby przejrzeli. I otworzyl Pan oczy ich, i widzieli, ze byli w posród Samaryi.
\par 21 I rzekl król Izraelski do Elizeusza, gdy je ujrzal;
\par 22 Mamze je pobic, ojcze mój? Ale on rzekl: Nie bij. Azazes je wzial przez miecz twój, albo przez luk twój, zebys je mial pobic? Polóz chleb i wode przed nie, aby jedli i pili, i wrócili sie do pana swego.
\par 23 A tak przygotowal dla nich dostatek wielki, i jedli i pili; i puscil je, i odeszli do pana swego. I nie wazyly sie wiecej wojska Syryjskie wpadac do ziemi Izraelskiej.
\par 24 Stalo sie potem, ze zebral Benadad, król Syryjski, wszystkie wojska swe, a przyciagnal i oblegl Samaryje.
\par 25 Przetoz byl glód wielki w Samaryi; albowiem ja bylo oblezono, tak, iz glowe osla sprzedawano za osmdziesiat srebrników, a czwarta czesc miary gnoju golebiego za piec srebrników.
\par 26 I przydalo sie, gdy król Izraelski przechadzal sie po murze, ze jedna niewiasta zawolala nan mówiac: Ratuj mie królu, panie mój!
\par 27 Który rzekl: nie ratujeli cie Pan, skadze ja ciebie poratuje? izali z gumna, czyli z prasy?
\par 28 Nadto rzekl jej król: Cóz ci? A ona rzekla: Ta niewiasta rzekla do mnie: Daj syna twego, zebysmy go zjadly dzisiaj, a jutro zjemy syna mego.
\par 29 I uwarzylysmy syna mego, i zjadlysmy go. Potem rzeklam jej dnia drugiego: Daj syna twego, abysmy go zjadly; ale ona skryla syna swego.
\par 30 A gdy król uslyszal slowa onej niewiasty, rozdarl odzienie swe; a gdy sie przechodzil po murze, widzial lud, ze wór byl na ciele jego od spodku.
\par 31 Tedy rzekl król: To niechaj mi uczyni Bóg, i to przyczyni, jezli sie glowa Elizeusza, syna Safatowego, na nim dzisiaj ostoi.
\par 32 (A Elizeusz siedzial w domu swoim, i starcy siedzieli z nim.) I poslal meza z tych, którzy przed nim stali; a pierwej, niz on posel przyszedl do niego, rzekl byl do starszych: Nie wieciez, iz poslal syn tego mezobójce, aby scieto glowe moje? Patrzciez, gdy przyjdzie ten posel, a zamknijciez drzwi, a zahamujcie go przede drzwiami; boc tenten nóg pana jego jest juz za nim.
\par 33 A gdy to jeszcze mówil z nimi, oto posel przychodzil ku niemu, i rzekl: Oto to zle jest od Pana; czegóz mam wiecej oczekiwac od Pana?

\chapter{7}

\par 1 Tedy rzekl Elizeusz: Sluchajcie slowa Panskiego. Tak mówi Pan: O tym czasie jutro miara maki pszennej bedzie za sykiel, a dwie miary jeczmienia za sykiel, w bramie Samaryjskiej.
\par 2 I odpowiedzial ksiaze, na którego sie rece król wspieral, mezowi Bozemu i rzekl: By tez Pan poczynil okna w niebie, izaliby to moglo byc? Który mu rzekl: Oto ty ujrzysz oczyma twemi; ale tego jesc nie bedziesz.
\par 3 A byli czterej mezowie tredowaci u wyjscia bramy, którzy rzekli jeden do drugiego: Pocóz tu mieszkamy, azbysmy pomarli?
\par 4 Jezli wnijdziemy do miasta, glód w miescie, i pomrzemy tam, a jezli tu zostaniemy, przecie pomrzemy. Teraz tedy pójdzcie, a zbiezmy do obozu Syryjskiego; jezli nas zywo zostawia, bedziemy zywi; jezli nas tez zabija, pomrzemy.
\par 5 Wstali tedy, gdy sie zmierzchac poczelo, aby szli do obozu Syryjskiego; a przyszedlszy na koniec obozu Syryjskiego, oto tam nie bylo nikogo.
\par 6 Albowiem sprawil Pan, ze slychac bylo w obozie Syryjskim grzmot wozów i tenten koni, i huk wojska wielkiego, i rzekli jeden do drugiego: Oto najal za pieniadze przeciwko nam król Izraelski króle Hetejskie, i króle Egipskie, aby przypadli na nas.
\par 7 A tak wstawszy uciekli w zmierzch, zostawiwszy namioty swe, i konie swe i osly swe, i obóz jaki byl, a uciekli, chcac zachowac dusze swoje.
\par 8 A gdy przyszli oni tredowaci az na przodek obozu, weszli do jednego namiotu, i jedli i pili, a nabrawszy stamtad srebra i zlota, i szat, szli i skryli. Potem sie wrócili, i weszli do drugiego namiotu, a nabrawszy takze stamtad, odeszli i pokryli.
\par 9 Zatem rzekl jeden do drugiego: Nie dobrze czynimy. Dzien ten jest dzien dobrej nowiny, a my milczymy? Jezli bedziemy czekali az do zaranku, bedziemy winni grzechu. Przetoz teraz pójdzcie, wnijdzmy, a opowiedzmy to domowi królewskiemu.
\par 10 A tak przyszedlszy zawolali na wrotnego miejskiego, i powiedzieli im mówiac: Przyszlismy do obozu Syryjskiego, a oto nie bylo tam nikogo, ani glosu ludzkiego, oprócz koni uwiazanych, i oslów uwiazanych, i namiotów, jako przedtem byly.
\par 11 Tedy on zawolal na inne wrotne, a ci opowiedzieli to w domu królewskim.
\par 12 Wstawszy tedy król w nocy, rzekl do slug swoich: Powiem ja wam, co nam uczynili Syryjczycy; wiedza, zesmy zglodniali, przetoz wyszli z obozu, a pokryli sie w polu, mówiac: Gdy wynijda z miasta, pojmiemy je zywo, i miasto ubiezemy.
\par 13 Tedy odpowiedzial jeden z slug jego, i rzekl: Prosze niech wezma piec koni pozostalych, które zostaly w miescie; (oto one sa jako wszystko mnóstwo Izraelskie, które zostalo w niem; oto one sa mówie jako wszystko mnóstwo Izraelskie, które ginie,)te wyslijmy a wywiedzmy sie.
\par 14 A tak wziawszy dwa wozy z konmi, poslal król do obozu Syryjskiego, mówiac: Idzcie a obaczcie.
\par 15 I szli za nimi az do Jordanu, a oto po wszystkiej drodze pelno bylo szat i naczynia, które porzucili Syryjczycy, kwapiac sie. Tedy wróciwszy sie oni poslowie, oznajmili to królowi.
\par 16 Przeto wyszedlszy lud, rozchwycil obóz Syryjski; a byla miara pszennej maki za sykiel, a dwie miary jeczmienia za sykiel, wedlug slowa Panskiego.
\par 17 A król postanowil byl onego ksiazecia, na którego sie rece wspieral, w bramie, którego lud podeptal w bramie, az umarl, jako mu byl powiedzial maz Bozy, który o tem mówil, gdy byl król przyszedl do niego.
\par 18 I stalo sie wedlug slowa, które byl rzekl maz Bozy królowi, mówiac: Dwie miary jeczmienia za sykiel, a miara pszennej maki bedzie za sykiel, jutro o tym czasie w bramie Samaryjskiej.
\par 19 Na co byl odpowiedzial on ksiaze mezowi Bozemu, mówiac: By tez Pan uczynil okna w niebie, izali to bedzie wedlug slowa tego? A on mu rzekl: Oto ty ujrzysz oczyma twemi, ale tego jesc nie bedziesz.
\par 20 I stalo mu sie tak; bo go podeptal lud w bramie, az umarl.

\chapter{8}

\par 1 Potem Elizeusz rzekl do onej niewiasty, której byl syna wskrzesil, mówiac: Wstan a idz, ty i dom twój, a badz gosciem, kedy bedziesz mogla byc; bo zawolal Pan glodu, i przyjdzie na ziemie przez siedm lat.
\par 2 Wstala tedy ona niewiasta, i uczynila wedlug slowa meza Bozego; a poszla ona i dom jej, i byla gosciem w ziemi Filistynskiej przez siedm lat.
\par 3 I stalo sie po wyjsciu siedmiu lat, ze sie wrócila ona niewiasta z ziemi Filistynskiej, i poszla, aby wolala na króla o dom swój, i o role swoje.
\par 4 A na ten czas król rozmawial z Giezym, sluga meza Bozego, mówiac: Powiedz mi prosze wszystkie zacne sprawy, które czynil Elizeusz.
\par 5 A gdy on powiadal królowi, jako wskrzesil umarlego, oto niewiasta, której byl wskrzesil syna, zawolala na króla o dom swój i o role swoje. I rzekl Giezy: Królu panie mój, tac to jest niewiasta, i ten syn jej, którego wskrzesil Elizeusz.
\par 6 I pytal król niewiasty, a ona mu powiedziala. I przydal jej król komornika jednego, mówiac: Przywróc jej wszystko, co jej bylo, i wszystkie dochody z pola od onego dnia, którego opuscila ziemie, az dotad.
\par 7 Potem przyszedl Elizeusz do Damaszku, a Benadad, król Syryjski, chorowal. I powiedziano mu, mówiac: Przyszedl tu maz Bozy.
\par 8 I rzekl król do Hazaela: Wezmij w reke swa upominek, a idz przeciwko mezowi Bozemu, i pytaj sie Pana przezen, mówiac: Wstaneli z tej choroby?
\par 9 Przetoz szedl Hazael przeciwko niemu, a wziawszy upominek w reke swa, i ze wszystkich dóbr Damaskich brzemion na czterdziesci wielbladów. I przyszedl, a stanal przed nim, mówiac: Syn twój Benadad, król Syryjski, poslal mie do ciebie, mówiac: Wstaneli z tej choroby?
\par 10 I odpowiedzial mu Elizeusz: Idz, powiedz mu: Wprawdziebysci mógl zyc; wszakze okazal mi Pan, ze pewnie umrzesz.
\par 11 Wtem pokazal mu, i stawil twarz swoje smutna, i plakal maz Bozy.
\par 12 Któremu rzekl Hazael: Czemuz pan mój placze? I odpowiedzial: Iz wiem, co uczynisz zlego synom Izraelskim. Twierdze ich popalisz ogniem, a mlodzience ich mieczem pomordujesz, i dzieci ich poroztracasz, i brzemienne ich porozcinasz.
\par 13 Tedy rzekl Hazael: Co? Izali sluga twój pies, zeby mial czynic tak wielka rzecz? I odpowiedzial Elizeusz: Okazal mi Pan, ze ty bedziesz królem nad Syria.
\par 14 I odszedl od Elizeusza, a przyszedl do pana swego, który rzekl do niego: Cóz ci powiedzial Elizeusz? A on rzekl: Powiedzial mi, zebys pewnie mógl zyc.
\par 15 A nazajutrz wzial Hazael koldre i zamaczal ja w wodzie, i rozciagnal na twarzy jego. I umarl (Benadad), a Hazael królowal miasto niego.
\par 16 A roku piatego Jorama, syna Achaba, króla Izraelskiego, i Jozafata króla Judzkiego, poczal królowac Joram, syn Jozafata, król Judzki.
\par 17 Trzydziesci i dwa lata mial, gdy królowac poczal, a osm lat królowal w Jeruzalemie.
\par 18 Ale chodzil drogami królów Izraelskich, sprawujac sie jako dom Achabowy; bo córke Achabowa mial za zone, i czynil zle przed oczyma Panskiemi.
\par 19 Wszakze nie chcial Pan wytracic Judy, dla Dawida, slugi swego, jako mu byl powiedzial, iz mu mial dac pochodnie miedzy synami jego, po wszystkie dni.
\par 20 Za dni jego odstapil Edom, aby nie byl pod moca Judy; i postanowili nad soba króla.
\par 21 Przetoz przyciagnal Joram do Seiru, i wszystkie wozy z nim; a wstawszy w nocy porazil Edomczyki, którzy go byli otoczyli, i hetmany wozów, tak iz lud uciekal do namiotów swoich.
\par 22 Wszakze odstapil Edom, aby nie byl pod moca Judy, az do dnia tego. Odstapilo takze i Lobne onegoz czasu.
\par 23 A inne sprawy Joramowe, i wszystko co czynil, izali nie jest napisane w kronikach o królach Judzkich?
\par 24 I zasnal Joram z ojcami swymi, a pogrzebiony jest z ojcami swymi w miescie Dawidowem; i królowal Ochozyjasz, syn jego, miasto niego.
\par 25 Roku dwunastego Jorama, syna Achaba, króla Izraelskiego, poczal królowac Ochozyjasz, syn Jorama, króla Judzkiego.
\par 26 We dwudziestu i dwóch latach byl Ochozyjasz, gdy królowac poczal, a rok jeden królowal w Jeruzalemie; a imie matki jego bylo Atalija, córka Amrego, króla Izraelskiego.
\par 27 Ten chodzil droga domu Achabowego, i czynil zle przed oczyma Panskiemi, jako i dom Achabowy; bo byl zieciem domu Achabowego.
\par 28 Przetoz wychadzal z Joramem, synem Achabowym, na wojne przeciw Hazaelowi, królowi Syryjskiemu, do Ramot Galaadskiego; ale porazili Syryjczycy Jorama.
\par 29 A tak wrócil sie król Joram, aby sie leczyl w Jezreelu na rany, które mu byli zadali Syryjczycy w Ramacie, gdy walczyl z Hazaelem, królem Syryjskim. A Ochozyjasz, syn Jorama, króla Judzkiego, przyjechal nawiedzac Jorama, syna Achabowego, do Jezreela; bo tam chorowal.

\chapter{9}

\par 1 A Elizeusz prorok zawolal jednego z synów prorockich, i rzekl mu: Przepasz biodra twoje, a wezmij te banke olejku w reke twa, a idz do Ramot Galaadskiego.
\par 2 A gdy tam przyjdziesz, ujrzysz tam Jehu, syna Jozafatowego, syna Namsy, a wszedlszy tam, odwiedziesz go z posrodku braci jego, i wprowadzisz go do gmachu najskrytszego.
\par 3 A wziawszy banke olejku, wylejesz na glowe jego, i rzeczesz: Tak mówi Pan: Pomazalem cie za króla nad Izraelem. A otworzywszy drzwi ucieczesz, i nie zabawisz sie tam.
\par 4 Tedy odszedl on mlodzieniec, sluga prorocki, do Ramot Galaadskiego.
\par 5 A gdy przyszedl, oto hetmani wojsk siedzieli. I rzekl: Hetmanie! mam nieco z toba mówic. I rzekl Jehu: Z którymze ze wszystkich nas? I odpowiedzial: Z toba, hetmanie!
\par 6 Tedy wstawszy wszedl do gmachu, a on wylal olejek na glowe jego, i rzekl mu: Tak mówi Pan, Bóg Izraelski: Pomazalem cie za króla nad ludem Panskim, nad Izraelem.
\par 7 I wytracisz dom Achaba, pana twego; albowiem pomszcze sie krwi slug moich proroków, i krwi wszystkich slug Panskich, z reki Jezabeli.
\par 8 A tak zginie wszystek dom Achabowy; i wykorzenie z domu Achabowego az do najmniejszego szczeniecia, i wieznia, i opuszczonego w Izraelu.
\par 9 I uczynie domowi Achabowemu, jako domowi Jeroboama, syna Nabatowego, i jako domowi Baazy, syna Achyjaszowego.
\par 10 Jezabele tez zjedza psy na polu Jezreelskim, a nie bedzie, ktoby ja pogrzebal. To rzeklszy otworzyl drzwi, i uciekl.
\par 11 A gdy Jehu wyszedl do slug pana swego, rzekl mu jeden: A dobrzez wszystko? Pocóz przychodzil ten szalony do ciebie? A on im odpowiedzial: Wy znacie tego meza, i mowe jego.
\par 12 Tedy rzekli: Nie prawda to; prosze powiedz nam. A on rzekl: Tak a tak rzekl do mnie, mówiac: Tak mówi Pan: Pomazalem cie za króla nad Izraelem.
\par 13 Pospieszyli sie tedy, a wziawszy kazdy szate swa, kladli je poden na najwyzszym stopniu, i zatrabiwszy w trabe, mówili: Króluje Jehu!
\par 14 Tedy sie sprzysiagl Jehu, syn Jozafata, syna Namsy, przeciw Joramowi. (A na ten czas Joram strzegl Ramot Galaadskiego, on i wszystek Izrael, przed Hazaelem, królem Syryjskim.
\par 15 Ale sie byl wrócil król Joram, aby sie leczyl na rany, które mu byli zadali Syryjczycy, gdy walczyl z Hazaelem, królem Syryjskim.) I rzekl Jehu: Jezli sie wam zda, niech nie wychodzi nikt z miasta, zeby szedl co oznajmic w Jezreelu.
\par 16 I wsiadl na wóz Jehu, i jechal do Jezreela, bo tam Joram lezal; Ochozyjasz takze, król Judzki, przyjechal byl, aby nawiedzil Jorama.
\par 17 Wtem stróz, który stal na wiezy w Jezreelu, ujrzawszy poczet Jehu przyjezdzajacy, rzekl: Poczet jakis widze. I rzekl Joram: Wezmij jezdnego, a wyslij przeciwko nim, aby sie spytal, jezli pokój.
\par 18 A tak biezal jezdny przeciwko niemu, i rzekl: Tak mówi król: A pokój? I odpowiedzial Jehu: Co tobie do pokoju? Obróc sie, jedz za mna. Przetoz oznajmil stróz mówiac: Dojechalci posel do nich, ale sie nie wraca.
\par 19 Zatem poslal drugiego jezdnego, który przyjechawszy do nich, rzekl: Tak mówi król: A pokój? Odpowiedzial Jehu: Co tobie do pokoju? Obróc sie, a jedz za mna.
\par 20 Znowu oznajmil to stróz, mówiac: Przyjechalci do nich, ale sie nie wraca. A przyjazd jego, jakoby przyjazd Jehu, syna Namsy; bo szalenie jedzie.
\par 21 Tedy rzekl Joram: Zaprzegaj. I zaprzezono w wóz jego. I wyjechal Joram, król Izraelski, i Ochozyjasz, król Judzki, kazdy na wozie swym. A wyjechawszy przeciw Jehu, trafili go na polu Nabota Jezreelskiego.
\par 22 A gdy ujrzal Joram Jehu, rzekl: Jestze pokój Jehu? I odpowiedzial: Co za pokój? poniewaz jeszcze cudzolóstwa Jezabeli, matki twojej, i czary jej wielkie sa.
\par 23 Przetoz obróciwszy sie Joram uciekl, mówiac do Ochozyjasza: Zdrada, Ochozyjaszu!
\par 24 Tedy Jehu wziawszy w rece swoje luk, postrzelil Jorama miedzy ramiona jego, az przeszla strzala przez serce jego, tak, ze padl na wozie swoim.
\par 25 Potem rzekl Jehu do Badakiera, hetmana swego: Wezmij go, a porzuc na polu Nabota Jezreelskiego; albowiem pamietasz, gdysmy ja i ty jechali spolu za Achabem, ojcem jego, ze Pan wydal byl przeciwko niemu te pogrózke.
\par 26 Zaiste krwi Nabota, i krwi synów jego, któram widzial wczoraj, rzekl Pan, pomszcze sie nad toba na tem polu. Pan to rzekl: przetoz teraz wezmij go, a porzuc go na polu wedlug slowa Panskiego.
\par 27 Co Ochozyjasz, król Judzki, ujrzawszy, uciekal droga do domu ogrodowego. Ale go gonil Jehu, i rzekl: I tego zabijcie na wozie jego. I zranili go na wstapie Guru, który jest podle Jeblaam. A uciekl do Magieddy, i tam umarl.
\par 28 I kazali go zawiesc sludzy jego do Jeruzalemu, a pogrzebli go w grobie jego z ojcami jego w miescie Dawidowem.
\par 29 A roku jedenastego Jorama, syna Achabowego, królowal Ochozyjasz nad Juda.
\par 30 Zatem przyszedl Jehu do Jezreel. Co gdy Jezabela uslyszala, ufarbowala twarz swoje, i ochedozyla glowe swa a patrzala z okna.
\par 31 A gdy Jehu wjezdzal w brame, rzekla: Jestze pokój, o Zymry, morderzu pana swego?
\par 32 A on podnióslszy twarz swoje ku oknu, rzekl: Któz ze mna trzyma, kto? Tedy wejrzeli nan dwaj albo trzej komornicy jej.
\par 33 Którym rzekl: Zrzuccie ja. I zrzucili ja, i popryskala sie sciana i konie krwia jej, i podeptal ja.
\par 34 A gdy tam wszedl, jadl i pil, i rzekl: Obaczcie prosze one przekleta, a pogrzebcie ja; boc córka królewska jest.
\par 35 Tedy szedlszy, aby ja pogrzebli, nie znalezli z niej jedno czaszke z glowy, i nogi, i dloni rak.
\par 36 A wróciwszy sie, oznajmili mu to. Który rzekl: Wypelnilo sie slowo Panskie, które powiedzial przez sluge swego Elijasza Tesbite, mówiac: Na polu Jezreel zjedza psy cialo Jezabeli.
\par 37 Niech bedzie trup Jezabeli, jako gnój na roli, na polu Jezreel, tak zeby nie mówiono. Tac jest Jezabela.

\chapter{10}

\par 1 A mial Achab siedmdziesiat synów w Samaryi. I napisal Jehu list, a poslal go do Samaryi do ksiazat Jezreelskich, i do starszych i do tych, którzy wychowywali syny Achabowe, w te slowa:
\par 2 Skoro was dojdzie ten list, gdyz u was sa synowie pana waszego, i u was wozy, i konie, i miasto obronne, i rynsztunek;
\par 3 Obierzciez najgodniejszego i najsposobniejszego z synów pana waszego, a posadzcie na stolicy ojca jego, i walczcie o dom pana waszego.
\par 4 Ale sie oni bardzo bojac rzekli: Oto dwaj królowie nie ostali sie przed nim, a jakoz my sie ostoimy?
\par 5 A tak poslal ten, który byl sprawca domu, i ten, który byl sprawca miasta, i starsi, i ci, którzy wychowywali syny królewskie, do Jehu, mówiac: Sludzysmy twoi, a co nam rozkazesz, uczynimy. Nie postanowiemy króla zadnego; co dobrego jest w oczach twoich, czyn.
\par 6 I napisal do nich list drugi, mówiac: Jezliscie moi, a glosu mego sluchacie, wezmijciez glowy synów pana waszego, a przyjdzcie do mnie jutro o tym czasie do Jezreel. A synów królewskich bylo siedmdziesiat mezów u najprzedniejszych w miescie, którzy je wychowywali.
\par 7 A gdy ich list doszedl, wziawszy syny królewskie, pobili onych siedmdziesiat mezów, a skladlszy glowy ich do koszów, poslali je do niego do Jezreela.
\par 8 I przyszedl posel, który mu oznajmil, mówiac: Przyniesiono glowy synów królewskich. A on rzekl: Skladzcie je na dwie kupie u wejscia bramy az do poranku.
\par 9 A gdy rano wyszedl, stanal, i rzekl do wszystkiego ludu: Sprawiedliwiscie wy. Otom sie ja sprzysiagl przeciwko panu memu, i zabilem go; ale te wszystkie któz pobil?
\par 10 Wiedzciez teraz, ze nie upadlo prózno zadne z slów Panskich na ziemie, które mówil Pan przeciwko domowi Achabowemu, gdyz uczynil Pan, co byl powiedzial przez sluge swego Elijasza.
\par 11 A tak pobil Jehu wszystkie, którzy pozostali z domu Achabowego w Jezreelu, i wszystkie najprzedniejsze jego, i przyjaciele jego, i kaplany jego, tak iz nie zostawil po nim zadnego zywego.
\par 12 Potem wstawszy odszedl, i pojechal do Samaryi. A gdy byl a domu, gdzie pasterze strzygali owce na drodze,
\par 13 Tedy Jehu znalazl u braci Ochozyjasza króla Judzkiego, i rzekl: Któscie wy? I odpowiedzieli: Braciasmy Ochozyjaszowi, a idziemy, abysmy pozdrowili syny królewskie, i syny królowej.
\par 14 Tedy rzekl: Pojmajcie je zywo. I pojmali je zywo, i pobili je u studni onegoz domu, gdzie strzygano owce, czterdziestu i dwóch mezów, i nie zostawil zadnego z nich.
\par 15 Potem odjechawszy stamtad, trafil Jonadaba, syna Rechabowego, idacego przeciwko sobie, a pozdrowil go i rzekl do niego: Jestze serce twoje szczere, jako serce moje z sercem twojem? I odpowiedzial mu Jonadab: Jest. A jest? rzekl Jehu, dajze mi reke twoje. Tedy mu dal reke swa; i kazal mu wsiasc do siebie na wóz.
\par 16 I rzekl: Jedz ze mna, a przypatrz sie gorliwosci mojej za Pana. A tak wiózl go na wozie swoim.
\par 17 A gdy przyjechal do Samaryi, bil wszystkie, którzy byli pozostali z domu Achabowego w Samaryi, i wytracil je wedlug slowa Panskiego, który mówil do Elijasza.
\par 18 Zatem zebral Jehu wszystek lud, i rzekl do niego: Achab sluzyl Baalowi malo, Jehu mu bedzie sluzyl wiecej.
\par 19 Przetoz teraz wszystkich proroków Baalowych, i wszystkich slug jego, i wszystkich kaplanów jego, zwolajcie do mnie az do jednego; albowiem ofiare wielka bede sprawowal Baalowi. Ktoby sie kolwiek nie stawil, nie zostanie zyw. A to Jehu chytrze czynil, chcac wytracic chwalce Baalowe.
\par 20 Nadto rzekl Jehu: Zapowiedzcie swieto Baalowi. I obwolano je.
\par 21 I rozeslal Jehu do wszystkiego Izraela. I zeszli sie wszyscy chwalcy Baalowi, tak ze nie zostal zaden, któryby nie przyszedl. I weszli do kosciola Baalowego, a napelniony byl dom Baalowy od konca az do konca.
\par 22 Tedy rzekl temu, który byl nad szatami: Wynies szaty wszystkim chwalcom Baalowym. I wyniósl im szaty.
\par 23 Zatem wszedl Jehu i Jonadab, syn Rechabowy, do domu Baalowego, i rzekl chwalcom Baalowym: Dowiedzcie sie, a obaczcie, by snac nie byl kto z wami z chwalców Panskich, oprócz samych chwalców Baalowych.
\par 24 A tak weszli, aby sprawowali ofiary, i calopalenia. Ale Jehu sporzadzil byl sobie na dworze osmdziesiat mezów, którym byl rzekl: Jezliby kto uszedl z ludu tego, który ja podawam w rece wasze, dusza wasza bedzie za dusze onego.
\par 25 A gdy sie dokonczyly ofiary calopalenia, rzekl Jehu zolnierzom i rotmistrzom swym: Wnijdzcie, a pomordujcie je, aby zaden nie uszedl. A tak pomordowali je ostrzem miecza, i rozrzucili je zolnierze i rotmistrze; potem odeszli do kazdego miasta, gdzie byl dom Baalowy.
\par 26 A wyrzuciwszy balwany z domu Baalowego, popalili je.
\par 27 Obalili tez slup Baalowy, obalili i dom jego, a uczynili z niego wychody, az do tego czasu.
\par 28 A tak wygladzil Jehu Baala z Izraela.
\par 29 Wszakze od grzechów Jeroboama, syna Nabatowego, który do grzechu przywiódl Izraela, nie odstapil Jehu, ani opuscil cielców zlotych, które byly w Betel, i które byly w Dan.
\par 30 Tedy rzekl Pan do Jehu: Poniewazes sie pilnie staral, abys uczynil, co dobrego jest w oczach moich, wedlug wszystkiego, co bylo w sercu mojem, uczyniles domowi Achabowemu: synowie twoi az do czwartego pokolenia siedziec beda na stolicy Izraelskiej.
\par 31 Ale Jehu nie strzegl tego, aby chodzil w zakonie Pana, Boga Izraelskiego, ze wszystkiego serca swego, ani odstapil od grzechów Jeroboamowych, który do grzechu przywiódl Izraela.
\par 32 W one dni poczal Pan umniejszac Izraela: bo je porazil Hazael po wszystkich granicach Izraelskich:
\par 33 Od Jordanu az na wschód slonca, wszystke ziemie Galaadska, Gadowa, i Rubenowa, i Manasesowa od Aroer, które jest u potoku Arnon, i Galaad, i Basan.
\par 34 Ale ostatek spraw Jehu, i wszystko, co czynil, i wszystka moc jego, azaz tego nie napisano w kronikach królów Izraelskich?
\par 35 I zasnal Jehu z ojcami swymi, i pochowali go w Samaryi; a królowal Joachaz, syn jego, miasto niego.
\par 36 A czas, którego królowal Jehu nad Izraelem w Samaryi, bylo dwadziescia i osm lat.

\chapter{11}

\par 1 Tedy Atalija, matka Ochozyjaszowa, widzac iz umarl syn jej, powstala, i wytracila wszystko nasienie królewskie.
\par 2 Ale wziawszy Josaba, córka króla Jorama, siostra Ochzyjaszowa, Joaza, syna Ochozyjaszowego, ukradla go z posrodku synów królewskich, które zabijano; tego i z mamka jego w pokoju loznicy skryla przed Atalija, i nie zabito go.
\par 3 I byl przy niej w domu Panskim skryty przez szesc lat, których Atalija królowala nad ziemia.
\par 4 Potem roku siódmego poslawszy Jojada, przyzwal rotmistrzów, hetmanów i zolnierzy, i wprowadzil je do siebie do domu Panskiego, a uczyniwszy z nimi przymierze, przywiódl, je do przysiegi w domu Panskim, i ukazal im syna królewskiego.
\par 5 I rozkazal im, mówiac: Toc jest co uczynicie: trzecia czesc z was, którzy przychodzicie w sabat, a trzymywacie straz, niech bedzie przy domu królewskim:
\par 6 A trzecia czesc z was zostanie w bramie Sur; trzecia czesc zasie bedzie w bramie, która jest za zolnierzami; a bedziecie trzymali straz przy tym domu dla jakiego gwaltu.
\par 7 A dwie czesci z was wszystkich wychodzacych w sabat niech trzymaja straz domu Panskiego okolo króla.
\par 8 A tak obstapicie króla okolo, kazdy majac bron swa w rekach swych; a ktobykolwiek przyszedl do waszego szyku, niech bedzie zabity, a wy bedziecie przy królu, gdy wychodzic i wchodzic bedzie.
\par 9 I uczynili rotmistrze oni wedlug wszystkiego, co im byl rozkazal Jojada kaplan; a wziawszy kazdy meze swe, którzy przychodzili w sabat, i którzy odchodzili w sabat, przyszli do Jojady kaplana.
\par 10 Tedy dal kaplan rotmistrzom wlócznie i tarcze, które byly króla Dawida, które byly w domu Panskim.
\par 11 I stali zolnierze, kazdy majac bron swoje w rekach swych, od prawej strony domu az do lewej strony domu przeciwko oltarzowi, i przeciwko domowi okolo króla zewszad.
\par 12 Tedy wywiódl syna królewskiego, i wlozyl nan korone, i swiadectwo. I uczynili go królem, i pomazali go, a klaskajac rekoma mówili: Niech zyje król!
\par 13 Wtem uslyszawszy Atalija krzyk zbiegajacego sie ludu, weszla do ludu do domu Panskiego.
\par 14 A gdy ujrzala, ze oto król stal na majestacie wedlug zwyczaju, a ksiazeta i traby okolo króla, a wszystek lud ziemi weselacy sie, i trabiacy w traby, rozdarla Atalija odzienie swoje, i wolala: Sprzysiezenie, sprzysiezenie!
\par 15 Przetoz rozkazal Jojada kaplan rotmistrzom, którzy byli nad wojskiem, i rzekl do nich: Wywiedzcie ja z zagrodzenia kosciola, a ktobykolwiek chcial isc za nia, niech zabity bedzie mieczem; bo rzekl byl kaplan: Niech nie bedzie zabita w domu Panskim.
\par 16 I uczynili jej plac; a gdy przyszla na droge, która wodzono konie do domu królewskiego, tamze jest zabita.
\par 17 Tedy uczynil Jojada przymierze miedzy Panem, i miedzy królem, i miedzy ludem, aby byli ludem Panskim; takze miedzy królem i miedzy ludem.
\par 18 I wszedl wszystek lud onej ziemi do domu Baalowego, i zburzyli go; oltarze jego i obrazy jego polamali do szczetu; nadto Matana, kaplana Baalowego, zabili przed oltarzami. I postanowil znowu kaplan przelozone nad domem Panskim.
\par 19 Potem wziawszy rotmistrze, i hetmany, i zolnierze, i wszystek lud onej ziemi, prowadzili króla z domu Panskiego, i przyszli droga az ku bramie zolnierzy, do domu królewskiego. I usiadl na stolicy królewskiej.
\par 20 I weselil sie wszystek lud onej ziemi, a miasto sie uspokoilo, gdy Atalija zabito mieczem podle domu królewskiego.
\par 21 A bylo siedm lat Joazowi, gdy poczal królowac.

\chapter{12}

\par 1 Roku siódmego Jehu poczal królowac Joaz, a czterdziesci lat królowal w Jeruzalemie; imie matki jego bylo Sebija z Beersaby.
\par 2 I czynil Joaz, co dobrego bylo w oczach Panskich, po wszystkie dni swoje, których go uczyl Jojada kaplan.
\par 3 Wszakze wyzyny nie byly zniesione; jeszcze lud ofiarowal i kadzil na onych wyzynach.
\par 4 I rzekl Joaz do kaplanów: Wszystkie pieniadze poswiecone, które przychodza do domu Panskiego, pieniadze tych, którzy ida w liczbe, pieniadze kazdego z osobna wedlug szacunku jego, i wszystkie pieniadze, które ktokolwiek dobrowolnie znosi do domu Panskiego,
\par 5 Te wezma do siebie kaplani kazdy od znajomego swego; a oni naprawia skaze domu Panskiego wszedy, gdzieby sie znalazla skaza.
\par 6 I stalo sie roku dwudziestego i trzeciego króla Joaza, gdy jeszcze nie poprawili byli kaplani skazy domu,
\par 7 Ze wezwal król Joaz Jojady kaplana, i innych kaplanów, i mówil do nich: Przecz nie oprawujecie skazy domu? Przetoz teraz nie bierzcie pieniedzy od znajomych waszych, ale one na poprawe skazy domu oddawajcie.
\par 8 I zezwolili na to kaplani; zeby nie brali pieniedzy od ludu, i zeby nie poprawiali skazy domu.
\par 9 Przetoz wziawszy Jojada kaplan skrzynie jedne, uczynil dziure w wieku jej, a postawil ja przy oltarzu po prawej stronie, kedy wchodzono do domu Panskiego. I kladli w nie kaplani, którzy strzegli progu, wszystkie pieniadze, które wnoszono do domu Panskiego.
\par 10 A gdy widzieli, ze bylo wiele pieniedzy w skrzyni, tedy przychodzil pisarz królewski, i kaplan najwyzszy, którzy zliczywszy chowali one pieniadze, które sie znajdowaly w domu Panskim.
\par 11 I dawali pieniadze gotowe w rece rzemieslników, przelozonych nad robota domu Panskiego; a ci je wydawali na ciesle, i na robotniki, którzy poprawiali domu Panskiego;
\par 12 I na murarze, i na te, co ciosali kamienie, i na kupowanie drzewa, i ciosanego kamienia, ku poprawie skazy domu Panskiego, i na wszystek naklad ku poprawie domu onego.
\par 13 Wszakze nie sprawowano do domu Panskiego kubków srebrnych, naczynia do muzyki, miednic, i trab, zadnego naczynia zlotego, i naczynia srebrnego, z pieniedzy, które przynoszono do domu Panskiego;
\par 14 Ale rzemieslnikom przelozonym nad robota dawali je, i poprawiali za nie domu Panskiego.
\par 15 A nie sluchano liczby tych ludzi, którym dawano pieniadze w rece ich, aby wydawali rzemieslnikom, poniewaz to oni wiernie odprawowali.
\par 16 Ale pieniadze za wystepek, i pieniadze za grzechy, nie byly wnoszone do domu Panskiego; kaplanom sie dostawaly.
\par 17 Tedy wyciagnal Hazael, król Syryjski, a walczyl przeciwko Giet, i wzial je. Potem obrócil Hazael twarz swoje, aby ciagnal przeciwko Jeruzalemowi.
\par 18 Przetoz wzial Joaz, król Judzki, wszystkie rzeczy poswiecone, które byli poswiecili Jozafat i Joram, i Ochozyjasz, ojcowie jego, królowie Judzcy, i to, co byl sam poswiecil, i wszystko zloto, które sie znalazlo w skarbach domu Panskiego, i domu królewskiego, a poslal to do Hazaela, króla Syryjskiego, i odciagnal od Jeruzalemu.
\par 19 Ale insze sprawy Joazowe, i wszystko co czynil, azaz to nie jest napisane w kronikach o królach Judzkich?
\par 20 Potem powstawszy sludzy jego sprzysiegli sie miedzy soba i zabili Joaza w Betmello, któredy chodza do Selli;
\par 21 To jest, zabili go Josachar, syn Semaatowy, i Jozabad, syn Sommerowy; ci sludzy jego zabili go, i umarl. A pochowali go z ojcami jego w miescie Dawidowem, i królowal Amazyjasz, syn jego, miasto niego.

\chapter{13}

\par 1 Roku dwudziestego i trzeciego Joaza, syna Ochozyjasza, króla Judzkiego, królowal Joachaz, syn Jehu, nad Izraelem w Samaryi siedmnascie lat.
\par 2 A czynil zle przed oczyma Panskiemi; bo nasladowal grzechów Jeroboama, syna Nabatowego, który przywiódl do grzechu Izraela, i nie odchylil sie od nich.
\par 3 I zapalil sie gniew Panski przeciw Izraelowi, i podal je w reke Hazaela, króla Syryjskiego, i w reke Benadada, syna Hazaelowego, po wszystkie dni.
\par 4 Ale gdy sie modlil Joachaz przed obliczem Panskiem, wysluchal go Pan; bo widzial scisnienie Izraela, ze go byl ucisnal król Syryjski.
\par 5 Przetoz dal Pan Izraelowi wybawiciela, a wyszli z reki Syryjczyków, i mieszkali synowie Izraelscy w przybytkach swych, jako i przedtem.
\par 6 Wszakze nie odstapili od grzechów domu Jeroboamowego, który przywiódl do grzechu Izraela, ale w nich chodzili; do tego jeszcze i gaj zostal w Samaryi.
\par 7 Aczkolwiek nie zostawil Joachazowi z ludu, jedno piecdziesiat jezdnych, i dziesiec wozów, i dziesiec tysiecy pieszych, gdyz je byl wytracil król Syryjski, i w proch je pomlócil.
\par 8 Ale inne sprawy Joachazowe, i wszystko, co czynil, i moc jego, azaz to nie jest napisane w kronikach o królach Izraelskich?
\par 9 I zasnal Joachaz z ojcami swymi, i pochowano go w Samaryi, a królowal Joaz, syn jego, miasto niego.
\par 10 Roku trzydziestego i siódmego Joaza, króla Judzkiego, królowal Joaz, syn Joachazowy, nad Izraelem w Samaryi szesnascie lat;
\par 11 I czynil zle przed oczyma Panskiemi, nie uchylajac sie od zadnych grzechów Jeroboama, syna Nabatowego, który przywiódl do grzechu Izraela; ale w nich chodzil.
\par 12 A inne sprawy Joazowe, i wszystko co czynil, i moc jego, jako walczyl przeciwko Amazyjaszowi, królowi Judzkiemu, azaz to nie jest napisane w kronikach o królach Izraelskich?
\par 13 I zasnal Joaz z ojcami swymi, a Jeroboam usiadl na stolicy jego. I pogrzebion jest Joaz w Samaryi z królami Izraelskimi.
\par 14 A Elizeusz wpadl w ciezka chorobe, w której tez umarl. I przyszedl do niego Joaz, król Izraelski, i plakal nad nim, mówiac: Ojcze mój, ojcze mój! wozie Izraelski, i jazdo jego.
\par 15 Tedy mu rzekl Elizeusz: Wemij luk i strzaly; a wziawszy przyniósl do niego luk i strzaly.
\par 16 I rzekl do króla Izraelskiego: Wezmij w reke twoje luk; i wzial go w reke swoje; wlozyl tez Elizeusz rece swe na rece królewskie.
\par 17 I rzekl: Otwórz to okno na wschód slonca. A gdy otworzyl, rzekl Elizeusz: Strzelze! i strzelil. I rzekl: Strzala zbawienia Panskiego, a strzala wybawienia przeciw Syryjczykom; albowiem porazisz Syryjczyki w Afeku az do szczetu.
\par 18 Rzekl powtóre: Wezmij strzaly! i wzial. Tedy rzekl do króla Izraelskiego: Uderz w ziemie! i uderzyl trzy kroc a potem przestal.
\par 19 Przetoz rozgniewal sie nan maz Bozy, i rzekl: Miales uderzyc piec albo szesc kroc, bobys byl porazil Syryjczyki az do szczetu: a teraz tylko po trzy kroc porazisz Syryjczyki.
\par 20 Potem umarl Elizeusz, i pogrzebiono go. A kupy swawolne Moabskie wtargnely do ziemi roku drugiego.
\par 21 I stalo sie, gdy chowano jednego czlowieka, tedy ujrzawszy swawolna kupe, rzucili onego czlowieka w grób Elizeuszowy, który gdy byl wrzucony, a dotknal sie kosci Elizeuszowych, ozyl i wstal na nogi swoje.
\par 22 A Hazael, król Syryjski, trapil lud Izraelski po wszystkie dni Joachazowe.
\par 23 Ale ulitowawszy sie ich Pan, zmilowal sie nad nimi, i nawrócil sie ku nim dla przymierza swego z Abrahamem, z Izaakiem, i z Jakóbem; i nie chcial ich wytracic, ani ich odrzucil od oblicza swego, az do tego czasu.
\par 24 I umarl Hazael, król Syryjski, a królowal Benadad, syn jego, miasto niego.
\par 25 Przetoz znowu Joaz, syn Joachazowy, odebral miasta z reki Benadada, syna Hazaelowego, które byl wzial z rak Joachaza, ojca jego, przez wojne; bo po trzy kroc porazil go Joaz, i przywrócil miasta Izraelowi.

\chapter{14}

\par 1 Roku wtórego Joaza, syna Joachaza, króla Izraelskiego, poczal królowac Amazyjasz, syn Joaza, króla Judzkiego.
\par 2 Dwadziescia i piec lat mial, gdy królowac poczal, a dwadziescia i dziewiec lat królowal w Jeruzalemie. Imie matki jego bylo Joadana z Jeruzalemu.
\par 3 Ten czynil, co dobrego jest przed oczyma Panskiemi, aczkolwiek nie tak jako Dawid, ojciec jego; wedlug wszystkiego, co czynil Joaz, ojciec jego, postepowal.
\par 4 Wszakze wyzyny nie byly zniesione; jeszcze lud ofiarowal i kadzil po wyzynach.
\par 5 A gdy zmocnione bylo królestwo w reku jego, pobil slugi swe, którzy byli zabili króla, ojca jego.
\par 6 Lecz synów onych morderców nie pobil, jako napisano w ksiegach zakonu Mojzeszowego, gdzie rozkazal Pan, mówiac: Nie pomra ojcowie za synów, ani synowie pomra za ojców, ale kazdy za grzech swój umrze.
\par 7 Ten tez porazil Edomczyków dziesiec tysiecy w dolinie solnej, i wzial moca Sele, a nazwal imie jej Jokteel, az do tego czasu.
\par 8 Tedy poslal Amazyjasz posly do Joaza, syna Joachaza, syna Jehu, króla Izraelskiego, mówiac: Pójdz, wejrzymy sobie w oczy.
\par 9 Poslal zasie Joaz, król Izraelski, do Amazyjasza, króla Judzkiego, mówiac: Oset, który jest na Libanie, poslal do cedru Libanskiego, mówiac: Daj córke twoje synowi memu za zone. Wtem przyszedl zwierz polny, który jest na Libanie, i podeptal on oset.
\par 10 Zes ty bardzo porazil Edomczyki, dlatego sie podnioslo serce twoje. Chlubze sie, a siedz w domu twoim; i przeczze sie masz wdawac w to zle, abys upadl ty, i Juda z toba?
\par 11 Ale nie usluchal Amazyjasz. Przetoz wyciagnal Joaz, król Izraelski, a wejrzeli sobie w oczy, on i Amazyjasz, król Judzki, w Betsemes, które jest w Judztwie.
\par 12 I porazony jest Juda od ludu Izraelskiego, a uciekl kazdy do przybytku swego.
\par 13 Lecz Amazyjasza, króla Judzkiego, syna Joaza, syna Ochozyjaszowego, pojmal Joaz, król Izraelski, w Betsemes, a przyciagnawszy do Jeruzalemu, zburzyl mur Jeruzalemski od bramy Efraim az do bramy naroznej, na cztery sta lokci.
\par 14 I zabral wszystko zloto i srebro i wszystkie naczynia, które sie znalazly w domu Panskim, i w skarbach domu królewskiego, i ludzie zastawne, i wrócil sie do Samaryi.
\par 15 A inne sprawy Joazowe, które czynil, i moc jego, i jako walczyl z Amazyjaszem, królem Judzkim, azaz tego nie zapisano w kronikach o królach Izraelskich?
\par 16 I zasnal Joaz z ojcami swymi, a pogrzebiony jest w Samaryi z królmi Izraelskimi, a królowal Jeroboam, syn jego, miasto niego.
\par 17 I zyl Amazyjasz, syn Joazowy, król Judzki, po smierci Joaza, syn Joachaza, króla Izraelskiego, pietnascie lat.
\par 18 A inne sprawy Amazyjaszowe, azaz nie sa opisane w kronikach o królach Judzkich?
\par 19 Potem sprzysiegli sie przeciwko niemu niektórzy w Jeruzalemie; ale uciekl do Lachys. Przetoz poslawszy za nim do Lachys, zabili go tam.
\par 20 Skad przyniesli go na koniach, i pogrzebiony jest w Jeruzalemie z ojcami swymi, w miescie Dawidowem.
\par 21 A tak wziawszy wszystek lud Judzki Azaryjasza, któremu bylo szesnascie lat, postanowili go królem na miejscu ojca jego Amazyjasza.
\par 22 Ten pobudowal Elat, i przywrócil je do Judy, gdy zasnal król z ojcami swymi.
\par 23 Roku pietnastego Amazyjasza, syna Joaza, króla Judzkiego, królowal Jeroboam, syn Joaza, króla Izraelskiego, w Samaryi czterdziesci lat i rok.
\par 24 A czynil zle przed oczyma Panskiemi, nie uchylajac sie od wszystkich grzechów Jeroboama, syn Nabatowego, który przywiódl do grzechu Izraela.
\par 25 Ten zasie przywrócil granice Izraelskie od wejscia do Emat az do morza pustego, wedlug slowa Pana, Boga Izraelskiego, które byl wyrzekl przez sluge swego Jonasza, syna Amaty, proroka; który byl z Gatefer.
\par 26 Albowiem widzial Pan utrapienie Izraelskie, im dalej tem wieksze, tak, ze i wiezien, i opuszczony zniszczeni byli, a nie byl, ktoby ratowal Izraela.
\par 27 A nie rzekl byl Pan, aby mial wygladzic imie Izraela, zeby nie zostalo pod niebem: przetoz je wybawil przez reke Jeroboama, syna Joazaowego.
\par 28 A inne sprawy Jeroboamowe, i wszystko co czynil, i moc jego, która walczyl, i która przywrócil Damaszek i Emat Judzkie Izraelowi, azaz tego nie zapisano w kronikach o królach Izraelskich?
\par 29 I zasnal Jeroboam z ojcami swymi, z królmi Izraelskimi, a królowal Zacharyjasz, syn jego, miasto niego.

\chapter{15}

\par 1 Roku dwudziestego i siódmego Jeroboama, króla Izraelskiego, królowal Azaryjasz, syn Amazyjasza, króla Judzkiego.
\par 2 Szesnascie mu lat bylo, gdy poczal królowac, a piecdziesiat i dwa lata królowal w Jeruzalemie. Imie matki jego bylo Jechelija z Jeruzalemu.
\par 3 Ten czynil, co dobrego jest w oczach Panskich, wedlug wszystkiego, jako czynil Amazyjasz, ojciec jego.
\par 4 Wszakze wyzyny nie byly zniesione: jeszcze lud ofiarowal i kadzil po wyzynach.
\par 5 I zarazil Pan króla, a byl tredowaty az do smierci swej, i mieszkal w domu osobnym; przetoz Joatam, syn królewski, rzadzil domem, sadzac lud ziemi.
\par 6 A inne sprawy Azaryjaszowe, i wszystko co czynil, azaz tego nie zapisano w kronikach o królach Judzkich?
\par 7 I zasnal Azaryjasz z ojcami swymi, a pochowano go z ojcami jego w miescie Dawidowem; a królowal Joatam, syn jego, miasto niego.
\par 8 Roku trzydziestego i ósmego Azaryjasza, króla Judzkiego, królowal Zacharyjasz, syn Jeroboamowy, nad Izraelem w Samaryi szesc miesiecy.
\par 9 I czynil zle przed oczyma Panskimi, jako czynili ojcowie jego, nie odstepujac od grzechów Jeroboama, syna Nabatowego, który przywiódl do grzechu Izraela.
\par 10 I sprzysiagl sie przeciw niemu Sellum, syn Jabesowy, i ranil go przed ludem, i zabil go, a królowal miasto niego.
\par 11 A inne sprawy Zacharyjaszowe, oto sa napisane w kronikach o królach Izraelskich.
\par 12 Toc jest ono slowo Panskie, które powiedzial do Jehu, mówiac: Synowie twoi do czwartego pokolenia beda siedzieli na stolicy Izraelskiej. I tak sie stalo.
\par 13 Tedy Sellum, syn Jabesowy, królowal roku trzydziestego i dziewiatego roku Uzyjasza, króla Judzkiego, a królowal przez jeden miesiac w Samaryi.
\par 14 Bo przyciagnawszy Manachem, syn Gady, z Tersy, a przyszedlszy do Samaryi, porazil Selluma, syna Jabesowego w Samaryi, a zabiwszy go, królowal miasto niego.
\par 15 A inne sprawy Sellumowe, i sprzysiezenie jego, którem sie byl sprzysiagl, oto zapisane w kronikach o królach Izraelskich.
\par 16 Tedy dobyl Manachem miasta Tafsy, i pobil wszystkie, którzy w niem byli, i wszystki granice jego od Tersy; przeto, ze mu nie otworzyli, pomordowal je, i wszystkie brzemienne w niem porozcinal.
\par 17 Roku trzydziestego i dziewiatego Azaryjasza, króla Judzkiego, królowal Manachem, syn Gady, nad Izraelem dziesiec lat w Samaryi.
\par 18 I czynil zle przed oczyma Panskiemi, nie odstepujac od grzechów Jeroboama, syna Nabatowego, który do grzechu przywodzil Izraela po wszystkie dni swoje.
\par 19 A gdy wyciagnal Ful, król Assyryjski, przeciw ziemi Izraelskiej, dal Manachem Fulowi tysiac talentów srebra, aby mu byl na pomocy ku umocnieniu królestwa w rekach jego.
\par 20 I ulozyl Manachem podatek na Izraela, na wszystkie najbogatsze, aby dawali królowi Asyryjskiemu, po piecdziesiat syklów srebra, kazdy z osobna; i wrócil sie król Assyryjski, a nie bawil sie tam w onej ziemi.
\par 21 A inne sprawy Manachemowe, i cokolwiek czynil, napisane sa w kronikach o królach Izraelskich.
\par 22 I zasnal Manachem z ojcami swymi, a królowal Facejasz, syn jego, miasto niego.
\par 23 Roku piecdziesiatego Azaryjasza, króla Judzkiego, królowal Facejasz, syn Manachemowy, nad Izraelem w Samaryi dwa lata.
\par 24 I czynil zle przed oczyma Panskiemi, nie odstepujac od grzechu Jeroboama, syna Nabatowego, który przywiódl do grzechu Izraela.
\par 25 Tedy sie zbuntowal przeciwko niemu Facejasz, syn Romelijasza, hetman jego, i zabil go w Samaryi w palacu domu królewskiego, z Argobem i z Aryjaszem, majac z soba piecdziesiat mezów Galaadczyków, a zabiwszy go królowal miasto niego.
\par 26 A inne sprawy Facejaszowe i wszystko co czynil, oto napisano w kronikach o królach Izeraelskich.
\par 27 Roku piecdziesiatego i wtórego Azaryjasza, króla Judzkiego, królowal Facejasz syn Romelijasza, nad Izraelem w Samaryi dwadziescia lat.
\par 28 I czynil zle przed oczyma Panskiemi, nie odstepujac od grzechu Jeroboama, syna Nabatowego, który przywiódl do grzechu Izraela.
\par 29 Za dni Facejasza, króla Izraelskiego, przyciagnal Teglet Falaser, król Assyryjski, i wzial Ajon i Abelbetmaacha, i Jonoe, i Kiedes, i Azor, i Galaad, i Galilee, wszystke ziemie Neftali, a przeniósl obywatele jej do Assyryi.
\par 30 Tedy sie zbuntowal Ozeasz, syn Eli, przeciw Facejaszowi, synowi Romelijaszowemu, a raniwszy go, zabil go, i królowal miasto niego roku dwudziestego Joatama, syna Uzyjaszowego.
\par 31 A inne sprawy Facejaszowe, i wszystko co czynil, oto zapisano w kronikach o królach Izraelskich.
\par 32 Roku wtórego Facejasza, syna Romelijaszowego, króla Izraelskiego królowal Joatam, syn Uzyjasza, króla Judzkiego.
\par 33 Dwadziescia i piec lat mial, gdy królowac poczal, i szesnascie lat królowal w Jeruzalemie. Imie matki jego Jerusa, córka Sadokowa.
\par 34 I czynil, co dobrego jest przed oczyma Panskiemi; wedlug wszystkiego, co czynil Uzyjasz, ojciec jego, postepowal.
\par 35 Wszakze wyzyny nie byly zniesione; jeszcze lud ofiarowal i kadzil na wyzynach. Tenze zbudowal brame najwyzsza domu Panskiego.
\par 36 A inne sprawy Joatamowe, i wszystko co czynil, zapisane w kronikach o królach Judzkich.
\par 37 Za onych dni poczal Pan posylac na Jude Rasyna, króla Syryjskiego, i Facejasza, syna Romelijaszowego.
\par 38 I zasnal Joatam z ojcami swymi, i pogrzebiony jest z ojcami swymi w miescie Dawida, ojca swego. A królowal Achaz, syn jego, miasto niego.

\chapter{16}

\par 1 Roku siedmnastego Facejasza, syna Romelijaszowego, królowal Achaz, syn Joatama, króla Judzkiego.
\par 2 Dwadziescia lat bylo Achazowi, gdy królowac poczal, a szesnascie lat królowal w Jeruzalemie; ale nie czynil, co dobrego jest przed oczyma Pana, Boga swego, jako Dawid, ojciec jego;
\par 3 Lecz chodzil drogami królów Izraelskich. Nadto i syna swego dal przewiesc przez ogien wedlug obrzydliwosci poganów, które byl Pan wygnal przed obliczem synów Izraeliskich.
\par 4 Ofiarowal tez i kadzil na wyzynach, i na pagórkach, i pod kazdem drzewem galezistem.
\par 5 Tedy wyciagnal Rasyn, król Syryjski, i Facejasz, syn Romelijasza, król Izraelski, przeciwko Jeruzalemowi na wojne, i oblegli Achaza; wszakze go dobyc nie mogli.
\par 6 Tegoz czasu Rasyn, król Syryjski, przywrócil zasie Elat do Syryi, a wykorzenil Zydy z Elat, ale Syryjczycy przyszedlszy do Elat, mieszkali tam az do dnia tego.
\par 7 I poslal Achaz posly do Teglat Falasera króla Assyryjskiego, mówiac: Sluga twój i syn twój jestem. Przyciagnij a wybaw mie z reki króla Syryjskiego, i z reki króla Izraelskiego, którzy powstali przeciwko mnie.
\par 8 Tedy wziawszy Achaz srebro i zloto, które sie znalazlo w domu Panskim i w skarbach domu królewskiego, poslal dar królowi Assyryjskiemu.
\par 9 Na co mu przyzwolil król Assyryjski; a przyciagnawszy król Assyryjski pod Damaszek wzial go, i przeniósl obywatele jego do Chyr, a Rasyna zabil.
\par 10 Zatem jechal król Achaz przeciw Teglat Falaserowi, królowi Assyryjskiemu, do Damaszku; a ujrzawszy król Achaz oltarz w Damaszku, poslal do Uryjasza kaplana wizerunek oltarza onego i ksztalt jego, wedlug wszystkiego jako byl urobiony.
\par 11 I zbudowal Uryjasz kaplan oltarz wedlug onego wszystkiego, jako byl poslal król Achaz z Damaszku; tak uczynil Uryjasz kaplan pierwej, nizeli sie wrócil król Achaz z Damaszku.
\par 12 A gdy sie wrócil król z Damaszku, ujrzawszy oltarz przystapil do niego, i sprawowal ofiary na nim.
\par 13 I zapalil calopalenie swoje, i ofiare sniedna swoje, i ofiarowal ofiare mokra swoje, i kropil krwia ofiar spokojnych swoich na oltarzu.
\par 14 Ale oltarz miedziany, który byl przed Panem, przeniósl z przedniej strony domu, aby nie stal miedzy oltarzem jego, a miedzy domem Panskim; a postawil go po bok oltarza ku pólnocy.
\par 15 I rozkazal król Achaz Uryjaszowi kaplanowi, mówiac: Na tym wiekszym oltarzu bedziesz zapalal calopalenie poranne i ofiare sniedna wieczorna, i calopalenie królewskie, i ofiare sniedna jego, i calopalenie wszystkiego ludu ziemi, i ofiare ich sniedna, i ofiary mokre ich, i wszelka krwia calopalenia, i wszelka krwia innych ofiar bedziesz kropil na nim; ale oltarz miedziany bedzie mi na radzenie sie Boga.
\par 16 I uczynil Uryjasz kaplan wedlug wszystkiego, jako byl rozkazal król Achaz.
\par 17 Nadto poodcinal król Achaz listwy podstawków, i pozbieral z nich wanny; do tego morze zdjal z wolów miedzianych, które byly pod niem, a polozyl je na tle kamiennem.
\par 18 Zaslone takze sabatnia, która bylo sprawiono w domu, i drzwi zewnetrzne, któremi król wchadzal, odjal od domu Panskiego dla bojazni króla Assyryjskiego.
\par 19 A inne sprawy Achazowe, które czynil, zapisane sa w kronikach o królach Judzkich.
\par 20 I zasnal Achaz z ojcami swymi, i pogrzebiony jest z ojcami swymi w miescie Dawidowem. A królowal Ezechyjasz, syn jego miasto niego.

\chapter{17}

\par 1 Roku dwunastego Achaza, króla Judzkiego, królowal Ozeasz, syn Eli, w Samaryi nad Izraelem dziewiec lat.
\par 2 I czynil zle przed oczyma Panskiemi, wszakze nie tak jak inni królowie Izraelscy, którzy byli przed nim.
\par 3 Przeciwko niemu wyciagnal Salmanasar, król Assyryjski; i stal sie Ozeasz niewolnikiem jego, i dawal mu dan.
\par 4 A gdy obaczyl król Assyryjski, iz sie Ozeasz buntowal przeciw niemu, a iz wyprawil posly do Sua, króla Egipskiego, i nie posylal dani dorocznej królowi Assyryjskiemu, oblegl go król Assyryjski, a zwiazawszy podal go do wiezienia.
\par 5 I ciagnal król Assyryjski przez wszystke ziemie, az przyciagnal do Samaryi, pod która lezal przez trzy lata.
\par 6 A roku dziewiatego Ozeasza wzial król Assyryjski Samaryje, i przeniósl Izraela do Assyryi, a osadzil je w Hala i w Habor nad rzeka Gozan i w miastach Medskich.
\par 7 A to sie stalo przeto, ze grzeszyli synowie Izraelscy przeciw Panu, Bogu swemu, który je wywiódl z ziemi Egipskiej, aby nie byli pod moca Faraona, króla Egipskiego; a bali sie bogów cudzych,
\par 8 Chodzac w ustawach poganów, które byl Pan wyrzucil przed obliczem synów Izraelskich, i w ustawach królów Izraelskich, które czynili.
\par 9 Obludnie synowie Izraelscy postepowali, czyniac co nie bylo rzecza dobra przed Panem, Bogiem swym, i pobudowali sobie wyzyny po wszystkich miastach swych, od wiezy strazników az do miasta obronnego;
\par 10 A nastawiali sobie slupów, i gajów na kazdym pagórku wynioslym, pod kazdem drzewem galezistem,
\par 11 Palac tam kadzidla po wszystkich górach, jako narody, które wypedzil Pan przed obliczem ich; i czynili rzeczy co najgorsze, pobudzajac Pana ku gniewu,
\par 12 A sluzyli brzydkim balwanom, o którym im powiedzial Pan, aby tego nie czynili.
\par 13 I oswiadczal sie Pan przeciwko Izraelowi, i przeciwko Judzie, przez wszystkie proroki, i przez wszystkie widzace, mówiac: Nawróccie sie od dróg waszych zlych, a strzezcie rozkazania mego, i wyroków moich wedlug wszystkiego zakonu, którym rozkazal ojcom waszym, a z którymem poslal do was proroki, slugi moje.
\par 14 Lecz nie byli posluszni; ale zatwardzili kark swój wedlug karku ojców swych, którzy nie wierzyli w Pana, Boga swego.
\par 15 I wzgardzili wyroki jego, i przymierze jego, które uczynil z ojcami ich, i oswiadczenia jego, któremi sie oswiadczal przeciwko nim, a chodzili za próznoscia, i stali sie próznymi, i nasladowali poganów, którzy byli okolo nich, o których im rozkazal Pan, aby nie czynili jako oni.
\par 16 I opusciwszy wszystkie rozkazania Pana, Boga swego, poczynili sobie lane balwany, mianowicie dwóch cielców; poczynili tez gaje, a klaniali sie wszystkiemu wojsku niebieskiemu, i sluzyli Baalowi.
\par 17 Przewodzili tez syny i córki swe przez ogien, i bawili sie wieszczbami i wrózkami, i zaprzedali sie, aby czynili zle przed oczyma Panskiemi, pobudzajac go do gniewu.
\par 18 Przetoz sie bardzo Pan rozgniewal na Izraela, a odrzucil je od oblicza swego, nic z nich nie zostawujac, oprócz samego pokolenia Judy.
\par 19 Alec i Juda nie strzegl przykazan Pana, Boga swego; lecz chodzil w ustawach Izaelskich, których naczynili.
\par 20 Przetoz odrzucil Pan wszystko nasienie Izraelskie, i utrapil je, a podal je w reke lupiezcom, az je odrzucil od oblicza swego.
\par 21 Albowiem oderwal sie Izrael od domu Dawidowego, a postanowili królem Jeroboama, syna Nabatowego; ale Jeroboam odwiódl Izraela od nasladowania Pana, a przywiódl je do grzeszenia grzechem wielkim.
\par 22 I chodzili synowie Izraelscy we wszystkich grzechach Jeroboamowych, które on czynil, a nie odstapili od nich,
\par 23 A odrzucil Pan Izraela od oblicza swego, jako powiedzial przez wszystkie slugi swe proroki; a tak przeniesiony jest Izrael z ziemi swej do Assyryi, az do dnia tego.
\par 24 Potem przyprowadzil król Assyryjski lud z Babilonu, i z Kuta, i z Awa, i z Emat, i z Sefarwaim, a osadzil je w miastach Samaryi miasto synów Izraelskch; którzy posiadlszy Samaryje, mieszkali w miastach jej.
\par 25 A gdy tam oni mieszkac poczeli a nie bali sie Pana, poslal Pan na nie lwy, którzy je zabijali.
\par 26 I powiedziano to królowi Assyryjskiemu, mówiac: Narodowie, któres przeniósl i osadzil w miastach Samaryi, nie wiedza obyczaju Boga onej ziemi; przetoz poslal na nie lwy, a oto je zabijaja, dla tego, iz nie wiedza obyczaju Boga onej ziemi.
\par 27 Tedy rozkazal król Assyryjski, mówiac: Zawiedzcie tam jednego z kaplanów, którescie stamtad przywiedli, aby poszedlszy mieszkal tam, i nauczal ich obyczaju Boga onej ziemi.
\par 28 Przyszedl tedy jeden z kaplanów, których bylo wzieto z Samryi, i mieszkal w Betel, a nauczal ich, jako sie mieli bac Pana.
\par 29 Wszakze naczynili sobie kazdy naród bogów swych, i postawili je w domu wyzyn, które byli pobudowali Samaryjczycy, kazdy naród w miastach swych, w których mieszkali,
\par 30 Albowiem mezowie Babilonscy uczynili Sukkotbenot, a mezowie Kutscy uczynili Nergiel, a mezowie Ematscy uczynili Asyma.
\par 31 A Hewejczycy uczynili Nebahaz, i Tartak; a Sefarwaiczycy palili syny swe w ogniu Adramelechowi, i Anamelechowi, bogom Sefarwaimskim.
\par 32 A tak bali sie Pana, naczyniwszy sobie z posrodku siebie kaplanów na wyzynach, którzy im uslugiwali w domach wyzyn.
\par 33 A choc sie Pana bali, wszakze przecie bogom swoim sluzyli wedlug zwyczajów onych narodów, skad byli przeniesieni.
\par 34 Ci az do dnia tego sprawuja sie wedlug zwyczajów starych, nie boja sie Pana, ani czynia wedlug wyroków jego, i wedlug ustaw jego, i wedlug zakonu, i wedlug rozkazania, które przykazal Pan synom Jakóbowym, którego przezwal Izraelem.
\par 35 Uczynil tez byl Pan z nimi przymierze, i rozkazal im, mówiac: Nie bójcie sie bogów cudzych, i nie klaniajcie sie im, ani im sluzcie, ani im ofiarujcie;
\par 36 Ale Pana, który was wywiódl z ziemi Egipskiej moca wielka i ramieniem wyciagnionem, tego sie bójcie, i jemu sie klaniajcie, i jemu ofiarujcie;
\par 37 Takze ustaw, i sadów, i zakonu, i przykazan, które wam napisal, strzezcie, czyniac je po wszystkie dni, a nie bójcie sie bogów cudzych.
\par 38 Wiec przymierza, którem czynil z wami, nie zapominajcie, ani sie bójcie bogów cudzych.
\par 39 Ale Pana, Boga waszego, sie bójcie, a on was wybawi z reki wszystkich nieprzyjaciól waszych;
\par 40 Lecz nie usluchali, ale owszem wedlug obyczaju swego dawnego czynili.
\par 41 A tak narodowie oni bali sie Pana, wszakze przecie rytym balwanom swoim sluzyli; a synowie ich, i synowie synów ich, wedlug wszystkiego, co czynili ojcowie ich, tak i oni czynia, az po dzis dzien.

\chapter{18}

\par 1 Roku trzeciego Ozeasza, syna Eli, króla Izraelskiego, królowal Ezechyjasz, syn Achaza, króla Judzkiego.
\par 2 Dwadziescia i piec lat mu bylo, gdy poczal królowac, a dwadziescia i dziewiec lat królowal w Jeruzalemie. Imie matki jego bylo Abi, córka Zacharyjaszowa.
\par 3 I czynil co bylo dobrego przed oczyma Panskiemi, wedlug wszystkiego, jako czynil Dawid, ojciec jego.
\par 4 On zniósl wyzyny, i skruszyl balwany, i powycinal gaje, a pokruszyl weza miedzianego, którego byl uczynil Mojzesz; bo az do onych dni Izraelczycy kadzili mu, i nazwal go Nehustan.
\par 5 W Panu Bogu Izraelskim ufal; a po nim nie byl zaden podobny jemu miedzy wszystkimi królami Judzkimi, i którzy byli przed nim.
\par 6 Bo sie trzymal Pana, nie odstepujac od niego, a strzegac przykazania jego, które byl przykazal Pan Mojzeszowi.
\par 7 A Pan byl z nim; i we wszystkiem, do czego sie obrócil, szczescilo mu sie. Wybil sie tez z mocy królowi Assyryjskiemu, i nie sluzyl mu.
\par 8 Tenze porazil Filistyny az do Gazy i granic jego, od wiezy strazników az do miasta obronnego.
\par 9 Roku czwartego króla Ezechyjasza, (który byl rok siódmy Ozeasza, syna Eli, króla Izraelskiego) wyciagnal Salmanaser, król Assyryjski, przeciwko Samaryi, i oblegl ja.
\par 10 A wzial ja przy dokonczeniu trzeciego roku; roku szóstego Ezechyjasza, (który byl rok dziewiaty Ozeasza, króla Izraelskiego) wzieta jest Samaryja.
\par 11 Tedy przeniósl król Assyryjski Izraela do Assyryi, i osadzil nimi Halach i Habor u rzeki Gazan, i miasta Medskie.
\par 12 Przeto, iz nie posluchali glosu Pana Boga swego, ale przestepowali przymierze jego, i tego wszystkiego, co rozkazal Mojzesz, sluga Panski, nie sluchali i nie czynili.
\par 13 Potem czternastego roku króla Ezechyjasza ruszyl sie Sennacheryb, król Assyryjski, przeciw wszystkim miastom Judzkim obronnym, i wzial je.
\par 14 A tak poslal Ezechyjasz, król Judzki, do króla Assyryjskiego, do Lachys, mówiac: Zgrzeszylem; odciagnij odemnie, cokolwiek na mie wlozysz, poniose. Tedy wlozyl król Assyryjski na Ezechyjasza, króla Judzkiego, dan trzy sta talentów srebra, i trzydziesci talentów zlota.
\par 15 I dal Ezechyjasz wszystko srebro, które sie znalazlo w domu Panskim i w skarbach domu królewskiego.
\par 16 Onegoz czasu oblupil Ezechyjasz drzwi domu Panskiego, i slupy, które samze Ezechyjasz, król Judzki, byl obyl, a dal je królowi Assyryjskiemu.
\par 17 Wszakze poslal król Assyryjski Tartana, i Rabsarysa, i Rabsacesa z Lachys do króla Ezechyjasza z wielkiem wojskiem do Jeruzalemu. Którzy wyciagnawszy przyjechali ku Jeruzalemowi, a przyciagnawszy przyszli i polozyli sie u rur sadzawki wyzszej, która jest podle drogi brukowanej na polu blecharzowem.
\par 18 A gdy wolali na króla, wyszedl do nich Elijakim, syn Helkijaszowy, przelozony nad domem, i Sobna pisarz, i Joach syn Asafowy, kanclerz.
\par 19 I rzekl do nich Rabsaces: Prosze powiedzcie Ezechyjaszowi: Tak mówi król wielki, król Assyryjski: Co to za ufnosc, na której sie wspierasz?
\par 20 Mówiles: (alec to slowa daremne) Rady i mocy mam dosyc do wojny. A teraz w kimze ufasz, zes mi sie sprzeciwil?
\par 21 Oto teraz spolegasz na Egipcie, jako na lasce trzcinnej, i to nalamanej, która jezliby sie kto podpieral, tedy wnijdzie w reke jego i przekole ja. Takic jest Farao, król Egipski, wszystkim, co w nim ufaja.
\par 22 A jezli mi rzeczecie: W Panu Bogu naszym ufnosc mamy: azaz nie ten jest, którego zniósl Ezechyjasz wyzyny i oltarze? i rozkazal Judzie i Jeruzalemowi, mówiac: Przed tym oltarzem klaniac sie bedziecie w Jeruzalemie.
\par 23 Przetoz teraz zarecz sie królowi Assyryjskiemu, panu memu, a dam ci dwa tysiace koni; bedzieszli mógl miec jezdnych tak wiele do dnich?
\par 24 I jakoz sie ty mozesz oprzec hetmanowi jednemu najmniejszemu z slug pana mego? choc pokladasz nadzieje w Egipcie dla wozów i jezdnych.
\par 25 Nadto, czy bez woli Panskiej przyciagnalem przeciw temu miejscu, abym je zburzyl? Pan mówi do mnie: Idz do tej ziemi, a spustosz ja.
\par 26 Tedy rzekl Elijakim, syn Helkijaszowy, i Sobna, i Joach do Rabsacesa: Prosze mów do slug twoich po syryjsku, boc rozumiemy; a nie mów z nami po zydowsku, gdzie slyszy lud, który jest na murze.
\par 27 Którym odpowiedzial Rabsaces: Azaz mie do pana twego, albo do ciebie przyslal Pan mój, abym te slowa mówil? Azaz nie do tych mezów, którzy siedza na murze, aby jedli lajna swoje, i pili mocz swój z wami.
\par 28 A tak stanawszy Rabsaces wolal glosem wielkim po zydowsku, a mówiac rzekl: Sluchajcie slów króla wielkiego, króla Assyryjskiego.
\par 29 Tak mówi król: Niech was nie zwodzi Ezechyjasz; bo was nie bedzie mógl wyrwac z reki mojej.
\par 30 A niech wam nie rozkazuje ufac Ezechyjasz w Panu, mówiac: Pewnie nas wyrwie Pan, a nie bedzie podane to miasto w rece króla Assyryjskiego.
\par 31 Nie sluchajcie Ezechyjasza. Albowiem tak mówi król Assyryjski: Uczyncie ze mna przymierze, a wynijdzcie do mnie, a jedzcie kazdy z winnicy swojej i kazdy z figi swojej, i pijcie kazdy wode z studnicy swojej,
\par 32 Az przyjde a pobiore was do ziemi podobnej ziemi waszej, do ziemi zyznej i obfitujacej winem, do ziemi chleba i winnic, do ziemi drzew oliwnych, i oliwy, i miodu; i bedziecie zyli, a nie pomrzecie. Nie sluchajciez Ezechyjasza; bo was zwodzi, mó wiac: Pan was wybawi.
\par 33 Izaz mogli bogowie narodów wybawic kazdy ziemie swoje, z reki króla Assyryjskiego?
\par 34 Gdziez jest bóg Emat i Arfad? gdziez jest bóg Sefarwaim, Ana i Awa? izali wyrwali Samaryje z rak moich?
\par 35 Któryz jest miedzy wszystkimi bogi tych ziem, któryby wyrwal ziemie swoje z reki mojej? A mialby Pan wyrwac Jeruzalem z reki mojej?
\par 36 Ale milczal lud, i nie odpowiedzieli mu slowa; bo takie bylo rozkazanie królewskie, mówiac: Nie odpowiadajcie mu.
\par 37 Przyszedl tedy Elijakim, syn Helkijaszowy, przelozony domu, i Sobna pisarz, i Joach, syn Asafowy, kanclerz, do Ezechyjasza, rozdarlszy szaty swe, i oznajmili mu slowa Rabsacesowe.

\chapter{19}

\par 1 A gdy to uslyszal król Ezechyjasz, rozdarl szaty swoje, a oblekl sie w wór, i wszedl do domu Panskiego;
\par 2 I poslal Elijakima, sprawce domu swego, i Sobne pisarza, i starsze z kaplanów, obleczone w wory, do Izajasza proroka, syna Amosowego.
\par 3 Którzy rzekli do niego: Tak mówi Ezechyjasz: Dzien utrapienia i lajania, i bluznierstwa jest ten dzien; albowiem synowie przyszli az do porodzenia, a sily niemasz ku rodzeniu.
\par 4 Oby uslyszal Pan, Bóg twój, wszystkie slowa Rabsacesowe, którego przyslal król Assyryjski, pan jego, uragac Bogu zywemu! aby sie pomscil onych slów, które slyszal Pan, Bóg twój. Przetoz uczyn modlitwe za te ostatki, które sie znajduja.
\par 5 Przyszli tedy sludzy króla Ezechyjasza do Izajasza.
\par 6 Którym odpowiedzial Izajasz: Tak powiedzcie panu waszemu: To mówi Pan: Nie bój sie tych slów, któres slyszal, któremi mie lzyli sludzy króla Assyryjskiego.
\par 7 Oto Ja puszcze nan ducha, i uslyszy wiesc, a wróci sie do ziemi swojej, i poloze go mieczem w ziemi jego.
\par 8 Ale wróciwszy sie Rabsaces znalazl króla Assyryjskiego dobywajacego Lebny; albowiem uslyszal, iz odciagnal byl od Lachys.
\par 9 A uslyszawszy o Tyraku, królu Etyjopskim, ze mówino: Oto wyciagnal na wojne przeciwko tobie, znowu pslal posly do Ezechyjasza, mówiac:
\par 10 To powiedzcie królowi Ezechyjaszowi, królowi Judzkiemu, mówiac: Niech cie nie zwodzi Bóg twój, któremu ty ufasz, a mówisz: Nie bedzie podane Jeruzalem w rece króla Assyryjskiego.
\par 11 Otos slyszal, co poczynili królowie Assyryjscy wszystkim ziemiom, burzac je; a tybys mial byc wybawiony?
\par 12 Izali wybawili bogowie narodów te, które wygubili ojcowie moi, Gozan, i Haran, i Resef, i syny Eden, którzy byli w Telassar?
\par 13 Gdziez jest król Emat, i król Arfad, i król miasta Sefarwaim, Ana i Awa?
\par 14 Przetoz wziawszy Ezechyjasz list z reki poslów, przeczytal go, i wszedlszy do domu Panskiego rozciagnal go Ezechyjasz przed Panem.
\par 15 I modlil sie Ezechyjasz przed Panem, mówiac: Panie, Boze Izraelski, siedzacy na Cherubinach! ty, tys sam jest Bóg wszystkich królestw ziemi, tys stworzyl niebo i ziemie.
\par 16 Naklonze, Panie! ucha twojego, a uslysz; otwórz, Panie! oczy twoje, a obacz; uslysz slowa Sennacheryba, który przyslal hanbic ciebie, Boga zywego.
\par 17 Prawdac jest, Panie! ze spustoszyli królowie Assyryjscy narody one, i ziemie ich.
\par 18 I powrzucali bogi ich w ogien; albowiem nie byli bogowie, ale robota rak ludzkich, drewno, i kamien; przetoz je wygubili.
\par 19 A teraz, Panie Boze nasz! wybaw nas prosze z reki jego, aby poznaly wszystkie królestwa ziemi, zes ty, Panie! sam Bogiem.
\par 20 Tedy poslal Izajasz, syn Amosowy, do Ezechyjasza, mówiac: Tak mówi Pan, Bóg Izraelski: O cos mie prosil z strony Sennacheryba, króla Assyryjskiego, wysluchalem cie.
\par 21 A tec sa slowa, które mówil Pan o nim: Panna, córka Syonska, wzgardzila cie, smiala sie z ciebie, kiwala glowa za toba córka Jeruzalemska.
\par 22 Kogozes hanbil, i kogo bluznil? przeciwko komuzes podniósl glos, a wyniosles ku górze oczy swoje? przeciw Swietemu Izraelskiemu.
\par 23 Przez posly twoje hanbiles Pana mego, i rzekles: W mnóstwie wozów moich wstapilem na wysokie góry, i na strony Libanskie, i podrabie wysokie cedry jego, i wyborne jodly jego, i przyjde az do ostatnich przybytków jego, do lasów, i wybornych ról jego.
\par 24 Jam wykopal zródla, i pilem wody cudze, a wysuszylem stopami nóg moich wszystkie potoki oblezonych.
\par 25 Izazes nie slyszal, zem ja zdawna uczynil a od dni starodawnych stworzylem je? a teraz mialzebym na nie przywiesc spustoszenie, i obrócic w gromady gruzu; jako insze miasta obronne?
\par 26 Których obywatele stali sie jako bez rak, przestraszeni sa i zawstydzeni, bywszy jako trawa polna, i jako ziola zielone, i trawy po dachach, które pierwej schna, niz sie dostaja,
\par 27 Mieszkanie twoje i wyjscie twoje, i wejscie twoje znam, takze popedliwosc twoje przeciwko mnie.
\par 28 Poniewazes sie przeciwko mnie zajuszyl, a zapedy twoje przyszly do uszów moich, przetoz zaloze kolce moje za nozdrza twoje, a wedzidlo moje wprawie w gebe twoje, i wróce cie ta droga, któras przyszedl,
\par 29 A to bedziesz mial, Ezechyjaszu! za znak: Tego roku bedziesz jadl samorodne zboze, i roku takze drugiego samorodne zboze; ale roku trzeciego bedziecie siac i zac, i sadzic winnice i jesc owoc ich.
\par 30 Ostatek bowiem domu Judy, który pozostal, wkorzeni sie gleboko, i wyda owoc ku górze.
\par 31 Albowiem z Jeruzalemu wynijda ostatki, i ci, którzy sa zachowani z góry Syonskiej. Gorliwosc Pana zastepów to uczyni.
\par 32 A przetoz tak mówi Pan o królu Assyryjskim: Nie wnijdzie do miasta tego, ani tam dojdzie strzala jego, ani go ubiezy tarcza, ani usypie szanców okolo niego;
\par 33 Droga, która przyszedl, wróci sie, a do miasta tego nie wnijdzie, mówi Pan.
\par 34 Bo bede bronil miasta tego, i zachowam je sam dla siebie, i dla Dawida, slugi mego.
\par 35 I stalo sie onej nocy, ze wyszedl Aniol Panski, a pobil w obozie Assyryjskim sto osmdziesiat i piec tysiecy. A gdy wstali rano, oto wszedy pelno trupów.
\par 36 Przetoz ruszywszy sie odjechal i wrócil sie Seneacheryb, król Assyryjski, a mieszkal w Niniwe.
\par 37 A gdy chwalil boga swego Nesrocha w domu, tedy Adramelech i Sarassar, synowie jego, zabili go mieczem, a sami uciekli do ziemi Ararat. I królowal Assarhaddon, syn jego, miasto niego.

\chapter{20}

\par 1 W one dni zachorowal Ezechyjasz az na smierc, i przyszedl do niego Izajasz prorok, syn Amosowy, i rzekl mu: Tak mówi Pan: Rozpraw dom twój; bo umrzesz, a nie bedziesz zyl.
\par 2 Tedy obrócil Ezechyjasz twarz swoje do sciany, i modlil sie Panu, mówiac:
\par 3 Prosze, o Panie! wspomnij teraz, zem chodzil przed toba w prwadzie, i w sercu calem, czyniac to, co dobrego jest w oczach twoich. I plakal Ezechyjasz placzem wielkiem.
\par 4 Ale jeszcze Izajasz nie wyszedl byl do pól sieni, gdy sie slowo Panskie stalo do niego, mówiac:
\par 5 Wróc sie, a mów do Ezechyjasza, wodza ludu mego: Tak mówi Pan, Bóg Dawida, ojca twego: Wysluchalem modlitwe twoje, a widzialem lzy twoje; oto Ja uzdrawiam cie, dnia trzeciego wnijdziesz do domu Panskiego;
\par 6 I przydam do dni twoich pietnascie lat, a z reki króla Assyryjskiego wyrwe ciebie, i to miasto; i bronic bede tego miasta dla siebie, i dla Dawida, slugi mego.
\par 7 Przytem rzekl Izajasz: Przyniescie bryle fig suchych. Która przynióslszy wlozyli na wrzód, i zgoil sie.
\par 8 I rzekl Ezechyjasz do Izajasza: Jaki znak tego, ze mie uzdrowi Pan, a iz pójde dnia trzeciego do domu Panskiego?
\par 9 Odpowiedzial Izajasz: Toc bedzie znakiem od Pana, iz uczyni Pan te rzecz, którac obiecal. Chceszze, zeby cien postapil na dziesiec stopni, albo zeby sie na wstecz nawrócil na dziesiec stopni?
\par 10 I rzekl Ezechyjasz: Snadniej moze cien postapic na dól na dziesiec stopni, tego nie chce; ale niech sie wróci cien na wstecz na dziesiec stopni.
\par 11 Tedy wolal Izajasz prorok do Pana; i nawrócil cien po onych stponiach, któremi byl postapil na zegarze slonecznym Achazowym, na wstecz na dziesiec stopni.
\par 12 Onegoz czasu poslal Berodach Baladan, syn Baladanowy, król Babilonski, list i dary do Ezechyjasza; bo zaslyszal, ze byl zaniemógl Ezechyjasz.
\par 13 I wysluchal ich Ezechyjasz i okazal im wszystkie skarbnice klejnotów swoich, srebro, i zloto, i rzeczy wonne, i olejki najwyborniejsze, i dom rynsztunków swoich, i wszystko, co sie znajdowalo w skarbach jego; nie bylo nic czego by im nie pokazal Ezechyjasz w domu swym, i we wszystkiem panstwie swojem.
\par 14 Przetoz przyszedl prorok Izajasz do króla Ezechyjasza, i rzekl mu: Coc powiedzieli ci mezowie, a skad przyszli do ciebie? I odpowiedzial Ezechyjasz: Z ziemi dalekiej przyszli z Babilonu.
\par 15 I rzekl: Cóz widzieli w domu twoim? Odpowiedzial Ezechyjasz: Wszystko, cokolwiek jest w domu moim, widzieli: nie bylo nic, czegobym im nie pokazal w skarbach moich.
\par 16 Ale Izajasz rzekl do Ezechyjasza: Sluchaj slowa Panskiego.
\par 17 Oto przyjda dni, w które zabiora wszystko do Babilonu, cokolwiek jest w domu twoim, i cokolwiek schowali ojcowie twoi az do dnia tego; nie zostanie ci nic, mówi Pan.
\par 18 Ale i syny twoje, którzy wynijda z ciebie, i które splodzisz, pobiora, i beda komornikami na dworze króla Babilonskiego.
\par 19 Tedy rzekl Ezechyjasz do Izajasza: Dobre jest slowo Panskie, któres mówil. Nadto rzekl: Zaiste dobre, jezli tylko pokój i prawda bedzie za dni moich.
\par 20 Ale inne sprawy Ezechyjaszowe, i wszystka moc jego, i jako uczynil sadzawke, i rury, którymi przywiódl wode do miasta to zapisano w kronikach o królach Judzkich.
\par 21 I zasnal Ezechyjasz z ojcami swymi, a królowal Manases, syn jego, miasto niego.

\chapter{21}

\par 1 We dwunastym roku byl Manases, gdy królowac poczal, a piecdziesiat i piec lat królowal w Jeruzalemie; a imie matki jego bylo Hadsyba.
\par 2 I czynil zle przed oczyma Panskiemi wedlug obrzydlosci tych narodów, które wygnal Pan przed obliczem synów Izraelskich.
\par 3 Albowiem znowu pobudowal wyzyny, które byl poburzyl Ezechyjasz, ojciec jego, i wystawil oltarze Baalowi, i nasadzil gaj jako byl uczynil Achab, król Izraelski, i klanial sie wszystkiemu wojsku niebieskiemu, i sluzyl mu.
\par 4 Pobudowal tez oltarze w domu Panskim, o którym powiedzial byl Pan: W Jeruzalemie poloze imie moje.
\par 5 Nadto nabudowal oltarzy wszystkiemu wojsku niebieskiemu w obu sieniach domu Panskiego.
\par 6 Syna takze swego przewiódl przez ogien, i przestrzegal czasów, i bawil sie wieszczba, i ustawil czarnoksiezniki, i guslarze, a bardzo wiele zlego czynil przed oczyma Panskiemi, drazniac go.
\par 7 Postawil takze balwana gajowego, którego byl uczynil w domu, o którym byl rzekl Pan do Dawida i do Salomona, syna jego: W domu tym i w Jeruzalemie, którem obral ze wszystkiego pokolenia Izraelskiego, poloze imie moje na wieki:
\par 8 A wiecej sie nie dopuszcze ruszyc nodze Izraela z ziemi, któram dal ojcom ich, by jedno skutecznie strzegli wszystkiego, com im rozkazal, i wszystkiego zakonu, który im przykazal sluga mój Mojzesz.
\par 9 Ale nie sluchali; bo je zwiódl Manases, tak iz sie gorzej sprawowali niz narody, które wygladzil Pan przed obliczem synów Izraelskich.
\par 10 Aczkolwiek powiedzial byl Pan przez slugi swoje proroki, mówiac:
\par 11 Przeto, ze czynil Manases, król Judzki, te obrzydliwosci, czyniac gorsze rzeczy nad one wszystkie, które czynili Amorejczycy, którzy byli przed nim, a ze przywiódl w grzech i Jude przez brzydkie balwany swoje;
\par 12 Przetoz tak mówi Pan, Bóg Izraelski: Oto Ja przywiode zle na Jeruzalem i na Jude, tak iz kazdemu, co to uslyszy, zabrzmi w obu uszach jego.
\par 13 Bo rozciagne nad Jeruzalem sznur Samaryjski, i wage domu Achabowego, a wytre Jeruzalem, jako kto wyciera mise, a wytarlszy przewraca ja dnem ku górze.
\par 14 I opuszcze ostatki dziedzictwa mego, a podam je w reke nieprzyjaciól ich; i beda na lup, i na rozproszenie wszystkim nieprzyjaciolom swoim.
\par 15 Przeto, iz sie dopuszczali zlego przed oczyma memi, a draznili mie ode dnia, którego wyszli ojcowie ich z Egiptu, az do dzisiejszego dnia.
\par 16 Nadto i krwi niewinnej Manases wylal bardzo wiele, tak iz nia napelnil Jeruzalem od konca do konca, oprócz grzechu swego, przez który przywiódl do grzechu Jude, aby czynil zle przed oczyma Panskiemi.
\par 17 A inne sprawy Manasesowe, i wszystko co czynil, i grzech jego, którego sie dopuscil, to zapisano w kronikach o królach Judzkich.
\par 18 I zasnal Manases z ojcami swymi, i pogrzebiony jest w ogrodzie domu swego, w ogrodzie Ozy; a królowal Amon, syn jego, miasto niego.
\par 19 Dwadziescia i dwa lata mial Amon, gdy królowac poczal, a dwa lata królowal w Jeruzalemie. A imie matki jego bylo Masallemet, córka Harusa z Jateby.
\par 20 I czynil zle przed oczyma Panskiemi, jako czynil Manases, ojciec jego.
\par 21 A chodzil wszystkimi drogami, któremi chodzil ojciec jego, sluzac brzydkim balwanom, którym sluzyl ojciec jego, i klanial sie im;
\par 22 I opuscil Pana, Boga ojców swoich, a nie chodzil droga Panska.
\par 23 Ale sie sprzysiegli sludzy Amonowi przeciwko niemu, i zabili króla w domu jego.
\par 24 Tedy pobil lud onej ziemi wszystkie, którzy sie byli sprzysiegli przeciwko królowi Amonowi; i postanowil lud onej ziemi królem Jozyjasza, syna jego, miasto niego.
\par 25 Ale inne sprawy Amonowe, które czynil, opisane sa w kronikach o królach Judzkich.
\par 26 I pochowano go w grobie jego w ogrodzie Ozy; a królowal Jozyjasz, syn jego, miasto niego.

\chapter{22}

\par 1 Osm lat bylo Jozyjaszowi, gdy poczal królowac, a trzydziesci jeden lat królowal w Jeruzalemie; a imie matki jego bylo Jedyda, córka Adaja z Besekatu.
\par 2 I czynil, co bylo dobrego przed oczyma Panskiemi, chodzac wszystkimi drogami Dawida, ojca swego, a nie uchylal sie ani na prawo ani na lewo.
\par 3 A osmnastego roku króla Jozyjasza poslal król Safana, syna Azalijaszowego, syna Mesulama, pisarza, do domu Panskiego, mówiac:
\par 4 Idz do Helkijasza, kaplana najwyzszego, aby zebral pieniadze, które wnoszono do domu Panskiego, które wybierali strózowie progu od ludu.
\par 5 A niech je dawaja w rece rzemieslników, przelozonych nad robota domu Panskiego, aby je dawali robotnikom, którzy robili w domu Panskim, naprawiajac skaze domu;
\par 6 To jest, budownikom i cieslom, i murarzom, i na zkupowanie drzewa, i kamienia ciosanego ku naprawie domu.
\par 7 Wszakze niech nie czynia liczby z pieniedzy, które dawaja do rak ich; bo oni wiernie nimi szafowac beda.
\par 8 I rzekl Helkijasz, kaplan najwyzszy, do Safana pisarza. Ksiegi zakonu znalazlem w domu Panskim. I dal Helkijasz one ksiegi Safanowi, i czytal je (Safan).
\par 9 Przyszedlszy tedy Safan pisarz do króla, odniósl to królowi, i rzekl: Zebrali sludzy twoi pieniadze, które sie znalazly w domu Panskim, i oddali je w rece rzemieslników przelozonych nad robota w domu Panskim.
\par 10 Oznajmil tez Safan pisarz królowi, mówiac: Dal mi ksiege Helkijasz kaplan; i czytal ja Safan przed królem.
\par 11 A gdy uslyszal król slowa ksiag zakonu, rozdarl szaty swe.
\par 12 I rozkazal król Helkijaszowi kaplanowi, i Ahykamowi, synowi Safanowemu, i Achborowi, synowi Micheaszowemu, i Safanowi pisarzowi, i Asajaszowi, sludze swemu, mówiac:
\par 13 Idzcie, poradzcie sie Pana o mie, i o lud, i o wszystkiego Jude z strony slów tych ksiag, które sa znalezione; bo wielki jest gniew Panski, który sie zapalil przeciwko nam, przeto iz nie posluchali ojcowie nasi slów tych ksiag, zeby czynili wedlug wszystkiego, co nam jest napisane.
\par 14 A tak poszedl Helkijasz kaplan, i Ahykam, i Achbor, i Safan, i Azajasz, do Huldy prorokini, zony Selluma, syna Tekui, syna Araaszowego, który byl strózem szat; a ona mieszkala w Jeruzalemie na drugiej stronie miasta; i mówili z nia.
\par 15 Która rzekla do nich: Tak mówi Pan, Bóg Izraelski. Powiedzcie mezowi, który was poslal do mnie;
\par 16 Tak mówi Pan: Oto Ja przywiode zle na to miejsce i na obywateli jego wedlug wszystkich slów ksiag tych, które czytal król Judzki.
\par 17 Przeto, ze mie opuscili, i kadzili bogom cudzym, aby mie draznili wszystkiemi sprawami rak swoich, dla czego rozpalila sie popedliwosc moja przeciwko miejscu temu, i nie bedzie ugaszona.
\par 18 A królowi Judzkiemu, który was poslal o rade do Pana, tak powiedzcie: Tak mówi Pan, Bóg Izraelski, o slowach, któres slyszal:
\par 19 Poniewaz zmiekczone jest serce twoje, a upokorzyles sie przed obliczem Panskiem, slyszac, com powiedzial przeciwko temu miejscu, i przeciwko obywatelom jego, iz ma przyjsc w spustoszenie i w przeklestwo; rozdarles szaty swe, a plakales przedemna, i Jam cie tez wysluchal, mówi Pan.
\par 20 Przetoz oto Ja cie zbiore do ojców twoich, a bedziesz zebrany do grobu twego w pokoju, aby nie ogladaly oczy twoje wszystkiego zlego, które Ja przywiode na to miejsce. I odniesiono te rzecz królowi.

\chapter{23}

\par 1 Tedy poslawszy król, aby sie zebrali do niego wszyscy starsi Judzcy i Jeruzalemscy,
\par 2 Wstapil król do domu Panskiego, i wszyscy mezowie Judzcy, i wszyscy obywatele Jeruzalemscy z nim, i kaplani, i prorocy, i wszystek lud od malego az do wielkiego; i czytal, gdzie wszyscy slyszeli wszystkie slowa ksiag przymierza, które byly znalezione w domu Panskim.
\par 3 Potem stanal król na majestacie, i uczynil przymierze przed Panem, ze chce chodzic za Panem, i strzedz rozkazania jego, i swiadectw jego, i wyroków jego ze wszystkiego serca, i ze wszystkiej duszy, i pelnic slowa przymierza tego, które byly napisa ne w onych ksiegach. I przestal lud na onem przymierzu.
\par 4 I przykazal król Helkijaszowi, kaplanowi najwyzszemu, i kaplanom wtórego rzedu, i odzwiernym, aby wyrzucili z kosciola Panskiego wszystko naczynie, które sprawione bylo Baalowi, i gajowi poswieconemu, i wszystkiemu wojsku niebieskiemu, i spalil je precz za Jeruzalemem na polu Cedron, a zaniósl popiól ich do Betela.
\par 5 Zlozyl tez z urzedu popów, których byli postanowili królowie Judzcy, aby kadzili po wyzynach w miastach Judzkich i okolo Jeruzalemu; przytem i onych, którzy kadzili Baalowi, sloncu i miesiacowi, i planetom, i wszystkiemu wojsku niebieskiemu.
\par 6 Kazal tez wyniesc gaj swiecony z domu Panskiego precz z Jeruzalemu ku potokowi Cedron, a spalil go u potoku Cedron, i starl go w proch, a popiól jego rozmiotal na groby synów onegoz ludu.
\par 7 Zburzyl tez domy Sodomczyków, które byly w domu Panskim, kedy niewiasty tkaly opony do gaju poswieconego.
\par 8 I zawolal wszystkich kaplanów z miast Judzkich, a splugawil wyzyny, na których kadzili, od Gabaa az do Beerseba, i popsul wyzyny przy bramach, które byly w wejsciu bramy Jozuego ksiazecia miasta, a bylo po lewej stronie wchodzacemu w brame miejska.
\par 9 Wszakze nie przystepowali kaplani wyzyn do oltarza Panskiego w Jeruzalemie, ale jadali chleby przasne miedzy bracmi swoimi.
\par 10 Splugawil tez i Tofet, które bylo w dolinie syna Hennomowego, aby wiecej nikt nie przewodzil syna swego, ani córki swojej przez ogien ku czci Molochowi.
\par 11 Zagubil tez one konie, które byli królowie Judzcy oddali sloncu, a staly, kedy wchodza do domu Panskiego, podle mieszkania Natanmelecha dworzanina, które bylo na przedmiesciu; i wozy slonca spalil ogniem.
\par 12 Takze oltarze, które byly na dachu sali Achazowej, które byli poczynili królowie Judzcy, i oltarze, które byl poczynil Manases w obu sieniach domu Panskiego, pokazil król; a pospieszywszy sie stamtad kazal wrzucic proch ich w potok Cedron.
\par 13 Wyzyny takze, które byly przed Jeruzalem, i które byly po prawej stronie góry Oliwnej, których byl nabudowal Salomon, król Izraelski, Astarotowi, obrzydlosci Sydonczyków, i Chamosowi, obrzydlosci Moabczyków, i Melchomowi, obrzydlosci synów Amm onowych, splugawil król.
\par 14 I pokruszyl slupy, a powycinal gaje, i napelnil miejsca ich kosciami ludzkiemi.
\par 15 Nadto i oltarz, który byl w Betel, i wyzyne, która byl uczynil Jeroboam, syn Nabatowy, który przywiódl do grzechu lud Izraelski, i ten oltarz, i wyzyne zepsul, a spaliwszy one wyzyne, starl na proch, i spalil gaj.
\par 16 A obróciwszy sie Jozyjasz, obaczyl groby, które tam byly na górze, a poslawszy pobral kosci z onych grobów, i popalil je na tymze oltarzu; a tak go splugawil wedlug slowa Panskiego, które mówil maz Bozy, który byl te rzeczy przepowiedzial.
\par 17 I rzekl: Cóz to jest za napis, który widze? I odpowiedzieli mu mezowie miasta: Grób to meza Bozego, który przyszedlszy z Judy opowiedzial te rzeczy, któres uczynil nad oltarzem w Betel.
\par 18 A on rzekl: Zaniechajcie go, niechaj nikt nie rucha kosci jego. I wybawili kosci jego, i kosci proroka onego, który byl przyszedl z Samaryi.
\par 19 Wszystkie tez domy wyzyn, które byly w miastach Samaryjskich, które byÜi pobudowali królowie Judzcy, drazniac Pana, zniósl Jozyjasz, i uczynil im wedlug wszystkiego, jako byl uczynil w Betel.
\par 20 Pozabijal takze wszystkich kaplanów wyzyn, którzy tam byli na oltarzach, i palil kosci ludzkie na nich, potem sie wrócil do Jeruzalemu.
\par 21 Rozkazal tez król wszystkiemu ludowi, mówiac: Obchodzcie swieto przejscia Panu, Bogu waszemu, jako napisano w ksiegach przymierza tego.
\par 22 Bo nie obchodzono takiego swieta przejscia ode dni sedziów, którzy sadzili Izraela, i przez wszystkie dni królów Izraelskich, i królów Judzkich.
\par 23 Jako osmnastego roku króla Jozyjasza, obchodzono takie swieto przejscia Panu w Jeruzalemie.
\par 24 Ale i wieszczków, i czarowników, i obrazy, i brzydkie balwany, i wszystkie obrzydlosci, co ich bylo widac w ziemi Judzkiej i w Jeruzalemie, wykorzenil Jozyjasz, aby wypelnil slowa zakonu napisane w ksiegach, które znalazl Helkijasz kaplan w dom u Panskim.
\par 25 I nie byl podobny jemu król przed nim, któryby sie nawrócil do Pana z calego serca swego, i ze wszystkiej duszy swojej, i ze wszystkiej sily swojej, wedlug wszystkiego zakonu Mojzeszowego, ani po nim powstal jemu podobny.
\par 26 Wszakze nie odwrócil sie Pan od popedliwosci wielkiego gniewu swego, która byl wzruszony gniew jego przeciw Judzie dla wszelkiego rozdraznienia, którem go byl rozdraznil Manases.
\par 27 Przetoz rzekl Pan: I Jude odrzuce od oblicznosci mojej, jakom odrzucil Izraela, i wzgardze to miasto, którem byl obral, to jest Jeruzalem, i dom ten, o którymem mówil: Bedzie tam imie moje.
\par 28 A inne sprawy Jozyjaszowe, i wszystko, co czynil, opisane jest w kronikach o królach Judzkich.
\par 29 Za dni jego wyciagnal Farao Necho, król Egipski, przeciw królowi Assyryjskiemu ku rzece Eufrates; wyjechal tez król Jozyjasz przeciwko niemu, którego on zabil w Megiddo, gdy go ujrzal.
\par 30 I przywiezli go sludzy jego umarlego z Megiddo, a przyprowadzili go do Jeruzalemu, i pogrzebli go w grobie jego. Potem wziawszy lud onej ziemi Joachaza, syna Jozyjaszowego, pomazali go, i królem go postanowili miasto ojca jego.
\par 31 Dwadziescia lat i trzy mial Joachaz, gdy królowac poczal, a trzy miesiace królowal w Jeruzalemie. A imie matki jego bylo Chamutal, córka Jeremijaszowa z Lebny.
\par 32 I czynil zle przed oczyma Panskiemi wedlug wszystkiego, co czynili ojcowie jego.
\par 33 I zwiazal go Farao Necho w Rebli w ziemi Emat, gdy królowal w Jeruzalemie, a ulozyl dan na one ziemie sto talentów srebra, i talent zlota.
\par 34 A królem postanowil Farao Necho Elijakima, syna Jozyjaszowego, miasto Jozyjasza, ojca jego, i odmienil imie jego, a nazwal go Joakim; ale Joachaza wzial, który, gdy przyszedl do Egiptu, tamze umarl.
\par 35 A to srebro i zloto dawal Joakim Faraonowi; przetoz szacowal ziemie, aby mógl oddawac srebro wedlug rozkazania Faraonowego; od kazdego wedlug szacunku jego, bral srebro i zloto od ludu ziemi, aby je oddawal Faraonowi Nechowi.
\par 36 Dwadziescia i piec lat mial Joakim, gdy królowac poczal, a jedenascie lat królowal w Jeruzalemie. A imie matki jego bylo Zebuda, córka Fadajowa z Rumy.
\par 37 I czynil zle przed oczyma Panskiemi wedlug wszystkiego, jako czynili ojcowie jego.

\chapter{24}

\par 1 Za dni jego wyciagnal Nabuchodonozor, król Babilonski. I stal sie Joakim niewolnikiem jego przez trzy lata, a potem wybil sie z mocy jego.
\par 2 Przetoz poslal Pan przeciwko niemu wojska Chaldejskie, i wojsko Syryjskie, i wojska Moabskie, i wojska synów Ammonowych; i poslal je na Jude, aby go wytracili wedlug slowa Panskiego, które byl powiedzial przez slugi swe proroki.
\par 3 Zaiste stalo sie to podlug slowa Panskiego przeciwko Judzie, aby go odrzucil od oblicza swego dla grzechów Manasesowych, wedlug wszystkiego, co byl uczynil;
\par 4 I dla krwi niewinnej, która wylewal, i napelnil Jeruzalem krwia niewinna, czego mu nie chcial Pan odpuscic.
\par 5 A inne sprawy Joakimowe, i wszystko co czynil, zapisane w kronikach o królach Judzkich.
\par 6 A tak zasnal Joakim z ojcami swymi, a królowal Joachyn, syn jego, miasto niego.
\par 7 Ale nie ruszal sie wiecej król Egipski z ziemi swej. Bo byl wzial król Babilonski wszystko od rzeki Egipskiej az do rzeki Eufrates, co przynalezalo królowi Egipskiemu.
\par 8 Osmnascie lat mial Joachyn, gdy królowac poczal, a trzy miesiace królowal w Jeruzalemie. Imie matki jego bylo Nehusta, córka Elnatanowa z Jeruzalemu.
\par 9 I czynil zle przed oczyma Panskiemi, wedlug wszystkiego, jako czynil ojciec jego.
\par 10 Czasu onego przyciagneli sludzy Nabuchodonozora, króla Babilonskiego, przeciwko Jeruzalemowi, i przyszlo miasto w oblezenie.
\par 11 Przyciagnal tez Nabuchodonozor, król Babilonski, przeciwko miastu, gdy sludzy jego lezeli okolo niego.
\par 12 Tedy wyszedl Joachyn, król Judzki, do króla Babilonskiego, on i matka jego, i sludzy jego, i ksiazeta jego, i dworzanie jego, i wzial go król Babilonski roku ósmego królowania swego.
\par 13 I wyniósl stamtad wszystkie skarby domu Panskiego, i skarby domu królewskiego, i potlukl wszystkie naczynia zlote, które byl sprawil Salomon, król Izraelski, w kosciele Panskim, jako byl powiedzial Pan.
\par 14 I przeniósl wszystko Jeruzalem, i wszystkich ksiazat, i wszystek lud rycerski, wiezniów dziesiec tysiecy, i wszystkich ciesli, i kowali, a nie zostal tam nikt, oprócz ubogiego ludu onej ziemi.
\par 15 Przeniósl i Joachyna do Babilonu, i matke królewska, i zony królewskie, i dworzan jego, i rycerski lud onej ziemi zawiódl w niewole z Jeruzalemu do Babilonu.
\par 16 Wszystkich tez mezów walecznych siedm tysiecy, i ciesli, takze i kowali tysiac, i wszystkich godnych ku bojowi, tych zawiódl w niewole król Babilonski do Babilonu.
\par 17 A królem postanowil król Babilonski króla Matanijasza, stryja jego, miasto niego, i odmienil mu imie, a nazwal go Sedekijasz.
\par 18 Dwadziescia i jeden lat mial Sedekijasz, gdy królowac poczal, a jedenascie lat królowal w Jeruzalemie; a imie matki jego bylo Chamutal, córka Jermijaszowa z Lebny.
\par 19 I czynil zle przed oczyma panskiemi wedlug wszystkiego, jako czynil Joakim.
\par 20 Albowiem sie to stalo dla rozgniewania Panskiego przeciwko Jeruzalemowi i Judzie, az je odrzucil od twarzy swojej. Wtem zasie odstapil Sedekijasz od króla Babilonskiego.

\chapter{25}

\par 1 I stalo sie roku dziewiatego królowania jego, miesiaca dziesiatego, dnia dziesiatego tegoz miesiaca, ze przyciagnal Nabuchodonozor, król Babilonski, on i wszystko wojsko jego przeciw Jeruzalemowi, i polozyl sie obozem u niego, a porobil przeciwko niemu szance w okolo.
\par 2 A tak oblezone bylo miasto az do jedenastego roku króla Sedekijasza.
\par 3 Tedy dnia dziewiatego czwartego miesiaca byl wielki glód w miescie, a nie mial chleba lud ziemi.
\par 4 I przelamano mur miejski, a wszyscy ludzie rycerscy uciekli w nocy droga, kedy ida do bramy, która jest miedzy dwoma murami, które byly podle ogrodu królewskiego; a Chaldejczycy lezeli okolo miasta, a król uszedl droga do pustyni.
\par 5 I gonilo wojsko Chaldejskie króla, i pojmalo go na polach Jerycho; a wszystko wojsko jego rozpierzchnelo sie od niego.
\par 6 A tak pojmawszy króla przywiedli go do króla Babilonskiego do Rebli, kedy o nim uczynili sad.
\par 7 A synów Sedekijaszowych pozabijali przed oczyma jego; potem Sedekijasza oslepiwszy zwiazali go lancuchami miedzianemi, i zawiedli go do Babilonu.
\par 8 Potem miesiaca piatego, siódmego dnia tegoz miesiaca, (ten jest rok dziewietnasty królowania Nabuchodonozora, króla Babilonskiego) przyciagnal Nabuzardan, hetman zolnierski, sluga króla Babilonskiego, do Jeruzalemu;
\par 9 I spalil do Panski, i dom królewski, i wszystkie domy w Jeruzalemie, owa wszystko budowanie kosztowne popalil ogniem.
\par 10 Mury takze Jeruzalemskie w okolo rozwalilo wszystko wojsko Chaldejskie, które bylo z onym hetmanem zolnierskim.
\par 11 A ostatek ludu, który byl zostal w miescie, i zbiegi, którzy byli zbiegli do króla Babilonskiego, i inne pospólstwo, przeniósl Nabuzardan, hetman zolnierski.
\par 12 Tylko z ubogich onej ziemi zostawil hetman zolnierski, aby byli winiarzami i oraczami.
\par 13 Nadto slupy miedziane, które byly w domu Panskim, i podstawki, i morze miedziane, które bylo w domu Panskim, potlukli Chaldejczycy, i przeniesli wszystke miedz do Babilonu.
\par 14 Kotly tez i lopaty, i naczynia muzyczne, i misy i wszystko naczynie miedziane, którem uslugiwano; pobrali.
\par 15 I kadzielnice, i miednice, i co bylo zlotego w zlocie, i co bylo srebrnego w srebrze, pobral hetman zolnierski.
\par 16 Slupy dwa, morze jedno, i podstawki, które byl sprawil Salomon w domu Panskim, a nie bylo wagi miedzi onego wszystkiego naczynia.
\par 17 Osmnascie lokci wzwyz bylo slupa jednego, a galka na nim miedziana; a galka miala na wzwyz trzy lokcie, a siatka i jablka granatowe na galce w okolo, wszystko miedziane. Takiz tez byl i drugi slup z siatka,
\par 18 Wzial tez hetman zolnierski Sarajego, kaplana przedniego, i Sofonijasza, kaplana wtórego, i trzech odzwiernych.
\par 19 Wzial tez z miasta dworzanina jednego, który byl przelozony nad ludem, rycerskim, i piec mezów z tych, którzy stawali przed królem, którzy byli znalezieni w miescie, i pisarza przedniego wojskowego, który spisywal lud onej ziemi, i szescdziesiat mezów ludu z onej ziemi, którzy sie znalezli w miescie.
\par 20 Pojmawszy ich tedy Nabuzardan, hetman zolnierski, zawiódl ich do króla Babilonskiego do Ryblaty.
\par 21 I pobil ich król Babilonski, a pomordowal ich w Ryblacie w ziemi Emat; a tak przeniesiony jest Juda z ziemi swojej.
\par 22 Ale nad ludem, który jeszcze byl zostal w ziemi Judzkiej, którego byl zostawil Nabuchodonozor, król Babilonski, przelozyl Godolijasza, syna Ahykamowego, syna Safanowego.
\par 23 A gdy uslyszeli wszyscy hetmani wojska, sami i mezowie ich, ze przelozyl król Babilonski Godolijasza, tedy przyszli do Godolijasza do Masfy; mianowicie, Izmael, syn Natanijaszowy, i Johanan, syn Kareaszowy, i Serajasz, syn Tanhumeta Netofatczyk a, i Jezonijasz, syn Maachatowy sami i mezowie ich.
\par 24 Którym przysiagl Godolijasz, i mezom ich, i rzekl im: Nie bójcie sie byc poddanymi Chaldejczykom; zostancie w ziemi, a sluzcie królowi Babilonskiemu, i bedzie wam dobrze.
\par 25 I stalo sie miesiaca siódmego, ze przyszedl Izmael, syn Natanijasza, syna Elisamowego, z nasienia królewskiego, i dziesiec mezów z nim, i zabili Godolijasza, i umarl; takze Zydów i Chaldejczyków, którzy z nim byli w Masfa.
\par 26 Tedy powstal wszystek lud od malego az do wielkiego, i hetmani wojsk, a poszli do Egiptu; bo sie bali Chaldejczyków.
\par 27 Stalo sie takze trzydziestego i siódmego roku pojmania Joachyna, króla Judzkiego, dwunastego miesiaca dnia dwudziestego siódmego tegoz miesiaca, ze wywyzszyl Ewilmerodach, król Babilonski, tegoz roku, gdy poczal królowac, glowe Joachyna, króla Judzkiego, uwolniwszy go z wiezienia.
\par 28 I rozmawial z nim laskawie, a wystawil stolice jego nad stolice innych królów, którzy z nim byli w Babilonie.
\par 29 Odmienil tez odzienie jego, w którem byl w wiezieniu, i jadl chleb zawsze przed obliczem jego po wszystkie dni zywota swego.
\par 30 Obrok tez jemu naznaczony ustawicznie mu dawano od króla, na kazdy dzien po wszystkie dni zywota jego.


\end{document}