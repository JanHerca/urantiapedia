\begin{document}

\title{Salmos}


\chapter{1}

\par 1 Blogoslawiony maz, który nie chodzi w radzie niepoboznych, a na drodze grzesznych nie stoi, i na stolicy nasniewców nie siedzi;
\par 2 Ale w zakonie Panskim jest kochanie jego, a w zakonie jego rozmysla we dnie i w nocy.
\par 3 Albowiem bedzie jako drzewo nad strumieniem wód w sadzone, które owoc swój wydaje czasu swego, a lisc jego nie opada; i wszystko, cokolwiek czynic bedzie, poszczesci sie.
\par 4 Lecz nie tak niepobozni; ale sa jako plewa, która wiatr rozmiata.
\par 5 Przetoz sie niepobozni na sadzie nie ostoja, ani grzesznicy w zgromadzeniu sprawiedliwych.
\par 6 Albowiem zna Pan droge sprawiedliwych; ale droga niepoboznych zginie.

\chapter{2}

\par 1 Przeczze sie poganie buntuja, a narody przemyslaja prózne rzeczy?
\par 2 Schodza sie królowie ziemscy, a ksiazeta radza spolem przeciwko Panu, i przeciw pomazancowi jego, mówiac:
\par 3 Potargajmy zwiazki ich, a odrzucmy od siebie powrozy ich.
\par 4 Ale ten, który mieszka w niebie, smieje sie; Pan szydzi z nich.
\par 5 Tedy bedzie mówil do nich w popedliwosci swojej, a w gniewie swoim przestraszy ich,
\par 6 Mówiac: Jamci postanowil króla mojego nad Syonem, góra swieta moja.
\par 7 Opowiem ten dekret: Pan rzekl do mnie: Syn mój jestes ty, Jam ciebie dzis splodzil.
\par 8 Zadaj odemnie, a dam ci narody dziedzictwo twoje; a osiadlosc twoje, granice ziemi.
\par 9 Potrzesz ich laska zelazna, a jako naczynie zdunskie pokruszysz ich.
\par 10 Terazze tedy zrozumiejcie, królowie, nauczcie sie sedziowie ziemi!
\par 11 Sluzcie Panu w bojazni, a rozradujcie sie ze drzeniem.
\par 12 Pocalujcie syna, by sie snac nie rozgniewal, i zginelibyscie w drodze, gdyby sie najmniej zapalila popedliwosc jego. Blogoslawieni wszyscy, którzy w nim ufaja.

\chapter{3}

\par 1 Psalm Dawidowy, gdy uciekal przed Absalomem, synem swoim.
\par 2 Panie, jako sie namnozylo nieprzyjaciól moich! wiele ich powstaje przeciwko mnie.
\par 3 Wiele ich mówia o duszy mojej: Niemac ten ratunku od Boga. Sela.
\par 4 Ale ty, Panie! jestes tarcza moja, chwala moja, i wywyzszajacym glowe moje.
\par 5 Glosem swym wolalem do Pana, a wysluchal mie z góry swietej swojej. Sela.
\par 6 Jam sie ukladl, i zasnalem, a ocucilem sie; bo mie Pan podpieral.
\par 7 Nie ulekne sie wielu tysiecy ludu, którzy sie na mie zewszad zasadzili.
\par 8 Powstan, Panie! wybaw mie, Boze mój! albowiemes ty uderzyl w lice wszystkich nieprzyjaciól moich, z zeby niezbozników pokruszyles.
\par 9 Od Panac jest wybawienie, a nad ludem twoim blogoslawienstwo twoje. Sela.

\chapter{4}

\par 1 Przedniejszemu spiewakowi na Neginot psalm Dawidowy.
\par 2 Wysluchaj mie, gdy cie wzywam, Boze sprawiedliwosci mojej! którys mi sprawil przestrzenstwo w ucisnieniu; zmiluj sie nademna, a wysluchaj modlitwe moje.
\par 3 Synowie ludzcy, i dokadze chwale moje lzyc bedziecie, milujac próznosci, a szukajac klamstwa? Sela.
\par 4 Wiedzciez, zec Pan odlaczyl sobie poboznego; wyslucha Pan, gdy zawolam do niego.
\par 5 Lekajciez sie, a nie grzeszcie; rozmyslajcie w sercach swych, na lozach waszych, a umilkniecie. Sela.
\par 6 Ofiarujciez ofiary sprawiedliwosci, a ufajcie w Panu.
\par 7 Wielec ich mówia: Któz nam da ogladac dobra? Ale ty, Panie! podnies nad nami swiatlosc oblicza twego.
\par 8 I sposobisz wieksza radosc w sercu mojem, niz oni miewaja, gdy sie im zboza ich i wina ich obficie zrodza.
\par 9 W pokoju sie i poloze i zasne, bo ty sam, Panie! czynisz, ze bezpiecznie mieszkam.

\chapter{5}

\par 1 Przedniejszemu spiewakowi na Nechylot psalm Dawidowy.
\par 2 Przyjmij, Panie! w uszy swe slowa moje, i wyrozumij doleglosci moje.
\par 3 Sluchaj pilnie glosu wolania mego; królu mój, i Boze mój! boc sie modle tobie.
\par 4 Panie! rano uslysz glos mój; ranoc przedloze modlitwe moje, i bede wygladal pomocy.
\par 5 Albowiem ty, o Boze! nie kochasz sie w nieprawosci, a nie zmieszka z toba zlosnik.
\par 6 Nieostoja sie szaleni przed oczyma twemi: ty masz w nienawisci wszystkich, którzy broja nieprawosci.
\par 7 Wyglubisz tych, którzy mówia klamstwo; mezem krwawym i zdradliwym brzydzi sie Pan.
\par 8 Ale ja w obfitosci milosierdzia twego wnijde do domu twego, a poklonie sie w kosciele twoim swietym, w bojazni twojej.
\par 9 Panie! prowadz mie w sprawiedliwosci twojej dla nieprzyjaciól moich, a wyprostuj przed obliczem mojem droge twoje.
\par 10 Bo niemasz nic szczerego w ustach ich; wnetrznosci ich zlosliwe, gardlo ich jako grób otwarty, jezykiem swym pochlebiaja.
\par 11 Spustosz ich, o Boze! Niech upadna od rad swoich; dla wielkosci przestepstwa ich rozpedz ich, poniewaz sa odpornymi tobie.
\par 12 A niechaj sie rozwesela wszyscy, co ufaja w tobie; na wieki niech wykrzykuja, gdyz ich ty szczycic bedziesz, i rozraduja sie w tobie, którzy miluja imie twoje.
\par 13 Albowiem ty, Panie! sprawiedliwemu blogoslawic bedziesz, a zastawisz go, jako tarcza, dobrotliwoscia twoja.

\chapter{6}

\par 1 Przedniejszemu spiewakowi na Neginot i Seminit psalm Dawidowy.
\par 2 Panie! w popedliwosci twojej nie nacieraj na mie, a w gniewie twoim nie karz mie.
\par 3 Zmiluj sie nademna, Panie! bomci mdly; uzdrów mie, Panie! boc sie strwozyly kosci moje,
\par 4 I dusza moja bardzo jest zatrwozona; ale ty, Panie! pokadze?
\par 5 Nawróc sie, Panie! wyrwij dusze moje; wybaw mie dla milosierdzia twego;
\par 6 Albowiem w smierci niemasz pamiatki o tobie, a w grobie któz cie wyznawac bedzie?
\par 7 Spracowalem sie od wzdychania mego; oplywa na kazda noc posciel moja, a loze moje mokre jest od lez.
\par 8 Zacmilo sie dla gniewu oko moje, a zstarzala sie twarz moja dla wszystkich nieprzyjaciól moich.
\par 9 Odstapcie odemnie wszyscy, krórzy czynicie nieprawosc; albowiem Pan uslyszal glos placzu mojego.
\par 10 Uslyszal Pan prosbe moje; Pan modlitwe moje przyjal.
\par 11 Niech sie zawstydza i bardzo zatrwoza wszyscy nieprzyjaciele moi; niech tyl podadza, a niech predko pohanbieni beda.

\chapter{7}

\par 1 Syggajon Dawidowe, które spiewal Panu dla slów Chusy, syna Jemini.
\par 2 Panie, Boze mój! w tobie ufam; wybawze mie od wszystkich przesladowców moich, i wyzwól mie;
\par 3 By snac duszy mojej nie porwal jako lew, a nie rozszarpal, gdyby nie bylo, ktoby ja wybawil.
\par 4 Panie, Boze mój! jezlim to uczynil, a jezli jest nieprawosc w rekach moich;
\par 5 Jezlim zle oddal temu, który ze mna w pokoju mieszkal; jezlizem nie wyrwal tego, który mie dreczyl bez przyczyny:
\par 6 Niechajze przesladuje nieprzyjaciel dusze moje, a niechaj pochwyci, i podepcze na ziemi zywot mój, a slawe moje niech zagrzebie w proch. Sela.
\par 7 Powstanze, Panie! w popedliwosci twojej, podnies sie przeciwko wscieklosci nieprzyjaciól moich, ocuc sie, a obróc sie ku mnie; bos ty sad postanowil;
\par 8 Tedy sie do ciebie zbiezy zgromadzenie narodów; dla nich tedy usiadz na wysokosci.
\par 9 Pan bedzie sadzil narody. Osadzze mie, Panie! wedlug sprawiedliwosci mojej, i wedlug niewinnosci mojej, która jest przy mnie.
\par 10 Niechze, prosze, ustanie zlosc niepoboznych, a umocnij sprawiedliwego, który doswiadczasz serc i wnetrznosci, o Boze sprawiedliwy!
\par 11 Bóg jest tarcza moja, który wybawia ludzi serca szczerego.
\par 12 Bóg jest sedzia sprawiedliwym; Bóg obrusza sie co dzien na niezboznego.
\par 13 Jezli sie nie nawróci, naostrzy miecz swój; luk swój wyciagnal, i nagotowal go.
\par 14 Zgotowal nan bron smiertelna, a strzaly swoje na przesladowników przyprawil.
\par 15 Oto rodzi nieprawosc, bo poczal bolesc; ale porodzi klamstwo.
\par 16 Kopal dól, i wykopal go; ale wpadnie w dól, który sam uczynil.
\par 17 Obróci sie bolesc jego na glowe jego, a na wierzch glowy jego nieprawosc jego spadnie.
\par 18 Bede wyslawial Pana wedlug sprawiedliwosci jego, a bede spiewal imieniowi Pana najwyzszego.

\chapter{8}

\par 1 Przedniejszemu spiewakowi na Gittyt psalm Dawidowy.
\par 2 Panie, Panie nasz! jakoz zacne jest imie twoje po wszystkiej ziemi! którys wyniósl chwale twoje nad niebiosa.
\par 3 Z ust niemowlatek i ssacych ugruntowales moc twa dla nieprzyjaciól twoich, abys wyniszczyl nieprzyjaciela i tego, który sie msci.
\par 4 Gdy sie przypatruje niebiosom twoim, dzielu palców twoich, miesiacowi i gwiazdom, któres wystawil,
\par 5 Tedy mówie: Cóz jest czlowiek, iz nan pamietasz? albo Syn czlowieczy, iz go nawiedzasz?
\par 6 Albowiem malo mniejszym uczyniles go od Aniolów, chwala i czcia ukoronowales go.
\par 7 Dales mu opanowac sprawy rak twoich, wszystkos poddal pod nogi jego.
\par 8 Owce i woly wszystkie, nadto i zwierzeta polne.
\par 9 Ptastwo niebieskie, i ryby morskie, i cokolwiek chodzi po scieszkach morskich.
\par 10 Panie, Panie nasz! jako zacne jest imie twoje po wsystkiej ziemi!

\chapter{9}

\par 1 Przedniejszemu spiewakowi, na Halmutlabben piesn Dawidowa.
\par 2 Bede wyslawial Pana ze wszystkiego serca mego; opowiadac bede wszysteki cuda twoje.
\par 3 Rozwesele sie, i rozraduje sie w tobie; bede spiewal imieniowi twemu, o Najwyzszy!
\par 4 Ze sie obrócili nieprzyjaciele moi na wstecz: upadli i pogineli od oblicznosci twojej.
\par 5 Bos ty odprawil sad mój, i sprawe moje; zasiadles na stolicy, Sedzia sprawiedliwy.
\par 6 Rozgromiles pogan, zatraciles zlosnika, imie ich wygladziles na wieki wieczne.
\par 7 O nieprzyjacielu! azaz wykonane sa spustoszenia twoie na wiecznosc? Poburzylzes miasta? i owszem ich samych pamiatka zginela z niemi.
\par 8 Ale Pan na wieki trwa; zgotowal stolice swoje na sad.
\par 9 On bedzie sadzil okrag ziemi w sprawiedliwosci, i osadzi narody w prawosci.
\par 10 I bedzie Pan ucieczka ubogiemu, ucieczka czasu ucisku.
\par 11 I beda ufac w tobie, którzy znaja imie twoje; albowiem nie opuszczasz tych, Panie! którzy cie szukaja.
\par 12 Spiewajciez Panu, który mieszka na Syonie; opowiadajcie miedzy narodami sprawy jego.
\par 13 Boc on szuka krwi, i maja w pamieci, a nie zapomina wolania utrapionych.
\par 14 Zmiluj sie nademna, Panie! obacz utrapienie moje od tych, którzy mie maja w nienawisci, ty, co mie wyrywasz z bram smierci.
\par 15 Abym opowiadal wszystkie chwaly twoje w bramach córki Syonskiej, weselac sie w zbawieniu twojem.
\par 16 Zanurzeni sa poganie w dole, który uczynili; w sieci, która skrycie zastawili, uwiezla noga ich.
\par 17 Oznajmil sie Pan, gdy uczynil sad; w sprawie rak swoich sie zlosnik usidlil. Rzecz godna rozmyslania! Sela.
\par 18 Niepobozni sie obróca do piekla, wszystkie narody, które zapominaja Boga.
\par 19 Bo nie bedzie na wieki zapamietany ubogi; oczekiwanie nedznych nie zginie na wieki.
\par 20 Powstanze, Panie! niech sie nie zmacnia smiertelny czlowiek; a niech narody osadzone beda przed toba.
\par 21 Panie! pusc na nie strach, aby poznaly narody, iz sa ludzmi smiertelnymi.Sela.

\chapter{10}

\par 1 Panie! przeczze stoisz z daleka? przeczze sie ukrywasz czasu ucisku?
\par 2 Zlosnik z hardosci przesladuje ubogiego; niechajze beda uchwyceni w chytrych zamyslach, które zamyslaja.
\par 3 Bo sie chlubi niezboznik w pozadliwosciach duszy swojej, a lakomy blogoslawi sobie a drazni Pana.
\par 4 Niepobozny dla pychy, która po sobie pokazuje, nie pyta sie o Boga; wszystka mysl jego, ze niemasz Boga.
\par 5 Darza mu sie drogi jego na kazdy czas; dalekie sa sady twoje od niego; sapa przeciwko wszystkim nieprzyjaciolom swym.
\par 6 Mówi w sercu swem: Nie bede wzruszony od narodu do narodu; bo sie nie boje zlego.
\par 7 Usta jego pelne sa zlorzeczenstwa, i chytrosci, i zdrady; pod jezykiem jego uprzykrzenie i nieprawosc.
\par 8 Siedzi, czyhajac we wsiach, w skrytosciach zabija niewinnego; oczy jego upatruja ubogiego.
\par 9 Czyha w skrytem miejscu, jako lew w jamie swojej; dybie jakoby uchwycil ubogiego, ulapiwszy go ciagnie do sieci swojej.
\par 10 Przypada, przytula sie, i rzuca sie moca swoja na wiele ubogich.
\par 11 Mówi w sercu swem: Zapomnalci tego Bóg; zakryl oblicze swoje, nie ujrzy na wieki.
\par 12 Powstanze, Panie Boze! podnies reke twoje; nie zapominajze ubogich.
\par 13 Przeczze niezboznik drazni Boga, mówiac w sercu swem: Nie bedziesz sie o tem pytal?
\par 14 Ale ty widzisz ucisk, i krzywde upatrujesz, abys im odplacil reka twa; na ciebiec sie spuscil ubogi, tys jest pomocnikiem sierocie.
\par 15 Potrzyj ramie niepoboznego i zlosnika, dowiaduj sie o jego niezboznosci, az go nie stanie.
\par 16 Pan jest królem na wieki wieczne; ale narody zgina z ziemi jego.
\par 17 Zadosci pokornych wysluchiwasz, Panie! utwierdzasz serca ich, nachylasz ku nim ucha twojego.
\par 18 Abys sad uczynil sierocie i chudzinie, aby go wiecej nie trapil czlowiek smiertelny na ziemi.

\chapter{11}

\par 1 Przedniejszemu spiewakowi psalm Dawidowy. W Panu ja ufam. Jakoz tedy mówicie duszy mojej: Ulatuj jako ptak z góry swojej?
\par 2 Bo oto niepobozni naciagaja luk, przykladaja strzale swa na cieciwe, aby strzelali w ciemnosci na uprzejmych sercem.
\par 3 Ale zamysly ich beda skazone; bo sprawiedliwy cóz uczynil?
\par 4 Pan jest w kosciele swietem swoim, stolica Panska jest na niebie; oczy jego upatruja, powieki jego doswiadczaja synów ludzkich.
\par 5 Pan doswiadcza sprawiedliwego; ale niepoboznego i milujacego nieprawosc ma w nienawisci dusza jego.
\par 6 Wyleje jako deszcz na niepoboznych sidla, ogien i siarke, a wicher bedzie czastka kielicha ich.
\par 7 Bo sprawiedliwy Pan, sprawiedliwosc miluje, na szczerego patrza oczy jego.

\chapter{12}

\par 1 Przedniejszemu spiewakowi Seminit, piesn Dawidowa.
\par 2 Ratuj, Panie! boc juz niestaje milosiernego, a wygineli uprzejmi z synów ludzkich.
\par 3 Kazdy mówi klamstwo z bliznim swoim: usty pochlebnemi, dwojakiem sercem mówia.
\par 4 Niechajze Pan wytraci wszystkie wargi pochlebne, i jezyk mówiacy rzeczy wyniosle.
\par 5 Którzy mówia: Jezykiem naszym przewiedziemy, wargi nasze za nami sa, któz jest panem naszym?
\par 6 Dla zniszczenia ubogich, i dla wolania nedznych teraz powstane, mówi Pan; postawie w bezpiecznosci tego, na którego sidla stawiaja.
\par 7 Slowa Panskie sa slowa czyste, jako srebro wyplawione w piecu glinianym, siedm kroc przelewane.
\par 8 Ty, Panie! zachowaj ich; strzez ich od rodzaju tego az na wieki.
\par 9 Ze wszystkich stron niepobozni kraza, gdy wywyzszeni bywaja najpodlejsi miedzy synami ludzkimi.

\chapter{13}

\par 1 Przedniejszemu spiewakowi piesn Dawidowa.
\par 2 Dokadze Panie? Zapomniszze mie na wieki? dokadze ukrywac bedziesz oblicza twego przedemna?
\par 3 Dokadze sie bede radzil w duszy swojej, a trapil w sercu mojem przez caly dzien? Dokadze sie bedzie wywyzszal nieprzyjaciel mój nademna?
\par 4 Wejrzyjze, wysluchaj mie, Panie, Boze mój! oswiec oczy me, bym snac nie zasnal w smierci;
\par 5 By snac nie rzekl nieprzyjaciel mój: Przemoglem go; azeby sie nieprzyjaciele moi nie rodowali, gdybym sie zachwial.
\par 6 Ale ja w milosierdziu twojem ufam: rozraduje sie serce moje w zbawieniu twojem; bede spiewal Panu, ze mi dal wiele dobrego.

\chapter{14}

\par 1 Przedniejszemu spiewakowi psalm Dawidowy.
\par 2 Glupi rzekl w sercu swojem: Niemasz Boga. Popsowali sie, obrzydliwymi sie stali w zabawach swoich: niemasz, ktoby czynil dobrze.
\par 3 Pan z niebios spojrzal na synów ludzkich, aby obaczyl, bylliby kto rozumny i szukajacy Boga.
\par 4 Alec wszyscy odstapili, jednako sie nieuzytecznymi stali; niemasz, ktoby czynil dobrze, niemasz i jednego.
\par 5 Azaz nie wiedza wszyscy czyniciele nieprawosci, ze pozeraja lud mój, jako wiec chleb jedza? ale Pana nie wzywaja.
\par 6 Tam sie bardzo ulekna, gdyz Bóg jest przy narodzie sprawiedliwego.
\par 7 Hanbicie rade ubogiego; ale Pan jest nadzieja jego.
\par 8 Któz da z Syonu wybawienie Izraelowi? Gdyz zasie wyprowadzi Pan z wiezienia lud swój; rozraduje sie Jakób, a Izrael sie rozweseli.

\chapter{15}

\par 1 Piesn Dawidowa. Panie! któz bedzie przebywal w przybytku twoim? Któz bedzie mieszkal na swietej górze twojej?
\par 2 Ten, który chodzi w niewinnosci, i czyni sprawiedliwosc, a mówi prawde w sercu swojem;
\par 3 Który nie obmawia jezykiem swoim, nic zlego nie czyni blizniemu swemu, ani zelzywosci kladzie na blizniego swego.
\par 4 Przed którego oczyma wzgardzony jest niezboznik, ale tych, którzy sie boja Pana, ma w uczciwosci; który, choc przysieze z szkoda swoja, nie odmienia;
\par 5 Który pieniedzy swych nie daje na lichwe, i darów przeciwko niewinnym nie przyjmuje. Kto to czyni, nie zachwieje sie na wieki.

\chapter{16}

\par 1 Zlota piesn Dawidowa.
\par 2 Strzez mie, o Boze! bo w tobie ufam. Rzecz, duszo moja! Panu: Tys Pan mój, a dobroc moja nic ci nie pomoze,
\par 3 Ale swietym, którzy sa na ziemi, i zacnym, w których wszystko kochanie moje.
\par 4 Rozmnoza sie bolesci tych, którzy sie za cudzym bogiem kwapia; nie ukusze ze krwi mokrych ofiar ich, ani wezme imion ich w usta moje.
\par 5 Pan jest czastka dziedzictwa mego, i kielicha mego, ty zatrzymujesz los mój.
\par 6 Sznury mi przypadly na miejscach wesolych, a dziedzictwo wdzieczne przyszlo na mie.
\par 7 Bede blogoslawil Pana, który mi dal rade, gdyz i w nocy cwicza mie nerki moje.
\par 8 Stawialem Pana zawsze przed oczyma swemi; a iz on jest po prawicy mojej, nie bede wzruszony.
\par 9 Przetoz uweselilo sie serce moje, a rozradowala sie chwala moja; dotego cialo moje mieszkac bedzie bezpiecznie.
\par 10 Bo nie zostawisz duszy mojej w grobie, ani dopuscisz swietemu twemu ogladac skazenia.
\par 11 Oznajmisz mi droge zywota; obfitosc wesela jest przed obliczem twojem, rozkoszy po prawicy twojej az na wieki.

\chapter{17}

\par 1 Modlitwa Dawidowa. Wysluchaj, Panie! sprawiedliwosc moje; miej wzglad na wolanie moje; przyjmij w uszy modlitwe moje, która czynie usty nieobludnemi.
\par 2 Od oblicznosci twojej sad mój niech wynijdzie; oczy twoje niech patrza na uprzejmosc.
\par 3 Doswiadczyles serca mego, nawiedziles je w nocy; doswiadczyles mie ogniem, ales nic nie znalazl; mysli moje nie uprzedzaja ust moich.
\par 4 Co sie tknie spraw ludzkich wedlug slowa ust twoich, chronilem sie drogi okrutnika.
\par 5 Zatrzymuj kroki moje na drogach twych, aby sie nie chwialy nogi moje.
\par 6 Ja cie wzywam, bo mie wysluchiwasz, Boze! Naklon ucha twego ku mnie, wysluchaj slowa moje.
\par 7 Okaz milosierdzie twoje, ty, który ochraniasz ufajacych w tobie od tych, którzy powstawaja przeciwko prawicy twojej.
\par 8 Strzez mie jako zrenicy oka; pod cieniem skrzydel twoich ukryj mie.
\par 9 Przed twarza niepoboznych, którzy mie niszcza, przed nieprzyjaciólmi duszy mojej, którzy mie ogarneli.
\par 10 Tukiem swoim okryli sie; hardzie mówia usty swemi.
\par 11 Gdziekolwiek idziemy, obtoczyli nas; oczy swe nasadzili, aby nas potracili ku ziemi.
\par 12 Kazdy z nich podobien jest lwowi pragnacemu lupu, i lwieciu siedzacemu w jamie.
\par 13 Powstanze, Panie! uprzedz twarz jego, potrac go, wyrwij dusze moje od niezboznego mieczem twoim.
\par 14 Wyrwij mie od ludzi reka twoja, o Panie! od ludzi tego swiata, których dzial jest w tym zywocie, a których brzuch z szpizarni twojej napelniasz, skad nasyceni bywaja, i synowie ich, a zostawiaja ostatki swoje dzieciom swoim.
\par 15 Ale ja w sprawiedliwosci ogladam oblicze twoje; gdy sie ocuce, nasycony bede obrazem oblicznosci twojej.

\chapter{18}

\par 1 Przedniejszemu spiewakowi piesn Dawida, slugi Panskiego, który mówil do Pana slowa tej piesni onego dnia, gdy go Pan wyrwal z rak wszystkich nieprzyjaciól jego, i z reki Saulowej; i rzekl:
\par 2 Rozmiluje sie ciebie, Panie, mocy moja!
\par 3 Pan opoka moja, twierdza moja, i wybawicielem moim; Bóg mój skala moja, w nim bede ufal; tarcza moja, i róg zbawienia mego, ucieczka moja.
\par 4 Wzywalem Pana chwalebnego, a od nieprzyjaciól moich bylem wybawiony.
\par 5 Ogarnely mie byly bolesci smierci, a potoki niepoboznych zatrwozyly mie.
\par 6 Bolesci grobu ogarnely mie byly, zachwycily mie sidla smierci.
\par 7 W utrapieniu mojem wzywalem Pana, i wolalem do Boga mego; wysluchal z kosciola swego glos mój, a wolanie moje przed oblicznoscia jego przyszlo do uszów jego.
\par 8 Tedy sie ziemia wzruszyla i zadrzala, a fundamenty gór zatrzasnely sie, i wzruszyly sie od gniewu jego.
\par 9 Wystepowal dym z nozdrzy jego, wegle sie rozpalily od niego.
\par 10 Naklonil niebios, i zstapil, a ciemnosc byla pod nogami jego.
\par 11 A wsiadlszy na Cheruba, latal; latal na skrzydlach wiatrowych.
\par 12 Uczynil sobie z ciemnosci ukrycie, okolo siebie namiot swój z ciemnych wód, i z gestych obloków.
\par 13 Od blasku przed nim rozeszly sie obloki jego, grad i wegle ogniste.
\par 14 I zagrzmial na niebie Pan, a Najwyzszy wydal glos swój, grad i wegle ogniste.
\par 15 Wypuscil strzaly swe, i rozproszyl ich, a blyskawicami gestemi rozgromil ich.
\par 16 I okazaly sie glebokosci wód, a odkryte sa grunty swiata na fukanie twoje, Panie! i na tchnienie wiatru nozdrzy twoich.
\par 17 Poslawszy z wysokosci zachwycil mie; wyciagnal mie z wód wielkich.
\par 18 Wyrwal mie od mocnego nieprzyjaciela mego, i od tych, którzy mie mieli w nienawisci, choc byli mocniejszymi nad mie.
\par 19 Uprzedzili mie byli w dzien utrapienia mego; ale Pan byl podpora moja.
\par 20 Wywiódl mie na przestrzenstwo; wyrwal mie, iz mie umilowal.
\par 21 Nagrodzil mi Pan wedlug sprawiedliwosci mojej; wedlug czystosci rak moich oddal mi.
\par 22 Bom strzegl dróg Panskich, anim odstapil niezboznie od Boga mego.
\par 23 Bom mial wszystkie sady jego przed oczyma memi, a ustaw jego nie odrzucalem od siebie.
\par 24 Owszem, bylem szczerym przed nim, a strzeglem sie od nieprawosci mojej.
\par 25 Przetoz oddal mi Pan wedlug sprawiedliwosci mojej, wedlug czystosci rak moich, która byla przed oczyma jego.
\par 26 Ty, Panie! z milosiernym milosiernie sie obejdziesz, a z mezem szczerym szczerze sobie postapisz.
\par 27 Z uprzejmym uprzejmie sie obejdziesz, a z przewrotnym przewrotnie sobie postapisz;
\par 28 Albowiem ty lud utrapiony wybawisz, a oczy wyniosle ponizysz.
\par 29 Ty zaiste rozswiecisz pochodnie moje; Pan, Bóg mój, oswieci ciemnosci moje.
\par 30 Gdyz z toba przebilem sie przez wojsko, a z Bogiem moim przeskoczylem mur.
\par 31 Droga Boza doskonala jest; slowo Panskie jest ogniem wyplawione. Tarcza jest wszystkich, którzy w nim ufaja.
\par 32 Bo któz jest Bóg, oprócz Pana? a kto opoka, oprócz Boga naszego?
\par 33 On jest Bogiem, który mie opasuje moca, a czyni prosta droge moje.
\par 34 Krzepi nogi moje jako jelenie, a na wysokich miejscach moich stawia mie.
\par 35 Çwiczy rece moje do boju, tak, iz krusze luk miedziany ramionami swemi.
\par 36 Dales mi tez tarcz zbawienia twego, a prawica twoja podpierala mie, i dobrotliwosc twoja uwielmozyla mie.
\par 37 Rozszerzyles kroki moje podemna, tak, ze sie nie zachwialy golenie moje.
\par 38 Gonilem nieprzyjaciól moich, a doscignalem ich: i nie wrócilem sie, azem ich wytracil.
\par 39 Porazilem ich tak, iz nie mogli powstac; upadli pod nogi moje.
\par 40 Tys mie opasal moca ku bitwie; powstawajacych przeciwko mnie obaliles pod mie.
\par 41 Podales mi tyl nieprzyjaciól moich, abym tych, którzy mie nienawidza, wykorzenil.
\par 42 Wolalic, ale nie byl, ktoby ich wybawil; do Pana, ale ich nie wysluchal.
\par 43 I potarlem ich, jako proch od wiatru; jako bloto na ulicach podeptalem ich.
\par 44 Wyrwales mie od zwad ludzkich, a postawiles mie glowa narodom; lud, któregom nie znal, sluzyl mi.
\par 45 Skoro uslyszeli, byli mi posluszni; cudzoziemcy obludnie mi sie poddawali.
\par 46 Cudzoziemcy opadli, a drzeli w zamknieniach swoich.
\par 47 Zyje Pan, blogoslawiona opoka moja; przetoz niech bedzie wywyzszony Bóg zbawienia mego.
\par 48 Bóg jest, który mi zleca pomsty, i podbija mi narody.
\par 49 Tys wybawiciel mój od nieprzyjaciól moich; tys mie nad powstawajacych przeciwko mnie wywyzszyl; od meza drapieznego wyrwales mie.
\par 50 Przetoz cie, Panie! bede wyznawal miedzy narodami, a bede spiewal imieniowi twemu.
\par 51 Bos zacnie wybawil króla swego, a czynisz milosierdzie pomazancowi swemu Dawidowi, i nasieniu jego, az na wieki.

\chapter{19}

\par 1 Przedniejszemu spiewakowi psalm Dawidowy.
\par 2 Niebiosa opowiadaja chwale Boza, a dzielo rak jego rozpostarcie oznajmuje.
\par 3 Dzien dniowi podaje slowo, a noc nocy pokazuje umiejetnosc,
\par 4 Niemasz jezyka ani mowy, gdzieby glosu ich slychac nie bylo.
\par 5 Na wszystke ziemie wyszedl porzadek ich, a na konczyny okregu ziemi slowa ich; sloncu na nich namiot wystawil.
\par 6 A to jako oblubieniec wychodzi z loznicy swojej; raduje sie jako olbrzym, który ma biezec w droge.
\par 7 Wychodzi od konczyn niebios, a obchodzi je az do konczyn ich, a niemasz nic, coby sie moglo ukryc przed goracem jego.
\par 8 Zakon Panski jest doskonaly, nawracajacy dusze; swiadectwo Panskie wierne, dawajace madrosc nieumiejetnemu.
\par 9 Przykazania Panskie sa prawe, uweselajace serce; przykazanie Panskie czyste, oswiecajace oczy.
\par 10 Bojazn Panska czysta, trwajaca na wieki; sady Panskie sa prawdziwe, a przytem i sprawiedliwe;
\par 11 Pozadliwsze nad zloto, i nad wiele najwyborniejszego zlota, i slodsze nad miód i nad plastr miodowy.
\par 12 Sluga tez twój bywa oswiecony przez nie; a kto ich przestrzega, odnosi zaplate wielka.
\par 13 Ale wystepki któz zrozumie? od tajemnych wystepków oczysc mie.
\par 14 I od swawolnych zachowaj sluge twego, aby nie panowali nademna; tedy doskonalym bede, a bede oczyszczony od przestepstwa wielkiego.
\par 15 Niechze beda przyjemne slowa ust moich, i rozmyslanie serca mego przed obliczem twojem, Panie, skalo moja, i odkupicielu mój!

\chapter{20}

\par 1 Przedniejszemu spiewakowi psalm Dawidowy.
\par 2 Niech cie Pan wyslucha w dzien utrapienia; niech cie wywyzszy imie Boga Jakóbowego.
\par 3 Niech ci zesle ratunek z swiatnicy, a z Syonu niech cie podeprze.
\par 4 Niech wspomni na wszystkie ofiary twoje, a calopalenia twoje niech w popiól obróci. Sela.
\par 5 Niech ci da wszystko wedlug serca twego, a wszelka rade twoje niech wypelni.
\par 6 Rozweselimy sie w wybawieniu twojem, a w imieniu Boga naszego choragiew podniesiemy; niech wypelni Pan wszystkie prosby twoje.
\par 7 Terazesmy poznali, iz Pan wybawil pomazanca swego, a iz go wysluchal z nieba swego swietego przez zbawienna moc prawicy swojej.
\par 8 Jedni w wozach, a drudzy w koniach ufaja; ale my na imie Pana, Boga naszego, wspominamy.
\par 9 Onic polegli i upadli, a mysmy powstali, i ostoimy sie.
\par 10 Panie! ty nas zachowaj, a król nas niech wyslucha w dzien wolania naszego.

\chapter{21}

\par 1 Przedniejszemu spiewakowi piesn Dawidowa.
\par 2 Panie! w mocy twojej raduje sie król, a w zbawieniu twojem wielce sie weseli.
\par 3 Dales mu zadosc serca jego, a prosby ust jego nie odmówiles mu. Sela.
\par 4 Albowiemes go uprzedzil blogoslawienstwy hojnemi; wlozyles na glowe jego korone ze zlota szczerego.
\par 5 Prosil cie o zywot, a dales mu przedluzenie dni na wieki wieków.
\par 6 Wielka jest chwala jego w zbawieniu twojem; chwala i zacnoscia przyodziales go;
\par 7 Bos go wystawil na rozmaite blogoslawienstwo az na wieki; rozweseliles go weselem oblicza twego.
\par 8 Gdyz król nadzieje ma w Panu, a z milosierdzia Najwyzszego nie bedzie poruszony.
\par 9 Znajdzie reka twoja wszystkich nieprzyjaciól twoich, prawica twoja dosieze wszystkich, co cie w nienawisci maja.
\par 10 Uczynisz ich jako piec ognisty czasu gniewu twego; Pan w popedliwosci swojej wytraci ich, a ogien ich pozre.
\par 11 Plemie ich z ziemi wygubisz, a nasienie ich z synów ludzkich.
\par 12 Albowiem czyhali na twoje zle; zmyslali rade, której dowiesc nie mogli.
\par 13 Przetoz wystawisz ich za cel; cieciwe twa wyciagniesz przeciwko twarzy ich.
\par 14 Podniesze sie, Panie! w mocy twojej, tedy bedziemy spiewac i wyslawiac moznosc twoje.

\chapter{22}

\par 1 Przedniejszemu spiewakowi na czas poranny psalm Dawidowy.
\par 2 Boze mój! Boze mój! czemus mie opuscil? oddaliles sie od wybawienia mego, od slów ryku mego.
\par 3 Boze mój! wolam we dnie, a nie ozywasz mi sie; i w nocy, a nie moge sie uspokoic.
\par 4 Ales ty Swiety, mieszkajacy w chwalach Izraelskich.
\par 5 W tobie nadzieje mieli ojcowie nasi; nadzieje mieli, a wybawiles ich.
\par 6 Do ciebie wolali, a wybawieni sa; w tobie nadzieje mieli, a nie byli pohanbieni.
\par 7 Alem ja robak, a nie czlowiek: posmiewisko ludzkie, i wzgarda pospólstwa.
\par 8 Wszyscy, którzy mie widza, szydza ze mnie; wykrzywiaja gebe, chwieja glowa, mówiac:
\par 9 Spuscil sie na Pana, niechze go wyrwie; niech go wybawi, poniewaz sie w nim kocha.
\par 10 Ales ty jest, którys mie wywiódl z zywota, czyniac mi dobra nadzieje jeszcze u piersi matki mojej.
\par 11 Na tobie spolegam od narodzenia swego; z zywota matki mojej tys Bogiem moim.
\par 12 Nie oddalajze sie odemnie; albowiem utrapienie bliskie jest, a niemasz, ktoby ratowal.
\par 13 Obtoczylo mie mnóstwo cielców; byki z Basan oblegly mie.
\par 14 Otworzyly na mie gebe swa jako lew szarpajacy i ryczacy.
\par 15 Rozplynalem sie jako woda, a rozstapily sie wszystkie kosci moje; stalo sie serce moje jako wosk, zstopnialo w posród wnetrznosci moich.
\par 16 Wyschla jako skorupa moc moja, a jezyk mój przysechl do podniebienia mego; nawet w prochu smierci polozyles mie.
\par 17 Albowiem psy mie obskoczyly, gromada zlosników oblegla mie; przebodli rece moje i nogi moje.
\par 18 Zliczylbym wszystkie kosci moje; lecz oni na mie patrzac, przypatruja mi sie.
\par 19 Rozdzielili odzienie moje miedzy sie, a o szaty moje los miotali.
\par 20 Ale ty, Panie! nie oddalaj sie: mocy moja! na ratunek mój pospiesz.
\par 21 Wyrwij od miecza dusze moje, z mocy psiej jedynaczke moje.
\par 22 Wybaw mie z paszczeki lwiej, a od rogów jednorozcowych wyzwól mie.
\par 23 Tedy opowiem imie twoje braciom mym; w posród zgromadzenia chwalic cie bede.
\par 24 Mówiac: Którzy sie boicie Pana, chwalcie go; wszystko potomstwo Jakóbowe wyslawiajcie go, a niech sie go boi wszystko nasienie Izraelskie.
\par 25 Albowiem nie wzgardzil, ani sie odwrócil od utrapienia ubogiego, ani skryl od niego oblicza swego; owszem, gdy do niego wolal, wysluchal go.
\par 26 O tobie chwala moja w zgromadzeniu wielkiem; sluby moje oddam przed tymi, którzy sie ciebie boja.
\par 27 Beda jesc ubodzy, i nasyca sie; chwalic beda Pana, którzy go szukaja; serce wasze zyc bedzie na wieki.
\par 28 Wspomna i nawróca sie do Pana wszystkie granice ziemi, i klaniac sie beda przed obliczem twojem wszystkie pokolenia narodów.
\par 29 Albowiem Panskie jest królestwo, a on panuje nad narodami.
\par 30 Wszyscy bogaci ziemi beda jesc, i upadac przed nim, przed oblicznoscia jego klaniac sie beda wszyscy zstepujacy w proch, i którzy duszy swej zywo zachowac nie moga.
\par 31 Nasienie ich sluzyc mu bedzie, a bedzie przywlaszczane Panu w kazdym wieku.
\par 32 Zbieza sie, a beda opowiadali sprawiedliwosc jego narodowi, który z nich wynijdzie, iz ja on wykonal.

\chapter{23}

\par 1 Psalm Dawidowy. Pan jest pasterzem moim, na niczem mi nie zejdzie.
\par 2 Na paszach zielonych postawil mie; a do wód cichych prowadzi mie.
\par 3 Dusze moje posila: prowadzi mie scieszkami sprawiedliwosci dla imienia swego.
\par 4 Chocbym tez chodzil w dolinie cienia smierci, nie bede sie bal zlego, albowiemes ty ze mna; laska twoja, i kij twój, te mie ciesza.
\par 5 Przed obliczem mojem gotujesz stól przeciwko nieprzyjaciolom moim; pomazales olejkiem glowe moje, kubek mój jest oplywajacy.
\par 6 Nadto dobrodziejstwo i milosierdzie twe pójda za mna po wszystkie dni zywota mego, a bede mieszkal w domu Panskim na dlugie czasy.

\chapter{24}

\par 1 Psalm Dawidowy. Panska jest ziemia, i napelnienie jej, okrag ziemi, i którzy mieszkaja na nim.
\par 2 Bo on na morzu ugruntowal ja, a na rzekach utwierdzil ja.
\par 3 Któz wstapi na góre Panska? a kto stanie na miejscu swietem jego?
\par 4 Czlowiek niewinnych rak i czystego serca, który nie sklania ku marnosci duszy swej, a nie przysiega zdradliwie.
\par 5 Ten wezmie blogoslawienstwo od Pana, i sprawiedliwosc od Boga zbawiciela swego.
\par 6 Tenci jest naród szukajacych go, szukajacych oblicza twego, Boze Jakóbowy! Sela.
\par 7 Podniesciez, o bramy! wierzchy wasze; podniescie sie, wy bramy wieczne! aby wszedl król chwaly!
\par 8 Któryz to jest król chwaly? Pan mocny i mozny, Pan mocny w boju.
\par 9 Podniesciez, o bramy! wierzchy wasze; podniescie sie, wy bramy wieczne! aby wszedl król chwaly.
\par 10 Który to jest król chwaly? Pan zastepów, tenci jest król chwaly. Sela.

\chapter{25}

\par 1 Psalm Dawidowy. Do ciebie, Panie! dusze moje podnosze.
\par 2 Boze mój! w tobie ufam; niech nie bede zawstydzony, niech sie nie wesela nieprzyjaciele moi ze mnie.
\par 3 A tak wszyscy, którzy oczekuja ciebie, nie beda zawstydzeni; zawstydzeni beda bez przyczyny nieprawosc czyniacy.
\par 4 Panie! daj mi poznac drogi twe, sciezek twoich naucz mie.
\par 5 Daj, abym chodzil w prawdzie twojej, i naucz mie; bos ty jest Bóg zbawienia mego; ciebie oczekuje dnia kazdego.
\par 6 Wspomnij na litosci twoje, Panie! i na milosierdzia twoje, które sa od wieku.
\par 7 Grzechów mlodosci mojej, i przestepstw moich nie racz pamietac; wedlug milosierdzia twego wspomnij na mie, dla dobroci twojej, Panie!
\par 8 Dobry i prawy jest Pan; przetoz drogi naucza grzeszników.
\par 9 Poprowadzi cichych w sadzie, a nauczy pokornych drogi swojej.
\par 10 Wszystkie sciezki Panskie sa milosierdzie i prawda tym, którzy strzega przymierza jego, i swiadectwa jego.
\par 11 Panie! dla imienia twego odpusc nieprawosc moje, bo wielka jest.
\par 12 Jestze czlowiek, co sie boi Pana? Nauczy go drogi, któraby mial obrac.
\par 13 Dusza jego w dobrem przemieszkiwac bedzie, a nasienie jego odziedziczy ziemie.
\par 14 Tajemnica Panska objawiona jest tym, którzy sie go boja, a przymierze swoje oznajmuje im.
\par 15 Oczy moje ustawicznie patrza na Pana; albowiem on wywodzi z sieci nogi moje.
\par 16 Wejrzyjze na mie, a zmiluj sie nademna; bom jest nedzny i opuszczony.
\par 17 Utrapienia serca mego rozmnozyly sie; z ucisków moich wywiedz mie.
\par 18 Obacz udreczenie moje, i prace moje, a odpusc wszystkie grzechy moje.
\par 19 Obacz nieprzyjaciól moich, jako sie rozmnozyli, a maja mie nieslusznie w nienawisci.
\par 20 Strzez duszy mojej, a wyrwij mie, abym nie byl pohanbiony; bo w tobie nadzieje mam.
\par 21 Niewinnosc i szczerosc niech mie strzega; bom na cie oczekiwal.
\par 22 O Boze! wybawze Izraela ze wszystkich ucisków jego.

\chapter{26}

\par 1 Psalm Dawidowy. Sadz mie, Panie! Boc ja w niewinnosci mojej chodze, a w Panu ufajac, nie zachwieje sie.
\par 2 Spróbuj mie, Panie! i doswiadcz mie: wyplaw ogniem nerki moje i serce moje.
\par 3 Albowiem milosierdzie twoje jest przed oczyma mojemi, a bede chodzil w prawdzie twojej.
\par 4 Nie zasiadalem z ludzmi klamliwymi a z obludnikami nie kumalem sie.
\par 5 Nienawidzialem zgromadzenia zlosników, a z niepoboznymi nie zasiadalem.
\par 6 Umylem w niewinnosci rece moje, a obchodze w okolo oltarz twój, Panie!
\par 7 Abym ci oddawal chwale glosna, a opowiadal wszystkie cuda twoje.
\par 8 Panie! umilowalem mieszkanie domu twego, i miejsce przybytku chwaly twojej.
\par 9 Nie zagarniajze z grzesznikami duszy mojej, ani z mezami krwawymi zywota mojego.
\par 10 W których rekach jest przewrotnosc, a prawica ich pelna podarków.
\par 11 Ale ja w niewinnosci mojej chodze: odkupze mie, a zmiluj sie nademna.
\par 12 Noga moja stanela na równinie; w zgromadzeniach bede blogoslawil Pana.

\chapter{27}

\par 1 Psalm Dawidowy. Pan jest swiatloscia moja, i zbawieniem mojem, kogóz sie bac bede? Pan jest moca zywota mego, kogóz sie mam lekac?
\par 2 Gdy sie zbiora przeciwko mnie zlosnicy, aby pozarli cialo moje; przeciwnicy moi, i nieprzyjaciele moi sami sie potkneli i upadli.
\par 3 Przetoz chocby wojsko przeciwko mnie stanelo, nie uleknie sie serce moje; chocby powstala przeciwko mnie wojna, przeciez ja w tym ufam.
\par 4 O jednem rzecz prosil Pana, i tej szukac bede; abym mieszkal w domu Panskim po wszystkie dni zywota mego, a zebym ogladal wdziecznosc Panska, i dowiadywal sie w kosciele jego.
\par 5 Bo mie skryje w dzien zly w przybytku swoim; zachowa mie w skrytosci namiotu swego, a na skale wywyzszy mie.
\par 6 A tak wywyzszona bedzie glowa moja nad nieprzyjaciolmi moimi, którzy sa okolo mnie; i bede ofiarowal w przybytku jego ofiary wykrzykania; bede spiewal i chwaly oddawal Panu.
\par 7 Wysluchaj, Panie! glos mój, kiedy wolam, a zmiluj sie nademna, i wysluchaj mie.
\par 8 O tobie przemysla serce moje, którys rzekl: Szukajcie twarzy mojej; przetoz twarzy twojej, Panie! szukac bede.
\par 9 Nie ukrywajze twarzy twojej przedemna, ani odrzucaj w gniewie slugi twego; tys bywal ratunkiem moim, nie opuszczajze mie, ani mie odstepuj, Boze zbawienia mego.
\par 10 Choc ojciec mój, i matka moja opuscili mie, wszakze Pan przyjal mie.
\par 11 Naucz mie, Panie! drogi twojej, a prowadz mnie sciezka dla tych, którzy mie podstrzegaja.
\par 12 Niepodawajze mie na wole nieprzyjaciól moich; albowiemci powstali przeciwko mnie swiadkowie falszywi, i ten, który tchnie okrucienstwem.
\par 13 Bym byl nie wierzyl, ze mam ogladac dobroc Panska w ziemi zyjacych, zleby o mnie bylo.
\par 14 Oczekujze Pana, zmacniaj sie, a on utwierdzi serce twoje; przetoz oczekuj Pana.

\chapter{28}

\par 1 Psalm Dawidowy. Do ciebie, Panie! wolam, skalo moja! nie milcz na wolanie moje, bym snac, jezli mi sie nie ozwiesz, nie stal sie podobnym zstepujacym do grobu.
\par 2 Wysluchajze glos prósb moich, gdy wolam do ciebie, gdy podnosze rece moje do swiatnicy swietej twojej.
\par 3 Nie zagarniaj mie z niezboznymi, i z czyniacymi nieprawosc, którzy mówia o pokoju z bliznimi swymi a mysla zle w sercach swoich.
\par 4 Oddajze im wedlug spraw ich i wedlug zlych uczynków ich; wedlug pracy rak ich oddaj im, oddaj im zaplate ich.
\par 5 Albowiem nie zrozumiewaja spraw Panskich, ani uczynków rak jego; przetoz ich popsuje, a nie pobuduje ich.
\par 6 Blogoslawiony Pan; albowiem wysluchal glos prósb moich.
\par 7 Pan jest moca moja i tarcza moja, w nim, nadzieje ma serce moje, a jestem poratowany; przetoz sie rozweselilo serce moje, a piesnia moja chwalic go bede.
\par 8 Pan jest moca swych, i moca zbawienia pomazanca swego on jest.
\par 9 Zbaw lud twój, Panie! a blogoslaw dziedzictwu twemu, i pas ich, i wywyzszaj az na wieki.

\chapter{29}

\par 1 Psalm Dawidowy. Oddawajcie Panu synowie mocarzów, oddawajcie Panu chwale i moc.
\par 2 Oddawajcie Panu chwale imienia jego; klaniajcie sie Panu w ozdobie swietobliwosci.
\par 3 Glos Panski nad wodami; Bóg chwalebny wzbudza gromy, Pan nad wodami wielkiemi.
\par 4 Glos Panski mocny, glos Panski wielmozny,
\par 5 Glos Panski cedry lamie; kruszy Pan cedry Libanskie,
\par 6 I czyni, ze skacza jako cieleta; Liban i Syryjon jako mlody jednorozec.
\par 7 Glos Panski krzesze plomien ognisty.
\par 8 Na glos Panski z bólem pustynia rodza; z bólem rodzi na glos Panski pustynai Kades.
\par 9 Na glos Panski z bólem rodza lanie, i odkrywaja sie lasy; ale w kosciele swym opowiada wszystke chwale swoje.
\par 10 Pan nad potopem siedzial, i bedzie siedzial Pan, bedac królem na wieki.
\par 11 Pan doda mocy ludowi swojemu; Pan bedzie blogoslawil ludowi swemu w pokoju.

\chapter{30}

\par 1 Psalm piesni przy poswieceniu domu Dwidowego.
\par 2 Panie! wywyzszac cie bede; albowiem wywyzszyles mie, a nie dales pociechy nieprzyjaciolom moim ze mnie.
\par 3 Panie, Boze mój! wolalem do ciebie, a uzdrowiles mie.
\par 4 Panie! wywiodles z piekla dusze moje; zachowales mie przy zywocie, abym nie zstapil do grobu.
\par 5 Spiewajciez Panu swieci jego, a wysluchajcie pamiatke swietobliwosci jego.
\par 6 Albowiem predko przemija gniew jego, ale po wszystek zywot trwa dobra wola jego; z wieczora bywa placz, ale z poranku wesele.
\par 7 Rzeklem w szczesciu swojem: Nie bede poruszony na wieki.
\par 8 Albowiem ty, Panie! wedlug woli twojej umocniles byl góre moje moca; ale skoros ukryl oblicze swoje, strwozylem sie;
\par 9 I wolalem do ciebie, Panie! a Panum sie modlil, mówiac:
\par 10 Co za pozytek ze krwi mojej, gdybym zstapil do dolu? Izali cie proch chwalic bedzie? Iazali opowie prawde twoje?
\par 11 Wysluchajze, Panie! a zmiluj sie nademna; Panie! badz pomocnikiem moim.
\par 12 Tedys odmienil placz mój w plasanie; zdjales ze mnie wór mój, a przepasales mie radoscia.
\par 13 Przetoz tobie spiewac bedzie chwala moja, a milczec nie bedzie. Panie, Boze mój! na wieki wyslawiac cie bede.

\chapter{31}

\par 1 Przedniejszemu spiewakowi psalm Dawidowy.
\par 2 W tobie, Panie! nadzieje mam, niech nie bede zawstydzony na wieki; w sprawiedliwosci twojej wybaw mie.
\par 3 Naklon ku mnie ucha twego, co rychlej wybaw mie; badzze mi mocna skala, domem obronnym, abys mie zachowal.
\par 4 Bos ty jest skala moja, i obrona moja; przetoz dla imienia twego prowadz mie, i zaprowadz mie.
\par 5 Wywiedz mie z sieci, która zastawili na mie; bos ty jest moca moja.
\par 6 W rece twoje poruczam ducha mego; odkupiles mie, Panie, Boze prawdziwy!
\par 7 Mam w nienawisci tych, którzy przestrzegaja próznych marnosci; bo ja w Panu nadzieje pokladam.
\par 8 Bede sie radowal i weselil w milosierdziu twojem, zes wejrzal na utrapienie moje, a poznales ucisnienie duszy mojej.
\par 9 Anis mie zawarl w rece nieprzyjaciela; ales postawil na przestrzenstwie nogi moje.
\par 10 Zmiluj sie nademna, Panie! bom jest ucisniony; wywiedla od zalosci twarz moja; takze i dusza moja i zywot mój.
\par 11 Albowiem zwatlalo od bolesci zdrowie moje, a lata moje od wzdychania; zemdlala dla utrapianie mego sila moja, a kosci moje wyschly.
\par 12 U wszystkich nieprzyjaciól moich jestem w pohanbieniu wielkiem, a najwiecej u sasiadów moich; stalem sie na postrach znajomym moim; którzy mie widza na dworze, uciekaja przedemna.
\par 13 Wypadlem z pamieci jako umarly; stalem sie jako naczynie stluczone.
\par 14 Albowiem naslucham sie uszczypków od wielu; strachu dosc zewszad, gdy sie naradzaja wespól przeciwko mnie, chytrze przemysliwajac, aby odjeli dusze moje.
\par 15 Ale ja w tobie mam nadzieje, Panie! Rzeklem: Tys jest Bogiem moim.
\par 16 W rekach twoich sa czasy moje; wyrwijze mie z reki nieprzyjaciól moich, i od tych, którzy mie przesladuja.
\par 17 Oswiec oblicze twoje nad sluga twoim; wybaw mie przez milosierdzie twoje.
\par 18 Panie! niech nie bede pohanbiony, poniewaz cie wzywam; niech sie zawstydza niezbozni, i zamilkna w grobie.
\par 19 Niech zaniemieja wargi klamliwe, które mówia przeciwko sprawiedliwemu rzeczy przykre z hardoscia i ze wzgarda.
\par 20 O jakoz jest wielka dobroc twoja, któras zachowal bojacym sie ciebie, któras pokazywal tym, którzy ufaja w tobie przed synami ludzkimi.
\par 21 Ukrywasz ich w skrytosci oblicza twego, przed hardoscia czlowiecza ukrywasz ich, jako w namiocie, przed swarliwemi jezykami.
\par 22 Blogoslawiony Pan! bo dziwnie okazal milosierdzie swoje przeciwko mnie, jakoby w miescie obronnem.
\par 23 Jam rzekl w uciekaniu mojem: Odrzuconym jest od oczów twych; ales ty wysluchal glos modlitw moich, gdym wolal do ciebie.
\par 24 Milujciez Pana wszyscy swieci jego; boc Pan wiernych strzeze, oddaje sowicie hardzie postepujacemu.
\par 25 Zmacniajcie sie (a posili Bóg serca wasze) wszyscy, którzy nadzieje macie w Panu.

\chapter{32}

\par 1 Piesn Dawidowa nauczajaca. Blogoslawiony czlowiek, któremu odpuszczono nieprawosc, a którego zakryty jest grzech.
\par 2 Blogoslawiony czlowiek, któremu nie poczyta Pan nieprawosci, a w którego duchu nie masz zdrady.
\par 3 Gdym milczal, schnely kosci moje w narzekaniu mojem na kazdy dzien.
\par 4 Poniewaz we dnie i w nocy ociezala nademna reka twoja, obrócila sie wilgotnosc moja w susze letnia. Sela.
\par 5 przetoz grzech mój oznajmilem tobie, a nieprawosci mojej nie krylem. Rzeklem: Wyznam na sie przestepstwa moje Panu, a tys odpuscil nieprawosc grzechu mego. Sela.
\par 6 Oto sie tobie bedzie modlil kazdy swiety, czasu, którego mozesz byc znaleziony, a choc wzbiora powodzi wód wielkich, przeciez go nie dosiegna.
\par 7 Tys jest ucieczka moja; od ucisnienia zachowasz mie, i piosnkami radosnego wybawienia uraczysz mie. Sela.
\par 8 Dam ci rozum, i naucze cie drogi, po której masz chodzic; dam ci rade, obróciwszy na cie oko moje.
\par 9 Nie badzciez jako kon, albo jako mul, którzy rozumu nie maja, których geby uzda i wedzidlem kielznac musisz, aby sie na cie nie porywaly.
\par 10 Wiele bolesci przypada na zlosnika; ale ufajacego w Panu milosierdzie ogarnie.
\par 11 Weselcie sie w Panu, i radujcie sie sprawiedliwi, a wykrzykajcie wszyscy, którzyscie serca szczerego.

\chapter{33}

\par 1 Weselcie sie w Panu sprawiedliwi; bo szczerym przystoi chwalic Pana.
\par 2 Wyslawiajcie Pana na harfie, na lutni, na instrumencie o dziesieciu stronach, spiewajcie mu.
\par 3 Spiewajciez mu piosnke nowa; dobrze mu i glosno grajcie.
\par 4 Albowiem szczere jest slowo Panskie, i wszystkie sprawy jego wierne.
\par 5 Miluje sad i sprawiedliwosc; pelna jest ziemia milosierdzia Panskiego.
\par 6 Slowem Panskiem sa niebiosa uczynione, a Duchem ust jego wszystko wojsko ich.
\par 7 Który zgromadzil jako na kupe wody morskie, i zlozyl do skarbu przepasci.
\par 8 Niech sie boi Pana wszystka ziemia; niech sie go lekaja wszyscy obywatele okregu ziemi.
\par 9 Albowiem on rzekl, i stalo sie; on rozkazal, a stanelo.
\par 10 Pan rozprasza rady narodów, a wniwecz obraca zamysly ludzkie;
\par 11 Ale rada Panska trwa na wieki, a mysli serca jego od narodu do narodu.
\par 12 Blogoslawiony naród, którego Pan jest Bogiem jego; lud, który sobie obral za dziedzictwo.
\par 13 Pan patrzy z nieba, i widzi wszystkich synów ludzkich.
\par 14 Z miejsca mieszkania swego spoglada na wszystkich obywateli ziemi.
\par 15 Który stworzyl serce kazdego z nich, upatruje wszystkie sprawy ich.
\par 16 Nie bywa król wybawiony przez wielkosc wojska, ani mocarz nie ujdzie przez wielka moc swoje.
\par 17 Omylnyc jest kon ku wybawieniu, a nie wyrywa wielkoscia mocy swojej.
\par 18 Oto oko Panskie nad tymi, którzy sie go boja, nad tymi, którzy ufaja w milosierdziu jego;
\par 19 Aby wyrwal od smierci dusze ich, a pozywil ich w glodzie.
\par 20 Dusza nasza oczekuje Pana; on ratunek nasz i tarcza nasza.
\par 21 W nim zaprawde rozweseli sie serce nasze; bo w imieniu jego swietem ufamy.
\par 22 Niechze bedzie milosierdzie twoje, Panie! nad nami, jakosmy nadzieje w tobie mieli.

\chapter{34}

\par 1 Psalm Dawidowy, gdy sobie odmienil postawe przed Abimelechem, od którego bedac wygnany, odszedl.
\par 2 Bede blogoslawil Pana na kazdy czas; zawzdy bedzie chwala jego w ustach moich.
\par 3 W Panu sie chlubic bedzie dusza moja, co uslyszawszy pokorni rozwesela sie.
\par 4 Wielbijcie Pana ze mna, a wywyzszajmy imie jego spolecznie.
\par 5 Bom szukal Pana, i wysluchal mie, a ze wszystkich strachów moich wyrwal mie.
\par 6 Którzy nan spogladaja, a zbiegaja sie do niego, oblicza ich nie beda zawstydzone.
\par 7 Ten chudzina wolal, a Pan wysluchal, i ze wszystkich ucisków jego wybawil go.
\par 8 Zatacza obóz Aniol Panski okolo tych, którzy sie go boja, i wyrywa ich.
\par 9 Skosztujciez, a obaczcie, jako jest dobry Pan: blogoslawiony czlowiek, który w nim ufa.
\par 10 Bójcie sie Pana swieci jego; bo niemasz niedostatku bojacym sie go.
\par 11 Lwieta niedostatek cierpia i glód; lecz szukajacym Pana nie bedzie schodzilo na wszelkiem dobrem.
\par 12 Pójdzciez synowie, sluchajcie mie; bojazni Panskiej was naucze.
\par 13 Któz jest, co chce dlugo zyc, a miluje dni, aby widzial dobra?
\par 14 Strzez jezyka twego od zlego, a warg twoich, aby nie mówily zdrady.
\par 15 Odwróc sie od zlego, a czyn dobrze; szukaj pokoju, a scigaj go.
\par 16 Oczy Panskie otworzone sa na sprawiedliwych, a uszy jego na wolanie ich;
\par 17 Ale oblicze Panskie przeciwko tym, którzy broja zlosci, aby wykorzenil z ziemi pamiatke ich.
\par 18 Wolaja sprawiedliwi, a Pan ich wysluchiwa, i ze wszystkich trudnosci ich wybawia ich.
\par 19 Bliski jest Pan tym, którzy sa skruszonego serca, a utrapionych w duchu zachowuje.
\par 20 Wiele zlego przychodzi na sprawiedliwego; ale z tego wszystkiego wyrywa go Pan.
\par 21 On strzeze wszystkich kosci jego, tak, iz i jedna z nich nie skruszy sie.
\par 22 Zabije zlosc niepoboznego, a którzy w nienawisci maja sprawiedliwego, beda spustoszeni;
\par 23 Ale Pan odkupi dusze slug swoich, a nie beda spustoszeni wszyscy, którzy w nim ufaja.

\chapter{35}

\par 1 Psalm Dawidowy. Rozpieraj sie, Panie! z tymi, którzy sie ze mna spieraja; a walcz przeciwko tym, którzy walcza przeciwko mnie.
\par 2 Porwij pukierz i tarcze, a powstan na ratunek mój.
\par 3 Dobadz wlóczni, a staw sie na drodze przeciwko tym, którzy mie przesladuja. Rzeczze duszy mojej: Jam jest zbawieniem twojem.
\par 4 Niech beda pohanbieni i zawstydzeni, którzy szukaja duszy mojej; niech tyl podadza, i niech beda zawstydzeni, którzy mi zle mysla.
\par 5 Niech beda jako plewy przed wiatrem, a Aniol Panski niechaj ich rozproszy.
\par 6 Niech bedzie droga ich ciemna i sliska, Aniol Panski niech ich goni.
\par 7 Albowiem bez przyczyny zastawili na mie w dole sieci swoje, i bez przyczyny ukopali dól duszy mojej.
\par 8 Niechaj na nich przyjdzie spustoszenie, którego sie nie spodziewali; a siec ich, która zastawili, niech ich ulowi na zginienie, a niech w nia wpadna.
\par 9 Ale dusza moja niech sie rozraduje w Panu, niech sie rozweseli w zbawieniu jego.
\par 10 Tedy wszystkie kosci moje rzeka: Panie! któz podobny tobie? który wyrywasz utrapionego od mocniejszego naden, a nedznego i ubogiego od drapiezcy jego.
\par 11 Powstawaja swiadkowie falszywi, a o czem nie wiem, pytaja mie.
\par 12 Oddawaja mi zlem za dobre, chcac mie pozbawic duszy mojej,
\par 13 Chociazem sie ja w wór oblóczyl, gdy oni chorowali; trapilem postem dusze moje, i modlilem sie czesto sam u siebie za nimi.
\par 14 Jako do przyjaciela, jako do brata mego, ustawiczniem chadzal; ponizalem sie jako ten, który sie smuci, chodzac po matce w zalobie.
\par 15 Lecz oni, gdym ja chorowal, weselili sie, i zbierali sie; zbierali sie przeciwko mnie, jakoby byli dla mnie utrapieni, czegom ja nie spostrzegl; szczypali mie, a nie milczeli.
\par 16 Z obludnikami, z nasmiewcami, z pochlebcami zgrzytali na mie zebami swemi.
\par 17 Panie! dlugoz na to patrzec bedziesz? wyrwijze dusze moje od zguby ich, od lwiat jedynaczke moje.
\par 18 Bede cie wyslawial w zgromadzeniu wielkiem; miedzy ludem wielkim bede cie chwalil.
\par 19 Niech sie nie wesela ze mnie, którzy mi sa nieprzyjaciólmi bez przyczyny; którzy mie maja w nienawisci nieslusznie, niech nie mrugaja okiem.
\par 20 Albowiem nie mówia o pokoju; ale przeciwko spokojnym na ziemi zdradliwe slowa zmyslaja.
\par 21 Owszem, rozdzieraja na mie gebe swa, mówiac: Ehej! ehej! widzic to oko nasze.
\par 22 Widzisz to, Panie! nie milczze Panie! nie oddalaj sie odemnie.
\par 23 Obudzze sie, a ocuc dla sadu mego, Boze mój i Panie mój! dla sprawy mojej.
\par 24 Sadz mnie wedlug sprawiedliwosci twojej, Panie Boze mój! a niech sie nie wesela nademna.
\par 25 Niech nie mówia w sercu swojem: Ehej, duszo masza! niech nie mówia: Pozarlismy go.
\par 26 Niechajze beda pohanbieni, i zawstydzeni wszyscy weselacy sie ze zlego mego; niech beda obleczeni w hanbe, i w sromote, którzy sie chlubia przeciwko mnie.
\par 27 Ale ci, którzy sie kochaja w sprawiedliwosci mojej, niech spiewaja i raduja sie, a niech mówia ustawicznie: Niech bedzie uwielbiony Pan, który zyczy pokoju sludze swemu.
\par 28 A jezyk mój bedzie opowiadal sprawiedliwosc twoje, i na kazdy dzien chwale twoje.

\chapter{36}

\par 1 Przedniejszemu spiewakowi piesn Dawida, slugi Panskiego.
\par 2 Przewrotnosc niepoboznego swiadczy w sercu mojem: Niemasz bojazni Bozej przed oczyma jego.
\par 3 Bo sobie poblaza w oczach swoich, aby wykonal nieprawosc swoje az do obmierzenia.
\par 4 Slowa ust jego sa nieprawosc i zdrada; nie chcial rozumiec, aby dobrze czynil.
\par 5 Nieprawosc rozmysla na lozu swojem, stoi na drodze nie dobrej, a zlego sie nie waruje.
\par 6 Panie! milosierdzie twoje niebios siega, prawda twoja az pod obloki,
\par 7 Sprawiedliwosc twoja, jako góry najwyzsze; sady twoje, jako przepasc wielka; ludzie i zwierzeta zachowuje, Panie!
\par 8 Jakoz drogie jest milosierdzie twoje, Boze! przetoz synowie ludzcy w cieniu skrzydel twoich ufaja.
\par 9 Beda upojeni hojnoscia domu twego, a strumieniem rozkoszy twoich napoisz ich.
\par 10 Albowiem u ciebie jest zródlo zywota, a w swiatlosci twojej ogladamy swiatlosc.
\par 11 Rozciagnij milosierdzie twoje nad tymi, którzy cie znaja, a sprawiedliwosc twoje nad uprzejmymi sercem.
\par 12 Niech nie nastepuje na mie noga pysznych, a reka niepoboznych niech mie nie uwodzi.
\par 13 Tam, gdzie upadli, którzy czynili nieprawosc, porazeni sa, i nie mogli powstac.

\chapter{37}

\par 1 Piesn Dawidowa. Nie obruszaj sie dla zlosników, ani zajrzyj czyniacym nieprawosc.
\par 2 Bo jako trawa predko podcieci beda, a jako liscie zielone opadna.
\par 3 Ufaj w Panu, a czyn dobrze; mieszkajze na ziemi, a zyw sie sprawiedliwie.
\par 4 Kochaj sie w Panu, a dac prosby serca twego,
\par 5 Spusc na Pana droge twoje, a ufaj w nim, a on wszystko uczyni;
\par 6 I wywiedzie jako swiatlosc sprawiedliwosc twoje, a sad twój jako poludnie.
\par 7 Poddaj sie Panu, a oczekuj go; nie obruszaj sie na tego, któremu sie szczesci w sprawach jego, na czlowieka, który dokazuje, cokolwiek zamysli.
\par 8 Przestan gniewu, a zaniechaj popedliwosci; nie zapalaj sie gniewem, abys mial zle czynic.
\par 9 Albowiem zlosnicy beda wykorzenieni: lecz którzy oczekuja Pana, ci odziedzicza ziemie.
\par 10 Po malej chwili alic niemasz niezboznika; spojrzyszli na miejsce jego, alic go juz niemasz.
\par 11 Lecz pokorni odziedzicza ziemie, i rozkochaja sie w wielkosci pokoju.
\par 12 Zle mysli niepobozny przeciwko sprawiedliwemu, i zgrzyta nan zebami swemi.
\par 13 Ale sie Pan smieje z niego; bo widzi, ze przychodzi dzien jego.
\par 14 Miecza dobyli niezbozni, a naciagneli luk swój, aby porazili ubogiego, i niedostatecznego, azeby pomordowali tych, którzy chodza prosta droga;
\par 15 Alec miecz ich przeniknie serce ich, a luki ich beda polamane.
\par 16 Lepsza jest trocha sprawiedliwego, niz wielkie bogactwa wielu niepoboznych;
\par 17 Albowiem ramiona niezbozników beda pokruszone; ale sprawiedliwych Pan podpiera.
\par 18 Zna Pan dni doskonalych; przetoz dziedzictwo ich na wieki zostanie.
\par 19 Nie beda zawstydzeni we zly czas, a we dni glodu beda nasyceni;
\par 20 Ale niezbozni pogina, a nieprzyjaciele Panscy, jako tlustosc barania z dymem niszczeje, tak oni zniszczeja.
\par 21 Niezboznik pozycza, a nie ma czem oddac; ale sprawiedliwy pokazuje laske, i rozdaje.
\par 22 Albowiem blogoslawieni od Pana odziedzicza ziemie; ale przekleci od niego beda wykorzenieni.
\par 23 Od Pana bywaja sprawowane drogi czlowieka dobrego, a droga jego, podoba mu sie.
\par 24 Gdy padnie, nie stlucze sie: albowiem Pan trzyma go za reke jego.
\par 25 Bylem mlodym, i zstarzalem sie, nie widzialem sprawiedliwego opuszczonego, ani nasienia jego zebrzacego chleba.
\par 26 Na kazdy dzien pokazuje milosierdzie i pozycza, a przeciez nasienie jego jest w blogoslawienstwie.
\par 27 Odstap od zlego a czyn dobrze, a bedziesz mieszkal na wieki.
\par 28 Albowiem Pan miluje sad, a nie opusci swietych swoich, na wieki w strazy jego beda; ale nasienie niepoboznych bedzie wykorzenione.
\par 29 Sprawiedliwi odziedzicza ziemie, i beda w niej mieszkali na wieki.
\par 30 Usta sprawiedliwego mówia madrosc, a jezyk jego sad opowiada.
\par 31 Zakon Boga jego jest w sercu jego; przetoz nie zachwieja sie nogi jego.
\par 32 Wypatruje niepobozny sprawiedliwego, i szuka jakoby go zabil;
\par 33 Ale Pan nie zostawi go w reku jego, i nie potepi go, gdy bedzie sadzony.
\par 34 Oczekuj Pana, i strzez drogi jego, a on cie wywyzszy, abys odziedziczyl ziemie; a ogladasz, gdy niepobozni, wytraceni beda.
\par 35 Widzialem niezboznika nader wynioslego, a rozlozonego jako drzewo zielone samorosle;
\par 36 Ale przeminal, a oto go nie bylo; szukalem go, alem go znalesc nie mógl.
\par 37 Spojrzyj na niewinnego, a przypatrz sie szczeremu, ze ostatnie rzeczy takiego czlowieka sa spokojne.
\par 38 Lecz przestepcy pospolu pogina, a niezboznicy na ostatek wykorzenieni beda.
\par 39 Wszakze zbawienie sprawiedliwych jest od Pana, który jest moca ich czasu ucisnienia.
\par 40 Wspomaga ich Pan, i wyrywa ich; wyrywa ich od niepoboznych, i zachowuje ich; bo w nim nadzieje maja.

\chapter{38}

\par 1 Psalm Dawidowy ku przypominaniu.
\par 2 Panie! w popedliwosci twojej nie nacieraj na mie, a w gniewie twoim nie karz mie.
\par 3 Albowiem strzaly twoje utknely we mnie, a reka twoja dolega mie.
\par 4 Niemasz nic calego w ciele mojem dla rozgniewania twego; niemasz odpoczynku kosciom moim dla grzechu mojego.
\par 5 Bo nieprawosci moje przycisnely glowe moje; jako brzemie ciezkie obciazyly mie.
\par 6 Zjatrzyly sie, i pognily rany moje, dla glupstwa mojego.
\par 7 Skurczylem sie, i skrzywilem sie bardzo, na kazdy dzien w zalobie chodze.
\par 8 Albowiem wnetrznosci moje pelne sa brzydkosci, a nie masz nic calego w ciele mojem.
\par 9 Zemdlalem, i startym jest bardzo, rycze dla trwogi serca mego.
\par 10 Panie! przed toba jest wszystka zadosc moja, a wzdychanie moje przed toba nie jest skryte.
\par 11 Serce moje skacze; opuscila mie sila moja, a jasnosci oczów moich nie masz przy mnie.
\par 12 Którzy mie miluja, i przyjaciele moi, stronia od ran moich, a powinowaci moi z daleka stoja.
\par 13 I zastawili sidla ci, którzy szukaja duszy mojej; a którzy mi szukaja zlego, mówili przewrotnie, i zdrady przez caly dzien zmyslali.
\par 14 Alem ja niby gluchy nie slyszal, a jako niemy, który ust swoich nie otwiera.
\par 15 I stalem sie jako czlowiek, który nic nie slyszy, i niema odporu w ustach swoich.
\par 16 Albowiem na cie, Panie! oczekuje; ty za mie odpowiesz, Panie, Boze mój!
\par 17 Bom rzekl: Niechaj sie nie ciesza ze mnie; gdyby szwankowala noga moja, niechaj sie hardzie nie podnosza przeciwko mnie.
\par 18 Bom ja upadku bliski, a bolesc moja zawzdy jest przedemna.
\par 19 Owszem, nieprawosc moje wyznaje, a frasuje sie dla grzechu mojego.
\par 20 Ale nieprzyjaciele moi wesela sie, zmacniaja sie, i rozmnazaja sie ci, którzy mie nienawidza bez przyczyny:
\par 21 A oddawajac mi zlem za dobre sprzeciwiaja mi sie, przeto, ze nasladuje tego, co jest dobrego.
\par 22 Nie opuszczajze mie, Panie, Boze mój! nie oddalajze sie odemnie.
\par 23 Pospiesz na ratunek mój, Panie zbawienia mego!

\chapter{39}

\par 1 Przedniejszemu spiewakowi Jedytunowi psalm Dawidowy.
\par 2 Rzeklem: Bede strzegl dróg moich, abym nie zgrzeszyl jezykiem swym; wloze munsztuk w usta moje, póki niepobozny bedzie przedemna.
\par 3 Zaniemialem milczac; zamilknalem i w dobrej sprawie; ale bolesc moja bardziej sie wzmagala.
\par 4 Rozpalilo sie serce moje we wnetrznosciach moich; w rozmyslaniu mojem rozzarzyl sie ogien, azem tak rzekl jezykiem swoim:
\par 5 Daj mi poznac, Panie! dokonczenie moje, i wymiar dni moich jaki jest, abym wiedzial, jak dlugo trwac bede.
\par 6 Otos na dloni wymierzyl dni moje, a wiek mój jest jako nic przed toba; zaprawde szczera marnoscia jest wszelki czlowiek, choc najduzszy. Sela.
\par 7 Zaprawde pomija czlowiek jako cien; zaprawde prózno sie klopocze, zgromadza, a nie wie, kto to pobierze.
\par 8 A teraz na cóz oczekuje, Panie? Tys jest sam oczekiwaniem mojem.
\par 9 Przetoz od wszystkich przestepstw moich wybaw mie; na posmiech glupiemu nie dawaj mie.
\par 10 Zaniemialem, i nie otworzylem ust moich, przeto, zes to ty uczynil.
\par 11 Odejmij odemnie karanie twoje; bom od smagania reki twojej ustal.
\par 12 Gdy ty gromiac karzesz czlowieka dla nieprawosci, wnet niszczysz jako mól grzecznosc jego; zaistec marnoscia jest wszelki czlowiek. Sela.
\par 13 Wysluchajze modlitwe moje, Panie! a wolanie moje przyjmij w uszy swoje, nie milcz na lzy moje; bomci ja przychodniem u ciebie, i komornikiem, jako wszyscy ojcowie moi.
\par 14 Sfolguj mi, abym sie posilil, pierwej nizeli odejde, a nie bedzie mie.

\chapter{40}

\par 1 Przedniejszemu spiewakowi psalm Dawidowy.
\par 2 Z zadoscia oczekiwalem Pana; a sklonil sie ku mnie, i wysluchal wolanie moje;
\par 3 I wyciagnal miecz z dolu szumiacego i z blota lgnacego, a postawil na skale nogi moje, i utwierdzil kroki moje;
\par 4 A wlozyl w usta moje piesn nowa, chwale nalezaca Bogu naszemu, co gdy wiele ich oglada, ulekna sie, a beda miec nadzieje w Panu.
\par 5 Blogoslawiony czlowiek, który poklada w Panu nadzieje swoje, a nie oglada sie na hardych, ani na tych, którzy sie unosza za klamstwem.
\par 6 Wieles uczynil, Panie, Boze mój! cudów twoich, a mysli twoich o nas nikt porzadnie wyliczyc nie moze przed toba; chciallibym je wypowiedziec i wymówic, daleko ich wiecej, nizby wypowiedziane byc mogly.
\par 7 Ofiary i obiaty nie chciales, ales mi przeklól uszy; calopalenia i ofiary za grzech nie zadales.
\par 8 Tedym rzekl: Oto ide; w ksiegach napisano o mnie;
\par 9 Abym czynil wole twoje, Boze mój! pragne, albowiem zakon twój jest w posrodku wnetrznosci moich.
\par 10 Opowiadalem sprawiedliwosc twoje w zgromadzeniu wielkiem; oto warg moich nie zawsciagnalem, ty wiesz, Panie!
\par 11 Sprawiedliwosci twojej nie ukrylem w posród serca mego, prawde twoje i zbawienie twoje opowiadalem; nie tailem milosierdzia twego i prawdy twojej w zgromadzeniu wielkiem.
\par 12 Przetoz ty, Panie! nie zawsciagaj odemnie litosci twoich; milosierdzie twoje i prawda twoja niech mie zawzdy strzega.
\par 13 Albowiem ogarnely mie nieszczescia, którym niemasz liczby; doscignely mie nieprawosci moje, tak, ze przejrzec nie moge; rozmnozyly sie nad wlosy glowy mojej, a serce moje opuscilo mie.
\par 14 Raczze mie, Panie! wyrwac; o Panie! na ratunek mój pospiesz.
\par 15 Niech beda pohanbieni, (a niech sie zawstydza wszyscy,)którzy szukaja duszy mojej, aby ja zatracili; niechajze sie na wstecz cofna, a niech sie zawstydza, którzy mi zycza zlego.
\par 16 Niech beda spustoszeni za to, ze mie shanbic usiluja, mówiac mi: Ehej! ehej!
\par 17 Ale niech sie rozraduja i rozwesela w tobie wszyscy, którzy cie szukaja, i miluja zbawienie twoje; niech mówia zawzdy: Niechaj bedzie Pan uwielbiony.
\par 18 Jamci wprawdzie ubogi i nedzny; alec Pan mysli o mnie. Tys jest pomocnikiem moim i wybawicielem moim; Boze mój! nie omieszkujze.

\chapter{41}

\par 1 Przedniejszemu spiewakowi piesn Dawidowa.
\par 2 Blogoslawiony, który ma baczenie na potrzebnego; w dzien zly wybawi go Pan.
\par 3 Pan go bedzie strzegl, i zywic go bedzie; blogoslawony bedzie na ziemi, ani go poda na wole nieprzyjaciól jego.
\par 4 Pan go posili na lozu niemocy jego; wszystko lezenie jego odmieni w chorobie jego.
\par 5 Jam rzekl: Panie! zmiluj sie nademna, uzdrów dusze moje, bom tobie zgrzeszyl.
\par 6 Nieprzyjaciele moi mówili zle o mnie: Kiedyz wzdy umrze, a zginie imie jego?
\par 7 Jezli tez który z nich przychodzi, aby mie nawiedzil, tedy na zdradzie mówi; serce jego zgromadza sobie nieprawosc, a precz odszedlszy roznosi.
\par 8 Spolem przeciwko mnie szepcza wszyscy, którzy mie maja w nienawisci, a mysla zle o mnie,
\par 9 Mówiac: Pomsta sie nan za niezboznosc wylala, a iz sie polozyl, wiecej nie wstanie.
\par 10 Takze i ten, z którymem zyl w pokoju, któremum ufal, który chleb mój jadal, podniósl piete przeciwko mnie.
\par 11 Ale ty, Panie! zmiluj sie nademna, a podnies mie, i oddam im.
\par 12 A przez to poznam, ze sie kochasz we mnie, gdy sie nie bedzie weselil nieprzyjaciel mój ze mnie.
\par 13 Ale ty w niewinnosci mojej wesprzesz mie i postawisz mie przed obliczem twojem na wieki.
\par 14 Blogoslawiony Pan, Bóg Izraelski, od wieku az na wieki. Amen, Amen.

\chapter{42}

\par 1 Przedniejszemu spiewakowi z synów Korego piesn cwiczaca.
\par 2 Jako jelen krzyczy do strumieni wód, tak dusza moja wola do ciebie, o Boze!
\par 3 Pragnie dusza moja do Boga, do Boga zywego, mówiac: Kiedyz przyjde, a okaze sie przed obliczem Bozem?
\par 4 Lzy moje sa mi miasto chleba we dnie i w nocy, gdy mi mówia co dzien: Kedyz jest Bóg twój?
\par 5 Na to wspominajac wylewam sam sobie dusze moje, zem bywal w poczcie innych, i chadzalem z nimi do domu Bozego, z wesolym glosem, i z chwala, w mnóstwie weselacych sie.
\par 6 Przeczze sie smucisz, duszo moja! a przecz soba trwozysz we mnie? Czekaj na Boga; albowiem go jeszcze bede wyslawial za wielkie wybawienie twarzy jego.
\par 7 Boze mój! dusza moja teskni sobie we mnie; przetoz na cie wspominam w ziemi Jordanskiej i Hermonskiej, na górze Mizar.
\par 8 Przepasc przepasci przyzywa, na szum upustów twoich: wszystkie powodzi twoje i nawalnosci twoje na mie sie zwalily.
\par 9 Wszakze we dnie udzieli mi Pan milosierdzia swego, a w nocy piosnka jego bedzie ze mna, i modlitwa do Boga zywota mego.
\par 10 Rzeke Bogu, skale mojej: Przeczzes mie zapomnial? I czemu smutno chodze dla ucisnienia od nieprzyjaciela?
\par 11 Jest jako rana w kosciach moich, gdy mi uragaja nieprzyjaciele moi, mówiac do mnie na kazdy dzien: Kedy jest Bóg twój?
\par 12 Przeczze sie smucisz, duszo moja? a przecz soba trwozysz we mnie? Czekaj na Boga; albowiem go jeszcze bede wyslawial, gdyz on jest wielkiem zbawieniem twarzy mojej, i Bogiem moim.

\chapter{43}

\par 1 Sadz mie, o Boze! a ujmij sie o sprawe moje; od narodu niemilosiernego, i od czlowieka zdradliwego i niezboznego wyrwij mie;
\par 2 Bos ty jest Bóg sily mojej. Przeczzes mie odrzucil? a przecz smutno chodze dla ucisnienia od nieprzyjaciela?
\par 3 Zeslij swiatlosc twoje, i prawde twoje; te mie poprowadza, i wprowadza mie na swieta góre twoje, i do przybytków twoich,
\par 4 Abym przystapil do oltarza Bozego, do Boga wesela i radosci mojej; i bede cie wyslawial na harfie, o Boze, Boze mój!
\par 5 Przeczze sie smucisz, duszo moja, a przecz trwozysz soba we mnie? Czekaj na Boga, albowiem go jeszcze bede wyslawial, gdyz on jest wielkiem zbawieniem twarzy mojej, i Bogiem moim.

\chapter{44}

\par 1 Przedniejszemu spiewakowi z synów Korego psalm nauczajacy.
\par 2 Boze! uszami naszemi slyszelismy; ojcowie nasi powiadali nam o sprawach, któres czynil za dni ich, za dni starodawnych.
\par 3 Tys reka swa wypedzil pogan, a onyches wszczepil; wytraciles narody, a onyches rozkrzewil.
\par 4 Bo nie przez miecz swój posiedli ziemie, i ramie ich nie wybawilo ich, ale prawica twoja i ramie twoje, a swiatlosc oblicza twego, przeto, zes ich upodobal sobie.
\par 5 Tys sam król mój, o Boze! sprawze wielkie wybawienie Jakóbowi.
\par 6 Przez cie nieprzyjaciól naszych porazalismy; w imieniu twojem deptalismy powstawajacych przeciwko nam.
\par 7 Bom w luku moim nie ufal, ani miecz mój obronil mie;
\par 8 Ales nas ty wybawial od nieprzyjaciól naszych, a nienawidzacych nas zawstydzales.
\par 9 Przetoz chlubimy sie w tobie, Boze! na kazdy dzien, a imie twoje na wieki wyslawiamy. Sela.
\par 10 Ale teraz odrzuciles i zawstydziles nas, a nie wychodzisz z wojskami naszemi.
\par 11 Sprawiles, zesmy tyl podali nieprzyjacielowi, a ci, którzy nas maja w nienawisci, rozchwycili miedzy sie dobra nasze.
\par 12 Podales nas jako owce na zer, a miedzy pogan rozproszyles nas.
\par 13 Sprzedales lud twój za nic, a nie podniosles ceny ich.
\par 14 Podales nas na wzgarde sasiadom naszym, na szyderstwo i na posmiech tym, którzy sa okolo nas.
\par 15 Wystawiles nas na przypowiesc miedzy poganami, tak, ze nad nami narody glowa kiwaja.
\par 16 Na kazdy dzien wstyd mój jest przedemna, a hanba twarzy mojej okrywa mie.
\par 17 Dla glosu tego, który mie sromoci i potwarza, dla nieprzyjaciela, i tego, który sie msci.
\par 18 To wszystko przyszlo na nas; a wzdysmy cie nie zapomnieli, anismy wzruszyli przymierza twego.
\par 19 Nie cofnelo sie nazad serce nasze, ani sie uchylily kroki nasze od sciezki twojej,
\par 20 Chociazes nas byl potarl, wrzuciwszy nas na miejsce smoków, i okryles nas cieniem smierci.
\par 21 Bysmyc byli zapomnieli imienia Boga naszego, a podniesli rece nasze do Boga cudzego,
\par 22 Iazliby sie byl Bóg o tem nie pytal? gdyz on wie skrytosci serca.
\par 23 Alec nas dla ciebie zabijaja na kazdy dzien; poczytaja nas jako owce na rzez zgotowane.
\par 24 Ocuc sie; przeczze spisz, Panie! Przebudz sie, nie odrzucaj nas na wieki.
\par 25 Przeczze oblicze twoje ukrywasz, a zapominasz utrapienia naszego i ucisku naszego?
\par 26 Albowiem potloczona jest az do prochu dusza nasza, a przylgnal do ziemi zywot nasz.
\par 27 Powstanze na ratunek nasz, a odkup nas dla milosierdzia twego.

\chapter{45}

\par 1 Przedniejszemu spiewakowi z synów Korego na Sosannim psalm nauczajacy, a piesn weselna.
\par 2 Wydalo serce moje slowo dobre; rozprawiac bede piesni moje, o królu! jezyk mój bedzie jako pióro predkiego pisarza.
\par 3 Piekniejszys nad synów ludzkich; rozlala sie wdziecznosc po wargach twoich, przeto, ze cie poblogoslawil Bóg az na wieki.
\par 4 Przypasz miecz twój na biodra, o mocarzu! pokaz chwale twoje, i zacnosci twoje.
\par 5 A w dostojnosci twojej szczesliwie wywiedz z slowem prawdy, cichosci, i sprawiedliwosci, a dokaze strasznych rzeczy prawica twoja.
\par 6 Strzaly twoje ostre; od nich narody pod cie upadna, a serce nieprzyjaciól królewskich przenikna.
\par 7 Stolica twoja, o Boze! na wieki wieków; laska sprawiedliwosci jest laska królestwa twego.
\par 8 Umilowales sprawiedliwosc, a nienawidziles nieprawosci; przetoz pomazal cie, o Boze! Bóg twój olejkiem wesela nad uczestników twoich.
\par 9 Myrra, aloe, i kassyja wszystkie szaty twoje pachna, gdy wychodzisz z palaców z kosci sloniowych urobionych, nad tych, którzy cie uweselaja.
\par 10 Córki królewskie sa miedzy twemi zacnemi bialemi glowami; stanela malzonka po prawicy twojej w kosztownem zlocie z Ofir.
\par 11 Sluchajze córko, a obacz, i naklon ucha twego, a zapomnij narodu twego, i domu ojca twojego.
\par 12 A zakocha sie król w pieknosci twojej, albowiem on jest Panem twoim; przetoz klaniaj sie przed nim.
\par 13 Tyryjczycy takze z upominkami przed obliczem twojem klaniac sie beda, najbogatsi z narodów.
\par 14 Wszystka zacnosc córki królewskiej jest wewnatrz, a szaty jej bramowane sa zlotem.
\par 15 W odzieniu haftowanem przywioda ja do króla; takze panny za nia, towarzyszki jej, przywioda do ciebie.
\par 16 Przywioda je z weselem i z radoscia, a wnijda na palac królewski.
\par 17 Miasto ojców twych bedziesz miec synów twych, których postanowisz ksiazetami po wszystkiej ziemi.
\par 18 Wspominac bede imie twoje od kazdego rodzaju do rodzaju: dlatego cie narody wyslawiac beda na wieki wieków.

\chapter{46}

\par 1 Przedniejszemu spiewakowi z synów Korego, na Alamot piesn.
\par 2 Bóg jest ucieczka i sila nasza, ratunkiem we wszelkim ucisku najpewniejszym.
\par 3 Przetoz sie bac nie bedziemy, chocby sie poruszyla ziemia, chocby sie przeniosly góry w posród morza;
\par 4 Chocby zaszumialy, a wzburzyly sie wody jego, i zatrzesly sie góry od nawalnosci jego. Sela.
\par 5 Strumienie rzeki jego rozweselaja miasto Boze, najswietsze z przybytków najwyzszego.
\par 6 Bóg jest w posrodku jego, nie bedzie poruszone; poratuje go Bóg zaraz z poranku.
\par 7 Gdy sie wzburzyly narody, a zatrzasnely sie królestwa, Pan wydal glos swój, i rozplynela sie ziemia.
\par 8 Pan zastepów jest z nami; twierdza wysoka jest nam Bóg Jakóbowy. Sela.
\par 9 Pójdzcie, ogladajcie sprawy Panskie, jakie uczynil spustoszenie na ziemi;
\par 10 Który usmierza wojny az do konczyn ziemi, luk kruszy, i oreze lamie, a wozy ogniem pali.
\par 11 Mówiac: Uspokójcie sie, a wiedzcie, zem Ja Bóg; bede wywyzszony miedzy narodami, bede wywyzszony na ziemi.
\par 12 Pan zastepów z nami; twierdza wysoka jest nam Bóg Jakóbowy. Sela.

\chapter{47}

\par 1 Przedniejszemu spiewakowi z synów Korego piesn.
\par 2 Wszystkie narody klaskajcie rekoma, wykrzykajcie Bogu glosem wesela.
\par 3 Albowiem Pan najwyzszy, straszny, jest królem wielkim nad wszystka ziemia.
\par 4 Podbija ludzi pod moc nasze, a narody pod nogi nasze.
\par 5 Obral nam za dziedzictwo nasze chwale Jakóba, którego umilowal. Sela.
\par 6 Wstapil Bóg z krzykiem; Pan wstapil z glosem traby.
\par 7 Spiewajciez Bogu, spiewajcie; spiewajciez królowi naszemu, spiewajcie.
\par 8 Albowiem Bóg królem wszystkiej ziemi; spiewajciez rozumnie.
\par 9 Króluje Bóg nad narodami; Bóg siedzi na swietej stolicy swojej.
\par 10 Ksiazeta narodów przylaczyli sie do ludu Boga Abrahamowego; albowiem Boze sa tarcze ziemskie; skad on zacnie jest wywyzszony.

\chapter{48}

\par 1 Piesn psalmu synów Korego.
\par 2 Wielki jest Pan, i bardzo chwalebny w miescie Boga naszego, na górze swietej swojej.
\par 3 Ozdoba krainy, uciecha wszystkiej ziemi jest góra Syon w stronach pólnocnych, miasto króla wielkiego.
\par 4 Bóg w palacach jego uznany jest za twierdze wysoka.
\par 5 Bo oto królowie, gdy sie zgromadzili i ciagneli wespól,
\par 6 Sami to ujrzawszy bardzo sie zadziwili, a przestraszeni bedac predko uciekali.
\par 7 Strach ich tam ogarnal i bolesc, jako niewiaste rodzaca.
\par 8 Wiatrem wschodnim pokruszysz okrety z Tarsys.
\par 9 Jakosmy slyszeli, takesmy widzieli w miescie Pana zastepów, w miescie Boga naszego; Bóg je ugruntowal az na wieki. Sela.
\par 10 Uwazamy, o Boze! milosierdzie twoje w posród kosciola twego.
\par 11 Jakie jest imie twoje, Boze! taka tez jest chwala twoja az do konczyn ziemi; sprawiedliwosci pelna jest prawica twoja.
\par 12 Niech sie rozweseli góra Syon: niech sie rozraduja córki Judzkie dla sadów twoich, Boze!
\par 13 Otoczcie Syon, i obstapcie go; policzcie wieze jego.
\par 14 Przypatrujcie sie pilnie basztom jego, a ogladajcie palace jego, abyscie umieli powiadac narodowi potomnemu.
\par 15 Ze ten Bóg jest Bogiem naszym na wieki wieczne, a iz on naszym hetmanem bedzie az do smierci.

\chapter{49}

\par 1 Przedniejszemu spiewakowi z synów Korego psalm.
\par 2 Sluchajcie tego wszystkie narody; bierzcie to w uszy wszyscy mieszkajacy na okregu ziemi!
\par 3 Tak z ludu pospolitego, jako z ludzi zacnych, tak bogaty jako ubogi!
\par 4 Usta moje beda opowiadaly madrosc, a mysl serca mego roztropnosc.
\par 5 Naklonie do przypowiesci ucha mego, wyloze przy harfie zagadke moje.
\par 6 Przeczze sie mam bac we zle dni, aby mie nieprawosc tych, którzy mie depcza, miala ogarnac?
\par 7 Którzy ufaja bogactwom swoim, a w mnóstwie dostatków swoich chlubia sie.
\par 8 Gdyz brata swego nikt zadnym sposobem nie odkupi, ani moze dac Bogu okupu jego zan.
\par 9 (Albowiem drogi jest okup duszy ich, i nie moze sie ostac na wieki.)
\par 10 Aby zyl na wieki, a nie ogladal grobu.
\par 11 Bo widzimy, iz i madrzy umieraja, glupi i szalony zarówno gina, a zostawiaja, obcym bogactwa swoje.
\par 12 Mysla, ze domy ich sa wieczne, a przybytki ich trwaja od narodu do narodu; przetoz je nazywaja od imion swych na ziemi.
\par 13 Ale czlowiek we czci nie zostaje, podobnym bedac bydletom, które gina.
\par 14 Takowa mysl ich glupstwem ich jest, a przeciez potomkowie ich pochwalaja to usty swemi. Sela.
\par 15 Jako owce w grobie zlozeni beda, smierc ich strawi; ale sprawiedliwi panowac beda nad nimi z poranku, a ksztalt ich zniszczony bedzie w grobie, gdy ustapia z mieszkania swego.
\par 16 Ale Bóg wykupi dusze moje z mocy grobu, gdy mie przyjmie. Sela.
\par 17 Nie bójze sie, gdy sie kto zbogaci, a gdy sie rozmnozy slawa domu jego.
\par 18 Bo umierajac nie wezmie nic z soba, ani za nim zstapi slawa jego.
\par 19 A choc duszy swej za zywota swego poblaza i chwalono go, gdy sobie dobrze czynil:
\par 20 Przeciez musi isc za rodzina ojców swych, a na wieki nie oglada swiatlosci.
\par 21 Owóz czlowiek, który jest we czci, a nie zrozumiewa tego, podobny jest bydletom, które gina.

\chapter{50}

\par 1 Psalm Asafowi podany. Bóg nad Bogami, Pan mówil i przyzwal ziemie od wschodu slonca az do zachodu jego.
\par 2 Objasnil sie Bóg z Syonu w doskonalej ozdobie.
\par 3 Przyjdzie Bóg nasz, a nie bedzie milczal; ogien przed twarza jego bedzie pozeral, a okolo niego powstanie wicher gwaltowny.
\par 4 Przyzwie z góry niebiosa i ziemie, aby sadzil lud swój.
\par 5 Mówiac: Zgromadzcie mi swietych moich, którzy ze mna uczynili przymierze przy ofierze.
\par 6 Tedy niebiosa opowiedza sprawiedliwosc jego; albowiem sam Bóg jest sedzia. Sela.
\par 7 Sluchaj, ludu mój! a bede mówil; sluchaj, Izraelu! a oswiadcze sie przed toba: Jam Bóg, Bóg twój Jam jest.
\par 8 Nie bede cie z ofiar twoich winil, ani calopalenia twego, które sa zawsze przedemna.
\par 9 Nie wezme z domu twojego cielca, ani z okolu twego kozlów.
\par 10 Albowiem mój jest wszelki zwierz lesny, i tysiace bydla po górach.
\par 11 Znam wszystko ptastwo po górach, i zwierz polny jest przedemna.
\par 12 Bedeli laknal, nie rzekec o to; bo mój jest okrag ziemi, i napelnienie jego.
\par 13 Izali jadam mieso wolowe? albo pijam krew kozlowa?
\par 14 Ofiaruj Bogu chwale, i oddaj Najwyzszemu sluby twoje;
\par 15 A wzywaj mie w dzien utrapienia: tedy cie wyrwe, a ty mie uwielbisz.
\par 16 Lecz niezboznemu rzekl Bóg: Cózci do tego, ze opowiadasz ustawy moje, a biezesz przymierze moje w usta twoje?
\par 17 Poniewaz masz w nienawisci karnosc, i zarzuciles slowa moje za sie.
\par 18 Widziszli zlodzieja, biezysz z nim, a z cudzoloznikami masz sklad twój.
\par 19 Usta twoje rozpuszczasz na zle, a jezyk twój sklada zdrady.
\par 20 Zasiadlszy mówisz przeciwko bratu twemu, a lzysz syna matki twojej.
\par 21 Tos czynil, a Jam milczal; dlategos mniemal, zem ja tobie podobny, ale bede cie karal, i stawiec to przed oczy twoje.
\par 22 Zrozumiejciez to wzdy teraz, którzy zapominacie Boga, bym was snac nie porwal, a nie bedzie ktoby was wyrwal.
\par 23 Kto mi ofiaruje chwale, uczci mie; a temu, który naprawia droge swa, ukaze zbawienie Boze.

\chapter{51}

\par 1 Przedniejszemu spiewakowi psalm Dawidowy.
\par 2 Gdy do niego przyszedl Natan prorok, potem jak byl wszedl do Betsaby.
\par 3 Zmiluj sie nademna, Boze! wedlug milosierdzia twego; wedlug wielkich litosci twoich zgladz nieprawosci moje.
\par 4 Omyj mie doskonale od nieprawosci mojej, a od grzechu mego oczysc mie.
\par 5 Albowiem ja znam nieprawosc moje, a grzech mój przedemna jest zawzdy.
\par 6 Tobie, tobiem samemu zgrzeszyl, i zlem przed oczyma twemi uczynil, abys byl sprawiedliwy w mowie twojej, i czystym w sadzie twoim.
\par 7 Oto w nieprawosci poczety jestem, a w grzechu poczela mie matka moja.
\par 8 Oto sie kochasz w prawdzie wewnetrznej, a skryta madrosc objawiles mi.
\par 9 Oczysc mie, isopem, a oczyszczon bede; omyj mie, a nad snieg wybielony bede.
\par 10 Daj mi slyszec radosc i wesele, a niech sie rozraduja kosci moje, któres pokruszyl.
\par 11 Odwróc oblicze twoje od grzechów moich, a zgladz wszystkie nieprawosci moje.
\par 12 Serce czyste stwórz we mnie, o Boze! a ducha prawego odnów we wnetrznosciach moich.
\par 13 Nie odrzucaj mie od oblicza twego, a Ducha swego swietego nie odbieraj odemnie.
\par 14 Przywróc mi radosc zbawienia twego, a duchem dobrowolnym podeprzyj mie.
\par 15 Tedy bede nauczal przestepców dróg twoich, aby sie grzesznicy do ciebie nawrócili.
\par 16 Wyrwij mie z pomsty za krew, o Boze, Boze zbawienia mojego! a jezyk mój bedzie wyslawial sprawiedliwosc twoje.
\par 17 Panie! otwórz wargi moje, a usta moje opowiadac beda chwale twoje.
\par 18 Albowiem nie pragniesz ofiar, chocbym ci je dal, ani calopalenia przyjmiesz.
\par 19 Ofiary Bogu przyjemne duch skruszony; sercem skruszonem i strapionem nie pogardzisz, o Boze!
\par 20 Dobrze uczyn wedlug upodobania twego Syonowi; pobubuj mury Jeruzalemskie.
\par 21 Tedy przyjmiesz ofiary sprawiedliwosci, ofiary ogniste, i calopalenia; tedy cielce ofiarowac beda na oltarzu twoim.

\chapter{52}

\par 1 Przedniejszemu spiewakowi piesn Dawidowa nauczajaca.
\par 2 Gdy przyszedl Doeg Edomczyk, i oznajmil Saulowi, mówiac: Dawid przyszedl do domu Achimelechowego.
\par 3 Przeczze sie chlubisz ze zlosci, o mocarzu! milosierdzie Boze trwa kazdego dnia.
\par 4 Zle rzeczy mysli jezyk twój, jako brzytwa ostra czyniac zdrade.
\par 5 Umilowales zle, bardziej niz dobre; klamstwo raczej mówisz, niz sprawiedliwosc. Sela.
\par 6 Umilowales wszystkie slowa szkodliwe, i jezyk zdradliwy.
\par 7 Przetoz cie Bóg zniszczy na wieki; porwie cie, i wyrwie cie z przybytku, i wykorzeni cie z ziemi zyjacych. Sela.
\par 8 To widzac sprawiedliwi beda sie bali, i bede sie z niego nasmiewali, mówiac:
\par 9 Otoz czlowiek, który nie pokladal w Bogu sily swojej; ale ufajac w mnóstwie bogactw swoich, zmacnial sie w zlosci swej.
\par 10 Alec ja bede jako oliwa zielona w domu Bozym, bom nadzieje polozyl w milosierdziu Bozem na wieki wieczne.
\par 11 Bede cie wyslawial, Panie! na wieki, zes to uczynil, a bede oczekiwal imienia twego, gdyz jest zacne przed oblicznoscia swietych twoich.

\chapter{53}

\par 1 Przedniejszemu spiewakowi na Machalat piesn Dawidowa nauczajaca.
\par 2 Glupi rzekl w sercu swem: Niemasz Boga. Popsowali sie, i obrzydliwa czynia nieprawosc; niemasz, ktoby czynil dobrze.
\par 3 Bóg z niebios spojrzal na synów ludzkich, aby obaczyl, bylliby kto rozumny i szukajacy Boga.
\par 4 Alec oni wszyscy odstapili, jednako sie nieuzytecznymi stali: niemasz, ktoby czynil dobrze, niemasz, i jednego.
\par 5 Azaz nie wiedza wszyscy czyniciele nieprawosci, ze pozeraja lud mój, jako wiec chleb jedza? ale Boga nie wzywaja.
\par 6 Tam sie bardzo ulekna, gdzie niemasz strachu; albowiem Bóg rozproszy kosci tych, którzy cie oblegli; ty ich pohanbisz, bo ich Bóg wzgardzi.
\par 7 Któz da z Syonu wybawienie Izraelowi? Gdy Bóg przywróci z wiezienia lud swój, rozraduje sie Jakób, rozweseli sie Izrael.

\chapter{54}

\par 1 Przedniejszemu spiewakowi na Neginot piesn Dawidowa nauczajaca.
\par 2 Gdy przyszli Zyfejczycy, i rzekli do Saula: Dawid sie kryje przed toba u nas.
\par 3 Boze! dla imienia twego wybaw mie, a w mocy twojej podejmij sie sprawy mojej.
\par 4 Boze! wysluchaj modlitwe moje; przyjmij w uszy slowa ust moich.
\par 5 Albowiem obcy powstali przeciwko mnie, a okrutnicy szukaja duszy mojej, nie stawiajac sobie Boga przed oczyma swemi. Sela.
\par 6 Oto Bóg jest pomocnikiem moim: Pan jest z tymi, którzy podpieraja zywot mój.
\par 7 Oddaj zlym nieprzyjaciolom moim, w prawdzie twojej wytrac ich, o Panie!
\par 8 Tedyc dobrowolnie bede ofiarowal; bede wyslawial imie twoje, Panie! przeto, ze jest dobre;
\par 9 Gdyz z kazdego utrapienia wyrwales mie, a pomste nad nieprzyjaciolmi mymi ogladalo oko moje.

\chapter{55}

\par 1 Przedniejszemu spiewakowi na Neginot piesn Dawidowa nauczajaca.
\par 2 W uszy swe przyjmij, o Boze! modlitwe moje, a nie kryj sie przed prosba moja:
\par 3 Posluchaj mie z pilnoscia, a wysluchaj mie; boc sie uskarzam w modlitwie swej, i trwoze soba:
\par 4 Dla glosu nieprzyjaciela, i dla ucisnienia od bezboznika; albowiem mie zarzucaja klamstwem, a w popedliwosci swej sprzeciwiaja mi sie.
\par 5 Serce moje boleje we mnie, a strachy smierci przypadly na mie.
\par 6 Bojazn ze drzeniem przyszla na mie, a okryla mie trwoga.
\par 7 I rzeklem: Obym mial skrzydla jako golebica, zalecialbym, a odpoczalbym.
\par 8 Otobym daleko zalecial, a mieszkalbym na puszczy. Sela.
\par 9 Pospieszylbym, abym uszedl przed wiatrem gwaltownym, i przed wichrem.
\par 10 Zatrac ich, Panie! rozdziel jezyk ich; bom widzial bezprawie i rozruch w miescie.
\par 11 We dnie i w nocy otaczaja ich po murach jego, a zlosc i przewrotnosc jest w posrodku jego.
\par 12 Ciezkosci sa w posrodku jego, a nie ustepuje z ulic jego chytrosc i zdrada.
\par 13 Albowiem nie nieprzyjaciel jaki zelzyl mie, inaczej znióslbym to byl; ani ten, który mie mial w nienawisci, powstal przeciwko mnie; bobym sie wzdy byl skryl przed nim;
\par 14 Ale ty, czlowiecze mnie równy, wodzu mój, i znajomy mój.
\par 15 Którzysmy sie z soba mile w tajnosci naradzali, i do domu Bozego spolecznie chadzali.
\par 16 Oby ich smierc z predka zalapila, tak aby zywo zstapili do piekla! albowiem zlosc jest w mieszkaniu ich, i w posrodku ich.
\par 17 Ale ja do Boga zawolam, a Pan mie wybawi.
\par 18 W wieczór i rano, i w poludnie modlic sie, i z trzaskiem wolac bede, az wyslucha glos mój.
\par 19 Odkupi dusze moje, abym byl w pokoju od wojny przeciwko mnie; bo ich wiele bylo przy mnie.
\par 20 Wyslucha Bóg i utrapi ich, (jako ten, który siedzi od wieku.Sela.) przeto, ze nie masz w nich poprawy, ani sie Boga boja.
\par 21 Wyciagnal rece swoje na tych, którzy z nim mieli pokój, wzruszyl przymierze swoje.
\par 22 Gladsze niz maslo byly slowa ust jego, ale walka w sercu jego: a mie kciejsze slowa jego niz olej, wszakze byly jako miecze dobyte:
\par 23 Wrzuc na Pana brzemie twoje, a on cie opatrzy, i nie dopusci, aby sie na wieki zachwiac mial sprawiedliwy.
\par 24 Ale ich ty, o Boze! wepchniesz w dól zginienia; mezowie krwawi i zdradliwi nie dojda do polowy dni swoich; ale ja w tobie nadzieje miec bede.

\chapter{56}

\par 1 Przedniejszemu spiewakowi o niemej golebicy, na miejscach odleglych, zloty psalm Dawidowy, gdy go w Gat Filistynowie pojmali.
\par 2 Zmiluj sie nademna, o Boze!, bo mie chce pochlonac czlowiek; kazdego dnia walczac trapi mie.
\par 3 Chca mie polknac nieprzyjaciele moi na kazdy dzien; zaprawdec wiele jest walczacych przeciwko mnie, o Najwyzszy!
\par 4 Któregokolwiek mie dnia strach ogarnia, ja w tobie ufam.
\par 5 Boga wyslawiac bede dla slowa jego; w Bogu nadzieje bede mial, ani sie bede bal, zeby mi co cialo uczynic moglo.
\par 6 Przez caly dzien slowa moje wykrecaja, a przeciwko mnie sa wszystkie mysli ich, na zle.
\par 7 Zbieraja sie, i ukrywaja sie, i slad mój upatruja, czyhajac na dusze moje.
\par 8 Izali za nieprawosc pomsty ujda? strac te narody, o Boze! w popedliwosci twojej.
\par 9 Tys tulanie moje policzyl; zbierzze tez lzy moje w wiadro twe; izaz nie sa spisane w ksiegach twoich?
\par 10 Tedy sie nazad cofna nieprzyjaciele moi, któregokolwiek dnia zawolam; bo to wiem, iz Bóg jest ze mna.
\par 11 Boga wyslawiac bede z slowa; Pana chwalic bede z slowa jego.
\par 12 W Bogu mam nadzieje, nie bede sie bal, aby mi co mial uczynic czlowiek.
\par 13 Tobiem, o Boze! sluby uczynil; przetoz tez tobie chwaly oddam.
\par 14 Albowiemes wyrwal dusze moje od smierci, a nogi moje od upadku, abym statecznie chodzil przed obliczem Bozem w swiatlosci zyjacych.

\chapter{57}

\par 1 Przedniejszemu spiewakowi, jako: Nie zatracaj, zloty psalm Dawidowy, kiedy uciekal przed Saulem do jaskini.
\par 2 Zmiluj sie nademna, o Boze! zmiluj sie nademna; albowiem w tobie ufa dusza moja, a do cienia skrzydel twoich uciekam sie; az przeminie utrapienie.
\par 3 Bede wolal do Boga najwyzszego, do Boga, który wykonywa sprawe moje.
\par 4 On posle z nieba, i wybawi mie od pohanbienia tego, który mie chce pochlonac. Sela. Posle mi Bóg milosierdzie swoje i prawde swa.
\par 5 Dusza moja jest w posród lwów; leze miedzy palacymi, miedzy synami ludzkimi, których zeby jako wlócznie i strzaly, i jezyk ich miecz ostry.
\par 6 Wywyzze sie nad niebiosa, o Boze! a nade wszystka ziemia chwala twoja.
\par 7 Sieci zastawili na nogi moje, nachylili dusze moje, wykopali dól przed obliczem mojem; ale sami wpadli wen. Sela.
\par 8 Gotowe jest serce moje, Boze! gotowe jest serce moje; spiewac i wychwalac cie bede.
\par 9 Ocuc sie chwalo moja! ocuc sie, lutnio i harfo! gdy na switaniu powstaje.
\par 10 Bede cie wyslawial miedzy ludem, Panie! a bedec spiewal miedzy narodami.
\par 11 Albowiem wielkie jest az do niebios milosierdzie twoje, i az pod obloki prawda twoja.
\par 12 Wywyzze sie nad niebiosa, o Boze! a nade wszystka ziemie wywyz chwale twoje.

\chapter{58}

\par 1 Przedniejszemu spiewakowi, jako: Nie zatracaj, piesn zlota Dawidowa.
\par 2 O zgromadzenie! Izali poprawdzie sprawiedliwosc mówicie? A uprzejmiez sadzicie, wy synowie ludzcy?
\par 3 Owszem, radniej w sercu nieprawosci knujecie, a gwalty rak waszych na ziemi odwazacie.
\par 4 Odlaczyli sie niezboznicy zaraz od narodzenia; pobladzili zaraz z zywota matki swej, mówiac klamstwo.
\par 5 Jad maja w sobie, jako wezowy, jako jad zmii gluchej, która zatula ucho swoje,
\par 6 Aby nie slyszala glosu zaklinacza, ani czarownika w czarach bieglego.
\par 7 O Boze! pokruszze zeby ich w ustach ich; polam, Panie! lwiat trzonowe zeby.
\par 8 Niech sie rozplyna jako woda, niech sie wniwecz obróca; niech beda jako ten, który naciaga luk, wszakze sie strzaly jego lamia.
\par 9 Jako slimak, który schodzi i niszczeje; jako martwy plód niewiesci niech nie ogladaja slonca.
\par 10 Ciernie wasze pierwej niz wypuszcza tarny swoje, za zielona w gniewie Bozym jako wichrem porwane beda.
\par 11 I bedzie sie weselil sprawiedliwy, gdy ujrzy pomste; nogi swoje umyje we krwi niepoboznego.
\par 12 I rzecze kazdy: Zaprawdec sprawiedliwy odniesie pozytek z sprawiedliwosci swojej; zaistec jest Bóg, który sadzi na ziemi.

\chapter{59}

\par 1 Przedniejszemu spiewakowi, jako: Nie zatracaj, zlota piesn Dawidowa, gdy poslal Saul, aby strzezono domu jego, a zabito go.
\par 2 Wyrwij mie od nieprzyjaciól moich, o Boze mój! a od powstawajacych przeciwko mnie uczyn mie bezpiecznym.
\par 3 Wyrwij mie od tych, którzy broja nieprawosc, a od mezów krwawych wybaw mie.
\par 4 Albowiem oto czyhaja na dusze moje; zbieraja sie przeciwko mnie mocarze bez przestepstwa mego i bez grzechu mego, o Panie!
\par 5 Bez wszelkiej nieprawosci mojej zbiegaja sie, i gotuja sie; powstanze, zabiez mi, a obacz.
\par 6 Ty sam, Panie, Boze zastepów, Boze Izraelski! ocuc sie, abys nawiedzil te wszystkie narody; a nie miej litosci nad zadnym z onych przestepców zlosliwych. Sela.
\par 7 Nawracaja sie pod wieczór, a warcza jako psy, i biegaja okolo miasta.
\par 8 Oto blegoca usty swemi, miecze sa w wargach ich; albowiem mówia: Któz slyszy?
\par 9 Ale ty, Panie! nasmiewasz sie z nich; nasmiewasz sie ze wszystkich narodów.
\par 10 Gdy on moc przewodzi, na ciebie pozór miec bede; bos ty, Boze! twierdza moja.
\par 11 Bóg mój milosierny uprzedzi mie; Bóg mi da ogladac pomste nad nieprzyjaciólmi mymi.
\par 12 Nie zabijajze ich, aby nie zapomnial lud mój; ale ich rozprosz moca twoja, i zrzuc ich, tarczo nasza, o Panie!
\par 13 Grzech ust swych, slowa warg swych (pojmani bedac w hardosci swej dla zlorzeczenstwa i klamstwa) niech wyznawaja.
\par 14 Wytracze ich w popedliwosci, wytrac ich, az ich nie stanie. Niech poznaja, ze Bóg panuje w Jakóbie, i po krajach ziemi. Sela.
\par 15 I niech sie zas nawróca pod wieczór; niech warcza jako psy, a biegaja okolo miasta.
\par 16 Niech oni ciekaja, chcac sie najesc, wszakze glodni bedac uklasc sie musza.
\par 17 Ale ja bede spiewal o mocy twojej; zaraz z poranku wyslawiac bede milosierdzie twoje; bos ty byl twierdza moja, i ucieczka w dzien ucisku mego.
\par 18 O mocy moja! tobie bede spiewal; bos ty, Boze! twierdza moja, Bóg mój milosierny.

\chapter{60}

\par 1 Przedniejszemu spiewakowi na Sussanedut zlota piesn Dawidowa do nauczania;
\par 2 Gdy walczyl przeciw Syryjczykom Nacharaim, i przeciw Syryjczykom Soby; gdy sie wrócil Joab, poraziwszy Edomczyków w dolinie solnej dwanascie tysiecy.
\par 3 Boze! odrzuciles nas, rozproszyles nas, i rozgniewales sie; nawrócze sie zasie do nas.
\par 4 Zatrzasnales byl ziemia, i rozsadziles ja; uleczze rozpadliny jej, boc sie chwieje.
\par 5 Okazywales ludowi twemu przykre rzeczy, napoiles nas winem zawrotu.
\par 6 Ale teraz dales choragiew tym, którzy sie ciebie boja, aby ja wyniesli dla prawdy twej. Sela.
\par 7 Aby byli wybawieni umilowani twoi; zachowajze ich prawica twoja, a wysluchaj mie.
\par 8 Bóg ci mówil w swietobliwosci swojej; przeto sie rozwesele, rozdziele Sychem, i doline Sukkotska pomierze.
\par 9 Mojec jest Galaad, mój i Manases, i Efraim moc glowy mojej; Juda zakonodawca moim.
\par 10 Moab miednica do umywania mego; na Edoma wrzuce buty moje; ty, Palestyno! wykrzykaj nademna.
\par 11 Któz mie wprowadzi do miasta obronnego? kto mie przyprowadzi az do Edom?
\par 12 Izali nie ty, o Boze! którys nas byl odrzucil, a nie wychodziles, Boze! z wojskami naszemi?
\par 13 Dajze nam ratunek w utrapieniu; boc omylny ratunek ludzki.
\par 14 W Bogu meznie sobie poczynac bedziemy, a on podepcze nieprzyjaciól naszych.

\chapter{61}

\par 1 Przedniejszemu spiewakowi na Neginot piesn Dawidowa.
\par 2 Wysluchaj, o Boze! wolanie moje, miej pozór na modlitwe moje.
\par 3 Od konca ziemi wolam do ciebie w zatrwozeniu serca mego; wprowadz mie na skale, która jest wywyza nad mie.
\par 4 Albowiemes ty byl ucieczka moja, i baszta mocna przed twarza nieprzyjaciela.
\par 5 Bede mieszkal w przybytku twoim na wieki, schraniajac sie pod zaslone skrzydel twoich. Sela.
\par 6 Albowiemes ty, Boze! wysluchal zadosci moje; tys dal dziedzictwo tym, którzy sie boja imienia twego.
\par 7 Dni do dni królewskich przydaj; niech beda lata jego od narodu do narodu.
\par 8 Niech mieszka na wieki przed obliczem Bozem; zgotuj milosierdzie i prawde, niech go strzega.
\par 9 Tak bede spiewal imieniowi twemu na wieki, a sluby moje oddawac bede na kazdy dzien.

\chapter{62}

\par 1 Przedniejszemu spiewakowi Jedytunowi psalm Dawidowy.
\par 2 Tylko na Boga spolega dusza moja, od niegoc jest zbawienie moje.
\par 3 Tylkoc on jest skala moja i wybawieniem mojem, twierdza moja; przeto sie bardzo nie zachwieje.
\par 4 Dokadze bedziecie myslic zle przeciwko czlowiekowi? Wszyscy wy zabici bedziecie; bedziecie jako sciana pochylona, a jako mur walacy sie.
\par 5 Przeciez jednak radza, jakoby go zepchnac z dostojenstwa jego; kochaja sie w klamstwie, usty swemi dobrorzecza, ale w sercu swem zlorzecza. Sela.
\par 6 Ty przeciez na Bogu spolegaj, duszo moja! bo od niego jest oczekiwanie moje.
\par 7 Onci sam jest skala moja zbawieniem mojem, i twierdza moja; przetoz nie zachwieje sie.
\par 8 W Bogu wybawienie moje, i chwala moja skala mocy mojej; nadzieja moja jest w Bogu.
\par 9 Ufajciez w nim na kazdy czas, o narody! Wylewajcie przed obliczem jego serca wasze: Bóg jest ucieczka nasza. Sela.
\par 10 Zaprawdec marnoscia sa synowie ludzcy, klamliwi synowie mocarzy; bedali pospolu wlozeni na wage, lekciejszymi beda nad marnosc.
\par 11 Nie ufajciez w krzywdzie ani w drapiestwie, a nie bedzcie marnymi; przybedzieli wam majetnosci, nie przykladajciez serca do nich.
\par 12 Razci rzekl Bóg, dwakrociem to slyszal, iz moc jest Boza,
\par 13 A ze Panie! twoje jest milosierdzie, a ze ty oddasz kazdemu wedlug uczynków jego.

\chapter{63}

\par 1 Psalm Dawidowy, gdy byl na puszczy Judzkiej.
\par 2 Boze! tys jest Bogiem moim; z poranku cie szukam; pragnie cie dusza moja, teskni po tobie cialo moje w ziemi suchej i upragnionej, w której nie masz wody;
\par 3 Abym cie ogladal w swiatnicy twojej, i widzial moc twoje i chwale twoje.
\par 4 (Albowiem lepsze jest milosierdzie twoje, niz zywot,)aby cie chwalily wargi moje,
\par 5 Abym cie blogoslawil za zywota mego, a w imieniu twojem abym podnosil rece moje.
\par 6 Jako tlustoscia i sadlem bylaby tu nasycona dusza moja, a radosnem warg spiewaniem wychwalalyby cie usta moje.
\par 7 Zaprawdec na cie wspominam, i na lozu mojem kazdej strazy nocnej rozmyslam o tobie.
\par 8 Bos mi bywal na pomocy; przetoz w cieniu skrzydel twoich spiewac bede.
\par 9 Przylgnela dusza moja do ciebie; prawica twoja podpiera mie.
\par 10 Ale ci, którzy szukaja upadku duszy mojej, sami wnijda do najglebszej niskosci ziemi.
\par 11 Zabije kazdego z nich ostrosc miecza, i przyjda liszkom na podzial.
\par 12 Lecz król bedzie sie weselil w Bogu, a bedzie sie chlubil kazdy, kto przezen przysiega; albowiem zatkane beda usta mówiacych klamstwo.

\chapter{64}

\par 1 Przedniejszemu spiewakowi psalm Dawidowy.
\par 2 Wysluchaj, o Boze! glos mój, gdy sie modle; od strachu nieprzyjaciela strzez zywot mój.
\par 3 Skryj mie przed skryta rada zlosników, przed zbuntowaniem czyniacych nieprawosc.
\par 4 Którzy zaostrzyli jezyk swój jako miecz, nalozyli strzale swoje, slowo jadowite,
\par 5 Aby strzelali z skrytosci na niewinnego; niespodzianie nan strzelaja, a nikogo sie nie boja.
\par 6 Stwierdzaja sie w zlem; zmawiaja sie, jakoby zakryc sidla, i mówia: Któz je obaczy?
\par 7 Szukaja pilnie nieprawosci; giniemy od rad zdradliwie wynalezionych. Takci wnetrznosc i serce czlowiecze glebokie jest.
\par 8 Ale gdy Bóg na nich wypusci predka strzale, porazeni beda;
\par 9 A do upadku przywiedzie ich wlasny jezyk ich; odlaczy sie od nich kazdy, kto ich ujrzy.
\par 10 I ulekna sie wszyscy ludzie, a beda opowiadali sprawe Boza, i dzielo jego zrozumieja.
\par 11 Ale sprawiedliwy sie bedzie weselil w Panu, a bedzie w nim ufal; i beda sie chlubili wszyscy, którzy sa uprzejmego serca.

\chapter{65}

\par 1 Przedniejszemu spiewakowi psalm i piesn Dawidowa.
\par 2 Tobie przynalezy, o Boze! chwala na Syonie, a tobie slub ma byc oddany.
\par 3 Ty wysluchiwasz modlitwy; przetoz do ciebie przychodzi wszelkie cialo.
\par 4 Wielkie nieprawosci, które wziely góre nad nami, i przestepstwa nasze ty oczyszczasz.
\par 5 Blogoslawiony, kogo ty obierasz a przyjmujesz, aby mieszkal w sieniach twoich; bedziemy nasyceni dobrami domu twego, w swiatnicy kosciola twego.
\par 6 Przedziwne rzeczy podlug sprawiedliwosci mówisz do nas, Boze zbawienia naszego, nadziejo wszystkich krajów ziemi, i morza dalekiego!
\par 7 Który utwierdzasz góry moca swoja, sila przepasany bedac;
\par 8 Który usmierzasz szum morski, szum nawalnosci jego, i wzruszenie narodów,
\par 9 Tak, ze sie bac musza cudów twoich, którzy mieszkaja na krajach ziemi; których nastawaniem poranku i wieczora do wesela pobudzasz.
\par 10 Nawiedzasz ziemie, i odwilzasz ja; obficie ja ubogacasz strumieniem Bozym, napelnionym wodami, i gotujesz zboze ich, gdy ja tak przyprawiasz.
\par 11 Zagony jej napawasz, bruzdy jej znizasz, dzdzami ja odmiekczasz, a urodzajom jej blogoslawisz.
\par 12 Koronujesz rok dobrocia twa, a sciezki twoje skrapiasz tlustoscia.
\par 13 Skrapiasz pastwiska na pustyniach; tak, ze i pagórki radoscia przepasane bywaja.
\par 14 Przyodziewaja sie pola stadami owiec, a doliny okrywaja sie zbozem; tak, ze wykrzykaja i spiewaja.

\chapter{66}

\par 1 Przedniejszemu spiewakowi piesn psalmu.
\par 2 Wykrzykuj Bogu wszystka ziemo! Spiewajciez psalmy na chwale imienia jego, oglaszajcie slawe i chwale jego.
\par 3 Rzeczciez Bogu: Jakos straszny w sprawach twoich! Dla wielkosci mocy twojej obludniec sie podadza nieprzyjaciele twoi.
\par 4 Wszystkac sie ziemia klaniac, i spiewac ci bedzie; psalm spiewac bedzie imieniowi twemu. Sela.
\par 5 Pójdzciez, a ogladajcie sprawy Boze; straszny jest w sprawach swoich przy synach ludzkich.
\par 6 Obrócil morze w ziemie sucha; rzeke przeszli sucha noga; tamesmyc sie weselili w nim.
\par 7 Panuje w mocy swej na wieki; oczy jego patrza na narody, odporni nie wywyzsza sie. Sela.
\par 8 Blogoslawciez narody Boga naszego, i oglaszajcie glos chwaly jego.
\par 9 Zachowal przy zdrowiu dusze nasze, a nie dal sie powinac nodze naszej.
\par 10 Albowiemes nas doswiadczyl, o Boze! wyplawiles nas ogniem, tak jako srebro plawione bywa.
\par 11 Nagnales nas byl w siec, a scisnales uciskiem biodra nasze.
\par 12 Wsadziles czlowieka na glowe nasze; weszlismy byli w ogien i w wode, wszakzes nas wywiódl na ochlode.
\par 13 Przetoz wnijde do domu twego z calopaleniem, a oddam ci sluby moje.
\par 14 Którec slubowaly wargi moje, i wyrzekly usta moje w utrapieniu mojem.
\par 15 Calopalenie z tlustych baranów bedec ofiarowal z kadzeniem, bedec ofiarowal woly i kozly. Sela.
\par 16 Pójdzcie, sluchajcie, a bede opowiadal wszystkim, którzy sie boicie Boga, co uczynil duszy mojej.
\par 17 Do niegom usty swemi wolal, a wywyzszalem go jezykiem moim.
\par 18 Bym byl patrzal na nieprawosc w sercu mojem, nie wysluchalby byl Pan.
\par 19 Alec zaiste wysluchal Bóg, a byl pilen glosu modlitwy mojej.
\par 20 Blogoslawiony Bóg, który nie odrzucil modlitwy mojej, ani odjal milosierdzia swego odemnie.

\chapter{67}

\par 1 Przedniejszemu spiewakowi na Neginot psalm ku spiewaniu.
\par 2 Boze! zmiluj sie nad nami, a blogoslaw nam, rozswiec oblicze twoje nad nami. Sela.
\par 3 Aby tak poznali na ziemi droge twoje, a po wszystkich narodach zbawienie twoje.
\par 4 Tedy cie beda wyslawialy narody o Boze! Beda cie wyslawiac wszyscy ludzie!
\par 5 Radowac sie beda i wykrzykac narody; bo ty bedziesz sadzil ludzi w sprawiedliwosci, a narody bedziesz sprawowal na ziemi. Sela.
\par 6 Beda cie wyslawiac narody, o Boze! Beda cie wyslawiac wszyscy ludzie.
\par 7 Ziemia takze wyda urodzaj swój; niech nam blogoslawi Bóg, Bóg nasz.
\par 8 Niech nam blogoslawi Bóg, a niech sie go boja wszystkie kraje ziemi.

\chapter{68}

\par 1 Przedniejszemu spiewakowi psalm Dawidowy ku spiewaniu.
\par 2 Powstanie Bóg, a beda rozproszeni nieprzyjaciele jego, i pouciekaja przed twarza jego ci, którzy go maja w nienawisci.
\par 3 Jako bywa dym rozpedzony, tak ich rozpedzasz: jako sie wosk rozplywa od ognia, tak niezboznicy pogina przed obliczem Bozem.
\par 4 Ale sprawiedliwi weselic sie i radowac beda przed obliczem Bozem, i plasac beda od radosci.
\par 5 Spiewajcie Bogu, spiewajcie psalmy imieniowi jego; gotujcie droge temu, który jezdzi na oblokach. Pan jest imie jego, radujciez sie przed obliczem jego.
\par 6 Ojcem jest sierót, i sedzia wdów, Bogiem w przybytku swym swietym.
\par 7 Bóg, który samotne w rodowite domy rozmnaza, wywodzi wiezniów z oków; ale odporni mieszkac musza w ziemi suchej.
\par 8 Boze! gdys wychodzil przed obliczem ludu twego, gdys chodzil po puszczy; Sela,
\par 9 Ziemia sie trzesla, takze i niebiosa rozplywaly sie przed obliczem Bozem, i ta góra Synaj drzala przed twarza Boga, Boga Izraelskiego.
\par 10 Deszcz obfity spuszczales hojnie, o Boze! na dziedzictwo twoje, a gdy omdlewalo, tys je zas otrzezwial.
\par 11 Zastepy twoje mieszkaja w niem, któres ty dla ubogiego nagotowal dobrocia twoja, o Boze!
\par 12 Pan dal slowo swe, i tych, którzy pociechy zwiastowali, zastep wielki mówiacych.
\par 13 Królowie z wojskami uciekali, uciekali: ale ta, która przygladala domu, dzielila lupy.
\par 14 Chociazescie lezec musieli miedzy kotlami, przeciez bedziecie jako golebica, majaca pióra posrebrzone, a której skrzydla jako zólte zloto.
\par 15 Gdy Wszechmogacy rozproszy królów w tej ziemi, wybielejesz jako snieg na górze Salmon.
\par 16 Na górze Bozej, na górze Basanskiej, na górze pogórczystej, na górze Basanskiej.
\par 17 Przeczze wyskakujecie góry pogórczyste? na tejci górze ulubil sobie Bóg mieszkanie, tamci Pan bedzie mieszkal na wieki.
\par 18 Wozów Bozych jest dwadziescia tysiecy, wiele tysiecy Aniolów; ale Pan miedzy nimi jako na Synaj w swiatnicy przebywa.
\par 19 Wstapiles na wysokosc, wiodles pojmanych wiezniów, nabrales darów dla ludzi, i najodporniejszych, Panie Boze! przywiodles, aby mieszkali z nami.
\par 20 Blogoslawiony Pan; na kazdy dzien hojnie nas opatruje dobrami swemi Bóg zbawienia naszego. Sela.
\par 21 On jest Bóg nasz, Bóg obfitego zbawienia; panujacy Pan z smierci wywodzi.
\par 22 Zaiste Bóg zrani glowe nieprzyjaciól swoich, i wierzch glowy wlosami nakryty chodzacego w grzechach swoich.
\par 23 Rzekl Pan: Wyprowadze zas swoich jako z Basan, wywiode ich zas z glebokosci morskiej.
\par 24 Przetoz bedzie noga twoja zbroczona we krwi, i jezyk psów twoich we krwi nieprzyjacielskiej.
\par 25 Widzieli ciagnienia twoje, Boze! ciagnienia Boga mego i króla mego w swiatnicy.
\par 26 Wprzód szli spiewacy, a za nimi grajacy na instrumentach, a w posrodku panienki bijac w bebny.
\par 27 W zgromadzeniach blogoslawcie Bogu, blogoslawcie Panu, którzyscie z narodu Izraelskiego. Tu niech bedzie Benjamin maluczki, który ich opanowal;
\par 28 Tu ksiazeta Judzcy, i hufy ich, ksiazeta Zabulonscy, i ksiazeta Neftalimscy.
\par 29 Obdarzyl cie Bóg twój sila; utwierdz, o Boze! to, cos w nas sprawil.
\par 30 Dla kosciola twego, który jest w Jeruzalemie, bedac królowie dary przynosic.
\par 31 Poraz poczet kopijników, zgromadzenie mocnych wodzów, i ludu bujnego, hardych, chlubiacych sie kesem srebra; rozprosz narody pragnace wojny.
\par 32 Przyjdac zacni ksiazeta z Egiptu: Murzynska ziemia pospieszy sie wyciagnac rece swe do Boga.
\par 33 Królestwa ziemi! spiewajciez Bogu, spiewajcie Panu. Sela.
\par 34 Temu, który jezdzi na najwyzszych niebiosach od wiecznosci; oto wydaje glos swój, glos mocy swojej.
\par 35 Przyznajcie moc Bogu, nad Izraelem dostojnosc jego, a wielmoznosc jego na oblokach.
\par 36 Strasznys jest, o Boze! z swietych przybytków twoich; Bóg Izraelski sam daje moc i sily ludowi swemu. Niechajze bedzie Bóg blogoslawiony.

\chapter{69}

\par 1 Przedniejszemu spiewakowi na Sosannim psalm Dawidowy.
\par 2 Wybaw mie, o Boze! boc przyszly wody az do duszy mojej.
\par 3 Pograzony jestem w glebokiem blocie, gdzie dna niemasz; przyszedlem w glebokosci wód, a nawalnosc ich porwala mie.
\par 4 Spracowalem sie wolajac, wyschlo gardlo moje; ustaly oczy moje, gdym oczekiwal Boga mojego.
\par 5 Wiecej jest tych, którzy mie maja w nienawisci bez przyczyny, niz wlosów na glowie mojej; zmocnili sie ci, którzy mie wygubic usiluja, a sa nieprzyjaciólmi mymi nieslusznie; czegom nie wydarl, musialem nagradzac.
\par 6 Boze! ty znasz glupstwo moje, a wystepki moje nie sa tajne przed toba.
\par 7 Niechajze nie beda zawstydzeni dla mnie ci, którzy na cie oczekuja, Panie, Panie zastepów! niech nie przychodza dla mnie do hanby ci, którzy cie szukaja, o Boze Izraelski!
\par 8 Bo dla ciebie ponosze uraganie, a zelzywosc okryla oblicze moje.
\par 9 Stalem sie obcym braciom moim, a cudzoziemcem synom matki mojej,
\par 10 Przeto, ze gorliwosc domu twego zzarla mie, a uraganie uragajacych tobie przypadlo na mie.
\par 11 Gdym plakal i trapil postem dusze moje, stalo mi sie to pohanbienie.
\par 12 Gdym wzial na sie wór miasto szaty, bylem u nich przypowiescia.
\par 13 Mówili o mnie ci, którzy siedzieli w bramie, a bylem piosnka u tych, którzy pili mocny napój.
\par 14 Ale jaobracam modlitwe moje do ciebie, Panie! czas jest upodobania twego; o Boze! wedlug wielkosci milosierdzia twego wysluchajze mie, dla prawdy zbawienia twego.
\par 15 Wyrwij mie z blota, abym nie byl pograzony; niech bede wyrwany od tych, którzy mie nienawidza, jako z glebokosci wód;
\par 16 Aby mie nie zatopily strumienie wód, i nie pozarla glebia i nie zawarla nademna studnia wierzchu swego.
\par 17 Wysluchajze mie, Panie! boc dobre jest milosierdzie twoje; wedlug wielkiej litosci twojej wejrzyj na mie.
\par 18 Nie zakrywajze oblicza twego od slugi swego, bom jest w utrapieniu; pospieszze sie, wysluchaj mie.
\par 19 Przybliz sie do duszy mojej, a wybaw ja; dla nieprzyjaciól moich odkup mie.
\par 20 Ty znasz pohanbienie moje, i zelzywosc moje, i wstyd mój: przed tobac sa wszyscy nieprzyjaciele moi.
\par 21 Pohanbienie pokruszylo serce moje, z czegom byl zalosny; oczekiwalem, azaliby sie mie kto uzalil, ale nikt nie byl; azaliby mie kto pocieszyl, alem nie znalazl.
\par 22 Owszem, miasto pokarmu podali mi zólc, a w pragnieniu mojem napoili mie octem.
\par 23 Niechajze im bedzie stól ich przed nimi sidlem, a szczescie ich na upadek.
\par 24 Niech sie zacmia oczy ich, aby nie widzieli, a biodra ich niech sie zawzdy chwieja.
\par 25 Wylij na nich rozgniewanie swoje, a popedliwosc gniewu twego niech ich ogarnie.
\par 26 Niech bedzie mieszkanie ich puste, w namiotach ich niech nikt nie mieszka.
\par 27 Bo tego, któregos ty ubil, przesladuja, a o bolesci poranionych twoich rozmawiaja.
\par 28 Przydajze nieprawosc ku nieprawosci ich, a niech nie przychodza do sprawiedliwosci twojej.
\par 29 Niech beda wymazani z ksiag zyjacych, a z sprawiedliwymi niech nie beda zapisani.
\par 30 Jamci utrapiony, i zbolaly; lecz zbawienie twoje, Boze! na miejscu bezpiecznem postawi mie.
\par 31 Tedy bede chwalil imie Boze piesnia, a bede je wielbil z dziekczynieniem.
\par 32 A bedzie to przyjemniejsze Panu, nizeli wól albo cielec rogaty z rozdzielonemi kopytami.
\par 33 To widzac pokorni rozraduja sie, szukajac Boga, a ozyje serce ich;
\par 34 Iz wysluchiwa Pan ubogich, a wiezniami swymi nie gardzi.
\par 35 Niech go chwala niebiosa i ziemia, morze i wszystko, co sie w nich rucha.
\par 36 Bógci zaiste zachowa Syon, i pobuduje miasta Judzkie; i beda tam mieszkac, a ziemie te dziedzicznie otrzymaja.
\par 37 Takze i nasienie slug jego dziedzicznie ja otrzyma, a którzy miluja imie jego, beda w niej mieszkac.

\chapter{70}

\par 1 Przedniejszemu spiewakowi psalm Dawidowy na wspominamie.
\par 2 Boze! pospiesz sie, abys mie wyrwal; Panie! pospiesz sie, abys mi dal ratunek.
\par 3 Niech beda zawstydzeni i pohanbieni, którzy szukaja duszy mojej; niech sie obróca na wstecz, i niech beda pohanbieni, którzy mi zlego zycza.
\par 4 Niech sie obróca nazad za to, ze mie chca pohanbic ci, którzy mi mówia: Ehej, ehej!
\par 5 Ale niech sie wesela i raduja w tobie wszyscy, którzy cie szukaja, a którzy miluja zbawienie twoje, niech mówia zawzdy: Uwielbiony badz, Boze nasz!
\par 6 Jamci nedzny i ubogi; o Boze! pospiesz sie ku mnie; tys jest pomoca moja, i wybawicielem moim! Panie! nie omieszkujze.

\chapter{71}

\par 1 W tobie, Panie! nadzieje mam: niech na wieki pohanbiony nie bede.
\par 2 Wedlug sprawiedliwosci twej wybaw mie, i wyrwij mie; naklon ku mnie ucha twego, i zachowaj mie.
\par 3 Badz mi skala mieszkania, gdziebym zawzdy uchodzil; przykazales, aby mie strzezono; bos ty skala moja i twierdza moja.
\par 4 Boze mój! wyrwij mie z reki niezboznika, z reki przewrotnego i gwaltownika;
\par 5 Albowiemes ty oczekiwaniem mojem, Panie! Panie! nadziejo moja od mlodosci mojej.
\par 6 Na tobiem spolegl zaraz z zywota; tys mie wywiódl z zywota matki mojej; w tobie chwala moja zawzdy.
\par 7 Jako dziwowisko bylem u wielu; wszakze tys jest mocna nadzieja moja.
\par 8 Niechaj beda napelnione usta moje chwala twoja, przez caly dzien slawa twoja.
\par 9 Nie odrzucajze mie w starosci mojej; gdy ustanie sila moja, nie opuszczaj mie.
\par 10 Albowiem mówili nieprzyjaciele moi przeciwko mnie, a ci, którzy czyhali na dusze moje, rade uczynili spolem,
\par 11 Mówiac: Bóg go opuscil, gonciez go, a pojmijcie go; boc nie ma, ktoby go wyrwal.
\par 12 Boze! nie oddalajze sie odemnie; o Boze mój! pospieszze sie na ratunek mój.
\par 13 Niechze beda zawstydzeni, niech zgina przeciwnicy duszy mojej; niech beda okryci zelzywoscia i wstydem, którzy mi szukaja zlego.
\par 14 Alec ja zawzdy oczekiwac bede, a tem wiecej rozszerze chwale twoje.
\par 15 Usta moje opowiadac beda sprawiedliwosc twoje, caly dzien zbawienie twoje, aczkolwiek liczby jego nie wiem.
\par 16 Przystapie do wyslawiania wszelakiej mocy Pana panujacego, bede wspominal wlasna sprawiedliwosc twoje.
\par 17 Boze! uczyles mie od mlodosci mojej, i opowiadam az po dzis dzien dziwne sprawy twoje.
\par 18 A przetoz az do starosci i sedziwosci nie opuszczaj mie, Boze! az opowiem ramie twoje temu narodowi, i wszystkim potomkom moc twoje.
\par 19 Bo sprawiedliwosc twoja, Boze! wywyzszona jest, czynisz zaiste rzeczy wielkie. Boze! któz jest podobien tobie?
\par 20 Który, aczes przypuscil na mie wielkie i ciezkie uciski, wszakze zasie do zywota przywracasz mie, a z przepasci ziemskich zasie wywodzisz mie.
\par 21 Rozmnozysz dostojnosc moje a zasie ucieszysz mie.
\par 22 A ja tez wyslawiac cie bede na instrumentach muzycznych, i prawde twoje, Boze mój! bedec spiewal przy harfie, o Swiety Izraelski!
\par 23 Rozraduja sie wargi moje, gdyc bede spiewal, i dusza moja, któras wykupil.
\par 24 Nadto i jezyk mój bedzie opowiadal przez caly dzien sprawiedliwosc twoje; bo sie zawstydzic, i hanbe odniesc musieli ci, którzy szukali nieszczescia mego.

\chapter{72}

\par 1 Salomonowi. Boze! daj królowi sady twoje, a sprawiedliwosc twoje synowi królewskiemu;
\par 2 Aby sadzil lud twój w sprawiedliwosci, a ubogich twoich w prawosci.
\par 3 Przyniosa góry ludowi pokój, a pagórki sprawiedliwosc.
\par 4 Bedzie sadzil ubogich z ludu, a wybawi synów ubogiego; ale gwaltownika pokruszy.
\par 5 Beda sie bac ciebie, póki slonce i miesiac trwac bedzie, od narodu az do narodu.
\par 6 Jako zstepuje deszcz na pokoszona trawe, a deszcz kroplisty skrapiajacy ziemie:
\par 7 Tak sprawiedliwy zakwitnie za dni jego, a bedzie obfitosc pokoju, dokad miesiaca staje.
\par 8 Bedzie panowal od morza az do morza, i od rzeki az do konczyn ziemi.
\par 9 Przed nim padac beda mieszkajacy na pustyniach, a nieprzyjaciele jego proch lizac beda.
\par 10 Królowie od morza i z wysep dary mu przyniosa; królowie Sebejscy i Sabejscy upominki oddadza.
\par 11 I beda mu sie klaniac wszyscy królowie; wszystkie narody sluzyc mu beda.
\par 12 Albowiem wyrwie ubogiego wolajacego, i nedznego, który nie ma pomocnika.
\par 13 Zmiluje sie nad ubogim, i nad niedostatecznym, a dusze nedznych wybawi.
\par 14 Od zdrady i gwaltu wybawi dusze ich; bo droga jest krew ich przed oczyma jego.
\par 15 I bedzie zyl, a dawac mu beda zloto sabejskie, i ustawicznie sie za nim modlic beda, caly dzien blogoslawic mu beda.
\par 16 Gdy sie wrzuci garsc zboza do ziemi na wierzchu gór, zaszumi jako Liban urodzaj jego, a mieszczanie zakwitna jako ziola polne.
\par 17 Imie jego bedzie na wieki; pokad slonce trwa, dziedziczyc bedzie imie jego, a blogoslawiac sobie w nim wszystkie narody wielbic go beda.
\par 18 Blogoslawiony Pan Bóg, Bóg Izraelski, który sam cuda czyni.
\par 19 I blogoslawione imie chwaly jego na wieki, a niech bedzie napelniona chwala jego wszystka ziemia. Amen, Amen.
\par 20 A tuc sie koncza modlitwy Dawida, syna Isajego.

\chapter{73}

\par 1 Psalm Asafowy. Zaistec dobry jest Bóg Izraelowi, tym, którzy sa czystego serca.
\par 2 Ale nogi moje malo sie byly nie potknely, a blisko tego bylo, ze malo nie szwankowaly kroki moje,
\par 3 Gdym byl zawiscia poruszon przeciwko szalonym, widzac szczescie niepoboznych.
\par 4 Bo nie maja zwiazków az do smierci, ale w calosci zostaje sila ich.
\par 5 W pracy ludzkiej nie sa, a kazni, jako inni ludzie, nie doznawaja.
\par 6 Przetoz otoczeni sa pycha, jako lancuchem zlotym, a przyodziani okrutnoscia, jako szata ozdobna.
\par 7 Wystapily od tlustosci oczy ich, a wiecej maja nad pomyslenie serca.
\par 8 Rozpuscili sie, i mówia zlosliwie, o ucisnieniu bardzo hardzie mówia.
\par 9 Wystawiaja przeciwko niebu usta swe, a jezyk ich krazy po ziemi.
\par 10 A przetoz na to przychodzi lud jego, gdy sie im wody juz wierzchem leja,
\par 11 Ze mówia:Jakoz ma Bóg o tem wiedziec? albo mali o tem wiadomosc Najwyzszy?
\par 12 Albowiem, oto ci niezboznymi bedac, maja pokój na swiecie, i nabywaja bogactw.
\par 13 Prózno tedy w czystosci chowam rece moje, a w niewinnosci serce moje omywam.
\par 14 Poniewaz mie caly dzien bija, a karanie cierpie na kazdy poranek.
\par 15 Rzekeli: Bede tez tak o tem mówil, tedy rodzaj synów twoich rzecze, zem im niepraw.
\par 16 Chcialemci tego rozumem doscignac, ale mi sie tu trudno zdalo;
\par 17 Azem wszedl do swiatnicy Bozej, a tum porozumial dokonczenie ich.
\par 18 Zprawdes ich na miejscach sliskich postawil, a podajesz ich na spustoszenie.
\par 19 Oto jakoc przychodza na spustoszenie! niemal w okamgnieniu niszczeja i gina od strachu.
\par 20 Sa jako sen temu, co ocucil; Panie! gdy ich obudzisz obraz ich lekce powazysz.
\par 21 Gdy zgorzklo serce moje, a nerki moje cierpialy klucie:
\par 22 Zglupialem byl, a nicem nie rozumial, bylem przed toba jako bydle.
\par 23 A wszakze zawzdy bylem z toba; bos mie trzymal za prawa reke moje.
\par 24 Wedlug rady swej prowadz mie, a potem do chwaly przyjmiesz mie.
\par 25 Kogozbym innego mial na niebie? I na ziemi oprócz ciebie w nikim innym upodobania nie mam.
\par 26 Choc cialo moje, i serce moje ustanie, jednak Bóg jest skala serca mego, i dzialem moim na wieki.
\par 27 Gdyz oto ci, którzy sie oddalaja od ciebie, zgina; wytracasz tych, którzy cudzoloza odstepowaniem od ciebie.
\par 28 Alec mnie najlepsza jest trzymac sie Boga; przetoz pokladam w Panu panujacym nadzieje moje, abym opowiadal wszystkie sprawy jego.

\chapter{74}

\par 1 Piesn wyuczajaca, podana Asafowi. Przeczzes nas, o Boze! do konca odrzucil? Przeczze sie rozpalila zapalczywosc twoja przeciwko owcom pastwiska twego?
\par 2 Wspomnij na zgromadzenie twoje, któres sobie zdawna nabyl i odkupil, na pret dziedzictwa twego, na te góre Syon, na której mieszkasz.
\par 3 Pospieszze sie na srogie popustoszenie; a jako wszystko poburzyl nieprzyjaciel w swiatnicy!
\par 4 Ryczeli nieprzyjaciele twoi w posrodku zgromadzenia twego, a na znak tego zostawili wiele choragwi swoich.
\par 5 Za rycerza miano tego, który sie z wysoka z siekiera zanosil, rabiac drzewo wiazania jego.
\par 6 A teraz juz i rzezania jego na porzad siekierami i mlotami tluka.
\par 7 Zalozyli ogien w swiatnicy twojej, a obaliwszy na ziemie, splugawili przybytek imienia twego.
\par 8 Mówili w sercu swojem: Zburzmy je pospolu; popalili wszystkie przybytki Boze w ziemi.
\par 9 Znaków naszych nie widzimy:juz niemasz proroka, i niemasz miedzy nami, któryby wiedzial, póki to ma trwac.
\par 10 Dokadze, o Boze! przeciwnik bedzie uragac? izali nieprzyjaciel bedzie bluznil imie twoje az na wieki?
\par 11 Przeczze zstrzymujesz reke twoje; a prawicy swej z zanadrza swego cale nie dobedziesz?
\par 12 Wszakes ty, Boze! zdawna królem moim; ty sprawujesz hojne zbawienie w posród ziemi.
\par 13 Tys moca twoja rozdzielil morze, a potarles glowy wielorybów w wodach.
\par 14 Tys skruszyl glowe Lewiatana, dales go za pokarm ludowi na puszczy.
\par 15 Tys przerwal zródla i potoki; tys osuszyl rzeki bystre.
\par 16 Twójci jest dzien, twoja tez i noc; tys uczynil swiatlo i slonce.
\par 17 Tys zalozyl wszystkie granice ziemi; lato i zime tys sprawil.
\par 18 Wspomnijze na to, ze nieprzyjaciel zelzyl Pana, a lud szalony jako uraga imieniowi twemu.
\par 19 Nie podawajze tej kupie duszy synogarlicy twojej; na stadko ubogich twoich nie zapominaj na wieki.
\par 20 Obejrzyj sie na przymierze twoje; albowiem i najciemniejsze katy ziemi pelne jaskin drapiestwa.
\par 21 Niechajze nedznik nie odchodzi z hanba; ubogi i zebrak niechaj chwali imie twoje.
\par 22 Powstanze, o Boze! ujmij sie o sprawe twoje; wspomnij na pohanbienie twoje, które sie dzieje od szalonych na kazdy dzien.
\par 23 Nie zapominajze wykrzykania nieprzyjaciól twoich, i huku tych, co przeciwko tobie powstawaja, który sie ustawicznie sili.

\chapter{75}

\par 1 Przedniejszemu spiewakowi, jako: Nie zatracaj, psalm i piesn Asafowa.
\par 2 Wyslawiamy cie, Boze! wyslawiamy: bo bliskie imie twoje; opowiadaja to dziwne sprawy twoje.
\par 3 Gdy przyjdzie czas ulozony, ja sprawiedliwie sadzic bede.
\par 4 Rozstapila sie ziemia, i wszyscy obywatele jej; ale ja utwierdze slupy jej. Sela.
\par 5 Rzeke szalonym: Nie szalejcie, a niepoboznym: Nie podnoscie rogów.
\par 6 Nie podnoscie przeciwko Najwyzszemu rogów swych, a nie mówcie krnabrnie,
\par 7 Bo nie od wschodu, ani od zachodu, ani od puszczy przychodzi wywyzszenie.
\par 8 Ale Bóg sedzia, tego poniza, a owego wywyzsza.
\par 9 Zaiste kielich jest w rekach Panskich, a ten wina metnego nalany; z tegoz nalewac bedzie, tak, ze i drozdze jego wyssa i wypija wszyscy niepobozni ziemi.
\par 10 Ale ja bede opowiadal sprawy Panskie na wieki, bede spiewal Bogu Jakóbowemu.
\par 11 A wszystkie rogi niezboznikom postracam; ale rogi sprawiedliwego beda wywyzszone.

\chapter{76}

\par 1 Przedniejszemu spiewakowi na Neginot, psalm i piesn Asafowi.
\par 2 Znajomy jest Bóg w Judzkiej ziemi, w Izraelu wielkie imie jego.
\par 3 W Salemie jest przybytek jego, a mieszkanie jego na Syonie.
\par 4 Tamci polamal ogniste strzaly luków, tarcze, i miecz, i wojne. Sela.
\par 5 Zacnymes sie stal i dostojnym z gór lupiestwa.
\par 6 Ci, którzy byli serca meznego, podani sa na lup, zasneli snem swoim, nie znalezli mezni rycerze sily w rekach swych.
\par 7 Od gromienia twego, o Boze Jakóbowy! twardo zasnely i wozy i konie.
\par 8 Tys jest, ty bardzo straszliwy; i któz jest, coby sie ostal przed obliczem twojem, gdy sie zapali gniew twój?
\par 9 Gdy z nieba dajesz slyszec sad swój, ziemia sie leka i ucicha;
\par 10 Gdy Bóg na sad powstaje, aby wybawil wszystkich pokornych na ziemi. Sela.
\par 11 Zaiste i gniew czlowieczy chwalic cie musi, a ty ostatek zagniewania skrócisz.
\par 12 Sluby czyncie, a oddawajcie je Panu, Bogu waszemu, wszyscy, którzyscie okolo niego, wszyscy przynoscie dary strasznemu.
\par 13 Onci odejmuje ducha ksiazetom, a on jest na postrach królom ziemskim.

\chapter{77}

\par 1 Przedniejszemu spiewakowi dla Jedytuna psalm Asafowy.
\par 2 Glos mój podnosze do Boga, kiedy wolam; glos mój podnosze do Boga, aby mie wysluchal.
\par 3 W dzien utrapienia mego szukalem Pana: wyciagalem w nocy rece moje bez przestania, a nie dala sie ucieszyc dusza moja.
\par 4 Wspominalem na Boga, a trwozylem soba; rozmyslalem, a utrapieniem scisniony byl duch mój. Sela.
\par 5 Zatrzymywales oczy moje, aby czuly; potartym byl, azem nie mógl mówic.
\par 6 Przychodzily mi na pamiec dni przeszle i lata dawne.
\par 7 Wspominalem sobie na spiewanie moje; w nocym w sercu swem rozmyslal, i wywiadywal sie o tem duch mój, mówiac:
\par 8 Izali mie na wieki odrzuci Pan, a wiecej mi juz laski nie ukaze?
\par 9 Izali do konca ustanie milosierdzie jego, i koniec wezmie slowo od rodzaju az do rodzaju? Izali zapomnial Bóg zmilowac sie?
\par 10 Izali zatrzymal w gniewie litosci swoje? Sela.
\par 11 I rzeklem: Toc jest smierc moja; wszakze prawica Najwyzszego uczyni odmiane.
\par 12 Wspominac sobie bede na sprawy Panskie, a przypominac sobie bede dziwne sprawy twoje, zdawna uczynione.
\par 13 I bede rozmyslal o wszelkiem dziele twojem, i o uczynkach twoich bede mówil:
\par 14 Boze! swieta jest droga twoja. Któryz Bóg jest tak wielki, jako Bóg nasz?
\par 15 Tys jest Bóg, który czynisz cuda; podales do znajomosci miedzy narody moc twoje.
\par 16 Odkupiles ramieniem twojem lud swój, syny Jakóbowe i Józefowe. Sela.
\par 17 Widzialy cie wody, o Boze! widzialy cie wody, i ulekly sie, i wzruszyly sie przepasci.
\par 18 Obloki wydaly powodzi; niebiosa wydaly gromy, a strzaly twoje tam i sam biegaly.
\par 19 Huczalo grzmienie twoje po oblokach, blyskawice oswiecily okrag ziemi, ziemia sie wzruszyla i zatrzesla.
\par 20 Przez morze byla droga twoja, a sciezki twoje przez wody wielkie, wszakze sladów twoich nie bylo.
\par 21 Prowadziles lud twój, jako stado owiec, przez reke Mojzesza i Aarona.

\chapter{78}

\par 1 Piesn wyuczajaca podana Asafowi. Sluchaj, ludu mój! zakonu mego; naklonciez uszów swych do slów ust moich.
\par 2 Otworze w podobienstwie usta moje, a bede opowiadal przypowiastki starodawne.
\par 3 Cosmy slyszeli, i poznali, i co nam ojcowie nasi opowiadali.
\par 4 Nie zataimy tego przed synami ich, którzy przyszlym potomkom swoim opowiadac beda chwaly Panskie, i moc jego, i cuda jego, które uczynil.
\par 5 Bo wzbudzil swiadectwo w Jakóbie, a zakon wydal w Izraelu; przykazal ojcom naszym, aby to do wiadomosci podawali synom swoim,
\par 6 Aby poznal wiek potomny, synowie, którzy sie narodzic mieli, a oni zas powstawszy, aby to opowiadali synom swoim;
\par 7 Aby pokladali w Bogu nadzieje swoje, a nie zapominali na sprawy Boze, ale strzegli przykazan jego;
\par 8 Aby sie nie stali jako ojcowie ich narodem odpornym i nieposlusznym, narodem, który nie wygotowal serca swego, aby byl wierny Bogu duch jego.
\par 9 Albo jako synowie Efraimowi zbrojni, którzy, choc umieli z luku strzelac, wszakze w dzien wojny tyl podali.
\par 10 Bo nie przestrzegali przymierza Bozego, a wedlug zakonu jego zbraniali sie chodzic.
\par 11 Zapomnieli na sprawy jego, i na dziwne dziela jego, które im pokazywal.
\par 12 Przed ojcami ich czynil cuda w ziemi Egipskiej, na polu Soan.
\par 13 Rozdzielil morze, i przeprowadzil ich, i sprawil, ze stanely wody jako kupa.
\par 14 Prowadzil ich w obloku we dnie, a kazdej nocy w jasnym ogniu.
\par 15 Rozszczepil skaly na puszczy, a napoil ich, jako z przepasci wielkich.
\par 16 Wywiódl strumienie ze skaly, a uczynil, ze wody ciekly jako rzeki.
\par 17 A wszakze oni przyczynili grzechów przeciwko niemu, a wzruszyli Najwyzszego na puszczy do gniewu;
\par 18 I kusili Boga w sercu swem, zadajac pokarmu wedlug lubosci swojej.
\par 19 A mówili przeciwko Bogu temi slowy: Izali moze Bóg zgotowac stól na tej puszczy?
\par 20 Oto uderzyl w skale, a wyplynely wody, i rzeki wezbraly; izali tez bedzie mógl dac chleb? Izali nagotuje miesa ludowi swemu?
\par 21 Przetoz uslyszawszy to Pan, rozgniewal sie, a ogien sie zapalil przeciw Jakóbowi, takze i popedliwosc powstala przeciw Izraelowi;
\par 22 Przeto, iz nie wierzyli Bogu, a nie mieli nadziei w zbawieniu jego.
\par 23 Choc byl rozkazal oblokom z góry, i forty niebieskie otworzyl.
\par 24 I spuscil im jako deszcz manne ku pokarmowi, a pszenice niebieska dal im.
\par 25 Chleb mocarzów jadl czlowiek, a zeslal im pokarmów do sytosci.
\par 26 Obrócil wiatr ze wschodu na powietrzu, a przywiódl moca swa wiatr z poludnia;
\par 27 I spuscil na nich mieso jako proch, i ptastwo skrzydlate jako piasek morski;
\par 28 Spuscil je w posród obozu ich, wszedy okolo namiotów ich.
\par 29 I jedli, a nasyceni byli hojnie, i dal im, czego zadali.
\par 30 A gdy jeszcze nie wypelnili zadosci swej, gdy jeszcze pokarm byl w ustach ich:
\par 31 Tedy zapalczywosc Boza przypadla na nich, i pobil tlustych ich, a przedniejszych z Izraela porazil.
\par 32 Ale w tem wszystkiem jeszcze grzeszyli, i nie wierzyli cudom jego;
\par 33 Przetoz sprawil, ze marnie dokonali dni swoich, i lat swoich w strachu.
\par 34 Gdy ich tracil, jezlize go szukali, i nawracali sie, a szukali z rana Boga,
\par 35 Przypominajac sobie, iz Bóg byl skala ich, a Bóg najwyzszy odkupicielem ich:
\par 36 (Aczkolwiek pochlebiali mu usty swemi, i jezykiem swoim klamali mu;
\par 37 A serce ich nie bylo szczere przed nim, ani wiernymi byli w przymierzu jego.)
\par 38 On jednak bedac milosierny odpuszczal nieprawosci ich, a nie zatracal ich, ale czestokroc odwracal gniew swój, a nie pobudzal wszystkiego gniewu swego;
\par 39 Bo pamietal, ze sa cialem, wiatrem, który odchodzi, a nie wraca sie zas.
\par 40 Jako go czesto draznili na puszczy, i do bolesci przywodzili na pustyniach?
\par 41 Bo coraz kusili Boga, a Swietemu Izraelskiemu granice zamierzali.
\par 42 Nie pamietali na reke jego, i na on dzien, w który ich wybawil z utrapienia;
\par 43 Gdy czynil w Egipcie znaki swoje, a cuda swe na polu Soan;
\par 44 Gdy obrócil w krew rzeki ich, i strumienie ich, tak, ze z nich pic nie mogli.
\par 45 Przepuscil na nich rozmaite muchy, aby ich kasaly, i zaby, aby ich gubily:
\par 46 I dal chrzaszczom urodzaje ich, a prace ich szaranczy.
\par 47 Potlukl gradem szczepy ich, a drzewa lesnych fig ich gradem lodowym.
\par 48 I podal gradowi bydlo ich, a majetnosc ich weglu ognistemu.
\par 49 Poslal na nich gniew zapalczywosci swojej, popedliwosc, i rozgniewanie, i ucisnienie, przypusciwszy na nich aniolów zlych.
\par 50 Wyprostowal sciezke gniewowi swemu, nie zachowal od smierci duszy ich, i na bydlo ich powietrze dopuscil;
\par 51 I pobil wszystko pierworodztwo w Egipcie, pierwiastki mocy ich w przybytkach Chamowych;
\par 52 Ale jako owce wyprowadzil lud swój, a wodzil ich jako stada po puszczy.
\par 53 Wodzil ich w bezpieczenstwie, tak, ze sie nie lekali, (a nieprzyjaciól ich okrylo morze,)
\par 54 Az ich przywiódl do swietej granicy swojej, na one góre, której nabyla prawica jego.
\par 55 I wyrzucil przed twarza ich narody, i sprawil, ze im przyszly na sznur dziedzictwa ich, azeby mieszkaly w przybytkach ich pokolenia Izraelskie.
\par 56 A wszakze przeciez kusili i draznili Boga najwyzszego, a swiadectwa jego nie strzegli.
\par 57 Ale sie odwrócili, i przewrotnie sie obchodzili, jako i ojcowie ich; wywrócili sie jako luk omylny.
\par 58 Bo go wzruszyli do gniewu wyzynami swemi, a rytemi balwanami swemi pobudzili go do zapalczywosci.
\par 59 Co slyszac Bóg rozgniewal sie, i zbrzydzil sobie bardzo Izraela,
\par 60 Tak, ze opusciwszy przybytek w Sylo, namiot, który postawil miedzy ludzmi,
\par 61 Podal w niewole moc swoje, i slawe swoje w rece nieprzyjacielskie.
\par 62 Dal pod miecz lud swój, a na dziedzictwo swoje rozgniewal sie.
\par 63 Mlodzienców jego ogien pozarl, a panienki jego nie byly uczczone.
\par 64 Kaplani jego od miecza polegli, a wdowy jego nie plakaly.
\par 65 Lecz potem ocucil sie Pan jako ze snu, jako mocarz wykrzykajacy od wina.
\par 66 I zarazil nieprzyjaciól swoich na posladkach, a na wieczna hanbe podal ich.
\par 67 Ale choc wzgardzil namiotem Józefowym, a pokolenia Efraimowego nie obral,
\par 68 Wszakze obral pokolenie Judowe, i góre Syon, która umilowal.
\par 69 I wystawil sobie jako palac wysoki swiatnice swoje, jako ziemie, która ugruntowal na wieki.
\par 70 I obral Dawida sluge swego, wziawszy go z obór owczych;
\par 71 Gdy chodzil za owcami kotnemi, przyprowadzil go, aby pasl Jakóba, lud jego, i Izraela, dziedzictwo jego;
\par 72 Który ich pasl w szczerosci serca swego, a w roztropnosci rak swoich prowadzil ich.

\chapter{79}

\par 1 Psalm podany Asafowi. O Boze! wtargneli poganie w dziedzictwo twoje, splugawili kosciól twój swiety, obrócili Jeruzalem w kupy gruzu.
\par 2 Dali trupy slug twoich na pokarm ptastwu powietrznemu, ciala swietych twoich bestyjom ziemskim.
\par 3 Wylali krew ich jako wode okolo Jeruzalemu, a nie byl, ktoby ich pochowal.
\par 4 Stalismy sie pohanbieniem u sasiadów naszych; smiechowiskiem i igrzyskiem u tych, którzy sa okolo nas.
\par 5 Dokadze, o Panie? azaz na wieki gniewac sie bedziesz? a jako ogien palac bedzie zapalczywosc twoja?
\par 6 Wylij gniew twój na pogan, którzy cie nie znaja, i na królestwa, które imienia twego nie wzywaja.
\par 7 Albowiemci pozarli Jakóba, a mieszkanie jego spustoszyli.
\par 8 Nie wspominajze nam przeszlych nieprawosci naszych; niech nas rychlo uprzedzi milosierdzie twoje, bosmy bardzo znedzeni.
\par 9 Wspomózze nas, o Boze zbawienia naszego! dla chwaly imienia twego, a wyrwij nas, i badz milosciw grzechom naszym dla imienia twego.
\par 10 Przeczzeby mieli mówic poganie: Gdziez jest Bóg ich? Badz znacznym miedzy poganami, przed oczyma naszemi, dla pomsty krwi slug twoich, która jest wylana.
\par 11 Niech przyjdzie przed oblicze twoje narzekanie wiezniów, a wedlug wielkosci ramienia twego zachowaj ostatki tych, co sa na smierc skazani.
\par 12 A oddaj sasiadom naszym siedmiorako na lono ich za pohanbienie ich, którec uczynili, o Panie!
\par 13 Ale my lud twój i owce pastwiska twego, bedziemy cie wyslawiali na wieki; od narodu do narodu bedziemy opowiadac chwale twoje.

\chapter{80}

\par 1 Przedniejszemu spiewakowi na Sosannim psalm swiadectwa Asafowi.
\par 2 O Pasterzu Izraelski! posluchaj, który prowadzisz Józefa jako stado owiec; który siedzisz na Cherubinach, rozjasnij sie.
\par 3 Wzbudz moc swoje przed Efraimem, i Benjaminem, i Manasesem, a przybadz na wybawienie nasze.
\par 4 O Boze! przywróc nas, a rozjasnij nad nami oblicze twoje, a bedziemy zbawieni.
\par 5 Panie, Boze zastepów! dokadze sie bedziesz gniewal na modlitwe ludu swego?
\par 6 Nakarmiles ich chlebem placzu, i napoiles ich lzami miara wielka.
\par 7 Wystawiles nas na zwade sasiadom naszym; a nieprzyjaciolom naszym, aby sobie z nas smiech stroili.
\par 8 O Boze zastepów; przywróc nas, a rozjasnij nad nami oblicze twoje, a bedziemy zbawieni.
\par 9 Tys macice winna z Egiptu przeniósl; wyrzuciles pogan, a wsadziles ja.
\par 10 Uprzatnales dla niej, i sprawiles, ze sie rozkorzenila i napelnila ziemie.
\par 11 Okryte sa góry cieniem jej, a galezie jej jako najwyzsze cedry.
\par 12 Rozpuscila latorosle swe az do morza, i az do rzeki galazki swe.
\par 13 Przeczzes tedy rozwalil plot winnicy, tak, ze ja szarpaja wszyscy, którzy mimo droga ida?
\par 14 Zniszczyl ja wieprz dziki, a zwierz polny spasl ja.
\par 15 O Boze zastepów! nawróc sie prosze, spojrzyj z nieba, i obacz, a nawiedz te winna macice;
\par 16 Te winnice, która szczepila prawica twoja, i latoroslki, któres sobie zmocnil.
\par 17 Spalona jest ogniem, i wyrabana; ginie od zapalczywosci oblicza twego.
\par 18 Niech bedzie reka twoja nad mezem prawicy twojej, nad synem czlowieczym, któregos sobie zmocnil.
\par 19 A nie odstapimy od ciebie; zachowaj nas przy zywocie, a imienia twego wzywac bedziemy.
\par 20 O Panie, Boze zastepów! nawrócze nas zasie; rozjasnij nad nami oblicze twoje, a bedziemy zbawieni.

\chapter{81}

\par 1 Przedniejszemu spiewakowi na Gittyt, Asafowi.
\par 2 Wesolo spiewajcie Bogu mocy naszej; wykrzykajcie Bogu Jakóbowemu.
\par 3 Wezmijcie psalm, przydajcie beben, i wdzieczna harfe z lutnia.
\par 4 Zatrabcie w trabe na nowiu miesiaca, czasu ulozonego, w dzien swieta naszego uroczystego.
\par 5 Albowiem jest postanowienie w Izraelu, prawo Boga Jakóbowego.
\par 6 Na swiadectwo w Józefie wystawil je, kiedy byl wyszedl przeciw ziemi egipskiej, kedym slyszal jezyk, któregom nie rozumial.
\par 7 Wybawilem, mówi Bóg, od brzemienia ramie jego, a rece jego od dzwigania kotlów uwolnione.
\par 8 Gdys mie w ucisku wzywal. wyrwalem cie, i wysluchalem cie w skrytosci gromu, doswiadczalem cie u wód poswarku. Sela.
\par 9 Tedym rzekl: Sluchaj, ludu mój! a oswiadcze sie przeciwko tobie, o Izraelu! bedzieszli mie sluchal.
\par 10 I nie bedziesz mial boga cudzego, ani sie bedziesz klanial bogu obcemu;
\par 11 (Albowiem Jam Pan, Bóg twój, którym cie wywiódl z ziemi Egipskiej;) otwórz usta twoje, a napelniec je.
\par 12 Ale lud mój nie usluchal glosu mego, a Izrael nie przestal na mnie.
\par 13 Przetoz puscilem ich za zadzami serca ich, i chodzili za radami swemi.
\par 14 Oby mie byl lud mój posluchal, a Izrael drogami mojemi chodzil!
\par 15 W krótkim czasie bym byl nieprzyjaciól ich ponizyl, a przeciw nieprzyjaciolom ich obrócilbym reke swa.
\par 16 Ci, którzy w nienawisci maja Pana, choc obludnie, poddacby sie im musieli, i bylby czas ich az na wieki.
\par 17 I karmilbym ich tlustoscia pszenicy, a miodem z opoki nasycilbym ich.

\chapter{82}

\par 1 Psalm Asafowy. Bóg stoi w zgromadzeniu Bozem, a w posród bogów sadzi i mówi:
\par 2 Dokadze bedziecie niesprawiedliwie sadzic, a osoby niezbozników przyjmowac? Sela.
\par 3 Czyncie sprawiedliwosc ubogiemu i sierotce; utrapionego i niedostatecznego usprawiedliwiajcie.
\par 4 Wyrwijcie chudzine i nedznego, a z reki niepoboznej wyrwijcie go.
\par 5 Lecz oni nic nie wiedza, ani rozumieja; w ciemnosciach ustawicznie chodza; zaczem sie zachwialy wszystkie grunty ziemi.
\par 6 Jam rzekl: Bogowiescie, a synami Najwyzszego wy wszyscy jestescie.
\par 7 A wszakze jako i inni ludzie pomrzecie, a jako jeden z ksiazat upadniecie.
\par 8 Powstanze, o Boze! a sadz ziemie; albowiem ty dziedzicznie trzymasz wszystkie narody.

\chapter{83}

\par 1 Piesn i psalm Asafowy.
\par 2 O Boze! nie milczze, nie badz jako ten, co nie slyszy, i nie chciej sie uspokoic, o Boze!
\par 3 Bo sie oto nieprzyjaciele twoi burza, a ci, którzy cie w nienawisci maja, podnosza glowe.
\par 4 Przeciwko ludowi twemu wymyslili chytra rade, a spikneli sie przeciw tym, których ty ochraniasz;
\par 5 Mówiac: Pójdzcie, a wytracmy ich, niech nie beda narodem, tak, zeby i nie wspominano wiecej imienia Izraelskiego.
\par 6 Albowiem spikneli sie jednomyslnie, przymierze przeciwko tobie uczynili:
\par 7 Namioty Edomczyków, i Ismaelczyków, Moabczyków, i Agarenczyków,
\par 8 Giebalczyków, i Ammonitczyków, i Amalekitczyków, takze Filistynczyków z tymi, którzy mieszkaja w Tyrze;
\par 9 Wiec i Assyryjczycy zlaczyli sie z nimi, bedac ramieniem synom Lotowym. Sela.
\par 10 Uczynze im tak jako Madyjanczykom, jako Sysarze, jako Jabinowi u potoku Cyson.
\par 11 Którzy sa wygladzeni w Endor; stali sie jako gnój na ziemi.
\par 12 Obchodzze sie z nimi, i z ich hetmanami, jako z Orebem, i jako z Zeba, i jako z Zebeem, i jako z Salmanem, ze wszystkimi ksiazetami ich;
\par 13 Bo rzekli: Posiadzmy dziedzicznie przybytki Boze.
\par 14 Boze mój! uczynze ich jako kolo, i jako zdzblo przed wiatrem.
\par 15 Jako ogien, który las pali, i jako plomien, który zapala góry.
\par 16 Tak ich ty wichrem twoim scigaj, a burza twa zatrwóz ich.
\par 17 Napelnij twarze ich pohanbieniem, aby szukali imienia twego, Panie!
\par 18 Niech beda zawstydzeni i ustraszeni az na wieki, a bedac pohanbieni niech zagina.
\par 19 A tak niech poznaja, zes ty, którego imie jest Pan, tys sam Najwyzszym nad wszystka ziemia.

\chapter{84}

\par 1 Przedniejszemu spiewakowi na Gittyt, synom Korego psalm.
\par 2 O jako sa mile przybytki twoje, Panie zastepów!
\par 3 Zada i bardzo teskni dusza moja do sieni Panskich; serce moje i cialo moje pochutniwa sobie do Boga zywego.
\par 4 Oto i wróbel znalazl sobie domek, i jaskólka gniazdo swoje, gdzie poklada ptaszeta swe, u oltarzów twoich, Panie zastepów, królu mój i Boze mój!
\par 5 Blogoslawieni, którzy mieszkaja w domu twoim; beda cie na wieki chwalic. Sela.
\par 6 Blogoslawiony czlowiek, który ma sile swoje w tobie, i w których sercu sa drogi twoje.
\par 7 Którzy idac przez doline morwów, za zródlo go sobie pokladaja, i deszcz pozegnania przychodzi na nich.
\par 8 I ida huf za hufem, a ukazuja sie przed Bogiem na Syonie.
\par 9 O Panie, Boze zastepów! wysluchaj modlitwe moje; przyjmij w uszy twe, o Boze Jakóbowy. Sela.
\par 10 O Boze, tarczo nasza! obacz, a wejrzyj na oblicze pomazanca twego.
\par 11 Albowiem lepszy jest dzien w sieniach twoich, niz gdzie indziej tysiac; obralem sobie raczej w progu siedziec w domu Boga swego, nizeli mieszkac w przybytkach niezbozników.
\par 12 Albowiem Pan Bóg jest sloncem i tarcza: tuc laski i chwaly Pan udziela, i nie odmawia, co jest dobrego, tym, którzy chodza w niewinnosci.
\par 13 Panie zastepów! blogoslawiony czlowiek, który ma nadzieje w tobie.

\chapter{85}

\par 1 Przedniejszemu spiewakowi synom Korego psalm.
\par 2 Laskes, Panie!niekiedy pokazywal ziemi twojej; przywróciles zasie z niewoli Jakóba.
\par 3 Odpusciles nieprawosc ludu twojego, pokryles wszelki grzech ich. Sela.
\par 4 Usmierzyles wszystek gniew twój, odwróciles od zapalczywosci popedliwosc twoje.
\par 5 Przywróc nas, o Boze zbawienia naszego; a uczyn wstret gniewowi swemu przeciwko nam.
\par 6 Izali na wieki gniewac sie bedziesz na nas? a rozciagniesz gniew twój od rodzaju do rodzaju?
\par 7 Izali ty obróciwszy sie, nie ozywisz nas, tak, aby sie lud twój rozradowal w tobie?
\par 8 Panie! okaz nam milosierdzie twoje, a daj nam zbawienie swoje.
\par 9 Ale poslucham, co rzecze Bóg, on Pan mocny; zaiste mówi pokój do ludu swego, i do swietych swoich, byle sie jedno zas do glupstwa nie wracali.
\par 10 Zaistec bliskie jest zbawienie jego tym, którzy sie go boja; a przebywac bedzie chwala jego w ziemi naszej.
\par 11 Milosierdzie i prawda spotkaja sie z soba; sprawiedliwosc i pokój pocaluja sie.
\par 12 Prawda z ziemi wyrosnie, a sprawiedliwosc z nieba wyjrzy.
\par 13 Da tez Pan i doczesne dobra, a ziemia nasza wyda owoc swój.
\par 14 Sprawi, ze sprawiedliwosc przed twarza jego pójdzie, gdy postawi na drodze nogi swoje.

\chapter{86}

\par 1 Modlitwa Dawidowa. Naklon, Panie! ucha twego, a wysluchaj mie; bomci nedzny i ubogi.
\par 2 Strzezze duszy mojej, bom jest ten, którego ty milujesz; zachowaj sluge twego, Boze mój! który ma nadzieje w tobie.
\par 3 Zmiluj sie nademna, Panie, albowiem do ciebie na kazdy dzien wolam.
\par 4 Rozwesel dusze slugi twego; bo do ciebie, o Panie! dusze swa podnosze.
\par 5 Bos ty, Panie! dobry i litosciwy, i wielce milosierny wszystkim, którzy cie wzywaja.
\par 6 Wysluchajze, Panie! modlitwe moje, a posluchaj pilnie glosu prosby mojej.
\par 7 Wzywam cie w dzien ucisku mego; bo mie ty wysluchasz.
\par 8 Nie masz zadnego podobnego tobie miedzy bogami, o Panie! i nie masz takowych spraw, jako sa twoje.
\par 9 Wszystkie narody, któres ty stworzyl, przychodzac klaniac sie beda przed obliczem twojem, Panie! i wielbic beda imie twoje.
\par 10 Bos ty jest wielki, a czynisz cuda; tys sam jest Bogiem.
\par 11 Naucz mie, Panie, drogi twojej, abym chodzil w prawdzie twojej, a ustanów serce moje w bojazni imienia twego;
\par 12 A bede cie chwalil, Panie, Boze mój! ze wszystkiego serca mego, i bede wielbil imie twoje na wieki,
\par 13 Poniewaz milosierdzie twoje wielkie jest nademna, a tys wyrwal dusze moje z dolu glebokiego.
\par 14 O Boze! powstali hardzi przeciwko mnie, a rota okrutników szukala duszy mojej, ci, którzy cie przed oczyma nie maja.
\par 15 Ale ty, Panie, Boze milosierny i litosciwy, i nierychly ku gniewu, i wielce milosierny, i prawdziwy!
\par 16 Wejrzyj na mie, a zmiluj sie nademna, dajze moc twoje sludze twemu, a zachowaj syna sluzebnicy twojej.
\par 17 Okaz mi znak dobroci twojej, aby to widzac ci, którzy mie maja w nienawisci, zawstydzeni byli, zes mie ty, Panie! poratowal, i pocieszyles mie.

\chapter{87}

\par 1 Synom Korego psalm i piesn. Fundament jego jest na górach swietych.
\par 2 Umilowal Pan bramy Syonskie nad wszystkie przybytki Jakóbowe.
\par 3 Slawne o tobie rzeczy powiadaja, o miasto Boze! Sela.
\par 4 Wspomne na Egipt, i na Babilon przed swymi znajomymi; oto i Filistynczycy, i Tyryjczycy, i Murzyni rzeka, ze sie tu kazdy z nich urodzil.
\par 5 Takze i o Syonie mówic beda:Ten i ów urodzil sie w nim; a sam Najwyzszy ugruntuje go.
\par 6 Pan policzy narody, gdy je popisywac bedzie, mówiac:Ten sie tu urodzil. Sela.
\par 7 Przetoz o tobie spiewac beda z plasaniem wszystkie sily zywota mego.

\chapter{88}

\par 1 Piesn a psalm synów Korego przedniejszemu spiewakowi na Machalat ku spiewaniu, nauczajacy, (zlozony)od Hemana Ezrahytczyka.
\par 2 Panie, Boze zbawienia mego! we dnie i w nocy wolam do ciebie.
\par 3 Niech przyjdzie przed oblicze twoje modlitwa moja; naklon ucha twego do wolania mego.
\par 4 Bo nasycona jest utrapieniem dusza moja, a zywot mój przyblizyl sie az do grobu.
\par 5 Poczytano mie miedzy tych, którzy zstepuja do dolu; bylem jako czlowiek bez wszelakiej mocy.
\par 6 Policzony jestem miedzy umarlymi; jestem jako pobici, lezacy w grobie, na których wiecej nie pamietasz, którzy sa od reki twojej wytraceni.
\par 7 Spusciles mie w dól najglebszy, do najciemniejszego i najglebszego miejsca.
\par 8 Dolegla mie zapalczywosc twoja, a wszystkiemi nawalnosciami twemi przytloczyles mie. Sela.
\par 9 Dalekos oddalil znajomych moich odemnie, którymes mie bardzo obrzydzil, a takiem zawarty, ze mi nie lza wynijsc.
\par 10 Oko moje zemdlalo od utrapienia mego; wzywam cie, Panie! na kazdy dzien, wyciagajac do ciebie rece moje.
\par 11 Izali przed umarlymi cuda czynic bedziesz? izali umarli powstana, aby cie wyslawiali? Sela.
\par 12 Izali opowiadane bedzie w grobie milosierdzie twoje? a prawda twoja w zginieniu?
\par 13 Izali poznaja w ciemnosciach cuda twoje? a sprawiedliwosc twoje w ziemi zapamietania?
\par 14 Lecz ja, Panie! do ciebie wolam, a z poranku uprzedza cie modlitwa moja.
\par 15 Przeczze, o Panie! odrzucasz dusze moje, a zakrywasz oblicze twoje przedemna?
\par 16 Jamci utrapiony, i prawie juz umierajacy od gwaltu; ponosze strachy twoje, i trwoze soba.
\par 17 Powstal przeciwko mnie srogi gniew twój, a strachy twoje wytracily mie.
\par 18 Ogarniaja mie jako woda przez caly dzien; otaczaja mie gromadno.
\par 19 Oddaliles odemnie przyjaciela i towarzysza, a znajomym moim jestem jako w ciemnosci.

\chapter{89}

\par 1 Nauczajacy (zlozony) od Etana Ezrahytczyka.
\par 2 O milosierdziach Panskich na wieki spiewac bede; od narodu do narodu opowiadac bede usty swemi prawde twoje.
\par 3 Rzeklem bowiem: Milosierdzie na wieki budowane bedzie; na niebiosach utwierdziles prawde twoje, o którejs rzekl:
\par 4 Postanowilem przymierze z wybranym moim: przysiaglem Dawidowi, sludze swemu,
\par 5 Ze az na wieki utwierdze nasienie twoje, a zbuduje od narodu do narodu stolice twoje.Sela.
\par 6 Przetoz, Panie! wyslawiaja niebiosa cud twój, i prawde twoje w zgromadzeniu swietych.
\par 7 Albowiem któz na niebie przyrównany moze byc Panu? kto podobien jest Panu miedzy synami mocarzów?
\par 8 I w zgromadzeniu swietych bardzo jest Bóg straszliwy, a straszny nade wszystkich, którzy sa okolo niego.
\par 9 Panie, Boze zastepów! któz jest jakos ty, Pan mocny? bo prawda twoja jest okolo ciebie.
\par 10 Ty panujesz nad nadetoscia morska; gdy sie podnosza nawalnosci jego, ty je skracasz.
\par 11 Tys potawrl Egipt jako zranionego; moca ramienia twego rozproszyles nieprzyjaciól twoich.
\par 12 Twojec sa niebiosa, twoja tez i ziemia; okrag swiata i pelnosc jego tys ugruntowal.
\par 13 Tys stworzyl pólnocy i poludnie; Tabor i Hermon spiewaja o imieniu twojem.
\par 14 Ramie twoje mocne jest; mozna jest reka twoja, a wywyzszona jest prawica twoja.
\par 15 Sprawiedliwosc i sad sa gruntem stolicy twojej; milosierdzie i prawda uprzedzaja oblicze twoje.
\par 16 Blogoslawiony lud, który zna dzwiek twój; Panie! w swiatlosci oblicza twego chodzic beda.
\par 17 W imieniu twojem weselic sie beda kazdego dnia, a w sprwiedliwosci twojej wywyzszac sie beda.
\par 18 Bos ty jest chwala mocy ich, a za wola twoja wywyzszy sie róg nasz.
\par 19 Bo od Pana jest tarcza nasza, a od swietego Izraelskiego król nasz.
\par 20 W on czas mówiac w widzeniu do swietego twego rzekles: Polozylem ratunek w reku mocarza, wywyzszylem wybranego z ludu.
\par 21 Znalazlem Dawida, sluge mego; olejkiem swietym moim pomazalem go.
\par 22 Przetoz reka moja bedzie stala przy nim, a ramie moje posili go.
\par 23 Nie ucisnie go nieprzyjaciel, a syn nieprawosci nie utrapi go.
\par 24 Bo potre przed twarza jego przeciwników jego, a tych, którzy go maja w nienawisci, poraze.
\par 25 Nadto prawda moja i milosierdzie moje z nim bedzie, a w imieniu mojem wywyzszony bedzie róg jego.
\par 26 I poloze na morzu reke jego i na rzekach prawice jego.
\par 27 On wolajac rzecze: Tys ojciec mój, Bóg mój, i skala zbawienia mego,
\par 28 Ja go tez za pierworodnego wystawie, i za wyzszego nad królami ziemi.
\par 29 Na wieki mu zachowam milosierdzie moje, a przymierze moje stale bedzie przy nim.
\par 30 I uczynie, ze na wieki bedzie trwalo nasienie jego, a stolica jego jako dni niebios.
\par 31 Ale jezliby synowie jego opuscili zakon mój, a w sadach moich nie chodzili;
\par 32 Jezliby ustawy moje splugawili, a przykazan moich nie przestrzegali:
\par 33 Tedy nawiedze rózga przestepstwo ich, a karaniem nieprawosc ich.
\par 34 Ale milosierdzia swego nie odejme od niego, ani sklamie przeciw prawdzie mojej.
\par 35 Nie splugawie przymierza mego, a tego, co wyszlo z ust moich, nie odmienie.
\par 36 Razem przysiagl przez swietobliwosc moje, ze nie sklamie Dawidowi,
\par 37 A ze nasienie jego zostanie na wieki, a stolica jego jako slonce przedemna;
\par 38 Jako miesiac bedzie utwierdzone na wieki, i jako swiadkowie na niebie godnowierni. Sela.
\par 39 Ales go ty odrzucil i wzgardzil; rozgniewales sie na pomazanca twego.
\par 40 Zrzuciles przymierze z sluga twoim; straciles na ziemie korone jego.
\par 41 Roztargales wszystkie ploty jego, i basztys jego rozwalil.
\par 42 Szarpaja go wszyscy, którzy droga mimo ida; posmiewiskiem jest i sasiadom swoim.
\par 43 Wywyzszyles prawice przeciwników jego; uweseliles wszysstkich nieprzyjaciól jego.
\par 44 I ostrze miecza jego stepiles, a nie ratowales go w bitwie.
\par 45 Zniosles ochedóstwo jego, a stolice jego uderzyles o ziemie.
\par 46 Ukróciles dni mlodosci jego, a przyodziales go hanba. Sela.
\par 47 Dokadze, Panie! na wiekiz sie kryc bedziesz? takze bedzie jako ogien palac zapalczywosc twoja?
\par 48 Wspomnijze na mie, jako krótki jest wiek mój; azas prózno stworzyl wszystkich synów ludzkich?
\par 49 Któz z ludzi tak zyc moze, aby nie ogladal smierci? któz wyrwie dusze swa z mocy grobu? Sela.
\par 50 Gdziez sa litosci twoje dawne, o Panie! któres przysiagl Dawidowi w prawdzie swej?
\par 51 Wspomnij, Panie! na zelzywosc slug twoich, a jakom ponosil wzgarde w zanadrzu swem od wszystkich narodów moznych.
\par 52 Panie! jako uragali nieprzyjaciele twoi, jako uragali sciezkom pomazanca twego.
\par 53 Niech bedzie blogoslawiony Pan az na wieki. Amen, Amen.

\chapter{90}

\par 1 Modlitwa Mojzesza, meza Bozego. Panie! tys bywal ucieczka nasza od narodu do narodu.
\par 2 Pierwej nizli góry stanely i nizlis wyksztaltowal ziemie, i okrag swiata, oto zaraz od wieku az na wieki tys jest Bogiem.
\par 3 Ty znowu czlowieka w proch obracasz, a mówisz: Nawróccie sie synowie ludzcy.
\par 4 Albowiem tysiac lat przed oczyma twemi sa jako dzien wczorajszy, który przeminal, i jako straz nocna.
\par 5 Powodzia porywasz ich; sa jako sen, i jako trawa, która z poranku rosnie.
\par 6 Z poranku kwitnie i rosnie; ale w wieczór bywa pokoszona, i usycha.
\par 7 Albowiem od gniewu twego giniemy, a popedliwoscia twoja jestesmy przestraszeni.
\par 8 Polozyles nieprawosci nasze przed soba, tajne wystepki nasze przed jasnoscia oblicza twego.
\par 9 Skad wszystkie dni nasze nagle przemijaja dla gniewu twego; jako slowa niszczeja lata nasze.
\par 10 Dni wieku naszego jest lat siedmdziesiat, a jezli kto duzszy, lat osmdziesiat, a to, co najlepszego w nich, tylko klopot i nedza, a gdy to pominie, tedy predko odlatujemy.
\par 11 Ale któz zna srogosc gniewu twego? albo kto bojac sie ciebie zna zapalczywosc twoje?
\par 12 Nauczze nas obliczac dni naszych, abysmy przywiedli serce do madrosci.
\par 13 Nawrócze sie, Panie! dokadze odwlaczasz? zlitujze sie nad slugami twymi.
\par 14 Nasycze nas z poranku milosierdziem twojem; tak, abysmy wesolo spiewac i radowac sie mogli po wszystkie dni nasze.
\par 15 Rozweselze nas wedlug dni, któryches nas utrapil, wedlug lat, którychesmy doznali zlego.
\par 16 Niech bedzie znaczna przy slugach twoich sprawa twoja, a chwala twoja przy synach ich.
\par 17 Niech bedzie przyjemnosc Pana, Boga naszego, przy nas, a sprawe rak naszych utwierdz miedzy nami, sprawe rak naszych utwierdz, Panie!

\chapter{91}

\par 1 Ten, który mieszka w ochronie Najwyzszego, i w cieniu Wszechmocnego przebywac bedzie;
\par 2 Rzecze Panu: Nadzieja moja i zamek mój, Bóg mój, w nim nadzieje miec bede.
\par 3 Onci zaiste wybawi cie z sidla lowczego, i z powietrza najjadowitszego.
\par 4 Pierzem swem okryje cie, a pod skrzydlami jego bezpiecznym bedziesz; prawda jego tarcza i puklerzem.
\par 5 Nie ulekniesz sie strachu nocnego, ani strzaly latajacej we dnie;
\par 6 Ani zarazy morowej, która przechodzi w ciemnosci, ani powietrza morowego, które zatraca w poludnie.
\par 7 Padnie po boku twym tysiac, a dziesiec tysiecy po prawej stronie twojej; ale sie do ciebie nie przyblizy.
\par 8 Tylko to oczyma twemi obaczysz, a nagrode niepoboznych ogladasz.
\par 9 Poniewazes ty Pana, który jest nadzieja moja, i Najwyzszego, za przybytek swój polozyl:
\par 10 Nie spotka cie nic zlego, ani jaka plaga przyblizy sie do namiotu twego.
\par 11 Albowiem Aniolom swoim przykazal o tobie, aby cie strzegli na wszystkich drogach twoich.
\par 12 Na rekach nosic cie beda, bys snac nie obrazil o kamien nogi twojej.
\par 13 Po lwie, i po bazyliszku deptac bedziesz, lwie i smoka podepczesz.
\par 14 Iz sie we mnie, mówi Pan, rozkochal, wyrwe go, i wywyzsze go, przeto, iz poznal imie moje.
\par 15 Bedzie mie wzywal, a wyslucham go; Ja z nim bede w utrapieniu, wyrwe go, i uwielbie go.
\par 16 Dlugoscia dni nasyce go, i okaze mu zbawienie moje.

\chapter{92}

\par 1 Psalm a piesn na dzien sobotni.
\par 2 Dobra rzecz jest wyslawiac Pana, a spiewac imieniowi twemu, o Najwyzszy.
\par 3 Opowiadac z poranku milosierdzie twoje, i prawde twoje na kazda noc,
\par 4 Na instrumencie o dziesieciu strunach, na lutni, i na harfie z spiewaniem.
\par 5 Albowiemes mie rozweselil, Panie! sprawami twemi; o sprawach rak twoich spiewac bede.
\par 6 O jako wielmozne sa sprawy twoje, Panie! bardzo glebokie sa mysli twoje.
\par 7 Czlowiek bydlecy nie zna, a glupi nie zrozumiewa tego,
\par 8 Iz wyrastaja niezboznicy jako ziele, a kwitna wszyscy, którzy czynia nieprawosc, aby byli wykorzenieni az na wieki;
\par 9 Ale ty, o Najwyzszy! jestes Panem na wieki.
\par 10 Albowiem, oto nieprzyjaciele twoi, Panie! albowiem oto nieprzyjaciele twoi zgina; rozproszeni beda wszyscy, którzy czynia nieprawosc.
\par 11 Ale róg mój wywyzszysz jako jednorozców; pokropiony bede olejkiem swiezym.
\par 12 I ujrzy oko moje nieszczescie tych, co na mie czyhaja; o zlosnikach, którzy powstawaja przeciwko mnie, uslysza uszy moje.
\par 13 Sprawiedliwy jako palma zakwitnie, jako cedra na Libanie rozmnozy sie.
\par 14 Wszczepieni w domu Panskim, w sieniach Boga naszego zakwitna.
\par 15 Nawet i w sedziwosci przyniosa owoc, czerstwymi i zielonymi beda;
\par 16 Aby to opowiadano, ze uprzejmym jest Pan, skala moja, a ze w nim nie masz zadnej nieprawosci.

\chapter{93}

\par 1 Pan króluje, oblekl sie w dostojnosc; oblekl sie Pan w moznosc, i przepasal sie; utwierdzil tez okrag swiata, aby sie nie poruszyl.
\par 2 Utwierdzona jest stolica twoja przed wszystkiemi czasy; tys jest od wiecznosci.
\par 3 Podniosly rzeki, o Panie! podniosly rzeki szum swój; podniosly rzeki nawalnosci swoje.
\par 4 Nad szum wielkich wód, nad mocne waly morskie mocniejszy jest Pan na wysokosci.
\par 5 Swiadectwa twoje sa bardzo pewne; swietobliwosc, Panie! jest domu twego ozdoba na wieczne dni.

\chapter{94}

\par 1 Boze pomst! Panie Boze pomst! rozjasnij sie!
\par 2 Podnies sie, o Sedzio wszystkiej ziemi! a daj zaplate pysznym.
\par 3 Dokadze niepobozni, Panie! dokadze niepobozni radowac sie beda?
\par 4 Dlugoz beda swiegotac i hardzie mówic, chlubiac sie wszyscy, którzy czynia nieprawosc?
\par 5 Lud twój, Panie! trzec, a dziedzictwo twoje trapic?
\par 6 Wdowy i przychodniów mordowac? a sierotki zabijac?
\par 7 Mówiac: Nie widzi tego Pan, ani tego rozumie Bóg Jakóbowy.
\par 8 Zrozumiciez, o wy bydlecy miedzy ludzmi! a wy szaleni kiedyz zrozumiecie?
\par 9 Izali ten, który szczepil ucho, nie slyszy? i który uksztaltowal oko, izali nie widzi?
\par 10 Izali ten, który cwiczy narody, nie bedzie karal? który uczy czlowieka umiejetnosci.
\par 11 Pan zna mysli ludzkie, iz sa szczera marnoscia.
\par 12 Blogoslawiony jest maz, którego ty cwiczysz, Panie! a zakonu twego uczysz go.
\par 13 Abys mu sprawil pokój od zlych dni, azby byl wykopany dól niezboznikowi.
\par 14 Albowiem, nie opusci Pan ludu swego, a dziedzictwa swego nie zaniecha.
\par 15 Ale az ku sprawiedliwosci obróci sie sad, a za nim wszyscy serca uprzejmego.
\par 16 Któzby sie byl zastawil za mna przeciwko zlosnikom? ktoby sie byl ujal o mnie przeciwko tym, którzy czynia nieprawosc?
\par 17 By mi byl Pan nie przybyl na pomoc, maloby byla nie mieszkala dusza moja w milczeniu.
\par 18 Juzem byl rzekl: Zachwiala sie noga moja; ale milosierdzie twoje, o Panie! zatrzymalo mie.
\par 19 W wielkosci utrapienia mego, we wnetrznosciach moich, pociechy twoje rozweselaly dusze moje.
\par 20 Izali z toba towarzyszy stolica nieprawosci tych, którzy stanowia krzywde miasto prawa?
\par 21 Którzy sie zbieraja przeciwko duszy sprawiedliwego, a krew niewinna potepiaja?
\par 22 Ale Pan jest twierdza moja, a Bóg mój skala ufnosci mojej.
\par 23 Onci obróci na nich nieprawosc ich, a dla zlosci ich wytraci ich; wytraci ich Pan, Bóg nasz.

\chapter{95}

\par 1 Pójdzciez, spiewajmy Panu; wykrzykujmy skale zbawienia naszego.
\par 2 Uprzedzmy oblicze jego z chwala; psalmy mu spiewajmy.
\par 3 Albowiem Pan jest Bóg wielki, i król wielki nade wszystkich bogów.
\par 4 W jegoz rekach sa glebokosci ziemi, i wierzchy gór jego sa.
\par 5 Jegoz jest morze, bo je on uczynil; i ziemia, która rece jego uksztaltowaly.
\par 6 Pójdzcie, klaniajmy sie, a upadajmy przed nim; klekajmy przed Panem, stworzycielem naszym.
\par 7 Onci jest zaiste Bóg nasz, a mysmy lud pastwiska jego, i owce rak jego. Dzis, jezli glos jego uslyszycie,
\par 8 Nie zatwardzajciez serca swego, jako w Meryba, a jako czasu kuszenia na puszczy.
\par 9 Kiedy mie kusili ojcowie wasi, doswiadczylic mie, i widzieli sprawy moje.
\par 10 Przez czterdziesci lat mialem spór z tym narodem, i rzeklem: Lud ten bladzi sercem, a nie poznali dróg moich;
\par 11 Którymem przysiagl w popedliwosci mojej, ze nie wnijda do odpocznienia mego.

\chapter{96}

\par 1 Spiewajcie Panu piesn nowa; spiewajcie Panu wszystka ziemia!
\par 2 Spiewajciez Panu, dobrorzeczcie imieniowi jego, opowiadajcie ode dnia do dnia zbawienie jego.
\par 3 Opowiadajcie miedzy narodami chwale jego, miedzy wszystkimi ludzmi cuda jego.
\par 4 Albowiem wielki Pan i wszelkiej chwaly godny, i straszliwy jest nad wszystkich bogów.
\par 5 Wszyscy bowiem bogowie narodów sa balwani; ale Pan niebiosa uczynil.
\par 6 Zacnosc i ochedóstwo przed obliczem jego, moc i pieknosc w swiatnicy jego.
\par 7 Oddajcie Panu, pokolenia narodów, oddajcie Panu chwale i moc.
\par 8 Oddajcie Panu chwale imienia jego; przyniescie dary, a wnijdzcie do sieni jego.
\par 9 Klaniajcie sie Panu w ozdobie swietobliwosci; niech sie leka oblicza jego wszystka ziemia.
\par 10 Powiadajcie miedzy poganami: Pan króluje, a ze i krag swiata utwierdzony bedzie, tak, aby sie nie poruszyl, a iz bedzie sadzil ludzi w sprawiedliwosci.
\par 11 Niech sie wesela niebiosa, a niech plasa ziemia; niech zaszumi morze, i co w niem jest.
\par 12 Niech plasaja pola, i wszystko co jest na nich; tedy niech wykrzykaja wszystkie drzewa lesne,
\par 13 Przed obliczem Panskiem; boc idzie, idzie zaiste, aby sadzil ziemie. Bedzie sadzil okrag swiata w sprawiedliwosci, a narody w prawdzie swojej.

\chapter{97}

\par 1 Pan króluje; wyskakuj ziemio, a wesel sie mnóstwo wysep!
\par 2 Oblok i ciemnosc okolo niego; sprawiedliwosc i sad sa gruntem stolicy jego.
\par 3 Ogien przed obliczem jego idzie, a zapala w okolo nieprzyjaciól jego.
\par 4 Blyskawice jego oswiecaja okrag swiata, co widzac ziemia zadrzala.
\par 5 Góry jako wosk rozplywaja sie przed obliczem Panskiem, przed obliczem Pana wszystkiej ziemi.
\par 6 Niebiosa opowiadaja sprawiedliwosc jego, a wszystkie narody ogladaja chwale jego.
\par 7 Niechze beda zawstydzeni wszyscy, którzy sluza obrazom, którzy sie chlubia w balwanach; klaniajciez mu sie wszyscy bogowie.
\par 8 To uslyszawszy Syon rozweseli sie, a radowac sie beda córki Judzkie, dla sadów twoich, Panie!
\par 9 Albowiemes ty Pan najwyzszy na wszystkiej ziemi, a bardzos wywyzszony nad wszystkich bogów.
\par 10 Wy, którzy milujecie Pana, miejcie zle w nienawisci; on strzeze swietych swoich, a z reki niepoboznych wyrywa ich.
\par 11 Swiatlosci nasiano sprawiedliwemu, a radosci tym, którzy sa uprzejmego serca.
\par 12 Weselcie sie sprawiedliwi w Panu, a wyslawiajcie pamiatke swietobliwosci jego.

\chapter{98}

\par 1 Psalm. Spiewajcie Panu piesn nowa, bo dziwne rzeczy uczynil; dopomogla mu prawica jego, i ramie swietobliwosci jego.
\par 2 Objawil Pan zbawienie swoje; przed oczyma pogan oznajmil sprawiedliwosc swoje.
\par 3 Wspomnial na milosierdzie swoje, i na prawde swoje przeciw domowi Izraelskiemu; ogladaly wszystkie granice ziemi zbawienie Boga naszego.
\par 4 Spiewajze Panu wszystka ziemio; wykrzykajcie, a weselcie sie i spiewajcie.
\par 5 Grajcie Panu na harfie; na harfie, glosem przyspiewujac.
\par 6 Na trabach i na kornetach krzykliwych glos wydawajcie przed Królem i Panem.
\par 7 Niech zaszumi morze, i co w niem jest, okrag swiata, i mieszkajacy na nim.
\par 8 Rzeki niech klaskaja rekoma; góry wespól niech sie rozraduja,
\par 9 Przed Panem, bo idzie sadzic ziemie. On bedzie sadzil okrag swiata w sprawiedliwosci, i narody w prawosci.

\chapter{99}

\par 1 Pan króluje, niechze zadrza narody; siedzi miedzy Cherubinami, niechze sie poruszy ziemia.
\par 2 Pan na Syonie wielki, a wywyzszony nad wszystkie narody.
\par 3 Niech wyslawiaja imie twoje wielkie i straszne; albowiem swiete jest.
\par 4 Moc zaiste królewska miluje sad; albowiemes ty ustanowil prawa; sad i sprawiedliwosc w Jakóbie ty wykonujesz.
\par 5 Wywyzszajcie Pana, Boga naszego, a klaniajcie sie u podnózka nóg jego; bo swiety jest.
\par 6 Mojzesz i Aaron miedzy kaplanami jego, a Samuel miedzy wzywajacymi imienia jego, wolali do Pana, a on ich wysluchal.
\par 7 W slupie oblokowym mówil do nich; a gdy strzegli swiadectw jego i ustaw, które im podal,
\par 8 Panie, Boze nasz! tys ich wysluchiwal; Boze!bywales im milosciwym, i gdys ich karal dla wystepków ich.
\par 9 Wywyzszajcie Pana, Boga naszego, a klaniajcie sie na górze swietej jego; albowiem swiety jest Pan, Bóg nasz.

\chapter{100}

\par 1 Psalm dla dziekczynienia. Wykrzykajcie Panu, wszystka ziemio!
\par 2 Sluzcie Panu z weselem, przychodzcie przed oblicze jego z radoscia.
\par 3 Wiedzciez, zec Pan jest Bogiem; on uczynil nas, a nie my samych siebie, abysmy byli ludem jego, i owcami pastwiska jego.
\par 4 Wnijdzciez w bramy jego z wyslawianiem, a do sieni jego z chwalami; wyslawiajciez go, dobrorzeczciez imieniowi jego;
\par 5 Albowiem dobry jest Pan, na wieki trwa milosierdzie jego, a od narodu az do narodu prawda jego.

\chapter{101}

\par 1 Psalm samego Dawida. O milosierdziu i o sadzie spiewac bede; tobie, o Panie! spiewac bede.
\par 2 Ostroznym bede na drodze uprzejmej, kiedy przyjdziesz do mnie; bede chodzil ustawicznie w szczerosci serca mego, w domu moim.
\par 3 Nie poloze przed oczy moje zlej rzeczy; kazda sprawe wystepników mam w nienawisci, a nie chwyci sie mnie.
\par 4 Serce przewrotne odstapi odemnie, a o zle nie bede dbal.
\par 5 Tego, który potajemnie obmawia blizniego swego, wytne; oczów wynioslych, i serca nadetego nie bede mógl cierpiec.
\par 6 Oczy moje obrócone beda na prawdomównych w ziemi, aby siadali zemna; kto chodzi droga uprzejma, ten mi sluzyc bedzie.
\par 7 Nie bedzie mieszkal w domu moim zdrajca, ten, który mówi klamstwo, nie ostoi sie przed oczyma memi.
\par 8 Co poranek tracic bede wszystkich niezboznych na ziemi, abym tak wykorzenil z miasta Panskiego wszystkich, którzy czynia nieprawosc.

\chapter{102}

\par 1 Modlitwa utrapionego, gdy bedac w ucisku, przed Panem wylewa zadosc swoje.
\par 2 Panie! wysluchaj modlitwe moje, a wolanie moje niechaj przyjdzie do ciebie.
\par 3 Nie ukrywaj oblicza twego przedemna; w dzien ucisku mego naklon ku mnie ucha twego; w dzien którego cie wzywam, predko mie wysluchaj.
\par 4 Albowiem niszczeja jako dym dni moje, a kosci moje jako ognisko wypalone sa.
\par 5 Porazone jest jako trawa, i uwiedlo serce moje, tak, zem zapomnial jesc chleba swego.
\par 6 Od glosu wzdychania mego przylgnely kosci moje do ciala mego.
\par 7 Stalem sie podobnym pelikanowi na puszczy; jestem jako puhacz na pustyniach.
\par 8 Czuje, a jestem jako wróbel samotny na dachu.
\par 9 Przez caly dzien uragaja mi nieprzyjaciele moi, a nasmiewcy moi przeklinaja mie.
\par 10 Bo jadam popiól jako chleb, a napój mój mieszam ze lzami,
\par 11 Dla rozgniewania twego, i dla zapalczywosci gniewu twego; albowiem podnióslszy mie porzuciles mie.
\par 12 Dni moje sa jako cien nachylony, a jam jako trawa uwiadl;
\par 13 Ale ty, Panie! trwasz na wieki, a pamiatka twoja od narodu do narodu.
\par 14 Ty powstawszy zmilujesz sie nad Syonem; boc czas, zebys sie zlitowal nad nim, gdyz przyszedl czas naznaczony.
\par 15 Albowiem upodobaly sie slugom twoim kamienie jego, i nad prochem jego zmiluja sie;
\par 16 Aby sie bali poganie imienia Panskiego, a wszyscy królowie ziemscy chwaly twojej;
\par 17 Gdy pobuduje Pan Syon, i okaze sie w chwale swojej;
\par 18 Gdy wejrzy na modlitwe ponizonych, nie gardzac modlitwa ich.
\par 19 To zapisza dla narodu potomnego, a lud, który ma byc stworzony, chwalic bedzie Pana,
\par 20 Ze wejrzal z wysokosci swiatnicy swojej, ze z nieba na ziemie spojrzal;
\par 21 Aby wysluchal wzdychania wiezniów, i rozwiazal na smierc skazanych;
\par 22 Aby opowiadali na Syonie imie Panskie, a chwale jego w Jeruzalemie,
\par 23 Gdy sie pospolu zgromadza narody i królestwa, aby sluzyly Panu.
\par 24 Utrapil w drodze sile moje, ukrócil dni moich;
\par 25 Azem rzekl; Boze mój! nie bierz mie w polowie dni moich; od narodu bowiem az do narodu trwaja lata twoje.
\par 26 I pierwej nizelis zalozyl ziemie, i niebiosa, dzielo rak twoich.
\par 27 One pomina, ale ty zostajesz; wszystkie te rzeczy jako szata zwiotszeja, jako odzienie odmienisz je, i odmienione beda.
\par 28 Ale ty tenzes zawzdy jest, a lata twoje nigdy nie ustana.
\par 29 Synowie slug twoich, u ciebie mieszkac beda, a nasienie ich zmocni sie przed toba.

\chapter{103}

\par 1 Psalm Dawidowy. Blogoslaw duszo moja Panu, i wszystkie wnetrznosci moje imieniowi jego swietemu.
\par 2 Blogoslawze duszo moja Panu, a nie zapominaj wszystkich dobrodziejstw jego.
\par 3 Który odpuszcza wszystkie nieprawosci twoje; który uzdrawia wszystkie choroby twoje;
\par 4 Który wybawia od smierci zywot twój; który cie koronuje milosierdziem i wielka litoscia:
\par 5 Który nasyca dobrem usta twoje, a odnawia jako orla mlodosc twoje.
\par 6 Pan czyni, co sprawiedliwego jest, i sady wszystkim ucisnionym.
\par 7 Oznajmil drogi swe Mojzeszowi, a synom Izraelskim sprawy swoje.
\par 8 Milosierny i litosciwy jest Pan, nierychly do gniewu, i wielkiego milosierdzia.
\par 9 Nie bedzie sie na wieki wadzil, a gniewu wiecznie chowal.
\par 10 Nie wedlug grzechów naszych obchodzi sie z nami, ani wedlug nieprawosci naszych odplaca nam.
\par 11 Albowiem jako sa niebiosa wysokie nad ziemia, tak jest utwierdzone milosierdzie jego nad tymi, którzy sie go boja;
\par 12 A jako daleko jest wschód od zachodu, tak daleko oddalil od nas przestepstwa nasze.
\par 13 Jako ma litosc ojciec nad dziatkami, tak ma litosc Pan nad tymi, którzy sie go boja.
\par 14 Onci zaiste zna, cosmy za ulepienie, pamieta, zesmy prochem.
\par 15 Dni czlowiecze sa jako trawa, a jako kwiat polny, tak kwitnie.
\par 16 Gdy nan wiatr powienie, alisci go niemasz, ani go wiecej pozna miejsce jego.
\par 17 Ale milosierdzie Panskie od wieków az na wieki nad tymi, którzy sie go boja, a sprawiedliwosc jego nad synami synów,
\par 18 Którzy strzega przymierza jego, i pamietaja na przykazanie jego, aby je czynili.
\par 19 Pan na niebiosach utwierdzil stolice; a królestwo jego nad wszystkimi panuje.
\par 20 Blogoslawciez Panu Aniolowie jego mocni w sile, którzy czynicie rozkazania jego, poslusznymi bedac glosowi slowa jego.
\par 21 Blogoslawcie Panu wszystkie wojska jego, sludzy jego, którzy czynicie wole jego.
\par 22 Blogoslawcie Panu wszystkie sprawy jego, na wszystkich miejscach panowania jego. Blogoslaw, duszo moja! Panu.

\chapter{104}

\par 1 Blogoslaw, duszo moja! Panu. Panie, Boze mój! wielces jest wielmoznym; chwale i ozdobe przyoblokles.
\par 2 Przyodziales sie swiatloscia jako szata; rozciagnales niebiosa jako opone.
\par 3 Którys zasklepil na wodach palace swoje; który uzywasz obloków miasto wozów; który chodzisz na skrzydlach wiatrowych;
\par 4 Który czynisz duchy poslami swymi; ty czynisz slugi swe ogniem palajacym.
\par 5 Ugruntowales ziemie na slupach jej, tak, ze sie nie poruszy na wieki wieczne.
\par 6 Przepascia jako szata przyodziales ja byl, tak, ze wody staly nad górami.
\par 7 Na zgromienie twojerozbiegly sie, a na glos pogromu twego predko zuciekaly.
\par 8 Wstapily góry, znizyly sie doliny na miejsce, któres im zalozyl.
\par 9 Zamierzyles im kres, aby go nie przestepowaly, ani sie wracaly na okrycie ziemi.
\par 10 Który wypuszczasz zródla po dolinach, aby plynely miedzy górami,
\par 11 A napój dawaly wszystkiemu zwierzowi polnemu; a z nich gasza lesne osly pragnienie swoje.
\par 12 Przy nich mieszka ptastwo niebieskie, a z posród galazek glos wydaje.
\par 13 Który pokrapiasz góry z palaców swoich, aby sie z owoców spraw twoich nasycala ziemia.
\par 14 Za twoja sprawa rosnie trawa dla bydla, a ziola na pozytek czlowieczy; ty wywodzisz chleb z ziemi:
\par 15 I wino, które uwesela serce czlowiecze, od którego sie lsni twarz jako od oleju; i chleb, który zatrzymuje zywot ludzki.
\par 16 Nasycone bywaja i drzewa Panskie, i cedry Libanu, których nasadzil;
\par 17 Na których ptaki gniazda swe maja, i bocian na jedlinach ma dom swój.
\par 18 Góry wysokie dzikim kozom, a skaly sa ucieczka królikom.
\par 19 Uczynil miesiac dla pewnych czasów, a slonce zna zachód swój.
\par 20 Przywodzisz ciemnosc, i bywa noc, w która wychodza wszystkie zwierzeta lesne.
\par 21 Lwieta rycza do lupu, i szukaja od Boga pokarmu swego.
\par 22 Lecz gdy slonce wznijdzie, zas sie zgromadzaja, i w jamach swoich klada sie.
\par 23 Tedy wychodzi czlowiek do roboty swojej, i do pracy swojej az do wieczora.
\par 24 O jakoz wielkie sa sprawy twoje, Panie! te wszystkie madrzes uczynil, a napelniona jest ziemia bogactwem twojem.
\par 25 W morzu zas wielkiem i bardzo szerokiem, tam sa plazy, którym nie masz liczby, i zwierzeta male i wielkie.
\par 26 Po niem okrety przechodza, i wieloryb, któregos ty stworzyl, aby w niem igral.
\par 27 Wszystko to na cie oczekuje, abys im dal pokarm czasu swego.
\par 28 Gdy im dajesz, zbieraja; gdy otwierasz reke twoje, nasycone bywaja dobremi rzeczami.
\par 29 Lecz gdy ukrywasz oblicze twoje, trwoza soba; gdy odbierasz ducha ich, gina, i w proch sie swój obracaja.
\par 30 Gdy wysylasz ducha twego, stworzone bywaja, i odnawiasz oblicze ziemi.
\par 31 Niechajze bedzie chwala Panska na wieki; niech sie rozweseli Pan w sprawach swoich.
\par 32 On gdy wejrzy na ziemie, zadrzy; dotknie sie gór, a zakurza sie.
\par 33 Bede spiewal Panu za zywota mego; bede spiewal Bogu memu, póki mie staje.
\par 34 O nim bedzie wdzieczna mowa moja, a ja sie rozwesele w Panu.
\par 35 Oby byli wytraceni grzesznicy z ziemi, a niezboznych aby juz nie bylo! Blogoslaw, duszo moja! Panu. Halleluja.

\chapter{105}

\par 1 Wyslawiajcie Pana; oglaszajcie imie jego; opowiadajcie miedzy narodami sprawy jego.
\par 2 Spiewajcie mu, spiewajcie mu psalmy, rozmawiajcie o wszystkich cudach jego.
\par 3 Chlubcie sie imieniem swietem jego; niech sie weseli serce szukajacych Pana.
\par 4 Szukajciez Pana i mocy jego; szukajcie oblicza jego zawsze.
\par 5 Przypominajcie sobie dziwy jego, które czynil, cuda jego i sady ust jego.
\par 6 Wy nasienie Abrahama, slugi jego! Wy synowie Jakóbowi, wybrani jego!
\par 7 Onci jest Pan, Bóg nasz, po wszystkiej ziemi sady jego.
\par 8 Pamieta wiecznie na przymierze swoje: na slowo, które przykazal az do tysiacznego pokolenia;
\par 9 Które postanowil z Abrahamem, i na przysiege swa uczyniona Izaakowi.
\par 10 Bo je postanowil Jakóbowi za ustawe, a Izraelowi za umowe wieczna.
\par 11 Mówiac: Tobie dam ziemie Chananejska za sznur dziedzictwa waszego;
\par 12 Kiedy ich byl maly poczet, prawie maly poczet, a jeszcze w niej byli przychodniami.
\par 13 Przechodzili zaiste od narodu do narodu, a z królestwa innego ludu;
\par 14 Nie dopuszczal nikomu, aby im mial krzywde czynic; nawet karal dla nich i królów, mówiac:
\par 15 Nie tykajcie pomazanców moich, a prorokom moim nie czyncie nic zlego.
\par 16 Gdy przywolawszy glód na ziemie, wszystke podpore chleba pokruszyl.
\par 17 Poslal przed nimi meza, który byl za niewolnika sprzedany, to jest Józefa;
\par 18 Którego nogi petami trapili, a zelazo scisnelo cialo jego,
\par 19 Az do onego czasu, gdy sie o nim wzmianka stala; mowa Panska doswiadczala go.
\par 20 Poslawszy król kazal go puscic; ten, który panowal nad narodami, wolnym go uczynil.
\par 21 Postanowil go panem domu swego, i ksiazeciem nad wszystka dzierzawa swoja,
\par 22 Aby wladal i ksiazetami jego wedlug zdania duszy swojej, i starców jego madrosci nauczal.
\par 23 Potem wszedl Izrael do Egiptu, a Jakób byl gosciem w ziemi Chamowej;
\par 24 Gdzie rozmnozyl Bóg lud swój bardzo, i uczynil go mozniejszym nad nieprzyjaciól jego.
\par 25 Odmienil serce ich, iz mieli w nienawisci lud jego, a zmyslali zdrady przeciw slugom jego.
\par 26 Poslal Mojzesza, sluge swego i Aarona, którego obral;
\par 27 Którzy im przedlozyli slowa znaków jego, i cuda w ziemi Chamowej.
\par 28 Poslal ciemnosci, i zacmilo sie, a nie byli odpornymi slowu jego.
\par 29 Obrócil wody ich w krew, a pomorzyl ryby w nich.
\par 30 Wydala ziemia ich mnóstwo zab, i byly w palacach królów ich.
\par 31 Rzekl, a przyszla rozmaita mucha, i mszyce we wszystkich granicach ich.
\par 32 Dal grad miasto deszczu, ogien palacy na ziemie ich.
\par 33 Takze potlukl winnice ich, i figi ich, a pokruszyl drzewa w granicach ich.
\par 34 Rzekl, a przyszla szarancza, i chrzaszczów niezliczone mnóstwo;
\par 35 I pozarly wszelkie ziele w ziemi ich, a pojadly urodzaje ziemi ich.
\par 36 Nawet pobil wszystko pierworodztwo w ziemi ich, poczatek wszystkiej sily ich.
\par 37 Tedy ich wywiódl ze srebrem i ze zlotem, a nie byl nikt slaby miedzy pokoleniem ich.
\par 38 Radowal sie Egipt, gdy oni wychodzili; albowiem byl przypadl na nich strach ich.
\par 39 Rozpostarl oblok na okrycie ich, a ogien na oswiecanie nocy.
\par 40 Na zadanie ich przywiódl przepiórki, a chlebem niebieskim nasycil ich.
\par 41 Otworzyl skale i wyplynely wody, a plynely po suchych miejscach jako rzeka.
\par 42 Albowiem wspomnial na slowo swietobliwosci swojej, które rzekl do Abrahama, slugi swego.
\par 43 Przetoz wywiódl lud swój z weselem, a z spiewaniem wybranych swoich.
\par 44 I podal im ziemie pogan, a posiedli prace narodów.
\par 45 Aby zachowali ustawy jego, a prawa jego przestrzegali. Halleluja.

\chapter{106}

\par 1 Halleluja. Wyslawiajcie Pana; albowiem dobry, albowiem na wieki milosierdzie jego.
\par 2 Któz wyslowi niezmierna moc Panska, a wypowie wszystke chwale jego?
\par 3 Blogoslawieni, którzy strzega sadu, a czynia sprawiedliwosc na kazdy czas.
\par 4 Pamietaj na mie, Panie! dla milosci ku ludowi swemu; nawiedzze mie zbawieniem swojem,
\par 5 Abym uzywal dobrego z wybranymi twoimi, a weselil sie w radosci narodu twego, i chlubil sie wespól z dziedzictwem twojem.
\par 6 Zgrzeszylismy z ojcami swymi; niesprawiedliwiesmy czynili, i nieprawosc popelniali.
\par 7 Ojcowie nasi w Egipcie nie zrozumieli cudów twoich, ani pamietali na wielkosc milosierdzia twego; ale odpornymi byli przy morzu Czerwonem.
\par 8 A wszakze ich wyswobodzil dla imienia swego, aby oznajmil moc swoje.
\par 9 Bo zgromil morze Czerwone, i wyschlo, a przewiódl ich przez przepasci, jako przez puszcze.
\par 10 A tak zachowal ich od reki tego, który ich mial w nienawisci, a wykupil ich z reki nieprzyjacielskiej.
\par 11 W tem okryly wody tych, którzy ich ciazyli; nie zostal ani jeden z nich.
\par 12 A choc uwierzyli slowom jego, i wyslawiali chwale jego:
\par 13 Przeciez predko zapomnieli na sprawy jego, i nie czekali na rady jego.
\par 14 Ale zdjeci bedac chciwoscia na puszczy, kusili Boga na pustyniach.
\par 15 I dal im, czego zadali, a wszakze przepuscil suchoty na nich.
\par 16 Zatem gdy sie wzruszyli zawiscia przeciw Mojzeszowi w obozie, i przeciw Aaronowi, swietemu Panskiemu:
\par 17 Otworzyla sie ziemia, i pozarla Datana, i okryla rote Abironowa,
\par 18 I zapalil sie ogien na zebranie ich; plomien spalil niepoboznych.
\par 19 Sprawili i cielca na Horebie, i klaniali sie balwanowi litemu,
\par 20 I odmienili chwale swa w podobienstwo wolu, jedzacego trawe.
\par 21 Zapomnieli na Boga, wybawiciela swego, który czynil wielkie rzeczy w Egipcie;
\par 22 Rzeczy dziwne w ziemi Chamowej, rzeczy straszne przy morzu Czerwonem.
\par 23 Przetoz rzekl, ze ich chcial wytracic, gdyby sie byl Mojzesz, wybrany jego, nie stawil w onem rozerwaniu przed nim, a nie odwrócil popedliwosci jego, aby ich nie tracil.
\par 24 Wzgardzili tez ziemia pozadana, nie wierzac slowu jego.
\par 25 I szemrzac w namiotach swoich, nie byli posluszni glosowi Panskiemu.
\par 26 Przetoz podniósl reke swoje przeciwko nim, aby ich pobil na puszczy;
\par 27 A zeby rozrzucil nasienie ich miedzy pogan, i rozproszyl ich po ziemiach.
\par 28 Sprzegli sie tez byli z balwanem Baalfegorem, a jedli ofiary umarlych.
\par 29 A tak draznili Boga sprawami swemi, ze sie na nich oborzyla plaga;
\par 30 Az sie zastawil Finees, a pomste uczynil, i rozerwana jest ona plaga;
\par 31 Co mu poczytano ku sprawiedliwosci od narodu do narodu, az na wieki.
\par 32 Znowu go byli wzruszyli do gniewu u wód Meryba, tak, iz sie zle dzialo i z Mojzeszem dla nich.
\par 33 Albowiem rozdraznili ducha jego, ze wyrzekl co niesluszne usty swemi.
\par 34 Nadto nie wytracili onych narodów, o których im byl Pan powiedzial.
\par 35 Ale pomieszawszy sie z onemi narodami, nauczyli sie spraw ich:
\par 36 I sluzyli balwanom ich, które im byly sidlem.
\par 37 Albowiem dyjablom ofiarowali synów swoich, i córki swoje,
\par 38 I wylewali krew niewinna, krew synów swoich, i córek swoich, których ofiarowali balwanom rytym Chananejskim, tak, ze splugawiona byla ziemia onem krwi rozlaniem.
\par 39 I zmazali sie sprawami swemi, a cudzolozyli wynalazkami swemi.
\par 40 Przetoz zapaliwszy sie Pan w popedliwosci przeciw ludowi swemu, obrzydzil sobie dziedzictwo swoje,
\par 41 I podal ich w rece poganom; a panowali nad nimi, którzy ich mieli w nienawisci;
\par 42 I uciskali ich nieprzyjaciele ich, tak, ze ponizeni byli pod reka ich.
\par 43 Czestokroc ich wybawial; wszakze go oni wzruszali do gniewu radami swemi, zaczem ponizeni byli dla nieprawosci swoich.
\par 44 A wszakze wejrzal na ucisk ich, i uslyszal wolanie ich.
\par 45 Bo sobie wspomnial na przymierze swoje z nimi, a zalowal tego wedlug wielkiej litosci swojej.
\par 46 Tak, ze im zjednal milosierdzie przed oczyma wszystkich, którzy ich byli pojmali.
\par 47 Wybawze nas, Panie, Boze nasz! a zgromadz nas z tych pogan, abysmy wyslawiali imie swietobliwosci twojej, a chlubili sie w chwale twojej.
\par 48 Blogoslawiony Pan, Bóg Izraelski, od wieków az na wieki; na co niech rzecze wszystek lud: Amen, Halleluja.

\chapter{107}

\par 1 Wyslawiajcie Pana: albowiem dobry; albowiem na wieki milosierdzie jego.
\par 2 Niech o tem powiedza ci, których odkupil Pan, jako ich wykupil z reki nieprzyjacielskiej,
\par 3 A zgromadzil ich z ziem, od wschodu i od zachodu, od pólnocy i od morza.
\par 4 Bladzili po puszczy, po pustyni bezdroznej, miasta dla mieszkania nie znajdujac.
\par 5 Byli glodnymi i pragnacymi, az w nich omdlewala dusza ich.
\par 6 A gdy wolali do Pana w utrapieniu swojem, z ucisku ich wyrywal ich;
\par 7 I prowadzil ich droga prosta, aby przyszli do miasta, w któremby mieszkali.
\par 8 Niechajze wyslawiaja przed Panem milosierdzie jego, a dziwne sprawy jego przed synami ludzkimi:
\par 9 Iz napoil dusze pragnaca, a dusze zglodniala napelnil dobrami.
\par 10 Którzy siedza w ciemnosci i w cieniu smierci, scisnieni bedac nedza i zelazem,
\par 11 Przeto, ze byli odpornymi wyrokom Bozym, a rada Najwyzszego pogardzili;
\par 12 Dla czego ponizyl bieda serce ich; upadli, a nie byl, ktoby ratowal.
\par 13 A gdy wolali do Pana w utrapieniu swojem, z ucisków ich wybawial ich.
\par 14 Wywodzil ich z ciemnosci, i z cienia smierci, a zwiazki ich potargal.
\par 15 Niechajze wyslawiaja przed Panem milosierdzie jego, a dziwne sprawy jego przed synami ludzkimi.
\par 16 Przeto, ze kruszy bramy miedziane, a zawory zelazne rabie.
\par 17 Szaleni dla drogi przewrotnosci swojej, i dla nieprawosci swej utrapieni bywaja.
\par 18 Wszelki pokarm brzydzi sobie dusza ich, az sie przyblizaja do bram smierci.
\par 19 Gdy wolaja do Pana w utrapieniu swojem, z ucisków ich wybawia ich.
\par 20 Posyla slowo swe, i uzdrawia ich, a wybawia ich z grobu.
\par 21 Niechajze wyslawiaja przed Panem milosierdzie jego, a dziwne sprawy jego przed synami ludzkimi;
\par 22 I ofiarujac ofiary chwaly, niech opowiadaja sprawy jego z wesolem spiewaniem.
\par 23 Którzy sie plawia na morzu w okretach, pracujacy na wodach wielkich:
\par 24 Ci widuja sprawy Panskie, i dziwy jego na glebi.
\par 25 Jako jedno rzecze, wnet powstanie wiatr gwaltowny, a podnosza sie nawalnosci morskie.
\par 26 Wstepuja az ku niebu, i zas zstepuja do przepasci, tak, iz sie dusza ich w niebezpieczenstwie rozplywa.
\par 27 Bywaja miotani, a potaczaja sie jako pijany, a wszystka umiejetnosc ich niszczeje.
\par 28 Gdy wolaja do Pana w utrapieniu swojem, z ucisków ich wybawia ich.
\par 29 Obraca burze w cisze, tak, ze umilkna nawalnosci ich.
\par 30 I wesela sie, ze ucichlo; a tak przywodzi ich do portu pozadanego.
\par 31 Niechajze wyslawiaja przed Panem milosierdzie jego, a dziwne sprawy jego przed synami ludzkimi.
\par 32 Niech go wywyzszaja w zgromadzeniu ludu, a w radzie starców niechaj go chwala.
\par 33 Obraca rzeki w pustynie, a potoki wód w susze;
\par 34 Ziemie urodzajna obraca w nieplodna dla zlosci tych, którzy w niej mieszkaja.
\par 35 Pustynie obraca w jeziora, a ziemie sucha w strumienie wód.
\par 36 I osadza w nich glodnych, aby zakladali miasta ku mieszkaniu;
\par 37 Którzy posiewaja pole, a sadza winnice, i zgromadzaja sobie pozytek z urodzaju.
\par 38 Takci im on blogoslawi, ze sie bardzo rozmnazaja, a dobytku ich nie umniejsza.
\par 39 Ale podczas umniejszeni i ponizeni bywaja okrucienstwem, nedza, i utrapieniem;
\par 40 Gdy wylewa wzgarde na ksiazat, dopuszczajac, aby bladzili po puszczy bezdroznej.
\par 41 Onci nedznego z utrapienia podnosi, i rozmnaza rodzine jego jako trzode.
\par 42 To widzac uprzejmi rozwesela sie, a wszelka nieprawosc zatka usta swe.
\par 43 Ale któz jest tak madry, aby to upatrywal, i wyrozumiewal wszystkie litosci Panskie?

\chapter{108}

\par 1 Piesn psalmu samego Dawida.
\par 2 Gotowe jest serce moje, Boze! spiewac i wyslawiac cie bede, takze i chwala moja.
\par 3 Ocucze sie lutnio i harfo! gdy na switaniu powstaje.
\par 4 Wyslawiac cie bede miedzy ludzmi, Panie! a bedec spiewal miedzy narodami.
\par 5 Albowiem wieksze jest nad niebiosa milosierdzie twoje, i az pod obloki prawda twoja.
\par 6 Wywyszze sie nad niebiosa, o Boze! a nad wszystke ziemie chwala twoja.
\par 7 Niech beda wybawieni umilowami twoi; zachowajze ich prawica swoja, a wysluchaj mie.
\par 8 Bóg mówil przez swietobliwosc swoje; dlatego sie weselic bede, ze rozdziele Sychem, a doline Sukkot rozmierze.
\par 9 Mojec jest Galaad, mój i Manases, a Efraim moca glowy mojej, Juda zakonodawca mój.
\par 10 Moab jest miednica do umywania mego, na Edoma porzuce obuwie moje: przeciwko Filistynom trabic bede.
\par 11 Któz mie zaprowadzi do miasta obronnego? Któz mie przywiedzie az do ziemi edomskiej?
\par 12 Izali nie ty, o Boze! którys nas byl odrzucil, a nie wychodziles, o Boze! z wojskami naszemi?
\par 13 Dajze nam pomoc z ucisku; albowiem omylna jest pomoc ludzka.
\par 14 W Bogu sobie meznie poczynac bedziemy, a on podepcze nieprzyjaciól naszych.

\chapter{109}

\par 1 Przedniejszemu spiewakowi psalm Dawidowy. O Boze chwaly mojej! nie milcz;
\par 2 Bo sie usta niepoboznego, i usta klamliwe przeciwko mnie otworzyly; mówili przeciwko mnie jezykiem klamliwym,
\par 3 A slowy jadowitemi ogarneli mie, walczac przeciwko mnie bez wszekiej przyczyny.
\par 4 Przeciwili mi sie za milosc moje, chociazem sie za nich modlil.
\par 5 Oddawaja mi zlem za dobre; a nienawiscia za milosc moje.
\par 6 Postawze nad nim bezboznika, a przeciwnik niech stoi po prawej rece jego.
\par 7 Gdy przed sadem stanie, niech wynijdzie potepionym, a modlitwa jego niech sie w grzech obróci.
\par 8 Niech beda dni jego krótkie, a przelozenstwo jego niech inny wezmie.
\par 9 Niech dzieci jego beda sierotami, a zona jego wdowa.
\par 10 Niech beda biegunami i tulaczami synowie jego, niech zebrza, a niech zebrza wychodzac z pustek swoich.
\par 11 Niech lichwiarz zalapi wszystko, co jest jego, a niech obcy rozchwyca prace jego.
\par 12 Niech nie bedzie ktoby mu milosierdzie pokazal, niech nie bedzie, ktoby sie zmilowal nad sierotami jego.
\par 13 Potomkowie jego niech z korzenia wycieci beda; w drugiem pokoleniu niech bedzie wygladzone imie ich.
\par 14 Niech przyjdzie na pamiec nieprawosc przodków jego przed Panem, a grzech matki jego niechaj nie bedzie zgladzony.
\par 15 Niech beda przed Panem ustawicznie, azby wygladzil z ziemi pamiatke ich,
\par 16 Przeto, ze nie pamietal, aby czynil milosierdzie, ale przesladowal czlowieka nedznego i ubogiego, a tego, który byl serca utrapionego, chcial zamordowac.
\par 17 Poniewaz umilowal przeklestwo, niechze przyjdzie na niego; niechcial blogoslawienstwa niechze bedzie oddalone od niego.
\par 18 A tak niech bedzie obleczony w przeklestwo, jako w szate swoje; a niech wnijdzie jako woda we wnetrznosci jego, a jako olej w kosci jego.
\par 19 Niech mu to bedzie jako plaszcz do przodziania, a jako pas dla ustawicznego opasywania.
\par 20 Takowa zaplata niech bedzie przeciwnikom moim od Pana, i tym, którzy zle mówia przeciwko duszy mojej.
\par 21 Ale ty, Panie! o Panie! uzyj nademna litosci dla imienia twego; a iz dobre jest milosierdzie twoje, wyrwijze mie.
\par 22 Bomci ja jest ubogi i nedzny, a serce moje zranione jest w wnetrznosciach moich.
\par 23 Jako cien, który ustepuje, uchodzic musze; zganiaja mie jako szarancze.
\par 24 Kolana moje upadaja od postu, a cialo moje wychudlo z tlustosci.
\par 25 Nadto stalem sie im posmiewiskiem; gdy mie widza, kiwaja glowami swemi.
\par 26 Wspomózze mie, o Panie, Boze mój! zachowaj mie wedlug milosierdzia swego,
\par 27 Tak, aby poznac mogli, iz to reka twoja, a zes ty, Panie! to uczynil.
\par 28 Niechze oni przeklinaja, ty blogoslaw; którzy powstali, niech beda zawstydzeni, aby sie weselil sluga twój.
\par 29 Niech beda przeciwnicy moi w hanbe obleczeni, a niech sie przyodzieja, jako plaszczem, zelzywoscia swoja.
\par 30 Bede Pana wielce wyslawial usty swemi, a w posród wielu chwalic go bede.
\par 31 Przeto, ze stoi po prawej stronie nedznemu, aby go wybawil od tych, którzy osadzaja dusze jego.

\chapter{110}

\par 1 Psalm Dawidowy. Rzekl Pan Panu memu: Siadz po prawicy mojej, dokad nie poloze nieprzyjaciól twoich podnózkiem nóg twoich.
\par 2 Laske mocy twojej posle Pan z Syonu, mówiac: Panuj w posród nieprzyjaciól twoich.
\par 3 Lud twój bedzie dobrowolny w dzien zwyciestwa twego, w ozdobie swietobliwosci, a rozrodzi sie plód twój z zywota jako rosa na switaniu.
\par 4 Przysiagl Pan, a nie bedzie tego zalowal, mówiac: Tys jest kaplanem na wieki wedlug porzadku Melchisedechowego.
\par 5 Pan po prawicy twojej potrze królów w dzien gniewu swego.
\par 6 Bedzie sadzil narody, i wszystko napelni trupami; potlucze glowe nad wiela ziem panujaca.
\par 7 Z strumienia na drodze pic bedzie; przetoz wywyzszy glowe.

\chapter{111}

\par 1 Halleluja. Bede wyslawial Pana calem sercem w radzie szczerych, i w zgromadzeniu.
\par 2 Wielkie sprawy Panskie, jawne u wszystkich, którzy sie w nich kochaja.
\par 3 Chwalebne i ozdobne dzielo jego, a sprawiedliwosc jego trwa na wieki.
\par 4 Pamiatke cudów swoich uczynil milosierny a litosciwy Pan.
\par 5 Dal pokarm tym, którzy sie go boja, pamietajac wiecznie na przymierze swoje.
\par 6 Moc spraw swoich oznajmil ludowi swemu, dawszy im dziedzictwo pogan.
\par 7 Uczynki rak jego prawda i sad; nieodmienne sa wszystkie przykazania jego,
\par 8 Utwierdzone na wieki wieczne, uczynione w prawdzie i w szczerosci.
\par 9 Wykupienie poslawszy ludowi swemu, przykazal na wieki strzedz przymierza swego; swiete i straszne jest imie jego.
\par 10 Poczatek madrosci jest bojazn Panska; rozumu dobrego nabywaja wszyscy, którzy rozkazanie Panskie czyna; chwala jego trwa na wieki.

\chapter{112}

\par 1 Halleluja. Blogoslawiony maz, który sie Pana boi, a w przykazaniach jego ma wielkie kochanie.
\par 2 Mozne bedzie na ziemi nasienie jego; rodzina szczerych blogoslawiona bedzie.
\par 3 Majetnosc i bogactwa sa w domu jego, a sprawiedliwosc jego trwa na wieki.
\par 4 Szczerym w ciemnosciach swiatlosc wschodzi; laskawy, milosierny, i sprawiedliwy jest Bóg.
\par 5 Dobry czlowiek litosciwym jest, i pozycza, a rzeczy swe miarkuje rozsadkiem.
\par 6 Bo na wieki nie bedzie poruszony; w pamieci wiecznej bedzie sprawiedliwy.
\par 7 Slyszac zle nowiny, nie boi sie; stateczne serce jego ufa w Panu.
\par 8 Umocnione serce jego nie boi sie, az oglada pomste nad nieprzyjaciolmi swymi.
\par 9 Rozprasza, i daje ubogim; sprawiedliwosc jego trwa na wieki; róg jego wywyzszy sie w slawie.
\par 10 Widzac to niepobozny, bedzie sie gniewal, i zebami swemi zgrzytal, i schnac bedzie; zadosc niepoboznych zginie.

\chapter{113}

\par 1 Halleluja. Chwalcie sludzy Panscy, chwalcie imie Panskie.
\par 2 Niechaj bedzie imie Panskie blogoslawione, odtad az na wieki.
\par 3 Od wschodu slonca, az do zachodu jego, niech bedzie chwalebne imie Panskie.
\par 4 Pan jest nad wszystkie narody wywyzszony; chwala jego nad niebiosa.
\par 5 Któz taki, jako Pan Bóg nasz, który mieszka na wysokosci?
\par 6 Który sie zniza, aby widzial, co jest na niebie i na ziemi.
\par 7 Podnosi z prochu nedznego, a z gnoju wywyzsza ubogiego,
\par 8 Aby go posadzil z ksiazetami, z ksiazetami ludu swego;
\par 9 Który sprawia, ze nieplodna w domu bywa matka weselaca sie z dziatek. Halleluja.

\chapter{114}

\par 1 Gdy wychodzil Izrael z Egiptu i dom Jakóbowy z narodu obcego,
\par 2 Stal sie Juda poswieceniem jego, Izrael panowaniem jego.
\par 3 To widzac morze, ucieklo a Jordan wrócil sie nazad.
\par 4 Góry skakaly jako barany, pogórki jako jagnieta.
\par 5 Morze! cóz ci sie stalo, izes ucieklo? O Jordanie! zes sie nazad wrócil?
\par 6 Góry! zescie skakaly jako barany? pagórki! jako jagnieta?
\par 7 Przed obliczem Panskiem zadrzala ziemia, przed obliczem Boga Jakóbowego.
\par 8 Który obraca opoke w jezioro wód, a krzemien w zródlo wód.

\chapter{115}

\par 1 Nie nam, Panie! nie nam, ale imieniowi twemu daj chwale dla milosierdzia twego, i dla prawdy twojej.
\par 2 Czemuz maja mówic poganie: Gdziez teraz jest Bóg ich?
\par 3 Wszakze Bóg nasz jest na niebie, czyniac wszystko, co mu sie podoba.
\par 4 Ale balwany ich sa srebro i zloto, robota rak ludzkich.
\par 5 Usta maja, a nie mówia; oczy maja, a nie widza.
\par 6 Uszy maja, a nie slysza; nozdrze maja, a nie wonieja.
\par 7 Rece maja, a nie macaja; nogi maja, a nie chodza, ani wolaja gardlem swojem.
\par 8 Niech im podobni beda, którzy je czynia, i wszyscy, którzy w nich ufaja.
\par 9 Izraelu! ufaj w Panu; bo on jest pomocnikiem ich i tarcza ich.
\par 10 Domie Aaronowy! ufajcie w Panu; on jest pomocnikiem, i tarcza ich.
\par 11 Którzy sie boicie Pana, ufajcie w Panu; on jest pomocnikiem i tarcza ich.
\par 12 Pan bedzie pamietal na nas, bedzie blogoslawil; bedzie blogoslawil domowi Izraelskiemu, bedzie blogoslawil domowi Aaronowemu.
\par 13 Bedzie blogoslawil tym, którzy sie boja Pana, malym i wielkim.
\par 14 Rozmnozy was Pan, was i synów waszych.
\par 15 Blogoslawieniscie wy od Pana, który stworzyl niebo i ziemie.
\par 16 Niebiosa sa niebiosa Panskie; ale ziemie dal synom ludzkim.
\par 17 Umarli nie beda chwalili Pana, ani kto z tych, co zstepuja do miejsca milczenia.
\par 18 Ale my bedziemy blogoslawili Panu, odtad az na wieki. Halleluja.

\chapter{116}

\par 1 Miluje Pana, iz wysluchal glos mój, i prosby moje.
\par 2 Albowiem naklonil ucha swego ku mnie, gdym go wzywal za dni moich.
\par 3 Ogarnely mie byly bolesci smierci, a utrapienia grobu zjely mie; ucisk i bolesc przyszla na mie.
\par 4 I wzywalem imienia Panskiego, mówiac: Prosze, o Panie! wybaw dusze moje.
\par 5 Milosciwy Pan i sprawiedliwy, Bóg nasz litosciwy.
\par 6 Pan prostaczków strzeze; bylem ucisniony, a wspomógl mie.
\par 7 Nawróc sie, duszo moja! do odpocznienia swego; albowiem ci Pan dobrze uczynil.
\par 8 Bo wyrwal dusze moje od smierci, oczy moje od placzu, noge moje od upadku.
\par 9 Bede chodzil ustawicznie przed oblicznoscia Panska w ziemi zyjacych.
\par 10 Uwierzylem, dlategom mówil, chociazem bardzo byl utrapiony.
\par 11 Jam byl rzekl w zatrwozeniu mojem: Wszelki czlowiek klamca.
\par 12 Cóz oddam Panu za wszystkie dobrodziejstwa jego, które mi uczynil?
\par 13 Kielich obfitego zbawienia wezme, a imienia Panskiego wzywac bede,
\par 14 Sluby moje oddam Panu, a to zaraz przed wszystkim ludem jego.
\par 15 Droga jest przed oczyma Panskiemi smierc swietych jego.
\par 16 O mój Panie! zem ja sluga twoim, jam sluga twoim, synem sluzebnicy twojej, rozwiazales zwiazki moje.
\par 17 Tobie ofiarowac bede ofiare chwaly, i imienia Panskiego wzywac bede.
\par 18 Sluby moje oddam Panu, a to zaraz przed wszystkim ludem jego,
\par 19 W przysionkach domu Panskiego, w posrodku ciebie, Jeruzalemie! Halleluja.

\chapter{117}

\par 1 Chwalcie Pana wszystkie narody! chwalcie go wszyscy ludzie!
\par 2 Albowiem rozszerzone jest nad nami milosierdzie jego, a prawda Panska trwa na wieki. Halleluja.

\chapter{118}

\par 1 Wyslawiajcie Pana, albowiem dobry; albowiem na wieki trwa milosierdzie jego;
\par 2 Rzecz teraz, Izraelu! ze na wieki milosierdzie jego.
\par 3 Rzecz teraz, domie Aaronowy! ze na wieki milosierdzie jego.
\par 4 Rzeczciez teraz, którzy sie boicie Pana, ze na wieki milosierdzie jego.
\par 5 W ucisku wzywalem Pana; wysluchal mie, i na przestrzenstwie postawil mie Pan.
\par 6 Pan jest zemna, nie bede sie bal, zeby mi co uczynil czlowiek.
\par 7 Pan jest zemna miedzy pomocnikami mymi; przetoz ja ogladam pomste nad tymi, którzy mie maja w nienawisci.
\par 8 Lepiej miec nadzieje w Panu, nizeli ufac w czlowieku.
\par 9 Lepiej miec nadzieje w Panu, nizeli ufac w ksiazetach.
\par 10 Wszystkie narody ogarnely mie; ale w imieniu Panskim wygubilem ich.
\par 11 Czestokroc mie ogarnely; ale w imieniu Panskiem wygubilem ich.
\par 12 Ogarnely mie jako pszczoly, ale zgasly jako ogien z ciernia; bo w imieniu Panskiem wytracilem ich.
\par 13 Bardzos poteznie na mie nacieral, abym upadl; ale Pan poratowal mie.
\par 14 Pan jest moca moja, i piesnia moja; on byl moim wybawicielem.
\par 15 Glos wykrzykania i zbawienia w przybytkach sprawiedliwych, prawica Panska dokazala mocy;
\par 16 Prawica Panska wywyzszyla sie; prawica Panska dokazala mocy.
\par 17 Nie umre, ale bede zyl, abym opowiadal sprawy Panskie.
\par 18 Pokaralci mie Pan srodze; ale mie na smierc nie podal.
\par 19 Otwórzcie mi bramy sprawiedliwosci, a wszedlszy w nie bede wyslawial Pana.
\par 20 Tac jest brama Panska, która sprawiedliwi wchodza.
\par 21 Tuc ja ciebie wyslawiac bede; bos mie wysluchal, i byles wybawicielem moim.
\par 22 Kamien, którzy odrzucili budujacy, uczyniony jest glowa wegielna.
\par 23 Od Pana sie to stalo, a jest dziwno w oczach naszych.
\par 24 Tenci to dzien, który uczynil Pan; rozweselmyz sie, a rozradujmy sie wen.
\par 25 Prosze, Panie! zachowajze teraz; prosze Panie! zdarz teraz.
\par 26 Blogoslawiony, który przychodzi w imie Panskie; blogoslawimy wam z domu Panskiego.
\par 27 Bógci Panem, onci nas oswiecil; przywiazcie baranki powrozami ku ofierze az do rogów oltarza.
\par 28 Tys jest Bóg mój; przetoz cie wyslawiac bede, Boze mój! wywyzszac cie bede.
\par 29 Wyslawiajciez Pana, albowiem jest dobry: albowiem na wieki milosierdzie jego.

\chapter{119}

\par 1 Blogoslawieni, którzy zyja bez nagany, którzy chodza w zakonie Panskim,
\par 2 Blogoslawieni, którzy strzega swiadectw jego, i którzy go ze wszystkiego serca szukaja;
\par 3 I którzy nie czynia nieprawosci, ale chodza drogami jego.
\par 4 Tys przykazal, aby pilnie strzezono rozkazan twoich.
\par 5 Oby wyprostowane byly drogi moje ku przestrzeganiu praw twoich!
\par 6 Tedy nie bede zawstydzony, gdy sie bede ogladal na wszystkie rozkazania twoje.
\par 7 Bede cie wyslawial w szczerosci serca, gdy sie naucze praw sprawiedliwosci twojej.
\par 8 Ustaw twoich z pilnoscia strzedz bede; tylko mie nie opuszczaj.
\par 9 Jakim sposobem oczysci mlodzieniec scieszke swoje? Gdy sie zachowa wedlug slowa twego.
\par 10 Ze wszystkiego serca mego szukam cie; nie dopuszczajze mi bladzic od rozkazan twoich.
\par 11 W sercu mojem skladam wyroki twoje, abym nie zgrzeszyl przeciwko tobie.
\par 12 Blogoslawionys ty, Panie! nauczze mie ustaw twoich.
\par 13 Wargami mojemi opowiadam wszystkie sady ust twoich.
\par 14 W drodze swiadectw twoich kocham sie wiecej, niz we wszystkich bogactwach.
\par 15 O przykazaniach twoich rozmyslam, i przypatruje sie drogom twoim.
\par 16 W ustawach twoich kocham sie, i nie zapominam slów twoich.
\par 17 Daruj to sludze twemu, abym zyl, a przestrzegal slów twoich.
\par 18 Odslon oczy moje, abym sie przypatrzyl dziwom z zakonu twego.
\par 19 Jestem gosciem na ziemi; nie ukrywaj przedemna rozkazan twoich.
\par 20 Omdlewa dusza moja, pragnac sadów twoich na kazdy czas.
\par 21 Wytraciles pysznych; przekleci sa ci, którzy bladza od rozkazan twoich.
\par 22 Oddal odemnie pohanbienie i wzgarde, gdyz strzege swiadecwt twoich.
\par 23 I ksiazeta zasiadaja, a mówia przeciwko mnie; wszakze sluga twój rozmysla w ustawach twoich.
\par 24 Swiadectwa twoje zaiste sa mojem kochaniem, i radcami mymi.
\par 25 Przylgnela do prochu dusza moja; ozywze mie wedlug slowa twego.
\par 26 Drogi moje rozpowiedzialem, a wysluchales mie; naucz mie ustaw twoich.
\par 27 Daj, abym zrozumial droge rozkazan twoich, azebym rozmyslal o dziwnych sprawach twoich.
\par 28 Rozplywa sie od smutku dusza moja; utwierdzze mie wedlug slowa twego.
\par 29 Droge klamliwa oddal odemnie, a zakonem twoim udaruj mie.
\par 30 Obralem droge prawdy, a sady twoje przekladam sobie.
\par 31 Przystalem do swiadectw twoich; Panie! nie zawstydzajze mie.
\par 32 Droga przykazan twoich pobieze, gdy rozszerzysz serce moje.
\par 33 Naucz mie, Panie! drogi ustaw twoich, a bede jej strzegl az do konca.
\par 34 Daj mi rozum, abym strzegl zakonu twego, azebym go przestrzegal ze wszystkiego serca.
\par 35 Daj, abym chodzil sciezka przykazan twoich, gdyz w tem jest upodobanie moje.
\par 36 Naklon serce moje do swiadectw twoich, a nie do lakomstwa.
\par 37 Odwróc oczy moje, aby nie patrzaly na marnosc; na drodze twojej ozyw mie.
\par 38 Utwierdz wyrok twój sludze twemu, który sie oddal bojazni twojej.
\par 39 Oddal odemnie pohanbienie moje, którego sie boje; bo sady twoje dobre.
\par 40 Oto pragne rozkazan twoich; w sprawiedliwosci twojej ozyw mie.
\par 41 Niech na mie przyjda litosci twoje, Panie! i zbawienie twoje wedlug wyroku twego.
\par 42 Tak abym odpowiedz mógl dac sama rzecza temu, który mi uraga, gdyz ufam w slowie twojem.
\par 43 A nie wyjmuj z ust moich slowa najprawdziwszego; albowiem sadów twoich oczekuje.
\par 44 I bede strzegl zakonu twego zawsze, az na wieki wieczne.
\par 45 A ustawicznie bede chodzil na przestrzenstwie, bom sie dopytal rozkazan twoich.
\par 46 Owszem, bede mówil o swiadectwach twoich przed królmi, a nie bede zawstydzony.
\par 47 Bom sie rozkochal w przykazaniach twoich, którem umilowal.
\par 48 Przyloze i rece moje do rozkazan twoich, które miluje, a bede rozmyslal o ustawach twoich.
\par 49 Wspomnij na slowo wyrzeczone do slugi twego, któremes mie ubezpieczyl.
\par 50 Toc pociecha moja w utrapieniu mojem, ze mie wyrok twój ozywia.
\par 51 Pyszni bardzo sie ze mnie nasmiewaja; wszakze sie od zakonu twego nie uchylam.
\par 52 Bo pamietam na sady twoje wieczne, Panie! któremi sie ciesze.
\par 53 Strach mie ogarnal nad niezboznymi, którzy opuszczaja zakon twój.
\par 54 Sa mi ustawy twoje piesniami w domu pielgrzymstwa mego.
\par 55 Wspominam sobie i w nocy na imie twoje, Panie! i strzege zakonu twego.
\par 56 Toc mam z tego, ze przestrzegam przykazan twoich.
\par 57 Rzeklem: Panie! to jest czastka moja, przestrzegac slów twoich.
\par 58 Modle sie przed obliczem twojem ze wszystkiego serca; zmilujze sie nademna wedlug slowa twego.
\par 59 Uwazylem w myslach drogi moje, a obrócilem nogi moje ku swiadectwom twoim.
\par 60 Spiesze sie, a nie omieszkuje przestrzegac rozkazan twoich.
\par 61 Hufy niepoboznych zlupily mie; ale na zakon twój nie zapominam.
\par 62 O pólnocy wstaje, abym cie wyslawial w sadach sprawiedliwosci twojej.
\par 63 Jestem towarzyszem wszystkich, którzy sie ciebie boja, i tych, którzy przestrzegaja przykazan twoich.
\par 64 Panie! pelna jest ziemia milosierdzia twego; nauczze mie ustaw twoich.
\par 65 Laskawies postapil ze sluga twoim, Panie! wedlug slowa twego.
\par 66 Dobrego rozumu i umiejetnosci naucz mie; bom przykazaniom twoim uwierzyl.
\par 67 Pierwej nizem sie byl unizyl, bladzilem; ale teraz wyroku twego przestrzegam.
\par 68 Dobrys ty i dobrotliwy; nauczze mie ustaw twoich.
\par 69 Uknowali hardzi klamstwo przeciwko mnie; ale ja ze wszystkiego serca strzege przykazan twoich.
\par 70 Serce ich zatylo jako sadlo; ale sie ja zakonem twoim ciesze.
\par 71 Jest mi to ku dobremu, zem byl utrapiony, abym sie nauczyl ustaw twoich.
\par 72 Lepszy mi jest zakon ust twoich, nizeli tysiace zlota i srebra.
\par 73 Rece twoje uczynily mie, i wyksztaltowaly mie; dajze mi rozum, abym sie nauczyl przykazan twoich;
\par 74 Aby sie radowali bojacy sie ciebie, ujrzawszy mie, ze na slowo twoje oczekuje.
\par 75 Znam, Panie! iz sa sprawiedliwe sady twoje, a izes mie slusznie utrapil.
\par 76 Niechajze mie, prosze, ucieszy milosierdzie twoje wedlug wyroku twego, którys uczynil sludze twemu.
\par 77 Niechze na mie przyjda litosci twoje, abym zyl; bo zakon twój jest kochaniem mojem.
\par 78 Niech beda zawstydzeni pyszni, przeto, ze mie chytrze podwrócic chcieli; ale ja rozmyslac bede w przykazaniach twoich.
\par 79 Niech sie obróca do mnie, którzy sie ciebie boja, i którzy znaja swiadectwa twoje.
\par 80 Niech bedzie serce moje uprzejme przy ustawach twoich, abym nie byl zawstydzony.
\par 81 Teskni dusza moja po zbawieniu twojem, oczekuje na slowo twoje.
\par 82 Ustaly oczy moje, czekajac wyroku twego, gdy mówie: Kiedyz mie pocieszysz?
\par 83 Chociazem jest jako naczynie skórzane w dymie, wszakzem ustaw twoich nie zapomnial.
\par 84 Wielez bedzie dni slugi twego? kiedyz sad wykonasz nad tymi, którzy mie przesladuja?
\par 85 Pyszni pokopali mi doly, co nie jest wedlug zakonu twojego.
\par 86 Wszystkie przykazania twoje sa prawda; bez przyczyny mie przesladuja; ratujze mie.
\par 87 Bez mala mie juz wniwecz nie obrócili na ziemi; a wszakzem ja nie opuscil przykazan twoich.
\par 88 Wedlug milosierdzia twego ozyw mie, abym strzegl swiadectwa ust twoich.
\par 89 O Panie! slowo twoje trwa na wieki na niebie.
\par 90 Od narodu do narodu prawda twoja; ugruntowales ziemie, i stoi.
\par 91 Wedlug rozrzadzenia twego trwa to wszystko az do dnia tego; wszystko to zaiste jest ku sluzbie twojej.
\par 92 By byl zakon twój nie byl kochaniem mojem, dawnobym byl zginal w utrapieniu mojem.
\par 93 Na wieki nie zapomne na przykazania twoje, gdyzes mie w nich ozywil.
\par 94 Twójcim ja, zachowajze mie; bo przykazan twoich szukam.
\par 95 Czekaja na mie niezboznicy, aby mie zatracili; ale ja swiadectwa twoje uwazam.
\par 96 Wszelkiej rzeczy koniec widze; ale przykazanie twoje bardzo szerokie.
\par 97 O jakom sie rozmilowal zakonu twego! tak, iz kazdego dnia jest rozmyslaniem mojem.
\par 98 Nad nieprzyjaciól moich medrszym mie czynisz przykazaniem twojem; bo je mam ustawicznie przed soba.
\par 99 Nad wszystkich nauczycieli moich stalem sie rozumniejszym; bo swiadectwa twoje sa rozmyslaniem mojem.
\par 100 Nad starców jestem roztropniejszy; bo przykazan twoich przestrzegam.
\par 101 Od wszelkiej zlej drogi zawsciagam nogi swoje, abym strzegl slowa twego.
\par 102 Od sadów twoich nie odstepuje, przeto, ze ich ty mnie uczysz.
\par 103 O jako sa slodkie slowa twoje podniebieniu memu! nad miód sa slodsze ustom moim.
\par 104 Z przykazan twoich nabylem rozumu: przetoz mam w nienawisci wszelka scieszke obledliwa.
\par 105 Slowo twe jest pochodnia noga moim, a swiatloscia scieszce mojej.
\par 106 Przysieglem i uczynie temu dosyc, ze bede strzegl sadów sprawiedliwosci twojej.
\par 107 Jestem bardzo utrapiony; o Panie! ozyw mie wedlug slowa twego.
\par 108 Panie! dobrowolne sluby ust moich przyjmij prosze za wdzieczne, a sadów twoich naucz mie.
\par 109 Dusza moja jest w ustawicznem niebezpieczenstwie; wszakze na zakon twój nie zapominam.
\par 110 Sidlo na mie niezboznicy zastawili; lecz ja sie od przykazan twoich nie obladze.
\par 111 Za dziedzictwo wieczne wzialem swiadectwa twoje; bo sa radoscia serca mego.
\par 112 Naklonilem serca mego ku wykonywaniu ustaw twoich ustawicznie, i az do konca (zywota).
\par 113 Wymysly mam w nienawisci, a zakon twój miluje.
\par 114 Tys jest ucieczka moja, i tarcza moja; na slowo twoje oczekuje.
\par 115 Odstapciez odemnie zlosnicy, abym strzegl rozkazania Boga mojego.
\par 116 Utwierdzze mie wedlug slowa twego, abym zyl, a nie zawstydzaj mie w oczekiwaniu mojem.
\par 117 Podpieraj mie, abym byl zachowany, i rozmyslal w ustawach twoich ustawicznie.
\par 118 Podeptales wszystkich, którzy sie obladzili od ustaw twoich; albowiem jest klamliwa zdrada ich.
\par 119 Odrzucasz jako zuzelice wszystkich niezbozników ziemi; dla tego miluje swiadectwa twoje.
\par 120 Drzy od strachu przed toba cialo moje; bo sie sadów twoich lekam.
\par 121 Czynie sady i sprawiedliwosc: nie podawajze mie tym, którzy mi gwalt czynia.
\par 122 Zastap sam sluge twego ku dobremu, aby mie hardzi nie potloczyli.
\par 123 Oczy moje ustaly, czekajac na zbawienie twoje, i na wyrok sprawiedliwosci twojej.
\par 124 Obchodz sie z sluga twoim wedlug milosierdzia twego, a ustaw twoich naucz mie.
\par 125 Slugamci ja twój, dajze mi zrozumienie; abym umial swiadectwa twoje.
\par 126 Czasci juz, abys czynil Panie! albowiem wzruszono zakon twój.
\par 127 Dlatego umilowalem rozkazania twoje nad zloto, a nad zloto najwyborniejsze.
\par 128 Przeto, ze wszystkie przykazania twoje, wszystkie prawdziwe byc uznaje, a wszelkie sciezki obledliwe mam w nienawisci.
\par 129 Dziwne sa swiadectwa twoje; przetoz ich strzeze dusza moja.
\par 130 Poczatek slów twoich oswieca i daje rozum prostakom.
\par 131 Usta moje otwieram i dysze; albowiemem przykazan twoich pragnal.
\par 132 Wejzyjze na mie, a zmiluj sie nademna wealug prawa tych, którzy miluja imie twoje.
\par 133 Drogi moje utwierdz w slowie twojem, a niech nademna nie panuje zadna nieprawosc.
\par 134 Wybaw mie od ucisnienia ludzkiego, abym strzegl rozkazan twoich.
\par 135 Rozswiec nad sluga twoim oblicze twoje, a naucz mie ustaw twoich.
\par 136 Strumienie wód plyna z oczów moich dla tych, którzy nie strzegli zakonu twego.
\par 137 Sprawiedliwys ty, Panie! i prawdziwy w sadach twoich.
\par 138 Przykazales sprawiedliwe swiadectwa twoje, i wielce prawdziwe.
\par 139 Zniszczyla mie gorliwosc moja, iz zapominaja na slowo twoje nieprzyjaciele moi.
\par 140 Doskonale sa doswiadczone slowa twoje; dlatego sie sluga twój w nich rozkochal.
\par 141 Jam maluczki i wzgardzony; wszakze przykazan twoich nie zapominam.
\par 142 Sprawiedliwosc twoja sprawiedliwosc wieczna, a zakon twój prawda.
\par 143 Ucisk i utrapienie przyszlo na mie; przykazania twoje sa kochaniem mojem.
\par 144 Sprawiedliwosc swiadectw twoich trwa na wieki; daj mi rozum, a zyc bede.
\par 145 Wolam ze wszystkiego serca, wysluchajze mie, o Panie! a bede strzegl ustaw twoich.
\par 146 Wolam do ciebie, zachowajze mie, a bede pilen swiadectw twoich.
\par 147 Uprzedzam cie na switaniu i wolam, na slowo twoje oczekujac.
\par 148 Uprzedzaja straz nocna oczy moje, przeto, abym rozmyslal o wyrokach twoich.
\par 149 Panie! glos mój uslysz wedlug milosierdzia twego; wedlug sadu twego ozyw mie.
\par 150 Przyblizaja sie, którzy nasladuja zlosci, ci, którzy sie od zakonu twego oddalili.
\par 151 Bliskos ty jest, Panie! a wszystkie przykazania twoje sa prawda.
\par 152 Dawno to wiem o swiadectwach twoich, zes je na wieki ugruntowal.
\par 153 Obacz utrapienie moje, a wyrwij mie; bom na zakon twój nie zapomnial.
\par 154 Stan przy sprawie mojej, a obron mie; dla slowa twego ozyw mie.
\par 155 Dalekoc jest od niezbozników zbawienie; bo sie nie badaja o ustawach twoich.
\par 156 Wielkie sa litosci twoje, Panie! wedlug sadów twoich ozyw mie.
\par 157 Wielec jest przesladowców moich i nieprzyjaciól moich; wszakze od swiadectw twoich nie uchylam sie.
\par 158 Widzialem przestepców, i mierzialo mie to, ze wyroku twego nie przestrzegali.
\par 159 Obaczze Panie! iz rozkazania twoje miluje; wedlug milosierdzia twego ozyw mie.
\par 160 Najprzedniejsza rzecz slowa twego jest prawda, a na wieki trwa wszelki sad sprawiedliwosci twojej.
\par 161 Ksiazeta mie przesladuja bez przyczyny; wszakze slów twoich boi sie serce moje.
\par 162 Ja sie wesele z wyroku twego, tak jako ten, który znajduje wielkie korzysci.
\par 163 Ale nienawidze klamstwa, i brzydza sie niem; ale zakon twój miluje.
\par 164 Chwale cie siedm kroc przez dzien, dla sadów twoich sprawiedliwych.
\par 165 Pokój wielki dajesz tym, którzy miluja zakon twój, a nie doznawaja zadnego obrazenia.
\par 166 Panie! oczekuje zbawienia twego; a przykazania twoje wykonywam.
\par 167 Przestrzega dusza moja swiadectw twoich; albowiem je bardzo miluje.
\par 168 Przestrzegam przykazan twoich i swiadectw twoich; albowiem wszystkie drogi moje sa przed toba.
\par 169 Panie! niech sie przyblizy wolanie moje przed oblicze twoje; wedlug slowa twego daj mi zrozumienie.
\par 170 Niech przyjdzie prosba moja przed twarz twoje, a wedlug obietnicy twojej wyrwij mie.
\par 171 Chwale wydadza wargi moje, gdy mie nauczysz ustaw twoich.
\par 172 Opowiadac bedzie jezyk mój wyroki twoje; bo wszystkie przykazania twoje sa sprawiedliwosc.
\par 173 Niech mi bedzie na pomocy reka twoja, gdyzem sobie obral przykazania twoje.
\par 174 Panie! zbawienia twego pragne, a zakon twój jest kochaniem mojem.
\par 175 Zyc bedzie dusza moja, i bedzie cie chwalila, a sady twoje beda mi na pomocy.
\par 176 Bladze jako owca zgubiona, szukajze slugi twego; boc przykazan twoich nie zapominam.

\chapter{120}

\par 1 Piesn stopni. Wolalem do Pana w utrapieniu mojem, a wysluchal mie.
\par 2 Wyzwól, Panie! dusze moje od warg klamliwych, i od jezyka zdradliwego.
\par 3 Cóz ci da, albo coc za pozytek przyniesie jezyk zdradliwy?
\par 4 Który jest jako strzaly ostre mocarza, i jako wegle jalowcowe.
\par 5 Niestetyz mnie, zem tak dlugo gosciem w Mesech, a mieszkam w namiotach Kedarskich.
\par 6 Dlugo mieszka dusza moja miedzy tymi, którzy pokój maja w nienawisci.
\par 7 Jac radze do pokoju; ale gdy o tem mówie, oni do wojny.

\chapter{121}

\par 1 Piesn stopni. Oczy moje podnosze na góry, skadby mi pomoc przyszla.
\par 2 Pomoc moja jest od Pana, który stworzyl niebo i ziemie.
\par 3 Nie dopusci, aby sie zachwiac miala noga twoja; nie drzemiec stróz twój.
\par 4 Oto nie drzemie ani spi ten, który strzeze Izraela.
\par 5 Pan jest strózem twoim; Pan jest cieniem twoim po prawej rece twojej.
\par 6 We dnie slonce nie uderzy na cie, ani miesiac w nocy.
\par 7 Pan cie strzec bedzie od wszystkiego zlego; on duszy twojej strzec bedzie.
\par 8 Pan strzec bedzie wyjscia twego i wejscia twego, odtad az na wieki.

\chapter{122}

\par 1 Piesn stopni Dawidowa. Wesele sie z tego, ze mi powiedziano: Do domu Panskiego pójdziemy.
\par 2 Ze stanely nogi nasze w bramach twoich, o Jeruzalemie!
\par 3 O Jeruzalem pieknie pobudowane jako miasto w sobie wespól spojone!
\par 4 Bo tam wstepuja pokolenia, pokolenia Panskie, do swiadectwa Izraelowego, aby wyslawialy imie Panskie.
\par 5 Albowiem tam sa postawione stolice na sad, stolice domu Dawidowego.
\par 6 Zadajciez pokoju Jeruzalemowi, mówiac: Niech sie szczesci tym, którzy cie miluja.
\par 7 Niech bedzie pokój w basztach twoich, a uspokojenie w palacach twoich.
\par 8 Dla braci moich i dla przyjaciól moich teraz ci bede zadal pokoju.
\par 9 Dla domu Pana, Boga naszego, bede szukal twego dobrego.

\chapter{123}

\par 1 Piesn stopni. Do ciebie oczy moje podnosze, który mieszkasz w niebie.
\par 2 Oto jako oczy slug pilnuja rak panów swych, i jako oczy dziewki pilnuja reki pani swej, tak oczy nasze pogladaja na Pana, Boga naszego, az sie zmiluje nad nami.
\par 3 Zmiluj sie nad nami, Panie, zmiluj sie nad nami; bosmy bardzo nasyceni wzgarda.
\par 4 Bardzo jest nasycona dusza nasza posmiewiskiem bezboznych, i wzgarda pysznych.

\chapter{124}

\par 1 Piesn stopni Dawidowa. Gdyby byl Pan z nami nie byl, (powiedz teraz Izraelu!)
\par 2 Gdyby byl Pan z nami nie byl, gdy ludzie powstawali przeciwko nam:
\par 3 Tedycby nas byli zywo pozarli w rozpaleniu gniewu swego przeciwko nam;
\par 4 Tedycby nas byly wody zabraly a strumien porwalby byl dusze nasze;
\par 5 Tedycby byly porwaly dusze nasze one wody gwaltowne.
\par 6 Blogoslawiony Pan, który nas nie podal na lup zebom ich.
\par 7 Dusza nasza jako ptaszek uszla z sidla ptaszników; sidlo sie potargalo, a mysmy uszli.
\par 8 Wspomozenie nasze w imieniu Panskiem, który stworzyl niebo i ziemie.

\chapter{125}

\par 1 Piesn stopni. Którzy ufaja w Panu, sa jako góra Syon, która sie nie poruszy, ale na wieki zostaje.
\par 2 Jako okolo Jeruzalemu sa góry, tak Pan jest okolo ludu swego, od tego czasu az i na wieki.
\par 3 Albowiem nie zostanie laska niezbozników nad losem sprawiedliwych, by snac nie sciagneli sprawiedliwi rak swych ku nieprawosci.
\par 4 Dobrze czyn, Panie! dobrym, i tym, którzy sa uprzejmego serca.
\par 5 Ale tych, którzy sie udawaja krzyewemi drogami swemi, niech zapedzi Pan z tymi, którzy czynia nieprawosc; lecz pokój niech bedzie nad Izraelem.

\chapter{126}

\par 1 Piesn stopni. Gdy zas Pan nawrócil pojmanych z Syonu, bylismy jako ci, którym sie sni.
\par 2 Tedy byly napelnione weselem usta nasze, a jezyk nasz radoscia; tedy mówiono miedzy narodami: Wielmozne rzeczy Pan uczynil z nimi.
\par 3 Wielmozne rzeczy Pan uczynil z nami, z czegosmy sie bardzo uradowali.
\par 4 Przywrócze zas, o Panie! pojmanie nasze, jako strumienie na poludnie.
\par 5 Którzy siali ze lzami, zac beda z wykrzykaniem;
\par 6 Tam i sam chodzac z placzem rozsiewa lud drogie nasienie; ale zas przyszedlszy z radoscia znosic bedzie snopy swoje.

\chapter{127}

\par 1 Piesn stopni dla Salomona. Jezli Pan domu nie zbuduje, prózno pracuja ci, którzy go buduja; jezli Pan nie bedzie strzegl miasta, prózno czuje ten, który go strzeze.
\par 2 Prózno macie rano wstawac, dlugo siadac, i jesc chleb bolesci, poniewaz Pan umilowanemu swemu sen daje.
\par 3 Oto dziatki sa dziedzictwem od Pana, a plód zywota nagroda.
\par 4 Jako strzaly w reku mocarza, tak sa dziatki, które sie darza.
\par 5 Blogoslawiony maz, który niemi napelnil sajdak swój; nie beda zawstydzani, gdy sie w bramie rozpierac beda z nieprzyjaciolmi swymi.

\chapter{128}

\par 1 Piesn stopni. Blogoslawiony wszelki, który sie boi Pana, który chodzi drogami jego.
\par 2 Bo prace rak twoich pozywac bedziesz; blogoslawionym bedziesz, i bedziec sie dobrze dzialo.
\par 3 Zona twoja bedzie jako winna macica plodna po bokach domu twego; dziatki twoje jako latorosle oliwne okolo stolu twego.
\par 4 Oto takci bedzie ublogoslawiony maz, który sie boi Pana.
\par 5 Niechzec Pan blogoslawi z Syonu, abys patrzyl na dobro Jeruzalemskie po wszystkie dni zywota twego.
\par 6 I ogladal synów synów twoich, i pokój nad Izraelem.

\chapter{129}

\par 1 Piesn stopni. Bardzoc mie utrapili zaraz od mlodosci mojej, powiedz teraz Izraelu.
\par 2 Bardzoc mie utrapili od mlodosci mojej, wszakze mie nie przemogli.
\par 3 Po grzbiecie moim orali oracze, i dlugie przeganiali brózdy swoje.
\par 4 Ale Pan sprawiedliwy poprzecinal powrozy niezbozników.
\par 5 Zawstydzeni i nazad obróceni beda wszyscy, którzy Syon maja w nienawisci.
\par 6 Beda jako trawa na dachu, która pierwej, niz odrosnie, usycha.
\par 7 Z której zenca nie moze garsci swej napelnic; ani narecza swego ten, który wiaze snopy.
\par 8 I mimo idacy nie rzeka: Blogoslawienstwo Panskie niech bedzie z wami; albo: Blogoslawimy wam w imieniu Panskiem.

\chapter{130}

\par 1 Piesn stopni. Z glebokosci wolam do ciebie, o Panie!
\par 2 Panie! wysluchaj glos mój: naklon uszów twych do glosu prosb moich.
\par 3 Panie! bedzieszli nieprawosci upatrywal, Panie! któz sie zostoi?
\par 4 Alec u ciebie jest odpuszczenie, aby sie ciebie bano.
\par 5 Oczekuje na Pana; oczekuje dusza moja, i jeszcze oczekuje na slowo jego.
\par 6 Dusza moja oczekuje Pana, pilniej niz straz switania, która strzeze az do poranku.
\par 7 Oczekujze, Izraelu! na Pana; albowiem u Pana jest milosierdzie, a obfite u niego odkupienie.
\par 8 Onci sam odkupi Izraela od wszystkich nieprawosci jego.

\chapter{131}

\par 1 Piesn stopni Dawidowa. Panie! nie wynioslo sie serce moje, ani sie wyniosly oczy moje, anim sie kusil o rzeczy wielkie, albo wyzsze nad to, niz mi nalezy.
\par 2 Izalim nie polozyl i nie uspokoil duszy mojej, jako dziecie odstawione od matki swej? odstawionemu dziecieciu byla podobna we mnie dusza moja.
\par 3 Miejze nadzieje w Panu, o Izraelu! odtad az na wieki.

\chapter{132}

\par 1 Piesn stopni. Na Dawida pomnij, Panie! na wszystkie utrapienia jego.
\par 2 Który przysiagl Panu, a slub uczynil mocarzowi Jakóbowemu, mówiac:
\par 3 Zaiste nie wnijde do przybytku domu mego, i nie wstapie na poslanie loza mego;
\par 4 I nie pozwole snu oczom moim, ani powiekom moim drzemania,
\par 5 Dokad nie znajde miejsca dla Pana, na mieszkania mocarzowi Jakóbowemu.
\par 6 Oto uslyszawszy o niej w Efracie, znalezlismy ja na polach lesnych.
\par 7 Wnijdzmyz do przybytków jego, a klaniajmy sie u podnózka nóg jego.
\par 8 Powstanze Panie! a wnijdz do odpocznienia twego, ty, i skrzynia moznosci twojej.
\par 9 Kaplani twoi niech sie obloka w sprawiedliwosc, a swieci twoi nie sie rozraduja.
\par 10 Dla Dawida, slugi twego, nie odwracaj oblicza pomazanca twego.
\par 11 Przysiagl Pan Dawidowi prawde, a nie uchyli sie od niej, mówiac: Z owocu zywota twego posadze na stolicy twojej.
\par 12 Bedali strzegli synowie twoi przymierza mojego, i swiadectw moich, których ich naucze: tedy i synowie ich az na wieki beda siedzieli na stolicy twojej
\par 13 Albowiem obral Pan Syon, i upodobal go sobie na mieszkanie, mówiac:
\par 14 Toc bedzie odpocznienie moje az na wieki; tu bede mieszkal, bom go siebie upodobal.
\par 15 Zywnosc jego bede obficie blogoslawil, a ubogich jego nasyce chlebem.
\par 16 Kaplanów jego przyobloke zbawieniem, a swieci jego weselac sie, radowac sie beda.
\par 17 Tam sprawie, ze zakwitnie róg Dawidowy; tam zgotuje pochodnie pomazancowi memu.
\par 18 Nieprzyjaciól jego przyobloke wstydem; ale nad nim rozkwitnie sie korona jego.

\chapter{133}

\par 1 Piesn stopni Dawidowa. Oto jako rzecz dobra, i jako wdzieczna, gdy bracia zgodnie mieszkaja.
\par 2 Jest jako olejek najwyborniejszy wylany na glowe, sciekajacy na brode, na brode Aaronowa, sciekajacy az i na podolek szat jego.
\par 3 Jako rosa Hermon, która zstepuje na góry Syonskie; albowiem tam daje Pan blogoslawienstwo i zywot az na wieki.

\chapter{134}

\par 1 Piesn stopni. Ej nuz blogoslawcie Panu wszyscy sludzy Panscy, którzy stawacie w domu Panskim na kazda noc.
\par 2 Podnoscie rece wasze ku swiatnicy, a blogoslawcie Panu, mówiac:
\par 3 Niechajzec blogoslawi Pan z Syonu, który stworzyl niebo i ziemie.

\chapter{135}

\par 1 Halleluja. Chwalcie imie Panskie, chwalcie sludzy Panscy.
\par 2 Którzy stawacie w domu Panskim, w sieniach domu Boga naszego.
\par 3 Chwalciez Pana, albowiem to Pan dobry; spiewajciez imieniowi jego, boc jest wdzieczne.
\par 4 Albowiem sobie Jakóba Pan obral, i Izraela za wlasnosc swoje.
\par 5 Jac zaiste uznaje, iz wielki jest Pan, a Pan nasz jest nad wszystkich bogów.
\par 6 Wszystko co chce Pan, to czyni, na niebie i na ziemi, w morzu i we wszystkich przepasciach.
\par 7 Który czyni, ze wystepuja pary od konczyn ziemi; blyskawice i dzdze przywodzi, wywodzi wiatr z skarbów swoich;
\par 8 Który pobil pierworodztwa w Egipcie, od czlowieka az do bydlecia.
\par 9 Poslal znaki i cuda w posród ciebie, Egipcie! na Faraona i na wszystkich slug jego.
\par 10 Który porazil wiele narodów, a pobil królów moznych;
\par 11 Sehona, króla Amorejskiego, i Oga, króla Basanskiego, i wszystkie królestwa Chananejskie.
\par 12 I dal ziemie ich w dziedzictwo, w dziedzictwo Izraelowi, ludowi swemu.
\par 13 Panie! imie twoje na wieki; Panie! pamiatka twoja od narodu do narodu.
\par 14 Zaite Pan sadzic bedzie lud swój, a nad slugami swymi zmiluje sie.
\par 15 Ale balwany poganskie, srebro i zloto, sa robota rak ludzkich.
\par 16 Usta maja, a nie mówia, oczy maja, a nie widza;
\par 17 Uszy maja, a nie slysza, ani maja tchnienia w ustach swoich.
\par 18 Niech im podobni beda, którzy je robia, i wszyscy, którzy w nich ufaja.
\par 19 Domie Izraelski! blogoslawcie Panu; domie Aaronowy! blogoslawcie Panu.
\par 20 Domie Lewiego! blogoslawcie Panu, którzy sie boicie Pana, blogoslawcie Panu.
\par 21 Blogoslawiony Pan z Syonu, który mieszka w Jeruzalemie. Halleluja.

\chapter{136}

\par 1 Wyslawiajciez Pana, albowiem jest dobry; albowiem na wieki milosierdzie jego.
\par 2 Wyslawiajciez Boga nad bogami; albowiem na wieki milosierdzie jego.
\par 3 Wyslawiajciez Pana nad panami; albowiem na wieki milosierdzie jego;
\par 4 Tego, który sam czyni cuda wielkie; albowiem na wieki milosierdzie jego.
\par 5 Który madrze niebiosa uczynil; albowiem na wieki milosierdzie jego;
\par 6 Który rozciagnal ziemie na wodach; albowiem na wieki milosierdzie jego;
\par 7 Który uczynil swiatla wielkie; albowiem na wieki milosierdzie jego;
\par 8 Slonce, aby panowalo we dnie; albowiem na wieki milosierdzie jego;
\par 9 Miesiac i gwiazdy, aby panowaly w nocy; albowiem na wieki milosierdzie jego.
\par 10 Który porazil Egipczan na pierworodnych ich; albowiem na wieki milosierdzie jego.
\par 11 Który wywiódl Izraela z posrodku ich; albowiem na wieki milosierdzie jego;
\par 12 W rece mocnej i w ramieniu wyciagnionem; albowiem na wieki milosierdzie jego.
\par 13 Który rozdzielil morze Czerwone na rozdzialy; albowiem na wieki milosierdzie jego;
\par 14 I przeprowadzil lud Izraelski posrodkiem jego; albowiem na wieki milosierdzie jego.
\par 15 I wrzucil Faraona z wojskiem jego w morze Czerwone; albowiem na wieki milosierdzie jego.
\par 16 Który prowadzil lud swój przez puszcze; albowiem na wieki milosierdzie jego.
\par 17 Który porazil królów wielkich; albowiem na wieki milosierdzie jego;
\par 18 I pobil królów moznych; albowiem na wieki milosierdzie jego;
\par 19 Sehona; króla Amorejskiego; albowiem na wieki milosierdzie jego;
\par 20 I Oga, króla Basanskiego; albowiem na wieki milosierdzie jego.
\par 21 I dal ziemie ich w dziedzictwo; albowiem na wieki milosierdzie jego;
\par 22 W dziedzictwo Izraelowi, sludze swemu; albowiem na wieki milosierdzie jego.
\par 23 Który w unizeniu naszem pamieta na nas; albowiem na wieki milosierdzie jego.
\par 24 I wybawil nas od nieprzyjaciól naszych; albowiem na wieki milosierdzie jego.
\par 25 Który daje pokarm wszelkiemu cialu; albowiem na wieki milosierdzie jego.
\par 26 Wyslawiajciez Boga niebios; albowiem na wieki milosierdzie jego.

\chapter{137}

\par 1 Nad rzekami Babilonskiemi, tamesmy siadali i plakali, wspominajac na Syon.
\par 2 Na wierzbach, które sa w nim, zawieszalismy harfy nasze.
\par 3 A gdy nas tam pytali ci, którzy nas zawiedli w niewole, o slowa piesni, (chociazesmy byli zawiesili piesni radosci,)mówiac: Spiewajcie nam piesn z piesni Syonskich,
\par 4 Odpowiedzielismy: Jakoz mamy spiewac piesn Panska w ziemi cudzoziemców?
\par 5 Jezlize cie zapomne, o Jeruzalemie! niech zapomni sama siebie prawica moja.
\par 6 Niech przylgnie jezyk mój do podniebienia mego, jezlibym na cie nie pomnial, jezlibym nie przelozyl Jeruzalemu nad najwieksze wesele moje.
\par 7 Wspomnij, Panie! na synów Edomskich, i na dzien Jeruzalemski, w który mówili: Poburzcie, poburzcie az do gruntu w nim.
\par 8 O córko Babilonska! i ty bedziesz spustoszona. Blogoslawiony, któryc odda nagrode twoje, za to, cos nam zlego uczynila.
\par 9 Blogoslawiony, który pochwyci i roztraci dziatki twe o skale.

\chapter{138}

\par 1 Psalm Dawidowy. Wyslawiac cie bede, Panie! ze wszystkiego serca mego; przed bogami spiewac ci bede.
\par 2 Bede sie klanial ku kosciolowi twemu swietemu, i bede wyslawial imie twoje dla milosierdzia twego, i dla prawdy twojej; bos nade wszystko uwielbil imie twoje i wyroki twoje.
\par 3 W dzien, któregom cie wzywal, wysluchales mie, a posililes moca dusze moje.
\par 4 Wyslawiac cie beda, Panie! wszyscy królowie ziemi, gdy uslysza wyroki ust twoich.
\par 5 I beda spiewali o drogach Panskich, a iz wielka jest chwala Panska.
\par 6 A choc wywyzszony jest Pan, wszakze na unizonego patrzy, a wysokomyslnego z daleka poznaje.
\par 7 Jezlibym chodzil w posród utrapienia, ozywisz mie; przeciw popedliwosci nieprzyjaciól moich wyciagniesz reke twoje, a prawica twoja wyswobodzi mie.
\par 8 Pan wszystko za mie wykona. O Panie! milosierdzie twoje trwa na wieki; sprawy rak twoich nie opuscisz.

\chapter{139}

\par 1 Przedniejszemu spiewakowi psalm Dawidowy. Panie! doswiadczyles i doznales mie.
\par 2 Tedy znasz siedzenie moje, i powstanie moje, wyrozumiewasz mysli moje z daleka.
\par 3 Tys chodzenie moje i lezenie moje ogarnal, swiadomes wszystkich dróg moich.
\par 4 Nim przyjdzie slowo na jezyk mój, oto Panie! ty to wszystko wiesz.
\par 5 Z tylu i z przodku otoczyles mie, a polozyles na mie reke twoje.
\par 6 Dziwniejsza umiejetnosc twoja nad dowcip mój; wysoka jest, nie moge jej pojac.
\par 7 Dokad ujde przed duchem twoim? a dokad przed obliczem twojem ucieke?
\par 8 Jezlibym wstapil do nieba, jestes tam; i jezlibym sobie poslal w grobie, i tames przytomny.
\par 9 Wziallibym skrzydla rannej zorzy, abym mieszkal na koncu morza,
\par 10 I tamby mie reka twoja prowadzila, a dosieglaby mie prawica twoja.
\par 11 Albo rzekllibym: Wzdyc ciemnosci zakryja mie; alec i noc jest swiatlem okolo mnie,
\par 12 Gdyz i ciemnosci nic nie zakryja przed toba; owszem tobie noc jako dzien swieci; ciemnoscic sa jako swiatlosc.
\par 13 Ty zaiste w nocy masz nerki moje; okryles mie w zywocie matki mojej.
\par 14 Wyslawiam cie dlatego, ze sie zdumiewam strasznym i dziwnym sprawom twoim, a dusza moja zna je wybornie.
\par 15 Nie zataila sie zadna kosc moja przed toba, chociazem byl uczyniony w skrytosci, i misternie zlozony w niskosciach ziemi.
\par 16 Niedoskonaly plód ciala mego widzialy oczy twoje; w ksiegi twoje wszystkie czlonki moje wpisane sa, i dni, w których ksztaltowane byly, gdy jeszcze zadnego z nich nie bylo.
\par 17 Przetoz o jako drogie sa u mnie mysli twoje, Boze! a jako ich jest wielka liczba.
\par 18 Jezlibym je chcial zliczyc, nad piasek rozmnozyly sie; ocuceli sie, jeszczem ci ja z toba.
\par 19 Zabillibys, o Boze! niezboznika, tedycby mezowie krwawi odstapili odemnie;
\par 20 Którzy mówia przeciwko tobie obrzydlosci, którzy prózno wynosza nieprzyjaciól twoich.
\par 21 Izali tych, którzy cie w nienawisci maja, o Panie! niemam w nienawisci? a ci, którzy przeciwko tobie powstawaja, izaz mi nie omierzli?
\par 22 Glówna nienawiscia nienawidze ich, a mam ich za nieprzyjaciól.
\par 23 Wyszpieguj mie, Boze! a poznaj serce moje; doswiadcz mie, a poznaj mysli moje,
\par 24 I obacz, jezli droga odpornosci jest we mnie, a prowadz mie droga wieczna.

\chapter{140}

\par 1 Przedniejszemu spiewakowi psalm Dawidowy.
\par 2 Wyrwij mie, Panie! od czlowieka zlego, od meza okrutnego strzez mie;
\par 3 Którzy mysla zle rzeczy w sercu, a na kazdy dzien zbieraja sie na wojne.
\par 4 Zaostrzaja jezyk swój, jako waz; jad zmij pod wargami ich. Sela.
\par 5 Zachowaj mie, Panie! od rak bezboznika; od meza okrutnego strzez mie, którzy myslili podwrócic nogi moje.
\par 6 Hardzi na mie zastawili sidlo, i powrozy; rozciagneli sieci przy scieszce, a sidla swe zastawili na mie. Sela.
\par 7 Rzeklem Panu: Tys jest Bóg mój! wysluchajze, Panie! glos modlitw moich.
\par 8 O Panie, Panie mocy zbawienia mego, który przykrywasz glowe moje w dzien bitwy!
\par 9 Nie dawaj, Panie! bezboznemu czego zada; ani mysli jego zlej góry nie dawaj, zeby sie nie podniósl. Sela.
\par 10 A wodza tych, którzy mie obstapili, nieprawosc warg ich niech ich okryje.
\par 11 Niech na nich spadna wegle rozpalone; do ognia niech wrzuceni beda, i do dolów glebokich, skadby nie powstali.
\par 12 Potwarca nie bedzie utwierdzony na ziemi, a maz okrutny zloscia ulowiony bedac upadnie.
\par 13 Wiem, ze Pan uczyni sad utrapionemu, i pomste nedznych.
\par 14 A tak sprawiedliwi beda wyslawiac imie twoje, a szczerzy beda mieszkac przed obliczem twojem.

\chapter{141}

\par 1 Piesn Dawidowa. Panie! wolam do ciebie, pospiesz sie do mnie: posluchaj glosu mego, gdy wolam do ciebie.
\par 2 Niech bedzie przyjemna modlitwa moja, jako kadzidlo przed obliczem twoim, a podnoszenie rak moich jako ofiara wieczorna.
\par 3 Panie! polóz straz ustom moim; strzez drzwi warg moich.
\par 4 Nie nachylaj serca mego do zlej rzeczy, abym nie czynil spraw niepoboznych z mezami czyniacymi nieprawosc, i zebym sie nie karmil rozkoszami ich.
\par 5 Niech mie bije sprawiedliwy, a przyjme to za milosierdzie; i niech mie gromi, a bedzie mi to za najwyborniejszy olejek, który nie zarazi glowy mojej; albowiem jeszczec modlitwa moja platna bedzie przeciwko zlosci ich.
\par 6 Niech beda zrzuceni do miejsc opoczystych sedziowie ich, aby slyszeli slowa moje, ze byly wdzieczne.
\par 7 Jako gdyby kto rabal i lupal drwa na ziemi, tak sie rozlatuja kosci nasze az do ust grobowych.
\par 8 Ale do ciebie, Panie, Panie! podnosze oczy moje; w tobie ufam, nie odpychaj duszy mojej.
\par 9 Strzez mie od sidla, które na mie zastawili, i od sidel czyniacych nieprawosc.
\par 10 Niech wpadna razem w sieci swoje niepobozni, a ja za tem przemine.

\chapter{142}

\par 1 Piesn wyuczajaca Dawidowa, gdy byl w jaskini, modlitwa jego.
\par 2 Glosem moim do Pana wolam; glosem moim Panu sie modle.
\par 3 Wylewam przed obliczem jego zadlosc moje, a utrapienie moje przed oblicznoscia jego oznajmuje.
\par 4 Gdy bywa scisniony duch mój we mnie, ty znasz scieszke moje; na drodze, która chodze, ukryli na mie sidlo.
\par 5 Ogladamli sie na prawa strone, a przypatruje sie, niemasz ktoby mie znal; zginela ucieczka moja, niemasz ktoby sie ujal o dusze moje.
\par 6 Panie! do ciebie wolam, mówiac: Tys nadzieja moja, tys dzial mój w ziemi zyjacych.
\par 7 Posluchaj pilnie wolania mego, bom bardzo znedzony; wyrwij mie od tych, którzy mie przesladuja, albowiem sa mocniejszymi nad mie.
\par 8 Wywiedzze z ciemnicy dusze moje, abym chwalil imie twoje; obstapia mie sprawiedliwi, gdy mi dobrodziejstwo uczynisz.

\chapter{143}

\par 1 Psalm Dawidowy. Panie! wysluchaj modlitwe moje, a przyjmij w uszy prosby moje; dla prawdy twojej wysluchaj mie i dla sprawiedliwosci twojej.
\par 2 A nie wchodz w sad z sluga twoim; albowiem nie bedzie usprawiedliwiony przed obliczem twoim zaden zyjacy.
\par 3 Gdyz przesladuje nieprzyjaciel dusze moje, potarl równo z ziemia zywot mój; sprawil to, ze musze mieszkac w ciemnosciach, jako ci, którzy z dawna pomarli.
\par 4 I scisniony jest we mnie duch mój, a we wnetrznosciach moich niszczeje serce moje.
\par 5 Wspominam sobie dni dawne, i rozmyslam o wszystkich sprawach twoich, i uczynki rak twoich rozbieram.
\par 6 Wyciagam rece moje ku tobie; dusza moja, jako sucha ziemia, ciebie pragnie. Sela.
\par 7 Pospiesz sie, a wysluchaj mie, Panie! ustaje duch mój; nie ukrywajze oblicza twego przedemna; bomci podobny zstepujacym do grobu.
\par 8 Spraw, abym rano slyszal milosierdzie twoje, bo w tobie ufam; oznajmij mi droge, którabym mial chodzic; bo do ciebie podnosze dusze moje.
\par 9 Wyrwij mie od nieprzyjaciól moich, Panie! do ciebie sie uciekam.
\par 10 Naucz mie czynic wole twoje, albowiemes ty Bóg mój; duch twój dobry niech mie prowadzi po ziemi prawej.
\par 11 Dla imienia twego, Panie!ozyw mie; dla sprawiedliwosci twojej wywiedz z utrapienia dusze moje.
\par 12 I dla milosierdzia twego wytrac nieprzyjaciól moich, a wygladz wszystkich przeciwników duszy mojej; bom ja sluga twój.

\chapter{144}

\par 1 Piesn Dawidowa. Blogoslawiony Pan, skala moja, który cwiczy rece moje do bitwy, a palce moje do wojny.
\par 2 Milosierdziem mojem, i twierdza moja, ucieczka moja, wybawicielem moim, i tarcza moja on mi jest, przetoz w nim ufam; onci podbija pod mie lud mój.
\par 3 Panie! cóz jest czlowiek, ze nan masz baczenie? a syn czlowieczy, ze go sobie powazasz?
\par 4 Czlowiek marnosci jest podobny; dni jego jako cien pomijajacy.
\par 5 Panie! naklon niebios twoich, a zstap; dotknij sie gór, a zakurza sie.
\par 6 Zablysnij blyskawica, a rozprosz ich; pusc strzaly twoje, a poraz ich.
\par 7 Sciagnij reke swa z wysokosci; wybaw mie, a wyrwij mie z wód wielkich, z reki cudzoziemców.
\par 8 Których usta klamstwo mówia, a prawica ich, prawica omylna.
\par 9 Boze! piesn nowa tobie zaspiewam; na lutni, i na instrumencie o dziesieciu stronach spiewac ci bede.
\par 10 Bóg daje zwyciestwo królom, a Dawida, sluge swego, wybawia od miecza srogiego.
\par 11 Wybawze mie, a wyrwij mie z reki cudzoziemców, których usta mówia klamstwo, a prawica ich prawica omylna;
\par 12 Aby synowie nasi byli jako szczepy rosnace w mlodosci swojej, a córki nasze, jako kamienie wegielne, wyciosane w budynku koscielnym.
\par 13 Szpizarnie nasze pelne niech wydawaja wszelakie potrzeby; trzody nasze niech rodza tysiace, niech rodza dziesiec tysiecy w oborach naszych.
\par 14 Woly nasze niech beda tluste; niech nie bedzie wtargnienia, ani zajecia, ani narzekania po ulicach naszych.
\par 15 Blogoslawiony lud, któremu sie tak dzieje. Blogoslawiony lud, którego Bogiem jest Pan.

\chapter{145}

\par 1 Chwalebna piesn Dawidowa. Wywyzszac cie bede, Boze mój, królu mój! i blogoslawic bede imieniowi twemu na wieki wieków.
\par 2 Na kazdy dzien blogoslawic cie bede, a chwalic imie twoje na wieki wieków.
\par 3 Pan wielki jest i bardzo chwalebny, a wielkosc jego nie moze byc doscigniona.
\par 4 Naród narodowi wychwalac bedzie sprawy twoje, a mocy twoje opowiadac beda.
\par 5 Ozdobe chwaly wielmoznosci twojej, i dziwne twe sprawy wyslawiac bede.
\par 6 I moc strasznych uczynków twoich oglaszac beda, i ja zacnosc twoje opowiadac bede,
\par 7 Pamiec obfitej dobroci twojej wyslawiac, o sprawiedliwosci twojej spiewac beda, mówiac:
\par 8 Dobrotliwy i milosierny jest Pan, nierychly do gniewu, i wielkiego milosierdzia.
\par 9 Dobryc jest Pan wszystkim, a milosierdzie jego nad wszystkiem sprawami jego.
\par 10 Niech cie wyslawiaja, Panie! wszystkie sprawy twoje, a swieci twoi niech ci blogoslawia.
\par 11 Slawe królestwa twego niech opowiadaja, a o moznosci twojej niech rozmawiaja;
\par 12 Aby oznajmili synom ludzkim mocy jego, a chwale i ozdobe królestwa jego.
\par 13 Królestwo twoje jest królestwo wszystkich wieków, a panowanie twoje nie ustaje nad wszystkimi narodami.
\par 14 Trzyma Pan wszystkich upadajacych, a podnosi wszystkich obalonych.
\par 15 Oczy wszystkich w tobie nadzieje maja, a ty im dajesz pokarm ich czasu swojego.
\par 16 Otwierasz reke twoje, a nasycasz wszystko, co zyje, wedlug upodobania twego.
\par 17 Sprawiedliwy jest Pan we wszystkich drogach swoich, i milosierny we wszystkich sprawach swoich.
\par 18 Bliski jest Pan wszystkim, którzy go wzywaja, wszystkim, którzy go wzywaja w prawdzie.
\par 19 Wole tych czyni, którzy sie go boja, a wolanie ich wysluchiwa, i ratuje ich.
\par 20 Strzeze Pan wszystkich, którzy go miluja; ale wszystkich niepoboznych wytraci.
\par 21 Chwale Panska wyslawiac beda usta moje; a blogoslawic bedzie wszelkie cialo imie swiete jego na wieki wieków.

\chapter{146}

\par 1 Halleluja.
\par 2 Chwal, duszo moja! Pana. Chwalic bede Pana, pókim zyw; bede spiewal Bogu memu, póki mie staje.
\par 3 Nie ufajcie w ksiazetach, ani w zadnym synu ludzkim, w którym nie masz wybawienia.
\par 4 Wynijdzie duch jego, i nawróci sie do ziemi swojej; w onze dzien zgina wszystkie mysli jego.
\par 5 Blogoslawiony, którego Bóg Jakóbowy jest pomocnikiem, którego nadzieja jest w Panu, Bogu jego;
\par 6 Który uczynil niebo, i ziemie, morze, i wszystko, co w nich jest, który przestrzega prawdy az na wieki;
\par 7 Który czyni sprawiedliwosc ukrzywdzonym, i daje chleb zglodnialym; Pan rozwiazuje wiezniów.
\par 8 Pan otwiera oczy slepych; Pan podnosi upadlych; Pan miluje sprawiedliwych.
\par 9 Pan strzeze przychodniów, sierotce i wdowie pomaga; ale droge niepoboznych podwraca.
\par 10 Pan bedzie królowal na wieki; Bóg twój, o Syonie! od narodu do narodu. Halleluja.

\chapter{147}

\par 1 Chwalcie Pana; albowiem dobra rzecz jest, spiewac Bogu naszemu; albowiem to wdzieczna i przystojna jest chwala.
\par 2 Pan Jeruzalem buduje, a rozproszonego Izraela zgromadza.
\par 3 Który uzdrawia skruszonych na sercu, a zawiazuje bolesci ich.
\par 4 Który rachuje liczbe gwiazd, a kazda z nich imieniem jej nazywa.
\par 5 Wielki jest Pan nasz, i wielki w mocy; rozumienia jego niemasz liczby.
\par 6 Pan pokornych podnosi; ale niepoboznych az ku ziemi uniza.
\par 7 Spiewajciez Panu z chwala; spiewajcie Bogu naszemu na harfie;
\par 8 Który okrywa niebiosa oblokami, a deszcz ziemi gotuje: który czyni, ze rosnie trawa po górach;
\par 9 Który daje bydlu pokarm ich, i kruczetom mlodym, które wolaja do niego.
\par 10 Nie kocha sie w mocy konskiej, ani sie kocha w goleniach meskich.
\par 11 Kocha sie Pan w tych, którzy sie go boja, a którzy ufaja w milosierdziu jego.
\par 12 Chwalze, Jeruzalemie! Pana; chwalze, Syonie! Boga twego.
\par 13 Albowiem on umacnia zawory bram twoich, a blogoslawi synów twoich w posrodku ciebie.
\par 14 On czyni pokój w granicach twoich, a najwyborniejsza pszenica nasyca cie.
\par 15 On wysyla slowo swe na ziemie; bardzo predko biezy wyrok jego.
\par 16 On daje snieg jako welne, szron jako popiól rozsypuje.
\par 17 Rzuca lód swój jako bryly; przed zimnem jego któz sie ostoi?
\par 18 Posyla slowo swoje, i roztapia je; powienie wiatrem swym, a rozlewaja wody.
\par 19 Oznajmuje slowo swe Jakóbowi, ustawy swe i sady swe Izraelowi.
\par 20 Nie uczynil tak zadnemu narodowi; przetoz nie poznali sadów jego. Halleluja.

\chapter{148}

\par 1 Halleluja. Chwalcie Pana na niebiosach; chwalciez go na wysokosciach.
\par 2 Chwalcie go wszyscy Aniolowie jego; chwalcie go wszystkie wojska jego.
\par 3 Chwalcie go slonce i miesiacu; chwalcie go wszystkie jasne gwiazdy.
\par 4 Chwalcie go niebiosa nad niebiosami, i wody, które sa nad niebem.
\par 5 Chwalcie imie Panskie; albowiem on rozkazal, a stworzone sa.
\par 6 I wystawil je na wieki wieczne; zalozyl im kres, którego nie przestepuja.
\par 7 Chwalcie Pana na ziemi, smoki i wszystkie przepasci.
\par 8 Ogien i grad, snieg i para, wiatr gwaltowny, wykonywajacy rozkaz jego;
\par 9 Góry, i wszystkie pagórki, drzewa rodzaje, i wszystkie cedry;
\par 10 Zwierzeta, i wszystko bydlo, gadziny, i ptastwo skrzydlaste.
\par 11 Królowie ziemscy, i wszystkie narody; ksiazeta i wszyscy sedziowie ziemi;
\par 12 Mlodziency, takze i panny, starzy i mlodzi,
\par 13 Chwalcie imie Panskie; albowiem wywyzszone jest imie jego samego, a chwala jego nad ziemia i niebem.
\par 14 I wywyzszyl róg ludu swego, chwale wszystkich swietych jego, mianowicie synów Izraelskich, ludu jemu najblizszego. Halleluja.

\chapter{149}

\par 1 Halleluja. Spiewajcie Panu piesn nowa; chwala jego niechaj zabrzmi w zgromadzeniu swietych.
\par 2 Wesel sie, Izraelu! w Twórcy swoim; synowie Syonscy! radujcie sie w królu swoim.
\par 3 Chwalcie imie jego na piszczalkach; na bebnie i na harfie grajcie mu.
\par 4 Albowiem sie kocha Pan w ludu swym; pokornych zbawieniem uwielbia.
\par 5 Radowac sie beda swieci w chwale Bozej, a spiewac beda w pokojach swych.
\par 6 Wyslawiania Boze beda w ustach ich, a miecz na obie strony ostry w rekach ich,
\par 7 Aby wykonywali pomste nad poganami, a karali narody;
\par 8 Aby wiazali petami królów ich, a szlachte ich okowami zelaznemi;
\par 9 Aby postapili z nimi wedlug prawa zapisanego.Tac jest slawa wszystkich swietych jego. Halleluja.

\chapter{150}

\par 1 Halleluja. Chwalcie Boga w swiatnicy jego; chwalcie go na rozpostarciu mocy jego.
\par 2 Chwalcie go ze wszelkiej mocy jego; chwalcie go wedlug wielkiej dostojnosci jego.
\par 3 Chwalcie go na glosnych trabach; chwalcie go na lutni i na harfie.
\par 4 Chwalcie go na bebnie, i na piszczalce; chwalcie go stronach i na organach.
\par 5 Chwalcie go na cymbalach glosnych; chwalcie go cymbalach krzykliwych.
\par 6 Niech wszelki duch chwali Pana! Halleluja.


\end{document}