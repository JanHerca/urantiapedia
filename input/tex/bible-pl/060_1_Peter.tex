\begin{document}

\title{1 List św. Piotra}


\chapter{1}

\par 1 Piotr, Apostol Jezusa Chrystusa, przychodniom rozproszonym w Poncie, w Galacyi, w Kapadocyi, w Azji i w Bitynii;
\par 2 Wybranym wedlug przejrzenia Boga Ojca przez poswiecenie Ducha, ku posluszenstwu i pokropieniu krwi Jezusa Chrystusa. Laska wam i pokój niech bedzie rozmnozony.
\par 3 Blogoslawiony niech bedzie Bóg i Ojciec Pana naszego, Jezusa Chrystusa, który wedlug wielkiego milosierdzia swego odrodzil nas ku nadziei zywej przez zmartwychwstanie Jezusa Chrystusa od umarlych,
\par 4 Ku dziedzictwu nieskazitelnemu i niepokalanemu, i niezwiedlemu, w niebiesiech dla was zachowanemu,
\par 5 Którzy moca Boza strzezeni bywacie przez wiare ku zbawieniu, które zgotowane jest, aby bylo objawione czasu ostatecznego.
\par 6 W czem weselicie sie teraz maluczko, (jezlize potrzeba) zasmuceni w rozmaitych pokusach,
\par 7 Aby doswiadczenie wiary waszej daleko drozsze niz zloto, które ginie, którego jednak przez ogien doswiadczaja, znalezione bylo wam ku chwale i ku czci, i ku slawie w objawienie Jezusa Chrystusa,
\par 8 Którego nie widziawszy, milujecie, którego teraz nie widzac, wszakze wen wierzac, weselicie sie radoscia niewymowna i chwalebna,
\par 9 Odnoszac koniec wiary waszej: zbawienie dusz.
\par 10 O którem zbawieniu wywiadywali sie i badali sie prorocy, którzy o tej lasce, która na was przyjsc miala, prorokowali.
\par 11 Badajac sie, na który albo na jaki czas objawial Duch Chrystusowy, który w nich byl, swiadczac pierwej o utrapieniach, które mialy przyjsc na Chrystusa i o wielkiej za tem chwale.
\par 12 Którym objawione jest, iz nie samym sobie, ale nam tem uslugiwali, co wam teraz zwiastowano przez tych, którzy wam kazali Ewangielije przez Ducha Swietego z nieba zeslanego, na które rzeczy pragna patrzyc Aniolowie.
\par 13 Przetoz przepasawszy biodra mysli waszej i trzezwymi bedac, doskonala miejcie nadzieje ku tej lasce, która wam dana bedzie w objawienie Jezusa Chrystusa,
\par 14 Jako synowie posluszni, którzy sie nie przypodobywacie przeszlym w nieumiejetnosci waszej pozadliwosciom;
\par 15 Ale jako ten, który was powolal, swiety jest, i wy badzcie swietymi we wszelkiem obcowaniu,
\par 16 Dlatego ze napisano: Swietymi badzcie, izem Ja jest swiety.
\par 17 A poniewaz Ojcem nazywacie tego, który bez braku osób kazdego sadzi wedlug uczynku, patrzciez, abyscie w bojazni czas pielgrzymowania waszego trawili,
\par 18 Wiedzac, iz nie skazitelnemi rzeczami, srebrem albo zlotem, wykupieni jestescie od marnego obcowania waszego, od ojców podanego.
\par 19 Ale droga krwia, jako baranka niewinnego i niepokalanego, Chrystusa;
\par 20 Przejrzanego przed zalozeniem swiata, a objawionego czasów ostatecznych dla was,
\par 21 Którzy przez niego wierzycie w Boga, który go wzbudzil od umarlych i dal mu chwale, aby wiara i nadzieja wasza byla w Bogu.
\par 22 Oczyszczajac dusze wasze w posluszenstwie prawdy przez Ducha Swietego ku nieobludnej braterskiej milosci, z czystego serca jedni drugich milujcie uprzejmie,
\par 23 Odrodzeni bedac nie z nasienia skazitelnego, ale z nieskazitelnego przez slowo Boze zywe i trwajace na wieki.
\par 24 Poniewaz wszelkie cialo jest jako trawa i wszelka chwala czlowieka jako kwiat trawy; uwiedla trawa i kwiat jej opadl;
\par 25 Ale slowo Panskie trwa na wieki. A toc jest slowo, które wam jest zwiastowane.

\chapter{2}

\par 1 Przetoz zlozywszy wszelka zlosc i wszelka zdrade, i oblude, i zazdrosc, i wszelakie obmowiska,
\par 2 Jako dopiero narodzone niemowlatka, szczerego mleka slowa Bozego pozadajcie, abyscie przez nie urosli,
\par 3 Jezliscie tylko skosztowali, ze dobrotliwy jest Pan.
\par 4 Do którego przystepujac, do kamienia zywego, acz od ludzi odrzuconego, ale od Boga wybranego i kosztownego,
\par 5 I wy jako zywe kamienie budujcie sie w dom duchowny, w kaplanstwo swiete, ku ofiarowaniu duchowych ofiar, przyjemnych Bogu przez Jezusa Chrystusa.
\par 6 A przetoz mówi Pismo: Oto klade na Syonie kamien narozny wegielny, wybrany, kosztowny; a kto w niego uwierzy, nie bedzie zawstydzony.
\par 7 Wam tedy wierzacym jest uczciwoscia, ale nieposlusznym, kamien, który odrzucili budujacy, ten sie stal glowa wegielna,
\par 8 I kamieniem obrazenia, i opoka zgorszenia tym, którzy sie obrazaja o slowo, nie wierzac, na co tez wystawieni sa.
\par 9 Ale wy jestescie rodzajem wybranym, królewskim kaplanstwem, narodem swietym, ludem nabytym, abyscie opowiadali cnoty tego, który was powolal z ciemnosci ku dziwnej swojej swiatlosci.
\par 10 Którzyscie niekiedy byli nie ludem, alescie teraz ludem Bozym; którzyscie niekiedy nie dostapili byli milosierdzia, alescie teraz milosierdzia dostapili.
\par 11 Najmilsi! prosze was, abyscie sie jako przychodniowie i goscie wstrzymywali od cielesnych pozadliwosci, które walcza przeciwko duszy,
\par 12 Obcowanie wasze majac poczciwe miedzy poganami, aby zamiast tego, w czem was pomawiaja jako zloczynców, dobrym sie uczynkom waszym przypatrujac, chwalili Boga w dzien nawiedzenia.
\par 13 Badzciez tedy poddani wszelkiemu ludzkiemu urzedowi dla Pana, badz królowi, jako najwyzszemu,
\par 14 Badz przelozonym, jako od niego poslanym ku pomscie zle czyniacych, a ku chwale dobrze czyniacych.
\par 15 Albowiem taka jest wola Boza, abyscie dobrze czyniac, usta zatkali nieumiejetnosci glupich ludzi.
\par 16 (Badzciez) jako wolni, a nie jako ci, którzy wolnosc zaslona zlosci maja, ale jako sludzy Bozy.
\par 17 Wszystkich czcijcie, braterstwo milujcie, Boga sie bójcie, króla w uczciwosci miejcie.
\par 18 Sludzy! badzcie poddani panom we wszelakiej bojazni, nie tylko dobrym i bacznym, ale i dziwnym.
\par 19 Boc to jest laska, jezli kto dla sumienia Bozego ponosi frasunki, cierpiac bezwinnie.
\par 20 Bo cóz jest za chwala, jezlibyscie grzeszac, cierpliwie znosili, by was i piesciami bito? Ale jezli dobrze czyniac i cierpiac znosicie, to jest laska u Boga.
\par 21 Albowiem na to tez powolani jestescie, poniewaz i Chrystus cierpial za was, zostawiwszy wam przyklad, abyscie nasladowali stóp jego.
\par 22 Który grzechu nie uczynil, ani znaleziona jest zdrada w ustach jego.
\par 23 Któremu gdy zlorzeczono, nie odzlorzeczyl; gdy cierpial, nie grozil, ale poruczyl krzywde temu, który sprawiedliwie sadzi.
\par 24 Który grzechy nasze na ciele swoim zaniósl na drzewo, abysmy obumarlszy grzechom sprawiedliwosci zyli, którego sinoscia uzdrowieni jestescie.
\par 25 Albowiemescie byli jako owce bladzace; ale teraz jestescie nawróceni do pasterza i biskupa dusz waszych.

\chapter{3}

\par 1 Takze i zony! badzcie poddane mezom swoim, aby i ci, którzy nie wierza slowu, przez pobozne obcowanie zon, bez slowa byli pozyskani,
\par 2 Obaczywszy czyste w bojazni Bozej obcowanie wasze.
\par 3 Których ochedostwo niech bedzie nie ono zwierzchne, w splecieniu wlosów i oblozeniu sie zlotem, albo w ubieraniu sie w szaty:
\par 4 Ale on skryty serdeczny czlowiek, zalezacy w nieskazeniu cichego i spokojnego ducha, który jest przed obliczem Bozem kosztowny.
\par 5 Albowiem tak niekiedy i one swiete malzonki, które nadzieje mialy w Bogu, zdobily sie, bedac poddane mezom swoim.
\par 6 Jako Sara byla posluszna Abrahamowi nazywajac go panem; której wy stalyscie sie córkami, gdy dobrze czynicie, nie bojac sie zadnego postrachu.
\par 7 Takze i wy, mezowie! mieszkajcie z niemi umiejetnie, a jako mdlejszemu naczyniu niewiesciemu oddawajcie uczciwosc, jako tez spóldziedziczkom laski zywota, aby sie modlitwy wasze nie przerywaly.
\par 8 A na koniec wszyscy badzcie jednomyslni, spólcierpiacy doleglosci, braterstwo milujacy, milosierni i dobrotliwi,
\par 9 Nie oddawajac zlego za zle, ani lajania za lajanie, lecz przeciwnym obyczajem dobrorzeczac, gdyz wiecie, iz na to powolani jestescie, abyscie blogoslawienstwo odziedziczyli.
\par 10 Albowiem kto chce zywot milowac i ogladac dni dobre, niech pohamuje jezyka swego od zlego, a usta jego niech nie mówia zdrady;
\par 11 Niech sie odwróci od zlego, a czyni dobre; niech szuka pokoju i sciga go.
\par 12 Albowiem oczy Panskie otworzone sa na sprawiedliwych, a uszy jego ku prosbie ich; lecz oblicze Panskie przeciwko tym, którzy czynia zle rzeczy.
\par 13 I któz jest, co by wam zle uczynil, jezlibyscie dobrego nasladowcami byli?
\par 14 Ale chociazbyscie tez cierpieli dla sprawiedliwosci, blogoslawieni jestescie, a strachu ich nie lekajcie sie, ani trwozcie soba, ale Pana Boga poswiecajcie w sercach waszych.
\par 15 Badzcie zawsze gotowi ku daniu odpowiedzi kazdemu domagajacemu sie od was rachunku o tej nadziei, która w was jest, z cichoscia i z bojaznia, majac sumienie dobre;
\par 16 Aby w tem, w czem was pomawiaja jako zloczynców, zawstydzili sie ci, którzy nagane dawaja waszemu dobremu obcowaniu w Chrystusie.
\par 17 Lepiej bowiem jest, abyscie dobrze czyniac, jezli sie tak podoba woli Bozej, cierpieli, nizeli zle czyniac.
\par 18 Bo i Chrystus raz za grzechy cierpial, sprawiedliwy za niesprawiedliwych, aby nas przywiódl do Boga, umartwiony bedac cialem, ale ozywiony duchem;
\par 19 Przez którego i tym duchom, którzy sa w wiezieniu, przyszedlszy kazal.
\par 20 Którzy niekiedy nieposluszni byli, gdy raz oczekiwala Boza cierpliwosc za dni Noego, kiedy korab gotowano, w którym malo (to jest osm) dusz zachowane sa w wodzie.
\par 21 Czego teraz chrzest wzorem bedac, zbawia nas (nie skladanie cielesnego plugastwa, ale obietnica spólna sumienia dobrego u Boga,)przez zmartwychwstanie Jezusa Chrystusa,
\par 22 Który jest na prawicy Bozej, szedlszy do nieba, podbiwszy sobie Aniolów i zwierzchnosci, i mocy.

\chapter{4}

\par 1 Poniewaz tedy Chrystus ucierpial za nas w ciele, i wy tez taz mysla badzcie uzbrojeni, ze ten, co cierpial w ciele, poprzestal grzechu,
\par 2 Aby juz wiecej nie cielesnym pozadliwosciom, ale woli Bozej zyl ostatek czasu w ciele.
\par 3 Albowiem dosyc nam, zesmy przeszlego czasu zywota popelniali lubosci pogan, chodzac w rozpustach, w pozadliwosciach, w opilstwach, w biesiadach, w pijanstwach i sprosnych balwochwalstwach.
\par 4 Przetoz, ze sie wy z nimi nie schadzacie na taka zbyteczna rozpuste, zda sie im rzecza obca i bluznia to.
\par 5 Ci dadza liczba temu, który gotowy jest sadzic zywych i umarlych.
\par 6 Dlatego bowiem i umarlym kazano Ewangielije, aby sadzeni byli wedlug ludzi z strony ciala, ale zyli wedlug Boga duchem.
\par 7 A wszystkiemuc sie koniec przybliza.
\par 8 Przetoz trzezwymi badzcie i czulymi ku modlitwom, a nade wszystko miejcie uprzejma milosc jedni ku drugim; albowiem milosc zakryje mnóstwo grzechów.
\par 9 Goscinnymi badzcie jedni ku drugim bez szemrania.
\par 10 Kazdy jako wzial dar, tak nim jeden drugiemu uslugujcie, jako dobrzy szafarze rozlicznej laski Bozej.
\par 11 Jezli kto mówi, niech mówi jako wyroki Boze, jezli kto posluguje, niech to czyni jako z sily, której Bóg dodaje, aby we wszystkiem chwalony byl Bóg przez Jezusa Chrystusa, któremu nalezy chwala i panowanie na wieki wieków. Amen.
\par 12 Najmilsi! niech wam nie bedzie rzecza dziwna ten ogien, który na was przychodzi ku doswiadczeniu waszemu, jakoby co obcego na was przychodzilo;
\par 13 Ale radujcie sie z tego, zescie uczestnikami ucierpienia Chrystusowego, abyscie sie i w objawienie chwaly jego z radoscia weselili.
\par 14 Jezli was lza dla imienia Chrystusowego, blogoslawieni jestescie, gdyz on Duch chwaly a Duch Bozy odpoczywa na was, który wzgledem nich bywa bluzniony, ale wzgledem was bywa uwielbiony.
\par 15 A zaden z was niech nie cierpi jako mezobójca, albo zlodziej, albo zloczynca, albo jako w cudzy urzad sie wtracajacy.
\par 16 Lecz jezli cierpi jako chrzescijanin, niech sie nie wstydzi, owszem niech chwali Boga w tej mierze.
\par 17 Albowiem czas jest, aby sie sad poczal od domu Bozego; a poniewaz najprzód zaczyna sie od nas, jakiz bedzie koniec tych, co sa nieposluszni Ewangielii Bozej?
\par 18 A poniewaz sprawiedliwy ledwie zbawiony bedzie, niezbozny i grzeszny gdziez sie okaze?
\par 19 Przetoz i ci, którzy cierpia wedlug woli Bozej, niechaj jemu, jako wiernemu Stworzycielowi, poruczaja dusze swoje, dobrze czyniac.

\chapter{5}

\par 1 Starszych, którzy sa miedzy wami, prosze ja spólstarszy i swiadek ucierpienia Chrystusowego, i uczestnik chwaly, która ma byc objawiona:
\par 2 Pascie trzode Boza, która jest miedzy wami, dogladajac jej nie poniewolnie, ale dobrowolnie; nie dla sprosnego zysku, ale ochotnym umyslem:
\par 3 Ani jako panujac nad dziedzictwem Panskiem, ale wzorami bedac trzody.
\par 4 A gdy sie okaze on ksiaze pasterzy, odniesiecie niezwiedla korone chwaly.
\par 5 Takze, mlodsi! badzcie poddani starszym, a wszyscy jedni drugim badzcie poddani. Pokora badzcie wewnatrz ozdobieni, gdyz Bóg pysznym sie sprzeciwia, a pokornym laske daje.
\par 6 Unizajciez sie tedy pod mocna reka Boza, aby was wywyzszyl czasu swego;
\par 7 Wszystko staranie wasze wrzuciwszy na niego, gdyz on ma piecze o was.
\par 8 Trzezwymi badzcie, czujcie; albowiem przeciwnik wasz dyjabel, jako lew ryczacy obchodzi, szukajac kogo by pozarl.
\par 9 Któremu dawajcie odpór, mocni bedac w wierze, wiedzac, iz sie takowez ucierpienia nad braterstwem waszem, które jest na swiecie, wykonywaja.
\par 10 A Bóg wszelkiej laski, który nas powolal do wiecznej chwaly swojej w Chrystusie Jezusie, gdy maluczko ucierpicie, ten niech was doskonalymi uczyni, utwierdzi, umocni i ugruntuje;
\par 11 Jemu niech bedzie chwala i panowanie na wieki wieków. Amen.
\par 12 Przez Sylwana wam wiernego brata, jako rozumiem, krótkom pisal, napominajac i swiadczac, iz ta jest prawdziwa laska Boza, w której stoicie.
\par 13 Pozdrawia was spólwybrany zbór, ten, który jest w Babilonie i Marek, syn mój
\par 14 Pozdrówcie jedni drugich w pocalowaniu milosci. Pokój niech bedzie wam wszystkim, którzyscie w Chrystusie Jezusie. Amen.


\end{document}