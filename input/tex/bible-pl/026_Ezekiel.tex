\begin{document}

\title{Ezechiela}


\chapter{1}

\par 1 I stalo sie trzydziestego roku, miesiaca czwartego, piatego dnia tegoz miesiaca, gdym byl w posrodku pojmanych u rzeki Chebar, ze sie otworzyly niebiosa, i widzialem widzenie Boze.
\par 2 Piatego dnia tegoz miesiaca, (ten jest rok piaty po zaprowadzeniu króla Joachyna.)
\par 3 Prawdziwie stalo sie slowo Panskie do Ezechyjela kaplana, syna Buzowego, w ziemi Chaldejskiej u rzeki Chebar, a byla nad nim reka Panska.
\par 4 I widzialem, a oto wiatr gwaltowny przychodzil od pólnocy, i oblok wielki, i ogien palajacy, a blask byl okolo niego, a z posrodku jego wynikala jakoby niejaka predka swiatlosc, z posrodku, mówie, onego ognia.
\par 5 Takze z posrodku jego ukazalo sie podobienstwo czworga zwierzat, których takowy byl ksztalt: Podobienstwo czlowieka mialy.
\par 6 A kazde po cztery twarze, takze po cztery skrzydla kazde z nich mialo;
\par 7 Nogi ich byly nogi proste, a stopa nóg ich jako stopa nogi cielecej, a lsnialy sie wlasnie jako miedz wypolerowana;
\par 8 Rece ludzkie byly pod skrzydlami ich po czterech stronach ich, a twarze ich i skrzydla ich na czterech onych stronach;
\par 9 Skrzydla ich spojone byly jedno z drugiem, nie obracaly sie, gdy chodzily, ale kazde w prost na swa strone chodzilo.
\par 10 A podobienstwo twarzy ich takie: Z przodku twarz ludzka a twarz lwia po prawej stronie kazdego z nich, a twarz wolowa po lewej stronie wszystkich czworga, takze twarz orla z tylu mialo wszystko czworo z nich:
\par 11 A twarze ich i skrzydla ich byly podniesione ku górze; kazde zwierze dwa skrzydla spajalo z dwoma skrzydlami drugiego, a dwoma przykrywaly cialo swoje;
\par 12 A kazde z nich wprost na swa strone chodzilo; kedykolwiek duch chcial, aby szly, tam szly, nie obracaly sie, gdy chodzily.
\par 13 Takze podobienstwo onych zwierzat na wejrzeniu bylo jako wegle w ogniu rozpalone, palace sie jako pochodnie; ten ogien ustawicznie chodzil miedzy zwierzetami, a on ogien mial blask, z którego ognia wychodzila blyskawica.
\par 14 Biegaly tez one zwierzeta, i wracaly sie jako predkie blyskawice.
\par 15 A gdym sie przypatrywal onym zwierzetom, a oto kolo jedno bylo na ziemi przy zwierzetach u czterech twarzy kazdego z nich;
\par 16 Na wejrzeniu byly kola, i robota ich jako barwa kamienia Tarsys, a podobienstwo bylo jednakie onych czterech kól, a byly na wejrzeniu i robota ich, jakoby bylo kolo w posrodku kola;
\par 17 Majac isc na cztery strony swoje chodzily, a nie obracaly sie, gdy chodzily.
\par 18 Dzwona taka wysokosc mialy, az strach z nich pochodzil; te dzwona w okolo wszystkich czterech kól pelne byly oczów.
\par 19 A gdy chodzily zwierzeta, kola tez chodzily podle nich; a gdy sie podnosily zwierzeta w góre od ziemi, podnosily sie i kola.
\par 20 Gdziekolwiek chcial duch, aby szly, tam szly; gdzie mówie duch chcial, aby szly; a kola podnosily sie przed niemi, bo duch zwierzat byl w kolach.
\par 21 Gdy one szly, szly i kola, a gdy one staly, staly; a gdy sie podniosly od ziemi, podniosly sie tez kola z niemi; bo duch zwierzat byl w kolach.
\par 22 Nad glowami zwierzat bylo podobienstwo rozpostarcia jako podobienstwo krysztalu przezroczystego rozciagnionego nad glowami ich z wierzchu;
\par 23 A pod onem rozpostarciem skrzydla ich byly podniesione, jedno z drugiem spojone; kazde mialo dwa, któremi sie przykrywalo, kazde, mówie, mialo dwa, któremi przykrywalo cialo swoje.
\par 24 I slyszalem szum skrzydel ich, jako szum wód wielkich, jako szum Wszechmocnego, gdy chodzily, i szum huku jako szum wojska; a gdy staly, spuscily skrzydla swoje.
\par 25 A gdy staly i spuszczaly skrzydla swoje, tedy byl szum z wierzchu nad rozpostarciem, które bylo nad glowa ich.
\par 26 A z wierzchu na rozpostarciu, które bylo nad glowa ich, bylo podobienstwo stolicy na wejrzeniu jako kamien szafirowy, a nad podobienstwem stolicy, na nim z wierzchu na wejrzeniu jako osoba czlowieka.
\par 27 I widzialem na wejrzeniu jakoby predka swiatlosc, a wewnatrz w niej w okolo na wejrzeniu jako ogien od biódr jego w góre; takze tez od biódr jego na dól widzialem na wejrzeniu jako ogien, i blask okolo niego.
\par 28 Jaka bywa tecza na wejrzeniu, która bywa na obloku czasu deszczu, taki byl na wejrzeniu blask w okolo. Toc bylo widzenie podobienstwa chwaly Panskiej, które gdym widzial, upadlem na oblicze swoje, i uslyszalem glos mówiacego.

\chapter{2}

\par 1 I rzekl do mnie: Synu czlowieczy! stan na nogi twe, a bede mówil z toba.
\par 2 I wstapil w mie duch, gdy przemówil do mnie, a postawil mie na nogi moje, i slyszalem mówiacego do mnie;
\par 3 Który rzekl do mnie: Synu czlowieczy! Ja cie posylam do synów Izraelskich, do narodów odpornych, którzy mi sie sprzeciwili; oni i ojcowie ich wystepowali przeciwko mnie az prawie do dnia tego:
\par 4 Do tych, mówie, synów niewstydliwej twarzy, i zatwardzialego serca Ja cie posylam, i rzeczesz im: Tak mówi panujacy Pan.
\par 5 Niech oni sluchaja albo nie, gdyz domem odpornym sa, przeciez niech wiedza, ze prorok byl w posrodku ich.
\par 6 Ale ty, synu czlowieczy! nie bój sie ich, ani sie lekaj slów ich, ze odporni a jako ciernie sa przeciwko tobie, a ze miedzy niedzwiadkami mieszkasz; slów ich nie bój sie, ani sie twarzy ich lekaj, przeto, ze domem odpornym sa.
\par 7 Ale mów slowa moje do nich, niech oni sluchaja albo nie, gdyz odpornymi sa.
\par 8 Lecz ty, synu czlowieczy! sluchaj, co Ja mówie do ciebie: Nie badz odporny, jako ten dom odporny; otwórz usta swe, a zjedz, coc dam.
\par 9 I widzialem, a oto reka byla wyciagniona do mnie, a oto w niej zwinione ksiegi,
\par 10 Które rozwinal przedemna; a byly popisane z przodku i z konca, a w nich byly napisane narzekania, i wzdychania i bieda.

\chapter{3}

\par 1 I rzekl do mnie: Synu czlowieczy! co przed toba jest, zjedz; zjedz te ksiegi, a idz i mów do domu Izraelskiego.
\par 2 Otworzylem tedy usta swe, i dal mi zjesc one ksiegi,
\par 3 A mówil do mnie: Synu czlowieczy! nakarm brzuch twój, a wnetrznosci twoje napelnij temi ksiegami, którec daje. I zjadlem je, i byly w ustach moich slodkie jako miód.
\par 4 Zatem rzekl do mnie: Synu czlowieczy! idz a wnijdz do domu Izraelskiego, i mów slowy mojemi do nich.
\par 5 Bo cie nie do ludu nieznajomej mowy, albo trudnego jezyka posylam, ale do domu Izraelskiego;
\par 6 Nie do wielu narodów nieznajomej mowy, i trudnego jezyka, którychbys slów nie zrozumial, którzy jednak, gdybym cie do nich poslal, usluchaliby cie.
\par 7 Lecz dom Izraelski nie bedzie cie chcial usluchac, poniewaz mnie samego usluchac nie chca; bo wszystek dom Izraelski jest twardego czola i zatwardzonego serca.
\par 8 Ale otom uczynil twarz twoje twarda przeciwko twarzy ich, a czolo twe twarde przeciwko czolu ich.
\par 9 Uczynilem czolo twe jako dyjament, i twardsze nad skale; nie bójze sie ich, ani sie lekaj twarzy ich, przeto, ze sa domem odpornym.
\par 10 I rzekl do mnie: Synu czlowieczy! wszystkie slowa moje, które mówic bede do ciebie, przyjmij do serca twego, a sluchaj uszyma twemi.
\par 11 Idz a wnijdz do pojmanych, do synów ludu twego, i mów do nich, a powiedz im: Tak mówi panujacy Pan, niech oni sluchaja, albo nie.
\par 12 Tedy mie duch podniósl, i slyszalem za soba glos grzmotu wielkiego: Blogoslawiona niech bedzie chwala Panska z miejsca swego;
\par 13 I szum skrzydel onych zwierzat, które sie naspól dotykaly, i glos kól naprzeciwko nim, i glos grzmotu wielkiego.
\par 14 A duch podniósl mie i wzial mie. I odszedlem z gorzkoscia w rozgniewaniu ducha mego; ale reka Panska nademna mocna byla.
\par 15 I przyszedlem do pojmanych do Telabib, którzy mieszkali przy rzece Chebar, i siadlem gdzie oni mieszkali, a siedzialem tam siedm dni w posrodku ich, zdumiawszy sie.
\par 16 A gdy minelo siedm dni, bylo slowo Panskie do mnie mówiace:
\par 17 Synu czlowieczy! Dalem cie strózem domowi Izraelskiemu, abys slyszac slowo z ust moich napomnial ich odemnie.
\par 18 Gdybym Ja rzekl niepoboznemu: Smiercia umrzesz, a nie napomnialbys go, i nie mówilbys, abys go odwiódl od niezboznej drogi jego, tak, zebys go przy zywocie zachowal, tedy onci niezbozny w nieprawosci swojej umrze; ale krwi jego z reki twojej szukac bede.
\par 19 Lecz jezlibys ty napomnial niezboznego, a nie odwrócilby sie od niezboznosci swej, i od drogi swej niezboznej, tedy onci w nieprawosci swojej umrze; ale ty dusze swoje wybawisz.
\par 20 Takze jezliby sie odwrócil sprawiedliwy od sprawiedliwosci swojej, a czynilby nieprawosc, a Jabym polozyl zawade przed nim, i takby umarl, a tybys go nie napomnial: w grzechu swym umrze, a nie przyjda na pamiec sprawiedliwosci jego, które czynil, lecz krwi jego z reki twojej szukac bede.
\par 21 Ale jezlibys ty napomnial sprawiedliwego, aby nie zgrzeszyl ten sprawiedliwy, i nie grzeszylby, zaiste zyc bedzie, bo napomnienie przyjal, a ty dusze swoje wybawisz.
\par 22 I byla tam nademna reka Panska, i rzekl do mnie: Wstawszy wyjdz w pole, a tam sie z toba rozmówie.
\par 23 A tak wstawszy szedlem w pole, a oto chwala Panska stala tam, jako chwala, któram widzial u rzeki Chebar, i upadlem na oblicze moje.
\par 24 Tedy duch wstapil w mie, a postawiwszy mie na nogi moje mówil do mnie, i rzekl mi: Wnijdz, zamknij sie w domu swym.
\par 25 Bo oto na cie, synu czlowieczy! wloza powrozy, i zwiaza cie niemi, a nie bedziesz mógl wynijsc miedzy nich.
\par 26 A Ja uczynie, ze jezyk twój przylgnie do podniebienia twego, i bedziesz niemy, a nie bedziesz im mezem strofujacym, przeto, ze sa domem odpornym.
\par 27 Ale gdy z toba mówic bede, otworze usta twoje; tedy rzeczesz do nich: Tak mówi panujacy Pan: Kto chce sluchac, niech slucha, a kto nie chce, niech nie slucha, gdyz domem odpornym sa.

\chapter{4}

\par 1 A ty, synu czlowieczy! wezmij sobie cegle, a polozywszy ja przed soba, wyryj na niej miasto Jeruzalem;
\par 2 I sporzadz na niej oblezenie, i zbuduj na niej szance, i usyp na niej wal, a postaw na niej wojska, i zasadz na niej tarany w okolo;
\par 3 Potem wezmij sobie panew zelazna, i postaw ja miasto muru zelaznego miedzy soba a miedzy miastem, a obróc twarz swoje przeciwko niemu, niech bedzie oblezone, a oblezesz je. To bedzie znakiem domowi Izraelskiemu;
\par 4 A ty ukladz sie na lewy bok twój, a wlóz nan nieprawosc domu Izraelskiego; jak wiele dni lezec bedziesz na nim, tak dlugo poniesiesz nieprawosc ich.
\par 5 A Ja tobie daje lata nieprawosci ich wedlug liczby dni, to jest trzy sta i dziewiecdziesiat dni, tak dlugo poniesiesz nieprawosc domu Izraelskiego.
\par 6 A gdy je wypelnisz, ukladziesz sie powtóre na prawy bok twój, a poniesiesz nieprawosc domu Judzkiego przez czterdziesci dni; dzien za rok, dzien, mówie, za rok daje tobie.
\par 7 Tak tedy obróc twarz swoje przeciwko oblezeniu Jeruzalemu, ramie swoje wysmuknij, a prorokuj przeciwko niemu.
\par 8 A oto klade na cie powrozy, abys sie nie obrócil z jednego boku na drugi, dokad nie wypelnisz dni oblezenia swego.
\par 9 Przetoz nabierz sobie pszenicy, i jeczmienia, i bobu, i soczewicy, i prosa, i wiki, a wlóz to do jednego naczynia, i uczyn sobie z tego pokarm, wedlug liczby dni, których lezec bedziesz na boku swym, to jest, przez trzysta i dziewiecdziesiat dni jesc go bedziesz.
\par 10 A ta waga bedzie pokarmu twego, który jesc bedziesz, mianowicie dwadziescia syklów na dzien, od czasu az do czasu jesc go bedziesz.
\par 11 Takze wode pod miara pic bedziesz; szósta czesc hynu od czasu do czasu pic bedziesz.
\par 12 A podplomyk jeczmienny, który jesc bedziesz, ten przy lajnach czlowieczych upieczesz przed oczyma ich.
\par 13 I rzekl Pan: Tak beda jesc synowie Izraelscy chleb swój plugawy dla pogan, których tam zgromadze.
\par 14 I rzeklem: Ach panujacy Panie! oto dusza moja nie jest splugawiona zdechlina, a tego, co zwierz rozszarpal, nie jadlem od dziecinstwa mego az dotad, i nie wchodzilo do ust moich zadne mieso obrzydle.
\par 15 Ale on rzekl do mnie: Oto daje ci lajna wolowe miasto lajen czlowieczych, abys sobie przy nich napiekl chleba.
\par 16 Zatem rzekl do mnie: Synu czlowieczy! Oto Ja zlamie laske chleba w Jeruzalemie, tak, ze chleb pod waga jesc beda, i to z frasunkiem, takze wode pod miara pic beda, i to z zdumieniem;
\par 17 Aby niedostatek majac chleba i wody, zdumiewal sie kazdy z nich, i uwiadl w nieprawosciach swoich.

\chapter{5}

\par 1 Potem ty, synu czlowieczy! wezmij sobie nóz ostry, to jest brzytwe barwierska, wezmij ja sobie, a ogól nia glowe swoje i brode swoje; potem wezmij sobie wage, i rozdziel je.
\par 2 Jedne trzecia czesc ogniem spal w posród miasta, gdy sie wypelnia dni oblezenia; potem wziawszy druga trzecia czesc, posiekaj mieczem okolo niego, a ostatnia trzecia czesc roztrzasnij na wiatr; bo miecza dobede za nimi.
\par 3 A wszakze wezmij z nich jaka troche, i zawiaz w kraje szaty twojej.
\par 4 A i z tych jeszcze wziawszy wrzuc je w posród ognia, i spal je w ogniu, skad wynijdzie ogien na wszystek dom Izraelski.
\par 5 Tak mówi panujacy Pan: Toc jest Jeruzalem, którem postawil w posród pogan, a zewszad otoczyl krainami;
\par 6 Ale odmienilo sady moje w niezboznosc, bardziej niz poganie, a ustawy moje bardziej niz inne krainy, które sa okolo niego; bo sadami mojemi pogardzili, i w ustawach moich nie chodzili.
\par 7 Przetoz tak mówi panujacy Pan: Dlatego, zescie poganstwo grzechami przewyzszyli, które jest okolo was, a w ustawach moich nie chodziliscie, i sadów moich nie zachowaliscie, nawet ani tak jako poganie, którzy sa okolo was, saduscie nie czynili;
\par 8 Przetoz tak mówi panujacy Pan: Otom Ja przeciwko tobie, otom Ja, i wykonam w posrodku ciebie sady przed oczyma pogan;
\par 9 Bo uczynie przy tobie to, czegom pierwej nie uczynil, i czego napotem nie uczynie dla wszystkich obrzydliwosci twoich,
\par 10 Tak, ze ojcowie jesc beda synów w posrodku ciebie, a synowie jesc beda ojców swoich; i wykonam sady przeciwko tobie, a rozprosze wszystkie ostatki twoje na wszystkie strony.
\par 11 Przetoz jako zyje Ja, mówi panujacy Pan: Poniewazes ty swiatnice moje splugawilo wszelakiemi nieczystosciami twojemi, i wszelakiemi obrzydliwosciami twemi, i Ja cie tez podam w lekkosc, a nie sfolgujec oko moje, ani sie zlituje.
\par 12 Trzecia czesc z ciebie morem pomrze i glodem wyginie w posrodku ciebie, a druga trzecia czesc od miecza padnie okolo ciebie, a trzecia ostatnia czesc na wszystkie strony rozprosze, i miecza dobede za nimi.
\par 13 A tak sie do konca wyleje zapalczywosc moja, i natre popedliwoscia swoja na nich, i uciesze sie, i doznaja, zem Ja Pan, którym to wyrzekl w zapalczywosci mojej, gdy wypelnie gniew swój nad nimi.
\par 14 A podam cie w spustoszenie i w pohanbienie narodom, które sa okolo ciebie, przed oczyma kazdego mijajacego.
\par 15 A tak bedziesz na hanbe, na posmiech, na srogi przyklad i na zdumienie narodom, które sa okolo ciebie, gdy wykonam przeciwko tobie sady w popedliwosci i w gniewie i w srogiem zagniewaniu. Ja Pan mówilem.
\par 16 Gdy wypuszcze srogie strzaly glodu na zgube wasze, które wypuszcze, abym was wygubil, a glód zgromadze przeciwko wam, i zlamie wam laske chleba.
\par 17 Posle zaiste na was glód, i zwierzeta okrutne, które cie osieroca; i mór i krew przyjdzie na cie, gdy na cie miecz przywiode. Ja Pan mówilem.

\chapter{6}

\par 1 I stalo sie slowo Panskie do mnie, mówiac:
\par 2 Synu czlowieczy! obróc twarz twoje ku górom Izraelskim, a prorokuj przeciwko nim,
\par 3 I rzecz: Góry Izraelskie! sluchajcie slowa panujacego Pana. Tak mówi panujacy Pan górom i pagórkom, strumieniom i dolinom: Oto Ja, Ja przywiode na was miecz, i pokaze wyzyny wasze.
\par 4 A tak spustoszone beda oltarze wasze, i zdruzgotane beda sloneczne obrazy wasze, a porozrzucam pobitych waszych przed plugawemi balwanami waszemi.
\par 5 Poloze tez trupy synów Izraelskich przed plugawemi balwanami ich, a rozrzuce kosci wasze okolo oltarzów waszych.
\par 6 Po wszystkich mieszkaniach waszych miasta spustoszone beda, i wyzyny spustoszeja, tak, iz beda skazone i rozwalone oltarze wasze, zdruzgotane beda, i ustana plugawe balwany wasze, a beda podrabione sloneczne obrazy wasze, a tak wygladzone beda dziela wasze.
\par 7 I padnie zraniony w posrodku was, a poznacie, zem Ja Pan.
\par 8 Wszakze z was niektórych pozostawie, którzyby uszli miecza miedzy poganami, gdy rozproszeni bedziecie po ziemiach.
\par 9 I wspomna na mie, którzy z was zachowani beda miedzy narodami, u krórych beda w pojmaniu, zem ubolewal nad sercem ich wszetecznem, które odstapilo odemnie, i nad oczyma ich, które nierzad plodzily, chodzac za plugawemi balwanami swojemi: i omierzna sami sobie dla zlosci, które czynili we wszystkich obrzydliwosciach swoich.
\par 10 I poznaja, zem Ja Pan, a izem nie darmo mówil, ze na nich to zle przywiode.
\par 11 Tak mówi panujacy Pan: Klasnij reka twoja, a tapnij noga twoja, i mów: Niestetyz na wszystkie zle obrzydliwosci domu Izraelskiego; bo od miecza, od glodu, od morowego powietrza polegna.
\par 12 Ten, co bedzie daleko, morem umrze, a ten, co blisko, od miecza polegnie, a ten, co zostanie, i oblezony, od glodu umrze; a tak wykonam popedliwosc moje nad nimi.
\par 13 Tedy poznacie, zem Ja Pan, gdy beda pobici ich lezeli w posrodku plugawych balwanów ich, i okolo oltarzów ich, na kazdym pagórku wysokim, po wszystkich wierzchach gór, i pod kazdem drzewem zielonem, i pod kazdym debem krzewistym, na któremkolwiek miejscu sprawowali wonnosc wdzieczna wszystkim plugawym balwanom swoim.
\par 14 Bo reke swoje wyciagne przeciwko nim, i uczynie te ziemie spustoszona i bardziej spustoszona niz pustynie Dyblat, we wszystkich mieszkaniach ich. A tak poznaja, zem Ja Pan.

\chapter{7}

\par 1 Potem stalo sie slowo Panskie do mnie, mówiac:
\par 2 Sluchaj ty, synu czlowieczy: Tak mówi panujacy Pan o ziemi Izraelskiej: Koniec, koniec przyszedl na wszystkie cztery strony ziemi.
\par 3 Teraz przyjdzie koniec na cie; posle na cie popedliwosc moje, i bede cie sadzil wedlug dróg twoich, i zwale na cie wszystkie obrzydliwosci twoje.
\par 4 A nie sfolgujec oko moje, i nie zmiluje sie, ale drogi twoje zwale na cie, a obrzydliwosci twoje beda w posrodku ciebie, i poznacie, zem Ja Pan.
\par 5 Tak mówi panujacy Pan: Utrapienie jedno, oto utrapienie straszne przychodzi;
\par 6 Koniec przychodzi, przychodzi koniec, ocknal sie przeciwko tobie, oto przychodzi.
\par 7 Przychodzi predki poranek na cie, o obywatelu ziemi! przychodzi ten czas, przybliza sie ten dzien grzmotu, a nie glosu rozlegajacego sie po górach.
\par 8 Juz predko, juz wyleje gniew mój na cie, a wykonam zapalczywosc moje nad toba, a osadze cie wedlug dróg twoich, i wloze na cie wszystkie obrzydliwosci twoje.
\par 9 Nie sfolgujec zaiste oko moje, ani sie zlituje, ale wedlug dróg twoich nagrodzec, i obrzydliwosci twoje w posrodku ciebie beda; a tak poznacie, zem Ja Pan, który bije.
\par 10 Oto ten dzien, oto przyszedl; przyszedl predki poranek, zakwitnela rózga, wybija sie pycha.
\par 11 Okrucienstwo wyroslo w rózge niezboznosci; nie zostanie z nich nic, ani z mnóstwa ich, ani z huku ich, i nie bedzie zadnego narzekania nad nimi.
\par 12 Idzie czas, przybliza sie dzien. Kto kupi, nie bedzie sie weselil, a kto sprzeda, nie bedzie zalowal; bo popedliwosc przyjdzie na wszystko mnóstwo jej.
\par 13 Bo kto sprzedal, nie wróci sie do rzeczy sprzedanej chocby jeszcze miedzy zyjacymi byl zywot ich; poniewaz widzenie na wszystko mnóstwo jej nie wróci sie, a zaden w nieprawosci zywota swego nie zmocni sie.
\par 14 Trabic beda w trabe, i wszystko przygotuja, jednak nie bedzie kto mial isc na wojne; bo popedliwosc moja oburzy sie na wszystko mnóstwo jej.
\par 15 Miecz zewnatrz, a mór i glód bedzie wewnatrz; kto bedzie na polu, od miecza umrze; a kto w miescie, glód i mór go pozre.
\par 16 A którzy z nich uciekna, ci beda na górach jako golebice w dolinie; wszyscy beda narzekali, kazdy nad nieprawoscia swoja.
\par 17 Wszystkie rece oslabieja, i wszystkie sie kolana rozplyna jako woda.
\par 18 I obloka sie w wory, i okryje ich strach, i na wszelkiej twarzy bedzie wstyd, i na wszystkich glowach ich lysina.
\par 19 Srebro swoje po ulicach rozrzuca, a zloto ich bedzie jako nieczystosc; srebro ich i zloto ich nie bedzie ich moglo wybawic w dzien popedliwosci Panskiej; nie nasyca duszy swojej, i wnetrznosci swych nie napelnia, przeto, ze im jest ku obrazeniu nieprawosc ich;
\par 20 A iz w slawie ozdoby swojej, która na chwale swoja Bóg wystawil, obrazów obrzydliwosci swoich i sprosnosci swoich naczynili, przetozem im je w nieczystosc obrócil;
\par 21 I podam je w rece cudzoziemców na rozchwycenie, i niezboznych w ziemi na lup, którzy ja splugawia;
\par 22 Odwróce tez twarz moje od nich, a zgwalca swiatnice moje, a wnijda do niej rozbójnicy, i splugawia ja.
\par 23 Uczyn lancuch; bo ziemia pelna jest krwawych sadów, a miasto pelne jest krzywd.
\par 24 Przetoz najgorszych z pogan przywiode, aby posiedli domy ich; i uczynie wstret pysze mocarzów, a splugawieni beda, którzy je poswiecaja.
\par 25 Zginienie przyszlo; przetoz szukac beda pokoju, ale go nie bedzie.
\par 26 Ucisk za uciskiem przyjdzie, a wiesc za wiescia przypadnie; i beda szukac widzenia od proroka! ale zakon zginie od kaplana, a rada od starców.
\par 27 Król plakac bedzie, a ksiaze w smutek obleczony bedzie, a rece ludu w ziemi przestraszone beda. Wedlug drogi ich uczynie im, i wedlug sadów ich osadze ich, i poznaja, zem Ja Pan.

\chapter{8}

\par 1 I stalo sie roku szóstego, piatego dnia, szóstego miesiaca, gdym siedzal w domu swym, a starsi Judzcy siedzieli przedemna, tedy tam przypadla na mie reka panujacego Pana.
\par 2 I widzalem, a oto podobienstwo na wejrzeniu jako ogien; od biódr jego na dól jako ogien, a od biódr jego w góre na wejrzeniu jako blask, i niby predka swiatlosc.
\par 3 Tedy sciagnawszy podobienstwo reki, uchwycil mie za kedzierze glowy mojej, i podniósl mie duch miedzy ziemia i miedzy niebem, a przywiódl mie do Jeruzalemu w widzeniach Bozych, do wejscia bramy wewnetrznej, która patrzy ku pólnocy, gdzie byla stolica balwana do gorliwosci i zapalczywosci pobudzajaca.
\par 4 A oto tam byla chwala Boga Izraelskiego na wejrzeniu jako ona, któram widzial na polu.
\par 5 I rzekl do mnie: Synu czlowieczy! teraz podnies oczy swe ku drodze na pólnocy; a tak podnioslem oczy swe ku drodze na pólnocy, a oto na pólnocy byl u bramy oltarzowej on balwan pobudzajacy do gorliwosci w samem prawie wejsciu.
\par 6 Wtem mi rzekl: Synu czlowieczy! widziszze ty, co ci czynia, obrzydliwosci wielkie, które tu czyni dom Izraelski? tak, ze sie oddalic musze od swiatnicy mojej; ale obróciwszy sie ujrzysz obrzydliwosci jeszcze wieksze.
\par 7 I przywiódl mie do drzwi sieni, gdziem ujrzal, a oto dziura jedna byla w scianie.
\par 8 I rzekl do mnie: Synu czlowieczy! przekop teraz te sciane: i przekopalem sciane, a oto drzwi jedne.
\par 9 I rzekl do mnie: Wnijdz, a obacz te zle obrzydliwosci, które tu oni czynia.
\par 10 Przetoz wszedlszy ujrzalem, a oto wszelakie podobienstwa plazu, i zwierzat obrzydlych, i wszelakich plugawych balwanów domu Izraelskiego wyryte byly na scianie wszedy w okolo.
\par 11 A siedmdziesiat mezów starszych z domu Izraelskiego, z Jazanijaszem, synem Safanowym, stojacym w posród nich, stali przed nimi, majac kazdy kadzielnice swoje w rece swej, tak, ze gesty oblok kadzenia w góre wstepowal.
\par 12 I rzekl do mnie: A widzialzes, synu czlowieczy! co starsi domu Izraelskiego w ciemnosci czynia, kazdy w pokojach swoich malowanych? Bo mówia: Nie widzi nas Pan, opuscil Pan te ziemie.
\par 13 Znowu rzekl do mnie: Jeszcze obróciwszy sie ujrzysz obrzydliwosci wieksze, które oni czynia.
\par 14 I przywiódl mie do wrót bramy domu Panskiego, która jest na pólnpcy, a oto tam niewiasty siedzialy, placzac Tammusa;
\par 15 I rzekl mi: A widzialzes, synu czlowieczy? Ale obróciwszy sie ujrzysz jeszcze wieksze obrzydliwosci, nizeli te.
\par 16 Tedy mie wprowadzil do wnetrznej sieni domu Panskiego, a oto u drzwi kosciola Panskiego miedzy przysionkim i oltarzem bylo dwadziescia i piec mezów, których tyly byly obrócone ku kosciolowi Panskiemu, a twarze ich na wschód, którzy sie klaniali przeciwko wschodowi slonca.
\par 17 I rzekl mi: A widzialzes, synu czlowieczy? Izali to lekka rzecz jest domowi Judzkiemu, czynic takie obrzydliwosci, jakie tu czynia? Bo napelniwszy ziemie nieprawoscia, obrócili sie, aby mie draznili, a oto przykladaja latorosl winna do nosów swoich.
\par 18 Przetoz i Ja postapie z nimi wedlu zapalczywosci; oko moje nie sfolguje ani sie zmiluje; i beda wolac do uszów moich glosem wielkim, a nie wyslucham ich.

\chapter{9}

\par 1 Potem zawolal, gdziem ja slyszal, glosem wielkim, mówiac: Hetmani, nastapcie na to miasto, majac kazdy bron swoje ku zabijaniu w rece swej.
\par 2 A oto szesc mezów przyszlo droga ku brami wyzszej, która patrzy na pólnocy, i mial kazdy bron swoje ku wytraceniu w rece swej; ale maz jeden byl w posród nich odziany szata lniana, a kalamarz pisarski przy biodrach jego; i przyszedlszy staneli u oltarza miedzanego.
\par 3 A chwala Boga Izraelskiego zstapila byla z Cherubina, na którym byla, do progu domu, i zawolala na meza onego odzianego szata lniana, przy którego biodrach byl kalamarz pisarski.
\par 4 I rzekl Pan do niego: Przejdz przez posrodek miasta, przez posrodek Jeruzalemu, a uczyn znak na czolach mezów, którzy wzdychaja i narzekaja nad wszystkiemi obrzydliwosciami, które sie dzieja w posród niego.
\par 5 A onym rzekl, gdziem ja slyszal: Idzcie po miescie za nim; a zabijajcie; niech nie folguje oko wasze, ani sie zmilujcie.
\par 6 Starca, mlodzienca, i panne, i maluczkich, i niewiasty wybijcie do szczetu; ale do zadnego meza, na którymby byl znak, nie przystepujcie, od swiatnicy mojej poczniecie. A tak poczeli od onych mezów starszych, którzy byli przed domem Panskim;
\par 7 (Bo im byl rzekl: Splugawcie ten dom, a napelnijcie sieni pobitymi; idzciez.) A wyszedlszy zabijali w miscie.
\par 8 A gdy ich pozabijali, a jam pozostal, tedym padl na oblicze moje, i zawolalem a rzeklem: Ach panujacy Panie! izali ty wytracisz wszystek ostatek Izraelski, wylewajac popedliwosc swoje na Jeruzalem?
\par 9 I rzekl do mnie: Nieprawosc domu Izraelskiego i Judzkiego nader jest bardzo wielka, i napelniona jest ziemia krwia, a miasto pelne jest przewrotnosci; bo mówili: Pan te ziemie opuscil, a Pan nie widzi nas.
\par 10 Przetoz i Ja toz uczynie, nie sfolguje oko moje, ani sie zmiluje, droge ich na glowe ich obróce.
\par 11 A oto maz on odziany szata lniana, przy którego biodrach byl kalamarz, oznajmil to mówiac: Uczynilem tak, jakos mi rozkazal.

\chapter{10}

\par 1 I widzialem; a oto na rozpostarciu, które bylo nad glowa Cherubinów, jakoby kamien szafirowy, na wejrzeniu jako podobienstwo stolicy, ukazalo sie nad nimi.
\par 2 Tedy rzekl do onego meza odzianego szata lniana, mówiac: Wnijdz miedzy kola pod Cherubinów, a napeln rece swe weglem ognistym z posród Cherubinów, i rozrzuc po miescie. I wszedl przed oczyma memi.
\par 3 (A Cherubowie stali po prawej stronie domu, gdy wchodzil on maz, a oblok napelnil sien wnetrzna.
\par 4 Bo gdy sie byla podniosla chwala Panska z Cherubinów, ku progowi domu, tedy napelniony byl dom oblokiem, a sien napelniona byla jasnoscia chwaly Panskiej;
\par 5 A szum skrzydel Cherubinów slyszany byl az do sieni zewnetrznej, jako glos Boga wszechmocnego, gdy mówi.)
\par 6 Gdy tedy rozkazal onemu mezowi odzianemu w szate lniana, mówiac: Wezmij ognia z posrodku kól, z posrodku Cherubinów: wszedl i stanal podle kól.
\par 7 Tedy wyciagnal Cherubin jeden reke swa w posród Cherubinów do onego ognia, który byl w posrodku Cherubinów, a wziawszy podal go w reke onego odzianego szata lniana, który wziawszy go wyszedl.
\par 8 Bo sie ukazalo na onych Cherubinach podobienstwo reki czlowieczej pod skrzydlami ich.
\par 9 Potemem wejrzal, a oto cztery kola podle Cherubinów, kolo jedno podle jednego Cherubina, a tak kazde kolo podle kazdego Cherubina, a podobienstwo kól jako barwa kanienia Tarsys;
\par 10 A na wejrzeniu mialy jednakie podobienstwa one kola, jakoby bylo kolo w posrodku kola.
\par 11 Gdy chodzily, na cztery strony swoje chodzily; nie uchylaly sie, gdy szly, ale do onego miejsca, do którego sie wódz obracal, za nim szly; nie uchylaly sie, gdy szly.
\par 12 A wszytko cialo ich, i grzbiety ich, i rece ich, i skrzydla ich, takze i kola pelne byly oczów okolo onych samych czterech, i kól ich.
\par 13 A kola one nazwal okregiem, gdziem ja slyszal.
\par 14 A kazde zwierze mialo cztery twarze; twarz pierwsza byla twarz Cherubinowa, druga twarz byla twarz czlowiecza, trzecia byla twarz lwia, a czwarta byla twarz orla.
\par 15 I podniesli sie Cherubinowie. Toc sa one zwierzeta, którem widzial nad rzeka Chebar.
\par 16 A gdy chodzili Cherubinowie, chodzily i kola podle nich; a gdy ponosili Cherubinowie skrzydla swoje, aby sie wzbili od ziemi, nie odwracaly sie tez kola od nich.
\par 17 Gdy oni stali, staly, a gdy sie ponosili, podnosily sie tez z nimi; bo duch zwierzat byl w nich.
\par 18 I odeszla chwala Panska od progu domu, i stanela nad Cherubinami.
\par 19 Gdy podniesli Cherubinowie skrzydla swoje, a wzbili sie od ziemi przed oczyma mojemi odchodzac, a kola przeciwko nim, i staneli w wejsciu bramy domu Panskiego wschodniej, tedy chwala Boga Izraelskiego z wierzchu nad nimi byla.
\par 20 Toc sa one zwierzeta, którem widzial pod Bogiem Izraelskim nad rzeka Chebar; i poznalem, iz to byli Cherubinowie.
\par 21 Po cztery twarze mial kazdy z nich, i po cztery skrzydla kazdy z nich, a podobienstwo rak ludzkich pod skrzydlami ich.
\par 22 A podobienstwo twarzy ich bylo jako twarzy, którem widzial u rzeki Chebar; takze i oblicze ich takiez bylo, i oni sami; kazdy z nich prosto ku swej stronie chodzil.

\chapter{11}

\par 1 I podniósl mie duch, a przywiódl mie do bramy domu Panskiego wschodniej, która patrzy na wschód slonca; a oto w wejsciu onej bramy bylo dwadziescia i piec mezów, miedzy którymi widzialem Jazanijasza, syna Assurowego, i Pelatyjasza, syna Banaja szowego, ksiazat ludu;
\par 2 Tedy mi rzekl: Synu czlowieczy! onic to sa mezowie, którzy zamyslaja o nieprawosci, a radza zla rade w tem miescie,
\par 3 Mówiac: Nie budujmy domów blisko; boby tak miasto bylo kotlem, a my miesem.
\par 4 Dlatego prorokuj przeciwko nim, prorokuj, synu czlowieczy!
\par 5 Tedy przypadl na mie duch Panski, i rzekl do mnie: Mów: Tak mówi Pan: Takescie mówili, domie Izraelski! bo co wam kolwiek przychodzi na mysl, to Ja znam;
\par 6 Wielkiescie mnóstwo pobili w tem miescie, a napelniliscie ulice jego pobitymi.
\par 7 Dlatego tak mówi panujacy Pan: Którzy sa pobici od was, którychescie skladali w posrodku jego, oni sa miesem, a miasto kotlem; ale was wywioda z posrodku jego.
\par 8 Baliscie sie miecza; ale miecz przywiode na was, mówi panujacy Pan.
\par 9 A wywiode was z posrodku jego, a podam was w rece obcych, i wykonam nad wami sady.
\par 10 Od miecza polezecie, na granicach Izraelskich osadze was, i dowiecie sie, zem Ja Pan.
\par 11 Miasto nie bedzie wam kotlem, ani wy bedziecie w posród jego miesem; na granicach Izraelskich osadze was,
\par 12 I dowiecie sie, zem Ja Pan; poniewazescie w ustawach moich nie chodzili, a sadów moich nie czynili, alescie wedlug sadów tych narodów, którzy okolo was sa, czynili.
\par 13 A gdym prorokowal, tedy Pelatyjasz, syn Banajaszowy, umarl: dlatego upadlem na twarz moje, a wolajac glosem wielkim, rzeklem: Ach, panujacy Panie! do gruntu wygladzisz ostatki Izraelskie.
\par 14 I stalo sie slowo Panskie do mnie, mówiac:
\par 15 Synu czlowieczy! bracia twoi, bracia twoi, powinowaci twoi, i wszystek dom Izraelski, wszystek mówie dom, cic sa, którym mówili obywatele Jeruzalemscy: Oddalcie sie od Pana; namci dana jest ta ziemia w osiadlosc.
\par 16 Przetoz mów: Tak mówi panujacy Pan: Chociazem ich daleko zagnal miedzy narody, i chociazem ich rozproszyl po ziemiach, wszakze bede im swiatnica i przez ten maly czas w ziemiach, do których przyjda.
\par 17 Przetoz mów: Tak mówi panujacy Pan: Zgromadze was z narodów, a zbiore was z ziem, do którychescie rozproszeni, i dam wam ziemie Izraelska.
\par 18 I wnijda tam, a zniosa wszystkie splugawienia jej, i wszystkie jej obrzydliwosci z niej.
\par 19 Bo im dam serce jedno, i ducha nowego dam do wnetrznosci waszych, i odejme sece kamienne z ciala ich, a dam im serce miesiste,
\par 20 Aby w ustawach moich chodzili, a sadów moich strzegli, i czynili je; i beda ludem moim, a Ja bede Bogiem ich.
\par 21 Ale którychby serce chodzilo za zadzami plugastw swoich, i obrzydliwosci swoich, tych droge obróce na glowy ich, mówi panujacy Pan.
\par 22 Tedy podniesli Cherubinowie skrzydla swoje, i kola z nimi, a chwala Boga Izraelskiego byla nad nimi z wierzchu.
\par 23 I odeszla chwala Panska z posrodku miasta, a stanela na górze, która jest na wschód miasta.
\par 24 A duch podniósl mie, i zas mie przywiódl do ziemi Chaldejskiej do pojmanych, w widzeniu w Duchu Bozym. I odeszlo odemnie widzenie, którem widzial.
\par 25 I mówilem do pojmanych te wszystkie rzeczy Panskie, które mi ukazal.

\chapter{12}

\par 1 I stalo sie slowo Panskie do mnie mówiac:
\par 2 Synu czlowieczy! ty w posrodku domu odpornego mieszkasz, którzy maja oczy, aby widzieli, a nie widza, uszy maja, aby slyszeli, a nie slysza; przeto, ze domem odpornym sa.
\par 3 Ty tedy, synu czlowieczy! spraw sobie sprzet przeprowadzenia, a przeprowadzaj sie we dnie przed oczyma ich, a przeprowadzaj sie z miejsca swego na miejsce inne przed oczyma ich, aza wzdy obacza, acz domem odpornym sa.
\par 4 I wyniesiesz sprzet swój, jako sprzet przeprowadzenia, we dnie przed oczyma ich; a sam wynijdz w wieczór przed oczyma ich, jako wychodza, którzy sie przeprowadzaja.
\par 5 Przed oczyma ich przekop sobie mur, a wynies przezen sprzet twój.
\par 6 Przed oczyma ich na ramionach nies, z zmierzkiem wynies, twarz swoje zakryj, a nie patrz na ziemie; bom cie dal za dziw domowi Izraelskiemu.
\par 7 I uczynilem tak, jako mi rozkazano; sprzet mój wynioslem jako sprzet prowadzacego sie we dnie, a w wieczór przekopalem sobie mur reka; z zmierzkiem wynioslem go na ramieniu niosac przed oczyma ich.
\par 8 Znowu stalo sie slowo Panskie do mnie rano, mówiac:
\par 9 Synu czlowieczy! Izali nie rzekl do ciebie dom Izraelski, dom ten odporny: Cóz to czynisz?
\par 10 Rzeczze im, tak mówi panujacy Pan: Na ksiecia, który jest w Jeruzalemie, sciaga sie to brzemie, i na wszystek dom Izraelski, którzy sa w posród jego.
\par 11 Rzeczze do nich: Jam jest dziwem waszym; jakom uczynil, tak sie im stanie; przeprowadza sie, i w niewole pójda.
\par 12 A ksiaze, który jest w posrodku nich, na ramieniu poniesie sprzet swój z zmierzkiem, i wynijdzie; mur przekopia, aby go wywiedli przezen; twarz swoje zakryje, tak, ze nie bedzie widzial ziemi okiem swojem.
\par 13 Bo rozciagne nan siec swoje, i pojmany bedzie niewodem moim, i przywiode go do Babilonu, do ziemi Chaldejskiej, a tej nie oglada, i tam umrze.
\par 14 Takze tez wszystkich, którzy sa okolo niego, pomoc jego, i wszystkie hufy jego rozprosze na wszystkie strony, i miecza dobede za nimi;
\par 15 I poznaja, zem Ja Pan, gdy ich rozprosze miedzy narody, i rozwieje ich po ziemiach.
\par 16 Jednak zostawie z nich troche mezów po mieczu, po glodzie i po morze, aby opowiadali wszystkie obrzydliwosci swe miedzy narodami, do których wnijda, i poznaja, zem Ja Pan.
\par 17 Znowu stalo sie slowo Panskie do mnie, mówiac:
\par 18 Synu czlowieczy! chleb swój z strachem jedz i wode twoje ze drzeniem i z smutkiem pij,
\par 19 A rzecz do ludu tej ziemi: Tak mówi panujacy Pan o tych, którzy mieszkaja w Jeruzalemie, o ziemi Izraelskiej: Chleb swój z smutkiem jesc, a wode swa z trwoga pic beda, aby byla ziemia jego zlupiona z dostatków swoich dla bezprawia wszystkich mieszkajacych w niej;
\par 20 Takze miasta, w których mieszkaja, spustoszone beda, i ziemia spustoszeje; i dowiecie sie, zem Ja Pan.
\par 21 Stalo sie zas slowo Panskie do mnie, mówiac:
\par 22 Synu czlowieczy! cóz to za przypowiesc u was o ziemi Izraelskiej, iz mówicie: Przedluzac sie dni, a z tego widzenia nic nie bedzie?
\par 23 Przetoz mów do nich: Tak mówi panujacy Pan: Sprawie Ja to, iz ustanie ta przypowiesc, a nie beda uzywac tej przypowiesci wiecej w Izraelu; powiedz im: I owszem, przyblizyly sie te dni, i spelnienie wszelkiego widzenia.
\par 24 Bo nie bedzie wiecej zadnego marnego widzenia, ani wieszczby pochlebcy w posrodku domu Izraelskiego.
\par 25 Przeto, ze Ja Pan mówic bede, a którekolwiek slowo wyrzeke, stanie sie; nie pójdzie w dluga, ale za dni waszych, domie odporny! wyrzeke slowo, i wypelnie je, mówi panujacy Pan.
\par 26 I stalo sie slowo Panskie do mnie, mówiac:
\par 27 Synu czlowieczy! oto dom Izraelski mówia: To widzenie, które ten widzi, odwlecze sie na wiele dni, a o dalekich czasach ten prorokuje;
\par 28 Przeto im rzecz: Tak mówi panujacy Pan: Nie pójdziec w dluga zadne slowo moje; ale slowo, które wyrzeke, stanie sie, mówi panujacy Pan.

\chapter{13}

\par 1 I stalo sie slowo Panskie do mnie, mówiac:
\par 2 Synu czlowieczy! prorokuj przeciw prorokom Izraelskim, którzy prorokuja, a rzecz prorokujacym z serca swego: Sluchajcie slowa Panskiego.
\par 3 Tak mówi panujacy Pan: Biada prorokom glupim, którzy ida za duchem swoim, choc nic nie widzieli!
\par 4 Izraelu!prorocy twoi sa jako liszki na puszczy.
\par 5 Nie wstepujcie na przerwane miejsca, ani grodzcie plotu okolo domu Izraelskiego, zeby sie mógl ostac w bitwie w dzien Panski.
\par 6 Widza marnosc i wieszczbe klamliwa; powiadaja: Pan mówi, choc ich Pan nie poslal; i ciesza lud, aby tylko utwierdzili slowo swe.
\par 7 Izali widzenia marnego nie widzicie, a wieszczby klamliwej nie opowiadacie? I mówicie: Pan mówil, chociazem Ja nie mówil.
\par 8 Przetoz tak mówi panujacy Pan: Poniewaz mówicie marnosc, a widzicie klamstwo, przetoz oto Ja jestem przeciwko wam, mówi panujacy Pan.
\par 9 Bo reka moja bedzie przeciwko prorokom, którzy widza marnosc, a opowiadaja klamstwo; w zgromadzeniu ludu mego nie beda, a w poczet domu Izraelskiego nie beda wpisani, i do ziemi Izraelskiej nie wnijda; a dowiecie sie, zem Ja panujacy Pan.
\par 10 Przeto, przeto mówie, ze w blad wprowadzili lud mój, mówiac: Pokój, choc nie bylo pokoju; jeden zaiste zbudowal sciane gliniana, drudzy ja tynkowali wapnem nieczynionem.
\par 11 Mówze do tych, którzy ja tynkuja wapnem nieczynionem: Upadnie to, przyjdzie deszcz gwaltowny, a wy, kamienie gradowe! spadniecie, i wiatr wichrowaty rozwali ja.
\par 12 A oto gdy upadnie ona sciana, izali wam nie rzeka: Gdziez jest ono tynkowanie, któremescie tynkowali?
\par 13 Prztoz tak mówi panujacy Pan! Rozwale ja, mówie, wiatrem wichrowatym w zapalczywosci mojej, i deszcz gwaltowny w popedliwosci mojej przyjdzie, a kamienie gradowe w rozgniewaniu mojem na zniszczenie jej.
\par 14 Bo obale te sciane, którascie potynkowali wapnem nieczynionem, a zrównam ja z ziemia, tak, ze odkryty bedzie grunt jej, i upadnie, i skazeni bedziecie w posrodku jej, i dowiecie sie, zem Ja Pan.
\par 15 A gdy wykonam popedliwosc moje nad ta sciana, i nad tymi, którzy ja tynkowali wapnem nieczynionem, rzeke do was: Niemasz juz onej sciany, niemasz i tych, którzy ja tynkowali,
\par 16 To jest, proroków Izraelskich, którzy prorokuja o Jeruzalemie, i oglaszaja mu widzenie pokoju, choc niemasz pokoju, mówi panujacy Pan.
\par 17 Ale ty, synu czlowieczy! obróc twarz twoje przeciwko córkom ludu swego, które prorokuja z serca swego, a prorokuj przeciwko nim,
\par 18 I rzecz: Tak mówi panujacy Pan: Biada tym, które szyja wezglówka pod wszelkie lokcie rak ludu mojego, a czynia duchny na glowy wszelkiego wzrostu, aby lowily dusze! izali lowic macie dusze ludu mego, abyscie sie zywic mogly?
\par 19 Bo mie podawacie w lekkosc u ludu mego dla garsci jeczmienia, i dla kesa chleba, zabijajac dusze, które nie umra, a ozywiajac dusze, które zywe nie beda, klamiac ludowi memu, którzy sluchaja klamstwa.
\par 20 Dlatego, tak mówi panujacy Pan: Oto Ja bede przeciwko wezglówkom waszym, któremi wy tam dusze lowicie, abyscie je zwiodly; bo je stargne z ramion waszych, a wypuszcze dusze, które wy lowicie, abyscie je zwiodly;
\par 21 I rozerwe duchny wasze, a wybawie lud mój z reki waszej, abyscie ich wiecej nie mogly lowic reka swoja; a dowiecie sie, zem Ja Pan.
\par 22 Przeto, ze zasmucacie serce sprawiedliwego klamstwy, chociazem go Ja nie zasmucil, a zmacniacie rece niezboznego, aby sie nie odwrócil od zlej drogi swojej, ozywiajac go;
\par 23 Przetoz nie bedziecie wiecej widywac marnosci, ani wieszczby wiecej prorokowac bedziecie; bo wyrwe lud mój z reki waszej, a dowiecie sie, zem Ja Pan.

\chapter{14}

\par 1 Potem przyszedlszy do mnie mezowie z starszych Izraelskich, usiedli przedemna.
\par 2 I stalo sie slowo Panskie do mnie, mówiac:
\par 3 Synu czlowieczy! ci mezowie zlozyli plugawe balwany swoje do serca swego, a nieprawosc, która im jest ku obrazeniu. polozyli przed twarza swoja; mniemaszze, iz mie uprzejmie pytaja o rade?
\par 4 Dlategoz powiedz im, i mów do nich: Tak mówi panujacy Pan: Ktokolwiek z domu Izraelsiego polozyl plugawe balwany swoje w sercu swojem, a nieprawosc, która mu jest ku obrazeniu, polozyl przed twarza swoja, a przyszedlby do proroka, Ja Pan odpowiem temu, który przyjdzie, o mnóstwie plugawych balwanów jego.
\par 5 Abym ulapal dom Izraelski w sercu ich, ze sie odwrócili odemnie od plugawych balwanów swoich wszyscy zgola.
\par 6 Przetoz rzecz do domu Izraelskiego: Tak mówi panujacy Pan: Nawróccie sie, a cofnijcie sie od plugawych balwanów waszych, i od wszystkich obrzydliwosci waszych odwróccie twarz swoje.
\par 7 Bo ktobykolwiek z domu Izraelskiego i z przychodniów, którzy mieszkaja w Izraelu, odwrócil sie od nasladowania mnie, a polozylby plugawe balwany swoje w sercu swem, i nieprawosc, która mu jest ku obrazeniu, polozylby przed twarz swoja, i przyszedl by do proroka, aby mie przezen pytal, Ja Pan odpowiem mu sam przez sie,
\par 8 I postawie oblicze moje przeciw temu mezowi, i dam go na znak, i na przyslowie, a wytrace go z posrodku ludu mojego; i dowiecie sie, zem Ja Pan.
\par 9 A gdyby sie prorok dal zwiesc, aby mówil slowo, Ja Pan zwiodlem proroka onego, i wyciagne nan reke swoje, i wygladze go z posrodku ludu mego Izraelskiego.
\par 10 I tak poniosa nieprawosc swoje; jaka jest kazn na tego, któryby sie pytal, taka tez kazn na proroka bedzie,
\par 11 Aby wiecej nie bladzil dom Izraelski odemnie, a nie mazali sie wiecej zadnemi przestepstwami swojemi, aby byli ludem moim, a Ja abym byl Bogiem ich, mówi panujacy Pan.
\par 12 Znowu stalo sie slowo Panskie do mnie, mówiac:
\par 13 Synu czlowieczy! gdyby ziemia zgrzeszyla przeciwko mnie, dopuszczajac sie przestepstwa, tedy jezlibym wyciagnal reke moje na nia, a zlamalbym jej laske chleba, i poslalbym na nia glód, i wytracilbym z niej ludzi i zwierzeta;
\par 14 Chocby byli w posrodku niej ci trzej mezowie, Noe, Danijel i Ijob, oni w sprawiedliwosci swojej wybawiliby dusze swe, mówi panujacy Pan.
\par 15 Takze jezlibym zly zwierz przepuscil na ziemie, a osierocilby ja, i bylaby spustoszona, zeby jej nikt przechodzic nie mógl dla zwierza,
\par 16 Jako zyje Ja, mówi panujacy Pan, ze chocby ci trzej mezowie byli w posrodku jej, zadna miaraby nie wybawili synów ani córek: oniby tylko sami wybawieni byli, leczby ziemia spustoszona byla.
\par 17 Albo jezlibym przywiódl miecz na te ziemie, a rzeklbym mieczowi: Przejdz przez te ziemie, abym wytracil z niej ludzi i zwierzeta,
\par 18 Jako zyje Ja, mówi panujacy Pan, ze chocby ci trzej mezowie byli w posrodku jej, zadna miaraby nie wybawili synów ani córek, aleby oni tylko sami byli wybawieni.
\par 19 Albo poslallibym mór na te ziemie, i wylalbym popedliwosc swoje na nia ku wytraceniu, aby z niej ludzie i zwierzeta byli wytraceni.
\par 20 Ze chocby Noe, Danijel i Ijob byli w posrodku jej, jako zyje Ja, mówi panujacy Pan, zadna miaraby ani syna ani córki nie wybawili, oniby tylko w sprawiedliwosci swojej wybawili dusze swe.
\par 21 Owszem, tak mówi panujacy Pan: Chocbym cztery kazni moje ciezkie, miecz, i glód, i zly zwierz, i mór poslal na Jeruzalem, abym wytracil z niego ludzi i zwierzeta,
\par 22 A oto jezliby zostali w nim, którzyby tego uszli, a wywiedzeni byli synowie albo córki, oto i oni musza isc do was; i ogladacie droge ich, i sprawy ich, i ucieszycie sie nad tem zlem, które przywiode na Jeruzalem, nad wszystkiem, co przywiode na nie.
\par 23 A tak uciesza was, gdy ujrzycie droge ich i sprawy ich; i zrozumiecie, zem to wszystko nie darmo uczynil, com uczynil w nim, mówi panujacy Pan.

\chapter{15}

\par 1 Tedy sie stalo slowo Panskie do mnie, mówiac:
\par 2 Synu czlowieczy! cóz jest drzewo macicy winnej przeciwko wszelkiemu innemu drzewu, albo przeciwko latoroslom drzewa lesnego?
\par 3 Izali wezma z niego drzewo ku urobieniu czego? Izali urobia z niego kolek do zawieszania na nim jakiego naczynia?
\par 4 Oto ogniowi podane bywa na strawienie; gdy oba konce jego ogien strawi, a posrodek jego ogore, azaz sie na co przyda?
\par 5 Oto póki bylo cale, nic nie moglo byc z niego urobione; dopieroz gdy je ogien strawil, a spalilo sie, na nic sie wiecej nie przyda.
\par 6 Przetoz tak mówi panujacy Pan: Jako jest drzewo macicy miedzy drzewem lesnem, którem podal ogniowi na strawienie, takem podal obywateli Jeruzalemskich.
\par 7 Bo postawie oblicze swoje przeciwko nim; z jednego ognia wyjda, a drugi ogien strawi ich: i dowiecie sie, zem Ja Pan, gdy postawie twarz swoje przeciwko nim,
\par 8 A podam ziemie ich na spustoszenie, przeto, iz sie dopuscili przestepstwa, mówi panujacy Pan.

\chapter{16}

\par 1 I stalo sie slowo Panskie do mnie, mówiac:
\par 2 Synu czlowieczy! oznajmij Jeruzalemowi obrzydliwosci jego, i rzecz:
\par 3 Tak mówi panujacy Pan do córki Jeruzalemskiej: Obcowanie twoje i ród twój jest z ziemi Chananejskiej; ojciec twój jest Amorejczyk, a matka twoja Hetejka.
\par 4 A narodzenie twoje takie: W dzien, któregos sie urodzila, nie urzniono pepka twego, i woda cie nie obmyto dla ochedozenia, ani cie sola posolono, ani w pieluchy uwiniono.
\par 5 Nie zlitowalo sie nad toba oko, abyc uczynilo jedno z tych, ulitowawszy sie ciebie; ale cie porzucono na polu, przeto, zes byla obmierzla w dzien, któregos sie urodzila.
\par 6 A idac mimo cie, i widzac cie ku podeptaniu podana we krwi twojej, rzeklem ci: Zyj we krwi twojej; rzeklem ci, mówie: Zyj we krwi twojej.
\par 7 Rozmnozylem cie na tysiace, jako urodzaj polny, i rozmnozonas, i stalas sie wielka, a przyszlas do bardzo wielkiej ozdoby; piersi twoje odely sie, a wlosy twoje urosly, chociazes byla naga i odkryta.
\par 8 Prztoz idac mimo cie, a widzac cie, ze oto czas twój, czas milosci, rozciagnalem podolek mój na cie, i nakrylem nagosc twoje, i obowiazalem ci sie przysiega, a wszedlem w przymierze z toba, mówi panujacy Pan, i stalas sie moja.
\par 9 I omylem cie woda, a splukawszy krew twoje z ciebie, pomazalem cie olejkiem;
\par 10 Nadto przyodzialem cie szata haftowana, i obulem cie w kosztowne trzewiki, i opasalem cie bisiorem, a przyodzialem cie szata jedwabna;
\par 11 I przybralem cie w ochedostwo, a dalem manele na rece twoje, i lancuch zloty na szyje twoje;
\par 12 Dalem tez naczelnik na czolo twoje, a nausznice na uszy twoje, i korone ozdobna na glowe twoje;
\par 13 A tak bylas ozdobiona zlotem i srebrem, a odzienie twoje bylo bisior, i szata jedwabna, i haftowana; jadalas bulke i miód, i oliwe, a bylas nader piekna, i szczesliwiec sie powodzilo w królestwie,
\par 14 Tak, ze sie rozeszla powiesc o tobie miedzy narodami dla pieknosci twojej; bos doskonala byla dla slawy mojej, któram byl wlozyl na cie, mówi panujacy Pan.
\par 15 Ales ufala w pieknosci twojej, i plodzilas nierzad, bedac tak slawna; bos plodzila nierzad z kazdym mimo cie idacym, kazdy snadnie uzyl pieknosci twojej.
\par 16 A nabrwaszy szat swoich naczynilas sobie wyzyn rozmaitych farb, a plodzilas nierzad przy nich, któremu podobny nigdy nie przyjdzie, ani bedzie.
\par 17 Nadto nabrawszy klejnotów ozdoby swojej ze zlota mego i z srebra mego, którem ci byl dal, naczynilas sobie obrazów poglowia meskiego, i plodzilas z nimi nierzad.
\par 18 Wzielas tez szaty swe haftowane, a przyodzialas je, olejek mój i kadzidlo moje kladlas przed nie;
\par 19 Nawet i chleb mój, którem ci dal, bulke i oliwe, i miód, którymem cie karmil, kladlas przed nie na wonnosc przyjemna, i bylo tak, mówi panujacy Pan.
\par 20 Bralas tez synów swoich, i córki swoje, któryches mi narodzila, a ones im ofiarowala ku pozarciu; izali to jeszcze male wszeteczenstwa twoje?
\par 21 Do tego i synówes moich zabijala, a dawalas ich, aby ich przenoszono przez ogien im kwoli.
\par 22 Nadto przy wszystkich obrzydliwosciach swoich, i wszeteczenstwach swoich nie pamietalas na dni mlodosci twojej, gdys byla naga i odkryta, podana na podeptanie we krwi twojej.
\par 23 Nadto mimo wszystkich onych zlosci twoich (Biada, biada tobie! mówi panujacy Pan.)
\par 24 Zbudowalas sobie dom nierzadny, i wystawilas sobie wyzyne w kazdej ulicy.
\par 25 Po wszystkich rozstaniach dróg pobudowalas wyzyny swoje, a uczynilas obmierzla pieknosc swoje, rozkladajac nogi swoje kazdemu mimo idacemu, i rozmnozylas wszeteczenstwa swoje.
\par 26 Bos nierzad plodzila z synami Egipskimi, sasiadami twymi wielkich cial, a rozmnozylas wszeteczenstwa twe, abys mie do gniewu pobudzala.
\par 27 Przetoz otom wyciagnal reke moje na cie, a uminiejszylem obroku twego, i podalem cie na wole nienawidzacych cie córek Filistynskich, które sie wstydza za zle drogi twoje.
\par 28 Nadto plodzilas nierzad z synami Assyryjskimi, przeto, zes sie nie mogla nasycic, a nierzad plodzac z nimi, i takes sie nie nasycila.
\par 29 A tak rozmnozylas wszeteczenstwo swe w ziemi Chananejskiej i Chaldejskiej, a i tak nie nasycilas sie.
\par 30 O jako jest zmamione serce twoje! mówi panujacy Pan, poniewaz sie dopuszczasz tych postepków niewiasty nierzadnej i wszetecznej.
\par 31 Budujac sobie nierzadne domy na rozstaniu kazdego goscinca, a wyzyne sobie stawiajac w kazdej ulicy, owszem pogardzajac zaplata nie jestes ani jako wszetecznica,
\par 32 Ani jako niewiasta cudzolozaca, która mimo meza swego obcych przypuszcza.
\par 33 Wszystkim wszetecznicom dawaja zaplate; ales ty dawala zaplate swoje wszystkim zalotnikom twoim, i dawalas im upominki, aby chodzili do ciebie zewszad na wszeteczenstwa z toba.
\par 34 A tak znajduje sie przy tobie przeciwna rzecz od innych niewiast we wszeteczenstwach twoich, poniewaz cie dla wszeteczenstwa nie szukaja; ale ty dajesz zaplate, a nie tobie zaplate dawaja, co sie opak dzieje.
\par 35 Przetoz, o wszetecznico! sluchaj slowa Panskiego.
\par 36 Tak mówi panujacy Pan: Dlatego, iz sie wylala sprosnosc twoja, i odkryla sie nagosc twoja we wszeteczenstwach twoich z zalotnikami twoimi i ze wszystkimi plugawemi balwanami obrzydliwosci twoich, i dla rozlania krwi synów twoich, któryches im dal a.
\par 37 Przetoz oto Ja zgromadze wszystkich zalotników twoich, z którymis obcowala, i wszystkich, któryches milowala, ze wszystkimi, cos ich nienawidziala, i zgromadze ich przeciwko tobie zewszad, i odkryje nagosc twoje przed nimi, aby widzieli wszystke na gosc twoje;
\par 38 I osadze cie sadem cudzoloznic i krew rozlewajacych, i podam cie na smierc gniewu i zapalczywosci.
\par 39 Bo cie podam w rece ich, i zburza dom twój nierzadny, a rozrzuca wyzyny twoje, i zewloka cie z szaty twojej, i pobiora klejnoty ozdoby twojej, i zostawia cie naga i odkryta;
\par 40 I przywioda na cie gromade, i ukamionuja cie kamieniem, i przebija cie mieczami swemi;
\par 41 I popala domy twoje ogniem, a uczynia nad toba sad przed oczyma wielu niewiast. A tak uczynie wstret wszeteczenstwu twemu, i nie bedziesz wiecej dawala zaplaty.
\par 42 A tak odpocznie sobie gniew mój na tobie, i odstapi zapalczywosc moja od ciebie, i uspokoje sie, a nie bede sie wiecej gniewal.
\par 43 Dlatego, zes nie pamietala na dni mlodosci twojej, ales mie tem wszystkiem draznila, przetoz oto i Ja obrócilem droge twoje na glowe twoje, mówi panujacy Pan, tak, ze nie bedziesz nierzadu plodzila, ani jakich obrzydliwosci swoich.
\par 44 Oto ktokolwiek przez przypowiesci mówi, na cie przypowiesci obróci, mówiac: Jaka matka, taka córka jej.
\par 45 Córka matki twojej jestes, która sobie zbrzydzila meza swego i dziatki swoje; a siostra obu sióstr swoich jestes, które sobie zbrzydzily mezów swoich i dziatki swoje. Matka wasza jest Hetejka, a ojciec wasz Amorejczyk.
\par 46 Ale siostra twoja starsza, która siedzi po lewicy twojej, jest Samaryja i córki jej; a siostra twoja mlodsza niz ty, która siedzi po prawicy twojej, jest Sodoma i córki jej.
\par 47 Aczkolwiekes po drogach ich nie chodzila, anis wedlug obrzydliwosci ich czynila, obrzydziwszy to sobie jako rzecz mala, przecies sie bardziej niz one popsowala na wszystkich drogach twoich.
\par 48 Jako zyje Ja, mówi panujacy Pan, ze Sodoma siostra twoja i córki jej nie czynily, jakos ty czynila i córki twoje.
\par 49 Oto ta byla nieprawosc Sodomy, siostry twojej, pycha, sytosc chleba, i obfitosc pokoju; co ona majac i córki jej, reki jednak ubogiego i nedznego nie posilala.
\par 50 Owszem, wynióslszy sie, czynily obrzydliwosc przed obliczem mojem; przetozem je zniósl, jako mi sie zdalo.
\par 51 Samaryja takze ani polowy grzechów twoich nie grzeszyla; bos rozmnozyla obrzydliwosci twoje nad nia; a tak usprawiedliwilas siostry twoje obrzydliwosciami twojemi, któres czynila.
\par 52 Ponosze i ty hanbe swoje, któras przysadzila siostrom swoim, dla grzechów swych, któremis obrzydliwosci czynila bardziej niz one, sprawiedliwszemic byly nizeli ty; i ty, mówie, zawstydz sie, a nos na sobie hanbe swoje, gdyzes usprawiedliwila sios try twoje.
\par 53 Przywróceli zaz wiezniów ich, to jest, wiezniów Sodomy i córek jej, i wiezniów Samaryji i córek jej; tedyc tez przywiode pojmanych wiezniów twoich w posrodku ich;
\par 54 Abys tak nosila hanbe twoje, a wstydzila sie za wszystko, cos czynila, a tak abys je ucieszyla.
\par 55 Jezlizec siostry twoje, Sodoma i córki jej, wróca sie do pierwszego stanu swego, takze Samaryja i córki jej wróca sie do pierwszego stanu swego; tedy sie tez i ty z córkami swemi nawrócisz do pierwszego stanu swego.
\par 56 Poniewaz Sodoma siostra twoja nie byla powiescia w ustach twoich w dzien pychy twojej.
\par 57 Pierwej niz byla objawiona zlosc twoja, jako za czasu oblezenia od córek Syryjskich, i wszystkich, którzy sa okolo nich, córek Filistynskich, które cie niszczyly ze wszystkich stron.
\par 58 Niecnote twoje i obrzydliwosc twoje ponosisz, mówi Pan.
\par 59 Bo tak mówi panujacy Pan: Tak uczynie z toba, jakos uczynila, gdys wzgardzila przysiega, i zlamala przymierze.
\par 60 Wszakze wspomne na przymierze moje z toba, uczynione za dni mlodosci twojej, i stwierdze z toba przymierze wieczne.
\par 61 I wspomnisz na drogi twoje, i zawstydzisz sie, gdy przyjmiesz siostry twoje starsze nad cie, i mlodsze niz ty, i dam ci je za córki, ale nie wedlug przymierza twego.
\par 62 A tak utwierdze przymierze moje z toba, a dowiesz sie, zem Ja Pan.
\par 63 Abys wspomniala, i zawstydzila sie, i nie mogla wiecej otworzyc ust dla wstydu swego, gdy cie oczyszcze od wszystkiego, cos czynila, mówi panujacy Pan.

\chapter{17}

\par 1 Stalo sie zas slowo Panskie do mnie, mówiac:
\par 2 Synu czlowieczy! zadaj zagadke, a mów w podobienstwie o domu Izraelskim,
\par 3 I rzecz: Tak mówi panujacy Pan: Orzel wielki z wielkiemi skrzydlami i z dlugiemi pióry, pelen pierza pstrego, przylecial na Liban, i wzial wierzch cedru.
\par 4 Wierzch mlodocianych latorosli jego ulamal, i zaniósl go do ziemi kupieckiej, a w miescie kupieckiem polozyl go.
\par 5 Wzial tez nasienia onej ziemi, a wsadzil je na polu urodzajnem, a wsadzil je bardzo ostroznie przy wodach wielkich;
\par 6 Które weszloby bylo, i byloby winna macica bujna, choc niskiego wzrostu; i bylyby latorosli jej ku niemu, a korzenie jej bylyby mu poddane. A tak byloby bylo macica winna, któraby byla wydala latorosli, i wypuscila galazki.
\par 7 Ale byl orzel jeden wielki z wielkiemi skrzydlami i z gestem pierzem, a oto ona winna macica przypoila korzenie swoje ku niemu, i galazki swe rozciagnela do niego, aby ja odwilzal z brózd sadu swego.
\par 8 Choc na polu dobrem przy wodach wielkich wsadzona byla, aby wypuscila latorosli, i przyniosla owoce, i byla macica winna wspaniala.
\par 9 Rzeczze tedy: Tak mówi panujacy Pan: Izali sie jej poszczesci? Izali orzel korzenia jej nie wyrwie, i owocu jej nie oberwie, i nie posuszy? Izali wszystkich latorosli wyroslych z niej nie ususzy? Izali z wielka moca a z obfitym ludem jej nie wygladzi z korzenia jej?
\par 10 Oto jakozkolwiek wsadzona jest, izali sie jej poszczesci? Izali do szczetu nie uschnie, skoro sie jej dotknie wiatr wschodni? Izali przy brózdach, przy którch sie przyjela, nie uschnie?
\par 11 Zatem stalo sie slowo Panskie do mnie, mówiac:
\par 12 Mów teraz do domu odpornego: Azaz nie wiecie, co to jest? Zatem mów: Oto przyciagnal król Babilonski do Jeruzalemu, i zabral króla jego, i ksiazat jego, i zawiódl ich z soba do Babilonu.
\par 13 Wzial tez z nasienia królewskiego, a uczyniwszy z nim przymierze, zawiazal go przysiega, i mocnych z onej ziemi zabral,
\par 14 Aby bylo królestwo znizone, przeto, zeby sie nie wynioslo, zeby tak strzegac przymierza jego, stalo.
\par 15 Ale mu sie sprzeciwil, poslawszy poslów swych do Egiptu, aby mu dal koni i wiele ludu. Izali sie to poszczesci? Izali pomsty ujdzie ten, co tak uczynil? Ten, który wzrusza przymierze, izali pomsty ujdzie?
\par 16 Jako zyje Ja, mówi panujacy Pan, ze na miejscu króla tego, który go królem uczynil, którego przysiega wzgardzil, a którego przymierze wzruszyl, u niego w Babilonie umrze.
\par 17 Ani mu Farao z wielkim wojskiem i z ludem gestym co pomoze na wojnie, gdy usypie wal, i porobi szance, aby wykorzenil mnóstwo ludu,
\par 18 Poniewaz pogardzil przysiega, zrzuciwszy przymierze; bo oto dal na to reke swa, a przecie to wszystko uczyni; nie ujdzie pomsty.
\par 19 Dlatego tak mówi panujacy Pan: Jako zyje Ja, ze przysiege swoje, która wzgardzil, i przymierze moje, które wzruszyl, pewnie obróce na glowe jego.
\par 20 Bo rozciagne nan siec moje, a bedzie niewodem moim pojmany, i zawiode go do Babilonu, a tam sie z nim rozsadze o wystepek jego, którym wystapil przeciwko mnie.
\par 21 A wszyscy tez, którzy uciekli od niego ze wszystkiemi hufami jego, od miecza polegna, a pozostali na wszystkie strony rozproszeni beda, a dowiecie sie, ze Ja Pan mówilem to.
\par 22 Tak mówi panujacy Pan: Wszakze wezme z wierzchu tego wysokiego cedru, i wsadze; z wierzchu mlodocianych latorosli jego mloda latorostke ulamie, a wszczepie ja na górze wysokiej i wynioslej;
\par 23 Na wysokiej górze Izraelskiej wszczepie ja, i wypusci galazki, i wyda owoc, i bedzie cedrem zacnym; i bedzie pod nim mieszkac wszelkie ptastwo, i wszystko, co ma skrzydla, pod cieniem galezia jego mieszkac bedzie.
\par 24 A tak poznaja wszystkie drzewa polne, zem Ja Pan ponizyl drzewo wysokie, a wywyzszylem drzewo niskie, ususzylem drzewo zielone, a uczynilem, ze zakwitlo drzewo suche. Ja Pan rzeklem to, i uczynie.

\chapter{18}

\par 1 I stalo sie slowo Panskie do mnie, mówiac:
\par 2 Cóz wam po tem, iz uzywacie tej przypowiesci o ziemi Izraelskiej mówiac: Ojcowie jedli jagode cierpka, a synów zeby dretwieja.
\par 3 Jako zyje Ja, mówi panujacy Pan, ze wy nie bedziecie wiecej mogli uzywac tej przypowiesci w Izraelu.
\par 4 Oto dusze wszystkie moje sa, jako dusza ojcowska tak i dusza synowska moje sa; dusza, która grzeszy, ta umrze.
\par 5 Bo bylliby maz sprawiedliwy, a czynilby sad i sprawiedliwosc,
\par 6 Któryby na górach nie jadal, a oczówby swych nie podnosil do plugawych balwanów domu Izraelskiego, a zonyby blizniego swego nie zmazal, i do niewiasty dla nieczystoty oddalonej nie przystapil;
\par 7 Któryby nikogo nie uciskal, zastawby dluznikowi swemu wracal, cudzegoby gwaltem nie bral, chlebaby swego laknacemu udzielal, a nagiegoby szata przyodziewal;
\par 8 Któryby na lichwe nie dawal, i platuby nie bral, od nieprawosciby odwracal reke swoje, a sadby sprawiedliwy czynil miedzy mezem a mezem;
\par 9 Któryby w ustawach moich chodzil, a sadówby moich przestrzegal, czyniac, co jest prawego: ten sprawiedliwy pewnie zyc bedzie, mówi panujacy Pan.
\par 10 A gdyby splodzil syna lotra, krew wylewajacego, któryby czemkolwiek z tych rzeczy bratu szkodzil,
\par 11 A tegoby wszystkiego nie czynil, owszemby i na górach jadal, i zoneby blizniego swego zmazal,
\par 12 Ubogiegoby i nedznego uciskal, co cudzego jest, gwaltemby zabieral, zastawuby nie wracal, a do plugawych balwanów podnosilby oczy swoje, i obrzydliwosciby czynil,
\par 13 Na lichweby dawal, i plat bral, izali zyc bedzie? Nie bedzie zyl; poniewaz te wszystkie obrzydliwosci czynil: smiercia umrze, krew jego na nim bedzie.
\par 14 A oto jezeliby splodzil syna, któryby widzial wszystkie grzechy ojca swego, które czynil, a widzac nie czynilby tak;
\par 15 Na górachby nie jadal, a oczówby swych nie podnosil do plugawych balwanów domu Izraelskiego, zonyby blizniego swego nie zmazal,
\par 16 I nikogoby nie uciskal, zastawuby nie zatrzymywal, a cudzegoby gwaltem nie bral, chlebaby swego laknacemu udzielal, a nagiegoby szata przyodzial,
\par 17 Od nieprawegoby odwrócil reke swoje, lichwyby i platu nie bral, sadyby moje czynil, w ustawachby moich chodzil: ten nie umrze dla nieprawosci ojca swego, ale pewnie zyc bedzie.
\par 18 Lecz ojciec jego, przeto, ze czynil krzywde, co jest cudzego, bratu gwaltem bral, a to, co jest dobrego, nie czynil w posrodku ludu swego: przetoz oto umrze dla nieprawosci swojej.
\par 19 Ale mówicie: Czemuz? Izali nie poniesie syn nieprawosci ojcowskiej? Gdy syn sad i sprawiedliwosc czyni, wszystkich ustaw moich strzeze i czyni je, pewnie zyc bedzie.
\par 20 Dusza, która grzeszy, ta umrze; ale syn nie poniesie nieprawosci ojcowskiej, ani ojciec poniesie nieprawosci synowskiej; sprawiedliwosc sprawiedliwego przy nim zostanie, a niepoboznosc niepoboznego nan przypadnie.
\par 21 A jezliby sie niepobozny odwrócil od wszystkich grzechów swoich, które czynil, a strzeglby wszystkich ustaw moich, i czynilby sad i sprawiedliwosc, pewnie zyc bedzie, nie umrze;
\par 22 Zadne przestepstwa jego, których sie dopuscil, nie beda mu przypominane; w sprawiedliwosci swej, któraby czynil, zyc bedzie.
\par 23 Azaz Ja sie kocham w smierci niepoboznego? mrwi panujacy Pan. Izali nie raczej, gdy sie odwróci od dróg swoich, aby zyl?
\par 24 Ale jezliby sie odwrócil sprawiedliwy od sprawiedliwosci swojej, a czynilby nieprawosc, czyniac wedlug wszystkich obrzydliwosci, które czyni niezbozny, izali taki zyc bedzie? Wszystkie sprawiedliwosci jego, które czynil, nie beda wspominane; dla przestepstwa swego, które popelnial, i dla grzechu swego, którego sie dopuscil, dla tych rzeczy umrze.
\par 25 A iz mówicie: Nie prosta jest droga Panska; sluchajciez teraz, o domie Izraelski! izali droga moja nie jest prosta? Izali drogi wasze nie sa krzywe?
\par 26 Gdyby sie odwrócil sprawiedliwy od sprawiedliwosci swojej, a czyniac nieprawosc w temby umarl, dla nieprawosci swojej, która czynil, umrze.
\par 27 Ale gdyby sie odwrócil niezbozny od niezboznosci swojej, której sie dopuscil, a czynilby sad i sprawiedliwosc, ten dusze swoje zachowa.
\par 28 Bo obaczywszy sie odwrócil sie od wszystkich wystepków swoich, których sie dopuszczal, pewnie zyc bedzie, nie umrze.
\par 29 A przecie mówi dom Izraelski: Nie prosta jest droga Panska; izali drogi moje nie sa proste, o domie Izraelski? Izali nie raczej drogi wasze sa krzywe?
\par 30 A przetoz kazdego z was wedlug dróg jego sadzic bede, o domie Izraelski! mówi panujacy Pan. Nawrócciez sie, a odwróccie sie od wszystkich wystepków waszych, aby wam nieprawosc nie byla na obrazenie.
\par 31 Odrzuccie od siebie wszystkie przestepstwa wasze, którychescie sie dopuszczali, a uczyncie sobie serce nowe i ducha nowego. I przeczze macie umrzec, o domie Izraelski?
\par 32 Albowiem sie Ja nie kocham w smierci umierajacego, mówi panujacy Pan; nawrócciez sie tedy, a zyc bedziecie.

\chapter{19}

\par 1 A ty uczyn narzekanie nad ksiazetami Izraelskimi,
\par 2 A mów: Cóz byla matka twoja? Lwica miedzy lwami lezaca, która w posrodku lwiat wychowywala szczenieta swoje.
\par 3 A gdy odchowala jedno z szczeniat swoich, stalo sie lwem, tak, ze nauczywszy sie chwytac lupu pozeral i ludzi.
\par 4 To gdy uslyszaly o nim narody, w jamie ich pojmany jest, a zawiedziony w lancuchach do ziemi Egipskiej.
\par 5 Co widzac lwica, ze nadzieja jej, która miala, zginela, wziawszy jedno z szczeniat swoich, lwem je uczynila;
\par 6 Który chodzac w posrodku lwów stal sie lwem, a nauczywszy sie chwytac lupu pozeral i ludzi.
\par 7 Poburzyl tez palace ich, i miasta ich spustoszyl, tak, iz i ziemia i pelnosc jej od glosu ryku jego spustoszala.
\par 8 I zeszly sie przeciwko niemu narody z okolicznych krain, i zarzucili nan sieci swoje; a tak w jamie ich pojmany jest.
\par 9 I wsadzili go do klatki w lancuchach, i przywiedli go do króla Babilonskiego, i wprowadzili go do wiezienia ciezkiego, aby wiecej nie byl slyszany glos jego po górach Izraelskich.
\par 10 Matka twoja byla czasu pokoju twego jako winna macica przy wodach szczepiona; plodna i galezista byla dla wód obfitych.
\par 11 I miala rózgi mocne na sceptry panujacych, a wywyzszyl sie wzrost jej miedzy gestemi galeziami, tak, ze byla okazala dla wysokosci swojej, i dla mnóstwa latorosli swoich.
\par 12 Ale w rozgniewaniu wyrwana bedac, na ziemie porzucona jest, a wiatr wschodni ususzyl owoc jej; oblamane sa i poschly rózgi mocy jej, ogien pozarl je.
\par 13 A teraz wszczepiona jest na puszczy w ziemi suchej i pragnacej.
\par 14 Nadto wyszedl ogien z rózgi latorosli jej, a pozarl owoc jej, tak, ze niemasz na niej rózgi mocnej dla sceptru panujacego. Toc jest narzekanie, i bedzie narzekaniem.

\chapter{20}

\par 1 I stalo sie roku siódmego, miesiaca piatego, dziesiatego dnia tegoz miesiaca, przyszli niektórzy z starszych Izraelskich, aby sie radzili Pana; i usiedli przedemna.
\par 2 Tedy sie stalo slowo Panskie do mnie, mówac:
\par 3 Synu czlowieczy! mów do starszych Izraelsich, a powiedz im: Tak mówi panujacy Pan: Izali wy przychodzicie, abyscie sie mnie radzili? Jako zyje Ja, ze sie wy mnie nie radzicie, mówi panujacy Pan;
\par 4 Izali sie za nich zastawiac bedziesz? Izali sie za nich zastawiac bedziesz, synu czlowieczy? Oznajmij im raczej obrzydliwosci ojców ich,
\par 5 A rzecz do nich: Tak mówi panujacy Pan: Tego dnia, któregom wybral Izraela, podnioslem reke moje nasieniu domu Jakóbowego, i dalem sie im poznac w ziemi Egipskiej; podnioslem reke moje dla nich, mówiac: Jam Pan, Bóg wasz.
\par 6 Onego dnia podnioslem reke moje dla nich, ze ich wywiode z ziemi Egipskiej do ziemi, któram im upatrzyl, oplywajacej mlekiem i miodem, która jest ozdoba wszystkich ziem.
\par 7 I rzeklem im: Kazdy z was niech porzuci obrzydliwosci oczów swoich, a nie kalajcie sie plugawemi balwanami egipskiemi; bom Ja Pan, Bóg wasz.
\par 8 Ale mi byli odpornymi, i nie chcieli mie sluchac; zaden z nich obrzydliwosci oczów swoich nie odrzucil, i plugawych balwanów egipskich nie opóscil; prztozem rzekl: Wyleje gniew mój na nich, a wypelnie popedliwosc moje nad nimi w posrodku ziemi egipskiej.
\par 9 A wszakzem uczynil dla imienia mego, aby nie bylo zelzone przed oczyma tych narodów, miedzy którymi oni byli, przed których oczyma dalem sie im poznac, ze ich chce wywiesc z ziemi egipskiej.
\par 10 A tak wywiodlem ich z ziemi egipskiej, i przyprowadzilem ich na puszcze;
\par 11 I dalem im ustawy moje, i sady moje podalem im do wiadomosci; które jezliby czlowiek zachowywal, zyc w nich bedzie.
\par 12 Nadto i sabaty moje dalem im, aby byly znakiem miedzy mna i miedzy nimi, aby wiedzieli, zem Ja jest Pan, który ich poswiecam.
\par 13 Ale mi sie odpornym stal dom Izraelski na puszczy; w ustawach moich nie chodzili, i sady moje odrzucili, które jezliby czlowiek czynil, zyc w nich bedzie. Sabaty tez moje bardzo zgwalcili; prztozem mówil, iz wleje popedliwosc moje na nich na pus zczy, abym ich wygubil.
\par 14 Leczem uczynil dla imienia mego, aby nie bylo zelzone przed oczyma tych narodów przed którychem ich oczyma wywiódl.
\par 15 A wszakzem podniósl reke moje dla nich na tej puszczy, ze ich nie wprowadze do ziemi, któram im byl dal, oplywajacej mlekiem i miodem, i która jest ozdoba wszystkich ziem;
\par 16 Poniewaz sady moje odrzucili, a w ustawach moich nie chodzili, i sabaty moje pogwalcili, a ze serce ich za plugawemi balwanami chodzi.
\par 17 Ale im przecie przepuscilo oko moje, tak, zem ich nie wytracil, i nie wygladzil do szczetu na puszczy.
\par 18 I mówilem do synów ich na tej puszczy: W ustawach ojców waszych nie chodzcie, i sadów ich nie przestrzegajcie, ani sie plugawemi balwanami ich kalajcie.
\par 19 Jam Pan, Bóg wasz; w ustawach moich chodzcie, a sadów moich strzezcie, i czyncie ich;
\par 20 Sabaty tez moje swieccie, i beda znakiem miedzy mna i miedzy wami, abyscie wiedzieli zem Ja Pan, Bóg wasz.
\par 21 Lecz mi byli odpornymi ci synowie; w ustawach moich nie chodzili, i sadów moich nie przestrzegali, aby je czynili, (które jezliby czynil czlowiek pewnieby zyl w nich) i sabaty moje pogwalcili. I rzeklem: Wyleje popedliwosc moje na nich, abym wyko nal gniew swój na nich na tej puszczy.
\par 22 Alem odwrócil reke moje, a uczynilem to dla imienia mego, aby nie bylo zelzone przed oczyma tych narodów, przed którychem ich oczyma wywiódl.
\par 23 Alem ja podniósl reke moje dla nich na puszczy, zem ich mial rozproszyc miedzy pogan, i rozwiac ich po ziemiach,
\par 24 Przeto, ze sadów moich nie czynili, i ustawy moje odrzucili, i sabaty moje pogwalcili, a za plugawemi balwanami ojców swoich oczy swe obrócili.
\par 25 Dlategoz i Jam im dal ustawy nie dobre i sady, w których zyc nie beda;
\par 26 I splugawilem ich z darami ich, gdy przewodzili przez ogien wszelkie otwarzajace zywot, abym ich spustoszyl, a zeby sie dowiedzieli zem ja Pan.
\par 27 Prztoz mów do domu Izraelskigo, synu czlowieczy! a powiedz im: Tak mówi panujacy Pan: Jeszcze i w tem lzyli mie ojcowie wasi, dopuszczajac sie przeciwko mnie przestepstwa,
\par 28 Ze gdym ich wwiódl do ziemi, o któram byl podniósl reke moje, zem ja im dac mial, gdzie ujrzeli jaki pagórek wysoki, albo jakie drzewo galeziste, zaraz tam ofiarowali ofiary swoje, i dawali tam drazniace dary swoje, tamze kladli i wdzieczna wonno sc swoje, tamze sprawowali mokre ofiary swoje.
\par 29 A chociazem mówil do nich: Cóz to za wyzyna, do której wy chadzacie? Przecie ja zowia wyzyna az do dnia tego.
\par 30 Przetoz rzecz domowi Izraelskiemu: Tak mówi panujacy Pan: Izali sie wy drogami ojców waszych kalac macie, a z obrzydliwosciami ich nierzad plodzic?
\par 31 I maciez sie kalac przy wszystkich plugawych balwanach waszych, przynoszac dary wasze i przewodzac synów waszych przez ogien az do dnia tego, a przecie odemnie rady szukac, o domie Izraelski? Jako zyje Ja, mówi panujacy Pan, ze sie wy mnie wiecej radzic nie bedziecie.
\par 32 A to, co wstepuje na mysl wasze, nigdy sie nie stanie. Wy mówicie: Bedziemy jako inne narody, jako pokolenia innych ziem, drewnu i kamieniowi sluzace;
\par 33 Jako zyje Ja, mówi panujacy Pan, ze reka mozna i ramieniem wyciagnionem, a w popedliwosci wylanej bede królowal nad wami;
\par 34 I wywiode was z narodów, a zgromadze was z ziem, do którychescie rozproszeni, reka mozna, i ramieniem wyciagnionem, i w popedliwosci wylanej;
\par 35 A prowadzac was po puszczy tych narodów, tam sie z wami twarza w twarz sadzic bede.
\par 36 Jakom sie sadzil z ojcami waszymi na puszczy ziemi Egipskiej, tak sie z wami sadzic bede, mówi panujacy Pan;
\par 37 I popedze was pod rózga, abym was przywiódl do zwiazku przymierza.
\par 38 Ale z was wybiore odpornych i wystepujacych przeciwko mnie; z ziemi pielgrzymstwa ich wywiode ich, wszakze do ziemi Izraelskiej nie wnijda; i poznacie, zem Ja Pan.
\par 39 Wy tedy, o domie Izraelski! tak mówi panujacy Pan: Idzciez, sluzcie kazdy plugawym balwanom swoim i na potem, poniewaz mnie nie sluchacie; a imienia mego swietego nie kalajcie wiecej darami waszemi, i plugawemi balwanami waszemi.
\par 40 Bo na swietej górze mojej, na wysokiej górze Izraeskiej, mówi panujacy Pan, tam mi sluzyc bedzie wszystek dom Izraeski, ile ich bedzie w onej ziemi; tam ich laskawie przyjme, i tam sie pytac bede o ofiarach waszych podnoszonych, i o pierwiastkach darów waszych ze wszystkiemi swietemi rzeczami waszemi.
\par 41 Z wdzieczna wonnoscia laskawie was przyjme, gdy was wywiode z narodów, a zgromadze was z onych ziem, do którychescie rozproszeni byli, a tak poswiecony bede w was przed oczyma onych narodów.
\par 42 A poznacie, zem Ja Pan, gdy was wprowadze do ziemi Izraelskiej, do ziemi onej, o któram podniósl reke moje, ze ja dam ojcom waszym;
\par 43 A wspomnicie tam na drogi wasze, i na wszystkie sprawy wasze, któremiscie sie splugawili, a obmierzniecie sami sobie przed obliczem waszem dla wszystkich zlosci waszych, którescie czynili.
\par 44 Tam sie dowiecie, zem Ja Pan, gdy wam to uczynie dla imienia mego, nie wedlug dróg waszych zlych, ani wedlug spraw waszych skazonych, o domie Izraelski! mówi panujacy Pan.
\par 45 I stalo sie slowo Panskie do mnie, mówiac:
\par 46 Synu czlowieczy! obróc twarz twoje ku stronie poludniowej, a krop jako rosa na poludnie, i prorokuj przeciwko losowi pola poludniowego,
\par 47 A rzecz lasowi poludniowemu: Sluchaj slowa Panskiego: Tak mówi panujacy Pan: Oto Ja rozniece w tobie ogien, który pozre w tobie wszelkie drzewo zielone, i wszelkie drzewo suche; nie bedzie ugaszony plomien palajacy, i zgoreja w nim wszystkie twar ze od poludnia az do pólnocy.
\par 48 I ujrzy wszelkie cialo, zem go Ja Pan zapalil; nie bedzie ugaszony.
\par 49 I rzeklem: Ach panujacy Panie! Oni mówia o mnie: Ten tylko w przypowiesciach mówi.

\chapter{21}

\par 1 I stalo sie slowo Panskie do mnie, mówiac:
\par 2 Synu czlowieczy! obróc twarz swoje ku Jeruzalemowi, a krop jako rosa przeciwko miejscom swietym, i prorokuj przeciwko ziemi Izraelskiej,
\par 3 A powiedz ziemi Izraelskiej: Tak mówi panujacy Pan: Otom Ja przeciwko tobie, i dobede miecza mego z pochew jego, i wytne z ciebie sprawiedliwego i niezboznego.
\par 4 Abym tedy wycial z ciebie sprawiedliwego, i niezboznego, przeto wynijdzie miecz mój z pochew swoich na wszelkie cialo od poludnia az ku pólnocy.
\par 5 I dowie sie wszelkie cialo, zem Ja Pan dobyl miecza mego z pochew jego; a nie obróci sie wiecej.
\par 6 A ty, synu czlowieczy! wzdychaj, jakobys mial biodro zlamane, a w gorzkosci wzdychaj przed oczyma ich.
\par 7 A gdyc rzeka: Dlaczego wzdychasz? Tedy odpowiesz: Dla wiesci, która idzie, na która sie rozplynie wszelkie serce, i oslabieja wszelkie rece, i scisniony bedzie wszelki duch, i wszelkie kolano rozplynie sie jako woda; oto idzie, i stanie sie, mówi panujacy Pan.
\par 8 I stalo sie slowo Panskie do mnie, mówiac:
\par 9 Synu czlowieczy! prorokuj, a rzecz: Tak mówi Pan: Powiedz, miecz, miecz naostrzony jest i wypolerowany;
\par 10 Wyostrzony jest na pomordowanie ku zabiciu naznaczonych, wypolerowany jest, aby sie lsnil. Izali sie weselic mamy? Gdyz rózga syna mego kazde drzewo lekce wazy.
\par 11 Dalci go na wypolerowanie, aby mógl byc ujety reka, jestci wyostrzony ten miecz, jest i wypolerowany, aby dany byl do reki zabijajacego.
\par 12 Wolaj a kwil, synu czlowieczy, przeto, ze ten miecz bedzie przeciwko ludowi memu, ten tez przeciwko wszystkim ksiazetom Izraelskim; strachy miecza przyjda na lud mój; przetoz sie w biodre uderz.
\par 13 Gdym ich karal, cóz to pomoglo? I nie mamze rózgi wszystko lekce powazajacej na nich wyciagnac? mówi panujacy Pan.
\par 14 Ty tedy, synu czlowieczy! prorokuj, a bij reka w reke; bo powtóre i potrzecie miecz przyjdzie, miecz mordujacych, ten miecz mordujacych bez litosci, przenikajacy az do pokojów ich.
\par 15 We wszystkich bramach ich dalem strach miecza, aby sie rozplynelo serce, i upadków sie namnozylo. Ach! wypolerowany jest, aby sie blyszczal, a wyostrzony, aby zabijal.
\par 16 Zbierz sie mieczu, udaj sie na prawo i na lewo, gdziekolwiek jest chec twarzy twojej.
\par 17 Boc i Ja uderze reka moja w reke moje, i uspokoje rozgniewanie moje. Ja Pan mówilem to.
\par 18 Wtem sie stalo slowo Panskie do mnie, mówiac:
\par 19 A ty, synu czlowieczy! polóz przed soba dwie drogi, któredyby isc mial miecz króla Babilonskiego; z jednej ziemi niech wychodza obiedwie, a na rozdrozu obierz te ku miastu, te obierz.
\par 20 Pokaz droge, któraby miecz isc mial, czyli ku Rabbacie synów Ammonowych, czyli ku ziemi Judzkiej na Jeruzalemskie twierdze.
\par 21 Albowiem stanie król Babilonski na rozdrozu, na poczatku dwóch dróg, pytajac sie wieszczby; bedzie polerowal strzaly, bedzie sie radzil balwanów, bedzie patrzyl na watrobe.
\par 22 Po prawej rece jego wieszczba ukaze Jeruzalem, aby szykowal hetmanów, którzyby pobudzali do mordowania, a podnosili glos z okrzykiem, aby zasadzili tarany przeciwko bramom, aby usypali wal, a urobili szance.
\par 23 I beda to mieli za prózna wieszczbe przed oczyma swemi ci, co sie obowiazali przysiegami; a to przywiedzie na pamiec nieprawosc ich, aby pojmani byli.
\par 24 Przetoz tak mówi panujacy Pan: Dlatego, ze na pamiec przywodzicie nieprawosc swoje, a odkrywa sie przestepstwo wasze, tak, ze jawne sa grzechy wasze we wszystkich sprawach waszych, przto, mówie, ze na pamiec przychodzicie, ta reka pojmani bedziecie.
\par 25 A ty, nieczysty bezbozniku, ksieciu Izraelski! którego dzien przychodzi, gdy nieprawosc skonczona bedzie:
\par 26 Tak mówi panujacy Pan: Zdejm te czapke, a zrzuc te korone, która juz nigdy takowa nie bedzie; tego, który w ponizenie przyszedl, wywyzsze, a wywyzszonego ponize.
\par 27 W niwecz, w niwecz, w niwecz ja obróce, czego pierwej nie bywalo, az przyjdzie ten, co do niej ma prawo, którem mu dal.
\par 28 Ale ty, synu czlowieczy! prorokuj, a mów: Tak mówi panujacy Pan o synach Ammonowych i o hanbie ich; rzecz mówie: Miecz, miecz dobyty jest, ku zabijaniu wypoleroawny jest, aby wytracil wszystko, i aby sie blyskal.
\par 29 I chociaz ci opowiadaja marnosc, i wróza klamstwo, aby cie przylozyli do szyi niezbozników pobitych, których dzien przychodzi, gdy nieprawosc, skonczona bedzie.
\par 30 Schowaj jednak miecz do pochew jego; na miejscu, na któremes splodzona, w ziemi mieszkania twego, sadzic cie bede.
\par 31 I wyleje na cie rozgniewanie moje; ogniem popedliwosci mojej na cie dmuchac bede, i podam cie w rece ludzi zuchwalych i przemyslnych na wytracenie.
\par 32 Ogniowi potrawa bedziesz, krew twoja bedzie w posrodku ziemi, nie bedziesz wiecej wspominana; bo Ja Pan mówilem to.

\chapter{22}

\par 1 I stalo sie slowo Panskie do mnie, mówiac:
\par 2 A ty, synu czlowieczy! izalibys sie zastawial, izalibys sie zastawial za to miasto krwi? Raczej mu oznajmij wszystkie obrzydliwosci jego.
\par 3 I rzecz: Tak mówi panujacy Pan: O miasto, które wylejesz krew w posrodku siebie; a czynisz plugwe balwany samo przeciwko sobie, abys sie splugawilo, przychodzic czas twój.
\par 4 Tys wylaniem krwi twojej zgrzeszylo, i plugawemi balwanami swemi, któryches naczynilo, siebies skalalo, i sprawilos, ze sie przyblizyly dni twoje, a zes przyszlo az do lat swoich; przetoz cie podam na pohanbienie narodom, i na posmiech wszystkim ziemiom.
\par 5 Biliskie i dalekie od ciebie beda sie nasmiewac z ciebie, o miasto zlej slawy i zwad pelne!
\par 6 Oto ksiazeta Izraelscy, kazdy wszystka sila na to sie udali, aby krew w tobie przelewali.
\par 7 Ojca i matke zniewazaja w tobie, przychodniowi krzywde czynia w posród ciebie, sierote i wdowe uciskaja w tobie.
\par 8 Swietemi rzeczami mojemi pogardzasz, a sabaty moje splugawiasz.
\par 9 Potwarcy sa w tobie, aby wylewali krew; i na górach jadaja w tobie, zlosci popelniaja w posrodku ciebie.
\par 10 Nagosc ojcowska syn odkrywa w tobie, a oddalone dla nieczystosci gwalca w tobie.
\par 11 A drugi z zona blizniego swego czyni obrzydliwosc, a inny z synowa swoja sprosnie sie maze; inny zas siostre swoje, córke ojca swego, gwalci w tobie.
\par 12 Podarki biora w tobie na wylewanie krwi; lichwe i plat bierzesz, a zysku szukasz z uciskiem blizniego swego, a na mie zapominasz, mówi panujacy Pan.
\par 13 Przetoz otom Ja klasnal rekami swemi nad zyskiem twoim, którego nabywasz, i nad krwia twoja, która byla w posrodku ciebie.
\par 14 Izali wytrzyma serce twoje? izali zdolaja rece twoje dniom, których Ja z toba bede mial sprawe? Ja Pan rzeklem, i uczynie.
\par 15 Bo cie rozprosze miedzy pogany, i rozwieje cie po ziemiach, i uprzatne do konca nieczystosc twoje z ciebie.
\par 16 I bedziesz splugawione przed oczyma pogan, az poznasz, zem Ja Pan.
\par 17 Potem sie stalo slowo Panskie do mnie, mówiac:
\par 18 Synu czlowieczy, dom Izraaelski mi sie obrócil w zuzelice; wszyscy zgola sa miedzia i cyna i zelazam i olowiem w posród pieca, zuzelica srebra sie stali.
\par 19 Przetoz tak mówi panujacy Pan: Dlatego, zescie wy sie wszyscy obrócili w zuzelice, przetoz oto Ja zgromadze was do Jeruzalemu.
\par 20 Jako zgromadzaja srebro, i miedz, i zelazo, i olów i cyne w posród pieca, aby rozdymano ogien kolo nich dla roztopienia, tak was zgromadze w zapalczywosci mojej i w gniewie moim, a zlozywszy roztapiac was bede.
\par 21 Owa zgromadze was a rozedme okolo was ogien popedliwosci mojej, i roztopieni bedziecie w posrodku niego.
\par 22 Jako sie srebro topi w posrodku pieca, tak sie roztopicie w posrodku niego, a dowiesie sie, ze Ja Pan wylalem na was popedliwosc swoje.
\par 23 Nadto stalo sie slowo Panskie do mnie mówiac:
\par 24 Synu czlowieczy! mów do tej ziemi: Ty ziemio nieczystas jest, nie bedziesz deszczem odwilzana w dzien zapalczywosci.
\par 25 Sprzysiezenie proroków jej jest w posrodku jej; podobni sa lwowi ryczacemu, oblów chwytajacemu; dusze pozeraja, bogactwa i drogie rzeczy zbieraja, a czynia wiele wdów w posrodku niej.
\par 26 Kaplani jej gwalca zakon mój, i swiete rzeczy moje splugawiaja; miedzy swietym i pospolitym róznosci nie czynia, a miedzy nieczystym a czystym nie rozsadzaja. Nadto od sabatów moich zakrywaja oczy swe, tak, iz zelzony bywam miedzy nimi.
\par 27 Ksiazeta jej w posrodku jej sa jako wilki chwytajace lup, wylewajacy krew, tracacy dusze, udawajacy sie za zyskiem.
\par 28 A prorocy ich tynkuja wapnem nieczynionem, prorokujac marnosc, a wrózac im klamstwo mówiac: Tak mówi panujacy Pan, choc Pan nie mówi panujacy Pan: Biada temu miastu krwawemu, garncowi, w którym zostaje przywara jego, z którego, mówie, przywara jego nie wychodzi; po sztukach, po sztukach wyciagaj z niego, a padniec nan los.
\par 29 Lud tej ziemi gwalt czyni, gwaltem biorac, co jest cudzego; ubogiemu i nedznemu krzywde czynia, a przychodnia bezprawnie uciskaja.
\par 30 Szukalem miedzy nimi meza, któryby plot ugrodzil, i stanal w przerwie przed twarza moja za ta ziemia, abym jej nie zburzyl; alem zadnego nie znalazl.
\par 31 Przetoz wyleje na nich gniew mój, ogniem popedliwosci mojej wyniszcze ich; droge ich na glowe ich obróce, mówi panujacy Pan.

\chapter{23}

\par 1 I stalo sie slowo Panskie do mnie, mówiac:
\par 2 Synu czlowieczy! byly dwie niewiasty, córki jednej matki;
\par 3 Te nierzad plodzily w Egipcie, w mlodosci swojej nierzad plodzily; tam sa omacane piersi ich, i tam sa zgniecione piersi panienstwa ich.
\par 4 Imiona ich te sa: Wiekszej Ahola, a siostry jej Aholiba. Tec byly moje, i zrodzily synów i córki; imiona, mówie, ich sa, Samaryja Ahola, a Jeruzalem Aholiba.
\par 5 Ale Ahola plodzila wszeteczenstwo przy mnie, a nierzadu pilnowala z milosnikami swojemi, z Assyryjkczykami bliskimi,
\par 6 Którzy byli obleczeni w hijacynt, z ksiazetami, i panami, i ze wszystkimi mlodziencami udatnymi, i z jezdnymi jezdzacymi na koniach;
\par 7 Udala sie, mówie, na wszeteczenstwo swoje z nimi, ze wszystkimi najprzedniejszymi synami Assyryjskimi, i ze wszystkimi, za którymi miloscia palala, a splugawiala sie wszystkiemi plugawemi balwanami ich.
\par 8 A tak wszeteczenstw swoich egipskich nie zaniechala: bo z nia sypiali w mlodosci jej, a oni omacali piersi panienstwa jej, i wylali wszeteczenstwo swe na nia.
\par 9 Dlatego podalem ja w reke zalotników jej, w reke synów Assyryjskich, za którymi miloscia palala.
\par 10 Oni odkryli nagosc jej, synów jej i córki jej zabrali, a same mieczem zabili; i stala sie oslawiona miedzy niewiastami, gdy sady wykonali przy niej.
\par 11 To widzac siostra jej Aholiba, bardziej sie niz ona zapalila miloscia, a wszeteczenstwo jej wieksze bylo nizeli siostry jej.
\par 12 Za synami Assyryjskimi palala miloscia, za ksiazetami, i panami bliskimi, ubranymi w szaty kosztowne, za jezdnymi jezdzacymi na koniach, i za wszystkimi mlodziencami urodziwymi.
\par 13 I widzialem, ze sie splugawila, a iz jednaka droga obu byla.
\par 14 Ale ta jeszcze to przydala do wszeteczenstw swoich, ze widzac mezów wymalowanych na scianie, obrazy Chaldejczyków malowane farbami,
\par 15 Opasane pasami na biodrach ich, i kolpaki farbowane na gowach ich, a ze wszyscy na wejrzeniu byli jako hetmani, podobni synom Babilonskim, w ziemi Chaldejskiej zrodzonym;
\par 16 I palala miloscia ku nim, ujrzawszy ich oczyma swemi, a wyprawila poslów do nich do ziemi Chaldejskiej.
\par 17 Tedy weszli do niej synowie Babilonscy do komory nierzadu, i zmazali ja wszeteczenstwem swojem; a gdy sie splugawila z nimi, odstapila dusza jej od nich.
\par 18 I gdy odkryla wszeteczenstwa swe, odkryla i nagosc swoje; odstapila dusza moja od niej, jako byla odstapila dusza moja od siostry jej.
\par 19 Bo rozmnozyla wszeteczenstwa swoje, wspominajac na dni mlodosci swojej, których nierzadu patrzyla w ziemi Egipskiej.
\par 20 I palala miloscia przeciwko nierzadnikom ich, których ciala sa jako ciala oslów, a przyrodzenie ich jako przyrodzenie konskie.
\par 21 A takes sie zas nawrócila do sprosnosci mlodosci twojej, gdy macali Egipczanie piersi twoje dla piersi mlodosci twojej.
\par 22 Przetoz, o Aholibo! tak mówi panujacy Pan: Oto ja pobudze zalotników twoich przeciwko tobie, tych, od których odstapila dusza twoja, i przywiode ich na cie zewszad;
\par 23 Synów Babilonskich, i wszystkich Chaldejczyków z Pekot, i z Soby, i z Kohy, i wszystkich synów Assyryjskich z nimi, mlodzienców udatnych, ksiazat i panów wszystkich, hetmanów i ludzi zacnych, wszystkich jezdzacych na koniach;
\par 24 I przyjada przeciwko tobie na wozach zelaznych, i na rydwanach, i na karach, a to z zgraja narodów, z tarczami, i z pancerzami, i z przylbicami, poloza  sie przeciwko tobie zewszad; i dam im prawo, aby cie sadzili wedlug praw swoich.
\par 25 I wyleje gorliwosc moje na cie, i obejda sie z toba zapalczywie, nos twój i uszy twoje oberzna, a ostatek twój od miecza polegnie; oni synów twoich i córki twoje pojmaja, a ostatek twój ogniem pozarty bedzie;
\par 26 I zewloka cie z szat twoich, a pobiora strój ozdoby twojej.
\par 27 A tak uczynie koniec sprosnosci twojej przy tobie, i wszeteczenstwu twemu, któres przyniosla z ziemi Egipskiej, a nie podniesiesz oczów twych do nich, ani na Egipt nie wspomnisz wiecej.
\par 28 Bo tak mówi panujacy Pan: Oto cie Ja podaje w rece tych, których masz w nienawisci, w rece tych, od których odstapila dusza twoja;
\par 29 I obejda sie z toba wedlug nienawisci, i zabiora wszystke prace twoje, a zostawia cie naga i obnazona, i bedzie jawna nagosc wszeteczenstw twoich, i sprosnosci twojej, i nierzadów twoich.
\par 30 Toc sie stanie przeto, zes nierzad plodzila nasladujac pogan, przeto, zes sie zmazala plugawemi balwanami ich;
\par 31 Chodzilas droga siostry swej, dlatego dam kubek jej w reke twoje.
\par 32 Tak mówi panujacy Pan: Kubek siostry twojej gleboki i szeroki pic bedziesz, spory bedzie; posmiech takze i igrzysko beda mieli z ciebie.
\par 33 Pijanstwem i bolescia napelniona bedziesz, kubkiem spustoszenia i smutku, kubkiem siostry swej Samaryi!
\par 34 I wypijesz go i wysaczysz, potem go pokruszysz, a piersi swe poobrywasz; bom Ja rzekl, mówi panujacy Pan.
\par 35 Dlatego tak mówi panujacy Pan: Poniewazes zapomniala na mie, a zarzucilas mie za tyl swój, i ty tez ponos niecnote, swoje, i wszeteczenstwa swoje.
\par 36 I rzekl Pan do mnie: Synu czlowieczy! izali sie bedziesz zastawial za Ahole albo za Aholibe? Oznajmij im raczej obrzydliwosci ich:
\par 37 Ze cudzolozyly, a krew jest na rekach ich, i z plugawemi balwanami swemi cudzolozyly; nadto i synów swych, których mi narodzily, przez ogien przeporowadzaly im na pozarcie.
\par 38 Wiec i to mi uczynily, ze swiatnice moje splugawily dnia onego, a sabaty moje pogwalcily.
\par 39 Bo gdy ofiarowaly synów swych plugawym balwanom swoich, wchodzily do swiatnicy mojej onegoz dnia, aby ja zmazaly; oto takci czynily w posrodku domu mego.
\par 40 Nadto posylaly tez do mezów, aby przyszli z daleka; którzy zaraz przychodzili, kiedy posel do nich wyslany byl. Tymes kwoli sie ty umywala, farbowalas twarz swoje, i zdobilas sie ochedóstwem swojem.
\par 41 Siadalas na lozu zacnem, przed którym byl stól przygotowany, na któremes i kadzeniw moje i olejek mój pokladala.
\par 42 A gdy glos onego mnóstwa ucichl, tedy i do mezów ludu pospolitego posylaly, których przywodzono ozartych z puszczy, i kladli manele na rece ich, i korony ozdobne ne rece ich.
\par 43 A chociazem przymawial onej cudzoloznicy zastarzalej, a iz oni raz z jedna, raz z druga nierzad plodza,
\par 44 I ze kazdy z nich wchodzi do niej, jako wchodza do niewiasty wszetecznej: przecie jednak wchodzili do Aholi i do Aholiby, niewiast niecnotliwych.
\par 45 Przetoz sprawiedliwi mezowie, ci je osadza sadem cudzoloznic, i sadem wylewajacych krew, przeto, ze cudzolozyly, a krew jest na rekach ich.
\par 46 Bo tak mówi panujacy Pan: Przywiode na nie wojsko, a poddam je na potlukanie i na lup;
\par 47 I ukamionuje je ono zgromadzenie kamieniem, i rozsieka je mieczami swemi, synów ich i córki ich pobija, a domy ich ogniem spala.
\par 48 A tak uprzatne sprosnosc z tej ziemi, i beda sie tem karac wszystkie niewiasty, a nie uczynia wedlug niecnoty waszej.
\par 49 Bo wlozona bedzie na was niecnota wasza, a grzechy plugawych balwanów waszych poniesiecie, i poznacie, zem Ja panujacy Pan.

\chapter{24}

\par 1 I stalo sie slowo Panskie do mnie roku dziewiatego, miesiac dziesiatego, dziesiatego dnia tegoz miesiaca, mówiac:
\par 2 Synu czlowieczy! napisz sobie imie dnia tego, tegoz wlasnie dnia; bo oblegl król Babilonski Jeruzalem prawie tegoz dnia.
\par 3 A rzecz przez przypowiesc do tego domu odpornego, podobienstwo mówiac do nich: Tak mówi panujacy Pan: Przystaw ten garniec, przystaw, a nalej wen wody;
\par 4 A zebrawszy sztuki nalezace do niego, kazda sztuke dobra, udziec i lopatke, najlepszemi kosciami napeln go.
\par 5 Wezmijze i co najwyborniejsze bydle, a nalóz ogien z kosci pod niem; sprawze aby to wrzalo i kipialo, zeby i kosci jego rozewrzaly w niem.
\par 6 Przetoz tak mówi panujacy Pan: Biada temu miastu krwawemu, garncowi, w którym zostaje przywara jego, z którego, mówie, przywara jego nie wychodzi: po sztukach, po sztukach wyciagaj z niego, nie padniec nan los.
\par 7 Bo iz krew jego jest w posrodku jego, na wierzchu skaly wystawilo ja, nie wylalo jej na ziemie, aby byla zakryta prochem;
\par 8 Tedy i Ja rozniece zapalczywosc na wykonanie pomsty, wystawie krew jego na wierzchu skaly, aby nie byla zakryta.
\par 9 Przetoz tak mówi panujacy Pan: Biada temu miastu krwawemu; bo i Ja naniece wielki ogien,
\par 10 Przykladajac drew, rozniecajac ogien, wniwecz obracajac mieso, i zaprawiajac korzeniem, tak, ze i kosci spalone beda;
\par 11 A postawie ten garniec na wegle jego prózny, aby sie zagrzala i rozpalila miedz jego, a zeby sie rozplynely w posród jego plugastwa jego, a izby zniesiona byla przywara jego.
\par 12 Utrudzilo mie klamstwami swemi, przetoz nie wynijdzie z niego wielkosc szumowin jego, do ognia musza szumowiny jego.
\par 13 W nieczystosci twojej jest sprosnosc; dlatego, zem cie oczyszczal; a przecies nie oczyszczona, i od nieczystosci twojej nie bedziesz wiecej oczyszczana, az uspokoje na tobie rozgniewanie moje.
\par 14 Ja Pan mówilem; przyjdzie to, i uczynie to, nie cofne sie, ani sfolguje, ani mi zal bedzie; wedlug dróg twoich, i wedlug spraw twoich sadzic cie bede, mówi panujacy Pan.
\par 15 I stalo sie slowo Panskie do mnie, mówiac:
\par 16 Synu czlowieczy! oto Ja od ciebie odejme zadnosc oczów twoich z predka; wszakze nie kwil, ani placz, a niech nie wychodza lzy twoje.
\par 17 Zaniechaj kwilenia, nie czyn zaloby, jako bywa nad umarlym; czapke twoje wlóz na sie, a obuwie twoje wzuj na nogi twoje, a nie zaslaniaj warg, a chleba niczyjego nie jedz.
\par 18 Co gdym z poranku ludowi powiedzial, tedy umarla zona moja w wieczór; i uczynilem rano, jako mi rozkazano.
\par 19 I mówil lud do mnie: I nie oznajmiszze nam, co nam te rzeczy znacza, które ty czynisz?
\par 20 Tedym rzekl do nich: Slowo Panskie stalo sie do mnie, mówiac:
\par 21 Powiedz domowi Izraelskiemu: Tak mówi panujacy Pan: Oto Ja splugawie swiatnice moje, wynioslosc mocy waszej, zadnosc oczów waszych, i to, w czem sie kocha dusza wasza, a synowie wasi i córki wasze, którescie zostawili, od miecza polegna.
\par 22 I uczynicie, jakom uczynil; nie zaslonicie wargi, a chleba niczyjego jesc nie bedziecie;
\par 23 A czapki swe na glowach swych, i bóty swoje na nogach swoich majac, nie bedziecie kwilic ani plakac; ale bedziecie schnac dla nieprawosci waszych, a wzdychac jeden z drugim.
\par 24 Bo wam jest Ezechyjel dziwem, wedlug wszystkiego, co on czyni, czynic bedziecie; a gdy to przyjdzie, dowiecie sie, zem ja panujacy Pan.
\par 25 A ty, synu czlowieczy! azaz w ten dzien, którego Ja odejme od nich moc ich, wesele ozdoby ich, zadnosc oczów ich, i to, po czem teskni dusza ich, synów ich i córki ich,
\par 26 Izali dnia onego nie przyjdzie do ciebie ten, co uciecze, oznajmujac ci to?
\par 27 Dnia onego otworza sie usta twoje przy tym, który ujdzie; a bedziesz mówil, i nie bedziesz wiecej niemym; i bedziesz im dziwem, a poznaja, zem Ja Pan.

\chapter{25}

\par 1 I stalo sie slowo Panskie do mnie, mówiac:
\par 2 Synu czlowieczy! obróc twarz twoje przeciwko synom Ammonowym, a prorokuj przeciwko nim.
\par 3 I rzecz synom Ammonowym: Sluchajcie slowa panujacego Pana. Tak mówi panujacy Pan: Przeto, zes wykrzykal mówiac: Hej, hej! nad swiatnica moja, gdy byla splugawiona, i nad ziemia Izraelska, i gdy byla spustoszona, i nad domem Judzkim, i gdy szedl w niewole;
\par 4 Przetoz oto Ja cie tez podam narodom wschodnim w dziedzictwo i pobuduja palace swoje w tobie, a wystawia mieszkanie swoje w tobie, one beda jesc urodzaje twoje, oni tez beda pic mleko twoje.
\par 5 I dam Rabbe na mieszkanie wielbladom, a miasta synów Ammonowych na legowisko trzodom; i dowiecie sie, zem Ja Pan.
\par 6 Bo tak mówi panujacy Pan: Przeto, izes klaskal reka, a tapal noga, i weseliles sie z serca, zes cale spustoszyl ziemie Izraelska:
\par 7 Przetoz oto Ja wyciagne reke swa przeciwko tobie, a dam cie w rozchwycenie narodom, i wytne cie z narodów, a wytrace cie z ziem, i wygladze cie, a dowiesz sie, zem Ja Pan.
\par 8 Tak mówi panujacy Pan: Dlatego, ze Moab i Seir mówil: Oto dom Judzki podobny jest wszystkim innym narodom.
\par 9 Dlatego oto Ja otworze strone Moabczyków od miast, od miast mówie ich, i od granic ich, ozdobe ziemi Betiesymot, Baalmeon, i Karyjataim,
\par 10 Narodom wschodnim z ziemia synów Ammonowych; bom ja dal w dziedzictwo, aby nie bylo pamiatki synów Ammonowych miedzy narodami.
\par 11 A tak nad Moabem sady wykonam, iz poznaja, zem Ja Pan.
\par 12 Tak mówi panujacy Pan: Przetoz, iz sie Edomczycy srodze mscili nad domem Judzkim, i przywiedli na sie wine wielka, mszczac sie nad nimi;
\par 13 Przetoz tak mówi panujacy Pan: Wyciagne tez reke moje na ziemie Edomczyków, a wytrce z niej ludzi i bydlo, i uczynie ja pustynia; od Teman az do Dedan od miecza polegna.
\par 14 A tak wykonam pomste moje nad Edomczykami przez rece ludu mojego Izraelskiego, a obejda sie z Edomczykami wedlug popedliwosci mojej, i wedlug gniewu mego; i poznaja pomste moje, mówi panujacy Pan.
\par 15 Tak mówi panujacy Pan: Przeto, iz sie Filistynczycy mscili, i pomste wykonywali, pustoszac ich z serca, a do zginienia przywodzac z nieprzyjazni starodawnej;
\par 16 Dlatego tak mówi panujacy Pan: Oto Ja wyciagne reke moje na Filistynczyków, i wykorzenie Cheretejczyków, i wytrace ostatek krainy pomorskiej.
\par 17 A tak uczynie nad nimi pomsty wielkie, karzac ich w zapalczywosci; i dowiedza sie, zem Ja Pan, gdy wykonam pomste moje nad nimi.

\chapter{26}

\par 1 I stalo sie roku jedenastego, pierwszego dnia miesiaca, stalo sie slowo Panskie do mnie, mówiac:
\par 2 Synu czlowieczy! przeto, iz Tyr mówil o Jeruzalemie wykrzykajac: Hej, hej! zniszczone jest miasto bram bardzo ludnych, obraca sie do mnie, teraz napelniony bede, gdyz to jest spustoszone;
\par 3 Dlatego tak mówi panujacy Pan: Otom Ja powstal przeciwko tobie, o Tyrze! a przywiode na cie wiele narodów, jakobym przywiódl morze z nawalnosciami jego;
\par 4 I zburza mury Tyrskie, i rozwala wieze jego; i wymiote z niego proch jego, i uczynie go wierzcholkiem skaly gladkiej,
\par 5 Tak, ze beda wysuszac sieci w posród morza; bom Ja rzekl, mówi panujacy Pan, przeto bedzie na rozchwycenie narodom.
\par 6 A córki jego, które beda na polu, mieczem pobite beda, a dowiedza sie, zem Ja Pan.
\par 7 Bo tak mówi panujacy Pan: Oto Ja przywiode przeciwko Tyrowi Nabuchodonozora, króla Babilonskiego, od pólnocy, króla nad królmi, z konmi i z wozami, i z jezdnymi i z zgraja, i z ludem wielkim.
\par 8 Córki twoje na polu mieczem pomorduje, i przeciwko tobie porobi baszty, i usypie wal przeciwko tobie, i postawi przeciwko tobie tarcz;
\par 9 I tarany zasadzi przeciwko murom twoim, a wieze twoje potlucze mlotami swemi.
\par 10 Od mnóstwa koni jego okryje cie proch ich; od grzmotu jezdnych i kar i wozów porusza sie mury twoje, gdy wchodzic bedzie w bramy twoje, jako wchodza do miasta zburzonego.
\par 11 Kopytami koni swoich zdepcze wszystkie ulice twoje, lud twój mieczem pobije, a mocne slupy twoje upadna na ziemie;
\par 12 I robiora majetnosc twoje, a rozchwyca towary twoje, i rozwala mury twoje, i domy twoje rozkoszne poburza, a kamienie twoje, i drzewo twoje, i proch twój do wody wrzuca.
\par 13 I uczynie. ze ustanie glos piesni twoich, a dzwiek harf twoich nie bedzie wiecej slyszany.
\par 14 I uczynie cie wierzcholkiem gladkiej skaly; staniesz sie miejscem ku wysuszaniu sieci, nie bedziesz wiecj zbudowany; bom Ja Pan powiedzial, mówi panujacy Pan.
\par 15 Tak mówi panujacy Pan do Tyru: Izali sie od trzasku upadku twego, gdy ranni wolac beda, gdy okrutne morderstwo bedzie w posrodku ciebie, wyspy sie nie porusza?
\par 16 I powstana z stolic swoich wszyscy ksiazeta pomorscy, i zloza z siebie plaszcze swoje, a szaty swe haftowane zewleka, strachem beda przyodziani, na ziemi usiada, a wzdrygajac sie co chwila zdumiewac sie beda nad toba,
\par 17 I podniosa nad toba lament i rzekna do ciebie: Jakos zginelo, o miasto! w którem mieszkano dla przyleglosci morza, miasto slawne, które bylo mocne na morzu, ono i z obywatelami swymi, którzy byli straszni wszystkim obywatelom jego.
\par 18 Tedy sie zatrwoza wyspy w dzien upadku twego; zatrwoza sie mówie wyspy morskie nad zginieniem twojem.
\par 19 Bo tak mówi panujacy Pan: Gdy cie uczynie miastem spustoszonem, jako miasta, w których nie mieszkaja, gdy na cie przepasc przywiode, tak, ze cie wody wielkie przykryja;
\par 20 Gdy uczynie, ze zstapisz z tymi, którzy zstepuja do dolu, do ludu dawnego, a poloze cie w najnizszych stronach ziemi, na pustyniach dawnych, z tymi, co zstepuja do dolu, aby nie mieszkano w tobie, tedy dokaze slawy w ziemi zyjacych.
\par 21 Bo uczynie to, ze bedziesz na wielki postrach, gdy cie nie stanie; a chocby cie szukano, nie znajda cie na wieki, mówi panujacy Pan.

\chapter{27}

\par 1 I stalo sie slowo Panskie do mnie, mówiac:
\par 2 A ty, synu czlowieczy! podnies nad Tyrem lament,
\par 3 A rzecz do Tyru, który lezy nad portami morskiemi i handluje z narodami na wielu wyspach: Tak mówi panujacy Pan: O Tyrze! tys mówil: Jam jest doskonaly w pieknosci.
\par 4 W posrodku morza byly granice twoje, budownicy twoi doskonala uczynili pieknosc twoje.
\par 5 Z jedliny z Sanir pobudowalic wszystkie pietra twoje, cedry z Libanu brali, aby czynili maszty twoje.
\par 6 Z debów Basanskich czynili wiosla twoje, lawy twoje urobili z kosci sloniowych i z bukszpanu z wysep Cytymskich.
\par 7 Bisior haftowany egipski bywal plótnem twojem, z któregos zagle miewal; hijacynt i szarlat z wysep Elisa byl nakryciem twojem.
\par 8 Obywatele Sydonu, i Arwadczycy bywali zeglarzami twymi; medrcy twoi, Tyrze! którzy bywali w tobie, ci byli sternikami twymi.
\par 9 Starcy z Giebal, i medrcy jego oprawiali w tobie rozpadliny twoje; wszystkie okrety morskie i zeglarze ich bywali w tobie, handlujac z toba.
\par 10 Persowie, i Ludczycy, i Putejczycy bywali w wojsku twojem, mezowie waleczni twoi; tarcz i przylbice zawieszali w tobie, ci przydawali tobie ozdoby.
\par 11 Synowie Arwad z wojskiem twojem na murach twoich w okolo, takze Gamadczycy na wiezach twoich bywali, tarcze twoje zawieszali na murach twoich w okolo; cic sa, którzy doskonala uczynili pieknosc twoje.
\par 12 Zamorscy kupcy twoi dla wielkosci wszelakich dostatków, srebrem, zelazem, cyna i olowiem kupczyli na jarmarkach twoich.
\par 13 Jawan, Tubal, i Mesech, kupcy twoi, ludzi i naczynie miedziane dawali na zamiane tobie.
\par 14 Z domu Togorma konmi, i jezdnymi, i mulami kupczyli na jarmarkach twoich.
\par 15 Synowie Dedanowi, kupcy twoi, i wiele wysep przekupowali towary reki twojej, rogi, kosci sloniowe, i drzewo hebanowe dawali na zamiane za zaplate twoje.
\par 16 Syryjczycy kupcy twoi dla mnóstwa przemyslnych robót twoich, karbunkulami, szarlatem, i haftarskiemi rzeczami, plótnem subtelnem, i koralami, i krysztalami handlowali na jarmarkach twoich.
\par 17 Juda i ziemia Izraelska, i ci kupcy twoi, pszenice z Minnit i z Pannag, i miód, i oliwe. i kadzidlo na zamianec dawali.
\par 18 Damaszczanie, kupcy twoi dla mnóstwa przemyslnych robót twoich, i dla mnóstwa wszelkich dostatków, winem z Helbonu i welne biala kupczyli.
\par 19 Takze Dan i Jawan, kramarze na jarmarkach twoich, sprzedawali zelazo polerowane, kassyje, i Tatarskie ziela na zamianec dawali.
\par 20 Dedan kupczyl w tobie suknami kosztownemi na wozy.
\par 21 Arabczycy, i wszyscy ksiazeta Kedarscy, i ci kupczyli z toba skopami i baranami, i kozlami, tem handlowali w tobie.
\par 22 Kupcy Sabejscy i z Ramy kupczyli z toba wszelakiemi przedniejszemi wonnemi rzeczami, i wszelakim kamieniem drogim i zlotem kupczyli na jarmarkach twoich;
\par 23 Haran, i Kanne, i Eden kupcy z Saby; Assur i Kilmad kupczyl w tobie.
\par 24 Ci kupcy twoi, sztukami hijacyntu, i rzeczami haftowanemi, i skrzyniami dla kosztownych szat, takze towarami, które powrozami obwiazuja albo w skrzyniach cedrowych zawieraja, kupczyli w tobie.
\par 25 Okrety morskie przodkowaly w kupiectwie twojem; i bylos napelnione i uwielbione bardzo w posród morza.
\par 26 Na wody wielkie zaprowadzili cie zeglarze twoi; wiatr wschodni rozbije cie w posród morza.
\par 27 Bogactwa twoje, i jarmarki twoje, kupiectwo twoje, zeglarze twoi, i sternicy twoi, i ci, którzy zaprawiali rozpadliny twoje, i kupcy towarów twoich, i wszyscy mezowie waleczni twoi, którzy byli w tobie, i wszystko mnóstwo twoje, które jest w posród ciebie, wpadna w posród morza w dzien upadku twego.
\par 28 Na glos krzyku sterników twoich zadrza waly morskie;
\par 29 I wystapia z okretów swoich wszyscy robiacy wioslem, zeglarze, i wszyscy sternicy morscy na ziemi stana.
\par 30 I beda narzekali nad toba glosem wielkim, i beda gorzko wolali, a sypiac proch na glowy swoje, w popiele sie walac beda.
\par 31 Nadto poczynia sobie dla ciebie lysiny, a opasza sie worami; i beda plakac nad toba w gorzkosci duszy swej placzem gorzkim.
\par 32 Uczynia, mówie, nad toba lament zalosny, a beda narzekali nad toba, mówiac: Którez miasto podobne jest Tyrowi, wycietemu w posrodku morza?
\par 33 Gdy wychodzily towary twoje z morza, nasycalos wiele narodów; mnóstwem bogactw twoich i handlów twoich bogacilos królów ziemskich.
\par 34 Ale gdy bedziesz podruzgotane od morza w glebokosciach wód, kupiectwo twoje i wszystko mnóstwo twoje w posrodku ciebie upadnie.
\par 35 Wszyscy na wyspach mieszkajacy zdumieja sie nad toba, a królowie ich strachem zdjeci bedac, bardzo sie zatrwoza.
\par 36 Kupcy miedzy narodami zaswisna nad toba; na postrach im bedziesz, a nie bedzie cie na wieki.

\chapter{28}

\par 1 I stalo sie slowo Panskie do mnie mówiac:
\par 2 Synu czlowieczy! mów ksiazeciu Tyrskiemu: Tak mówi panujacy Pan: Dlatego, iz sie podnioslo serce twoje, a mówisz: Jam jest Bóg, siedze w posród morza na stolicy Boskiej, gdyzes ty czlowiek, a nie Bóg, choc serce swoje stawiasz jako serce Boze;
\par 3 Otos medrszym nad Danijela, zadna tajemnica nie jest zakryta przed toba,
\par 4 Madroscia twoja i roztropnoscia twoja nazbierales sobie bogactw, i nabyles zlota i srebra do skarbów toich;
\par 5 Wielkoscia madrosci twojej w kupiectwie twojem rozmnozyles bogactwa twoje; a tak podnioslo sie serce twoje dla bogactw twoich.
\par 6 Przetoz tak mówi panujacy Pan: Poniewaz stawiasz serce twoje jako serce Boze,
\par 7 Dlatego oto Ja przywiode na cie cudzoziemców najsrozszych z narodów, którzy dobywszy mieczów swoich na pieknosc madrosci twojej splugawia jasnosc twoje;
\par 8 W dól cie wepchna, i umrzesz sroga smiercia w posród morza.
\par 9 Izali mówiac rzeczesz przed tym, który cie zabijac bedzie: Jestem Bóg? gdyzes czlowiek a nie Bóg w reku mordercy twego.
\par 10 Smiercia nieobrzezanców umrzesz od reki cudzoziemców; bo Ja mówilem, mówi panujacy Pan.
\par 11 I stalo sie slowo Panskie do mnie mówiac:
\par 12 Synu czlowieczy! podnies lament nad królem Tyrskim, a mów do niego: Tak mówi panujacy Pan: Ty, co pieczetujesz sumy, pelen madrosci i doskonalej pieknosci;
\par 13 Byles w Eden, ogrodzie Bozym; wszelki kamien drogi byl nakryciem twojem, sardyjusz, topazyjusz, i jaspis, chrysolit, onyks, i beryl, szafir, kabunkul, i szmaragd, i zloto; w ten dzien, któregos ty stworzony, zgotowane sa u ciebie narzedy bebnów twoich i piszczalek twoich.
\par 14 Tys byl Cherubinem pomazanym, nakrywajacym; Jam cie wystawil, byles na górze Bozej swietej, w posród kamienia ognistego przechadzales sie,
\par 15 Byles doskonalym na drogach twoich ode dnia tego, któregos jest stworzony, az sie znalazla nieprawosc w tobie.
\par 16 Dla wielkosci kupiectwa twego pelno w posród ciebie bezprawia, i zgrzeszyles; dlatego wytrace cie z góry Bozej, o Cherubinie nakrywajacy! a z posrodku kamienia ognistego wygubie cie.
\par 17 Podnioslo cie serce twoje dla pieknosci twojej, na zles uzywal madrosci swojej dla jasnosci twojej; przetoz cie uderze o ziemie, a przed obliczem królów poloze cie, abyc sie dziwowali.
\par 18 Dla mnóstwa nieprawosci twoich, i dla sprawiedliwosci kupiectwa twego splugwiles swiatnice twoje; przetoz wywiode ogien z posrodku ciebie, który cie pozre, a obróce cie w popiól na ziemi przed oczyma wszystkich, co na cie patrza, .
\par 19 Wszyscy, co cie znali miedzy narodami, zdumiewaja sie nad toba; bedziesz na wielki postrach, a nie bedzie cie az na wieki.
\par 20 I stalo sie slowo Panskie do mnie mówiac:
\par 21 Synu czlowieczy! obróc twarz swoje przeciw Sydonowi, a prorokuj przeciw niemu,
\par 22 I mów: Tak mówi panujacy Pan: Otom Ja przeciwko tobie, o Sydonie! a bede uwielbiony w posród ciebie; i dowiedza sie, zem Ja Pan, gdy nad nimi sady wykonam, i w nim poswiecony bede.
\par 23 I posle nan mór, i krew na ulice jego, i upadna zranieni w posrodku niego od miecza, który na nich przyjdzie ze wszystkich stron; a dowiedza sie, zem Ja Pan.
\par 24 A tak nie bedzie mial wiecej dom Izraelski ciernia kolacego, i glogu bolesci zadawajacego z wszystkich okolicznych swych, którzy ich pustosza; i dowiedza sie, zem Ja panujacy Pan.
\par 25 Tak mówi panujacy Pan: Gdy zgromadze dom Izraelski z narodów, miedzy którymi sa rozproszeni, i poswiecony bede w nich przed oczyma pogan, i beda mieszkali w ziemi swojej, któram byl dal sludze memu Jakóbowi;
\par 26 Tedy w niej beda bezpiecznie mieszkali, a pobuduja domy, i naszczepia winnic; mieszkac mówie beda bezpiecznie, gdy wykonam sady nad wszystkimi okolicznymi ich, którzy ich pustoszyli; i dowiedza sie, zem Ja Pan, Bóg ich.

\chapter{29}

\par 1 Roku dziesatego, dziesatego miesiaca, dwunastego dnia tegoz miesiaca stalo sie slowo Panskie do mnie mówiac:
\par 2 Synu czlowieczy! obróc twarz swoje przeciwko Faraonowi, królowi Egipskiemu, a prorokuj przeciw niemu i przeciwko wszystkiemu Egiptowi;
\par 3 Mów, a rzecz: Tak mówi panujacy Pan: Otom Ja przeciwko tobie o Faraonie, królu Egipski, wielorybie wielki, który lezysz w posrodku rzek twoich, i mówisz: Mojac jest rzeka, i jam ja sobie uczynil;
\par 4 Przetoz wloze wede w czelusci twoje, i uczynie, ze powiezna ryby rzek twoich na luskach twych, i wywleke cie z posrodku rzek twoich i wszystkie ryby rzek twoich, które na luskach twoich powiezna;
\par 5 I zostawie cie na puszczy, ciebie i wszystkie ryby rzek twoich: polezesz na polu, i nie bedziesz zebrany ani zgromadzony; dam cie bestyjom ziemskim i ptastwu niebieskiemu ku pozarciu;
\par 6 I dowiedza sie wszyscy mieszkajacy w Egipcie, zem Ja Pan, przeto zescie laska trzciniana domowi Izraelskiemu.
\par 7 Gdy sie ciebie reka chwytaja, lamiesz sie i rozcinasz im wszystko ramie; a gdy sie podpieraja toba, kruszysz sie, choc im nadstawiasz wszystkich biódr.
\par 8 Dlategoz tak mówi panujacy Pan: Oto Ja przywiode na cie miecz, i wygladze z ciebie czlowieka i bydle;
\par 9 A ziemia Egipska bedzie pustynia i spustoszeniem, i dowiedza sie, zem Ja Pan, dlatego, zes mówil: Rzeka moja, i jam ja uczynil;
\par 10 Przetoz oto Ja bede przeciwko tobie i przeciwko rzece twojej, i podam ziemie Egipska w spustoszenie, i we srogie poburzenie, od wiezy Sewene az do granic Murzynskich.
\par 11 Nie przejdzie przez nia noga czlowiecza, i noga bydleca nie przejdzie przez nia, ani w niej beda mieszkac przez czterdziesci lat.
\par 12 A tak uczynie ziemie Egipska pustnia nad inne ziemie spustoszone, a miasta jej nad inne miasta spustoszone, beda spustoszone przez czterdziesci lat: gdyz rozprosze Egipczan miedzy narody, i rozwieje ich po ziemiach.
\par 13 A wszakze tak mówi panujacy Pan: Gdy sie skonczy czterdziesci lat, zgromadze Egipczan z narodów, do których rozproszeni beda.
\par 14 I przywróce zasie wiezniów Egipskich, i przywiode ich do ziemi Patros, do ziemi mieszkania ich, i beda tam królestwem podlem.
\par 15 Miedzy innemi królestwami bedzie najpodlejszen, a nie wyniesie sie wiecej nad inne narody, i umniejsze ich, aby nie panowali nad narodami.
\par 16 I nie bedzie wiecej domowi Izraelskiemu ufnoscia, któraby mi na pamiec przywodzila nieprawosc, gdyby sie ogladali na nie; i dowiedza sie, zem Ja panujacy Pan.
\par 17 Potem stalo sie dwudziestego i siódmego roku, pierwszego miesiaca, pierwszego dnia tegoz miesiaca, stalo sie slowo Panskie do mnie, mówiac:
\par 18 Synu czlowieczy! Nabuchodonozor, król Babilonski, przyniewolil gwaltem wojsko swe do sluzby ciezkiej przeciwko Tyrowi; kazda glowa oblysiala, i kazde ramie obnazone, a przecie nie ma zaplaty on, ani wojsko jego z Tyru za one sluzbe, która podejmo wal, walczac przeciwko jemu.
\par 19 Przetoz tak mówi panujacy Pan: Oto Ja daje Nabuchodonozorowi, królowi Babilonskiemu, ziemie Egipska, aby zabral dostatki jej, i rozszarpal lupy jej, i rozchwycil korzysci jej, aby mialo zaplate wojsko jego.
\par 20 Za prace ich, która dla mnie podjeli, dam im ziemie Egipska, przeto, ze mnie pracowali, mówi panujacy Pan.
\par 21 Dnia onego uczynie, ze wyrosnie róg domu Izraelskiego, tobie tez usta twoje otworze w posrodku ich; i dowiedza sie, zem Ja Pan.

\chapter{30}

\par 1 I stalo sie slowo Panskie do mnie mówiac:
\par 2 Synu czlowieczy! prorokuj a mów: Tak mówi panujacy Pan; kwilcie mówiac: Ach niestetyz na ten dzien!
\par 3 Bo bliski jest dzien, bliski jest mówie dzien Panski; ten bedzie dzien chmury, i czas narodów.
\par 4 I przyjdzie miecz na Egipt, a bedzie wielka trwoga w ziemi Murzynskiej, gdy polegna pobici w Egipcie, a zabiora dostatki jego, i podwrócone beda grunty jego.
\par 5 Murzyni i Putejczycy, i Ludczycy, i wszystko pospólstwo, i Kubejczycy, i obywatele innych ziem, w przymierzu bedacych, z nimi od miecza upadna.
\par 6 Tak mówi Pan, ze upadna, którzy podpieraja Egipt, i stracona bedzie pycha mocy jego; od wiezy Sewene od miecza upadna w niej, mówi panujacy Pan.
\par 7 I beda spustoszeni nad inne ziemie spustoszone, a miasta ich nad inne miasta poburzone beda;
\par 8 I dowiedza sie, zem Ja Pan, gdy zapale ogien w Egipcie, i beda potarci wszyscy pomocnicy jego.
\par 9 Dnia onego wynijda poslowie od oblicza mego w okretach na postrach ziemi Murzynskiej bezpiecznej; i bedzie wielka trwoga miedzy nimi, jaka byla w dzien porazki Egipskiej; bo oto pewnie przychodzi.
\par 10 Tak mówi panujacy Pan: Uczynie zaiste koniec mnóstwu Egipskiemu przez reke Nabuchodonozora, króla Babilonskiego.
\par 11 On i lud jego z nimi, najsrozsi z narodów, przywiedzeni beda na wytracenie tej ziemi; bo dobeda mieczów swych przeciw Egiptowi, i napelnia ziemie pobitymi.
\par 12 I wysusze rzeki, a zaprzedam ziemie w reke zlosników; a tak spustosze ziemie, i pelnosc jej przez reke cudzoziemców. Ja Pan mówilem.
\par 13 Tak mówi panujacy Pan: Wytrace tez plugawe balwany, i zniose obraz z Nof, a ksiazecia ziemi Egipskiej wiecej nie bedzie, gdyz puszcze strach na ziemie Egipska;
\par 14 Bo spustosze Patros, a rozniece ogien w Soan, i wykonam sad nad No;
\par 15 Wyleje tez popedliwosc moje na Syn, obronne miejsce Egipskie, a wytrace mnóstwo z No.
\par 16 Gdyz rozniece ogien w Egipcie, Syn bolejac bolec bedzie, i No rozwalone bedzie, a Nof na kazdy dzien udreczone bedzie.
\par 17 Mlodziency miasta On i Bubasto od miecza polegna, a panny w pojmanie pójda.
\par 18 Takze w Tachpanches zacmi sie dzien, gdy tam pokrusze zawory Egipskie, i ustanie w niem pycha mocy jego, chmura je okryje, a córki jego w pojmanie pójda.
\par 19 A tak wykonam sady nad Egiptem, i dowiedza sie, zem Ja Pan.
\par 20 I stalo sie jedenastego roku, pierwszego miesiaca, siódmego dnia stalo sie slowo Panskie do mnie, mówiac:
\par 21 Synu czlowieczy! zlamalem ramie Faraona, króla Egipskiego; a oto nie bedzie zawiazane, aby bylo uleczone, ani bedzie chustkami obwinione, ani bedzie zwiazane, aby bylo zmocnione do trzymania miecza.
\par 22 Przetoz tak mówi panujacy Pan: Otom Ja przeciwko Faraonowi, królowi Egipskiemu, a skrusze ramiona jego, tak mocne jako i zlamane, i wytrace miecz z reki jego;
\par 23 I rozprosze Egipczan miedzy narody, a rozwieje ich po ziemiach.
\par 24 Umocnie zasie ramiona króla Babilonskiego, i dam miecz mój w rece jego, a ramiona Faraonowe zlamie, i bedzie stekal przed obliczem jego, jako steka zraniony na smierc.
\par 25 Umocnie, mówie, ramiona króla Babilonskiego, a ramiona Faraonowe upadna; i dowiedza sie, zem Ja Pan, gdy dam miecz mój w reke króla Babilonskiego, aby go wyciagnal na ziemie Egipska.
\par 26 A tak rozprosze Egipczan miedzy narody, i rozwieje ich po ziemiach: i dowiedza sie, zem Ja Pan.

\chapter{31}

\par 1 Potem jedenastego roku, trzeciego miesiaca, pierwszego dnia tegoz miesiaca, stalo sie slowo Panskie do mnie, mówiac:
\par 2 Synu czlowieczy! mów do Faraona, króla Egipskiego, i do ludu jego: Komuzes podobnym w wielmoznosci twojej?
\par 3 Oto Assur byl jako cedr na Libanie, pieknych galezi i szeroko cien podawajacy i wysokiego wzrostu, a miedzy gestwina galezi byl wierzch jego.
\par 4 Wody mu wzrost daly, glebokosc go wywyzszyla, a rzekami jej otoczony byl w okolo korzen jego, a strumienie tylko swoje wypuszczala na wszystkie drzewa polne,
\par 5 Tak, ze sie wywyzszyl wzrost jego nad wszystkie drzewa polne, i rozkrzewily sie latorosle jego, a dla obfitosci wód rozszerzyly sie galezie jego, które wypuscil.
\par 6 Na galeziach jego czynilo gniazda wszelakie ptastwo niebieskie, a pod latoroslami jego mnozyly sie wszelkie zwierzeta polne, i pod cieniem jego siadaly wszystkie narody zacne.
\par 7 I byl piekny dla wielkosci swojej, i dla dlugosci galezi swoich; bo korzen jego byl przy wodach obfitych.
\par 8 Cedry go nie przewyzszaly w ogrodzie Bozym, jedliny nie byly równe latoroslom jego, a kasztanowe drzewa nie byly podobne galeziom jego; zadne drzewo w ogrodzie Bozym nie bylo mu równe w pieknosci swojej.
\par 9 Jam go pieknym uczynil dla mnóstwa galezi jego, i zajrzaly mu wszystkie drzewa w Eden, które byly w ogrodzie Bozym.
\par 10 Przetoz tak mówi panujacy Pan: Dlatego, ze wysoko wzrósl, a wywyzszyl wierzch swój miedzy gestwina galezi, i podnioslo sie serce jego dla wysokosci jego:
\par 11 Przetozen go podal w reke najmocniejszego z narodów, aby sie z nim srogo obchodzil; dla niezboznosci jego odrzucilem go.
\par 12 A tak wygladzili go cudzoziemcy najsrozsi z narodów, i porzucili go; na górach i na wszystkich dolinach odpadly galezie jego, i polamane sa latorosli jego przy wszystkich strumieniach tej ziemi; dlatego ustapily z cienia jego wszystkie narody zie mskie, i opuscily go.
\par 13 Na obaleniu jego mieszka wszelkie ptastwo niebieskie, a na galeziach jego jest wszelki zwierz polny,
\par 14 Dlatego, aby sie na potem nie wywyzszalo wzrostem swoim zadne drzewo przy wodach, i zeby nie wypuszczalo wierzchów swoich, miedzy gestwina galezi, i nie wspinalo sie nad inne wysokoscia swoja zadne drzewo wodami opojone. Albowiem ci wszyscy podan i sa na smierc, i wrzuceni w niskosci ziemi w posród synów ludzkich z tymi, którzy zstepuja do dolu.
\par 15 Tak mówi panujacy Pan: Dnia, którego on zstapil do grobu, uczynilem lament, zawarlem dla niego przepasc, i zahamowalem strumienie jej, aby sie zastanowily wody wielkie; i uczynilaem, ze sie zacmil dla niego Liban, a wszystkie drzewa polne dla nie go zemdlaly.
\par 16 Od grzmotu upadku jego zatrwozylem narody, gdym go wepchnal do grobu z tymi, co w dól zstepuja, nad czem sie ucieszyly na ziemi na dole wszystkie drzewa Eden, i co jest najwyborniejszego, i najlepszego na Libanie, i wszystko, co jest opojone woda.
\par 17 I ci z nim zstapili do grobu, do pobitych mieczem, którzy byli ramieniem jego, i którzy siadali w cieniu jego w posrodku narodów.
\par 18 Komuzes podobny byl slawa i wielkoscia miedzy drzewami Eden? Oto zrzucony bedziesz z drzewami Eden na dól na ziemie; w posrodku nieobrzezanców polezesz miedzy pobitymi mieczem. Toc jest Farao i wszystka zgraja jego, mówi panujacy Pan.

\chapter{32}

\par 1 A dwunastego roku, miesiaca dwunastego, pierwszego dnia tegoz miesiaca, stalo sie slowo Panskie do mnie, mówiac:
\par 2 Synu czlowieczy! podnies lament nad Faraonem, królem Egipskim, a powiedz mu: Podobnys ty lwowi mlodemu miedzy narodami, tys jest jako wieloryb w morzu, gdyz bujajac po rzekach twoich macisz wode nogami twemi, i mieszasz rzeki jego.
\par 3 Tak mówi panujacy Pan: Rozciagne tez na cie siec moje przez zebranie wielu narodów, którzy cie wyciagna niewodem moim;
\par 4 I zostawie cie na ziemi, na polu porzuce cie, i sprawie, ze mieszkac bedzie na tobie wszelkie ptastwo niebieskie, i nakarmie toba zwierza wszystkiej ziemi;
\par 5 I rozrzuce mieso twoje po górach, i napelnie doliny wysokoscia twoja,
\par 6 I napoje ziemie twoje, w której plywasz, krwia twoja az do gór, tak, ze i rzeki beda napelnione toba.
\par 7 A gdy cie zgasze, zakryje niebiosa, i ciemne uczynie gwiazdy ich, slonce oblokiem zaslonie, a ksiezyc nie da swiatla swego.
\par 8 Wszystkie swiatla jasne na niebiosach zacmie dla ciebie, i przywiode ciemnosc na ziemie twoje, mówi panujacy Pan.
\par 9 Nadto zasmuce serce wielu narodów, gdy za sprawa moja przyjdzie wiesc o starciu twojem miedzy narody, do ziem, któryches nie znal.
\par 10 Uczynie, mówie, ze sie zdumieje nad toba wiele narodów, a królowie ich zatrwoza sie bardzo dla ciebie, gdy szermowac bede mieczem swoim przed twarza ich; beda sie zaiste lekac co chwila kazdy o dusze swoje w dzien upadku twego.
\par 11 Bo tak mówi panujacy Pan: Miecz króla Babilonskiego przyjdzie na cie.
\par 12 Mieczami mocarzów poraze zgraje twoje; najokrutniejsi ze wszystkich narodów, ci skaza pyche Egipska, i bedzie zgladzone wszystko mnóstwo jego.
\par 13 Zagladze i wszystko bydlo jego, które jest przy wodach wielkich, tak, ze ich wiecej nie zmaci noga czlowiecza, ani kopyta bydlece macic ich beda.
\par 14 Tedy uczynie, ze sie wody ich ustoja, a rzeki ich jako oliwa pójda, mówi panujacy Pan.
\par 15 Gdyz obróce ziemie Egipska w spustoszenie, a ziemia bedzie wyprózniona z pelnosci swojej, gdy pobije wszystkich mieszkajacych w niej; i dowiedza sie, zem Ja Pan.
\par 16 Toc jest lament, którym nad nia lamentowac beda; córki narodów narzekac beda nad nia, nad Egiptem i nad wszystkiem mnóstwem jego narzekac beda, mówi panujacy Pan.
\par 17 Potem dwunastego roku, pietnastego dnia tegoz miesiaca, stalo sie slowo Panskie do mnie, mówiac;
\par 18 Synu czlowieczy! narzekaj nad mnóstwem Egipskiem, a zepchnij je, i córki tych narodów slawnych az do najnizszych miejsc ziemi, do tych, co zstepuja do dolu,
\par 19 I mów: Nad kogozes wdzieczniejszy? Zstap, a polóz sie z nieobrzezancami.
\par 20 W posród pobitych mieczem upadna; pod miecz podany jest, wywleczciez go ze wszystka zgraja jego.
\par 21 Mówic do niego beda najmocniejsi z mocarzów z posrodku grobu z pomocnikami jego, którzy tam zstapiwszy, leza z nieobrzezancami pobitymi mieczem.
\par 22 Tam jest Assur, i wszystka zgraja jego, w okolo niego sa groby jego; wszyscy ci pobici upadli od miecza.
\par 23 Którego groby polozone sa przy stronach dolu, aby byla zgraja jego w okolo grobu jego; ci wszyscy pobici polegli od miecza, którzy puszczali strach w ziemi zyjacych.
\par 24 Tam Elam i wszystka zgraja jego okolo grobu jego, ci wszyscy pobici upadli od miecza, którzy zstapili w nieobrzezce do niskosci ziemi, którzy puszczali strach swój w ziemi zyjacych; juzci odnosza hanbe swoja z tymi, którzy zstepuja do dolu.
\par 25 W posrodku pobitych postawili mu loze, i wszystkiej zgrai jego, w okolo niego sa groby jego; wszyscy ci nieobrzezancy pobici sa mieczem, przychodzil strach ich na ziemie zyjacych, juzci odnosza hanbe swoje z tymi, którzy zstapili do grobu, a w pos rodku pobitych polozeni sa.
\par 26 Tam Mesech, Tubal i wszystka zgraja jego, i w okolo niego groby jego, ci wszyscy nieobrzezancy pobici mieczem, choc puszczali strach swój w ziemi zyjacych.
\par 27 Aczkolwiekci jeszcze nie polegli z mocarzami, którzy upadli z nieobrzezanców, co zstapili do grobu z wojennym orezem swoim, i polozyli miecze swe pod glowy swe; a wszakze przyjdzie nieprawosc ich na kosci ich, chociaz strach tych mocarzów byl w ziemi zyjacych.
\par 28 I ty w posrodku nieobrzezanców starty bedziesz, i bedziesz lezal miedzy pobitymi mieczem.
\par 29 Tam Edom, i królowie jego, i wszyscy ksiazeta jego, którzy polozeni sa, z moca swoja i z pobitymi mieczem; i ci z nieobrzezancami lezec beda, i z tymi, którzy zstepuja do dolu.
\par 30 Tam wszyscy zgola ksiazeta pólnocnej strony, i wszyscy Sydonczycy, którzy zstepuja do pobitych z strachem swoim, za moc swoje wstydzic sie beda, i lezec beda ci nieobrzezancy z pobitymi mieczem, a odniosa hanbe swoje z tymi, którzy zstepuja do dolu.
\par 31 Tych ujrzawszy Farao ucieszy sie nad wszystka zgraja swoja, która jest mieczem pobita, Farao i wszystko wojsko jego, mówi panujacy Pan.
\par 32 Bo puszcze strach mój w ziemi zyjacych, i polozony bedzie miedzy nieobrzezancami z pobitymi mieczem Farao i wszystka zgraja jego, mówi panujacy Pan.

\chapter{33}

\par 1 I stalo sie slowo Panskie do mnie, mówiac:
\par 2 Synu czlowieczy! mów do synów ludu twego, a rzecz do nich: Gdy przywiode miecz na która ziemie, jezlize lud onej ziemi wezmie meza jednego z granic swoich, a postanowi go sobie za stróza,
\par 3 A on widzac miecz przychodzacy na one ziemie, zatrabilby w trabe i przestrzeglby lud,
\par 4 A któryby slyszal glos traby, i nie dbalby na przestroge, a wtem przyszedlszy miecz, zgladzilby go; krew jego bedzie na glowie jego;
\par 5 Bo glos traby slyszal, wszakze nie dbal na przestroge, dlatego krew jego na nim bedzie; byc byl przyjal przestroge, zachowalby byl dusze swoje.
\par 6 Ale jezliby stróz ujrzal miecz przychodzacy, a nie zatrabilby w trabe, a luduby nie przestrzegl, i przyszedlby miecz, i znióslby którego z nich, takowy bedzie w nieprawosci swojej zachwycony; ale krwi jego z reki onego stróza szukac bede.
\par 7 Ciebiec, synu czlowieczy! ciebiem postanowil strózem domu Izraelskiego, abys slyszac slowo z ust moich, przestrzegl ich odemnie.
\par 8 Gdybym Ja tedy rzekl niezboznemu: Niezbozniku! smiercia, umrzesz, a tybys mu tego nie powiedzial, przestrzegajac niezboznika od drogi jego; tenci niezboznik dla nieprawosci swojej umrze; ale krwi jego z reki twojej szukac bede.
\par 9 Ale jezlibys ty przestrzegl niezboznego od drogi jego, aby sie od niej odwrócil, wszakze nie odwrócilby sie od drogi swojej, onci dla nieprawosci swojej umrze; ale ty dusze swoje wybawisz.
\par 10 A tak ty, synu czlowieczy! mów do domu Izraelskiego: Tak powiadacie mówiac: Przeto, ze wystepki nasze i grzechy nasze sa na nas, tak, ze w nich schniemy, i jakozbysmy zyc mogli?
\par 11 Rzeczze tedy do nich: Jako zyje Ja, mówi panujacy Pan: Nie chce smierci niepoboznego, ale aby sie odwrócil niepobozny od drogi sojej, a zyl. Odwrócciez sie, odwrócciez sie od zlych dróg swoich, przeczze macie umrzec, o domie Izraelski!
\par 12 Ty tedy, synu czlowieczy! mów do synów ludu swego: Sprawiedliwosc sprawiedliwego nie wybawi go w dzien przestepstwa jego, a niezbozny nie upadnie w swojej niezboznosci w dzien, któregoby sie odwrócil od niezboznosci swojej; takze sprawiedliwy nie bedzie mógl zyc dla sprawiedliwosci swojej w dzien, któregoby zgrzeszyl.
\par 13 Jezlibym zas rzekl sprawiedliwemu: Pewnie zyc bedziesz, a onby ufajac sprawiedliwosci swojej czynil nieprawosc, zadna sprawiedliwosc jego nie przyjdzie na pamiec; ale dla tej nieprawosci swojej, która czynil, umrze.
\par 14 Zasie, rzekeli niepoboznemu: Smiercia umrzesz, a onby sie odwrócil od grzechu swego, i czynilby sad i sprawiedliwosc,
\par 15 Wrócilby niezbozny zastaw, a co wydarl, oddalliby, i chodzilby w ustawach zywota, nie czyniac nieprawosci, pewnie zyc bedzie, a nie umrze.
\par 16 Zadne grzechy jego, któremi grzeszyl, nie beda mu wspominane; sad i sprawiedliwosc czynil, pewnie zyc bedzie.
\par 17 A wzdy mówia synowie ludu twego: Nie prawa jest droga Panska, choc onych samych droga nie jest prawa.
\par 18 Gdyby sie odwrócil sprawiedliwy od sprawiedliwosci swojej, a czynilby nieprawosc, umrze dla niej;
\par 19 Ale gdyby sie odwrócil niezbozny od niezboznosci swojej, a czynilby sad i sprawiedliwosc, dlatego zyc bedzie.
\par 20 A przecie mówicie: Nie prawa jest droga Panska; kazdego z was wedlug drogi jego sadzic bede, o domie Izraelski!
\par 21 I stalo sie dwunastego roku, dziesiatego miesiaca, piatego dnia tegoz miesiaca od zaprowadzenia naszego, ze przyszedl do mnie jeden, który uszedl z Jeruzalemu, mówiac: Dobyto miasta.
\par 22 A reka Panska byla przy mnie w wieczór przedtem, niz przyszedl ten, który uciekl, i otworzyla usta moje, az do mnie rano przyszedl; otworzyla mówie usta moje, abym niemym dalej nie byl.
\par 23 I stalo sie slowo Panskie do mnie, mówiac:
\par 24 Synu czlowieczy! Obywatele tych spustoszonych miejsc w ziemi Izraelskiej powiadaja, mówiac: Abraham sam jeden byl, a wzdy posiadl te ziemie; ale nas jest wiele, namci dana jest ta ziemia w osiadlosc.
\par 25 Dlategoz mów do nich: Tak mówi panujacy Pan: Ze krwia jadacie, i oczy swe podnosicie do plugawych balwanów swoich, i krew wylewacie, a chcielibyscie te ziemie posiasc?
\par 26 Stoicie na mieczu waszym, czynicie obrzydliwosc, a kazdy zone blizniego swego plugawi; izali ziemie posiadziecie?
\par 27 Tak rzecz do nich: Tak mówi panujacy Pan: Jako zyje Ja, ze ci, którzy sa na miejscach spustoszonych, od miecza polegna; a kto jest na polu, tego podam bestyjom na pozarcie; a którzy sa na zamkach albo w jaskiniach, morem pomra;
\par 28 I podam ziemie na wielkie spustoszenie, i ustanie pycha mocy jej; i spustoszeja góry Izraelskie, a nie bedzie, ktoby po nich chodzil.
\par 29 I dowiedza sie, zem Ja Pan, gdy podam ziemie ich na wielkie spustoszenie dla wszystkich obrzydliwosci ich, które czynili.
\par 30 Ale ty, synu czlowieczy! sluchaj. Synowie ludu twojego czesto mówia o tobie okolo scian i we drzwiach domów, i mówi jeden do drugiego, i kazdy do blizniego swego, mówiac: Pójdzcie, a posluchajcie, co za slowo od Pana wyszlo.
\par 31 I schodza sie do ciebie, tak, jako sie schodzi lud, i siadaja przed obliczem twojem, jako lud mój, i sluchaja slów twoich, ale ich nie czynia; a choc je sobie usty swemi smakuja, wszakze za szkaradnym zyskiem swoim serce ich chodzi.
\par 32 A oto tys im jest jako piesn wdzieczna pieknego glosu, i dobrze umiejetnego spiewaka; sluchajac w prawdzie slów twoich, ale ich nie czynia.
\par 33 Lecz gdy to przyjdzie, (jakoz oto przychodzi) dopiero sie dowiedza, ze prorok byl w posrodku nich.

\chapter{34}

\par 1 I stalo sie slowo Panskie do mnie mówiac:
\par 2 Synu czlowieczy! prorokuj przeciwko pasterzom Izraelskim, prorokuj, a mówi do tych pasterzy: Tak mówi panujacy Pan: Biada pasterzom Izraelskim, którzy sami siebie pasa!
\par 3 Izali pasterze trzody pasc nie maja? Tlustosc jadacie, a welna sie przyodziewacie, to, co jest tlustego zabijacie, a trzody nie pasiecie;
\par 4 Slabych nie posilacie, a chorego nie leczycie, i zlamanego nie zawiazujecie, a zaploszonego nie przywodzicie, ani zgubionego nie szukacie, owszem, nad nimi surowie i srodze panujecie.
\par 5 Tak, ze rozproszone bedac sa bez paterza i staly sie na pozarcie wszelkiemu zwierzowi polnemu. poniewaz sie rozpierzchnely.
\par 6 Blakaja sie owce moje po wszystkich górach, i po kazdym pagórku wysokim; owszem, po wszystkiej ziemi rozproszyly sie owce moje, a nie byl, ktoby ich szukal, i ktoby sie za niemi pytal.
\par 7 Dlatego, wy pasterze! sluchajcie slowa Panskiego,
\par 8 Jako zyje Ja, mówi panujacy Pan: Przeto, iz trzoda moje jest na lup dana, a owce moje sa na pozarcie wszelkiemu zwierzowi polnemu, bedac bez pasterza a iz nie szukaja pasterze moi owiec moich, ale tylko pasterze samych siebie pasa, a owiec moich nie pasa;
\par 9 Przetoz o pasterze! sluchajcie slowo Panskiego.
\par 10 Tak mówi panujacy Pan: Otom Ja przeciwko pasterzom, i szukac bede owiec moich z rak ich, a uczynie, ze oni przestana pasc owiec moich, aby nie pasli wiecej pasterze samych siebie; wydre zaiste owce moje z geby ich, i nie beda im wiecej pokarmem.
\par 11 Bo tak mówi panujacy Pan: Oto Ja, Ja szukac bede owiec moich, i pytac sie za niemi.
\par 12 Jako sie pyta pasterz o trzode swoje, kiedy bywa w posrodku owiec swoich rozproszonych: tak sie Ja bede pytal za owcami mojemi, i wyrwe je ze wszystkich miejsc, kedy byly rozproszone w dzien obloku i chmury;
\par 13 I wywiode je z narodów, a zgromadze je z ziem, i przywiode je do ziemi ich, a pasc je bede na górach Izraelskich nad strumieniami, i po wszystkich mieszkaniach tej ziemi.
\par 14 Na pastwiskach dobrych pasc je bede, a na górach wysokich Izraelskich beda pastwiska ich; tam odpoczywac beda w trawach bujnych, a w paszach tlustych pasc sie beda na górach Izraelskich.
\par 15 Ja sam pasc bede owce moje, i Ja im poczynie legowiska, mówi panujacy Pan.
\par 16 Zgubionej szukac bede, a zaploszona przywiode, i polamana zawiaze, a slaba posile; ale tlusta i mocna wytrace; bo je bede pasl w sadzie.
\par 17 A wy, trzodo moja! tak mówi panujacy Pan: Oto Ja uczynie sad miedzy owca a owca, miedzy barany a kozly.
\par 18 Azaz wam na tem malo, pasc sie na dobrej paszy, ze jeszcze ostatek pastwisk waszychdepczecie nogami swojemi? a czysta wode pic, ze ostatek nogami swemi macicie?
\par 19 Tak, ze sie owce moje tem, co bylo podeptane nogami waszemi, pasc, a meciny nóg waszych pic musza.
\par 20 Przetoz tak mówi panujacy Pan do nich: Oto Ja, Ja sad uczynie miedzy bydleciem tlustym i miedzy bydleciem chudem,
\par 21 Dlatego, ze wy bokami i plecami tracacie, a rogami waszemi bodziecie wszystkie slabe, tak zescieje precz rozegnali.
\par 22 Przetoz wyzwole owce moje, ze juz dalej lupem nie beda, i uczynie sad miedzy owca i owca;
\par 23 I wzbudze nad niemi pasterza jednego, który je pasc bedzie, sluge mego Dawida, on je pasc bedzie, i on bedzie pasterzem ich.
\par 24 A Ja Pan bede im Bogiem, a sluga mój Dawid ksiazeciem w posrodku nich, Ja Pan mówilem to.
\par 25 I uczynie z nimi przymierze pokoju, a wygubie zly zwierz z ziemi; i beda na puszczy bezpiecznie mieszkac, a w lasach sypiac beda;
\par 26 Nadto dam im, i okolicy pagórka mego, blogaslawienstwo, i spuszczac bede deszcz czasu swego; deszcze to blogoslwienstwa beda;
\par 27 I wypusci drzewo polne owoc swój, a ziemia wyda urodzaj swój, i beda na ziemi swojej bezpieczni, a dowiedza sie, zem Ja Pan, gdy polamie zawory jarzma ich, a wyrwie je z reki tych, którzy je zniewalaja.
\par 28 I nie beda wiecej lupem narodom, a zwierz ziemiski pozerac ich nie bedzie; ale mieszkac beda bezpiecznie, a nie bedzie, ktoby je straszyl.
\par 29 I wzbudze im latorosl slawna, ze nie beda wiecej glodem niszczeni na ziemi, ani poniosa pohanbienia od pogan.
\par 30 I dowiedza sie, zem Ja Pan, Bóg ich, z nimi, a oni lud mój, dom Izraelski, mówi panujacy Pan.
\par 31 Ale wy owce moje, owce pastwiska mego, wyscie lud mój, a Jam Bóg wasz, mówi panujacy Pan.

\chapter{35}

\par 1 I stalo sie slowo Panskie do mnie, mówiac:
\par 2 Synu czlowieczy! obróc twarz swoje przeciwko górze Seir, a prorokuj przeciwko niej,
\par 3 I mów do niej: Tak mówi panujacy Pan: Otom Ja przeciwko tobie, góro Seir! a wyciagne reke moje na cie, i podam cie na wielkie spustoszenie.
\par 4 Miasta twoje w spustoszenie obróce, ze bedziesz spustoszona, i dowiesz sie, zem Ja Pan.
\par 5 Przeto, iz wieczna nieprzyjazn wiedziesz, i rozpraszasz synów Izraelskich ostrzem miecza czasu utrapienia ich, czasu wykonywania kazni ich;
\par 6 Dlatego, jako zyje Ja, mówi panujacy pan, ze cie podam na zabicie, a krew cie scigac bedzie; poniewaz krwi rozlewania w nienawisci nie masz, przetoz cie krew scigac bedzie.
\par 7 I obróce góre Seir w wielkie spustoszenie, a wygubie z niej przechodzacego, i wracajacego sie;
\par 8 I napelnie góry jej pobitymi jej; na pagórkach twoich, i na dolinach twoich, i przy wszystkich strumieniach twoich pobici mieczem polegna na nich;
\par 9 Na pustynie wieczne podam cie, i w miastach twoich mieszkac nie beda; a dowiecie sie, zem Ja Pan.
\par 10 Przetoz iz mówisz: Te dwa narody, i te dwie ziemie moje beda a posiadziemy te, w której Pan mieszkal;
\par 11 Dlategoz, jako zyje Ja, mówi panujacy Pan, ze uczynie wedlug gniewu twego i wedlug zawisci twojej, któras czynila w nienawisci swej przeciwko nim; i bede poznany od nich, gdy cie sadzic bede;
\par 12 I dowiesz sie, zem Ja Pan slyszal wszystkie uragania twoje, któres wyrzekla przeciw górom Izraelskim, mówiac: Spustoszone sa, a nam podane sa ku pozarciu.
\par 13 Boscie sie przeciwko mnie wynosili usty waszemi, a rozmnozyliscie slowa swoje przeciwko mnie, com Ja slyszal.
\par 14 Tak mówi panujacy Pan: Jako sie ta wszystka ziemia weseli, tak cie obróce w pustnie.
\par 15 Jako sie ty weselisz nad dziedzictwem domu Izraelskiego, przeto, ze jest spustoszone, tak uczynie i tobie; bedziesz spustoszona, o góro Seir! a zgola wszystka ziemia Edomska; i dowiedza sie zem Ja Pan.

\chapter{36}

\par 1 A ty synu czlowieczy! prorokuj o górach Izraelskich, a mów: Góry Izraelskie! sluchajcie slowa Panskiego;
\par 2 Tak mówi panujacy Pan: Przeto, iz nieprzyjaciel rzekl o was: Hej, hej! i wysokosci wieczne dostaly sie nam w dziedzictwo;
\par 3 Przetoz prorokuj a mów: Tak mówi panujacy Pan: Dlatego, dlatego, mówie, iz was zburzyli, a pochloneli zewszad, i staliscie sie dziedzictwem pozostalym narodom, i przyszliscie na jezyk i na obmowisko ludzkie;
\par 4 Przetoz, góry Izraelskie! sluchajcie slowa panujacego Pana. Tak mówi panujacy Pan górom i pagórkom, strumieniom i dolinom, pustyniom, obalinom, i miastom opuszczonym, które sa na splundrowanie, i na posmiewisko ostatkowi narodów okolicznych.
\par 5 Dlatego, tak mówi panujacy Pan, zaprawde w ogniu zapalczywosci mojej mówic bede przeciw ostatkom tych narodów, i przeciwko wszystkiej ziemi Edomskiej, którzy sobie przywlaszczyli ziemie moje za dziedzictwo z weselem calego serca, i z ochotnem pust oszeniem, aby siedlisko wygnanych jego bylo na rozszarpanie,
\par 6 Przetoz prorokuj o ziemi Izrelskiej, a mów górom, i pagórkom, strumieniom i dolinom: Tak mówi panujacy Pan: Oto Ja w zapalczywosci mojej, i w popedliwosci mojej mówie, dlatego, iz hanbe od narodów ponosicie.
\par 7 Przetoz tak mówi panujacy Pan: Jam podniósl reke moje, iz te narody, które sa zewszad okolo was, sami hanbe swoje poniosa.
\par 8 A wy, góry Izraelskie! wypuscie galazki swe, i owoc swój przyniescie ludowi memu Izraelskiemu, gdy sie przybliza a przyjda.
\par 9 Bo oto Ja ide do was, i nawróce sie do was, a uprawiane i posiewane bedziecie;
\par 10 I rozmnoze na was ludzi, wszystek zgola dom Izarelski; i beda mieszkac w miastach, a miejsca zburzone pobudowane beda.
\par 11 Rozmnoze, mówie, na was ludzi i dobytek, a rozmnoza sie i urosna; i sprawie, ze mieszkac bedziecie jako za dawnych czasów waszych, i lepiej wam czynic bede niz przedtem, i dowiecie sie, zem Ja Pan.
\par 12 Bo przyprowadze na was ludzi, lud mój Izraelski, i posiada cie, i bedziesz im dziedzictwem, a wiecej ich nie osierocisz.
\par 13 Tak mówi panujacy Pan: Dlatego, ze o was powiadaja: Tys jest ta ziemia, która pozerasz ludzi, i osieracasz narody twoje;
\par 14 Przetoz nie bedziesz wiecej ludzi pozerala, ani narodów twoich wiecej osieracala, mówi panujacy Pan.
\par 15 I nie dopuszcze w tobie wiecej slyszec hanby narodów, ani zelzywosci ludzkiej nie poniesiesz wiecej, i narodów twoich nie przywiedziesz wiecej do upadku, mówi panujacy Pan.
\par 16 I stalo sie slowo Panskie do mnie, mówiac:
\par 17 Synu czlowieczy! dom Izraelski mieszkajac w ziemi swej splugawial ja drogami swemi i sprawami swemi, tak, ze droga ich przed obliczem mojem byla jako nieczystosc niewiasty odlaczonej.
\par 18 Przetoz wylalem gniew mój na nich dla krwi, która wylali na ziemie, i dla plugawych balwanów ich, któremi ja splugawili.
\par 19 I rozproszylem ich miedzy narody, a rozwiani sa po ziemiach; wedlug dróg ich i wedlug spraw ich sadzilem ich.
\par 20 A gdy weszli do narodów, do których przyszli, pomazali tam imie swiatobliwosci mojej, gdy o nich mówiono: Lud to Panski, a z ziemi jego wyszli.
\par 21 Alem im sfolgowal dla imienia swietobliwosci mojej, które splugawil dom Izraelski miedzy narodami, do których przyszli;
\par 22 Przetoz mów do domu Izraelskiego: Tak mówi panujacy Pan: Nie dla was Ja to czynie, o domie Izraelski! ale dla imienia swietobliwosci mojej, którescie splugawili miedzy narodami, do którychescie przyszli;
\par 23 Abym poswiecil wielkie imie moje, które bylo splugawione miedzy narodami, którescie wy zmazali w posrodku ich; i dowiedza sie narody, zem Ja Pan, mówi panujacy Pan, gdy poswiecony bede w was przed oczyma ich;
\par 24 Bo was zbiore z narodów, i zgromadze was ze wszystkich ziem, i przywiode was do ziemi waszej;
\par 25 I pokropie was woda czysta, a oczyszczeni bedziecie od wszystkich nieczystót waszych, i od wszystkich plugawych balwanów waszych oczyszcze was;
\par 26 I dam wam serce nowe, a ducha nowego dam do wnetrznosci waszych, i odejme serce kamienne z ciala waszego, a dam wam serce miesiste.
\par 27 Ducha mego, mówie, dam do wnetrznosci waszej, a uczynie, ze w ustawach moich chodzic, a sadów moich przestrzegac, i czynic je bedziecie.
\par 28 A bedziecie mieszkac w ziemi, któram dal ojcom waszym, i bedziecie ludem moim, a Ja bede Bogiem waszym.
\par 29 Bo was wyzwole od wszystkich nieczystót waszych, i przywolam zboza, i rozmnoze je, a nie dopuszcze na was glodu.
\par 30 Rozmnoze tez owoc drzew, i urodzaje polne, abyscie wiecej nie nosili hanby glodu miedzy narodami.
\par 31 I wspomnicie na zle drogi wasze, i na sprawy wasze, które nie byly dobre, i omierzniecie sami sobie w oczach swoich dla nieprawosci waszych i dla obrzydliwosci waszych.
\par 32 Nie dla wasci Ja to czynie, mówi panujacy Pan, niech wam to jawno bedzie; sromajcie sie, a wstydzcie sie za drogi wasze, o domie Izraelski!
\par 33 Tak mówi panujacy Pan: Którego was dnia oczyszcze od wszystkich nieprawosci waszych, osadze miasta, a miejsca zburzone beda pobudowane.
\par 34 A tak ziemia spustoszala sprawowana bedzie, która przedtem byla spustoszona przed oczyma wszystkich przechodzacych.
\par 35 I rzeka: Ziemia ta spustoszala stala sie jako ogród Eden; takze miasta puste i opuszczone i rozwalone, obronne sa i osadzone.
\par 36 I dowiedza sie narody, którekolwiek zostna okolo was, zem Ja Pan pobudowal rozwaliny, a nasadzil miejsca spustoszone. Ja Pan mówilem, i uczynie.
\par 37 Tak mówi panujacy Pan: Jeszcze tego bedzie u mnie szukal dom Izraelski, abym to im uczynil, abym ich rozmnozyl w ludzi jako trzode.
\par 38 Jako trzode na ofiary, jako trzode Jeruzalemska na swieta jego uroczyste, tak spustoszone miasta beda napelnione trzodami ludzi; i dowiedza sie, zem Ja Pan.

\chapter{37}

\par 1 Byla nademna reka Panska, i wywiódl mie Pan w duchu, i postawil mie w posrodku pola, które bylo pelne kosci;
\par 2 I przewiódl mie przez nie w okolo a w okolo, a oto bylo ich bardzo wiele na onem polu, a oto byly bardzo suche.
\par 3 I rzekl do mnie: Synu czlowieczy! ozyjali te kosci? I rzeklem: Panujacy Panie! ty wiesz.
\par 4 W tem rzekl do mnie: Prorokuj o tych kosciach, a mów do nich: Kosci suche, sluchajcie slowa Panskiego!
\par 5 Tak mówi panujacy Pan o tych kosciach: Oto ja wprowadze w was ducha, a ozyjecie;
\par 6 A wloze na was zyly, i uczynie, ze porosnie na was mieso, i powleke was skóra, a dam wam ducha, i ozyjecie, i poznacie, zem Ja Pan.
\par 7 Prorokowalem tedy, jako mi rozkazano; i stal sie szum, gdym ja prorokowal, a oto poruszenie; i przystapily kosci, kosc do kosci swojej.
\par 8 I ujrzalem, a oto na nich zyly, i mieso poroslo, i powleczone byly skóra po wierzchu; ale ducha nie bylo w nich.
\par 9 I rzekl do mnie: Prorokuj do ducha, prorokuj, synu czlowieczy! i rzecz do ducha: Tak mówi panujacy Pan: Od czterech wiatrów przyjdz, duchu! i natchnij te pobite, a niech ozyja.
\par 10 Prorokowalem tedy jako mi byl rozkazal, i wstapil w nie duch, a ozyly, i stanely na nogach swoich, wojsko nader bardzo wielkie.
\par 11 I rzekl do mnie: Synu czlowieczy! te kosci sa wszystek dom Izraelski. Oto mówia: Wyschly kosci nasze, i zginela nadzieja nasza, wygladzeni jestesmy.
\par 12 Dlategoz prorokuj, a mów do nich: Tak mówi panujacy Pan: Oto Ja otworze groby wasze, i wywiode was z grobów waszych, ludu mój! i przywiode was do ziemi Izraelskiej!
\par 13 I dowiecie sie, zem Ja Pan, gdy Ja otworze groby wasze, a wywiode was z grobów waszych, ludu mój!
\par 14 I dam w was ducha mego, a ozyjecie, i dam wam odpoczac w ziemi waszej; i dowiecie sie, ze Ja Pan mówie to i uczynie, mówi Pan.
\par 15 I stalo sie slowo Panskie do mnie, mówiac:
\par 16 A ty, synu czlowieczy! wezmij sobie jedno drewno, i napisz na niem: Judzie i synom Izraelskim, towarzyszom jego. Wezmij tez i drugie drewno, a napisz na niem: Józefowi drewnu Efraimowemu, i wszystkiemu domowi Izraelskiemu, towarzyszom jego!
\par 17 I zlóz je sobie jedno do drugiego w jedno drewno, aby byly jako jedno w rece twojej.
\par 18 A gdy rzekna do ciebie synowie ludu twego, mówiac: Izali nam nie oznajmisz, co przez to rozumiesz?
\par 19 Rzeczesz im: Tak mówi panujacy Pan: Oto Ja wezme drewno Józefowe, które jest w rece Efraimowej i pokolenia Izraelskie, towarzyszów jego, i przyloze je z nim do drewna Judowego, i uczynie je jednem drewnem, a beda jednem w rece mojej.
\par 20 A gdy beda one drewna, na których napiszesz, w rece twojej przed oczyma ich,
\par 21 Tedy rzecz do nich: Tak mówi panujacy Pan: Oto Ja wezme synów Izraelskich z posrodku tych narodów, do których byli zaszli, i zgromadze ich zewszad, a przywiode ich do ziemi ich;
\par 22 I uczynie ich narodem jednym w ziemi, na górach Izraelskich, i bedzie król jeden nad onymi wszystkimi za króla; a nie beda wiecej dwa narody, nie rozdziela sie nigdy wiecej na dwoje królestw;
\par 23 Nie splugwia sie wiecej plugawemi balwanami swemi, i obrzydliwosciami swemi, ani jakiemi przestepstwy swemi; i wybawie ich z kazdego mieszkania ich, gdzie zgrzeszyli, i oczyszcze ich, i beda ludem moim, a Ja bede Bogiem ich.
\par 24 A sluga mój Dawid bedzie królem nad nimi, i pasterza jednego wszyscy miec beda, aby w sadach moich chodzili, i ustaw moich przestrzegali, i czynili je.
\par 25 I beda mieszkac w onej ziemi, któram byl dal sludze memu Jakóbowi, w której mieszkali ojcowie wasi; beda, mówie, w niej mieszkali oni i synowie ich, i synowie synów ich az na wieki, a Dawid, sluga mój, bedzie ksiazeciem ich na wieki.
\par 26 I uczynie tez z nimi przymierze pokoju, a przymierze wieczne bedzie z nimi; i osadze ich i rozmnoze ich, i zaloze swiatnice moje w posrodku ich na wieki.
\par 27 I bedzie przybytek mój miedzy nimi, i bede Bogiem ich, a oni beda ludem moim.
\par 28 I dowiedza sie narody, zem Ja Pan, który poswiecam Izraela, gdy bedzie swiatnica moja w posrodku ich na wieki.

\chapter{38}

\par 1 I stalo sie slowo Panskie do mnie mówiac:
\par 2 Synu czlowieczy! obróc twarz swoje przeciw Gogowi w ziemi Magog, ksiazeciu glównemu w Mesech i Tubal, a prorokuj przeciw niemu.
\par 3 I rzecz: Tak mówi panujacy Pan: Otom Ja przeciw tobie, o Gogu, ksieciu glówny w Mesechu i w Tubalu!
\par 4 Bo cie zawróce, i wloze wedzidlo w czelusci twoje, i wywiode cie, i wszystko wojsko twoje, konie i jezdnych wszystkich poubieranych w zupelny kirys, hufy wielkie z tarczami i z przylbicami, wszystkich tych, którzy wladaja mieczem.
\par 5 Persów, Murzynów i Putejczyków z nimi, tych wszystkich z tarczami i z przylbicami;
\par 6 Gomer i wszystkie hufy jego, dom Togormy mieszkajacego w stronach pólnocnych, i wszystkie poczty jego, narodów wiele z toba.
\par 7 Gotuj sie, a wypraw sie, ty i wszystkie pólki twoje, które sie zebraly do ciebie, a badz strózem ich.
\par 8 Po wielu dniach nawiedziony bedziesz, a w ostatnie lata przyciagniesz na lud uwolniony od miecza, i zebrany z wielu narodów, na góry Izraelskie, które byly pustynia ustawiczna, gdyz oni z narodów bedac wywiedzieni, wszyscy bezpiecznie mieszkac beda.
\par 9 W tem przyciagniesz i przyjdziesz jako burza, bedziesz jako oblok okrywajacy ziemie, ty i wszystkie poczty twoje, i wiele narodów z toba.
\par 10 Tak mówi panujacy Pan: Dnia onego wstapia dziwne rzeczy na serce twoje, a bedziesz zle zamysly myslal,
\par 11 I rzeczesz: Wtargne do ziemi, w której sa wsi; przypadne na spokojnych i bezpiecznie mieszkajacych, na wszystkich, którzy mieszkaja bez muru, a zawór i bram nie maja;
\par 12 Abym wzial lupy, a rozchwycil korzysci; abym obrócil reke swoje na spustoszone miejsca juz znowu osadzone, i na lud zgromadzony z narodów, którzy sie bydlem i kupiectwem bawia, a mieszkaja w posrodku ziemi.
\par 13 Seba, i Dedan, i kupcy morscy, i wszystkie lwieta jego rzekna do ciebie: Izali ty na branie lupów idziesz? Izali na rozchwycenie korzysci zebrales pólki twoje, abys wybral srebro i zloto, i zabral dobytek i majetnosci, a zebys zebral lup wielki?
\par 14 Przetoz prorokuj, synu czlowieczy! a mów do Goga: Tak mówi panujacy Pan: Izali sie w on dzien, gdy lud mój Izraelski bezpiecznie mieszkac bedzie, nie dowiesz?
\par 15 I przyjdziesz z miejsca swego z stron pólnocnych, ty i narodów wiele z toba, wszyscy wsiadajacy na konie, lud wielki i wojsko gwaltowne;
\par 16 I przypadniesz na lud mój Izraelski jako oblok, abys okryl te ziemie. W ostatnie dni przywiode cie do ziemi mojej, aby mie poznaly narody, gdy bede poswiecony w tobie, przed oczyma ich, o Gogu!
\par 17 Tak mówi panujacy Pan: Azazes ty nie jest on, o którymem powiedzial za dni dawnych przez slug moich, proroków Izraelskich, którzy prorokowali za dni onych lat, zem cie mial przywiesc na nich?
\par 18 Wszakze w on dzien, w dzien, którego przyciagnie Gog na ziemie Izraelska, mówi panujacy Pan, wzruszy sie gniew mój w popedliwosci mojej;
\par 19 A w gorliwosci mojej, i w ogniu gniewu mego mówic bede, ze dnia onego bedzie wielki rozruch w ziemi Izraelskiej;
\par 20 I zadrza od oblicznosci mojej ryby morskie, i ptastwo niebieskie, i zwierz polny, i wszelka gadzina plazajaca sie po ziemi, i wszyscy ludzie, którzy sa na obliczu ziemi; i porozwalaja sie góry, i upadna wysokie wieze, i kazdy mur obali sie na zie mie.
\par 21 Bo przywolam przeciwko niemu po wszystkich górach moich miecz, mówi panujacy Pan; miecz kazdego obróci sie na brata jego.
\par 22 I bede sie z nim sadzil morem i krwia, a deszcz gwaltowny i grad kamienny, ogien i siarke spuszcze nan, i na wojska jego, i na wiele narodów, które z nim beda.
\par 23 I pokaze sie wielmoznym, i poswiece sie, i znajomym sie uczynie przed oczyma wielu narodów, i dowiedza sie, zem Ja Pan.

\chapter{39}

\par 1 A ty, synu czlowieczy! prorokuj przeciwko Gogowi, a mów: Tak mówi panujacy Pan: Otom Ja przeciwko tobie, Gogu, ksieciu glówny w Mesechu i w Tubalu!
\par 2 I zawróce cie, a szescioraka plaga scisne cie, i wywiode cie z stron pólnocych, a przywiode cie na góry Izraelskie;
\par 3 A wytrace luk twój z reki twojej lewej, i strzaly twoje z prawej reki twojej wybije.
\par 4 Na górach Izraelskich upadniesz, ty i wszystkie hufy twoje, i narody, które z toba beda; ptastwu i wszelkiej rzeczy skrzydlastej, i zwierzowi polnenu podam cie na pozarcie.
\par 5 Na obliczu pola upadniesz; bom Ja to wyrzekl, mówi panujacy Pan.
\par 6 I puszcze ogien na Magoga, i na tych, co bezpiecznie na wyspach mieszkaja; a dowiedza sie, zem Ja Pan.
\par 7 A imie swietobliwosci mojej oznajmie w posrodku ludu mego Izraelskiego, i nie dopuszcze wiecej zmazac imienia swietobliwosci mojej; i dowiedza sie narody, zem Ja Pan, swiety w Izraelu.
\par 8 Oto przyjdzie i stanie sie to, mówi panujacy Pan, tegoz dnia, o któremem mówil.
\par 9 Tedy wynijda obywatele miast Izraelskich, a zapaliwszy spala oreze i tarcze, i drzewca, luki i strzaly, kije reczne i wlócznie, i beda z nich niecic ogien przez siedm lat;
\par 10 A nie beda drew nosic z pola, ani ich rabac w lasach, ale z oreza ogien niecic beda a zlupia tych, którzy ich lupili, i splundruja tych, którzy ich plundrowali, mówi panujacy Pan.
\par 11 I stanie sie dnia onego, ze dam Gogowi miejsce na grób, tam w Izraelu, doline, któredy chodza na wschód slonca ku morzu, która zatka usta mimo idacych; i pogrzebia tam Goga i wszystkie zgraje jego, a beda ja zwac dolina mnóstwa Gogowego;
\par 12 Bo ich grzesc bedzie dom Izrelski przez siedm miesiecy aby oczyscili ziemie.
\par 13 A tak pogrzebie ich wszystek lud onej ziemi, i obróci sie im to w slawe, w dzien, którego Ja bede uwielbiony, mówi panujacy Pan.
\par 14 I obiora mezów ustwaicznych, którzyby sie przechodzili po onej ziemi, przechodzili, mówie, a chowali tych, którzyby zostali na ziemi, aby ja wyczyscili, a po wyjsciu siedmiu miesiecy szukac poczna.
\par 15 A ci przechodzac chodzic beda po ziemi, a ujrzawszy kosci czlowiecze postawia przy nich znak, aby je pochowali ci, co chowaja umarlych, w dolini mnóstwa Gogowego.
\par 16 Owszem, to mnóstwo jego bedzie ku slawie miastu, gdy oczyszcza one ziemie.
\par 17 A ty synu czlowieczy! tak mówi panujacy Pan: Rzecz ptastwu i wszelkiej rzeczy skrzydlastej, i kazdej bestyi polnej: Zbierzcie sie, a przyjdzcie, zgromadzcie sie zewszad na ofiare moje, która Ja wam sprawuje, ofiare wielka na górach Izraelskiech, zebyscie jedli mieso, i pili krew.
\par 18 Mieso mocarzy jesc bedziecie, a krew ksiazat ziemskich pic bedziecie, baranów, baranków, i kozlów, i cielców, którzy wszyscy potyli w Basan.
\par 19 Najecie sie tlustosci do sytosci a napijecie sie krwi do upicia z tej ofiary mojej, która wam nagotuje.
\par 20 I nasycicie sie z stolu mego konmi, i jezdzcami, mocarzami, i kazdym mezem walecznym, mówi panujacy Pan.
\par 21 A tak objawie chwale moje miedzy narodami, i ogladaja wszystkie narody sad mój, którym uczynil, i reke moje, któram na nie wyciagnal;
\par 22 A dowie sie dom Izraelski, zem Ja Pan, Bóg ich od onego dnia i na potem.
\par 23 Poznaja tez i narody, iz dla nieprawosci swojej zawiedzieni sa do wiezienia dom Izraelski, dlatego, iz wystapili przeciwko mnie. Dlategom tez byl zakryl oblicze swoje przed nimi, a podalem ich w rece nieprzyjaciól ich, aby wszyscy od miecza polegli.
\par 24 Wedlug nieczystosci ich, i wedlug przewrotnosci ich obszedlem sie z nimi, i zakrylem oblicze moje przed nimi.
\par 25 Przetoz tak mówi panujacy Pan: Juzci przywróce wiezniów Jakóbowych; a zmiluje sie nad wszystkim domem Izraelskim; i gorliwym bede dla imienia swietobliwosci mojej,
\par 26 Gdy odniosa pohanbienie swoje i wszystko przestepstwo swoje, którem wystapili przeciwko mnie, gdy bezpiecznie mieszkali w ziemi swojej, a nie byl, ktoby ich trwozyl;
\par 27 A gdy ich przywróce z narodów, i zgromadze ich z ziem nieprzyjaciól ich, i bede poswiecony w nich przed oczyma wielu narodów.
\par 28 Tedy sie dowiedza, zem Ja Pan, Bóg ich, gdy zawiódlszy ich do narodów zasie ich zgromadze do ziemi ich, a nie zostawie tam wiecej zadnego z nich.
\par 29 I nie zakryje wiecej oblicza mego przed nimi, gdyz wyleje ducha mojego na dom Izraelski, mówi panujacy Pan.

\chapter{40}

\par 1 Dwudziestego i piatego roku zaprowadzenia naszego, na poczatku roku, dziesiatego dnia miesiaca, czternastego roku po zburzeniu miasta, tegoz prawie dnia byla nademna reka Panska, a przywiódl mie tam.
\par 2 W widzeniach Bozych przywiódl mie do ziemi Izraelskiej, a postawil mie na górze bardzo wysokiej, na której bylo jakoby budowanie miasta na poludnie.
\par 3 I przywiódl mie tam, a oto maz, który byl na wejrzeniu jako ksztalt miedzi, majac sznur lniany w rece swej i laske ku rozmierzaniu, a ten stal w bramie.
\par 4 I mówil do mnie ten maz: Synu czlowieczy! patrz oczyma swemi, a uszyma swemi sluchaj, i przylóz serce swoje do wszystkiego, coc okaze; bos tu na to przywiedziony, abyc to ukazano, a ty oznajmisz wszystko, co widzisz, domowi Izraelskiemu.
\par 5 A oto mur zewnatrz przy domu zewszad w okolo; a w rece onego meza byla laska ku rozmierzaniu na szesc lokci, (a kazdy lokiec na dlon byl nad zwyczajny dluzszy.) i wymierzyl szerokosc onego budowania na laske jedne, i wysokosc na laske jedne,
\par 6 Potem wszedlszy do bramy, która byla na drodze wschodniej, wstapil po schodach jej, i wymierzyl próg bramy na laske jedne wszerz, a próg drugiej na jedne laske wszerz;
\par 7 Kazda tez komore na jedne laske wdluz, a na jedne laske wszerz; a miedzy komorami byl plac na piec lokci, próg tez bramy podle przysionku bramy wewnatrz byl na jedne laske.
\par 8 I wymierzyl przysionek bramy wewnatrz na jedne laske.
\par 9 Wymierzyl tez przysionek bramy na osm lokci, a podwoje jej na dwa lokcie, a ten przysionek bramy byl wewnatrz.
\par 10 Komory tez bramy ku drodze wschodniej byly trzy z jednej a trzy z drugiej strony; jednaka miara byla wszystkich trzech, jednaka tez miara podwoi ich z obu stron.
\par 11 Wymierzyl tez szerokosc drzwi onej bramy na dziesiec lokci, a dlugosc bramy na trzynascie lokci.
\par 12 Byla tez wystawa przed komorami na jeden lokiec, takze wystawa z drugiej strony na jeden lokiec, a kazda tez komora na szesc lokci z jednej, a na szesc lokci z drugiej strony.
\par 13 Potem wymierzyl brame od dachu komory jednej az do dachu drugiej, szerokosc na dwadziescia i piec lokci, a drzwi byly przeciwko drzwiom.
\par 14 I uczynil podwoje na szescdziesiat lokci, a kazdy podwój u sieni i u bramy zewszad w okolo byl pod jedna miara.
\par 15 A od przodku bramy, gdzie sie wchodzi do przodku przysionka bramy wnetrznej, bylo piecdziesiet lokci.
\par 16 Okna tez pochodziste byly w komorach, i nad podwojami ich wewnatrz bramy zewszad w okolo, takze tez i w przysionkach, a na oknach zewszad w okolo wewnatrz, i na podwojach byly palmy.
\par 17 Potem mie przywiódl do sieni zewnetrznej, a oto komory i tlo uczynione bylo w sieni wszedy w okolo, a trzydziesci komór bylo na onem tle.
\par 18 A to tlo bylo po stronach bram, jako dlugie byly bramy, a toc bylo tlo nizsze.
\par 19 Wymierzyl takze szerokosc od przodku bramy nizszej az do przodku sieni wewnetrznej z dworu na sto lokci ku wschodowi i ku pólnocy.
\par 20 Brame tez, która byla ku pólnocy przy sieni zewnetrznej, wymierzy wdluz i wszerz:
\par 21 (A komory jej byly trzy z jednej, a trzy z drugiej strony, a podwoje jej i przysionki jej byly wedlug pomiaru pierwszej bramy;) na piecdziesiat lokci byla dlugosc jej, a szerokosc na dwadziescia i piec lokci.
\par 22 Okna tez jej, i przysionki jej, i palmy jej byly wedlug pomiaru bramy onej, która patrzyla na wschód, a po siedmiu stopniach wstepowano na nia, a przysionki jej byly tuz przed schodami.
\par 23 Takze brama sieni wewnetrznej byla przeciw bramie ku pólnocy i ku wschodowi, a wymierzyl od bramy do bramy sto lokci.
\par 24 Potem mie wywiódl na droge poludniowa, a oto brama byla ku drodze poludniowej, i wymierzyl podwoje jej, i przysionki jej wedlug tejze miary.
\par 25 (A okna jej, i przysionki jej wszedy w okolo byly, takze jako i drugie) na piecdziesiet lokci wdluz a wszerz na dwadziescia i piec lokci,
\par 26 Wschód tez byl do niej o siedmiu stopniach, a przysionki jej byly przed nimi, takze i palmy, jedna z jednej a druga z drugiej strony przy podwojach jej.
\par 27 Rozmierzyl tez brame sieni wewnetrznej ku poludniu, od bramy do bramy ku poludniu sto lokci.
\par 28 Potem mie wwiódl do sieni wewnetrznej przez poludniowa brame, i rozmierzyl one brame poludniowa wedlug tychze miar.
\par 29 A komory jej i podwoje jej i przysionki jej byly wedlug tychze miar, a okna jej i przysionki jej okolo niej zewszad majace na piecdziesiat lokci wdluz a wszerz na dwadziescia i piec lokci.
\par 30 A przysionki zewszad w okolo na dwadziescia i piec lokci wdluz, a wszerz na piecdziesiat lokci.
\par 31 A przysionki jej byly jako sien zewnetrzna, majac palmy na podwojach; wchód tez byl do niej o osmiu stopniach.
\par 32 Wwiódl mie takze do sieni wewnetrznej droga wschodnia, i wymierzyl one brame wedlug onychze miar;
\par 33 Taze komory jej i podwoje jej i przysionki jej wedlug onychze miar, i okna jej i przysionki jej wszedy w okolo; wdluz na piecdziesiat lokci, a wszerz na dwadziescia i piec lokci.
\par 34 Taze przysionki jej przy sieni zewnetrznej, i palmy przy podwojach jej z obu stron; wchód tez byl do niej o osmiu stopniach.
\par 35 Potem mie wwiódl do bramy pólnocnej, i wymierzyl ja wedlug onychze miar.
\par 36 Komory jej, podwoje jej, i przysionki jej i okna jej byly wszedy w okolo wdluz na piecdziesiat lokci, a wszerz na dwadziescia i piec lokci.
\par 37 A podwoje jej przy sieni zewnetrznej, i palmy przy podwojach jej z obu stron, a wchód byl do niej o osmiu stopniach.
\par 38 Byly tez komory i drzwi ich przy podwojach bram, a tam omywano calopalenia.
\par 39 W przysionku tez bramy byly dwa stoly z jednej strony, a dwa stoly z drugiej strony, na których bito calopalenia, i ofiary za grzech, i ofiary za wystepek.
\par 40 Na stronie tez przed wchodem przy drzwiach bramy pólnocnej byly dwa stoly, takze i przy drugiej stronie, która jest u przysionku bramy, byly dwa stoly.
\par 41 Cztery stoly z jednej, a cztery stoly z drugiej strony byly przy stronie bramy; osm bylo wszystkich stolów, na których bito ofiary.
\par 42 A cztery stoly do calopalenia byly z ciosanego kamienia, na póltora lokcia wdluz, a wszerz na póltora lokcia, a wzwyz na jeden lokiec; na których kladziono naczynia, któremi bito calopalenia i inne ofiary.
\par 43 Haki tez wmiesz na jedne dlon w domu wszedy w okolo byly zgotowane, a mieso na stolach dla ofiar.
\par 44 Byly tez zewnatrz przed brama wnetrzna komory spiewaków w sieni wnetrznej, których rzad jeden byl przy stronie bramy pólnocnej, patrzacy na poludnie; drugi rzad byl przy stronie bramy wschodniej, patrzacy na pólnoc.
\par 45 I rzel do mnie: Te komory, które patrza ku drodze poludniowej, sa dla kaplanów straz trzymajacych w domu.
\par 46 A te zas komory, których przodek jest ku drodze pólnocnej, sa dla kaplanów straz trzymajacych kolo oltarza; ci sa synowie Sadokowi, którzy sie przyblizaja z synów Lewiego do Pana, aby mu sluzyli.
\par 47 I wymierzyl te sien na cztery granie, wdluz na sto lokci, a wszerz na sto lokci, a oltarz byl przed domem.
\par 48 Wwiódl mie potem do przysionka domu, i rozmierzyl podwoje przysionka na piec lokci z jednej, a na piec lokci z drugiej strony; szerz zasie bramy byla na trzy lokcie z jednej, a na trzy lokcie z drugiej strony.
\par 49 A dlugosc przysionka byla na dwadziescia lokci, a szerokosc na jedenascie lokci, a po stopniach wchodzono do niego; slupy tez byly przy podwojach, jeden z jednej, a drugi z drugiej strony.

\chapter{41}

\par 1 I wwiódl mie do kosciola, a wymierzyl podwoje, szesc lokci wszerz z jednej strony, a szesc lokci wszerz z drugiej strony, wedlug szerokosci namiotu.
\par 2 A szerokosc drzwi byla na dziesiec lokci, a boki drzwi na piec lokci z jednej, a na piec lokci z drugiej strony; i rozmierzyl dlugosc ich na czterdziesci lokci, a szerokosc na dwadziescia lokci.
\par 3 A wszedlszy wewnatrz wymierzyl podwoje na dwa lokcie, a drzwi na szesc lokci, a szerokosc drzwi na siedm lokci.
\par 4 Wymierzyl tez dlugosc swiatnicy na dwadziescia lokci, a szerokosc jej na dwadziescia lokci w kosciele, i rzekl do mnie: To jest swiatnica swietych.
\par 5 Wymierzyl tez mur domu na szesc lokci, a szerokosc komory na cztery lokcie wszedy okolo domu.
\par 6 A te komory, komora nad komora, byly na trzydziesci i na trzy stopy, a schodzily sie przy murze domu spolem, tak, ze sie komory wszedy w okolo trzymaly, a nie trzymaly sie na murze domu.
\par 7 Bo sie rozszerzal mur w okolo im dalej tem bardziej z wierzchu dla komór, które byly okolo domu, od wierzchu az do dolu zewszad w okolo domu, poniewaz on dom im dalej tem szerszy byl od wierzchu, a tak najnizsze komory rozszerzaly sie ku wierzchow i dla sredniej komory.
\par 8 Widzialem tez przy domu i najwyzsze komory wszedy w okolo, a podlogi onych komór byly wymierzone na cala laske, to jest, na szesc lokci az do konca.
\par 9 Szerokosc tez muru przy komorach z dworu byla na piec lokci, i plac prózny pod komorami, które byly przy domu.
\par 10 A miedzy komorami i komórkami byla szerokosc na dwadziescia lokci okolo domu zewszad.
\par 11 A drzwi do komór byly ku placowi onemu próznemu, i byly drzwi jedne ku drodze pólnocnej, a drugie ku drodze poludniowej, a szerokosc onego próznego placu byla na piec lokci zewszad w okolo.
\par 12 A budowanie, które bylo przed pietrem w kacie drogi zachodniej, bylo na siedmdziesiat lokci na szerz, a mur onego budowania byl na piec lokci wszerz wszedy w okolo, a dlugosc jego na dziewiecdziesiat lokci.
\par 13 Potem rozmierzyl dom, którego dlugosc byla na sto lokci; takze i pietro i budowanie, i mury jego byly wdluz na sto lokci;
\par 14 Takze szerokosc przodku domu i pietra ku wschodowi slonca byla na sto lokci.
\par 15 Wymierzyl tez i dlugosc budowania przed pietrem, które bylo za niem, takze i ganki jego z jednej i z drugiej strony, a bylo tego na sto lokci, takze kosciól wnetrzny i z przysionkami sieni.
\par 16 Progi i okna pochodziste, i ganki, które byly w okolo po trzech stronach ich przeciwko progom, obite byly deskami wszedy w okolo od ziemi az do okien, takze i okna deskami obite byly.
\par 17 Od wierzchu drzwi az do wnetrznej i zewnetrznej strony domu, wszystek mur zewszad w okolo wewnatrz i zewnatrz dobrze wymierzony.
\par 18 Która robota byla uczyniona z Cherubinami, i z palmami, a kazda palma byla miedzy Cherubinem a Cherubinem, a kazdy Cherubin mial dwie twarze;
\par 19 Minowicie twarz ludzka byla naprzeciwko palmy z jednej strony, a twarz lwieca naprzeciwko palmy z drugiej strony; tak uczyniono po wszystkim domu wszedy w okolo.
\par 20 Od ziemi az do wierzchu drzwi byli Cherubinowie i palmy wyryte, taze i na scianie koscielnej.
\par 21 Podwoje koscielne byly na cztery granie, a ksztalt swiatnicy byl jako ksztalt koscielny.
\par 22 Oltarz drewniany byl na trzy lokcie wzwyz, a wdluz na dwa lokcie z weglami swemi; którego dlugosc i sciany jego byly drewniane. Tedy rzekl do mnie: Oto ten jest stól, który stoi przed obliczem Panskiem.
\par 23 A byly dwoiste drzwi u kosciola i u swiatnicy.
\par 24 A dwoiste drzwi we wrotach, to jest, dwoiste drzwi obracajace sie, dwoiste we wrotach jednych i dwoiste we wrotach drugich.
\par 25 A uczyniono na nich, to jest na tych drzwiach koscielnych, Cherubiny i palmy, tak jako je bylo uczyniono na scianach; belki takze drewniane byly przed przedsionkiem z dworu.
\par 26 Takze na oknach pochodzistych byly palmy z obu stron na bokach przysionku, takze i na komorach domu onego i na belkach.

\chapter{42}

\par 1 I wwiódl mie do sieni zewnetrznej droga, która idzie ku pólnocy, i wwiódl mie do onych komórek, które byly przeciwko pietru, a które byly przeciwko budowaniu na pólnocy.
\par 2 Którego dlugosc przy drzwiach na pólnocy byla na wejrzeniu na sto lokci, a szerokosc na piecdziesiat lokci.
\par 3 Przeciwko sieni wewnetrzej, która miala dwadziescia lokci i przeciwko tlu, które bylo w sieni zewnetrznej z dworu, byl ganek przeciwko gankowi trzema rzedami.
\par 4 A przed komórkami byl plac do przechadzki na dziesiec lokci wszerz wewnatrz, scieszka do nich na jednym lokciu, a drzwi ich byly na pólnocy.
\par 5 A komórki najwyzsze byly ciasniejsze, przeto, ze ganki byly szersze niz one, nizeli spodnie i srednie budowania.
\par 6 Bo bylo o trzech pietrach, ale nie mialo slupów, jakie slupy byly w sieniach; przetoz wazsze bylo niz spodnie i niz srednie od ziemi.
\par 7 Ogrodzenia tez, które bylo z dworu przeciwko onym komorom ku sieni zewnetrznej przed komórkami, dlugosc byla na piecdziesiat lokci.
\par 8 Bo dlugosc komórek, które byly w sieni zewnetrzej, byla na piecdziesiat lokci, a przed kosciolem sto lokci.
\par 9 A pod temi komórkami bylo wejscie od wschodu slonca, przez które wchodzono do nich z onej sieni zewnetrznej.
\par 10 Na szerokosci ogrodzenia onej sieni ku wschodowi przed pietrem i przed budowaniem byly komórki.
\par 11 A scieszka przed niemi byla podobna scieszce onych komórek, które byly na pólnocy; a jako byla dlugosc ich, taka tez byla szerokosc ich, a wszystkie wyjscia ich i drzwi ich byly im podobne.
\par 12 A drzwi onych komórek, które byly na poludnie, podobne byly drzwiom na poczatku drogi, drogi, mówie, przed ogrodzeniem prosto na wschód slonca, kedy sie wchodzi do nich.
\par 13 Tedy rzekl do mnie: Komórki na pólnocy, i komórki na poludnie, które sa przed pietrem, sa komórki swiete, gdzie beda jadali kaplani, którzy przystepuja do Pana, przynoszac rzeczy najswietsze; tam klasc beda rzeczy najswietsze, i ofiary sniedne, i ofiary za grzech, i za wystepek; bo to miejsce swiete jest.
\par 14 Tam gdy wnijda kaplani, nie wynijda z swiatnicy do sieni zewnetrznej, ale tam zostawia odzienia swoje, w których sluzyli (bo swiete jest) a obloka sie w insze szaty, gdy beda mieli przystapic do tego, co nalezy ludowi.
\par 15 A gdy odprawil wymiar domu wewnetrznego, wywiódl mie droga ku bramie, która idzie ku wschodowi, i wymierzyl go wszedy w okolo.
\par 16 Wymierzyl strone od wschodu slonca laska pomiaru na piecset lasek laski pomiarowej w okolo.
\par 17 Wymierzyl tez strone pólnocna na piecset lasek laska pomiaru w okolo.
\par 18 Takze strone od poludnia wymierzyl na piecset laska pomiaru.
\par 19 A obróciwszy sie ku stronie zachodniej, wymierzyl ja na piecset lasek laska pomiaru.
\par 20 Na cztery strony wymierzyl to, to jest mur zewszad w okolo, dlugosc na piecset, a szerokosc takze na piecset lasek, aby sie dzielilo swiete miejsce od pospolitego.

\chapter{43}

\par 1 Wiódl mie potem ku bramie, która brama patrzyla ku drodze na wschód slonca.
\par 2 A oto chwala Boga Izraelskiego przychodzila droga od wschodu, a szum jej byl jako szum wód wielkich, a ziemia sie lsnila od chwaly jego.
\par 3 A podobne bylo ono widzenie, którem widzial, cale onemu widzeniu, którem widzial, gdym przychodzil, abym psul miasto; widzenie, mówie, podobne onemu widzeniu, którem widzial u rzeki Chebar, i upadlem na twarz moje.
\par 4 A gdy chwala Panska wchodzila do domu droga bramy, która patrzala ku drodze na wschód slonca,
\par 5 Tedy mie podniósl duch, i wwiódl mie do sieni wewnetrznej, a oto dom pelen byl chwaly Panskiej.
\par 6 I uslyszalem, a oto mówiono do mnie z domu, a on maz stal podle mnie,
\par 7 I mówil do mnie: Synu czlowieczy! miejsce stolicy mojej, i miejsce stóp nóg moich, na którem mieszkac bede w posród synów Izraelskich na wieki; nie splugawi wiecej dom Izraelski imienia swietobliwosci mojej, ani oni, ani królowie ich w szeteczens twy swemi i trupami królów swych, ani wyzynami swemi.
\par 8 Gdy kladli próg swój podle progu mego, a podwoje swoje podle podwoi moich, a sciane miedzy mna i miedzy soba; a tak splugawiali imie swietobliwosci mojej obrzydliwosciami swemi, które czynili, przetozem ich zniszczyl w popedliwosci mojej.
\par 9 Ale teraz niech odrzuca wszeteczenstwo swoje, i trupy królów swoich odemnie, a bede mieszkal w posrodku ich na wieki.
\par 10 Ty, synu czlowieczy! powiedz domowi Izraelskiemu o tym domu, a niech sie wstydza za nieprawosci swoje, i niech sobie rozmierza wizerunek jego.
\par 11 A jezliby sie zawstydzili za wszystko, co czynili, tydy im oznajmij wizerunek domu tego, i wymiar jego, wyjscie jego, i wejscie jego, i wszystkie ksztalty jego, i wszystkie ustawy jego, wszystkie, mówie, ksztalty, i wszystkie prawa jego, a napisz przed oczyma ich, aby przestrzegali wszystkiego ksztaltu jego, i wszystkich ustaw jego, i czynili je.
\par 12 A tenci jest zakon domu tego: Na wierzchu góry wszystko ogrodzenie jego wszedzie w okolo najswietsze jest; oto tenci jest zakon domu tego.
\par 13 A tec sa pomiary oltarza wedlug tychze lokci, a miara lokcia na lokiec i na dlon; podstawek jego na lokiec wzwyz, a wszerz takze na lokiec, a kraniec jego az do kraju jego w okolo byl na piedz jedna; a tac byla wystawa oltarza;
\par 14 To jest od podstawku, który byl przy ziemi, az do przepasania nizszego, dwa lokcie, a szerokosc na jeden lokiec; a od mniejszego przepasania az do przepasania wiekszego cztery lokcie, a szerokosc na lokiec;
\par 15 Ale sam oltarz niech bedzie na cztery lokcie, a z oltarza w góre cztery rogi.
\par 16 A oltarz na dwanascie lokci wdluz, a na dwanascie wszerz czworograniasty po czterech stronach swoich;
\par 17 A przepasanie jego na czternascie lokci wdluz, a na czternascie wszerz, po czterech stronach jego, a kraniec okolo niego na pól lokcia, a podstawek jego na lokiec w okolo, a wschód jego na wschód slonca.
\par 18 I rzekl do mnie: Synu czlowieczy! tak mówi panujacy Pan: Tec sa ustawy okolo oltarza w dzien, w który zbudowany bedzie, aby na nim ofiarowano calopalenia, i krwia na nim kropiono.
\par 19 A kaplanom Lewitom, którzy sa z nasienia Sadokowego, a przystepuja do mnie, mówi panujacy Pan, aby mi sluzyli, dasz cielca mlodego na ofiare za grzech;
\par 20 Wezmiesz tedy ze krwi jego, a wlozysz na cztery rogi jego, i na cztery wegly przepasania, i na kraniec w okolo; a tak go oczyscisz, i poswiecisz go.
\par 21 Potem wezmiesz onego cielca za grzech, a spalisz go na miejscu postanowionem w onym domu zewnatrz przed swiatnica.
\par 22 A wtórego dnia bedziesz ofiarowal kozla z kóz bez wady za grzech, i oczyszcza oltarz, tak jako go cielcem oczyscili.
\par 23 A gdy dokonczysz oczyszczania, bedziesz ofiarowal cielca mlodego bez wady, i barana z trzody bez wady.
\par 24 Które gdy ofiarowac bedziesz przed Panem, wrzuca kaplani na nie soli, i uczynia z nich ofiare calopalenia Panu.
\par 25 Przez siedm dni bedziesz ofiarowal kozla za grzech na kazdy dzien, takze i cielca mlodego, i barana z trzody bez wady ofiarowac beda,
\par 26 Przez siedm dni oczyszczac beda oltarz, i oczyszcza go, a poswieca rece swoje.
\par 27 A wypelniwszy te dni, ósmego dnia napotem sprawowac beda kaplani na oltarzu calopalenia wasze, i spokojne ofiary wasze, i przyjme was laskawie, mówi panujacy Pan.

\chapter{44}

\par 1 Tedy mie zas przywiódl droga ku bramie swiatnicy zewnetrznej, która patrzy na wschód slonca, a ta byla zamkniona.
\par 2 I rzekl do mnie Pan: Ta brama zamkniona bedzie, nie bedzie otworzona, a zaden nie wnijdzie przez nia; bo Pan, Bóg Izraelski, przeszedl przez nia, przetoz bedzie zamkniona.
\par 3 Ksiazeca jest; ksiaze sam bedzie siadal w niej, aby jadl chleb przed obliczem Panskiem; droga przysionku tej bramy wchodzic, a droga jej wychodzic bedzie.
\par 4 I przywiódl mie droga bramy pólnocnej ku przedniej stronie domu; i ujrzalem, a oto napelnila chwala Panska dom Panski, i upadlem na oblicze swoje.
\par 5 A Pan rzekl do mnie: Synu czlowieczy! uwazaj to, a obacz oczyma twemi, i uszyma twemi sluchaj wszystkiego, co Ja mówie tobie o wszystkich ustawach domu Panskiego, i o wszystkich prawach jego; uwazaj, mówie, abys obaczyl wejscie w dom, i wszystkie wyjscia z swiatnicy.
\par 6 A rzecz odpornemu domowi Izraelskiemu: Tak mówi panujacy Pan: Dosyc miejcie na wszystkich obrzydliwosciach waszych, o domie Izraelski!
\par 7 Zescie tu przywodzili cudzoziemców nieobrzezanych na sercu, i nie obrzezanych na ciele, zeby bywali w swiatnicy mojej a splugawili dom mój; wyscie tez ofiarowali chleb mój, tlustosc, i krew, gdy oni lamali przymierze moje mimo wszystkie obrzydliwosc i wasze;
\par 8 A nie trzymaliscie strazy nad swietymi rzeczami mojemi, alescie postawili strózów na strazy mojej w swiatnicy mojej miasto siebie.
\par 9 Tak mówi panujacy Pan: Zaden cudzoziemiec nieobrzezany na sercu i nieobrzezany na ciele nie wnijdzie do swiatnicy mojej ze wszystkich cudzoziemców, którzy sa miedzy synami Izraelskimi.
\par 10 Takze i Lewitowie, którzy sie oddalali odemnie, gdy bladzil Izrael, którzy sie obladzili odemnie za plugawemi balwanami swemi, ci poniosa nieprawosc swoje.
\par 11 Bo beda w swiatnicy mojej za slug w powinnosciach przy bramach domu, i za slug przy domu; onic beda bic ofiary na calopalenie, i ofiary za lud, i oni stac beda przed obliczem ich, aby im sluzyli.
\par 12 Dlatego, ze im sluzyli przed plugawemi balwanami ich, a byli domowi Izraelskiemi przyczyna upadku w nieprwosc, przetozem podniósl reke moje dla nich, mówi panujacy Pan, ze poniosa nieprawosc swoje.
\par 13 A nie przystapia do mnie, aby mi sprawowali urzad kaplanski, ani przystapia do jakich swietych rzeczy moich, albo do najswietszych, ale poniosa pohanbienie swoje i obrzydliwosci swoje, które czynili.
\par 14 Przetoz ich postanowie za strózów obrzedów domu na kazda posluge jego, i na wszystko, co w nim bedzie sprawowane.
\par 15 Lecz kaplani z Lewitów, synowie Sadokowi, którzy trzymali straz nad swiatnica moja, gdy sie obladzili synowie Izraelscy odemnie, ci przstapia do mnie, aby mi sluzyli, i stana przed twarza moja, aby mi ofiarowali tlustosc i krew, mówi panujacy Pan.
\par 16 Ci wchodzic beda do swiatnicy mojej, ci tez przystapia do stolu mego, aby mi sluzyli i straz moje trzymali.
\par 17 A gdy beda mieli wchodzic do bramy sieni wewnetrznej, tedy sie obleka w szaty lniane, a nie wezma na sie nic welnianego, gdy sluzyc beda w bramach sieni wnetrznej, i wewnatrz.
\par 18 Czapki lniane beda mieli na glowie swojej, i ubiory plócienne beda na biodrach ich, a nie beda sie przepasywac niczem, coby pot czynilo.
\par 19 A gdy wychodzic beda do sieni zewnetrznej, do sieni mówie zewnetrznej do ludu, zewleka szaty swe, w których sluzyli, a poloza je w komorach swiatnicy, i obloka sie w inne szaty, a nie beda poswiecali ludu szatami swemi.
\par 20 Glowy tez swojej nie beda golic, ani wlosów zapuszczac, ale je równo przystrzygac beda na glowach swoich.
\par 21 Wina tez nie bedzie pil zaden kaplan, gdy bedzie mial wchodzic do sieni wnetrznej.
\par 22 Wdowy tez i porzuconej nie beda sobie pojmowac za zony; ale panny z nasienia domu Izraelskiego, albo wdowe, któraby pozostala wdowa po kaplanie; pojmowac beda.
\par 23 A ludu mego uczyc beda róznosci miedzy rzecza swieta i nieswieta, takze miedzy nieczystem i czystem niech ich ucza róznosci.
\par 24 A gdy bedzie jaki spór, oni sie niech stawia do rozsadzania, a wedlug sadów moich rozsadza go; praw tez moich i ustaw moich we wszystkie uroczyste swieta moje strzedz beda, a sabaty moje swiecic beda.
\par 25 Do umarlego tez czlowieka nie wnijdzie kaplan, aby sie nie zmazal; chyba do ojca i do matki; i do syna, i do córki, i do brata, i do siostry, któraby jeszcze nie byla za mezem; przy tych moze sie zmazac.
\par 26 A po oczyszczeniu jego, (siedm dni nalicza mu.)
\par 27 W ten dzien, którego wnijdzie do swiatnicy, do sieni wewnetrznej, aby sluzyl w swiatnicy, uczyni ofiare za grzech swój, mówi panujacy Pan.
\par 28 A miasto dziedzictwa ich Jam jest dziedzictwem ich; przetoz osiadlosci nie dawajcie im w Izraelu, Jam jest osiadloscia ich.
\par 29 Ofiare sniedna i ofiare za grzech i za wystepek, to oni jesc beda; takze wszelka rzecz ofiarowana Bogu w Izraelu ich bedzie.
\par 30 Takze tez najprzedniejsze rzeczy wszystkich pierwocin ze wszystkiego, i kazda ofiara podnoszenia wszystkich rzeczy ze wszystkich ofiar waszych kaplanska bedzie; pierwiastki tez ciast waszych dacie kaplanowi, aby wlozyl blogoslawienstwo na dom twój.
\par 31 Zadnego scierwu, i rozszarpanego od zwierza ani z ptastwa ani z bydlat kaplani jesc nie beda.

\chapter{45}

\par 1 A gdy podzielicie losem te ziemie w dziedzictwo, oddacie za ofiare Panu dzial swiety z tej ziemi, wdluz na dwadziescia i piec tysiecy lokci, a wszerz na dziesiec tysiecy; ten dzial bedzie swiety po wszystkich granicach swoich w okolo.
\par 2 Z niego bedzie miejsce swiete na piec set wdluz, i na piec set wszerz, czworograniaste w okolo; a niech ma piecdziesiat lokci wolnego placu w okolo.
\par 3 Z tegoz wymiaru odmierzysz wdluz dwadziescia i piec tysiecy lokci, a wszerz dziesiec tysiecy lokci, aby na nim byla swiatnica, i swiatnica najswietsza.
\par 4 Ten dzial ziemi swiety jest; kaplanom, slugom swiatnicy, nalezec bedzie, którzy przystepuja, aby sluzyli Panu, aby mieli miejsce dla domów, i miejsce swiete dla swiatnicy.
\par 5 A tych dwadziescia i piec tysiecy lokci wdluz, a dziesiec tysiecy wszerz niech bedzie takze Lewitom, którzy sluza w domu onym, w dzierzawe po dwadziescia komórek.
\par 6 A na osadzenie miasta dacie piec tysiecy lokci wszerz, a wdluz dwadziescia i piec tysiecy przeciwko ofierze miejsca swietego; a to bedzie dla wszystkiego domu Izraelskiego.
\par 7 A ksiazeciu dacie z obu stron tej ofiary miejsca swietego, i polozenia miasta przed ofiara miejsca swietego, i przed polozeniem miasta od strony zachodniej dzial ku zachodowi, a od strony wschodniej dzial ku wschodowi, a dlugosc naprzeciwko kazdem u z tych dzialów od granicy zachodniej az do granicy wschodniej.
\par 8 Ten dzial ziemi bedzie mu za osiadlosc w Izraelu, a nie beda wiecej uciskali ksiazeta moi ludu mego; ale wydziela ziemie domowi Izraelskiemu wedlug pokolenia ich.
\par 9 Tak mówi panujacy Pan: Dosyc miejsca na tem, o ksiazeta Izraelscy! Gwaltu i lupiestwa zaniechajcie, sad i sprawiedliwosc czyncie, a odejmijcie obciazenia wasze od ludu mego, mówi panujacy Pan;
\par 10 Wage sprawiedliwa i Efa sprawiedliwe, i Bat sprawiedliwy miec bedziecie.
\par 11 Efa i Bat pod jedna miara niech beda, aby Bat bral w sie dziesiata czesc Chomeru, takze Efa dziesiata czesc Chomeru; wedlug Chomeru jednaka obojga miara bedzie.
\par 12 A sykiel niech ma dwadziescia pieniedzy a dwadziescia syklów, dwadziescia piec syklów, a pietnascie syklów grzywna wam bedzie.
\par 13 A tak bedzie ofiara podnoszenia, która ofiarowac bedziecie szósta czesc Efy z Chomeru pszenicy, takze szósta czesc Efy dacie z Chomeru jeczmienia.
\par 14 Ustawa zas okolo oliwy ta jest: Bat jest miara oliwy; dziesiata czesc Batu dacie z miary Chomeru, dziesieciu Batów; bo dziesiec Batów jest Chomer.
\par 15 Owce tez jedne z trzody dwóch set z obfitych pastwisk Izraelskich na ofiare sniedna, i na calopalenie, i na ofiary spokojne ku oczyszczeniu was, mówi panujacy Pan.
\par 16 Wszystek lud tej ziemi obowiazany bedzie do tej ofiary podnoszenia i z ksiazeciem w Izraelu.
\par 17 Bo ksiaze bedzie powinien dawac calopalenia, i sniedne i mokre ofiary na swieta, i na nowie miesiecy, i na sabaty, i na wszystkie swieta uroczyste domu Izraelskiego; on sprawowac bedzie ofiare za grzech, i sniedna i palona ofiare, i ofiary spokoj ne na oczyszczenie za dom Iazraelski.
\par 18 Tak mówi panujacy Pan: Pierwszego dnia pierwszego miesiaca wezmiesz mlodego cielca bez wady, a oczyscisz swiatnice.
\par 19 Wezmie tez kaplan ze krwi ofiary za grzech, i pomaze podwoje domu, i cztery wegly onego przepasania na oltarzu, i podwoje bramy sieni wnetrznej.
\par 20 Takze tez uczyni siódmego dnia tegoz miesiaca za kazdego, który z omylki i z prostoty zgrzeszyl; tak oczyscicie dom.
\par 21 Pierwszego miesiaca, czternastego dnia tegoz miesiaca, bedziecie miec swieto przejscia, swieto siedm dni, których chleby przasne jedzone beda.
\par 22 I bedzie ksiaze ofiarowal dnia onego za sie, i za wszystek lud onej ziemi cielca na ofiare za grzech.
\par 23 A przez siedm dni onego swieta uroczystego ofiarowac bedzie calopalenie Panu, siedm cielców i siedm baranów bez wady na kazdy dzien, przez siedm dni, a na ofiare za grzech kozla z kóz na kazdy dzien;
\par 24 A ofiare sniedna Efy przy cielcu, i Efe przy baranie, takze oliwy hyn przy Efie.
\par 25 Siódmego miesiaca, dnia pietnastego tegoz miesiaca w swieto takze wlasnie ofiarowac bedzie przez siedm dni, jako ofiare za grzech, tak calopalenie, tak i ofiare sniedna i oliwe.

\chapter{46}

\par 1 Tak mówi panujacy Pan: Brama sieni wnetrznej, która patrzy na wschód slonca, zamkniona bedzie przez szesc dni robotnych; ale bedzie otworzona w dzien sabatu, takze i w dzien nowiu miesiaca bedzie otworzona.
\par 2 I przyjdzie ksiaze droga przysionku bramy zewnatrz, a stanie u podwoi onej bramy, a kaplani sprawowac beda calopalenie jego, i spokojne ofiary jego, a pokloniwszy sie na progu bramy, potem wynijdzie, a brama nie bedzie zamkniona az do wieczora,
\par 3 Aby sie klanial lud onej ziemi u drzwi bramy we dni sabatu i na nowiu miesiaca przed obliczem Panskiem.
\par 4 Ale calopalenie, które bedzie ksiaze sprawowal Panu w dzien sabatu, bedzie szesc baranków zupelnych i baran zupelny;
\par 5 I ofiara sniedna, efa na barana, i na baranki ofiara sniedna wedlug przemozenia reki jego, a oliwy hyn na efa.
\par 6 A na dzien nowiu miesiaca niech bedzie cielec mlody zupelny, i szesc baranków i baran zupelny.
\par 7 Takze niech ofiaruje efa ofiary suchej przy cielcu, i efa przy baranie, a przy barankach wedlug przemozenia reki swojej, a oliwy hyn na efa.
\par 8 A ksiaze wchodzac droga przysionku i bramy pójdzie, i droga jej odejdzie.
\par 9 Ale gdy bedzie lud onej ziemi wchodzil przed oblicznosc Panska na swieta uroczyste, tedy ten, co wnijdzie droga bramy od pólnocy, aby sie klanial, wynijdzie droga bramy poludniowej; a kto wnijdzie droga bramy poludniowej, wynijdzie droga bramy pól nocnej; nie wróci sie droga onej bramy, która wszedl, ale przeciwko niej wynijdzie.
\par 10 A gdy oni wchodzic beda, ksiaze miedzy nimi wchodzic bedzie; a gdy odchodzic beda, odejdzie.
\par 11 Takze na swieta i na uroczyste swieta niech bedzie ofiara sniedna efa na cielca, i efa na barana, a na baranki, co przemoze reka jego, a oliwy hyn na efa.
\par 12 A bedzieli ksiaze ofiarowal ofiare dobrowolna, calopalenie albo spokojne dobrowolne ofiary Panu, tedy mu niech bedzie otworzona brama, która patrzy na wschód slonca, a niech sprawuje calopalenie swoje albo spokojne ofiary swoje, jako ofiare w dzien sabatu; potem odejdzie, i zamkna brame, gdy wynijdzie.
\par 13 Nadto baranka rocznego zupelnego Panu ofiarowac bedzie co dzien na calopalenie; na kazdy poranek baranka ofiarowac bedzie.
\par 14 Takze ofiare sniedna bedzie ofiarowal przy nim na kazdy poranek szósta czesc efy, a oliwy trzecia czesc hynu na skropienie pszennej maki, sniedna mówie ofiare Panu postanowieniem wiecznem ustwicznie.
\par 15 Tak tedy ofiarowac beda baranka i ofiare sniedna i oliwe na kazdy poranek, calopalenie ustwiczne.
\par 16 Tak mówi panujacy Pan: Jezli komu da ksiaze dar z synów swoich, póki dziedzictwem jego jest, synów jego niech bedzie ku osiadlosci i ku dziedzicwu ich.
\par 17 Ale jezli da dar z dziedzictwa swego któremu z slug swoich, tedy bedzie jego az do roku wolnosci, a potem wróci sie na onego ksiecia; a wszakze dziedzictwo jego miec beda synowie jego.
\par 18 Nie bedzie tez nic bral ksiaze z dziedzictwa ludu, gwaltem ich wyrzucajac z osiadlosci ich; ale z osiadlosci swojej da dziedzictwo synom swoim, zeby nie byl rozproszony lud mój, nikt z osiadlosci swojej.
\par 19 Tedy mie wwiódl przez wejscie, które jest przy stronie bramy, do kaplanów, do komórek swietych, które patrzyly na pólnocy, a oto tu bylo miejsce po obu stronach ku zachodowi;
\par 20 I rzekl do mnie: To jest miejsce, gdzie warza kaplani ofiare za wystepek i za grzech, i smaza ofiare sniednia, aby nic nie wynosili do sieni zewnetrznej ku poswiecaniu ludu.
\par 21 Potem mie wywiódl do sieni zewnetrznej, i obwiódl mie po czterech katach sieni, a oto sien byla w kazdym kacie onej sieni.
\par 22 Na czterech weglach onej sieni byly sieni z kominami na czterdziesci lokci wdluz a na trzydziesci lokci wszerz, jednaz miara onych czterech sieni naroznych.
\par 23 A w onych czterech byly kuchnie w okolo, takze ogniska poczynione w onych kuchniach w okolo.
\par 24 I rzekl mi: Te miejsca sa tych, którzy warza, gdzie sludzy domu warza ofiary ludu.

\chapter{47}

\par 1 Potem mie przywiódl ku drzwiom domu, a oto wody wychodzily z pod progu domu na wschód slonca; bo przednia strona domu byla na wschód slonca, a wody one schodzily spodkiem po prawej stronie domu po stronie poludniowej oltarza.
\par 2 Stamtad mie wywiódl droga bramy pólnocnej, i obwiódl mie droga zewnetrzna do bramy zewnetrznej, droga, która patrzy na wschód slonca; a oto wody wynikly po prawej stronie.
\par 3 A gdy wychodzil on maz na wschód slonca, w którego reku byla miara, i wymierzyl tysiac lokci, i przewiódl mie przez wode, przez wode az do kostek;
\par 4 Potem wymierzyl drugi tysiac, a przewiódl mie przez wode, przez wode az do kolan; i zas wymierzyl trzeci tysiac, i przewiódl mie przez wode az do biódr.
\par 5 A gdy zas wymierzyl tysiac, byl potok, któregom nie mógl przebrnac; bo byly wezbraly wody, wody, które trzeba bylo przeplynac, potok, któregom nie mógl przebrnac.
\par 6 Tedy rzekl do mnie: Widzialzes, synu czlowieczy? I wywiódl mie, i obrócil mie na brzeg onego potoku.
\par 7 A gdym sie obrócil, oto na brzegu onego potoku bylo drzewo bardzo wielkie po obu stronach;
\par 8 I rzekl do mnie: Te wody wychodza od Galilei pierwszej, a schodza po równinie, i wchodza w morze; a gdy do morza wpadna, uzdrowione bywaja wody.
\par 9 I stanie sie, ze kazda dusza zyjaca, która sie plaza, gdziekolwiek przyjda potoki, zyc bedzie, i bedzie ryb bardzo wiele, przeto, ze gdy przyjda tam one wody, oczerstwieja, i zyc beda wszedy, kedykolwiek przyjdzie ten potok.
\par 10 Stanie sie i to, ze stana podle niego rybitwi od Engaddy az do zdroju Eglaim, tam beda rozciagac sieci; ryb bedzie bardzo wiele rozmaitego rodzaju, jako ryb morza wielkiego.
\par 11 Blota jego i kaluze jego nie beda uzdrowine, ale soli oddane beda.
\par 12 A nad potokiem wyrosnie na brzegu jego po obu stronach wszelkie drzewo owoce przynoszace, którego lisc nie opada, ani owoc jego ustaje, w miesiacach swoich przynosi pierwociny; bo wody jego z swiatnicy wychodza, przetoz owoc jego jest na pokarm, a liscie jego na lekarstwo.
\par 13 Tak mówi panujacy Pan: Tac jest granica, w której sobie dziedzicznie przywlaszczycie ziemie wedlug dwunastu pokolen Izraelskich; Józefowi sie dostana dwa sznury.
\par 14 Dziedzicznie, mówie, posiadziecie ja, równie jeden jako drugi, o która podnioslem reke moje, ze ja dam ojcom waszym; i przypadnie wam ta ziemia w dziedzictwo.
\par 15 Tac jest tedy granica tej ziemi ku stronie pólnocnej, od morza wielkiego, droga do Hetlonu, kedy wchodza do Sedad.
\par 16 Emat, Berota, Sybraim, które sa miedzy granica Damaszku i miedzy granica Emat, wsi posrednie, które sa przy granicy Hawran.
\par 17 A tak bedzie granica od morza Chatzar Enon, bedzie granica Damaszek, a pólnocna strona na pólnocy, i granica Emat; a toc jest strona pólnocna.
\par 18 A strona wschodnia miedzy Hawran i miedzy Damaszkiem i miedzy Galaad i miedzy ziemia Izraelska przy Jordanie; od tej granicy przy morzu wschodniem mierzyc bedziecie; a toc jest strona wschodnia.
\par 19 A strona poludniowa na poludnie od Tamar az do wód poswarków w Kades, od potoku az do morza wielkiego; a toc jest strona poludniwoa na poludnie.
\par 20 Strona zas zachodnia morze wielkie od granicy az przeciwko kedy sie wchodzi do Emat, tac jest strona zachodnia.
\par 21 A tak rozmierzycie sobie te ziemie, wedlug pokolen Izraelskich.
\par 22 A gdy ja rozmierzycie, bedzie wam w dziedzictwo, i przychodniom, którzyby mieszkali miedzy wami, którzyby splodzili synów miedzy wami, bo wam beda jako tu zrodzeni miedzy synami Izraelskimi; z wami dziedziczyc beda miedzy pokoleniami Izraelskiemi.
\par 23 A w któremkolwiek pokoleniu przychodzien przychodniem bedzie, tam mu dacie dziedzictwo jego, mówi panujacy Pan.

\chapter{48}

\par 1 A tec sa imiona pokolen: W granicach na pólnocna strone podle drogi Hetlon, kedy wchodza do Emat Chatzar Enon, ku granicy Damaszku na pólnocna strone podle Emat, od wschodniej strony az na zachód osadzi sie pokolenie jedno, to jest Dan.
\par 2 A przy granicy Dan, od strony wschodniej az do strony zachodniej, jedno, to jest Aser.
\par 3 A przy granicy Aser, od strony wschodniej az do strony zachodniej, jedno, to jest Neftalim.
\par 4 A przy granicy Neftalim, od strony wschodniej az do strony zachodniej, jedno, to jest Manase.
\par 5 A przy granicy Manase, od strony wschodniej az do strony zachodniej, jedno, to jest Efraim.
\par 6 A przy granicy Efraim, od strony wschodniej az do strony zachodniej, jedno, to jest Ruben.
\par 7 A przy granicy Rubenowej, od strony wschodniej az do strony zachodniej, jedno, to jest Juda.
\par 8 A przy granicy Judy, od strony wschodniej az do strony zachodniej bedzie ofiara, która ofiarowac beda, dwadziescia i piec tysiecy lokci wszerz, a wdluz zarówno z jednym z innych dzialów od strony wschodniej az do strony zachodniej, i bedzie swiatnica w posrodku niego.
\par 9 Ta ofiara, która ofiarowac macie Panu, bedzie wdluz dwadziescia i piec tysiecy lokci, a wszerz dziesiec tysiecy.
\par 10 A tym sie dostanie ta ofiara swieta, to jest kaplanom, na pólnocy dwadziescia i piec tysiecy lokci, a na zachód wszerz dziesiec tysiecy, a na wschód wszerz dziesiec tysiecy, a na poludnie wdluz dwadziescia i piec tysiecy, a swiatnica Panska bedzie w posród niego.
\par 11 To ma byc kazdemu kaplanowi poswieconemu z synów Sadokowych, którzy trzymaja straz moje, którzy nie bladzili, gdy bladzili synowie Izraelscy, jako bladzili inni Lewitowie.
\par 12 I bedzie dzial ich ofiarowany z ofiary onej ziemi, rzecz najswietsza, przy granicy Lewitów.
\par 13 A Lewitów dzial bedzie na przeciwko granicy kaplanskiej dwadziescia i piec tysiecy lokci wdluz, a wszerz dziesiec tysiecy; kazda dlugosc dwadziescia i piec tysiecy, a szerokosc dziesiec tysiecy.
\par 14 I nie beda go sprzedawac, ani frymarczyc, ani przynosic pierwocin ziemi, przeto ze jest poswiecona Panu.
\par 15 A piec tysiecy lokci, które pozostana wszerz przeciwko onym dwudziestu i pieciu tysiecy, bedzie miejsce pospolite dla miasta na mieszkanie i dla przedmiescia, a miasto bedzie w posrodku niego.
\par 16 A tec sa pomiary jego: Strona pólnocna na cztery tysiace i na piec set lokci, takze strona poludniowa na cztery tysiace i na piec set; od strony tez wschodniej cztery tysiace i piec set, a strona zachodnia na cztery tysiace i na piec set.
\par 17 A bedzie przedmiescia miejskiego na pólnocy dwiescie i piecdziesiat lokci; a na poludnie dwiescie i piecdziesiat, takze na wschód slonca dwiescie i piecdziesiat, a na zachód slonca dwiescie i piecdziesiat;
\par 18 A co zbedzie wdluz przeciw ofierze swietej, dziesiec tysiecy lokci na wschód, i dziesiec tysiecy na zachód; a z tego, co bedzie naprzeciw onej ofierze swietej, beda miec dochody ku wychowaniu sludzy miasta.
\par 19 A ci sludzy miasta sluzyc beda miastu ze wszystkich pokolen Izraelskich.
\par 20 Wszystke te ofiare na dwadziescia i piec tysiecy lokci wedlug tych dwudziestu i pieciu tysiecy, czworograniasta ofiarowac bedziecie na ofiare swieta ku osiadlosci miastu.
\par 21 A to, co zostani, ksiazece bedzie z obu stron ofiary swietej, i osiadlosci mejskiej, przed onemi dwudziestu i pieciu tysiacami lokci ofiary az ku granicy wschodniej, i od zachodu przeciwko tymze dwudziestu i pieciu tysiacom lokci, podle granicy zachodniej przciwko tym dzialom, ksiazeciu bedzie; a to bedzie ofiara swieta, a swiatnica domu bedzie w posrodku niego.
\par 22 A od osiadlosci Lewitów i od osiadlosci miejskiej w posród tego, co jest ksiazecego, miedzy granica Judowa i miedzy granica Benijaminowa, to ksiazece bedzie,
\par 23 A ostatnie pokolenia, od strony wschodniej az do strony zachodniej osadzi sie pokolenie jedno, to jest Benjamin.
\par 24 A przy granicy Benjaminowej od strony wschodniej az do strony zachodniej, jedno, to jest Symeon.
\par 25 A przy granicy Symeonowej, od strony wschodniej az do strony zachodniej, jedno, to jest Isaschar.
\par 26 A przy granicy Isascharowej, od strony wschodniej az do strony zachodniej, jedno to jest Zabulon.
\par 27 A przy granicy Zabulonowej, od strony wschodniej az do strony zachodniej, jedno, to jest Gad.
\par 28 A przy granicy Gadowej, ku stronie poludniowej na poludnie, tu bedzie granica od Tamar az do wód poswarku w Kades, ku potokowi przy morzu wielkiem.
\par 29 Toc jest ona ziemia, która losem rozdzielicie od potoku wedlug pokolen Izraelskich, i tec dzialy ich, mówi panujacy Pan.
\par 30 Tec tez sa granice miejskie od strony pólnocnej cztery tysiace i piec set lokci miary.
\par 31 A bramy miasta wedlug imion pokolen Izraelskich; trzy bramy na pólnocy, brama Rubenowa jedna, brama Judowa jedna, brama Lewiego jedna.
\par 32 A od strony wschodniej cztery tysiace i piec set, a bramy trzy, to jest brama Józefowa jedna, brama Benjaminowa jedna, bramam Danowa jedna.
\par 33 Od strony tez poludniowej cztery tysiace i piec set lokci miary, i trzy bramy: brama Symeonowa jedna, brama Isascharowa jedna, brama Zabulonowa jedna.
\par 34 Od strony zachodniej cztery tysiacy i piec set, bramy ich trzy: Brama Gadowa jedna, brama Aserowa jedna, brama Neftalimowa jedna.
\par 35 W okrag osmnascie tysiecy lokci; a imie miasta ode dnia tego bedzie: Pan tam mieszka.


\end{document}