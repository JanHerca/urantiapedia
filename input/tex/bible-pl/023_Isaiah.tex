\begin{document}

\title{Izajasza}


\chapter{1}

\par 1 Widzenie Izajasza, syna Amosowego, które widzial nad Juda i nad Jeruzalemem, za dni Ozeasza, Joatama, Achaza, i Ezechijasza, królów Judzkich.
\par 2 Sluchajcie niebiosa, a ty ziemio przyjmij w uszy swe! Albowiem Pan mówi: Synówem wychowal i wywyzszyl; ale oni odstapili odemnie.
\par 3 Zna wól gospodarza swego, i osiel zlób pana swego; ale Izrael mie nie zna, lud mój nie zrozumiewa.
\par 4 Biada narodowi grzesznemu, ludowi obciazonemu nieprawoscia, nasieniu zlosliwych; synom skazonym! Opuscili Pana, do gniewu pobudzili swietego Izraelskiego, odwrócili sie nazad.
\par 5 Przeczze tem wiecej przyczyniacie przestepstwa, im wiecej was bija? Wszystka glowa chora, i wszystko serce mdle.
\par 6 Od stopy nogi az do wierzchu glowy niemasz na nim nic calego; rana i sinosc, i rany zagnile nie sa wycisnione, ani zawiazane, ani olejkiem odmiekczone.
\par 7 Ziemia wasza spustoszona, miasta wasze popalone ogniem. Ziemie wasze cudzoziemcy przed wami pozeraja i pustosza, jako zwykli cudzoziemcy.
\par 8 I zostala córka Syonska jako chlodnik na winnicy, jako budka w ogrodzie ogórczanym, i jako miasto zburzone.
\par 9 By nam byl Pan zastepów nie zostawil trochy ostatków, bylibysmy jako Sodoma, stalibysmy sie byli Gomorze podobnymi.
\par 10 Sluchajcie slowa Panskiego, ksiazeta Sodemscy! przyjmujcie w uszy zakon Boga naszego, ludzie Gomorscy!
\par 11 Cóz mi po mnóstwie ofiar waszych? mówi Pan. Juzem syty calopalenia baranów, i loju tlustego bydla; a krwi cielców, i baranków, i kozlów nie pragne.
\par 12 Gdy przychodzicie, abyscie sie okazywali przed twarza moja, któz tego zadal z rak waszych, abyscie deptali sieni moje?
\par 13 Nie ofiarujciez wiecej ofiary daremnej. Kadzenie jest mi obrzydloscia; nowiu miesiaca i sabatu, gdy zwolywacie zgromadzenia, nie moge scierpiec (bo nieprawoscia jest) ani dnia zapowiedzianego.
\par 14 Nowych miesiecy waszych, i uroczystych swiat waszych nienawidzi dusza moja; staly mi sie ciezarem; upracowalem sie noszac je.
\par 15 Przetoz gdy wyciagniecie rece wasze, skryje oczy moje przed wami; a gdy rozmnozycie modlitwe, nie wyslucham; bo rece wasze krwi sa pelne.
\par 16 Omyjcie sie, czystymi badzcie, odejmijcie zlosc uczynków waszych od oczów moich; przestancie zle czynic.
\par 17 Uczcie sie dobrze czynic; szukajcie sadu, podzwignijcie ucisnionego, sad czyncie sierocie, ujmujcie sie o krzywde wdowy.
\par 18 Przyjdzciez teraz, a rozpierajmy sie z soba, mówi Pan: Chocby byly grzechy wasze jako szarlat, jako snieg zbieleja; chocby byly czerwone jako karmazyn, jako welna biale beda.
\par 19 Bedziecieli powolni, a posluchacie mie, dóbr ziemi pozywac bedziecie.
\par 20 Lecz jezli nie bedziecie poslusznymi, ale odpornymi, od miecza pozarci bedziecie; bo usta Panskie mówily.
\par 21 Jakoc sie stalo nierzadnica to miasto wierne, pelne sadu? Sprawiedliwosc mieszkala w niem; lecz teraz mezobójcy.
\par 22 Srebro twoje obrócilo sie w zuzel; wino twoje pomieszalo sie z woda.
\par 23 Ksiazeta twoi sa uporni, i towarzysze zlodziei; kazdy z nich miluje dary, a jada za nagroda; sierocie nie czynia sprawiedliwosci, a sprawa wdowy nie przychodzi przed nich.
\par 24 Przetoz mówi Pan, Pan zastepów, mozny Izraelski: Oto uciesze sie nad nieprzyjaciólmi moimi, a pomszcze sie nad przeciwnikami swymi.
\par 25 I obróce reke moje na cie, a wypale az do czysta zuzelice twoje, i odpedze wszystke cene twoje.
\par 26 A przywróce sedziów twoich, jako przedtem byli, i radców twoich, jako na poczatku. Potem cie nazywac beda miastem sprawiedliwosci, miastem wiernem.
\par 27 Syon w sadzie okupione bedzie, a ci, co sie do niego nawróca, w sprawiedliwosci.
\par 28 Ale przewrotnicy i grzesznicy wespól starci beda, a ci, co opuscili Pana, zniszczeja.
\par 29 Albowiem zawstydzeni bedziecie dla gajów, którychescie pozadali; i pohanbieni dla ogrodów, którescie sobie obrali.
\par 30 Gdy sie staniecie jako dab, z którego liscie opadly, a jako ogród, w którym wody niemasz.
\par 31 I bedzie mocarz jako zgrzebia, a ten, który go uczynil, jako iskra; i zapala sie oboje pospolu, a nie bedzie, ktoby zagasil.

\chapter{2}

\par 1 Slowo, które widzial Izajasz, syn Amosowy, nad Juda i nad Jeruzalemem.
\par 2 I stanie sie w ostateczne dni, ze bedzie przygotowana góra domu Panskiego na wierzchu gór, i wywyzszy sie nad pagórkami, a zbieza sie do niej wszystkie narody.
\par 3 I pójdzie wiele ludzi, mówiac: Pójdzcie a wstapmy na góre Panska, do domu Boga Jakóbowego, a bedzie nas uczyl dróg swoich, i bedziemy chodzili scieszkami jego; albowiem z Syonu wyjdzie zakon, a slowo Panskie z Jeruzalemu.
\par 4 I bedzie sadzil miedzy narodami, a bedzie karal wiele ludzi.I przekuja miecze swe na lemiesze, a wlócznie swe na sierpy; nie podniesie naród przeciw narodowi miecza, ani sie beda cwiczyc do bitwy.
\par 5 Domie Jakóbowy! pójdzcie, a chodzmy w swiatlosci Panskiej.
\par 6 Ales ty opuscil lud swój, dom Jakóbowy! gdyz sa pelni obrzydliwosci narodów wschodnich, i sa wieszczkami jako Filistynowie, a w synach cudzych sie kochali.
\par 7 I napelniona jest ziemia ich srebrem i zlotem, a konca niemasz skarbom ich.
\par 8 Napelniona jest ziemia ich konmi, a konca niemasz wozom ich. Napelniona tez jest ziemia ich balwanami, robocie rak swoich klaniaja sie, które poczynily palce ich.
\par 9 I klania sie pospolity czlowiek, a uniza sie i zacny maz; przetoz nie odpuszczaj im.
\par 10 Wnijdz w skale, a skryj sie w prochu przed strachem Panskim, i przed chwala majestatu jego.
\par 11 Oczy wyniosle czlowiecze znizone beda, a wysokosc ludzka nachylona bedzie; ale sam Pan wywyzszony bedzie dnia onego.
\par 12 Albowiem dzien Pana zastepów przyjdzie na wszelkiego pysznego i wynioslego, i na kazdego wywyzszonego, ze bedzie ponizony;
\par 13 I na wszystkie cedry Libanskie wysokie a podniosle, i na wszystkie deby Basanskie;
\par 14 I na wszystkie góry wysokie, i na wszystkie pagórki wyniosle;
\par 15 I na kazda wieze wysoka, i na kazdy mur obronny;
\par 16 I na wszystkie okrety morskie, i na wszystkie malowania rozkoszne.
\par 17 I bedzie nachylona wynioslosc czlowiecza, a wywyzszenie ludzkie znizone bedzie; ale sam Pan wywyzszony bedzie dnia onego.
\par 18 Lecz balwany ich do szczetu pokruszone beda.
\par 19 Tedy wnijda do jaskin skalnych, i do jam podziemnych przed strachem Panskim, i przed chwala majestatu jego, gdy powstanie, aby ziemie potarl.
\par 20 Dnia onego wrzuci czlowiek balwany swe srebrne i balwany swe zlote, które mu naczyniono, aby sie im klanial, w dziury kretów i nietoperzy.
\par 21 I wnijdzie w rozpadliny skalne, i na wierzcholki opok przed strachem Panskim, i przed chwala majestatu jego, gdy powstanie, aby potarl ziemie.
\par 22 Przestanciez ufac w czlowieku, którego dech jest w nozdrzach jego; bo za cóz on ma byc poczytany?

\chapter{3}

\par 1 Albowiem oto panujacy Pan zastepów odejmie od Jeruzalemu i od Judy laske, i podpore, wszelaka podpore chleba, i wszelaka podpore wody.
\par 2 Mocarza i meza walecznego, i sedziego, i proroka, i medrca, i starca;
\par 3 Rotmistrza nad piecdziesiat, a meza powaznego, i radce, i madrego rzemieslnika, i krasomówce.
\par 4 I dam im dzieci za ksiazeta; dzieci mówie panowac beda nad nimi.
\par 5 I bedzie uciskal miedzy ludem jeden drugiego, i blizni blizniego swego: powstanie dziecie przeciwko starcowi, a podly przeciwko zacnemu.
\par 6 A gdy sie uchwyci kazdy brata swego z domu ojca swego, i rzecze: Masz odzienie, badzze ksiazeciem naszym, a upadek ten zatrzymaj reka swa:
\par 7 Tedy on przysieze dnia onego, mówiac: Nie bede zawiazywal tych ran: albowiem w domu moim niemasz chleba, ani odzienia; nie stanowciez mie ksiazeciem nad ludem.
\par 8 Bo Jeruzalem upada, a Juda sie wali, dlatego, ze jezyk ich, i sprawy ich sa przeciwko Panu, pobudzajac do gniewu oczy majestatu jego.
\par 9 Postawa oblicza ich swiadczy przeciwko nim; grzech swój, jako Sodomczycy, opowiadaja, a nie taja go. Biada duszy ich! albowiem sami na sie zle przywodza.
\par 10 Powiedzcie sprawiedliwemu, ze mu dobrze bedzie; bo owocu uczynków swoich pozywac bedzie.
\par 11 Ale biada niepoboznemu! zle mu bedzie; albowiem odplata rak jego dana mu bedzie.
\par 12 Ksiazeta ludu mego sa dziecmi, a niewiasty panuja nad nimi. O ludu mój! ci, którzy cie wodza, zwodza cie, a droge sciezek twoich ukrywaja.
\par 13 Powstal Pan, aby sadzil, stoi, aby sadzil lud.
\par 14 Pan przyjdzie na sad przeciwko starszym ludu swego, i przeciwko ksiazetom ich, a rzecze: Wyscie spustoszyli winnice moje, zdzierstwo z ubogiego w domach waszych.
\par 15 Przeczze trzecie lud mój, a oblicza ubogich bijecie? mówi Pan, Pan zastepów.
\par 16 I rzekl Pan: Iz sie wynosza córki Syonskie, a chodza szyje wyciagnawszy, i mrugajac oczyma przechodza sie, a drobno postepujac nogami swemi szelest czynia:
\par 17 Przetoz oblysi Pan wierzch glowy córek Syonskich, a Pan sromote ich obnazy.
\par 18 Dnia onego odejmie Pan ochedóstwo podwiazek, takze czepce i zawieszenia,
\par 19 Pizmowe jablka, i manele, i zatyczki,
\par 20 Bieretki, i zapony, i bindy, i przedniczki, i nausznice;
\par 21 Pierscionki, i naczelniki,
\par 22 Odmienne szaty, i plaszczyki, i podwiki, i wacki,
\par 23 Zwierciadla, i rantuszki, i tkanki, i letniki.
\par 24 I bedzie miasto wonnych rzeczy smród, a miasto pasa rozpasanie, a miasto utrefionych wlosów lysina, a miasto szerokiej szaty opasanie worem, a miasto pieknosci ogorzelina.
\par 25 Mezowie twoi od miecza upadna, a mocarze twoi w bitwie.
\par 26 I zasmuca sie, a plakac beda bramy jego, a spustoszony na ziemi siedziec bedzie.

\chapter{4}

\par 1 A w on dzien uchwyci sie siedm niewiast meza jednego, mówiac: Chleb swój jesc bedziemy, i odzieniem swem przyodziewac sie bedziemy; tylko niech nas zowia od imienia twego, a odejmij pohanbienie nasze.
\par 2 W on dzien latorosl Panska zacna i slawna bedzie, a owoc ziemi bujny i pozorny tym, którzy zachowani beda z Izraela.
\par 3 I stanie sie, ze kto zostanie na Syonie, i który zostawiony bedzie w Jeruzalemie, swietym slynac bedzie, kazdy, który jest napisany do zywota w Jeruzalemie.
\par 4 Gdy omyje Pan plugastwo córek Syonskich, a krew Jeruzalemska oplócze z niego w duchu sadu, i w duchu zapalenia.
\par 5 I stworzy Pan nad kazdem miejscem góry Syonskiej, i nad kazdem zgromadzeniem jej oblok we dnie, a dym i jasnosc palajacego ognia w nocy: bo nad wszystka slawa bedzie ochrona.
\par 6 A bedzie namiotem na zaslone we dnie od goraca, a na ucieczke i ukrycie przede dzdzem i powodzia.

\chapter{5}

\par 1 Zaspiewam teraz milemu memu piosnke milego mego o winnicy jego. Winnice ma mily mój na pagórku urodzajnym;
\par 2 Która ogrodzil, i wybral z niej kamienie, a nasadzil ja macicami wybornemi, i zbudowal wieze w posrodku niej, takze i prase postawil w niej, a czekal, aby wydala grona; ale ona zrodzila plonne wino.
\par 3 A tak, obywatele Jeruzalemscy i mezowie Judzcy! prosze, rozsadzcie teraz miedzy mna i miedzy winnica moja.
\par 4 Cóz dalej czynic bylo winnicy mojej, czegobym jej nie uczynil? Gdym rzekal, aby wydala grona, czemuz zrodzila plonne wino?
\par 5 A przetoz oznajmie wam, co ja uczynie winnicy mojej: Rozbiore plot jej, a bedzie spustoszona; rozwale ogrodzenie jej, a bedzie podeptana.
\par 6 I uczynie ja pusta; nie bedzie obrzezywana, ani okopywana, ale porosnie ostem i cierniem; oblokom tez przykaze, aby na nia wiecej dzdzu nie spuszczaly.
\par 7 Winnica zaiste Pana zastepów jest dom Izraelski, a maz Judzki szczepieniem jego rozkosznem. Oczekiwal sadu, a oto ucisnienie; oczekiwal sprawiedliwosci, a oto krzyk.
\par 8 Biada wam, którzy przylaczacie dom do domu, a role do roli przyczyniacie, tak, ze miejsca innym nie staje, jakobyscie tylko sami mieszkac mieli na ziemi!
\par 9 Pan zastepów rzekl w uszy moje: Zaiste wiele domów spustoszeje, a wielkie i piekne domy beda bez obywatela.
\par 10 Do tego dziesiec stajan winnicy przyniosa jedne baryle wina, a jeden chomer nasienia wyda efa.
\par 11 Biada tym, którzy rano wstawajac chodza za pijanstwem, a trwaja na niem do wieczora, az ich wino rozpali!
\par 12 A cytra, i lutnia, beben i piszczalka, i wino bywa na biesiadach ich; ale na sprawy Panskie nie patrza, a na uczynki rak jego nie ogladaja sie.
\par 13 Przetoz w niewole pójdzie lud mój, iz nie ma umiejetnosci; a zacni jego beda glodnymi, i pospólstwo jego wyschnie od pragnienia.
\par 14 Dlatego rozszerzylo pieklo gardlo swoje, a rozdarlo nad miare paszczeke swoje, i zstapia do niego szlachta i pospólstwo jego, i zgielk jego, i ci, którzy sie wesela w niem.
\par 15 A tak bedzie nachylony czlowiek, a zacny maz ponizony bedzie, i oczy wynioslych znizone beda.
\par 16 Ale Pan zastepów wywyzszony bedzie w sadzie, a Bóg swiety ukaze sie swietym w sprawiedliwosci.
\par 17 I beda sie pasc baranki wedlug zwyczaju swego, a przychodniowie pustyn bogaczów pozywac beda.
\par 18 Biada tym, którzy ciagna nieprawosc powrozami marnosci, a grzech jako powrozem wozowym!
\par 19 Którzy mówia: Niech sie pospieszy, a niechaj nie omieszkuje sprawa jego, abysmy ja widzieli; niech sie przyblizy i przyjdzie rada swietego Izraelskiego, zebysmy sie dowiedzieli.
\par 20 Biada tym, którzy nazywaja zle dobrem a dobre zlem; którzy pokladaja ciemnosc za swietlosc, a swiatlosc za ciemnosc; którzy pokladaja gorzkosc za slodkosc, a slodkosc za gorzkosc!
\par 21 Biada tym, którzy sie sobie zdadza byc madrymi, a sami u siebie roztropnymi!
\par 22 Biada tym, którzy sa mocni na picie wina, a mezom duzym ku nalewaniu napoju mocnego!
\par 23 Którzy usprawiedliwiaja niezboznego za podarki, a sprawiedliwosc sprawiedliwych odejmuja od nich!
\par 24 Przetoz jako plomien ogniowy pozera parzdzieze, i jako plomien plewy trawi: tak korzen ich bedzie jako zgnilizna, a kwiat ich jako proch ku górze pójdzie; albowiem odrzucili zakon Pana zastepów, a wyrokiem swietego Izraelskiego pogardzili.
\par 25 Dlatego sie zapalila popedliwosc Panska przeciw ludowi swemu, a wyciagnawszy nan reke swa porazil go, tak ze sie zatrzasnely góry, i byly trupy ich jako gnój po ulicach. W tem jednak wszystkiem nie odwrócila sie zapalczywosc jego, ale jeszcze rek a jego jest wyciagniona.
\par 26 Bo podniesie choragiew do narodu dalekiego, a zaswisnie nan od konczyn ziemi, a oto rychlo i predko przyjdzie.
\par 27 Zadnego spracowanego i upadajacego nie bedzie miedzy nimi; nie bedzie drzemiacego ani spiacego, ani sie rozepnie pas na biodrach jego, ani sie rozerwie rzemyk u trzewików jego.
\par 28 Strzaly jego ostre, i wszystkie luki jego naciagnione; kopyta koni jego jako krzemien poczytane beda, a kola jego jako burza.
\par 29 Ryk jego jako lwi; bedzie ryczal jako szczenieta lwie; bedzie zgrzytal, i porwie lup, i uciecze z nim, a nie bedzie ktoby go wydarl.
\par 30 I zaszumi nad nim dnia onego jako szum morski. Tedy spojrzymy na ziemie, a oto ciemnosc i ucisk; bo i swiatlo zacmi sie przy wytraceniu jego.

\chapter{6}

\par 1 Roku, którego umarl król Uzyjasz, widzialem Pana, siedzacego na stolicy wysokiej i wynioslej, a podolek jego napelnial kosciól.
\par 2 Serafinowie stali nad nim, szesc skrzydel mial kazdy z nich; dwoma zakrywal twarz swoje, a dwoma przykrywal nogi swoje, a dwoma latal.
\par 3 I wolal jeden do drugiego, mówiac: Swiety, swiety, swiety, Pan zastepów; pelna jest wszystka ziemia chwaly jego.
\par 4 I poruszyly sie podwoje u drzwi od glosu wolajacego, a dom pelny byl dymu.
\par 5 I rzeklem: Biada mnie! juzem zginal, przeto, zem czlowiek splugawionych warg, a mieszkam w posrodku ludu, który ma splugawione wargi; a iz króla, Pana zastepów, widzialy oczy moje.
\par 6 I przylecial do mnie jeden z Serafinów, majac w rece swej wegiel rozpalony, który kleszczykami wzial z oltarza;
\par 7 I dotknal sie ust moich, a rzekl: Oto sie dotknal ten wegiel warg twoich, a odejdzie nieprawosc twoja, a grzech twój zgladzony bedzie.
\par 8 Potemem slyszal glos Pana mówiacego: Kogoz posle? a kto nam pójdzie? Tedym rzekl: Otom ja, poslij mie.
\par 9 A on rzekl: Idz, a powiedz ludowi temu: Sluchajcie sluchajac, a nie rozumijcie, a widzac patrzajcie, a nie poznawajcie.
\par 10 Zatwardz serce ludu tego, a uszy jego obciaz, i oczy jego zawrzyj, aby nie widzial oczyma swemi, a uszyma swemi nie slyszal, i sercem swem nie zrozumial, a nie nawrócil sie, i nie byl uzdrowion.
\par 11 A gdym rzekl: Dokadze Panie? A on rzekl: Dokad nie spustoszeja miasta, tak aby nie bylo obywatela; i domy, aby nie bylo w nich czlowieka, a ziemia do szczetu nie spustoszeje;
\par 12 Dokad Pan daleko nie zapedzi wszelkiego czlowieka, a nie bedzie doskonale spustoszenie w posród ziemi;
\par 13 Dokad jeszcze na nia dziesiata zguba nie przyjdzie, a dopiero skazona bedzie, A wszakze jako one deby, które sa przy bramie Zallechet podpora, tak nasienie swiete jest podpora jej.

\chapter{7}

\par 1 I stalo sie za dni Achaza, syna Joatamowego, syna Uzyjasza, króla Judzkiego, ze przyciagnal Rasyn, król Syryjski, i Facejasz, syn Romelijasza, króla Izraelskiego, pod Jeruzalem, aby walczyl przeciw niemu: ale go nie mógl dobyc.
\par 2 I oznajmiono domowi Dawidowemu, mówiac: Zmówila sie Syryja z Efraimem. Tedy sie poruszylo serce jego, i serce ludu jego, jako sie poruszaja drzewa lesne od wiatru.
\par 3 Tedy rzekl Pan do Izajasza: Wyjdz teraz przeciw Achazowi, ty, i Sear Jasub, syn twój, na koniec rur sadzawki wyzszej, na droge pola farbierzowego;
\par 4 A powiedz mu: Patrz, abys sie nie frasowal; nie bój sie, a serce twoje niechaj sie nie leka tych dwóch ostatków glowien kurzacych sie, to jest, zapalczywosci gniewu Rasyna z Syryjczykami, i syna Romelijaszowego,
\par 5 Przeto, ze zla rade uradzili przeciw tobie Syryjczyk, Efraim, i syn Romelijaszowy, mówiac:
\par 6 Ciagnijmy przeciwko ziemi Judzkiej, a utrapmy ja leza, i oderwijmy ja do siebie, a postanówmy króla w posród niej, syna Tabealowego.
\par 7 Tak mówi Pan Panujacy: Nie stanie sie, i nie bedzie to.
\par 8 Albowiem glowa Syryi jest Damaszek, a glowa Damaszku Rasyn; a po szescdziesieciu i pieciu latach bedzie potarty Efraim, tak, iz wiecej ludem nie bedzie.
\par 9 Miedzy tem glowa Efraimowa bedzie Samaryja, a glowa Samaryi syn Romelijaszowy. Jezli nie uwierzycie, pewnie sie nie ostoicie.
\par 10 Nadto jeszcze rzekl Pan do Achaza, mówiac:
\par 11 Zadaj sobie znaku od Pana, Boga twego, badz na dole nisko, badz wysoko w górze.
\par 12 Tedy odpowiedzial Achaz: Nie bede zadal, ani bede kusil Pana.
\par 13 A prorok rzekl: Sluchaj teraz, domie Dawidowy! Maloz sie wam zda, uprzykrzac sie ludziom, ze sie uprzykrzacie i Bogu mojemu?
\par 14 Przetoz wam sam Pan znak da. Oto panna pocznie i porodzi syna, a nazwie imie jego Immanuel.
\par 15 Maslo i miód jesc bedzie, azby umial odrzucac zle, a obierac dobre. -
\par 16 Owszem, pierwej niz bedzie umialo to dziecie odrzucac zle i obierac dobre, ziemia, która sie ty brzydzisz, opuszczona bedzie od dwóch królów swoich.
\par 17 Ale na cie Pan przywiedzie i na lud twój, i na dom ojca twego, dni, jakich nie bylo ode dnia, którego odstapil Efraim od Judy, a to przez króla Assyryjskiego.
\par 18 Albowiem stanie sie dnia onego, ze zaswisnie Pan na muchy, które sa na koncu rzek Egipskich, i na pszczoly, które sa w ziemi Assyryjskiej.
\par 19 I przyjda a usiada wszystkie w dolinach pustych, i w rozpadlinach skalnych, i na wszystkich drzewach urodzajnych.
\par 20 Dnia onego ogoli Pan brzytwa najeta przez tych, którzy sa za rzeka, to jest (przez króla Assyryjskiego) glowe, i wlosy na nogach, takze i brode wszczat ogoli.
\par 21 I stanie sie dnia onego, ze ledwie czlowiek zywo krówke, albo dwie owce zachowa.
\par 22 A wszakze dla obfitosci mleka, którego nadoi, bedzie jadl maslo; maslo zaiste i miód bedzie jadl, ktokolwiek pozostanie w ziemi.
\par 23 Stanie sie tez onegoz dnia, iz kazde miejsce, gdzie bylo tysiac winnych macic za tysiac srebrników, ostem i cierniem porosnie.
\par 24 Tedy z strzalami i z lukiem tam chodzic beda; bo ostem i cierniem zarosnie wszystka ziemia.
\par 25 Na wszystkie tez góry, które motyka kopane byc moga, nie przyjdzie strach ostu i ciernia; ale beda na pastwisko wolom, i na podeptanie owcom.

\chapter{8}

\par 1 I rzekl Pan do mnie: Wezmij sobie ksiegi wielkie, a napisz na nich pismem czlowieczem: Pospiesz sie do lupu, pokwap sie do korzysci.
\par 2 Tedym wzial sobie za swiadków wiernych Uryjasza kaplana, i Zacharyjasza, syna Jeberechyjaszowego.
\par 3 Wtemem przystapil do prorokini, która poczawszy porodzila syna. I rzekl Pan do mnie: Nazów imie jego: Pospiesz sie do lupu, pokwap sie do korzysci.
\par 4 Albowiem nizeli bedzie umialo to dziecie wolac: Ojcze mój i matko moja, lud króla Assyryjskiego pobierze bogactwa Damaszku, i lupy Samaryi.
\par 5 Nadto rzekl jeszcze Pan do mnie, mówiac:
\par 6 Poniewaz wzgardzil lud ten wody Syloe, które cicho plyna, a waseli sie z Rasyna, i syna Romelijaszowego:
\par 7 Przetoz oto Pan przywiedzie na nich wody rzeki gwaltownej i wielkiej, to jest króla Assyryjskiego, i wszystke slawe jego, tak, ze wystapi ze wszystkich strumieni swoich, a wyleje ze wszystkich brzegów swoich.
\par 8 Pociecze i przez ziemie Judzka, wyleje a rozejdzie sie, az do szyi wzbierze; a rozszerzone skrzydla jego napelnia szerokosc ziemi twojej, o Immanuelu!
\par 9 Zbierajcie sie narody, wszakze potlumione bedziecie. Przyjmujcie w uszy wszyscy w dalekiej ziemi; przepaszcie sie, wszakze potlumieni bedziecie; przepaszcie sie, wszakze potlumieni bedziecie.
\par 10 Wnijdzcie w rade, a bedzie rozerwana; namówcie sie, a nie ostoi sie; bo Bóg z nami.
\par 11 Tak bowiem Pan rzekl do mnie, ujawszy mie za reke, i dal mi przestroge, zebym nie chodzil droga ludu tego, mówiac:
\par 12 Nie mówcie: Sprzysiezenie. Kiedykolwiek ten lud mówi: Sprzysiezenie, nie strachajcie sie jako oni, ani sie lekajcie.
\par 13 Pana zastepów samego poswiecajcie; a on niech bedzie bojaznia wasza, i on strachem waszym.
\par 14 A bedzie wam poswieceniem; ale kamieniem obrazenia i opoka otracenia obydwom domom Izraelskim, sidlem i siecia obywatelom Jeruzalemskim.
\par 15 I otraci sie wielu ich o nie, upadna i skruszeni beda, usidla sie a pojmani beda.
\par 16 Zawiaz to swiadectwo, zapieczetuj zakon miedzy uczniami moimi.
\par 17 Tedy bede oczekiwal Pana, który skryl oblicze swoje od domu Jakóbowego, i poczekam go.
\par 18 Oto ja i dzieci, które mi dal Pan, sa na znaki i na cuda w Izraelu, od Pana zastepów, który mieszka na górze Syon.
\par 19 A tak jezliby wam rzekli: Dowiadujcie sie od czarowników i od wieszczków, którzy szepca i markoca, rzeczcie: Izali sie nie ma dowiadywac lud u Boga swego? azaz umarlych miasto zywych radzic sie ma?
\par 20 Do zakonu raczej i do swiadectwa; ale jezli nie chca, niechze mówia wedlug slowa tego, w którem niemasz zadnej zorzy.
\par 21 Dlaczego kazdy z nich utrapiony i zglodnialy tulacby sie musial; a bedac zglodnialym, sam w sobie gniewac sie bedzie, i zlorzeczyc królowi swemu, i Bogu swemu, w góre pogladajac.
\par 22 A gdy na ziemie spojrzy, oto ucisk i ciemnosc, zacmienie, bieda, i obaczy, ze jest wrazony do ciemnosci.

\chapter{9}

\par 1 Ale jednak nie tak zacmiona bedzie ona ziemia, która ucisniona bedzie, jako pierwszego czasu, gdy Bóg dotknal ziemie Zabulon, i ziemie Neftalim; ani jako potem, gdy obciazyl ku drodze morskiej przy Jordanie Galilee ludna.
\par 2 Bo lud on, który chodzi w ciemnosci, ujrzy swiatlosc wielka, a tym, którzy mieszkaja w ziemi cienia smierci, swiatlosc swiecic bedzie.
\par 3 Rozmnozyles ten naród, ales nie uczynil wielkiego wesela; wszakze weselic sie beda przed toba, jako sie wesela czasu zniwa, jako sie raduja, którzy lupy dziela;
\par 4 Gdyz jarzmo brzemienia jego, a laske ramienia jego, i pret poborcy jego zlamiesz, jako za dni Madyjanczyków,
\par 5 Gdzie sie wszystka bitwa bojujacych z trzaskiem stala, i szaty byly we krwi zbroczone, a co sie spalic moglo, ogniem spalono.
\par 6 Albowiem dziecie narodzilo sie nam, a syn dany jest nam; i bedzie panowanie na ramieniu jego, a nazwia imie jego: Dziwny, Radny, Bóg mocny, Ojciec wiecznosci, Ksiaze pokoju;
\par 7 A ku rozmnozeniu tego panstwa i pokoju, któremu konca nie bedzie, usiadzie na stolicy Dawidowej, i na królestwie jego, az je postanowi i utwierdzi w sadzie i w sprawiedliwosci, odtad az na wieki. Uczyni to zawisna milosc Pana zastepów.
\par 8 Poslal Pan slowo do Jakóba, a upadlo w Izraelu.
\par 9 I dowie sie wszystek lud Efraim, i mieszkajacy w Samaryi, którzy w hardosci i w wynioslosci serca mówia:
\par 10 Cegly upadly, ale my ciosanym kamieniem budowac bedziemy, podrabano plonne figi, ale my to w cedry odmienimy.
\par 11 Alec Pan wywyzszy nieprzyjaciól Rasynowych naden, a nieprzyjaciól jego zbierze;
\par 12 Syryjczyków z przodku, a Filistynczyków z tylu, i pozra Izraela cala geba. A wszakze w tem wszystkiem nie odwróci sie zapalczywosc jego, ale jeszcze reka jego bedzie wyciagniona.
\par 13 Przeto, ze sie lud ten nie nawraca do tego, który go bije, a Pana zastepów nie szuka:
\par 14 Dlatego Pan odetnie od Izraela glowe i ogon, Gala? i sitowie, dnia jednego.
\par 15 (Starzec i uczciwy czlowiek, ten jest glowa, a prorok, który uczy klamstwa, ten jest ogonem.)
\par 16 Albowiem wodzowie ludu tego sa zwodziciele, a którzy sie im wodzic dadza, zgineli.
\par 17 Dlatego z mlodzienców jego Pan sie nie ucieszy, a nad sierotami jego, i nad wdowami jego nie zmiluje sie; albowiem wszyscy sa obludni i zlosliwi, a kazde usta mówia sprosnosc. A wszakze w tem wszystkiem nie odwróci sie zapalczywosc jego; ale jeszcze reka jego bedzie wyciagniona.
\par 18 Albowiem gdy sie niepoboznosc jako ogien roznieci, pozre glóg i ciernie: potem zapali gestwine lasu, skad sie rozwieja jako dym na powietrzu.
\par 19 Albowiem dla rozgniewania Pana zastepów zacmi sie ziemia, a ten lud bedzie jako strawa ognia, i zaden bratu swemu nie przepusci.
\par 20 A porwieli co po prawej stronie, przecie laknac bedzie; a bedzieli zarl po lewej, przecie sie nie nasyci; kazdy z nich cialo ramienia swego zrec bedzie;
\par 21 Manases Efraima, a Efraim Manasesa, a obaj spolu beda przeciwko Judzie. Wszakze w tem wszystkiem nie odwróci sie zapalczywosc jego; ale jeszcze reka jego bedzie wyciagniona.

\chapter{10}

\par 1 Biada tym, którzy stanowia prawa niesprawiedliwe! i pisarzom, którzy ucisk na innych spisuja!
\par 2 Aby odpychali ubogiego od sadu, a wydzierali sprawiedliwosc ubogich ludu mego; aby wdowy byly korzyscia ich, a sierotki lupem ich.
\par 3 Cóz uczynicie w dzien nawiedzenia, i spustoszenia, które z daleka przyjdzie? do kogoz sie ucieczecie o wspomozenie? a gdzie zostawicie slawe wasze?
\par 4 Aby sie nie miala miedzy wiezniami unizyc, i miedzy pobitymi upasc. A wszakze w tem wszystkiem nie odwróci sie zapalczywosc jego; ale jeszcze reka jego bedzie wyciagniona.
\par 5 Biada Assurowi, rózdze gniewu mego! chociaz kij rozgniewania mego jest w reku jego.
\par 6 Na naród obludny posle go, a o ludu zapalczywosci mojej przykaze mu, aby bral lup i wydzieral korzysci a polozyl go na podeptanie, jako bloto na ulicach.
\par 7 Lecz on nie tak bedzie mniemal, i serce jego nie tak bedzie myslalo, poniewaz w sercu swem ulozyl, aby wytracil i wykorzenil niemalo narodów.
\par 8 Albowiem rzecze: Izali ksiazeta moi nie sa tez i królmi?
\par 9 Izali Chalmo nie jest jako Karchemis? Izali Arfat nie jest jako Emat? Izali Samaryja nie jest jako Damaszek?
\par 10 Jako reka moja znalazla królestwa balwanskie, chociaz balwany ich wieksze byly, niz w Jeruzalemie i w Samaryi.
\par 11 Izali Jeruzalemowi i balwanom jego tak nie uczynie, jakom uczynil Samaryi i balwanom jej?
\par 12 I stanie sie, gdy Pan wykona wszystke sprawe swoje na górze Syonskiej i w Jeruzalemie, ze nawiedze owoc wynioslego serca króla Assyryjskiego, i pyche wysokich oczów jego;
\par 13 Bo rzecze: W mocy reki mojej uczynilem to, i w madrosci mojej; bom byl madry, i odjalem granice narodów, a skarby ich zabralem, i wytracilem obywateli jako mocarz.
\par 14 Owszem reka moja znalazla majetnosc narodów jako gniazdo; a jako zbieraja jajka, które sa opuszczone, takiem ja wszystke ziemie zebral, a nie byl ktoby skrzydlem ruszyl, albo otworzyl usta, i coby mruczal.
\par 15 Izali sie bedzie przechwalala siekiera przeciw temu, który nia rabie? Izali sie bedzie wynosila pila przeciw temu, który nia trze? jakozby sie wynosila rózga przeciw temu, który ja podniósl? jakozby sie przechwalal kij, ze nie jest drewnem?
\par 16 Przetoz Pan, Pan zastepów, posle na tlustych jego suchoty, a pod slawa jego z predka sie zapali, jako gwaltowny ogien;
\par 17 Bo Swiatlosc Izraelowa bedzie ogniem, a Swiety jego plomieniem, który spali i pozre ciernie jego i oset jego dnia jednego.
\par 18 Takze wspanialosc lasu jego i urodzajnych pól jego, od duszy az do ciala zniszczy, i stanie sie jako chorazy od strachu uciekajacy.
\par 19 A pozostalych drzew lasu jego mala liczba bedzie, tak, ze je i dziecie bedzie popisac moglo.
\par 20 I stanie sie dnia onego, ze ostatki Izraelskie, i ci, którzy zostali z domu Jakóbowego, nie beda wiecej spolegac na tym, co ich bije; ale prawdziwie spolegac beda na Panu, Swietym Izraelskim.
\par 21 Ostatek nawróci sie, ostatek Jakóbowy do Boga mocnego.
\par 22 Bo chocby lud twój, o Izraelu! byl jako piasek morski, ostatek tylko z niego nawróci sie. Wytracenie naznaczone sprawi, ze ziemia bedzie oplywala sprawiedliwoscia.
\par 23 Wytracenie mówie naznaczone uczyni Pan, Pan zastepów, w posrodku tej wszystkiej ziemi.
\par 24 Przetoz tak mówi Pan, Pan zastepów: Nie bój sie Assyryjczyka, ludu mój! który mieszkasz w Syonie; rózga ubije cie, a laske swa podniesie na cie, jako na drodze Egipskiej.
\par 25 Albowiem po maluczkim czasie skonczy sie gniew mój przeciwko tobie, a na wygladzenie ich zapalczywosc moja powstanie.
\par 26 Gdyz bicz nan wzbudzi Pan zastepów, jako porazke Madyjanczyków na skale Horeb; a jako podniósl rózge swoje na morze na drodze Egipskiej tak ja nan podniesie.
\par 27 A dnia onego zdjete bedzie brzemie jego z ramienia twego, i jarzmo jego z szyi twojej; owszem, skazone bedzie jarzmo od przytomnosci pomazanego.
\par 28 Przyciagnie do Ajat, przejdzie przez Migron, w Machmas zlozy orez swój.
\par 29 Przejda bród, w Gieba jako w gospodzie nocowac beda; uleknie sie Rama, Gabaa Saulowe uciecze.
\par 30 Podnies glos twój, córko Gallim! niech slysza w Lais, o ubogie Anatot!
\par 31 Ustapi Madmena; obywatele Gabim zbiora sie do uciekania.
\par 32 Jeszcze przez dzien zastanowiwszy sie w Nobie, pogrozi reka swa górze córki Syonskiej, i pagórkowi Jeruzalemskiemu.
\par 33 Oto Pan, Pan zastepów, okrzesze wszystke sile latorosli, a te, którzy sa wysokiego wzrostu, podetnie; i beda wysocy ponizeni.
\par 34 Gestwiny takze lasów siekiera wytnie, a Liban od wielmoznego upadnie.

\chapter{11}

\par 1 Ale wyjdzie rószczka ze pnia Isajego, a latorostka z korzenia jego wyrosnie.
\par 2 I odpocznie na nim Duch Panski, Duch madrosci i rozumu, Duch rady i mocy, Duch umiejetnosci i bojazni Panskiej.
\par 3 I bedzie czulym w bojazni Panskiej, nie bedzie wedlug widzenia oczów swoich sadzil, ani wedlug slyszenia uszów swoich karal.
\par 4 Ale bedzie ubogich sadzil w sprawiedliwosci, a w prawosci bedzie karal cichych na ziemi. I uderzy ziemie rózga ust swoich, a duchem warg swoich zabije niezboznika.
\par 5 Albowiem sprawiedliwosc bedzie pasem biódr jego, a prawda przepasaniem nerek jego.
\par 6 I bedzie mieszkal wilk z barankiem, a lampart z kozleciem bedzie lezal; takze ciele i szczenie lwie, i karmne bydla pospolu beda, a male dziecie rzadzic ich bedzie.
\par 7 Krowa i niedzwiedzica spolem pasc sie beda, a plód ich pospolu lezec bedzie, a lew jako wól plewy jesc bedzie.
\par 8 A dziecie ssace bedzie gralo nad dziura zmijowa; a to, które odstawione jest, wpusci reke swoje do dziury bazyliszkowej.
\par 9 Nie beda szkodzic ani zabijac na wszystkiej górze mojej swietej; bo ziemia bedzie napelniona znajomoscia Panska, tak jako morze wodami napelnione jest.
\par 10 I stanie sie dnia onego, ze sie za korzeniem Isajego, który stanie za choragiew narodom, poganie pytac beda; albowiem odpocznienie jego slawne bedzie.
\par 11 Stanie sie tez dnia onego, iz Pan powtóre reke swa przylozy, aby posiadl ostatek ludu swego, który pozostanie od Assyryjczyków i od Egiptu, i od Patros, i od Chus, i od Elam, i od Senaar, i od Emat, i od wysep morskich.
\par 12 I podniesie choragwie miedzy poganami, a zgromadzi wygnanych z Izraela, a rozproszonych z Judy zbierze ze czterech stron ziemi.
\par 13 I ustanie nienawisc Efraimowa, a nieprzyjaciele Judowi wykorzenieni beda. Efraim nie bedzie nienawidzil Judy, a Juda nie bedzie trapil Efraima;
\par 14 Ale poleca na ramie Filistynów na zachód, a pospolu lupic beda narody na wschód slonca; na Edomczyków i Moabczyków sciagna reke swa, a synowie Amonowi posluszni im beda.
\par 15 Zniszczy tez Pan odnoge morza Egipskiego, i podniesie reke swoje przeciwko rzece mocnym wiatrem swym, a rozdzieli ja na siedm potoków, i sprawi to, ze ja w obuwiu przechodzic beda.
\par 16 A bedzie droga bita ostatkowi ludu jego, który pozostanie od Assyryjczyków, jako byla Izraelowi dnia onego, kiedy wychodzil z ziemi Egipskiej.

\chapter{12}

\par 1 I rzeczesz dnia onego: Wyslawiac cie bede, Panie! przeto, ze bedac rozgniewany na mie, odwróciles zapalczywosc gniewu twego, a ucieszyles mie.
\par 2 Oto Bóg zbawienie moje, ufac bede, a nie ulekne sie; albowiem Pan, Bóg mój, jest moca moja, i piesnia moja, i zbawieniem mojem.
\par 3 I bedziecie z radoscia czerpac wody ze zdrojów tegoz zbawienia.
\par 4 I rzeczecie dnia onego: Wyslawiajcie Pana wzywajcie imienia jego, opowiadajcie miedzy narodami sprawy jego, przypominajcie, ze wysokie jest imie jego.
\par 5 Spiewajcie Panu, albowiem wielkie rzeczy uczynil; niech to bedzie wiadomo po wszystkiej ziemi.
\par 6 Wykrzykaj a spiewaj, obywatelko Syonska! albowiem wielki jest w posrodku ciebie Swiety Izraelski.

\chapter{13}

\par 1 Brzemie Babilonu, które widzial Izajasz, syn Amosowy.
\par 2 Na górze wysokiej podniescie choragiew, podwyzcie glos do nich, dajcie znac reka, a niechaj wnijda w bramy ksiazece.
\par 3 Jam przykazal poswieconym moim; przyzwalem tez i mocarzów moich do wykonania gniewu mego, którzy sie wesela z wywyzszenia mego.
\par 4 Glos zgrai na górach, jako ludu gestego, glos i dzwiek królestw i narodów zgromadzonych: Pan zastepów spisuje wojsko na wojne.
\par 5 Ciagna z ziemi dalekiej, od konczyn niebios, mianowicie Pan i naczynia popedliwosci jego, aby zburzyl wszystke ziemie.
\par 6 Kwilcie! albowiem blisko jest dzien Panski, który przyjdzie jako spustoszenie od Wszechmocnego.
\par 7 Dlatego wszelkie rece oslabieja, a wszelkie serce czlowiecze stopnieje.
\par 8 I beda przestraszeni, uciski i trapienia ogarna ich, jako rodzaca bolec beda. Kazdy nad bliznim swoim zdumieje sie, oblicza ich plomieniowi podobne beda.
\par 9 Oto dzien Panski srogi idzie w zapalczywosci i popedliwosci gniewu, aby obrócil te ziemie w pustynie, a grzeszników jej aby z niej wygladzil.
\par 10 Bo gwiazdy niebieskie i planety ich nie dopuszcza swiecic swiatlu swemu; zacmi sie slonce, gdy wschodzic bedzie, a miesiac nie wyda swiatla swego.
\par 11 I nawiedze na okregu ziemskim zlosc, a na niezboznych nieprawosci ich; i uczynie koniec pysze hardych, a hardosc okrutników znize.
\par 12 Meza drozszym uczynie nad szczere zloto, a czlowieka nad zloto z Ofir.
\par 13 Dlatego zatrzasne niebem, a poruszy sie ziemia z miejsca swego w rozgniewaniu Pana zastepów, i w dzien popedliwego gniewu jego.
\par 14 I bedzie jako lani przeploszona, i jako trzoda, której nie ma kto zgromadzic; kazdy sie do ludu swego obróci, i kazdy do ziemi swojej uciecze.
\par 15 Ktokolwiek znaleziony bedzie, przebity bedzie; a kazdy, którzy sie kolwiek do nich przylaczy, od miecza poleze.
\par 16 Nadto i dziatki ich roztracane beda przed oczyma ich; domy ich splundrowane beda, a zony ich pogwalcone beda.
\par 17 Oto Ja pobudze przeciwko nim Medów, którzy o srebro nie beda dbali, a w zlocie nie beda sie kochali;
\par 18 Ale z luków dziatki postrzelaja, a nad plodem zywota nie zmiluja sie, oko ich synom nie przepusci.
\par 19 I bedzie Babilon, który byl ozdoba królestw i slawa zacnosci Chaldejczyków, jako podwrócenie od Boga Sodomy i Gomory.
\par 20 Nie beda sie w nim osadzac na wieki, ani mieszkac od narodu az do narodu; ani tam rozbije namiotu Arabczyk, ani tam pasterze z stadami odpoczywac beda.
\par 21 Ale tam zwierz odpoczywac bedzie, a domy ich bestyjami napelnione beda; i beda tam mieszkac sowy, a pokusy tam skakac beda.
\par 22 I beda sie sobie ozywac straszne potwory na palacach ich, a smoki na zamkach rozkosznych. A blisko tego ze przyjdzie czas jego, a dni jego nie odwloka sie.

\chapter{14}

\par 1 Albowiem zlituje sie Pan nad Jakóbem, a obierze zasie Izraela, i da im odpocznac w ziemi ich; a przylaczy sie do nich cudzoziemiec, i przystana do domu Jakóbowego.
\par 2 Bo wezma z soba narody, i przywioda je do miejsca swego; i wezmie je sobie dom Izraelski w ziemi Panskiej w dziedzictwo za slugi i za sluzebnice; i imac beda tych, którzy ich imali, a panowac beda nad tymi, którzy ich ciemiezyli.
\par 3 A dnia onego, któregoc Pan da odpocznienie od pracy twojej i od strachu twego, i od niewoli ciezkiej, w któras byl podbity,
\par 4 Wezmiesz te przypowiesc przeciw królowi Babilonskiemu, i rzeczesz: O jako ustal poborca, ustal podatek zlota!
\par 5 Pan zlamal kij niezboznych, i rózge panujacych;
\par 6 Tego, który ludzi bijal w zapalczywosci biciem ustawicznem, panowal w gniewie nad narodami bez litosci dreczonemi;
\par 7 Teraz odpoczywa i jest w pokoju wszystka ziemia, wszyscy glosno spiewaja;
\par 8 I jodly sie wesela nad toba, i cedry Libanskie, mówiac: Od tego czasu, jakos ty polegl, nie powstal, ktoby nas podcinal.
\par 9 I pieklo ze spodku wzruszylo sie dla ciebie, aby tobie przychodzacemu zaszlo; wzbudzilo dla ciebie umarlych, wszystkich ksiazat ziemi; rozkazalo powstac z stolic swoich i wszystkim królom narodów.
\par 10 Ci wszyscy odpowiadajac mówia do ciebie: I tys zemdlony jako i my, a stales sie nam podobnym.
\par 11 Stracona jest do piekla, pycha twoja i dzwiek muzyki twojej; podeslanoc mole, a robaki cie przykrywaja.
\par 12 Jakoz to, zes spadl z nieba, o jutrzenko! która wschodzisz rano? powalonys az na ziemie, którys watlil narody!
\par 13 Wszakies ty mawial w sercu swem: Wstapie na niebo, nad gwiazdy Boze wywyzsze stolice moje, a usiade na górze zgromadzenia, na stronach pólnocnych:
\par 14 Wstapie na wysokosc obloków, bede równy Najwyzszemu.
\par 15 Wszakze stracon jestes az do piekla, w glebokosc dolu.
\par 16 Którzy cie ujrza, za toba sie ogladac, i przypatrywac ci sie beda mówiac: Onze to maz, który trwozyl ziemie? który trzasal królestwami?
\par 17 Który obrócil okrag swiata w pustynie, a miasta jego poburzyl, a wiezniom swoim nie otwarzal ciemnicy?
\par 18 Wszyscy królowie narodów, cokolwiek ich bylo, pochowani sa w slawie, kazdy w domu swoim.
\par 19 Ales ty odrzucony od grobu swego, jako latorosl obrzydla, jako szata zabitych, których poprzebijano mieczem, którzy zstepuja do grobu kamienistego, jako scierw podeptany.
\par 20 Nie bedzisz mial uczestnictwa z nimi w pogrzebie; bos ziemie twoje pokazil, i lud swój pomordowal; albowiem nasienie zlosników nie bedzie wspominane na wieki.
\par 21 Gotujcie synów jego na zamordowanie dla nieprawosci ojców ich, aby nie powstali, i nie odziedziczyli ziemi, nie napelnili miastami okregu ziemskiego.
\par 22 Bo powstane przeciwko nim, mówi Pan zastepów, a wykorzenie imie Babilonu, i ostatki jego, tak syna jako i wnuka, mówi Pan;
\par 23 I uczynie je osiadloscia baków, i kaluzami wód, i wymiote go miotla spustoszenia, mówi Pan zastepów.
\par 24 Przysiagl Pan zastepów, mówiac: Zaiste, jakom umyslil, tak bedzie, a jakom uradzil, tak sie stanie;
\par 25 Iz potre Assyryjczyka w ziemi mojej, a na górach moich podepcze go; a odejdzie od nich jarzmo jego, i brzemie jego z ramienia jego zdjete bedzie.
\par 26 Tac jest rada uradzona przeciw onej wszystkiej ziemi; a tac jest reka wyciagniona przeciwko tym wszystkim narodom.
\par 27 A poniewaz Pan zastepów postanowil, któz to wzruszy? a reke jego wyciagniona któz odwróci?
\par 28 Roku, którego umarl król Achaz, stalo sie to proroctwo;
\par 29 Nie raduj sie, ty wszystka ziemio Filistynska! iz zlamana jest rózga tego, który cie bil; bo z korzenia wezowego wynijdzie bazyliszek, a plód jego bedzie smok ognisty latajacy.
\par 30 I beda sie pasc pierworodni nedznych, a ubodzy bezpiecznie odpoczna; ale korzen twój glodem wygubie a ostatki twoje wybije.
\par 31 Kwilze bramo! krzycz miasto! juzes sie rozplynela wszystka ty ziemio Filistynska; bo od pólnocy ogien przyjdzie, a nie bedzie, coby stronil od pocztów jego.
\par 32 A cóz odpowiedza poslom narodu? To, ze Pan ugruntowal Syon, a do niego sie uciekac beda ubodzy ludu jego.

\chapter{15}

\par 1 Brzemie Moabczyków. Poniewaz w nocy zburzone i spustoszone bedzie Ar Moabskie, poniewaz w nocy zburzone i spustoszone bedzie Kir Moabskie:
\par 2 Wstapi do Bait, i do Dybon, i do Bamot z placzem; nad Nebo, i nad Medeba Moab kwilic bedzie; na kazdej glowie jego bedzie lysina, i kazda broda ogolona bedzie.
\par 3 Na ulicach jego przepasza sie worem; na dachach jego i na rynkach jego kazdy kwilic bedzie, wracajac sie z placzem.
\par 4 I bedzie wolal Hesebon i Eleale, az w Jahas slyszany bedzie glos ich; owszem i zbrojni Moabscy narzekac beda, a dusza kazdego z nich porzewniac sobie bedzie, mówiac:
\par 5 Serce moje ryczy nad Moabem i nad twierdzami jego, az slyszec w Zoar, jako jalowica trzyletnia; bo droga Luchytska z placzem pójdzie, a na drodze Choronaim krzyk jako w porazce podniosa;
\par 6 Przeto, ze wody Nymrym zgina, ze poschna ziola, uwiednie trawa, a nic nie bedzie zielonego.
\par 7 Przetoz, cokolwiek sobie zachowali, i majetnosci ich, odniose do potoku Arabskiego.
\par 8 Bo obejdzie krzyk granice Moabska, az do Eglaim narzekanie jego, i az do Beer Elim kwilenie jego;
\par 9 Poniewaz i wody Dymonskie krwi pelne beda: bo przyloze Dymonowi przydatki, a na tych, którzyby uszli z Moabczyków, posle lwy, i na ostatki w tej ziemi.

\chapter{16}

\par 1 Poslijcie baranki Panujacemu nad ziemia, od skaly az do pustyni, do góry córki Syonskiej.
\par 2 Bo inaczej Moab bedzie jako ptak tulajacy sie, i z gniazda wyploszony; tak beda córki Moabskie przy brodach Arnon.
\par 3 Wnijdz w rade, uczyn sad, wystaw cien swój w posród poludnia jako noc, skryj wygnanców, a tulajacego sie nie wydawaj.
\par 4 Niech mieszkaja u ciebie wygnancy moi. O Moabie! badz ich ochrona przed pustoszycielem; albowiem ustanie gwaltownik, ustanie pustoszyciel, a wygladzony bedzie z ziemi ten, który innych depcze.
\par 5 I bedzie zgotowana stolica w milosierdziu, a usiadzie na niej w prawdzie w przybytku Dawidowym ten, któryby sadzil i szukal sadu, a czynil predka sprawiedliwosc.
\par 6 Lecz slyszelismy o pysze Moabowej, ze bardzo pyszny jest, o hardosci, i wynioslosci jego, i o zapalczywosci jego; wszakze nie przyjda do skutku zamysly jego.
\par 7 Przetoz narzekac bedzie Moabczyk przed Moabczykiem, wszyscy kwilic beda; nad gruntami miasta Kirchareset wzdychac beda, mówiac: Juzci sa skazone.
\par 8 Owszem i pola Hesebonskie spustoszone sa, i winna macica Sabama. Panowie narodów potarli najwyborniejsze macice jego, które az do Jazer siegaly, a szerzyly sie po puszczy; latorosli jego rozlozyly sie, i przesiegly morze.
\par 9 Przetoz placze dla placzu Jazerczyków, i dla winnicy Sabama oblewam sie lzami mojemi, o Hesebonie, i Eleale! bo na letni owoc twój, i na zniwo twoje przypadl okrzyk wojenny.
\par 10 I ustalo wesele i radosc nad polem urodzajnem; na winnicach nie spiewaja ani wykrzykaja; wina w prasach nie tloczy ten, który je tloczyl; i jac wykrzykania poprzestaje.
\par 11 Dlatego brzmia wnetrznosci moje nad Moabem jako lutnia, a trzewa moje nad Kircharesem.
\par 12 I stanie sie, gdy sie pokaze, ze sie spracowal Moab nad wyzynami, tedy wnijdzie do swiatnicy swojej, aby sie modlil, ale nic nie sprawi.
\par 13 Toc jest slowo, które Pan z dawna powiedzial o Moabie.
\par 14 Ale teraz powiedzial Pan, mówiac: Po trzech latach, jakie sa lata najemnicze, slawa Moabowa zelzona bedzie ze wszystka zgraja jego wielka, a ostatek jego lichy, maluczki i mdly bedzie.

\chapter{17}

\par 1 Brzemie Damaszku. Oto Damaszek przestanie byc miastem, a stanie sie kupa rumu.
\par 2 Miasta Aroer opuszczone beda; dla trzód beda, które tam odpoczywac beda, a nie bedzie, ktoby je straszyl.
\par 3 I ustanie obrona od Efraima, i królestwo od Damaszku, i od ostatka Syryjczyków, i jako slawa synów Izraelskich zniszczeja, mówi Pan zastepów.
\par 4 I stanie sie dnia onego, ze umniejszona bedzie slawa Jakóbowa, a tlustosc ciala jego schudnie.
\par 5 Albowiem Azur bedzie jako ten, który w zniwa zboze zbiera, a ramie jego znie klosy; i bedzie jako ten, co zbiera klosy w dolinie Refaim.
\par 6 Wszakze zostana na nim pominione grona, jako na otrzesnionej oliwie dwie albo trzy oliwiki zostana na wierzchu drzewa, a cztery albo piec na rodzajnych galeziach jego, mówi Pan, Bóg Izraelski.
\par 7 Dnia onego obejrzy sie czlowiek na stworzyciela swego, a oczy jego na Swietego Izraelskiego pogladac beda;
\par 8 A nie obejrzy sie na oltarze, sprawe rak swoich, ani na to, co uczynily palce jego, patrzyc bedzie, ani na gaje poswiecone, ani na obrazy sloneczne.
\par 9 Dnia onego miasta mocy jego beda opuszczone, jako chroscinka i rószczka, które opuszczone beda od synów Izraelskich, i bedziesz ziemia spustoszona.
\par 10 Bos zapomniala na Boga zbawienia swego, i na skale mocy twojej nie wspominalas. Przetoz choc szczepy rozkoszne szczepisz, i latorosli winne obce sadzisz;
\par 11 Czasu szczepienia twego szczepy aby rosly, opatrujesz; nawet tegoz poranku, co siejesz, aby sie puscilo, starasz sie: wszakze w dzien pozytku gromadno bolesc i rozpacz zac bedziesz.
\par 12 Biada zgrai ludu wielkiego, którzy hucza jako szum morski, i zgielkowi narodów, które szumia jako szum wód gwaltownych;
\par 13 Narodom, które szumia jako szum wód wielkich; bo je Pan sfuka, i uciekna daleko, i gonione beda od wiatru jako plewy po wierzchach gór, a jako wiechec od wichru.
\par 14 Bo czasu wieczornego nastapi trwoga, a niz poranek przyjdzie, alic go niemasz. Tenci jest dzial tych, którzy nas pustosza, i los tych, którzy nas plundruja.

\chapter{18}

\par 1 Biada ziemi, która zacmiaja skrzydla, która jest przy rzekach ziemi Murzynskiej!
\par 2 Która posyla poslów przez morze po wodach w lodziach z sitowia, mówiac: Idzcie, poslowie predcy! do narodu rozszarpanego i splundrowanego, do ludu strasznego z dawna i dotad, do narodu do szczetu podeptanego, którego ziemie rzeki rozerwaly.
\par 3 Wszyscy obywatele swiata i mieszkajacy na ziemi ujrzycie, gdy bedzie choragiew podniesiona na górach, i gdy w traby trabic beda, uslyszycie.
\par 4 Albowiem tak mówi Pan do mnie: Uspokoje sie, a przypatrywac sie bede z przybytku mojego, a bede jako cieplo jasne po deszczu, a jako oblok wypuszczajacy rose goracosci zniwa.
\par 5 Bo przed zbieraniem wina, gdy sie pusci paczki, a kwiat wyda grono cierpkie jeszcze rosnace, tedy oberznie latorostki nozami, a galezie odejmie i obetnie.
\par 6 I beda zostawione wszystkie wespól ptastwu na górach i zwierzetom ziemskim; i bedzie na nich przez lato ptastwo, a wszelaki zwierz ziemski na nich zimowac bedzie.
\par 7 Czasu onego przyniesiony bedzie dar Panu zastepów od ludu rozszarpanego i splundrowanego, od ludu strasznego z dawna i dotad, od narodu do szczetu podeptanego, którego ziemie rzeki rozrywaly; a przeniesiony bedzie na miejsce imienia Pana zastepów, na górze Syon.

\chapter{19}

\par 1 Brzemie Egiptu. Oto Pan jedzie na obloku lekkim, i przyciagnie do Egiptu, a porusza sie balwany Egipskie przed oblicznoscia jego, a serce Egipczan rozplynie sie w posrodku ich.
\par 2 Bo spuszcze Egipczan z Egipczanami, tak, iz walczyc bedzie kazdy przeciw bratu swemu, i kazdy przeciw przyjacielowi swemu, miasto przeciwko miastu, królestwo przeciwko królestwu.
\par 3 I zniszczony bedzie duch w Egipczanach, a rade ich w niwecz obróce; i beda sie radzic balwanów i wieszczków, i czarowników, i wrózków swoich.
\par 4 I podam Egipt w rece panów okrutnych, a król srogi panowac bedzie nad nimi, mówi Pan, Pan zastepów.
\par 5 I zgina wody z morza, a rzeka osiaknie i wyschnie.
\par 6 I pójda na wstecz rzeki, opadna i powysychaja potoki groblami ujete, trzcina i sitowie powiednie.
\par 7 Trawa okolo rzeki i przy brzegu jej, i wszelakie siewy przy potokach poschna, i zniszczeja i zgina.
\par 8 I beda sie smucic rybitwi, i zalosni beda wszyscy, którzy zarzucaja do rzeki wede; a którzy rozciagaja sieci po wodzie, do nedzy przyjda.
\par 9 Takze zawstydza sie ci, którzy tkaja rzeczy lniane, i subtelne, i którzy siatki robia.
\par 10 Albowiem sieci jego zepsowane beda, i wszyscy, którzy robia sadzawki dla ryb.
\par 11 Zaistec zglupieli ksiazeta Soanscy, madrych radców Faraonowych rada zglupiala. Jakoz rzeczecie do Faraona: Jam jest syn madrych, a syn królów starodawnych?
\par 12 Gdziez teraz sa medrkowie twoi? niech ci teraz oznajmia, jezli wiedza, co uradzil Pan zastepów przeciw Egiptowi.
\par 13 Zglupieli ksiazeta Soanscy, zwiedzieni sa ksiazeta Nofscy; zwiedli Egipt przedniejsi w pokoleniu jego.
\par 14 Pan puscil miedzy nich ducha wichrowatego, i sprawi to, ze pobladzi Egipt w kazdej sprawie swojej, tak jako bladzi pijany przy zwracaniu swojem.
\par 15 I nie bedzie zadna sprawa w Egipcie, któraby uczynic miala glowa albo ogon, Gala? albo sitowie
\par 16 Dnia onego bedzie Egipt podobny niewiastom; bo sie lekac i strachac bedzie przed podniesieniem reki Pana zastepów, która on podniesie przeciwko niemu.
\par 17 I bedzie ziemia Judzka Egiptowi na postrach; kazdy, kto wspomni na nia, bedzie sie lekal dla rady Pana zastepów, która postanowil o nim.
\par 18 Dnia onego bedzie piec miast w ziemi Egipskiej, mówiacych jezykiem Chananejskim, a przysiegajacych przez Pana zastepów; lecz jedno z nich miastem spustoszenia nazwane bedzie.
\par 19 Dnia onego stanie oltarz Panski w posród ziemi Egipskiej, a slup wystawiony bedzie Panu przy granicy jego.
\par 20 A bedzie na znak i na swiadectwo Panu zastepów w ziemi Egipskiej. A gdy zawolaja do Pana dla tych, którzy ich ciemiezyli, tedy im posle wybawiciela i ksiazecia, i wybawi ich.
\par 21 I bedzie Pan w Egipcie poznany, bo poznaja Pana Egipczanie dnia onego, a beda go czcic ofiarami i darami, i poslubia sluby Panu, a wypelnia je.
\par 22 A tak uderzy Pan Egipt, aby go zbiwszy uzdrowil go; bo sie nawróca do Pana, a on sie im da ublagac, i uzdrowi ich.
\par 23 Dnia onego bedzie gosciniec z Egiptu do Assyryi, i beda chodzic Assyryjczycy do Egiptu, a Egipczanie do Assyryi, i beda sluzyc Panu Egipczanie z Assyryjczykami.
\par 24 Dnia onego bedzie Izrael jako trzeci miedzy Egipczanem i Assyryjczykiem, a blogoslawienstwo bedzie w posrodku ziemi.
\par 25 Albowiem bedzie im blogoslawil Pan zastepów, mówiac: Blogoslawiony lud mój Egipski, a sprawa rak moich Assyryjczykowie a Izrael dziedzictwo moje.

\chapter{20}

\par 1 Roku, którego Tartan przyciagnal do Azotu, poslany bedac od Sargona, króla Assyryjskiego, i walczyl przeciw Azotowi, i dobyl go;
\par 2 Onegoz czasu rzekl Pan przez Izajasza, syna Amosowego, mówiac: Idz, a rozwiaz wór z biódr twoich, a bóty twoje zzuj z nóg twoich; i uczynil tak, i chodzil nago i boso.
\par 3 I rzekl Pan: Jako chodzi sluga mój Izajasz nago i boso, na znak i na cud tego, co sie ma stac trzeciego roku Egiptowi i Murzynskiej ziemi:
\par 4 Tak powiedzie król Assyryjski wiezniów Egipskich, i pojmanych Murzynskich, mlodych i starych, nagich i bosych, z obnazonemi zadkami na hanbe Egipczyków.
\par 5 I przelekna sie, i wstydzic sie beda za Murzynów, na których sie ogladali, i za Egipczanów, z których sie chlubili.
\par 6 Tedy rzecze dnia onego obywatel tej wyspy: Oto toc jest ucieczka nasza, do którejsmy uciekali o pomoc, abysmy wyswobodzeni byli z mocy króla Assyryjskiego; jakozbysmy tedy ujsc mogli?

\chapter{21}

\par 1 Brzemie pustego morza. Jako wicher na poludnie biezy, tak przyjdzie z puszczy, z ziemi strasznej.
\par 2 Widzenie srogie jest mi okazane. Przewrotny przewrotnosc broi, a pustoszyciel pustoszy. Przyciagnijze, Elamie! Oblez, Medzie! Babilon; wszelkiemu wzdychaniu jego koniec uczynie.
\par 3 Dlatego napelnione sa biodra moje bolescia, a ucisk ogarnal mie, jako ucisk rodzaca. Skrzywilem sie slyszac, a strwozylem sie widzac.
\par 4 Uleklo sie serce moje, strach mie ogarnal; noc rozkoszy moich obrócila mi sie w lekanie.
\par 5 Przygotuj stól; niech straz na strazy bedzie; jedz, pij; wstancie ksiazeta, smarujcie tarcze.
\par 6 Albowiem tak mi rzekl Pan: Idz, postaw stróza, któryby powiedzial, cokolwiek ujrzy.
\par 7 I ujrzal wozy, i dwa rzedy jezdnych; wozy, które osly, i wozy, które wielblady ciagnely: i przypatrywal sie im z wielka bardzo pilnoscia.
\par 8 Tedy zawolal jako lew: Panie mój! jac stoje na strazy ustawicznie we dnie; nawet na strazy mojej staje na kazda noc.
\par 9 (A oto wtem przyjechali mezowie na wozach, i jazda dwoma rzedami.) I zawolal straznik, a rzekl: Upadl, upadl Babilon, i wszystkie ryte obrazy bogów jego pokruszone o ziemie.
\par 10 Babilon jest gumno moje, i zboze bojewiska mego. Com slyszal od Pana zastepów, Boga Izraelskiego, tom wam opowiedzial.
\par 11 Brzemie Dumy. Wola na mie ktos z Seiru: Hej, strózu! co sie stalo w nocy? Strózu! co sie stalo w nocy?
\par 12 Rzekl stróz: Przyszedl poranek, takze i noc. Chcecieli szukac, szukajcie, nawróccie sie a przyjdzcie.
\par 13 Brzemie na Arabije. Po lasach Arabii noclegi miewac bedziecie, o podrózni Dedanscy!
\par 14 Niech zabieza pragnacemu, niosac wode obywatele ziemi Temanskiej; z chlebem jego niech wynijda przeciw uciekajacemu.
\par 15 Bo przed mieczami uciekac beda, przed mieczem dobytym, przed lukiem napietym, przed ciezkoscia bitwy.
\par 16 Gdyz tak rzekl Pan do mnie: Ze po roku, jaki jest rok najemniczy, ustanie wszystka slawa Kedar.
\par 17 A ostatek pocztu strzelców meznych synów Kedar bedzie umniejszony; albowiem to Pan Bóg Izraelski mówil.

\chapter{22}

\par 1 Brzemie doliny widzenia. Cóz ci sie stalo, zes wszystka na dachy wystapila?
\par 2 Miasto pelne wrzasku, i zgielku, miasto weselace sie! Pobici twoi nie sa pobici mieczem, ani zgineli w bitwie.
\par 3 Wszyscy ksiazeta twoi naporzad sie rozpierzchneli, od strzelców powiazani sa wespól, i ci, którzy z daleka uciekaja.
\par 4 Dlategom rzekl: Odstapcie odemnie, abym gorzko plakal; nie kwapcie sie, cieszyc mie w spustoszeniu córki ludu mojego.
\par 5 Albowiem to jest dzien ucisku i podeptania, i zamieszania od Pana, Pana zastepów, w dolinie widzenia, dzien burzenia murów, i wolania na góry.
\par 6 Elam tez wzial sajdak z wozami ludu wojennego, a Kir okazal tarcze swoje.
\par 7 I stalo sie, ze wyborne doliny twoje napelnione byly wozami, a jezdni sie poteznie zaszancowali u bramy.
\par 8 I odkryta byla zaslona Judowa; a pogladales dnia onego na zbrojownie w domu lasu.
\par 9 I pogladaliscie na rozwaliny miasta Dawidowego, bo ich wiele bylo; i zgromadzily sie wody sadzawki dolnej.
\par 10 Takze policzyliscie domy w Jeruzalemie, a rozwaliliscie domy na oprawe murów.
\par 11 Uczyniliscie tez przekop miedzy dwoma murami, dla wód stawu starego, a nie ogladaliscie sie na tego, co go sprawil, a tego, który go zdawna zbudowal, nie widzieliscie.
\par 12 Nadto, gdy wolal Pan, Pan zastepów, dnia onego do placzu i do narzekania, i do oblysienia sie, i do przepasania sie worem;
\par 13 A oto radosc i wesele wasze, zabijac woly, i bic owce, a jedzac mieso, i pijac wino, mówic: Jedzmy, pijmy, bo jutro pomrzemy.
\par 14 Alec to doszlo uszów moich, mówi Pan zastepów. Przetoz wam ta nieprawosc nie bedzie odpuszczona; az pomrzecie, mówi Pan, Pan zastepów.
\par 15 Tak mówi Pan, Pan zastepów: Idz, wnijdz do tego podskarbiego, do Sobny, który jest przelozonym w domu, i rzecz:
\par 16 Co ty tu masz? albo kogo tu masz, zes tu sobie wykowal grób? Wykowales sobie na wysokiem miejscu grób swój, a wystawiles na skale przybytek swój?
\par 17 Oto Pan, który cie przykryl jako zacnego meza, a który cie kosztownie przyodzial,
\par 18 Predko cie zatoczy jako kule do ziemi szerokiej i przestronnej; tam umrzesz, tam i wozy slawy twojej zgina, o hanbo domu Pana swego!
\par 19 A tak wypedze cie z stanowiska twego, a z urzedu twego zloze cie.
\par 20 A dnia onego przyzwie sluge swego Elijakima, syna Helkijaszowego;
\par 21 I obleke go w szate twoje, i pasem twoim potwierdze go, panowanie tez twoje dam w reke jego; i bedzie za ojca obywatelom Jeruzalemskim, i domowi Judzkiemu.
\par 22 I poloze klucz domu Dawidowego na ramieniu jego; gdy otworzy, nikt nie zawrze, a gdy zawrze, nikt nie otworzy.
\par 23 I wbije go jako gwózdz na miejscu pewnem, a bedzie stolica chwaly domu ojca swego.
\par 24 A zawisnie na nim wszystka slawa domu ojca jego, synowie i córki, i wszystko naczynie by najmniejsze, od naczynia, z którego pija, az do kazdego naczynia winnego.
\par 25 Dnia onego, mówi Pan zastepów, bedzie wyjety gwózdz, który byl wbity na miejscu pewnem, a bedzie przyciety i upadnie; odciete bedzie i brzemie, które jest na nim; bo Pan mówil.

\chapter{23}

\par 1 Brzemie Tyru. Kwilcie okrety morskie! albowiem zburzony jest, tak, iz niemasz ani domu, ani ktoby przychodzil z ziemi Cytym.
\par 2 To mi o nich objawiono. Umilknijciez, obywatele wyspy! która kupcy Sydonscy plywajac przez morze napelniali.
\par 3 A którego dochody na wielkich wodach, nasienie Sychor, zniwo jego dochód z rzeki, a w którym byl sklad narodów.
\par 4 Zawstydz sie, Sydonie! bo rzeklo morze, moc morska, mówiac: Nie pracuje w porodzeniu, i nie rodza, i nie wychowuje mlodzienców, ani odchowuje panien.
\par 5 Jako nad powiescia o Egipcie, tak beda zalosni o Tyrze.
\par 6 Przeprawcie sie przez morze, kwilcie obywatele wyspy!
\par 7 Toz to jest miasto wasze weselace sie? Jego starozytnosc jestci ode dni dawnych; ale go zawioda nogi jego na daleka wedrówke.
\par 8 Któz to postanowil o Tyrze, który koronuje insze? którego kupcy sa ksiazetami, a kramarze jego slawnymi na ziemi?
\par 9 Pan zastepów postanowil to, aby ohydzil pyche wszelkiej slawy, a zeby do zniewagi przywiódl wszystkich zacnych na ziemi.
\par 10 Nawróc sie do ziemi swej, jako rzeka, o córko morska; niemaszci tam wiecej pasa.
\par 11 Reke swoje wyciagnal na morze, zatrwozyl królestwa. Pan rozkazal o Chanaanie, aby zburzone byly twierdze jego;
\par 12 I rzekl: Juz sie nie bedziesz wiecej weselila, ty zgwalcona panno, córko Syonska! Powstan, przepraw sie do Cytym; lecz i tam nie bedziesz miala odpoczynku.
\par 13 Oto ziemia Chaldejska, ten lud nie byl ludem. Assyryjczyk zalozyl ja dla obywateli pustyn, którzy wyslawili zamki jej, pobudowali palace jej; ale on ja w gruz obrócil.
\par 14 Kwilcie okrety morskie! albowiem zburzona jest twierdza wasza.
\par 15 I stanie sie dnia onego, ze w zapamietaniu bedzie Tyr przez siedmdziesiat lat, przez wiek króla jednego. A po siedmdziesieciu latach Tyr znowu bedzie mial piosnke, jako piosnke nierzadnicy.
\par 16 Wezmij lutnie, obchodz miasto, o nierzadnico w zapomnienie podana! graj dobrze, dlugo spiewaj, abys na pamiec przyszla.
\par 17 I stanie sie po wyjsciu siedmdziesieciu lat, ze Tyr Pan nawiedzi; ale sie on zas wróci do nierzadniczego zysku swego, i bedzie nierzad plodzil ze wszystkiemi królestwami ziemi, na obliczu ziemi.
\par 18 Wszakze kupiectwo jego, i zysk jego bedzie poswiecony Panu. Do skarbu odlozony, i schowany nie bedzie; ale tym, którzy mieszkaja przed Panem, pozyteczne bedzie kupiectwo jego, aby jedli do sytosci, a mieli odzienie dobre.

\chapter{24}

\par 1 Oto Pan obnazy ziemie, i spustoszy ja, i przemieni oblicze jej, a rozproszy obywateli jej.
\par 2 I bedzie jako lud pospolity tak i ksiaze; jako sluga, tak pan jego; jako dziewka, tak pani jej; jako kupujacy, tak sprzedawajacy; jako pozyczajacy, tak i ten, co u drugiego pozycza; jako lichwiarz, tak ten, co lichwe daje.
\par 3 Wielce obnazona bedzie ziemia, i bardzo zlupiona; albowiem Pan mówil to slowo.
\par 4 Plakac bedzie i upadnie ziemia, zwatleje i obali sie okrag ziemski; zemdleja wysokie narody ziemskie,
\par 5 Przeto, ze ta ziemia splugawiona jest pod obywatelami swoimi; albowiem przestapili prawa, odmienili ustawy, wzruszyli przymierze wieczne.
\par 6 Dla tego przeklestwo pozre ziemie, a zniszczeja obywatele jej; dlatego popaleni beda obywatele ziemi, a malo ludzi zostanie.
\par 7 Smucic sie bedzie moszcz, uwiednie winna macica, wzdychac beda wszyscy wesolego serca.
\par 8 Ustanie wesele bebnów, ustanie wykrzykanie weselacych sie, ucichnie wesele cytry.
\par 9 Nie beda pic wina z spiewaniem; gorzki bedzie napój mocny pijacym go.
\par 10 Starte bedzie miasto próznosci; kazdy dom zawarty bedzie, aby do niego nie wchodzono.
\par 11 Narzekanie bedzie na ulicach dla wina; zacmione bedzie wszelkie wesele, a przeniesie sie radosc ziemi.
\par 12 Spustoszenie w miescie zostanie, a bramy zburzone beda.
\par 13 Albowiem tak bedzie w posród ziemi, w posrodku narodów, jako gdy otrzesa oliwy, i jako bywa z gronami, gdy sie dokona zbieranie wina.
\par 14 Ci podniosa glos swój, wykrzykac beda, w zacnosci Panskiej wykrzykac beda, i przy morzu.
\par 15 Przetoz w dolinach wyslawiajcie Pana, na wyspach morskich imie Pana, Boga Izraelskiego.
\par 16 Od konczyn ziemi slyszymy piosnke o slawie sprawiedliwego. Alem ja rzekl: Wychudlem, wychudlem, biada mnie! Przewrotni przewrotnosc broja, przewrotnosc, mówie, bez wszelkiego wstydu broja.
\par 17 Strach, i dól, i sidlo przyjdzie na cie, który mieszkasz na ziemi.
\par 18 I stanie sie, ze kto uciecze przed wiescia strachu, wpadnie w dól, a kto wylizie z dolu, pojmany bedzie sidlem; bo upusty z wysokosci otworzone beda a zatrzasna sie grunty ziemi.
\par 19 Rozstepujac rozstapi sie ziemia; rozsiadajac rozsiadzie sie ziemia; poruszajac poruszy sie ziemia.
\par 20 Chwiejac chwiac sie bedzie ziemia jako pijany a przeniesiona bedzie jako budka; bo ja obciazy nieprawosc jej, i upadnie, a wiecej nie powstanie.
\par 21 A dnia onego nawiedzi Pan wojsko wysokie na wysokosci, takze i królów ziemskich na ziemi.
\par 22 I beda zgromadzeni, jako zgromadzeni bywaja wiezniowie do ciemnicy, a beda zamknieni w tarasie; po wielu, mówie, dniach, nawiedzeni beda.
\par 23 I zasromi sie miesiac, a zawstydzi sie slonce, gdy królowac bedzie Pan zastepów, na górze Syonskiej, i w Jeruzalemie, i przed starcami swymi w wielkiej slawie.

\chapter{25}

\par 1 Panie! tys Bóg mój, wywyzszac cie bede i wyslawiac bede imie twoje, bos uczynil rzeczy dziwne; rady twe, z dawna postanowione, sa wierna prawda.
\par 2 Albowiemes miasta obrócil w mogile; miasto obronne w rozwaliny; palace cudzoziemców, aby nie byly miastem, i aby nie byly znowu na wieki budowane.
\par 3 Dlatego cie wielbic bedzie lud mozny; miasta narodów srogich ciebie sie bac beda.
\par 4 Albowiemes ty byl twierdza ubogiemu, zamkiem nedznemu w ucisku jego, ucieczka przed powodzia, zaslona przed goracem, gdyz wscieklosc okrutników byla jako powódz podwracajaca sciane.
\par 5 Huk cudzoziemców potlumiles, jako goracosc w susze; jako goracosc cieniem obloku, tak okrucienstwo okrutników potlumione.
\par 6 I sprawi Pan zastepów na wszystkie narody na tej górze uczte z rzeczy tlustych, uczte z wystalego wina, z rzeczy tlustych, szpik w sobie majacych, z wina wystalego i czystego.
\par 7 I skazi na tej górze zaslone, która zaslania wszystkich ludzi, i przykrycie, którem sa przykryte wszystkie narody.
\par 8 Polknie smierc w zwyciestwie, a Pan panujacy otrze lze z kazdego oblicza, i pohanbienie ludu swego odejmie ze wszystkiej ziemi; bo Pan mówil.
\par 9 I rzecze dnia onego lud Panski: Oto Bóg nasz ten jest; oczekiwalismy go, i wybawil nas. Tenci jest Pan, któregosmy oczekiwali; weselic i radowac sie bedziemy w zbawieniu jego.
\par 10 Albowiem na tej górze odpocznie reka Panska, a Moab podeptany od niego bedzie, jako plewa w gnój wdeptana bywa.
\par 11 I wyciagnie rece swoje w posród jego, jako je wyciaga plywacz ku plywaniu, a ponizy wynioslosc jego lokciami rak swoich.
\par 12 A tak obrone i wysokosc murów twoich pochyli, ponizy i powali na ziemie az do prochu.

\chapter{26}

\par 1 Dnia onego spiewana bedzie ta piesn w ziemi Judzkiej: Mamy miasto obronne, Bóg zbawieniem opatrzyl mury i baszty jego.
\par 2 Otwórzcie bramy, a niech wnijdzie naród sprawiedliwy, który strzeze prawdy.
\par 3 Czlowieka spolegajacego na tobie zachowywasz w pokoju, w pokoju mówie: bo w tobie ufa.
\par 4 Miejcie nadzieje w Panu az na wieki; boc w Panu, w Panu jest skala wieczna.
\par 5 Ale poniza mieszkajacego na wysokosci; miasto wyniosle poniza, poniza je az do ziemi, i straca je az do prochu;
\par 6 Depcze je noga; nogi ubogiego, stopa nedzników.
\par 7 Scieszka sprawiedliwego jest prosta; prosta droga sprawiedliwego wyrównywasz.
\par 8 Na drodze sadów twoich, Panie! oczekujemy cie; zadnosc duszy naszej jest do imienia twego, i do wspominania na cie.
\par 9 Dusza moja zada cie w nocy, owszem, duchem swym, który jest we mnie, rano cie szukam; albowiem gdy sie sady twoje odprawiaja na ziemi, sprawiedliwosci sie ucza obywatele okregu ziemskiego.
\par 10 Gdy sie laska pokazuje niepoboznemu, nie uczy sie sprawiedliwosci; w ziemi prawosci nieprawosc czyni, a nie dba nic na majestat Panski.
\par 11 Panie! choc wywyzszona jest reka twoja, przecie tego nie widza; ujrzac, ale pohanbieni beda, zajrzac ludowi twemu; nadto i ogien tych nieprzyjaciól twoich pozre.
\par 12 Panie! zrzadzisz nam pokój; bo wszystko, co sie dzialo przy nas, czyniles ku dobremu naszemu.
\par 13 Panie Boze nasz! panowalic nad nami inni panowie oprócz ciebie; ale mysmy tylko, w tobie ufajac, wspominali na imie twoje.
\par 14 Pomarli, nie ozyja; martwymi bedac nie powstana, przeto, zes ich nawiedzil i wykorzenil, i wygladzil wszystke pamiatke ich.
\par 15 Rozmnozyles naród; o Panie! rozmnozyles naród; uwielbionys jest, aczes go byl zapedzil na wszystkie granice ziemi.
\par 16 Panie! w ucisku szukali cie; gdys ich karal, wylewali modlitwy swe.
\par 17 Jako brzemienna, gdy sie przybliza ku rodzeniu, boleje i wola w bolesciach swoich, takesmy byli przed obliczem twojem, Panie!
\par 18 Poczelismy, bolelismy; alesmy tylko jakoby wiatr porodzili, a zadnegosmy wybawienia ziemi nie sprawili, i nie upadli mieszkajacy na okregu ziemskim.
\par 19 Ozyja umarli twoi, trupy moje wstana, gdy rzeczesz: Ocuccie sie, a spiewajcie mieszkajacy w prochu! Albowiem rosa twoja bedzie jako rosa na ziolach; ale niezboznych o ziemie uderzysz.
\par 20 Idz, ludu mój! wnijdz do komór swoich, a zamknij drzwi twoje za soba; skryj sie na maluczka chwilke, dokad nie przeminie rozgniewanie.
\par 21 Albowiem oto Pan wychodzi z miejsca swego, aby nawiedzil nieprawosc mieszkajacych na ziemi; tedy ziemia odkryje krew swoje, a nie zakryje dalej pobitych swoich.

\chapter{27}

\par 1 Dnia onego nawiedzi Pan mieczem swoim srogim, wielkim i mocnym, Lewiatana, weza dlugiego, i Lewiatana, weza skreconego, a zabije smoka, który jest w morzu.
\par 2 Dnia onego spiewajcie o winnicy wybornego wina.
\par 3 Ja Pan, który jej strzege, co chwilka odwilzac ja bede, a zeby jej kto nie psul, w nocy i we dnie strzedz jej bede.
\par 4 Zapalczywosci zadnej we mnie niemasz. Któz mi da oset albo ciernie, abym przeciwko niej walczyl, i spalil ja do szczetu?
\par 5 Izali kto ujmie sile moje, aby uczynil pokój zemna? aby pokój, mówie, uczynil zemna?
\par 6 Przyjdzie do tego, ze sie Jakób rozkorzeni, zakwitnie i rozrodzi sie Izrael, i napelni okrag ziemski owocem.
\par 7 Bo izali go tak uderzy, jako uderzyl nieprzyjaciela jego? albo izali go zamordowal, jako inni sa zamordowani od niego?
\par 8 Owszem, miernie go karal, i w ten czas, gdy go wypychal i gdy go nieprzyjaciel wiatrem swoim gwaltownym w dzien wschodniego wiatru, zabieral.
\par 9 Przetoz tym sposobem oczyszczona bedzie nieprawosc Jakóbowa; a tenci jest wszystek pozytek, ze odejmie grzech jego, gdy rozrzuci wszystkie kamienie oltarza, jako kamienie wapienne rozszarpane, a nie ostoja sie gaje i obrazy sloneczne.
\par 10 Gdy miasto obronne spustoszeje, a bedzie mieszkaniem porzuconem i spustoszonem jako pustynia. Tam sie pasc, i tam legac bedzie cielec, i ogryzie latoroslki jego.
\par 11 Gdy poschna galazki jego, pokruszone beda, a niewiasty przyszedlszy zapala je. Albowiem ten lud nie ma zadnego rozumu; przetoz nie zmiluje sie nad nim, który go uczynil, a który go stworzyl, nie zlituje sie nad nim.
\par 12 Dnia onego, gdy sie Pan bedzie mscil od lozyska rzeki az do potoku Egipskiego, wy synowie Izraelscy po jednemu zebrani bedziecie.
\par 13 Stanie sie tez dnia onego, ze zatrabia w trabe wielka, i przyjda, którzy byli pogineli w ziemi Assyryjskiej, i którzy byli zagnani do ziemi Egipskiej; i beda sie Panu klaniali na górze swietej w Jeruzalemie.

\chapter{28}

\par 1 Biada pysznej koronie, pijanicom z Efraima, i kwiatowi opadlemu z ozdoby slawy swojej! Biada tym, którzy rzadza dolina bardzo urodzajna, i znikczemnialym od wina!
\par 2 Oto mozny i silny Panski bedac jako nawalnosc gradu, jako wicher wywracajacy, jako bystrosc wód gwaltownej powodzi uderzy ja o ziemie reka swa.
\par 3 Nogami podeptana bedzie pyszna korona, pijanicy Efraimscy!
\par 4 Tedy sie stanie, ze kwiat opadajacy z ozdoby i z slawy swojej, tych, którzy rzadza dolina bardzo urodzajna, bedzie jako owoc skorozrzy, pierwej niz lato bywa; który skoro kto obaczy, nie pusci go z reki, az go zje.
\par 5 Dnia onego bedzie Pan zastepów korona ozdoby, i korona slawy ostatkowi ludu swego,
\par 6 I duchem sadu siedzacemu na sadzie, a moca tym, którzy odpieraja bitwe az do bramy.
\par 7 Ale i ci od wina bladza, i od mocnego napoju potaczaja sie. Ksiaze i prorok bladza od mocnego napoju, utoneli w winie, potaczaja sie od mocnego napoju, bladza w widzeniu, potykaja sie w sadzie.
\par 8 Albowiem wszystkie stoly ich pelne sa zwracania i plugastwa, tak, az miejsca nie staje.
\par 9 Kogozby uczyc mial umiejetnosci? a komu da zrozumiec co slyszal? Izali odstawionym od mleka, a odsadzonym od piersi?
\par 10 Poniewaz podawal im przykazanie za przykazaniem, przykazanie za przykazaniem, przepis za przepisem, przepis za przepisem, troche tu, troche owdzie:
\par 11 A wszakze jakoby nieznajoma mowa, i jezykiem obcym mówil do ludu twego.
\par 12 A gdy im rzekl: Toc jest odpocznienie, sprawcie odpoczynek spracowanemu, toc jest odpocznienie; ale oni nie chcieli sluchac.
\par 13 I bedzie im slowo Panskie: przykazanie za przykazaniem, przykazanie za przykazaniem, przepis za przepisem, przepis za przepisem; troche tu, troche owdzie, na to, aby szli, a padlszy wznak stlukli sie, a uwiklani bedac, pochwytani byli.
\par 14 Przetoz sluchajcie slowa Panskiego, mezowie nasmiewcy! panujacy nad tym ludem, który jest w Jeruzalemie.
\par 15 Dlatego, ze mówicie: Uczynilismy przymierze z smiercia, i z pieklem mamy porozumienie, bicz gwaltowny nas nie dojdzie, gdy przechodzic bedzie; bosmy polozyli klamstwo za ucieczke swoje, a pod falszem utailismy sie;
\par 16 Dlategoz tak powiedzial panujacy Pan: Oto Ja za grunt klade w Syonie kamien, kamien doswiadczony, wegielny, kosztowny, gruntownie ugruntowany; kto wierzy, nie pokwapi sie.
\par 17 A wykonam sad wedlug sznuru, a sprawiedliwosc wedlug wagi; i potlucze grad nadzieje omylna, a ucieczke wody zatopia.
\par 18 A tak zgladzone bedzie przymierze wasze z smiercia, a porozumienie wasze z pieklem nie ostoi sie; gdy bicz gwaltowny przechodzic bedzie, bedziecie od niego podeptani.
\par 19 Kiedy jedno pocznie przechodzic, pochwyci was; bo na kazdy poranek przechodzic bedzie we dnie i w nocy. A sam postrach przywiedzie was ku zrozumieniu tego, coscie slyszeli;
\par 20 Zwlaszcza iz krótsze bedzie loze, nizby sie kto mógl rozciagnac, i nakrycie waskie, chocby sie skurczyl.
\par 21 Albowiem Pan powstanie jako na górze Perazym, a rozgniewa sie jako w dolinie Gabaon, aby wykonal sprawe swoje, niezwyczajna sprawe swoje, i aby dokonczyl sprawy swojej, niezwyczajnej sprawy swojej.
\par 22 A tak teraz nie nasmiewajcie sie, aby sie niezmocnily zwiazki wasze, bom o pewnem zepsowaniu wszystkiej ziemi slyszal od Pana, Pana zastepów,
\par 23 Nadstawiajcie uszów, a sluchajcie glosu mego; badzcie pilni, a sluchajcie mowy mojej.
\par 24 Izali kazdego dnia oracz orze, aby sial? przegania brozdy, a wlóczy role swoje?
\par 25 Izali zrównawszy wierzch jej, nie rozsiewa wyki, i nie roztrzasa kminu, i nie sieje pszenicy wybornej, i jeczmienia przedniego, i orkiszu na miejscu sposobnem?
\par 26 Bo go uczy roztropnosci Bóg jego, i naucza go.
\par 27 Wyki nie mlóca okowanem naczyniem, ani taczaja kola wozowego po kminie; ale kijem wybijaja wyke, a kmin laska.
\par 28 Pszenica mlócona bywa; wszakze i tej nie zawzdy mlócic bedzie, ani jej potrze kolem woza swego, ani jej zebami jego pokruszy.
\par 29 I toc od Pana zastepów wyszlo, który jest dziwny w radzie, a wielmozny w rzeczy samej.

\chapter{29}

\par 1 Biada Aryjelowi! Aryjelowi miastu, w którem mieszkal Dawid. Przydajcie rok do roku, niechaj rzeza barany.
\par 2 Jednak ucisne Aryjela, i bedzie smutek i zalosc, bo mi bedzie jako Aryjel.
\par 3 Poloze sie zaiste obozem w okolo przeciwko tobie, i scisne cie walami, i wystawie przeciwko tobie baszty.
\par 4 Tedy bedac znizone, z ziemi mówic bedziesz, i z prochu szeptac bedzie mowa twoja; bedzie mówil glos twój, jako wieszczka z ziemi, a z prochu mowa twoja szeptac bedzie.
\par 5 Bo mnóstwo nieprzyjaciól twoich bedzie jako proszku drobnego, a zgraja okrutników jako plew latajacych; a to sie nagle w okamgnieniu stanie.
\par 6 Od Pana zastepów nawiedzione bedzie gromem i trzesieniem ziemi, i glosem wielkim, wichrem i burza, i plomieniem ognia pozerajacego.
\par 7 Ale jako sen widzenia nocnego, tak bedzie zgraja wszystkich narodów walczacych przeciwko Aryjelowi, i wszystkich bojujacych przeciwko niemu i twierdzom jego, i tych, którzy go uciskaja.
\par 8 Bedzie, mówie, jako gdy sie sni glodnemu, jakoby jadl; ale gdy sie ocuci, alic czczy zywot jego; i jako gdy sie sni pragnacemu, jakoby pil, a gdy sie ocuci, alic zemdlony zostaje, a dusza jego pragnie; tak bedzie zgraja wszystkich narodów walczacych przeciwko górze Syonskiej.
\par 9 Jakoz tedy odwlaczacie, chocbyscie sie zdumiewac mieli; rozkoszujecie, chocbyscie mieli na pomoc wolac. Opili sie, ale nie winem; potaczaja sie, ale nie od mocnego napoju.
\par 10 Bo was napelnil Pan duchem snu twardego, i zawarl oczy wasze; proroków i ksiazat waszych najopatrzniejszych oczy zaslonil.
\par 11 Przetoz wam wszelkie widzenie podobne jest slowom ksiag zapieczetowanych, które danoliby temu, co zna pismo, a rzeczono: Czytaj to prosze, tedy odpowie: Nie moge, bo sa zapieczetowane.
\par 12 A danoliby ksiegi temu, co nie zna pisma, a rzeczono: Czytaj to prosze, tedy odpowie: Nie znam pisma.
\par 13 Bo mówi Pan: Przeto, ze ten lud przybliza sie do mnie usty swemi, a serce jego dalekie jest odemnie, a bojazni, która sie mnie boja, z przykazan ludzkich nauczyli sie:
\par 14 Dlatego Ja tez sobie dziwnie poczne z tym ludem, dziwnie i cudownie, i zginie madrosc madrych jego, a rozum roztropnych jego skryje sie.
\par 15 Biada tym, którzy gleboko przed Panem ukrywaja rade! których kazda sprawa dzieje sie w ciemnosci, i mówia: Któz widzi? Kto wie o nas?
\par 16 Przewrotne mysli wasze sa jako glina garncarska. Izali rzecze robota o tym, co ja urobil: Nie urobil mie? i ulepienie izali rzecze o tym, co je ulepil: Nie rozumial?
\par 17 Izali po maluczkim i króciuchnym czasie nie obróci sie Liban w pole? a pole za las poczytane nie bedzie?
\par 18 I uslysza dnia onego glusi slowa ksiag, a z mroku i z ciemnosci oczy slepych patrzac beda.
\par 19 Ale cisi nader sie rozwesela w Panu, a ubodzy ludzie rozwesela sie w Swietym Izraelskim.
\par 20 Gdy ustanie okrutnik, a zniszczeje nasmiewca, wykorzenieni beda wszyscy, którzy pilnowali nieprawosci;
\par 21 Którzy winuja czlowieka dla slowa, a na tego, który ich strofuje, w bramie sidla stawiaja, i bez przyczyny do upadku przywodza sprawiedliwego.
\par 22 Przetoz tak mówi o domu Jakóbowym Pan, który odkupil Abrahama: Juz dalej nie bedzie zawstydzony Jakób, ani wiecej twarz jego zblednie.
\par 23 Albowiem gdy ujrzy synów swoich, dzielo rak moich, w posrodku siebie, poswiecajacych imie moje; tedy beda poswiecac Swietego Jakóbowego, a Boga Izraelskiego bac sie beda.
\par 24 I stana sie rozumnymi bladzacy duchem, a szemracze naucza sie umiejetnosci.

\chapter{30}

\par 1 Biada synom odpornym, mówi Pan, którzy czynia rade, ale nie ze mnie, i nakrywaja ja nakryciem, ale nie z ducha mojego, aby przyczyniali grzechu do grzechu.
\par 2 Którzy chodza a zstepuja do Egiptu, nie radzac sie ust moich, aby sie zmocnili moca Faraonowa, i ukryli sie w cieniu Egipskim.
\par 3 Bo moc Faraonowa bedzie wam ku zawstydzeniu, a ucieczka do cieniu Egipskiego ku pohanbieniu.
\par 4 Przeto, ze byli w Soan ksiazeta jego, a poslowie jego do Chanes chodzili.
\par 5 Wszystkich do hanby przywiedzie przez lud, który im nie bedzie ku dogodzie, ani ku pomocy, ani ku pozytkowi, ale tylko ku zelzywosci i ku hanbie.
\par 6 Brzemie odniosa na bydletach poludniowych do ziemi ucisku i utrapienia, (skad pochodzi lew i szcznie lwie, zmija i smok ognisty latajacy;) odniosa mówie na grzbietach bydlatek bogactwa swoje, i na garbie wielbladów skarby swoje, do ludu, który im nic nie pomoze;
\par 7 Bo Egipczanie daremno i prózno pomagac beda. Dlatego to oglaszam, ze ich moc jest, siedziec w pokoju.
\par 8 Terazze idz, napisz to na tablicy przed oczyma ich, a na ksiegach to wyrysuj, aby to trwalo do dnia ostatniego, i az na wieki wieków:
\par 9 Ze ten lud jest odporny, synowie klamliwi, synowie, którzy nie chca sluchac zakonu Panskiego.
\par 10 Którzy mówia widzacym: Nie miewajcie widzenia; a prorokom: Nie prorokujcie nam, co prawego jest; mówcie nam rzeczy przyjemne, prorokujcie oszukanie;
\par 11 Ustapcie z drogi, zejdzcie z scieszki; niech bedzie daleki od oblicza naszego Swiety Izraelski.
\par 12 Przetoz tak mówi Swiety Izraelski: Iz gardzicie tem slowem, a ufacie w potwarzy i w przewrotnosci, i spolegacie na niej:
\par 13 Dlatego wam ta nieprawosc bedzie jako mur przerwany upadajacy, i jako wydecie na murze wysokim, którego bywa nagle i predkie obalenie;
\par 14 I pokruszy ja, jako sie kruszy stluczone naczynie garncarskie; a tak mu nie sfolguje, iz sie nie znajdzie po stluczeniu jego i skorupa, któraby mógl nabrac ognia z ogniska, albo naczerpac wody z kaluzy.
\par 15 Albowiem tak mówi panujacy Pan, Swiety, Izraelski: Jezli sie nawrócicie i uspokoicie sie, zachowani bedziecie; w milczeniu i w nadziei bedzie moc wasza. Ale nie chcecie;
\par 16 Owszem mówicie: Nie tak, ale na koniach ucieczemy; przetoz uciekac bedziecie. Na predkich koniach ujedziemy; ale predsi beda ci, którzy was gonic beda.
\par 17 Tysiac ich uciecze przed okrzyknieniem jednego, a przed okrzyknieniem pieciu wszyscy ucieczecie, az zostaniecie jako maszt na wierzchu góry, a jako choragiew na pagórku.
\par 18 A dlategoc Pan czekac bedzie, aby sie zmilowal nad wami, i dlatego sie wywyzszy, aby sie zlitowal nad wami; albowiem Pan jest Bogie sadu; blogoslawieni wszyscy, którzy nan oczekuja.
\par 19 Bo lud na Syonie i w Jeruzalemie bedzie mieszkac; plakac wiecej nie bedziesz. Zapewne zlituje sie nad toba na glos wolania twego (Pan), a skoro uslyszy, ozwiec sie.
\par 20 A choc wam Pan da chleb utrapienia, i wode ucisku, jednak nie odleca wiecej od ciebie nauczyciele twoi, ale oczy twoje patrzac beda na nauczycieli twoich;
\par 21 I uszy twoje uslysza slowo z tylu do ciebie mówiacego: Tac jest droga, chodzcie po niej, lubbyscie sie w prawo albo w lewo udali.
\par 22 Tedy zarzucicie okrycie srebrnych swoich balwanów rytych, i odzienie zlotych swoich balwanów odlewanych; rozproszysz je jako plugastwo niewiasty przyrodzona niemoc cierpiacej, a rzeczesz im: Precz stad.
\par 23 Da Bóg i deszcz na siewy twoje, któremibys posial ziemie, a chleb z urodzaju ziemi bedzie syty i obfity; dnia onego pasc sie beda i bydla twoje na pastwisku szerokiem.
\par 24 Woly takze i osly sprawujace ziemie, pastwe czysta jesc beda, która opalka i lopata wywiana bywa.
\par 25 I beda na kazdej górze wysokiej, i na kazdym pagórku wynioslym strumienie i potoki wód w dzien porazki wielkiej, gdy wieze upadna.
\par 26 Swiatlosc tez miesiaca bedzie jako swiatlosc sloneczna; a swiatlosc sloneczna bedzie w siedmiornasób, jako swiatlosc siedmiu dni, dnia, którego zawiaze Pan zlamanie ludu swego, a rane zbicia jego uleczy.
\par 27 Oto imie Panskie przychodzi z daleka, zapalila sie popedliwosc jego, i ciezka jest ku znoszeniu; wargi jego pelne sa gniewu, a jezyk jego jako ogien pozerajacy.
\par 28 A duch jego jest jako rzeka wylewajaca, która az do gardla siega, aby przewiewal narody, azby sie wniwecz obrócily, a wedzidlem kielznal czelusci narodów.
\par 29 Tedy zaspiewacie, jako gdy sie w nocy obchodzi uroczyste swieto, a rozweselicie sie w sercu jako ten, który idzie z piszczalka, idac na góre Panska, do skaly Izraelskiej;
\par 30 Gdy da uslyszec Pan wielmoznosc glosu swego, i wyciagnione ramie swoje okaze w popedliwosci gniewu swojego, i w plomieniu ognia pozerajacego z rozproszeniem, z gwaltownym dzdzem, i z gradem kamiennym.
\par 31 Bo od glosu Panskiego starty bedzie Assyryjczyk, który innym kijem bijal.
\par 32 I stanie sie, ze na samym kazde uderzenie kijowe, którem go Pan uderzy, znaczne bedzie, gdy z bebnami i z lutniami, i z bitwa wesola walczyc bedzie przeciwko niemu.
\par 33 Albowiem dawno juz jest nagotowane pieklo, i dla samego króla nagotowane jest; które glebokie i szerokie uczynil, podniaty jego ognia i drew sila jest; poddymanie Panskie jako rzeka siarczana zapala je.

\chapter{31}

\par 1 Biada tym, którzy zstepuja do Egiptu o pomoc, a na koniach spolegaja, i ufaja w wozach, ze ich wiele, i w jezdnych, iz sa mocni bardzo, a nie ogladaja sie na Swietego Izraelskiego, a Pana nie szukaja!
\par 2 Alec on tez jest madry, przetoz przywiedzie zle, a slów swoich nie odmieni; lecz powstanie przeciw domowi zlosników i przeciwko ratunkowi tych, którzy broja nieprawosc.
\par 3 Albowiem Egipczanie sa ludzie a nie Bóg, a konie ich cialo, a nie duch. Przetoz skoro Pan wyciagnie reke swa, padnie i pomocnik, padnie i ten, któremu dawaja pomoc; a tak wszyscy spolem zgina.
\par 4 Bo tak rzekl Pan do mnie: Jako gdy ryczy lew, i szczenie lwie nad lupem swym, a choc zwolywaja przeciwko niemu gromade pasterzy, przecie sie on wrzasku ich nie leka, ani sie korzy przed hukiem ich; tak Pan zastepów zstapi, aby walczyl o góre Syonska, i o pagórek jej.
\par 5 Jako ptaki lataja okolo gniazda swego, tak obroni Pan zastepów Jeruzalem, i owszem, broni i wybawia, a przechodzac z pomste zachowa.
\par 6 Nawróccie sie do tego, od którego gleboko zabrneli synowie Izraelscy,
\par 7 Albowiem dnia onego odrzuci kazdy balwany swe srebrne, i balwany swe zlote, które wam naczynily rece wasze na grzech.
\par 8 I upadnie Assyryjczyk od miecza nie meskiego, a miecz nie czlowieczy pozre go: i uciecze przed mieczem, a mlodziency jego holdownikami beda.
\par 9 A tak opoke swoje od strachu minie, a ksiazeta jego ulekna sie przed choragwia, mówi Pan, którego ogien jest na Syonie, a piec w Jeruzalemie.

\chapter{32}

\par 1 Oto król bedzie królowal w sprawiedliwosci, a ksiazeta w sadzie panowac beda.
\par 2 Bo maz on bedzie jako zaslona od wiatru, i jako zakrycie przed powodzia; jako strumienie wód na miejscu suchem, jako cien skaly wielkiej w ziemi upragnionej;
\par 3 I nie beda sie blakac oczy widzacych, i uszy sluchajacych pilnie sluchac beda.
\par 4 Serce glupich zrozumie umiejetnosc, a jezyk jakajacych sie predko i rzetelnie mówic bedzie.
\par 5 I nie beda wiecej zwac nieszlachetnego szlachetnym, a skapy nie bedzie slyna szczodrym.
\par 6 Przeto, ze nieszlachetny o nieszlachetnosci mówi, a serce jego zmysla nieprawosc, aby wykonal obludnosc, a mówil przeciwko Panu zdroznie; aby wyniszczyl dusze laknacego, a napój pragnacego odjal.
\par 7 Skapego tez usilowania zle sa: bo chytrze obmysla, jakoby wniwecz obrócil utrapionych slowy klamliwemi, i mówil przeciwko nedznemu przed sadem.
\par 8 Ale szczodrobliwy o szczodrobliwosci mysli, a przy szczodrobliwosci stac bedzie.
\par 9 Niewiasty spokojne! powstancie, sluchajcie glosu mego; córki bezpieczne! bierzcie w uszy swe powiesci moje.
\par 10 Przez wiele dni i lat trwozyc sie bedziecie, wy bezpieczne! albowiem ustanie zbieranie wina, a sprzatania urodzajów nie bedzie.
\par 11 Zatrwozcie sie, a uleknijcie sie, bezpieczne! zewleczcie sie, i obnazcie sie, a przepaszcie biodra wasze.
\par 12 Kwilac nad piersiami, nad rolami rozkosznemi, i nad winna macica urodzajna.
\par 13 Na ziemi ludu mojego ciernie i oset wyrosnie, owszem, na wszystkich domach wesolych miasta radujacego sie.
\par 14 Albowiem palac opuszczony bedzie, huk miasta ustanie, zamek i baszty jaskiniami zostana az na wieki, na radosc dzikim oslom i na pastwiska trzodom.
\par 15 Póki nie bedzie wylany na nas duch z wysokosci, a nie obróci sie pustynia w pole urodzajne, a pole urodzajne za las poczytane nie bedzie.
\par 16 I bedzie sad przemieszkiwal na puszczy, a sprawiedliwosc pole urodzajne osiadzie.
\par 17 I bedzie pokój dzielo sprawiedliwosci, a skutek sprawiedliwosci odpocznienie i bezpiecznosc az na wieki.
\par 18 Bo bedzie mieszkal lud mój w przybytku pokoju, i w przybytkach bezpiecznych, i w odpoczywaniu spokojnem.
\par 19 Chocby i grad spadl na las, a miasto bardzo ponizone bylo.
\par 20 Blogoslawieni jestescie, którzy siejecie na wszelakich miejscach urodzajnych, wpuszczajac tam woly i osly.

\chapter{33}

\par 1 Biada tobie, który lupisz, chociazes sam nie zlupiony, i który zdradzasz, chociazes sam nie byl zdradzony! Gdy lupic przestaniesz, bedziesz tez zlupiony; gdy zdradzac przestaniesz, beda cie tez zdradzac.
\par 2 Panie! zmiluj sie nad nami, ciebie oczekujemy. Badz ramieniem swoich na kazdy poranek, a zbawieniem naszem czasu utrapienia.
\par 3 Przed glosem ogromnym rozpierzchna sie narody; przed wywyzszeniem twojem rozprosza sie poganie.
\par 4 I beda zebrane lupy wasze, jako zbieraja chrzaszcze; a jako przypada szarancza, tak oni przypadna na nie.
\par 5 Pan bedzie wywyzszony, bo mieszka na wysokosci; napelni Syon sadem i sprawiedliwoscia.
\par 6 Madrosc i umiejetnosc beda utwierdzeniem czasów twoich, sila i obfitem zbawieniem twem, a bojazn Panska skarbem twoim.
\par 7 Oto mocarze ich wolaja na dworze, poslowie pokoju gorzko placza.
\par 8 Spustoszaly drogi, przestano scieszka chodzic; zlamal przymierze, zniewazyl miasta, a czlowieka za nic sobie nie ma.
\par 9 Plakala i zwatlala ziemia; zawstydzony jest Liban i uwiadl; Saron sie stal jako pustynia, i otluczono Basan i Karmel.
\par 10 Teraz powstane, mówi Pan, teraz sie wywyzsze, teraz sie podniose.
\par 11 Poczawszy slome, urodzicie mierzwe; duch wasz was pozre jako ogien.
\par 12 I beda narody, jako wypalone wapno; beda jako ciernie wyciete, ogniem spalone.
\par 13 Sluchajcie, którzyscie daleko, com uczynil, a bliscy poznajcie moc moje.
\par 14 Zlekli sie na Syonie grzesznicy, strach zdjal obludników mówiacych: Któz z nas ostac sie moze przed ogniem pozerajacym? Któz z nas ostac sie moze przed plomieniem wiecznym?
\par 15 Ten, który chodzi w sprawiedliwosci, a mówi, co jest prawego; który sie zyskiem niesprawiedliwym brzydzi; który otrzasa rece swe, aby darów nie bral; który zatula uszy swe, aby nie sluchal o rozlaniu krwi, i zamruza oczy swoje, aby nie patrzal na zle:
\par 16 Ten na wysokosciach mieszkac bedzie, zamki na skalach beda ucieczka jego; chleb jego dany mu bedzie, wody jego nie ustana.
\par 17 Króla w pieknosci jego ogladaja oczy twoje, ujrza i ziemie daleka.
\par 18 Serce twoje bedzie rozmyslalo o starchu, mówiac: Gdzie teraz jest pisarz? gdziez teraz jest poborca? gdziez jest obliczajacy wieze?
\par 19 Ludu okrutnego nie ogladasz, ludu glebokiej mowy, któregos nie slyszal, i jezyka obcego, któregobys nie rozumial.
\par 20 Wejrzyj na Syon, miasto uroczystych swiat naszych, oczy twoje niechaj patrza na Jeruzalem, na mieszkanie spokojne, na namiot, który nie bedzie przeniesion; kolki jego na wieki sie nie porusza, a zaden powróz jego nie zerwie sie,
\par 21 Przeto, ze nam na tem miejscu Pan wielmozny jest rzekami strumieni szerokich, po których nie pójda z wioslami, ani okret wielki po nich przechodzic bedzie.
\par 22 Bo Pan jest sedzia nasz, Pan zakonodawca nasz; Pan król nasz; on nas zbawi.
\par 23 Oslabialy powrozy twoje, nie beda mogly w klubie zatrzymac masztu twego, ani rozciagna zaglów. Tedy rozdzielone beda lupy korzysci wielkiej, ze i chromi rozchwyca lupy.
\par 24 A nie rzecze zaden z obywateli: Zachorowalem; lud, który mieszka w nim, uwolniony bedzie od nieprawosci.

\chapter{34}

\par 1 Przystapcie, narody! ku sluchaniu, a wy ludzie pilnie uwazajcie! Niech slucha ziemia, i pelnosc jej, okrag ziemi, i wszystko, co sie rodzi na niej.
\par 2 Bo rozgniewanie Panskie jest na wszystkie narody, a popedliwosc jego na wszystko wojsko ich; wytraci je jako przeklete, a poda je na zabicie.
\par 3 I beda wyrzuceni pobici ich, a z trupów ich smród wynijdzie, a krew ich z gór poplynie.
\par 4 I niszczec bedzie wszystko wojsko niebieskie, a niebiosa jako ksiegi zwinione beda, i wszystko wojsko ich opadnie, jako opada lisc z winnej macicy, i jako opada niedojrzaly owoc z figowego drzewa.
\par 5 Albowiem opojony jest na niebie miecz mój; oto zstapi na Edomczyków, i na sad ludu przekletego odemnie.
\par 6 Miecz Panski pelny bedzie krwi, utlusci sie w loju i we krwi baranków i kozlów, w loju nerek baranich; bo ofiara Panska bedzie w Bocra, a porazka wielka w ziemi Edomskiej.
\par 7 Zstapia z nimi i jednorozce, i byki z wolami, i opojona bedzie krwia ziemia ich, a proch ich bedzie opojony tukiem.
\par 8 Albowiem to dzien pomsty Panskiej bedzie, i rok odplaty, aby sie pomszczono Syonu.
\par 9 I obróca sie potoki jej w smole, a proch jej w siarke, a ziemia jej obróci sie w smole gorejaca;
\par 10 Ani w nocy ani we dnie nie zagasnie, na wieki bedzie wystepowal dym jej; od narodu do narodu pusta zostanie; na wieki wieczne nie bedzie, ktoby szedl przez nia.
\par 11 Ale ja pelikan i bak posieda, a sowa i kruk mieszkac w niej beda; i rozciagnie po niej sznur spustoszenia, i wagi próznosci.
\par 12 Szlachty jej na królestwo wzywac beda, ale nie bedzie tam zadnego; bo wszyscy ksiazeta jej wniwecz sie obróca.
\par 13 I urosna na palacach ich ciernie, pokrzywy i oset na zamkach ich; i bedzie przybytkiem smoków, a mieszkaniem strusiów.
\par 14 Tam sie beda potykaly dzikie zwierzeta z koczkodanami, i pokusa jedna drugiej ozywac sie bedzie; tam lezec bedzie jedza, a znajdzie sobie odpocznienie.
\par 15 Tam sobie sep gniazdo uczyni, zniesie jajka, i wyleze, a schowa pod cien swój; tamze sie zleca kanie jedna do drugiej.
\par 16 Szukajciez w ksiegach Panskich, a czytajcie; ani jedno z tych nie uchybi, a jedno bez drugiego nie bedzie; albowiem usta Panskie to rozkazaly, a duch jego sam zgromadzi je.
\par 17 Bo im on los rzucil, a reka jego one im sznurem rozmierzyla; az na wieki dziedzicznie ja posiada, od narodu do narodu mieszkac w niej beda.

\chapter{35}

\par 1 Weselic sie z tego bedzie pustynia i miejsce lesne, a rozraduje sie i zakwitnie jako róza.
\par 2 Slicznie zakwitnie, i radujac sie weselic sie bedzie z wykrzykaniem; chwala Libanu bedzie jej dana, i ozdoba Karmelu i Saronu. One ujrza chwale Panska i ozdobe Boga naszego.
\par 3 Umacniajcie rece oslabiale, a kolana zemdlale posilajcie.
\par 4 Mówcie do zatrwozonych w sercu: Zmocnijcie sie, nie bójcie sie; oto Bóg wasz z pomsta przyjdzie; z nagroda Bóg sam przyjdzie, i zbawi was.
\par 5 Tedy sie otworza oczy slepych, a uszy gluchych tworzone beda.
\par 6 Tedy poskoczy chromy jako jelen, a niemych jezyk spiewac bedzie; albowiem wody na puszczy wynikna, a potoki na pustyniach.
\par 7 I stanie sie miejsce suche jeziorem, a bezwodne zródlami wód; w lozyskach smoków, kedy legali, trawa, trzcina, i sitowie rosc bedzie.
\par 8 I bedzie tam droga i scieszka, która droga swieta slynac bedzie; nie pójdzie po niej nieczysty, ale bedzie dla onych samych. Którzy ta droga pójda, i glupi nawet, nie zbladza.
\par 9 Nie bedzie tam lwa, a okrutny zwierz nie bedzie chodzil po niej, ani sie tam znajdzie; ale wybawieni po niej chodzic beda.
\par 10 Odkupieni, mówie, Panscy nawróca sie, i przyjda na Syon z spiewaniem, a wesele wieczne bedzie na glowie ich; radosc i wesele otrzymaja, a zalosc i smutek uciecze.

\chapter{36}

\par 1 I stlo sie czternastego roku królowania Ezechyjasza, ze przyciagnal Sennacheryb, król Assyryjski, przeciwko wszystkim miastom Judzkim obronnym, i pobral je.
\par 2 I poslal król Assyryjski Rabsacesa z Lachys, do Jeruzalemu, do króla Ezechyjasza z wielkiem wojskiem, który stanal u rur sadzawki wyzszej przy drodze pola blecharzowego.
\par 3 Tedy wyszedl do niego Elijakim, syn Helkijaszowy, przelozony nad domem, i Sobna pisarz, i Joach, syn Asafowy, kanclerz.
\par 4 I rzekl do nich Rabsaces: Prosze, powiedzcie Ezechyjaszowi: Tak mówi król wielki, król Assyryjski: Cóz to za ufnosc, która ufasz?
\par 5 Rzeklbym: (Acz to rzecz daremna) Snac rady i mocy do wojny dosyc masz; ale w kimze ufasz, ze mi sie sprzeciwiasz?
\par 6 Otos spolegl na lasce tej trzciny nalamanej, na Egipcie, która jezliby sie kto podparl, wnijdzie w reke jego, i przekole ja. Takic jest Farao, król Egipski, wszystkim, którzy w nim ufaja.
\par 7 A jezli mi rzeczesz: W Panu, Bogu naszym, ufamy; azaz nie ten jest, którego zniósl Ezechyjasz wyzyny i oltarze, i przykazal Judzie i Jeruzalemowi mówiac: Przed tym oltarzem klaniac sie bedziecie?
\par 8 Przetoz teraz prosze, zarecz sie Panu memu, królowi Assyryjskiemu, a ja tobie dam dwa tysiace koni, bedzieszli je mógl osadzic jezdnymi.
\par 9 I jakoz sie ty mozesz oprzec hetmanowi jednemu najmniejszemu z slug pana mego, choc ufasz w Egipcie dla wozów i jezdnych?
\par 10 Nadto czy bez woli Panskiej przyciagnalem do tej ziemi, abym ja spustoszyl? Pan rzekl do mnie: Ciagnij do tej ziemi, a spustosz ja.
\par 11 Tedy rzekl Elijakim, i Sobna, i Joach do Rabsacesa: Prosze, mów do slug twoich po syryjsku, wszak rozumiemy, a nie mów do nas po zydowsku przed tym ludem, który jest na murze.
\par 12 I odpowiedzial Rabsaces: Azaz mie do Pana twego albo do ciebie poslal Pan mój, abym te slowa mówil? Poslal mie raczej do mezów, którzy siedza na murze, aby jedli lajna swoje, a mocz swój pospolu z wami pili.
\par 13 A tak stanal Rabsaces i wolal glosem wielkim po zydowsku, mówiac: Sluchajcie slów króla wielkiego, króla Assyryjskiego.
\par 14 Tak mówi król: Niech was nie zwodzi Ezechyjasz: bo was nie bedzie mógl wybawic.
\par 15 A niech wam nie rozkazuje Ezechyjasz ufac w Panu, mówiac: Zapewne nas Pan wybawi, a nie bedzie to miasto podane w reke króla Assyryjskiego.
\par 16 Nie sluchajciez Ezechyjasza; albowiem tak powiedzial król Assyryjski: Uczyncie zemna przymierze, a wynijdzcie do mnie, a jedz kazdy z was z winnicy swojej, i kazdy z figowego drzewa swego, a pij kazdy z was wode z studni swojej;
\par 17 Az przyjde a pobiore was do ziemi podobnej ziemi waszej, do ziemi zboza i wina, do ziemi chleba i winnic.
\par 18 Niech was nie zwodzi Ezechyjasz, mówiac: Pan nas wybawi. Izaz mogli bogowie narodów wybawic kazdy ziemie swoje z reki króla Assyryjskiego?
\par 19 Gdziez sa bogowie Emat i Arfad? Gdzie sa bogowie Sefarwaim? Azaz wybawil Samaryje z reki mojej?
\par 20 Którzyz sa miedzy wszystkimi bogami tych ziem, którzyby wydarli ziemie swoje z reki mojej? A mialby Pan wybawic Jeruzalem z reki mojej?
\par 21 Ale oni milczeli, i nie odpowiedzieli mu i slowa; bo takie bylo rozkazanie królewskie, mówiac: Nie odpowiadajcie mu.
\par 22 I przyszedl Elijakim, syn Helkijaszowy, przelozony domu, i Sobna pisarz, i Joach, syn Asafowy, kanclerz, do Ezechyjasza, rozdarlszy szaty swe, i oznajmili mu slowa Rabsacesowe.

\chapter{37}

\par 1 A gdy to uslyszal król Ezechyjasz, rozdarl szaty swoje, a oblóklszy sie w wór, wszedl do domu Panskiego.
\par 2 I poslal Elijakima, sprawce domu swego, i Sobne pisarza, i starszych z kaplanów obleczonych w wory, do Izajasza proroka, syna Amosowego.
\par 3 Którzy rzekli do niego: Tak mówi Ezechyjasz: Dzien ten jest dzien utrapienia, i lajania, i bluznienia; albowiem synowie przyszli az do porodzenia, ale sily niemasz ku rodzeniu.
\par 4 Oby uslyszal Pan, Bóg twój, slowa Rabsacesowe, którego poslal król Assyryjski, pan jego, aby uragal Bogu zyjacemu, i pomscil sie Pan Bóg twój, tych slów, które slyszal! Przetoz uczyn modlitwe za te ostatki ludu, które sie znajduja.
\par 5 Przyszly tedy sludzy króla Ezechyjasza do Izajasza;
\par 6 Którym odpowiedzial Izajasz: Tak powiedzcie Panu waszemu, tak mówi Pan: Nie bój sie tych slów, któres slyszal, któremi mie lzyli sludzy króla Assyryjskiego.
\par 7 Oto ja mu dam innego ducha, aby uslyszawszy wiesc nawrócil sie do ziemi swojej; i sprawie to, ze polegnie od miecza w ziemi swojej.
\par 8 Ale Rabsaces wróciwszy sie znalazl króla Assyryjskiego dobywajacego Lebny; albowiem uslyszal, iz odciagnal byl od Lachys.
\par 9 A uslyszawszy o Tyraku, królu Etyjopskim, ze mówiono: Oto ciagnie, aby walczyl przeciwko tobie; uslyszawszy to, mówie, przecie poslal poslów do Ezechyjasza z temi slowy:
\par 10 To powiedzcie Ezechyjaszowi, królowi Judzkiemu, mówiac: Niech cie nie zwodzi Bóg twój, któremu ty ufasz, a mówisz: Nie bedzie podane Jeruzalem w rece króla Assyryjskiego.
\par 11 Otos slyszal, co poczynili królowie Assyryjscy wszystkim ziemiom, które wygladzili; a tybys mial byc wybawiony?
\par 12 Izali je wybawili bogowie tych narodów, które wygubili ojcowie moi: Gozan, i Haran, i Resef, i synów Eden, którzy byli w Telassar?
\par 13 Gdziez jest król Elmat, i król Arfad, i król miasta Sefarwaim, Ana, i Awa?
\par 14 Przetoz wziawszy Ezechyjasz list z reki poslów, przeczytal go, a wszedlszy do domu Panskiego, rozciagnal go Ezechyjasz przed Panem.
\par 15 I modlil sie Ezechyjasz Panu, mówiac:
\par 16 Panie zastepów, Boze Izraelski, siedzacy na Cherubinach! Ty, tys sam jest Bóg wszystkich królestw ziemi, tys stworzyl niebo i ziemie.
\par 17 Naklonze, Panie! ucha twego, a uslysz; otwórz, Panie! oczy twoje, a obacz; uslysz wszystkie slowa Sennacherybowe, który przyslal hanbic ciebie, Boga zywego.
\par 18 Prawdac jest, Panie! ze sputoszyli królowie Assyryjscy wszystkie te krainy, i ziemie ich;
\par 19 I powrzucali bogów ich w ogien; albowiem nie byli bogami, ale robota rak ludzkich, drewno i kamien; przetoz ich wygubili.
\par 20 A teraz, o Panie, Boze nasz! wybaw nas z reki jego, aby poznaly wszystkie królestwa ziemi, zes ty, Panie! sam Bogiem.
\par 21 Tedy poslal Izajasz, syn Amosowy, do Ezechyjasza, mówiac: Tak mówi Pan Bóg Izraelski: O cos mie prosil z strony Sennacheryba, króla Assyryjskiego,
\par 22 Tedy to jest slowo, które mówil Pan o nim: Panna, córka Syonska, wzgardzila cie, smiala sie z ciebie, kiwala glowa za toba córka Jeruzalemska.
\par 23 Kogozes hanbil, i kogos bluznil? przeciwko komuzes podniósl glos, i wyniosles ku górze oczy swe? przeciwko Swietemu Izraelskiemu.
\par 24 Przez slugi twoje hanbiles Pana, i mówiles: W mnóstwie wozów moich wstapilem ja na wysokie góry, na strony Libanskie, i porabie wysokie cedry jego, i wyborne jodly jego; i wnijde na same wysokosc wierzchu jego, do lasów, i urodzajnych ról jego.
\par 25 Jam wykopal zródla i pilem wody, a wysuszylem stopami nóg moich wszystkie potoki miejsc oblezonych.
\par 26 Izazes nie slyszal, zem to z dawna uczynil, i ode dni starodawnych to sprawil? A teraz do tego przywodze, aby w pustynie i w kupy rumu miasta obronne obrócone byly.
\par 27 A obywatele ich rece skurczone majac, przestraszeni sa i zawstydzeni, stali sie jako trawa polna, i jako ziele wschodzace, i trawy na dachach, a siewy rdza zepsowane, pierwej nizeliby dorosly.
\par 28 Mieszkanie twoje, i wyjscie twoje, i wejscie twoje znam, i popedliwosc twoje przeciwko sobie.
\par 29 Poniewazes sie przeciwko mnie zajuszyl, a zapedy twoje przyszly do uszów moich, przetoz zaloze kolce moje za nozdrza twoje, a wedzidlo moje wprawie w gebe twoje, i wróce cie ta droga, któras przyszedl.
\par 30 A to miej za znak, Ezechyjaszu! Tego roku jesc bedziesz samorodne zboze, takze i drugiego roku samorodne zboze; ale roku trzeciego bedziecie siac i zac, i winnice sadzic, i pozywac owoce ich.
\par 31 Ostatek bowiem domu Judy, który pozostal, wkorzeni sie gleboko, i wyda owoc ku górze.
\par 32 Albowiem z Jeruzalemu wyjda ostatki, i zachowani z góry Syonskiej. Gorliwosc Pana zastepów to uczyni.
\par 33 Przetoz tak mówi Pan o królu Assyryjskim: Nie wnijdzie do miasta tego, ani tam strzaly wystrzeli, ani go zaprzatnie tarcza, ani usypie okolo niego szanców.
\par 34 Droga, która przyszedl, zas sie wróci, a do miasta tego nie wnijdzie, mówi Pan.
\par 35 Bo bede bronil miasta tego, i zachowam je sam dla siebie, i dla Dawida, slugi mego.
\par 36 Tedy wyszedl Aniol Panski, i pobil w obozie Assyryjskim sto osmdziesiat, i piec tysiecy; a gdy wstali bardzo rano, oto wszedy pelno trupów.
\par 37 Przetoz ruszywszy sie, odjechal, i wrócil sie Sennacheryb, król Assyryjski, a mieszkal w Niniwie.
\par 38 A gdy chwalil Nesrocha, boga swego, w domu, tedy Adramelach i Sarasar, synowie jego, zabili go mieczem, a sami uciekli do ziemi Ararat; a królowal Assarhaddon, syn jego, miasto niego.

\chapter{38}

\par 1 W one dni zachorowal Ezechyjasz az na smierc. I przyszedl do niego Izajasz prorok, syn Amosowy, a rzekl do niego: Tak mówi Pan: Rozpraw dom swój; albowiem umrzesz, a nie zostaniesz zyw.
\par 2 Tedy obrócil Ezechyjasz twarz swoje do sciany, a modlil sie Panu.
\par 3 I rzekl: Prosze, o Panie! wspomnij teraz, zem chodzil przed toba w prawdzie i w sercu uprzejmem, czyniac to, co dobrego jest w oczach twoich. I plakal Ezechyjasz placzem wielkim.
\par 4 I stalo sie slowo Panskie do Izajasza, mówiac:
\par 5 Idz, a powiedz Ezechyjaszowi: Tak mówi Pan, Bóg Dawida, ojca twego: Wysluchalem modlitwe twoje, widzialem lzy twoje; oto Ja przyczynie do dni twoich pietnascie lat;
\par 6 I z reki króla Assyryjskiego wyrwe ciebie i to miasto, a bede bronil miasta tego.
\par 7 A to bedziesz mial za znak od Pana, ze Pan uczyni to, co mówil.
\par 8 Oto Ja wróce nazad cien po stopniach po których szedl, na zegarze slonecznym Achazowym na dziesiec stopni po tychze stopniach, po których bylo zeszlo.
\par 9 Pisanie Ezechyjasza, króla Judzkiego, gdy byl zachorowal i wyzdrowial z niemocy swojej:
\par 10 Jam rzekl w ukróceniu dni moich: Wnijde do bram grobu, pozbawion bede ostatka lat swoich;
\par 11 Rzeklem, ze nie ujrze Pana, Pana w ziemi zyjacych; nie ogladam wiecej czlowieka miedzy obywatelami na swiecie.
\par 12 Pobyt mój pomija, a przenosi sie odemnie, jako namiot pasterski; oderznalem zywot swój, jako tkacz; od krosien oderznie mie; dzis, pierwej niz noc nadejdzie, dokonasz mie.
\par 13 Rozmyslalem sobie z poranku, ze jako lew potrze wszystkie kosci moje, dzis, pierwej niz noc nadejdzie, dokonasz mie.
\par 14 Jako zóraw i jaskólka szczebiotalem, stekalem jako golebica; oczy moje ku górze podniesione byly, i rzeklem: Panie! gwalt cierpie, przedluz mi zywota.
\par 15 Ale cóz mam wiecej rzec? Onci mi odpowiedzial, i sam uczynil, ze zyc bede mimo wszystkie lata swe po gorzkosci duszy mojej.
\par 16 Panie! kto po nich i w nich zyc bedzie, wszystkim znajomy bedzie zywot dychania mego, zes mi zdrowie przywrócil, a zachowales mie przy zywocie.
\par 17 Oto czasu pokoju przyszla na mie byla gorzkosc najgorzciejsza; ale sie tobie podobalo wyrwac dusze moje z przepasci skazenia, przeto, zes zarzucil w tyl swój wszystkie grzechy moje.
\par 18 Albowiem nie grób wyslawia cie, ani smierc chwali cie, ani ci, którzy w dól wstepuja, oczekuja prawdy twojej.
\par 19 Zywy, zywy, ten cie wyslawiac bedzie, jako ja dzisiaj, a ojciec synom oznajmi prawde twoje.
\par 20 Pan mie wybawil; przetoz piesn moje spiewac bedziemy po wszystkie dni zywota naszego w domu Panskim.
\par 21 I rzekl byl Izajasz: Niech wezma bryle suchych fig, i przyloza na wrzód, a bedzie uzdrowiony.
\par 22 I rzekl byl Ezechyjasz: Cóz jest za znak, ze wstapie do domu Panskiego?

\chapter{39}

\par 1 Onego czasu poslal Merodach Baladan, syn Baladanowy, król Babilonski, list i dary do Ezechyjasza; bo zaslyszal, ze zachorowawszy zas ozdrowial.
\par 2 I weselil sie z tego Ezechyjasz, i ukazal im skarbnice klejnotów swoich srebra i zlota, i rzeczy wonnych, i olejki najwyborniejsze, takze i dom rynsztunków swoich, i cokolwiek sie znajdowalo w skarbach jego: nie bylo nic, czegoby im nie ukazal Ezechyjasz w domu swym, i we wszystkiem panstwie swojem.
\par 3 Wtem przyszedl Izajasz prorok do króla Ezechyjasza, i rzekl mu: Coc powiedzieli ci mezowie, i skad przyszli do ciebie? I odpowiedzial Ezechyjasz: Z ziemi dalekiej przyszli do mnie, z Babilonu.
\par 4 Nadto rzekl: Cóz widzieli w domu twoim? Odpowiedzial Ezechyjasz: Wszystko, co jest w domu moim, widzieli; niemasz nic, czegobym im nie ukazal w skarbach moich.
\par 5 Tedy rzekl Izajasz do Ezechyjasza: Sluchaj slowa Pana zastepów:
\par 6 Oto przyjda te dni, w które zabiora wszystko do Babilonu, cokolwiek jest w domu twoim, i cokolwiek zachowali ojcowie twoi az do dnia tego; nie zostanie nic, mówi Pan;
\par 7 Ale i synów twoich, którzy wyjda z ciebie, których splodzisz, pobiora, i beda komornikami na dworze króla Babilinskiego.
\par 8 Tedy rzekl Ezechyjasz do Izajasza: Dobre jest slowo Panskie, któres mówil; (i dolozyl: Dobre,)przeto, ze pokój i prawda bedzie za dni moich.

\chapter{40}

\par 1 Cieszcie, cieszcie lud mój! mówi Bóg wasz.
\par 2 Mówcie do serca Jeruzalemu: oglaszajcie mu, ze sie juz dopelnil czas postanowiony jego, ze jest odpuszczona nieprawosc jego, i ze wzial z reki Panskiej w dwójnasób za wszystkie grzechy swoje.
\par 3 Glos wolajacego na puszczy: Gotujcie droge Panska, prosta czyncie na pustyni scieszke Boga naszego.
\par 4 Kazda dolina niech podniesiona bedzie, a kazda góra i pagórek niech ponizony bedzie; co jest krzywego, niech sie wyprostuje, a miejsca nierówne niech beda równina.
\par 5 Bo sie objawi chwala Panska, a ujrzy wszelkie cialo spolem, iz usta Panskie mówily.
\par 6 Glos mówiacego: Wolaj. I rzekl: Cóz mam wolac? To: Wszelkie cialo jest trawa, a wszystka zacnosc jego jako kwiat polny.
\par 7 Trawa usycha, kwiat opada; skoro wiatr Panski powionie nan; zaprawdec ludzie sa ta trawa.
\par 8 Trawa usycha, kwiat opada; ale slowo Boga naszego trwa na wieki.
\par 9 Wstap sobie na góre wysoka, Syonie! który opowiadasz rzeczy ucieszne. Podnies mocno glos twój, Jeruzalemie! które opowiadasz rzeczy pocieszne; podnies, nie bój sie, rzecz miastom Judzkim: Oto Bóg wasz.
\par 10 Oto panujacy Pan przyjdzie przeciwko mocnemu, a ramie jego panowac bedzie nad nim; oto zaplata jego z nim, a dzielo jego przed nim.
\par 11 Jako pasterz trzode swoje pasc bedzie; do narecza swego zgromadzi baranki, i na lonie swem piastowac je bedzie, a kotne zwolna poprowadzi.
\par 12 Kto zmierzyl wody garscia swoja, a niebiosa piedzia rozmierzyl? a kto proch ziemi miara zmierzyl? kto zwazyl na wadze góry, a pagórki na szalach?
\par 13 Któz doscignal ducha Panskiego, a kto radca jego byl, zeby mu oznajmil?
\par 14 Z kim wszedl w rade, zeby mu rozumu przydal, a nauczyl go sciezek sadu? Kto go nauczyl umiejetnosci, a droge wszelakiej roztropnosci ukazal mu?
\par 15 Oto narody sa jako kropla z wiadra, a jako proszek na szalach poczytane sa; wyspy jako najmniejsza rzecz porywa.
\par 16 I Liban nie wystarczylby ku wznieceniu ognia, i zwierzeta jego nie wystarczylyby na calopalenie.
\par 17 Wszystkie narody sa jako nic przed nim; za nic i za marnosc poczytane sa u niego.
\par 18 Komuz tedy podobnym uczynicie Boga? A jakie podobienstwo przyrównacie mu?
\par 19 Rzemieslnik uleje balwana a zlotnik zlotem go powlecze, i lancuszki srebrne do niego odleje.
\par 20 A ten, który dla ubóstwa nie ma co ofiarowac, obiera drzewo, któreby nie próchnialo, i rzemieslnika umiejetnego sobie szuka, aby wygotowal balwana rytego, któryby sie nie poruszyl.
\par 21 Izali nie wiecie? Izali nie slyszycie? Izali sie wam nie opowiada od poczatku? Izali nie zrozumiewacie od zalozenia gruntów ziemi?
\par 22 Ten, który siedzi nad okregiem ziemi, której obywatele sa jako szarancza; ten, który rozpostarl niebiosa jako cienkie plótno, a rozciagnal je, jako namiot ku mieszkaniu:
\par 23 Tenci ksiazat w niwecz obraca, sedziów ziemskich jako nic rozprasza.
\par 24 Ze nie bywaja szczepieni ani wsiani, ani sie tez wkorzeni w ziemi pien ich; i jako jedno powienie na nich, wnet usychaja, a wicher jako zdzblo unosi ich.
\par 25 Komuz mie tedy przyrównacie, abym mu byl podobny? mówi Swiety.
\par 26 Podniescie ku górze oczy wasze, a obaczcie! Kto to stworzyl? kto wywiódl w poczcie wojsko ich, a to wszystko z imienia przyzywa, wedlug wielkosci sily, i wielkiej mocy, tak, ze ani jedno z nich nie zginie?
\par 27 Przeczze tedy powiadasz, Jakóbie! przeczze tak mówisz Izraelu: Skryta jest droga moja przed Panem, a sprawa moja przed Boga mego nie przychodzi?
\par 28 Izali nie wiesz? izalis nie slyszal, ze Bóg wieczny Pan, który stworzyl granice ziemi, nie ustanie, ani sie spracuje, i ze nie moze byc doscigniona madrosc jego?
\par 29 Który dodaje spracowanemu sily, a tego, który nie ma zadnej sily, moc rozmnaza.
\par 30 Mlódz ustaje i omdlewa, a mlodziency w mlodosci upadaja:
\par 31 Ale którzy oczekuja Pana, nabywaja nowej sily; podnosza sie piórami jako orly, bieza a nie spracuja sie, chodza a nie ustawaja.

\chapter{41}

\par 1 Umilknijcie przedemna, wyspy! a narody niech sie posila. Niech przystapia a niech mówia: Przystapmy spolem do sadu
\par 2 Któz wzbudzil od wschodu slonca sprawiedliwego, i wezwal go, aby go nasladowal? Któz mu podbil narody, aby nad królami panowal, podawszy je jako proch pod miecz jego, a jako plewy rozproszone pod luk jego?
\par 3 Uganial sie z nimi, przeszedl spokojnie scieszke, po której nogami swemi nie chadzal.
\par 4 Któz to sprawil i uczynil? któz wzywal rodzaje od poczatku? Ja Pan, pierwszy i ostatni, Ja sam.
\par 5 Widzialy wyspy, i ulekly sie; konczyny ziemi zdumialy sie; zgromadzily sie, i zeszly sie.
\par 6 Jeden drugiemu pomagal, a bratu swemu mówil: Zmacniaj sie!
\par 7 A tak zmacnial teszarz zlotnika, blache mlotem gladzacego, kujacego na kowadle, mówiac: Do lutowania to dobre. Potem to stwierdzil gwozdziami, aby sie nie ruszylo.
\par 8 Ale ty, Izraelu, slugo mój! ty Jakóbie, któregom obral, nasienie Abrahama, przyjaciela mego!
\par 9 Ty, któregom pochwycil od konczyn ziemi, owszem, pominawszy przedniejszych ich, powolalem cie mówiac do ciebie: Slugas ty mój, obralem cie, a nie odrzucilem cie.
\par 10 Nie bój sie! bom Ja z toba. Nie lekaj sie! bom Ja Bogiem twoim. Zmocnie cie, a dam ci pomoc, i podepre cie prawica sprawiedliwosci swojej.
\par 11 Oto zawstydza sie, a beda pohanbieni wszyscy gniewem palajacy przeciwko tobie: stana sie jako nic, i zgina ci, którzy sie tobie sprzeciwiaja.
\par 12 Szukallibys ich, nie znajdziesz ich; ci, którzy sie sprzeciwiaja tobie, beda jako nic, a ci, którzy walcza z toba, w niwecz obróceni beda.
\par 13 Bom ja Pan, Bóg twój, trzymam cie za prawice twoje, a mówiec: Nie bój sie! Ja cie wspomoge.
\par 14 Nie bój sie, robaczku Jakóbie, garstko ludu Izraelskiego! Jac bede na pomocy, mówi Pan a odkupiciel twój, Swiety Izraelski.
\par 15 Otom cie uczynil jako wóz z zebami nowemi po obu stronach; i pomlócisz góry, a potrzesz je, a pagórki jako plewe polozysz.
\par 16 Przewiejesz je, wtem je wiatr porwie, a wicher rozproszy je; ale sie ty rozradujesz w Panu, w Swietym Izraelskim bedziesz sie chlubil.
\par 17 Ubogich i nedznych, którzy szukaja wody, a niemasz jej, których jezyk usechl od pragnienia, Ja Pan wyslucham ich; Ja, Bóg Izraelski, nie opuszcze ich.
\par 18 Otworze rzeki na miejscach wysokich, a zródla w posród równin; obróce pustynie w jeziora wód, a ziemie sucha w strumienie wód.
\par 19 Nasadze na puszczy cedrów, wybornych cedrów, sosien, i oliwnych drzew; nasadze pustynie jedlina, wiazem, i bukszpanem;
\par 20 Aby widzieli, i poznali, i uwazali, i zrozumieli, ze to reka Panska uczynila, i ze to Swiety Izraelski stworzyl.
\par 21 Przedlózcie sprawe wasze, mówi Pan; ukazcie mocne dowody swoje, mówi król Jakóbowy.
\par 22 Niech przystapi, a niech nam oznajmi to, co sie ma stac; rzeczy pierwsze, które byly, powiedzcie, abysmy uwazyli w sercu swem, a poznali cel ich; albo przynajmniej nam przyszle rzeczy oznajmijcie.
\par 23 Oznajmijcie, co ma przyjsc napotem, a poznamy, zescie bogowie; albo uczyncie co dobrego lub zlego, abysmy sie zdumiewali, gdybysmy to spolem widzieli.
\par 24 Otoscie wy zgola na nic, a sprawa wasza takze na nic nie jest; przetoz obrzydly jest ten, co was sobie obiera.
\par 25 Wzbudze od pólnocy lud, ten przyciagnie; i od wschodu slonca, ten wzywac bedzie imienia mego; oborzy sie na ksiazat jako na bloto, a podepcze ich, jako garncarz gline.
\par 26 Kto oznajmi od poczatku? tedy bedziemy wiedzieli; albo co bylo od dawnych czasów? tedy rzeczemi: Tys jest sprawiedliwy? Niemasz zgola nikogo, coby oznajmil, ani jest, ktoby sie dal slyszec, albo ktoby slyszal mowy wasze.
\par 27 Jam pierwszy, który Syonowi opowiadam: Oto, oto sa; a Jeruzalemowi dam opowiadaczy rzeczy pociesznych.
\par 28 Bo widze, ze niemasz nikogo, niemasz nikogo miedzy nimi, coby dal rade; acz sie ich pytaja, wszakze nie odpowiadaja i slowa.
\par 29 Oto ci wszyscy sa marnoscia, za nic nie stoja uczynki ich; wiatrem i próznoscia sa odlewane balwany ich.

\chapter{42}

\par 1 Oto sluga mój, spolegac bede na nim, wybrany mój, którego sobie upodobala dusza moja. Dam mu Ducha swego, on sad narodom wyda.
\par 2 Nie bedzie wolal, ani sie bedzie wywyzszal, ani bedzie slyszany na ulicy glos jego.
\par 3 Trzciny nalamanej nie dolamie, a lnu kurzacego sie nie dogasi; ale sad wyda wedlug prawdy.
\par 4 Nie zamroczy sie, ani ustanie, dokad nie wykona sadu na ziemi, a nauki jego wyspy oczekiwac beda.
\par 5 Tak mówi Bóg, Pan, który stworzyl niebiosa i rozpostarl je; który rozszerzyl ziemie, i co sie rodzi z niej; który daje tchnienie ludowi mieszkajacemu na niej, a ducha tym, co chodza po niej.
\par 6 Ja Pan wezwalem cie w sprawiedliwosci, i ujalem cie za reke twa; przetoz strzedz cie bede, i dam cie za przymierze ludowi, i za swiatlosc narodom.
\par 7 Aby otwieral oczy slepych, a wywodzil wiezniów z ciemnicy, i z domu wiezienia siedzacych w ciemnosciach.
\par 8 Ja Pan, toc jest imie moje, a chwaly mojej nie dam innemu, ani slawy mojej balwanom rytym.
\par 9 Oto pierwsze rzeczy przyszly, Ja tez nowe opowiadam, pierwej, niz sie zaczna, dam wam o nich slyszec.
\par 10 Spiewajcie Panu piesn nowa, chwala jego jest od konczyn ziemi, którzy sie plawicie po morzu, i wszystko, co w niem jest, wyspy i obywatele ich.
\par 11 Podniescie glos pustynie, i miasta jej, i wsi, w których mieszka Kedar; wykrzykajcie obywatele skal, z wierzchu gór wolajcie.
\par 12 Oddajcie czesc Panu, a chwale jego na wyspach opowiadajcie.
\par 13 Pan wynijdzie jako mocarz, jako maz waleczny wzruszy sie gorliwoscia; trabic, owszem krzyczec bedzie, a przeciw nieprzyjaciolom swoim meznie sobie pocznie,
\par 14 Mówiac: Milczalem dosc dlugo, jakobym nie slyszal, wstrzymywalem sie; ale juz jako rodzaca krzyczec bede, spustosze, i wszystkich oraz polkne.
\par 15 W pustynie góry i pagórki obróce, i wszystkie ziola ich posusze; obróce i rzeki w wyspy, a jeziora wysusze.
\par 16 I powiode slepych droga, której nie znali, a scieszkami, o których nie wiedzieli, poprowadze ich; obróce przed nimi ciemnosci w swiatlosc, a co nierównego, w równine. Toc jest, co im uczynie, a nie opuszcze ich.
\par 17 Cofna sie nazad, i zawstydza sie bardzo, którzy ufaja w balwanach rytych, którzy mówia obrazom litym: Wyscie bogowie nasi.
\par 18 O glusi! sluchajcie; a wy slepi! przejrzyjcie, abyscie widzieli.
\par 19 Któz slepy, jedno sluga mój? a kto gluchy, jedno posel mój, którego posylam? Któz tak slepy jako doskonaly, slepy, mówie, jako sluga Panski?
\par 20 Widzi wiele rzeczy, a wszakze nie zrozumiewa; otworzone ma uszy, wszakze nie slyszy.
\par 21 Pan go sobie upodobal dla sprawiedliwosci swojej; uwielbil go zakonem, i slawnym go uczynil.
\par 22 Ale ten lud jest zlupiony i rozszarpany, którego mlodzienców ile ich kolwiek jest, imaja, i do ciemnic podawaja; podani sa na lup, a niemasz ktoby ich wybawil; podani sa na rozchwycenie, ani jest, ktoby rzekl: Wróc ich zas.
\par 23 Któz to z was w uszy przyjmuje? kto zrozumiewa, aby czulszym byl napotem?
\par 24 Kto podal na rozszarpanie Jakóba, a Izraela lupiezcom? Izali nie Pan, przeciwko któremusmy zgrzeszyli? Bo nie chcieli drogami jego chodzic, ani sluchac zakonu jego.
\par 25 Dlatego nan Pan wylal popedliwosc gniewu swego, i gwaltowna wojne, a zapalil go w okolo, a wszakze nie poznal tego; zapalil go, mówie, a wszakze tego do serca nie przypuscil.

\chapter{43}

\par 1 Ale teraz tak mówi Pan, który cie stworzyl, o Jakóbie; i który cie uczynil, o Izraelu! Nie bój sie, bom cie odkupil, a wezwalem cie imieniem twojem; mójes ty.
\par 2 Gdy pójdziesz przez wody, bede z toba, a jezli przez rzeki, nie zaleja cie; pójdzieszli przez ogien, nie spalisz sie, a plomien nie imie sie ciebie.
\par 3 Bom Ja Pan, Bóg twój, Swiety Izraelski, zbawiciel twój. Dalem za cie na okup Egipt, ziemie Murzynska, i Sabe miasto ciebie.
\par 4 Zaraz jakos drogim uczyniony przed oczyma memi, jestes uwielbionym, a Jam cie umilowal; przetoz dalem ludzi za cie, i narody za zywot twój.
\par 5 Nie bój sie, bom Ja z toba; od wschodu slonca przyprowadze zas nasienie twoje, i od zachodu zgromadze cie.
\par 6 Rzeke pólnocnej stronie: Wróc: a poludniowi: Nie zabraniaj. Przywiedz zasie synów moich z daleka, a córki moje od konczyn ziemi;
\par 7 Kazdego, który sie nazywa imieniem mojem, i któregom ku chwale swojej stworzyl, któregom uksztaltowal, i któregom uczynil.
\par 8 Wywiedz lud slepy, który juz ma oczy i gluchy, który juz ma uszy.
\par 9 Wszystkie narody niech sie spolu zejda, i niech sie zgromadza ludzie. Któz jest miedzy nimi, coby to opowiedzial, a przeszle rzeczy nam oznajmil? Niech stawia swiadków swoich, a beda usprawiedliwieni; albo niech slysza i rzekna: Prawdac jest!
\par 10 Wyscie swiadkowie moi, mówi Pan, i sluga mój, któregom obral, abyscie wiedzieli i wierzyli mi, i zrozumieli, zem Ja jest, a ze przedemna nie byl stworzony Bóg, ani po mnie bedzie.
\par 11 Ja, Jam jest Pan, a niemasz oprócz mnie zbawiciela.
\par 12 Ja oznajmuje i wyswabadzam, i opowiadam, a niemasz nikogo miedzy wami z obcych bogów; i wyscie mi tego swiadkami, mówi Pan, zem ja Bóg.
\par 13 Pierwej niz dzien byl, Jam jest, a niemasz, ktoby wyrwal z reki mojej; gdy co uczynie, i któz to odwróci?
\par 14 Tak mówi Pan, odkupiciel wasz, Swiety Izraelski: Dla was posle do Babilonu, i oderwe wszystkie zawory, i Chaldejczyków z okretami, w których sie oni chlubia.
\par 15 Jam jest Pan, Swiety wasz; Stworzyciel Izraelowy, Król wasz.
\par 16 Tak mówi Pan, który sposobil na morzu droge, i scieszke na bystrych wodach.
\par 17 Który wywodzi wozy i konie, wojsko i sile; czyni, ze oraz upadaja, a nie powstawaja: gasna jako knot gasnie.
\par 18 Nie wspominajcie pierwszych rzeczy, a starodawnych nie uwazajcie.
\par 19 Oto Ja czynie rzecz nowa, a zaraz sie zjawi; izali tego nie poznacie Nadto sposobie na puszczy droge, a na pustyni rzeki.
\par 20 Chwalic mie bedzie zwierz polny, smoki, i sowy, zem wywiódl na puszczy wody, a rzeki na pustyni, abym dal napój ludowi memu, wybranemu ludowi memu.
\par 21 Lud ten, którym sobie stworzyl, chwale moje opowiadac bedzie;
\par 22 A tys mie nie wzywal, o Jakóbie! owszemes sobie utesknil ze mna, o Izraelu!
\par 23 Nie przywiodles mi bydlatka na calopalenie twoje, i ofiarami twemi nie uczciles mie; nie przymuszalem cie, abys mi sluzyl ofiarami sniednemi, anim cie obciazal tem, abys mi kadzil;
\par 24 Nie kupiles mi za pieniadze wonnych rzeczy, anis mie tlustoscia ofiar twoich opoil; ales mie obciazyl grzechami twemi, a zadales mi prace nieprawosciami twojemi.
\par 25 Ja, Ja sam gladze przestepstwa twoje dla siebie, a grzechów twoich nie wspomne.
\par 26 Przywiedz mi na pamiec, sadzmy sie spolu; powiedz ty, maszli, czembys sie usprawiedliwil?
\par 27 Ojciec twój pierwszy zgrzeszyl, a nauczyciele twoi wystapili przeciwko mnie.
\par 28 A tak zrzuce ksiazat z miejsc swietych, i podam na przeklestwo Jakóba, a Izraela na pohanbienie.

\chapter{44}

\par 1 A teraz sluchaj Jakóbie slugo mój! i ty, Izraelu! któregom wybral.
\par 2 Tak mówi Pan, który cie uczynil, i który cie uksztaltowal zaraz z zywota matki, i który cie wspomaga: Nie bój sie Jakóbie, slugo mój! i uprzejmy, któregom wybral.
\par 3 Bo wyleje wody na pragnacego, a potoki na sucha ziemie; wyleje Ducha mego na nasienie twoje, i blogoslawienstwo moje na potomki twoje.
\par 4 I rozkrzewia sie jako miedzy trawa, i jako wierzby przy ciekacych wodach.
\par 5 Ten rzecze: Jam jest Panski, a ów sie ozowie do imienia Jakóbowego, a inny sie zapisze reka swa Panu, i imieniem Izraelskim bedzie sie nazywal.
\par 6 Tak mówi Pan, król Izraelski, i odkupiciel jego, Pan zastepów: oprócz mnie niemasz Boga.
\par 7 Bo któz jako Ja oglasza i opowiada to, i sporzadza mi to, zaraz od onego czasu, jakom rozsadzil lud na swiecie? a kto przyszle rzeczy, i to, co ma byc, oznajmi im?
\par 8 Nie bójciez sie, ani soba trwozcie. Izalim wam tego z dawna nie oznajmil, i nie opowiedzial? Tegoscie wy mnie sami swiadkami. Izali jest Bóg oprócz mnie? Niemasz zaiste skaly; Ja o zadnej nie wiem.
\par 9 Tworzyciele balwanów wszyscy nic nie sa, i te najmilsze rzeczy ich nic im nie pomoga; czego oni sobie sami swiadkami bedac, nic nie widza, ani rozumieja, zeby sie wstydzic mogli.
\par 10 Kto tworzy boga, i balwana leje, do niczego sie to nie przygodzi.
\par 11 Oto wszyscy, i uczestnicy ich beda pohanbieni; owszem, rzemieslnicy ich, ci nad innych ludzi, chocby sie wszyscy zebrali i staneli, lekac sie musza, i spolem pohanbieni beda.
\par 12 Kowal kleszczami robi przy weglu, a mlotami ksztaltuje balwana; gdy go robi moca ramienia swego, az od glodu w nim i sily ustaja, ani pije wody, az i omdlewa.
\par 13 Ciesla zas rozciega sznur, znaczy sznurem farbowanym, i ociosuje toporem, i cyrklem rozmierza go, i czyni go na podobienstwo meza, i na podobienstwo pieknego czlowieka, aby mieszkal w domu.
\par 14 Narabie sobie cedrów, i bierze cyprys i dab, albo to, co jest najmocniejszego miedzy drzewem lesnem, albo wsadzi jawór, który za deszczem odrasta;
\par 15 I uzywa tego czlowiek do palenia, albo wziawszy z niego, ogrzewa sie przy nim, takze roznieca ogien, aby napiekl chleba, nadto z tegoz drzewa robi sobie boga, i klania mu sie; czyni z niego balwana, i kleka przed nim.
\par 16 Czesc jego pali ogniem, przy drugiej czesci jego mieso je, piecze pieczen i nasycony bywa; takze rozgrzewa sie, i mówi: Ehej! rozgrzalem sie, widzialem ogien.
\par 17 A z ostatku jego czyni boga, balwana swego; kleka przed nim, klania sie, i modli mu sie, mówiac: Wybaw mie, bos ty bóg mój.
\par 18 Nie wiedza, ani rozumieja, przeto, ze Bóg zaslepil oczy ich, aby nie widzieli, i serca ich, aby nie rozumieli.
\par 19 I nie uwazaja tego w sercu swojem, nie majaz to umiejetnosci ani baczenia, aby rzekli: Czesc z niego spalilem ogniem, a przy weglu jego napieklem chleba, upieklem mieso, i najadlem sie; i mamze ja z ostatku jego obrzydliwosc uczynic, a przed kloc em drewnianym klekac?
\par 20 Taki sie karmi popiolem, serce jego zwiedzione unosi go, aby nie mógl wybawic duszy swojej, ani rzec: Izali to nie jest oszukanie, co jest w prawicy mojej?
\par 21 Pomnijze na to, Jakóbie i Izraelu! przeto, zes ty jest sluga moim. Stworzylem cie, slugas ty mój; o Izraelu! nie zapomne na cie.
\par 22 Gladze nieprawosci twoje jako oblok, a grzechy twoje jako mgle; nawróc sie do mnie, bom cie odkupil.
\par 23 Spiewajcie niebiosa, bo to Pan uczynil; wykrzykajcie niskosci ziemi, zabrzmijcie chwale góry, las, i wszystkie drzewa w nim; albowiem Pan odkupil Jakóba, a w Izraelu slawnym sie uczynil.
\par 24 Tak mówi Pan, odkupiciel twój, i który cie utworzyl wnet z zywota matki: Ja Pan wszystko czynie, sam rozciagam niebiosa, rozposcieram ziemie moca swoja.
\par 25 Wniwecz obracam znamiona praktykarzów, i wieszczków do szalenstwa przywodze; i medrców na wstecz obracam, a umiejetnosc ich glupia czynie.
\par 26 Potwierdzam slowa slugi swego, a rade poslów swych wykonywam. Który mówie o Jeruzalemie: Mieszkac w niem bede; a o miastach Judzkich: Pobudowane beda; bo spustoszenia ich pobuduje;
\par 27 Który mówie glebinie: Wyschnij, Ja potoki twe wysusze;
\par 28 Który mówie o Cyrusie: On pasterz mój, bo wszystke wole moje wykona; i rzecze Jeruzalemowi: Bedziesz zbudowane; a kosciolowi: Bedziesz zalozony.

\chapter{45}

\par 1 To mówi Pan pomazancowi swemu Cyrusowi, którego prawice ujme, a poraze przed nim narody, i biodra królów rozpasze, a pootwieram przed nim wrota, i bramy nie beda zamkniete.
\par 2 Ja przed toba pójde, a krzywe drogi wyprostuje, wrota miedziane skrusze, a zawory zelazne porabie;
\par 3 I dam ci skarby skryte, i klejnoty schowane, abys poznal, zem Ja Pan, Bóg Izraelski, który cie przyzywam imieniem twojem.
\par 4 Dla slugi mego Jakóba, i dla Izraela, wybranego mego, nazwalem cie imieniem twojem, przezwiskiem twojem, chociaz mie nie znasz.
\par 5 Jam Pan, a niemasz zadnego wiecej, oprócz mnie niemasz zadnego Boga; przepasalem cie, aczkolwiek mie nie znasz:
\par 6 Aby poznali od wschodu slonca, i od zachodu slonca, iz niemasz zadnego oprócz mnie, Jam Pan, a niemasz zadnego wiecej;
\par 7 Który czynie swiatlosc, i stwarzam ciemnosci; sprawuje pokój, i stwarzam zle. Ja Pan czynie to wszystko.
\par 8 Spuscie niebiosa rose z góry, a obloki niech kropia sprawiedliwosc; niech sie otworzy ziemia, a niech wyrosnie zbawienie, a sprawiedliwosc niech wespól zakwitnie. Ja Pan sprawie to.
\par 9 Biada temu, który sie spiera z stworzycielem swoim, bedac skorupa, jako inne skorupy gliniane. Izali glina rzecze garncarzowi swemu: Cóz czynisz? Robota twoja zaprawde nikczemna jest.
\par 10 Biada temu, który mówi ojcu: Cóz plodzisz? a niewiescie: Cóz porodzisz?
\par 11 Tak mówi Pan, Swiety Izraelski, i Twórca jego: O przyszle rzeczy pytajcie mie, a synów moich, i sprawe rak moich poruczajcie mi.
\par 12 Jam uczynil ziemie, i czlowiekam na niej stworzyl. Jam jest, którego rece rozciagnely niebiosa, a wszystkiemu wojsku ich rozkazuje.
\par 13 Jam go wzbudzil w sprawiedliwosci, i wszystkie drogi jego wyprostuje. Onci zbuduje miasto moje, a wiezniów moich wypusci, nie za okup, ani za dar, mówi Pan zastepów.
\par 14 Tak mówi Pan: Praca Egipska, i kupiectwo Murzynów, i Sebejczyków, mezowie wysocy do ciebie przyjda, a twoi beda; za toba chodzic beda, w petach pójda, tobie sie klaniac, i tobie sie korzyc beda, mówiac: Tylko w tobie jest Bóg, a niemasz zadnego wiecej, oprócz tego Boga.
\par 15 Zaprawde tys jest Bóg skryty, Bóg Izraelski, zbawiciel.
\par 16 Wszyscy sie oni zawstydza, i pohanbieni beda; czyniciele balwanów spolu z hanba odstapia.
\par 17 Ale Izrael zbawion bedzie przez Pana zbawieniem wiecznem; nie zawstydzicie sie, ani bedziecie pohanbieni, az na wieki wieczne.
\par 18 Bo tak mówi Pan, który stworzyl niebiosa (ten Bóg, który stworzyl ziemie, i uczynil ja! który ja utwierdzil, nie na prózno stworzyl ja, na mieszkanie utworzyl ja): Jam Pan, a niemasz zadnego wiecej.
\par 19 Nie mówilem potajemnie na miejscu ziemskiem ciemnem; nie na prózno mówie nasieniu Jakóbowemu: Szukajcie mie. Ja Pan mówie sprawiedliwosc, a zwiastuje prawosc.
\par 20 Zgromadzcie sie, a przyjdzcie; przyblizcie sie wespól, wy, którzyscie pozostali miedzy poganami. Nic nie wiedza, którzy sie z drewnianemi balwanami swemi nosza; bo sie modla bogu, który nie moze wybawic.
\par 21 Oznajmujciez a przywiedzcie innych, a niech pospolu w rade wnijda, a ukaza, kto to od dawnego czasu przepowiedzial? kto od onego czasu oznajmil? Izali nie Ja Pan? Boc niemasz zadnego innego Boga oprócz mnie. Niemasz Boga sprawiedliwego, i zbawiciela innego oprócz mnie.
\par 22 Obejrzyjciez sie na mie, abyscie zbawione byly wszystkie konczyny ziemi; bom Ja Bóg, a niemasz zadnego wiecej.
\par 23 Przysiaglem sam przez sie; wyszlo z ust moich slowo sprawiedliwe, które sie nazad nie wróci: Ze sie mnie klaniac bedzie wszelkie kolano, i przysiegac kazdy jezyk.
\par 24 Mówiac: Tylko w Panu mam wszelka sprawiedliwosc i sile. Takowi az do niego przyjda; ale pohanbieni beda wszyscy, którzy sie gniewem zapalaja przeciwko niemu.
\par 25 W Panu usprawiedliwione bedzie, i przechwalac sie bedzie wszystko nasienie Izraelskie.

\chapter{46}

\par 1 Pochylil sie Bel, upadl Nebo; balwany ich wlozone sa na bestyje, i na bydleta; tem zaiste, co wy nosicie, beda bardzo obciazone az do ustania.
\par 2 Pochylily sie, i upadly spolem, i Babilonczycy nie beda mogli ratowac brzemion; owszem, i dusza ich w niewole pójdzie.
\par 3 Sluchajcie mie, domie Jakóbowy, i wszystkie ostatki domu Izraelskiego! które nosze zaraz z zywota, które piastuje zaraz od narodzenia;
\par 4 Ja sam az do starosci, i owszem az do sedziwosci was nosic bede. Jam was uczynil, Ja tez nosic bede; Ja mówie nosic was bede, i wybawie.
\par 5 Komuz mie przypodobacie, i przyrównacie, albo podobnym uczynicie, zebym mu byl podobny?
\par 6 Ci, którzy marnie wydawaja zloto z worka, a srebro na szalach waza, najmuja za zaplate zlotnika, aby uczynil z niego boga, przed którym padaja i klaniaja sie.
\par 7 Nosza go na ramieniu, dzwigaja go, i stawiaja go na miejscu jego. I stoi, a z miejsca swego sie nie ruszy; jezli kto zawola do niego, nie ozywa sie, ani go z utrapienia jego wybawia.
\par 8 Pamietajciez na to, a wstydzcie sie; przypusccie to do serca, o przestepnicy!
\par 9 Wspomnijcie sobie na rzeczy pierwsze, które sie dzialy od wieku; bom Ja Bóg, a niemasz zadnego Boga wiecej, i niemasz mnie podobnego;
\par 10 Który opowiadam od poczatku rzeczy ostatnie, i zdawna to, co sie jeszcze nie stalo; rzekeli co, rada moja ostoi sie, i wszystke wole moje uczynie.
\par 11 Który zawolam od wschodu slonca ptaka, z ziemi dalekiej tego, któryby wykonal rade moje. Rzeklem, a dowiode tego; umyslilem, a uczynie to.
\par 12 Sluchajcie mie, wy upornego serca, którzy jestescie dalekimi od sprawiedliwosci.
\par 13 Sprawie, ze sie przyblizy sprawiedliwosc moja, nie pójdzie w dluga, a zbawienie moje nie omieszka; bo poloze w Syonie zbawienie, a w Izraelu slawe moje.

\chapter{47}

\par 1 Zstap, a usiadz w prochu, panno, córko Babilonska! siadz na ziemi, a nie na stolicy, córko Chaldejska! bo cie nie beda wiecej nazywac kochanka i rozkosznica.
\par 2 Wezmij zarna, a miel make; odkryj warkocze swoje, obnaz nogi, odkryj golenie, brnij przez rzeke.
\par 3 Odkryta bedzie nagosc twoja, a hanba twoja widziana bedzie; wezme pomste z ciebie, a nie dam sie nikomu zahamowac.
\par 4 To mówi odkupiciel nasz, imie jego Pan zastepów, Swiety Izraelski.
\par 5 Siedz milczac, a wnijdz do ciemnosci, córko Chaldejska! bo cie wiecej nie beda nazywac pania królestw.
\par 6 Rozgniewalem sie byl na lud mój, splugawilem dziedzictwo moje, a dalem je w rece twoje; ales im ty nie okazala milosierdzia, i starców obciazalas jarzmem twojem bardzo,
\par 7 I rzeklas: Na wieki pania bede; i tak nie przypuscilas tego do serca swego, anis sobie przywodzila na pamiec dokonczenia tego.
\par 8 Przetoz sluchaj tego teraz, rozkosznico! (która mieszkasz bezpiecznie, a mówisz w sercu swem: Jam jest, a niemasz oprócz mnie innej, nie bede wdowa, ani uznam sieroctwa;)
\par 9 Ze to oboje przyjdzie na cie nagle dnia jednego, sieroctwo i wdowstwo, a doskonale przypanie na cie dla mnóstwa gusel twoich, i dla wielkosci czarów twoich.
\par 10 Bo ufasz w zlosci twojej, a mówisz: Nie widzi mie nikt. Madrosc twoja i umiejetnosc twoja, ta cie przewrotna uczynila, abys mówila w sercu swem: Jam jest, a niemasz oprócz mnie innej.
\par 11 Dlatego przyjdzie na cie zle, którego wyjscia nie wiesz, i przypadnie na cie bieda, której nie bedziesz mogla zbyc; a przyjdzie na cie nagle spustoszenie, nim wzwiesz.
\par 12 Stanze teraz z czarami swemi, i z mnóstwem gusel twoich, któremis sie parala od mlodosci twojej, azazbys co sobie mogla pomódz, albo sie snac czem zmocnic.
\par 13 Ustawasz z mnóstwem rad twoich; niechajze teraz stana praktykarze, którzy sie przypatruja gwiazdom, którzy dawaja znac, co ma byc kazdego miesiaca, a niech cie wybawia z tego, co ma przyjsc na cie.
\par 14 Oto sa jako plewa; ogien popali ich, nie wybawia ani duszy swej z mocy plomienia; nie zostanie wegla do ogrzania sie, ani ognia, coby posiedziec przy nim.
\par 15 Takci sie stanie kupcom twoim, z którymis sie zabawiala od mlodosci twojej: kazdy sie z nich w swa strone uda, nie bedzie, ktoby cie wybawil.

\chapter{48}

\par 1 Sluchajcie tego, domie Jakóbowy! którzy sie nazywacie imieniem Izraelowem, a poszliscie z wód Judzkich; którzy przysiegacie przez imie Panskie, a Boga Izraelskiego przypominacie, ale nie w prawdzie ani w sprawiedliwosci;
\par 2 Aczkolwiek od miasta swietego mianujecie sie, a na Bogu Izraelskim spolegacie, Pan zastepów imie jego.
\par 3 Pierwsze rzeczy z dawnam opowiadal, a co z ust moich wyszlo i com oglaszal, naglem czynil, i przychodzilo.
\par 4 Wiedzialem, zes ty twardy, a szyja twoja zyla zelazna, a czolo twoje miedziane.
\par 5 Przetoz oznajmialem ci z dawna; pierwej niz sie co stalo, oglaszalem, bys snac nie rzekl: Balwan mój uczynil to, a obraz mój albo ulanie moje rozkazalo to.
\par 6 Slyszales o tem, spojrzyjze na to wszystko; a wy izali tego nie opowiecie? Teraz juz oglaszam nowe i tajemne rzeczy, i o któryches nie wiedzial.
\par 7 Teraz stworzone sa, a nie onego czasu, o któryches przed tym dniem nic nie slyszal, bys snac nie rzekl: Otom wiedzial o tem.
\par 8 Owszem anis slyszal, anis wiedzial; ani sie to w on czas donioslo ucha twego; bom wiedzial, ze zapewne wystapisz, a ze przestepca bedziesz zaraz z zywota matki twojej.
\par 9 Dla imienia mego zatrzymam popedliwosc moje, a dla chwaly mojej zahamuje gniew przeciwko tobie, abym cie nie wygladzil.
\par 10 Oto wyplawie cie, ale nie jako srebro; przebiore cie w piecu utrapienia.
\par 11 Sam dla siebie, dla siebie to uczynie; bo jakozby mialo byc splugawione imie moje? Zaiste chwaly mojej nie dam innemu.
\par 12 Sluchaj mie, Jakóbie i Izraelu, wezwany mój! Jam jest, Jam pierwszy, Jam i ostateczny.
\par 13 A reka moja zalozyla ziemie, i prawica moja piedzia rozmierzyla niebiosa; zawolalem je, a zaraz stanely.
\par 14 Zbierzcie sie wszyscy, a sluchajcie. Któz z nich to opowiedzial? Pan umilowal go, on wykona wole jego nad Babilonem, a ramie jego przeciw Chaldejczykom.
\par 15 Ja, Jam mówil; przetoz wezwe go, przywiode go, a poszczesci mu sie droga jego.
\par 16 Przyblizcie sie do mnie, a sluchajcie tego! Nie mówilem od poczatku w skrytosci; ale od onegoz czasu, którego sie to dzialo, tamem byl. A teraz panujacy Pan poslal mie, i duch jego.
\par 17 Tak mówi Pan, odkupiciel twój, Swiety Izraelski: Jam Pan, Bóg twój, który cie ucze, abys postepowal; a prowadze cie droga, po której chodzic masz.
\par 18 Obyzes byl pilnowal przykazania mego! bylby jako rzeka pokój twój, a sprawiedliwosc twoja jako waly morskie;
\par 19 A nasienie twoje byloby jako piasek, a plód zywota twego jako drzastwo jego; a nie byloby wyciete ani wygladzone imie jego przed obliczem mojem.
\par 20 Wynijdzcie z Babilonu, ucieczcie od Chaldejczyków; glosem to rozslawiajcie, rozglaszajcie to, roznaszajcie to, az do konczyn ziemi; mówcie: Pan odkupil sluge swego Jakóba.
\par 21 Nie upragna, gdy ich przez pustynie powiedzie; wody z skaly wywiedzie im; bo rozszczepi opoke, i wyplyna wody.
\par 22 Niemasz pokoju niepoboznym, mówi Pan.

\chapter{49}

\par 1 Sluchajcie mie wyspy, a narody dalekie pilnujcie! Pan zaraz z zywota wezwal mie, zaraz z zywota matki mojej uczynil wzmianke imienia mego:
\par 2 I uczynil usta moje jako miecz ostry, w cieniu reki swej zakryl mie, a uczyniwszy mie strzala wypolerowana, do sajdaku swego schowal mie;
\par 3 I rzekl mi: Slugas ty mój, w Izraelu toba sie chlubic bede.
\par 4 A Jam rzekl: Nadarmom pracowal, próznom i daremnie zniszczyl sile moje; wszakze sad mój jest u Pana, a praca moja u Boga mego.
\par 5 A teraz mówi Pan, który mie zaraz z zywota za sluge sobie utworzyl, abym zas przywiódl do niego Jakóba. (Chocby Izrael nie byl zebrany, slawnym jednak bede przed oczyma Panskiemi; albowiem Bóg mój jest sila moja.)
\par 6 I rzekl: Maloby mi to bylo, abys mi byl sluga ku podzwignieniu pokolen Jakóbowych, i ku nawróceniu ostatków z Izraela; przetoz dalem cie za swiatlosc poganom, abys byl zbawieniem mojem az do konczyn ziemi.
\par 7 Tak mówi Pan, odkupiciel Izraelowy, Swiety jego, do tego, którym kazdy gardzi, a którym sie brzydza narody, do slugi panujacych: Królowie widzac cie powstana, a ksiazeta klaniac ci sie beda dla Pana, który jest wierny, dla Swietego Izraelskiego, który cie obral.
\par 8 Tak mówi Pan: Czasu przyjemnego wyslucham cie, a w dzien zbawienia poratuje cie; nadto strzedz cie bede, i dam cie za przymierze ludowi, abys utwierdzil ziemie, a podal w osiadlosc dziedzictwa spustoszale;
\par 9 Abys mówil wiezniom: Wynijdzcie; a tym, co sa w ciemnosciach: Okazcie sie. Podle dróg pasc sie beda, a po wszystkich miejscach wysokich beda pastwiska ich.
\par 10 Nie beda laknac, ani pragnac, i nie uderzy na nich goracosc, ani slonce, bo ten, który ma litosc nad nimi, poprowadzi ich, i podle zródel wód powiedzie ich.
\par 11 Nadto sposobie na wszystkich górach moich droge, a goscince moje beda powyzszone.
\par 12 Oto ci z daleka przyjda, a oto drudzy od pólnocy i od morza, a drudzy z ziemi Synim.
\par 13 Spiewajcie niebiosa, rozraduj sie ziemio, i glosno zabrzmijcie góry! albowiem Pan pocieszyl lud swój, a nad ubogimi swoimi zmilowal sie.
\par 14 Ale Syon rzekl: Opuscil mie Pan, a Pan zapomnial na mie.
\par 15 Izali moze zapomniec niewiasta niemowlatka swego, aby sie nie zlitowala nad plodem zywota swego? A chocby tez i one zapomnialy, wszakze Ja ciebie nie zapomne.
\par 16 Oto na dloniach swoich wyrysowalem cie; mury twoje zawzdy sa przedemna.
\par 17 Pospiesza sie do ciebie synowie twoi, a ci, którzy cie burzyli i kazili, odejda od ciebie.
\par 18 Podnies w okolo oczy swe, a obacz; ci wszyscy zgromadziwszy sie przyjda do ciebie. Jakom zywy Ja, mówi Pan, ze tymi wszystkimi jako ochedóstwem przyodziejesz sie, i oblozysz sie nimi jako oblubienica;
\par 19 Przeto, ze pustynie twoje, i spustoszale miejsca twoje, i ziemia zburzenia twego teraz beda ciasne dla obywateli, gdyz oddaleni beda ci, którzy cie pozerali.
\par 20 Tak, ze rzeka w uszy twoje synowie sieroctwa twego: Ciasne mi jest to miejsce; ustapze mi, abym mieszkac mógl.
\par 21 I rzeczesz w sercu swem: Któz mi tych naplodzil? bom ja byla osierociala, i samotna, wygnanam byla, i tulalam sie; któz wzdy tych odchowal? Otom Ja tylko sama pozostala byla, gdziez ci byli?
\par 22 Tak mówi panujacy Pan: Oto wzniose na narody reke moje, a do ludzi podniose choregiew moje, aby przyniesli synów twoich na reku, i córki twoje aby na ramionach przynoszone byly.
\par 23 I beda królowie piastunami twoimi, a ksiezny ich mamkami twemi; twarza ku ziemi klaniac ci sie beda, i proch nóg twoich lizac beda; a dowiesz sie, zem Ja Pan, a iz nie bywaja zawstydzeni, którzy na mie oczekuja.
\par 24 I rzeczesz: Izali korzysc od mocarza odjeta bedzie? Izali pojmany lud sprawiedliwego wybawiony bedzie?
\par 25 Owszem, tak mówi Pan: I pojmany lud mocarzowi odjety bedzie, i korzysc okrutnikowi wydarta bedzie; albowiem przeciwnikowi twemu Ja sie sprzeciwie, a synów twoich Ja wyswobodze.
\par 26 I tych, którzy cie pustosza, wlasnem ich cialem nakarmie, a krwia swoja jako moszczem upija sie. I pozna wszelkie cialo, zem Ja Pan, zbawiciel twój, i odkupiciel twój, mocny Jakóbowy.

\chapter{50}

\par 1 Tak mówi Pan: Gdzie jest list rozwodny matki waszej, którymem ja wolno puscil? albo kto jest z pozyczalników moich, któremum was zaprzedal? Otoscie nieprawosciami swojemi sami siebie zaprzedali, a dla przestepstw waszych wolno puszczona jest matka wasza.
\par 2 Przeczze, gdy przychodze, niemasz nikogo? a gdy wolam, nikt sie nie ozywa? Izali tak jest ukrócona reka moja, aby nie mogla odkupic? Izali niemasz we mnie mocy ku wybawieniu? Oto fukiem moim osuszam morze, obracam rzeki w pustynie, tak iz zasmier dna ryby ich dla niedostatku wody, i zdychaja od pragnienia.
\par 3 Oblocze niebiosa w ciemnosci, a wór daje za odzienie ich.
\par 4 Panujacy Pan dal mi jezyk umiejetny, abym umial czasu przygodnego mówic slowo upracowanemu. Budzi mie na kazdy zaranek, pobudza uszy moje, abym sluchal tak jako uczacy sie pilnie.
\par 5 Panujacy Pan otwiera mi uszy, a Ja sie nie sprzeciwiam, ani sie na wstecz wracam.
\par 6 Ciala mego nadstawiam bijacym, a policzków moich tym, którzy mie targaja; twarzy mojej nie zakrywam od obelzenia i plwania.
\par 7 Bo panujacy Pan wspomaga mie; przetoz nie bywam pohanbiony. Dla tego postawilem twarz moje jako krzemien, gdyz wiem, ze pohanbiony nie bede.
\par 8 Bliskoc jest ten, który mie usprawiedliwia. Któz sie sprzeczac bedzie ze mna? Stanmy spolem; kto ma prawo ze mna, niech przystapi ku mnie.
\par 9 Oto panujacy Pan pomagac mi bedzie; któz jest, coby mie potepil? Oto wszyscy takowi jako odzienie zwiotszeja, a mól zgryzie ich.
\par 10 Kto jest miedzy wami bojacy sie Pana, posluchaj glosu slugi jego; kto jest, co chodzi w ciemnosciach a nie ma swiatlosci? ufaj w imieniu Panskiem, a spolegaj na Bogu swoim.
\par 11 Oto wy wszyscy, którzy rozniecacie ogien, a przepasujecie sie iskrami, chodzciez w swiatlosci ognia waszego, i w iskrach, którescie rozniecili; z reki mojej wam sie to stanie, ze w bolesci lezec bedziecie.

\chapter{51}

\par 1 Sluchajcie mie, którzy nasladujecie sprawiedliwosci, którzy szukacie Pana. Spojrzyjcie na skale, z którejscie wycieci, i na glebokosc dolu, skadescie wykopani.
\par 2 Spojrzyjcie na Abrahama, ojca waszego, i na Sare, która was porodzila, zem go jednego wezwal, i poblogoslawilem mu, a rozmnozylem go.
\par 3 Gdyz pocieszy Pan Syon, pocieszy wszystkie pustynie jego, a uczyni puszcze jego bardzo rozkoszna, a pustynie jego jako ogród Panski, radosc i wesele znajdzie sie w nim, dziekczynienie, i glos spiewania.
\par 4 Pilnujcie mie, ludu mój i rodzino moja! nadstawcie mi uszów; bo zakon odemnie wyjdzie, a sad mój za swiatlosc narodom wystawie.
\par 5 Blisko jest sprawiedliwosc moja, wynijdzie zbawienie moje, a ramiona moje narody sadzic beda. Na mie wyspy oczekuja, a po ramieniu mojem tesknia.
\par 6 Podniescie ku niebu oczy wasze, a spojrzyjcie na ziemie na dól. Niebiosa jako dym zniszczeja, a ziemia jako odzienie zwiotszeje, i obywatele jej, jako i ona zgina; ale zbawienie moje na wieki bedzie, a sprawiedliwosc moja nie ustanie.
\par 7 Sluchajcie mie, którzy znacie sprawiedliwosc ludu, w którego sercu jest zakon mój! Nie bójcie sie uragania ludzkiego, a sromocenia ich nie lekajcie sie.
\par 8 Albowiem ich mól jako szate pozre, a robak ich jako welne pogryzie; ale sprawiedliwosc moja na wieki bedzie, a zbawienie moje od narodu do narodu.
\par 9 Ocuc sie, ocuc sie, oblecz sie w sile, o ramie Panskie! Ocuc sie jako za dni dawnych, i za rodzajów przeszlych! Izalis nie ty jest, któres zgladzilo Egipt, i zranilo smoka?
\par 10 Izalis nie ty jest, któres wysuszylo morze, wody przepasci wielkiej? któres obrócilo glebokosci morskie w droge, aby przeszli wybawieni?
\par 11 A tak ci, których odkupil Pan, niech sie nawróca, i przyjda do Syonu z spiewaniem, a wesele wieczne niech bedzie nad glowa ich; wesela i radosci niech dostapia, a niech uciecze smutek i wzdychanie.
\par 12 Ja, Jam jest pocieszyciel wasz. Któzes ty, ze sie boisz czlowieka smiertelnego, i syna czlowieczego trawie podobnego?
\par 13 Ze zapominasz na Pana stworzyciela swego, który rozciagnal niebiosa, i zalozyl ziemie? a ze sie lekasz ustawicznie kazdego dnia popedliwosci trapiacego, gdy sie gotuje, aby zatracal? Ale gdziez jest ta popedliwosc trapiacego?
\par 14 Pospieszy sie, aby wiezien byl uwolniony; bo nie umrze w dole, ani bedzie mial jaki niedostatek chleba swego.
\par 15 Ja zaiste jestem Pan, Bóg twój, który rozdzielam morze, tak, ze szumia waly jego; Pan zastepów jest imie moje.
\par 16 Jam wlozyl slowa moje w usta twoje, a cieniem reki mojej zakrylem cie, abys szczepil niebiosa, a zalozyl ziemie, i rzekl Syonowi: Tys jest lud mój.
\par 17 Ocuc sie, ocuc sie, powstan Jeruzalemie! któres pilo z reki Panskiej kubek zapalczywosci jego, drozdze z kubka trucizny smiertelnej wypilos i wysaczylos.
\par 18 Nikt go nie prowadzil ze wszystkich synów, których naplodzilo, i nikt go nie ujal za reke jego ze wszystkich synów, które wychowalo.
\par 19 Dwie rzeczy sa, które cie spotkaly; (któz sie ciebie uzalil?) Spustoszenie i skruszenie, glód i miecz; któz cie pocieszy?
\par 20 Synowie twoi pomdlawszy lezeli na rogach wszystkich ulic, jako bawól w sieci, pelni bedac popedliwosci Panskiej, gromienia Boga twego.
\par 21 A przetoz sluchaj teraz tego, o utrapiona i pijana, ale nie winem!
\par 22 Tak mówi Pan twój, Pan i Bóg twój, który sie zastawia za lud swój: Oto biore z reki twojej kubek trucizny smiertelnej, i drozdze kubka popedliwosci mojej; nie bedziesz wiecej pic z niego;
\par 23 Ale podam go w reke tych, którzy cie trapia, którzy mówili duszy twojej: Nachyl sie, niech przez cie przejdziemy; a tys pokladalo jako ziemie grzbiet swój, i jako ulice przechodzacym.

\chapter{52}

\par 1 Ocuc sie, ocuc sie, oblecz sie w moc twoje, Syonie! oblecz sie w szate ochedóstwa twego, o Jeruzalemie, miasto swiete! Albowiem nie natrze na cie nieobrzezany i nieczysty.
\par 2 Otrzasnij sie z prochu, powstan, siadz, Jeruzalemie! dobadz sie z oków szyi swojej, o pojmana córko Syonska!
\par 3 Tak zaista Pan mówi: Darmoscie sie zaprzedali, przetoz bez pieniedzy odkupieni bedziecie.
\par 4 Bo tak mówi panujacy Pan: Do Egiptu wstapil lud mój przedtem, aby tam pielgrzymowal; ale Assyryjczyk bez przyczyny go trapi.
\par 5 A teraz cóz mam czynic? mówi Pan, poniewaz lud mój darmo jest pojmany, a ci, którzy panuja nad nim, do wzdychania go przywodza, mówi Pan; nadto ustawicznie kazdego dnia imie moje bluznione bywa.
\par 6 Przetoz pozna lud mój imie moje, przetoz pozna, mówie, dnia onego, zem Ja jest ten, który mówie; otom Ja przytomny.
\par 7 O jako piekne sa na górach nogi tego, co pocieszne rzeczy zwiastuje, i opowiada pokój; tego, co zwiastuje dobre, i opowiada zbawienie, a mówi do Syonu: Bóg twój króluje!
\par 8 Wynosza glos strózowie twoi, glos wynosza, a spolem wykrzykac beda; bo okiem w oko ujrza, ze zasie Pan Syon przywiedzie.
\par 9 Wykrzykajcie a spiewajcie spolem, pustynie Jeruzalemskie! bo pocieszyl Pan lud swój, odkupil Jeruzalem.
\par 10 Wysmuknal Pan ramie swietobliwosci swojej przed oczyma wszystkich narodów, aby ogladaly wszystkie konczyny ziemi zbawienie Boga naszego.
\par 11 Odstapcie, odstapcie wynijdzcie z Babilonu, nieczystego sie nie dotykajcie, wynijdzcie z posrodku jego; oczyscie sie wy, którzy nosicie naczynie Panskie.
\par 12 Bo nie z trzaskiem wynijdziecie, ani uciekajac pójdziecie; pójdzie zaiste Pan przed wami, a zgromadzi was Bóg Izraelski.
\par 13 Oto sie szczesliwie powiedzie sludze memu. Wywyzszony i podniesiony i bardzo uwielbiony bedzie.
\par 14 Jako wiele ich zdumieja sie nad nim, ze przemierzla jest nad innych ludzi osoba jego, a ksztalt jego nad synów ludzkich:
\par 15 Tak zasie pokropi wiele narodów, i królowie przed nim zatula usta swe, przeto, ze czego im nie powiadano, to ogladaja, a to, o czem nie slyszeli, wyrozumieja.

\chapter{53}

\par 1 Któz uwierzyl kazaniu naszemu, a ramie Panskie komu objawione jest?
\par 2 Bo wyrósl jako latorostka przed nim, a jako korzen z ziemi suchej, nie majac ksztaltu ani pieknosci; i widzielismy go; ale nic nie bylo widziec, czemubysmy go zadac mieli.
\par 3 Najwzgardzenszy byl, i najpodlejszy z ludzi, maz bolesci, a swiadomy niemocy, i jako zakrywajacy twarz swoje; najwzgardzenszy mówie, skadesmy go za nic nie mieli.
\par 4 Zaiste on niemocy nasze wzial na sie, a bolesci nasze wlasne nosil; a mysmy mniemali, ze jest zraniony, ubity od Boga i utrapiony.
\par 5 Lecz on zraniony jest dla wystepków naszych, starty jest dla nieprawosci naszych; kazn pokoju naszego jest na nim, a sinoscia jego jestesmy uzdrowieni.
\par 6 Wszyscysmy jako owce zbladzili, kazdy na droge swa obrócilismy sie, a Pan wlozyl nan nieprawosc wszystkich nas.
\par 7 Ucisniony jest i utrapiony, a nie otworzyl ust swoich; jako baranek na zabicie wiedziony byl, i jako owca przed tymi, którzy ja strzyga, oniemial, i nie otworzyl ust swoich.
\par 8 Z wiezienia i z sadu wyjety jest; przetoz rodzaj jego któz wypowie? Albowiem wyciety jest z ziemi zyjacych, a zraniony dla przestepstwa ludu mojego;
\par 9 Który to lud podal niezboznym grób jego, a bogatemu smierc jego, choc jednak nieprawosci nie uczynil, ani zdrada znaleziona jest w ustach jego.
\par 10 Takci sie Panu upodobalo zetrzec go, i niemoca utrapic, aby polozywszy ofiara za grzech dusze swa, ujrzal nasienie swoje, przedluzyl dni swoich; a to, co sie podoba Panu, przez reke jego aby sie szczesliwie wykonalo.
\par 11 Z pracy duszy swej ujrzy owoc, którym nasycon bedzie. Znajomoscia swoja wielu usprawiedliwi sprawiedliwy sluga mój; bo nieprawosci ich on sam poniesie.
\par 12 Przetoz mu dam dzial dla wielu, aby sie dzielil korzyscia z mocarzami, poniewaz wylal na smierc dusze swoje, a z przestepcami policzon bedac, on sam grzech wielu odniósl, i za przestepców sie modlil.

\chapter{54}

\par 1 Spiewaj nieplodna! która nie rodzisz, spiewaj glosno, a krzycz, która w porodzeniu nie pracujesz; bo wiecej bedzie synów opuszczonej, niz synów tej, która ma meza, mówi Pan.
\par 2 Rozprzestrzen miejsce namiotu swego, a opon przybytków swych nie zabraniaj rozciagnac: wyciagnij powrozy twoje, a kolki twoje utwierdz.
\par 3 Bo sie na prawo i na lewo rozsilisz, a nasienie twoje narody odziedziczy, i miasta spustoszone osadzi.
\par 4 Nie bój sie, bo pohanbiona nie bedziesz; a nie zapalaj sie, bo nie przyjdziesz na posromocenie; owszem na zelzywosc mlodosci twojej zapomnisz, a na pohanbienie wdowstwa twego wiecej nie wspomnisz.
\par 5 Albowiem malzonkiem twoim jest stworzycie twój, Pan zastepów imie jego, a odkupiciel twój, Swiety Izraelski, Bogiem wszystkiej ziemi zwany bedzie.
\par 6 Bo cie jako zony opuszczonej i strapionej w duchu, Pan powola, a jako zony mlodej, gdy odrzucona bedziesz, mówi Bóg twój.
\par 7 Na mala chwilke opuscilem cie; ale zas w litosciach wielkich zgromadze cie.
\par 8 W maluczkim gniewie skrylem maluczko twarz swoje przed toba; ale w milosierdziu wiecznem zlituje sie nad toba, mówi Pan, odkupiciel twój.
\par 9 Bo to jest u mnie, co przy potopie Noego; jakom przysiagl, ze sie wiecej nie beda rozlewac wody Noego po ziemi: takem przysiagl, ze sie nie rozgniewam na cie, ani cie zgromie.
\par 10 A chocby sie i góry poruszyly, i pagórki sie zachwialy: jednak milosierdzie moje od ciebie nie odstapi, a przymierze pokoju mego nie wzruszy sie, mówi twój milosciwy Pan.
\par 11 O utrapiona, wichrem rozmiotana, z pociechy obrana! oto Ja poloze na karbunkulach kamienie twoje, a na szafirach zaloze cie.
\par 12 I uczynie z krysztalu okna twoje, a bramy twoje z kamienia rubinowego, i wszystkie granice twoje z kamienia kosztownego.
\par 13 A wszyscy synowie twoi beda wyuczeni od Pana, i obfitosc pokoju beda mieli synowie twoi.
\par 14 Na sprawiedliwosci ugruntowana bedziesz; od ucisku sie oddalisz, przetoz sie go bac nie bedziesz; i od starcia; bo sie nie przyblizy do ciebie.
\par 15 Oto nie jeden mieszkac bedzie z toba, który nie jest mój; ale ktoby mieszkajac z toba, byl przeciwnym tobie, upadnie.
\par 16 Otom ja stworzyl kowala poddymajacego wegle w ogniu, a wyjmujacego naczynie ku robocie swojej: Jam tez stworzyl pustoszyciela, aby wytracal.
\par 17 Zadne naczynie urobione przeciw tobie nie zdarzy sie, a kazdy jezyk powstawajacy przeciw tobie na sadzie potepisz. Toc jest dziedzictwo slug Panskich, a sprawiedliwosc ich odemnie, mówi Pan.

\chapter{55}

\par 1 Nuz wszyscy pragnacy pójdzcie do wód, i wy, co niemacie pieniedzy, pójdzcie, kupujcie a jedzcie; pójdzcie, mówie, kupujcie bez pieniedzy i bez zaplaty, wino i mleko.
\par 2 Przecz wynakladacie pieniadze nie za chleb, a prace swa na to, co nie nasyca? Sluchajac sluchajcie mie, a jedzcie to, co jest dobrego, i niech sie rozkocha w tlustosci dusza wasza.
\par 3 Nakloncie ucha swego, a pójdzcie do mnie; sluchajcie, a bedzie zyla dusza wasza. I postanowie z wami przymierze wieczne, milosierdzie Dawidowe pewne wyleje na was.
\par 4 Oto dalem go za swiadka narodom, za wodza i za nauczyciela narodom.
\par 5 Oto naród, któregos nie znal, powolasz, a narody, które cie nie znaly, zbieza sie do ciebie dla Pana, Boga twego, i Swietego Izraelskiego; bo cie uwielbi.
\par 6 Szukajcie Pana, póki moze byc znaleziony; wzywajcie go, póki blisko jest.
\par 7 Niech opusci niepobozny droge swoje, a czlowiek nieprawy mysli swoje i niech sie nawróci do Pana, a zmiluje sie; i do Boga naszego, gdyz jest hojnym w odpuszczaniu.
\par 8 Boc zaiste mysli moje nie sa jako mysli wasze, ani drogi wasze jako drogi moje, mówi Pan;
\par 9 Ale jako wyzsze sa niebiosa niz ziemia, tak przewyzszaja drogi moje drogi wasze, a mysli moje mysli wasze.
\par 10 Bo jako zstepuje deszcz i snieg z nieba, a tam sie wiecej nie wraca, ale napawa ziemie, a czyni ja plodna, czyni ja tez urodzajna, tak ze wydaje nasienie siejacemu, a chleb jedzacemu:
\par 11 Takci bedzie slowo moje, które wynijdzie z ust moich; nie wróci sie do mnie prózno, ale uczyni to, co mi sie podoba, i poszczesci mu sie w tem, na co je posle.
\par 12 Przetoz w weselu wynijdziecie, a w pokoju doprowadzeni bedziecie. Góry i pagórki chwale przed wami glosno zaspiewaja, a wszystkie drzewa polne rekami klaskac beda.
\par 13 Miasto ciernia wyrosnie jedlina, a miasto pokrzywy wyrosnie mirt; a to bedzie Panu ku slawie, na znak wieczny, który nigdy nie bedzie wygladzony.

\chapter{56}

\par 1 Tak mówi Pan: Strzezcie sadu, a czyncie sprawiedliwosc; bo blisko tego, ze zbawienie moje przyjdzie, a sprawiedliwosc moja objawiona bedzie.
\par 2 Blogoslawiony czlowiek, który to czyni, i syn czlowieczy, który sie trzyma tego, przestrzegajac sabatu, aby go nie splugawil, a strzegac reki swej, aby nie uczynila nic zlego.
\par 3 Niech tedy nie mówi cudzoziemiec, który przystaje do Pana, mówiac: Zaiste Pan mie odlaczyl od ludu swego; niech tez nie mówi trzebieniec: Otom ja drzewo suche.
\par 4 Albowiem tak mówi Pan o trzebiencach, którzyby przestrzegali sabatów moich, a obrali to, co mi sie podoba, i trzymali przymierze moje:
\par 5 Zec im dam w domu swym i miedzy murami mojemi miejsce, i imie lepsze nizeli synów i córek; dam im imie wieczne, które nie bedzie wygladzone.
\par 6 A cudzoziemców, którzyby przystali do Pana, aby mu sluzyli, a milowali imie Panskie, bedac u niego za slugi, wszystkich przestrzegajacych sabaty, aby go nie splugawili, i zachowujacych przymierze moje;
\par 7 Tych przywiode na góre swietobliwosci mojej, a uwesele ich w domu modlitwy mojej; calopalenia ich i ofiary ich przyjemne beda na oltarzu moim; bo dom mój domem modlitwy nazwany bedzie u wszystkich narodów.
\par 8 Tak mówi panujacy Pan, który zgromadza rozpedzonych z Izraela: Jeszcze zgromadze do niego, i do zgromadzonych jego.
\par 9 Wszystkie zwierzeta polne przyjdzcie na pozarcie, i wszystkie zwierzeta lesne.
\par 10 Strózowie jego slepi, wszyscy zgola nic nie umieja, wszyscy sa psami niememi, nie moga szczekac; ospalymi sa, leza, kochaja sie w drzemaniu.
\par 11 A sa psami obzartemi, nie moga sie nigdy nasycic; sami sie pasac nie umieja nauczac. Wszyscy sie za droga swoja udali, kazdy za lakomstwem swojem z strony swej, mówiac:
\par 12 Pójdzcie, nabiore wina, a upijemy sie mocnym napojem, a bedzie nam jako dzis tak i jutro, i jeszcze daleko obficiej.

\chapter{57}

\par 1 Sprawiedliwy ginie, a nikt tego do serca nieprzypuszcza; i mezowie pobozni schodza, a nikt tego nie uwaza, ze przed przyjsciem zlego sprawiedliwy zebrany bywa;
\par 2 Ze wschodzi do pokoju, a odpoczywa na lozu swojem, ktokolwiek chodzi w uprzejmosci.
\par 3 Ale wy sami przystapcie, synowie czarownicy, nasienie cudzoloznika i wszetecznicy!
\par 4 Nad kimze sie cieszycie? przeciwko komuz rozdzieracie gebe, i wywieszacie jezyk? Izali nie jestescie synowie nierzadu, nasienie klamliwe?
\par 5 Którzy nierzad plodzicie w gajach pod kazdem drzewem zielonem zabijajac synów swych przy potokach, pod wysokiemi skalami.
\par 6 Miedzy gladkim kamieniem potokowym jest dzial twój. Cic sa, ci losem twoim, na które tez wylewasz ofiare mokra, a ofiarujesz ofiare sniedna, i w temze bym sie Ja kochal?
\par 7 Na górze wysokiej i wynioslej postawiles loze twoje, a tam wstepujesz ku sprawowaniu ofiar.
\par 8 A za drzwiami i za podwojem polozylas pamiatke twoje, gdyz odemnie odchodzac odkrywasz sie, a wstapiwszy rozszerzasz loze swe, czyniac je przestworniejsze, nizeli poganie; umilowalas loze ich, gdziekolwiek miejsce upatrzysz.
\par 9 Chodzisz i do króla, z olejkiem i z rozmaitemi wonnemi masciami twemi; posylasz bowiem poslów swych daleko, a ponizasz sie az do grobu.
\par 10 Mnóstwem dróg swoich spracowalas sie, a nie mówisz: Daremnac to. Znalazlas pomoc rece swojej, dlategos nie zemdlala.
\par 11 Kogozes sie obawiala i lekala, izes klamala? Na mies nie pomniala, anis tego przypuscila do serca swego: dlategoz to, zem Ja milczal, a to z dawna, nie boisz sie mnie?
\par 12 Ja opowiem sprawiedliwosc twoje i sprawy twoje, którec nic nie pomoga.
\par 13 Gdy zawolasz, niech cie wybawi zgraja twoja; ale wszystkie one rozniesie wiatr, i pochwyci marnosc. Lecz ten, co we mnie ufa, odziedziczy ziemie, a posiadzie góre swieta moje.
\par 14 Bo rzeka: Wyrównajcie, wyrównajcie, zgotujcie droge, uprzatnijcie zawady z drogi ludu mojego.
\par 15 Bo tak mówi on najdostojniejszy i najwyzszy, który mieszka w wiecznosci, a swiete jest imie jego: Ja, który mieszkam na wysokosci na miejscu swietem, mieszkam i z tym, który jest skruszonego i unizonego ducha, ozywiajac ducha pokornych, ozywiajac serce skruszonych.
\par 16 Nie bede sie zaiste na wieki wadzil, ani sie wiecznie gniewal; bocby duch przed obliczem mojem zemdlal, i dusze, którem Ja uczynil.
\par 17 Dla nieprawosci lakomstwa jego rozgniewalem sie, a uderzylem go; ukrylem sie, a rozgniewalem sie, przeto, ze odpornym bedac, poszedl droga serca swego.
\par 18 Widze drogi jego, wszakze uzdrowie go; doprowadze go; i przywróce mu pociechy, i tym, którzy z nim placza.
\par 19 Stworze owoc warg, pokój dalekiemu i bliskiemu, mówi Pan; a tak uzdrowie go.
\par 20 Lecz niepobozni beda jako morze wzburzone, gdy sie uspokoic nie moze, a którego wody wymiataja kal i bloto.
\par 21 Niemasz pokoju niepoboznym, mówi Bóg mój.

\chapter{58}

\par 1 Wolaj wszystkiem gardlem, nie zawsciagaj; wynos glos swój jako traba, a opowiedz ludowi mojemu przestepstwa ich, a domowi Jakóbowemu grzechy ich;
\par 2 Chociaz mie kazdego dnia szukaja, a znac chca drogi moje, jako naród, który sprawiedliwosc czyni, a sadu Boga swego nie opuszcza; pytaja mie o sadach sprawiedliwosci a przgna sie przyblizyc do Boga mówiac:
\par 3 Przeczze poscimy, gdyz na to nie patrzysz? trapimy dusze nasze, a nie widzisz? Oto w dzien postu waszego przewodzicie wole swoje, a wszystkie prace swoje wyciagacie.
\par 4 Oto poscicie na swary, i na zwady, i bijecie piescia niemilosciwie; nie poscicie, jak sie godzi tych dni, aby byl slyszany na wysokosci glos wasz.
\par 5 Izali to jest takowy post, jakim obral, a dzien, w któryby trapil czlowiek dusze swoje? zeby zwiesil jako sitowie glowe swoje, a wór i popiól sobie podscielal? Toz to nazwiesz postem, i dniem przyjemnym Panu?
\par 6 Ale to jest post, którym obral: Rozwiaz zwiazki niepoboznosci, rozwiaz brzemiona ciezkie, i wolno pusc skruszonych, a tak wszelakie jarzmo rozerwij;
\par 7 Ulamuj laknacemu chleba twego, a ubogich wygnanców wprowadz do domu twego; ujrzyszli nagiego, przyodziej go, a przed cialem swojem nie ukrywaj sie.
\par 8 Tedy wyniknie jako zorza ranna swiatlosc twoja, a zdrowie twoje predko zakwitnie! i pójdzie przed toba sprawiedliwosc twoja, a chwala Panska zbierze cie.
\par 9 Tedy wzywac bedziesz, a Pan wyslucha; zawolasz, a odpowiec: Owom Ja. Jezli odejmiesz z posrodku siebie i jarzmo, a przestaniesz palca wyciagac, i mówic nieprawosci;
\par 10 Jezli wylejesz laknacemu dusze swoje, a dusze utrapiona nasycisz: tedy wejdzie w ciemnosci swiatlosc twoja, a zmierzk twój bedzie jako poludnie.
\par 11 Bo cie Pan ustawicznie poprowadzi, i nasyci pod najwieksza susza dusze twoje, a kosci twoje utuczy, i bedziesz jako ogród wilgotny, a jako zdrój wód, którego wody nie ustawaja.
\par 12 I pobuduja splodzeni od ciebie pustynie starodawne, grunty od narodu do narodu wywiedziesz; i nazwa cie naprawca obalin, i przeprawca sciezek ku mieszkaniu.
\par 13 Jezlize odwrócisz od sabatu noge swoje, abys nie przewodzil woli swojej w dzien mój swiety; i jezeli nazwiesz sabat rozkosza, dniem swietym a Panu slawnym, i bedzieszli go mial w uczciwosci, tak, abys wen nie czynil dróg swoich, i nie przewodzil woli swej, i nie mówil slowo próznego:
\par 14 Tedy bedziesz rozkoszowal w Panu; i wprowadza cie na wysokie miejsca ziemi, i sprawie to, abys pozywal dziedzictwa Jakóba, ojca twego; bo usta Panskie mówily.

\chapter{59}

\par 1 Oto nie jest ukrócona reka Panska, aby zbawic nie mogla; a nie jest obciazone ucho jego, aby wysluchac nie moglo.
\par 2 Ale nieprawosci wasze rozdzial uczynily miedzy wami i miedzy Bogiem waszym, a grzechy wasze sprawily, ze ukryl twarz przed wami, aby nie slyszal.
\par 3 Bo rece wasze krwia sa zmazane, a palce wasze nieprawoscia; wargi wasze mówia klamstwo, a jezyk wasz nieprawosc swiegoce.
\par 4 Niemasz ktoby sie zastawial o sprawiedliwosc, ani jest ktoby sie zasadzal o prawde. Ufaja w próznosci, a mówia klamstwo; poczynaja ucisk, a rodza nieprawosc.
\par 5 Jaja bazyliszkowe wylegli, a plótna pajeczego natkali. Ktoby jadl jaja ich, umrze, a jezli je stlucze, wynijdzie jaszczórka.
\par 6 Plótna ich nie godza sie na szate, ani sie przyodzieja robotami swemi. Uczynki ich sa uczynki nieprawosci, a sprawa lupiestwa jest w rekach ich.
\par 7 Nogi ich bieza do zlego, i kwapia sie na wylanie krwi niewinnej. Mysli ich sa mysli nieprawosci; spustoszenie i starcie jest na drogach ich.
\par 8 Drogi pokoju nieznaja, i niemasz sprawiedliwosci w drogach ich; scieszki swe sami pokrzywili u siebie; kazdy, kto po nich chodzi, nie zna pokoju.
\par 9 Dlatego oddalil sie sad od nas, a nie dochodzi nas sprawiedliwosc; czekamy na swiatlosc, a oto ciemnosc; na jasnosc, ale w cmie chodzimy.
\par 10 Macamy sciany jako slepi, a macamy, jakobysmy oczów nie mieli. Potykamy sie w poludnie jako w zmierzk; w wielkich dostatkach podobnismy umarlym.
\par 11 Mruczymy wszyscy jako niedzwiedz, jako golebica ustawicznie stekamy; oczekujemy na sad, ale go niemasz; na wybawienie, ale dalekie jest od nas.
\par 12 Bo sie rozmnozyly przestepstwa nasze przed toba, a grzechy nasze swiadcza przeciwko nam, poniewaz nieprawosci nasze sa przy nas, i zlosci nasze uznajemy;
\par 13 Zesmy wystapili, i klamali przeciw Panu, i odwrócilismy sie, abysmy nie szli za Bogiem naszym; zesmy mówili o potwarzy i o odstapieniu, zesmy zmyslali i wywierali z serca swego slowa klamliwe.
\par 14 Tak, ze sie sad opak obrócil, a sprawiedliwosc z daleka stoi; bo na ulicy prawda szwankowala, a prawosc przejscia nie ma.
\par 15 Owszem, prawda zginela, a ten, co odstepuje od zlego, na lup podany bywa. To widzi Pan, i nie podoba sie to w oczach jego, ze niemasz sadu.
\par 16 Gdy tedy widzial, ze niemasz zadnego meza, az sie zdumial, ze niemasz zadnego, coby sie zastawil, a przetoz wybawienie sprawilo mu ramie jego, a sprawiedliwosc jego sama go podparla.
\par 17 Bo sie przyoblókl w sprawiedliwosc jako w pancerz, a helm zbawienia na glowie jego; oblókl sie w odzienie pomsty jako w szate, a odzial sie zapalczywoscia jako plaszczem;
\par 18 Aby wedlug uczynków, aby wedlug nich odplacil popedliwoscia przeciwnikom swoim, aby nagrode nieprzyjaciolom swoim, a wyspom zaplate oddal.
\par 19 I beda sie bali, którzy sa na zachód, imienia Panskiego, i którzy na wschód slonca, slawy jego. Gdy przypadnie nieprzyjaciel jako rzeka, tedy go duch Panski precz zapedzi.
\par 20 Bo przyjdzie do Syonu odkupiciel, i do tych, którzy sie odwracaja od wystepków w Jakóbie, mówi Pan.
\par 21 A toc bedzie przymierze moje z nimi, mówi Pan: Duch mój, który jest w tobie, i slowa moje, którem wlozyl w usta twoje, nie odstapia od ust twoich, ani od ust nasienia twego, ani od ust potomków nasienia twego, mówi Pan, odtad az na wieki.

\chapter{60}

\par 1 Powstan, objasnij sie! poniewaz przyszla swiatlosc twoja, a chwala Panska weszla nad toba.
\par 2 Bo oto ciemnosci okryja ziemie, a zacmienie narody; ale nad toba wejdzie Pan, a chwala jego nad toba widziana bedzie.
\par 3 I beda chodzic narody w swiatlosci twojej, a królowie w jasnosci, która wejdzie nad toba.
\par 4 Podnies w okolo oczy twe, a spojrzyj; ci wszyscy, którzy sie zgromadzili, pójda do ciebie; synowie twoi z daleka przyjda, a córki twoje przy boku twoim chowane beda.
\par 5 Tedy ogladasz to, a rozweselisz sie; tedy sie zdumieje i rozszerzy serce twoje, gdy sie obróci ku tobie zgraja morska, a moc narodów przyjdzie do ciebie.
\par 6 Obfitosc wielbladów okryje cie, takze dromedarze z Madyjan i z Efy. Wszyscy ci przyjda z Saby, zloto i kadzidlo przyniosa, a chwaly Panskie opowiadac beda.
\par 7 Wszystkie stada z Kedar zgromadza sie do ciebie; barany z Nebajotu sluzyc ci beda, a ofiarowane bedac na oltarzu moim, przyjemne beda; a tak dom majestatu mego ozdobie.
\par 8 I rzeczesz: Którzyz to sa, co sie jako obloki zlatuja, i jako golebie do okien swoich?
\par 9 Na miec zaiste wyspy oczekuja, i okrety morskie zdawna, aby przywiedli synów twoich z daleka, takze srebro swoje z soba, i zloto swoje imieniowi Pana, Boga twego, i Swietego Izraelskiego; bo cie uwielbi.
\par 10 I pobuduja cudzoziemcy mury twoje, a królowie ich sluzyc ci beda, gdyz w rozgniewaniu mojem uderze cie, a w upodobaniu mojem zlituje sie nad toba.
\par 11 I beda otworzone bramy twoje ustawicznie; we dnie i w nocy nie beda zatkane, aby przywiedziono do ciebie moc pogan, i królowie ich aby byli przywiedzieni.
\par 12 Naród ten i królestwo, którecby nie sluzylo, zginie; narody takie, mówie, do szczetu spustoszone beda.
\par 13 Slawa Libanu do ciebie przyjdzie, jedlina, sosna, takze bukszpan, dla ozdoby miejsca swiatnicy mojej, abym miejsce nóg moich uwielbil.
\par 14 Przyjda takze do ciebie w pokorze synowie tych, którzy cie trapili, i beda sie klaniac stopom nóg twoich, którzykolwiek pogardzili toba, i nazwia cie miastem Panskiem, Syonem Swietego Izraelskiego.
\par 15 Miasto tego, cos opuszczona i w nienawisci byla, tak, ze nie bylo, ktoby przez cie chodzil, wystawie cie za dostojnosc wieczna, i wesele od narodu do narodu.
\par 16 Bo ssac bedziesz mleko narodów, i piersiami królów karmiona bedziesz; i poznasz, izem Ja Pan, zbawieciel twój i odkupiciel twój, mocarz Jakóbowy;
\par 17 Miasto miedzi naniose zlota, a miasto zelaza naniose srebra, a miasto drew miedzi, a miasto kamienia zelaza; i postawie nad toba dozorców spokojnych, i urzedników sprawiedliwych.
\par 18 Nie bedzie wiecej slychac o drapiestwie w ziemi twojej, o zburzeniu i spustoszeniu na granicach twoich; ale oglaszac bedziesz zbawienie na murach twoich, a chwale w bramach twoich.
\par 19 Nie bedziesz mial wiecej slonca za swiatlosc dzienna, a jasnosc miesiaca nie oswieci cie, ale Pan bedzie swiatloscia twoja wieczna, a Bóg twój slawa twoja.
\par 20 Nie zajdzie wiecej slonce twoje, a miesiac twój nie skryje sie; bo Pan bedzie wieczna swiatloscia twoja; a tak dokonaja sie dni smutku twego.
\par 21 Lud takze twój, którzybykolwiek byli sprawiedliwi, na wieki odziedzicza ziemie; beda latorosla szczepienia mego, dzielem rak moich, abym w niem byl uwielbiony.
\par 22 Najmniejszy rozmnozy sie na tysiace, a maluczki poczet w naród niezliczony. Ja Pan czasu swego predko to uczynie.

\chapter{61}

\par 1 Duch Panujacego Pana jest nademna; przeto mie pomazal Pan, abym opowiadal Ewangelije cichym, poslal mie, abym zwiazal rany tych, którzy sa skruszonego serca, abym zwiastowal pojmanym wyzwolenie, a wiezniom otworzenie ciemnicy;
\par 2 Abym oglosil milosciwy rok Panski, i dzien pomsty Boga naszego; abym cieszyl wszystkich placzacych;
\par 3 Abym sprawil radosc placzacym w Syonie, a dal im ozdobe miasto popiolu, olejek wesela miasto smutku, odzienie chwaly miasto scisnionego; i beda nazwani drzewami sprawiedliwosci, szczepieniem Panskiem, abym byl uwielbiony.
\par 4 Tedy pobuduja spustoszenie starodawne, pustynie stare naprawia, a odnowia miasta spustoszone, puste za wielu narodów.
\par 5 Bo sie stawia cudzoziemcy, a pasc beda stada wasze, a synowie cudzoziemców oraczami waszymi i winiarzami waszymi beda.
\par 6 Ale wy kaplanami Panskimi nazwani bedziecie, slugami Boga naszego zwac was beda; majetnosci pogan uzywac bedziecie, a w slawie ich wywyzszeni bedziecie.
\par 7 Za dwojakie pohanbienie i zelzywosc wasze spiewac bedziecie; z dzialu ich, i w ziemi ich dwojakie dziedzictwo posiadziecie, a tak wesele wieczne miec bedziecie.
\par 8 Ja Pan miluje sad, a mam w nienawisci lupiestwo przy calopaleniu; przetoz sprawie, aby uczynki ich dzialy sie w prawdzie, a przymierze wieczne postanowie z nimi.
\par 9 I znajome bedzie miedzy poganami nasienie ich, a potomstwo ich w posrodku narodów; wszyscy, którzy ich ujrza, poznaja ich, ze sa nasieniem, któremu Pan poblogoslawil.
\par 10 Weselac weselic sie beda w Panu, a dusza moja rozraduje sie w Bogu moim; bo mie oblókl w szaty zbawienia, a plaszczem sprawiedliwosci przyodzial mie, jako oblubienca ozdobnego chwala, i jako oblubienice ozdobiona w klejnoty swoje.
\par 11 Bo jako ziemia wydaje plód swój, a jako ogród nasienie swoje wywodzi, tak panujacy Pan wywiedzie sprawiedliwosc i chwale swoje przed wszystkie narody.

\chapter{62}

\par 1 Dla Syonu milczec nie bede, a dla Jeruzalemu nie uspokoje sie, dokad sprawiedliwosc jego nie wynijdzie jako jasnosc, a zbawienie jego jako pochodnia gorzec nie bedzie.
\par 2 I ogladaja narody sprawiedliwosc twoje, i wszyscy królowie slawe twoje i nazwa cie imieniem nowem, które usta Panskie mianowac beda.
\par 3 I bedziesz korona ozdobna w rece Panskiej, i korona królestwa w rece Boga twego.
\par 4 Nie beda cie wiecej zwac opuszczona, i ziemia twoja nie bedzie wiecej zwana spustoszona; ale ty nazywana bedziesz rozkosza moja, a ziemia twoja mezatka; bo Pan bedzie mial rozkosz w tobie, a ziemia twoja bedzie zamezna.
\par 5 Albowiem jako mlodzieniec panne pojmuje, tak cie sobie pojma synowie twoi; a jako sie oblubieniec weseli z oblubienicy, tak sie weselic bedzie z ciebie Bóg twój.
\par 6 Na murach twoich, o Jeruzalem! postawie strózów, którzy przez caly dzien cala noc nigdy nie umilkna; którzy wspominacie Pana, nie milczcie;
\par 7 A nie dawajcie mu odpocznienia, dokad nie utwierdzi, i dokad nie sposobi, aby Jeruzalem bylo slawne na ziemi.
\par 8 Przysiagl Pan przez prawice swoje i przez ramie mocy swojej, mówiac: Nie podam wiecej pszenicy twojej na pokarm nieprzyjaciolom twoim, i nie beda pic cudzoziemcy wina twego, okolo któregos pracowal.
\par 9 Ale ci, którzy je zgromadza, pozywac go, i chwalic Pana beda; a którzy je zbieraja, beda je pic w sieniach swiatnicy mojej.
\par 10 Przechodzcie, przechodzcie przez bramy! gotujcie droge ludowi; wyrównajcie, wyrównajcie goscince; wybierzcie kamienie, podniescie choragiew do narodów.
\par 11 Oto Pan rozkaze obwolac az do konczyn ziemi; powiedzcie córce Syonskiej: Oto zbawiciel twój idzie, oto zaplata jego z nim, a dzielo jego przed nim.
\par 12 I nazwia synów twoich ludem swietym, odkupionymi Panskimi, a ciebie nazwia miastem zacnem i nie opuszczonem.

\chapter{63}

\par 1 Któz to jest, który idzie z Edom, w szatach ubroczonych we krwi z Bocra? Ten przyozdobiony szata swoja, postepujacy w wielkosci mocy swojej? Jam jest, który mówie sprawiedliwosc, dostateczny do wybawienia.
\par 2 Przeczze jest czerwone odzienie twoje? a szaty twoje jako tego, który tloczy w prasie?
\par 3 Prase tloczylem Ja sam, a nikt z ludu nie byl zemne; Ja, mówie, tloczylem nieprzyjaciól w gniewie swym, i podeptalem ich w popedliwosci mojej, az pryskala krew mocarzów ich na szaty moje; a tak wszystko odzienie moje spluskalem.
\par 4 Albowiem dzien pomsty byl w sercu mojem, a rok odkupionych moich przyszedl.
\par 5 Lecz gdym widzial, ze nie bylo pomocnika, azem sie zdumal, ze nikogo nie bylo, coby mie podparl, przetoz mi wybawienie sprawilo ramie moje, a popedliwosc moja, ta mie podparla.
\par 6 I podeptalem narody w gniewie swym, a opoilem je w zapalczywosci mojej, i uderzylem o ziemie mocarzy ich.
\par 7 Milosierdzia Panskie wspominac bede, i chwaly Panskie za wszystko, cokolwiek nam uczynil Pan, i hojnosc dóbr, które pokazal domowi Izraelskiemu wedlug milosierdzia swego, i wedlug wielkich litosci swoich.
\par 8 Bo rzekl: Wzdyc sa ludem moim, sa synami, nie przeniewierza mi sie; przetoz byl ich zbawicielem.
\par 9 We wszelakiem ucisnieniu ich i on byl ucisniony: ale Aniol oblicza jego wybawil ich. Z milosci swej, i z litosci swojej on sam odkupil ich, piastowal ich i nosil ich po wszystkie dni wieków.
\par 10 Ale oni odpornymi byli, i zasmucali Ducha jego Swietego; dla tego obrócil sie im w nieprzyjaciela, a sam walczyl przeciwko nim.
\par 11 I wspominal sobie lud jego na dni starodawne, i na Mojzesza, mówiac: Gdziez jest ten, który ich wywiódl z morza, z pasterzem trzody swojej? Gdziez jest ten, który polozyl w posrodku jego Ducha swego Swietego?
\par 12 Który ich wiódl po prawicy Mojzeszowej ramieniem wielmoznosci swojej? który rozdzielil wody przed nimi, aby sobie uczynil imie wieczne?
\par 13 Który ich przeprowadzil przez przepasci, jako konia po puszczy, a nie szwankowali?
\par 14 Jako gdy bydle na dól zstepuje: tak Duch Panski zwolna prowadzil z nich kazdego; takes wiódl lud swój, abys sobie uczynil imie slawne.
\par 15 Spojrzyjze z nieba, a obacz z mieszkania swietobliwosci twojej, i ozdoby twojej. Gdziez jest gorliwosc twoja, i wielka sila twoja? Gdzie wzruszenie wnetrznosci twoich, i litosci twoich? Przedemnaz zawsciagnione beda?
\par 16 Tys zaiste ojciec nasz: bo Abraham nie wie o nas, a Izrael nie zna nas. Tys, Panie! ojciec nasz, odkupiciel nasz; toc jest od wieku imie twoje.
\par 17 Przeczzes nam, Panie! dopuscil bladzic z dróg twoich? przeczzes zatwardzil serce nasze, abysmy sie ciebie nie bali? Nawrócze sie dla slug twoich, dla pokolenia dziedzictwa twego.
\par 18 Na maly czas posiadl ziemie lud swietobliwosci twojej; nieprzyjaciele nasi podeptali swiatnice twoje.
\par 19 Mysmy twoi od wieku, a nad tymi nigdys nie panowal, ani wzywano imienia twego nad nimi.

\chapter{64}

\par 1 Obys rozdarl niebiosa, i zstapil, aby sie od oblicza twego góry rozplynely!
\par 2 (Jako od gorejacego ognia, ognia roztapiajacego, woda wre,)abys oznajmil imie twoje nieprzyjaciolom twoim, azeby sie od oblicza twego narody zatrwozyly.
\par 3 Jako gdys czynil dziwy, którychesmy sie nie spodziewali; zstapiles, a od oblicza twego góry sie rozplywaly.
\par 4 Czego od wieków nie slyszano ani to do uszów przychodzilo; oko nie widzialo Boga innego oprócz ciebie, coby tak uczynil temu, co nan oczekuje.
\par 5 Zabiezales weselacemu sie i czyniacemu sprawiedliwosc, i tym, którzy na drogach twoich wspominali na cie. Otos sie ty rozgniewal, przeto zesmy grzeszyli na tych drogach ustawicznie, wszakze zachowani bedziemy,
\par 6 Aczkolwiek jestesmy jako nieczysty my wszyscy, i jako szata splugawiona sa wszystkie sprawiedliwosci nasze; przetoz wszyscy opadamy jako lisc, a nieprawosci nasze jako wiatr unosza nas.
\par 7 Nadto niemasz, ktoby wzywal imienia twego, i pobudzil sie do tego, aby sie chwycil ciebie, przynajmniej teraz, gdys zakryl twarz swoje przed nami, a sprawiles, abysmy niszczeli dla nieprawosci naszych.
\par 8 Ale teraz, o Panie! tys jest ojciec nasz, mysmy glina, a tys twórca nasz; a takesmy wszyscy dzielem reki twojej.
\par 9 Nie gniewaj sie, Panie! tak bardzo, a nie na wieki pomnij nieprawosci naszej: oto wejrzyj prosze, mysmy wszyscy ludem twoim.
\par 10 Miasta swietobliwosci twojej obrócone sa w pustynie, Syon w pustynie, a Jeruzalem w spustoszenie obrócone.
\par 11 Dom swietobliwosci naszej i ozdoby naszej, w którym cie chwalili ojcowie nasi, ogniem jest spalony, i wszystkie najkosztowniejsze rzeczy nasze obrócily sie w pustki.
\par 12 Izali nad tem zatrzymasz sie Panie? izali milczec a nas tak bardzo trapic bedziesz?

\chapter{65}

\par 1 Objawilem sie tym, którzy sie o mie nie pytali; znalezionym jest od tych, którzy mie nie szukali; do narodu, który sie nie nazywal imieniem mojem. rzeklem: Otom Ja! otom Ja!
\par 2 Rozciagnalem rece moje na kazdy dzien do ludu upornego, który chodzi droga nie dobra za myslami swemi;
\par 3 Do ludu, który mie jawnie wzrusza do gniewu, ustawicznie ofiarujac w ogrodach, a kadzac na ceglach;
\par 4 Którzy siadaja przy grobach, a przy balwanach swoich nocuja; którzy jedza swinie mieso, i polewke obrzydla z naczynia swego.
\par 5 Mówiac: Odstap precz, nie przystepuj do mnie; bom jest swietobliwszy nizeli ty. Cic sa dymem w nozdrzach moich, i ogniem palajacym przez caly dzien.
\par 6 Oto zapisano to przedemna: Nie zamilcze, ale oddam i odplace na lono ich.
\par 7 Nieprawosci wasze, takze i nieprawosci ojców waszych, mówi Pan, którzy kadzili po górach, a na pagórkach hanbili mie; przetoz odmierze sprawe ich pierwsza na lono ich.
\par 8 Tak mówi Pan: Jako gdyby kto znalazl wino w gronie, i rzeklby: Nie psuj go, bo blogoslawienstwo jest w niem; tak i Ja uczynie dla slug moich, ze ich wszystkich nie wygubie.
\par 9 Bo wywiode z Jakóba nasienie, a z Judy dzierzawce gór moich; i posieda ja wybrani moi, a sludzy moi tam mieszkac beda.
\par 10 A Saron bedzie za pastwisko owcom, a dolina Achor za legowisko wolów ludu mojego, którzy mie szukali.
\par 11 Ale was, którzyscie opuscili Pana, którzy zapominacie na góre swietobliwosci mojej, którzy gotujecie temu wojsku stól, a którzy oddawacie temu pocztowi mokre ofiary:
\par 12 Was, mówie policze pod miecz, tak, ze wy wszyscy do zabicia schylac sie bedziecie, przeto, zem wolal, a nie ozwaliscie mi sie, mówilem, a nie slyszeliscie, alescie czynili, co zlego jest przed oczyma mojemi, a czegom Ja nie chcial, obieraliscie.
\par 13 Przetoz tak mówi panujacy Pan: Oto sludzy moi jesc beda, a wy laknac bedziecie; oto sludzy moi pic beda, a wy pragnac bedziecie; oto sludzy moi weselic sie beda, a wy zawstydzeni bedziecie.
\par 14 Oto sludzy moi wykrzykac beda od radosci serdecznej, a wy bedziecie wolac od bolesci serca, i od skruszenia ducha wyc bedziecie.
\par 15 I zostawicie imie wasze na przeklinanie wybranym moim, gdy was pomorduje panujacy Pan, a slugi swe nazwie innem imieniem.
\par 16 Ten, który sobie bedzie blogoslawil na ziemi, bedzie sobie blogoslawil w Bogu prawdziwym; a kto bedzie przysiegal na ziemi, bedzie przysiegal przez Boga prawdziwego; w zapomnienie zaiste przyjda te uciski pierwsze, a beda zakryte od oczów moich.
\par 17 Albowiem oto Ja tworze niebiosa nowe, i ziemie nowa, a nie beda wspominane rzeczy pierwsze, ani wstapia na serce.
\par 18 Owszem weselcie sie, a radujcie sie na wieki wieków z tego, co Ja stworze; bo oto Ja stworze Jeruzalem na radosc, a lud jego na wesele.
\par 19 I rozraduje sie w Jeruzalemie, a weselic sie bede w ludu moim; a nie bedzie slychac w nim glosu placzu i glosu narzekania.
\par 20 Nie bedzie tam wiecej nikogo w wieku dziecinnym, ani starca, któryby nie dopelnil dni swoich; bo dziecie we stu latach umrze; ale grzesznik, chocby mial i sto lat, przeklety bedzie.
\par 21 Pobuduja tez domy, a beda w nich mieszkali; nasadza tez winnic, a beda jesc owoce ich.
\par 22 Nie beda budowac tak, aby tam inszy mieszkal; nie beda szczepic, aby inny jadl; bo dni ludu mojego beda jako dni drzewa, a dziela rak swoich do zwietszenia uzywac beda wybrani moi.
\par 23 Nie beda robic prózno, ani plodzic beda na postrach; bo beda nasieniem blogoslawionych od Pana, oni i potomkowie ich z nimi.
\par 24 Nadto stanie sie, ze pierwej niz zawolaja, Ja sie ozwe; jeszcze mówic beda, a Ja wyslucham.
\par 25 Wilk z barankiem pasc sie beda spolem; lew jako wól plewy jesc bedzie, a wezowi proch bedzie chlebem jego; nie beda szkodzic ani zatracac na wszystkiej górze swietej mojej, mówi Pan.

\chapter{66}

\par 1 Tak mówi Pan: Niebo jest stolica moja, a ziemia podnózkiem nóg moich. Gdziez tedy bedzie ten dom, który mi zbudujecie? albo gdzie bedzie miejsce odpocznienia mego?
\par 2 Bo to wszystko reka moja uczynila, i nia stanelo to wszystko, mówi Pan. Wszakze Ja na tego patrze, który jest utrapionego i skruszonego ducha, a który drzy na slowo moje.
\par 3 Inaczej ten, kto zabija wolu na ofiare, jakoby zabil czlowieka; kto zabija na ofiare bydlatko, jakoby psa scial; kto ofiaruje ofiare sucha, jakoby krew swinia ofiarowal; kto kadzi kadzidlem, jakoby balwanowi blogoslawil. A jako oni sobie obrali drogi swoje, i w obrzydliwosciach swoich kochala sie dusza ich.
\par 4 Tak i Ja obiore za wynalazki ich, a to, czego sie boja, przywiode na nich, przeto, ze gdym wolal, zaden sie nie ozwal, gdym mówil, nie sluchali, ale czynili to, co zlego jest przed oczyma mojemi, a to, czegom nie chcial, obierali.
\par 5 Sluchajcie slowa Panskiego, wy którzy drzycie na slowo jego. Bracia wasi nienawidzacy was, a wyganiajacy was dla imienia mego, mówia: Niech sie okaze slawa Panska. Okazec sie zaiste ku pociesze waszej; ale oni pohanbieni beda.
\par 6 Glos grzmotu z miasta slyszany bedzie, glos z kosciola, glos Pana oddawajacego zaplate nieprzyjaciolom swoim.
\par 7 Pierwej niz pracowala ku porodzeniu, porodzila, pierwej niz ja ogarnela bolesc, porodzila mezczyzne.
\par 8 Któz slyszal co takowego? Kto widzial co podobnego? Mozez to byc, aby ziemia narodzila ludu za jeden dzien? Izali naród splodzony bywa jednym razem? Ale Syon ledwie poczal pracowac ku porodzeniu, alic porodzil synów swych.
\par 9 Cózbym Ja, który otwieram zywot, rodzic nie mial? mówi Pan. Cózbym Ja, który to czynie, ze rodza, zawartym byl? mówi Bóg twój.
\par 10 Weselcie sie z Jeruzalemem a radujcie sie w nim wszyscy, którzy go milujecie. Wesewlcie sie z nim wielce, wszyscy którzykolwiek plakali nad nim.
\par 11 Przeto, ze ssac bedziecie, i sycic sie piersiami pociech jego, ssac bedziecie, i rozkoszami oplywac w jasnosci chwaly jego.
\par 12 Bo tak mówi Pan: Oto Ja obróce na nich pokój jako rzeke, a slawe narodów jako strumien zalewajacy, i bedziecie ssac; na reku noszeni, i na kolanach rozkosznie piastowani bedziecie.
\par 13 Jako ten, którego cieszy matka jego, tak Ja was cieszyc bede; a tak w Jeruzalemie uciechy miewac bedziecie.
\par 14 Ujrzycie zaiste, a radowac sie bedzie serce wasze, a kosci wasze jako trawa zakwitna. I poznana bedzie reka Panska przy slugach jego; ale sie gniewem zapali przeciwko nieprzyjaciolom swoim.
\par 15 Bo oto Pan w ogniu przyjdzie, a poczwórne jego jako wicher, aby wylal gniew swój w popedliwosci, a lajanie swoje w plomieniu ognia.
\par 16 Pan, mówie, przez ogien sadzic bedzie, i przez miecz swój wszelkie cialo, a pobitych od Pana wiele bedzie.
\par 17 I ci, którzy sie poswiecaja i oczyszczaja w ogrodach, jeden za drugim jawnie; którzy jedza mieso swinie, i inna obrzydlosc, i myszy, koniec takze wezma, mówi Pan.
\par 18 Albowiem Ja znam sprawy ich, i mysli ich; i przyjdzie ten czas, ze zgromadze wszystkie narody, i jezyki, i przyjda a ogladaja chwale moje.
\par 19 I poloze na nich znak, a posle z tych, którzy zachowani beda, do narodów przy morzu do Pul i Lud, którzy ciagna luk do Tubala, i do Jawanu, na wyspy dalekie, które nic o mnie nie slyszaly, i nie widzialy chwaly mojej; i beda opowiadaly chwale moje miedzy narodami.
\par 20 I przywiode wszystkich braci waszych ze wszystkich narodów Panu w dary, na koniach i na wozach, i na lektykach, i na mulach, i na zawidnikach, na góre swietobliwosci mojej do Jeruzalemu, mówi Pan, tak jako przynosza synowie Izraelscy dar w naczyniu czystem do domu Panskiego.
\par 21 I z tych tez nabiore kaplanów i Lewitów, mówi Pan.
\par 22 Bo jako te niebiosa nowe, i ta ziemia nowa, która Ja uczynie, stanie przedemna, mówi Pan, tak stanie nasienie wasze i imie wasze.
\par 23 I stanie sie, ze od nowiu miesiaca do nowiu miesiaca, i od sabatu do sabatu przychodzic bedzie wszelkie cialo, aby sie klanialo przed oblicznoscia moja, mówi Pan.
\par 24 I wynijda a ogladaja trupy ludzi tych, którzy wystapili przeciwko mnie; albowiem robak ich nie zdechnie, a ogien ich nie zgasnie, a beda obrzydliwoscia wszelkiemu cialu.


\end{document}