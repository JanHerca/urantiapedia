\begin{document}

\title{List do Kolosan}


\chapter{1}

\par 1 Pawel, Apostol Jezusa Chrystusa przez wole Boza, i Tymoteusz brat,
\par 2 Tym, którzy sa w Kolosiech, swietym i wiernym braciom w Chrystusie Jezusie. Laska wam i pokój niech bedzie od Boga, Ojca naszego, i od Pana Jezusa Chrystusa.
\par 3 Dziekujemy Bogu i Ojcu Pana naszego Jezusa Chrystusa, zawsze modlac sie za was,
\par 4 Uslyszawszy o wierze waszej w Chrystusie Jezusie i milosci przeciwko wszystkim swietym,
\par 5 Dla nadziei onej wam odlozonej w niebiesiech, o którejscie przedtem slyszeli przez slowo prawdy, to jest Ewangielii,
\par 6 Która przyszla do was jako i na wszystek swiat, i przynosi owoc, jako i u was, od onego dnia, któregoscie uslyszeli i poznali laske Boza w prawdzie.
\par 7 Jakoscie sie tez nauczyli od Epafry, milego spólslugi naszego, który jest wiernym sluga Chrystusowym dla was;
\par 8 Który tez oznajmil nam milosc wasze w duchu.
\par 9 Dlatego i my od onego dnia, któregosmy to uslyszeli, nie przestajemy sie za was modlic i prosic, abyscie byli napelnieni znajomoscia woli jego we wszelkiej madrosci i w wyrozumieniu duchownem.
\par 10 Abyscie chodzili przystojnie przed Panem ku wszelkiemu jego upodobaniu, w kazdym uczynku dobrym owoc przynoszac i rosnac w znajomosci Bozej,
\par 11 Wszelka moca umocnieni bedac wedlug chwalebnej mocy jego, ku wszelkiej cierpliwosci i nieskwapliwosci z radoscia,
\par 12 Dziekujac Ojcu, który nas godnymi uczynil, abysmy byli uczestnikami dziedzictwa swietych w swiatlosci;
\par 13 Który nas wyrwal z mocy ciemnosci i przeniósl do królestwa Syna swego milego,
\par 14 W którym mamy odkupienie przez krew jego, to jest odpuszczenie grzechów;
\par 15 Który jest obrazem Boga niewidzialnego i pierworodny wszystkich rzeczy stworzonych.
\par 16 Albowiem przezen stworzone sa wszystkie rzeczy, które sa na niebie i na ziemi, widzialne i niewidzialne, badz trony, badz panstwa, badz ksiestwa, badz zwierzchnosci, wszystko przezen i dla niego stworzone jest.
\par 17 A on jest przed wszystkimi i wszystko w nim stoi.
\par 18 On tez jest glowa ciala, to jest kosciola, który jest poczatkiem i pierworodnym z umarlych, aby on miedzy wszystkimi przodkowal;
\par 19 Poniewaz sie upodobalo Ojcu, aby w nim wszystka zupelnosc mieszkala.
\par 20 I zeby przezen z soba pojednal wszystko, uczyniwszy pokój przez krew krzyza jego; przezen, mówie, tak to, co jest na ziemi, jako i to, co jest na niebiesiech.
\par 21 I was, którzyscie niekiedy byli oddaleni i nieprzyjaciele umyslem w zlosliwych uczynkach, teraz tez pojednal.
\par 22 Cialem wlasnem swojem przez smierc, aby was wystawil swietemi i niepokalanemi, i bez nagany przed obliczem swojem;
\par 23 Jezli tylko trwacie w wierze ugruntowani i utwierdzeni, a nie poruszeni od nadziei Ewangielii, którascie slyszeli, która jest kazana wszelkiemu stworzeniu, które jest pod niebem, której ja Pawel stalem sie sluga;
\par 24 Który sie teraz raduje w doleglosciach moich dla was i dopelniam ostatków ucisków Chrystusowych na ciele mojem za cialo jego, które jest kosciól.
\par 25 Któregom sie ja stal sluga wedlug daru Bozego, który mi jest dany dla was, abym wypelnil slowo Boze,
\par 26 To jest, tajemnice one, która byla zakryta od wieków i od rodzajów, ale teraz objawiona jest swietym jego.
\par 27 Którym chcial Bóg oznajmic, jakie jest bogactwo tej tajemnicy chwalebnej miedzy poganami, która jest Chrystus miedzy wami, nadzieja ona chwaly;
\par 28 Którego my opowiadamy, napominajac kazdego czlowieka i uczac kazdego czlowieka we wszelkiej madrosci, abysmy wystawili kazdego czlowieka doskonalym w Chrystusie Jezusie;
\par 29 W czem tez pracuje, bojujac wedlug skutecznej mocy jego, która we mnie dzielo swoje poteznie sprawuje.

\chapter{2}

\par 1 Albowiem chce, abyscie wiedzieli jako wielka trudnosc mam o was i o tych, którzy sa w Laodycei i którzykolwiek nie widzieli oblicza mego w ciele;
\par 2 Aby pocieszone byly serca ich, bedac spojone miloscia, a to ku wszelkiemu bogactwu zupelnego i pewnego wyrozumienia, ku poznaniu tajemnicy Boga i Ojca, i Chrystusa,
\par 3 W którym skryte sa wszystkie skarby madrosci i umiejetnosci.
\par 4 A toc mówie, aby was nikt falszywemi dowodami nie oszukal przez wystawna mowe.
\par 5 Bo aczkolwiek nie jestem obecny cialem, ale duchem jestem z wami, radujac sie i widzac porzadek wasz i utwierdzenie wiary waszej w Chrystusa;
\par 6 Przetoz jakoscie przyjeli Pana Jezusa Chrystusa, tak w nim chodzcie,
\par 7 Bedac wkorzenieni i wybudowani na nim, i utwierdzeni w wierze, jakoscie sie nauczyli, obfitujac w niej z dziekowaniem.
\par 8 Patrzciez, zeby was kto sobie w korzysc nie obrócil przez filozofije i przez prózne oszukanie, uczac wedlug ustawy ludzkiej, wedlug zywiolów swiata, a nie wedlug Chrystusa,
\par 9 Gdyz w nim mieszka wszystka zupelnosc bóstwa cielesnie.
\par 10 I jestescie w nim dopelnieni, który jest glowa wszelkiego ksiestwa i zwierzchnosci,
\par 11 W którym i obrzezani jestescie obrzezka nie reka uczyniona, to jest, zewleklszy cialo wszystkich grzechów ciala przez obrzezke Chrystusowa.
\par 12 Pogrzebieni z nim bedac w chrzcie; w którymescie tez spolem z nim wzbudzeni przez wiare, która sprawuje Bóg, który go wzbudzil od umarlych.
\par 13 I was, gdyscie byli umarlymi w grzechach i nieobrzezce ciala waszego, wespól z nim ozywil, odpusciwszy wam wszystkie grzechy.
\par 14 Zmazawszy on, który byl przeciwko nam, cyrograf w ustawach zalezacy, który nam byl przeciwny, zniósl go z posrodku, przybiwszy go do krzyza;
\par 15 I zlupiwszy ksiestwa i mocy, wiódl je na podziw, jawnie tryumfujac z nich sam przez sie.
\par 16 Niechajze was tedy nikt nie sadzi dla pokarmu, albo dla napoju; albo z strony swieta, albo nowiu miesiaca, albo sabatów,
\par 17 Które sa cieniem rzeczy przyszlych, ale prawda jest cialo Chrystusowe.
\par 18 Niechaj was nikt wygranego zakladu nie osadza, który sie dobrowolnie w pokore i w sluzbe Aniolów, których nie widzial, wdaje, prózno sie nadymajac z umyslu ciala swego.
\par 19 A nie trzymajac sie glowy Chrystusa, z którego wszystko cialo przez stawy i zwiazania posilek biorac i wespól spojone bedac, rosnie wzrostem Bozym.
\par 20 Jezliscie tedy umarli z Chrystusem zywiolom swiata tego, przeczze sie, jakobyscie jeszcze zyli swiatu, ustawami bawicie?
\par 21 Mówia niektórzy: Nie dotykaj sie, ani kosztuj, ani ruszaj;
\par 22 Co wszystko kazi sie samem uzywaniem, wedlug przykazan i nauk ludzkich;
\par 23 Które maja ksztalt madrosci w nabozenstwie dobrowolnie obranem i w pokorze, i w niefolgowaniu cialu; wszakze nie maja zadnej wagi, tylko do nasycenia ciala sluza.

\chapter{3}

\par 1 A tak jezliscie powstali z Chrystusem, tego, co jest w górze, szukajcie, gdzie Chrystus na prawicy Bozej siedzi;
\par 2 O tem, co jest w górze, myslcie, nie o tem, co jest na ziemi.
\par 3 Albowiemescie umarli i zywot wasz skryty jest z Chrystusem w Bogu.
\par 4 Ale gdy sie Chrystus, on zywot nasz, pokaze, tedy i wy z nim okazecie sie w chwale.
\par 5 Umartwiajciez tedy czlonki wasze, które sa na ziemi; wszeteczenstwo, nieczystosc, namietnosc, zla pozadliwosc i lakomstwo, które jest balwochwalstwem,
\par 6 Dla których rzeczy przychodzi gniew Bozy na syny odporne.
\par 7 W którychescie i wy niekiedy chodzili, gdyscie zyli w nich.
\par 8 Lecz teraz zlózcie i wy to wszystko: gniew, zapalczywosc, zlosc, bluznierstwo i sprosna mowe z ust waszych.
\par 9 Nie klamcie jedni przeciwko drugim, gdyzescie zewlekli czlowieka starego z uczynkami jego,
\par 10 A oblekliscie nowego tego, który sie odnawia w znajomosc, podlug obrazu tego, który go stworzyl.
\par 11 Gdzie nie masz Greka i Zyda, obrzezki i nieobrzezki, cudzoziemca i Tatarzyna, niewolnika i wolnego; ale wszystko i we wszystkich Chrystus.
\par 12 Przetoz przyobleczcie jako wybrani Bozy, swieci i umilowani, wnetrznosci milosierdzia, dobrotliwosc, pokore, cichosc, cierpliwosc,
\par 13 Znaszajac jedni drugich i odpuszczajac sobie wzajemnie, jezli ma kto przeciw komu skarge: jako i Chrystus odpuscil wam, tak i wy.
\par 14 A nad to wszystko (przyobleczcie) milosc, która jest zwiazka doskonalosci.
\par 15 A pokój Bozy niech rzad prowadzi w sercach waszych, do któregoscie tez powolani w jedno cialo; a badzcie wdziecznymi.
\par 16 Slowo Chrystusowe niechaj mieszka w was obficie ze wszelka madroscia, nauczajac i napominajac samych siebie przez psalmy i hymny, i piesni duchowne, wdziecznie spiewajac w sercach waszych Panu.
\par 17 A wszystko, cokolwiek czynicie w slowie albo w uczynku, wszystko czyncie w imieniu Pana Jezusa, dziekujac Bogu i Ojcu przezen.
\par 18 Zony! Badzcie poddane mezom swym, tak jako przystoi w Panu.
\par 19 Mezowie! Milujcie zony wasze, a nie badzcie surowymi przeciwko nim.
\par 20 Dziatki! Posluszne badzcie rodzicom we wszystkiem; albowiem sie to podoba Panu.
\par 21 Ojcowie! Nie pobudzajcie do gniewu dzieci waszych, aby serca nie tracily.
\par 22 Sludzy! Posluszni badzcie we wszystkiem panom cielesnym, nie sluzac na oko jako ci, co sie ludziom podobac chca, ale w szczerosci serca, bojac sie Boga.
\par 23 A wszystko, cokolwiek czynicie, z duszy czyncie, jako Panu, a nie ludziom.
\par 24 Wiedzac, iz od Pana wezmiecie zaplate dziedzictwa; albowiem Panu Chrystusowi sluzycie.
\par 25 A ten, co krzywde czyni, odniesie zaplate ukrzywdzenia, a nie maszci wzgledu na osoby u Boga.

\chapter{4}

\par 1 Panowie! Sprawiedliwie i slusznie sie z slugami obchodzcie, wiedzac, iz i wy Pana macie w niebiesiech.
\par 2 W modlitwach trwajcie, czujac w nich z dziekowaniem,
\par 3 Modlac sie spolecznie i za nami, aby nam Bóg otworzyl drzwi slowa, zebysmy mówili o tajemnicy Chrystusowej dla której tez jestem zwiazany.
\par 4 Abym ja objawil, jako mi sie godzi mówic.
\par 5 Madrze chodzcie przed obcymi, czas odkupujac.
\par 6 Mowa wasza niech zawsze bedzie przyjemna, sola okraszona, abyscie wiedzieli, jakobyscie kazdemu z osobna odpowiedziec mieli.
\par 7 O wszystkiem, co sie ze mna dzieje oznajmi wam Tychykus, mily brat i wierny sluga, i spólsluga w Panu,
\par 8 Któregom poslal do was dla tego samego, aby sie wywiedzial, co sie z wami dzieje i pocieszyl serca wasze;
\par 9 Z Onezymem, wiernym a milym bratem, który jest z was; ci wam wszystko oznajmia, co sie tu dzieje.
\par 10 Pozdrawia was Arystarchus spólwiezien mój i Marek, siostrzeniec Barnabaszowy, (o któremescie wzieli rozkazanie: Jezliby do was przyszedl, przyjmijciez go.)
\par 11 I Jezus, którego zowia Justem, którzy sa z obrzezki. Ci tylko sa pomocnikami moimi w królestwie Bozem, którzy byli pociecha moja.
\par 12 Pozdrawia was Epafras, który z was jest sluga Chrystusowy, który zawsze bojuje za was w modlitwach, abyscie stali doskonalymi i zupelnymi we wszelkiej woli Bozej.
\par 13 Bo mu daje swiadectwo, iz gorliwa milosc ma przeciwko wam i przeciwko tym, którzy sa w Laodycei i którzy sa w Hijerapolu.
\par 14 Pozdrawia was Lukasz, lekarz mily, takze i Demas.
\par 15 Pozdrówcie braci, którzy sa w Laodycei, i Nymfasa, i zbór, który jest w domu jego.
\par 16 A gdy ten list u was przeczytany bedzie, sprawcie to, aby tez byl w Laodycenskim zborze przeczytany; a ten, który jest pisany z Laodycei i wy tez przeczytajcie.
\par 17 A powiedzcie Archipowi: Patrzaj na to poslugiwanie, któres przyjal w Panu, abys je wypelnil.
\par 18 Pozdrowienie reka moja Pawlowa. Pamietajcie na wiezienie moje. Laska niech bedzie z wami. Amen.


\end{document}