\begin{document}

\title{3 List św. Jana}


\chapter{1}

\par 1 Starszy Gajowi milemu, którego ja miluje w prawdzie.
\par 2 Najmilszy! najprzód zadam, aby ci sie dobrze powodzilo i abys byl zdrów, tak jako sie dobrze powodzi duszy twojej.
\par 3 Albowiem wielcem sie uradowal, gdy przyszli bracia, i dali swiadectwo o twojej prawdzie, jako ty w prawdzie chodzisz.
\par 4 Wiekszej nad te radosci nie mam, jako gdy slysze, iz dziatki moje chodza w szczerosci.
\par 5 Najmilszy! Wiernie czynisz, cokolwiek czynisz przeciwko braciom i przeciw gosciom,
\par 6 Którzy swiadectwo wydali o milosci twojej przed zborem; i dobrze uczynisz, jezli ich odprowadzisz, jako przystoi przed Bogiem.
\par 7 Albowiem dla imienia jego wyszli, nic nie wziawszy od pogan.
\par 8 My tedy takowych powinnismy przyjmowac, abysmy byli pomocnikami prawdzie.
\par 9 Pisalem do zboru waszego; ale Dyjotrefes, który chce byc przedniejszy miedzy nimi, nie przyjmuje nas.
\par 10 Przeto jezli przyjde, przypomne uczynki jego, które czyni, slowami zlemi obmawiajac nas, a nie majac dosyc na tem, i sam braci nie przyjmuje, i tym, co by przyjac chcieli, zabrania i ze zboru ich wylacza.
\par 11 Najmilszy! nie nasladuj zlego, ale dobrego. Kto dobrze czyni, z Boga jest; ale kto zle czyni, nie widzial Boga.
\par 12 Demetryjuszowi swiadectwo jest dane od wszystkich, i od samej prawdy; lecz i my swiadectwo o nim dajemy, a wiecie, iz swiadectwo nasze prawdziwe jest.
\par 13 Wielem mial pisac; lecz nie chce pisac inkaustem i piórem;
\par 14 Bo mam nadzieje, ze cie w rychle ujrze, a tedy ustnie mówic bedziemy.
\par 15 Pokój tobie. Pozdrawiaja cie przyjaciele. Pozdrów i ty przyjaciól z imienia.


\end{document}