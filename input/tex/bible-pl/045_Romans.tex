\begin{document}

\title{List do Rzymian}


\chapter{1}

\par 1 Pawel, sluga Jezusa Chrystusa, powolany Apostol, odlaczony ku opowiadaniu Ewangielii Bozej;
\par 2 Która przedtem obiecal przez proroki swoje w pismach swietych,
\par 3 O Synu swoim, który sie narodzil z nasienia Dawidowego wedlug ciala;
\par 4 A pokazal sie Synem Bozym moznie, wedlug Ducha poswiecenia, przez zmartwychwstanie, to jest o Jezusie Chrystusie, Panu naszym.
\par 5 Przez którego wzielismy laske i urzad apostolski ku posluszenstwu wiary miedzy wszystkimi narody, dla imienia jego;
\par 6 Miedzy którymi jestescie i wy powolani od Jezusa Chrystusa.
\par 7 Wszystkim, którzy jestescie w Rzymie, umilowanym Bozym, powolanym swietym, laska niech bedzie wam i pokój od Boga, Ojca naszego, i od Pana Jezusa Chrystusa.
\par 8 Najprzód tedy dziekuje Bogu mojemu przez Jezusa Chrystusa za was wszystkich, iz wiara wasza slynie po wszystkim swiecie.
\par 9 Swiadkiem mi bowiem jest on Bóg, któremu sluze w duchu moim w Ewangielii Syna jego, iz bez przestanku wzmianke o was czynie,
\par 10 Zawsze w modlitwach moich proszac, izby mi sie wzdy kiedykolwiek droga zdarzyla za wola Boza przyjsc do was.
\par 11 Albowiem pragne was widziec, abym wam mógl udzielic jakiego daru duchownego ku utwierdzeniu waszemu;
\par 12 To jest, abysmy sie u was zobopólnie ucieszyli przez spoleczna wiare, i wasze, i moje.
\par 13 A nie chce, abyscie i wy wiedziec nie mieli, bracia! zem czesto zamyslal pójsc do was; (alem byl dotad zawsciagniony), abym mial jaki pozytek i miedzy wami, jako i miedzy inszymi pogany.
\par 14 I Grekom, i grubym narodom, i madrym, i glupim jestem dluznikiem,
\par 15 Tak, iz ile ze mnie jest, gotowym jest i wam, którzyscie w Rzymie, Ewangielije opowiadac.
\par 16 Albowiem nie wstydze sie za Ewangielije Chrystusowa, poniewaz jest moca Boza ku zbawieniu kazdemu wierzacemu, Zydowi najprzód, potem i Greczynowi.
\par 17 Bo sprawiedliwosc Boza w niej bywa objawiona z wiary w wiare, jako napisano: Ze sprawiedliwy z wiary zyc bedzie.
\par 18 Bo gniew Bozy objawia sie z nieba przeciwko wszelkiej niepoboznosci i niesprawiedliwosci tych ludzi, którzy zatrzymuja prawde Boza w niesprawiedliwosci.
\par 19 Przeto iz co moze byc wiedziano o Bogu, jest w nich jawno, gdyz im Bóg objawil.
\par 20 Bo rzeczy jego niewidzialne od stworzenia swiata, przez rzeczy uczynione widzialne bywaja, to jest ona wieczna jego moc i bóstwo, na to, aby oni byli bez wymówki.
\par 21 Przeto iz poznawszy Boga, nie chwalili jako Boga, ani mu dziekowali, owszem znikczemnieli w myslach swoich i zacmilo sie bezrozumne serce ich;
\par 22 Mieniac sie byc madrymi, zglupieli;
\par 23 I odmienili chwale nieskazitelnego Boga w podobienstwo obrazu skazitelnego czlowieka i ptaków, i czworonogich zwierzat, i plazów.
\par 24 A przetoz podal je Bóg pozadliwosciom serc ich ku nieczystosci, aby lzyli ciala swoje miedzy soba,
\par 25 Jako te, którzy odmienili prawde Boza w klamstwo i chwalili stworzenie, i sluzyli mu raczej niz Stworzycielowi, który jest blogoslawiony na wieki. Amen.
\par 26 Dlatego podal je Bóg w namietnosci sromotne, gdyz i niewiasty ich odmienily uzywanie przyrodzone w ono, które jest przeciwko przyrodzeniu.
\par 27 Takze i mezczyzni opusciwszy przyrodzone uzywanie niewiasty, zapalili sie w swej pozadliwosci jedni ku drugim, mezczyzna z mezczyzna hanbe plodzac, a nagrode nalezaca bledowi swemu na sie biorac.
\par 28 A jako sie im nie upodobalo miec w znajomosci Boga, tak tez Bóg je podal w umysl opaczny, aby czynili, co nie przystoi;
\par 29 Napelnieni bedac wszelakiej nieprawosci, wszeteczenstwa, przewrotnosci, lakomstwa, zlosci, pelni zazdrosci, morderstwa, sporu, zdrady, zlych obyczajów;
\par 30 Zausznicy, obmówcy, Boga nienawidzacy, potwarcy, pyszni, chlubni, wynalazcy zlych rzeczy, rodzicom nieposluszni,
\par 31 Bezrozumni, przymierza nie trzymajacy, bez przyrodzonej milosci, nieprzejednani i niemilosierni;
\par 32 Którzy poznawszy prawo Boze, iz ci, co takowe rzeczy czynia, godni sa smierci, nie tylko sami je czynia, ale tez przestawaja z tymi, co je czynia.

\chapter{2}

\par 1 Przetoz jestes bez wymówki, o czlowiecze! który osadzasz; bo w czem drugiego osadzasz, samego siebie osadzasz, poniewaz toz czynisz, który drugiego osadzasz.
\par 2 Lecz wiemy, iz sad Bozy jest wedlug prawdy przeciwko tym, którzy takowe rzeczy czynia.
\par 3 Czy mniemasz, o czlowiecze! który osadzasz tych, co takowe rzeczy czynia, a sam je czynisz, ze ty ujdziesz sadu Bozego?
\par 4 Czy bogactwy dobrotliwosci jego i cierpliwosci, i nieskwapliwosci pogardzasz, nie wiedzac, iz cie dobrotliwosc Boza do pokuty prowadzi?
\par 5 Ale podlug zatwardzialosci twojej i serca niepokutujacego skarbisz sobie samemu gniew na dzien gniewu i objawienia sprawiedliwego sadu Bozego.
\par 6 Który odda kazdemu podlug uczynków jego;
\par 7 Tym, którzy przez wytrwanie w uczynku dobrym szukaja slawy i czci i nieskazitelnosci, odda zywot wieczny;
\par 8 A zas swarliwym i prawdzie nieposlusznym, lecz poslusznym niesprawiedliwosci, odda zapalczywosc i gniew;
\par 9 Utrapienie i ucisk duszy kazdego czlowieka, który zlosc popelnia, Zyda najprzód, potem i Greka;
\par 10 A chwale i czesc, i pokój wszelkiemu czyniacemu dobre, Zydowi najprzód, potem i Grekowi.
\par 11 Albowiem nie masz wzgledu na osoby u Boga.
\par 12 A którzykolwiek bez zakonu zgrzeszyli, bez zakonu tez pogina; a którzykolwiek w zakonie zgrzeszyli, przez zakon sadzeni beda.
\par 13 (Gdyz nie sluchacze zakonu sprawiedliwymi sa u Boga; ale czyniciele zakonu usprawiedliwieni beda.)
\par 14 Bo poniewaz poganie nie majacy zakonu, z przyrodzenia czynia, co jest w zakonie, ci, zakonu nie majac, sami sobie sa zakonem;
\par 15 Którzy ukazuja skutek zakonu, napisany na sercach swych, z poswiadczaniem sumienia ich i mysli wespól siebie oskarzajacych albo tez wymawiajacych,
\par 16 W dzien, gdy sadzic bedzie Bóg skryte rzeczy ludzkie wedlug Ewangielii mojej przez Jezusa Chrystusa.
\par 17 Oto sie ty nazywasz Zydem i polegasz na zakonie, a chlubisz sie Bogiem.
\par 18 I znasz wole jego i rozeznajesz rzeczy rózne od niej, wycwiczony bedac z zakonu;
\par 19 I masz za to, zes jest wodzem slepych, swiatloscia tych, którzy sa w ciemnosci;
\par 20 Mistrzem bezrozumnych, nauczycielem niemowlatek, majac ksztalt znajomosci i prawdy w zakonie.
\par 21 Który tedy uczysz drugiego, siebie samego nie uczysz? Który opowiadasz, zeby nie kradziono, kradniesz?
\par 22 Który mówisz, zeby nie cudzolozono, cudzolozysz? który sie brzydzisz balwany, swiete rzeczy kradniesz?
\par 23 Który sie chlubisz zakonem, przez przestepstwo zakonu Boga lzysz?
\par 24 Albowiem imie Boze dla was bluznione bywa miedzy pogany, jako napisano.
\par 25 Boc obrzezanie jest pozyteczne, jezlibys pelnil zakon; ale jezlibys byl przestepca zakonu, twoje obrzezanie stalo sie nieobrzezka.
\par 26 Jezliby tedy nieobrzezka przestrzegala praw zakonnych, azaz jego nieobrzezka nie bedzie przyczyna za obrzezke?
\par 27 I osadzi nieobrzezka z przyrodzenia zakon pelniaca ciebie, który przez litere i obrzezke jestes przestepca zakonu.
\par 28 Albowiem nie ten jest Zydem, który jest Zydem na jawie, ani to jest obrzezka, która jest na jawie na ciele;
\par 29 Ale który jest w skrytosci Zydem i obrzezka serca, która jest w duchu, nie w literze, której chwala nie jest z ludzi, ale z Boga.

\chapter{3}

\par 1 Czemze tedy zacniejszy Zyd? albo co za pozytek obrzezki?
\par 2 Wielki z kazdej miary. Albowiem to najpierwsza, iz im zwierzone byly wyroki Boze.
\par 3 Bo cóz na tem, jezli niektórzy nie uwierzyli? Azaz niedowiarstwo ich niszczy wiare Boza?
\par 4 Nie daj tego Boze! I owszem niech Bóg bedzie prawdziwy, a wszelki czlowiek klamca, jako napisano: A abys byl usprawiedliwiony w mowach twoich, a zebys zwyciezyl, gdybys sadzil.
\par 5 Jezli tedy niesprawiedliwosc nasza Boza sprawiedliwosc zaleca, cóz rzeczemy? Azaz niesprawiedliwy jest Bóg, który gniew przywodzi? (Po ludzku mówie.)
\par 6 Nie daj tego Boze! albowiem jakozby Bóg sadzil swiat?
\par 7 Bo jezli prawda Boza przez moje klamstwo obfitowala ku chwale jego, czemuz jeszcze i ja bywam sadzony jako grzesznik?
\par 8 A nie raczej tak mówimy: (jako nas szkaluja i jako niektórzy udawaja, zebysmy mówili:) Bedziemy czynic zle rzeczy, aby przyszly dobre? Których potepienie jest sprawiedliwe.
\par 9 Cóz tedy? Mamyz nad nie? Zadnym sposobem; gdyzesmy przedtem dowiedli, ze Zydowie i Grekowie, wszyscy sa pod grzechem,
\par 10 Jako napisano: Nie masz sprawiedliwego ani jednego;
\par 11 Nie masz rozumnego i nie masz, kto by szukal Boga.
\par 12 Wszyscy sie odchylili, wespól sie stali nieuzytecznymi, nie masz kto by czynil dobre, nie masz az do jednego.
\par 13 Grobem otworzonym jest gardlo ich, jezykami swemi zdradzali, jad zmiji pod wargami ich.
\par 14 Których usta napelnione sa przeklinania i gorzkosci;
\par 15 Nogi ich predkie sa ku wylewaniu krwi;
\par 16 Skruszenie z bieda w drogach ich,
\par 17 A drogi pokoju nie poznali;
\par 18 Nie masz bojazni Bozej przed oczami ich.
\par 19 A wiemy, iz cokolwiek zakonowi mówi, tym którzy sa pod zakonem, mówi, aby wszelkie usta byly zatulone i aby wszystek swiat podlegal karaniu Bozemu.
\par 20 Przeto z uczynków zakonu nie bedzie usprawiedliwione zadne cialo przed oblicznoscia jego, gdyz przez zakon jest poznanie grzechu.
\par 21 Lecz teraz bez zakonu sprawiedliwosc Boza objawiona jest, majaca swiadectwo z zakonu i z proroków;
\par 22 Sprawiedliwosc, mówie, Boza przez wiare Jezusa Chrystusa ku wszystkim i na wszystkie wierzace; boc róznosci nie masz.
\par 23 Albowiem wszyscy zgrzeszyli i nie dostaje im chwaly Bozej.
\par 24 A bywaja usprawiedliwieni darmo z laski jego przez odkupienie, które sie stalo w Chrystusie Jezusie.
\par 25 Którego Bóg wystawil ublaganiem przez wiare we krwi jego, ku okazaniu sprawiedliwosci swojej przez odpuszczenie przedtem popelnionych grzechów w cierpliwosci Bozej,
\par 26 Ku okazaniu sprawiedliwosci swojej w terazniejszym czasie, na to, aby on byl sprawiedliwym i usprawiedliwiajacym tego, który jest z wiary Jezusowej.
\par 27 Gdziez tedy jest chluba? Odrzucona jest. Przez któryz zakon? Czyli uczynków? Nie, ale przez zakon wiary.
\par 28 Przetoz mamy za to, ze czlowiek bywa usprawiedliwiony wiara bez uczynków zakonu.
\par 29 Izali Bóg jest tylko Bogiem Zydów? izali tez nie pogan? Zaiste i pogan.
\par 30 Poniewaz jeden jest Bóg, który usprawiedliwi obrzezke z wiary i nieobrzezke przez wiare.
\par 31 To tedy zakon niszczymy przez wiare? Nie daj tego Boze! i owszem zakon stanowimy.

\chapter{4}

\par 1 Cóz tedy, rzeczemy, znalazl Abraham, ojciec nasz wedlug ciala?
\par 2 Bo jezli Abraham z uczynków jest usprawiedliwiony, ma sie czem chlubic, ale nie u Boga.
\par 3 Albowiem cóz Pismo mówi? Uwierzyl Abraham Bogu i przyczytano mu to za sprawiedliwosc.
\par 4 A robiacemu zaplata nie bywa przyczytana podlug laski, ale podlug dlugu;
\par 5 Nie robiacemu zas, lecz wierzacemu w tego, który usprawiedliwia niepoboznego, przyczytana bywa wiara jego za sprawiedliwosc.
\par 6 Jako i Dawid powiada, ze blogoslawienstwo czlowieka jest, któremu Bóg przyczyta sprawiedliwosc bez uczynków, mówiac:
\par 7 Blogoslawieni, których odpuszczone sa nieprawosci, a których zakryte sa grzechy;
\par 8 Blogoslawiony maz, któremu Pan grzechu nie przyczyta.
\par 9 To tedy blogoslawienstwo tylko na obrzezke przychodzi, czy tez na nieobrzezke? Gdyz mówimy, iz wiara Abrahamowi jest przyczytana za sprawiedliwosc.
\par 10 Jakoz mu tedy jest przyczytano? Gdy byl w obrzezce, czyli w nieobrzezce? Nie w obrzezce, ale w nieobrzezce.
\par 11 I przyjal znak obrzezki za pieczec sprawiedliwosci onej wiary, która byla w nieobrzezce, na to, aby byl ojcem wszystkich wierzacych w nieobrzezce, aby i onym przyczytana byla sprawiedliwosc;
\par 12 I aby byl ojcem obrzezki, nie tylko tych, którzy sa z obrzezki, ale tez i tych, którzy chodza stopami wiary ojca naszego Abrahama, która byla w nieobrzezce.
\par 13 Albowiem nie przez zakon sie stala obietnica Abrahamowi albo nasieniu jego, aby byl dziedzicem swiata, ale przez sprawiedliwosc wiary.
\par 14 Bo jezli ci, którzy sa z zakonu, dziedzicami sa, tedyc zniszczala wiara i wniwecz sie obrócila obietnica.
\par 15 Gdyz zakon gniew sprawuje; albowiem gdzie zakonu nie masz, tam ani przestepstwa.
\par 16 Przetoz z wiary jest dziedzictwo, aby bylo z laski, i zeby byla warowna obietnica wszystkiemu nasieniu, nie tylko temu, które jest z zakonu, ale i temu, które jest z wiary Abrahamowej, który jest ojcem nas wszystkich;
\par 17 (Jako napisano: Ojcem wielu narodów wystawilem cie) przed Bogiem, któremu uwierzyl, który ozywia umarle i który przywoluje te rzeczy, których nie masz, i jakoby byly.
\par 18 Który (Abraham) przeciwko nadziei w nadzieje uwierzyl, ze sie stanie ojcem wielu narodów wedlug tego, co mu powiedziano: Tak bedzie nasienie twoje.
\par 19 A nie bedac slabym w wierze, nie patrzyl na cialo swoje juz obumarle, majac okolo stu lat, ani na obumarly zywot Sary.
\par 20 O obietnicy tedy Bozej nie watpil z niedowiarstwa; ale sie umocnil wiara i dal chwale Bogu,
\par 21 Bedac tez tego pewien, ze cokolwiek on obiecal, mocen jest i uczynic.
\par 22 Przetoz przyczytano mu to za sprawiedliwosc.
\par 23 A nie napisano tego dla niego samego, iz mu to przyczytano,
\par 24 Ale i dla nas, którym ma byc przyczytano, którzy wierzymy w tego, który wzbudzil Jezusa, Pana naszego, z martwych;
\par 25 Który wydany jest dla grzechów naszych, a wstal z martwych dla usprawiedliwienia naszego.

\chapter{5}

\par 1 Bedac tedy usprawiedliwieni z wiary, pokój mamy z Bogiem przez Pana naszego Jezusa Chrystusa;
\par 2 Przez któregosmy tez przystep otrzymali wiara ku lasce, w której stoimy i chlubimy sie nadzieja chwaly Bozej.
\par 3 A nie tylko to, ale sie tez chlubimy z ucisków, wiedzac, iz ucisk cierpliwosc sprawuje,
\par 4 A cierpliwosc doswiadczenie, a doswiadczenie nadzieje,
\par 5 A nadzieja nie pohanbia, przeto iz milosc Boza rozlana jest w sercach naszych przez Ducha Swietego, który nam jest dany.
\par 6 Albowiem Chrystus, gdy jeszcze bylismy mdlymi, wedlug czasu umarl za niepobozne.
\par 7 Choc ledwie by kto umarl za sprawiedliwego; wszakze za dobrego snacby sie kto umrzec wazyl.
\par 8 Lecz zaleca Bóg milosc swoje ku nam, ze gdy jeszcze bylismy grzesznymi, Chrystus za nas umarl.
\par 9 Daleko tedy wiecej teraz usprawiedliwieni bedac krwia jego, zachowani bedziemy przez niego od gniewu.
\par 10 Bo jezlize bedac nieprzyjaciolmi, pojednanismy z Bogiem przez smierc Syna jego; daleko wiecej bedac pojednani, zachowani bedziemy przez zywot jego.
\par 11 A nie tylko to, ale sie tez chlubimy Bogiem przez Pana naszego Jezusa Chrystusa, przez któregosmy teraz pojednanie otrzymali.
\par 12 Przetoz jako przez jednego czlowieka grzech wszedl na swiat, a przez grzech smierc; tak tez na wszystkich ludzi smierc przyszla, poniewaz wszyscy zgrzeszyli.
\par 13 Albowiem az do zakonu grzech byl na swiecie; ale grzech nie bywa przyczytany, gdy zakonu nie masz.
\par 14 Lecz smierc królowala od Adama az do Mojzesza i nad tymi, którzy nie zgrzeszyli, na podobienstwo przestepstwa Adamowego, który jest wzorem onego, który mial przyjsc.
\par 15 Ale nie jako upadek, tak i dar z laski; albowiem jezli przez upadek jednego wiele ich pomarlo, daleko wiecej laska Boza i dar z laski onego jednego czlowieka Jezusa Chrystusa na wiele ich oplywala.
\par 16 A dar nie jest taki, jako to, co przyszlo przez jednego, który zgrzeszyl. Albowiem wina jest z jednego upadku ku potepieniu, ale dar z laski z wielu upadków ku usprawiedliwieniu.
\par 17 Albowiem jezli dla jednego upadku smierc królowala przez jednego, daleko wiecej, którzy obfitosc onej laski i dar sprawiedliwosci przyjmuja, w zywocie królowac beda przez tegoz jednego Jezusa Chrystusa.
\par 18 Przetoz tedy jako przez jednego upadek na wszystkich ludzi przyszla wina ku potepieniu; tak tez przez jednego usprawiedliwienie na wszystkich ludzi przyszedl dar ku usprawiedliwieniu zywota.
\par 19 Bo jako przez nieposluszenstwo jednego czlowieka wiele sie ich stalo grzesznymi; tak przez posluszenstwo jednego czlowieka wiele sie ich stalo sprawiedliwymi.
\par 20 A zakon przytem nastapil, aby obfitowal grzech; lecz gdzie sie grzech rozmnozyl, tam laska tem wiecej obfitowala.
\par 21 Aby jako grzech królowal ku smierci, tak tez aby laska królowala przez sprawiedliwosc ku zywotowi wiecznemu przez Jezusa Chrystusa, Pana naszego.

\chapter{6}

\par 1 Cóz tedy rzeczemy? Zostaniemyz w grzechu, aby laska obfitowala?
\par 2 Nie daj tego Boze! albowiem którzysmy umarli grzechowi, jakoz jeszcze w nim zyc bedziemy?
\par 3 Azaz nie wiecie, iz którzykolwiek ochrzczeni jestesmy w Chrystusa Jezusa, w smierc jego ochrzczeni jestesmy?
\par 4 Pogrzebienismy tedy z nim przez chrzest w smierc, aby jako Chrystus wzbudzony jest z martwych przez chwale ojcowska, tak zebysmy i my w nowosci zywota chodzili.
\par 5 Bo jezlizesmy z nim wszczepieni w podobienstwo smierci jego, tedy tez i w podobienstwo zmartwychwstania wszczepieni z nim bedziemy.
\par 6 To wiedzac, ze stary nasz czlowiek pospolu jest z nim ukrzyzowany, aby cialo grzechu bylo zniszczone, zebysmy juz wiecej nie sluzyli grzechowi;
\par 7 Bo ktoc umarl, usprawiedliwiony jest od grzechu.
\par 8 Jezlismy tedy z Chrystusem umarli, wierzymy, iz tez z nim zyc bedziemy,
\par 9 Wiedzac, ze Chrystus powstawszy z martwych, wiecej nie umiera i smierc mu wiecej nie panuje.
\par 10 Bo iz umarl, grzechowi raz umarl, a iz zyje, zyje Bogu.
\par 11 Tak tez i wy rozumiejcie, zescie wy umarlymi grzechowi, alescie zywymi Bogu w Chrystusie Jezusie, Panu naszym.
\par 12 Niechze tedy nie króluje grzech w smiertelnem ciele waszem, zebyscie mu posluszni byli w pozadliwosciach jego.
\par 13 Ani stawiajcie czlonków waszych orezem niesprawiedliwosci grzechowi: ale stawiajcie siebie samych Bogu, jako z umarlych zywi, i czlonki wasze orezem sprawiedliwosci Bogu.
\par 14 Albowiem grzech panowac nad wami nie bedzie; bo jestescie nie pod zakonem, ale pod laska.
\par 15 Cóz tedy? Bedziemyz grzeszyli, zesmy nie pod zakonem, ale pod laska? Nie daj tego Boze!
\par 16 Azaz nie wiecie, ze komu sie stawiacie za slugi ku posluszenstwu, tegoscie slugami, komuscie posluszni; badz grzechowi ku smierci, badz posluszenstwu ku sprawiedliwosci?
\par 17 Ale chwala Bogu, ze bywszy slugami grzechu, usluchaliscie z serca sposobu onej nauki, którejscie sie poddali.
\par 18 A bedac uwolnieni od grzechu, staliscie sie niewolnikami sprawiedliwosci.
\par 19 Po ludzku mówie dla mdlosci ciala waszego. Albowiem jakoscie stawiali czlonki wasze na sluzbe nieczystosci i nieprawosci ku czynieniu nieprawosci: tak teraz stawiajcie czlonki wasze na sluzbe sprawiedliwosci ku poswieceniu.
\par 20 Bo pókiscie byli slugami grzechu, byliscie wolnymi od sprawiedliwosci:
\par 21 Jakizescie tedy naonczas pozytek mieli onych rzeczy, za które sie teraz wstydzicie? Bo koniec onych jest smierc.
\par 22 Lecz teraz, bedac uwolnieni od grzechu, a zniewoleni Bogu, macie pozytek swój ku poswieceniu, a koniec zywot wieczny.
\par 23 Albowiem zaplata za grzech jest smierc; ale dar z laski Bozej jest zywot wieczny, w Chrystusie Jezusie, Panu naszym.

\chapter{7}

\par 1 Azaz nie wiecie, bracia! (bo powiadomym zakonu mówie), iz zakon panuje nad czlowiekiem, póki zyje?
\par 2 Albowiem niewiasta, która jest za mezem, póki zyw maz, obowiazana mu jest zakonem; a jezliby maz umarl, uwolniona jest od zakonu mezowego.
\par 3 Przetoz tedy, póki maz zyje, bedzie zwana cudzoloznica, jezliby zona inszego meza zostala; a jezliby maz jej umarl, wolna jest od zakonu onego, aby nie byla cudzoloznica, chocby sie inszego meza zona stala.
\par 4 A tak, bracia moi! i wyscie umartwieni zakonowi przez cialo Chrystusowe, abyscie sie stali inszego, to jest tego, który wzbudzony jest z martwych, abysmy owoc przynosili Bogu.
\par 5 Albowiem gdysmy byli w ciele, namietnosci grzechów, które sie wzniecaly przez zakon, mocy dokazywaly w czlonkach naszych ku przynoszeniu owocu smierci.
\par 6 Lecz teraz stalismy sie wolni od zakonu, umarlszy temu, w którymesmy byli zatrzymani, abysmy Bogu sluzyli w nowosci ducha, a nie w starosci litery.
\par 7 Cóz tedy rzeczemy? Iz zakon jest grzechem? Nie daj tego Boze! I owszemem grzechu nie poznal, tylko przez zakon; bo i o pozadliwosci bym byl nie wiedzial, by byl zakon nie rzekl: Nie bedziesz pozadal.
\par 8 Lecz grzech wziawszy przyczyne przez przykazanie, sprawil we mnie wszelka pozadliwosc; albowiem bez zakonu grzech jest martwy.
\par 9 I jam zyl niekiedy bez zakonu; lecz gdy przyszlo przykazanie, grzech ozyl, a jam umarl.
\par 10 I znalazlo sie, ze to przykazanie, które mialo byc ku zywotowi, jest mi ku smierci.
\par 11 Gdyz grzech, wziawszy przyczyne przez przykazanie, zawiódl mie i przez nie zabil mie.
\par 12 A tak zakon jest swiety i przykazanie swiete, i sprawiedliwe, i dobre.
\par 13 To tedy dobre staloz mi sie smiercia? Nie daj tego Boze! I owszem grzech, aby sie pokazal byc grzechem, sprawil mi smierc przez dobre, zeby sie stal nader grzeszacym on grzech przez ono przykazanie.
\par 14 Bo wiemy, iz zakon jest duchowny, alem ja cielesny, zaprzedany pod grzech.
\par 15 Albowiem tego, co czynie, nie pochwalam; bo nie, co chce, to czynie, ale czego nienawidze, to czynie.
\par 16 A jezli czego nie chce, to czynie, przyzwalam zakonowi, ze dobry jest.
\par 17 Juz tedy teraz nie ja to czynie, ale grzech we mnie mieszkajacy;
\par 18 Gdyz wiem, ze nie mieszka we mnie (to jest w ciele mojem) dobre; albowiem chec jest we mnie, ale wykonac to, co jest dobrego, nie znajduje.
\par 19 Bo nie czynie dobrego, które chce; ale zle, którego nie chce, to czynie.
\par 20 A jezliz ja to czynie, czego nie chce, juz ja wiecej nie czynie tego, ale grzech, który we mnie mieszka.
\par 21 Znajduje tedy ten zakon w sobie, gdy chce dobre czynic, ze sie mnie zle trzyma.
\par 22 Albowiem kocham sie w zakonie Bozym wedlug wewnetrznego czlowieka.
\par 23 Lecz widze inszy zakon w czlonkach moich, odporny zakonowi umyslu mego i który mie zniewala pod zakon grzechu, który jest w czlonkach moich.
\par 24 Nedznyz ja czlowiek! Któz mie wybawi z tego ciala smierci?
\par 25 Dziekuje Bogu przez Jezusa Chrystusa, Pana naszego. Przetoz tedy ja sam umyslem sluze zakonowi Bozemu, lecz cialem zakonowi grzechu.

\chapter{8}

\par 1 Przetoz teraz zadnego potepienia nie masz tym, który bedac w Chrystusie Jezusie nie wedlug ciala chodza, ale wedlug Ducha.
\par 2 Albowiem zakon Ducha zywota, który jest w Chrystusie Jezusie, uwolnil mie od zakonu grzechu smierci.
\par 3 Bo co niemoznego bylo zakonowi, w czem on byl slaby dla ciala, Bóg poslawszy Syna swego w podobienstwie grzesznego ciala i dla grzechu, potepil grzech w ciele,
\par 4 Aby ona sprawiedliwosc zakonu byla wypelniona w nas, którzy nie wedlug ciala chodzimy, ale wedlug Ducha.
\par 5 Albowiem, którzy sa wedlug ciala, o tem mysla, co jest cielesnego; ale którzy sa wedlug Ducha, mysla o tem, co jest duchowego.
\par 6 Gdyz zmysl ciala jest smierc; ale zmysl ducha jest zywot i pokój,
\par 7 Przeto iz zmysl ciala jest nieprzyjacielem Bogu; bo sie zakonowi Bozemu nie poddaje, gdyz tez i nie moze.
\par 8 Przetoz którzy sa w ciele, Bogu sie podobac nie moga.
\par 9 Lecz wy nie jestescie w ciele, ale w duchu, gdyz Duch Bozy mieszka w was: a jezli kto Ducha Chrystusowego nie ma, ten nie jest jego.
\par 10 Ale jezli Chrystus w was jest, tedy cialo jest martwe dla grzechu, a duch jest zywy dla sprawiedliwosci.
\par 11 A jezli Duch tego, który Jezusa wzbudzil z martwych, mieszka w was, ten który wzbudzil Chrystusa z martwych, ozywi i smiertelne ciala wasze przez Ducha swego, który w was mieszka.
\par 12 A tak, bracia! dluznikami jestesmy nie cialu, abysmy wedlug ciala zyli.
\par 13 Albowiem jezlibyscie wedlug ciala zyli, pomrzecie; ale jezlibyscie Duchem sprawy ciala umartwili, zyc bedziecie.
\par 14 Bo którzykolwiek Duchem Bozym prowadzeni bywaja, ci sa synami Bozymi.
\par 15 Gdyzescie nie wzieli ducha niewoli znowu ku bojazni, alescie wzieli Ducha przysposobienia synowskiego, przez którego wolamy: Abba, to jest Ojcze!
\par 16 Tenze duch poswiadcza duchowi naszemu, iz jestesmy dziecmi Bozymi.
\par 17 A jezliz dziecmi, tedy i dziedzicami, dziedzicami wprawdzie Bozymi, a spóldziedzicami Chrystusowymi, jezli tylko z nim cierpimy, abysmy tez z nim byli uwielbieni.
\par 18 Albowiem, (bracia!) mam za to, iz utrapienia terazniejszego czasu nie sa godne onej przyszlej chwaly, która sie ma objawic w nas.
\par 19 Bo troskliwe wygladanie stworzenia oczekuje objawienia synów Bozych.
\par 20 Gdyz stworzenie marnosci jest poddane, nie dobrowolnie, ale dla tego, który je poddal,
\par 21 Pod nadzieja, ze i samo stworzenie bedzie uwolnione z niewoli skazenia na wolnosc chwaly dziatek Bozych.
\par 22 Bo wiemy, iz wszystko stworzenie wespól wzdycha i wespól boleje az dotad.
\par 23 A nie tylko ono stworzenie, ale i my, którzy mamy pierwiastki Ducha, i my sami w sobie wzdychamy, oczekujac przysposobienia synowskiego, odkupienia ciala naszego.
\par 24 Albowiem nadziejasmy zbawieni. A nadzieja widoma nie jest nadzieja; bo co kto widzi, przecz sie tego spodziewa?
\par 25 Ale czego nie widzimy, tego sie spodziewamy i tego przez cierpliwosc oczekujemy.
\par 26 Takze tez i Duch dopomaga mdlosciom naszym. Albowiem o co bysmy sie modlic mieli, jako potrzeba, nie wiemy; ale tenze Duch przyczynia sie za nami wzdychaniem niewymownem.
\par 27 A ten, który sie serc bada, wie, który jest zmysl Ducha, poniewaz wedlug Boga przyczynia sie za swietymi.
\par 28 A wiemy, iz tym, którzy miluja Boga, wszystkie rzeczy dopomagaja ku dobremu, to jest tym, którzy wedlug postanowienia Bozego powolani sa.
\par 29 Albowiem, które on przejrzal, te tez przenaznaczyl, aby byli przypodobani obrazowi Syna jego, zeby on byl pierworodnym miedzy wieloma bracmi,
\par 30 A które przenaznaczyl, te tez powolal; a które powolal, te tez usprawiedliwil; a które usprawiedliwil, te tez uwielbil.
\par 31 Cóz tedy rzeczemy na to? Jezli Bóg za nami, któz przeciwko nam?
\par 32 Który ani wlasnemu Synowi nie przepuscil, ale go za nas wszystkich wydal: jakoz by wszystkiego z nim nie darowal nam?
\par 33 Któz bedzie skarzyl na wybrane Boze? Bóg jest, który usprawiedliwia.
\par 34 Któz jest, co by je potepil? Chrystus jest, który umarl, owszem i zmartwychwstal, który tez jest na prawicy Bozej, który sie tez przyczynia za nami.
\par 35 Któz nas odlaczy od milosci Chrystusowej? czyli utrapienie? czyli ucisk? czyli przesladowanie? czyli glód? czyli nagosc? czyli niebezpieczenstwo? czyli miecz?
\par 36 Jako napisano: Dla ciebie caly dzien zabijani bywamy, poczytanismy jako owce na rzez naznaczone;
\par 37 Ale w tem wszystkiem przezwyciezamy przez tego, który nas umilowal.
\par 38 Albowiem pewienem tego, iz ani smierc, ani zywot, ani Aniolowie, ani ksiestwa, ani mocarstwa, ani terazniejsze ani przyszle rzeczy,
\par 39 Ani wysokosc, ani glebokosc, ani zadne insze stworzenie nie bedzie nas moglo odlaczyc od milosci Bozej, która jest w Jezusie Chrystusie, Panu naszym.

\chapter{9}

\par 1 Prawde mówie w Chrystusie, a nie klamie, w czem mi poswiadcza sumienie moje przez Ducha Swietego:
\par 2 Ze mam wielki smutek i nieustawajacy ból w sercu mojem.
\par 3 Albowiem zadalbym sam, abym sie stal odlaczonym od Chrystusa za braci moich, za pokrewnych moich wedlug ciala.
\par 4 Którzy sa Izraelczycy, których jest przysposobienie synowskie i chwala, i przymierza, i zakonu danie, i sluzba Boza, i obietnice;
\par 5 Których sa ojcowie i z których poszedl Chrystus ile wedlug ciala, który jest nad wszystkimi Bóg blogoslawiony na wieki. Amen.
\par 6 Lecz nie mozna, zeby mialo upasc slowo Boze; albowiem nie wszyscy, którzy sa z Izraela, sa Izraelem;
\par 7 Ani iz sa nasieniem Abrahamowem, wszyscy sa dziecmi; ale rzeczono: W Izaaku bedzie tobie nazwane nasienie;
\par 8 To jest, nie dzieci ciala sa dziecmi Bozymi; ale dzieci obietnicy bywaja w nasienie policzone.
\par 9 Albowiem obietnicy slowo to jest: O tym wlasnie czasie przyjde, a Sara bedzie miala syna;
\par 10 A nie tylko to, ale i Rebeka, gdy z jednego ojca naszego Izaaka brzemienna zostala.
\par 11 Gdy sie jeszcze byly dziatki nie narodzily, ani co dobrego albo zlego uczynily, aby sie ostalo postanowienie Boze wedlug wybrania, nie z uczynków, ale z tego, który powoluje,
\par 12 Rzeczono jej, ze wiekszy bedzie sluzyl mniejszemu;
\par 13 Jako napisano: Jakóbam umilowal, alem Ezawa mial w nienawisci.
\par 14 Cóz tedy rzeczemy? Jestze niesprawiedliwosc u Boga? Nie daj tego Boze!
\par 15 Albowiem do Mojzesza mówi: Zmiluje sie, nad kim sie zmiluje; a zlituje sie, nad kim sie zlituje.
\par 16 A przetoz nie zalezy na tym co chce, ani na tym, co biezy, ale na Bogu, który sie zmilowywa.
\par 17 Albowiem mówi Pismo do Faraona: Na tom cie samo wzbudzil, abym okazal moc moje na tobie, a izby opowiadane bylo imie moje po wszystkiej ziemi.
\par 18 A tak nad kim chce, zmilowywa sie, a kogo chce, zatwardza.
\par 19 Ale mi rzeczesz: Przeczze sie jeszcze uskarza? bo któz sie sprzeciwil woli jego?
\par 20 I owszem, o czlowiecze! któzes ty jest, który spór wiedziesz z Bogiem? Izali lepianka rzecze lepiarzowi: Przeczzes mie tak uczynil?
\par 21 Izali nie ma mocy garncarz nad glina, zeby z tejze gliny uczynil jedno naczynie ku uczciwosci, a drugie ku zelzywosci?
\par 22 A jezliz Bóg chcac okazac gniew i znajoma uczynic moznosc swoje, znosil w wielkiej cierpliwosci naczynia gniewu na zginienie zgotowane,
\par 23 A izby znajome uczynil bogactwo chwaly swojej nad naczyniem milosierdzia, które zgotowal ku chwale;
\par 24 Których i powolal, to jest nas, nie tylko z Zydów, ale i z poganów.
\par 25 Jako tez u Ozeasza mówi: Nazwie lud, który nie byl moim, ludem moim, a one, która nie byla umilowana, nazwie umilowana.
\par 26 I stanie sie, ze na tem miejscu, gdzie im mawiano: Nie jestescie wy ludem moim, tam nazwani beda synami Boga zywego.
\par 27 A Izajasz wola nad Izraelem, mówiac: Chocby liczba synów Izraelskich byla jako piasek morski, ostatki zachowane beda.
\par 28 Albowiem sprawe skonczy i skróci w sprawiedliwosci; sprawe zaiste skrócona uczyni Pan na ziemi.
\par 29 I jako przedtem powiedzial Izajasz: By nam byl Pan zastepów nie zostawil nasienia, bylibysmy sie stali jako Sodoma i Gomorze bylibysmy podobni.
\par 30 Cóz tedy rzeczemy? To, iz poganie, którzy nie szukali sprawiedliwosci, dostapili sprawiedliwosci, a sprawiedliwosci, która jest z wiary.
\par 31 A Izrael szukajac zakonu sprawiedliwosci, nie doszedl zakonu sprawiedliwosci.
\par 32 Dlaczegoz? Iz nie z wiary, ale jako z uczynków zakonu jej szukali; albowiem sie obrazili o kamien obrazenia,
\par 33 Jako napisano: Oto klade w Syonie kamien obrazenia i opoke otracenia, a wszelki, który w niego wierzy, nie bedzie pohanbiony.

\chapter{10}

\par 1 Bracia! przychylna wola serca mego i modlitwa, która czynie do Boga za Izraelem, jestci ku zbawieniu.
\par 2 Albowiem daje im swiadectwo, iz gorliwosc ku Bogu maja, ale nie wedlug wiadomosci.
\par 3 Bo nie znajac sprawiedliwosci Bozej, a chcac wlasna sprawiedliwosc wystawic, sprawiedliwosci Bozej nie byli poddani.
\par 4 Albowiem koniec zakonu jest Chrystus ku sprawiedliwosci kazdemu wierzacemu.
\par 5 Gdyz Mojzesz pisze o sprawiedliwosci, która jest z zakonu, iz ktobykolwiek te rzeczy czynil, przez nie zyc bedzie.
\par 6 Ale sprawiedliwosc, która jest z wiary, tak mówi: Nie mów w sercu swem: Kto wstapi na niebo? to jest Chrystusa na dól sprowadzic:
\par 7 Albo kto zstapi do przepasci? to jest Chrystusa od umarlych wyprowadzic.
\par 8 Ale (Mojzesz) cóz mówi: Blisko ciebie jest slowo w ustach twoich i w sercu twojem. Toc jest slowo wiary, które opowiadamy:
\par 9 Ze jezlibys usty wyznal Pana Jezusa i uwierzylbys w sercu twojem, ze go Bóg z martwych wzbudzil, zbawiony bedziesz.
\par 10 Albowiem sercem wierzono bywa ku sprawiedliwosci, ale sie usty wyznanie dzieje ku zbawieniu.
\par 11 Bo Pismo mówi: Wszelki, kto w niego wierzy, nie bedzie pohanbiony;
\par 12 Gdyz nie masz róznosci miedzy Zydem i Grekiem; bo tenze Pan wszystkich, bogaty jest ku wszystkim, którzy go wzywaja.
\par 13 Kazdy bowiem, kto by wzywal imienia Panskiego, zbawiony bedzie.
\par 14 Jakoz tedy wzywac beda tego, w którego nie uwierzyli? a jako uwierza w tego, o którym nie slyszeli? a jako uslysza bez kaznodziei?
\par 15 Jakoz tez beda kazac, jezliby nie byli poslani? Jako napisano: O jako sliczne sa nogi tych, którzy opowiadaja pokój, tych, którzy opowiadaja dobre rzeczy.
\par 16 Alec nie wszyscy posluszni byli Ewangielii; albowiem Izajasz mówi: Panie! któz uwierzyl kazaniu naszemu?
\par 17 Wiara tedy jest z sluchania, a sluchanie przez slowo Boze.
\par 18 Ale mówie: Izali nie slyszeli? i owszem na wszystke ziemie wyszedl glos ich i na konczyny okregu ziemi slowa ich.
\par 19 Ale mówie: Izali tego nie poznal Izrael? Pierwszy Mojzesz mówi: Ja was do zawisci pobudze przez naród, który nie jest narodem, przez naród nierozumny rozdraznie was.
\par 20 A Izajasz smialosci uzywajac mówi: Jestem znaleziony od tych, którzy mnie nie szukali, i jestem objawiony tym, którzy sie o mnie nie pytali.
\par 21 Ale przeciwko Izraelowi mówi: Caly dzien wyciagalem rece moje do ludu upornego i sprzeciwiajacego sie.

\chapter{11}

\par 1 Mówie tedy: Izali Bóg odrzucil lud swój? Nie daj tego Boze! Albowiem i jam jest Izraelczyk z nasienia Abrahamowego, z pokolenia Benjaminowego.
\par 2 Nie odrzucilci Bóg ludu swego, który przejrzal. Azaz nie wiecie, co mówi Pismo o Elijaszu? jako sie przyczynia do Boga przeciwko Izraelowi mówiac:
\par 3 Panie! Proroki twoje pomordowali i oltarze twoje zburzyli, a zostalem ja sam i szukaja duszy mojej.
\par 4 Ale cóz mu mówi Boska odpowiedz? Zostawilem sobie siedm tysiecy mezów, którzy nie sklonili kolana Baalowi.
\par 5 Tak tedy i terazniejszego czasu ostatki podlug wybrania z laski zostaly.
\par 6 A poniewaz z laski, tedyc juz nie z uczynków, inaczej laska juz by nie byla laska; a jezli z uczynków, juzci nie jest laska; inaczej uczynek juz by nie byl uczynkiem.
\par 7 Cóz tedy? Czego Izrael szuka, tego nie dostapil; ale wybrani dostapili, a inni zatwardzeni sa,
\par 8 (Jako napisano: Dal im Bóg ducha twardego snu, oczy, aby nie widzieli i uszy, aby nie slyszeli), az do dzisiejszego dnia.
\par 9 A Dawid mówi: Niechaj im bedzie stól ich sidlem i ulowieniem i otraceniem i odplata.
\par 10 Niech zacmione beda oczy ich, aby nie widzieli, a grzbietu ich zawsze nachylaj.
\par 11 Mówie tedy: Azaz sie potkneli, aby padli? Nie daj tego Boze! Ale przez ich upadek doszlo zbawienie pogan, aby je do zawisci przywiódl.
\par 12 A poniewaz upadek ich jest bogactwem swiata, a umniejszenie ich bogactwem pogan, jakoz daleko wiecej ich zupelnosc?
\par 13 Albowiem mówie wam poganom, ilem ja jest Apostolem pogan, uslugiwanie moje zalecam,
\par 14 Azazbym jako ku zawisci pobudzil cialo moje i zbawilbym niektóre z nich.
\par 15 Albowiem jezlic odrzucenie ich jest pojednaniem swiata, cóz bedzie przyjecie ich, tylko ozycie od umarlych?
\par 16 Poniewaz jezli pierwiastki swiete, tedyc i zaczynienie; a jezli korzen swiety, tedyc i galezie.
\par 17 A jezli niektóre z galezi odlamane sa, a ty, którys byl plonna oliwa, jestes wszczepiony zamiast nich i stales sie uczestnikiem korzenia i tlustosci oliwnego drzewa:
\par 18 Nie chlubze sie przeciw galeziom, bo jezli sie chlubisz, wiedzze, iz nie ty korzenia nosisz, ale korzen ciebie.
\par 19 Ale rzeczesz: Odlamane sa galezie, abym ja byl wszczepiony.
\par 20 Dobrze; dla niedowiarstwa odlamane sa, ale ty wiara stoisz; nie badzze hardej mysli, ale sie bój.
\par 21 Albowiem jezli Bóg przyrodzonym galeziom nie przepuscil, wiedz, ze i tobie nie przepusci.
\par 22 Obaczze tedy dobrotliwosc i srogosc Boza; przeciwko tym wprawdzie, którzy upadli, srogosc, ale przeciwko tobie dobrotliwosc, jezlibys trwal w dobroci; inaczej i ty bedziesz wyciety.
\par 23 Alec i oni, jezli nie beda trwali w niedowiarstwie, wszczepieni zas beda, gdyz mocny jest Bóg one zasie wszczepic.
\par 24 Albowiem jezlis ty jest wyciety z oliwy, z przyrodzenia plonnej, a przeciwko przyrodzeniu jestes wszczepiony w dobra oliwe, jakoz daleko wiecej, którzy sa wedlug przyrodzenia, wszczepieni beda w swoje wlasna oliwe!
\par 25 Bo nie chce, abyscie nie mieli wiedziec, bracia! tej tajemnicy, (zebyscie nie byli sami u siebie madrymi), iz zatwardzenie z czesci przyszlo na Izraela, póki by nie weszla zupelnosc pogan.
\par 26 A tak wszystek Izrael bedzie zbawiony, jako napisano: Przyjdzie z Syonu wybawiciel i odwróci niepoboznosci od Jakóba.
\par 27 A toc bedzie przymierze moje z nimi, gdy odejme grzechy ich.
\par 28 A tak wedlug Ewangielii nieprzyjaciólmi sa dla was; lecz wedlug wybrania sa milymi dla ojców.
\par 29 Albowiem darów swoich i wezwania Bóg nie zaluje.
\par 30 Bo jako i wy niekiedy nie wierzyliscie Bogu, ale teraz dostapiliscie milosierdzia dla ich niedowiarstwa,
\par 31 Tak i oni teraz stali sie nieposlusznymi, aby dla milosierdzia wam okazanego i oni milosierdzia dostapili.
\par 32 Albowiem zamknal je Bóg wszystkie w niedowiarstwo, aby sie nad wszystkimi zmilowal.
\par 33 O glebokosci bogactwa i madrosci, i znajomosci Bozej! Jako sa niewybadane sady jego i niedoscignione drogi jego!
\par 34 Bo któz poznal umysl Panski? albo kto byl rajca jego?
\par 35 Albo kto mu co pierwej dal, a bedzie mu zasie oddano?
\par 36 Albowiem z niego i przez niego i w nim sa wszystkie rzeczy; jemu niech bedzie chwala na wieki. Amen.

\chapter{12}

\par 1 Prosze was tedy, bracia! przez litosci Boze, abyscie stawiali ciala wasze ofiara zywa, swieta, przyjemna Bogu, to jest rozumna sluzbe wasze.
\par 2 A nie przypodobywajcie sie temu swiatu, ale sie przemiencie przez odnowienie umyslu waszego na to, abyscie doswiadczyli, która jest wola Boza dobra, przyjemna i doskonala.
\par 3 Albowiem powiadam przez laske, która mi jest dana, kazdemu, co jest miedzy wami, aby wiecej o sobie nie rozumial, nizeli potrzeba rozumiec; ale zeby o sobie rozumial skromnie, tak jako komu Bóg udzielil miare wiary.
\par 4 Albowiem jako w jednem ciele wiele czlonków mamy, ale wszystkie czlonki nie jednoz dzielo maja:
\par 5 Tak wiele nas jest jednem cialem w Chrystusie, alesmy z osobna jedni drugich czlonkami.
\par 6 Majac tedy rózne dary wedlug laski, która nam jest dana; jezli proroctwo, niech bedzie wedlug sznuru wiary;
\par 7 Jezli poslugowanie, niech bedzie w poslugowaniu; jezli kto naucza, niech trwa w nauczaniu;
\par 8 Jezli kto napomina, w napominaniu; kto rozdaje, w szczerosci; kto przelozony jest, w pilnosci; kto czyni milosierdzie, niech czyni z ochota.
\par 9 Milosc niech bedzie nieobludna; miejcie w obrzydliwosci zle; imajac sie dobrego.
\par 10 Miloscia braterska jedni ku drugim sklonni badzcie, uczciwoscia jedni drugich uprzedzajac.
\par 11 W pracy nie leniwi, duchem palajacy, Panu sluzacy;
\par 12 W nadziei sie weselacy, w ucisku cierpliwi, w modlitwie ustawiczni;
\par 13 Potrzebom swietych udzielajacy, goscinnosci nasladujacy.
\par 14 Dobrorzeczcie tym, którzy was przesladuja; dobrorzeczcie, a nie przeklinajcie.
\par 15 Weselcie sie z weselacymi, a placzcie z placzacymi.
\par 16 Badzcie miedzy soba jednomyslni, wysoko o sobie nie rozumiejac, ale sie do niskich naklaniajac.
\par 17 (Bracia!) nie badzcie madrymi sami u siebie; zadnemu zlem za zle nie oddawajcie, obmysliwajac to, co jest uczciwego przed wszystkimi ludzmi.
\par 18 Jezli mozna, ile z was jest, ze wszystkimi ludzmi pokój miejcie.
\par 19 Nie mscijcie sie sami, najmilsi: ale dajcie miejsce gniewowi; albowiem napisano: Mnie pomsta, a Ja oddam, mówi Pan.
\par 20 Jezli tedy laknie nieprzyjaciel twój, nakarm go; jezli pragnie, napój go: bo to czyniac, wegle rozpalone zgarniesz na glowe jego.
\par 21 Nie daj sie zwyciezyc zlemu, ale zle dobrem zwyciezaj.

\chapter{13}

\par 1 Kazda dusza niech bedzie zwierzchnosciom wyzszym poddana: boc nie masz zwierzchnosci, tylko od Boga; a te, które sa zwierzchnosci, od Boga sa postanowione.
\par 2 A tak, kto sie zwierzchnosci sprzeciwia, Bozemu sie postanowieniu sprzeciwia; a którzy sie sprzeciwiaja, sami sobie potepienie zjednaja.
\par 3 Albowiem przelozoni nie sa na postrach dobrym uczynkom, ale zlym. A chcesz sie nie bac zwierzchnosci, czyn, co jest dobrego, a bedziesz mial pochwale od niej;
\par 4 Bozym bowiem jest sluga tobie ku dobremu. Ale jezli uczynisz, co jest zlego, bój sie; boc nie darmo miecz nosi, gdyz jest sluga Bozym, mszczacym sie w gniewie nad czyniacym, co jest zlego.
\par 5 Przetoz trzeba byc poddanym nie tylko dla gniewu, ale i dla sumienia.
\par 6 Albowiem dla tego tez podatki dajecie, gdyz sa slugami Bozymi, którzy tego samego ustawicznie pilnuja.
\par 7 Oddawajciez tedy kazdemu, cobyscie powinni: komu podatek, temu podatek, komu clo, temu clo, komu bojazn, temu bojazn; komu czesc, temu czesc.
\par 8 Nikomu nic winni nie badzcie, tylko abyscie sie spolecznie milowali; bo kto miluje blizniego, zakon wypelnil.
\par 9 Gdyz to przykazanie: Nie bedziesz cudzolozyl, nie bedziesz zabijal, nie bedziesz kradl, nie bedziesz falszywie swiadczyl, nie bedziesz pozadal, i jezli które insze jest przykazanie, w tem slowie sumownie sie zamyka, mianowicie: Bedziesz milowal blizniego twego, jako siebie samego.
\par 10 Milosc blizniemu zlosci nie wyrzadza; a tak wypelnieniem zakonu jest milosc.
\par 11 A to czyncie, wiedzac czas, iz juz przyszla godzina, abysmy sie ze snu ocucili; albowiem teraz blizsze nas jest zbawienie, anizeli kiedysmy uwierzyli.
\par 12 Noc przeminela, a dzien sie przyblizyl; odrzucmyz tedy uczynki ciemnosci, a obleczmy sie w zbroje swiatlosci.
\par 13 Chodzmy uczciwie jako we dnie, nie w biesiadach i w pijanstwach, nie we wszeteczenstwach i rozpustach, nie w poswarkach ani w zazdrosci;
\par 14 Ale obleczcie sie w Pana Jezusa Chrystusa, a nie czyncie starania o ciele ku wykonywaniu pozadliwosci.

\chapter{14}

\par 1 A tego, który jest w wierze slaby, przyjmujcie, nie na sprzeczania okolo sporów.
\par 2 Boc jeden wierzy, iz moze jesc wszystko, a drugi bedac slaby, jarzyne jada.
\par 3 Ten, który je, niech lekce nie wazy tego, który nie je; a który nie je, niech nie potepia tego, który je; albowiem go Bóg przyjal.
\par 4 Ktos ty jest, co sadzisz cudzego sluge? Panu wlasnemu stoi, albo upada, a ostoi sie; albowiem go Bóg moze utwierdzic.
\par 5 Bo jeden róznosc czyni miedzy dniem a dniem, a drugi kazdy dzien za równo sadzi; kazdy niech bedzie dobrze upewniony w zmysle swoim.
\par 6 Kto przestrzega dnia, Panu przestrzega; a kto nie przestrzega dnia, Panu nie przestrzega; kto je, Panu je, bo dziekuje Bogu; a kto nie je, Panu nie je, a dziekuje Bogu.
\par 7 Albowiem nikt z nas sobie nie zyje, i nikt sobie nie umiera.
\par 8 Bo choc zyjemy, Panu zyjemy; choc umieramy, Panu umieramy; przetoz choc i zyjemy, choc i umieramy, Panscy jestesmy.
\par 9 Gdyz na to Chrystus i umarl i powstal i ozyl, aby i nad umarlymi i nad zywymi panowal.
\par 10 Ale ty przeczze potepiasz brata twego? Albo tez ty czemu lekcewazysz brata twego, gdyz wszyscy staniemy przed stolica Chrystusowa?
\par 11 Bo napisano: Jako zyje Ja, mówi Pan, iz mi sie kazde kolano ukloni, i kazdy jezyk wyslawiac bedzie Boga.
\par 12 A przeto kazdy z nas sam za sie odda rachunek Bogu.
\par 13 A tak juz nie sadzmy jedni drugich; ale raczej to rozsadzajcie, abyscie nie kladli obrazenia, ani dawali zgorszenia bratu.
\par 14 Wiem i upewnionym jest przez Pana Jezusa, iz nie masz nic przez sie nieczystego, tylko temu, który mniema co byc nieczystem, to temu nieczyste jest.
\par 15 Lecz jezli dla pokarmu brat twój bywa zasmucony, juz nie postepujesz wedlug milosci; nie zatracaj pokarmem twoim tego, za którego Chrystus umarl.
\par 16 Niechze tedy dobro wasze bluznione nie bedzie.
\par 17 Albowiem królestwo Boze nie jest pokarm ani napój, ale sprawiedliwosc i pokój i radosc w Duchu Swietym:
\par 18 Bo kto w tych rzeczach sluzy Chrystusowi, mily jest Bogu, a przyjemny ludziom.
\par 19 Przetoz tedy nasladujmy tego, co nalezy do pokoju i do spolecznego budowania.
\par 20 Dla pokarmu nie psuj sprawy Bozej. Wszystkoc wprawdzie jest czyste; ale zle jest czlowiekowi, który je z obrazeniem.
\par 21 Dobrac jest, nie jesc miesa i nie pic wina, ani zadnej rzeczy, która sie brat twój obraza albo gorszy albo slabieje.
\par 22 Ty wiare masz? miejze ja sam u siebie przed Bogiem. Blogoslawiony, który samego siebie nie sadzi w tem, co ma za dobre.
\par 23 Ale kto jest watpliwy, jezliby jadl, potepiony jest, iz nie je z wiary; albowiem cokolwiek nie jest z wiary, grzechem jest.

\chapter{15}

\par 1 A tak powinnismy znosic, my, którzysmy mocni, mdlosci slabych, a nie podobac sie samym sobie.
\par 2 Przetoz kazdy z nas niech sie blizniemu podoba ku dobremu dla zbudowania;
\par 3 Poniewaz i Chrystus nie podobal sie samemu sobie, ale jako napisano: Uragania uragajacych tobie przypadly na mie.
\par 4 Bo cokolwiek przedtem napisano, ku naszej nauce napisano, abysmy przez cierpliwosc i przez pocieche Pism nadzieje mieli.
\par 5 A Bóg cierpliwosci i pociechy niech wam da, abyscie byli jednomyslni miedzy soba wedlug Chrystusa Jezusa.
\par 6 Abyscie jednomyslnie jednemi usty wyslawiali Boga, Ojca Pana naszego Jezusa Chrystusa.
\par 7 Przetoz przyjmujcie jedni drugich, jako i Chrystus przyjal nas do chwaly Bozej.
\par 8 Bo powiadam, iz Jezus Chrystus byl sluga obrzezki dla prawdy Bozej, aby potwierdzil obietnice ojcom uczynione,
\par 9 A poganie zeby za milosierdzie chwalili Boga, jako napisano: Dlatego bede cie wyslawial miedzy pogany i imieniowi twemu spiewac bede.
\par 10 I zasie mówi: Weselcie sie poganie z ludem jego.
\par 11 I zasie: Chwalcie Pana wszyscy poganie, a wyslawiajcie go wszyscy ludzie.
\par 12 I zasie Izajasz mówi: Bedzie korzen Jessego, a który powstanie, aby panowal nad pogany, w nim poganie nadzieje pokladac beda.
\par 13 A Bóg nadziei niech was napelni wszelka radoscia i pokojem w wierze, abyscie obfitowali w nadziei przez moc Ducha Swietego.
\par 14 A pewienem, bracia moi! i ja sam o was, ze jestescie i wy sami pelni dobroci, napelnieni wszelka znajomoscia, i mozecie jedni drugich napominac.
\par 15 A pisalem do was, bracia! poniekad smielej, jakoby was napominajac przez laske, która mi jest dana od Boga.
\par 16 Na to, abym byl sluga Jezusa Chrystusa miedzy pogany, swietobliwie pracujac w Ewangielii Bozej, aby ofiara pogan stala sie przyjemna, poswiecona przez Ducha Swietego.
\par 17 Mam sie tedy czem chlubic w Chrystusie Jezusie, w rzeczach Bozych.
\par 18 Albowiem nie smialbym mówic tego, czego by nie sprawowal Chrystus przez mie w przywodzeniu ku posluszenstwu pogan, przez slowo i przez uczynek,
\par 19 Przez moc znamion i cudów, przez moc Ducha Bozego, tak izem od Jeruzalemu i okolicznych krain az do Iliryku napelnil Ewangielija Chrystusowa;
\par 20 A to tak usilujac kazac Ewangielije, gdzie i mianowany nie byl Chrystus, abym na cudzym fundamencie nie budowal.
\par 21 Ale jako napisano: Którym nie powiadano o nim, ogladaja; a którzy o nim nie slyszeli, zrozumieja.
\par 22 Dlaczegom tez czesto miewal przeszkody, zem do was przyjsc nie mógl.
\par 23 Lecz teraz nie mam wiecej miejsca w tych samych krainach, a majac chec przyjsc do was od wielu lat.
\par 24 Kiedykolwiek pójde do Hiszpanii, przyjde do was: bo sie spodziewam, ze tamtedy idac ujrze was, a ze wy mie tam poprowadzicie, kiedy sie pierwej z wami troszeczke uciesze.
\par 25 A teraz ide do Jeruzalemu, uslugujac swietym.
\par 26 Albowiem sie upodobalo Macedonii i Achai, nieco spólnie zlozyc na ubogich swietych, którzy sa w Jeruzalemie.
\par 27 Owa podobalo sie im i sa ich dluznikami; bo poniewaz dóbr ich duchownych poganie sie uczestnikami stali, powinni im tez sa cielesnemi uslugiwac.
\par 28 Przetoz gdy to wykonam, a onym jako zapieczetowany ten pozytek oddam, pójde przez was do Hiszpanii;
\par 29 A wiem, iz gdy przyjde do was, z hojnem blogoslawienstwem Ewangielii Chrystusowej przyjde.
\par 30 A prosze was, bracia! przez Pana naszego Jezusa Chrystusa i przez milosc Ducha, abyscie wespól ze mna pracowali w modlitwach za mie do Boga,
\par 31 Abym byl wybawiony od tych, którzy sa niewiernymi w ziemi Judzkiej, a izby usluga moja, która wykonywam przeciw Jeruzalemowi, przyjemna byla swietym;
\par 32 Abym z radoscia przyszedl do was za wola Boza i z wami sie wespól ucieszyl.
\par 33 A Bóg pokoju niech bedzie z wami wszystkimi. Amen.

\chapter{16}

\par 1 A zalecam wam Febe, siostre nasze, która jest sluzebnica zboru Kienchreenskiego;
\par 2 Abyscie ja przyjeli w Panu, jako przystoi swietym, i stali przy niej, w którejkolwiek by was rzeczy potrzebowala; albowiem i ona wielom gospody uzyczala, az i mnie samemu.
\par 3 Pozdrówcie Pryscylle i Akwile, pomocniki moje w Chrystusie Jezusie;
\par 4 (Którzy za dusze moje swojej wlasnej szyi nadstawiali; którym nie ja sam dziekuje, ale i wszystkie zbory poganskie.)
\par 5 Takze zbór, który jest w domu ich. Pozdrówcie Epeneta milego mojego, który jest pierwiastkiem Achai do Chrystusa.
\par 6 Pozdrówcie Maryje, która wiele pracowala dla nas.
\par 7 Pozdrówcie Andronika i Junijasza, krewnych moich i spólwiezni moich, którzy znacznymi sa miedzy Apostolami, którzy i przede mna byli w Chrystusie.
\par 8 Pozdrówcie Amplijasa, milego mojego w Panu.
\par 9 Pozdrówcie Urbana, pomocnika naszego w Chrystusie i Stachyna mnie milego.
\par 10 Pozdrówcie Apellesa doswiadczonego w Chrystusie. Pozdrówcie tych, którzy sa z domu Arystobulowego.
\par 11 Pozdrówcie Herodijona, pokrewnego mojego. Pozdrówcie tych, którzy sa z domu Narcyssowego, tych, którzy sa w Panu.
\par 12 Pozdrówcie Tryfene i Tryfose, które pracuja w Panu. Pozdrówcie Persyde mila, która wiele pracowala w Panu.
\par 13 Pozdrówcie Rufa, wybranego w Panu i matke jego, i moje.
\par 14 Pozdrówcie Asynkryta, Flegonta, Hermana, Patrobe, Hermena i braci, którzy sa z nimi.
\par 15 Pozdrówcie Filologa i Julije, Nerego i siostre jego, i Olimpa, i wszystkich swietych, którzy sa z nimi.
\par 16 Pozdrówcie jedni drugich z pocalowaniem swietem. Pozdrawiaja was zbory Chrystusowe.
\par 17 A prosze was, bracia! abyscie upatrywali tych, którzy czynia rozerwania i zgorszenia przeciwko tej nauce, którejscie sie wy nauczyli; i chroncie sie ich.
\par 18 Albowiem takowi Panu naszemu Jezusowi Chrystusowi nie sluza, ale wlasnemu brzuchowi swemu, a przez lagodna mowe i pochlebstwo serca prostych zwodza.
\par 19 Bo posluszenstwo wasze wszystkich doszlo. A przetoz raduje sie z was; ale chce, abyscie byli madrymi na dobre, a prostymi na zle.
\par 20 A Bóg pokoju zetrze szatana pod nogi wasze w rychle. Laska Pana naszego Jezusa Chrystusa niech bedzie z wami. Amen.
\par 21 Pozdrawiaja was Tymoteusz, pomocnik mój, i Lucyjusz, i Jazon, i Sosypater, pokrewni moi.
\par 22 Pozdrawiam was w Panu ja Tercyjusz, którym ten list pisal.
\par 23 Pozdrawia was Gajus, gospodarz mój i wszystkiego zboru. Pozdrawia was Erastus, szafarz miejski, i Kwartus brat.
\par 24 Laska Pana naszego Jezusa Chrystusa niech bedzie z wami wszystkimi. Amen.
\par 25 A temu, który was moze utwierdzic wedlug Ewangielii mojej i opowiadania Jezusa Chrystusa, wedlug objawienia tajemnicy od czasów wiecznych zamilczanej,
\par 26 Lecz teraz objawionej i przez Pisma prorockie wedlug postanowienia wiecznego Boga ku posluszenstwu wiary miedzy wszystkimi narody oznajmionej;
\par 27 Temu, samemu madremu Bogu niech bedzie chwala przez Jezusa Chrystusa na wieki. Amen.


\end{document}