\begin{document}

\title{Zachariasza}


\chapter{1}

\par 1 Miesiaca ósmego roku wtórego Daryjusza stalo sie slowo Panskie do mnie Zacharyjasza, syna Barachyjaszowego, syna Iddowego, proroka, mówiac:
\par 2 Rozgniewal sie Pan na ojców waszych bardzo.
\par 3 Przetoz rzecz do nich: Tak mówi Pan zastepów: Nawróccie sie do mnie, mówi Pan zastepów, a nawróce sie do was, mówi Pan zastepów.
\par 4 Nie badzciez jako ojcowie wasi, na których wolali ojcowie pierwsi, mówiac: Tak mówi Pan zastepów: Nawrócciez sie teraz od zlych dróg waszych, i od zlych spraw waszych; ale nie usluchali, ani dbali na mie, mówi Pan.
\par 5 Gdziez sa ojcowie wasi, i prorocy? Izali na wieki zyc beda?
\par 6 Wszakze izali slowa moje i ustawy moje, którem rozkazal prorokom, slugom moim, nie zasiegly ojców waszych? tak, ze nawróciwszy sie rzekli: Jako nam Pan zastepów uczynic umyslil wedlug dróg naszych i wedlug uczynków naszych, tak nam uczynil.
\par 7 Dnia dwudziestego i czwartego jedenastego miesiaca (ten jest miesiac Sebat) roku wtórego Daryjuszowego stalo sie slowo Panskie do Zacharyjasza, syna Barachyjaszowego, syna Iddowego, proroka, mówiac:
\par 8 Widzialem w nocy, a oto maz jechal na koniu rydzym, który stal miedzy mirtami, które byly w nizinie, a za nim konie rydze, czarne i biale.
\par 9 Tedym rzekl: Co zacz sa ci, panie mój? I rzekl do mnie Aniol rozmawiajacy zemna: Ja tobie okaze, co zacz sa.
\par 10 I odpowiedzial maz, który stal miedzy mirtami, i rzekl: Ci sa, których poslal Pan, aby przeszli ziemie.
\par 11 I odpowiedzieli Aniolowi Panskiemu stojacemu miedzy mirtami, i rzekli: Przeszlismy ziemie, a oto wszystka ziemia bezpieczenstwa i pokoju uzywa.
\par 12 Tedy odpowiedzial Aniol Panski, i rzekl: O Panie zastepów! i dokadze sie nie zmilujesz nad Jeruzalemem, i nad miastami Judzkiemi, na któres sie gniewal juz siedmdziesiat lat?
\par 13 I odpowiedzial Pan Aniolowi onemu, który mówil zemna, slowy dobremi, slowy pociesznemi.
\par 14 I rzekl do mnie Aniol, który mówil zemna: Wolaj a rzecz: Tak mówi Pan zastepów: Zapalilem sie za Jeruzalemem i za Syonem gorliwoscia wielka.
\par 15 A gniewam sie bardzo na te narody, które uzywaja pokoju; bo gdym sie Ja troche zagniewal, tedy one pomagaly do zlego.
\par 16 Przetoz tak mówi Pan: Nawrócilem sie do Jeruzalemu w milosierdziu, dom mój zbudowany bedzie w niem, mówi Pan zastepów, i sznur rozciagniony bedzie na Jeruzalem.
\par 17 Jeszcze wolaj, mówiac: Tak mówi Pan zastepów: Jeszcze sie osadza miasta moje dla obfitosci dobrego; bo jeszcze Pan Syon pocieszy, i obierze jeszcze Jeruzalem.
\par 18 Tedym podniósl oczy swe i ujrzalem, a oto cztery rogi.
\par 19 I rzeklem do Aniola, który mówil zemna: Cóz to jest? I rzekl do mnie: Tec sa rogi, które rozrzucily Jude i Izraela i Jeruzalem.
\par 20 Ukazal mi tez Pan czterech kowali.
\par 21 I rzeklem: Cóz ci ida czynic? I odpowiedzial, mówiac: Tec sa rogi, które rozrzucily Jude, tak, iz zaden nie mógl podniesc glowy swojej; przetoz ci przyszli, aby je przestraszyli, i stracili rogi tych narodów, które podniosly róg przeciwko ziemi Judzkiej, aby ja rozrzucily.

\chapter{2}

\par 1 Potem podnioslem oczy swoje i ujrzalem, a oto maz, w którego rece byl sznur pomiarowy.
\par 2 I rzeklem: Dokad idziesz? I rzekl do mnie: Abym rozmierzyl Jeruzalem, i obaczyl, jako wielka jest szerokosc jego, i jako wielka dlugosc jego.
\par 3 A oto gdy on Aniol, który rozmawial zemna, wychodzil, inszy Aniol wychodzil przeciwko niemu.
\par 4 I rzekl do niego: Biez, rzecz do tego mlodzienca, mówiac: Jeruzalemczycy po wsiach mieszkac beda dla mnóstwa ludu i bydla w posrodku jego.
\par 5 A ja bede, mówi Pan, murem jego ognistym w okolo, i bede slawa w posrodku jego.
\par 6 Nuze, nuze! Ucieczcie juz z ziemi pólnocnej, mówi Pan, poniewaz na cztery strony swiata, mówi Pan, rozproszylem was.
\par 7 Nuze Syonie! który mieszkasz u córki Babilonskiej, wyswobódz sie.
\par 8 Bo tak mówi Pan zastepów: Poslal mie po slawe przeciwko tym narodom, którzy was zlupili; bo kto sie was dotyka, dotyka sie zrenicy oka mego.
\par 9 Albowiem oto Ja podniose reke moje przeciwko nim, i beda lupem slugom swoim, a dowiecie sie, iz mie Pan zastepów poslal.
\par 10 Zaspiewaj a rozraduj sie, córko Syonska! bo oto Ja przyjde, a mieszkac bede w posrodku ciebie, mówi Pan.
\par 11 I przylaczy sie w on dzien wiele narodów do Pana, i beda ludem moim, i bede mieszkal w posród ciebie, a dowiesz sie, iz Pan zastepów poslal mie do ciebie.
\par 12 Tedy Pan Jude wezmie w osiadlosc za dzial swój w ziemi swietej, i obierze zas Jeruzalem.
\par 13 Niech umilknie wszelkie cialo przed obliczem Panskiem; albowiem sie ocuci z mieszkania swietobliwosci swojej.

\chapter{3}

\par 1 Zatem mi okazal Jesuego, kaplana najwyzszego, stojacego przed Aniolem Panskim, i szatana stojacego po prawicy jego, aby mu sie sprzeciwil.
\par 2 Ale Pan rzekl do szatana: Niech cie Pan zgromi, szatanie! niech cie, mówie, zgromi Pan, który obral Jeruzalem. Izali ten nie jest jako glownia wyrwana z ognia?
\par 3 Ale Jesua byl obleczon w szate plugawa, a stal przed Aniolem.
\par 4 A odpowiadajac rzekl do tych, którzy stali przed nim, mówiac: Zdejmijcie te szate plugawa z niego. I rzekl do niego: Otom przeniósl z ciebie nieprawosc twoje, a obleklem cie w szaty odmienne.
\par 5 Znowu rzekl: Niech wloza piekna czapke na glowe jego; i wlozyli piekna czapke na glowe jego, i oblekli go w szaty, a Aniol Panski stal przytem.
\par 6 I oswiadczal sie Aniol Panski przed Jesuem, mówiac:
\par 7 Tak mówi Pan zastepów: Jezli drogami mojemi chodzic bedziesz, a jezli ustaw moich przestrzegac bedziesz, tedy tez ty bedziesz sadzil dom mój, i bedziesz strzegl sieni moich; i dam ci to zapewne, abys chodzil miedzy tymi, którzy tu stoja.
\par 8 Sluchajze tedy teraz, Jesue, kaplanie najwyzszy! ty i towarzysze twoi, którzy siedza przed toba: Aczkolwiek ci mezowie sa dziwem, wszakze oto Ja przywiode sluge mego Latorosl.
\par 9 Albowiem oto ten kamien, który klade przed Jesuego, na ten kamien jeden obrócone beda siedm oczów; oto Ja wyrzeze na nim rzezanie, mówi Pan zastepów, a odejme nieprawosc tej ziemi, dnia jednego.
\par 10 Dnia onego, mówi Pan zastepów, wezwie kazdy blizniego swego pod macice winna i pod figowe drzewo.

\chapter{4}

\par 1 Potem nawrócil sie Aniol, który mówil zemna, i obudzil mie, jako gdy kto budzony bywa ze snu swego;
\par 2 I rzekl do mnie: Cóz widzisz? I rzeklem: Widze, a oto swiecznik wszystek zloty, a czasza na wierzchu jego, i siedm lamp jego na nim, siedm tez nalewek do onych siedmiu lamp, które sa na wierzchu jego;
\par 3 Dwie tez oliwy przytem, jedna po prawej stronie czaszy, a druga po lewej stronie jej.
\par 4 Tedy odpowiadajac rzeklem do Aniola, który mówil zemna, mówiac: Cóz to jest, panie mój?
\par 5 I odpowiedzial Aniol, który mówil zemna, i rzekl mi: Izaz nie wiesz, co to jest? I rzeklem: Nie wiem, panie mój!
\par 6 Tedy odpowiadajac rzekl do mnie, mówiac: Toc jest slowo Panskie do Zorobabela mówiace: Nie wojskiem ani sila stanie sie to, ale duchem moim, mówi Pan zastepów.
\par 7 Cózes ty jest, o góro wielka! przeciwko Zorobabelowi? Równina; bo on wywiedzie kamien glówny z glosnym okrzykiem: Laska, laska nad nim.
\par 8 I stalo sie slowo Panskie do mnie, mówiac:
\par 9 Rece Zorobabelowe zalozyly dom ten, i rece jego dokonaja go; a dowiesz sie, ze Pan zastepów poslal mie do was.
\par 10 Bo któzby wzgardzil dniem malych poczatków? poniewaz sie wesela, patrzac na ten kamien, to jest, na prawidlo w rece Zorobabelowej, na te siedm oczów Panskich przechodzacych wszystke ziemie.
\par 11 Tedy odpowiadajac rzeklem mu: Cóz sa te dwie oliwy po prawej stronie tego swiecznika, i po lewej stronie jego?
\par 12 Znowu odpowiadajac rzeklem mu: Cóz sa te dwie oliwki, które sa miedzy dwoma rurkami zlotemi, które z siebie zloto wylewaja?
\par 13 Tedy rzekl do mnie, mówiac: Izali nie wiesz, co to jest? Rzeklem: Nie wiem, Panie mój.
\par 14 I rzekl: Tec sa one dwie oliwy, które sa u Panujacego na wszystkiej ziemi.

\chapter{5}

\par 1 Potemem sie obrócil, a podnióslszy oczu swych ujrzalem, a oto ksiega leciala.
\par 2 I rzekl do mnie: Cóz widzisz? I rzeklem: Widze ksiege lecaca, której dlugosc na dwadziescia lokci, a szerokosc na dziesiec lokci.
\par 3 I rzekl do mnie: Toc jest przeklestwo, które wyjdzie na oblicze wszystkiej ziemi: bo kazdy zlodziej wedlug tego przeklestwa, jako i ta ziemia, wygladzony, i kazdy falszywie przysiegajacy wedlug niego, jako i ona, wygladzony bedzie.
\par 4 Wywiode je, mówi Pan zastepów, aby przyszlo na dom zlodzieja, i na dom klamliwie przez imie moje przysiegajacego; owszem, mieszkac bedzie w posrodku domu jego, i zniszczy go, i drzewo jego i kamienie jego.
\par 5 Wyszedl tedy Aniol on, który zemna mówil, i rzekl mi: Podniesze teraz oczu swych, a obacz, co to jest, co wychodzi.
\par 6 I rzeklem: Cóz jest? A on odpowiedzial: To jest efa wychodzace. Potem rzekl: Toc jest oko ich przypatrujace sie wszystkiej ziemi.
\par 7 A oto sztuke olowiu niesiono, a przytem byla niewiasta jedna, która siedziala w posrodku efa.
\par 8 Tedy rzekl Aniol: Toc jest ona niezboznosc; i wrzucil ja w posród efa, wrzucil i one sztuke olowiu na wierzch efy.
\par 9 A podnióslszy oczu swych ujrzalem, a oto dwie niewiasty wychodzily, majace wiatr w skrzydlach swych, a mialy skrzydla, jako skrzydla bocianie, i podniosly ono efa miedzy ziemie i miedzy niebo.
\par 10 Tedym rzekl do onego Aniola, który mówil zemna: Dokadze niosa to efa?
\par 11 I rzekl do mnie: Aby mu zbudowano dom w ziemi Senaar, gdzieby umocnione bylo i postawione na podstawku swoim.

\chapter{6}

\par 1 Potem obróciwszy sie podnioslem oczu swych i ujrzalem, a oto cztery wozy wychodzily z posrodku dwóch gór, a góry one byly góry miedziane.
\par 2 W pierwszym wozie byly konie rydze, a w drugim wozie konie wrone (kare):
\par 3 W trzecim wozie konie biale, a w czwartym wozie byly konie strokate, a wszystkie mocne.
\par 4 Tedy odpowiadajac rzeklem do Aniola, który mówil zemna: Co to jest, Panie mój?
\par 5 I odpowiedzial Aniol a rzekl do mnie: Te sa cztery wiatry niebieskie, wychodzace z miejsca, gdzie staly, przed panujacym nad wszystka ziemia.
\par 6 Konie wrone zaprzezone wychodza do ziemi pólnocnej, a biale wychodza za nimi, strokate zas wychodza do ziemi poludniowej.
\par 7 Te tedy mocne konie wyszedlszy chcialy isc, aby obeszly ziemie; tedy rzekl: Idzcie, a obejdzcie ziemie! I obeszly ziemie.
\par 8 A zawolawszy mie rzekl do mnie, mówiac: Oto te, które wyszly do ziemi pólnocnej, uspokoily ducha mego w ziemi pólnocnej.
\par 9 I stalo sie slowo Panskie do mnie, mówiac:
\par 10 Wezmij od tych, co byli pojmani od Cheldajego i od Tobijasza, i od Jedajasza; (a ty przyjdziesz tegoz dnia, i wnijdziesz do domu Josyjasza, syna Sofonijaszowego) którzy ida z Babilonu;
\par 11 Wezmij, mówie, srebro i zloto, a uczyn korony, a wlóz je na glowe Jesuego, syna Jozedekowego, kaplana najwyzszego.
\par 12 I rzecz do niego, mówiac: Tak powiada Pan zastepów, mówiac: Oto maz, którego imie jest Latorosl, który z miejsca swego wyrosnie, ten wystawi kosciól Panu.
\par 13 Bo ten ma wystawic kosciól Panu, ten zas przyniesie slawe, i siedziec i panowac bedzie na stolicy swojej, i bedzie kaplanem na stolicy swojej, a rada pokoju bedzie miedzy nimi obiema.
\par 14 A te korony zostana Chelemowi, i Tobijaszowi, i Jedajaszowi, i Chenowi, synowi Sofonijaszowemu, na pamiatke w kosciele Panskim.
\par 15 Bo dalecy przyjda, a beda budowac kosciól Panski; i dowiecie sie, ze Pan zastepów poslal mie do was; a to sie stanie, jezli pilnie sluchac bedziecie glosu Pana, Boga swego.

\chapter{7}

\par 1 Potem stalo sie roku czwartego Daryjusza króla, stalo sie slowo Panskie do Zacharyjasza dnia czwartego, miesiaca dziewiatego, który jest Kislew;
\par 2 Gdy poslal lud do domu Bozego Sarassara i Regiemmelecha, i mezów jego, aby sie modlili przed obliczem Panskiem;
\par 3 I aby mówili do kaplanów, którzy byli w domu Pana zastepów, takze i do proroków, mówiac: Izali jeszcze plakac bede miesiaca piatego, wylaczywszy sie tak, jakom juz czynil przez kilka lat?
\par 4 I stalo sie slowo Pana zastepów do mnie, mówiac:
\par 5 Rzecz do wszystkiego ludu tej ziemi, i do kaplanów, mówiac: Gdyscie poscili i plakali piatego i siódmego miesiaca przez te siedmdziesiat lat, izazescie mnie, mnie, mówie, post poscili?
\par 6 A gdy jecie albo pijecie, izali nie sobie jecie i nie sobie pijecie?
\par 7 Izaliscie nie tak czynic mieli wedlug slowa, które przepowiedzial Pan przez proroków przeszlych, gdy jeszcze Jeruzalem bezpieczenstwa i pokoju uzywalo, i miasta jego okolo niego, i lud w stronie poludniowej i po polach mieszkal (w pokoju?)
\par 8 I stalo sie slowo Panskie do Zacharyjasza, mówiac:
\par 9 Tak powiedzial Pan zastepów, mówiac: Sprawiedliwie sadzcie, a milosierdzie i litosc pokazujcie kazdy nad bliznim swoim;
\par 10 A wdowy i sieroty, i przychodnia, i ubogiego nie uciskajcie, i zlego jeden przeciwko drugiemu nie myslcie w sercu swojem.
\par 11 Ale nie chcieli dbac; i obrócili sie tylem, a uszy swe zatulili, aby nie sluchali.
\par 12 Serca tez swe zatwardzili jako dyjament, aby nie sluchali zakonu tego i slów, które posylal Pan zastepów duchem swoim przez proroków przeszlych, skad przyszedl wielki gniew od Pana zastepów.
\par 13 Bo jako oni, gdy ich wolano, nie sluchali, tak tez, gdy oni wolali, nie wysluchalem, mówi Pan zastepów.
\par 14 I rozproszylem ich jako wicher miedzy wszystkie narody, które nie znali, i ta ziemia spustoszala po nich, tak, ze nie byl przechodzacy i wracajacy sie, a tak ziemie pozadana w spustoszenie obrócili.

\chapter{8}

\par 1 Potem stalo sie slowo Pana zastepów, mówiac:
\par 2 Tak mówi Pan zastepów: Zapalilem sie nad Syonem gorliwoscia wielka, owszem, rozgniewaniem wielkiem zapalilem sie.
\par 3 Tak mówi Pan: Nawrócilem sie do Syonu, i mieszkam w posród Jeruzalemu, aby Jeruzalem zwano miastem wiernem, a góre Pana zastepów, góra swietobliwosci.
\par 4 Tak mówi Pan zastepów: Jeszcze siadac beda starcy i baby na ulicach Jeruzalemskich, majac kazdy z nich laske w rece swej dla zeszlosci wieku.
\par 5 Ulice takze miasta pelne beda chlopiat i dziewczat grajacych na ulicach jego.
\par 6 Tak mówi Pan zastepów: Izali, ze sie to niepodobna widzi przed oczyma ostatków ludu tego tych dni, bedzie tez to niepodobna przed oczyma mojemi? mówi Pan zastepów.
\par 7 Tak mówi Pan zastepów: Oto Ja wybawie lud mój z ziemi na wschód, i z ziemi na zachód slonca.
\par 8 I przywiode ich zas; a beda mieszkac w posród Jeruzalemu, i beda ludem moim, a Ja bede Bogiem ich w prawdzie i w sprawiedliwosci.
\par 9 Tak mówi Pan zastepów: Niech sie zmocnia rece wasze, którzyscie sluchali w tych dniach slów tych z ust proroków, którzy byli ode dnia, którego zalozony jest dom Pana zastepów, ze kosciól ma byc dobudowany.
\par 10 Bo sie przed temi dniami praca ludzka i praca bydlat nie nagradzala, nawet wychodzacemu i wchodzacemu nie bylo pokoju dla nieprzyjaciela; bom Ja spuscil wszystkich ludzi jednego z drugim.
\par 11 Lecz teraz nie tak jako za dni przeszlych czynie ostatkowi ludu tego, mówi Pan zastepów:
\par 12 Ale siewy macie spokojne; winna macica wydaje owoc swój, i ziemia wydaje urodzaj swój, niebiosa takze wydawaja rose twoje, a to wszystko daje w osiadlosc ostatkom ludu tego.
\par 13 I stanie sie, ze jakoscie byli przeklestwem miedzy poganami, o domie Judzki i domie Izraelski! tak was zas bede ochranial, i bedziecie blogoslawienstwem; nie bójcie sie, niech sie zmacniaja rece wasze.
\par 14 Bo tak mówi Pan zastepów: Jakom wam byl umyslil zle uczynic, gdy mie do gniewu pobudzali ojcowie wasi, mówi Pan zastepów, a nie zalowalem tego,
\par 15 Tak nawróciwszy sie umyslilem w te dni dobrze czynic Jeruzalemowi i domowi Judzkiemu; nie bójciez sie.
\par 16 Tec sa rzeczy, które czynic bedziecie: Prawde mówcie kazdy z bliznim swoim, prawy i spokojny sad czyncie w bramach waszych;
\par 17 A jeden drugiemu nic zlego nie myslcie w sercach waszych, a w krzywoprzysiestwie sie nie kochajcie; bo to wszystko jest, czego nienawidze, mówi Pan.
\par 18 I stalo sie slowo Pana zastepów do mnie, mówiac:
\par 19 Tak mówi Pan zastepów: Post czwartego, i post piatego, i post siódmego, i post dziesiatego miesiaca obróci sie domowi Judzkiemu w radosc i wesele, i w rozkoszne uroczyste swieta; ale prawde i pokój milujcie.
\par 20 Tak mówi Pan zastepów: Jeszczec beda przychodzic narody i obywatele wielu miast;
\par 21 Przychodzic, mówie, beda obywatele jednego miasta do drugiego, mówiac: Pójdzmy ochotnie blagac oblicze Panskie, a szukac Pana zastepów; i rzecze kazdy: Pójde i ja.
\par 22 A tak wiele ludu i narodów niezliczonych przyjdzie szukac Pana zastepów w Jeruzalemie, i blagac oblicze Panskie.
\par 23 Tak mówi Pan zastepów: W one dni uchwyca sie dziesiec mezów ze wszystkich jezyków onych narodów; uchwyca sie, mówie, podolka jednego Zyda, mówiac: Pójdziemy z wami, bo slyszymy, ze Bóg jest z wami.

\chapter{9}

\par 1 Brzemie slowa Panskiego przeciwko ziemi, która jest w okolo ciebie, a Damaszek bedzie odpocznieniem jego; albowiem oko Panskie przypatruje sie ludziom i wszystkim pokoleniom Izraelskim;
\par 2 Nawet i do Emat dosieze, i do Tyru i do Sydonu, choc jest madry bardzo.
\par 3 Bo sobie Tyr obrone zbudowal, i nazbieral srebra jako prochu, a zlota jako blota po ulicach.
\par 4 Oto Pan go wypedzi, a wrazi w morze sile jego, a sam od ognia pozarty bedzie.
\par 5 Co widzac Aszkalon, uleknie sie, takze Gaza wielce zalosne bedzie, i Akaron, przeto, ze je zawstydzila nadzieja ich; i zginie król z Gazy, a Aszkalon nie bedzie osadzone;
\par 6 I bedzie mieszkal bekart w Azocie, a tak wykorzenie pyche Filistynczyków.
\par 7 I odejme krew kazdego od ust jego, i obrzydliwosci jego od zebów jego; zostawiony tez bedzie i on Bogu naszemu, aby byl jako ksiaze w Judzie, a Akaron jako Jebuzejczycy.
\par 8 I poloze sie obozem u domu swego dla wojska, i dla przechodzacego a wracajacego sie; i nie przejdzie wiecej przez nich lupiezca, przeto, ze sie tak teraz podoba w oczach moich.
\par 9 Wesel sie bardzo, córko Syonska! wykrzykaj, córko Jeruzalemska! Oto król twój przyjdzie tobie sprawiedliwy i zbawiciel ubogi i siedzacy na osle, to jest, na osleciu, zrebiatku oslicy.
\par 10 Bo wytrace wozy z Efraima, i konie z Jeruzalemu, i bedzie polamany luk wojenny; i oglosi pokój narodom, a wladza jego (bedzie) od morza az do morza, i od rzeki az do konczyn ziemi.
\par 11 Owszem, ty wesel sie dla krwi przymierza swego; albowiem wypuscilem wiezniów twoich z dolu, w którym niemasz wody.
\par 12 Wrócciez sie tedy do twierdzy, o wiezniowie, którzy nadzieje macie! albowiemci i dzis dwojako opowiadam i nagrodze.
\par 13 Gdyz sobie naciagne Jude, a luk napelnie Efraimem; i wzbudze synów twoich, o Syonie! przeciwko synom twoim, o Jawanie! i zgotuje cie jako miecz mocarza.
\par 14 Bo sie Pan ukaze przeciwko nim, a jako blask wyniknie strzala jego; panujacy, mówie, Pan zatrabi w trabe, a pójdzie w wichrach poludniowych.
\par 15 Pan zastepów ochraniac bedzie lud swój, aby podbiwszy sobie kamienie z procy, jedli i pili wykrzykajac jako od wina; i napelnia, jako miednice, tak i rogi oltarza.
\par 16 A tak wybawi ich dnia onego Pan, Bóg ich, jako trzode ludu swego; bo kamienie wiencami ozdobione, wystawione beda miasto choragwi w ziemi jego.
\par 17 Albowiem oto o jakie blogoslawienstwo jego! i jako wielka ozdoba jego! Zboze mlodzienców, a moszcz panny mowne uczyni.

\chapter{10}

\par 1 Zadajcie od Pana dzdzu czasu potrzebnego, a Pan uczyni obloki dzdzyste, a deszcz obfity da wam i kazdemu trawe na polu.
\par 2 Bo obrazy mówia próznosc, a wieszczkowie prorokuja klamstwo i sny prózne opowiadaja, daremnie ciesza; dlatego poszli w niewole, jako trzoda, utrapieni sa, ze nie mieli pasterza.
\par 3 Przeciwko takim pasterzom zapalila sie popedliwosc moja, a te kozly nawiedze; ale trzode swoje, dom Judzki, nawiedzi Pan zastepów, i wystawi ich jako ubranego konia do boju.
\par 4 Od niego wegiel, od niego gwózdz, od niego luk wojenny, od niego takze wynijdzie wszelki poborca;
\par 5 I beda jako mocarze depczacy w bloto po ulicach w bitwie, i walczyc beda, bo Pan z nimi; a zawstydza tych, którzy wsiadaja na kon.
\par 6 I umocnie dom Judowy, a dom Józefowy wybawie, i w pokoju ich osadze, bo mam litosc nad nimi; i beda, jakobym ich nie odrzucil; bom Ja jest Pan, Bóg ich, a wyslucham ich.
\par 7 I beda Efraimczycy jako mocarz, a rozweseli sie jako od wina serce ich; a synowie ich widzac to weselic sie beda, i rozraduje sie serce ich w Panu.
\par 8 Zaswisne na nich, a zgromadze ich, bo ich odkupie; i beda rozmnozeni, jako przedtem rozmnozeni byli.
\par 9 I rozsieje ich miedzy narody, aby na miejscach dalekich wspomnieli na mie, a zywi bedac z synami swoimi nawrócili sie.
\par 10 A tak ich przywiode z ziemi Egipskiej, i z Assyryi zgromadze ich, a do ziemi Galaad i do Libanu przywiode ich; ale im miejsca stawac nie bedzie.
\par 11 Przetoz dla ciasnosci przez morze przejdzie, i rozbije na morzu waly, i wyschna wszystkie glebokosci rzeki; tedy bedzie znizona pycha Assyryi, a sceptr od Egiptu odjety bedzie.
\par 12 Zmocnie ich tez w Panu, a w imieniu jego chodzic beda, mówi Pan.

\chapter{11}

\par 1 Otwórz, Libanie! wrota swe, niech pozre ogien cedry twoje.
\par 2 Rozkwil sie jodlo! bo upadl cedr, bo wielmozni spustoszeni sa; kwilcie deby Basanskie, bo wyciety jest las ogrodzony.
\par 3 Glos narzekania pasterzy slyszany jest, iz zburzona jest wielmoznosc ich; glos ryku lwiat, iz zburzona jest pycha Jordanu.
\par 4 Tak mówi Pan, Bóg mój: Pas owce na rzez zgotowane;
\par 5 Które dzierzawcy ich zabijaja, a nie bywaja obwinieni, i owszem, sprzedawajacy je mówia: Blogoslawiony Pan, zesmy sie zbogacili, a którzy je pasa, nie maja litosci nad niemi.
\par 6 Przetoz nie sfolguje wiecej obywatelom tej ziemi, mówi Pan; bo oto Ja podam tych ludzi kazdego w reke blizniego jego, i w reke króla ich, i potra ziemie, a nie wyrwie jej z rak ich.
\par 7 Bom pasl owce zgotowane na rzez, was, mówie, o nedzne owce! i wziawszy sobie dwie laski, jednem nazwal Uciecha, a drugam nazwal Zwiazujacych, a paslem one owce.
\par 8 I zgladzilem trzech pasterzy w jednym miesiacu; ale utesknila sobie dusza moja z nimi, przeto, ze dusza ich brzydzila sie mna.
\par 9 Rzeklem tedy: Nie bedec was pasl; co zdycha, niech zdechnie, a co ma byc wygladzone, niech bedzie wygladzone, a które pozostana, niech pozera mieso jedna drugiej.
\par 10 Przetoz wziawszy laske moje Uciechy, porabalem ja, wzruszywszy przymierze moje, którem postanowil z tym wszystkim ludem.
\par 11 A dnia onego, gdy wzruszone bylo, pewnie poznali nedzni z trzody, którzy sie na mie ogladali, ze to slowo Panskie.
\par 12 Bom rzekl do nich: Jezli to jest dobre w oczach waszych, dajcie zaplate moje, a jezli nie, zaniechajciez; tedy odwazyli zaplate moje trzydziesci srebrników.
\par 13 Zatem rzekl Pan do mnie: Porzuc je przed garncarza; zacnaz to zaplata, któram jest od nich tak drogo oszacowany! Wzialem tedy trzydziesci srebrników, a porzucilem je w domu Panskim przed garncarza.
\par 14 Potem porabalem laske moje druga Zwiazujacych, wzruszywszy braterstwo miedzy Juda i miedzy Izraelem.
\par 15 I rzekl Pan do mnie: Wezmij sobie jeszcze orez pasterza glupiego.
\par 16 Bo oto Ja wzbudze pasterza w tej ziemi, który nie bedzie oblakanych nawiedzal, ani bedzie jagniatek szukal, zlamanego tez leczyc, i tego, co ustanie, nosic nie bedzie; ale mieso tlustych jesc bedzie, a kopyta ich postraca.
\par 17 Biada pasterzowi niepozytecznemu, który opuszcza trzode! miecz nad ramieniem jego i nad prawem okiem jego; ramie jego cale uschnie, a prawe oko jego cale zacmione bedzie.

\chapter{12}

\par 1 Brzemie slowa Panskiego nad Izraelem. Tak mówi Pan, który rozpostarl niebiosa, a ugruntowal ziemie, który tworzy ducha czlowieczego we wnetrznosciach jego:
\par 2 Oto Ja postawie Jeruzalem kubkiem opojenia wszystkim narodom okolicznym, którzy beda przeciwko Judzie na oblezenie, i przeciwko Jeruzalemowi.
\par 3 Owszem, stanie sie dnia onego, ze uczynie Jeruzalem kamieniem ciezkim wszystkim narodom; wszyscy, którzy go dzwigac beda, bardzo sie uraza, chocby sie zgromadzily przeciwko niemu wszystkie narody ziemi.
\par 4 Dnia onego, mówi Pan, zaraze kazdego konia zdretwieniem i jezdzca jego szalenstwem; ale nad domem Juda otworze oczy moje, a kazdego konia narodów zaraze slepota.
\par 5 I rzekna ksiazeta Judzcy w sercu swem: Mamy sile i obywatele Jeruzalemscy w Panu zastepów, Bogu swoim.
\par 6 Dnia onego poloze ksiazat Judzkich jako wegle ogniste miedzy drwy, a jako pochodnie gorejaca miedzy snopy; i pozra na prawa i na lewa strone wszystkie narody okoliczne, i zostanie jeszcze Jeruzalem na miejscu swem w Jeruzalemie.
\par 7 Zachowa Pan i namioty Judzkie pierwej, aby sie nie wywyzszala chwala domu Dawidowego, i chwala obywateli Jeruzalemskich przeciwko Judzie.
\par 8 Dnia onego Pan bedzie bronil obywateli Jeruzalemskich, a któryby byl miedzy nimi najslabszy, stanie sie dnia onego podobny Dawidowi, a dom Dawidowy podobny bogom, podobny Aniolowi Panskiemu przed nimi.
\par 9 Bo sie stanie dnia onego, ze szukac bede wszystkie narody, które przyciagna przeciwko Jeruzalemowi, abym je wytracil.
\par 10 I wyleje na dom Dawidowy, i na obywateli Jeruzalemskich Ducha laski i modlitw, a patrzyc beda na mie, którego przebodli; i plakac beda nad nim placzem, jako nad jednorodzonym; gorzko, mówie, plakac beda nad nim, jako gorzko placza nad pierworodnym.
\par 11 Dnia onego bedzie wielkie kwilenie w Jeruzalemie, jako kwilenie w Adadrymon na polu Magieddon;
\par 12 Bo ziemia kwilic bedzie, kazde pokolenie osobno, pokolenie domu Dawidowego osobno, i niewiasty ich osobno; pokolenie domu Natanowego osobno, i niewiasty ich osobno;
\par 13 Pokolenie domu Lewiego osobno, i niewiasty ich osobno; pokolenie Semejego osobno, i niewisty ich osobno;
\par 14 Wszystkie insze pokolenia, kazde pokolenie osobno, i niewiasty ich osobno.

\chapter{13}

\par 1 W on dzien bedzie otworzona studnica domowi Dawidowemu i obywatelom Jeruzalemskim na omycie grzechu i nieczystosci.
\par 2 I stanie sie dnia onego, mówi Pan zastepów, ze wykorzenie imiona balwanów z ziemi, tak, ze nie beda wiecej wspominane; dotego i tych proroków i ducha nieczystego zniose z ziemi.
\par 3 I stanie sie, gdyby kto dalej prorokowal, ze mu rzekna ojciec jego i matka jego, którzy go splodzili: Nie bedziesz zyl, przeto zes klamstwo mówil w imieniu Panskiem; i przebija go ojciec jego i matka jego, którzy go splodzili, ze prorokowal.
\par 4 I stanie sie dnia onego, ze sie zawstydza oni prorocy, kazdy za widzenie swoje, gdyby prorokowali, i nie obleka sie w suknie kosmata, aby klamali;
\par 5 Ale kazdy rzecze: Nie jestem ja prorokiem, ale rolnikiem; bo mie tego nauczono od dziecinstwa mego.
\par 6 A jezeli mu kto rzecze: Cóz to masz za rany na rekach twoich? Tedy rzecze: Temi jestem zraniony w domu tych, którzy mie miluja.
\par 7 O mieczu! ocknij sie na pasterza mego, i na meza towarzysza mego, mówi Pan zastepów; uderz pasterza, a owce rozproszone beda; ale zas obróce reke moje ku maluczkim.
\par 8 Bo sie stanie po tej wszystkiej ziemi, mówi Pan, ze dwie czesci w niej wytracone beda i pomra, a trzecia zostanie w niej.
\par 9 I wwiode i one trzecia czesc do ognia, a wyplawie je jako plawia srebro, a doswiadczac ich bede, jako doswiadczaja zlota; kazdy bedzie wzywal imienia mego, a Ja go wyslucham: rzeke: Tys lud mój, a on rzecze: Tys Pan Bóg mój.

\chapter{14}

\par 1 Oto przychodzi dzien Panski, a rozdzielone beda korzysci twoje w posrodku ciebie.
\par 2 Bo zgromadze wszystkie narody przeciwko Jeruzalemowi na wojne, a miasto wziete bedzie, i domy rozchwycone beda, i niewiasty pogwalcone beda; a gdy pójdzie czesc miasta w pojmanie, ostatek ludu nie bedzie wygladzony z miasta.
\par 3 Bo wyjdzie Pan, i bedzie walczyl przeciwko onym narodom, jako zwykl wojowac w dzien potykania.
\par 4 I stana nogi jego w on dzien na górze Oliwnej, która jest przeciwko Jeruzalemowi na wschód slonca, a góra Oliwna sie na poly rozszczepi, na wschód i na zachód slonca rozpadlina bardzo wielka, i odwali sie polowa onej góry na pólnoc, a polowa jej na poludnie.
\par 5 Tedy ucieczecie przed dolina gór; (bo dolina tych gór dosieze az do Azal) bedziecie, mówie, uciekac, jakoscie uciekali przed trzesieniem ziemi za dni Ozyjasza, króla Judzkiego, gdy przyjdzie Pan, Bóg mój, i wszyscy swieci z nim.
\par 6 I stanie sie dnia onego, ze nie bedzie swiatlosci drogiej, ani ciemnosci gestej;
\par 7 Lecz bedzie dzien jeden, który jest wiadomy Panu, a nie bedzie dnia ani nocy; wszakze czasu wieczornego bedzie swiatlo.
\par 8 A dnia onego wyjda wody zywe z Jeruzalemu; polowa ich do morza na wschód slonca, a polowa ich do morza ostatniego, a to bedzie w lecie i w zimie.
\par 9 A Pan bedzie królem nad wszystka ziemia; w on dzien bedzie Pan jeden, i imie jego jedno.
\par 10 I uczyniona bedzie ta wszystka ziemia, jako równina od Gabaa az do Remmon na poludnie ku Jeruzalemowi, który wywyzszony bedac, stac bedzie na miejscu swojem od bramy Benjaminowej az do miejsca bramy pierwszej i az do bramy wegielnej, a od wiezy Chananeel az do pras królewskich.
\par 11 I beda w nim mieszkac, a nie bedzie wiecej przeklestwem, a Jeruzalem bezpiecznie mieszkac bedzie.
\par 12 A tac bedzie plaga, która uderzy Pan wszystkie narody, któreby walczyly przeciwko Jeruzalemowi: Cialo kazdego stojacego na nogach swoich schnac bedzie, a oczy ich wyplyna z dolków swoich, i jezyk ich uschnie w ustach ich.
\par 13 I stanie sie dnia onego wielkie ucisnienie Panskie miedzy nimi, tak, iz reke jeden drugiego uchwyci, a reka jego podniesie sie na reke blizniego swego.
\par 14 I tyc tez, Judo! walczyc bedziesz w Jeruzalemie, a zgromadzone beda bogactwa wszystkich narodów okolicznych, zloto i srebro i szat obfitosc wielka.
\par 15 A takaz bedzie plaga na konie, muly, wielblady, i osly, i na wszystkie bydleta, które beda w onym obozie, jako i ta plaga.
\par 16 A ile ich pozostanie z onych wszystkich narodów, któreby przyciagnely przeciwko Jeruzalemowi, beda przychodzic od roku do roku, poklon oddawac królowi, Panu zastepów, i obchodzic swieto Kuczek;
\par 17 A ktoby nie szedl z pokolenia ziemi do Jeruzalemu, poklon oddawac królowi, Panu zastepów, na tych deszcz padac nie bedzie.
\par 18 A jesli pokolenie Egipskie nie wstapi, i nie przyjdzie, choc na nich deszcz nie pada, przyjdzie jednak ta plaga, która uderzy Pan narody, które nie przyszly, obchodzic swieta Kuczek.
\par 19 A tac bedzie kazn grzechu Egipskiego, i kazn grzechu wszystkich narodów, któreby nie przychodzily ku obchodzeniu swieta Kuczek.
\par 20 Dnia onego bedzie na rzedach konskich napisane: Swietobliwosc Panska; a kotlów bedzie w domu Panskim, jako miednic przed oltarzem.
\par 21 Owszem, kazdy kociel w Jeruzalemie i w Judzie poswiecony bedzie Panu zastepów; a przychodzac wszyscy, którzy ofiarowac maja, brac je i warzyc w nich beda, a nie bedzie Chananejczyka wiecej w domu Pana zastepów dnia onego.


\end{document}