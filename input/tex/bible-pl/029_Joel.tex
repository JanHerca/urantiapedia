\begin{document}

\title{Joela}


\chapter{1}

\par 1 Slowo Panskie, które sie stalo do Joela syna Patuelowego.
\par 2 Sluchajcie tego starcy, a bierzcie w uszy wszyscy obywatele tej ziemi! Izali sie to stalo za dni waszych, albo za dni ojców waszych?
\par 3 Powiadajcie o tem synom waszym, a synowie wasi synom swoim, a synowie ich rodzajowi potomnemu.
\par 4 Co zostalo po gasienicach, pojadla szarancza, a co zostalo po szaranczy, pojadl chrzaszcz, a co zostalo po chrzaszczu, pojadl czerw.
\par 5 Ocuccie sie pijani a placzcie, i narzekajcie wszyscy, którzy pijecie wino, dla moszczu; bo wydarty jest od ust waszych.
\par 6 Albowiem naród przyciagnal do ziemi mojej mocny a niezliczony; zeby jego zeby lwie, a trzonowe zeby jako lwa srogiego.
\par 7 Winna macice moje podal na spustoszenie, a figowe drzewo moje na oblupienie; w szczat je obnazyl i porzucil, tak, ze zbielaly galezie ich.
\par 8 Narzekaj, jako panna przepasana worem nad mezem mlodosci swojej.
\par 9 Odjeta jest sniedna i mokra ofiara od domu Panskiego; placza kaplani, sludzy Panscy.
\par 10 Spustoszone jest pole, i smuci sie ziemia, przeto, ze popsowano zboze; wysechl moszcz, oliwa zginela.
\par 11 Wstydza sie oracze, narzekaja winiarze dla pszenicy i dla jeczmienia; bo zginelo zniwo polne.
\par 12 Winna macica uschla, a figowe drzewo uwiedlo; drzewo granatowe i palma, i jablon, i wszystkie drzewa polne poschly, i wesele zginelo od synów ludzkich.
\par 13 Przepaszcie sie, a placzcie, o kaplani! narzekajcie sludzy oltarza; wnijdzcie a legajcie w nocy w worach, sludzy Boga mojego! bo zawsciagniona jest od domu Boga waszego ofiara sniedna i ofiara mokra.
\par 14 Poswieccie post, zwolajcie zgromadzenia, zbierzcie starców i wszystkich obywateli ziemi do domu Pana, Boga waszego, i wolajcie do Pana:
\par 15 Ach biada na ten dzien! bo bliski jest dzien Panski, a przychodzi jako spustoszenie od Wszechmocnego.
\par 16 Izali przed oczyma naszemi nie zginela zywnosc, a z domu Boga naszego radosc i wesele?
\par 17 Pognily ziarna pod skibami swemi, spustoszone sa gumna, zburzone sa szpichlerze; bo wyschlo zboze.
\par 18 Czemu wzdycha bydlo? Blakaja sie stada wolów, ze nie maja pastwisk, nawet i trzody owiec wyginely.
\par 19 Do ciebie wolam, o Panie! bo ogien pozarl pastwiska na puszczy a plomien popalil wszystkie drzewa polne;
\par 20 Takze i zwierzeta polne rycza do ciebie, przeto, ze wyschly strumienie wód, a ogien pozarl pastwiska na puszczy.

\chapter{2}

\par 1 Trabcie w trabe na Syonie, a krzyczcie na swietej górze mojej! niechaj zadrza wszyscy obywatele ziemi; bo przychodzi dzien Panski, bo juz bliski jest;
\par 2 Dzien ciemnosci i mroku, dzien obloku i chmury, jako ranna zorza rozciagniona po górach; lud wielki a mocny, któremu równego nie bylo od wieku, i nie bedzie po nim nigdy az do lat rodzaju i rodzaju.
\par 3 Przed obliczem jako ogien pozerajacy, a za nim plomien palajacy; ta ziemia jest przed nim jako ogród Eden, ale po nim bedzie pustynia pusta i nie ujdzie nikt przed nim.
\par 4 Ksztalt ich jest jako ksztalt koni, a tak pobieza jako jezdni.
\par 5 Po wierzchu gór skakac beda jako grzmot wozów, a jako szum plomienia ognistego pozerajacego sciernisko, jako lud mozny uszykowany do bitwy.
\par 6 Ulekna sie narody przed obliczem jego, wszystkie twarze ich jako garniec poczernieja.
\par 7 Pobieza jako mocarze, a wbieza na mury jako mezowie waleczni; kazdy z nich droga swoja pójdzie, a nie ustapia z sciezek swoich.
\par 8 Jeden drugiego nie scisnie, kazdy droga swoja pójdzie; a choc i na miecz upadna, nie beda zranieni.
\par 9 Po miescie chodzic beda, po murze biegac, na domy wstapia, a oknami wleza jako zlodziej.
\par 10 Przed obliczem jego ziemia zadrzy, niebiosa sie porusza, slonce i miesiac sie zacmi, a gwiazdy zawsciagna jasnosc swoje.
\par 11 A Pan wyda glos swój przed wojskiem swojem, przeto, ze bardzo wielki bedzie obóz jego, przeto, ze mocny ten, co wykona slowo jego; wielki bowiem dzien Panski bedzie i straszliwy bardzo, i któz go zniesie?
\par 12 A przetoz jeszcze i teraz mówi Pan: Nawróccie sie do mnie samego calem sercem swojem, i w poscie i w placzu i w kwileniu.
\par 13 Rozedrzyjcie serce wasze a nie szaty wasze, i nawróccie sie do Pana, Boga waszego; boc on jest laskawy i milosierny, nierychly ku gniewu, a hojny w milosierdziu, i zalujacy zlego.
\par 14 Któz wie, nie obrócili sie, a nie bedzieli mu zal, i nie zostawili po sobie blogoslawienstwa na sniedna i mokra ofiare Panu, Bogu waszemu.
\par 15 Trabcie w trabe na Syonie, poswieccie post, zwolajcie zgromadzenie.
\par 16 Zgromadzcie lud, poswieccie zgromadzenie, zbierzcie starców, zniescie maluczkie i ssace piersi; niech wynijdzie oblubieniec z loznicy swojej, a oblubienica z pokoju swego.
\par 17 Kaplani, sludzy Panscy, miedzy przysionkiem a oltarzem niech placza i mówia: Przepusc, Panie! ludowi twemu, a nie daj dziedzictwa swego na pohanbienie, aby nad nimi poganie panowac mieli. Przeczzeby mówiono miedzy narodami: Gdziez jest Bóg ich?
\par 18 I zapali sie Pan miloscia ku ziemi swojej, a zmiluje sie nad ludem swoim.
\par 19 I ozwie sie Pan, a rzecze do ludu swego: Oto Ja posle wam zboze, i moszcz, i oliwe, a bedziecie niemi nasyceni, i nie podam was wiecej na pohanbienie miedzy pogan.
\par 20 Bo pólnocne wojsko oddale od was, a zapedze je do ziemi suchej i spustoszonej; przedni huf jego obróci sie ku morzu wschodniemu, a koniec jego ku morzu ostatecznemu, i wynijdzie z niego smród i zgnilosc, choc sobie hardzie poczyna.
\par 21 Nie bój sie, ziemio! wesel sie a raduj sie; bo Pan wielkie rzeczy uczyni.
\par 22 Nie bójcie sie zwierzeta pól moich; boc wzroslo pastwisko na pustyni, a drzewa przyniosa owoce swoje, figowe drzewo i macica winna wydadza moc swoje.
\par 23 I wy, synowie Syonscy! weselcie sie i radujcie sie w Panu, Bogu waszym; bo wam da deszcz wczesny, a zesle wam deszcz obfity w jesieni i na wiosne.
\par 24 I beda gumna zbozem napelnione, a prasy oplywac beda moszczem i oliwa.
\par 25 A tak nagrodze wam lata, które zjadla szarancza, czerw, chrzaszcze i gasienice, wojsko moje wielkie, którem posylal na was.
\par 26 Tedy jedzac jesc bedziecie, a nasyceni bedac chwalic bedziecie imie Pana, Boga swego, który uczynil z wami dziwne rzeczy, i nie bedzie pohanbiony lud mój na wieki.
\par 27 I dowiecie sie, zem Ja jest w posród Izraela, a zem Ja Panem, Bogiem waszym, a ze niemasz inszego; boc nie bedzie pohanbiony lud mój na wieki.
\par 28 A potem wyleje Ducha mego na wszelkie cialo, a prorokowac beda synowie wasi i córki wasze; starcom waszym sny sie snic beda, a mlodziency wasi widzenia widziec beda.
\par 29 Nawet i na slugi i na sluzebnice wyleje w one dni Ducha mego.
\par 30 I dam cuda na niebie i na ziemi, krew i ogien i slupy dymowe.
\par 31 Slonce obróci sie w ciemnosc, a miesiac w krew, pierwej niz dzien Panski wielki a straszny przyjdzie.
\par 32 Wszakze stanie sie, ze ktobykolwiek wzywal imienia Panskiego, wybawiony bedzie; bo na górze Syon i w Jeruzalemie bedzie wybawienie, jako rzekl Pan, to jest w ostatkach, które Pan powola.

\chapter{3}

\par 1 Bo oto w one dni i w on czas, gdy nawróce pojmany lud Judzki i Jeruzalemski,
\par 2 Zgromadze tez wszystkie narody, i sprowadze je na doline Jozafat, i bede sie tam z nimi sadzil o lud swój, i o dziedzictwo swoje Izraelskie, które rozproszyli miedzy pogan, i ziemie moje rozdzielili.
\par 3 O lud tez mój los miotali, a dawali mlodzieniaszka za wszetecznice, a dzieweczke sprzedawali za wino, aby pili.
\par 4 Ale wy cóz przeciwko mnie macie, o Tyryjczycy i Sydonczycy i wszystkie granice Filistynskie? Izali wy mnie nagrode czynicie? Jezli mi tak nagrode czynicie, snadniec i predko i Ja obróce nagrode wasze na glowe wasze,
\par 5 Którzy srebro moje i zloto moje zabieracie, a klejnoty moje wyborne wnosicie do kosciolów swoich;
\par 6 A synów Judzkich i synów Jeruzalemskich sprzedawacie synom Jawanowym, abyscie ich oddalili od granic ich.
\par 7 Oto Ja wzbudze ich z tego miejsca, na którescie ich zaprzedali, a obróce nagrode wasze na glowe wasze;
\par 8 I zaprzedam synów waszych i córki wasze w rece synów Judzkich, i zaprzedadza ich Sebejcykom do narodu dalekiego; bo Pan mówil.
\par 9 Obwolajcie to miedzy narodami, ogloscie wojne, pobudzcie mocarzów, niech przyciagna a dadza sie najac wszyscy mezowie waleczni.
\par 10 Przekujcie lemiesze wasze na miecze, a kosy wasze na oszczepy; kto slaby, niech rzecze: Mocnym ja.
\par 11 Zgromadzcie sie, a zbiezcie sie wszystkie narody okoliczne, zbierzcie sie; spraw to, o Panie! ze tam zstapia mocarze twoi.
\par 12 Niech sie ocuca i przyciagna te narody na doline Jozafat; bo tam siedziec bede, abym sadzil wszystkie narody okoliczne.
\par 13 Zapuscciesz sierp, bo sie dostalo zniwo; pójdzcie, zstapcie, bo pelna jest prasa; oplywaja kadzi, bo wiele jest zlosci ich.
\par 14 Gromady, gromady lezec beda w dolinie posieczenia; bo bliski jest dzien Panski w dolinie posieczenia.
\par 15 Slonce i miesiac zacmia sie, a gwiazdy straca jasnosc swoje.
\par 16 Nadto Pan z Syonu zaryczy, a z Jeruzalemu wyda glos swój, tak, ze zadrza niebiosa i ziemia; Ale Pan jest ucieczka ludu swego i sila synów Izraelskich.
\par 17 I dowiecie sie, zem Ja Pan, Bóg wasz, mieszkajacy na Syonie, górze swietobliwosci swojej; a tak Jeruzalem bedzie swiete, a obcy nie przejda wiecej przez nie.
\par 18 I stanie sie dnia onego, ze góry kropic beda moszczem a pagórki oplywac mlekiem, i wszystkie strumienie Judzkie beda pelne wody, a z domu Panskiego wynijdzie zródlo, które obleje doline Syttym.
\par 19 Egipt przyjdzie na spustoszenie, a ziemia Edomska w straszna sie pustynie obróci dla gwaltu synom Judzkim uczynionego; bo wylewali krew niewinna w ziemi ich.
\par 20 Ale Juda na wieki trwac bedzie, a Jeruzalem od narodu do narodu;
\par 21 I oczyszcze tych, którychem krwi nie oczyscil; a Pan mieszka na Syonie.


\end{document}