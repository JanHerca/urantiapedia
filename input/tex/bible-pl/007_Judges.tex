\begin{document}

\title{Sędziów}


\chapter{1}

\par 1 I stalo sie po smierci Jozuego, iz pytali synowie Izraelscy Pana, mówiac: Któz z nas wprzód pójdzie przeciw Chananejczykowi, aby walczyl z nim?
\par 2 I rzekl Pan: Juda pójdzie; otom podal ziemie w reke jego.
\par 3 I rzekl Juda do Symeona, brata swego: Pójdz ze mna do losu mego, a bedziemy walczyli przeciw Chananejczykowi; wszak ja tez pójde z toba do losu twego. I szedl z nim Symeon.
\par 4 Tedy poszedl Juda, i podal Pan Chananejczyka, i Ferezejczyka w rece ich, a porazili z nich w Bezeku dziesiec tysiecy mezów.
\par 5 Bo nalezli Adonibezeka w Bezeku, i walczyli przeciwko niemu, a porazili Chananejczyka i Ferezejczyka.
\par 6 I uciekal Adonibezek, którego oni gonili; a pojmawszy go, poucinali palce wielkie u rak jego, i u nóg jego.
\par 7 Tedy rzekl Adonibezek: Siedmdziesiat królów z palcami wielkiemi obcietemi u rak swych i u nóg swych, zbierali odrobiny pod stolem moim; jakom czynil, tak mi oddal Bóg. I przywiedli go do Jeruzalemu, i tamze umarl.
\par 8 Bo walczyli przedtem synowie Judowi przeciwko Jeruzalemowi i wzieli je, i wysiekli je ostrzem miecza, i miasto spalili ogniem.
\par 9 Potem ciagneli synowie Juda, aby walczyli przeciw Chananejczykowi mieszkajacemu na górach, i na poludnie, i w polach.
\par 10 Ciagnal tedy Juda przeciwko Chananejczykowi, który mieszkal w Hebronie, (a imie Hebronu bylo przedtem Karyjatarbe) i porazil Sesai i Ahymana, i Talmaja.
\par 11 Stamtad zasie ciagneli do mieszkajacych w Dabir, (a imie Dabir bylo przedtem Karyjatsefer.)
\par 12 I rzekl Kaleb: Kto by dobyl Karyjatsefer, a wzialby je, dam mu Achse, córke moje, za zone.
\par 13 I wzial je Otonijel, syn Keneza, mlodszego brata Kalebowego.; a dal mu Achse, córke swa, za zone.
\par 14 I stalo sie, gdy przyszla do niego, namawiala go, aby prosil ojca jej o pole; i zsiadla z osla, i rzekl do niej Kaleb: Cóz ci?
\par 15 A ona rzekla: Daj mi blogoslawienstwo; gdyzes mi dal ziemie sucha, daj mi tez zródla wód. I dal jej Kaleb zródla wyzsze i zródla dolne.
\par 16 Synowie tez Ceni, swiekra Mojzeszowego, wyszli z miasta Palm z synami Judowymi na puszcza Judowe, która jest na poludnie od Arad, i przyszedlszy mieszkali z ludem.
\par 17 Potem ciagnal Juda z Symeonem, bratem swym, a porazili Chananejczyka, mieszkajacego w Sefat, a zburzyli je, i nazwali imie miasta onego Horma.
\par 18 Wzial tez Juda Gaze z granicami jego, i Akkaron z granicami jego.
\par 19 I byl Pan z Juda, i posiadl one góre; ale nie wypedzil mieszkajacych w dolinie, bo mieli wozy zelazne.
\par 20 A tak oddano Kalebowi Hebron, jako byl rozkazal Mojzesz, skad on wygnal trzech synów Enakowych.
\par 21 Ale Jebuzejczyka, mieszkajacego w Jeruzalemie, nie wygnali synowie Benjaminowi; przetoz mieszkal Jebuzejczyk z synami Benjaminowymi w Jeruzalemie az do dnia tego.
\par 22 Udal sie tez dom Józefów do Betel, a Pan byl z nimi.
\par 23 I szpiegowal dom Józefów Betel; (a imie miasta tego bylo przedtem Luz.)
\par 24 A ujrzawszy oni szpiegowie czlowieka wychodzacego z miasta, rzekli do niego: Ukaz nam prosimy wejscie do miasta, a uczynimy z toba milosierdzie.
\par 25 I ukazal im wejscie do miasta; i wysiekli miasto ostrzem miecza, a czlowieka onego ze wszystkim domem jego puscili wolno.
\par 26 A tak poszedl on czlowiek do ziemi Hetejczyków, i zbudowal miasto, a nazwal imie jego Luz; to jest imie jego az do dnia tego.
\par 27 Nie wypedzil tez Manases obywateli z Betsean i z miasteczek jego, ani z Tanach i z miasteczek jego, ani obywateli z Dor i z miasteczek jego, ani obywateli z Jeblam i z miasteczek jego, ani obywateli z Megiddo i z miasteczek jego; i poczal Chananejczyk mieszkac w onej ziemi.
\par 28 A gdy sie zmocnil Izrael, uczynil Chananejczyka holdownikiem, a nie wygnal go.
\par 29 Takze i Efraim nie wypedzil Chananejczyka mieszkajacego w Gazer; przetoz mieszkal Chananejczyk miedzy nimi w Gazer.
\par 30 Zabulon tez nie wypedzil mieszkajacych w Cetron, i mieszkajacych w Nahalol; przetoz mieszkal Chananejczyk miedzy nimi, bedac holdownikiem ich.
\par 31 Aser tez nie wypedzil mieszkajacych w Acho, i mieszkajacych w Sydonie, i w Ahalab, i w Achsyb, i w Helba, i w Afek, i w Rohob.
\par 32 I mieszkal Aser w posrodku Chananejczyka, mieszkajacego w onej ziemi; bo go nie wypedzil.
\par 33 Neftalim tez nie wypedzil obywateli z Betsemes, ani obywateli z Betanat, i mieszkal miedzy Chananejczykami mieszkajacymi w onej ziemi; jednak obywatele Betsemes i Betanat byli holdownikami ich.
\par 34 I scisneli Amorejczycy syny Danowe na górach, tak iz im nie dopuscili schodzic na doline.
\par 35 I poczal mieszkac Amorejczyk na górze Hares, w Ajalon i w Salebim; i wzmocnila sie reka domu Józefowego, i byli holdownikami ich.
\par 36 A byla granica Amorejczykowa od góry, gdzie wstepuja do niedzwiadków, od skaly ich i wyzej.

\chapter{2}

\par 1 I przyszedl Aniol Panski z Galgal do Bochym, mówiac: Wywiodlem was z Egiptu, i wywiodlem was do ziemi, o któram przysiagl ojcom waszym, i mówilem: Nie wzrusze przymierza mego z wami na wieki;
\par 2 Jedno wy nie wchodzcie w przymierze z mieszkajacymi w tej ziemi, owszem oltarze ich porozwalajcie; alescie nie sluchali glosu mojego. Przeczzescie to uczynili?
\par 3 Przetozem tak rzekl: Nie wypedze ich od oblicznosci waszej; ale beda wam jako ciernie na boki, a bogowie ich beda wam jako sidlo.
\par 4 I stalo sie, gdy mówil Aniol Panski te slowa do wszystkich synów Izraelskich, ze lud podniósl glos swój, i plakali.
\par 5 I nazwali imie miejsca onego Bochym, a tamze ofiarowali Panu.
\par 6 A Jozue rozpuscil byl lud, i rozeszli sie synowie Izraelscy kazdy do dziedzictwa swego, aby posiedli ziemie.
\par 7 Tedy sluzyl lud Panu po wszystkie dni Jozuego, i po wszystkie dni starszych, którzy dlugo zyli po smierci Jozuego, a którzy widzieli wszystkie sprawy Panskie wielkie, które uczynil Izraelowi.
\par 8 Ale gdy umarl Jozue syn Nunów, sluga Panski, bedac we stu i w dziesiec lat;
\par 9 I gdy go pogrzebli na granicy dziedzictwa jego w Tamnatheres na górze Efraim, od pólnocy góry Gaas;
\par 10 Takze gdy wszystek on rodzaj przylaczony jest do ojców swoich, i powstal po nich inszy naród, który nie znal Pana, ani tez spraw, który uczynil Izraelowi;
\par 11 Tedy uczynili synowie Izraelscy zle przed oczyma Panskiemi, a sluzyli Baalom;
\par 12 I opuscili Pana, Boga ojców swoich, który je wywiódl z ziemi Egipskiej, i szli za bogami cudzymi, którzy byli z bogów onych narodów okolicznych, i klaniali sie im, a tak rozdraznili Pana.
\par 13 Bo opuscili Pana, a sluzyli Baalowi i Astarotowi.
\par 14 I rozpalil sie gniew Panski przeciw Izraelowi, i podal je w rece lupiezcom, którzy je lupili; a zaprzedal je w rece nieprzyjaciól ich okolicznych, tak iz sie nie mogli dalej ostac przed nieprzyjacioly swymi.
\par 15 A gdzie sie kolwiek ruszyli, reka Panska byla przeciwko nim ku zlemu, jako powiedzial Pan, i jako im przysiagl Pan; i byli scisnieni bardzo.
\par 16 Potem Pan wzbudzil sedzie, którzy je wyzwalali z rak lupiezców ich;
\par 17 Ale i sedziów swych nie sluchali, owszem sie scudzolozyli z bogami obcymi, klaniajac sie im, i ustepowali predko z drogi, która chodzili ojcowie ich, a sluchajac przykazan Panskich, nie czynili tak.
\par 18 A gdy im wzbudzal Pan sedzie, bywal Pan z kazdym sedzia, i wybawial je z rak nieprzyjaciól ich po wszystkie dni onego sedziego; bo sie uzalil Pan narzekania ich, do którego je przywodzili ci, którzy je uciskali i trapili.
\par 19 Wszakze po smierci sedziego odwracali sie, i psowali sie bardziej niz ojcowie ich, chodzac za bogami cudzymi, a sluzac im, i klaniajac sie im, nic nie opuszczali z spraw swoich i z drogi swojej upornej.
\par 20 Przetoz wzruszyla sie popedliwosc Panska przeciw Izraelowi, i rzekl: Dla tego, ze przestapil ten naród przymierze moje, którem przykazal ojcom ich, a nie byli posluszni glosowi mojemu,
\par 21 Ja tez na potem nie wypedze zadnego od twarzy ich z tych narodów, które pozostawil Jozue, kiedy umarl.
\par 22 Abym przez nie doswiadczal Izraela, bedali strzedz drogi Panskiej, chodzac po niej, jako jej strzegli ojcowie ich, czyli nie.
\par 23 I zostawil Pan one narody, nie wyganiajac ich rychlo, ani ich podal w reke Jozuego.

\chapter{3}

\par 1 A tec sa narody, które pozostawil Pan, aby kusil przez nie Izraela, wszystkie, którzy nie wiedzieli o zadnych walkach Chananejskich;
\par 2 Aby wzdy wiedzieli potomkowie synów Izraelskich, i poznali, co jest walka, którzy jej zgola przedtem nie znali.
\par 3 Piecioro ksiazat Filistynskich i wszystkie Chananejczyki, i Sydonczyki i Hewejczyki, mieszkajace na górze Libanie, od góry Baal Hermon az tam, gdzie wchodza do Hemat.
\par 4 Cic byli, przez które doswiadczal Izraela, aby sie dowiedzial, bedali posluszni przykazaniom Panskim, które rozkazal ojcom ich przez Mojzesza.
\par 5 A tak synowie Izraelscy mieszkali w posród Chananejczyków, Hetejczyków, i Amorejczyków, i Ferezejczyków i Hewejczyków, i Jebuzejczyków.
\par 6 I brali sobie córki ich za zony, a córki swe dawali synom ich, i sluzyli bogom ich.
\par 7 I czynili Izraelscy synowie zle przed oczyma Panskiemi: zapomniawszy Pana, Boga swego, sluzyli Baalom, i swieconym gajom.
\par 8 Tedy sie zapalil gniew Panski przeciw Izraelowi, i podal je w reke Chusanrasataima, króla Syrskiego w Mezopotamii: A sluzyli synowie Izraelscy Chusanrasataimowi przez osiem lat.
\par 9 Potem wolali synowie Izraelscy do Pana; I wzbudzil Pan wybawiciela synom Izraelskim, aby je wybawil, Otonijela, syna Kenezowego, brata Kalebowego mlodszego.
\par 10 I byl nad nim Duch Panski, a sadzil Izraela; a gdy sie ruszyl na wojne, podal Pan w reke jego Chusanrasataima, króla Syryjskiego, i zmocnila sie reka jego nad Chusanrasataimem.
\par 11 A tak byla w pokoju ziemia przez czterdziesci lat, az umarl Otonijel, syn Kenezów.
\par 12 Potem znowu synowie Izraelscy czynili zle przed oczyma Panskiemi. I zmocnil Pan Eglona, króla Moabskiego, przeciw Izraelowi, przeto iz czynili zle przed oczyma Panskiemi.
\par 13 Bo zebrawszy do siebie syny Ammonowe i Amalekowe ruszyl sie, i porazil Izraela, i opanowal miasto Palm.
\par 14 Sluzyli tedy synowie Izraelscy Eglonowi, królowi Moabskiemu, osiemnascie lat.
\par 15 Potem wolali synowie Izraelscy do Pana. I wzbudzil im Pan wybawiciela, Aoda, syna Gery, syna Jemini, meza reka prawa niewladajacego; i poslali synowie Izraelscy przezen dar Eglonowi, królowi Moabskiemu.
\par 16 I uczynil sobie Aod miecz z obu stron ostry, na lokiec wzdluz, i przypasal go pod szaty swe do prawego biodra swojego.
\par 17 I przyniósl dar Eglonowi, królowi Moabskiemu: a Eglon byl czlowiek bardzo otyly.
\par 18 A gdy oddal dar, odprawil lud, który byl dar przyniósl;
\par 19 A sam wróciwszy sie do gór kamiennych, które byly w Galgal, rzekl: Rzecz tajemna mam do ciebie, o królu! któremu on odpowiedzial: Milcz; i wyszli od niego wszyscy, którzy stali przed nim.
\par 20 I wszedlszy Aod do niego, (a on sam siedzial na sali letniej, która mial sam dla siebie,)i rzekl Aod: Mam rozkazanie Boze do ciebie. I powstal z stolicy swojej.
\par 21 Tedy Aod wyciagnawszy lewa reke swa, dobyl miecza od prawego biodra swego, i wrazil go w brzuch jego,
\par 22 Tak iz wpadla i rekojesc za zelazem, i zawarlo sie w sadle zelazo; bo byl nie wyjal miecza z brzucha jego, az sie i gnój rzucil.
\par 23 Wyszedl potem Aod przez przysionek, a zamknal drzwi gmachu za soba, i zawarl zamkiem.
\par 24 A gdy on wyszedl, sludzy jego przyszli, a widzac, iz drzwi gmachu zamknione byly, rzekli: Podobno sobie król czyni wczas na sali letniej.
\par 25 A naczekawszy sie, az sie wstydzili, widzac, ze on nie otwiera drzwi sali, wziawszy klucz otworzyli; a oto, pan ich lezal na ziemi umarly.
\par 26 Lecz Aod uszedl, póki sie oni bawili, a minawszy góry kamienne, szedl do Seiratu.
\par 27 A gdy przyszedl, zatrabil w trabe na górze Efraim; i zstapili z nim synowie Izraelscy z góry, a on przed nimi.
\par 28 I rzekl do nich: Pójdzcie za mna: albowiem podal Pan nieprzyjacioly wasze Moabity w rece wasze. Tedy szli za nim, a odjawszy bród Jordanski Moabitom, nie dopuszczali nikomu przeprawy.
\par 29 I pobili Moabitów na ten czas okolo dziesieciu tysiecy mezów, wszystko bogatych, i wszystko mezów duzych, a nikt nie uszedl.
\par 30 I ponizony jest Moab dnia onego pod reka Izraela, a byla w pokoju ziemia przez osiemdziesiat lat.
\par 31 A po nim byl Samgar, syn Anatów, który porazil z Filistynów szesc set mezów stykiem wolowym, a wybawil i ten Izraela.

\chapter{4}

\par 1 Potem znowu synowie Izraelscy czynili zle przed oczyma Panskiemi po smierci Aodowej.
\par 2 I podal je Pan w rece Jabina, króla Chananejskiego, który królowal w Hasor, a hetman wojska jego byl Sysara, a sam mieszkal w Haroset poganskiem.
\par 3 Tedy wolali synowie Izraelscy do Pana; albowiem mial dziewiec set wozów zelaznych, a srodze uciskal syny Izraelskie przez dwadziescia lat.
\par 4 A Debora, niewiasta prorokini, zona Lapidotowa, sadzila Izraela na on czas.
\par 5 I mieszkala pod palma Debora miedzy Rama i miedzy Betel na górze Efraim, i chodzili do niej synowie Izraelscy na sad.
\par 6 Która poslawszy przyzwala Baraka, syna Abinoemowego, z Kades Neftalim, mówiac do niego: Izali nie rozkazal Pan, Bóg Izraelski: idz, a zbierz lud na górze Tabor, a wezmij z soba dziesiec tysiecy mezów z synów Neftalimowych, i z synów Zabulonowych?
\par 7 I przywiode do ciebie ku rzece Cyson Sysare, hetmana wojska Jabinowego, i wozy jego, i mnóstwo jego, a podam go w rece twoje.
\par 8 I rzekl do niej Barak: Jezli pójdziesz ze mna, pójde, a jezli nie pójdziesz ze mna, nie pójde.
\par 9 Która odpowiedziala: Jac w prawdzie pójde z toba, ale nie bedzie z slawa twoja ta droga, która ty pójdziesz; albowiem w reke niewiescia poda Pan Sysare. A tak wstawszy Debora, szla z Barakiem do Kades.
\par 10 Zebral tedy Barak Zabulona i Neftalima do Kades, a wywiódl z soba dziesiec tysiecy mezów, z którym tez szla i Debora.
\par 11 Ale Heber Cynejczyk odlaczyl sie od Cynejczyków, od synów Hobaby, swiekra Mojzeszowego, i rozbil namiot swój az do Elon w Sananim, które jest w Kades.
\par 12 I powiedziano Sysarze, iz wyszedl Barak, syn Abinoemów na góre Tabor.
\par 13 Przetoz zebral Sysara wszystkie wozy swoje, dziewiec set wozów zelaznych, i wszystek lud, który mial ze soba od Haroset poganskiego az do rzeki Cyson.
\par 14 Tedy rzekla Debora do Baraka: Wstan; albowiem tenci jest dzien, w który podal Pan Sysare w rece twoje; izali Pan nie idzie przed toba? A tak zszedl Barak z góry Tabor, a dziesiec tysiecy mezów za nim.
\par 15 I porazil Pan Sysare, i wszystkie wozy, i wszystko wojsko jego ostrzem miecza przed Barakiem, a skoczywszy Sysara z wozu, uciekal pieszo.
\par 16 Ale Barak gonil wozy i wojsko az do Haroset poganskiego; i poleglo wszystko wojsko Sysarowe od ostrza miecza, tak iz z nich i jeden nie zostal.
\par 17 A Sysara uciekl pieszo do namiotu Jaeli, zony Hebera Cynejczyka; albowiem byl pokój miedzy Jabinem, królem Hasor, i miedzy domem Hebera Cynejczyka.
\par 18 A wyszedlszy Jael przeciwko Sysarze, rzekla do niego: Sklon sie panie mój, sklon sie do mnie, nie bój sie; i sklonil sie do niej do namiotu, i przykryla go kocem.
\par 19 Tedy rzekl do niej: Daj mi prosze, napic sie troche wody, bom upragnal; a ona otworzywszy lagiew mleka, dala mu sie napic, i przykryla go.
\par 20 I rzekl do niej: Stój we drzwiach u namiotu; a jezliby kto przyszedl, i pytal cie, mówiac: Jestze tu kto? tedy rzeczesz: Nie masz.
\par 21 Potem wziela Jael, zona Heberowa, gwózdz od namiotu, wziela tez i mlot w reke swa, a wszedlszy do niego po cichu, przebila gwozdziem skron jego az utknal w ziemi, (bo byl twardo usnal, bedac spracowanym,)i umarl.
\par 22 A oto, Barak gonil Sysare, i wyszla Jael przeciwko niemu, i rzekla mu: Pójdz, a ukazec meza, którego szukasz. I wyszedl do niej, a oto Sysara lezal umarly, a gwózdz w skroni jego.
\par 23 A tak ponizyl Bóg dnia onego Jabina, króla Chananejskiego, przed syny Izraelskimi.
\par 24 I nacierala reka synów Izraelskich tem bardziej, a ciezka byla Jabinowi, królowi Chananejskiemu, az zgladzili tegoz Jabina, króla Chananejskiego.

\chapter{5}

\par 1 I spiewala Debora i Barak, syn Abinoemów, dnia onego, mówiac:
\par 2 Dla pomsty uczynionej w Izraelu, a iz sie na to dobrowolnie lud ofiarowal, blogoslawcie Pana.
\par 3 Sluchajcie królowie, bierzcie w uszy ksiazeta, ja, ja Panu zaspiewam, spiewac bede Panu, Bogu Izraelskiemu.
\par 4 Panie, gdys wyszedl z Seir, a przechodziles przez pole Edom, ziemia sie wzruszyla, nieba tez kropily, a obloki wydawaly wody.
\par 5 Góry sie rozplynely od oblicza Panskiego, a góra Synaj od oblicza Pana, Boga Izraelskiego.
\par 6 Za dni Samgara, syna Anatowego, i za dni Jaeli zaginely sciezki, a którzy szli w droge, chodzili sciezkami krzywemi.
\par 7 Spustoszaly wsi w Izraelu, spustoszaly, azem powstala ja Debora, azem powstala matka w Izraelu.
\par 8 Gdy Izrael obieral sobie bogi nowe, tedy bywala wojna w bramach; tarczy jednak nie bylo widac, ani drzewca miedzy czterdziesta tysiecy w Izraelu.
\par 9 Serce moje naklonione do ksiazat Izraelskich. Ochotni z ludu blogoslawciez Pana.
\par 10 Którzy jezdzicie na oslicach bialych, i zasiadacie na sadach, i którzy chodzicie po drogach, rozmawiajcie z soba,
\par 11 Ze ucichl trzask strzelców miedzy miejscami, gdzie czerpia wode; tam niech opowiadaja sprawiedliwosci Panskie, sprawiedliwosci we wsiach jego w Izraelu; tedy zstapi do bram lud Panski.
\par 12 Powstan Deboro, powstan, powstan, a zaspiewaj piesn; powstan Baraku, a pojmaj wieznie twoje, synu Abinoemów.
\par 13 Teraz panowac bedzie potloczony nad moznymi z ludu; Pan dopomógl mi panowac nad mocarzami.
\par 14 Z Efraima wyszedl korzen ich przeciw Amalekowi, za toba (Efraimie,)Benjamin miedzy ludem twoim; z Machyru wyszli zakonodawcy, a z Zabulonu pisarze.
\par 15 Ksiazeta tez Isaschar byly z Debora; Isaschar tez jako i Barak w doline poslan jest pieszo; ale w dziale Rubenitów byli ludzie wysokich mysli.
\par 16 Czemus siedzial miedzy dwiema oborami, sluchajac wrzasku trzód? w dziale Rubenitów byli ludzie wysokich mysli.
\par 17 Galaad za Jordanem odpoczynal, a Dan przecz sie bawil okretami? Aser czemu siedzial na brzegu morskim, a w skalach swoich mieszkal?
\par 18 Zabulon jest lud, który wydal dusze swa na smierc, takze i Neftalim, a to na wysokich polach.
\par 19 Przyszli królowie, walczyli; na ten czas walczyli królowie Chananejscy w Tanach u wód Magieddo, jednak korzysci srebra nie odniesli.
\par 20 Z nieba walczono: gwiazdy z miejsc swoich walczyly z Sysara.
\par 21 Potok Cyson porwal je, potok Kiedumim, potok Cyson; podeptalas, o duszo moja, mozne.
\par 22 Tedy sie popadaly kopyta konskie od wielkiego tapania mocarzów jego.
\par 23 Przeklinajcie Meroz, rzekl Aniol Panski, przeklinajcie przeklinajac obywatele jego; albowiem nie przyszli na ratunek Panu, na ratunek Panu z mocarzami.
\par 24 Blogoslawiona miedzy niewiastami Jael, zona Hebera Cynejczyka; nad niewiasty w namiocie mieszkajace blogoslawiona bedzie.
\par 25 Prosil o wode, a ona mleka dala, a na przystawce ksiazecej przyniosla masla.
\par 26 Lewa reke swa do gwozdzia sciagnela, a prawice swoje do mlota kowalskiego, i uderzyla Sysare, przebila glowe jego, i przerazila, i przeklula skronie jego.
\par 27 U nóg jej skurczyl sie, padl, lezal; u nóg jej skurczyl sie, padl; kedy sie skurczyl, tam upadl zabity.
\par 28 Oknem wygladala, a wolala matka Sysary przez krate: Przeczze omieszkiwa wrócic sie wóz jego? przecz sie nie spiesza nogi wozników jego?
\par 29 Przedniejsze i medrsze niewiasty odpowiedzialy, jako tez i sama sobie odpowiadala:
\par 30 Snac trafili na lup, i dziela go? Panienka jedna albo dwie dostana sie mezowi jednemu; lupy rozlicznych barw oddawaja Sysarze, a lupy pstro haftowane, i lupy pstro z obu stron tkane dostawaja sie na szyje lupy bioracych.
\par 31 Tak niechaj zgina wszyscy nieprzyjaciele twoi, Panie, a ci, którzy ciebie miluja, niech beda jako slonce, gdy wschodzi w mocy swojej. I byla w pokoju ziemia przez czterdziesci lat.

\chapter{6}

\par 1 Potem czynili synowie Izraelscy zle przed oczyma Panskiemi, i podal je Pan w rece Madyjanitów przez siedem lat.
\par 2 A zmocnila sie reka Madyjanitów nad Izraelem, tak iz przed Madyjanitami kopali sobie synowie Izraelscy lochy, które byly w górach, i jaskinie i twierdze.
\par 3 A bywalo, gdy czego nasial Izrael, ze przychodzil Madyjan i Amalek, i ludzie ze wschodu slonca, a najezdzali go;
\par 4 I polozywszy sie obozem przeciwko nim, psowali zboza ziemi, az gdzie chodza do Gazy, nic nie zostawujac na pozywienie Izraelczykom, ani owiec, ani wolów, ani oslów.
\par 5 Albowiem oni i stada ich przyciagali, i namioty ich, a przychodzili jako szarancza w mnóstwie, i nie bylo im i wielbladom ich liczby; tak przychodzac do ziemi spustoszyli ja.
\par 6 Tedy znedzony byl Izrael bardzo od Madyjanitów, i wolali synowie Izraelscy do Pana.
\par 7 A gdy wolali synowie Izraelscy do Pana z przyczyny Madyjanitów.
\par 8 Poslal Pan meza proroka do synów Izraelskich, i mówil do nich: Tak mówi Pan, Bóg Izraelski: Jam was wywiódl z ziemi Egipskiej, a wywiodlem was z domu niewoli.
\par 9 I wyrwalem was z reki Egipczanów, i z reki wszystkich, którzy was trapili, którem przed wami wygnal, i dalem wam ziemie ich;
\par 10 A powiedzialem wam: Jam Pan, Bóg wasz, nie bójciez sie bogów Amorejskich, w których ziemi wy mieszkacie; alescie nie usluchali glosu mego.
\par 11 Przyszedl potem Aniol Panski i stanal pod debem, który byl w Efra, w dziedzictwie Joasa, ojca Esrowego. A Giedeon, syn jego, mlócil zboze na bojewisku, aby z niem uciekl przed Madyjanitami.
\par 12 Tedy mu sie ukazal Aniol Panski, i rzekl do niego: Pan z toba, mezu waleczny.
\par 13 I odpowiedzial mu Giedeon: Prosze Panie mój, jezli Pan jest z nami, a czemuz na nas przyszlo to wszystko? gdziez teraz sa wszystkie cuda jego, które nam opowiadali ojcowie nasi, mówiac: Izali z Egiptu nie wywiódl nas Pan? a teraz opuscil nas Pan, i podal nas w rece Madyjanitów.
\par 14 Tedy wejrzawszy nan Pan rzekl: Idzze z ta twoja moca, a wybawisz Izraela z reki Madyjanczyków; izalim cie nie poslal?
\par 15 A on rzekl do niego: Prosze Panie mój, czemze wybawie Izraela? oto naród mój podly jest w Manase, a jam najmniejszy w domu ojca mego.
\par 16 I rzekl do niego Pan: Poniewaz Ja bede z toba, przetoz porazisz Madyjanity, jako meza jednego.
\par 17 A on mu odpowiedzial: Jezlizem prosze znalazl laske przed oczyma twemi, daj mi znak, ze ty mówisz ze mna.
\par 18 Nie odchodz prosze stad, az zas przyjde do ciebie, a przyniosec ofiare moje, i poloze ja przed toba. I odpowiedzial: Ja poczekam, az sie wrócisz.
\par 19 Odszedlszy tedy Giedeon zgotowal kozlatko z stada a z miary maki przasne chleby, a mieso wlozyl w kosz, a polewke miesna wlal w garnek, i przyniósl to do niego pod dab, i ofiarowal.
\par 20 I rzekl do niego Aniol Bozy: Wezmij to mieso i te chleby niekwaszone, a polóz na onej skale polewka polawszy; i uczynil tak.
\par 21 Zatem sciagnal Aniol Panski koniec laski, która mial w rece swojej, i dotknal sie miesa i przasników, i wyszedl ogien z skaly, a spalil mieso i chleby przasne; a miedzy tem Aniol Panski odszedl od oczu jego.
\par 22 A widzac Giedeon, iz to byl Aniol Panski, rzekl: Ach, Panie Boze, czemuzem widzial Aniola Panskiego twarza w twarz?
\par 23 I rzekl mu Pan: Pokój z toba; nie bój sie, nie umrzesz.
\par 24 Przetoz zbudowal tam Giedeon oltarz Panu, i nazwal go: Pan pokoju; az do dnia tego ten jeszcze jest w Efracie, ojca Esrowego.
\par 25 I stalo sie onej nocy, ze mu rzekl Pan: Wezmij cielca doroslego, który jest ojca twego, tego cielca drugiego siedmioletniego, a rozwal oltarz Baalów, który jest ojca twego, i gaj, który jest okolo niego, wysiecz;
\par 26 A zbuduj oltarz Panu, Bogu twemu, na wierzchu tej skaly na równinie, a wezmij tego cielca drugiego, i uczyn z niego calopalenie przy drwach z gaju, który wysieczesz.
\par 27 Wziawszy tedy Giedeon dziesiec mezów z slug swoich, uczynil jako mu rozkazal Pan; a iz sie bal domu ojca swego i mezów miasta, nie uczynil tego we dnie, ale uczynil w nocy.
\par 28 A gdy wstali mezowie miasta rano, ujrzeli rozwalony oltarz Baalów, i gaj, który byl podle niego, wyrabany, i cielca onego drugiego ofiarowanego na calopalenie na oltarzu zbudowanym.
\par 29 Zatem rzekl jeden do drugiego: Któz to wzdy uczynil? A gdy sie pytali i dowiadowali, powiedziano: Giedeon, syn Joasów, uczynil to.
\par 30 Tedy rzekli mezowie miasta do Joasa: Wywiedz syna twego, niech umrze, iz rozrzucil oltarz Baalów, a iz wycial gaj, który byl okolo niego.
\par 31 I odpowiedzial Joas wszystkim, którzy stali okolo niego: A wyz sie to swarzyc macie o Baala? Izali wy go wybawicie? Kto by sie on zastawial, niech umrze tegoz poranku; jezli bogiem jest, niech sie msci tego, ze rozwalono oltarz jego.
\par 32 I nazwano go onegoz dnia Jerubaal, mówiac: Niech sie msci nad nim Baal, iz rozwalil oltarz jego.
\par 33 Tedy wszyscy Madyjanitowie, i Amalekitowie, i ludzie od wschodu slonca zebrali sie wespól, a przeprawiwszy sie przez Jordan, polozyli sie obozem w dolinie Jezreel.
\par 34 Ale Duch Panski przyoblókl Giedeona, który zatrabiwszy w trabe zwolal domu Abiezerowego do siebie.
\par 35 I wyprawil posly do wszystkiego pokolenia Manasesowego, i zebrali sie do niego; posly tez poslal do Asera, i do Zabulona, i do Neftalima, i zajechali im.
\par 36 Tedy rzekl Giedeon do Boga: Jezli wybawisz przez reke moje Izraela, jakos powiedzial.
\par 37 Oto, ja poloze runo welny na bojewisku; jezliz rosa tylko na runo upadnie, a wszystka ziemia sucha bedzie, tedy bede wiedzial, iz wybawisz przez reke moje Izraela, jakos powiedzial.
\par 38 I stalo sie tak; bo wstawszy nazajutrz, scisnal runo, i wyzdzal rosy z runa pelna czasze wody.
\par 39 Nadto rzekl Giedeon do Boga: Niech sie nie wzrusza gniew twój przeciwko mnie, ze przemówie jeszcze raz. Niech doswiadcze prosze jeszcze raz na tem runie; niech bedzie, prosze, suche samo runo tylko, a na wszystkiej ziemi niech bedzie rosa.
\par 40 I uczynil tak Bóg onej nocy, ze bylo samo runo suche, a na wszystkiej ziemi byla rosa.

\chapter{7}

\par 1 Wstal tedy bardzo rano Jerobaal, który jest Giedeon, i wszystek lud, który byl z nim, i polozyli sie obozem u zródla Harod, a obóz Madyjanski byl im na pólnocy od pagórka More w dolinie.
\par 2 I rzekl Pan do Giedeona: Wielki jest lud z toba; przetoz nie dam Madyjanitów w rece ich, by sie snac nie chlubil przeciw mnie Izrael, mówiac: Reka moja wybawila mie.
\par 3 A tak zawolaj teraz, aby slyszal lud, mówiac: Kto jest lekliwym i bojazliwym, niech sie wróci, a rano niechaj idzie precz ku górze Galaad. Tedy sie wrócilo z ludu dwadziescia i dwa tysiace, a dziesiec tysiecy ich zostalo.
\par 4 I rzekl Pan do Giedeona: Jeszcze lud wielki. Zaprowadz ich do wody, a tam go doswiadcze; albowiem o kim ci powiem: Ten niech idzie z toba, ten pójdzie z toba, a o kimcikolwiek powiem: Ten niech nie chodzi z toba, ten nie pójdzie.
\par 5 Tedy zaprowadzil lud do wód; i rzekl Pan do Giedeona: Kazdego, który leptac bedzie jezykiem swoim wode, jako pies lepce, postawisz go osobno; takze kazdego, który ukleknie na kolana swoje, aby pil, stanie osobno.
\par 6 I byla liczba tych, którzy chwytali reka swoja do ust swoich wode, trzy sta mezów; a wszystek inny lud ukleknawszy na kolana swoje, pil wode.
\par 7 Tedy rzekl Pan do Giedeona: Przez tych trzy sta mezów, którzy leptali wode, wybawie was, a podam Madyjanity w rece twoje, a inny wszystek lud, kazdy niech idzie na miejsce swoje.
\par 8 A tak on lud wzial zywnosci z soba, i traby swe; a inne wszystkie meze Izraelskie rozpuscil kazdego do namiotu swego, trzy sta tylko mezów zostawiwszy; a obóz Madyjanski byl pod nim w dolinie.
\par 9 I stalo sie onej nocy, ze rzekl do niego Pan. Wstan, znijdz do obozu, bom go dal w rece twoje;
\par 10 A jezli sie ty sam isc boisz, znijdzze z Fara, sluga twoim, do obozu.
\par 11 I uslyszysz, co beda mówic; a potem posila sie rece twoje, i pociagniesz na obóz. A tak szedl on sam, i Fara, sluga jego, az na koniec zbrojnego ludu, który byl w obozie.
\par 12 A Madyjanitowie i Amalekitowie, i wszystek lud od wschodu slonca lezeli w dolinie, jako szarancza przemnóstwo, i wielbladów ich nie bylo liczby, jako piasek, który jest na brzegu morskim niezliczony.
\par 13 Tam gdy przyszedl Giedeon, oto, niektóry powiadal towarzyszowi swemu sen, i rzekl: Oto snil mi sie sen, a zdalo mi sie, ze bochen chleba jeczmiennego toczyl sie do obozu Madyjanskiego, i przytoczyl sie az do namiotu, i uderzyl wen, az polegl, i wywrócil go z wierzchu, i upadl namiot.
\par 14 Któremu odpowiedzial towarzysz, jego i rzekl: Nic to nie jest innego, jedno miecz Giedeona, syna Joasowego, meza Izraelskiego; dal Bóg w reke jego Madyjanity ze wszystkim obozem.
\par 15 I stalo sie, gdy uslyszal Giedeon powiesc snu onego, i wyklad jego, podziekowal Bogu, a wróciwszy sie do obozu Izraelskiego, rzekl: Wstancie; albowiem dal Pan w rece wasze obóz Madyjanski.
\par 16 Rozdzielil tedy one trzy sta mezów na trzy hufce, a dal traby w rece kazdemu z nich, i dzbany czcze, i pochodnie w posrodek dzbanów
\par 17 I rzekl do nich: Co ujrzycie, ze ja czynie, toz czyncie; bo oto ja wnijde w przodek obozu, a co ja czynic bede, toz wy czyncie.
\par 18 Gdy zatrabie w trabe, ja i wszyscy, którzy sa ze mna, tedy wy tez zatrabicie w traby okolo wszystkiego obozu, i bedziecie mówili: Miecz Panski i Giedeonów.
\par 19 A tak szedl Giedeon, i sto mezów, którzy z nim byli, w przodek obozu, gdy sie zaczela srednia straz; zaraz skoro przemieniono straz i trabili w traby, a potlukli dzbany, które w rekach swych mieli.
\par 20 Zatrabily tez one trzy hufy w traby, i potlukly dzbany; a wziawszy w lewa reke swoje pochodnie, a w prawa reke swoje traby, aby trabili, wolali: Miecz Panski i Giedeonów.
\par 21 I staneli kazdy na miejscu swojem kolo obozu, a strwozyl sie wszystek obóz, i krzyczac uciekali.
\par 22 Gdy tedy trabili oni trzy sta mezów w traby, obrócil Pan miecz jednego przeciw drugiemu we wszystkim obozie; a tak ucieklo wojsko az do Betseta w Sererat, az do granicy Abelmehola w Tabbat.
\par 23 A zebrawszy sie mezowie Izraelscy z Neftalim, i z Aser, i ze wszystkiego pokolenia Manasesowego, gonili Madyjanczyki.
\par 24 Zatem posly rozeslal Giedeon na wszystke góre Efraimska, mówiac: Zabiegajcie Madyjanitom, a ubiezcie przed nimi wody az do Betabara, i do Jordanu. Zebrali sie tedy wszyscy mezowie z Efraima, i ubiezeli wody az do Betabara i do Jordanu.
\par 25 Przytem pojmali dwoje ksiazat Madyjanskich, Oreba i Zeba; a zabili Oreba na skale Oreb, a Zeba zabili u prasy Zeb, i gonili Madyjanity, a glowe Oreba i Zeba przyniesli do Giedeona za Jordan.

\chapter{8}

\par 1 I rzekli do niego mezowie z Efraim: Przeczzes to nam uczynil, izes nas nie wezwal, gdys szedl walczyc przeciwko Madyjanitom? i swarzyli sie z nim srodze.
\par 2 A on rzekl: Cózem ja takiego uczynil, jako wy? izali nie lepsze jest poslednie zbieranie wina Efraimowe, niz pierwsze zbieranie Abieserowe?
\par 3 W reke wasza podal Bóg ksiazeta Madyjanskie, Oreba i Zeba; i cózem mógl takiego uczynic, jako wy? Tedy sie usmierzyl duch ich przeciw niemu, gdy mówil te slowa.
\par 4 A gdy przyszedl Giedeon do Jordanu, przeprawil sie przezen sam, i trzy sta mezów, którzy z nim byli spracowani w pogoni.
\par 5 I rzekl do mieszczan w Sokot: Dajcie prosze po bochenku chleba ludowi, który idzie za mna, bo sa spracowani, a ja bede gonil Zebee i Salmana, króle Madyjanskie.
\par 6 Ale mu rzekli przedniejsi z Sokot: Izali juz moc Zeby i Salmana jest w rekach twoich, zebysmy dac mieli wojsku twemu chleba?
\par 7 Którym rzekl Giedeon: Wiec kiedy poda Pan Zebee i Salmana w reke moje, tedy bede mlócil ciala wasze cierniem z tej puszczy i ostem.
\par 8 Szedl zasie stamtad do Fanuel, i mówil takze do nich; ale mu odpowiedzieli mezowie z Fanuel, jako odpowiedzieli mezowie w Sokot.
\par 9 Tedy tez powiedzial mezom z Fanuel, mówiac: Gdy sie wróce w pokoju, rozwale te wieze.
\par 10 Ale Zebee i Salmana byli w Karkor, i wojska ich z nimi okolo pietnastu tysiecy, wszyscy, którzy byli pozostali ze wszystkiego wojska z ludzi od wschodu slonca; a pobitych bylo sto i dwadziescia tysiecy mezów walecznych.
\par 11 Tedy ciagnal Giedeon droga tych, co mieszkali w namiociech, od wschodu slonca Nobe i Jegbaa, i uderzyl na obóz, (a obóz sie byl ubezpieczyl,)
\par 12 A uciekli Zebee i Salmana, i gonil je, i pojmal onych dwóch królów Madyjanskich, Zebee i Salmana, i wszystko wojsko ich strwozyl.
\par 13 Potem sie wrócil Giedeon, syn Joasów, z bitwy, niz weszlo slonce;
\par 14 A pojmawszy mlodzienca z mezów Sokot, wypytal go, który spisal mu przedniejszych w Sokot i starszych jego, siedemdziesiat i siedem mezów.
\par 15 A przyszedlszy do mezów Sokot, rzekl: Otóz Zebee i Salmana, którymiscie mi uragali, mówiac: Izali moc Zeby i Salmana jest w rekach twoich, abysmy mieli dac mezom twoim spracowanym chleba?
\par 16 Przetoz wziawszy starsze miasta onego, i ciernia z onej pustyni i ostu, dal na nich przyklad innym mezom Sokot.
\par 17 Wieze tez Fanuel rozwalil, i pobil meze miasta.
\par 18 Rzekl potem do Zeby i do Salmana: Co zacz byli mezowie oni, którescie pobili w Tabor? A oni odpowiedzieli: Takowi byli jakos ty; kazdy z nich na wejrzeniu byl, jako syn królewski.
\par 19 I rzekl: Braciac to moi, synowie matki mojej byli; zywie Pan, byscie je byli zywo zachowali, nie pobilbym was.
\par 20 I rzekl do Jetra, pierworodnego swego: Wstan, a pobij je. Ale nie dobyl mlodzienczyk miecza swego, przeto iz sie bal; bo jeszcze byl pacholeciem.
\par 21 Tedy rzekli Zebee i Salmana: Wstan ty, a rzuc sie na nas; bo jaki maz, taka sila jego. A tak wstawszy Giedeon, zabil Zebee i Salmana, i pobral klejnoty, które byly na szyjach wielbladów ich.
\par 22 I rzekli Izraelczycy do Giedeona: Panuj nad nami, i ty, i syn twój, i syn syna twego; bos nas wybawil z reki Madyjanczyków.
\par 23 Na to odpowiedzial im Giedeon: Nie bede ja panowal nad wami, ani bedzie panowal syn mój nad wami; Pan panowac bedzie nad wami.
\par 24 Nadto rzekl do nich Giedeon: Bede was prosil o jedne rzecz, aby mi kazdy z was dal nausznice z lupu swego; (bo nausznice zlote mieli, bedac Ismaelczykami.)
\par 25 I rzekli: Radzic damy; i rozpostarlszy szate rzucali na nie kazdy nausznice z lupów swoich.
\par 26 I byla waga nausznic onych zlotych, które sobie uprosil, tysiac i siedemset syklów zlota, oprócz klejnotów i zawieszenia i szat szarlatowych, które byly na królach Madyjanskich, i oprócz lancuchów, które byly na szyjach wielbladów ich.
\par 27 I sprawil z tego Giedeon Efod, a polozyl go w miescie swem w Efra; i scudzolozyl sie tam wszystek Izrael, chodzac za nim, a bylo to Giedeonowi i domowi jego sidlem.
\par 28 A tak byli ponizeni Madyjanczycy przed synami Izraelskimi, i nie podniesli wiecej glowy swojej; i byla w pokoju ziemia przez czterdziesci lat za dni Giedeona.
\par 29 Wrócil sie tedy Jerobaal, syn Joasów, i mieszkal w domu swoim.
\par 30 A mial Giedeon siedmdziesiat synów, którzy poszli z biódr jego; albowiem mial wiele zon.
\par 31 Mial tez zaloznice, która byla z Sychem, a ta mu urodzila syna, i dala mu imie Abimelech.
\par 32 Umarl potem Giedeon, syn Joasów, w starosci dobrej, a pogrzebion jest w grobie Joasa, ojca swego w Efra, które jest ojca Esrowego.
\par 33 A gdy umarl Giedeon, odwrócili sie synowie Izraelscy, i scudzolozyli sie, idac za Baalem, i postawili sobie Baalberyt za boga.
\par 34 I nie pamietali synowie Izraelscy na Pana, Boga swego, który je wyrwal z rak wszystkich nieprzyjaciól ich okolicznych;
\par 35 I nie uczynili milosierdzia z domem Jerobaala Giedeona wedlug wszystkich dobrodziejstw, które on byl uczynil Izraelowi.

\chapter{9}

\par 1 Tedy odszedl Abimelech, syn Jerobaalów, do Sychem, do braci matki swojej, i mówil do nich, i do wszystkiego narodu domu ojca matki swej, a rzekl:
\par 2 Mówcie prosze, gdzieby slyszeli wszyscy przelozeni Sychem: Co wam lepszego, aby nad wami panowalo siedmdziesiat mezów, wszyscy synowie Jerobaalowi, czyli zeby panowal nad wami maz jeden? Wzdy pamietajcie, zem ja kosc wasza, i cialo wasze.
\par 3 Tedy mówili bracia matki jego o nim, gdzie slyszeli wszyscy przelozeni Sychem, wszystkie te slowa, i naklonilo sie serce ich za Abimelechem, bo rzekli: Brat nasz jest.
\par 4 I dali mu siedmdziesiat srebrników z domu Baalberyt, i naprzyjmowal za nie Abimelech ludzi lekkomyslnych, i tulaczów, którzy chodzili za nim.
\par 5 I przyszedl w dom ojca swego do Efra, i pobil bracia swa, syny Jerobaalowe, siedmdziesiat mezów na jednym kamieniu; tylko zostal Jotam, syn Jerobaalów, najmniejszy, iz sie byl skryl.
\par 6 I zebrali sie wszyscy mezowie Sychem, i wszystek dom Mello, a szedlszy obrali Abimelecha królem na równinie, kedy stal slup w Sychem.
\par 7 Co gdy powiedziano Jotamowi, szedlszy stanal na wierzchu góry Garyzym, a podnióslszy glos swój wolal, i rzekl im: Posluchajcie mie mezowie Sychem, a was tez Bóg uslyszy.
\par 8 Zeszly sie drzewa, aby pomazaly nad soba króla, i rzekly do oliwnego drzewa: Króluj nad nami.
\par 9 Którym odpowiedzialo oliwne drzewo: Izali opuszcze tlustosc moje, przez która uczczony bywa Bóg i ludzie, a pójde, abym wystawione bylo nad drzewy?
\par 10 Rzekly potem drzewa do figowego drzewa: Pójdz ty, króluj nad nami.
\par 11 Którym odpowiedzialo figowe drzewo: Izali opuszcze slodkosc moje, i owoc mój wyborny, a pójde, abym wystawione bylo nad drzewy?
\par 12 Potem rzekly drzewa do macicy winnej: Pójdz ty, króluj nad nami.
\par 13 Tedy im odpowiedziala macica: Izali opuszcze moszcz mój, który uwesela Boga i ludzie, a pójde, abym wystawiona byla nad drzewy?
\par 14 I rzekly wszystkie drzewa do ostu: Pójdz ty, króluj nad nami.
\par 15 Tedy odpowiedzial oset drzewom: jezli wy prawdziwie chcecie pomazac mie za króla nad soba, pójdzciez, a odpoczywajcie pod cieniem moim, a jezliz nie, niech wynijdzie ogien z ostu, a spali Cedry Libanskie.
\par 16 A tak teraz jezliscie prawdziwie a szczerze uczynili, obrawszy sobie królem Abimelecha; a jezliscie sie dobrze obeszli z Jerobaalem i z domem jego, a jezliscie podlug dobrodziejstw reki jego uczynili z nim;
\par 17 (Albowiem walczyl ojciec mój za was, i podal dusze swa w niebezpieczenstwo, aby was wyrwal z reki Madyjanczyków;
\par 18 Alescie wy powstali przeciw domowi ojca mego dzis, i pobiliscie syny jego, siedmdziesiat mezów na jednym kamieniu, i obraliscie królem Abimelecha, syna sluzebnicy jego, nad mezami Sychem, iz bratem waszym jest.)
\par 19 Jezliscie prawdziwie a szczerze obeszli sie z Jerobaalem, i z domem jego dnia tego, weselciez sie z Abimelecha, a on niech sie tez weseli z was.
\par 20 Ale jezliz nie, niechajze wynijdzie ogien z Abimelecha, a pozre meze Sychem, i dom Mello; niechajze tez wynijdzie ogien od mezów Sychem, i z domu Mello, a pozre Abimelecha.
\par 21 Tedy uciekl Jotam, a ucieklszy poszedl do Beer, i mieszkal tam, bojac sie Abimelecha, brata swego.
\par 22 A tak panowal Abimelech nad Izraelem przez trzy lata.
\par 23 I poslal Bóg ducha zlego miedzy Abimelecha i miedzy meze Sychemskie, a zlamali wiare mezowie Sychem Abimelechowi;
\par 24 Aby sie zemszczono krzywdy siedemdziesieciu synów Jerobaalowych, aby krew ich przyszla na Abimelecha, brata ich, który je pobil, i na meze Sychemskie, którzy zmocnili rece jego, aby pobil bracia swoje.
\par 25 I zasadzili sie nan mezowie Sychemscy na wierzchu gór, a rozbijali kazdego, który jedno szedl tamta droga. I powiedziano to Abimelechowi.
\par 26 Nadto przyszedl Gaal, syn Obedów, i bracia jego, i przyszedl do Sychem, a poufali mu mezowie Sychem.
\par 27 A wyszedlszy na pola zbierali wina swoje, i tloczyli, weselac sie; a wszedlszy w dom bogów swoich, jedli i pili, a zlorzeczyli Abimelechowi.
\par 28 Tedy rzekl Gaal, syn Obedów: Cóz jest Abimelech? i co jest Sychem, zebysmy mu sluzyli? azaz nie jest syn Jerobaalów, a Zebul urzednikiem jego? raczej sluzcie mezom Hemora, ojca Sychemowego; bo czemuz bysmy onemu sluzyc mieli?
\par 29 O by kto podal ten lud w rece moje, aby sprzatnal Abimelecha! I rzekl Abimelechowi: Zbierz swoje wojsko, a wynijdz.
\par 30 A uslyszawszy Zebul, przelozony miasta onego, slowa Gaala, syna Obedowego, zapalil sie gniew jego.
\par 31 I wyprawil posly do Abimelecha potajemnie, mówiac: Oto Gaal, syn Obedów, i bracia jego przyszli do Sychem, a oto chca walczyc z miastem przeciwko tobie.
\par 32 A tak teraz wstan noca, ty i lud, który jest z toba, a uczyn zasadzke w polu.
\par 33 A rano, gdy slonce wznijdzie, wstawszy uderzysz na miasto; a gdy on i lud, który jest z nim, wynijdzie przeciw tobie, uczynisz z nim, co bedzie chciala reka twoja.
\par 34 Tedy wstawszy Abimelech, i wszystek lud, który z nim byl, w nocy, zasadzili sie przeciw Sychem na czterech miejscach.
\par 35 A wyszedlszy Gaal, syn Obedów, stanal w samej bramie miasta; wstal tez i Abimelech, i lud, który z nim byl, z zasadzki.
\par 36 A widzac Gaal lud, rzekl do Zebula: Oto, lud idzie z wierzchu gór. Któremu odpowiedzial Zebul: Cien góry ty widzisz, jakoby ludzie.
\par 37 Tedy po wtóre rzekl Gaal, mówiac: Oto, lud zstepuje z góry a jeden huf idzie droga równiny Meonenim.
\par 38 Tedy rzekl do niego Zebul: Gdziez teraz usta twoje, które mówily: Co jest Abimelech, abysmy mu sluzyc mieli? izali to nie ten lud, którys wzgardzil? wnijdz teraz, a walcz przeciw niemu.
\par 39 A tak wyszedl Gaal przed mezami Sychem, a walczyl przeciw Abimelechowi.
\par 40 I gonil go Abimelech, gdy przed nim uciekal, a poleglo wiele rannych az do samej bramy.
\par 41 I zostal Abimelech w Aruma; a Zebul wygnal Gaala z bracia jego, aby nie mieszkali w Sychem.
\par 42 Ale nazajutrz wyszedl lud w pole, i powiedziano to Abimelechowi.
\par 43 Wziawszy tedy lud, rozdzielil go na trzy hufce, i zasadzil sie w polu; a widzac, a oto lud wychodzil z miasta, wypadl na nie, i pobil je.
\par 44 Bo Abimelech i hufy, które z nim byly, uderzyly na nie, i staneli u samej bramy miasta, a inne dwa hufy uderzyly na wszystkie, którzy byli w polu, i pobili je.
\par 45 A Abimelech dobywal miasta przez on wszystek dzien, i wzial je; a lud, który w niem byl, pomordowal, a zburzywszy miasto, posial je sola.
\par 46 A uslyszawszy wszyscy mezowie, którzy byli na wiezy Sychem, przyszli do twierdzy domu boga Beryt.
\par 47 I opowiedziano Abimelechowi, ze sie tam zgromadzili wszyscy mezowie wiezy Sychem.
\par 48 Tedy szedl Abimelech na góre Salmon, on i wszystek lud, który byl z nim; a nabrawszy z soba siekier, ucial Gala? z drzewa, a wziawszy ja, wlozyl na ramiona swoje, i rzekl do ludu, który z nim byl: Co widzicie, zem uczynil, predko czyncie tak, jak o ja.
\par 49 A tak uciawszy z onego wszystkiego ludu kazdy Gala? swoje, szli za Abimelechem, a kladli je okolo twierdzy i spalili niemi twierdza ogniem, i poginelo tam wszystkich mezów wiezy Sychemskiej okolo tysiaca mezów i niewiast.
\par 50 Potem szedl Abimelech do Tebes, a polozywszy sie przeciwko Tebes, dobyl go.
\par 51 Ale wieza byla mocna w posrodku miasta, na która uciekli wszyscy mezowie, i niewiasty, i wszyscy przedniejsi miasta, a zamknawszy ja za soba, weszli na dach wiezy.
\par 52 Tedy przyszedl Abimelech az do samej wiezy, i dobywal jej, a stanawszy u samych drzwi wiezy, chcial ja spalic ogniem.
\par 53 Miedzy tem zrzucila niewiasta niektóra sztuke kamienia od zarn na glowe Abimelechowe, i rozbila wierzch glowy jego.
\par 54 Który natychmiast zawolawszy pacholika, co nosil bron jego, rzekl do niego: Dobadz miecza twego, a zabij mie, by snac nie rzeczono o mnie: Niewiasta go zabila: a tak przebil go pacholik jego, i umarl.
\par 55 A widzac mezowie Izraelscy, iz umarl Abimelech, rozeszli sie kazdy do miejsca swego.
\par 56 I oddal Bóg ono zle Abimelechowi, które uczynil nad ojcem swoim, zabiwszy siedmdziesiat braci swych.
\par 57 I wszystko zle mezów Sychem obrócil Bóg na glowe ich; a przyszlo na nie przeklestwo Jotama, syna Jerobaalowego.

\chapter{10}

\par 1 I powstal po Abimelechu na obrone Izraela Tola, syn Fui, syna Dodowego, maz z pokolenia Isascharowego, a ten mieszkal w Samir na górze Efraim.
\par 2 I sadzil Izraela przez dwadziescia i trzy lata, potem umarl, i pogrzbion jest w Samir.
\par 3 A po nim powstal Jair Galaadczyk, który sadzil Izraela przez dwadziescia i dwa lata.
\par 4 A ten mial trzydziesci synów, którzy jezdzili na trzydziestu oslat, a mieli trzydziesci miast, które zwano Awot Jair az po dzisiejszy dzien w ziemi Galaadskiej.
\par 5 I umarl Jair, a pogrzebion jest w Kamon.
\par 6 Lecz znowu synowie Izraelscy czynili zle przed oczyma Panskiemi, a sluzac Baalowi, i Astaratowi, i bogom Syryjskim, i bogom Sydonskim, i bogom Moabskim, i bogom synów Ammon, i nawet bogom Filistynskim, a opusciwszy Pana, nie sluzyli mu.
\par 7 Przetoz sie wzruszyl gniewem Pan przeciw Izraelowi, i podal je w rece Filistynów, i w rece synów Ammonowych.
\par 8 Którzy trapili i uciskali syny Izraelskie od onego roku przez osiemnascie lat, wszystkie syny Izraelskie, którzy byli przed Jordanem w ziemi Amorejczyka, która jest w Galaad.
\par 9 Przeprawili sie tez synowie Ammonowi za Jordan, aby walczyli z Juda, i z Benjaminem, takze i z domem Efraimowym, i scisneli Izraelczyki bardzo.
\par 10 Tedy wolali synowie Izraelscy do Pana, mówiac: Zgrzeszylismy tobie, zesmy opuscili ciebie, Boga naszego, i sluzylismy Baalom.
\par 11 Ale Pan rzekl do synów Izraelskich: Izalim was od Egipczyków, i od Amorejczyków, od synów Ammonowych, i od Filistynów,
\par 12 I od Sydonczyków, i Amalekitów, i Mahanitów, którzy was trapili, gdyscie wolali do mnie, nie wybawil z reki ich?
\par 13 Alescie wy mie opuscili, a sluzyliscie bogom cudzym; przetoz was wiecej nie wybawie.
\par 14 Idzciez, a wolajcie do bogów, którescie sobie obrali; oni niechaj was wybawia czasu ucisku waszego.
\par 15 I odpowiedzieli synowie Izraelscy Panu: Zgrzeszylismy; uczynze ty z nami, co sie zda dobrego w oczach twoich, tylko wybaw nas prosimy dnia tego.
\par 16 I wyrzucili bogi cudze z posrodku siebie, a sluzyli Panu, i uzalil sie Pan utrapienia Izraelskiego.
\par 17 Zebrali sie tedy synowie Ammonowi, a polozyli sie obozem w Galaad; zebrali sie tez i synowie Izraelscy, a polozyli sie obozem w Masfa.
\par 18 Tedy rzekl lud i przelozeni w Galaad miedzy soba: Kto sie naprzód pocznie potykac z syny Ammonowymi, ten bedzie hetmanem nad wszystkimi mieszkajacymi w Galaad.

\chapter{11}

\par 1 Ale Jefte Galaadczyk byl czlowiekiem bardzo meznym, a byl synem niewiasty nierzadnej, z która splodzil Galaad tegoz Jeftego.
\par 2 Ale i zona Galaadowa narodzila mu synów; a doróslszy synowie tej zony, wygnali Jeftego, mówiac mu: Nie bedziesz bral dziedzictwa w domu ojca naszego, bos ty jest synem inszej niewiasty.
\par 3 Uciekl tedy Jefte przed bracia swoja, a mieszkal w ziemi Tob: i zebrali sie do niego ludzie ogoloceni, i poszli z nim.
\par 4 I stalo sie potem, ze walczyli synowie Ammonowi z Izraelem.
\par 5 A gdy poczeli walczyc Ammonitowie z Izraelem, tedy poszli starsi z Galaad, aby wzieli Jeftego z ziemi Tob.
\par 6 I rzekli do niego: Pójdz, a badz nam za hetmana, a bedziemy walczyli przeciwko synom Ammonowym.
\par 7 Ale Jefte odpowiedzial starszym Galaad: Izazescie wy mnie nie mieli w nienawisci, i wygnaliscie mie z domu ojca mego? przeczzescie przyszli teraz do mnie, gdy ucisk przyszedl na was?
\par 8 Tedy rzekli starsi z Galaad do Jeftego: Dla tegosmy sie teraz wrócili do ciebie, abys szedl z nami, a walczyl przeciwko synom Ammonowym, a byl nam za hetmana, wszystkim mieszkajacym w Galaad.
\par 9 I odpowiedzial Jefte starszym z Galaad: Poniewaz wy mnie przywracacie, a chcecie, abym walczyl przeciwko synom Ammonowym, a jezli mi je poda Pan, bedez wam za hetmana?
\par 10 I rzekli starsi z Galaad do Jeftego: Pan bedzie swiadkiem miedzy nami, jezliz tak wedlug slowa twego nie uczynimy.
\par 11 Tedy szedl Jefte z starszymi z Galaad, i postanowil go lud hetmanem i ksiazeciem nad soba; i mówil Jefte wszystkie te slowa przed Panem w Masfa.
\par 12 Potem wyprawil Jefte posly do króla synów Ammonowych, mówiac: Co ja mam z toba, zes przyciagnal na mie, abys walczyl przeciwko ziemi mojej?
\par 13 Na co odpowiedzial król synów Ammonowych poslom Jeftego: Ze wzial Izrael ziemie moje, gdy szedl z Egiptu, od Arnon az do Jabok i az do Jordanu: przetoz teraz wróc mi ja w pokoju.
\par 14 Po wtóre jeszcze Jefte wyprawil posly do króla synów Ammonowych.
\par 15 I rzekl mu: Tak mówi Jefte: Nie wzial Izrael ziemi Moabskiej, ani ziemi synów Ammonowych.
\par 16 Ale gdy z Egiptu szedl Izrael przez puszcza az ku morzu czerwonemu, a przyszedl do Kades.
\par 17 Skad wyprawil Izrael posly do króla Edomskiego, mówiac: Prosze niech przejde przez ziemie twoje, i nie pozwolil król Edomski, takze i do króla Moabskiego poslal, i nie pozwolil; a tak zostal Izrael w Kades.
\par 18 A gdy szedl przez puszcza, obszedl ziemie Edomska, i ziemie Moabska, a przyszedl od wschodu slonca ziemi Moabskiej, i polozyli sie obozem za Arnon, a nie wszedl w granice Moabskie; bo Arnon jest granica Moabska.
\par 19 Dlategoz wyprawil Izrael posly do Sehona, króla Amorejskiego, króla w Hesebon, i rzekl mu Izrael: Niech przejde prosze przez ziemie twoje az do miejsca mego.
\par 20 Ale nie dowierzal Sehon Izraelowi, aby isc mial przez granice jego; owszem zebral Sehon wszystek lud swój, i polozyl sie obozem w Jasa, i zwiódl bitwe z Izraelem.
\par 21 I dal Pan, Bóg Izraelski, Sehona, i wszystek lud jego w rece Izraelowe, i porazil je, a posiadl Izrael wszystke ziemie Amorejczyka, który mieszkal w onej ziemi.
\par 22 A tak posiedli wszystkie granice Amorejskie od Arnon az do Jabok, a od puszczy az do Jordanu.
\par 23 Poniewaz tedy Pan, Bóg Izraelski wypedzil Amorejczyka przed ludem swym Izraelskim, przecz ty chcesz panowac nad nim?
\par 24 Azaz, coc dal posiesc Kamos, bóg twój, tego nie posiedziesz? tak, kogo Pan, Bóg nasz, wygnal przed oblicznoscia nasza, tego tez dziedzictwo posiadamy.
\par 25 Do tego czemzes ty lepszy nad Balaka syna Seferowego, króla Moabskiego? zaz sie on kiedy wadzil z Izrealem? zaz kiedy walczyl przeciwko niemu?
\par 26 Oto przez trzy sta lat mieszkal Izrael w Hesebon, i we wsiach jego, takze w Aroer, i we wsiach jego, i we wszystkich miastach, które sa przy granicy Arnon; czemuzescie ich nie odjeli przez ten czas?
\par 27 A tak nie jam tobie winien, ale ty mnie zle czynisz, ze walczysz przeciwko mnie; niech Pan, który jest Sedzia, rozsadzi dzis miedzy syny Izraelskimi i miedzy syny Ammonowymi.
\par 28 Ale nie usluchal król synów Ammonowych slów Jeftego, które wskazal do niego.
\par 29 I byl nad Jeftem duch Panski, a przeszedl przez Galaad, i przez Manase; przeszedl tez przez Masfa w Galaad, a z Masfy w Galaad ciagnal przeciw synom Ammonowym.
\par 30 Tamze uczynil Jefte slub Panu, mówiac: Jezli pewnie podasz syny Ammonowe w rece moje,
\par 31 Tedy to, cobykolwiek wyszlo ze drzwi domu mego przeciwko mnie, gdy sie wróce w pokoju od synów Ammonowych, to mówie bedzie Panu, albo ofiarowac je bede na calopalenie.
\par 32 A tak Jefte ciagnal przeciwko synom Ammonowym, aby walczyl z nimi, i podal je Pan w rece jego.
\par 33 I porazil je od Aroer az idac do Menit, dwadziescia miast, i az do równiny winnic porazka bardzo wielka, a ponizeni sa synowie Ammonowi przed syny Izraelskimi.
\par 34 A gdy sie wracal Jefte do Masfa do domu swego, oto, córka jego wyszla przeciw niemu, z bebny, i z muzyka; a ta byla jedynaczka, bo nie mial zadnego syna ani innej córki.
\par 35 I stalo sie, gdy ja ujrzal, rozdarl odzienie swoje, i rzekl: Ach, córko moja, bardzos mie ponizyla! i tys jest z tych, którzy mie frasuja, gdyzem slub uczynil Panu, a nie bede mógl odmienic.
\par 36 Któremu ona odpowiedziala: Ojcze mój, uczyniles slub Panu, uczynze ze mna tak, jakos wyrzekl usty twojemi, gdyc tylko dal Pan pomste nad nieprzyjacioly twemi, nad syny Ammonowymi.
\par 37 Nadto rzekla do ojca swego: To mi tylko uczyn: pusc mie na dwa miesiace, ze pójde a wstapie na góry i oplakiwac bede panienstwo moje, ja i towarzyszki moje.
\par 38 A on rzekl: Idz; i puscil ja na dwa miesiace. Poszla tedy ona i towarzyszki jej, a oplakiwala panienstwo swoje na górach.
\par 39 A gdy wyszly dwa miesiace, wrócila sie od ojca swego, i wypelnil nad nia slub swój, który byl uczynil; a tak ona nie poznala meza. I weszlo to w zwyczaj w Izraelu,
\par 40 Iz na kazdy rok schodzily sie córki Izraelskie aby sie rozmawialy z córka Jeftego Galaadczyka, przez cztery dni w rok.

\chapter{12}

\par 1 I zebrali sie mezowie Efraimscy, a przyszedlszy ku pólnocy, rzekli do Jeftego: Przeczzes szedl walczyc przeciwko synom Ammonowym, a nie wezwales nas, abysmy szli z toba? przetoz dom twój i ciebie spalimy ogniem.
\par 2 I rzekl Jefte do nich: Mialem nie maly spór ja i lud mój z syny Ammonowymi, i wzywalem was, a nie wybawiliscie mie z rak ich.
\par 3 A widzac, zescie mie wybawic nie chcieli, odwazylem zdrowie swoje, i ciagnalem przeciw synom Ammonowym, a podal je Pan w rece moje, i przeczzescie przyszli do mnie dnia tego, abyscie walczyli przeciwko mnie?
\par 4 A tak zebrawszy Jefte wszystkie meze z Galaad, walczyl z Efraimem; i porazili mezowie z Galaad Efraima, przeto iz mówili: Wy Galaadczycy, którzy sie bawicie miedzy Efraimitami i miedzy Manasesytami, zbiegowiescie od Efraimitów.
\par 5 I odjeli Galaadczycy brody Jordanskie Efraimowi; a gdy mówili uciekajacy z Efraimczyków: Niech przejde, tedy pytali mezowie Galaadscy: A Efratejczykies ty: A jezli rzekl: Nie.
\par 6 Tedy mu mówili: Wymówze teraz Szybolet; jezli rzekl: Sybolet, a inaczej nie mógl wymówic, tedy pojmawszy go, zabijali go u brodu Jordanskiego. I poleglo na on czas z Efraima czterdziesci i dwa tysiace.
\par 7 A tak sadzil Jefte Galaadczyk Izraela przez szesc lat; potem umarl Jefte Galaadczyk, a pogrzebion jest w jednem z miast Galaadskich.
\par 8 Potem sadzil po nim Izraela Abesan z Betlehem.
\par 9 A mial trzydziesci synów, i trzydziesci córek, które powydawal od siebie, trzydziesci zon przywiódl synom swoim zinad, i sadzil Izraela przez siedem lat.
\par 10 Umarl potem Abesan, i pogrzebion jest w Betlehem.
\par 11 A po nim sadzil Izraela Elon Zabulonczyk, i sadzil Izraela przez dziesiec lat.
\par 12 Potem umarl Elon Zabulonczyk, i pogrzebiony jest w Ajalon w ziemi Zabulon.
\par 13 A po nim sadzil Izraela Abdon, syn Hellelów, Faratonczyk.
\par 14 A ten mial czterdziesci synów, i trzydziesci wnuków, którzy jezdzili na siedemdziesieciu osletach; i sadzil Izraela przez osiem lat.
\par 15 Umarl potem Abdon, syn Hellelów, Faratonczyk, i pogrzebiony jest w Faratonie w ziemi Efraimskiej, na górze Amalekitów.

\chapter{13}

\par 1 Potem znowu synowie Izraelscy czynili zle przed oczyma Panskiemi, i podal je Pan w rece Filistynów przez czterdziesci lat.
\par 2 Tedy byl maz niektóry z Saraa, z pokolenia Dan, imieniem Manue, a zona jego byla nieplodna, i nie rodzila.
\par 3 I ukazal sie Aniol Panski onej niewiescie, a rzekl do niej: Otos teraz nieplodna, anis rodzila; ale poczniesz i porodzisz syna.
\par 4 Przetoz sie teraz strzez, abys nie pila wina, i napoju mocnego, i abys nie jadla nic nieczystego;
\par 5 Bo oto poczniesz i porodzisz syna, a brzytwa nie postoi na glowie jego, bo Nazarejczykiem Bozym bedzie to dziecie zaraz z zywota; a on pocznie wybawiac Izraela z reki Filistynów.
\par 6 Tedy przyszla niewiasta, i powiedziala to mezowi swemu, mówiac: Maz Bozy przyszedl do mnie, którego oblicze bylo jako oblicze Aniola Bozego, bardzo straszne, i nie pytalam go, skad byl, ani mi imienia swego oznajmil.
\par 7 Tylko mi rzekl: Oto, poczniesz i porodzisz syna; przetoz teraz nie pij wina, ani napoju mocnego, ani jedz co nieczystego; bo Nazarejczykiem Bozym bedzie to dziecie zaraz z zywota az do dnia smierci swojej.
\par 8 Tedy sie modlil Manue Panu, mówiac: Prosze Panie mój, maz Bozy, któregos poslal, niech przyjdzie prosze znowu do nas, a nauczy nas, co czynic mamy z dziecieciem, które sie narodzi?
\par 9 I wysluchal Bóg glos Manuego; bo przyszedl Aniol Bozy znowu do niewiasty onej, gdy siedziala na polu; ale Manue, maz jej, nie byl z nia.
\par 10 Tedy kwapiac sie ona niewiasta, biezala, i opowiedziala mezowi swemu, i rzekla mu: Oto mi sie ukazal maz on, który byl przyszedl przedtem do mnie.
\par 11 A wstawszy Manue szedl za zona swoja; a przyszedlszy do onego meza, rzekl mu: Tyzes jest ten maz, którys mówil z zona moja? A on rzekl: Jam jest.
\par 12 I rzekl Manue: Niech sie teraz spelni slowo twoje; ale cóz bedzie za obyczaj dzieciecia, i co za sprawa jego?
\par 13 I odpowiedzial Aniol Panski Manuemu: Wszystkiego, com powiedzial zonie twojej, niech sie strzeze.
\par 14 Zadnej rzeczy, która pochodzi z winnej macicy, niechaj nie je; takze wina ani napoju mocnego, niech nie pije, ani zadnej rzeczy nieczystej niech nie je, a com jej kolwiek przykazal, tego niech przestrzega.
\par 15 Tedy rzekl Manue do Aniola Panskiego: Daj sie prosze zatrzymac, a nagotujemy przed cie kozlatko z stada.
\par 16 Ale Aniol Panski odpowiedzial Manuemu: Chocbys mie zatrzymal, nie bede jadl chleba twego; ale jezli bedziesz chcial sprawic calopalenie, ofiarujze je Panu; bo nie wiedzial Manue, zeby on byl Aniol Panski.
\par 17 Tedy rzekl Manue do Aniola Panskiego: Cóz za imie twoje? abysmy, gdy sie spelni slowo twoje, uczcili cie.
\par 18 Któremu odpowiedzial Aniol Panski: Przeczze pytasz o imie moje, które jest dziwne?
\par 19 Wzial tedy Manue kozle z stada, i ofiare sniedna, i ofiarowal to na opoce Panu, i uczynil cud, a Manue i zona jego patrzyli na to.
\par 20 A gdy wstepowal plomien z oltarza ku niebu, tedy wstapil Aniol Panski w plomieniu oltarzowym, a Manue, i zona jego widzac to, upadli na twarze swe na ziemie.
\par 21 A potem nie ukazal sie wiecej Aniol Panski Manuemu, ani zonie jego; i poznal Manue, ze to byl Aniol Panski.
\par 22 I rzekl Manue do zony swojej: Koniecznie pomrzemy, bosmy Boga widzieli.
\par 23 Któremu odpowiedziala zona jego: Gdyby nas chcial Pan zabic, nie przyjalby z rak naszych calopalenia, i ofiary sniednej, aniby nam byl okazal tego wszystkiego, aniby nam na ten czas byl objawil takowych rzeczy.
\par 24 Porodzila tedy ona niewiasta syna, i nazwala imie jego Samson; i roslo dziecie, a blogoslawil mu Pan.
\par 25 I poczal go Duch Panski umacniac w obozie Dan miedzy Saraa i miedzy Estaol.

\chapter{14}

\par 1 Szedl tedy Samson do Tamnaty, a ujrzal tam niewiaste z córek Filistynskich.
\par 2 A przyszedlszy oznajmil ojcu swemu i matce swojej, mówiac; Niewiastem widzial w Tamnacie z córek Filistynskich; przetoz teraz wezmijcie mi ja za zone.
\par 3 I rzekl mu ojciec jego, i matka jego; Azaz nie masz miedzy córkami braci twych, i we wszystkim ludu moim niewiasty, ze chcesz isc a wziac sobie zone z Filistynów nieobrzezanych? Odpowiedzial Samson ojcu swemu: Te mi wezmijcie, bo sie podobala oczom moim.
\par 4 A ojciec jego i matka jego nie wiedzieli, ze to bylo od Pana; bo on przyczyny szukal na Filistyny, gdy na on czas Filistyni panowali nad Izraelem.
\par 5 Tedy szedl Samson z ojcem swym i z matka swoja do Tamnaty, a przychodzac ku winnicom Tamnaty, oto, lew mlody ryczacy zabiezal mu.
\par 6 I przypadl nan Duch Panski, a rozdarl go, jakoby rozdarl kozle, choc nic nie mial w rekach swych; i nie oznajmil ojcu swemu i matce swojej, co uczynil.
\par 7 Przyszedlszy tedy mówil z ona niewiasta, a podobala sie oczom Samsonowym.
\par 8 A wróciwszy sie po kilku dniach, aby ja pojal, zstapil, aby ogladal on scierw lwi, a oto, rój pszczól byl w scierwie lwim, i miód.
\par 9 A wziawszy go w rece swoje szedl droga i jadl, a przyszedlszy do ojca swego i do matki swojej, dal im, i jedli; ale im nie powiedzial, ze z scierwu lwiego nabral miodu.
\par 10 tedy szedl ojciec jego do onej niewiasty, i sprawil tam Samson wesele; bo tak czyniwali mlodziency.
\par 11 A gdy go ujrzeli Filistyni, wzieli trzydziesci towarzyszów, aby byli przy nim.
\par 12 Do których rzekl Samson: Zadam wam zagadke, a jezli ja zgadniecie przez siedem dni wesela i wylozycie mi ja, tedy wam dam trzydziesci przescieradel, i trzydziesci szat odmiennych.
\par 13 A jezliz mi jej nie zgadniecie, tedy wy mnie dacie trzydziesci przescieradel, i trzydziesci szat odmiennych; którzy mu odpowiedzieli: Zadaj zagadke twoje; a bedziemy jej sluchali.
\par 14 I rzekl do nich: Z pozerajacego wyszedl pokarm, a z mocnego wyszla slodkosc; i nie mogli zgadnac onej zagadki przez trzy dni.
\par 15 I rzekli dnia siódmego do zony Samsonowej: Namów meza twego, aby nam powiedzial zagadke, bysmy snac nie spalili ciebie, i domu ojca twego ogniem; na tozescie nas wezwali, abyscie posiedli majetnosc nasze, czy nie na to?
\par 16 Plakala tedy zona Samsonowa nan, mówiac: Zaprawde mie masz w nienawisci, a nie milujesz mie; zadales zagadke synom ludu mego, a nie chcesz mi jej oznajmic. I rzekl do niej: Otom jej ojcu memu i matce mojej nie oznajmil, a tobie bym mial oznajmic?
\par 17 I plakala nan przez one siedem dni, póki mieli wesele. Stalo sie tedy dnia siódmego, ze jej oznajmil, bo mu sie uprzykrzala. A ona powiedziala one zagadke synom ludu swego.
\par 18 Przetoz rzekli do niego mezowie onego miasta siódmego dnia przed zachodem slonca: Cóz slodszego nad miód, a co mocniejszego nad lwa? którym on odpowiedzial: Byscie byli nie orali jalowica moja, nie zgadlibyscie byli zagadki mojej.
\par 19 I przypadl nan Duch Panski, a szedlszy do Aszkalonu, zabil z nich trzydziesci mezów a wziawszy lupy z nich, dal szaty odmienne onym, którzy zgadli zagadke, i rozgniewawszy sie bardzo poszedl do domu ojca swego.
\par 20 I dostala sie zona Samsonowa towarzyszowi jego, z którym mial towarzystwo.

\chapter{15}

\par 1 I stalo sie po kilku dni, pod czas zniwo pszenicznego, ze nawiedzil Samson zone swoje, wziawszy kozle z stada, i mówil: Wnijde do zony mojej do komory; ale mu nie dopuscil ojciec jej wnijsc.
\par 2 Bo rzekl ojciec jej, mówiac: Mniemalem, zes ja mial w nienawisci; przetoz dalem ja towarzyszowi twemu; azaz siostra jej mlodsza nie jest cudniejsza nad nie? wezmijze ja sobie miasto niej.
\par 3 I odpowiedzial im Samson: Juz teraz nie bede winien napotem Filistynom, choc im uczynie co zlego.
\par 4 Odszedlszy tedy Samson ulapal trzy sta liszek, a nabrawszy pochodni, przywiazal ogon do ogona, i uwiazal pochodnia jedne miedzy dwoma ogonami w posrodku.
\par 5 Potem zapaliwszy ogniem pochodnie, rozpuscil je miedzy zboza Filistynskie, i popalil tak stogi jako zboza stojace, i winnice z oliwnicami.
\par 6 Tedy rzekli Filistynowie: Któz to uczynil? I odpowiedziano: Samson, ziec Tamnatczyków, przeto ze mu wzial zone jego, a dal ja towarzyszowi jego. Poszli tedy Filistynowie, i spalili ja i ojca jej ogniem.
\par 7 Którym rzekl Samson: Chociascie to uczynili, przeciec sie ja pomszcze nad wami, a potem przestane.
\par 8 A tak potlukl je okrutnie od biódr az do goleni, a odszedlszy mieszkal na wierzchu opoki Etam.
\par 9 Przyciagneli tedy Filistynowie, a polozywszy sie obozem w Juda, rozciagneli sie az do Lechy.
\par 10 Tedy rzekli mezowie Juda: Przeczzescie wyciagneli przeciwko nam? I odpowiedzieli: Przyszlismy, abysmy zwiazali Samsona, i uczynili mu, jako on nam uczynil.
\par 11 A tak wyszlo trzy tysiace mezów z Juda na wierzch opoki Etam, i mówili do Samsona: Azaz nie wiesz, ze panuja nad nami Filistynowie? I cózes nam to uczynil? I odpowiedzial im: Jako mi uczynili, i takiem im uczynil.
\par 12 I rzekli mu: Przyszlismy, abysmy cie zwiazali, i wydali w rece Filistynów; którym odpowiedzial Samson: Przysiezcie mi, ze sie na mie sami nie targniecie.
\par 13 A oni mu rzekli, mówiac: Nie, tylko zwiazawszy cie, wydamy cie w rece ich; ale cie nie zabijemy. A tak zwiazali go dwoma powrozami nowemi, i sprowadzili go z opoki.
\par 14 Który gdy przyszedl az do Lechy, tedy Filistynowie krzyczac biezeli przeciw niemu; ale Duch Panski przypadl nan, i staly sie powrozy, które byly na ramionach jego, jako nici lniane ogniem spalone, i rozerwaly sie zwiazki z rak jego.
\par 15 Tedy znalazlszy czelusc osla swieza, a wyciagnawszy po nie reke swoje, wzial ja, i zabil nia tysiac mezów.
\par 16 Zatem rzekl Samson: Czeluscia osla kupe jedne albo dwie kupy, a czeluscia osla zabilem tysiac mezów.
\par 17 A gdy przestal mówic, porzucil czelusc z reki swej, i nazwal miejsce ono Ramat Lechy.
\par 18 Zatem upragnal bardzo, i zawolal do Pana mówiac: Tys dal przez reke slugi twego to wybawienie wielkie, a teraz umre od pragnienia, albo wpadne w rece nieobrzezanców.
\par 19 A tak rozszczepil Bóg skale w Lechy, i wyszly z niej wody, i napil sie, i wrócil sie duch jego a ozyl; przetoz nazwal imie onego zródla: zródlo wzywajacego, które jest w Lechy az do dnia dzisiejszego.
\par 20 I sadzil lud Izraelski za dni Filistynów przez dwadziescia lat.

\chapter{16}

\par 1 Potem szedl Samson do Gazy, a ujrzawszy tam niewiaste nierzadna, wszedl do niej.
\par 2 I powiedziano mieszczanom w Gazie: Przyszedl tu Samson; którzy obstapiwszy go, strzegli nan cala noc w bramie miejskiej, a sprawujac sie cicho przez one cala noc, mówili: Gdy sie pocznie rozedniwac, zabijemy go.
\par 3 Ale Samson spal az do pólnocy, a wstawszy o pólnocy, ujal wrota bramy miejskiej ze dwiema podwojami, i wyrwal je z zawora, i wlozyl na ramiona swoje, a zaniósl je na wierzch góry, która byla przeciw Hebronowi.
\par 4 I stalo sie potem, ze sie rozmilowal niewiasty w dolinie Sorek, której imie Dalila.
\par 5 I przyszli do niej ksiazeta Filistynskie, i mówili jej: Oszukaj go, a wywiedz sie, w czem jest moc jego wielka, a jako bysmy go przemóc i zwiazawszy utrapic mogli? a dac kazdy z nas tysiac i sto srebrników.
\par 6 Tedy rzekla Dalila do Samsona: Powiedz mi prosze, w czem jest moc twoja wielka, a czem bys zwiazany i utrapiony byc mógl?
\par 7 I odpowiedzial jej Samson: Jezliby mie zwiazano siedmia wici surowych, które jeszcze nie uschly, tedy oslabieje, i bede jako inny czlowiek.
\par 8 I przyniosly jej ksiazeta Filistynskie siedem wici surowych, które jeszcze nie byly uschly, i zwiazala go niemi.
\par 9 A oni sie byli nan zasadzili w komorze, i rzekla mu: Filistynowie nad toba, Samsonie; ale on zerwal wici, jakoby kto zerwal nic zgrzebna, ogniem napalona; i nie poznano, w czem byla moc jego.
\par 10 Rzekla potem Dalila do Samsona: Otos mie oszukal, i sklamales przede mna; teraz powiedz mi prosze, czem by cie zwiazac?
\par 11 A on jej odpowiedzial: Jezliby mie zwiazano powrozami nowemi, których jeszcze nie uzywano, tedy oslabieje, i bede jako inny czlowiek.
\par 12 A tak wziela Dalila powrozy nowe, i zwiazala go niemi, i rzekla do niego: Filistynowie nad toba, Samsonie; (a oni sie byli nan zasadzili w komorze,)ale porwal je na ramionach swych jako nici.
\par 13 Rzekla zatem Dalila do Samsona: Pókiz ze mnie szydzic bedziesz, i klamac przede mna? powiedzze mi, czem bys mógl byc zwiazany? I powiedzial jej? Gdybys przywila siedem kedzierzy glowy mojej do walu tkackiego.
\par 14 Ona tedy przybiwszy gwozdziem do walu tkackiego rzekla do niego: Filistynowie nad toba, Samsonie; ale on ocuciwszy sie ze snu swego, wyrwal gwózdz z osnowa i z walem.
\par 15 Znowu rzekla do niego: Jakoz mówisz, miluje cie? a serce twoje nie jest ze mna. Juzes mie po trzy kroc oszukal, i nie powiedziales mi, w czem jest twoja moc wielka.
\par 16 A gdy mu sie uprzykrzala slowy swemi na kazdy dzien, i trapila go, az zemdlala dusza jego na smierc,
\par 17 Tedy jej otworzyl cale serce swoje, i powiedzial jej: Brzytwa nigdy nie postala na glowie mojej, gdyzem jest Nazarejczykiem Bozym zaraz z zywota matki mojej; gdyby mie ogolono, odejdzie ode mnie moc moja, i oslabieje, i bede jako inny czlowiek.
\par 18 Widzac tedy Dalila, ze jej otworzyl cale serce swoje, poslala i wezwala ksiazat Filistynskich, mówiac: Pójdzciez jeszcze raz, boc mi otworzyl cale serce swoje; i przyszly do niej ksiazeta Filistynskie, niosac srebro w rekach swych.
\par 19 Tedy go uspila na lonie swojem, a przyzwawszy niektórego czlowieka, dala ogolic siedem kedzierzy glowy jego; potem go jela draznic, gdy odeszla moc jego od niego,
\par 20 I rzekla: Filistynowie nad toba, Samsonie. A ocuciwszy sie ze snu swego, rzekl: Wynijde jako i pierwej, a wybije sie; a nie wiedzial, ze Pan odstapil od niego.
\par 21 Tedy pojmawszy go Filistynowie, wylupili mu oczy, i wiedli go do Gazy, zwiazawszy go dwoma miedzianymi lancuchami, i musial mlec w domu wiezniów.
\par 22 Potem poczely wlosy na glowie jego odrastac po onem goleniu.
\par 23 A ksiazeta Filistynskie zebrali sie sprawowac ofiary wielkie Dagonowi, bogu swemu, weselili sie, i mówili: Podal bóg nasz w rece nasze Samsona, nieprzyjaciela naszego.
\par 24 Którego tez ujrzawszy lud chwalili boga swego, bo mówili: Podal bóg nasz w rece nasze nieprzyjaciela naszego, a tego, który pustoszyl ziemie nasze, i który wiele z naszych pozabijal.
\par 25 I stalo sie, gdy byli dobrej mysli, ze rzekli: Zawolajcie Samsona, aby blaznowal przed nami. A tak zawolano Samsona z domu wiezniów, aby blaznowal przed nimi; i postawili go miedzy dwoma slupami.
\par 26 Zatem rzekl Samson do chlopca, który go trzymal za reke jego. Przywiedz mie, abym pomacal slupów, na których dom stoi, i podparl sie na nich.
\par 27 A dom pelen byl mezów i niewiast; tamze byly wszystkie ksiazeta Filistynskie, a na dachu okolo trzech tysiecy mezów i niewiast, którzy sie przypatrowali, gdy blaznowal Samson.
\par 28 Wzywal tedy Samson Pana, i rzekl: Panie Boze, wspomnij na mie, prosze, a zmocnij mie prosze tylko ten raz; Boze, abym sie raz pomscil obu oczu moich nad Filistynami.
\par 29 A ujawszy Samson oba slupy posrednie, na których dom stal, wsparl sie o nie, o jeden prawa reka swoja a o drugi lewa reka swoja.
\par 30 Zatem rzekl Samson: Niech umrze dusza moja z Filistynami; a gdy sie o nie mocno oparl, upadl dom na ksiazeta, i na wszystek lud, który w nim byl, i bylo umarlych, które on pobil umierajac, wiecej niz onych, które pobil za zywota swego.
\par 31 A przyszedlszy bracia jego, i wszystek dom ojca jego, wzieli go, a wróciwszy sie pogrzebli go miedzy Saraa, i miedzy Estaol, w grobie Manue, ojca jego. A on sadzil Izraela przez dwadziescia lat.

\chapter{17}

\par 1 A byl niektóry maz z góry Efraim, imieniem Michas.
\par 2 Ten rzekl do matki swojej: Tysiac i sto srebrników, którec bylo ukradziono, o któres przeklinala, i mówilas, gdym i ja slyszal, oto srebro to u mnie jest, jam je wzial. I rzekl matka jego: Blogoslawionys, synu mój, od Pana.
\par 3 A tak wrócil tysiac i sto srebrników matce swojej; i rzekla matka jego: Zaiste poswiecilam to srebro Panu z reki mojej dla ciebie, synu mój, aby uczyniono z niego ryty i lany obraz, przetoz teraz oddawam ci je.
\par 4 I wrócil ono srebro matce swojej. Tedy wziawszy matka jego dwiescie srebrników, dala je zlotnikowi; i uczynil z nich obraz ryty i lany, który byl w domu Michasowym.
\par 5 A mial ten Michas kaplice bogów, sprawil tez byl Efod i Terafim, a poswiecil rece jednego z synów swych, aby mu byl za kaplana.
\par 6 W one dni nie bylo króla w Izraelu; kazdy, co byl dobrego w oczach jego, czynil.
\par 7 I byl mlodzieniec z Betlehem Juda, które bylo w pokoleniu Juda, a ten bedac Lewita byl tam przychodniem.
\par 8 Wyszedl tedy on maz z miasta Betlehem Juda, aby mieszkal, gdzieby mu sie trafilo; i przyszedl na góre Efraim az do domu Michasowego, idac droga swoja.
\par 9 Tedy rzekl do niego Michas: Skad idziesz? I odpowiedzial mu: Jam jest Lewita z Betlehem Juda, a ide, abym mieszkal gdzieby mi sie trafilo.
\par 10 I rzekl mu Michas: Zostan u mnie, a badz mi za ojca i za kaplana, a jac dam dziesiec srebrników do roku, i dwie szaty, i pozywienie twoje; i szedl za nim on Lewita.
\par 11 I upodobalo sie Lewicie mieszkac z mezem onym; a byl przy nim on mlodzieniec jako jeden z synów jego.
\par 12 I poswiecil Michas rece Lewity, i byl mu on mlodzieniec za kaplana, i mieszkal w domu Michasowym.
\par 13 Tedy rzekl Michas: Teraz wiem, ze mi bedzie Pan blogoslawil, gdyz mam Lewite za kaplana.

\chapter{18}

\par 1 W one dni nie bylo króla w Izraelu, a tegoz czasu pokolenie Dan szukalo sobie dziedzictwa do mieszkania; albowiem nie przypadlo im bylo az do onego dnia w posrodku pokolen Izraelskich dziedzictwo.
\par 2 Przetoz wyprawili synowie Dan z pokolenia swego pieciu mezów z granic swoich, mezów walecznych z Saraa i Estaol, aby przepatrzyli ziemie i wyszpiegowali ja, i rzekli do nich: Idzciez, wyszpiegujcie ziemie; i przyszli na góre Efraim az do domu Michasowego i nocowali tam.
\par 3 A gdy byli blisko domu Michasowego, poznali glos mlodzienca Lewity, i zstapiwszy tam, rzekli mu: Którz cie tu przywiódl? a co tu czynisz? i co tu masz za sprawe?
\par 4 A on im odpowiedzial: Tak a tak postanowil ze mna Michas, i najal mie, abym u niego byl za kaplana.
\par 5 I rzekli do niego: Prosimy poradz sie Boga, abysmy wiedzieli, poszczescili sie nam ta droga nasza, która idziemy.
\par 6 I odpowiedzial im kaplan: Idzcie w pokoju; albowiem sprawuje Pan droge wasze, która idziecie.
\par 7 A tak poszedlszy onych piec mezów, przyszli do Lais, a ujrzeli lud, który w nim byl, mieszkajacy bezpiecznie wedlug zwyczaju Sydonczyków w próznowaniu i w bezpieczenstwie; bo nie byl, kto by ich trapil w onej ziemi, albo posiadal królestwo ich; nadto odleglymi byli od Sydonczyków, i zadnej sprawy z nikim nie mieli.
\par 8 Gdy sie tedy wrócili do braci swych do Saraa i do Estaol, rzekli im bracia ich: Cózescie sprawili?
\par 9 I rzekli: Wstancie, a ciagnijmy przeciwko nim; bosmy widzieli ziemie, a oto, bardzo dobra. A wy nie dbacie? NIe lenciez sie isc, a przyszedlszy osiesc te ziemie.
\par 10 Gdy wnijdziecie, przyjdziecie do ludu bezpiecznego, do ziemi przestronnej; bo ja dal Bóg w rece wasze, miejsce, kedy nie masz zadnego niedostatku wszystkich rzeczy, które sa na ziemi.
\par 11 I wyszlo stamtad z pokolenia Dan, z Saraa i z Estaol, szesc set mezów gotowych do boju.
\par 12 A idac polozyli sie obozem u Karyjatyjarym w Juda; przetoz nazwali ono miejsce obóz Danów az do dnia dzisiejszego, a jest za Karyjatyjarym.
\par 13 A ruszywszy sie stamtad na góre Efraim przyszli az do domu Michasowego;
\par 14 I mówili oni piec mezowie, którzy chodzili na szpiegi do ziemi Lais, i rzekli do braci swych: Wieciez, iz w tym domu jest Efod i Terafim, i obraz ryty i lany? przetoz teraz wiedzcie, co macie czynic.
\par 15 A zstapiwszy tam, przyszli do domu mlodzienca Lewity, w dom Michasów, i pozdrowili go w pokoju.
\par 16 Ale szesc set mezów gotowych do boju, którzy byli z synów Danowych, stali przed drzwiami.
\par 17 A tak wszedlszy tam oni piec mezów, którzy chodzili na wyszpiegowanie ziemi, wzieli obraz ryty, i Efod, i Terafim, i obraz lany; a kaplan stal przede drzwiami bramy z szescia set mezów gotowych do boju.
\par 18 A ci, którzy weszli do domu Michasowego, wzieli obraz ryty, Efod i Terafim, i obraz lany; i rzekl do nich kaplan: Cóz to czynicie?
\par 19 A oni mu odpowiedzieli: Milcz, wlóz reke twa na usta twoje, a pójdz z nami, a badz nam za ojca i za kaplana; cózci lepiej, byc kaplanem w domu meza jednego, czyli byc kaplanem pokolenia i domu Izraelskiego?
\par 20 I uradowalo sie serce kaplanowe, a wziawszy Efod i Terafim, i obraz ryty, wszedl w posrodek onego ludu.
\par 21 A oni obróciwszy sie poszli, a puscili przed soba dziatki i bydlo, i co bylo kosztowniejszego.
\par 22 A gdy byli opodal od domu Michasowego, tedy mezowie, którzy mieszkali w domach bliskich domu Michasowego, zebrawszy sie gonili syny Dan.
\par 23 I wolali za synami Dan, którzy obejrzawszy sie rzekli do Michasa: Cóz ci, zes sie tak skupil?
\par 24 I odpowiedzial: Bogi moje, którem sprawil, pobraliscie, i kaplana, a odeszliscie, i cóz wiecej miec bede? a jeszcze mówicie: Cóz ci?
\par 25 Na to mu odpowiedzieli synowie Dan: Niech nie slyszymy glosu twego za soba, by sie snac nie rzucili na was mezowie rozgniewani, a stracilbys dusze twoje i dusze domu twego.
\par 26 I poszli synowie Dan droga swoja; a widzac Michas, ze byli mozniejsi nizli on, wrócil sie i szedl do domu swego.
\par 27 Tedy oni wziawszy to, co byl sprawil Michas, i z kaplanem, którego mial, przyszli do Lais, do ludu próznujacego i bezpiecznego, i wysiekli je ostrzem miecza, a miasto spalili ogniem.
\par 28 A nie byl, kto by ich ratowal; albowiem byli daleko od Sydonu, i nie mieli zadnej sprawy z nikim, a to miasto lezalo w dolinie, która jest w Betrohob, które znowu pobudowawszy mieszkali w niem.
\par 29 I nazwali imie miasta onego Dan wedlug imienia Dana, ojca swego, który sie byl urodzil Izraelowi; a przedtem imie miasta onego bylo Lais.
\par 30 A tak postawili sobie synowie Dan obraz ryty; a Jonatan, syn Gersona Manasesowego, on i synowie jego, byli kaplanami w pokoleniu Dan az do czasu pojmania obywateli onej ziemi.
\par 31 Wystawili tedy sobie on obraz ryty, który byl uczynil Michas, po wszystkie dni, póki byl dom Bozy w Sylo.

\chapter{19}

\par 1 I stalo sie w one dni, gdy króla nie bylo w Izraelu, ze maz niektóry Lewita, mieszkajacy przy stronie góry Efraim, pojal sobie zone zaloznice z Betlehem Juda.
\par 2 A bawila sie nierzadem przy nim zaloznica jego; potem odeszla od niego do domu ojca swego, do Betlehem Juda; i byla tam u niego przez cztery miesiace.
\par 3 Wstawszy tedy maz jej, szedl za nia, aby ja ublagawszy zasie ja przywiódl, majac z soba sluge swego, i pare oslów. Tedy ona wwiodla go w dom ojca swego, którego gdy ujrzal ojciec onej dziewki, radowal sie z przyjscia jego.
\par 4 I przyjal go wdziecznie swiekier jego, ojciec dziewki onej, a mieszkal u niego, przez trzy dni, i jedli i pili i nocowali tam.
\par 5 A dnia czwartego, gdy wstali bardzo rano, wstal i on, aby odszedl. Ale rzekl ojciec onej dziewki do ziecia swego: Posil serce twoje trocha chleba, a potem pójdziecie.
\par 6 Tedy siedli i jedli oboje wespól, i napili sie. Zatem rzekl ojciec onej dziewki do meza jej: Zostan prosze, a przenocuj tu, i badz dobrej mysli.
\par 7 A gdy wstal on maz, chcac przecie isc w droge, gwaltem przymusil go swiekier jego, iz sie wróciwszy zostal tam na noc.
\par 8 Wstal potem bardzo rano dnia piatego, chcac isc; ale mówil ojciec onej dziewki: Posil prosze serce twoje; i zabawili sie, az sie dzien nachylil, a jedli oba spolu.
\par 9 Wstal tedy on maz, aby szedl sam i zaloznica jego, i sluga jego, któremu rzekl swiekier jego, ojciec onej dziewki: Oto sie juz dzien nachylil ku wieczorowi, przenocujciez tu prosze; oto schodzi dzien, przenocuje tu, a badz dobrej mysli, a jutro rano wyprawicie sie w droge swa, i pójdziesz do przybytku twego.
\par 10 Tedy on maz nie chcial zostac na noc, ale wstal i odszedl, a przyszedl az ku Jebus, (które jest Jeruzalem) majac z soba dwóch oslów z brzemiony, i zaloznice swoje.
\par 11 A gdy byli blisko Jebus, a dzien sie juz bardzo nachylil, tedy rzekl sluga do pana swego: Pójdz prosze, a wstapmy do tego miasta Jebuzejczyków, i przenocujmy w niem.
\par 12 Któremu odpowiedzial pan jego: Nie wstepujmy do miasta cudzoziemców, które nie jest z synów Izraelskich, ale idzmy az do Gabaa.
\par 13 Nadto rzekl do slugi swego: Pójdz, abysmy przyszli na jedno z tych miejsc, i przenocowali albo w Gabaa albo w Rama.
\par 14 A minawszy poszli: i zaszlo im slonce u Gabaa, które jest pokolenia Benjaminowego.
\par 15 I udali sie tam, aby wszedlszy przenocowali w Gabaa; a gdy wszedl, usiadl na ulicy w miescie, przeto ze nie byl, kto by je przyjal w dom i przenocowal.
\par 16 A oto, maz stary szedl od roboty swojej z pola w wieczór; a ten maz byl z góry Efraim, bedac przychodniem w Gabaa, ale ludzie miejsca onego byli synowie Jemini.
\par 17 Ten podnióslszy oczy swe ujrzal meza onego podróznego na ulicy miasta, i rzekl do niego starzec: Dokad idziesz, i skades przyszedl?
\par 18 Któremu on odpowiedzial: Idziemy z Betlehem Juda az ku stronie góry Efraimowej, skadem jest; bom chodzil do Betlehem Judskiego; a teraz ide do domu Panskiego, ale nie masz nikogo, coby mie przyjal w dom;
\par 19 Choc i plewy i siano mam dla oslów naszych, takze chleb i wino mam dla siebie i dla sluzebnicy twej i dla slugi, który jest ze mna, sluga twoim; nie mam niedostatku z zadnej rzeczy.
\par 20 Tedy mu rzekl on maz stary: Nie frasuj sie; czegockolwiek nie dostanie, to ja opatrze; tylko na ulicy nie zostawaj przez noc.
\par 21 Wwiódl go tedy do domu swego, i dal oslom obrok; potem umywszy nogi swoje, jedli i pili.
\par 22 A gdy rozweselili serce swoje, oto, mezowie miasta tego, mezowie niepobozni, obstapili dom, kolacac we drzwi, i rzekli do gospodarza domu onego, do meza starego, mówiac: Wywiedz meza, który wszedl w dom twój, abysmy go poznali.
\par 23 A wyszedlszy do nich on maz, gospodarz domu, rzekl im: Nie tak bracia moi: nie czyncie prosze tej zlosci, gdyz wszedl ten maz do domu mego, nie czynciez tej sprosnosci.
\par 24 Oto córka moja panna, i zaloznica jego, wywiode je zaraz, ze je obelzycie, a uczynicie z niemi, co sie wam bedzie dobrego zdalo; tylko mezowi temu nie czyncie tej zelzywosci.
\par 25 Ale nie chcieli oni mezowie sluchac glosu jego; przetoz wziawszy on maz zaloznice swoje, wywiódl ja do nich na dwór; i poznali ja, a czynili jej gwalt przez cala noc az do zaranku, a potem puscili ja, gdy wschodzila zorza.
\par 26 A przyszedlszy ona niewiasta na switaniu, upadla u drzwi domu onegoz meza, gdzie byl pan jej, az sie rozednialo.
\par 27 Potem wstawszy pan jej rano, otworzyl drzwi u domu, i wyszedl, chcac isc w droge swoje, a oto, ona niewiasta, zaloznica jego, lezala u drzwi domu, a rece jej byl na progu.
\par 28 I rzekl do niej: Wstan a pójdzmy; ale nic nie odpowiedziala. Wziawszy ja tedy na osla, wstal on maz, i szedl do miejsca swego.
\par 29 Tam przyszedlszy w dom swój, porwal miecz, a zdjawszy zaloznice swoje rozrabal ja z kosciami jej na dwanascie sztuk, i rozeslal ja po wszystkich granicach Izraelskich.
\par 30 A ktokolwiek to widzial, mówil: Nigdy sie to nie stalo, ani co takowego widziano od onego dnia, jako wyszli synowie Izraelscy z ziemi Egipskiej, az do tego dnia; uwazajciez to z pilnoscia, a radzcie i mówcie o tem.

\chapter{20}

\par 1 Wyszli tedy wszyscy synowie Izraelscy, a zgromadzilo sie wszystko pospólstwo jednomyslnie od Dan az do Beerseba, i do ziemi Galaad do Pana do Masfy.
\par 2 I staneli przedniejsi wszystkiego ludu i wszystkie pokolenia Izraelskie w zgromadzeniu ludu Bozego, cztery kroc sto tysiecy ludu pieszego, godnego do boju.
\par 3 (I uslyszeli synowie Benjamin, iz sie zebrali synowie Izraelscy w Masfa.) Rzekli tedy synowie Izraelscy: Powiedzcie, jako sie stal ten zly uczynek?
\par 4 I odpowiedzial on maz Lewita, malzonek niewiasty zabitej, i rzekl: Do Gabaa, które jest w Benjamin, przyszedlem, ja i zaloznica moja, abym tam przenocowal.
\par 5 I powstali przeciwko mnie mezowie z Gabaa, a obstapili okolo mnie dom w nocy, umysliwszy mie zabic; ale zaloznice moje tak gwalcili, az umarla.
\par 6 Wzialem tedy zaloznice moje, i rozrabalem ja na sztuki, i rozeslalem ja do wszystkich krain dziedzictwa Izraelskiego; albowiem sie dopuscili w Izraelu haniebnego i sprosnego uczynku.
\par 7 Otoscie wy wszyscy synowie Izraelscy; uwazciez to miedzy soba, a radzcie o tem.
\par 8 I powstal wszystek lud jednostajnie, mówiac: Nie pójdzie nikt do namiotu swego, ani odejdzie kto do domu swego.
\par 9 Ale teraz to uczynimy miastu Gabaa, rzuciwszy los przeciwko niemu;
\par 10 Wezmiemy dziesiec mezów ze sta w kazdem pokoleniu Izraelskiem, a sto z tysiaca, a tysiac z dziesieciu tysiecy, zeby dodawali zywnosci ludowi, który przyciagnie do Gabaa Benjamin, i pomsci sie nad nim wszystkiej sprosnosci, której sie dopuscili w Izraelu.
\par 11 A tak zebral sie wszystek lud Izraelski przeciwko miastu, zmówiwszy sie jednostajnie.
\par 12 I poslaly pokolenia Izraelskie posly do wszystkich domów synów Benjaminowych, mówiac: Co to za zly uczynek, który sie stal miedzy wami?
\par 13 Przetoz teraz wydajcie meze niepobozne, którzy sa w Gabaa, abysmy je pozabijali, a uprzatneli zle z Izraela; ale nie chcieli synowie Benjaminowi sluchac glosu braci swych, synów Izraelskich.
\par 14 Owszem zgromadzili sie synowie Benjaminowi z miast swoich do Gabaa, aby walczyli przeciw synom Izraelskim.
\par 15 I naliczono synów Benjaminowych dnia onego z miast ich dwadziescia i szesc tysiecy mezów godnych do boju, oprócz obywateli Gabaa, których naliczono siedem set mezów na wybór.
\par 16 Miedzy tym wszystkim ludem bylo siedem set mezów na wybór, którzy nie uzywali reki swej prawej, a kazdy z nich ciskajac z procy kamieniem, i wlosa nie chybial.
\par 17 Mezów zasie Izraelskich naliczono, oprócz synów Benjaminowych, cztery kroc sto tysiecy mezów walecznych, i wszystko godnych do boju.
\par 18 Wstawszy tedy szli do domu Bozego, i radzili sie Boga, a mówili synowie Izraelscy: Któz za nas pójdzie wprzód na wojne przeciw synom Benjaminowym? I odpowiedzial Pan: Juda wprzód pójdzie.
\par 19 A tak wstawszy synowie Izraelscy rano, polozyli sie obozem przeciw Gabaa.
\par 20 A wyszedlszy mezowie Izraelscy ku bitwie przeciw synom Benjaminowym, uszykowali sie mezowie Izraelscy ku potykaniu przeciw Gabaa.
\par 21 Ale wyszedlszy synowie Benjaminowi z Gabaa, porazili z Izraela dnia onego dwadziescia i dwa tysiace mezów na glowe.
\par 22 Potem pokrzepiwszy sie mezowie ludu Izraelskiego znowu sie uszykowali ku bitwie na onemze miejscu, gdzie sie byli uszykowali dnia pierwszego.
\par 23 Pierwej jednak poszli synowie Izraelscy, i plakali przed Panem az do wieczora, i pytali sie Pana, mówiac: Izali jeszcze mamy isc walczyc przeciwko synom Benjamina, brata naszego? I rzekl Pan: Idzcie przeciwko nim.
\par 24 I ruszyli sie synowie Izraelscy przeciwko synom Benjaminowym drugiego dnia.
\par 25 A wypadlszy synowie Benjaminowi przeciwko nim z Gabaa drugiego dnia, porazili synów Izraelskich znowu osiemnascie tysiecy mezów na glowe, wszystko mezów walecznych.
\par 26 Przetoz szli wszyscy synowie Izraelscy, i wszystek lud, a przyszli do domu Bozego, i placzac trwali tam przed Panem, i poscili dnia onego az do wieczora, ofiarujac calopalenia, i ofiary spokojne przed obliczem Panskiem.
\par 27 I pytali synowie Izraelscy Pana, (bo tam byla skrzynia przymierza Bozego na on czas;
\par 28 A Finees, syn Eleazara, syna Aaronowego, stal przed nia na ten czas,)mówiac: Mamyli jeszcze wynijsc na wojne przeciwko synom Benjamina, brata naszego, czyli zaniechac? I odpowiedzial Pan: Idzcie; bo jutro dam je w rece wasze.
\par 29 Tedy poczynil Izrael zasadzki przeciw Gabaa zewszad w okolo.
\par 30 A ruszywszy sie synowie Izraelscy przeciwko synom Benjaminowym dnia trzeciego, uszykowali sie przeciw Gabaa, jako pierwszy i wtóry raz.
\par 31 Wyszli tez synowie Benjaminowi przeciwko ludowi, o odsadziwszy sie od miasta, poczeli bic lud i siec, jako pierwszy i wtóry raz po drogach, (z których jedna szla do Betel, a druga do Gabaa,)i po polu, a zabili okolo trzydziestu mezów z Izraela.
\par 32 I rzekli synowie Benjaminowi: Porazeni beda od nas jako i pierwej; lecz synowie Izraelscy mówili: Uciekajmy, a uwiedzmy je od miasta, az na wielkie drogi.
\par 33 Zatem wszyscy synowie Izraelscy wstawszy z miejsca swego uszykowali sie w Baaltamar; zasadzki tez Izraelskie wyszly z miejsca swego, z lak Gabaa.
\par 34 A tak przeszlo przez Gabaa dziesiec tysiecy mezów na wybór ze wszystkiego Izraela, i byla bitwa sroga, a oni nie widzieli, ze ich nieszczescie potkac mialo.
\par 35 I porazil Pan Benjamina przed twarza Izraela, a zabili synowie Izraelscy z Benjamina dnia onego dwadziescia i piec tysiecy i sto mezów, wszystko godnych do boju.
\par 36 A widzac synowie Benjaminowi, ze byli porazeni, (bo mezowie Izraelscy ustepowali z placu przed Benjaminem, ufajac zasadzkom, które byli uczynili przeciw Gabaa;
\par 37 A ci, co byli na zasadzce, pospieszyli sie, i uderzyli na Gabaa, a wpadlszy pobili ostrzem miecza wszystkie, którzy byli w miescie.
\par 38 Albowiem znak postawiony mieli mezowie Izraelscy z onymi, co byli w zasadzce, mianowicie, ze gdyby dym wielki wypuscili z miasta.
\par 39 Tedy sie mezowie Izraelscy obrócili ku bitwie. Synowie zas Benjaminowi poczeli bic i siec, i zabili z mezów Izraelskich okolo trzydziestu mezów; bo rzekli: Zaiste porazeni sa przed nami jako i w pierwszej bitwie.
\par 40 Ale gdy plomien i dym jako slup poczal wzgóre wstepowac z miasta, tedy obejrzawszy sie synowie Benjaminowi nazad, uderzyli, a oto, ogien z miasta, wstepowal az ku niebu.)
\par 41 A iz sie mezowie Izraelscy obrócili, potrwozyli sie mezowie Benjaminowi, widzac, ze nieszczescie nastepowalo na nie.
\par 42 I uciekli przed mezami Izraelskimi droga ku puszczy; a wojsko doganialo ich, i ci, którzy wybiezeli z miast, bili je miedzy soba.
\par 43 Ogarneli tedy Benjamina, i gonili je bez przestanku, a wparli je az do Gabaa na wschód slonca.
\par 44 Poleglo tedy z Benjamina, osiemnascie tysiecy mezów, wszystko mezów duzych.
\par 45 A z tych, którzy obróciwszy sie uciekali na puszcza, na skale Remmon, lapiac je po drogach, zabili piec tysiecy mezów, a gonili je az do Giedeon, gdzie zamordowali z nich dwa tysiace mezów.
\par 46 A tak bylo wszystkich, którzy polegli z Benjamina dnia onego, dwadziescia i piec tysiecy mezów walecznych, wszystko mezów duzych.
\par 47 Tylko sie obrócilo i ucieklo na puszcze, na skale Remmon, szesc set mezów, i zostali na skale Remmon przez cztery miesiace.
\par 48 Potem mezowie Izraelscy wróciwszy sie do synów Benjaminowych, wybili je ostrzem miecza, w miescie, poczawszy od ludzi az do bydlecia, i do wszystkiego, co znalezli; przytem i wszystkie miasta, które pozostaly, popalili ogniem.

\chapter{21}

\par 1 Nadto przysiegli mezowie Izraelscy w Masfa, mówiac: Zaden z nas nie da córki swej Benjaminczykom za zone.
\par 2 A tak poszedl lud do domu Bozego, i trwali tam az do wieczora przed Bogiem, a podnióslszy glos swój, plakali placzem wielkim.
\par 3 I rzekli: O Panie, Boze Izraelski, czemuz sie to stalo w Izraelu, ze ubylo dzisiaj z Izraela jedno pokolenie?
\par 4 Tedy nazajutrz wstawszy rano lud zbudowali tam oltarz, a sprawowali calopalone i spokojne ofiary.
\par 5 Zatem rzekli synowie Izraelscy: Któz jest, co nie przyszedl do zgromadzenia ze wszystkich pokolen Izraelskich do Pana? (Bo sie byli wielka przysiega zawiazali przeciw temu, któryby nie przyszedl do Pana do Masfa, mówiac: Smiercia umrze.
\par 6 I zalowali synowie Izraelscy Benjamina, brata swego, a mówili: Wygladzone jest dzis pokolenie jedno z Izraela.
\par 7 Cóz uczynimy tym, co pozostali, aby mieli zony, gdyzesmy przysiegli przez Pana, ze im nie mamy dac córek naszych za zony?)
\par 8 Rzekli tedy: Jestze kto z pokolen Izraelskich, coby nie przyszedl do Pana do Masfa? a oto, nie przyszedl byl nikt do obozu z Jabes Galaad do zgromadzenia.
\par 9 Bo gdy liczono lud, tedy nikogo tam nie bylo z obywateli Jabes Galaad.
\par 10 I poslalo tam zgromadzenie dwanascie tysiecy mezów walecznych, rozkazujac im i mówiac: Idzcie, a pobijcie obywatele Jabes Galaad ostrzem miecza, i niewiasty i dzieci.
\par 11 A tak sobie postapicie: Kazdego mezczyzne, i kazda niewiaste, która meza uznala, zabijecie.
\par 12 Nalezli tedy z obywateli Jabes Galaad cztery sta dzieweczek, panien, które nie uznaly meza obcujac z nim, i przywiedli je do obozu do Sylo, które bylo w ziemi Chananejskiej.
\par 13 Potem poslalo wszystko zgromadzenie, a mówilo do synów Benjaminowych, którzy byli na skale Remmon, i przyzwali ich w pokoju.
\par 14 Przetoz wrócil sie Benjamin onego czasu, i dali im zony, które byli zywo zachowali z niewiast Jabes Galaad, ale im sie ich jeszcze nie dostawalo.
\par 15 A lud zalowal Benjamina, iz uczynil Pan przerwe w pokoleniach Izraelskich.
\par 16 Tedy rzekli starsi zgromadzenia tego: A z tymi drugimi cóz uczynimy, aby mieli zony, gdyz niewiasty wygladzone sa z Benjamina?
\par 17 Nadto rzekli: Dziedzictwo Benjamina pozostalym nalezy, aby nie zaginelo pokolenie z Izraela.
\par 18 A my nie mozemy im dac zon z córek naszych, (gdyz byli przysiegli synowie Izraelscy, mówiac: Przeklety, kto da zone Benjaminczykowi.)
\par 19 Potem rzekli: Oto swieto Panskie uroczyste bywa na kazdy rok w Sylo, które jest ku pólnocy od Betel, a na wschód slonca ku drodze, która idzie od Betel ku Sychem, a pod poludnia lezy ku Lebnie.
\par 20 A tak rozkazali synom Benjaminowym, mówiac: Idzcie, a zasadzcie sie w winnicach,
\par 21 A patrzajcie, gdy córki Sylo wynijda gromada do tanca; tam wyszedlszy z winnic, porwij kazdy z was sobie zone z córek Sylo, a potem idzcie do ziemi Benjamin.
\par 22 A gdy przyjda ojcowie ich, albo bracia ich skarzyc sie przed nami, tedy im rzeczemy: Zmilujcie sie nad nimi dla nas; bosmy nie wzieli dla kazdego z nich zony na wojnie, a wyscie im ich tez nie dali; przetoz nie jestescie winni.
\par 23 Tedy uczynili tak synowie Benjamin, i nabrali zon wedlug liczby swojej z onych co tancowaly, które porwawszy odeszli, i wrócili sie do dziedzictwa swego, a pobudowawszy miasta mieszkali w nich.
\par 24 A tak rozeszli sie stamtad synowie Izraelscy onego czasu, kazdy do pokolenia swego i do domu swego, a szedl stamtad kazdy do dziedzictwa swego.
\par 25 W one dni nie bylo króla w Izraelu; kazdy, co mu sie dobrego zdalo to czynil.


\end{document}