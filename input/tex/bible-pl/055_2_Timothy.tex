\begin{document}

\title{2 List do Tymoteusza}


\chapter{1}

\par 1 Pawel, Apostol Jezusa Chrystusa przez wole Boza, wedlug obietnicy zywota onego, który jest w Chrystusie Jezusie;
\par 2 Tymoteuszowi, milemu synowi, niech bedzie laska, milosierdzie i pokój od Boga Ojca i Chrystusa Jezusa, Pana naszego.
\par 3 Dziekuje Bogu, któremu sluze z przodków w czystem sumieniu, ze cie bez przestanku wspominam w prosbach moich, w nocy i we dnie,
\par 4 Zadajac cie widziec, wspominajac na twoje lzy, abym byl radoscia napelniony,
\par 5 Przywodzac sobie na pamiec one, która w tobie jest, nieobludna wiare, która pierwej mieszkala w babce twojej Loidzie i w matce twojej Eunice, a pewienem, ze i w tobie mieszka.
\par 6 Dla której przyczyny przypominam ci, abys wzniecal dar Bozy, który w tobie jest przez wlozenie rak moich.
\par 7 Albowiem nie dal nam Bóg Ducha bojazni, ale mocy i milosci, i zdrowego zmyslu.
\par 8 Przetoz nie wstydz sie za swiadectwo Pana naszego, ani za mie, wieznia jego, ale cierp zle z Ewangielija wedlug mocy Bozej.
\par 9 Który nas zbawil i powolal powolaniem swietem, nie wedlug uczynków naszych, ale wedlug postanowienia swego i laski, która nam jest dana w Chrystusie Jezusie przed czasy wiecznemi.
\par 10 A teraz objawiona jest przez okazanie sie zbawiciela naszego, Jezusa Chrystusa, który i smierc zgladzil, i zywot na jasnie wywiódl, i niesmiertelnosc przez Ewangielije,
\par 11 Której jam jest postanowiony kaznodzieja i Apostolem, i nauczycielem pogan.
\par 12 Dla której tez przyczyny te rzeczy cierpie; alec sie nie wstydze, gdyz wiem, komum uwierzyl i pewienem, iz on mocen jest tego, czego mi sie powierzyl, strzec az do onego dnia.
\par 13 Zatrzymaj wzór zdrowych slów, któres ode mnie uslyszal, w wierze i w milosci, która jest w Chrystusie Jezusie.
\par 14 Strzez dobrego pokladu przez Ducha Swietego, który w nas mieszka.
\par 15 Wiesz to, iz mie odstapili wszyscy, którzy sa w Azyi, z których jest Fygellus i Hermogenes.
\par 16 Niech da Pan milosierdzie swoje Onezyforowemu domowi, iz mie czesto ochlodzil i za lancuch mój sie nie wstydzil;
\par 17 Ale bedac w Rzymie, bardzo mie pilno szukal i znalazl.
\par 18 Niech mu Pan da, aby znalazl milosierdzie u Pana w on dzien; a ty lepiej wiesz, jako mi wiele uslugiwal w Efezie.

\chapter{2}

\par 1 Przetoz ty, synu mój! zmacniaj sie w lasce, która jest w Chrystusie Jezusie;
\par 2 A cos slyszal ode mnie przed wieloma swiadkami, tegoz sie powierz wiernym ludziom, którzy by sposobni byli i inszych nauczac.
\par 3 Przetoz ty cierp zle, jako dobry zolnierz Jezusa Chrystusa.
\par 4 Zaden, który zolnierke sluzy, nie wikle sie sprawami tego zywota, aby sie temu, od którego za zolnierza przyjety jest, podobal.
\par 5 A chocby sie tez kto potykal, nie bywa koronowany, jezliby sie przystojnie nie potykal.
\par 6 Oracz, który pracuje, ma najprzód pozytki odbierac.
\par 7 Rozumiej, co mówie, a Pan niech ci da we wszystkiem wyrozumienie.
\par 8 Pamietaj, iz Jezus Chrystus powstal z martwych, który jest z nasienia Dawidowego, wedlug Ewangielii mojej,
\par 9 W której cierpie zle, jakoby zloczynca, az do zwiazek; alec slowo Boze nie jest zwiazane.
\par 10 Przetoz wszystko znosze dla wybranych, aby i oni zbawienia dostapili, które jest w Chrystusie Jezusie, z chwala wieczna.
\par 11 Wierna jest ta mowa; albowiem jezlismy z nim umarli, z nim tez zyc bedziemy.
\par 12 Jezli cierpimy, z nim tez królowac bedziemy; jezli sie go zapieramy, i on sie nas zaprze.
\par 13 Jezlismy niewiernymi, on wiernym zostaje i zaprzec samego siebie nie moze.
\par 14 Te rzeczy przypominaj, oswiadczajac przed obliczem Panskiem, aby sie nie wdawali w spory okolo slów, co ku niczemu nie jest pozyteczne, tylko ku podwróceniu tych, którzy sluchaja.
\par 15 Staraj sie, abys sie doswiadczonym stawil Bogu robotnikiem, który by sie nie zawstydzil i który by dobrze rozbieral slowo prawdy.
\par 16 A swieckim próznomównosciom czyn wstret; albowiem postepuja ku wiekszej niepoboznosci.
\par 17 A mowa ich szerzy sie jako kancer (rak), z których jest Hymeneusz i Filetus,
\par 18 Którzy wzgledem prawdy celu uchybili, gdy powiadaja, iz sie juz stalo zmartwychwstanie i podwracaja wiare niektórych.
\par 19 A wszakze mocny stoi grunt Bozy, majac te pieczec: Zna Pan, którzy sa jego; i Niech odstapi od niesprawiedliwosci wszelki, który mianuje imie Chrystusowe.
\par 20 A w wielkim domu nie tylko sa naczynia zlote i srebrne, ale tez drewniane i gliniane, a niektóre ku uczciwosci, drugie zasie ku zelzywosci.
\par 21 Jezliby tedy kto samego siebie oczyscil od tych rzeczy, bedzie naczyniem ku uczciwosci, poswieconem i uzytecznem Panu, do wszelkiej dobrej sprawy zgotowanem.
\par 22 Chron sie tez pozadliwosci mlodzienczych, a nasladuj sprawiedliwosci, wiary, milosci, pokoju z tymi, którzy wzywaja Pana z czystego serca.
\par 23 Chron sie tez gadek glupich i nieumiejetnych, wiedzac, iz rodza zwady.
\par 24 Ale sluga Panski nie ma byc zwadliwy, lecz ma byc ukladny przeciwko wszystkim, sposobny ku nauczaniu, zlych cierpliwie znaszajacy;
\par 25 Który by w cichosci nauczal tych, którzy sie sprzeciwiaja, owaby im kiedy Bóg dal pokute ku uznaniu prawdy,
\par 26 Aby obaczywszy sie, wywiklali sie z sidla dyjabelskiego, od którego pojmani sa ku czynieniu woli jego.

\chapter{3}

\par 1 A to wiedz, iz w ostateczne dni nastana czasy trudne.
\par 2 Albowiem beda ludzie sami siebie milujacy, lakomi, chlubni, pyszni, bluzniercy, rodzicom nieposluszni, niewdzieczni, niepobozni,
\par 3 Bez przyrodzonej milosci, przymierza nie trzymajacy, potwarcy, niepowsciagliwi, nieskromni, dobrych nie milujacy,
\par 4 Zdrajcy, skwapliwi, nadeci, rozkoszy raczej milujacy niz milujacy Boga;
\par 5 Którzy maja ksztalt poboznosci, ale sie skutku jej zaparli; i tych sie chron.
\par 6 Albowiem z tych sa ci, którzy sie wrywaja w domy i pojmane wioda niewiastki grzechami obciazone, które uwodza rozmaite pozadliwosci;
\par 7 Które sie zawsze ucza, a nigdy do znajomosci prawdy przyjsc nie moga.
\par 8 A jako Jannes i Jambres sprzeciwiali sie Mojzeszowi, tak i ci sprzeciwiaja sie prawdzie, ludzie rozumu skazonego, odrzuceni z strony wiary.
\par 9 Ale nie postapia dalej; albowiem glupstwo ich jawne bedzie wszystkim, jako i onych bylo.
\par 10 Ales ty doszedl nauki mojej, sposobu zywota mego, przedsiewziecia wiary, nieskwapliwosci, milosci i cierpliwosci,
\par 11 Przesladowania, ucierpienia, które mie spotkaly w Antyjochii, w Ikonii i w Listrze, jakiem przesladowania podejmowal; a ze wszystkich wyrwal mie Pan.
\par 12 Alec i wszyscy, którzy chca poboznie zyc w Chrystusie Jezusie, przesladowani beda.
\par 13 Lecz zli ludzie i zwodziciele postapia w gorsze, jako zwodzacy tak i zwiedzeni.
\par 14 Ale ty trwaj w tem, czegos sie nauczyl i czegoc powierzono, wiedzac, od kogos sie tego nauczyl.
\par 15 A iz z dziecinstwa umiesz Pisma swiete, które cie moga uczynic madrym ku zbawieniu przez wiare, która jest w Chrystusie Jezusie.
\par 16 Wszystko Pismo od Boga jest natchnione i pozyteczne ku nauce, ku strofowaniu, ku naprawie, ku cwiczeniu, które jest w sprawiedliwosci;
\par 17 Aby czlowiek Bozy byl doskonaly, ku wszelkiej sprawie dobrej dostatecznie wycwiczony.

\chapter{4}

\par 1 Ja tedy oswiadczam sie przed Bogiem i Panem Jezusem Chrystusem, który ma sadzic zywych i umarlych w slawnem przyjsciu swojem i królestwie swojem;
\par 2 Kaz slowo Boze, nalegaj w czas albo nie w czas, strofuj, grom i napominaj ze wszelka cierpliwoscia i nauka.
\par 3 Albowiem przyjdzie czas, gdy zdrowej nauki nie scierpia, ale wedlug swoich pozadliwosci zgromadza sobie sami nauczycieli, majac swierzbiace uszy,
\par 4 A odwróca uszy od prawdy, a ku basniom je obróca.
\par 5 Ale ty badz czulym we wszystkiem, cierp zle, wykonywaj uczynek kaznodziei, uslugiwania twego zupelnie dowódz.
\par 6 Albowiem ja juz bywam ofiarowany, a czas rozwiazania mego nadchodzi.
\par 7 Dobrym bój bojowal, biegem wykonal, wiarem zachowal;
\par 8 Zatem odlozona mi jest korona sprawiedliwosci, która mi odda w on dzien Pan, sedzia sprawiedliwy, a nie tylko mnie, ale i wszystkim, którzy umilowali slawne przyjscie jego.
\par 9 Staraj sie, abys do mnie przyszedl rychlo.
\par 10 Albowiem Demas mie opuscil, umilowawszy ten swiat, i poszedl do Tesaloniki, Krescens do Galacyi, Tytus do Dalmacyi;
\par 11 Sam tylko Lukasz ze mna jest. Marka wziawszy, przywiedz ze soba; bo mi jest bardzo pozyteczny ku posludze.
\par 12 A Tychykam poslal do Efezu.
\par 13 Oponcze, któram zostawil w Troadzie u Karpusa, gdy przyjdziesz, przynies z soba i ksiegi, zwlaszcza membrany.
\par 14 Aleksander kotlarz wiele mi zlego wyrzadzil; niech mu Pan odda wedlug uczynków jego.
\par 15 Którego i ty sie strzez; albowiem sie bardzo sprzeciwil slowom naszym.
\par 16 W pierwszej obronie mojej zaden przy mnie nie stal, ale mie wszyscy opuscili; niech im to nie bedzie przyczytane.
\par 17 Ale Pan przy mnie stal i umocnil mie, aby przez mie zupelnie utwierdzone bylo kazanie, a izby je slyszeli wszyscy poganie, i bylem wyrwany z paszczeki lwiej.
\par 18 A wyrwie mie Pan z kazdego uczynku zlego i zachowa do królestwa swego niebieskiego; któremu chwala na wieki wieków. Amen.
\par 19 Pozdrów Pryszke i Akwile, i dom Onezyforowy.
\par 20 Erastus zostal w Koryncie, a Trofimam zostawil w Milecie chorego.
\par 21 Staraj sie, abys przyszedl przed zima. Pozdrawia cie Eubulus i Pudens, i Linus, i Klaudyja, i bracia wszyscy.
\par 22 Pan Jezus Chrystus niech bedzie z duchem twoim. Laska Boza niech bedzie z wami. Amen.


\end{document}