\begin{document}

\title{List do Filipian}


\chapter{1}

\par 1 Pawel i Tymoteusz, sludzy Jezusa Chrystusa, wszystkim swietym w Chrystusie Jezusie, którzy sa w miescie Filipis, z biskupami i z dyjakonami.
\par 2 Laska wam i pokój niech bedzie od Boga, Ojca naszego, i od Pana Jezusa Chrystusa:
\par 3 Dziekuje Bogu memu, ilekroc na was wspominam,
\par 4 (Zawsze w kazdej modlitwie mojej za wszystkich was z radoscia prosbe czyniac).
\par 5 Za spolecznosc wasze w Ewangielii, od pierwszego dnia az dotad;
\par 6 Pewien tego bedac, iz ten, który poczal w was dobra sprawe, dokona az do dnia Jezusa Chrystusa.
\par 7 Jakoz sprwiedliwa jest, abym ja to rozumial o was wszystkich, dlatego iz was mam w sercu mojem i w wiezieniu mojem, i w obronie, i w utwierdzeniu Ewangielii, was, mówie, wszystkich, którzy jestescie ze mna uczestnikami laski.
\par 8 Albowiem swiadkiem mi jest Bóg, jako was wszystkich pragne we wnetrznosciach Jezusa Chrystusa.
\par 9 I o to sie modle, aby milosc wasza im dalej tem wiecej pomnazala sie w znajomosci i we wszelkim zmysle,
\par 10 Abyscie mogli rozeznac rzeczy rózne, zebyscie byli szczerymi i bez obrazenia na dzien Chrystusowy,
\par 11 Bedac napelnieni owocami sprawiedliwosci, które przynosicie przez Jezusa Chrystusa ku slawie i chwale Bozej.
\par 12 A chce, bracia! abyscie wiedzieli, iz to, co sie ze mna dzieje, na wieksze pomnozenie Ewangielii wyszlo.
\par 13 Tak iz zwiazki moje dla Chrystusa rozgloszone sa po wszystkim palacu cesarskim i u wszystkich inszych.
\par 14 A wiele ich z braci w Panu serca nabywszy z moich zwiazek, smielszymi sa, bez bojazni mówic slowo.
\par 15 Wszakze niektórzy z zazdrosci i z sporu, a niektórzy tez z dobrej woli Chrystusa kaza.
\par 16 A ci, którzy z sporu Chrystusa opowiadaja nieszczerze, mniemaja, iz przydawaja ucisku zwiazkom moim;
\par 17 A którzy z milosci, wiedza, zem jest wystawiony ku obronie Ewangielii,
\par 18 Ale cóz na tem? Owszem jakimkolwiek sposobem, lub postawnie, lub w prawdzie Chrystus bywa opowiadany, i z tego sie raduje, i jeszcze sie radowac bede;
\par 19 Gdyz wiem, iz mi to wynijdzie na zbawienie przez modlitwe wasze i pomoc Ducha Jezusa Chrystusa,
\par 20 Wedlug troskliwego oczekiwania i nadziei mojej, iz sie w niczem nie zawstydze; ale z wszelakiem bezpieczenstwem, jako zawsze, tak i teraz, uwielbionym bedzie Chrystus w ciele mojem, lub przez zywot, lub przez smierc.
\par 21 Albowiem mnie zyciem jest Chrystus, a umrzec zysk.
\par 22 A jezliz zyc w ciele jest mi to owocem pracy mojej, jednak nie wiem, co bym obrac mial.
\par 23 Albowiem jestem scisniony od tego obojga, pragnac byc rozwiazany, a byc z Chrystusem, bo to daleko lepiej:
\par 24 Ale zostac w ciele potrzebniej jest dla was.
\par 25 A bedac tego pewien, wiem, iz zostane i z wami wszystkimi pomieszkam ku waszemu pomnozeniu i weselu wiary,
\par 26 Aby obfitowala chluba wasza w Chrystusie Jezusie ze mnie, gdy do was zasie przybede.
\par 27 Tylko sie tak sprawujcie, jako przystoi Ewangielii Chrystusowej, abym, lub przyjde i ogladam was, lub nie przyjde, slyszal o was, iz stoicie w jednym duchu, jednomyslnie bojujac w wierze Ewangielii.
\par 28 Ani w czem nie strachajac sie przeciwników, co onym jest pewnym znakiem zginienia, a wam zbawienia, a to od Boga;
\par 29 Gdyz wam to dane dla Chrystusa, abyscie nie tylko wen wierzyli, ale abyscie tez dla niego cierpieli,
\par 30 Tenze bój majac, jakiscie widzieli we mnie, i jaki teraz o mnie slyszycie.

\chapter{2}

\par 1 Jezli tedy macie jaka pocieche w Chrystusie, jezli jaka ucieche milosci, jezli jaka spolecznosc ducha, jezli sa jakie wnetrznosci i zlitowania w was,
\par 2 Dopelnijciez wesela mojego, abyscie jednoz rozumieli, jednostajna milosc majac, bedac jednomyslni i jednoz rozumiejacy;
\par 3 Nic nie czyniac spornie, albo przez prózna chwale, ale w pokorze jedni drugich majac za wyzszych nad sie.
\par 4 Nie upatrujcie kazdy tylko, co jest jego, ale kazdy tez, co jest drugich.
\par 5 Tego tedy badzcie o sobie rozumienia, które bylo i w Chrystusie Jezusie.
\par 6 Który, bedac w ksztalcie Bozym, nie poczytal sobie tego za drapiestwo równym byc Bogu,
\par 7 Ale wyniszczyl samego siebie, przyjawszy ksztalt niewolnika, stawszy sie podobny ludziom;
\par 8 I postawa znaleziony jako czlowiek, sam sie ponizyl, bedac poslusznym az do smierci, a to smierci krzyzowej.
\par 9 Dlatego tez Bóg nader go wywyzszyl i darowal mu imie, które jest nad wszystkie imie;
\par 10 Aby w imieniu Jezusowem wszelkie sie kolano sklanialo, tych, którzy sa na niebiesiech i tych, którzy sa na ziemi, i tych, którzy sa pod ziemia.
\par 11 A wszelki jezyk aby wyznawal, ze Jezus Chrystus jest Panem ku chwale Boga Ojca.
\par 12 Przetoz, moi mili! jakoscie zawsze posluszni byli, nie tylko w przytomnosci mojej, ale teraz daleko wiecej w niebytnosci mojej, z bojaznia i ze drzeniem zbawienie swoje sprawujcie.
\par 13 Albowiem Bóg jest, który sprawuje w was chcenie i skuteczne wykonanie wedlug upodobania swego.
\par 14 Wszystko czyncie bez szemrania i poswarków,
\par 15 Abyscie byli bez nagany i szczeremi dziatkami Bozemi, nienaganionymi w posrodku narodu zlego i przewrotnego, miedzy którymi swiecicie jako swiatla na swiecie.
\par 16 Zachowywujac slowa zywota ku chlubie mojej w dzien Chrystusowy, zem darmo nie biezal i darmo nie pracowal.
\par 17 Ale chocbym ofiarowany byl dla ofiary i uslugi wiary waszej, wesele sie i spólwesele sie ze wszystkimi wami;
\par 18 Z tegoz tedy i wy weselcie sie i spólweselcie sie ze mna.
\par 19 A mam nadzieje w Panu Jezusie, iz Tymoteusza w rychle posle do was, abym sie i ja ucieszyl, dowiedziawszy sie, co sie z wami dzieje.
\par 20 Albowiem nie mam nikogo w umysle jemu równego, który by sie uprzejmie o rzeczy wasze starac chcial;
\par 21 Bo wszyscy swoich rzeczy szukaja, a nie tych, które sa Jezusa Chrystusa.
\par 22 Ale wiecie, iz on jest doswiadczonym, a iz jako syn z ojcem ze mna sluzyl w Ewangielii.
\par 23 Mam tedy nadzieje, ze tego do was posle, skoro obacze, co sie ze mna dalej dziac bedzie;
\par 24 A mam ufnosc w Panu, ze i sam w rychle do was przyjde.
\par 25 Alem rozumial rzecza potrzebna, Epafrodyta, brata i pomocnika i spólbojownika mego, a waszego Apostola i sluge w potrzebie mojej, poslac do was,
\par 26 Poniewaz pragnal was wszystkich i bardzo sie frasowal, zescie slyszeli, iz zachorowal.
\par 27 Bo wprawdzie chorowal malo nie na smierc; ale sie Bóg zmilowal nad nim, a nie tylko nad nim, ale i nade mna, abym nie mial smutku na smutek.
\par 28 Przetoz tem ochotniej poslalem go, abyscie zasie ujrzawszy go, uweselili sie, a ja abym mial mniej smutku.
\par 29 Przyjmijciez go tedy w Panu ze wszystkiem weselem; a takich w poczciwosci miejcie;
\par 30 Boc dla dziela Chrystusowego bliskim byl smierci, odwazywszy zdrowie swoje, aby dopelnil tego, czego nie dostawalo w usludze waszej przeciwko mnie.

\chapter{3}

\par 1 Dalej mówiac, bracia moi! radujcie sie w Panu. Jednez rzeczy wam pisac mnie nie mierzi, a wam jest bezpiecznie.
\par 2 Upatrujcie psy, upatrujcie zlych robotników, upatrujcie rozerwanie.
\par 3 Albowiem my jestesmy obrzezaniem, którzy duchem sluzymy Bogu i chlubimy sie w Chrystusie Jezusie, a w ciele nie ufamy.
\par 4 Aczci i ja w ciele mam ufanie; jezli kto inszy zda sie miec ufanie w ciele, bardziej ja,
\par 5 Obrzezany bedac ósmego dnia, z narodu Izraelskiego, z pokolenia Benjaminowego, Zyd z Zydów, wedlug zakonu Faryzeusz;
\par 6 Wedlug gorliwosci przesladowca kosciola, wedlug sprawiedliwosci onej, która jest z zakonu, bedac bez przygany.
\par 7 Ale to, co mi bylo zyskiem, tom poczytal dla Chrystusa za szkode.
\par 8 Owszem i wszystko poczytam sobie za szkode dla zacnosci znajomosci Chrystusa Jezusa, Pana mojego, dla któregom wszystko utracil i mam to sobie za gnój, abym Chrystusa zyskal,
\par 9 I byl znaleziony w nim, nie majac sprawiedliwosci mojej, tej która jest z zakonu, ale te, która jest przez wiare Chrystusowa, to jest sprawiedliwosc z Boga, która jest przez wiare;
\par 10 Zebym go poznal i moc zmartwychwstania jego, i spolecznosc ucierpienia jego, przyksztaltowany bedac smierci jego,
\par 11 Owabym jakimkolwiek sposobem doszedl do powstania z martwych.
\par 12 Nie izbym juz uchwycil, albo juz doskonalym byl; ale scigam, azbym tez uchwycil to, na com tez od Chrystusa Jezusa uchwycony.
\par 13 Bracia! jac o sobie nie rozumiem, zebym juz uchwycil.
\par 14 Ale jedno czynie, ze tego, co za mna jest, zapamietywajac, a do tego sie, co przede mna jest, spieszac, bieze do kresu ku zakladowi powolania onego Bozego, które jest z góry w Chrystusie Jezusie.
\par 15 Ile tedy nas doskonalych, toz rozumiejmy; a jezli co inaczej rozumiecie, i toc wam Bóg objawi.
\par 16 Wszakze w tem, czegosmy doszli, wedlug jednegoz sznuru postepujmy i jednoz rozumiejmy.
\par 17 Badzciez wespól nasladowcami moimi, bracia! a upatrujcie tych, którzy tak chodza, jako nas za wzór macie.
\par 18 Albowiem wiele ich chodzi, o którychem wam czesto powiadal, a teraz i z placzem mówie, iz sa nieprzyjaciolmi krzyza Chrystusowego;
\par 19 Których koniec jest zatracenie, których Bóg jest brzuch, a chwala w hanbie ich, którzy sie o rzeczy ziemskie staraja.
\par 20 Alec nasza rzeczpospolita jest w niebiesiech, skad tez zbawiciela oczekujemy, Pana Jezusa Chrystusa.
\par 21 Który przemieni cialo nasze podle, aby sie podobne stalo chwalebnemu cialu jego, wedlug skutecznej mocy, która tez wszystkie rzeczy sobie podbic moze.

\chapter{4}

\par 1 Przetoz, bracia moi mili i pozadani! radosci i korono moja! tak stójcie w Panu, najmilsi moi!
\par 2 Ewodyi prosze i Syntychy prosze, aby jednegoz rozumienia byly w Panu.
\par 3 Prosze tez i cie, towarzyszu wierny! badz tym na pomoc, które w Ewangielii wespól ze mna pracowaly, i z Klemensem i z innymi pomocnikami moimi, których imiona sa w ksiegach zywota.
\par 4 Radujcie sie zawsze w Panu; znowu mówie, radujcie sie.
\par 5 Skromnosc wasza niech bedzie wiadoma wszystkim ludziom; Pan blisko jest.
\par 6 Nie troszczcie sie o zadna rzecz, ale we wszystkiem przez modlitwe i prosbe z dziekowaniem zadnosci wasze niech beda znajome u Boga.
\par 7 A pokój Bozy, który przewyzsza wszelki rozum, bedzie strzegl serc waszych i mysli waszych w Chrystusie Jezusie.
\par 8 A dalej mówiac, bracia, cokolwiek jest prawdziwego, cokolwiek poczciwego, cokolwiek sprawiedliwego, cokolwiek czystego, cokolwiek przyjemnego, cokolwiek chwalebnego, jezli która cnota i jezli która chwala, o tem przemyslajcie.
\par 9 Czegoscie sie tez nauczyli i coscie przyjeli, i slyszeli, i widzieli przy mnie, to czyncie, a Bóg pokoju bedzie z wami.
\par 10 A uradowalem sie wielce w Panu, zescie sie juz wzdy znowu zazielenili w swojem staraniu o mie, jakoz i staraliscie sie o to, lecz wam na sposobnym czasie schodzilo.
\par 11 Nie zebym to mówil dla niedostatku; bomci sie ja nauczyl, na tem przestawac, co mam.
\par 12 Umiem i unizac sie, umiem i obfitowac; wszedy i we wszystkich rzeczach jestem wycwiczony i nasyconym byc, i laknac, i obfitowac, i niedostatek cierpiec;
\par 13 Wszystko moge w Chrystusie, który mie posila.
\par 14 Wszakze dobrzescie uczynili, zescie spolecznie dogodzili uciskowi mojemu.
\par 15 A wiecie i wy Filipensowie, iz na poczatku Ewangielii, gdym wyszedl z Macedonii, zaden mi zbór nie udzielil na rachunek dawania i brania, tylko wy sami;
\par 16 Poniewaz i do Tesaloniki raz i drugi, czego potrzeba bylo, poslaliscie mi,
\par 17 Nie przeto, zebym datku szukal; ale szukam pozytku, który by obfitowal na rachunku waszym.
\par 18 Gdyzem odebral wszystko i mam dostatek, pelenem, wziawszy od Epafrodyta, co poslano od was, wonnosc dobrego zapachu, ofiare przyjemna i Bogu sie podobajaca.
\par 19 A Bóg mój napelni wszelka potrzebe wasze wedlug bogactwa swego, chwalebnie, w Chrystusie Jezusie.
\par 20 A Bogu i Ojcu naszemu niech bedzie chwala na wieki wieków. Amen.
\par 21 Pozdrówcie wszystkich swietych w Chrystusie Jezusie. Pozdrawiaja was bracia, którzy sa ze mna.
\par 22 Pozdrawiaja was wszyscy swieci; ale osobliwie, którzy sa z cesarskiego domu.
\par 23 Laska Pana naszego Jezusa Chrystusa niech bedzie z wami wszystkimi. Amen.


\end{document}