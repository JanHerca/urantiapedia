\begin{document}

\title{Pieśń nad Pieśniami}


\chapter{1}

\par 1 Piesn najprzedniejsza z piesni Salomonowych.
\par 2 Niech mie pocaluje pocalowaniem ust swoich; albowiem lepsze sa milosci twoje niz wino.
\par 3 Dla wonnosci wyborne sa masci twoje; imie twoje jest jako olejek rozlany; przetoz cie panienki umilowaly.
\par 4 Pociagnijze mie, a pobiezymy za toba. Wprowadzil mie król do pokojów swoich; przetoz sie w tobie radowac i weselic bedziemy milosci twoje raczej niz wino; bo uprzejmi miluja cie.
\par 5 Czarnamci, alem wdzieczna, o córki Jeruzalemskie! Jestem jako namioty Kedarskie, jako opony Salomonowe.
\par 6 Nie patrzajcie na mie, zem jest sniada; bo mie opalilo slonce.Synowie matki mojej rozpaliwszy sie przeciwko mnie, postanowili mie, abym strzegla winnic; a winnicy mojej, któram miala, nie strzeglam.
\par 7 Oznajmijze mi ty, którego miluje dusza moja, gdzie pasiesz? gdzie trzodzie dajesz odpoczywac w poludnie? albowiem przeczzebym miala byc jako oblakana przy trzodach towarzyszów twoich?
\par 8 Jesli nie wiesz, o najpiekniejsza miedzy niewiastami! wynijdzze sladem trzody, a pas kozlatka twoje przy budach pasterzy.
\par 9 Przyrównywam cie, o przyjaciólko moja! jezdzie w wozach Faraonowych.
\par 10 Jagody lica twego klejnotami sa ozdobione, a szyja twoja lancuchami.
\par 11 Naczynimyc klejnotów zlotych z nakrapianiem srebrnem.
\par 12 Dotad, pokad król jest u stolu, szpikanard mój wydaje wonnosc swoje.
\par 13 Jako snopek myrry jest mi mily mój na piersiach moich odpoczywajacy.
\par 14 Mily mój jest mi jako grono cyprowe na winnicach, w Engaddy.
\par 15 O jakos ty piekna, przyjaciólko moja, o jakos ty piekna! oczy twoje jako oczy golebicy.
\par 16 O jakos ty jest piekny, mily mój! i jako wdzieczny! nawet i to loze nasze zieleni sie.
\par 17 Belki domów naszych sa cedrowe, a stropy nasze jodlowe.

\chapter{2}

\par 1 Jam jest jako róza Saronska, a lilija przy dolinach.
\par 2 Jako lilija miedzy cierniem, tak przyjaciólka moja miedzy pannami.
\par 3 Jako jablon miedzy drzewem lesnem, tak mily mój miedzy mlodziencami. Pragnelam siedziec w cieniu jego, i siedze; bo owoc jego slodki jest ustom moim.
\par 4 Wprowadzil mie w dom wina, majac za choragiew milosc przeciwko mnie.
\par 5 Oczerstwijcie mie temi flaszami, posilcie mie temi jablkami; boc omdlewam od milosci.
\par 6 Lewica jego pod glowa moja, a prawica jego oblapia mie.
\par 7 Poprzysiegam was, córki Jeruzalemskie! przez sarny i lanie polne, abyscie nie budzily i nie przerywaly snu milego mego, dokad nie zechce.
\par 8 Glos milego mego! oto on idzie skaczac po tych górach, a poskakujac po tych pagórkach.
\par 9 Mily mój podobny jest sarnie, albo mlodemu jelonkowi; oto on stoi za sciana nasza, wyglada z okien, patrzy przez kraty.
\par 10 Ozwal sie mily mój, a rzekl mi: Wstan, przyjaciólko moja! piekna moja! a pójdz.
\par 11 Albowiem oto minela zima! deszcz przeszedl, i przestal.
\par 12 Kwiatki sie ukazuja na ziemi; czas spiewania przyszedl, a glos synogarlicy slychac w ziemi naszej.
\par 13 Figowe drzewo wypuscilo niedojrzale figi swoje, a macice winne rozkwitle, wonia wydaly; wstanze przyjaciólko moja, piekna moja! a pójdz.
\par 14 Golebico moja mieszkajaca w rozpadlinach skalnych, w skrytosciach przykrych! okaz mi oblicze twoje, niech uslysze glos twój; albowiem glos twój wdzieczny, a oblicze twoje pozadane.
\par 15 Polapcie nam liszki, liszki male, które psuja winnice; poniewaz winnice nasze kwitna.
\par 16 Mily mój jest mój, a jam jest jego, który pasie miedzy lilijami;
\par 17 Azby sie okazal ten dzien, a cienie przeminely. Nawróc sie, badz podobny, mily mój! sarnie albo jelonkowi mlodemu na górach Beter.

\chapter{3}

\par 1 Na lozu mojem w nocy szukalam tego, którego miluje dusza moja; szukalam go, alem go nie znalazla.
\par 2 Juz tedy wstane, a obieze miasto; po rynkach i po ulicach bede szukac tego, którego miluje dusza moja; szukalam go, alem go nie znalazla.
\par 3 Natrafili mie strózowie, którzy chodzili po miescie; i spytalem: Widzielizescie tego któego miluje dusza moja?
\par 4 A gdym maluczko odeszla od nich, zarazem znalazla tego, którego miluje dusza moja. Uchwycilam sie go, a nie puszcze go, az go wprowadze do domu matki mojej, i do pokoju rodzicielki mojej.
\par 5 Poprzysiegam was, córki Jeruzalemskie! przez sarny i lanie polne, abyscie nie budzily ani przerywaly snu milego mojego, dokadby nie zechcial.
\par 6 Któraz to jest, co wystepuje z puszczy jako sluoy dymu, okurzona bedac myrra i kadzidlem drozszem nad wszelaki proszek aptekarski?
\par 7 Oto loze Salomonowe, okolo którego stoi szescdziesiat mocarzów z mocarzów Izraelskich.
\par 8 Wszyscy ci trzymaja miecz, bedac wycwiczeni do bitwy; kazdy z nich ma swój miecz przy boku swym dla strachu nocnego.
\par 9 Palac sobie król Salomon wystawil z drzewa Libanskiego.
\par 10 Slupy jego poczynil srebrne, a poklad jego zloty, podniebienie szarlatne, a wewnatrz uslany jest miloscia córek Jerozolimskich.
\par 11 Wynijdzcie, córki Syonskie! a ogladajcie króla Salomona w koronie, która go ukoronowala matka jego w dzien zrekowin jego, i w dzien wesela serca jego.

\chapter{4}

\par 1 O jakos ty piekna, przyjaciólko moja; o jakos ty piekna! Oczy twoje jako oczy golebicy miedzy kedzierzami twemi; wlosy twoje jako trzoda kóz, które widac na górze Galaad.
\par 2 Zeby twoje jako stado owiec jednakich, gdy wychodza z kapieli, z których kazda miewa po dwojgu, a nieplodnej niemasz miedzy niemi.
\par 3 Wargi twoje jako sznur karmazynowy, a wymowa twoja wdzieczna; skronie twoje miedzy kedzierzami twemi sa jako sztuka jablka granatowego.
\par 4 Szyja twoja jako wieza Dawidowa z obronami wystawiona, w której tysiac tarczy wisi, i wszystka bron mocarzów.
\par 5 Obie piersi twoje jako dwoje blizniat sarnich, które sie pasa miedzy lilijami;
\par 6 Azby sie okazal ten dzien, a cienie przeminely, wnijde na góre myrry, i na pagórek kadzidla.
\par 7 Wszystkas ty jest piekna, przyjaciólko moja! a zmazy niemasz na tobie.
\par 8 Pójdziesz ze mna z Libanu, o oblubienico moja! ze mna z Libanu pójdziesz, a spojrzysz z wierzchu góry Amana, z wierzchu góry Sanir i Hermon, z jaskin lwich, i z gór lampartowych.
\par 9 Ujelas serce moje, siostro moja, oblubienico moja! ujelas serce moje jednem okiem twojem, i jednym lancuszkiem na szyi twojej.
\par 10 O jakoz sa ucieszne milosci twoje, siostro moja! oblubienico moja! O jako daleko zacniejsze milosci twoje, niz wino, a wonnosc masci twoich nad wszystkie rzeczy wonne!
\par 11 Plastrem miodu oplywaja wargi twoje, oblubienico moja! miód i mleko pod jezykiem twoim, a wonnosc szat twoich, jako wonnosc Libanu.
\par 12 Ogrodem zamknionym jestes, siostro moja, oblubienico moja! zródlo zamknione, zdrój zapieczetowany.
\par 13 Szczepki twoje sa sadem jablek granatowych z owocem wdziecznym cyprysu i szpikanardu;
\par 14 Szpikanardu, i szafranu, kasyi, i cynamonu, ze wszystkiemi drzewami kadzidlo przynoszacemi! myrry, i aloesu, ze wszystkiemi osobliwemi rzeczami wonnemi.
\par 15 O zródlo ogrodne, zdroju wód zywych, które plyna z Libanu!
\par 16 Powstan wietrze pólnocny, a przyjdz wietrze z poludnia, przewiej ogród mój; niech plyna wonnosci jego, niech przjdzie mily mój do ogrodu swego, a niech je rozkoszne owoce swoje.

\chapter{5}

\par 1 Przyszedlem do ogrodu mego, siostro moja, oblubienico moja! zbieram myrre moje z rzeczami wonnemi mojemi; jem plastr mój z miodem moim, pije wino moje z mlekiem mojem. Jedzciez, przyjaciele! pijcie, a pijcie dostatkiem, mili moi!
\par 2 Jac spie; ale serce moje czuje, i slyszy glos milego mego, kolaczacego i mówiacego: Otwórz mi, siostro moja, przyjaciólko moja, golebico moja, uprzejma moja! albowiem glowa moja pelna jest rosy, a kedzierze moje kropli nocnych.
\par 3 I odpowiedzialem: Zewleklam suknie moje, jakoz ja oblec mam? umylam nogi moje, jakoz je zmazac mam?
\par 4 Mily mój sciagnal reke swoje dziura, a wnetrznosci moje wzruszyly sie we mnie.
\par 5 I wstalam, abym otworzyla milemu memu, a oto z rak mych kapala myrra, a z palców moich myrra ciekaca na rekojesc zawory.
\par 6 Otworzylam milemu memu; ale mily mój juz byl odszedl i minal. Omdlalam byla na glos jego; szukalam go, alem go nie znalazla; wolalam go, ale mi sie nieozwal.
\par 7 Natrafili mie stróze, co chodza po miescie; ubili mie, zranili mie, wzieli i plaszcz mój ze mnie stróze murów.
\par 8 Poprzysiegam was, córki Jeruzalemskie! Jeslibyscie znalazly milego mego, abyscie mu powiedzialy, zem od milosci zachorowala.
\par 9 Cóz ma mily twój nad innych milych, o najpiekniejsza miedzy niewiastami? co ma mily twój nad innych milych, ze nas tak poprzysiegasz?
\par 10 Mily mój bialy i rumiany, i zacniejszy nad innych dziesiec tysiecy.
\par 11 Glowa jego jako bryla szczerego zlota; wlosy jego kedzierzawe, czarne jako kruk;
\par 12 Oczy jego jako golebicy nad strumieniami wody, jako umyte w mleku, stojace w mierze swojej;
\par 13 Policzki jego jako zagonki ziól wonnych, jako kwiatki wonnych rzeczy; wargi jego jako lilije wypuszczajace myrre ciekaca;
\par 14 Rece jego jako pierscienie zlote, osadzone drogim kamieniem, hiacyntem; brzuch jego jako glanc kosci sloniowej, safirem osadzonej;
\par 15 Golenie jego jako slupy marmurowe, postawione na podstawkach zlota wybornego; oblicze jego jako Liban, wyborne jako cedry;
\par 16 Usta jego nader slodkie, a wszystek jest pozadany. Taki ci jest mily mój, i taki przyjaciel mój, o córki Jeruzalemskie!
\par 17 Gdziez poszedl mily twój, o najpiekniejsza miedzy niewiastami? Gdzie sie obrócil mily twój? a szukac go bedziemy z toba.

\chapter{6}

\par 1 Mily mój wstapil do ogrodu swego miedzy zagonki ziól wonnych, aby pasl w ogrodach, i zeby zbieral lilije.
\par 2 Jam jest milego mego, a mily mój jest mój, który pasie miedzy lilijami.
\par 3 Pieknas ty, przyjaciólko moja! jako Tersa; pieknas, jako Jeruzalem; ogromna, jako wojsko uszykowane.
\par 4 (Odwróc oczy twoje odemnie, gdyz mie one srogim czynia). Wlosy twoje sa jako stada kóz, które wychodza z Galaad.
\par 5 Zeby twoje sa jako stado owiec, które wychodza z kapieli, z których kazda miewa po dwojgu, a nieplodnej niemasz miedzy niemi.
\par 6 Skronie twoje miedzy kedzierzami twemi sa jako sztuka jablka granatowego.
\par 7 Aczkolwiek jest szescdziesiat zon królewskich, a osmdziesiat zaloznic, a panien bez liczby:
\par 8 Wszakze jednaz jest golebica moja, uprzejma moja, jedynaczka u matki swojej, bez zmazy u rodzicielki swojej. Ujrzawszy ja córki, blogoslawiona ja nazwaly; takze i zony królewskie i zaloznice, i chwalily ja, mówiac:
\par 9 Któraz to jest, co sie pokazuje jako zorza, piekna jako miesiac, czysta jako slonce, ogromna jako wojsko uszykowane z choragwiami?
\par 10 Zstapilam do ogrodu orzechowego, abym ogladala owoce rosnace w dolinach; abym obaczyla, jezli kwitna winne macice, a wypuszczajali paczki jablonie granatowe.
\par 11 Nizem sie dowiedziala, dusza moja wsadzila mie na wóz przedniejszych z ludu mego.
\par 12 Nawróc sie, nawróc sie, o Sulamitko! nawróc sie, nawróc sie, niech na cie patrzymy. Cóz widzicie na Sulamitce? Widzimy, jakoby hufy wojenne.

\chapter{7}

\par 1 O jako piekne sa nogi twoje w trzewikach, o córko ksiazeca! Opasania biódr twoich sa jako zawieszenia, reka dobrego rzemieslnika urobione.
\par 2 Pepek twój jako czasza okragla, która nie jest bez napoju; brzuch twój jest jako bróg pszenicy osadzony lilijami.
\par 3 Obie piersi twoje sa jako dwoje blizniat mlodych sarniat.
\par 4 Szyja twoja jako wieza z kosci sloniowych; oczy twoje jako sadzawki w Hesebon podle bramy Batrabim; nos twój jako wieza na Libanie, która patrzy ku Damaszkowi.
\par 5 Glowa twoja na tobie jako Karmel, a wlosy glowy twojej jako szarlat.Król widzac cie bylby jako przywiazany na gankach swoich.
\par 6 O jakozes piekna, i jako wdzieczna, o milosci przerozkoszna!
\par 7 Ten twój wzrost podobny jest palmie, a piersi twoje gronom.
\par 8 Rzeklem: Wstapie na palme, dosiegne wierzchów jej. Niechajze mi tedy beda piersi twoje jako grona winne, a wonnosc nozdrzy twoich jako jablek wonnych;
\par 9 A usta twoje jako wino wyborne, które na prost bardzo mile plynie i sprawóje, ze mówia wargi spiacych.
\par 10 Jam jest milego mego, a do mnie jest rzadza jego.
\par 11 Przyjdz, mily mój; wyjdziemy na pole, a przenocujemy we wsiach.
\par 12 Rano wstaniemy do winnic; ogladamy, jezli kwitnie winna macica, jezli sie zawiazuja gronka, kwitnali jablka granatowe; tam ci oswiadcze milosci moje.
\par 13 Polne jabluszka wydaly wonnosc swoje, a przede drzwiami naszemi sa wszystkie owoce wdzieczne, nowe i stare, którem tobie, mily mój! zachowala.

\chapter{8}

\par 1 Obyzes byl jako bratem moim, pozywajac piersi matki mojej! abym cie znalazlszy na dworzu, pocalowala cie, a nie byla wzgardzona.
\par 2 Prowadzilabym cie, i wprowadzila do domu matki mojej, gdziebys mie uczyl; a jabym ci dala pic wino przyprawne i moszcz z jablek moich granatowych.
\par 3 Lewica jego pod glowa moja, a prawica swoja oblapia mie.
\par 4 Poprzysiegam was, córki Jeruzalemskie! abyscie nie budzily ani przerywaly snu milego mego, dokad nie zechce.
\par 5 Któraz to jest, co wystepuje z puszczy, podparlszy sie milego swego? Pod jablonia wzbudzilam cie, tam cie poczela matka twoja, tam cie poczela rodzicielka twoja.
\par 6 Przylóz mie jako pieczec na serce swoje, jako sygnet do ramienia swego! albowiem milosc mocna jest jako smierc, twarda jako grób zawistna milosc; wegle jej jako wegle ogniste i jako plomien gwaltowny.
\par 7 Wody wielkie nie moglyby zagasic tej milosci, ani rzeki zatopic; chocby kto wszystke majetnosc domu swego dal za takowa milosc, bylby pewnie wzgardzony.
\par 8 Mamy sistre maluczka, która jeszcze nie ma piersi. Cóz uczynimy z siostra nasza w dzien, którego o niej mowa bedzie?
\par 9 Jezlize jest murem, zbudujmyz na niej palac srebrny; a jezli jest drzwiami, oprawmyz ja deszczkami cedrowymi.
\par 10 Jam jest mur, a piersi moje jako wieze. Wtenczas bylam przed oczyma jego, jako ta, która znajduje pokój.
\par 11 Winnice mial Salomon w Baalhamon, która winnice najal strózom, aby kazdy przynosil za owoc jej tysiac srebrników.
\par 12 Ale winnica moja, która mam, jest przedemna. Miej sobie tysiac srebrników, Salomonie, a dwiescie ci którzy strzega owocu jej.
\par 13 O ty, która mieszkasz w ogrodach! przyjaciele slyuchaja glosu twego; ozwijze mi sie!
\par 14 Pospiesz sie, mily mój! a badz podobnym sarnie, albo mlodemu jelonkowi na górach ziól wonnych.


\end{document}