\begin{document}

\title{Habakuka}


\chapter{1}

\par 1 Brzemie, które widzial prorok Abakuk.
\par 2 Dokadze wolac bede, o Panie! a nie wysluchasz? Dokadze do ciebie przed gwaltem krzyczec bede, a nie wybawisz?
\par 3 Przeczze dopuszczasz, abym patrzyl na nieprawosc, i widzial bezprawie, i zgube, i gwalt przeciwko sobie? i przecz sie znajduje ten, który swar i niezgode roznieca?
\par 4 Dlatego naruszony bywa zakon, a prawu sie nigdy dosyc nie dzieje; albowiem niepobozny otacza sprawiedliwego, dlatego wychodzi sad przewrotny.
\par 5 Spojrzyjcie na narody a obaczcie i dziwujcie sie z zdumieniem, przeto, iz czynie nieco za dni waszych, o czem gdy wam powiadac beda, nie uwierzycie.
\par 6 Albowiem oto Ja wzbudze Chaldejczyków, naród srogi i predki, który szeroko pójdzie przez ziemie, aby posiadl mieszkania cudze.
\par 7 Straszny jest i ogromny; od niego samego wynijdzie sad jego, i wywyzszenie jego.
\par 8 Konie jego predsze beda niz lamparty, a srozsze nad wilki wieczorne; szeroko rozciagna sie jezdni jego, a jezdni jego z daleka przyjda, przyleca jako orzel spieszacy sie do zeru.
\par 9 Kazdy z nich dla lupiestwa przyjdzie; obróca twarze swoje na wschód slonca, a wiezniów zgromadza jako piasek.
\par 10 Ten i z królów szydzic bedzie, a ksiazeta beda na posmiech u niego; ten tez z kazdej twierdzy nasmiewac sie bedzie, a usypawszy waly wezmie ja.
\par 11 Tedy sie odmieni duch jego, a wystapi i przewini, myslac, ze ta moc jego jest boga jego.
\par 12 Izalis ty nie jest od wieku, Panie, Boze mój, swiety mój? myc nie pomrzemy; o Panie! postawiles go na sad; ty, o skalo nasza! na karanies go ugruntowal.
\par 13 Czyste sa oczy twoje, tak, ze na zle patrzyc i bazprawia widziec nie moga; przeczzebys mial patrzyc na czyniacych przewrotnosc? Przeczzebys mial milczec, poniewaz niezboznik pozera sprawiedliwszego nizeli sam?
\par 14 Mialzebys zaniechac ludzi jako ryb morskich, jako plazu, który nie ma pana?
\par 15 Wszystkie weda wyciaga, zagarnia je niewodem swoim, i zgromadza je do sieci swoich; dlategoz sie weseli i raduje.
\par 16 Przeto ofiaruje niewodowi swemu i kadzi sieci swojej; albowiem przez nie utyl dzial jego, a pozywienie jego hojniejsze.
\par 17 Izali dlatego bedzie zapuszczal niewód swój, a ustawicznie zabijal narody bez litosci?

\chapter{2}

\par 1 Na strazy swej stac bede, i stane na baszcie, wygladajac, abym obaczyl, co bedzie Bóg mówil, cobym mial odpowiedziec po karaniu mojem.
\par 2 Tedy mi odpowiedzial Pan, mówiac: Napisz widzenie, a napisz rzetelnie na tablicach, aby je predko czytelnik przeczytal,
\par 3 Przeto, ze jeszcze do pewnego czasu odlozone jest widzenie, które wypowie na skonczeniu jego, a nie sklamie; a jezliby na chwile odwlaczal, oczekuj nan; boc zapewne przyjdzie, a nie omieszka.
\par 4 Oto kto sobie hardzie poczyna, tego dusza nie jest szczera w nim; ale sprawiedliwy z wiary swej zyc bedzie.
\par 5 Dopieroz czlowiek opily, przewrotny i hardy nie ostoi sie w mieszkaniu swojem, który rozszerza jako pieklo dusze swoje, a jest jako smierc, która sie nie moze nasycic, chocby zgromadzil do siebie wszystkie narody, a zebral do siebie wszystkich lud zi.
\par 6 Izali ci wszyscy o nim przypowiesci nie uczynia i wykladów i gadanin o nim? mówiac: Biada temu, który rozmnaza rzeczy nie swoje, (a dokadze?) i obciaza sie gestem blotem!
\par 7 Izali nie powstana z predka, którzy cie kasac beda, i nie ocuca sie, którzy cie szarpac beda? i staniesz sie im lupem.
\par 8 Bo izes ty zlupil wiele narodów, zlupia cie tez wszystkie ostatki narodów, dla krwi ludzkiej i dla gwaltu uczynionego ziemi i miastu i wszystkim, którzy mieszkaja w niem.
\par 9 Biada temu, który lakomie szuka zysku szkaradnego domowi swemu, aby wystawil wysoko gniazdo swoje, a tak uszedl z mocy zlego!
\par 10 Uradziles hanbe domowi swemu, abys wytracil wiele narodów, a grzeszyl przeciwko duszy swojej.
\par 11 Albowiem kamien z muru wolac bedzie, i sek z drzewa wyda o tem swiadectwo.
\par 12 Biada temu, który krwia buduje miasto, a utwierdza miasta nieprawoscia!
\par 13 Azaz to nie jest od Pana zastepów, iz kolo czego ludzie pracuja, to ogien skazi, a nad czem sie narody spracowaly, to nadaremno bedzie?
\par 14 Albowiem ziemia bedzie napelniona znajomoscia chwaly Panskiej, jako morze wody napelniaja.
\par 15 Biada temu, który poi blizniego swego, przystawiajac naczynia swego, tak aby go upoil, i napatrzyl sie nagosci jego!
\par 16 Nasycisz sie hanby dla slawy; pic bedziesz i ty, a obnazony bedziesz; obróci sie do ciebie kielich prawicy Panskiej, i zwrócenie sromotne przyjdzie na slawe twoje.
\par 17 Bo cie lupiestwo Libanu okryje i spustoszenie zwierzat, które ich straszylo: dla krwi ludzkiej i dla gwaltu ziemi i miasta, i wszystkich, którzy mieszkaja w niem.
\par 18 Cóz pomoze ryty obraz, ze go wyryl rzemieslnik jego? albo odlewany obraz i nauczyciel klamstwa, ze ufa rzemieslnik w robocie swojej, czyniac balwany nieme?
\par 19 Biada temu! który mówi drewnu: Ocuc sie, a kamieniowi niememu: Obudz sie! Tenze to ma uczyc? Spojrzyj nan, powleczonyc jest zlotem i srebrem; ale w nim niemasz zgola zadnego ducha.
\par 20 Pan jest w kosciele swietobliwosci swojej; umilknij przed obliczem jego wszystka ziemio!

\chapter{3}

\par 1 Modlitwa Abakuka proroka wedlug rozmaitych piesni zlozona.
\par 2 O Panie! uslyszawszy wyrok twój uleklem sie. O Panie! zachowaj sprawe twoje w posrodku lat, i objaw ja w posrodku lat; w gniewie wspomnij na milosierdzie.
\par 3 Gdy Bóg szedl od poludnia, a Swiety z góry Faran, Sela! okryla niebiosa slawa jego, a chwaly jego ziemia pelna byla.
\par 4 Jasnosc jego byla jako swiatlosc, rogi byly na bokach jego, a tam byla skryta sila jego.
\par 5 Przed obliczem jego szedl mór, a wegle palajace szly przed nogami jego.
\par 6 Stanal i rozmierzyl ziemie, wejrzal i rozproszyl narody, skruszone sa góry wieczne, i sklonily sie pagórki dawne: drogi jego sa wieczne.
\par 7 Widzialem namioty Chusan próznosci poddane, a opony ziemi Madyjanskiej drzaly.
\par 8 Izali sie na rzeki, o Panie! izali sie na rzeki rozpalil gniew twój? Izali na morze rozgniewanie twoje, gdys jechal na koniach twoich, i na wozach twoich zbawiennych?
\par 9 Jawnie odkryty jest luk twój dla przysiegi pokoleniom wyrzeczonej, Sela!
\par 10 Rozdzieliles rzeki ziemi: widzialy cie góry i zadrzaly, powódz wód przeminela; przepasc wydala glos swój, glebokosc rece swoje podniosla.
\par 11 Slonce i miesiac zastanowil sie w mieszkaniu swojem, przy jegoz swietle lataly strzaly twe, i przy blasku lsniacej sie wlóczni twojej.
\par 12 W zagniewaniu podeptales ziemie, w zapalczywosci mlóciles pogan;
\par 13 Wyszedles na wybawienie ludu swego, na wybawienie z pomazancem twoim; przebiles glowe z domu niezboznika, odkrywszy grunt az do szyi, Sela!
\par 14 Potlukles kijmi jego glowe wsi jego, gdy sie burzyli jako wicher, aby mie rozproszyli; weselili sie, jakoby pozrec mieli ubogiego w skrytosci.
\par 15 Jechales przez morze na koniach twoich, przez gromade wód wielkich.
\par 16 Gdym to slyszal, zatrzasnal sie brzuch mój! na ten glos drzaly wargi moje, zgnilosc weszla w kosci moje, i wszystekem sie trzasl, slyszac, ze mam odpoczac w dzien utrapienia, gdy przyciagnie na ten lud nieprzyjaciel, aby go przez wojne wygladzil.
\par 17 Chocby figowe drzewo nie zakwitnelo i nie bylo urodzaju na winnicach, chocby i owoc oliwy pochybil, i role nie przynioslyby pozytku, i z owczarniby owce wybite byly, a nie byloby bydla w oborach;
\par 18 Wszakze sie ja w Panu weselic bede, rozraduje sie w Bogu zbawienia mego.
\par 19 Panujacy Pan jest sila moja, który czyni nogi moje, jako nogi lani, i po miejscach wysokich poprowadzi mie. Przedniejszemu nad spiewakami na muzycznem naczyniu mojem.


\end{document}