\begin{document}

\title{Jonasza}


\chapter{1}

\par 1 I stalo sie slowo Panskie do Jonasza, syna Amaty, mówiac:
\par 2 Wstan, idz do Niniwy miasta tego wielkiego, a wolaj przeciwko niemu; bo wstapila zlosc ich przed oblicze moje.
\par 3 Ale Jonasz wstal, aby uciekl do Tarsu od oblicza Panskiego; a przyszedlszy do Joppen, znalazl okret, który mial isc do Tarsu, a zaplaciwszy od niego wstapil nan, aby plynal z nimi do Tarsu od oblicza Panskiego.
\par 4 Ale Pan wzruszyl wiatr wielki na morzu, i powstal wicher wielki na morzu; i zdalo sie, jakoby sie okret rozbic mial.
\par 5 A zeglarze uleklszy sie wolali kazdy do boga swego, a wyrzucali do morza to, co mieli na okrecie, aby tem lzejszy byl; ale Jonasz zszedl byl na strone okretu, a polozywszy sie spal twardo.
\par 6 Tedy przystapil do niego sternik, i rzekl mu: Cóz czynisz ty, ospalcze? wstan, wolaj do Boga swego, owa snac wspomni Bóg na nas, abysmy nie zgineli.
\par 7 Tedy rzekl jeden do drugiego: Chodzcie, rzucmy losy, abysmy sie dowiedzieli, dla kogo to zle na nas przyszlo; rzucili tedy losy, i padl los na Jonasza.
\par 8 I rzekli do niego: Powiedz nam prosze, dla kogo to zle przyszlo na nas? cos za rzemiosla? skad idziesz? z którejs ziemi i z któregos narodu?
\par 9 I rzekl do nich: Jestem Hebrejczyk, a boje sie Pana, Boga niebieskiego, który stworzyl morze i ziemie.
\par 10 Tedy sie zlekli mezowie strachem wielkim; a dowiedziawszy sie mezowie oni, ze od oblicza Panskiego ucieka, (bo im byl oznajmil) rzekli do niego: Cózes to uczynil?
\par 11 Nadto rzekli do niego: Cóz z toba uczynimy, aby sie morze uspokoilo? Bo sie morze im dalej tem bardziej burzylo.
\par 12 Tedy rzekl do nich: Wezmijcie mie, a wrzuccie mie w morze, a uspokoi sie morze przed wami, gdyz ja wiem, iz dla mnie to wzruszenie wielkie na was przyszlo.
\par 13 Ale oni mezowie robili wioslami, chcac sie do brzegu dostac, wszakze nie mogli; bo sie morze im dalej tem wiecej burzylo przeciwko nim.
\par 14 Wolali tedy do Pana, mówiac: O Panie! prosimy, abysmy nie zgineli dla smierci meza tego, ani wkladaj na nas krwi niewinnej; bo ty, o Panie! jako chcesz, tak czynisz.
\par 15 Zatem wzieli Jonasza i wrzucili go w morze; i uspokoilo sie morze od wzburzenia swego.
\par 16 Bali sie tedy mezowie strachem wielkim Pana, i ofiarowali ofiare Panu, i sluby czynili.

\chapter{2}

\par 1 Lecz Pan byl nagotowal rybe wielka, zeby pozarla Jonasza; i byl Jonasz we wnetrznosciach onej ryby trzy dni i trzy nocy.
\par 2 I modlil sie Jonasz Panu, Bogu swemu, we wnetrznosciach onej ryby,
\par 3 I rzekl: Wolalem z ucisku swego do Pana, a ozwal mi sie; z glebokosci grobu wolalem, a wysluchales glos mój.
\par 4 Bos mie wrzucil w glebokosc w posrodek morza, i rzeka ogarnela mie; wszystkie nawalnosci twoje i powodzi twoje zwalily sie na mie.
\par 5 Juzem byl rzekl: Wygnanym jest od oczów twoich, wszakze jeszcze bede patrzal na kosciól twój swiety.
\par 6 Ogarnely mie wody az do duszy, przepasc mie ogarnela, rogozem obwiniona byla glowa moja.
\par 7 Zstapilem az do spodku gór, ziemia sie zaworami swemi zawarla nademna na wieki; tys jednak wywiódl z dolu zywot mój, o Panie, Boze mój!
\par 8 Gdy ustawala we mnie dusza moja, wspomnialem na Pana; modlitwa moja przyszla do ciebie, do swietego kosciola twego.
\par 9 Którzy pilnuja marnosci nikczemnych, pozbawiaja sie milosierdzia Bozego;
\par 10 Ale ja z glosem dziekczynienia ofiarowac ci bede, com slubowal, spelnie; od Pana jest obfite wybawienie.
\par 11 I rozkazal Pan onej rybie, a wyrzucila Jonasza na brzeg.

\chapter{3}

\par 1 Tedy sie stalo slowo Panskie do Jonasza powtóre, mówiac:
\par 2 Wstan, idz do Niniwy, tego miasta wielkiego, a kaz przeciwko niemu to, coc rozkazuje.
\par 3 Wstal tedy Jonasz, i poszedl do Niniwy wedlug slowa Panskiego. (A Niniwe bylo miasto bardzo wielkie na trzy dni drogi.)
\par 4 Tedy Jonasz poczal chodzic po miescie, ile mógl za jeden dzien ujsc, i wolal mówiac: Po czterdziestu dniach Niniwe bedzie wywrócone.
\par 5 I uwierzyli Niniwczycy Bogu; a zapowiedziawszy post oblekli sie w wory, od najwiekszego z nich az do najmniejszego z nich.
\par 6 Bo gdy ta rzecz przyszla do króla Niniwskiego, powstawszy z stolicy swojej zlozyl z siebie odzienie swoje, a obleklszy sie w wór, siedzial w popiele.
\par 7 I rozkazal wywolac i opowiadac w Niniwie z dekretu królewskiego, i ksiazat swoich, tak mówiac: Ludzie i bydlo, woly i owce niech nic nie ukuszaja, i niech sie nie pasa, i wody nie pija;
\par 8 Ale sie niech okryja worami ludzie i bydlo, a niech do Boga gorliwie wolaja, a niech sie odwróci kazdy od zlej drogi swojej i od lupiestwa, które jest w reku jego.
\par 9 Kto wie, jezli sie nie obróci Bóg, a nie uzali sie tego, nie odwrócili sie, mówie, od popedliwosci gniewu swego, abysmy nie zgineli.
\par 10 I widzial Bóg sprawy ich, iz sie odwrócili od zlej drogi swej i uzalil sie Bóg nad tem zlem, które rzekl, ze im mial uczynic, a nie uczynil.

\chapter{4}

\par 1 I nie podobalo sie to bardzo Jonaszowi, i rozpalil sie gniew jego.
\par 2 Przetoz sie modlil Panu, i rzekl: Prosze Panie! azazem tego nie mówil, gdym jeszcze byl w ziemi mojej? Dlategom sie pospieszyl, abym uciekl do Tarsu, gdyzem wiedzial, zes ty Bóg laskawy i litosciwy, dlugo cierpliwy i wielkiego milosierdzia, a który zalujesz zlego.
\par 3 A teraz, o Panie! prosze, odbierz dusze moje odemnie: bo mi lepiej umrzec, nizeli zyc.
\par 4 I rzekl Pan: A dobrzez to, ze sie tak gniewasz?
\par 5 Bo wyszedl byl Jonasz z miasta, i siedzial na wschód slonca przeciwko miastu; a uczyniwszy tam sobie bude, usiadl pod nia w cieniu, azby ujrzal, coby sie dzialo z onem miastem.
\par 6 A Pan Bóg byl zgotowal banie, która wyrosla nad Jonaszem, aby zaslaniala glowe jego, i zastawiala go od goraca; tedy sie Jonasz bardzo z onej bani radowal.
\par 7 Wtem nazajutrz na switaniu nagotowal Bóg robaka, który podgryzl one banie, tak, ze uschla.
\par 8 I stalo sie, gdy weszlo slonce, wzbudzil Bóg wiatr suchy od wschodu slonca, i bilo slonce na glowe Jonaszowa, tak, iz omdlewal, i zyczyl sobie smierci, mówiac: Lepiej mi umrzec, nizeli zyc.
\par 9 I rzekl Bóg do Jonasza: Dobrzez to, ze sie tak gniewasz o te banie? I rzekl: Dobrze, ze sie gniewam az na smierc.
\par 10 Tedy mu rzekl Pan: Ty zalujesz tej bani, okolo którejs nie pracowal, anis jej dal wzrost, która jednej nocy urosla, i jednej nocy zginela;
\par 11 A Jabym nie mial zalowac Niniwy, miasta tak wielkiego? w którem jest wiecej nizeli sto i dwadziescia tysiecy ludzi, którzy nie umieja rozeznac miedzy prawica swoja i lewica swoja, i bydla wiele.


\end{document}