\begin{document}

\title{Jozuego}


\chapter{1}

\par 1 I stalo sie po smierci Mojzesza, slugi Panskiego, ze mówil Pan do Jozuego syna Nunowego, slugi Mojzeszowego, i rzekl:
\par 2 Mojzesz, sluga mój, umarl; przetoz teraz wstan, przepraw sie przez ten Jordan, ty, i wszystek lud ten, do ziemi, która Ja im, synom Izraelskim, dawam.
\par 3 Kazde miejsce, po którem deptac bedzie stopa nogi waszej, dalem wam, jakom obiecal Mojzeszowi.
\par 4 Od puszczy i od Libanu tego, i az do rzeki wielkiej, rzeki Eufrates, i wszystka ziemia Hetejczyków, i az do morza wielkiego na zachód slonca, bedzie granica wasza.
\par 5 Nie ostoi sie nikt przed toba po wszystkie dni zywota twego; jakom byl z Mojzeszem, tak bede z toba, nie odstapie cie, ani cie opuszcze.
\par 6 Zmacniajze sie i meznie sobie poczynaj; albowiem ty podasz w dziedzictwo ludowi temu ziemie, o któram przysiagl ojcom ich, ze im ja dam.
\par 7 Tylko sie zmacniaj, i bardzo meznie sobie poczynaj, abys strzegl, i czynil wszystko wedlug zakonu, któryc rozkazal Mojzesz, sluga mój; nie uchylaj sie od niego ani na prawo ani na lewo, zebys sie roztropnie sprawowal we wszystkiem, do czego sie udasz.
\par 8 Niech nie odstepuja ksiegi zakonu tego od ust twoich; ale rozmyslaj w nich we dnie i w nocy, abys strzegl i czynil wszystko, co napisano w nim; albowiem na ten czas poszczescia sie drogi twoje, i na ten czas roztropnym bedziesz.
\par 9 Azazemci nie rozkazal: Zmocnij sie, i meznie sobie poczynaj, nie lekaj sie, a nie trwóz soba, albowiem z toba jest Pan, Bóg twój, we wszystkiem, do czegokolwiek sie obrócisz?
\par 10 A przetoz rozkazal Jozue przelozonym nad ludem, mówiac:
\par 11 Przejdzcie przez posrodek obozu, a rozkazcie ludowi, mówiac: Gotujcie sobie zywnosc; albowiem po trzech dniach przejdziecie przez ten Jordan, abyscie weszli, a posiedli ziemie, która Pan, Bóg wasz, dawa wam w osiadlosc.
\par 12 Rubenitom tez, i Gadytom, i polowie pokolenia Manasesowego rzekl Jozue mówiac:
\par 13 Pamietajcie na slowo, które wam rozkazal Mojzesz, sluga Panski, mówiac: Pan, Bóg wasz, sprawil wam odpoczynienie, i dal wam te ziemie;
\par 14 Zony wasze, dziatki wasze, i bydla wasze niech zostana w ziemi, która wam dal Mojzesz z tej strony Jordanu; ale wy pójdziecie zbrojni przed bracia wasza, wszyscy duzy w sile, a bedziecie ich posilkowac,
\par 15 Az odpoczynienie sprawi Pan braci waszym jako i wam, a oni posieda ziemie, która im dawa Pan, Bóg wasz; potem sie wrócicie do ziemi osiadlosci waszej, a bedziecie ja trzymac, która wam dal Mojzesz, sluga Panski, z tej strony Jordanu na wschód slonca.
\par 16 I odpowiedzieli Jozuemu, mówiac: Wszystko, cos nam rozkazal, uczynimy, a gdziekolwiek nas poslesz, pójdziemy.
\par 17 Jakosmy byli posluszni Mojzeszowi, tak posluszni bedziemy i tobie; tylko niech bedzie Pan, Bóg twój, z toba, jako byl z Mojzeszem.
\par 18 Ktobykolwiek przeciwil sie ustom twoim, a nie bylby poslusznym slowom twoim we wszystkiem, co mu rozkazesz, niechaj umrze; tylko sie zmacniaj, a meznie sobie poczynaj.

\chapter{2}

\par 1 A tak poslal Jozue, syn Nunów, z Syttim dwóch szpiegów potajemnie, mówiac: Idzcie, wypatrujcie ziemie, i Jerycho. Szli tedy i weszli do niektórej niewiasty wszetecznej, której imie Rachab, i odpoczeli tam.
\par 2 I powiedziano to królowi Jerycha, mówiac: Oto, mezowie jacys przyszli tu tej nocy z synów Izraelskich, aby przeszpiegowali te ziemie.
\par 3 Tedy poslal król Jerycha do Rachaby, mówiac: Wywiedz meze, którzy przyszli do ciebie, a weszli do domu twego; bo na przeszpiegowanie wszystkiej ziemi przyszli.
\par 4 Ale wziawszy ona niewiasta tych dwóch mezów, skryla je, i rzekla: Prawdac jest, przyszli do mnie mezowie; alem nie wiedziala, skad byli.
\par 5 A gdy brame zamykano w zmierzch, oni mezowie wyszli; i nie wiem, dokad poszli; gonciez ich co najrychlej, bo ich doscigniecie.
\par 6 A ona wwiodla je byla na dach, i tam je przykryla lnem nietartym, który byla rozstawila na dachu.
\par 7 Mezowie tedy wyslani gonili je droga ku Jordanu az do brodu; a brame zamkniono, skoro wyszli ci, którzy szli za nimi w pogon.
\par 8 A tak pierwej niz posneli, ona wstapila do nich na dach;
\par 9 I rzekla do onych mezów: Wiem, ze wam dal Pan ziemie te; bo strach wasz przypadl na nas, i oslabiali wszyscy obywatele tej ziemi przed wami.
\par 10 Bosmy slyszeli, jako wysuszyl Pan wody morza czerwonego przed wami, gdyscie wychodzili z Egiptu, i coscie uczynili dwom królom Amorejskim, którzy byli z onej strony Jordanu, Sehonowi, i Ogowi, którescie pobili.
\par 11 Co gdysmy uslyszeli, upadlo serce nasze i nie ostal sie wiecej duch w nikim przed wami; albowiem Pan, Bóg wasz, jest Bogiem na niebie wzgóre, i na ziemi nisko.
\par 12 Przetoz teraz przysiezcie mi prosze przez Pana, iz jakom ja uczynila z wami milosierdzie, takze uczyncie i wy z domem ojca mego milosierdzie, a dajcie mi znak pewny,
\par 13 Iz zachowacie zywo ojca mego i matke moje, i bracia moje, i siostry moje, i wszystko, co ich jest, a wybawicie dusze nasze od smierci.
\par 14 I odpowiedzieli jej oni mezowie: Dusza nasza bedzie za was na smierc, jezli nie wydacie tej sprawy naszej, i bedzie to, gdy nam poda Pan te ziemie, ze uczynimy z toba milosierdzie i prawde.
\par 15 I spuscila je na powrozie z okna; bo dom jej byl przy murze i ona na murze mieszkala.
\par 16 I rzekla im: Na góre idzcie, by sie snac nie spotkali z wami, którzy was gonia: i tam sie kryjcie przez trzy dni, az sie wróca, którzy was gonia, a potem pójdziecie droga wasza.
\par 17 I rzekli jej mezowie oni: Bedziemy wolni od przysiegi tej, któras nas poprzysiegla,
\par 18 Jezli, gdy wnijdziemy do ziemi, tego czerwonego sznuru nie uwiazesz u okna, po którymes nas spuscila, a ojca twego, i matki twojej, i braci twojej, i wszystkiego domu ojca twego nie zbierzeszli do siebie w dom;
\par 19 Albowiem ktobykolwiek wyszedl ze drzwi domu twego, krew jego bedzie na glowe jego, a my bedziemy bez winy; ale kazdego, ktokolwiek bedzie z toba w domu, krew jego obróci sie na glowe nasze, jezli sie go kto reke dotknie.
\par 20 Lecz jezli wydasz te sprawe nasze, tedy bedziemy wolni od przysiegi twojej, któras nas poprzysiegla.
\par 21 I odpowiedziala; Jakoscie powiedzieli, niechze tak bedzie. Tedy je wypuscila, i poszli; i uwiazala sznur czerwony w onem oknie.
\par 22 A odszedlszy przyszli na góre, i zostali tam przez trzy dni, az sie wrócili, którzy je gonili; bo ich szukali ci, którzy je gonili, po wszystkich drogach, ale nie znalezli.
\par 23 I wrócili sie oni dwaj mezowie, a zstapiwszy z góry, przeprawili sie, i przyszli do Jozuego, syna Nunowego, i powiedzieli mu wszystko, co sie z nimi dzialo;
\par 24 I mówili do Jozuego: Dal Pan w rece nasze te wszystke ziemie; bo sie strwozyli wszyscy obywatele ziemi przed twarza nasza.

\chapter{3}

\par 1 Tedy Jozue wstal bardzo rano, i ruszyli sie z Syttim, a przyszli az do Jordanu, on i wszyscy synowie Izraelscy, i tamze przenocowali, nizli sie przeprawili.
\par 2 A po trzecim dniu przeszli przelozeni przez posrodek obozu.
\par 3 I rozkazali ludowi, mówiac: Gdy ujrzycie skrzynie przymierza Pana, Boga waszego, i kaplany Lewity, niosace ja, wy tez ruszycie sie z miejsca swego, a pójdziecie za nia;
\par 4 Wszakze plac miedzy wami i miedzy nia bedzie na dwa tysiace lokci miary zwyczajnej; nie przystepujcie blisko do niej, abyscie wiedzieli droge, która isc macie; albowiem nie chodziliscie ta droga przedtem.
\par 5 Tedy rzekl Jozue do ludu: Poswieccie sie; albowiem jutro uczyni Pan miedzy wami dziwne rzeczy.
\par 6 Przytem rzekl Jozue do kaplanów, mówiac: Wezmijcie skrzynie przymierza, a idzcie przed ludem; i wzieli skrzynie przymierza, i szli przed ludem.
\par 7 I rzekl Pan do Jozuego: Dzis cie poczne wywyzszac przed oczyma wszystkiego Izraela, aby poznali, iz jakom byl z Mojzeszem, tak bede i z toba.
\par 8 Rozkazze ty kaplanom, niosacym skrzynie przymierza, i rzecz im: Gdy wnijdziecie w brzeg wód Jordanskich, w Jordanie staniecie.
\par 9 Rzekl tez Jozue do synów Izraelskich: Przystapcie sam, a sluchajcie slów Pana, Boga waszego.
\par 10 I rzekl Jozue: W tem poznacie, ze Bóg zyjacy jest w posrodku was, a iz koniecznie wypedzi przed twarza wasza Chananejczyka, i Hetejczyka, i Hewejczyka, i Ferezejczyka, i Gergezejczyka, i Amorejczyka, i Jebuzejczyka.
\par 11 Oto, skrzynia przymierza Panujacego nad wszystka ziemia pójdzie przed wami przez Jordan.
\par 12 Przetoz teraz obierzcie sobie dwanascie mezów z pokolen Izraelskich, po jednym mezu z kazdego pokolenia;
\par 13 A gdy sie zastanowia stopy nóg kaplanów, niosacych skrzynie Pana, Panujacego nad wszystka ziemia, w wodzie Jordanskiej, tedy sie wody jordanskie rozstapia, tak iz woda plynaca z góry stanie w jednej kupie.
\par 14 I stalo sie, gdy sie ruszyl lud z namiotów swych, aby sie przeprawili przez Jordan, a kaplani, niosacy skrzynie przymierza, szli przed ludem;
\par 15 A gdy przyszli niosacy skrzynie az do Jordanu, a nogi kaplanów, którzy niesli skrzynie, omoczyly sie w brzegu wód, (bo Jordan wzbiera i wylewa na wszystkie brzegi swoje, na kazdy czas zniwa.)
\par 16 Tedy sie zastanowily wody plynace z gór, a stanely w jednej kupie bardzo daleko od Adama, miasta, które jest ku stronie Sartan; a które plynely na dól do morza pustego, morza slonego, zginely i ustaly; a tak lud przeprawial sie przeciwko Jerychu.
\par 17 A kaplani, którzy niesli skrzynie przymierza Panskiego, stali na suszy w posród Jordanu porzadnie, a wszyscy Izraelczycy szli po suszy, az sie lud wszystek przeprawil przez Jordan.

\chapter{4}

\par 1 A gdy sie wszystek lud przeprawil za Jordan, (bo rzekl byl Pan do Jozuego, mówiac:
\par 2 Obierzcie sobie z ludu dwanascie mezów, po jednym mezu z kazdego pokolenia.
\par 3 I rozkazcie im, mówiac: Wezmijcie sobie stad z posrodku Jordanu, z tego miejsca, gdzie staly nogi kaplanów porzadnie, dwanascie kamieni, które z soba wynióslszy postawicie na stanowisku, gdzie bedziecie lezec przez te noc.
\par 4 Tedy wezwal Jozue dwanascie mezów, które byl wybral z synów Izraelskich, po jednym mezu z kazdego pokolenia.)
\par 5 I rzekl do nich Jozue: Idzcie przed skrzynia Pana, Boga waszego, w posrodek Jordanu, a wezmij kazdy kamien jeden na ramie swoje wedlug liczby pokolenia synów Izraelskich,
\par 6 Aby to bylo na znak miedzy wami, gdyby potem pytali synowie wasi mówiac: Co wam znaczy ten kamien?
\par 7 Tedy im powiecie, iz sie rozstapily wody w Jordanie przed skrzynia przymierza Panskiego; albowiem gdy szla przez Jordan, rozstapily sie wody Jordanskie; i bedzie ten kamien na pamiatke synom Izraelskim az na wieki.
\par 8 I uczynili tak synowie Izraelscy, jako rozkazal Jozue, i wzieli dwanascie kamieni z posród Jordanu, jako mówil Pan do Jozuego, wedlug liczby pokolenia synów Izraelskich, a zaniesli je z soba az do stanowiska, i tam je zlozyli.
\par 9 Jozue tez wystawil dwanascie kamieni w posród Jordanu, na miejscu, kedy staly nogi kaplanów, niosacych skrzynie przymierza, które tam zostaly az po dzis dzien.
\par 10 A tak kaplani niosacy skrzynie stali w posród Jordanu, az sie wypelnilo to wszystko, co byl rozkazal Pan Jozuemu mówic do ludu wedlug wszystkiego, co byl przykazal Mojzesz Jozuemu. Spieszyl sie tedy lud i przeszedl Jordan.
\par 11 I stalo sie gdy wszystek lud przeszedl, ze tez przeszla i skrzynia Panska i kaplani przed oblicznoscia ludu.
\par 12 Przeszli tez synowie Rubenowi, i synowie Gadowi, z polowa pokolenia Manasesowego, zbrojno przed syny Izraelskimi, jako im byl powiedzial Mojzesz.
\par 13 Okolo czterdziestu tysiecy ludu zbrojnego przeszlo przed Panem do boju na polu Jerycha.
\par 14 Dnia onego wywyzszyl Pan Jozuego przed oczyma wszystkiego Izraela, i bali sie go, jako sie bali Mojzesza po wszystkie dni zywota jego.
\par 15 Potem rzekl Pan do Jozuego, mówiac:
\par 16 Rozkaz kaplanom, niosacym skrzynie swiadectwa, aby wystapili z Jordanu.
\par 17 I rozkazal Jozue kaplanom, mówiac: Wystapcie z Jordanu.
\par 18 I stalo sie, gdy wystapili kaplani, niosacy skrzynie przymierza Panskiego, z posrodku Jordanu, a stanely stopy nóg kaplanów na suszy, wrócily sie wody Jordanskie na miejsce swoje, a plynely, jako przedtem, we wszystkich brzegach swoich.
\par 19 A lud, wyszedlszy z Jordanu dziesiatego dnia miesiaca pierwszego, polozyli sie obozem w Galgal ku stronie wschodniej Jerycha.
\par 20 A dwanascie onych kamieni, które wyniesli z Jordanu, postawil Jozue w Galgal.
\par 21 I rzekl do synów Izraelskich, mówiac: Co znaczy ten kamien?
\par 22 Tedy oznajmijcie synom waszym, mówiac: Po suszy przeszedl Izrael ten Jordan;
\par 23 Albowiem osuszyl Pan Bóg wody Jordanskie przed wami, azescie przeszli, jako uczynil Pan, Bóg wasz, morzu czerwonemu, które wysuszyl przed nami, azesmy przeszli;
\par 24 Aby poznali wszyscy narodowie ziemi reke Panska, ze mozna jest, zebyscie sie bali Pana, Boga waszego, po wszystkie dni.

\chapter{5}

\par 1 I stalo sie, gdy uslyszeli wszyscy królowie Amorejscy, którzy mieszkali za Jordanem ku zachodowi, i wszyscy królowie Chananejscy, którzy mieszkali nad morzem, ze wysuszyl Pan wody Jordanskie przed syny Izraelskimi, az sie przeprawili, upadlo serce ich, tak iz nie zostal wiecej w nich duch przed oblicznoscia synów Izraelskich.
\par 2 Onegoz czasu rzekl Pan do Jozuego: Uczyn sobie noze ostre, a znowu obrzez syny Izraelskie po wtóre.
\par 3 I uczynil sobie Jozue noze ostre i obrzezal syny Izraelskie na pagórku nieobrzezek.
\par 4 A tac byla przyczyna, dla czego je obrzezal Jozue: Wszystek lud, który wyszedl z Egiptu, plci meskiej, wszyscy mezowie wojenni, pomarli byli na puszczy, w drodze, gdy wyszli z Egiptu.
\par 5 Bo obrzezan byl wszystek on lud, co wyszedl; ale wszystek lud, który sie zrodzil na puszczy, w drodze po wyjsciu z Egiptu, nie byl obrzezany.
\par 6 (Albowiem przez czterdziesci lat chodzili synowie Izraelscy po puszczy, az poginal wszystek on naród mezów wojennych, którzy byli wyszli z Egiptu, którzy nie sluchali glosu Panskiego, którym przysiagl Pan, iz im nie mial okazac ziemi, o która przysiagl Pan ojcom ich, iz nam ja dac mial, ziemie oplywajaca mlekiem i miodem.)
\par 7 Ale syny ich, które wystawil na miejsca ich, te obrzezal Jozue, bo byli w nieobrzezce; bo ich nie obrzezano w drodze.
\par 8 A gdy juz wszystek lud byl obrzezany, mieszkal na miejscu swem w obozie, az sie wygoili.
\par 9 Potem rzekl Pan do Jozuego: Dzisiajm zdjal pohanbienie Egipskie z was; i nazwano imie miejsca onego Galgal, az do dnia tego.
\par 10 Tedy polozyli sie obozem synowie Izraelscy w Galgal, a obchodzili swieto przejscia czternastego dnia miesiaca w wieczór na polach Jerycha.
\par 11 I jedli z urodzajów onej ziemi nazajutrz po swiecie przejscia chleby przasne, i klosy prazone onegoz dnia.
\par 12 I przestala manna nazajutrz, gdy poczeli jesc zboza onej ziemi; i nie mieli wiecej synowie Izraelscy manny, ale jedli z urodzajów ziemi Chananejskiej onegoz roku.
\par 13 I stalo sie, gdy Jozue byl u Jerycha, ze podniósl oczu swych a ujrzal, a oto maz stal przeciwko niemu, majac miecz swój dobyty w rece swej; i przystapiwszy do niego Jozue, rzekl mu: Z naszychzes ty, czy z nieprzyjaciól naszych?
\par 14 A on rzekl: Nie; alem Ja hetman wojska Panskiego, terazem przyszedl. Tedy upadlszy Jozue obliczem swem na ziemie, poklonil sie, i rzekl mu: Cóz Pan mój mówi do slugi swego?
\par 15 I rzekl hetman wojska Panskiego do Jozuego: Zzuj obuwie twoje z nóg twoich, bo miejsce, na którem stoisz, swiete jest; i uczynil tak Jozue.

\chapter{6}

\par 1 A Jerycho bylo zamknione, i opatrzone przed synami Izraelskimi, i nikt z niego nie wychodzil, ani do niego wchodzil.
\par 2 Tedy rzekl Pan do Jozuego: Otom dal w rece twoje Jerycho, i króla jego, i mozne wojska jego.
\par 3 A tak obchodzic bedziecie miasto, wszyscy mezowie waleczni, okolo miasta chodzac raz na dzien; tak uczynicie po szesc dni.
\par 4 Przytem siedem kaplanów poniosa siedem trab z rogów baranich, przed skrzynia; a dnia siódmego obejdziecie miasto siedem kroc, a kaplani trabic beda w traby.
\par 5 A gdy przewlocznie trabic beda w traby z rogów baranich, skoro uslyszycie glos traby, wszystek lud uczyni okrzyk bardzo wielki, i upadnie mur miasta na miejscu swem, i wnijdzie lud do miasta, kazdy przeciw miejscu, gdzie stal.
\par 6 Tedy wezwawszy Jozue, syn Nunów, kaplanów, rzekl do nich: wezmijcie skrzynie przymierza, a siedem kaplanów niech niosa siedem trab z baranich rogów przed skrzynia Panska.
\par 7 Potem rzekl do ludu: Idzcie a obejdzcie miasto, a zbrojni niech ida przed skrzynia Panska.
\par 8 A gdy to Jozue ludowi powiedzial, siedem kaplanów wziawszy siedem trab z rogów baranich, szli przed skrzynia Panska, i trabili w traby, a skrzynia przymierza Panskiego szla za nimi.
\par 9 A zbrojni szli przed kaplany trabiacymi w traby; ostatek tez ludu pospolitego szedl za skrzynia, gdy idac trabiono w traby.
\par 10 A ludowi przykazal Jozue, mówiac: Nie bedziecie wolac, ani bedzie slyszan glos wasz, ani wynijdzie z ust waszych slowo, az do dnia, którego wam rzeke: Wolajcie; i uczynicie okrzyk.
\par 11 Tedy obeszla skrzynia Panska miasto w okolo raz; i wrócili sie do obozu, i zostali w obozie przez noc.
\par 12 Wstal zasie Jozue rano, a kaplani wzieli skrzynie Panska.
\par 13 A siedem kaplanów wziawszy siedem trab z rogów baranich, przed skrzynia Panska szli, idac i trabiac w traby; a zbrojni szli przed nimi, ostatek tez ludu pospolitego szedl za skrzynia Panska, gdy idac trabiono w traby.
\par 14 A tak obeszli miasto drugi raz dnia wtórego, i wrócili sie do obozu; i tak czynili po szesc dni.
\par 15 Ale dnia siódmego wstali rano na switaniu, i obeszli miasto tymze sposobem siedem kroc; tylko dnia tego obeszli miasto siedem kroc.
\par 16 I stalo sie, gdy siódmy raz obchodzili, a kaplani trabili w traby, rzekl Jozue do ludu: Krzyczciez teraz; albowiem Pan podal wam miasto.
\par 17 I niech bedzie to miasto przeklestwem Panu, ono, i wszystko co w niem jest; tylko Rachab wszetecznica zywo zostanie, ona i wszyscy, którzy z nia sa w domu, gdyz utaila poslów, któresmy byli poslali.
\par 18 A wszakze sie wy strzezcie od rzeczy przekletych, abyscie sie nie stali przeklestwem, biorac co z rzeczy przekletych, abyscie nie wprawili obozu Izraelskiego w przeklestwo, i nie zamieszali go.
\par 19 Ale wszystko srebro i zloto i naczynia miedziane i zelazne, swiete beda Panu; do skarbu Panskiego zlozone beda.
\par 20 Tedy krzyczal lud, gdy zatrabiono w traby, albowiem gdy uslyszal lud glos trab, krzyczal i lud wielkim glosem, i upadl mur na miejscu swem, i wszedl lud do miasta, kazdy przeciw miejscu, gdzie stal, i wzieli miasto;
\par 21 I wytracili wszystko, co bylo w miescie, meze i niewiasty, dzieci i starce; woly tez i owce, i osly ostrzem miecza pobili.
\par 22 Ale dwom mezom, którzy szpiegowali one ziemie, rzekl Jozue: Wnijdzcie do domu niewiasty wszetecznej, a wywiedzcie stamtad niewiaste, i wszystko, co jej jest, jakoscie jej przysiegli.
\par 23 Tedy wszedlszy mlodziency oni, co byli wyszpiegowali ziemie, Rachabe, i ojca jej, matke jej i bracia jej, i wszystko co bylo jej, i wszystke rodzine jej wywiedli, i zostawili je za obozem Izraelskim.
\par 24 Ale miasto spalili ogniem, i wszystko, co w niem bylo; tylko srebro i zloto, i naczynie miedziane, i zelazne, zlozyli do skarbu domu Panskiego.
\par 25 Rachabe takze wszetecznice, i dom ojca jej, i wszystko, co bylo jej, Jozue zywo zostawil, i mieszkala w posrodku Izraela az do terazniejszego dnia, dla tego, iz utaila poslów, które byl poslal Jozue ku przeszpiegowaniu Jerycha.
\par 26 I wydal klatwe Jozue onego czasu, mówiac: Przeklety maz przed Panem, któryby powstal a budowal to miasto Jerycho; na pierworodnym swoim zalozy je, a na najmniejszym postawi bramy jego.
\par 27 I byl Pan z Jozuem, a rozchodzila sie slawa jego po wszystkiej ziemi.

\chapter{7}

\par 1 Ale zgrzeszyli synowie Izraelscy przestepstwem przy rzeczach przekletych; albowiem Achan, syn Charmiego, syna Zabdy, syna Zare, z pokolenia Juda, wzial nieco z rzeczy przekletych; zaczem zapalil sie gniew Panski przeciw synom Izraelskim.
\par 2 Bo gdy poslal Jozue kilka mezów z Jerycha do Haj, które bylo blisko Betawen na wschód slonca od Betel, i rzekl do nich, mówiac: Idzcie, a wyszpiegujcie ziemie; tedy szedlszy oni mezowie, wyszpiegowali Haj.
\par 3 A wróciwszy sie do Jozuego, rzekli mu: Niech nie ciagnie wszystek lud; okolo dwóch tysiecy mezów, albo okolo trzech tysiecy mezów niech ida, a zburza Haj; nie trudz tam wszystkiego ludu, bo ich tam trocha.
\par 4 Poszlo tedy okolo trzech tysiecy mezów z ludu, i uciekli przed mezami z Haj.
\par 5 A porazili z nich mezowie z Haj okolo trzydziestu i szesciu mezów, goniac je od bramy az do Sabarym, a porazili je, gdy uciekali z góry, i dla tego rozplynelo sie serce ludu, i bylo jako woda.
\par 6 Tedy rozdarlszy Jozue odzienie swoje, upadl twarza swoja na ziemie przed skrzynia Panska, a lezal az do wieczora, on i starsi Izraelscy, posypawszy prochem glowy swoje.
\par 7 Zatem rzekl Jozue: Ach! Panie Panujacy, przeczzes przeprowadzil lud ten za Jordan, abys nas podal w reke Amorejczyka na wytracenie? O bysmy byli raczej mieszkali za Jordanem!
\par 8 O Panie, cóz rzeke, poniewaz podawa Izrael tyl nieprzyjaciolom swoim?
\par 9 Bo uslyszawszy Chananejczycy, i wszyscy obywatele tej ziemi, obtocza nas zewszad, a wytraca imie nasze z ziemi. I cóz to uczynisz imieniowi twemu wielkiemu?
\par 10 Tedy rzekl Pan do Jozuego: Wstan; przeczzes upadl na oblicze twoje?
\par 11 Zgrzeszyl Izrael, i przestapili przymierze moje, którem im przykazal; albowiem wzieli z rzeczy przekletych, a ukradli je, i sklamali, i schowali je miedzy naczynie swoje.
\par 12 A dla tegoc synowie Izraelscy nie beda sie mogli ostac przed nieprzyjacioly swymi, tyl beda podawali nieprzyjaciolom swym, bo sie zmazali rzecza przekleta; nie bede wiecej z wami, jezli nie wykorzenicie przeklestwa tego z posrodku was.
\par 13 Wstan, poswiec lud i rzeczy: Poswieccie sie na jutro; bo tak mówi Pan, Bóg Izraelski: Przeklestwo jest w posrodku ciebie, Izraelu; nie ostoisz sie przed nieprzyjacioly twymi, az odejmiecie przeklestwo z posrodku siebie.
\par 14 A tak przystapcie rano wedlug pokolen waszych; a pokolenie, które okaze Pan, przystapi wedlug familii; a familija, która okaze Pan, przystapi wedlug domów; a dom, który okaze Pan, przystapi wedlug osób.
\par 15 A kto bedzie znaleziony w przeklestwie, bedzie spalony ogniem, on, i wszystko, co jego jest, dla tego ze przestapil przymierze Panskie, a dopuscil sie niegodnej rzeczy w Izraelu.
\par 16 Przetoz wstawszy Jozue rano, rozkazal przystepowac Izraelowi wedlug pokolen ich; i znalazlo sie pokolenie Juda.
\par 17 I kazal przystapic familii Juda, i znalazla sie familija Zare, i kazal przystapic familii Zarego wedlug osób, i znalazl sie dom Zabdy.
\par 18 I kazal przystapic domowi jego wedlug osób, i znalazl sie Achan, syn Charmiego, syna Zabdy, syna Zare, z pokolenia Juda.
\par 19 I rzekl Jozue do Achana: Synu mój, daj prosze chwale Panu, Bogu Izraelskiemu, i wyznaj przed nim, a oznajmij mi prosze, cos uczynil, nie taj przede mna.
\par 20 Tedy odpowiedzial Achan Jozuemu, mówiac: Zaprawde, jam zgrzeszyl Panu, Bogu Izraelskiemu, tak a tak uczynilem.
\par 21 Widzialem miedzy lupy plaszcz babilonski jeden, piekny, i dwiescie syklów srebra, i pret zloty jeden, piecdziesiat syklów wazacy, i pozadalem tego, i wzialem to, a oto, te rzeczy sa zakopane w ziemi, w posród namiotu mego, a srebro pod niemi.
\par 22 Tedy poslal Jozue posly, którzy biezeli do namiotu, a oto te rzeczy byly skryte w namiocie jego, a srebro pod niemi.
\par 23 A wziawszy je z namiotu przyniesli je do Jozuego, i do wszystkich synów Izraelskich, a polozyli je przed obliczem Panskiem.
\par 24 A tak wziawszy Jozue, i wszystek Izrael z nim, Achana, syna Zarego, i srebro, i plaszcz, i pret zloty, i syny jego, i córki jego, i woly jego, i osly jego, i owce jego, i namiot jego, i wszystko co mial, wywiedli je na doline Achor.
\par 25 I rzekl Jozue: Przeczzes nas potrwozyl? niechze cie tez Pan zatrwozy dnia tego. I ukamionowal go wszystek Izrael, i spalili je ogniem, ukamionowawszy je kamienmi;
\par 26 Potem wystawili na nim wielka kupe kamieni, która trwa az do dnia tego. I odwrócil sie Pan od gniewu zapalczywosci swojej; przetoz nazwane jest imie miejsca onego, dolina Achor, az do dnia dzisiejszego.

\chapter{8}

\par 1 Potem rzekl Pan do Jozuego: Nie bój sie, ani sie lekaj; wezmij z soba wszystek lud wojenny, a wstawszy ciagnij do Haj, otom dal w rece twoje króla Haj, i lud jego, i ziemie jego.
\par 2 A uczynisz Hajowi i królowi jego, jakos uczynil Jerychu i królowi jego, wszakze lupy jego, i bydla jego rozbierzecie miedzy sie; uczynze zasadzke na miasto z tylu jego.
\par 3 A tak wstal Jozue i wszystek lud waleczny, aby ciagneli ku Haj; i przebral Jozue trzydziesci tysiecy mezów bardzo mocnych, i poslal je noca.
\par 4 I rozkazal im, mówiac: Patrzajcie wy, abyscie uczynili zasadzke za miastem; nie oddalajcie sie od miasta daleko bardzo, a badzcie wszyscy pogotowiu.
\par 5 A ja, i wszystek lud, który ze mna jest, przyciagniemy pod miasto; a gdy oni wynijda przeciwko nam, jako pierwej ucieczemy przed nimi.
\par 6 A oni pójda za nami, az je uwiedziemy od miasta; bo rzeka: Uciekaja przed nami, jako i pierwej, gdyz uciekac bedziemy przed nimi.
\par 7 Tedy wy wstaniecie z zasadzki, i wyprzecie ostatek ludu z miasta, i da je Pan, Bóg wasz, w reke wasze.
\par 8 A wziawszy miasto, zapalicie je ogniem, wedlug slowa Panskiego uczynicie; patrzajciez, rozkazalem wam.
\par 9 Poslal je tedy Jozue, i szli na zasadzke; a zostali miedzy Betel, i miedzy Haj na zachód Hajowi; a Jozue przez one noc zostal w posrodku ludu.
\par 10 Potem wstawszy Jozue bardzo rano, obliczyl lud, a szedl sam i starsi z Izraela przed ludem przeciw Haj.
\par 11 Wszystek tez lud wojenny, który z nim byl, ruszyli sie, i przyciagnawszy przyszli pod miasto, i polozyli sie obozem na stronie pólnocnej ku Haj; a byla dolina miedzy nim, i miedzy Haj.
\par 12 Nadto wzial okolo pieciu tysiecy mezów, które postawil na zasadzce miedzy Betel, i miedzy Haj, od strony zachodniej miasta.
\par 13 I przyblizyl sie lud, to jest, wszystko wojsko, które bylo od pólnocy miasta, i którzy byli na zasadzce jego od zachodu miasta; i przyciagnal Jozue onej nocy w posrodek doliny.
\par 14 I stalo sie, gdy je ujrzal król Haj, pospieszyli sie i wstali rano, i wyszli ludzie z miasta przeciw Izraelowi ku bitwie, sam król, i wszystek lud jego, na czas naznaczony przed równine, nie wiedzac, ze zasadzka byla uczyniona nan za miastem.
\par 15 Tedy Jozue i wszystek Izrael, jakoby od nich porazeni, uciekali droga ku puszczy.
\par 16 I zwolany jest wszystek lud, który byl w miescie, aby je gonili, i gonili Jozuego; i tak uwiedzieni byli od miasta.
\par 17 I nie zostal nikt w Haj i w Betel, któryby nie wyszedl za Izraelem; i zostawili miasto otworzone, a gonili Izraela.
\par 18 Tedy rzekl Pan do Jozuego: Podnies choragiew, która masz w rece swej, przeciwko Haj; bo je w rece twoje dam. I podniósl Jozue choragiew, która mial w rece swej, przeciwko miastu.
\par 19 A oni, co byli na zasadzce, wstawszy predko z miejsca swego; biezali, gdy on podniósl reke swa, a ubiezawszy miasto, wzieli je, i zaraz je zapalili ogniem.
\par 20 A obejrzawszy sie mezowie miasta Haj ujrzeli, a oto, wstepowal dym miasta ku niebu, i nie mieli miejsca do uciekania, ani tam, ani sam; bo lud, który uciekal ku puszczy, obrócil sie na one, co je gonili.
\par 21 Tedy Jozue i wszystek lud Izraelski, widzac, iz oni, co byli na zasadzce, wzieli miasto, a iz wychodzil dym z miasta, obrócili sie i pobili meze miasta Haj.
\par 22 Oni tez drudzy wyszli z miasta przeciwko nim, i obtoczyli je Izraelczycy, jedni stad, a drudzy zowad, i porazili je tak, iz z nich zaden nie zostal, ani uszedl.
\par 23 Tamze króla Haj pojmali zywo, i przywiedli go przed Jozuego.
\par 24 Gdy tedy Izraelczycy pobili wszystkie obywatele Haj na polu przy puszczy, tedy za nimi szli w pogon, a polegli oni wszyscy od miecza, az wygladzeni sa; obrócili sie wszyscy Izraelczycy do Haj, i wysiekli je ostrzem miecza.
\par 25 I bylo wszystkich, którzy polegli dnia onego, od meza az do niewiasty dwanascie tysiecy, wszystkich obywateli Haj.
\par 26 A Jozue nie spuscil reki swej, która byl podniósl z choragwia, az pobil wszystkie obywatele Haj.
\par 27 Tylko bydlo, i lupy miasta onego rozebrali miedzy sie Izraelczycy wedlug slowa Panskiego, które rozkazal Jozuemu.
\par 28 Tedy zapalil Jozue Haj, i uczynil je mogila wieczna, i pustynia az do dnia tego.
\par 29 A króla Haj obwiesil na drzewie az do wieczora; a gdy slonce zaszlo, rozkazal Jozue, aby zdjeto trupa jego z drzewa, a porzucono go w samem wejsciu bramy miejskiej, i namiotali nan kupe wielka kamieni, która jest az do dnia tego.
\par 30 Tedy Jozue zbudowal oltarz Panu, Bogu Izraelskiemu, na górze Hebal.
\par 31 Jako byl rozkazal Mojzesz, sluga Panski, synom Izraelskim, a jako napisano w ksiegach zakonu Mojzeszowego, oltarz z calego kamienia, na którym zadne zelazo nie postalo; i sprawowali na nim calopalenia Panu, ofiarowali tez spokojne ofiary.
\par 32 Tamze napisal na kamieniach powtórzenie zakonu Mojzeszowego, który napisal przed oblicznoscia synów Izraelskich.
\par 33 A wszystek Izrael, i starsi jego, i przelozeni, i sedziowie jego, stali po obu stronach skrzyni przed kaplanami Lewitami, którzy nosili skrzynie przymierza Panskiego, tak przychodzien, jako w domu zrodzony, polowa ich przeciw górze Garyzym, a polowa ich przeciw górze Hebal, jako byl przedtem rozkazal Mojzesz, sluga Panski, aby blogoslawiono ludowi Izraelskiemu.
\par 34 A potem czytal wszystkie slowa zakonu, blogoslawienstwo, i przeklestwo, wedlug wszystkiego, co napisano w ksiegach zakonu.
\par 35 Nie bylo i slowa ze wszystkiego, co rozkazal Mojzesz, czego by nie czytal Jozue przed wszystkiem zgromadzeniem Izraelskiem, przed niewiastami, i przed dziatkami, i przed przychodniami, którzy mieszkali miedzy nimi.

\chapter{9}

\par 1 A gdy uslyszeli wszyscy królowie, którzy byli za Jordanem na górach, i na równinach, i nad wszystkiem brzegiem morza wielkiego przeciw Libanowi, Hetejczyk, i Amorejczyk, i Chananejczyk, Ferezejczyk, Hewejczyk, i Jebuzejczyk;
\par 2 Zebrali sie pospolu, aby walczyli przeciw Jozuemu, i przeciw Izraelowi, jednomyslnie.
\par 3 Ale obywatele Gabaon, uslyszawszy, co uczynil Jozue Jerychowi i Hajowi,
\par 4 Postapili sobie i oni chytrze, a poszedlszy zmyslili sie byc poslami, i wzieli wory stare na osly swe, i lagwie winne stare, i potarte, i latane;
\par 5 I obuwie stare i latane na nogi swoje, i szaty stare na sie, a wszystek chleb, co go z soba nabrali w droge, suchy byl i splesnialy.
\par 6 Tedy przyszli do Jozuego, do obozu w Galgal, i rzekli do niego, i do mezów Izraelskich: Z ziemismy dalekiej przyszli; przetoz teraz uczyncie z nami przymierze.
\par 7 Ale odpowiedzieli mezowie Izraelscy Hewejczykowi: Podobno ty mieszkasz miedzy nami, a jakoz z toba mozemy uczynic przymierze?
\par 8 A oni rzekli do Jozuego: Sludzy twoi jestesmy. I rzekl do nich Jozue: Coscie wy zacz, a skadescie przyszli?
\par 9 I odpowiedzieli mu: Z ziemi dalekiej bardzo przyszli sludzy twoi w imieniu Pana, Boga twego; bosmy slyszeli slawe jego, i wszystko, co uczynil w Egipcie;
\par 10 I wszystko, co uczynil dwom królom Amorejskim, którzy byli za Jordanem, Sehonowi królowi Hesebon, i Ogowi królowi Basan, którzy mieszkali w Astarot.
\par 11 I rozkazali nam starsi nasi, i wszyscy obywatele ziemi naszej, mówiac: Nabierzcie sobie zywnosci na droge, a idzcie przeciwko nim, i mówcie im: Sludzy wasi jestesmy, przetoz teraz uczyncie z nami przymierze.
\par 12 Ten chleb nasz cieplysmy na droge wzieli z domów naszych tego dnia, gdysmy wyszli, abysmy szli do was; a teraz oto wysechl, i poplesnial.
\par 13 I te lagwie winne, któresmy byli napelnili, byly nowe, a oto sie popekaly; takze te szaty nasze, i obuwie nasze zwiotszaly dla bardzo dalekiej drogi.
\par 14 A tak wzieli oni mezowie Izraelscy z onej zywnosci ich, a ust sie Panskich nie pytali.
\par 15 Tedy z nimi uczynil Jozue pokój, i postanowil z nimi przymierze, aby ich zachowal przy zywocie; takze przysiegly im ksiazeta zgromadzenia.
\par 16 Ale po trzech dniach po uczynieniu z nimi przymierza, uslyszeli, ze blisko ich byli, a iz w posrodku ich mieszkali.
\par 17 A ruszywszy sie synowie Izraelscy przyciagneli do miast ich dnia trzeciego, a miasta ich te byly: Gabaon, i Kafira, i Beerot, i Karyjatyjarym.
\par 18 I nie wytracili ich synowie Izraelscy; albowiem przysiegly im byly ksiazeta zgromadzenia przez Pana, Boga Izraelskiego, skad szemralo wszystko zgromadzenie przeciw ksiazetom.
\par 19 I rzekly wszystkie ksiazeta do calego zgromadzenia: Mysmy im przysiegli przez Pana, Boga Izraelskiego; przetoz teraz nie mozemy sie ich tknac.
\par 20 To im uczynimy, a zachowamy je zywo, izby nie przyszlo na nas rozgniewanie dla przysiegi, którasmy im przysiegli.
\par 21 Nadto rzekly do nich ksiazeta: Niech zyja, a niech rabia drwa, i niech nosza wode wszystkiemu zgromadzeniu; i przestali na tem, jako im powiedzialy ksiazeta.
\par 22 Potem wezwal ich Jozue, i rzekl do nich, mówiac: Przeczzescie nas oszukali, powiadajac: Dalekimismy od was bardzo, a wy w posrodku nas mieszkacie?
\par 23 A tak teraz przekleci jestescie, i nie ustana z was sludzy, i rabiacy drwa, i noszacy wode do domu Boga mego.
\par 24 Którzy odpowiedzieli Jozuemu, i rzekli: Zapewne oznajmiono bylo slugom twoim, jako byl rozkazal Pan, Bóg twój, Mojzeszowi sludze swemu, aby wam dal wszystke ziemie, a izby wygladzil wszystkie mieszkajace w tej ziemi przed twarza wasza; przetoz balismy sie bardzo o zywot nasz przed wami, i uczynilismy te rzecz.
\par 25 A teraz otosmy w rekach twoich; coc sie dobrego i slusznego widzi uczynic z nami, uczyn.
\par 26 I uczynil im tak, a wybawil je z rak synów Izraelskich, ze ich nie pobili.
\par 27 I postanowil je Jozue dnia onego, aby rabali drwa, i nosili wode zgromadzeniu, i do oltarza Panskiego az do tego dnia, na miejscu, które by obral.

\chapter{10}

\par 1 A gdy uslyszal Adonisedek, król Jerozolimski, iz wzial Jozue Haj, i zburzyl je, (bo jako uczynil Jerychowi i królowi jego, tak uczynil Hajowi i królowi jego,)a iz uczynili pokój obywatele Gabaon z Izraelem, i mieszkaja w posrodku ich;
\par 2 Tedy sie ulakl bardzo, przeto ze miasto wielkie bylo Gabaon, jako jedno z miast królewskich, a ze bylo wieksze niz Haj, a wszyscy mezowie jego waleczni.
\par 3 Przetoz poslal Adonisedek, król Jerozolimski, do Hohama, króla Hebron, i do Faran, króla Jerymota, i do Jafija, króla Lachys, i do Dabir, króla Eglon, mówiac:
\par 4 Przyjedzcie do mnie, a dajcie mi pomoc, abysmy pobili Gabaonity, którzy uczynili pokój z Jozuem, i z syny Izraelskimi.
\par 5 Zebralo sie tedy, a wyciagnelo piec królów Amorejskich, król Jerozolimski, król Hebron, król Jerymot, król Lachys, król Eglon, sami, i wszystkie wojska ich, i polozyli sie obozem u Gabaon, i dobywali go.
\par 6 Tedy poslali obywatele Gabaon do Jozuego, i do obozu w Galgal, mówiac: Nie zawsciagaj reki swej od slug twoich; przyciagnij do nich rychlo, a wybaw nas i pomóz nam; boc sie zebrali przeciwko nam wszyscy królowie Amorejscy, którzy mieszkaja po górach.
\par 7 Ruszyl sie tedy Jozue z Galgal, sam i wszystek lud wojenny z nim, i wszyscy mezowie waleczni.
\par 8 (Bo byl rzekl Pan do Jozuego: Nie bój sie ich; albowiem w rece twoje podalem je, a nie ostoi sie zaden z nich przed toba.)
\par 9 I przypadl na nie Jozue nagle; bo cala noc ciagnal z Galgal.
\par 10 I potrwozyl je Pan przed obliczem Izraela, który je porazil porazka wielka w Gabaon, i gonil je droga, która chodza ku Betoron, a bil je az do Aseka i az do Maceda.
\par 11 I stalo sie, gdy uciekali przed Izraelem, biezac z góry do Betoron, ze Pan spuscil na nie kamienie wielkie z nieba az do Aseka, i umierali; wiecej ich pomarlo od kamienia gradowego, niz ich pobili synowie Izraelscy mieczem.
\par 12 Tedy mówil Jozue do Pana, dnia, którego podal Pan Amorejczyka w rece synom Izraelskim, i rzekl przed oczyma Izraela: Slonce w Gabaon zastanów sie, a miesiacu w dolinie Ajalon!
\par 13 I zastanowilo sie slonce, a miesiac stanal, az sie lud pomscil nad nieprzyjacioly swymi. Izali to nie jest napisano w ksiegach sprawiedliwego? Tedy stanelo slonce w posród nieba, a nie pospieszylo sie zachodzic, jakoby przez caly dzien.
\par 14 I nie byl takowy dzien przedtem, ani potem, w któryby usluchac mial Pan glosu czlowieczego, bo Pan walczyl za Izraelem.
\par 15 Potem sie wrócil Jozue, i wszystek Izrael z nim, do obozu do Galgal.
\par 16 A ucieklo bylo onych piec królów, i skryli sie w jaskinia przy Maceda.
\par 17 I dano znac Jozuemu, mówiac: Znaleziono piec królów, którzy sie pokryli w jaskini w Maceda.
\par 18 I rzekl Jozue: Przywalcie kamienie wielkie do dziury jaskini, a postawcie u niej meze, aby ich strzegli.
\par 19 A wy nie stójcie, goncie nieprzyjacioly wasze, a bijcie ostatek ich, ani im dajcie uchodzic do miast ich; boc je podal Pan, Bóg wasz, w reke wasze.
\par 20 A gdy przestal Jozue z syny Izraelskimi bic ich porazka bardzo wielka, az je do szczetu wytracili, a którzy zywo zostali z nich, uszli do miast obronnych;
\par 21 Tedy wrócil sie wszystek lud zdrowo do obozu, do Jozuego w Maceda, a nie ruszyl przeciwko synom Izraelskim nikt jezykiem swoim.
\par 22 Potem rzekl Jozue: Otwórzcie te dziure jaskini, a wywiedzcie do mnie tych pieciu królów z jaskini.
\par 23 I uczynili tak, i wywiedli do niego pieciu królów onych z jaskini, króla Jerozolimskiego, króla Hebron, króla Jerymot, króla Lachys, króla Eglon.
\par 24 A gdy wywiedli one króle do Jozuego, tedy przyzwal Jozue wszystkich mezów Izraelskich, i rzekl do rotmistrzów, mezów walecznych, którzy z nim chodzili: Przystapcie sam, a nastapcie nogami waszemi na szyje tych królów; którzy przystapiwszy nastapili nogami swemi na szyje ich.
\par 25 Zatem rzekl do nich Jozue: Nie bójcie sie, ani sie lekajcie; zmacniajcie sie, i meznie sobie poczynajcie; boc tak uczyni Pan wszystkim nieprzyjaciolom waszym, przeciw którym walczycie.
\par 26 Potem pobil je Jozue, i pomordowal je, i zawiesil je na pieciu drzewach, a wisieli na drzewach az do wieczora.
\par 27 A gdy zaszlo slonce, rozkazal Jozue, ze je zlozono z drzewa, i wrzucono je do jaskini, w której sie byli skryli, a zawalono kamienmi wielkiemi dziure u jaskini, które tam sa jeszcze i do dnia tego.
\par 28 Tegoz dnia wzial Jozue Maceda, i wysiekl je ostrzem miecza, i króla ich zamordowal wespól z nimi, i wszelka dusze, która byla w niem; nie zostawil zadnego zywo, i uczynil królowi Maceda, jako uczynil królowi Jerycha.
\par 29 Potem ciagnal Jozue, i wszystek Izrael z nim, z Maceda do Lebny, i dobywal Lebny.
\par 30 A podal Pan i ono w rece Izraela, i króla jego, i wysiekl je ostrzem miecza, i wszelka dusze, która byla w niem; nie zostawil w niem zadnego zywo, i uczynil królowi jego, jako uczynil królowi Jerycha.
\par 31 Potem ciagnal Jozue, i wszystek Izrael z nim, z Lebny do Lachys, a polozywszy sie przy niem obozem, dobywal go.
\par 32 I podal Pan Lachys w rece Izraela, i wzial je dnia drugiego, i wysiekli je ostrzem miecza, i wszelka dusze, która byla w niem, tak wlasnie jako uczynil Lebnie.
\par 33 Tedy przyszedl Horam, król Gazer, na ratunek Lachysowi, ale go porazil Jozue, i lud jego, tak iz nie zostawil mu zadnego zywo.
\par 34 Potem ciagnal Jozue, i wszystek Izrael z nim, z Lachys do Eglon, i polozyli sie obozem przeciwko niemu, i dobywali go;
\par 35 Które wziawszy onegoz dnia, wysiekli je ostrzem miecza, i wszelka dusze, która byla w niem, onegoz dnia zabil, tak wlasnie jako uczynil Lachys.
\par 36 Potem sie ruszyl Jozue, i wszystek Izrael z nim, z Eglonu do Hebronu, i dobywal go;
\par 37 I wzieli je, a wysiekli je ostrzem miecza, i króla jego, i wszystkie miasta jego, i wszelka dusze, która byla w niem; nie zostawil zadnego zywo, tak wlasnie jako uczynil Eglonowi, i wytracil je, i wszelka dusze, która w niem byla.
\par 38 Stamtad obrócil sie Jozue, i wszystek Izrael z nim, do Dabir, i dobywal go.
\par 39 I wzial je, i króla jego, i wszystkie miasta jego, i wysiekli je ostrzem miecza, i pomordowal wszystkie dusze, które w niem byly; nie zostawil zadnego zywo; jako uczynil Hebronowi tak uczynil Dabirowi i królowi jego, i jako uczynil Lebnie i królowi jego.
\par 40 A tak pobil Jozue wszystke ziemie górna, i poludniowa, i polna, i podgórna, i wszystkie króle ich; nie zostawil zadnego zywo, ale wszystkie dusze wytracil, jako mu byl przykazal Pan, Bóg Izraelski.
\par 41 I porazil je Jozue od Kades Barny az do Gazy, i wszystke ziemie Gosen, i az do Gabaon.
\par 42 A wszystkie te króle, i ziemie ich, wzial Jozue jednym razem; albowiem Pan, Bóg Izraelski, walczyl za Izraelem.
\par 43 Zatem sie wrócil Jozue i wszystek Izrael z nim do obozu do Galgal.

\chapter{11}

\par 1 To gdy uslyszal Jabin, król Hasor, poslal do Johaba, króla Madon, i do króla Symron, i do króla Achsaf,
\par 2 I do królów, którzy byli na pólnocy, na górach i na polach, na poludnie Cynerot, i w równinach, i w krainach Dor, ku zachodowi;
\par 3 Do Chananejczyka na wschód i na zachód slonca, i do Amorejczyka, i do Hetejczyka, i do Ferezejczyka, i do Jebuzejczyka po górach, i do Hewejczyka pod góra Hermon, w ziemi Maswa.
\par 4 I wyciagneli sami, i wszystkie wojska ich z nimi, lud wielki, jako piasek, który jest na brzegu morskim, i koni i wozów bardzo wiele.
\par 5 A zgromadziwszy sie wszyscy oni królowie przyszli, i polozyli sie pospolu obozem u wód Merom, aby zwiedli bitwe z Izraelem.
\par 6 I rzekl Pan do Jozuego: Nie bój sie ich, albowiem jutro o tym czasie Ja podam te wszystkie pobite przed Izraelem; koniom ich zyly poderzniesz, a wozy ich ogniem spalisz.
\par 7 Wyciagnal tedy Jozue, i wszystek lud waleczny z nim, przeciwko nim ku wodom Merom z nagla, i uderzyli na nie.
\par 8 I podal je Pan w reke Izraelowi, i porazili je, a gonili je az do Sydonu wielkiego, i az do wód goracych, i az do pola Masfa na wschód slonca, i pobili je, tak iz jednego z nich nie zostawili zywego.
\par 9 I uczynil im Jozue, jako mu byl rozkazal Pan; koniom ich zyly poderznal, a wozy ich popalil ogniem.
\par 10 Potem wróciwszy sie Jozue tego czasu wzial Hasor, a króla jego zabil mieczem; a Hasor bylo przedtem glowa wszystkich tych królestw.
\par 11 Zabili tez kazda dusze, która byla w niem, ostrzem miecza mordujac, tak iz nie zostalo nic zywego; a Hasor spalil ogniem.
\par 12 Takze uczynil wszystkim miastom królów onych, i wszystkie króle ich pojmal Jozue, i pobil je ostrzem miecza, mordujac je, jako byl rozkazal Mojzesz, sluga Panski.
\par 13 Tylko tych wszystkich miast, które byly obronne, nie palil Izrael, oprócz samego Hasora, które spalil Jozue.
\par 14 Wszystkie tez lupy miast onych, i bydla, pobrali sobie synowie Izraelscy, tylko wszystkie ludzie zabijali ostrzem miecza, az je wytracili, nie zostawujac nikogo zywego.
\par 15 Jako byl rozkazal Pan Mojzeszowi, sludze swemu, tak rozkazal Mojzesz Jozuemu; tak tez uczynil Jozue, nie opuscil niczego ze wszystkiego, co byl rozkazal Pan Mojzeszowi.
\par 16 A tak wzial Jozue wszystke one ziemie górna, i wszystke na poludnie lezaca, i wszystke ziemie Gosen, i równiny, i pola, i góre Izrael z równina jej;
\par 17 Od góry Halak, która idzie ku Seir, az do Baalgad, w równinie Libanskiej pod góra Hermon; i wszystkie króle ich pojmal, i porazil je, i pozabijal je.
\par 18 Przez wiele dni prowadzil Jozue z onymi wszystkimi królami wojne.
\par 19 A nie bylo miasta, które by pokój uczynilo z syny Izraelskimi, oprócz Hewejczyków, którzy mieszkali w Gabaon; wszystkie insze wzieli przez wojne.
\par 20 Albowiem od Pana sie to stalo, ze zatwardzil serca ich, aby szli ku bitwie przeciw Izraelowi, zeby je wyniszczyl, nie majac nad nimi milosierdzia, ale zeby je wytracil, jako byl rozkazal Pan Mojzeszowi.
\par 21 I ciagnal Jozue onegoz czasu, i wybil syny Enakowe z gór, z Hebronu, z Dabiru, z Anab, i ze wszystkich gór Judzkich, i ze wszystkich gór Izraelskich, pospolu z miasty ich wykorzenil je Jozue.
\par 22 Nie zostal nikt z Enakitów, w ziemi synów Izraelskich; tylko w Gazie, w Gad, i w Azdod zostali.
\par 23 Wzial tedy Jozue wszystke one ziemie, tak jako mówil Pan do Mojzesza: i podal ja Jozue w dziedzictwo Izraelowi wedlug dzialów ich, i wedlug pokolenia ich, a uspokoila sie ziemia od wojen.

\chapter{12}

\par 1 A ci sa królowie ziemi, które pobili synowie Izraelscy, i posiedli ziemie ich za Jordanem ku wschodowi slonca, od potoku Arnon az do góry Hermon, i wszystke równine ku wschodowi slonca:
\par 2 Sehon, król Amorejski, który mieszkal w Hesebon, a panowal od Aroer, które lezy nad brzegiem potoku Arnon, i od polowy tegoz potoku i polowy Galaadu az do potoku Jabok, gdzie sa granice synów Ammonowych,
\par 3 A od równin az do morza Cynerot na wschód slonca, i az do morza pustyni, do morza slonego na wschód, idac ku Betsemot, i od poludnia pod góre Fazga.
\par 4 I granice Oga, króla Basanskiego, który byl pozostal z Refaimów, a mieszkal w Astarot i w Edrej.
\par 5 Który tez panowal na górze Hermon, i w Selecha, i we wszystkiem Basan, az do granic Gessurytów, i Mahachatytów, i nad polowa Galaad ku granicy Sehona, króla Hesebonskiego.
\par 6 Mojzesz, sluga Panski, i synowie Izraelscy, pobili je; i podal te ziemie Mojzesz, sluga Panski, w dziedzictwo Rubenitom, i Gadytom, i polowie pokolenia Manasesowego.
\par 7 Ci tez sa królowie ziemi, które pobil Jozue, i synowie Izraelscy za Jordanem na zachód slonca, od Baalgad na polu Libanskiem, i az do Halak, która idzie ku Seir, która podal Jozue pokoleniom Izraelskim w dziedzictwo wedlug dzialu ich.
\par 8 Na górach, i na równinach, i w polach, i w nizynach, i na puszczy, i na poludnie ziemi Hetejczyka, Amorejczyka, i Chananejczyka, Ferezejczyka, Hewejczyka, i Jebuzejczyka.
\par 9 Król Jerycha jeden; król Haj, które jest w bok Betel, jeden.
\par 10 Król Jeruzalem jeden; król Hebron jeden.
\par 11 Król Jerymot jeden; król Lachys jeden.
\par 12 Król Heglon jeden; król Gazer jeden.
\par 13 Król Dabir jeden; król Gader jeden.
\par 14 Król Horma jeden; król Hered jeden.
\par 15 Król Lebni jeden; król Adullam jeden.
\par 16 Król Maceda jeden; król Betel jeden.
\par 17 Król Taffua jeden; król Hefer jeden.
\par 18 Król Afek jeden; król Saron jeden.
\par 19 Król Madon jeden; król Hasor jeden.
\par 20 Król Symron Meron jeden; król Aksaf jeden.
\par 21 Król Tenach jeden; król Mageddo jeden.
\par 22 Król Kades jeden; król Jachanam z Karmelu jeden.
\par 23 Król Dor z krainy Dor jeden; król Goim w Galgal jeden;
\par 24 Król Torsa jeden. Wszystkich królów trzydziesci i jeden.

\chapter{13}

\par 1 I zstarzal sie Jozue, a byl zeszly w leciech. I rzekl Pan do niego: Tys sie zstarzal, a zszedles w leciech, a ziemi zostawa bardzo wiele ku posiadaniu.
\par 2 Tac jest ziemia, która pozostawa: Wszystkie granice Filistynów, i wszyscy Gessurytowie,
\par 3 Od Nilu, który oblewa Egipt, az do granicy Akaronu na pólnocy, przynalezy Chananejczykowi, piecioro ksiestw Filistynskich; Asackie, i Asdodziejskie, Askalonskie, Getejskie, i Akaronicki, i Hawejczycy.
\par 4 Od poludnia wszystka ziemia Chananejska, i Mara, które jest Sydonczyków az do Afeka, i az do granicy Amorejczyka;
\par 5 I ziemia Giblitów ze wszystkim Libanem na wschód slonca, od Baalgad pod góre Hermon, az gdzie chodza do Emat.
\par 6 Wszystkie mieszkajace na górach od Libanu az do wód goracych, wszystkie Sydonczyki Ja wypedze przed syny Izraelskimi; tylko ja podziel Izraelitom w dziedzictwo, jakom ci rozkazal.
\par 7 Przetoz teraz rozdziel te ziemie w dziedzictwo, dziewieciorgu pokoleniu, i polowie pokolenia Manasesowego.
\par 8 Gdyz z druga polowa Rubenitowie i Gadytowie wzieli dziedzictwo swoje, które im dal Mojzesz za Jordanem na wschód slonca, jako im dal Mojzesz, sluga Panski;
\par 9 Od Aroer, które jest nad brzegiem potoku Arnon, i miasto, które jest w posrodku potoku, i wszystke równine Medeba az do Dybon;
\par 10 I wszystkie miasta Sehona, króla Amorejskiego, który królowal w Hesebon, az do granicy synów Ammonowych;
\par 11 Takze Galaad, i granice Gessurytów, i Machatytów, i wszystke góre Hermon, i wszystko Basan az do Salecha;
\par 12 Wszystko królestwo Oga w Basan, który królowal w Astarot, i w Edrej; ten byl pozostal z Refaimitów, a pobil je Mojzesz i wygladzil je.
\par 13 Ale nie wygnali synowie Izraelscy Gessurytów i Machatytów; przetoz mieszkal Gessur i Machat w posród Izraelczyków az do dnia tego.
\par 14 Tylko pokoleniu Lewi nie dal dziedzictwa; ofiary ogniste Pana Boga Izraelskiego sa dziedzictwem jego, jako mu powiedzial Bóg.
\par 15 A tak oddal Mojzesz pokoleniu synów Rubenowych dziedzictwo wedlug familii ich.
\par 16 I byla granica ich od Aroer, które jest nad brzegiem potoku Arnon, i miasto, które jest w posród potoku, i wszystka równina ku Medeba.
\par 17 Hesebon, i wszystkie miasta jego, które byly w równinie; Dybon i Bamot Baal, i Bet Baal Meon;
\par 18 I Jassa, i Cedymot, i Mefaat;
\par 19 I Karyjataim, i Sebama, i Saratasar na górze w dolinie;
\par 20 I Betfegor, i Asdod, Fazga, i Betyjesymot.
\par 21 Wszystkie tez miasta w równinie, i wszystko królestwo Sehona, króla Amorejskiego, który królowal w Hesebon, którego zabil Mojzesz, i ksiazeta Madyjanskie Ewi, i Recem, i Sur, i Hur, i Reba; ksiazeta Sehonowe, obywatele ziemi.
\par 22 I Balaama, syna Boerowego, wieszczka, zabili synowie Izraelscy mieczem z innymi pobitymi.
\par 23 Byla tedy granica synów Rubenowych Jordan z granicami swemi. Toc jest dziedzictwo synów Rubenowych wedlug domów ich, miast i wsi ich.
\par 24 Dal tez Mojzesz pokoleniu Gad, synom Gadowym, wedlug domów ich dziedzictwo.
\par 25 A byly ich granice Jazer i wszystkie miasta Galaad, i polowa ziemi synów Ammonowych az do Aroer, które jest przeciw Rabba;
\par 26 I od Hesebon az do Ramat Massa i Betonim, a od Mahanaim az do granicy Dabir.
\par 27 W dolinie tez Beram, i Betnimra, i Sochot, i Safon, ostatek królestwa Sehona, króla Hesebonskiego, Jordan i pogranicze jego az do konca morza Cynneret za Jordanem na wschód slonca.
\par 28 Toc jest dziedzictwo synów Gad wedlug domów ich, miast i wsi ich.
\par 29 Nadto dal Mojzesz osiadlosc polowie pokolenia Manasesowego, i byla ta polowa pokolenia synów Manasesowych wedlug domów ich.
\par 30 Byla granica ich od Machanaim wszystko Basan i wszystko królestwo Oga, króla Basanskiego, i wszystkie wsi Jairowe, które sa w Basan, szescdziesiat miast.
\par 31 I polowe Galaad, i Astarot, i Erdej, miasta królestwa Oga w Basan, dal synom Machyra, syna Manasesowego, polowie synów Machyrowych wedlug domów ich.
\par 32 Tec sa osiadlosci, które podzielil Mojzesz w polach Moabskich za Jordanem przeciw Jerychu na wschód slonca.
\par 33 Ale pokoleniu Lewi nie dal Mojzesz dziedzictwa; bo Pan, Bóg Izraelski, sam jest dziedzictwem ich, jako im powiedzial.

\chapter{14}

\par 1 A toc jest, co dziedzictwem wzieli synowie Izraelscy w ziemi Chananejskiej, a co prawem dziedzicznem oddali im w osiadlosc Eleazar kaplan i Jozue, syn Nunów, i przedniejsi z ojców z pokolenia synów Izraelskich.
\par 2 Losem dzielac dziedzictwo ich, jako byl rozkazal Pan przez Mojzesza, dziewieciorgu pokoleniu i polowie pokolenia.
\par 3 Albowiem Mojzesz byl oddal dziedzictwo dwom pokoleniom i polowie pokolenia za Jordanem; ale Lewitom nie dal byl dziedzictwa miedzy nimi.
\par 4 Bo bylo synów Józefowych dwa pokolenia, Manasesowe i Efraimowe; ani dali dzialu Lewitom w ziemi, oprócz miast ku mieszkaniu, z przedmiesciami ich dla bydla ich i dla trzód ich.
\par 5 Jako rozkazal Pan Mojzeszowi, tak uczynili synowie Izraelscy, i podzielili ziemie.
\par 6 Tedy przyszli synowie Judowi do Jozuego w Galgal; i rzekl do niego Kaleb, syn Jefuna Kenezejskiego: Ty wiesz, co mówil Pan do Mojzesza, meza Bozego, o mnie i o tobie w Kades Barnie.
\par 7 Czterdziesci mi lat bylo, gdy mnie slal Mojzesz, sluga Panski, z Kades Barny ku przeszpiegowaniu ziemi, i odnioslem mu te rzecz, jako bylo w sercu mojem.
\par 8 Lecz bracia moi, którzy chodzili ze mna, skazili serce ludowi; alem ja przecie szedl statecznie za Panem, Bogiem moim.
\par 9 I przysiagl Mojzesz dnia onego, mówiac: Zaiste ziemia, która deptala noga twoja, przyjdzie tobie w dziedzictwo, i synom twoim az na wieki, przeto zes statecznie chodzil za Panem, Bogiem moim.
\par 10 A teraz oto przedluzyl zywota mego Pan, jako powiedzial; juz sa czterdziesci i piec lat od onego czasu, jako to mówil Pan do Mojzesza, a jako chodzili Izraelczycy po puszczy; a teraz oto ja dzis mam osiemdziesiat i piec lat:
\par 11 A jeszcze i dzis takiem duzy, jakom byl w on czas, gdy mie wyslal Mojzesz; a jako moc moja byla na on czas, taka jest moc moja i teraz ku bojowaniu, i ku wychodzeniu i przychodzeniu.
\par 12 A tak teraz daj mi te góre, o której powiedzial Pan dnia onego; bos ty slyszal dnia onego, iz tam sa Enakitowie, i miasta wielki a obronne; bedzieli Pan ze mna, wypedze je, jako mi obiecal Pan.
\par 13 I blogoslawil mu Jozue, a dal Hebron Kalebowi, synowi Jefunowemu, w dziedzictwo.
\par 14 A tak dostal sie Hebron Kalebowi, synowi Jefuna Kenezejskiego, w dziedzictwo az do dnia tego, przeto ze statecznie chodzil za Panem, Bogiem Izraelskim.
\par 15 A zwano przedtem Hebron miasto Arba, który Arba byl czlowiekiem wielkim miedzy Enakity; i uspokoila sie ziemia od wojen

\chapter{15}

\par 1 I byl los pokolenia synów Judowych wedlug domów ich przy granicach Edom, i przy puszczy Syn na poludnie od ostatecznej granicy poludniowej.
\par 2 A byla ich granica od poludnia, od konca morza slonego, i od skaly, która jest ku poludniowi.
\par 3 I wychodzi ku poludniowi, ku pagórkowi niedzwiadkowemu, a ciagnie sie az do Syn; a idac od poludnia do Kades Barny biezy az ku Efronowi, i ciagnie sie az do Adar, obtaczajac Karkaa.
\par 4 Stamtad idac do Asemona idzie ku rzece Egipskiej, a idzie koniec tych granic na zachód; tac bedzie granica na poludniu.
\par 5 Granica zasie od wschodu slonca jest morze slone az do konca Jordanu, a granica z strony pólnocnej jest od skaly morskiej, od konca Jordanu.
\par 6 A ciagnie sie ta granica do Betaglu, i biezy od pólnocy az do Betaraba; a stamtad idzie ta granica az do kamienia Boen, syna Rubenowego.
\par 7 Idzie takze ta granica do Dabir od doliny Achor, a ku pólnocy sie udawa do Galgal, które jest przeciw górze, gdzie wstepuja do Adommim, która jest na poludnie od rzeki, a idzie ta granica do wód Ensemes, a konczy sie u studnicy Rogiel.
\par 8 Biezy tez ta granica przez doline syna Hennomowego po bok Jebuzejczyka od poludnia, co jest Jeruzalem. Stamtad biezy ta granica na wierzch góry, która jest przeciwko dolinie Hennom na zachód, a która jest na koncu doliny Refaimitów na pólnocy.
\par 9 Obtacza tez ta granica od wierzchu góry az do zródla wody Neftoa, i biezy az do miast góry Hefron; potem sie ciagnie ta granica ku Baala, które jest Karyjatyjarym.
\par 10 Potem kolem biezy ta granica od Baala na zachód do góry Seir, a stamtad przechodzi po bok góry Jarym od pólnocy, która jest Cheslon, i spuszcza sie do Betsemes, i przychodzi do Tamna.
\par 11 I wychodzi ta granica po bok Akaronu na pólnocy, a idzie kolem ta granica, az do Sechronu, i biezy przez góre Baala; stamtad wychodzi do Jabneel, i koncza sie te granica u morza.
\par 12 A granica zachodnia jest przy morzu wielkiem, i przy granicach jego; tac jest granica synów Juda w okrag podlug domów ich.
\par 13 Ale Kalebowi synowi Jefunowemu, dal Jozue dzial miedzy syny Juda, jak Pan powiedzial Jozuemu, miasto Arba, ojca olbrzymów, to jest Hebron.
\par 14 I wypedzil stamtad Kaleb trzech synów Enakowych: Sesaja, i Ahymana, i Talmaja, syny Enakowe.
\par 15 A wyszedl stamtad do mieszkajacych w Dabir, które zwano przedtem Karyjatsefer.
\par 16 I rzekl Kaleb: Kto by dobyl Karyjatsefer, a wzial je, tedy mu dam Achse, córke swoje, za zone.
\par 17 I dobyl go Otonijel, syn Keneza, brata Kalebowego; i dal mu Achse, córke swoje, za zone.
\par 18 I stalo sie, gdy ona przyszla do niego, namawiala go, aby prosil ojca jej o pole; przetoz zsiadla z osla, i rzekl do niej Kaleb: Cóz ci?
\par 19 A ona odpowiedziala: Daj mi blogoslawienstwo; gdyzes mi dal ziemie sucha, przydaj mi tez zródla wód. I dal jej zródla wyzsze, i zródla dolne.
\par 20 Toc jest dziedzictwo pokolenia synów Judowych wedlug domów ich.
\par 21 I byly miasta w granicach pokolenia synów Judowych podle granicy Edom ku poludniowi: Kabseel, i Eder, i Jagur.
\par 22 I Cyna, i Dymona, i Adada;
\par 23 I Kades, i Hasor, i Jetnan;
\par 24 I Zyf, i Telem, i Balot;
\par 25 I Hasor Hadata, i Karyjot Chesron, toc jest Hasor;
\par 26 Amam, i Sama, i Molada;
\par 27 I Asorgadda, i Hessemon, i Betfalet;
\par 28 I Hasersual, i Beersaba, i Bazotyja;
\par 29 Baala, i Ijim, i Esem;
\par 30 I Eltolad, i Kesyl, i Horma;
\par 31 I Syceleg, i Medemena, i Sensenna;
\par 32 I Lebaot, i Selim, Ain, i Remmon; wszystkich miast dwadziescia i dziewiec, i wsi ich.
\par 33 W równinach zas Estaol, i Sarea, i Asena;
\par 34 I Zanoe, i Engannim, Tepnach, i Enaim;
\par 35 Jerymot, i Adullam, Socho, i Aseka;
\par 36 I Saraim, i Adytaim, i Gedera, i Gederotaim, miast czternascie, i wsi ich.
\par 37 Sanany, i Hadasa, i Mygdalgad;
\par 38 I Delean, i Mesfa, i Jektel;
\par 39 Lachys, i Bassekat, i Eglon;
\par 40 I Chabbon, i Lachmas, i Chytlis;
\par 41 I Kiederot, Bet Dagon, i Naama, i Maceda, miast szesnascie, i wsi ich.
\par 42 Labana, i Eter, i Asan;
\par 43 I Iftach, i Esna, i Nesyb;
\par 44 I Ceila, i Achzyb, i Maresa, miast dziewiec, i wsi ich;
\par 45 Akkaron, i miasteczka jego, i wioski jego;
\par 46 Od Akkaronu az do morza wszystko, co lezy po bok Asotu, i ze wsiami ich;
\par 47 Azot, miasteczka jego, i wsi jego; Gaza, miasteczka jego, i wsi jego, az do potoku Egipskiego, i morze wielkie za granica jego.
\par 48 A na górze leza Sam, i Jeter, i Soko;
\par 49 I Danna, i Karyjatsenna, które jest Dabir;
\par 50 I Anab, i Istemo, i Anim;
\par 51 I Gosen, i Holon, i Gilo, miast jedenascie, i wsi ich;
\par 52 Arab, i Duma, i Esaan;
\par 53 I Janum, i Bet Tafua, i Afeka;
\par 54 I Chumta, i Karyjat Arbe, toc jest Hebron, i Syjor, miast dziewiec, i wsi ich.
\par 55 Maon, Karmel, i Zyf, i Juta.
\par 56 I Jezrael, i Jukiedam, i Zanoe;
\par 57 Kain, Gabaa, i Tamna, miast dziesiec, i wsi ich.
\par 58 Halhul, Betsur i Giedor;
\par 59 I Maret, i Bet Anot, i Eltekon, miast szesc, i wsi ich.
\par 60 Karyjat Baal, które jest Karyjatyjarym, i Rabba, miasta dwa, i wsi ich.
\par 61 A na puszczy: Bet Araba, Meddyn, i Sechacha;
\par 62 I Nebsan, i miasto Soli, i Engaddy, miast szesc, i wsi ich.
\par 63 Ale Jebuzejczyka, obywatela Jeruzalemskiego, nie mogli synowie Judowi wypedzic; przetoz mieszkal Jebuzejczyk z syny Juda w Jeruzalemie az do dnia tego.

\chapter{16}

\par 1 Padl tez los synom Józefowym od Jordanu ku Jerychu przy wodach Jerycha na wschód slonca, puszcza, która idzie od Jerycha przez góre Betel.
\par 2 A wychodzi od Betel do Luzy, a idzie do granicy Archy, do Attarot.
\par 3 Potem sie ciagnie ku morzu do granicy Jaflety, az do granicy Bet Horonu dolnego, i az do Gazer, a konczy sie az u morza.
\par 4 I wzieli dziedzictwo synowie Józefowi, Manase i Efraim.
\par 5 A byla granica synów Efraimowych wedlug domów ich; byla mówie granica dziedzictwa ich na wschód slonca od Attarot Adar az do Bet Horon wyzszego.
\par 6 I wychodzi ta granica do morza od Machmeta ku pólnocy, a idzie kolem ta granica pod wschód slonca do Tanat Selo, i przechodzi ja od wschodu az do Janoe;
\par 7 I ciagnie sie od Janoe do Attarot i Naarata, a przychodzi do Jerycha, a wychodzi ku Jordanowi.
\par 8 Od Tafua biezy ta granica ku zachodowi do potoku Kana, a konczy sie przy morzu. Toc jest dziedzictwo pokolenia synów Efraimowych wedlug domów ich.
\par 9 Miasta tez oddzielone synom Efraimowym byly w posród dziedzictwa synów Manasesowych, wszystkie miasta i wsi ich.
\par 10 I nie wygnali Chananejczyka, mieszkajacego w Gazer; i mieszkal Chananejczyk w posrodku Efraimitów az do dnia tego, i holdowal im, dan dawajac.

\chapter{17}

\par 1 Padl tez los pokoleniu Manasesowemu (bo on jest pierworodny Józefów.) Machyrowi pierworodnemu Manasesowemu, ojcu Galaada, przeto, ze byl mezem walecznym, i dostal mu sie Galaad i Basan.
\par 2 Dostalo sie tez innym synom Manasesowym wedlug domów ich, synom Abiezer, i synom Helek, i synom Esryjel, i synom Sychem, i synom Hefer, i synom Semida. Cic sa synowie Manasesowi, syna Józefowego, mezczyzny wedlug domów ich.
\par 3 Ale Salfaad, syn Heferów, syna Galaadowego, syna Machyrowego, syna Manasesowego, nie mial synów, jedno córki, a te imiona córek jego: Machla, i Noa, Hegla, Melcha, i Tersa.
\par 4 Te przyszedlszy przed Eleazara kaplana, i przed Jozuego, syna Nunowego, i przed ksiazeta, rzekly: Pan rozkazal Mojzeszowi, aby nam dal dziedzictwo w posród braci naszych; i dal im Jozue wedlug rozkazania Panskiego dziedzictwo w posrodku braci ojca ich.
\par 5 I przypadlo sznurów na Manasesa dziesiec, oprócz ziemi Galaad i Basan, które byly za Jordanem.
\par 6 Albowiem córki Manasesowe otrzymaly dziedzictwo miedzy syny jego, a ziemia Galaad dostala sie drugim synom Manasesowym.
\par 7 I byla granica Manasesowa od Aser do Machmatat, które jest przeciwko Sychem, a idzie granica ta po prawej stronie do mieszkajacych w En Tafua.
\par 8 (Manasesowa byla ziemia Tafua; ale Tafua przy granicy Manasesowej byla synów Efraimowych.)
\par 9 I biezy ta granica do potoku Kana na poludnie tegoz potoku; a miasta Efraimitów sa miedzy miasty Manasesowemi; ale granica Manasesowa idzie od pólnocy onego potoku, a konczy sie u morza.
\par 10 Na poludnie byl dzial Efraimów, a na pólnocy Manasesów, a morze jest granica jego; a w pokoleniu Aser schodza sie na pólnocy, a w Isaschar na wschód slonca.
\par 11 I dostalo sie Manasesowi w pokoleniu Isaschar i w Aser, Betsan i miasteczka jego, i Jeblaam i miasteczka jego; przytem mieszkajacy w Dor i miasteczka ich, takze mieszkajacy w Endor i miasteczka ich; i mieszkajacy tez w Tanach i miasteczka ich, i mieszkajacy w Magiedda i miasteczka ich; trzy powiaty.
\par 12 Ale nie mogli synowi Manasesowi wypedzic z onych miast obywateli; przetoz poczal Chananejczyk mieszkac w onej ziemi.
\par 13 A gdy sie zmocnili synowie Izraelscy, uczynili Chananejczyka holdownikiem: ale go nie wygnali do szczetu.
\par 14 Tedy rzekli synowie Józefowi do Jozuego, mówiac: Przeczzes nam dal w dziedzictwo los jeden, i sznur jeden? a mysmy lud wielki i dotad blogoslawil nam Pan.
\par 15 I rzekl do nich Jozue: Jezlis jest ludem wielkiem, idzze do lasu, a wysiecz sobie tam miejsca w ziemi Ferezejskiej, i Refaimskiej, jezlic ciasna góra Efraimowa.
\par 16 Któremu odpowiedzieli synowie Józefowi: Nie dosyc nam na tej górze; do tego wozy zelazne sa u wszystkich Chananejczyków, którzy mieszkaja w ziemi nadolnej, i u tych, którzy mieszkaja w Betsan i w miasteczkach jego, takze u tych, którzy mieszkaja w dolinie Jezreel.
\par 17 RZekl tedy Jozue do domu Józefowego, do Efraima i do Manasesa, mówiac: Ludes ty wielki, i moc twoja wielka, nie bedziesz mial tylko losu jednego.
\par 18 Ale góre bedziesz mial; a iz tam jest las, tedy go wyrabiesz, i bedziesz mial granice jego; bo wypedzisz Chananejczyka, choc ma wozy zelazne i choc jest potezny.

\chapter{18}

\par 1 Tedy sie zebralo wszystko zgromadzenie synów Izraelskich do Sylo, i postawili tam namiot zgromadzenia, gdy ziemia byla od nich opanowana.
\par 2 A zostalo bylo z synów Izraelskich, którym bylo nie oddzielono dziedzictwa ich, siedmioro pokolenia.
\par 3 Tedy rzekl Jozue do synów Izraelskich: Dokadze zaniedbywacie wnijsc, abyscie posiedli ziemie, która wam dal Pan, Bóg ojców waszych?
\par 4 Obierzcie miedzy soba po trzech meza z kazdego pokolenia, które posle, aby wstawszy obeszli ziemie, a rozpisali ja wedlug dziedzictwa ich, potem sie wróca do mnie.
\par 5 I rozdziela ja na siedem czesci: Juda stanie na granicach swoich od poludnia, a dom Józefów stanie na granicach swoich od pólnocy.
\par 6 Wy tedy rozpiszecie ziemie na siedem czesci a przyniesiecie tu do mnie: tedy wam rzuce los tu przed Panem, Bogiem naszym.
\par 7 Albowiem Lewitowie nie maja dzialu miedzy wami, gdyz kaplanstwo Panskie jest dziedzictwo ich; ale Gad, i Ruben, i polowa pokolenia Manasesowego wzieli dziedzictwa swe za Jordanem na wschód slonca, które im oddal Mojzesz, sluga Panski.
\par 8 Przetoz wstawszy mezowie oni odeszli; a Jozue rozkazal tym, którzy szli, aby rozpisali ziemie, mówiac: Idzcie a obejdzcie ziemie, i popiszcie ja, a potem wróccie sie do mnie, a tu wam rzuce los przed PAnem w Sylo.
\par 9 Odeszli tedy mezowie oni i obchodzili ziemie, i opisywali ja wedlug miast na siedem czesci w ksiegi; potem sie wrócili do Jozuego, do obozu w Sylo.
\par 10 Rzucil im los Jozue w Sylo przed Panem, a podzielil tam Jozue ziemie synom Izraelskim wedlug dzialów ich.
\par 11 Tedy padl los pokoleniu synów Benjaminowych wedlug domów ich, a przyszla granica losu ich miedzy syny Judowe, i miedzy syny Józefowe.
\par 12 I byla granica ich ku stronie pólnocnej od Jordanu, a szla taz granica po bok Jerycha od pólnocy, ciagnac sie na góre ku zachodowi, a konczyla sie przy puszczy Betawen.
\par 13 A stamtad idzie ta granica do Luz, od strony poludniowej Luzy, która jest Betel, a puszcza sie ta granica do Attarot Adar podle góry, która jest od poludnia Betoron dolnego.
\par 14 I biezy ta granica kolem po bok morza na poludnie od góry, która jest przeciw Betoron, na poludnie, i konczy sie w Karyjat Baal, które jest Karyjat Jarym, miasto synów Judowych; a toc jest strona zachodnia.
\par 15 Strona zasie na poludnie od konca Karyjat Jarym; a wychodzi ta granica ku morzu, i biezy ku zródlu wód Neftoa.
\par 16 I ciagnie sie ta granica do konca góry, która jest przeciwko dolinie synów Ennon, a jest w dolinie Refaim na pólnocy, i idzie przez doline Refaim na pólnocy, i idzie przez doline Ennon po stronie Jebuzejczyka na poludnie, stamtad biezy do zródla Rogiel.
\par 17 A idzie kolem od pólnocy, a dochodzi do Ensemes, a wychodzi do Gelilot, które jest przeciwko górze, wstepujac do Adommim, biezac stamtad do kamienia Bohena, syna Rubenowego.
\par 18 Stamtad idzie ku stroni, która jest przeciwko równinom na pólnocy, i ciagnie sie ku Araba.
\par 19 Stamtad biezy ta granica ku stronie Betogla na pólnocy, a konczy sie u skaly morza slonego na pólnocy, ku koncowi Jordanu na poludnie; toc jest granica poludniowa.
\par 20 Jordan zas konczy ja ku stronie na wschód slonca; a toc jest dziedzictwo synów Benjaminowych wedlug granic ich w okrag, wedle domów ich.
\par 21 Byly tedy te miasta pokolenia synów Benjaminowych wedlug domów ich: Jerycho i Betagal, i dolina Kasys.
\par 22 I Betaraba, i Samraim, i Betel;
\par 23 I Awim, i Afara, i Ofera;
\par 24 I Kafar Hammonaj, i Ofni, i Gaba, i miast dwanascie, i wsi ich;
\par 25 Gabon, i Rama, i Berot;
\par 26 I Misfe, i Kafara, i Mosa;
\par 27 I Rekiem, i Jerefel, i Tarela;
\par 28 I Sela, Elef, i Jebuz (które jest Jeruzalem), Gibeat, Kiryjat, miast czternascie, i wsi ich. toc jest dziedzictwo synów Benjaminowych wedlug domów ich.

\chapter{19}

\par 1 Potem padl los wtóry Symeonowi, pokoleniu synów Symeonowych wedlug domów ich, a bylo dziedzictwo ich w posród dziedzictwa synów Judowych.
\par 2 A dostalo sie im w dziedzictwo ich Beerseba, i Seba, i Molada;
\par 3 I Hasersual, i Bala, i Asem;
\par 4 I Etolat, i Betul, i Horma;
\par 5 I Syceleg, i Bet Marchabot, i Hasersusa,
\par 6 I Betlebaot, i Serohem, i trzynascie miast, i wsi ich;
\par 7 Ain, Remmon, i Atar, i Asan, miasta cztery, i wsi ich;
\par 8 I wszystkie wsi, które byly okolo tych miast, az do Baalatbeer, i Ramat ku stronie poludniowej. Toc jest dziedzictwo pokolenia synów Symeonowych wedlug domów ich.
\par 9 Z dzialu synów Judowych dostalo sie dziedzictwo synom Symeonowym, bo dzial synów Judowych byl wielki dla nich; przetoz wzieli dziedzictwo synowie Symeonowi posród dziedzictwa ich.
\par 10 Potem padl los trzeci synom Zabulonowym wedlug domów ich, a jest granica dziedzictwa ich.
\par 11 A idzie granica ich morza Marala, i przychodzi do Debbaset, ciagnac sie az do potoku, który jest przeciw Jeknoam.
\par 12 I wraca sie od Saryd na wschód slonca ku granicy Chasalek Tabor, a stamtad biezy do Daberet, i ciagnie sie do Jafije;
\par 13 Potem stamtad biezy na wschód slonca do Gethefer i do Itakasyn, a wychodzi w Rymmon, i kolem idzie do Nehy.
\par 14 Idzie takze kolem taz granica od pólnocy ku Hannaton, a konczy sie u doliny Jeftael.
\par 15 I Katet, i Nahalal, i Symeron, i Jedala, i Betlehem, miast dwanascie, i wsi ich.
\par 16 Toc jest dziedzictwo synów Zabulonowych wedlug domów ich, te miasta i wsi ich.
\par 17 Isascharowi tez padl los czwarty, to jest, synom Isascharowym wedlug domów ich.
\par 18 A byla granica ich Jezreel, i Chasalot, i Sunem.
\par 19 I Hafaraim, i Seon, i Anaharat;
\par 20 I Rabbot, i Cesyjom, i Abes;
\par 21 I Ramet, i Engannim, i Enhadda, i Betfeses.
\par 22 A przychodzi granica ich do Taboru, i do Sehesyma, i do Betsemes, a koncza sie granice ich u Jordanu, miast szesnascie, i wsi ich.
\par 23 Toc jest dziedzictwo pokolenia synów Isascharowych wedlug domów ich, te miasta i wsi ich.
\par 24 Potem padl los piaty pokoleniu synów Asur wedlug domów ich.
\par 25 I byla granica ich: Helkat, i Chali, i Beten, i Achsaf;
\par 26 I Elmelech, i Amaad, i Aessal, a idzie na Karmel do morza, i do Sychor, i Lobanat.
\par 27 Stamtad sie obraca na wschód slonca ku Betdagon, i biezy az do Zabulon, i do doliny Jeftach El na pólnocy, Betemek i do Nehyjel, wychodzac do Kabul ku lewej stronie;
\par 28 I do Hebronu, i Rohob, i Hamon, i Kana, az do Sydonu wielkiego.
\par 29 A wraca sie ta granica od Rama az do miasta Zor obronnego; stamtad sie obraca ta granica az do Hosa, a konczy sie u morza podle dzialu Achsyba.
\par 30 I Amma, i Afek, i Rohob, miast dwadziescia i dwa, i wsi ich.
\par 31 Toc jest dziedzictwo pokolenia synów Aser wedlug domów ich; te miasta i wsi ich.
\par 32 Potem synom Neftalimowym padl los szósty, synom Neftalimowym wedlug domów ich.
\par 33 I byla granica ich od Helef, i od Helon, do Saannanim, i Adami, które jest Necheb, i Jebnael, az ku Lekum, i konczy sie u Jordanu.
\par 34 Potem sie obraca ta granica ku morzu do Asanot Tabor; a stamtad biezy ku Hukoka, i idzie do Zabulonu na poludnie, a do Asar przychodzi ku zachodu, a do Juda ku Jordanowi na wschód slonca.
\par 35 A miasta obronne sa: Assedym Ser, i Emat, Rekat, i Cyneret;
\par 36 I Edama, i Arama, i Asor,
\par 37 I Kiedes, i Edrej, i Enhasor;
\par 38 I Jeron, i Magdalel, Horem, i Betanat, i Betsemes, miast dziewietnascie, i wsi ich.
\par 39 Toc jest dziedzictwo pokolenia synów Neftalimowych wedlug domów ich; te miasta i wsi ich.
\par 40 Potem pokolen synów Dan wedlug domów ich, padl los siódmy.
\par 41 A byla granica dziedzictwa ich: Saraa, i Estaol, i Isremes;
\par 42 I Selebim, i Ajalon, i Jetela;
\par 43 I Elon, i Temnata, i Ekron;
\par 44 I Eltekie, i Gebbeton i Baalat;
\par 45 I Jehut, i Bane Barak, i Getremmon;
\par 46 I Mehajarkon, i Rakon z granica przeciwko Joppie.
\par 47 Ale granica synów Danowych byla bardzo mala; przetoz wyszedlszy synowie Dan dobywali Lesem, i wzieli je, i wysiekli je ostrzem miecza, i wziawszy je w dziedzictwo mieszkali w niem; i przezwali Lesem Dan wedlug imienia Dana, ojca swego.
\par 48 Toc jest dziedzictwo pokolenia synów Danowych wedlug domów ich; te miasta, i wsi ich.
\par 49 A gdy przestali dzielic ziemie wedlug granic jej, tedy dali synowie Izraelscy dziedzictwo Jozuemu, synowi Nunowemu, w posród siebie.
\par 50 Wedlug rozkazania Panskiego dali mu miasto, którego zadal, Tamnat Saraa na górze Efraim, gdzie zbudowal miasto, i mieszkal w niem.
\par 51 Tec sa dziedzictwa, które losem podzielili w osiadlosc Eleazar kaplan, i Jozue, syn Nunów, i przedniejsi z ojców pokolenia synów Izraelskich w Sylo przed Panem, u drzwi namiotu zgromadzenia, i dokonczyli podzialu ziemi.

\chapter{20}

\par 1 POtem rzekl Pan do Jozuego, mówiac:
\par 2 Powiedz synom Izraelskim, i rzecz: Oddzielcie sobie miasta ucieczki, o którychem mówil do was przez Mojzesza;
\par 3 Aby tam uciekl mezobójca, coby zabil czlowieka nie chcac, z niewiadomosci; i beda wam dla ucieczki przed tym, któryby sie krwi chcial mscic.
\par 4 I uciecze do jednego z miast, a stanie u wrót bramy miejskiej, i opowie starszym miasta onego sprawe swoje; i przyjma go do miasta miedzy sie, i dadza mu miejsce, a bedzie mieszkal z nimi.
\par 5 A gdy go bedzie gonil ten, któryby sie chcial mscic krwi, tedy nie wydadza mezobójcy w rece jego; albowiem nie chcac zabil blizniego swego, a nie majac zadnej wasni, z nim przedtem.
\par 6 I bedzie mieszkal w onem miescie, a stanie przed zebraniem na sad, i az do smierci kaplana wielkiego, który bedzie za onych dni; tedy sie wróci mezobójca, i przyjdzie do miasta swego i do domu swego, do miasta, z którego uciekl.
\par 7 I oddzieli Kades w Galilei na górze Neftali, a Sychem na górze Efraim, i miasto Arba, które jest Hebron, na górze Juda.
\par 8 Z drugiej zasie strony Jordanu, gdzie lezy Jerycho od wschodu slonca, oddzieli Bosor na puszczy, w równinie z pokolenia Rubenowego, i Ramot w Galaad z pokolenia Gad, przytem Golan w Basen z pokolenia Manasesowego.
\par 9 Tec byly miasta dla ucieczki wszystkim synom Izraelskim, i cudzoziemcom, którzy mieszkali w posrodku ich, aby tam uciekl kazdy, kto by kogo zabil z nieobaczenia, a nie byl zamordowan przez tego, któryby sie krwi chcial mscic, azby pierwej stanal przed zgromadzeniem.

\chapter{21}

\par 1 Przystapili tedy przedniejsi z ojców Lewitów do Eleazara kaplana, i do Jozuego, syna Nunowego, i do przedniejszych z ojców w pokoleniach synów Izraelskich.
\par 2 I rzekli do nich w Sylo, w ziemi Chananejskiej, mówiac: Pan rozkazal przez Mojzesza, abyscie nam dali miasta ku mieszkaniu z przedmiesciami ich dla dobytków naszych.
\par 3 Przetoz dali synowie Izraelscy Lewitom z dziedzictwa swego wedlug slowa Panskiego te miasta, i przedmiescia ich.
\par 4 Padl tedy los na domy Kaatytów; i dostalo sie synom Aarona kaplana, Lewitom z pokolenia Judowego i z pokolenia Symeonowego, i z pokolenia Benjaminowego, losem miast trzynascie.
\par 5 A drugim synom Kaatowym z domów pokolenia Efraimowego, i z pokolenia Danowego, i z polowy pokolenia Manasesowego, dostalo sie losem miast dziesiec.
\par 6 A synom Gersonowym z domów pokolenia Isascharowego, i z pokolenia Aserowego, i z pokolenia Neftalimowego, i z polowy pokolenia Manasesowego w Basan dostalo sie losem miast trzynascie.
\par 7 Takze synom Merarego wedlug domów ich, z pokolenia Rubenowego, i z pokolenia Gadowego, i z pokolenia Zabulonowego miast dwanascie.
\par 8 Dali tedy synowie Izraelscy Lewitom te miasta, i przedmiescia ich, jako byl rozkazal Pan przez Mojzesza, losem.
\par 9 A tak dali z pokolenia synów Judowych, i z pokolenia synów Symeonowych te miasta, których tu imiona polozone sa.
\par 10 I dostaly sie synom Aaronowym z domów Kaatowych z synów Lewiego; bo im padl los pierwszy.
\par 11 I dano im miasto Arba, ojca Enakowego, które jest Hebron na górze Juda, i przedmiescia jego okolo niego;
\par 12 Ale role miasta tego, i wsi jego dano Kalebowi, synowi Jefunowemu w osiadlosc jego.
\par 13 Synom tedy Aarona kaplana dano miasto dla ucieczki mezobójcy, Hebron i przedmiescia jego; takze Lobne i przedmiescia jego;
\par 14 I Jeter, i przedmiescia jego; Estemon, i przedmiescia jego;
\par 15 I Helon, i przedmiescia jego, i Dabir, i przedmiescia jego.
\par 16 I Ain, i przedmiescia jego, i Jeta, i przedmiescia jego; Betsemes i przedmiescia jego; miast dziewiec z tegoz dwojga pokolenia.
\par 17 A z pokolenia Benjaminowego Gabaon i przedmiescia jego; Gabae i przedmiescia jego;
\par 18 Anatot i przedmiescia jego; i Almon i przedmiescia jego; miasta cztery.
\par 19 Owa wszystkich miast synów Aaronowych, kaplanów, trzynascie miast i przedmiescia ich.
\par 20 Ale domom synów Kaatowych, Lewitom, którzy byli zostali z synów Kaatowych, dane byly miasta losu ich z pokolenia Efraimowego.
\par 21 A dano im miasto ku ucieczce mezobójcy, Sychem i przedmiescia jego na górze Efraim; i Gazer i przedmiescia jego.
\par 22 I Kibsaim i przedmiescia jego; i Betoron, i przedmiescia jego; miasta cztery.
\par 23 Takze z pokolenia Danowego Elteko i przedmiescia jego; Gabaton i przedmiescia jego;
\par 24 Ajalon i przedmiescia jego; Gatrymon i przedmiescia jego; miasta cztery.
\par 25 A z polowy pokolenia Manasesowego Tanach i przedmiescia jego; i Gatrymon i przedmiescia jego; dwa miasta.
\par 26 Wszystkich miast dziesiec i przedmiescia ich dano domom synów Kaatowych pozostalym.
\par 27 Synom zas Gersonowym z pokolenia Lewiego, od polowy pokolenia Manasesowego, dano miasta dla ucieczki mezobójcy: Golan w Basan i przedmiescia jego, i Bozran i przedmiescia jego; dwa miasta.
\par 28 Z pokolenia Isaschar: Kiesyjon i przedmiescia jego; Daberet i przedmiescia jego;
\par 29 Jaramot i przedmiescia jego, i Engannim i przedmiescia jego; miasta cztery.
\par 30 A z pokolenia Aser: Masaa i przedmiescia jego; Abdon i przedmiescia jego;
\par 31 Helkat i przedmiescia jego, Rohob i przedmiescia jego; miasta cztery.
\par 32 A z pokolenia Neftalimowego dano miasto dla ucieczki mezobójcy, Kades w Galilei i przedmiescia jego; i Hamotdor i przedmiescia jego, takze Kartan i przedmiescia jego; trzy miasta.
\par 33 Wszystkich miast Gersonitów wedlug domów ich bylo trzynascie miast i przedmiescia ich.
\par 34 Potem domom synów Merarego Lewitom ostatnim, z pokolenia Zabulonowego dano Jeknam i przedmiescia jego; Karta i przedmiescia jego.
\par 35 Damna i przedmiescia jego; Nahalol i przedmiescia jego; miasta cztery.
\par 36 A z pokolenia Rubenowego Besor i przedmiescia jego; i Jahasa i przedmiescia jego;
\par 37 Kedemot i przedmiescia jego; i Mefaat i przedmiescia jego; miasta cztery.
\par 38 Nadto z pokolenia Gadowego dano miasta dla ucieczki mezobójcy, Ramod w Galaad i przedmiescia jego, i Mahanaim i przedmiescia jego;
\par 39 Hesebon i przedmiescia jego; Jazer i przedmiescia jego; wszystkich miast cztery.
\par 40 Wszystkich miast synów Merarego wedlug domów ich, którzy jeszcze byli pozostali z domów Lewitów, przyszlo im losem miast dwanascie.
\par 41 A tak wszystkich miast Lewitów w posrodku dziedzictwa synów Izraelskich miast czterdziesci osiem i przedmiescia ich.
\par 42 A mialy te wszystkie miasta, kazde z osobna, przedmiescia okolo siebie; a tak bylo okolo wszystkich onych miast.
\par 43 Dal tedy Pan Izraelowi wszystke ziemie, o która przysiagl, ze ja dac mial ojcom ich; i posiedli ja, a mieszkali w niej.
\par 44 Dal im tez odpoczynek Pan zewszad w okolo, tak jako byl przysiagl ojcom ich; a nie byl nikt, kto by sie im oprzec mógl ze wszystkich nieprzyjaciól ich; wszystkie nieprzyjacioly ich dal Pan w reke ich.
\par 45 Nie chybilo zadne slowo ze wszystkich slów dobrych, które obiecal Pan domowi Izraelskiemu; wszystko sie wypelnilo.

\chapter{22}

\par 1 Tedy przyzwal Jozue Rubenitów i Gadytów, i polowe pokolenia Manasesowego,
\par 2 I rzekl do nich: Wyscie strzegli wszystkiego, co wam rozkazal Mojzesz, sluga Panski, i byliscie posluszni glosowi memu we wszystkiem, com wam rozkazal.
\par 3 Nie opusciliscie braci waszej przez dlugi czas az do dnia tego; alescie strzegli pilnie rozkazania Pana, Boga waszego.
\par 4 A teraz, poniewaz odpoczynek dal Pan, Bóg wasz, braciom waszym, jako im byl obiecal, przetoz teraz wróccie sie, a idzcie do przybytków waszych i do ziemi osiadlosci waszej, która wam dal Mojzesz sluga Panski, przed Jordanem.
\par 5 Tylko strzezcie pilnie, abyscie zachowali przykazanie, i Zakon, który wam rozkazal Mojzesz, sluga Panski: abyscie milowali Pana Boga waszego, a chodzili wszystkiemi drogami jego, chowajac rozkazania jego, dzierzac sie go i sluzac mu ze wszystkiego serca waszego, i ze wszystkiej duszy waszej.
\par 6 I blogoslawil im Jozue, a rozpuscil je; i odeszli do przybytków swoich.
\par 7 Ale polowie pokolenia Manasesowego dal byl Mojzesz osiadlosc w Basan, a drugiej polowie jego dal Jozue dzial z bracia ich z tej strony Jordanu na zachód slonca; a gdy je rozpuszczal Jozue do przybytku ich, blogoslawil im.
\par 8 I rzekl do nich, mówiac: Z wielkiemi bogactwy wracacie sie do przybytków waszych, i z majetnoscia bardzo wielka, ze srebrem i z zlotem, i z miedzia, i z zelazem, i szat bardzo wiela; dzielciez sie lupem nieprzyjaciól waszych z bracia swoja.
\par 9 Tedy wracajac sie odeszli synowie Rubenowi, i synowie Gadowi, i polowa pokolenia Manasesowego od synów Izraelskich z Sylo, które jest w ziemi Chananejskiej, aby szli do ziemi Galaad, do ziemi osiadlosci swojej, która dziedzicznie otrzymali wedlug slowa Panskiego przez Mojzesza.
\par 10 I przyszli do granic Jordanu, które byly w ziemi Chananejskiej, i zbudowali tam synowie Rubenowi, i synowie Gadowi, i polowa pokolenia Manasesowego oltarz nad Jordanem, oltarz wielki na podziw.
\par 11 I uslyszeli synowie Izraelscy, iz powiadano: Oto zbudowali synowie Rubenowi, i synowie Gadowi, i polowa pokolenia Manasesowego oltarz przeciw ziemi Chananejskiej na granicach nad Jordanem, kedy przeszli synowie Izraelscy.
\par 12 To gdy uslyszeli synowie Izraelscy, zeszlo sie wszystko zgromadzenie ich do Sylo, aby sie ruszyli przeciwko nim na wojne.
\par 13 I poslali synowie Izraelscy do synów Rubenowych, i do synów Gadowych, i do polowy pokolenia Manasesowego, do ziemi Galaad, Fineesa, syna Eleazara kaplana.
\par 14 A z nim dziesiec ksiazat, po jednym ksiazeciu z kazdego domu ojcowskiego ze wszystkich pokolen Izraelskich, a kazdy ksiaze z tych byl przedniejszym w domu ojców swoich, w tysiacach Izraelskich.
\par 15 Tedy ci przyszli do synów Rubenowych, i do synów Gadowych, i do polowy pokolenia Manasesowego, do ziemi Galaad, i rzekli do nich, mówiac:
\par 16 Tak mówi wszystko zgromadzenie Panskie: Cóz to jest za przestepstwo, któremescie wystapili przeciwko Bogu Izraelskiemu, zescie sie dzis odwrócili od Pana, budujac sobie oltarz, abyscie dzis byli przeciwnymi Panu?
\par 17 Azaz nam malo na zlosci Fegorowej, od której nie jestesmy oczyszczeni i po dzis dzien, skad byla pomsta w zgromadzeniu Panskiem,
\par 18 Zescie sie dzis odwrócili, zebyscie nie szli za Panem? Zaczem stanie sie, poniewazescie wy dzis odpornymi Panu, ze sie on jutro na wszystko zgromadzenie Izraelskie rozgniewa.
\par 19 A jezliz jest nieczysta ziemia osiadlosci waszej, przeprowadzcie sie do ziemi dziedzictwa Panskiego, w której przebywa przybytek Panski, i wezmiecie osiadlosci w posrodku nas; tylko Panu nie badzcie odpornymi, a nie odpadajcie od nas, budujac sobie oltarz oprócz oltarza Pana, Boga naszego.
\par 20 Azaz przez Achana, syna Zarego, gdy sie dopuscil przestepstwa w rzeczy przekletej, na wszystko zgromadzenie Izraelskie nie przypadl gniew? a nie on sam jeden umarl dla nieprawosci swojej.
\par 21 Tedy odpowiedzieli synowie Rubenowi, i synowie Gadowi, i polowa pokolenia Manasesowego, a mówili do ksiazat tysiaców Izraelskich:
\par 22 Bóg nad Bogi, Pan, Bóg nad Bogi, Pan, on to wie, i Izrael sam pozna, jezli sie to stalo z uporu, albo jezli z przestepstwa przeciw Panu, niechze nas nie zywi dnia tego.
\par 23 Jezlizesmy sobie zbudowali oltarz, abysmy sie odwrócili od Pana, a jezliz ku ofiarowaniu na nim calopalenia, i ofiar sniednych, albo ku sprawowaniu na nim ofiar spokojnych, Pan niech to rozezna;
\par 24 Jezlizesmy nie raczej obawiajac sie tej rzeczy, uczynili to, mówiac: Napotem rzeka synowie wasi synom naszym, mówiac: Cóz wam do Pana, Boga Izraelskiego?
\par 25 Oto granice polozyl Pan miedzy nami i miedzy wami, synowie Rubenowi i synowie Gadowi, Jordan; nie macie wy dzialu w Panu, i odwróca synowie wasi syny nasze od bojazni Panskiej.
\par 26 Przetosmy rzekli: Uczynmy tak, a zbudujmy sobie oltarz, nie dla calopalenia, ani innych ofiar:
\par 27 Ale izby byl swiadkiem miedzy nami i miedzy wami, i miedzy potomstwy naszemi po nas, i abysmy sluzyli Panu przed obliczem jego w calopaleniach naszych, i w sniednych ofiarach naszych, i w spokojnych ofiarach naszych, a izby nie rzekli synowie wasi napotem synom naszym: Nie macie czastki w Panu.
\par 28 Nadtosmy rzekli: Gdyby napotem rzekli nam, albo potomstwu naszemu, tedy rzeczemy: Patrzajcie na podobienstwo oltarza Panskiego, który uczynili ojcowie nasi, nie dla calopalenia, ani innych ofiar, ale zeby on byl swiadkiem miedzy nami i miedzy wami.
\par 29 Boze nas uchowaj, zebysmy mieli przeciwnymi byc Panu, a odstapic dzis od Pana, zbudowawszy oltarz dla calopalonych, dla sniednych i dla innych ofiar, oprócz oltarza Pana, Boga naszego, który jest przed przybytkiem jego.
\par 30 A uslyszawszy Finees kaplan, i ksiazeta zgromadzenia, i przelozeni nad tysiacmi Izraelskimi, którzy z nim byli, slowa, które mówili synowie Rubenowi, i synowie Gadowi, i synowie Manasesowi, podobalo sie im to.
\par 31 I rzekl Finees, syn Eleazara kaplana, do synów Rubenowych i do synów Gadowych i do synów Manasesowych: Dzisiajsmy poznali, iz w posrodku nas jest Pan, izescie sie nie dopuscili przeciw Panu przestepstwa tego, i wyswobodziliscie syny Izraelskie z reki Panskiej.
\par 32 A tak wrócili sie Finees, syn Eleazara kaplana, z onymi ksiazety od synów Rubenowych i do synów Gadowych z ziemi Galaad do ziemi Chananejskiej do synów Izraelskich, i odniesli im te rzecz.
\par 33 I podobalo sie to synom Izraelskim; a blogoslawili Boga synowie Izraelscy, i nie mówili wiecej, zeby mieli isc przeciwko nim na wojne, i wytracic ziemie, w której synowie Rubenowi i synowie Gadowi mieszkali.
\par 34 Przezwali tedy synowie Rubenowi i synowie Gadowi oltarz on Ed, mówiac: Swiadkiem bedzie miedzy nami, ze Pan jest Bogiem.

\chapter{23}

\par 1 I stalo sie po niemalym czasie, gdy odpoczynek dal Pan Izraelowi od wszystkich nieprzyjaciól ich okolicznie, a Jozue sie zstarzal, i byl zeszlym w leciech,
\par 2 Ze przyzwal Jozue wszystkiego Izraela, starszych jego, i przedniejszych jego, i sedziów jego, i przelozonych jego, i rzekl do nich. Jam sie zstarzal a zszedlem w leciech.
\par 3 A wyscie widzieli wszystko, co uczynil Pan, Bóg wasz, wszystkim tym narodom przed obliczem waszem: bo Pan, Bóg wasz sam walczyl za was.
\par 4 Obaczciez, rozdzielilem wam losem te narody pozostale w dziedzictwo miedzy pokolenia wasze od Jordanu, i wszystkie narody, którem wytracil, az do morza wielkiego na zachód slonca.
\par 5 A Pan, Bóg wasz, sam je wypedzi od twarzy waszej, i wyzenie je od oblicznosci waszej, i posiedziecie dziedzicznie ziemie ich, jako wam to powiedzial Pan, Bóg wasz.
\par 6 Zmacniajciez sie bardzo, abyscie strzegli a czynili wszystko, co napisano w ksiegach Zakonu Mojzeszowego, nie odstepujac od niego na prawo ani na lewo.
\par 7 Ani sie tez mieszajcie z temi narodami, które zostaly z wami; ani imienia bogów ich nie wspominajcie, ani przysiegajcie przez nie, ani im sluzcie, ani sie im klaniajcie;
\par 8 Ale sie Pana, Boga waszego, trzymajcie, jakoscie czynili az do dnia tego.
\par 9 Bo jako wypedzil Pan od oblicza waszego narody wielkie i mozne, i nie oparl sie wam nikt az do dnia tego:
\par 10 Tak maz jeden z was bedzie uganial tysiac; albowiem Pan, Bóg wasz, on walczy za wami, jako wam obiecal.
\par 11 Przetoz przestrzegajcie z pilnoscia, abyscie milowali Pana, Boga waszego.
\par 12 Bo jezli sie cale odwrócicie, a przystaniecie do tych pozostalych narodów, do tych, które zostawaja miedzy wami, i spowinowacicie sie z nimi, a bedziecie sie mieszac z nimi, one tez z wami:
\par 13 Wiedzciez wiedzac, zec nie bedzie wiecej Pan, Bóg wasz, wyganial tych narodów od twarzy waszej; ale beda wam sidlem, i zawada, i biczem na boki wasze, i cierniem na oczy wasze, póki nie wyginiecie z tej przewybornej ziemi, która wam dal Pan, Bóg wasz.
\par 14 A oto, ja ide dzis w droge wszystkiej ziemi; poznajciez tedy ze wszystkiego serca waszego, i ze wszystkiej duszy waszej, zec nie chybilo zadne slowo ze wszystkich slów najlepszych, które mówil Pan, Bóg wasz o was; wszystkie sie nad wami wypelnily, a nie chybilo z nich zadne slowo.
\par 15 Przetoz jako sie wypelnilo nad wami kazde slowo dobre, które mówil Pan, Bóg wasz, do was, tak przywiedzie Pan na was kazde slowo zle, az was wytraci z ziemi tej przewybornej, która wam dal Pan, Bóg wasz.
\par 16 Jezli przestapicie przymierze Pana, Boga waszego, który wam rozkazal, a szedlszy sluzyc bedziecie bogom obcym, i klaniac sie im bedziecie, tedy sie rozpali popedliwosc Panska przeciwko wam, i zginiecie predko z tej przewybornej ziemi, która wam dal.

\chapter{24}

\par 1 Tedy zebral Jozue wszystkie pokolenia Izraelskie do Sychem, i zwolal starszych z Izraela, i przedniejszych z nich, i sedziów ich, i przelozonych ich, i staneli przed obliczem Bozem.
\par 2 I rzekl Jozue do wszystkiego ludu: Tak mówi Pan, Bóg Izraelski: Za rzeka mieszkali ojcowie wasi od dawnych czasów, Tare, ojciec Abrahamów, i ojciec Nachorów, i sluzyli bogom obcym.
\par 3 I wzialem ojca waszego Abrahama z miejsca, które jest za rzeka, i prowadzilem go przez wszystke ziemie Chananejska, i rozmnozylem nasienie jego, dawszy mu Izaaka.
\par 4 Dalem tez Izaakowi Jakóba i Ezawa, a podalem Ezawowi góre Seir, aby ja posiadl; ale Jakób i synowie jego zaszli do Egiptu.
\par 5 I poslalem Mojzesza i Aarona, a trapilem Egipt; a gdym to uczynil w posród niego, potemem was wywiódl.
\par 6 I wywiodlem ojce wasze z Egiptu, a przyszliscie az do morza, i gonili Egipczanie ojce wasze z wozami i z jezdnymi az do morza czerwonego.
\par 7 Tedy wolali do Pana, a on polozyl ciemnosci miedzy wami i miedzy Egipczany, i przywiódl na nie morze, a okrylo je; i widzialy oczy wasze, com uczynil w Egipcie, i mieszkaliscie na puszczy przez dlugi czas.
\par 8 Potemem przywiódl was do ziemi Amorejczyka, mieszkajacego za Jordanem, i walczyli przeciwko wam; alem je podal w reke wasze, i posiedliscie ziemie ich, a wygladzilem je przed wami.
\par 9 Powstal tez Balak, syn Seforów, król Moabski, aby walczyl przeciw Izraelowi; a poslawszy przyzwal Balaama, syna Beorowego, aby was przeklinal.
\par 10 I nie chcialem sluchac Balaama; przetoz blogoslawiac blogoslawil wam, a tak wybawilem was z rak jego.
\par 11 Przeprawiliscie sie potem przez Jordan, i przyszliscie do Jerycha, i walczyli przeciwko wam mezowie z Jerycha, Amorejczyk, i Ferezejczyk, i Chananejczyk, i Hetejczyk, i Gargiezejczyk, i Hewejczyk, i Jebuzejczyk; alem je podal w rece wasze.
\par 12 I poslalem przed wami sierszenie, którzy je wypedzili przed obliczem waszem, dwu królów Amorejskich, nie mieczem twoim ani lukiem twoim.
\par 13 I dalem wam ziemie, w którejscie nie robili, i miasta, którychescie nie budowali, w których mieszkacie, a winnic i oliwnic, którychescie nie sadzili, pozywacie.
\par 14 Przetoz teraz bójcie sie Pana, a sluzcie mu w doskonalosci i w prawdzie, a zniescie bogi, którym sluzyli ojcowie wasi za rzeka, i w Egipcie, a sluzcie Panu.
\par 15 A jezli sie wam zda zle sluzyc Panu, obierzciez sobie dzis, komu byscie sluzyli, chociaz bogi, którym sluzyli ojcowie wasi, co byli za rzeka, chociaz bogi Amorejskie, w których wy ziemi mieszkacie; alec ja i dom mój bedziemy sluzyli Panu.
\par 16 I odpowiedzial lud mówiac: Nie daj Boze, abysmy mieli odstapic Pana, a sluzyc bogom cudzym:
\par 17 Albowiem Pan, Bóg nasz, on jest, który nas wywiódl, i ojce nasze z ziemi Egipskiej, z domu niewoli, a który uczynil przed oczyma naszemi te znaki wielkie, i strzegl nas we wszystkiej drodze, którasmy szli, i miedzy wszystkimi narody, przez któresmy przeszli;
\par 18 I wypedzil Pan wszystkie narody, i Amorejczyka mieszkajacego w ziemi przed twarza nasza. A tak my bedziemy sluzyli Panu; bo on jest Bóg nasz.
\par 19 Tedy rzekl Jozue do ludu: Nie mozecie wy sluzyc Panu; bo Bóg swiety jest, Bóg zapalczywy jest, nie przepusci zlosciom waszym, ani grzechom waszym.
\par 20 Jezliz opuscicie Pana, a bedziecie sluzyli bogom cudzym, obróci sie, i utrapi was, i zniszczy was, choc wam przedtem dobrze czynil.
\par 21 I odpowiedzial lud Jozuemu: Nie tak; ale Panu sluzyc bedziemy.
\par 22 Tedy rzekl Jozue do ludu: Swiadkami bedziecie sami przeciwko, sobie, izescie sobie obrali Pana, abyscie mu sluzyli; a oni rzekli: Swiadkami jestesmy.
\par 23 I rzekl: Terazze zniescie bogi cudze, którzy sa w posrodku was, a nakloncie serca wasze ku Panu, Bogu Izraelskiemu.
\par 24 I odpowiedzial lud Jozuemu: Panu, Bogu naszemu, sluzyc bedziemy, i glosowi jego posluszni byc chcemy.
\par 25 A tak uczynil Jozue przymierze z ludem dnia onego, i przelozyl im rozkazanie i sad w Sychem.
\par 26 I napisal Jozue slowa te w ksiegi Zakonu Bozego; wzial tez kamien wielki, i postawil go tam pod debem, który byl u swiatnicy Panskiej.
\par 27 Tedy rzekl Jozue do wszystkiego ludu: Oto kamien ten bedzie nam swiadectwem; albowiem on slyszal wszystkie slowa Panskie, które mówil z nami i bedzie przeciwko wam na swiadectwo, byscie snac nie sklamali przeciwko Bogu waszemu.
\par 28 Zatem rozpuscil Jozue lud, kazdego do dziedzictwa swego.
\par 29 I stalo sie potem, ze umarl Jozue, syn Nunów, sluga Panski, we stu i w dziesieciu lat.
\par 30 I pogrzebali go na granicy dziedzictwa jego w Tamnat Sare, które jest na górze Efraim, ku pólnocy góry Gaas.
\par 31 I sluzyl Izrael Panu po wszystkie dni Jozuego, i po wszystkie dni starszych, którzy dlugo zyli po Jozuem, a którzy wiedzieli o wszystkich sprawach Panskich, które czynil Izraelowi.
\par 32 Kosci tez Józefowe, które byli przeniesli synowie Izraelscy z Egiptu, pogrzebali w Sychem, na czesci pola, które byl kupil Jakób od synów Hemora, ojca Sychemowego, za sto jagniat; i byly u synów Józefowych w dziedzictwie ich.
\par 33 Eleazar takze, syn Aaronów, umarl; i pogrzebali go na pagórku Fineesa, syna jego, który mu byl dany na górze Efraim.


\end{document}