\begin{document}

\title{Przysłów}


\chapter{1}

\par 1 Przypowiesci Salomona, syna Dawidowego, króla Izraelskiego,
\par 2 Dla poznania madrosci i cwiczenia, ku wyrozumieniu powiesci roztropnych;
\par 3 Dla pojecia cwiczenia w rozumie, w sprawiedliwosci, w sadzie i w prawosci;
\par 4 Dla podania prostakom ostroznosci, mlodemu umiejetnosci, i opatrznosci.
\par 5 Tych gdy madry sluchac bedzie, przybedzie mu nauki, a roztropny w radach opatrzniejszy bedzie,
\par 6 Aby zrozumial przypowiesci, i wyklady ich, slowa madrych i zagadki ich.
\par 7 Bojazn Panska jest poczatkiem umiejetnosci; ale glupi madroscia i cwiczeniem gardza.
\par 8 Sluchaj, synu mój! cwiczenia ojca twego, a nie opuszczaj nauki matki twojej.
\par 9 Bo to przyda wdziecznosci glowie twojej, i bedzie lancuchem kosztownym szyi twojej.
\par 10 Synu mój! jezliby cie namawiali grzesznicy, nie przyzwalaj.
\par 11 Jezlicby rzekli: Pójdz z nami, czyhajmy na krew, zasadzmy sie na niewinnego bez przyczyny;
\par 12 Pozremyz ich zywo, jako grób, a calkiem, jako zstepujacych w dól;
\par 13 Wszelkiej majetnosci kosztownej nabedziemy, napelnimy domy nasze korzyscia;
\par 14 Rzuc miedzy nas los twój; mieszek jeden wszyscy miec bedziemy.
\par 15 Synu mójâ nie chodzze z nimi w droge; zawsciagnij nogi twojej od sciezek ich.
\par 16 Albowiem nogi ich ku zlemu bieza, i spiesza sie na wylanie krwi.
\par 17 Bo jako prózno zastawiaja sieci przed oczyma wszelkiego ptaka skrzydlastego:
\par 18 Tak i ci na krew swoje czyhaja, a zasadzaja sie na dusze swoje.
\par 19 Takiec sa scieszki kazdego czyhajacego na zysk, który dusze pana swego odbiera.
\par 20 Madrosc na dworzu wola, glos swój na ulicach wydaje.
\par 21 W najwiekszym zgielku wola, u wrót bram, w miastach powiesci swoje opowiada, mówiac:
\par 22 Prostacy! dokadze sie kochac bedziecie w prostocie? a nasmiewcy posmiewisko milowac bedziecie? a glupi nienawidziec umiejetnosci bedziecie?
\par 23 Nawrócciez sie na karanie moje; oto wam wydam ducha mojego, a podam wam do znajomosci slowa moje.
\par 24 Poniewazem wolala, a nie chcieliscie; wyciagalam reke moje, a nie byl, ktoby uwazal;
\par 25 Owszem odrzuciliscie wszystke rade moje, a karnosci mojej nie chcieliscie przyjac;
\par 26 Przetoz ja w zginieniu waszem smiac sie bede, bede z was szydzila, gdy przyjdzie, czego sie strachacie.
\par 27 Gdy przyjdzie jako spustoszenie, czego sie strachacie, i gdy zginienie wasze przypadnie jako wicher, gdy przyjdzie na was ucisk i utrapienie;
\par 28 Tedy mie wzywac beda, a nie nie wyslucham; szukac mie beda z poranku, a nie znajda mie.
\par 29 Przeto, iz mieli w nienawisci umiejetnosc, a bojazni Panskiej nie obrali sobie,
\par 30 Ani przestawali na radzie mijej, ale gardzili wszelka karnoscia moja:
\par 31 Przetoz beda uzywac owocu dróg swoich, a radami swemi nasyceni beda.
\par 32 Bo odwrócenie prostaków pozabija ich, a szczescie glupich wytraci ich.
\par 33 Ale kto mie slucha, bezpiecznie mieszkac bedzie, a bedzie wolny od strachu zlych rzeczy.

\chapter{2}

\par 1 Synu mój! jezli przyjmiesz slowa moje, a przykazanie moje zachowasz u siebie;
\par 2 Nadstawiszli madrosci ucha twego, i nakloniszli serca twego do roztropnosci;
\par 3 Owszem, jezli na rozum zawolasz, a roztropnosci wezwieszli glosem swoim;
\par 4 Jezli jej szukac bedziesz jako srebra, a jako skarbów skrytych pilnie szukac bedziesz:
\par 5 Tedy zrozumiesz bojazn Panska, a znajomosc Boza znajdziesz.
\par 6 Albowiem Pan daje madrosc, z ust jego pochodzi umiejetnosc i roztropnosc.
\par 7 On zachowuje uprzejmym prawdziwa madrosc; on jest tarcza chodzacym w szczerosci,
\par 8 Aby strzegli sciezek sadu; on drogi swietych swoich strzeze.
\par 9 Tedy wyrozumiesz sprawiedliwosc, i sad, i prawosc, i wszelka scieszke dobra.
\par 10 Gdy wnijdzie madrosu w serce twoje, a umiejetnosc duszy twojej wdzieczna bedzie:
\par 11 Tedy cie ostroznosc strzedz bedzie, a opatrznosc zachowa cie.
\par 12 Wyrywajac cie od drogi zlej, i od czlowieka mówiacego przewrotnosci;
\par 13 Od tych, którzy opuszczaja scieszki proste, udawajac sie drogami ciemnemi;
\par 14 Którzy sie raduja, gdy czynia zle, a wesela sie w zlosliwych przewrotnosciach;
\par 15 Których scieszki sa krzywe, a sami sa przewrotnymi na drogach swoich;
\par 16 Wyrywajac cie od niewiasty postronnej i obcej, która pochlebia lagodnemi slowy;
\par 17 Która opuszcza wodza mlodosci swojej, a przymierza Boga swojego zapomina.
\par 18 Bo sie nachyla ku smierci dom jej, a do umarlych scieszki jej.
\par 19 Wszyscy, którzy do niej wchodza, nie wracaja sie, ani trafiaja na scieszke zywota.
\par 20 A przetoz bedziesz chodzil droga dobrych, a sciezek sprawiedliwych bedziesz przestrzegal.
\par 21 Albowiem cnotliwi beda mieszkali na ziemi, a szczerzy trwac beda na niej;
\par 22 Ale niepobozni z ziemi wykorzenieni beda, a przewrotni beda z niej wygladzeni.

\chapter{3}

\par 1 Synu mój! nie zapominaj zakonu mego, a przykazan moich niech strzeze serce twoje.
\par 2 Boc dlugosci dni i lat zywota, i pokoju przyczynia.
\par 3 Milosierdzie i prawda niech cie nie opuszczaja; uwiaz je u szyi twojej, napisz je na tablicy serca twojego.
\par 4 Tedy znajdziesz laske i rozum dobry przed oczyma Bozemi i ludzkiemi.
\par 5 Ufaj w Panu ze wszystkiego serca twego, a na rozumie twoim nie spolegaj.
\par 6 We wszystkich drogach twoich znaj go, a on prostowac bedzie scieszki twoje.
\par 7 Nie badz madrym sam u siebie; ale sie bój Pana, a odstap od zlego.
\par 8 To bedzie zdrowiem zywotowi twemu, a odwilzeniem kosciom twoim.
\par 9 Czcij Pana z majetnosci twojej, i z pierwiastek wszystkich dochodów twoich.
\par 10 A gumna twoje napelnione beda obfitoscia, i od wina nowego prasy twoje rozpadac sie beda.
\par 11 Synu mój! karania Panskiego nie odrzucaj, i nie uprzykrzaj sobie cwiczenia jego.
\par 12 Bo kogo Pan miluje, tego karze, a to jako ojciec, który sie w synu kocha.
\par 13 Blogoslawiony czlowiek, który znajduje madrosc, i czlowiek, który dostanie roztropnosci.
\par 14 Bo lepiej nia kupczyc, nizeli kupczyc srebrem: owszem pozyteczniejszy nad zloto dochód jej.
\par 15 Drozsza jest nad perly, a wszystkie najmilsze rzeczy twoje nie zrównaja sie z nia.
\par 16 Przedluzenie dni w prawicy jej, a w lewicy jej bogactwa i zacnosc.
\par 17 Drogi jej rozkoszne, i wszystkie scieszki jej spokojne.
\par 18 Drzewem zywota jest tym, którzyby sie jej chwycili; a którzy sie jej trzymaja, sa blogoslawionymi.
\par 19 Pan madroscia ugruntowal ziemie, a roztropnoscia umocnil niebiosa.
\par 20 Umiejetnoscia jego rozstapily sie przepasci, a obloki rosa kropia.
\par 21 Synu mój! niech to nie odstepuje od oczów twych: strzez prawdziwej madrosci i roztropnosci;
\par 22 I beda zywotem duszy twojej, a ozdoba szyi twojej.
\par 23 Tedy bedziesz chodzil bezpiecznie droga twoja, a noga twoja nie potknie sie.
\par 24 Jezli sie ukladziesz, nie bedziesz sie lekal; a gdy sie uspokoisz, wdzieczny bedzie sen twój.
\par 25 Nie ulekniesz sie strachu naglego, ani spustoszenia bezbozników, gdy przyjdzie.
\par 26 Albowiem Pan bedzie ufaniem twojem, a nogi twojej bedzie strzegl od samolówki.
\par 27 Nie zbraniaj sie dobrze czynic potrzebujacemu, gdy cie na to stanie, abys dobrze czynil.
\par 28 Nie mów blizniemu twemu: Idz, a wróc sie, a jutroc dam; gdyz to masz u siebie.
\par 29 Nie knuj zlego przeciwko blizniemu twemu, gdyz on z toba dowiernie mieszka.
\par 30 Nie wadz sie z czlowiekiem bez przyczyny, jezlizec nic zlego nie wyrzadzil.
\par 31 Nie zajrzyj mezowi gwalt czyniacemu, a nie obieraj zadnej drogi jego.
\par 32 Albowiem przewrotny jest obrzydliwoscia przed Panem; ale z szczerymi tajemnica jego;
\par 33 Przeklestwo Panskie jest w domu niezboznika; ale przybytkowi sprawiedliwych blogoslawi,
\par 34 Poniewaz on szydzi z posmiewców, ale pokornym laske daje.
\par 35 Madrzy dziedzicznie slawe osieda, ale glupi odniosa zelzywosc.

\chapter{4}

\par 1 Sluchajcie synowie! cwiczenia ojcowskiego, a pilnujcie, abyscie umieli roztropnosc;
\par 2 Albowiem wam nauke dobra daje; zakonu mego nie opuszczajcie.
\par 3 Gdybym byl mlodziuchnym synem u ojca mego, i jedynakiem u matki mojej,
\par 4 On mie uczyl, powiadajac mi: Niech sie chwyci powiesci moich serce twoje, strzez przytkazan moich, a bedziesz zyl.
\par 5 Nabywaj madrosci, nabywaj roztropnosci; nie zapominaj, ani sie uchylaj od powiesci ust moich.
\par 6 Nie opuszczaj jej, a bedzie cie strzegla; rozmiluj sie jej, a zachowa cie.
\par 7 Poczatkiem wszystkiego jest madrosc, nabywajze madrosci, a za wzystke majetnosc twoje nabywaj roztropnosci.
\par 8 Wywyzszaj ja, a wywyzszy cie, rozslawi cie, gdy ja przyjmiesz.
\par 9 Przyda glowie twojej wdziecznosci, korona ozdoby obdarzy cie.
\par 10 Sluchaj, synu mój! a przyjmij powiesci moje, a rozmnozac lata zywota.
\par 11 Drogi madrosci nauczam cie; po scieszkach prostych wiode cie;
\par 12 Któremi gdy pójdziesz, nie bedzie scisniony chód twój; a jezli pobiezysz, nie potkniesz sií.
\par 13 Przyjmij owiczenie, nie puszczaj sie go, strzez go; albowiem ono jest zywotem twoim.
\par 14 Scieszka niepoboznych nie chodz, a nie udawaj sie droga zlosliwych.
\par 15 Opusc ja, nie chodz po niej; uchyl sie od niej, a omin ja.
\par 16 Boc oni nie zasna, az co zlego zbroja; ani sie uspokoja, az kogo do upadku przywioda;
\par 17 Albowiem jedza chleb niezboznosci, a wino drapiestwa pija,
\par 18 Ale scieszka sprawiedliwych jako swiatlosc jasna, która im dalej tem bardziej swieci, az do dnia doskonalego.
\par 19 Droga zas niepoboznych jest jako ciemnosc; nie wiedza, o co sií otracic moga.
\par 20 Synu mój! slów moich pilnuj; ku powiesciom moim naklon ucha twojego.
\par 21 Niech nie odchodza od oczów twoich, zachowaj je w posród serca twego.
\par 22 Albowiem zywotem sa tym, którzy je znajduja, a wszystkiemu cialu ich lekarstwem.
\par 23 Nad wszystko, czego ludzie strzega, strzez serca twego; bo z niego zywot pochodzi.
\par 24 Oddal od siebie przewrotnosc ust, a zlosliwe wargi oddal od siebie.
\par 25 Oczy twoje niechaj na dobre rzeczy patrza, a powieki twoje niech droge przed toba prostuja.
\par 26 Umiarkuj sciezke nóg twoich, aby wszystkie drogi twoje pewne byly.
\par 27 Nie uchylaj sie na prawo ani na lewo; owszem, odwróc noge twoje od zlego.

\chapter{5}

\par 1 Synu mój! badz pilen madrosci mojej, a ku mojej roztropnosci naklon ucha twego,
\par 2 Abys strzegl ostroznosci, a umiejetnosc aby wargi twoje zachowala.
\par 3 Bo choc niewiasty obcej wargi miodem oplywaja, a gladsze niz oliwa usta jej:
\par 4 Ale ostatnie rzeczy jej gorzkie jak piolun, a ostre jako miecz na obie strony ostry.
\par 5 Nogi jej zstepuja do smierci, a do piekla chód jej prowadzi.
\par 6 Jezlibys zwazyc chcial scieszke zywota jej, nie pewne sa drogi jej, nie poznasz ich.
\par 7 Przetoz teraz, synowie! sluchajcie mie, a nie odstepujcie od powiesci ust moich.
\par 8 Oddal od niej droge twoje, a nie przyblizaj sie ku drzwiom domu jej.
\par 9 Bys snac nie podal obcym slawy twojej, a lat twoich okrutnikowi;
\par 10 By sie snac nie nasycili obcy sila twoja, a prace twoje nie zostaly w domu cudzym;
\par 11 I narzekalbys w ostateczne czasy twoje, gdybys zniszczyl czerstwosc twoje i cialo twoje;
\par 12 I rzeklbys: O jakozem mial cwiczenie w nienawisci, a strofowaniem gardzilo serce moje!
\par 13 Nie sluchalem glosu cwiczacych mie, a tym, którzy mie uczyli, nie naklanialem ucha mego!
\par 14 Maluczkom nie przyszedl we wszystko nieszczescie, w posród zebrania i zgromadzenia.
\par 15 Pij wode ze zdroju twego, a wody plynace ze zródla twego!
\par 16 Niech sie precz rozchodza zródla twoje, a po ulicach strumienie wód.
\par 17 Miej je sam dla siebie, a nie obcy z toba.
\par 18 Niech nie bedzie zdrój twój blogoslawiony, a wesel sie z zony mlodosci twojej.
\par 19 Niechzec bedzie jako lani wdzieczna, i sarna rozkodzna; niech cie nasycaja piersi jej na kazdy czas, w milosci jej kochaj sie ustawicznie.
\par 20 Bo przeczze sie masz kochac w obcej, synu mój! i odpoczywac na lonie cudzej?
\par 21 Gdyz przed oczyma Panskiemi sa drogi czlowiecze, a on wszystkie scieszki jego wazy.
\par 22 Nieprawosci wlasne pojmaja niezboznika, a w powrozach grzechu swego uwikle sie.
\par 23 Onci umrze, przeto, ze nie przyjmowal cwiczenia, a dla wielkosci glupstwa swego bedzie bladzil.

\chapter{6}

\par 1 Synu mój! jezlibys reczyl za przyjaciela twego, a dalbys obcemu reke twoje:
\par 2 Usidliles sie slowy ust twoich, pojmanys mowami ust twoich.
\par 3 Przetoz uczyn tak, synu mój! a wyzwól sie, gdyzes wpadl w reke przyjaciela twego; idzze, upokórz sie, a nalegaj na przyjaciela twego.
\par 4 Nie dawaj snu oczom twoim, ani drzemania powiekom twoim.
\par 5 Wyrwij sie jako lani z rak mysliwca i jako ptak z reki ptasznika.
\par 6 Idz do mrówki, leniwcze! obacz drogi jej, a nabadz madrosci;
\par 7 Która, choc nie ma wodza, ani przelozonego, ani pana,
\par 8 Przeciez w lecie gotuje pokarm swój, a zgromadza w zniwa zywnosc swoje.
\par 9 Leniwcze! dokadze lezec bedziesz? kiedyz wstaniesz ze snu swego?
\par 10 Troche sie przespisz, troche podrzemiesz, troche zlozysz rece, abys odpoczywal.
\par 11 A wtem ubóstwo twoje przyjdzie jako podrózny, a niedostatek twój, jako maz zbrojny.
\par 12 Czlowiek niepobozny, maz zlosliwy chodzi w przewrotnosci ust;
\par 13 Mruga oczyma swemi, mówi nogami swemi, ukazuje palcami swemi;
\par 14 Przewrotnosci sa w sercu jego, mysli zle na kazdy czas, a zwady rozsiewa.
\par 15 Przetoz predko przyjdzie upadek jego; nagle skruszony bedzie bez uleczenia.
\par 16 Szesc jest rzeczy, których nienawidzi Pan, a siódma jest obrzydliwoscia duszy jego;
\par 17 Oczów wynioslych, jezyka klamliwego, i rak wylewajacych krew niewinna;
\par 18 Serca, które knuje mysli zle; nóg, które sie kwapia biezec ku zlemu;
\par 19 Swiadka falszywego, który mówi klamstwo, i tego, który sieje rosterki miedzy bracmi.
\par 20 Strzezze, synu mój! przykazania ojca twego, a nie opuszczaj nauki matki twojej.
\par 21 Wiazze je zawzdy u serca twego, a wieszaj je u szyi twojej.
\par 22 Gdziekolwiek pójdziesz, poprowadzi cie; gdy, zasniesz strzedz cie bedzie, a gdy sie ocucisz, rozmawiac z toba bedzie,
\par 23 (Bo przykazanie jest pochodnia, nauka swiatloscia, a droga zywota sa karnosci cwiczenia.)
\par 24 Aby cie strzegly od niewiasty zlej, i od lagodnego jezyka niewiasty obcej.
\par 25 Nie pozadaj pieknosci jej w sercu twojem, a niech cie nie lowi powiekami swemi.
\par 26 Albowiem dla niewiasty wszetecznej zubozeje czlowiek az do kesa chleba; owszem zona cudzolozna droga dusze lowi.
\par 27 Izaz moze kto brac ogien do zanadrzy swoich, aby szaty jego nie zgorzaly?
\par 28 Izaz moze kto chodzic po rozpalonym weglu, aby sie nogi jego nie poparzyly?
\par 29 Tak kto wchodzi do zony blizniego swego, nie bedzie bez winy, ktokolwiek sie jej dotknie.
\par 30 Nie klada hanby na zlodzieja, jezliz co ukradnie, chcac nasycic dusze swoje, bedac glodnym;
\par 31 Ale gdy go zastana, nagradza siedmiorako, albo wszystke majetnosc domu swego daje.
\par 32 Lecz cudzolozacy z niewiasta glupi jest, a kto chce zatracic dusze swoje, ten to czyni.
\par 33 Karanie i zelzywosc odniesie, a hanba jego nie bedzie zgladzona.
\par 34 Bo zawisna milosc jest zapalczywoscia meza, a nie sfolguje w dzien pomsty.
\par 35 Nie bedzie mial wzgledu na zaden okup, ani przyjmie, chociazby mu najwiecej darów dawano.

\chapter{7}

\par 1 Synu mój! strzez slów moich, a przykazanie moje chowaj u siebie.
\par 2 Strzez przykazan moich, a zyc bedziesz; a nauki mojej, jako zrenicy oczów swych.
\par 3 Uwiaz je na palcach twoich, napisz je na tablicy serca twego.
\par 4 Mów madrosci: Siostras ty moja, a roztropnosc przyjaciólka nazywaj,
\par 5 Aby cie strzegly od zony cudzej, i od obcej, która mówi lagodne slowa.
\par 6 Bom oknem domu swego przez krate moje wygladal;
\par 7 I widzialem miedzy prostakami, obaczylem miedzy synami mlodzienca glupiego,
\par 8 Który szedl ulica przy rogu jej, droga postepujac ku domowi jej.
\par 9 Ze zmierzkiem pod wieczór, w ciemnosci nocnej, i w mroku.
\par 10 A oto niewiasta spotkala go, w ubiorze wszetecznicy, chytrego serca,
\par 11 Swiegotliwa i nie ukrócona, a w domu wlasnym nie mogly sie ostac nogi jej;
\par 12 Raz na dworzu, raz na ulicach i po wszystkich katach zasadzki czyniaca;
\par 13 I uchwycila go, i pocalowala go, a zlozywszy wstyd z twarzy swojej, rzekla mu:
\par 14 Ofiary spokojne sa u mnie; dzisiajm oddala sluby moje.
\par 15 Przetozem wyszla przeciw tobie, abym pilnie szukala twarzy twojej, i znalazlam cie.
\par 16 Obilam kobiercami loze moje, ozdobione rzezaniem i przescieradlami egipskiemi.
\par 17 Potrzasnelam pokój swój myrra, aloesem, i cynamonem.
\par 18 Pójdzze, opójmy sie miloscia az do poranku, ucieszmy sie miloscia.
\par 19 Boc meza mego w domu niemasz; pojechal w droge daleka.
\par 20 Worek pieniedzy wzial z soba; dnia pewnego wróci sie do domu swego.
\par 21 I naklonila go wiela slów swoich, a lagodnoscia warg swoich zniewolila go.
\par 22 Wnet poszedl za nia, jako wól, gdy go na rzez wioda, a jako glupi do peta, którem karany bywa.
\par 23 I przebila strzala watrobe jego; kwapil sie jako ptak do sidla, nie wiedzac, iz je zgotowano na dusze jego.
\par 24 Przetoz teraz, synowie! sluchajcie mie, a badzcie pilni powiesci ust moich.
\par 25 Niechaj sie nie uchyla za drogami jej serce twoje, ani sie tulaj po scieszkach jej.
\par 26 Albowiem wielu zraniwszy porazila, i mocarze wszyscy pozabijani sa od niej.
\par 27 Dom jej jest jako drogi piekielne, wiodace do gmachów smierci.

\chapter{8}

\par 1 Izali madrosc nie wola, i roztropnosc nie wydaje glosu swego?
\par 2 Na wierzchu wysokich miejsc, przy drodze i na rozstaniu dróg stoi.
\par 3 U bram, kedy sie chodzi do miasta, i w wejsciu u drzwi wola, mówiac:
\par 4 Na was wolam, o mezowie! a glos mój obracam do synów ludzkich.
\par 5 Zrozumijcie prostacy ostroznosc, a glupi zrozumijcie sercem.
\par 6 Sluchajcie; bo o wielkich rzeczach bede mówil, a otworzenie warg moich opowie szczerosc.
\par 7 Zaistec prawde mówia usta moje, a niezboznosc obrzydliwoscia jest wargom moim.
\par 8 Sprawiedliwe sa wszystkie slowa ust moich; nie masz w nich nic nieprawego ani przewrotnego.
\par 9 Wszystkie sa prawe rozumnemu, a uprzejme tym, którzy znajduja umiejetnosc.
\par 10 Przyjmijciez cwiczenie moje, a nie srebro, a umiejetnosc raczej, niz zloto wyborne.
\par 11 Albowiem lepsza jest madrosc niz perly, takze wszystkie pozadane rzeczy nie porównaja z nia.
\par 12 Ja madrosc mieszkam z roztropnoscia, i umiejetnosc ostroznosci wynajduje.
\par 13 Bojazn Panska jest, miec w nienawisci zle. Ja nienawidze pychy, wysokomyslnosci, i drogi zlej, i ust przewrotnych.
\par 14 Przy mnie jest rada, i prawdziwa madrosc; jam jest roztropnosc, a moc jest moja.
\par 15 Przez mie królowie króluja, i ksiazeta stanowia sprawiedliwosc.
\par 16 Przez mie ksiazeta panuja, i wielmoznymi sa wszyscy sedziowie ziemi.
\par 17 Ja miluje tych, którzy mie miluja; a którzy mie szukaja rano, znajduja mie.
\par 18 Bogactwo i slawa przy mnie jest; majetnosc trwala i sprawiedliwosc.
\par 19 Lepszy jest owoc mój, niz zloto, i niz najkosztowniejsze zloto, a dochody moje lepsze, niz srebro wyborne.
\par 20 Prowadze scieszka sprawiedliwosci, posrodkiem sciezek sadu,
\par 21 Abym tym, którzy mie miluja, dala w dziedzictwo majetnosc wieczna, i skarby ich napelnila.
\par 22 Pan mie mial przy poczatku drogi swej, przed sprawami swemi, przed wszystkiemi czasy.
\par 23 Przed wieki jestem zrzadzona, przed poczatkiem; pierwej niz byla ziemia;
\par 24 Gdy jeszcze nie bylo przepasci, splodzonam jest, gdy jeszcze nie bylo zródel oplywajacych wodami.
\par 25 Pierwej niz góry zalozone byly, niz byly pagórki, splodzonam jest.
\par 26 Jeszcze byl nie uczynil ziemi, i równin, ani poczatku prochu okregu ziemskiego.
\par 27 Gdy gotowal niebiosa, tamem byla; gdy rozmierzal okraglosc nad przepasciami;
\par 28 Gdy utwierdzal obloki w górze, i umacnial zródla przepasci;
\par 29 Gdy zakladal morzu granice jego, i wodom, aby nie przestepowaly rozkazania jego; gdy rozmierzal grunty ziemi:
\par 30 Tedym byla u niego jako wychowaniec, i bylam uciecha jego na kazdy dzien, grajac przed nim na kazdy czas.
\par 31 Gram na okregu ziemi jego, a rozkoszy moje, mieszkac z synami ludzkimi.
\par 32 Sluchajciez mie tedy teraz, synowie! albowiem blogoslawieni, którzy strzega dróg moich.
\par 33 Sluchajcie cwiczenia, nabadzcie rozumu, a nie cofajcie sie.
\par 34 Blogoslawiony czlowiek, który mie slucha, czujac u wrót moich na kazdy dzien, a strzegac podwoi drzwi moich.
\par 35 Bo kto mie znajduje, znajduje zywot, a otrzymuje laske od Pana.
\par 36 Ale kto grzeszy przeciwko mnie, krzywde czyni duszy swojej; wszyscy, którzy mie nienawidza, miluja smierc.

\chapter{9}

\par 1 Madrosc zbudowala dom swój, i wyciosala siedm slupów swoich;
\par 2 Pobila bydlo swoje, roztworzyla wino swoje, i stól swój przygotowala;
\par 3 A rozeslala dzieweczki swoje, wola na wierzchach najwyzszych miejsc w miescie, mówiac:
\par 4 Ktokolwiek jest prostakiem, wstap sam; a do glupich mówi:
\par 5 Pójdzcie, jedzcie chleb mój, i pijcie wino, którem roztworzyla.
\par 6 Opusccie prostote, a bedziecie zyli, a chodzcie droga roztropnosci.
\par 7 Kto strofuje nasmiewce, odnosi hanbe; a kto strofuje niezboznika, odnosi zelzywosc.
\par 8 Nie strofuj nasmiewcy, aby cie nie mial w nienawisci; strofuj madrego, a bedzie cie milowal.
\par 9 Uczyn to madremu, a medrszym bedzie; naucz sprawiedliwego, a bedzie umiejetniejszym.
\par 10 Poczatek madrosci jest bojazn Panska, a umiejetnosc swietych jest rozum.
\par 11 Bo przez mie rozmnoza sie dni twoje, i przedluza sie lata zywota.
\par 12 Bedzieszli madrym, sobie bedziesz madrym; a jezli nasmiewca, ty sam szkode odniesiesz.
\par 13 Niewiasta glupia swiegotliwa jest, prostaczka, i nic nieumiejaca;
\par 14 A siedzi u drzwi domu swego na stolku, na miejscach wysokich w miescie,
\par 15 Aby wolala na idacych droga, którzy prosto ida scieszkami swemi, mówiac:
\par 16 Ktokolwiek jest prostakiem, wstap sam; a do glupiego mówi:
\par 17 Wody kradzione slodsze sa, a chleb pokatny smaczniejszy.
\par 18 Ale prostak nie wie, ze tam sa umarli, a ci, których wezwala, sa w glebokosciach grobu.

\chapter{10}

\par 1 Syn madry rozwesela ojca: ale syn glupi smutkiem jest matki swojej.
\par 2 Nie pomoga skarby niezboznosci; ale sprawiedliwosc wyrywa od smierci.
\par 3 Nie dopusci Pan laknac duszy sprawiedliwego; ale majetnosc niezbozników rozproszy.
\par 4 Do nedzy przywodzi reka zdradliwa; ale reka pracowita ubogaca.
\par 5 Kto zbiera w lecie, jest syn roztropny; kto dosypia we zniwa, jest syn pohanbienia.
\par 6 Blogoslawienstwo jest nad glowa sprawiedliwego; ale usta bezboznych pokrywaja nieprawosc.
\par 7 Blogoslawiona jest pamiatka sprawiedliwego; ale imie niezboznych smierdzi.
\par 8 Madre serce przyjmuje przykazanie; ale glupi od warg swoich upadnie.
\par 9 Kto chodzi w szczerosci, chodzi bezpiecznie; ale kto jest przewrotnym w drogach swoich, wyjawion bedzie.
\par 10 Kto mruga okiem, przynosi frasunek, ale glupi od warg swoich upadnie.
\par 11 Usta sprawiedliwego sa zródlo zywota; ale usta niezbozników pokrywaja nieprawosc.
\par 12 Nienawisc wzbudza swary; ale milosc wszystkie przestepstwa pokrywa.
\par 13 W wargach roztropnego znajduje sie madrosc; ale kij na grzbiecie szalonego.
\par 14 Madrzy taja umiejetnosc; ale usta glupiego bliskie upadku.
\par 15 Majetnosc bogatego jest miastem jego mocnem; ale nedza jest ubogich zniszczeniem.
\par 16 Praca sprawiedliwego jest ku zywotowi; ale dochód niepoboznych jest ku grzechowi.
\par 17 Scieszka zywota idzie, kto przyjmuje karnosc; ale kto gardzi strofowaniem, w blad sie zawodzi.
\par 18 Kto pokrywa nienawisc wargami klamliwemi, i kto rozglasza hanbe, glupi jest.
\par 19 Wielomownosc nie bywa bez grzechu; ale kto powsciaga wargi swoje, ostrozny jest.
\par 20 Srebro wyborne jest jezyk sprawiedliwego; ale serce niezboznych za nic nie stoi.
\par 21 Wargi sprawiedliwego wiele ich zywia; ale glupi dla glupstwa umieraja.
\par 22 Blogoslawienstwo Panskie ubogaca, a nie przynosi z soba utrapienia.
\par 23 Za smiech sobie ma glupi, popelnic niecnote, ale maz roztropny dzierzy sie madrosci.
\par 24 Czego sie boi niezboznik, to nan przychodzi; ale czego zadaja sprawiedliwi, Bóg im daje.
\par 25 Jako przemija wicher, tak sie niepobozni nie ostoja; ale sprawiedliwy ma grunt wieczny.
\par 26 Jako ocet zebom, i jako dym oczom, tak jest leniwy tym, którzy go posylaja.
\par 27 Bojazn Panska dni przyczynia; ale lata niezboznego ukrócone bywaja.
\par 28 Oczekiwanie sprawiedliwych jest wesele, ale nadzieja niezboznych zginie.
\par 29 Droga Panska jest moca szczeremu; ale strachem tym, którzy broja zlosci.
\par 30 Sprawiedliwy sie na wieki nie poruszy; ale niezboznicy nie beda mieszkali na ziemi.
\par 31 Usta sprawiedliwego rozmnazaja madrosc; ale jezyk przewrotny bedzie wyciety.
\par 32 Wargi sprawiedliwego znaja, co sie Bogu podoba; ale usta niepoboznych sa przewrotne.

\chapter{11}

\par 1 Waga falszywa obrzydliwoscia jest Panu; ale gwichty sprawiedliwe podobaja mu sie.
\par 2 Za pycha przychodzi hanba; ale przy pokornych jest madrosc.
\par 3 Szczerosc ludzi cnotliwych prowadzi ich; ale przewrotnosc przestepców potraci ich.
\par 4 Niepomoga bogactwa w dzien gniewu; ale sprawiedliwosc wybawia od smierci.
\par 5 Sprawiedliwosc uprzejmego sprawuje droge jego; lecz bezbozny dla bezboznosci swojej upada.
\par 6 Sprawiedliwosc uprzejmych wybawia ich: ale przewrotni w zlosciach pojmani bywaja.
\par 7 Gdy umiera czlowiek niepobozny, ginie nadzieja jego, a oczekiwanie mocarzy niszczeje.
\par 8 Sprawiedliwy z ucisku wybawiony bywa; ale niepobozny przychodzi na miejsce jego.
\par 9 Obludnik usty kazi przyjaciela swego; ale sprawiedliwi umiejetnoscia wybawieni bywaja.
\par 10 Z szczescia sprawiedliwych miasto sie weseli; a gdy gina niezbozni, bywa radosc.
\par 11 Dla blogoslawienstwa sprawiedliwych bywa wywyzszone miasto; ale dla ust niepoboznych bywa wywrócone.
\par 12 Glupi gardzi bliznim swym; ale maz roztropny milczy.
\par 13 Obmówca obchodzac objawia tajemnice; ale kto jest wiernego serca, tai zwierzonej rzeczy.
\par 14 Gdzie niemasz dostatecznej rady, lud upada; ale gdzie wiele radców, tam jest wybawienie.
\par 15 Bardzo sobie szkodzi, kto za obcego reczy; ale kto sie chroni rekojemstwa, bezpieczen jest.
\par 16 Niewiasta uczciwa dostepuje slawy, a mocarze maja bogactwa.
\par 17 Czlowiek uczynny dobrze czyni duszy swej; ale okrutnik trapi cialo swoje.
\par 18 Niezboznik czyni dzielo omylne; ale kto sieje sprawiedliwosc, ma zaplate trwala.
\par 19 Jako sprawiedliwosc jest ku zywotowi, tak kto nasladuje zlosci, bliski jest smierci.
\par 20 Obrzydliwoscia sa Panu przewrotni sercem; ale mu sie podobaja, którzy zyja bez zmazy.
\par 21 Zlosnik, choc sobie innych na pomoc wezmie, pomsty nie ujdzie; ale nasienie sprawiedliwych zachowane bedzie.
\par 22 Niewiasta piekna a glupia jest jako kolce zlote w pysku u swini.
\par 23 Zadza sprawiedliwych jest zawzdy ku dobremu; ale oczekiwanie niepoboznych, popedliwosc.
\par 24 Nie jeden udziela szczodrze, a wzdy mu przybywa; a drugi skapi wiecej niz trzeba, a wzdy ubozeje.
\par 25 Czlowiek szczodrobliwy bywa bogatszy; a kto nasyca, sam tez bedzie nasycony.
\par 26 Kto zatrzymuje zboze, tego lud przeklina; ale blogoslawienstwo nad glowa tego, który je sprzedaje.
\par 27 Kto pilnie szuka dobrego, nabywa przyjazni; ale kto szuka zlego, przyjdzie nan.
\par 28 Kto ufa w bogactwach swych, ten upadnie; ale sprawiedliwi jako latorosl zieleniec sie beda.
\par 29 Kto czyni zamieszanie w domu swoim, odziedziczy wiatr, a glupi musi sluzyc madremu.
\par 30 Owoc sprawiedliwego jest drzewo zywota; a kto naucza ludzi, madry jest.
\par 31 Oto jezli sie sprawiedliwemu na ziemi nagroda staje, tedy daleko wiecej niezboznemu i grzesznikowi.

\chapter{12}

\par 1 Kto miluje cwiczenie, miluje umiejetnosc; a kto ma w nienawisci karnosc, glupim jest.
\par 2 Dobry odniesie laske od Pana; ale meza który zle mysli, Bóg potepi.
\par 3 Nie zmocni sie czlowiek z niezboznosci; ale korzen sprawiedliwych nie bedzie poruszony.
\par 4 Zona stateczna korona jest meza swego; ale która go do hanby przywodzi, jest jako zgnilosc w kosciach jego.
\par 5 Mysli sprawiedliwych sa prawe: ale rady niepoboznych zdradliwe.
\par 6 Slowa niepoboznych czyhaja na krew; ale usta sprawiedliwych wybawiaja ich.
\par 7 Niepobozni podwróceni bywaja, tak, ze ich niestaje; ale dom sprawiedliwych zostaje.
\par 8 Z rozumu swego maz chwalony bywa; ale kto jest przewrotnego serca, wzgardzony bedzie.
\par 9 Lepszy jest czlowiek podly, który ma sluge, nizeli chlubny, któremu nie staje chleba.
\par 10 Sprawiedliwy ma na pieczy zywot bydlatka swego; ale serce niepoboznych okrutne jest.
\par 11 Kto sprawuje ziemie swoje, chlebem nasycony bywa; ale kto nasladuje próznujacych, glupi jest.
\par 12 Niepobozny pragnie obrony przeciw nieszczesciu; ale korzen sprawiedliwych daje ja.
\par 13 W przestepstwie warg uplata sie zlosnik; ale sprawiedliwy z ucisku wychodzi.
\par 14 Z owocu ust kazdy bedzie nasycony dobrem, a nagrode spraw rak jego Bóg mu odda.
\par 15 Droga glupiego zda sie prosta przed oczyma jego; ale kto slucha rady, madrym jest.
\par 16 Gniew glupiego zaraz poznany bywa; ale ostrozny pokrywa hanbe swoje.
\par 17 Kto mówi prawde, opowiada sprawiedliwosc; ale swiadek klamliwy mówi zdrade.
\par 18 Znajdzie takowego, co mówi slowa jako miecz przerazajace; ale jezyk madrych jest lekarstwem.
\par 19 Wargi prawdomówne utwierdzone beda na wieki; ale króciuchno trwa jezyk klamliwy.
\par 20 Zdrada jest w sercu tych, którzy zle mysla; ale którzy radza do pokoju, maja wesele.
\par 21 Nie spotka sprawiedliwego zadne nieszczescie; ale niezboznicy pelni beda zlego.
\par 22 Obrzydliwoscia sa Panu wargi klamliwe; ale czyniacy prawde podobaja mu sie.
\par 23 Czlowiek ostrozny tai umiejetnosc; ale serce glupich wywoluje glupstwo.
\par 24 Reka pracowitych bedzie panowala; ale zdradliwa bedzie dan dawala.
\par 25 Frasunek w sercu czlowieczem poniza je; ale powiesc dobra uwesela je.
\par 26 Zacniejszy jest nad blizniego swego sprawiedliwy; ale droga niezboznych zawodzi ich.
\par 27 Nie upiecze chytry oblowu swojego; ale czlowiek pilny majetnosci kosztownych nabedzie.
\par 28 Na scieszce sprawiedliwosci zywot, a na drodze scieszki jej niemasz smierci.

\chapter{13}

\par 1 Syn madry przyjmuje cwiczenie ojcowskie, ale nasmiewca nie slucha strofowania.
\par 2 Kazdy bedzie pozywal dobrego z owocu ust swoich; ale dusza przewrotnych krzywdy pozywac bedzie.
\par 3 Kto strzeze ust swych, strzeze duszy swojej; kto lekkomyslnie otwiera wargi swe, bedzie starty.
\par 4 Dusza leniwego zada, a nic nie ma; ale dusza pracowitych zbogaci sie.
\par 5 Slowa klamliwego nienawidzi sprawiedliwy; ale niezbozny staje sie obrzydliwym i shanbionym.
\par 6 Sprawiedliwosc strzeze tego, który zyje bez zmazy; ale niezboznosc podwraca grzesznika.
\par 7 Znajduje sie taki co sie czyni bogatym, a nie ma nic; i taki, co sie czyni ubogim, choc ma wiele bogactw.
\par 8 Okup zywota czlowieczego jest bogactwo jego; ale ubogi nie slucha lajania.
\par 9 Swiatlosc sprawiedliwych jasna: ale pochodnia bezboznych zgasnie.
\par 10 Sama tylko pycha czlowiek zwady wszczyna, ale przy tych, co rade przyjmuja, jest madrosc.
\par 11 Bogactwa zle nabyte umniejsza sie; ale kto je zgromadza reka swa, przyczynia ich.
\par 12 Nadzieja dluga watli serce; ale zadosc wypelniona jest drzewem zywota.
\par 13 Kto gardzi slowem Bozem, sam sobie szkodzi; ale kto sie boi przykazania jego, odniesie nagrode.
\par 14 Nauka madrego jest zródlem zywota ku ochronieniu sie sidel smierci.
\par 15 Rozum dobry daje laske; ale droga przewrotnych jest przykra.
\par 16 Kazdy ostrozny umiejetnie sobie poczyna; ale glupi rozposciera glupstwo.
\par 17 Posel niezbozny upada we zle; ale posel wierny jest lekarstwem.
\par 18 Ubóstwo i zelzywosc przyjdzie na tego, który sie wylamuje z karnosci; ale kto przestrzega upominania, wyslawiony bedzie.
\par 19 Zadnosc wypelniona slodka jest duszy; ale odstapic od zlego, glupim jest obrzydliwoscia.
\par 20 Kto chodzi z madrymi, madrym bedzie; ale kto towarzyszy z glupimi, startym bedzie.
\par 21 Nieszczescie grzeszników sciga; ale sprawiedliwym Bóg dobrem nagrodzi.
\par 22 Dobry czlowiek zostawia dziedzictwo synom synów swoich; ale majetnosc grzesznika sprawiedliwemu zachowana bywa.
\par 23 Obfita zywnosc na roli ubogich, a drugi ginie przez nieroztropnosc.
\par 24 Kto zawsciaga rózgi swej, ma w nienawisci syna swego; ale kto go miluje, wczas go karze.
\par 25 Sprawiedliwy je, i nasyca dusze swoje; ale zoladek niezboznych niedostatek cierpi.

\chapter{14}

\par 1 Madra niewiasta buduje dom swój; ale go glupia rekami swemi rozwala.
\par 2 Kto chodzi w szczerosci swojej, boi sie Pana; ale przewrotny w drogach swoich gardzi nim.
\par 3 W ustach glupiego jest rózga hardosci; ale wargi madrych strzega ich.
\par 4 Gdzie niemasz wolów, zlób jest prózny; ale sila wolów mnozy sie obfitosc zboza.
\par 5 Swiadek prawdziwy nie klamie; ale swiadek falszywy mówi klamstwo.
\par 6 Nasmiewca szuka madrosci, a nie znajduje; ale umiejetnosc roztropnemu jest snadna.
\par 7 Idz precz od oblicza meza glupiego, gdyz nie znajdziesz przy nim warg umiejetnosci.
\par 8 Madrosc ostroznego jest rozumiec droge swoje, ale glupstwo glupich jest zdrada.
\par 9 Kazdy glupi nakrywa grzech, a miedzy uprzejmymi mieszka przyjazn.
\par 10 Serce kazdego uznaje gorzkosc duszy swojej, a do wesela jego nie przymiesza sie obcy.
\par 11 Dom niezboznych zgladzony bedzie; ale przybytek cnotliwych zakwitnie.
\par 12 Zda sie pod czas droga byc prosta czlowiekowi; wszakze dokonczenie jej jest droga na smierc.
\par 13 Takze i w smiechu boleje serce, a koniec wesela bywa smutek.
\par 14 Drogami swemi nasyci sie czlowiek przewrotnego serca; ale sie go chroni maz dobry.
\par 15 Prostak wierzy kazdemu slowu; ale ostrozny zrozumiewa postepki swoje.
\par 16 Madry sie boi, i odstepuje od zlego; ale glupi dociera, i smialym jest.
\par 17 Porywczy czlowiek dopuszcza sie glupstwa, a maz zlych mysli w nienawisci bywa.
\par 18 Glupstwo prostacy dziedzicznie trzymaja; ale ostrozni bywaja koronowani umiejetnoscia.
\par 19 zli sie klaniaja przed dobrymi, a niepobozni stoja u drzwi sprawiedliwego.
\par 20 Ubogi bywa i u przyjaciela swego w nienawisci; ale wiele jest tych, którzy bogatego miluja.
\par 21 Bliznim swym grzesznik pogardza; ale kto ma litosc nad ubogimi, blogoslawionym jest.
\par 22 Izali nie bladza, którzy wymyslaja zle? a milosierdzie i prawda nalezy tym, którzy wymyslaja dobre.
\par 23 W kazdej pracy bywa pozytek; ale gole slowo warg tylko do nedzy sluzy.
\par 24 Bogactwo madrych jest korona ich; ale glupstwo glupich zostaje glupstwem.
\par 25 Swiadek prawdziwy wyzwala dusze; ale falszywy klamstwo mówi.
\par 26 Kto sie boi Pana, ma ufanie mocne; a synowie jego ucieczke miec beda.
\par 27 Bojazn Panska jest zródlo zywota ku uchronieniu sie sidel smierci.
\par 28 W mnóstwie ludu jest zacnosc królewska; ale w trosze ludu zniszczenie hetmana.
\par 29 Nierychly do gniewu jest bogaty w rozum; ale porywczy pokazuje glupstwo.
\par 30 Serce zdrowe jest zywotem ciala; ale zazdrosc jest zgniloscia w kosciach.
\par 31 Kto ciemiezy ubogiego, uwlacza stworzycielowi jego; ale go czci, kto ma litosc nad ubogim.
\par 32 Dla zlosci swojej wygnany bywa niepobozny; ale sprawiedliwy nadzieje ma i przy smierci swojej.
\par 33 W sercu madrego odpoczywa madrosc, ale wnet poznac, co jest w sercu glupich.
\par 34 Sprawiedliwosc wywyzsza naród; ale grzech jest ku pohanbieniu narodów.
\par 35 Król laskaw bywa na sluge roztropnego; ale sie gniewa na tego, który mu hanbe czyni.

\chapter{15}

\par 1 Odpowiedz lagodna usmierza gniew; ale slowa przykre wzruszaja popedliwosc.
\par 2 Jezyk madrych zdobi umiejetnosc; ale usta glupich wywieraja glupstwo.
\par 3 Na kazdem miejscu oczy Panskie upatruja zle i dobre.
\par 4 Zdrowy jezyk jest drzewo zywota; ale przewrotnosc z niego jest jako zdruzgotanie od wiatru.
\par 5 Glupi gardzi karaniem ojca swego; ale kto przyjmuje napomnienie, stanie sie ostroznym.
\par 6 W domu sprawiedliwego jest dostatek wielki; ale w dochodach niepoboznego zamieszanie.
\par 7 Wargi madrych sieja umiejetnosc; ale serce glupich nie tak.
\par 8 Ofiara niepoboznych jest obrzydliwoscia Panu; ale modlitwa szczerych podoba mu sie.
\par 9 Obrzydliwoscia Panu jest droga bezboznego, ale tego, co idzie za sprawiedliwoscia, miluje.
\par 10 Karanie srogie nalezy temu, co opuszcza droge; a kto ma w nienawisci karnosc, umrze.
\par 11 Pieklo i zatracenie sa przed Panem; jakoz daleko wiecej serca synów ludzkich.
\par 12 Nasmiewca nie miluje tego, który go karze, ani do madrych przychodzi.
\par 13 Serce wesole uwesela twarz; ale dla zalosci serca duch strapiony bywa.
\par 14 Serce rozumne szuka umiejetnosci; ale usta glupich karmia sie glupstwem
\par 15 Wszystkie dni ubogiego sa zle; ale kto jest wesolego serca, ma gody ustawiczne.
\par 16 Lepsza jest trocha w bojazni Panskiej, nizeli skarb wielki z klopotem.
\par 17 Lepszy jest pokarm z jarzyny, gdzie jest milosc, nizeli z karmnego wolu, gdzie jest nienawisc.
\par 18 Maz gniewliwy wszczyna swary; ale nierychly do gniewu usmierza zwady.
\par 19 Droga leniwego jest jako plot cierniowy, ale scieszka szczerych jest równa.
\par 20 Syn madry uwesela ojca; ale glupi czlowiek lekce wazy matke swoje.
\par 21 Glupstwo jest weselem glupiemu, ale czlowiek roztropny prostuje droge swoje.
\par 22 Gdzie niemasz rady, rozsypuja sie mysli; ale w mnóstwie radców ostoja sie.
\par 23 Weseli sie czlowiek z odpowiedzi ust swoich: bo slowo wedlug czasu wyrzeczone, o jako jest dobre!
\par 24 Droge zywota rozumny ma ku górze, aby sie uchronil piekla glebokiego.
\par 25 Pan wywróci dom pysznych; ale wdowy granice utwierdzi.
\par 26 Mysli zlego sa obrzydliwoscia Panu! ale powiesci czystych sa przyjemne.
\par 27 Kto chciwie nasladuje lakomstwa, zamieszanie czyni w domu swoim; ale kto ma w nienawisci dary, bedzie zyl.
\par 28 Serce sprawiedliwego przemysliwa, co ma mówic; ale usta niepoboznych wywieraja zle rzeczy.
\par 29 Dalekim jest Pan od niepoboznych; ale modlitwe sprawiedliwych wysluchiwa.
\par 30 Swiatlosc oczów uwesela serce, a wiesc dobra tuczy kosci.
\par 31 Ucho, które slucha karnosci zywota, w posrodku madrych mieszkac bedzie.
\par 32 Kto uchodzi cwiczenia, zaniedbywa duszy swojej; ale kto przyjmuje karanie, ma rozum.
\par 33 Bojazn Panska jest cwiczenie sie w madrosci, a slawe uprzedza ponizenie.

\chapter{16}

\par 1 Czlowiek sporzadza mysli serca swego; ale od Pana jest odpowiedz jezyka.
\par 2 Wszystkie drogi czlowiecze zdadza sie byc czyste przed oczyma jego; ale Pan jest, który wazy serca.
\par 3 Wlóz na Pana sprawy twe, a beda utwierdzone zamysly twoje.
\par 4 Pan dla siebie samego wszystko sprawil, nawet i niezboznika na dzien zly.
\par 5 Obrzydliwoscia jest Panu kazdy wynioslego serca; który choc sobie innych na pomoc wezmie, nie ujdzie pomsty.
\par 6 Milosierdziem i prawda oczyszczona bywa nieprawosc, a w bojazni Panskiej odstepujemy od zlego.
\par 7 Gdy sie podobaja Panu drogi czlowieka, i nieprzyjaciól jego do zgody z nim przywodzi.
\par 8 Lepsza jest trocha z sprawiedliwoscia, niz wiele dochodów niesprawiedliwych.
\par 9 Serce czlowiecze rozrzadza drogi swe; ale Pan sprawuje kroki jego.
\par 10 Sprawiedliwy rozsadek jest w wargach królewskich; w sadzie nie bladza usta jego.
\par 11 Waga i szale sa ustawa Panska, a wszystkie gwichty sprawiedliwe w worku sa za sprawa jego.
\par 12 Obrzydliwoscia jest królom czynic niezboznosc; bo sprawiedliwoscia stolica umocniona bywa.
\par 13 Przyjemne sa królom wargi sprawiedliwe, a szczerych w mowie miluja.
\par 14 Gniew królewski jest poslem smierci; ale maz madry ublaga go.
\par 15 W jasnosci twarzy królewskiej jest zywot, a laska jego jest jako oblok z deszczem póznym.
\par 16 Daleko lepiej jest nabyc madrosci, nizeli zlota najczystszego; a nabyc roztropnosci lepiej, niz srebra.
\par 17 Gosciniec uprzejmych jest odstapic od zlego; strzeze duszy swej, kto strzeze drogi swojej.
\par 18 Przed zginieniem przychodzi pycha, a przed upadkiem wynioslosc ducha.
\par 19 Lepiej jest byc unizonego ducha z pokornymi, nizeli dzielic korzysci z pysznymi.
\par 20 Kto ma wzglad na slowa, znajduje dobre; a kto ufa w Panu, blogoslawiony jest.
\par 21 Kto jest madrego serca, slynie rozumnym, a slodkosc warg przydaje nauki.
\par 22 Zdrój zywota jest roztropnosc tym, którzy ja maja; ale umiejetnosc glupich jest glupstwem.
\par 23 Serce madrego roztropnie sprawuje usta swoje, a wargami swemi przydaje nauki.
\par 24 Powiesci wdzieczne sa jako plastr miodu, slodkoscia duszy, a lekarstwem kosciom.
\par 25 Zda sie podczas droga byc prosta czlowiekowi; wszakze dokonczenie jej pewna droga na smierc.
\par 26 Czlowiek pracowity pracuje sobie; bo go pobudzaja usta jego.
\par 27 Czlowiek niezbozny wykopuje zle, a w wargach jego jako ogien palajacy.
\par 28 Maz przewrotny rozsiewa zwady, a klatecznik rozlacza przyjaciól.
\par 29 Maz okrutny przewabia blizniego swego, i wprowadza go na droge niedobra.
\par 30 Kto mruga oczyma swemi, zmysla przewrotnosci; a kto rucha wargami swemi, broi zle.
\par 31 Korona chwaly jest sedziwosc; znajduje sie na drodze sprawiedliwosci.
\par 32 Lepszy jest nierychly do gniewu, nizeli mocarz; a kto panuje sercu swemu, lepszy jest, nizeli ten, co dobyl miasta.
\par 33 Los na lono rzucaja; ale od Pana jest wszystko rozrzadzenie jego.

\chapter{17}

\par 1 Lepszy jest kes suchego chleba a w pokoju, nizeli pelen dom nabitego bydla ze swarem.
\par 2 Sluga roztropny bedzie panowal nad synem, który jest ku hanbie; a miedzy bracmi bedzie dzielil dziedzictwo.
\par 3 Tygiel srebra a piec zlota doswiadcza; ale Pan serc doswadcza.
\par 4 Zly pilnuje warg zlosliwych, a klamca slucha jezyka przewrotnego.
\par 5 Kto sie nasmiewa z ubogiego, uwlacza stworzycielowi jego; a kto sie raduje z upadku czyjego, nie ujdzie pomsty.
\par 6 Korona starców sa synowie synów ich, a ozdoba synów sa ojcowie ich.
\par 7 Nie przystoi mowa powazna glupiemu, dopieroz ksieciu usta klamliwe.
\par 8 Jako kamien drogi, tak bywa dar wdzieczny temu, który go bierze; do czegokolwiek zmierzy, zdarzy mu sie.
\par 9 Kto pokrywa przestepstwo, szuka laski; ale kto wznawia rzeczy, rozlacza przyjaciól.
\par 10 Wiecej wazy gromienie u roztropnego, nizeli sto plag u glupiego.
\par 11 Uporny tylko zlego szuka, dla tego posel okrutny bedzie nan zeslany.
\par 12 Lepiej jest czlowiekowi spotkac sie z niedzwiedzica osierociala, nizeli z glupim w glupstwie jego.
\par 13 Kto oddaje zlem za dobre, nie wynijdzie zle z domu jego.
\par 14 Kto zaczyna zwade, jest jako ten, co przekopuje wode; przetoz niz sie zwada rozsili, zaniechaj go.
\par 15 Kto usprawiedliwia niezboznego, a winnym czyni sprawiedliwego, oba jednako sa obrzydliwoscia Panu.
\par 16 Cóz po dostatku w reku glupiego, poniewaz do nabycia madrosci rozumu nie ma?
\par 17 Wszelkiego czasu miluje przyjaciel, a w ucisku stawia sie jako brat.
\par 18 Czlowiek glupi daje reke, czyniac rekojemstwo przed twarza przyjaciela swego.
\par 19 Kto miluje zwade, miluje grzech; a kto wynosi usta swe, szuka upadku.
\par 20 Przewrotny w sercu nie znajduje dobrego; a kto jest przewrotnego jezyka, wpadnie we zle.
\par 21 Kto splodzil glupiego, na smutek swój splodzil go, ani sie rozweseli ojciec niemadrego.
\par 22 Serce wesole oczerstwia jako lekarstwo; ale duch sfrasowany wysusza kosci.
\par 23 Niezbozny potajemnie dar bierze, aby podwrócil scieszki sadu.
\par 24 Na twarzy roztropnego znac madrosc; ale oczy glupiego az na kraju ziemi.
\par 25 Syn glupi zaloscia jest ojcu swemu, a gorzkoscia rodzicielce swojej.
\par 26 Zaiste nie dobra, winowac sprawiedliwego, albo zeby przelozeni kogo dla cnoty bic mieli.
\par 27 Kto zawsciaga mowy swe, jest umiejetnym; drogiego ducha jest maz rozumny.
\par 28 Gdy glupi milczy, za madrego poczytany bywa; a który zatula wargi swoje, za rozumnego.

\chapter{18}

\par 1 Czlowiek swej mysli, szuka tego, co mu sie podoba, a w kazda rzecz wtraca sie.
\par 2 Nie kocha sie glupi w roztropnosci, ale w tem, co mu objawia serce jego.
\par 3 Gdy przychodzi niezbozny, przychodzi tez wzgarda, a z mezem lekkomyslnym uraganie.
\par 4 Slowa ust meza madrego sa jako wody glebokie, a zródlo madrosci jako potok wylewajacy.
\par 5 Nie dobra to, miec wzglad na osobe niezboznego, aby byl podwrócony sprawiedliwy w sadzie.
\par 6 Wargi glupiego zmierzaja do swaru, a usta jego do bitwy wyzywaja.
\par 7 Usta glupiego sa upadkiem jego, a wargi jego sidlem duszy jego.
\par 8 Slowa obmówcy sa jako slowa zranionych, a wszakze przenikaja do wnetrznosci zywota.
\par 9 Kto niedbaly w sprawach swoich, bratem jest utratnika.
\par 10 Imie Panskie jest mocna wieza; sprawiedliwy sie do niej uciecze, a wywyzszony bedzie.
\par 11 Majetnosc bogatego jest miastem jego mocnem, a jako mur wysoki w mysli jego.
\par 12 Przed upadkiem podnosi sie serce czlowiecze, a slawe uprzedza ponizenie.
\par 13 Kto odpowiada, pierwej niz wyslucha, glupstwo to jego i zelzywosc.
\par 14 Duch meza znosi niemoc swoje; ale ducha utrapionego któz zniesie?
\par 15 Serce rozumne nabywa umiejetnosci, a ucho madrych szuka jej.
\par 16 Dar czlowieczy plac mu czyni, i przed wielmoznych przywodzi go.
\par 17 Sprawiedliwym zda sie ten, kto pierwszy w sprawie swojej; ale gdy przychodzi blizni jego, dochodzi go.
\par 18 Los usmierza zwady, i miedzy moznymi rozsadek czyni.
\par 19 Brat krzywda urazony trudniejszy nad miasto niedobyte, a swary sa jako zawory u palacu.
\par 20 Z owocu ust kazdego nasycon bywa zywot jego; urodzajem warg swych bedzie nasycony.
\par 21 Smierc i zywot jest w mocy jezyka, a kto go miluje, bedzie jadl owoce jego.
\par 22 Kto znalazl zone, znalazl rzecz dobra, i dostapil laski od Pana.
\par 23 Ubogi pokornie mówi; ale bogaty odpowiada surowie.
\par 24 Czlowiek, który ma przyjaciól, ma sie obchodzic po przyjacielsku, poniewaz przyjaciel bywa przychylniejszy nad brata.

\chapter{19}

\par 1 Lepszy jest ubogi, który chodzi w uprzejmosci swej, nizeli przewrotny w wargach swoich, który jest glupim.
\par 2 Zaiste duszy bez umiejetnosci nie dobrze, a kto jest predkich nóg, potknie sie.
\par 3 Glupstwo czlowiecze podwraca droge jego, a przecie przeciwko Panu zapala sie gniewem serce jego.
\par 4 Bogactwa przyczyniaja wiele przyjaciól; ale ubogi od przyjaciela swego odlaczony bywa.
\par 5 Falszywy swiadek nie bedzie bez pomsty; a kto mówi klamstwo, nie ujdzie.
\par 6 Wielu sie ich uniza przed ksieciem, a kazdy jest przyjacielem mezowi szczodremu.
\par 7 Wszyscy bracia ubogiego nienawidza go; daleko wiecej inni przyjaciele jego oddalaja sie od niego; wola za nimi, a niemasz ich.
\par 8 Nabywa rozumu, kto miluje dusze swoje, a strzeze roztropnosci, aby znalazl co dobrego.
\par 9 Swiadek falszywy nie bedzie bez pomsty; a kto mówi klamstwo, zginie.
\par 10 Nie przystoi glupiemu rozkosz, ani sludze panowac nad ksiazetami.
\par 11 Rozum czlowieczy zawsciaga gniew jego, a ozdoba jego jest mijac przestepstwo.
\par 12 Zapalczywosc królewska jest jako ryk lwiecia; ale laska jego jest jako rosa na trawie.
\par 13 Syn glupi jest utrapieniem ojcu swemu, a zona swarliwa jest jako ustawiczne kapanie przez dach.
\par 14 Dom i majetnosc dziedzictwem przypada po rodzicach; ale zona roztropna jest od Pana.
\par 15 Lenistwo przywodzi twardy sen, a dusza gnusna bedzie laknela.
\par 16 Kto strzeze przykazania, strzeze duszy swojej; ale kto gardzi drogami swemi, zginie.
\par 17 Panu pozycza, kto ma litosc nad ubogim, a on mu za dobrodziejstwo jego odda.
\par 18 Karz syna swego, póki o nim nadzieja, a zabiegajac zginieniu jego niech mu nie folguje dusza twoja.
\par 19 Wielki gniew okazuj, kiedy odpuszczasz karanie, grozac mu, poniewaz odpuszczasz, ze potem srozej karac bedziesz.
\par 20 Sluchaj rady, a przyjmuj karnosc, abys kiedyzkolwiek byl madrym.
\par 21 Wiele jest mysli w sercu czlowieczem; ale rada Panska, ta sie ostoi.
\par 22 Pozadana rzecz czlowiekowi dobroczynnosc jego, ale lepszy jest ubogi, niz maz klamliwy.
\par 23 Bojazn Panska prowadzi do zywota, a kto ja ma, w obfitosci mieszka, i nie spotka go nieszczescie.
\par 24 Leniwy kryje reke swa pod pache, i do ust swych nie podnosi jej.
\par 25 Bij nasmiewce, zeby prostak byl ostrozniejszym; a roztropnego sfukaj, zeby zrozumial umiejetnosc.
\par 26 Syn wstyd i hanbe zadawajacy, ojca gubi i matke wygania.
\par 27 Synu mój! przestan sluchac nauki, któraby cie odwodzila od mów rozumnych.
\par 28 Swiadek zlosliwy posmiewa sie z sadu, a usta niezboznych polykaja nieprawosc.
\par 29 Sady sa na posmiewców zgotowane, a guzy na grzbiet glupich.

\chapter{20}

\par 1 Wino czyni posmiewce, a napój mocny zwajce; przetoz kazdy, co sie w nim kocha, nie bywa madrym.
\par 2 Strach królewski jest jako, ryk lwiecia; kto go rozgniewa, grzeszy przeciwko duszy swojej.
\par 3 Uczciwa rzecz kazdemu, poprzestac zwady; ale glupim jest, co sie w nia wdaje.
\par 4 Dla zimna leniwy nie orze; przetoz zebrac bedzie we zniwa, ale nic nie otrzyma.
\par 5 Rada w sercu meza jest jako woda gleboka: jednak maz rozumny naczerpnie jej.
\par 6 Wieksza czesc ludzi przechwala sie uczynnoscia swoja; ale w samej rzeczy, któz takiego znajdzie?
\par 7 Sprawiedliwy chodzi w uprzejmosci swojej; blogoslawieni synowie jego po nim.
\par 8 Król siedzac na stolicy sadowej rozgania oczyma swemi wszystko zle.
\par 9 Któz rzecze: Oczyscilem serce moje? czystym jest od grzechu mego?
\par 10 Dwojaki gwicht i dwojaka miara, to oboje obrzydliwoscia jest Panu.
\par 11 Po zabawach swych poznane bywa i dziecie, jezli czysty i prawy uczynek jego.
\par 12 Ucho, które slyszy, i oko, które widzi, Pan to oboje uczynil.
\par 13 Nie kochaj sie w spaniu, bys snac nie zubozal, otwórz oczy swoje, a nasycisz sie chlebem.
\par 14 Zle to, zle to, mówi ten, co kupuje, a odszedlszy, ali sie chlubi.
\par 15 Wargi umiejetne sa jako zloto i obfitosc perel, i kosztowne klejnoty.
\par 16 Wezmij szate tego, któryc reczyl za obcego; a od tego, który reczyl za cudzoziemke, wezmij zastaw jego.
\par 17 Smaczny jest drugiemu chleb klamstwa; ale potem piaskiem napelnione beda usta jego.
\par 18 Mysli radami utwierdzaj, a wojne prowadz opatrznie.
\par 19 Kto objawia tajemnice, zdradliwie sie obchodzi; przetoz z tymy, którzy pochlebiaja wargami swemi, nie miej towarzystwa.
\par 20 Kto zlorzeczy ojcu swemu albo matce swojej, zgasnie pochodnia jego w gestych ciemnosciach.
\par 21 Dziedzictwu predko z poczatku nabytemu naostatek blogoslawic nie beda.
\par 22 Nie mów: Oddam zlem. Oczekuj na Pana, a wybawi cie.
\par 23 Obrzydliwoscia Panu dwojaki gwicht, a szale falszywe nie podobaja mu sie.
\par 24 Od Pana bywaja sprawowane drogi meza; ale czlowiek jakoz zrozumie droge jego?
\par 25 Pozrec rzecz poswiecona, jest czlowiekowi sidlem; a poslubiwszy co, tego zas szukac, jakoby tego ujsc.
\par 26 Król madry rozprasza niezboznych, i przywodzi na nich pomste.
\par 27 Dusza ludzka jest pochodnia Panska, która doswiadcza wszystkich skrytosci wnetrznych.
\par 28 Milosierdzie i prawda króla strzega, a stolica jego milosierdziem wsparta bywa.
\par 29 Ozdoba mlodzienców jest sila ich, a sedziwosc poczciwoscia starców.
\par 30 Zlemu sa lekarstwem sinosci ran, i razy przenikajace do wnetrznosci zywota jego.

\chapter{21}

\par 1 Serce królewskie jest w rece Panskiej jako potoki wód; kedy chce, nakloni je.
\par 2 Wszelka droga czlowieka prosta jest przed oczyma jego; ale Pan, jest który serca wazy.
\par 3 Czynic sprawiedliwosc i sad, bardziej sie Panu podoba, nizeli ofiara.
\par 4 Wynioslosc oczu i nadetosc serca, i oranie niepoboznych sa grzechem.
\par 5 Mysli pracowitego pewne dostatki przynosza; ale kazdego skwapliwego przynosza pewna nedze.
\par 6 Zebrane skarby jezykiem klamliwym sa marnoscia pomijajaca tych, którzy szukaja smierci.
\par 7 Drapiestwo niezboznych potrwozy ich; bo nie chcieli czynic to, co bylo sprawiedliwego.
\par 8 Maz, którego droga przewrotna, obcym jest; ale sprawa czystego jest prosta.
\par 9 Lepiej jest mieszkac w kacie pod dachem, nizeli z zona swarliwa w domu przestronnym.
\par 10 Dusza niezboznego pragnie zlego, a przyjaciel jego nie bywa wdzieczny w oczach jego.
\par 11 Gdy karza nasmiewce, prostak medrszym bywa; a gdy roztropnie postepuja z madrym, przyjmuje nauke.
\par 12 Bóg daje przestroge sprawiedliwemu na domie niezboznika, który podwraca niezboznych dla zlosci ich.
\par 13 Kto zatula ucho swe na wolanie ubogiego, i on sam bedzie wolal, a nie bedzie wysluchany.
\par 14 Dar potajemnie dany usmierza zapalczywosc, i upominek w zanadrza wlozony gniew wielki uspokaja.
\par 15 Radosc sie mnozy sprawiedliwemu, gdy sie sad odprawuje; ale strach tym, którzy czynia nieprawosc.
\par 16 Czlowiek bladzacy z drogi madrosci w zebraniu umarlych odpoczywac bedzie.
\par 17 Maz, który dobra mysl miluje, staje sie ubogim; a kto miluje wino i olejki, nie zbogaci sie.
\par 18 Niezboznik bedzie okupem za sprawiedliwego, a za uprzejmych przewrotnik.
\par 19 Lepiej mieszkac w ziemi pustej, niz z zona swarliwa i gniewliwa.
\par 20 Skarb pozadany i olej sa w przybytku madrego; ale glupi czlowiek pozera go.
\par 21 Kto nasladuje sprawiedliwosci i milosierdzia, znajduje zywot, sprawiedliwosc i slawe.
\par 22 Madry ubiega miasto mocarzy, a burzy potege ufnosci ich.
\par 23 Kto strzeze ust swoich i jezyka swego, strzeze od ucisków duszy swojej.
\par 24 Hardego i pysznego imie jest nasmiewca, który wszysko poniewoli i z pycha czyni.
\par 25 Leniwego zadosc zabija; bo rece jego robic nie chca.
\par 26 Kazdego dnia pala pozadliwoscia; ale sprawiedliwy udziela, a nie szczedzi.
\par 27 Ofiara niepoboznych jest obrzydliwoscia, a dopieroz gdyby ja w grzechu ofiarowal.
\par 28 Swiadek falszywy zaginie; ale maz dobry to, co slyszy, statecznie mówic bedzie.
\par 29 Maz niezbozny zatwardza twarz swoje; ale uprzejmy sam sprawuje droge swoje.
\par 30 Niemasz madrosci, ani rozumu, ani rady przeciwko Panu.
\par 31 Konia gotuja na dzien bitwy; ale od Pana jest wybawienie.

\chapter{22}

\par 1 Lepsze jest dobre imie, niz bogactwa wielkie; a przyjazn lepsza, niz srebro i zloto.
\par 2 Bogaty i ubogi spotkali sie z soba; ale Pan jest obydwóch stworzycielem.
\par 3 Ostrozny widzac zle ukrywa sie; ale prostacy wprost idac wpadaja w szkode.
\par 4 Pokory i bojazni Panskiej nagroda jest bogactwo, i slawa i zywot.
\par 5 Ciernie i sidla sa na drodze przewrotnego; kto strzeze duszy swej, oddala sie od nich.
\par 6 Cwicz mlodego wedlug potrzeby drogi jego; bo gdy sie zstarzeje, nie odstapi od niej.
\par 7 Bogaty nad ubogimi panuje; ale ten, co pozycza, sluga bywa tego, który mu pozycza.
\par 8 Kto sieje nieprawosc, zac bedzie utrapienie, a rózga gniewu jego ustanie.
\par 9 Oko dobrotliwe, toc bedzie ublogoslawione; bo udziela chleba swego ubogiemu.
\par 10 Wyrzuc nasmiewce, a ustanie zwada; owszem uspokoi sie swar i pohanbienie.
\par 11 Kto miluje czystosc serca, a jest wdziecznosc w wargach jego, temu król przyjacielem bedzie.
\par 12 Oczy Panskie strzega umiejetnosci; ale przedsiewziecia przewrotnego podwraca.
\par 13 Leniwiec mówi: Lew na dworzu, w posród ulicy bym byl zabity.
\par 14 Usta obcych niewiast sa dól gleboki; na kogo sie Pan gniewa, wpadnie tam.
\par 15 Glupstwo przywiazane jest do serca mlodego; ale rózga karnosci oddali je od niego.
\par 16 Kto ciemiezy ubogiego, aby sobie przysporzyl, takze kto daje bogatemu: pewnie zubozeje.
\par 17 Naklon ucha twego, a sluchaj slów madrych, a serce twoje przylóz ku nauce mojej;
\par 18 Boc to bedzie uciecha, gdy je zachowasz w sercu twojem, gdy beda spolem sporzadzone w wargach twoich;
\par 19 Aby bylo w Panu ufanie twoje; oznajmujec to dzis, a ty tak czyn.
\par 20 Izalim ci nie napisal znamienitych rzeczy z strony rad i umiejetnosci,
\par 21 Abym ci do wiadomosci podal pewnosc powiesci prawdziwych, abys umial odnosic slowa prawdy tym, którzy cie poslali.
\par 22 Nie odzieraj nedznego, przeto ze nedzny jest; ani ubogiego w bramie uciskaj.
\par 23 Albowiem Pan sie podejmie sprawy ich, i wydrze dusze tym, którzy im wydzieraja.
\par 24 Nie badz przyjacielem gniewliwemu, a z mezem popedliwym nie obcuj,
\par 25 Bys snac nie przywykl scieszkom jego, a nie wlozyl sidla na dusze swoje.
\par 26 Nie bywaj miedzy tymi, którzy recza; ani miedzy rekojmiami za dlugi;
\par 27 Bo jezlibys nie mial czem zaplacic, przeczzeby kto mial brac posciel twoja pod toba?
\par 28 Nie przenos starej granicy, która uczynili ojcowie twoi.
\par 29 Widzialzes meza ratszego w sprawach swoich? Takowyc przed królami staje, a nie staje przed podlymi.

\chapter{23}

\par 1 Gdy siadziesz, abys jadl z panem, uwazaj pilnie, kto jest przed toba;
\par 2 Inaczej wrazilbys nóz w gardlo swoje, jezlibys byl chciwy pokarmu.
\par 3 Nie pragnij lakoci jego; bo sa pokarmem obludnym.
\par 4 Nie staraj sie, abys sie zbogacil; owszem, zaniechaj opatrznosci twojej.
\par 5 I mialzebys obrócic oczy twoje na bogactwo, które predko niszczeje? bo sobie uczyni skrzydla podobne orlim, i uleci do nieba.
\par 6 Nie jedz chleba czlowieka zazdrosnego, a nie zadaj lakoci jego.
\par 7 Albowiem jako on ciebie wazy w mysli swej, tak ty waz pokarm jego. Mówic: Jedz i pij, ale serce jego nie jest z toba.
\par 8 Sztuczke twoje, któras zjadl, zwrócisz, a utracisz wdzieczne slowa twoje.
\par 9 Przed glupim nie mów; albowiem wzgardzi roztropnoscia powiesci twoich.
\par 10 Nie przenos granicy starej, a na role sierotek nie wchodz.
\par 11 Bo obronca ich mozny; onci sie podejmuje sprawy ich przeciwko tobie.
\par 12 Obróc do nauki serce twoje, a uszy twoje do powiesci umiejetnosci.
\par 13 Nie odejmuj od mlodego karnosci; bo jezli go ubijesz rózga, nie umrze.
\par 14 Ty go bij rózga, a dusze jego z piekla wyrwiesz.
\par 15 Synu mój! bedzieli madre serce twoje, bedzie sie weselilo serce moje, serce moje we mnie;
\par 16 I rozwesela sie nerki moje, gdy beda mówily wargi twoje, co jest prawego.
\par 17 Niech nie zajrzy serce twoje grzesznikom; ale raczej chodz w bojanni Panskiej na kazdy dzien;
\par 18 Bo iz jest zaplata, przeto nadzieja twoja nie bedzie wykorzeniona.
\par 19 Sluchaj, synu mój! a badz madry, i nawiedz na droge serce twoje.
\par 20 Nie bywaj miedzy pijanicami wina, ani miedzy zarlokami miesa;
\par 21 Boc pijanica i zarlok zubozeje, a ospaly w latach chodzic bedzie.
\par 22 Sluchaj ojca twego, który cie splodzil, a nie pogardzaj matka twoja, gdy sie zstarzeje.
\par 23 Kupuj prawde, a nie sprzedawaj jej; kupuj madrosc, umiejetnosc i rozum.
\par 24 Bardzo sie raduje ojciec sprawiedliwego, a kto splodzil madrego, weseli sie z niego.
\par 25 Niech sie tedy weseli ojciec twój, i matka twoja; i niech sie rozraduje rodzicielka twoja.
\par 26 Synu mój! daj mi serce twoje, a oczy twoje niechaj strzega dróg moich.
\par 27 Bo nierzadnica jest dól gleboki, a cudza zona jest studnia ciasna.
\par 28 Ona tez jako zbojca zasadzki czyni, a zuchwalców miedzy ludzmi rozmnaza.
\par 29 Komu biada? Komu niestety? Komu zwady? Komu krzyk? Komu rany daremne? Komu zapalenie oczów?
\par 30 Tym, którzy siadaja na winie; tym, którzy chodza, szukajac przyprawnego wina.
\par 31 Nie zapatruj sie na wino, gdy sie rumieni, i gdy wydaje w kubku lune swoje, a prosto wyskakuje.
\par 32 Bo na koniec jako waz ukasi, a jako zmija uszczknie;
\par 33 Oczy twoje patrzyc beda na cudze zony, a serce twe bedzie mówilo przewrotnosci;
\par 34 I bedziesz jako ten, który lezy w posród morza, a jako ten, który spi na wierzchu masztu;
\par 35 Rzeczesz:Ubito mie, a nie stekalem, potluczono mie, a nie czulem. Gdy sie ocuce, udam sie zas do tego.

\chapter{24}

\par 1 Nie nasladuj ludzi zlych, ani zadaj przebywac z nimi;
\par 2 Albowiem serce ich mysli o drapiestwie, a wargi ich mówia o ucisnieniu.
\par 3 Madroscia bywa dom zbudowany, a roztropnoscia umocniony.
\par 4 Zaiste przez umiejetnosc komory napelnione bywaja wszelakiemi bogactwami kosztownemi i wdziecznemi.
\par 5 Czlowiek madry mocny jest, a maz umiejetny przydaje sily.
\par 6 Albowiem przez madra rade zwiedziesz bitwe, a wybawienie przez mnóstwo radców miec bedziesz.
\par 7 Wysokie sa glupiemu madrosci; w bramie nie otworzy ust swoich.
\par 8 Kto mysli zle czynic, tego zlosliwym zwac beda.
\par 9 Zla mysl glupiego jest grzechem, a posmiewca jest obrzydliwoscia ludzka.
\par 10 Jezli bedziesz gnusnym, tedy w dzien ucisku slaba bedzie sila twoja.
\par 11 Wybawiaj pojmanych na smierc; a od tych, którzy ida na stracenie, nie odwracaj sie.
\par 12 Jezli rzeczesz: Otosmy o tem nie wiedzieli; izali ten, który wazy serca, nie rozumie? a ten, który strzeze duszy twojej, nie rozezna? i nie odda czlowiekowi wedlug uczynków jego?
\par 13 Jedz miód, synu mój! bo dobry, i plastr slodki podniebieniu twemu;
\par 14 Tak umiejetnosc madrosci duszy twojej, jezlize ja znajdziesz; onac bedzie nagroda, a nadzieja twoja nie bedzie wycieta.
\par 15 Nie czyn zasadzki, niezbozniku! na przybytek sprawiedliwego, a nie przeszkadzaj odpocznieniu jego.
\par 16 Bo choc siedm kroc upada sprawiedliwy, przecie zas powstaje; ale niezbozni wpadna w nieszczescie.
\par 17 Gdy upadnie nieprzyjaciel twój, nie ciesz sie; i gdy sie potknie, niech sie nie raduje serce twoje;
\par 18 Aby snac nie ujrzal Pan, a nie podobaloby sie to w oczach jego, i odwrócilby od niego gniew swój na cie.
\par 19 Nie gniewaj sie dla zlosników, ani sie udawaj za niepoboznymi;
\par 20 Boc nie wezmie zlosnik nagrody; pochodnia niepoboznych zgasnie.
\par 21 Synu mój! bój sie Pana i króla, a z niestatecznymi nie mieszaj sie;
\par 22 Boc znagla powstanie zginienie ich, a upadek obydwóch któz wie?
\par 23 I toc tez madrym nalezy: wzglad miec na osobe u sadu, nie dobra.
\par 24 Tego, który mówi niepoboznemu; Jestes sprawiedliwy, beda ludzie przeklinac, a narody sie nim brzydzic beda.
\par 25 Ale którzy go karza, szczesliwi beda, a przyjdzie na nich blogoslawienstwo kazdego dobrego.
\par 26 Pocaluja wargi tego, co mówi slowa prawdziwe.
\par 27 Rozrzadz na polu robote twoje, a sprawuj pilnie role swoje; a potem bedziesz budowal dom twój.
\par 28 Nie badz swiadkiem lekkomyslnym przeciw blizniemu swemu, ani czyn lagodnych namów wargami swemi.
\par 29 Nie mów: Jako mi uczynil, tak mu uczynie; oddam mezowi temu wedlug uczynku jego.
\par 30 Szedlem przez pole meza leniwego a przez winnice czlowieka glupiego;
\par 31 A oto porosla wszedzie ostem; pokrzywy wszystko pokryly, a plot kamienny jej rozwalil sie.
\par 32 Co ja ujrzawszy zlozylem to do serca mego, a widzac to wzialem to ku przestrodze.
\par 33 Troche sie przespisz, troche podrzemiesz, troche zlozysz rece, abys odpoczywal;
\par 34 A wtem ubóstwo twoje przyjdzie jako podrózny, a niedostatek twój jako maz zbrojny.

\chapter{25}

\par 1 Tec sa przypowiesci Salomonowe, które zebrali mezowie Ezechyjasza, króla Judzkiego.
\par 2 Slawa to Boza, taic sprawe; ale slawa królów, wywiadywac sie rzeczy.
\par 3 Wysokosc niebios, i glebokosc ziemi, i serce królów nie sa doscignione.
\par 4 Odejm zuzelice od srebra, a wynijdzie odlewajacemu naczynie kosztowne.
\par 5 Odejm niezboznika od oblicza królewskiego, a umocni sie w sprawiedliwosci stolica jego.
\par 6 Nie udawaj sie za zacnego przed królem, a na miejscu wielmoznych nie stawaj;
\par 7 Bo lepiej jest, iz ci rzeka: Postap sam: a nizeliby cie znizyc miano przed ksieciem; co widuja oczy twoje.
\par 8 Nie pokwapiaj sie do swaru, bys snac na ostatek nie wiedzial, co masz czynic, gdyby cie zawstydzil blizni twój.
\par 9 Prowadz do konca sprawe swoje z przyjacielem twoim, a tajemnicy drugiego nie objawiaj;
\par 10 By cie snac nie zelzyl ten, co cie slucha, a nieslawa twoja zostalaby na tobie.
\par 11 Jakie jest jablko zlote z wyrzezaniem srebrnem, takiec jest slowo do rzeczy powiedziane.
\par 12 Ten, który madrze napomina, jest u tego, co slucha, jako nausznica zlota, i klejnot z szczerego zlota.
\par 13 Jako zimno sniezne czasu zniwa: tak posel wierny tym, którzy go posylaja; bo dusze panów swych ochladza.
\par 14 Czlowiek, który sie chlubi darem zmyslonym, jest jako wiatr i obloki bez deszczu.
\par 15 Ksiaze cierpliwoscia bywa zmiekczony, a jezyk lagodny kosci lamie.
\par 16 Znajdzieszli miód, jedzze, ilec potrzeba, by snac objadlszy sie go nie zwrócil.
\par 17 Powsciagnij noge twoje od domu blizniego twego, by snac bedac ciebie syt, nie mial cie w nienawisci.
\par 18 Kazdy, kto mówi falszywe swiadectwo przeciw blizniemu swemu, jest jako mlot, i miecz, i strzala ostra.
\par 19 Ufnosc w czlowieku przewrotnym jest w dzien ucisku jako zab wylamany i noga wywiniona.
\par 20 Jako ten, który zewloczy odzienie czasu zimy, albo leje ocet na saletre, taki jest ten, który spiewa piesni sercu smutnemu.
\par 21 Jezlizeby laknal ten, co cie nienawidzi, nakarm go chlebem; a jezliby pragnal, daj mu sie napic wody;
\par 22 Bo wegle rozpalone zgromadzisz na glowe jego, a Pan ci nagrodzi.
\par 23 Jako wiatr pólnocny deszcz przynosi: tak przynosi twarz gniewliwa jezyk uszczypliwy.
\par 24 Lepiej mieszkac w kacie pod dachem, nizeli z zona swarliwa w domu przestronnym.
\par 25 Jako woda chlodna duszy pragnacej: tak wiesc dobra z ziemi dalekiej.
\par 26 Jako zródlo nogami pomacone, albo zdrój zepsuty: tak sprawiedliwy, który upada przed niezboznym.
\par 27 Jako jesc wiele miodu nie jest rzecz dobra: tak szukanie wlasnej slawy jest nieslawne.
\par 28 Maz, który nie ma mocy nad duchem swoim, jest jako miasto rozwalone bez muru.

\chapter{26}

\par 1 Jako snieg w lecie, i jako deszcz we zniwa; tak glupiemu nie przystoi chwala.
\par 2 Jako sie ptak tam i sam tula, i jako jaskólka lata: tak przeklestwo niezasluzone nie przyjdzie.
\par 3 Bicz na konia, oglów na osla, a kij potrzebny jest na grzbiet glupiego.
\par 4 Nie odpowiadaj glupiemu wedlug glupstwa jego, abys mu i ty nie byl podobny.
\par 5 Odpowiedz glupiemu wedlug glupstwa jego, aby sie sobie nie zdal byc madrym.
\par 6 Jakoby nogi obcial, tak sie bezprawia dopuszcza, kto sie glupiemu poselstwa powierza.
\par 7 Jako nierówne sa golenie u chromego: tak jest powiesc w ustach glupich.
\par 8 Jako kiedy kto przywiazuje kamien drogi do procy: tak czyni ten, który uczciwosc glupiemu wyrzadza.
\par 9 Jako ciernie, gdy sie dostana w reke pijanego: tak przypowiesc jest w ustach glupich.
\par 10 Wielki Pan stworzyl wszystko, a daje zaplate glupiemu, daje takze zaplate przestepcom.
\par 11 Jako pies wraca sie do zwrócenia swego: tak glupi powtarza glupstwo swoje.
\par 12 Ujrzyszli czlowieka, co sie sobie zda byc madrym, nadzieja o glupim lepsza jest, nizeli o nim.
\par 13 Leniwy mówi: lew na drodze, lew na ulicach.
\par 14 Jako sie drzwi obracaja na zawiasach swoich: tak leniwiec na lózku swojem.
\par 15 Leniwiec reke kryje do zanadrzy swych, a ciezko mu jej podnosic do ust swoich.
\par 16 Leniwiec zda sie sobie byc medrszym, nizeli siedm odpowiadajacych z rozsadkiem.
\par 17 Jakoby tez psa za uszy lapal, kto sie mimo idac w cudza zwade wdaje.
\par 18 Jako szalony wypuszcza iskry i strzaly smiertelne:
\par 19 Tak jest kazdy, który podchodzi przyjaciela swego, a mówi: Azam ja nie zartowal?
\par 20 Gdy niestaje drew, gasnie ogien; tak gdy nie bedzie klatecznika, ucichnie zwada.
\par 21 Jako wegiel martwy sluzy do rozniecenia, i drwa do ognia; tak czlowiek swarliwy do rozniecenia zwady.
\par 22 Slowa obmówcy sa jako slowa zranionych; a wszakze przenikaja do wnetrznosci zywota.
\par 23 Wargi nieprzyjacielskie i serce zle sa jako srebrna piana, która polewaja naczynie gliniane.
\par 24 Ten, co kogo nienawidzi, za inszego sie udaje wargami swemi; ale w sercu swojem mysli o zdradzie.
\par 25 Gdyc sie ochotnym mowa swa ukazuje, nie wierz mu: bo siedmioraka obrzydliwosc jest w sercu jego.
\par 26 Nienawisc zdradliwie bywa pokryta; ale odkryta bywa zlosc jej w zgromadzeniu.
\par 27 Kto drugiemu dól kopie, wpada wen; a kto kamien toczy, na niego sie obraca.
\par 28 Czlowiek jezyka klamliwego ma utrapienie w nienawisci, a usta lagodne przywodza do upadku.

\chapter{27}

\par 1 Nie chlub sie ze dnia jutrzejszego; bo nie wiesz, coc przyniesie dzien dzisiejszy.
\par 2 Niechaj cie kto inny chwali, a nie usta twoje; obcy, a nie wargi twoje.
\par 3 Ciezkic jest kamien, i piasek wazny; ale gniew glupiego ciezszy, niz to oboje.
\par 4 Okrutnyc jest gniew, i nagla popedliwosc; ale przed zazdroscia któz sie ostoi?
\par 5 Lepsza jest przymówka jawna, nizeli milosc skryta.
\par 6 Lepsze sa rany od przyjaciela, niz lagodne calowanie czlowieka nienawidzacego.
\par 7 Dusza nasycona i plastr miodu podepcze; ale glodnej duszy i gorzkosc wszelaka slodka.
\par 8 Jako ptak odlatuje od gniazda swojego: tak czlowiek odchodzi od miejsca swego.
\par 9 Jako masc i kadzenie uwesela serce: tak slodkosc przyjaciela uwesela wiecej, niz wlasna rada.
\par 10 Przyjaciela twego, i przyjaciela ojca twego nie opuszczaj, a do domu brata twego nie wchodz w dzien utrapienia twego; bo lepszy sasiad bliski, niz brat daleki.
\par 11 Badz madrym, synu mój! a uweselaj serce moje, abym mial co odpowiedziec temu, któryby mi uragal.
\par 12 Ostrozny, upatrujac zle, ukrywa sie; ale prostak wprost idac, w szkode popada.
\par 13 Wezmij szate tego, któryc reczyl za obcego; a od tego, który reczyl za cudzoziemke, wezmij zastaw jego.
\par 14 Temu, który przyjacielowi swemu wielkim glosem rano wstawajac blogoslawi, poczytane to bedzie za przeklestwo.
\par 15 Kapanie ustawiczne w dzien gwaltownego deszczu, i zona swarliwa sa sobie podobni;
\par 16 Kto ja kryje, kryje wiatr, a wonia wyda; jako olejek wonny w prawej rece jego.
\par 17 Zelazo zelazem bywa naostrzone; tak maz zaostrza oblicze przyjaciela swego.
\par 18 Kto strzeze drzewa figowego, pozywa owocu jego; takze kto strzeze pana swego, uczczony bedzie.
\par 19 Jako sie w wodzie twarz przeciwko twarzy ukazuje: tak serce czlowiecze przeciw czlowiekowi.
\par 20 Pieklo i zatracenie nie moga byc nasycone; takze i oczy ludzkie nasycic sie nie moga.
\par 21 Tygiel srebra a piec zlota doswiadcza, a czlowieka wiesc slawy jego.
\par 22 Chocbys glupiego i w stepie miedzy krupami staporem stlukl, nie odejdzie od niego glupstwo jego.
\par 23 Dogladaj pilnie dobytku twego, a miej piecze o trzodach twoich.
\par 24 Boc nie na wieki trwa bogactwo, ani korona od narodu do narodu.
\par 25 Gdy wzrasta trawa, a ukazuja sie ziola, tedy z gór siano zbieraja.
\par 26 Owce beda na szaty twoje, a nagroda polna kozly.
\par 27 Nadto dostatek mleka koziego na pokarm twój, na pokarm domu twego, i na pozywienie dziewek twoich.

\chapter{28}

\par 1 Uciekaja niezbozni, choc ich nikt nie goni: ale sprawiedliwi jako lwie mlode sa bez bojazni.
\par 2 Dla przestepstwa ziemi wiele bywa ksiazat jej; ale dla czlowieka roztropnego i umiejetnego trwale bywa panstwo.
\par 3 Maz ubogi, który uciska nedznych, podobny jest dzdzowi gwaltownemu, po którym chleba nie bywa.
\par 4 Którzy opuszczaja zakon, chwala niezboznika; ale ci, którzy strzega zakonu, sa im odpornymi.
\par 5 Ludzie zli nie zrozumiewaja sadu; ale którzy Pana szukaja, rozumieja wszystko.
\par 6 Lepszy jest ubogi, który chodzi w uprzejmosci swojej, nizeli przewrotny na drogach swych, chociaz jest bogaty.
\par 7 Kto strzeze zakonu, jest synem roztropnym; ale kto karmi obzerce, czyni zelzywosc ojcu swemu.
\par 8 Kto rozmnaza majetnosc swoje z lichwy i z platu, temu ja zbiera, który ubogiemu szczodrze bedzie dawal.
\par 9 Kto odwraca ucho swe, aby nie sluchal zakonu, i modlitwa jego jest obrzydliwoscia.
\par 10 Kto zawodzi uprzejmych na droge zla, w dól swój sam wpadnie; ale uprzejmi odziedzicza rzeczy dobre.
\par 11 Maz bogaty zda sie sobie byc madrym; ale ubogi roztropny dochodzi go.
\par 12 Gdy sie raduja sprawiedliwi, wielka jest slawa; ale gdy powstawaja niepobozni, kryje sie czlowiek.
\par 13 Kto pokrywa przestepstwa swe, nie poszczesci mu sie; ale kto je wyznaje i opuszcza, milosierdzia dostapi.
\par 14 Blogoslawiony czlowiek, który sie zawsze boi; ale kto zatwardza serce swoje, wpada w zle.
\par 15 Pan niezbozny, panujacy nad ludem ubogim jest jako lew ryczacy, i jako niedzwiedz glodny.
\par 16 Ksiaze bezrozumny wielkim jest drapiezca: ale kto nienawidzi lakomstwa, przedluzy dni swoje.
\par 17 Czlowieka, który gwalt czyni krwi ludzkiej, chocby i do dolu uciekal, nikt nie zatrzyma.
\par 18 Kto chodzi w uprzejmosci, zachowany bedzie; ale przewrotny na drogach swoich oraz upadnie.
\par 19 Kto sprawuje ziemie swoje, chlebem nasycony bywa; ale kto nasladuje próznujacych, ubóstwem nasycony bywa.
\par 20 Maz wierny przyczyni blogoslawienstwa; ale kto sie predko chce zbogacic, nie bywa bez winy.
\par 21 Miec wzglad na osobe, rzecz niedobra; bo nie jeden dla kesa chleba staje sie przewrotnym.
\par 22 Predko chce czlowiek zazdrosciwy zbogatniec, a nie wie, iz nac niedostatek przyjdzie.
\par 23 Kto strofuje czlowieka, wieksza potem laske znajduje, niz ten, co pochlebia jezykiem.
\par 24 Kto lupi ojca swego, albo matke swoja, a mówi, iz to nie grzech: towarzyszem jest mezobójcy.
\par 25 Wysokomyslny wszczyna zwade; ale kto nadzieje ma w Panu, dostatek miec bedzie.
\par 26 Kto ufa w sercu swem, glupi jest; ale kto sobie madrze poczyna, ten ujdzie nieszczescia.
\par 27 Kto daje ubogiemu, nie bedzie mial niedostatku; ale kto od niego odwraca oczy swe, wielkie przeklestwa nan przyjda.
\par 28 Gdy niepobozni powstawaja, kryje sie czlowiek; ale gdy gina, sprawiedliwi sie rozmnazaja.

\chapter{29}

\par 1 Czlowiek, który na czeste karanie zatwardza kark swój, nagle zniszczeje, i nie wskóra.
\par 2 Gdy sie rozmnazaja sprawiedliwi, weseli sie lud; ale gdy panuje bezboznik, wzdycha lud.
\par 3 Maz, który miluje madrosc, uwesela ojca swego; ale kto chowa nierzadnice, traci majetnosc.
\par 4 Król sadem ziemie utwierdza; ale maz, który dary bierze, podwraca ja.
\par 5 Czlowiek, który pochlebia przyjacielowi swemu, rozciaga siec przed nogami jego.
\par 6 Wystepek zlego jest mu sidlem; ale sprawiedliwy spiewa i weseli sie.
\par 7 Sprawiedliwy wyrozumiewa sprawe nedznych; ale niezboznik nie ma na to rozumu i umiejetnosci.
\par 8 Mezowie nasmiewcy zawodza miasto; ale madrzy odwracaja gniew.
\par 9 Maz madry, wiedzieli spór z mezem glupim, chocby sie gniewal, chocby sie tez smial, nie bedzie mial pokoju.
\par 10 Mezowie krwawi nienawidza uprzejmego; ale uprzejmi staranie wioda o dusze jego.
\par 11 Wszystkiego ducha swego wywiera glupi, ale madry na dalszy czas go zawsciaga.
\par 12 Pana, który rad slucha slów klamliwych, wszyscy sludzy jego sa niepobozni.
\par 13 Ubogi i zdzierca spotkali sie; a wszakze obydwóch oczy Pan oswieca.
\par 14 Króla, który sadzi ucisnionych wedlug prawdy, stolica jego na wieki utwierdzona bedzie.
\par 15 Rózga i karnosc madrosc daje; ale dziecie swawolne zawstydza matke swoje.
\par 16 Gdy sie rozmnazaja niezbozni, rozmnaza sie i przestepstwo; ale sprawiedliwi upadek ich ogladaja.
\par 17 Karz syna twego, a sprawic odpocznienie, i sposobi rozkosz duszy twojej.
\par 18 Gdy proroctwo ustaje, lud bywa rozproszony; ale kto strzeze zakonu, blogoslawiony jest.
\par 19 Sluga nie bywa slowami naprawiony; bo choc rozumie, jednak nie odpowiada.
\par 20 Ujrzysz czlowieka skwapliwego w sprawach swoich; ale lepsza jest nadzieja o glupim, niz o nim.
\par 21 Kto w rozkoszy chowa z dziecinstwa sluge swego, na ostatek bedzie chcial byc za syna.
\par 22 Czlowiek gniewliwy wszczyna zwade, a pierzchliwy wiele grzeszy.
\par 23 Pycha czlowiecza poniza go; ale pokorny w duchu slawy dostepuje.
\par 24 Kto spólkuje ze zlodziejem, ma w nienawisci dusze swoje; takze tez kto przeklestwa slyszy, a nie objawia go.
\par 25 Strach czlowieczy stawia sobie sidlo; ale kto ma nadzieje w Panu, wywyzszony bedzie.
\par 26 Wiele tych, co szukaja twarzy panów; alec od Pana jest sad kazdego.
\par 27 Sprawiedliwym jest maz niezbozny obrzydliwoscia; a zasie kto w uprzejmosci chodzi, jest niezboznym obrzydliwoscia.

\chapter{30}

\par 1 Te sa slowa Agóra, syna Jakiego, i zebranie mów tegoz meza do Ityjela, do Ityjela i Uchala.
\par 2 Zaistem jest glupszy nad innych, a rozumu czlowieczego nie mam.
\par 3 I nie nauczylem sie madrosci, a umiejetnosci swietych nie umiem.
\par 4 Któz wstapil na niebo, i zasie zstapil? któz zgromadzil wiatr do garsci swych? Któz zagarnal wody do szaty swej? któz utwierdzil wszystkie konczyny ziemi? Cóz za imie jego? i co za imie syna jego? Wieszze?
\par 5 Wszelka mowa Boza jest czysta; on jest tarcza tym, którzy ufaja w nim.
\par 6 Nie przydawaj do slów jego, aby cie nie karal, a bylbys znaleziony w klamstwie.
\par 7 Dwóch rzecz zadam od ciebie, nie odmawiajze mi pierwej niz umre.
\par 8 Marnosc i slowo klamliwe oddal odemnie; ubóstwa i bogactwa nie dawaj mi; zyw mie tylko pokarmem wedlug potrzeby mojej;
\par 9 Abym snac nasyconym bedac nie zaprzal sie ciebie, i nie rzekl: Któz jest Pan? Albo zubozawszy zebym nie kradl, i nie bral nadaremno imienia Boga mego.
\par 10 Nie podwodz na sluge przed Panem jego, byc snac nie zlorzeczyl, a ty abys nie zgrzeszyl.
\par 11 Jest rodzaj, który ojcu swemu zlorzeczy, a matce swojej nie blogoslawi.
\par 12 Jest rodzaj, który sie zda sobie byc czystym, choc od plugastwa swego nie jest omyty.
\par 13 Jest rodzaj, którego sa wyniosle oczy, i powieki jego wywyzszone sa.
\par 14 Jest rodzaj, którego zeby sa jako miecze, a trzonowe zeby jego jako noze na pozarcie ubogich na ziemi, a nedzników miedzy ludzmi.
\par 15 Pijawka ma dwie córki, które mówia: Przynies, przynies.Trzy rzeczy sa, które nie bywaja nasycone, owszem cztery, które nie mówia: Dosyc.
\par 16 Grób, i zywot nieplodny, ziemia tez nie bywa nasycona woda, a ogien nie mówi: Dosyc.
\par 17 Oko, które sie nasmiewa z ojca, i wzgardza posluszenstwem macierzynskiem, wykluja kruki u potoków, i orleta je wyjedza.
\par 18 Te trzy rzeczy sa ukryte przedemna, owszem cztery, których nie wiem:
\par 19 Drogi orlej na powietrzu, drogi wezowej na skale, drogi okretowej w posród morza, i drogi mezowej z panna.
\par 20 Takac jest droga niewiasty cudzolozacej: je, a uciera usta swoje, i mówi: Nie popelnilam zlego uczynku.
\par 21 Dla trzech rzeczy porusza sie ziemia, owszem dla czterech, których zniesc nie moze:
\par 22 Dla slugi, kiedy panuje, i dla glupiego, kiedy sie nasyci chleba;
\par 23 Dla przemierzlej niewiasty, kiedy za maz idzie, i dla dziewki, kiedy dziedziczy po pani swojej.
\par 24 Tec sa cztery rzeczy najmniejsze na ziemi, wszakze sa medrsze nad medrców:
\par 25 Mrówki, huf slaby, które sobie jednak w lecie gotuja pokarm swój;
\par 26 Króliki, twór slaby, którzy jednak buduja w skale dom swój;
\par 27 Szarancze króla nie maja, a wszakze wszystkie hufami wychodza;
\par 28 Pajak rekoma robi, a bywa w palacach królewskich.
\par 29 Te trzy rzeczy sa, które wspaniale chodza, owszem cztery, które zmezyle chodza:
\par 30 Lew najmocniejszy miedzy zwierzetami, który przed nikim nie ustepuje:
\par 31 Kon na biodrach przepasany, i koziel, i król, przeciw któremu zaden nie powstaje.
\par 32 Jezlis glupio uczynil, gdys sie wynosil, albo jezlis zle myslil, polózze reke na usta.
\par 33 Kto tlucze smietane, wybija maslo; a kto bardzo nos wyciera, wyciska krew; tak kto wzbudza gniew, wszczyna zwade.

\chapter{31}

\par 1 Te sa slowa Lemuela króla, i zebranie mów, któremi go cwiczyla matka jego.
\par 2 Cóz rzeke, synu mój? cóz rzeke, synu zywota mego? i cóz rzeke, synu slubów moich?
\par 3 Nie dawaj niewiastom sily twojej, ani dróg twoich tym, którzy do zginienia królów przywodza.
\par 4 Nie królom, o Lemuelu! nie królom nalezy pic wino, a nie panom bawic sie napojem mocnym;
\par 5 By snac pijac nie zapomnial na ustawy, a nie odmienil spraw wszystkich ludzi ucisnionych.
\par 6 Dajcie napój mocny ginacemu, a wino tym, którzy sa ducha sfrasowanego.
\par 7 Niech sie napije, a zapomni ubóstwa swego, a na utrapienie swoje niech wiecej nie wspomni.
\par 8 Otwórz usta swe za niemym w sprawie wszystkich osadzonych na smierc.
\par 9 Otwórz usta swe, sadz sprawiedliwie, a podejmij sie sprawy ubogiego i nedznego.
\par 10 Któz znajdzie niewiaste stateczna, gdyz nad perly daleko wieksza jest cena jej?
\par 11 Serce meza jej ufa jej, a na korzysciach schodzic mu nie bedzie.
\par 12 Dobrze mu czyni, a nie zle, po wszystkie dni zywota swego.
\par 13 Szuka welny i lnu, a pracuje ochotnie rekami swemi.
\par 14 Podobna jest okretom kupieckim; z daleka przywodzi zywnosc swoje.
\par 15 I wstaje bardzo rano, a daje pokarm czeladzi swej, a obrok sluszny dziewkom swym.
\par 16 Obmysla role, i ujmuje ja; z zarobku rak swoich szczepi winnice.
\par 17 Przepasuje moca biodra swe, a posila ramiona swoje.
\par 18 Doswiadcza, ze jest dobra skrzetnosc jej, a nie gasnie w nocy pochodnia jej.
\par 19 Rece swoje obraca do kadzieli, a palcami swemi trzyma wrzeciono.
\par 20 Reke swa otwiera ubogiemu, a rece swoje wyciaga ku nedznemu.
\par 21 Nie boi sie o czeladz swoje czasu sniegu; albowiem wszystka czeladz jej obloczy sie w szate dwoista.
\par 22 Kobierce sobie robi; plótno subtelne i szarlat jest odzieniem jej.
\par 23 Znaczny jest w bramach maz jej, gdy siedzi miedzy starszymi ziemi.
\par 24 Plótno robi, i sprzedaje, takze pasy sprzedaje kupcowi.
\par 25 Moc i przystojnosc jest odzieniem jej; nie frasuje sie o czasy przyszle.
\par 26 Madrze otwiera usta swe, a nauka milosierdzia jest na jezyku jej.
\par 27 Doglada rzadu w domu swym, a chleba próznujac nie je.
\par 28 Powstawszy synowie jej blogoslawia jej; takze i maz jej chwali ja,
\par 29 Mówiac: Wiele niewiast grzecznie sobie poczynaly; ale je ty przechodzisz wszystkie.
\par 30 Omylna jest wdziecznosc, i marna pieknosc; ale niewiasta, która sie Pana boi, ta pochwaly godna.
\par 31 Dajcie jej z owocu reku jej, a niechaj ja chwala w bramach uczynki jej.


\end{document}