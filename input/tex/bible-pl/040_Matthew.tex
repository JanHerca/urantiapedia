\begin{document}

\title{Mateusza}


\chapter{1}

\par 1 Ksiega o rodzie Jezusa Chrystusa, syna Dawidowego, syna Abrahamowego.
\par 2 Abraham splodzil Izaaka, a Izaak splodzil Jakóba, a Jakób splodzil Jude, i braci jego.
\par 3 A Juda splodzil Faresa i Zare z Tamary, a Fares splodzil Hesrona, a Hesron splodzil Arama.
\par 4 A Aram splodzil Aminadaba, a Aminadab splodzil Naasona, a Naason splodzil Salmona.
\par 5 A Salmon splodzil Booza z Rachaby, a Booz splodzil Obeda z Ruty, a Obed splodzil Jessego.
\par 6 A Jesse splodzil Dawida króla, a Dawid król splodzil Salomona z tej, która byla zona Uryjaszowa.
\par 7 A Salomon splodzil Roboama, a Roboam splodzil Abijasza, a Abijasz splodzil Aze.
\par 8 A Aza splodzil Jozafata, a Jozafat splodzil Jorama, a Joram splodzil Ozyjasza.
\par 9 A Ozyjasz splodzil Joatama, a Joatam splodzil Achaza, a Achaz splodzil Ezechijasza.
\par 10 A Ezechijasz splodzil Manasesa, a Manases splodzil Amona, a Amon splodzil Jozyjasza.
\par 11 A Jozyjasz splodzil Jechonijasza i braci jego podczas zaprowadzenia do Babilonu.
\par 12 A po zaprowadzeniu do Babilonu Jechonijasz splodzil Salatyjela, a Salatyjel splodzil Zorobabela.
\par 13 A Zorobabel splodzil Abijuda, a Abijud splodzil Elijakima, a Elijakim splodzil Azora.
\par 14 A Azor splodzil Sadoka, a Sadok splodzil Achima, a Achim splodzil Elijuda.
\par 15 A Elijud splodzil Eleazara, a Eleazar splodzil Matana, a Matan splodzil Jakóba.
\par 16 A Jakób splodzil Józefa, meza Maryi, z której sie narodzil Jezus, którego zowia Chrystus.
\par 17 A tak wszystkiego pokolenia od Abrahama az do Dawida jest pokolen czternascie, a od Dawida az do zaprowadzenia do Babilonu, pokolen czternascie, a od zaprowadzenia do Babilonu az do Chrystusa, pokolen czternascie.
\par 18 A narodzenie Jezusa Chrystusa takie bylo: Albowiem gdy Maryja, matka jego, poslubiona byla Józefowi, pierwej nizeli sie zeszli, znaleziona jest brzemienna z Ducha Swietego.
\par 19 Ale Józef, maz jej, bedac sprawiedliwym i nie chcac jej oslawic, chcial ja potajemnie opuscic.
\par 20 A gdy on o tem zamyslal, oto mu sie Aniol Panski we snie ukazal, mówiac: Józefie, synu Dawidów! nie bój sie przyjac Maryi, zony twojej; albowiem, co sie w niej poczelo, z Ducha Swietego jest.
\par 21 A urodzi syna, i nazowiesz imie jego Jezus; albowiem on zbawi lud swój od grzechów ich.
\par 22 A to sie wszystko stalo, aby sie wypelnilo, co powiedziano od Pana przez proroka, mówiacego:
\par 23 Oto panna bedzie brzemienna i porodzi syna, a nazowia imie jego Emanuel, co sie wyklada: Bóg z nami.
\par 24 Tedy Józef ocuciwszy sie ze snu, uczynil, jako mu rozkazal Aniol Panski, i przyjal zone swoje;
\par 25 Ale jej nie uznal, az porodzila onego syna swego pierworodnego, i nazwal imie jego Jezus.

\chapter{2}

\par 1 A gdy sie Jezus narodzil w Betlehemie Judzkiem za dnia Heroda króla, oto medrcy ze wschodu slonca przybyli do Jeruzalemu, mówiac:
\par 2 Gdziez jest ten, który sie narodzil, król zydowski? Bosmy widzieli gwiazde jego na wschód slonca, i przyjechalismy, abysmy mu sie poklonili.
\par 3 Co gdy król Herod uslyszal, zatrwozyl sie, i wszystko Jeruzalem z nim.
\par 4 Przetoz zebrawszy wszystkie przedniejsze kaplany i nauczyciele ludu, dowiadywal sie od nich, gdzie by sie mial Chrystus narodzic.
\par 5 A oni mu rzekli: W Betlehemie Judzkiem: bo tak napisano przez proroka:
\par 6 I ty Betlehemie, ziemio Judzka! zadna miara nie jestes najmniejsze miedzy ksiazety Judzkimi; albowiem z ciebie wynijdzie wódz, który rzadzic bedzie lud mój Izraelski.
\par 7 Tedy Herod wezwawszy potajemnie onych medrców, pilnie sie wywiadywal od nich o czasie, którego sie gwiazda ukazala.
\par 8 A poslawszy je do Betlehemu, rzekl: Jechawszy, pilnie sie wywiadujcie o tem dzieciatku; a gdy znajdziecie, oznajmijcie mi, abym i ja przyjechawszy, poklonil mu sie.
\par 9 Oni tedy, wysluchawszy króla, poszli; a oto ona gwiazda, która widzieli na wschód slonca, prowadzila je, az przyszedlszy, stanela nad miejscem, gdzie bylo dzieciatko;
\par 10 A gdy ujrzeli one gwiazde, uradowali sie radoscia bardzo wielka;
\par 11 I wszedlszy w dom, znalezli dzieciatko z Maryja, matka jego, a upadlszy, poklonili mu sie, i otworzywszy skarby swoje, ofiarowali mu dary: zloto i kadzidlo i myrre.
\par 12 Lecz bedac upomnieni od Boga we snie, aby sie nie wracali do Heroda, insza droga wrócili sie do krainy swojej.
\par 13 A gdy oni odeszli, oto Aniol Panski ukazal sie we snie Józefowi mówiac: Wstawszy, wezmij to dzieciatko i matke jego, a uciecz do Egiptu, a badz tam, az ci powiem; albowiem Herod bedzie szukal dzieciatka, aby je zatracil.
\par 14 Który wstawszy, wzial dzieciatko i matke jego w nocy, i uszedl do Egiptu;
\par 15 I byl tam az do smierci Herodowej, aby sie wypelnilo, co powiedziano od Pana przez proroka, mówiacego: Z Egiptum wezwal syna mego.
\par 16 Tedy Herod ujrzawszy, ze byl oszukany od medrców, rozgniewal sie bardzo, a poslawszy pobil wszystkie dziatki, które byly w Betlehemie i po wszystkich granicach jego, od dwóch lat i nizej, wedlug czasu, o którym sie byl pilnie wywiedzial od medrców.
\par 17 Tedy sie wypelnilo, co powiedziano przez Jeremijasza proroka, mówiacego:
\par 18 Glos w Ramie slyszany jest, lament, i placz, i narzekanie wielkie: Rachel placzaca synów swoich nie dala sie pocieszyc, przeto, ze ich nie masz.
\par 19 A gdy umarl Herod, oto Aniol Panski ukazal sie we snie Józefowi w Egipcie,
\par 20 Mówiac: Wstawszy, wezmij dzieciatko i matke jego, a idz do ziemi Izraelskiej; albowiem pomarli ci, którzy szukali duszy dzieciecej.
\par 21 A on wstawszy, wzial do siebie dzieciatko i matke jego, i przyszedl do ziemi Izraelskiej.
\par 22 Lecz gdy uslyszal, iz Archelaus królowal w Judzkiej ziemi na miejscu Heroda, ojca swego, bal sie tam isc; ale napomniany bedac od Boga we snie, ustapil w strony Galilejskie;
\par 23 I przyszedlszy mieszkal w miescie, które zowia Nazaret, aby sie wypelnilo co powiedziano przez proroki: Iz Nazarejczykiem nazwany bedzie.

\chapter{3}

\par 1 W one dni przyszedl Jan Chrzciciel, kazac na puszczy w ziemi Judzkiej,
\par 2 A mówiac: Pokutujcie; albowiem sie przyblizylo królestwo niebieskie.
\par 3 Tenci bowiem jest on, o którym powiedziano przez Izajasza proroka, mówiacego: Glos wolajacego na puszczy: Gotujcie droge Panska, proste czyncie sciezki jego.
\par 4 A ten Jan mial odzienie z siersci wielbladziej, i pas skórzany okolo biódr swoich, a pokarm jego byl szarancza i miód lesny.
\par 5 Tedy wychodzilo do niego Jeruzalem i wszystka Judzka ziemia i wszystka kraina okolo Jordanu;
\par 6 I byli chrzczeni od niego w Jordanie, wyznawajac grzechy swoje.
\par 7 A gdy ujrzal wiele z Faryzeuszów i Saduceuszów przychodzacych do chrztu swego, rzekl im: Rodzaju jaszczurczy! któz wam pokazal, zebyscie uciekali przed przyszlym gniewem?
\par 8 Przynosciez tedy owoce godne pokuty;
\par 9 A nie mniemajcie, ze mozecie mówic sami o sobie: Ojca mamy Abrahama; albowiemci powiadam wam, iz Bóg i z tych kamieni wzbudzic moze dzieci Abrahamowi.
\par 10 A juz i siekiera do korzenia drzew przylozona jest; wszelkie tedy drzewo, które nie przynosi owocu dobrego, bywa wyciete, i w ogien wrzucone.
\par 11 Jac was chrzcze woda ku pokucie; ale ten, który idzie za mna, mocniejszy jest nad mie; któregom obuwia nosic nie jest godzien; ten was chrzcic bedzie Duchem Swietym i ogniem.
\par 12 Którego lopata jest w reku jego, a wyczysci bojewisko swoje, i zgromadzi pszenice swoje do gumna, ale plewy spali ogniem nieugaszonym.
\par 13 Tedy Jezus przyszedl od Galilei nad Jordan do Jana, aby byl ochrzczony od niego;
\par 14 Ale mu Jan bardzo zabranial, mówiac: Ja potrzebuje, abym byl ochrzczony od ciebie, a ty idziesz do mnie?
\par 15 A odpowiadajac Jezus, rzekl do niego: Zaniechaj teraz; albowiem tak przystoi na nas, abysmy wypelnili wszelka sprawiedliwosc; tedy go zaniechal.
\par 16 A Jezus ochrzczony bedac, wnet wystapil z wody, a oto sie mu otworzyly niebiosa, i widzial Ducha Bozego, zstepujacego jako golebice, i przychodzacego na niego;
\par 17 A oto glos z niebios mówiacy: Ten jest on Syn mój mily, w którym mi sie upodobalo.

\chapter{4}

\par 1 Tedy Jezus zawiedziony jest na puszcze od Ducha, aby byl kuszony od dyjabla.
\par 2 A gdy poscil czterdziesci dni i czterdziesci nocy, potem laknal.
\par 3 I przystapiwszy do niego kusiciel, rzekl: Jezlis jest Syn Bozy, rzecz, aby sie te kamienie staly chlebem.
\par 4 A on odpowiadajac rzekl: Napisano: Nie samym chlebem czlowiek zyc bedzie, ale kazdem slowem pochodzacem przez usta Boze.
\par 5 Tedy go wzial dyjabel do miasta swietego, i postawil go na ganku koscielnym,
\par 6 I rzekl mu: Jezlis jest Syn Bozy, spusc sie na dól, albowiem napisano: Iz Aniolom swoim przykazal o tobie, i beda cie na rekach nosili, abys snac nie obrazil o kamien nogi swojej.
\par 7 Rzekl mu Jezus: Zasie napisano: Nie bedziesz kusil Pana, Boga twego.
\par 8 Wzial go zasie dyjabel na góre bardzo wysoka, i pokazal mu wszystkie królestwa swiata i slawe ich,
\par 9 I rzekl mu: To wszystko dam tobie, jezli upadlszy, poklonisz mi sie.
\par 10 Tedy mu rzekl Jezus: Pójdz precz, szatanie! albowiem napisano: Panu Bogu twemu klaniac sie bedziesz, i jemu samemu sluzyc bedziesz.
\par 11 Tedy go opuscil dyjabel, a oto Aniolowie przystapili i sluzyli mu.
\par 12 A gdy uslyszal Jezus, iz Jan byl podany do wiezienia, wrócil sie do Galilei;
\par 13 A opusciwszy Nazaret, przyszedl, i mieszkal w Kapernaum, które jest nad morzem w granicach Zabulonowych i Neftalimowych;
\par 14 Aby sie wypelnilo, co powiedziano przez Izajasza proroka, mówiacego:
\par 15 Ziemia Zabulonowa i ziemia Neftalimowa przy drodze morskiej za Jordanem, Galilea poganów;
\par 16 Lud, który siedzial w ciemnosci, widzial swiatlosc wielka, a siedzacym w krainie i w cieniu smierci weszla im swiatlosc.
\par 17 Od onego czasu poczal Jezus kazac i mówic: Pokutujcie, albowiem sie przyblizylo królestwo niebieskie.
\par 18 A gdy Jezus chodzil nad morzem Galilejskiem, ujrzal dwóch braci: Szymona, którego zowia Piotrem, i Andrzeja, brata jego, którzy zapuszczali siec w morze; albowiem byli rybitwy.
\par 19 I rzekl im: Pójdzcie za mna, a uczynie was rybitwami ludzi.
\par 20 A oni zaraz opusciwszy sieci, szli za nim.
\par 21 A postapiwszy stamtad, ujrzal drugich dwóch braci, Jakóba, syna Zebedeuszowego, i Jana, brata jego, w lodzi z Zebedeuszem, ojcem ich, poprawiajacych sieci swoje, i wezwal ich.
\par 22 A oni wnetze opusciwszy lódz i ojca swego, poszli za nim.
\par 23 I obchodzil Jezus wszystke Galileja, uczac w bóznicach ich, i kazac Ewangielija królestwa, a uzdrawiajac wszelka chorobe i wszelka niemoc miedzy ludem.
\par 24 I rozeszla sie wiesc o nim po wszystkiej Syryi; i przywodzono do niego wszystkie zle sie majace, a rozmaitemi chorobami i mekami zdjete, takze i opetane, i lunatyki i powietrzem ruszone; i uzdrawial je.
\par 25 A szedl za nim lud wielki z Galilei, z dziesieciu miast, i z Jeruzalemu, i z Judzkiej ziemi, i zza Jordanu.

\chapter{5}

\par 1 A Jezus widzac lud, wstapil na góre; a gdy usiadl, przystapili do niego uczniowie jego.
\par 2 A otworzywszy usta swe, uczyl je, mówiac:
\par 3 Blogoslawieni ubodzy w duchu; albowiem ich jest królestwo niebieskie.
\par 4 Blogoslawieni, którzy sie smeca; albowiem pocieszeni beda.
\par 5 Blogoslawieni cisi; albowiem oni odziedzicza ziemie.
\par 6 Blogoslawieni, którzy lakna i pragna sprawiedliwosci; albowiem oni nasyceni beda.
\par 7 Blogoslawieni milosierni: albowiem oni milosierdzia dostapia.
\par 8 Blogoslawieni czystego serca; albowiem oni Boga ogladaja.
\par 9 Blogoslawieni pokój czyniacy; albowiem oni synami Bozymi nazwani beda.
\par 10 Blogoslawieni, którzy cierpia przesladowanie dla sprawiedliwosci; albowiem ich jest królestwo niebieskie.
\par 11 Blogoslawieni jestescie, gdy wam zlorzeczyc beda, i przesladowac was, i mówic wszystko zle przeciwko wam, klamajac dla mnie.
\par 12 Radujcie sie, i weselcie sie; albowiem zaplata wasza obfita jest w niebiesiech; tak bowiem przesladowali proroki, którzy byli przed wami.
\par 13 Wy jestescie sól ziemi; jezli tedy sól zwietrzeje, czemze solic beda? Do niczego sie juz nie zgodzi, tylko aby byla precz wyrzucona i od ludzi podeptana.
\par 14 Wy jestescie swiatlosc swiata, nie moze sie miasto ukryc na górze lezace.
\par 15 Ani zapalaja swiecy, i stawiaja jej pod korzec, ale na swiecznik, i swieci wszystkim, którzy sa w domu.
\par 16 Tak niechaj swieci swiatlosc wasza przed ludzmi, aby uczynki wasze dobre widzieli, a chwalili ojca waszego, który jest w niebiesiech.
\par 17 Nie mniemajcie, abym przyszedl rozwiazywac zakon albo proroki; nie przyszedlem rozwiazywac, ale wypelnic.
\par 18 Zaprawde bowiem powiadam wam: Az przeminie niebo i ziemia, jedna jota albo jedna kreska nie przeminie z zakonu, azby sie wszystko stalo.
\par 19 Kto by tedy rozwiazal jedno z tych przykazan najmniejszych, i uczylby tak ludzi, najmniejszym bedzie nazwany w królestwie niebieskiem; a ktokolwiek by czynil i uczyl, ten bedzie wielkim nazwany w królestwie niebieskiem.
\par 20 Albowiem powiadam wam: Jezli nie bedzie obfitsza sprawiedliwosc wasza, niz nauczonych w Pismie i Faryzeuszów, zadnym sposobem nie wnijdziecie do królestwa niebieskiego.
\par 21 Slyszeliscie, iz rzeczono starym: Nie bedziesz zabijal; a ktobykolwiek zabil, bedzie winien sadu;
\par 22 Ale ja wam powiadam: Iz kazdy, kto sie gniewa na brata swego bez przyczyny, bedzie winien sadu; a ktokolwiek rzecze bratu swemu: Racha; bedzie winien rady, a ktokolwiek rzecze: Blaznie! bedzie winien ognia piekielnego.
\par 23 A tak jezlibys ofiarowal dar twój na oltarzu, a tam bys wspomnial, iz brat twój ma co przeciwko tobie,
\par 24 Zostaw tam dar twój przed oltarzem, a odejdz, pierwej sie pojednaj z bratem twoim; a potem przyszedlszy ofiaruj dar twój.
\par 25 Zgódz sie z przeciwnikiem twoim rychlo, pókis jest z nim w drodze, by cie snac przeciwnik nie podal sedziemu, a sedzia by cie podal sludze, i bylbys wrzucony do wiezienia.
\par 26 Zaprawde ci powiadam: Nie wynijdziesz stamtad, póki bys nie oddal do ostatniego pieniazka.
\par 27 Slyszeliscie, iz rzeczono starym: Nie bedziesz cudzolozyl;
\par 28 Alec Ja wam powiadam: Iz kazdy, który patrzy na niewiaste, aby jej pozadal, juz z nia cudzolóstwo popelnil w sercu swojem.
\par 29 Jezli cie tedy oko twoje prawe gorszy, wylup je, a zarzuc od siebie; albowiem pozyteczniej jest tobie, aby zginal jeden z czlonków twoich, a wszystko cialo twoje nie bylo wrzucone do ognia piekielnego.
\par 30 A jezliz cie prawa reka twoja gorszy, odetnij ja, i zarzuc od siebie; albowiem pozyteczniej jest tobie, aby zginal jeden z czlonków twoich, a wszystko cialo twoje nie bylo wrzucone do ognia piekielnego.
\par 31 Zasie rzeczono: Ktobykolwiek opuscil zone swoja, niech jej da list rozwodny;
\par 32 Ale Ja wam powiadam: Ktobykolwiek opuscil zone swoje oprócz przyczyny cudzolóstwa, przywodzi ja w cudzolóstwo, a kto by opuszczona pojal, cudzolozy.
\par 33 Slyszeliscie zasie, iz rzeczono starym: Nie bedziesz krzywo przysiegal, ale oddasz Panu przysiegi twoje;
\par 34 Ale Ja wam powiadam, abyscie zgola nie przysiegali, ani na niebo, gdyz jest stolica Boza;
\par 35 Ani na ziemie, gdyz jest podnózkiem nóg jego; ani na Jeruzalem, gdyz jest miasto wielkiego króla;
\par 36 Ani na glowe twoje bedziesz przysiegal, gdyz nie mozesz jednego wlosa bialym albo czarnym uczynic.
\par 37 Ale mowa wasza niech bedzie: Tak, tak; nie, nie; a co wiecej nadto jest, to od zlego jest.
\par 38 Slyszeliscie, iz rzeczono: Oko za oko, a zab za zab;
\par 39 Ale Ja wam powiadam: Zebyscie sie nie sprzeciwiali zlemu, ale kto by cie uderzyl w prawy policzek twój, nadstaw mu i drugi;
\par 40 I temu, który sie z toba chce prawowac, a suknie twoje wziac, pusc mu i plaszcz;
\par 41 A kto by cie przymuszal isc mile jedne, idz z nim i dwie;
\par 42 Temu, co cie prosi, daj, a od tego, co chce u ciebie pozyczyc, nie odwracaj sie.
\par 43 Slyszeliscie, iz rzeczono: Bedziesz milowal blizniego twego, a bedziesz mial w nienawisci nieprzyjaciela twego;
\par 44 Alec Ja wam powiadam: Milujcie nieprzyjacioly wasze; blogoslawcie tym, którzy was przeklinaja; dobrze czyncie tym, którzy was maja w nienawisci, i módlcie sie za tymi, którzy wam zlosc wyrzadzaja i przesladuja was;
\par 45 Abyscie byli synami Ojca waszego, który jest w niebiesiech; bo on to czyni, ze slonce jego wschodzi na zle i na dobre, i deszcz spuszcza na sprawiedliwe i na niesprawiedliwe,
\par 46 Albowiem jezli milujecie te, którzy was miluja, jakaz zaplate macie? azaz i celnicy tego nie czynia?
\par 47 A jezlibyscie tylko braci waszych pozdrawiali, cóz osobliwego czynicie? azaz i celnicy tak nie czynia?
\par 48 Badzciez wy tedy doskonalymi, jako i Ojciec wasz, który jest w niebiesiech, doskonaly jest.

\chapter{6}

\par 1 Strzezcie sie, abyscie jalmuzny waszej nie czynili przed ludzmi dlatego, abyscie byli widziani od nich; inaczej nie bedziecie mieli zaplaty u Ojca waszego, który jest w niebiesiech.
\par 2 Przetoz, gdy czynisz jalmuzne, nie trab przed soba, jako obludnicy czynia w bóznicach i na ulicach, aby byli chwaleni od ludzi; zaprawde powiadam wam, odbieraja zaplate swoje.
\par 3 Ale ty gdy czynisz jalmuzne, niechaj nie wie lewica twoja, co czyni prawica twoja.
\par 4 Aby jalmuzna twoja byla w skrytosci, a Ojciec twój, który widzi w skrytosci, ten ci jawnie odda.
\par 5 A gdy sie modlisz, nie badz jako obludnicy; albowiem sie oni radzi w bóznicach i na rogach ulic stojac, modla, aby byli widziani od ludzi; zaprawde powiadam wam, iz odbieraja zaplate swoje.
\par 6 Ale ty, gdy sie modlisz, wnijdz do komory swojej, a zawarlszy drzwi swoje, módl sie Ojcu twemu, który jest w skrytosci; a Ojciec twój, który widzi w skrytosci, odda ci jawnie.
\par 7 A modlac sie, nie badzcie wielomówni, jako poganie; albowiem oni mniemaja, ze dla swojej wielomównosci wysluchani beda.
\par 8 Nie badzciez tedy im podobni, gdyz wie Ojciec wasz, czego potrzebujecie, pierwej nizbyscie wy go prosili.
\par 9 Wy tedy tak sie módlcie; Ojcze nasz, którys jest w niebiesiech! Swiec sie imie twoje;
\par 10 Przyjdz królestwo twoje; badz wola twoja jako w niebie, tak i na ziemi.
\par 11 Chleba naszego powszedniego daj nam dzisiaj.
\par 12 I odpusc nam nasze winy, jako i my odpuszczamy naszym winowajcom;
\par 13 I nie wwódz nas na pokuszenie, ale nas zbaw ode zlego; albowiem twoje jest królestwo i moc i chwala na wieki. Amen.
\par 14 Bo jezli odpuscicie ludziom upadki ich, odpusci i wam Ojciec wasz niebieski;
\par 15 A jezli nie odpuscicie ludziom upadków ich, i Ojciec wasz nie odpusci wam upadków waszych.
\par 16 A gdy poscicie, nie badzciez smetnej twarzy, jako obludnicy; szpeca bowiem twarzy swoje, aby byli widziani od ludzi, ze poszcza; zaprawde powiadam wam, odbieraja zaplate swoje.
\par 17 Ale ty, gdy poscisz, namaz glowe twoje, i umyj twarz twoje,
\par 18 Abys nie byl widziany od ludzi, ze poscisz, ale od Ojca twojego, który jest w skrytosci; a Ojciec twój, który widzi w skrytosci, odda ci jawnie.
\par 19 Nie skarbcie sobie skarbów na ziemi, gdzie mól i rdza psuje, i gdzie zlodzieje podkopywuja i kradna;
\par 20 Ale sobie skarbcie skarby w niebie, gdzie ani mól ani rdza psuje, i gdzie zlodzieje nie podkopywuja, ani kradna.
\par 21 Albowiem gdzie jest skarb wasz, tam jest i serce wasze.
\par 22 Oko twoje jestci swieca ciala twego; jezliby tedy oko twoje bylo szczere, wszystko cialo twoje jasne bedzie;
\par 23 Jezliby zas oko twoje zle bylo, wszystko cialo twoje ciemne bedzie; jezli tedy swiatlosc, która jest w tobie, ciemnoscia jest, sama ciemnosc jakaz bedzie?
\par 24 Nikt nie moze dwom panom sluzyc, gdyz albo jednego bedzie mial w nienawisci, a drugiego bedzie milowal; albo jednego trzymac sie bedzie, a drugim pogardzi; nie mozecie Bogu sluzyc i mamonie.
\par 25 Dlatego powiadam wam: Nie troszczcie sie o zywot wasz, co byscie jedli, albo co byscie pili, ani o cialo wasze, czem byscie sie odziewali; azaz zywot nie jest zacniejszy niz pokarm, i cialo niz odzienie?
\par 26 Spojrzyjcie na ptaki niebieskie, iz nie sieja ani zna, ani zbieraja do gumien, a wzdy Ojciec wasz niebieski zywi je; izali wy nie jestescie daleko zacniejsi nad nie?
\par 27 I któz z was troskliwie myslac, moze przydac do wzrostu swego lokiec jeden?
\par 28 A o odzienie przeczze sie troszczycie? Przypatrzcie sie liliom polnym, jako rosna; nie pracuja, ani przeda.
\par 29 A Ja wam powiadam, iz ani Salomon we wszystkiej slawie swojej nie byl tak przyodziany, jako jedna z tych.
\par 30 Jezli tedy trawe polna, która dzis jest, a jutro bywa w piec wrzucona, Bóg tak przyodziewa, azaz nie daleko wiecej was! o malowierni!
\par 31 Nie troszczcie sie tedy, mówiac: Cóz bedziemy jesc? albo co bedziemy pic? albo czem sie bedziemy przyodziewac?
\par 32 Boc tego wszystkiego poganie szukaja; wiec bowiem Ojciec wasz niebieski, ze tego wszystkiego potrzebujecie.
\par 33 Ale szukajcie naprzód królestwa Bozego, i sprawiedliwosci jego, a to wszystko bedzie wam przydano.
\par 34 Przetoz nie troszczcie sie o jutrzejszy dzien: albowiem jutrzejszy dzien troskac sie bedzie o swoje potrzeby. Dosycci ma dzien na swojem utrapieniu.

\chapter{7}

\par 1 Nie sadzcie, abyscie nie byli sadzeni;
\par 2 Albowiem jakim sadem sadzicie, takim sadzeni bedziecie, i jaka miara mierzycie, taka wam odmierzono bedzie.
\par 3 A czemuz widzisz zdzblo w oku brata twego, a balki, która jest w oku twojem, nie baczysz?
\par 4 Albo jakoz rzeczesz bratu twemu: Dopusc, iz wyjme zdzblo z oka twego, a oto balka jest w oku twojem.
\par 5 Obludniku! wyjmij pierwej balke z oka twego, tedy przejrzysz, abys wyjal zdzblo z oka brata twego.
\par 6 Nie dawajcie swietego psom, ani mieccie perel waszych przed swinie, by ich snac nie podeptaly nogami swemi, i obróciwszy sie nie rozszarpaly was.
\par 7 Proscie, a bedzie wam dano, szukajcie, a znajdziecie; kolaczcie, a bedzie wam otworzono.
\par 8 Kazdy bowiem, kto prosi, bierze; a kto szuka, znajduje; a temu, co kolacze, bedzie otworzono.
\par 9 I któryz z was jest czlowiek, którego prosilliby syn jego o chleb, izali mu da kamien?
\par 10 A prosilby o rybe, izali mu da weza?
\par 11 Jezli wy tedy bedac zlymi, umiecie dary dobre dawac dzieciom waszym, czemze wiecej Ojciec wasz, który jest w niebiesiech, da rzeczy dobre tym, którzy go prosza.
\par 12 Wszystko tedy, co byscie chcieli, aby wam ludzie czynili, tak i wy czyncie im; tenci bowiem jest zakon i prorocy.
\par 13 Wchodzcie przez ciasna brame; albowiem przestronna jest brama i szeroka droga, która prowadzi na zatracenie, a wiele ich jest, którzy przez nie wchodza.
\par 14 A ciasna jest brama i waska droga, która prowadzi do zywota; a malo ich jest, którzy ja znajduja.
\par 15 A strzezcie sie falszywych proroków, którzy przychodza do was w odzieniu owczem, ale wewnatrz sa wilcy drapiezni.
\par 16 Z owoców ich poznacie je; izali zbieraja z ciernia grona winne, albo z ostu figi?
\par 17 Tak ci wszelkie drzewo dobre owoce dobre przynosi; ale zle drzewo owoce zle przynosi.
\par 18 Nie moze dobre drzewo owoców zlych przynosic, ani drzewo zle owoców dobrych przynosic.
\par 19 Wszelkie drzewo, które nie przynosi owocu dobrego, bywa wyciete i w ogien wrzucone.
\par 20 A tak z owoców ich poznacie je.
\par 21 Nie kazdy, który mi mówi: Panie, Panie! wnijdzie do królestwa niebieskiego; ale który czyni wole Ojca mojego, który jest w niebiesiech.
\par 22 Wiele ich rzecze mi dnia onego: Panie, Panie! izazesmy w imieniu twojem nie prorokowali, i w imieniu twojem dyjablów nie wyganiali, i w imieniu twojem wiele cudów nie czynili?
\par 23 A tedy im wyznam: Zem was nigdy nie znal; odstapcie ode mnie, którzy czynicie nieprawosc.
\par 24 Wszelkiego tedy, który slucha tych slów moich i czyni je, przypodobam mezowi madremu, który zbudowal dom swój na opoce;
\par 25 I spadl gwaltowny deszcz, i przyszla powódz, i wiatry wialy, i uderzyly na on dom, ale nie upadl, bo byl zalozony na opoce.
\par 26 A wszelki, który slucha tych slów moich, a nie czyni ich, przypodobany bedzie mezowi glupiemu, który zbudowal dom swój na piasku;
\par 27 I spadl deszcz gwaltowny, i przyszla powódz, i wiatry wialy, a uderzyly na on dom, i upadl, a byl wielki upadek jego.
\par 28 I stalo sie, gdy dokonczyl Jezus tych slów, ze sie zdumiewal lud nad nauka jego.
\par 29 Albowiem je uczyl jako moc majacy, a nie jako nauczeni w Pismie.

\chapter{8}

\par 1 A gdy zstepowal z góry, szedl za nim wielki lud;
\par 2 A oto tredowaty przyszedlszy, poklonil mu sie, mówiac: Panie! jezli chcesz, mozesz mnie oczyscic.
\par 3 I wyciagnawszy Jezus reke, dotknal sie go, mówiac: Chce, badz oczyszczony; i zaraz oczyszczony jest trad jego.
\par 4 Tedy mu rzekl Jezus: Patrz, abys nikomu nie powiadal, ale idz, ukaz sie kaplanowi, i ofiaruj dar on, który przykazal Mojzesz na swiadectwo przeciwko nim.
\par 5 A gdy Jezus wszedl do Kapernaum, przyszedl do niego setnik, proszac go,
\par 6 I mówiac: Panie! sluga mój lezy w domu powietrzem ruszony, i ciezko sie trapi.
\par 7 I rzekl mu Jezus: Ja przyjde i uzdrowie go.
\par 8 A odpowiadajac setnik rzekl: Panie! nie jestem godzien, abys wszedl pod dach mój; ale tylko rzecz slowo, a bedzie uzdrowiony sluga mój.
\par 9 Bomci i ja czlowiek pod moca innego, majacy pod soba zolnierze; i mówie temu: Idz, a idzie; a drugiemu: Przyjdz, a przychodzi; a sludze memu: Czyn to, a czyni.
\par 10 A gdy to uslyszal Jezus, zadziwil sie, i rzekl tym, którzy szli za nim: Zaprawde powiadam wam: Anim w Izraelu tak wielkiej wiary nie znalazl.
\par 11 A powiadam wam: Iz wiele ich od wschodu i od zachodu slonca przyjdzie, a usiada za stolem z Abrahamem i z Izaakiem i z Jakóbem w królestwie niebieskiem.
\par 12 Ale synowie królestwa beda wyrzuceni w ciemnosci zewnetrzne, tam bedzie placz i zgrzytanie zebów.
\par 13 I rzekl Jezus setnikowi: Idz, a jakos uwierzyl, niech ci sie stanie; i uzdrowiony jest sluga jego onejze godziny.
\par 14 A gdy Jezus przyszedl do domu Piotrowego, ujrzal swiekre jego, lezaca na lozu i majaca goraczke.
\par 15 I dotknal sie reki jej, i opuscila ja goraczka; i wstala, a poslugowala im.
\par 16 A gdy byl wieczór, przywiedli do niego wiele opetanych: i wyganial duchy slowem; i wszystkie, którzy sie zle mieli, uzdrawial;
\par 17 Aby sie wypelnilo, co powiedziano przez Izajasza proroka, mówiacego: On niemocy nasze na sie wzial, a choroby nasze nosil.
\par 18 A widzac Jezus wielki lud okolo siebie, kazal sie przeprawic na druga strone morza.
\par 19 Tedy przystapiwszy niektóry z nauczonych w Pismie, rzekl mu: Mistrzu! pójde za toba, gdziekolwiek pójdziesz.
\par 20 I rzekl mu Jezus: Liszki maja jamy, a ptaki niebieskie gniazda; ale Syn czlowieczy nie ma, gdzie by glowe sklonil.
\par 21 A drugi z uczniów jego rzekl mu: Panie! dopusc mi pierwej odejsc i pogrzesc ojca mego;
\par 22 Ale mu Jezus rzekl: Pójdz za mna, a niechaj umarli grzebia umarle swoje.
\par 23 A gdy on wstapil w lódz, wstapili za nim i uczniowie jego.
\par 24 A oto sie wzruszenie wielkie stalo na morzu, tak iz sie lódz walami okrywala; a on spal.
\par 25 A przystapiwszy uczniowie jego, obudzili go, mówiac: Panie! ratuj nas, giniemy.
\par 26 I rzekl do nich: Przeczze jestescie bojazliwi? o malowierni! Tedy wstawszy, zgromil wiatry i morze, i stalo sie uciszenie wielkie.
\par 27 A ludzie sie dziwowali, mówiac: Jakiz to jest ten, ze mu i wiatry i morze posluszne sa?
\par 28 A gdy sie on przewiózl na druga strone do krainy Giergiezenczyków, zabiezeli mu dwaj opetani z grobów wychodzacy, bardzo okrutni, tak iz nie mógl nikt przechodzic ona droga.
\par 29 A oto zakrzykneli, mówiac: Cóz my z toba mamy, Jezusie, Synu Bozy? Przyszedles tu przed czasem, dreczyc nas?
\par 30 I byla daleko od nich trzoda wielka swin pasacych sie.
\par 31 Tedy go dyjabli prosili, mówiac: Jezli nas wyganiasz, dopusc nam wnijsc w trzode tych swin.
\par 32 I rzekl im: Idzcie. A oni wyszedlszy, weszli w one trzode swin, a oto porwawszy sie ona wszystka trzoda swin, z przykra wpadla w morze, i pozdychala w wodach.
\par 33 Lecz pasterze uciekli, a poszedlszy do miasta, opowiedzieli wszystko, i to, co sie z onymi opetanymi stalo.
\par 34 A oto wszystko miasto wyszlo przeciwko Jezusowi, a ujrzawszy go prosili, aby z ich granic odszedl.

\chapter{9}

\par 1 Tedy wstapiwszy w lódz, przewiózl sie, i przyszedl do miasta swego;
\par 2 A oto przyniesli mu powietrzem ruszonego, na lozu lezacego. A widzac Jezus wiare ich, rzekl powietrzem ruszonemu: Ufaj, synu! odpuszczone sa tobie grzechy twoje.
\par 3 A oto niektórzy z nauczonych w Pismie mówili sami w sobie: Ten bluzni.
\par 4 A widzac Jezus mysli ich, rzekl: Przeczze wy myslicie zle rzeczy w sercach waszych?
\par 5 Albowiem cóz latwiej rzec: Odpuszczone sa tobie grzechy, czyli rzec: Wstan, a chodz?
\par 6 Ale abyscie wiedzieli, iz ma moc Syn czlowieczy na ziemi odpuszczac grzechy, tedy rzekl powietrzem ruszonemu: Wstawszy, wezmij loze twoje, a idz do domu twego.
\par 7 Tedy wstawszy, poszedl do domu swego.
\par 8 Co ujrzawszy lud, dziwowal sie, i chwalil Boga, który dal taka moc ludziom.
\par 9 A odchodzac stamtad Jezus, ujrzal czlowieka siedzacego na cle, którego zwano Mateusz, i rzekl mu: Pójdz za mna; tedy wstawszy, szedl za nim.
\par 10 I stalo sie, gdy Jezus siedzial za stolem w domu jego, ze oto wiele celników i grzeszników przyszedlszy, usiedli z Jezusem i z uczniami jego.
\par 11 Co widzac Faryzeuszowie, rzekli uczniom jego: Przeczze z celnikami i grzesznikami je nauczyciel wasz?
\par 12 A Jezus uslyszawszy to, rzekl im: Nie potrzebujac zdrowi lekarza, ale ci, co sie zle maja.
\par 13 Owszem idzcie, a nauczcie sie, co to jest: Milosierdzia chce, a nie ofiary; bom nie przyszedl wzywac sprawiedliwych, ale grzesznych do pokuty.
\par 14 Tedy przyszli do niego uczniowie Janowi, mówiac: Przecz my i Faryzeuszowie czesto poscimy, a uczniowie twoi nie poszcza?
\par 15 I rzekl im Jezus: Izali sie moga synowie loznicy malzenskiej smecic, póki z nimi jest oblubieniec? Ale przyjda dni, gdy od nich bedzie oblubieniec odjety, a tedy poscic beda.
\par 16 A zaden nie wprawuje laty sukna nowego w szate wiotcha; albowiem ono zalatanie ujmuje nieco od szaty, i stawa sie gorsze rozdarcie;
\par 17 Ani leja wina mlodego w stare statki; bo inaczej pukaja sie statki, a wino wycieka, i statki sie psuja; ale mlode wino leja w nowe statki, i oboje bywaja zachowane.
\par 18 To gdy on do nich mówil, oto niektóry przelozony bóznicy przyszedlszy poklonil mu sie, mówiac: Córka moja dopiero skonala; ale pójdz, a wlóz na nia reke twoje, a ozyje.
\par 19 Tedy wstawszy Jezus, szedl za nim, i uczniowie jego.
\par 20 (A oto niewiasta, która plynienie krwi ode dwunastu lat cierpiala, przystapiwszy z tylu, dotknela sie podolka szat jego;
\par 21 Bo rzekla sama w sobie: Jezli sie tylko dotkne szaty jego, bede uzdrowiona.
\par 22 Ale Jezus obróciwszy sie i ujrzawszy ja, rzekl: Ufaj, córko! wiara twoja ciebie uzdrowila; i uzdrowiona byla niewiasta od onej godziny.)
\par 23 A gdy przyszedl Jezus w dom przelozonego, i ujrzal piszczki i lud zgielk czyniacy,
\par 24 Rzekl im: Ustapcie; albowiem dzieweczka nie umarla, ale spi. I nasmiewali sie z niego.
\par 25 Ale gdy wygnany byl on lud, wszedlszy, ujal ja za reke jej, i wstala dzieweczka.
\par 26 I rozeszla sie ta wiesc po wszystkiej ziemi.
\par 27 A gdy Jezus odchodzil stamtad, szli za nim dwaj slepi, wolajac i mówiac: Synu Dawidowy! zmiluj sie nad nami.
\par 28 A gdy on wszedl do domu, przyszli do niego slepi; i rzekl im Jezus: Wierzyciez, iz to moge uczynic? Rzekli mu: Owszem Panie!
\par 29 Tedy sie dotknal oczu ich, mówiac: Wedlug wiary waszej niechaj sie wam stanie.
\par 30 I otworzyly sie oczy ich; i przygrozil im srodze Jezus, mówiac: Patrzciez, aby nikt o tem nie wiedzial.
\par 31 Lecz oni wyszedlszy, rozslawili go po wszystkiej onej ziemi.
\par 32 A gdy oni wychodzili, oto przywiedli mu czlowieka niemego, opetanego od dyjabla.
\par 33 A gdy byl wygnany on dyjabel, przemówil niemy; i dziwowal sie lud, mówiac: Nigdy sie taka rzecz nie pokazala w Izraelu.
\par 34 Ale Faryzeuszowie mówili: Przez ksiazecia dyjabelskiego wygania dyjably.
\par 35 I obchodzil Jezus wszystkie miasta i miasteczka, nauczajac w bóznicach ich, i kazac Ewangielije królestwa, a uzdrawiajac wszelka chorobe, i wszelka niemoc miedzy ludem.
\par 36 A widzac on lud, uzalil sie go, iz byl strudzony i rozproszony jako owce nie majace pasterza.
\par 37 Tedy rzekl uczniom swoim: Zniwoc wprawdzie wielkie, ale robotników malo.
\par 38 Proscie tedy Pana zniwa, aby wypchnal robotniki na zniwo swoje.

\chapter{10}

\par 1 A zwolawszy dwunastu uczniów swoich, dal im moc nad duchy nieczystymi, aby je wyganiali, i uzdrawiali wszelka chorobe i wszelka niemoc.
\par 2 A dwunastu Apostolów te sa imiona: Pierwszy Szymon, którego zowia Piotr, i Andrzej, brat jego; Jakób, syn Zebedeusza, i Jan, brat jego;
\par 3 Filip i Bartlomiej, Tomasz i Mateusz on celnik, Jakób, syn Alfeusza, i Lebeusz, nazwany Tadeusz;
\par 4 Szymon Kananejczyk, i Judasz Iszkaryjot, który go tez wydal.
\par 5 Tych dwunastu poslal Jezus, rozkazujac im i mówiac: Na droge poganów nie zachodzcie, i do miasta Samarytanczyków nie wchodzcie;
\par 6 Ale raczej idzcie do owiec, które zginely z domu Izraelskiego;
\par 7 A idac kazcie, mówiac: Przyblizylo sie królestwo niebieskie.
\par 8 Chore uzdrawiajcie, tredowate oczyszczajcie, umarle wskrzeszajcie, dyjably wyganiajcie; darmoscie wzieli, darmo dawajcie.
\par 9 Nie bierzcie z soba zlota, ani srebra, ani miedzi w trzosy wasze;
\par 10 Ani taistry na droge, ani dwóch sukien, ani butów, ani laski; albowiem godzien jest robotnik zywnosci swojej.
\par 11 A do któregokolwiek miasta albo miasteczka wnijdziecie, wywiadujcie sie, kto by w niem tego byl godzien, a tamze mieszkajcie, póki nie wynijdziecie;
\par 12 A wszedlszy w dom, pozdrówcie go.
\par 13 A jezliby on dom tego byl godny, niech na niego przyjdzie pokój wasz; a jezliby nie byl godny, pokój wasz niech sie wróci do was.
\par 14 A kto by was nie przyjal, ani sluchal slów waszych, wychodzac z domu albo z miasta onego, otrzasnijcie proch z nóg waszych.
\par 15 Zaprawde wam powiadam: Lzej bedzie ziemi Sodomskiej i Gomorskiej w dzien sadny, nizeli miastu onemu.
\par 16 Oto Ja was posylam jako owce miedzy wilki; badzciez tedy roztropnymi jako weze, a szczerymi jako golebice,
\par 17 A strzezcie sie ludzi; albowiem was beda wydawac do rady, i w zgromadzeniach swoich was biczowac beda.
\par 18 Takze przed starosty i przed króle wodzeni bedziecie dla mnie, na swiadectwo przeciwko nim i poganom.
\par 19 Ale gdy was podadza, nie troszczcie sie, jako i co byscie mówili; albowiem wam dano bedzie onejze godziny, co byscie mówili;
\par 20 Bo wy nie jestescie, którzy mówicie, ale duch Ojca waszego, który mówi w was.
\par 21 I wyda brat brata na smierc, i ojciec syna, i powstana dzieci przeciwko rodzicom, i beda je zabijac.
\par 22 I bedziecie w nienawisci u wszystkich dla imienia mego; ale kto wytrwa do konca, ten bedzie zbawion.
\par 23 A gdy was przesladowac beda w tem miescie, uciekajcie do drugiego; bo zaprawde powiadam wam, ze nie obejdziecie miast Izraelskich, az przyjdzie Syn czlowieczy.
\par 24 Nie jestci uczen nad mistrza, ani sluga nad Pana swego;
\par 25 Dosyc uczniowi, aby byl jako mistrz jego, a sluga jako Pan jego; jezlic gospodarza Beelzebubem nazywali, czem wiecej domowniki jego nazywac beda.
\par 26 Przetoz nie bójcie sie ich; albowiem nic nie jest skrytego, co by nie mialo byc objawiono, i nic tajemnego, czego by sie dowiedziec nie miano.
\par 27 Co wam w ciemnosci mówie, powiadajcie na swietle; a co w ucho slyszycie, obwolywajcie na dachach;
\par 28 A nie bójcie sie tych, którzy zabijaja cialo, lecz duszy zabic nie moga; ale raczej bójcie sie tego, który moze i dusze i cialo zatracic w piekielnym ogniu.
\par 29 Izali dwóch wróbelków za pieniazek nie sprzedaja, a wzdy jeden z nich nie upadnie na ziemie oprócz woli Ojca waszego.
\par 30 Nawet i wlosy wszystkie na glowie waszej policzone sa.
\par 31 Nie bójcie sie tedy; nad wiele wróbelków wy zacniejszymi jestescie.
\par 32 Wszelki tedy, który by mie wyznal przed ludzmi, wyznam go Ja tez przed Ojcem moim, który jest w niebiesiech;
\par 33 A kto by sie mnie zaparl przed ludzmi, zapre sie go i Ja przed Ojcem moim, który jest w niebiesiech.
\par 34 Nie mniemajcie, zem przyszedl dawac pokój na ziemie; nie przyszedlem dawac pokoju, ale miecz.
\par 35 Bom przyszedl, abym rozerwanie uczynil miedzy synem a ojcem jego, i miedzy córka a matka jej, takze miedzy synowa i swiekra jej;
\par 36 I nieprzyjaciolmi beda czlowiekowi domownicy jego.
\par 37 Kto miluje ojca albo matke nad mie, nie jest mie godzien; a kto miluje syna albo córke nad mie, nie jest mie godzien;
\par 38 A kto nie bierze krzyza swego, i nie idzie za mna, nie jest mie godzien.
\par 39 Kto by znalazl dusze swoje, straci ja; a kto by stracil dusze swoje dla mnie, znajdzie ja.
\par 40 Kto was przyjmuje, mnie przyjmuje; a kto mnie przyjmuje, przyjmuje tego, który mie poslal.
\par 41 Kto przyjmuje proroka w imieniu proroka, zaplate proroka wezmie; a kto przyjmuje sprawiedliwego w imieniu sprawiedliwego, sprawiedliwego zaplate wezmie.
\par 42 Kto by tez napoil jednego z tych to malych tylko kubkiem zimnej wody w imie ucznia, zaprawde powiadam wam, nie straci zaplaty swojej.

\chapter{11}

\par 1 I stalo sie, gdy Jezus przestal rozkazywac dwunastu uczniom swoim, poszedl z onad, aby uczyl i kazal w miastach ich.
\par 2 A Jan uslyszawszy w wiezieniu o uczynkach Chrystusowych, poslawszy dwóch z uczniów swoich,
\par 3 Rzekl mu: Tyzes jest on, który ma przyjsc, czyli inszego czekac mamy?
\par 4 A odpowiadajac Jezus, rzekl im: Szedlszy, oznajmijcie Janowi, co slyszycie i widzicie.
\par 5 Slepi widza, a chromi chodza, tredowaci biora oczyszczenie, a glusi slysza, umarli zmartwychwstaja, i ubogim Ewangielija opowiadana bywa;
\par 6 A blogoslawiony jest, który sie nie zgorszy ze mnie.
\par 7 A gdy oni odeszli, poczal Jezus mówic do ludu o Janie: Coscie wyszli na puszcze widziec? Izali trzcine chwiejaca sie od wiatru?
\par 8 Ale coscie wyszli widziec? Izali czlowieka w miekkie szaty obleczonego? oto którzy miekkie szaty nosza, w domach królewskich sa.
\par 9 Ale coscie wyszli widziec? Izali proroka? zaiste powiadam wam, i wiecej niz proroka.
\par 10 Boc ten jest, o którym napisano: Oto ja posylam Aniola mego przed obliczem twojem, który zgotuje droge twoje przed toba.
\par 11 Zaprawde powiadam wam: Nie powstal z tych, którzy sie z niewiast rodza, wiekszy nad Jana Chrzciciela; ale który jest najmniejszym w królestwie niebieskiem, wiekszy jest, nizeli on.
\par 12 A ode dni Jana Chrzciciela az dotad królestwo niebieskie gwalt cierpi, a gwaltownicy porywaja je.
\par 13 Bo wszyscy prorocy i zakon az do Jana prorokowali.
\par 14 A jezli chcecie przyjac, onci jest Elijasz, który mial przyjsc.
\par 15 Kto ma uszy ku sluchaniu, niechaj slucha.
\par 16 Ale komuz przypodobam ten naród? podobny jest dziatkom, które siedza na rynkach, i wolaja na towarzysze swoje,
\par 17 I mówia: Gralysmy wam na piszczalce, a nie tancowalyscie; spiewalysmy piesni zalobne, a nie plakalyscie.
\par 18 Albowiem przyszedl Jan ani jedzac ani pijac, a mówia: Iz dyjabelstwo ma.
\par 19 Przyszedl Syn czlowieczy jedzac i pijac, a mówia: Oto czlowiek obzerca i pijanica wina, przyjaciel celników i grzeszników; i usprawiedliwiona jest madrosc od synów swoich.
\par 20 Tedy poczal przymawiac miastom, w których sie najwiecej dzialo cudów jego, ze nie pokutowaly, mówiac:
\par 21 Biada tobie Chorazynie! biada tobie Betsaido! bo gdyby sie byly w Tyrze i w Sydonie te cuda staly, które sie w was staly, dawno by byly w worze i w popiele pokutowaly.
\par 22 Wszakze powiadam wam: Lzej bedzie Tyrowi i Sydonowi w dzien sadny, nizeli wam.
\par 23 A ty Kapernaum! któres az do nieba wywyzszone, az do piekla stracone bedziesz; bo gdyby sie byly w Sodomie te cuda dzialy, które sie dzialy w tobie, zostalaby byla az do dnia dzisiejszego.
\par 24 Nawet powiadam wam: Iz lzej bedzie ziemi Sodomskiej w dzien sadny, nizeli tobie.
\par 25 W on czas odpowiadajac Jezus, rzekl: Wyslawiam cie, Ojcze, Panie nieba i ziemi! zes te rzeczy zakryl przed madrymi i roztropnymi, a objawiles je niemowlatkom.
\par 26 Zaprawde, Ojcze! tak sie upodobalo tobie.
\par 27 Wszystkie rzeczy dane mi sa od Ojca mego, i nikt nie zna Syna, tylko Ojciec, ani Ojca kto zna, tylko Syn, a komu by chcial Syn objawic.
\par 28 Pójdzcie do mnie wszyscy, którzyscie spracowani i obciazeni, a Ja wam sprawie odpocznienie;
\par 29 Wezmijcie jarzmo moje na sie, a uczcie sie ode mnie, zem Ja cichy i pokornego serca; a znajdziecie odpocznienie duszom waszym;
\par 30 Albowiem jarzmo moje wdzieczne jest, a brzemie moje lekkie jest.

\chapter{12}

\par 1 W on czas szedl Jezus w sabat przez zboza, a uczniowie jego lakneli, i poczeli rwac klosy i jesc.
\par 2 A ujrzawszy to Faryzeuszowie, rzekli mu: Oto uczniowie twoi czynia, czego sie nie godzi czynic w sabat.
\par 3 A on im rzekl: Izascie nie czytali, co uczynil Dawid, gdy laknal, on i ci, którzy z nim byli?
\par 4 Jako wszedl do domu Bozego, i chleby pokladne jadl, których mu sie nie godzilo jesc, ani tym, którzy z nim byli, tylko samym kaplanom.
\par 5 Alboscie nie czytali w zakonie, ze w sabat i kaplani w kosciele sabat gwalca, a bez winy sa?
\par 6 Ale mówie wam, iz tu wiekszy jest niz kosciól.
\par 7 A gdybyscie wiedzieli, co to jest: Milosierdzia chce, a nie ofiary, nie potepialibyscie niewinnych;
\par 8 Albowiem Syn czlowieczy Panem jest i sabatu.
\par 9 A odszedlszy stamtad przyszedl do bóznicy ich;
\par 10 A oto byl tam czlowiek majacy reke uschla; i pytali go, mówiac: Godzili sie w sabat uzdrawiac? aby go oskarzyli.
\par 11 A on im rzekl: Któryz czlowiek z was bedzie, który by mial owce jedna, a gdyby mu ta w sabat w dól wpadla, izali jej nie dobedzie i nie wyciagnie?
\par 12 A czemze zacniejszy jest czlowiek nizeli owca? Przetoz godzi sie w sabat dobrze czynic.
\par 13 Tedy rzekl czlowiekowi onemu: Wyciagnij reke twoje; a on wyciagnal, i przywrócona jest do zdrowia jako i druga.
\par 14 A wyszedlszy Faryzeuszowie, uczynili rade przeciwko niemu, jakoby go stracili.
\par 15 Ale Jezus poznawszy to, odszedl stamtad, i szedl za nim lud wielki; i uzdrowil one wszystkie,
\par 16 I przygrozil im, aby go nie objawiali,
\par 17 Zeby sie wypelnilo, co powiedziano przez Izajasza proroka, mówiacego:
\par 18 Oto ten sluga mój, któregom obral, ten umilowany mój, w którym sie upodobalo duszy mojej; poloze ducha mojego na nim, a sad narodom opowie;
\par 19 Nie bedzie sie wadzil, ani bedzie wolal, i nikt na ulicach nie uslyszy glosu jego;
\par 20 Trzciny nalamanej nie dolamie, a lnu kurzacego sie nie zagasi, az wystawi sad ku zwyciestwu;
\par 21 A w imieniu jego narodowie beda nadzieje mieli.
\par 22 Tedy przywiedziono do niego opetanego, slepego i niemego, i uzdrowil go, tak iz on slepy i niemy i mówil i widzial.
\par 23 I zdumial sie wszystek lud, i mówili: Nie tenze jest on syn Dawidowy?
\par 24 Ale Faryzeuszowie uslyszawszy to, rzekli: Ten nie wygania dyjablów, tylko przez Beelzebuba, ksiazecia dyjabelskiego.
\par 25 Lecz Jezus widzac mysli ich, rzekl im: Kazde królestwo rozdzielone samo przeciwko sobie pustoszeje, i kazde miasto albo dom, sam przeciwko sobie rozdzielony, nie ostoi sie.
\par 26 A jezliz szatan szatana wygania, sam przeciwko sobie rozdzielony jest; jakoz sie tedy ostoi królestwo jego?
\par 27 A jezliz ja przez Beelzebuba wyganiam dyjably, synowie wasi przez kogoz wyganiaja? Przetoz oni sedziami waszymi beda;
\par 28 A jezliz ja duchem Bozym wyganiam dyjably, tedyz do was przyszlo królestwo Boze.
\par 29 Albo jakoz moze kto wnijsc do domu mocarza, i sprzet jego rozchwycic, jezliby pierwej nie zwiazal mocarza onego? toz dopiero dom jego rozchwyci.
\par 30 Kto nie jest ze mna, przeciwko mnie jest, a kto nie zbiera ze mna, rozprasza.
\par 31 Dlatego powiadam wam: Wszelki grzech i bluznierstwo ludziom odpuszczone bedzie; ale bluznierstwo przeciwko Duchowi Swietemu nie bedzie odpuszczone ludziom.
\par 32 I ktobykolwiek rzekl slowo przeciwko Synowi czlowieczemu, bedzie mu odpuszczono; ale kto by mówil przeciwko Duchowi Swietemu, nie bedzie mu odpuszczono, ani w tym wieku ani w przyszlym.
\par 33 Czynciez albo drzewo dobre, i owoc jego dobry; albo czyncie drzewo zle, i owoc jego zly; albowiem z owocu drzewo poznane bywa.
\par 34 Rodzaju jaszczurczy! jakoz mozecie mówic dobre rzeczy, bedac zlymi, gdyz z obfitosci serca usta mówia?
\par 35 Dobry czlowiek z dobrego skarbu serca wynosi rzeczy dobre, a zly czlowiek ze zlego skarbu wynosi rzeczy zle.
\par 36 Ale powiadam wam, iz z kazdego slowa próznego, które by mówili ludzie, dadza z niego liczbe w dzien sadny;
\par 37 Albowiem z mów twoich bedziesz usprawiedliwiony, i z mów twoich bedziesz osadzony.
\par 38 Tedy odpowiedzieli niektórzy z nauczonych w Pismie i Faryzeuszów, mówiac: Nauczycielu, chcemy od ciebie znamie widziec.
\par 39 A on odpowiadajac rzekl im: Rodzaj zly i cudzolozny znamienia szuka; ale mu nie bedzie znamie dane, tylko ono znamie Jonasza proroka.
\par 40 Albowiem jako Jonasz byl w brzuchu wieloryba trzy dni i trzy noce, tak bedzie Syn czlowieczy w sercu ziemi trzy dni i trzy noce.
\par 41 Mezowie Niniwiccy stana na sadzie z tym rodzajem, i potepia go, przeto ze pokutowali na kazanie Jonaszowe; a oto tu wiecej nizeli Jonasz.
\par 42 Królowa z poludnia stanie na sadzie z tym rodzajem, i potepi go; iz przyszla od krajów ziemi, aby sluchala madrosci Salomonowej; a oto tu wiecej nizeli Salomon.
\par 43 A gdy nieczysty duch od czlowieka wychodzi, przechadza sie po miejscach suchych, szukajac odpocznienia, ale nie znajduje.
\par 44 Tedy mówi: Wróce sie do domu mego, skadem wyszedl; a przyszedlszy znajduje go prózny i umieciony i ochedozony.
\par 45 Tedy idzie, i bierze z soba siedm inszych duchów gorszych, nizeli sam: a wszedlszy mieszkaja tam, i bywaja ostatnie rzeczy czlowieka onego gorsze, nizeli pierwsze. Tak sie stanie i temu rodzajowi zlemu.
\par 46 A gdy on jeszcze mówil do ludu, oto matka i bracia jego stali przed domem chcac z nim mówic.
\par 47 I rzekl mu niektóry: Oto matka twoja i bracia twoi stoja przed domem, chcac z toba mówic.
\par 48 A on odpowiadajac, rzekl temu, co mu to powiedzial: Któraz jest matka moja? i którzy sa bracia moi?
\par 49 A wyciagnawszy reke swoje na uczniów swoich, rzekl: Oto matka moja i bracia moi!
\par 50 Albowiem ktobykolwiek czynil wole Ojca mojego, który jest w niebiesiech, ten jest bratem moim, i siostra i matka.

\chapter{13}

\par 1 A dnia onego wyszedlszy Jezus z domu, usiadl nad morzem:
\par 2 I zebral sie do niego wielki lud, tak iz wstapiwszy w lódz, siedzial, a wszystek lud stal na brzegu.
\par 3 I mówil do nich wiele w podobienstwach i rzekl: Oto wyszedl rozsiewca, aby rozsiewal;
\par 4 A gdy on rozsiewal, niektóre padlo podle drogi; i przylecialy ptaki, a podziobaly je.
\par 5 Drugie zasie padlo na miejsce opoczyste, gdzie nie mialo wiele ziemi; i wnet weszlo, iz nie mialo glebokosci ziemi.
\par 6 Ale gdy slonce weszlo, wygorzalo, a iz nie mialo korzenia, uschlo.
\par 7 A drugie padlo miedzy ciernie, i wzrosly ciernie, a zadusily je.
\par 8 A drugie padlo na ziemie dobra i wydalo pozytek, jedno setny, drugie szescdziesiatny, a drugie trzydziestny.
\par 9 Kto ma uszy ku sluchaniu, niechaj slucha.
\par 10 Tedy przystapiwszy uczniowie, rzekli mu: Dlaczegoz im w podobienstwach mówisz?
\par 11 A on odpowiadajac, rzekl im: Wam dano wiedziec tajemnice królestwa niebieskiego, ale onym nie dano;
\par 12 Albowiem kto ma, bedzie mu dano, i obfitowac bedzie, ale kto nie ma, i to, co ma, bedzie od niego odjeto.
\par 13 Dlategoc im w podobienstwach mówie, iz widzac nie widza, i slyszac nie slysza, ani rozumieja.
\par 14 I pelni sie w nich proroctwo Izajaszowe, które mówi: Sluchem sluchac bedziecie, ale nie zrozumiecie; i widzac widziec bedziecie, ale nie ujrzycie;
\par 15 Albowiem zatylo serce ludu tego, a uszyma ciezko slyszeli, i oczy swe zamruzyli, zeby kiedy oczyma nie widzieli i uszyma nie slyszeli, a sercem nie zrozumieli, i nie nawrócili sie, a uzdrowilbym je.
\par 16 Ale oczy wasze blogoslawione, ze widza, i uszy wasze, ze slysza;
\par 17 Bo zaprawde powiadam wam, iz wiele proroków i sprawiedliwych zadalo widziec to, co wy widzicie, ale nie widzieli, i slyszec to, co slyszycie, ale nie slyszeli.
\par 18 Wy tedy sluchajcie podobienstwa onego rozsiewcy.
\par 19 Gdy kto slucha slowa o tem królestwie, a nie rozumie, przychodzi on zly i porywa to, co wsiano w serce jego; tenci jest on, który podle drogi posiany jest.
\par 20 A na opoczystych miejscach posiany, ten jest, który slucha slowa i zaraz je z radoscia przyjmuje;
\par 21 Ale nie ma korzenia w sobie, lecz doczesny jest; a gdy przychodzi ucisk, albo przesladowanie dla slowa, wnet sie gorszy.
\par 22 A miedzy ciernie posiany, ten jest, który slucha slowa; ale pieczolowanie swiata tego i omamienie bogactw zadusza slowo, i staje sie bez pozytku.
\par 23 A na dobrej ziemi posiany, jest ten, który slucha slowa i rozumie, tenci pozytek przynosi; a przynosi jeden setny, drugi szescdziesiatny, a drugi trzydziestny.
\par 24 Drugie podobienstwo przelozyl im, mówiac: Podobne jest królestwo niebieskie czlowiekowi, rozsiewajacemu dobre nasienie na roli swojej.
\par 25 A gdy ludzie zasneli, przyszedl nieprzyjaciel jego, i nasial kakolu miedzy pszenica, i odszedl.
\par 26 A gdy urosla trawa i pozytek przyniosla, tedy sie pokazal i kakol.
\par 27 Tedy przystapiwszy sludzy gospodarscy, rzekli mu: Panie! izalis dobrego nasienia nie nasial na roli twojej? Skadze tedy ma kakol?
\par 28 A on im rzekl: Nieprzyjaciel czlowiek to uczynil. I rzekli sludzy do niego: A chceszze, iz pójdziemy, a zbierzemy go?
\par 29 A on rzekl: Nie! byscie snac zbierajac kakol, nie wykorzenili zaraz z nim i pszenicy.
\par 30 Dopusccie obojgu spolem rosc az do zniwa; a czasu zniwa rzeke zencom: Zbierzcie pierwej kakol, a zwiazcie go w snopki ku spaleniu; ale pszenice zgromadzcie do gumna mojego.
\par 31 Insze podobienstwo przelozyl im, mówiac: Podobne jest królestwo niebieskie ziarnu gorczycznemu, które wziawszy czlowiek, wsial na roli swojej.
\par 32 Które najmniejszec jest ze wszystkich nasion; ale kiedy urosnie, najwieksze jest ze wszystkich jarzyn, i staje sie drzewem, tak iz ptaki niebieskie przylatujac, gniazda sobie czynia na galazkach jego.
\par 33 Insze podobienstwo powiedzial im: Podobne jest królestwo niebieskie kwasowi, który wziawszy niewiasta, zakryla we trzy miary maki, azby wszystka skwasniala.
\par 34 To wszystko mówil Jezus w podobienstwach do ludu, a bez podobienstwa nie mówil do nich;
\par 35 Aby sie wypelnilo, co powiedziano przez proroka mówiacego: Otworze w podobienstwach usta moje, wypowiem skryte rzeczy od zalozenia swiata.
\par 36 Tedy rozpusciwszy on lud, przyszedl Jezus do domu; i przystapili do niego uczniowie jego, mówiac: Wylóz nam podobienstwo o kakolu onej roli.
\par 37 A on odpowiadajac, rzekl im: Ten, który rozsiewa dobre nasienie, jest Syn czlowieczy;
\par 38 A rola jest swiat, a dobre nasienie sa synowie królestwa; ale kakol sa synowie onego zlego;
\par 39 Nieprzyjaciel zasie, który go rozsial, jestci dyjabel, a zniwo jest dokonanie swiata, a zency sa Aniolowie.
\par 40 Jako tedy zbieraja kakol, a pala go ogniem, tak bedzie przy dokonaniu swiata tego.
\par 41 Posle Syn czlowieczy Anioly swoje, a oni zbiora z królestwa jego wszystkie zgorszenia, i te, którzy nieprawosc czynia;
\par 42 I wrzuca je w piec ognisty, tam bedzie placz i zgrzytanie zebów.
\par 43 Tedy sprawiedliwi lsnic sie beda jako slonce w królestwie Ojca swego. Kto ma uszy ku sluchaniu, niechaj slucha.
\par 44 Zasie podobne jest królestwo niebieskie skarbowi skrytemu w roli, który znalazlszy czlowiek skryl, i od radosci, która mial z niego, odchodzi, i wszystko, co ma, sprzedaje, i kupuje one role.
\par 45 Zasie podobne jest królestwo niebieskie czlowiekowi kupcowi, szukajacemu pieknych perel;
\par 46 Który znalazlszy jedne perle bardzo droga, odszedl, i posprzedawal wszystko, co mial, i kupil ja.
\par 47 Zasie podobne jest królestwo niebieskie niewodowi zapuszczonemu w morze, i ryby wszelkiego rodzaju zagarniajacemu.
\par 48 Który gdy byl pelen, wyciagneli rybitwi na brzeg, a usiadlszy, wybierali dobre ryby w naczynia, a zle precz wyrzucali.
\par 49 Takci bedzie przy dokonaniu swiata; wynijda Aniolowie, i wylacza zle z posrodku sprawiedliwych,
\par 50 I wrzuca je w piec ognisty; tam bedzie placz i zgrzytanie zebów.
\par 51 Rzekl im Jezus: Wyrozumieliscie to wszystko? Rzekli mu: Tak, Panie!
\par 52 A on im rzekl: Przetoz kazdy nauczony w Pismie, wycwiczony w królestwie niebieskiem, podobny jest czlowiekowi gospodarzowi, który wynosi z skarbu swego nowe i stare rzeczy.
\par 53 I stalo sie, gdy Jezus dokonczyl tych podobienstw, puscil sie stamtad.
\par 54 A przyszedlszy do ojczyzny swojej, nauczal je w bóznicy ich, tak iz sie bardzo zdumiewali i mówili: Skadze temu ta madrosc, i ta moc?
\par 55 Izaz ten nie jest on syn ciesli? Izaz matki jego nie zowia Maryja, a bracia jego Jakób, i Jozes, i Szymon, i Judas?
\par 56 A siostry jego izali wszystkie u nas nie sa? Skadze tedy temu to wszystko?
\par 57 I gorszyli sie z niego; ale Jezus rzekl im: Nie jest prorok beze czci, tylko w ojczyznie swojej i w domu swoim.
\par 58 I nie uczynil tam wiele cudów dla niedowiarstwa ich.

\chapter{14}

\par 1 W on czas uslyszal Herod Tetrarcha, wiesc o Jezusie.
\par 2 I rzekl slugom swoim: Tenci jest Jan Chrzciciel; on to zmartwychwstal, i dlatego sie cuda przez niego dzieja.
\par 3 Albowiem Herod pojmawszy Jana, zwiazal go byl i wsadzil do wiezienia dla Herodyjady, zony Filipa, brata swego.
\par 4 Bo mu Jan mówil: Nie godzi ci sie jej miec.
\par 5 Ale gdy go on chcial zabic, bal sie ludu: albowiem go za proroka mieli.
\par 6 Gdy tedy obchodzono dzien narodzenia Herodowego, tancowala córka Herodyjady w posrodku gosci, i podobala sie Herodowi.
\par 7 Skad pod przysiega obiecal jej dac, czegobykolwiek zadala.
\par 8 A ona przedtem bedac naprawiona od matki swojej, rzekla: Daj mi tu na misie glowe Jana Chrzciciela.
\par 9 I zasmucil sie król; ale dla przysiegi i dla spólsiedzacych kazal jej dac.
\par 10 A poslawszy kata, scial Jana w wiezieniu.
\par 11 I przyniesiono glowe jego na misie, i oddano dzieweczce, i odniosla ja matce swojej.
\par 12 A przyszedlszy uczniowie jego wzieli cialo i pogrzebli je, a szedlszy powiedzieli Jezusowi.
\par 13 To uslyszawszy Jezus, ustapil stamtad w lodzi na miejsce puste osobno; a uslyszawszy lud, szli za nim z miast pieszo.
\par 14 Wyszedlszy tedy Jezus ujrzal wielki lud, i uzalil sie ich, a uzdrawial chore ich.
\par 15 A gdy nadchodzil wieczór, przystapili do niego uczniowie jego, mówiac: Puste jest to miejsce, a czas juz przeminal; rozpusc ten lud, aby odszedlszy do miasteczek, kupili sobie zywnosci.
\par 16 A Jezus im rzekl: Nie potrzeba im odchodzic, dajcie wy im co jesc.
\par 17 Ale mu oni rzekli: Nie mamy tu, tylko piec chlebów i dwie ryby.
\par 18 A on rzekl: Przyniescie mi je tu.
\par 19 I rozkazawszy ludowi usiasc na trawie, wzial onych piec chlebów i dwie ryby, a wejrzawszy w niebo, blogoslawil, a lamiac dawal uczniom chleby, a uczniowie ludowi.
\par 20 I jedli wszyscy, a nasyceni byli; i zebrali, co zbywalo ulomków, dwanascie koszów pelnych.
\par 21 A tych, którzy jedli, bylo okolo pieciu tysiecy mezów, oprócz niewiast i dziatek.
\par 22 A wnetze przymusil Jezus uczniów swoich, aby wstapili w lódz, i uprzedzili go na druga strone, azby rozpuscil lud.
\par 23 A rozpusciwszy lud, wstapil na góre z osobna, aby sie modlil; a gdy byl wieczór, sam tam byl.
\par 24 A lódz juz w posrodku morza bedac, miotana byla od walów; albowiem byl wiatr przeciwny.
\par 25 Lecz o czwartej strazy nocnej szedl do nich Jezus, chodzac po morzu.
\par 26 A ujrzawszy go uczniowie po morzu chodzacego, zatrwozyli sie, mówiac: Obluda to jest! i od bojazni krzykneli.
\par 27 Lecz wnet rzekl do nich Jezus, mówiac: Ufajcie! Jam ci to jest; nie bójcie sie.
\par 28 A odpowiadajac mu Piotr rzekl: Panie! Jezlizes ty jest, kaz mi przyjsc do ciebie po wodzie.
\par 29 A on rzekl: Pójdz! A Piotr, wystapiwszy z lodzi, szedl po wodzie, aby przyszedl do Jezusa;
\par 30 Ale widzac wiatr gwaltowny, zlakl sie; a gdy poczal tonac, zakrzyknal, mówiac: Panie, ratuj mie!
\par 31 A Jezus zaraz wyciagnawszy reke, uchwycil go i rzekl mu: O malowierny! przeczzes watpil?
\par 32 A gdy oni wstapili w lódz, uciszyl sie wiatr.
\par 33 A ci, którzy byli w lodzi, przystapiwszy poklonili mu sie, mówiac: Prawdziwie jestes Synem Bozym.
\par 34 I przeprawiwszy sie, przyszli do ziemi Gienezaret.
\par 35 A poznawszy go mezowie miejsca onego, poslali do wszystkiej onej okolicznej krainy; i przyniesiono do niego wszystkie, którzy sie zle mieli.
\par 36 I prosili go, aby sie tylko podolku szaty jego dotykali; a którzykolwiek sie dotkneli, uzdrowieni sa.

\chapter{15}

\par 1 Tedy przystapili do Jezusa z Jeruzalemu nauczeni w Pismie i Faryzeuszowie, mówiac:
\par 2 Czemu uczniowie twoi przestepuja ustawe starszych? albowiem nie umywaja rak swych, gdy maja jesc chleb.
\par 3 A on odpowiadajac, rzekl im: Czemuz i wy przestepujecie przykazanie Boze dla ustawy waszej?
\par 4 Albowiem Bóg przykazal, mówiac: Czcij ojca twego i matke; i kto by zlorzeczyl ojcu albo matce, smiercia niechaj umrze.
\par 5 Ale wy powiadacie: Kto by rzekl ojcu albo matce: Dar, którykolwiek jest ode mnie, tobie pozyteczny bedzie; a nie uczcilby ojca swego albo matki swojej, bez winy bedzie.
\par 6 I wzruszyliscie przykazania Boze dla ustawy waszej.
\par 7 Obludnicy! dobrze o was prorokowal Izajasz, mówiac:
\par 8 Lud ten przybliza sie do mnie usty swemi, i wargami czci mie; ale serce ich daleko jest ode mnie.
\par 9 Lecz prózno mie czcza, nauczajac nauk, które sa przykazania ludzkie.
\par 10 A zawolawszy do siebie ludu, rzekl im: Sluchajcie, a rozumiejcie.
\par 11 Nie to, co wchodzi w usta, pokala czlowieka; ale co wychodzi z ust, to pokala czlowieka.
\par 12 Tedy przystapiwszy uczniowie jego, rzekli mu: Wiesz, iz Faryzeuszowie, uslyszawszy te mowe, zgorszyli sie?
\par 13 A on odpowiadajac rzekl: Wszelki szczep, którego nie szczepil Ojciec mój niebieski, wykorzeniony bedzie.
\par 14 Zaniechajcie ich; slepi sa wodzowie slepych, a slepy jezliby slepego prowadzil, obadwa w dól wpadna.
\par 15 A odpowiadajac Piotr, rzekl mu: Wylóz nam to podobienstwo.
\par 16 I rzekl Jezus: Jeszczez i wy bezrozumni jestescie?
\par 17 Jeszczez nie rozumiecie, iz wszystko, co wchodzi w usta, w brzuch idzie, i do wychodu bywa wyrzucono?
\par 18 Ale co z ust pochodzi, z serca wychodzi, a toc pokala czlowieka.
\par 19 Albowiem z serca wychodza zle mysli, mezobójstwa, cudzolóstwa, wszeteczenstwa, zlodziejstwa, falszywe swiadectwa, bluznierstwa.
\par 20 Toc jest, co pokala czlowieka: ale jesc nieumytemi rekoma, toc nie pokala czlowieka.
\par 21 A wyszedlszy Jezus stamtad, ustapil w strony Tyru i Sydonu.
\par 22 A oto niewiasta Chananejska z onych granic wyszedlszy, wolala, mówiac do niego: Zmiluj sie nade mna Panie, synu Dawidowy! córka moja ciezko bywa od dyjabla dreczona.
\par 23 A on jej nie odpowiedzial i slowa. Tedy przystapiwszy uczniowie jego, prosili go, mówiac: Odpraw ja, boc wola za nami.
\par 24 A on odpowiadajac rzekl: Nie jestem poslany, tylko do owiec, które zginely z domu Izraelskiego.
\par 25 Lecz ona przystapiwszy, poklonila mu sie, mówiac: Panie, ratuj mie!
\par 26 A on odpowiadajac rzekl: Niedobra jest brac chleb dziecinny, a miotac szczenietom.
\par 27 A ona rzekla: Tak jest, Panie! a wszakze i szczenieta jedza odrobiny, które padaja z stolu panów ich.
\par 28 Tedy odpowiadajac Jezus rzekl jej: O niewiasto! wielka jest wiara twoja; niechaj ci sie stanie, jako chcesz. I uzdrowiona jest córka jej od onejze godziny.
\par 29 A Jezus poszedlszy stamtad, przyszedl nad morze Galilejskie, a wstapiwszy na góre, siedzial tam.
\par 30 I przyszedl do niego wielki lud, majac z soba chrome, slepe, nieme, ulomne i inszych wiele, i kladli je u nóg Jezusowych, i uzdrawial je,
\par 31 Tak iz sie on lud dziwowal, widzac, ze niemi mówia, ulomni uzdrowieni sa, chromi chodza, a slepi widza; i wielbili Boga Izraelskiego.
\par 32 Lecz Jezus zwolawszy uczniów swoich, rzekl: Zal mi tego ludu; albowiem juz trzy dni przy mnie trwaja, a nie maja, co by jedli, a nie chce ich rozpuscic glodnych, by snac nie pomdleli na drodze.
\par 33 Tedy mu rzekli uczniowie jego: Skadze bysmy wzieli tak wiele chleba na tej puszczy, abysmy tak wielki lud nasycili?
\par 34 I rzekl im Jezus: Wielez macie chlebów? A oni rzekli: Siedm, i troche rybek.
\par 35 Tedy rozkazal ludowi, aby siedli na ziemi.
\par 36 A wziawszy one siedm chlebów i one ryby, uczyniwszy dzieki, lamal i dal uczniom swoim, a uczniowie ludowi.
\par 37 I jedli wszyscy i nasyceni sa, i zebrali, co zbylo ulomków, siedm koszów pelnych.
\par 38 A bylo tych, którzy jedli, cztery tysiace mezów, oprócz niewiast i dziatek.
\par 39 Tedy rozpusciwszy lud, wstapil w lódz, i przyszedl na granice Magdalanskie.

\chapter{16}

\par 1 A przystapiwszy Faryzeuszowie i Saduceuszowie, kuszac prosili go, aby im znamie z nieba ukazal.
\par 2 A on odpowiadajac, rzekl im: Gdy bywa wieczór, mówicie: Pogoda bedzie; bo sie niebo czerwieni.
\par 3 A rano: Dzis bedzie niepogoda; albowiem sie niebo pochmurne czerwieni. Obludnicy! postawe nieba rozsadzic umiecie, a znamion tych czasów nie mozecie.
\par 4 Rodzaj zly i cudzolozny znamienia szuka; ale mu znamie nie bedzie dane, tylko ono znamie Jonasza proroka. I opusciwszy je, odszedl.
\par 5 A gdy sie przeprawili uczniowie jego na druga strone morza, zapamietali wziac chleba.
\par 6 I rzekl im Jezus: Patrzcie, a strzezcie sie kwasu Faryzeuszów i Saduceuszów.
\par 7 A oni rozmawiali miedzy soba, mówiac: Nie wzielismy chleba.
\par 8 Co obaczywszy Jezus, rzekl im: O czemze rozmawiacie miedzy soba, o malowierni, zescie chleba nie wzieli?
\par 9 Jeszczez nie rozumiecie, ani pamietacie onych pieciu chlebów, a onych pieciu tysiecy ludzi, i jakoscie wiele koszów zebrali?
\par 10 Ani onych siedmiu chlebów i czterech tysiecy ludzi, a jakoscie wiele koszów nazbierali?
\par 11 Jakoz nie rozumiecie, zem wam nie o chlebie powiedzial, mówiac: Abyscie sie strzegli kwasu Faryzeuszów i Saduceuszów?
\par 12 Tedy zrozumieli, ze nie mówil, aby sie strzegli kwasu chleba, ale nauki Faryzeuszów i Saduceuszów.
\par 13 A gdy przyszedl Jezus w strony Cezaryi Filippowej, pytal uczniów swoich, mówiac: Kimze mie powiadaja byc ludzie Syna czlowieczego?
\par 14 A oni rzekli: Jedni Janem Chrzcicielem, a drudzy Elijaszem, insi tez Jeremijaszem, albo jednym z proroków.
\par 15 I rzekl im: A wy kim mie byc powiadacie?
\par 16 A odpowiadajac Szymon Piotr rzekl: Tys jest Chrystus, on Syn Boga zywego.
\par 17 Tedy odpowiadajac Jezus rzekl mu: Blogoslawiony jestes Szymonie, synu Jonaszowy! bo tego cialo i krew nie objawily tobie, ale Ojciec mój, który jest w niebiesiech.
\par 18 A Ja ci tez powiadam, zes ty jest Piotr; a na tej opoce zbuduje kosciól mój, a bramy piekielne nie przemoga go:
\par 19 I tobie dam klucze królestwa niebieskiego; a cokolwiek zwiazesz na ziemi, bedzie zwiazane i w niebiesiech; a cokolwiek rozwiazesz na ziemi, bedzie rozwiazane i w niebiesiech.
\par 20 Tedy przykazal uczniom swoim, aby nikomu nie powiadali, ze on jest Jezus Chrystus.
\par 21 I odtad poczal Jezus pokazywac uczniom swoim, iz musi odejsc do Jeruzalemu, i wiele cierpiec od starszych i od przedniejszych kaplanów i nauczonych w Pismie, a byc zabitym i trzeciego dnia zmartwychwstac.
\par 22 A wziawszy go Piotr na strone, poczal go strofowac, mówiac: Zmiluj sie sam nad soba, Panie! nie przyjdzie to na cie.
\par 23 A on obróciwszy sie, rzekl Piotrowi: Idz ode mnie, szatanie! jestes mi zgorszeniem; albowiem nie pojmujesz tego, co jest Bozego, ale co jest ludzkiego.
\par 24 Tedy rzekl Jezus do uczniów swoich: Jezli kto chce isc za mna, niechajze samego siebie zaprze, a wezmie krzyz swój, i nasladuje mie!
\par 25 Bo kto by chcial dusze swoje zachowac, straci ja; a kto by stracil dusze swoje dla mnie, znajdzie ja.
\par 26 Albowiem cóz pomoze czlowiekowi, chocby wszystek swiat pozyskal, a na duszy swojej szkodowal? albo co za zamiane da czlowiek za dusze swoje?
\par 27 Albowiem Syn czlowieczy przyjdzie w chwale Ojca swego z Anioly swoimi, a tedy odda kazdemu wedlug uczynków jego.
\par 28 Zaprawde powiadam wam: Sa niektórzy z tych, co tu stoja, którzy nie ukusza smierci, azby ujrzeli Syna czlowieczego, idacego w królestwie swojem.

\chapter{17}

\par 1 A po szesciu dniach wzial Jezus Piotra i Jakóba i Jana, brata jego, i wprowadzil je na góre wysoka osobno.
\par 2 I przemieniony jest przed nimi, a rozjasnilo sie oblicze jego jako slonce, a szaty jego staly sie biale jako swiatlosc.
\par 3 A oto ukazali sie im Mojzesz i Elijasz, z nim rozmawiajacy.
\par 4 I odpowiadajac Piotr, rzekl do Jezusa: Panie! dobrze nam tu byc; jezli chcesz, uczynimy tu trzy namioty, tobie jeden, i Mojzeszowi jeden, i Elijaszowi jeden.
\par 5 A gdy on jeszcze mówil, oto oblok jasny zacienil je; a oto glos z obloku mówiacy: Ten jest Syn mój mily, w którym mi sie upodobalo, tego sluchajcie.
\par 6 To uslyszawszy uczniowie, upadli na twarz swoje i bali sie bardzo.
\par 7 Tedy przystapiwszy Jezus dotknal sie ich i rzekl: Wstancie, a nie bójcie sie.
\par 8 A oni podnióslszy oczy swoje, nikogo nie widzieli, tylko Jezusa samego.
\par 9 A gdy zstepowali z góry, przykazal im Jezus, mówiac: Nikomu nie powiadajcie tego widzenia, az Syn czlowieczy zmartwychwstanie.
\par 10 I pytali go uczniowie jego, mówiac: Cóz tedy nauczeni w Pismie powiadaja, ze ma Elijasz pierwej przyjsc?
\par 11 A Jezus odpowiadajac, rzekl im: Elijaszci pierwej przyjdzie i naprawi wszystko;
\par 12 Ale wam powiadam: Iz Elijasz juz przyszedl, wszakze nie poznali go, ale uczynili mu, cokolwiek chcieli; takci i Syn czlowieczy ma ucierpiec od nich.
\par 13 Tedy zrozumieli uczniowie, ze o Janie Chrzcicielu mówil do nich.
\par 14 A gdy przyszli do ludu, przystapil do niego czlowiek, i upadl przed nim na kolana,
\par 15 I rzekl: Panie! zmiluj sie nad synem moim: albowiem lunatykiem jest, i ciezko sie trapi; czestokroc bowiem wpada w ogien, i czestokroc w wode.
\par 16 I przywiodlem go do uczniów twoich, ale go nie mogli uzdrowic.
\par 17 A odpowiadajac Jezus, rzekl: O rodzaju niewierny i przewrotny! Dokadze bede z wami? Dokadze was bede cierpial? przywiedzcie mi go sam.
\par 18 I zgromil onego dyjabla Jezus; i wyszedl od niego, i uzdrowiony jest on mlodzieniec od onejze godziny.
\par 19 Tedy przystapiwszy uczniowie do Jezusa osobno, rzekli mu: Czemuzesmy go my wygnac nie mogli?
\par 20 Lecz Jezus rzekl do nich: Dla niedowiarstwa waszego; zaprawde bowiem powiadam wam: Jezlibyscie, majac wiare jako ziarno gorczyczne, rzekli tej górze: Przenies sie stad na ono miejsce, tedy sie przeniesie, a nic niemozebnego wam nie bedzie.
\par 21 Ale ten rodzaj nie wychodzi, tylko przez modlitwe i przez post.
\par 22 A gdy przebywali w Galilei, rzekl do nich Jezus: Syn czlowieczy bedzie wydany w rece ludzkie;
\par 23 I zabija go, ale trzeciego dnia zmartwychwstanie. I zasmucili sie bardzo.
\par 24 A gdy przyszli do Kapernaum, przystapili do Piotra ci, którzy podatek wybierali, i rzekli: Izali nauczyciel wasz nie daje podatku?
\par 25 I rzekl: Tak. A gdy wchodzil w dom, uprzedzil go Jezus, mówiac: Cóz ci sie zda, Szymonie? Królowie ziemscy od kogoz biora clo albo czynsz? od synów swoich, czyli od obcych?
\par 26 Rzekl mu Piotr: Od obcych. I rzekl mu Jezus: Toc tedy synowie sa wolni.
\par 27 Wszakze abysmy ich nie zgorszyli, szedlszy do morza, zarzuc wedke, a te rybe, która najpierwej uwieznie, wezmij, a otworzywszy gebe jej, znajdziesz stater, który wziawszy, daj im za mie i za sie.

\chapter{18}

\par 1 Onej godziny przystapili uczniowie do Jezusa, mówiac: Któz wzdy najwiekszy jest w królestwie niebieskiem?
\par 2 A zawolawszy Jezus dzieciecia, postawil je w posrodku ich,
\par 3 I rzekl: Zaprawde powiadam wam: Jezli sie nie nawrócicie i nie staniecie sie jako dzieci, zadnym sposobem nie wnijdziecie do królestwa niebieskiego.
\par 4 Kto sie tedy unizy jako to dziecie, tenci jest najwiekszym w królestwie niebieskiem.
\par 5 A kto by przyjal jedno dziecie takie w imieniu mojem, mnie przyjmuje.
\par 6 Kto by zas zgorszyl jednego z tych malych, którzy we mie wierza, pozyteczniej by mu bylo, aby zawieszony byl kamien mlynski na szyi jego, a utopiony byl w glebokosci morskiej.
\par 7 Biada swiatu dla zgorszenia! albowiem musza zgorszenia przyjsc; wszakze biada czlowiekowi onemu, przez którego przychodzi zgorszenie!
\par 8 Przetoz jezli reka twoja albo noga twoja gorszy cie, odetnij ja i zarzuc od siebie; lepiej jest tobie wnijsc do zywota chromym albo ulomnym, nizeli dwie rece albo dwie nogi majac, wrzuconym byc do ognia wiecznego.
\par 9 A jezli cie oko twoje gorszy, wylup je i zarzuc od siebie; lepiej jest tobie jednookim wnijsc do zywota, nizeli oba oczy majac, byc wrzuconym do ognia piekielnego.
\par 10 Patrzajciez, abyscie nie gardzili zadnym z tych maluczkich; albowiem wam powiadam, iz Aniolowie ich w niebiesiech zawsze patrza na oblicze Ojca mojego, który jest w niebiesiech.
\par 11 Przyszedl bowiem Syn czlowieczy, aby zbawil to, co bylo zginelo.
\par 12 Co sie wam zda? Gdyby który czlowiek mial sto owiec, a zablakalaby sie jedna z nich, azaz nie zostawia onych dziewiecdziesieciu i dziewieciu, a poszedlszy na góry, nie szuka zblakanej?
\par 13 A jezli mu sie zdarzy, znalezc ja, zaprawde powiadam wam, ze sie z niej bardziej raduje, niz z onych dziewiecdziesieciu i dziewieciu nie zblakanych.
\par 14 Tak nie jest wola Ojca waszego, który jest w niebiesiech, aby zginal jeden z tych maluczkich.
\par 15 A jezliby zgrzeszyl przeciwko tobie brat twój, idz, strofuj go miedzy toba i onym samym: jezli cie uslucha, pozyskales brata twego.
\par 16 Ale jezli cie nie uslucha, przybierz do siebie jeszcze jednego albo dwóch, aby w usciech dwóch albo trzech swiadków stanelo kazde slowo.
\par 17 A jezliby ich nie usluchal, powiedz zborowi; a jezliby zboru nie usluchal, niech ci bedzie jako poganin i celnik.
\par 18 Zaprawde powiadam wam: Cokolwiek byscie zwiazali na ziemi, bedzie zwiazane i na niebie; a co byscie rozwiazali na ziemi; bedzie rozwiazane i na niebie.
\par 19 Zasie powiadam wam: Iz gdyby sie z was dwaj zgodzili na ziemi o wszelka rzecz, o która by prosili, stanie sie im od Ojca mego, który jest w niebiesiech.
\par 20 Albowiem gdzie sa dwaj albo trzej zgromadzeni w imie moje, tam jestem w posrodku ich.
\par 21 Tedy przystapiwszy do niego Piotr, rzekl: Panie! wielekroc zgrzeszy przeciwko mnie brat mój, a odpuszcze mu? czyz az do siedmiu kroc?
\par 22 I rzekl mu Jezus: Nie mówie ci az do siedmiu kroc, ale az do siedmdziesiat siedmiu kroc.
\par 23 Dlatego podobne jest królestwo niebieskie czlowiekowi królowi, który sie chcial rachowac z slugami swymi.
\par 24 A gdy sie poczal rachowac, stawiono mu jednego, który byl winien dziesiec tysiecy talentów.
\par 25 A gdy nie mial skad oddac, kazal go pan jego zaprzedac, i zone jego, i dzieci, i wszystko, co mial, i dlug oddac.
\par 26 Upadlszy tedy sluga on, poklonil mu sie, mówiac: Panie! miej cierpliwosc nade mna, a wszystko ci oddam.
\par 27 A uzaliwszy sie pan onego slugi, uwolnil go, i dlug mu odpuscil.
\par 28 A wyszedlszy on sluga, znalazl jednego z spólslug swoich, który mu byl winien sto groszy; a porwawszy go, dusil go, mówiac: Oddaj mi, cos winien.
\par 29 Przypadlszy tedy on spólsluga jego do nóg jego, prosil go, mówiac: Miej cierpliwosc nade mna, a oddam ci wszystko.
\par 30 Lecz on nie chcial, ale szedlszy wrzucil go do wiezienia, azby oddal, co byl winien.
\par 31 Ujrzawszy tedy spólsludzy jego, co sie stalo, zasmucili sie bardzo, a szedlszy oznajmili panu swemu wszystko, co sie stalo.
\par 32 Tedy zawolawszy go pan jego, rzekl mu: Slugo zly! wszystek on dlug odpuscilem ci, zes mie prosil.
\par 33 Azazes sie i ty nie mial zmilowac nad spólsluga twoim, jakom sie i ja zmilowal nad toba?
\par 34 A rozgniewawszy sie pan jego, podal go katom, azby oddal to wszystko, co mu byl winien.
\par 35 Tak i Ojciec mój niebieski uczyni wam, jezli nie odpuscicie kazdy bratu swemu z serc waszych upadków ich.

\chapter{19}

\par 1 I stalo sie, gdy dokonczyl Jezus tych mów, odszedl z Galilei, a przyszedl na granice Judzkie nad Jordan.
\par 2 I szedl za nim wielki lud, i uzdrawial je tam.
\par 3 Tedy przyszli do niego Faryzeuszowie, kuszac go i mówiac mu: Godzili sie czlowiekowi opuscic zone swoja dla kazdej przyczyny?
\par 4 A on odpowiadajac rzekl im: Nie czytaliscie, iz ten, który stworzyl na poczatku czlowieka, mezczyzne i niewiaste uczynil je?
\par 5 I rzekl: Dlatego opusci czlowiek ojca i matke, a przylaczy sie do zony swojej, i beda dwoje jednem cialem.
\par 6 A tak juz nie sa dwoje, ale jedno cialo; co tedy Bóg zlaczyl, czlowiek niechaj nie rozlacza.
\par 7 Rzekli mu: Przeczze tedy Mojzesz kazal dac list rozwodny i opuscic ja?
\par 8 Rzekl im: Mojzesz dla zatwardzenia serca waszego dopuscil wam, opuscic zony wasze, lecz z poczatku nie bylo tak.
\par 9 Ale ja powiadam wam: Iz ktobykolwiek opuscil zone swoje, (oprócz dla wszeteczenstwa), a insza by pojal, cudzolozy; a kto by opuszczona pojal, cudzolozy.
\par 10 Rzekli mu uczniowie jego: Jezlic taka jest sprawa meza z zona, tedy nie jest dobrze zenic sie.
\par 11 A on im rzekl: Nie wszyscy pojmuja tej rzeczy, ale tylko ci, którym to dano.
\par 12 Albowiem sa rzezancy, którzy sie tak z zywota matki narodzili; sa tez rzezancy, którzy od ludzi sa urzezani; sa tez rzezancy, którzy sie sami urzezali dla królestwa niebieskiego. Kto moze pojac, niechaj pojmuje!
\par 13 Tedy mu przynoszono dziatki, aby na nie rece wkladal i modlil sie; ale uczniowie gromili je.
\par 14 Lecz Jezus rzekl: Zaniechajcie dziatek, a nie zabraniajcie im przychodzic do mnie; albowiem takich jest królestwo niebieskie.
\par 15 A wlozywszy na nie rece, poszedl stamtad.
\par 16 A oto jeden przystapiwszy, rzekl mu: Nauczycielu dobry! co dobrego mam czynic, abym mial zywot wieczny?
\par 17 Ale mu on rzekl: Przecz mie zowiesz dobrym? nikt nie jest dobry, tylko jeden, to jest Bóg; a jezli chcesz wnijsc do zywota, przestrzegaj przykazan.
\par 18 I rzekl mu: Których? A Jezus rzekl: Nie bedziesz zabijal, nie bedziesz cudzolozyl, nie bedziesz kradl, nie bedziesz mówil falszywego swiadectwa;
\par 19 Czcij ojca twego i matke, i milowac bedziesz blizniego swego, jako siebie samego.
\par 20 Rzekl mu mlodzieniec: Tegom wszystkiego przestrzegal od mlodosci swojej; czegoz mi jeszcze nie dostaje?
\par 21 Rzekl mu Jezus: Jezli chcesz byc doskonalym, idz, sprzedaj majetnosci twoje, i rozdaj ubogim, a bedziesz mial skarb w niebie, a przyszedlszy, nasladuj mie.
\par 22 A gdy mlodzieniec te slowa uslyszal, odszedl smutny; albowiem wiele mial majetnosci.
\par 23 Tedy Jezus rzekl uczniom swoim: Zaprawde powiadam wam, ze z trudnoscia bogaty wnijdzie do królestwa niebieskiego.
\par 24 I zasie powiadam wam: Ze snadniej wielbladowi przez ucho igielne przejsc, niz bogatemu wnijsc do królestwa Bozego.
\par 25 Co uslyszawszy uczniowie jego, zdumieli sie bardzo, mówiac: Któz tedy moze byc zbawion?
\par 26 A Jezus wejrzawszy na nie, rzekl im: U ludzic to nie mozna; lecz u Boga wszystko jest mozebne.
\par 27 Tedy odpowiadajac Piotr, rzekl mu: Otosmy my opuscili wszystko, i poszlismy za toba; cóz nam tedy za to bedzie?
\par 28 A Jezus rzekl im: Zaprawde powiadam wam: Iz wy, którzyscie mie nasladowali w odrodzeniu, gdy usiadzie Syn czlowieczy na stolicy chwaly swojej, usiadziecie i wy na dwunastu stolicach, sadzac dwanascie pokolen Izraelskich.
\par 29 A kazdy, kto by opuscil domy, albo braci, albo siostry, albo ojca, albo matke, albo zone, albo dzieci, albo role, dla imienia mego, stokroc wiecej wezmie, i zywot wieczny odziedziczy.
\par 30 A wiele pierwszych beda ostatnimi, a ostatnich pierwszymi.

\chapter{20}

\par 1 Albowiem podobne jest królestwo niebieskie czlowiekowi gospodarzowi, który wyszedl bardzo rano najmowac robotników do winnicy swojej.
\par 2 A zmówiwszy sie z robotnikami z grosza na dzien poslal je do winnicy swojej.
\par 3 A wyszedlszy o trzeciej godzinie, ujrzal drugich, którzy stali na rynku próznujacy;
\par 4 I rzekl im: Idzcie i wy do winnicy, a co bedzie sprawiedliwego, dam wam.
\par 5 A oni poszli. Zasie wyszedlszy o szóstej i dziewiatej godzinie, takze uczynil.
\par 6 Potem o jedenastej godzinie wyszedlszy, znalazl drugie, którzy stali próznujacy, i rzekl im: Przecz tu stoicie caly dzien próznujacy?
\par 7 Rzekli mu: Iz nas nikt nie najal; i rzekl im: Idzcie i wy do winnicy, a co bedzie sprawiedliwego, wezmiecie.
\par 8 A gdy byl wieczór, rzekl pan winnicy sprawcy swemu: Zawolaj robotników, a oddaj im zaplate, poczawszy od ostatnich az do pierwszych.
\par 9 A gdy przyszli oni, którzy o jedenastej godzinie byli najeci, wzial kazdy z nich po groszu.
\par 10 Przyszedlszy tez i pierwsi, mniemali, ze wiecej wezma; ale wzieli i oni, kazdy z nich, po groszu.
\par 11 A wziawszy, szemrali przeciwko gospodarzowi,
\par 12 Mówiac: Ci ostatni jedne godzine robili, a uczyniles je nam równymi, którzysmy znosili ciezar dnia i upalenie.
\par 13 A on odpowiadajac rzekl jednemu z nich: Przyjacielu! nie czynie ci krzywdy; azaz sie nie z grosza zmówil ze mna?
\par 14 Wezmij, co twojego jest, a idz; chce bowiem temu ostatniemu dac jako i tobie.
\par 15 Azaz mi sie nie godzi czynic z mojem, co chce? Czyli oko twoje zlosliwe jest, izem ja jest dobry?
\par 16 Takci beda ostatni pierwszymi, a pierwsi ostatnimi; albowiem wiele jest wezwanych, ale malo wybranych.
\par 17 A wstepujac Jezus do Jeruzalemu, wzial z soba dwanascie uczniów na osobne miejsce w drodze, i rzekl im:
\par 18 Oto wstepujemy do Jeruzalemu, Syn czlowieczy bedzie wydany przedniejszym kaplanom i nauczonym w Pismie, i osadza go na smierc.
\par 19 I wydadza go poganom na posmiewanie i na ubiczowanie i na ukrzyzowanie; ale trzeciego dnia zmartwychwstanie.
\par 20 Tedy przystapila do niego matka synów Zebedeuszowych z synami swoimi, klaniajac mu sie, i proszac nieco od niego.
\par 21 A on jej rzekl; Czegóz chcesz? Rzekla mu: Rzecz, aby siedzieli ci dwaj synowie moi, jeden po prawicy twojej a drugi po lewicy w królestwie twojem.
\par 22 Ale Jezus odpowiadajac rzekl: Nie wiecie, o co prosicie; mozeciez pic kielich, który ja bede pil? i chrztem, którym sie ja chrzcze, byc ochrzczeni? Rzekli mu: Mozemy.
\par 23 Tedy im rzekl: Kielichci mój pic bedziecie, i chrztem, którym sie ja chrzcze, ochrzczeni bedziecie; ale siedziec po prawicy mojej i po lewicy mojej, nie jest moja rzecz dac wam, ale tym, którym jest zgotowano od Ojca mojego.
\par 24 A uslyszawszy to oni dziesieciu, rozgniewali sie na onych dwóch braci.
\par 25 Ale Jezus zwolawszy ich, rzekl: Wiecie, iz ksiazeta narodów panuja nad nimi, a którzy wielcy sa, mocy dokazuja nad nimi.
\par 26 Lecz nie tak bedzie miedzy wami: ale ktobykolwiek miedzy wami chcial byc wielkim, niech bedzie sluga waszym.
\par 27 A ktobykolwiek miedzy wami chcial byc pierwszym, niech bedzie sluga waszym.
\par 28 Jako i Syn czlowieczy nie przyszedl, aby mu sluzono, ale aby sluzyl, i aby dal dusze swa na okup za wielu.
\par 29 A gdy oni wychodzili z Jerycha, szedl za nim wielki lud.
\par 30 A oto dwaj slepi, siedzacy przy drodze, uslyszawszy, iz Jezus przechodzil, zawolali, mówiac: Zmiluj sie nad nami, Panie, synu Dawidowy!
\par 31 Ale on lud gromil ich, aby milczeli; lecz oni tem wiecej wolali, mówiac: Zmiluj sie nad nami, Panie, synu Dawidowy!
\par 32 A zastanowiwszy sie Jezus, zawolal ich i rzekl: Cóz chcecie, abym wam uczynil?
\par 33 Rzekli mu: Panie! aby byly otworzone oczy nasze.
\par 34 A uzaliwszy sie ich Jezus, dotknal sie oczu ich, a zaraz przejrzaly oczy ich; i szli za nim.

\chapter{21}

\par 1 A gdy sie przyblizyli do Jeruzalemu, i przyszli do Betfagie, do góry oliwnej, tedy Jezus poslal dwóch uczniów,
\par 2 Mówiac im: Idzcie do miasteczka, które jest przeciwko wam, a zaraz znajdziecie oslice uwiazana i osle z nia; odwiazciez je, a przywiedzcie do mnie.
\par 3 A jezliby wam co kto rzekl, powiedzcie, iz Pan ich potrzebuje; a zarazem pusci je.
\par 4 A to sie wszystko stalo, aby sie wypelnilo, co powiedziano przez proroka, mówiacego:
\par 5 Powiedzcie córce Syonskiej: Oto król twój idzie tobie cichy, a siedzacy na oslicy, i na osleciu, synu oslicy pod jarzmem bedacej.
\par 6 Szedlszy tedy uczniowie, a uczyniwszy tak, jako im byl rozkazal Jezus,
\par 7 Przywiedli oslice i osle, i wlozyli na nie szaty swoje, i wsadzili go na nie.
\par 8 A wielki lud slal szaty swoje na drodze, a drudzy obcinali galazki z drzew, i slali na drodze.
\par 9 A lud wprzód i pozad idacy wolal, mówiac: Hosanna synowi Dawidowemu! blogoslawiony, który idzie w imieniu Panskiem, Hosanna na wysokosciach!
\par 10 A gdy on wjechal do Jeruzalemu, wzruszylo sie wszystko miasto, mówiac: Któz ten jest?
\par 11 A lud mówil: Tenci jest Jezus, on prorok z Nazaretu Galilejskiego.
\par 12 Tedy wszedl Jezus do kosciola Bozego, i wygnal wszystkie sprzedawajace i kupujace w kosciele, a stoly tych, co pieniedzmi handlowali, i stolki sprzedawajacych golebie poprzewracal,
\par 13 I rzekl im: Napisano: Dom mój dom modlitwy nazwany bedzie; alescie wy uczynili z niego jaskinie zbójców.
\par 14 Tedy przystapili do niego slepi i chromi w kosciele, i uzdrowil je.
\par 15 A obaczywszy przedniejsi kaplani i nauczeni w Pismie cuda, które czynil, i dzieci wolajace w kosciele, i mówiace: Hosanna synowi Dawidowemu: rozgniewali sie.
\par 16 I rzekli mu: Slyszyszze, co ci mówia? A Jezus im rzekl: I owszem. Nigdysciez nie czytali, iz z ust niemowlatek i ssacych wykonales chwale?
\par 17 A opusciwszy je, wyszedl z miasta do Betanii, i tam zostal;
\par 18 A rano wracajac sie do miasta, laknal.
\par 19 I ujrzawszy jedno figowe drzewo przy drodze, przyszedl do niego, i nie znalazl nic na niem, tylko same liscie, i rzekl mu: Niechaj sie wiecej z ciebie owoc nie rodzi na wieki. I uschlo zarazem one figowe drzewo.
\par 20 A ujrzawszy to uczniowie, dziwowali sie, mówiac: Jakoc predko uschlo to figowe drzewo!
\par 21 Tedy odpowiadajac Jezus, rzekl im: Zaprawde powiadam wam: Jezlibyscie mieli wiare, a nie watpilibyscie, nie tylko to, co sie stalo z figowem drzewem, uczynicie, ale gdybyscie i tej górze rzekli: Podnies sie, a rzuc sie w morze, stanie sie.
\par 22 I wszystko, o cobysciekolwiek prosili w modlitwie wierzac, wezmiecie.
\par 23 A gdy on przyszedl do kosciola, przystapili do niego, gdy uczyl, przedniejsi kaplani i starsi ludu, mówiac: Któraz moca to czynisz? a kto ci dal te moc?
\par 24 A odpowiadajac Jezus, rzekl im: Spytam i ja was o jedne rzecz, która jezli mi powiecie, i ja wam powiem, która moca to czynie.
\par 25 Chrzest Jana skad byl? z nieba czyli z ludzi? A oni mysleli sami w sobie, mówiac: Jezli powiemy z nieba, rzecze nam: Czemuzescie mu tedy nie uwierzyli?
\par 26 Jezli zas powiemy z ludzi, boimy sie ludu; bo Jana wszyscy maja za proroka.
\par 27 A odpowiadajac Jezusowi rzekli: Nie wiemy. Rzekl im i on: I ja wam nie powiem, która moca to czynie.
\par 28 Ale cóz sie wam zda? Czlowiek niektóry mial dwóch synów; a przystapiwszy do pierwszego, rzekl: Synu! idz, rób dzis na winnicy mojej.
\par 29 Ale on odpowiadajac rzekl: Nie chce, a potem obaczywszy sie, poszedl.
\par 30 A przystapiwszy do drugiego, rzekl takze; a on odpowiadajac rzekl: Ja ide, panie! ale nie szedl.
\par 31 Któryz z tych dwóch uczynil wole ojcowska? Rzekli mu: On pierwszy. Rzekl im Jezus: Zaprawde powiadam wam, ze was celnicy i wszetecznice uprzedzaja do królestwa Bozego.
\par 32 Albowiem przyszedl do was Jan droga sprawiedliwosci, a nie uwierzyliscie mu, ale celnicy i wszetecznice uwierzyli mu: a wy widzac to, przeciez sie nie obaczyliscie, abyscie mu uwierzyli.
\par 33 Drugiego podobienstwa sluchajcie: Czlowiek niektóry byl gospodarzem, który nasadzil winnice, i plotem ja ogrodzil, i wkopal w niej prase, i zbudowal wieze, i najal ja winiarzom, i odjechal precz.
\par 34 A gdy sie przyblizyl czas odbierania pozytków, poslal slugi swoje do onych winiarzy, aby odebrali pozytki jej.
\par 35 Ale winiarze pojmawszy slugi jego, jednego ubili, a drugiego zabili, a drugiego ukamionowali.
\par 36 Zasie poslal inszych slug, wiecej niz pierwszych; i takze im uczynili.
\par 37 Ale na ostatek poslal syna swego, mówiac: Beda sie wstydzic syna mego.
\par 38 Lecz winiarze, ujrzawszy onego syna, rzekli miedzy soba: Tenci jest dziedzic; pójdzcie, zabijmy go, a otrzymamy dziedzictwo jego.
\par 39 Tedy porwawszy go, wyrzucili go precz z winnicy i zabili.
\par 40 Gdy tedy pan winnicy przyjdzie, cóz uczyni onym winiarzom?
\par 41 Rzekli mu: Zle, zle potraci, a winnice najmie inszym winiarzom, którzy mu oddawac beda pozytki czasów swoich.
\par 42 Rzekl im Jezus: Nie czytalisciez nigdy w Pismach: Kamien, który odrzucili budujacy, ten sie stal glowa wegielna: od Panac sie to stalo, i dziwne jest w oczach naszych?
\par 43 Przetoz powiadam wam: Iz od was odjete bedzie królestwo Boze, i bedzie dane narodowi czyniacemu pozytki jego.
\par 44 A kto by padl na ten kamien, roztraci sie, a na kogo by on upadl, zetrze go.
\par 45 A uslyszawszy przedniejsi kaplani i Faryzeuszowie podobienstwa jego, domyslili sie, iz o nich mówil;
\par 46 A chcac go pojmac, bali sie ludu, poniewaz go mieli za proroka.

\chapter{22}

\par 1 A odpowiadajac Jezus, zasie im rzekl w podobienstwach, mówiac:
\par 2 Podobne jest królestwo niebieskie czlowiekowi królowi, który sprawil wesele synowi swemu;
\par 3 I poslal slugi swe, aby wezwali zaproszonych na wesele; ale nie chcieli przyjsc.
\par 4 Znowu poslal insze slugi, mówiac: Powiedzcie zaproszonym: Otom obiad mój nagotowal, woly moje i co bylo karmnego, pobito, i wszystko gotowe, pójdzciez na wesele.
\par 5 Ale oni zaniedbawszy odeszli, jeden do roli swojej, a drugi do kupiectwa swego;
\par 6 A drudzy pojmawszy slugi jego, zelzyli i pobili je.
\par 7 Co gdy król uslyszal, rozgniewal sie, a poslawszy wojska swoje, wytracil one morderce, i miasto ich zapalil.
\par 8 Tedy rzekl slugom swoim: Weselec wprawdzie jest gotowe; lecz zaproszeni nie byli godni.
\par 9 Przetoz idzcie na rozstania dróg, kogokolwiek znajdziecie, wezwijcie na wesele.
\par 10 Tedy wyszedlszy oni sludzy na drogi, zgromadzili wszystkie, którekolwiek znalezli, zle i dobre, i napelnione jest wesele goscmi.
\par 11 A wszedlszy król, aby ogladal goscie, obaczyl tam czlowieka nie odzianego szata weselna;
\par 12 I rzekl mu: Przyjacielu! jakos tu wszedl, nie majac szaty weselnej? A on zamilknal.
\par 13 Tedy rzekl król slugom: Zwiazawszy nogi i rece jego, wezmijcie go, a wrzuccie do ciemnosci zewnetrznych, tam bedzie placz i zgrzytanie zebów.
\par 14 Albowiem wiele jest wezwanych, ale malo wybranych.
\par 15 Tedy odszedlszy Faryzeuszowie uczynili rade, jako by go usidlili w mowie.
\par 16 I poslali do niego ucznie swoje z Herodyjany, mówiac: Nauczycielu! wiemy, zes jest prawdziwy, i drogi Bozej w prawdzie uczysz, a nie dbasz na nikogo; albowiem nie patrzysz na osobe ludzka.
\par 17 Przetoz powiedz nam, co ci sie zda? Godzili sie dac czynsz cesarzowi, czyli nie?
\par 18 Ale Jezus poznawszy zlosc ich, rzekl im: Czemuz mie kusicie, obludnicy?
\par 19 Pokazcie mi monete czynszowa; a oni mu podali grosz.
\par 20 I rzekl im: Czyjze to obraz i napis?
\par 21 Rzekli mu: Cesarski. Tedy im rzekl: Oddawajciez tedy, co jest cesarskiego, cesarzowi, a co jest Bozego, Bogu.
\par 22 To uslyszawszy, zadziwili sie, a opusciwszy go, odeszli.
\par 23 Dnia onego przyszli do niego Saduceuszowie, którzy mówia, iz nie masz zmartwychwstania, i pytali go,
\par 24 Mówiac: Nauczycielu! Mojzesz powiedzial: Jezliby kto umarl, nie majac dzieci, aby brat jego prawem powinowactwa pojal zone jego, i wzbudzil nasienie bratu swemu.
\par 25 Bylo tedy u nas siedm braci; a pierwszy pojawszy zone, umarl, a nie majac nasienia, zostawil zone swoje bratu swemu.
\par 26 Takze tez wtóry i trzeci, az do siódmego.
\par 27 A na ostatek po wszystkich umarla i ona niewiasta.
\par 28 Przetoz przy zmartwychwstaniu, któregoz z tych siedmiu bedzie zona, gdyz ja wszyscy mieli?
\par 29 A odpowiadajac Jezus rzekl im: Bladzicie, nie bedac powiadomieni Pisma, ani mocy Bozej.
\par 30 Albowiem przy zmartwychwstaniu ani sie zenic, ani za maz chodzic nie beda, ale beda jako Aniolowie Bozy w niebie.
\par 31 A o powstaniu umarlych nie czytaliscie, co wam powiedziano od Boga mówiacego:
\par 32 Jam jest Bóg Abrahama, i Bóg Izaaka, i Bóg Jakóba? Bóg nie jestci Bogiem umarlych, ale zywych.
\par 33 A uslyszawszy to lud, zdumial sie nad nauka jego.
\par 34 Lecz gdy uslyszeli Faryzeuszowie, ze zawarl usta Saduceuszom, zeszli sie wespól.
\par 35 I spytal go jeden z nich, zakonnik, kuszac go i mówiac:
\par 36 Nauczycielu! które jest najwieksze przykazanie w zakonie?
\par 37 A Jezus mu rzekl: Bedziesz milowal Pana, Boga twego, ze wszystkiego serca twego, i ze wszystkiej duszy twojej i ze wszystkiej mysli twojej.
\par 38 To jest pierwsze i najwieksze przykazanie.
\par 39 A wtóre podobne jest temuz: Bedziesz milowal blizniego twego, jako samego siebie.
\par 40 Na tych dwóch przykazaniach wszystek zakon i prorocy zawisneli.
\par 41 A gdy sie Faryzeuszowie zebrali, spytal ich Jezus,
\par 42 Mówiac: Co sie wam zda o Chrystusie? Czyim jest synem? Rzekli mu: Dawidowym.
\par 43 I rzekl im: Jakoz tedy Dawid w duchu nazywa go Panem? mówiac:
\par 44 Rzekl Pan Panu memu: Siadz po prawicy mojej, az poloze nieprzyjacioly twoje podnózkiem nóg twoich.
\par 45 Poniewaz go tedy Dawid nazywa Panem, jakoz jest synem jego?
\par 46 A zaden mu nie mógl odpowiedziec i slowa, i nie smial go nikt wiecej od onego dnia pytac.

\chapter{23}

\par 1 Tedy Jezus rzekl do ludu i do uczniów swoich, mówiac:
\par 2 Na stolicy Mojzeszowej usiedli nauczeni w Pismie i Faryzeuszowie.
\par 3 Przetoz wszystkiego, czegokolwiek by wam rozkazali przestrzegac, przestrzegajcie i czyncie, ale wedlug uczynków ich nie czyncie; albowiem oni mówia, ale nie czynia.
\par 4 Bo wiaza brzemiona ciezkie i nieznosne, i klada je na ramiona ludzkie, lecz palcem swoim nie chca ich ruszyc.
\par 5 A wszystkie uczynki swoje czynia, aby byli widziani od ludzi, i rozszerzaja bramy swoje, i rozpuszczaja podolki plaszczów swoich.
\par 6 Nadto miluja pierwsze miejsca na wieczerzach, i pierwsze stolki w bóznicach.
\par 7 I pozdrawiania na rynkach, i aby je nazywali ludzie: Mistrzu, mistrzu!
\par 8 Ale wy nie nazywajcie sie mistrzami; albowiem jeden jest mistrz wasz, Chrystus; ale wy jestescie wszyscy bracmi.
\par 9 I nikogo nie zówcie ojcem waszym na ziemi; albowiem jeden jest Ojciec wasz, który jest w niebiesiech.
\par 10 A niechaj was nie zowia mistrzami, gdyz jeden jest mistrz wasz, Chrystus.
\par 11 Ale kto z was najwiekszy jest, bedzie sluga waszym.
\par 12 A kto by sie wywyzszal, bedzie ponizony; a kto by sie ponizal, bedzie wywyzszony.
\par 13 Lecz biada wam, nauczeni w Pismie i Faryzeuszowie obludni! iz zamykacie królestwo niebieskie przed ludzmi; albowiem tam sami nie wchodzicie, ani tym, którzy by wnijsc chcieli, wchodzic nie dopuszczacie.
\par 14 Biada wam, nauczeni w Pismie i Faryzeuszowie obludni! iz pozeracie domy wdów, a to pod pokrywka dlugich modlitw, dlatego ciezszy sad odniesiecie.
\par 15 Biada wam, nauczeni w Pismie i Faryzeuszowie obludni! iz obchodzicie morze i ziemie, abyscie uczynili jednego nowego Zyda; a gdy sie stanie, czynicie go synem zatracenia, dwakroc wiecej nizeliscie sami.
\par 16 Biada wam, wodzowie slepi! którzy powiadacie: kto by przysiagl na kosciól, nic nie jest: ale kto by przysiagl na zloto koscielne, winien jest.
\par 17 Glupi i slepi! albowiem cóz jest wiekszego, zloto czy kosciól, który poswieca zloto?
\par 18 A kto by przysiagl na oltarz, nic nie jest; lecz kto by przysiagl na dar, który jest na nim, winien jest.
\par 19 Glupi i slepi! albowiem cóz wiekszego jest? dar, czyli oltarz, który poswieca dar?
\par 20 Kto tedy przysiega na oltarz, przysiega na niego, i na to wszystko, co na nim jest;
\par 21 A kto przysiega na kosciól, przysiega na niego, i na tego, który w nim mieszka;
\par 22 I kto przysiega na niebo, przysiega na stolice Boza, i na tego, który siedzi na niej.
\par 23 Biada wam, nauczeni w Pismie i Faryzeuszowie obludni! iz dawacie dziesiecine z mietki i z anyzu i z kminu, a opuszczacie powazniejsze rzeczy w zakonie, sad i milosierdzie i wiare; te rzeczy mieliscie czynic, a onych nie opuszczac.
\par 24 Wodzowie slepi! którzy przecedzacie komara, i wielblada polykacie.
\par 25 Biada wam, nauczeni w Pismie i Faryzeuszowie obludni! iz oczyszczacie kubek z wierzchu i mise, a wewnatrz pelne sa drapiestwa i zbytku.
\par 26 Faryzeuszu slepy! oczysc pierwej to, co jest wewnatrz w kubku i w misie, aby i to, co jest z wierzchu, czystem bylo.
\par 27 Biada wam, nauczeni w Pismie i Faryzeuszowie obludni! izescie podobni grobom pobielanym, które sie zdadza z wierzchu byc cudne, ale wewnatrz pelne sa kosci umarlych i wszelakiej nieczystosci.
\par 28 Takze i wy z wierzchu zdacie sie byc ludziom sprawiedliwi; ale wewnatrz jestescie pelni obludy i nieprawosci.
\par 29 Biada wam, nauczeni w Pismie i Faryzeuszowie obludni! iz budujecie groby proroków, i zdobicie nagrobki sprawiedliwych,
\par 30 I mówicie: Bysmy byli za dni ojców naszych, nie bylibysmy uczestnikami ich we krwi proroków.
\par 31 A tak swiadczycie sami przeciwko sobie, ze jestescie synowie tych, którzy proroki pozabijali.
\par 32 I wy tez dopelniacie miary ojców waszych.
\par 33 Wezowie! rodzaju jaszczurczy! i jakoz bedziecie mogli ujsc przed sadem ognia piekielnego?
\par 34 Przetoz oto ja posylam do was proroki, i medrce, i nauczone w Pismie, a z nich niektóre zabijecie i ukrzyzujecie, a niektóre z nich ubiczujecie w bóznicach waszych, i bedziecie je przesladowac od miasta do miasta;
\par 35 Aby przyszla na was wszystka krew sprawiedliwa, wylana na ziemi, ode krwi Abla sprawiedliwego az do krwi Zacharyjasza, syna Barachyjaszowego, któregoscie zabili miedzy kosciolem i oltarzem.
\par 36 Zaprawde powiadam wam: Przyjdzie to wszystko na ten naród.
\par 37 Jeruzalem! Jeruzalem! które zabijasz proroki, i które kamionujesz te, którzy do ciebie byli posylani: ilekroc chcialem zgromadzic dzieci twoje, tak jako zgromadza kokosz kurczeta swoje pod skrzydla, a nie chcieliscie?
\par 38 Oto wam dom wasz pusty zostanie.
\par 39 Albowiem powiadam wam, ze mie nie ujrzycie od tego czasu, az rzeczecie: Blogoslawiony, który idzie w imieniu Panskiem.

\chapter{24}

\par 1 A wyszedlszy Jezus z kosciola szedl; i przystapili uczniowie jego, aby mu ukazali budowanie koscielne.
\par 2 I rzekl im Jezus: Izaz nie widzicie tego wszystkiego? Zaprawde powiadam wam, nie zostanie tu kamien na kamieniu, który by nie byl rozwalony.
\par 3 A gdy siedzial na górze oliwnej, przystapili do niego uczniowie osobno, mówiac: Powiedz nam, kiedy sie to stanie, i co za znak przyjscia twego i dokonania swiata?
\par 4 I odpowiadajac Jezus, rzekl im: Patrzcie, aby was kto nie zwiódl.
\par 5 Albowiem wiele ich przyjdzie pod imieniem mojem, mówiac: Jam jest Chrystus; i wiele ich zwioda.
\par 6 I uslyszycie wojny i wiesci o wojnach; patrzciez, abyscie soba nie trwozyli; albowiem musi to wszystko byc, ale jeszcze nie tu jest koniec.
\par 7 Albowiem powstanie naród przeciwko narodowi, i królestwo przeciwko królestwu, i beda glody i mory i trzesienia ziemi miejscami.
\par 8 Ale to wszystko jest poczatkiem bolesci.
\par 9 Tedy was podadza w udreczenie, i beda was zabijac, i bedziecie w nienawisci u wszystkich narodów dla imienia mego.
\par 10 A tedy wiele sie ich zgorszy, a jedni drugich wydadza, i jedni drugich nienawidziec beda.
\par 11 I wiele falszywych proroków powstanie, i zwioda wielu.
\par 12 A iz sie rozmnozy nieprawosc, oziebnie milosc wielu.
\par 13 Ale kto wytrwa az do konca, ten zbawion bedzie.
\par 14 I bedzie kazana ta Ewangielija królestwa po wszystkim swiecie, na swiadectwo wszystkim narodom. A tedyc przyjdzie koniec.
\par 15 Przetoz gdy ujrzycie obrzydliwosc spustoszenia, opowiedziana przez Danijela proroka, stojaca na miejscu swietem, (kto czyta, niechaj uwaza),
\par 16 Tedy ci, co beda w ziemi Judzkiej, niech uciekaja na góry;
\par 17 A kto na dachu, niechaj nie zstepuje, aby co wzial z domu swego;
\par 18 A kto na roli, niech sie nazad nie wraca, aby wzial szaty swe.
\par 19 A biada brzemiennym i piersiami karmiacym w one dni!
\par 20 Przetoz módlcie sie, aby nie bylo uciekanie wasze w zimie, albo w sabat.
\par 21 Albowiem naonczas bedzie wielki ucisk, jaki nie byl od poczatku swiata az dotad, ani potem bedzie.
\par 22 A gdyby nie byly skrócone one dni, nie byloby zbawione zadne cialo: ale dla wybranych beda skrócone one dni.
\par 23 Tedy jezliby wam kto rzekl: Oto tu jest Chrystus, albo tam, nie wierzcie.
\par 24 Albowiem powstana falszywi Chrystusowie, i falszywi prorocy, i czynic beda znamiona wielkie i cuda, tak izby zwiedli (by mozna) i wybrane.
\par 25 Otom wam przepowiedzial.
\par 26 Jezliby wam tedy rzekli: Oto na puszczy jest, nie wychodzcie; oto w komorach, nie wierzcie.
\par 27 Albowiem jako blyskawica wychodzi od wschodu slonca i ukazuje sie az na zachód, tak bedzie i przyjscie Syna czlowieczego.
\par 28 Bo gdziekolwiek bedzie scierw, tam sie zgromadza i orly.
\par 29 A zaraz po utrapieniu onych dni slonce sie zacmi, a ksiezyc nie da jasnosci swojej, i gwiazdy beda padac z nieba, i mocy niebieskie porusza sie.
\par 30 Tedyc sie ukaze znamie Syna czlowieczego na niebie, a tedy beda narzekac wszystkie pokolenia ziemi, i ujrza Syna czlowieczego, przychodzacego na oblokach niebieskich, z moca i z chwala wielka;
\par 31 I posle Anioly swoje z traba glosu wielkiego, i zgromadza wybrane jego od czterech wiatrów, od krajów niebios az do krajów ich.
\par 32 A od drzewa figowego nauczcie sie tego podobienstwa: Gdy sie juz Gala? jego odmladza i liscie wypuszcza, poznajecie, iz blisko jest lato.
\par 33 Takze i wy, gdy ujrzycie to wszystko, poznawajcie, iz blisko jest, a we drzwiach.
\par 34 Zaprawde powiadam wam, ze nie przeminie ten wiek, azby sie to wszystko stalo.
\par 35 Niebo i ziemia przemina, ale slowa moje nie przemina.
\par 36 A o onym dniu i godzinie nikt nie wie, ani Aniolowie niebiescy, tylko sam Ojciec mój.
\par 37 Ale jako bylo za dni Noego, tak bedzie i przyjscie Syna czlowieczego.
\par 38 Albowiem jako za dni onych przed potopem jedli, i pili, i ozeniali sie i za maz wydawali, az do onego dnia, którego wszedl Noe do korabia,
\par 39 I nie spostrzegli sie, az przyszedl potop i zabral wszystkie: tak bedzie i przyjscie Syna czlowieczego.
\par 40 Tedy beda dwaj na roli; jeden bedzie wziety, a drugi zostawiony;
\par 41 Dwie beda mlec we mlynie; jedna bedzie wzieta, a druga zostawiona;
\par 42 Czujciez tedy, poniewaz nie wiecie, której godziny Pan wasz przyjdzie.
\par 43 A to wiedzcie, ze, gdyby wiedzial gospodarz, o której strazy zlodziej ma przyjsc, wzdyby czul, i nie dalby podkopac domu swego.
\par 44 Przetoz i wy badzcie gotowi; bo której sie godziny nie spodziejecie, Syn czlowieczy przyjdzie.
\par 45 Któryz tedy jest sluga wierny i roztropny, którego postanowil pan jego nad czeladzia swoja, aby im dawal pokarm na czas sluszny?
\par 46 Blogoslawiony on sluga, którego by, przyszedlszy Pan jego, znalazl tak czyniacego;
\par 47 Zaprawde powiadam wam, ze go nad wszystkiemi dobrami swemi postanowi.
\par 48 A jezliby rzekl on zly sluga w sercu swojem: Odwlacza pan mój z przyjsciem swojem;
\par 49 I poczalby bic spólslugi, a jesc i pic z pijanicami:
\par 50 Przyjdzie pan slugi onego, dnia, którego sie nie spodzieje, i godziny, której nie wie;
\par 51 I odlaczy go, a czesc jego polozy z obludnikami; tam bedzie placz i zgrzytanie zebów.

\chapter{25}

\par 1 Tedy podobne bedzie królestwo niebieskie dziesieciu pannom, które wziawszy lampy swoje, wyszly przeciwko oblubiencowi.
\par 2 A bylo z nich piec madrych, a piec glupich.
\par 3 One glupie wziawszy lampy swoje, nie wziely oleju z soba.
\par 4 Lecz madre wziely oleju w naczynia swoje z lampami swemi.
\par 5 A gdy oblubieniec odwlaczal z przyjsciem, zdrzemaly sie wszystkie i posnely.
\par 6 A o pólnocy stal sie krzyk: Oto oblubieniec idzie; wynijdzcie przeciwko niemu!
\par 7 Tedy wstaly one wszystkie panny i ochedozyly lampy swoje.
\par 8 Ale glupie rzekly do madrych: Dajcie nam z oleju waszego, boc lampy nasze gasna.
\par 9 I odpowiedzialy one madre, mówiac: Nie damy, by snac nam i wam nie dostalo; idzcie raczej do sprzedawajacych, a kupcie sobie.
\par 10 A gdy odeszly kupowac, przyszedl oblubieniec; a te, które byly gotowe, weszly z nim na wesele; i zamknione sa drzwi.
\par 11 Lecz potem przyszly i one drugie panny, mówiac: Panie, Panie, otwórz nam!
\par 12 A on odpowiadajac, rzekl: Zaprawde powiadam wam, nie znam was.
\par 13 Czujciez tedy; bo nie wiecie dnia ani godziny, której Syn czlowieczy przyjdzie.
\par 14 Albowiem jako czlowiek precz odjezdzajacy zwolal slug swoich i oddal im dobra swoje;
\par 15 I dal jednemu piec talentów, a drugiemu dwa, a drugiemu jeden, kazdemu wedlug przemozenia jego, i zaraz precz odjechal.
\par 16 A poszedlszy on, który wzial piec talentów, robil niemi, i zyskal drugie piec talentów.
\par 17 Takze i on, który wzial dwa, zyskal i ten drugie dwa.
\par 18 Ale ten, który wzial jeden, odszedlszy wykopal dól w ziemi, i skryl pieniadze pana swego.
\par 19 A po dlugim czasie przyszedl pan onych slug, i rachowal sie z nimi.
\par 20 Tedy przystapiwszy on, który byl wzial piec talentów, przyniósl drugie piec talentów, mówiac: Panie! oddales mi piec talentów, otom drugie piec talentów zyskal niemi.
\par 21 I rzekl mu pan jego: To dobrze, slugo dobry i wierny! nad malem byles wiernym, nad wielem cie postanowie; wnijdz do radosci pana twego.
\par 22 A przystapiwszy i on, który byl dwa talenty wzial, rzekl: Panie! oddales mi dwa talenty, otom drugie dwa talenty zyskal niemi.
\par 23 Rzekl mu pan jego: To dobrze, slugo dobry i wierny! gdyzes byl wierny nad malem, nad wielem cie postanowie; wnijdz do radosci pana twego.
\par 24 A przystapiwszy i ten, który byl wzial jeden talent, rzekl: Panie! wiedzialem, ze jestes czlowiek srogi, który zniesz, gdzies nie rozsiewal, i zbierasz, gdzies nie rozsypywal;
\par 25 Bojac sie tedy, szedlem i skrylem talent twój w ziemie; oto masz, co twego jest.
\par 26 A odpowiadajac pan jego, rzekl mu: Slugo zly i gnusny! wiedziales, iz zne, gdziem nie rozsiewal, i zbieram, gdziem nie rozsypywal;
\par 27 Przetozes mial pieniadze moje dac tym, co pieniedzmi handluja, a ja przyszedlszy, wzialbym byl, co jest mojego, z lichwa.
\par 28 Przetoz wezmijcie od niego ten talent, a dajcie temu, który ma dziesiec talentów.
\par 29 (Albowiem kazdemu, który ma, bedzie dano, i obfitowac bedzie; a od tego, który nie ma, i to, co ma, bedzie od niego odjeto.)
\par 30 A niepozytecznego sluge wrzuccie do onych ciemnosci zewnetrznych, tam bedzie placz i zgrzytanie zebów.
\par 31 A gdy przyjdzie Syn czlowieczy w chwale swojej, i wszyscy swieci Aniolowie z nim, tedy usiadzie na stolicy chwaly swojej,
\par 32 I beda zgromadzone przed niego wszystkie narody, i odlaczy je, jedne od drugich, jako pasterz odlacza owce od kozlów.
\par 33 A postawi owce zaiste po prawicy swojej, a kozly po lewicy.
\par 34 Tedy rzecze król tym, którzy beda po prawicy jego: Pójdzcie, blogoslawieni Ojca mego! odziedziczcie królestwo wam zgotowane od zalozenia swiata.
\par 35 Albowiem laknalem, a daliscie mi jesc; pragnalem, a daliscie mi pic; bylem gosciem, a przyjeliscie mie;
\par 36 Bylem nagim, a przyodzialiscie mie; bylem chorym, a nawiedziliscie mie; bylem w wiezieniu, a przychodziliscie do mnie.
\par 37 Tedy mu odpowiedza sprawiedliwi, mówiac: Panie! kiedyzesmy cie widzieli laknacym, a nakarmilismy cie? albo pragnacym, a napoilismy cie?
\par 38 I kiedysmy cie widzieli gosciem, a przyjelismy cie? albo nagim, a przyodzialismy cie?
\par 39 Albo kiedysmy cie widzieli chorym, albo w wiezieniu, a przychodzilismy do ciebie?
\par 40 A odpowiadajac król, rzecze im: Zaprawde powiadam wam, cokolwiekiescie uczynili jednemu z tych braci moich najmniejszych, mniescie uczynili.
\par 41 Potem rzecze i tym, którzy beda po lewicy: Idzcie ode mnie, przekleci! w ogien wieczny, który zgotowany jest dyjablu i Aniolom jego.
\par 42 Albowiem laknalem, a nie daliscie mi jesc; pragnalem, a nie daliscie mi pic;
\par 43 Bylem gosciem, a nie przyjeliscie mie; nagim, a nie przyodzialiscie mie; chorym i w wiezieniu, a nie nawiedziliscie mie.
\par 44 Tedy mu odpowiedza i oni, mówiac: Panie! kiedysmy cie widzieli laknacym, albo pragnacym, albo gosciem, albo nagim, albo chorym, albo w wiezieniu, a nie sluzylismy tobie?
\par 45 Tedy im odpowie, mówiac: Zaprawde powiadam wam, czegosciekolwiek nie uczynili jednemu z tych najmniejszych, i mniescie nie uczynili.
\par 46 I pójda ci na meki wieczne; ale sprawiedliwi do zywota wiecznego.

\chapter{26}

\par 1 I stalo sie, gdy dokonczyl Jezus tych wszystkich mów, rzekl do uczniów swoich:
\par 2 Wiecie, iz po dwóch dniach bedzie wielkanoc, a Syn czlowieczy bedzie wydany, aby byl ukrzyzowany.
\par 3 Tedy sie zebrali przedniejsi kaplani i nauczeni w Pismie i starsi ludu do dworu najwyzszego kaplana, którego zwano Kaifasz;
\par 4 I naradzali sie, jakoby Jezusa zdrada pojmali i zabili;
\par 5 Lecz mówili: Nie w swieto, aby nie byl rozruch miedzy ludem.
\par 6 A gdy Jezus byl w Betanii, w domu Szymona tredowatego,
\par 7 Przystapila do niego niewiasta, majaca sloik alabastrowy masci bardzo kosztownej, i wylala ja na glowe jego, gdy siedzial u stolu.
\par 8 Co widzac uczniowie jego, rozgniewali sie, mówiac: I na cóz ta utrata?
\par 9 Albowiem mogla byc ta masc drogo sprzedana, i moglo sie to dac ubogim.
\par 10 Co gdy poznal Jezus, rzekl im: Przecz sie przykrzycie tej niewiescie? Dobry zaprawde uczynek uczynila przeciwko mnie.
\par 11 Albowiem ubogie zawsze macie z soba, ale mnie nie zawsze miec bedziecie.
\par 12 Bo ona wylawszy te masc na cialo moje, uczynila to, gotujac mie ku pogrzebowi.
\par 13 Zaprawde powiadam wam: Gdziekolwiek bedzie kazana ta Ewangielija po wszystkim swiecie, i to bedzie powiadano, co ona uczynila, na pamiatke jej.
\par 14 Tedy odszedlszy jeden ze dwunastu, którego zwano Judaszem Iszkaryjotem, do przedniejszych kaplanów,
\par 15 Rzekl im: Co mi chcecie dac, a ja go wam wydam? A oni mu odwazyli trzydziesci srebrników.
\par 16 A odtad szukal czasu sposobnego, aby go wydal.
\par 17 A pierwszego dnia przasników przystapili uczniowie do Jezusa, mówiac mu: Gdziez chcesz, zec nagotujemy, abys jadl baranka?
\par 18 A on rzekl: Idzcie do miasta, do niektórego czlowieka, a rzeczcie mu: Kazalci nauczyciel powiedziec: Czas mój blisko jest, u ciebie jesc bede baranka z uczniami moimi.
\par 19 I uczynili uczniowie, jako im rozkazal Jezus, i nagotowali baranka.
\par 20 A gdy byl wieczór, usiadl za stolem ze dwunastoma.
\par 21 A gdy jedli, rzekl: Zaprawde powiadam wam, iz jeden z was wyda mie.
\par 22 I zasmuciwszy sie bardzo, poczeli mówic do niego kazdy z nich: Azazem ja jest Panie?
\par 23 A on odpowiadajac, rzekl: Który macza ze mna reke w misie, ten mie wyda.
\par 24 Synci czlowieczy idzie, jako napisano o nim; ale biada czlowiekowi temu, przez którego Syn czlowieczy wydany bywa! dobrze by mu bylo, by sie byl nie narodzil ten czlowiek.
\par 25 A odpowiadajac Judasz, który go wydawal, rzekl: Izalim ja jest, Mistrzu? Mówi mu: Tys powiedzial.
\par 26 A gdy oni jedli, wziawszy Jezus chleb, a poblogoslawiwszy lamal, i dal uczniom i rzekl: Bierzcie, jedzcie, to jest cialo moje.
\par 27 A wziawszy kielich i podziekowawszy, dal im, mówiac: Pijcie z tego wszyscy;
\par 28 Albowiem to jest krew moja nowego testamentu, która sie za wielu wylewa na odpuszczenie grzechów.
\par 29 Ale powiadam wam, iz nie bede pil odtad z tego rodzaju winnej macicy, az do dnia onego, gdy go bede pil z wami nowy w królestwie Ojca mego.
\par 30 I zaspiewawszy piesn, wyszli na góre oliwna.
\par 31 Tedy im rzekl Jezus: Wy wszyscy zgorszycie sie ze mnie tej nocy; albowiem napisano: Uderze pasterza, i beda rozproszone owce trzody.
\par 32 Lecz gdy ja zmartwychwstane, poprzedze was do Galilei.
\par 33 A odpowiadajac Piotr, rzekl mu: Chocby sie wszyscy zgorszyli z ciebie, ja sie nigdy nie zgorsze.
\par 34 Rzekl mu Jezus: Zaprawde powiadam ci, iz tej nocy, pierwej niz kur zapieje, trzykroc sie mnie zaprzesz.
\par 35 Rzekl mu Piotr: Chocbym z toba mial i umrzec, nie zapre sie ciebie. Takze i wszyscy uczniowie mówili.
\par 36 Tedy przyszedl Jezus z nimi na miejsce, które zwano Gietsemane, i rzekl uczniom: Siadzciez tu, az odszedlszy, bede sie tam modlil.
\par 37 A wziawszy z soba Piotra i dwóch synów Zebedeuszowych, poczal sie smecic i tesknic.
\par 38 Tedy im rzekl Jezus: Smetna jest dusza moja az do smierci; zostanciez tu, a czujcie ze mna.
\par 39 A postapiwszy troche, padl na oblicze swoje, modlac sie i mówiac: Ojcze mój, jezli mozna, niech mie ten kielich minie; a wszakze nie jako ja chce, ale jako ty.
\par 40 Tedy przyszedl do uczniów, i znalazl je spiace, i rzekl Piotrowi: Takzescie nie mogli przez jedne godzine czuc ze mna?
\par 41 Czujciez, a módlcie sie, abyscie nie weszli w pokuszenie; duchci jest ochotny, ale cialo mdle.
\par 42 Zasie po wtóre odszedlszy, modlil sie, mówiac: Ojcze mój, jezli mie nie moze ten kielich minac, tylko abym go pil, niech sie stanie wola twoja.
\par 43 A przyszedlszy, znalazl je zasie spiace; albowiem oczy ich byly obciazone.
\par 44 A zaniechawszy ich, znowu odszedl i modlil sie po trzecie, tez slowa mówiac.
\par 45 Tedy przyszedl do uczniów swoich i rzekl im: Spijciez juz i odpoczywajcie; oto sie przyblizyla godzina, a Syn czlowieczy bedzie wydany w rece grzeszników.
\par 46 Wstancie, pójdzmy! oto sie przyblizyl ten, który mie wydaje.
\par 47 A gdy on jeszcze mówil, oto Judasz, jeden z dwunastu, przyszedl, a z nim wielka zgraja z mieczami i z kijami, od przedniejszych kaplanów i starszych ludu;
\par 48 Ale ten, który go wydawal, dal byl im znak, mówiac: Któregokolwiek pocaluje, tenci jest; imajciez go.
\par 49 A wnet przystapiwszy do Jezusa, rzekl: Badz pozdrowiony, Mistrzu! i pocalowal go.
\par 50 Ale mu rzekl Jezus: Przyjacielu! na cos przyszedl? Tedy przystapiwszy, rzucili rece na Jezusa i pojmali go.
\par 51 A oto jeden z tych, którzy byli z Jezusem, wyciagnal reke, i dobyl miecza swego, a uderzywszy sluge kaplana najwyzszego, ucial mu ucho.
\par 52 Tedy mu rzekl Jezus: Obróc miecz swój na miejsce jego; albowiem wszyscy, którzy miecz biora, od miecza pogina.
\par 53 Azaz mniemasz, ze bym nie mógl teraz prosic Ojca mego, a stawilby mi wiecej niz dwanascie wojsk Aniolów?
\par 54 Ale jakozby sie wypelnily Pisma, które mówia, iz sie tak musi stac?
\par 55 Onejze godziny mówil Jezus do onej zgrai: Wyszliscie jako na zbójce z mieczami i z kijmi, pojmac mie; na kazdy dzien siadalem u was, uczac w kosciele, a nie pojmaliscie mie.
\par 56 Alec sie to wszystko stalo, aby sie wypelnily Pisma prorockie. Tedy uczniowie jego wszyscy opusciwszy go, uciekli.
\par 57 A oni pojmawszy Jezusa, wiedli go do Kaifasza, najwyzszego kaplana, gdzie sie byli zebrali nauczeni w Pismie i starsi.
\par 58 Ale Piotr szedl za nim z daleka az do dworu najwyzszego kaplana; a wszedlszy tam, siedzial z slugami, aby ujrzal koniec.
\par 59 Ale przedniejsi kaplani i starsi, i wszelka rada szukali falszywego swiadectwa przeciwko Jezusowi, aby go na smierc wydali.
\par 60 Ale nie znalezli; i choc wiele falszywych swiadków przychodzilo, przecie nie znalezli. A na ostatek wystapiwszy dwaj falszywi swiadkowie,
\par 61 Rzekli: Ten mówil: Moge rozwalic kosciól Bozy, a za trzy dni zbudowac go.
\par 62 A wstawszy najwyzszy kaplan, rzekl mu: Nic nie odpowiadasz? Cóz to jest, co ci przeciwko tobie swiadcza?
\par 63 Lecz Jezus milczal. A odpowiadajac najwyzszy kaplan rzekl: Poprzysiegam cie przez Boga zywego, abys nam powiedzial, jezlis ty jest Chrystus, on Syn Bozy?
\par 64 Rzekl mu Jezus: Tys powiedzial; wszakze powiadam wam: Odtad ujrzycie Syna czlowieczego siedzacego na prawicy mocy Bozej, i przychodzacego na oblokach niebieskich.
\par 65 Tedy najwyzszy kaplan rozdarl szaty swoje, mówiac: Bluznil! Cóz jeszcze potrzebujemy swiadków? Otoscie teraz slyszeli bluznierstwo jego.
\par 66 Cóz sie wam zda? A oni odpowiadajac, rzekli: Winien jest smierci.
\par 67 Tedy plwali na oblicze jego, i piesciami go bili, a drudzy go policzkowali,
\par 68 Mówiac: Prorokuj nam, Chrystusie! kto jest ten, co cie uderzyl?
\par 69 Ale Piotr siedzial przed domem na podwórzu. Tedy przystapila do niego jedna dziewka, mówiac: I tys byl z tym Jezusem Galilejskim.
\par 70 A on sie zaprzal przed wszystkimi, mówiac: Nie wiem, co powiadasz.
\par 71 A gdy on wychodzil do przysionka, ujrzala go insza dziewka, i rzekla do tych, co tam byli: I tenci byl z tym Jezusem Nazarenskim.
\par 72 Tedy po wtóre zaprzal sie z przysiega, mówiac: Nie znam tego czlowieka.
\par 73 A przystapiwszy po malej chwilce ci, co tam stali, rzekli Piotrowi: Prawdziwie i tys jest z nich; bo i mowa twoja ciebie wydaje.
\par 74 Tedy sie poczal przeklinac i przysiegac, mówiac: Nie znam tego czlowieka; a zarazem kur zapial.
\par 75 I wspomnial Piotr na slowa Jezusowe, który mu byl powiedzial: Pierwej niz kur zapieje, trzykroc sie mnie zaprzesz; a wyszedlszy precz, gorzko plakal.

\chapter{27}

\par 1 A gdy bylo rano, weszli w rade wszyscy przedniejsi kaplani i starsi ludu przeciwko Jezusowi, aby go zabili;
\par 2 I zwiazawszy go, wiedli i podali Ponckiemu Pilatowi, staroscie.
\par 3 Tedy Judasz, który go byl wydal, widzac, iz byl osadzony, zalujac tego, wrócil trzydziesci srebrników, przedniejszym kaplanom i starszym ludu.
\par 4 Mówiac: Zgrzeszylem, wydawszy krew niewinna! A oni rzekli: Cóz nam do tego? ty ujrzysz!
\par 5 A porzuciwszy one srebrniki w kosciele, odszedl, a odszedlszy powiesil sie.
\par 6 Ale przedniejsi kaplani wziawszy one srebrniki, mówili: Nie godzi sie ich klasc do skarbu koscielnego, gdyz zaplata jest krwi.
\par 7 I naradziwszy sie, kupili za nie role garncarzowa na pogrzeb gosciom.
\par 8 Dlatego ona rola nazwana jest rola krwi, az do dnia dzisiejszego.
\par 9 Tedy sie wypelnilo, co powiedziano przez Jeremijasza proroka, mówiacego: I wzieli trzydziesci srebrników, zaplate oszacowanego, który byl oszacowany od synów Izraelskich;
\par 10 I dali je za role garncarzowa, jako mi postanowil Pan.
\par 11 A Jezus stal przed starosta; i pytal go starosta, mówiac: Tyzes jest on król zydowski? A Jezus mu rzekl: Ty powiadasz.
\par 12 A gdy nan skarzyli przedniejsi kaplani i starsi, nic nie odpowiedzial.
\par 13 Tedy mu rzekl Pilat: Nie slyszyszze jako wiele przeciwko tobie swiadcza?
\par 14 Lecz mu nie odpowiedzial i na jedno slowo, tak iz sie starosta bardzo dziwowal.
\par 15 Ale na swieto zwykl byl starosta wypuszczac ludowi jednego wieznia, którego by chcieli.
\par 16 I mieli natenczas wieznia znacznego, którego zwano Barabbasz.
\par 17 A gdy sie zebrali, rzekl do nich Pilat: Któregoz chcecie, abym wam wypuscil? Barabbasza, czyli Jezusa, którego zowia Chrystusem?
\par 18 Bo wiedzial, iz go z nienawisci wydali.
\par 19 A gdy on siedzial na sadowej stolicy, poslala do niego zona jego, mówiac: Nie miej zadnej sprawy z tym sprawiedliwym; bom wiele ucierpiala dzis we snie dla niego.
\par 20 Ale przedniejsi kaplani i starsi namówili lud, aby prosili o Barabbasza, a Jezusa, aby stracili.
\par 21 A odpowiadajac starosta, rzekl im: Którego chcecie, abym wam z tych dwóch wypuscil? a oni odpowiedzieli: Barabbasza.
\par 22 Rzekl im Pilat: Cóz tedy uczynie z Jezusem, którego zowia Chrystusem? Rzekli mu wszyscy: Niech bedzie ukrzyzowany.
\par 23 A starosta rzekl: Cóz wzdy zlego uczynil? Ale oni tem bardziej wolali, mówiac: Niech bedzie ukrzyzowany!
\par 24 A widzac Pilat, iz to nic nie pomagalo, ale owszem sie wiekszy rozruch wszczynal, wziawszy wode, umyl rece przed ludem, mówiac: Nie jestem ja winien krwi tego sprawiedliwego; wy ujrzycie.
\par 25 A odpowiadajac wszystek lud, rzekl: Krew jego na nas i na dziatki nasze.
\par 26 Tedy im wypuscil Barabbasza; ale Jezusa ubiczowawszy, wydal go, aby byl ukrzyzowany.
\par 27 Tedy zolnierze staroscini przywiódlszy Jezusa na ratusz, zebrali do niego wszystke rote;
\par 28 A zewleklszy go, przyodziali go plaszczem szarlatowym;
\par 29 I uplótlszy korone z ciernia, wlozyli na glowe jego, i dali trzcine w prawa reke jego, a upadajac przed nim na kolana, nasmiewali sie z niego, mówiac: Badz pozdrowiony, królu zydowski!
\par 30 A plujac na niego, wzieli one trzcine, i bili go w glowe jego.
\par 31 A gdy sie z niego nasmiali, zewlekli go z onego plaszcza, i oblekli go w szate jego, i wiedli go, aby byl ukrzyzowany.
\par 32 A wychodzac znalezli czlowieka Cyrenejczyka, imieniem Szymon; tego przymusili, aby niósl krzyz jego.
\par 33 A przyszedlszy na miejsce rzeczone Golgota, które zowia miejscem trupich glów,
\par 34 Dali mu pic ocet z zólcia zmieszany; a skosztowawszy, nie chcial pic.
\par 35 A ukrzyzowawszy go, rozdzielili szaty jego, i miotali los, aby sie wypelnilo, co powiedziano przez proroka: Rozdzielili sobie szaty moje, a o odzienie moje los miotali.
\par 36 A siedzac, strzegli go tam.
\par 37 I przybili nad glowa jego wine jego napisana: Ten jest Jezus, król zydowski.
\par 38 Byli tez ukrzyzowani z nim dwaj zbójcy, jeden po prawicy, a drugi po lewicy.
\par 39 A ci, którzy mimo chodzili, bluznili go, chwiejac glowami swojemi,
\par 40 I mówiac: Ty, co rozwalasz kosciól, a w trzech dniach budujesz go, ratuj samego siebie; jezlis jest Syn Bozy, zstap z krzyza.
\par 41 Takze i przedniejsi kaplani z nauczonymi w Pismie, i z starszymi, nasmiewajac sie, mówili:
\par 42 Inszych ratowal, a samego siebie ratowac nie moze; jezliz jest król Izraelski, niech teraz zstapi z krzyza, a uwierzymy mu.
\par 43 Dufal w Bogu, niechze go teraz wybawi, jezli sie w nim kocha; boc powiedzial: Jestem Synem Bozym.
\par 44 Takze tez i zbójcy, którzy byli z nim ukrzyzowani, uragali mu.
\par 45 A od szóstej godziny stala sie ciemnosc po wszystkiej ziemi az do dziewiatej godziny.
\par 46 A okolo dziewiatej godziny zawolal Jezus glosem wielkim, mówiac: Eli, Eli, Lama Sabachtani! to jest, Boze mój! Boze mój! czemus mie opuscil?
\par 47 Tedy niektórzy z tych, co tam stali, uslyszawszy to, mówili: Elijasza ten wola.
\par 48 A zarazem biezawszy jeden z nich, wzial gabke, i napelnil ja octem, a wlozywszy na trzcine, dal mu pic.
\par 49 A drudzy mówili: Zaniechaj; patrzajmy, jezli przyjdzie Elijasz, aby go wybawil.
\par 50 Ale Jezus zawolawszy po wtóre glosem wielkim, oddal ducha.
\par 51 A oto zaslona koscielna rozerwala sie na dwoje od wierzchu az do dolu, i trzesla sie ziemia, a skaly sie rozpadaly.
\par 52 I groby sie otwieraly, a wiele cial swietych, którzy byli zasneli, powstalo:
\par 53 A wyszedlszy z grobów po zmartwychwstaniu jego, weszli do miasta swietego, i pokazali sie wielom.
\par 54 Tedy setnik i ci, co z nim Jezusa strzegli, widzac trzesienie ziemi, i to, co sie dzialo, zlekli sie bardzo, mówiac: Prawdziwiec ten byl Synem Bozym.
\par 55 A bylo tam wiele niewiast z daleka sie przypatrujacych, które byly przyszly za Jezusem od Galilei, poslugujac mu;
\par 56 Miedzy któremi byla Maryja Magdalena, i Maryja, matka Jakóbowa i Jozesowa, i matka synów Zebedeuszowych.
\par 57 A gdy byl wieczór, przyszedl czlowiek bogaty z Arymatyi, imieniem Józef, który tez byl uczniem Jezusowym.
\par 58 Ten przyszedlszy do Pilata, prosil o cialo Jezusowe.Tedy Pilat rozkazal, aby mu bylo ono cialo oddane;
\par 59 A Józef wziawszy ono cialo, uwinal je w czyste przescieradlo;
\par 60 I polozyl je w nowym grobie swoim, który byl w opoce wykowal; a przywaliwszy do drzwi grobowych kamien wielki, odszedl.
\par 61 A byla tam Maryja Magdalena, i druga Maryja, które siedzialy przeciwko grobowi.
\par 62 A drugiego dnia, który byl pierwszy po przygotowaniu, zgromadzili sie przedniejsi kaplani i Faryzeuszowie do Pilata.
\par 63 Mówiac: Panie! wspomnielismy, iz on zwodziciel powiedzial, gdy jeszcze zyw byl: Po trzech dniach zmartwychwstane.
\par 64 Rozkaz tedy obwarowac grób az do dnia trzeciego, by snac przyszedlszy uczniowie jego w nocy, nie ukradli go, i nie powiedzieli ludowi, iz powstal od umarlych; i bedzie posledni blad gorszy niz pierwszy.
\par 65 Rzekl im Pilat: Macie straz, idzciez, obwarujcie, jako umiecie.
\par 66 A oni poszedlszy, osadzili grób straza, zapieczetowawszy kamien.

\chapter{28}

\par 1 A gdy sie skonczyl sabat, i juz switalo na pierwszy dzien onego tygodnia, przyszla Maryja Magdalena i druga Maryja, aby grób ogladaly.
\par 2 A oto stalo sie wielkie trzesienie ziemi; albowiem Aniol Panski zstapiwszy z nieba, przystapil i odwalil kamien ode drzwi, i usiadl na nim.
\par 3 A bylo wejrzenie jego jako blyskawica, a szata jego biala jako snieg.
\par 4 A ci, którzy strzegli grobu, drzeli, bojac sie go, i stali sie jako umarli.
\par 5 Ale Aniol odpowiadajac, rzekl do niewiast: Nie bójcie sie wy; boc wiem, iz Jezusa ukrzyzowanego szukacie.
\par 6 Nie maszci go tu; albowiem powstal, jako powiedzial; chodzcie, ogladajcie miejsce, gdzie lezal Pan.
\par 7 A predko idac, powiedzcie uczniom jego, ze zmartwychwstal; a oto uprzedza was do Galilei, tam go ujrzycie; otom wam powiedzial.
\par 8 Tedy wyszedlszy predko od grobu z bojaznia i z radoscia wielka, biezaly, aby to opowiedzialy uczniom jego.
\par 9 A gdy szly, aby to opowiedzialy uczniom jego, oto Jezus spotkal sie z niemi, mówiac: Badzcie pozdrowione. A one przystapiwszy, uchwycily sie nóg jego i poklonily mu sie.
\par 10 Tedy im rzekl Jezus: Nie bójcie sie; idzcie, opowiedzcie braciom moim, aby poszli do Galilei, a tam mie ujrza.
\par 11 A gdy one poszly, oto niektórzy z strazy przyszedlszy do miasta, oznajmili przedniejszym kaplanom wszystko, co sie stalo.
\par 12 Którzy zgromadziwszy sie z starszymi, i naradziwszy sie, dali niemalo pieniedzy zolnierzom,
\par 13 Mówiac: Powiadajcie, iz uczniowie jego w nocy przyszedlszy, ukradli go, gdysmy spali.
\par 14 A jezliby sie to do starosty donioslo, my go namówimy, a was bezpiecznymi uczynimy.
\par 15 A oni wziawszy pieniadze, uczynili, jako ich nauczono. I rozniosla sie ta powiesc miedzy Zydy az do dnia dzisiejszego.
\par 16 Lecz jedenascie uczniów poszli do Galilei na góre, gdzie im byl naznaczyl Jezus.
\par 17 Ale ujrzawszy go, poklonili mu sie; lecz niektórzy watpili.
\par 18 Ale Jezus przystapiwszy, mówil do nich, a rzekl: Dana mi jest wszelka moc na niebie i na ziemi.
\par 19 Idac tedy, nauczajcie wszystkie narody, chrzczac je w imie Ojca, i Syna, i Ducha Swietego;
\par 20 Uczac je przestrzegac wszystkiego, com wam przykazal. A oto Jam jest z wami po wszystkie dni, az do skonczenia swiata. Amen.


\end{document}