\begin{document}

\title{Koheleta}


\chapter{1}

\par 1 Slowa kaznodziei, syna Dawidowego, króla w Jeruzalemie.
\par 2 Marnosc nad marnosciami, powiedzial kaznodzieja; marnosc nad marnosciami, i wszystko marnosc.
\par 3 Cóz za pozytek ma czlowiek ze wszystkiej pracy swej, która prowadzi pod sloncem?
\par 4 Jeden rodzaj przemija, a drugi rodzaj nastaje; lecz ziemia na wieki stoi.
\par 5 Slonce wschodzi i slonce zachodzi, a spieszy sie do miejsca swego, kedy wschodzi;
\par 6 Idzie na poludnie, a obraca sie na pólnocy; wiatr ustawicznie krazac idzie, a po okregach swoich wraca sie wiatr.
\par 7 Wszystkie rzeki ida do morza, wszakze morze nie wylewa; do miejsca, z którego rzeki plyna, wracaja sie, aby zas stamtad wychodzily.
\par 8 Wszystkie rzeczy sa pelne zabaw, a czlowiek nie moze ich wymówic; oko nie nasyci sie widzeniem, a ucho nie napelni sie slyszeniem.
\par 9 Co bylo, jest to, co byc ma; a co sie teraz dzieje, jest to, co sie dziac bedzie, a niemasz nic nowego pod sloncem.
\par 10 Jestze jaka rzecz, o którejby kto rzec mógl: Wej! to cos nowego? I toc juz bylo za onych wieków, które byly przed nami.
\par 11 Niemasz pamiatki pierwszych rzeczy; takze tez i potomnych, które beda, nie bedzie pamiatki u tych, którzy potem nastana.
\par 12 Ja kaznodzieja bylem królem Izraelskim w Jeruzalemie;
\par 13 I przylozylem do tego serce swe, abym szukal, i doszedl madroscia swoja wszystkiego, co sie dzieje pod niebem.(Te zabawe trudna dal Bóg synom ludzkim, aby sie nia trapili.)
\par 14 Widzialem wszystkie sprawy, które sie dzieja pod sloncem, a oto wszystko jest marnoscia i utrapieniem ducha.
\par 15 Co jest krzywego, nie moze byc wyprostowane, a niedostatki nie moga byc policzone.
\par 16 Przetoz takiem myslal w sercu swem, mówiac: Otom ja uwielbil i rozszerzyl madrosc nad wszystkich, którzy byli przedemna w Jeruzalemie, a serce moje dostapilo wielkiej madrosci i umiejetnosci.
\par 17 I przylozylem do tego serce moje, abym poznal madrosc i umiejetnosc, szalenstwo i glupstwo; alem doznal, iz to jest utrapieniem ducha.
\par 18 Bo gdzie wiele madrosci, tam jest wiele gniewu; a kto przyczynia umiejetnosci, przyczynia bolesci.

\chapter{2}

\par 1 Rzeklem ja do serca mego: Nuze teraz doswiadcze cie w weselu, uzywajze dobrych rzeczy; ale i toc marnosc.
\par 2 Smiechowi rzeklem: Szalejesz, a weselu: Cóz to czynisz?
\par 3 Przemyslalem w sercu swem, abym pozwolil wina cialu memu, (serce jednak swoje sprawujac madroscia) i abym sie trzymal glupstwa dotad, azbym obaczyl, coby lepszego bylo synom ludzkim czynic pod niebem, przez wszystkie dni zywota ich.
\par 4 Wielkiem sprawy wykonal; pobudowalem sobie domy, nasadzilem sobie winnic;
\par 5 Naczynilem sobie ogrodów, i sadów, i naszczepilem w nich drzew wszelakiego owocu;
\par 6 Pobudowalem sobie stawy ku odwilzaniu przez nie lasu, w którym rosnie drzewo;
\par 7 Nabylem sobie slug i dziewek, i mialem czeladz w domu moim; do tego i stada wolów, i wielkie trzody owiec mialem nad wszystkich, którzy byli przedemna w Jeruzalemie.
\par 8 Zgromadzilem tez sobie srebro i zloto, i klejnoty od królów i krain. Sporzadzilem tez sobie spiewaków i spiewaczki, i inne rozkosze synów ludzkich, i muzyczne naczynia rozliczne.
\par 9 A tak stalem sie wielkim i mozniejszym nad wszystkich, którzy byli przede mna w Jeruzalemie; nadto madrosc moja zostawala przy mnie.
\par 10 A wszystkiego, czego pozadaly oczy moje, nie zabranialem im, anim odmawial sercu memu zadnego wesela; ale serce moje weselilo sie ze wszystkiej pracy mojej. A toc byl dzial mój ze wszystkiej pracy mojej.
\par 11 Lecz gdym sie obejrzal na wszystkie sprawy swoje, które czynily rece moje, i na prace, którem podejmowal pracujac: oto wszystko marnosc, i utrapienie ducha, i niemasz nic pozytecznego pod sloncem.
\par 12 Przetoz obrócilem sie do tego, abym sie przypatrywal madrosci, i szalenstwu, i glupstwu; (bo cózby czlowiek czynil ten, który nastanie po królu? to, co juz inni czynili.)
\par 13 I obaczylem, iz jest pozyteczniejsza madrosc nizeli glupstwo, tak jako jest pozyteczniejsza swiatlosc, nizeli ciemnosc.
\par 14 Madry ma oczy w glowie swej, ale glupi w ciemnosciach chodzi; a wszakzem poznal, ze jednakie przygody na wszystkich przychdza.
\par 15 Dlategom rzekl w sercu mojem: Mali mi sie tak dziac, jako sie glupiemu dzieje, przeczzem go ja tedy madroscia przeszedl? Przetozem rzekl w sercu mojem: I toc jest marnosc.
\par 16 Albowiem nie na wieki bedzie pamiatki madrego i glupiego, dlatego, iz to, co teraz jest, we dni przyszle wszystkiego zapomna; a jako umiera madry, tak i glupi.
\par 17 Przetoz mi zywot omierzl; bo mi sie nie podoba zadna rzecz, która sie dzieje pod sloncem; albowiem wszystkie sa marnoscia, i utrapieniem ducha.
\par 18 Nawet omierzla mi i wszystka praca moja, któram podejmowal pod sloncem, przeto, ze ja zostawic musze czlowiekowi, który nastanie po mnie.
\par 19 A kto wie, bedzieli madrym, czyli glupim? a wszakze bedzie panowal nad wszystka praca moja, któram prowadzil, i w którejm byl madry pod sloncem. Alec i to marnosc.
\par 20 I przypadlem na to, abym zwatpil w sercu mojem o wszystkiej pracy, któram sie madrze bawil pod sloncem.
\par 21 Nie jeden zaiste czlowiek pracuje madrze, i umiejetnie, i sprawiedliwie; a wszakze to innemu, który nie robil na to, za dzial zostawi. I toc marnosc i wielka bieda.
\par 22 Bo cóz ma czlowiek ze wszystkiej pracy swej, i z usilowania serca swego, które podejmuje pod sloncem?
\par 23 Poniewaz wszystkie dni jego sa bolesne, a zabawa jego jest frasunek, tak iz i w nocy nie odpoczywa serce jego. I toc jest marnosc.
\par 24 Izali nie lepsza czlowiekowi, aby jadl i pil, i dobrze uczynil duszy swojej z pracy swojej? alemci widzial, ze i to z reki Bozej pochodzi.
\par 25 Albowiem któzby sluszniej mial jesc, i pozywac tego nad mie?
\par 26 Bo czlowiekowi, który mu sie podoba, daje madrosc, umiejetnosc, i wesele; ale grzesznikowi daje frasunek, aby zbieral i zgromadzal, coby zostawil temu, który sie podoba Bogu. I toc jest marnosc, a utrapienie ducha.

\chapter{3}

\par 1 Kazda rzecz ma swój czas, i kazde przedsiewziecie ma swój czas pod niebem.
\par 2 Jest czas rodzenia i czas umierania; czas sadzenia, i czas wycinania tego, co sadzono;
\par 3 Czas zabijania, i czas leczenia; czas rozwalania, i czas budowania;
\par 4 Czas placzu, i czas smiechu; czas smutku, i czas skakania;
\par 5 Czas rozrzucania kamieni, i czas zbierania kamieni; czas oblapiania, i czas oddalenia sie od oblapiania;
\par 6 Czas szukania, i czas stracenia; czas chowania, i czas odrzucenia;
\par 7 Czas rozdzierania, i czas zszywania; czas milczenia, i czas mówienia;
\par 8 Czas milowania, i czas nienawidzenia; czas wojny, i czas pokoju.
\par 9 Cóz tedy ma ten, co pracuje, z tego, okolo czego pracuje?
\par 10 Widzialem prace, która dal Bóg synom ludzkim, aby sie nia bawili.
\par 11 Wszystko dobrze czyni czasu swego; owszem i zadosc swiata dal do serca ich, choc czlowiek dziela tego, które Bóg sprawuje, ani poczatku, ani konca nie dochodzi.
\par 12 Stad wiem, ze nic lepszego nie maja, jedno aby sie weselili, a czynili dobrze za zywota swego.
\par 13 Acz i to, gdy kazdy czlowiek je i pije, i uzywa dobrze wszystkiej pracy swojej, jest dar Bozy.
\par 14 Wiem i to, ze cokolwiek Bóg czyni, trwa na wieki; i ze sie do tego nic nie moze przydac, ani z tego co ujac; a czyni to Bóg, aby sie bali oblicza jego.
\par 15 To, co bylo, teraz jest, a co bedzie, juz bylo; albowiem Bóg odnawia to, co przeminelo.
\par 16 Nadtom jeszcze widzial pod sloncem na miejscu sadu niepoboznosc, a na miejscu sprawiedliwosci niesprawiedliwosc.
\par 17 I rzeklem w sercu swem: Sprawiedliwego i niezboznego Bóg sadzic bedzie; bo czas kazdemu przedsiewzieciu i kazdej sprawy tam bedzie.
\par 18 Nadto rzeklem w sercu swem o sprawie synów ludzkich, ze im Bóg okazal, aby wiedzieli, ze sa podobni bydlu.
\par 19 Bo przypadek synów ludzkich, i przypadek bydla, jest przypadek jednaki. Jako umiera ono, tak umiera i ten, i ducha jednakiego wszyscy maja, a nie ma czlowiek nic wiecej nad bydle; bo wszystko jest marnosc.
\par 20 Wszystko to idzie na jedno miejsce; a wszystko jest z prochu, i wszystko sie zas w proch obraca.
\par 21 A któz wie, ze duch synów ludzkich wstepuje w góre? a duch bydlecy, ze zstepuje pod ziemie?
\par 22 Przetoz obaczylem, ze czlowiekowi niemasz nic lepszego, jedno weselic sie z pracy swej, gdyz to jest dzial jego; albowiem któz go do tego przywiedzie, aby poznal to, co ma byc po nim?

\chapter{4}

\par 1 Potemem sie obrócil i ujrzalem wszystkie uciski, które sie dzieja pod sloncem, a oto widzialem lzy ucisnionych, którzy nie maja pocieszyciela, ani mocy, aby uszli rak tych, którzy ich ciemieza; a nie maja, mówie, pocieszyciela.
\par 2 Dlategom ja umarlych, którzy juz zeszli, wiecej chwalil, nizeli zywych, którzy jeszcze az dotad zyja.
\par 3 Owszem szczesliwy jest nad tych obydwóch ten, który jeszcze nie byl, który nie widzial nic zlego, które sie dzieje pod sloncem.
\par 4 Bom widzial, ze wszelaka praca i kazde dzielo dobre jest ku zazdrosci jednych drugim. I toc jest marnosc i utrapienie ducha.
\par 5 Glupi sklada rece swe, a je cialo swoje.
\par 6 Lepsza jest pelna garsc z pokojem, nizeli obie garsci pelne z praca i z udreczeniem ducha.
\par 7 Znowu obróciwszy sie ujrzalem druga marnosc pod sloncem.
\par 8 Jest kto samotny, niemajac zadnego, ani syna, ani brata, a wzdy niemasz konca wszelakiej pracy jego, ani oczy jego moga sie nasycic bogactwem. Nie mysli: Komuz ja pracuje, tak ze i zywotowi swemu ujmuje dobrego. I toc jest marnosc, i ciezkie udrec zenie.
\par 9 Lepiej jest we dwóch byc, niz jednemu; maja zaiste dobry pozytek z pracy swojej.
\par 10 Bo jezli jeden upadnie, drugi podzwignie towarzysza swego. A tak biada samotnemu, gdyby upadl! bo nie ma drugiego, coby go podzwignal.
\par 11 Takze bedali dwaj spolu lezec, zagrzeja sie; ale jeden jakoz sie zagrzeje?
\par 12 Owszem jezliby kto jednego przemagal, dwaj mu sie zastawia; a sznur troisty nie lacno sie zerwie.
\par 13 Lepszy jest chlopiec ubogi a madry, nizeli król stary a glupi, który juz nie umie przyjmowac napominania.
\par 14 Bo ów z wiezienia wychodzi, aby królowal, a ten i w królestwie swojem zubozec moze.
\par 15 Widzialem wszystkich zyjacych, którzy chodza pod sloncem, ze przestawali z chlopieciem, potomkiem onego, który mial nastapic na królestwo po nim.
\par 16 Nie bylo konca niestatkowi wszystkiego ludu, którykolwiek byl przed nimi; nie bedac sie potomkowie cieszyc z niego. A tak i to jest marnosc, i utrapienie ducha.
\par 17 Strzez nogi twojej, gdy idziesz do domu Bozego, a badz sklonniejszym ku sluchaniu, nizeli ku dawaniu ofiar ludzi glupich; boc oni nie wiedza, ze zle czynia.

\chapter{5}

\par 1 Nie bywaj porywczy do mówienia, ani serce twoje predkie na wymówienie slowa przed obliczem Bozem, albowiem Bóg jest na niebie, a ty na ziemi; przeto niech slów twoich malo bedzie.
\par 2 Bo jako sen przychodzi z wielkiej pracy, tak glos glupiego z wielu slów.
\par 3 Gdy co Bogu poslubisz, nie omieszkiwaj tego oddac, boc mu sie glupi nie podobaja; cokolwiek poslubisz, oddaj.
\par 4 Lepiej jest nie slubowac, nizeli poslubiwszy co, nie oddac.
\par 5 Nie dopuszczaj ustom twoim, aby do grzechu przywodzily cialo twoje, ani mów przed aniolem, ze to jest blad. Przeczze masz Boga gniewac mowa swa, któryby wniwecz obrócil sprawe rak twoich?
\par 6 Bo gdzie jest wiele snów, tam i marnosci i slów wiele; ale sie ty Boga bój.
\par 7 Jezli ucisk ubogiego, i zatrzymanie sadu i sprawiedliwosci ujrzysz w której krainie, nie dziwuj sie temu; bo wyzszy wysokiego upatruje, a jeszcze wyzsi sa nad nimi.
\par 8 Zabawa kolo ziemi ma pierwsze miejsce u wszystkich; i król roli sluzy.
\par 9 Kto miluje pieniadze, nie nasyci sie pieniedzy, a kto miluje bogactwa, nie bedzie mial pozytku. I toc jest marnosc.
\par 10 Gdzie wiele majetnosci, wiele bywa tych, co ja jedza. Cóz tedy za pozytek Panu z tego? jedno ze na nie patrzy oczyma swemi.
\par 11 Slodki jest sen pracowitemu, chociaz malo, chociaz wiele jadl; ale nasycenie bogatego spac mu nie dopusci.
\par 12 Jest ciezka bieda, któram widzial pod sloncem; bogactwa zachowane na zle pana swego.
\par 13 Bo takowe bogactwo zla sprawa gina, a syn, którego splodzi, nie bedzie mial nic w rekach swych.
\par 14 Jako nagi wyszedl z zywota matki swojej, tak sie wraca, jako byl przyszedl, a nie odnosi nic z pracy swojej, coby mial wziac w reke swoje.
\par 15 A tak i toc jest ciezka bieda, ze jako przyszedl, tak odejdzie. Cóz tedy za pozytek, ze na wiatr pracowal?
\par 16 Dotego, ze po wszystkie dni swoje w ciemnosci jadal z wielkim klopotem, z bolescia i z gniewem.
\par 17 Toc jest, com ja obaczyl, ze dobra i osobliwa rzecz jest, jesc i pic, i uzywac dobrego ze wszystkiej pracy swej, która czlowiek podejmuje pod sloncem po wszystkie dni zywota swego, które mu dal Bóg; albowiem to jest dzial jego.
\par 18 A któremukolwiek czlowiekowi dal Bóg majetnosc i bogactwo, i dal mu w moc, aby ich uzywal, i odbieral dzial swój, a weselil sie z pracy swojej: to jest dar Bozy.
\par 19 Bo nie bedzie wiele pamietal na dni zywota swego; przeto, ze mu Bóg zyczy wesela serca jego.

\chapter{6}

\par 1 Jest zle, którem widzial pod sloncem, a jest ludziom zwyczajne.
\par 2 Gdy któremu czlowiekowi Bóg dal bogactwa, i majetnosc, i slawe, tak ze na niczem nie schodzi duszy jego, czegokolwiek zada, jednak nie daje mu Bóg mocy pozywac tego: ale obcy czlowiek pozera je. Toc jest marnosc i bieda ciezka.
\par 3 Jezli kto splodzil sto synów, a zylby wiele lat, i przedluzylyby sie dni lat jego, a jezliby dusza jego nie byla nasycona dobrem, a nie mialby ani pogrzebu: powiadam, ze lepszy jest martwy plód, nizeli on.
\par 4 Bo ten prózno przyszedlszy do ciemnosci odchodzi, a ciemnosciami imie jego okryte bywa.
\par 5 Owszem, slonca nie widzial, i nic nie poznaje; a tak odpocznienie lepsze ma, nizeli ów.
\par 6 A chocby tez zyl przez dwa tysiace lat, a dobregoby nie uzyl, azaz do jednego miejsca wszyscy nie ida?
\par 7 Wszystka praca czlowiecza jest dla geby jego, a wszakze dusza jego nie moze sie nasycic.
\par 8 Albowiem co ma wiecej madry nad glupiego? albo co ma wiecej ubogi, który sobie umie poczynac miedzy ludzmi?
\par 9 Lepiej jest co oczyma widziec, nizeli tego zadac; alec i to marnosc i utrapienie ducha.
\par 10 Czemkolwiek kto jest, juz tak nazwano imie jego; i wiadomo bylo, ze czlowiekiem byc mial, i ze sie nie moze sadzic z mocniejszym nad sie.
\par 11 Poniewaz tedy wiele rzeczy jest, które rozmnazaja marnosc, cóz z nich za pozytek ma czlowiek?
\par 12 Albowiem któz wie, co jest dobrego czlowiekowi w tym zywocie po wszystkie dni zywota marnosci jego, które jako cien pomijaja? Albo kto oznajmi czlowiekowi, co po nim bedzie pod sloncem?

\chapter{7}

\par 1 Lepsze jest imie dobre, nizeli masc wyborna; a dzien smierci, niz dzien narodzenia.
\par 2 Lepiej isc do domu zaloby, niz isc do domu biesiady, przeto, iz tam widzimy koniec kazdego czlowieka, a zyjacy sklada to do serca swego.
\par 3 Lepszy jest smutek, nizeli smiech; bo przez smutek twarzy naprawia sie serce.
\par 4 Serce madrych w domu zaloby; ale serce glupich w domu wesela.
\par 5 Lepiej jest sluchac gromienia madrego, nizeli sluchac piesni glupich.
\par 6 Bo jaki jest trzask ciernia pod garncem, tak jest smiech glupiego; i toc jest marnosc.
\par 7 Zaiste ucisk przywodzi madrego do szalenstwa, a dar zaslepia serce.
\par 8 Lepsze jest dokonczenie rzeczy, nizeli poczatek jej; lepszy jest czlowiek cierpliwego ducha, niz ducha wynioslego.
\par 9 Nie badz porywczy w duchu twym do gniewu; bo gniew w zanadrzyu glupich odpoczywa.
\par 10 Nie mów: Cóz to jest, ze dni pierwsze lepsze byly, niz terazniejsze? Bobys sie o tem nie madrze pytal.
\par 11 Dobra jest madrosc przy majetnosci, i jest pozteczna tym, którzy widza slonce.
\par 12 Albowiem pod cieniem madrosci, i pod cieniem srebra odpoczywa czlowiek, a wszakze przedniejsza jest umiejetnosc madrosci; bo przynosi zywot tym, którzy ja maja.
\par 13 Przypatrz sie sprawie Bozej; bo któz moze wyprostowac, co on skrzywi?
\par 14 W dzien dobry zazywaj dobra, w dzien zly miej sie na pieczy: boc ten uczynil Bóg przeciwko owemu, dlatego, aby nie doszedl czlowiek tego, co nastanie po nim.
\par 15 Tom wszystko widzial za dni marnosci mojej: Bywa sprawiedliwy, który ginie z sprawiedliwoscia swoja; takze bywa niezboznik, który dlugo zyje we zlosci swojej.
\par 16 Nie badz nazbyt sprawiedliwym, ani nazbyt madrym; przeczzebys mial do zguby przychodzic?
\par 17 Nie badz nader niepoboznym, ani nazbyt glupim; przeczzebys mial umrzec przed czasem swoim?
\par 18 Dobra jest, abys sie owego trzymal, a tego sie nie puszczal; kto sie boi Boga, uchodzi tego wszystkiego.
\par 19 Madrosc umacnia madrego wiecej, nizeli dziesiec ksiazat, którzy sa w miescie.
\par 20 Zaiste niemasz czlowieka sprawiedliwego na ziemi, któryby czynil dobrze, a nie grzeszyl.
\par 21 Nie do wszystkich tez slów, które mówia ludzie przykladaj serca twego; i niech cie to nie obchodzi, chocciby i sluga twój zlorzeczyl.
\par 22 Boc wie serce twoje, zes i ty czestokroc drugim zlorzeczyl.
\par 23 Wszystkiegom tego doswiadczyl madroscia, i rzeklem: Bede madrym; alec sie madrosc oddalila odemnie.
\par 24 A co dalekiego, i co bardzo glebokiego jest, któz to znajdzie?
\par 25 Wszystkom ja przeszedl mysla swoja, abym poznal i wybadal sie, i wynalazl madrosc i rozum, a zebym poznal niezboznosc, glupstwo, i blad, i szalenstwo.
\par 26 I znalazlem rzecz gorzciejsza nad smierc, to jest, taka niewiaste, której serce jest jako sieci i sidlo, a rece jej jako peta. Kto sie Bogu podoba, wolny bedzie od niej; ale grzesznik bedzie od niej pojmany.
\par 27 Otom to znalazl, (mówi kaznodzieja,)stosujac jedno z drugiem, abym doszedl umiejetnosci.
\par 28 Czego zas nad to szukala dusza moja, tedym nie znalazl. Meza jednego z tysiaca znalazlem; alem niewiasty miedzy temi wszystkiemi nie znalazl.
\par 29 To tylko obacz, com znalazl, ze stworzyl Bóg czlowieka dobrego; ale oni udali sie za rozmaitemi myslami.
\par 30 Któz moze z madrym porównac? a kto moze wylozyc kazda rzecz?

\chapter{8}

\par 1 Madrosc czlowieka oswieca oblicze jego, a hardosc twarzy jego odmienia.
\par 2 Jac radze, abys wyroku królewskiego przestrzegal a wszakze wedlug przysiegi Bozej.
\par 3 Nie skwapiaj sie odejsc od oblicza jego, ani trwaj w uporze; albowiem cobykolwiek chcial, uczynilciby.
\par 4 Bo gdzie slowo królewskie, tam i moc jego: a któz mu rzecze: Co czynisz?
\par 5 Kto strzeze przykazania, nie uzna nic zlego; i czas i przyczyny zna serce madrego.
\par 6 Albowiem wszelki zamysl ma czas i przyczyny; alec wielka bieda trzyma sie czlowieka,
\par 7 Ze nie wie, co ma byc; bo kiedy sie co stanie, któz mu oznajmi?
\par 8 Niemasz czlowieka, coby mial moc nad zywotem, zeby zahamowal dusze, ani ma mocy nade dniem smierci; ani ma, czemby sie bronil w tym boju, ani wyswobodzi niezboznego niepoboznosc.
\par 9 Tom wszystko widzial, gdym przylozyl serce swoje do tego wszystkiego, co sie pod sloncem dzieje; widzialem ten czas, którego panuje czlowiek nad czlowiekiem na jego zle.
\par 10 Tedym widzial niezboznych pogrzebionych, ze sie zas nawrócili; ale którzy z miejsca swietego odeszli, przyszli w zapamietanie w onem miescie, w którem dobrze czynili. I toc jest marnosc.
\par 11 Bo iz nie zaraz wychodzi dekret na zle sprawy, przetoz na tem jest wszystko serce synów ludzkich, aby czynili zle rzeczy.
\par 12 A chociaz grzesznik sto kroc zle czyni, i odwlacza mu sie, wszakze ja wiem, ze dobrze bedzie bojacym sie Boga, którzy sie boja oblicza jego.
\par 13 Ale niezboznemu nie dobrze bedzie, ani sie przedluza dni jego, owszem pomija jako cien, przeto, iz sie nie boi oblicza Bozego.
\par 14 Jest tez marnosc, która sie dzieje na ziemi, ze bywaja sprawiedliwi, którym sie tak powodzi, jakoby czynili uczynki niepoboznych; zasie bywaja niepobozni, którym sie tak powodzi, jakoby czynili uczynki sprawiedliwych. Przetozem rzekl: I toc jest marnosc.
\par 15 A tak chwalilem wesele, przeto, iz niemasz nic lepszego czlowiekowi pod sloncem, jedno jesc, i pic, i weselic sie, a iz mu jedno to zostaje z pracy jego po wszystkie dni zywota jego, które mu Bóg dal pod sloncem.
\par 16 A chociazem udal serce swe na to, abym doszedl madrosci, i zrozumial klopoty, które sie dzieja na ziemi, dla których czlowiek ani we dnie ani w nocy nie spi;
\par 17 A wszakze widzialem przy kazdym uczynku Bozym, ze nie moze czlowiek doscignac sprawy, która sie dzieje pod sloncem. Starac sie czlowiek chcac tego dojsc, ale nie dochodzi; owszem chocby rzekl madry, ze sie chce dowiedziec, nie bedzie mógl znalesc.

\chapter{9}

\par 1 Zaprawdem to wszystko uwazal w sercu swem, abym to wszystko objasnil, ze sprawiedliwi i madrzy z sprawami swemi sa w rekach Bozych, a iz ani milosci, ani nienawisci nie zna czlowiek ze wszystkich rzeczy, które sa przed obliczem jego.
\par 2 Wszystko sie dzieje jednakowo wszystkim; jednoz przychodzi na sprawiedliwego i niezboznego, na dobrego i na czystego i nieczystego, na ofiarujacego i na tego, który nie ofiaruje; na dobrego, i na grzesznego, na przysiegajacego, i na tego, co sie przysiegi boi.
\par 3 A toc jest najgorsza miedzy wszystkiem, co sie dzieje pod sloncem, iz jednoz przychodzi na wszystkich; a owszem, ze serce synów ludzkich pelne jest zlego, a iz glupstwo trzyma sie serca ich za zywota ich, a potem ida do umarlych.
\par 4 Albowiem ktokolwiek sie towarzyszy ze wszystkimi zywymi, ma nadzieje, (Gdyz i pies zywy lepszy jest, niz lew zdechly;)
\par 5 Boc ci, co zyja, wiedza, ze umrzec maja; ale umarli o niczem nie wiedza, i nie maja wiecej zadnej zaplaty, gdyz w zapamietanie przyszla pamiatka ich.
\par 6 Owszem i milosc ich, i zazdrosc ich i nienawisc ich juz zginela, a nie maja wiecej dzialu na wieki we wszystkiem, co sie dzieje pod sloncem.
\par 7 Idzze tedy, jedz z radoscia chleb twój, a pij z dobra mysla wino twoje; albowiem juz wdzieczne sa Bogu sprawy twoje.
\par 8 Na kazdy czas niech beda szaty twoje biale, a olejku na glowie twojej niech sie nie przebiera.
\par 9 Zazywaj zywota z zona, któras umilowal, po wszystkie dni zywota marnosci twojej, którec dal Bóg pod sloncem po wszystkie dni marnosci twojej; boc ten jest dzial twój w zywocie twoim i w pracy twojej, która podejmujesz pod sloncem.
\par 10 Wszystko, co przedsiewezmie reka twoja do czynienia, czyn wedlug moznosci twojej, albowiem niemasz zadnej pracy, ani mysli, ani umiejetnosci, ani madrosci w grobie, do którego ty idziesz.
\par 11 Potem obróciwszy sie ujrzalem pod sloncem, ze bieg nie jest w mocy predkich, ani wojna w mocy meznych, ani zywnosc w mocy madrych, ani bogactwo w mocy roztropnych, ani laska w mocy pomyslnych; ale czas i trafunek wszystko przynosi.
\par 12 Bo czlowiek nie wie czasu swego; ale jako ryby, które bywaja lowione siecia szkodliwa, i jako ptaki lapane bywaja sidlem; tak ulowieni bywaja synowie ludzcy we zly czas, gdy na nie nagle przypada.
\par 13 Nadto widzialem i te madrosc pod sloncem, która jest wielka u mnie:
\par 14 Miasto male, a w niem ludzi malo, przeciw któremu przyciagnal król mozny, i oblegl je, i usypal przeciwko niemu waly wielkie;
\par 15 I znalazl sie w niem maz ubogi madry, który wybawil miasto ono madroscia swoja; choc nikt nie wspomnial na onego meza ubogiego.
\par 16 Przetozem ja rzekl: Lepsza jest madrosc, nizeli moc, aczkolwiek madrosc onego ubogiego byla wzgardzona, i slów jego nie sluchali.
\par 17 Slów ludzi madrych spokojnie sluchac nalezy, raczej niz krzyku panujacego miedzy glupimi.
\par 18 Lepsza jest madrosc niz oreze wojenne; ale jeden grzesznik psuje wiele dobrego.

\chapter{10}

\par 1 Jako muchy zdechle zasmradzaja i psuja olejek aptekarski: tak czlowieka z madrosci i z slawy zacnago troche glupstwa oszpeca.
\par 2 Serce madrego jest po prawej stronie jego; ale serce glupiego po lewej stronie jego.
\par 3 I na ten czas, gdy glupi droga idzie, serce jego niedostatek cierpi; bo pokazuje wszystkim, ze glupim jest.
\par 4 Jezliby duch panujacego powstal przeciwko tobie, nie opuszczaj miejsca twego; albowiem pokora wstret czyni grzechom wielkim.
\par 5 Jest zle, którem widzial pod sloncem, to jest, blad, który pochodzi od zwierzchnosci:
\par 6 Ze glupi wywyzszani bywaja w godnosci wielkiej, a bogaci w madrosc nisko siadaja;
\par 7 Widzialem slugi na koniach, a ksiazat chodzacych piechota jako slugi.
\par 8 kto kopie dól, sam wen wpada; a kto rozrzuca plot, waz go ukasi.
\par 9 Kto przenosi kamienie, urazi sie niemi; a kto lupie drwa, niebezpieczen jest od nich.
\par 10 Jezlize sie stepi zelazo, a nie naostrzylby ostrza jego, tedy mocy przylozyc musi; ale to daleko lepiej madrosc sprawic moze.
\par 11 Jezli ukasi waz przed zakleciem, nic nie pomoga slowa zaklinacza.
\par 12 Slowa ust madrego sa wdzieczne; ale wargi glupiego pozeraja go.
\par 13 Poczatek slów ust jego glupstwo, a koniec powiesci jego wielkie blazenstwo.
\par 14 Bo glupi wiele mówi, choc nie wie ten czlowiek, co ma byc. Albowiem któz mu oznajmi, co po nim nastanie?
\par 15 Glupi pracuja az do ustania, a przecie nie moga dojsc do miasta.
\par 16 Biada tobie, ziemio! której król jest dziecieciem, i której ksiazeta rano biesiaduja.
\par 17 Blogoslawionas ty, ziemio! której król jest synem zacnych, a której ksiazeta czasu slusznego jadaja dla posilenia, a nie dla opilstwa.
\par 18 Dla lenistwa sie dach pochyla, a dla oslabialych rak przecieka dom.
\par 19 Dla uweselenia gotuja uczty, i wino rozwesela zywot; ale pieniadze do wszystkiego dopomagaja.
\par 20 Ani w mysli twojej królowi nie zlorzecz, ani w skrytym pokoju twoim nie przeklinaj bogatego; albowiem i ptak niebieski donióslby ten glos; a to, co ma skrzydla, objawiloby powiesc twoje.

\chapter{11}

\par 1 Puszczaj chleb twój po wodzie; bo po wielu dniach znajdziesz go.
\par 2 Daj czastke siedmiom albo osmiom; bo nie wiesz, co zlego bedzie na ziemi.
\par 3 Gdy sie napelniaja obloki, deszcz na ziemie wypuszcaja; a gdy upada drzewo na poludnie, albo na pólnocy, na któremkolwiek miejscu upadnie to drzewo, tam zostanie.
\par 4 Kto upatruje wiatr, nigdy nie bedzie sial; a kto sie przypatruje oblokom, nie bedzie_zal.
\par 5 Jako ty nie wiesz, która jest droga wiatru, i jako sie zrastaja kosci w zywocie brzemiennej: tak nie wiesz sprawy Bozej, który wszystko czyni.
\par 6 Poranu siej nasienie twoje, a w wieczór nie dawaj odpoczynku rece twojej, gdyz ty nie wiesz, co jest lepszego, toli, czy owo, czyli tez oboje jednako dobre.
\par 7 Zaprawde wdzieczna jest swiatlosc, i mila rzecz oczom widziec slonce.
\par 8 A wszakze, chocby przez wiele lat zyw byl czlowiek, a przez te wszystkie weselilby sie, tedy przywiódlszy sobie na pamiec dni ciemnosci, jako ich wiele bedzie, cokolwiek przeszlo, uzna byc marnoscia.
\par 9 Przetoz wesel sie, mlodziencze! w mlodosci twojej, a niech uzywa dobrej mysli serce twoje za dni mlodosci twojej, a chodz drogami serca twego, i wedlug zdania oczu twoich; ale wiedz, ze cie dla tego wszystkiego Bóg na sad przywiedzie.
\par 10 A tak oddal gniew od serca twego, i odrzuc zlosc od ciala twego, gdyz dziecinstwo i mlodosc sa marnoscia.

\chapter{12}

\par 1 Pamietaj tedy na stworzyciela swego we dni mlodosci twojej, pierwej nizeli nastana zle dni, i nadejda lata, o których rzeczesz: Nie podobaja mi sie.
\par 2 Pierwej niz sie zacmi slonce, i swiatlo, i miesiac i gwiazdy, a nawróca sie obloki po dzdzu.
\par 3 W dzien, którego sie porusza stróze domowi, i zachwieja sie mezowie duzy i ustana melacy, przeto, iz ich malo bedzie, i zacmia sie wygladajacy oknami;
\par 4 I zawra sie drzwi z dworu z slabym glosem melcia; i powstanie na glos ptaszy, i ustana wszystkie córki spiewajace.
\par 5 Nawet i wysokiego miejsca bac sie beda, i beda sie lekac na drodze, gdy zakwitnie migdalowe drzewo, takze i szarancza bedzie mu ciezka, i zadza go ominie; bo czlowiek idzie do domu wiecznego, a placzacy po ulicach chodzic beda,
\par 6 Pierwej niz sie przerwie sznur srebrny, i niz sie stlucze czasza zlota, a rozsypie sie wiadro nad zdrojem, a skruszy sie kolo nad studnia;
\par 7 I wróci sie proch do ziemi, jako przedtem byl, a duch wróci sie do Boga, który go dal.
\par 8 Marnosc nad marnosciami, mówi kaznodzieja, a wszystko marnosc.
\par 9 A czem wiecej kaznodzieja byl medrszym, tem wiecej nauczal umiejetnosci ludu, a rozwazal i wywiadywal sie, i skladal wiele przypowiesci.
\par 10 Staral sie kaznodzieja, jakoby znalazl powiesci wdzieczne, i napisal, co jest dobrego, i slowa prawdziwe.
\par 11 Slowa madrych podobne oscieniom, i podobne gwozdziom wbitym; slowa tych, którzy je zlozyli, podane sa od pasterza jednego.
\par 12 A tak, synu mój! z tych slów sie dostatecznie upomniec miozesz; albowim skladaniu wielu ksiag konca niemasz, a wiele czytac, jest spracowanie ciala.
\par 13 Suma wszystkiego, cos slyszal: Boga sie bój, a przykazan jego przestrzegaj, bo na tem czlowiekowi wszystko zalezy;
\par 14 Poniewaz kazdy uczynek, i kazda rzecz tajna, lub dobra, lub zla, Bóg na sad przywiedzie.


\end{document}