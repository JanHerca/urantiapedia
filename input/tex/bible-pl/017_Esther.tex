\begin{document}

\title{Estery}


\chapter{1}

\par 1 I stalo sie za dni Aswerusa, (który Aswerus królowal od Indyi az do Murzynskiej ziemi, nad stem i dwudziesta i siedmia krain.)
\par 2 Ze za onych dni, gdy siedzial król Aswerus na stolicy królestwa swego, która byla w Susan, miescie stolecznem,
\par 3 Roku trzeciego królowania swego sprawil u siebie uczte na wszystkich ksiazat swoich, i slug swoich, na hetmanów z Persów i z Medów, na przelozonych i na starostów onych krain,
\par 4 Pokazujac bogactwa, i chwale królestwa swego, i zacnosc a ozdobe wielmoznosci swojej przez wiele dni, mianowicie przez sto i osmdziesiat dni.
\par 5 (A gdy sie dokonczyly dni one, uczynil król na wszystek lud, co go kolwiek bylo w Susan, w miescie stolecznem, od najwiekszego az do najmniejszego, uczte przez siedm dni na sali w ogrodzie przy palacu królewskim.)
\par 6 Opony biale, zielone i hijacyntowe zawieszono na sznurach bisiorowych i szarlatnych, na kolcach srebrnych i na slupach marmurowych; loza zlote i srebrne na tle krysztalowem, i marmurowem, i paryjowem, i socharowem.
\par 7 A napój dawano w naczyniu zlotem, a to w naczyniu co raz innem, i wina królewskiego dostatkiem, jako przystalo na króla.
\par 8 Ale do picia, wedlug ustawy, nikt nie przymuszal. Albowiem tak byl rozkazal król wszystkim rzadcom domu swego, aby czynili wedlug woli kazdego.
\par 9 Wasty tez królowa sprawila uczte na bialeglowy w domu królewskim króla Aswerusa.
\par 10 A dnia siódmego, gdy sobie król podweselil winem, rzekl do Mechumana, Bysyta, Herbona, Bygta, i Abagta, Zetara, i Charchasa, do siedmiu komorników, którzy sluzyli przed obliczem króla Aswerusa,
\par 11 Aby przywiedli Wasty królowe przed oblicze królewskie w koronie królewskiej, chcac pokazac narodom i ksiazetom pieknosc jej; bo bardzo piekna byla.
\par 12 Ale nie chciala królowa Wasty przyjsc na rozkazanie królewskie, opowiedziane przez komorników. Przetoz rozgniewal sie król bardzo, a gniew jego zapalil sie w nim.
\par 13 Tedy rzekl król do medrców, rozumiejacych czasy: (bo taki byl zwyczaj przedkladac sprawy królewskie wszystkim bieglym w prawach i w sadach;
\par 14 A najblizszymi jego byli Charsena, Setar, Admata, Tarsys, Meres, Marsena, Memuchan, siedm ksiazat Perskich i Medskich, którzy patrzali na oblicze królewskie, i siadali na pierwszem miejscu w królestwie.)
\par 15 Co czynic podlug prawa z królowa Wasty, przeto, iz nie uczynila rozkazania króla Aswerusa, opowiedzianego przez komorników?
\par 16 Tedy odpowiedzial Memuchan przed królem i ksiazetami: Nie przeciwko królowi samemu wystapila Wasty królowa, ale przeciwko wszystkim ksiazetom, i przeciwko wszystkim narodom, którzy sa po wszystkich krainach króla Aswerusa.
\par 17 Albowiem gdy sie ta sprawa królowej doniesie do wszystkich niewiast, zniewaza sobie mezów swoich w oczach swoich, i rzeka: Król Aswerus rozkazal przywiesc Wasty królowe przed oblicze swoje, a nie przyszla.
\par 18 Owszem dzisiaj toz rzeka ksiezny Perskie i Medskie, (które slyszaly postepek królowej) wszystkim ksiazetom królewskim, a bedzie dosyc wzgardy i wasni.
\par 19 Przetoz, jesli sie za dobre widzi królowi, niech wynijdzie wyrok królewski od oblicza jego, a niech bedzie wpisan miedzy prawa Perskie i Medskie, których sie przestepowac nie godzi: Ze nie chciala przyjsc Wasty przed oblicznosc króla Aswerusa, przetoz królestwo jej da król innej, lepszej niz ona.
\par 20 A gdy uslysza ten wyrok królewski, który wydasz po wszystkiem królestwie swojem, jako wielkie jest, tedy wszystkie zony beda wyrzadzaly uczciwosc malzonkom swoim od wielkiego az do malego.
\par 21 I podobala sie ta rada królowi i ksiazetom. I uczynil król wedlug rady Memuchanowej;
\par 22 A rozeslal listy do wszystkich krain królewskich, do kazdej krainy pismem jej wlasnem, i do kazdego narodu jezykiem jego, aby kazdy maz byl panem w domu swoim. A to obwolano jezykiem kazdego narodu.

\chapter{2}

\par 1 To gdy sie stalo, a usmierzyl sie gniew króla Aswerusa, wspomnial na Wasty, i na to, co byla uczynila, i na dekret, który byl wydan przeciwko niej.
\par 2 I rzekli dworzanie królewscy, sludzy jego: Niech poszukaja królowi dzieweczek, panienek pieknej urody;
\par 3 A niech postanowi król starostów po wszystkich krainach królestwa swego, którzyby zebrali wszystkie dzieweczki, panienki pieknej urody, do Susan miasta stolecznego, do domu bialych glów, pod dozór Hegaja, komornika królewskiego, stróza bialych gló w, a dali im ochedostwa ich.
\par 4 A panienka, któraby sie upodobala w oczach królewskich, niech króluje miasto Wasty. I podobala sie ta rzecz w oczach królewskich, i uczynil tak.
\par 5 A byl Zyd w Susan, w miescie stolecznem, imieniem Mardocheusz, syn Jaira, syna Symhy, syna Cysowego, z pokolenia Benjaminowego.
\par 6 A ten byl przeniesiony z Jeruzalemu z innymi pojmanymi, którzy byli przeniesieni z Jechonijaszem, królem Judzkim, których byl zawiódl w niewole Nabuchodonozor, król Babilonski.
\par 7 Ten chowal Hadasse, która tez zwano Ester, córke stryja swego, przeto, iz nie miala ojca, ani matki; a byla panienka pieknej urody, i wdziecznej twarzy, która Mardocheusz po smierci ojca jej i matki jej za córke przyjal.
\par 8 A gdy sie rozglosilo rozkazanie królewskie, i wyrok jego, i gdy zgromadzono panienek wiele do Susan, miasta stolecznego, pod dozór Hegaja, wzieto tez i Estere do domu królewskiego pod dozór Hegaja, stróza bialych glów.
\par 9 I podobala mu sie ona dzieweczka, a znalazla laske w oczach jego. Przetoz jej zaraz kazal dac ochedostwo jej, i dzial jej, i siedm panienek nadobnych kazal jej dac z domu królewskiego; nadto opatrzenia jej i panienek jej polepszyl w domu bialoglowskim.
\par 10 Ale nie oznajmila Ester ludu swego, ani rodziny swej; albowiem jej byl Mardocheusz przykazal, aby nie oznajmowala.
\par 11 Ale Mardocheusz na kazdy dzien przechadzal sie przed sienia domu bialoglowskiego, chcac sie dowiedziec, jakoby sie miala Ester, i coby sie z nia dzialo.
\par 12 A gdy przychodzil pewny czas kazdej panny, aby weszla do króla Aswerusa, gdy sie wypelnilo przy niej wszystko wedlug prawa bialych glów przez dwanascie miesiecy; (bo sie tak wypelnialy dni ochedazania ich, mazac sie przez szesc miesiecy olejkiem z myrry, a przez drugie szesc miesiecy rzeczami wonnemi, i innem ochedostwem bialoglowskiem.)
\par 13 Zatem panna wchodzila do króla, a o cokolwiek rzekla, to jej dano, aby z tem poszla z domu bialoglowskiego az do pokoju królewskiego.
\par 14 W wieczór wchadzala, a rano sie zas wracala do drugiego domu bialoglowskiego pod straz Saasgazy, komornika królewskiego, stróza zaloznic; nie wchadzala wiecej do króla, ale jezli sie upodobala królowi, przyzywano jej z imienia.
\par 15 A gdy przyszedl czas pewny Esterze, córce Abihaila, stryja Mardocheuszowego, (który ja byl sobie wzial za córke.) aby weszla do króla, nie zadala niczego, tylko co jej rzekl Hegaj, komornik królewski, stróz bialych glów. I miala Ester laske w ocz ach wszystkich, którzy ja widzieli.
\par 16 A tak wzieta jest Ester do króla Aswerusa, do domu jego królewskiego, miesiaca dziesiatego, (ten jest miesiac Tebet,)roku siódmego królowania jego.
\par 17 I rozmilowal sie król Estery nad wszystkie biale glowy, a miala laske i milosc u niego nad wszystkie panny, tak, iz wlozyl korone królewska na glowe jej, a uczynil ja królowa miasto Wasty.
\par 18 Nadto sprawil król uczte wielka na wszystkich ksiazat swoich, i slug swoich, to jest uczte Estery, i dal odpoczynek krainom, i rozdawal dary, tak jako przystoi królowi.
\par 19 A gdy powtóre zebrane byly panny, a Mardocheusz siedzial u bramy królewskiej;
\par 20 (A Ester nie oznajmila byla narodu swego, ani ludu swego, jako jej byl rozkazal Mardocheusz; bo rozkazaniu Mardocheuszowemu dosyc czynila Ester, jako gdy ja wychowywal u siebie.)
\par 21 W onez dni, gdy Mardocheusz siedzial u bramy królewskiej, rozgniewal sie Bigtan i Teres, dwaj komornicy królewscy, z tych, którzy strzegli progu, i szukali jakoby sciagnac reke na króla Aswerusa.
\par 22 Czego dowiedziawszy sie Mardocheusz, oznajmil to królowej Esterze, a Estera to oznajmila królowi imieniem Mardocheuszowem.
\par 23 A gdy sie tego dowiadywano, znalazlo sie tak; i powieszono obu na szubienicy, a napisano to w ksiegach kroniki przed królem.

\chapter{3}

\par 1 Po tych sprawach wielmoznym uczynil król Aswerus Hamana, syna Hamedatowego, Agagiejczyka, i wywyzszyl go, i wystawil stolice jego nad wszystkich ksiazat, którzy byli przy nim.
\par 2 A wszyscy sludzy królewscy, którzy byli u bramy królewskiej, klaniali mu sie, i upadali przed Hamanem: albowiem tak byl rozkazal król o nim. Ale Mardocheusz nie klanial sie, ani upadal przed nim.
\par 3 Przetoz rzekli sludzy królewscy, którzy byli w bramie królewskiej, do Mardocheusza: Czemuz ty przestepujesz rozkazanie królewskie?
\par 4 A gdy tak do niego na kazdy dzien mawiali, a nie usluchal ich, oznajmili to Hamanowi, chcac widziec, jezli sie ostoja slowa Mardocheuszowe; bo im byl powiedzial, ze byl Zydem.
\par 5 A widzac Haman, iz sie Mardocheusz nie klanial, ani upadal przed nim, napelniony jest Haman popedliwoscia.
\par 6 I mial to sobie za rzecz lekka, targnac sie na samego Mardocheusza; (bo mu bylo oznajmiono, z którego ludu byl Mardocheusz,)przetoz sie staral Haman, aby wytracil wszystkich Zydów, którzy byli po wszystkiem królestwie Aswerusowem, naród Mardocheu szowy.
\par 7 A tak miesiaca pierwszego (ten jest miesiac Nisan) roku dwunastego króla Aswerusa rozkazal Haman miotac Pur (to jest los) przed soba ode dnia do dnia, i od miesiaca az do miesiaca dwunastego; (ten jest miesiac Adar.)
\par 8 Bo byl rzekl Haman do króla Aswerusa: Jest lud niektóry rozproszony i rozsypany miedzy ludem po wszystkich krainach królestwa twego, którego prawa rózne sa od praw wszystkich narodów, a praw królewskich nie przestrzegaja; przetoz nie jest pozyteczno królowi, zaniechac ich.
\par 9 Jezli sie tedy królowi zda, niech bedzie napisano, aby byli wytraceni. A ja dziesiec tysiecy talentów srebra odwaze do rak przelozonych nad ta praca, aby je odniesli do skarbu królewskiego.
\par 10 Tedy zdjal król pierscien swój z reki swej, i dal go Hamanowi Agagiejczykowi, synowi Hamedatowemu, nieprzyjacielowi zydowskiemu.
\par 11 I rzekl król do Hamana: Srebroc to daruje, i ten lud, abys z nim czynil, coc sie podoba.
\par 12 Przetoz przyzwano pisarzy królewskich miesiaca pierwszego, trzynastego dnia tegoz miesiaca, i napisano wszystko, jako byl rozkazal Haman, do ksiazat królewskich, i do starostów, którzy byli nad kazda kraina, i do hetmanów kazdego narodu, do kazdej krainy wedlug pisma jej, i do kazdego narodu wedlug jezyka jego. Imieniem króla Aswerusa napisano, i zapieczetowano sygnetem królewskim.
\par 13 I rozeslano listy przez poslów do wszystkich krain królewskich, aby wygladzono, wymordowano, i wytracono wszystkich Zydów, od mlodego az do starca, dziatki i niewiasty, dnia jednego, trzynastego dnia miesiaca dwunastego, (ten jest miesiac Adar.) a korzysc ich aby rozchwycono.
\par 14 A tac suma byla tych listów, aby obwolano po wszystkich krainach, i oznajmiono wszystkim narodom, zeby byli gotowi na on dzien.
\par 15 Tedy wyjechali poslowie spieszno z rozkazaniem królewskiem; przybito tez wyrok w Susan, w miescie stolecznem, a król i Haman siedzieli pijac; ale miasto Susan bylo zatrwozone.

\chapter{4}

\par 1 A Mardocheusz, dowiedziawszy sie wszystkiego, co sie bylo stalo, rozdarl szaty swe, i oblekl sie w wór, a posypawszy sie popiolem, wyszedl w posród miasta, i wolal glosem wielkim i zalosnym.
\par 2 I przyszedl az przed brame królewska; bo sie nie godzilo wnijsc w brame królewska obleczonemu w wór.
\par 3 W kazdej takze krainie i miejscu, gdziekolwiek rozkazanie królewskie, i wyrok jego przyszedl, byla wielka zalosc miedzy Zydami, post, i placz, i narzekanie, a w worze, i na popiele wiele ich lezalo.
\par 4 Przetoz przyszedlszy panny Estery, i komornicy jej, oznajmili jej to; i zasmucila sie królowa bardzo i poslala szaty, aby obleczono Mardocheusza, zdjawszy z niego wór jego. Ale ich on nie przyjal.
\par 5 Tedy zawolawszy Estera Atacha, jednego z komorników królewskich, którego jej byl dal za sluge, rozkazala mu z strony Mardocheusza, aby sie dowiedzial, co i przeczby to bylo.
\par 6 Wyszedl tedy Atach do Mardocheusza na ulice miejska, która byla przed brama królewska;
\par 7 I oznajmil mu Mardocheusz wszystko, co mu sie przydalo, i o tej sumie srebra, która obiecal Haman odwazyc do skarbu królewskiego przeciwko Zydom, aby byli wytraceni.
\par 8 Nadto dal mu przepis wyroku, który byl przybity w Susan na wytracenie ich, aby okazal Esterze, i oznajmil jej; a zeby jej rozkazal, aby szla do króla, i prosila go, a przyczynila sie do niego za ludem swoim.
\par 9 Tedy przyszedlszy Atach oznajmil Esterze slowa Mardocheuszowe.
\par 10 I rzekla Estera do Atacha, wskazujac przezen do Mardocheusza:
\par 11 Wszyscy sludzy królewscy, i lud krain królewskich wiedza, ze ktobykolwiek (maz albo biala glowa) wszedl do króla do sieni wnetrznej, nie bedac wezwany, to prawo o nim jest, aby byl zabity, oprócz na kogoby wyciagnal król sceptr zloty, ten zyw z ostanie. Alem ja nie byla wezwana, abym weszla do króla, juz przez trzydziesci dni.
\par 12 A gdy oznajmiono Mardocheuszowi slowa Estery.
\par 13 Rzekl Mardocheusz, aby zasie powiedziano Esterze: Nie mniemaj w umysle twoim, abys zachowana byc miala w domu królewskim mimo wszystkich Zydów.
\par 14 Albowiem, jezli ty tak cale milczec bedziesz na ten czas, ulzenie i wybawienie przyjdzie Zydom skad inad, ale ty i dom ojca twego zginiecie; a któz wie, jezlis nie dla tego czasu dostapila królestwa?
\par 15 I rzekla Estera, aby zasie oznajmiono Mardocheuszowi:
\par 16 Idz, zbierz wszystkich Zydów, którzy sie znajduja w Susan, a posccie za mie, a nie jedzcie ani pijcie przez trzy dni, w nocy ani we dnie. Ja tez, i panny moje takze, bede poscila; tedy wnijde do króla, choc to nie wedlug prawa, a jezli zgine, niech zgine.
\par 17 Tedy szedl Mardocheusz, i uczynil wszystko, co mu byla rozkazala Estera.

\chapter{5}

\par 1 A dnia trzeciego ubrawszy sie Ester w ubiór królewski, stanela w sieni wewnetrznej domu królewskiego przeciw palacowi królewskiemu. A król siedzial na stolicy królewskiej swojej w palacu królewskim przeciwko drzwiom domu.
\par 2 A gdy ujrzal król Estere królowe stojaca w sieni, znalazla laske w oczach jego, i wyciagnal król do Estery sceptr zloty, który trzymal w rece swej. Tedy przystapiwszy Ester dotknela sie konca sceptru.
\par 3 I rzekl do niej król: Cóz ci królowa Ester? a co za prosba twoja? Chocbys tez i o polowe królestwa prosila, tedyc bedzie dano.
\par 4 I odpowiedziala Ester: Jezli sie królowi podoba, niech przyjdzie król i Haman dzisiaj na uczte, któram dla niego nagotowala.
\par 5 I rzekl król: Zawolajcie co rychlej Hamana, aby dosyc uczynil woli Estery. Przyszedl tedy król i Haman na one uczte, która byla sprawila Ester.
\par 6 Potem król rzekl do Estery, napiwszy sie wina: Cóz za prosba twoja? a bedziec dano; co za zadosc twoja? Chocbys i o polowe królestwa prosila, bedziec dano.
\par 7 Na to odpowiedziala Ester, i rzekla: Zadosc moja, i prosba moja ta jest:
\par 8 Jezlim znalazla laske w oczach królewskich, a jezli sie królowi podoba, aby przyzwolil na prosbe moje, i wypelnil zadosc moje, aby jeszcze przyszedl król i Haman na uczte, która im zgotuje, a jutro uczynie wedlug slowa królewskiego.
\par 9 A tak wyszedl Haman dnia onego wesoly, i z dobra mysla; ale gdy ujrzal Haman Mardocheusza w bramie królewskiej, ze ani powstal, ani sie ruszyl przed nim, napelniony byl Haman przeciwko Mardocheuszowi popedliwoscia.
\par 10 Wszakze zatrzymal sie Haman, az przyszedl do domu swego, a poslawszy wezwal przyjaciól swoich, i Zeres, zony swej.
\par 11 I powiadal im Haman o slawie bogactw swoich, i o mnóstwie synów swych, i o wszystkiem, jako go uwielbil król, i jako go wywyzszyl nad innych ksiazat i slug królewskich.
\par 12 Nadto rzekl Haman: Nawet nie wezwala Ester królowa z królem na uczte, która nagotowala, tylko mnie a jeszcze i na jutro jestem od niej z królem wezwany.
\par 13 Ale mi to wszystko za nic, pokad ja widze Mardocheusza Zyda, siedzacego u bramy królewskiej.
\par 14 I rzekla mu Zeres, zona jego, i wszyscy przyjaciele jego: Niech postawia szubienice wysoka na piecdziesiat lokci, a rano mów do króla, aby powieszono Mardocheusza na niej, a idz z królem na uczte z weselem. I upodobala sie ta rada Hamanowi, i kazal postawic szubienice.

\chapter{6}

\par 1 Onej nocy król nie mogac spac, kazal przyniesc ksiegi historyi pamieci godnych, i kroniki; i czytano je przed królem.
\par 2 I znalezli napisane, ze oznajmil Mardocheusz zdrade Bigtana i Teresa, dwóch komorników królewskich z tych, którzy strzegli progu, ze szukali sciagnac reke na króla Aswerusa.
\par 3 Tedy rzekl król: Jakiejz dostapil czci i zacnosci Mardocheusz dla tego? Na co odpowiedzieli sludzy królewscy, dworzanie jego: Nic za to nie odniósl.
\par 4 I rzekl król: Któz jest w sieni? (a Haman przyszedl byl do sieni zewnetrznej palacu królewskiego, chcac mówic z królem, aby powieszono Mardocheusza na szubienicy, która mu byl nagotowal.)
\par 5 Tedy odpowiedzieli królowi sludzy jego: Oto Haman stoi w sieni. I rzekl król: Niech sam wnijdzie.
\par 6 I wszedl Haman. Któremu król rzekl: Coby uczynic mezowi temu, którego król chce uczcic? (a Haman myslil w sercu swem: Komuzby chcial król uczciwosc wieksza wyrzadzic nad mie?)
\par 7 I odpowiedzial Haman królowi: Mezowi, którego król chce uczcic,
\par 8 Niech przyniosa szate królewska, w która sie ubiera król, i przywioda konia, na którym jezdza król, a niech wloza korone królewska na glowe jego;
\par 9 A dawszy one szate i onego konia do reki którego z ksiazat królewskich, z ksiazat najprzedniejszych, niech ubiora meza onego, którego król chce uczcic, a niech go prowadza na koniu po ulicy miejskiej, a niech wolaja przed nim: Tak sie ma stac mezowi, którego król chce uczcic.
\par 10 Tedy rzekl król do Hamana: Spiesz sie, wezmij szate i konia, jakos powiedzial, a uczyn tak Mardocheuszowi Zydowi, który siedzi w bramie królewskiej, a nie opuszczaj nic z tego wszystkiego, cos mówil.
\par 11 Przetoz wziawszy Haman szate i konia, ubral Mardocheusza, i prowadzil go na koniu po ulicy miejskiej, wolajac przed nim: Tak sie ma stac mezowi, którego król chce uczcic.
\par 12 Wrócil sie potem Mardocheusz do bramy królewskiej, a Haman pokwapil sie do domu swego z zaloscia, majac glowe nakryta.
\par 13 I powiedzial Haman Zeresie, zonie swej, i wszystkim przyjaciolom swoim wszystko, co mu sie przydalo. I rzekli do niego medrcy jego, i Zeres, zona jego: Poniewaz z narodu Zydowskiego jest Mardocheusz, przed któregos obliczem poczal upadac, nie przemozesz go, ale pewnie upadniesz przed obliczem jego.
\par 14 A gdy oni jeszcze mówili z nim, oto komornicy królewscy przyszli, a przymusili Hamana, aby szedl na uczte, która byla Ester sprawila.

\chapter{7}

\par 1 A tak przyszedl król i Haman na uczte do Estery królowej.
\par 2 I rzekl zasie król do Estery drugiego dnia, napiwszy sie wina: Cóz za prosba twoja, królowo Ester? a bedziec dano; co za zadosc twoja? Chocbys tez i o polowe królestwa prosila, staniec sie.
\par 3 Tedy odpowiedziala królowa Ester, i rzekla: Jezlim znalazla laske przed oczyma twemi, o królu! a jezli sie królowi podoba, niech mi bedzie darowany zywot mój na prosbe moje, i naród mój na zadosc moje.
\par 4 Albowiemesmy zaprzedani, ja i naród mój, abysmy byli wygladzeni, wymordowani i wytraceni. Gdybysmy za niewolników i niewolnice sprzedani byli, milczalabym, chocby i tak ten nieprzyjaciel nasz nie mógl nagrodzic tej szkody królowi.
\par 5 Tedy odpowiedzial król Aswerus, i rzekl do Estery królowej: Któz to jest? a gdzie ten jest, którego serce tak nadete jest, aby to smial uczynic?
\par 6 Irzekla Ester: Maz przeciwnik, a nieprzyjaciel najgorszy jest ten Haman. I strwozyl sie Haman przed królem i królowa.
\par 7 Tedy król wstal w popedliwosci swojej od onej uczty, a szedl do ogrodu przy palacu; ale Haman zostal, aby prosil o zywot swój Estery królowej; bo wiedzial, ze mu zgotowane bylo nieszczescie od króla.
\par 8 Potem król wrócil sie z ogrodu, który byl przy palacu, do domu, gdzie pil wino; a Haman upadl byl na loze, na którem siedziala Tedy rzekl król: Izali jeszcze i gwalt chce uczynic królowej u mnie w domu? A gdy te slowa wyszly z ust królewskich, zaraz twarz Hamanowa nakryto.
\par 9 Wtem rzekl Harbona, jeden z komorników, przed królem: Oto jeszcze szubienica, która byl zgotowal Haman na Mardocheusza, który sie staral o dobre królewskie, stoi przy domu Hamanowym wzwyz na piecdziesiat lokci. I rzekl król: Powiescie go na niej.
\par 10 I powieszono Hamana na onej szubienicy, która byl zgotowal Mardocheuszowi. A tak uspokoil sie gniew królewski.

\chapter{8}

\par 1 Onegoz dnia dal król Aswerus Esterze królowej dom Hamana, nieprzyjaciela zydowskiego; a Mardocheusz przyszedl przed króla; bo mu byla oznajmila Ester, ze byl jej pokrewnym.
\par 2 Tedy zdjal król pierscien swój, który byl wzial od Hamana, i dal go Mardocheuszowi, a Ester postanowila Mardocheusza nad domem Hamanowym.
\par 3 Potem jeszcze Ester mówila do króla, upadlszy u nóg jego, i plakala, i prosila go, aby wniwecz obrócil zlosc Hamana Agagiejczyka, i zamysl jego, który byl wymyslil przeciwko Zydom.
\par 4 Tedy wyciagnal król na Estere sceptr zloty, a Estera wstawszy stenela przed królem.
\par 5 I rzekla: Jezli sie królowi podoba, a jezlim znalazla laske przed obliczem jego, i jezli sie to za sluszne zda byc królowi, i jezlim ja przyjemna w oczach jego, niech napisza, aby byly odwolane listy zamyslów Hamana, syna Hamedata Agagiejczyka, które rozpisal na wytracenie Zydów, którzy sa po wszystkich krainach królewskich.
\par 6 Albowiem jakozbym mogla patrzec na to zle, któreby przyszlo na lud mój? albo jakobym mogla widziec zginienie rodziny mojej?
\par 7 I rzekl król Aswerus do Estery królowej, i do Mardocheusza Zyda: Otom dom Hamanowy dal Esterze, a onego powieszono na szubienicy, przeto, iz sciagnal reke swoje na Zydów.
\par 8 Wy tedy piszcie do Zydów, jako sie wam podoba, imieniem królewskiem, i zapieczetujcie pierscieniem królewskim; albowiem to, co sie pisze imieniem królewskiem, i pieczetuje sie pierscieniem królewskim, nie moze byc odwolane.
\par 9 A tak zwolano pisarzy królewskich onego czasu, miesiaca trzeciego, (ten jest miesiac Sywan) dwudziestego i trzeciego dnia tegoz miesiaca, a pisano wszystko, jako rozkazal Mardocheusz, do Zydów i do ksiazat, i do starostów, i do przelozonych nad krain a mi, którzy sa od Indyi az do Murzynskiej ziemi nad stem dwudziesta i siedmia krain, do kazdej krainy pismem jej, i do kazdego narodu jezykiem jego, i do Zydów pismen ich i jezykiem ich.
\par 10 A gdy napisal imieniem króla Aswerusa, i zapieczetowal pierscieniem królewskim, rozeslal listy przez poslów, którzy jezdzali na koniach predkich, i na mulach mlodych:
\par 11 Iz król dal wolnosc Zydom, którzy byli we wszystkich miastach, aby sie zgromadzili, a zastawiali sie o dusze swoje, a zeby wytracili, wymordowali, i wygubili wszystkie wojska ludu onego, i krain tych, którzyby im gwalt czynili, dziatkom ich, i zon om ich, a lupy ich zeby rozchwycili;
\par 12 A to jednego dnia po wszystkich krainach króla Aswerusa, to jest trzynastego dnia, miesiaca dwunastego, ten jest miesiac Adar.
\par 13 Suma tych listów byla: Zeby wydano wyrok w kazdej krainie, i oznajmiono wszystkim narodom, aby byli Zydzi gotowi na on dzien ku pomscie nad nieprzyjaciolmi swymi.
\par 14 Tedy poslowie, którzy jezdzali na koniach predkich i na mulach, biezeli jak najpredzej z rozkazaniem królewskiem, a przybity byl ten wyrok w Susan na palacu królewskim.
\par 15 A Mardocheusz wyszedl od króla w szacie królewskiej hijacyntowej i bialej, i w wielkiej koronie zlotej, i w plaszczu bisiorowem, i szarlatnym; a miasto Susan weselilo i radowalo sie.
\par 16 A Zydom weszla swiatlosc i wesele, i radosc i czesc.
\par 17 Takze w kazdej krainie, i w kazdem miescie, i na wszelkiem miejscu, gdziekolwiek rozkaz królewski, i wyrok jego doszedl, mieli Zydzi wesele, radosc, uczty, i dzien ucieszny; a wiele z narodów onych krain zostawali Zydami; albowiem strach byl przypadl od Zydów na nie.

\chapter{9}

\par 1 Potem dwunastego miesiaca, który jest miesiac Adar, dnia trzynastego tegoz miesiaca, gdy przyszedl czas rozkazania królewskiego i wyroku jego, aby sie wypelnil onegoz dnia, którego sie spodziewali nieprzyjaciele zydowscy panowac nad nimi, stala sie rzecz przeciwna, ze panowali Zydowie nad tymi, którzy ich mieli w nienawisci.
\par 2 Bo sie byli zebrali Zydowie w miastach swych po wszystkich krainach króla Aswerusa, aby sciagneli reke na tych którzy zlego ich szukali; a nikt sie nie ostal przed nimi, bo byl przypadl strach ich na wszystkie narody.
\par 3 A wszyscy przelozeni nad krainami, i ksiazeta, i starostowie, i sprawcy robót królewskich, mieli w uczciwosci Zydów; bo przypadl strach Mardocheuszowy na nich.
\par 4 Albowiem Mardocheusz byl wielkim w domu królewskim, a slawa jego rozchodzila sie po wszystkich krainach, gdyz on maz Mardocheusz postepowal, i wielkim urósl.
\par 5 A tak pobili Zydzi wszystkich nieprzyjaciól swoich, mieczem ich mordujac, i tracac, i niszczac, a czyniac z tymi, co ich nienawidzieli, wedlug upodobania swego.
\par 6 Nawet i w Susan, miescie stolecznem, zabili i wytracili Zydzi piec set mezów;
\par 7 I Parsandata, i Dalfona, i Aspata,
\par 8 I Porata, i Adalijasza, i Arydata,
\par 9 I Parymasta, i Arysaja, i Arydaja, i Wajzata,
\par 10 Dziesieciu synów Hamana, syna Hamedatowego, nieprzyjaciela zydowskiego, zabili; ale na lupy ich nie sciagneli reki swojej.
\par 11 Onegoz dnia, gdy przyniesiono liczbe pobitych w Susan, miescie królewskiem, przed króla,
\par 12 Rzekl król do Estery królowej: W Susan, miescie stolecznem, zabili Zydzi i wytracili piec set mezów, i dziesiec synów Hamanowych; a w innych krainach królewskich cóz uczynili? cóz jeszcze za prosba twoja? a bedziec dana; a co jeszcze za zadosc twoja? a staniec sie.
\par 13 I rzekla Ester: Jezli sie królowi podoba, niech bedzie pozwolono i jutro Zydom, którzy sa w Susan, aby uczynili wedlug wyroku dzisiejszego, a dziesiec synów Hamanowych aby zawiesili na szubienicy.
\par 14 I rozkazal król, aby tak bylo. A tak przybity byl wyrok w Susan, i powieszono dziesiec synów Hamanowych.
\par 15 A zgromadziwszy sie Zydowie, którzy byli w Susan, i dnia czternastego miesiaca Adar, zabili w Susan trzysta mezów; wszakze na lupy ich nie sciagneli reki swojej.
\par 16 Inni takze Zydzi, którzy byli w krainach królewskich, i zebrawszy sie zastawiali sie za dusze swe; a póty mieli pokój od nieprzyjaciól swych. Bo zabili nieprzyjaciól swoich siedmdziesiat i piec tysiecy; wszakze na lupy ich nie sciagneli reki swo jej.
\par 17 Poczeli dnia trzynastego miesiaca Adar, a przestali dnia czternastego tegoz miesiaca, a sprawowali tegoz dnia uczty i wesela.
\par 18 Ale Zydzi, którzy byli w Susan, zebrali sie dnia trzynastego i czternastego tegoz miesiaca; a odpoczeli pietnastego dnia tegoz miesiaca, i sprawowali dnia onego uczty i wesela.
\par 19 Przetoz Zydzi mieszkajacy po wsiach, i po miasteczkach niemurowanych, obchodza dzien czternasty miesiaca Adar z weselem, i z ucztami i z dobra mysla, posylajac upominki jeden drugiemu.
\par 20 Bo pisal Mardocheusz o tem, i rozeslal listy do wszystkich Zydów, którzy byli po wszystkich krainach króla Aswerusa, do bliskich i do dalekich.
\par 21 Stanowiac im, aby obchodzili dzien czternasty miesiaca Adar, i dzien pietnasty tegoz miesiaca na kazdy rok,
\par 22 Wedlug onych dni, w których odpoczeli Zydzi od nieprzyjaciól swoich, a miesiaca tego, który sie im byl obrócil z smutku w wesele, a z placzu w dzien radosci; aby obchodzili one dni z ucztami i z weselem, jeden drugiemu upominki, a ubogim dary po sylajac.
\par 23 I przyjeli to wszyscy Zydzi, ze co zaczeli, czynic beda, i co pisal Mardocheusz do nich;
\par 24 Jako Haman, syn Hamadetowy, Agagiejczyk, nieprzyjaciel wszystkich Zydów, umyslil o Zydach, aby ich wytracil, i miotal pur, to jest los, na wytracenie ich i na wygubienie ich:
\par 25 A jako Ester weszla przed oblicze królewskie, i mówila o listy; a jako obrócone byly zle zamysly jego, które byl wymyslil przeciwko Zydom na glowe jego; i jako go powieszono i synów jego na szubienicy.
\par 26 Przetoz nazwali one dni Purym, od imienia tego pur, a to za przyczyna wszystkich slów listu tego, i co widzieli przy tem, i co przyszlo na nich.
\par 27 Postanowili tez i przyjeli to Zydowie na sie, i na nasienie swoje, i na wszystkich, którzy sie do nich przylaczyli, aby tego nie przestepowano, ale zeby obchodzono te dwa dni wedlug opisania ich, i wedlug postanowionego czasu ich na kazdy rok.
\par 28 A iz te dni beda pamietne i slawne od wieku do wieku, od rodzaju do rodzaju w kazdej krainie, i w kazdem miescie. Nadto, ze te dni Purym nie zagina z posrodku Zydów, a pamiatka ich nie ustanie u potomstwa ich.
\par 29 Napisala tez Ester królowa, córka Abihajlowa, i Mardocheusz Zyd, ze wszelka pilnoscia, aby potwierdzili tym listem wtórym tych dni Purym.
\par 30 Który list Mardocheusz poslal do wszystkich Zydów, do stu i dwudziestu i siedmiu krain królestwa Aswerusowego, pozdrawiajac ich laskawie i uprzejmie.
\par 31 A zeby statecznie przestrzegali tych dni Purym czasów swoich, jako im je postanowil Mardocheusz Zyd, i Ester królowa, i jako obowiazali siebie samych, i nasienie swoje, na pamiatke postu i narzekania ich.
\par 32 A tak wyrok Estery potwierdzil ustawy tych dni Purym, co zapisano w tej ksiedze.

\chapter{10}

\par 1 Potem ulozyl król Aswerus podatek na ziemie swoje, i na wyspy morskie.
\par 2 Awszystkie sprawy mocy jego, i moznosci jego, z opisaniem zacnosci Mardocheuszowej, która go wielmoznym uczynil król, to zapisano w ksiegach kronik o królach Medskich i Perskich.
\par 3 Albowiem Mardocheusz Zyd byl wtórym po królu Aswerusie, i wielkim u Zydów, i zacny u mnóstwa braci swych, starajac sie o dobro ludu swego, i sprawujac pokój wszystkiemu narodowi swemu.


\end{document}