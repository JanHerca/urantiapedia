\begin{document}

\title{1 Kronik}


\chapter{1}

\par 1 Adam, Set, Enos.
\par 2 Kienan, Mahalaleel, Jared.
\par 3 Eonch, Matusalem, Lamech.
\par 4 Noe, Sem, Cham, i Jafet.
\par 5 Synowie Jafetowi: Gomer, i Magog, i Madaj, i Jawan, i Tubal, i Mesech, i Tyras,
\par 6 A synowie Gomerowi: Aschenaz, i Ryfat, i Togorma.
\par 7 Synowie tez Jawanowi: Elisa, i Tarsys, Cytym, i Dodanin.
\par 8 Synowie Chamowi: Chus i Misraim, Put i Chanaan.
\par 9 A synowie Chusowi: Seba, i Hewila, i Sabta, i Regma, i Sabtacha; a synowie Regmy: Seba i Dedan.
\par 10 Splodzil tez Chus Neroda; ten poczal byc moznym na ziemi.
\par 11 Misraim tez splodzil Ludyma, i Hananima, i Laubima, i Naftuhyma,
\par 12 I Patrusyma, i Chasluchyma, (z których poszli Filistynowie) i Kaftoryma.
\par 13 Potem Chanaan splodzil Sydona, pierworodnego swego, i Hetejczyka.
\par 14 I Jebuzejczyka, i Amorejczyka, i Giergiezejczyka,
\par 15 I Hewejczyka, i Archajczyka, i Symejczyka,
\par 16 I Aradejczyka, i Samarejczyka, i Chamatejczyka.
\par 17 Synowie Semowi: Elam, i Assur, i Arfachsad, i Lud, i Aram, i Chus, i Hul, i Gieter, i Mesech.
\par 18 A Arfachsad splodzil Selecha, a Selech splodzil Hebera.
\par 19 A Heberowi urodzili sie dwaj synowie, z których jednemu imie bylo Faleg, przeto, ze za jego czasów rozdzielona jest ziemia; a imie brata jego Jektan.
\par 20 A Jektan splodzil Elmodada, i Salefa, i Hassarmota, i Jarecha,
\par 21 I Adorama, i Uzala, i Dekla,
\par 22 I Hebala, i Abimaela, i Sebaja,
\par 23 I Ofira, i Hewila, i Jobaba. Ci wszyscy byli synowie Jektanowi.
\par 24 Sem, Arfachsad, Selech.
\par 25 Heber, Peleg, Rechu,
\par 26 Sarug, Nachor, Tare,
\par 27 Abram; ten jest Abraham.
\par 28 Synowie Abrahamowi: Izaak i Ismael.
\par 29 A tec sa rodzaje ich: Pierworodny Ismaelowy Nebajot, i Kiedar, i Abdeel, i Mabsam.
\par 30 Masma, i Duma, Massa, Hadad, i Tema,
\par 31 Jetur, Nafis i Kiedma. Cic sa synowie Ismaelowi.
\par 32 A synowie Cetury, zaloznicy Abrahamowej, których porodzila: Zamram i Joksan, i Madan, i Midyjan, i Jesbok, i Suach. A synowie Joksanowi; Saba i Dedan.
\par 33 Synowie tez Madyjanowi: Hefa, i Hefer, i Henoch, i Abida, i Eldaa. Cic wszyscy sa synowie Cetury.
\par 34 I splodzil Abraham Izaaka. A synowie Izaakowi byli: Ezaw i Izrael.
\par 35 A synowie Ezawowi: Elifas, Rehuel, i Jehus, i Jelom, i Kore.
\par 36 Synowie Elifasowi: Teman i Omar, Sefo i Gaatan, Kienaz i syn Tamny, to jest, Amalek.
\par 37 Synowie Rehuelowi: Nahat, Zara, Samma, i Meza.
\par 38 A synowie Seirowi: Lotan, i Sobal, i Sebeon, i Hana, i Dysson, i Eser, i Dysan.
\par 39 A synowie Lotanowi: Chory, i Heman; a siostra Lotanowa byla Tamna.
\par 40 Synowie Sobalowi: Halman, i Manaat, i Hewal, Sefo, i Onam; a synowie Sebeonowi: Ajai Ana.
\par 41 Synowie Ana: Dyson; a synowie Dysona: Hamdan, i Eseban, i Jetran, i Charan.
\par 42 Synowie Eserowi: Balaan, i Zawan, Akan. Synowie Dysanowi: Hus i Aran.
\par 43 Cic sa królowie, którzy królowali w ziemi Edomskiej, przedtem niz królowal król nad synami Izraelskimi: Bela, syn Beorowy, a imie miasta jego Dynhaba.
\par 44 A gdy umarl Bela, królowal miasto niego Jobab, syn Zerachowy z Bosry.
\par 45 A gdy umarl Jobab, królowal miasto niego Chusam z ziemi Temanskiej.
\par 46 A gdy umarl Chusam, królowal miasto niego Hadad, syn Badadowy, który porazil Madyjanczyków na polu Moabskiem; a imie miasta jego Hawid.
\par 47 A gdy umarl Hadad, królowal miasto niego Samla z Masreki.
\par 48 A gdy umarl Samla, królowal miasto niego Saul z Rechobot nad rzeka
\par 49 A gdy umarl Saul, królowal miasto niego Balanan, syn Achoborowy.
\par 50 A gdy umarl Balanan, królowal miasto niego Hadar; a imie miasta jego Pehu, imie tez zony jego Mehetabel, córka Matredy, córki Mezaabowej.
\par 51 A gdy umarl Hadar, byli ksiazetami w Edon: ksiaze Tamna, ksiaze Halwa, ksiaze Jetet,
\par 52 Ksiaze Oolibama, ksiaze Ela, ksiaze Pinon,
\par 53 Ksiaze Kienaz, ksiaze Teman, ksiaze Mabsar,
\par 54 Ksiaze Magdyjel, ksiaze Hyram. Toc byli ksiazeta Edomscy.

\chapter{2}

\par 1 Cic sa synowie Izraelowi: Ruben, Symeon, Lewi, i Juda, Isaschar i Zabulon,
\par 2 Dan, Józef, i Benjamin, Neftali, Gad i Aser.
\par 3 Synowie Judy: Her, i Onan, i Sela. Ci trzej urodzili mu sie z córki Sui Chananejskiej. Ale Her, pierworodny Judy, byl zlym przed oczyma Panskiemi; przetoz go zabil.
\par 4 Tamar zasie, niewiastka jego, urodzila mu Faresa i Zare. Wszystkich synów Judowych piec.
\par 5 Synowie Faresowi: Hesron i Hamuel.
\par 6 Synowie zas Zary: Zamry, i Etan, i Heman, i Chalkol, i Darda; wszystkich tych bylo piec.
\par 7 Asynowie Zamrego: Charmi, wnuk Acharowy, który zamieszanie uczynil w Izraelu, zgrzeszywszy kradzieza rzeczy przekletych.
\par 8 Asynowie Etanowi: Azaryjasz.
\par 9 A synowie Esronowi, którzy mu sie urodzili: Jerameel, i Ram, i Chalubaj.
\par 10 Ale Ram splodzil Aminadaba, a Aminadab splodzil Naasona, ksiazecia synów Judzkich.
\par 11 A Naason splodzil Salmona, a Salmon splodzil Booza.
\par 12 A Booz splodzil Obeda, a Obed splodzil Isajego.
\par 13 A Isaj splodzil pierworodnego swego Elijaba, i Abinadaba wtórego, i Samma trzeciego.
\par 14 Natanaela czwartego, Raddaja piatego.
\par 15 Ozema szóstego, Dawida siódmego.
\par 16 A siostry ich: Sareija, i Abigail; a synowie Sarwii: Abisaj, i Joab, i Asael, trzej.
\par 17 A Abigail urodzila Amaze, a ojciec Amazy byl Jeter Ismaelczyk.
\par 18 A Kaleb, syn Hesronowy, splodzil z Azuba, malzonka swoja, i z Jeryjota synów. A ci byli synowie jego: Jeser, i Sobab, i Ardon.
\par 19 A gdy umarla Azuba, pojal sobie Kaleb Esrate, która mu urodzila Hura.
\par 20 A Hur splodzil Ury, a Ury splodzil Besaleela.
\par 21 Potem wszedl Hesron do córki Machyra, ojca Galaadowego, a pojal ja, bedac w szescdizesiat lat; która mu urodzila Seguba.
\par 22 A Segub splodzil Jaira, który mial dwadziescia trzy miast w ziemi Galaadskiej.
\par 23 Bo wzial Giessurytom, i Assyryjczykom wsi Jairowe, i Kanat z miasteczkami jego, szescdziesiat miast. To wszystko pobrali synowie Machyra, ojca Galaadowego.
\par 24 A gdy umarl Hesron w Kaleb Efrata, tedy zona Hesronowa Abija porodzila mu Assura, ojca Tekui.
\par 25 Byli tez synowie Jerameelowi, pierworodnego Hesronowego: Pierworodny Ram, po nim Buana, i Oren, i Osem z Abii.
\par 26 Mial takze druga zone Jerameel, imieniem Atara; ta jest matka Onamowa.
\par 27 Ale synowie Ramowi, pierworodnego Jerameelowego, byli: Maas, i Jamin, i Achar.
\par 28 Byli tez synowie Onamowi: Semaj, i Jada, a synowie Semejego: Nadad i Abisur;
\par 29 A imie zony Abisurowej Abihail, która mu urodzila Achobbana i Molida.
\par 30 Synowie Nadabowi: Saled i Affaim; lecz Saled umarl bez potomstwa.
\par 31 A synowie Affaimowi Jesy; a synowie Jesy Sesan, a córka Sesana Achialaj.
\par 32 A synowie Jady, brata Semejego, Jeter i Jonatan; ale Jeter umarl bez potomstwa.
\par 33 A synowie Jonatanowi: Falet i Zyza. Cic byli synowie Jerameelowi.
\par 34 Lecz nie mial Sesan synów, jedno córki; mial tez Sesan sluge Egipczanina, imieniem Jeracha.
\par 35 I dal Sesan córke Jerachowi, sludze swemu, za zone, która mu urodzila Etaja.
\par 36 Etaj splodzil Natana, a Natan splodzil Zabada.
\par 37 A Zabad splodzil Efijala, a Efijal splodzil Obeda.
\par 38 A obed splodzil Jehu, a Jehu splodzil Azaryjasza.
\par 39 A Azaryjasz splodzil Helesa, a Heles splodzil Elasa.
\par 40 A Elas splodzil Sysmaja, a Sysmaj splodzil Selluma.
\par 41 A Sellum splodzil Ikamijasza, a Ikamijasz splodzil Elisama.
\par 42 A synowie Kaleba, brata Jerameelowego: Mesa pierworodny jego, który byl ojcem Zyfejczyków i synów Maresy, ojca Hebrowowego.
\par 43 A synowie Hebronowi: Kore i Tafua, i Rechem, i Semma.
\par 44 A Semma splodzil Rahama, syna Jerkaamowego, a Rechem splodzil Sammajego.
\par 45 A Sammaj byl synem Maonowym, a Maon byl ojcem Betsurczyków.
\par 46 Efa tez, zaloznica Kalebowa, urodzila Harana, i Moze, i Giezeza; a Haran splodzil Giezeza.
\par 47 A synowie Jachdajowi: Regiem, i Jotam, i Giesan, i Falet, i Efa, i Saaf,
\par 48 Zaloznica zas druga Kalebowa Maacha urodzila Sabera, i Tyrchana.
\par 49 A zona Saafowa urodzila ojca Madmenczyków, i Sewa, ojca Machbenczyków, i ojca Gabaonczyków; a córka Kalebowa byla Achsa.
\par 50 Cic byli synowie Kaleba, syna Hurowego, pierworodnego Efraty: Sobal, ojciec Karyjatyjarymczyków.
\par 51 Salma, ojciec Betlehemczyków, Charef, ojciec Betgaderczyków.
\par 52 Mial tez synów Sobal, ojciec Karyjatyjarymczyków, który dogladal polowy Menuchoty.
\par 53 A domy Karyjatyjarymskie byly Jetrejczycy, i Futejczycy, i Sematejczycy, i Maserejczycy, z których tez poszli Saraitowie, o Estaolitowie.
\par 54 A synowie Salmy: Betlehemczycy, i Netofatczycy, ozdoby domu Joabowego, i polowa Manachaty, ojca Sorygo.
\par 55 A domy pisarzów mieszkajacych w Jabez: Tyryjatejczycy, Symatejczycy, Suchatejczycy. Cic sa Cynejczycy, którzy poszli z Hemata, ojca domu Rechabowego.

\chapter{3}

\par 1 Cic sa synowie Dawidowi, którzy mu sie urodzili w Hebronie: Pierworodny Ammon z Achynoamy Jezreelitki; wtóry Danijel z Abigaili Karmelitki;
\par 2 Trzeci Absalom, syn Maachy, córki Tolmaja, króla Giessur; czwarty Adonijasz, syn Haggity;
\par 3 Piaty Sefatyjasz z Abitaili; szósty Jetraam z Egli, zony jego.
\par 4 Tych szesc urodzilo mu sie w Hebronie, kedy królowal przez siedm lat, i przez szesc miesiecy; a trzydziesci i trzy lata królowal w Jeruzalemie.
\par 5 A ci urodzili mu sie w Jeruzalemie; Samna, i Sobab, i Natan, i Salomon, cztery synowie z Betsui, córki Ammielowej;
\par 6 I Ibchar, i Elisama, i Elifet;
\par 7 I Noge, i Nefeg, i Jafija;
\par 8 I Elizama, i Elijada, i Elifelet, dziewiec synów.
\par 9 A cic wszyscy sa synowie Dawidowi, oprócz synów z zaloznic; a Tamar byla siostra ich.
\par 10 Syn Salomonowy Roboam; Abiam syn jego, Aza syn jego, Jozafat syn jego.
\par 11 Joram syn jego, Ochozyjasz syn jego, Joaz syn jego;
\par 12 Amazyjasz syn jego, Azaryjasz syn jego, Joatam syn jego;
\par 13 Achaz syn jego, Ezechyjasz syn jego, Manases syn jego;
\par 14 Amon syn jego, Jozyjasz syn jego;
\par 15 A synowie Jozyjaszowi: Pierworodny Johanan, wtóry Joakim, trzeci Sedekijasz, czwarty Sellum.
\par 16 A synowie Joakimowi: Jechonijasz syn jego, Sedekijasz syn jego.
\par 17 A synowie Jechonijasza wieznia: Salatyjel syn jego.
\par 18 A salatyjelowi: Malchiram, i Fadajasz, i Seneser, Jekiemija, Hosama, i Nadabija.
\par 19 A synowie Fadajaszowi: Zorobabel, i Semej; a syn Zorobabelowy Mesollam, i Hananijasz, i Selomit, siostra ich.
\par 20 A Mesollamowi: Hasuba, i Ohol, i Barachyjasz, i Hazadyjasz Josabchesed, piec synów.
\par 21 A syn Hananijaszowy: Faltyjasz, i Jesajasz; synowie Rafajaszowi, synowie Arnanaszowi, synowie Obadyjaszowi, synowie Sechenijaszowi.
\par 22 A synowie Sechenijaszowi: Semejasz; a synowie Semejaszowi: Chattus, i Igal, i Baryja, i Naaryjasz, i Safat; szesc synów.
\par 23 A synowie Naaryjaszowi: Elijenaj, i Ezechyjasz, i Esrykam, trzej synowie.
\par 24 A synowie Elijenajego: Hodawijasz i Elijasub, i Felejasz, i Akkub, i Jochanan, i Dalajasz, i Anani, siedm synów.

\chapter{4}

\par 1 Synowie Judowi: Fares, Hesron, i Charmi, i Hur, i Sobal.
\par 2 A Rejasz, syn Sobalowy splodzil Jahata, a Jahat splodzil Achuma, i Laada. Tec sa rodzaje Zaratego.
\par 3 Ci tez sa z ojca Etama: Jezreel, i Jesema, i Idbas; a imie siostry ich Selelfuni.
\par 4 A Faunuel ojciec Giedory, i Ezer ojciec Hosy. Cic sa synowie Hura, pierworodnego Efraty, ojca Betlehemczyków.
\par 5 A Assur, ojciec Tekui, mial dwie zony: Chele i Naare.
\par 6 I urodzila mu Naara Achusama, i Hefera, i Temana, i Achastara. Cic sa synowie Naary.
\par 7 Synowie zasie Cheli: Seret, Jesochar, i Etnan.
\par 8 A Kos splodzil Anuba, i Hasoboba, i dom Acharchela, syna Harumy.
\par 9 A Jabes byl zacniejszy nad braci swych, któremu matka jego dala imie Jabes, mówiac: Bom go w bolesci urodzila.
\par 10 I wzywal Jabes Boga Izraelskiego, mówiec: Jezli blogoslawiac blogoslawic mi bedziesz, a rozszerzysz granice moje, a bedzie reka twoja ze mna, a wybawisz mie od zlego, abym utrapiony nie byl. I wypelnil to Bóg, o co go prosil.
\par 11 A Chelub, brat Sucha, splodzil Mechyra; ten jest ojcem Estona.
\par 12 A Eston splodzil Betrafa, i Paseacha, i Techynna, ojca miasta Nahas. Cic sa mezowie Rechy.
\par 13 A synowie Kienazowi: Otonijel, i Seraja; a synowie Otonijelowi: Hatat i Meanataj.
\par 14 A Meanataj splodzil Ofra, a Serajasz splodzil Joaba, ojca w dolinie mieszkajacych rzemieslników; bo rzemieslnicy byli.
\par 15 Synowie zasie Kaleba, syna Jefumi: Hyru, Ela, i Nahain; a syn Ela jest Kienaz.
\par 16 A synowie Jehaleleela: Zyf i Zyfa, Tyryja, i Azarel.
\par 17 A synowie Ezry: Jeter, i Mered, i Efer, i Jalon; a zona Merodowa urodzila Miryjama, i Samaja, i Isbacha, ojca Estemoa.
\par 18 Zona tez jego Judaja urodzila Jereda, ojca Giedor, i Hebera, ojca Socho, i Jekutyjela, ojca Zanoach. Cic sa synowie Betii, córki Faraonowej, która byl pojal Mered.
\par 19 A synowie zony Hodyjaszowej, siostry Nahama, ojca Ceili: Garmi i Estemoa Mahatczyk.
\par 20 A synowie Symonowi: Amnon, i Rynna, Benchanan i Tylon; a synowie Isy: Zochet i Bensochet.
\par 21 Synowie Seli, syna Judowego: Her, ojciec Lecha i Laada, ojciec Maraa; i rodzaje domów tych, którzy robili okolo bisioru w domu Asbeat,
\par 22 I Joakim i mezowie Chozeby, i Joaz i Saraf, którzy panowali w Moab, i Jasubi Lechem; ale te rzeczy sa dawne.
\par 23 Cic sa garncarze, którzy mieszkali w sadach i miedzy plotami, którzy tam przy królu dla robót jego mieszkali.
\par 24 Synowie Symeonowi: Namuel, i Jamin, i Jaryb, Zera, Saul.
\par 25 Sallum syn jego, Mabsam syn jego, Misma syn jego.
\par 26 A synowie Mismy: Hamuel syn jego, Zachur syn jego, Semej syn jego.
\par 27 Ten Semej mial synów szesnascie, i córek szesc; ale bracia jego nie mieli wiele synów, tak, ze wszystkiej rodziny ich nie bylo tak wiele, jako synów Judowych.
\par 28 I mieszkali w Beerseba i w Molada, i w Hasersual,
\par 29 I w Bela, i w Asem, i w Etolad,
\par 30 I w Betul, i w Horma, i w Sycelegu,
\par 31 I w Bet Marchabot, i w Hasersusa, i w Betbirze, i w Saaraim. Te miasta ich byly, póki królowal Dawid.
\par 32 A wsi ich byly: Etam, i Hain, Remnon, i Tochen i Asan przy tych pieciu miastach;
\par 33 I wszystkie wsi ich, które byly okolo tych miast az do Baal; tec byly mieszkaniem ich wedlug narodu ich.
\par 34 A Mosobab, i Jamlech, i Josa, syn Amazyjaszowy;
\par 35 I Joel, i Jehu, syn Josabijasza, syna Serajaszowego, syna Asyjelowego;
\par 36 I Elichenaj, i Jakóba, i Jesochaja, i Asaja, i Adyjel, i Jesymijel, i Banajas;
\par 37 I Sysa, syn Syfy, syna Allonowego, syna Jodajaszowego, syna Symry, i syna Semajaszowego.
\par 38 Ci mianowani postanowieni sa za ksiazeta w narodach swych, a domy ojców ich rozmnozyly sie bardzo.
\par 39 Przetoz ruszyli sie, aby szli do Gador, az na wschód slonca onej doliny, aby szukali paszy bydlu swemu.
\par 40 I znalezli obfite i wyborne pasze, a ziemie przestronna i spokojna i rodzajna; bo tam naród Chamów mieszkal przedtem.
\par 41 Przetoz przyszedlszy ci z imienia opisani za dni Ezechyjasza, króla Judzkiego, poburzyli namioty ich i przybytki ich, które tam byly znalezione; a wymordoealiich, i nie masz ich az do dnia tego, i osiedli miejsce ich; bo tam mieli paszy dla bydel swoich.
\par 42 A z onych synów Symeonowych niektórzy szli na góre Seir, piec set mezów, a Faltyjasz, i Necharyjasz, i Rafajasz, i Husel, synowie Isy, byli wodzami ich.
\par 43 I wymordowali ostatki, które byly uszly z Amalekitów, a mieszkali tam az po dzis dzien.

\chapter{5}

\par 1 A synowie Rubena pierworodnego Izraelowego, (ten bowiem byl pierworodny; ale gdy zgwalcil loze ojca swego, dane jest pierworodztwo jego synom Józefa, syna Izraelowego, tak jednak, ze go nie poczytano za pierworodnego:
\par 2 Bo Judas byl najmezniejsz miedzy bracmi swymi, a ksiazeciem miedzy nimi; ale pierworodztwo nalezalo Józefowi.)
\par 3 Synowie mówie Rubena, pierworodnego Izraelowego, byli: Henoch i Fallu, Hesron i Charmi.
\par 4 Synowie Joelowi: Samajasz syn jego, Gog syn jego, Semej syn jego;
\par 5 Michas syn jego, Reajasz syn jego, Baal syn jego;
\par 6 Bera syn jego, którego wzial w niewole Teglat Falasar, król Asyryjski; ten byl ksiazeciem Rubenitów.
\par 7 A bracia jego wedlug domów swych, gdy byli policzeni wedlug ich narodów, mieli ksiazeta Jehiela i Zacharyjasza.
\par 8 A Bela, syn Azazowy, syna Semmy, syna Joelowego; ten mieszkal w Aroer az ku Nebo i Baalmeon.
\par 9 Takze i na wschód slonca mieszkal, az kedy wchodza na puszcze od rzeki Eufrates; albowiem stada ich rozmnozyly sie w ziemi Galaadskiej.
\par 10 Ci za dni Saulowych walczyli z Agarenczykami, którzy porazeni sa od reki ich; a tak mieszkali w namiotach ich po wszystkiej krainie wschodniej ziemi Galaadskiej.
\par 11 A synowie Gadowi na przeciwko nich mieszkali w ziemi Bazan, az do Selchy.
\par 12 Joel byl przedniejszym ich, a Safam wtóry, a Janaj i Safat zostali w Bazan.
\par 13 A braci ich wedlug domów ojców swych: Michael i Mesullam, i Seba, i Joraj, i Jachan, i Zyja, i Heber, siedm.
\par 14 (Cic sa synowie Abihaila, syna Hurowego, syna Jaroachowego, syna, Galaadowego, syna Michaelowego, syna Jesysowego, syna Jachdowego, syna Buzowego;)
\par 15 Achy, syn Abdyjela, syna Gunowego, ksiaze w domu ojców ich.
\par 16 I mieszkali w Galaad, w Bazan, i w miasteczkach jego, i po wszystkich przedmiesciach Saron az do ich granic.
\par 17 Wszyscy ci policzeni byli za dni Jotama, króla Judzkiego, i za dni Jeroboama, króla Izraelskiego.
\par 18 Synów Rubenowych, i Gadowych, i polowy pokolenia Manasesowego, ludzi walecznych, mezów noszacych tarcze i miecz, i ciagnacych luk, i cwiczonych ku bojowi, czterdziesci i cztery tysiace, siedm set i szescdziesiat, wychodzacych do bitwy.
\par 19 Ci wiedli wojne z Agarenczykami, z Jeturejczykami, i z Nafejczykami, i Nodabczykami,
\par 20 A mieli pomoc przeciwko nim. I podani sa w reke ich Agarenczycy ze wszystkim, co mieli, przeto iz do Boga wolali w bitwie, a on ich wysluchal, iz ufali w nim.
\par 21 I zabrali dobytki ich, wielbladów ich piecdziesiat tysiecy, a owiec dwiescie i piecdziesiat tysiecy, oslów dwa tysiace, a ludzi sto tysiecy.
\par 22 Albowiem rannych wiele poleglo, iz od Boga byla ona porazka. I mieszkali na miejscu ich, az ich zabrano w niewole.
\par 23 Ale synowie polowy pokolenia Manasesowego mieszkali w onej ziemi od Bazan az do Baal-Hermon i Sanir, które jest góra Hermon; bo i oni rozmnozeni byli.
\par 24 A cic sa ksiazeta domów ojców ich: Efer i Jesy, i Elijel, i Azryjel, i Jeremijasz, i Hodawijasz, i Jachdyjel, mezowie bardzo mocni, mezowie slawni, ksiazeta domu ojców swoich.
\par 25 Ale gdy zgrzeszyli przeciw Bogu ojców swych i cudzolozyli nasladujac bogów narodów onej ziemi, które wykorzenil Bóg przed twarza ich:
\par 26 Wzbudzil Bóg Izraelski ducha Fula króla Assyryjskiego, i ducha Teglat Falasera, króla Assyryjskiego, i przeniósl ich: Rubenitów, i Gadydtów, i polowe pokolenia Manasesowego, a zawiódl ich do Hela, i Haboru, i do Ara, i do rzeki Gozan, az do dnia tego.

\chapter{6}

\par 1 Synowie Lewiego: Gerson, Kaat, i Merary.
\par 2 A synowie Kaatowi: Amram, Izaar, i Hebron, i Husyjel.
\par 3 A synowie Amramowi: Aaron, i Mojzesz, i córka Maryja; a synowie Aaronowi: Nadab, i Abiju, Eleazar, i Itamar.
\par 4 Eleazer splodzil Fineesa; Finees splodzil Abisua.
\par 5 Abissue splodzil Bokki, a Bokki splodzil Uzy.
\par 6 A Uzy splodzil Zerachyjasza, a Zerachyjasz splodzil Merajota.
\par 7 Merajot splodzil Amaryjasza, a Amaryjasz splodzil Achytoba.
\par 8 A Achytob splodzil Sadoka, a Sadok splodzil Achymaasa.
\par 9 Achymaas splodzil Azaryjasza, a Azaryjasz splodzi× Johanana.
\par 10 A Johanan splodzil Azaryjasz; tenci jest, który kaplanski urzad sprawowal w domu, który zbudowal Salomon w Jeruzalemie.
\par 11 Splodzil tez Azaryjasz Amaryjasza, a Amaryjasz splodzil Achytoba.
\par 12 A Achytob splodzil Sadoka, a Sadok splodzil Salluma.
\par 13 A Sallum splodzil Helkijasza, a Helkijasz splodzil Azaryjasza.
\par 14 A Azaryjasz splodzil Sarajasza, a Sarajasz splodzil Jozedeka.
\par 15 Ale Jozedek poszedl w niewole, gdy Pan przeniósl Jude i Jeruzalem przez Nabuchodonozora.
\par 16 Synowie Lewi: Gierson, Kaat, i Merary.
\par 17 A tec sa imiona synów Giersonowych: Lobni i Semei.
\par 18 A synowie Kaatowi: Amram i Izaar, i Hebron, i Husyjel.
\par 19 Synowie Merarego: Macheli, i Muzy. A tec sa domy Lewitów wedlug ojców ich.
\par 20 Giersonowi: Lobni syn jego, Jachat syn jego, Zamma syn jego;
\par 21 Joach syn jego, Iddo syn jego, Zara syn jego, Jetraj syn jego.
\par 22 Synowie Kaatowi: Aminadab syn jego, Kore syn jego, Aser syn jego.
\par 23 Elkana syn jego, i Abiazaf syn jego, i Assyr syn jego;
\par 24 Tachat syn jego, Uryjel syn jego, Ozyjasz syn jego, i Saul syn jego.
\par 25 A synowie Elkamowi: Amasaj i Achymot.
\par 26 Elkana. Synowie Elkanowi: Sofaj syn jego, i Nahat syn jego;
\par 27 Elijab syn jego, Jerobam syn jego, Elkana syn jego.
\par 28 A synowie Samuelowi: Pierworodny Wassni i Abijas.
\par 29 Synowie Merarego: Mahali; Lobni syn jego, Symej syn jego, Uza syn jego;
\par 30 Symha syn jego, Haggijasz syn jego, Asajasz syn jego.
\par 31 Ci sa, których postanowil do spiewania w domu Panskim, gdy tam postawiono skrzynie.
\par 32 I sluzyli przed przybytkiem namiotu zgromadzenia, spiewajac, az zbudowal Salomon dom Panski w Jeruzalemie, i stali wedlug porzadku swego na sluzbie swojej.
\par 33 A cic sa, którzy stali i synowie ich z synów Kaatowych: Heman spiewak syn Joela, syna Samuelowego,
\par 34 Syna Elkanowego, syna Jerohamowego, syna Elijelowego, syna Tohu,
\par 35 Syna Sufowego, syna Elkanowego, syna Machatowego, syna Amasajowego,
\par 36 Syna Elkanowego, syna Joelowego, syna Azaryjaszowego, syna Sofonijaszowego,
\par 37 Syna Tachatowego, syna Assyrowego, syna Abijasowego,
\par 38 Syna Korego, syna Isarowego, syna Kaatowego, syna Lewiego, syna Izraelowego.
\par 39 A brat jego Asaf, który stawal po prawicy jego. Asaf, syn Barachyjaszowy, syna Samaowego,
\par 40 Syna Michaelowego, syna Basejaszowego, syna Malchyjaszowego,
\par 41 Syna Etny, syna Zerachowego, syna Adajowego,
\par 42 Syna Etanowgo, syna Symmowego,
\par 43 Syna Semejowego, syna Jachatowego, syna Giersonowego, syna Lewiego.
\par 44 A synowie Merarego i bracia ich stawali po lewej stronie: Etan, syn Kuzego, syna Abdego, syna Malluchowego,
\par 45 Syna Hasabijaszowego, syna Amazyjaszowego, syna Helkijaszowego.
\par 46 Syna Amsego, syna Banego, syna Semmerowego,
\par 47 Syna Moholi, syna Musego, syna Merarego, syna Lewiego.
\par 48 A bracia ich Lewitowie postawieni sa ku wszelakiej posludze przybytku domu Bozego.
\par 49 Ale Aaron i synowie jego palili na oltarzu calopalenia, i na oltarzu kadzenia przy kazdej posludze swiatyni swietych, i ku oczyszczaniu Izraela podlug wszystkiego, jako byl przykazal Mojzesz, sluga Bozy.
\par 50 A ci sa synowie Aaronowi: Eleazar syn jego, Finees syn jego,
\par 51 Abisua syn jego, Bokki syn jego, Uzy syn jego, Zerachyjasz syn jego,
\par 52 Merajot syn jego, Amaryjasz syn jego, Achytob syn jego,
\par 53 Sadok syn jego, Achymaas syn jego.
\par 54 A te sa mieszkania ich, wedlug palaców ich w granicy ich, to jest, synów Aaronowych wedlug rodzaju Kaatytów: bo to byl ich los.
\par 55 Przetoz dali im Hebron w ziemi Judzkiej, i przedmiescia jego okolo niego;
\par 56 Ale pole miejskie i wsi ich dali Kalebowi, synowie Jefunowemu.
\par 57 Synom zas Aaronowym dali z miast Judzkich miasta ucieczki Hebron, i Lobne i przedmiescia jego, i Jeter, i Estemoa, i z przedmiesciami jego;
\par 58 I Holon i przedmiescia jego, i Dabir i przedmiescia jego;
\par 59 I Asan i przedmiescia jego, i Betsemes i przedmiescia jego.
\par 60 A z pokolenia Benjaminowego: Gabae i przedmiescia jego, i Almat i przedmiescia jego, i Anatot i przedmiescia jego. Wszystkich miast ich trzynascie miast wedlug domów ich.
\par 61 A synom Kaatowym, pozostalym z rodzaju tegoz pokolenia, dostalo sie w polowie pokolenia Manasesowego losem miast dziesiec.
\par 62 A synom Giersonowym wedlug domów ich dostalo sie w pokoleniu Isascharowem, i w pokoleniu Aserowem, i w pokoleniu Neftalimowem, i w pokoleniu Manasesowem w Bazan miast trzynascie.
\par 63 Synom Merarego wedlug domów ich dostalo sie w pokoleniu Rubenowem, i w pokoleniu Gadowem, i w pokoleniu Zabulonowem losem miast dwanascie.
\par 64 Dali tez synowie Izraelscy Lewitom miasta i przedmiescia ich;
\par 65 A dali je losem w pokoleniu synów Judowych, i w pokoleniu synów Symeonowych, i w pokoleniu synów Benjaminowych, miasta te, które nazwali imiony swemi.
\par 66 A tym, którzy byli z rodu synów Kaatowych, (a byly miasta i granice ich w pokoleniu Efraim.)
\par 67 Tym dali z miast ucieczki Sychem i przedmiescia jego na górze Efraim, i Gazer i przedmiescia jego.
\par 68 I Jekmaan i przedmiescia jego, i Betoron i przedmiescia jego;
\par 69 I Ajalon i przedmiescia jego, i Gatrymon i przedmiescia jego.
\par 70 A w polowie pokolenia Manasesowego: Aner i przedmiescia jego, Balam i przedmiescia jego. To dali rodzajowi pozostalemu synów Kaatowych.
\par 71 Synom tez Giersonowym z rodu polowy pokolenia Manasesowego dali Golan w Bazan i przedmiescia jego, i Astarot i przedmiescia jego;
\par 72 A w pokoleniu Isascharowem Kades i przedmiescia jego, Daberet i przedmiescia jego.
\par 73 I Ramot i przedmiescia jego, i Anam i przedmiescia jego.
\par 74 A w pokoleniu Aserowem Masal i przedmiescia jego, i Abdon i przedmiescia jego,
\par 75 I Hukok i przedmiescia jego, i Rohob i przedmiescia jego.
\par 76 A w pokoleniu Neftalimowem: Kades w Galilei, i przedmiescia jego, i Hammon i przedmiescia jego, i Kiryjataim i przedmiescia jego.
\par 77 Synom Merarego, pozostalym z pokolenia Zabulon, dane sa Remmon i przedmiescia jego, Tabor i przedmiescia jego.
\par 78 A za Jordanem u Jerycha na wschód slonca od Jordanu, dane sa w pokoleniu Rubenowem: Besor na puszczy, i przedmiescia jego, i Jahasa i przedmiescia jego.
\par 79 I Kiedemot i przedmiescia jego, i Mefaat i przedmiescia jego.
\par 80 A w pokoleniu Gadowem Ramot w Galaad i przedmiescia jego; i Mahanaim i przedmiescia jego.
\par 81 Hesebon i przedmiescia jego, i Jazer i przedmiescia jego.

\chapter{7}

\par 1 A synowie Isascharowi: Tola i Fua, Jasub, i Semram, czterej.
\par 2 A synowie Tolego: Uzy, i Rafajasz, i Jeryjel, i Jachamaj, i Jebsam, i Samuel; a cic byli ksiazetami wedlug domów ojców swych, którzy poszli z Tole, mezowie bardzo duzy wedlug narodów swych; poczet ich byl za dni Dawodowych dwadziescia i dwa tysiace i szesc set.
\par 3 A synowie Uzego: Izrahyjasz; a synowie Izrahyjaszowi: Michael i Obadyjasz, i Joel, Jesyjasz; piec ksiazat wszystkich.
\par 4 A z nimi w narodach ich, wedlug domów ojców ich, poczet mezów walecznych trzydziesci i szesc tysiecy; bo mieli wiele zon i synów.
\par 5 A braci ich wedlug wszystkich rodzajów Isascharowych, mezów duzych bylo osmdziesiat i siedm tysiecy, wszystkich policzonych.
\par 6 Synowie Benjaminowi: Bela i Bechor, i Jedyjael, trzej.
\par 7 Synowie zas Belego: Esbon i Uzy, i Uzyjel, i Jerymot, i Iry; piec ksiazat domów ojcowskich, mezów duzych; naliczono dwadziescia i dwa tysiace, i trzydziesci i cztery.
\par 8 A synowie Bechorowi; Zamirai i Joaz, i Eliezer, i Elienaj, i Amry, i Jerymot, i Abijasz, i Anatot, i Alamat; wszyscy ci synowie Bechorowi.
\par 9 A naliczono ich wedlug rodzajów ich, ksiazat domów ojców ich, mezów udatnych dwadziescia tysiecy i dwiescie.
\par 10 A synowie Jedyjaelowi: Bilan; a synowie Bilanowi: Jehus, i Banjamin, i Ehod, i Chanaan, i Zetan, i Tarsys, i Achysachar.
\par 11 Tych wszystkich synów Jadyjaelowych wedlug ksiazat domów ojcowskich, mezów bardzo duzych siedmnascie tysiecy i dwiescie, wychodzacych na wojne do bitwy;
\par 12 Oprócz Suppim i Ofim, synów w domu zrodzonych, i Husym, i synów w obcym kraju zrodzonych.
\par 13 A synowie Neftalimowi: Jachsel, i Guni, i Jesser, i Selem, synowie Bali.
\par 14 A synowie Manasesowi: Asryjel, którego mu urodzila Zona, (a zaloznica jego Syryjanka urodzila Machyra, ojca Galaadowego.
\par 15 A Machyr wzial sobie za zone siostre Ofimowa i Suppimowa, której imie bylo Maacha;) a imie drugiego Salfaad, i mial Salfaad córki.
\par 16 A urodzila Maacha, zona Machyrowa, syna, i nazwala imie jego Fares; a imie brata jego Sares, a synowie jego Ulam i Rekiem.
\par 17 A synowie Ulamowi Bedon. Cic sa synowie Galaada, syna Machyrowego, syna Manasesowego.
\par 18 A siostra jego Molechet urodzila Isoda, i Abiezera, i Machala.
\par 19 A synowie Semidowi byli Ahyjan, i Sechem, i Likchy, i Anijam.
\par 20 A synowie Efraimowi: Sutala, i Bered syn jego, i Tachat syn jego, i Elada syn jego, i Tachat syn jego;
\par 21 I Zabad syn jego, i Sutala syn jego, i Eser, i Elad. A pobili ich mezowie z Get, co sie byli zrodzili w onej ziemi; albowiem byli wtargneli, aby pobrali dobytki ich.
\par 22 Przetoz plakal Efraim, ojciec ich, przez wiele dni; i przyszli bracia jego, aby go cieszyli.
\par 23 Potem wszedl do zony swej, która poczela i porodzila syna, i nazwal imie jego Beryja, przeto iz sie urodzil w utrapieniu domu jego.
\par 24 Córke tez jego Seere, która pobudowala Betoron nizsze i wyzsze, i Uzenzeera;
\par 25 I Refacha syna jego, i Resefa, i Telacha syna jego, i Techena syna jego;
\par 26 I Laadana syna jego, Ammiuda syna jego, Elisama syna jego;
\par 27 Nuna syna jego, Jozuego syna jego.
\par 28 A osiadlosc ich i mieszkania ich, Betel i wsi jego; a na wschód slonca Naaran; a na zachód slonca Gazer i wsi jego, i Sychem i wsi jego, az do Aza i wsi jego.
\par 29 A podle miejsc synów Manasesowych: Betsan, i wsi jego, Tanach i wsi jego, Magieddon i wsi jego, Dor i wsi jego. W tych mieszkali synowie Józefowi, syna Izraelowego.
\par 30 Synowie Aserowi: Jemna i Jesua, Iswy i Beryja, i Sera, siostra ich.
\par 31 A synowie Beryjaszowi: Heber i Melchyjel; ten jest ojciec Birsawitów,
\par 32 A Heber splodzil Jafleta, i Somera, i Hotama, i Sue, siostre ich.
\par 33 A synowie Jafletowi: Pasach i Bimhal i Aswat. Cic sa synowie Jafletowi.
\par 34 A synowie Somerowi: Ahy i Rohaga, Jechuba i Aram.
\par 35 A synowie Helema, brata jego: Sofach, Jemna, i Seles, i Amal.
\par 36 Synowie Sofachowi: Suach, Harnefer, i Sual, i Bery, i Imra.
\par 37 Beser, i Hod, i Sema, i Silsa, i Jetram, i Bera.
\par 38 A synowie Jeterowi: Jefone, i Fispa, i Ara, i Ulla.
\par 39 A synowie Ullowi: Arach, i Haniel, i Rysyjasz.
\par 40 Ci wszyscy sa synowie Aserowi, ksiazeta domów rodzajów swych, wybrani i duzy mezowie, przedniejsi z ksiazat, którzy policzeni sa na wojne do bitwy; poczet tych mezów dwadziescia i szesc tysiecy.

\chapter{8}

\par 1 A Benjamin splodzil Bele, pierworodnego swego, Asbela wtórego, i Abracha trzeciego.
\par 2 Nocha czwartego, a Rafajasza piatego.
\par 3 A synowie Beli byli Addar i Giera i Abihud.
\par 4 I Abisua i Noaman i Achoach.
\par 5 I Giera i Sufam i Churam.
\par 6 A cic sa synowie Echudowi: ci sa ksiazetami narodów mieszkajacych w Gabaa, którzy je przeniesli do Manakat;
\par 7 To jest Noaman, i Achija, i Giera; on je przeniósl, a splodzil Uze, i Ahyhuda, i Sacharaima.
\par 8 A Sacharaim splodzil dzieci w krainie Moabskiej, gdy one byl odprawil, z Chusyma, i Bara, zonami swemi.
\par 9 Splodzil tedy z Hodes, zona swa, Jobaba, i Sebijasza, i Meze, i Malchama.
\par 10 I Jehusa, i Sachyjasza, i Mirme. Cic sa synowiejego, ksiazeta domów ojcowskich.
\par 11 A z Chysyma splodzil Abituba i Elfaala.
\par 12 A synowie Elfaalowi: Eber, i Misaam, i Samed, który zbudowal Ono, i Lod i wsi jego.
\par 13 A Beryja i Sama byli ksiazetami narodíw mieszkajacych w Ajalon; ci wygnali obywateli z Get.
\par 14 A Achyjo, Sesak i Jerymot,
\par 15 I Zabadyjasz, i Arad, i Hader,
\par 16 I MIchael, i Isfa, i Jocha, synowie Berajaszowi.
\par 17 A Zabadyjasz, i Mesullam, i Hyszki, i Heber,
\par 18 I Ismaraj, i Islijasz, i Jobab, synowie Elfaalowi.
\par 19 A Jakim, i Zychry, i Zabdy,
\par 20 I Elienaj, i Selataj, i Eliel,
\par 21 I Adajasz, i Berajasz, i Symrat, synowie Synchy.
\par 22 A Isfan, i Eber, i Eliel,
\par 23 I Abdon, i Zychry, i Chanan,
\par 24 I Hananijasz, i Eleam, i Anatotyjasz,
\par 25 I Ifdajasz, i Fanuel, synowie Sesakowi.
\par 26 I Samseraj, i Zecharyjasz, i Atalijasz,
\par 27 I Jaresyjasz, i Elijasz, i Zychry, synowie Jerochamowi.
\par 28 Ci sa ksiazeta domów ojcowskich wedlug rodzajów swych, a ci ksiazeta mieszkali w Jeruzalemie.
\par 29 A w Gabaonie mieszkal ojciec Gabaonczyków, a imie zony jego bylo Maacha.
\par 30 A syn jego pierworodny Abdon; po nim Sur, i Cys, i Baal, i Nadab.
\par 31 I Giedor, i Achyjo, i Zechar.
\par 32 Ale Michlot splodzil Symejasza; a ci takze naprzeciwko braci swych mieszkali w Jeruzalemie z bracmi swymi.
\par 33 A Ner splodzil Cysa, a Cys splodzil Saula; Saul zas splodzil Jonatana i Melchisuego, i Abinadaba i Esbaala.
\par 34 A syn Jonatanowy byl Merybbaal, a Merybbaal splodzil Michasa.
\par 35 A synowie Michasowi: Fiton i Melech i Tarea i Achaz.
\par 36 A Achaz splodzil Joada, a Joada splodzil Alemeta i Asmaweta i Zymrego, a Zymry splodzil Mose;
\par 37 A Mosa splodzil Bine; Refajasz syn jego, Elasa syn jego, Asel syn jego.
\par 38 Ten Asel mial szesc synów, a tec imiona ich: Asrykam, Bochru, i Ismael, i Searyjasz, i Obadyjasz, i Hanan; ci wszyscy synowie Aselowi.
\par 39 A syowie Eseka, brata jego: Ulam pierworodny jego. Jehus wtóry, i Elifelet trzeci.
\par 40 A synowie Ulamowi byli mezowie duzy, i mocno luk ciagnacy, którzy mieli wiele synów i wnuków, az do stu i piecdziesiat. Ci wszyscy poszli z synów Benjaminowych.

\chapter{9}

\par 1 A tak wszyscy Izraelczycy obliczeni sa; a oto zapisani sa w ksiegach królów Izraelskich i Judzkich, a przeniesieni sa do Babilonu dla przestepstwa swego.
\par 2 Lecz którzy pierwsi mieszkali w osiadlosciach swych i w miastach swoich, Izraelczycy, kaplani, Lewitowie, i Netynejczycy.
\par 3 W Jeruzalemie mieszkali z synów Judowych, i z synów Benjaminowych, i z synów Efraimowych, i Manasesowych;
\par 4 Uttaj, syn Ammiuda, syna Amry, syna Imry, syna Bonny, z synów Faresa, syna Judowego.
\par 5 A z Sylona: Asajasz pierworodny, i synowie jego.
\par 6 A z synów Zerachowych: Jehuel i braci ich szesc set i dziewiecdziesiat.
\par 7 A z synów Benjaminowych: Salu, syn Mesullama, syna Hodowiego, syna Asenuowego.
\par 8 A Ibnijasz, syn Jerochamowy, i Ela, syn Uzego, syna Michry, i Mesullam, syn Sefatyjasza, syna Rehuelowego, syna Ibnijaszowego.
\par 9 Takze braci ich wedlug narodów ich bylo dziewiec set i piecdziesiat i szesc: ci wszyscy mezowie byli ksiazetami rodzajów wedlug domów ojców swoich.
\par 10 Z kaplanów zasie: Jedejasz, i Jechojaryk, i Jachyn,
\par 11 I Azaryjasz, syn Helkijasza, syna Mesullamowego, syna Sadokowego, syna Merajatowego, syna Achytobowego, byl ksiazeciam domu Bozego.
\par 12 I Adajasz, syn Jerohama, syna Fassurowego, syna Melchyjaszowego, i Maasaj, syn Adyjela, syna Jechserowego, syna Mesulla, owego, syna Mesullemitowego, syna Immerowego.
\par 13 A braci ich ksiazet wedlug domów ojców ich bylo tysiac i siedm set i szescdziesiat, mazów duzych ku sprawowaniu poslugi w domu Bozym.
\par 14 A z Lewitów: Semejasz, syn Hassuba, syna Asrykamowego, syna Hasabijaszowego, z synów Merarego;
\par 15 I Bakkabar, Cheres, i Galal, i Matanijasz, syn Michy, syna Zychrego, syna Asafowego;
\par 16 Obadyjasz tez, syn Semahajasza, syna Galalowego, syna Jedytunowego, i Barachyjasz, syn Asy, syna Elkanowego, który mieszkal we wsiach Netofatyckich.
\par 17 A odzwierni: Sallum, i Akkub, i Talmon, i Ahyman, i nracia ich; Sallum ksiaze miedzy nimi.
\par 18 Który az dotad w bramie królewskiej stawal na wschód slonca; ci byli odzwiernymi wedlug pocztów synów Lewiego.
\par 19 Ale Sallum, syn Korego, syna Abijazafowego, syna Korego, i bracia jego z domu ojca jego, Korytczycy byli ku odprawowaniu poslugi, strózami progów namiotu; a ojcowie ich byli nad obozem Panskim strózami wejscia.
\par 20 A Finees syn Eleazarowy, byl ksiazeciem nad nimi, a Pan byl z nimi.
\par 21 Zacharyjasz zasie, syn Mesellemijaszowy, byl odzwiernym drzwi u namiotu zgromadzenia.
\par 22 Cic wszyscy sa obrani za odzwiernych do drzwi, dwiescie osób i dwanascie: ci we wsiach swych policzeni sa, których postanowil Dawid i Samuel widzacy, dla wiernosci ich,
\par 23 Aby oni i synowie ich byli we drzwiach domu Panskiego, w domu namiotu na strazy.
\par 24 I byli odzwierni po czterech stronach, na wschód, na zachód, na pólnocy, i na poludnie.
\par 25 Bracia zasie ich byli we wsiach swych, przychodzac kazdego siódmego dnia, od czasu az do czasu odmieniajac sie z nimi.
\par 26 Albowiem pod sprawa tych czterech przedniejszych odzwiernych byli Lewitowie, a byli przelozeni nad gmachami i nad skarbami domu Bozego;
\par 27 A okolo domu Bozego nocowali, gdyz im nalezala straz jego; a oni go na kazdy poranek otwierali.
\par 28 A z nich niektórzy byli nad naczyniem ku poslugiwaniu; albowiem pod liczba wnosili je, i pod liczba wynosili je.
\par 29 Niektózry zasie z nich byli postanowieni nad innem naczyniem, i nad wszystkiem naczyniem swiatnicy, i nad maka pszenna i winem, i oliwa, i kadzidlem, i nad rzeczami wonnemi.
\par 30 A niektórzy z synów kaplanskich sprawowali masci z rzeczy wonnych.
\par 31 Matatyjasz tez z Lewitów, pierworodny Salluma Korytczyka, byl przelozony nad rzeczami, które w panwiach smazono.
\par 32 A z synów Kaatowych z braci ich, byli niektórzy przelozonymi nad chlebami pokladnemi, aby je gotowali na kazdy sabat.
\par 33 A z tych byli spiewacy, przedniejsi z domów ojcowskich, miedzy Lewitami mieszkajacy w gmachach, od inszych prac wolni; bo we dnie i w nocy powinnosci swej pilnowac musieli.
\par 34 Ci przedniejsi z domów ojcowskich miedzy Lewitami, wedlug narodów swych przedniejsi; ci mieszkali w Jeruzalemie.
\par 35 A w Gabaonie mieszkali ojciec Gabaonczyków Jehyjel, a imie zony jego Maacha;
\par 36 A syn jego pierworodny Abdon, po nim Sur, i Cys, i Baal, i Neer, i Nadab,
\par 37 I Giedeor, i Achyjo, i Zacharyjasz, i Michlot;
\par 38 (A Michlot splodzil Symmama) a ci takze przweciw braci swych mieszkali w Jeruzalemie z bracmi swymi.
\par 39 A Neer splodzil Cysa, a Cys splodzil Saula, a Saul splodzil Jonatana, i Melchisuego, i Abidanaba, i Esbaala.
\par 40 A syn Jonatana Merybbaal; a Merybbaal splodzil Michasa.
\par 41 Synowie zas Michasowi: Fiton, i Melech, i Tarea, i Achaz.
\par 42 A Achaz splodzil Jare; a Jara splodzil Alemeta, i Asmaweta, i Zynrego; a Zymry splodzil Mose.
\par 43 A Mosa splodzil Bine; a Refajasz syn jego, Elasa syn jego, Asel syn jego.
\par 44 A Asel mial szesc synów; a tec imiona ich: Asrykam, i Bochru, i Ismael, i Searyjasz, i Obadyjasz, i Hanan. Cic sa synowie Aselowi.

\chapter{10}

\par 1 A gdy Filistynowie walczyli z Izraelem, uciekli mezowie Izraelscy przed Filistynami a polegli, bedac porazeni na górze Gielboe.
\par 2 I gonili Filistynowie Saula i synów jego; i zabili Filistynowie Jonatana, i Abinadaba, i Melchisuego, synów Saulowych.
\par 3 A gdy sie zmocnila bitwa przeciw Saulowi, trafili na niego strzelcy, i z luku zraniony jest od strzelców.
\par 4 Rzekl tedy Saul do slugi swego, co za nim bron nosil: Dobadz miecza twego, a przebij mie nim, by snac nie przyszli ci nieobrzezancy, a nie posmiewali sie ze mnie. Ale nie chcial sluga, który nosil bron jego; bo sie bardzo bal. Przetoz porwawszy Saul miech, padl nan.
\par 5 A widzac sluga, co nosil bron jego, ze umarl Saul, padlszy tez i sam na miecz umarl.
\par 6 A tak umarl Saul, i trzej synowie jego, i wszystek dom jego z nim pospolu zginal.
\par 7 Co gdy ujrzeli wszyscy mezowie Izraelscy, którzy mieszkali na dolinie, iz uciekli Izraelczycy, a iz pomarli Saul i synowie jego, opusciwszy miasta swe takze uciekli. I przyszli Filistynowie, i mieszkali w nich.
\par 8 A gdy nazajutrz przyszli Filistynowie brac lupy z pobitych, znalezli Saula, i synów jego, lezacych na górze Gielboe;
\par 9 A zlupiwszy go wzieli glowe jego, i zbroje jego, i poslali po ziemi Filistynskiej w okolo, aby to ogloszone bylo przed balwany ich, i przed ludem.
\par 10 I polozyli zbroje jego w domu boga swego, a glowe jego zawiesili w domu Dagonowym.
\par 11 Uslyszawszy tedy wszyscy mezowie Jabes Galaad wszystko, co uczynili Filistynowie Saulowi,
\par 12 Powstali wszyscy mezowie mocni, i wzieli cialo Saulowe, i cialo synów jego, a przynióslszy do Jabes pogrzebli kosci ich pod debem w Jabes, i poscili przez siedm dni.
\par 13 A tak umarl Saul dla przestepstwa swego, którem byl wystapil przeciwko Panu, i przeciwko slowu Panskiemu, któego nie przestrzegal, iz sie radzil ducha wieszczego, pytajac sie go;
\par 14 A iz sie nie radzil Pana, zabil go, a przeniósl królestwo na Dawida, syna Isajego.

\chapter{11}

\par 1 Zebral sie tedy wszystek Izrael do Dawida do Hebronu, mówiac: Otosmy kosc twoja i cialo twoje!
\par 2 Jako i przedtem, gdy jeszcze byl Saul królem, tez wywodzil i wwodzil Izraela. Tak Pan, Bóg twój, rzekl tobie: Ty bedziesz pasl lud mój Izraelski, a ty bedziesz wodzem nad ludem moim Izraelskim.
\par 3 A tak przyszli wszyscy starsi Izraelscy do króla do Hebronu, i uczynil Dawid z nimi przymierze w Hebronie przed Panem, i pomazali Dawida za króla nad Izraelem wedlug slowa Panskiego, które powiedzial przez Samuela.
\par 4 Jechal tedy Dawid ze wszystkim Izraelem do Jeruzalemu, które jest Jebus, gdzie byli Jebuzejczycy obywatelami ziemi.
\par 5 I rzekli obywatele Jebuzejscy do Dawida: Nie wnijdziesz sam. Ale Dawid wzial zamek Syonski, który jest miastem Dawidowem.
\par 6 Bo byl rzekl Dawid: Ktobykolwiek porazil Jebuzejczzyka najpierwej, ten bedzie skiazeciem i hetmanem. Przetoz wstapil najpierw Joab, syn Sarwii, i zostal hetmanem.
\par 7 I mieszkal Dawid na onym zamku; dla tego nazwano go miastem Dawidowem.
\par 8 I zbudowal miasto w okolo, od Mello az w okrag; a Joab pobudowal ostatek miasta.
\par 9 A tak Dawid im dalej, tem wiecej rozmnazal sie, i rosl; albowiem Pan zastepów byl z nim.
\par 10 A cic sa najorzedniejsi rycerze, których mial Dawid, którzy sie meznie starali z nim o królestwo jego ze wszystkim Izraelem, aby go królem uczynili wedlug slowa Panskiego nad Izraelem.
\par 11 A tenci jest poczet rycerzy, których mial Dawid: Jasobam, syn Chachmonowy, przedniejszy miedzy trzydziestoma; ten podnióslszy oszczep swój na trzystu, jednym razem ich zabil.
\par 12 A po nim Elezar, syn Dodonowy, Achochytczyk; ten byl jednym miedzy trzema mocarzami.
\par 13 Ten byl z Dawidem w Pasdamim, gdzie sie zebrali byli Filistynowie ku bitwie; a byla ona czesc pola pelna jeczmienia, a lud byl uciekl przed Filistynami.
\par 14 I staneli w posród onego pola, i obronili go, a porazili Filistynów: i wybawil Pan lud wybawieniem wielkiem.
\par 15 Ci takze trzej ze trzydziestu przedniejszych wstapili na skale do Dawida do jaskini Odollam, gdyz wojsko Filistynskie lezalo obozem w dolinie Rafaim;
\par 16 Albowiem Dawid natenczas mieszkal na zamku, a stanowisko Filistynskie bylo natenczas w Betlehem.
\par 17 Pragnel tedy Dawid: Oby mi sie kto dal napic wody z studni Betlehemskiej, która jest u bramy!
\par 18 Przetoz przebiwszy sie ci trzej przez wojsko Filistynskie, naczerpali wody z studni Betlehemskiej, która jest u bramy, a wziawszy przyniesli do Dawida. Lecz jej nie chcial Dawid pic, ale ja wylal na ofiare Panu.
\par 19 I rzekl: Nie daj mi tego, Boze mój, abym to uczynic mial! Izali krew tych mezów pic beda, którzy odwazyli zywot swój? albowiem z odwaga zywota swego przyniesli ja; i nie chcial jej pic. Toc uczynili trzej oni mocarze.
\par 20 A Abisaj, brat Joabowy, byl przedniejszy z onych trzech; tenze podniósl wlócznie swa na trzysta ludu, które pobil, i otrzymal slawe miedzy onymi trzema.
\par 21 Z tych trzech nad innych dwóch byl slawniejszy, a byl ich ksiazeciem; jednak onych trzech pierwszych nie doszedl.
\par 22 Banajas tez, syn Jojady, syn meza duzego, wielkich spraw, z Kabseela, ten zabil dwóch mocarzów Moabskich; ten tez zszedlszy zabil lwa w posród jamy, gdy byl snieg.
\par 23 Ten tez zabil meza Egipczanina, meza, którego wzrost byl na piec lokci. A chociaz Egipczanin mial w reku oszczep jako nawój tkacki, wszkze przyszedl do niego z kijem, i wydarl oszczep z reki Egipczanina, i zabil go oszczepem jego.
\par 24 To uczynil Banajas, syn Jojady, który takze slawnym zostal miedzy onymi trzema mocarzami.
\par 25 A choc byl miedzy onymi trzydziestoma slawnym, wszakze nie doszedl onych trzech. I postanowil go Dawid nad drabantami swymi.
\par 26 A w wojsku co mocniejsi byli: Asael, brat Joabowy, Elkanan, syn Dodonowy z Betlehem;
\par 27 Sammot Harodczyk, Heles Felonitczyk;
\par 28 Hyra, syn Ikkiesowy, Tekuitczyk, Abiezer Anatotczyk;
\par 29 Sybbechaj Husatczyk, Ilaj Ahohytczyk;
\par 30 Maharaj Netofatczyk, Heled, syn Baamy, Netofatczyk;
\par 31 Itaj, syna Rybajego, z Gabaat synów Benjaminowych, Banajas Faratonczyk;
\par 32 Hutaj od potoku Gaas; Abiel Arbatczyk;
\par 33 Asmawet Bacharomczyk; Elijachba Salabonczyk.
\par 34 Synowie Asema Gisonczyka: Jonatan, syn Sagii, Hororczyk;
\par 35 Ahijam, syn Zacharowy, Ararytczyk, Elifal, syn Urowy;
\par 36 Hefer Mecheratczyk, Achijas Felonitczyk;
\par 37 Hesro Karmelczyk, Naaraj, syn Ezbajowy;
\par 38 Joel, brat Natanowy, Michbar, syn Gierego.
\par 39 Selek Ammonitczyk, Nacharaj Berotczyk, który nosil bron Koaba, syna Sarwii;
\par 40 Hyra Itrejczyk, Gareb Itrejczyk;
\par 41 Uryjasz Hetejczyk, Zabad, syn Achalajego.
\par 42 Adyna, syn Sysy, Rubenitczyk, ksiaze Rubenitów, a z nim trzydziesci mezów.
\par 43 Hanan, syn Maachy, i Jozafat Mitnitczyk.
\par 44 Uzyjasz Asteratczyk, Sama i Jehijel, synowie Hotamy Aroerytczyka.
\par 45 Jedinael, syn Symry, i Jocha, brat jego, Tysytczyk.
\par 46 Eliel Machawimczyk, i Jerybaj, i Josawijasz, synowie Elnaamowi, i Itma Moabczyk.
\par 47 Eliel, i Obed, i Jaasyjel z Mezobaj.

\chapter{12}

\par 1 A cic sa, co byli przyszli do Dawida do Sycelegu, gdy sie jeszcze kryl przed Saulem, synem Cysowym; a ci byli miedzy mocarzami posilek dawajacy w bitwie,
\par 2 Noszacy luk, a prawa i lewa reka ciskajacy kamienmi, i strzelajacy z luku, a byli z braci Saulowych z pokolenia Benjaminowego:
\par 3 Ksiaze Achyjezer, i Joaz, synowie Semmai Gabatczyka, i Jezyjel, i Falet, synowie Azmawetowi, i Baracha, i Jehu Anatotczyk;
\par 4 Ismajasz tez Gabaonczyk, mezny miedzy trzydziestoma, a byl przelozony nad trzydziestoma; i Jeremijasz, i Jahazyjel, Johanan, i Jozabad Gliederatczyk;
\par 5 Eluzaj, i Jerymot, i Bealijasz, i Semaryjasz, i Sefatyjasz Harufitszyk;
\par 6 Elkana, i Jesyjasz, i Asareel i Joezer, i Jasobam Korchytczyk;
\par 7 I Joela, i Zebadyjasz, synowie Jerohamowi z Giedor.
\par 8 A z pokolenia Gadowego zbiegli byli do Dawida na miejsce obronne na puszcze mezowie duzy, mezowie sposobni do boju, noszacy tarcz i kopije, których twarze byly jako lwie tarze, a jako sarny po górach predcy;
\par 9 Eser przedniejszy, Obadyjasz wtóry, Elijab trzeci,
\par 10 Mismanna czwarty, Jeremijasz piaty,
\par 11 Ataj szósty, Eliel siódmy,
\par 12 Jochanan ósmy, Elzebad dziewiaty,
\par 13 Jeremijasz dziesiaty, Machbanajasz jedenasty.
\par 14 Cic byli z synów Gadowych, hetmani wojska, jeden nad stem mniejszy, a wiekszy nad tysiacem.
\par 15 Cic sa, którzy przeszli Jordan miesiaca pierwszego, który byl wylal ze wszystkich brzegów swoich; i wygnali wszystkich mieszkajacych w dolinach na wschód i na zachód slonca.
\par 16 Przyszli takze niektórzy z synów Benjaminowych i z Judowych, do miejsca obronnego, do Dawida.
\par 17 I wyszedl Dawid przeciwko nim a odpowiadajac, rzekl im: Jezliscie spokojnie przyszli do mnie, abyscie mie ratowali, serce tez moje zlaczy sie z wami; ale jezliscie przyszli, abyscie mie wydali nieprzyjaciolom moim, (choc nie masz nieprawosci przy mnie) niech w to wejrzy Bóg ojców naszych, a niech sadzi.
\par 18 Tedy Duch przyoblekl Amazyjasza, przedniejszego miedzy hetmanami, i rzekl: Twoismy, o Dawidzie! a z toba przestajemy, synu Isajego. Pokój, pokój tobie, i pokój pomocnikom twoim! gdyz ci pomaga Bóg twój. A tak przyjal ich Dawid, i postanowil ich hetmanami wojska.
\par 19 A z pokolenia Manasesowego odpadli niektórzy do Dawida, gdy ciagnal z Filistynami przeciwko Saulowi na wojne; ale im nie byli na pomocy, gdyz naradziwszy sie ksiazeta Filistynscy odeslali go, mówiac: Ten z niebezpieczenstwem glów naszych odpadnie do Saula, pana swego.
\par 20 Gdy tedy szedl do Syceleu, uciekli do niego niektórzy z pokolenia Manasesowego: Adnach i Josabad, i Jediael, i Michael, i Jozabad i Elihu, i Sylletaj, i hetmani nad tysiacami w pokoleniu Manasesowem.
\par 21 A ci posilkowali Dawida przeciw onemu hufowi; bo mezni byli wszyscy, przetoz byli hetmanami w wojsku jego.
\par 22 Nawet na kazdy dzien sciagali sie do Dawida na pomoc jemu, az bylo wojsko wielkie jako wojsko Boze.
\par 23 A tac jest liczba przedniejszych gotowych do boju, którzy przyszli do Dawida do Hebronu, aby przeniesli królestwo Saulowe do niego wedlug slowa Panskiego.
\par 24 Z synów Judowych, noszacych tarcz i wlócznie, szesc tysiecy i osm set gotowych do boju.
\par 25 Z synów Symeonowych, meznych do boju, siedm tysiecy i sto.
\par 26 Z synów Lewiego cztery tysiace i szesc set.
\par 27 Jojada takze przedniejszy z synów Aaronowych, a z nim trzy tysiace i siedm set.
\par 28 A Sadok mlodzieniec, rycerz mezny, i z domu ojca jego ksiazat dwadziescia i dwóch.
\par 29 A z synów Benjaminowych, braci Saulowych, trzy tysiace; bo jeszcze wielka czesc ich przestawala z domem Saulowym.
\par 30 A z synów Efraimowych dwadziescia tysiecy i osm set, ludzi meznych, mezów slawnych w domach ojców ich.
\par 31 A z polowy pokolenia Manasesowego osmnascie tysiecy, którzy byli mianowani wedlug imion, aby przyszli i postanowili Dawida królem.
\par 32 A z synów Isascharowych, umiejacych rozeznawac czasy, tak iz wiedzieli, co kiedy czynic mial Izrael, ksiazat ich dwiescie; a wszyscy bracia ich przestawali na radzie ich.
\par 33 Z pokolenia Zabulonowego, którzy wychodzili na wojne, gotowych do boju z kazdym orezem wojennym, piecdziesiat tysiecy, stawajacych w szyku jednostajnem sercem.
\par 34 A z pokolenia Neftalimowego ksiazat tysiac, a z nimi z tarczami i z kopijami trzydziesci i siedm tysiecy.
\par 35 A z pokolenia Danowego, gotowych do boju, dwadziescia i osm tysiecy i szesc set.
\par 36 A z pokolenia Aserowego, którzy wychodzili na wojne, i umieli sie szykowac do bitwy, czterdziesci tysiecy.
\par 37 A z Za-Jordania z pokolenia Rubenowego i Gadowego, i z polowy pokolenia Manasesowego ze wszystkim orezemwojennym sto i dwadziescia tysiecy.
\par 38 Ci wszyscy mezowie waleczni sprawni ku bitwie, sercem uprzejmem przyszl do Hebronu, aby postanowili Dawida królem nad wszystkim Izraelem. Nadto i wszyscy inni z Izraela jednego serca byli, aby postanowili królem Dawida.
\par 39 I byli tam z Dawidem przez trzy dni jedzac i pijac: bo im byli nagotowali bracia ich.
\par 40 Takze i którzy blisko ich byli az do Isaschar i Zabulon i Neftalim, przynosili chleby na oslach, i na wielbladach, i na mulach, i na wolach, potrawy, maki, figi, rodzynki, i wino, i oliwe, i wolów, i owiec wielkim dostatkiem; bo byla radosc w Izraelu.

\chapter{13}

\par 1 A Dawid wszedl w rade z hetmanami nad tysiacami, i z setnikami, i ze wszystkimi rotmistrzami.
\par 2 I mówil Dawid do wszystkiego zgromadzenia Izraelkskiego: Jezli sie wam podoba, i jezli to jest od Pana Boga naszego, rozeslijmy wszedy do braci naszych pozostalych po wszystkich krainach Izraelskich; przytem tez do kaplanów i Lewitów po miastach i przedmiesciach ich, a niech sie zgromadza do nas:
\par 3 Abysmy przeprowadzili skrzynie Boga naszego do nas; albowiem nie pytalismy sie o niej za dni Saulowych.
\par 4 I rzeklo wszystko zgromadzenie, aby sie tak stalo; bo sie ta rzecz podobala wszystkiemu ludowi.
\par 5 Zebral tedy Dawid wszystkiego Izraela od Nilu Egipskiego az gdzie sie chodzi do Emat, aby przyprowadzili skrzynie Boza z Karyjatyjarym.
\par 6 A tak przyszedl Dawid, i wszystek Izrael do Baala w Karyjatyjarym, które jest w Judzie, aby przyprowadzili stamtad skrzynie Pana Boga siedzacego nad Cherubinami, gdzie wzywane bywa imie jego.
\par 7 I wiezli skrzynie Boza na wozie nowym z domu Abinadabowego, a Oza i Achyjo prowadzili wóz.
\par 8 Lecz Dawid i wszystek Izrael grali przed Bogiem ze wszystkiej mocy, i piesniami, i na harfach, i na cytrach, i na bebnach, i na cymbalach, i na trabach.
\par 9 A gdy przyszli na bojewisko Chydon, sciagnal Oza reke swa, aby zadzierzal skrzynie; bo woly byly wystapily z drogi.
\par 10 I rozgniewal sie bardzo Pan na Oze, i zabil go, przeto iz sciagnal reke swa ku skrzyni; i umarl tamze przed Bogiem.
\par 11 I zafrasowal sie Dawid, iz to rozerwanie Pan uczynil w Ozie; a przetoz nazwal ono miejsce Peres Oza, az do dnia tego.
\par 12 I ulakl sie Dawid Boga dnia onego, a rzekl: Jakoz mam wprowadzic do siebie skrzynie Boza?
\par 13 Przetoz nie wprowadzil Dawid skrzyni do siebie, do miasta Dawidowego; ale ja wprowadzil do domu Obededoma Gietejczyka.
\par 14 I zostala skrzynia Boza miedzy domownikami Obededomowymi w domu jego przez trzy miesiace. I blogoslawil Pan domowi Obededomowemu i wszystkiemu, co mial.

\chapter{14}

\par 1 Potem poslal Hiram, król Tyrski, poslów do Dawida, i drzewa cedrowe, i murarzy i ciesli, aby mu zbudowali dom.
\par 2 I poznal Dawid, iz go utwierdzil Pam królem nad Izraelem, a iz wywyzszyl królestwo jego dla ludu swego Izraelskiego.
\par 3 (I pojal Dawid wiecej zon w Jeruzalemie, a splodzil Dawid wiecej synów i córek.
\par 4 A tec sa imiona tych, którzy mu sie urodzili w Jeruzalemie: Samna, i Sobab, Natan, i Salomon,
\par 5 I Ibchar, i Elisua, i Elfalet,
\par 6 I Noga, i Nefeg, i Jafija,
\par 7 I Elisama, i Beelijada, i Elifelet.)
\par 8 Wtem uslyszawszy Filistynowie, ze byl pomazany Dawid za króla nad wszystkim Izraelem, wyciagneli wszyscy Filistynowie, aby szukali Dawida. Co gdy uslyszal Dawid, wyszedl przeciwko nim.
\par 9 Bo Filistynowie przyciagnawszy rozpostarli sie w dolinie Rafaim.
\par 10 I radzil sie Dawid Boga, mówiac: Mamli isc przeciw Filistynom? a podaszli ich w rece moje? I odpowiedzial mu Pan: Idz a podam ich w rece twoje.
\par 11 A gdy oni przyciagneli do Baal Perazym, porazil ich tam Dawid, i rzekl Dawid: Rozerwal Bóg nieprzyjacioly moje przez reke moje, jako sie rozrywaja wody; a przetoz nazwano imie miejsca onego Baal Perazym.
\par 12 I zostawili tam bogi swoje; a Dawid rozkazal, aby je spalono ogniem.
\par 13 Lecz zebrawszy sie znowu Filistynowie rozpostarli sie w dolinie.
\par 14 Przetoz radzil sie znowu Dawid Boga. I rzekl mu Bóg: Nie ciagnij za nimi, ale sie odwróc od nich, abys na nich natarl przeciwko morwom.
\par 15 A gdy uslyszysz szum idacego po wierzchach morwowych, tedy wynijdziesz do bitwy; bo wyszedl Bóg przed toba, aby porazil wojska Filistynskie.
\par 16 I uczynil Dawid, jako mu byl rozkazal Bóg; i porazili wojska Filistynskie od Gabaon az do Gazer.
\par 17 A tak rozslawilo sie imie Dawidowe po wszystkich ziemiach: i sprawil to Pan, ze byl straszny wszystkim narodom.

\chapter{15}

\par 1 A gdy sobie pobudowal Dawid domy w miescie swojem, i nagotowal miejsce skrzyni Bozej, i rozbil jej namiot,
\par 2 Tedy rzekl Dawid: Niegodzi sie nosic skrzyni Bozej jedno Lewitom. Tych bowiem obral Pan, aby nosili skrzynie Boza, i sluzyli mu az na wieki.
\par 3 Przetoz zebral Dawid wszystkiego Izraela do Jeruzalemu, aby przeniósl skrzynie Panska na miejsce jej, które jej byl zgotowal.
\par 4 I zgromadzil Dawid synów Aaronowych i Lewitów.
\par 5 Z synów Kaatowych: Uryjela przedniejszego, i braci jego sto i dwadziescia,
\par 6 Z synów Merarego: Asajasza przedniejszego, i braci jego dwiescie i dwadziescia.
\par 7 Z synów Giersonowych: Joela przedniejszego, i braci jego sto i trzydziesci.
\par 8 Z synów Elisafanowych: Semejasza przedniejszego, i braci jego dwiescie.
\par 9 Z synów Hebronowych: Elijela przedniejszego, i braci jego osmdziesiat.
\par 10 Z synów Hasyjelowych: Aminadaba przedniejszego, i braci jego sto i dwanascie.
\par 11 Tedy wezwal Dawia Sadoka i Abijatara, kaplanów, takze Lewitów Uryjela, Asajasza, i Joela, Semejasza, i Elijela, i Aminadaba;
\par 12 I rzekl do nich: Wyscie przedniejsi z domów ojcowskich miedzy Lewitami; poswiecciez sie i z bracmi swoimi, abyscie przyniesli skrzynie Pana, Boga Izraelskiego, na miejsce, którem jej nagotowal.
\par 13 Albowiem izescie tego pierwej nie uczynili, uczynil rozerwanie Pan, Bóg nasz, miedzy nami; bosmy go nie szukali wedlug przystojnosci.
\par 14 Poswiecili sie tedy kaplani i Lewitowie, aby przyniesli skrzynie Pana, Boga Izraelskiego.
\par 15 I niesli synowie Lewitów skrzynie Boza, jako byl rozkazal Mojzesz wedlug slowa Panskiego, na ramionach swych, na drazkach które przy niej byly.
\par 16 I rzekl Dawid przedniejszym z Lewitów, aby postanowili z braci swoich spiewaków z instrumentami muzycznemi, z lutniami, z cytrami, i z cymbalami, aby slyszany byl wyniesiony glos z weselem.
\par 17 I postanowili Lewitowie Hemana, syna Joelowego, a z braci jego Asafa, syna Barachyjaszowego, a z synów Merarego, braci ich, Etana syna Chysajowego;
\par 18 A z nimi braci ich w rzedzie wtórym: Zacharyjasza, Bena, i Jazyjela, i Semiramota, i Jechyjela, i Unni, Elijaba, i Benajasza, Maasejasza, i Matytyjasza, i Elifelego, i Miknejasza, i Obededoma, i Jehijela, odzwiernych.
\par 19 A spiewacy Heman, Asaf, i Etan na cymbalach miedzianych glosno grali.
\par 20 A Zacharyjasz, i Jazyjel, i Semiramot, i Jechyjel, i Unni, i Elijab, i Maasejasz, i Benajasz grali na lutniach przy spiewaniu wysokiem.
\par 21 A Matytyjasz, i Elifele, i Miknejasz, i Obededom, i Jechyjel, i Azazyjasz grali na cytrach przy spiewaniu niskiem.
\par 22 A Kienanijasz, przedniejszy z Lewitów, którzy niesli skrzynie, rozrzadzal, jakoby niesc miano; bo byl roztropny.
\par 23 Ale Bacharyjasz i Elkana byli odzwiernymi u skrzyni.
\par 24 A Sebanijasz i Jozafat, i Natanael, i Amasaj, i Zacharyjasz, i Benajasz, i Eliezer, kaplani, trabili w traby przed skrzynia Boza; ale Obededom i Jechyjasz byli odzwiernymi u skrzyni.
\par 25 A tak Dawid i starsi Izraelscy, i hetmani nad tysiacami szli, aby przeprowadzili skrzynie przymierza Panskiego z domu Obededomowego z weselem.
\par 26 I stalo sie, gdy Bóg wspomógl Lewitów, niosacych skrzynie przymierza Panskiego, ze ofiarowali siedm wolów, i siedm baranów.
\par 27 A Dawid byl obleczony w szate bisiorowa, takze i wszyscy Lewitowie, którzy niesli skrzynie, i spiewacy, i Kienanijasz, rzadca tych, co niesli miedzy spiewakami; a Dawid mial na sobie efod lniany.
\par 28 A tak wszystek Izrael prowadzil skrzynie przymierza PaOskiego z weselem, i z hukiem kornetu, i traby, i cymbalów, grajac na lutniach i na cytrach.
\par 29 I stalo sie, gdy skrzynia przymierza Panskiego wchodzila do miasta Dawidowego, ze Michol, córka Saulowa, wygladajac oknem ujrzala króla Dawida skaczacego, i grajacego, i wzgardzila go w sercu swojem.

\chapter{16}

\par 1 A gdy przyniesli skrzynie Boza, i postawili ja w posród namiotu, który byl Dawid rozbil, tedy sprawowali calopalenia i ofiary spokojne przed Bogiem.
\par 2 A gdy dokonczyl Dawid ofiarowac calopalenia, i ofiar spokojnych, blogoslawil ludowi w imie Panskie.
\par 3 I rozdzielil wszystkim mezom Izraelskim, od meza az do niewiasty, kazdemu po bochenku chleba, i po sztuce miesa, i po lagiewce wina.
\par 4 A postanowil przed skrzynia Panska z Lewitów slugu, aby wspominali, i wyznawali, i chwalili Pana, Boga Izraelskiego.
\par 5 Asaf byl przedniejszy, a wtóry po nim Zacharyjasz, Jehyjel, i Semiramot, i Jechyjel i Matytyjasz, i Elijab, i Benajasz, i Obededom, i Jechyjel; ci na instrumentach, na lutniach, na harfach, ale Asaf na cymbalach, grali.
\par 6 Benajasz zas i Jachazyjel kaplani z trabami ustawiczne byli przed skrzynia przymierza Bozego.
\par 7 Dopiero dnia onego najpierwej postanowil Dawid, aby tym psalmem chwalony byl Pan przez Asafa i braci jego:
\par 8 Wyslawiajcie Pana, wzywajcie imienia jego, a opowiadajcie miedzy narodami sprawy jego.
\par 9 Spiewajcie mu, grajcie mu, rozmawiajcie o wszystkich cudach jego.
\par 10 Chlubcie sie w imieniu swietem jego, a niech sie rozraduje serce szukajacych Pana.
\par 11 Szukajcie Pana, i mocy jego; szukajcie oblicza jego zawzdy.
\par 12 Wspominajcie dziwne sprawy jego, które czynil, i cuda jego, i sady ust jego.
\par 13 O nasienie Izraelskie, sludzy jego! O synowie Jakóbowi, wybrani jego!
\par 14 On jest Pan, Bóg nasz; po wszystkiej ziemi sady jego.
\par 15 Pamietajcie az na wieki na przymierze jego, na slowow, które przykazal do tysiacznego pokolenia;
\par 16 Które postanowil z Abrahamem, i na przysiege jego z Izaakiem;
\par 17 I postanowil to Jakóbowi za prawo, a Izraelowi za przymierze wieczne,
\par 18 Mówiac: Tobie dam ziemie Chananejska za sznur dziedzictwa waszego.
\par 19 Choc was byla mala liczba, a przez krótki czas byliscie przychodniami w niej;
\par 20 I przechodzili od narodu do narodu, i od królestwa do innego ludu:
\par 21 Nie dopuscil nikomu, aby ich uciskac mial, i karal dla nich królów,
\par 22 Mówiac:Nie tykajcie pomazanców moich, a prorokom moim nie czyncie nic zlego.
\par 23 Spiewajcie Panu wszystka ziemio; opowiadajcie ode dnia do dnia zbawienie jego.
\par 24 Opowiadajcie miedzy narodami chwale jego, i miedzy wszystkimi ludzmi dziwne sprawy jego;
\par 25 Bo wielki jest Pan, i chwalebny bardzo, i straszniejszy nad wszystkich bogów;
\par 26 Gdyz bogowie poganscy sa balwanami; ale Pan niebiosa uczynil.
\par 27 Slawa i zacnosc przed nim, moc i wesele na miejscu jego.
\par 28 Przyniescie Panu pokolenia narodów, przyniescie Panu chwale i moc.
\par 29 Przyniescie Panu chwale imienia jego, przyniescie dary, a przychodzcie przed oblicznosci jego; klaniajcie sie Panu w ozdobie swietobliwosci.
\par 30 Bójcie sie oblicza jego wszystka ziemio, a bedzie utwierdzony okrag ziemi, aby sie nie poruszyl.
\par 31 Niech sie rozraduja niebiosa, a niech sie rozweseli ziemia, a niech mówia w narodach: Pan króluje!
\par 32 Niech zaszumi morze, i ze wszystkiem, co w niem jest; niech sie rozraduje pole, i wszystko, co na niem jest.
\par 33 Tedy sie rozwesela drzewa lesne przed Panem; albowiem przyszedl sadzic ziemie.
\par 34 Wyslwiajcie Pana; albowiem dobry, bo na wieki trwa milosierdzie jego.
\par 35 A mówcie: Zachowaj nas, Boze zbawienia naszego! i zgromadz nas, a wyrwij nas od pogan, abysmy wielbili imie swiete twoje, i chlubili sie w chwale twojej.
\par 36 Blogoslawiony Pan, Bóg Izraelski, od wieków az na wieki. I rzekl wszystek lud Amen, i chwalil Pana.
\par 37 I zostawil tam Dawid przed skrzynia przymierza Panskiego Asafa i braci jego, aby sluzyli przed skrzynia ustawicznie wedlug potrzeby dnia kazdego.
\par 38 Lecz Obededoma i braci ich szesdziesiat i osm, Obededoma mówie, syna Jedytunowego, i Hose, uczynil odzwiernymi.
\par 39 A Sadoka kaplana, i braci jego kaplanów postawil przed przybytkiem Panskim na wyzynie, która byla w Gabaon,
\par 40 Aby ofiarowali calopalenia Panu na oltarzu calopalenia ustawicznie rano i w wieczór, a to wedlug wszystkiego, co napisano w zakonie Panskim, który przykazal Izraelowi.
\par 41 A z nimi Hemana i Jedytuna, i innych na to obranych, którzy byli z imienia mianowani, aby chwalili Pana, przeto iz na wieki trwa milosierdzie jego.
\par 42 A miedzy nimi Heman i Jedytun, trabili i grali na trabach, na cymbalach, i na innych instrumentach muzycznych Bogu; ale synów Jedytunowych postawil u wrót.
\par 43 A tak rozszedl sie wszystek lud, kazdy do domu swego. Dawid sie tez wrócil, aby blogoslawil domowi swemu.

\chapter{17}

\par 1 I stalo sie, gdy mieszkal Dawid w domu swym, ze rzekl do Natana proroka: Oto ja mieszkam w domu cedrowym, a skrzynia przymierza Panskiego pod kortynami.
\par 2 I rzekl Natan do Dawida: Cokolwiek jest w sercu twem, uczyn, gdyz Bóg jest z toba.
\par 3 Potem onej nocy stalo sie slowo Boze do Natana, mówiac:
\par 4 Idz, a mów do Dawida, slugi mego: Tak mówi Pan: Nie ty mi bedziesz budowal domu do mieszkania;
\par 5 Poniewazem nie mieszkal w domu ode dnia, któregom wywiódl synów Izraelskich, az do dnia tego: alem sie przechadzal z namiotu do namiotu, i z przybytku do przybytku.
\par 6 Wszedzie gdziemkolwiek chodzil ze wszystkim Izraelem, izalim i slowo rzekl któremu z sedziów Izraelskich, którymem rozkazal, aby pasli lud mój, mówiac: Przeczzescie mi nie zbudowali domu cedrowego?
\par 7 Przetoz teraz tak powiesz sludze memu Dawidowi: Tak mówi Pan zastepów: Jam ciebie wzial z owczarni, gdys chodzil za trzoda, abys byl wodzem nad ludem moim Izraelskim;
\par 8 I bylem z toba wszedzie, gdzieskolwiek chodzil, a wygladzilem wszystkich nieprzyjaciól twoich przed twarza twoja, i uczynilem ci imie, jako imie wielkich ludzi, którzy sa za ziemi.
\par 9 A postanowilem miejsce ludowi memu Izraelskiemu, i wszczepilem go; i bedzie mieszkal na miejscu swem, a nie bedzie wiecej poruszowny, ani go wiecej synowie nieprawosci trapic, jako przedtem;
\par 10 Zaraz ode dni, którychem postanowil sedziów nad ludem moim Izraelskim, i ponizylem wszystkich nieprzyjaciól twoich, i oznajmilem ci, zec Pan dom zbuduje.
\par 11 A gdy sie wypelnia dni twoje, abys szedl za ojcami twoimi, wzbudze nasienie twoje po tobie, które bedzie z synów twoich, i umocnie królestwo jego.
\par 12 Ten mi zbuduje dom, i utwierdze stolice jego az na wieki.
\par 13 Ja mu bede za ojca a on mi bedzie za syna, a milosierdzia mego nie odejme od niego, jakom je odjal od tego, który byl przed toba;
\par 14 Owszem postanowie go w domu moim, i w królestwie mojem az na wieki, a stolica jego bedzie trwala az na wieki.
\par 15 Wedlug tych wszystkich slów i wedlug wszystkiego widzenia tego, tak mówil Natan do Dawida.
\par 16 Zatem wszedlszy król Dawid, siadl przed obliczem Panskiem, i rzekl: Cózem ja jest, Panie Boze! co jest dom mój, zes mie przywiódl az dotad?
\par 17 Lecz i to malo bylo przed oczyma twemi, o Boze! ales tez obietnice uczynil o domie slugi twego na czas daleki, i wejrzales na mie wedlug obyczaju ludzkiego, wywyzszajac mie, o Panie Boze!
\par 18 Cóz jeszcze wiecej ma mówic Dawid przed toba o uwielbieniu slugi twego? albowiem ty znasz sluge twego.
\par 19 Panie! dla slugi twego i wedlug serca twego uczyniles te wszystkie wielkie rzeczy, abys znajome uczynil te wszystkie wielmozne sprawy.
\par 20 Panie! nie masz podobnego tobie, i nie masz Boga oprócz ciebie, wedle wszystkiego, cosmy slyszeli w uszy nasze.
\par 21 I któz jest jako lud twój, jako Izrael, naród jedyny na ziemi, dla któregoby Bóg szedl, aby go sobie odkupil za lud, a uczynil sobie imie, czyniac wielkie rzeczy i straszne, wyganiajac pogany przed twarza ludu twego, którys wykupil z Egiptu?
\par 22 I uczyniles lud twój Izraelski sobie za lud az na wieki, a ty Panie! stales sie im za Boga.
\par 23 Przetoz teraz, o Panie! slowo, któres powiedzial o sludze twoim, i o domu jego, niech bedzie stwierdzone az na wieki, a uczyn, jakos powiedzial.
\par 24 Niechajze sie tak stanie, i niechaj bedzie uwielbione imie twoje az na wieki, aby mówiono: Pan zastepów, Bóg Izraelski, Bogiem jest nad Izraelem; a dom Dawida, slugi twego, niechaj umocniony bedzie przed twarza twoja.
\par 25 Albowiemes ty, Boze mój! objawil sludze twemu, iz mu zbudujesz dom; przetoz znalazl sluga twój u siebie, aby sie modlil przed toba.
\par 26 A tak o Panie! tys sam Bóg, a mówiles to dobre o sludze swym.
\par 27 Teraz tedy poczales blogoslawic domowi slugi twego, aby trwal na wieki przed toba; bos ty Panie! jemu blogoslawil, a bedzie ublogoslawiony na wieki.

\chapter{18}

\par 1 I stalo sie potem, ze porazil Dawid Filistynów i ponizyl ich, a wzial Get i wsi jego z rak Filistynów.
\par 2 Porazil tez Moabczyki, i byli Moabczyki slugami Dawidowymi, przynoszac mu hold.
\par 3 Porazil tez Dawid Hadarezera, króla Soby w Emat, gdy byl wyjechal, aby rozprzestrzenial panstwo swoje nad rzeka Eufrates.
\par 4 Zabral mu tedy Dawid tyziac wozów, i siedm tysiecy jezdnych, i dwadziescia tysiecy mezów pieszych, i poderznal Dawid zyly wszystkich wozników, zachowawszy z nich koni do sta wozów.
\par 5 Przyciagneli tez Syryjczycy z Damaszku na pomoc Hadarezerowi, królowi Soby; lecz porazil Dawid z Syryjczyków dwadziescia i dwa tysiace mezów.
\par 6 Tedy Dawid osadzil zolnierzem Syryje Damaska, a byli Syryjczycy slugami Dawidowymi, oddawajac mu hold; i zachowywal Pan Dwida, gdzie sie kolwiek obrócil.
\par 7 Pobral tez Dawid tarcze zlote, które mieli sludzy Hadarezerowi, i wniósl je do Jeruzalemu.
\par 8 Przytem z Tybchat i z Chun, miast Hadarezerowych, nabral Dawid miedzi bardzo wiele, z której Salomon sprawil morze miedziane, i slupy, i naczynia miedziane.
\par 9 A gdy uslyszal Tohy, król Emat, ze porazil Dawid wszystko wojsko Hadarezera, króla Soby.
\par 10 Poslal Adorama, syna swego, do króla Dawida, aby go pozdrowil w pokoju, i aby mu powinszowal, przeto, ze zwalczyl Hadarezera, i porazil go; (albowiem walczyl Tohy z Hadarezerem) który przyniósl z soba wszelakie naczynie zlote, i srebrne, i miedziane.
\par 11 Które tez poswiecil król Dawid Panu z srebrem i ze zlotem, które byl pobral od wszystkich narodów, od Edomczyków, i od Moabczyków, o od synów Ammonowych, i od Filistynów, i od Amalekitów.
\par 12 A Abisaj, syn Sarwii, porazil Edomczyków w dolinie solnej osmnascie tysiecy.
\par 13 I osadzil Edomska ziemie zolnierzem: a byli wszyscy Edomczycy slugami Dawidowymi: i zachowywal Pan Dawida wszedzie, gdzie sie obrócil.
\par 14 A tak królowal Dawid nad wszystkim Izraelem, czyniac sad i sprawiedliwosc wszystkiemu ludowi swemu.
\par 15 A byl Joab, syn Sarwii, nad wojskiem, a Jozafat, syn Ahiludowy, kanclerzem.
\par 16 A Sadok, syn Achitobowy, i Abimelech, syn Abijatara, byli kaplanami, a Susa byl pisarzem.
\par 17 Benajasz tez, syn Jojady, byl przelozonym nad Cheretczykami i Feletczykami; a synowie Dawidowi byli pierwszymi przy boku królewskim.

\chapter{19}

\par 1 I stalo sie potem, ze umarl Nahas, król synów Ammonowych, a syn jego królowal miasto niego.
\par 2 Tedy rzekl Dawid: Uczynie milosierdzie nad Hanonem, synem Nahasowym; bo ojciec jego uczynil milosierdzie nademna. I poslal Dawid poslów, aby go cieszyli po ojcu jego; a tak przyszli sludzy Dawidowi do ziemi synów Ammonowych, do Hanona, aby go cieszyli.
\par 3 Ale rzekli ksiazeta synów Ammonowych do Hanona: Mniemasz, zeby Dawid czynil uczciwosc ojcu twemu, iz przyslal do ciebie tych, którzyby cie cieszyli; azaz nie dla tego, aby wypatrzyli i wyszpiegowali, i zburzyli te ziemie, przyszli sludzy jego do ciebie?
\par 4 Przetoz wziawszy Hanon slugi Dawidowe, ogolil je, i pobrzynal szaty ich od polowy az do zadków, i puscil ich.
\par 5 Poszli tedy niektórzy, i oznajmili Dawidowi o tych mezach. I poslal przeciwko nim, (poniewaz byli oni mezowie zelzeni bardzo,)i rzekl im król: Zostancie w Jerycho, az odrosna brody wasze, potem sie wrócicie.
\par 6 A widzac synowie Ammonowi, ze sie obrzydlymi stali Dawidowi, poslal Hanon i synowie Ammonowi tysiac talentów srebra, aby sobie najeli za te pieniadze z Mezopotamii i z Syryi Maacha, i z Soby wozy i jezdnych.
\par 7 I najeli sobie za one pieniadze trzydziesci i dwa tysiace wozów, i króla Maacha z ludem jego. Którzy przyciagnawswzy polozyli sie obozem przeciw Medeba; a synowie Ammonowi zebrawszy sie z miast swych, stawili sie do bitwy.
\par 8 Co gdy uslyszal Dawid, poslal Joaba ze wszystkiem wojskiem ludu rycerskiego.
\par 9 A tak wyciagnawszy synowie Ammonowi uszykowali sie do bitwy przed brama miejska. Królowie zasie, którzy byli przyszli na pomoc, osobno w polu byli.
\par 10 Przetoz widzac Joab uszykowane przeciwko sobie wojsko do bitwy z przodku i z tylu, wybral niektórych ze wszystkich przebranych z Izraela, i uszykowal wojsko przeciw Syryjczykom.
\par 11 A ostatek ludu dal pod reke Abisajemu, bratu swemu; i uszykowali sie przeciw synom Ammonowym.
\par 12 I rzekl Joab: Jezli mi beda silnymi Syryjczycy, przyjdziesz mi na pomoc, i jezli tobie synowie Ammonowi beda silnymi, ja tobie dam pomoc.
\par 13 Zmacniej sie, a badzmy maznymi za lud nasz, i za miasta Boga naszego, a Pan, co dobrego jest w oczach jego, niech uczyni.
\par 14 Nastapil tedy Joab, i lud, który z ni byl, do bitwy przeciwko Syryjczykom; ale oni uciekli przed nim.
\par 15 Tedy synowie Ammonowi ujrzawszy, ze uciekali Syryjczycy, uciekli i oni przed Abisaim, bratem jego, i uszli do miasta; a Joab wrócil sie do Jeruzalemu.
\par 16 A tak widzac Syryjczycy, iz byli porazeni od Izraela, wyprawili poslów, i wywiedli Syryjczyków, którzy byli za rzeka, a Sobach, hetman wojska Hadarezerowego, prowadzil ich.
\par 17 I oznajmiono to Dawidowi, który zebrawszy wszystkiego Izraela przeprawil sie przez Jordan, a przyciagnawszy do nich uszykowal wojsko przeciwko nim; a gdy uszykowal wojsko Dawid przeciwko Syryjczykom ku bitwie, zwiedli z nim bitwe.
\par 18 Tedy uciekli Syryjczycy przed Izraelem, i porazil Dawid z Syryjczyków siedm tysiecy wozów, i czterdziesci tysiecy mezów pieszych, i Sobacha, hetmana wojska onego, zabil.
\par 19 Przetoz gdy ujrzeli sludzy Hadarezerowi, iz byli porazeni od Izraela, uczynili pokój z Dawidem, i sluzyli mu. I nie chcieli napotem Syryjczycy dawac pomocy synom Ammonowym.

\chapter{20}

\par 1 I stalo sie po roku tego czasu, gdy królowie zwykli wyjezdzac na wojne, iz wywiódl Joab co mezniejsze rycerstwo, i pustoszyl ziemie synów Ammonowych, a przyciagnawszy oblegl Rabbe; (lecz Dawid zostawal w Jeruzalemie) i dobyl Joab Rabby, i zburzyl ja.
\par 2 I wzial Dawid korone króla ich z glowy jego, a znalazl w niej talent zlota, i kamienie bardzo drogie. I wlozono ja na glowe Dawidowa, i wywiózl lupów z miasta bardzo wiele.
\par 3 Lud tez, który byl w nim, wywiódl, i dal ich potrzec pilami i wozami zelaznemi, i porabac siekierami. Takci uczynil Dawid wszystkim miastom synów Ammonowych, i wrócil sie Dawid ze wszystkim ludem do Jeruzalemu.
\par 4 Potem znowu gdy byla wojna w Gazer z Filistynami, zabil Sobbochaj Husatczyk Syfe, który byl z narodu olbrzymów; a tak Filistynowie ponizeni sa.
\par 5 Byla tez jeszcze wojna z Filistynami, gdzie zabil Elchana, syn Jairowy, Lachmiego, brata Golijata Gietejczyka, którego drzewce u wlóczni bylo nawój tkacki.
\par 6 Nadto jeszcze byla wojna w Get, gdzie byl maz wzrostu wielkiego, majac po szesc palców, wszystkich dwadziescia i cztery; a ten tez byl z narodu tegoz olbrzyma.
\par 7 Ten gdy uragal Izraelowi, zabil go Jonatan, syn Samaja, brata Dawidowego.
\par 8 Ci byli synowie jednego olbrzyma z Get, którzy polegli od reki Dawidowej, i od reki slug jego.

\chapter{21}

\par 1 Ale szatan powstal przeciw Izraelowi a pobudzil Dawida, aby policzyl Izraela.
\par 2 Przetoz rzekl Dawid do Joaba i do przelozonych nad ludem: Idzcie, obliczcie Izraela od Beerseba az do Dan, a odniescie do mnie, zebym wiedzial poczet ich.
\par 3 Ale rzekl Joab: Niech przymnozy Pan ludu swego, jako teraz jest, tyle sto kroc; izali królu, panie mój! nie sa wszyscy oni slugami pana mego? Przeczze sie tego dowiaduje pan mój? Przeczzeby to mialo byc na upadek Izraelowi?
\par 4 Wszakze slowo królewskie przemoglo Joaba; przetoz wyszedl Joab, a obszedlszy wszystkiego Izraela, wrócil sie potem do Jeruzalem.
\par 5 I oddal Joab poczet porachowanego ludu Dawidowi. A bylo wszystkiego Izraela tysiac tysiecy i sto tysiecy, mezów godnych ku bojowi; a z Judy bylo cztery kroc sto tysiecy, i siedmdziesiet tysiecy mezów walecznych.
\par 6 Lecz Lwitów i Benjamitów nie policzyl miedzy nich, gdyz przykre bylo rozkazanie królewskie Joabowi.
\par 7 Owszem nie podobala sie Bogu ta rzecz; przetoz pokaral Izraela.
\par 8 I rzekl Dawid do Boga: Zgrzeszylem bardzo, zem to uczynil; ale teraz oddal, prosze, nieprawosc slugi twego; bom bardzo glupio uczynil.
\par 9 Zatem rzekl Pan do Gada, proroka Dawidowego, mówiac:
\par 10 Idz, powiedz Dawidowi, a rzecz: Tak mówi Pan: Trzyc rzeczy podaje; obierz sobie jedne z nich, abym ci czynil.
\par 11 Tedy przyszedl Gad do Dawida, i rzekl mu: Tak mówi Pan: Obierz sobie:
\par 12 Albo przez trzy lata glód, albo zebys przez trzy miesiace ginal od nieprzyjaciól twych, a miecz nieprzyjaciól twoich zeby cie scigal, albo zeby przez trzy dni miecz Panski i mor byl w ziemi, a Aniol Panski zeby niszczyl wszystkie granice Izraelskie. Przetoz teraz uwaz, co mam odpowiedziec temu, który mie poslal.
\par 13 I rzekl Dawid do Gada: Bardzom scisniony; niech wpadne, prosze, w rece Panskie, gdyz bardzo wielkie sa zlitowania jego, a w rece ludzkie niechaj nie wpadam.
\par 14 Tedy przepuscil Pan powietrze morowe na Izraela. I poleglo z Izraela siedmdziesiat tysiecy mezów.
\par 15 Poslal tez Bóg Aniola do Jeruzalemu, aby ich tracil. A gdy ich tracil, ujrzal Pan, i uzalil sie nad tem zlem, i rzekl Aniolowi tracacemu: Dosyc juz, zawsciagnij reke twa. A Aniol Panski stal podle bojewiska Ornana Jebuzejczyka.
\par 16 Wtem podnióslszy Dawid oczy swe ujrzal Aniola Panskiego, który stal miedzy ziemia i miedzy niebem, a w rece jego miecz jego dobyty, wyciagniony przeciw Jeruzalemowi. I upadl Dawid i starsi, obleklszy sie w wory, na twarze swoje.
\par 17 Zatem rzekl Dawid do Boga: Izalim nie ja rozkazal liczyc ludu? Jamci jest sam, którym zgrzeszyl, i bardzo zle uczynil; ale te owce cóz uczynily? Panie, Boze mój! niech sie obróci, prosze, reka twoja na mie i na dom ojca mego; ale przeciwko ludowi twemu niech sie nie srozy ta plaga.
\par 18 Zatem Aniol Panski rzekl do Gada, aby mówil Dawidowi, zeby szedl i zbudowal oltarz Panu na bojewisku Ornana Jebuzejczyka.
\par 19 A tak szedl Dawid wedlug slowa Gadowego, które mówil imieniem Panskiem.
\par 20 Tedy obejrzawszy sie Ornan ujrzal onego Aniola; a czterej synowie jego, którzy byli z nim, skryli sie; a Ornan mlócil pszenice.
\par 21 Wtem przyszedl Dawid do Ornana; a spojrzawszy Ornan obaczyl Dawida, i wyszedlszy z bojewiska, poklonil sie Dawidowi twarza do ziemi.
\par 22 I rzekl Dawid do Ornana: Daj mi plac tego bojewiska, abym zbudowal na nim oltarz Panu; za sluszne pieniadze spusc mi je, a bedzie odwrócona ta plaga od ludu.
\par 23 I rzekl Ornan do Dawida: Wezmij je sobie, a niech uczyni król, pan mój, co mu sie dobrego widzi; otoc przydaje i woly na calopalenia, i wóz na drwa, i pszenice na ofiare sniedna: toc to wszystko daje.
\par 24 I rzekl król Dawid do Ornana: Nie tak, ale raczej kupie za sluszne pieniadze; bo nie wezme co twego jest, ani bede ofiarowal Panu calopalenia darowanego.
\par 25 A tak Dawid dal Ornanowi za on plac szesc syklów zlota dobrej wagi.
\par 26 I zbudowal tam Dawid oltarz Panu, a ofiarowal calopalenia i ofiary spokojne, i wzywal Pana, który go wysluchal, spusciwszy ogien z nieba na oltarz calopalenia.
\par 27 I rzekl Pan do Aniola, aby obrócil miecz swój w pochwy swoje.
\par 28 Onego czasu widzac Dawid, iz go wysluchal Pan na bojewisku Ornana Jebuzejczyka, ofiarowal tam ofiary.
\par 29 Albowiem przybytek Panski, który uczynil Mojzesz na puszczy, i oltarz calopalenia, naonczas byl na wyzynie w Gabaonie.
\par 30 A nie mógl Dawid isc do niego, aby sie radzil Boga; bo przestraszony byl mieczem Aniola Panskiego.

\chapter{22}

\par 1 I rzekl Dawid: Toc jest miejsce domu Pana Boga, i to oltarz na calopalenie Izraelowi.
\par 2 Przetoz rozkazal Dawid, aby zgromadzono cudzoziemców, którzy byli w ziemi Izraelskiej; i postanowil z nich kamienników, aby ciosali kamienie czworograniaste na budowanie domu Bozego.
\par 3 Zelaza taz bardzo wiele na gwozdzie, i na drzwi w bramach, i na spajanie nagotowal Dawid, i miedzi wage niezliczona.
\par 4 Drzewa taz cedrowego bez liczby, albowiem nawiezli Dawidowi Sydonczycy i Tyryjczycy drzewa cedrowego bardzo wiele.
\par 5 Bo rzekl byl Dawid: Salomon, syn mój, jest mlodzienczykiem malym, a dom ma byc zbudowany Panu wielki i znamienity, któregoby imie i slawa po wszystkiej ziemi byla; przetoz teraz nagotuje mu potrzeb. I nagotowal Dawid przed smiercia swa bardzo wiele potrzeb.
\par 6 Tedy zawolal Salomona, syna swego, a przykazal mu, aby zbudowal dom Panu, Bogu Izraelskiemu.
\par 7 I rzekl Dawid do Salomona: Synu mój! Umyslilem byl w sercu mojem, zbudowac dom imieniowi Pana Boga mego.
\par 8 Ale sie stalo do mnie slowo Panskie, mówiac: Wieles krwi rozlal, i wielkies wojny prowadzil; nie bedziesz budowal domu imieniowi memu, przeto zes wiele krwi rozlal na ziemie przedemna,
\par 9 Oto syn, któryc sie urodzi, bedzie mezem spokojnym; bo mu dam odpocznienie od wszystkich nieprzyjaciól jego zewszad. Przetoz Salomon bedzie imie jego; albowiem pokój i odpocznienie dam Izraelowi za dni jego.
\par 10 On zbuduje dom imieniowi memu; on mi bedzie za syna, a ja mu bede za ojca, i utwierdze stolice królestwa jego nad Izraelem az na wieki.
\par 11 Przetoz Pan bedzie z toba, synu mój! I bedziec sie szczescilo, i zbudujesz dom Pana, Boga twego, jako mówil o tobie.
\par 12 Wszakze niech ci da Pan roztropnosc, i zmysl, a niech cie postanowi nad Izraelem, abys strzegl zakonu Pana Boga twego.
\par 13 Tedy szczesliwym bedziesz, jezli strzedz i czynic bedziesz przykazania i sady, które rozkazal Pan przez Mojzesza Izraelowi. Zmacniajze sie, a badz mezem, nie bój sie ani sie lekaj.
\par 14 A otom ja w utrapieniu mojem nagotowal na dom Panski zlota sto tysiecy talentów, i srebra tysiac tysiecy talentów do tego miedzi i zelaza bez wagi, bo tego wiele jest: drzewa takze, i kamienia nagotowalem, a ty do tego przyczynisz.
\par 15 Masz tez u siebie wiele rzemieslników, kamiennikówi i murarzy, i ciesli, i wszelkich bieglych w kazdem rzemiesle.
\par 16 Zlota, srebra, i miedzi, i zelaza niemasz liczby; wstanze a czyn, a Pan bedzie z toba.
\par 17 I przykazal Dawid wszystkim ksiazetom Izraelakim, aby pomagali Salomonowi synowi jego;
\par 18 Mówiac: Izali Pan Bóg wasz nie jest z wami, który wam dal odpocznienie zewszad? Bo dal w reke moje obywateli tej ziemi, i poddane jest ta ziemia Panu i ludowi jego.
\par 19 Teraz tedy oddajcie serce swe i dusze swoje, abyscie szukali Pana, Boga waszego; i wstancie, a budujcie swiatnice Panu Bogu, zebyscie tam wnieszli skrzynie przymierza Panskiego, i naczynia swiete Boze, do domu, który bedzie zbudowany imieniowi Panskiemu.

\chapter{23}

\par 1 A tak Dawid bedac stary i pelen dni, postanowil królem Salomona, syna swego nad Izraelem.
\par 2 I zgromadzil wszystkich ksiazat Izraelskich, i kaplanów, i Lewitów;
\par 3 A policzono Lewitów od trzydziestu lat i wyzej; i byl poczet ich wedlug glów ich osób trzydziesci i osm tysiecy.
\par 4 Z których postanowiono na posluge domu Panskiego dwadziescia i cztery tysiace, a przelozonych i sedziów szesc tysiecy.
\par 5 Nadto cztery tysiace odzwiernych, i cztery tysiace chwalacych Pana na instrumentach, których nasprawial Dawid ku chwaleniu Boga.
\par 6 I rozdzielil ich Dawid na pewne hufy wedlug synów Lewiego, to jest, Giersona, Kaata, i Merarego.
\par 7 Z Giersona byli Laadam, i Semej.
\par 8 Synowie Laadanowi: przedniejszy Jachijel, i Zetam, i Joel, ci trzej.
\par 9 Synowie Semejowi: Salomit, i Hazyjel, i Haran, ci trzej. Cic byli przedniejsi domów ojcowskich z Laadana.
\par 10 A synowie Semejowi: Jachat, Zyna, i Jehus, i Baryjasz; cic synowie Semejowi czterej.
\par 11 A Jachat byl pierwszym, a Zyza wtóry; ale Jehus i Baryjasz nie mieli wiele synów; przetoz byli w domu ojcowskim policzeni za jedne familije.
\par 12 Synowie Kaatowi: Amram, Izaar, Hebron, i Husyjel, czterej.
\par 13 Synowie Amramowi: Aaron i Mojzesz. Lecz Aaron byl odlaczony, aby sluzyl w swiatnicy najswietszej, sam i synowie jego az na wieki, i aby kadzili przed Panem, a sluzyli mu, i blogoslawili w imieniu jego az na wieki.
\par 14 Ale synowie Mojzesza, meza Bozego, policzeni sa w pokoleniu Lewiego.
\par 15 Synowie Mojzeszowi: Gierson i Eliezer.
\par 16 Synowie Giersonowi: Sebujel pierwszy.
\par 17 A synowie Eliezerowi byli Rechabijasz pierwszy. I nie mial Eliezer synów innych; ale synowie Rechabijaszowi rozmnozyli sie bardzo.
\par 18 Synowie Izaarowi: Salomit pierwszy.
\par 19 Synowie Hebronowi: Jeryjasz pierwszy, Amaryjasz wtóry, Jehazyjel trzeci, a Jekmaan czwarty.
\par 20 Synowie Husyjelowi: Micha pierwszy, a Jesyjasz wtóry.
\par 21 Synowie Merarego: Mahelin i Musy; a synowie Mahelego: Eleazar i Cys.
\par 22 I umarl Eleazar, a nie mial synów, tylko córki, które pojmowali synowie Cysowi, bracia ich.
\par 23 Synowie Musy: Maheli, i Eder, i Jerymot, trzej.
\par 24 Cic sa synowie Lewiego wedlug domów ojców swych, przedniejsi domów ojcowskich, którzy policzeni byli wedlug pocztu imion i osób swych z osobna, którzy odprawowali prace uslugiwania w domu Panskim od dwudziestu lat i wyzej.
\par 25 Albowiem rzekl Dawid: Dal odpocznienie Pan, Bóg Izraelski, ludowi swemu, i bedzie mieszkal w Jeruzalemie az na wieki.
\par 26 Do tego i Lewitowie nie beda wiecej nosic przybytku i wszystkiego naczynia jego ku poslugiwaniu jego;
\par 27 Ale wedlug postanowienia Dawidowego ostatniego, byli policzeni synowie Lewiego od dwudziestu lat i wyzej;
\par 28 Aby zostawali pod reka synów Aaronowych ku usludze domu Panskiego w przysionkach i w gmachach, i ku oczyszczaniu wszelkich rzeczy poswieconych, i ku pracy okolo uslugi domu Bozego;
\par 29 I okolo chleba pokladnego, i okolo maki na ofiare, i okolo placków niekwaszonych, i okolo panewek, i okolo rzeczy smazonych, i okolo wszelkiej miary i odmierzania;
\par 30 A izby stali na kazdy poranek ku wyslawianiu, i ku chwaleniu Pana, takze i w wieczór;
\par 31 Nadto, przy kazdem ofiarowaniu calopalenia Panu w sabaty, na nowiu miesiaca, i w uroczyste swieta, wedlug liczby i porzadku ich ustawicznie przed Panem;
\par 32 A tak aby pilnowali strazy namiotu zgromadzenia, i strazy swiatnicy, i strazy synów Aaronowych, braci swych, w usludze domu Panskiego.

\chapter{24}

\par 1 A synowie Aaronowi tym sposobem rozdzieleni byli: Synowie Aaronowi byli Nedab i Abiju, Eleazar i Itamar;
\par 2 Ale iz Nadab i Abiju umarli przed obliczem ojca swego, a synów nie mieli: przetoz odprawowali urzad kaplanski Eleazar i Itamar.
\par 3 I podzielil ich Dawid, to jest Sadoka z synów Eleazarowych, i Achimelecha z synów Itamarrowych, wedlug urzedu ich w uslugach ich.
\par 4 I znalazlo sie synów Eleazarowych wiecej przedniejszych mezów, niz synów Itamarowych, gdy ich podzielil. Z synów Eleazarowych bylo przedniejszych wedlug domów ojcowskich szesnascie; ale synów Itamarowych wedlug domów ojcowskich osm.
\par 5 A rozdzieleni sa losem jedni od drugich; bo byli przelozonymi nad swiatnica, i przedniejszymi przed Bogiem, tak z synów Eleazarowych, jako i z synów Itamarowych.
\par 6 A popisal ich Senejasz, syn Natanaelowy, pisarz, z pokolenia Lewiego, przed królem i psiazetami, i przed Sadokiem kaplanem, i Achimelechem, synem Abijatarowym, i przedniejszymi z domów ojcowskich, kaplanów i Lewitów; a naznaczono jeden dom ojcowsk i Eleazarowi, a drugi naznaczono Itamarowi.
\par 7 I padl los pierwszy na Jehojaryba, na Jedajasza wtóry;
\par 8 Na Haryma trzeci, na Seoryma czwarty;
\par 9 Na Malchyjasza piaty, na Mijamana szósty;
\par 10 Na Akkosa siódmy, na Abijasza ósmy;
\par 11 Na Jesuego dziewiaty, na Sechenijasza dziesiaty;
\par 12 Na Eliasyba jedenasty, na Jakima dwunasty;
\par 13 Na Huppa trzynasty, na Jesebaba czternasty;
\par 14 Na Bilge pietnasty, na Immera szesnasty;
\par 15 Na Chezyra siedmnasty, na Happisesa osmnasty;
\par 16 Na Petachyjasza dziewietnasty, na Ezechyjela dwudziesty;
\par 17 Na Jachyna dwudziesty i pierwszy, na Gamuela dwudziesty i wtóry;
\par 18 Na Delajasza dwudziesty i trzeci, na Maazyjasza dwudziesty i czwarty.
\par 19 Cic sa sporzadzeni w poslugiwaniu swojem, aby wchodzili do domu Panskiego w przemianach swych, jako zwykli pod rzadem Aarona, ojca ich, jako mu byl rozkazal Pan, Bóg Izraelski.
\par 20 A z synów Lewiego, którzy byli pozostali z synów Amramowych, Subajel; z synów Subajelowych Jechdejasz.
\par 21 Z Rechabijasza, z synów Rechabijaszowych byl przednjiejszy Jesyjasz.
\par 22 Z Isaary Salomit, z synów Salomitowych Jachat.
\par 23 A synowie Jeryjaszowi: Amaryjasz wtóry, Jehazylej trzeci, Jekmaan czwarty.
\par 24 Synowie Husyjelowi Micha; z synów Michy Samir.
\par 25 Brat Michasowy Jesyjasz; z synów Jesyjaszowych Zacharyjasz.
\par 26 Synowie Merarego: Maheli i Musy: synowie Jahasyjaszowi Beno.
\par 27 Sunowie Merarego z Jahasyjasza: Beno, i Soam, i Zachur, i Hybry.
\par 28 Z Mahalego Eleazar, który nie mial synów.
\par 29 Z Cysa, synowie Cysowi Jerahmeel.
\par 30 A synowie Musy: Maheli i Eder, i Jerymot. Cic byli synowie Lewitów wedlug domów ojców ich.
\par 31 I ci tez miotali losy naprzeciwko braci swoich, synom Aaronowym, przed Dawidem królem, i Sadokiem, i Achimelechem, i przedniejszymi domów ojcowskich, z kaplanów i Lewitów, z domów ojcowskich, kazdy przedniejszy przeciwko bratu swemu mlodszemu.

\chapter{25}

\par 1 I odlaczyl Dawid i hetmani wojska na poslugiwanie synów Asafowych i Hemanowych, i Jedytunowych, którzy prorokowali przy cytrach, i przy harfach, i przy cymbalach. A byla liczba ich, to jest mazów pracujacych w usludze swej:
\par 2 Z synów Asafowych: Zachur, i Józef, i Natanijasz, i Asarela. Synowie Asafowi byli pod reka Asafowa, który prorokowal na rozkazanie królewskie.
\par 3 Z Jedytuna: Synowie Jedytunowi: Godolijasz, i Zery, i Jesajasz, Hasabijasz, i Matytyjasz, i Symej, szesc, pod reka ojca ich Jedytuna, który prorokowal przy harfie, wyznawajac i chwalac Pana.
\par 4 Z Hemana: Synowie Hemanowi: Bukkijasz, Matanijasz, Husyjel, Zebuel, i Jerymot, Chananijasz, Chanani, Eliata, Gieddalty, i Romantyjeser, i Jasbekassa, Malloty, Hotyr, Machazyjot.
\par 5 Ci wszyscy byli synowie Hemana, widzacego królewskiego w slowach Bozych, ku wywyzszeniu rogu: bo dal Bóg czternascie synów Hemanowych, i trzy córki.
\par 6 Ci wszyscy byli pod sprawa ojca swego przy spiewaniu w domu Panskim na cymbalach, na lutniach, i na cytrach ku sluzbie w domu Bozym, jako rozkazal król, i Asaf, Jedytun, i Heman.
\par 7 A byl poczet ich z braci ich, którzy byli cwiczonymi w piesniach Panskich, wszystkich mistrzów dwiescie osmdziesiat i osm.
\par 8 I miotali losy, straz przeciwko strazy, tak maly jako i wielki, tak mistrz jako i uczen.
\par 9 I padl los pierwszy w domu Asafowym na Józefa; na Godolijasza wtóry, z bracmi jego i z synami jego, których bylo dwanascie.
\par 10 Na Zachura trzeci, na synów jego i na braci jego dwanascie.
\par 11 Czwarty na Isrego, na synów jego i na braci jego dwanascie.
\par 12 Piaty na Natanijasza, na synów jego i na braci jego dwanascie.
\par 13 Szósty na Bukkijasza, na synów jego i na braci jego dwanascie.
\par 14 Siódmy na Jesarela, na synów jego i na braci jego dwanascie.
\par 15 Osmy na Jesajasza, na synów jego i na braci jego dwanascie.
\par 16 Dziewiaty na Matanijasza, na synów jego i na braci jego dwanascie.
\par 17 Dziesiaty na Symejasza, na synów jego i na braci jego dwanascie.
\par 18 Jedenasty na Asarela, na synów jego i na braci jego dwanascie.
\par 19 Dwunasty na Hasabijasza, na synów jego i na braci jego dwanascie.
\par 20 Trzynasty na Subajela, na synów jego i na braci jego dwanascie.
\par 21 Czternasty na Matytyjasza, na synów jego i na braci jego dwanascie.
\par 22 Pietnasty na Jerymota, na synów jego i na braci jego dwanascie.
\par 23 Szesnasty naChananijasza, na synów jego i na braci jego dwanascie.
\par 24 Siedemnasty na Jesbekassa, na synów jego i na braci jego dwanascie.
\par 25 Osmnasty na Chananijego, na synów jego i na braci jego dwanascie.
\par 26 Dziewietnasty na Mallotego, na synów jego i na braci jego dwanascie.
\par 27 Dwudziesty na Elijata, na synów jego i na braci jego dwanascie.
\par 28 Dwudziesty i pierwszy na Hotyra, na synów jego i na braci jego dwanascie.
\par 29 Dwudziesty i wtóry na Gieddaltego, na synów jego i na braci jego dwanascie.
\par 30 Dwudziesty i trzeci na Machazyjota, na synów jego i na braci jego dwanascie.
\par 31 Dwudziesty i czwarty na Romantyjesera, na synów jego i na braci jego dwanascie.

\chapter{26}

\par 1 Rozdzialy zas odzwiernych byly z Korejczyków: Meselemijasz, syn Korego, z synów Asafowych.
\par 2 A z Meselemijaszowych synów: Zacharyjasz pierworodny, Jadyjael wtóry, Zabadyjasz trzeci, Jatnijel czwarty.
\par 3 Elam piaty, Jachanan szósty, a Elienaj siódmy.
\par 4 A z Obededomowych synów: Semajasz pierworodny, Jozabad wtóry, Joach trzeci, i Sachar czwarty, a Natanael piaty,
\par 5 Ammijel szósty, Isaschar siódmy, Pechulletaj ósmy; bo mu Bóg blogoslawil.
\par 6 Semajaszowi tez, synowi jego, zrodzili sie synowie, którzy panowali w domu ojca swego; bo byli mezowie bardzo mocni.
\par 7 Synowie Semajaszowi: Otni, i Rafael i Obed, Elzabed, bracia jego, mezowie mocni, Elihu i Semachyjasz.
\par 8 Wszyscy ci z synów Obededomowych, sami i synowie ich, i bracia ich, kazdy z nich bardzo mocny i sposobny ku poslugiwaniu, szesdziesiat i dwa wszystkich z Obededoma.
\par 9 A z Meselemijaszowych synów i braci, mezów mocnych, osmnascie.
\par 10 A z Hosy, który byl z synów Merarego, synowie byli: Semry przedniejszy; nie izby byl pierworodny, ale iz go ojciec jego uczynil przedniejszym;
\par 11 Helkijasz wtóry, Tebalijasz trzeci, Zacharyjasz czwarty; wszystkich synów i braci trzynascie.
\par 12 Ci sa rozdzieleni na odzwiernych, aby byli wrotnymi z mezów przedniejszych, trzymajac straz na przemiany z bracmi swymi przy sluzbie w domu Panskim.
\par 13 Albowiem miotali losy, tak maly jako wielki wedlug domów ojców swych, o kazda brame.
\par 14 I padl los na wschód slonca Selemijaszowy; Zacharyjaszowi takze synowi jego, radcy madremu, rzucili losy, i padl los jego na pólnocy;
\par 15 A Obededomowi na poludnie; ale synom jego na dom skarbów.
\par 16 Suppimowi i Hozie na zachód z brama Zallechet, przy scieszce usypanej, idacej ku górze; a tak byla straz na przeciwko strazy.
\par 17 Na wschód slonca bylo Lewitów szesc, na pólnocy na dzien czterech, na poludnie na dzien czterech, a przy domu skarbów dwóch a dwóch.
\par 18 Przy stronie zewnetrznej na zachód bylo czterech na drodze sypanej, a dwóch przy stronie zewnetrznej.
\par 19 Tec sa rozdziely odzwiernych z synów Korego, i z synów Merarego.
\par 20 A z drugich Lewitów Achyjasz byl nad skarbami domu Bozego, to jest nad skarbami rzeczy poswieconych.
\par 21 Synowie Laadanowi, którzy byli z synów Giersonickich: z Laadana Giersonczyka przedniejsi w domach ojcowskich, Jehyjel.
\par 22 A synowie Jehyjlowi byli Zetam, i Joel, brat jego; ci byli nad skarbami domu Panskiego.
\par 23 Z Amramczyków, i z Izaarczyków, z Hebronczyków, i Hysyjelczyków.
\par 24 Byl Sebuel, syn Giersona, syna Mojzeszowego, przelozony nad skarbami.
\par 25 Ale bracia jego z Eleazara byli Rechabijasz syn jego, i Jesajasz syn jego, i Joram syn jego, i Zychry syn jego, i Salomit syn jego.
\par 26 Ten Salomit i bracia jego byli na wszystkiemi skarbami rzeczy poswieconych, które byl poswiecil Dawid król, i przedniejsi z domów ojcowskich, i pólkownicy, i rotmistrze, i hetmani wojska.
\par 27 Bo z wojen i z lupów poswiecali na poprawe domu Panskiego;
\par 28 I wszystko, co byl poswiecil Samuel widzacy, i Saul, syn Cysowy, i Abner, syn Nera, i Joab, syn Sarwii, i ktokolwiek co poswiecal, oddawal do rak Salomitowych i braci jego.
\par 29 Z Izaarytów: Kienanijasz i synowie jego nad robota zewnetrzna w Izraelu, byli za urzedników i za sedziów.
\par 30 Z Hebronczyków: Hasabijasz i braci jego, mezów duzych, bylo tysiac i siedm set przelozonych nad Izraelem za Jordanem na zachód slonca, w kazdej robocie Panskiej i w posludze królewskiej.
\par 31 A z Hebronczyków byl Jeryjasz przedniejszy nad Hebronczykami wedlug narodów ich i domów ojcowskich. Bo roku czterdziestego królestwa Dawidowego szukano i znaleziono miedzy nimi mezów bardzo mocnych w Jazer Galaadskiem.
\par 32 A braci jego, mezów godnych, bylo dwa tysiace i siedm set przedniejszych w domach ojcowskich. I postanowil ich Dawid król nad Rubenczykami, i nad Gadczykami, i nad polowa pokolenia Manasesowego, nad wszystkiemi sprawami Bozemi i sprawami królewskiemi.

\chapter{27}

\par 1 A synów Izraelskich wedlug liczby ich, przedniejszych w domach ojcowskich, i pólkowników, i rotmistrzów, i przelozonych nad tymi, którzy sluzyli królowi we wszelkiej potrzebie, w podzialach swoich, przychodzacych i odchodzacych na kazdy miesiac przez wszystkie miesiace w roku; w kazdym podziale bylo dwadziescia i cztery tysiace.
\par 2 Nad hufem pierwszego miesiaca byl Jasobeam, syn Sabdyjelowy, a w podziele jego bylo dwadziescia i cztery tysiace.
\par 3 Ten byl z synów Faresowych przedniejszym nad wszystkiemi przelozonymi w wojsku miesiaca pierwszego.
\par 4 A nad podzialem wtórego miesiaca byl Dodaj Achochytczyk i z podzialem swym; po nim Michlot, ksiaze, a w podziale jego bylo ludu dwadziescia i cztery tysiace.
\par 5 Przelozony trzeci wojska miesiaca trzeciego byl Banajas, syn Jojady kaplana, przedniejszym, a w podziale jego dwadziescia i cztery tysiace.
\par 6 Ten Banajas byl mocarz miedzy trzydziestoma, i nad trzydziestoma, a nad podzialem jego byl Ammisadab, syn jego.
\par 7 Czwarty czwartego miesiaca byl Asael, brat Joabowy, a po nim Zabadyjasz, syn jego, a w podziale jego dwadziescia i cztery tysiace.
\par 8 Piaty miesiaca piatego byl przelozonym Samut Jezrahytczyk, a w podziale jego dwadziescia i cztery tysiace.
\par 9 Szósty miesiaca szóstego byl Hyra, syn Ikkiessa Tekuitczyka, a w podziale jego dwadziescia i cztery tysiace.
\par 10 Siódmy miesiaca siódmego byl Heles Felonitczyk z synów Efraimowych, a w podziale jego dwadziescia i cztery tysiace.
\par 11 Osmy miesiaca ósmego byl Sobochaj Husatczyk z Zarchejczyków, a w podziale jego dwadziescia i cztery tysiace.
\par 12 Dziewiaty miesiaca dziewiatego byl Abijezer Anatotczyk z synów Jemini, a w podziale jego dwadziescia i cztery tysiace.
\par 13 Dziesiaty miesiaca dziesiatego byl Mahary Netofatczyk z Zarchejczyków, a w podziale jego dwadziescia i cztery tysiace.
\par 14 Jedenasty miesiaca jedenastego byl Banajas Faratonszyk z synów Efraimowych, a w podziale jego dwadziescia i cztery tysiace.
\par 15 Dwunasty miesiaca dwunastego byl Haldaj Netofatczyk z Otonijela, a w podziale jego dwadziescia i cztery tysiace.
\par 16 Nadto nad pokoleniem Izraelskiem byli: Nad Rubenczykami byl ksiazeciem Elijezer, syn Zychrego; nad Symeonczykami Sefatyjasz, syn Maachowy;
\par 17 Nad pokoleniem Lewiego Chasabijasz, syn Chemuelowy; nad Aaronowem Sadok;
\par 18 Nad Judowem Elihu z braci Dawidowych; nad Isascharowem Amry, syn Michaelowy;
\par 19 Nad Zabulonowem Jesmajasz, syn Abdyjaszowy; nad Neftalimowem Jerymot, syn Asryjelowy;
\par 20 Nad synami Efraimowymi Hosejasz, syn Azazyjaszowy; nad polowa pokolenia Manasesowego Joel, syn Fadajaszowy;
\par 21 Nad druga polowa pokolenia Manasesowego w Galaadzie Iddo, syn Zacharyjaszowy; nad Benjaminowem Jaasyjel, syn Abnerowy.
\par 22 Nad Danowem Azraeel, syn Jerohamowy. Cic sa ksiazeta pokolen Izraelskich.
\par 23 A nie wlozyl Dawid w liczbe ich zadnego, co mial dwadziescia lat i nizej; albowiem Pan byl powiedzial, iz mial rozmnozyc Izraela jako gwiazdy niebieskie.
\par 24 A Joab, syn Sarwii, poczal byl liczyc, ale nie dokonczyl, dlatego ze byl gniew przypadl na Izraela. I nie weszla ta liczba w liczbe kronik o sprawach króla Dawida.
\par 25 A nad skarbami królewskiemi byl Asmawet, syn Abdyjelowy; a nad dochodami z pól, z miast, i ze wsi i z zamków byl Jonatan, syn Uzyjaszowy;
\par 26 A nad oraczami, którzy uprawiali ziemie, byl Ezer, syn Chalubowy.
\par 27 A nad winnicami byl Semejasz Ramatczyk; a nad urodzajami winnic i nad piwnicami winnemi Zabdyjasz Zyfmejczyk.
\par 28 A nad oliwnicami, i nad drzewami figowemi, które sa w polach, byl Balanan Giedertczyk, a nad piwnicami oliwnemi Joas.
\par 29 A nad bydlem, które paszono w Saron, Sytraj Saronitczyk; a nad bydlem po dolinach Safat, syn Adlajego.
\par 30 A nad wielbladami byl Obil Ismaelitczyk, a nad oslicami byl Jechdejasz Meronczyk.
\par 31 A nad drobnem bydlem Jazys Agiertczyk. Cic wszysct byli przelozonymi nad majetnosciami króla Dawida.
\par 32 Ale Jonatan, stryj Dawidowy, byl radca, maz madry, i nauczony; ten, i Jehijel, syn Chachmonowy, byl z synami królewskimi.
\par 33 Achitofel tez byl radca królewskim, a Chusaj Archytczyk przyjacielem królewsskim.
\par 34 A po Achitofelu byl Jojada, syn Banajasowy, i Abujatar. A Joab byl hetmanem wojska królewskiego.

\chapter{28}

\par 1 Tedy zgromadzil Dawid wszystkich ksiazet Izraelskich, i przedniejszych z kazdego pokolenia, i przelozonych nad hyfcami, którzy sluzyli królowi, i pólkowników, i rotmistrzów, i przelozonych nad wszystka majetnoscia i osiadloscia królewska; synów tez swoich z komornikami, i z innymi moznymi, i ze wszystkim ludem rycerskim do Jeruzalemu.
\par 2 A powstawszy król Dawid na nogi swoje, rzekl: Sluchajcie mie, bracia moi, i ludu mój! Jam byl umyslil w sercu swem, budowac dom, gdzieby odpoczywala skrzynia przymierza Pacskiego, i na podnózek nóg Boga naszego, i zgotowalem byl potrzeby ku budowaniu;
\par 3 Ale Bóg rzekl do mnie: Nie bedziesz budowal domu imieniowi memu, przeto zes maz waleczny, i rozlewales krew.
\par 4 Ale obral mie Pan, Bóg Izraelski, ze wszystkiego domu ojca mego, abym byl królem nad Izraelem na wieki; bo z Judy obral ksiecia, a z narodu Judzkiego dom ojca mego; i z synów ojca mego upodobal mie sobie za króla nad wszystkim Izraelem.
\par 5 A ze wszystkich synów moich (bo mi wiele synów Pan dal) obral Salomona, syna mego, aby siedzial na stolicy królestwa Panskiego nad Izraelem.
\par 6 I mówil do mnie: Salomon, syn twój, ten zbydyje dom mój, i przysionki moje; albowiemem go sobie obral za syna, a Ja mu bede za ojca.
\par 7 I umocnie królestwo jego az na wieki, bedzieli statecznym w pelnieniu przykazan moich i sadów moich, jako i dzis.
\par 8 Teraz tedy mówie wam pezed obliczem wszystkiego Izraela, zgromadzenia tego Panskiego, gdzie slyszy Bóg nasz: Strzezcie a szukajcie wszystkich rozkazan Pana, Boga waszego, abyscie osiedli ziemie dobra, i zostawili ja w dziedzictwo synom swoim po so bie az na wieki.
\par 9 A ty Salomonie, synu mój! znaj Boga, ojca twojego, i sluz mu sercem doskonalem, i umyslem dobrowolnym; bo wszystkie serca przeglada Pan, i wszystkie zamusly mysli zna. Jezli go szukac badziesz, znajdziesz go, a jezli go opuscisz, odrzuci cie na wieki.
\par 10 Obaczze teraz, iz cie Pan obral, abys zbudowal dom swiatnicy; zmacniajze sie a wykonaj to.
\par 11 Tedy oddal Dawid Salomonowi, synowi swemu, wizerunek przysionka, i gmachów jego, i komor jego, i sal jego, i wnetrznych pokojów jego, i domu ublagalni.
\par 12 Przytem wizerunek wszystkiego, co byl umyslil o sieni domu Pacskiego, i o wszystkich gmachach dla skarbów domu Bozego, i dla skarbów rzeczy swietych;
\par 13 I dla pocztów kaplanskich, i Lewitów, i dla wszystkiej pracy w usludze domu Panskiego, i dla wszystkiego naczynia sluzby domu Panskiego.
\par 14 Takze zlota pewna wage na wszystkie naczynia zlote, od wszystkiej uslugi; srebra takze na wszystkie naczynia srebrne pewna wage, na wszystkie naczynia ku wszelakiej usludze;
\par 15 Mianowicie pewna wage na swieczniki zlote i na lampy ich zlote wedlug wagi kazdego swiecznika i lamp jego, i na swieczniki srebrne wedlug wagi swiecznika kazdego i lamp jego, wedlug potrzeby kazdego swiecznika.
\par 16 Takze pewnawage zlota na stoly chlebów pokladbych, na kazdy stól, przytem srebra na stoly srebrne.
\par 17 A na widelki, i na kociolki, i na kadzielnice szczerego zlota, i na czasze zlote, pewnawage na kazda czasze, i na czasze srebrne, pewna wage na kazda czasze.
\par 18 Takzw na oltarz do kadzeania dal zlota szczerego pewna wage, i zlota ku wystawieniu woza Cherubinów, którzyby rozciagnionemi skrzydlami okrywali skrzynie przymierza Panskiego.
\par 19 To wszystko, rzekl Dawid, opisane z reki Panskiej mie doszlo, abym zrozumial wszystko, jako co urobic miano.
\par 20 A tak rzekl Dawid do Salomona, syna swego: Zmacniaj sie, a badz meznym, czyn to; nie bój sie, ani sie lekaj; bo Pan Bóg, Bóg mój, bedzie z toba, nie opusci cie, ani cie odstapi, az dokonczysz wszystkiej roboty sluzby domu Panskiego.
\par 21 A oto poczty kaplanów i Lewitów do kazdej poslugi w domu Bozym bada z toba w kazdej pracy; kazdy ochotny i roztropny przy wszelkiej posludze, takze ksiazeta, i wszystek lud stana na kazde rozkazanie twoje.

\chapter{29}

\par 1 Potem mówil król Dawid do wszystkoego zgromadzenia: Salomona, syn mego jedynego, obral Bóg mlodzienczyka malego. Ale to wielka sprawa; do nie czlowiekowi palac ten, ale Panu Bogu bedzie.
\par 2 Ja wedlug najwyzszego przemozenia mego nagotowalem na dom Boga mego zlota, na naczynie zlote, i srebra na srebrne, i miedzi na miedziane, zelaza na zelazne, i drzewa na drewniane, kamienia onychynowego na osadzanie, i kamienia karbunkulowego, i ro zlicznych farb, a wszelakiego kamienia drogiego, i kamienia marmurowego dostatek wielki.
\par 3 Nadto z checi mojej ku domowi Boga mego osobne zloto i srebro, które mam, oddaje na dom Boga mego, oprócz tego wszystkiego, com zgotowal na dom swiatnicy;
\par 4 To jest trzy tysiace talentów zlota, zlota z Ofir, i siedm tysiecy talentów srebra najczystszego na okrycie scian gmachów;
\par 5 Zlota na naczynie zlote, a srebra na srebrne, i na wszystkie roboty rak rzemieslniczych; i jezliby jeszcze kto chcial co dobrowolnie dzis ofiarowac Panu?
\par 6 Tedy dobrowolnie ofiarowali przedniejsi z domów i przedniejsi z pokolen Izraelskich, i pólkownicy, i rotmistrze, i przelozeni nad robota królewska.
\par 7 I zlozyli na usluga domu Bozego zlota talentíw piec tysiecy, i zlotych dziesiec tysiecy, a srebra talentów dziesiec tysiecy, i midzi osmnascie tysiecy talentów, a zelaza sto tysiecy talentów.
\par 8 Ci tez co mieli drogie kamienie, dawali je do sksarbu domu Panskiego, do rak Jehijela Giersonczyka.
\par 9 I weselil sie lud, ze tak ochotnie ofiaroweali. Albowiem sercem doskonalem chetnie ofiarowali Panu; takze i król Dawid weselil sie weselem wielkiem.
\par 10 Przetoz blogoslawil Dawid Panu przed obliczem wszystkiego zgromadzenia, i rzekl: Blogoslawionys ty Panie, Boze Izraela, ojca naszego, od wieku az na wieki.
\par 11 Twoja jest, Panie! wielmoznosc, i moc, i slawa, i zwyciastwo, i czesc, i wszystko na niebie i na ziemi; twoje jest, Panie! królestwo, a tysjest wywzszony nad wszelka zwierzchnosc.
\par 12 I bogactwa, i slawa od ciebie sa, a ty panujesz nad wszystkimi, a w rekach twych jest moc i sila, i w rece twojej jest wywyzszyc i utwierdzic wszystko.
\par 13 Teraz tedy, Boze nasz! wyznajemy cie, a chwalimy imie slawy twojej.
\par 14 Albowiem cuzem ja, i co jest lud mój, zebysmyto sily mieli, tobie to dobrowolnie ofiarowac? gdyz od ciebie jest wszystko, a z rak twoich wziawszy dalismy tobie.
\par 15 Bosmy my pielgrzymami i przychodniami przed toba, jako i wszyscy ojcowie nasi; dni nasze na ziemi jako cien, a nie masz czego oczekiwac.
\par 16 O Panie, Boze nasz! ten wszystek dostatek którysmy zgotowali tobie na budowanie domu imieniowi twemu swietemu, z reki twojej jest, i twoje jest wszystko.
\par 17 Wiemci ja, Boze mój! iz ty doswiadczasz serc a kochasz sie w szczerosci; przetoz ja w szczerosci serca mego, ochotniem pofiarowal to wszystko, nawet i lud twój, który sie tu znalazl, widzialem z weselem i z ochota ofiarujacy tobie.
\par 18 Panie, Boze Abrahama, Izaaka, i Izraela, ojców naszych! zachowajze na wieki te chec, i umysl serca ludu twego, a przygotuj sobie serce ich.
\par 19 Salomonowi tez, synowi memu daj serce doskonale, aby strzegl przykazan twoich, swiadectw twoich, i ustaw twoich, i czynil wszystko, i aby zbudowal dom, dla któregom potrzebu zgotowal.
\par 20 Potem mówil Dawid do wszystkiego zgromadzenia: Blogoslawciez teraz Panu, Bogu waszemu. I blogoslawilo wszystko zgromadzenie Panu, Bogu ojców sowich, a naczyliwszy sie poklonili sie Panu i królowi.
\par 21 I zatem ofiarowali Panu ofiary. Ofiarowali tez calopalenia Panu nazajutrz po onym dniu, wolów tysiac, baranów tysiac, baranków tysiac z mokremi ofiarami ich, i inszych ofiar wielkie mnóstwo, za wszystkiego Izraela.
\par 22 I jedli a pili przed Panem dnia onego z weselem wielkiem, a postanowili powtóre królem Salomona, syna Dawidowego, i pomazali go Panu za ksiecia, a Sadoka za kaplana.
\par 23 A tak usiadl Salomon na stolicy Panskiej za króla miasto Dawida, ojca swego, i szczescilo mu sie, a byl mu posluszny wszystek Izrael.
\par 24 I wszyscy ksiazeta, i mozni, takze i wszyscy synowie króla Dawida, dali rece na poddanstwo Salomonowi królowi.
\par 25 I uwielbil Pan Salomona bardzo zacnie przed oczyma wszystkiego Izraela, a dal mu slawe królewskoa, jakiej zaden król przed nim nie mial w Izraelu.
\par 26 A tak Dawid, syn Isajego, królowal nad wszystkim Izraelem.
\par 27 A dni, których królowal nad Izraelem, bylo czterdziesci lat; w Hebronie królowal siedm lat; a w Jeruzalemie królowal trzydziesci i trzy lata.
\par 28 I umarl w starosci dobrej, pelen dni, bogactw i slawy: a królowal Salomon, syn jego, miasto niego.
\par 29 A sprawy króla Dawida pierwsze i ostatnie, oto sa zapisane w ksiegach Samuela widzacego, i w ksiedze Natana proroka, i w skiedze Gada widzacego;
\par 30 Ze wszystkiem królowaniem jego, i moznoscia jego, i z czasami, które za niego i za Izraela, i za wszystkich królestw ziemskich przeszly.


\end{document}