\begin{document}

\title{Amosa}


\chapter{1}

\par 1 Slowa Amosa, który byl miedzy pasterzami z Tekua, które widzial o Izraelu za dni Uzyjasza, króla Judzkiego, i za dni Jeroboama, syna Joazowego, króla Izraelskiego, dwa lata przed trzesieniem ziemi.
\par 2 i rzekl: Zaryczy Pan z Syonu, z Jeruzalemu wyda glos swój; i beda plakaly mieszkania pasterzy, a wyschna pola najwyborniejsze.
\par 3 Tak mówi Pan: Dla trzech wystepków Damaszku, owszem, dla czterech, nie przepuszcze mu, przeto, ze mlócili wozami zelaznemi Galaada;
\par 4 Ale posle ogien na dom Hazaela, który pozre palace Benadadowe.
\par 5 Polamie tez zawore w Damaszku, a wykorzenie obywatela z doliny Awen, i tego, który trzyma sceptr z domu Heden; i pójdzie w niewole lud Syryjski do Kir, mówi Pan.
\par 6 Tak mówi Pan: Dla trzech wystepków Gazy, owszem, dla czterech, nie przepuszcze im, przeto, ze ich pojmawszy, w wiezienie wieczne podawali Edomczykom;
\par 7 Ale posle ogien na mur Gazy, który pozre palace jej.
\par 8 Wykorzenie tez obywatela z Azotu, i tego, który trzyma sceptr z Aszkalonu, i obróce reke moje przeciwko Akkaronowi, ze zginie ostatek Filistynów, mówi panujacy Pan.
\par 9 Tak mówi Pan: Dla trzech wystepków Tyru, owszem, dla czterech, nie przepuszcze mu, przeto, ze ich w wiezienie wieczne podali Edomczykom, a nie wspomnieli na przymierze braterskie;
\par 10 Ale posle ogien na mur Tyrski, który pozre palace jego.
\par 11 Tak mówi Pan: Dla trzech wystepków Edoma, owszem, dla czterech, nie przepuszcze mu, przeto, ze zepsowawszy w sobie wszelaka litosc swoje przesladuje mieczem brata swego, a gniewem swym ustawicznie pala, owszem, zapalczywosc jego rozsila sie bez przestania;
\par 12 Ale posle ogien na Teman, i pozre palace w Bocra.
\par 13 Tak mówi Pan: Dla trzech wystepków synów Amonowych, owszem, dla czterech, nie przepuszcze mu, przeto, iz rozcinali brzemienne w Galaad, tylko aby rozszerzali granice swoje;
\par 14 Ale rozniece ogien na murze Rabby, który pozre palace jego z krzykiem w dzien wojny, i z wichrem w dzien niepogody.
\par 15 I pójdzie król ich w niewole, on i ksiazeta jego z nim, mówi Pan.

\chapter{2}

\par 1 Tak mówi Pan: Dla trzech wystepków Moaba, owszem, dla czterech, nie przepuszcze mu, przeto, iz spalil kosci króla Edomskiego na popiól;
\par 2 Ale posle ogien na Moaba, który pozre palace Karyjot; i umrze Moab w huku, w krzyku i w glosie traby.
\par 3 I wygladze sedziów z posrodku jego, i wszystkich ksiazat jego pobije z nim, mówi Pan.
\par 4 Tak mówi Pan: Dla trzech wystepków Judzkich, owszem, dla czterech, nie przepuszcze mu, przeto, ze odrzucaja zakon Panski, i ustaw jego nie przestrzegaja, a dadza sie zwodzic klamstwom swoim, których nasladowali ojcowie ich;
\par 5 Ale posle ogien na Jude, który pozre palace Jeruzalemskie.
\par 6 Tak mówi Pan: Dla trzech wystepków Izraelskich, owszem, dla czterech, nie przepuszcze mu, przeto, ze sprawiedliwego za pieniadze sprzedawaja, a ubogiego za pare trzewików;
\par 7 Którzy usiluja, aby na proch potarli glowy ubogich, a droge pokornych podwracaja; nadto syn i ojciec jego wchodza do jednejze dziewki, aby splugawili imie swietobliwosci mojej;
\par 8 I na szatach zastawionych klaniaja sie przy kazdym oltarzu, a wino tych, co podpadli pod kazn, pija w domu bogów swoich.
\par 9 Chociazem Ja wytracil Amorejczyka od oblicza ich, którego wysokosc byla jako wysokosc cedrów, aczkolwiek warownie stal jako dab, wszakzem skazil owoc jego z wierzchu, a korzenie jego ze spodku.
\par 10 A was wywiodlem z ziemi Egipskiej, i prowadzilem was po puszczy czterdziesci lat, zebyscie posiedli ziemie Amorejczyka.
\par 11 Nadto wzbudzalem z synów waszych proroków, a z mlodzienców waszych Nazarejczyków; izali nie tak jest, o synowie Izraelowi? mówi Pan.
\par 12 Alescie wy napawali Nazarejczyków winem, a prorokom zakazywaliscie, mówiac: Nie prorokujcie.
\par 13 Oto Ja scisne ziemie wasze, tak jako cisnie wóz napelniony snopami.
\par 14 I zginie ucieczka od predkiego, a mocarz nie pokrzepi mocy swojej, i duzy nie wybawi duszy swojej;
\par 15 A ten, który trzyma luk, nie ostoi sie, i predki na nogi swe nie uciecze, a ten, który jezdzi na koniu, nie zachowa duszy swej,
\par 16 Ale i rycerz serca zmezalego miedzy mocarzami nago uciecze w on dzien, mówi Pan.

\chapter{3}

\par 1 Sluchajcie slowa tego, które mówi Pan przeciwko wam, synowie Izraelscy! przeciwko wszystkiemu rodzajowi, którym wywiódl z ziemi Egipskiej, mówiac:
\par 2 Tylkom was samych poznal ze wszystkich rodzajów ziemi; przetoz was nawiedze dla wszystkich nieprawosci waszych.
\par 3 Izali dwa spolem pójda nie zgodziwszy sie?
\par 4 Izali zaryczy lew w lesie, gdyby nie mial lupu? Izali wyda lwie glos swój z jaskini swojej, gdyby lapac nie mialo?
\par 5 Izali ptak wpadnie w sidlo na ziemi, gdyby sidla nie bylo? Izali bedzie podniesione sidlo z ziemi, gdyby nic nie uwiezlo?
\par 6 Izali sie ludzie nie lekaja, gdy traba w miescie zabrzmi? izali jest w miescie co zlego, którego by Pan nie uczynil?
\par 7 Zaiste nic nie czyni panujacy Pan, chyba zeby objawil tajemnice swoje slugom swoim, prorokom.
\par 8 Lew ryczy, któzby sie nie bal? Panujacy Pan mówi, któzby nie prorokowal?
\par 9 Obwolajcie w palacach w Azocie i w palacach ziemi Egipskiej, a mówcie: Zbierzcie sie na góry Samaryi, a obaczcie wielkie zamieszania w posrodku jej, i ucisk cierpiacych w niej;
\par 10 I ze nie umieja czynic, co jest prawego, mówi Pan, zbierajac na palacach swoich skarby z zdzierstwa i z lupiestwa.
\par 11 Przetoz tak mówi panujacy Pan: Oto nieprzyjaciel okolo tej ziemi, a ten odejmie od ciebie sile twoje, i rozchwycone beda palace twoje.
\par 12 Tak mówi Pan: Jako wyrywa pasterz z paszczeki lwiej dwa gnaty albo kes ucha, tak wyrwani beda synowie Izraelscy, którzy mieszkaja w Samaryi na stronie loza, i na stronie poscieli.
\par 13 Sluchajcie a oswiadczajcie w domu Jakóbowym, mówi panujacy Pan, Bóg zastepów.
\par 14 Bo dnia onego, którego Izraela nawiedze dla przestepstwa jego, nawiedze tez oltarze w Betel, i odciete beda rogi oltarza, tak, ze na ziemie upadna;
\par 15 I uderze dom zimy o dom lata, a zgina domy z kosci sloniowych, a domy zacne koniec wezma, mówi Pan.

\chapter{4}

\par 1 Sluchajcie slowa tego, o krowy Basanskie! którescie na górach Samaryi, które uciskacie nedzników a niszczycie ubogich, które mówicie panom ich: Przyniescie, abysmy pily.
\par 2 Przysiagl panujacy Pan przez swietobliwosc swoje, iz oto dni ida na was, których nieprzyjaciel wezmie was na haki, a potomki wasze na wedy rybackie, i wyjdziecie przerwami, kazda tak jako stoi;
\par 3 I bedziecie rozrzucac cokolwiek bylo w palacach waszych, mówi Pan.
\par 4 Idzciez do Betel a badzcie tulaczami w Galgal; rozmnózcie przestepstwa, a przynoscie na kazdy poranek ofiary wasze, i trzeciego roku dziesieciny wasze;
\par 5 A palac ofiare chwaly z kwaszonych rzeczy, obwolajcie ofiary dobrowolne, i rozgloscie, poniewaz sie wam tak podoba, o synowie Izraelscy! mówi panujacy Pan.
\par 6 A chociazem Ja wam dal czystosc zebów we wszystkich miastach waszych, to jest, niedostatek chleba po wszystkich miejscach waszych, wszakzescie sie nie nawrócili do mnie, mówi Pan.
\par 7 Jam tez zahamowal od was deszcz, gdy jeszcze byly trzy miesiece do zniwa, a spuscilem deszcz na jedne miasto, a na drugiem miasto nie spuscil; jedna dziedzina byla deszczem odwilzona, a druga dziedzina, na która deszcz nie padal, uschla.
\par 8 I chodzily dwa i trzy miasta do jednego miasta, aby pily wode, a nie mogly sie napic; a wszakzescie sie nie nawrócili do mnie, mówi Pan.
\par 9 Uderzylem was susza i rdza; obfitosc, która przynosily ogrody wasze, i winnice wasze, i figowe sady wasze, i oliwnice wasze, gasienice pozarly, a wszakzescie sie nie nawrócili do mnie, mówi Pan.
\par 10 Poslalem na was mór, tak jako na Egipt, pobilem mieczem mlodzienców waszych, w pojmaniem podal konie wasze, i sprawilem, ze smród wojsk waszych wystepowal w nozdrza wasze; a wszakzescie sie nie nawrócili do mnie, mówi Pan.
\par 11 Wywrócilem was, jako Bóg wywrócil Sodome i Gomore, tak, zescie byli jako glownia wyrwana z ognia; a wszakzescie sie nie nawrócili do mnie, mówi Pan.
\par 12 Przetoz tak ci uczynie, o Izraelu! a iz ci tak uczynic chce, badzze gotowym na zabiezenie Bogu swemu, o Izraelu!
\par 13 Albowiem oto on jest, który ksztaltuje góry, a tworzy wiatry, i który oznajmuje czlowiekowi, jaka jest mysl jego; on z rannej zorzy ciemnosc czyni, a depcze wysokosci ziemi; Pan Bóg zastepów jest imie jego.

\chapter{5}

\par 1 Sluchajcie slowa tego, które Ja wydaje przeciwko wam, to jest narzekania, o domie Izraelski!
\par 2 Upadnie, a nie powstanie wiecej panna Izraelska; opuszczona bedzie w ziemi swej, a nie bedzie, ktoby ja podniósl.
\par 3 Bo tak mówi panujacy Pan: W miescie, z którego wychodzilo tysiac, zostanie sto, a w tem, z którego wychodzilo sto, zostanie dziesiec domowi Izraelskiemu.
\par 4 Bo tak mówi Pan domowi Izraelskiemu: Szukajcie mie, a zyc bedziecie;
\par 5 A nie szukajcie Betela, ani chodzcie do Galgal, i do Beerseby nie udawajcie sie, ;bo Galgal w niewole zawiedzione bedzie, a Betel sie wniwecz obróci.
\par 6 Szukajcie Pana, a zyc bedziecie, by snac domu Józefowego nie przeniknal jako ogien, i nie pochlonal Betel, a nie bylby, ktoby ugasil;
\par 7 Którzy obracacie sad w piolun, a sprawiedliwosc na ziemi opuszczacie: Szukajcie, mówie.
\par 8 Tego, który uczynil Baby na niebie i Oriona, który cien smierci w poranek odmienia i dzien w ciemnosci nocne; który przywoluje wody morskie, a wylewa je na oblicze ziemi, Pan jest imie jego;
\par 9 Który pokrzepia slabego przeciwko mocarzowi, tak ze ten oslabialy do twierdzy uchodzi.
\par 10 Maja w nienawisci tego, który ich w bramie karze; a tym, co mówi rzeczy dobre, brzydza sie.
\par 11 Przetoz, iz uciskacie ubogiego, a brzemie zboza bierzecie od niego, domuwescie wprawdzie z ciosanego kamienia nabudowali, ale nie bedziecie w nich mieszkac; winnic rozkosznych nasadziliscie, ale wina z nich pic nie bedziecie.
\par 12 Bo wiem o wielkich przestepstwach waszych, i srogich grzechach waszych, ze ciemiezycie sprawiedliwego, biorac poczty, a ubogich sprawy w bramie podwracacie.
\par 13 Przetoz roztropny czasu onego milczec musi; bo czas zly jest.
\par 14 Szukajcie dobrego a nie zlego, abyscie zyli; a bedzie tak Pan Bóg zastepów z wami, jako mówicie.
\par 15 Miejcie w nienawisci zle, a milujcie dobre, a sad postanówcie w bramie; owa sie snac Pan, Bóg zastepów, nad ostatkiem Józefa zmiluje.
\par 16 Przetoz tak mówi panujacy Pan, Bóg zastepów: Po wszystkich ulicach bedzie narzekanie, a po wszystkich stronach zakrzykna: Biada, biada! i zawolaja oracza do placzu i do kwilenia z tymi, którzy narzekac umieja.
\par 17 Owszem, i po wszystkich winnicach bedzie narzekanie, gdy przejde przez posrodek ciebie, mówi Pan.
\par 18 Biada tym, którzy zadaja dnia Panskiego! cóz wam po tym dniu Panskim, poniewaz jest ciemnoscia, a nie swiatloscia?
\par 19 Jako gdyby kto uciekal przede lwem, a zabiezal mu niedzwiedz; albo gdyby wszedl do domu, a podparl sie reka swa na scianie, ukasilby go waz.
\par 20 Izali dzien Panski nie jest dzien ciemnosci, a nie swiatlosci, w którym niemasz jasnosci, ale chmura?
\par 21 Mam w nienawisci i odrzucilem uroczyste swieta wasze, ani sie kocham w ofiarach zgromadzenia waszego.
\par 22 Bo jezli mi ofiarowac bedziecie calopalenia, i sniedne ofiary wasze, nie przyjme ich, a na spokojne ofiary tlustych bydel waszych nie wejrze.
\par 23 Odejmij odemnie wrzask piesni swoich; bo ich i dzwieku harf waszych sluchac nie chce.
\par 24 Ale sad nawalnie poplynie, jako woda, a sprawiedliwosc jako strumien gwaltowny.
\par 25 Izaliscie mi ofiary i dar ofiarowali na puszczy przez czterdziesci lat, domie Izraelski?
\par 26 Owszem, nosiliscie namiot Molocha waszego i Kijuna, obrazy wasze, gwiazde bogów waszych, którychescie sobie naczynili.
\par 27 Przetoz was zaprowadze za Damaszek, mówi Pan, Bóg zastepów imie jego.

\chapter{6}

\par 1 Biada bezpiecznym na Syonie, i ufajacym w górze Samaryjskiej! którzy sa slawni mimo innych u tych narodów, do których sie schodzi dom Izraelski.
\par 2 Zajdzcie do Chalny, i idzcie z onad do Emat wielkiego, a zstapcie do Giet Filistynskiego, a obaczcie, sali które królestwa lepsze nizeli te, i jezeli szersza jest granica ich, niz granica wasza.
\par 3 (Biada wam) którzy mniemacie, ze daleki jest dzien zly, a przystawiacie stolice drapiestwa!
\par 4 Którzy sypiacie na lozach sloniowych, a rozciagacie sie na poscielach waszych; którzy jadacie barany z trzody, a cielce tuczone ze stani;
\par 5 Którzy spiewacie przy lutni, wymyslajac sobie naczynia muzyczne, jako Dawid;
\par 6 Którzy pijacie wino czaszami, a drogiemi sie masciami namazujecie, i nie bolejecie nad utrapieniem Józefowem.
\par 7 Przetoz teraz pójda w niewole na czele pojmanych; a tak odstapi biesiada od zbyteczników.
\par 8 Przysiagl panujacy Pan sam przez sie, mówi Pan, Bóg zastepów: Zbrzydzilem sobie pyche Jakóbowa i palace jego mam w nienawisci; przetoz podam miasto i wszystko, co w niem jest, nieprzyjacielowi;
\par 9 A zostanieli dziesiec osób w domu jednym, i ci pomra.
\par 10 I wezmie kazdego z nich stryj jego, i spali go, aby wyniósl kosci z domu, a rzecze temu, który jest w gmachach domu: Jestze kto wiecej z toba? I odpowie: Niemasz. Tedy rzecze: Milcz; przeto, ze nie wspominali imienia Panskiego.
\par 11 Bo oto Pan rozkaze, i uderzy na dom wielki rozstapieniem, a na dom mniejszy rozpadlinami.
\par 12 Izali konie moga biegac po skale? Izali tam wolami orac moga? Boscie obrócili sad w trucizne, a owoc sprawiedliwosci w piolun;
\par 13 Biada wam! którzy sie weselicie, a niemasz z czego, mówiac: Izalismy sobie nie nasza moca wzieli rogi?
\par 14 Ale oto Ja wzbudze przeciwko wam, o domie Izraelski! mówi Pan, Bóg zastepów, naród, który was ucisnie od wejscia do Emat az do strumienia pustyni.

\chapter{7}

\par 1 To mi ukazal panujacy Pan. Oto tworzyl szarancze, gdy najpierwej poczal odrastac potraw, gdy oto potraw byl po pokoszeniu królewskiem.
\par 2 A gdy zjadly trawe ziemi, rzeklem: Panujacy Panie! sfolguj prosze; bo któz zostanie Jakóbowi, gdyz maluczki jest?
\par 3 I zalowal Pan tego; a rzekl Pan: Nie stanie sie.
\par 4 Tedy mi ukazal panujacy Pan, a oto panujacy Pan wolal, ze sprawe swoje powiedzie ogniem, a spaliwszy przepasc wielka, spalil i czesc królestwa Izraelskiego.
\par 5 Tedym rzekl: Panujacy Panie! przestan prosze; bo któz zostanie Jakóbowi, gdyz maluczki jest?
\par 6 I zalowal Pan tego, a rzekl panujacy Pan: I toc sie nie stanie.
\par 7 Potem ukazal mi, a oto Pan stal na murze wedlug sznuru zbudowanym, w którego reku bylo prawidlo.
\par 8 I rzekl Pan do mnie: Cóz widzisz Amosie? I rzeklem: Prawidlo. Tedy rzekl Pan: Oto Ja poloze prawidlo w posrodku ludu mego Izraelskiego, a juz mu wiecej nie bede przegladal.
\par 9 Bo wyzyny Izaakowe spustoszone beda, a swiatnice Izraelskie zburzone beda, gdy powstane przeciwko domowi Jeroboamowemu z mieczem.
\par 10 Tedy poslal Amazyjasz, kaplan Betelski, do Jeroboama, króla Izraelskiego, mówiac: Sprzysiagl sie przeciwko tobie w posrodku domu Izraelskiego, tak, iz ziemia nie moze zniesc wszystkich slów jego.
\par 11 Bo tak mówi Amos: Jeroboam od miecza umrze, a Izrael zapewne do wiezienia z ziemi swojej zaprowadzony bedzie.
\par 12 Potem rzekl Amazyjasz do Amosa: O widzacy! uchodz, uciekaj do ziemi Judzkiej, a jedz tam chleb, i tam prorokuj;
\par 13 Ale w Betelu wiecej nie prorokuj; bo to jest swiatnica królewska, i dom królewski.
\par 14 Tedy odpowiedzial i rzekl do Amazyjasza: Nie bylem ja prorokiem, nawet ani synem prorockim; alem byl skotarzem, a zbieralem figi lesne.
\par 15 Ale mie Pan wzial, gdym chodzil za bydlem, i rzekl do mnie Pan: Idz, prorokuj ludowi memu Izraelskiemu.
\par 16 Teraz tedy sluchaj slowa Panskiego. Ty mówisz: Nie prorokuj w Izraelu, i nie kaz w domu Izaakowym;
\par 17 Przetoz tak mówi Pan: Zona twoja w miescie nierzad plodzic bedzie, a synowie twoi i córki twoje od miecza polegna, a ziemia twoja sznurem bedzie podzielona, a ty w ziemi splugawionej umrzesz; lecz Izrael zapewne zaprowadzony bedzie do wiezienia z ziemi swojej.

\chapter{8}

\par 1 To mi jeszcze ukazal panujacy Pan, oto byl kosz letniego owocu.
\par 2 Tedy rzekl: Cóz widzisz Amosie? I rzeklem: Kosz letniego owocu. Znowu rzekl Pan do mnie: Przyszedl koniec ludowi memu Izraelskiemu, nie bede mu juz wiecej przegladal.
\par 3 Tedy sie obróca w kwilenie piesni koscielne dnia onego, mówi panujacy Pan, mnóstwa trupów na kazde miejsce po cichu narzucaja.
\par 4 Sluchajciez tego, którzy pozeracie ubogiego, abyscie wygubili chudziny z ziemi;
\par 5 I mawiacie: Kiedyz przeminie nów miesiaca, abysmy sprzedawali zboze? i sabat, abysmy otworzyli spichlerze? abysmy umniejszyli miary efa, a podwyzszyli wagi, a szale zdradliwie sfalszowali.
\par 6 Kupujac ubogich za pieniadze, a chudzine za pare trzewików; nadto abysmy odmieciny zbóz sprzedawali.
\par 7 Przysiagl Pan przez zacnosc Jakóbowa, ze nie zapomne na wieki wszystkich spraw ich.
\par 8 Izali by sie i ziemia nad tem nie poruszyla, i nie plakalby kazdy, kto mieszka na niej? i owszem, wzbierze wszystka jako rzeka, i porwana i zatopiona bedzie jako rzeka Egipska.
\par 9 A dnia onego, mówi panujacy Pan, sprawie, ze slonce zajdzie o poludniu, i przywiode ciemnosc na ziemie w dzien jasny;
\par 10 I obróce w placz swieta wasze, a wszystkie piesni wasze w narzekanie, i sprawie to, ze bedzie na wszystkich biodrach wór, i na kazdej glowie oblysienie; i bedzie w tej ziemi kwilenie, jako nad jednorocznym, a ostateczne rzeczy jej jako dzien gorz kosci.
\par 11 Oto dni przychodza, mówi panujacy Pan, ze posle glód na ziemie, nie glód chleba, ani pragnienie wody, ale sluchania slów Panskich,
\par 12 Tak, ze sie tulac beda od morza az do morza, i od pólnocy az na wschód biegac beda, szukajac slowa Panskiego, wszakze nie znajda.
\par 13 Dnia onego pomdleja panienki piekne, nawet i mlodziency od onego pragnienia;
\par 14 Którzy przysiegaja przez obrzydliwosc Samaryi, i mówia: Jako zyje Bóg twój, o Dan! i jako zyje droga Beerseba; i upadna, a nie powstana wiecej.

\chapter{9}

\par 1 Widzialem Pana stojacego na oltarzu, który rzekl: Uderz w galke, az zadrza podwoje, a rozetnij je wszystkie od wierzchu ich, a ostatek mieczem pobije; zaden z nich nie uciecze, i nie bedzie z nich nikt, coby tego uszedl.
\par 2 Chocby sie zakopali w ziemie, i stamtadby ich reka moja wziela; chocby wstapili az do nieba, i stamtadby ich stargnal.
\par 3 A chocby sie skryli na wierzchu Karmelu, wyszpieguje i wezme ich stamtad; a chocby sie skryli przed oczyma mojemi na dnie morskiem, przykaze wezowi, aby ich i stamtad wykasal;
\par 4 A chocby poszli w niewole przed nieprzyjaciólmi swymi, i tam przykaze mieczowi, aby ich pomordowal; obróce zaiste przeciwko nim oko swe na zle, a nie na dobre.
\par 5 Bo panujacy Pan zastepów, gdy sie dotknie ziemi, rozplywa sie, a placza wszyscy mieszkajacy na niej, i wzbiera wszystka jako rzeka, a zatopiona bywa jako rzeka Egipska.
\par 6 Który na niebiesiech zbudowal palace swoje, a zastep swój na ziemi uszykowal; który moze zawolac wody morskie, a wylac je na oblicze ziemi; Pan jest imie jego.
\par 7 Izali nie jestescie podobni synom Murzynskim przedemna, o synowie Izraelscy? mówi Pan; izalim Izraela nie wywiódl z ziemi Egipskiej jako Filistynczyków z Kaftor, i Syryjczyków z Kir?
\par 8 Oto oczy panujacego Pana przeciwko temu królestwu grzeszacemu, abym je wygladzil z oblicza ziemi; wszakze nie wygladze do szczetu domu Jakóbowemu, mówi Pan.
\par 9 Bom oto Ja rozkazal, a rozmiece miedzy wszystkie narody dom Izraelski jako miotana bywa pszenica na przetaku, tak, iz nie przepadnie i kamyk na ziemie.
\par 10 Wszyscy grzesznicy z ludu mojego od miecza pomra, którzy mówia: Nie przyblizy sie do nas, ani nas zachwyci to zle.
\par 11 Dnia onego wystawie upadly przybytek Dawidowy, a zagrodze rozerwanie jego, i obaliny jego naprawie, a pobuduje go, jako za dni dawnych;
\par 12 Aby posiedli ostatki Edomczyków i wszystkie narody nad którymi wzywano imienia mojego, mówi Pan, który to czyni.
\par 13 Oto dni ida, mówi Pan, ze oracz zence zajmie, a ten, co tloczy winne jagody, rozsiewajacego nasienie; a góry moszczem kropic beda, a wszystkie pagórki sie rozplyna.
\par 14 I nawróce zas z wiezienia lud mój Izraelski, i pobuduja miasta spustoszone, a mieszkac w nich beda; sadzic tez beda winnice, i wino z nich pic beda; sadów tez naszczepia, i owoc ich jesc beda.
\par 15 A tak ich wszczepie w ziemi ich, ze nie beda wiecej wykorzenieni z ziemi swojej, któram im dal, mówi Pan, Bóg twój.


\end{document}