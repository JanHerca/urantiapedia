\begin{document}

\title{Lamentacje}


\chapter{1}

\par 1 W biblii siedmdziesieciu tlómaczów, ta ksiega swieta tak sie zaczyna. I stalo sie, gdy Izrael pojmany byl, a Jeruzalem spustoszone, ze Jeremijasz siedzial placzac, i narzekal narzekaniem takiem nad Jeruzalemem, a rzekl: Ach miasto tak ludne jakoz siedzi samotne! stalo sie jako wdowa; zacne miedzy narodami, przednie miedzy krainami stalo sie holdowne.
\par 2 Ustawicznie w nocy placze, a lzy jego na jagodach jego; niemasz, ktoby je cieszyl ze wszystkich milosników jego; wszyscy przyjaciele jego przeniewierzyli mu sie, stali mu sie nieprzyjaciolmi.
\par 3 Przeniósl sie Juda dla utrapienia i dla wielkiej niewoli; wszakze mieszkajac miedzy narodami nie znajduje odpocznienia; wszyscy, którzy je gonia polapali je w ciesni.
\par 4 Drogi Syonskie placza, ze nikt nie przychodzi na swieto uroczyste. Wszystkie bramy jego spustoszaly, kaplani jego wzdychaja, panny jego smutne sa, a samo pelne jest gorzkosci.
\par 5 Nieprzyjaciele jego sa glowa, przeciwnikom jego szczesliwie sie powodzi; bo go Pan utrapil dla mnóstwa przestepstwa jego; maluczcy jego poszli w niewole przed obliczem trapiacego.
\par 6 A tak odjeta jest od córki Syonskiej wszystka ozdoba jej; ksiazeta jej staly sie jako jelenie nie znajdujacy paszy, i uchodza bez sily przed tym, który je goni.
\par 7 Wspomina córka Jeruzalemska we dni utrapienia swego i kwilenia swego na wszystkie uciechy swoje, które miewala ode dni dawnych, gdy pada lud jej od reki nieprzyjacielskiej, nie majac, ktoby jej ratowal; widzac ja nieprzyjaciele nasmiewali sie z sa batów jej.
\par 8 Ciezko zgrzeszyla córka Jeruzalemska, przetoz jako nieczysta odlaczona jest. Wszyscy, którzy ja w uczciwosci mieli, lekce ja sobie waza, przeto, ze widza nagosc jej, a ona wzdycha, i tylem sie obraca.
\par 9 Nieczystota jej na podolkach jej, a nie pomniala na koniec swój; przetoz znacznie jest znizona, nie majac, ktoby ja pocieszyl. Wejrzyj, Panie! na utrapienie moje; boc sie wyniósl nieprzyjaciel.
\par 10 Reke swoje wyciagnal nieprzyjaciel na wszystkie kochania jej; bo musi patrzyc na pogan wchodzacych do swiatnicy jej, o czemes byl przykazal, aby nie wchodzili do zgromadzenia twego.
\par 11 Wszystek lud jej wzdychajac chleba szuka, daje kosztowne rzeczy swoje za pokarm ku posileniu duszy. Wejrzyj, Panie! a obacz; bom zniewazona.
\par 12 Nicze was to nie obchodzi? o wszyscy, którzy mimo idziecie droga! Obaczcie, a ogladajcie, jezli jest bolesc, jako moja bolesc, która mi jest zadana, jako mie zasmucil Pan w dzien gniewu zapalczywosci swojej.
\par 13 Z wysokosci poslal ogien w kosci moje, który je opanowal; rozciagnal siec nogom moim, obrócil mie na wstecz, podal mie na spustoszenie, przez caly dzien zalosna.
\par 14 Zwiazane jest jarzmo nieprawosci moich reka jego, splotly sie, wstapily na szyje moje; toc porazilo sile moje; podal mie Pan w rece nieprzyjaciól, nie moge powstac.
\par 15 Pan podeptal wszystkich mocarzy moich w posród mnie, zwolal przeciwko mnie gromady, aby starl mlodzienców moich, Pan tloczyl jako w prasie panne, córke Judzka.
\par 16 Przetoz ja placze; z oczów moich, z oczów moich, mówie, wody cieka, ze jest daleko odemnie pocieszyciel, któryby ochlodzil dusze moje, synowie moi wytraceni sa, przeto, iz wzial góre nieprzyjaciel.
\par 17 Rozciaga córka Sydonska rece swoje, nie ma, ktoby ja cieszyl; wzbudzil Pan na Jakóba zewszad w okolo nieprzyjaciól jego; córka Jeruzalemska jest miedzy nimi, niby dla nieczystosci oddalona.
\par 18 Sprawiedliwy jest Pan; bom ustom jego odporna byla. Sluchajcie, prosze, wszyscy ludzie, a obaczcie bolesc moje; panny moje, i mlodziency moi poszli w niewole.
\par 19 Wolalam na przyjaciól moich, oni mie zdradzili; kaplani moi i starcy moi w miescie zgineli, szukajac sobie pokarmu, aby posilili dusze swoje.
\par 20 Wejrzyz, Panie, bomci utrapiona, wnetrznosci moje strwozone sa, wywrócilo sie serce moje we mnie, przeto, zem byla bardzo odporna; na dworze miecz osieraca, a w domu nic niemasz jedno smierc.
\par 21 Slyszac, ze ja wzdycham, ale niemasz, ktoby mie pocieszyl; wszyscy nieprzyjaciele moi slyszac o nieszczesciu mojem wesela sie, zes ty to uczynil, a przywiodles dzien przedtem ogloszony; alec beda mnie podobni.
\par 22 Niech przyjdzie wszystka zlosc ich przed oblicznosc twoje, a uczyn im, jakos mnie uczynil dla wszystkich przestepstw moich; bo wielkie sa wzdychania moje, a serce moje zalosne.

\chapter{2}

\par 1 Jokoz zacmil Pan w zapalczywosci swoje córke Syjonska! zrzucil z nieba na ziemie slawe Izraelska, a nie wspomnial na podnózek nóg swoich w dzien zapalczywosci swojej.
\par 2 Polknal Pan bez wszelkiej litosci wszystkie przybytki Jakóbowe, zburzyl w popedliwosci swojej twierdze córki Judzkiej, uderzyl je o ziemie, w hanbe oddal królestwo i ksiazat jej.
\par 3 Odcial w gniewie zapalczywosci wszystek róg Izraelski, odwrócil nazad prawice swoje od nieprzyjaciela, a rozpaliwszy sie przeciwko Jakóbowi, jako ogien palajacy pozera do szczetu w okolo.
\par 4 Naciagnal luk swój, jako nieprzyjaciel, postawil prawice swoje jako przeciwnik, i pozabijal wszystkich najpozorniejszych z ludu, a w namiocie córki Syonskiej wylal jako ogien popedliwosc swoje.
\par 5 Pan sie stal jako nieprzyjaciel, polknal Izraela, polknal wszystkie palace jego, popsul twierdze jego, i rozmnozyl w ludu Judzkim placz i narzekanie.
\par 6 Oderwal moca plot swój jako od ogrodu, zepsul namiot swój; Pan przywiódl w zapamietanie w Syonie uroczyste swieta i sabaty, a odrzucil w gniewie popedliwosci swojej króla i kaplana.
\par 7 Pan odrzucil oltarz swój, zbrzydzil sobie swiatnice swoje, podal do rak nieprzyjacielskich mury i palace Syonskie; krzyczeli w domu Panskim jako w dzien swieta uroczystego.
\par 8 Umyslil Pan rozwalic mur córki Syonskiej, rozciagnal sznur, a nie odwrócil reki swojej od skazenia; rozkwilil baszty, i mur, tak ze wespól omdlewaja.
\par 9 Zapadly w ziemie bramy jej, polamal i pokruszyl zawory jej; król jej i ksiazeta jej sa miedzy poganami; niemasz ani zakonu, takze ani prorocy jej nie miewaja widzenia od Pana.
\par 10 Starcy córki Syonskiej usiadlszy na ziemi umilkneli, posypali prochem glowe swoje, a przepasuja sie worami; panny Jeruzalemskie zwiszaja ku ziemi glowy swe.
\par 11 Oczy moje od lez ustaly; strwozyly sie wnetrznosci moje, wylala sie na ziemie watroba moja dla starcia córki ludu mojego, gdy i niemowlatka, i dziatki ssace na ulicach miasta omdlewaja;
\par 12 Matkom swoim mówia: Gdziez jest zboze i wino? Gdy mdleja jako zranieni po ulicach miasta, i wypuszczaja dusze swoje na lonie matek swych.
\par 13 Kogoc za swiadka stawie? Kogo tobie przyrównam, o córko Jeruzalemska? Kogoc przypodobam, abym cie ucieszyl, panno, córko Syonska? bo skruszenie twoje wielkie jako morze, któz cie uleczy?
\par 14 Prorocy twoi opowiadalic klamstwo i marnosc, a nie odkrywali nieprawosci twojej, aby odwrócili pojmanie twoje; alec przepowiadali ciezary, klamstwa i wygnanie.
\par 15 Klaskaja nad toba rekoma wszyscy, którzy ida droga, swistaja, a chwieja glowa swoja nad córka Jeruzalemska, mówiac: A onoz to miasto, o którem powiadano, ze jest doskonalej pieknosci, i weselem wszystkiej ziemi?
\par 16 Otworzyli na cie usta swe wszyscy nieprzyjaciele twoi, swistaja i zgrzytaja zebami, mówiac: Pozryjmy je; tenci jest zaiste on dzien, któregosmy czekali, znalezlismy i ogladalismy go.
\par 17 Uczynil Pan, co byl umyslil, wypelnil slowo swoje, które byl przykazal ode dni dawnych; zburzyl bez litosci, a rozweselil nad toba nieprzyjaciela, wywyzszyl róg przeciwników twoich.
\par 18 Wolalo serce ich do Pana. O murze córki Syonskiej! wylewaj lzy we dnie i w nocy jako strumien, nie dawaj sobie odpocznienia, a niech sie nie uspokaja zrenica oka twego.
\par 19 Wstan, wolaj w nocy na poczatku strazy, wylewaj serce twoje przed obliczem Panskiem jako wode; podnos do niego rece swoje za dusze dziatek swych, które omdlewaja od glodu na rogu wszystkich ulic, a rzecz: Wejrzyj Panie! a obacz, komus tak kiedy u czynil?
\par 20 Izali maja niewiasty jesc plód swój, niemowlatka ucieszne? Izali zamordowany byc ma w swiatnicy Panskiej kaplan i prorok?
\par 21 Lezy na ziemi po ulicach dziecie, i starzec; panny moje, i mlodziency moi polegli od miecza; pobiles ich w dzien zapalczywosci twojej, pomordowales ich, a nie sfolgowales.
\par 22 Zwolales strachów moich zewszad, jako w dzien uroczystego swieta, a nie byl w dzien zapalczywosci Panskiej, ktoby uszedl a zyw zostal; którychem na reku piastowala i wychowywala, tych nieprzyjaciel mój wyniszczyl.

\chapter{3}

\par 1 Jam jest ten maz, którym widzial utrapienie od rózgi rozgniewania Bozego.
\par 2 Zaprowadzil mie, i zawiódl do ciemnosci, a nie do swiatlosci;
\par 3 Tylko sie na mie obórzyl, a obrócil reke swoje przez caly dzien.
\par 4 Do starosci przywiódl cialo moje i skóre moje, a polamal kosci moje.
\par 5 Obudowal mie a ogarnal zólcia i praca;
\par 6 W ciemnych miejscach posadzil mie, jako tych, którzy dawno pomarli.
\par 7 Ogrodzil mie, abym nie wyszedl, obciazyl okowy moje;
\par 8 A choc wolam i krzycze, zatula uszy na modlitwe moje.
\par 9 Ogrodzil droge moje ciosanym kamieniem, scieszki moje wywrócil.
\par 10 Jest jako niedzwiedziem czyhajacym na mie, jako lwem w skrytosciach.
\par 11 Drogi moje odwrócil, owszem, rozszarpal mie, i uczynil mie spustoszona.
\par 12 Naciagnal luk swój, a postawil mie jako cel strzalom swym.
\par 13 Przestrzelil nerki moje strzalami z sajdaka swego.
\par 14 Jestem posmiewiskiem ze wszystkim ludem moim, piesnia ich przez caly dzien.
\par 15 Nasyca mie gorzkosciami; upija mie piolunem.
\par 16 Nadto pokruszyl o kamyczki zeby moje, i pograzyl mie w popiele.
\par 17 Takes oddalil, o Boze! od pokoju dusze moje, az na wczasy zapominam.
\par 18 I mówie: Zginela sila moja, i nadzieja moja, któram mial w Panu.
\par 19 Wszakze wspominajac na utrapienie moje, i na placz mój, na piolun, i na zólc.
\par 20 Wspominajac ustawicznie, uniza sie we mnie dusza moja.
\par 21 Przywodzac to sobie do serca swego, mam nadzieje.
\par 22 Wielkie jest milosierdzie Panskie, zesmy do szczetu nie zgineli; nie ustawaja zaiste litosci jego.
\par 23 Ale sie na kazdy poranek odnawiaja; wielka jest prawda twoja.
\par 24 Pan jest dzialem moim, mówi dusza moja, dlatego mam w nim nadzieje.
\par 25 Dobry jest Pan tym, którzy nan oczekuja, duszy takowej, która go szuka.
\par 26 Dobrze jest, cierpliwie oczekiwac na zbawienie Panskie.
\par 27 Dobrze jest mezowi nosic jarzmo od dziecinstwa swego;
\par 28 Który bedac opuszczony, cierpliwym jest w tem, co nan wlozono;
\par 29 Kladzie w prochu usta swe, azby sie okazala nadzieja;
\par 30 Nadstawia bijacemu policzka, a nasycony bywa obelzeniem.
\par 31 Bo Pan na wieki nie odrzuca;
\par 32 Owszem, jezli zasmuca, zasie sie zmiluje wedlug mnóstwa milosierdzia swego.
\par 33 Zaiste nie z serca trapi i zasmuca synów ludzkich.
\par 34 Aby kto starl nogami swemi wszystkich wiezniów w ziemi;
\par 35 Aby kto niesprawiedliwie sadzil meza przed obliczem Najwyzszego;
\par 36 Aby kto wywrócil czlowieka w sprawie jego, Pan sie w tem nie kocha.
\par 37 Któz jest, coby rzekl: Stalo sie, a Pan nie przykazal?
\par 38 Izali z ust Najwyzszego nie pochodzi zle i dobre?
\par 39 Przeczzeby tedy sobie utyskiwac mial czlowiek zyjacy, a maz nad kaznia za grzechy swoje.
\par 40 Dowiadujmy sie raczej, a badajmy sie dróg naszych, nawrócmy sie do Pana;
\par 41 Podniesmy serca i rece nasze w niebo do Boga.
\par 42 Mysmy wstapili i stalismy sie odpornymi; przetoz ty nie odpuszczasz.
\par 43 Okryles sie zapalczywoscia, i gonisz nas, mordujesz, a nie szanujesz.
\par 44 Okryles sie oblokiem, aby cie nie dochodzila modlitwa.
\par 45 Za smieci i za pomiotlo polozyles nas w posrodku tych narodów.
\par 46 Otworzyli na nas usta swoje wszyscy nieprzyjaciele nasi.
\par 47 Strach i dól przyszedl na nas, spustoszenie i skruszenie.
\par 48 Strumienie wód plyna z oczów moich, dla skruszenia córki ludu mojego.
\par 49 Oczy moje plyna bez przestanku, przeto, ze niemasz zadnej ulgi,
\par 50 Azby wejrzal i obaczyl Pan z nieba.
\par 51 Oczy moje trapia dusze moje dla wszystkich córek miasta mojego.
\par 52 Lowili mie ustawicznie jako ptaka nieprzyjaciele moi bez przyczyny.
\par 53 Wrzucili do dolu zywot mój, a przywalili mie kamieniem.
\par 54 Wezbraly wody nad glowa moja, i rzeklem: Juzci po mnie!
\par 55 Wzywam imienia twego, o Panie! z dolu bardzo glebokiego.
\par 56 Glos mój wysluchiwales; nie zatulajze ucha twego przed wzdychaniem mojem, i przed wolaniem mojem.
\par 57 Przyblizajac sie do mnie w dzien, któregom cie wzywal, mawiales: Nie bój sie.
\par 58 Zastawiales sie, Panie! o sprawe duszy mojej, a wybawiales zywot mój.
\par 59 Widzisz, o Panie! bezprawie, które mi sie dzieje, osadzze sprawe moje.
\par 60 Widzisz wszystke pomste ich, i wszystkie zamysly ich przeciwko mnie.
\par 61 Slyszysz uraganie ich, o Panie! i wszystkie zamysly ich przeciwko mnie.
\par 62 Slyszysz wargi powstawajacych przeciwko mnie, i przemysliwanie ich przeciwko mnie przez caly dzien.
\par 63 Obacz siadanie ich, i wstawanie ich; jam zawzdy jest piesnia ich.
\par 64 Oddajze im nagrode, Panie! wedlug sprawy rak ich;
\par 65 Dajze im zatwardziale serce, i przeklestwo swe na nich;
\par 66 Gon ich w zapalczywosci, a zgladz ich, aby nie byli pod niebem twojem, o Panie!

\chapter{4}

\par 1 O jakoz posniedzialo zloto! zmienilo sie wyborne zloto, rozmiotano kamienie swiatnicy, po rogach wszystkich ulic.
\par 2 Szlachetni synowie Syonscy, którzy byli przyrównani do zlota szczerego, jakoz sa poczytani za naczynie gliniane, za dzielo rak garncarskich!
\par 3 I smoki wiec podawajac piersi, karmia mlode swoje; ale córka ludu mojego dla okrutnika podobna jest sowie na puszczy.
\par 4 Przylgnal jezyk ssacego do podniebienia jego dla upragnienia, dzieci prosza o chleb: ale niemasz, ktoby im go ulamal.
\par 5 Ci, którzy jadali potrawy rozkoszne, gina na ulicach, a którzy byli wychowani w szarlacie, przytulaja sie do gnoju.
\par 6 Wieksze jest karanie córki ludu mojego, nizeli pomsta Sodomy, która jest podwrócona w jednem okamgnieniu, i nie zostaly na niej rece.
\par 7 Czystsi byli Nazarejczycy jego nad snieg, jasniejsi nad mleko, rumiensze ciala ich, nizeli drogie kamienie, jakoby z szafiru wyciosani byli;
\par 8 Ale teraz wejrzenie ich czerniejsze jest niz czarnosc, nie moga poznani byc na ulicach; przyschla skóra ich do kosci ich, wyschla jest jako drzewo.
\par 9 Lepiej sie tym stalo, którzy sa pobici mieczem, nizeli tym, co umieraja glodem, gdyz oni zgineli przebitymi bedac, ale ci dla niedostatku urodzajów polnych.
\par 10 Rece niewiast milosiernych warzyly synów swych, aby im byli za pokarm w potarciu córki ludu mego.
\par 11 Wypelnil Pan popedliwosc swoje, i wylal gniew zapalczywosci swojej, i zapalil ogien na Syonie, który pozarl grunty jego.
\par 12 Nigyby byli nie wierzyli królowie ziemscy, i wszysscy obywatele swiata, zeby byl mial wnijsc przeciwnik, i nieprzyjaciel w bramy Jeruzalemskie.
\par 13 Ale sie to stalo dla grzechów proroków jego, i nieprawosci kaplanów jego, którzy wylewali w posrodku jego krew sprawiedliwych.
\par 14 Tulali sie jako slepi po ulicach, mazac sie krwia, której nie mogli, tylko sie dotykac szatami swemi.
\par 15 Przetoz wolali na nich: Ustepujcie, nieczysci! ustepujcie, ustepujcie, nie dotykajcie sie! Prawiec ustapili, i tulaja sie; dlatego mówia miedzy narodami: Nie beda juz wiecej mieli wlasnego mieszkania.
\par 16 Oblicze Panskie rozproszylo ich, a nie wejrzy na nich wiecej; nieprzyjaciele kaplanów nie szanuja, a nad starcami milosierdzia nie uzywaja.
\par 17 A wzdzy jeszcze az do ustania oczów swych wygladamy próznego ratunku swego; ogladajac sie na naród, który wybawic nie moze.
\par 18 Szlakuja stopy nasze, tak, ze ani po ulicach naszych chodzic nie mozemy; przyblizyl sie koniec nasz, wypelnily sie dni nasze, zaiste przyszlo dokonczenie nasze.
\par 19 Predsi sa ci, którzy nas gonia, niz orly niebieskie; po górach nas gonia, na pustyniach czyhaja na nas.
\par 20 Tchnienie nozdrzy naszych, to jest pomazaniec Panski, pojmany jest w jamach ich, o którymesmy mówili: W cieniu jego zyc bedziemy miedzy narodami.
\par 21 Raduj sie i wesel sie córko Edomska! która mieszkasz w ziemi Hus; przyjdzie tez do ciebie kubek, upijesz sie, i obnazysz sie.
\par 22 Wzielo koniec karanie twoje, o córko Syonska! nie zaniecha cie Bóg dluzej w pojmaniu twojem; ale twoje nieprawosc nawiedzi, o córko Edomska! a odkryje grzechy twoje.

\chapter{5}

\par 1 Wspomnij, Panie! na to, co sie nam przydalo; wejrzyj a obacz pohanbienie nasze.
\par 2 Dziedzictwo nasze obrócone jest do obcych, a domy nasze do cudzoziemców.
\par 3 Sierotamismy a bez ojca; matki nasze sa jako wdowy.
\par 4 Wody nasze za pieniadze pijemy, drwa nasze za pieniadze kupujemy.
\par 5 Na szyi swej przesladowanie cierpiemy, pracujemy, a nie dadza nam odpoczac.
\par 6 Egipczykom podajemy reke i Assyryjczykom, zebysmy sie nasycili chleba.
\par 7 Ojcowie nasi zgrzeszyli, niemasz ich, a my nieprawosc ich ponosimy.
\par 8 Niewolnicy panuja nad nami, niemasz, ktoby nas wybawil z reki ich.
\par 9 Z odwaga duszy naszej szukamy chleba swego dla strachu miecza i na puszczy.
\par 10 Skóra nasza jako piec zczerniala od srogosci glodu.
\par 11 Niewiasty w Syonie pogwalcono; i panny w miastach Judzkich.
\par 12 Ksiazeta reka ich powieszeni sa, a osoby starszych nie maja w uczciwosci.
\par 13 Mlodzieców do zarn biora, a mlodzieniaszkowie po drwami padaja.
\par 14 Starcy w bramach wiecej nie siadaja, a mlodziency przestali piesni swoje.
\par 15 Ustalo wesele serca naszego, plasanie nasze w kwilenie sie obrócilo.
\par 16 Spadla korona z glowy naszej; biada nam, zesmy zgrzeszyli!
\par 17 Dlategoz mdle jest serce nasze, dlatego zacmione sa oczy nasze;
\par 18 Dla góry Syonskiej, ze jest spustoszona, liszki chodza po niej.
\par 19 Ty, Panie! trwasz na wieki, a stolica twoja od narodu do narodu.
\par 20 Przeczze nas na wieki zapominasz, a opuszczasz nas przez tak dlugi czas?
\par 21 Nawróc nas do siebie, o Panie! a nawróceni bedziemy; odnów dni nasze, jako z dawna byly.
\par 22 Bo izali nas cale odrzucisz, a gniewac sie bedziesz na nas tak bardzo?


\end{document}