\begin{document}

\title{Job}

 
\chapter{1}

\par 1 در زمین عوص، مردی بود که ایوب نام داشت و آن مرد کامل و راست و خداترس بود و از بدی اجتناب می‌نمود.
\par 2 و برای او، هفت پسر و سه دختر زاییده شدند.
\par 3 و اموال او هفت هزار گوسفند و سه هزار شتر و پانصد جفت گاو وپانصد الاغ ماده بود و نوکران بسیار کثیر داشت وآن مرد از تمامی بنی مشرق بزرگتر بود.
\par 4 و پسرانش می‌رفتند و در خانه هر یکی ازایشان، در روزش مهمانی می‌کردند و فرستاده، سه خواهر خود را دعوت می‌نمودند تا با ایشان اکل و شرب بنمایند.
\par 5 و واقع می‌شد که چون دوره روزهای مهمانی‌ایشان بسر می‌رفت، ایوب فرستاده، ایشان را تقدیس می‌نمود و بامدادان برخاسته، قربانی های سوختنی، به شماره همه ایشان می‌گذرانید، زیرا ایوب می‌گفت: «شایدپسران من گناه کرده، خدا را در دل خود ترک نموده باشند» و ایوب همیشه چنین می‌کرد.
\par 6 و روزی واقع شد که پسران خدا آمدند تا به حضور خداوند حاضر شوند و شیطان نیز در میان ایشان آمد.
\par 7 و خداوند به شیطان گفت: «از کجاآمدی؟» شیطان در جواب خداوند گفت: «ازتردد کردن در زمین و سیر کردن در آن.»
\par 8 خداوندبه شیطان گفت: «آیا در بنده من ایوب تفکر کردی که مثل او در زمین نیست؟ مرد کامل وراست و خداترس که از گناه اجتناب می‌کند.»
\par 9 شیطان در جواب خداوند گفت: «آیا ایوب مجان از خدا می‌ترسد؟
\par 10 آیا تو گرد او و گردخانه او و گرد همه اموال او، به هر طرف حصارنکشیدی و اعمال دست او را برکت ندادی ومواشی او در زمین منتشر نشد؟
\par 11 لیکن الان دست خود را دراز کن و تمامی مایملک او رالمس نما و پیش روی تو، تو را ترک خواهدنمود.»
\par 12 خداوند به شیطان گفت: «اینک همه اموالش در دست تو است، لیکن دستت را بر خوداو دراز مکن.» پس شیطان از حضور خداوندبیرون رفت.
\par 13 و روزی واقع شد که پسران و دخترانش درخانه برادر بزرگ خود می‌خوردند و شراب می‌نوشیدند.
\par 14 و رسولی نزد ایوب آمده، گفت: «گاوان شیار می‌کردند و ماده الاغان نزد آنهامی چریدند.
\par 15 و سابیان بر آنها حمله آورده، بردند و جوانان را به دم شمشیر کشتند و من به تنهایی رهایی یافتم تا تو را خبر دهم.»
\par 16 و اوهنوز سخن می‌گفت که دیگری آمده، گفت: «آتش خدا از آسمان افتاد و گله و جوانان راسوزانیده، آنها را هلاک ساخت و من به تنهایی رهایی یافتم تا تو را خبر دهم.»
\par 17 و او هنوزسخن می‌گفت که دیگری آمده، گفت: «کلدانیان سه فرقه شدند و بر شتران هجوم آورده، آنها را بردند و جوانان را به دم شمشیر کشتند و من به تنهایی رهایی یافتم تا تو را خبر دهم.»
\par 18 و اوهنوز سخن می‌گفت که دیگری آمده، گفت: «پسران و دخترانت در خانه برادر بزرگ خودمی خوردند و شراب می‌نوشیدند
\par 19 که اینک بادشدیدی از طرف بیابان آمده، چهار گوشه خانه رازد و بر جوانان افتاد که مردند و من به تنهایی رهایی یافتم تا تو را خبر دهم.»
\par 20 آنگاه ایوب برخاسته، جامه خود را درید وسر خود را تراشید و به زمین افتاده، سجده کرد
\par 21 و گفت: «برهنه از رحم مادر خود بیرون آمدم وبرهنه به آنجا خواهم برگشت؛ خداوند داد وخداوند گرفت و نام خداوند متبارک باد.»دراین همه، ایوب گناه نکرد و به خدا جهالت نسبت نداد.
\par 22 دراین همه، ایوب گناه نکرد و به خدا جهالت نسبت نداد.
 
\chapter{2}

\par 1 و روزی واقع شد که پسران خدا آمدند تا به حضور خداوند حاضر شوند، و شیطان نیزدر میان ایشان آمد تا به حضور خداوند حاضرشود.
\par 2 و خداوند به شیطان گفت: «از کجاآمدی؟» شیطان در جواب خداوند گفت: «ازتردد نمودن در جهان و از سیر کردن در آن.»
\par 3 خداوند به شیطان گفت: «آیا در بنده من ایوب تفکر نمودی که مثل او در زمین نیست؟ مرد کامل و راست و خداترس که از بدی اجتناب می‌نمایدو تا الان کاملیت خود را قایم نگاه می‌دارد، هرچند مرا بر آن واداشتی که او را بی‌سبب اذیت رسانم.»
\par 4 شیطان در جواب خداوند گفت: «پوست به عوض پوست، و هر‌چه انسان داردبرای جان خود خواهد داد.
\par 5 لیکن الان دست خود را دراز کرده، استخوان و گوشت او را لمس نما و تو را پیش روی تو ترک خواهد نمود.»
\par 6 خداوند به شیطان گفت: «اینک او در دست تواست، لیکن جان او را حفظ کن.»
\par 7 پس شیطان از حضور خداوند بیرون رفته، ایوب را از کف پا تا کله‌اش به دملهای سخت مبتلا ساخت.
\par 8 و او سفالی گرفت تا خود را با آن بخراشد و در میان خاکستر نشسته بود.
\par 9 و زنش او را گفت: «آیا تا بحال کاملیت خود را نگاه می‌داری؟ خدا را ترک کن و بمیر.»
\par 10 او وی را گفت: «مثل یکی از زنان ابله سخن می‌گویی! آیا نیکویی را از خدا بیابیم و بدی رانیابیم؟» در این همه، ایوب به لبهای خود گناه نکرد.
\par 11 و چون سه دوست ایوب، این همه بدی راکه بر او واقع شده بود شنیدند، هر یکی از مکان خود، یعنی الیفاز تیمانی و بلدد شوحی و سوفرنعماتی روانه شدند و با یکدیگر همداستان گردیدند که آمده، او را تعزیت گویند و تسلی دهند.
\par 12 و چون چشمان خود را از دور بلندکرده، او را نشناختند، آواز خود را بلند نموده، گریستند و هر یک جامه خود را دریده، خاک بسوی آسمان بر سر خود افشاندند.و هفت روز و هفت شب همراه او بر زمین نشستند و کسی با وی سخنی نگفت چونکه دیدند که درد او بسیارعظیم است.
\par 13 و هفت روز و هفت شب همراه او بر زمین نشستند و کسی با وی سخنی نگفت چونکه دیدند که درد او بسیارعظیم است.
 
\chapter{3}

\par 1 و بعد از آن ایوب دهان خود را باز کرده، روز خود را نفرین کرد.
\par 2 و ایوب متکلم شده، گفت:
\par 3 «روزی که در آن متولد شدم، هلاک شود و شبی که گفتند مردی در رحم قرار گرفت،
\par 4 آن روز تاریکی شود. و خدا از بالا بر آن اعتنانکند و روشنایی بر او نتابد.
\par 5 تاریکی و سایه موت، آن را به تصرف آورند. ابر بر آن ساکن شود. کسوفات روز آن را بترساند.
\par 6 و آن شب را ظلمت غلیظ فرو‌گیرد و در میان روزهای سال شادی نکند، و به شماره ماهها داخل نشود.
\par 7 اینک آن شب نازاد باشد. و آواز شادمانی در آن شنیده نشود.
\par 8 لعنت کنندگان روز، آن را نفرین نمایند، که در برانگیزانیدن لویاتان ماهر می‌باشند.
\par 9 ستارگان شفق آن، تاریک گردد و انتظار نوربکشد و نباشد، و مژگان سحر را نبیند،
\par 10 چونکه درهای رحم مادرم را نبست، و مشقت را ازچشمانم مستور نساخت.
\par 11 «چرا از رحم مادرم نمردم؟ و چون از شکم بیرون آمدم چرا جان ندادم؟
\par 12 چرا زانوها مراقبول کردند، و پستانها تا مکیدم؟
\par 13 زیرا تا بحال می‌خوابیدم و آرام می‌شدم. در خواب می‌بودم واستراحت می‌یافتم.
\par 14 با پادشاهان و مشیران جهان، که خرابه‌ها برای خویشتن بنا نمودند،
\par 15 یابا سروران که طلا داشتند، و خانه های خود را ازنقره پر ساختند.
\par 16 یا مثل سقط پنهان شده نیست می‌بودم، مثل بچه هایی که روشنایی را ندیدند.
\par 17 در آنجا شریران از شورش باز می‌ایستند، و درآنجا خستگان می‌آرامند،
\par 18 در آنجا اسیران دراطمینان با هم ساکنند، و آواز کارگذاران رانمی شنوند.
\par 19 کوچک و بزرگ در آنجا یک‌اند. وغلام از آقایش آزاد است.
\par 20 چرا روشنی به مستمند داده شود؟ و زندگی به تلخ جانان؟
\par 21 که انتظار موت را می‌کشند و یافت نمی شود، و برای آن حفره می‌زنند بیشتر از گنجها.
\par 22 که شادی وابتهاج می‌نمایند و مسرور می‌شوند چون قبر رامی یابند؟
\par 23 چرا نور داده می‌شود به کسی‌که راهش مستور است، که خدا اطرافش را مستورساخته است؟
\par 24 زیرا که ناله من، پیش از خوراکم می‌آید و نعره من، مثل آب ریخته می‌شود،
\par 25 زیرا ترسی که از آن می‌ترسیدم، بر من واقع شد. و آنچه از آن بیم داشتم بر من رسید.مطمئن و آرام نبودم و راحت نداشتم وپریشانی بر من آمد.»
\par 26 مطمئن و آرام نبودم و راحت نداشتم وپریشانی بر من آمد.»
 
\chapter{4}

\par 1 و الیفاز تیمانی در جواب گفت:
\par 2 «اگر کسی جرات کرده، با تو سخن گوید، آیا تورا ناپسند می‌آید؟ لیکن کیست که بتواند از سخن‌گفتن بازایستد؟
\par 3 اینک بسیاری را ادب آموخته‌ای و دستهای ضعیف را تقویت داده‌ای.
\par 4 سخنان تو لغزنده را قایم داشت، و تو زانوهای لرزنده را تقویت دادی.
\par 5 لیکن الان به تو رسیده است و ملول شده‌ای، تو را لمس کرده است وپریشان گشته‌ای.
\par 6 آیا توکل تو بر تقوای تونیست؟ و امید تو بر کاملیت رفتار تو نی؟
\par 7 الان فکر کن! کیست که بی‌گناه هلاک شد؟ و راستان درکجا تلف شدند؟
\par 8 چنانکه من دیدم آنانی که شرارت را شیار می‌کنند و شقاوت را می‌کارندهمان را می‌دروند.
\par 9 از نفخه خدا هلاک می‌شوندو از باد غضب او تباه می‌گردند.
\par 10 غرش شیر ونعره سبع و دندان شیربچه‌ها شکسته می‌شود.
\par 11 شیر نر از نابودن شکار هلاک می‌شود وبچه های شیر ماده پراکنده می‌گردند.
\par 12 «سخنی به من در خفا رسید، و گوش من آواز نرمی از آن احساس نمود.
\par 13 در تفکرها ازرویاهای شب، هنگامی که خواب سنگین بر مردم غالب شود،
\par 14 خوف و لرز بر من مستولی شد که جمیع استخوانهایم را به جنبش آورد.
\par 15 آنگاه روحی از پیش روی من گذشت، و مویهای بدنم برخاست.
\par 16 در آنجا ایستاد، اما سیمایش راتشخیص ننمودم. صورتی در‌پیش نظرم بود. خاموشی بود و آوازی شنیدم
\par 17 که آیا انسان به حضور خدا عادل شمرده شود؟ و آیا مرد در نظرخالق خود طاهر باشد؟
\par 18 اینک بر خادمان خوداعتماد ندارد، و به فرشتگان خویش، حماقت نسبت می‌دهد.
\par 19 پس چند مرتبه زیاده به ساکنان خانه های گلین، که اساس ایشان در غبار است، که مثل بید فشرده می‌شوند!
\par 20 از صبح تا شام خردمی شوند، تا به ابد هلاک می‌شوند و کسی آن را به‌خاطر نمی آورد.آیا طناب خیمه ایشان ازایشان کنده نمی شود؟ پس بدون حکمت می‌میرند.
\par 21 آیا طناب خیمه ایشان ازایشان کنده نمی شود؟ پس بدون حکمت می‌میرند.
 
\chapter{5}

\par 1 «الان استغاثه کن و آیا کسی هست که تو راجواب دهد؟ و به کدامیک از مقدسان توجه خواهی نمود؟
\par 2 زیرا غصه، احمق رامی کشد و حسد، ابله را می‌میراند.
\par 3 من احمق رادیدم که ریشه می‌گرفت و ناگهان مسکن او رانفرین کردم.
\par 4 فرزندان او از امنیت دور هستند ودر دروازه پایمال می‌شوند و رهاننده‌ای نیست.
\par 5 که گرسنگان محصول او را می‌خورند، و آن رانیز از میان خارها می‌چینند، و دهان تله برای دولت ایشان باز است.
\par 6 زیرا که بلا از غبار درنمی آید، و مشقت از زمین نمی روید.
\par 7 بلکه انسان برای مشقت مولود می‌شود، چنانکه شراره‌ها بالامی پرد.
\par 8 و لکن من نزد خدا طلب می‌کردم، ودعوی خود را بر خدا می‌سپردم،
\par 9 که اعمال عظیم و بی‌قیاس می‌کند و عجایب بی‌شمار؛
\par 10 که بر روی زمین باران می‌باراند، و آب بر روی صخره‌ها جاری می‌سازد،
\par 11 تا مسکینان را به مقام بلند برساند، و ماتمیان به سلامتی سرافراشته شوند.
\par 12 که فکرهای حیله گران را باطل می‌سازد، به طوری که دستهای ایشان هیچ کار مفیدنمی تواند کرد.
\par 13 که حکیمان را در حیله ایشان گرفتار می‌سازد، و مشورت مکاران مشوش می‌شود.
\par 14 در روز به تاریکی برمی خورند و به وقت ظهر، مثل شب کورانه راه می‌روند.
\par 15 که مسکین را از شمشیر دهان ایشان، و از دست زورآور نجات می‌دهد.
\par 16 پس امید، برای ذلیل پیدا می‌شود و شرارت دهان خود را می‌بندد.
\par 17 «هان، خوشابحال شخصی که خدا تنبیهش می‌کند. پس تادیب قادر مطلق را خوار مشمار.
\par 18 زیرا که او مجروح می‌سازد و التیام می‌دهد، ومی کوبد و دست او شفا می‌دهد.
\par 19 در شش بلا، تو را نجات خواهد داد و در هفت بلا، هیچ ضرر برتو نخواهد رسید.
\par 20 در قحط تو را از موت فدیه خواهد داد، و در جنگ از دم شمشیر.
\par 21 ازتازیانه زبان پنهان خواهی ماند، و چون هلاکت آید، از آن نخواهی ترسید.
\par 22 بر خرابی وتنگسالی خواهی خندید، و از وحوش زمین بیم نخواهی داشت.
\par 23 زیرا با سنگهای صحراهمداستان خواهی بود، و وحوش صحرا با توصلح خواهند کرد. 
\par 24 و خواهی دانست که خیمه تو ایمن است، و مسکن خود را تجسس خواهی کرد و چیزی مفقود نخواهی یافت.
\par 25 وخواهی دانست که ذریتت کثیر است و اولاد تومثل علف زمین.
\par 26 و در شیخوخیت به قبرخواهی رفت، مثل بافه گندم که در موسمش برداشته می‌شود.اینک این را تفتیش نمودیم وچنین است، پس تو این را بشنو و برای خویشتن بدان.»
\par 27 اینک این را تفتیش نمودیم وچنین است، پس تو این را بشنو و برای خویشتن بدان.»
 
\chapter{6}

\par 1 و ایوب جواب داده، گفت:
\par 2 «کاش که غصه من سنجیده شود. و مشقت مرا درمیزان با آن بگذارند.
\par 3 زیرا که الان از ریگ دریاسنگینتر است. از این سبب سخنان من بیهوده می‌باشد.
\par 4 زیرا تیرهای قادرمطلق در اندرون من است. و روح من زهر آنها را می‌آشامد، و ترسهای خدا بر من صف آرایی می‌کند.
\par 5 آیا گورخر باداشتن علف عرعر می‌کند؟ و یا گاو بر آذوقه خودبانگ می‌زند؟
\par 6 آیا چیز بی‌مزه، بی‌نمک خورده می‌شود؟ و یا در سفیده تخم، طعم می‌باشد؟
\par 7 جان من از لمس نمودن آنها کراهت دارد. آنهابرای من مثل خوراک، زشت است.
\par 8 «کاش که مسالت من برآورده شود، و خداآرزوی مرا به من بدهد!
\par 9 و خدا راضی شود که مرا خرد کند، و دست خود را بلند کرده، مرامنقطع سازد!
\par 10 آنگاه معهذا مرا تسلی می‌شد ودر عذاب الیم شاد می‌شدم، چونکه کلمات حضرت قدوس را انکار ننمودم.
\par 11 من چه قوت دارم که انتظار بکشم و عاقبت من چیست که صبرنمایم؟
\par 12 آیا قوت من قوت سنگها است؟ و یاگوشت من برنج است؟
\par 13 آیا بالکل بی‌اعانت نیستم؟ و مساعدت از من مطرود نشده است؟
\par 14 حق شکسته دل از دوستش ترحم است، اگر‌چه هم ترس قادر مطلق را ترک نماید.
\par 15 اما برادران من مثل نهرها مرا فریب دادند، مثل رودخانه وادیها که می‌گذرند.
\par 16 که از یخ سیاه فام می‌باشند، و برف در آنها مخفی است.
\par 17 وقتی که آب از آنها می‌رود، نابود می‌شوند. و چون گرماشود، از جای خود ناپدید می‌گردند.
\par 18 کاروانیان از راه خود منحرف می‌شوند، و در بیابان داخل شده، هلاک می‌گردند.
\par 19 کاروانیان تیما به آنهانگران بودند. قافله های سبا امید آن را داشتند.
\par 20 از امید خود خجل گردیدند. به آنجا رسیدند وشرمنده گشتند.
\par 21 زیرا که الان شما مثل آنهاشده‌اید، مصیبتی دیدید و ترسان گشتید.
\par 22 آیاگفتم که چیزی به من ببخشید؟ یا ارمغانی از اموال خود به من بدهید؟
\par 23 یا مرا از دست دشمن رهاکنید؟ و مرا از دست ظالمان فدیه دهید؟
\par 24 «مرا تعلیم دهید و من خاموش خواهم شد، و مرا بفهمانید که در چه چیز خطا کردم.
\par 25 سخنان راستی چقدر زورآور است! اما تنبیه شما چه نتیجه می‌بخشد؟
\par 26 آیا گمان می‌برید که سخنان را تنبیه می‌نمایید و سخنان مایوس را که مثل باد است؟
\par 27 یقین برای یتیم قرعه می‌اندازیدو دوست خود را مال تجارت می‌شمارید.
\par 28 پس الان التفات کرده، بر من توجه نمایید، و روبه‌روی شما دروغ نخواهم گفت.
\par 29 برگردید و بی‌انصافی نباشد، و باز برگردید زیرا عدالت من قایم است.آیا در زبان من بی‌انصافی می‌باشد؟ و آیا کام من چیزهای فاسد را تمیز نمی دهد؟
\par 30 آیا در زبان من بی‌انصافی می‌باشد؟ و آیا کام من چیزهای فاسد را تمیز نمی دهد؟
 
\chapter{7}

\par 1 «آیا برای انسان بر زمین مجاهده‌ای نیست؟ و روزهای وی مثل روزهای مزدور نی؟
\par 2 مثل غلام که برای سایه اشتیاق دارد، و مزدوری که منتظر مزد خویش است،
\par 3 همچنین ماههای بطالت نصیب من شده است، و شبهای مشقت برای من معین گشته.
\par 4 چون می‌خوابم می‌گویم: کی برخیزم؟ و شب بگذرد و تا سپیده صبح ازپهلو به پهلو گردیدن خسته می‌شوم.
\par 5 جسدم ازکرمها و پاره های خاک ملبس است، و پوستم تراکیده و مقروح می‌شود.
\par 6 روزهایم از ماکوی جولا تیزروتر است، و بدون امید تمام می‌شود.
\par 7 به یاد آور که زندگی من باد است، و چشمانم دیگر نیکویی را نخواهد دید.
\par 8 چشم کسی‌که مرامی بیند دیگر به من نخواهد نگریست، وچشمانت برای من نگاه خواهد کرد و نخواهم بود.
\par 9 مثل ابر که پراکنده شده، نابود می‌شود. همچنین کسی‌که به گور فرو می‌رود، برنمی آید.
\par 10 به خانه خود دیگر نخواهد برگشت، و مکانش باز او را نخواهد شناخت.
\par 11 پس من نیز دهان خود را نخواهم بست. از تنگی روح خود سخن می‌رانم، و از تلخی جانم شکایت خواهم کرد.
\par 12 آیا من دریا هستم یا نهنگم که بر من کشیکچی قرار می‌دهی؟
\par 13 چون گفتم که تخت خوابم مراتسلی خواهد داد و بسترم شکایت مرا رفع خواهد کرد،
\par 14 آنگاه مرا به خوابها ترسان گردانیدی، و به رویاها مرا هراسان ساختی.
\par 15 به حدی که جانم خفه شدن را اختیار کرد و مرگ رابیشتر از این استخوانهایم.
\par 16 کاهیده می‌شوم ونمی خواهم تا به ابد زنده بمانم. مرا ترک کن زیراروزهایم نفسی است.
\par 17 «انسان چیست که او را عزت بخشی، و دل خود را با او مشغول سازی؟
\par 18 و هر بامداد از اوتفقد نمایی و هرلحظه او را بیازمایی؟
\par 19 تا به کی چشم خود را از من برنمی گردانی؟ مرا واگذار تاآب دهان خود را فرو برم.
\par 20 من گناه کردم، اما با تو‌ای پاسبان بنی آدم چه کنم؟ برای چه مرا به جهت خود هدف ساخته‌ای، به حدی که برای خود بار سنگین شده‌ام؟و چرا گناهم رانمی آمرزی، و خطایم را دور نمی سازی؟ زیرا که الان در خاک خواهم خوابید، و مرا تفحص خواهی کرد و نخواهم بود.»
\par 21 و چرا گناهم رانمی آمرزی، و خطایم را دور نمی سازی؟ زیرا که الان در خاک خواهم خوابید، و مرا تفحص خواهی کرد و نخواهم بود.»
 
\chapter{8}

\par 1 پس بلدد شوحی در جواب گفت:
\par 2 «تا به کی این چیزها را خواهی گفت و سخنان دهانت باد شدید خواهد بود؟
\par 3 آیا خداوندداوری را منحرف سازد؟ یا قادر مطلق انصاف رامنحرف نماید؟
\par 4 چون فرزندان تو به او گناه ورزیدند، ایشان را به‌دست عصیان ایشان تسلیم نمود.
\par 5 اگر تو به جد و جهد خدا را طلب می‌کردی و نزد قادر مطلق تضرع می‌نمودی،
\par 6 اگر پاک و راست می‌بودی، البته برای تو بیدارمی شد، و مسکن عدالت تو را برخوردارمی ساخت.
\par 7 و اگر‌چه ابتدایت صغیر می‌بود، عاقبت تو بسیار رفیع می‌گردید.
\par 8 زیرا که ازقرنهای پیشین سوال کن، و به آنچه پدران ایشان تفحص کردند توجه نما،
\par 9 چونکه ما دیروزی هستیم و هیچ نمی دانیم، و روزهای ما سایه‌ای برروی زمین است.
\par 10 آیا ایشان تو را تعلیم ندهند وبا تو سخن نرانند؟ و از دل خود کلمات بیرون نیارند؟
\par 11 آیا نی، بی‌خلاب می‌روید، یا قصب، بی‌آب نمو می‌کند؟
\par 12 هنگامی که هنوز سبزاست و بریده نشده، پیش از هر گیاه خشک می‌شود.
\par 13 همچنین است راه جمیع فراموش کنندگان خدا. و امید ریاکار ضایع می شود،
\par 14 که امید او منقطع می‌شود، واعتمادش خانه عنکبوت است.
\par 15 بر خانه خودتکیه می‌کند و نمی ایستد، به آن متمسک می‌شودو لیکن قایم نمی ماند.
\par 16 پیش روی آفتاب، تر وتازه می‌شود و شاخه هایش در باغش پهن می‌گردد.
\par 17 ریشه هایش بر توده های سنگ درهم بافته می‌شود، و بر سنگلاخ نگاه می‌کند.
\par 18 اگر ازجای خود کنده شود، او را انکار کرده، می‌گوید: تو را نمی بینم.
\par 19 اینک خوشی طریقش همین است و دیگران از خاک خواهند رویید.
\par 20 هماناخدا مرد کامل را حقیر نمی شمارد، و شریر رادستگیری نمی نماید،
\par 21 تا دهان تو را از خنده پرکند، و لبهایت را از آواز شادمانی.خصمان توبه خجالت ملبس خواهند شد، و خیمه شریران نابود خواهد گردید.»
\par 22 خصمان توبه خجالت ملبس خواهند شد، و خیمه شریران نابود خواهد گردید.»
 
\chapter{9}

\par 1 پس ایوب در جواب گفت:
\par 2 «یقین می‌دانم که چنین است. لیکن انسان نزد خدا چگونه عادل شمرده شود؟
\par 3 اگر بخواهد با وی منازعه نماید، یکی از هزار او را جواب نخواهد داد.
\par 4 اودر ذهن حکیم و در قوت تواناست. کیست که با اومقاومت کرده و کامیاب شده باشد؟
\par 5 آنکه کوههارا منتقل می‌سازد و نمی فهمند، و در غضب خویش آنها را واژگون می‌گرداند،
\par 6 که زمین را ازمکانش می‌جنباند، و ستونهایش متزلزل می‌شود.
\par 7 که آفتاب را امر می‌فرماید و طلوع نمی کند وستارگان را مختوم می‌سازد.
\par 8 که به تنهایی، آسمانها را پهن می‌کند و بر موجهای دریامی خرامد.
\par 9 که دب اکبر و جبار و ثریا را آفرید، و برجهای جنوب را
\par 10 که کارهای عظیم بی‌قیاس را می‌کند و کارهای عجیب بی‌شمار را.
\par 11 اینک از من می‌گذرد و او را نمی بینم، و عبور می‌کند واو را احساس نمی نمایم.
\par 12 اینک او می‌رباید وکیست که او را منع نماید؟ و کیست که به او تواندگفت: چه می‌کنی؟
\par 13 خدا خشم خود را بازنمی دارد و مددکاران رحب زیر او خم می‌شوند.
\par 14 «پس به طریق اولی، من کیستم که او راجواب دهم و سخنان خود را بگزینم تا با اومباحثه نمایم؟
\par 15 که اگر عادل می‌بودم، او راجواب نمی دادم، بلکه نزد داور خود استغاثه می‌نمودم.
\par 16 اگر او را می‌خواندم و مرا جواب می‌داد، باور نمی کردم که آواز مرا شنیده است.
\par 17 زیرا که مرا به تندبادی خرد می‌کند و بی‌سبب، زخمهای مرا بسیار می‌سازد.
\par 18 مرا نمی گذارد که نفس بکشم، بلکه مرا به تلخیها پر می‌کند.
\par 19 اگردرباره قوت سخن گوییم، اینک او قادر است؛ واگر درباره انصاف، کیست که وقت را برای من تعیین کند؟
\par 20 اگر عادل می‌بودم دهانم مرا مجرم می‌ساخت، و اگر کامل می‌بودم مرا فاسق می‌شمرد.
\par 21 اگر کامل هستم، خویشتن رانمی شناسم، و جان خود را مکروه می‌دارم.
\par 22 این امر برای همه یکی است. بنابراین می‌گویم که اوصالح است و شریر را هلاک می‌سازد.
\par 23 اگرتازیانه ناگهان بکشد، به امتحان بی‌گناهان استهزامی کند.
\par 24 جهان به‌دست شریران داده شده است و روی حاکمانش را می‌پوشاند. پس اگر چنین نیست، کیست که می‌کند؟
\par 25 و روزهایم از پیک تیزرفتار تندروتر است، می‌گریزد و نیکویی رانمی بیند.
\par 26 مثل کشتیهای تیزرفتار می‌گریزد و مثل عقاب که بر شکار فرود آید.
\par 27 اگر فکر کنم که ناله خود را فراموش کنم و ترش رویی خود رادور کرده، گشاده رو شوم،
\par 28 از تمامی مشقتهای خود می‌ترسم و می‌دانم که مرا بی‌گناه نخواهی شمرد،
\par 29 چونکه ملزم خواهم شد. پس چرا بیجازحمت بکشم؟
\par 30 اگر خویشتن را به آب برف غسل دهم، و دستهای خود را به اشنان پاک کنم،
\par 31 آنگاه مرا در لجن فرو می‌بری، و رختهایم مرامکروه می‌دارد.
\par 32 زیرا که او مثل من انسان نیست که او را جواب بدهم و با هم به محاکمه بیاییم.
\par 33 در میان ما حکمی نیست که بر هر دوی مادست بگذارد.
\par 34 کاش که عصای خود را از من بردارد، و هیبت او مرا نترساند.آنگاه سخن می‌گفتم و از او نمی ترسیدم، لیکن من در خودچنین نیستم.
\par 35 آنگاه سخن می‌گفتم و از او نمی ترسیدم، لیکن من در خودچنین نیستم.
 
\chapter{10}

\par 1 «جانم از حیاتم بیزار است. پس ناله خود را روان می‌سازم و در تلخی جان خود سخن می‌رانم.
\par 2 به خدا می‌گویم مرا ملزم مساز، و مرا بفهمان که از چه سبب با من منازعت می‌کنی؟
\par 3 آیا برای تو نیکو است که ظلم نمایی وعمل دست خود را حقیر شماری، و بر مشورت شریران بتابی؟
\par 4 آیا تو را چشمان بشر است؟ یامثل دیدن انسان می‌بینی؟
\par 5 آیا روزهای تو مثل روزهای انسان است؟ یا سالهای تو مثل روزهای مرد است؟
\par 6 که معصیت مرا تفحص می‌کنی وبرای گناهانم تجسس می‌نمایی؟
\par 7 اگر‌چه می‌دانی که شریر نیستم و از دست تو رهاننده‌ای نیست.
\par 8 «دستهایت مرا جمیع و تمام سرشته است، و مرا آفریده است و آیا مرا هلاک می‌سازی؟
\par 9 به یادآور که مرا مثل سفال ساختی و آیا مرا به غباربرمی گردانی؟
\par 10 آیا مرا مثل شیر نریختی و مثل پنیر، منجمد نساختی؟
\par 11 مرا به پوست و گوشت ملبس نمودی و مرا با استخوانها و پیها بافتی. 
\par 12 حیات و احسان به من عطا فرمودی و لطف توروح مرا محافظت نمود.
\par 13 اما این چیزها را دردل خود پنهان کردی، و می‌دانم که اینها در فکرتو بود.
\par 14 اگر گناه کردم، مرا نشان کردی و مرا ازمعصیتم مبرا نخواهی ساخت.
\par 15 اگر شریر هستم وای بر من! و اگر عادل هستم سر خود رابرنخواهم افراشت، زیرا از اهانت پر هستم ومصیبت خود را می‌بینم!
\par 16 و اگر (سرم )برافراشته شود، مثل شیر مرا شکار خواهی کرد و باز عظمت خود را بر من ظاهر خواهی ساخت.
\par 17 گواهان خود را بر من پی درپی می‌آوری و غضب خویش را بر من می‌افزایی وافواج متعاقب یکدیگر به ضد منند.
\par 18 پس برای چه مرا از رحم بیرون آوردی؟ کاش که جان می‌دادم و چشمی مرا نمی دید.
\par 19 پس می‌بودم، چنانکه نبودم و از رحم مادرم به قبر برده می‌شدم.
\par 20 آیا روزهایم قلیل نیست؟ پس مراترک کن، و از من دست بردار تا اندکی گشاده روشوم،
\par 21 قبل از آنکه بروم به‌جایی که از آن برنخواهم گشت، به زمین ظلمت و سایه موت!به زمین تاریکی غلیظ مثل ظلمات، زمین سایه موت و بی‌ترتیب که روشنایی آن مثل ظلمات است.»
\par 22 به زمین تاریکی غلیظ مثل ظلمات، زمین سایه موت و بی‌ترتیب که روشنایی آن مثل ظلمات است.»
 
\chapter{11}

\par 1 و صوفر نعماتی در جواب گفت:
\par 2 «آیابه کثرت سخنان جواب نباید داد و مردپرگو عادل شمرده شود؟
\par 3 آیا بیهوده‌گویی تومردمان را ساکت کند و یا سخریه کنی و کسی تورا خجل نسازد؟
\par 4 و می‌گویی تعلیم من پاک است، و من در نظر تو بی‌گناه هستم.
\par 5 و لیکن کاش که خدا سخن بگوید و لبهای خود را بر تو بگشاید،
\par 6 و اسرار حکمت را برای تو بیان کند. زیرا که درماهیت خود دو طرف دارد. پس بدان که خدا کمتراز گناهانت تو را سزا داده است.
\par 7 آیا عمق های خدا را می‌توانی دریافت نمود؟ یا به کنه قادرمطلق توانی رسید؟
\par 8 مثل بلندیهای آسمان است؛ چه خواهی کرد؟ گودتر از هاویه است؛ چه توانی دانست؟
\par 9 پیمایش آن از جهان طویل تر واز دریا پهن تر است.
\par 10 اگر سخت بگیرد و حبس نماید و به محاکمه دعوت کند کیست که او راممانعت نماید؟
\par 11 زیرا که بطالت مردم را می‌داندو شرارت را می‌بیند اگرچه در آن تامل نکند.
\par 12 ومرد جاهل آنوقت فهیم می‌شود که بچه خروحشی، انسان متولد شود.
\par 13 اگر تو دل خود راراست سازی و دستهای خود را بسوی او درازکنی،
\par 14 اگر در دست تو شرارت باشد، آن را ازخود دور کن، و بی‌انصافی در خیمه های تو ساکن نشود.
\par 15 پس یقین روی خود را بی‌عیب برخواهی افراشت، و مستحکم شده، نخواهی ترسید.
\par 16 زیرا که مشقت خود را فراموش خواهی کرد، و آن را مثل آب رفته به یاد خواهی آورد،
\par 17 و روزگار تو از وقت ظهر روشن ترخواهد شد، و اگرچه تاریکی باشد، مثل صبح خواهد گشت.
\par 18 و مطمئن خواهی بود چونکه امید داری، و اطراف خود را تجسس نموده، ایمن خواهی خوابید.
\par 19 و خواهی خوابید وترساننده‌ای نخواهد بود، و بسیاری تو راتملق خواهند نمود.لیکن چشمان شریران کاهیده می‌شود و ملجای ایشان از ایشان نابود می‌گردد و امید ایشان جان کندن ایشان است.»
\par 20 لیکن چشمان شریران کاهیده می‌شود و ملجای ایشان از ایشان نابود می‌گردد و امید ایشان جان کندن ایشان است.»
 
\chapter{12}

\par 1 پس ایوب در جواب گفت:
\par 2 «به درستی که شما قوم هستید، و حکمت با شما خواهد مرد.
\par 3 لیکن مرا نیز مثل شما فهم هست، و از شما کمتر نیستم. و کیست که مثل این چیزها را نمی داند؟
\par 4 برای رفیق خود مسخره گردیده‌ام. کسی‌که خدا را خوانده است و او رامستجاب فرموده، مرد عادل و کامل، مسخره شده است.
\par 5 در افکار آسودگان، برای مصیبت اهانت است. مهیا شده برای هرکه پایش بلغزد.
\par 6 خیمه های دزدان به سلامت است و آنانی که خدا را غضبناک می‌سازند ایمن هستند، که خدای خود را در دست خود می‌آورند.
\par 7 «لیکن الان از بهایم بپرس و تو را تعلیم خواهند داد. و از مرغان هوا و برایت بیان خواهندنمود.
\par 8 یا به زمین سخن بران و تو را تعلیم خواهدداد، و ماهیان دریا به تو خبر خواهند رسانید.
\par 9 کیست که از جمیع این چیزها نمی فهمد که دست خداوند آنها را به‌جا آورده است،
\par 10 که جان جمیع زندگان در دست وی است، و روح جمیع افراد بشر؟
\par 11 آیا گوش سخنان رانمی آزماید، چنانکه کام خوراک خود رامی چشد؟
\par 12 نزد پیران حکمت است، و عمردراز فطانت می‌باشد.
\par 13 لیکن حکمت و کبریایی نزد وی است. مشورت و فطانت از آن او است.
\par 14 اینک او منهدم می‌سازد و نمی توان بنا نمود. انسان را می‌بندد و نمی توان گشود.
\par 15 اینک آبهارا باز می‌دارد و خشک می‌شود، و آنها را رهامی کند و زمین را واژگون می‌سازد.
\par 16 قوت ووجود نزد وی است. فریبنده و فریب خورده از آن او است.
\par 17 مشیران را غارت زده می‌رباید، وحاکمان را احمق می‌گرداند.
\par 18 بند پادشاهان را می گشاید و در کمر ایشان کمربند می‌بندد.
\par 19 کاهنان را غارت زده می‌رباید، و زورآوران راسرنگون می‌سازد.
\par 20 بلاغت معتمدین را نابودمی گرداند، و فهم پیران را برمی دارد.
\par 21 اهانت رابر نجیبان می‌ریزد و کمربند مقتدران را سست می‌گرداند.
\par 22 چیزهای عمیق را از تاریکی منکشف می‌سازد، و سایه موت را به روشنایی بیرون می‌آورد.
\par 23 امت‌ها را ترقی می‌دهد و آنهارا هلاک می‌سازد، امت‌ها را وسعت می‌دهد وآنها را جلای وطن می‌فرماید.
\par 24 عقل روسای قوم های زمین را می‌رباید، و ایشان را در بیابان آواره می‌گرداند، جایی که راه نیست.درتاریکی کورانه راه می‌روند و نور نیست. و ایشان را مثل مستان افتان و خیزان می‌گرداند.
\par 25 درتاریکی کورانه راه می‌روند و نور نیست. و ایشان را مثل مستان افتان و خیزان می‌گرداند.
 
\chapter{13}

\par 1 «اینک چشم من همه این چیزها رادیده، و گوش من آنها را شنیده وفهمیده است.
\par 2 چنانکه شما می‌دانید من هم می‌دانم. و من کمتر از شما نیستم.
\par 3 لیکن می‌خواهم با قادر مطلق سخن گویم. و آرزو دارم که با خدا محاجه نمایم.
\par 4 اما شما دروغها جعل می‌کنید، و جمیع شما طبیبان باطل هستید.
\par 5 کاش که شما به کلی ساکت می‌شدید که این برای شما حکمت می‌بود.
\par 6 پس حجت مرابشنوید. و دعوی لبهایم را گوش گیرید.
\par 7 آیابرای خدا به بی‌انصافی سخن خواهید راند؟ و به جهت او با فریب تکلم خواهید نمود؟
\par 8 آیا برای او طرف داری خواهید نمود؟ و به جهت خدادعوی خواهید کرد؟
\par 9 آیا نیکو است که او شما راتفتیش نماید؟ یا چنانکه انسان را مسخره می‌نمایند او را مسخره می‌سازید.
\par 10 البته شما را توبیخ خواهد کرد. اگر در خفا طرف داری نمایید.
\par 11 آیا جلال او شما را هراسان نخواهد ساخت؟ و هیبت او بر شما مستولی نخواهد شد؟
\par 12 ذکرهای شما، مثل های غبار است. وحصارهای شما، حصارهای گل است.
\par 13 «از من ساکت شوید و من سخن خواهم گفت. و هرچه خواهد، بر من واقع شود.
\par 14 چراگوشت خود را با دندانم بگیرم و جان خود را دردستم بنهم؟
\par 15 اگرچه مرا بکشد، برای او انتظارخواهم کشید. لیکن راه خود را به حضور او ثابت خواهم ساخت.
\par 16 این نیز برای من نجات خواهدشد. زیرا ریاکار به حضور او حاضر نمی شود.
\par 17 بشنوید! سخنان مرا بشنوید. و دعوی من به گوشهای شما برسد.
\par 18 اینک الان دعوی خود رامرتب ساختم. و می‌دانم که عادل شمرده خواهم شد.
\par 19 کیست که بامن مخاصمه کند؟ پس خاموش شده جان را تسلیم خواهم کرد.
\par 20 فقطدو چیز به من مکن. آنگاه خود را از حضور توپنهان نخواهم ساخت.
\par 21 دست خود را از من دورکن. و هیبت تو مرا هراسان نسازد.
\par 22 آنگاه بخوان و من جواب خواهم داد، یا اینکه من بگویم و مراجواب بده.
\par 23 خطایا و گناهانم چقدر است؟ تقصیر و گناه مرا به من بشناسان.
\par 24 چرا روی خود را از من می‌پوشانی؟ و مرا دشمن خودمی شماری؟
\par 25 آیا برگی را که از باد رانده شده است می‌گریزانی؟ و کاه خشک را تعاقب می‌کنی؟
\par 26 زیرا که چیزهای تلخ را به ضد من می‌نویسی، و گناهان جوانی‌ام را نصیب من می‌سازی.
\par 27 و پایهای مرا در کنده می‌گذاری، وجمیع راههایم را نشان می‌کنی و گرد کف پاهایم خط می‌کشی؛و حال آنکه مثل چیز گندیده فاسد، و مثل جامه بید خورده هستم.
\par 28 و حال آنکه مثل چیز گندیده فاسد، و مثل جامه بید خورده هستم.
 
\chapter{14}

\par 1 قلیل الایام و پر از زحمات است.
\par 2 مثل گل می‌روید و بریده می‌شود. و مثل سایه می‌گریزد و نمی ماند.
\par 3 و آیا بر چنین شخص چشمان خود را می‌گشایی و مرا با خود به محاکمه می‌آوری؟
\par 4 کیست که چیز طاهر را ازچیز نجس بیرون آورد؟ هیچکس نیست.
\par 5 چونکه روزهایش مقدر است و شماره ماههایش نزد توست و حدی از برایش گذاشته‌ای که از آن تجاوز نتواند نمود.
\par 6 از او رو بگردان تاآرام گیرد. و مثل مزدور روزهای خود را به انجام رساند.
\par 7 «زیرا برای درخت امیدی است که اگر بریده شود باز خواهد رویید، و رمونهایش نابودنخواهد شد.
\par 8 اگر‌چه ریشه‌اش در زمین کهنه شود، و تنه آن در خاک بمیرد.
\par 9 لیکن از بوی آب، رمونه می‌کند و مثل نهال نو، شاخه‌ها می‌آورد.
\par 10 اما مرد می‌میرد و فاسد می‌شود و آدمی چون جان را سپارد کجا است؟
\par 11 چنانکه آبها از دریازایل می‌شود، و نهرها ضایع و خشک می‌گردد.
\par 12 همچنین انسان می‌خوابد و برنمی خیزد، تانیست شدن آسمانها بیدار نخواهند شد و ازخواب خود برانگیخته نخواهند گردید.
\par 13 کاش که مرا در هاویه پنهان کنی؛ و تا غضبت فرو نشیند، مرا مستور سازی؛ و برایم زمانی تعیین نمایی تا مرا به یاد آوری.
\par 14 اگر مرد بمیرد باردیگر زنده شود؛ در تمامی روزهای مجاهده خودانتظار خواهم کشید، تا وقت تبدیل من برسد.
\par 15 تو ندا خواهی کرد و من جواب خواهم داد، و به صنعت دست خود مشتاق خواهی شد.
\par 16 اما الان قدمهای مرا می‌شماری و آیا برگناه من پاسبانی نمی کنی؟
\par 17 معصیت من در کیسه مختوم است. و خطای مرا مسدود ساخته‌ای.
\par 18 به درستی کوهی که می‌افتد فانی می‌شود وصخره از مکانش منتقل می‌گردد.
\par 19 آب سنگهارا می‌ساید، و سیلهایش خاک زمین را می‌برد. همچنین امید انسان را تلف می‌کنی،
\par 20 بر او تا به ابد غلبه می‌کنی، پس می‌رود. روی او را تغییرمی دهی و او را رها می‌کنی.
\par 21 پسرانش به عزت می‌رسند و او نمی داند. یا به ذلت می‌افتند و ایشان را به نظر نمی آورد.برای خودش فقط جسد اواز درد بی‌تاب می‌شود. و برای خودش جان اوماتم می‌گیرد.»
\par 22 برای خودش فقط جسد اواز درد بی‌تاب می‌شود. و برای خودش جان اوماتم می‌گیرد.»
 
\chapter{15}

\par 1 پس الیفاز تیمانی در جواب گفت:
\par 2 «آیامرد حکیم از علم باطل جواب دهد؟ وبطن خود را از باد شرقی پر سازد؟
\par 3 آیا به سخن بی‌فایده محاجه نماید؟ و به کلماتی که هیچ نفع نمی بخشد؟
\par 4 اما تو خداترسی را ترک می‌کنی وتقوا را به حضور خدا ناقص می‌سازی.
\par 5 زیرا که دهانت، معصیت تو را ظاهر می‌سازد و زبان حیله گران را اختیار می‌کنی.
\par 6 دهان خودت تو راملزم می‌سازد و نه من، و لبهایت بر تو شهادت می‌دهد.
\par 7 آیا شخص اول از آدمیان زاییده شده‌ای؟ و پیش از تلها به وجود آمده‌ای؟
\par 8 آیامشورت مخفی خدا را شنیده‌ای و حکمت را برخود منحصر ساخته‌ای؟
\par 9 چه می‌دانی که ما هم نمی دانیم؟ و چه می‌فهمی که نزد ما هم نیست؟
\par 10 نزد ما ریش سفیدان و پیران هستند که درروزها از پدر تو بزرگترند.
\par 11 آیا تسلی های خدابرای تو کم است و کلام ملایم با تو؟
\par 12 چرا دلت تو را می‌رباید؟ و چرا چشمانت را بر هم می‌زنی
\par 13 که روح خود را به ضد خدا بر می‌گردانی، وچنین سخنان را از دهانت بیرون می‌آوری؟ 
\par 14 «انسان چیست که پاک باشد، و مولود زن که عادل شمرده شود؟
\par 15 اینک بر مقدسان خوداعتماد ندارد، و آسمانها در نظرش پاک نیست.
\par 16 پس از طریق اولی انسان مکروه و فاسد که شرارت را مثل آب می‌نوشد.
\par 17 من برای تو بیان می‌کنم پس مرا بشنو. و آنچه دیده‌ام حکایت می‌نمایم.
\par 18 که حکیمان آن را از پدران خودروایت کردند و مخفی نداشتند،
\par 19 که به ایشان به تنهایی زمین داده شد، و هیچ غریبی از میان ایشان عبور نکرد،
\par 20 شریر در تمامی روزهایش مبتلای درد است. و سالهای شمرده شده برای مرد ظالم مهیا است.
\par 21 صدای ترسها در گوش وی است. در وقت سلامتی تاراج کننده بر وی می‌آید.
\par 22 باور نمی کند که از تاریکی خواهد برگشت وشمشیر برای او مراقب است.
\par 23 برای نان می‌گردد و می‌گوید کجاست. و می‌داند که روزتاریکی نزد او حاضر است.
\par 24 تنگی و ضیق او رامی ترساند، مثل پادشاه مهیای جنگ بر او غلبه می‌نماید.
\par 25 زیرا دست خود را به ضد خدا درازمی کند و بر قادر مطلق تکبر می‌نماید.
\par 26 با گردن بلند بر او تاخت می‌آورد، با گل میخهای سخت سپر خویش،
\par 27 چونکه روی خود را به پیه پوشانیده، و کمر خود را با شحم ملبس ساخته است.
\par 28 و در شهرهای ویران و خانه های غیرمسکون که نزدیک به خراب شدن است ساکن می‌شود.
\par 29 او غنی نخواهد شد و دولتش پایدارنخواهد ماند، و املاک او در زمین زیاد نخواهد گردید.
\par 30 از تاریکی رها نخواهد شد، و آتش، شاخه هایش را خواهد خشکانید، و به نفخه دهان او زائل خواهد شد.
\par 31 به بطالت توکل ننماید وخود را فریب ندهد، والا بطالت اجرت او خواهدبود.
\par 32 قبل از رسیدن وقتش تمام ادا خواهد شدو شاخه او سبز نخواهد ماند.
\par 33 مثل مو، غوره خود را خواهد افشاند، و مثل زیتون، شکوفه خود را خواهد ریخت،
\par 34 زیرا که جماعت ریاکاران، بی‌کس خواهند ماند، و خیمه های رشوه خواران را آتش خواهد سوزانید.به شقاوت حامله شده، معصیت را می‌زایند و شکم ایشان فریب را آماده می‌کند.»
\par 35 به شقاوت حامله شده، معصیت را می‌زایند و شکم ایشان فریب را آماده می‌کند.»
 
\chapter{16}

\par 1 پس ایوب در جواب گفت:
\par 2 «بسیارچیزها مثل این شنیدم. تسلی دهندگان مزاحم همه شما هستید.
\par 3 آیا سخنان باطل راانتها نخواهد شد؟ و کیست که تو را به جواب دادن تحریک می‌کند؟
\par 4 من نیز مثل شمامی توانستم بگویم، اگر جان شما در جای جان من می‌بود، و سخنها به ضد شما ترتیب دهم، و سرخود را بر شما بجنبانم،
\par 5 لیکن شما را به دهان خود تقویت می‌دادم و تسلی لبهایم غم شما رارفع می‌نمود.
\par 6 «اگر من سخن گویم، غم من رفع نمی گردد؛ واگر ساکت شوم مرا چه راحت حاصل می‌شود؟
\par 7 لیکن الان او مرا خسته نموده است. تو تمامی جماعت مرا ویران ساخته‌ای.
\par 8 مرا سخت گرفتی و این بر من شاهد شده است. و لاغری من به ضدمن برخاسته، روبرویم شهادت می‌دهد.
\par 9 در غضب خود مرا دریده و بر من جفا نموده است. دندانهایش را بر من افشرده و مثل دشمنم چشمان خود را بر من تیز کرده است.
\par 10 دهان خود را بر من گشوده‌اند، بر رخسار من به استحقارزده‌اند، به ضد من با هم اجتماع نموده‌اند.
\par 11 خدامرا به‌دست ظالمان تسلیم نموده، و مرا به‌دست شریران افکنده است.
\par 12 چون در راحت بودم مراپاره پاره کرده است، و گردن مرا گرفته، مرا خردکرده، و مرا برای هدف خود نصب نموده است.
\par 13 تیرهایش مرا احاطه کرد. گرده هایم را پاره می‌کند و شفقت نمی نماید. و زهره مرا به زمین می‌ریزد.
\par 14 مرا زخم بر زخم، مجروح می‌سازد ومثل جبار، بر من حمله می‌آورد.
\par 15 بر پوست خود پلاس دوخته‌ام، و شاخ خود را در خاک خوار نموده‌ام.
\par 16 روی من از گریستن سرخ شده است، و بر مژگانم سایه موت است.
\par 17 اگر‌چه هیچ بی‌انصافی در دست من نیست، و دعای من پاک است.
\par 18 ‌ای زمین خون مرا مپوشان، واستغاثه مرا آرام نباشد.
\par 19 اینک الان نیز شاهد من در آسمان است، و گواه من در اعلی علیین.
\par 20 دوستانم مرا استهزا می‌کنند، لیکن چشمانم نزد خدا اشک می‌ریزد.
\par 21 و آیا برای انسان نزدخدا محاجه می‌کند، مثل بنی آدم که برای همسایه خود می‌نماید؟زیرا سالهای اندک سپری می‌شود، پس به راهی که برنمی گردم، خواهم رفت.
\par 22 زیرا سالهای اندک سپری می‌شود، پس به راهی که برنمی گردم، خواهم رفت.
 
\chapter{17}

\par 1 «روح من تلف شده، و روزهایم تمام گردیده، و قبر برای من حاضر است.
\par 2 به درستی که استهزاکنندگان نزد منند، و چشم من در منازعت ایشان دائم می‌ماند.
\par 3 الان گرو بده و به جهت من نزد خود ضامن باش. والا کیست که به من دست دهد؟
\par 4 چونکه دل ایشان را از حکمت منع کرده‌ای، بنابراین ایشان را بلند نخواهی ساخت.
\par 5 کسی‌که دوستان خود را به تاراج تسلیم کند، چشمان فرزندانش تار خواهد شد.
\par 6 مرا نزدامت‌ها مثل ساخته است، و مثل کسی‌که بر رویش آب دهان اندازند شده‌ام.
\par 7 چشم من از غصه کاهیده شده است، و تمامی اعضایم مثل سایه گردیده.
\par 8 راستان به‌سبب این، حیران می‌مانند وصالحان خویشتن را بر ریاکاران برمی انگیزانند.
\par 9 لیکن مرد عادل به طریق خود متمسک می‌شود، و کسی‌که دست پاک دارد، در قوت ترقی خواهدنمود.
\par 10 «اما همه شما برگشته، الان بیایید و در میان شما حکیمی نخواهم یافت.
\par 11 روزهای من گذشته، و قصدهای من و فکرهای دلم منقطع شده است.
\par 12 شب را به روز تبدیل می‌کنند و باوجود تاریکی می‌گویند روشنایی نزدیک است.
\par 13 وقتی که امید دارم هاویه خانه من می‌باشد، وبستر خود را در تاریکی می‌گسترانم،
\par 14 و به هلاکت می‌گویم تو پدر من هستی و به کرم که تومادر و خواهر من می‌باشی.
\par 15 پس امید من کجااست؟ و کیست که امید مرا خواهد دید؟تابندهای هاویه فرو می‌رود، هنگامی که با هم درخاک نزول (نماییم ).»
\par 16 تابندهای هاویه فرو می‌رود، هنگامی که با هم درخاک نزول (نماییم ).»
 
\chapter{18}

\par 1 پس بلدد شوحی در جواب گفت:
\par 2 «تابه کی برای سخنان، دامها می‌گسترانید؟ تفکر کنید و بعد از آن تکلم خواهیم نمود.
\par 3 چرا مثل بهایم شمرده شویم؟ و در نظر شما نجس نماییم؟
\par 4 ‌ای که در غضب خود خویشتن را پاره می‌کنی، آیا به‌خاطر تو زمین متروک شود، یاصخره از جای خود منتقل گردد؟
\par 5 البته روشنایی شریران خاموش خواهد شد، و شعله آتش ایشان نور نخواهد داد.
\par 6 در خیمه اوروشنایی به تاریکی مبدل می‌گردد، و چراغش براو خاموش خواهد شد.
\par 7 قدمهای قوتش تنگ می‌شود. و مشورت خودش او را به زیر خواهدافکند.
\par 8 زیرا به پایهای خود در دام خواهد افتاد، و به روی تله‌ها راه خواهد رفت.
\par 9 تله پاشنه او راخواهد گرفت. و دام، او را به زور نگاه خواهدداشت.
\par 10 دام برایش در زمین پنهان شده است، وتله برایش در راه.
\par 11 ترسها از هر طرف او راهراسان می‌کند، و به او چسبیده، وی رامی گریزاند.
\par 12 شقاوت، برای او گرسنه است، وذلت، برای لغزیدن او حاضر است.
\par 13 اعضای جسد او را می‌خورد. نخست زاده موت، جسد اورا می‌خورد.
\par 14 آنچه بر آن اعتماد می‌داشت، ازخیمه او ربوده می‌شود، و خود او نزد پادشاه ترسها رانده می‌گردد.
\par 15 کسانی که از وی نباشنددر خیمه او ساکن می‌گردند، و گوگرد بر مسکن اوپاشیده می‌شود.
\par 16 ریشه هایش از زیرمی خشکد، و شاخه‌اش از بالا بریده خواهد شد.
\par 17 یادگار او از زمین نابود می‌گردد، و در کوچه هااسم نخواهد داشت.
\par 18 از روشنایی به تاریکی رانده می‌شود. و او را از ربع مسکون خواهندگریزانید.
\par 19 او را در میان قومش نه اولاد و نه ذریت خواهد بود، و در ماوای او کسی باقی نخواهد ماند.
\par 20 متاخرین از روزگارش متحیر خواهند شد، چنانکه بر متقدمین، ترس مستولی شده بود.به درستی که مسکن های شریران چنین می‌باشد، و مکان کسی‌که خدا رانمی شناسد مثل این است.»
\par 21 به درستی که مسکن های شریران چنین می‌باشد، و مکان کسی‌که خدا رانمی شناسد مثل این است.»
 
\chapter{19}

\par 1 پس ایوب در جواب گفت:
\par 2 «تا به کی جان مرا می‌رنجانید؟ و مرا به سخنان خود فرسوده می‌سازید؟
\par 3 این ده مرتبه است که مرا مذمت نمودید، و خجالت نمی کشید که با من سختی می‌کنید؟
\par 4 و اگر فی الحقیقه خطا کرده‌ام، خطای من نزد من می‌ماند.
\par 5 اگر فی الواقع بر من تکبر نمایید و عار مرا بر من اثبات کنید،
\par 6 پس بدانید که خدا دعوی مرا منحرف ساخته، و به دام خود مرا احاطه نموده است.
\par 7 اینک از ظلم، تضرع می‌نمایم و مستجاب نمی شوم و استغاثه می‌کنم و دادرسی نیست.
\par 8 طریق مرا حصارنموده است که از آن نمی توانم گذشت و برراههای من تاریکی را گذارده است.
\par 9 جلال مرا ازمن کنده است و تاج را از سر من برداشته،
\par 10 مرا ازهر طرف خراب نموده، پس هلاک شدم. و مثل درخت، ریشه امید مرا کنده است.
\par 11 غضب خودرا بر من افروخته، و مرا یکی از دشمنان خودشمرده است.
\par 12 فوجهای او با هم می‌آیند و راه خود را بر من بلند می‌کنند و به اطراف خیمه من اردو می‌زنند.
\par 13 «برادرانم را از نزد من دور کرده است وآشنایانم از من بالکل بیگانه شده‌اند.
\par 14 خویشانم مرا ترک نموده و آشنایانم مرا فراموش کرده‌اند.
\par 15 نزیلان خانه‌ام و کنیزانم مرا غریب می‌شمارند، و در نظر ایشان بیگانه شده‌ام.
\par 16 غلام خود راصدا می‌کنم و مرا جواب نمی دهد، اگر‌چه او را به دهان خود التماس بکنم.
\par 17 نفس من نزد زنم مکروه شده است و تضرع من نزد اولاد رحم مادرم.
\par 18 بچه های کوچک نیز مرا حقیرمی شمارند و چون برمی خیزم به ضد من حرف می‌زنند.
\par 19 همه اهل مشورتم از من نفرت می‌نمایند، و کسانی را که دوست می‌داشتم از من برگشته‌اند.
\par 20 استخوانم به پوست و گوشتم چسبیده است، و با پوست دندانهای خودخلاصی یافته‌ام.
\par 21 بر من ترحم کنید! ترحم کنیدشما‌ای دوستانم! زیرا دست خدا مرا لمس نموده است.
\par 22 چرا مثل خدا بر من جفا می‌کنید وازگوشت من سیر نمی شوید.
\par 23 کاش که سخنانم الان نوشته می‌شد! کاش که در کتابی ثبت می‌گردید،
\par 24 و با قلم آهنین و سرب بر صخره‌ای تا به ابد کنده می‌شد!
\par 25 و من می‌دانم که ولی من زنده است، و در ایام آخر، بر زمین خواهدبرخاست.
\par 26 و بعد از آنکه این پوست من تلف شود، بدون جسدم نیز خدا را خواهم دید.
\par 27 ومن او را برای خود خواهم دید. و چشمان من بر اوخواهد نگریست و نه چشم دیگری. اگر‌چه گرده هایم در اندرونم تلف شده باشد.
\par 28 اگربگویید چگونه بر او جفا نماییم و حال آنگاه اصل امر در من یافت می‌شود.پس از شمشیربترسید، زیرا که سزاهای شمشیر غضبناک است، تا دانسته باشید که داوری خواهد بود.»
\par 29 پس از شمشیربترسید، زیرا که سزاهای شمشیر غضبناک است، تا دانسته باشید که داوری خواهد بود.»
 
\chapter{20}

\par 1 پس صوفر نعماتی در جواب گفت:
\par 2 «از این جهت فکرهایم مرا به جواب دادن تحریک می‌کند و به این سبب، من تعجیل می نمایم.
\par 3 سرزنش توبیخ خود را شنیدم، و ازفطانتم روح من مرا جواب می‌دهد.
\par 4 آیا این را ازقدیم ندانسته‌ای، از زمانی که انسان بر زمین قرارداده شد،
\par 5 که شادی شریران، اندک زمانی است، و خوشی ریاکاران، لحظه‌ای؟
\par 6 اگر‌چه شوکت اوتا به آسمان بلند شود، و سر خود را تا به فلک برافرازد.
\par 7 لیکن مثل فضله خود تا به ابد هلاک خواهد شد، و بینندگانش خواهند گفت: کجااست؟
\par 8 مثل خواب، می‌پرد و یافت نمی شود. ومثل رویای شب، او را خواهند گریزانید.
\par 9 چشمی که او را دیده است دیگر نخواهد دید، ومکانش باز بر او نخواهد نگریست.
\par 10 فرزندانش نزد فقیران تذلل خواهند کرد، و دستهایش دولت او را پس خواهد داد. 
\par 11 استخوانهایش از جوانی پر است، لیکن همراه او در خاک خواهد خوابید.
\par 12 اگر‌چه شرارت در دهانش شیرین باشد، و آن را زیر زبانش پنهان کند.
\par 13 اگر‌چه او را دریغ داردو از دست ندهد، و آن را در میان کام خود نگاه دارد.
\par 14 لیکن خوراک او در احشایش تبدیل می‌شود، و در اندرونش زهرمار می‌گردد.
\par 15 دولت را فرو برده است و آن را قی خواهد کرد، و خدا آن را از شکمش بیرون خواهد نمود.
\par 16 اوزهر مارها را خواهد مکید، و زبان افعی او راخواهد کشت.
\par 17 بر رودخانه‌ها نظر نخواهندکرد، بر نهرها و جویهای شهد و شیر.
\par 18 ثمره زحمت خود را رد کرده، آن را فرو نخواهد برد، وبرحسب دولتی که کسب کرده است، شادی نخواهد نمود.
\par 19 زیرا فقیران را زبون ساخته وترک کرده است. پس خانه‌ای را که دزدیده است، بنا نخواهد کرد.
\par 20 «زیرا که در حرص خود قناعت را ندانست. پس از نفایس خود، چیزی استرداد نخواهد کرد.
\par 21 چیزی نمانده است که نخورده باشد. پس برخورداری او دوام نخواهد داشت.
\par 22 هنگامی که دولت او بی‌نهایت گردد، در تنگی گرفتار می‌شود، و دست همه ذلیلان بر او استیلاخواهد یافت.
\par 23 در وقتی که شکم خود را پرمی کند، خدا حدت خشم خود را بر او خواهدفرستاد، و حینی که می‌خورد آن را بر او خواهدبارانید.
\par 24 از اسلحه آهنین خواهد گریخت وکمان برنجین، او را خواهد سفت.
\par 25 آن رامی کشد و از جسدش بیرون می‌آید، و پیکان براق از زهره‌اش درمی رود و ترسها بر او استیلامی یابد.
\par 26 تمامی تاریکی برای ذخایر او نگاه داشته شده است. و آتش ندمیده آنها را خواهدسوزانید، و آنچه را که در چادرش باقی است، خواهد خورد.
\par 27 آسمانها عصیانش را مکشوف خواهد ساخت، و زمین به ضد او خواهدبرخاست.
\par 28 محصول خانه‌اش زایل خواهد شد، و در روز غضب او نابود خواهد گشت.این است نصیب مرد شریر از خدا و میراث مقدر او ازقادر مطلق.»
\par 29 این است نصیب مرد شریر از خدا و میراث مقدر او ازقادر مطلق.»
 
\chapter{21}

\par 1 پس ایوب در جواب گفت:
\par 2 «بشنوید، کلام مرا بشنوید. و این، تسلی شماباشد.
\par 3 با من تحمل نمایید تا بگویم، و بعد ازگفتنم استهزا نمایید.
\par 4 و اما من، آیا شکایتم نزدانسان است؟ پس چرا بی‌صبر نباشم؟
\par 5 به من توجه کنید و تعجب نمایید، و دست به دهان بگذارید.
\par 6 هرگاه به یاد می‌آورم، حیران می‌شوم. و لرزه جسد مرا می‌گیرد.
\par 7 چرا شریران زنده می‌مانند، پیر می‌شوند و در توانایی قوی می‌گردند؟
\par 8 ذریت ایشان به حضور ایشان، با ایشان استوار می‌شوند و اولاد ایشان در نظرایشان.
\par 9 خانه های ایشان، از ترس ایمن می‌باشد وعصای خدا بر ایشان نمی آید.
\par 10 گاو نر ایشان جماع می‌کند و خطا نمی کند و گاو ایشان می‌زایدو سقط نمی نماید.
\par 11 بچه های خود را مثل گله بیرون می‌فرستند و اطفال ایشان رقص می‌کنند.
\par 12 با دف وعود می‌سرایند، و با صدای نای شادی می‌نمایند.
\par 13 روزهای خود را در سعادتمندی صرف می‌کنند، و به لحظه‌ای به هاویه فرودمی روند.
\par 14 و به خدا می‌گویند: از ما دور شو زیراکه معرفت طریق تو را نمی خواهیم.
\par 15 قادرمطلق کیست که او را عبادت نماییم، و ما را چه فایده که از او استدعا نماییم.
\par 16 اینک سعادتمندی ایشان در دست ایشان نیست. کاش که مشورت شریران از من دور باشد.
\par 17 «بسا چراغ شریران خاموش می‌شود وذلت ایشان به ایشان می‌رسد، و خدا در غضب خود دردها را نصیب ایشان می‌کند.
\par 18 مثل سفال پیش روی باد می‌شوند و مثل کاه که گردبادپراکنده می‌کند.
\par 19 خدا گناهش را برای فرزندانش ذخیره می‌کند، و او را مکافات می‌رساند و خواهد دانست.
\par 20 چشمانش هلاکت او را خواهد دید، و از خشم قادر مطلق خواهدنوشید.
\par 21 زیرا که بعد از او در خانه‌اش او را چه شادی خواهد بود، چون عدد ماههایش منقطع شود؟
\par 22 آیا خدا را علم توان آموخت؟ چونکه اوبر اعلی علیین داوری می‌کند.
\par 23 یکی در عین قوت خود می‌میرد، در حالی که بالکل در امنیت وسلامتی است.
\par 24 قدحهای او پر از شیر است، ومغز استخوانش تر و تازه است.
\par 25 و دیگری درتلخی جان می‌میرد و از نیکویی هیچ لذت نمی برد.
\par 26 اینها باهم در خاک می‌خوابند و کرمها ایشان را می‌پوشانند.
\par 27 اینک افکار شما رامی دانم و تدبیراتی که ناحق بر من می‌اندیشید.
\par 28 زیرا می‌گویید کجاست خانه امیر، و خیمه های مسکن شریران؟
\par 29 آیا از راه گذریان نپرسیدید؟ ودلایل ایشان را انکار نمی توانید نمود،
\par 30 که شریران برای روز ذلت نگاه داشته می‌شوند و درروز غضب، بیرون برده می‌گردند.
\par 31 کیست که راهش را پیش رویش بیان کند، و جزای آنچه راکه کرده است به او برساند؟
\par 32 که آخر او را به قبرخواهند برد، و بر مزار او نگاهبانی خواهند کرد.
\par 33 کلوخهای وادی برایش شیرین می‌شود وجمیع آدمیان در عقب او خواهند رفت، چنانکه قبل از او بیشماره رفته‌اند.پس چگونه مراتسلی باطل می‌دهید که در جوابهای شما محض خیانت می‌ماند!»
\par 34 پس چگونه مراتسلی باطل می‌دهید که در جوابهای شما محض خیانت می‌ماند!»
 
\chapter{22}

\par 1 پس الیفاز تیمانی در جواب گفت:
\par 2 «آیا مرد به خدا فایده برساند؟ البته مرد دانا برای خویشتن مفید است.
\par 3 آیا اگر توعادل باشی، برای قادر مطلق خوشی رخ می‌نماید؟ یا اگر طریق خود را راست سازی، او رافایده می‌شود؟
\par 4 آیا به‌سبب ترس تو، تو را توبیخ می‌نماید؟ یا با تو به محاکمه داخل خواهد شد؟
\par 5 آیا شرارت تو عظیم نیست و عصیان تو بی‌انتهانی،
\par 6 چونکه از برادران خود بی‌سبب گرو گرفتی و لباس برهنگان را کندی،
\par 7 به تشنگان آب ننوشانیدی، و از گرسنگان نان دریغ داشتی؟
\par 8 امامرد جبار، زمین از آن او می‌باشد و مرد عالیجاه، در آن ساکن می‌شود.
\par 9 بیوه‌زنان را تهی‌دست رد نمودی، و بازوهای یتیمان شکسته گردید.
\par 10 بنابراین دامها تو را احاطه می‌نماید و ترس، ناگهان تو را مضطرب می‌سازد.
\par 11 یا تاریکی که آن را نمی بینی و سیلابها تو را می‌پوشاند.
\par 12 آیاخدا مثل آسمانها بلند نیست؟ و سر ستارگان رابنگر چگونه عالی هستند.
\par 13 و تو می‌گویی خداچه می‌داند و آیا از تاریکی غلیظ داوری تواندنمود؟
\par 14 ابرها ستر اوست پس نمی بیند، و بردایره افلاک می‌خرامد.
\par 15 آیا طریق قدما را نشان کردی که مردمان شریر در آن سلوک نمودند،
\par 16 که قبل از زمان خود ربوده شدند، و اساس آنهامثل نهر ریخته شد
\par 17 که به خدا گفتند: از ما دورشو و قادر مطلق برای ما چه تواند کرد؟
\par 18 و حال آنگاه او خانه های ایشان را از چیزهای نیکو پرساخت. پس مشورت شریران از من دور شود.
\par 19 «عادلان چون آن را بینند، شادی خواهندنمود و بی‌گناهان بر ایشان استهزا خواهند کرد.
\par 20 آیا مقاومت کنندگان ما منقطع نشدند؟ و آتش بقیه ایشان را نسوزانید؟
\par 21 پس حال با او انس بگیر و سالم باش. و به این منوال نیکویی به توخواهد رسید.
\par 22 تعلیم را از دهانش قبول نما، وکلمات او را در دل خود بنه.
\par 23 اگر به قادرمطلق بازگشت نمایی، بنا خواهی شد. و اگر شرارت رااز خیمه خود دور نمایی
\par 24 و اگر گنج خود را درخاک و طلای اوفیر را در سنگهای نهرهابگذاری،
\par 25 آنگاه قادر مطلق گنج تو و نقره خالص برای تو خواهد بود،
\par 26 زیرا در آنوقت ازقادر مطلق تلذذ خواهی یافت، و روی خود را به طرف خدا برخواهی افراشت.
\par 27 نزد او دعاخواهی کرد و او تو را اجابت خواهد نمود، و نذرهای خود را ادا خواهی ساخت.
\par 28 امری راجزم خواهی نمود و برایت برقرار خواهد شد، وروشنایی بر راههایت خواهد تابید.
\par 29 وقتی که ذلیل شوند، خواهی گفت: رفعت باشد، و فروتنان را نجات خواهد داد.کسی را که بی‌گناه نباشدخواهد رهانید، و به پاکی دستهای تو رهانیده خواهد شد.»
\par 30 کسی را که بی‌گناه نباشدخواهد رهانید، و به پاکی دستهای تو رهانیده خواهد شد.»
 
\chapter{23}

\par 1 پس ایوب در جواب گفت:
\par 2 «امروز نیز شکایت من تلخ است، وضرب من از ناله من سنگینتر.
\par 3 کاش می‌دانستم که او را کجا یابم، تا آنکه نزد کرسی او بیایم.
\par 4 آنگاه دعوی خود را به حضور وی ترتیب می‌دادم، ودهان خود را از حجت‌ها پر می‌ساختم.
\par 5 سخنانی را که در جواب من می‌گفت می‌دانستم، و آنچه راکه به من می‌گفت می‌فهمیدم.
\par 6 آیا به عظمت قوت خود با من مخاصمه می‌نمود؟ حاشا! بلکه به من التفات می‌کرد.
\par 7 آنگاه مرد راست با اومحاجه می‌نمود و از داور خود تا به ابد نجات می‌یافتم.
\par 8 اینک به طرف مشرق می‌روم و اویافت نمی شود و به طرف مغرب و او را نمی بینم.
\par 9 به طرف شمال جایی که او عمل می‌کند، و او رامشاهده نمی کنم. و او خود را به طرف جنوب می‌پوشاند و او را نمی بینم،
\par 10 زیرا او طریقی راکه می‌روم می‌داند و چون مرا می‌آزماید، مثل طلابیرون می‌آیم.
\par 11 پایم اثر اقدام او را گرفته است وطریق او را نگاه داشته، از آن تجاوز نمی کنم.
\par 12 از فرمان لبهای وی برنگشتم و سخنان دهان اورا زیاده از رزق خود ذخیره کردم.
\par 13 لیکن اوواحد است و کیست که او را برگرداند؟ و آنچه دل او می‌خواهد، به عمل می‌آورد.
\par 14 زیرا آنچه را که بر من مقدر شده است بجا می‌آورد، و چیزهای بسیار مثل این نزد وی است.
\par 15 از این جهت ازحضور او هراسان هستم، و چون تفکر می‌نمایم از او می‌ترسم،
\par 16 زیرا خدا دل مرا ضعیف کرده است، و قادرمطلق مرا هراسان گردانیده.چونکه پیش از تاریکی منقطع نشدم، و ظلمت غلیظ را از نزد من نپوشانید.
\par 17 چونکه پیش از تاریکی منقطع نشدم، و ظلمت غلیظ را از نزد من نپوشانید.
 
\chapter{24}

\par 1 «چونکه زمانها از قادرمطلق مخفی نیست. پس چرا عارفان او ایام او راملاحظه نمی کنند؟
\par 2 بعضی هستند که حدود رامنتقل می‌سازند و گله‌ها را غصب نموده، می‌چرانند.
\par 3 الاغهای یتیمان را می‌رانند و گاوبیوه‌زنان را به گرو می‌گیرند.
\par 4 فقیران را از راه منحرف می‌سازند، و مسکینان زمین جمیع خویشتن را پنهان می‌کنند.
\par 5 اینک مثل خروحشی به جهت کار خود به بیابان بیرون رفته، خوراک خود را می‌جویند و صحرا به ایشان نان برای فرزندان ایشان می‌رساند.
\par 6 علوفه خود را درصحرا درو می‌کنند و تاکستان شریران راخوشه چینی می‌نمایند.
\par 7 برهنه و بی‌لباس شب رابه‌سر می‌برند و در سرما پوششی ندارند.
\par 8 ازباران کوهها تر می‌شوند و از عدم پناهگاه، صخره‌ها را در بغل می‌گیرند
\par 9 و کسانی هستند که یتیم را از پستان می‌ربایند و از فقیر گرو می‌گیرند.
\par 10 پس ایشان بی‌لباس و برهنه راه می‌روند وبافه‌ها را برمی دارند و گرسنه می‌مانند.
\par 11 دردروازه های آنها روغن می‌گیرند و چرخشت آنهارا پایمال می‌کنند و تشنه می‌مانند.
\par 12 ازشهرآباد، نعره می‌زنند و جان مظلومان استغاثه می‌کند. اما خدا حماقت آنها را به نظر نمی آورد.
\par 13 «و دیگرانند که از نور متمردند و راه آن رانمی دانند. و در طریق هایش سلوک نمی نمایند.
\par 14 قاتل در صبح برمی خیزد و فقیر و مسکین رامی کشد. و در شب مثل دزد می‌شود.
\par 15 چشم زناکار نیز برای شام انتظار می‌کشد و می‌گوید که چشمی مرا نخواهد دید، و بر روی خود پرده می‌کشد.
\par 16 در تاریکی به خانه‌ها نقب می‌زنند ودر روز، خویشتن را پنهان می‌کنند و روشنایی رانمی دانند،
\par 17 زیرا صبح برای جمیع ایشان مثل سایه موت است، چونکه ترسهای سایه موت رامی دانند.
\par 18 آنها بر روی آبها سبک‌اند و نصیب ایشان بر زمین ملعون است، و به راه تاکستان مراجعت نمی کنند.
\par 19 چنانکه خشکی و گرمی آب برف را نابود می‌سازد، همچنین هاویه خطاکاران را.
\par 20 رحم (مادرش ) او را فراموش می‌نماید و کرم، او را نوش می‌کند. و دیگر مذکورنخواهد شد، و شرارت مثل درخت بریده خواهدشد. 
\par 21 زن عاقر را که نمی زاید می‌بلعد و به زن بیوه احسان نمی نماید،
\par 22 و اما خدا جباران را به قوت خود محفوظ می‌دارد. برمی خیزند اگرچه امید زندگی ندارند،
\par 23 ایشان را اطمینان می‌بخشد و بر آن تکیه می‌نمایند، اما چشمان اوبر راههای ایشان است.
\par 24 اندک زمانی بلندمی شوند، پس نیست می‌گردند و پست شده، مثل سایرین برده می‌شوند و مثل سر سنبله‌ها بریده می‌گردند،و اگر چنین نیست پس کیست که مراتکذیب نماید و کلام مرا ناچیز گرداند؟»
\par 25 و اگر چنین نیست پس کیست که مراتکذیب نماید و کلام مرا ناچیز گرداند؟»
 
\chapter{25}

\par 1 پس بلدد شوحی در جواب گفت:
\par 2 «سلطنت و هیبت از آن اوست وسلامتی را در مکان های بلند خود ایجاد می‌کند.
\par 3 آیا افواج او شمرده می‌شود و کیست که نور اوبر وی طلوع نمی کند؟
\par 4 پس انسان چگونه نزد خدا عادل شمرده شود؟ و کسی‌که از زن زاییده شود، چگونه پاک باشد؟
\par 5 اینک ماه نیز روشنایی ندارد و ستارگان در نظر او پاک نیستند.پس چندمرتبه زیاده انسان که مثل خزنده زمین و بنی آدم که مثل کرم می‌باشد.»
\par 6 پس چندمرتبه زیاده انسان که مثل خزنده زمین و بنی آدم که مثل کرم می‌باشد.»
 
\chapter{26}

\par 1 پس ایوب در جواب گفت:
\par 2 «شخص بی‌قوت را چگونه اعانت کردی؟ و بازوی ناتوان را چگونه نجات دادی؟
\par 3 شخص بی‌حکمت را چه نصیحت نمودی؟ وحقیقت امر را به فراوانی اعلام کردی!
\par 4 برای که سخنان را بیان کردی؟ و نفخه کیست که از توصادر شد؟
\par 5 ارواح مردگان می‌لرزند، زیر آبها وساکنان آنها.
\par 6 هاویه به حضور او عریان است، وابدون را ستری نیست.
\par 7 شمال را بر جو پهن می‌کند، و زمین را بر نیستی آویزان می‌سازد.
\par 8 آبها را در ابرهای خود می‌بندد، پس ابر، زیرآنها چاک نمی شود.
\par 9 روی تخت خود رامحجوب می‌سازد و ابرهای خویش را پیش آن می‌گستراند.
\par 10 به اطراف سطح آبها حد می‌گذاردتا کران روشنایی و تاریکی.
\par 11 ستونهای آسمان متزلزل می‌شود و از عتاب او حیران می‌ماند.
\par 12 به قوت خود دریا را به تلاطم می‌آورد، و به فهم خویش رهب را خرد می‌کند.
\par 13 به روح اوآسمانها زینت داده شد، و دست او مار تیز رو راسفت.اینک اینها حواشی طریق های او است. و چه آواز آهسته‌ای درباره او می‌شنویم، لکن رعد جبروت او را کیست که بفهمد؟»
\par 14 اینک اینها حواشی طریق های او است. و چه آواز آهسته‌ای درباره او می‌شنویم، لکن رعد جبروت او را کیست که بفهمد؟»
 
\chapter{27}

\par 1 و ایوب دیگرباره مثل خود را آورده، گفت:
\par 2 «به حیات خدا که حق مرابرداشته و به قادرمطلق که جان مرا تلخ نموده است.
\par 3 که مادامی که جانم در من باقی است ونفخه خدا در بینی من می‌باشد،
\par 4 یقین لبهایم به بی‌انصافی تکلم نخواهد کرد، و زبانم به فریب تنطق نخواهد نمود.
\par 5 حاشا از من که شما راتصدیق نمایم، و تا بمیرم کاملیت خویش را ازخود دور نخواهم ساخت.
\par 6 عدالت خود را قایم نگاه می‌دارم و آن را ترک نخواهم نمود، و دلم تازنده باشم، مرا مذمت نخواهد کرد.
\par 7 دشمن من مثل شریر باشد، و مقاومت کنندگانم مثل خطاکاران.
\par 8 زیرا امید شریر چیست هنگامی که خدا او را منقطع می‌سازد؟ و حینی که خدا جان اورا می‌گیرد؟
\par 9 آیا خدا فریاد او را خواهد شنید، هنگامی که مصیبت بر او عارض شود؟
\par 10 آیا درقادرمطلق تلذذ خواهد یافت، و در همه اوقات ازخدا مسالت خواهد نمود؟
\par 11 «شما را درباره دست خدا تعلیم خواهد دادو از اعمال قادرمطلق چیزی مخفی نخواهم داشت.
\par 12 اینک جمیع شما این را ملاحظه کرده‌اید، پس چرا بالکل باطل شده‌اید.
\par 13 این است نصیب مرد شریر از جانب خدا، و میراث ظالمان که آن را از قادرمطلق می‌یابند.
\par 14 اگرفرزندانش بسیار شوند شمشیر برای ایشان است، و ذریت او از نان سیر نخواهند شد.
\par 15 بازماندگان او از وبا دفن خواهند شد، و بیوه‌زنانش گریه نخواهند کرد.
\par 16 اگر‌چه نقره را مثل غبار اندوخته کند، و لباس را مثل گل آماده سازد.
\par 17 او آماده می‌کند لیکن مرد عادل آن را خواهد پوشید، وصالحان نقره او را تقسیم خواهند نمود.
\par 18 خانه خود را مثل بید بنا می‌کند، و مثل سایبانی که دشتبان می‌سازد
\par 19 او دولتمند می‌خوابد اما دفن نخواهد شد. چشمان خود را می‌گشاید و نیست می‌باشد.
\par 20 ترسها مثل آب او را فرو می‌گیرد، و گردباد او را در شب می‌رباید.
\par 21 باد شرقی او رابرمی دارد و نابود می‌شود و او را از مکانش دورمی اندازد،
\par 22 زیرا (خدا) بر او تیر خواهدانداخت و شفقت نخواهد نمود. اگر‌چه اومی خواهد از دست وی فرار کرده، بگریزد.مردم کفهای خود را بر او بهم می‌زنند و او را ازمکانش صفیر زده، بیرون می‌کنند.
\par 23 مردم کفهای خود را بر او بهم می‌زنند و او را ازمکانش صفیر زده، بیرون می‌کنند.
 
\chapter{28}

\par 1 «یقین برای نقره معدنی است، و به جهت طلا جایی است که آن را قال می‌گذارند.
\par 2 آهن از خاک گرفته می‌شود و مس ازسنگ گداخته می‌گردد.
\par 3 مردم برای تاریکی حدمی گذارند و تا نهایت تمام تفحص می‌نمایند، تابه سنگهای ظلمت غلیظ و سایه موت.
\par 4 کانی دور از ساکنان زمین می‌کنند، از راه گذریان فراموش می‌شوند و دور از مردمان آویخته شده، به هر طرف متحرک می‌گردند.
\par 5 از زمین نان بیرون می‌آید، و ژرفیهایش مثل آتش سرنگون می‌شود.
\par 6 سنگهایش مکان یاقوت کبود است. وشمشهای طلا دارد.
\par 7 آن راه را هیچ مرغ شکاری نمی داند، و چشم شاهین آن را ندیده است،
\par 8 وجانوران درنده بر آن قدم نزده‌اند، و شیر غران برآن گذر نکرده.
\par 9 دست خود را به سنگ خارا درازمی کنند، و کوهها را از بیخ برمی کنند.
\par 10 نهرها ازصخره‌ها می‌کنند و چشم ایشان هر چیز نفیس رامی بیند.
\par 11 نهرها را از تراوش می‌بندند وچیزهای پنهان شده را به روشنایی بیرون می‌آورند.
\par 12 اما حکمت از کجا پیدا می‌شود؟ وجای فطانت کجا است؟
\par 13 انسان قیمت آن رانمی داند و در زمین زندگان پیدا نمی شود.
\par 14 لجه می‌گوید که در من نیست، و دریا می‌گوید که نزدمن نمی باشد.
\par 15 زر خالص به عوضش داده نمی شود و نقره برای قیمتش سنجیده نمی گردد.
\par 16 به زر خالص اوفیر آن را قیمت نتوان کرد، و نه به جزع گرانبها و یاقوت کبود.
\par 17 با طلا و آبگینه آن را برابر نتوان کرد، و زیورهای طلای خالص بدل آن نمی شود.
\par 18 مرجان و بلور مذکورنمی شود و قیمت حکمت از لعل گرانتر است.
\par 19 زبرجد حبش با آن مساوی نمی شود و به زرخالص سنجیده نمی گردد.
\par 20 پس حکمت ازکجا می‌آید؟ و مکان فطانت کجا است؟
\par 21 ازچشم تمامی زندگان پنهان است، و از مرغان هوامخفی می‌باشد.
\par 22 ابدون و موت می‌گویند که آوازه آن را به گوش خود شنیده‌ایم.
\par 23 خدا راه آن را درک می‌کند و او مکانش را می‌داند.
\par 24 زیراکه او تا کرانه های زمین می‌نگرد و آنچه را که زیرتمامی آسمان است می‌بیند.
\par 25 تا وزن از برای بادقرار دهد، و آبها را به میزان بپیماید.
\par 26 هنگامی که قانونی برای باران قرار داد، و راهی برای سهام رعد،
\par 27 آنگاه آن را دید و آن را بیان کرد. آن رامهیا ساخت و هم تفتیشش نمود.و به انسان گفت: اینک ترس خداوند حکمت است، و ازبدی اجتناب نمودن، فطانت می‌باشد.»
\par 28 و به انسان گفت: اینک ترس خداوند حکمت است، و ازبدی اجتناب نمودن، فطانت می‌باشد.»
 
\chapter{29}

\par 1 و ایوب باز مثل خود را آورده، گفت:
\par 2 «کاش که من مثل ماههای پیش می‌بودم و مثل روزهایی که خدا مرا در آنها نگاه می‌داشت.
\par 3 هنگامی که چراغ او بر سر من می‌تابید، و با نور او به تاریکی راه می‌رفتم.
\par 4 چنانکه در روزهای کامرانی خود می‌بودم، هنگامی که سر خدا بر خیمه من می‌ماند.
\par 5 وقتی که قادر مطلق هنوز با من می‌بود، و فرزندانم به اطراف من می‌بودند.
\par 6 حینی که قدمهای خود را باکره می‌شستم و صخره، نهرهای روغن را برای من می ریخت.
\par 7 چون به دروازه شهر بیرون می‌رفتم وکرسی خود را در چهار سوق حاضر می‌ساختم.
\par 8 جوانان مرا دیده، خود را مخفی می‌ساختند، وپیران برخاسته، می‌ایستادند.
\par 9 سروران از سخن‌گفتن بازمی ایستادند، و دست به دهان خودمی گذاشتند.
\par 10 آواز شریفان ساکت می‌شد وزبان به کام ایشان می‌چسبید.
\par 11 زیرا گوشی که مرا می‌شنید، مرا خوشحال می‌خواند و چشمی که مرا می‌دید، برایم شهادت می‌داد.
\par 12 زیرافقیری که استغاثه می‌کرد او را می‌رهانیدم، ویتیمی که نیز معاون نداشت.
\par 13 برکت شخصی که در هلاکت بود، به من می‌رسید و دل بیوه‌زن راخوش می‌ساختم.
\par 14 عدالت را پوشیدم و مراملبس ساخت، و انصاف من مثل ردا و تاج بود.
\par 15 من به جهت کوران چشم بودم، و به جهت لنگان پای.
\par 16 برای مسکینان پدر بودم، و دعوایی را که نمی دانستم، تفحص می‌کردم.
\par 17 دندانهای آسیای شریر را می‌شکستم و شکار را ازدندانهایش می‌ربودم.
\par 18 «و می‌گفتم، در آشیانه خود جان خواهم سپرد و ایام خویش را مثل عنقا طویل خواهم ساخت.
\par 19 ریشه من به سوی آبها کشیده خواهدگشت، و شبنم بر شاخه هایم ساکن خواهد شد.
\par 20 جلال من در من تازه خواهد شد، و کمانم دردستم نو خواهد ماند.
\par 21 مرا می‌شنیدند و انتظارمی کشیدند، و برای مشورت من ساکت می‌ماندند.
\par 22 بعد از کلام من دیگر سخن نمی گفتند و قول من بر ایشان فرو می‌چکید.
\par 23 و برای من مثل باران انتظار می‌کشیدند و دهان خویش را مثل باران آخرین باز می‌کردند.
\par 24 اگر بر ایشان می‌خندیدم باور نمی کردند، و نور چهره مرا تاریک نمی ساختند.راه را برای ایشان اختیار کرده، به ریاست می‌نشستم، و در میان لشکر، مثل پادشاه ساکن می‌بودم، و مثل کسی‌که نوحه‌گران را تسلی می‌بخشد.
\par 25 راه را برای ایشان اختیار کرده، به ریاست می‌نشستم، و در میان لشکر، مثل پادشاه ساکن می‌بودم، و مثل کسی‌که نوحه‌گران را تسلی می‌بخشد.
 
\chapter{30}

\par 1 «و اما الان کسانی که از من خردسالترندبر من استهزا می‌کنند، که کراهت می‌داشتم از اینکه پدران ایشان را با سگان گله خود بگذارم.
\par 2 قوت دستهای ایشان نیز برای من چه فایده داشت؟ کسانی که توانایی ایشان ضایع شده بود،
\par 3 از احتیاج و قحطی بی‌تاب شده، زمین خشک را در ظلمت خرابی و ویرانی می‌خاییدند.
\par 4 خبازی را در میان بوته‌ها می‌چیدند، و ریشه شورگیاه نان ایشان بود.
\par 5 از میان (مردمان ) رانده می‌شدند. از عقب ایشان مثل دزدان، هیاهومی کردند.
\par 6 در گریوه های وادیها ساکن می‌شدند. در حفره های زمین و در صخره‌ها.
\par 7 در میان بوته‌ها عرعر می‌کردند، زیر خارها با هم جمع می‌شدند.
\par 8 ابنای احمقان و ابنای مردم بی‌نام، بیرون از زمین رانده می‌گردیدند.
\par 9 و اما الان سرود ایشان شده‌ام و از برای ایشان ضرب‌المثل گردیده‌ام.
\par 10 مرا مکروه داشته، از من دورمی شوند، و از آب دهان بر رویم انداختن، بازنمی ایستند.
\par 11 چونکه زه را بر من باز کرده، مرامبتلا ساخت. پس لگام را پیش رویم رها کردند.
\par 12 از طرف راست من انبوه عوام الناس برخاسته، پاهایم را از پیش در می‌برند، و راههای هلاکت خویش را بر من مهیا می‌سازند.
\par 13 راه مرا خراب کرده، به اذیتم اقدام می‌نمایند، و خود معاونی ندارند.
\par 14 گویا از ثلمه های وسیع می‌آیند، و ازمیان خرابه‌ها بر من هجوم می‌آورند.
\par 15 ترسها برمن برگشته، آبروی مرا مثل باد تعاقب می‌کنند، و فیروزی من مثل ابر می‌گذرد.
\par 16 و الان جانم بر من ریخته شده است، و روزهای مصیبت، مرا گرفتارنموده است.
\par 17 شبانگاه استخوانهایم در اندرون من سفته می‌شود، و پیهایم آرام ندارد.
\par 18 ازشدت سختی لباسم متغیر شده است، و مرا مثل گریبان پیراهنم تنگ می‌گیرد. 
\par 19 مرا در گل انداخته است، که مثل خاک و خاکستر گردیده‌ام.
\par 20 «نزد تو تضرع می‌نمایم و مرا مستجاب نمی کنی، و برمی خیزم و بر من نظر نمی اندازی.
\par 21 خویشتن را متبدل ساخته، بر من بیرحم شده‌ای، با قوت دست خود به من جفا می‌نمایی.
\par 22 مرا به باد برداشته، برآن سوار گردانیدی، و مرادر تندباد پراکنده ساختی.
\par 23 زیرا می‌دانم که مرابه موت باز خواهی گردانید، و به خانه‌ای که برای همه زندگان معین است.
\par 24 یقین بر توده ویران دست خود را دراز نخواهد کرد، و چون کسی دربلا گرفتار شود، آیا به این سبب استغاثه نمی کند؟
\par 25 آیا برای هر مستمندی گریه نمی کردم، و دلم به جهت مسکین رنجیده نمی شد.
\par 26 لکن چون امید نیکویی داشتم بدی آمد، و چون انتظار نورکشیدم ظلمت رسید.
\par 27 احشایم می‌جوشد وآرام نمی گیرد، و روزهای مصیبت مرا درگرفته است.
\par 28 ماتم‌کنان بی‌آفتاب گردش می‌کنم و درجماعت برخاسته، تضرع می‌نمایم.
\par 29 برادرشغالان شده‌ام، و رفیق شترمرغ گردیده‌ام.
\par 30 پوست من سیاه گشته، از من می‌ریزد، واستخوانهایم از حرارت سوخته گردیده است.بربط من به نوحه گری مبدل شده و نای من به آواز گریه کنندگان.
\par 31 بربط من به نوحه گری مبدل شده و نای من به آواز گریه کنندگان.
 
\chapter{31}

\par 1 «با چشمان خود عهد بسته‌ام، پس چگونه بر دوشیزه‌ای نظر افکنم؟
\par 2 زیراقسمت خدا از اعلی چیست؟ و نصیب قادرمطلق، از اعلی علیین؟
\par 3 آیا آن برای شریران هلاکت نیست؟ و به جهت عاملان بدی مصیبت نی؟
\par 4 آیا او راههای مرا نمی بیند؟ و جمیع قدمهایم را نمی شمارد؟
\par 5 اگر با دروغ راه می‌رفتم یا پایهایم با فریب می‌شتابید،
\par 6 مرا به میزان عدالت بسنجد، تا خدا کاملیت مرا بداند.
\par 7 اگرقدمهایم از طریق آواره گردیده، و قلبم در‌پی چشمانم رفته، و لکه‌ای به‌دستهایم چسبیده باشد،
\par 8 پس من کشت کنم و دیگری بخورد، ومحصول من از ریشه‌کنده شود.
\par 9 اگر قلبم به زنی فریفته شده، یا نزد در همسایه خود در کمین نشسته باشم،
\par 10 پس زن من برای شخصی دیگرآسیا کند، و دیگران بر او خم شوند.
\par 11 زیرا که آن قباحت می‌بود و تقصیری سزاوار حکم داوران.
\par 12 چونکه این آتشی می‌بود که تا ابدون می‌سوزانید، و تمامی محصول مرا از ریشه می‌کند.
\par 13 اگر دعوی بنده و کنیز خود را ردمی کردم، هنگامی که بر من مدعی می‌شدند.
\par 14 پس چون خدا به ضد من برخیزد چه خواهم کرد؟ و هنگامی که تفتیش نماید به او چه جواب خواهم داد؟
\par 15 آیا آن کس که مرا در رحم آفریداو را نیز نیافرید؟ و آیا کس واحد، ما را در رحم نسرشت؟
\par 16 «اگر مراد مسکینان را از ایشان منع نموده باشم، و چشمان بیوه‌زنان را تار گردانیده،
\par 17 اگرلقمه خود را به تنهایی خورده باشم، و یتیم از آن تناول ننموده،
\par 18 و حال آنکه او از جوانیم با من مثل پدر پرورش می‌یافت، و از بطن مادرم بیوه‌زن را رهبری می‌نمودم.
\par 19 اگر کسی را از برهنگی هلاک دیده باشم، و مسکین را بدون پوشش،
\par 20 اگر کمرهای او مرا برکت نداده باشد، و از پشم گوسفندان من گرم نشده،
\par 21 اگر دست خود را بر یتیم بلند کرده باشم، هنگامی که اعانت خود را دردروازه می‌دیدم،
\par 22 پس بازوی من از کتفم بیفتد، و ساعدم از قلم آن شکسته شود.
\par 23 زیرا که هلاکت از خدا برای من ترس می‌بود و به‌سبب کبریایی او توانایی نداشتم.
\par 24 اگر طلا را امیدخود می‌ساختم و به زر خالص می‌گفتم تو اعتمادمن هستی،
\par 25 اگر از فراوانی دولت خویش شادی می‌نمودم، و از اینکه دست من بسیار کسب نموده بود،
\par 26 اگر چون آفتاب می‌تابید بر آن نظرمی کردم و بر ماه، هنگامی که با درخشندگی سیرمی کرد.
\par 27 و دل من خفیه فریفته می‌شد و دهانم دستم را می‌بوسید.
\par 28 این نیز گناهی مستوجب قصاص می‌بود زیرا خدای متعال را منکرمی شدم.
\par 29 اگر از مصیبت دشمن خود شادی می‌کردم یا حینی که بلا به او عارض می‌شد وجدمی نمودم،
\par 30 و حال آنکه زبان خود را از گناه ورزیدن بازداشته، بر جان او لعنت را سوال ننمودم.
\par 31 اگر اهل خیمه من نمی گفتند: کیست که از گوشت او سیر نشده باشد،
\par 32 غریب درکوچه شب را به‌سر نمی برد و در خود را به روی مسافر می‌گشودم.
\par 33 اگر مثل آدم، تقصیر خود رامی پوشانیدم و عصیان خویش را در سینه خودمخفی می‌ساختم،
\par 34 از این جهت که از انبوه کثیرمی ترسیدم و اهانت قبایل مرا هراسان می‌ساخت، پس ساکت مانده، از در خود بیرون نمی رفتم.
\par 35 کاش کسی بود که مرا می‌شنید، اینک امضای من حاضر است. پس قادر مطلق مرا جواب دهد. و اینک کتابتی که مدعی من نوشته است.
\par 36 یقین که آن را بر دوش خود برمی داشتم و مثل تاج برخود می‌بستم.
\par 37 شماره قدمهای خود را برای اوبیان می‌کردم و مثل امیری به او تقرب می‌جستم.
\par 38 اگر زمین من بر من فریاد می‌کرد و مرزهایش با هم گریه می‌کردند،
\par 39 اگر محصولاتش را بدون قیمت می‌خوردم و جان مالکانش را تلف می‌نمودم،پس خارها به عوض گندم وکرکاس به عوض جو بروید.»سخنان ایوب تمام شد.
\par 40 پس خارها به عوض گندم وکرکاس به عوض جو بروید.»سخنان ایوب تمام شد.
 
\chapter{32}

\par 1 پس آن سه مرد از جواب دادن به ایوب باز ماندند چونکه او در نظر خود عادل بود.
\par 2 آنگاه خشم الیهو ابن برکئیل بوزی که ازقبیله رام بود مشتعل شد، و غضبش بر ایوب افروخته گردید، از این جهت که خویشتن را ازخدا عادلتر می‌نمود.
\par 3 و خشمش بر سه رفیق خود افروخته گردید، از این جهت که هر‌چندجواب نمی یافتند، اما ایوب را مجرم می‌شمردند.
\par 4 و الیهو از سخن‌گفتن با ایوب درنگ نموده بود زیرا که ایشان در عمر، از وی بزرگتربودند.
\par 5 اما چون الیهو دید که به زبان آن سه مردجوابی نیست، پس خشمش افروخته شد.
\par 6 و الیهو ابن برکئیل بوزی به سخن آمده، گفت: «من در عمر صغیر هستم، و شما موسفید. بنابراین ترسیده، جرات نکردم که رای خود را برای شمابیان کنم.
\par 7 و گفتم روزها سخن گوید، و کثرت سالها، حکمت را اعلام نماید.
\par 8 لیکن در انسان روحی هست، و نفخه قادرمطلق، ایشان را فطانت می‌بخشد.
\par 9 بزرگان نیستند که حکمت دارند، و نه پیران که انصاف را می‌فهمند.
\par 10 بنابراین می‌گویم که مرا بشنو. و من نیز رای خود را بیان خواهم نمود.
\par 11 اینک از سخن‌گفتن با شما درنگ نمودم، و براهین شما را گوش گرفتم. تا سخنان را کاوش گردید.
\par 12 و من در شما تامل نمودم و اینک کسی از شما نبود که ایوب را ملزم سازد. یا سخنان او را جواب دهد.
\par 13 مبادا بگویید که حکمت رادریافت نموده‌ایم، خدا او را مغلوب می‌سازد و نه انسان.
\par 14 زیرا که سخنان خود را به ضد من ترتیب نداده است، و به سخنان شما او را جواب نخواهم داد.
\par 15 ایشان حیران شده، دیگر جواب ندادند، وسخن از ایشان منقطع شد.
\par 16 پس آیا من انتظاربکشم چونکه سخن نمی گویند؟ و ساکت شده، دیگر جواب نمی دهند؟
\par 17 پس من نیز از حصه خود جواب خواهم داد، و من نیز رای خود رابیان خواهم نمود.
\par 18 زیرا که از سخنان، مملوهستم. و روح باطن من، مرا به تنگ می‌آورد.
\par 19 اینک دل من مثل شرابی است که مفتوح نشده باشد، و مثل مشکهای تازه نزدیک است بترکد.
\par 20 سخن خواهم راند تا راحت یابم و لبهای خودرا گشوده، جواب خواهم داد.
\par 21 حاشا از من که طرفداری نمایم و به احدی کلام تملق‌آمیز گویم.چونکه به گفتن سخنان تملق‌آمیز عارف نیستم. والا خالقم مرا به زودی خواهد برداشت.
\par 22 چونکه به گفتن سخنان تملق‌آمیز عارف نیستم. والا خالقم مرا به زودی خواهد برداشت.
 
\chapter{33}

\par 1 «لیکن‌ای ایوب، سخنان مرا استماع نما. و به تمامی کلام من گوش بگیر.
\par 2 اینک الان دهان خود را گشودم، و زبانم در کامم متکلم شد.
\par 3 کلام من موافق راستی قلبم خواهدبود. و لبهایم به معرفت خالص تنطق خواهدنمود.
\par 4 روح خدا مرا آفریده، و نفخه قادرمطلق مرا زنده ساخته است.
\par 5 اگر می‌توانی مرا جواب ده، و پیش روی من، کلام را ترتیب داده بایست.
\par 6 اینک من مثل تو از خدا هستم. و من نیز از گل سرشته شده‌ام.
\par 7 اینک هیبت من تو را نخواهدترسانید، و وقار من بر تو سنگین نخواهد شد.
\par 8 «یقین در گوش من سخن گفتی و آواز کلام تو را شنیدم
\par 9 که گفتی من زکی و بی‌تقصیر هستم. من پاک هستم و در من گناهی نیست.
\par 10 اینک او علتها برمن می‌جوید. و مرا دشمن خودمی شمارد.
\par 11 پایهایم را در کنده می‌گذارد و همه راههایم را مراقبت می‌نماید.
\par 12 هان در این امر توصادق نیستی. من تو را جواب می‌دهم، زیرا خدااز انسان بزرگتر است.
\par 13 چرا با او معارضه می‌نمایی، از این جهت که از همه اعمال خوداطلاع نمی دهد؟
\par 14 زیرا خدا یک دفعه تکلم می‌کند، بلکه دو دفعه و انسان ملاحظه نمی نماید.
\par 15 در خواب، در رویای شب، چون خواب سنگین بر انسان مستولی می‌شود، حینی که دربستر خود در خواب می‌باشد.
\par 16 آنگاه گوشهای انسان را می‌گشاید و تادیب ایشان را ختم می‌سازد.
\par 17 تا انسان را از اعمالش برگرداند وتکبر را از مردمان بپوشاند.
\par 18 جان او را از حفره نگاه می‌دارد و حیات او را از هلاکت شمشیر.
\par 19 بادرد در بستر خود سرزنش می‌یابد، و اضطراب دایمی در استخوانهای وی است.
\par 20 پس جان اونان را مکروه می‌دارد و نفس او خوراک لطیف را.
\par 21 گوشت او چنان فرسوده شد که دیده نمی شودو استخوانهای وی که دیده نمی شد برهنه گردیده است.
\par 22 جان او به حفره نزدیک می‌شود و حیات او به هلاک کنندگان.
\par 23 «اگر برای وی یکی به منزله هزار فرشته یامتوسطی باشد، تا آنچه را که برای انسان راست است به وی اعلان نماید،
\par 24 آنگاه بر او ترحم نموده، خواهد گفت: او را از فرو رفتن به هاویه برهان، من کفاره‌ای پیدا نموده‌ام.
\par 25 گوشت او ازگوشت طفل لطیف تر خواهد شد. و به ایام جوانی خود خواهد برگشت.
\par 26 نزد خدا دعا کرده، او رامستجاب خواهد فرمود، و روی او را با شادمانی خواهد دید. و عدالت انسان را به او رد خواهدنمود.
\par 27 پس در میان مردمان سرود خوانده، خواهد گفت: گناه کردم و راستی را منحرف ساختم، و مکافات آن به من نرسید.
\par 28 نفس مرا ازفرورفتن به هاویه فدیه داد، و جان من، نور رامشاهده می‌کند.
\par 29 اینک همه این چیزها را خدابه عمل می‌آورد، دو دفعه و سه دفعه با انسان.
\par 30 تا جان او را از هلاکت برگرداند و او را از نورزندگان، منور سازد.
\par 31 ‌ای ایوب متوجه شده، مرااستماع نما، و خاموش باش تا من سخن رانم.
\par 32 اگر سخنی داری به من جواب بده، متکلم شوزیرا می‌خواهم تو را مبری سازم.و اگر نه، تومرا بشنو. خاموش باش تا حکمت را به تو تعلیم دهم.»
\par 33 و اگر نه، تومرا بشنو. خاموش باش تا حکمت را به تو تعلیم دهم.»
 
\chapter{34}

\par 1 پس الیهو تکلم نموده، گفت:
\par 2 «ای حکیمان سخنان مرا بشنوید، و‌ای عارفان، به من گوش گیرید.
\par 3 زیرا گوش، سخنان را امتحان می‌کند، چنانکه کام، طعام را ذوق می‌نماید.
\par 4 انصاف را برای خود اختیار کنیم، و درمیان خود نیکویی را بفهمیم.
\par 5 چونکه ایوب گفته است که بی‌گناه هستم. و خدا داد مرا از من برداشته است.
\par 6 هرچند انصاف با من است دروغگو شمرده شده‌ام، و هرچند بی‌تقصیرم، جراحت من علاج ناپذیر است.
\par 7 کدام شخص مثل ایوب است که سخریه را مثل آب می‌نوشد
\par 8 که در رفاقت بدکاران سالک می‌شود، و با مردان شریر رفتار می‌نماید.
\par 9 زیرا گفته است انسان رافایده‌ای نیست که رضامندی خدا را بجوید.
\par 10 پس الان‌ای صاحبان فطانت مرا بشنوید، حاشااز خدا که بدی کند. و از قادرمطلق، که ظلم نماید.
\par 11 زیرا که انسان را به حسب عملش مکافات می‌دهد، و بر هرکس موافق راهش می‌رساند. 
\par 12 وبه درستی که خدا بدی نمی کند، و قادر مطلق انصاف را منحرف نمی سازد.
\par 13 کیست که زمین رابه او تفویض نموده، و کیست که تمامی ربع مسکون را به او سپرده باشد.
\par 14 اگر او دل خود رابه وی مشغول سازد، اگر روح و نفخه خویش رانزد خود بازگیرد،
\par 15 تمامی بشر با هم هلاک می‌شوند و انسان به خاک راجع می‌گردد.
\par 16 پس اگر فهم داری این را بشنو، و به آواز کلام من گوش ده.
\par 17 آیا کسی‌که از انصاف نفرت دارد سلطنت خواهد نمود؟ و آیا عادل کبیر را به گناه اسنادمی دهی؟
\par 18 آیا به پادشاه گفته می‌شود که لئیم هستی، یا به نجیبان که شریر می‌باشید؟
\par 19 پس چگونه به آنکه امیران را طرفداری نمی نماید ودولتمند را بر فقیر ترجیح نمی دهد. زیرا که جمیع ایشان عمل دستهای وی‌اند.
\par 20 درلحظه‌ای در نصف شب می‌میرند. قوم مشوش شده، می‌گذرند، و زورآوران بی‌واسطه دست انسان هلاک می‌شوند.
\par 21 «زیرا چشمان او بر راههای انسان می‌باشد، و تمامی قدمهایش را می‌نگرد.
\par 22 ظلمتی نیست و سایه موت نی، که خطاکاران خویشتن را درآن پنهان نمایند.
\par 23 زیرا اندک زمانی بر احدی تامل نمی کند تا او پیش خدا به محاکمه بیاید.
\par 24 زورآوران را بدون تفحص خرد می‌کند، ودیگران را به‌جای ایشان قرار می‌دهد.
\par 25 هرآینه اعمال ایشان را تشخیص می‌نماید، و شبانگاه ایشان را واژگون می‌سازد تا هلاک شوند.
\par 26 به‌جای شریران ایشان را می‌زند، در مکان نظرکنندگان.
\par 27 از آن جهت که از متابعت اومنحرف شدند، و در همه طریقهای وی تامل ننمودند.
\par 28 تا فریاد فقیر را به او برسانند، و اوفغان مسکینان را بشنود.
\par 29 چون او آرامی دهدکیست که در اضطراب اندازد، و چون روی خودرا بپوشاند کیست که او را تواند دید. خواه به امتی خواه به انسانی مساوی است،
\par 30 تا مردمان فاجرسلطنت ننمایند و قوم را به دام گرفتار نسازند.
\par 31 لیکن آیا کسی هست که به خدا بگوید: سزایافتم، دیگر عصیان نخواهم ورزید.
\par 32 و آنچه راکه نمی بینم تو به من بیاموز، و اگر گناه کردم باردیگر نخواهم نمود.
\par 33 آیا برحسب رای تو جزاداده، خواهد گفت: چونکه تو رد می‌کنی پس تواختیار کن و نه من. و آنچه صواب می‌دانی بگو.
\par 34 صاحبان فطانت به من خواهند گفت، بلکه هرمرد حکیمی که مرا می‌شنود
\par 35 که ایوب بدون معرفت حرف می‌زند و کلام او از روی تعقل نیست.
\par 36 کاش که ایوب تا به آخر آزموده شود، زیرا که مثل شریران جواب می‌دهد.چونکه برگناه خود طغیان را مزید می‌کند و در میان مادستک می‌زند و به ضد خدا سخنان بسیارمی گوید.»
\par 37 چونکه برگناه خود طغیان را مزید می‌کند و در میان مادستک می‌زند و به ضد خدا سخنان بسیارمی گوید.»
\par 2 «آیا این را انصاف می‌شماری که گفتی من از خدا عادل تر هستم؟
\par 3 زیرا گفته‌ای برای توچه فایده خواهد شد، و به چه چیز بیشتر از گناهم منفعت خواهم یافت.
\par 4 من تو را جواب می‌گویم ورفقایت را با تو.
\par 5 به سوی آسمانها نظر کن و ببین وافلاک را ملاحظه نما که از تو بلندترند.
\par 6 اگر گناه کردی به او چه رسانیدی؟ و اگر تقصیرهای تو بسیار شد برای وی چه کردی؟
\par 7 اگر بی‌گناه شدی به او چه بخشیدی؟ و یا از دست تو چه چیز را گرفته است؟
\par 8 شرارت تو به مردی چون تو (ضرر می‌رساند) و عدالت تو به بنی آدم (فایده می‌رساند).
\par 9 از کثرت ظلمها فریاد برمی آورند واز دست زورآوران استغاثه می‌کنند،
\par 10 و کسی نمی گوید که خدای آفریننده من کجا است که شبانگاه سرودها می‌بخشد
\par 11 و ما را از بهایم زمین تعلیم می‌دهد، و از پرندگان آسمان حکمت می‌بخشد.
\par 12 پس به‌سبب تکبر شریران فریادمی کنند اما او اجابت نمی نماید،
\par 13 زیرا خدابطالت را نمی شنود و قادر مطلق برآن ملاحظه نمی فرماید.
\par 14 هرچند می‌گویی که او رانمی بینم، لیکن دعوی در حضور وی است پس منتظر او باش.
\par 15 و اما الان از این سبب که درغضب خویش مطالبه نمی کند و به کثرت گناه اعتنا نمی نماید،از این جهت ایوب دهان خودرا به بطالت می‌گشاید و بدون معرفت سخنان بسیار می‌گوید.»
\par 16 از این جهت ایوب دهان خودرا به بطالت می‌گشاید و بدون معرفت سخنان بسیار می‌گوید.»
 
\chapter{36}

\par 1 و الیهو باز‌گفت:
\par 2 «برای من‌اندکی صبرکن تا تو را اعلام نمایم، زیرا از برای خدا هنوز سخنی باقی است.
\par 3 علم خود را از دورخواهم آورد و به خالق خویش، عدالت راتوصیف خواهم نمود.
\par 4 چونکه حقیقت کلام من دروغ نیست، و آنکه در علم کامل است نزد توحاضر است.
\par 5 اینک خدا قدیر است و کسی رااهانت نمی کند و در قوت عقل قادر است.
\par 6 شریررا زنده نگاه نمی دارد و داد مسکینان را می‌دهد.
\par 7 چشمان خود را از عادلان برنمی گرداند، بلکه ایشان را با پادشاهان بر کرسی تا به ابد می‌نشاند، پس سرافراشته می‌شوند.
\par 8 اما هرگاه به زنجیرهابسته شوند، و به بندهای مصیبت گرفتار گردند،
\par 9 آنگاه اعمال ایشان را به ایشان می‌نمایاند وتقصیرهای ایشان را از اینکه تکبر نموده‌اند،
\par 10 وگوشهای ایشان را برای تادیب باز می‌کند، و امرمی فرماید تا از گناه بازگشت نمایند.
\par 11 پس اگربشنوند و او را عبادت نمایند، ایام خویش را درسعادت بسر خواهند برد، و سالهای خود را درشادمانی.
\par 12 و اما اگر نشنوند از تیغ خواهند افتاد، و بدون معرفت، جان را خواهند سپرد.
\par 13 اماآنانی که در دل، فاجرند غضب را ذخیره می‌نمایند، و چون ایشان را می‌بندد استغاثه نمی نمایند.
\par 14 ایشان در عنفوان جوانی می‌میرندو حیات ایشان با فاسقان (تلف می‌شود).
\par 15 مصیبت کشان را به مصیبت ایشان نجات می‌بخشد و گوش ایشان را در تنگی باز می‌کند.
\par 16 «پس تو را نیز از دهان مصیبت بیرون می‌آورد، در مکان وسیع که در آن تنگی نمی بود وزاد سفره تو از فربهی مملو می‌شد،
\par 17 و تو ازداوری شریر پر هستی، لیکن داوری و انصاف باهم ملتصقند.
\par 18 باحذر باش مبادا خشم تو را به تعدی ببرد، و زیادتی کفاره تو را منحرف سازد.
\par 19 آیا او دولت تو را به حساب خواهد آورد؟ نی، نه طلا و نه تمامی قوای توانگری را.
\par 20 برای شب آرزومند مباش، که امت‌ها را از جای ایشان می‌برد.
\par 21 با حذر باش که به گناه مایل نشوی، زیرا که تو آن را بر مصیبت ترجیح داده‌ای.
\par 22 اینک خدا در قوت خود متعال می‌باشد. کیست که مثل او تعلیم بدهد؟
\par 23 کیست که طریق او را به او تفویض کرده باشد؟ و کیست که بگویدتو بی‌انصافی نموده‌ای؟
\par 24 به یاد داشته باش که اعمال او را تکبیر گویی که درباره آنها مردمان می‌سرایند.
\par 25 جمیع آدمیان به آنها می‌نگرند. مردمان آنها را از دور مشاهده می‌نمایند.
\par 26 اینک خدا متعال است و او را نمی شناسیم. و شماره سالهای او را تفحص نتوان کرد.
\par 27 زیرا که قطره های آب را جذب می‌کند و آنها باران را ازبخارات آن می‌چکاند.
\par 28 که ابرها آن را به شدت می‌ریزد و بر انسان به فراوانی می‌تراود.
\par 29 آیاکیست که بفهمد ابرها چگونه پهن می‌شوند، یارعدهای خیمه او را بداند؟
\par 30 اینک نور خود رابر آن می‌گستراند. و عمق های دریا را می‌پوشاند.
\par 31 زیرا که به واسطه آنها قوم‌ها را داوری می‌کند، و رزق را به فراوانی می‌بخشد.
\par 32 دستهای خودرا با برق می‌پوشاند، و آن را بر هدف مامورمی سازد.رعدش از او خبر می‌دهد و مواشی از برآمدن او اطلاع می‌دهند.
\par 33 رعدش از او خبر می‌دهد و مواشی از برآمدن او اطلاع می‌دهند.
 
\chapter{37}

\par 1 «از این نیز دل می‌لرزد و از جای خودمتحرک می‌گردد.
\par 2 گوش داده، صدای آواز او را بشنوید، و زمزمه‌ای را که از دهان وی صادر می‌شود،
\par 3 آن را در زیر تمامی آسمانهامی فرستد، و برق خویش را تا کرانهای زمین.
\par 4 بعد از آن صدای غرش می‌کند و به آواز جلال خویش رعد می‌دهد و چون آوازش شنیده شدآنها را تاخیر نمی نماید.
\par 5 خدا از آواز خودرعدهای عجیب می‌دهد. اعمال عظیمی که ماآنها را ادراک نمی کنیم به عمل می‌آورد،
\par 6 زیرابرف را می‌گوید: بر زمین بیفت. و همچنین بارش باران را و بارش بارانهای زورآور خویش را.
\par 7 دست هر انسان را مختوم می‌سازد تا جمیع مردمان اعمال او را بدانند.
\par 8 آنگاه وحوش به ماوای خود می‌روند و در بیشه های خویش آرام می‌گیرند.
\par 9 از برجهای جنوب گردباد می‌آید و ازبرجهای شمال برودت.
\par 10 از نفخه خدا یخ بسته می‌شود و سطح آبها منجمد می‌گردد.
\par 11 ابرها رانیز به رطوبت سنگین می‌سازد و سحاب، برق خود را پراکنده می‌کند.
\par 12 و آنها به دلالت او به هر سو منقلب می‌شوند تا هرآنچه به آنها امرفرماید بر روی تمامی ربع مسکون به عمل آورند.
\par 13 خواه آنها را برای تادیب بفرستد یا به جهت زمین خود یا برای رحمت.
\par 14 «ای ایوب این را استماع نما. بایست و دراعمال عجیب خدا تامل کن.
\par 15 آیا مطلع هستی وقتی که خدا عزم خود را به آنها قرار می‌دهد وبرق، ابرهای خود را درخشان می‌سازد؟
\par 16 آیا تواز موازنه ابرها مطلع هستی؟ یا از اعمال عجیبه اوکه در علم، کامل است.
\par 17 که چگونه رختهای توگرم می‌شود هنگامی که زمین از باد جنوبی ساکن می‌گردد.
\par 18 آیا مثل او می‌توانی فلک رابگسترانی که مانند آینه ریخته شده مستحکم است؟
\par 19 ما را تعلیم بده که با وی چه توانیم گفت، زیرا به‌سبب تاریکی سخن نیکو نتوانیم آورد.
\par 20 آیا چون سخن گویم به او خبر داده می‌شود یاانسان سخن گوید تا هلاک گردد.
\par 21 و حال آفتاب را نمی توان دید هرچند در سپهر درخشان باشد تاباد وزیده آن را پاک کند.
\par 22 درخشندگی طلایی از شمال می‌آید و نزد خدا جلال مهیب است.
\par 23 قادر مطلق را ادراک نمی توانیم کرد، او در قوت وراستی عظیم است و در عدالت کبیر که بی‌انصافی نخواهد کرد.لهذا مردمان از او می‌ترسند، امااو بر جمیع دانادلان نمی نگرد.»
\par 24 لهذا مردمان از او می‌ترسند، امااو بر جمیع دانادلان نمی نگرد.»
 
\chapter{38}

\par 1 و خداوند ایوب را از میان گردبادخطاب کرده، گفت:
\par 2 «کیست که مشورت را از سخنان بی‌علم تاریک می‌سازد؟
\par 3 الان کمر خود را مثل مرد ببند، زیرا که از توسوال می‌نمایم پس مرا اعلام نما.
\par 4 وقتی که زمین را بنیاد نهادم کجا بودی؟ بیان کن اگر فهم داری.
\par 5 کیست که آن را پیمایش نمود؟ اگر می‌دانی! وکیست که ریسمانکار را بر آن کشید؟
\par 6 پایه هایش بر چه چیز گذاشته شد؟ و کیست که سنگ زاویه‌اش را نهاد،
\par 7 هنگامی که ستارگان صبح باهم ترنم نمودند، و جمیع پسران خدا آوازشادمانی دادند؟
\par 8 و کیست که دریا را به درهامسدود ساخت، وقتی که به در جست و از رحم بیرون آمد؟
\par 9 وقتی که ابرها را لباس آن گردانیدم و تاریکی غلیظ را قنداقه آن ساختم.
\par 10 و حدی برای آن قرار دادم و پشت بندها و درها تعیین نمودم.
\par 11 و گفتم تا به اینجا بیا و تجاوز منما. و دراینجا امواج سرکش تو بازداشته شود.
\par 12 «آیا تو از ابتدای عمر خود صبح را فرمان دادی، و فجر را به موضعش عارف گردانیدی؟
\par 13 تا کرانه های زمین را فرو‌گیرد و شریران از آن افشانده شوند.
\par 14 مثل گل زیر خاتم مبدل می‌گردد. و همه‌چیز مثل لباس صورت می‌پذیرد.
\par 15 و نور شریران از ایشان گرفته می‌شود، و بازوی بلند شکسته می‌گردد.
\par 16 آیا به چشمه های دریا داخل شده، یا به عمقهای لجه رفته‌ای؟
\par 17 آیا درهای موت برای تو باز شده است؟ یا درهای سایه موت را دیده‌ای؟
\par 18 آیاپهنای زمین را ادراک کرده‌ای؟ خبر بده اگر این همه را می‌دانی.
\par 19 راه مسکن نور کدام است، ومکان ظلمت کجا می‌باشد،
\par 20 تا آن را به حدودش برسانی، و راههای خانه او را درک نمایی؟
\par 21 البته می‌دانی، چونکه در آنوقت مولودشدی، و عدد روزهایت بسیار است! 
\par 22 «آیا به مخزن های برف داخل شده، و خزینه های تگرگ را مشاهده نموده‌ای،
\par 23 که آنهارا به جهت وقت تنگی نگاه داشتم، به جهت روزمقاتله و جنگ؟
\par 24 به چه طریق روشنایی تقسیم می‌شود، و باد شرقی بر روی زمین منتشرمی گردد؟
\par 25 کیست که رودخانه‌ای برای سیل کند، یا طریقی به جهت صاعقه‌ها ساخت.
\par 26 تا برزمینی که کسی در آن نیست ببارد و بر بیابانی که در آن آدمی نباشد،
\par 27 تا (زمین ) ویران و بایر راسیراب کند، و علفهای تازه را از آن برویاند؟
\par 28 آیا باران را پدری هست؟ یا کیست که قطرات شبنم را تولید نمود؟
\par 29 از رحم کیست که یخ بیرون آمد؟ و ژاله آسمان را کیست که تولیدنمود؟
\par 30 آبها مثل سنگ منجمد می‌شود، وسطح لجه یخ می‌بندد.
\par 31 آیا عقد ثریا رامی بندی؟ یا بندهای جبار را می‌گشایی؟
\par 32 آیابرجهای منطقه البروج را در موسم آنها بیرون می‌آوری؟ و دب اکبر را با بنات او رهبری می‌نمایی؟
\par 33 آیا قانون های آسمان را می‌دانی؟ یا آن را بر زمین مسلط می‌گردانی؟
\par 34 آیا آوازخود را به ابرها می‌رسانی تا سیل آبها تو رابپوشاند؟
\par 35 آیا برقها را می‌فرستی تا روانه شوند، و به تو بگویند اینک حاضریم؟
\par 36 کیست که حکمت را در باطن نهاد یا فطانت را به دل بخشید؟
\par 37 کیست که با حکمت، ابرها رابشمارد؟ و کیست که مشکهای آسمان را بریزد؟
\par 38 چون غبار گل شده، جمع می‌شود و کلوخها باهم می‌چسبند.
\par 39 آیا شکار را برای شیر ماده صید می‌کنی؟ و اشتهای شیر ژیان را سیرمی نمایی؟
\par 40 حینی که در ماوای خود خویشتن را جمع می‌کنند و در بیشه در کمین می‌نشینند؟کیست که غذا را برای غراب آماده می‌سازد، چون بچه هایش نزد خدا فریاد برمی آورند، و به‌سبب نبودن خوراک آواره می‌گردند؟
\par 41 کیست که غذا را برای غراب آماده می‌سازد، چون بچه هایش نزد خدا فریاد برمی آورند، و به‌سبب نبودن خوراک آواره می‌گردند؟
 
\chapter{39}

\par 1 «آیا وقت زاییدن بز کوهی را می‌دانی؟ یا زمان وضع حمل آهو را نشان می‌دهی؟
\par 2 آیا ماههایی را که کامل می‌سازندحساب توانی کرد؟ یا زمان زاییدن آنهارامی دانی؟
\par 3 خم شده، بچه های خود را می‌زایند واز دردهای خود فارغ می‌شوند.
\par 4 بچه های آنهاقوی شده، در بیابان نمو می‌کنند، می‌روند و نزدآنها برنمی گردند.
\par 5 کیست که خر وحشی را رهاکرده، آزاد ساخت. و کیست که بندهای گورخر راباز نمود.
\par 6 که من بیابان را خانه او ساختم، وشوره زار را مسکن او گردانیدم.
\par 7 به غوغای شهراستهزاء می‌کند و خروش رمه بان را گوش نمی گیرد.
\par 8 دایره کوهها چراگاه او است و هرگونه سبزه را می‌طلبد.
\par 9 آیا گاو وحشی راضی شود که تو را خدمت نماید، یا نزد آخور تو منزل گیرد؟
\par 10 آیا گاو وحشی را به ریسمانش به شیار توانی بست؟ یا وادیها را از عقب تو مازو خواهد نمود؟
\par 11 آیا از اینکه قوتش عظیم است بر او اعتمادخواهی کرد؟ و کار خود را به او حواله خواهی نمود؟
\par 12 آیا براو توکل خواهی کرد که محصولت را باز آورد و آن را به خرمنگاهت جمع کند؟
\par 13 «بال شترمرغ به شادی متحرک می‌شود واما پر و بال او مثل لقلق نیست.
\par 14 زیرا که تخمهای خود را به زمین وامی گذارد و بر روی خاک آنها را گرم می‌کند
\par 15 و فراموش می‌کند که پا آنها را می‌افشرد، و وحوش صحرا آنها راپایمال می‌کنند.
\par 16 با بچه های خود سختی می کند که گویا از آن او نیستند، محنت او باطل است و متاسف نمی شود.
\par 17 زیرا خدا او را ازحکمت محروم ساخته، و از فطانت او را نصیبی نداده است.
\par 18 هنگامی که به بلندی پرواز می‌کنداسب و سوارش را استهزا می‌نماید.
\par 19 «آیا تو اسب را قوت داده و گردن او را به یال ملبس گردانیده‌ای؟
\par 20 آیا او را مثل ملخ به جست وخیز آورده‌ای؟ خروش شیهه او مهیب است.
\par 21 در وادی پا زده، از قوت خود وجدمی نماید و به مقابله مسلحان بیرون می‌رود.
\par 22 برخوف استهزاء کرده، هراسان نمی شود، و از دم شمشیر برنمی گردد.
\par 23 ترکش بر او چکچک می‌کند، و نیزه درخشنده و مزراق
\par 24 با خشم وغیض زمین را می‌نوردد. و چون کرنا صدا می‌کندنمی ایستد،
\par 25 وقتی که کرنا نواخته شود هه هه می‌گوید و جنگ را از دور استشمام می‌کند، وخروش سرداران و غوغا را.
\par 26 آیا از حکمت توشاهین می‌پرد؟ و بالهای خود را بطرف جنوب پهن می‌کند؟
\par 27 آیا از فرمان تو عقاب صعودمی نماید و آشیانه خود را به‌جای بلند می‌سازد؟
\par 28 بر صخره ساکن شده، ماوا می‌سازد. بر صخره تیز و بر ملاذ منیع.
\par 29 از آنجا خوراک خود را به نظر می‌آورد و چشمانش از دور می‌نگرد.بچه هایش خون را می‌مکند و جایی که کشتگانند او آنجا است.»
\par 30 بچه هایش خون را می‌مکند و جایی که کشتگانند او آنجا است.»
\par 2 «آیا مجادله کننده با قادرمطلق مخاصمه نماید؟ کسی‌که با خدا محاجه کند آن را جواب بدهد.»
\par 3 آنگاه ایوب خداوند را جواب داده، گفت:
\par 4 «اینک من حقیر هستم و به تو چه جواب دهم؟ دست خود را به دهانم گذاشته‌ام.
\par 5 یک مرتبه گفتم و تکرار نخواهم کرد. بلکه دو مرتبه و نخواهم افزود.»
\par 6 پس خداوند ایوب را از گردباد خطاب کرد وگفت:
\par 7 «الان کمر خود را مثل مرد ببند. از توسوال می‌نمایم و مرا اعلام کن.
\par 8 آیا داوری مرانیز باطل می‌نمایی؟ و مرا ملزم می‌سازی تاخویشتن را عادل بنمایی؟
\par 9 آیا تو را مثل خدابازویی هست؟ و به آواز مثل او رعد توانی کرد؟
\par 10 الان خویشتن را به جلال و عظمت زینت بده. و به عزت و شوکت ملبس ساز.
\par 11 شدت غضب خود را بریز و به هرکه متکبر است نظر افکنده، اورا به زیر انداز.
\par 12 بر هرکه متکبر است نظر کن و اورا ذلیل بساز. و شریران را در جای ایشان پایمال کن.
\par 13 ایشان را با هم در خاک پنهان نما و رویهای ایشان را درجای مخفی محبوس کن.
\par 14 آنگاه من نیز درباره تو اقرار خواهم کرد، که دست راستت تو را نجات تواند داد.
\par 15 اینک بهیموت که او را باتو آفریده‌ام که علف را مثل گاو می‌خورد،
\par 16 همانا قوت او در کمرش می‌باشد، و توانایی وی در رگهای شکمش.
\par 17 دم خود را مثل سروآزاد می‌جنباند. رگهای رانش بهم پیچیده است.
\par 18 استخوانهایش مثل لوله های برنجین واعضایش مثل تیرهای آهنین است.
\par 19 او افضل صنایع خدا است. آن که او را آفرید حربه‌اش را به او داده است.
\par 20 به درستی که کوهها برایش علوفه می‌رویاند، که در آنها تمامی حیوانات صحرابازی می‌کنند
\par 21 زیر درختهای کنار می‌خوابد. در سایه نیزار و در خلاب.
\par 22 درختهای کنار او را به سایه خود می‌پوشاند، و بیدهای نهر، وی رااحاطه می‌نماید.
\par 23 اینک رودخانه طغیان می‌کند، لیکن او نمی ترسد و اگر‌چه اردن دردهانش ریخته شود ایمن خواهد بود.آیا چون نگران است او را گرفتار توان کرد؟ یا بینی وی را باقلاب توان سفت؟
\par 24 آیا چون نگران است او را گرفتار توان کرد؟ یا بینی وی را باقلاب توان سفت؟
 
\chapter{41}

\par 1 «آیا لویاتان را با قلاب توانی کشید؟ یازبانش را با ریسمان توانی فشرد؟
\par 2 آیادر بینی او مهار توانی کشید؟ یا چانه‌اش را باقلاب توانی سفت؟
\par 3 آیا او نزد تو تضرع زیادخواهد نمود؟ یا سخنان ملایم به تو خواهدگفت؟
\par 4 آیا با تو عهد خواهد بست یا او را برای بندگی دایمی خواهی گرفت؟
\par 5 آیا با او مثل گنجشک بازی توانی کرد؟ یا او را برای کنیزان خود توانی بست؟
\par 6 آیا جماعت (صیادان ) از اوداد و ستد خواهند کرد؟ یا او را در میان تاجران تقسیم خواهند نمود؟
\par 7 آیا پوست او را با نیزه هامملو توانی کرد؟ یا سرش را با خطافهای ماهی گیران؟
\par 8 اگر دست خود را بر او بگذاری جنگ را به یاد خواهی داشت و دیگر نخواهی کرد.
\par 9 اینک امید به او باطل است. آیا از رویتش نیز آدمی به روی درافکنده نمی شود؟
\par 10 کسی اینقدر متهور نیست که او را برانگیزاند. پس کیست که در حضور من بایستد؟
\par 11 کیست که سبقت جسته، چیزی به من داده، تابه او رد نمایم؟ هرچه زیر آسمان است از آن من می‌باشد.
\par 12 «درباره اعضایش خاموش نخواهم شد و ازجبروت و جمال ترکیب او خبر خواهم داد.
\par 13 کیست که روی لباس او را باز تواند نمود؟ وکیست که در میان دو صف دندانش داخل شود؟
\par 14 کیست که درهای چهره‌اش را بگشاید؟ دایره دندانهایش هولناک است.
\par 15 سپرهای زورآورش فخر او می‌باشد، با مهر محکم وصل شده است.
\par 16 با یکدیگر چنان چسبیده‌اند که باد از میان آنهانمی گذرد.
\par 17 با همدیگر چنان وصل شده‌اند و باهم ملتصقند که جدا نمی شوند.
\par 18 از عطسه های او نور ساطع می‌گردد و چشمان او مثل پلکهای فجر است.
\par 19 از دهانش مشعلها بیرون می‌آید وشعله های آتش برمی جهد.
\par 20 از بینی های او دودبرمی آید مثل دیگ جوشنده و پاتیل.
\par 21 از نفس او اخگرها افروخته می‌شود و از دهانش شعله بیرون می‌آید.
\par 22 بر گردنش قوت نشیمن دارد، وهیبت پیش رویش رقص می‌نماید.
\par 23 طبقات گوشت او بهم چسبیده است، و بر وی مستحکم است که متحرک نمی شود.
\par 24 دلش مثل سنگ مستحکم است، و مانند سنگ زیرین آسیا محکم می‌باشد.
\par 25 چون او برمی خیزد نیرومندان هراسان می‌شوند، و از خوف بی‌خود می‌گردند.
\par 26 اگر شمشیر به او انداخته شود اثر نمی کند، و نه نیزه و نه مزراق و نه تیر.
\par 27 آهن را مثل کاه می‌شمارد و برنج را مانند چوب پوسیده.
\par 28 تیرهای کمان او را فرار نمی دهد و سنگهای فلاخن نزد او به کاه مبدل می‌شود.
\par 29 عمود مثل کاه شمرده می‌شود و بر حرکت مزراق می‌خندد.
\par 30 در زیرش پاره های سفال تیز است و گردون پرمیخ را بر گل پهن می‌کند.
\par 31 لجه را مثل دیگ می‌جوشاند و دریا را مانند پاتیلچه عطاران می‌گرداند.
\par 32 راه را در عقب خویش تابان می‌سازد به نوعی که لجه را سفیدمو گمان می‌برند.
\par 33 بر روی خاک نظیر او نیست، که بدون خوف آفریده شده باشد.بر هرچیز بلند نظر می‌افکندو بر جمیع حیوانات سرکش پادشاه است.»
\par 34 بر هرچیز بلند نظر می‌افکندو بر جمیع حیوانات سرکش پادشاه است.»
 
\chapter{42}

\par 1 و ایوب خداوند را جواب داده، گفت:
\par 2 «می‌دانم که به هر چیز قادر هستی، وابدا قصد تو را منع نتوان نمود.
\par 3 کیست که مشورت را بی‌علم مخفی می‌سازد. لکن من به آنچه نفهمیدم تکلم نمودم. به چیزهایی که فوق ازعقل من بود و نمی دانستم.
\par 4 الان بشنو تا من سخن گویم. از تو سوال می‌نمایم مرا تعلیم بده.
\par 5 ازشنیدن گوش درباره تو شنیده بودم لیکن الان چشم من تو را می‌بیند.
\par 6 از این جهت از خویشتن کراهت دارم و در خاک و خاکستر توبه می‌نمایم.»
\par 7 و واقع شد بعد از اینکه خداوند این سخنان را به ایوب گفته بود که خداوند به الیفاز تیمانی فرمود: «خشم من بر تو و بر دو رفیقت افروخته شده، زیرا که درباره من آنچه راست است مثل بنده‌ام ایوب نگفتید.
\par 8 پس حال هفت گوساله وهفت قوچ برای خود بگیرید و نزد بنده من ایوب رفته، قربانی سوختنی به جهت خویشتن بگذرانید و بنده‌ام ایوب به جهت شما دعا خواهدنمود، زیرا که او را مستجاب خواهم فرمود، مباداپاداش حماقت شما را به شما برسانم چونکه درباره من آنچه راست است مثل بنده‌ام ایوب نگفتید.»
\par 9 پس الیفاز تیمانی و بلدد شوحی و صوفرنعماتی رفته، به نوعی که خداوند به ایشان امر فرموده بود عمل نمودند و خداوند ایوب رامستجاب فرمود.
\par 10 و چون ایوب برای اصحاب خود دعا کرد خداوند مصیبت او را دور ساخت وخداوند به ایوب دو چندان آنچه پیش داشته بودعطا فرمود.
\par 11 و جمیع برادرانش و همه خواهرانش و تمامی آشنایان قدیمش نزد وی آمده، در خانه‌اش با وی نان خوردند و او را درباره تمامی مصیبتی که خداوند به او رسانیده بودتعزیت گفته، تسلی دادند و هرکس یک قسیطه وهرکس یک حلقه طلا به او داد. 
\par 12 و خداوند آخرایوب را بیشتر از اول او مبارک فرمود، چنانکه او را چهارده هزار گوسفند و شش هزار شتر و هزارجفت گاو و هزار الاغ ماده بود.
\par 13 و او را هفت پسر و سه دختر بود.
\par 14 و دختر اول را یمیمه ودوم را قصیعه و سوم را قرن هفوک نام نهاد.
\par 15 و در تمامی زمین مثل دختران ایوب زنان نیکوصورت یافت نشدند و پدر ایشان، ایشان رادر میان برادرانشان ارثی داد.و بعد از آن ایوب صد و چهل سال زندگانی نمود و پسران خود و پسران پسران خود را تا پشت چهارم دید.
\par 16 و بعد از آن ایوب صد و چهل سال زندگانی نمود و پسران خود و پسران پسران خود را تا پشت چهارم دید.


\end{document}