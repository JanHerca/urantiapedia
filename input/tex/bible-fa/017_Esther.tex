\begin{document}

\title{استر}

 
\chapter{1}

\par 1 در ایام اخشورش (این امور واقع شد). این همان اخشورش است که از هند تا حبش، بر صد و بیست و هفت ولایت سلطنت می‌کرد.
\par 2 در آن ایام حینی که اخشورش پادشاه، بر کرسی سلطنت خویش در دارالسلطنه شوشن نشسته بود.
\par 3 در سال سوم از سلطنت خویش، ضیافتی برای جمیع سروران و خادمان خود برپا نمود و حشمت فارس ومادی از امرا و سروران ولایتها، به حضور او بودند.
\par 4 پس مدت مدید صد و هشتاد روز، توانگری جلال سلطنت خویش و حشمت مجد عظمت خود را جلوه می‌داد.
\par 5 پس بعد از انقضای آنروزها، پادشاه برای همه کسانی که دردارالسلطنه شوشن از خرد و بزرگ یافت شدند، ضیافت هفت روزه در عمارت باغ قصر پادشاه برپا نمود.
\par 6 پرده‌ها از کتان سفید و لاجورد، با ریسمانهای سفید و ارغوان در حلقه های نقره بر ستونهای مرمر سفید آویخته وتختهای طلا و نقره بر سنگفرشی از سنگ سماق و مرمر سفید و در و مرمر سیاه بود.
\par 7 و آشامیدن، از ظرفهای طلابود و ظرفها را اشکال مختلفه بود و شرابهای ملوکانه برحسب کرم پادشاه فراوان بود.
\par 8 و آشامیدن برحسب قانون بود که کسی بر کسی تکلف نمی نمود، زیرا پادشاه درباره همه بزرگان خانه‌اش چنین امر فرموده بود که هر کس موافق میل خود رفتار نماید.
\par 9 و وشتی ملکه نیز ضیافتی برای زنان خانه خسروی اخشورش پادشاه برپا نمود.
\par 10 در روز هفتم، چون دل پادشاه از شراب خوش شد، هفت خواجه‌سرا یعنی مهومان و بزتا و حربونا و بغتا و ابغتا و زاتر و کرکس را که درحضور اخشورش پادشاه خدمت می‌کردند، امر فرمود
\par 11 که وشتی ملکه را با تاج ملوکانه به حضور پادشاه بیاورندتا زیبایی او را به خلایق و سروران نشان دهد، زیرا که نیکو منظر بود.
\par 12 اما وشتی ملکه نخواست که برحسب فرمانی که پادشاه به‌دست خواجه‌سرایان فرستاده بود بیاید. پس پادشاه بسیار خشمناک شده، غضبش در دلش مشتعل گردید.
\par 13 آنگاه پادشاه به حکیمانی که از زمانها مخبر بودند تکلم نموده، (زیرا که عادت پادشاه با همه کسانی که به شریعت و احکام عارف بودند چنین بود.
\par 14 و مقربان او کرشنا و شیتار و ادماتا و ترشیش و مرس و مرسنا ومموکان، هفت رئیس فارس و مادی بودند که روی پادشاه را می‌دیدند و در مملکت به درجه اول می‌نشستند)
\par 15 گفت: «موافق شریعت، به وشتی ملکه چه باید کرد؟ چونکه به فرمانی که اخشورش پادشاه به‌دست خواجه‌سرایان فرستاده است، عمل ننموده.»
\par 16 آنگاه مموکان به حضور پادشاه و سروران عرض کرد که «وشتی ملکه، نه‌تنها به پادشاه تقصیر نموده، بلکه به همه روسا و جمیع طوایفی که در تمامی ولایتهای اخشورش پادشاه می‌باشند،
\par 17 زیرا چون این عمل ملکه نزد تمامی زنان شایع شود، آنگاه شوهرانشان در نظر ایشان خوار خواهند شد، حینی که مخبر شوند که اخشورش پادشاه امر فرموده است که وشتی ملکه را به حضورش بیاورند و نیامده است.
\par 18 و در آنوقت، خانمهای فارس و مادی که این عمل ملکه را بشنوند، به جمیع روسای پادشاه چنین خواهند گفت و این مورد بسیار احتقار و غضب خواهد شد.
\par 19 پس اگر پادشاه این رامصلحت داند، فرمان ملوکانه‌ای از حضور وی صادر شود و در شرایع فارس و مادی ثبت گردد، تا تبدیل نپذیرد، که وشتی به حضور اخشورش پادشاه دیگر نیاید و پادشاه رتبه ملوکانه او را به دیگری که بهتر از او باشد بدهد.
\par 20 وچون فرمانی که پادشاه صادر گرداند در تمامی مملکت عظیم او مسموع شود، آنگاه همه زنان شوهران خود را ازبزرگ و کوچک، احترام خواهند نمود.»
\par 21 و این سخن در نظر پادشاه و روسا پسندآمد و پادشاه موافق سخن مموکان عمل نمود.و مکتوبات به همه ولایتهای پادشاه به هرولایت، موافق خط آن و به هر قوم، موافق زبانش فرستاد تا هر مرد در خانه خود مسلط شود و درزبان قوم خود آن را بخواند.
\par 22 و مکتوبات به همه ولایتهای پادشاه به هرولایت، موافق خط آن و به هر قوم، موافق زبانش فرستاد تا هر مرد در خانه خود مسلط شود و درزبان قوم خود آن را بخواند.
 
\chapter{2}

\par 1 بعد از این وقایع، چون غضب اخشورش پادشاه فرو نشست، وشتی و آنچه را که او کرده بود و حکمی که درباره او صادر شده بود، به یاد آورد.
\par 2 و ملازمان پادشاه که او را خدمت می‌کردند، گفتند که «دختران باکره نیکو منظربرای پادشاه بطلبند.
\par 3 و پادشاه در همه ولایتهای مملکت خود وکلا بگمارد که همه دختران باکره نیکو منظر را به دارالسلطنه شوشن در خانه زنان زیر دست هیجای که خواجه‌سرای پادشاه ومستحفظ زنان می‌باشد، جمع کنند و به ایشان اسباب طهارت داده شود.
\par 4 و دختری که به نظرپادشاه پسند آید، در جای وشتی ملکه بشود.» پس این سخن در نظر پادشاه پسند آمد و همچنین عمل نمود.
\par 5 شخصی یهودی در دارالسلطنه شوشن بودکه به مردخای بن یائیر ابن شمعی ابن قیس بنیامینی مسمی بود.
\par 6 و او از اورشلیم جلای وطن شده بود، با اسیرانی که همراه یکنیا پادشاه یهودا جلای وطن شده بودند که نبوکدنصر پادشاه بابل ایشان را به اسیری آورده بود.
\par 7 و او هدسه، یعنی استر، دختر عموی خود را تربیت می‌نمودچونکه وی را پدر و مادر نبود و آن دختر، خوب صورت و نیکومنظر بود و بعد از وفات پدر ومادرش، مردخای وی را به‌جای دختر خودگرفت.
\par 8 پس چون امر و فرمان پادشاه شایع گردید ودختران بسیار در دارالسلطنه شوشن زیر دست هیجای جمع شدند، استر را نیز به خانه پادشاه، زیر دست هیجای که مستحفظ زنان بود آوردند.
\par 9 و آن دختر به نظر او پسند آمده، در حضورش التفات یافت. پس به زودی، اسباب طهارت وتحفه هایش را به وی داد و نیز هفت کنیز را که از خانه پادشاه برگزیده شده بودند که به وی داده شوند و او را با کنیزانش به بهترین خانه زنان نقل کرد.
\par 10 و استر، قومی و خویشاوندی خود رافاش نکرد، زیرا که مردخای او را امر فرموده بودکه نکند.
\par 11 و مردخای روز به روز پیش صحن خانه زنان گردش می‌کرد تا از احوال استر و ازآنچه به وی واقع شود، اطلاع یابد.
\par 12 و چون نوبه هر دختر می‌رسید که نزداخشورش پادشاه داخل شود، یعنی بعد از آنکه آنچه را که برای زنان مرسوم بود که در مدت دوازده ماه کرده شود چونکه ایام تطهیر ایشان بدین منوال تمام می‌شد، یعنی شش ماه به روغن مر و شش ماه به عطریات و اسباب تطهیر زنان.
\par 13 آنگاه آن دختر بدین طور نزد پادشاه داخل می‌شد که هر‌چه را می‌خواست به وی می‌دادند تاآن را از خانه زنان به خانه پادشاه با خود ببرد.
\par 14 در وقت شام داخل می‌شد و صبحگاهان به خانه دوم زنان، زیر دست شعشغاز که خواجه‌سرای پادشاه و مستحفظ متعه‌ها بود، برمی گشت و بار دیگر، نزد پادشاه داخل نمی شد، مگر اینکه پادشاه در او رغبت کرده، او را بنام بخواند.
\par 15 و چون نوبه استر، دختر ابیحایل، عموی مردخای که او را بجای دختر خود گرفته بودرسید که نزد پادشاه داخل شود، چیزی سوای آنچه هیجای، خواجه‌سرای پادشاه و مستحفظزنان گفته بود نخواست و استر در نظر هر‌که او رامی دید، التفات می‌یافت.
\par 16 پس استر را نزداخشورش پادشاه، به قصر ملوکانه‌اش در ماه دهم که ماه طیبیت باشد، در سال هفتم سلطنت اوآوردند.
\par 17 و پادشاه، استر را از همه زنان زیاده دوست داشت و از همه دوشیزگان، در حضور وی نعمت و التفات زیاده یافت. لهذا تاج ملوکانه را بر سرش گذاشت و او را در جای وشتی ملکه ساخت.
\par 18 و پادشاه ضیافت عظیمی یعنی ضیافت استر را برای همه روسا و خادمان خودبرپا نمود و به ولایتها راحت بخشیده، برحسب کرم ملوکانه خود، عطایا ارزانی داشت.
\par 19 و چون دوشیزگان، بار دیگر جمع شدند، مردخای بر دروازه پادشاه نشسته بود.
\par 20 و استرهنوز خویشاوندی و قومی خود را بر وفق آنچه مردخای به وی امر فرموده بود فاش نکرده بود، زیرا که استر حکم مردخای را مثل زمانی که نزدوی تربیت می‌یافت بجا می‌آورد.
\par 21 در آن ایام، حینی که مردخای در دروازه پادشاه نشسته بود، دونفر از خواجه‌سرایان پادشاه و حافظان آستانه یعنی بغتان و تارش غضبناک شده، خواستند که بر اخشورش پادشاه دست بیندازند.
\par 22 و چون مردخای از این امر اطلاع یافت، استر ملکه را خبر داد و استر، پادشاه را اززبان مردخای مخبر ساخت.پس این امر راتفحص نموده، صحیح یافتند و هر دوی ایشان رابر دار کشیدند و این قصه در حضور پادشاه، درکتاب تواریخ ایام مرقوم شد.
\par 23 پس این امر راتفحص نموده، صحیح یافتند و هر دوی ایشان رابر دار کشیدند و این قصه در حضور پادشاه، درکتاب تواریخ ایام مرقوم شد.
 
\chapter{3}

\par 1 بعد از این وقایع، اخشورش پادشاه، هامان بن همداتای اجاجی را عظمت داده، به درجه بلند رسانید و کرسی او را از تمامی روسایی که با او بودند بالاتر گذاشت.
\par 2 و جمیع خادمان پادشاه که در دروازه پادشاه می‌بودند، به هامان سر فرود آورده، وی را سجده می‌کردند، زیرا که پادشاه درباره‌اش چنین امر فرموده بود. لکن مردخای سر فرود نمی آورد و او را سجده نمی کرد.
\par 3 و خادمان پادشاه که در دروازه پادشاه بودند، از مردخای پرسیدند که «تو چرا از امرپادشاه تجاوز می‌نمایی؟»
\par 4 اما هر‌چند، روز به روز این سخن را به وی می‌گفتند، به ایشان گوش نمی داد. پس هامان راخبر دادند تا ببینند که آیا کلام مردخای ثابت می‌شود یا نه، زیرا که ایشان را خبر داده بود که من یهودی هستم.
\par 5 و چون هامان دید که مردخای سر فرود نمی آورد و او را سجده نمی نماید، هامان از غضب مملو گردید.
\par 6 و چونکه دست انداختن بر مردخای، تنها به نظر وی سهل آمد واو را از قوم مردخای اطلاع داده بودند، پس هامان قصد هلاک نمودن جمیع یهودیانی که در تمامی مملکت اخشورش بودند کرد، زانرو که قوم مردخای بودند.
\par 7 در ماه اول از سال دوازدهم سلطنت اخشورش که ماه نیسان باشد، هر روز در حضورهامان و هر ماه تا ماه دوازدهم که ماه اذار باشد، فور یعنی قرعه می‌انداختند.
\par 8 پس هامان به اخشورش پادشاه گفت: «قومی هستند که در میان قوم‌ها در جمیع ولایتهای مملکت تو پراکنده ومتفرق می‌باشند و شرایع ایشان، مخالف همه قومها است و شرایع پادشاه را به‌جا نمی آورند. لهذا ایشان را چنین واگذاشتن برای پادشاه مفیدنیست.
\par 9 اگر پادشاه را پسند آید، حکمی نوشته شود که ایشان را هلاک سازند. و من ده هزار وزنه نقره به‌دست عاملان خواهم داد تا آن را به خزانه پادشاه بیاورند.»
\par 10 آنگاه پادشاه انگشترخود را از دستش بیرون کرده، آن را به هامان بن همداتای اجاجی که دشمن یهود بود داد.
\par 11 و پادشاه به هامان گفت: «هم نقره و هم قوم را به تودادم تا هرچه در نظرت پسند آید به ایشان بکنی.»
\par 12 پس کاتبان پادشاه را در روز سیزدهم ماه اول احضار نمودند و بر وفق آنچه هامان امرفرمود، به امیران پادشاه و به والیانی که بر هرولایت بودند و بر سروران هر قوم مرقوم شد. به هر ولایت، موافق خط آن و به هر قوم موافق زبانش، به اسم اخشورش پادشاه مکتوب گردید وبه مهر پادشاه مختوم شد.
\par 13 و مکتوبات به‌دست چاپاران به همه ولایتهای پادشاه فرستاده شد تاهمه یهودیان را از جوان و پیر و طفل و زن در یک روز، یعنی سیزدهم ماه دوازدهم که ماه آذارباشد، هلاک کنند و بکشند و تلف سازند و اموال ایشان را غارت کنند.
\par 14 و تا این حکم در همه ولایتها رسانیده شود، سوادهای مکتوب به همه قومها اعلان شد که در همان روز مستعد باشند.پس چاپاران بیرون رفتند و ایشان را برحسب فرمان پادشاه شتابانیدند و این حکم دردارالسلطنه شوشن نافذ شد و پادشاه و هامان به نوشیدن نشستند. اما شهر شوشن مشوش بود.
\par 15 پس چاپاران بیرون رفتند و ایشان را برحسب فرمان پادشاه شتابانیدند و این حکم دردارالسلطنه شوشن نافذ شد و پادشاه و هامان به نوشیدن نشستند. اما شهر شوشن مشوش بود.
 
\chapter{4}

\par 1 و چون مردخای از هرآنچه شده بود اطلاع یافت، مردخای جامه خود را دریده، پلاس با خاکستر در بر کرد و به میان شهر بیرون رفته، به آواز بلند فریاد تلخ برآورد.
\par 2 و تاروبروی دروازه پادشاه آمد، زیرا که جایز نبود که کسی با لباس پلاس داخل دروازه پادشاه بشود.
\par 3 و در هر ولایتی که امر و فرمان پادشاه به آن رسید، یهودیان را ماتم عظیمی و روزه و گریه ونوحه گری بود و بسیاری در پلاس و خاکسترخوابیدند. 
\par 4 پس کنیزان و خواجه‌سرایان استر آمده، اورا خبر دادند و ملکه بسیار محزون شد و لباس فرستاد تا مردخای را بپوشانند و پلاس او را ازوی بگیرند، اما او قبول نکرد.
\par 5 آنگاه استر، هتاک را که یکی از خواجه‌سرایان پادشاه بود و او را به جهت خدمت وی تعیین نموده بود، خواند و او راامر فرمود که از مردخای بپرسد که این چه امراست و سببش چیست.
\par 6 پس هتاک به سعه شهرکه پیش دروازه پادشاه بود، نزد مردخای بیرون رفت.
\par 7 و مردخای او را از هرچه به او واقع شده واز مبلغ نقره‌ای که هامان به جهت هلاک ساختن یهودیان وعده داده بود که آن را به خزانه پادشاه بدهد، خبر داد.
\par 8 و سواد نوشته فرمان را که درشوشن به جهت هلاکت ایشان صادر شده بود، به او داد تا آن را به استر نشان دهد و وی را مخبرسازد و وصیت نماید که نزد پادشاه داخل شده، ازاو التماس نماید و به جهت قوم خویش از وی درخواست کند.
\par 9 پس هتاک داخل شده، سخنان مردخای را به استر بازگفت.
\par 10 و استر هتاک را جواب داده، او راامر فرمود که به مردخای بگوید
\par 11 که «جمیع خادمان پادشاه و ساکنان ولایتهای پادشاه می‌دانند که به جهت هرکس، خواه مرد و خواه زن که نزد پادشاه به صحن اندرونی بی‌اذن داخل شود، فقط یک حکم است که کشته شود، مگرآنکه پادشاه چوگان زرین را بسوی او دراز کند تازنده بماند. و سی روز است که من خوانده نشده‌ام که به حضور پادشاه داخل شوم.»
\par 12 پس سخنان استر را به مردخای باز‌گفتند.
\par 13 و مردخای گفت به استر جواب دهید: «در دل خود فکر مکن که تو در خانه پادشاه به خلاف سایر یهود، رهایی خواهی یافت.
\par 14 بلکه اگر در این وقت تو ساکت بمانی، راحت و نجات برای یهود از جای دیگر پدید خواهد شد. اما تو وخاندان پدرت هلاک خواهید گشت و کیست بداند که به جهت چنین وقت به سلطنت نرسیده‌ای.»
\par 15 پس استر فرمود به مردخای جواب دهید
\par 16 که «برو و تمامی یهود را که در شوشن یافت می‌شوند جمع کن و برای من روزه گرفته، سه شبانه‌روز چیزی مخورید و میاشامید و من نیز باکنیزانم همچنین روزه خواهیم داشت. و به همین طور، نزد پادشاه داخل خواهم شد، اگر‌چه خلاف حکم است. و اگر هلاک شدم، هلاک شدم.»پس مردخای رفته، موافق هرچه استر وی را وصیت کرده بود، عمل نمود.
\par 17 پس مردخای رفته، موافق هرچه استر وی را وصیت کرده بود، عمل نمود.
 
\chapter{5}

\par 1 و در روز سوم، استر لباس ملوکانه پوشیده، به صحن دروازه اندرونی پادشاه، در مقابل خانه پادشاه بایستاد و پادشاه، بر کرسی خسروی خود در قصر سلطنت، روبروی دروازه خانه نشسته بود.
\par 2 و چون پادشاه، استر ملکه را دید که در صحن ایستاده است، او در نظر وی التفات یافت. و پادشاه چوگان طلا را که در دست داشت، به سوی استر دراز کرد و استر نزدیک آمده، نوک عصا را لمس کرد.
\par 3 و پادشاه او را گفت: «ای استر ملکه، تو را چه شده است و درخواست تو چیست؟ اگر‌چه نصف مملکت باشد، به تو داده خواهد شد.»
\par 4 استر جواب داد که «اگر به نظر پادشاه پسندآید، پادشاه با هامان امروز به ضیافتی که برای اومهیا کرده‌ام بیاید.»
\par 5 آنگاه پادشاه فرمود که «هامان را بشتابانید، تابرحسب کلام استر کرده شود.» پس پادشاه وهامان، به ضیافتی که استر برپا نموده بود آمدند.
\par 6 و پادشاه در مجلس شراب به استر گفت: «مسئول تو چیست که به تو داده خواهد شد ودرخواست تو کدام؟ اگرچه نصف مملکت باشد، برآورده خواهد شد.»
\par 7 استر در جواب گفت: «مسول و درخواست من این است،
\par 8 که اگر در نظر پادشاه التفات یافتم و اگر پادشاه مصلحت داند که مسول مرا عطافرماید و درخواست مرا بجا آورد، پادشاه وهامان به ضیافتی که به جهت ایشان مهیا می‌کنم بیایند و فردا امر پادشاه را بجا خواهم آورد.»
\par 9 پس در آن روز هامان شادمان و مسرور شده، بیرون رفت. لیکن چون هامان، مردخای را نزددروازه پادشاه دید که به حضور او برنمی خیزد وحرکت نمی کند، آنگاه هامان بر مردخای به شدت غضبناک شد.
\par 10 اما هامان خودداری نموده، به خانه خود رفت و فرستاده، دوستان خویش و زن خود زرش را خواند.
\par 11 و هامان برای ایشان، فراوانی توانگری خود و کثرت پسران خویش را و تمامی عظمتی را که پادشاه به او داده و او را بر سایر روسا و خدام پادشاه برتری داده بود، بیان کرد.
\par 12 و هامان گفت: «استر ملکه نیز کسی را سوای من به ضیافتی که برپا کرده بود، همراه پادشاه دعوت نفرمود و فردا نیز او مراهمراه پادشاه دعوت کرده است.
\par 13 لیکن همه این چیزها نزد من هیچ است، مادامی که مردخای یهود را می‌بینم که در دروازه پادشاه نشسته است.»آنگاه زوجه‌اش زرش و همه دوستانش اورا گفتند: «داری به بلندی پنجاه ذراع بسازند وبامدادان، به پادشاه عرض کن که مردخای را بر آن مصلوب سازند. پس با پادشاه با شادمانی به ضیافت برو.» و این سخن به نظر هامان پسند آمده، امر کرد تا دار را حاضر کردند.
\par 14 آنگاه زوجه‌اش زرش و همه دوستانش اورا گفتند: «داری به بلندی پنجاه ذراع بسازند وبامدادان، به پادشاه عرض کن که مردخای را بر آن مصلوب سازند. پس با پادشاه با شادمانی به ضیافت برو.» و این سخن به نظر هامان پسند آمده، امر کرد تا دار را حاضر کردند.
 
\chapter{6}

\par 1 در آن شب، خواب از پادشاه برفت و امرفرمود که کتاب تذکره تواریخ ایام رابیاورند تا آن را در حضور پادشاه بخوانند.
\par 2 و درآن، نوشته‌ای یافتند که مردخای درباره بغتان وترش خواجه‌سرایان پادشاه و حافظان آستانه وی که قصد دست درازی بر اخشورش پادشاه کرده بودند، خبر داده بود.
\par 3 و پادشاه پرسید که «چه حرمت و عزت به عوض این (خدمت ) به مردخای عطا شد؟» بندگان پادشاه که او را خدمت می‌کردند جواب دادند که «برای او چیزی نشد.»
\par 4 پادشاه گفت: «کیست در حیاط؟» (و هامان به حیاط بیرونی خانه پادشاه آمده بود تا به پادشاه عرض کند که مردخای را برداری که برایش حاضر ساخته بود مصلوب کنند. )
\par 5 و خادمان پادشاه وی را گفتند: «اینک هامان در حیاطایستاده است.» پادشاه فرمود تا داخل شود.
\par 6 و چون هامان داخل شد پادشاه وی را گفت: «با کسی‌که پادشاه رغبت دارد که او را تکریم نماید، چه باید کرد؟» و هامان در دل خود فکرکرد: «کیست غیر از من که پادشاه به تکریم نمودن او رغبت داشته باشد؟»
\par 7 پس هامان به پادشاه گفت: «برای شخصی که پادشاه به تکریم نمودن او رغبت دارد،
\par 8 لباس ملوکانه را که پادشاه می‌پوشد و اسبی را که پادشاه بر آن سوار می‌شودو تاج ملوکانه‌ای را که بر سر او نهاده می‌شود، بیاورند.
\par 9 و لباس و اسب را به‌دست یکی ازامرای مقرب ترین پادشاه بدهند و آن را به شخصی که پادشاه به تکریم نمودن او رغبت داردبپوشانند و بر اسب‌سوار کرده، و در کوچه های شهر بگردانند و پیش روی او ندا کنند که با کسی‌که پادشاه به تکریم نمودن او رغبت دارد، چنین کرده خواهد شد.»
\par 10 آنگاه پادشاه به هامان فرمود: «آن لباس واسب را چنانکه گفتی به تعجیل بگیر و با مردخای یهود که در دروازه پادشاه نشسته است، چنین معمول دار و از هرچه گفتی چیزی کم نشود.»
\par 11 پس هامان آن لباس و اسب را گرفت ومردخای را پوشانیده و او را سوار کرده، درکوچه های شهر گردانید و پیش روی او ندامی کرد که «با کسی‌که پادشاه به تکریم نمودن اورغبت دارد چنین کرده خواهد شد.»
\par 12 ومردخای به دروازه پادشاه مراجعت کرد. اماهامان ماتم‌کنان و سرپوشیده، به خانه خودبشتافت.
\par 13 و هامان به زوجه خود زرش و همه دوستان خویش، ماجرای خود را حکایت نمود وحکیمانش و زنش زرش او را گفتند: «اگر این مردخای که پیش وی آغاز افتادن نمودی از نسل یهود باشد، بر او غالب نخواهی آمد، بلکه البته پیش او خواهی افتاد.»و ایشان هنوز با اوگفتگو می‌کردند که خواجه‌سرایان پادشاه رسیدند تا هامان را به ضیافتی که استر مهیا ساخته بود، به تعجیل ببرند.
\par 14 و ایشان هنوز با اوگفتگو می‌کردند که خواجه‌سرایان پادشاه رسیدند تا هامان را به ضیافتی که استر مهیا ساخته بود، به تعجیل ببرند.
 
\chapter{7}

\par 1 پس پادشاه و هامان نزد استر ملکه به ضیافت حاضر شدند.
\par 2 و پادشاه در روزدوم نیز در مجلس شراب به استر گفت: «ای استرملکه، مسول تو چیست که به تو داده خواهد شد ودرخواست تو کدام؟ اگر‌چه نصف مملکت باشدبجا آورده خواهد شد.»
\par 3 استر ملکه جواب داد و گفت: «ای پادشاه، اگر در نظر تو التفات یافته باشم و اگر پادشاه راپسند آید، جان من به مسول من و قوم من به درخواست من، به من بخشیده شود.
\par 4 زیرا که من و قومم فروخته شده‌ایم که هلاک و نابود و تلف شویم و اگر به غلامی و کنیزی فروخته می‌شدیم، سکوت می‌نمودم با آنکه مصیبت ما نسبت به ضرر پادشاه هیچ است.»
\par 5 آنگاه اخشورش پادشاه، استر ملکه راخطاب کرده، گفت: «آن کیست و کجا است که جسارت نموده است تا چنین عمل نماید؟»
\par 6 استر گفت: «عدو و دشمن، همین هامان شریراست.» آنگاه هامان در حضور پادشاه و ملکه به لرزه درآمد.
\par 7 و پادشاه غضبناک شده، از مجلس شراب برخاسته، به باغ قصر رفت. و چون هامان دید که بلا از جانب پادشاه برایش مهیا است برپا شد تانزد استر ملکه برای جان خود تضرع نماید.
\par 8 وچون پادشاه از باغ قصر به‌جای مجلس شراب برگشت، هامان بر بستری که استر بر آن می‌بودافتاده بود، پس پادشاه گفت: «آیا ملکه را نیز به حضور من در خانه بی‌عصمت می‌کند؟» سخن هنوز بر زبان پادشاه می‌بود که روی هامان راپوشانیدند.
\par 9 آنگاه حربونا، یکی ازخواجه‌سرایانی که در حضور پادشاه می‌بودند، گفت: «اینک دار پنجاه ذراعی نیز که هامان آن رابه جهت مردخای که آن سخن نیکو را برای پادشاه گفته است مهیا نموده، در خانه هامان حاضر است.» پادشاه فرمود که «او را بر آن مصلوب سازید.»پس هامان را بر داری که برای مردخای مهیا کرده بود، مصلوب ساختند و غضب پادشاه فرو نشست.
\par 10 پس هامان را بر داری که برای مردخای مهیا کرده بود، مصلوب ساختند و غضب پادشاه فرو نشست.
 
\chapter{8}

\par 1 در آنروز اخشورش پادشاه، خانه هامان، دشمن یهود را به استر ملکه ارزانی داشت. و مردخای در حضور پادشاه داخل شد، زیرا که استر او را از نسبتی که با وی داشت خبر داده بود.
\par 2 و پادشاه انگشتر خود را که از هامان گرفته بودبیرون کرده، به مردخای داد و استر مردخای را برخانه هامان گماشت.
\par 3 و استر بار دیگر به پادشاه عرض کرد و نزد پایهای او افتاده، بگریست و از اوالتماس نمود که شر هامان اجاجی و تدبیری را که برای یهودیان کرده بود، باطل سازد.
\par 4 پس پادشاه چوگان طلا را بسوی استر دراز کرد و استربرخاسته، به حضور پادشاه ایستاد
\par 5 و گفت: «اگرپادشاه را پسند آید و من در حضور او التفات یافته باشم و پادشاه این امر را صواب بیند و اگر من منظور نظر او باشم، مکتوبی نوشته شود که آن مراسله را که هامان بن همداتای اجاجی تدبیرکرده و آنها را برای هلاکت یهودیانی که در همه ولایتهای پادشاه می‌باشند نوشته است، باطل سازد.
\par 6 زیرا که من بلایی را که بر قومم واقع می‌شود چگونه توانم دید؟ و هلاکت خویشان خود را چگونه توانم نگریست؟»
\par 7 آنگاه اخشورش پادشاه به استر ملکه و مردخای یهودی فرمود: «اینک خانه هامان را به استر بخشیدم و او را به‌سبب دست درازی به یهودیان به دار کشیده‌اند.
\par 8 و شما آنچه را که درنظرتان پسند آید، به اسم پادشاه به یهودیان بنویسید و آن را به مهر پادشاه مختوم سازید، زیراهرچه به اسم پادشاه نوشته شود و به مهر پادشاه مختوم گردد کسی نمی تواند آن را تبدیل نماید.»
\par 9 پس در آن ساعت، در روز بیست و سوم ماه سوم که ماه سیوان باشد، کاتبان پادشاه را احضارکردند و موافق هر‌آنچه مردخای امر فرمود، به یهودیان و امیران و والیان و روسای ولایتها یعنی صد و بیست و هفت ولایت که از هند تا حبش بودنوشتند. به هر ولایت، موافق خط آن و به هر قوم، موافق زبان آن و به یهودیان، موافق خط و زبان ایشان.
\par 10 و مکتوبات را به اسم اخشورش پادشاه نوشت و به مهر پادشاه مختوم ساخته، آنها را به‌دست چاپاران اسب‌سوار فرستاد و ایشان براسبان تازی که مختص خدمت پادشاه و کره های مادیانهای او بودند، سوار شدند. 
\par 11 و در آنهاپادشاه به یهودیانی که در همه شهرها بودند، اجازت داد که جمع شده، به جهت جانهای خودمقاومت نمایند و تمامی قوت قومها و ولایتها راکه قصد اذیت ایشان می‌داشتند، با اطفال و زنان ایشان هلاک سازند و بکشند و تلف نمایند واموال ایشان را تاراج کنند.
\par 12 در یک روز یعنی در سیزدهم ماه دوازدهم که ماه آذار باشد در همه ولایتهای اخشورش پادشاه.
\par 13 و تا این حکم درهمه ولایتها رسانیده شود، سوادهای مکتوب به همه قومها اعلان شد که در همان روز یهودیان مستعد باشند تا از دشمنان خود انتقام بگیرند.
\par 14 پس چاپاران بر اسبان تازی که مختص خدمت پادشاه بود، روانه شدند و ایشان را بر حسب حکم پادشاه شتابانیده، به تعجیل روانه ساختند و حکم، در دارالسلطنه شوشن نافذ شد.
\par 15 و مردخای از حضور پادشاه با لباس ملوکانه لاجوردی و سفید و تاج بزرگ زرین و ردای کتان نازک ارغوانی بیرون رفت و شهر شوشن شادی ووجد نمودند،
\par 16 و برای یهودیان، روشنی وشادی و سرور و حرمت پدید آمد.و در همه ولایتها و جمیع شهرها در هر جایی که حکم وفرمان پادشاه رسید، برای یهودیان، شادمانی وسرور و بزم و روز خوش بود و بسیاری ازقوم های زمین به دین یهود گرویدند زیرا که ترس یهودیان بر ایشان مستولی گردیده بود.
\par 17 و در همه ولایتها و جمیع شهرها در هر جایی که حکم وفرمان پادشاه رسید، برای یهودیان، شادمانی وسرور و بزم و روز خوش بود و بسیاری ازقوم های زمین به دین یهود گرویدند زیرا که ترس یهودیان بر ایشان مستولی گردیده بود.
 
\chapter{9}

\par 1 و در روز سیزدهم ماه دوازدهم که ماه آذارباشد، هنگامی که نزدیک شد که حکم وفرمان پادشاه را جاری سازند و دشمنان یهودمنتظر می‌بودند که بر ایشان استیلا یابند، این همه برعکس شد که یهودیان بر دشمنان خویش استیلا یافتند.
\par 2 و یهودیان در شهرهای خود درهمه ولایتهای اخشورش پادشاه جمع شدند تا برآنانی که قصد اذیت ایشان داشتند، دست بیندازندو کسی با ایشان مقاومت ننمود زیرا که ترس ایشان بر همه قومها مستولی شده بود.
\par 3 و جمیع روسای ولایتها و امیران و والیان و عاملان پادشاه، یهودیان را اعانت کردند زیرا که ترس مردخای برایشان مستولی شده بود،
\par 4 چونکه مردخای درخانه پادشاه معظم شده بود و آوازه او در جمیع ولایتها شایع گردیده و این مردخای آن فان بزرگتر می‌شد.
\par 5 پس یهودیان جمیع دشمنان خود را به دم شمشیر زده، کشتند و هلاک کردند و با ایشان هرچه خواستند، به عمل آوردند.
\par 6 و یهودیان دردارالسلطنه شوشن پانصد نفر را به قتل رسانیده، هلاک کردند.
\par 7 و فرشنداطا و دلفون و اسفاتا،
\par 8 وفوراتا و ادلیا و اریداتا،
\par 9 و فرمشتا و اریسای واریدای و یزاتا،
\par 10 یعنی ده پسر هامان بن همداتای، دشمن یهود را کشتند، لیکن دست خود را به تاراج نگشادند.
\par 11 در آن روز، عدد آنانی را که در دارالسلطنه شوشن کشته شدند به حضور پادشاه عرضه داشتند.
\par 12 و پادشاه به استر ملکه گفت که «یهودیان در دارالسلطنه شوشن پانصد نفر و ده پسر هامان را کشته و هلاک کرده‌اند. پس در سایرولایتهای پادشاه چه کرده‌اند؟ حال مسول توچیست که به تو داده خواهد شد و دیگر‌چه درخواست داری که برآورده خواهد گردید؟»
\par 13 استر گفت: «اگر پادشاه را پسند آید به یهودیانی که در شوشن می‌باشند، اجازت داده شود که فردا نیز مثل فرمان امروز عمل نمایند وده پسر هامان را بردار بیاویزند.»
\par 14 و پادشاه فرمود که چنین بشود و حکم در شوشن نافذگردید و ده پسر هامان را به دار آویختند.
\par 15 ویهودیانی که در شوشن بودند، در روز چهاردهم ماه آذار نیز جمع شده، سیصد نفر را در شوشن کشتند، لیکن دست خود را به تاراج نگشادند.
\par 16 وسایر یهودیانی که در ولایتهای پادشاه بودندجمع شده، برای جانهای خود مقاومت نمودند وچون هفتاد و هفت هزار نفر از مبغضان خویش راکشته بودند، از دشمنان خود آرامی یافتند. امادست خود را به تاراج نگشادند.
\par 17 این، در روز سیزدهم ماه آذار (واقع شد) ودر روز چهاردهم ماه، آرامی یافتند و آن را روزبزم و شادمانی نگاه داشتند.
\par 18 و یهودیانی که در شوشن بودند، در سیزدهم و چهاردهم آن ماه جمع شدند و در روز پانزدهم ماه آرامی یافتند وآن را روز بزم و شادمانی نگاه داشتند.
\par 19 بنابراین، یهودیان دهاتی که در دهات بی‌حصار ساکنند، روز چهاردهم ماه آذار را روز شادمانی و بزم وروز خوش نگاه می‌دارند و هدایا برای یکدیگرمی فرستند.
\par 20 و مردخای این مطالب را نوشته، مکتوبات را نزد تمامی یهودیانی که در همه ولایتهای اخشورش پادشاه بودند، از نزدیک و دور فرستاد،
\par 21 تا بر ایشان فریضه‌ای بگذارد که روز چهاردهم و روز پانزدهم ماه آذار را سال به سال عید نگاه دارند.
\par 22 چونکه در آن روزها، یهودیان ازدشمنان خود آرامی یافتند و در آن ماه، غم ایشان به شادی و ماتم ایشان به روز خوش مبدل گردید. لهذا آنها را روزهای بزم و شادی نگاه بدارند وهدایا برای یکدیگر و بخششها برای فقیران بفرستند.
\par 23 پس یهودیان آنچه را که خود به عمل نمودن آن شروع کرده بودند و آنچه را که مردخای به ایشان نوشته بود، بر خود فریضه ساختند.
\par 24 زیرا که هامان بن همداتای اجاجی، دشمن تمامی یهود، قصد هلاک نمودن یهودیان کرده و فور یعنی قرعه برای هلاکت و تلف نمودن ایشان انداخته بود.
\par 25 اما چون این امر به سمع پادشاه رسید، مکتوب حکم داد که قصد بدی که برای یهود اندیشیده بود، بر سر خودش برگردانیده شود و او را با پسرانش بر دار کشیدند.
\par 26 از این جهت آن روزها را از اسم فور، فوریم نامیدند، و موافق تمامی مطلب این مکتوبات و آنچه خود ایشان در این امر دیده بودند و آنچه برایشان وارد آمده بود،
\par 27 یهودیان این را فریضه ساختند و آن را بر ذمه خود و ذریت خویش وهمه کسانی که به ایشان ملصق شوند، گرفتند که تبدیل نشود و آن دو روز را برحسب کتابت آنها وزمان معین آنها سال به سال نگاه دارند.
\par 28 و آن روزها را در همه طبقات و قبایل وولایتها و شهرها بیاد آورند و نگاه دارند و این‌روزهای فوریم، از میان یهود منسوخ نشود ویادگاری آنها از ذریت ایشان نابود نگردد.
\par 29 واستر ملکه، دختر ابیحایل و مردخای یهودی، به اقتدار تمام نوشتند تا این مراسله دوم را درباره فوریم برقرار نمایند.
\par 30 و مکتوبات، مشتمل برسخنان سلامتی و امنیت نزد جمیع یهودیانی که در صد و بیست و هفت ولایت مملکت اخشورش بودند، فرستاد،
\par 31 تا این دو روز فوریم را در زمان معین آنها فریضه قرار دهند، چنانکه مردخای یهودی و استر ملکه بر ایشان فریضه قرار دادند و ایشان آن را بر ذمه خود و ذریت خویش گرفتند، به یادگاری ایام روزه و تضرع ایشان.پس سنن این فوریم، به فرمان استر فریضه شد و در کتاب مرقوم گردید.
\par 32 پس سنن این فوریم، به فرمان استر فریضه شد و در کتاب مرقوم گردید.
 
\chapter{10}

\par 1 و اخشورش پادشاه بر زمینها و جزایردریا جزیه گذارد،و جمیع اعمال قوت و توانایی او و تفصیل عظمت مردخای که چگونه پادشاه او را معظم ساخت، آیا در کتاب تواریخ ایام پادشاهان مادی و فارس مکتوب نیست؟
\par 2 و جمیع اعمال قوت و توانایی او و تفصیل عظمت مردخای که چگونه پادشاه او را معظم ساخت، آیا در کتاب تواریخ ایام پادشاهان مادی و فارس مکتوب نیست؟


\end{document}