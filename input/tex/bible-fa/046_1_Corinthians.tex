\begin{document}

\title{اول قرنتيان}


\chapter{1}

\par 1 پولس به اراده خدا رسول خوانده شده عیسی مسیح و سوستانیس برادر،
\par 2 به کلیسای خدا که در قرنتس است، ازمقدسین در مسیح عیسی که برای تقدس خوانده شده‌اند، با همه کسانی که در هرجا نام خداوند ماعیسی مسیح را می‌خوانند که (خداوند) ما و(خداوند) ایشان است.
\par 3 فیض و سلامتی ازجانب پدر ما خدا و عیسی مسیح خداوند بر شماباد.
\par 4 خدای خود را پیوسته شکر می‌کنم درباره شما برای آن فیض خدا که در مسیح عیسی به شما عطا شده است،
\par 5 زیرا شما از هرچیز دروی دولتمند شده‌اید، در هر کلام و در هرمعرفت.
\par 6 چنانکه شهادت مسیح در شما استوارگردید،
\par 7 بحدی که در هیچ بخشش ناقص نیستیدو منتظر مکاشفه خداوند ما عیسی مسیح می‌باشید.
\par 8 که او نیز شما را تا آخر استوار خواهدفرمود تا در روز خداوند ما عیسی مسیح بی‌ملامت باشید.
\par 9 امین است خدایی که شما را به شراکت پسر خود عیسی مسیح خداوند ماخوانده است.
\par 10 لکن‌ای برادران از شما استدعا دارم به نام خداوند ما عیسی مسیح که همه یک سخن گوییدو شقاق در میان شما نباشد، بلکه در یک فکر ویک رای کامل شوید.
\par 11 زیرا که‌ای برادران من، از اهل خانه خلوئی درباره شما خبر به من رسیدکه نزاعها در میان شما پیدا شده است.
\par 12 غرض اینکه هریکی از شما می‌گوید که من از پولس هستم، و من از اپلس، و من از کیفا، و من از مسیح.
\par 13 آیا مسیح منقسم شد؟ یا پولس در راه شمامصلوب گردید؟ یا به نام پولس تعمید یافتید؟
\par 14 خدا را شکر می‌کنم که هیچ‌یکی از شما راتعمید ندادم جز کرسپس و قایوس،
\par 15 که مباداکسی گوید که به نام خود تعمید دادم.
\par 16 وخاندان استیفان را نیز تعمید دادم و دیگر یادندارم که کسی را تعمید داده باشم.
\par 17 زیرا که مسیح مرا فرستاد، نه تا تعمید دهم بلکه تا بشارت رسانم، نه به حکمت کلام مبادا صلیب مسیح باطل شود.
\par 18 زیرا ذکر صلیب برای هالکان حماقت است، لکن نزد ما که ناجیان هستیم قوت خداست.
\par 19 زیرا مکتوب است: «حکمت حکما را باطل سازم و فهم فهیمان را نابود گردانم.»
\par 20 کجا است حکیم؟ کجا کاتب؟ کجا مباحث این دنیا؟ مگرخدا حکمت جهان را جهالت نگردانیده است؟
\par 21 زیرا که چون برحسب حکمت خدا، جهان ازحکمت خود به معرفت خدا نرسید، خدا بدین رضا داد که بوسیله جهالت موعظه، ایمانداران رانجات‌بخشد.
\par 22 چونکه یهود آیتی می‌خواهند ویونانیان طالب حکمت هستند.
\par 23 لکن ما به مسیح مصلوب وعظ می‌کنیم که یهود را لغزش و امت هارا جهالت است.
\par 24 لکن دعوت‌شدگان را خواه یهود و خواه یونانی مسیح قوت خدا و حکمت خدا است.
\par 25 زیرا که جهالت خدا از انسان حکیمتر است و ناتوانی خدا از مردم، تواناتر.
\par 26 زیرا‌ای برادران دعوت خود را ملاحظه نمایید که بسیاری بحسب جسم حکیم نیستند وبسیاری توانا نی و بسیاری شریف نی.
\par 27 بلکه خدا جهال جهان را برگزید تا حکما را رسوا سازدو خدا ناتوانان عالم را برگزید تا توانایان را رسواسازد،
\par 28 و خسیسان دنیا و محقران را خدابرگزید، بلکه نیستیها را تا هستیها را باطل گرداند.
\par 29 تا هیچ بشری در حضور او فخر نکند.
\par 30 لکن از او شما هستید در عیسی مسیح که از جانب خدابرای شما حکمت شده است و عدالت قدوسیت و فدا.تا چنانکه مکتوب است هر‌که فخر کنددر خداوند فخر نماید.
\par 31 تا چنانکه مکتوب است هر‌که فخر کنددر خداوند فخر نماید.

\chapter{2}

\par 1 و من‌ای برادران، چون به نزد شما آمدم، با فضیلت کلام یا حکمت نیامدم چون شمارا به‌سر خدا اعلام می‌نمودم.
\par 2 زیرا عزیمت نکردم که چیزی در میان شما دانسته باشم جزعیسی مسیح و او را مصلوب.
\par 3 و من در ضعف وترس و لرزش بسیار نزد شما شدم،
\par 4 و کلام ووعظ من به سخنان مقنع حکمت نبود، بلکه به برهان روح و قوت،
\par 5 تا ایمان شما در حکمت انسان نباشد بلکه در قوت خدا.
\par 6 لکن حکمتی بیان می‌کنیم نزد کاملین، اماحکمتی که از این عالم نیست و نه از روسای این عالم که زایل می‌گردند.
\par 7 بلکه حکمت خدا را درسری بیان می‌کنیم، یعنی آن حکمت مخفی را که خدا پیش از دهرها برای جلال ما مقدر فرمود،
\par 8 که احدی از روسای این عالم آن را ندانست زیرا اگر می‌دانستند خداوند جلال را مصلوب نمی کردند.
\par 9 بلکه چنانکه مکتوب است: «چیزهایی را که چشمی ندید و گوشی نشنید و به‌خاطر انسانی خطور نکرد، یعنی آنچه خدا برای دوستداران خود مهیا کرده است.»
\par 10 اما خدا آنهارا به روح خود بر ما کشف نموده است، زیرا که روح همه‌چیز حتی عمقهای خدا را نیز تفحص می‌کند.
\par 11 زیرا کیست از مردمان که امور انسان رابداند جز روح انسان که در وی می‌باشد. همچنین نیز امور خدا را هیچ‌کس ندانسته است، جز روح خدا.
\par 12 لیکن ما روح جهان را نیافته‌ایم، بلکه آن روح که از خداست تا آنچه خدا به ما عطا فرموده است بدانیم.
\par 13 که آنها را نیز بیان می‌کنیم نه به سخنان آموخته شده از حکمت انسان، بلکه به آنچه روح‌القدس می‌آموزد و روحانیها را باروحانیها جمع می‌نماییم.
\par 14 اما انسان نفسانی‌امور روح خدا را نمی پذیرد زیرا که نزد او جهالت است و آنها را نمی تواند فهمید زیرا حکم آنها ازروح می‌شود.
\par 15 لکن شخص روحانی در همه‌چیز حکم می‌کند و کسی را در او حکم نیست.«زیرا کیست که فکر خداوند را دانسته باشد تااو را تعلیم دهد؟» لکن ما فکر مسیح را داریم.
\par 16 «زیرا کیست که فکر خداوند را دانسته باشد تااو را تعلیم دهد؟» لکن ما فکر مسیح را داریم.

\chapter{3}

\par 1 و من‌ای برادران نتوانستم به شما سخن گویم چون روحانیان، بلکه چون جسمانیان و چون اطفال در مسیح.
\par 2 و شما را به شیر خوراک دادم نه به گوشت زیرا که هنوزاستطاعت آن نداشتید بلکه الحال نیز ندارید،
\par 3 زیرا که تا به حال جسمانی هستید، چون در میان شما حسد و نزاع و جدایی‌ها است. آیا جسمانی نیستید و به طریق انسان رفتار نمی نمایید؟
\par 4 زیراچون یکی گوید من از پولس و دیگری من از اپلس هستم، آیا انسان نیستید؟
\par 5 پس کیست پولس و کیست اپلس؟ جزخادمانی که بواسطه ایشان ایمان آوردید و به اندازه‌ای که خداوند به هرکس داد.
\par 6 من کاشتم واپلس آبیاری کرد لکن خدا نمو می‌بخشید.
\par 7 لهذانه کارنده چیزی است و نه آب دهنده بلکه خدای رویاننده.
\par 8 و کارنده و سیرآب کننده‌یک هستند، لکن هر یک اجرت خود را بحسب مشقت خودخواهند یافت.
\par 9 زیرا با خدا همکاران هستیم وشما زراعت خدا و عمارت خدا هستید.
\par 10 بحسب فیض خدا که به من عطا شد، چون معمار دانا بنیاد نهادم و دیگری بر آن عمارت می‌سازد؛ لکن هرکس با‌خبر باشد که چگونه عمارت می‌کند.
\par 11 زیرا بنیادی دیگر هیچ‌کس نمی تواند نهاد جز آنکه نهاده شده است، یعنی عیسی مسیح.
\par 12 لکن اگر کسی بر آن بنیاد، عمارتی از طلا یا نقره یا جواهر یا چوب یا گیاه یاکاه بنا کند،
\par 13 کار هرکس آشکار خواهد شد، زیرا که آن روز آن را ظاهر خواهد نمود، چونکه آن به آتش به ظهور خواهد رسید و خود آتش، عمل هرکس را خواهد آزمود که چگونه است.
\par 14 اگر کاری که کسی بر آن گذارده باشد بماند، اجر خواهد یافت.
\par 15 و اگر عمل کسی سوخته شود، زیان بدو وارد آید، هرچند خود نجات یابداما چنانکه از میان آتش.
\par 16 آیا نمی دانید که هیکل خدا هستید و روح خدا در شما ساکن است؟
\par 17 اگر کسی هیکل خدارا خراب کند، خدا او را هلاک سازد زیرا هیکل خدا مقدس است و شما آن هستید.
\par 18 زنهارکسی خود را فریب ندهد! اگر کسی از شما خودرا در این جهان حکیم پندارد، جاهل بشود تاحکیم گردد.
\par 19 زیرا حکمت این جهان نزد خداجهالت است، چنانکه مکتوب است: «حکما را به مکر خودشان گرفتار می‌سازد.»
\par 20 و ایض: «خداوند افکار حکما را می‌داند که باطل است.»
\par 21 پس هیچ‌کس در انسان فخر نکند، زیراهمه‌چیز از آن شما است:
\par 22 خواه پولس، خواه اپلس، خواه کیفا، خواه دنیا، خواه زندگی، خواه موت، خواه چیزهای حال، خواه چیزهای آینده، همه از آن شما است،و شما از مسیح و مسیح از خدا می‌باشد.
\par 23 و شما از مسیح و مسیح از خدا می‌باشد.

\chapter{4}

\par 1 هرکس ما را چون خدام مسیح و وکلای اسرار خدا بشمارد.
\par 2 و دیگر در وکلابازپرس می‌شود که هر یکی امین باشد.
\par 3 امابجهت من کمتر چیزی است که از شما یا از یوم بشر حکم کرده شوم، بلکه برخود نیز حکم نمی کنم.
\par 4 زیرا که در خود عیبی نمی بینم، لکن ازاین عادل شمرده نمی شوم، بلکه حکم کننده من خداوند است.
\par 5 لهذا پیش از وقت به چیزی حکم مکنید تا خداوند بیاید که خفایای ظلمت راروشن خواهد کرد و نیتهای دلها را به ظهورخواهد آورد؛ آنگاه هرکس را مدح از خدا خواهدبود.
\par 6 اما‌ای برادران، این چیزها را بطور مثل به خود و اپلس نسبت دادم به‌خاطر شما تا درباره ماآموخته شوید که از آنچه مکتوب است تجاوزنکنید و تا هیچ‌یکی از شما تکبر نکند برای یکی بر دیگری.
\par 7 زیرا کیست که تو را برتری داد وچه چیز داری که نیافتی؟ پس چون یافتی، چرافخر می‌کنی که گویا نیافتی.
\par 8 الحال سیر شده و دولتمند گشته‌اید و بدون ما سلطنت می‌کنید؛ و کاشکه سلطنت می‌کردیدتا ما نیز با شما سلطنت می‌کردیم.
\par 9 زیرا گمان می‌برم که خدا ما رسولان را آخر همه عرضه داشت مثل آنانی که فتوای موت بر ایشان شده است، زیرا که جهان و فرشتگان و مردم راتماشاگاه شده‌ایم.
\par 10 ما به‌خاطر مسیح جاهل هستیم، لکن شما در مسیح دانا هستید؛ ما ضعیف لکن شما توانا؛ شما عزیز اما ما ذلیل.
\par 11 تا به همین ساعت گرسنه و تشنه و عریان و کوبیده وآواره هستیم،
\par 12 و به‌دستهای خود کار کرده، مشقت می‌کشیم و دشنام شنیده، برکت می‌طلبیم و مظلوم گردیده، صبر می‌کنیم.
\par 13 چون افترا بر مامی زنند، نصیحت می‌کنیم و مثل قاذورات دنیا وفضلات همه‌چیز شده‌ایم تا به حال.
\par 14 و این را نمی نویسم تا شما را شرمنده سازم بلکه چون فرزندان محبوب خود تنبیه می‌کنم.
\par 15 زیرا هرچند هزاران استاد در مسیح داشته باشید، لکن پدران بسیار ندارید، زیرا که من شما رادر مسیح عیسی به انجیل تولید نمودم،
\par 16 پس ازشما التماس می‌کنم که به من اقتدا نمایید.
\par 17 برای همین تیموتاوس را نزد شما فرستادم که اوست فرزند محبوب من و امین در خداوند تا راههای مرا در مسیح به یاد شما بیاورد، چنانکه در هرجاو در هرکلیسا تعلیم می‌دهم.
\par 18 اما بعضی تکبرمی کنند به گمان آنکه من نزد شما نمی آیم.
\par 19 لکن به زودی نزد شما خواهم آمد، اگرخداوند بخواهد و خواهم دانست نه سخن متکبران را بلکه قوت ایشان را.
\par 20 زیرا ملکوت خدا به زبان نیست بلکه در قوت است.چه خواهش دارید آیا با چوب نزد شما بیایم یا بامحبت و روح حلم؟
\par 21 چه خواهش دارید آیا با چوب نزد شما بیایم یا بامحبت و روح حلم؟

\chapter{5}

\par 1 فی الحقیقه شنیده می‌شود که در میان شمازنا پیدا شده است، و چنان زنایی که درمیان امت‌ها هم نیست؛ که شخصی زن پدر خود را داشته باشد.
\par 2 و شما فخر می‌کنید بلکه ماتم هم ندارید، چنانکه باید تا آن کسی‌که این عمل را کرداز میان شما بیرون شود.
\par 3 زیرا که من هرچند درجسم غایبم، اما در روح حاضرم؛ و الان چون حاضر، حکم کردم در حق کسی‌که این را چنین کرده است.
\par 4 به نام خداوند ما عیسی مسیح، هنگامی که شما با روح من با قوت خداوند ماعیسی مسیح جمع شوید،
\par 5 که چنین شخص به شیطان سپرده شود بجهت هلاکت جسم تا روح در روز خداوند عیسی نجات یابد.
\par 6 فخر شما نیکو نیست. آیا آگاه نیستید که اندک خمیرمایه، تمام خمیر را مخمر می‌سازد؟
\par 7 پس خود را از خمیرمایه کهنه پاک سازید تافطیر تازه باشید، چنانکه بی‌خمیرمایه هستید زیراکه فصح ما مسیح در راه ما ذبح شده است.
\par 8 پس عید را نگاه داریم نه به خمیرمایه کهنه و نه به خمیرمایه بدی و شرارت، بلکه به فطیر ساده دلی و راستی.
\par 9 در آن رساله به شما نوشتم که با زانیان معاشرت نکنید.
\par 10 لکن نه مطلق با زانیان این جهان یا طمعکاران و یا ستمکاران یا بت‌پرستان، که در این صورت می‌باید از دنیا بیرون شوید.
\par 11 لکن الان به شما می‌نویسم که اگر کسی‌که به برادر نامیده می‌شود، زانی یا طماع یا بت‌پرست یافحاش یا میگسار یا ستمگر باشد، با چنین شخص معاشرت مکنید بلکه غذا هم مخورید.
\par 12 زیرا مرا چه‌کار است که بر آنانی که خارج اندداوری کنم. آیا شما بر اهل داخل داوری نمی کنید؟لکن آنانی را که خارج‌اند خدا داوری خواهد کرد. پس آن شریر را از میان خودبرانید.
\par 13 لکن آنانی را که خارج‌اند خدا داوری خواهد کرد. پس آن شریر را از میان خودبرانید.

\chapter{6}

\par 1 آیا کسی از شما چون بر دیگری مدعی باشد، جرات دارد که مرافعه برد پیش ظالمان نه نزد مقدسان؟
\par 2 یا نمی دانید که مقدسان، دنیا را داوری خواهند کرد؛ و اگر دنیا از شما حکم یابد، آیا قابل مقدمات کمتر نیستید؟
\par 3 آیانمی دانید که فرشتگان را داوری خواهیم کرد تاچه رسد به امور روزگار؟
\par 4 پس چون در مقدمات روزگار مرافعه دارید، آیا آنانی را که در کلیساحقیر شمرده می‌شوند، می‌نشانید؟
\par 5 بجهت انفعال شما می‌گویم، آیا در میان شما یک نفر دانانیست که بتواند در میان برادران خود حکم کند؟
\par 6 بلکه برادر با برادر به محاکمه می‌رود و آن هم نزد بی‌ایمانان!
\par 7 بلکه الان شما را بالکلیه قصوری است که بایکدیگر مرافعه دارید. چرا بیشتر مظلوم نمی شوید و چرا بیشتر مغبون نمی شوید؟
\par 8 بلکه شما ظلم می‌کنید و مغبون می‌سازید و این را نیزبه برادران خود!
\par 9 آیا نمی دانید که ظالمان وارث ملکوت خدا نمی شوند؟ فریب مخورید، زیرافاسقان و بت‌پرستان و زانیان و متنعمان و لواط
\par 10 و دزدان و طمعکاران و میگساران و فحاشان وستمگران وارث ملکوت خدا نخواهند شد.
\par 11 وبعضی از شما چنین می‌بودید لکن غسل یافته ومقدس گردیده و عادل کرده شده‌اید به نام عیسی خداوند و به روح خدای ما.
\par 12 همه‌چیز برای من جایز است لکن هرچیزمفید نیست. همه‌چیز برای من رواست، لیکن نمی گذارم که چیزی بر من تسلط یابد.
\par 13 خوراک برای شکم است و شکم برای خوراک، لکن خدااین و آن را فانی خواهد ساخت. اما جسم برای زنانیست، بلکه برای خداوند است و خداوند برای جسم.
\par 14 و خدا خداوند را برخیزانید و ما را نیز به قوت خود خواهد برخیزانید.
\par 15 آیا نمی دانید که بدنهای شما اعضای مسیح است؟ پس آیااعضای مسیح را برداشته، اعضای فاحشه گردانم؟ حاشا!
\par 16 آیا نمی دانید که هرکه با فاحشه پیوندد، با وی یکتن باشد؟ زیرا می‌گوید «هردویک تن خواهند بود».
\par 17 لکن کسی‌که با خداوندپیوندد یکروح است.
\par 18 از زنا بگریزید. هر گناهی که آدمی می‌کند بیرون از بدن است، لکن زانی بربدن خود گناه می‌ورزد.
\par 19 یا نمی دانید که بدن شما هیکل روح‌القدس است که در شما است که از خدا یافته‌اید و از آن خود نیستید؟زیرا که به قیمتی خریده شدید، پس خدا را به بدن خودتمجید نمایید.
\par 20 زیرا که به قیمتی خریده شدید، پس خدا را به بدن خودتمجید نمایید.

\chapter{7}

\par 1 اما درباره آنچه به من نوشته بودید: مرد رانیکو آن است که زن را لمس نکند.
\par 2 لکن بسبب زنا، هر مرد زوجه خود را بدارد و هر زن شوهر خود را بدارد.
\par 3 و شوهر حق زن را ادانماید و همچنین زن حق شوهر را.
\par 4 زن بر بدن خود مختار نیست بلکه شوهرش، و همچنین مرد نیز اختیار بدن خود ندارد بلکه زنش،
\par 5 از یکدیگرجدایی مگزینید مگر مدتی به رضای طرفین تابرای روزه و عبادت فارغ باشید؛ و باز با هم پیوندید مبادا شیطان شما را به‌سبب ناپرهیزی شما در تجربه اندازد،
\par 6 لکن این را می‌گویم به طریق اجازه نه به طریق حکم.
\par 7 اما می‌خواهم که همه مردم مثل خودم باشند. لکن هرکس نعمتی خاص از خدا دارد، یکی چنین و دیگری چنان.
\par 8 لکن به مجردین و بیوه‌زنان می‌گویم که ایشان را نیکو است که مثل من بمانند.
\par 9 لکن اگر پرهیزندارند، نکاح بکنند زیرا که نکاح از آتش هوس بهتر است.
\par 10 اما منکوحان را حکم می‌کنم و نه من بلکه خداوند که زن از شوهر خود جدا نشود؛
\par 11 و اگر جدا شود، مجرد بماند یا با شوهر خودصلح کند؛ و مرد نیز زن خود را جدا نسازد.
\par 12 و دیگران را من می‌گویم نه خداوند که اگرکسی از برادران زنی بی‌ایمان داشته باشد و آن زن راضی باشد که با وی بماند، او را جدا نسازد.
\par 13 وزنی که شوهر بی‌ایمان داشته باشد و او راضی باشد که با وی بماند، از شوهر خود جدا نشود.
\par 14 زیرا که شوهر بی‌ایمان از زن خود مقدس می‌شود و زن بی‌ایمان از برادر مقدس می‌گردد واگرنه اولاد شما ناپاک می‌بودند، لکن الحال مقدسند.
\par 15 اما اگر بی‌ایمان جدایی نماید، بگذارش که بشود زیرا برادر یا خواهر در این صورت مقید نیست و خدا ما را به سلامتی خوانده است.
\par 16 زیرا که تو چه دانی‌ای زن که شوهرت را نجات خواهی داد؟ یا چه دانی‌ای مرد که زن خود را نجات خواهی داد؟
\par 17 مگر اینکه به هرطور که خداوند به هرکس قسمت فرموده و به همان حالت که خدا هرکس راخوانده باشد، بدینطور رفتار بکند؛ و همچنین درهمه کلیساها امر می‌کنم.
\par 18 اگر کسی در مختونی خوانده شود، نامختون نگردد و اگر کسی درنامختونی خوانده شود، مختون نشود.
\par 19 ختنه چیزی نیست و نامختونی هیچ، بلکه نگاه داشتن امرهای خدا.
\par 20 هرکس در هر حالتی که خوانده شده باشد، در همان بماند.
\par 21 اگر در غلامی خوانده شدی تو را باکی نباشد، بلکه اگر هم می‌توانی آزاد شوی، آن را اولی تر استعمال کن.
\par 22 زیرا غلامی که در خداوند خوانده شده باشد، آزاد خداوند است؛ و همچنین شخصی آزاد که خوانده شد، غلام مسیح است.
\par 23 به قیمتی خریده شدید، غلام انسان نشوید.
\par 24 ‌ای برادران هرکس در هرحالتی که خوانده شده باشد، در آن نزد خدا بماند.
\par 25 اما درباره باکره‌ها حکمی از خداوند ندارم. لکن چون از خداوند رحمت یافتم که امین باشم، رای می‌دهم.
\par 26 پس گمان می‌کنم که بجهت تنگی این زمان، انسان را نیکو آن است که همچنان بماند.
\par 27 اگر با زن بسته شدی، جدایی مجوی واگر از زن جدا هستی دیگر زن مخواه.
\par 28 لکن هرگاه نکاح کردی، گناه نورزیدی و هرگاه باکره منکوحه گردید، گناه نکرد. ولی چنین در جسم زحمت خواهند کشید، لیکن من بر شما شفقت دارم.
\par 29 اما‌ای برادران، این را می‌گویم وقت تنگ است تا بعد از این آنانی که زن دارند مثل بی‌زن باشند
\par 30 و گریانان چون ناگریانان و خوشحالان مثل ناخوشحالان و خریدارن چون غیرمالکان باشند،
\par 31 و استعمال کنندگان این جهان مثل استعمال کنندگان نباشند، زیرا که صورت این جهان درگذر است.
\par 32 اما خواهش این دارم که شما بی‌اندیشه باشید. شخص مجرد در امور خداوند می‌اندیشدکه چگونه رضامندی خداوند را بجوید؛
\par 33 وصاحب زن در امور دنیا می‌اندیشد که چگونه زن خود را خوش بسازد.
\par 34 در میان زن منکوحه وباکره نیز تفاوتی است، زیرا باکره در امور خداوندمی اندیشد تا هم در تن و هم در روح مقدس باشد؛ اما منکوحه در امور دنیا می‌اندیشد تا شوهر خودرا خوش سازد.
\par 35 اما این را برای نفع شمامی گویم نه آنکه دامی بر شما بنهم بلکه نظر به شایستگی و ملازمت خداوند، بی‌تشویش.
\par 36 لکن هرگاه کسی گمان برد که با باکره خودناشایستگی می‌کند، اگر به حد بلوغ رسید و ناچاراست از چنین شدن، آنچه خواهد بکند؛ گناهی نیست؛ بگذار که نکاح کنند.
\par 37 اما کسی‌که در دل خود پایدار است و احتیاج ندارد بلکه در اراده خود مختار است و در دل خود جازم است که باکره خود را نگاه دارد، نیکو می‌کند.
\par 38 پس هم کسی‌که به نکاح دهد، نیکو می‌کند و کسی‌که به نکاح ندهد، نیکوتر می‌نماید.
\par 39 زن مادامی که شوهرش زنده است، بسته است. اما هرگاه شوهرش مرد آزاد گردید تا به هرکه بخواهد منکوحه شود، لیکن در خداوندفقط.اما بحسب رای من خوشحال تر است، اگر چنین بماند و من نیز گمان می‌برم که روح خدارا دارم.
\par 40 اما بحسب رای من خوشحال تر است، اگر چنین بماند و من نیز گمان می‌برم که روح خدارا دارم.

\chapter{8}

\par 1 اما درباره قربانی های بتها: می‌دانیم که همه علم داریم. علم باعث تکبر است، لکن محبت بنا می‌کند.
\par 2 اگر کسی گمان برد که چیزی می‌داند، هنوز هیچ نمی داند، بطوری که بایددانست.
\par 3 اما اگر کسی خدا را محبت نماید، نزد اومعروف می‌باشد.
\par 4 پس درباره خوردن قربانی های بتها، می‌دانیم که بت در جهان چیزی نیست و اینکه خدایی دیگر جز یکی نیست.
\par 5 زیرا هرچند هستند که به خدایان خوانده می‌شوند، چه در آسمان و چه درزمین، چنانکه خدایان بسیار و خداوندان بسیارمی باشند،
\par 6 لکن ما را یک خداست یعنی پدر که همه‌چیز از اوست و ما برای او هستیم، و یک خداوند یعنی عیسی مسیح که همه‌چیز از اوست و ما از او هستیم.
\par 7 ولی همه را این معرفت نیست، زیرا بعضی تابه حال به اعتقاد اینکه بت هست، آن را چون قربانی بت می‌خورند و ضمیر ایشان چون ضعیف است نجس می‌شود.
\par 8 اما خوراک، ما رامقبول خدا نمی سازد زیرا که نه به خوردن بهتریم و نه به ناخوردن بدتر.
\par 9 لکن احتیاط کنید مبادااختیار شما باعث لغزش ضعفا گردد.
\par 10 زیرا اگرکسی تو را که صاحب علم هستی بیند که در بتکده نشسته‌ای، آیا ضمیر آن کس که ضعیف است به خوردن قربانی های بتها بنا نمی شود؟
\par 11 و از علم تو آن برادر ضعیف که مسیح برای او مرد هلاک خواهد شد.
\par 12 و همچنین چون به برادران گناه ورزیدید و ضمایر ضعیفشان را صدمه رسانیدید، همانا به مسیح خطا نمودید.بنابراین، اگرخوراک باعث لغزش برادر من باشد، تا به ابدگوشت نخواهم خورد تا برادر خود را لغزش ندهم.
\par 13 بنابراین، اگرخوراک باعث لغزش برادر من باشد، تا به ابدگوشت نخواهم خورد تا برادر خود را لغزش ندهم.

\chapter{9}

\par 1 آیا رسول نیستم؟ آیا آزاد نیستم؟ آیا عیسی مسیح خداوند ما را ندیدم؟ آیا شماعمل من در خداوند نیستید؟
\par 2 هرگاه دیگران رارسول نباشم، البته شما را هستم زیرا که مهررسالت من در خداوند شما هستید.
\par 3 حجت من بجهت آنانی که مرا امتحان می‌کنند این است
\par 4 که آیا اختیار خوردن و آشامیدن نداریم؟
\par 5 آیااختیار نداریم که خواهر دینی را به زنی گرفته، همراه خود ببریم، مثل سایر رسولان و برادران خداوند و کیفا؟
\par 6 یا من و برنابا به تنهایی مختارنیستیم که کار نکنیم؟
\par 7 کیست که هرگز از خرجی خود جنگ کند؟ یا کیست که تاکستانی غرس نموده، از میوه‌اش نخورد؟ یا کیست که گله‌ای بچراند و از شیر گله ننوشد؟
\par 8 آیا این را بطور انسان می‌گویم یاشریعت نیز این را نمی گوید؟
\par 9 زیرا که در تورات موسی مکتوب است که «گاو را هنگامی که خرمن را خرد می‌کند، دهان مبند». آیا خدا در فکر گاوان می‌باشد؟
\par 10 آیا محض خاطر ما این رانمی گوید؟ بلی برای ما مکتوب است که شخم کننده می‌باید به امید، شخم نماید و خردکننده خرمن در امید یافتن قسمت خود باشد.
\par 11 چون ما روحانیها را برای شما کاشتیم، آیا امربزرگی است که ما جسمانیهای شما را درو کنیم؟
\par 12 اگر دیگران در این اختیار بر شما شریکند آیانه ماه بیشتر؟ لیکن این اختیار را استعمال نکردیم، بلکه هرچیز را متحمل می‌شویم، مبادا انجیل مسیح را تعویق اندازیم.
\par 13 آیا نمی دانید که هرکه در امور مقدس مشغول باشد، از هیکل می‌خورد و هرکه خدمت مذبح کند، از مذبح نصیبی می‌دارد.
\par 14 و همچنین خداوند فرمود که «هرکه به انجیل اعلام نماید، ازانجیل معیشت یابد».
\par 15 لیکن من هیچیک از اینهارا استعمال نکردم و این را به این قصد ننوشتم تا بامن چنین شود، زیرا که مرا مردن بهتر است از آنکه کسی فخر مرا باطل گرداند.
\par 16 زیرا هرگاه بشارت دهم، مرا فخر نیست چونکه مرا ضرورت افتاده است، بلکه وای بر من اگر بشارت ندهم.
\par 17 زیراهرگاه این را طوع کنم اجرت دارم، لکن اگر کره باشد وکالتی به من سپرده شد.
\par 18 در این صورت، مرا چه اجرت است تا آنکه چون بشارت می‌دهم، انجیل مسیح را بی‌خرج سازم و اختیار خود را درانجیل استعمال نکنم؟
\par 19 زیرا با اینکه از همه کسی آزاد بودم، خودرا غلام همه گردانیدم تا بسیاری را سود برم.
\par 20 ویهود را چون یهود گشتم تا یهود را سود برم واهل شریعت را مثل اهل شریعت تا اهل شریعت را سود برم؛
\par 21 و بی‌شریعتان را چون بی‌شریعتان شدم، هرچند نزد خدا بی‌شریعت نیستم، بلکه شریعت مسیح در من است، تا بی‌شریعتان را سودبرم؛
\par 22 ضعفا را ضعیف شدم تا ضعفا را سود برم؛ همه کس را همه‌چیز گردیدم تا به هرنوعی بعضی را برهانم.
\par 23 اما همه کار را بجهت انجیل می‌کنم تا در آن شریک گردم.
\par 24 آیا نمی دانید آنانی که در میدان می‌دوند، همه می‌دوند لکن یک نفر انعام را می‌برد. به اینطور شما بدوید تا به‌کمال ببرید.
\par 25 و هرکه ورزش کند در هرچیز ریاضت می‌کشد؛ اما ایشان تا تاج فانی را بیابند لکن ما تاج غیرفانی را.
\par 26 پس من چنین می‌دوم، نه چون کسی‌که شک دارد؛ ومشت می‌زنم نه آنکه هوا را بزنم.بلکه تن خودرا زبون می‌سازم و آن را در بندگی می‌دارم، مباداچون دیگران را وعظ نمودم، خود محروم شوم.
\par 27 بلکه تن خودرا زبون می‌سازم و آن را در بندگی می‌دارم، مباداچون دیگران را وعظ نمودم، خود محروم شوم.

\chapter{10}

\par 1 زیرا‌ای برادران نمی خواهم شما بی خبر باشید از اینکه پدران ما همه زیرابر بودند و همه از دریا عبور نمودند
\par 2 و همه به موسی تعمید یافتند، در ابر و در دریا؛
\par 3 و همه همان خوراک روحانی را خوردند
\par 4 و همه همان شرب روحانی را نوشیدند، زیرا که می‌آشامیدنداز صخره روحانی که از عقب ایشان می‌آمد و آن صخره مسیح بود.
\par 5 لیکن از اکثر ایشان خداراضی نبود، زیرا که در بیابان انداخته شدند.
\par 6 و این امور نمونه‌ها برای ما شد تا ماخواهشمند بدی نباشیم، چنانکه ایشان بودند؛
\par 7 ونه بت‌پرست شوید، مثل بعضی از ایشان، چنانکه مکتوب است: «قوم به اکل و شرب نشستند وبرای لهو و لعب برپا شدند.»
\par 8 و نه زنا کنیم چنانکه بعضی از ایشان کردند و در یک روزبیست و سه هزار نفر هلاک گشتند.
\par 9 و نه مسیح راتجربه کنیم، چنانکه بعضی از ایشان کردند و ازمارها هلاک گردیدند.
\par 10 و نه همهمه کنید، چنانکه بعضی از ایشان کردند و هلاک کننده ایشان را هلاک کرد.
\par 11 و این همه بطور مثل بدیشان واقع شد وبرای تنبیه ما مکتوب گردید که اواخر عالم به ما رسیده است.
\par 12 پس آنکه گمان برد که قایم است، باخبر باشد که نیفتد.
\par 13 هیچ تجربه جز آنکه مناسب بشر باشد، شما را فرو نگرفت. اما خداامین است که نمی گذارد شما فوق طاقت خودآزموده شوید، بلکه باتجربه مفری نیز می‌سازد تایارای تحمل آن را داشته باشید.
\par 14 لهذا‌ای عزیزان من از بت‌پرستی بگریزید.
\par 15 به خردمندان سخن می‌گویم: خود حکم کنیدبرآنچه می‌گویم.
\par 16 پیاله برکت که آن را تبرک می‌خوانیم، آیا شراکت در خون مسیح نیست؟ ونانی را که پاره می‌کنیم، آیا شراکت در بدن مسیح نی؟
\par 17 زیرا ما که بسیاریم، یک نان و یکتن می‌باشیم چونکه همه از یک نان قسمت می‌یابیم.
\par 18 اسرائیل جسمانی را ملاحظه کنید! آیاخورندگان قربانی‌ها شریک قربانگاه نیستند؟
\par 19 پس چه گویم؟ آیا بت چیزی می‌باشد؟ یا که قربانی بت چیزی است؟
\par 20 نی! بلکه آنچه امت هاقربانی می‌کنند، برای دیوها می‌گذرانند نه برای خدا؛ و نمی خواهم شما شریک دیوها باشید.
\par 21 محال است که هم از پیاله خداوند و هم از پیاله دیوها بنوشید؛ و هم از مایده خداوند و هم ازمایده دیوها نمی توانید قسمت برد.
\par 22 آیاخداوند را به غیرت می‌آوریم یا از او تواناترمی باشیم؟
\par 23 همه‌چیز جایز است، لیکن همه مفیدنیست؛ همه رواست، لیکن همه بنا نمی کند.
\par 24 هرکس نفع خود را نجوید، بلکه نفع دیگری را.
\par 25 هرآنچه را در قصابخانه می‌فروشند، بخوریدو هیچ مپرسید به‌خاطر ضمیر.
\par 26 زیرا که جهان و پری آن از آن خداوند است.
\par 27 هرگاه کسی ازبی ایمانان از شما وعده خواهد و می‌خواهیدبروید، آنچه نزد شما گذارند بخورید و هیچ مپرسید بجهت ضمیر.
\par 28 اما اگر کسی به شماگوید «این قربانی بت است»، مخورید به‌خاطر آن کس که خبر داد و بجهت ضمیر، زیرا که جهان وپری آن از آن خداوند است.
\par 29 اما ضمیرمی گویم نه از خودت بلکه ضمیر آن دیگر؛ زیراچرا ضمیر دیگری بر آزادی من حکم کند؟
\par 30 واگر من به شکر بخورم، چرا بر من افترا زنند به‌سبب آن چیزی که من برای آن شکر می‌کنم؟
\par 31 پس خواه بخورید، خواه بنوشید، خواه هرچه کنید، همه را برای جلال خدا بکنید.
\par 32 یهودیان و یونانیان و کلیسای خدا را لغزش مدهید.چنانکه من نیز در هرکاری همه را خوش می‌سازم و نفع خود را طالب نیستم، بلکه نفع بسیاری را تا نجات یابند.
\par 33 چنانکه من نیز در هرکاری همه را خوش می‌سازم و نفع خود را طالب نیستم، بلکه نفع بسیاری را تا نجات یابند.

\chapter{11}

\par 1 پس اقتدا به من نمایید چنانکه من نیز به مسیح می‌کنم.
\par 2 اما‌ای برادران شما را تحسین می‌نمایم ازاین جهت که در هرچیز مرا یاد می‌دارید و اخبار رابطوری که به شما سپردم، حفظ می‌نمایید.
\par 3 اما می‌خواهم شما بدانید که سر هر مرد، مسیح است و سر زن، مرد و سر مسیح، خدا.
\par 4 هرمردی که سرپوشیده دعا یا نبوت کند، سر خود رارسوا می‌نماید.
\par 5 اما هر زنی که سر برهنه دعا یانبوت کند، سر خود را رسوا می‌سازد، زیرا این چنان است که تراشیده شود.
\par 6 زیرا اگر زن نمی پوشد، موی را نیز ببرد؛ و اگر زن را موی بریدن یا تراشیدن قبیح است، باید بپوشد.
\par 7 زیراکه مرد را نباید سر خود بپوشد چونکه او صورت و جلال خداست، اما زن جلال مرد است.
\par 8 زیراکه مرد از زن نیست بلکه زن از مرد است.
\par 9 و نیزمرد بجهت زن آفریده نشد، بلکه زن برای مرد.
\par 10 از این جهت زن می‌باید عزتی بر سر داشته باشد به‌سبب فرشتگان.
\par 11 لیکن زن از مرد جدانیست و مرد هم جدا از زن نیست در خداوند.
\par 12 زیرا چنانکه زن از مرد است، همچنین مرد نیزبوسیله زن، لیکن همه‌چیز از خدا.
\par 13 در دل خود انصاف دهید: آیا شایسته است که زن ناپوشیده نزد خدا دعا کند؟
\par 14 آیا خودطبیعت شما را نمی آموزد که اگر مرد موی درازدارد، او را عار می‌باشد؟
\par 15 و اگر زن موی درازدارد، او را فخر است، زیرا که موی بجهت پرده بدو داده شد؟
\par 16 و اگر کسی ستیزه گر باشد، ما وکلیساهای خدا را چنین عادتی نیست.
\par 17 لیکن چون این حکم را به شما می‌کنم، شمارا تحسین نمی کنم، زیرا که شما نه از برای بهتری بلکه برای بدتری جمع می‌شوید.
\par 18 زیرا اولاهنگامی که شما در کلیسا جمع می‌شوید، می‌شنوم که در میان شما شقاقها روی می‌دهد وقدری از آن را باور می‌کنم.
\par 19 از آن جهت که لازم است در میان شما بدعتها نیز باشد تا که مقبولان از شما ظاهر گردند.
\par 20 پس چون شما دریک جا جمع می‌شوید، ممکن نیست که شام خداوند خورده شود،
\par 21 زیرا در وقت خوردن هرکس شام خود را پیشتر می‌گیرد و یکی گرسنه و دیگری مست می‌شود.
\par 22 مگر خانه‌ها برای خوردن و آشامیدن ندارید؟ یا کلیسای خدا راتحقیر می‌نمایید و آنانی را که ندارند شرمنده می‌سازید؟ به شما چه بگویم؟ آیا در این امر شمارا تحسین نمایم؟ تحسین نمی نمایم!
\par 23 زیرا من از خداوند یافتم، آنچه به شما نیزسپردم که عیسی خداوند در شبی که او را تسلیم کردند، نان را گرفت
\par 24 و شکر نموده، پاره کرد وگفت: «بگیرید بخورید. این است بدن من که برای شما پاره می‌شود. این را به یادگاری من به‌جاآرید.»
\par 25 و همچنین پیاله را نیز بعد از شام وگفت: «این پیاله عهد جدید است در خون من. هرگاه این را بنوشید، به یادگاری من بکنید.»
\par 26 زیرا هرگاه این نان را بخورید و این پیاله رابنوشید، موت خداوند را ظاهر می‌نمایید تاهنگامی که بازآید.
\par 27 پس هرکه بطور ناشایسته نان را بخورد وپیاله خداوند را بنوشد، مجرم بدن و خون خداوند خواهد بود.
\par 28 اما هر شخص خود راامتحان کند و بدینطرز از آن نان بخورد و از آن پیاله بنوشد.
\par 29 زیرا هرکه می‌خورد و می‌نوشد، فتوای خود را می‌خورد و می‌نوشد اگر بدن خداوند را تمییز نمی کند.
\par 30 از این سبب بسیاری از شما ضعیف و مریض‌اند و بسیاری خوابیده‌اند.
\par 31 اما اگر برخود حکم می‌کردیم، حکم بر مانمی شد.
\par 32 لکن هنگامی که بر ما حکم می‌شود، از خداوند تادیب می‌شویم مبادا با اهل دنیا بر ماحکم شود.
\par 33 لهذا‌ای برادران من، چون بجهت خوردن جمع می‌شوید، منتظر یکدیگر باشید.و اگرکسی گرسنه باشد، در خانه بخورد، مبادا بجهت عقوبت جمع شوید. و چون بیایم، مابقی را منتظم خواهم نمود.
\par 34 و اگرکسی گرسنه باشد، در خانه بخورد، مبادا بجهت عقوبت جمع شوید. و چون بیایم، مابقی را منتظم خواهم نمود.

\chapter{12}

\par 1 اما درباره عطایای روحانی، ای برادران نمی خواهم شما بی‌خبر باشید.
\par 2 می دانید هنگامی که امت‌ها می‌بودید، به سوی بتهای گنگ برده می‌شدید بطوری که شما رامی بردند.
\par 3 پس شما را خبر می‌دهم که هرکه متکلم به روح خدا باشد، عیسی را اناتیمانمی گوید و احدی جزیه روح‌القدس عیسی راخداوند نمی تواند گفت.
\par 4 و نعمتها انواع است ولی روح همان.
\par 5 وخدمتها انواع است اما خداوند همان.
\par 6 و عملهاانواع است لکن همان خدا همه را در همه عمل می‌کند.
\par 7 ولی هرکس را ظهور روح بجهت منفعت عطا می‌شود.
\par 8 زیرا یکی را بوساطت روح، کلام حکمت داده می‌شود و دیگری را کلام علم، بحسب همان روح.
\par 9 و یکی را ایمان به همان روح و دیگری را نعمتهای شفا دادن به همان روح.
\par 10 و یکی را قوت معجزات ودیگری را نبوت و یکی را تمییز ارواح و دیگری را اقسام زبانها و دیگری را ترجمه زبانها.
\par 11 لکن در جمیع اینها همان یک روح فاعل است که هرکس را فرد بحسب اراده خود تقسیم می‌کند.
\par 12 زیرا چنانکه بدن یک است و اعضای متعدددارد و تمامی اعضای بدن اگرچه بسیار است یکتن می‌باشد، همچنین مسیح نیز می‌باشد.
\par 13 زیرا که جمیع ما به یک روح در یک بدن تعمیدیافتم، خواه یهود، خواه یونانی، خواه غلام، خواه آزاد و همه از یک روح نوشانیده شدیم.
\par 14 زیرابدن یک عضو نیست بلکه بسیار است.
\par 15 اگر پاگوید چونکه دست نیستم از بدن نمی باشم، آیابدین سبب از بدن نیست؟
\par 16 و اگر گوش گویدچونکه چشم نیم از بدن نیستم، آیا بدین سبب ازبدن نیست؟
\par 17 اگر تمام بدن چشم بودی، کجامی بود شنیدن و اگر همه شنیدن بودی کجا می‌بودبوییدن؟
\par 18 لکن الحال خدا هریک از اعضا را دربدن نهاد برحسب اراده خود.
\par 19 و اگر همه یک عضو بودی بدن کجا می‌بود؟
\par 20 اما الان اعضابسیار است لیکن بدن یک.
\par 21 و چشم دست رانمی تواند گفت که محتاج تو نیستم یا سر پایها رانیز که احتیاج به شما ندارم.
\par 22 بلکه علاوه بر این، آن اعضای بدن که ضعیفتر می‌نمایند، لازم ترمی باشند.
\par 23 و آنها را که پست‌تر اجزای بدن می‌پندارم، عزیزتر می‌داریم و اجزای قبیح ماجمال افضل دارد.
\par 24 لکن اعضای جمیله ما رااحتیاجی نیست، بلکه خدا بدن را مرتب ساخت بقسمی که ناقص را بیشتر حرمت داد،
\par 25 تا که جدایی در بدن نیفتد، بلکه اعضا به برابری در فکریکدیگر باشند.
\par 26 و اگر یک عضو دردمند گردد، سایر اعضا با آن همدرد باشند و اگر عضوی عزت یابد، باقی اعضا با او به خوشی آیند.
\par 27 اما شما بدن مسیح هستید و فرد اعضای آن می‌باشید.
\par 28 و خدا قرارداد بعضی را درکلیسا: اول رسولان، دوم انبیا، سوم معلمان، بعدقوات، پس نعمتهای شفا دادن و اعانات و تدابیر واقسام زبانها.
\par 29 آیا همه رسول هستند، یا همه انبیا، یا همه معلمان، یا همه قوات؟
\par 30 یا همه نعمتهای شفا دارند، یا همه به زبانها متکلم هستند، یا همه ترجمه می‌کنند؟لکن نعمتهای بهتر را به غیرت بطلبید و طریق افضلتر نیز به شمانشان می‌دهم.
\par 31 لکن نعمتهای بهتر را به غیرت بطلبید و طریق افضلتر نیز به شمانشان می‌دهم.

\chapter{13}

\par 1 اگر به زبانهای مردم و فرشتگان سخن گویم و محبت نداشته باشم، مثل نحاس صدادهنده و سنج فغان کننده شده‌ام.
\par 2 و اگرنبوت داشته باشم و جمیع اسرار و همه علم رابدانم و ایمان کامل داشته باشم بحدی که کوهها رانقل کنم و محبت نداشته باشم، هیچ هستم.
\par 3 واگر جمیع اموال خود را صدقه دهم و بدن خود رابسپارم تا سوخته شود و محبت نداشته باشم، هیچ سود نمی برم.
\par 4 محبت حلیم و مهربان است؛ محبت حسد نمی برد؛ محبت کبر و غرور ندارد؛
\par 5 اطوار ناپسندیده ندارد و نفع خود را طالب نمی شود؛ خشم نمی گیرد و سوءظن ندارد؛
\par 6 ازناراستی خوشوقت نمی گردد، ولی با راستی شادی می‌کند؛
\par 7 در همه‌چیز صبر می‌کند و همه را باور می‌نماید؛ در همه حال امیدوار می‌باشد وهر چیز را متحمل می‌باشد.
\par 8 محبت هرگز ساقط نمی شود و اما اگر نبوتهاباشد، نیست خواهد شد و اگر زبانها، انتها خواهد پذیرفت و اگر علم، زایل خواهد گردید.
\par 9 زیراجزئی علمی داریم و جزئی نبوت می‌نماییم،
\par 10 لکن هنگامی که کامل آید، جزئی نیست خواهد گردید.
\par 11 زمانی که طفل بودم، چون طفل حرف می‌زدم و چون طفل فکر می‌کردم و مانندطفل تعقل می‌نمودم. اما چون مرد شدم، کارهای طفلانه را ترک کردم.
\par 12 زیرا که الحال در آینه بطور معما می‌بینم، لکن آن وقت روبرو؛ الان جزئی معرفتی دارم، لکن آن وقت خواهم شناخت، چنانکه نیز شناخته شدم.و الحال این سه چیز باقی است: یعنی‌ایمان و امید ومحبت. اما بزرگتر از اینها محبت است.
\par 13 و الحال این سه چیز باقی است: یعنی‌ایمان و امید ومحبت. اما بزرگتر از اینها محبت است.

\chapter{14}

\par 1 در‌پی محبت بکوشید و عطایای روحانی را به غیرت بطلبید، خصوص اینکه نبوت کنید.
\par 2 زیرا کسی‌که به زبانی سخن می‌گوید، نه به مردم بلکه به خدا می‌گوید، زیراهیچ‌کس نمی فهمد لیکن در روح به اسرار تکلم می‌نماید.
\par 3 اما آنکه نبوت می‌کند، مردم را برای بنا و نصیحت و تسلی می‌گوید.
\par 4 هرکه به زبانی می‌گوید، خود را بنا می‌کند، اما آنکه نبوت می‌نماید، کلیسا را بنا می‌کند.
\par 5 و خواهش دارم که همه شما به زبانها تکلم کنید، لکن بیشتر اینکه نبوت نمایید زیرا کسی‌که نبوت کند بهتر است ازکسی‌که به زبانها حرف زند، مگر آنکه ترجمه کندتا کلیسا بنا شود.
\par 6 اما الحال‌ای برادران اگر نزد شما آیم و به زبانها سخن رانم، شما را چه سود می‌بخشم؟ مگرآنکه شما را به مکاشفه یا به معرفت یا به نبوت یا به تعلیم گویم.
\par 7 و همچنین چیزهای بیجان که صدا می‌دهد چون نی یا بربط اگر در صداها فرق نکند، چگونه آواز نی یا بربط فهمیده می‌شود؟
\par 8 زیر اگر کرنا نیز صدای نامعلوم دهد که خود رامهیای جنگ می‌سازد؟
\par 9 همچنین شما نیز به زبان، سخن مفهوم نگویید، چگونه معلوم می‌شودآن چیزی که گفته شد زیرا که به هوا سخن خواهید گفت؟
\par 10 زیرا که انواع زبانهای دنیاهرقدر زیاده باشد، ولی یکی بی‌معنی نیست.
\par 11 پس هرگاه قوت زبان را نمی دانم، نزد متکلم بربری می‌باشم و آنکه سخن گوید نزد من بربری می‌باشد.
\par 12 همچنن شما نیز چونکه غیورعطایای روحانی هستید بطلبید اینکه برای بنای کلیسا افزوده شوید.
\par 13 بنابراین کسی‌که به زبانی سخن می‌گوید، دعا بکند تا ترجمه نماید.
\par 14 زیرا اگر به زبانی دعاکنم، روح من دعا می‌کند لکن عقل من برخوردارنمی شود.
\par 15 پس مقصود چیست؟ به روح دعاخواهم کرد و به عقل نیز دعا خواهم نمود؛ به روح سرود خواهم خواند و به عقل نیز خواهم خواند.
\par 16 زیرا اگر در روح تبرک می‌خوانی، چگونه آن کسی‌که به منزلت امی است، به شکر تو آمین گوید و حال آنکه نمی فهمد چه می‌گویی؟
\par 17 زیرا تو البته خوب شکر می‌کنی، لکن آن دیگربنا نمی شود.
\par 18 خدا را شکر می‌کنم که زیادتر ازهمه شما به زبانها حرف می‌زنم.
\par 19 لکن در کلیسابیشتر می‌پسندم که پنج کلمه به عقل خود گویم تادیگران را نیز تعلیم دهم از آنکه هزاران کلمه به زبان بگویم.
\par 20 ‌ای برادران، در فهم اطفال مباشید بلکه در بدخویی اطفال باشید و در فهم رشید.
\par 21 درتورات مکتوب است که «خداوند می‌گوید به زبانهای بیگانه و لبهای غیر به این قوم سخن خواهم گفت و با این همه مرا نخواهند شنید.»
\par 22 پس زبانها نشانی است نه برای ایمان داران بلکه برای بی‌ایمانان؛ اما نبوت برای بی‌ایمان نیست بلکه برای ایمانداران است.
\par 23 پس اگر تمام کلیسادر جایی جمع شوند و همه به زبانها حرف زنند وامیان یا بی‌ایمانان داخل شوند، آیا نمی گویند که دیوانه‌اید؟
\par 24 ولی اگر همه نبوت کنند و کسی ازبی ایمانان یا امیان درآید، از همه توبیخ می‌یابد واز همه ملزم می‌گردد،
\par 25 و خفایای قلب او ظاهرمی شود و همچنین به روی درافتاده، خدا راعبادت خواهد کرد و ندا خواهد داد که «فی الحقیقه خدا در میان شما است.»
\par 26 پس‌ای برادران مقصود این است که وقتی که جمع شوید، هریکی از شما سرودی دارد، تعلیمی دارد، زبانی دارد، مکاشفه‌ای دارد، ترجمه‌ای دارد، باید همه بجهت بنا بشود.
\par 27 اگرکسی به زبانی سخن گوید، دو دو یا نهایت سه سه باشند، به ترتیب و کسی ترجمه کند.
\par 28 اما اگرمترجمی نباشد، در کلیسا خاموش باشد و با خودو با خدا سخن گوید.
\par 29 و از انبیا دو یا سه سخن بگویند و دیگران تمیز دهند.
\par 30 و اگر چیزی به دیگری از اهل مجلس مکشوف شود، آن اول ساکت شود.
\par 31 زیرا که همه می‌توانید یک یک نبوت کنید تا همه تعلیم یابند و همه نصیحت‌پذیرند.
\par 32 و ارواح انبیا مطیع انبیا می‌باشند.
\par 33 زیرا که او خدای تشویش نیست بلکه خدای سلامتی، چنانکه در همه کلیساهای مقدسان.
\par 34 و زنان شما در کلیساها خاموش باشند زیرا که ایشان را حرف زدن جایز نیست بلکه اطاعت نمودن، چنانکه تورات نیز می‌گوید.
\par 35 اما اگرمی خواهند چیزی بیاموزند، در خانه از شوهران خود بپرسند، چون زنان را در کلیسا حرف زدن قبیح است.
\par 36 آیا کلام خدا از شما صادر شد یا به شما به تنهایی رسید؟
\par 37 اگر کسی خود را نبی یاروحانی پندارد، اقرار بکند که آنچه به شمامی نویسم، احکام خداوند است.
\par 38 اما اگر کسی جاهل است، جاهل باشد.
\par 39 پس‌ای برادران، نبوت را به غیرت بطلبید و از تکلم نمودن به زبانهامنع مکنید.لکن همه‌چیز به شایستگی و انتظام باشد.
\par 40 لکن همه‌چیز به شایستگی و انتظام باشد.

\chapter{15}

\par 1 الان‌ای برادران، شما را از انجیلی که به شما بشارت دادم اعلام می‌نمایم که آن را هم پذیرفتید و در آن هم قایم می‌باشید،
\par 2 وبوسیله آن نیز نجات می‌یابید، به شرطی که آن کلامی را که به شما بشارت دادم، محکم نگاه دارید والا عبث ایمان آوردید.
\par 3 زیرا که اول به شما سپردم، آنچه نیز یافتم که مسیح برحسب کتب در راه گناهان ما مرد،
\par 4 و اینکه مدفون شد ودر روز سوم برحسب کتب برخاست؛
\par 5 و اینکه به کیفا ظاهر شد و بعد از آن به آن دوازده،
\par 6 و پس ازآن به زیاده از پانصد برادر یک بار ظاهر شد که بیشتر از ایشان تا امروز باقی هستند اما بعضی خوابیده‌اند.
\par 7 از آن پس به یعقوب ظاهر شد و بعدبه جمیع رسولان.
\par 8 و آخر همه بر من مثل طفل سقطشده ظاهر گردید.
\par 9 زیرا من کهترین رسولان هستم و لایق نیستم که به رسول خوانده شوم، چونکه برکلیسای خدا جفا می‌رسانیدم.
\par 10 لیکن به فیض خدا آنچه هستم هستم و فیض او که بر من بود باطل نگشت، بلکه بیش از همه ایشان مشقت کشیدم، اما نه من بلکه فیض خدا که با من بود.
\par 11 پس خواه من و خواه ایشان بدین طریق وعظمی کنیم و به اینطور ایمان آوردید.
\par 12 لیکن اگر به مسیح وعظ می‌شود که ازمردگان برخاست، چون است که بعضی از شمامی گویند که قیامت مردگان نیست؟
\par 13 اما اگرمردگان را قیامت نیست، مسیح نیز برنخاسته است.
\par 14 و اگر مسیح برنخاست، باطل است وعظما و باطل است نیز ایمان شما.
\par 15 و شهود کذبه نیز برای خدا شدیم، زیرا درباره خدا شهادت دادیم که مسیح را برخیزانید، و حال آنکه او رابرنخیزانید در صورتی که مردگان برنمی خیزند.
\par 16 زیرا هرگاه مردگان برنمی خیزند، مسیح نیزبرنخاسته است.
\par 17 اما هرگاه مسیح برنخاسته است، ایمان شما باطل است و شما تاکنون درگناهان خود هستید،
\par 18 بلکه آنانی هم که درمسیح خوابیده‌اند هلاک شدند.
\par 19 اگر فقط در این جهان در مسیح امیدواریم، از جمیع مردم بدبخت تریم.
\par 20 لیکن بالفعل مسیح از مردگان برخاسته ونوبر خوابیدگان شده است.
\par 21 زیرا چنانکه به انسان موت آمد، به انسان نیز قیامت مردگان شد.
\par 22 و چنانکه در آدم همه می‌میرند در مسیح نیزهمه زنده خواهند گشت.
\par 23 لیکن هرکس به رتبه خود؛ مسیح نوبر است و بعد آنانی که در وقت آمدن او از آن مسیح می‌باشند.
\par 24 و بعد از آن انتها است وقتی که ملکوت را به خدا و پدر سپارد. و در آن زمان تمام ریاست و تمام قدرت و قوت را نابود خواهد گردانید.
\par 25 زیرا مادامی که همه دشمنان را زیر پایهای خود ننهد، می‌باید اوسلطنت بنماید.
\par 26 دشمن آخر که نابود می‌شود، موت است.
\par 27 زیرا «همه‌چیز را زیر پایهای وی انداخته است». اما چون می‌گوید که «همه‌چیز رازیر انداخته است»، واضح است که او که همه رازیر او انداخت مستثنی است.
\par 28 اما زمانی که همه مطیع وی شده باشند، آنگاه خود پسر هم مطیع خواهد شد او را که همه‌چیز را مطیع وی گردانید، تا آنکه خدا کل در کل باشد.
\par 29 والا آنانی که برای مردگان تعمید می‌یابندچکنند؟ هرگاه مردگان مطلق برنمی خیزند، پس چرا برای ایشان تعمید می‌گیرند؟
\par 30 و ما نیز چراهر ساعت خود را در خطر می‌اندازیم؟
\par 31 به آن فخری درباره شما که مرا در خداوند ما مسیح عیسی هست قسم، که هرروزه مرا مردنی است.
\par 32 چون بطور انسان در افسس با وحوش جنگ کردم، مرا چه سود است؟ اگر مردگان برنمی خیزند، بخوریم و بیاشامیم چون فردامی میریم.
\par 33 فریفته مشوید معاشرات بد، اخلاق حسنه را فاسد می‌سازد.
\par 34 برای عدالت بیدارشده، گناه مکنید زیرا بعضی معرفت خدا راندارند. برای انفعال شما می‌گویم.
\par 35 اما اگر کسی گوید: «مردگان چگونه برمی خیزند و به کدام بدن می‌آیند؟»،
\par 36 ‌ای احمق آنچه تو می‌کاری زنده نمی گردد جز آنکه بمیرد.
\par 37 و آنچه می‌کاری، نه آن جسمی را که خواهد شد می‌کاری، بلکه دانه‌ای مجرد خواه ازگندم و یا از دانه های دیگر.
\par 38 لیکن خدا برحسب اراده خود، آن را جسمی می‌دهد و به هر یکی ازتخمها جسم خودش را.
\par 39 هر گوشت از یک نوع گوشت نیست، بلکه گوشت انسان، دیگر است وگوشت بهایم، دیگر و گوشت مرغان، دیگر وگوشت ماهیان، دیگر.
\par 40 و جسمهای آسمانی هست و جسمهای زمینی نیز، لیکن‌شان آسمانی‌ها، دیگر و‌شان زمینی‌ها، دیگر است،
\par 41 و‌شان آفتاب دیگر و‌شان ماه دیگر و‌شان ستارگان، دیگر، زیرا که ستاره از ستاره در‌شان، فرق دارد.
\par 42 به همین نهج است نیز قیامت مردگان. درفساد کاشته می‌شود، و در بی‌فسادی برمی خیزد؛
\par 43 در ذلت کاشته می‌گردد و در جلال برمی خیزد؛ در ضعف کاشته می‌شود و در قوت برمی خیزد؛
\par 44 جسم نفسانی کاشته می‌شود و جسم روحانی برمی خیزد. اگر جسم نفسانی هست، هرآینه روحانی نیز هست.
\par 45 و همچنین نیز مکتوب است که انسان اول یعنی آدم نفس زنده گشت، اماآدم آخر روح حیات‌بخش شد.
\par 46 لیکن روحانی مقدم نبود بلکه نفسانی و بعد از آن روحانی.
\par 47 انسان اول از زمین است خاکی؛ انسان دوم خداوند است از آسمان.
\par 48 چنانکه خاکی است، خاکیان نیز چنان هستند و چنانکه آسمانی است آسمانی‌ها همچنان می‌باشند.
\par 49 و چنانکه صورت خاکی را گرفتیم، صورت آسمانی را نیزخواهیم گرفت.
\par 50 لیکن‌ای برادران این را می‌گویم که گوشت و خون نمی تواند وارث ملکوت خدا شود و فاسدوارث بی‌فسادی نیز نمی شود.
\par 51 همانا به شماسری می‌گویم که همه نخواهیم خوابید، لیکن همه متبدل خواهیم شد.
\par 52 در لحظه‌ای، درطرفه العینی، به مجرد نواختن صور اخیر، زیراکرنا صدا خواهد داد، و مردگان، بی‌فساد خواهندبرخاست و ما متبدل خواهیم شد.
\par 53 زیرا که می‌باید این فاسد بی‌فسادی را بپوشد و این فانی به بقا آراسته گردد.
\par 54 اما چون این فاسد بی‌فسادی را پوشید و این فانی به بقا آراسته شد، آنگاه این کلامی که مکتوب است به انجام خواهد رسید که «مرگ در ظفر بلعیده شده است.
\par 55 ‌ای موت نیش تو کجا است و‌ای گور ظفر تو کجا؟»
\par 56 نیش موت گناه است و قوت گناه، شریعت.
\par 57 لیکن شکر خدا راست که ما را بواسطه خداوند ماعیسی مسیح ظفر می‌دهد.بنابراین‌ای برادران حبیب من پایدار وبی تشویش شده، پیوسته در عمل خداوندبیفزایید، چون می‌دانید که زحمت شما درخداوند باطل نیست.
\par 58 بنابراین‌ای برادران حبیب من پایدار وبی تشویش شده، پیوسته در عمل خداوندبیفزایید، چون می‌دانید که زحمت شما درخداوند باطل نیست.

\chapter{16}

\par 1 اما درباره جمع کردن زکات برای مقدسین، چنانکه به کلیساهای غلاطیه فرمودم، شما نیز همچنین کنید.
\par 2 در روز اول هفته، هر یکی از شما بحسب نعمتی که یافته باشد، نزد خود ذخیره کرده، بگذارد تا در وقت آمدن من زحمت جمع کردن نباشد.
\par 3 و چون برسم، آنانی را که اختیار کنید با مکتوبها خواهم فرستاد تا احسان شما را به اورشلیم ببرند.
\par 4 و اگرمصلحت باشد که من نیز بروم، همراه من خواهندآمد.
\par 5 و چون از مکادونیه عبور کنم، به نزد شماخواهم آمد، زیرا که از مکادنیه عبور می‌کنم،
\par 6 واحتمال دارد که نزد شما بمانم بلکه زمستان را نیزبسر برم تا هرجایی که بروم، شما مرا مشایعت کنید.
\par 7 زیرا که الان اراده ندارم در بین راه شما راملاقات کنم، چونکه امیدوارم مدتی با شما توقف نمایم، اگر خداوند اجازت دهد.
\par 8 لیکن من تاپنطیکاست در افسس خواهم ماند،
\par 9 زیرا که دروازه بزرگ و کارساز برای من باز شد و معاندین، بسیارند.
\par 10 لیکن اگر تیموتاوس آید، آگاه باشیدکه نزد شما بی‌ترس باشد، زیرا که در کار خداوندمشغول است چنانکه من نیز هستم.
\par 11 لهذاهیچ‌کس او را حقیر نشمارد، بلکه او را به سلامتی مشایعت کنید تا نزد من آید زیرا که او را با برادران انتظار می‌کشم.
\par 12 اما درباره اپلس برادر، از اوبسیار درخواست نمودم که با برادران به نزد شمابیاید، لیکن هرگز رضا نداد که الحال بیاید ولی چون فرصت یابد خواهد آمد.
\par 13 بیدار شوید، در ایمان استوار باشید و مردان باشید و زورآور شوید.
\par 14 جمیع کارهای شما با محبت باشد.
\par 15 و‌ای برادران به شماالتماس دارم (شما خانواده استیفان را می‌شناسیدکه نوبر اخائیه هستند و خویشتن را به خدمت مقدسین سپرده اند)،
\par 16 تا شما نیز چنین اشخاص را اطاعت کنید و هرکس را که در کار و زحمت شریک باشد.
\par 17 و از آمدن استفان و فرتوناتس واخائیکوس مرا شادی رخ نمود زیرا که آنچه ازجانب شما ناتمام بود، ایشان تمام کردند.
\par 18 چونکه روح من و شما را تازه کردند. پس چنین اشخاص را بشناسید.
\par 19 کلیساهای آسیا به شما سلام می‌رسانند واکیلا و پرسکلا با کلیسایی که در خانه ایشانند، به شما سلام بسیار در خداوند می‌رسانند.
\par 20 همه برادران شما را سلام می‌رسانند. یکدیگر را به بوسه مقدسانه سلام رسانید.
\par 21 من پولس ازدست خود سلام می‌رسانم.
\par 22 اگر کسی عیسی مسیح خداوند را دوست ندارد، اناتیما باد. ماران اتا.فیض عیسی مسیح خداوند با شما باد.
\par 23 فیض عیسی مسیح خداوند با شما باد.



\end{document}