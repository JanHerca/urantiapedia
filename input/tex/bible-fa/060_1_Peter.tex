\begin{document}

\title{1 Peter}


\chapter{1}

\par 1 به غریبانی که پراکنده‌اند در پنطس وغلاطیه و قپدوقیه و آسیا و بطانیه؛
\par 2 برگزیدگان برحسب علم سابق خدای پدر، به تقدیس روح برای اطاعت و پاشیدن خون عیسی مسیح.فیض و سلامتی بر شما افزون باد.
\par 3 متبارک باد خدا و پدر خداوند ما عیسی مسیح که بحسب رحمت عظیم خود ما رابوساطت برخاستن عیسی مسیح از مردگان از نوتولید نمود برای امید زنده،
\par 4 بجهت میراث بی‌فساد و بی‌آلایش و ناپژمرده که نگاه داشته شده است در آسمان برای شما؛
\par 5 که به قوت خدامحروس هستید به ایمان برای نجاتی که مهیاشده است تا در ایام آخر ظاهر شود.
\par 6 و در آن وجد می‌نمایید، هرچند در حال، اندکی از راه ضرورت در تجربه های گوناگون محزون شده‌اید،
\par 7 تا آزمایش ایمان شما که از طلای فانی باآزموده شدن در آتش، گرانبهاتر است، برای تسبیح و جلال و اکرام یافت شود در حین ظهورعیسی مسیح.
\par 8 که او را اگرچه ندیده‌اید محبت می‌نمایید و الان اگرچه او را نمی بینید، لکن بر اوایمان آورده، وجد می‌نمایید با خرمی‌ای که نمی توان بیان کرد و پر از جلال است.
\par 9 و انجام ایمان خود یعنی نجات جان خویش را می‌یابید.
\par 10 که درباره این نجات، انبیایی که از فیضی که برای شما مقرر بود، اخبار نمودند، تفتیش وتفحص می‌کردند
\par 11 و دریافت می‌نمودند که کدام و چگونه زمان است که روح مسیح که درایشان بود از آن خبر می‌داد، چون از زحماتی که برای مسیح مقرر بود و جلالهایی که بعد از آنهاخواهد بود، شهادت می‌داد؛
\par 12 و بدیشان مکشوف شد که نه به خود بلکه به ما خدمت می‌کردند، در آن اموری که شما اکنون از آنها خبریافته‌اید از کسانی که به روح‌القدس که از آسمان فرستاده شده است، بشارت داده‌اند و فرشتگان نیز مشتاق هستند که در آنها نظر کنند.
\par 13 لهذا کمر دلهای خود را ببندید و هشیارشده، امید کامل آن فیضی را که در مکاشفه عیسی مسیح به شما عطا خواهد شد، بدارید.
\par 14 و چون ابنای اطاعت هستید، مشابه مشوید بدان شهواتی که در ایام جهالت می‌داشتید.
\par 15 بلکه مثل آن قدوس که شما را خوانده است، خود شما نیز درهر سیرت، مقدس باشید.
\par 16 زیرا مکتوب است: «مقدس باشید زیرا که من قدوسم.»
\par 17 و چون او را پدر می‌خوانید که بدون ظاهربینی برحسب اعمال هرکس داوری می‌نماید، پس هنگام غربت خود را با ترس صرف نمایید.
\par 18 زیرا می‌دانید که خریده شده‌اید از سیرت باطلی که از پدران خود یافته‌اید نه به چیزهای فانی مثل نقره و طلا،
\par 19 بلکه به خون گرانبها چون خون بره بی‌عیب و بی‌داغ یعنی خون مسیح،
\par 20 که پیش از بنیاد عالم معین شد، لکن درزمان آخر برای شما ظاهر گردید،
\par 21 که بوساطت او شما بر آن خدایی که او را از مردگان برخیزانیدو او را جلال داد، ایمان آورده‌اید تا ایمان و امیدشما بر خدا باشد.
\par 22 چون نفسهای خود را به اطاعت راستی طاهر ساخته‌اید تا محبت برادرانه بی‌ریا داشته باشید، پس یکدیگر را از دل بشدت محبت بنمایید.
\par 23 از آنرو که تولد تازه یافتید نه از تخم فانی بلکه از غیرفانی یعنی به کلام خدا که زنده وتا ابدالاباد باقی است.
\par 24 زیرا که «هر بشری مانندگیاه است و تمام جلال او چون گل گیاه. گیاه پژمرده شد و گلش ریخت.لکن کلمه خدا تاابدالاباد باقی است.» و این است آن کلامی که به شما بشارت داده شده است.
\par 25 لکن کلمه خدا تاابدالاباد باقی است.» و این است آن کلامی که به شما بشارت داده شده است.

\chapter{2}

\par 1 لهذا هر نوع کینه و هر مکر و ریا و حسد وهرقسم بدگویی را ترک کرده،
\par 2 چون اطفال نوزاده، مشتاق شیر روحانی و بی‌غش باشید تا از آن برای نجات نمو کنید،
\par 3 اگرفی الواقع چشیده‌اید که خداوند مهربان است.
\par 4 و به او تقرب جسته، یعنی به آن سنگ زنده رد شده از مردم، لکن نزد خدا برگزیده و مکرم.
\par 5 شما نیز مثل سنگهای زنده بنا کرده می‌شوید به عمارت روحانی و کهانت مقدس تا قربانی های روحانی و مقبول خدا را بواسطه عیسی مسیح بگذرانید.
\par 6 بنابراین، در کتاب مکتوب است که «اینک می‌نهم در صهیون سنگی سر زاویه برگزیده و مکرم و هر‌که به وی ایمان آورد خجل نخواهد شد.»
\par 7 پس شما را که ایمان دارید اکرام است، لکن آنانی را که ایمان ندارند، «آن سنگی که معماران رد کردند، همان سر زاویه گردید»،
\par 8 و «سنگ لغزش دهنده و صخره مصادم»، زیرا که اطاعت کلام نکرده، لغزش می‌خورند که برای همین معین شده‌اند.
\par 9 لکن شما قبیله برگزیده و کهانت ملوکانه وامت مقدس و قومی که ملک خاص خدا باشدهستید تا فضایل او را که شما را از ظلمت، به نورعجیب خود خوانده است، اعلام نمایید.
\par 10 که سابق قومی نبودید و الان قوم خدا هستید. آن وقت از رحمت محروم، اما الحال رحمت کرده شده‌اید.
\par 11 ‌ای محبوبان، استدعا دارم که چون غریبان و بیگانگان از شهوات جسمی که با نفس در نزاع هستند، اجتناب نمایید؛
\par 12 و سیرت خود را درمیان امت‌ها نیکو دارید تا در همان امری که شمارا مثل بدکاران بد می‌گویند، از کارهای نیکوی شما که ببینند، در روز تفقد، خدا را تمجیدنمایند.
\par 13 لهذا هر منصب بشری را بخاطر خداونداطاعت کنید، خواه پادشاه را که فوق همه است،
\par 14 و خواه حکام را که رسولان وی هستند، بجهت انتقام کشیدن از بدکاران و تحسین نیکوکاران.
\par 15 زیرا که همین است اراده خدا که به نیکوکاری خود، جهالت مردمان بی‌فهم را ساکت نمایید،
\par 16 مثل آزادگان، اما نه مثل آنانی که آزادی خود را پوشش شرارت می‌سازند بلکه چون بندگان خدا.
\par 17 همه مردمان را احترام کنید. برادران را محبت نمایید. از خدا بترسید. پادشاه رااحترام نمایید.
\par 18 ‌ای نوکران، مطیع آقایان خود باشید با کمال ترس؛ و نه فقط صالحان و مهربانان را بلکه کج خلقان را نیز.
\par 19 زیرا این ثواب است که کسی بجهت ضمیری که چشم بر خدا دارد، در وقتی که ناحق زحمت می‌کشد، دردها را متحمل شود.
\par 20 زیرا چه فخر دارد هنگامی که گناهکار بوده، تازیانه خورید و متحمل آن شوید. لکن اگرنیکوکار بوده، زحمت کشید و صبر کنید، این نزدخدا ثواب است.
\par 21 زیرا که برای همین خوانده شده‌اید، چونکه مسیح نیز برای ما عذاب کشید وشما را نمونه‌ای گذاشت تا در اثر قدمهای وی رفتار نمایید،
\par 22 «که هیچ گناه نکرد و مکر درزبانش یافت نشد.»
\par 23 چون او را دشنام می‌دادند، دشنام پس نمی داد و چون عذاب می‌کشید تهدیدنمی نمود، بلکه خویشتن را به داور عادل تسلیم کرد.
\par 24 که خود گناهان ما را در بدن خویش بردارمتحمل شد تا از گناه مرده شده، به عدالت زیست نماییم که به ضربهای او شفا یافته‌اید.از آنروکه مانند گوسفندان گمشده بودید، لکن الحال به سوی شبان و اسقف جانهای خود برگشته‌اید.
\par 25 از آنروکه مانند گوسفندان گمشده بودید، لکن الحال به سوی شبان و اسقف جانهای خود برگشته‌اید.

\chapter{3}

\par 1 همچنین‌ای زنان، شوهران خود را اطاعت نمایید تا اگر بعضی نیز مطیع کلام نشوند، سیرت زنان، ایشان را بدون کلام دریابد،
\par 2 چونکه سیرت طاهر و خداترس شما را بینند.
\par 3 و شما رازینت ظاهری نباشد، از بافتن موی و متحلی شدن به طلا و پوشیدن لباس،
\par 4 بلکه انسانیت باطنی قلبی در لباس غیر فاسد روح حلیم و آرام که نزدخدا گرانبهاست.
\par 5 زیرا بدینگونه زنان مقدسه درسابق نیز که متوکل به خدا بودند، خویشتن رازینت می‌نمودند و شوهران خود را اطاعت می‌کردند.
\par 6 مانند ساره که ابراهیم را مطیع می‌بودو او را آقا می‌خواند و شما دختران او شده‌اید، اگر نیکویی کنید و از هیچ خوف ترسان نشوید.
\par 7 و همچنین‌ای شوهران، با فطانت با ایشان زیست کنید، چون با ظروف ضعیف تر زنانه، وایشان را محترم دارید چون با شما وارث فیض حیات نیز هستند تا دعاهای شما بازداشته نشود.
\par 8 خلاصه همه شما یکرای و همدرد و برادردوست و مشفق و فروتن باشید.
\par 9 و بدی بعوض بدی و دشنام بعوض دشنام مدهید، بلکه برعکس برکت بطلبید زیرا که می‌دانید برای این خوانده شده‌اید تا وارث برکت شوید.
\par 10 زیرا «هرکه می‌خواهد حیات را دوست دارد و ایام نیکو بیند، زبان خود را از بدی و لبهای خود را از فریب گفتن باز بدارد؛
\par 11 از بدی اعراض نماید و نیکویی رابه‌جا آورد؛ سلامتی را بطلبد و آن را تعاقب نماید.
\par 12 از آنرو که چشمان خداوند بر عادلان است وگوشهای او به سوی دعای ایشان، لکن روی خداوند بر بدکاران است.»
\par 13 و اگر برای نیکویی غیور هستید، کیست که به شما ضرری برساند؟
\par 14 بلکه هرگاه برای عدالت زحمت کشیدید، خوشابحال شما. پس ازخوف ایشان ترسان و مضطرب مشوید.
\par 15 بلکه خداوند مسیح را در دل خود تقدیس نمایید وپیوسته مستعد باشید تا هرکه سبب امیدی را که دارید از شما بپرسد، او را جواب دهید، لیکن باحلم و ترس.
\par 16 و ضمیر خود را نیکو بدارید تاآنانی که بر سیرت نیکوی شما در مسیح طعن می‌زنند، در همان چیزی که شما را بد می‌گویندخجالت کشند،
\par 17 زیرا اگر اراده خدا چنین است، نیکوکار بودن و زحمت کشیدن، بهتر است ازبدکردار بودن.
\par 18 زیرا که مسیح نیز برای گناهان یک بار زحمت کشید، یعنی عادلی برای ظالمان، تا ما را نزد خدا بیاورد؛ در حالیکه بحسب جسم مرد، لکن بحسب روح زنده گشت،
\par 19 و به آن روح نیز رفت و موعظه نمود به ارواحی که درزندان بودند،
\par 20 که سابق نافرمانبردار بودندهنگامی که حلم خدا در ایام نوح انتظار می‌کشید، وقتی که کشتی بنا می‌شد، که در آن جماعتی قلیل یعنی هشت نفر به آب نجات یافتند،
\par 21 که نمونه آن یعنی تعمید اکنون ما را نجات می‌بخشد (نه دور کردن کثافت جسم بلکه امتحان ضمیر صالح به سوی خدا) بواسطه برخاستن عیسی مسیح،که به آسمان رفت و بدست راست خدا است وفرشتگان و قدرتها و قوات مطیع او شده‌اند.
\par 22 که به آسمان رفت و بدست راست خدا است وفرشتگان و قدرتها و قوات مطیع او شده‌اند.

\chapter{4}

\par 1 لهذا چون مسیح بحسب جسم برای مازحمت کشید، شما نیز به همان نیت مسلح شوید زیرا آنکه بحسب جسم زحمت کشید، ازگناه بازداشته شده است.
\par 2 تا آنکه بعد از آن مابقی عمر را در جسم نه بحسب شهوات انسانی بلکه موافق اراده خدا بسر برد.
\par 3 زیرا که عمر گذشته کافی است برای عمل نمودن به خواهش امت‌ها ودر فجور و شهوات و می‌گساری و عیاشی و بزمهاو بت‌پرستیهای حرام رفتار نمودن.
\par 4 و در این متعجب هستند که شما همراه ایشان به سوی همین اسراف اوباشی نمی شتابید و شما را دشنام می‌دهند.
\par 5 و ایشان حساب خواهند داد بدو که مستعد است تا زندگان و مردگان را داوری نماید.
\par 6 زیرا که از اینجهت نیز به مردگان بشارت داده شد تا بر ایشان موافق مردم بحسب جسم حکم شود و موافق خدا بحسب روح زیست نمایند.
\par 7 لکن انتهای همه‌چیز نزدیک است. پس خرداندیش و برای دعا هشیار باشید.
\par 8 و اول همه با یکدیگر بشدت محبت نمایید زیرا که محبت کثرت گناهان را می‌پوشاند.
\par 9 و یکدیگر را بدون همهمه مهمانی کنید.
\par 10 و هریک بحسب نعمتی که یافته باشد، یکدیگر را در آن خدمت نماید، مثل وکلاء امین فیض گوناگون خدا.
\par 11 اگر کسی سخن گوید، مانند اقوال خدا بگوید و اگر کسی خدمت کند، برحسب توانایی که خدا بدو داده باشد بکند تا در همه‌چیز، خدا بواسطه عیسی مسیح جلال یابد که او را جلال و توانایی تاابدالاباد هست، آمین.
\par 12 ‌ای حبیبان، تعجب منمایید از این آتشی که در میان شماست و بجهت امتحان شما می‌آید که گویا چیزی غریب بر شما واقع شده باشد.
\par 13 بلکه بقدری که شریک زحمات مسیح هستید، خشنودشوید تا در هنگام ظهور جلال وی شادی و وجدنمایید.
\par 14 اگر بخاطر نام مسیح رسوایی می‌کشید، خوشابحال شما زیرا که روح جلال وروح خدا بر شما آرام می‌گیرد.
\par 15 پس زنهار هیچ‌یکی از شما چون قاتل یا دزد یا شریر یا فضول عذاب نکشد.
\par 16 لکن اگر چون مسیحی عذاب بکشد، پس شرمنده نشود بلکه به این اسم خدا راتمجید نماید.
\par 17 زیرا این زمان است که داوری ازخانه خدا شروع شود؛ و اگر شروع آن از ماست، پس عاقبت کسانی که انجیل خدا را اطاعت نمی کنند چه خواهد شد؟
\par 18 و اگر عادل به دشواری نجات یابد، بی‌دین و گناهکار کجا یافت خواهد شد؟پس کسانی نیز که برحسب اراده خدا زحمت کشند، جانهای خود را در نیکوکاری به خالق امین بسپارند.
\par 19 پس کسانی نیز که برحسب اراده خدا زحمت کشند، جانهای خود را در نیکوکاری به خالق امین بسپارند.

\chapter{5}

\par 1 پیران را در میان شما نصیحت می‌کنم، من که نیز با شما پیر هستم و شاهد بر زحمات مسیح و شریک در جلالی که مکشوف خواهدشد.
\par 2 گله خدا را که در میان شماست بچرانید ونظارت آن را بکنید، نه به زور بلکه به رضامندی ونه بجهت سود قبیح بلکه به رغبت؛
\par 3 و نه‌چنانکه بر قسمت های خود خداوندی بکنید بلکه بجهت گله نمونه باشید،
\par 4 تا در وقتی که رئیس شبانان ظاهر شود، تاج ناپژمرده جلال را بیابید.
\par 5 همچنین‌ای جوانان، مطیع پیران باشید بلکه همه با یکدیگر فروتنی را بر خود ببندید زیرا خدابا متکبران مقاومت می‌کند و فروتنان را فیض می‌بخشد.
\par 6 پس زبردست زورآور خدا فروتنی نمایید تا شما را در وقت معین سرافراز نماید.
\par 7 وتمام اندیشه خود را به وی واگذارید زیرا که اوبرای شما فکر می‌کند.
\par 8 هشیار و بیدار باشید زیراکه دشمن شما ابلیس مانند شیر غران گردش می‌کند و کسی را می‌طلبد تا ببلعد.
\par 9 پس به ایمان استوار شده، با او مقاومت کنید، چون آگاه هستیدکه همین زحمات بر برادران شما که در دنیاهستند، می‌آید.
\par 10 و خدای همه فیضها که ما را به جلال ابدی خود در عیسی مسیح خوانده، است، شما را بعداز کشیدن زحمتی قلیل کامل و استوار و تواناخواهد ساخت.
\par 11 او را تا ابدالاباد جلال وتوانایی باد، آمین.
\par 12 به توسط سلوانس که او را برادر امین شمامی شمارم، مختصری نوشتم و نصیحت وشهادت می‌دهم که همین است فیض حقیقی خداکه بر آن قائم هستید.خواهر برگزیده با شما که در بابل است و پسر من مرقس به شما سلام می‌رسانند.
\par 13 خواهر برگزیده با شما که در بابل است و پسر من مرقس به شما سلام می‌رسانند.



\end{document}