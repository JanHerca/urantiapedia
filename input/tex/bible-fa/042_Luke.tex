\begin{document}

\title{Luke}


\chapter{1}

\par 1 از آنجهت که بسیاری دست خود را درازکردند به سوی تالیف حکایت آن اموری که نزد ما به اتمام رسید،
\par 2 چنانچه آنانی که از ابتدانظارگان و خادمان کلام بودند به ما رسانیدند،
\par 3 من نیز مصلحت چنان دیدم که همه را من البدایه به تدقیق در‌پی رفته، به ترتیب به تو بنویسم‌ای تیوفلس عزیز،
\par 4 تا صحت آن کلامی که در آن تعلیم یافته‌ای دریابی.
\par 5 در ایام هیرودیس پادشاه یهودیه، کاهنی زکریا نام از فرقه ابیا بود که زن او از دختران هارون بود و الیصابات نام داشت.
\par 6 و هر دو در حضورخدا صالح و به جمیع احکام و فرایض خداوند، بی‌عیب سالک بودند.
\par 7 و ایشان را فرزندی نبودزیرا که الیصابات نازاد بود و هر دو دیرینه سال بودند.
\par 8 و واقع شد که چون به نوبت فرقه خود درحضور خدا کهانت می‌کرد،
\par 9 حسب عادت کهانت، نوبت او شد که به قدس خداوند درآمده، بخور بسوزاند.
\par 10 و در وقت بخور، تمام جماعت قوم بیرون عبادت می‌کردند.
\par 11 ناگاه فرشته خداوند به طرف راست مذبح بخور ایستاده، بر وی ظاهر گشت.
\par 12 چون زکریااو را دید، در حیرت افتاده، ترس بر او مستولی شد.
\par 13 فرشته بدو گفت: «ای زکریا ترسان مباش، زیرا که دعای تو مستجاب گردیده است وزوجه ات الیصابات برای تو پسری خواهد زاییدو او را یحیی خواهی نامید.
\par 14 و تو را خوشی وشادی رخ خواهد نمود و بسیاری از ولادت اومسرور خواهند شد.
\par 15 زیرا که در حضورخداوند بزرگ خواهد بود و شراب و مسکری نخواهد نوشید و از شکم مادر خود، پر ازروح‌القدس خواهد بود.
\par 16 و بسیاری ازبنی‌اسرائیل را، به سوی خداوند خدای ایشان خواهد برگردانید.
\par 17 و او به روح و قوت الیاس پیش روی وی خواهد خرامید، تا دلهای پدران رابه طرف پسران و نافرمانان را به حکمت عادلان بگرداند تا قومی مستعد برای خدا مهیا سازد.»
\par 18 زکریا به فرشته گفت: «این را چگونه بدانم وحال آنکه من پیر هستم و زوجه‌ام دیرینه سال است؟»
\par 19 فرشته در جواب وی گفت: «من جبرائیل هستم که در حضور خدا می‌ایستم وفرستاده شدم تا به تو سخن گویم و از این امور تورا مژده دهم.
\par 20 و الحال تا این امور واقع نگردد، گنگ شده یارای حرف زدن نخواهی داشت، زیراسخن های مرا که در وقت خود به وقوع خواهدپیوست، باور نکردی.»
\par 21 و جماعت منتظر زکریامی بودند و از طول توقف او در قدس متعجب شدند.
\par 22 اما چون بیرون آمده نتوانست با ایشان حرف زند، پس فهمیدند که در قدس رویایی دیده است، پس به سوی ایشان اشاره می‌کرد و ساکت ماند.
\par 23 و چون ایام خدمت او به اتمام رسید، به خانه خود رفت.
\par 24 و بعد از آن روزها، زن او الیصابات حامله شده، مدت پنج ماه خود را پنهان نمود و گفت:
\par 25 «به اینطور خداوند به من عمل نمود درروزهایی که مرا منظور داشت، تا ننگ مرا از نظرمردم بردارد.»
\par 26 و در ماه ششم جبرائیل فرشته از جانب خدا به بلدی از جلیل که ناصره نام داشت، فرستاده شد.
\par 27 نزد باکره‌ای نامزد مردی مسمی به یوسف از خاندان داود و نام آن باکره مریم بود.
\par 28 پس فرشته نزد او داخل شده، گفت: «سلام برتو‌ای نعمت رسیده، خداوند با توست و تو درمیان زنان مبارک هستی.»
\par 29 چون او را دید، ازسخن او مضطرب شده، متفکر شد که این چه نوع تحیت است.
\par 30 فرشته بدو گفت: «ای مریم ترسان مباش زیرا که نزد خدا نعمت یافته‌ای.
\par 31 واینک حامله شده، پسری خواهی زایید و او راعیسی خواهی نامید.
\par 32 او بزرگ خواهد بود و به پسر حضرت اعلی، مسمی شود، و خداوند خداتخت پدرش داود را بدو عطا خواهد فرمود.
\par 33 واو بر خاندان یعقوب تا به ابد پادشاهی خواهد کردو سلطنت او را نهایت نخواهد بود.»
\par 34 مریم به فرشته گفت: «این چگونه می‌شود وحال آنکه مردی را نشناخته‌ام؟»
\par 35 فرشته درجواب وی گفت: «روح‌القدس بر تو خواهد آمد وقوت حضرت اعلی بر تو سایه خواهد افکند، از آنجهت آن مولود مقدس، پسر خدا خوانده خواهد شد.
\par 36 و اینک الیصابات از خویشان تونیز در‌پیری به پسری حامله شده و این ماه ششم است، مر او را که نازاد می‌خواندند.
\par 37 زیرا نزدخدا هیچ امری محال نیست.»
\par 38 مریم گفت: «اینک کنیز خداوندم. مرا برحسب سخن تو واقع شود.» پس فرشته از نزد او رفت.
\par 39 در آن روزها، مریم برخاست و به بلدی ازکوهستان یهودیه بشتاب رفت.
\par 40 و به خانه زکریادرآمده، به الیصابات سلام کرد.
\par 41 و چون الیصابات سلام مریم را شنید، بچه در رحم او به حرکت آمد و الیصابات به روح‌القدس پر شده،
\par 42 به آواز بلند صدا زده گفت: «تو در میان زنان مبارک هستی و مبارک است ثمره رحم تو.
\par 43 و ازکجا این به من رسید که مادر خداوند من، به نزد من آید؟
\par 44 زیرا اینک چون آواز سلام تو گوش زدمن شد، بچه از خوشی در رحم من به حرکت آمد.
\par 45 و خوشابحال او که ایمان آورد، زیرا که آنچه از جانب خداوند به وی گفته شد، به انجام خواهدرسید.»
\par 46 پس مریم گفت: «جان من خداوند راتمجید می‌کند،
\par 47 و روح من به رهاننده من خدابوجد آمد،
\par 48 زیرا بر‌حقارت کنیز خود نظرافکند. زیرا هان از کنون تمامی طبقات مراخوشحال خواهند خواند،
\par 49 زیرا آن قادر، به من کارهای عظیم کرده و نام او قدوس است،
\par 50 ورحمت او نسلا بعد نسل است. بر آنانی که از او می ترسند.
\par 51 به بازوی خود، قدرت را ظاهرفرمود و متکبران را به خیال دل ایشان پراکنده ساخت.
\par 52 جباران را از تختها به زیر افکند. وفروتنان را سرافراز گردانید.
\par 53 گرسنگان را به چیزهای نیکو سیر فرمود و دولتمندان راتهیدست رد نمود.
\par 54 بنده خود اسرائیل را یاری کرد، به یادگاری رحمانیت خویش،
\par 55 چنانکه به اجداد ما گفته بود، به ابراهیم و به ذریت او تاابدالاباد.»
\par 56 و مریم قریب به سه ماه نزد وی ماند. پس به خانه خود مراجعت کرد.
\par 57 اما چون الیصابات را وقت وضع حمل رسید، پسری بزاد.
\par 58 و همسایگان و خویشان اوچون شنیدند که خداوند رحمت عظیمی بر وی کرده، با او شادی کردند.
\par 59 و واقع شد در روزهشتم چون برای ختنه طفل آمدند، که نام پدرش زکریا را بر او می‌نهادند.
\par 60 اما مادرش ملتفت شده، گفت: «نی بلکه به یحیی نامیده می‌شود.»
\par 61 به وی گفتند: «از قبیله تو هیچ‌کس این اسم راندارد.»
\par 62 پس به پدرش اشاره کردند که «او را چه نام خواهی نهاد؟»
\par 63 او تخته‌ای خواسته بنوشت که «نام او یحیی است» و همه متعجب شدند.
\par 64 در ساعت، دهان و زبان او باز گشته، به حمدخدا متکلم شد.
\par 65 پس بر تمامی همسایگان ایشان، خوف مستولی گشت و جمیع این وقایع در همه کوهستان یهودیه شهرت یافت.
\par 66 و هرکه شنید، در خاطر خود تفکر نموده، گفت: «این چه نوع طفل خواهد بود؟» و دست خداوند با وی می بود.
\par 67 و پدرش زکریا از روح‌القدس پر شده نبوت نموده، گفت:
\par 68 «خداوند خدای اسرائیل متبارک باد، زیرا از قوم خود تفقد نموده، برای ایشان فدایی قرار داد.
\par 69 و شاخ نجاتی برای مابرافراشت، در خانه بنده خود داود.
\par 70 چنانچه به زبان مقدسین گفت که از بدو عالم انبیای اومی بودند،
\par 71 رهایی از دشمنان ما و از دست آنانی که از ما نفرت دارند،
\par 72 تا رحمت را برپدران ما به‌جا آرد و عهد مقدس خود را تذکرفرماید.
\par 73 سوگندی که برای پدر ما ابراهیم یادکرد،
\par 74 که ما را فیض عطا فرماید، تا از دست دشمنان خود رهایی یافته، او را بی‌خوف عبادت کنیم.
\par 75 در حضور او به قدوسیت و عدالت، درتمامی روزهای عمر خود.
\par 76 و تو‌ای طفل نبی حضرت اعلی خوانده خواهی شد، زیرا پیش روی خداوند خواهی خرامید، تا طرق او را مهیاسازی،
\par 77 تا قوم او را معرفت نجات دهی، درآمرزش گناهان ایشان.
\par 78 به احشای رحمت خدای ما که به آن سپیده از عالم اعلی از ما تفقدنمود،
\par 79 تا ساکنان در ظلمت و ظل موت را نوردهد. و پایهای ما را به طریق سلامتی هدایت نماید.»پس طفل نمو کرده، در روح قوی می‌گشت. و تا روز ظهور خود برای اسرائیل، دربیابان بسر می‌برد.
\par 80 پس طفل نمو کرده، در روح قوی می‌گشت. و تا روز ظهور خود برای اسرائیل، دربیابان بسر می‌برد.

\chapter{2}

\par 1 و در آن ایام حکمی از اوغسطس قیصرصادر گشت که تمام ربع مسکون را اسم نویسی کنند.
\par 2 و این اسم نویسی اول شد، هنگامی که کیرینیوس والی سوریه بود.
\par 3 پس همه مردم هر یک به شهر خود برای اسم نویسی می‌رفتند.
\par 4 و یوسف نیز از جلیل از بلده ناصره به یهودیه به شهر داود که بیت لحم نام داشت، رفت. زیرا که او از خاندان و آل داود بود.
\par 5 تا نام او بامریم که نامزد او بود و نزدیک به زاییدن بود، ثبت گردد.
\par 6 و وقتی که ایشان در آنجا بودند، هنگام وضع حمل او رسیده،
\par 7 پسر نخستین خود رازایید. و او را در قنداقه پیچیده، در آخورخوابانید. زیرا که برای ایشان در منزل جای نبود.
\par 8 و در آن نواحی، شبانان در صحرا بسرمی بردند و در شب پاسبانی گله های خویش می‌کردند.
\par 9 ناگاه فرشته خداوند بر ایشان ظاهرشد و کبریایی خداوند بر گرد ایشان تابید و بغایت ترسان گشتند.
\par 10 فرشته ایشان را گفت: «مترسید، زیرا اینک بشارت خوشی عظیم به شما می‌دهم که برای جمیع قوم خواهد بود.
\par 11 که امروز برای شما در شهر داود، نجات‌دهنده‌ای که مسیح خداوند باشد متولد شد.
\par 12 و علامت برای شمااین است که طفلی در قنداقه پیچیده و در آخورخوابیده خواهید یافت.»
\par 13 در همان حال فوجی از لشکر آسمانی با فرشته حاضر شده، خدا راتسبیح‌کنان می‌گفتند:
\par 14 «خدا را در اعلی علیین جلال و بر زمین سلامتی و در میان مردم رضامندی باد.»
\par 15 و چون فرشتگان از نزد ایشان به آسمان رفتند، شبانان با یکدیگر گفتند: «الان به بیت لحم برویم و این چیزی را که واقع شده و خداوند آن را به ما اعلام نموده است ببینیم.»
\par 16 پس به شتاب رفته، مریم و یوسف و آن طفل رادر آخور خوابیده یافتند.
\par 17 چون این را دیدند، آن سخنی را که درباره طفل بدیشان گفته شده بود، شهرت دادند.
\par 18 و هر‌که می‌شنید از آنچه شبانان بدیشان گفتند، تعجب می‌نمود.
\par 19 امامریم در دل خود متفکر شده، این همه سخنان رانگاه می‌داشت.
\par 20 و شبانان خدا را تمجید وحمدکنان برگشتند، به‌سبب همه آن اموری که دیده و شنیده بودند چنانکه به ایشان گفته شده بود.
\par 21 و چون روز هشتم، وقت ختنه طفل رسید، او را عیسی نام نهادند، چنانکه فرشته قبل از قرارگرفتن او در رحم، او را نامیده بود.
\par 22 و چون ایام تطهیر ایشان برحسب شریعت موسی رسید، او رابه اورشلیم بردند تا به خداوند بگذرانند.
\par 23 چنانکه در شریعت خداوند مکتوب است که هر ذکوری که رحم را گشاید، مقدس خداوندخوانده شود.
\par 24 و تا قربانی گذرانند، چنانکه درشریعت خداوند مقرر است، یعنی جفت فاخته‌ای یا دو جوجه کبوتر.
\par 25 و اینک شخصی شمعون نام در اورشلیم بود که مرد صالح و متقی و منتظر تسلی اسرائیل بود و روح‌القدس بر وی بود.
\par 26 و از روح‌القدس بدو وحی رسیده بود که، تا مسیح خداوند را نبینی موت را نخواهی دید.
\par 27 پس به راهنمایی روح، به هیکل درآمد وچون والدینش آن طفل یعنی عیسی را آوردند تا رسوم شریعت را بجهت او بعمل آورند،
\par 28 او رادر آغوش خود کشیده و خدا را متبارک خوانده، گفت:
\par 29 «الحال‌ای خداوند بنده خود را رخصت می‌دهی، به سلامتی برحسب کلام خود.
\par 30 زیراکه چشمان من نجات تو را دیده است،
\par 31 که آن راپیش روی جمیع امت‌ها مهیا ساختی.
\par 32 نوری که کشف حجاب برای امت‌ها کند و قوم تواسرائیل را جلال بود.»
\par 33 و یوسف و مادرش ازآنچه درباره او گفته شد، تعجب نمودند.
\par 34 پس شمعون ایشان را برکت داده، به مادرش مریم گفت: «اینک این طفل قرار داده شد، برای افتادن وبرخاستن بسیاری از آل اسرائیل و برای آیتی که به خلاف آن خواهند گفت.
\par 35 و در قلب تو نیزشمشیری فرو خواهد رفت، تا افکار قلوب بسیاری مکشوف شود.»
\par 36 و زنی نبیه بود، حنا نام، دختر فنوئیل ازسبط اشیر بسیار سالخورده، که از زمان بکارت هفت سال با شوهر بسر برده بود.
\par 37 و قریب به هشتاد و چهار سال بود که او بیوه گشته ازهیکل جدا نمی شد، بلکه شبانه‌روز به روزه ومناجات در عبادت مشغول می‌بود.
\par 38 او درهمان ساعت در‌آمده، خدا را شکر نمود ودرباره او به همه منتظرین نجات در اورشلیم، تکلم نمود.
\par 39 و چون تمامی رسوم شریعت خداوند را به پایان برده بودند، به شهر خود ناصره جلیل مراجعت کردند.
\par 40 و طفل نمو کرده، به روح قوی می‌گشت و از حکمت پر شده، فیض خدا بروی می‌بود.
\par 41 و والدین او هر ساله بجهت عید فصح، به اورشلیم می‌رفتند.
\par 42 و چون دوازده ساله شد، موافق رسم عید، به اورشلیم آمدند.
\par 43 وچون روزها را تمام کرده مراجعت می‌نمودند، آن طفل یعنی عیسی، در اورشلیم توقف نمودو یوسف و مادرش نمی دانستند.
\par 44 بلکه چون گمان می‌بردند که او در قافله است، سفریکروزه کردند و او را در میان خویشان وآشنایان خود می‌جستند.
\par 45 و چون او را نیافتند، در طلب او به اورشلیم برگشتند.
\par 46 و بعد ازسه روز، او را در هیکل یافتند که در میان معلمان نشسته، سخنان ایشان را می‌شنود و ازایشان سوال همی کرد.
\par 47 و هر‌که سخن او رامی شنید، از فهم و جوابهای او متحیرمی گشت.
\par 48 چون ایشان او را دیدند، مضطرب شدند. پس مادرش به وی گفت: «ای فرزند چرا با ماچنین کردی؟ اینک پدرت و من غمناک گشته تو را جستجو می‌کردیم.»
\par 49 او به ایشان گفت: «از بهر‌چه مرا طلب می‌کردید، مگرندانسته‌اید که باید من در امور پدر خود باشم؟»
\par 50 ولی آن سخنی را که بدیشان گفت، نفهمیدند.
\par 51 پس با ایشان روانه شده، به ناصره آمد و مطیع ایشان می‌بود و مادر او تمامی این امور را درخاطر خود نگاه می‌داشت.و عیسی درحکمت و قامت و رضامندی نزد خدا و مردم ترقی می‌کرد.
\par 52 و عیسی درحکمت و قامت و رضامندی نزد خدا و مردم ترقی می‌کرد.

\chapter{3}

\par 1 و در سال پانزدهم از سلطنت طیباریوس قیصر، در وقتی که پنطیوس پیلاطس، والی یهودیه بود و هیرودیس، تیترارک جلیل وبرادرش فیلپس تیترارک ایطوریه تراخونیتس ولیسانیوس تیترارک آبلیه
\par 2 و حنا و قیافا روسای کهنه بودند، کلام خدا به یحیی ابن زکریا در بیابان نازل شده،
\par 3 به تمامی حوالی اردن آمده، به تعمیدتوبه بجهت آمرزش گناهان موعظه می‌کرد.
\par 4 چنانچه مکتوب است در صحیفه کلمات اشعیای نبی که می‌گوید: «صدای ندا کننده‌ای دربیابان که راه خداوند را مهیا سازید و طرق او راراست نمایید.
\par 5 هر وادی انباشته و هر کوه و تلی پست و هر کجی راست و هر راه ناهموار صاف خواهد شد
\par 6 و تمامی بشر نجات خدا را خواهنددید.»
\par 7 آنگاه به آن جماعتی که برای تعمید وی بیرون می‌آمدند، گفت: «ای افعی‌زادگان، که شمارا نشان داد که از غضب آینده بگریزید؟
\par 8 پس ثمرات مناسب توبه بیاورید و در خاطر خود این سخن را راه مدهید که ابراهیم پدر ماست، زیرا به شما می‌گویم خدا قادر است که از این سنگها، فرزندان برای ابراهیم برانگیزاند.
\par 9 و الان نیز تیشه بر ریشه درختان نهاده شده است، پس هر درختی که میوه نیکو نیاورد، بریده و در آتش افکنده می‌شود.»
\par 10 پس مردم از وی سوال نموده گفتند: «چه کنیم؟»
\par 11 او در جواب ایشان گفت: «هر‌که دوجامه دارد، به آنکه ندارد بدهد. و هرکه خوراک دارد نیز چنین کند.»
\par 12 و باجگیران نیز برای تعمید آمده، بدو گفتند: «ای استاد چه کنیم؟»
\par 13 بدیشان گفت: «زیادتر از آنچه مقرر است، مگیرید.»
\par 14 سپاهیان نیز از او پرسیده، گفتند: «ماچه کنیم؟» به ایشان گفت: «بر کسی ظلم مکنید وبر هیچ‌کس افترا مزنید و به مواجب خود اکتفاکنید.»
\par 15 و هنگامی که قوم مترصد می‌بودند و همه در خاطر خود درباره یحیی تفکر می‌نمودندکه این مسیح است یا نه،
\par 16 یحیی به همه متوجه شده گفت: «من شما را به آب تعمیدمی دهم، لیکن شخصی تواناتر از من می‌آید که لیاقت آن ندارم که بند نعلین او را باز کنم. اوشما را به روح‌القدس و آتش تعمید خواهد داد.
\par 17 او غربال خود را به‌دست خود دارد وخرمن خویش را پاک کرده، گندم را در انبارخود ذخیره خواهد نمود و کاه را در آتشی که خاموشی نمی پذیرد خواهد سوزانید.»
\par 18 وبه نصایح بسیار دیگر، قوم را بشارت می‌داد.
\par 19 اما هیرودیس تیترارک چون به‌سبب هیرودیا، زن برادر او فیلپس و سایر بدیهایی که هیرودیس کرده بود از وی توبیخ یافت،
\par 20 این رانیز بر همه افزود که یحیی را در زندان حبس نمود.
\par 21 اما چون تمامی قوم تعمید یافته بودند وعیسی هم تعمید گرفته دعا می‌کرد، آسمان شکافته شد
\par 22 و روح‌القدس به هیات جسمانی، مانند کبوتری بر او نازل شد. و آوازی از آسمان در‌رسید که تو پسر حبیب من هستی که به توخشنودم.
\par 23 و خود عیسی وقتی که شروع کرد، قریب به سی ساله بود. و حسب گمان خلق، پسر یوسف ابن هالی
\par 24 ابن متات، بن لاوی، بن ملکی، بن ینا، بن یوسف،
\par 25 ابن متاتیا، بن آموس، بن ناحوم، بن حسلی، بن نجی،
\par 26 ابن مات، بن متاتیا، بن شمعی، بن یوسف، بن یهودا،
\par 27 ابن یوحنا، بن ریسا، بن زروبابل، بن سالتیئیل، بن نیری،
\par 28 ابن ملکی، بن ادی، بن قوسام، بن ایلمودام، بن عیر،
\par 29 ابن یوسی، بن ایلعاذر، بن یوریم، بن متات، بن لاوی،
\par 30 ابن شمعون، بن یهودا، بن یوسف، بن یونان، بن ایلیاقیم،
\par 31 ابن ملیا، بن مینان، بن متاتا بن ناتان، بن داود،
\par 32 ابن یسی، بن عوبید، بن بوعز، بن شلمون، بن نحشون،
\par 33 ابن عمیناداب، بن ارام، بن حصرون، بن فارص، بن یهودا،
\par 34 ابن یعقوب، بن اسحق، بن ابراهیم، بن تارح، بن ناحور،
\par 35 ابن سروج، بن رعور، بن فالج، بن عابر، بن صالح،
\par 36 ابن قینان، بن ارفکشاد، بن سام، بن نوح، بن لامک،
\par 37 ابن متوشالح، بن خنوخ، بن یارد، بن مهللئیل، بن قینان،ابن انوش، بن شیث، بن آدم، بن الله.
\par 38 ابن انوش، بن شیث، بن آدم، بن الله.

\chapter{4}

\par 1 اما عیسی پر از روح‌القدس بوده، از اردن مراجعت کرد و روح او را به بیابان برد.
\par 2 ومدت چهل روز ابلیس او را تجربه می‌نمود و درآن ایام چیزی نخورد. چون تمام شد، آخر گرسنه گردید.
\par 3 و ابلیس بدو گفت: «اگر پسر خدا هستی، این سنگ را بگو تا نان گردد.»
\par 4 عیسی در جواب وی گفت: «مکتوب است که انسان به نان فقط زیست نمی کند، بلکه به هر کلمه خدا.»
\par 5 پس ابلیس او رابه کوهی بلند برده، تمامی ممالک جهان را درلحظه‌ای بدو نشان داد.
\par 6 و ابلیس بدو گفت: «جمیع این قدرت و حشمت آنها را به تو می‌دهم، زیرا که به من سپرده شده است و به هر‌که می‌خواهم می‌بخشم.
\par 7 پس اگر تو پیش من سجده کنی، همه از آن تو خواهد شد.»
\par 8 عیسی در جواب او گفت: «ای شیطان، مکتوب است، خداوند خدای خود را پرستش کن و غیر او راعبادت منما.»
\par 9 پس او را به اورشلیم برده، برکنگره هیکل قرار داد و بدو گفت: «اگر پسر خداهستی، خود را از اینجا به زیر انداز.
\par 10 زیرامکتوب است که فرشتگان خود را درباره تو حکم فرماید تا تو را محافظت کنند.
\par 11 و تو را به‌دستهای خود بردارند، مبادا پایت به سنگی خورد.»
\par 12 عیسی در جواب وی گفت که «گفته شده است، خداوند خدای خود را تجربه مکن.»
\par 13 و چون ابلیس جمیع تجربه را به اتمام رسانید، تا مدتی از او جدا شد.
\par 14 و عیسی به قوت روح، به جلیل برگشت وخبر او در تمامی آن نواحی شهرت یافت.
\par 15 و اودر کنایس ایشان تعلیم می‌داد و همه او را تعظیم می‌کردند.
\par 16 و به ناصره جایی که پرورش یافته بود، رسید و بحسب دستور خود در روز سبت به کنیسه درآمده، برای تلاوت برخاست.
\par 17 آنگاه صحیفه اشعیا نبی را بدو دادند و چون کتاب راگشود، موضعی را یافت که مکتوب است
\par 18 «روح خداوند بر من است، زیرا که مرا مسح کرد تا فقیران را بشارت دهم و مرا فرستاد، تاشکسته دلان را شفا بخشم و اسیران را به رستگاری و کوران را به بینایی، موعظه کنم و تاکوبیدگان را، آزاد سازم،
\par 19 و از سال پسندیده خداوند موعظه کنم.»
\par 20 پس کتاب را به هم پیچیده، به خادم سپرد وبنشست و چشمان همه اهل کنیسه بر وی دوخته می‌بود.
\par 21 آنگاه بدیشان شروع به گفتن کرد که «امروز این نوشته در گوشهای شما تمام شد.»
\par 22 و همه بر وی شهادت دادند و از سخنان فیض آمیزی که از دهانش صادر می‌شد، تعجب نموده، گفتند: «مگر این پسر یوسف نیست؟»
\par 23 بدیشان گفت: «هرآینه این مثل را به من خواهیدگفت، ای طبیب خود را شفا بده. آنچه شنیده‌ایم که در کفرناحوم از تو صادر شد، اینجا نیز در وطن خویش بنما.»
\par 24 و گفت: «هرآینه به شمامی گویم که هیچ نبی در وطن خویش مقبول نباشد.
\par 25 و به تحقیق شما را می‌گویم که بسا بیوه‌زنان در اسرائیل بودند، در ایام الیاس، وقتی که آسمان مدت سه سال و شش ماه بسته ماند، چنانکه قحطی عظیم در تمامی زمین پدید آمد،
\par 26 و الیاس نزد هیچ کدام از ایشان فرستاده نشد، مگر نزد بیوه‌زنی در صرفه صیدون.
\par 27 و بساابرصان در اسرائیل بودند، در ایام الیشع نبی واحدی از ایشان طاهر نگشت، جز نعمان سریانی.»
\par 28 پس تمام اهل کنیسه چون این سخنان راشنیدند، پر از خشم گشتند.
\par 29 و برخاسته او را ازشهر بیرون کردند و بر قله کوهی که قریه ایشان برآن بنا شده بود بردند، تا او را به زیر افکنند.
\par 30 ولی از میان ایشان گذشته، برفت.
\par 31 و به کفرناحوم شهری از جلیل فرود شده، در روزهای سبت، ایشان را تعلیم می‌داد.
\par 32 و ازتعلیم او در حیرت افتادند، زیرا که کلام او باقدرت می‌بود.
\par 33 و در کنیسه مردی بود، که روح دیو خبیث داشت و به آواز بلند فریادکنان می‌گفت:
\par 34 «آه‌ای عیسی ناصری، ما را با تو چه‌کار است، آیا آمده‌ای تا ما را هلاک سازی؟ تو را می شناسم کیستی، ای قدوس خدا.»
\par 35 پس عیسی او را نهیب داده، فرمود: «خاموش باش و ازوی بیرون آی.» در ساعت دیو او را در میان انداخته، از او بیرون شد و هیچ آسیبی بدونرسانید.
\par 36 پس حیرت بر همه ایشان مستولی گشت و یکدیگر را مخاطب ساخته، گفتند: «این چه سخن است که این شخص با قدرت و قوت، ارواح پلید را امر می‌کند و بیرون می‌آیند!»
\par 37 وشهرت او در هر موضعی از آن حوالی، پهن شد.
\par 38 و از کنیسه برخاسته، به خانه شمعون درآمد. و مادر‌زن شمعون را تب شدیدی عارض شده بود، برای او از وی التماس کردند.
\par 39 پس برسر وی آمده، تب را نهیب داده، تب از او زایل شد. در ساعت برخاسته، به خدمتگذاری ایشان مشغول شد.
\par 40 و چون آفتاب غروب می‌کرد، همه آنانی که اشخاص مبتلا به انواع مرضها داشتند، ایشان را نزد وی آوردند و به هر یکی از ایشان دست گذارده، شفا داد.
\par 41 و دیوها نیز از بسیاری بیرون می‌رفتند و صیحه زنان می‌گفتند که «تو مسیح پسرخدا هستی.» ولی ایشان را قدغن کرده، نگذاشت که حرف زنند، زیرا که دانستند او مسیح است.
\par 42 و چون روز شد، روانه شده به مکانی ویران رفت و گروهی کثیر در جستجوی او آمده، نزدش رسیدند و او را باز می‌داشتند که از نزد ایشان نرود.
\par 43 به ایشان گفت: «مرا لازم است که به شهرهای دیگر نیز به ملکوت خدا بشارت دهم، زیرا که برای همین کار فرستاده شده‌ام.»پس در کنایس جلیل موعظه می‌نمود.
\par 44 پس در کنایس جلیل موعظه می‌نمود.

\chapter{5}

\par 1 و هنگامی که گروهی بر وی ازدحام می نمودند تا کلام خدا را بشنوند، او به کناردریاچه جنیسارت ایستاده بود.
\par 2 و دو زورق رادر کنار دریاچه ایستاده دید که صیادان از آنهابیرون آمده، دامهای خود را شست و شومی نمودند.
\par 3 پس به یکی از آن دو زورق که مال شمعون بود سوار شده، از او درخواست نمود که از خشکی اندکی دور ببرد. پس در زورق نشسته، مردم را تعلیم می‌داد.
\par 4 و چون از سخن‌گفتن فارغ شد، به شمعون گفت: «به میانه دریاچه بران و دامهای خود رابرای شکار بیندازید.»
\par 5 شمعون در جواب وی گفت: «ای استاد، تمام شب را رنج برده چیزی نگرفتیم، لیکن به حکم تو، دام را خواهیم انداخت.»
\par 6 و چون چنین کردند، مقداری کثیر ازماهی صید کردند، چنانکه نزدیک بود دام ایشان گسسته شود.
\par 7 و به رفقای خود که در زورق دیگربودند اشاره کردند که آمده ایشان را امداد کنند. پس آمده هر دو زورق را پر کردند بقسمی که نزدیک بود غرق شوند.
\par 8 شمعون پطرس چون این را بدید، بر پایهای عیسی افتاده، گفت: «ای خداوند از من دور شو زیرا مردی گناهکارم.»
\par 9 چونکه به‌سبب صیدماهی که کرده بودند، دهشت بر او و همه رفقای وی مستولی شده بود.
\par 10 و هم چنین نیز بریعقوب و یوحنا پسران زبدی که شریک شمعون بودند. عیسی به شمعون گفت: «مترس. پس از این مردم را صید خواهی کرد.»
\par 11 پس چون زورقهارا به کنار آوردند همه را ترک کرده، از عقب اوروانه شدند.
\par 12 و چون او در شهری از شهرها بود ناگاه مردی پر از برص آمده، چون عیسی را بدید، به روی در‌افتاد و از او درخواست کرده، گفت: «خداوندا، اگر خواهی می‌توانی مرا طاهرسازی.»
\par 13 پس او دست آورده، وی را لمس نمود و گفت: «می‌خواهم. طاهر شو.» که فور برص از او زایل شد.
\par 14 و او را قدغن کرد که «هیچ‌کس را خبر مده، بلکه رفته خود را به کاهن بنما و هدیه‌ای بجهت طهارت خود، بطوری که موسی فرموده است، بگذران تا بجهت ایشان شهادتی شود.»
\par 15 لیکن خبر او بیشتر شهرت یافت و گروهی بسیار جمع شدند تا کلام او رابشنوند و از مرضهای خود شفا یابند،
\par 16 و او به ویرانه‌ها عزلت جسته، به عبادت مشغول شد.
\par 17 روزی از روزها واقع شد که او تعلیم می‌دادو فریسیان و فقها که از همه بلدان جلیل و یهودیه و اورشلیم آمده، نشسته بودند وقوت خداوندبرای شفای ایشان صادر می‌شد،
\par 18 که ناگاه چندنفر شخصی مفلوج را بر بستری آوردند ومی خواستند او را داخل کنند تا پیش روی وی بگذارند.
\par 19 و چون به‌سبب انبوهی مردم راهی نیافتند که او را به خانه درآورند بر پشت بام رفته، او را با تختش از میان سفالها در وسط پیش عیسی گذاردند.
\par 20 چون او ایمان ایشان را دید، به وی گفت: «ای مرد، گناهان تو آمرزیده شد.»
\par 21 آنگاه کاتبان و فریسیان در خاطر خود تفکرنموده، گفتن گرفتند: «این کیست که کفر می‌گوید. جز خدا و بس کیست که بتواند گناهان رابیامرزد؟»
\par 22 عیسی افکار ایشان را درک نموده، در جواب ایشان گفت: «چرا در خاطر خود تفکرمی کنید؟
\par 23 کدام سهلتر است، گفتن اینکه گناهان تو آمرزیده شد، یا گفتن اینکه برخیز و بخرام؟
\par 24 لیکن تا بدانید که پسر انسان را استطاعت آمرزیدن گناهان بر روی زمین هست، مفلوج راگفت، تو را می‌گویم برخیز و بستر خود رابرداشته، به خانه خود برو.»
\par 25 در ساعت برخاسته، پیش ایشان آنچه بر آن خوابیده بودبرداشت و به خانه خود خدا را حمدکنان روانه شد.
\par 26 و حیرت همه را فرو گرفت و خدا راتمجید می‌نمودند و خوف بر ایشان مستولی شده، گفتند: «امروز چیزهای عجیب دیدیم.»
\par 27 از آن پس بیرون رفته، باجگیری را که لاوی نام داشت، بر باجگاه نشسته دید. او را گفت: «از عقب من بیا.»
\par 28 در حال همه‌چیز را ترک کرده، برخاست و در عقب وی روانه شد.
\par 29 و لاوی ضیافتی بزرگ در خانه خود برای او کرد و جمعی بسیار از باجگیران و دیگران با ایشان نشستند.
\par 30 اما کاتبان ایشان و فریسیان همهمه نموده، به شاگردان او گفتند: «برای چه با باجگیران وگناهکاران اکل و شرب می‌کنید؟»
\par 31 عیسی درجواب ایشان گفت: «تندرستان احتیاج به طبیب ندارند بلکه مریضان.
\par 32 و نیامده‌ام تا عادلان بلکه تا عاصیان را به توبه بخوانم.»
\par 33 پس به وی گفتند: «از چه سبب شاگردان یحیی روزه بسیار می‌دارند و نماز می‌خوانند وهمچنین شاگردان فریسیان نیز، لیکن شاگردان تواکل و شرب می‌کنند.»
\par 34 بدیشان گفت: «آیامی توانید پسران خانه عروسی را مادامی که دامادبا ایشان است روزه‌دار سازید؟
\par 35 بلکه ایامی می‌آید که داماد از ایشان گرفته شود، آنگاه در آن روزها روزه خواهند داشت.»
\par 36 و مثلی برای ایشان آورد که «هیچ‌کس پارچه‌ای از جامه نو را بر جامه کهنه وصله نمی کند والا آن نو را پاره کند و وصله‌ای که از نوگرفته شد نیز در خور آن کهنه نبود.
\par 37 و هیچ‌کس شراب نو را در مشکهای کهنه نمی ریزد والاشراب نو، مشکها را پاره می‌کند و خودش ریخته و مشکها تباه می‌گردد.
\par 38 بلکه شراب نو را درمشکهای نو باید ریخت تا هر دو محفوظ بماند.و کسی نیست که چون شراب کهنه را نوشیده فی الفور نو را طلب کند، زیرا می‌گوید کهنه بهتراست.»
\par 39 و کسی نیست که چون شراب کهنه را نوشیده فی الفور نو را طلب کند، زیرا می‌گوید کهنه بهتراست.»

\chapter{6}

\par 1 و واقع شد در سبت دوم اولین که او از میان کشت زارها می‌گذشت و شاگردانش خوشه‌ها می‌چیدند و به کف مالیده می‌خوردند.
\par 2 و بعضی از فریسیان بدیشان گفتند: «چرا کاری می‌کنید که کردن آن در سبت جایز نیست.»
\par 3 عیسی در جواب ایشان گفت: «آیا نخوانده ایدآنچه داود و رفقایش کردند در وقتی که گرسنه بودند،
\par 4 که چگونه به خانه خدا درآمده نان تقدمه را گرفته بخورد و به رفقای خود نیز داد که خوردن آن جز به کهنه روا نیست.»
\par 5 پس بدیشان گفت: «پسر انسان مالک روز سبت نیز هست.»
\par 6 و در سبت دیگر به کنیسه درآمده تعلیم می‌داد و در آنجا مردی بود که دست راستش خشک بود.
\par 7 و کاتبان و فریسیان چشم بر اومی داشتند که شاید در سبت شفا دهد تا شکایتی بر او یابند.
\par 8 او خیالات ایشان را درک نموده، بدان مرد دست خشک گفت: «برخیز و در میان بایست.» در حال برخاسته بایستاد.
\par 9 عیسی بدیشان گفت: «از شما چیزی می‌پرسم که در روزسبت کدام رواست، نیکویی‌کردن یا بدی، رهانیدن جان یا هلاک کردن؟»
\par 10 پس چشم خود را بر جمیع ایشان گردانیده، بدو گفت: «دست خود را دراز کن.» او چنان کرد و فور دستش مثل دست دیگر صحیح گشت.
\par 11 اما ایشان ازحماقت پر گشته به یکدیگر می‌گفتند که «باعیسی چه کنیم؟»
\par 12 و در آن روزها برفراز کوه برآمد تا عبادت کند وآن شب را در عبادت خدا به صبح آورد.
\par 13 و چون روز شد، شاگردان خود را پیش طلبیده دوازده نفر از ایشان را انتخاب کرده، ایشان را نیزرسول خواند.
\par 14 یعنی شمعون که او را پطرس نیز نام نهاد و برادرش اندریاس، یعقوب و یوحنا، فیلپس و برتولما،
\par 15 متی و توما، یعقوب ابن حلفی و شمعون معروف به غیور.
\par 16 یهودا برادریعقوب و یهودای اسخریوطی که تسلیم‌کننده وی بود.
\par 17 و با ایشان به زیر آمده، بر جای همواربایستاد. و جمعی از شاگردان وی و گروهی بسیاراز قوم، از تمام یهودیه و اورشلیم و کناره دریای صور و صیدون آمدند تا کلام او را بشنوند و ازامراض خود شفا یابند.
\par 18 و کسانی که از ارواح پلید معذب بودند، شفا یافتند.
\par 19 و تمام آن گروه می‌خواستند او را لمس کنند. زیرا قوتی از وی صادر شده، همه را صحت می‌بخشید.
\par 20 پس نظر خود را به شاگردان خویش افکنده، گفت: «خوشابحال شما‌ای مساکین زیراملکوت خدا از آن شما است.
\par 21 خوشابحال شماکه اکنون گرسنه‌اید، زیرا که سیر خواهید شد. خوشابحال شما که الحال گریانید، زیرا خواهیدخندید.
\par 22 خوشابحال شما وقتی که مردم بخاطرپسر انسان از شما نفرت گیرند و شما را از خودجدا سازند و دشنام دهند و نام شما را مثل شریربیرون کنند.
\par 23 در آن روز شاد باشید و وجدنمایید زیرا اینک اجر شما در آسمان عظیم می‌باشد، زیرا که به همینطور پدران ایشان با انبیاسلوک نمودند.
\par 24 «لیکن وای بر شما‌ای دولتمندان زیرا که تسلی خود را یافته‌اید.
\par 25 وای بر شما‌ای سیرشدگان، زیرا گرسنه خواهید شد. وای بر شماکه الان خندانید زیرا که ماتم و گریه خواهید کرد.
\par 26 وای بر شما وقتی که جمیع مردم شما راتحسین کنند، زیرا همچنین پدران ایشان با انبیای کذبه کردند.
\par 27 «لیکن‌ای شنوندگان شما را می‌گویم دشمنان خود را دوست دارید و با کسانی که ازشما نفرت کنند، احسان کنید.
\par 28 و هر‌که شما رالعن کند، برای او برکت بطلبید و برای هرکه با شماکینه دارد، دعای خیر کنید.
\par 29 و هرکه بر رخسارتو زند، دیگری را نیز به سوی او بگردان و کسی‌که ردای تو را بگیرد، قبا را نیز از او مضایقه مکن.
\par 30 هرکه از تو سوال کند بدو بده و هر‌که مال تو راگیرد از وی باز مخواه.
\par 31 و چنانکه می‌خواهیدمردم با شما عمل کنند، شما نیز به همانطور باایشان سلوک نمایید.
\par 32 «زیرا اگر محبان خود را محبت نمایید، شما را چه فضیلت است؟ زیرا گناهکاران هم محبان خود را محبت می‌نمایند.
\par 33 و اگر احسان کنید با هر‌که به شما احسان کند، چه فضیلت دارید؟ چونکه گناهکاران نیز چنین می‌کنند.
\par 34 واگر قرض دهید به آنانی که امید بازگرفتن از ایشان دارید، شما را چه فضیلت است؟ زیرا گناهکاران نیز به گناهکاران قرض می‌دهند تا از ایشان عوض گیرند.
\par 35 بلکه دشمنان خود را محبت نمایید واحسان کنید و بدون امید عوض، قرض دهید زیراکه اجر شما عظیم خواهد بود و پسران حضرت اعلی خواهید بود چونکه او با ناسپاسان وبدکاران مهربان است.
\par 36 پس رحیم باشید چنانکه پدر شما نیز رحیم است.
\par 37 «داوری مکنید تا بر شما داوری نشود وحکم مکنید تا بر شما حکم نشود و عفو کنید تاآمرزیده شوید.
\par 38 بدهید تا به شما داده شود. زیرا پیمانه نیکوی افشرده و جنبانیده و لبریز شده را در دامن شما خواهند گذارد. زیرا که به همان پیمانه‌ای که می‌پیمایید برای شما پیموده خواهدشد.»
\par 39 پس برای ایشان مثلی زد که «آیا می‌تواند کور، کور را راهنمایی کند؟ آیا هر دو در حفره‌ای نمی افتند؟
\par 40 شاگرد از معلم خویش بهتر نیست لیکن هر‌که کامل شده باشد، مثل استاد خود بود.
\par 41 و چرا خسی را که در چشم برادر تو است می‌بینی و چوبی را که در چشم خود داری نمی یابی؟
\par 42 و چگونه بتوانی برادر خود را گویی‌ای برادر اجازت ده تا خس را از چشم تو برآورم و چوبی را که در چشم خود داری نمی بینی. ای ریاکار اول چوب را از چشم خود بیرون کن، آنگاه نیکو خواهی دید تا خس را از چشم برادر خودبرآوری.
\par 43 «زیرا هیچ درخت نیکو میوه بد بارنمی آورد و نه درخت بد، میوه نیکو آورد.
\par 44 زیراکه هر درخت از میوه‌اش شناخته می‌شود از خارانجیر را نمی یابند و از بوته، انگور را نمی چینند.
\par 45 آدم نیکو از خزینه خوب دل خود چیز نیکوبرمی آورد و شخص شریر از خزینه بد دل خویش چیز بد بیرون می‌آورد. زیرا که از زیادتی دل زبان سخن می‌گوید.
\par 46 «و چون است که مرا خداوندا خداوندامی گویید و آنچه می‌گویم بعمل نمی آورید.
\par 47 هر‌که نزد من آید و سخنان مرا شنود و آنها را به‌جا آورد، شما را نشان می‌دهم که به چه کس مشابهت دارد.
\par 48 مثل شخصی است که خانه‌ای می‌ساخت و زمین را کنده گود نمود و بنیادش رابر سنگ نهاد. پس چون سیلاب آمده، سیل بر آن خانه زور آورد، نتوانست آن را جنبش دهد زیراکه بر سنگ بنا شده بود.لیکن هر‌که شنید وعمل نیاورد مانند شخصی است که خانه‌ای برروی زمین بی‌بنیاد بنا کرد که چون سیل بر آن صدمه زد، فور افتاد و خرابی آن خانه عظیم بود.»
\par 49 لیکن هر‌که شنید وعمل نیاورد مانند شخصی است که خانه‌ای برروی زمین بی‌بنیاد بنا کرد که چون سیل بر آن صدمه زد، فور افتاد و خرابی آن خانه عظیم بود.»

\chapter{7}

\par 1 و چون همه سخنان خود را به سمع خلق به اتمام رسانید، وارد کفرناحوم شد.
\par 2 ویوزباشی را غلامی که عزیز او بود مریض ومشرف بر موت بود.
\par 3 چون خبر عیسی را شنید، مشایخ یهود را نزد وی فرستاده از او خواهش کرد که آمده غلام او را شفا بخشد.
\par 4 ایشان نزدعیسی آمده به الحاح نزد او التماس کرده گفتند: «مستحق است که این احسان را برایش به‌جاآوری.
\par 5 زیرا قوم ما را دوست می‌دارد و خودبرای ما کنیسه را ساخت.»
\par 6 پس عیسی با ایشان روانه شد و چون نزدیک به خانه رسید، یوزباشی چند نفر از دوستان خودرا نزد او فرستاده بدو گفت: «خداوندا زحمت مکش زیرا لایق آن نیستم که زیر سقف من درآیی.
\par 7 و از این سبب خود را لایق آن ندانستم که نزد تو آیم، بلکه سخنی بگو تا بنده من صحیح شود.
\par 8 زیرا که من نیز شخصی هستم زیر حکم ولشکریان زیر دست خود دارم. چون به یکی گویم برو، می‌رود و به دیگری بیا، می‌آید و به غلام خود این را بکن، می‌کند.»
\par 9 چون عیسی این راشنید، تعجب نموده به سوی آن جماعتی که از عقب او می‌آمدند روی گردانیده، گفت: «به شمامی گویم چنین ایمانی، در اسرائیل هم نیافته‌ام.»
\par 10 پس فرستادگان به خانه برگشته، آن غلام بیماررا صحیح یافتند.
\par 11 و دو روز بعد به شهری مسمی به نائین می‌رفت و بسیاری از شاگردان او و گروهی عظیم، همراهش می‌رفتند.
\par 12 چون نزدیک به دروازه شهر رسید، ناگاه میتی را که پسر یگانه بیوه‌زنی بود می‌بردند و انبوهی کثیر از اهل شهر، با وی می‌آمدند.
\par 13 چون خداوند او را دید، دلش بر اوبسوخت و به وی گفت: «گریان مباش.»
\par 14 و نزدیک آمده تابوت را لمس نمود وحاملان آن بایستادند. پس گفت: «ای جوان تو رامی گویم برخیز.»
\par 15 در ساعت آن مرده راست بنشست و سخن‌گفتن آغاز کرد و او را به مادرش سپرد.
\par 16 پس خوف همه را فراگرفت و خدا راتمجیدکنان می‌گفتند که «نبی‌ای بزرگ در میان مامبعوث شده و خدا از قوم خود تفقد نموده است.»
\par 17 پس این خبر درباره او در تمام یهودیه و جمیع آن مرز و بوم منتشر شد.
\par 18 و شاگردان یحیی او را از جمیع این وقایع مطلع ساختند.
\par 19 پس یحیی دو نفر از شاگردان خود را طلبیده، نزد عیسی فرستاده، عرض نمودکه «آیا تو آن آینده هستی یا منتظر دیگری باشیم؟»
\par 20 آن دو نفر نزد وی آمده، گفتند: «یحیی تعمید‌دهنده ما را نزد تو فرستاده، می‌گوید آیا تو آن آینده هستی یا منتظر دیگری باشیم.»
\par 21 در همان ساعت، بسیاری را از مرضهاو بلایا و ارواح پلید شفا داد و کوران بسیاری رابینایی بخشید.
\par 22 عیسی در جواب ایشان گفت: «بروید و یحیی را از آنچه دیده و شنیده‌اید خبردهید که کوران، بینا و لنگان خرامان و ابرصان طاهر و کران، شنوا و مردگان، زنده می‌گردند و به فقرا بشارت داده می‌شود.
\par 23 و خوشابحال کسی‌که در من لغزش نخورد.»
\par 24 و چون فرستادگان یحیی رفته بودند، درباره یحیی بدان جماعت آغاز سخن نهاد که «برای دیدن چه چیز به صحرا بیرون رفته بودید، آیا نی را که از باد در جنبش است؟
\par 25 بلکه بجهت دیدن چه بیرون رفتید، آیا کسی را که به لباس نرم ملبس باشد؟ اینک آنانی که لباس فاخر می‌پوشندو عیاشی می‌کنند، در قصرهای سلاطین هستند.
\par 26 پس برای دیدن چه رفته بودید، آیا نبی‌ای را؟ بلی به شما می‌گویم کسی را که از نبی هم بزرگتراست.
\par 27 زیرا این است آنکه درباره وی مکتوب است، اینک من رسول خود را پیش روی تومی فرستم تا راه تو را پیش تو مهیا سازد.
\par 28 زیراکه شما را می‌گویم از اولاد زنان نبی‌ای بزرگتر ازیحیی تعمید‌دهنده نیست، لیکن آنکه در ملکوت خدا کوچکتر است از وی بزرگتر است.»
\par 29 وتمام قوم و باجگیران چون شنیدند، خدا راتمجید کردند زیرا که تعمید از یحیی یافته بودند.
\par 30 لیکن فریسیان و فقها اراده خدا را از خود ردنمودند زیرا که از وی تعمید نیافته بودند.
\par 31 آنگاه خداوند گفت: «مردمان این طبقه را به چه تشبیه کنم و مانند چه می‌باشند؟
\par 32 اطفالی را می مانند که در بازارها نشسته، یکدیگر را صدازده می‌گویند، برای شما نواختیم رقص نکردید ونوحه گری کردیم گریه ننمودید.
\par 33 زیرا که یحیی تعمید‌دهنده آمد که نه نان می‌خورد و نه شراب می‌آشامید، می‌گویید دیو دارد.
\par 34 پسر انسان آمد که می‌خورد و می‌آشامد، می‌گویید اینک مردی است پرخور و باده پرست و دوست باجگیران و گناهکاران.
\par 35 اما حکمت از جمیع فرزندان خود مصدق می‌شود.
\par 36 و یکی از فریسیان از او وعده خواست که بااو غذا خورد پس به خانه فریسی درآمده بنشست.
\par 37 که ناگاه زنی که در آن شهر گناهکاربود، چون شنید که در خانه فریسی به غذا نشسته است شیشه‌ای از عطر آورده،
\par 38 در پشت سر اونزد پایهایش گریان بایستاد و شروع کرد به شستن پایهای او به اشک خود و خشکانیدن آنها به موی سر خود و پایهای وی را بوسیده آنها را به عطرتدهین کرد.
\par 39 چون فریسی‌ای که از او وعده خواسته بوداین را بدید، با خود می‌گفت که «این شخص اگرنبی بودی هرآینه دانستی که این کدام و چگونه زن است که او را لمس می‌کند، زیرا گناهکاری است.»
\par 40 عیسی جواب داده به وی گفت: «ای شمعون چیزی دارم که به تو گویم.» گفت: «ای استاد بگو.»
\par 41 گفت: «طلبکاری را دو بدهکار بودکه از یکی پانصد و از دیگری پنجاه دینار طلب داشتی.
\par 42 چون چیزی نداشتند که ادا کنند، هر دو را بخشید. بگو کدام‌یک از آن دو او را زیادترمحبت خواهد نمود.»
\par 43 شمعون در جواب گفت: «گمان می‌کنم آنکه او را زیادتر بخشید.» به وی گفت: «نیکو گفتی.»
\par 44 پس به سوی آن زن اشاره نموده به شمعون گفت: «این زن را نمی بینی، به خانه تو آمدم آب بجهت پایهای من نیاوردی، ولی این زن پایهای مرا به اشکها شست و به مویهای سر خود آنها راخشک کرد.
\par 45 مرا نبوسیدی، لیکن این زن ازوقتی که داخل شدم از بوسیدن پایهای من بازنایستاد.
\par 46 سر مرا به روغن مسح نکردی، لیکن اوپایهای مرا به عطر تدهین کرد.
\par 47 از این جهت به تو می‌گویم، گناهان او که بسیار است آمرزیده شد، زیرا که محبت بسیار نموده است. لیکن آنکه آمرزش کمتر یافت، محبت کمتر می‌نماید.»
\par 48 پس به آن زن گفت: «گناهان تو آمرزیده شد.»
\par 49 و اهل مجلس در خاطر خود تفکر آغاز کردندکه این کیست که گناهان را هم می‌آمرزد.پس به آن زن گفت: «ایمانت تو را نجات داده است به سلامتی روانه شو.»
\par 50 پس به آن زن گفت: «ایمانت تو را نجات داده است به سلامتی روانه شو.»

\chapter{8}

\par 1 و بعد از آن واقع شد که او در هر شهری ودهی گشته، موعظه می‌نمود و به ملکوت خدا بشارت می‌داد و آن دوازده با وی می‌بودند.
\par 2 و زنان چند که از ارواح پلید و مرضها شفا یافته بودند، یعنی مریم معروف به مجدلیه که از اوهفت دیو بیرون رفته بودند،
\par 3 و یونا زوجه خوزا، ناظر هیرودیس و سوسن و بسیاری از زنان دیگرکه از اموال خود او را خدمت می‌کردند.
\par 4 و چون گروهی بسیار فراهم می‌شدند و از هرشهر نزد او می‌آمدند مثلی آورده، گفت
\par 5 که «برزگری بجهت تخم کاشتن بیرون رفت. و وقتی که تخم می‌کاشت بعضی بر کناره راه ریخته شد وپایمال شده، مرغان هوا آن را خوردند.
\par 6 وپاره‌ای بر سنگلاخ افتاده چون رویید از آنجهت که رطوبتی نداشت خشک گردید.
\par 7 و قدری درمیان خارها افکنده شد که خارها با آن نمو کرده آن را خفه نمود.
\par 8 و بعضی در زمین نیکو پاشیده شده رویید و صد چندان ثمر آورد.» چون این بگفت ندا در‌داد «هر‌که گوش شنوا دارد بشنود.»
\par 9 پس شاگردانش از او سوال نموده، گفتند که «معنی‌این مثل چیست؟»
\par 10 گفت: «شما رادانستن اسرار ملکوت خدا عطا شده است و لیکن دیگران را به واسطه مثلها، تا نگریسته نبینند وشنیده درک نکنند.
\par 11 اما مثل این است که تخم کلام خداست.
\par 12 و آنانی که در کنار راه هستندکسانی می‌باشند که چون می‌شنوند، فور ابلیس آمده کلام را از دلهای ایشان می‌رباید، مبادا ایمان آورده نجات یابند.
\par 13 و آنانی که بر سنگلاخ هستند کسانی می‌باشند که چون کلام رامی شنوند آن را به شادی می‌پذیرند و اینها ریشه ندارند پس تا مدتی ایمان می‌دارند و در وقت آزمایش، مرتد می‌شوند.
\par 14 اما آنچه در خارهاافتاد اشخاصی می‌باشند که چون شنوند می‌روند و اندیشه های روزگار و دولت و لذات آن ایشان راخفه می‌کند و هیچ میوه به‌کمال نمی رسانند.
\par 15 اما آنچه در زمین نیکو واقع گشت کسانی می‌باشند که کلام را به دل راست و نیکو شنیده، آن را نگاه می‌دارند و با صبر، ثمر می‌آورند.
\par 16 «و هیچ‌کس چراغ را افروخته، آن را زیرظرفی یا تختی پنهان نمی کند بلکه بر چراغدان می‌گذارد تا هر‌که داخل شود روشنی را ببیند.
\par 17 زیرا چیزی نهان نیست که ظاهر نگردد و نه مستور که معلوم و هویدا نشود.
\par 18 پس احتیاطنمایید که به چه طور می‌شنوید، زیرا هر‌که داردبدو داده خواهد شد و از آنکه ندارد آنچه گمان هم می‌برد که دارد، از او گرفته خواهد شد.»
\par 19 و مادر وبرادران او نزد وی آمده به‌سبب ازدحام خلق نتوانستند او را ملاقات کنند.
\par 20 پس او را خبر داده گفتند: «مادر و برادرانت بیرون ایستاده می‌خواهند تو را ببینند.»
\par 21 در جواب ایشان گفت: «مادر و برادران من اینانند که کلام خدا را شنیده آن را به‌جا می‌آورند.»
\par 22 روزی از روزها او با شاگردان خود به کشتی سوار شده، به ایشان گفت: «به سوی آن کناردریاچه عبور بکنیم.» پس کشتی را حرکت دادند.
\par 23 و چون می‌رفتند، خواب او را در ربود که ناگاه طوفان باد بر دریاچه فرود آمد، بحدی که کشتی از آب پر می‌شد و ایشان در خطر افتادند.
\par 24 پس نزد او آمده او را بیدار کرده، گفتند: «استادا، استادا، هلاک می‌شویم.» پس برخاسته باد وتلاطم آب را نهیب داد تا ساکن گشت و آرامی پدید آمد.
\par 25 پس به ایشان گفت: «ایمان شما کجااست؟» ایشان ترسان و متعجب شده با یکدیگرمی گفتند که «این چطور آدمی است که بادها وآب را هم امر می‌فرماید و اطاعت او می‌کنند.»
\par 26 و به زمین جدریان که مقابل جلیل است، رسیدند.
\par 27 چون به خشکی فرود آمد، ناگاه شخصی از آن شهر‌که از مدت مدیدی دیوهاداشتی و رخت نپوشیدی و در خانه نماندی بلکه در قبرها منزل داشتی دچار وی گردید.
\par 28 چون عیسی را دید، نعره زد و پیش او افتاده به آواز بلندگفت: «ای عیسی پسر خدای تعالی، مرا با تو چه‌کار است؟ از تو التماس دارم که مرا عذاب ندهی.»
\par 29 زیرا که روح خبیث را امر فرموده بودکه از آن شخص بیرون آید. چونکه بارها او راگرفته بود، چنانکه هر‌چند او را به زنجیرها وکنده‌ها بسته نگاه می‌داشتند، بندها را می‌گسیخت و دیو او را به صحرا می‌راند.
\par 30 عیسی از اوپرسیده، گفت: «نام تو چیست؟» گفت: «لجئون.» زیرا که دیوهای بسیار داخل او شده بودند.
\par 31 واز او استدعا کردند که ایشان را نفرماید که به هاویه روند.
\par 32 و در آن نزدیکی گله گراز بسیاری بودند که در کوه می‌چریدند. پس از او خواهش نمودند که بدیشان اجازت دهد تا در آنها داخل شوند. پس ایشان را اجازت داد.
\par 33 ناگاه دیوها از آن آدم بیرون شده، داخل گرازان گشتند که آن گله ازبلندی به دریاچه جسته، خفه شدند.
\par 34 چون گرازبانان ماجرا را دیدند فرار کردند و در شهر واراضی آن شهرت دادند.
\par 35 پس مردم بیرون آمده تا آن واقعه را ببینندنزد عیسی رسیدند و چون آدمی را که از او دیوهابیرون رفته بودند، دیدند که نزد پایهای عیسی رخت پوشیده و عاقل گشته نشسته است ترسیدند.
\par 36 و آنانی که این را دیده بودند ایشان راخبر دادند که آن دیوانه چطور شفا یافته بود.
\par 37 پس تمام خلق مرزوبوم جدریان از او خواهش نمودند که از نزد ایشان روانه شود، زیرا خوفی شدید بر ایشان مستولی شده بود. پس او به کشتی سوار شده مراجعت نمود.
\par 38 اما آن شخصی که دیوها از وی بیرون رفته بودند از او درخواست کرد که با وی باشد. لیکن عیسی او را روانه فرموده، گفت:
\par 39 «به خانه خود برگرد و آنچه خدا با تو کرده است حکایت کن.» پس رفته درتمام شهر از آنچه عیسی بدو نموده بود موعظه کرد.
\par 40 و چون عیسی مراجعت کرد خلق او راپذیرفتند زیرا جمیع مردم چشم به راه اومی داشتند.
\par 41 که ناگاه مردی، یایرس نام که رئیس کنیسه بود به پایهای عیسی افتاده، به اوالتماس نمود که به خانه او بیاید.
\par 42 زیرا که او رادختر یگانه‌ای قریب به دوازده ساله بود که مشرف بر موت بود. و چون می‌رفت خلق بر اوازدحام می‌نمودند.
\par 43 ناگاه زنی که مدت دوازده سال به استحاضه مبتلا بود و تمام مایملک خود را صرف اطبانموده و هیچ‌کس نمی توانست او را شفا دهد،
\par 44 از پشت سر وی آمده، دامن ردای او را لمس نمود که در ساعت جریان خونش ایستاد.
\par 45 پس عیسی گفت: «کیست که مرا لمس نمود.» چون همه انکار کردند، پطرس و رفقایش گفتند: «ای استاد مردم هجوم آورده بر تو ازدحام می‌کنند ومی گویی کیست که مرا لمس نمود؟»
\par 46 عیسی گفت: «البته کسی مرا لمس نموده است، زیرا که من درک کردم که قوتی از من بیرون شد.»
\par 47 چون آن زن دید که نمی تواند پنهان ماند، لرزان شده، آمد و نزد وی افتاده پیش همه مردم گفت که به چه سبب او را لمس نمود و چگونه فور شفا یافت.
\par 48 وی را گفت: «ای دختر خاطرجمع دار، ایمانت تو را شفا داده است، به سلامتی برو.»
\par 49 و این سخن هنوز بر زبان او بود که یکی ازخانه رئیس کنیسه آمده به وی گفت: «دخترت مرد. دیگر استاد را زحمت مده.»
\par 50 چون عیسی این را شنید توجه نموده به وی گفت: «ترسان مباش، ایمان آور و بس که شفا خواهد یافت.»
\par 51 و چون داخل خانه شد، جز پطرس و یوحنا ویعقوب و پدر و مادر دختر هیچ‌کس را نگذاشت که به اندرون آید.
\par 52 و چون همه برای او گریه وزاری می‌کردند او گفت: «گریان مباشید نمرده بلکه خفته است.»
\par 53 پس به او استهزا کردندچونکه می‌دانستند که مرده است.
\par 54 پس او همه را بیرون کرد و دست دختر را گرفته صدا زد وگفت: «ای دختر برخیز.»
\par 55 و روح او برگشت وفور برخاست. پس عیسی فرمود تا به وی خوراک دهند.و پدر و مادر او حیران شدند. پس ایشان را فرمود که هیچ‌کس را از این ماجراخبر ندهند.
\par 56 و پدر و مادر او حیران شدند. پس ایشان را فرمود که هیچ‌کس را از این ماجراخبر ندهند.

\chapter{9}

\par 1 پس دوازده شاگرد خود را طلبیده، به ایشان قوت و قدرت بر جمیع دیوها و شفادادن امراض عطا فرمود.
\par 2 و ایشان را فرستاد تا به ملکوت خدا موعظه کنند و مریضان را صحت بخشند.
\par 3 و بدیشان گفت: «هیچ‌چیز بجهت راه برمدارید نه عصا و نه توشه‌دان و نه نان و نه پول ونه برای یک نفر دو جامه.
\par 4 و به هرخانه‌ای که داخل شوید همان جا بمانید تا از آن موضع روانه شوید.
\par 5 و هر‌که شما را نپذیرد، وقتی که از آن شهر بیرون شوید خاک پایهای خود را نیزبیفشانید تا بر ایشان شهادتی شود.»
\par 6 پس بیرون شده در دهات می‌گشتند و بشارت می‌دادند و درهرجا صحت می‌بخشیدند.
\par 7 اما هیرودیس تیترارک، چون خبر تمام این وقایع را شنید مضطرب شد، زیرا بعضی می‌گفتندکه یحیی از مردگان برخاسته است،
\par 8 و بعضی که الیاس ظاهر شده و دیگران، که یکی از انبیای پیشین برخاسته است.
\par 9 اما هیرودیس گفت «سریحیی را از تنش من جدا کردم ولی این کیست که درباره او چنین خبر می‌شنوم» و طالب ملاقات وی می‌بود.
\par 10 و چون رسولان مراجعت کردند، آنچه کرده بودند بدو بازگفتند. پس ایشان را برداشته به ویرانه‌ای نزدیک شهری که بیت صیدا نام داشت به خلوت رفت.
\par 11 اما گروهی بسیار اطلاع یافته در عقب وی شتافتند. پس ایشان را پذیرفته، ایشان را از ملکوت خدا اعلام می‌نمود و هر‌که احتیاج به معالجه می‌داشت صحت می‌بخشید.
\par 12 و چون روز رو به زوال نهاد، آن دوازده نزدوی آمده، گفتند: «مردم را مرخص فرما تا به دهات و اراضی این حوالی رفته منزل و خوراک برای خویشتن پیدا نمایند، زیرا که در اینجا درصحرا می‌باشیم.»
\par 13 او بدیشان گفت: «شماایشان را غذا دهید.» گفتند: «ما را جز پنج نان و دوماهی نیست مگر برویم و بجهت جمیع این گروه غدا بخریم.»
\par 14 زیرا قریب به پنجهزار مرد بودند. پس به شاگردان خود گفت که ایشان را پنجاه پنجاه، دسته دسته، بنشانند.»
\par 15 ایشان همچنین کرده همه را نشانیدند.
\par 16 پس آن پنج نان و دوماهی را گرفته، به سوی آسمان نگریست و آنها رابرکت داده، پاره نمود و به شاگردان خود داد تاپیش مردم گذارند.
\par 17 پس همه خورده سیرشدند. و دوازده سبد پر از پاره های باقی‌مانده برداشتند.
\par 18 و هنگامی که او به تنهایی دعا می‌کرد وشاگردانش همراه او بودند، از ایشان پرسیده، گفت: «مردم مرا که می‌دانند؟»
\par 19 در جواب گفتند: «یحیی تعمید‌دهنده و بعضی الیاس ودیگران می‌گویند که یکی از انبیای پیشین برخاسته است.»
\par 20 بدیشان گفت: «شما مرا که می‌دانید؟» پطرس در جواب گفت: «مسیح خدا.»
\par 21 پس ایشان را قدغن بلیغ فرمود که هیچ‌کس را از این اطلاع مدهید.
\par 22 و گفت: «لازم است که پسر انسان زحمت بسیار بیند و از مشایخ وروسای کهنه و کاتبان رد شده کشته شود و روزسوم برخیزد.»
\par 23 پس به همه گفت: «اگر کسی بخواهد مراپیروی کند می‌باید نفس خود را انکار نموده، صلیب خود را هر روزه بردارد و مرا متابعت کند.
\par 24 زیرا هر‌که بخواهد جان خود را خلاصی دهدآن را هلاک سازد و هر کس جان خود را بجهت من تلف کرد، آن را نجات خواهد داد.
\par 25 زیراانسان را چه فایده دارد که تمام جهان را ببرد ونفس خود را بر باد دهد یا آن را زیان رساند.
\par 26 زیرا هر‌که از من و کلام من عار دارد پسر انسان نیز وقتی که در جلال خود و جلال پدر و ملائکه مقدسه آید از او عار خواهد داشت.
\par 27 لیکن هرآینه به شما می‌گویم که بعضی از حاضرین دراینجا هستند که تا ملکوت خدا را نبینند ذائقه موت را نخواهند چشید.»
\par 28 و از این کلام قریب به هشت روز گذشته بود که پطرس و یوحنا و یعقوب را برداشته برفراز کوهی برآمد تا دعا کند.
\par 29 و چون دعامی کرد هیات چهره او متبدل گشت و لباس اوسفید و درخشان شد.
\par 30 که ناگاه دو مرد یعنی موسی و الیاس با وی ملاقات کردند.
\par 31 و به هیات جلالی ظاهر شده درباره رحلت او که می‌بایست به زودی در اورشلیم واقع شود گفتگومی کردند.
\par 32 اما پطرس و رفقایش را خواب در ربود. پس بیدار شده جلال او و آن دو مرد را که با وی بودند، دیدند.
\par 33 و چون آن دو نفر از او جدامی شدند، پطرس به عیسی گفت که «ای استاد، بودن ما در اینجا خوب است. پس سه سایبان بسازیم یکی برای تو و یکی برای موسی ودیگری برای الیاس.» زیرا که نمی دانست چه می‌گفت.
\par 34 و این سخن هنوز بر زبانش می‌بود که ناگاه ابری پدیدار شده بر ایشان سایه افکند وچون داخل ابر می‌شدند، ترسان گردیدند.
\par 35 آنگاه صدایی از ابر برآمد که «این است پسرحبیب من، او را بشنوید.»
\par 36 و چون این آوازرسید عیسی را تنها یافتند و ایشان ساکت ماندندو از آنچه دیده بودند هیچ‌کس را در آن ایام خبرندادند.
\par 37 و در روز بعد چون ایشان از کوه به زیرآمدند، گروهی بسیار او را استقبال نمودند.
\par 38 که ناگاه مردی از آن میان فریادکنان گفت: «ای استادبه تو التماس می‌کنم که بر پسر من لطف فرمایی زیرا یگانه من است.
\par 39 که ناگاه روحی او رامی گیرد و دفعه صیحه می‌زند و کف کرده مصروع می‌شود و او را فشرده به دشواری رها می‌کند.
\par 40 و از شاگردانت درخواست کردم که او را بیرون کنند نتوانستند.»
\par 41 عیسی در جواب گفت: «ای فرقه بی‌ایمان کج روش، تا کی با شما باشم و متحمل شما گردم. پسر خود را اینجا بیاور.»
\par 42 و چون او می‌آمددیو او را دریده مصروع نمود. اما عیسی آن روح خبیث را نهیب داده طفل را شفا بخشید وبه پدرش سپرد.
\par 43 و همه از بزرگی خدامتحیر شدند و وقتی که همه از تمام اعمال عیسی متعجب شدند به شاگردان خودگفت:
\par 44 «این سخنان را در گوشهای خود فراگیریدزیرا که پسر انسان به‌دستهای مردم تسلیم خواهدشد.»
\par 45 ولی این سخن را درک نکردند و از ایشان مخفی داشته شد که آن را نفهمند و ترسیدند که آن را از وی بپرسند.
\par 46 و در میان ایشان مباحثه شد که کدام‌یک ازما بزرگتر است.
\par 47 عیسی خیال دل ایشان راملتفت شده طفلی بگرفت و او را نزد خود برپاداشت.
\par 48 و به ایشان گفت: «هر‌که این طفل را به نام من قبول کند، مرا قبول کرده باشد و هر‌که مراپذیرد، فرستنده مرا پذیرفته باشد. زیرا هر‌که ازجمیع شما کوچکتر باشد، همان بزرگ خواهدبود.»
\par 49 یوحنا جواب داده گفت: «ای استادشخصی را دیدیم که به نام تو دیوها را اخراج می‌کند و او را منع نمودیم، از آن رو که پیروی مانمی کند.»
\par 50 عیسی بدو گفت: «او را ممانعت مکنید. زیرا هر‌که ضد شما نیست با شماست.»
\par 51 و چون روزهای صعود او نزدیک می‌شدروی خود را به عزم ثابت به سوی اورشلیم نهاد.
\par 52 پس رسولان پیش از خود فرستاده، ایشان رفته به بلدی از بلاد سامریان وارد گشتند تا برای اوتدارک بینند.
\par 53 اما او را جای ندادند از آن رو که عازم اورشلیم می‌بود.
\par 54 و چون شاگردان او، یعقوب و یوحنا این را دیدند گفتند: «ای خداوند آیا می‌خواهی بگوییم که آتش از آسمان باریده اینها را فرو‌گیرد چنانکه الیاس نیز کرد؟
\par 55 آنگاه روی گردانیده بدیشان گفت: «نمی دانید که شما ازکدام نوع روح هستید.
\par 56 زیرا که پسر انسان نیامده است تا جان مردم را هلاک سازد بلکه تانجات دهد.» پس به قریه‌ای دیگر رفتند.
\par 57 و هنگامی که ایشان می‌رفتند در اثنای راه شخصی بدو گفت: «خداوندا هر جا روی تو رامتابعت کنم.»
\par 58 عیسی به وی گفت: «روباهان راسوراخها است و مرغان هوا را آشیانه‌ها، لیکن پسر انسان را جای سر نهادن نیست.»
\par 59 و به دیگری گفت: «از عقب من بیا.» گفت: «خداوندااول مرا رخصت ده تا بروم پدر خود را دفن کنم.»
\par 60 عیسی وی را گفت: «بگذار مردگان مردگان خود را دفن کنند. اما تو برو و به ملکوت خداموعظه کن.»
\par 61 و کسی دیگر گفت: «خداوندا تورا پیروی می‌کنم لیکن اول رخصت ده تا اهل خانه خود را وداع نمایم.»عیسی وی را گفت: «کسی‌که دست را به شخم زدن دراز کرده از پشت سر نظر کند، شایسته ملکوت خدا نمی باشد.»
\par 62 عیسی وی را گفت: «کسی‌که دست را به شخم زدن دراز کرده از پشت سر نظر کند، شایسته ملکوت خدا نمی باشد.»

\chapter{10}

\par 1 و بعد از این امور، خداوند هفتاد نفر دیگر را نیز تعیین فرموده، ایشان را جفت جفت پیش روی خود به هر شهری و موضعی که خود عزیمت آن داشت، فرستاد.
\par 2 پس بدیشان گفت: «حصاد بسیار است و عمله کم. پس ازصاحب حصاد درخواست کنید تا عمله ها برای حصاد خود بیرون نماید.
\par 3 بروید، اینک من شما را چون بره‌ها در میان گرگان می‌فرستم.
\par 4 وکیسه و توشه‌دان و کفشها با خود برمدارید وهیچ‌کس را در راه سلام منمایید،
\par 5 و در هرخانه‌ای که داخل شوید، اول گویید سلام بر این خانه باد.
\par 6 پس هرگاه ابن السلام در آن خانه باشد، سلام شما بر آن قرار گیرد والا به سوی شما راجع شود.
\par 7 و در آن خانه توقف نمایید و از آنچه دارند بخورید و بیاشامید، زیرا که مزدور مستحق اجرت خود است و از خانه به خانه نقل مکنید.
\par 8 ودر هر شهری که رفتید و شما را پذیرفتند، از آنچه پیش شما گذارند بخورید.
\par 9 و مریضان آنجا راشفا دهید و بدیشان گویید ملکوت خدا به شمانزدیک شده است.
\par 10 لیکن در هر شهری که رفتیدو شما را قبول نکردند، به کوچه های آن شهربیرون شده بگویید،
\par 11 حتی خاکی که از شهرشما بر ما نشسته است، بر شما می‌افشانیم. لیکن این را بدانید که ملکوت خدا به شما نزدیک شده است.
\par 12 و به شما می‌گویم که حالت سدوم در آن روز، از حالت آن شهر سهل تر خواهد بود.
\par 13 وای بر تو‌ای خورزین؛ وای بر تو‌ای بیت صیدا، زیرا اگر معجزاتی که در شما ظاهر شددر صور و صیدون ظاهر می‌شد، هرآینه مدتی درپلاس و خاکستر نشسته، توبه می‌کردند.
\par 14 لیکن حالت صور و صیدون در روز جزا، از حال شماآسانتر خواهد بود.
\par 15 و تو‌ای کفرناحوم که سر به آسمان افراشته‌ای، تا به حهنم سرنگون خواهی شد.
\par 16 آنکه شما را شنود، مرا شنیده و کسی‌که شما را حقیر شمارد مرا حقیر شمرده و هر‌که مراحقیر شمارد فرستنده مرا حقیر شمرده باشد.»
\par 17 پس آن هفتاد نفر با خرمی برگشته گفتند: «ای خداوند، دیوها هم به اسم تو اطاعت مامی کنند.»
\par 18 بدیشان گفت: «من شیطان را دیدم که چون برق از آسمان می‌افتد.
\par 19 اینک شما راقوت می‌بخشم که ماران و عقربها و تمامی قوت دشمن را پایمال کنید و چیزی به شما ضرر هرگزنخواهد رسانید.
\par 20 ولی از این شادی مکنید که ارواح اطاعت شما می‌کنند بلکه بیشتر شاد باشیدکه نامهای شما در آسمان مرقوم است.»
\par 21 در همان ساعت، عیسی در روح وجدنموده گفت: «ای پدر مالک آسمان و زمین، تو راسپاس می‌کنم که این امور را از دانایان وخردمندان مخفی داشتی و بر کودکان مکشوف ساختی. بلی‌ای پدر، چونکه همچنین منظور نظرتو افتاد.»
\par 22 و به سوی شاگردان خود توجه نموده گفت: «همه‌چیز را پدر به من سپرده است. وهیچ‌کس نمی شناسد که پسر کیست، جز پدر و نه که پدر کیست، غیر از پسر و هر‌که پسر بخواهدبرای او مکشوف سازد.»
\par 23 و در خلوت به شاگردان خود التفات فرموده گفت: «خوشابحال چشمانی که آنچه شما می‌بینید، می‌بینند.
\par 24 زیرابه شما می‌گویم بسا انبیا و پادشاهان می‌خواستندآنچه شما می‌بینید، بنگرند و ندیدند و آنچه شمامی شنوید، بشنوند و نشنیدند.»
\par 25 ناگاه یکی از فقها برخاسته از روی امتحان به وی گفت: «ای استاد چه کنم تا وارث حیات جاودانی گردم؟»
\par 26 به وی گفت: «در تورات چه نوشته شده است و چگونه می‌خوانی؟»
\par 27 جواب داده، گفت: «اینکه خداوند خدای خود را به تمام دل و تمام نفس و تمام توانایی و تمام فکر خودمحبت نما و همسایه خود را مثل نفس خود.»
\par 28 گفت: «نیکو جواب گفتی. چنین بکن که خواهی زیست.»
\par 29 لیکن او چون خواست خودرا عادل نماید، به عیسی گفت: «و همسایه من کیست؟»
\par 30 عیسی در جواب وی گفت: «مردی که ازاورشلیم به سوی اریحا می‌رفت، به‌دستهای دزدان افتاد و او را برهنه کرده مجروح ساختند واو را نیم مرده واگذارده برفتند.
\par 31 اتفاق کاهنی ازآن راه می‌آمد، چون او را بدید از کناره دیگررفت.
\par 32 همچنین شخصی لاوی نیز از آنجا عبورکرده نزدیک آمد و بر او نگریسته از کناره دیگربرفت.
\par 33 «لیکن شخصی سامری که مسافر بود نزدوی آمده چون او را بدید، دلش بر وی بسوخت.
\par 34 پس پیش آمده بر زخمهای او روغن و شراب ریخته آنها را بست واو را بر مرکب خود سوارکرده به‌کاروانسرای رسانید و خدمت او کرد.
\par 35 بامدادان چون روانه می‌شد، دو دینار درآورده به‌سرایدار داد و بدو گفت این شخص را متوجه باش و آنچه بیش از این خرج کنی، در حین مراجعت به تو دهم.
\par 36 «پس به نظر تو کدام‌یک از این سه نفرهمسایه بود با آن شخص که به‌دست دزدان افتاد؟»
\par 37 گفت: «آنکه بر او رحمت کرد.» عیسی وی را گفت: «برو و تو نیز همچنان کن.»
\par 38 و هنگامی که می‌رفتند او وارد بلدی شد وزنی که مرتاه نام داشت، او را به خانه خودپذیرفت.
\par 39 و او را خواهری مریم نام بود که نزدپایهای عیسی نشسته کلام او را می‌شنید.
\par 40 امامرتاه بجهت زیادتی خدمت مضطرب می‌بود. پس نزدیک آمده، گفت: «ای خداوند آیا تو راباکی نیست که خواهرم مرا واگذارد که تنهاخدمت کنم، او را بفرما تا مرا یاری کند.»
\par 41 عیسی در جواب وی گفت: «ای مرتاه، ای مرتاه تو در چیزهای بسیار اندیشه و اضطراب داری.لیکن یک چیز لازم است و مریم آن نصیب خوب را اختیار کرده است که از او گرفته نخواهد شد.»
\par 42 لیکن یک چیز لازم است و مریم آن نصیب خوب را اختیار کرده است که از او گرفته نخواهد شد.»

\chapter{11}

\par 1 و هنگامی که او در موضعی دعا می‌کردچون فارغ شد، یکی از شاگردانش به وی گفت: «خداوندا دعا کردن را به ما تعلیم نما، چنانکه یحیی شاگردان خود را بیاموخت.»
\par 2 بدیشان گفت: «هرگاه دعا کنید گویید‌ای پدرما که در آسمانی، نام تو مقدس باد. ملکوت توبیاید. اراده تو چنانکه در آسمان است در زمین نیزکرده شود.
\par 3 نان کفاف ما را روز به روز به ما بده.
\par 4 و گناهان ما را ببخش زیرا که ما نیز هر قرضدار خود را می‌بخشیم. و ما را در آزمایش میاور، بلکه ما را از شریر رهایی ده.»
\par 5 و بدیشان گفت: «کیست از شما که دوستی داشته باشد و نصف شب نزد وی آمده بگوید‌ای دوست سه قرص نان به من قرض ده،
\par 6 چونکه یکی از دوستان من از سفر بر من وارد شده وچیزی ندارم که پیش او گذارم.
\par 7 پس او از اندرون در جواب گوید مرا زحمت مده، زیرا که الان دربسته است و بچه های من در رختخواب با من خفته‌اند نمی توانم برخاست تا به تو دهم.
\par 8 به شمامی گویم هر‌چند به علت دوستی برنخیزد تا بدودهد، لیکن بجهت لجاجت خواهد برخاست و هرآنچه حاجت دارد، بدو خواهد داد.
\par 9 «و من به شما می‌گویم سوال کنید که به شماداده خواهد شد. بطلبید که خواهید یافت. بکوبیدکه برای شما بازکرده خواهد شد.
\par 10 زیرا هر‌که سوال کند، یابد و هر‌که بطلبد، خواهد یافت وهرکه کوبد، برای او باز کرده خواهد شد.
\par 11 وکیست از شما که پدر باشد و پسرش از او نان خواهد، سنگی بدو دهد یا اگر ماهی خواهد، به عوض ماهی ماری بدو بخشد،
\par 12 یا اگرتخم‌مرغی بخواهد، عقربی بدو عطا کند.
\par 13 پس اگر شما با آنکه شریر هستید می‌دانید چیزهای نیکو را به اولاد خود باید داد، چند مرتبه زیادترپدر آسمانی شما روح‌القدس را خواهد داد به هرکه از او سوال کند.»
\par 14 و دیوی را که گنگ بود بیرون می‌کرد وچون دیو بیرون شد، گنگ گویا گردید و مردم تعجب نمودند.
\par 15 لیکن بعضی از ایشان گفتند که «دیوها را به یاری بعلزبول رئیس دیوها بیرون می‌کند.»
\par 16 و دیگران از روی امتحان آیتی آسمانی از او طلب نمودند.
\par 17 پس او خیالات ایشان را درک کرده بدیشان گفت: «هر مملکتی که برخلاف خود منقسم شود، تباه گردد و خانه‌ای که بر خانه منقسم شود، منهدم گردد.
\par 18 پس شیطان نیز اگر به ضد خود منقسم شود سلطنت اوچگونه پایدار بماند. زیرا می‌گویید که من به اعانت بعلزبول دیوها را بیرون می‌کنم.
\par 19 پس اگرمن دیوها را به وساطت بعلزبول بیرون می‌کنم، پسران شما به وساطت که آنها را بیرون می‌کنند؟ از اینجهت ایشان داوران بر شما خواهند بود.
\par 20 لیکن هرگاه به انگشت خدا دیوها را بیرون می‌کنم، هرآینه ملکوت خدا ناگهان بر شما آمده است.
\par 21 وقتی که مرد زورآور سلاح پوشیده خانه خود را نگاه دارد، اموال او محفوظ می‌باشد.
\par 22 اما چون شخصی زورآورتر از او آید بر او غلبه یافته همه اسلحه او را که بدان اعتماد می‌داشت، از او می‌گیرد و اموال او را تقسیم می‌کند.
\par 23 کسی‌که با من نیست، برخلاف من است و آنکه با من جمع نمی کند، پراکنده می‌سازد.
\par 24 چون روح پلید از انسان بیرون آید به مکانهای بی‌آب بطلب آرامی گردش می‌کند وچون نیافت می‌گوید به خانه خود که از آن بیرون آمدم برمی گردم.
\par 25 پس چون آید، آن را جاروب کرده شده و آراسته می‌بیند.
\par 26 آنگاه می‌رود وهفت روح دیگر، شریرتر از خود برداشته داخل شده در آنجا ساکن می‌گردد و اواخر آن شخص ازاوائلش بدتر می‌شود.»
\par 27 چون او این سخنان را می‌گفت، زنی از آن میان به آواز بلند وی را گفت «خوشابحال آن رحمی که تو را حمل کرد و پستانهایی که مکیدی.»
\par 28 لیکن او گفت: «بلکه خوشابحال آنانی که کلام خدا را می‌شنوند و آن را حفظمی کنند.»
\par 29 و هنگامی که مردم بر او ازدحام می‌نمودند، سخن‌گفتن آغاز کرد که «اینان فرقه‌ای شریرند که آیتی طلب می‌کنند و آیتی بدیشان عطا نخواهدشد، جز آیت یونس نبی.
\par 30 زیرا چنانکه یونس برای اهل نینوا آیت شد، همچنین پسر انسان نیزبرای این فرقه خواهد بود.
\par 31 ملکه جنوب در روزداوری با مردم این فرقه برخاسته، بر ایشان حکم خواهد کرد زیرا که از اقصای زمین آمد تا حکمت سلیمان را بشنود و اینک در اینجا کسی بزرگتر ازسلیمان است.
\par 32 مردم نینوا در روز داوری با این طبقه برخاسته بر ایشان حکم خواهند کرد زیرا که به موعظه یونس توبه کردند و اینک در اینجا کسی بزرگتر از یونس است.
\par 33 و هیچ‌کس چراغی نمی افروزد تا آن را درپنهانی یا زیر پیمانه‌ای بگذارد، بلکه بر چراغدان، تا هر‌که داخل شود روشنی را بیند.
\par 34 چراغ بدن چشم است، پس مادامی که چشم تو بسیط است تمامی جسدت نیز روشن است و لیکن اگر فاسدباشد، جسد تو نیز تاریک بود.
\par 35 پس باحذر باش مبادا نوری که در تو است، ظلمت باشد.
\par 36 بنابراین هرگاه تمامی جسم تو روشن باشد وذره‌ای ظلمت نداشته باشد همه‌اش روشن خواهد بود، مثل وقتی که چراغ به تابش خود، تورا روشنایی می‌دهد.»
\par 37 و هنگامی که سخن می‌گفت یکی ازفریسیان از او وعده خواست که در خانه اوچاشت بخورد. پس داخل شده بنشست.
\par 38 امافریسی چون دید که پیش از چاشت دست نشست، تعجب نمود.
\par 39 خداوند وی را گفت: «همانا شما‌ای فریسیان بیرون پیاله و بشقاب راطاهر می‌سازید ولی درون شما پر از حرص وخباثت است.
\par 40 ‌ای احمقان آیا او که بیرون راآفرید، اندرون را نیز نیافرید؟
\par 41 بلکه از آنچه دارید، صدقه دهید که اینک همه‌چیز برای شماطاهر خواهد گشت.
\par 42 وای بر شما‌ای فریسیان که ده‌یک از نعناع و سداب و هر قسم سبزی رامی دهید و از دادرسی و محبت خدا تجاوزمی نمایید، اینها را می‌باید به‌جا آورید و آنها رانیز ترک نکنید.
\par 43 وای بر شما‌ای فریسیان که صدر کنایس و سلام در بازارها را دوست می‌دارید.
\par 44 وای بر شما‌ای کاتبان و فریسیان ریاکار زیرا که مانند قبرهای پنهان شده هستید که مردم بر آنها راه می‌روند و نمی دانند.»
\par 45 آنگاه یکی از فقها جواب داده گفت: «ای معلم، بدین سخنان ما را نیز سرزنش می‌کنی؟»
\par 46 گفت «وای بر شما نیز‌ای فقها زیرا که بارهای گران را بر مردم می‌نهید و خود بر آن بارها، یک انگشت خود را نمی گذارید.
\par 47 وای بر شما زیراکه مقابر انبیا را بنا می‌کنید و پدران شما ایشان راکشتند.
\par 48 پس به‌کارهای پدران خود شهادت می‌دهید و از آنها راضی هستید، زیرا آنها ایشان را کشتند و شما قبرهای ایشان را می‌سازید.
\par 49 ازاین‌رو حکمت خدا نیز فرموده است که به سوی ایشان انبیا و رسولان می‌فرستم و بعضی از ایشان را خواهند کشت و بر بعضی جفا کرد،
\par 50 تا انتقام خون جمیع انبیا که از بنای عالم ریخته شد از این طبقه گرفته شود.
\par 51 از خون هابیل تا خون زکریاکه در میان مذبح و هیکل کشته شد. بلی به شمامی گویم که از این فرقه بازخواست خواهد شد.
\par 52 وای بر شما‌ای فقها، زیرا کلید معرفت رابرداشته‌اید که خود داخل نمی شوید و داخل‌شوندگان را هم مانع می‌شوید.»
\par 53 و چون او این سخنان را بدیشان می‌گفت، کاتبان و فریسیان با او بشدت درآویختند و درمطالب بسیار سوالها از او می‌کردند.و در کمین او می‌بودند تا نکته‌ای از زبان او گرفته مدعی اوبشوند.
\par 54 و در کمین او می‌بودند تا نکته‌ای از زبان او گرفته مدعی اوبشوند.

\chapter{12}

\par 1 و در آن میان، وقتی که هزاران از خلق جمع شدند، به نوعی که یکدیگر راپایمال می‌کردند، به شاگردان خود به سخن‌گفتن شروع کرد. «اول آنکه از خمیرمایه فریسیان که ریاکاری است احتیاط کنید.
\par 2 زیرا چیزی نهفته نیست که آشکار نشود و نه مستوری که معلوم نگردد.
\par 3 بنابراین آنچه در تاریکی گفته‌اید، درروشنایی شنیده خواهد شد و آنچه در خلوتخانه در گوش گفته‌اید بر پشت بامها ندا شود.
\par 4 لیکن‌ای دوستان من، به شما می‌گویم از قاتلان جسم که قدرت ندارند بیشتر از این بکنند، ترسان مباشید.
\par 5 بلکه به شما نشان می‌دهم که از که باید ترسید، از او بترسید که بعد از کشتن، قدرت دارد که به جهنم بیفکند. بلی به شما می‌گویم از او بترسید.
\par 6 آیا پنج گنجشک به دو فلس فروخته نمی شود وحال آنکه یکی از آنها نزد خدا فراموش نمی شود؟
\par 7 بلکه مویهای سر شما همه شمرده شده است. پس بیم مکنید، زیرا که از چندان گنجشک بهتر هستید.
\par 8 «لیکن به شما می‌گویم هر‌که نزد مردم به من اقرار کند، پسر انسان نیز پیش فرشتگان خدا او رااقرار خواهد کرد.
\par 9 اما هر‌که مرا پیش مردم انکارکند، نزد فرشتگان خدا انکار کرده خواهد شد.
\par 10 و هر‌که سخنی برخلاف پسر انسان گوید، آمرزیده شود. اما هر‌که به روح‌القدس کفر گویدآمرزیده نخواهد شد.
\par 11 و چون شما را درکنایس و به نزد حکام و دیوانیان برند، اندیشه مکنید که چگونه و به چه نوع حجت آورید یا چه بگویید.
\par 12 زیرا که در همان ساعت روح‌القدس شما را خواهد آموخت که چه باید گفت.»
\par 13 و شخصی از آن جماعت به وی گفت: «ای استاد، برادر مرا بفرما تا ارث پدر را با من تقسیم کند.»
\par 14 به وی گفت: «ای مرد که مرا بر شما داوریا مقسم قرار داده است؟»
\par 15 پس بدیشان گفت: «زنهار از طمع بپرهیزید زیرا اگرچه اموال کسی زیاد شود، حیات او از اموالش نیست.»
\par 16 و مثلی برای ایشان آورده، گفت: «شخصی دولتمند را از املاکش محصول وافر پیدا شد.
\par 17 پس با خود اندیشیده، گفت چه کنم؟ زیراجایی که محصول خود را انبار کنم، ندارم.
\par 18 پس گفت چنین می‌کنم انبارهای خود را خراب کرده، بزرگتر بنا می‌کنم و در آن تمامی حاصل و اموال خود را جمع خواهم کرد.
\par 19 و نفس خود راخواهم گفت که‌ای جان اموال فراوان اندوخته شده بجهت چندین سال داری. الحال بیارام و به اکل و شرب و شادی بپرداز.
\par 20 خدا وی را گفت‌ای احمق در همین شب جان تو را از تو خواهندگرفت، آنگاه آنچه اندوخته‌ای، از آن که خواهدبود؟
\par 21 همچنین است هر کسی‌که برای خودذخیره کند و برای خدا دولتمند نباشد.»
\par 22 پس به شاگردان خود گفت: «از این جهت به شما می‌گویم که اندیشه مکنید بجهت جان خودکه چه بخورید و نه برای بدن که چه بپوشید.
\par 23 جان از خوراک و بدن از پوشاک بهتر است.
\par 24 کلاغان را ملاحظه کنید که نه زراعت می‌کنندو نه حصاد و نه گنجی و نه انباری دارند و خدا آنهارا می‌پروراند. آیا شما به چند مرتبه از مرغان بهترنیستید؟
\par 25 و کیست از شما که به فکر بتواندذراعی بر قامت خود افزاید.
\par 26 پس هرگاه توانایی کوچکترین کاری را ندارید چرا برای مابقی می اندیشید.
\par 27 سوسنهای چمن را بنگریدچگونه نمو می‌کنند و حال آنکه نه زحمت می‌کشند و نه می‌ریسند، اما به شما می‌گویم که سلیمان با همه جلالش مثل یکی از اینها پوشیده نبود.
\par 28 پس هرگاه خدا علفی را که امروز درصحرا است و فردا در تنور افکنده می‌شود چنین می‌پوشاند، چقدر بیشتر شما را‌ای سست‌ایمانان.
\par 29 پس شما طالب مباشید که چه بخوریدیا چه بیاشامید و مضطرب مشوید.
\par 30 زیرا که امت های جهان، همه این چیزها را می‌طلبند، لیکن پدر شما می‌داند که به این چیزها احتیاج دارید.
\par 31 بلکه ملکوت خدا را طلب کنید که جمیع این چیزها برای شما افزوده خواهد شد.
\par 32 ترسان مباشید‌ای گله کوچک، زیرا که مرضی پدر شما است که ملکوت را به شما عطافرماید.
\par 33 آنچه دارید بفروشید و صدقه دهید وکیسه‌ها بسازید که کهنه نشود و گنجی را که تلف نشود، در آسمان جایی که دزد نزدیک نیاید و بیدتباه نسازد.
\par 34 زیرا جایی که خزانه شما است، دل شما نیز در آنجا می‌باشد.
\par 35 کمرهای خود را بسته چراغهای خود راافروخته بدارید.
\par 36 و شما مانند کسانی باشید که انتظار آقای خود را می‌کشند که چه وقت ازعروسی مراجعت کند تا هروقت آید و در رابکوبد، بی‌درنگ برای او بازکنند.
\par 37 خوشابحال آن غلامان که آقای ایشان چون آید ایشان را بیداریابد. هر آینه به شما می‌گویم که کمر خود را بسته ایشان را خواهد نشانید و پیش آمده، ایشان راخدمت خواهد کرد.
\par 38 و اگر در پاس دوم یا سوم از شب بیاید و ایشان را چنین یابد، خوشا بحال آن غلامان.
\par 39 اما این را بدانید که اگر صاحب‌خانه می‌دانست که دزد در چه ساعت می‌آید، بیدارمی ماند و نمی گذاشت که به خانه‌اش نقب زنند.
\par 40 پس شما نیز مستعد باشید، زیرا در ساعتی که گمان نمی برید پسر انسان می‌آید.»
\par 41 پطرس به وی گفت: «ای خداوند، آیا این مثل را برای ما زدی یا بجهت همه.»
\par 42 خداوندگفت: «پس کیست آن ناظر امین و دانا که مولای اووی را بر سایر خدام خود گماشته باشد تا آذوقه را در وقتش به ایشان تقسیم کند.
\par 43 خوشابحال آن غلام که آقایش چون آید، او را در چنین کارمشغول یابد.
\par 44 هرآینه به شما می‌گویم که او رابر همه مایملک خود خواهد گماشت.
\par 45 لیکن اگر آن غلام در خاطر خود گوید، آمدن آقایم به طول می‌انجامد و به زدن غلامان و کنیزان و به خوردن و نوشیدن و میگساریدن شروع کند،
\par 46 هرآینه مولای آن غلام آید، در روزی که منتظر او نباشد و در ساعتی که او نداند و او را دوپاره کرده نصیبش را با خیانتکاران قرار دهد.
\par 47 «اما آن غلامی که اراده مولای خویش رادانست و خود را مهیا نساخت تا به اراده او عمل نماید، تازیانه بسیار خواهد خورد.
\par 48 اما آنکه نادانسته کارهای شایسته ضرب کند، تازیانه کم خواهد خورد. و به هر کسی‌که عطا زیاده شود ازوی مطالبه زیادتر گردد و نزد هر‌که امانت بیشترنهند از او بازخواست زیادتر خواهند کرد.
\par 49 من آمدم تا آتشی در زمین افروزم، پس چه می‌خواهم اگر الان در‌گرفته است.
\par 50 اما مراتعمیدی است که بیابم و چه بسیار در تنگی هستم، تا وقتی که آن بسر‌آید.
\par 51 آیا گمان می‌برید که من آمده‌ام تا سلامتی بر زمین بخشم؟ نی بلکه به شما می‌گویم تفریق را.
\par 52 زیرا بعد ازاین پنج نفر که در یک خانه باشند دو از سه و سه ازدو جدا خواهند شد،
\par 53 پدر از پسر و پسر از پدرو مادر از دختر و دختر از مادر و خارسو ازعروس و عروس از خارسو مفارقت خواهندنمود.»
\par 54 آنگاه باز به آن جماعت گفت: «هنگامی که ابری بینید که از مغرب پدید آید، بی‌تامل می‌گویید باران می‌آید و چنین می‌شود.
\par 55 و چون دیدید که باد جنوبی می‌وزد، می‌گویید گرما خواهد شد و می‌شود.
\par 56 ‌ای ریاکاران، می‌توانید صورت زمین و آسمان راتمیز دهید، پس چگونه این زمان رانمی شناسید؟
\par 57 و چرا از خود به انصاف حکم نمی کنید؟
\par 58 و هنگامی که با مدعی خود نزد حاکم می‌روی، در راه سعی کن که از او برهی، مبادا تو رانزد قاضی بکشد و قاضی تو را به‌سرهنگ سپاردو سرهنگ تو را به زندان افکند.تو را می‌گویم تا فلس آخر را ادا نکنی، از آنجا هرگز بیرون نخواهی آمد.»
\par 59 تو را می‌گویم تا فلس آخر را ادا نکنی، از آنجا هرگز بیرون نخواهی آمد.»

\chapter{13}

\par 1 در آن وقت بعضی آمده او را از جلیلیانی خبر دادند که پیلاطس خون ایشان را با قربانی های ایشان آمیخته بود.
\par 2 عیسی در جواب ایشان گفت: «آیا گمان می‌برید که این جلیلیان گناهکارتر بودند از سایر سکنه جلیل ازاین‌رو که چنین زحمات دیدند؟
\par 3 نی، بلکه به شما می‌گویم اگر توبه نکنید، همگی شماهمچنین هلاک خواهید شد.
\par 4 یا آن هجده نفری که برج در سلوام بر ایشان افتاده ایشان را هلاک کرد، گمان می‌برید که از جمیع مردمان ساکن اورشلیم، خطاکارتر بودند؟
\par 5 حاشا، بلکه شما رامی گویم که اگر توبه نکنید همگی شما همچنین هلاک خواهید شد.»
\par 6 پس این مثل را آورد که «شخصی درخت انجیری در تاکستان خود غرس نمود و چون آمدتا میوه از آن بجوید، چیزی نیافت.
\par 7 پس به باغبان گفت اینک سه سال است می‌آیم که از این درخت انجیر میوه بطلبم و نمی یابم، آن را ببر چرا زمین رانیز باطل سازد.
\par 8 در جواب وی گفت، ای آقاامسال هم آن را مهلت ده تا گردش را کنده کودبریزم،
\par 9 پس اگر ثمر آورد والا بعد از آن، آن راببر.»
\par 10 و روز سبت در یکی از کنایس تعلیم می‌داد.
\par 11 و اینک زنی که مدت هجده سال روح ضعف می‌داشت و منحنی شده ابد نمی توانست راست بایستد، در آنجا بود.
\par 12 چون عیسی او را دید وی را خوانده گفت: «ای زن از ضعف خودخلاص شو.»
\par 13 و دست های خود را بر وی گذارد که در ساعت راست شده، خدا را تمجیدنمود.
\par 14 آنگاه رئیس کنیسه غضب نمود، از آنروکه عیسی او را در سبت شفا داد. پس به مردم توجه نموده، گفت: «شش روز است که باید کاربکنید در آنها آمده شفا یابید، نه در روزسبت.»
\par 15 خداوند در جواب او گفت: «ای ریاکار، آیاهر یکی از شما در روز سبت گاو یا الاغ خود را ازآخور باز کرده بیرون نمی برد تا سیرآبش کند؟
\par 16 و این زنی که دختر ابراهیم است و شیطان او رامدت هجده سال تا به حال بسته بود، نمی بایست او را در روز سبت از این بند رها نمود؟»
\par 17 وچون این را بگفت همه مخالفان او خجل گردیدندو جمیع آن گروه شاد شدند، بسبب همه کارهای بزرگ که از وی صادر می‌گشت.
\par 18 پس گفت: «ملکوت خدا چه چیز را می‌ماندو آن را به کدام شی تشبیه نمایم.
\par 19 دانه خردلی راماند که شخصی گرفته در باغ خود کاشت، پس رویید و درخت بزرگ گردید، بحدی که مرغان هوا آمده در شاخه هایش آشیانه گرفتند.»
\par 20 باز‌گفت: «برای ملکوت خدا چه مثل آورم؟
\par 21 خمیرمایه‌ای را می‌ماند که زنی گرفته در سه پیمانه آرد پنهان ساخت تا همه مخمرشد.»
\par 22 و در شهرها و دهات گشته، تعلیم می‌داد وبه سوی اورشلیم سفر می‌کرد،
\par 23 که شخصی به وی گفت: «ای خداوند آیا کم هستند که نجات یابند؟» او به ایشان گفت:
\par 24 «جد و جهد کنید تااز در تنگ داخل شوید. زیرا که به شما می‌گویم بسیاری طلب دخول خواهند کرد و نخواهندتوانست.
\par 25 بعد از آنکه صاحب‌خانه برخیزد ودر را ببندد و شما بیرون ایستاده در را کوبیدن آغاز کنید و گویید خداوندا خداوندا برای ما بازکن. آنگاه وی در جواب خواهد گفت شما رانمی شناسم که از کجا هستید.
\par 26 در آن وقت خواهید گفت که در حضور تو خوردیم وآشامیدیم و در کوچه های ما تعلیم دادی.
\par 27 بازخواهد گفت، به شما می‌گویم که شما رانمی شناسم از کجا هستید؟ ای همه بدکاران از من دور شوید.
\par 28 در آنجا گریه و فشار دندان خواهدبود، چون ابراهیم واسحق و یعقوب و جمیع انبیارا در ملکوت خدا بینید و خود را بیرون افکنده یابید
\par 29 و از مشرق و مغرب و شمال و جنوب آمده در ملکوت خدا خواهند نشست.
\par 30 و اینک آخرین هستند که اولین خواهند بود و اولین که آخرین خواهند بود.»
\par 31 در همان روز چند نفر از فریسیان آمده به وی گفتند: «دور شو و از اینجا برو زیرا که هیرودیس می‌خواهد تو را به قتل رساند.»
\par 32 ایشان را گفت: «بروید و به آن روباه گوییداینک امروز و فردا دیوها را بیرون می‌کنم ومریضان را صحت می‌بخشم و در روز سوم کامل خواهم شد.
\par 33 لیکن می‌باید امروز و فردا و پس‌فردا راه روم، زیرا که محال است نبی بیرون ازاورشلیم کشته شود.
\par 34 ‌ای اورشلیم، ای اورشلیم که قاتل انبیا و سنگسار کننده مرسلین خودهستی، چند کرت خواستم اطفال تو را جمع کنم، چنانکه مرغ جوجه های خویش را زیر بالهای خود می‌گیرد و نخواستید.اینک خانه شمابرای شما خراب گذاشته می‌شود و به شمامی گویم که مرا دیگر نخواهید دید تا وقتی آیدکه گویید مبارک است او که به نام خداوندمی آید.»
\par 35 اینک خانه شمابرای شما خراب گذاشته می‌شود و به شمامی گویم که مرا دیگر نخواهید دید تا وقتی آیدکه گویید مبارک است او که به نام خداوندمی آید.»

\chapter{14}

\par 1 و واقع شد که در روز سبت، به خانه یکی از روسای فریسیان برای غذاخوردن درآمد و ایشان مراقب او می‌بودند،
\par 2 واینک شخصی مستسقی پیش او بود،
\par 3 آنگاه عیسی ملتفت شده فقها و فریسیان را خطاب کرده، گفت: «آیا در روز سبت شفا دادن جایزاست؟»
\par 4 ایشان ساکت ماندند. پس آن مرد راگرفته، شفا داد و رها کرد.
\par 5 و به ایشان روی آورده، گفت: «کیست از شما که الاغ یا گاوش روزسبت در چاهی افتد و فور آن را بیرون نیاورد؟»
\par 6 پس در این امور از جواب وی عاجزماندند.
\par 7 و برای مهمانان مثلی زد، چون ملاحظه فرمود که چگونه صدر مجلس را اختیارمی کردند. پس به ایشان گفت:
\par 8 «چون کسی تو رابه عروسی دعوت کند، در صدر مجلس منشین، مبادا کسی بزرگتر از تو را هم وعده خواسته باشد.
\par 9 پس آن کسی‌که تو و او را وعده خواسته بودبیاید و تو را گوید این کس را جای بده و تو باخجالت روی به صف نعال خواهی نهاد.
\par 10 بلکه چون مهمان کسی باشی، رفته در پایین بنشین تاوقتی که میزبانت آید به تو گوید، ای دوست برترنشین آنگاه تو را در حضور مجلسیان عزت خواهد بود.
\par 11 زیرا هر‌که خود را بزرگ سازدذلیل گردد و هر‌که خویشتن را فرود آرد، سرافرازگردد.»
\par 12 پس به آن کسی‌که از او وعده خواسته بود نیز گفت: «وقتی که چاشت یا شام دهی دوستان یا برادران یا خویشان یا همسایگان دولتمند خود را دعوت مکن مبادا ایشان نیز تو رابخوانند و تو را عوض داده شود.
\par 13 بلکه چون ضیافت کنی، فقیران و لنگان و شلان و کوران رادعوت کن.
\par 14 که خجسته خواهی بود زیرا ندارندکه تو را عوض دهند و در قیامت عادلان، به توجزا عطا خواهد شد.»
\par 15 آنگاه یکی از مجلسیان چون این سخن راشنید گفت: «خوشابحال کسی‌که در ملکوت خداغذا خورد.»
\par 16 به وی گفت: «شخصی ضیافتی عظیم نمود و بسیاری را دعوت نمود.
\par 17 پس چون وقت شام رسید، غلام خود را فرستاد تادعوت‌شدگان را گوید، بیایید زیرا که الحال همه‌چیز حاضر است.
\par 18 لیکن همه به یک رای عذرخواهی آغاز کردند. اولی گفت: مزرعه‌ای خریدم و ناچار باید بروم آن را ببینم، از توخواهش دارم مرا معذور داری.
\par 19 و دیگری گفت: پنج جفت گاو خریده‌ام، می‌روم تا آنها رابیازمایم، به تو التماس دارم مرا عفو نمایی.
\par 20 سومی گفت: زنی گرفته‌ام و از این سبب نمی توانم بیایم.
\par 21 پس آن غلام آمده مولای خود را از این امور مطلع ساخت. آنگاه صاحب‌خانه غضب نموده به غلام خود فرمود: به بازارهاو کوچه های شهر بشتاب و فقیران و لنگان وشلان و کوران را در اینجا بیاور.
\par 22 پس غلام گفت: ای آقا آنچه فرمودی شد و هنوز جای باقی است.
\par 23 پس آقا به غلام گفت: به راهها و مرزهابیرون رفته، مردم را به الحاح بیاور تا خانه من پرشود.
\par 24 زیرا به شما می‌گویم هیچ‌یک از آنانی که دعوت شده بودند، شام مرا نخواهد چشید.»
\par 25 و هنگامی که جمعی کثیر همراه اومی رفتند، روی گردانیده بدیشان گفت:
\par 26 «اگرکسی نزد من آید و پدر، مادر و زن و اولاد وبرادران و خواهران، حتی جان خود را نیز دشمن ندارد، شاگرد من نمی تواند بود.
\par 27 و هر‌که صلیب خود را برندارد و از عقب من نیاید، نمی تواندشاگرد من گردد.
\par 28 «زیرا کیست از شما که قصد بنای برجی داشته باشد و اول ننشیند تا برآورد خرج آن رابکند که آیا قوت تمام کردن آن دارد یا نه؟
\par 29 که مبادا چون بنیادش نهاد و قادر بر تمام کردنش نشد، هر‌که بیند تمسخرکنان گوید،
\par 30 این شخص عمارتی شروع کرده نتوانست به انجامش رساند.
\par 31 یا کدام پادشاه است که برای مقاتله باپادشاه دیگر برود، جز اینکه اول نشسته تامل نماید که آیا با ده هزار سپاه، قدرت مقاومت کسی را دارد که با بیست هزار لشکر بر وی می‌آید؟
\par 32 والا چون او هنوز دور است ایلچی‌ای فرستاده شروط صلح را ازاو درخواست کند.
\par 33 «پس همچنین هر یکی از شما که تمام مایملک خود را ترک نکند، نمی تواند شاگرد من شود.
\par 34 «نمک نیکو است ولی هرگاه نمک فاسدشد به چه چیز اصلاح پذیرد؟نه برای زمین مصرفی دارد و نه برای مزبله، بلکه بیرونش می‌ریزند. آنکه گوش شنوا دارد بشنود.»
\par 35 نه برای زمین مصرفی دارد و نه برای مزبله، بلکه بیرونش می‌ریزند. آنکه گوش شنوا دارد بشنود.»

\chapter{15}

\par 1 و چون همه باجگیران و گناهکاران به نزدش می‌آمدند تا کلام او را بشنوند،
\par 2 فریسیان و کاتبان، همهمه‌کنان می‌گفتند: «این شخص، گناهکارن را می‌پذیرد و با ایشان می‌خورد.»
\par 3 پس برای ایشان این مثل را زده، گفت:
\par 4 «کیست از شما که صد گوسفند داشته باشد و یکی از آنها گم شود که آن نود و نه را درصحرا نگذارد و از عقب آن گمشده نرود تا آن رابیابد؟
\par 5 پس چون آن را یافت به شادی بر دوش خود می‌گذارد،
\par 6 و به خانه آمده، دوستان وهمسایگان را می‌طلبد و بدیشان می‌گوید با من شادی کنید زیرا گوسفند گمشده خود را یافته‌ام.
\par 7 به شما می‌گویم که بر این منوال خوشی درآسمان رخ می‌نماید به‌سبب توبه یک گناهکاربیشتر از برای نود و نه عادل که احتیاج به توبه ندارند.
\par 8 یا کدام زن است که ده درهم داشته باشدهرگاه یک درهم گم شود، چراغی افروخته، خانه را جاروب نکند و به دقت تفحص ننماید تا آن رابیابد؟
\par 9 و چون یافت دوستان و همسایگان خودرا جمع کرده می‌گوید با من شادی کنید زیرادرهم گمشده را پیدا کرده‌ام.
\par 10 همچنین به شمامی گویم شادی برای فرشتگان خدا روی می‌دهدبه‌سبب یک خطاکار که توبه کند.»
\par 11 باز‌گفت: «شخصی را دو پسر بود.
\par 12 روزی پسر کوچک به پدر خود گفت: ای پدر، رصداموالی که باید به من رسد، به من بده. پس اومایملک خود را بر این دو تقسیم کرد.
\par 13 و چندی نگذشت که آن پسر کهتر، آنچه داشت جمع کرده، به ملکی بعید کوچ کرد و به عیاشی ناهنجار، سرمایه خود را تلف نمود.
\par 14 و چون تمام راصرف نموده بود، قحطی سخت در آن دیارحادث گشت و او به محتاج شدن شروع کرد.
\par 15 پس رفته خود را به یکی از اهل آن ملک پیوست. وی او را به املاک خود فرستاد تاگرازبانی کند.
\par 16 و آرزو می‌داشت که شکم خودرا از خرنوبی که خوکان می‌خوردند سیر کند وهیچ‌کس او را چیزی نمی داد.
\par 17 «آخر به خود آمده، گفت چقدر از مزدوران پدرم نان فراوان دارند و من از گرسنگی هلاک می‌شوم،
\par 18 برخاسته نزد پدر خود می‌روم و بدوخواهم گفت‌ای پدر به آسمان و به حضور تو گناه کرده‌ام،
\par 19 و دیگر شایسته آن نیستم که پسر توخوانده شوم، مرا چون یکی از مزدوران خودبگیر.
\par 20 در ساعت برخاسته به سوی پدر خودمتوجه شد. اما هنوز دور بود که پدرش او را دیده، ترحم نمود و دوان دوان آمده او را در آغوش خود کشیده، بوسید.
\par 21 پسر وی را گفت، ای پدربه آسمان و به حضور تو گناه کرده‌ام و بعد از این لایق آن نیستم که پسر تو خوانده شوم.
\par 22 لیکن پدر به غلامان خود گفت، جامه بهترین را از خانه آورده بدو بپوشانید و انگشتری بر دستش کنید ونعلین بر پایهایش،
\par 23 و گوساله پرواری را آورده ذبح کنید تا بخوریم و شادی نماییم.
\par 24 زیرا که این پسر من مرده بود، زنده گردید و گم شده بود، یافت شد. پس به شادی کردن شروع نمودند.
\par 25 اما پسر بزرگ او در مزرعه بود. چون آمده نزدیک به خانه رسید، صدای ساز و رقص راشنید.
\par 26 پس یکی از نوکران خود را طلبیده پرسید این چیست؟
\par 27 به وی عرض کرد برادرت آمده و پدرت گوساله پرواری را ذبح کرده است زیرا که او را صحیح باز‌یافت.
\par 28 ولی او خشم نموده نخواست به خانه درآید تا پدرش بیرون آمده به او التماس نمود.
\par 29 اما او در جواب پدرخود گفت، اینک سالها است که من خدمت توکرده‌ام و هرگز از حکم تو تجاوز نورزیده و هرگزبزغاله‌ای به من ندادی تا با دوستان خود شادی کنم.
\par 30 لیکن چون این پسرت آمد که دولت تو رابا فاحشه‌ها تلف کرده است، برای او گوساله پرواری را ذبح کردی.
\par 31 او وی را گفت، ای فرزند تو همیشه با من هستی و آنچه از آن من است، مال تو است.ولی می‌بایست شادمانی کرد و مسرور شد زیرا که این برادر تو مرده بود، زنده گشت و گم شده بود، یافت گردید.»
\par 32 ولی می‌بایست شادمانی کرد و مسرور شد زیرا که این برادر تو مرده بود، زنده گشت و گم شده بود، یافت گردید.»

\chapter{16}

\par 1 و به شاگردان خود نیز گفت: «شخصی دولتمند را ناظری بود که از او نزد وی شکایت بردند که اموال او را تلف می‌کرد.
\par 2 پس او را طلب نموده، وی را گفت، این چیست که درباره تو شنیده‌ام؟ حساب نظارت خود را باز بده زیرا ممکن نیست که بعد از این نظارت کنی.
\par 3 ناظر با خود گفت چکنم؟ زیرا مولایم نظارت رااز من می‌گیرد. طاقت زمین‌کندن ندارم و از گدایی نیز عار دارم.
\par 4 دانستم چکنم تا وقتی که از نظارت معزول شوم، مرا به خانه خود بپذیرند.
\par 5 پس هریکی از بدهکاران آقای خود را طلبیده، به یکی گفت آقایم از تو چند طلب دارد؟
\par 6 گفت صدرطل روغن. بدو گفت سیاهه خود را بگیر ونشسته پنجاه رطل بزودی بنویس.
\par 7 باز دیگری را گفت از تو چقدر طلب دارد؟ گفت صد کیل گندم. وی را گفت سیاهه خود را بگیر و هشتادبنویس.
\par 8 «پس آقایش، ناظر خائن را آفرین گفت، زیراعاقلانه کار کرد. زیرا ابنای این جهان در طبقه خویش از ابنای نور عاقل تر هستند.
\par 9 و من شمارا می‌گویم دوستان از مال بی‌انصافی برای خودپیدا کنید تا چون فانی گردید شما را به خیمه های جاودانی بپذیرند.
\par 10 آنکه در اندک امین باشد درامر بزرگ نیز امین بود و آنکه در قلیل خائن بود در کثیر هم خائن باشد.
\par 11 و هرگاه در مال بی‌انصافی امین نبودید، کیست که مال حقیقی را به شمابسپارد؟
\par 12 و اگر در مال دیگری دیانت نکردید، کیست که مال خاص شما را به شما دهد؟
\par 13 هیچ خادم نمی تواند دو آقا را خدمت کند. زیرا یا از یکی نفرت می‌کند و با دیگری محبت، یا با یکی می‌پیوندد و دیگری را حقیر می‌شمارد. خدا و مامونا را نمی توانید خدمت نمایید.
\par 14 و فریسیانی که زر دوست بودند همه این سخنان را شنیده، او را استهزا نمودند.
\par 15 به ایشان گفت، شما هستید که خود را پیش مردم عادل می‌نمایید، لیکن خدا عارف دلهای شماست. زیرا که آنچه نزد انسان مرغوب است، نزد خدا مکروه است.
\par 16 تورات و انبیا تا به یحیی بود و از آن وقت بشارت به ملکوت خدا داده می‌شود و هر کس بجد و جهد داخل آن می‌گردد.
\par 17 لیکن آسانتر است که آسمان و زمین زایل شود، از آنکه یک نقطه از تورات ساقط گردد.
\par 18 هر‌که زن خود را طلاق دهد و دیگری را نکاح کند زانی بود و هر‌که زن مطلقه مردی را به نکاح خویش درآورد، زنا کرده باشد.
\par 19 شخصی دولتمند بود که ارغوان و کتان می‌پوشید و هر روزه در عیاشی با جلال بسرمی برد.
\par 20 و فقیری مقروح بود ایلعازر نام که او رابر درگاه او می‌گذاشتند،
\par 21 و آرزو می‌داشت که از پاره هایی که از خوان آن دولتمند می‌ریخت، خود را سیر کند. بلکه سگان نیز آمده زبان برزخمهای او می‌مالیدند.
\par 22 باری آن فقیر بمرد و فرشتگان، او را به آغوش ابراهیم بردند و آن دولتمند نیز مرد و او را دفن کردند.
\par 23 پس چشمان خود را در عالم اموات گشوده، خود رادر عذاب یافت. و ابراهیم را از دور و ایلعازر را درآغوشش دید.
\par 24 آنگاه به آواز بلند گفت، ای پدرمن ابراهیم، بر من ترحم فرما و ایلعازر را بفرست تا سر انگشت خود را به آب تر ساخته زبان مراخنک سازد، زیرا در این نار معذبم.
\par 25 ابراهیم گفت‌ای فرزند به‌خاطر آور که تو در ایام زندگانی چیزهای نیکوی خود را یافتی و همچنین ایلعازرچیزهای بد را، لیکن او الحال در تسلی است و تودر عذاب.
\par 26 و علاوه بر این در میان ما و شماورطه عظیمی است، چنانچه آنانی که می‌خواهنداز اینجا به نزد شما عبور کنند، نمی توانند و نه نشینندگان آنجا نزد ما توانند گذشت.
\par 27 گفت‌ای پدر به تو التماس دارم که او را به خانه پدرم بفرستی.
\par 28 زیرا که مرا پنج برادر است تا ایشان راآگاه سازد، مبادا ایشان نیز به این مکان عذاب بیایند.
\par 29 ابراهیم وی را گفت موسی و انبیا رادارند سخن ایشان را بشنوند.
\par 30 گفت نه‌ای پدرما ابراهیم، لیکن اگر کسی از مردگان نزد ایشان رود، توبه خواهند کرد.وی را گفت هرگاه موسی و انبیا را نشنوند اگر کسی از مردگان نیزبرخیزد، هدایت نخواهند پذیرفت.»
\par 31 وی را گفت هرگاه موسی و انبیا را نشنوند اگر کسی از مردگان نیزبرخیزد، هدایت نخواهند پذیرفت.»

\chapter{17}

\par 1 و شاگردان خود را گفت: «لابد است ازوقوع لغزشها، لیکن وای بر آن کسی‌که باعث آنها شود.
\par 2 او را بهتر می‌بود که سنگ آسیایی بر گردنش آویخته شود و در دریا افکنده شود از اینکه یکی از این کودکان را لغزش دهد.
\par 3 احتراز کنید و اگر برادرت به تو خطا ورزد او راتنبیه کن و اگر توبه کند او را ببخش.
\par 4 و هرگاه درروزی هفت کرت به تو گناه کند و در روزی هفت مرتبه، برگشته به تو گوید توبه می‌کنم، او راببخش.»
\par 5 آنگاه رسولان به خداوند گفتند: «ایمان ما رازیاد کن.»
\par 6 خداوند گفت: «اگر ایمان به قدر دانه خردلی می‌داشتید، به این درخت افراغ می‌گفتیدکه کنده شده در دریا نشانده شود اطاعت شمامی کرد.
\par 7 «اما کیست از شماکه غلامش به شخم کردن یا شبانی مشغول شود و وقتی که از صحرا آید به وی گوید، بزودی بیا و بنشین.
\par 8 بلکه آیا بدونمی گوید چیزی درست کن تا شام بخورم و کمرخود را بسته مرا خدمت کن تا بخورم و بنوشم وبعد از آن تو بخور و بیاشام؟
\par 9 آیا از آن غلام منت می‌کشد از آنکه حکمهای او را به‌جا آورد؟ گمان ندارم.
\par 10 همچنین شما نیز چون به هر چیزی که مامور شده‌اید عمل کردید، گویید که غلامان بی‌منفعت هستیم زیرا که آنچه بر ما واجب بود به‌جا آوردیم.»
\par 11 و هنگامی که سفر به سوی اورشلیم می‌کرداز میانه سامره و جلیل می‌رفت.
\par 12 و چون به قریه‌ای داخل می‌شد ناگاه ده شخص ابرص به استقبال او آمدند و از دور ایستاده،
\par 13 به آواز بلندگفتند: «ای عیسی خداوند بر ما ترحم فرما.»
\par 14 اوبه ایشان نظر کرده گفت: «بروید و خود را به کاهن بنمایید.» ایشان چون می‌رفتند، طاهر گشتند.
\par 15 و یکی از ایشان چون دید که شفا یافته است، برگشته به صدای بلند خدا را تمجید می‌کرد.
\par 16 وپیش قدم او به روی در‌افتاده وی را شکر کرد. و اواز اهل سامره بود.
\par 17 عیسی ملتفت شده گفت «آیا ده نفر طاهر نشدند، پس آن نه کجا شدند؟
\par 18 آیا هیچ‌کس یافت نمی شود که برگشته خدا راتمجید کند جز این غریب؟»
\par 19 و بدو گفت: «برخاسته برو که ایمانت تو را نجات داده است.»
\par 20 و چون فریسیان از او پرسیدند که ملکوت خدا کی می‌آید، او در جواب ایشان گفت: «ملکوت خدا با مراقبت نمی آید.
\par 21 و نخواهندگفت که در فلان یا فلان جاست. زیرا اینک ملکوت خدا در میان شما است.»
\par 22 و به شاگردان خود گفت: «ایامی می‌آید که آرزو خواهید داشت که روزی از روزهای پسر انسان را بینید ونخواهید دید.
\par 23 و به شما خواهند گفت، اینک در فلان یا فلان جاست، مروید و تعاقب آن مکنید.
\par 24 زیرا چون برق که از یک جانب زیرآسمان لامع شده تا جانب دیگر زیر آسمان درخشان می‌شود، پسر انسان در یوم خودهمچنین خواهد بود.
\par 25 لیکن اول لازم است که اوزحمات بسیار بیند و از این فرقه مطرود شود.
\par 26 و چنانکه در ایام نوح واقع شد، همانطوردر زمان پسر انسان نیز خواهد بود،
\par 27 که می‌خوردند و می‌نوشیدند و زن و شوهرمی گرفتند تا روزی که چون نوح داخل کشتی شد، طوفان آمده همه را هلاک ساخت.
\par 28 وهمچنان‌که در ایام لوط شد که به خوردن وآشامیدن و خرید و فروش و زراعت و عمارت مشغول می‌بودند،
\par 29 تا روزی که چون لوط ازسدوم بیرون آمد، آتش و گوگرد از آسمان باریدو همه را هلاک ساخت.
\par 30 بر همین منوال خواهدبود در روزی که پسر انسان ظاهر شود.
\par 31 در آن روز هر‌که بر پشت بام باشد و اسباب او در خانه نزول نکند تا آنها را بردارد و کسی‌که در صحراباشد همچنین برنگردد.
\par 32 زن لوط را بیاد آورید.
\par 33 هر‌که خواهد جان خود را برهاند آن را هلاک خواهد کرد و هر‌که آن را هلاک کند آن را زنده نگاه خواهد داشت.
\par 34 به شما می‌گویم در آن شب دو نفر بر یک تخت خواهند بود، یکی برداشته و دیگری واگذارده خواهد شد.
\par 35 و دوزن که در یک جا دستاس کنند، یکی برداشته ودیگری واگذارده خواهد شد.
\par 36 و دونفر که درمزرعه باشند، یکی برداشته و دیگری واگذارده خواهد شد.»در جواب وی گفتند: «کجا‌ای خداوند.» گفت: «در هر جایی که لاش باشد درآنجا کرکسان جمع خواهند شد.»
\par 37 در جواب وی گفتند: «کجا‌ای خداوند.» گفت: «در هر جایی که لاش باشد درآنجا کرکسان جمع خواهند شد.»

\chapter{18}

\par 1 و برای ایشان نیز مثلی آورد در اینکه می باید همیشه دعا کرد و کاهلی نورزید.
\par 2 پس گفت که «در شهری داوری بود که نه ترس از خدا و نه باکی از انسان می‌داشت.
\par 3 ودر همان شهر بیوه‌زنی بود که پیش وی آمده می‌گفت، داد مرا از دشمنم بگیر.
\par 4 و تا مدتی به وی اعتناننمود ولکن بعد از آن با خود گفت هرچند از خدا نمی ترسم و از مردم باکی ندارم،
\par 5 لیکن چون این بیوه‌زن مرا زحمت می‌دهد، به داد او می‌رسم، مبادا پیوسته آمده مرا به رنج آورد.
\par 6 خداوند گفت بشنوید که این داوربی انصاف چه می‌گوید؟
\par 7 و آیا خدا برگزیدگان خود را که شبانه‌روز بدو استغاثه می‌کنند، دادرسی نخواهد کرد، اگرچه برای ایشان دیرغضب باشد؟
\par 8 به شما می‌گویم که به زودی دادرسی ایشان را خواهد کرد. لیکن چون پسرانسان آید، آیا ایمان را بر زمین خواهدیافت؟
\par 9 و این مثل را آورد برای بعضی که بر خوداعتماد می‌داشتند که عادل بودند و دیگران راحقیر می‌شمردند.
\par 10 که «دو نفر یکی فریسی ودیگری باجگیر به هیکل رفتند تا عبادت کنند.
\par 11 آن فریسی ایستاده بدینطور با خود دعا کرد که خدایا تو را شکر می‌کنم که مثل سایر مردم حریص و ظالم و زناکار نیستم و نه مثل این باجگیر،
\par 12 هر هفته دو مرتبه روزه می‌دارم و ازآنچه پیدا می‌کنم ده‌یک می‌دهم.
\par 13 اما آن باجگیر دور ایستاده نخواست چشمان خود را به سوی آسمان بلند کند بلکه به سینه خود زده گفت، خدایا بر من گناهکار ترحم فرما.
\par 14 به شمامی گویم که این شخص، عادل کرده شده به خانه خود رفت به خلاف آن دیگر، زیرا هر‌که خود رابرافرازد، پست گردد و هرکس خویشتن را فروتن سازد، سرافرازی یابد.»
\par 15 پس اطفال را نیز نزد وی آوردند تا دست برایشان گذارد. اما شاگردانش چون دیدند، ایشان رانهیب دادند.
\par 16 ولی عیسی ایشان را خوانده، گفت: «بچه‌ها را واگذارید تا نزد من آیند و ایشان را ممانعت مکنید، زیرا ملکوت خدا برای مثل اینها است.
\par 17 هرآینه به شما می‌گویم هر‌که ملکوت خدا را مثل طفل نپذیرد داخل آن نگردد.»
\par 18 و یکی از روسا از وی سوال نموده، گفت: «ای استاد نیکو چه کنم تا حیات جاودانی راوارث گردم؟»
\par 19 عیسی وی را گفت: «از بهر‌چه مرا نیکو می‌گویی و حال آنکه هیچ‌کس نیکونیست جز یکی که خدا باشد.
\par 20 احکام رامی دانی زنا مکن، قتل مکن، دزدی منما، شهادت دروغ مده و پدر و مادر خود را محترم دار.»
\par 21 گفت: «جمیع اینها را از طفولیت خود نگاه داشته‌ام.»
\par 22 عیسی چون این را شنید بدو گفت: «هنوز تو را یک چیز باقی است. آنچه داری بفروش و به فقرا بده که در آسمان گنجی خواهی داشت، پس آمده مرا متابعت کن.»
\par 23 چون این راشنید محزون گشت، زیرا که دولت فراوان داشت.
\par 24 اما عیسی چون او را محزون دید گفت: «چه دشوار است که دولتمندان داخل ملکوت خداشوند.
\par 25 زیرا گذشتن شتر از سوراخ سوزن آسانتر است از دخول دولتمندی در ملکوت خدا.»
\par 26 اما شنوندگان گفتند: «پس که می‌تواندنجات یابد؟»
\par 27 او گفت: «آنچه نزد مردم محال است، نزد خدا ممکن است.»
\par 28 پطرس گفت: «اینک ما همه‌چیز را ترک کرده پیروی تو می‌کنیم.»
\par 29 به ایشان گفت: «هرآینه به شما می‌گویم، کسی نیست که خانه یاوالدین یا زن یا برادران یا اولاد را بجهت ملکوت خدا ترک کند،
\par 30 جز اینکه در این عالم چند برابربیابد و در عالم آینده حیات جاودانی را.»
\par 31 پس آن دوازده را برداشته، به ایشان گفت: «اینک به اورشلیم می‌رویم و آنچه به زبان انبیادرباره پسر انسان نوشته شده است، به انجام خواهد رسید.
\par 32 زیرا که او را به امت‌ها تسلیم می‌کنند و استهزا و بی‌حرمتی کرده آب دهان بروی انداخته
\par 33 و تازیانه زده او را خواهند کشت ودر روز سوم خواهد برخاست.»
\par 34 اما ایشان چیزی از این امور نفهمیدند و این سخن از ایشان مخفی داشته شد و آنچه می‌گفت، درک نکردند.
\par 35 و چون نزدیک اریحا رسید، کوری بجهت گدایی بر سر راه نشسته بود.
\par 36 و چون صدای گروهی را که می‌گذشتند شنید، پرسید چه چیزاست؟
\par 37 گفتندش عیسی ناصری در گذر است.
\par 38 در حال فریاد برآورده گفت: «ای عیسی، ای پسر داود، بر من ترحم فرما.»
\par 39 و هرچند آنانی که پیش می‌رفتند او را نهیب می‌دادند تا خاموش شود، او بلندتر فریاد می‌زد که پسر داودا بر من ترحم فرما.
\par 40 آنگاه عیسی ایستاده فرمود تا او رانزد وی بیاورند. و چون نزدیک شد از وی پرسیده،
\par 41 گفت: «چه می‌خواهی برای توبکنم؟» عرض کرد: «ای خداوند، تا بینا شوم.»
\par 42 عیسی به وی گفت: «بینا شو که ایمانت تو راشفا داده است.»در ساعت بینایی یافته، خدا راتمجید‌کنان از عقب او افتاد و جمیع مردم چون این را دیدند، خدا را تسبیح خواندند.
\par 43 در ساعت بینایی یافته، خدا راتمجید‌کنان از عقب او افتاد و جمیع مردم چون این را دیدند، خدا را تسبیح خواندند.

\chapter{19}

\par 1 پس وارد اریحا شده، از آنجا می گذشت.
\par 2 که ناگاه شخصی زکی نام که رئیس باجگیران و دولتمند بود،
\par 3 خواست عیسی را ببیند که کیست و از کثرت خلق نتوانست، زیرا کوتاه قد بود.
\par 4 پس پیش دویده بردرخت افراغی برآمد تا او را ببیند. چونکه اومی خواست از آن راه عبور کند.
\par 5 و چون عیسی به آن مکان رسید، بالا نگریسته او را دید و گفت: «ای زکی بشتاب و به زیر بیا زیرا که باید امروز درخانه تو بمانم.»
\par 6 پس به زودی پایین شده او را به خرمی پذیرفت.
\par 7 و همه چون این را دیدند، همهمه‌کنان می‌گفتند که در خانه شخصی گناهکار به میهمانی رفته است.
\par 8 اما زکی برپا شده به خداوند گفت: «الحال‌ای خداوند نصف مایملک خود را به فقرامی دهم و اگر چیزی ناحق از کسی گرفته باشم، چهار برابر بدو رد می‌کنم.»
\par 9 عیسی به وی گفت: «امروز نجات در این خانه پیدا شد. زیرا که این شخص هم پسر ابراهیم است.
\par 10 زیرا که پسرانسان آمده است تا گمشده را بجوید و نجات‌بخشد.»
\par 11 و چون ایشان این را شنیدند او مثلی زیادکرده آورد چونکه نزدیک به اورشلیم بود و ایشان گمان می‌بردند که ملکوت خدا می‌باید در همان زمان ظهور کند.
\par 12 پس گفت: «شخصی شریف به دیار بعید سفر کرد تا ملکی برای خود گرفته مراجعت کند.
\par 13 پس ده نفر از غلامان خود راطلبیده ده قنطار به ایشان سپرده فرمود، تجارت کنید تا بیایم.
\par 14 اما اهل ولایت او، چونکه او رادشمن می‌داشتند ایلچیان در عقب او فرستاده گفتند، نمی خواهیم این شخص بر ما سلطنت کند.
\par 15 و چون ملک را گرفته مراجعت کرده بود، فرمود تا آن غلامانی را که به ایشان نقد سپرده بودحاضر کنند تا بفهمد هر یک چه سود نموده است.
\par 16 پس اولی آمده گفت، ای آقا قنطار تو ده قنطار دیگر نفع آورده است.
\par 17 بدو گفت آفرین‌ای غلام نیکو. چونکه بر چیز کم امین بودی بر ده شهر حاکم شو.
\par 18 و دیگری آمده گفت، ای آقاقنطار تو پنج قنطار سود کرده است.
\par 19 او را نیزفرمود بر پنج شهر حکمرانی کن.
\par 20 و سومی آمده گفت، ای آقا اینک قنطار تو موجود است، آن را در پارچه‌ای نگاه داشته‌ام.
\par 21 زیرا که از توترسیدم چونکه مرد تندخویی هستی. آنچه نگذارده‌ای، برمی داری و از آنچه نکاشته‌ای درومی کنی.
\par 22 به وی گفت، از زبان خودت بر توفتوی می‌دهم، ای غلام شریر. دانسته‌ای که من مرد تندخویی هستم که برمیدارم آنچه رانگذاشته‌ام و درو می‌کنم آنچه را نپاشیده‌ام.
\par 23 پس برای چه نقد مرا نزد صرافان نگذاردی تاچون آیم آن را با سود دریافت کنم؟
\par 24 پس به حاضرین فرمود قنطار را از این شخص بگیرید وبه صاحب ده قنطار بدهید.
\par 25 به او گفتند‌ای خداوند، وی ده قنطار دارد.
\par 26 زیرا به شمامی گویم به هر‌که دارد داده شود و هر‌که نداردآنچه دارد نیز از او گرفته خواهد شد.
\par 27 اما آن دشمنان من که نخواستند من بر ایشان حکمرانی نمایم، در اینجا حاضر ساخته پیش من به قتل رسانید.»
\par 28 و چون این را گفت، پیش رفته متوجه اورشلیم گردید.
\par 29 و چون نزدیک بیت‌فاجی وبیت عنیا بر کوه مسمی به زیتون رسید، دو نفر ازشاگردان خود را فرستاده،
\par 30 گفت: «به آن قریه‌ای که پیش روی شما است بروید و چون داخل آن شدید، کره الاغی بسته خواهید یافت که هیچ‌کس بر آن هرگز سوار نشده. آن را باز کرده بیاورید.
\par 31 و اگر کسی به شما گوید، چرا این راباز می‌کنید، به وی گویید خداوند او را لازم دارد.»
\par 32 پس فرستادگان رفته آن چنانکه بدیشان گفته بود یافتند.
\par 33 و چون کره را باز می‌کردند، مالکانش به ایشان گفتند چرا کره را باز می‌کنید؟
\par 34 گفتند خداوند او را لازم دارد.
\par 35 پس او را به نزد عیسی آوردند و رخت خود را بر کره افکنده، عیسی را سوار کردند.
\par 36 و هنگامی که او می‌رفت جامه های خود را در راه می‌گستردند.
\par 37 و چون نزدیک به‌سرازیری کوه زیتون رسید، تمامی شاگردانش شادی کرده، به آوازبلند خدا را حمد گفتن شروع کردند، به‌سبب همه قواتی که از او دیده بودند.
\par 38 و می‌گفتند مبارک باد آن پادشاهی که می‌آید، به نام خداوند سلامتی در آسمان و جلال در اعلی علیین باد.
\par 39 آنگاه بعضی از فریسیان از آن میان بدو گفتند: «ای استاد شاگردان خود را نهیب نما.»
\par 40 او درجواب ایشان گفت: «به شما می‌گویم اگراینها ساکت شوند، هرآینه سنگها به صداآیند.»
\par 41 و چون نزدیک شده، شهر را نظاره کرد برآن گریان گشته،
\par 42 گفت: «اگر تو نیز می‌دانستی هم در این زمان خود آنچه باعث سلامتی تومیشد، لاکن الحال از چشمان تو پنهان گشته است.
\par 43 زیرا ایامی بر تو می‌آید که دشمنانت گرد تو سنگرها سازند و تو را احاطه کرده از هرجانب محاصره خواهند نمود.
\par 44 و تو را وفرزندانت را در اندرون تو بر خاک خواهند افکندو در تو سنگی بر سنگی نخواهند گذاشت زیرا که ایام تفقد خود را ندانستی.»
\par 45 و چون داخل هیکل شد، کسانی را که درآنجا خرید و فروش می‌کردند، به بیرون نمودن آغاز کرد.
\par 46 و به ایشان گفت: «مکتوب است که خانه من خانه عبادت است لیکن شما آن را مغاره دزدان ساخته‌اید.»
\par 47 و هر روز در هیکل تعلیم می‌داد، اما روسای کهنه و کاتبان و اکابر قوم قصدهلاک نمودن او می‌کردند.و نیافتند چه کنندزیرا که تمامی مردم بر او آویخته بودند که از اوبشنوند.
\par 48 و نیافتند چه کنندزیرا که تمامی مردم بر او آویخته بودند که از اوبشنوند.

\chapter{20}

\par 1 روزی از آن روزها واقع شد هنگامی که او قوم را در هیکل تعلیم و بشارت می‌داد که روسا کهنه و کاتبان با مشایخ آمده،
\par 2 به وی گفتند: «به ما بگو که به چه قدرت این کارها رامی کنی و کیست که این قدرت را به تو داده است؟»
\par 3 در جواب ایشان گفت: «من نیز از شماچیزی می‌پرسم. به من بگویید.
\par 4 تعمید یحیی ازآسمان بود یا از مردم؟»
\par 5 ایشان با خود اندیشیده، گفتند که اگر گوییم از آسمان، هرآینه گوید چرا به او ایمان نیاوردید؟
\par 6 و اگر گوییم از انسان، تمامی قوم ما را سنگسار کنند زیرا یقین می‌دارند که یحیی نبی است.»
\par 7 پس جواب دادند که «نمی دانیم از کجا بود.»
\par 8 عیسی به ایشان گفت: «من نیز شما را نمی گویم که این کارها را به چه قدرت به‌جا می‌آورم.»
\par 9 و این مثل را به مردم گفتن گرفت که «شخصی تاکستانی غرس کرد و به باغبانش سپرده مدت مدیدی سفر کرد.
\par 10 و در موسم غلامی نزدباغبانان فرستاد تا از میوه باغ بدو سپارند. اماباغبانان او را زده، تهی‌دست بازگردانیدند.
\par 11 پس غلامی دیگر روانه نمود. او را نیز تازیانه زده بی‌حرمت کرده، تهی‌دست بازگردانیدند.
\par 12 و بازسومی فرستاد. او را نیز مجروح ساخته بیرون افکندند.
\par 13 آنگاه صاحب باغ گفت چه کنم؟ پسرحبیب خود را می‌فرستم شاید چون او را بینند احترام خواهند نمود.
\par 14 اما چون باغبانان او رادیدند، با خود تفکرکنان گفتند، این وارث می‌باشد، بیایید او را بکشیم تا میراث از آن ماگردد.
\par 15 در حال او را از باغ بیرون افکنده کشتند. پس صاحب باغ بدیشان چه خواهد کرد؟
\par 16 اوخواهد آمد و باغبانان را هلاک کرده باغ را به دیگران خواهد سپرد.» پس چون شنیدند گفتندحاشا.
\par 17 به ایشان نظر افکنده گفت: «پس معنی‌این نوشته چیست، سنگی را که معماران ردکردند، همان سر زاویه شده است.
\par 18 و هر‌که برآن سنگ افتد خرد شود، اما اگر آن بر کسی بیفتداو را نرم خواهد ساخت؟»
\par 19 آنگاه روسای کهنه و کاتبان خواستند که در همان ساعت او را گرفتارکنند. لیکن از قوم ترسیدند زیرا که دانستند که این مثل را درباره ایشان زده بود.
\par 20 و مراقب او بوده جاسوسان فرستادند که خود را صالح می‌نمودند تا سخنی از او گرفته، اورا به حکم و قدرت والی بسپارند.
\par 21 پس از اوسوال نموده گفتند: «ای استاد می‌دانیم که تو به راستی سخن می‌رانی و تعلیم می‌دهی و از کسی روداری نمی کنی، بلکه طریق خدا را به صدق می‌آموزی،
\par 22 آیا بر ما جایز هست که جزیه به قیصر بدهیم یا نه؟»
\par 23 او چون مکر ایشان را درک کرد، بدیشان گفت: «مرا برای چه امتحان می‌کنید؟
\par 24 دیناری به من نشان دهید. صورت ورقمش از کیست؟ «ایشان در جواب گفتند: «از قیصر است.»
\par 25 او به ایشان گفت: «پس مال قیصررا به قیصر رد کنید و مال خدا را به خدا.»
\par 26 پس چون نتوانستند او را به سخنی در نظر مردم ملزم سازند، از جواب او در عجب شده ساکت ماندند.
\par 27 و بعضی از صدوقیان که منکر قیامت هستند، پیش آمده از وی سوال کرده،
\par 28 گفتند: «ای استاد، موسی برای ما نوشته است که اگرکسی را برادری که زن داشته باشد بمیرد وبی اولاد فوت شود، باید برادرش آن زن را بگیردتا برای برادر خود نسلی آورد.
\par 29 پس هفت برادربودند که اولی زن گرفته اولاد ناآورده، فوت شد.
\par 30 بعد دومین آن زن را گرفته، او نیز بی‌اولاد بمرد.
\par 31 پس سومین او را گرفت و همچنین تا هفتمین وهمه فرزند ناآورده، مردند.
\par 32 و بعد از همه، آن زن نیز وفات یافت.
\par 33 پس در قیامت، زن کدام‌یک از ایشان خواهد بود، زیرا که هر هفت او راداشتند؟»
\par 34 عیسی در جواب ایشان گفت: «ابنای این عالم نکاح می‌کنند و نکاح کرده می‌شوند.
\par 35 لیکن آنانی که مستحق رسیدن به آن عالم و به قیامت از مردگان شوند، نه نکاح می‌کنند و نه نکاح کرده می‌شوند.
\par 36 زیرا ممکن نیست که دیگربمیرند از آن جهت که مثل فرشتگان و پسران خدامی باشند، چونکه پسران قیامت هستند.
\par 37 و امااینکه مردگان برمی خیزند، موسی نیز در ذکر بوته نشان داد، چنانکه خداوند را خدای ابراهیم وخدای اسحاق و خدای یعقوب خواند.
\par 38 و حال آنکه خدای مردگان نیست بلکه خدای زندگان است. زیرا همه نزد او زنده هستند.»
\par 39 پس بعضی از کاتبان در جواب گفتند: «ای استاد. نیکوگفتی.»
\par 40 و بعد از آن هیچ‌کس جرات آن نداشت که از وی سوالی کند.
\par 41 پس به ایشان گفت: «چگونه می‌گویند که مسیح پسر داود است
\par 42 و خود داود در کتاب زبور می‌گوید، خداوند به خداوند من گفت به‌دست راست من بنشین
\par 43 تا دشمنان تو راپای انداز تو سازم؟
\par 44 پس چون داود او راخداوند می‌خواند چگونه پسر او می‌باشد؟»
\par 45 و چون تمامی قوم می‌شنیدند، به شاگردان خود گفت:
\par 46 «بپرهیزید از کاتبانی که خرامیدن در لباس دراز را می‌پسندند و سلام در بازارها وصدر کنایس و بالا نشستن در ضیافتها را دوست می‌دارند.و خانه های بیوه‌زنان را می‌بلعند ونماز را به ریاکاری طول می‌دهند. اینها عذاب شدیدتر خواهند یافت.»
\par 47 و خانه های بیوه‌زنان را می‌بلعند ونماز را به ریاکاری طول می‌دهند. اینها عذاب شدیدتر خواهند یافت.»

\chapter{21}

\par 1 و نظر کرده دولتمندانی را دید که هدایای خود را در بیت‌المال می‌اندازند.
\par 2 و بیوه‌زنی فقیر را دید که دو فلس در آنجا انداخت.
\par 3 پس گفت: «هرآینه به شمامی گویم این بیوه فقیر از جمیع آنها بیشترانداخت.
\par 4 زیرا که همه ایشان از زیادتی خود درهدایای خدا انداختند، لیکن این زن از احتیاج خود تمامی معیشت خویش را انداخت.
\par 5 و چون بعضی ذکر هیکل می‌کردند که به سنگهای خوب و هدایا آراسته شده است گفت:
\par 6 «ایامی می‌آید که از این چیزهایی که می‌بینید، سنگی بر سنگی گذارده نشود، مگر اینکه به زیرافکنده خواهد شد.
\par 7 و از او سوال نموده، گفتند: «ای استاد پس این امور کی واقع می‌شود وعلامت نزدیک شدن این وقایع چیست؟»
\par 8 گفت: «احتیاط کنید که گمراه نشوید. زیرا که بسا به نام من آمده خواهند گفت که من هستم و وقت نزدیک است. پس از عقب ایشان مروید.
\par 9 و چون اخبارجنگها و فسادها را بشنوید، مضطرب مشویدزیرا که وقوع این امور اول ضرور است لیکن انتهادر ساعت نیست.»
\par 10 پس به ایشان گفت: «قومی با قومی ومملکتی با مملکتی مقاومت خواهند کرد.
\par 11 وزلزله های عظیم در جایها و قحطیها و وباها پدیدو چیزهای هولناک و علامات بزرگ از آسمان ظاهر خواهد شد.
\par 12 و قبل از این همه، بر شمادست اندازی خواهند کرد و جفا نموده شما را به کنایس و زندانها خواهند سپرد و در حضورسلاطین و حکام بجهت نام من خواهند برد.
\par 13 واین برای شما به شهادت خواهد انجامید.
\par 14 پس در دلهای خود قرار دهید که برای حجت آوردن، پیشتر اندیشه نکنید،
\par 15 زیرا که من به شما زبانی وحکمتی خواهم داد که همه دشمنان شما با آن مقاومت و مباحثه نتوانند نمود.
\par 16 و شما راوالدین و برادران و خویشان و دوستان تسلیم خواهند کرد و بعضی از شما را به قتل خواهندرسانید.
\par 17 و جمیع مردم به جهت نام من شما رانفرت خواهند کرد.
\par 18 ولکن مویی از سر شما گم نخواهد شد.
\par 19 جانهای خود را به صبر دریابید.
\par 20 «و چون بینید که اورشلیم به لشکرهامحاصره شده است آنگاه بدانید که خرابی آن رسیده است.
\par 21 آنگاه هر‌که در یهودیه باشد، به کوهستان فرار کند و هر‌که در شهر باشد، بیرون رود و هر‌که در صحرا بود، داخل شهر نشود.
\par 22 زیرا که همان است ایام انتقام، تا آنچه مکتوب است تمام شود.
\par 23 لیکن وای بر آبستنان وشیردهندگان در آن ایام، زیرا تنگی سخت برروی زمین و غضب بر این قوم حادث خواهد شد.
\par 24 و به دم شمشیر خواهند افتاد و در میان جمیع امت‌ها به اسیری خواهند رفت و اورشلیم پایمال امت‌ها خواهد شد تا زمانهای امت‌ها به انجام رسد.
\par 25 و در آفتاب و ماه و ستارگان علامات خواهد بود و بر زمین تنگی و حیرت از برای امت‌ها روی خواهد نمود به‌سبب شوریدن دریا وامواجش.
\par 26 و دلهای مردم ضعف خواهد کرد از خوف و انتظار آن وقایعی که برربع مسکون ظاهرمی شود، زیرا قوات آسمان متزلزل خواهد شد.
\par 27 و آنگاه پسر انسان را خواهند دید که بر ابری سوار شده با قوت و جلال عظیم می‌آید.
\par 28 «و چون ابتدای این چیزها بشود راست شده، سرهای خود را بلند کنید از آن جهت که خلاصی شما نزدیک است.»
\par 29 و برای ایشان مثلی گفت که «درخت انجیر و سایر درختان راملاحظه نمایید،
\par 30 که چون می‌بینید شکوفه می‌کند خود می‌دانید که تابستان نزدیک است.
\par 31 و همچنین شما نیز چون بینید که این امور واقع می‌شود، بدانید که ملکوت خدا نزدیک شده است.
\par 32 هرآینه به شما می‌گویم که تا جمیع این امور واقع نشود، این فرقه نخواهد گذشت.
\par 33 آسمان و زمین زایل می‌شود لیکن سخنان من زایل نخواهد شد.
\par 34 پس خود را حفظ کنید مبادا دلهای شما ازپرخوری و مستی و اندیشه های دنیوی، سنگین گردد و آن روز ناگهان بر شما آید.
\par 35 زیرا که مثل دامی بر جمیع سکنه تمام روی زمین خواهد آمد.
\par 36 پس در هر وقت دعا کرده، بیدار باشید تاشایسته آن شوید که از جمیع این چیزهایی که به وقوع خواهد پیوست نجات یابید و در حضورپسر انسان بایستید.»
\par 37 و روزها را در هیکل تعلیم می‌داد و شبها بیرون رفته، در کوه معروف به زیتون به‌سر می‌برد.و هر بامداد قوم نزد وی در هیکل می‌شتافتند تا کلام او را بشنوند.
\par 38 و هر بامداد قوم نزد وی در هیکل می‌شتافتند تا کلام او را بشنوند.

\chapter{22}

\par 1 و چون عید فطیر که به فصح معروف است نزدیک شد،
\par 2 روسای کهنه وکاتبان مترصد می‌بودند که چگونه او را به قتل رسانند، زیرا که از قوم ترسیدند.
\par 3 اما شیطان در یهودای مسمی به اسخریوطی که از‌جمله آن دوازده بود داخل گشت،
\par 4 و اورفته با روسای کهنه و سرداران سپاه گفتگو کرد که چگونه او را به ایشان تسلیم کند.
\par 5 ایشان شادشده با او عهد بستند که نقدی به وی دهند.
\par 6 و اوقبول کرده در صدد فرصتی برآمد که اورا درنهانی از مردم به ایشان تسلیم کند.
\par 7 اما چون روز فطیر که در آن می‌بایست فصح را ذبح کنند رسید،
\par 8 پطرس و یوحنا را فرستاده، گفت: «بروید و فصح را بجهت ما آماده کنید تابخوریم.»
\par 9 به وی گفتند: «در کجا می‌خواهی مهیا کنیم؟»
\par 10 ایشان را گفت: «اینک هنگامی که داخل شهر شوید، شخصی با سبوی آب به شمابرمی خورد. به خانه‌ای که او درآید، از عقب وی بروید،
\par 11 و به صاحب‌خانه گویید، استاد تو را می گوید مهمانخانه کجا است تا در آن فصح را باشاگردان خود بخورم.
\par 12 او بالاخانه‌ای بزرگ ومفروش به شما نشان خواهد داد در آنجا مهیاسازید.»
\par 13 پس رفته چنانکه به ایشان گفته بودیافتند و فصح را آماده کردند.
\par 14 و چون وقت رسید با دوازده رسول بنشست.
\par 15 و به ایشان گفت: «اشتیاق بی‌نهایت داشتم که پیش از زحمت دیدنم، این فصح را باشما بخورم.
\par 16 زیرا به شما می‌گویم از این دیگرنمی خورم تا وقتی که در ملکوت خدا تمام شود.»
\par 17 پس پیاله‌ای گرفته، شکر نمود و گفت: «این رابگیرید و در میان خود تقسیم کنید.
\par 18 زیرا به شما می‌گویم که تا ملکوت خدا نیاید، از میوه مودیگر نخواهم نوشید.»
\par 19 و نان را گرفته، شکرنمود و پاره کرده، به ایشان داد و گفت: «این است جسد من که برای شما داده می‌شود، این را به یادمن به‌جا آرید.»
\par 20 و همچنین بعد از شام پیاله راگرفت و گفت: «این پیاله عهد جدید است در خون من که برای شما ریخته می‌شود.
\par 21 لیکن اینک دست آن کسی‌که مرا تسلیم می‌کند با من در سفره است.
\par 22 زیرا که پسر انسان برحسب آنچه مقدراست، می‌رود لیکن وای بر آن کسی‌که او را تسلیم کند.»
\par 23 آنگاه از یکدیگر شروع کردند به پرسیدن که کدام‌یک از ایشان باشد که این کار بکند؟
\par 24 و در میان ایشان نزاعی نیز افتاد که کدام‌یک ازایشان بزرگتر می‌باشد؟
\par 25 آنگاه به ایشان گفت: «سلاطین امت‌ها بر ایشان سروری می‌کنند وحکام خود را ولی‌نعمت می‌خوانند.
\par 26 لیکن شماچنین مباشید، بلکه بزرگتر از شما مثل کوچکترباشد و پیشوا چون خادم.
\par 27 زیرا کدام‌یک بزرگتراست آنکه به غذا نشیند یا آنکه خدمت کند آیانیست آنکه نشسته است؟ لیکن من در میان شماچون خادم هستم.
\par 28 و شما کسانی می‌باشید که در امتحانهای من با من به‌سر بردید.
\par 29 و من ملکوتی برای شما قرار می‌دهم چنانکه پدرم برای من مقرر فرمود.
\par 30 تا در ملکوت من از خوان من بخورید و بنوشید و بر کرسیها نشسته بردوازده سبط اسرائیل داوری کنید.»
\par 31 پس خداوند گفت: «ای شمعون، ای شمعون، اینک شیطان خواست شما را چون گندم غربال کند،
\par 32 لیکن من برای تو دعا کردم تاایمانت تلف نشود و هنگامی که تو بازگشت کنی برادران خود را استوار نما.»
\par 33 به وی گفت: «ای خداوند حاضرم که با تو بروم حتی در زندان و درموت.»
\par 34 گفت: «تو را می‌گویم‌ای پطرس امروزخروس بانگ نزده باشد که سه مرتبه انکار خواهی کرد که مرا نمی شناسی.»
\par 35 و به ایشان گفت: «هنگامی که شما را بی‌کیسه و توشه‌دان و کفش فرستادم به هیچ‌چیز محتاج شدید؟» گفتند هیچ.
\par 36 پس به ایشان گفت: «لیکن الان هر‌که کیسه دارد، آن را بردارد و همچنین توشه‌دان را و کسی‌که شمشیر ندارد جامه خود را فروخته آن رابخرد.
\par 37 زیرا به شما می‌گویم که این نوشته در من می‌باید به انجام رسید، یعنی با گناهکاران محسوب شد زیرا هر‌چه در خصوص من است، انقضا دارد.
\par 38 گفتند: «ای خداوند اینک دوشمشیر.» به ایشان گفت: «کافی است.»
\par 39 و برحسب عادت بیرون شده به کوه زیتون رفت و شاگردانش از عقب او رفتند.
\par 40 و چون به آن موضع رسید، به ایشان گفت: «دعا کنید تا درامتحان نیفتید.»
\par 41 و او از ایشان به مسافت پرتاپ سنگی دور شده، به زانو درآمد و دعا کرده، گفت:
\par 42 «ای پدر اگر بخواهی این پیاله را از من بگردان، لیکن نه به خواهش من بلکه به اراده تو.»
\par 43 وفرشته‌ای از آسمان بر او ظاهر شده او را تقویت می‌نمود.
\par 44 پس به مجاهده افتاده به سعی بلیغتردعا کرد، چنانکه عرق او مثل قطرات خون بود که بر زمین می‌ریخت.
\par 45 پس از دعا برخاسته نزدشاگردان خود آمده ایشان را از حزن در خواب یافت.
\par 46 به ایشان گفت: «برای چه در خواب هستید؟ برخاسته دعا کنید تا در امتحان نیفتید.»
\par 47 و سخن هنوز بر زبانش بود که ناگاه جمعی آمدند و یکی از آن دوازده که یهودا نام داشت بردیگران سبقت جسته نزد عیسی آمد تا او راببوسد.
\par 48 و عیسی بدو گفت: «ای یهودا آیا به بوسه پسر انسان را تسلیم می‌کنی؟»
\par 49 رفقایش چون دیدند که چه می‌شود عرض کردند خداوندابه شمشیر بزنیم.
\par 50 و یکی از ایشان، غلام رئیس کهنه را زده، گوش راست او را از تن جدا کرد.
\par 51 عیسی متوجه شده گفت: «تا به این بگذارید.» وگوش او را لمس نموده، شفا داد.
\par 52 پس عیسی به روسای کهنه و سرداران سپاه هیکل و مشایخی که نزد او آمده بودند گفت: «گویا بر دزد با شمشیرها و چوبها بیرون آمدید.
\par 53 وقتی که هر روزه در هیکل با شما می‌بودم دست بر من دراز نکردید، لیکن این است ساعت شما و قدرت ظلمت.»
\par 54 پس او را گرفته بردند و به‌سرای رئیس کهنه آوردند و پطرس از دور از عقب می‌آمد.
\par 55 و چون در میان ایوان آتش افروخته گردش نشسته بودند، پطرس در میان ایشان بنشست.
\par 56 آنگاه کنیزکی چون او را در روشنی آتش نشسته دید بر او چشم دوخته گفت: «این شخص هم با او می‌بود.»
\par 57 او وی را انکار کرده گفت: «ای زن او را نمی شناسم.»
\par 58 بعد از زمانی دیگری او را دیده گفت: «تو از اینها هستی.» پطرس گفت: «ای مرد، من نیستم.»
\par 59 و چون تخمین یک ساعت گذشت یکی دیگر با تاکیدگفت: «بلاشک این شخص از رفقای او است زیراکه جلیلی هم هست.»
\par 60 پطرس گفت: «ای مردنمی دانم چه می‌گویی؟» در همان ساعت که این رامی گفت خروس بانگ زد.
\par 61 آنگاه خداوندروگردانیده به پطرس نظر افکند پس پطرس آن کلامی را که خداوند به وی گفته بود به‌خاطرآورد که قبل از بانگ زدن خروس سه مرتبه مراانکار خواهی کرد.
\par 62 پس پطرس بیرون رفته زارزار بگریست.
\par 63 و کسانی که عیسی را گرفته بودند، او راتازیانه زده استهزا نمودند.
\par 64 و چشم او را بسته طپانچه بر رویش زدند و از وی سوال کرده، گفتند: «نبوت کن که تو را زده است؟»
\par 65 و بسیارکفر دیگر به وی گفتند.
\par 66 و چون روز شد اهل شورای قوم یعنی روسای کهنه و کاتبان فراهم آمده در مجلس خوداو را آورده،
\par 67 گفتند: «اگر تو مسیح هستی به مابگو: «او به ایشان گفت: «اگر به شما گویم مراتصدیق نخواهید کرد.
\par 68 و اگر از شما سوال کنم جواب نمی دهید و مرا رها نمی کنید.
\par 69 لیکن بعداز این پسر انسان به طرف راست قوت خداخواهد نشست.»
\par 70 همه گفتند: «پس تو پسر خداهستی؟» او به ایشان گفت: «شما می‌گویید که من هستم.»گفتند: «دیگر ما را چه حاجت به شهادت است، زیرا خود از زبانش شنیدیم.»
\par 71 گفتند: «دیگر ما را چه حاجت به شهادت است، زیرا خود از زبانش شنیدیم.»

\chapter{23}

\par 1 پس تمام جماعت ایشان برخاسته، اورا نزد پیلاطس بردند.
\par 2 و شکایت بر اوآغاز نموده، گفتند: «این شخص را یافته‌ایم که قوم را گمراه می‌کند و از جزیه دادن به قیصر منع می‌نماید و می‌گوید که خود مسیح و پادشاه است.»
\par 3 پس پیلاطس از او پرسیده، گفت: «آیا توپادشاه یهود هستی؟» او در جواب وی گفت: «تومی گویی.»
\par 4 آنگاه پیلاطس به روسای کهنه وجمیع قوم گفت که «در این شخص هیچ عیبی نمی یابم.»
\par 5 ایشان شدت نموده گفتند که «قوم رامی شوراند و در تمام یهودیه از جلیل گرفته تا به اینجا تعلیم می‌دهد.»
\par 6 چون پیلاطس نام جلیل را شنید پرسید که «آیا این مرد جلیلی است؟»
\par 7 و چون مطلع شد که از ولایت هیرودیس است او را نزد وی فرستاد، چونکه هیرودیس در آن ایام در اورشلیم بود.
\par 8 اما هیرودیس چون عیسی را دید، بغایت شادگردید زیرا که مدت مدیدی بود می‌خو. است اورا ببیند چونکه شهرت او را بسیار شنیده بود ومترصد می‌بود که معجزه‌ای از او بیند.
\par 9 پس چیزهای بسیار از وی پرسید لیکن او به وی هیچ جواب نداد.
\par 10 و روسای کهنه و کاتبان حاضرشده به شدت تمام بر وی شکایت می‌نمودند.
\par 11 پس هیرودیس با لشکریان خود او را افتضاح نموده و استهزا کرده لباس فاخر بر او پوشانید ونزد پیلاطس او را باز فرستاد.
\par 12 و در همان روزپیلاطس و هیرودیس با یکدیگر مصالحه کردند، زیرا قبل از آن در میانشان عداوتی بود.
\par 13 پس پیلاطس روسای کهنه و سرادران وقوم را خوانده،
\par 14 به ایشان گفت: «این مرد را نزدمن آوردید که قوم را می‌شوراند. الحال من او رادر حضور شما امتحان کردم و از آنچه بر او ادعامی کنید اثری نیافتم.
\par 15 و نه هیرودیس هم زیراکه شما را نزد او فرستادم و اینک هیچ عمل مستوجب قتل از او صادر نشده است.
\par 16 پس اورا تنبیه نموده رها خواهم کرد.»
\par 17 زیرا او را لازم بود که هر عیدی کسی را برای ایشان آزاد کند.
\par 18 آنگاه همه فریاد کرده، گفتند: «او را هلاک کن وبرابا را برای ما رها فرما.»
\par 19 و او شخصی بود که به‌سبب شورش و قتلی که در شهر واقع شده بود، در زندان افکنده شده بود.
\par 20 باز پیلاطس نداکرده خواست که عیسی را رها کند.
\par 21 لیکن ایشان فریاد زده گفتند: «او را مصلوب کن، مصلوب کن.»
\par 22 بار سوم به ایشان گفت: «چرا؟ چه بدی کرده است؟ من در او هیچ علت قتل نیافتم. پس او را تادیب کرده رها می‌کنم.»
\par 23 اماایشان به صداهای بلند مبالغه نموده خواستند که مصلوب شود و آوازهای ایشان و روسای کهنه غالب آمد.
\par 24 پس پیلاطس فرمود که برحسب خواهش ایشان بشود.
\par 25 و آن کس را که به‌سبب شورش و قتل در زندان حبس بود که خواستند رها کرد و عیسی را به خواهش ایشان سپرد.
\par 26 و چون او را می‌بردند شمعون قیروانی را که از صحرا می‌آمد مجبور ساخته صلیب را بر اوگذاردند تا از عقب عیسی ببرد.
\par 27 و گروهی بسیاراز قوم و زنانی که سینه می‌زدند و برای او ماتم می‌گرفتند، در عقب او افتادند.
\par 28 آنگاه عیسی به سوی آن زنان روی گردانیده، گفت: «ای دختران اورشلیم برای من گریه مکنید، بلکه بجهت خود واولاد خود ماتم کنید.
\par 29 زیرا اینک ایامی می‌آیدکه در آنها خواهند گفت، خوشابحال نازادگان ورحمهایی که بار نیاوردند و پستانهایی که شیرندادند.
\par 30 و در آن هنگام به کوهها خواهند گفت که بر ما بیفتید و به تلها که ما را پنهان کنید.
\par 31 زیرااگر این کارها را به چوب تر کردند به چوب خشک چه خواهد شد؟»
\par 32 و دو نفر دیگر را که خطاکار بودند نیزآوردند تا ایشان را با او بکشند.
\par 33 و چون به موضعی که آن را کاسه سر می‌گویند رسیدند، اورا در آنجا با آن دو خطاکار، یکی بر طرف راست و دیگری بر چپ او مصلوب کردند.
\par 34 عیسی گفت: «ای پدر اینها را بیامرز، زیراکه نمی دانند چه می‌کنند.» پس جامه های او راتقسیم کردند و قرعه افکندند.
\par 35 و گروهی به تماشا ایستاده بودند. و بزرگان نیز تمسخرکنان با ایشان می‌گفتند: «دیگران را نجات داد. پس اگر او مسیح و برگزیده خدا می‌باشد خود رابرهاند.»
\par 36 و سپاهیان نیز او را استهزا می‌کردند و آمده او را سرکه می‌دادند،
\par 37 و می‌گفتند: «اگر توپادشاه یهود هستی خود را نجات ده.»
\par 38 و بر سراو تقصیرنامه‌ای نوشتند به خط یونانی و رومی وعبرانی که «این است پادشاه یهود.»
\par 39 و یکی از آن دو خطاکار مصلوب بر وی کفر گفت که «اگر تو مسیح هستی خود را و ما رابرهان.»
\par 40 اما آن دیگری جواب داده، او را نهیب کرد و گفت: «مگر تو از خدا نمی ترسی؟ چونکه تو نیز زیر همین حکمی.
\par 41 و اما ما به انصاف، چونکه جزای اعمال خود را یافته‌ایم، لیکن این شخص هیچ کار بی‌جا نکرده است.»
\par 42 پس به عیسی گفت: «ای خداوند، مرا به یاد آور هنگامی که به ملکوت خود آیی.»
\par 43 عیسی به وی گفت: «هرآینه به تو می‌گویم امروز با من در فردوس خواهی بود.»
\par 44 و تخمین از ساعت ششم تا ساعت نهم، ظلمت تمام روی زمین را فرو گرفت.
\par 45 وخورشید تاریک گشت و پرده قدس از میان بشکافت.
\par 46 و عیسی به آواز بلند صدا زده گفت: «ای پدر به‌دستهای تو روح خود را می‌سپارم.» این را بگفت و جان را تسلیم نمود.
\par 47 امایوزباشی چون این ماجرا را دید، خدا را تمجیدکرده، گفت: «در حقیقت، این مرد صالح بود.»
\par 48 وتمامی گروه که برای این تماشا جمع شده بودندچون این وقایع را دیدند، سینه زنان برگشتند.
\par 49 و جمیع آشنایان او از دور ایستاده بودند، با زنانی که از جلیل او را متابعت کرده بودند تا این امور راببینند.
\par 50 و اینک یوسف نامی از اهل شورا که مردنیکو و صالح بود،
\par 51 که در رای و عمل ایشان مشارکت نداشت و از اهل رامه بلدی از بلاد یهودبود و انتظار ملکوت خدا را می‌کشید،
\par 52 نزدیک پیلاطس آمده جسد عیسی را طلب نمود.
\par 53 پس آن را پایین آورده در کتان پیچید و در قبری که ازسنگ تراشیده بود و هیچ‌کس ابد در آن دفن نشده بود سپرد.
\par 54 و آن روز تهیه بود و سبت نزدیک می‌شد.
\par 55 و زنانی که در عقب او از جلیل آمده بودند از پی او رفتند و قبر و چگونگی گذاشته شدن بدن او را دیدند.پس برگشته، حنوط و عطریات مهیا ساختند و روز سبت را به حسب حکم آرام گرفتند.
\par 56 پس برگشته، حنوط و عطریات مهیا ساختند و روز سبت را به حسب حکم آرام گرفتند.

\chapter{24}

\par 1 پس در روز اول هفته هنگام سپیده صبح، حنوطی را که درست کرده بودندبا خود برداشته به‌سر قبر‌آمدند و بعضی دیگران همراه ایشان.
\par 2 و سنگ را از سر قبر غلطانیده دیدند.
\par 3 چون داخل شدند، جسد خداوند عیسی را نیافتند
\par 4 و واقع شد هنگامی که ایشان از این امر متحیر بودند که ناگاه دو مرد در لباس درخشنده نزد ایشان بایستادند.
\par 5 و چون ترسان شده سرهای خود را به سوی زمین افکنده بودند، به ایشان گفتند: «چرا زنده را از میان مردگان می‌طلبید؟
\par 6 در اینجا نیست، بلکه برخاسته است. به یاد آورید که چگونه وقتی که در جلیل بود شمارا خبر داده،
\par 7 گفت ضروری است که پسر انسان به‌دست مردم گناهکار تسلیم شده مصلوب گرددو روز سوم برخیزد.»
\par 8 پس سخنان او را به‌خاطرآوردند.
\par 9 و از سر قبر برگشته، آن یازده و دیگران را ازهمه این امور مطلع ساختند.
\par 10 و مریم مجدلیه ویونا و مریم مادر یعقوب و دیگر رفقای ایشان بودند که رسولان را از این چیزها مطلع ساختند.
\par 11 لیکن سخنان زنان را هذیان پنداشته باورنکردند.
\par 12 اما پطرس برخاسته، دوان دوان به سوی قبر رفت و خم شده کفن را تنها گذاشته دیدو از این ماجرا در عجب شده به خانه خود رفت.
\par 13 و اینک در همان روز دو نفر از ایشان می‌رفتند به سوی قریه‌ای که از اورشلیم به مسافت، شصت تیر پرتاب دور بود و عمواس نام داشت.
\par 14 و با یک دیگر از تمام این وقایع گفتگومی کردند.
\par 15 و چون ایشان در مکالمه و مباحثه می‌بودند، ناگاه خود عیسی نزدیک شده، با ایشان همراه شد.
\par 16 ولی چشمان ایشان بسته شد تا اورا نشناسند.
\par 17 او به ایشان گفت: «چه حرفها است که با یکدیگر می‌زنید و راه را به کدورت می‌پیمایید؟»
\par 18 یکی که کلیوپاس نام داشت درجواب وی گفت: «مگر تو در اورشلیم غریب وتنها هستی و از آنچه در این ایام در اینجا واقع شدواقف نیستی؟»
\par 19 به ایشان گفت: «چه چیزاست؟» گفتندش: «درباره عیسی ناصری که مردی بود نبی و قادر در فعل و قول در حضورخدا و تمام قوم،
\par 20 و چگونه روسای کهنه وحکام ما او را به فتوای قتل سپردند و او رامصلوب ساختند.
\par 21 اما ما امیدوار بودیم که همین است آنکه می‌باید اسرائیل را نجات دهد وعلاوه بر این همه، امروز از وقوع این امور روزسوم است،
\par 22 و بعضی از زنان ما هم ما را به حیرت انداختند که بامدادان نزد قبر رفتند،
\par 23 وجسد او را نیافته آمدند و گفتند که فرشتگان را دررویا دیدیم که گفتند او زنده شده است.
\par 24 وجمعی از رفقای ما به‌سر قبر رفته، آن چنانکه زنان گفته بودند یافتند لیکن او را ندیدند.»
\par 25 او به ایشان گفت: «ای بی‌فهمان وسست دلان از ایمان آوردن به انچه انبیا گفته‌اند.
\par 26 آیا نمی بایست که مسیح این زحمات را بیند تابه جلال خود برسد؟»
\par 27 پس از موسی و سایرانبیا شروع کرده، اخبار خود را در تمام کتب برای ایشان شرح فرمود.
\par 28 و چون به آن دهی که عازم آن بودندرسیدند، او قصد نمود که دورتر رود.
\par 29 و ایشان الحاح کرده، گفتند که «با ما باش. چونکه شب نزدیک است و روز به آخر رسیده.» پس داخل گشته با ایشان توقف نمود.
\par 30 و چون با ایشان نشسته بود نان را گرفته برکت داد و پاره کرده به ایشان داد.
\par 31 که ناگاه چشمانشان باز شده، او راشناختند و در ساعت از ایشان غایب شد.
\par 32 پس با یکدیگر گفتند: «آیا دل در درون مانمی سوخت، وقتی که در راه با ما تکلم می‌نمود وکتب را بجهت ما تفسیر می‌کرد؟»
\par 33 و در آن ساعت برخاسته به اورشلیم مراجعت کردند و آن یازده را یافتند که با رفقای خود جمع شده
\par 34 می گفتند: «خداوند در حقیقت برخاسته و به شمعون ظاهر شده است.»
\par 35 و آن دو نفر نیز ازسرگذشت راه و کیفیت شناختن او هنگام پاره کردن نان خبر دادند.
\par 36 و ایشان در این گفتگو می‌بودند که ناگاه عیسی خود در میان ایشان ایستاده، به ایشان گفت: «سلام بر شما باد.»
\par 37 اما ایشان لرزان وترسان شده گمان بردند که روحی می‌بینند.
\par 38 به ایشان گفت: «چرا مضطرب شدید و برای چه دردلهای شما شبهات روی می‌دهد؟
\par 39 دستها وپایهایم را ملاحظه کنید که من خود هستم و دست بر من گذارده ببینید، زیرا که روح گوشت واستخوان ندارد، چنانکه می‌نگرید که در من است.»
\par 40 این را گفت و دستها و پایهای خود رابدیشان نشان داد.
\par 41 و چون ایشان هنوز ازخوشی تصدیق نکرده، در عجب مانده بودند، به ایشان گفت: «چیز خوراکی در اینجا دارید؟»
\par 42 پس قدری از ماهی بریان و از شانه عسل به وی دادند.
\par 43 پس آن را گرفته پیش ایشان بخورد.
\par 44 و به ایشان گفت: «همین است سخنانی که وقتی با شما بودم گفتم ضروری است که آنچه درتورات موسی و صحف انبیا و زبور درباره من مکتوب است به انجام رسد.»
\par 45 و در آن وقت ذهن ایشان را روشن کرد تا کتب را بفهمند.
\par 46 و به ایشان گفت: «بر همین منوال مکتوب است وبدینطور سزاوار بود که مسیح زحمت کشد و روزسوم از مردگان برخیزد.
\par 47 و از اورشلیم شروع کرده، موعظه به توبه و آمرزش گناهان در همه امت‌ها به نام او کرده شود.
\par 48 و شما شاهد بر این امور هستید.
\par 49 و اینک، من موعود پدر خود را برشما می‌فرستم. پس شما در شهر اورشلیم بمانیدتا وقتی که به قوت از اعلی آراسته شوید.»
\par 50 پس ایشان را بیرون از شهر تا بیت عنیا بردو دستهای خود را بلند کرده، ایشان را برکت داد.
\par 51 و چنین شد که در حین برکت دادن ایشان، ازایشان جدا گشته، به سوی آسمان بالا برده شد.پس او را پرستش کرده، با خوشی عظیم به سوی اورشلیم برگشتند.
\par 52 پس او را پرستش کرده، با خوشی عظیم به سوی اورشلیم برگشتند.



\end{document}