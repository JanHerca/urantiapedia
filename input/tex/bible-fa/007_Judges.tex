\begin{document}

\title{داوران}

 
\chapter{1}

\par 1 و بعد از وفات یوشع واقع شد که بنی‌اسرائیل از خداوند سوال کرده، گفتند: «کیست که برای ما بر کنعانیان، اول برآید و باایشان جنگ نماید؟»
\par 2 خداوند گفت: «یهودابرآید، اینک زمین را به‌دست او تسلیم کرده‌ام.»
\par 3 و یهودا به برادر خود شمعون گفت: «به قرعه من همراه من برآی، و با کنعانیان جنگ کنیم، و من نیزهمراه تو به قرعه تو خواهم آمد.» پس شمعون همراه او رفت.
\par 4 و یهودا برآمد، و خداوندکنعانیان و فرزیان را به‌دست ایشان تسلیم نمود، وده هزار نفر از ایشان را در بازق کشتند.
\par 5 و ادونی بازق را در بازق یافته، با او جنگ کردند، وکنعانیان و فرزیان را شکست دادند.
\par 6 و ادونی بازق فرار کرد و او را تعاقب نموده، گرفتندش، وشستهای دست و پایش را بریدند.
\par 7 و ادونی بازق گفت: «هفتاد ملک با شستهای دست و پا بریده زیر سفره من خورده‌ها برمی چیدند، موافق آنچه من کردم خدا به من مکافات رسانیده است.» پس او را به اورشلیم آوردند و در آنجامرد.
\par 8 و بنی یهودا با اورشلیم جنگ کرده، آن راگرفتند، و آن را به دم شمشیر زده، شهر را به آتش سوزانیدند.
\par 9 و بعد از آن بنی یهودا فرود شدند تابا کنعانیانی که در کوهستان و جنوب و بیابان ساکن بودند، جنگ کنند.
\par 10 و یهودا بر کنعانیانی که در حبرون ساکن بودند برآمد، و اسم حبرون قبل از آن قریه اربع بود، و شیشای و اخیمان وتلمای را کشتند.
\par 11 و از آنجا بر ساکنان دبیر برآمد و اسم دبیرقبل از آن، قریه سفیر بود.
\par 12 و کالیب گفت: «آنکه قریه سفیر را زده، فتح نماید، دختر خود عکسه رابه او به زنی خواهم داد.»
\par 13 و عتنیئیل بن قنازبرادر کوچک کالیب آن را گرفت، پس دختر خودعکسه را به او به زنی داد.
\par 14 و چون دختر نزد وی آمد او را ترغیب کرد که از پدرش زمینی طلب کند و آن دختر از الاغ خود پیاده شده، کالیب وی را گفت: «چه می‌خواهی؟»
\par 15 به وی گفت: «مرابرکت ده زیرا که مرا در زمین جنوب ساکن گردانیدی، پس مرا چشمه های آب بده.» وکالیب چشمه های بالا و چشمه های پایین را به اوداد.
\par 16 و پسران قینی پدر زن موسی از شهرنخلستان همراه بنی یهودا به صحرای یهودا که به جنوب عراد است برآمده، رفتند و با قوم ساکن شدند.
\par 17 و یهودا همراه برادر خود شمعون رفت، و کنعانیانی را که در صفت ساکن بودند، شکست دادند، و آن را خراب کرده، اسم شهر راحرما نامیدند.
\par 18 و یهودا غزه و نواحی‌اش واشقلون و نواحی‌اش و عقرون و نواحی‌اش راگرفت.
\par 19 و خداوند با یهودا می‌بود، و او اهل کوهستان را بیرون کرد، لیکن ساکنان وادی رانتوانست بیرون کند، زیرا که ارابه های آهنین داشتند.
\par 20 و حبرون را برحسب قول موسی به کالیب دادند، و او سه پسر عناق را از آنجا بیرون کرد.
\par 21 و بنی بنیامین یبوسیان را که در اورشلیم ساکن بودند بیرون نکردند، و یبوسیان بابنی بنیامین تا امروز در اورشلیم ساکنند.
\par 22 و خاندان یوسف نیز به بیت ئیل برآمدند، وخداوند با ایشان بود.
\par 23 و خاندان یوسف بیت ئیل را جاسوسی کردند، و نام آن شهر قبل ازآن لوز بود.
\par 24 و کشیکچیان مردی را که از شهربیرون می‌آمد دیده، به وی گفتند: «مدخل شهر رابه ما نشان بده که با تو احسان خواهیم نمود.»
\par 25 پس مدخل شهر را به ایشان نشان داده، پس شهر را به دم شمشیر زدند، و آن مرد را با تمامی خاندانش رها کردند.
\par 26 و آن مرد به زمین حتیان رفته، شهری بنا نمود و آن را لوز نامید که تا امروزاسمش همان است.
\par 27 و منسی اهل بیت‌شان و دهات آن را و اهل تعنک و دهات آن و ساکنان دور و دهات آن وساکنان یبلعام و دهات آن و ساکنان مجدو ودهات آن را بیرون نکرد، و کنعانیان عزیمت داشتند که در آن زمین ساکن باشند.
\par 28 و چون اسرائیل قوی شدند، بر کنعانیان جزیه نهادند، لیکن ایشان را تمام بیرون نکردند.
\par 29 و افرایم کنعانیانی را که در جازر ساکن بودند، بیرون نکرد، پس کنعانیان در میان ایشان درجازر ساکن ماندند.
\par 30 و زبولون ساکنان فطرون وساکنان نهلول را بیرون نکرد، پس کنعانیان در میان ایشان ساکن ماندند، و جزیه بر آنها گذارده شد.
\par 31 و اشیر ساکنان عکو و ساکنان صیدون واحلب و اکزیب و حلبه و عفیق و رحوب را بیرون نکرد.
\par 32 پس اشیریان در میان کنعانیانی که ساکن آن زمین بودند سکونت گرفتند، زیرا که ایشان را بیرون نکردند.
\par 33 و نفتالی ساکنان بیت شمس و ساکنان بیت عنات را بیرون نکرد، پس در میان کنعانیانی که ساکن آن زمین بودند، سکونت گرفت. لیکن ساکنان بیت شمس و بیت عنات به ایشان جزیه می‌دادند.
\par 34 و اموریان بنی دان را به کوهستان مسدودساختند زیرا که نگذاشتند که به بیابان فرود آیند.
\par 35 پس اموریان عزیمت داشتند که در ایلون وشعلبیم در کوه حارس ساکن باشند، و لیکن چون دست خاندان یوسف قوت گرفت، جزیه برایشان گذارده شد.و حد اموریان از سر بالای عقربیم و از سالع تا بالاتر بود.
\par 36 و حد اموریان از سر بالای عقربیم و از سالع تا بالاتر بود.
 
\chapter{2}

\par 1 و فرشته خداوند از جلجال به بوکیم آمده، گفت: «شما را از مصر برآوردم و به زمینی که به پدران شما قسم خوردم داخل کردم، و گفتم عهد خود را با شما تا به ابد نخواهم شکست.
\par 2 پس شما با ساکنان این زمین عهد مبندید ومذبح های ایشان را بشکنید، لیکن شما سخن مرانشنیدید. این چه‌کار است که کرده‌اید؟
\par 3 لهذا من نیز گفتم ایشان را از حضور شما بیرون نخواهم کرد، و ایشان در کمرهای شما خارها خواهندبود، و خدایان ایشان برای شما دام خواهند بود.»
\par 4 و چون فرشته خداوند این سخنان را به تمامی اسرائیل گفت، قوم آواز خود را بلند کرده، گریستند.
\par 5 و آن مکان را بوکیم نام نهادند، و درآنجا برای خداوند قربانی گذرانیدند.
\par 6 و چون یوشع قوم را روانه نموده بود، بنی‌اسرائیل هر یکی به ملک خود رفتند تا زمین رابه تصرف آورند.
\par 7 و در تمامی ایام یوشع و تمامی ایام مشایخی که بعد از یوشع زنده ماندند، و همه کارهای بزرگ خداوند را که برای اسرائیل کرده بود، دیدند، قوم، خداوند را عبادت نمودند.
\par 8 ویوشع بن نون، بنده خداوند، چون صد و ده ساله بود، مرد.
\par 9 و او را در حدود ملکش در تمنه حارس در کوهستان افرایم به طرف شمال کوه جاعش دفن کردند.
\par 10 و تمامی آن طبقه نیز به پدران خودپیوستند، و بعد از ایشان طبقه دیگر برخاستند که خداوند و اعمالی را که برای اسرائیل کرده بود، ندانستند.
\par 11 و بنی‌اسرائیل در نظر خداوند شرارت ورزیدند، و بعلها را عبادت نمودند.
\par 12 و یهوه خدای پدران خود را که ایشان را از زمین مصربیرون آورده بود، ترک کردند، و خدایان غیر را ازخدایان طوایفی که در اطراف ایشان بودند پیروی نموده، آنها را سجده کردند. و خشم خداوند رابرانگیختند.
\par 13 و یهوه را ترک کرده، بعل وعشتاروت را عبادت نمودند.
\par 14 پس خشم خداوند بر اسرائیل افروخته شده، ایشان را به‌دست تاراج کنندگان سپرد تا ایشان را غارت نمایند، و ایشان را به‌دست دشمنانی که به اطراف ایشان بودند، فروخت، به حدی که دیگرنتوانستند با دشمنان خود مقاومت نمایند.
\par 15 و به هرجا که بیرون می‌رفتند، دست خداوند برای بدی بر ایشان می‌بود، چنانکه خداوند گفته، و چنانکه خداوند برای ایشان قسم خورده بود و به نهایت تنگی گرفتار شدند.
\par 16 و خداوند داوران برانگیزانید که ایشان را ازدست تاراج کنندگان نجات دادند.
\par 17 و باز داوران خود را اطاعت ننمودند، زیرا که در عقب خدایان غیر زناکار شده، آنها را سجده کردند، و از راهی که پدران ایشان سلوک می‌نمودند، و اوامرخداوند را اطاعت می‌کردند، به زودی برگشتند، و مثل ایشان عمل ننمودند.
\par 18 و چون خداوندبرای ایشان داوران برمی انگیخت خداوند با داورمی بود، و ایشان را در تمام ایام آن داور از دست دشمنان ایشان نجات می‌داد، زیرا که خداوند به‌خاطر ناله‌ای که ایشان از دست ظالمان وستم کنندگان خود برمی آوردند، پشیمان می‌شد.
\par 19 و واقع می‌شد چون داور وفات یافت که ایشان برمی گشتند و از پدران خود بیشتر فتنه انگیز شده، خدایان غیر را پیروی می‌کردند، و آنها را عبادت نموده، سجده می‌کردند، و از اعمال بد و راههای سرکشی خود چیزی باقی نمی گذاشتند.
\par 20 لهذاخشم خداوند بر اسرائیل افروخته شد و گفت: «چونکه این قوم از عهدی که با پدران ایشان امرفرمودم، تجاوز نموده، آواز مرا نشنیدند،
\par 21 من نیز هیچ‌یک از امتها را که یوشع وقت وفاتش واگذاشت، از حضور ایشان دیگر بیرون نخواهم نمود.
\par 22 تا اسرائیل را به آنها بیازمایم که آیاطریق خداوند را نگهداشته، چنانکه پدران ایشان نگهداشتند، در آن سلوک خواهند نمود یا نه.»پس خداوند آن طوایف را واگذاشته، به‌سرعت بیرون نکرد و آنها را به‌دست یوشع تسلیم ننمود.
\par 23 پس خداوند آن طوایف را واگذاشته، به‌سرعت بیرون نکرد و آنها را به‌دست یوشع تسلیم ننمود.
 
\chapter{3}

\par 1 پس اینانند طوایفی که خداوند واگذاشت تا به واسطه آنها اسرائیل را بیازماید، یعنی جمیع آنانی که همه جنگهای کنعان را ندانسته بودند.
\par 2 تا طبقات بنی‌اسرائیل دانشمند شوند وجنگ را به ایشان تعلیم دهد، یعنی آنانی که آن راپیشتر به هیچ وجه نمی دانستند.
\par 3 پنج سردارفلسطینیان و جمیع کنعانیان و صیدونیان و حویان که در کوهستان لبنان از کوه بعل حرمون تا مدخل حمات ساکن بودند.
\par 4 و اینها برای آزمایش بنی‌اسرائیل بودند، تا معلوم شود که آیا احکام خداوند را که به واسطه موسی به پدران ایشان امرفرموده بود، اطاعت خواهند کرد یا نه.
\par 5 پس بنی‌اسرائیل در میان کنعانیان و حتیان و اموریان وفرزیان و حویان و یبوسیان ساکن می‌بودند.
\par 6 دختران ایشان را برای خود به زنی می‌گرفتند، ودختران خود را به پسران ایشان می‌دادند، وخدایان آنها را عبادت می‌نمودند.
\par 7 و بنی‌اسرائیل آنچه در نظر خداوند ناپسندبود، کردند، و یهوه خدای خود را فراموش نموده، بعلها و بتها را عبادت کردند.
\par 8 و غضب خداوند بر اسرائیل افروخته شده، ایشان را به‌دست کوشان رشعتایم، پادشاه ارام نهرین، فروخت، و بنی‌اسرائیل کوشان رشعتایم را هشت سال بندگی کردند.
\par 9 و چون بنی‌اسرائیل نزدخداوند فریاد کردند، خداوند برای بنی‌اسرائیل نجات‌دهنده‌ای یعنی عتنئیل بن قناز برادر کوچک کالیب را برپا داشت، و او ایشان را نجات داد.
\par 10 وروح خداوند بر او نازل شد پس بنی‌اسرائیل راداوری کرد، و برای جنگ بیرون رفت، و خداوندکوشان رشعتایم، پادشاه ارام را به‌دست او تسلیم کرد، و دستش بر کوشان رشعتایم مستولی گشت.
\par 11 و زمین چهل سال آرامی یافت. پس عتنئیل بن قناز مرد.
\par 12 و بنی‌اسرائیل بار دیگر در نظر خداوندبدی کردند، و خداوند عجلون، پادشاه موآب رابر اسرائیل مستولی ساخت، زیرا که در نظرخداوند شرارت ورزیده بودند.
\par 13 و او بنی عمون و عمالیق را نزد خود جمع کرده، آمد، وبنی‌اسرائیل را شکست داد، و ایشان شهرنخلستان را گرفتند.
\par 14 و بنی‌اسرائیل عجلون، پادشاه موآب را هجده سال بندگی کردند.
\par 15 و چون بنی‌اسرائیل نزد خداوند فریادبرآوردند، خداوند برای ایشان نجات‌دهنده‌ای یعنی‌ایهود بن جیرای بنیامینی را که مردچپ دستی بود، برپا داشت، و بنی‌اسرائیل به‌دست او برای عجلون، پادشاه موآب، ارمغانی فرستادند.
\par 16 و ایهود خنجر دودمی که طولش یک ذراع بود، برای خود ساخت و آن را در زیرجامه بر ران راست خود بست.
\par 17 و ارمغان را نزدعجلون، پادشاه موآب عرضه داشت. و عجلون مرد بسیار فربهی بود.
\par 18 و چون از عرضه داشتن ارمغان فارغ شد، آنانی را که ارمغان را آورده بودند، روانه نمود.
\par 19 و خودش از معدنهای سنگ که نزد جلجال بود، برگشته، گفت: «ای پادشاه سخنی مخفی برای تو دارم.» گفت: «ساکت باش.» و جمیع حاضران از پیش او بیرون رفتند.
\par 20 و ایهود نزد وی داخل شد و او بتنهایی در بالاخانه تابستانی خود می‌نشست. ایهود گفت: «کلامی از خدا برای تو دارم.» پس از کرسی خودبرخاست. 
\par 21 و ایهود دست چپ خود را دراز کرده، خنجر را از ران راست خویش کشید و آن را در شکمش فرو برد.
\par 22 و دسته آن با تیغه‌اش نیز فرو رفت و پیه، تیغه را پوشانید زیرا که خنجررا از شکمش بیرون نکشید و به فضلات رسید.
\par 23 و ایهود به دهلیز بیرون رفته، درهای بالاخانه رابر وی بسته، قفل کرد.
\par 24 و چون او رفته بود، نوکرانش آمده، دیدندکه اینک درهای بالاخانه قفل است. گفتند، یقین پایهای خود را در غرفه تابستانی می‌پوشاند.
\par 25 وانتظار کشیدند تا خجل شدند، و چون او درهای بالاخانه را نگشود پس کلید را گرفته، آن را بازکردند، و اینک آقای ایشان بر زمین مرده افتاده بود.
\par 26 و چون ایشان معطل می‌شدند، ایهود به دررفت و از معدنهای سنگ گذشته، به سعیرت به سلامت رسید.
\par 27 و چون داخل آنجا شد کرنا رادر کوهستان افرایم نواخت و بنی‌اسرائیل همراهش از کوه به زیر آمدند، و او پیش روی ایشان بود.
\par 28 و به ایشان گفت: «از عقب من بیاییدزیرا خداوند، موآبیان، دشمنان شما را به‌دست شما تسلیم کرده است.» پس از عقب او فرودشده، معبرهای اردن را پیش روی موآبیان گرفتند، و نگذاشتند که احدی عبور کند.
\par 29 و درآن وقت به قدر ده هزار نفر از موآبیان را، یعنی هرزورآور و مرد جنگی را کشتند و کسی رهایی نیافت.
\par 30 و در آن روز موآبیان زیر دست اسرائیل ذلیل شدند، و زمین هشتاد سال آرامی یافت.و بعد از او شمجر بن عنات بود که ششصدنفر از فلسطینیان را با چوب گاورانی کشت، و اونیز اسرائیل را نجات داد.
\par 31 و بعد از او شمجر بن عنات بود که ششصدنفر از فلسطینیان را با چوب گاورانی کشت، و اونیز اسرائیل را نجات داد.
 
\chapter{4}

\par 1 و بنی‌اسرائیل بعد از وفات ایهود بار دیگردر نظر خداوند شرارت ورزیدند.
\par 2 وخداوند ایشان را به‌دست یابین، پادشاه کنعان، که در حاصور سلطنت می‌کرد، فروخت، و سردارلشکرش سیسرا بود که در حروشت امتهاسکونت داشت.
\par 3 و بنی‌اسرائیل نزد خداوندفریاد کردند، زیرا که او را نهصد ارابه آهنین بود وبر بنی‌اسرائیل بیست سال بسیار ظلم می‌کرد.
\par 4 و در آن زمان دبوره نبیه زن لفیدوت اسرائیل را داوری می‌نمود.
\par 5 و او زیر نخل دبوره که درمیان رامه و بیت ئیل در کوهستان افرایم بود، می‌نشست، و بنی‌اسرائیل به جهت داوری نزدوی می‌آمدند.
\par 6 پس او فرستاده، باراق بن ابینوعم را از قادش نفتالی طلبید و به وی گفت: «آیا یهوه، خدای اسرائیل، امر نفرموده است که برو و به کوه تابور رهنمایی کن، و ده هزار نفر از بنی نفتالی وبنی زبولون را همراه خود بگیر؟
\par 7 و سیسرا، سردار لشکر یابین را با ارابه‌ها و لشکرش به نهرقیشون نزد تو کشیده، او را به‌دست تو تسلیم خواهم کرد.»
\par 8 باراق وی را گفت: «اگر همراه من بیایی می‌روم و اگر همراه من نیایی نمی روم.»
\par 9 گفت: «البته همراه تو می‌آیم، لیکن این سفر که می‌روی برای تو اکرام نخواهد بود، زیرا خداوندسیسرا را به‌دست زنی خواهد فروخت.» پس دبوره برخاسته، همراه باراق به قادش رفت.
\par 10 وباراق، زبولون و نفتالی را به قادش جمع کرد وده هزار نفر در رکاب او رفتند، و دبوره همراهش برآمد.
\par 11 و حابر قینی خود را از قینیان یعنی ازبنی حوباب برادر زن موسی جدا کرده خیمه خویش را نزد درخت بلوط در صعنایم که نزد قادش است، برپا داشت.
\par 12 و به سیسرا خبر دادند که باراق بن ابینوعم به کوه تابور برآمده است.
\par 13 پس سیسرا همه ارابه هایش، یعنی نهصد ارابه آهنین و جمیع مردانی را که همراه وی بودند از حروشت امتها تانهر قیشون جمع کرد.
\par 14 و دبوره به باراق گفت: «برخیز، این است روزی که خداوند سیسرا را به‌دست تو تسلیم خواهد نمود، آیا خداوند پیش روی تو بیرون نرفته است؟» پس باراق از کوه تابور به زیر آمد و ده هزار نفر از عقب وی.
\par 15 وخداوند سیسرا و تمامی ارابه‌ها و تمامی لشکرش را به دم شمشیر پیش باراق منهزم ساخت، وسیسرا از ارابه خود به زیر آمده، پیاده فرار کرد.
\par 16 و باراق ارابه‌ها و لشکر را تا حروشت امتهاتعاقب نمود، و جمیع لشکر سیسرا به دم شمشیرافتادند، به حدی که کسی باقی نماند.
\par 17 و سیسرا به چادر یاعیل، زن حابر قینی، پیاده فرار کرد، زیرا که در میان یابین، پادشاه حاصور، و خاندان حابرقینی صلح بود.
\par 18 ویاعیل به استقبال سیسرا بیرون آمده، وی را گفت: «برگرد‌ای آقای من؛ به سوی من برگرد،
\par 19 ومترس.» پس به سوی وی به چادر برگشت و او رابه لحافی پوشانید. و او وی را گفت: «جرعه‌ای آب به من بنوشان، زیرا که تشنه هستم.» پس مشک شیر را باز کرده، به وی نوشانید و او راپوشانید.
\par 20 او وی را گفت: «به در چادر بایست واگر کسی بیاید و از تو سوال کرده، بگوید که آیاکسی در اینجاست، بگو نی.»
\par 21 و یاعیل زن حابرمیخ چادر را برداشت، و چکشی به‌دست گرفته، نزد وی به آهستگی آمده، میخ را به شقیقه اش کوبید، چنانکه به زمین فرو رفت، زیرا که او ازخستگی در خواب سنگین بود و بمرد.
\par 22 و اینک باراق سیسرا را تعاقب نمود و یاعیل به استقبالش بیرون آمده، وی را گفت: «بیا تا کسی را که می‌جویی تو را نشان بدهم.» پس نزد وی داخل شد و اینک سیسرا مرده افتاده، و میخ درشقیقه‌اش بود.
\par 23 پس در آن روز خدا یابین، پادشاه کنعان راپیش بنی‌اسرائیل ذلیل ساخت.و دست بنی‌اسرائیل بر یابین پادشاه کنعان زیاده و زیاده استیلا می‌یافت تا یابین، پادشاه کنعان را هلاک ساختند.
\par 24 و دست بنی‌اسرائیل بر یابین پادشاه کنعان زیاده و زیاده استیلا می‌یافت تا یابین، پادشاه کنعان را هلاک ساختند.
 
\chapter{5}

\par 1 و در آن روز دبوره و باراق بن ابینوعم سرودخوانده، گفتند:
\par 2 «چونکه پیش روان در اسرائیل پیشروی کردند، چونکه قوم نفوس خود را به ارادت تسلیم نمودند، خداوند را متبارک بخوانید.
\par 3 ‌ای پادشاهان بشنوید! ای زورآوران گوش دهید! من خود برای خداوند خواهم سرایید. برای یهوه خدای اسرائیل سرود خواهم خواند.
\par 4 ‌ای خداوند وقتی که از سعیر بیرون آمدی، وقتی که از صحرای ادوم خرامیدی، زمین متزلزل شد و آسمان نیز قطره‌ها ریخت. و ابرها هم آبهابارانید.
\par 5 کوهها از حضور خداوند لرزان شد و این سینا ازحضور یهوه، خدای اسرائیل.
\par 6 در ایام شمجر بن عنات، در ایام یاعیل شاهراههاترک شده بود، و مسافران از راههای غیرمتعارف می رفتند.
\par 7 حاکمان در اسرائیل نایاب و نابود شدند، تا من، دبوره، برخاستم، در اسرائیل، مادر برخاستم.
\par 8 خدایان جدید را اختیار کردند. پس جنگ دردروازه‌ها رسید. در میان چهل هزار نفر دراسرائیل، سپری و نیزه‌ای پیدا نشد.
\par 9 قلب من به حاکمان اسرائیل مایل است، که خودرا در میان قوم به ارادت تسلیم نمودند. خداوند رامتبارک بخوانید.
\par 10 ‌ای شما که بر الاغهای سفید سوارید و برمسندها می‌نشینید، و بر طریق سالک هستید، این را بیان کنید.
\par 11 دور از آواز تیراندازان، نزد حوضهای آب درآنجا اعمال عادله خداوند را بیان می‌کنند، یعنی احکام عادله او را در حکومت اسرائیل. آنگاه قوم خداوند به دروازه‌ها فرود می‌آیند.
\par 12 بیدار شو بیدار شو‌ای دبوره. بیدار شو بیدارشو و سرود بخوان. برخیز‌ای باراق و‌ای پسرابینوعم، اسیران خود را به اسیری ببر.
\par 13 آنگاه جماعت قلیل بر بزرگان قوم تسلطیافتند. و خداوند مرا بر جباران مسلط ساخت.
\par 14 از افرایم آمدند، آنانی که مقر ایشان در عمالیق است. در عقب تو بنیامین با قومهای تو، و از ماکیرداوران آمدند. و از زبولون آنانی که عصای صف آرا را به‌دست می‌گیرند.
\par 15 و سروران یساکار همراه دبوره بودند. چنانکه باراق بود همچنان یساکار نیز بود. در عقب او به وادی هجوم آوردند. فکرهای دل نزد شعوب روبین عظیم بود.
\par 16 چرا در میان آغلها نشستی. آیا تا نی گله‌ها رابشنوی؟ مباحثات دل، نزد شعوب روبین عظیم بود.
\par 17 جلعاد به آن طرف اردن ساکن ماند. و دان چرانزد کشتیها درنگ نمود. اشیر به کناره دریانشست. و نزد خلیجهای خود ساکن ماند.
\par 18 و زبولون قومی بودند که جان خود را به خطرموت تسلیم نمودند. و نفتالی نیز در بلندیهای میدان.
\par 19 پادشاهان آمده، جنگ کردند. آنگاه پادشاهان کنعان مقاتله نمودند. در تعنک نزد آبهای مجدو. و هیچ منفعت نقره نبردند.
\par 20 از آسمان جنگ کردند. ستارگان از منازل خودبا سیسرا جنگ کردند.
\par 21 نهر قیشون ایشان را در ربود. آن نهر قدیم یعنی نهر قیشون. ای جان من قوت را پایمال نمودی.
\par 22 آنگاه اسبان، زمین را پازدن گرفتند. به‌سبب تاختن یعنی تاختن زورآوران ایشان.
\par 23 فرشته خداوند می‌گوید میروز را لعنت کنید، ساکنانش را به سختی لعنت کنید، زیرا که به امدادخداوند نیامدند تا خداوند را در میان جباران اعانت نمایند.
\par 24 یاعیل، زن حابرقینی، از سایر زنان مبارک باد! از زنان چادرنشین مبارک باد!
\par 25 او آب خواست و شیر به وی داد، و سرشیر رادر ظرف ملوکانه پیش آورد.
\par 26 دست خود را به میخ دراز کرد، و دست راست خود را به چکش عمله. و به چکش سیسرا را زده، سرش را سفت. و شقیقه او را شکافت و فرودوخت.
\par 27 نزد پایهایش خم شده، افتاد و دراز شد. نزدپایهایش خم شده، افتاد. جایی که خم شد درآنجا کشته افتاد.
\par 28 از دریچه نگریست و نعره زد، مادر سیسرا از شبکه (نعره زد): چرا ارابه‌اش در‌آمدن تاخیرمی کند؟ و چرا چرخهای ارابه هایش توقف می‌نماید؟
\par 29 خاتونهای دانشمندش در جواب وی گفتند. لیکن او سخنان خود را به خود تکرار کرد.
\par 30 آیا غنیمت را نیافته، و تقسیم نمی کنند؟ یک دختر دو دختر برای هر مرد. و برای سیسراغنیمت رختهای رنگارنگ، غنیمت رختهای رنگارنگ قلابدوزی، رخت رنگارنگ قلابدوزی دورو. بر گردنهای اسیران.همچنین‌ای خداوند جمیع دشمنانت هلاک شوند. و اما محبان او مثل آفتاب باشند، وقتی که در قوتش طلوع می‌کند.»و زمین چهل سال آرامی یافت.
\par 31 همچنین‌ای خداوند جمیع دشمنانت هلاک شوند. و اما محبان او مثل آفتاب باشند، وقتی که در قوتش طلوع می‌کند.»و زمین چهل سال آرامی یافت.
 
\chapter{6}

\par 1 و بنی‌اسرائیل در نظر خداوند شرارت ورزیدند. پس خداوند ایشان را به‌دست مدیان هفت سال تسلیم نمود.
\par 2 و دست مدیان براسرائیل استیلا یافت، و به‌سبب مدیان بنی‌اسرائیل شکافها و مغاره‌ها و ملاذها را که درکوهها می‌باشند، برای خود ساختند.
\par 3 و چون اسرائیل زراعت می‌کردند، مدیان و عمالیق وبنی مشرق آمده، بر ایشان هجوم می‌آوردند.
\par 4 وبر ایشان اردو زده، محصول زمین را تا به غزه خراب کردند، و در اسرائیل آذوقه و گوسفند وگاو و الاغ باقی نگذاشتند.
\par 5 زیرا که ایشان بامواشی و خیمه های خود برآمده، مثل ملخ بی‌شمار بودند، و ایشان و شتران ایشان را حسابی نبود و به جهت خراب ساختن زمین داخل شدند.
\par 6 و چون اسرائیل به‌سبب مدیان بسیار ذلیل شدند، بنی‌اسرائیل نزد خداوند فریاد برآوردند.
\par 7 و واقع شد چون بنی‌اسرائیل از دست مدیان نزد خداوند استغاثه نمودند،
\par 8 که خداوند نبی‌ای برای بنی‌اسرائیل فرستاد، و او به ایشان گفت: «یهوه خدای اسرائیل چنین می‌گوید: من شما رااز مصر برآوردم و شما را از خانه بندگی بیرون آوردم،
\par 9 و شما را از دست مصریان و از دست جمیع ستمکاران شما رهایی دادم، و اینان را ازحضور شما بیرون کرده، زمین ایشان را به شمادادم.
\par 10 و به شما گفتم، من، یهوه، خدای شماهستم. از خدایان اموریانی که در زمین ایشان ساکنید، مترسید لیکن آواز مرا نشنیدید.»
\par 11 و فرشته خداوند آمده، زیر درخت بلوطی که در عفره است که مال یوآش ابیعزری بود، نشست و پسرش جدعون گندم رادر چرخشت می‌کوبید تا آن را از مدیان پنهان کند.
\par 12 پس فرشته خداوند بر او ظاهر شده، وی را گفت: «ای مرد زورآور، یهوه با تو است.»
\par 13 جدعون وی راگفت: «آه‌ای خداوند من، اگر یهوه با ماست پس چرا این همه بر ما واقع شده است، و کجاست جمیع اعمال عجیب او که پدران ما برای ما ذکرکرده، و گفته‌اند که آیا خداوند ما را از مصر بیرون نیاورد، لیکن الان خداوند ما را ترک کرده، و به‌دست مدیان تسلیم نموده است.» 
\par 14 آنگاه یهوه بروی نظر کرده، گفت: «به این قوت خود برو واسرائیل را از دست مدیان رهایی ده، آیا من تو رانفرستادم.»
\par 15 او در جواب وی گفت: «آه‌ای خداوند چگونه اسرائیل را رهایی دهم، اینک خاندان من در منسی ذلیل تر از همه است و من درخانه پدرم کوچکترین هستم.»
\par 16 خداوند وی راگفت: «یقین من با تو خواهم بود و مدیان را مثل یک نفر شکست خواهی داد.»
\par 17 او وی را گفت: «اگر الان در نظر تو فیض یافتم، پس آیتی به من بنما که تو هستی آنکه با من حرف می‌زنی.
\par 18 پس خواهش دارم که از اینجا نروی تا نزد تو برگردم، و هدیه خود را آورده، به حضور تو بگذرانم.» گفت: «من می‌مانم تا برگردی.»
\par 19 پس جدعون رفت و بزغاله‌ای را باقرصهای نان فطیر از یک ایفه آرد نرم حاضرساخت، و گوشت را در سبدی و آب گوشت را درکاسه‌ای گذاشته، آن را نزد وی، زیر درخت بلوطآورد و پیش وی نهاد.
\par 20 و فرشته خدا او را گفت: «گوشت و قرصهای فطیر را بردار و بر روی این صخره بگذار، و آب گوشت را بریز.» پس چنان کرد.
\par 21 آنگاه فرشته خداوند نوک عصا را که دردستش بود، دراز کرده، گوشت و قرصهای فطیررا لمس نمود که آتش از صخره برآمده، گوشت وقرصهای فطیر را بلعید، و فرشته خداوند ازنظرش غایب شد.
\par 22 پس جدعون دانست که اوفرشته خداوند است. و جدعون گفت: «آه‌ای خداوند یهوه، چونکه فرشته خداوند را روبرودیدم.»
\par 23 خداوند وی را گفت: «سلامتی بر توباد! مترس نخواهی مرد.»
\par 24 پس جدعون درآنجا برای خداوند مذبحی بنا کرد و آن را یهوه شالوم نامید که تا امروز در عفره ابیعزریان باقی است.
\par 25 و در آن شب، خداوند او را گفت: «گاو پدرخود، یعنی گاو دومین را که هفت ساله است بگیر، و مذبح بعل را که از آن پدرت است منهدم کن، وتمثال اشیره را که نزد آن است، قطع نما.
\par 26 وبرای یهوه، خدای خود، بر سر این قلعه مذبحی موافق رسم بنا کن، و گاو دومین را گرفته، با چوب اشیره که قطع کردی برای قربانی سوختنی بگذران.»
\par 27 پس جدعون ده نفر از نوکران خود رابرداشت و به نوعی که خداوند وی را گفته بود، عمل نمود، اما چونکه از خاندان پدر خود ومردان شهر می‌ترسید، این کار را در روز نتوانست کرد، پس آن را در شب کرد.
\par 28 و چون مردمان شهر در صبح برخاستند، اینک مذبح بعل منهدم شده، و اشیره که در نزد آن بود، بریده، و گاو دومین بر مذبحی که ساخته شده بود، قربانی گشته.
\par 29 پس به یکدیگر گفتند، کیست که این کار را کرده است، و چون دریافت وتفحص کردند، گفتند جدعون بن یوآش این کار راکرده است.
\par 30 پس مردان شهر به یوآش گفتند: «پسر خود را بیرون بیاور تا بمیرد زیرا که مذبح بعل را منهدم ساخته، و اشیره را که نزد آن بود، بریده است.»
\par 31 اما یوآش به همه کسانی که برضد او برخاسته بودند، گفت: «آیا شما برای بعل محاجه می‌کنید؟ و آیا شما او را می‌رهانید؟ هرکه برای او محاجه نماید همین صبح کشته شود، واگر او خداست، برای خود محاجه نماید چونکه کسی مذبح او را منهدم ساخته است.
\par 32 پس درآن روز او را یربعل نامید و گفت: «بگذارید تا بعل با او محاجه نماید زیرا که مذبح او را منهدم ساخته است.»
\par 33 آنگاه جمیع اهل مدیان و عمالیق وبنی مشرق با هم جمع شدند و عبور کرده، دروادی یزرعیل اردو زدند.
\par 34 و روح خداوندجدعون را ملبس ساخت، پس کرنا را نواخت واهل ابیعزر در عقب وی جمع شدند.
\par 35 ورسولان در تمامی منسی فرستاد که ایشان نیز درعقب وی جمع شدند و در اشیر و زبولون و نفتالی رسولان فرستاد و به استقبال ایشان برآمدند.
\par 36 و جدعون به خدا گفت: «اگر اسرائیل رابرحسب سخن خود به‌دست من نجات خواهی داد،
\par 37 اینک من در خرمنگاه، پوست پشمینی می گذارم و اگر شبنم فقط بر پوست باشد و برتمامی زمین خشکی بود، خواهم دانست که اسرائیل را برحسب قول خود به‌دست من نجات خواهی داد.»
\par 38 و همچنین شد و بامدادان به زودی برخاسته، پوست را فشرد و کاسه‌ای پر ازآب شبنم از پوست بیفشرد.
\par 39 و جدعون به خداگفت: «غضب تو بر من افروخته نشود و همین یک مرتبه خواهم گفت، یک دفعه دیگر فقط با پوست تجربه نمایم، این مرتبه پوست به تنهایی خشک باشد و بر تمامی زمین شبنم.»و خدا در آن شب چنان کرد که بر پوست فقط، خشکی بود و برتمامی زمین شبنم.
\par 40 و خدا در آن شب چنان کرد که بر پوست فقط، خشکی بود و برتمامی زمین شبنم.
 
\chapter{7}

\par 1 و یربعل که جدعون باشد با تمامی قوم که باوی بودند صبح زود برخاسته، نزد چشمه حرود اردو زدند، و اردوی مدیان به شمال ایشان نزد کوه موره در وادی بود.
\par 2 و خداوند به جدعون گفت: «قومی که با توهستند، زیاده از آنند که مدیان را به‌دست ایشان تسلیم نمایم، مبادا اسرائیل بر من فخر نموده، بگویند که دست ما، ما را نجات داد.
\par 3 پس الان به گوش قوم ندا کرده، بگو: هر کس که ترسان وهراسان باشد از کوه جلعاد برگشته، روانه شود.» وبیست و دو هزار نفر از قوم برگشتند و ده هزارباقی ماندند.
\par 4 و خداوند به جدعون گفت: «باز هم قوم زیاده‌اند، ایشان را نزد آب بیاور تا ایشان را آنجابرای تو بیازمایم، و هر‌که را به تو گویم این با توبرود، او همراه تو خواهد رفت، و هر‌که را به تو گویم این با تو نرود، او نخواهد رفت.»
\par 5 و چون قوم را نزد آب آورده بود، خداوند به جدعون گفت: «هر‌که آب را به زبان خود بنوشد چنانکه سگ می‌نوشد او را تنها بگذار، و همچنین هر‌که بر زانوی خود خم شده، بنوشد.»
\par 6 و عدد آنانی که دست به دهان آورده، نوشیدند، سیصد نفر بودو جمیع بقیه قوم بر زانوی خود خم شده، آب نوشیدند.
\par 7 و خداوند به جدعون گفت: «به این سیصد نفر که به کف نوشیدند، شما را نجات می‌دهم، و مدیان را به‌دست تو تسلیم خواهم نمود. پس سایر قوم هر کس به‌جای خود بروند.»
\par 8 پس آن گروه توشه و کرناهای خود را به‌دست گرفتند و هر کس را از سایر مردان اسرائیل به خیمه خود فرستاد، ولی آن سیصد نفر را نگاه داشت. و اردوی مدیان در وادی پایین دست اوبود.
\par 9 و در همان شب خداوند وی را گفت: «برخیزو به اردو فرود بیا زیرا که آن را به‌دست تو تسلیم نموده‌ام.
\par 10 لیکن اگر از رفتن می‌ترسی، با خادم خود فوره به اردو برو.
\par 11 و چون آنچه ایشان بگویند بشنوی، بعد از آن دست تو قوی خواهدشد، و به اردو فرود خواهی آمد.» پس او وخادمش، فوره به کناره سلاح دارانی که در اردوبودند، فرود آمدند.
\par 12 و اهل مدیان و عمالیق وجمیع بنی مشرق مثل ملخ، بی‌شمار در وادی ریخته بودند، و شتران ایشان را مثل ریگ که برکناره دریا بی‌حساب است، شماره‌ای نبود.
\par 13 پس چون جدعون رسید، دید که مردی به رفیقش خوابی بیان کرده، می‌گفت که «اینک خوابی دیدم، و هان گرده‌ای نان جوین در میان اردوی مدیان غلطانیده شده، به خیمه‌ای برخوردو آن را چنان زد که افتاد و آن را واژگون ساخت، چنانکه خیمه بر زمین پهن شد.»
\par 14 رفیقش درجواب وی گفت که «این نیست جز شمشیرجدعون بن یوآش، مرد اسرائیلی، زیرا خدامدیان و تمام اردو را به‌دست او تسلیم کرده است.»
\par 15 و چون جدعون نقل خواب و تعبیرش راشنید، سجده نمود، و به لشکرگاه اسرائیل برگشته، گفت: «برخیزید زیرا که خداوند اردوی مدیان را به‌دست شما تسلیم کرده است.»
\par 16 و آن سیصد نفر را به سه فرقه منقسم ساخت، و به‌دست هر یکی از ایشان کرناها و سبوهای خالی داد و مشعلها در سبوها گذاشت.
\par 17 و به ایشان گفت: «بر من نگاه کرده، چنان بکنید. پس چون به کنار اردو برسم، هر‌چه من می‌کنم، شما هم چنان بکنید.
\par 18 و چون من و آنانی که با من هستند کرناهارا بنوازیم، شما نیز از همه اطراف اردو کرناها رابنوازید و بگویید (شمشیر) خداوند و جدعون.»
\par 19 پس جدعون و صد نفر که با وی بودند درابتدای پاس دوم شب به کنار اردو رسیدند و درهمان حین کشیکچی‌ای تازه گذارده بودند، پس کرناها را نواختند و سبوها را که در دست ایشان بود، شکستند.
\par 20 و هر سه فرقه کرناها را نواختندو سبوها را شکستند و مشعلها را به‌دست چپ وکرناها را به‌دست راست خود گرفته، نواختند، وصدا زدند شمشیر خداوند و جدعون.
\par 21 و هرکس به‌جای خود به اطراف اردو ایستادند وتمامی لشکر فرار کردند و ایشان نعره زده، آنها رامنهزم ساختند.
\par 22 و چون آن سیصد نفر کرناها را نواختند، خداوند شمشیر هر کس را بر رفیقش وبر تمامی لشکر گردانید، و لشکر ایشان تابیت شطه به سوی صریرت و تا سر حد آبل محوله که نزد طبات است، فرار کردند.
\par 23 و مردان اسرائیل از نفتالی و اشیر و تمامی منسی جمع شده، مدیان را تعاقب نمودند.
\par 24 و جدعون به تمامی کوهستان افرایم، رسولان فرستاده، گفت: «به جهت مقابله با مدیان به زیر آیید و آبها را تا بیت باره و اردن پیش ایشان بگیرید.» پس تمامی مردان افرایم جمع شده، آبهارا تا بیت باره و اردن گرفتند.و غراب و ذئب، دو سردار مدیان را گرفته، غراب را بر صخره غراب و ذئب را در چرخشت ذئب کشتند، ومدیان را تعاقب نمودند، و سرهای غراب و ذئب را به آن طرف اردن، نزد جدعون آوردند.
\par 25 و غراب و ذئب، دو سردار مدیان را گرفته، غراب را بر صخره غراب و ذئب را در چرخشت ذئب کشتند، ومدیان را تعاقب نمودند، و سرهای غراب و ذئب را به آن طرف اردن، نزد جدعون آوردند.
 
\chapter{8}

\par 1 و مردان افرایم او را گفتند: «این چه‌کاراست که به ما کرده‌ای که چون برای جنگ مدیان می‌رفتی ما را نخواندی و به سختی با وی منازعت کردند.»
\par 2 او به ایشان گفت: «الان من بالنسبه به‌کار شما چه کردم؟ مگر خوشه چینی افرایم از میوه چینی ابیعزر بهتر نیست؟
\par 3 به‌دست شما خدا دو سردار مدیان، یعنی غراب و ذئب راتسلیم نمود و من مثل شما قادر بر چه‌کار بودم؟» پس چون این سخن را گفت، خشم ایشان بروی فرو نشست.
\par 4 و جدعون با آن سیصد نفر که همراه او بودندبه اردن رسیده، عبور کردند، و اگر‌چه خسته بودند، لیکن تعاقب می‌کردند.
\par 5 و به اهل سکوت گفت: «تمنا این که چند نان به رفقایم بدهید زیراخسته‌اند، و من زبح و صلمونع، ملوک مدیان راتعاقب می‌کنم.»
\par 6 سرداران سکوت به وی گفتند: «مگر دستهای زبح و صلمونع الان در دست تومی باشد تا به لشکر تو نان بدهیم.»
\par 7 جدعون گفت: «پس چون خداوند زبح و صلمونع را به‌دست من تسلیم کرده باشد، آنگاه گوشت شما رابا شوک و خار صحرا خواهم درید.»
\par 8 و از آنجا به فنوعیل برآمده، به ایشان همچنین گفت، و اهل فنوعیل مثل جواب اهل سکوت او را جواب دادند.
\par 9 و به اهل فنوعیل نیز گفت: «وقتی که به سلامت برگردم این برج را منهدم خواهم ساخت.»
\par 10 و زبح و صلمونع در قرقور با لشکر خود به قدر پانزده هزار نفر بودند. تمامی بقیه لشکربنی مشرق این بود، زیرا صد و بیست هزار مردجنگی افتاده بودند.
\par 11 و جدعون به راه چادرنشینان به طرف شرقی نوبح و یجبهاه برآمده، لشکر ایشان را شکست داد، زیرا که لشکر مطمئن بودند.
\par 12 و زبح و صلمونع فرارکردند و ایشان را تعاقب نموده، آن دو ملک مدیان یعنی زبح و صلمونع را گرفت و تمامی لشکرایشان را منهزم ساخت.
\par 13 و جدعون بن یوآش از بالای حارس ازجنگ برگشت.
\par 14 و جوانی از اهل سکوت راگرفته، از او تفتیش کرد و او برای وی نامهای سرداران سکوت و مشایخ آن را که هفتاد و هفت نفر بودند، نوشت.
\par 15 پس نزد اهل سکوت آمده، گفت: «اینک زبح و صلمونع را که درباره ایشان مرا طعنه زده، گفتید مگر دست زبح و صلمونع الان در دست تو است تا به مردان خسته تو نان بدهیم.»
\par 16 پس مشایخ شهر و شوک و خارهای صحرا را گرفته، اهل سکوت را به آنها تادیب نمود.
\par 17 و برج فنوعیل را منهدم ساخته، مردان شهر را کشت.
\par 18 و به زبح و صلمونع گفت: «چگونه مردمانی بودند که در تابور کشتید.» گفتند: «ایشان مثل توبودند، هر یکی شبیه شاهزادگان.»
\par 19 گفت: «ایشان برادرانم و پسران مادر من بودند، به خداوند حی قسم اگر ایشان را زنده نگاه می‌داشتید، شما را نمی کشتم.»
\par 20 و به نخست زاده خود، یتر، گفت: «برخیز و ایشان رابکش.» لیکن آن جوان شمشیر خود را از ترس نکشید چونکه هنوز جوان بود. 
\par 21 پس زبح وصلمونع گفتند: «تو برخیز و ما را بکش زیراشجاعت مرد مثل خود اوست.» پس جدعون برخاسته، زبح و صلمونع را بکشت و هلالهایی که بر گردن شتران ایشان بود، گرفت.
\par 22 پس مردان اسرائیل به جدعون گفتند: «بر ماسلطنت نما، هم پسر تو و پسر پسر تو نیز چونکه ما را از دست مدیان رهانیدی.»
\par 23 جدعون درجواب ایشان گفت: «من بر شما سلطنت نخواهم کرد، و پسر من بر شما سلطنت نخواهد کرد، خداوند بر شما سلطنت خواهد نمود.»
\par 24 وجدعون به ایشان گفت: «یک چیز از شما خواهش دارم که هر یکی از شما گوشواره های غنیمت خود را به من بدهد.» زیرا که گوشواره های طلاداشتند چونکه اسمعیلیان بودند.
\par 25 در جواب گفتند: «البته می‌دهیم». پس ردایی پهن کرده، هریکی گوشواره های غنیمت خود را در آن انداختند.
\par 26 و وزن گوشواره های طلایی که طلبیده بود هزار و هفتصد مثقال طلا بود، سوای آن هلالها و حلقه‌ها و جامه های ارغوانی که برملوک مدیان بود، و سوای گردنبندهایی که برگردن شتران ایشان بود.
\par 27 و جدعون از آنهاایفودی ساخت و آن را در شهر خود عفره برپاداشت، و تمامی اسرائیل به آنجا در عقب آن زناکردند، و آن برای جدعون و خاندان او دام شد.
\par 28 پس مدیان در حضور بنی‌اسرائیل مغلوب شدند و دیگر سر خود را بلند نکردند، و زمین درایام جدعون چهل سال آرامی یافت.
\par 29 و یربعل بن یوآش رفته، در خانه خود ساکن شد.
\par 30 و جدعون را هفتاد پسر بود که از صلبش بیرون آمده بودند، زیرا زنان بسیار داشت.
\par 31 وکنیز او که در شکیم بود او نیز برای وی پسری آورد، و او را ابیملک نام نهاد.
\par 32 و جدعون بن یوآش پیر و سالخورده شده، مرد، و در قبرپدرش یوآش در عفره ابیعزری دفن شد.
\par 33 و واقع شد بعد از وفات جدعون که بنی‌اسرائیل برگشته، در‌پیروی بعلها زنا کردند، وبعل بریت را خدای خود ساختند.
\par 34 وبنی‌اسرائیل یهوه، خدای خود را که ایشان را ازدست جمیع دشمنان ایشان از هر طرف رهایی داده بود، به یاد نیاوردند.و با خاندان یربعل جدعون موافق همه احسانی که با بنی‌اسرائیل نموده بود، نیکویی نکردند.
\par 35 و با خاندان یربعل جدعون موافق همه احسانی که با بنی‌اسرائیل نموده بود، نیکویی نکردند.
 
\chapter{9}

\par 1 و ابیملک بن یربعل نزد برادران مادر خود به شکیم رفته، ایشان و تمامی قبیله خاندان پدر مادرش را خطاب کرده، گفت:
\par 2 «الان درگوشهای جمیع اهل شکیم بگویید، برای شماکدام بهتر است؟ که هفتاد نفر یعنی همه پسران یربعل بر شما حکمرانی کنند؟ یا اینکه یک شخص بر شما حاکم باشد؟ و بیاد آورید که من استخوان و گوشت شما هستم.»
\par 3 و برادران مادرش درباره او در گوشهای جمیع اهل شکیم همه این سخنان را گفتند، و دل ایشان به پیروی ابیملک مایل شد، زیرا گفتند او برادر ماست.
\par 4 وهفتاد مثقال نقره از خانه بعل بریت به او دادند، وابیملک مردان مهمل و باطل را به آن اجیر کرد که او را پیروی نمودند.
\par 5 پس به خانه پدرش به عفره رفته، برادران خود پسران یربعل را که هفتاد نفربودند بر یک سنگ بکشت، لیکن یونام پسرکوچک یربعل زنده ماند، زیرا خود را پنهان کرده بود.
\par 6 و تمامی اهل شکیم و تمامی خاندان ملوجمع شده، رفتند، و ابیملک را نزد بلوط ستون که در شکیم است، پادشاه ساختند.
\par 7 و چون یوتام را از این خبر دادند، او رفته، به‌سر کوه جرزیم ایستاد و آواز خود را بلند کرده، ندا در‌داد و به ایشان گفت: «ای مردان شکیم مرابشنوید تا خدا شما را بشنود.
\par 8 وقتی درختان رفتند تا بر خود پادشاهی نصب کنند، و به درخت زیتون گفتند بر ما سلطنت نما.
\par 9 درخت زیتون به ایشان گفت: آیا روغن خود را که به‌سبب آن خداو انسان مرا محترم می‌دارند ترک کنم و رفته، بردرختان حکمرانی نمایم؟
\par 10 و درختان به انجیرگفتند که تو بیا و بر ما سلطنت نما.
\par 11 انجیر به ایشان گفت: آیا شیرینی و میوه نیکوی خود راترک بکنم و رفته، بر درختان حکمرانی نمایم؟
\par 12 و درختان به مو گفتند که بیا و بر ما سلطنت نما.
\par 13 مو به ایشان گفت: آیا شیره خود را که خدا وانسان را خوش می‌سازد ترک بکنم و رفته، بردرختان حکمرانی نمایم؟
\par 14 و جمیع درختان به خار گفتند که تو بیا و بر ما سلطنت نما.
\par 15 خار به درختان گفت: اگر به حقیقت شما مرا بر خود پادشاه نصب می‌کنید، پس بیایید و درسایه من پناه گیرید، و اگر نه آتش از خار بیرون بیاید و سروهای آزاد لبنان را بسوزاند.
\par 16 و الان اگر براستی و صداقت عمل نمودید در اینکه ابیملک را پادشاه ساختید، و اگر به یربعل وخاندانش نیکویی کردید و برحسب عمل دستهایش رفتار نمودید.
\par 17 زیرا که پدر من به جهت شما جنگ کرده، جان خود را به خطرانداخت و شما را از دست مدیان رهانید.
\par 18 وشما امروز بر خاندان پدرم برخاسته، پسرانش، یعنی هفتاد نفر را بر یک سنگ کشتید، و پسر کنیزاو ابیملک را چون برادر شما بود بر اهل شکیم پادشاه ساختید.
\par 19 پس اگر امروز به راستی وصداقت با یربعل و خاندانش عمل نمودید، ازابیملک شاد باشید و او از شما شاد باشد.
\par 20 واگرنه آتش از ابیملک بیرون بیاید، و اهل شکیم وخاندان ملو را بسوزاند، و آتش از اهل شکیم وخاندان ملو بیرون بیاید و ابیملک را بسوزاند.»
\par 21 پس یوتام فرار کرده، گریخت و به بئیر آمده، درآنجا از ترس برادرش، ابیملک، ساکن شد.
\par 22 و ابیملک بر اسرائیل سه سال حکمرانی کرد.
\par 23 و خدا روحی خبیث در میان ابیملک واهل شکیم فرستاد، و اهل شکیم با ابیملک خیانت ورزیدند.
\par 24 تا انتقام ظلمی که بر هفتاد پسریربعل شده بود، بشود، و خون آنها را از برادرایشان ابیملک که ایشان را کشته بود، و از اهل شکیم که دستهایش را برای کشتن برادران خودقوی ساخته بودند، گرفته شود.
\par 25 پس اهل شکیم بر قله های کوهها برای او کمین گذاشتند، و هرکس را که از طرف ایشان در راه می‌گذشت، تاراج می‌کردند. پس ابیملک را خبر دادند.
\par 26 و جعل بن عابد با برادرانش آمده، به شکیم رسیدند و اهل شکیم بر او اعتماد نمودند.
\par 27 و به مزرعه‌ها بیرون رفته، موها را چیدند و انگور رافشرده، بزم نمودند، و به خانه خدای خود داخل شده، اکل و شرب کردند و ابیملک را لعنت نمودند.
\par 28 و جعل بن عابد گفت: «ابیملک کیست و شکیم کیست که او را بندگی نماییم؟ آیا او پسریربعل و زبول، وکیل او نیست؟ مردان حامور پدرشکیم را بندگی نمایید. ما چرا باید او را بندگی کنیم؟
\par 29 کاش که این قوم زیر دست من می‌بودندتا ابیملک را رفع می‌کردم، و به ابیملک گفت: لشکر خود را زیاد کن و بیرون بیا.»
\par 30 و چون زبول، رئیس شهر، سخن جعل بن عابد را شنید خشم او افروخته شد.
\par 31 پس به حیله قاصدان نزد ابیملک فرستاده، گفت: «اینک جعل بن عابد با برادرانش به شکیم آمده‌اند وایشان شهر را به ضد تو تحریک می‌کنند.
\par 32 پس الان در شب برخیز، تو و قومی که همراه توست، و در صحرا کمین کن.
\par 33 و بامدادان در وقت طلوع آفتاب برخاسته، به شهر هجوم آور، و اینک چون او و کسانی که همراهش هستند بر تو بیرون آیند، آنچه در قوت توست، با او خواهی کرد.»
\par 34 پس ابیملک و همه کسانی که با وی بودند، در شب برخاسته، چهار دسته شده، در مقابل شکیم در کمین نشستند.
\par 35 و جعل بن عابد بیرون آمده، به دهنه دروازه شهر ایستاد، و ابیملک وکسانی که با وی بودند از کمینگاه برخاستند.
\par 36 وچون جعل آن گروه را دید به زبول گفت: «اینک گروهی از سر کوهها به زیر می‌آیند.» زبول وی راگفت: «سایه کوهها را مثل مردم می‌بینی.»
\par 37 باردیگر جعل متکلم شده، گفت: «اینک گروهی ازبلندی زمین به زیر می‌آیند و جمعی دیگر از راه بلوط معونیم می‌آیند.»
\par 38 زبول وی را گفت: «الان زبان تو کجاست که گفتی ابیملک کیست که او را بندگی نماییم؟ آیا این آن قوم نیست که حقیرشمردی؟ پس حال بیرون رفته، با ایشان جنگ کن.»
\par 39 و جعل پیش روی اهل شکیم بیرون شده، با ابیملک جنگ کرد.
\par 40 و ابیملک او را منهزم ساخت که از حضور وی فرار کرد و بسیاری تادهنه دروازه مجروح افتادند.
\par 41 و ابیملک درارومه ساکن شد، و زبول، جعل و برادرانش رابیرون کرد تا در شکیم نباشند.
\par 42 و در فردای آن روز واقع شد که مردم به صحرا بیرون رفتند، و ابیملک را خبر دادند.
\par 43 پس مردان خود را گرفته، ایشان را به سه فرقه تقسیم نمود، و در صحرا در کمین نشست، و نگاه کرد و اینک مردم از شهر بیرون می‌آیند، پس برایشان برخاسته، ایشان را شکست داد.
\par 44 وابیملک با فرقه‌ای که با وی بودند حمله برده، دردهنه دروازه شهر ایستادند، و آن دو فرقه برکسانی که در صحرا بودند هجوم آوردند، و ایشان را شکست دادند.
\par 45 و ابیملک در تمامی آن روزبا شهر جنگ کرده، شهر را گرفت و مردم را که درآن بودند، کشت، و شهر را منهدم ساخته، نمک در آن کاشت.
\par 46 و چون همه مردان برج شکیم این راشنیدند، به قلعه بیت ئیل بریت داخل شدند.
\par 47 وبه ابیملک خبر دادند که همه مردان برج شکیم جمع شده‌اند.
\par 48 آنگاه ابیملک با همه کسانی که باوی بودند به کوه صلمون برآمدند، و ابیملک تبری به‌دست گرفته، شاخه‌ای از درخت بریده، آن را گرفت و بر دوش خود نهاده، به کسانی که باوی بودند، گفت: «آنچه مرا دیدید که کردم تعجیل نموده، مثل من بکنید.»
\par 49 و تمامی قوم، هر کس شاخه خود را بریده، در عقب ابیملک افتادند و آنها را به اطراف قلعه نهاده، قلعه را بر سرایشان به آتش سوزانیدند، به طوری که همه مردمان برج شکیم که تخمین هزار مرد و زن بودند، بمردند.
\par 50 و ابیملک به تاباص رفت و بر تاباص اردوزده، آن را گرفت.
\par 51 و در میان شهر برج محکمی بود و همه مردان و زنان و تمامی اهل شهر درآنجا فرار کردند، و درها را بر خود بسته، به پشت بام برج برآمدند.
\par 52 و ابیملک نزد برج آمده، با آن جنگ کرد، و به دروازه برج نزدیک شد تا آن را به آتش بسوزاند.
\par 53 آنگاه زنی سنگ بالائین آسیایی گرفته، بر سر ابیملک انداخت و کاسه سرش را شکست.
\par 54 پس جوانی را که سلاحدارش بود به زودی صدا زده، وی را گفت: «شمشیر خود را کشیده، مرا بکش، مبادا درباره من بگویند زنی او را کشت.» پس غلامش شمشیررا به او فرو برد که مرد.
\par 55 و چون مردان اسرائیل دیدند که ابیملک مرده است، هر کس به مکان خود رفت.
\par 56 پس خدا شر ابیملک را که به پدرخود به کشتن هفتاد برادر خویش رسانیده بود، مکافات کرد.و خدا تمامی شر مردم شکیم رابر سر ایشان برگردانید، و لعنت یوتام بن یربعل برایشان رسید.
\par 57 و خدا تمامی شر مردم شکیم رابر سر ایشان برگردانید، و لعنت یوتام بن یربعل برایشان رسید.
 
\chapter{10}

\par 1 و بعد از ابیملک تولع بن فواه بن دودا، مردی از سبط یساکار، برخاست تااسرائیل را رهایی دهد، و او در شامیر در کوهستان افرایم ساکن بود.
\par 2 و او بر اسرائیل بیست و سه سال داوری نمود، پس وفات یافته، در شامیر مدفون شد.
\par 3 و بعد از او یائیر جلعادی برخاسته، براسرائیل بیست و دو سال داوری نمود.
\par 4 و او راسی پسر بود که بر سی کره الاغ سوار می‌شدند، وایشان را سی شهر بود که تا امروز به حووت یائیرنامیده است، و در زمین جلعاد می‌باشد.
\par 5 و یائیروفات یافته، در قامون دفن شد.
\par 6 و بنی‌اسرائیل باز در نظر خداوند شرارت ورزیده، بعلیم و عشتاروت و خدایان ارام وخدایان صیدون و خدایان موآب و خدایان بنی عمون و خدایان فلسطینیان را عبادت نمودند، و یهوه را ترک کرده، او را عبادت نکردند.
\par 7 وغضب خداوند بر اسرائیل افروخته شده، ایشان را به‌دست فلسطینیان و به‌دست بنی عمون فروخت.
\par 8 و ایشان در آن سال بر بنی‌اسرائیل ستم و ظلم نمودند، و بر جمیع بنی‌اسرائیل که به آن طرف اردن در زمین اموریان که در جلعادباشد، بودند، هجده سال ظلم کردند.
\par 9 وبنی عمون از اردن عبور کردند، تا با یهودا و بنیامین و خاندان افرایم نیز جنگ کنند، و اسرائیل درنهایت تنگی بودند.
\par 10 و بنی‌اسرائیل نزد خداوند فریاد برآورده، گفتند: «به تو گناه کرده‌ایم، چونکه خدای خود راتر ک کرده، بعلیم را عبادت نمودیم.» 
\par 11 خداوندبه بنی‌اسرائیل گفت: «آیا شما را از مصریان واموریان و بنی عمون و فلسطینیان رهایی ندادم؟
\par 12 و چون صیدونیان و عمالیقیان و معونیان برشما ظلم کردند، نزد من فریاد برآورید و شما را از دست ایشان رهایی دادم.
\par 13 لیکن شما مرا ترک کرده، خدایان غیر را عبادت نمودید، پس دیگرشما را رهایی نخواهم داد.
\par 14 بروید و نزدخدایانی که اختیار کرده‌اید، فریاد برآورید، وآنها شما را در وقت تنگی شما رهایی دهند.»
\par 15 بنی‌اسرائیل به خداوند گفتند: «گناه کرده‌ایم، پس برحسب آنچه درنظر تو پسند آید به ما عمل نما، فقط امروز ما را رهایی ده.»
\par 16 پس ایشان خدایان غیر را از میان خود دور کرده، یهوه راعبادت نمودند، و دل او به‌سبب تنگی اسرائیل محزون شد.
\par 17 پس بنی عمون جمع شده، در جلعاد اردوزدند، و بنی‌اسرائیل جمع شده، در مصفه اردوزدند.و قوم یعنی سروران جلعاد به یکدیگرگفتند: «کیست آن که جنگ را با بنی عمون شروع کند؟ پس وی سردار جمیع ساکنان جلعادخواهد بود.»
\par 18 و قوم یعنی سروران جلعاد به یکدیگرگفتند: «کیست آن که جنگ را با بنی عمون شروع کند؟ پس وی سردار جمیع ساکنان جلعادخواهد بود.»
 
\chapter{11}

\par 1 و یفتاح جلعادی مردی زورآور، شجاع، وپسر فاحشه‌ای بود، و جلعاد یفتاح را تولیدنمود.
\par 2 و زن جلعاد پسران برای وی زایید، وچون پسران زنش بزرگ شدند یفتاح را بیرون کرده، به وی گفتند: «تو در خانه پدر ما میراث نخواهی یافت، زیرا که تو پسر زن دیگر هستی.»
\par 3 پس یفتاح از حضور برادران خود فرار کرده، درزمین طوب ساکن شد، و مردان باطل نزد یفتاح جمع شده، همراه وی بیرون می‌رفتند.
\par 4 و واقع شد بعد از مرور ایام که بنی عمون بااسرائیل جنگ کردند.
\par 5 و چون بنی عمون بااسرائیل جنگ کردند، مشایخ جلعاد رفتند تایفتاح را از زمین طوب بیاروند.
\par 6 و به یفتاح گفتندبیا سردار ما باش تا با بنی عمون جنگ نماییم.»
\par 7 یفتاح به مشایخ جلعاد گفت: «آیا شما به من بغض ننمودید؟ و مرا از خانه پدرم بیرون نکردید؟ و الان چونکه در تنگی هستید چرا نزدمن آمده‌اید؟»
\par 8 مشایخ جلعاد به یفتاح گفتند: «از این سبب الان نزد تو برگشته‌ایم تا همراه ما آمده، بابنی عمون جنگ نمایی، و بر ما و بر تمامی ساکنان جلعاد سردار باشی.»
\par 9 یفتاح به مشایخ جلعادگفت: «اگر مرا برای جنگ کردن با بنی عمون بازآورید و خداوند ایشان را به‌دست من بسپارد، آیامن سردار شما خواهم بود.»
\par 10 و مشایخ جلعاد به یفتاح گفتند: «خداونددر میان ما شاهد باشد که البته برحسب سخن توعمل خواهیم نمود.
\par 11 پس یفتاح با مشایخ جلعاد رفت و قوم او را بر خود رئیس و سردارساختند، و یفتاح تمام سخنان خود را به حضورخداوند در مصفه گفت.
\par 12 و یفتاح قاصدان نزد ملک بنی عمون فرستاده، گفت: «تو را با من چه‌کار است که نزد من آمده‌ای تا با زمین من جنگ نمایی؟»
\par 13 ملک بنی عمون به قاصدان یفتاح گفت: «از این سبب که اسرائیل چون از مصر بیرون آمدند زمین مرا ازارنون تا یبوق و اردن گرفتند، پس الان آن زمینهارا به سلامتی به من رد نما.»
\par 14 و یفتاح بار دیگر قاصدان نزد ملک بنی عمون فرستاد.
\par 15 و او را گفت که «یفتاح چنین می‌گوید: اسرائیل زمین موآب و زمین بنی عمون را نگرفت.
\par 16 زیرا که چون اسرائیل از مصر بیرون آمدند، در بیابان تا بحر قلزم سفر کرده، به قادش رسیدند.»
\par 17 و اسرائیل رسولان نزد ملک ادوم فرستاده، گفتند: «تمنا اینکه از زمین تو بگذریم. اما ملک ادوم قبول نکرد، و نزد ملک موآب نیز فرستادند و او راضی نشد، پس اسرائیل در قادش ماندند.
\par 18 پس در بیابان سیر کرده، زمین ادوم و زمین موآب را دور زدند و به‌جانب شرقی زمین موآب آمده، به آن طرف ارنون اردو زدند، وبه حدود موآب داخل نشدند، زیرا که ارنون حدموآب بود.
\par 19 و اسرائیل رسولان نزد سیحون، ملک اموریان، ملک حشبون، فرستادند، واسرائیل به وی گفتند: تمنا اینکه از زمین تو به مکان خود عبور نماییم.
\par 20 اما سیحون بر اسرائیل اعتماد ننمود تا از حدود او بگذرند، بلکه سیحون تمامی قوم خود را جمع کرده، در یاهص اردوزدند و با اسرائیل جنگ نمودند.
\par 21 و یهوه خدای اسرائیل سیحون و تمامی قومش را به‌دست اسرائیل تسلیم نمود که ایشان را شکست دادند، پس اسرائیل تمامی زمین اموریانی که ساکن آن ولایت بودند در تصرف آوردند.
\par 22 و تمامی حدود اموریان را از ارنون تا بیوق و از بیابان تااردن به تصرف آوردند.
\par 23 پس حال یهوه، خدای اسرائیل، اموریان را از حضور قوم خود اسرائیل اخراج نموده است، و آیا تو آنها را به تصرف خواهی آورد؟
\par 24 آیا آنچه خدای تو، کموش به تصرف تو بیاورد، مالک آن نخواهی شد؟ وهمچنین هرکه را یهوه، خدای ما از حضور مااخراج نماید آنها را مالک خواهیم بود.
\par 25 و حال آیا تو از بالاق بن صفور، ملک موآب بهتر هستی وآیا او با اسرائیل هرگز مقاتله کرد یا با ایشان جنگ نمود؟
\par 26 هنگامی که اسرائیل در حشبون ودهاتش و عروعیر و دهاتش و در همه شهرهایی که بر کناره ارنون است، سیصد سال ساکن بودند پس در آن مدت چرا آنها را باز نگرفتید؟
\par 27 من به تو گناه نکردم بلکه تو به من بدی کردی که با من جنگ می‌نمایی. پس یهوه که داور مطلق است امروز در میان بنی‌اسرائیل و بنی عمون داوری نماید.»
\par 28 اما ملک بنی عمون سخن یفتاح را که به او فرستاده بود، گوش نگرفت.
\par 29 و روح خداوند بر یفتاح آمد و او از جلعادو منسی گذشت و از مصفه جلعاد عبور کرد و ازمصفه جلعاد به سوی بنی عمون گذشت.
\par 30 ویفتاح برای خداوند نذر کرده، گفت: «اگربنی عمون را به‌دست من تسلیم نمایی،
\par 31 آنگاه وقتی که به سلامتی از بنی عمون برگردم، هر‌چه به استقبال من از در خانه‌ام بیرون آید از آن خداوندخواهد بود، و آن را برای قربانی سوختنی خواهم گذرانید.»
\par 32 پس یفتاح به سوی بنی عمون گذشت تا با ایشان جنگ نماید، و خداوند ایشان را به‌دست او تسلیم کرد.
\par 33 و ایشان را از عروعیرتا منیت که بیست شهر بود و تا آبیل کرامیم به صدمه بسیار عظیم شکست داد، و بنی عمون ازحضور بنی‌اسرائیل مغلوب شدند.
\par 34 و یفتاح به مصفه به خانه خود آمد و اینک دخترش به استقبال وی با دف و رقص بیرون آمدو او دختر یگانه او بود و غیر از او پسری یادختری نداشت.
\par 35 و چون او را دید، لباس خودرا دریده، گفت: «آه‌ای دختر من، مرا بسیار ذلیل کردی و تو یکی از آزارندگان من شدی، زیرا دهان خود را به خداوند باز نموده‌ام و نمی توانم برگردم.»
\par 36 و او وی را گفت: «ای پدر من دهان خود را نزد خداوند باز کردی پس با من چنانکه ازدهانت بیرون آمد عمل نما، چونکه خداوند انتقام تو را از دشمنانت بنی عمون کشیده است.»
\par 37 و به پدر خود گفت: «این کار به من معمول شود. دو ماه مرا مهلت بده تا رفته بر کوهها گردش نمایم وبرای بکریت خود با رفقایم ماتم گیرم.»
\par 38 اوگفت: «برو». و او را دو ماه روانه نمود پس او بارفقای خود رفته، برای بکریتش بر کوهها ماتم گرفت.
\par 39 و واقع شد که بعد از انقضای دو ماه نزدپدر خود برگشت و او موافق نذری که کرده بود به او عمل نمود، و آن دختر مردی را نشناخت، پس در اسرائیل عادت شد،که دختران اسرائیل سال به سال می‌رفتند تا برای دختر یفتاح جلعادی چهار روز در هر سال ماتم گیرند.
\par 40 که دختران اسرائیل سال به سال می‌رفتند تا برای دختر یفتاح جلعادی چهار روز در هر سال ماتم گیرند.
 
\chapter{12}

\par 1 و مردان افرایم جمع شده، به طرف شمال گذشتند، و به یفتاح گفتند: «چرابرای جنگ کردنت با بنی عمون رفتی و ما رانطلبیدی تا همراه تو بیاییم؟ پس خانه تو را بر سرتو خواهیم سوزانید.»
\par 2 و یفتاح به ایشان گفت: «مرا و قوم مرا با بنی عمون جنگ سخت می‌بود، وچون شما را خواندم مرا از دست ایشان رهایی ندادید.
\par 3 پس چون دیدم که شما مرا رهایی نمی دهید جان خود را به‌دست خود گرفته، به سوی بنی عمون رفتم و خداوند ایشان را به‌دست من تسلیم نمود، پس چرا امروز نزد من برآمدید تابا من جنگ نمایید؟»
\par 4 پس یفتاح تمامی مردان جلعاد را جمع کرده، با افرایم جنگ نمود و مردان جلعاد افرایم را شکست دادند، چونکه گفته بودندای اهل جلعاد شما فراریان افرایم در میان افرایم و در میان منسی هستید.
\par 5 و اهل جلعاد معبرهای اردن را پیش روی افرایم گرفتند و واقع شد که چون یکی از گریزندگان افرایم می‌گفت: «بگذارید عبور نمایم.» اهل جلعاد می‌گفتند: «آیاتو افرایمی هستی؟» و اگر می‌گفت نی،
\par 6 پس او رامی گفتند: بگو شبولت، و او می‌گفت سبولت، چونکه نمی توانست به درستی تلفظ نماید، پس او را گرفته، نزد معبرهای اردن می‌کشتند، و در آن وقت چهل و دو هزار نفر از افرایم کشته شدند.
\par 7 و یفتاح بر اسرائیل شش سال داوری نمود. پس یفتاح جلعادی وفات یافته، در یکی ازشهرهای جلعاد دفن شد.
\par 8 و بعد از او ابصان بیت لحمی بر اسرائیل داوری نمود.
\par 9 و او را سی پسر بود و سی دختر که بیرون فرستاده بود و از بیرون سی دختر برای پسران خود آورد و هفت سال بر اسرائیل داوری نمود.
\par 10 و ابصان مرد و در بیت لحم دفن شد.
\par 11 وبعد از او ایلون زبولونی بر اسرائیل داوری نمود وداوری او بر اسرائیل ده سال بود.
\par 12 و ایلون زبولونی مرد و در ایلون در زمین زبولون دفن شد.
\par 13 و بعد از او عبدون بن هلیل فرعتونی براسرائیل داوری نمود.
\par 14 و او را چهل پسر و سی نواده بود، که بر هفتاد کره الاغ سوار می‌شدند وهشت سال بر اسرائیل داوری نمود.و عبدون بن هلیل فرعتونی مرد و در فرعتون در زمین افرایم در کوهستان عمالیقیان دفن شد.
\par 15 و عبدون بن هلیل فرعتونی مرد و در فرعتون در زمین افرایم در کوهستان عمالیقیان دفن شد.
 
\chapter{13}

\par 1 و بنی‌اسرائیل بار دیگر در نظر خداوندشرارت ورزیدند، و خداوند ایشان را به‌دست فلسطینیان چهل سال تسلیم کرد.
\par 2 و شخصی از صرعه از قبیله دان، مانوح نام بود، و زنش نازاد بوده، نمی زایید.
\par 3 و فرشته خداوند به آن زن ظاهر شده، او را گفت: «اینک توحال نازاد هستی و نزاییده‌ای لیکن حامله شده، پسری خواهی زایید.
\par 4 و الان باحذر باش و هیچ شراب و مسکری منوش و هیچ‌چیز نجس مخور.
\par 5 زیرا یقین حامله شده، پسری خواهی زایید، واستره بر سرش نخواهد آمد، زیرا آن ولد از رحم مادر خود برای خدا نذیره خواهد بود، و او به رهانیدن اسرائیل از دست فلسطینیان شروع خواهد کرد.»
\par 6 پس آن زن آمده، شوهر خود را خطاب کرده، گفت: «مرد خدایی نزد من آمد، و منظر اومثل منظر فرشته خدا بسیار مهیب بود. و نپرسیدم که از کجاست و از اسم خود مرا خبر نداد.
\par 7 و به من گفت اینک حامله شده، پسری خواهی زایید، و الان هیچ شراب و مسکری منوش، و هیچ‌چیزنجس مخور زیرا که آن ولد از رحم مادر تا روزوفاتش برای خدا نذیره خواهد بود.»
\par 8 و مانوح از خداوند استدعا نموده، گفت: «آه‌ای خداوند تمنا اینکه آن مرد خدا که فرستادی بار دیگر نزد ما بیاید و ما را تعلیم دهد که با ولدی که مولود خواهد شد، چگونه رفتار نماییم.»
\par 9 و خدا آواز مانوح را شنید و فرشته خدا باردیگر نزد آن زن آمد و او در صحرا نشسته بود، اماشوهرش مانوح نزد وی نبود.
\par 10 و آن زن به زودی دویده، شوهر خود را خبر داده، به وی گفت: «اینک آن مرد که در آن روز نزد من آمد، باردیگر ظاهر شده است.»
\par 11 و مانوح برخاسته، در عقب زن خود روانه شد، و نزد آن شخص آمده، وی را گفت: «آیا توآن مرد هستی که با این زن سخن گفتی؟» او گفت: «من هستم.»
\par 12 مانوح گفت: «کلام تو واقع بشوداما حکم آن ولد و معامله با وی چه خواهد بود؟»
\par 13 و فرشته خداوند به مانوح گفت: «از هر‌آنچه به زن گفتم اجتناب نماید.
\par 14 از هر حاصل مو زنهارنخورد و هیچ شراب و مسکری ننوشد، و هیچ‌چیز نجس نخورد و هر‌آنچه به او امر فرمودم، نگاهدارد.»
\par 15 و مانوح به فرشته خداوند گفت: «تو راتعویق بیندازیم و برایت گوساله‌ای تهیه بینیم.» 
\par 16 فرشته خداوند به مانوح گفت: «اگر‌چه مراتعویق اندازی، از نان تو نخواهم خورد، و اگرقربانی سوختنی بگذرانی آن را برای یهوه بگذران.» زیرا مانوح نمی دانست که فرشته خداوند است.
\par 17 و مانوح به فرشته خداوند گفت: «نام تو چیست تا چون کلام تو واقع شود، تو رااکرام نماییم.»
\par 18 فرشته خداوند وی را گفت: «چرا درباره اسم من سوال می‌کنی چونکه آن عجیب است.»
\par 19 پس مانوح گوساله و هدیه آردی را گرفته، بر آن سنگ برای خداوند گذرانید، و فرشته کاری عجیب کرد و مانوح و زنش می‌دیدند.
\par 20 زیراواقع شد که چون شعله آتش از مذبح به سوی آسمان بالا می‌رفت، فرشته خداوند در شعله مذبح صعود نمود، و مانوح و زنش چون دیدند، رو به زمین افتادند.
\par 21 و فرشته خداوند بر مانوح وزنش دیگر ظاهر نشد، پس مانوح دانست که فرشته خداوند بود.
\par 22 و مانوح به زنش گفت: «البته خواهیم مرد، زیرا خدا را دیدیم.»
\par 23 امازنش گفت: «اگر خداوند می‌خواست ما را بکشدقربانی سوختنی و هدیه آردی را از دست ما قبول نمی کرد، و همه این چیزها را به ما نشان نمی داد، ودر این وقت مثل این امور را به سمع ما نمی رسانید.»
\par 24 و آن زن پسری زاییده، او را شمشون نام نهاد، و پسر نمو کرد و خداوند او را برکت داد.و روح خداوند در لشکرگاه دان در میان صرعه و اشتاول به برانگیختن او شروع نمود.
\par 25 و روح خداوند در لشکرگاه دان در میان صرعه و اشتاول به برانگیختن او شروع نمود.
 
\chapter{14}

\par 1 و شمشون به تمنه فرود آمده، زنی ازدختران فلسطینیان در تمنه دید.
\par 2 وآمده، به پدر و مادر خود بیان کرده، گفت: «زنی ازدختران فلسطینیان در تمنه دیدم، پس الان او رابرای من به زنی بگیرید.»
\par 3 پدر و مادرش وی راگفتند: «آیا از دختران برادرانت و در تمامی قوم من دختری نیست که تو باید بروی و از فلسطینیان نامختون زن بگیری؟» شمشون به پدر خود گفت: «او را برای من بگیر زیرا در نظر من پسند آمد.»
\par 4 اما پدر و مادرش نمی دانستند که این از جانب خداوند است، زیرا که بر فلسطینیان علتی می‌خواست، چونکه در آن وقت فلسطینیان براسرائیل تسلط می‌داشتند.
\par 5 پس شمشون با پدر و مادر خود به تمنه فرودآمد، و چون به تاکستانهای تمنه رسیدند، اینک شیری جوان بر او بغرید.
\par 6 و روح خداوند بر اومستقر شده، آن را درید به طوری که بزغاله‌ای دریده شود، و چیزی در دستش نبود و پدر و مادرخود را از آنچه کرده بود، اطلاع نداد.
\par 7 و رفته، باآن زن سخن گفت وبه نظر شمشون پسند آمد.
\par 8 وچون بعد از چندی برای گرفتنش برمی گشت، ازراه به کنار رفت تا لاشه شیر را ببیند و اینک انبوه زنبور، و عسل در لاشه شیر بود.
\par 9 و آن را به دست خود گرفته، روان شد و در رفتن می‌خورد تابه پدر و مادر خود رسیده، به ایشان داد وخوردند. اما به ایشان نگفت که عسل را از لاشه شیر گرفته بود.
\par 10 و پدرش نزد آن زن آمد و شمشون در آنجامهمانی کرد، زیرا که جوانان چنین عادت داشتند.
\par 11 و واقع شد که چون او را دیدند، سی رفیق انتخاب کردند تا همراه او باشند.
\par 12 و شمشون به ایشان گفت: «معمایی برای شما می‌گویم، اگر آن را برای من در هفت روز مهمانی حل کنید و آن رادریافت نمایید، به شما سی جامه کتان و سی دست رخت می‌دهم.
\par 13 و اگر آن را برای من نتوانید حل کنید آنگاه شما سی جامه کتان و سی دست رخت به من بدهید.» ایشان به وی گفتند: «معمای خود را بگو تا آن را بشنویم.»
\par 14 به ایشان گفت: «از خورنده خوراک بیرون آمد، و اززورآور شیرینی بیرون آمد.» و ایشان تا سه روزمعما را نتوانستند حل کنند.
\par 15 و واقع شد که در روز هفتم به زن شمشون گفتند: «شوهر خود را ترغیب نما تا معمای خودرا برای ما بیان کند مبادا تو را و خانه پدر تو را به آتش بسوزانیم، آیا ما را دعوت کرده‌اید تا ما راتاراج نمایید یا نه.»
\par 16 پس زن شمشون پیش اوگریسته، گفت: «به درستی که مرا بغض می‌نمایی و دوست نمی داری زیرا معمایی به پسران قوم من گفته‌ای و آن را برای من بیان نکردی.» او وی راگفت: «اینک برای پدر و مادر خود بیان نکردم، آیابرای تو بیان کنم.»
\par 17 و در هفت روزی که ضیافت ایشان می‌بود پیش او می‌گریست، و واقع شد که در روز هفتم چونکه او را بسیار الحاح می‌نمود، برایش بیان کرد و او معما را به پسران قوم خودگفت.
\par 18 و در روز هفتم مردان شهر پیش از غروب آفتاب به وی گفتند که «چیست شیرین تراز عسل و چیست زورآورتر از شیر.» او به ایشان گفت: «اگر با گاو من خیش نمی کردید، معمای مرادریافت نمی نمودید.»
\par 19 و روح خداوند بر وی مستقر شده، به اشقلون رفت و از اهل آنجا سی نفر را کشت، و اسباب آنها را گرفته، دسته های رخت را به آنانی که معما را بیان کرده بودند، داد وخشمش افروخته شده، به خانه پدر خودبرگشت.و زن شمشون به رفیقش که او رادوست خود می‌شمرد، داده شد.
\par 20 و زن شمشون به رفیقش که او رادوست خود می‌شمرد، داده شد.
 
\chapter{15}

\par 1 و بعد از چندی، واقع شد که شمشون در روزهای درو گندم برای دیدن زن خود با بزغاله‌ای آمد و گفت نزد زن خود به حجره خواهم درآمد. لیکن پدرش نگذاشت که داخل شود.
\par 2 و پدرزنش گفت: «گمان می‌کردم که او رابغض می‌نمودی، پس او را به رفیق تو دادم، آیاخواهر کوچکش از او بهتر نیست؛ او را به عوض وی برای خود بگیر.»
\par 3 شمشون به ایشان گفت: «این دفعه از فلسطینیان بی‌گناه خواهم بود اگرایشان را اذیتی برسانم.»
\par 4 و شمشون روانه شده، سیصد شغال گرفت، و مشعلها برداشته، دم بر دم گذاشت، و در میان هر دو‌دم مشعلی گذارد.
\par 5 ومشعلها را آتش زده، آنها را در کشتزارهای فلسطینیان فرستاد، و بافه‌ها و زرعها و باغهای زیتون را سوزانید.
\par 6 و فلسطینیان گفتند: «کیست که این را کرده است؟» گفتند: «شمشون دامادتمنی، زیرا که زنش را گرفته، او را به رفیقش داده است.» پس فلسطینیان آمده، زن و پدرش را به آتش سوزانیدند.
\par 7 و شمشون به ایشان گفت: «اگربه اینطور عمل کنید، البته از شما انتقام خواهم کشید و بعد از آن آرامی خواهم یافت.»
\par 8 و ایشان را از ساق تا ران به صدمه‌ای عظیم کشت. پس رفته، در مغاره صخره عیطام ساکن شد.
\par 9 و فلسطینیان برآمده، در یهودا اردو زدند ودر لحی متفرق شدند.
\par 10 و مردان یهودا گفتند: «چرا بر ما برآمدید.» گفتند: «آمده‌ایم تا شمشون را ببندیم و برحسب آنچه به ما کرده است به اوعمل نماییم.»
\par 11 پس سه هزار نفر از یهودا به مغاره صخره عیطام رفته، به شمشون گفتند: «آیاندانسته‌ای که فلسطینیان بر ما تسلط دارند، پس این چه‌کار است که به ما کرده‌ای؟» در جواب ایشان گفت: «به نحوی که ایشان به من کردند، من به ایشان عمل نمودم.»
\par 12 ایشان وی را گفتند: «ماآمده‌ایم تا تو را ببندیم و به‌دست فلسطینیان بسپاریم.» شمشون در جواب ایشان گفت: «برای من قسم بخورید که خود بر من هجوم نیاورید.»
\par 13 ایشان در جواب وی گفتند: «حاشا! بلکه تو رابسته، به‌دست ایشان خواهیم سپرد، و یقین تو رانخواهیم کشت.» پس او را به دو طناب نو بسته، ازصخره برآوردند.
\par 14 و چون او به لحی رسید، فلسطینیان ازدیدن او نعره زدند و روح خداوند بر وی مستقرشده، طنابهایی که بر بازوهایش بود مثل کتانی که به آتش سوخته شود گردید، و بندها از دستهایش فروریخت.
\par 15 و چانه تازه الاغی یافته، دست خود را دراز کرد و آن را گرفته، هزار مرد با آن کشت.
\par 16 و شمشون گفت: «با چانه الاغ توده برتوده با چانه الاغ هزار مرد کشتم.»
\par 17 و چون ازگفتن فارغ شد، چانه را از دست خود انداخت وآن مکان را رمت لحی نامید.
\par 18 پس بسیار تشنه شده، نزد خداوند دعاکرده، گفت که «به‌دست بنده ات این نجات عظیم را دادی و آیا الان از تشنگی بمیرم و به‌دست نامختونان بیفتم؟»
\par 19 پس خدا کفه‌ای را که درلحی بود شکافت که آب از آن جاری شد و چون بنوشید جانش برگشته، تازه روح شد. از این سبب اسمش عین حقوری خوانده شد که تا امروز درلحی است.و او در روزهای فلسطینیان بیست سال بر اسرائیل داوری نمود.
\par 20 و او در روزهای فلسطینیان بیست سال بر اسرائیل داوری نمود.
 
\chapter{16}

\par 1 و شمشون به غزه رفت و در آنجافاحشه‌ای دیده، نزد او داخل شد.
\par 2 و به اهل غزه گفته شد که شمشون به اینجا آمده است. پس او را احاطه نموده، تمام شب برایش نزددروازه شهر کمین گذاردند، و تمام شب خاموش مانده، گفتند، چون صبح روشن شود او رامی کشیم.
\par 3 و شمشون تا نصف شب خوابید. ونصف شب برخاسته، لنکهای دروازه شهر و دوباهو را گرفته، آنها را با پشت بند کند و بر دوش خود گذاشته، بر قله کوهی که در مقابل حبرون است، برد.
\par 4 و بعد از آن واقع شد که زنی را در وادی سورق که اسمش دلیله بود، دوست می‌داشت.
\par 5 وسروران فلسطینیان نزد او برآمده، وی را گفتند: «او را فریفته، دریافت کن که قوت عظیمش در چه چیز است، و چگونه بر او غالب آییم تا او را بسته، ذلیل نماییم، و هریکی از ما هزار و صد مثقال نقره به تو خواهیم داد.»
\par 6 پس دلیله به شمشون گفت: «تمنا اینکه به من بگویی که قوت عظیم تو در چه چیز است و چگونه می‌توان تو را بست و ذلیل نمود.»
\par 7 شمشون وی را گفت: «اگر مرا به هفت ریسمان تر و تازه که خشک نباشد ببندند، من ضعیف و مثل سایر مردم خواهم شد.»
\par 8 و سروران فلسطینیان هفت ریسمان تر و تازه که خشک نشده بود، نزد او آوردند و او وی را به آنهابست.
\par 9 و کسان نزد وی در حجره در کمین می‌بودند و او وی را گفت: «ای شمشون فلسطینیان بر تو آمدند.» آنگاه ریسمانها رابگسیخت چنانکه ریسمان کتان که به آتش برخورد گسیخته شود، لهذا قوتش دریافت نشد.
\par 10 و دلیله به شمشون گفت: «اینک استهزاکرده، به من دروغ گفتی، پس الان مرا خبر بده که به چه چیز تو را توان بست.»
\par 11 او وی را گفت: «اگر مرا با طنابهای تازه که با آنها هیچ کار کرده نشده است، ببندند، ضعیف و مثل سایر مردان خواهم شد.»
\par 12 و دلیله طنابهای تازه گرفته، او رابا آنها بست و به وی گفت: «ای شمشون فلسطینیان بر تو آمدند.» و کسان در حجره درکمین می‌بودند. آنگاه آنها را از بازوهای خود مثل نخ بگسیخت.
\par 13 و دلیله به شمشون گفت: «تابحال مرااستهزا نموده، دروغ گفتی. مرا بگو که به چه چیزبسته می‌شوی.» او وی را گفت: «اگر هفت گیسوی سر مرا با تار ببافی.»
\par 14 پس آنها را به میخ قایم بست و وی را گفت: «ای شمشون فلسطینیان بر تو آمدند.» آنگاه از خواب بیدار شده، هم میخ نورد نساج و هم تار را برکند.
\par 15 و او وی را گفت: «چگونه می‌گویی که مرادوست می‌داری و حال آنکه دل تو با من نیست. این سه مرتبه مرا استهزا نموده، مرا خبر ندادی که قوت عظیم تو در چه چیز است.»
\par 16 و چون اووی را هر روز به سخنان خود عاجز می‌ساخت واو را الحاح می‌نمود و جانش تا به موت تنگ می‌شد،
\par 17 هر‌چه در دل خود داشت برای او بیان کرده، گفت که «استره بر سر من نیامده است، زیرا که از رحم مادرم برای خداوند نذیره شده‌ام، واگر تراشیده شوم، قوتم از من خواهد رفت وضعیف و مثل سایر مردمان خواهم شد.»
\par 18 پس چون دلیله دید که هرآنچه در دلش بود، برای او بیان کرده است، فرستاد و سروران فلسطینیان را طلبیده، گفت: «این دفعه بیایید زیراهرچه در دل داشت مرا گفته است.» آنگاه سروران فلسطینیان نزد او آمدند و نقد را به‌دست خود آوردند.
\par 19 و او را بر زانوهای خودخوابانیده، کسی را طلبید و هفت گیسوی سرش را تراشید. پس به ذلیل نمودن او شروع کرد وقوتش از او برفت.
\par 20 و گفت: «ای شمشون فلسطینیان بر تو آمدند.» آنگاه از خواب بیدارشده، گفت: «مثل پیشتر بیرون رفته، خود رامی افشانم. اما او ندانست که خداوند از او دورشده است.
\par 21 پس فلسطینیان او را گرفته، چشمانش را کندند و او را به غزه آورده، به زنجیرهای برنجین بستند و در زندان دستاس می‌کرد.
\par 22 و موی سرش بعد از تراشیدن باز به بلند شدن شروع نمود. 
\par 23 و سروران فلسطینیان جمع شدند تا قربانی عظیمی برای خدای خود، داجون بگذرانند، وبزم نمایند زیرا گفتند خدای ما دشمن ما شمشون را به‌دست ما تسلیم نموده است.
\par 24 و چون خلق او را دیدند خدای خود را تمجید نمودند، زیراگفتند خدای ما دشمن ما را که زمین ما را خراب کرد و بسیاری از ما را کشت، به‌دست ما تسلیم نموده است.
\par 25 و چون دل ایشان شاد شد، گفتند: «شمشون را بخوانید تا برای ما بازی کند.» پس شمشون را از زندان آورده، برای ایشان بازی می کرد، و او را در میان ستونها برپا داشتند.
\par 26 وشمشون به پسری که دست او را می‌گرفت، گفت: «مرا واگذار تا ستونهایی که خانه بر آنها قایم است، لمس نموده، بر آنها تکیه نمایم.»
\par 27 و خانه از مردان و زنان پر بود و جمیع سروران فلسطینیان در آن بودند و قریب به سه هزار مرد و زن برپشت بام، بازی شمشون را تماشا می‌کردند.
\par 28 و شمشون از خداوند استدعا نموده، گفت: «ای خداوند یهوه مرا بیاد آور و‌ای خدا این مرتبه فقط مرا قوت بده تا یک انتقام برای دو چشم خوداز فلسطینیان بکشم.»
\par 29 و شمشون دو ستون میان را که خانه بر آنها قایم بود، یکی را به‌دست راست و دیگری را به‌دست چپ خود گرفته، بر آنها تکیه نمود.
\par 30 و شمشون گفت: «همراه فلسطینیان بمیرم و با زور خم شده، خانه بر سروران و برتمامی خلقی که در آن بودند، افتاد. پس مردگانی که در موت خود کشت از مردگانی که درزندگی‌اش کشته بود، زیادتر بودند.آنگاه برادرانش و تمامی خاندان پدرش آمده، او رابرداشتند و او را آورده، در قبر پدرش مانوح درمیان صرعه و اشتاول دفن کردند. و او بیست سال بر اسرائیل داوری کرد.
\par 31 آنگاه برادرانش و تمامی خاندان پدرش آمده، او رابرداشتند و او را آورده، در قبر پدرش مانوح درمیان صرعه و اشتاول دفن کردند. و او بیست سال بر اسرائیل داوری کرد.
 
\chapter{17}

\par 1 و از کوهستان افرایم شخصی بود که میخا نام داشت.
\par 2 و به مادر خود گفت: «آن هزار و یکصد مثقال نقره‌ای که از تو گرفته شد، و درباره آن لعنت کردی و در گوشهای من نیز سخن گفتی، اینک آن نقره نزد من است، من آن را گرفتم.» مادرش گفت: «خداوند پسر مرا برکت دهد.»
\par 3 پس آن هزار و یکصد مثقال نقره را به مادرش رد نمود و مادرش گفت: «این نقره را برای خداوند از دست خود به جهت پسرم بالکل وقف می‌کنم تا تمثال تراشیده و تمثال ریخته شده‌ای ساخته شود، پس الان آن را به تو باز می‌دهم.»
\par 4 وچون نقره را به مادر خود رد نمود، مادرش دویست مثقال نقره گرفته، آن را به زرگری داد که او تمثال تراشیده، و تمثال ریخته شده‌ای ساخت و آنها در خانه میخا بود.
\par 5 و میخا خانه خدایان داشت، و ایفود و ترافیم ساخت، و یکی از پسران خود را تخصیص نمود تا کاهن او بشود.
\par 6 و در آن ایام در اسرائیل پادشاهی نبود و هر کس آنچه درنظرش پسند می‌آمد، می‌کرد.
\par 7 و جوانی از بیت لحم یهودا از قبیله یهودا و ازلاویان بود که در آنجا ماوا گزید.
\par 8 و آن شخص ازشهر خود، یعنی از بیت لحم یهودا روانه شد، تاهر جایی که بیابد ماوا گزیند، و چون سیر می‌کردبه کوهستان افرایم به خانه میخا رسید.
\par 9 و میخا اورا گفت: «از کجا آمده‌ای؟» او در جواب وی گفت: «من لاوی هستم از بیت لحم یهودا، ومی روم تا هر جایی که بیابم ماوا گزینم.»
\par 10 میخااو را گفت: «نزد من ساکن شو و برایم پدر و کاهن باش، و من تو را هر سال ده مثقال نقره و یک دست لباس و معاش می‌دهم.» پس آن لاوی داخل شد.
\par 11 و آن لاوی راضی شد که با او ساکن شود، و آن جوان نزد او مثل یکی از پسرانش بود.
\par 12 و میخاآن لاوی را تخصیص نمود و آن جوان کاهن اوشد، و در خانه میخا می‌بود.و میخا گفت: «الان دانستم که خداوند به من احسان خواهد نمود زیرالاوی‌ای را کاهن خود دارم.»
\par 13 و میخا گفت: «الان دانستم که خداوند به من احسان خواهد نمود زیرالاوی‌ای را کاهن خود دارم.»
 
\chapter{18}

\par 1 و در آن ایام در اسرائیل پادشاهی نبود، و در آن روزها سبط دان، ملکی برای سکونت خود طلب می‌کردند، زیرا تا در آن روزملک ایشان در میان اسباط اسرائیل به ایشان نرسیده بود.
\par 2 و پسران دان از قبیله خویش پنج نفر از جماعت خود که مردان جنگی بودند ازصرعه و اشتاول فرستادند تا زمین را جاسوسی وتفحص نمایند، و به ایشان گفتند: «بروید و زمین را تفحص کنید.» پس ایشان به کوهستان افرایم به خانه میخا آمده، در آنجا منزل گرفتند.
\par 3 و چون ایشان نزد خانه میخا رسیدند، آواز جوان لاوی راشناختند و به آنجا برگشته، او را گفتند: «کیست که تو را به اینجا آورده است و در این مکان چه می‌کنی و در اینجا چه داری؟»
\par 4 او به ایشان گفت: «میخا با من چنین و چنان رفتار نموده است، و مرااجیر گرفته، کاهن او شده‌ام.»
\par 5 وی را گفتند: «ازخدا سوال کن تا بدانیم آیا راهی که در آن می‌رویم خیر خواهد بود.»
\par 6 کاهن به ایشان گفت: «به سلامتی بروید. راهی که شما می‌روید منظورخداوند است.»
\par 7 پس آن پنج مرد روانه شده، به لایش رسیدند. و خلقی را که در آن بودند، دیدند که درامنیت و به رسم صیدونیان در اطمینان و امنیت ساکن بودند. و در آن زمین صاحب اقتداری نبودکه اذیت رساند و از صیدونیان دور بوده، با کسی کار نداشتند.
\par 8 پس نزد برادران خود به صرعه واشتاول آمدند. و برادران ایشان به ایشان گفتند: «چه خبر دارید؟»
\par 9 گفتند: «برخیزیم و بر ایشان هجوم آوریم، زیرا که زمین را دیده‌ایم که اینک بسیار خوب است، و شما خاموش هستید، پس کاهلی مورزید بلکه رفته، داخل شوید و زمین رادر تصرف آورید.
\par 10 و چون داخل شوید به قوم مطمئن خواهید رسید، و زمین بسیار وسیع است، و خدا آن را به‌دست شما داده است، و آن جایی است که از هرچه در جهان است، باقی ندارد.»
\par 11 پس ششصد نفر از قبیله دان مسلح شده، به آلات جنگ از آنجا یعنی از صرعه و اشتاول روانه شدند.
\par 12 و برآمده، در قریه یعاریم دریهودا اردو زدند. لهذا تا امروز آن مکان را محنه دان می‌خوانند و اینک در پشت قریه یعاریم است.
\par 13 و از آنجا به کوهستان افرایم گذشته، به خانه میخا رسیدند.
\par 14 و آن پنج نفر که برای جاسوسی زمین لایش رفته بودند، برادران خود را خطاب کرده، گفتند: «آیا می‌دانید که در این خانه‌ها ایفود و ترافیم وتمثال تراشیده و تمثال ریخته شده‌ای هست؟ پس الان فکر کنید که چه باید بکنید.»
\par 15 پس به آنسو برگشته، به خانه جوان لاوی، یعنی به خانه میخا آمده، سلامتی او را پرسیدند.
\par 16 و آن ششصد مرد مسلح شده، به آلات جنگ که ازپسران دان بودند، در دهنه دروازه ایستاده بودند.
\par 17 و آن پنج نفر که برای جاسوسی زمین رفته بودند برآمده، به آنجا داخل شدند، و تمثال تراشیده و ایفود و ترافیم و تمثال ریخته شده راگرفتند، و کاهن با آن ششصد مرد مسلح شده، به آلات جنگ به دهنه دروازه ایستاده بود.
\par 18 وچون آنها به خانه میخا داخل شده، تمثال تراشیده و ایفود و ترافیم و تمثال ریخته شده راگرفتند، کاهن به ایشان گفت: «چه می‌کنید؟»
\par 19 ایشان به وی گفتند: «خاموش شده، دست را بر دهانت بگذار و همراه ما آمده، برای ما پدر وکاهن باش. کدام برایت بهتر است که کاهن خانه یک شخص باشی یا کاهن سبطی و قبیله‌ای دراسرائیل شوی؟»
\par 20 پس دل کاهن شاد گشت. وایفود و ترافیم و تمثال تراشیده را گرفته، در میان قوم داخل شد.
\par 21 پس متوجه شده، روانه شدند، و اطفال ومواشی و اسباب را پیش روی خود قرار دادند.
\par 22 و چون ایشان از خانه میخا دور شدند، مردانی که در خانه های اطراف خانه میخا بودند جمع شده، بنی دان را تعاقب نمودند.
\par 23 و بنی دان راصدا زدند، و ایشان رو برگردانیده، به میخا گفتند: «تو را چه شده است که با این جمعیت آمده‌ای؟»
\par 24 او گفت: «خدایان مرا که ساختم با کاهن گرفته، رفته‌اید؛ و مرا دیگر‌چه چیز باقی است؟ پس چگونه به من می‌گویید که تو را چه شده است؟»
\par 25 و پسران دان او را گفتند: «آواز تو در میان ماشنیده نشود مبادا مردان تند خو بر شما هجوم آورند، و جان خود را با جانهای اهل خانه ات هلاک سازی.»
\par 26 و بنی دان راه خود را پیش گرفتند. و چون میخا دید که ایشان از او قوی ترند، رو گردانیده، به خانه خود برگشت.
\par 27 و ایشان آنچه میخا ساخته بود و کاهنی راکه داشت برداشته، به لایش بر قومی که آرام ومطمئن بودند، برآمدند، و ایشان را به دم شمشیرکشته، شهر را به آتش سوزانیدند.
\par 28 و رهاننده‌ای نبود زیرا که از صیدون دور بود و ایشان را با کسی معامله‌ای نبود و آن شهر در وادی‌ای که نزدبیت رحوب است، واقع بود. پس شهر را بنا کرده، در آن ساکن شدند.
\par 29 و شهر را به اسم پدر خود، دان که برای اسرائیل زاییده شد، دان نامیدند. امااسم شهر قبل از آن لایش بود.
\par 30 و بنی دان آن تمثال تراشیده را برای خود نصب کردند ویهوناتان بن جرشوم بن موسی و پسرانش تا روزاسیر شدن اهل زمین، کهنه بنی دان می‌بودند.پس تمثال تراشیده میخا را که ساخته بودتمامی روزهایی که خانه خدا در شیلوه بود، برای خود نصب نمودند.
\par 31 پس تمثال تراشیده میخا را که ساخته بودتمامی روزهایی که خانه خدا در شیلوه بود، برای خود نصب نمودند.
 
\chapter{19}

\par 1 و در آن ایام که پادشاهی در اسرائیل نبود، مرد لاوی در پشت کوهستان افرایم ساکن بود، و کنیزی از بیت لحم یهودا ازبرای خود گرفته بود.
\par 2 و کنیزش بر او زنا کرده، ازنزد او به خانه پدرش در بیت لحم یهودا رفت، ودر آنجا مدت چهار ماه بماند.
\par 3 و شوهرش برخاسته، از عقب او رفت تا دلش را برگردانیده، پیش خود باز آورد، و غلامی با دو الاغ همراه اوبود، و آن زن او را به خانه پدر خود برد. و چون پدر کنیز او را دید، از ملاقاتش شاد شد.
\par 4 و پدرزنش، یعنی پدر کنیز او را نگاه داشت. پس سه روز نزد وی توقف نمود و اکل و شرب نموده، آنجا بسر بردند.
\par 5 و در روز چهارم چون صبح زود بیدار شدند او برخاست تا روانه شود، اما پدرکنیز به داماد خود گفت که دل خود را به لقمه‌ای نان تقویت ده، و بعد از آن روانه شوید.
\par 6 پس هردو با هم نشسته، خوردند و نوشیدند. و پدر کنیزبه آن مرد گفت: «موافقت کرده، امشب را بمان ودلت شاد باشد.»
\par 7 و چون آن مرد برخاست تا روانه شود، پدرزنش او را الحاح نمود و شب دیگر در آنجا ماند.
\par 8 و در روز پنجم صبح زود برخاست تا روانه شود، پدر کنیز گفت: «دل خود را تقویت نما و تازوال روز تاخیر نمایید.» و ایشان هردو خوردند.
\par 9 و چون آن شخص با کنیز و غلام خود برخاست تا روانه شود، پدر زنش یعنی پدر کنیز او را گفت: «الان روز نزدیک به غروب می‌شود، شب رابمانید اینک روز تمام می‌شود، در اینجا شب رابمان و دلت شاد باشد و فردا بامدادان روانه خواهید شد و به خیمه خود خواهی رسید.»
\par 10 اما آن مرد قبول نکرد که شب را بماند، پس برخاسته، روانه شد و به مقابل یبوس که اورشلیم باشد، رسید، و دو الاغ پالان شده و کنیزش همراه وی بود.
\par 11 و چون ایشان نزد یبوس رسیدند، نزدیک به غروب بود. غلام به آقای خود گفت: «بیا و به این شهر یبوسیان برگشته، شب را در آن بسر بریم.»
\par 12 آقایش وی را گفت: «به شهر غریب که احدی از بنی‌اسرائیل در آن نباشدبرنمی گردیم بلکه به جبعه بگذریم.»
\par 13 و به غلام خود گفت: «بیا و به یکی از این‌جاها، یعنی به جبعه یا رامه نزدیک بشویم و در آن شب رابمانیم.»
\par 14 پس از آنجا گذشته، برفتند و نزد جبعه که از آن بنیامین است، آفتاب بر ایشان غروب کرد.
\par 15 پس به آن طرف برگشتند تا به جبعه داخل شده، شب را در آن بسر برند. و او درآمد در کوچه شهر نشست، اما کسی نبود که ایشان را به خانه خود ببرد و منزل دهد.
\par 16 و اینک مردی پیر در شب از کار خود ازمزرعه می‌آمد، و این شخص از کوهستان افرایم بوده، در جبعه ماوا گزیده بود، اما مردمان آن مکان بنیامینی بودند.
\par 17 و او نظر انداخته، شخص مسافری را در کوچه شهر دید، و آن مرد پیر گفت: «کجا می‌روی و از کجا می‌آیی؟» 
\par 18 او وی را گفت: «ما از بیت لحم یهودا به آن طرف کوهستان افرایم می‌رویم، زیرا از آنجا هستم و به بیت لحم یهودا رفته بودم، و الان عازم خانه خداوند هستم، و هیچ‌کس مرا به خانه خود نمی پذیرد،
\par 19 و نیزکاه و علف به جهت الاغهای ما هست، و نان وشراب هم برای من و کنیز تو و غلامی که همراه بندگانت است، می‌باشد و احتیاج به چیزی نیست.»
\par 20 آن مرد پیر گفت: «سلامتی بر تو باد، تمامی حاجات تو بر من است، اما شب را درکوچه بسر مبر.»
\par 21 پس او را به خانه خود برده، به الاغها خوراک داد و پایهای خود را شسته، خوردند و نوشیدند.
\par 22 و چون دلهای خود را شاد می‌کردند، اینک مردمان شهر، یعنی بعضی اشخاص بنی بلیعال خانه را احاطه کردند، و در را زده، به آن مرد پیرصاحب‌خانه خطاب کرده، گفتند: «آن مرد را که به خانه تو داخل شده است بیرون بیاور تا او رابشناسیم.»
\par 23 و آن مرد صاحب‌خانه نزد ایشان بیرون آمده، به ایشان گفت: «نی‌ای برادرانم شرارت مورزید، چونکه این مرد به خانه من داخل شده است، این عمل زشت را منمایید.
\par 24 اینک دختر باکره من و کنیز این مرد، ایشان را نزد شمابیرون می‌آورم. ایشان را ذلیل ساخته، آنچه درنظر شما پسند آید به ایشان بکنید. لیکن با این مرداین کار زشت را مکنید.»
\par 25 اما آن مردمان نخواستند که او را بشنوند. پس آن شخص کنیزخود را گرفته، نزد ایشان بیرون آورد و او راشناختند و تمامی شب تا صبح او را بی‌عصمت می‌کردند، و در طلوع فجر او را رها کردند.
\par 26 وآن زن در سپیده صبح آمده، به در خانه آن شخص که آقایش در آن بود، افتاد تا روشن شد.
\par 27 و در وقت صبح آقایش برخاسته، بیرون آمد تا به راه خود برود و اینک کنیزش نزد در خانه افتاده، و دستهایش بر آستانه بود.
\par 28 و او وی راگفت: «برخیز تا برویم.» اما کسی جواب نداد، پس آن مرد او را بر الاغ خود گذاشت و برخاسته، به مکان خود رفت.
\par 29 و چون به خانه خود رسید، کاردی برداشت و کنیز خود را گرفته، اعضای اورا به دوازده قطعه تقسیم کرد، و آنها را در تمامی حدود اسرائیل فرستاد.و هر‌که این را دیدگفت: «از روزی که بنی‌اسرائیل از مصر بیرون آمده‌اند تا امروز عمل مثل این کرده و دیده ونشده است. پس در آن تامل کنید و مشورت کرده، حکم نمایید.»
\par 30 و هر‌که این را دیدگفت: «از روزی که بنی‌اسرائیل از مصر بیرون آمده‌اند تا امروز عمل مثل این کرده و دیده ونشده است. پس در آن تامل کنید و مشورت کرده، حکم نمایید.»
 
\chapter{20}

\par 1 و جمیع بنی‌اسرائیل بیرون آمدند وجماعت مثل شخص واحد از دان تابئرشبع با اهل زمین جلعاد نزد خداوند در مصفه جمع شدند.
\par 2 و سروران تمام قوم و جمیع اسباطاسرائیل یعنی چهارصد هزار مرد شمشیر زن پیاده در جماعت قوم خدا حاضر بودند.
\par 3 وبنی بنیامین شنیدند که بنی‌اسرائیل در مصفه برآمده‌اند. و بنی‌اسرائیل گفتند: «بگویید که این عمل زشت چگونه شده است.»
\par 4 آن مرد لاوی که شوهر زن مقتوله بود، در جواب گفت: «من با کنیزخود به جبعه که از آن بنیامین باشد، آمدیم تا شب را بسر بریم.
\par 5 و اهل جبعه بر من برخاسته، خانه رادر شب، گرد من احاطه کردند، و مرا خواستندبکشند و کنیز مرا ذلیل نمودند که بمرد.
\par 6 و کنیز خود را گرفته، او را قطعه قطعه کردم و او را درتمامی ولایت ملک اسرائیل فرستادم، زیرا که کارقبیح و زشت در اسرائیل نمودند.
\par 7 هان جمیع شما، ای بنی‌اسرائیل حکم و مشورت خود رااینجا بیاورید.»
\par 8 آنگاه تمام قوم مثل شخص واحد برخاسته، گفتند: «هیچ کدام از ما به خیمه خود نخواهیم رفت، و هیچ کدام از ما به خانه خود بر نخواهیم گشت.
\par 9 و حال کاری که به جبعه خواهیم کرد این است که به حسب قرعه بر آن برآییم.
\par 10 و ده نفراز صد و صد از هزار و هزار از ده هزار از تمامی اسباط اسرائیل بگیریم تا آذوقه برای قوم بیاورند، و تا چون به جبعه بنیامینی برسند با ایشان موافق همه قباحتی که در اسرائیل نموده‌اند، رفتارنمایند.»
\par 11 پس جمیع مردان اسرائیل بر شهر جمع شده، مثل شخص واحد متحد شدند.
\par 12 و اسباطاسرائیل اشخاصی چند در تمامی سبط بنیامین فرستاده، گفتند: «این چه شرارتی است که در میان شما واقع شده است؟
\par 13 پس الان آن مردان بنی بلیعال را که در جبعه هستند، تسلیم نمایید تاآنها را به قتل رسانیم، و بدی را از اسرائیل دورکنیم.» اما بنیامینیان نخواستند که سخن برادران خود بنی‌اسرائیل را بشنوند.
\par 14 و بنی بنیامین ازشهرهای خود به جبعه جمع شدند تا بیرون رفته، با بنی‌اسرائیل جنگ نمایند.
\par 15 و از بنی بنیامین درآن روز بیست و ششهزار مرد شمشیرزن ازشهرها سان دیده شد، غیر از ساکنان جبعه که هفتصد نفر برگزیده، سان دیده شد.
\par 16 و از تمام این گروه هفتصد نفر چپ دست برگزیده شدند که هر یکی از آنها مویی را به سنگ فلاخن می‌زدند وخطا نمی کردند.
\par 17 و از مردان اسرائیل سوای بنیامینیان چهارصد هزار مرد شمشیرزن سان دیده شد که جمیع اینها مردان جنگی بودند.
\par 18 و بنی‌اسرائیل برخاسته، به بیت ئیل رفتند واز خدا مشورت خواسته، گفتند: «کیست که اولااز ما برای جنگ نمودن با بنی بنیامین برآید؟» خداوند گفت: «یهودا اول برآید.»
\par 19 و بنی‌اسرائیل بامدادان برخاسته، در برابرجبعه اردو زدند.
\par 20 و مردان اسرائیل بیرون رفتندتا با بنیامینیان جنگ نمایند، و مردان اسرائیل برابر ایشان در جبعه صف آرایی کردند.
\par 21 وبنی بنیامین از جبعه بیرون آمده، در آن روز بیست و دو هزار نفر از اسرائیل را بر زمین هلاک کردند.
\par 22 و قوم، یعنی مردان اسرائیل خود را قوی‌دل ساخته، بار دیگر صف آرایی نمودند، در مکانی که روز اول صف آرایی کرده بودند.
\par 23 وبنی‌اسرائیل برآمده، به حضور خداوند تا شام گریه کردند، و از خداوند مشورت خواسته، گفتند: «آیا بار دیگر نزدیک بشوم تا با برادران خود بنی بنیامین جنگ نمایم؟» خداوند گفت: «به مقابله ایشان برآیید.»
\par 24 و بنی‌اسرائیل در روز دوم به مقابله بنی بنیامین پیش آمدند.
\par 25 و بنیامینیان در روزدوم به مقابله ایشان از جبعه بیرون شده، بار دیگرهجده هزار نفر از بنی‌اسرائیل را بر زمین هلاک ساختند که جمیع اینها شمشیرزن بودند.
\par 26 آنگاه تمامی بنی‌اسرائیل، یعنی تمامی قوم برآمده، به بیت ئیل رفتند و گریه کرده، در آنجا به حضور خداوند توقف نمودند، و آن روز را تا شام روزه داشته، قربانی های سوختنی و ذبایح سلامتی به حضور خداوند گذرانیدند.
\par 27 و بنی‌اسرائیل از خداوند مشورت خواستند. وتابوت عهد خدا آن روزها در آنجا بود.
\par 28 وفینحاس بن العازار بن هارون در آن روزها پیش آن ایستاده بود، و گفتند: «آیا بار دیگر بیرون روم و بابرادران خود بنی بنیامین جنگ کنم یا دست بردارم؟» خداوند گفت: «برآی زیرا که فردا او رابه‌دست تو تسلیم خواهم نمود.»
\par 29 پس اسرائیل در هر طرف جبعه کمین ساختند.
\par 30 و بنی‌اسرائیل در روز سوم به مقابله بنی بنیامین برآمدند، و مثل سابق در برابر جبعه صف آرایی نمودند.
\par 31 و بنی بنیامین به مقابله قوم بیرون آمده، از شهر کشیده شدند و به زدن و کشتن قوم در راهها که یکی از آنها به سوی بیت ئیل ودیگری به سوی جبعه می‌رود مثل سابق شروع کردند، و به قدر سی نفر از اسرائیل در صحراکشته شدند.
\par 32 و بنی بنیامین گفتند که «ایشان مثل سابق پیش ما منهزم شدند.» اما بنی‌اسرائیل گفتند: «بگریزیم تا ایشان را از شهر به راههابکشیم.»
\par 33 و تمامی مردان اسرائیل از مکان خودبرخاسته، در بعل تامار صف آرایی نمودند، وکمین کنندگان اسرائیل از مکان خود یعنی از معره جبعه به در جستند.
\par 34 و ده هزار مرد برگزیده ازتمام اسرائیل در برابر جبعه آمدند و جنگ سخت شد، و ایشان نمی دانستند که بلا بر ایشان رسیده است.
\par 35 و خداوند بنیامین را به حضور اسرائیل مغلوب ساخت و بنی‌اسرائیل در آن روز بیست وپنجهزار و یکصد نفر را از بنیامین هلاک ساختندکه جمیع ایشان شمشیرزن بودند.
\par 36 و بنی بنیامین دیدند که شکست یافته اندزیرا که مردان اسرائیل به بنیامینیان جا داده بودند، چونکه اعتماد داشتند بر کمینی که به اطراف جبعه نشانده بودند.
\par 37 و کمین کنندگان تعجیل نموده، بر جبعه هجوم آوردند و کمین کنندگان خود را پراکنده ساخته، تمام شهر را به دم شمشیرزدند.
\par 38 و در میان مردان اسرائیل و کمین کنندگان علامتی قرار داده شد که تراکم دود بسیار بلند ازشهر برافرازند.
\par 39 پس چون مردان اسرائیل درجنگ رو گردانیدند، بنیامینیان شروع کردند به زدن و کشتن قریب سی نفر از مردان اسرائیل زیراگفتند یقین ایشان مثل جنگ اول از حضور ماشکست یافته‌اند.
\par 40 و چون آن تراکم ستون دوداز شهر بلند شدن گرفت، بنیامینیان از عقب خودنگریستند و اینک تمام شهر به سوی آسمان به دود بالا می‌رود.
\par 41 و بنی‌اسرائیل برگشتند وبنیامینیان پریشان شدند، زیرا دیدند که بلا برایشان رسیده است.
\par 42 پس از حضور مردان اسرائیل به راه صحرا روگردانیدند. اما جنگ، ایشان را در‌گرفت و آنانی که از شهر بیرون آمدندایشان را در میان، هلاک ساختند.
\par 43 پس بنیامینیان را احاطه کرده، ایشان را تعاقب نمودند، و در منوحه در مقابل جبعه به سوی طلوع آفتاب ایشان را پایمال کردند.
\par 44 و هجده هزار نفر ازبنیامین که جمیع ایشان مردان جنگی بودند، افتادند.
\par 45 و ایشان برگشته، به سوی صحرا تاصخره رمون بگریختند. و پنج هزار نفر از ایشان را به‌سر راهها هلاک کردند، و ایشان را تا جدعوم تعاقب کرده، دو هزار نفر از ایشان را کشتند.
\par 46 پس جمیع کسانی که در آن روز از بنیامین افتادند، بیست و پنج هزار مرد شمشیرزن بودندکه جمیع آنها مردان جنگی بودند.
\par 47 اما ششصدنفر برگشته، به سوی بیابان به صخره رمون فرارکردند، و در صخره رمون چهار ماه بماندند.ومردان اسرائیل بر بنیامینیان برگشته، ایشان را به دم شمشیر کشتند، یعنی تمام اهل شهر و بهایم وهرچه را که یافتند و همچنین همه شهرهایی را که به آنها رسیدند، به آتش سوزانیدند.
\par 48 ومردان اسرائیل بر بنیامینیان برگشته، ایشان را به دم شمشیر کشتند، یعنی تمام اهل شهر و بهایم وهرچه را که یافتند و همچنین همه شهرهایی را که به آنها رسیدند، به آتش سوزانیدند.
 
\chapter{21}

\par 1 و مردان اسرائیل در مصفه قسم خورده، گفتند که «احدی از ما دختر خود را به بنیامینیان به زنی ندهند.»
\par 2 و قوم به بیت ئیل آمده، در آنجا به حضور خدا تا شام نشستند و آواز خودرا بلند کرده، زار‌زار بگریستند.
\par 3 و گفتند: «ای یهوه، خدای اسرائیل، این چرا در اسرائیل واقع شده است که امروز یک سبط از اسرائیل کم شود؟»
\par 4 و در فردای آن روز قوم به زودی برخاسته، مذبحی در آنجا بنا کردند، وقربانی های سوختنی و ذبایح سلامتی گذرانیدند.
\par 5 و بنی‌اسرائیل گفتند: «کیست از تمامی اسباطاسرائیل که در جماعت نزد خداوند بر نیامده است.» زیرا قسم سخت خورده، گفته بودند که هرکه به حضور خداوند به مصفه نیاید، البته کشته شود.
\par 6 و بنی‌اسرائیل درباره برادر خود بنیامین پشیمان شده، گفتند: «امروز یک سبط از اسرائیل منقطع شده است.
\par 7 برای بقیه ایشان درباره زنان چه کنیم؟ زیرا که ما به خداوند قسم خورده‌ایم که از دختران خود به ایشان به زنی ندهیم.»
\par 8 و گفتند: «کدام‌یک از اسباط اسرائیل است که به حضور خداوند به مصفه نیامده است؟» و اینک از یابیش جلعاد کسی به اردو و جماعت نیامده بود.
\par 9 زیرا چون قوم شمرده شدند اینک ازساکنان یابیش جلعاد احدی در آنجا نبود.
\par 10 پس جماعت دوازده هزار نفر از شجاع ترین قوم را به آنجا فرستاده، و ایشان را امر کرده، گفتند: «بروید و ساکنان یابیش جلعاد را با زنان و اطفال به دم شمشیر بکشید.
\par 11 و آنچه باید بکنید این است که هر مردی را و هر زنی را که با مرد خوابیده باشد، هلاک کنید.»
\par 12 و در میان ساکنان یابیش جلعادچهارصد دختر باکره که با ذکوری نخوابیده ومردی را نشناخته بودند یافتند، و ایشان را به اردو در شیلوه که در زمین کنعان است، آوردند. 
\par 13 و تمامی جماعت نزد بنی بنیامین که درصخره رمون بودند فرستاده، ایشان را به صلح دعوت کردند.
\par 14 و در آن وقت بنیامینیان برگشتند و دخترانی را که از زنان یابیش جلعادزنده نگاه داشته بودند به ایشان دادند، و باز ایشان را کفایت نکرد.
\par 15 و قوم برای بنیامین پشیمان شدند، زیراخداوند در اسباط اسرائیل شقاق پیدا کرده بود.
\par 16 و مشایخ جماعت گفتند: «درباره زنان به جهت باقی ماندگان چه کنیم چونکه زنان از بنیامین منقطع شده‌اند؟
\par 17 و گفتند: میراثی به جهت نجات‌یافتگان بنیامین باید باشد تا سبطی ازاسرائیل محو نشود.
\par 18 اما ما دختران خود را به ایشان به زنی نمی توانیم داد زیرا بنی‌اسرائیل قسم خورده، گفته‌اند ملعون باد کسی‌که زنی به بنیامین دهد.
\par 19 و گفتند: «اینک هر سال در شیلوه که به طرف شمال بیت ئیل و به طرف مشرق راهی که ازبیت ئیل به شکیم می‌رود، و به سمت جنوبی لبونه است، عیدی برای خداوند می‌باشد.»
\par 20 پس بنی بنیامین را امر فرموده، گفتند: «بروید درتاکستانها در کمین باشید،
\par 21 و نگاه کنید و اینک اگر دختران شیلوه بیرون آیند تا با رقص کنندگان رقص کنند، آنگاه از تاکستانها درآیید، و ازدختران شیلوه هرکس زن خود را ربوده، به زمین بنیامین برود.
\par 22 و چون پدران و برادران ایشان آمده، نزد ما شکایت کنند، به ایشان خواهیم گفت: «ایشان را به‌خاطر ما ببخشید، چونکه مابرای هر کس زنش را در جنگ نگاه نداشتیم، وشما آنها را به ایشان ندادید، الان مجرم می‌باشید.»
\par 23 پس بنی بنیامین چنین کردند، و ازرقص کنندگان، زنان را برحسب شماره خودگرفتند، و ایشان را به یغما برده، رفتند، و به ملک خود برگشته، شهرها را بنا کردند و در آنها ساکن شدند.و در آن وقت بنی‌اسرائیل هر کس به سبط خود و به قبیله خود روانه شدند، و از آنجاهرکس به ملک خود بیرون رفتند.
\par 24 و در آن وقت بنی‌اسرائیل هر کس به سبط خود و به قبیله خود روانه شدند، و از آنجاهرکس به ملک خود بیرون رفتند.


\end{document}