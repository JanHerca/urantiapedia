\begin{document}

\title{Daniel}


\chapter{1}

\par 1 در سال سوم سلطنت یهویاقیم پادشاه یهودا، نبوکدنصر پادشاه بابل به اورشلیم آمده، آن را محاصره نمود.
\par 2 و خداوند یهویاقیم پادشاه یهودا را با بعضی از ظروف خانه خدا به‌دست او تسلیم نمود و او آنها را به زمین شنعار به خانه خدای خود آورد و ظروف را به بیت‌المال خدای خویش گذاشت.
\par 3 و پادشاه اشفناز رئیس خواجه‌سرایان خویش را امر فرمود که بعضی ازبنی‌اسرائیل و از اولاد پادشاهان و از شرفا رابیاورد.
\par 4 جوانانی که هیچ عیبی نداشته باشند ونیکومنظر و در هرگونه حکمت ماهر و به علم داناو به فنون فهیم باشند که قابلیت برای ایستادن درقصر پادشاه داشته باشند و علم و زبان کلدانیان رابه ایشان تعلیم دهند.
\par 5 و پادشاه وظیفه روزینه ازطعام پادشاه و از شرابی که او می‌نوشید تعیین نمود و (امر فرمود) که ایشان را سه سال تربیت نمایند و بعد از انقضای آن مدت در حضورپادشاه حاضر شوند.
\par 6 و در میان ایشان دانیال وحننیا و میشائیل و عزریا از بنی یهودا بودند.
\par 7 ورئیس خواجه‌سرایان نامها به ایشان نهاد، امادانیال را به بلطشصر و حننیا را به شدرک ومیشائیل را به میشک و عزریا را به عبدنغو مسمی ساخت.
\par 8 اما دانیال در دل خود قصد نمود که خویشتن را از طعام پادشاه و از شرابی که او می‌نوشید نجس نسازد. پس از رئیس خواجه‌سرایان درخواست نمود که خویشتن را نجس نسازد.
\par 9 وخدا دانیال را نزد رئیس خواجه‌سرایان محترم ومکرم ساخت.
\par 10 پس رئیس خواجه‌سرایان به دانیال گفت: «من از آقای خود پادشاه که خوراک و مشروبات شما را تعیین نموده است می‌ترسم. چرا چهره های شما را از سایر جوانانی که ابنای جنس شما می‌باشند، زشتتر بیند و همچنین سرمرا نزد پادشاه در خطر خواهید انداخت.»
\par 11 پس دانیال به رئیس ساقیان که رئیس خواجه‌سرایان اورا بر دانیال و حننیا و میشائیل و عزریا گماشته بودگفت:
\par 12 «مستدعی آنکه بندگان خود را ده روزتجربه نمایی و به ما بقول برای خوردن بدهند وآب به جهت نوشیدن.
\par 13 و چهره های ما وچهره های سایر جوانانی را که طعام پادشاه رامی خورند به حضور تو ملاحظه نمایند و به نهجی که خواهی دید با بندگانت عمل نمای.»
\par 14 و او ایشان را در این امر اجابت نموده، ده روزایشان را تجربه کرد.
\par 15 و بعد از انقضای ده روزمعلوم شد که چهره های ایشان از سایر جوانانی که طعام پادشاه را می‌خوردند نیکوتر و فربه تربود.
\par 16 پس رئیس ساقیان طعام ایشان و شراب راکه باید بنوشند برداشت و بقول به ایشان داد.
\par 17 اما خدا به این چهار جوان معرفت و ادراک در هر گونه علم و حکمت عطا فرمود و دانیال درهمه رویاها و خوابها فهیم گردید.
\par 18 و بعد از انقضای روزهایی که پادشاه امر فرموده بود که ایشان را بیاورند، رئیس خواجه‌سرایان ایشان رابه حضور نبوکدنصر آورد.
\par 19 و پادشاه با ایشان گفتگو کرد و از جمیع ایشان کسی مثل دانیال وحننیا و میشائیل و عزریا یافت نشد پس درحضور پادشاه ایستادند.
\par 20 و در هر مسئله حکمت و فطانت که پادشاه از ایشان استفسارکرد، ایشان را از جمیع مجوسیان و جادوگرانی که در تمام مملکت او بودند ده مرتبه بهتر یافت.ودانیال بود تا سال اول کورش پادشاه.
\par 21 ودانیال بود تا سال اول کورش پادشاه.

\chapter{2}

\par 1 و در سال دوم سلطنت نبوکدنصر، نبوکدنصر خوابی دید و روحش مضطرب شده، خواب از وی دور شد.
\par 2 پس پادشاه امرفرمود که مجوسیان و جادوگران و فالگیران وکلدانیان را بخوانند تا خواب پادشاه را برای اوتعبیر نمایند و ایشان آمده، به حضور پادشاه ایستادند.
\par 3 و پادشاه به ایشان گفت: «خوابی دیده‌ام وروحم برای فهمیدن خواب مضطرب است.»
\par 4 کلدانیان به زبان ارامی به پادشاه عرض کردند که «پادشاه تا به ابد زنده بماند! خواب را برای بندگانت بیان کن و تعبیر آن را خواهیم گفت.»
\par 5 پادشاه در جواب کلدانیان فرمود: «فرمان ازمن صادر شد که اگر خواب و تعبیر آن را برای من بیان نکنید پاره پاره خواهید شد و خانه های شمارا مزبله خواهند ساخت.
\par 6 و اگر خواب وتعبیرش را بیان کنید، بخششها و انعامها و اکرام عظیمی از حضور من خواهید یافت. پس خواب و تعبیرش را به من اعلام نمایید.»
\par 7 ایشان باردیگر جواب داده، گفتند که «پادشاه بندگان خودرا از خواب اطلاع دهد و آن را تعبیر خواهیم کرد.»
\par 8 پادشاه در جواب گفت: «یقین می‌دانم که شما فرصت می‌جویید، چون می‌بینید که فرمان ازمن صادر شده است.
\par 9 لیکن اگر خواب را به من اعلام ننمایید برای شما فقط یک حکم است. زیراکه سخنان دروغ و باطل را ترتیب داده‌اید که به حضور من بگویید تا وقت تبدیل شود. پس خواب را به من بگویید و خواهم دانست که آن راتعبیر توانید نمود.»
\par 10 کلدانیان به حضور پادشاه جواب داده، گفتند، که «کسی بر روی زمین نیست که مطلب پادشاه را بیان تواند نمود، لهذا هیچ پادشاه یاحاکم یا سلطانی نیست که چنین امری را از هرمجوسی یا جادوگر یا کلدانی بپرسد.
\par 11 و مطلبی که پادشاه می‌پرسد، چنان بدیع است که احدی غیراز خدایانی که مسکن ایشان با انسان نیست، نمی تواند آن را برای پادشاه بیان نماید.»
\par 12 از این جهت پادشاه خشم نمود و به شدت غضبناک گردیده، امر فرمود که جمیع حکیمان بابل را هلاک کنند.
\par 13 پس فرمان صادر شد و به صدد کشتن حکیمان برآمدند و دانیال و رفیقانش را می‌طلبیدند تا ایشان را به قتل رسانند.
\par 14 آنگاه دانیال با حکمت و عقل به اریوک رئیس جلادان پادشاه که برای کشتن حکیمان بابل بیرون می‌رفت، سخن گفت.
\par 15 و اریوک سردار پادشاه راخطاب کرده، گفت: «چرا فرمان از حضور پادشاه چنین سخت است؟» آنگاه اریوک دانیال را از کیفیت امر مطلع ساخت.
\par 16 و دانیال داخل شده، از پادشاه درخواست نمود که مهلت به وی داده شود تا تعبیر را برای پادشاه اعلام نماید.
\par 17 پس دانیال به خانه خودرفته، رفقای خویش حننیا و میشائیل و عزریا رااز این امر اطلاع داد،
\par 18 تا درباره این راز از خدای آسمانها رحمت بطلبند مبادا که دانیال و رفقایش با سایر حکیمان بابل هلاک شوند.
\par 19 آنگاه آن رازبه دانیال در رویای شب کشف شد. پس دانیال خدای آسمانها را متبارک خواند.
\par 20 و دانیال متکلم شده، گفت: «اسم خدا تا ابدلاباد متبارک بادزیرا که حکمت و توانایی از آن وی است.
\par 21 و اووقتها و زمانها را تبدیل می‌کند. پادشاهان رامعزول می‌نماید و پادشاهان را نصب می‌کند. حکمت را به حکیمان می‌بخشد وفطانت پیشه گان را تعلیم می‌دهد.
\par 22 اوست که چیزهای عمیق و پنهان را کشف می‌نماید. به آنچه در ظلمت است عارف می‌باشد و نور نزدوی ساکن است.
\par 23 ‌ای خدای پدران من تو راشکر می‌گویم و تسبیح می‌خوانم زیرا که حکمت و توانایی را به من عطا فرمودی و الان آنچه را که از تو درخواست کرده‌ایم به من اعلام نمودی چونکه ما را از مقصود پادشاه اطلاع دادی.»
\par 24 و از این جهت دانیال نزد اریوک که پادشاه او را به جهت هلاک ساختن حکمای بابل مامورکرده بود رفت، و به وی رسیده، چنین گفت که «حکمای بابل را هلاک مساز. مرا به حضورپادشاه ببر و تعبیر را برای پادشاه بیان خواهم نمود.»
\par 25 آنگاه اریوک دانیال را بزودی به حضورپادشاه رسانید و وی را چنین گفت که «شخصی را از اسیران یهودا یافته‌ام که تعبیر را برای پادشاه بیان تواند نمود.»
\par 26 پادشاه دانیال را که به بلطشصر مسمی بود خطاب کرده، گفت: «آیا تومی توانی خوابی را که دیده‌ام و تعبیرش را برای من بیان نمایی؟»
\par 27 دانیال به حضور پادشاه جواب داد و گفت: «رازی را که پادشاه می‌طلبد، نه حکیمان و نه جادوگران و نه مجوسیان و نه منجمان می‌توانندآن را برای پادشاه حل کنند.
\par 28 لیکن خدایی در آسمان هست که کاشف اسرارمی باشد و او نبوکدنصر پادشاه را از آنچه در ایام آخر واقع خواهد شد اعلام نموده است. خواب تو و رویای سرت که در بسترت دیده‌ای این است:
\par 29 ‌ای پادشاه فکرهای تو بر بسترت درباره آنچه بعد از این واقع خواهد شد به‌خاطرت آمد وکاشف الاسرار، تو را از آنچه واقع خواهد شدمخبر ساخته است.
\par 30 و اما این راز بر من ازحکمتی که من بیشتر از سایر زندگان دارم مکشوف نشده است، بلکه تا تعبیر بر پادشاه معلوم شود و فکرهای خاطر خود را بدانی.
\par 31 توای پادشاه می‌دیدی و اینک تمثال عظیمی بود واین تمثال بزرگ که درخشندگی آن بی‌نهایت ومنظر آن هولناک بود پیش روی تو برپا شد.
\par 32 سراین تمثال از طلای خالص و سینه و بازوهایش ازنقره و شکم و رانهایش از برنج بود.
\par 33 وساقهایش از آهن و پایهایش قدری از آهن وقدری از گل بود.
\par 34 و مشاهده می‌نمودی تاسنگی بدون دستها جدا شده، پایهای آهنین وگلین آن تمثال را زد و آنها را خرد ساخت.
\par 35 آنگاه آهن و گل و برنج و نقره و طلا با هم خردشد و مثل کاه خرمن تابستانی گردیده، باد آنها راچنان برد که جایی به جهت آنها یافت نشد. و آن سنگ که تمثال را زده بود کوه عظیمی گردید وتمامی جهان را پر ساخت.
\par 36 خواب همین است و تعبیرش را برای پادشاه بیان خواهیم نمود.
\par 37 ‌ای پادشاه، تو پادشاه پادشاهان هستی زیراخدای آسمانها سلطنت و اقتدار و قوت وحشمت به تو داده است.
\par 38 و در هر جایی که بنی آدم سکونت دارند حیوانات صحرا و مرغان هوا را به‌دست تو تسلیم نموده و تو را بر جمیع آنها مسلط گردانیده است. آن سر طلا تو هستی.
\par 39 و بعد از تو سلطنتی دیگر پست‌تر از تو خواهدبرخاست و سلطنت سومی دیگر از برنج که برتمامی جهان سلطنت خواهد نمود.
\par 40 و سلطنت چهارم مثل آهن قوی خواهد بود زیرا آهن همه‌چیز را خرد و نرم می‌سازد. پس چنانکه آهن همه‌چیز را نرم می‌کند، همچنان آن نیز خرد و نرم خواهد ساخت.
\par 41 و چنانکه پایها و انگشتها رادیدی که قدری از گل کوزه‌گر و قدری از آهن بود، همچنان این سلطنت منقسم خواهد شد وقدری از قوت آهن در آن خواهد ماند موافق آنچه دیدی که آهن با گل سفالین آمیخته شده بود.
\par 42 و اما انگشتهای پایهایش قدری از آهن وقدری از گل بود، همچنان این سلطنت قدری قوی و قدری زودشکن خواهد بود.
\par 43 و چنانکه دیدی که آهن با گل سفالین آمیخته شده بود، همچنین اینها خویشتن را با ذریت انسان آمیخته خواهند کرد. اما به نحوی که آهن با گل ممزوج نمی شود، همچنین اینها با یکدیگر ملصق نخواهند شد.
\par 44 و در ایام این پادشاهان خدای آسمانها سلطنتی را که تا ابدالاباد زایل نشود، برپاخواهد نمود و این سلطنت به قومی دیگر منتقل نخواهد شد، بلکه تمامی آن سلطنتها را خردکرده، مغلوب خواهد ساخت و خودش تا ابدالاباد استوار خواهد ماند.
\par 45 و چنانکه سنگ را دیدی که بدون دستها از کوه جدا شده، آهن وبرنج و گل و نقره و طلا را خرد کرد، همچنین خدای عظیم پادشاه را از آنچه بعد از این واقع می‌شود مخبر ساخته است. پس خواب صحیح وتعبیرش یقین است.»
\par 46 آنگاه نبوکدنصر پادشاه به روی خوددرافتاده، دانیال را سجده نمود و امر فرمود که هدایا و عطریات برای او بگذرانند.
\par 47 و پادشاه دانیال را خطاب کرده، گفت: «به درستی که خدای شما خدای خدایان و خداوند پادشاهان و کاشف اسرار است، چونکه تو قادر بر کشف این رازشده‌ای.»
\par 48 پس پادشاه دانیال را معظم ساخت وهدایای بسیار و عظیم به او داد و او را بر تمامی ولایت بابل حکومت داد و رئیس روسا بر جمیع حکمای بابل ساخت.و دانیال از پادشاه درخواست نمود تا شدرک و میشک و عبدنغو رابر کارهای ولایت بابل نصب کرد و اما دانیال دردروازه پادشاه می‌بود.
\par 49 و دانیال از پادشاه درخواست نمود تا شدرک و میشک و عبدنغو رابر کارهای ولایت بابل نصب کرد و اما دانیال دردروازه پادشاه می‌بود.

\chapter{3}

\par 1 نبوکدنصر پادشاه تمثالی از طلا که ارتفاعش شصت ذراع و عرضش شش ذراع بود ساخت و آن را در همواری دورا درولایت بابل نصب کرد.
\par 2 و نبوکدنصر پادشاه فرستاد که امرا و روسا و والیان و داوران وخزانه‌داران و مشیران و وکیلان و جمیع سروران ولایتها را جمع کنند تا به جهت تبرک تمثالی که نبوکدنصر پادشاه نصب نموده بود بیایند.
\par 3 پس امرا و روسا و والیان و داوران و خزانه‌داران ومشیران و وکیلان و جمیع سروران ولایتها به جهت تبرک تمثالی که نبوکدنصر پادشاه نصب نموده بود جمع شده، پیش تمثالی که نبوکدنصرنصب کرده بود ایستادند.
\par 4 و منادی به آواز بلندندا کرده، می‌گفت: «ای قومها و امت‌ها و زبانهابرای شما حکم است؛
\par 5 که چون آواز کرنا و سرناو عود و بربط و سنتور و کمانچه و هر قسم آلات موسیقی را بشنوید، آنگاه به رو افتاده، تمثال طلارا که نبوکدنصر پادشاه نصب کرده است سجده نمایید.
\par 6 و هر‌که به رو نیفتد و سجده ننماید درهمان ساعت در میان تون آتش ملتهب افکنده خواهد شد.»
\par 7 لهذا چون همه قومها آواز کرنا و سرنا و عودو بربط و سنتور و هر قسم آلات موسیقی راشنیدند، همه قومها و امت‌ها و زبانها به رو افتاده، تمثال طلا را که نبوکدنصر پادشاه نصب کرده بودسجده نمودند.
\par 8 اما در آنوقت بعضی کلدانیان نزدیک آمده، بر یهودیان شکایت آوردند،
\par 9 و به نبوکدنصر پادشاه عرض کرده، گفتند: «ای پادشاه تا به ابد زنده باش!
\par 10 تو‌ای پادشاه فرمانی صادرنمودی که هر‌که آواز کرنا و سرنا و عود و بربط وسنتور و کمانچه و هر قسم آلات موسیقی رابشنود به رو افتاده، تمثال طلا را سجده نماید.
\par 11 و هر‌که به رو نیفتد و سجده ننماید در میان تون آتش ملتهب افکنده شود.
\par 12 پس چند نفر یهودکه ایشان را بر کارهای ولایت بابل گماشته‌ای هستند، یعنی شدرک و میشک و عبدنغو. این اشخاص‌ای پادشاه، تو را احترام نمی نمایند وخدایان تو را عبادت نمی کنند و تمثال طلا را که نصب نموده‌ای سجده نمی نمایند.»
\par 13 آنگاه نبوکدنصر با خشم و غضب فرمود تاشدرک و میشک و عبدنغو را حاضر کنند. پس این اشخاص را در حضور پادشاه آوردند.
\par 14 پس نبوکدنصر ایشان را خطاب کرده، گفت: «ای شدرک و میشک و عبدنغو! آیا شما عمد خدایان مرا نمی پرستید وتمثال طلا را که نصب نموده‌ام سجده نمی کنید؟
\par 15 الان اگر مستعد بشوید که چون آواز کرنا و سرنا و عود و بربط و سنتور وکمانچه و هر قسم آلات موسیقی را بشنوید به روافتاده، تمثالی را که ساخته‌ام سجده نمایید، (فبها) و اما اگر سجده ننمایید، در همان ساعت در میان تون آتش ملتهب انداخته خواهید شد وکدام خدایی است که شما را از دست من رهایی دهد.»
\par 16 شدرک و میشک و عبدنغو در جواب پادشاه گفتند: «ای نبوکدنصر! درباره این امر ما راباکی نیست که تو را جواب دهیم.
\par 17 اگر چنین است، خدای ما که او را می‌پرستیم قادر است که ما را از تون آتش ملتهب برهاند و او ما را از دست تو‌ای پادشاه خواهد رهانید.
\par 18 و اگر نه، ای پادشاه تو را معلوم باد که خدایان تو را عبادت نخواهیم کرد و تمثال طلا را که نصب نموده‌ای سجده نخواهیم نمود.»
\par 19 آنگاه نبوکدنصر از خشم مملو گردید وهیئت چهره‌اش بر شدرک و میشک و عبدنغومتغیر گشت و متکلم شده، فرمود تا تون را هفت چندان زیاده تر از عادتش بتابند.
\par 20 و به قویترین شجاعان لشکر خود فرمود که شدرک و میشک وعبدنغو را ببندند و در تون آتش ملتهب بیندازند.
\par 21 پس این اشخاص را در رداها و جبه‌ها وعمامه‌ها و سایر لباسهای ایشان بستند و در میان تون آتش ملتهب افکندند.
\par 22 و چونکه فرمان پادشاه سخت بود و تون بی‌نهایت تابیده شده، شعله آتش آن کسان را که شدرک و میشک وعبدنغو را برداشته بودند کشت.
\par 23 و این سه مردیعنی شدرک و میشک و عبدنغو در میان تون آتش ملتهب بسته افتادند.
\par 24 آنگاه نبوکدنصر پادشاه در حیرت افتاد وبزودی هر‌چه تمامتر برخاست و مشیران خود راخطاب کرده، گفت: «آیا سه شخص نبستیم و درمیان آتش نینداختیم؟» ایشان در جواب پادشاه عرض کردند که «صحیح است‌ای پادشاه!»
\par 25 اودر جواب گفت: «اینک من چهار مرد می‌بینم که گشاده در میان آتش می‌خرامند و ضرری به ایشان نرسیده است و منظر چهارمین شبیه پسرخدا است.»
\par 26 پس نبوکدنصر به دهنه تون آتش ملتهب نزدیک آمد و خطاب کرده، گفت: «ای شدرک و میشک و عبدنغو! ای بندگان خدای تعالی بیرون شوید و بیایید.» پس شدرک و میشک و عبدنغو از میان آتش بیرون آمدند.
\par 27 و امرا وروسا و والیان و مشیران پادشاه جمع شده، آن مردان را دیدند که آتش به بدنهای ایشان اثری نکرده و مویی از سر ایشان نسوخته و رنگ ردای ایشان تبدیل نشده، بلکه بوی آتش به ایشان نرسیده است.
\par 28 آنگاه نبوکدنصر متکلم شده، گفت: «متبارک باد خدای شدرک و میشک و عبدنغو که فرشته خود را فرستاد و بندگان خویش را که بر اوتوکل داشتند و به فرمان پادشاه مخالفت ورزیدندو بدنهای خود را تسلیم نمودند تا خدای دیگری سوای خدای خویش را عبادت و سجده ننمایند، رهایی داده است.
\par 29 بنابراین فرمانی از من صادرشد که هر قوم و امت و زبان که حرف ناشایسته‌ای به ضد خدای شدرک و میشک و عبدنغو بگویند، پاره پاره شوند و خانه های ایشان به مزبله مبدل گردد، زیرا خدایی دیگر نیست که بدین منوال رهایی تواند داد.»آنگاه پادشاه (منصب ) شدرک و میشک و عبدنغو را در ولایت بابل برتری داد.
\par 30 آنگاه پادشاه (منصب ) شدرک و میشک و عبدنغو را در ولایت بابل برتری داد.

\chapter{4}

\par 1 از نبوکدنصر پادشاه، به تمامی قومها و امت‌ها و زبانها که بر تمامی زمین ساکنندسلامتی شما افزون باد!
\par 2 من مصلحت دانستم که آیات و عجایبی را که خدای تعالی به من نموده است بیان نمایم.
\par 3 آیات او چه قدر بزرگ وعجایب او چه قدر عظیم است. ملکوت اوملکوت جاودانی است و سلطنت او تا ابدالاباد.
\par 4 من که نبوکدنصر هستم در خانه خود مطمئن ودر قصر خویش خرم می‌بودم.
\par 5 خوابی دیدم که مرا ترسانید و فکرهایم در بسترم و رویاهای سرم مرا مضطرب ساخت.
\par 6 پس فرمانی از من صادرگردید که جمیع حکیمان بابل را به حضورم بیاورند تا تعبیر خواب را برای من بیان نمایند.
\par 7 آنگاه مجوسیان و جادوگران و کلدانیان ومنجمان حاضر شدند و من خواب را برای ایشان بازگفتم لیکن تعبیرش را برای من بیان نتوانستندنمود.
\par 8 بالاخره دانیال که موافق اسم خدای من به بلطشصر مسمی است و روح خدایان قدوس دراو می‌باشد، درآمد و خواب را به او باز‌گفتم.
\par 9 که «ای بلطشصر، رئیس مجوسیان، چون می‌دانم که روح خدایان قدوس در تو می‌باشد و هیچ سری برای تو مشکل نیست، پس خوابی که دیده‌ام وتعبیرش را به من بگو.
\par 10 رویاهای سرم در بسترم این بود که نظرکردم و اینک درختی در وسط زمین که ارتفاعش عظیم بود.
\par 11 این درخت بزرگ و قوی گردید وبلندیش تا به آسمان رسید و منظرش تا اقصای تمامی زمین بود.
\par 12 برگهایش جمیل و میوه‌اش بسیار و آذوقه برای همه در آن بود. حیوانات صحرا در زیر آن سایه گرفتند و مرغان هوا برشاخه هایش ماوا گزیدند و تمامی بشر از آن پرورش یافتند.
\par 13 در رویاهای سرم در بسترم نظر کردم و اینک پاسبانی و مقدسی از آسمان نازل شد،
\par 14 که به آواز بلند ندا درداد و چنین گفت: درخت را ببرید و شاخه هایش را قطع نمایید و برگهایش را بیفشانید و میوه هایش راپراکنده سازید تا حیوانات از زیرش و مرغان ازشاخه هایش آواره گردند.
\par 15 لیکن کنده ریشه هایش را با بند آهن و برنج در زمین در میان سبزه های صحرا واگذارید و از شبنم آسمان ترشود و نصیب او از علف زمین با حیوانات باشد.
\par 16 دل او از انسانیت تبدیل شود و دل حیوان را به او بدهند و هفت زمان براو بگذرد.
\par 17 این امر ازفرمان پاسبانان شده و این حکم از کلام مقدسین گردیده است تا زندگان بدانند که حضرت متعال بر ممالک آدمیان حکمرانی می‌کند و آن را بهرکه می‌خواهد می‌دهد و پست‌ترین مردمان را بر آن نصب می‌نماید.
\par 18 این خواب را من که نبوکدنصرپادشاه هستم دیدم و تو‌ای بلطشصر تعبیرش رابیان کن زیرا که تمامی حکیمان مملکتم نتوانستندمرا از تعبیرش اطلاع دهند، اما تو می‌توانی چونکه روح خدایان قدوس در تو می‌باشد.»
\par 19 آنگاه دانیال که به بلطشصر مسمی می‌باشد، ساعتی متحیر ماند و فکرهایش او رامضطرب ساخت. پس پادشاه متکلم شده، گفت: «ای بلطشصر خواب و تعبیرش تو را مضطرب نسازد.» بلطشصر در جواب گفت: «ای آقای من! خواب از برای دشمنانت و تعبیرش از برای خصمانت باشد.
\par 20 درختی که دیدی که بزرگ وقوی گردید و ارتفاعش تا به آسمان رسید ومنظرش به تمامی زمین
\par 21 و برگهایش جمیل و میوه‌اش بسیار و آذوقه برای همه در آن بود وحیوانات صحرا زیرش ساکن بودند و مرغان هوادر شاخه هایش ماوا گزیدند،
\par 22 ‌ای پادشاه آن درخت تو هستی زیرا که تو بزرگ و قوی گردیده‌ای و عظمت تو چنان افزوده شده است که به آسمان رسیده و سلطنت تو تا به اقصای زمین.
\par 23 و چونکه پادشاه پاسبانی و مقدسی را دید که ازآسمان نزول نموده، گفت: درخت را ببرید و آن راتلف سازید، لیکن کنده ریشه هایش را با بند آهن و برنج در زمین در میان سبزه های صحراواگذارید و از شبنم آسمان تر شود و نصیبش باحیوانات صحرا باشد تا هفت زمان بر آن بگذرد،
\par 24 ‌ای پادشاه تعبیر این است و فرمان حضرت متعال که بر آقایم پادشاه وارد شده است همین است،
\par 25 که تو را از میان مردمان خواهند راند ومسکن تو با حیوانات صحرا خواهد بود و تو رامثل گاوان علف خواهند خورانید و تو را از شبنم آسمان تر خواهند ساخت و هفت زمان بر توخواهد گذشت تا بدانی که حضرت متعال برممالک آدمیان حکمرانی می‌کند و آن را بهر‌که می‌خواهد عطا می‌فرماید.
\par 26 و چون گفتند که کنده ریشه های درخت را واگذارید، پس سلطنت تو برایت برقرار خواهد ماند بعد از آنکه دانسته باشی که آسمانها حکمرانی می‌کنند.
\par 27 لهذا‌ای پادشاه نصیحت من تو را پسند آید و گناهان خودرا به عدالت و خطایای خویش را به احسان نمودن بر فقیران فدیه بده که شاید باعث طول اطمینان تو باشد.»
\par 28 این همه بر نبوکدنصرپادشاه واقع شد.
\par 29 بعد از انقضای دوازده ماه او بالای قصرخسروی در بابل می‌خرامید.
\par 30 و پادشاه متکلم شده، گفت: «آیا این بابل عظیم نیست که من آن را برای خانه سلطنت به توانایی قوت و حشمت جلال خود بنا نموده‌ام؟»
\par 31 این سخن هنوز برزبان پادشاه بود که آوازی از آسمان نازل شده، گفت: «ای پادشاه نبوکدنصر به تو گفته می‌شود که سلطنت از تو گذشته است.
\par 32 و تو را از میان مردم خواهند راند و مسکن تو با حیوانات صحراخواهد بود و تو را مثل گاوان علف خواهندخورانید و هفت زمان بر تو خواهد گذشت تابدانی که حضرت متعال بر ممالک آدمیان حکمرانی می‌کند و آن را بهر‌که می‌خواهدمی دهد.»
\par 33 در همان ساعت این امر بر نبوکدنصر واقع شد و از میان مردمان رانده شده، مثل گاوان علف می‌خورد و بدنش از شبنم آسمان تر می‌شد تامویهایش مثل پرهای عقاب بلند شد وناخنهایش مثل چنگالهای مرغان گردید.
\par 34 و بعد از انقضای آن ایام من که نبوکدنصرهستم، چشمان خود را بسوی آسمان برافراشتم و عقل من به من برگشت و حضرت متعال رامتبارک خواندم و حی سرمدی را تسبیح و حمدگفتم زیرا که سلطنت او سلطنت جاودانی وملکوت او تا ابدالاباد است.
\par 35 و جمیع ساکنان جهان هیچ شمرده می‌شوند و با جنود آسمان وسکنه جهان بر وفق اراده خود عمل می‌نماید وکسی نیست که دست او را باز‌دارد یا او را بگویدکه چه می‌کنی.
\par 36 در همان زمان عقل من به من برگشت و به جهت جلال سلطنت من حشمت وزینتم به من باز داده شد و مشیرانم و امرایم مراطلبیدند و بر سلطنت خود استوار گردیدم وعظمت عظیمی بر من افزوده شد.الان من که نبوکدنصر هستم پادشاه آسمانها را تسبیح وتکبیر و حمد می‌گویم که تمام کارهای او حق و طریق های وی عدل است و کسانی که با تکبر راه می‌روند، او قادر است که ایشان را پست نماید.
\par 37 الان من که نبوکدنصر هستم پادشاه آسمانها را تسبیح وتکبیر و حمد می‌گویم که تمام کارهای او حق و طریق های وی عدل است و کسانی که با تکبر راه می‌روند، او قادر است که ایشان را پست نماید.

\chapter{5}

\par 1 بلشصر پادشاه ضیافت عظیمی برای هزارنفر از امرای خود برپا داشت و در حضورآن هزار نفر شراب نوشید.
\par 2 بلشصر در کیف شراب امر فرمود که ظروف طلا و نقره را که جدش نبوکدنصر از هیکل اورشلیم برده بودبیاورند تا پادشاه و امرایش و زوجه‌ها ومتعه هایش از آنها بنوشند.
\par 3 آنگاه ظروف طلا راکه از هیکل خانه خدا که در اورشلیم است گرفته شده بود آوردند و پادشاه و امرایش و زوجه‌ها ومتعه هایش از آنها نوشیدند.
\par 4 شراب می‌نوشیدندو خدایان طلا و نقره و برنج و آهن و چوب وسنگ را تسبیح می‌خواندند.
\par 5 در همان ساعت انگشتهای دست انسانی بیرون آمد و در برابر شمعدان بر گچ دیوار قصرپادشاه نوشت و پادشاه کف دست را که می‌نوشت دید.
\par 6 آنگاه هیئت پادشاه متغیر شد و فکرهایش او را مضطرب ساخت و بندهای کمرش سست شده، زانوهایش بهم می‌خورد.
\par 7 پادشاه به آوازبلند صدا زد که جادوگران و کلدانیان و منجمان رااحضار نمایند. پس پادشاه حکیمان بابل راخطاب کرده، گفت: «هر‌که این نوشته را بخواند وتفسیرش را برای من بیان نماید به ارغوان ملبس خواهد شد و طوق زرین بر گردنش (نهاده خواهدشد) و حاکم سوم در مملکت خواهد بود.»
\par 8 آنگاه جمیع حکمای پادشاه داخل شدند امانتوانستند نوشته را بخوانند یا تفسیرش را برای پادشاه بیان نمایند.
\par 9 پس بلشصر پادشاه، بسیارمضطرب شد و هیئتش در او متغیر گردید و امرایش مضطرب شدند.
\par 10 اما ملکه به‌سبب سخنان پادشاه و امرایش به مهمانخانه درآمد وملکه متکلم شده، گفت: «ای پادشاه تا به ابد زنده باش! فکرهایت تو را مضطرب نسازد و هیئت تومتغیر نشود.
\par 11 شخصی در مملکت تو هست که روح خدایان قدوس دارد و در ایام پدرت روشنایی و فطانت و حکمت مثل حکمت خدایان دراو پیدا شد و پدرت نبوکدنصر پادشاه یعنی پدر تو‌ای پادشاه او را رئیس مجوسیان وجادوگران و کلدانیان و منجمان ساخت.
\par 12 چونکه روح فاضل و معرفت و فطانت و تعبیرخوابها و حل معماها و گشودن عقده‌ها در این دانیال که پادشاه او را به بلطشصر مسمی نمودیافت شد. پس حال دانیال طلبیده شود و تفسیر رابیان خواهد نمود.»
\par 13 آنگاه دانیال را به حضور پادشاه آوردند وپادشاه دانیال را خطاب کرده، فرمود: «آیا توهمان دانیال از اسیران یهود هستی که پدرم پادشاه از یهودا آورد؟
\par 14 و درباره تو شنیده‌ام که روح خدایان در تو است و روشنایی و فطانت وحکمت فاضل در تو پیدا شده است.
\par 15 و الان حکیمان و منجمان را به حضور من آوردند تا این نوشته را بخوانند و تفسیرش را برای من بیان کننداما نتوانستند تفسیر کلام را بیان کنند.
\par 16 و من درباره تو شنیده‌ام که به نمودن تعبیرها و گشودن عقده‌ها قادر می‌باشی. پس اگر بتوانی الان نوشته را بخوانی و تفسیرش را برای من بیان کنی به ارغوان ملبس خواهی شد و طوق زرین بر گردنت (نهاده خواهد شد) و در مملکت حاکم سوم خواهی بود.»
\par 17 پس دانیال به حضور پادشاه جواب داد وگفت: «عطایای تو از آن تو باشد و انعام خود را به دیگری بده، لکن نوشته را برای پادشاه خواهم خواند و تفسیرش را برای او بیان خواهم نمود.
\par 18 اما تو‌ای پادشاه، خدای تعالی به پدرت نبوکدنصر سلطنت و عظمت و جلال و حشمت عطا فرمود.
\par 19 و به‌سبب عظمتی که به او داده بودجمیع قومها و امت‌ها و زبانها از او لرزان و ترسان می‌بودند. هر‌که را می‌خواست می‌کشت و هر‌که را می‌خواست زنده نگاه می‌داشت و هر‌که رامی خواست بلند می‌نمود و هر‌که را می‌خواست پست می‌ساخت.
\par 20 لیکن چون دلش مغرور وروحش سخت گردیده، تکبر نمود آنگاه ازکرسی سلطنت خویش به زیر افکنده شد وحشمت او را از او گرفتند.
\par 21 و از میان بنی آدم رانده شده، دلش مثل دل حیوانات گردید ومسکنش با گورخران شده، او را مثل گاوان علف می‌خورانیدند و جسدش از شبنم آسمان ترمی شد تا فهمید که خدای تعالی بر ممالک آدمیان حکمرانی می‌کند و هر‌که را می‌خواهد بر آن نصب می‌نماید.
\par 22 و تو‌ای پسرش بلشصر! اگرچه این همه را دانستی لکن دل خود را متواضع ننمودی.
\par 23 بلکه خویشتن را به ضد خداوندآسمانها بلند ساختی و ظروف خانه او را به حضور تو آوردند و تو و امرایت و زوجه‌ها ومتعه هایت از آنها شراب نوشیدید و خدایان نقره و طلا و برنج و آهن و چوب و سنگ را که نمی بینند و نمی شنوند و (هیچ ) نمی دانند تسبیح خواندی، اما آن خدایی را که روانت در دست او وتمامی راههایت از او می‌باشد، تمجید ننمودی.
\par 24 پس این کف دست از جانب او فرستاده شد واین نوشته مکتوب گردید.
\par 25 و این نوشته‌ای که مکتوب شده است این است: منامنا ثقیل و فرسین.
\par 26 و تفسیر کلام این است: منا؛ خدا سلطنت تو را شمرده و آن را به انتها رسانیده است.
\par 27 ثقیل؛ درمیزان سنجیده شده و ناقص درآمده‌ای.
\par 28 فرس؛ سلطنت تو تقسیم گشته و به مادیان و فارسیان بخشیده شده است.»
\par 29 آنگاه بلشصر امر فرمود تا دانیال را به ارغوان ملبس ساختند و طوق زرین بر گردنش (نهادند) و درباره‌اش ندا کردند که در مملکت حاکم سوم می‌باشد.در همان شب بلشصرپادشاه کلدانیان کشته شد.
\par 30 در همان شب بلشصرپادشاه کلدانیان کشته شد.

\chapter{6}

\par 1 و داریوش مادی در حالی که شصت و دوساله بود سلطنت را یافت.
\par 2 و داریوش مصلحت دانست که صد و بیست والی بر مملکت نصب نماید تا بر تمامی مملکت باشند.
\par 3 و بر آنهاسه وزیر که یکی از ایشان دانیال بود تا آن والیان به ایشان حساب دهند و هیچ ضرری به پادشاه نرسد.
\par 4 پس این دانیال بر سایر وزراء و والیان تفوق جست زیرا که روح فاضل دراو بود وپادشاه اراده داشت که او را بر تمامی مملکت نصب نماید.
\par 5 پس وزیران و والیان بهانه می‌جستند تا شکایتی در امور سلطنت بر دانیال بیاورند اما نتوانستند که هیچ علتی یا تقصیری بیابند، چونکه او امین بود و خطایی یا تقصیری دراو هرگز یافت نشد.
\par 6 پس آن اشخاص گفتند که «در این دانیال هیچ علتی پیدا نخواهیم کرد مگراینکه آن را درباره شریعت خدایش در او بیابیم.»
\par 7 آنگاه این وزراء و والیان نزد پادشاه جمع شدند و او را چنین گفتند: «ای داریوش پادشاه تابه ابد زنده باش.
\par 8 جمیع وزرای مملکت و روسا و والیان و مشیران و حاکمان با هم مشورت کرده‌اند که پادشاه حکمی استوار کند و قدغن بلیغی نماید که هر کسی‌که تا سی روز از خدایی یا انسانی سوای تو‌ای پادشاه مسالتی نماید درچاه شیران افکنده شود.
\par 9 پس‌ای پادشاه فرمان رااستوار کن و نوشته را امضا فرما تا موافق شریعت مادیان و فارسیان که منسوخ نمی شود تبدیل نگردد.»
\par 10 بنابراین داریوش پادشاه نوشته وفرمان را امضا نمود.
\par 11 اما چون دانیال دانست که نوشته امضا شده است به خانه خود درآمد و پنجره های بالاخانه خود را به سمت اورشلیم باز نموده، هر روز سه مرتبه زانو می‌زد و دعا می‌نمود و چنانکه قبل ازآن عادت می‌داشت نزد خدای خویش دعامی کرد و تسبیح می‌خواند.
\par 12 پس آن اشخاص جمع شده، دانیال را یافتند که نزد خدای خودمسالت و تضرع می‌نماید.
\par 13 آنگاه به حضور پادشاه نزدیک شده، درباره فرمان پادشاه عرض کردند که «ای پادشاه آیافرمانی امضا ننمودی که هر‌که تا سی روز نزدخدایی یا انسانی سوای تو‌ای پادشاه مسالتی نماید در چاه شیران افکنده شود؟» پادشاه درجواب گفت: «این امر موافق شریعت مادیان وفارسیان که منسوخ نمی شود صحیح است.»
\par 14 پس ایشان در حضور پادشاه جواب دادند وگفتند که «این دانیال که از اسیران یهودا می‌باشد به تو‌ای پادشاه و به فرمانی که امضا نموده‌ای اعتنانمی نماید، بلکه هر روز سه مرتبه مسالت خود رامی نماید.»
\par 15 آنگاه پادشاه چون این سخن راشنید بر خویشتن بسیار خشمگین گردید و دل خود را به رهانیدن دانیال مشغول ساخت و تاغروب آفتاب برای استخلاص او سعی می‌نمود.
\par 16 آنگاه آن اشخاص نزد پادشاه جمع شدند و به پادشاه عرض کردند که «ای پادشاه بدان که قانون مادیان و فارسیان این است که هیچ فرمان یاحکمی که پادشاه آن را استوار نماید تبدیل نشود.»
\par 17 پس پادشاه امر فرمود تا دانیال را بیاورند واو را در چاه شیران بیندازند و پادشاه دانیال راخطاب کرده، گفت: «خدای تو که او را پیوسته عبادت می‌نمایی تو را رهایی خواهد داد.»
\par 18 وسنگی آورده، آن را بر دهنه چاه نهادند و پادشاه آن را به مهر خود و مهر امرای خویش مختوم ساخت تا امر درباره دانیال تبدیل نشود.
\par 19 آنگاه پادشاه به قصر خویش رفته، شب را به روزه بسربرد و به حضور وی اسباب عیش او را نیاوردند وخوابش از او رفت.
\par 20 پس پادشاه صبح زود وقت طلوع فجر برخاست و به تعجیل به چاه شیران رفت.
\par 21 و چون نزد چاه شیران رسید به آوازحزین دانیال را صدا زد و پادشاه دانیال را خطاب کرده، گفت: «ای دانیال بنده خدای حی آیاخدایت که او را پیوسته عبادت می‌نمایی به رهانیدنت از شیران قادر بوده است؟»
\par 22 آنگاه دانیال به پادشاه جواب داد که «ای پادشاه تا به ابد زنده باش!
\par 23 خدای من فرشته خود را فرستاده، دهان شیران را بست تا به من ضرری نرسانند چونکه به حضور وی در من گناهی یافت نشد و هم درحضور تو‌ای پادشاه تقصیری نورزیده بودم.»
\par 24 آنگاه پادشاه بی‌نهایت شادمان شده، امر فرمود که دانیال را از چاه برآورند و دانیال را از چاه برآوردند و از آن جهت که بر خدای خود توکل نموده بود در اوهیچ ضرری یافت نشد.
\par 25 و پادشاه امر فرمود تاآن اشخاص را که بر دانیال شکایت آورده بودندحاضر ساختند و ایشان را با پسران و زنان ایشان در چاه شیران انداختند و هنوز به ته چاه نرسیده بودند که شیران بر ایشان حمله آورده، همه استخوانهای ایشان را خرد کردند.
\par 26 بعد از آن داریوش پادشاه به جمیع قومها و امت‌ها وزبانهایی که در تمامی جهان ساکن بودند نوشت که «سلامتی شما افزون باد!
\par 27 از حضور من فرمانی صادر شده است که در هر سلطنتی ازممالک من (مردمان ) به حضور خدای دانیال لرزان و ترسان باشند زیرا که او خدای حی و تاابدالاباد قیوم است. و ملکوت او بی‌زوال وسلطنت او غیرمتناهی است.او است که نجات می‌دهد و می‌رهاند و آیات و عجایب را درآسمان و در زمین ظاهر می‌سازد و اوست که دانیال را از چنگ شیران رهایی داده است.» پس این دانیال در سلطنت داریوش و درسلطنت کورش فارسی فیروز می‌بود.
\par 28 او است که نجات می‌دهد و می‌رهاند و آیات و عجایب را درآسمان و در زمین ظاهر می‌سازد و اوست که دانیال را از چنگ شیران رهایی داده است.» پس این دانیال در سلطنت داریوش و درسلطنت کورش فارسی فیروز می‌بود.

\chapter{7}

\par 1 در سال اول بلشصر پادشاه بابل، دانیال دربسترش خوابی و رویاهای سرش را دید. پس خواب را نوشت و کلیه مطالب را بیان نمود.
\par 2 پس دانیال متکلم شده، گفت: «شبگاهان در عالم رویا شده، دیدم که ناگاه چهار باد آسمان بر روی دریای عظیم تاختند.
\par 3 و چهار وحش بزرگ که مخالف یکدیگر بودند از دریا بیرون آمدند.
\par 4 اول آنها مثل شیر بود و بالهای عقاب داشت و من نظرکردم تا بالهایش کنده گردید و او از زمین برداشته شده، بر پایهای خود مثل انسان قرار داده گشت ودل انسان به او داده شد.
\par 5 و اینک وحش دوم دیگر مثل خرس بود و بر یک طرف خود بلند شدو در دهانش در میان دندانهایش سه دنده بود ووی را چنین گفتند: برخیز و گوشت بسیار بخور.
\par 6 بعد از آن نگریستم و اینک دیگری مثل پلنگ بود که بر پشتش چهار بال مرغ داشت و این وحش چهار سر داشت و سلطنت به او داده شد.
\par 7 بعد ازآن در رویاهای شب نظر کردم و اینک وحش چهارم که هولناک و مهیب و بسیار زورآور بود ودندانهای بزرگ آهنین داشت و باقی‌مانده رامی خورد و پاره پاره می‌کرد و به پایهای خویش پایمال می‌نمود و مخالف همه وحوشی که قبل ازاو بودند بود و ده شاخ داشت.
\par 8 پس در این شاخهاتامل می‌نمودم که اینک از میان آنها شاخ کوچک دیگری برآمد و پیش رویش سه شاخ از آن شاخهای اول از ریشه‌کنده شد و اینک این شاخ چشمانی مانند چشم انسان و دهانی که به سخنان تکبرآمیز متکلم بود داشت.
\par 9 «و نظر می‌کردم تا کرسیها برقرار شد وقدیم الایام جلوس فرمود و لباس او مثل برف سفید و موی سرش مثل پشم پاک و عرش اوشعله های آتش و چرخهای آن آتش ملتهب بود.
\par 10 نهری از آتش جاری شده، از پیش روی اوبیرون آمد. هزاران هزار او را خدمت می‌کردند وکرورها کرور به حضور وی ایستاده بودند. دیوان برپا شد و دفترها گشوده گردید.
\par 11 آنگاه نظر کردم به‌سبب سخنان تکبرآمیزی که آن شاخ می‌گفت. پس نگریستم تا آن وحش کشته شد وجسد او هلاک گردیده، به آتش مشتعل تسلیم شد.
\par 12 اما سایر وحوش سلطنت را از ایشان گرفتند، لکن درازی عمر تا زمانی و وقتی به ایشان داده شد.
\par 13 و در رویای شب نگریستم و اینک مثل پسر انسان با ابرهای آسمان آمد و نزدقدیم الایام رسید و او را به حضور وی آوردند.
\par 14 و سلطنت و جلال و ملکوت به او داده شد تاجمیع قوم‌ها و امت‌ها و زبانها او را خدمت نمایند. سلطنت او سلطنت جاودانی و بی‌زوال است و ملکوت او زایل نخواهد شد.
\par 15 «اما روح من دانیال در جسدم مدهوش شدو رویاهای سرم مرا مضطرب ساخت.
\par 16 و به یکی از حاضرین نزدیک شده، حقیقت این همه امور را از وی پرسیدم و او به من تکلم نموده، تفسیر امور را برای من بیان کرد،
\par 17 که این وحوش عظیمی که (عدد) ایشان چهار است چهار پادشاه می‌باشند که از زمین خواهند برخاست.
\par 18 امامقدسان حضرت اعلی سلطنت را خواهند یافت و مملکت را تا به ابد و تا ابدالاباد متصرف خواهند بود.
\par 19 آنگاه آرزو داشتم که حقیقت امررا درباره وحش چهارم که مخالف همه دیگران وبسیار هولناک بود و دندانهای آهنین و چنگالهای برنجین داشت و سایرین را می‌خورد و پاره پاره می‌کرد و به پایهای خود پایمال می‌نمود بدانم.
\par 20 و کیفیت ده شاخ را که بر سر او بود و آن دیگری را که برآمد و پیش روی او سه شاخ افتادیعنی آن شاخی که چشمان و دهانی را که سخنان تکبرآمیز می‌گفت داشت و نمایش او از رفقایش سختتر بود.
\par 21 پس ملاحظه کردم و این شاخ بامقدسان جنگ کرده، بر ایشان استیلا یافت.
\par 22 تاحینی که قدیم الایام آمد و داوری به مقدسان حضرت اعلی تسلیم شد و زمانی رسید که مقدسان ملکوت را به تصرف آوردند.
\par 23 پس اوچنین گفت: وحش چهارم سلطنت چهارمین برزمین خواهد بود و مخالف همه سلطنتها خواهدبود و تمامی جهان را خواهد خورد و آن راپایمال نموده، پاره پاره خواهد کرد.
\par 24 و ده شاخ از این مملکت، ده پادشاه می‌باشند که خواهندبرخاست و دیگری بعد از ایشان خواهدبرخاست و او مخالف اولین خواهد بود و سه پادشاه را به زیر خواهد افکند.
\par 25 و سخنان به ضدحضرت اعلی خواهد گفت و مقدسان حضرت اعلی را ذلیل خواهد ساخت و قصد تبدیل نمودن زمانها و شرایع خواهد نمود و ایشان تا زمانی ودو زمان و نصف زمان به‌دست او تسلیم خواهندشد.
\par 26 پس دیوان برپا خواهد شد و سلطنت او رااز او گرفته، آن را تا به انتها تباه و تلف خواهندنمود.
\par 27 و ملکوت و سلطنت و حشمت مملکتی که زیر تمامی آسمانهاست به قوم مقدسان حضرت اعلی داده خواهد شد که ملکوت اوملکوت جاودانی است و جمیع ممالک او راعبادت و اطاعت خواهند نمود.انتهای امر تابه اینجا است. فکرهای من دانیال مرا بسیارمضطرب نمود و هیئتم در من متغیر گشت، لیکن این امر را در دل خود نگاه داشتم.»
\par 28 انتهای امر تابه اینجا است. فکرهای من دانیال مرا بسیارمضطرب نمود و هیئتم در من متغیر گشت، لیکن این امر را در دل خود نگاه داشتم.»

\chapter{8}

\par 1 در سال سوم سلطنت بلشصر پادشاه، رویایی بر من دانیال ظاهر شد بعد از آنکه اول به من ظاهر شده بود.
\par 2 و در رویا نظر کردم ومی دیدم که من در دارالسلطنه شوشن که درولایت عیلام می‌باشد بودم و در عالم رویا دیدم که نزد نهر اولای می‌باشم.
\par 3 پس چشمان خود رابرافراشته، دیدم که ناگاه قوچی نزد نهر ایستاده بود که دو شاخ داشت و شاخهایش بلند بود ویکی از دیگری بلندتر و بلندترین آنها آخربرآمد.
\par 4 و قوچ را دیدم که به سمت مغرب وشمال و جنوب شاخ می‌زد و هیچ وحشی با اومقاومت نتوانست کرد و کسی نبود که از دستش رهایی دهد و برحسب رای خود عمل نموده، بزرگ می‌شد.
\par 5 و حینی که متفکر می‌بودم اینک بز نری ازطرف مغرب بر روی تمامی زمین می‌آمد و زمین را لمس نمی کرد و در میان چشمان بز نر شاخی معتبر بود.
\par 6 و به سوی آن قوچ صاحب دو شاخ که آن را نزد نهر ایستاده دیدم آمد و بشدت قوت خویش نزد او دوید.
\par 7 و او را دیدم که چون نزدقوچ رسید با او بشدت غضبناک شده، قوچ را زد وهر دو شاخ او را شکست و قوچ را یارای مقاومت با وی نبود پس وی را به زمین انداخته، پایمال کردو کسی نبود که قوچ را از دستش رهایی دهد.
\par 8 وبز نر بی‌نهایت بزرگ شد و چون قوی گشت آن شاخ بزرگ شکسته شد و در جایش چهار شاخ معتبر بسوی بادهای اربعه آسمان برآمد.
\par 9 و ازیکی از آنها یک شاخ کوچک برآمد و به سمت جنوب و مشرق و فخر زمینها بسیار بزرگ شد.
\par 10 و به ضد لشکر آسمانها قوی شده، بعضی ازلشکر و ستارگان را به زمین انداخته، پایمال نمود.
\par 11 و به ضد سردار لشکر بزرگ شد و قربانی دایمی از او گرفته شد و مکان مقدس او منهدم گردید.
\par 12 و لشکری به ضد قربانی دایمی، به‌سبب عصیان (قوم به وی ) داده شد و آن (لشکر)راستی را به زمین انداختند و او (موافق رای خود) عمل نموده، کامیاب گردید.
\par 13 و مقدسی را شنیدم که سخن می‌گفت و مقدس دیگری از آن یک که سخن می‌گفت، پرسید که رویا درباره قربانی دایمی و معصیت مهلک که قدس و لشکررا به پایمال شدن تسلیم می‌کند تا بکی خواهدبود.
\par 14 و او به من گفت: «تا دو هزار و سیصد شام وصبح، آنگاه مقدس تطهیر خواهد شد.»
\par 15 و چون من دانیال رویا را دیدم و معنی آن راطلبیدم، ناگاه شبیه مردی نزد من بایستاد.
\par 16 وآواز آدمی را از میان (نهر) اولای شنیدم که نداکرده، می‌گفت: «ای جبرائیل این مرد را از معنی‌این‌رویا مطلع ساز.»
\par 17 پس او نزد جایی که ایستاده بودم آمد و چون آمد من ترسان شده، به روی خود درافتادم و او مرا گفت: «ای پسر انسان بدانکه این‌رویا برای زمان آخر می‌باشد.»
\par 18 و حینی که او با من سخن می‌گفت، من برروی خود بر زمین در خواب سنگین می‌بودم و اومرا لمس نموده، در جایی که بودم برپا داشت.
\par 19 و گفت: «اینک من تو را از آنچه در آخر غضب واقع خواهد شد اطلاع می‌دهم زیرا که انتها درزمان معین واقع خواهد شد.
\par 20 اما آن قوچ صاحب دو شاخ که آن را دیدی پادشاهان مادیان و فارسیان می‌باشد.
\par 21 و آن بز نر ستبر پادشاه یونان می‌باشد و آن شاخ بزرگی که در میان دوچشمش بود پادشاه اول است.
\par 22 و اما آن شکسته شدن و چهار در جایش برآمدن، چهارسلطنت از قوم او اما نه از قوت او برپا خواهندشد.
\par 23 و در آخر سلطنت ایشان چون گناه عاصیان به اتمام رسیده باشد، آنگاه پادشاهی سخت روی و در مکرها ماهر، خواهد برخاست.
\par 24 و قوت او عظیم خواهد شد، لیکن نه از توانایی خودش. و خرابیهای عجیب خواهد نمود وکامیاب شده، (موافق رای خود) عمل خواهدنمود و عظما و قوم مقدسان را هلاک خواهدنمود.
\par 25 و از مهارت او مکر در دستش پیش خواهد رفت و در دل خود مغرور شده، بسیاری را بغته هلاک خواهد ساخت و با امیر امیران مقاومت خواهد نمود، اما بدون دست، شکسته خواهد شد.
\par 26 پس رویایی که درباره شام و صبح گفته شد یقین است اما تو رویا را بر هم نه زیرا که بعد از ایام بسیار واقع خواهد شد.»آنگاه من دانیال تا اندک زمانی ضعیف وبیمار شدم. پس برخاسته، به‌کارهای پادشاه مشغول گردیدم، اما درباره رویا متحیر ماندم واحدی معنی آن را نفهمید.
\par 27 آنگاه من دانیال تا اندک زمانی ضعیف وبیمار شدم. پس برخاسته، به‌کارهای پادشاه مشغول گردیدم، اما درباره رویا متحیر ماندم واحدی معنی آن را نفهمید.

\chapter{9}

\par 1 در سال اول داریوش بن اخشورش که ازنسل مادیان و بر مملکت کلدانیان پادشاه شده بود.
\par 2 در سال اول سلطنت او، من دانیال، عدد سالهایی را که کلام خداوند درباره آنها به ارمیای نبی نازل شده بود از کتب فهمیدم که هفتادسال در خرابی اورشلیم تمام خواهد شد.
\par 3 پس روی خود را بسوی خداوند خدا متوجه ساختم تا با دعا و تضرعات و روزه و پلاس و خاکسترمسالت نمایم؛
\par 4 و نزد یهوه خدای خود دعا کردم و اعتراف نموده، گفتم: «ای خداوند خدای عظیم و مهیب که عهد و رحمت را با محبان خویش وآنانی که فرایض تو را حفظ می‌نمایند نگاه می‌داری!
\par 5 ما گناه و عصیان و شرارت ورزیده وتمرد نموده و از اوامر و احکام تو تجاوز کرده‌ایم.
\par 6 و به بندگانت انبیایی که به اسم تو به پادشاهان وسروران و پدران ما و به تمامی قوم زمین سخن‌گفتند گوش نگرفته‌ایم.
\par 7 ‌ای خداوند عدالت ازآن تو است و رسوایی از آن ما است. چنانکه امروز شده است از مردان یهودا و ساکنان اورشلیم و همه اسرائیلیان چه نزدیک و چه دوردر همه زمینهایی که ایشان را به‌سبب خیانتی که به تو ورزیده‌اند در آنها پراکنده ساخته‌ای.
\par 8 ‌ای خداوند رسوایی از آن ما و پادشاهان و سروران وپدران ما است زیرا که به تو گناه ورزیده‌ایم.
\par 9 خداوند خدای ما را رحمتها و مغفرتها است هرچند بدو گناه ورزیده‌ایم.
\par 10 و کلام یهوه خدای خود را نشنیده‌ایم تا در شریعت او که به وسیله بندگانش انبیا پیش ما گذارد سلوک نماییم.
\par 11 و تمامی اسرائیل از شریعت توتجاوز نموده و روگردان شده، به آواز تو گوش نگرفته‌اند بنابراین لعنت و سوگندی که در تورات موسی بنده خدا مکتوب است بر ما مستولی گردیده، چونکه به او گناه ورزیده‌ایم.
\par 12 و اوکلام خود را که به ضد ما و به ضد داوران ما که برما داوری می‌نمودند گفته بود استوار نموده وبلای عظیمی بر ما وارد آورده است زیرا که زیر تمامی آسمان حادثه‌ای واقع نشده، مثل آنکه بر اورشلیم واقع شده است.
\par 13 تمامی این بلا بر وفق آنچه در تورات موسی مکتوب است برما وارد شده است، معهذا نزد یهوه خدای خود مسالت ننمودیم تا از معصیت خودبازگشت نموده، راستی تو را بفهمیم.
\par 14 بنابراین خداوند بر این بلا مراقب بوده، آن را بر ما واردآورد زیرا که یهوه خدای ما در همه کارهایی که می‌کند عادل است اما ما به آواز او گوش نگرفتیم.
\par 15 «پس الان‌ای خداوند خدای ما که قوم خود را به‌دست قوی از زمین مصر بیرون آورده، اسمی برای خود پیدا کرده‌ای، چنانکه امروزشده است، ما گناه ورزیده و شرارت نموده‌ایم.
\par 16 ‌ای خداوند مسالت آنکه برحسب تمامی عدالت خود خشم و غضب خویش را از شهرخود اورشلیم و از کوه مقدس خود برگردانی زیرابه‌سبب گناهان ما و معصیتهای پدران ما اورشلیم و قوم تو نزد همه مجاوران ما رسوا شده است.
\par 17 پس حال‌ای خدای ما دعا و تضرعات بنده خود را اجابت فرما و روی خود را بر مقدس خویش که خراب شده است به‌خاطر خداوندیت متجلی فرما.
\par 18 ‌ای خدایم گوش خود را فراگیر وبشنو و چشمان خود را باز کن و به خرابیهای ما وشهری که به اسم تو مسمی است نظر فرما، زیراکه ما تضرعات خود را نه برای عدالت خویش بلکه برای رحمتهای عظیم تو به حضور تومی نماییم.
\par 19 ‌ای خداوند بشنو! ای خداوندبیامرز! ای خداوند استماع نموده، به عمل آور! ای خدای من به‌خاطر خودت تاخیر منمازیرا که شهر تو و قوم تو به اسم تو مسمی می‌باشند.»
\par 20 و چون من هنوز سخن می‌گفتم و دعامی نمودم و به گناهان خود و گناهان قوم خویش اسرائیل اعتراف می‌کردم و تضرعات خود رابرای کوه مقدس خدایم به حضور یهوه خدای خویش معروض می‌داشتم،
\par 21 چون هنوز در دعامتکلم می‌بودم، آن مرد جبرائیل که او را دررویای اول دیده بودم بسرعت پرواز نموده، به وقت هدیه شام نزد من رسید،
\par 22 و مرا اعلام نمودو با من متکلم شده، گفت: «ای دانیال الان من بیرون آمده‌ام تا تو را فطانت و فهم بخشم.
\par 23 درابتدای تضرعات تو امر صادر گردید و من آمدم تاتو را خبر دهم زیرا که تو بسیار محبوب هستی، پس در این کلام تامل کن و رویا را فهم نما.
\par 24 هفتاد هفته برای قوم تو و برای شهر مقدست مقرر می‌باشد تا تقصیرهای آنها تمام شود وگناهان آنها به انجام رسد و کفاره به جهت عصیان کرده شود و عدالت جاودانی آورده شود و رویا ونبوت مختوم گردد و قدس‌الاقداس مسح شود.
\par 25 پس بدان و بفهم که از صدور فرمان به جهت تعمیر نمودن و بناکردن اورشلیم تا (ظهور) مسیح رئیس، هفت هفته و شصت و دو هفته خواهد بودو (اورشلیم ) با کوچه‌ها و حصار در زمانهای تنگی تعمیر و بنا خواهد شد.
\par 26 و بعد از آن شصت و دو هفته، مسیح منقطع خواهد گردید واز آن او نخواهد بود، بلکه قوم آن رئیس که می‌آید شهر و قدس را خراب خواهند ساخت وآخر او در آن سیلاب خواهد بود و تا آخر جنگ خرابیها معین است.و او با اشخاص بسیار دریک هفته عهد را استوار خواهد ساخت و در نصف آن هفته قربانی و هدیه را موقوف خواهدکرد و بر کنگره رجاسات خراب کننده‌ای خواهدآمد والی النهایت آنچه مقدر است بر خراب کننده ریخته خواهد شد.»
\par 27 و او با اشخاص بسیار دریک هفته عهد را استوار خواهد ساخت و در نصف آن هفته قربانی و هدیه را موقوف خواهدکرد و بر کنگره رجاسات خراب کننده‌ای خواهدآمد والی النهایت آنچه مقدر است بر خراب کننده ریخته خواهد شد.»

\chapter{10}

\par 1 در سال سوم کورش پادشاه فارس، امری بر دانیال که به بلطشصر مسمی بود کشف گردید و آن امر صحیح و مشقت عظیمی بود. پس امر را فهمید و رویا را دانست.
\par 2 در آن ایام من دانیال سه هفته تمام ماتم گرفتم.
\par 3 خوراک لذیذ نخوردم و گوشت و شراب به دهانم داخل نشد و تا انقضای آن سه هفته خویشتن را تدهین ننمودم.
\par 4 و در روز بیست وچهارم ماه اول من بر کنار نهر عظیم یعنی دجله بودم.
\par 5 و چشمان خود را برافراشته دیدم که ناگاه مردی ملبس به کتان که کمربندی از طلای او فاز بر کمر خود داشت،
\par 6 و جسد او مثل زبرجد وروی وی مانند برق و چشمانش مثل شعله های آتش و بازوها و پایهایش مانند رنگ برنج صیقلی و آواز کلام او مثل صدای گروه عظیمی بود.
\par 7 و من دانیال تنها آن رویا را دیدم و کسانی که همراه من بودند رویا را ندیدند لیکن لرزش عظیمی بر ایشان مستولی شد و فرار کرده، خودرا پنهان کردند.
\par 8 و من تنها ماندم و آن رویای عظیم را مشاهده می‌نمودم و قوت در من باقی نماند و خرمی من به پژمردگی مبدل گردید ودیگر هیچ طاقت نداشتم.
\par 9 اما آواز سخنانش راشنیدم و چون آواز کلام او را شنیدم به روی خودبر زمین افتاده، بیهوش گردیدم.
\par 10 که ناگاه دستی مرا لمس نمود و مرا بر دو زانو و کف دستهایم برخیزانید.
\par 11 و او مرا گفت: «ای دانیال مرد بسیارمحبوب! کلامی را که من به تو می‌گویم فهم کن وبر پایهای خود بایست زیرا که الان نزد تو فرستاده شده‌ام.» و چون این کلام را به من گفت لرزان بایستادم.
\par 12 و مرا گفت: «ای دانیال مترس زیرا از روزاول که دل خود را بر آن نهادی که بفهمی و به حضور خدای خود تواضع نمایی سخنان تومستجاب گردید و من به‌سبب سخنانت آمده‌ام.
\par 13 اما رئیس مملکت فارس بیست و یک روز بامن مقاومت نمود و میکائیل که یکی از روسای اولین است به اعانت من آمد و من در آنجا نزدپادشاهان فارس ماندم.
\par 14 و من آمدم تا تو را ازآنچه در ایام آخر بر قوم تو واقع خواهد شداطلاع دهم زیرا که این‌رویا برای ایام طویل است.»
\par 15 و چون اینگونه سخنان را به من گفته بود به روی خود بر زمین افتاده، گنگ شدم.
\par 16 که ناگاه کسی به شبیه بنی آدم لبهایم را لمس نمود و من دهان خود را گشوده، متکلم شدم و به آن کسی‌که پیش من ایستاده بود گفتم: «ای آقایم از این‌رویادرد شدیدی مرا در‌گرفته است و دیگر هیچ قوت نداشتم.
\par 17 پس چگونه بنده آقایم بتواند با آقایم گفتگو نماید و حال آنکه از آن وقت هیچ قوت درمن برقرار نبوده، بلکه نفس هم در من باقی نمانده است.»
\par 18 پس شبیه انسانی بار دیگر مرا لمس نموده، تقویت داد،
\par 19 و گفت: «ای مرد بسیارمحبوب مترس! سلام بر تو باد و دلیر و قوی باش!» چون این را به من گفت تقویت یافتم و گفتم: «ای آقایم بگو زیرا که مرا قوت دادی.»
\par 20 پس گفت: «آیا می‌دانی که سبب آمدن من نزد توچیست؟ و الان برمی گردم تا با رئیس فارس جنگ نمایم و به مجرد بیرون رفتنم اینک رئیس یونان خواهد آمد.لیکن تو را از آنچه در کتاب حق مرقوم است اطلاع خواهم داد و کسی غیر ازرئیس شما میکائیل نیست که مرا به ضد اینها مددکند.
\par 21 لیکن تو را از آنچه در کتاب حق مرقوم است اطلاع خواهم داد و کسی غیر ازرئیس شما میکائیل نیست که مرا به ضد اینها مددکند.

\chapter{11}

\par 1 ایستاده بودم تا او را استوار سازم وقوت دهم.
\par 2 «و الان تو را به راستی اعلام می‌نمایم. اینک سه پادشاه بعد از این در فارس خواهند برخاست و چهارمین از همه دولتمندتر خواهد بود و چون به‌سبب توانگری خویش قوی گردد همه را به ضد مملکت یونان برخواهد انگیخت.
\par 3 وپادشاهی جبار خواهد برخاست و بر مملکت عظیمی سلطنت خواهد نمود و برحسب اراده خود عمل خواهد کرد.
\par 4 و چون برخیزد سلطنت او شکسته خواهد شد و بسوی بادهای اربعه آسمان تقسیم خواهد گردید. اما نه به ذریت او ونه موافق استقلالی که او می‌داشت، زیرا که سلطنت او از ریشه‌کنده شده و به دیگران غیر ازایشان داده خواهد شد.
\par 5 و پادشاه جنوب با یکی از سرداران خود قوی شده، بر او غلبه خواهدیافت و سلطنت خواهد نمود و سلطنت اوسلطنت عظیمی خواهد بود.
\par 6 و بعد از انقضای سالها ایشان همداستان خواهند شد و دختر پادشاه جنوب نزد پادشاه شمال آمده، با اومصالحه خواهد نمود. لیکن قوت بازوی خود رانگاه نخواهد داشت و او و بازویش برقرار نخواهدماند و آن دختر و آنانی که او را خواهند‌آورد وپدرش و آنکه او را تقویت خواهد نمود در آن زمان تسلیم خواهند شد.
\par 7 «و کسی از رمونه های ریشه هایش در جای او خواهد برخاست و با لشکری آمده، به قلعه پادشاه شمال داخل خواهد شد و با ایشان (جنگ ) نموده، غلبه خواهد یافت.
\par 8 و خدایان وبتهای ریخته شده ایشان را نیز با ظروف گرانبهای ایشان از طلا و نقره به مصر به اسیری خواهد برد وسالهایی چند از پادشاه شمال دست خواهدبرداشت.
\par 9 و به مملکت پادشاه جنوب داخل شده، باز به ولایت خود مراجعت خواهد نمود.
\par 10 و پسرانش محاربه خواهند نمود و گروهی ازلشکرهای عظیم را جمع خواهند کرد و ایشان داخل شده، مثل سیل خواهند آمد و عبورخواهند نمود و برگشته، تا به قلعه او جنگ خواهند کرد.
\par 11 و پادشاه جنوب خشمناک شده، بیرون خواهد آمد و با وی یعنی با پادشاه شمال جنگ خواهد نمود و وی گروه عظیمی برپاخواهد کرد و آن گروه به‌دست وی تسلیم خواهند شد.
\par 12 و چون آن گروه برداشته شود، دلش مغرور خواهد شد و کرورها را هلاک خواهد ساخت اما قوت نخواهد یافت.
\par 13 پس پادشاه شمال مراجعت کرده، لشکری عظیم تر ازاول برپا خواهد نمود و بعد از انقضای مدت سالهابا لشکر عظیمی و دولت فراوانی خواهد آمد.
\par 14 و در آنوقت بسیاری با پادشاه جنوب مقاومت خواهند نمود و بعضی از ستمکیشان قوم توخویشتن را خواهند برافراشت تا رویا را ثابت نمایند اما ایشان خواهند افتاد.
\par 15 «پس پادشاه شمال خواهد آمد و سنگرهابرپا نموده، شهر حصاردار را خواهد گرفت و نه افواج جنوب و نه برگزیدگان او یارای مقاومت خواهند داشت بلکه وی را هیچ یارای مقاومت نخواهد بود.
\par 16 و آنکس که به ضد وی می‌آیدبرحسب رضامندی خود عمل خواهد نمود وکسی نخواهد بود که با وی مقاومت تواند نمودپس در فخر زمینها توقف خواهد نمود و آن به‌دست وی تلف خواهد شد.
\par 17 و عزیمت خواهدنمود که با قوت تمامی مملکت خویش داخل بشود و با وی مصالحه خواهد کرد و او دختر زنان را به وی خواهد داد تا آن را هلاک کند. اما او ثابت نخواهد ماند و از آن او نخواهد بود.
\par 18 پس بسوی جزیره‌ها توجه خواهد نمود و بسیاری ازآنها را خواهد گرفت. لیکن سرداری سرزنش او راباطل خواهد کرد، بلکه انتقام سرزنش او را از اوخواهد گرفت.
\par 19 پس بسوی قلعه های زمین خویش توجه خواهد نمود اما لغزش خواهدخورد و افتاده، ناپدید خواهد شد.
\par 20 «پس در جای او عاملی خواهد برخاست که جلال سلطنت را از میان خواهد برداشت لیکن در اندک ایامی او نیز هلاک خواهد شد نه به غضب و نه به جنگ.
\par 21 و در جای او حقیری خواهدبرخاست اما جلال سلطنت را به وی نخواهند دادو او ناگهان داخل شده، سلطنت را با حیله هاخواهد گرفت.
\par 22 و سیل افواج و رئیس عهد نیزاز حضور او رفته و شکسته خواهند شد.
\par 23 و ازوقتی که ایشان با وی همداستان شده باشند او به حیله رفتار خواهد کرد و با جمعی قلیل افراشته وبزرگ خواهد شد.
\par 24 و ناگهان به برومندترین بلادوارد شده، کارهایی را که نه پدرانش و نه پدران پدرانش کرده باشند بجا خواهد آورد و غارت وغنیمت و اموال را به ایشان بذل خواهد نمود و به ضد شهرهای حصاردار تدبیرها خواهد نمود، لیکن اندک زمانی خواهد بود.
\par 25 و قوت و دل خود را با لشکر عظیمی به ضد پادشاه جنوب برخواهد انگیخت و پادشاه جنوب با فوجی بسیار عظیم و قوی تهیه جنگ خواهد دید امایارای مقاومت نخواهد داشت زیرا که به ضد اوتدبیرها خواهند نمود.
\par 26 و آنانی که خوراک او رامی خورند او را شکست خواهند داد و لشکر اوتلف خواهد شد و بسیاری کشته خواهند افتاد.
\par 27 و دل این دو پادشاه به بدی مایل خواهد شد وبر یک سفره دروغ خواهند گفت، اما پیش نخواهد رفت زیرا که هنوز انتها برای وقت معین خواهد بود.
\par 28 پس با اموال بسیار به زمین خودمراجعت خواهد کرد ودلش به ضد عهد مقدس جازم خواهد بود پس (برحسب اراده خود) عمل نموده، به زمین خود خواهد برگشت.
\par 29 و دروقت معین مراجعت نموده، به زمین جنوب واردخواهد شد لیکن آخرش مثل اولش نخواهد بود.
\par 30 و کشتیها از کتیم به ضد او خواهند آمد لهذامایوس شده، رو خواهد تافت و به ضد عهدمقدس خشمناک شده، (برحسب اراده خود)عمل خواهد نمود و برگشته به آنانی که عهدمقدس را ترک می‌کنند توجه خواهد نمود.
\par 31 وافواج از جانب او برخاسته، مقدس حصین رانجس خواهند نمود و قربانی سوختنی دایمی راموقوف کرده، رجاست ویرانی را برپا خواهندداشت.
\par 32 و آنانی را که به ضد عهد شرارت می‌ورزند با مکرها گمراه خواهد کرد. اما آنانی که خدای خویش را می‌شناسند قوی شده، (کارهای عظیم ) خواهند کرد.
\par 33 و حکیمان قوم بسیاری را تعلیم خواهند داد، لیکن ایامی چند به شمشیر و آتش و اسیری و تاراج خواهند افتاد.
\par 34 و چون بیفتند نصرت کمی خواهند یافت وبسیاری با فریب به ایشان ملحق خواهند شد.
\par 35 وبعضی از حکیمان به جهت امتحان ایشان لغزش خواهند خورد که تا وقت آخر طاهر و سفیدبشوند زیرا که زمان معین هنوز نیست.
\par 36 «و آن پادشاه موافق اراده خود عمل نموده، خویشتن را بر همه خدایان افراشته و بزرگ خواهد نمود و به ضد خدای خدایان سخنان عجیب خواهد گفت و تا انتهای غضب کامیاب خواهد شد زیرا آنچه مقدر است به وقوع خواهدپیوست.
\par 37 و به خدای پدران خود و به فضیلت زنان اعتنا نخواهد نمود، بلکه بهیچ خدا اعتنانخواهد نمود زیرا خویشتن را از همه بلندترخواهد شمرد.
\par 38 و در جای او خدای قلعه‌ها راتکریم خواهد نمود و خدایی را که پدرانش او رانشناختند با طلا و نقره و سنگهای گرانبها ونفایس تکریم خواهد نمود.
\par 39 و با قلعه های حصین مثل خدای بیگانه عمل خواهد نمود وآنانی را که بدو اعتراف نمایند در جلال ایشان خواهد افزود و ایشان را بر اشخاص بسیار تسلطخواهد داد و زمین را برای اجرت (ایشان ) تقسیم خواهد نمود.
\par 40 «و در زمان آخر پادشاه جنوب با وی مقاتله خواهد نمود و پادشاه شمال با ارابه‌ها و سواران و کشتیهای بسیار مانند گردباد به ضد اوخواهد آمد و به زمینها سیلان کرده، از آنها عبورخواهد کرد.
\par 41 و به فخر زمینها وارد خواهد شدو بسیاری خواهند افتاد اما اینان یعنی ادوم وموآب و روسای بنی عمون از دست او خلاصی خواهند یافت.
\par 42 و دست خود را بر کشورهادراز خواهد کرد و زمین مصر رهایی نخواهدیافت.
\par 43 و بر خزانه های طلا و نقره و بر همه نفایس مصر استیلا خواهد یافت و لبیان وحبشیان در موکب او خواهند بود.
\par 44 لیکن اخباراز مشرق و شمال او را مضطرب خواهد ساخت، لهذا با خشم عظیمی بیرون رفته، اشخاص بسیاری را تباه کرده، بالکل هلاک خواهد ساخت.و خیمه های ملوکانه خود را در کوه مجیدمقدس در میان دو دریا برپا خواهد نمود، لیکن به اجل خود خواهد رسید و معینی نخواهد داشت.
\par 45 و خیمه های ملوکانه خود را در کوه مجیدمقدس در میان دو دریا برپا خواهد نمود، لیکن به اجل خود خواهد رسید و معینی نخواهد داشت.

\chapter{12}

\par 1 «و در آن زمان میکائیل، امیر عظیمی که برای پسران قوم تو ایستاده است خواهد برخاست و چنان زمان تنگی خواهد شدکه از حینی که امتی به وجود آمده است تا امروزنبوده و در آنزمان هر یک از قوم تو که در دفترمکتوب یافت شود رستگار خواهد شد.
\par 2 وبسیاری از آنانی که در خاک زمین خوابیده اندبیدار خواهند شد، اما اینان به جهت حیات جاودانی و آنان به جهت خجالت و حقارت جاودانی.
\par 3 و حکیمان مثل روشنایی افلاک خواهند درخشید و آنانی که بسیاری را به راه عدالت رهبری می‌نمایند مانند ستارگان خواهندبود تا ابدالاباد.
\par 4 اما تو‌ای دانیال کلام را مخفی دار و کتاب را تا زمان آخر مهر کن. بسیاری بسرعت تردد خواهند نمود و علم افزوده خواهدگردید.»
\par 5 پس من دانیال نظر کردم و اینک دو نفر دیگریکی به اینطرف و دیگری به آنطرف نهر ایستاده بودند.
\par 6 و یکی از ایشان به آن مرد ملبس به کتان که بالای آبهای نهر ایستاده بود گفت: «انتهای این عجایب تا بکی خواهد بود؟»
\par 7 و آن مرد ملبس به کتان را که بالای آبهای نهر ایستاده بود شنیدم که دست راست و دست چپ خود را بسوی آسمان برافراشته، به حی ابدی قسم خورد که برای زمانی و دو زمان و نصف زمان خواهد بود و چون پراکندگی قوت قوم مقدس به انجام رسد، آنگاه همه این امور به اتمام خواهد رسید.
\par 8 و من شنیدم اما درک نکردم پس گفتم: «ای آقایم آخر این امور چه خواهد بود؟»
\par 9 او جواب داد که «ای دانیال برو زیرا این کلام تا زمان آخرمخفی و مختوم شده است.
\par 10 بسیاری طاهر وسفید و مصفی خواهند گردید و شریران شرارت خواهند ورزید و هیچ کدام از شریران نخواهندفهمید لیکن حکیمان خواهند فهمید.
\par 11 و ازهنگام موقوف شدن قربانی دایمی و نصب نمودن رجاست ویرانی، هزار و دویست و نود روزخواهد بود.خوشابه‌حال آنکه انتظار کشد و به هزار و سیصد و سی و پنج روز برسد.
\par 12 خوشابه‌حال آنکه انتظار کشد و به هزار و سیصد و سی و پنج روز برسد.



\end{document}