\begin{document}

\title{1 Kings}

 
\chapter{1}

\par 1 چند او را به لباس می‌پوشانیدند، لیکن گرم نمی شد.
\par 2 و خادمانش وی را گفتند: «به جهت آقای ما، پادشاه، باکره‌ای جوان بطلبند تا به حضور پادشاه بایستد و او را پرستاری نماید، و در آغوش تو بخوابد تا آقای ما، پادشاه، گرم بشود.»
\par 3 پس در تمامی حدود اسرائیل دختری نیکو منظر طلبیدند و ابیشک شونمیه را یافته، او را نزد پادشاه آوردند.
\par 4 و آن دختر بسیار نیکو منظر بود و پادشاه را پرستاری نموده، او را خدمت می‌کرد، اما پادشاه او رانشناخت.
\par 5 آنگاه ادنیا پسر حجیت، خویشتن را برافراشته، گفت: «من سلطنت خواهم نمود.» و برای خود ارابه‌ها وسواران و پنجاه نفر را که پیش روی وی بدوند، مهیا ساخت.
\par 6 و پدرش او را در تمامی ایام عمرش نرنجانیده، ونگفته بود چرا چنین و چنان می‌کنی، و او نیز بسیار خوش اندام بود و مادرش او را بعد از ابشالوم زاییده بود.
\par 7 و بایوآب بن صرویه و ابیاتار کاهن مشورت کرد و ایشان ادنیا را اعانت نمودند.
\par 8 و اما صادوق کاهن و بنایاهو ابن یهویاداع و ناتان نبی و شمعی و ریعی و شجاعانی که از آن داود بودند، با ادنیا نرفتند.
\par 9 و ادنیا گوسفندان و گاوان و پرواریها نزد سنگ زوحلت که به‌جانب عین روجل است، ذبح نمود، و تمامی برادرانش، پسران پادشاه را با جمیع مردان یهودا که خادمان پادشاه بودند، دعوت نمود.
\par 10 اما ناتان نبی و بنایاهو و شجاعان و برادر خود، سلیمان را دعوت نکرد.
\par 11 و ناتان به بتشبع، مادر سلیمان، عرض کرده، گفت: «آیا نشنیدی که ادنیا، پسر حجیت، سلطنت می‌کند وآقای ما داود نمی داند.
\par 12 پس حال بیا تو را مشورت دهم تا جان خود و جان پسرت، سلیمان را برهانی.
\par 13 بروونزد داود پادشاه داخل شده، وی را بگو که‌ای آقایم پادشاه، آیا تو برای کنیز خود قسم خورده، نگفتی که پسر توسلیمان، بعد از من پادشاه خواهد شد؟ و او بر کرسی من خواهد نشست؟ پس چرا ادنیا پادشاه شده است؟
\par 14 اینک وقتی که تو هنوز در آنجا با پادشاه سخن گویی، من نیز بعد از تو خواهم آمد و کلام تو را ثابت خواهم کرد.»
\par 15 پس بتشبع نزد پادشاه به اطاق درآمد و پادشاه بسیار پیر بود و ابیشک شونمیه، پادشاه را خدمت می‌نمود.
\par 16 و بتشبع خم شده، پادشاه را تعظیم نمود و پادشاه گفت: «تو را چه شده است؟»
\par 17 او وی را گفت: «ای آقایم توبرای کنیز خود به یهوه خدای خویش قسم خوردی که پسر تو، سلیمان بعد از من پادشاه خواهد شد و او بر کرسی من خواهد نشست.
\par 18 و حال اینک ادنیا پادشاه شده است و آقایم پادشاه اطلاع ندارد.
\par 19 و گاوان و پرواریها وگوسفندان بسیار ذبح کرده، همه پسران پادشاه و ابیاتار کاهن و یوآب، سردار لشکر را دعوت کرده، اما بنده ات سلیمان را دعوت ننموده است.
\par 20 و اما‌ای آقایم پادشاه، چشمان تمامی اسرائیل به سوی توست تا ایشان را خبر دهی که بعد از آقایم، پادشاه، کیست که بر کرسی وی خواهد نشست.
\par 21 والا واقع خواهد شد هنگامی که آقایم پادشاه با پدران خویش بخوابد که من و پسرم سلیمان مقصر خواهیم بود.»
\par 22 و اینک چون او هنوز با پادشاه سخن می‌گفت، ناتان نبی نیز داخل شد.
\par 23 و پادشاه را خبر داده، گفتند که «اینک ناتان نبی است.» و او به حضور پادشاه درآمده، رو به زمین خم شده، پادشاه را تعظیم نمود.
\par 24 و ناتان گفت: «ای آقایم پادشاه، آیا تو گفته‌ای که ادنیا بعد از من پادشاه خواهد شد و او بر کرسی من خواهد نشست؟
\par 25 زیرا که امروز او روانه شده، گاوان و پرواریها و گوسفندان بسیار ذبح نموده، و همه پسران پادشاه و سرداران لشکر و ابیاتارکاهن را دعوت کرده است، و اینک ایشان به حضورش به اکل و شرب مشغولند و می‌گویند ادنیای پادشاه زنده بماند.
\par 26 لیکن بنده ات مرا و صادوق کاهن و بنایاهو ابن یهویاداع و بنده ات، سلیمان را دعوت نکرده است.
\par 27 آیااین کار از جانب آقایم، پادشاه شده و آیا به بنده ات خبر ندادی که بعد از آقایم، پادشاه کیست که بر کرسی وی بنشیند؟»
\par 28 و داود پادشاه در جواب گفت: «بتشبع رانزد من بخوانید.» پس او به حضور پادشاه درآمد وبه حضور پادشاه ایستاد.
\par 29 و پادشاه سوگندخورده، گفت: «قسم به حیات خداوند که جان مرا از تمام تنگیها رهانیده است.
\par 30 چنانکه برای تو، به یهوه خدای اسرائیل، قسم خورده، گفتم که پسر تو، سلیمان بعد از من پادشاه خواهد شد، و اوبه‌جای من بر کرسی من خواهد نشست، به همان طور امروز به عمل خواهم آورد.»
\par 31 و بتشبع روبه زمین خم شده، پادشاه را تعظیم نمود و گفت: «آقایم، داود پادشاه تا به ابد زنده بماند!»
\par 32 و داود پادشاه گفت: «صادوق کاهن و ناتان نبی و بنایاهو بن یهویاداع را نزد من بخوانید.» پس ایشان به حضور پادشاه داخل شدند.
\par 33 و پادشاه به ایشان گفت: «بندگان آقای خویش را همراه خود بردارید و پسرم، سلیمان را بر قاطر من سوارنموده، او را به جیحون ببرید.
\par 34 و صادوق کاهن و ناتان نبی او را در آنجا به پادشاهی اسرائیل مسح نمایند و کرنا را نواخته، بگویید: سلیمان پادشاه زنده بماند!
\par 35 و شما در عقب وی برایید تااو داخل شده، بر کرسی من بنشیند و او به‌جای من پادشاه خواهد شد، و او را مامور فرمودم که براسرائیل و بر یهودا پیشوا باشد.»
\par 36 و بنایاهو ابن یهویاداع در جواب پادشاه گفت: «آمین! یهوه، خدای آقایم، پادشاه نیز چنین بگوید.
\par 37 چنانکه خداوند با آقایم، پادشاه بوده است، همچنین باسلیمان نیز باشد، و کرسی وی را از کرسی آقایم داود پادشاه عظیم تر گرداند.»
\par 38 و صادوق کاهن و ناتان نبی و بنایاهو ابن یهویاداع و کریتیان و فلیتیان رفته، سلیمان را برقاطر داود پادشاه سوار کردند و او را به جیحون آوردند.
\par 39 و صادوق کاهن، حقه روغن را ازخیمه گرفته، سلیمان را مسح کرد و چون کرنا رانواختند تمامی قوم گفتند: «سلیمان پادشاه زنده بماند.»
\par 40 و تمامی قوم در عقب وی برآمدند وقوم نای نواختند و به فرح عظیم شادی نمودند، به حدی که زمین از آواز ایشان منشق می‌شد.
\par 41 و ادنیا و تمامی دعوت‌شدگانی که با اوبودند، چون از خوردن فراغت یافتند، این راشنیدند و چون یوآب آواز کرنا را شنید، گفت: «چیست این صدای اضطراب در شهر؟»
\par 42 وچون او هنوز سخن می‌گفت، اینک یوناتان بن ابیاتار کاهن رسید و ادنیا گفت: «بیا زیرا که تو مردشجاع هستی و خبر نیکو می‌آوری.»
\par 43 یوناتان در جواب ادنیا گفت: «به درستی که آقای ما، داودپادشاه، سلیمان را پادشاه ساخته است.
\par 44 وپادشاه، صادوق کاهن و ناتان نبی و بنایاهو ابن یهویاداع و کریتیان و فلیتیان را با او فرستاده، او رابر قاطر پادشاه سوار کرده‌اند.
\par 45 و صادوق کاهن وناتان نبی، او را در جیحون به پادشاهی مسح کرده‌اند و از آنجا شادی‌کنان برآمدند، چنانکه شهر به آشوب درآمد. و این است صدایی که شنیدید.
\par 46 و سلیمان نیز بر کرسی سلطنت جلوس نموده است.
\par 47 و ایض بندگان پادشاه به جهت تهنیت آقای ما، داود پادشاه آمده، گفتند: خدای تو اسم سلیمان را از اسم تو افضل و کرسی او را از کرسی تو اعظم گرداند. و پادشاه بر بسترخود سجده نمود.
\par 48 و پادشاه نیز چنین گفت: متبارک باد یهوه، خدای اسرائیل، که امروز کسی را که بر کرسی من بنشیند، به من داده است وچشمان من، این را می‌بیند.»
\par 49 آنگاه تمامی مهمانان ادنیا ترسان شده، برخاستند و هرکس به راه خود رفت.
\par 50 و ادنیا ازسلیمان ترسان شده، برخاست و روانه شده، شاخهای مذبح را گرفت.
\par 51 و سلیمان را خبرداده، گفتند که «اینک ادنیا از سلیمان پادشاه می ترسد و شاخهای مذبح را گرفته، می‌گوید که سلیمان پادشاه امروز برای من قسم بخورد که بنده خود را به شمشیر نخواهد کشت.»
\par 52 وسلیمان گفت: «اگر مرد صالح باشد، یکی ازمویهایش بر زمین نخواهد افتاد اما اگر بدی در اویافت شود، خواهد مرد.»و سلیمان پادشاه فرستاد تا او را از نزد مذبح آوردند و او آمده، سلیمان پادشاه را تعظیم نمود و سلیمان گفت: «به خانه خود برو.»
\par 53 و سلیمان پادشاه فرستاد تا او را از نزد مذبح آوردند و او آمده، سلیمان پادشاه را تعظیم نمود و سلیمان گفت: «به خانه خود برو.»
 
\chapter{2}

\par 1 و چون ایام وفات داود نزدیک شد، پسرخود سلیمان را وصیت فرموده، گفت:
\par 2 «من به راه تمامی اهل زمین می‌روم. پس تو قوی و دلیر باش.
\par 3 وصایای یهوه، خدای خود را نگاه داشته، به طریق های وی سلوک نما، و فرایض واوامر و احکام و شهادات وی را به نوعی که درتورات موسی مکتوب است، محافظت نما تا درهر کاری که کنی و به هر جایی که توجه نمایی، برخوردار باشی.
\par 4 و تا آنکه خداوند، کلامی را که درباره من فرموده و گفته است، برقرار دارد که اگرپسران تو راه خویش را حفظ نموده، به تمامی دل و به تمامی جان خود در حضور من به راستی سلوک نمایند، یقین که از تو کسی‌که بر کرسی اسرائیل بنشیند، مفقود نخواهد شد.
\par 5 و دیگر تو آنچه را که یوآب بن صرویه به من کرد می‌دانی، یعنی آنچه را با دو سردار لشکراسرائیل ابنیر بن نیر و عماسا ابن یتر کرد و ایشان را کشت و خون جنگ را در حین صلح ریخته، خون جنگ را بر کمربندی که به کمر خود داشت و بر نعلینی که به پایهایش بود، پاشید.
\par 6 پس موافق حکمت خود عمل نما و مباد که موی سفید او به سلامتی به قبر فرو رود.
\par 7 و اما با پسران برزلای جلعادی احسان نما و ایشان از‌جمله خورندگان بر سفره تو باشند، زیرا که ایشان هنگامی که از برادر تو ابشالوم فرار می‌کردم، نزدمن چنین آمدند.
\par 8 و اینک شمعی ابن جیرای بنیامینی از بحوریم نزد توست و او مرا در روزی که به محنایم رسیدم به لعنت سخت لعن کرد، لیکن چون به استقبال من به اردن آمد برای او به خداوند قسم خورده، گفتم که تو را با شمشیرنخواهم کشت.
\par 9 پس الان او را بی‌گناه مشمارزیرا که مرد حکیم هستی و آنچه را که با او بایدکرد، می‌دانی. پس مویهای سفید او را به قبر باخون فرود آور.»
\par 10 پس داود با پدران خود خوابید و در شهرداود دفن شد.
\par 11 و ایامی که داود بر اسرائیل سلطنت می‌نمود، چهل سال بود. هفت سال درحبرون سلطنت کرد و در اورشلیم سی و سه سال سلطنت نمود.
\par 12 و سلیمان بر کرسی پدر خودداود نشست و سلطنت او بسیار استوار گردید.
\par 13 و ادنیا پسر حجیت نزد بتشبع، مادر سلیمان آمد و او گفت: «آیا به سلامتی آمدی؟» او جواب داد: «به سلامتی.»
\par 14 پس گفت: «با تو حرفی دارم.» او گفت: «بگو.»
\par 15 گفت: «تو می‌دانی که سلطنت با من شده بود و تمامی اسرائیل روی خود را به من مایل کرده بودند تا سلطنت نمایم، اما سلطنت منتقل شده، از آن برادرم گردید زیراکه از جانب خداوند از آن او بود.
\par 16 و الان خواهشی از تو دارم؛ مسالت مرا رد مکن.» او وی را گفت: «بگو.»
\par 17 گفت: «تمنا این که به سلیمان پادشاه بگویی زیرا خواهش تو را رد نخواهد کردتا ابیشک شونمیه را به من به زنی بدهد.»
\par 18 بتشبع گفت: «خوب، من نزد پادشاه برای تو خواهم گفت.»
\par 19 پس بتشبع نزد سلیمان پادشاه داخل شد تابا او درباره ادنیا سخن گوید. و پادشاه به استقبالش برخاسته، او را تعظیم نمود و بر کرسی خودنشست و فرمود تا به جهت مادر پادشاه کرسی بیاورند و او به‌دست راستش بنشست.
\par 20 و اوعرض کرد: «یک مطلب جزئی دارم که از تو سوال نمایم. مسالت مرا رد منما.» پادشاه گفت: «ای مادرم بگو زیرا که مسالت تو را رد نخواهم کرد.»
\par 21 و او گفت: «ابیشک شونمیه به برادرت ادنیا به زنی داده شود.»
\par 22 سلیمان پادشاه، مادر خود راجواب داده، گفت: «چرا ابیشک شونمیه را به جهت ادنیا طلبیدی؟ سلطنت را نیز برای وی طلب کن چونکه او برادر بزرگ من است، هم به جهت او و هم به جهت ابیاتار کاهن و هم به جهت یوآب بن صرویه.»
\par 23 و سلیمان پادشاه به خداوند قسم خورده، گفت: «خدا به من مثل این بلکه زیاده از این عمل نماید اگر ادنیا این سخن رابه ضرر جان خود نگفته باشد.
\par 24 و الان قسم به حیات خداوند که مرا استوار نموده، و مرا برکرسی پدرم، داود نشانیده، و خانه‌ای برایم به طوری که وعده نموده بود، برپا کرده است که ادنیا امروز خواهد مرد.» 
\par 25 پس سلیمان پادشاه به‌دست بنایاهو ابن یهویاداع فرستاد و او وی را زدکه مرد.
\par 26 و پادشاه به ابیاتار کاهن گفت: «به مزرعه خود به عناتوت برو زیرا که تو مستوجب قتل هستی، لیکن امروز تو را نخواهم کشت، چونکه تابوت خداوند، یهوه را در حضور پدرم داودبرمی داشتی، و در تمامی مصیبت های پدرم مصیبت کشیدی.»
\par 27 پس سلیمان، ابیاتار را ازکهانت خداوند اخراج نمود تا کلام خداوند را که درباره خاندان عیلی در شیلوه گفته بود، کامل گرداند.
\par 28 و چون خبر به یوآب رسید، یوآب به خیمه خداوند فرار کرده، شاخهای مذبح را گرفت زیراکه یوآب، ادنیا را متابعت کرده، هرچند ابشالوم رامتابعت ننموده بود.
\par 29 و سلیمان پادشاه را خبردادند که یوآب به خیمه خداوند فرار کرده، واینک به پهلوی مذبح است. پس سلیمان، بنایاهوابن یهویاداع را فرستاده، گفت: «برو و او را بکش.»
\par 30 و بنایاهو به خیمه خداوند داخل شده، او راگفت: «پادشاه چنین می‌فرماید که بیرون بیا.» اوگفت: «نی، بلکه اینجا می‌میرم.» و بنایاهو به پادشاه خبر رسانیده، گفت که «یوآب چنین گفته، و چنین به من جواب داده است.»
\par 31 پادشاه وی را فرمود: «موافق سخنش عمل نما و او را کشته، دفن کن تا خون بی‌گناهی را که یوآب ریخته بود از من و از خاندان پدرم دورنمایی.
\par 32 و خداوند خونش را بر سر خودش ردخواهد گردانید به‌سبب اینکه بر دو مرد که از اوعادلتر و نیکوتر بودند هجوم آورده، ایشان را با شمشیر کشت و پدرم، داود اطلاع نداشت، یعنی ابنیر بن نیر، سردار لشکر اسرائیل و عماسا ابن یتر، سردار لشکر یهودا.
\par 33 پس خون ایشان برسر یوآب و بر سر ذریتش تا به ابد برخواهد گشت و برای داود و ذریتش و خاندانش و کرسی‌اش سلامتی از جانب خداوند تا ابدالاباد خواهد بود.»
\par 34 پس بنایاهو ابن یهویاداع رفته، او را زد و کشت و او را در خانه‌اش که در صحرا بود، دفن کردند.
\par 35 و پادشاه بنایاهو ابن یهویاداع را به‌جایش به‌سرداری لشکر نصب کرد و پادشاه، صادوق کاهن را در جای ابیاتار گماشت.
\par 36 و پادشاه فرستاده، شمعی را خوانده، وی راگفت: «به جهت خود خانه‌ای در اورشلیم بناکرده، در آنجا ساکن شو و از آنجا به هیچ طرف بیرون مرو.
\par 37 زیرا یقین در روزی که بیرون روی و از نهر قدرون عبور نمایی، بدان که البته خواهی مرد و خونت بر سر خودت خواهد بود.»
\par 38 وشمعی به پادشاه گفت: «آنچه گفتی نیکوست. به طوری که آقایم پادشاه فرموده است، بنده ات چنین عمل خواهد نمود.» پس شمعی روزهای بسیار در اورشلیم ساکن بود.
\par 39 اما بعد از انقضای سه سال واقع شد که دوغلام شمعی نزد اخیش بن معکه، پادشاه جت فرار کردند و شمعی را خبر داده، گفتند که «اینک غلامانت در جت هستند.»
\par 40 و شمعی برخاسته، الاغ خود را بیاراست و به جستجوی غلامانش، نزد اخیش به جت روانه شد، و شمعی رفته، غلامان خود را از جت بازآورد.
\par 41 و به سلیمان خبر دادند که شمعی از اورشلیم به جت رفته وبرگشته است.
\par 42 و پادشاه فرستاده، شمعی را خواند و وی را گفت: «آیا تو را به خداوند قسم ندادم و تو را نگفتم در روزی که بیرون شوی و به هر جا بروی یقین بدان که خواهی مرد، و تو مراگفتی سخنی که شنیدم نیکوست.
\par 43 پس قسم خداوند و حکمی را که به تو امر فرمودم، چرانگاه نداشتی؟»
\par 44 و پادشاه به شمعی گفت: «تمامی بدی را که دلت از آن آگاهی دارد که به پدر من داود کرده‌ای، می‌دانی و خداوند شرارت تو را به‌سرت برگردانیده است.
\par 45 و سلیمان پادشاه، مبارک خواهد بود و کرسی داود درحضور خداوند تا به ابد پایدار خواهد ماند.»پس پادشاه بنایاهو ابن یهویاداع را امر فرمود واو بیرون رفته، او را زد که مرد.و سلطنت در دست سلیمان برقرار گردید.
\par 46 پس پادشاه بنایاهو ابن یهویاداع را امر فرمود واو بیرون رفته، او را زد که مرد.و سلطنت در دست سلیمان برقرار گردید.
 
\chapter{3}

\par 1 و سلیمان با فرعون، پادشاه مصر، مصاهرت نموده، دختر فرعون را گرفت، واو را به شهر داود آورد تا بنای خانه خود و خانه خداوند و حصار اورشلیم را به هر طرفش تمام کند.
\par 2 لیکن قوم در مکانهای بلند قربانی می‌گذرانیدند زیرا خانه‌ای برای اسم خداوند تاآن زمان بنا نشده بود.
\par 3 و سلیمان خداوند را دوست داشته، به فرایض پدر خود، داود رفتار می‌نمود، جز اینکه در مکانهای بلند قربانی می‌گذرانید و بخورمی سوزانید.
\par 4 و پادشاه به جبعون رفت تا در آنجاقربانی بگذراند زیرا که مکان بلند عظیم، آن بود وسلیمان بر آن مذبح هزار قربانی سوختنی گذرانید.
\par 5 و خداوند به سلیمان در جبعون درخواب شب ظاهر شد. و خدا گفت: «آنچه را که به تو بدهم، طلب نما.»
\par 6 سلیمان گفت: «تو با بنده ات، پدرم داود، هرگاه در حضور تو با راستی و عدالت و قلب سلیم با تو رفتار می‌نمود، احسان عظیم می‌نمودی، و این احسان عظیم را برای اونگاه داشتی که پسری به او دادی تا بر کرسی وی بنشیند، چنانکه امروز واقع شده است.
\par 7 و الان‌ای یهوه، خدای من، تو بنده خود را به‌جای پدرم داود، پادشاه ساختی و من طفل صغیر هستم که خروج و دخول را نمی دانم.
\par 8 و بنده ات در میان قوم تو که برگزیده‌ای هستم، قوم عظیمی که کثیرند به حدی که ایشان را نتوان شمرد و حساب کرد.
\par 9 پس به بنده خود دل فهیم عطا فرما تا قوم تو را داوری نمایم و در میان نیک و بد تمیز کنم، زیرا کیست که این قوم عظیم تو را داوری تواندنمود؟»
\par 10 و این امر به نظر خداوند پسند آمد که سلیمان این چیز را خواسته بود.
\par 11 پس خدا وی را گفت: «چونکه این چیز را خواستی و طول ایام برای خویشتن نطلبیدی، و دولت برای خودسوال ننمودی، و جان دشمنانت را نطلبیدی، بلکه به جهت خود حکمت خواستی تا انصاف رابفهمی،
\par 12 اینک بر‌حسب کلام تو کردم و اینک دل حکیم و فهیم به تو دادم به طوری که پیش از تومثل تویی نبوده است و بعد از تو کسی مثل تونخواهد برخاست.
\par 13 و نیز آنچه را نطلبیدی، یعنی هم دولت و هم جلال را به تو عطا فرمودم به حدی که در تمامی روزهایت کسی مثل تو درمیان پادشاهان نخواهد بود.
\par 14 و اگر در راههای من سلوک نموده، فرایض و اوامر مرا نگاه داری به طوری که پدر تو داود سلوک نمود، آنگاه روزهایت را طویل خواهم گردانید.»
\par 15 پس سلیمان بیدار شد و اینک خواب بود وبه اورشلیم آمده، پیش تابوت عهد خداوند ایستاد، و قربانی های سوختنی گذرانید و ذبایح سلامتی ذبح کرده، برای تمامی بندگانش ضیافت نمود.
\par 16 آنگاه دو زن زانیه نزد پادشاه آمده، درحضورش ایستادند.
\par 17 و یکی از آن زنان گفت: «ای آقایم، من و این زن در یک خانه ساکنیم و درآن خانه با او زاییدم.
\par 18 و روز سوم بعد از زاییدنم واقع شد که این زن نیز زایید و ما با یکدیگر بودیم و کسی دیگر با ما در خانه نبود و ما هر دو در خانه‌تنها بودیم.
\par 19 و در شب، پسر این زن مرد زیرا که بر او خوابیده بود.
\par 20 و او در نصف شب برخاسته، پسر مرا وقتی که کنیزت در خواب بود از پهلوی من گرفت و در بغل خود گذاشت و پسر مرده خودرا در بغل من نهاد.
\par 21 و بامدادان چون برخاستم تاپسر خود را شیر دهم‌اینک مرده بود اما چون دروقت صبح بر او نگاه کردم، دیدم که پسری که من زاییده بودم، نیست.»
\par 22 زن دیگر گفت: «نی، بلکه پسر زنده از آن من است و پسر مرده از آن توست.» و آن دیگر گفت: «نی، بلکه پسر مرده ازآن توست و پسر زنده از آن من است.» و به حضورپادشاه مکالمه می‌کردند.
\par 23 پس پادشاه گفت: «این می‌گوید که این پسرزنده از آن من است و پسر مرده از آن توست و آن می‌گوید نی، بلکه پسر مرده از آن توست و پسرزنده از آن من است.»
\par 24 و پادشاه گفت: «شمشیری نزد من بیاورید.» پس شمشیری به حضور پادشاه آوردند.
\par 25 و پادشاه گفت: «پسرزنده را به دو حصه تقسیم نمایید و نصفش را به این و نصفش را به آن بدهید.»
\par 26 و زنی که پسرزنده از آن او بود چونکه دلش بر پسرش می سوخت به پادشاه عرض کرده، گفت: «ای آقایم! پسر زنده را به او بدهید و او را هرگزمکشید.» اما آن دیگری گفت: «نه از آن من و نه ازآن تو باشد؛ او را تقسیم نمایید.»
\par 27 آنگاه پادشاه امر فرموده، گفت: «پسر زنده را به او بدهید و او راالبته مکشید زیرا که مادرش این است.»و چون تمامی اسرائیل حکمی را که پادشاه کرده بود، شنیدند از پادشاه بترسیدند زیرا دیدند که حکمت خدایی به جهت داوری کردن در دل اوست.
\par 28 و چون تمامی اسرائیل حکمی را که پادشاه کرده بود، شنیدند از پادشاه بترسیدند زیرا دیدند که حکمت خدایی به جهت داوری کردن در دل اوست.
 
\chapter{4}

\par 1 و سلیمان پادشاه بر تمامی اسرائیل پادشاه بود.
\par 2 و سردارانی که داشت اینانند: عزریاهو ابن صادوق کاهن،
\par 3 و الیحورف و اخیاپسران شیشه کاتبان و یهوشافاط بن اخیلود وقایع نگار،
\par 4 و بنایاهو ابن یهویاداع، سردار لشکر، وصادوق و ابیاتار کاهنان،
\par 5 و عزریاهو بن ناتان، سردار وکلاء و زابود بن ناتان کاهن و دوست خالص پادشاه،
\par 6 و اخیشار ناظر خانه و ادونیرام بن عبدا، رئیس باجگیران.
\par 7 و سلیمان دوازده وکیل بر تمامی اسرائیل داشت که به جهت خوراک پادشاه و خاندانش تدارک می‌دیدند، که هریک از ایشان یک ماه درسال تدارک می‌دیدند.
\par 8 و نامهای ایشان این است: بنحور در کوهستان افرایم
\par 9 و بندقر درماقص و شعلبیم و بیت شمس و ایلون بیت حانان
\par 10 و بنحسد در اربوت که سوکوه و تمامی زمین حافر به او تعلق داشت
\par 11 و بنئبینداب در تمامی نافت دور که تافت دختر سلیمان زن او بود
\par 12 وبعنا ابن اخیلود در تعنک و مجدو و تمامی بیتشان که به‌جانب صرتان زیر یزرعیل است از بیتشان تا آبل محوله تا آن طرف یقمعام
\par 13 و بنجابر درراموت جلعاد که قرای یاعیر بن منسی که درجلعاد می‌باشد و بلوک ارجوب که در باشان است به او تعلق داشت، یعنی شصت شهر بزرگ حصاردار با پشت بندهای برنجین
\par 14 و اخیناداب بن عدو در محنایم
\par 15 و اخیمعص در نفتالی که اونیز باسمت، دختر سلیمان را به زنی گرفته بود
\par 16 وبعنا ابن حوشای در اشیر و بعلوت
\par 17 و یهوشافاطبن فاروح در یساکار
\par 18 و شمعی ابن ایلا دربنیامین
\par 19 و جابر بن اوری در زمین جلعاد که ولایت سیحون پادشاه اموریان و عوج پادشاه باشان بود و او به تنهایی در آن زمین وکیل بود.
\par 20 و یهودا و اسرائیل مثل ریگ کناره دریابیشمار بودند و اکل و شرب نموده، شادی می‌کردند.
\par 21 و سلیمان بر تمامی ممالک، از نهر(فرات ) تا زمین فلسطینیان و تا سرحد مصرسلطنت می‌نمود، و هدایا آورده، سلیمان را درتمامی ایام عمرش خدمت می‌کردند.
\par 22 و آذوقه سلیمان برای هر روز سی کر آردنرم و شصت کر بلغور بود.
\par 23 و ده گاو پرواری وبیست گاو از چراگاه و صد گوسفند سوای غزالهاو آهوها و گوزنها و مرغهای فربه.
\par 24 زیرا که برتمام ماورای نهر از تفسح تا غزه بر جمیع ملوک ماورای نهر حکمرانی می‌نمود و او را از هرجانب به همه اطرافش صلح بود.
\par 25 و یهودا واسرائیل، هرکس زیر مو و انجیر خود از دان تابئرشبع در تمامی ایام سلیمان ایمن می‌نشستند.
\par 26 و سلیمان را چهل هزار آخور اسب به جهت ارابه هایش و دوازده هزار سوار بود.
\par 27 و آن وکلا از برای خوراک سلیمان پادشاه و همه کسانی که بر سفره سلیمان پادشاه حاضر می‌بودند، هر یک در ماه خود تدارک می‌دیدند و نمی گذاشتند که به هیچ‌چیز احتیاج باشد.
\par 28 و جو و کاه به جهت اسبان و اسبان تازی به مکانی که هر کس بر‌حسب وظیفه‌اش مقرر بود، می‌آوردند.
\par 29 و خدا به سلیمان حکمت و فطانت از حدزیاده و وسعت دل مثل ریگ کناره دریا عطافرمود.
\par 30 و حکمت سلیمان از حکمت تمامی بنی مشرق و از حکمت جمیع مصریان زیاده بود.
\par 31 و از جمیع آدمیان از ایتان ازراحی و از پسران ماحول، یعنی حیمان و کلکول و دردع حکیم تربود و اسم او در میان تمامی امتهایی که به اطرافش بودند، شهرت یافت.
\par 32 و سه هزار مثل گفت وسرودهایش هزار و پنج بود. 
\par 33 و درباره درختان سخن گفت، از سرو آزاد لبنان تا زوفائی که بردیوارها می‌روید و درباره بهایم و مرغان وحشرات و ماهیان نیز سخن گفت.و از جمیع طوایف و از تمام پادشاهان زمین که آوازه حکمت او را شنیده بودند، می‌آمدند تا حکمت سلیمان را استماع نمایند.
\par 34 و از جمیع طوایف و از تمام پادشاهان زمین که آوازه حکمت او را شنیده بودند، می‌آمدند تا حکمت سلیمان را استماع نمایند.
 
\chapter{5}

\par 1 و حیرام، پادشاه صور، خادمان خود را نزدسلیمان فرستاد، چونکه شنیده بود که او رابه‌جای پدرش به پادشاهی مسح کرده‌اند، زیرا که حیرام همیشه دوست داود بود.
\par 2 و سلیمان نزد حیرام فرستاده، گفت
\par 3 که «تو پدر من داود رامی دانی که نتوانست خانه‌ای به اسم یهوه، خدای خود بنا نماید به‌سبب جنگهایی که او را احاطه می‌نمود تا خداوند ایشان را زیر کف پایهای اونهاد.
\par 4 اما الان یهوه، خدای من، مرا از هر طرف آرامی داده است که هیچ دشمنی و هیچ واقعه بدی وجود ندارد.
\par 5 و اینک مراد من این است که خانه‌ای به اسم یهوه، خدای خود، بنا نمایم چنانکه خداوند به پدرم داود وعده داد و گفت که پسرت که او را به‌جای تو بر کرسی خواهم نشانید، خانه را به اسم من بنا خواهد کرد.
\par 6 و حال امر فرما که سروهای آزاد از لبنان برای من قطع نمایند و خادمان من همراه خادمان تو خواهندبود، و مزد خادمانت را موافق هرآنچه بفرمایی به تو خواهم داد، زیرا تو می‌دانی که در میان ما کسی نیست که مثل صیدونیان در قطع نمودن درختان ماهر باشد.»
\par 7 پس چون حیرام سخنان سلیمان را شنید، به غایت شادمان شده، گفت: «امروز خداوند متبارک باد که به داود پسری حکیم بر این قوم عظیم عطانموده است.»
\par 8 و حیرام نزد سلیمان فرستاده، گفت: «پیغامی که نزد من فرستادی اجابت نمودم و من خواهش تو را درباره چوب سرو آزاد وچوب صنوبر بجا خواهم آورد.
\par 9 خادمان من آنهارا از لبنان به دریا فرود خواهند‌آورد و من آنها رابستنه خواهم ساخت در دریا، تا مکانی که برای من معین کنی و آنها را در آنجا از هم باز خواهم کرد تا آنها را ببری و اما تو درباره دادن آذوقه به خانه من اراده مرا به‌جا خواهی آورد.»
\par 10 پس حیرام چوبهای سرو آزاد و چوبهای صنوبر را موافق تمامی اراده‌اش به سلیمان داد.
\par 11 وسلیمان بیست هزار کر گندم و بیست هزار کرروغن صاف به حیرام به جهت قوت خانه‌اش داد، و سلیمان هرساله اینقدر به حیرام می‌داد.
\par 12 وخداوند سلیمان را به نوعی که به او وعده داده بود، حکمت بخشید و در میان حیرام و سلیمان صلح بود و با یکدیگر عهد بستند.
\par 13 و سلیمان پادشاه از تمامی اسرائیل سخره گرفت و آن سخره سی هزار نفر بود.
\par 14 و از ایشان ده هزار نفر، هر ماهی به نوبت به لبنان می‌فرستاد. یک ماه در لبنان و دو ماه در خانه خویش می‌ماندند. و ادونیرام رئیس سخره بود.
\par 15 وسلیمان را هفتاد هزار مرد باربردار و هشتاد هزارنفر چوب بر در کوه بود.
\par 16 سوای سروران گماشتگان سلیمان که ناظر کار بود، یعنی سه هزارو سیصد نفر که بر عاملان کار ضابط بودند.
\par 17 وپادشاه امر فرمود تا سنگهای بزرگ و سنگهای گرانبها و سنگهای تراشیده شده به جهت بنای خانه کندند.و بنایان سلیمان و بنایان حیرام وجبلیان آنها را تراشیدند، پس چوبها و سنگها را به جهت بنای خانه مهیا ساختند.
\par 18 و بنایان سلیمان و بنایان حیرام وجبلیان آنها را تراشیدند، پس چوبها و سنگها را به جهت بنای خانه مهیا ساختند.
 
\chapter{6}

\par 1 و واقع شد در سال چهارصد و هشتاد ازخروج بنی‌اسرائیل از زمین مصر در ماه زیوکه ماه دوم از سال چهارم سلطنت سلیمان براسرائیل بود که بنای خانه خداوند را شروع کرد.
\par 2 و خانه خداوند که سلیمان پادشاه بنا نمودطولش شصت ذراع و عرضش بیست و بلندیش سی ذراع بود.
\par 3 و رواق پیش هیکل خانه موافق عرض خانه، طولش بیست ذراع و عرضش روبروی خانه ده ذراع بود.
\par 4 و برای خانه پنجره های مشبک ساخت.
\par 5 و بر دیوار خانه به هر طرفش طبقه‌ها بنا کرد، یعنی به هر طرف دیوارهای خانه هم بر هیکل و هم بر محراب و به هر طرفش غرفه‌ها ساخت.
\par 6 و طبقه تحتانی عرضش پنج ذراع و طبقه وسطی عرضش شش ذراع و طبقه سومی عرضش هفت ذراع بود زیراکه به هر طرف خانه از خارج پشته‌ها گذاشت تاتیرها در دیوار خانه متمکن نشود.
\par 7 و چون خانه بنا می‌شد از سنگهایی که در معدن مهیا شده بود، بنا شد به طوری که در وقت بنا نمودن خانه نه چکش و نه تبر و نه هیچ آلات آهنی مسموع شد.
\par 8 و در غرفه های وسطی در جانب راست خانه بود و به طبقه وسطی و از طبقه وسطی تا طبقه سومی از پله های پیچاپیچ بالا می‌رفتند.
\par 9 و خانه را بنا کرده، آن را به اتمام رسانید و خانه را با تیرهاو تخته های چوب سرو آزاد پوشانید.
\par 10 و برتمامی خانه طبقه‌ها را بنا نمود که بلندی هر یک از آنها پنج ذراع بود و با تیرهای سرو آزاد در خانه متمکن شد.
\par 11 و کلام خداوند بر سلیمان نازل شده، گفت:
\par 12 «این خانه‌ای که تو بنا می‌کنی اگر در فرایض من سلوک نموده، احکام مرا به‌جا آوری و جمیع اوامر مرا نگاه داشته، در آنها رفتار نمایی، آنگاه سخنان خود را که با پدرت، داود، گفته‌ام با تواستوار خواهم گردانید.
\par 13 و در میان بنی‌اسرائیل ساکن شده، قوم خود اسرائیل را ترک نخواهم نمود.»
\par 14 پس سلیمان خانه را بنا نموده، آن را به اتمام رسانید.
\par 15 و اندرون دیوارهای خانه را به تخته های سرو آزاد بنا کرد، یعنی از زمین خانه تادیوار متصل به سقف را از اندرون با چوب پوشانید و زمین خانه را به تخته های صنوبر فرش کرد.
\par 16 و از پشت خانه بیست ذراع با تخته های سرو آزاد از زمین تا سر دیوارها بنا کرد و آنها رادر اندرون به جهت محراب، یعنی به جهت قدس‌الاقداس بنا نمود.
\par 17 و خانه، یعنی هیکل پیش روی محراب چهل ذراع بود.
\par 18 و دراندرون خانه چوب سرو آزاد منبت به شکل کدوها و بسته های گل بود چنانکه همه‌اش سروآزاد بود و هیچ سنگ پیدا نشد.
\par 19 و در اندرون خانه، محراب را ساخت تا تابوت عهد خداوند رادر آن بگذارد.
\par 20 و اما داخل محراب طولش بیست ذراع و عرضش بیست ذراع و بلندیش بیست ذراع بود و آن را به زر خالص پوشانید ومذبح را با چوب سرو آزاد پوشانید.
\par 21 پس سلیمان داخل خانه را به زر خالص پوشانید وپیش روی محراب زنجیرهای طلا کشید و آن رابه طلا پوشانید.
\par 22 و تمامی خانه را به طلاپوشانید تا همگی خانه تمام شد و تمامی مذبح راکه پیش روی محراب بود، به طلا پوشانید.
\par 23 و در محراب دو کروبی از چوب زیتون ساخت که قد هر یک از آنها ده ذراع بود.
\par 24 و بال یک کروبی پنج ذراع و بال کروبی دیگر پنج ذراع بود و از سر یک بال تا به‌سر بال دیگر ده ذراع بود.
\par 25 و کروبی دوم ده ذراع بود که هر دو کروبی رایک اندازه و یک شکل بود.
\par 26 بلندی کروبی اول ده ذراع بود و همچنین کروبی دیگر.
\par 27 و کروبیان را در اندرون خانه گذاشت و بالهای کروبیان پهن شد به طوری که بال یک کروبی به دیوار می‌رسیدو بال کروبی دیگر به دیوار دیگر می‌رسید و درمیان خانه بالهای آنها با یکدیگر برمی خورد.
\par 28 وکروبیان را به طلا پوشانید.
\par 29 و بر تمامی دیوارهای خانه، به هر طرف نقشهای تراشیده شده کروبیان و درختان خرما وبسته های گل در اندرون و بیرون کند.
\par 30 وزمین خانه را از اندرون و بیرون به طلاپوشانید.
\par 31 و به جهت در محراب دو لنگه از چوب زیتون، و آستانه و باهوهای آن را به اندازه پنج یک دیوار ساخت.
\par 32 پس آن دو لنگه از چوب زیتون بود و بر آنها نقشهای کروبیان و درختان خرما و بسته های گل کند و به طلا پوشانید. وکروبیان و درختان خرما را به طلا پوشانید.
\par 33 و همچنین به جهت در هیکل باهوهای چوب زیتون به اندازه چهار یک دیوار ساخت.
\par 34 و دو لنگه این در از چوب صنوبر بود و دو تخته لنگه اول تا می‌شد و دو تخته لنگه دوم تا می‌شد.
\par 35 و بر آنها کروبیان و درختان خرما و بسته های گل کند و آنها را به طلایی که موافق نقشها ساخته بود، پوشانید.
\par 36 و صحن اندرون را از سه صف سنگهای تراشیده، و یک صف تیرهای سرو آزادبنا نمود.
\par 37 و بنیاد خانه خداوند در ماه زیو از سال چهارم سلطنت نهاده شد.و در سال یازدهم درماه بول که ماه هشتم باشد، خانه با تمامی متعلقاتش بر وفق تمامی قانون هایش تمام شد. پس آن را در هفت سال بنا نمود.
\par 38 و در سال یازدهم درماه بول که ماه هشتم باشد، خانه با تمامی متعلقاتش بر وفق تمامی قانون هایش تمام شد. پس آن را در هفت سال بنا نمود.
 
\chapter{7}

\par 1 اما خانه خودش را سلیمان در مدت سیزده سال بنا نموده، تمامی خانه خویش را به اتمام رسانید.
\par 2 و خانه جنگل لبنان را بنا نمود که طولش صد ذراع و عرضش پنجاه ذراع و بلندیش سی ذراع بود و آن را بر چهار صف تیرهای سروآزاد بنا کرد و بر آن ستونها، تیرهای سرو آزادگذاشت.
\par 3 و آن بر زبر چهل و پنج غرفه که بالای ستونهابود به‌سرو آزاد پوشانیده شد که در هر صف پانزده بود.
\par 4 و سه صف تخته پوش بود و پنجره مقابل پنجره در سه طبقه بود.
\par 5 و جمیع درها وباهوها مربع و تخته پوش بود و پنجره مقابل پنجره در سه طبقه بود.
\par 6 و رواقی از ستونها ساخت که طولش پنجاه ذراع و عرضش سی ذراع بود و رواقی پیش آنها.
\par 7 و ستونها و آستانه پیش آنها و رواقی به جهت کرسی خود، یعنی رواق داوری که در آن حکم نماید، ساخت و آن را به‌سرو آزاد از زمین تاسقف پوشانید.
\par 8 و خانه‌اش که در آن ساکن شود در صحن دیگر در اندرون رواق به همین ترکیب ساخته شد. و برای دختر فرعون که سلیمان او را به زنی گرفته بود، خانه‌ای مثل این‌رواق ساخت.
\par 9 همه این عمارات از سنگهای گرانبهایی که به اندازه تراشیده و از اندرون و بیرون با اره‌ها بریده شده بود از بنیاد تا به‌سر دیوار و از بیرون تا صحن بزرگ بود.
\par 10 و بنیاد از سنگهای گرانبها وسنگهای بزرگ، یعنی سنگهای ده ذراعی و سنگهای هشت ذراعی بود.
\par 11 و بالای آنهاسنگهای گرانبها که به اندازه تراشیده شده، وچوبهای سرو آزاد بود.
\par 12 و گرداگرد صحن بزرگ سه صف سنگهای تراشیده و یک صف تیرهای سرو آزاد بود و صحن اندرون خانه خداوند ورواق خانه همچنین بود.
\par 13 و سلیمان پادشاه فرستاده، حیرام را از صورآورد.
\par 14 و او پسر بیوه‌زنی از سبط نفتالی بود وپدرش مردی از اهل صور و مسگر بود و او پر ازحکمت و مهارت و فهم برای کردن هر صنعت مسگری بود. پس نزد سلیمان پادشاه آمده، تمامی کارهایش را به انجام رسانید.
\par 15 و دو ستون برنج ریخت که طول هر ستون هجده ذراع بود و ریسمانی دوازده ذراع ستون دوم را احاطه داشت.
\par 16 و دو تاج از برنج ریخته شده ساخت تا آنها را بر سر ستونها بگذارد که طول یک تاج پنج ذراع و طول تاج دیگر پنج ذراع بود.
\par 17 و شبکه های شبکه کاری و رشته های زنجیر کاری بود به جهت تاجهایی که بر سرستونها بود، یعنی هفت برای تاج اول و هفت برای تاج دوم.
\par 18 پس ستونها را ساخت و گرداگرد یک شبکه کاری دو صف بود تا تاجهایی را که بر سرانارها بود بپوشاند. و به جهت تاج دیگر همچنین ساخت.
\par 19 و تاجهایی که بر سر ستونهایی که دررواق بود، از سوسنکاری به مقدار چهار ذراع بود.
\par 20 و تاجها از طرف بالا نیز بر سر آن دو ستون بودنزد بطنی که به‌جانب شبکه بود، و انارها در صفهاگرداگرد تاج دیگر دویست بود.
\par 21 و ستونها را دررواق هیکل برپا نمود و ستون راست را برپانموده، آن را یاکین نام نهاد. پس ستون چپ را برپا نموده، آن را بوعز نامید.
\par 22 و بر سر ستونهاسوسنکاری بود. پس کار ستونها تمام شد.
\par 23 و دریاچه ریخته شده را ساخت که از لب تالبش ده ذراع بود و از هر طرف مدور بود، وبلندیش پنج ذراع و ریسمانی سی ذراعی آن راگرداگرد احاطه داشت.
\par 24 و زیر لب آن از هرطرف کدوها بود که آن را احاطه می‌داشت برای هر ذراع ده، و آنها دریاچه را از هر جانب احاطه داشت و آن کدوها در دو صف بود و در حین ریخته شدن آن، ریخته شده بود. 
\par 25 و آن بردوازده گاو قایم بود که روی سه از آنها به سوی شمال بود و روی سه به سوی مغرب و روی سه به سوی جنوب و روی سه به سوی مشرق بود، ودریاچه بر فوق آنها بود و همه موخرهای آنها به طرف اندرون بود.
\par 26 و حجم آن یک وجب بود ولبش مثل لب کاسه مانند گل سوسن ساخته شده بود که گنجایش آن دو هزار بت می‌داشت.
\par 27 و ده پایه‌اش را از برنج ساخت که طول هرپایه چهار ذراع بود و عرضش چهار ذراع وبلندیش سه ذراع بود.
\par 28 و صنعت پایه‌ها اینطوربود که حاشیه‌ها داشت و حاشیه‌ها در میان زبانه هابود.
\par 29 و بر آن حاشیه‌ها که درون زبانه‌ها بودشیران و گاوان و کروبیان بودند و همچنین برزبانه‌ها به طرف بالا بود. و زیر شیران و گاوان بسته های گل کاری آویزان بود.
\par 30 و هر پایه چهارچرخ برنجین با میله های برنجین داشت و چهارپایه آن را دوشها بود و آن دوشها زیر حوض ریخته شده بود و بسته‌ها به‌جانب هریک طرف ازآنها بود.
\par 31 و دهنش در میان تاج و فوق آن یک ذراع بود و دهنش مثل کار پایه مدور و یک ذراع ونیم بود. و بر دهنش نیز نقشها بود و حاشیه های آنها مربع بود نه مدور.
\par 32 و چهار چرخ زیر حاشیه‌ها بود و تیره های چرخها در پایه بود وبلندی هر چرخ یک ذراع و نیم بود.
\par 33 و کارچرخها مثل کار چرخهای ارابه بود و تیره‌ها وفلکه‌ها و پره‌ها و قبه های آنها همه ریخته شده بود.
\par 34 و چهار دوش بر چهار گوشه هر پایه بود ودوشهای پایه از خودش بود.
\par 35 و در سر پایه، دایره‌ای مدور به بلندی نیم ذراع بود و بر سر پایه، تیرهایش و حاشیه هایش از خودش بود.
\par 36 و برلوحه های تیره‌ها و بر حاشیه هایش، کروبیان وشیران و درختان خرما را به مقدار هریک نقش کرد و بسته‌ها گرداگردش بود.
\par 37 به این طور آن ده پایه را ساخت که همه آنها را یک ریخت و یک پیمایش و یک شکل بود.
\par 38 و ده حوض برنجین ساخت که هر حوض گنجایش چهل بت داشت. و هر حوض چهارذراعی بود و بر هر پایه‌ای از آن ده پایه، یک حوض بود.
\par 39 و پنج پایه را به‌جانب راست خانه و پنج را به‌جانب چپ خانه گذاشت و دریاچه رابه‌جانب راست خانه به سوی مشرق از طرف جنوب گذاشت.
\par 40 و حیرام، حوضها و خاک اندازها و کاسه هارا ساخت. پس حیرام تمام کاری که برای سلیمان پادشاه به جهت خانه خداوند می‌کرد به انجام رسانید.
\par 41 دو ستون و دو پیاله تاجهایی که بر سردو ستون بود و دو شبکه به جهت پوشانیدن دوپیاله تاجهایی که بر سر ستونها بود.
\par 42 وچهارصد انار برای دو شبکه که دو صف انار برای هر شبکه بود به جهت پوشانیدن دو پیاله تاجهایی که بالای ستونها بود،
\par 43 و ده پایه و ده حوضی که بر پایه‌ها بود،
\par 44 و یک دریاچه و دوازده گاو زیردریاچه.
\par 45 و دیگها و خاک اندازها و کاسه‌ها، یعنی همه این ظروفی که حیرام برای سلیمان پادشاه در خانه خداوند ساخت از برنج صیقلی بود.
\par 46 آنها را پادشاه در صحرای اردن در کل رست که در میان سکوت و صرطان است، ریخت.
\par 47 و سلیمان تمامی این ظروف را بی‌وزن واگذاشت زیرا چونکه از حد زیاده بود، وزن برنج دریافت نشد.
\par 48 و سلیمان تمامی آلاتی که در خانه خداوندبود ساخت، مذبح را از طلا و میز را که نان تقدمه بر آن بود از طلا.
\par 49 و شمعدانها را که پنج از آنهابه طرف راست و پنج به طرف چپ روبروی محراب بود، از طلای خالص و گلها و چراغها وانبرها را از طلا،
\par 50 و طاسها و گلگیرها و کاسه‌ها وقاشقها و مجمرها را از طلای خالص و پاشنه‌ها راهم به جهت درهای خانه اندرونی، یعنی به جهت قدس‌الاقداس و هم به جهت درهای خانه، یعنی هیکل، از طلا ساخت.پس تمامی کاری که سلیمان پادشاه برای خانه خداوند ساخت تمام شد و سلیمان چیزهایی را که پدرش داود وقف کرده بود، ازنقره و طلا و آلات درآورده، در خزینه های خانه خداوند گذاشت.
\par 51 پس تمامی کاری که سلیمان پادشاه برای خانه خداوند ساخت تمام شد و سلیمان چیزهایی را که پدرش داود وقف کرده بود، ازنقره و طلا و آلات درآورده، در خزینه های خانه خداوند گذاشت.
 
\chapter{8}

\par 1 آنگاه سلیمان، مشایخ اسرائیل و جمیع روسای اسباط و سروران خانه های آبای بنی‌اسرائیل را نزد سلیمان پادشاه در اورشلیم جمع کرد تا تابوت عهد خداوند را از شهر داودکه صهیون باشد، برآورند.
\par 2 و جمیع مردان اسرائیل در ماه ایتانیم که ماه هفتم است در عیدنزد سلیمان پادشاه جمع شدند.
\par 3 و جمیع مشایخ اسرائیل آمدند و کاهنان تابوت را برداشتند.
\par 4 وتابوت خداوند و خیمه اجتماع و همه آلات مقدس را که در خیمه بود آوردند و کاهنان ولاویان آنها را برآوردند.
\par 5 و سلیمان پادشاه وتمامی جماعت اسرائیل که نزد وی جمع شده بودند، پیش روی تابوت همراه وی ایستادند، واینقدر گوسفند و گاو را ذبح کردند که به شمار وحساب نمی آمد.
\par 6 و کاهنان تابوت عهد خداوندرا به مکانش در محراب خانه، یعنی درقدس‌الاقداس زیر بالهای کروبیان درآوردند.
\par 7 زیرا کروبیان بالهای خود را بر مکان تابوت پهن می‌کردند و کروبیان تابوت و عصاهایش را از بالامی پوشانیدند.
\par 8 و عصاها اینقدر دراز بود که سرهای عصاها از قدسی که پیش محراب بود، دیده می‌شد اما از بیرون دیده نمی شد و تا امروزدر آنجا هست.
\par 9 و در تابوت چیزی نبود سوای آن دو لوح سنگ که موسی در حوریب در آن گذاشت، وقتی که خداوند با بنی‌اسرائیل در حین بیرون آمدن ایشان از زمین مصر عهد بست.
\par 10 وواقع شد که چون کاهنان از قدس بیرون آمدند ابر، خانه خداوند را پر ساخت.
\par 11 و کاهنان به‌سبب ابر نتوانستند به جهت خدمت بایستند، زیرا که جلال یهوه، خانه خداوند را پر کرده بود.
\par 12 آنگاه سلیمان گفت: «خداوند گفته است که در تاریکی غلیظ ساکن می‌شوم.
\par 13 فی الواقع خانه‌ای برای سکونت تو و مکانی را که در آن تا به ابد ساکن شوی بنا نموده‌ام.»
\par 14 و پادشاه روی خود را برگردانیده، تمامی جماعت اسرائیل را برکت داد و تمامی جماعت اسرائیل بایستادند.
\par 15 پس گفت: «یهوه خدای اسرائیل متبارک باد که به دهان خود به پدر من داود وعده داده، و به‌دست خود آن را به‌جاآورده، گفت:
\par 16 از روزی که قوم خود اسرائیل رااز مصر برآوردم، شهری از جمیع اسباط اسرائیل برنگزیدم تا خانه‌ای بنا نمایم که اسم من در آن باشد، اما داود را برگزیدم تا پیشوای قوم من اسرائیل بشود.
\par 17 و در دل پدرم، داود بود که خانه‌ای برای اسم یهوه، خدای اسرائیل، بنانماید.
\par 18 اما خداوند به پدرم داود گفت: چون دردل تو بود که خانه‌ای برای اسم من بنا نمایی، نیکوکردی که این را در دل خود نهادی.
\par 19 لیکن توخانه را بنا نخواهی نمود بلکه پسر تو که از صلب تو بیرون آید، او خانه را برای اسم من بنا خواهدکرد.
\par 20 پس خداوند کلامی را که گفته بود ثابت گردانید، و من به‌جای پدر خود داود برخاسته، وبر وفق آنچه خداوند گفته بود بر کرسی اسرائیل نشسته‌ام، و خانه را به اسم یهوه، خدای اسرائیل، بنا کرده‌ام.
\par 21 و در آن، مکانی مقرر کرده‌ام برای تابوتی که عهد خداوند در آن است که آن را باپدران ما حین بیرون آوردن ایشان از مصر بسته بود.»
\par 22 و سلیمان پیش مذبح خداوند به حضورتمامی جماعت اسرائیل ایستاده، دستهای خودرا به سوی آسمان برافراشت
\par 23 و گفت: «ای یهوه، خدای اسرائیل، خدایی مثل تو نه بالا درآسمان و نه پایین بر زمین هست که با بندگان خودکه به حضور تو به تمامی دل خویش سلوک می‌نمایند، عهد و رحمت را نگاه می‌داری.
\par 24 وآن وعده‌ای که به بنده خود، پدرم داود داده‌ای، نگاه داشته‌ای زیرا به دهان خود وعده دادی و به‌دست خود آن را وفا نمودی چنانکه امروز شده است.
\par 25 پس الان‌ای یهوه، خدای اسرائیل، بابنده خود، پدرم داود، آن وعده‌ای را نگاه دار که به او داده و گفته‌ای کسی‌که بر کرسی اسرائیل بنشیند برای تو به حضور من منقطع نخواهد شد، به شرطی که پسرانت طریق های خود را نگاه داشته، به حضور من سلوک نمایند چنانکه تو به حضورم رفتار نمودی.
\par 26 و الان‌ای خدای اسرائیل تمنا اینکه کلامی که به بنده خود، پدرم داود گفته‌ای، ثابت بشود.
\par 27 «اما آیا خدا فی الحقیقه بر زمین ساکن خواهد شد؟ اینک فلک و فلک الافلاک تو راگنجایش ندارد تا چه رسد به این خانه‌ای که من بناکرده‌ام.
\par 28 لیکن‌ای یهوه، خدای من، به دعا وتضرع بنده خود توجه نما و استغاثه و دعایی راکه بنده ات امروز به حضور تو می‌کند، بشنو،
\par 29 تاآنکه شب و روز چشمان تو بر این خانه باز شود وبر مکانی که درباره‌اش گفتی که اسم من در آنجاخواهد بود و تا دعایی را که بنده ات به سوی این مکان بنماید، اجابت کنی.
\par 30 و تضرع بنده ات و قوم خود اسرائیل را که به سوی این مکان دعامی نمایند، بشنو و از مکان سکونت خود، یعنی ازآسمان بشنو و چون شنیدی عفو نما.
\par 31 «اگر کسی به همسایه خود گناه ورزد وقسم بر او عرضه شود که بخورد و او آمده پیش مذبح تو در این خانه قسم خورد،
\par 32 آنگاه ازآسمان بشنو و عمل نموده، به جهت بندگانت حکم نما و شریران را ملزم ساخته، راه ایشان را به‌سر ایشان برسان و عادلان را عادل شمرده، ایشان را بر‌حسب عدالت ایشان جزا ده.
\par 33 «و هنگامی که قوم تو اسرائیل به‌سبب گناهی که به تو ورزیده باشند به حضور دشمنان خود مغلوب شوند اگر به سوی تو بازگشت نموده، اسم تو را اعتراف نمایند و نزد تو در این خانه دعا و تضرع نمایند،
\par 34 آنگاه از آسمان بشنوو گناه قوم خود، اسرائیل را بیامرز و ایشان را به زمینی که به پدران ایشان داده‌ای بازآور.
\par 35 «هنگامی که آسمان بسته شود و به‌سبب گناهی که به تو ورزیده باشند باران نبارد، اگر به سوی این مکان دعا کنند و اسم تو را اعتراف نمایند و به‌سبب مصیبتی که به ایشان رسانیده باشی از گناه خویش بازگشت کنند،
\par 36 آنگاه ازآسمان بشنو و گناه بندگانت و قوم خود اسرائیل را بیامرز و ایشان را به راه نیکو که در آن باید رفت، تعلیم ده و به زمین خود که آن را به قوم خویش برای میراث بخشیده‌ای، باران بفرست.
\par 37 «اگر در زمین قحطی باشد و اگر وبا یا بادسموم یا یرقان باشد و اگر ملخ یا کرم باشد و اگردشمنان ایشان، ایشان را در شهرهای زمین ایشان محاصره نمایند، هر بلایی یا هر مرضی که بوده باشد،
\par 38 آنگاه هر دعا و هر استغاثه‌ای که از هرمرد یا از تمامی قوم تو، اسرائیل، کرده شود که هریک از ایشان بلای دل خود را خواهند دانست، و دستهای خود را به سوی این خانه دراز نمایند،
\par 39 آنگاه از آسمان که مکان سکونت تو باشد بشنوو بیامرز و عمل نموده، به هر کس که دل او رامی دانی به حسب راههایش جزا بده، زیرا که تو به تنهایی عارف قلوب جمیع بنی آدم هستی.
\par 40 تاآنکه ایشان در تمام روزهایی که به روی زمینی که به پدران ما داده‌ای زنده باشند، از توبترسند.
\par 41 «و نیز غریبی که از قوم تو، اسرائیل، نباشدو به‌خاطر اسم تو از زمین بعید آمده باشد،
\par 42 زیراکه آوازه اسم عظیمت و دست قویت و بازوی دراز تو را خواهند شنید، پس چون بیاید و به سوی این خانه دعا نماید،
\par 43 آنگاه از آسمان که مکان سکونت توست بشنو و موافق هر‌چه آن غریب از تو استدعا نماید به عمل آور تا جمیع قومهای جهان اسم تو را بشناسند و مثل قوم تو، اسرائیل، از تو بترسند و بدانند که اسم تو بر این خانه‌ای که بنا کرده‌ام، نهاده شده است.
\par 44 «اگر قوم تو برای مقاتله با دشمنان خود به راهی که ایشان را فرستاده باشی بیرون روند وایشان به سوی شهری که تو برگزیده‌ای و خانه‌ای که به جهت اسم تو بنا کرده‌ام، نزد خداوند دعانمایند،
\par 45 آنگاه دعا و تضرع ایشان را از آسمان بشنو و حق ایشان را بجا آور.
\par 46 «و اگر به تو گناه ورزیده باشند زیرا انسانی نیست که گناه نکند و تو بر ایشان غضبناک شده، ایشان را به‌دست دشمنان تسلیم کرده باشی واسیرکنندگان ایشان، ایشان را به زمین دشمنان خواه دور و خواه نزدیک به اسیری ببرند،
\par 47 پس اگر ایشان در زمینی که در آن اسیر باشند به خودآمده، بازگشت نمایند و در زمین اسیری خود نزدتو تضرع نموده، گویند که گناه کرده، و عصیان ورزیده، و شریرانه رفتار نموده‌ایم، 
\par 48 و در زمین دشمنانی که ایشان را به اسیری برده باشند به تمامی دل و به تمامی جان خود به تو بازگشت نمایند، و به سوی زمینی که به پدران ایشان داده‌ای و شهری که برگزیده و خانه‌ای که برای اسم تو بنا کرده‌ام، نزد تو دعا نمایند،
\par 49 آنگاه ازآسمان که مکان سکونت توست، دعا و تضرع ایشان را بشنو و حق ایشان را بجا آور.
\par 50 و قوم خود را که به تو گناه ورزیده باشند، عفو نما وتمامی تقصیرهای ایشان را که به تو ورزیده باشندبیامرز و ایشان را در دل اسیرکنندگان ایشان ترحم عطا فرما تا بر ایشان ترحم نمایند.
\par 51 زیراکه ایشان قوم تو و میراث تو می‌باشند که ازمصر از میان کوره آهن بیرون آوردی.
\par 52 تاچشمان تو به تضرع بنده ات و به تضرع قوم تواسرائیل گشاده شود و ایشان را در هر‌چه نزدتو دعا نمایند، اجابت نمایی.
\par 53 زیرا که توایشان را از جمیع قومهای جهان برای ارثیت خویش ممتاز نموده‌ای چنانکه به واسطه بنده خود موسی وعده دادی هنگامی که تو‌ای خداوند یهوه پدران ما را از مصر بیرون آوردی.»
\par 54 و واقع شد که چون سلیمان از گفتن تمامی این دعا و تضرع نزد خداوند فارغ شد، از پیش مذبح خداوند از زانو زدن و دراز نمودن دستهای خود به سوی آسمان برخاست،
\par 55 و ایستاده، تمامی جماعت اسرائیل را به آواز بلند برکت دادو گفت:
\par 56 «متبارک باد خداوند که قوم خود، اسرائیل را موافق هر‌چه وعده کرده بود، آرامی داده است زیرا که از تمامی وعده های نیکو که به واسطه بنده خود، موسی داده بود، یک سخن به زمین نیفتاد.
\par 57 یهوه خدای ما با ما باشد چنانکه با پدران مامی بود و ما را ترک نکند و رد نماید.
\par 58 و دلهای مارا به سوی خود مایل بگرداند تا در تمامی طریق هایش سلوک نموده، اوامر و فرایض واحکام او را که به پدران ما امر فرموده بود، نگاه داریم.
\par 59 و کلمات این دعایی که نزد خداوندگفته‌ام، شب و روز نزدیک یهوه خدای ما باشد تاحق بنده خود و حق قوم خویش اسرائیل را برحسب اقتضای هر روز بجا آورد.
\par 60 تا تمامی قوم های جهان بدانند که یهوه خداست و دیگری نیست.
\par 61 پس دل شما با یهوه خدای ما کامل باشد تا در فرایض او سلوک نموده، اوامر او رامثل امروز نگاه دارید.»
\par 62 پس پادشاه و تمامی اسرائیل با وی به حضور خداوند قربانی‌ها گذرانیدند.
\par 63 و سلیمان به جهت ذبایح سلامتی که برای خداوند گذارنید، بیست و دو هزار گاو و صد و بیست هزار گوسفندذبح نمود و پادشاه و جمیع بنی‌اسرائیل، خانه خداوند را تبریک نمودند.
\par 64 و در آن روز پادشاه وسط صحن را که پیش خانه خداوند است تقدیس نمود زیرا چونکه مذبح برنجینی که به حضور خداوند بود به جهت گنجایش قربانی های سوختنی و هدایای آردی و پیه قربانی های سلامتی کوچک بود، از آن جهت قربانی های سوختنی و هدایای آردی و پیه ذبایح سلامتی را در آنجا گذرانید.
\par 65 و در آن وقت سلیمان و تمامی اسرائیل باوی عید را نگاه داشتند و آن انجمن بزرگ ازمدخل حمات تا وادی مصر هفت روز و هفت روز یعنی چهارده روز به حضور یهوه، خدای مابودند.و در روز هشتم، قوم را مرخص فرمودو ایشان برای پادشاه برکت خواسته، و با شادمانی و خوشدلی به‌سبب تمامی احسانی که خداوند به بنده خود، داود و به قوم خویش اسرائیل نموده بود، به خیمه های خود رفتند.
\par 66 و در روز هشتم، قوم را مرخص فرمودو ایشان برای پادشاه برکت خواسته، و با شادمانی و خوشدلی به‌سبب تمامی احسانی که خداوند به بنده خود، داود و به قوم خویش اسرائیل نموده بود، به خیمه های خود رفتند.
 
\chapter{9}

\par 1 و واقع شد که چون سلیمان از بنا نمودن خانه خداوند و خانه پادشاه و از بجا آوردن هر مقصودی که سلیمان خواسته بود، فارغ شد،
\par 2 خداوند بار دیگر به سلیمان ظاهر شد، چنانکه در جبعون بر وی ظاهر شده بود.
\par 3 و خداوند وی را گفت: «دعا و تضرع تو را که به حضور من کردی، اجابت نمودم، و این خانه‌ای را که بنانمودی تا نام من در آن تا به ابد نهاده شود تقدیس نمودم، و چشمان و دل من همیشه اوقات در آن خواهد بود.
\par 4 پس اگر تو با دل کامل و استقامت به طوری که پدرت داود رفتار نمود به حضور من سلوک نمایی، و هر‌چه تو را امر فرمایم بجا آوری و فرایض و احکام مرا نگاه داری،
\par 5 آنگاه کرسی سلطنت تو را بر اسرائیل تا به ابد برقرار خواهم گردانید، چنانکه به پدر تو داود وعده دادم و گفتم که از تو کسی‌که بر کرسی اسرائیل بنشیند، مفقودنخواهد شد.
\par 6 «اما اگر شما و پسران شما از متابعت من روگردانیده، اوامر و فرایضی را که به پدران شمادادم نگاه ندارید و رفته، خدایان دیگر را عبادت نموده، آنها را سجده کنید،
\par 7 آنگاه اسرائیل را ازروی زمینی که به ایشان دادم منقطع خواهم ساخت، و این خانه را که به جهت اسم خودتقدیس نمودم از حضور خویش دور خواهم‌انداخت، و اسرائیل در میان جمیع قومهاضرب‌المثل و مضحکه خواهد شد.
\par 8 و این خانه عبرتی خواهد گردید به طوری که هر‌که نزد آن بگذرد، متحیر شده، صفیر خواهد زد و خواهندگفت: خداوند به این زمین و به این خانه چرا چنین عمل نموده است؟
\par 9 و خواهند گفت: از این جهت که یهوه، خدای خود را که پدران ایشان را از زمین مصر بیرون آورده بود، ترک کردند و به خدایان دیگر متمسک شده، آنها را سجده و عبادت نمودند. لهذا خداوند تمامی این بلا را بر ایشان آورده است.»
\par 10 و واقع شد بعد از انقضای بیست سالی که سلیمان این دو خانه، یعنی خانه خداوند و خانه پادشاه را بنا می‌کرد،
\par 11 و حیرام، پادشاه صور، سلیمان را به چوب سرو آزاد و چوب صنوبر وطلا موافق هر‌چه خواسته بود اعانت کرده بود، آنگاه سلیمان پادشاه بیست شهر در زمین جلیل به حیرام داد.
\par 12 و حیرام به جهت دیدن شهرهایی که سلیمان به او داده بود، از صور بیرون آمد اماآنها به نظرش پسند نیامد.
\par 13 و گفت: «ای برادرم این شهرهایی که به من بخشیده‌ای چیست؟» وآنها را تا امروز زمین کابول نامید.
\par 14 و حیرام صد و بیست وزنه طلا برای پادشاه فرستاد.
\par 15 و این است حساب سخره‌ای که سلیمان پادشاه گرفته بود به جهت بنای خانه خداوند وخانه خود و ملو و حصارهای اورشلیم و حاصورو مجدو و جازر.
\par 16 زیرا که فرعون، پادشاه مصربرآمده، جازر را تسخیر نموده، و آن را به آتش سوزانیده، و کنعانیان را که در شهر ساکن بودندکشته بود، و آن را به دختر خود که زن سلیمان بودبه جهت مهر داده بود.
\par 17 و سلیمان، جازر وبیت حورون تحتانی را بنا کرد.
\par 18 و بعلت و تدمررا در صحرای زمین،
\par 19 و جمیع شهرهای مخزنی که سلیمان داشت و شهرهای ارابه‌ها وشهرهای سواران را و هر‌آنچه را که سلیمان میل داشت که در اورشلیم و لبنان و تمامی زمین مملکت خود بنا نماید (بنا نمود).
\par 20 و تمامی مردمانی که از اموریان و حتیان و فرزیان و حویان و یبوسیان باقی‌مانده، و از بنی‌اسرائیل نبودند،
\par 21 یعنی پسران ایشان که در زمین باقی ماندند بعداز آنانی که بنی‌اسرائیل نتوانستند ایشان را بالکل هلاک سازند، سلیمان ایشان را تا امروزخراج گذار و غلامان ساخت.
\par 22 اما ازبنی‌اسرائیل، سلیمان احدی را به غلامی نگرفت، بلکه ایشان مردان جنگی و خدام و سروران وسرداران و روسای ارابه‌ها و سواران او بودند.
\par 23 و اینانند ناظران خاصه که بر کارهای سلیمان بودند، پانصد و پنجاه نفر که بر اشخاصی که در کار مشغول می‌بودند، سرکاری داشتند.
\par 24 پس دختر فرعون از شهر داود به خانه خودکه برایش بنا کرده بود، برآمد، و در آن زمان ملو رابنا می‌کرد.
\par 25 و سلیمان هر سال سه مرتبه قربانی های سوختنی و ذبایح سلامتی بر مذبحی که به جهت خداوند بنا کرده بود می‌گذرانید، و بر مذبحی که پیش خداوند بود، بخور می‌گذرانید. پس خانه رابه اتمام رسانید.
\par 26 و سلیمان پادشاه در عصیون جابر که به‌جانب ایلوت بر کناره بحر قلزم در زمین ادوم است، کشتیها ساخت.
\par 27 و حیرام، بندگان خود راکه ملاح بودند و در دریا مهارت داشتند، درکشتیها همراه بندگان سلیمان فرستاد.پس به اوفیر رفتند و چهارصد و بیست وزنه طلا از آنجاگرفته، برای سلیمان پادشاه آوردند.
\par 28 پس به اوفیر رفتند و چهارصد و بیست وزنه طلا از آنجاگرفته، برای سلیمان پادشاه آوردند.
 
\chapter{10}

\par 1 و چون ملکه سبا آوازه سلیمان رادرباره اسم خداوند شنید، آمد تا او رابه مسائل امتحان کند.
\par 2 پس با موکب بسیار عظیم و با شترانی که به عطریات و طلای بسیار وسنگهای گرانبها بار شده بود به اورشلیم واردشده، به حضور سلیمان آمد و با وی از هر‌چه دردلش بود، گفتگو کرد.
\par 3 و سلیمان تمامی مسائلش را برایش بیان نمود و چیزی از پادشاه مخفی نماند که برایش بیان نکرد.
\par 4 و چون ملکه سبا تمامی حکمت سلیمان را دید و خانه‌ای را که بنا کرده بود،
\par 5 و طعام سفره او و مجلس بندگانش را و نظام و لباس خادمانش را و ساقیانش وزینه‌ای را که به آن به خانه خداوند برمی آمد، روح در او دیگر نماند.
\par 6 و به پادشاه گفت: «آوازه‌ای که درباره کارها وحکمت تو در ولایت خود شنیدم، راست بود.
\par 7 اما تا نیامدم و به چشمان خود ندیدم، اخبار راباور نکردم، و اینک نصفش به من اعلام نشده بود؛ حکمت و سعادتمندی تو از خبری که شنیده بودم، زیاده است.
\par 8 خوشابه‌حال مردان تو وخوشابه‌حال این بندگانت که به حضور تو همیشه می‌ایستند و حکمت تو را می‌شنوند.
\par 9 متبارک بادیهوه، خدای تو، که بر تو رغبت داشته، تو را برکرسی اسرائیل نشانید.
\par 10 از این سبب که خداوند، اسرائیل را تا به ابد دوست می‌دارد، تو رابر پادشاهی نصب نموده است تا داوری و عدالت را بجا آوری.»
\par 11 و به پادشاه صد و بیست وزنه طلا وعطریات از حد زیاده و سنگهای گرانبها داد، ومثل این عطریات که ملکه سبا به سلیمان پادشاه داد، هرگز به آن فراوانی دیگر نیامد.
\par 12 وکشتیهای حیرام نیز که طلا از اوفیر آوردند، چوب صندل از حد زیاده، و سنگهای گرانبها ازاوفیر آوردند.
\par 13 و پادشاه از این چوب صندل، ستونها به جهت خانه خداوند و خانه پادشاه و عودها وبربطها برای مغنیان ساخت، و مثل این چوب صندل تا امروز نیامده و دیده نشده است.
\par 14 و سلیمان پادشاه به ملکه سبا، تمامی اراده او را که خواسته بود داد، سوای آنچه سلیمان ازکرم ملوکانه خویش به وی بخشید. پس او بابندگانش به ولایت خود توجه نموده، رفت.
\par 15 و وزن طلایی که در یک سال نزد سلیمان رسید ششصد و شصت و شش وزنه طلا بود.
\par 16 سوای آنچه از تاجران و تجارت بازرگانان وجمیع پادشاهان عرب و حاکمان مملکت می‌رسید.
\par 17 و سلیمان پادشاه دویست سپرطلای چکشی ساخت که برای هر سپر ششصدمثقال طلا به‌کار برده شد، و سیصد سپر کوچک طلای چکشی ساخت که برای هر سپر سه منای طلا به‌کار برده شد، و پادشاه آنها را در خانه جنگل لبنان گذاشت.
\par 18 و پادشاه تخت بزرگی ازعاج ساخت و آن را به زر خالص پوشانید.
\par 19 وتخت را شش پله بود و سر تخت از عقبش مدوربود، و به این طرف و آن طرف کرسی‌اش دسته هابود و دو شیر به پهلوی دستها ایستاده بودند.
\par 20 وآنجا دوازده شیر از این طرف و آن طرف بر آن شش پله ایستاده بودند که در هیچ مملکت مثل این ساخته نشده بود.
\par 21 و تمامی ظروف نوشیدنی سلیمان پادشاه از طلا و تمامی ظروف خانه جنگل لبنان از زر خالص بود و هیچ‌یکی ازآنها از نقره نبود زیرا که آن در ایام سلیمان هیچ به حساب نمی آمد.
\par 22 زیرا پادشاه کشتیهای ترشیشی با کشتیهای حیرام به روی دریا داشت. وکشتیهای ترشیشی هر سال یک مرتبه می‌آمدند وطلا و نقره و عاج و میمونها و طاووسهامی آوردند.
\par 23 پس سلیمان پادشاه در دولت و حکمت ازجمیع پادشاهان جهان بزرگتر شد.
\par 24 و تمامی اهل جهان، حضور سلیمان را می‌طلبیدند تاحکمتی را که خداوند در دلش نهاده بود، بشنوند.
\par 25 و هر یکی از ایشان هدیه خود را از آلات نقره و آلات طلا و رخوت و اسلحه و عطریات واسبان و قاطرها، سال به سال می‌آوردند.
\par 26 و سلیمان ارابه‌ها و سواران جمع کرده، هزار و چهارصد ارابه و دوازده هزار سوار داشت و آنها را در شهرهای ارابه‌ها و نزد پادشاه دراورشلیم گذاشت. 
\par 27 و پادشاه نقره را در اورشلیم مثل سنگها و چوب سرو آزاد را مثل چوب افراغ که در صحراست، فراوان ساخت.
\par 28 و اسبهای سلیمان از مصر آورده می‌شد، و تاجران پادشاه دسته های آنها را می‌خریدند هر دسته را به قیمت معین.و یک ارابه را به قیمت ششصد مثقال نقره از مصر بیرون آوردند، و می‌رسانیدند و یک اسب را به قیمت صد و پنجاه، و همچنین برای جمیع پادشاهان حتیان و پادشاهان ارام به توسطآنها بیرون می‌آوردند.
\par 29 و یک ارابه را به قیمت ششصد مثقال نقره از مصر بیرون آوردند، و می‌رسانیدند و یک اسب را به قیمت صد و پنجاه، و همچنین برای جمیع پادشاهان حتیان و پادشاهان ارام به توسطآنها بیرون می‌آوردند.
 
\chapter{11}

\par 1 و سلیمان پادشاه سوای دختر فرعون، زنان غریب بسیاری را از موآبیان و عمونیان وادومیان و صیدونیان و حتیان دوست می‌داشت.
\par 2 از امتهایی که خداوند درباره ایشان بنی‌اسرائیل را فرموده بود که شما به ایشان درنیایید و ایشان به شما درنیایند مبادا دل شما را به پیروی خدایان خود مایل گردانند. و سلیمان با اینها به محبت ملصق شد.
\par 3 و او را هفتصد زن بانو و سیصد متعه بود و زنانش دل او را برگردانیدند.
\par 4 و در وقت پیری سلیمان واقع شد که زنانش دل او را به پیروی خدایان غریب مایل ساختند، و دل او مثل دل پدرش داود با یهوه، خدایش کامل نبود.
\par 5 پس سلیمان در عقب عشتورت، خدای صیدونیان، ودر عقب ملکوم رجس عمونیان رفت.
\par 6 و سلیمان در نظر خداوند شرارت ورزیده، مثل پدر خود داود، خداوند را پیروی کامل ننمود.
\par 7 آنگاه سلیمان در کوهی که روبروی اورشلیم است مکانی بلند به جهت کموش که رجس موآبیان است، و به جهت مولک، رجس بنی عمون بنا کرد.
\par 8 و همچنین به جهت همه زنان غریب خود که برای خدایان خویش بخور می‌سوزانیدند وقربانی‌ها می‌گذرانیدند، عمل نمود.
\par 9 پس خشم خداوند بر سلیمان افروخته شداز آن جهت که دلش از یهوه، خدای اسرائیل منحرف گشت که دو مرتبه بر او ظاهر شده،
\par 10 اورا در همین باب امر فرموده بود که پیروی خدایان غیر را ننماید اما آنچه خداوند به او امر فرموده بود، به‌جا نیاورد.
\par 11 پس خداوند به سلیمان گفت: «چونکه این عمل را نمودی و عهد وفرایض مرا که به تو امر فرمودم نگاه نداشتی، البته سلطنت را از تو پاره کرده، آن را به بنده ات خواهم داد.
\par 12 لیکن در ایام تو این را به‌خاطر پدرت، داودنخواهم کرد اما از دست پسرت آن را پاره خواهم کرد.
\par 13 ولی تمامی مملکت را پاره نخواهم کردبلکه یک سبط را به‌خاطر بنده‌ام داود و به‌خاطراورشلیم که برگزیده‌ام به پسر تو خواهم داد.»
\par 14 و خداوند دشمنی برای سلیمان برانگیزانید، یعنی هدد ادومی را که از ذریت پادشاهان ادوم بود.
\par 15 زیرا هنگامی که داود درادوم بود و یوآب که سردار لشکر بود، برای دفن کردن کشتگان رفته بود و تمامی ذکوران ادوم راکشته بود.
\par 16 (زیرا یوآب و تمامی اسرائیل شش ماه در آنجا ماندند تا تمامی ذکوران ادوم را منقطع ساختند).
\par 17 آنگاه هدد با بعضی ادومیان که ازبندگان پدرش بودند، فرار کردند تا به مصر بروند، و هدد طفلی کوچک بود.
\par 18 پس، از مدیان روانه شده، به فاران آمدند، و چند نفر از فاران با خودبرداشته، به مصر نزد فرعون، پادشاه مصر آمدند، و او وی را خانه‌ای داد و معیشتی برایش تعیین نمود و زمینی به او ارزانی داشت.
\par 19 و هدد درنظر فرعون التفات بسیار یافت و خواهر زن خود، یعنی خواهر تحفنیس ملکه را به وی به زنی داد.
\par 20 و خواهر تحفنیس پسری جنوبت نام برای وی زایید و تحفنیس او را در خانه فرعون از شیربازداشت و جنوبت در خانه فرعون در میان پسران فرعون می‌بود.
\par 21 و چون هدد در مصر شنید که داود با پدران خویش خوابیده، و یوآب، سردارلشکر مرده است، هدد به فرعون گفت: «مرارخصت بده تا به ولایت خود بروم.»
\par 22 فرعون وی را گفت: «اما تو را نزد من چه چیز کم است که اینک می‌خواهی به ولایت خود بروی؟» گفت: «هیچ، لیکن مرا البته مرخص نما.»
\par 23 و خدا دشمنی دیگر برای وی برانگیزانید، یعنی رزون بن الیداع را که از نزد آقای خویش، هددعزر، پادشاه صوبه فرار کرده بود.
\par 24 و مردان چندی نزد خود جمع کرده، سردار فوجی شدهنگامی که داود بعضی ایشان را کشت. پس به دمشق رفتند و در آنجا ساکن شده، در دمشق حکمرانی نمودند.
\par 25 و او در تمامی روزهای سلیمان، دشمن اسرائیل می‌بود، علاوه بر ضرری که هدد می‌رسانید و از اسرائیل نفرت داشته، برارام سلطنت می‌نمود.
\par 26 و یربعام بن نباط افرایمی از صرده که بنده سلیمان و مادرش مسمی به صروعه و بیوه‌زنی بود، دست خود را نیز به ضد پادشاه بلند کرد.
\par 27 و سبب آنکه دست خود را به ضد پادشاه بلند کرد، این بود که سلیمان ملو را بنا می‌کرد، و رخنه شهرپدر خود داود را تعمیر می‌نمود.
\par 28 و یربعام مردشجاع جنگی بود. پس چون سلیمان آن جوان رادید که در کار مردی زرنگ بود او را بر تمامی امور خاندان یوسف بگماشت.
\par 29 و در آن زمان واقع شد که یربعام از اورشلیم بیرون می‌آمد واخیای شیلونی نبی در راه به او برخورد، و جامه تازه‌ای در برداشت و ایشان هر دو در صحرا تنهابودند.
\par 30 پس اخیا جامه تازه‌ای که در برداشت گرفته، آن را به دوازده قسمت پاره کرد.
\par 31 و به یربعام گفت: «ده قسمت برای خود بگیر زیرا که یهوه، خدای اسرائیل چنین می‌گوید، اینک من مملکت را از دست سلیمان پاره می‌کنم و ده سبطبه تو می‌دهم.
\par 32 و به‌خاطر بنده من، داود و به‌خاطر اورشلیم، شهری که از تمامی اسباطبنی‌اسرائیل برگزیده‌ام، یک سبط از آن او خواهدبود.
\par 33 چونکه ایشان مرا ترک کردند و عشتورت، خدای صیدونیان، و کموش، خدای موآب، وملکوم، خدای بنی عمون را سجده کردند، و درطریقهای من سلوک ننمودند و آنچه در نظر من راست است، بجا نیاوردند و فرایض و احکام مرامثل پدرش، داود نگاه نداشتند.
\par 34 لیکن تمام مملکت را از دست او نخواهم گرفت بلکه به‌خاطر بنده خود داود که او را برگزیدم، از آنرو که اوامر و فرایض مرا نگاه داشته بود، او را در تمامی ایام روزهایش سرور خواهم ساخت.
\par 35 اماسلطنت را از دست پسرش گرفته، آن را یعنی ده سبط به تو خواهم داد.
\par 36 و یک سبط به پسرش خواهم بخشید تا بنده من، داود در اورشلیم، شهری که برای خود برگزیده‌ام تا اسم خود را درآن بگذارم، نوری در حضور من همیشه داشته باشد.
\par 37 و تو را خواهم گرفت تا موافق هر‌چه دلت آرزو دارد، سلطنت نمایی و بر اسرائیل پادشاه شوی.
\par 38 و واقع خواهد شد که اگر هر‌چه تو را امر فرمایم، بشنوی و به طریق هایم سلوک نموده، آنچه در نظرم راست است بجا آوری وفرایض و اوامر مرا نگاه داری چنانکه بنده من، داود آنها را نگاه داشت، آنگاه با تو خواهم بود وخانه‌ای مستحکم برای تو بنا خواهم نمود، چنانکه برای داود بنا کردم و اسرائیل را به توخواهم بخشید.
\par 39 و ذریت داود را به‌سبب این امر ذلیل خواهم ساخت اما نه تا به ابد.»
\par 40 پس سلیمان قصد کشتن یربعام داشت و یربعام برخاسته، به مصر نزد شیشق، پادشاه مصر فرارکرد و تا وفات سلیمان در مصر ماند.
\par 41 و بقیه امور سلیمان و هر‌چه کرد و حکمت او، آیا آنها در کتاب وقایع سلیمان مکتوب نیست؟
\par 42 و ایامی که سلیمان در اورشلیم برتمامی اسرائیل سلطنت کرد، چهل سال بود.پس سلیمان با پدران خود خوابید و در شهرپدر خود داود دفن شد و پسرش رحبعام در جای او سلطنت نمود
\par 43 پس سلیمان با پدران خود خوابید و در شهرپدر خود داود دفن شد و پسرش رحبعام در جای او سلطنت نمود
 
\chapter{12}

\par 1 و رحبعام به شکیم رفت زیرا که تمامی اسرائیل به شکیم آمدند تا او را پادشاه بسازند.
\par 2 و واقع شد که چون یربعام بن نباط شنید(و او هنوز در مصر بود که از حضور سلیمان پادشاه به آنجا فرار کرده، و یربعام در مصر ساکن می‌بود.
\par 3 و ایشان فرستاده، او را خواندند)، آنگاه یربعام و تمامی جماعت اسرائیل آمدند و به رحبعام عرض کرده، گفتند:
\par 4 «پدر تو یوغ ما راسخت ساخت اما تو الان بندگی سخت و یوغ سنگینی را که پدرت بر ما نهاد سبک ساز، و تو راخدمت خواهیم نمود.»
\par 5 به ایشان گفت: «تا سه روز دیگر بروید و بعد از آن نزد من برگردید.» پس قوم رفتند.
\par 6 و رحبعام پادشاه با مشایخی که در حین حیات پدرش، سلیمان به حضورش می‌ایستادندمشورت کرده، گفت: «که شما چه صلاح می‌بینیدتا به این قوم جواب دهم؟»
\par 7 ایشان او را عرض کرده، گفتند: «اگر امروز این قوم را بنده شوی وایشان را خدمت نموده، جواب دهی و سخنان نیکو به ایشان گویی همانا همیشه اوقات بنده توخواهند بود.»
\par 8 اما مشورت مشایخ را که به اودادند ترک کرد، و با جوانانی که با او تربیت یافته بودند و به حضورش می‌ایستادند، مشورت کرد.
\par 9 و به ایشان گفت: «شما چه صلاح می‌بینید که به این قوم جواب دهیم؟ که به من عرض کرده، گفته‌اند یوغی را که پدرت بر ما نهاده است، سبک ساز.»
\par 10 و جوانانی که با او تربیت یافته بودند او راخطاب کرده، گفتند که به این قوم که به تو عرض کرده، گفته‌اند که پدرت یوغ ما را سنگین ساخته است و تو آن را برای ما سبک ساز، به ایشان چنین بگو: انگشت کوچک من از کمر پدرم کلفت تراست.
\par 11 و حال پدرم یوغ سنگین بر شما نهاده است اما من یوغ شما را زیاده خواهم گردانید. پدرم شما را به تازیانه‌ها تنبیه می‌نمود اما من شمارا به عقربها تنبیه خواهم نمود.»
\par 12 و در روز سوم، یربعام و تمامی قوم به نزدرحبعام باز آمدند، به نحوی که پادشاه فرموده و گفته بود که در روز سوم نزد من باز آیید.
\par 13 وپادشاه، قوم را به سختی جواب داد، و مشورت مشایخ را که به وی داده بودند، ترک کرد.
\par 14 وموافق مشورت جوانان ایشان را خطاب کرده، گفت: «پدرم یوغ شما را سنگین ساخت، اما من یوغ شما را زیاده خواهم گردانید. پدرم شما را به تازیانه تنبیه می‌نمود اما من شما را به عقربها تنبیه خواهم کرد.»
\par 15 و پادشاه، قوم را اجابت نکردزیرا که این امر از جانب خداوند شده بود تاکلامی را که خداوند به واسطه اخیای شیلونی به یربعام بن نباط گفته بود، ثابت گرداند.
\par 16 و چون تمامی اسرائیل دیدند که پادشاه، ایشان را اجابت نکرد آنگاه قوم، پادشاه را جواب داده، گفتند: «ما را در داود چه حصه است؟ و درپسر یسا چه نصیب؟ ای اسرائیل به خیمه های خود بروید! و اینک‌ای داود به خانه خود متوجه باش!»
\par 17 اما رحبعام بر بنی‌اسرائیل که در شهرهای یهودا ساکن بودند، سلطنت می‌نمود.
\par 18 ورحبعام پادشاه ادورام را که سردار باج گیران بود، فرستاد و تمامی اسرائیل، او را سنگسار کردند که مرد و رحبعام پادشاه تعجیل نموده، بر ارابه خودسوار شد و به اورشلیم فرار کرد.
\par 19 پس اسرائیل تا به امروز بر خاندان داود عاصی شدند.
\par 20 و چون تمامی اسرائیل شنیدند که یربعام مراجعت کرده است، ایشان فرستاده، او را نزدجماعت طلبیدند و او را بر تمام اسرائیل پادشاه ساختند، و غیر از سبط یهودا فقط، کسی خاندان داود را پیروی نکرد.
\par 21 و چون رحبعام به اورشلیم رسید، تمامی خاندان یهودا و سبط بنیامین، یعنی صد و هشتادهزار نفر برگزیده جنگ آزموده را جمع کرد تا باخاندان اسرائیل مقاتله نموده، سلطنت را به رحبعام بن سلیمان برگرداند.
\par 22 اما کلام خدا برشمعیا، مرد خدا نازل شده، گفت:
\par 23 «به رحبعام بن سلیمان، پادشاه یهودا و به تمامی خاندان یهوداو بنیامین و به بقیه قوم خطاب کرده، بگو:
\par 24 خداوند چنین می‌گوید: مروید و با برادران خود بنی‌اسرائیل جنگ منمایید، هر کس به خانه خود برگردد زیرا که این امر از جانب من شده است.» و ایشان کلام خداوند را شنیدند وبرگشته، موافق فرمان خداوند رفتار نمودند.
\par 25 و یربعام شکیم را در کوهستان افرایم بناکرده، در آن ساکن شد و از آنجا بیرون رفته، فنوئیل را بنا نمود.
\par 26 و یربعام در دل خود فکرکرد که حال سلطنت به خاندان داود خواهدبرگشت. 
\par 27 اگر این قوم به جهت گذرانیدن قربانی‌ها به خانه خداوند به اورشلیم بروند همانادل این قوم به آقای خویش، رحبعام، پادشاه یهودا خواهد برگشت و مرا به قتل رسانیده، نزدرحبعام، پادشاه یهودا خواهند برگشت.
\par 28 پس پادشاه مشورت نموده، دو گوساله طلا ساخت وبه ایشان گفت: «برای شما رفتن تا به اورشلیم زحمت است، هان‌ای اسرائیل خدایان تو که تو رااز زمین مصر برآوردند!»
\par 29 و یکی را در بیت ئیل گذاشت و دیگری را در دان قرار داد.
\par 30 و این امرباعث گناه شد و قوم پیش آن یک تا دان می‌رفتند.
\par 31 و خانه‌ها در مکانهای بلند ساخت و از تمامی قوم که از بنی لاوی نبودند، کاهنان تعیین نمود.
\par 32 و یربعام عیدی در ماه هشتم در روز پانزدهم ماه مثل عیدی که در یهوداست برپا کرد و نزد آن مذبح می‌رفت و در بیت ئیل به همان طور عمل نموده، برای گوساله هایی که ساخته بود، قربانی می‌گذرانید. و کاهنان مکانهای بلند را که ساخته بود، در بیت ئیل قرار داد.و در روز پانزدهم ماه هشتم، یعنی در ماهی که از دل خود ابداع نموده بود، نزد مذبح که در بیت ئیل ساخته بود می‌رفت، و برای بنی‌اسرائیل عید برپا نموده، نزد مذبح برآمده، بخور می‌سوزانید.
\par 33 و در روز پانزدهم ماه هشتم، یعنی در ماهی که از دل خود ابداع نموده بود، نزد مذبح که در بیت ئیل ساخته بود می‌رفت، و برای بنی‌اسرائیل عید برپا نموده، نزد مذبح برآمده، بخور می‌سوزانید.
 
\chapter{13}

\par 1 و اینک مرد خدایی به فرمان خداوند ازیهودا به بیت ئیل آمد و یربعام به جهت سوزانیدن بخور نزد مذبح ایستاده بود.
\par 2 پس به فرمان خداوند مذبح را ندا کرده، گفت: «ای مذبح! ای مذبح! خداوند چنین می‌گوید: اینک پسری که یوشیا نام دارد به جهت خاندان داود زاییده می‌شود و کاهنان مکانهای بلند را که بر تو بخورمی سوزانند، بر تو ذبح خواهد نمود واستخوانهای مردم را بر تو خواهند سوزانید.»
\par 3 ودر آن روز علامتی نشان داده، گفت: «این است علامتی که خداوند فرموده است، اینک این مذبح چاک خواهد شد و خاکستری که بر آن است، ریخته خواهد گشت.»
\par 4 و واقع شد که چون پادشاه، سخن مرد خدا را که مذبح را که دربیت ئیل بود، ندا کرده بود، شنید، یربعام دست خود را از جانب مذبح دراز کرده، گفت: «او را بگیرید.» و دستش که به سوی او دراز کرده بود، خشک شد به طوری که نتوانست آن را نزد خودباز بکشد.
\par 5 و مذبح چاک شد و خاکستر از روی مذبح ریخته گشت بر‌حسب علامتی که آن مردخدا به فرمان خداوند نشان داده بود.
\par 6 و پادشاه، مرد خدا را خطاب کرده، گفت: «تمنا اینکه نزدیهوه، خدای خود تضرع نمایی و برای من دعاکنی تا دست من به من باز داده شود.» پس مرد خدانزد خداوند تضرع نمود، و دست پادشاه به او بازداده شده، مثل اول گردید.
\par 7 و پادشاه به آن مردخدا گفت: «همراه من به خانه بیا و استراحت نما وتو را اجرت خواهم داد.»
\par 8 اما مرد خدا به پادشاه گفت: «اگر نصف خانه خود را به من بدهی، همراه تو نمی آیم، و در اینجا نه نان می‌خورم و نه آب می‌نوشم.
\par 9 زیرا خداوند مرا به کلام خود چنین امر فرموده و گفته است نان مخور و آب منوش وبه راهی که آمده‌ای بر مگرد.»
\par 10 پس به راه دیگربرفت و از راهی که به بیت ئیل آمده بود، مراجعت ننمود.
\par 11 و نبی سالخورده‌ای در بیت ئیل ساکن می‌بود و پسرانش آمده، او را از هر کاری که آن مرد خدا آن روز در بیت ئیل کرده بود، مخبرساختند، و نیز سخنانی را که به پادشاه گفته بود، برای پدر خود بیان کردند.
\par 12 و پدر ایشان به ایشان گفت: «به کدام راه رفته است؟» و پسرانش دیده بودند که آن مرد خدا که از یهودا آمده بود به کدام راه رفت.
\par 13 پس به پسران خود گفت: «الاغ را برای من بیارایید.» و الاغ را برایش آراستند و برآن سوار شد.
\par 14 و از عقب مرد خدا رفته، او را زیردرخت بلوط نشسته یافت. پس او را گفت: «آیا توآن مرد خدا هستی که از یهودا آمده‌ای؟» گفت: «من هستم.»
\par 15 وی را گفت: «همراه من به خانه بیا و غذا بخور.»
\par 16 او در جواب گفت که «همراه تونمی توانم برگردم و با تو داخل شوم، و در اینجا باتو نه نان می‌خورم و نه آب می‌نوشم.
\par 17 زیرا که به فرمان خداوند به من گفته شده است که در آنجانان مخور و آب منوش و از راهی که آمده‌ای مراجعت منما.»
\par 18 او وی را گفت: «من نیز مثل تونبی هستم و فرشته‌ای به فرمان خداوند با من متکلم شده، گفت او را با خود به خانه ات برگردان تا نان بخورد و آب بنوشد.» اما وی را دروغ گفت.
\par 19 پس همراه وی در خانه‌اش برگشته، غذا خوردو آب نوشید.
\par 20 و هنگامی که ایشان بر سفره نشسته بودند، کلام خداوند به آن نبی که او را برگردانیده بودآمد،
\par 21 و به آن مرد خدا که از یهودا آمده بود، نداکرده، گفت: «خداوند چنین می‌گوید: چونکه ازفرمان خداوند تمرد نموده، حکمی را که یهوه، خدایت به تو امر فرموده بود نگاه نداشتی،
\par 22 وبرگشته، در جایی که به تو گفته شده بود غذامخور و آب منوش، غذا خوردی و آب نوشیدی، لهذا جسد تو به قبر پدرانت داخل نخواهد شد.»
\par 23 پس بعد از اینکه او غذا خورد وآب نوشید الاغ را برایش بیاراست، یعنی به جهت نبی که برگردانیده بود.
\par 24 و چون رفت، شیری اورا در راه یافته، کشت و جسد او در راه انداخته شد، و الاغ به پهلویش ایستاده، و شیر نیز نزد لاش ایستاده بود.
\par 25 و اینک بعضی راه گذران جسد رادر راه انداخته شده، و شیر را نزد جسد ایستاده دیدند، پس آمدند و در شهری که آن نبی پیر درآن ساکن می‌بود، خبر دادند.
\par 26 و چون نبی که او را از راه برگردانیده بودشنید، گفت: «این آن مرد خداست که از حکم خداوند تمرد نمود، لهذا خداوند او را به شیر داده که او را دریده و کشته است، موافق کلامی که خداوند به او گفته بود.
\par 27 پس پسران خود راخطاب کرده، گفت: «الاغ را برای من بیارایید.» وایشان آن را آراستند.
\par 28 و او روانه شده، جسد اورا در راه انداخته، و الاغ و شیر را نزد جسدایستاده یافت، و شیر جسد را نخورده و الاغ راندریده بود.
\par 29 و آن نبی جسد مرد خدا رابرداشت و بر الاغ گذارده، آن را بازآورد و آن نبی پیر به شهر آمد تا ماتم گیرد و او را دفن نماید.
\par 30 وجسد او را در قبر خویش گذارد و برای او ماتم گرفته، گفتند: «وای‌ای برادر من!»
\par 31 و بعد ازآنکه او را دفن کرد به پسران خود خطاب کرده، گفت: «چون من بمیرم مرا در قبری که مرد خدا درآن مدفون است، دفن کنید، و استخوانهایم را به پهلوی استخوانهای وی بگذارید.
\par 32 زیرا کلامی را که درباره مذبحی که در بیت ئیل است و درباره همه خانه های مکانهای بلند که در شهرهای سامره می‌باشد، به فرمان خداوند گفته بود، البته واقع خواهد شد.
\par 33 و بعد از این امر، یربعام از طریق ردی خودبازگشت ننمود، بلکه کاهنان برای مکانهای بلند ازجمیع قوم تعیین نمود، و هرکه می‌خواست، او راتخصیص می‌کرد تا از کاهنان مکانهای بلند بشود.و این کار باعث گناه خاندان یربعام گردید تا آن را از روی زمین منقطع و هلاک ساخت.
\par 34 و این کار باعث گناه خاندان یربعام گردید تا آن را از روی زمین منقطع و هلاک ساخت.
 
\chapter{14}

\par 1 در آن زمان ابیا پسر یربعام بیمار شد.
\par 2 و یربعام به زن خود گفت که «الان برخیز و صورت خود را تبدیل نما تا نشناسند که تو زن یربعام هستی، و به شیلوه برو. اینک اخیای نبی که درباره من گفت که براین قوم پادشاه خواهم شد در آنجاست.
\par 3 و در دست خود ده قرص نان وکلیچه‌ها و کوزه عسل گرفته، نزد وی برو و او تورا از آنچه بر طفل واقع می‌شود، خبر‌خواهدداد.»
\par 4 پس زن یربعام چنین کرده، برخاست و به شیلوه رفته، به خانه اخیا رسید و اخیانمی توانست ببیند زیرا که چشمانش از پیری تارشده بود.
\par 5 و خداوند به اخیا گفت: «اینک زن یربعام می‌آید تا درباره پسرش که بیمار است، چیزی از تو بپرسد. پس به او چنین و چنان بگو وچون داخل می‌شود به هیات، متنکره خواهدبود.»
\par 6 و هنگامی که اخیا صدای پایهای او را که به در داخل می‌شد شنید، گفت: «ای زن یربعام داخل شو. چرا هیات خود را متنکر ساخته‌ای؟ زیرا که من باخبر سخت نزد تو فرستاده شده‌ام.
\par 7 برو و به یربعام بگو: یهوه، خدای اسرائیل چنین می‌گوید: چونکه تو را از میان قوم ممتاز نمودم، وتو را بر قوم خود، اسرائیل رئیس ساختم،
\par 8 وسلطنت را از خاندان داود دریده، آن را به تو دادم، و تو مثل بنده من، داود نبودی که اوامر مرا نگاه داشته، با تمامی دل خود مرا پیروی می‌نمود، وآنچه در نظر من راست است، معمول می‌داشت وبس.
\par 9 اما تو از همه کسانی که قبل از تو بودندزیاده شرارت ورزیدی و رفته، خدایان غیر وبتهای ریخته شده به جهت خود ساختی و غضب مرا به هیجان آوردی و مرا پشت سر خودانداختی.
\par 10 بنابراین اینک من بر خاندان یربعام بلا عارض می‌گردانم و از یربعام هر مرد را و هر محبوس و آزاد را که در اسرائیل باشد، منقطع می‌سازم، و تمامی خاندان یربعام را دور می‌اندازم چنانکه سرگین را بالکل دور می‌اندازند.
\par 11 هرکه از یربعام در شهر بمیرد، سگان بخورند و هرکه درصحرا بمیرد، مرغان هوا بخورند، زیرا خداونداین را گفته است.
\par 12 پس تو برخاسته به خانه خودبرو و به مجرد رسیدن پایهایت به شهر، پسرخواهد مرد.
\par 13 و تمامی اسرائیل برای او نوحه نموده، او را دفن خواهند کرد زیرا که او تنها ازنسل یربعام به قبر داخل خواهد شد، به علت اینکه با او چیز نیکو نسبت به یهوه، خدای اسرائیل در خاندان یربعام یافت شده است.
\par 14 وخداوند امروز پادشاهی بر اسرائیل خواهدبرانگیخت که خاندان یربعام را منقطع خواهدساخت و چه (بگویم ) الان نیز (واقع شده است ).
\par 15 و خداوند اسرائیل را خواهد زد مثل نی‌ای که در آب متحرک شود، و ریشه اسرائیل را از این زمین نیکو که به پدران ایشان داده بود، خواهد کندو ایشان را به آن طرف نهر پراکنده خواهدساخت، زیرا که اشیریم خود را ساخته، خشم خداوند را به هیجان آوردند.
\par 16 و اسرائیل را به‌سبب گناهانی که یربعام ورزیده، و اسرائیل را به آنها مرتکب گناه ساخته است، تسلیم خواهدنمود.»
\par 17 پس زن یربعام برخاسته، و روانه شده، به ترصه آمد و به مجرد رسیدنش به آستانه خانه، پسر مرد.
\par 18 و تمامی اسرائیل او را دفن کردند وبرایش ماتم گرفتند، موافق کلام خداوند که به واسطه بنده خود، اخیای نبی گفته بود.
\par 19 و بقیه وقایع یربعام که چگونه جنگ کرد و چگونه سلطنت نمود اینک در کتاب تواریخ ایام پادشاهان اسرائیل مکتوب است.
\par 20 و ایامی که یربعام سلطنت نمود، بیست و دو سال بود. پس باپدران خود خوابید و پسرش ناداب به‌جایش پادشاه شد.
\par 21 و رحبعام بن سلیمان در یهودا سلطنت می‌کرد، و رحبعام چون پادشاه شد چهل و یک ساله بود و در اورشلیم، شهری که خداوند از تمام اسباط اسرائیل برگزید تا اسم خود را در آن بگذارد، هفده سال پادشاهی کرد. و اسم مادرش نعمه عمونیه بود.
\par 22 و یهودا در نظر خداوندشرارت ورزیدند، و به گناهانی که کردند، بیشتر ازهر‌آنچه پدران ایشان کرده بودند، غیرت او را به هیجان آوردند.
\par 23 و ایشان نیز مکانهای بلند وستونها و اشیریم بر هر تل بلند و زیر هر درخت سبز بنا نمودند.
\par 24 و الواط نیز در زمین بودند وموافق رجاسات امتهایی که خداوند از حضوربنی‌اسرائیل اخراج نموده بود، عمل می‌نمودند.
\par 25 و در سال پنجم رحبعام پادشاه واقع شد که شیشق پادشاه مصر به اورشلیم برآمد.
\par 26 وخزانه های خانه خداوند و خزانه های خانه پادشاه را گرفت و همه‌چیز را برداشت و جمیع سپرهای طلایی که سلیمان ساخته بود، برد.
\par 27 و رحبعام پادشاه به عوض آنها سپرهای برنجین ساخت وآنها را به‌دست سرداران شاطرانی که در خانه پادشاه را نگاهبانی می‌کردند، سپرد.
\par 28 و هر وقتی که پادشاه به خانه خداوند داخل می‌شد، شاطران آنها را برمی داشتند و آنها را به حجره شاطران باز می آوردند. 
\par 29 و بقیه وقایع رحبعام و هرچه کرد، آیا درکتاب تواریخ ایام پادشاهان یهودا مکتوب نیست؟
\par 30 و در میان رحبعام و یربعام درتمامی روزهای ایشان جنگ می‌بود.ورحبعام با پدران خویش خوابید و در شهر داود باپدران خود دفن شد، و اسم مادرش نعمه عمونیه بود و پسرش ابیام در جایش پادشاهی نمود.
\par 31 ورحبعام با پدران خویش خوابید و در شهر داود باپدران خود دفن شد، و اسم مادرش نعمه عمونیه بود و پسرش ابیام در جایش پادشاهی نمود.
 
\chapter{15}

\par 1 و در سال هجدهم پادشاهی یربعام بن نباط، ابیام، بر یهودا پادشاه شد.
\par 2 سه سال در اورشلیم سلطنت نمود و اسم مادرش معکه دختر ابشالوم بود.
\par 3 و در تمامی گناهانی که پدرش قبل از او کرده بود، سلوک می‌نمود، ودلش با یهوه، خدایش مثل دل پدرش داود کامل نبود.
\par 4 اما یهوه، خدایش به‌خاطر داود وی رانوری در اورشلیم داد تا پسرش را بعد از اوبرقرار گرداند، و اورشلیم را استوار نماید.
\par 5 چونکه داود آنچه در نظر خداوند راست بود، بجا می‌آورد و از هرچه او را امر فرموده، تمام روزهای عمرش تجاوز ننموده بود، مگر در امراوریای حتی.
\par 6 و در میان رحبعام و یربعام تمام روزهای عمرش جنگ بود.
\par 7 و بقیه وقایع ابیام وهرچه کرد، آیا در کتاب تواریخ ایام پادشاهان یهودا مکتوب نیست؟ و در میان ابیام و یربعام جنگ بود.
\par 8 و ابیام با پدران خویش خوابید و او رادر شهر دفن کردند و پسرش آسا در جایش سلطنت نمود.
\par 9 و در سال بیستم یربعام پادشاه اسرائیل، آسابر یهودا پادشاه شد.
\par 10 و در اورشلیم چهل و یک سال پادشاهی کرد و اسم مادرش معکه دخترابشالوم بود.
\par 11 و آسا آنچه در نظر خداوند راست بود، مثل پدرش، داود عمل نمود.
\par 12 و الواط را ازولایت بیرون کرد و بت هایی را که پدرانش ساخته بودند، دور نمود.
\par 13 و مادر خود، معکه را نیز ازملکه بودن معزول کرد، زیرا که او تمثالی به جهت اشیره ساخته بود. و آسا تمثال او را قطع نموده، آن را در وادی قدرون سوزانید.
\par 14 اما مکان های بلند برداشته نشد لیکن دل آسا در تمام ایامش باخداوند کامل می‌بود.
\par 15 و چیزهایی را که پدرش وقف کرده و آنچه خودش وقف نموده بود، از نقره و طلا و ظروف، در خانه خداونددرآورد.
\par 16 و در میان آسا و بعشا، پادشاه اسرائیل، تمام روزهای ایشان جنگ می‌بود.
\par 17 و بعشاپادشاه اسرائیل بر یهودا برآمده، رامه را بنا کرد تانگذارد که کسی نزد آسا، پادشاه یهودا رفت و آمدنماید.
\par 18 آنگاه آسا تمام نقره و طلا را که درخزانه های خانه خداوند و خزانه های خانه پادشاه باقی‌مانده بود گرفته، آن را به‌دست بندگان خودسپرد و آسا پادشاه ایشان را نزد بنهدد بن طبرمون بن حزیون، پادشاه ارام که در دمشق ساکن بودفرستاده، گفت:
\par 19 «در میان من و تو و در میان پدرمن و پدر تو عهد بوده است، اینک هدیه‌ای از نقره و طلا نزد تو فرستادم، پس بیا و عهد خود را بابعشا، پادشاه اسرائیل بشکن تا او از نزد من برود.»
\par 20 و بنهدد، آسا پادشاه را اجابت نموده، سرداران افواج خود را بر شهرهای اسرائیل فرستاد و عیون ودان و آبل بیت معکه و تمامی کنروت را با تمامی زمین نفتالی مغلوب ساخت.
\par 21 و چون بعشا این را شنید بنا نمودن رامه را ترک کرده، در ترصه اقامت نمود.
\par 22 و آسا پادشاه در تمام یهودا ندادرداد که احدی از آن مستثنی نبود تا ایشان سنگهای رامه و چوب آن را که بعشا بنا می‌کردبرداشتند، و آسا پادشاه جبع بنیامین و مصفه را باآنها بنا نمود.
\par 23 و بقیه تمامی وقایع آسا و تهور اوو هرچه کرد و شهرهایی که بنا نمود، آیا در کتاب تواریخ ایام پادشاهان یهودا مذکور نیست؟ اما درزمان پیریش درد پا داشت.
\par 24 و آسا با پدران خویش خوابید و او را در شهر داود با پدرانش دفن کردند، و پسرش یهوشافاط در جایش سلطنت نمود.
\par 25 و در سال دوم آسا، پادشاه یهودا، ناداب بن یربعام بر اسرائیل پادشاه شد، و دو سال براسرائیل پادشاهی کرد.
\par 26 و آنچه در نظر خداوندناپسند بود، بجا می‌آورد. و به راه پدر خود و به گناه او که اسرائیل را به آن مرتکب ساخته بود، سلوک می‌نمود.
\par 27 و بعشا ابن اخیا که از خاندان یساکار بود، بروی فتنه انگیخت و بعشا او را در جبتون که از آن فلسطینیان بود، کشت و ناداب و تمامی اسرائیل، جبتون را محاصره نموده بودند.
\par 28 و در سال سوم آسا، پادشاه یهودا، بعشا او را کشت و درجایش سلطنت نمود.
\par 29 و چون او پادشاه شد، تمام خاندان یربعام را کشت و کسی را برای یربعام زنده نگذاشت تا همه را هلاک کرد موافق کلام خداوند که به واسطه بنده خود اخیای شیلونی گفته بود.
\par 30 و این به‌سبب گناهانی شد که یربعام ورزیده، و اسرائیل را به آنها مرتکب گناه ساخته، و خشم یهوه، خدای اسرائیل را به آنها به هیجان آورده بود.
\par 31 و بقیه وقایع ناداب و هرچه کرد، آیا درکتاب تواریخ ایام پادشاهان اسرائیل مکتوب نیست؟
\par 32 و در میان آسا و بعشا، پادشاه اسرائیل، در تمام روزهای ایشان جنگ می‌بود.
\par 33 در سال سوم آسا، پادشاه یهودا، بعشا ابن اخیا بر تمامی اسرائیل در ترصه پادشاه شد وبیست و چهار سال سلطنت نمود.و آنچه درنظر خداوند ناپسند بود، می‌کرد و به راه یربعام وبه گناهی که اسرائیل را به آن مرتکب گناه ساخته بود، سلوک می‌نمود.
\par 34 و آنچه درنظر خداوند ناپسند بود، می‌کرد و به راه یربعام وبه گناهی که اسرائیل را به آن مرتکب گناه ساخته بود، سلوک می‌نمود.
 
\chapter{16}

\par 1 و کلام خداوند بر ییهو ابن حنانی درباره بعشا نازل شده، گفت:
\par 2 «چونکه تو را ازخاک برافراشتم و تو را بر قوم خود، اسرائیل پیشوا ساختم اما تو به راه یربعام سلوک نموده، قوم من، اسرائیل را مرتکب گناه ساخته، تا ایشان خشم مرا از گناهان خود به هیجان آورند.
\par 3 اینک من بعشا و خانه او را بالکل تلف خواهم نمود وخانه تو را مثل خانه یربعام بن نباط خواهم گردانید.
\par 4 آن را که از بعشا در شهر بمیرد، سگان بخورند و آن را که در صحرا بمیرد، مرغان هوابخورند.»
\par 5 و بقیه وقایع بعشا و آنچه کرد و تهور او، آیا در کتاب تواریخ ایام پادشاهان اسرائیل مکتوب نیست؟
\par 6 پس بعشا با پدران خود خوابید و درترصه مدفون شد و پسرش ایله در جایش پادشاه شد.
\par 7 و نیز کلام خداوند بر ییهوابن حنانی نبی نازل شد، درباره بعشا و خاندانش هم به‌سبب تمام شرارتی که در نظر خداوند بجا آورده، خشم او را به اعمال دستهای خود به هیجان آورد و مثل خاندان یربعام گردید و هم از این سبب که او راکشت.
\par 8 و در سال بیست و ششم آسا، پادشاه یهودا، ایله بن بعشا در ترصه بر اسرائیل پادشاه شد و دوسال سلطنت نمود.
\par 9 و بنده او، زمری که سردارنصف ارابه های او بود، بر او فتنه انگیخت و او درترصه در خانه ارصا که ناظر خانه او در ترصه بود، می‌نوشید و مستی می‌نمود.
\par 10 و زمری داخل شده، او را در سال بیست و هفتم آسا، پادشاه یهودا زد و کشت و در جایش سلطنت نمود.
\par 11 و چون پادشاه شد و بر کرسی وی بنشست، تمام خاندان بعشا را زد چنانکه یک مرد از اقربا و اصحاب او را برایش باقی نگذاشت.
\par 12 پس زمری تمامی خاندان بعشا را موافق کلامی که خداوند به واسطه ییهوی نبی درباره بعشا گفته بود، هلاک کرد.
\par 13 به‌سبب تمامی گناهانی که بعشا و گناهانی که پسرش ایله کرده، و اسرائیل را به آنها مرتکب گناه ساخته بودند، به طوری که ایشان به اباطیل خویش خشم یهوه، خدای اسرائیل رابه هیجان آورد.
\par 14 و بقیه وقایع ایله و هرچه کرد، آیا درکتاب تواریخ ایام پادشاهان اسرائیل مکتوب نیست.
\par 15 در سال بیست و هفتم آسا، پادشاه یهودا، زمری در ترصه هفت روز سلطنت نمود و قوم دربرابر جبتون که از آن فلسطینیان بود، اردو زده بودند.
\par 16 و قومی که در اردو بودند، شنیدند که زمری فتنه برانگیخته و پادشاه را نیز کشته است. پس تمامی اسرائیل، عمری را که سردار لشکربود، در همان روز بر تمامی اسرائیل در اردوپادشاه ساختند.
\par 17 آنگاه عمری و تمام اسرائیل با وی از جبتون برآمده، ترصه را محاصره نمودند.
\par 18 و چون زمری دید که شهر گرفته شد، به قصر خانه پادشاه داخل شده، خانه پادشاه را برسر خویش به آتش سوزانید و مرد.
\par 19 و این به‌سبب گناهانی بود که ورزید و آنچه را که در نظرخداوند ناپسند بود بجا آورد، و به راه یربعام و به گناهی که او ورزیده بود، سلوک نموده، اسرائیل را نیز مرتکب گناه ساخت.
\par 20 و بقیه وقایع زمری و فتنه‌ای که او برانگیخته بود، آیا در کتاب تواریخ ایام پادشاهان اسرائیل مکتوب نیست؟
\par 21 آنگاه قوم اسرائیل به دو فرقه تقسیم شدندو نصف قوم تابع تبنی پسر جینت گشتند تا او راپادشاه سازند و نصف دیگر تابع عمری.
\par 22 اماقومی که تابع عمری بودند بر قومی که تابع تبنی پسر جینت بودند، غالب آمدند پس تبنی مرد وعمری سلطنت نمود.
\par 23 در سال سی و یکم آسا، پادشاه یهودا، عمری بر اسرائیل پادشاه شد ودوازده سال سلطنت نمود؛ شش سال در ترصه سلطنت کرد.
\par 24 پس کوه سامره را از سامر به دو وزنه نقره خرید و در آن کوه بنایی ساخت و شهری را که بنا کرد به نام سامر که مالک کوه بود، سامره نامید.
\par 25 و عمری آنچه در نظر خداوند ناپسند بود، به عمل آورد و از همه آنانی که پیش از او بودند، بدتر کرد.
\par 26 زیرا که به تمامی راههای یربعام بن نباط و به گناهانی که اسرائیل را به آنها مرتکب گناه ساخته بود به طوری که ایشان به اباطیل خویش خشم یهوه، خدای اسرائیل را به هیجان آورد، سلوک می‌نمود.
\par 27 و بقیه اعمال عمری که کرد و تهوری که نمود، آیا در کتاب تواریخ ایام پادشاهان اسرائیل مکتوب نیست؟
\par 28 پس عمری با پدران خویش خوابید و در سامره مدفون شد وپسرش اخاب در جایش سلطنت نمود.
\par 29 و اخاب بن عمری در سال سی و هشتم آسا، پادشاه یهودا، بر اسرائیل پادشاه شد، واخاب بن عمری بر اسرائیل در سامره بیست و دوسال سلطنت نمود.
\par 30 و اخاب بن عمری از همه آنانی که قبل از او بودند در نظر خداوند بدتر کرد.
\par 31 و گویا سلوک نمودن او به گناهان یربعام بن نباط سهل می‌بود که ایزابل، دختر اتبعل، پادشاه صیدونیان را نیز به زنی گرفت و رفته، بعل راعبادت نمود و او را سجده کرد.
\par 32 و مذبحی به جهت بعل در خانه بعل که در سامره ساخته بود، برپا نمود.
\par 33 و اخاب اشیره را ساخت و اخاب در اعمال خود افراط نموده، خشم یهوه، خدای اسرائیل را بیشتر از جمیع پادشاهان اسرائیل که قبل از او بودند، به هیجان آورد.و در ایام او، حیئیل بیت ئیلی، اریحا را بنا کرد و بنیادش را برنخست زاده خود ابیرام نهاد و دروازه هایش را بر پسر کوچک خود سجوب برپا کرد موافق کلام خداوند که به واسطه یوشع بن نون گفته بود.
\par 34 و در ایام او، حیئیل بیت ئیلی، اریحا را بنا کرد و بنیادش را برنخست زاده خود ابیرام نهاد و دروازه هایش را بر پسر کوچک خود سجوب برپا کرد موافق کلام خداوند که به واسطه یوشع بن نون گفته بود.
 
\chapter{17}

\par 1 و ایلیای تشبی که از ساکنان جلعاد بود، به اخاب گفت: «به حیات یهوه، خدای اسرائیل که به حضور وی ایستاده‌ام قسم که دراین سالها شبنم و باران جز به کلام من نخواهدبود.»
\par 2 و کلام خداوند بر وی نازل شده، گفت:
\par 3 «ازاینجا برو و به طرف مشرق توجه نما و خویشتن را نزد نهر کریت که در مقابل اردن است، پنهان کن.
\par 4 و از نهر خواهی نوشید و غرابها را امر فرموده‌ام که تو را در آنجا بپرورند.»
\par 5 پس روانه شده، موافق کلام خداوند عمل نموده، و رفته نزد نهرکریت که در مقابل اردن است، ساکن شد.
\par 6 وغرابها در صبح، نان و گوشت برای وی و در شام، نان و گوشت می‌آوردند و از نهر می‌نوشید.
\par 7 وبعد از انقضای روزهای چند، واقع شد که نهرخشکید زیرا که باران در زمین نبود.
\par 8 و کلام خداوند بر وی نازل شده، گفت:
\par 9 «برخاسته، به صرفه که نزد صیدون است برو ودر آنجا ساکن بشو، اینک به بیوه‌زنی در آنجا امرفرموده‌ام که تو را بپرورد.» 
\par 10 پس برخاسته، به صرفه رفت و چون نزد دروازه شهر رسید اینک بیوه‌زنی در آنجا هیزم برمی چید، پس او را صدازده، گفت: «تمنا اینکه جرعه‌ای آب در ظرفی برای من بیاوری تا بنوشم.»
\par 11 و چون به جهت آوردن آن می‌رفت وی را صدا زده، گفت: «لقمه‌ای نان برای من در دست خود بیاور.»
\par 12 اوگفت: «به حیات یهوه، خدایت قسم که قرص نانی ندارم، بلکه فقط یک مشت آرد در تاپو و قدری روغن در کوزه، و اینک دو چوبی برمی چینم تارفته، آن را برای خود و پسرم بپزم که بخوریم وبمیرم.»
\par 13 ایلیا وی را گفت: «مترس، برو و به طوری که گفتی بکن. لیکن اول گرده‌ای کوچک ازآن برای من بپز و نزد من بیاور، و بعد از آن برای خود و پسرت بپز.
\par 14 زیرا که یهوه، خدای اسرائیل، چنین می‌گوید که تا روزی که خداوندباران بر زمین نباراند، تاپوی آرد تمام نخواهدشد، و کوزه روغن کم نخواهد گردید.»
\par 15 پس رفته، موافق کلام ایلیا عمل نمود. و زن و او وخاندان زن، روزهای بسیار خوردند.
\par 16 و تاپوی آرد تمام نشد و کوزه روغن کم نگردید، موافق کلام خداوند که به واسطه ایلیا گفته بود.
\par 17 و بعد از این امور واقع شد که پسر آن زن که صاحب‌خانه بود، بیمار شد. و مرض او چنان سخت شد که نفسی در او باقی نماند.
\par 18 و به ایلیاگفت: «ای مرد خدا مرا با تو چه‌کار است؟ آیا نزدمن آمدی تا گناه مرا بیاد آوری و پسر مرابکشی؟»
\par 19 او وی را گفت: «پسرت را به من بده.» پس او را از آغوش وی گرفته، به بالاخانه‌ای که درآن ساکن بود، برد و او را بر بستر خود خوابانید.
\par 20 و نزد خداوند استغاثه نموده، گفت: «ای یهوه، خدای من، آیا به بیوه‌زنی نیز که من نزد او ماواگزیده‌ام بلا رسانیدی و پسر او را کشتی؟»
\par 21 آنگاه خویشتن را سه مرتبه بر پسر دراز کرده، نزد خداوند استغاثه نموده، گفت: «ای یهوه، خدای من، مسالت اینکه جان این پسر به وی برگردد.»
\par 22 و خداوند آواز ایلیا را اجابت نمود وجان پسر به وی برگشت که زنده شد.
\par 23 و ایلیاپسر را گرفته، او را از بالاخانه به خانه به زیر آوردو به مادرش سپرد و ایلیا گفت: «ببین که پسرت زنده است!پس آن زن به ایلیا گفت: «الان از این دانستم که تو مرد خدا هستی و کلام خداوند دردهان تو راست است.»
\par 24 پس آن زن به ایلیا گفت: «الان از این دانستم که تو مرد خدا هستی و کلام خداوند دردهان تو راست است.»
 
\chapter{18}

\par 1 و بعد از روزهای بسیار، کلام خداونددر سال سوم، به ایلیا نازل شده، گفت: «برو و خود را به اخاب بنما و من بر زمین باران خواهم بارانید.»
\par 2 پس ایلیا روانه شد تا خود را به اخاب بنماید و قحط در سامره سخت بود.
\par 3 واخاب عوبدیا را که ناظر خانه او بود، احضار نمودو عوبدیا از خداوند بسیار می‌ترسید.
\par 4 و هنگامی که ایزابل انبیای خداوند را هلاک می‌ساخت، عوبدیا صد نفر از انبیا را گرفته، ایشان را پنجاه پنجاه در مغاره پنهان کرد و ایشان را به نان و آب پرورد.
\par 5 و اخاب به عوبدیا گفت: «در زمین نزدتمامی چشمه های آب و همه نهرها برو که شایدعلف پیدا کرده، اسبان و قاطران را زنده نگاه داریم و همه بهایم از ما تلف نشوند.»
\par 6 پس زمین را درمیان خود تقسیم کردند تا در آن عبور نمایند؛ اخاب به یک راه تنها رفت، و عوبدیا به راه دیگر، تنها رفت.
\par 7 و چون عوبدیا در راه بود، اینک ایلیا بدوبرخورد و او وی را شناخته، به روی خود درافتاده، گفت: «آیا آقای من ایلیا، تو هستی؟»
\par 8 او را جواب داد که «من هستم، برو و به آقای خودبگو که اینک ایلیاست.»
\par 9 گفت: «چه گناه کرده‌ام که بنده خود را به‌دست اخاب تسلیم می‌کنی تامرا بکشد.
\par 10 به حیات یهوه، خدای تو قسم که قومی و مملکتی نیست، که آقایم به جهت طلب تو آنجا نفرستاده باشد و چون می‌گفتند که اینجانیست به آن مملکت و قوم قسم می‌داد که تو رانیافته‌اند.
\par 11 و حال می‌گویی برو به آقای خودبگو که اینک ایلیاست؟
\par 12 و واقع خواهد شد که چون از نزد تو رفته باشم، روح خداوند تو را به‌جایی که نمی دانم، بردارد و وقتی که بروم و به اخاب خبر دهم و او تو را نیابد، مرا خواهد کشت. و بنده ات از طفولیت خود از خداوند می‌ترسد.
\par 13 مگر آقایم اطلاع ندارد از آنچه من هنگامی که ایزابل انبیای خداوند را می‌کشت کردم، که چگونه صد نفر از انبیای خداوند را پنجاه پنجاه در مغاره‌ای پنهان کرده، ایشان را به نان و آب پروردم.
\par 14 و حال تو می‌گویی برو و آقای خود رابگو که اینک ایلیاست؟ و مرا خواهد کشت.»
\par 15 ایلیا گفت: «به حیات یهوه، صبایوت که به حضور وی ایستاده‌ام قسم که خود را امروز به وی ظاهر خواهم نمود.»
\par 16 پس عوبدیا برای ملاقات اخاب رفته، او را خبر داد و اخاب به جهت ملاقات ایلیا آمد.
\par 17 و چون اخاب ایلیا را دید، اخاب وی راگفت: «آیا تو هستی که اسرائیل را مضطرب می‌سازی؟»
\par 18 گفت: «من اسرائیل را مضطرب نمی سازم، بلکه تو و خاندان پدرت، چونکه اوامرخداوند را ترک کردید و تو پیروی بعلیم رانمودی.
\par 19 پس الان بفرست و تمام اسرائیل را نزد من بر کوه کرمل جمع کن و انبیای بعل را نیزچهارصد و پنجاه نفر، و انبیای اشیریم راچهارصد نفر که بر سفره ایزابل می‌خورند.»
\par 20 پس اخاب نزد جمیع بنی‌اسرائیل فرستاده، انبیا را بر کوه کرمل جمع کرد.
\par 21 و ایلیابه تمامی قوم نزدیک آمده، گفت: «تا به کی درمیان دو فرقه می‌لنگید؟ اگر یهوه خداست، او راپیروی نمایید! و اگر بعل است، وی را پیروی نمایید!» اما قوم در جواب او هیچ نگفتند.
\par 22 پس ایلیا به قوم گفت: من تنها نبی یهوه باقی‌مانده‌ام وانبیای بعل چهارصد و پنجاه نفرند.
\par 23 پس به مادو گاو بدهند و یک گاو به جهت خود انتخاب کرده، و آن را قطعه قطعه نموده، آن را بر هیزم بگذارند و آتش ننهند و من گاو دیگر را حاضرساخته، بر هیزم می‌گذارم و آتش نمی نهم.
\par 24 وشما اسم خدای خود را بخوانید و من نام یهوه راخواهم خواند و آن خدایی که به آتش جواب دهد، او خدا باشد.» و تمامی قوم در جواب گفتند: «نیکو گفتی.»
\par 25 پس ایلیا به انبیای بعل گفت: «یک گاو برای خود انتخاب کرده، شما اول آن را حاضر سازید زیرا که بسیار هستید و به نام خدای خود بخوانید، اما آتش نگذارید.»
\par 26 پس گاو را که به ایشان داده شده بود، گرفتند و آن راحاضر ساخته، نام بعل را از صبح تا ظهر خوانده، می‌گفتند: «ای بعل ما را جواب بده.» لیکن هیچ صدا یا جوابی نبود و ایشان بر مذبحی که ساخته بودند، جست و خیز می‌نمودند.
\par 27 و به وقت ظهر، ایلیا ایشان را مسخره نموده، گفت: «به آوازبلند بخوانید زیرا که او خداست! شاید متفکراست یا به خلوت رفته، یا در سفر می‌باشد، یاشاید که در خواب است و باید او را بیدار کرد!»
\par 28 و ایشان به آواز بلند می‌خواندند و موافق عادت خود خویشتن را به تیغها و نیزه‌ها مجروح می‌ساختند به حدی که خون بر ایشان جاری می‌شد.
\par 29 و بعد از گذشتن ظهر تا وقت گذرانیدن هدیه عصری ایشان نبوت می‌کردند لیکن نه آوازی بود و نه کسی‌که جواب دهد یا توجه نماید.
\par 30 آنگاه ایلیا به تمامی قوم گفت: «نزد من بیایید.» و تمامی قوم نزد وی آمدند و مذبح یهوه را که خراب شده بود، تعمیر نمود.
\par 31 و ایلیاموافق شماره اسباط بنی یعقوب که کلام خداوندبر وی نازل شده، گفته بود که نام تو اسرائیل خواهد بود، دوازده سنگ گرفت.
\par 32 و به آن سنگها مذبحی به نام یهوه بنا کرد و گرداگرد مذبح خندقی که گنجایش دو پیمانه بزر داشت، ساخت.
\par 33 و هیزم را ترتیب داد و گاو را قطعه قطعه نموده، آن را بر هیزم گذاشت. پس گفت: «چهار خم از آب پر کرده، آن را بر قربانی سوختنی و هیزم بریزید.»
\par 34 پس گفت: «بار دیگربکنید، » و گفت: «بار سوم بکنید.» و بار سوم کردند.
\par 35 و آب گرداگرد مذبح جاری شد وخندق نیز از آب پر گشت.
\par 36 و در وقت گذرانیدن هدیه عصری، ایلیای نبی نزدیک آمده، گفت: «ای یهوه، خدای ابراهیم و اسحاق و اسرائیل، امروز معلوم بشود که تو دراسرائیل خدا هستی و من بنده تو هستم و تمام این کارها را به فرمان تو کرده‌ام.
\par 37 مرا اجابت فرما‌ای خداوند! مرا اجابت فرما تا این قوم بدانند که تویهوه خدا هستی و اینکه دل ایشان را باز پس گردانیدی.»
\par 38 آنگاه آتش یهوه افتاده، قربانی سوختنی و هیزم و سنگها و خاک را بلعید و آب را که در خندق بود، لیسید.
\par 39 و تمامی قوم چون این را دیدند به روی خود افتاده، گفتند: «یهوه، اوخداست! یهوه او خداست!»
\par 40 و ایلیا به ایشان گفت: «انبیای بعل را بگیرید و یکی از ایشان رهایی نیابد.» پس ایشان را گرفتند و ایلیا ایشان رانزد نهر قیشون فرود آورده، ایشان را در آنجاکشت.
\par 41 و ایلیا به اخاب گفت: «برآمده، اکل وشرب نما.» زیرا که صدای باران بسیار می‌آید.
\par 42 پس اخاب برآمده، اکل و شرب نمود. و ایلیا برقله کرمل برآمد و به زمین خم شده، روی خود رابه میان زانوهایش گذاشت.
\par 43 و به خادم خودگفت: «بالا رفته، به سوی دریا نگاه کن.» و او بالارفته، نگریست و گفت که چیزی نیست و او گفت: «هفت مرتبه دیگر برو.»
\par 44 و در مرتبه هفتم گفت که «اینک ابری کوچک به قدر کف دست آدمی ازدریا برمی آید.» او گفت: «برو و به اخاب بگو که ارابه خود را ببند و فرود شو مبادا باران تو را مانع شود.»
\par 45 و واقع شد که در اندک زمانی آسمان ازابر غلیظ و باد، سیاه فام شد، و باران سخت بارید واخاب سوار شده، به یزرعیل آمد.و دست خداوند بر ایلیا نهاده شده، کمر خود را بست وپیش روی اخاب دوید تا به یزرعیل رسید.
\par 46 و دست خداوند بر ایلیا نهاده شده، کمر خود را بست وپیش روی اخاب دوید تا به یزرعیل رسید.
 
\chapter{19}

\par 1 و اخاب، ایزابل را از آنچه ایلیا کرده، وچگونه جمیع انبیا را به شمشیر کشته بود، خبر داد.
\par 2 و ایزابل رسولی نزد ایلیا فرستاده، گفت: «خدایان به من مثل این بلکه زیاده از این عمل نمایند اگر فردا قریب به این وقت، جان تو را مثل جان یکی از ایشان نسازم.»
\par 3 و چون این رافهمید، برخاست و به جهت جان خود روانه شده، به بئرشبع که در یهوداست آمد و خادم خود را درآنجا واگذاشت.
\par 4 و خودش سفر یک روزه به بیابان کرده، رفت و زیر درخت اردجی نشست و برای خویشتن مرگ را خواسته، گفت: «ای خداوند بس است! جان مرا بگیر زیرا که از پدرانم بهتر نیستم.»
\par 5 وزیر درخت اردج دراز شده، خوابید. و اینک فرشته‌ای او را لمس کرده، به وی گفت: «برخیز وبخور.»
\par 6 و چون نگاه کرد، اینک نزد سرش قرصی نان بر ریگهای داغ و کوزه‌ای از آب بود. پس خورد و آشامید و بار دیگر خوابید.
\par 7 وفرشته خداوند بار دیگر برگشته، او را لمس کرد وگفت: «برخیز و بخور زیرا که راه برای تو زیاده است.»
\par 8 پس برخاسته، خورد و نوشید و به قوت آن خوراک، چهل روز و چهل شب تا حوریب که کوه خدا باشد، رفت.
\par 9 و در آنجا به مغاره‌ای داخل شده، شب را در آن بسر برد.
\par 10 او در جواب گفت: «به جهت یهوه، خدای لشکرها، غیرت عظیمی دارم زیرا که بنی‌اسرائیل عهد تو را ترک نموده، مذبح های تو را منهدم ساخته، و انبیای تو را به شمشیر کشته‌اند، و من به تنهایی باقی‌مانده‌ام و قصد هلاکت جان من نیزدارند.»
\par 11 او گفت: «بیرون آی و به حضور خداوند درکوه بایست.» و اینک خداوند عبور نمود و باد عظیم سخت کوهها را منشق ساخت و صخره هارا به حضور خداوند خرد کرد اما خداوند در بادنبود. و بعد از باد، زلزله شد اما خداوند در زلزله نبود.
\par 12 و بعد از زلزله، آتشی، اما خداوند درآتش نبود و بعد از آتش، آوازی ملایم و آهسته.
\par 13 و چون ایلیا این را شنید، روی خود را به ردای خویش پوشانیده، بیرون آمد و در دهنه مغاره ایستاد و اینک هاتفی به او گفت: «ای ایلیا تو را دراینجا چه‌کار است؟»
\par 14 او در جواب گفت: «به جهت یهوه، خدای لشکرها، غیرت عظیمی دارم زیرا که بنی‌اسرائیل عهد تو را ترک کرده، مذبح های تو را منهدم ساخته‌اند و انبیای تو را به شمشیر کشته‌اند و من به تنهایی باقی‌مانده‌ام وقصد هلاکت جان من نیز دارند.» 
\par 15 پس خداوندبه او گفت: «روانه شده، به راه خود به بیابان دمشق برگرد، و چون برسی، حزائیل را به پادشاهی ارام مسح کن.
\par 16 و ییهو ابن نمشی را به پادشاهی اسرائیل مسح نما، و الیشع بن شافاط را که از آبل محوله است، مسح کن تا به‌جای تو نبی بشود.
\par 17 و واقع خواهد شد هر‌که از شمشیر حزائیل رهایی یابد، ییهو او را به قتل خواهد رسانید و هرکه از شمشیر ییهو رهایی یابد، الیشع او رابه قتل خواهد رسانید.
\par 18 اما در اسرائیل هفت هزار نفررا باقی خواهم گذاشت که تمامی زانوهای ایشان نزد بعل خم نشده، و تمامی دهنهای ایشان او رانبوسیده است.»
\par 19 پس از آنجا روانه شده، الیشع بن شافاط را یافت که شیار می‌کرد و دوازده جفت گاو پیش وی و خودش با جفت دوازدهم بود. و چون ایلیااز او می‌گذشت، ردای خود را بر وی انداخت.
\par 20 و او گاوها را ترک کرده، از عقب ایلیا دوید وگفت: «بگذار که پدر و مادر خود را ببوسم و بعد ازآن در عقب تو آیم.» او وی را گفت: «برو و برگردزیرا به تو چه کرده‌ام!پس از عقب او برگشته، یک جفت گاو را گرفت و آنها را ذبح کرده، گوشت را با آلات گاوان پخت، و به کسان خود داد که خوردند و برخاسته، از عقب ایلیا رفت و به خدمت او مشغول شد.
\par 21 پس از عقب او برگشته، یک جفت گاو را گرفت و آنها را ذبح کرده، گوشت را با آلات گاوان پخت، و به کسان خود داد که خوردند و برخاسته، از عقب ایلیا رفت و به خدمت او مشغول شد.
 
\chapter{20}

\par 1 و بنهدد، پادشاه ارام، تمامی لشکر خودرا جمع کرد، و سی و دو پادشاه و اسبان و ارابه‌ها همراهش بودند. پس برآمده، سامره رامحاصره کرد و با آن جنگ نمود.
\par 2 و رسولان نزداخاب پادشاه اسرائیل به شهر فرستاده، وی راگفت: «بنهدد چنین می‌گوید:
\par 3 نقره تو و طلای تواز آن من است و زنان و پسران مقبول تو از آن منند.»
\par 4 و پادشاه اسرائیل در جواب گفت: «ای آقایم پادشاه! موافق کلام تو، من و هر‌چه دارم ازآن تو هستیم.»
\par 5 و رسولان بار دیگر آمده، گفتند: «بنهدد چنین امر فرموده، می‌گوید: به درستی که من نزد تو فرستاده، گفتم که نقره و طلا و زنان وپسران خود را به من بدهی.
\par 6 پس‌فردا قریب به این وقت، بندگان خود را نزد تو می‌فرستم تا خانه تو را و خانه بندگانت را جستجو نمایند و هر‌چه در نظر تو پسندیده است به‌دست خود گرفته، خواهند برد.»
\par 7 آنگاه پادشاه اسرائیل تمامی مشایخ زمین راخوانده، گفت: «بفهمید و ببینید که این مرد چگونه بدی را می‌اندیشد، زیرا که چون به جهت زنان وپسرانم و نقره و طلایم فرستاده بود، او را انکارنکردم.»
\par 8 آنگاه جمیع مشایخ و تمامی قوم وی راگفتند: او را مشنو و قبول منما.»
\par 9 پس به رسولان بنهدد گفت: «به آقایم، پادشاه بگویید: هر‌چه باراول به بنده خود فرستادی بجا خواهم آورد، امااینکار را نمی توانم کرد.» پس رسولان مراجعت کرده، جواب را به او رسانیدند.
\par 10 آنگاه بنهدد نزدوی فرستاده، گفت: «خدایان، مثل این بلکه زیاده از این به من عمل نمایند اگر گرد سامره کفایت مشتهای همه مخلوقی را که همراه من باشندبکند.»
\par 11 و پادشاه اسرائیل در جواب گفت: «وی را بگویید: آنکه اسلحه می‌پوشد مثل آنکه می‌گشاید فخر نکند.»
\par 12 و چون این جواب راشنید در حالی که او و پادشاهان در خیمه هامیگساری می‌نمودند، به بندگان خود گفت: «صف آرایی بنمایید.» پس در برابر شهرصف آرایی نمودند.
\par 13 و اینک نبی‌ای نزد اخاب، پادشاه اسرائیل آمده، گفت: «خداوند چنین می‌گوید: آیا این گروه عظیم را می‌بینی؟ همانا من امروز آن را به‌دست تو تسلیم می‌نمایم تا بدانی که من یهوه هستم.»
\par 14 اخاب گفت: «به واسطه که؟» او در جواب گفت: «خداوند می‌گوید به واسطه خادمان سروران کشورها.» گفت: «کیست که جنگ راشروع کند؟» جواب داد: «تو.»
\par 15 پس خادمان سروران کشورها را سان دید که ایشان دویست وسی و دو نفر بودند و بعد از ایشان، تمامی قوم، یعنی تمامی بنی‌اسرائیل را سان دید که هفت هزار نفر بودند.
\par 16 و در وقت ظهر بیرون رفتند و بنهدد با آن پادشاهان یعنی آن سی و سه پادشاه که مددکار اومی بودند، در خیمه‌ها به میگساری مشغول بودند.
\par 17 و خادمان سروران کشورها اول بیرون رفتند وبنهدد کسان فرستاد و ایشان او را خبر داده، گفتندکه «مردمان از سامره بیرون می‌آیند.»
\par 18 او گفت: «خواه برای صلح بیرون آمده باشند، ایشان رازنده بگیرید، و خواه به جهت جنگ بیرون آمده باشند، ایشان را زنده بگیرید.»
\par 19 پس ایشان از شهر بیرون آمدند، یعنی خادمان سروران کشورها و لشکری که در عقب ایشان بود.
\par 20 هر کس از ایشان حریف خود راکشت و ارامیان فرار کردند و اسرائیلیان ایشان راتعاقب نمودند و بنهدد پادشاه ارام بر اسب‌سوارشده، با چند سوار رهایی یافتند.
\par 21 و پادشاه اسرائیل بیرون رفته، سواران و ارابه‌ها را شکست داد، و ارامیان را به کشتار عظیمی کشت.
\par 22 و آن نبی نزد پادشاه اسرائیل آمده، وی راگفت: «برو و خویشتن را قوی ساز و متوجه شده، ببین که چه می‌کنی زیرا که در وقت تحویل سال، پادشاه ارام بر تو خواهد برآمد.»
\par 23 و بندگان پادشاه ارام، وی را گفتند: «خدایان ایشان خدایان کوهها می‌باشند و از این سبب بر ما غالب آمدند اما اگر با ایشان درهمواری جنگ نماییم، هر آینه بر ایشان غالب خواهیم آمد.
\par 24 پس به اینطور عمل نما که هریک از پادشاهان را ازجای خود عزل کرده، به‌جای ایشان سرداران بگذار.
\par 25 و تو لشکری رامثل لشکری که از تو تلف شده است، اسب به‌جای اسب و ارابه به‌جای ارابه برای خود بشمارتا با ایشان در همواری جنگ نماییم و البته برایشان غالب خواهیم آمد.» پس سخن ایشان را اجابت نموده، به همین طور عمل نمود.
\par 26 و در وقت تحویل سال، بنهدد ارامیان راسان دیده، به افیق برآمد تا با اسرائیل جنگ نماید.
\par 27 و بنی‌اسرائیل را سان دیده، زاد دادند وبه مقابله ایشان رفتند و بنی‌اسرائیل در برابر ایشان مثل دو گله کوچک بزغاله اردو زدند، اما ارامیان زمین را پر کردند.
\par 28 و آن مرد خدا نزدیک آمده، پادشاه اسرائیل را خطاب کرده، گفت: «خداوندچنین می‌گوید: چونکه ارامیان می‌گویند که یهوه خدای کوههاست و خدای وادیها نیست، لهذاتمام این گروه عظیم را به‌دست تو تسلیم خواهم نمود تا بدانید که من یهوه هستم.»
\par 29 و اینان درمقابل آنان، هفت روز اردو زدند و در روز هفتم جنگ، با هم پیوستند و بنی‌اسرائیل صد هزارپیاده ارامیان را در یک روز کشتند.
\par 30 و باقی ماندگان به شهر افیق فرار کردند و حصار بر بیست و هفت هزار نفر از باقی ماندگان افتاد.
\par 31 و بندگانش وی را گفتند: «هماناشنیده‌ایم که پادشاهان خاندان اسرائیل، پادشاهان حلیم می‌باشند، پس بر کمر خود پلاس و بر سر خود ریسمانها ببندیم و نزد پادشاه اسرائیل بیرون رویم شاید که جان تو را زنده نگاه دارد.
\par 32 و پلاس بر کمرهای خود و ریسمانها برسر خود بسته، نزد پادشاه اسرائیل آمده، گفتند: «بنده تو، بنهدد می‌گوید: تمنا اینکه جانم زنده بماند.» او جواب داد: «آیا او تا حال زنده است؟ او برادر من می‌باشد.»
\par 33 پس آن مردان تفال نموده، آن را به زودی از دهان وی گرفتند وگفتند: «برادر تو بنهدد!» پس او گفت: «بروید و اورا بیاورید.» و چون بنهدد نزد او بیرون آمد، او رابر ارابه خود سوار کرد.
\par 34 و (بنهدد) وی را گفت: «شهرهایی را که پدر من از پدر تو گرفت، پس می‌دهم و برای خود در دمشق کوچه‌ها بساز، چنانکه پدر من در سامره ساخت.» (در جواب گفت ): «من تو را به این عهد رها می‌کنم.» پس با اوعهد بست و او را رها کرد.
\par 35 و مردی از پسران انبیا به فرمان خداوند به رفیق خود گفت: «مرا بزن.» اما آن مرد از زدنش ابانمود.
\par 36 و او وی را گفت: «چونکه آواز خداوندرا نشنیدی همانا چون از نزد من بروی شیری تو راخواهد کشت.» پس چون از نزد وی رفته بود، شیری او را یافته، کشت.
\par 37 و او شخصی دیگر راپیدا کرده، گفت: «مرا بزن.» و آن مرد او را ضربتی زده، مجروح ساخت.
\par 38 پس آن نبی رفته، به‌سرراه منتظر پادشاه ایستاد، و عصابه خود را برچشمان خود کشیده، خویشتن را متنکر نمود.
\par 39 و چون پادشاه درگذر می‌بود او به پادشاه ندا درداد و گفت که «بنده تو به میان جنگ رفت و اینک شخصی میل کرده، کسی را نزد من آورد و گفت: این مرد را نگاه دار و اگر مفقود شود جان تو به عوض جان او خواهد بود یا یک وزنه نقره خواهی داد.
\par 40 و چون بنده تو اینجا و آنجا مشغول می‌بود او غایب شد.» پس پادشاه اسرائیل وی راگفت: «حکم تو چنین است. خودت فتوی دادی.»
\par 41 پس به زودی عصابه را از چشمان خودبرداشت و پادشاه اسرائیل او را شناخت که یکی از انبیاست.
\par 42 او وی را گفت: «خداوند چنین می‌گوید: چون تو مردی را که من به هلاکت سپرده بودم از دست خود رها کردی، جان تو به عوض جان او و قوم تو به عوض قوم او خواهند بود.»پس پادشاه اسرائیل پریشان حال و مغموم شده، به خانه خود رفت و به سامره داخل شد.
\par 43 پس پادشاه اسرائیل پریشان حال و مغموم شده، به خانه خود رفت و به سامره داخل شد.
 
\chapter{21}

\par 1 و بعد از این امور، واقع شد که نابوت یزرعیلی، تاکستانی در یزرعیل به پهلوی قصر اخاب، پادشاه سامره، داشت.
\par 2 واخاب، نابوت را خطاب کرده، گفت: «تاکستان خود را به من بده تا باغ سبزی کاری، برای من بشود زیرا نزدیک خانه من است، و به عوض آن تاکستانی نیکوتر از آن به تو خواهم داد، یا اگر درنظرت پسند آید قیمتش را نقره خواهم داد.»
\par 3 نابوت به اخاب گفت: «حاشا بر من از خداوندکه ارث اجداد خود را به تو بدهم.»
\par 4 پس اخاب به‌سبب سخنی که نابوت یزرعیلی به او گفته بود، پریشان حال و مغموم شده، به خانه خود رفت زیرا گفته بود ارث اجداد خود را به تو نخواهم داد. و بر بستر خود دراز شده، رویش را برگردانیدو طعام نخورد.
\par 5 و زنش، ایزابل نزد وی آمده، وی را گفت: «روح تو چرا پریشان است که طعام نمی خوری؟»
\par 6 او وی را گفت: «از این جهت که نابوت یزرعیلی را خطاب کرده، گفتم: تاکستان خود را به نقره به من بده یا اگر بخواهی به عوض آن، تاکستان دیگری به تو خواهم داد، و او جواب داد که تاکستان خود را به تو نمی دهم.»
\par 7 زنش ایزابل به او گفت: «آیا تو الان بر اسرائیل سلطنت می‌کنی؟ برخیز و غذا بخور و دلت خوش باشد. من تاکستان نابوت یزرعیلی را به تو خواهم داد.»
\par 8 آنگاه مکتوبی به اسم اخاب نوشته، آن را به مهر او مختوم ساخت و مکتوب را نزد مشایخ ونجبایی که با نابوت در شهرش ساکن بودند، فرستاد.
\par 9 و در مکتوب بدین مضمون نوشت: «به روزه اعلان کنید و نابوت را به صدر قوم بنشانید.
\par 10 و دو نفر از بنی بلیعال را پیش او وا دارید که براو شهادت داده، بگویند که تو خدا و پادشاه را کفرگفته‌ای. پس او را بیرون کشیده، سنگسار کنید تابمیرد.»
\par 11 پس اهل شهرش، یعنی مشایخ ونجبایی که در شهر ساکن بودند، موافق پیغامی که ایزابل نزد ایشان فرستاده، و بر‌حسب مضمون مکتوبی که نزد ایشان ارسال کرده بود، به عمل آوردند.
\par 12 و به روزه اعلان کرده، نابوت را درصدر قوم نشانیدند.
\par 13 و دو نفر از بنی بلیعال درآمده، پیش وی نشستند و آن مردان بلیعال به حضور قوم بر نابوت شهادت داده، گفتند که نابوت بر خدا و پادشاه کفر گفته است، و او را ازشهر بیرون کشیده، وی را سنگسار کردند تا بمرد.
\par 14 و نزد ایزابل فرستاده، گفتند که نابوت سنگسارشده و مرده است.
\par 15 و چون ایزابل شنید که نابوت سنگسارشده، و مرده است، ایزابل به اخاب گفت: «برخیزو تاکستان نابوت یزرعیل را که او نخواست آن رابه تو به نقره بدهد، متصرف شو، زیرا که نابوت زنده نیست بلکه مرده است.»
\par 16 و چون اخاب شنید که نابوت مرده است اخاب برخاسته، به جهت تصرف تاکستان نابوت یزرعیلی فرود آمد.
\par 17 و کلام خداوند نزد ایلیای تشبی نازل شده، گفت:
\par 18 «برخیز و برای ملاقات اخاب، پادشاه اسرائیل که در سامره است فرود شو اینک او در تاکستان نابوت است که به آنجا فرود شد تا آن رامتصرف شود. 
\par 19 و او را خطاب کرده، بگوخداوند چنین می‌گوید: آیا هم قتل نمودی و هم متصرف شدی؟ و باز او را خطاب کرده، بگوخداوند چنین می‌گوید: در جایی که سگان خون نابوت را لیسیدند سگان خون تو را نیز خواهندلیسید.»
\par 20 اخاب به ایلیا گفت: «ای دشمن من، آیا مرایافتی؟» او جواب داد: «بلی تو را یافتم زیرا توخود را فروخته‌ای تا آنچه در نظر خداوند بداست، بجا آوری.
\par 21 اینک من بر تو بلا آورده، تورا بالکل هلاک خواهم ساخت، و از اخاب هرمرد را خواه محبوس و خواه آزاد در اسرائیل منقطع خواهم ساخت.
\par 22 و خاندان تو را مثل خاندان یربعام بن نباط و مانند خاندان بعشا ابن اخیا خواهم ساخت به‌سبب اینکه خشم مرا به هیجان آورده، و اسرائیل را مرتکب گناه ساخته‌ای.»
\par 23 و درباره ایزابل نیز خداوند تکلم نموده، گفت: «سگان ایزابل را نزد حصار یزرعیل خواهند خورد.
\par 24 هر‌که را از کسان اخاب درشهر بمیرد، سگان بخورند و هر‌که را در صحرابمیرد مرغان هوا بخورند.»
\par 25 و کسی نبود مثل اخاب که خویشتن رابرای بجا آوردن آنچه در نظر خداوند بد است فروخت، و زنش ایزابل او را اغوا نمود.
\par 26 و درپیروی بتها رجاسات بسیار می‌نمود، برحسب آنچه اموریانی که خداوند ایشان را از حضوربنی‌اسرائیل اخراج نموده بود، می‌کردند.
\par 27 و چون اخاب این سخنان را شنید، جامه خود را چاک زده، پلاس در بر کرد و روزه گرفته، بر پلاس خوابید و به سکوت راه می‌رفت.
\par 28 آنگاه کلام خداوند بر ایلیای تشبی نازل شده، گفت:«آیا اخاب را دیدی چگونه به حضورمن متواضع شده است؟ پس از این جهت که درحضور من تواضع می‌نماید، این بلا را در ایام وی نمی آوردم، لیکن در ایام پسرش، این بلا را برخاندانش عارض خواهم گردانید.»
\par 29 «آیا اخاب را دیدی چگونه به حضورمن متواضع شده است؟ پس از این جهت که درحضور من تواضع می‌نماید، این بلا را در ایام وی نمی آوردم، لیکن در ایام پسرش، این بلا را برخاندانش عارض خواهم گردانید.»
 
\chapter{22}

\par 1 و سه سال گذشت که در میان ارام واسرائیل جنگ نبود.
\par 2 و در سال سوم، یهوشافاط، پادشاه یهودا نزد پادشاه اسرائیل فرودآمد.
\par 3 و پادشاه اسرائیل به خادمان خود گفت: «آیا نمی دانید که راموت جلعاد از آن ماست و مااز گرفتنش از دست پادشاه ارام غافل می‌باشیم؟»
\par 4 پس به یهوشافاط گفت: «آیا همراه من به راموت جلعاد برای جنگ خواهی آمد؟» و یهوشافاطپادشاه اسرائیل را جواب داد که «من، چون تو وقوم من، چون قوم تو و سواران من، چون سواران تو می‌باشند.»
\par 5 و یهوشافاط به پادشاه اسرائیل گفت: «تمنااینکه امروز از کلام خداوند مسالت نمایی.»
\par 6 وپادشاه اسرائیل به قدر چهارصد نفر از انبیا جمع کرده، به ایشان گفت: «آیا به راموت جلعاد برای جنگ بروم یا باز ایستم؟» ایشان گفتند: «برآی وخداوند آن را به‌دست پادشاه تسلیم خواهدنمود.»
\par 7 اما یهوشافاط گفت: «آیا در اینجا غیر ازاینها نبی خداوند نیست تا از او سوال نماییم؟»
\par 8 وپادشاه اسرائیل به یهوشافاط گفت: «یک مرددیگر، یعنی میکایا ابن یمله هست که به واسطه او از خداوند مسالت توان کرد، لیکن من از او نفرت دارم زیرا که درباره من به نیکویی نبوت نمی کند، بلکه به بدی.» و یهوشافاط گفت: «پادشاه چنین نگوید.»
\par 9 پس پادشاه اسرائیل یکی از خواجه‌سرایان خود را خوانده، گفت: «میکایا ابن یمله رابه زودی حاضر کن.»
\par 10 و پادشاه اسرائیل ویهوشافاط، پادشاه یهودا، هر یکی لباس خود راپوشیده، بر کرسی خود در جای وسیع، نزد دهنه دروازه سامره نشسته بودند، و جمیع انبیا به حضور ایشان نبوت می‌کردند.
\par 11 و صدقیا ابن کنعنه شاخهای آهنین برای خود ساخته، گفت: «خداوند چنین می‌گوید: ارامیان را به اینهاخواهی زد تا تلف شوند.»
\par 12 و جمیع انبیا نبوت کرده، چنین می‌گفتند: «به راموت جلعاد برآی وفیروز شو زیرا خداوند آن را به‌دست پادشاه تسلیم خواهد نمود.»
\par 13 و قاصدی که برای طلبیدن میکایا رفته بود، او را خطاب کرده، گفت: «اینک انبیا به یک زبان درباره پادشاه نیکو می‌گویند. پس کلام تو مثل کلام یکی از ایشان باشد و سخنی نیکو بگو.»
\par 14 میکایا گفت: «به حیات خداوند قسم که هرآنچه خداوند به من بگوید همان را خواهم گفت.»
\par 15 پس چون نزد پادشاه رسید، پادشاه وی راگفت: «ای میکایا آیا به راموت جلعاد برای جنگ برویم یا باز ایستیم.» او در جواب وی گفت: «برآی و فیروز شو. و خداوند آن را به‌دست پادشاه تسلیم خواهد کرد.»
\par 16 پادشاه وی راگفت: «چند مرتبه تو را قسم بدهم که به اسم یهوه، غیر از آنچه راست است به من نگویی.»
\par 17 اوگفت: «تمامی اسرائیل را مثل گله‌ای که شبان ندارد بر کوهها پراکنده دیدم و خداوند گفت: اینها صاحب ندارند، پس هر کس به سلامتی به خانه خود برگردد.»
\par 18 و پادشاه اسرائیل به یهوشافاط گفت: «آیا تو را نگفتم که درباره من به نیکویی نبوت نمی کند بلکه به بدی.»
\par 19 او گفت: «پس کلام خداوند را بشنو: من خداوند را بر کرسی خود نشسته دیدم و تمامی لشکر آسمان نزد وی به طرف راست و چپ ایستاده بودند.
\par 20 و خداوند گفت: کیست که اخاب را اغوا نماید تا به راموت جلعاد برآمده، بیفتد. و یکی به اینطور سخن راند و دیگری به آنطور تکلم نمود.
\par 21 و آن روح (پلید) بیرون آمده، به حضور خداوند بایستاد و گفت: من او رااغوا می‌کنم.
\par 22 و خداوند وی را گفت: به چه چیز؟ او جواب داد که من بیرون می‌روم و در دهان جمیع انبیایش روح کاذب خواهم بود. او گفت: وی را اغوا خواهی کرد و خواهی توانست. پس برو و چنین بکن.
\par 23 پس الان خداوند روحی کاذب در دهان جمیع این انبیای تو گذاشته است وخداوند درباره تو سخن بد گفته است.»
\par 24 آنگاه صدقیا ابن کنعنه نزدیک آمده، به رخسار میکایا زد و گفت: «روح خداوند به کدام راه از نزد من به سوی تو رفت تا به تو سخن گوید؟»
\par 25 میکایا جواب داد: «اینک در روزی که به حجره اندرونی داخل شده، خود را پنهان کنی آن را خواهی دید.»
\par 26 و پادشاه اسرائیل گفت: «میکایا را بگیر و او را نزد آمون، حاکم شهر ویوآش، پسر پادشاه ببر.
\par 27 و بگو پادشاه چنین می‌فرماید: این شخص را در زندان بیندازید و اورا به نان تنگی و آب تنگی بپرورید تا من به سلامتی برگردم.»
\par 28 میکایا گفت: «اگر فی الواقع به سلامتی مراجعت کنی خداوند به من تکلم ننموده است، و گفت‌ای قوم جمیع بشنوید.»
\par 29 و پادشاه اسرائیل و یهوشافاط، پادشاه یهودا به راموت جلعاد برآمدند.
\par 30 و پادشاه اسرائیل به یهوشافاط گفت: «من خود را متنکرساخته، به جنگ می‌روم و تو لباس خود رابپوش.» پس پادشاه اسرائیل خود را متنکرساخته، به جنگ رفت.
\par 31 و پادشاه ارام سی و دوسردار ارابه های خود را امر کرده، گفت: «نه باکوچک و نه با بزرگ بلکه با پادشاه اسرائیل فقطجنگ نمایید.»
\par 32 و چون سرداران ارابه هایهوشافاط را دیدند، گفتند: «یقین این پادشاه اسرائیل است.» پس برگشتند تا با او جنگ نمایندو یهوشافاط فریاد برآورد.
\par 33 و چون سرداران ارابه‌ها دیدند که او پادشاه اسرائیل نیست، ازتعاقب او برگشتند.
\par 34 اما کسی کمان خود رابدون غرض کشیده، پادشاه اسرائیل را میان وصله های زره زد، و او به ارابه ران خود گفت: «دست خود را بگردان و مرا از لشکر بیرون ببرزیرا که مجروح شدم.»
\par 35 و در آن روز جنگ سخت شد و پادشاه را در ارابه‌اش به مقابل ارامیان برپا می‌داشتند، و وقت غروب مرد و خون زخمش به میان ارابه ریخت.
\par 36 و هنگام غروب آفتاب در لشکر ندا در‌داده، گفتند: «هر کس به شهر خود و هر کس به ولایت خویش برگردد.»
\par 37 و پادشاه مرد و او را به سامره آوردند و پادشاه را در سامره دفن کردند.
\par 38 و ارابه را در برکه سامره شستند و سگان خونش را لیسیدند واسلحه او را شستند، برحسب کلامی که خداوندگفته بود.
\par 39 و بقیه وقایع اخاب و هر‌چه او کرد وخانه عاجی که ساخت و تمامی شهرهایی که بناکرد، آیا در کتاب تواریخ ایام پادشاهان اسرائیل مکتوب نیست.
\par 40 پس اخاب با اجدادخود خوابید و پسرش، اخزیا به‌جایش سلطنت نمود.
\par 41 و یهوشافاط بن آسا در سال چهارم اخاب، پادشاه اسرائیل بر یهودا پادشاه شد.
\par 42 ویهوشافاط سی و پنج ساله بود که آغاز سلطنت نمود و بیست و پنج سال در اورشلیم سلطنت کردو اسم مادرش عزوبه دختر شلحی، بود.
\par 43 و درتمامی طریقهای پدرش، آسا سلوک نموده، ازآنها تجاوز نمی نمود و آنچه در نظر خداوندراست بود، بجا می‌آورد. مگر اینکه مکانهای بلندبرداشته نشد و قوم در مکانهای بلند قربانی همی گذرانیدند و بخور همی سوزانیدند.
\par 44 ویهوشافاط با پادشاه اسرائیل صلح کرد.
\par 45 و بقیه وقایع یهوشافاط و تهوری که نمود وجنگهایی که کرد، آیا در کتاب تواریخ ایام پادشاهان یهودا مکتوب نیست؟
\par 46 و بقیه الواطی که از ایام پدرش، آسا باقی‌مانده بودند، آنها را اززمین نابود ساخت.
\par 47 و در ادوم، پادشاهی نبود. لیکن وکیلی پادشاهی می‌کرد.
\par 48 و یهوشافاط کشتیهای ترشیشی ساخت تا به جهت آوردن طلا به اوفیربروند، اما نرفتند زیرا کشتیها در عصیون جابرشکست.
\par 49 آنگاه اخزیا ابن اخاب به یهوشافاطگفت: «بگذار که بندگان من با بندگان تو در کشتیهابروند.» اما یهوشافاط قبول نکرد.
\par 50 و یهوشافاطبا اجداد خود خوابید و با اجدادش در شهرپدرش، داود دفن شد و پسرش، یهورام در جایش سلطنت نمود.
\par 51 و اخزیا ابن اخاب در سال هفدهم یهوشافاط، پادشاه یهودا بر اسرائیل در سامره پادشاه شد، و دو سال بر اسرائیل پادشاهی نمود.و آنچه درنظر خداوند ناپسند بود، بجامی آورد و به طریق پدرش و طریق مادرش وطریق یربعام بن نباط که اسرائیل را مرتکب گناه ساخته بود، سلوک می‌نمود.
\par 52 و آنچه درنظر خداوند ناپسند بود، بجامی آورد و به طریق پدرش و طریق مادرش وطریق یربعام بن نباط که اسرائیل را مرتکب گناه ساخته بود، سلوک می‌نمود.


\end{document}