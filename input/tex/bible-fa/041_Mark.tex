\begin{document}

\title{Mark}


\chapter{1}

\par 1 ابتدا انجیل عیسی مسیح پسر خدا.
\par 2 چنانکه در اشعیا نبی مکتوب است، «اینک رسول خود را پیش روی تو می‌فرستم تاراه تو را پیش تو مهیا سازد.
\par 3 صدای ندا کننده‌ای در بیابان که راه خداوند را مهیا سازید و طرق او راراست نمایید.»
\par 4 یحیی تعمید‌دهنده در بیابان ظاهر شد وبجهت آمرزش گناهان به تعمید توبه موعظه می‌نمود.
\par 5 و تمامی مرز و بوم یهودیه و جمیع سکنه اورشلیم نزد وی بیرون شدند و به گناهان خود معترف گردیده، در رود اردون از او تعمیدمی یافتند.
\par 6 و یحیی را لباس از پشم شتر وکمربند چرمی بر کمر می‌بود و خوراک وی ازملخ و عسل بری.
\par 7 و موعظه می‌کرد و می‌گفت که «بعد از من کسی تواناتر از من می‌آید که لایق آن نیستم که خم شده، دوال نعلین او را باز کنم.
\par 8 من شما را به آب تعمید دادم. لیکن او شما را به روح‌القدس تعمید خواهد داد.»
\par 9 و واقع شد در آن ایام که عیسی از ناصره جلیل آمده در اردن از یحیی تعمید یافت.
\par 10 وچون از آب برآمد، در ساعت آسمان را شکافته دید و روح را که مانند کبوتری بروی نازل می‌شود.
\par 11 و آوازی از آسمان در‌رسید که «توپسر حبیب من هستی که از تو خشنودم.»
\par 12 پس بی‌درنگ روح وی را به بیابان می‌برد.
\par 13 و مدت چهل روز در صحرا بود و شیطان او راتجربه می‌کرد و با وحوش بسر می‌برد و فرشتگان او را پرستاری می‌نمودند.
\par 14 و بعد از گرفتاری یحیی، عیسی به جلیل آمده، به بشارت ملکوت خدا موعظه کرده،
\par 15 می گفت: «وقت تمام شد و ملکوت خدانزدیک است. پس توبه کنید و به انجیل ایمان بیاورید.»
\par 16 و چون به کناره دریای جلیل می‌گشت، شمعون و برادرش اندریاس را دید که دامی دردریا می‌اندازند زیرا که صیاد بودند.
\par 17 عیسی ایشان را گفت: «از عقب من آیید که شما را صیادمردم گردانم.»
\par 18 بی‌تامل دامهای خود را گذارده، از پی او روانه شدند.
\par 19 و از آنجا قدری پیشتررفته، یعقوب بن زبدی و برادرش یوحنا را دید که در کشتی دامهای خود را اصلاح می‌کنند.
\par 20 درحال ایشان را دعوت نمود. پس پدر خود زبدی را با مزدوران در کشتی گذارده، از عقب وی روانه شدند.
\par 21 و چون وارد کفرناحوم شدند، بی‌تامل درروز سبت به کنیسه درآمده، به تعلیم دادن شروع کرد،
\par 22 به قسمی که از تعلیم وی حیران شدند، زیرا که ایشان را مقتدرانه تعلیم می‌داد نه مانندکاتبان.
\par 23 و در کنیسه ایشان شخصی بود که روح پلیدداشت. ناگاه صیحه زده،
\par 24 گفت: «ای عیسی ناصری ما را با تو چه‌کار است؟ آیا برای هلاک کردن ما آمدی؟ تو را می‌شناسم کیستی، ای قدوس خدا!»
\par 25 عیسی به وی نهیب داده، گفت: «خاموش شو و از او درآی!»
\par 26 در ساعت آن روح خبیث او را مصروع نمود و به آواز بلند صدازده، از او بیرون آمد.
\par 27 و همه متعجب شدند، بحدی که از همدیگر سوال کرده، گفتند: «این چیست و این چه تعلیم تازه است که ارواح پلید رانیز با قدرت امر می‌کند و اطاعتش می‌نمایند؟»
\par 28 و اسم او فور در تمامی مرز و بوم جلیل شهرت یافت.
\par 29 و از کنیسه بیرون آمده، فور با یعقوب ویوحنا به خانه شمعون و اندریاس درآمدند.
\par 30 ومادر‌زن شمعون تب کرده، خوابیده بود. درساعت وی را از حالت او خبر دادند.
\par 31 پس نزدیک شده، دست او را گرفته، برخیزانیدش که همان وقت تب از او زایل شد و به خدمت گذاری ایشان مشغول گشت.
\par 32 شامگاه چون آفتاب به مغرب شد، جمیع مریضان و مجانین را پیش او آوردند.
\par 33 و تمام شهر بر در خانه ازدحام نمودند.
\par 34 و بسا کسانی را که به انواع امراض مبتلا بودند، شفا داد ودیوهای بسیاری بیرون کرده، نگذارد که دیوهاحرف زنند زیرا که او را شناختند.
\par 35 بامدادان قبل از صبح برخاسته، بیرون رفت و به ویرانه‌ای رسیده، در آنجا به دعا مشغول شد.
\par 36 و شمعون و رفقایش در‌پی او شتافتند.
\par 37 چون او را دریافتند، گفتند: «همه تو رامی طلبند.»
\par 38 بدیشان گفت: «به دهات مجاورهم برویم تا در آنها نیز موعظه کنم، زیرا که بجهت این کار بیرون آمدم.»
\par 39 پس در تمام جلیل درکنایس ایشان وعظ می‌نمود و دیوها را اخراج می‌کرد.
\par 40 و ابرصی پیش وی آمده، استدعا کرد وزانو زده، بدو گفت: «اگر بخواهی، می‌توانی مراطاهر سازی!»
\par 41 عیسی ترحم نموده، دست خودرا دراز کرد و او را لمس نموده، گفت: «می‌خواهم. طاهر شو!»
\par 42 و چون سخن گفت، فی الفور برص از او زایل شده، پاک گشت.
\par 43 و اورا قدغن کرد و فور مرخص فرموده،
\par 44 گفت: «زنهار کسی را خبر مده، بلکه رفته خود را به کاهن بنما و آنچه موسی فرموده، بجهت تطهیر خودبگذران تا برای ایشان شهادتی بشود.»لیکن اوبیرون رفته، به موعظه نمودن و شهرت دادن این امر شروع کرد، بقسمی که بعد از آن او نتوانست آشکارا به شهر درآید بلکه در ویرانه های بیرون بسر می‌برد و مردم از همه اطراف نزد وی می‌آمدند.
\par 45 لیکن اوبیرون رفته، به موعظه نمودن و شهرت دادن این امر شروع کرد، بقسمی که بعد از آن او نتوانست آشکارا به شهر درآید بلکه در ویرانه های بیرون بسر می‌برد و مردم از همه اطراف نزد وی می‌آمدند.

\chapter{2}

\par 1 و بعد از چندی، باز وارد کفرناحوم شده، چون شهرت یافت که در خانه است،
\par 2 بی درنگ جمعی ازدحام نمودند بقسمی که بیرون در نیز گنجایش نداشت و برای ایشان کلام را بیان می‌کرد.
\par 3 که ناگاه بعضی نزد وی آمده مفلوجی را به‌دست چهار نفر برداشته، آوردند.
\par 4 و چون به‌سبب جمعیت نتوانستند نزد او برسند، طاق جایی را که او بود باز کرده و شکافته، تختی را که مفلوج بر آن خوابیده بود، به زیر هشتند.
\par 5 عیسی چون ایمان ایشان را دید، مفلوج را گفت: «ای فرزند، گناهان تو آمرزیده شد.»
\par 6 لیکن بعضی از کاتبان که در آنجا نشسته بودند، در دل خود تفکر نمودند
\par 7 که «چرا این شخص چنین کفر می‌گوید؟ غیر از خدای واحد، کیست که بتواند گناهان را بیامرزد؟»
\par 8 در ساعت عیسی در روح خود ادراک نموده که با خود چنین فکر می‌کنند، بدیشان گفت: «از بهر‌چه این خیالات را به‌خاطر خود راه می‌دهید؟
\par 9 کدام سهل تر است؟ مفلوج را گفتن گناهان تو آمرزیده شد؟ یا گفتن برخیز و بستر خود را برداشته بخرام؟
\par 10 لیکن تا بدانید که پسر انسان رااستطاعت آمرزیدن گناهان بر روی زمین هست...» مفلوج را گفت:
\par 11 «تو را می‌گویم برخیز و بستر خود را برداشته، به خانه خود برو!»
\par 12 او برخاست و بی‌تامل بستر خود را برداشته، پیش روی همه روانه شد بطوری که همه حیران شده، خدا را تمجید نموده، گفتند: «مثل این امرهرگز ندیده بودیم!»
\par 13 و باز به کناره دریا رفت و تمام آن گروه نزداو آمدند و ایشان را تعلیم می‌داد.
\par 14 و هنگامی که می‌رفت لاوی ابن حلفی را بر باجگاه نشسته دید. بدو گفت: «از عقب من بیا!» پس برخاسته، درعقب وی شتافت.
\par 15 و وقتی که او در خانه وی نشسته بود، بسیاری از باجگیران و گناهکاران باعیسی و شاگردانش نشستند زیرا بسیار بودند وپیروی او می‌کردند.
\par 16 و چون کاتبان و فریسیان او را دیدند که با باجگیران و گناهکاران می‌خورد، به شاگردان او گفتند: «چرا با باجگیران وگناهکاران اکل و شرب می‌نماید؟»
\par 17 عیسی چون این را شنید، بدیشان گفت: «تندرستان احتیاج به طبیب ندارند بلکه مریضان. و من نیامدم تا عادلان را بلکه تا گناهکاران را به توبه دعوت کنم.»
\par 18 و شاگردان یحیی و فریسیان روزه می‌داشتند. پس آمده، بدو گفتند: «چون است که شاگردان یحیی و فریسیان روزه می‌دارند وشاگردان تو روزه نمی دارند؟»
\par 19 عیسی بدیشان گفت: «آیا ممکن است پسران خانه عروسی مادامی که داماد با ایشان است روزه بدارند؟ زمانی که داماد را با خود دارند، نمی توانند روزه‌دارند.
\par 20 لیکن ایامی می‌آید که داماد از ایشان گرفته شود. در آن ایام روزه خواهند داشت.
\par 21 وهیچ‌کس بر جامه کهنه، پاره‌ای از پارچه نو وصله نمی کند، والا آن وصله نو از آن کهنه جدامی گردد و دریدگی بدتر می‌شود.
\par 22 و کسی شراب نو را در مشکهای کهنه نمی ریزد وگرنه آن شراب نو مشکها را بدرد و شراب ریخته، مشکهاتلف می‌گردد. بلکه شراب نو را در مشکهای نوباید ریخت.»
\par 23 و چنان افتاد که روز سبتی از میان مزرعه‌ها می‌گذشت و شاگردانش هنگامی که می‌رفتند، به چیدن خوشه‌ها شروع کردند.
\par 24 فریسیان بدو گفتند: «اینک چرا در روزسبت مرتکب عملی می‌باشند که روا نیست؟»
\par 25 او بدیشان گفت: «مگر هرگز نخوانده ایدکه داود چه کرد چون او و رفقایش محتاج وگرسنه بودند؟
\par 26 چگونه در ایام ابیتار رئیس کهنه به خانه خدا درآمده، نان تقدمه راخورد که خوردن آن جز به کاهنان روا نیست و به رفقای خود نیز داد؟»
\par 27 و بدیشان گفت: «سبت بجهت انسان مقرر شد نه انسان برای سبت.بنابراین پسر انسان مالک سبت نیزهست.»
\par 28 بنابراین پسر انسان مالک سبت نیزهست.»

\chapter{3}

\par 1 و باز به کنیسه درآمده، در آنجا مرد دست خشکی بود.
\par 2 و مراقب وی بودند که شاید او را در سبت شفا دهد تا مدعی او گردند.
\par 3 پس بدان مرد دست خشک گفت: «در میان بایست!»
\par 4 و بدیشان گفت: «آیا در روز سبت کدام جایز است؟ نیکویی‌کردن یا بدی؟ جان رانجات دادن یا هلاک کردن؟» ایشان خاموش ماندند.
\par 5 پس چشمان خود را بر ایشان باغضب گردانیده، زیرا که از سنگدلی ایشان محزون بود، به آن مرد گفت: «دست خود رادراز کن!» پس دراز کرده، دستش صحیح گشت.
\par 6 در ساعت فریسیان بیرون رفته، با هیرودیان درباره او شورا نمودند که چطور او را هلاک کنند.
\par 7 و عیسی با شاگردانش به سوی دریا آمد وگروهی بسیار از جلیل به عقب او روانه شدند،
\par 8 واز یهودیه و از اورشلیم و ادومیه و آن طرف اردن و از حوالی صور و صیدون نیز جمعی کثیر، چون اعمال او را شنیدند، نزد وی آمدند.
\par 9 و به شاگردان خود فرمود تا زورقی به‌سبب جمعیت، بجهت او نگاه دارند تا بر وی ازدحام ننمایند،
\par 10 زیرا که بسیاری را صحت می‌داد، بقسمی که هر‌که صاحب دردی بود بر او هجوم می‌آورد تااو را لمس نماید.
\par 11 و ارواح پلید چون او رادیدند، پیش او به روی در‌افتادند و فریادکنان می‌گفتند که «تو پسر خدا هستی.»
\par 12 و ایشان را به تاکید بسیار فرمود که او را شهرت ندهند.
\par 13 پس بر فراز کوهی برآمده، هر‌که راخواست به نزد خود طلبید و ایشان نزد او آمدند.
\par 14 و دوازده نفر را مقرر فرمود تا همراه او باشند وتا ایشان را بجهت وعظ نمودن بفرستد،
\par 15 وایشان را قدرت باشد که مریضان را شفا دهند ودیوها را بیرون کنند.
\par 16 و شمعون را پطرس نام نهاد.
\par 17 و یعقوب پسر زبدی و یوحنا برادریعقوب؛ این هر دو را بوانرجس یعنی پسران رعد نام گذارد.
\par 18 و اندریاس و فیلپس و برتولما و متی و توما و یعقوب بن حلفی و تدی وشمعون قانوی،
\par 19 و یهودای اسخریوطی که او راتسلیم کرد.
\par 20 و چون به خانه درآمدند، باز جمعی فراهم آمدند بطوری که ایشان فرصت نان خوردن هم نکردند.
\par 21 و خویشان او چون شنیدند، بیرون آمدند تا او را بردارند زیرا گفتند بی‌خود شده است.
\par 22 و کاتبانی که از اورشلیم آمده بودند، گفتند که بعلزبول دارد و به یاری رئیس دیوها، دیوها را اخراج می‌کند.
\par 23 پس ایشان را پیش طلبیده، مثلها زده، بدیشان گفت: «چطور می‌تواندشیطان، شیطان را بیرون کند؟
\par 24 و اگر مملکتی برخلاف خود منقسم شود، آن مملکت نتواندپایدار بماند.
\par 25 و هرگاه خانه‌ای به ضد خویش منقسم شد، آن خانه نمی تواند استقامت داشته باشد.
\par 26 و اگر شیطان با نفس خود مقاومت نمایدو منقسم شود، او نمی تواند قائم ماند بلکه هلاک می‌گردد.
\par 27 و هیچ‌کس نمی تواند به خانه مردزورآور درآمده، اسباب او را غارت نماید، جزآنکه اول آن زورآور را ببندد و بعد از آن خانه اورا تاراج می‌کند.
\par 28 هرآینه به شما می‌گویم که همه گناهان از بنی آدم آمرزیده می‌شود و هر قسم کفر که گفته باشند،
\par 29 لیکن هر‌که به روح‌القدس کفر گوید، تا به ابد آمرزیده نشود بلکه مستحق عذاب جاودانی بود.»
\par 30 زیرا که می‌گفتند روحی پلید دارد.
\par 31 پس برادران و مادر او آمدند و بیرون ایستاده، فرستادند تا او را طلب کنند.
\par 32 آنگاه جماعت گرد او نشسته بودند و به وی گفتند: «اینک مادرت و برادرانت بیرون تو را می‌طلبند.»
\par 33 در جواب ایشان گفت: «کیست مادر من وبرادرانم کیانند؟»
\par 34 پس بر آنانی که گرد وی نشسته بودند، نظر افکنده، گفت: «اینانند مادر وبرادرانم،زیرا هر‌که اراده خدا را به‌جا آردهمان برادر و خواهر و مادر من باشد.»
\par 35 زیرا هر‌که اراده خدا را به‌جا آردهمان برادر و خواهر و مادر من باشد.»

\chapter{4}

\par 1 و باز به کناره دریا به تعلیم دادن شروع کردو جمعی کثیر نزد او جمع شدند بطوری که به کشتی سوار شده، بر دریا قرار گرفت و تمامی آن جماعت بر ساحل دریا حاضر بودند.
\par 2 پس ایشان را به مثلها چیزهای بسیار می‌آموخت و درتعلیم خود بدیشان گفت:
\par 3 «گوش گیرید! اینک برزگری بجهت تخم پاشی بیرون رفت.
\par 4 و چون تخم می‌پاشید، قدری بر راه ریخته شده، مرغان هوا آمده آنها را برچیدند.
\par 5 و پاره‌ای بر سنگلاخ پاشیده شد، در جایی که خاک بسیار نبود. پس چون که زمین عمق نداشت به زودی رویید،
\par 6 وچون آفتاب برآمد، سوخته شد و از آنرو که ریشه نداشت خشکید.
\par 7 و قدری در میان خارها ریخته شد و خارها نمو کرده، آن را خفه نمود که ثمری نیاورد.
\par 8 و مابقی در زمین نیکو افتاد و حاصل پیدا نمود که رویید و نمو کرد و بارآورد، بعضی سی وبعضی شصت و بعضی صد.»
\par 9 پس گفت: «هر‌که گوش شنوا دارد، بشنود!»
\par 10 و چون به خلوت شد، رفقای او با آن دوازده شرح این مثل را از او پرسیدند.
\par 11 به ایشان گفت: «به شما دانستن سر ملکوت خدا عطاشده، اما به آنانی که بیرونند، همه‌چیز به مثلهامی شود،
\par 12 تا نگران شده بنگرند و نبینند و شنواشده بشنوند و نفهمند، مبادا بازگشت کرده گناهان ایشان آمرزیده شود.»
\par 13 و بدیشان گفت: «آیا این مثل رانفهمیده‌اید؟ پس چگونه سایر مثلها را خواهیدفهمید؟
\par 14 برزگر کلام را می‌کارد.
\par 15 و اینانند به کناره راه، جایی که کلام کاشته می‌شود؛ و چون شنیدند فور شیطان آمده کلام کاشته شده درقلوب ایشان را می‌رباید.
\par 16 و ایض کاشته شده درسنگلاخ، کسانی می‌باشند که چون کلام رابشنوند، در حال آن را به خوشی قبول کنند،
\par 17 ولکن ریشه‌ای در خود ندارند بلکه فانی می‌باشند؛ و چون صدمه‌ای یا زحمتی به‌سبب کلام روی دهد در ساعت لغزش می‌خورند.
\par 18 و کاشته شده در خارها آنانی می‌باشند که چون کلام راشنوند،
\par 19 اندیشه های دنیوی و غرور دولت وهوس چیزهای دیگر داخل شده، کلام را خفه می‌کند و بی‌ثمر می‌گردد.
\par 20 و کاشته شده درزمین نیکو آنانند که چون کلام را شنوند آن رامی پذیرند و ثمر می‌آورند، بعضی سی و بعضی شصت و بعضی صد.»
\par 21 پس بدیشان گفت: «آیا چراغ را می‌آورند تازیر پیمانه‌ای یا تختی و نه بر چراغدان گذارند؟
\par 22 زیرا که چیزی پنهان نیست که آشکارا نگردد وهیچ‌چیز مخفی نشود، مگر تا به ظهور آید.
\par 23 هرکه گوش شنوا دارد بشنود.»
\par 24 و بدیشان گفت: «باحذر باشید که چه می‌شنوید، زیرا به هر میزانی که وزن کنید به شما پیموده شود، بلکه از برای شماکه می‌شنوید افزون خواهد گشت.
\par 25 زیرا هر‌که دارد بدو داده شود و از هر‌که ندارد آنچه نیز داردگرفته خواهد شد.»
\par 26 و گفت: «همچنین ملکوت خدا مانند کسی است که تخم بر زمین بیفشاند،
\par 27 و شب و روزبخوابد و برخیزد و تخم بروید و نمو کند. چگونه؟ او نداند.
\par 28 زیرا که زمین به ذات خودثمر می‌آورد، اول علف، بعد خوشه، پس از آن دانه کامل در خوشه.
\par 29 و چون ثمر رسید، فور داس را بکار می‌برد زیرا که وقت حصاد رسیده است.»
\par 30 و گفت: «به چه چیز ملکوت خدا را تشبیه کنیم و برای آن چه مثل بزنیم؟
\par 31 مثل دانه خردلی است که وقتی که آن را بر زمین کارند، کوچکترین تخمهای زمینی باشد.
\par 32 لیکن چون کاشته شد، می‌روید و بزرگتر از جمیع بقول می‌گردد و شاخه های بزرگ می‌آورد، چنانکه مرغان هوا زیر سایه‌اش می‌توانند آشیانه گیرند.»
\par 33 و به مثلهای بسیار مانند اینهابقدری که استطاعت شنیدن داشتند، کلام رابدیشان بیان می‌فرمود،
\par 34 و بدون مثل بدیشان سخن نگفت. لیکن در خلوت، تمام معانی را برای شاگردان خود شرح می‌نمود.
\par 35 و در همان روز وقت شام، بدیشان گفت: «به کناره دیگر عبور کنیم.»
\par 36 پس چون آن گروه رارخصت دادند، او را همانطوری که در کشتی بودبرداشتند و چند زورق دیگر نیز همراه او بود.
\par 37 که ناگاه طوفانی عظیم از باد پدید آمد وامواج بر کشتی می‌خورد بقسمی که برمی گشت.
\par 38 و او در موخر کشتی بر بالشی خفته بود. پس اورا بیدار کرده گفتند: «ای استاد، آیا تو را باکی نیست که هلاک شویم؟»
\par 39 در ساعت اوبرخاسته، باد را نهیب داد و به دریا گفت: «ساکن شو و خاموش باش!» که باد ساکن شده، آرامی کامل پدید آمد.
\par 40 و ایشان را گفت: «از بهر‌چه چنین ترسانید و چون است که ایمان ندارید؟»پس بی‌نهایت ترسان شده، به یکدیگر گفتند: «این کیست که باد و دریا هم او را اطاعت می‌کنند؟»
\par 41 پس بی‌نهایت ترسان شده، به یکدیگر گفتند: «این کیست که باد و دریا هم او را اطاعت می‌کنند؟»

\chapter{5}

\par 1 پس به آن کناره دریا تا به‌سرزمین جدریان آمدند.
\par 2 و چون از کشتی بیرون آمد، فی الفور شخصی که روحی پلید داشت از قبوربیرون شده، بدو برخورد.
\par 3 که در قبور ساکن می‌بود و هیچ‌کس به زنجیرها هم نمی توانست اورا بند نماید،
\par 4 زیرا که بارها او را به کنده‌ها وزنجیرها بسته بودند و زنجیرها را گسیخته وکنده‌ها را شکسته بود و احدی نمی توانست او رارام نماید،
\par 5 و پیوسته شب وروز در کوهها وقبرها فریا می‌زد و خود را به سنگها مجروح می‌ساخت.
\par 6 چون عیسی را از دور دید، دوان دوان آمده، او را سجده کرد،
\par 7 و به آواز بلندصیحه زده، گفت: «ای عیسی، پسر خدای تعالی، مرا با تو چه‌کار است؟ تو را به خدا قسم می‌دهم که مرا معذب نسازی.»
\par 8 زیرا بدو گفته بود: «ای روح پلید از این شخص بیرون بیا!»
\par 9 پس از اوپرسید: «اسم تو چیست؟» «به وی گفت: «نام من لجئون است زیرا که بسیاریم.»
\par 10 پس بدوالتماس بسیار نمود که ایشان را از آن سرزمین بیرون نکند.
\par 11 و در حوالی آن کوهها، گله گرازبسیاری می‌چرید.
\par 12 و همه دیوها از وی خواهش نموده، گفتند: «ما را به گرازها بفرست تادر آنها داخل شویم.»
\par 13 فور عیسی ایشان رااجازت داد. پس آن ارواح خبیث بیرون شده، به گرازان داخل گشتند و آن گله از بلندی به دریاجست و قریب بدو هزار بودند که در آب خفه شدند.
\par 14 و خوک با نان فرار کرده، در شهر ومزرعه‌ها خبر می‌دادند و مردم بجهت دیدن آن ماجرا بیرون شتافتند.
\par 15 و چون نزد عیسی رسیده، آن دیوانه را که لجئون داشته بود دیدند که نشسته و لباس پوشیده و عاقل گشته است، بترسیدند.
\par 16 و آنانی که دیده بودند، سرگذشت دیوانه و گرازان را بدیشان بازگفتند.
\par 17 پس شروع به التماس نمودند که از حدود ایشان روانه شود.
\par 18 و چون به کشتی سوار شد، آنکه دیوانه بود ازوی استدعا نمود که با وی باشد.
\par 19 اما عیسی وی را اجازت نداد بلکه بدو گفت: «به خانه نزدخویشان خود برو و ایشان را خبر ده از آنچه خداوند با تو کرده است و چگونه به تو رحم نموده است.»
\par 20 پس روانه شده، در دیکاپولس به آنچه عیسی با وی کرده، موعظه کردن آغاز نمودکه همه مردم متعجب شدند.
\par 21 و چون عیسی باز به آنطرف، در کشتی عبور نمود، مردم بسیار بر وی جمع گشتند و برکناره دریا بود.
\par 22 که ناگاه یکی از روسای کنیسه، یایرس نام آمد و چون او را بدید بر پایهایش افتاده،
\par 23 بدو التماس بسیار نموده، گفت: «نفس دخترک من به آخر رسیده. بیا و بر او دست گذار تاشفا یافته، زیست کند.»
\par 24 پس با او روانه شده، خلق بسیاری نیز از پی او افتاده، بر وی ازدحام می‌نمودند.
\par 25 آنگاه زنی که مدت دوازده سال به استحاضه مبتلا می‌بود،
\par 26 و زحمت بسیار ازاطبای متعدد دیده و آنچه داشت صرف نموده، فایده‌ای نیافت بلکه بدتر می‌شد،
\par 27 چون خبرعیسی را بشنید، میان آن گروه از عقب وی آمده ردای او را لمس نمود،
\par 28 زیرا گفته بود: «اگرلباس وی را هم لمس کنم، هرآینه شفا یابم.»
\par 29 در ساعت چشمه خون او خشک شده، در تن خود فهمید که از آن بلا صحت یافته است.
\par 30 فی الفور عیسی از خود دانست که قوتی از اوصادر گشته. پس در آن جماعت روی برگردانیده، گفت: «کیست که لباس مرا لمس نمود؟»
\par 31 شاگردانش بدو گفتند: «می‌بینی که مردم بر توازدحام می‌نمایند! و می‌گویی کیست که مرا لمس نمود؟ !»
\par 32 پس به اطراف خود می‌نگریست تا آن زن را که این کار کرده، ببیند.
\par 33 آن زن چون دانست که به وی چه واقع شده، ترسان و لرزان آمد و نزد او به روی در‌افتاده، حقیقت امر رابالتمام به وی باز‌گفت.
\par 34 او وی را گفت: «ای دختر، ایمانت تو را شفا داده است. به سلامتی بروو از بلای خویش رستگار باش.»
\par 35 او هنوز سخن می‌گفت که بعضی از خانه رئیس کنیسه آمده، گفتند: «دخترت فوت شده؛ دیگر برای چه استاد را زحمت می‌دهی؟»
\par 36 عیسی چون سخنی را که گفته بودند شنید، درساعت به رئیس کنیسه گفت: «مترس ایمان آور وبس!»
\par 37 و جز پطرس و یعقوب و یوحنا برادریعقوب، هیچ‌کس را اجازت نداد که از عقب اوبیایند.
\par 38 پس چون به خانه رئیس کنیسه رسیدند، جمعی شوریده دید که گریه و نوحه بسیار می‌نمودند.
\par 39 پس داخل شده، بدیشان گفت: «چرا غوغا و گریه می‌کنید؟ دختر نمرده بلکه در خواب است.»
\par 40 ایشان بر وی سخریه کردند لیکن او همه را بیرون کرده، پدر و مادردختر را با رفیقان خویش برداشته، به‌جایی که دختر خوابیده بود، داخل شد.
\par 41 پس دست دختر را گرفته، به وی گفت: «طلیتا قومی.» که معنی آن این است: «ای دختر، تو را می‌گویم برخیز.»
\par 42 در ساعت دختر برخاسته، خرامید زیرا که دوازده ساله بود. ایشان بینهایت متعجب شدند.پس ایشان را به تاکید بسیار فرمود: «کسی از این امر مطلع نشود.» و گفت تا خوراکی بدو دهند.
\par 43 پس ایشان را به تاکید بسیار فرمود: «کسی از این امر مطلع نشود.» و گفت تا خوراکی بدو دهند.

\chapter{6}

\par 1 پس از آنجا روانه شده، به وطن خویش آمد و شاگردانش از عقب او آمدند.
\par 2 چون روز سبت رسید، در کنیسه تعلیم دادن آغاز نمودو بسیاری چون شنیدند، حیران شده گفتند: «ازکجا بدین شخص این چیزها رسیده و این چه حکمت است که به او عطا شده است که چنین معجزات از دست او صادر می‌گردد؟
\par 3 مگر این نیست نجار پسر مریم و برادر یعقوب و یوشا ویهودا و شمعون؟ و خواهران او اینجا نزد مانمی باشند؟» و از او لغزش خوردند.
\par 4 عیسی ایشان را گفت: «نبی بی‌حرمت نباشد جز در وطن خود و میان خویشان و در خانه خود.
\par 5 و در آنجاهیچ معجزه‌ای نتوانست نمود جز اینکه دستهای خود را بر چند مریض نهاده، ایشان را شفا داد.
\par 6 واز بی‌ایمانی‌ایشان متعجب شده، در دهات آن حوالی گشته، تعلیم همی داد.
\par 7 پس آن دوازده را پیش خوانده، شروع کرد به فرستادن ایشان جفت جفت و ایشان را بر ارواح پلید قدرت داد،
\par 8 و ایشان را قدغن فرمود که «جزعصا فقط، هیچ‌چیز برندارید، نه توشه‌دان و نه پول در کمربند خود،
\par 9 بلکه موزه‌ای در پا کنید ودو قبا در بر نکنید.»
\par 10 و بدیشان گفت: «در هر جاداخل خانه‌ای شوید، در آن بمانید تا از آنجا کوچ کنید.
\par 11 و هرجا که شما را قبول نکنند و به سخن شما گوش نگیرند، از آن مکان بیرون رفته، خاک پایهای خود را بیفشانید تا بر آنها شهادتی گردد. هرآینه به شما می‌گویم حالت سدوم و غموره درروز جزا از آن شهر سهل تر خواهد بود.»
\par 12 پس روانه شده، موعظه کردند که توبه کنند،
\par 13 وبسیار دیوها را بیرون کردند و مریضان کثیر راروغن مالیده، شفا دادند.
\par 14 و هیرودیس پادشاه شنید زیرا که اسم اوشهرت یافته بود و گفت که «یحیی تعمید‌دهنده از مردگان برخاسته است و از این جهت معجزات از او به ظهور می‌آید.
\par 15 اما بعضی گفتند که الیاس است و بعضی گفتند که نبی‌ای است یا چون یکی از انبیا.
\par 16 اما هیرودیس چون شنید گفت: «این همان یحیی است که من سرش را از تن جدا کردم که از مردگان برخاسته است.»
\par 17 زیرا که هیرودیس فرستاده، یحیی را گرفتار نموده، او رادر زندان بست بخاطر هیرودیا، زن برادر او فیلپس که او را در نکاح خویش آورده بود.
\par 18 از آن جهت که یحیی به هیرودیس گفته بود: «نگاه داشتن زن برادرت بر تو روا نیست.»
\par 19 پس هیرودیا از او کینه داشته، می‌خواست اور ا به قتل رساند اما نمی توانست،
\par 20 زیرا که هیرودیس از یحیی می‌ترسید چونکه او را مرد عادل و مقدس می‌دانست و رعایتش می‌نمود و هرگاه از اومی شنید بسیار به عمل می‌آورد و به خوشی سخن او را اصغا می‌نمود.
\par 21 اما چون هنگام فرصت رسید که هیرودیس در روز میلاد خودامرای خود و سرتیبان و روسای جلیل را ضیافت نمود؛
\par 22 و دختر هیرودیا به مجلس درآمده، رقص کرد و هیرودیس و اهل مجلس را شادنمود. پادشاه بدان دختر گفت: «آنچه خواهی ازمن بطلب تا به تو دهم.»
\par 23 و از برای او قسم خوردکه آنچه از من خواهی حتی نصف ملک مراهرآینه به تو عطا کنم.»
\par 24 او بیرون رفته، به مادرخود گفت: «چه بطلبم؟» گفت: «سر یحیی تعمیددهنده را.»
\par 25 در ساعت به حضور پادشاه درآمده، خواهش نموده، گفت: «می‌خواهم که الان سر یحیی تعمید‌دهنده را در طبقی به من عنایت فرمایی.»
\par 26 پادشاه به شدت محزون گشت، لیکن بجهت پاس قسم و خاطر اهل مجلس نخواست او را محروم نماید.
\par 27 بی‌درنگ پادشاه جلادی فرستاده، فرمود تا سرش رابیاورد.
\par 28 و او به زندان رفته سر او را از تن جداساخته و بر طبقی آورده، بدان دختر داد و دخترآن را به مادر خود سپرد.
\par 29 چون شاگردانش شنیدند، آمدند و بدن او را برداشته، دفن کردند.
\par 30 و رسولان نزد عیسی جمع شده، از آنچه کرده و تعلیم داده بودند او را خبر دادند.
\par 31 بدیشان گفت شما به خلوت، به‌جای ویران بیایید و اندکی استراحت نمایید زیرا آمد و رفت چنان بود که فرصت نان خوردن نیز نکردند.
\par 32 پس به تنهایی در کشتی به موضعی ویران رفتند.
\par 33 و مردم ایشان را روانه دیده، بسیاری اورا شناختند و از جمیع شهرها بر خشکی بدان سوشتافتند و از ایشان سبقت جسته، نزد وی جمع شدند.
\par 34 عیسی بیرون آمده، گروهی بسیار دیده، بر ایشان ترحم فرمود زیرا که چون گوسفندان بی‌شبان بودند و بسیار به ایشان تعلیم دادن گرفت.
\par 35 و چون بیشتری از روز سپری گشت، شاگردانش نزد وی آمده، گفتند: «این مکان ویرانه است و وقت منقضی شده.
\par 36 اینها را رخصت ده تا به اراضی و دهات این نواحی رفته، نان بجهت خود بخرند که هیچ خوراکی ندارند.»
\par 37 درجواب ایشان گفت: «شما ایشان را غذا دهید!» وی را گفتند: «مگر رفته، دویست دینار نان بخریم تا اینها را طعام دهیم!»
\par 38 بدیشان گفت: «چند نان دارید؟ رفته، تحقیق کنید.» پس دریافت کرده، گفتند: «پنج نان و دو ماهی.»
\par 39 آنگاه ایشان رافرمود که «همه را دسته دسته بر سبزه بنشانید.»
\par 40 پس صف صف، صد صد و پنجاه پنجاه نشستند.
\par 41 و آن پنج نان و دو ماهی را گرفته، به سوی آسمان نگریسته، برکت داد و نان را پاره نموده، به شاگردان خود بسپرد تا پیش آنهابگذارند و آن دو ماهی را بر همه آنها تقسیم نمود.
\par 42 پس جمیع خورده، سیر شدند.
\par 43 و ازخرده های نان و ماهی، دوازده سبد پر کرده، برداشتند.
\par 44 و خورندگان نان، قریب به پنج هزارمرد بودند.
\par 45 فی الفور شاگردان خود را الحاح فرمود که به کشتی سوار شده، پیش از او به بیت صیدا عبورکنند تا خود آن جماعت را مرخص فرماید.
\par 46 وچون ایشان را مرخص نمود، بجهت عبادت به فراز کوهی برآمد.
\par 47 و چون شام شد، کشتی درمیان دریا رسید و او تنها بر خشکی بود.
\par 48 وایشان را در راندن کشتی خسته دید زیرا که بادمخالف بر ایشان می‌وزید. پس نزدیک پاس چهارم از شب بر دریا خرامان شده، به نزد ایشان آمد و خواست از ایشان بگذرد.
\par 49 اما چون او رابر دریا خرامان دیدند، تصور نمودند که این خیالی است. پس فریاد برآوردند،
\par 50 زیرا که همه او را دیده، مضطرب شدند. پس بی‌درنگ بدیشان خطاب کرده، گفت: «خاطر جمع دارید! من هستم، ترسان مباشید!»
\par 51 و تا نزد ایشان به کشتی سوار شد، باد ساکن گردید چنانکه بینهایت درخود متحیر و متعجب شدند،
\par 52 زیرا که معجزه نان را درک نکرده بودند زیرا دل ایشان سخت بود.
\par 53 پس از دریا گذشته، به‌سرزمین جنیسارت آمده، لنگر انداختند.
\par 54 و چون از کشتی بیرون شدند، مردم در حال او را شناختند،
\par 55 و در همه آن نواحی بشتاب می‌گشتند و بیماران را بر تختهانهاده، هر جا که می‌شنیدند که او در آنجا است، می‌آوردند.و هر جایی که به دهات یا شهرها یا اراضی می‌رفت، مریضان را بر راهها می‌گذاردندو از او خواهش می‌نمودند که محض دامن ردای او را لمس کنند و هر‌که آن را لمس می‌کرد شفامی یافت.
\par 56 و هر جایی که به دهات یا شهرها یا اراضی می‌رفت، مریضان را بر راهها می‌گذاردندو از او خواهش می‌نمودند که محض دامن ردای او را لمس کنند و هر‌که آن را لمس می‌کرد شفامی یافت.

\chapter{7}

\par 1 و فریسیان و بعضی کاتبان از اورشلیم آمده، نزد او جمع شدند.
\par 2 چون بعضی ازشاگردان او را دیدند که با دستهای ناپاک یعنی ناشسته نان می‌خورند، ملامت نمودند،
\par 3 زیرا که فریسیان و همه یهود تمسک به تقلید مشایخ نموده، تا دستها را بدقت نشویند غذا نمی خورند،
\par 4 و چون از بازارها آیند تا نشویند چیزی نمی خورند و بسیار رسوم دیگر هست که نگاه می‌دارند چون شستن پیاله‌ها و آفتابه‌ها و ظروف مس و کرسیها.
\par 5 پس فریسیان و کاتبان از اوپرسیدند: «چون است که شاگردان تو به تقلیدمشایخ سلوک نمی نمایند بلکه به‌دستهای ناپاک نان می‌خورند؟»
\par 6 در جواب ایشان گفت: «نیکو اخبار نموداشعیا درباره شما‌ای ریاکاران، چنانکه مکتوب است: این قوم به لبهای خود مرا حرمت می‌دارندلیکن دلشان از من دور است.
\par 7 پس مرا عبث عبادت می‌نمایند زیرا که رسوم انسانی را به‌جای فرایض تعلیم می‌دهند،
\par 8 زیرا حکم خدا را ترک کرده، تقلید انسان را نگاه می‌دارند، چون شستن آفتابه‌ها و پیاله‌ها و چنین رسوم دیگر بسیار بعمل می‌آورید.»
\par 9 پس بدیشان گفت که «حکم خدا رانیکو باطل ساخته‌اید تا تقلید خود را محکم بدارید.
\par 10 از اینجهت که موسی گفت پدر و مادرخود را حرمت دار و هر‌که پدر یا مادر را دشنام دهد، البته هلاک گردد.
\par 11 لیکن شما می‌گویید که هرگاه شخصی به پدر یا مادر خود گوید: "آنچه ازمن نفع یابی قربان یعنی هدیه برای خداست "
\par 12 وبعد از این او را اجازت نمی دهید که پدر یا مادرخود را هیچ خدمت کند.
\par 13 پس کلام خدا را به تقلیدی که خود جاری ساخته‌اید، باطل می‌سازید و کارهای مثل این بسیار به‌جامی آورید.»
\par 14 پس آن جماعت را پیش خوانده، بدیشان گفت: «همه شما به من گوش دهید و فهم کنید.
\par 15 هیچ‌چیز نیست که از بیرون آدم داخل او گشته، بتواند او را نجس سازد بلکه آنچه از درونش صادر شود آن است که آدم را ناپاک می‌سازد.
\par 16 هر‌که گوش شنوا دارد بشنود.»
\par 17 و چون از نزد جماعت به خانه در‌آمد، شاگردانش معنی مثل را از او پرسیدند.
\par 18 بدیشان گفت: «مگر شما نیز همچنین بی‌فهم هستید ونمی دانید که آنچه از بیرون داخل آدم می‌شود، نمی تواند او را ناپاک سازد،
\par 19 زیرا که داخل دلش نمی شود بلکه به شکم می‌رود و خارج می‌شود به مزبله‌ای که این همه خوراک را پاک می‌کند.»
\par 20 وگفت: «آنچه از آدم بیرون آید، آن است که انسان را ناپاک می‌سازد،
\par 21 زیرا که از درون دل انسان صادر می‌شود، خیالات بد و زنا و فسق و قتل ودزدی
\par 22 و طمع و خباثت و مکر و شهوت‌پرستی و چشم بد و کفر و غرور و جهالت.
\par 23 تمامی این چیزهای بد از درون صادر می‌گردد و آدم را ناپاک می‌گرداند.»
\par 24 پس از آنجا برخاسته به حوالی صور وصیدون رفته، به خانه درآمد و خواست که هیچ‌کس مطلع نشود، لیکن نتوانست مخفی بماند،
\par 25 از آنرو که زنی که دخترک وی روح پلیدداشت، چون خبر او را بشنید، فور آمده برپایهای او افتاد.
\par 26 و او زن یونانی از اهل فینیقیه صوریه بود. پس از وی استدعا نمود که دیو را ازدخترش بیرون کند.
\par 27 عیسی وی را گفت: «بگذار اول فرزندان سیر شوند زیرا نان فرزندان راگرفتن و پیش سگان انداختن نیکو نیست.»
\par 28 آن زن در جواب وی گفت: «بلی خداوندا، زیرا سگان نیز پس خرده های فرزندان را از زیر سفره می‌خورند.»
\par 29 وی را گفت؛ «بجهت این سخن برو که دیو از دخترت بیرون شد.»
\par 30 پس چون به خانه خود رفت، دیو را بیرون شده و دختر را بربستر خوابیده یافت.
\par 31 و باز از نواحی صور روانه شده، از راه صیدون در میان حدود دیکاپولس به دریای جلیل آمد.
\par 32 آنگاه کری را که لکنت زبان داشت نزد وی آورده، التماس کردند که دست بر او گذارد.
\par 33 پس او را از میان جماعت به خلوت برده، انگشتان خود را در گوشهای او گذاشت و آب دهان انداخته، زبانش را لمس نمود؛
\par 34 و به سوی آسمان نگریسته، آهی کشید و بدو گفت: «افتح!» یعنی باز شو
\par 35 در ساعت گوشهای او گشاده وعقده زبانش حل شده، به درستی تکلم نمود.
\par 36 پس ایشان را قدغن فرمود که هیچ‌کس را خبرندهند؛ لیکن چندان‌که بیشتر ایشان را قدغن نمود، زیادتر او را شهرت دادند.و بینهایت متحیر گشته می‌گفتند: «همه کارها را نیکوکرده است؛ کران را شنوا و گنگان را گویامی گرداند!»
\par 37 و بینهایت متحیر گشته می‌گفتند: «همه کارها را نیکوکرده است؛ کران را شنوا و گنگان را گویامی گرداند!»

\chapter{8}

\par 1 و در آن ایام باز جمعیت، بسیار شده و خوراکی نداشتند. عیسی شاگردان خود راپیش طلبیده، به ایشان گفت:
\par 2 «بر این گروه دلم بسوخت زیرا الان سه روز است که با من می‌باشندو هیچ خوراک ندارند.
\par 3 و هرگاه ایشان را گرسنه به خانه های خود برگردانم، هرآینه در راه ضعف کنند، زیرا که بعضی از ایشان از راه دور آمده‌اند.»
\par 4 شاگردانش وی را جواب دادند: «از کجا کسی می‌تواند اینها را در این صحرا از نان سیر گرداند؟»
\par 5 از ایشان پرسید: «چند نان دارید؟» گفتند: «هفت.»
\par 6 پس جماعت را فرمود تا بر زمین بنشینند؛ و آن هفت نان را گرفته، شکر نمود و پاره کرده، به شاگردان خود داد تا پیش مردم گذارند. پس نزد آن گروه نهادند.
\par 7 و چند ماهی کوچک نیز داشتند. آنها را نیز برکت داده، فرمود تا پیش ایشان نهند.
\par 8 پس خورده، سیر شدند و هفت زنبیل پر از پاره های باقی‌مانده برداشتند.
\par 9 و عددخورندگان قریب به چهار هزار بود. پس ایشان رامرخص فرمود.
\par 10 و بی‌درنگ با شاگردان به کشتی سوار شده، به نواحی دلمانوته آمد.
\par 11 و فریسیان بیرون آمده، با وی به مباحثه شروع کردند. و از راه امتحان آیتی آسمانی از او خواستند.
\par 12 او از دل آهی کشیده، گفت: «از برای چه این فرقه آیتی می‌خواهند؟ هرآینه به شما می‌گویم آیتی بدین فرقه عطا نخواهد شد.»
\par 13 پس ایشان را گذارد و باز به کشتی سوارشده، به کناره دیگر عبور نمود.
\par 14 و فراموش کردند که نان بردارند و با خود در کشتی جز یک نان نداشتند.
\par 15 آنگاه ایشان را قدغن فرمود که «باخبر باشید و از خمیر مایه فریسیان و خمیرمایه هیرودیس احتیاط کنید!»
\par 16 ایشان با خوداندیشیده، گفتند: «از آن است که نان نداریم.»
\par 17 عیسی فهم کرده، بدیشان گفت: «چرا فکرمی کنید از آنجهت که نان ندارید؟ آیا هنوزنفهمیده و درک نکرده‌اید و تا حال دل شما سخت است؟
\par 18 آیا چشم داشته نمی بینید و گوش داشته نمی شنوید و به یاد ندارید؟
\par 19 وقتی که پنج نان را برای پنج هزار نفر پاره کردم، چند سبدپر از پاره‌ها برداشتید؟» بدو گفتند: «دوازده.»
\par 20 «و وقتی که هفت نان را بجهت چهار هزار کس؛ پس زنبیل پر از ریزه‌ها برداشتید؟» گفتندش: «هفت.»
\par 21 پس بدیشان گفت: «چرا نمی فهمید؟»
\par 22 چون به بیت صیدا آمد، شخصی کور را نزد او آوردند و التماس نمودند که او را لمس نماید.
\par 23 پس دست آن کور را گرفته، او را از قریه بیرون برد و آب دهان بر چشمان او افکنده، و دست بر اوگذارده از او پرسید که «چیزی می‌بینی؟»
\par 24 اوبالا نگریسته، گفت: «مردمان را خرامان، چون درختها می‌بینم.»
\par 25 پس بار دیگر دستهای خودرا بر چشمان او گذارده، او را فرمود تا بالانگریست و صحیح گشته، همه‌چیز را به خوبی دید.
\par 26 پس او را به خانه‌اش فرستاده، گفت: «داخل ده مشو و هیچ‌کس را در آن جا خبر مده.»
\par 27 و عیسی با شاگردان خود به دهات قیصریه فیلپس رفت. و در راه از شاگردانش پرسیده، گفت که «مردم مرا که می‌دانند؟»
\par 28 ایشان جواب دادند که «یحیی تعمید‌دهنده و بعضی الیاس وبعضی یکی از انبیا.»
\par 29 او از ایشان پرسید: «شمامرا که می‌دانید؟» پطرس در جواب او گفت: «تومسیح هستی.»
\par 30 پس ایشان را فرمود که «هیچ‌کس را از او خبر ندهند.»
\par 31 آنگاه ایشان را تعلیم دادن آغاز کرد که «لازم است پسر انسان بسیار زحمت کشد و ازمشایخ و روسای کهنه و کاتبان رد شود و کشته شده، بعد از سه روز برخیزد.»
\par 32 و چون این کلام را علانیه فرمود، پطرس او را گرفته، به منع کردن شروع نمود.
\par 33 اما او برگشته، به شاگردان خودنگریسته، پطرس را نهیب داد و گفت: «ای شیطان از من دور شو، زیرا امور الهی را اندیشه نمی کنی بلکه چیزهای انسانی را.»
\par 34 پس مردم را با شاگردان خود خوانده، گفت: «هر‌که خواهد از عقب من آید، خویشتن را انکارکند و صلیب خود را برداشته، مرا متابعت نماید.
\par 35 زیرا هر‌که خواهد جان خود را نجات دهد، آن را هلاک سازد؛ و هر‌که جان خود را بجهت من وانجیل بر باد دهد آن را برهاند.
\par 36 زیراکه شخص را چه سود دارد هر گاه تمام دنیا را ببرد و نفس خود را ببازد؟
\par 37 یا آنکه آدمی چه چیز را به عوض جان خود بدهد؟زیرا هر‌که در این فرقه زناکار و خطاکار از من و سخنان من شرمنده شود، پسر انسان نیز وقتی که با فرشتگان مقدس درجلال پدر خویش آید، از او شرمنده خواهدگردید.»
\par 38 زیرا هر‌که در این فرقه زناکار و خطاکار از من و سخنان من شرمنده شود، پسر انسان نیز وقتی که با فرشتگان مقدس درجلال پدر خویش آید، از او شرمنده خواهدگردید.»

\chapter{9}

\par 1 و بدیشان گفت: «هرآینه به شما می‌گویم بعضی از ایستادگان در اینجا می‌باشند که تاملکوت خدا را که به قوت می‌آید نبینند، ذائقه موت را نخواهند چشید.»
\par 2 و بعد از شش روز، عیسی پطرس و یعقوب و یوحنا را برداشته، ایشان را تنها بر فراز کوهی به خلوت برد و هیاتش در نظر ایشان متغیر گشت.
\par 3 و لباس او درخشان و چون برف بغایت سفید گردید، چنانکه هیچ گازری بر روی زمین نمی تواند چنان سفید نماید.
\par 4 و الیاس با موسی بر ایشان ظاهر شده، با عیسی گفتگو می‌کردند.
\par 5 پس پطرس ملتفت شده، به عیسی گفت: «ای استاد، بودن ما در اینجا نیکو است! پس سه سایبان می‌سازیم، یکی برای تو و دیگری برای موسی و سومی برای الیاس!»
\par 6 از آنرو که نمی دانست چه بگوید، چونکه هراسان بودند.
\par 7 ناگاه ابری بر ایشان سایه انداخت و آوازی از ابردر‌رسید که «این است پسر حبیب من، از اوبشنوید.»
\par 8 در ساعت گرداگرد خود نگریسته، جزعیسی تنها با خود هیچ‌کس را ندیدند.
\par 9 و چون از کوه به زیر می‌آمدند، ایشان راقدغن فرمود که تا پسر انسان از مردگان برنخیزد، از آنچه دیده‌اند کسی را خبر ندهند.
\par 10 و این سخن را در خاطر خود نگاه داشته، از یکدیگرسوال می‌کردند که برخاستن از مردگان چه باشد.
\par 11 پس از او استفسار کرده، گفتند: «چرا کاتبان می‌گویند که الیاس باید اول بیاید؟»
\par 12 او درجواب ایشان گفت که «الیاس البته اول می‌آید وهمه‌چیز را اصلاح می‌نماید و چگونه درباره پسرانسان مکتوب است که می‌باید زحمت بسیارکشد و حقیر شمرده شود.
\par 13 لیکن به شمامی گویم که الیاس هم آمد و با وی آنچه خواستندکردند، چنانچه در حق وی نوشته شده است.»
\par 14 پس چون نزد شاگردان خود رسید، جمعی کثیر گرد ایشان دید و بعضی از کاتبان را که باایشان مباحثه می‌کردند.
\par 15 در ساعت، تمامی خلق چون او را بدیدند در حیرت افتادند و دوان دوان آمده، او را سلام دادند.
\par 16 آنگاه از کاتبان پرسید که «با اینها چه مباحثه دارید؟»
\par 17 یکی ازآن میان در جواب گفت: «ای استاد، پسر خود رانزد تو آوردم که روحی گنگ دارد،
\par 18 و هر جا که او را بگیرد می‌اندازدش، چنانچه کف برآورده، دندانهایم بهم می‌ساید و خشک می‌گردد. پس شاگردان تو را گفتم که او را بیرون کنند، نتوانستند.»
\par 19 او ایشان را جواب داده، گفت: «ای فرقه بی‌ایمان تا کی با شما باشم و تا چه حدمتحمل شما شوم! او را نزد من آورید.»
\par 20 پس اورا نزد وی آوردند. چون او را دید، فور آن روح او را مصروع کرد تا بر زمین افتاده، کف برآورد وغلطان شد.
\par 21 پس از پدر وی پرسید: «چند وقت است که او را این حالت است؟» گفت: «ازطفولیت.
\par 22 و بارها او را در آتش و در آب انداخت تا او را هلاک کند. حال اگر می‌توانی بر ماترحم کرده، ما را مدد فرما.»
\par 23 عیسی وی راگفت: «اگر می‌توانی‌ایمان آری، مومن را همه‌چیزممکن است.»
\par 24 در ساعت پدر طفل فریادبرآورده، گریه‌کنان گفت: «ایمان می‌آورم‌ای خداوند، بی‌ایمانی مرا امداد فرما.»
\par 25 چون عیسی دید که گروهی گرد او به شتاب می‌آیند، روح پلید را نهیب داده، به وی فرمود: «ای روح گنگ و کر من تو را حکم می‌کنم از او در آی ودیگر داخل او مشو!»
\par 26 پس صیحه زده و او رابشدت مصروع نموده، بیرون آمد و مانند مرده گشت، چنانکه بسیاری گفتند که فوت شد.
\par 27 اما عیسی دستش را گرفته، برخیزانیدش که برپاایستاد.
\par 28 و چون به خانه در‌آمد، شاگردانش درخلوت از او پرسیدند: «چرا ما نتوانستیم او رابیرون کنیم؟»
\par 29 ایشان را گفت: «این جنس به هیچ وجه بیرون نمی رود جز به دعا.»
\par 30 و از آنجا روانه شده، در جلیل می‌گشتند ونمی خواست کسی او را بشناسد،
\par 31 زیرا که شاگردان خود را اعلام فرموده، می‌گفت: «پسرانسان به‌دست مردم تسلیم می‌شود و او راخواهند کشت و بعد از مقتول شدن، روز سوم خواهد برخاست.»
\par 32 اما این سخن را درک نکردند و ترسیدند که از او بپرسند.
\par 33 و وارد کفرناحوم شده، چون به خانه درآمد، از ایشان پرسید که «در بین راه با یک دیگرچه مباحثه می‌کردید؟»
\par 34 اما ایشان خاموش ماندند، از آنجا که در راه با یکدیگر گفتگومی کردند در اینکه کیست بزرگتر.
\par 35 پس نشسته، آن دوازده را طلبیده، بدیشان گفت: «هر‌که می‌خواهد مقدم باشد موخر و غلام همه بود.
\par 36 پس طفلی را برداشته، در میان ایشان بر پا نمودو او را در آغوش کشیده، به ایشان گفت:
\par 37 «هرکه یکی از این کودکان را به اسم من قبول کند، مراقبول کرده است و هر‌که مرا پذیرفت نه مرا بلکه فرستنده مرا پذیرفته باشد.
\par 38 آنگاه یوحنا ملتفت شده، بدو گفت: «ای استاد، شخصی را دیدیم که به نام تو دیوها بیرون می‌کرد و متابعت ما نمی نمود؛ و چون متابعت مانمی کرد، او را ممانعت نمودیم.»
\par 39 عیسی گفت: «او را منع مکنید، زیرا هیچ‌کس نیست که معجزه‌ای به نام من بنماید و بتواند به زودی درحق من بد گوید.
\par 40 زیرا هر‌که ضد ما نیست باماست.
\par 41 و هر‌که شما را از این‌رو که از آن مسیح هستید، کاسه‌ای آب به اسم من بنوشاند، هرآینه به شما می‌گویم اجر خود را ضایع نخواهد کرد.
\par 42 و هر‌که یکی از این کودکان را که به من ایمان آورند، لغزش دهد، او را بهتر است که سنگ آسیایی بر گردنش آویخته، در دریا افکنده شود.
\par 43 پس هرگاه دستت تو را بلغزاند، آن را ببرزیرا تو را بهتر است که شل داخل حیات شوی، ازاینکه با دو دست وارد جهنم گردی، در آتشی که خاموشی نپذیرد؛
\par 44 جایی که کرم ایشان نمیرد وآتش، خاموشی نپذیرد.
\par 45 و هرگاه پایت تو رابلغزاند، قطعش کن زیرا تو را مفیدتر است که لنگ داخل حیات شوی از آنکه با دو پا به جهنم افکنده شوی، در آتشی که خاموشی نپذیرد؛
\par 46 آنجایی که کرم ایشان نمیرد و آتش، خاموش نشود.
\par 47 و هر گاه چشم تو تو را لغزش دهد، قلعش کن زیرا تو را بهتر است که با یک چشم داخل ملکوت خدا شوی، از آنکه با دو چشم درآتش جهنم انداخته شوی،
\par 48 جایی که کرم ایشان نمیرد و آتش خاموشی نیابد.
\par 49 زیرا هرکس به آتش، نمکین خواهد شد و هر قربانی به نمک، نمکین می‌گردد.نمک نیکو است، لیکن هر گاه نمک فاسد گردد به چه چیز آن را اصلاح می‌کنید؟ پس در خود نمک بدارید و با یکدیگرصلح نمایید.»
\par 50 نمک نیکو است، لیکن هر گاه نمک فاسد گردد به چه چیز آن را اصلاح می‌کنید؟ پس در خود نمک بدارید و با یکدیگرصلح نمایید.»

\chapter{10}

\par 1 و از آنجا برخاسته، از آن طرف اردن به نواحی یهودیه آمد. و گروهی باز نزدوی جمع شدند و او برحسب عادت خود، بازبدیشان تعلیم می‌داد.
\par 2 آنگاه فریسیان پیش آمده، از روی امتحان ازاو سوال نمودند که «آیا مرد را طلاق دادن زن خویش جایز است.»
\par 3 در جواب ایشان گفت: «موسی شما را چه فرموده است؟»
\par 4 گفتند: «موسی اجازت داد که طلاق نامه بنویسند و رهاکنند.»
\par 5 عیسی در جواب ایشان گفت: «به‌سبب سنگدلی شما این حکم را برای شما نوشت.
\par 6 لیکن از ابتدای خلقت، خدا ایشان را مرد و زن آفرید.
\par 7 از آن جهت باید مرد پدر و مادر خود را ترک کرده، با زن خویش بپیوندد،
\par 8 و این دو یک تن خواهند بود چنانکه از آن پس دو نیستند بلکه یک جسد.
\par 9 پس آنچه خدا پیوست، انسان آن راجدا نکند.»
\par 10 و در خانه باز شاگردانش از این مقدمه ازوی سوال نمودند.
\par 11 بدیشان گفت: «هر‌که زن خود را طلاق دهد و دیگری را نکاح کند، بر‌حق وی زنا کرده باشد.
\par 12 و اگر زن از شوهر خود جداشود و منکوحه دیگری گردد، مرتکب زنا شود.»
\par 13 و بچه های کوچک را نزد او آوردند تاایشان را لمس نماید؛ اما شاگردان آورندگان رامنع کردند.
\par 14 چون عیسی این را بدید، خشم نموده، بدیشان گفت: «بگذارید که بچه های کوچک نزد من آیند و ایشان را مانع نشوید، زیراملکوت خدا از امثال اینها است.
\par 15 هرآینه به شما می‌گویم هر‌که ملکوت خدا را مثل بچه کوچک قبول نکند، داخل آن نشود.»
\par 16 پس ایشان را در آغوش کشید و دست بر ایشان نهاده، برکت داد.
\par 17 چون به راه می‌رفت، شخصی دوان دوان آمده، پیش او زانو زده، سوال نمود که «ای استادنیکو چه کنم تا وارث حیات جاودانی شوم؟
\par 18 عیسی بدو گفت: «چرا مرا نیکو گفتی و حال آنکه کسی نیکو نیست جز خدا فقط؟
\par 19 احکام را می دانی، زنا مکن، قتل مکن، دزدی مکن، شهادت دروغ مده، دغابازی مکن، پدر و مادر خود راحرمت دار.»
\par 20 او در جواب وی گفت: «ای استاد، این همه را از طفولیت نگاه داشتم.»
\par 21 عیسی به وی نگریسته، او را محبت نمود وگفت: «تو را یک چیز ناقص است: برو و آنچه داری بفروش و به فقرا بده که در آسمان گنجی خواهی یافت و بیا صلیب را برداشته، مرا پیروی کن.»
\par 22 لیکن او از این سخن ترش رو و محزون گشته، روانه گردید زیرا اموال بسیار داشت.
\par 23 آنگاه عیسی گرداگرد خود نگریسته، به شاگردان خود گفت: «چه دشوار است که توانگران داخل ملکوت خدا شوند.»
\par 24 چون شاگردانش از سخنان او در حیرت افتادند، عیسی باز توجه نموده، بدیشان گفت: «ای فرزندان، چه دشوار است دخول آنانی که به مال و اموال توکل دارند در ملکوت خدا!
\par 25 سهل تر است که شتر به سوراخ سوزن درآید از اینکه شخص دولتمند به ملکوت خدا داخل شود!»
\par 26 ایشان بغایت متحیرگشته، با یکدیگر می‌گفتند: «پس که می‌تواندنجات یابد؟»
\par 27 عیسی به ایشان نظر کرده، گفت: «نزد انسان محال است لیکن نزد خدا نیست زیراکه همه‌چیز نزد خدا ممکن است.»
\par 28 پطرس بدوگفتن گرفت که «اینک ما همه‌چیز را ترک کرده، تورا پیروی کرده‌ایم.»
\par 29 عیسی جواب فرمود: «هرآینه به شما می‌گویم کسی نیست که خانه یابرادران یا خواهران یا پدر یا مادر یا زن یا اولاد یااملاک را بجهت من و انجیل ترک کند،
\par 30 جزاینکه الحال در این زمان صد چندان یابد ازخانه‌ها و برادران و خواهران و مادران و فرزندان و املاک با زحمات، و در عالم آینده حیات جاودانی را.
\par 31 اما بسا اولین که آخرین می‌گردندو آخرین اولین.»
\par 32 و چون در راه به سوی اورشلیم می‌رفتند وعیسی در جلو ایشان می‌خرامید، در حیرت افتادند و چون از عقب او می‌رفتند، ترس بر ایشان مستولی شد. آنگاه آن دوازده را باز به کنارکشیده، شروع کرد به اطلاع دادن به ایشان از آنچه بر وی وارد می‌شد،
\par 33 که «اینک به اورشلیم می‌رویم و پسر انسان به‌دست روسای کهنه وکاتبان تسلیم شود و بر وی فتوای قتل دهند و اورا به امتها سپارند،
\par 34 و بر وی سخریه نموده، تازیانه‌اش زنند و آب دهان بر وی افکنده، او راخواهند کشت و روز سوم خواهد برخاست.»
\par 35 آنگاه یعقوب و یوحنا دو پسر زبدی نزدوی آمده، گفتند: «ای استاد، می‌خواهیم آنچه ازتو سوال کنیم برای ما بکنی.»
\par 36 ایشان را گفت: «چه می‌خواهید برای شما بکنم؟»
\par 37 گفتند: «به ما عطا فرما که یکی به طرف راست و دیگری برچپ تو در جلال تو بنشینیم.»
\par 38 عیسی ایشان راگفت: «نمی فهمید آنچه می‌خواهید. آیا می‌توانیدآن پیاله‌ای را که من می‌نوشم، بنوشید و تعمیدی را که من می‌پذیرم، بپذیرید؟»
\par 39 وی را گفتند: «می‌توانیم.» عیسی بدیشان گفت: «پیاله‌ای را که من می‌نوشم خواهید آشامید و تعمیدی را که من می‌پذیرم خواهید پذیرفت.
\par 40 لیکن نشستن به‌دست راست و چپ من از آن من نیست که بدهم جز آنانی را که از بهر ایشان مهیا شده است.»
\par 41 وآن ده نفر چون شنیدند بر یعقوب و یوحنا خشم گرفتند.
\par 42 عیسی ایشان را خوانده، به ایشان گفت: «می‌دانید آنانی که حکام امتها شمرده می‌شوند برایشان ریاست می‌کنند و بزرگانشان بر ایشان مسلطند.
\par 43 لیکن در میان شما چنین نخواهد بود، بلکه هر‌که خواهد در میان شما بزرگ شود، خادم شما باشد.
\par 44 و هر‌که خواهد مقدم بر شما شود، غلام همه باشد.
\par 45 زیرا که پسر انسان نیز نیامده تامخدوم شود بلکه تا خدمت کند و تا جان خود رافدای بسیاری کند.»
\par 46 و وارد اریحا شدند. و وقتی که او باشاگردان خود و جمعی کثیر از اریحا بیرون می‌رفت، بارتیمائوس کور، پسر تیماوس بر کناره راه نشسته، گدایی می‌کرد.
\par 47 چون شنید که عیسی ناصری است، فریاد کردن گرفت و گفت: «ای عیسی ابن داود بر من ترحم کن.»
\par 48 و چندان‌که بسیاری او را نهیب می‌دادند که خاموش شود، زیادتر فریاد برمی آورد که پسر داودا بر من ترحم فرما.
\par 49 پس عیسی ایستاده، فرمود تا او را بخوانند. آنگاه آن کور را خوانده، بدو گفتند: «خاطر جمع دار برخیز که تو را می‌خواند.»
\par 50 درساعت ردای خود را دور انداخته، بر پا جست ونزد عیسی آمد.
\par 51 عیسی به وی التفات نموده، گفت: «چه می‌خواهی از بهر تو نمایم؟» کور بدوگفت: «یا سیدی آنکه بینایی یابم.»عیسی بدوگفت: «برو که ایمانت تو را شفا داده است.» درساعت بینا گشته، از عقب عیسی در راه روانه شد.
\par 52 عیسی بدوگفت: «برو که ایمانت تو را شفا داده است.» درساعت بینا گشته، از عقب عیسی در راه روانه شد.

\chapter{11}

\par 1 و چون نزدیک به اورشلیم به بیت‌فاجی و بیت عنیا بر کوه زیتون رسیدند، دو نفراز شاگردان خود را فرستاده،
\par 2 بدیشان گفت: «بدین قریه‌ای که پیش روی شما است بروید وچون وارد آن شدید، درساعت کره الاغی را بسته خواهید یافت که تا به حال هیچ‌کس بر آن سوارنشده؛ آن را باز کرده، بیاورید.
\par 3 و هرگاه کسی به شما گوید چرا چنین می‌کنید، گویید خداوندبدین احتیاج دارد؛ بی‌تامل آن را به اینجا خواهدفرستاد.»
\par 4 پس رفته کره‌ای بیرون دروازه درشارع عام بسته یافتند و آن را باز می‌کردند،
\par 5 که بعضی از حاضرین بدیشان گفتند: «چه‌کار داریدکه کره را باز می‌کنید؟»
\par 6 آن دو نفر چنانکه عیسی فرموده بود، بدیشان گفتند. پس ایشان را اجازت دادند.
\par 7 آنگاه کره را به نزد عیسی آورده، رخت خود را بر آن افکندند تا بر آن سوار شد.
\par 8 وبسیاری رختهای خود و بعضی شاخه‌ها ازدرختان بریده، بر راه گسترانیدند.
\par 9 و آنانی که پیش و پس می‌رفتند، فریادکنان می‌گفتند: «هوشیعانا، مبارک باد کسی‌که به نام خداوندمی آید.
\par 10 مبارک باد ملکوت پدر ما داود که می‌آید به اسم خداوند. هوشیعانا در اعلی علیین.»
\par 11 و عیسی وارد اورشلیم شده، به هیکل درآمد وبه همه‌چیز ملاحظه نمود. چون وقت شام شد باآن دوازده به بیت عنیا رفت.
\par 12 بامدادان چون از بیت عنیا بیرون می‌آمدند، گرسنه شد.
\par 13 ناگاه درخت انجیری که برگ داشت از دور دیده، آمد تا شاید چیزی بر آن بیابد. اما چون نزد آن رسید، جز برگ بر آن هیچ نیافت زیرا که موسم انجیر نرسیده بود.
\par 14 پس عیسی توجه نموده، بدان فرمود: «از این پس تا به ابد، هیچ‌کس از تو میوه نخواهد خورد.» وشاگردانش شنیدند.
\par 15 پس وارد اورشلیم شدند. و چون عیسی داخل هیکل گشت، به بیرون کردن آنانی که درهیکل خرید و فروش می‌کردند شروع نمود وتخت های صرافان و کرسیهای کبوترفروشان راواژگون ساخت،
\par 16 و نگذاشت که کسی با ظرفی از میان هیکل بگذرد،
\par 17 و تعلیم داده، گفت: «آیامکتوب نیست که خانه من خانه عبادت تمامی امتها نامیده خواهد شد؟ اما شما آن را مغاره دزدان ساخته‌اید.»
\par 18 چون روسای کهنه و کاتبان این را بشنیدند، در صدد آن شدند که او را چطور هلاک سازندزیرا که از وی ترسیدند چون که همه مردم ازتعلیم وی متحیر می‌بودند.
\par 19 چون شام شد، از شهر بیرون رفت.
\par 20 صبحگاهان، در اثنای راه، درخت انجیر رااز ریشه خشک یافتند.
\par 21 پطرس به‌خاطر آورده، وی را گفت: «ای استاد، اینک درخت انجیری که نفرینش کردی خشک شده!»
\par 22 عیسی در جواب ایشان گفت: «به خدا ایمان آورید،
\par 23 زیرا که هرآینه به شما می‌گویم هر‌که بدین کوه گویدمنتقل شده، به دریا افکنده شو و در دل خود شک نداشته باشد بلکه یقین دارد که آنچه گویدمی شود، هرآینه هر‌آنچه گوید بدو عطا شود.
\par 24 بنابراین به شما می‌گویم آنچه در عبادت سوال می‌کنید، یقین بدانید که آن را یافته‌اید و به شماعطا خواهد شد.
\par 25 و وقتی که به دعا بایستید، هرگاه کسی به شما خطا کرده باشد، او را ببخشید تاآنکه پدر شما نیز که در آسمان است، خطایای شما را معاف دارد.
\par 26 اما هرگاه شما نبخشید، پدر شما نیز که درآسمان است تقصیرهای شما را نخواهد بخشید.»
\par 27 و باز به اورشلیم آمدند. و هنگامی که او درهیکل می‌خرامید، روسای کهنه و کاتبان و مشایخ نزد وی آمده،
\par 28 گفتندش: «به چه قدرت این کارها را می‌کنی و کیست که این قدرت را به توداده است تا این اعمال را به‌جا آری؟»
\par 29 عیسی در جواب ایشان گفت: «من از شما نیز سخنی می پرسم، مرا جواب دهید تا من هم به شما گویم به چه قدرت این کارها را می‌کنم.
\par 30 تعمید یحیی ازآسمان بود یا از انسان؟ مرا جواب دهید.»
\par 31 ایشان در دلهای خود تفکر نموده، گفتند: «اگرگوییم از آسمان بود، هرآینه گوید پس چرا بدوایمان نیاوردید.
\par 32 و اگر گوییم از انسان بود، » ازخلق بیم داشتند از آنجا که همه یحیی را نبی‌ای برحق می‌دانستند.پس در جواب عیسی گفتند: «نمی دانیم.» عیسی بدیشان جواب داد: «من هم شما را نمی گویم که به کدام قدرت این کارها را به‌جا می‌آورم.»
\par 33 پس در جواب عیسی گفتند: «نمی دانیم.» عیسی بدیشان جواب داد: «من هم شما را نمی گویم که به کدام قدرت این کارها را به‌جا می‌آورم.»

\chapter{12}

\par 1 پس به مثل‌ها به ایشان آغاز سخن نمودکه «شخصی تاکستانی غرس نموده، حصاری گردش کشید و چرخشتی بساخت وبرجی بنا کرده، آن را به دهقانان سپرد و سفر کرد.
\par 2 و در موسم، نوکری نزد دهقانان فرستاد تا ازمیوه باغ از باغبانان بگیرد.
\par 3 اما ایشان او را گرفته، زدند و تهی‌دست روانه نمودند.
\par 4 باز نوکری دیگرنزد ایشان روانه نمود. او را نیز سنگسار کرده، سراو را شکستند و بی‌حرمت کرده، برگردانیدندش.
\par 5 پس یک نفر دیگر فرستاده، او را نیز کشتند و بسادیگران را که بعضی را زدند و بعضی را به قتل رسانیدند.
\par 6 و بالاخره یک پسر حبیب خود راباقی داشت. او را نزد ایشان فرستاده، گفت: پسرمرا حرمت خواهند داشت.
\par 7 لیکن دهقانان با خودگفتند: این وارث است؛ بیایید او را بکشیم تامیراث از آن ما گردد.
\par 8 پس او را گرفته، مقتول ساختند و او را بیرون از تاکستان افکندند.
\par 9 پس صاحب تاکستان چه خواهد کرد؟ او خواهدآمد و آن باغبان را هلاک ساخته، باغ را به دیگران خواهد سپرد.
\par 10 آیا این نوشته رانخوانده‌اید: سنگی که معمارانش رد کردند، همان سر زاویه گردید؟
\par 11 این از جانب خداوند شد و در نظر ما عجیب است.
\par 12 آنگاه خواستند او را گرفتار سازند، اما از خلق می‌ترسیدند، زیرا می‌دانستند که این مثل رابرای ایشان آورد. پس او را واگذارده، برفتند.
\par 13 و چند نفر از فریسیان و هیرودیان را نزدوی فرستادند تا او را به سخنی به دام آورند.
\par 14 ایشان آمده، بدو گفتند: «ای استاد، ما را یقین است که تو راستگو هستی و از کسی باک نداری، چون که به ظاهر مردم نمی نگری بلکه طریق خدارا به راستی تعلیم می‌نمایی. جزیه دادن به قیصرجایز است یا نه؟ بدهیم یا ندهیم؟
\par 15 اما اوریاکاری ایشان را درک کرده، بدیشان گفت: «چرامرا امتحان می‌کنید؟ دیناری نزد من آرید تا آن راببینم.»
\par 16 چون آن را حاضر کردند، بدیشان گفت: «این صورت و رقم از آن کیست؟» وی راگفتند: «از آن قیصر.»
\par 17 عیسی در جواب ایشان گفت: «آنچه از قیصر است، به قیصر رد کنید وآنچه از خداست، به خدا.» و از او متعجب شدند.
\par 18 و صدوقیان که منکر هستند نزد وی آمده، از او سوال نموده، گفتند:
\par 19 «ای استاد، موسی به ما نوشت که هرگاه برادر کسی بمیرد و زنی بازگذاشته، اولادی نداشته باشد، برادرش زن او رابگیرد تا از بهر برادر خود نسلی پیدا نماید.
\par 20 پس هفت برادر بودند که نخستین، زنی گرفته، بمرد واولادی نگذاشت.
\par 21 پس ثانی او را گرفته، هم بی‌اولاد فوت شد و همچنین سومی.
\par 22 تا آنکه آن هفت او را گرفتند و اولادی نگذاشتند و بعد ازهمه، زن فوت شد.
\par 23 پس در قیامت چون برخیزند، زن کدام‌یک از ایشان خواهد بود ازآنجهت که هر هفت، او را به زنی گرفته بودند؟»
\par 24 عیسی در جواب ایشان گفت: «آیا گمراه نیستید از آنرو که کتب و قوت خدا را نمی دانید؟
\par 25 زیرا هنگامی که از مردگان برخیزند، نه نکاح می‌کنند و نه منکوحه می‌گردند، بلکه مانندفرشتگان، در آسمان می‌باشند.
\par 26 اما در باب مردگان که برمی خیزند، در کتاب موسی در ذکربوته نخوانده‌اید چگونه خدا او را خطاب کرده، گفت که منم خدای ابراهیم و خدای اسحاق وخدای یعقوب.
\par 27 و او خدای مردگان نیست بلکه خدای زندگان است. پس شما بسیار گمراه شده‌اید.»
\par 28 و یکی از کاتبان، چون مباحثه ایشان را شنیده، دید که ایشان را جواب نیکو داد، پیش آمده، از او پرسید که «اول همه احکام کدام است؟»
\par 29 عیسی او را جواب داد که «اول همه احکام این است که بشنو‌ای اسرائیل، خداوندخدای ما خداوند واحد است.
\par 30 و خداوندخدای خود را به تمامی دل و تمامی جان وتمامی خاطر و تمامی قوت خود محبت نما، که اول از احکام این است.
\par 31 و دوم مثل اول است که همسایه خود را چون نفس خود محبت نما. بزرگتر از این دو، حکمی نیست.»
\par 32 کاتب وی راگفت: «آفرین‌ای استاد، نیکو گفتی، زیرا خداواحد است و سوای او دیگری نیست،
\par 33 و او رابه تمامی دل و تمامی فهم و تمامی نفس و تمامی قوت محبت نمودن و همسایه خود را مثل خودمحبت نمودن، از همه قربانی های سوختنی وهدایا افضل است.»
\par 34 چون عیسی بدید که عاقلانه جواب داد، به وی گفت: «از ملکوت خدادور نیستی.» و بعد از آن، هیچ‌کس جرات نکردکه از او سوالی کند.
\par 35 و هنگامی که عیسی در هیکل تعلیم می‌داد، متوجه شده، گفت: «چگونه کاتبان می‌گویند که مسیح پسر داود است؟
\par 36 و حال آنکه خود داود در روح‌القدس می‌گوید که خداوند به خداوند من گفت برطرف راست من بنشین تا دشمنان تو را پای انداز تو سازم؟
\par 37 خودداود او را خداوند می‌خواند؛ پس چگونه او راپسر می‌باشد؟» و عوام الناس کلام او را به خشنودی می‌شنیدند.
\par 38 پس در تعلیم خود گفت: «از کاتبان احتیاطکنید که خرامیدن در لباس دراز و تعظیم های دربازارها
\par 39 و کرسی های اول در کنایس و جایهای صدر در ضیافت‌ها را دوست می‌دارند.
\par 40 اینان که خانه های بیوه‌زنان را می‌بلعند و نماز را به ریاطول می‌دهند، عقوبت شدیدتر خواهند یافت.»
\par 41 و عیسی در مقابل بیت‌المال نشسته، نظاره می‌کرد که مردم به چه وضع پول به بیت‌المال می‌اندازند؛ و بسیاری از دولتمندان، بسیارمی انداختند.
\par 42 آنگاه بیوه‌زنی فقیر آمده، دوفلس که یک ربع باشد انداخت.
\par 43 پس شاگردان خود را پیش خوانده، به ایشان گفت: «هرآینه به شما می‌گویم این بیوه‌زن مسکین از همه آنانی که در خزانه انداختند، بیشتر داد.زیرا که همه ایشان از زیادتی خود دادند، لیکن این زن ازحاجتمندی خود، آنچه داشت انداخت، یعنی تمام معیشت خود را.»
\par 44 زیرا که همه ایشان از زیادتی خود دادند، لیکن این زن ازحاجتمندی خود، آنچه داشت انداخت، یعنی تمام معیشت خود را.»

\chapter{13}

\par 1 و چون او از هیکل بیرون می‌رفت، یکی از شاگردانش بدو گفت: «ای استادملاحظه فرما چه نوع سنگها و چه عمارت ها است!»
\par 2 عیسی در جواب وی گفت: «آیا این عمارت های عظیمه را می‌نگری؟ بدان که سنگی بر سنگی گذارده نخواهد شد، مگر آنکه به زیرافکنده شود!»
\par 3 و چون او بر کوه زیتون، مقابل هیکل نشسته بود، پطرس و یعقوب و یوحنا و اندریاس سر ازوی پرسیدند:
\par 4 «ما را خبر بده که این امور کی واقع می‌شود و علامت نزدیک شدن این امورچیست؟»
\par 5 آنگاه عیسی در جواب ایشان سخن آغازکرد که «زنهار کسی شما را گمراه نکند!
\par 6 زیرا که بسیاری به نام من آمده، خواهند گفت که من هستم و بسیاری را گمراه خواهند نمود.
\par 7 اما چون جنگها و اخبار جنگها را بشنوید، مضطرب مشوید زیرا که وقوع این حوادث ضروری است لیکن انتها هنوز نیست.
\par 8 زیرا که امتی بر امتی ومملکتی بر مملکتی خواهند برخاست و زلزله هادر جایها حادث خواهد شد و قحطی‌ها واغتشاش‌ها پدید می‌آید؛ و اینها ابتدای دردهای زه می‌باشد.
\par 9 «لیکن شما از برای خود احتیاط کنید زیرا که شما را به شوراها خواهند سپرد و در کنایس تازیانه‌ها خواهند زد و شما را پیش حکام وپادشاهان بخاطر من حاضر خواهند کرد تا برایشان شهادتی شود.
\par 10 و لازم است که انجیل اول بر تمامی امتها موعظه شود.
\par 11 و چون شما راگرفته، تسلیم کنند، میندیشید که چه بگویید ومتفکر مباشید بلکه آنچه در آن ساعت به شما عطاشود، آن را گویید زیرا گوینده شما نیستید بلکه روح‌القدس است.
\par 12 آنگاه برادر، برادر را و پدر، فرزند را به هلاکت خواهند سپرد و فرزندان بروالدین خود برخاسته، ایشان را به قتل خواهندرسانید.
\par 13 و تمام خلق بجهت اسم من شما رادشمن خواهند داشت. اما هر‌که تا به آخر صبرکند، همان نجات یابد.
\par 14 «پس چون مکروه ویرانی را که به زبان دانیال نبی گفته شده است، در جایی که نمی بایدبرپا بینید - آنکه می‌خواند بفهمد - آنگاه آنانی که در یهودیه می‌باشند، به کوهستان فرار کنند،
\par 15 و هر‌که بر بام باشد، به زیر نیاید و به خانه داخل نشود تا چیزی از آن ببرد،
\par 16 و آنکه در مزرعه است، برنگردد تا رخت خود را بردارد.
\par 17 اما وای بر آبستنان و شیر دهندگان در آن ایام.
\par 18 و دعاکنید که فرار شما در زمستان نشود،
\par 19 زیرا که درآن ایام، چنان مصیبتی خواهد شد که از ابتدای خلقتی که خدا آفرید تاکنون نشده و نخواهد شد.
\par 20 و اگر خداوند آن روزها را کوتاه نکردی، هیچ بشری نجات نیافتی. لیکن بجهت برگزیدگانی که انتخاب نموده است، آن ایام را کوتاه ساخت.
\par 21 «پس هرگاه کسی به شما گوید اینک مسیح در اینجاست یا اینک در آنجا، باور مکنید.
\par 22 زانرو که مسیحان دروغ و انبیای کذبه ظاهرشده، آیات و معجزات از ایشان صادر خواهدشد، بقسمی که اگر ممکن بودی، برگزیدگان را هم گمراه نمودندی.
\par 23 لیکن شما برحذر باشید!
\par 24 و درآن روزهای بعد از آن مصیبت خورشید تاریک گردد و ماه نور خود را بازگیرد،
\par 25 و ستارگان ازآسمان فرو ریزند و قوای افلاک متزلزل خواهدگشت.
\par 26 آنگاه پسر انسان را بینند که با قوت وجلال عظیم بر ابرها می‌آید.
\par 27 در آن وقت، فرشتگان خود را از جهات اربعه از انتهای زمین تابه اقصای فلک فراهم خواهد آورد.
\par 28 «الحال از درخت انجیر مثلش را فراگیریدکه چون شاخه‌اش نازک شده، برگ می‌آوردمی دانید که تابستان نزدیک است.
\par 29 همچنین شما نیز چون این چیزها را واقع بینید، بدانید که نزدیک بلکه بر در است.
\par 30 هرآینه به شمامی گویم تا جمیع این حوادث واقع نشود، این فرقه نخواهند گذشت.
\par 31 آسمان و زمین زایل می‌شود، لیکن کلمات من هرگز زایل نشود.
\par 32 ولی از آن روز و ساعت غیر از پدرهیچ‌کس اطلاع ندارد، نه فرشتگان در آسمان و نه پسر هم.
\par 33 «پس برحذر و بیدار شده، دعا کنیدزیرا نمی دانید که آن وقت کی می‌شود.
\par 34 مثل کسی‌که عازم سفر شده، خانه خود را واگذارد وخادمان خود را قدرت داده، هر یکی را به شغلی خاص مقرر نماید و دربان را امر فرماید که بیداربماند.
\par 35 پس بیدار باشید زیرا نمی دانید که در چه وقت صاحب‌خانه می‌آید، در شام یا نصف شب یا بانگ خروس یا صبح.
\par 36 مبادا ناگهان آمده شما را خفته یابد.اما آنچه به شما می‌گویم، به همه می‌گویم: بیدار باشید!»
\par 37 اما آنچه به شما می‌گویم، به همه می‌گویم: بیدار باشید!»

\chapter{14}

\par 1 و بعد از دو روز، عید فصح و فطیر بودکه روسای کهنه و کاتبان مترصد بودندکه به چه حیله او را دستگیر کرده، به قتل رسانند.
\par 2 لیکن می‌گفتند: «نه در عید مبادا در قوم اغتشاشی پدید آید.»
\par 3 و هنگامی که او در بیت عنیا در خانه شمعون ابرص به غذا نشسته بود، زنی با شیشه‌ای از عطرگرانبها از سنبل خالص آمده، شیشه را شکسته، برسر وی ریخت.
\par 4 و بعضی در خود خشم نموده، گفتند: «چرا این عطر تلف شد؟
\par 5 زیرا ممکن بوداین عطر زیادتر از سیصد دینار فروخته، به فقراداده شود.» و آن زن را سرزنش نمودند.
\par 6 اماعیسی گفت: «او را واگذارید! از برای چه او رازحمت می‌دهید؟ زیرا که با من کاری نیکو کرده است،
\par 7 زیرا که فقرا را همیشه با خود دارید وهرگاه بخواهید می‌توانید با ایشان احسان کنید، لیکن مرا با خود دائم ندارید.
\par 8 آنچه در قوه اوبود کرد، زیرا که جسد مرا بجهت دفن، پیش تدهین کرد.
\par 9 هرآینه به شما می‌گویم در هر جایی از تمام عالم که به این انجیل موعظه شود، آنچه این زن کرد نیز بجهت یادگاری وی مذکور خواهد شد.»
\par 10 پس یهودای اسخریوطی که یکی از آن دوازده بود، به نزد روسای کهنه رفت تا او رابدیشان تسلیم کند.
\par 11 ایشان سخن او را شنیده، شاد شدند و بدو وعده دادند که نقدی بدو بدهند. و او در صدد فرصت موافق برای گرفتاری وی برآمد.
\par 12 و روز اول از عید فطیر که در آن فصح راذبح می‌کردند، شاگردانش به وی گفتند: «کجامی خواهی برویم تدارک بینیم تا فصح رابخوری؟»
\par 13 پس دو نفر از شاگردان خود رافرستاده، بدیشان گفت: «به شهر بروید و شخصی با سبوی آب به شما خواهد برخورد. از عقب وی بروید،
\par 14 و به هرجایی که درآید صاحب‌خانه راگویید: استاد می‌گوید مهمانخانه کجا است تافصح را با شاگردان خود آنجا صرف کنم؟
\par 15 و اوبالاخانه بزرگ مفروش و آماده به شما نشان می‌دهد. آنجا از بهر ما تدارک بینید.»
\par 16 شاگردانش روانه شدند و به شهر رفته، چنانکه او فرموده بود، یافتند و فصح را آماده ساختند.
\par 17 شامگاهان با آن دوازده آمد.
\par 18 و چون نشسته غذا می‌خوردند، عیسی گفت: «هرآینه به شما می‌گویم که، یکی از شما که با من غذامی خورد، مرا تسلیم خواهد کرد.»
\par 19 ایشان غمگین گشته، یک یک گفتن گرفتند که آیا من آنم و دیگری که آیا من هستم.
\par 20 او در جواب ایشان گفت: «یکی از دوازده که با من دست در قاب فروبرد!
\par 21 به درستی که پسر انسان بطوری که درباره او مکتوب است، رحلت می‌کند. لیکن وای بر آن کسی‌که پسر انسان به واسطه او تسلیم شود. او رابهتر می‌بود که تولد نیافتی.»
\par 22 و چون غذا می‌خوردند، عیسی نان راگرفته، برکت داد و پاره کرده، بدیشان داد و گفت: «بگیرید و بخورید که این جسد من است.»
\par 23 وپیاله‌ای گرفته، شکر نمود و به ایشان داد و همه ازآن آشامیدند
\par 24 و بدیشان گفت: «این است خون من از عهد جدید که در راه بسیاری ریخته می‌شود.
\par 25 هرآینه به شما می‌گویم بعد از این ازعصیر انگور نخورم تا آن روزی که در ملکوت خدا آن را تازه بنوشم.
\par 26 و بعد از خواندن تسبیح، به سوی کوه زیتون بیرون رفتند.
\par 27 عیسی ایشان را گفت: «همانا همه شما امشب در من لغزش خورید، زیرامکتوب است شبان را می‌زنم و گوسفندان پراکنده خواهند شد.
\par 28 اما بعد از برخاستنم، پیش از شمابه جلیل خواهم رفت.
\par 29 پطرس به وی گفت: «هرگاه همه لغزش خورند، من هرگز نخورم.»
\par 30 عیسی وی را گفت: «هرآینه به تو می‌گویم که امروز در همین شب، قبل از آنکه خروس دومرتبه بانگ زند، تو سه مرتبه مرا انکار خواهی نمود.»
\par 31 لیکن او به تاکید زیادتر می‌گفت: «هرگاه مردنم با تو لازم افتد، تو را هرگز انکار نکنم.» ودیگران نیز همچنان گفتند.
\par 32 و چون به موضعی که جتسیمانی نام داشت رسیدند، به شاگردان خود گفت: «در اینجابنشینید تا دعا کنم.»
\par 33 و پطرس و یعقوب ویوحنا را همراه برداشته، مضطرب و دلتنگ گردید
\par 34 و بدیشان گفت: «نفس من از حزن، مشرف بر موت شد. اینجا بمانید و بیدار باشید.»
\par 35 و قدری پیشتر رفته، به روی بر زمین افتاد ودعا کرد تا اگر ممکن باشد آن ساعت از او بگذرد.
\par 36 پس گفت: «یا ابا پدر، همه‌چیز نزد تو ممکن است. این پیاله را از من بگذران، لیکن نه به خواهش من بلکه به اراده تو.»
\par 37 پس چون آمد، ایشان را در خواب دیده، پطرس را گفت: «ای شمعون، در خواب هستی؟ آیا نمی توانستی یک ساعت بیدار باشی؟
\par 38 بیدار باشید و دعا کنید تادر آزمایش نیفتید. روح البته راغب است لیکن جسم ناتوان.»
\par 39 و باز رفته، به همان کلام دعانمود.
\par 40 و نیز برگشته، ایشان را در خواب یافت زیرا که چشمان ایشان سنگین شده بود وندانستند او را چه جواب دهند.
\par 41 و مرتبه سوم آمده، بدیشان گفت: «مابقی را بخوابید واستراحت کنید. کافی است! ساعت رسیده است. اینک پسر انسان به‌دستهای گناهکاران تسلیم می شود.
\par 42 برخیزید برویم که اکنون تسلیم‌کننده من نزدیک شد.»
\par 43 در ساعت وقتی که او هنوز سخن می‌گفت، یهودا که یکی از آن دوازده بود، با گروهی بسیار باشمشیرها و چوبها از جانب روسای کهنه و کاتبان و مشایخ آمدند.
\par 44 و تسلیم‌کننده او بدیشان نشانی داده، گفته بود: «هر‌که را ببوسم، همان است. او را بگیرید و با حفظ تمام ببرید.»
\par 45 و درساعت نزد وی شده، گفت: «یا سیدی، یا سیدی.» و وی را بوسید.
\par 46 ناگاه دستهای خود را بر وی انداخته، گرفتندش.
\par 47 و یکی از حاضرین شمشیر خود را کشیده، بر یکی از غلامان رئیس کهنه زده، گوشش را ببرید.
\par 48 عیسی روی بدیشان کرده، گفت: «گویا بر دزد با شمشیرها وچوبها بجهت گرفتن من بیرون آمدید!
\par 49 هر روزدر نزد شما در هیکل تعلیم می‌دادم و مرا نگرفتید. لیکن لازم است که کتب تمام گردد.»
\par 50 آنگاه همه او را واگذارده بگریختند.
\par 51 و یک جوانی باچادری بر بدن برهنه خود پیچیده، از عقب اوروانه شد. چون جوانان او را گرفتند،
\par 52 چادر راگذارده، برهنه از دست ایشان گریخت.
\par 53 و عیسی را نزد رئیس کهنه بردند و جمیع روسای کاهنان و مشایخ و کاتبان بر او جمع گردیدند.
\par 54 و پطرس از دور در عقب او می‌آمد تا به خانه رئیس کهنه درآمده، با ملازمان بنشست ونزدیک آتش خود را گرم می‌نمود.
\par 55 و روسای کهنه و جمیع اهل شورا در جستجوی شهادت برعیسی بودند تا او را بکشند و هیچ نیافتند،
\par 56 زیراکه هرچند بسیاری بر وی شهادت دروغ می‌دادند، اما شهادت های ایشان موافق نشد.
\par 57 وبعضی برخاسته شهادت دروغ داده، گفتند:
\par 58 «ماشنیدیم که او می‌گفت: من این هیکل ساخته شده به‌دست را خراب می‌کنم و در سه روز، دیگری راناساخته شده به‌دست، بنا می‌کنم.»
\par 59 و در این هم باز شهادت های ایشان موافق نشد.
\par 60 پس رئیس کهنه از آن میان برخاسته، ازعیسی پرسیده، گفت: «هیچ جواب نمی دهی؟ چه چیز است که اینها در حق تو شهادت می‌دهند؟»
\par 61 اما او ساکت مانده، هیچ جواب نداد. باز رئیس کهنه از او سوال نموده، گفت: «آیا تو مسیح پسرخدای متبارک هستی؟»
\par 62 عیسی گفت: «من هستم؛ و پسر انسان را خواهید دید که برطرف راست قوت نشسته، در ابرهای آسمان می‌آید.»
\par 63 آنگاه رئیس کهنه جامه خود را چاک زده، گفت: «دیگر‌چه حاجت به شاهدان داریم؟
\par 64 کفر او را شنیدید! چه مصلحت می‌دانید؟» پس همه بر او حکم کردند که مستوجب قتل است.
\par 65 و بعضی شروع نمودند به آب دهان بروی انداختن و روی او را پوشانیده، او را می‌زدندو می‌گفتند نبوت کن. ملازمان او را می‌زدند.
\par 66 و در وقتی که پطرس در ایوان پایین بود، یکی از کنیزان رئیس کهنه آمد
\par 67 و پطرس راچون دید که خود را گرم می‌کند، بر او نگریسته، گفت: «تو نیز با عیسی ناصری می‌بودی؟
\par 68 اوانکار نموده، گفت: «نمی دانم و نمی فهمم که توچه می‌گویی!» و چون بیرون به دهلیز خانه رفت، ناگاه خروس بانگ زد.
\par 69 و بار دیگر آن کنیزک اورا دیده، به حاضرین گفتن گرفت که «این شخص از آنها است!»
\par 70 او باز انکار کرد. و بعد از زمانی حاضرین بار دیگر به پطرس گفتند: «در حقیقت تو از آنها می‌باشی زیرا که جلیلی نیز هستی ولهجه تو چنان است.»
\par 71 پس به لعن کردن و قسم خوردن شروع نمود که «آن شخص را که می‌گویید نمی شناسم.»ناگاه خروس مرتبه دیگر بانگ زد. پس پطرس را به‌خاطر آمد آنچه عیسی بدو گفته بود که «قبل از آنکه خروس دومرتبه بانگ زند، سه مرتبه مرا انکار خواهی نمود.» و چون این را به‌خاطر آورد، بگریست.
\par 72 ناگاه خروس مرتبه دیگر بانگ زد. پس پطرس را به‌خاطر آمد آنچه عیسی بدو گفته بود که «قبل از آنکه خروس دومرتبه بانگ زند، سه مرتبه مرا انکار خواهی نمود.» و چون این را به‌خاطر آورد، بگریست.

\chapter{15}

\par 1 بامدادان، بی‌درنگ روسای کهنه بامشایخ و کاتبان و تمام اهل شورامشورت نمودند و عیسی را بند نهاده، بردند و به پیلاطس تسلیم کردند.
\par 2 پیلاطس از او پرسید: «آیا تو پادشاه یهودهستی؟» او در جواب وی گفت: «تو می‌گویی.»
\par 3 و چون روسای کهنه ادعای بسیار بر او می نمودند،
\par 4 پیلاطس باز از او سوال کرده، گفت: «هیچ جواب نمی دهی؟ ببین که چقدر بر توشهادت می‌دهند!»
\par 5 اما عیسی باز هیچ جواب نداد، چنانکه پیلاطس متعجب شد.
\par 6 و در هر عید یک زندانی، هر‌که رامی خواستند، بجهت ایشان آزاد می‌کرد.
\par 7 و برابانامی با شرکای فتنه او که در فتنه خونریزی کرده بودند، در حبس بود.
\par 8 آنگاه مردم صدازده، شروع کردند به‌خواستن که برحسب عادت با ایشان عمل نماید.
\par 9 پیلاطس درجواب ایشان گفت: «آیا می‌خواهید پادشاه یهود را برای شما آزاد کنم؟»
\par 10 زیرا یافته بودکه روسای کهنه او را از راه حسد تسلیم کرده بودند.
\par 11 اما روسای کهنه مردم را تحریض کرده بودند که بلکه برابا را برای ایشان رهاکند.
\par 12 پیلاطس باز ایشان را در جواب گفت: «پس چه می‌خواهید بکنم با آن کس که پادشاه یهودش می‌گویید؟»
\par 13 ایشان بار دیگر فریادکردند که «او را مصلوب کن!»
\par 14 پیلاطس بدیشان گفت: «چرا؟ چه بدی کرده است؟» ایشان بیشتر فریاد برآوردند که «او را مصلوب کن.»
\par 15 پس پیلاطس چون خواست که مردم راخشنود گرداند، برابا را برای ایشان آزاد کرد وعیسی را تازیانه زده، تسلیم نمود تا مصلوب شود.
\par 16 آنگاه سپاهیان او را به‌سرایی که دارالولایه است برده، تمام فوج را فراهم آوردند
\par 17 وجامه‌ای قرمز بر او پوشانیدند و تاجی از خاربافته، بر سرش گذاردند
\par 18 و او را سلام کردن گرفتند که «سلام‌ای پادشاه یهود!»
\par 19 و نی بر سراو زدند و آب دهان بر وی انداخته و زانو زده، بدو تعظیم می‌نمودند.
\par 20 و چون او را استهزاکرده بودند، لباس قرمز را از وی کنده جامه خودش را پوشانیدند و او را بیرون بردند تامصلوبش سازند.
\par 21 و راهگذری را شمعون نام، از اهل قیروان که از بلوکات می‌آمد، و پدر اسکندر و رفس بود، مجبور ساختند که صلیب او را بردارد.
\par 22 پس اورا به موضعی که جلجتا نام داشت یعنی محل کاسه سر بردند
\par 23 و شراب مخلوط به مر به وی دادند تا بنوشد لیکن قبول نکرد.
\par 24 و چون او رامصلوب کردند، لباس او را تقسیم نموده، قرعه برآن افکندند تا هر کس چه برد.
\par 25 و ساعت سوم بود که اورا مصلوب کردند.
\par 26 و تقصیر نامه وی این نوشته شد: «پادشاه یهود.»
\par 27 و با وی دو دزد را یکی از دست راست و دیگری از دست چپ مصلوب کردند.
\par 28 پس تمام گشت آن نوشته‌ای که می‌گوید: «ازخطاکاران محسوب گشت.»
\par 29 و راهگذاران او رادشنام داده و سر خود را جنبانیده، می‌گفتند: «هان‌ای کسی‌که هیکل را خراب می‌کنی و در سه روز آن را بنا می‌کنی،
\par 30 از صلیب به زیرآمده، خود را برهان!»
\par 31 و همچنین روسای کهنه و کاتبان استهزاکنان با یکدیگر می‌گفتند؛ «دیگران را نجات داد و نمی تواند خود را نجات دهد.
\par 32 مسیح، پادشاه اسرائیل، الان از صلیب نزول کند تا ببینیم و ایمان آوریم.» و آنانی که با وی مصلوب شدند او را دشنام می‌دادند.
\par 33 و چون ساعت ششم رسید تا ساعت نهم تاریکی تمام زمین را فرو گرفت.
\par 34 و در ساعت نهم، عیسی به آواز بلند ندا کرده، گفت: «ایلوئی ایلوئی، لما سبقتنی؟» یعنی «الهی الهی چرا مراواگذاردی؟»
\par 35 و بعضی از حاضرین چون شنیدند گفتند: «الیاس را می‌خواند.»
\par 36 پس شخصی دویده، اسفنجی را از سرکه پر کرد و برسر نی نهاده، بدو نوشانید و گفت: «بگذارید ببینیم مگر الیاس بیاید تا او را پایین آورد.»
\par 37 پس عیسی آوازی بلند برآورده، جان بداد.
\par 38 آنگاه پرده هیکل از سر تا پا دوپاره شد.
\par 39 و چون یوزباشی که مقابل وی ایستاده بود، دید که بدینطور صدا زده، روح را سپرد، گفت؛ «فی الواقع این مرد، پسر خدا بود.»
\par 40 و زنی چند از دور نظر می‌کردند که ازآنجمله مریم مجدلیه بود و مریم مادر یعقوب کوچک و مادر یوشا و سالومه،
\par 41 که هنگام بودن او در جلیل پیروی و خدمت او می‌کردند. و دیگرزنان بسیاری که به اورشلیم آمده بودند.
\par 42 و چون شام شد، از آن جهت روز تهیه یعنی روز قبل از سبت بود،
\par 43 یوسف نامی ازاهل رامه که مرد شریف از اعضای شورا و نیزمنتظر ملکوت خدا بود آمد و جرات کرده نزدپیلاطس رفت و جسد عیسی را طلب نمود.
\par 44 پیلاطس تعجب کرد که بدین زودی فوت شده باشد، پس یوزباشی را طلبیده، از او پرسید که «آیاچندی گذشته وفات نموده است؟»
\par 45 چون ازیوزباشی دریافت کرد، بدن را به یوسف ارزانی داشت.
\par 46 پس کتانی خریده، آن را از صلیب به زیر آورد و به آن کتان کفن کرده، در قبری که ازسنگ تراشیده بود نهاد و سنگی بر سر قبرغلطانید.و مریم مجدلیه و مریم مادر یوشادیدند که کجا گذاشته شد.
\par 47 و مریم مجدلیه و مریم مادر یوشادیدند که کجا گذاشته شد.

\chapter{16}

\par 1 پس چون سبت گذشته بود، مریم مجدلیه و مریم مادر یعقوب و سالومه حنوط خریده، آمدند تا او را تدهین کنند.
\par 2 وصبح روز یکشنبه را بسیار زود وقت طلوع آفتاب بر سر قبر‌آمدند.
\par 3 و با یکدیگر می‌گفتند: «کیست که سنگ را برای ما از سر قبر بغلطاند؟»
\par 4 چون نگریستند، دیدند که سنگ غلطانیده شده است زیرا بسیار بزرگ بود.
\par 5 و چون به قبر درآمدند، جوانی را که جامه‌ای سفید دربرداشت بر جانب راست نشسته دیدند. پس متحیر شدند.
\par 6 اوبدیشان گفت: «ترسان مباشید! عیسی ناصری مصلوب را می‌طلبید؟ او برخاسته است! در اینجانیست. آن موضعی را که او را نهاده بودند، ملاحظه کنید.
\par 7 لیکن رفته، شاگردان او و پطرس را اطلاع دهید که پیش از شما به جلیل می‌رود. اورا در آنجا خواهید دید، چنانکه به شما فرموده بود.»
\par 8 پس بزودی بیرون شده از قبر گریختندزیرا لرزه و حیرت ایشان را فرو گرفته بود و به کسی هیچ نگفتند زیرا می‌ترسیدند.
\par 9 و صبحگاهان، روز اول هفته چون برخاسته بود، نخستین به مریم مجدلیه که از او هفت دیوبیرون کرده بود ظاهر شد.
\par 10 و او رفته اصحاب اورا که گریه و ماتم می‌کردند خبر داد.
\par 11 و ایشان چون شنیدند که زنده گشته و بدو ظاهر شده بود، باور نکردند.
\par 12 و بعد از آن به صورت دیگر به دو نفر ازایشان در هنگامی که به دهات می‌رفتند، هویداگردید.
\par 13 ایشان رفته، دیگران را خبر دادند، لیکن ایشان را نیز تصدیق ننمودند.
\par 14 و بعد از آن بدان یازده هنگامی که به غذانشسته بودند ظاهر شد و ایشان را به‌سبب بی‌ایمانی و سخت دلی ایشان توبیخ نمود زیرا به آنانی که او را برخاسته دیده بودند، تصدیق ننمودند.
\par 15 پس بدیشان گفت: «در تمام عالم بروید وجمیع خلایق را به انجیل موعظه کنید.
\par 16 هر‌که ایمان آورده، تعمید یابد نجات یابد و اما هر‌که ایمان نیاورد بر او حکم خواهد شد.
\par 17 و این آیات همراه ایمانداران خواهد بود که به نام من دیوها را بیرون کنند و به زبانهای تازه حرف زنند
\par 18 و مارها را بردارند و اگر زهر قاتلی بخورندضرری بدیشان نرساند و هرگاه دستها بر مریضان گذارند شفا خواهند یافت.»و خداوند بعد از آنکه به ایشان سخن گفته بود، به سوی آسمان مرتفع شده، به‌دست راست خدا بنشست.
\par 19 و خداوند بعد از آنکه به ایشان سخن گفته بود، به سوی آسمان مرتفع شده، به‌دست راست خدا بنشست.



\end{document}