\begin{document}

\title{يشوع}

 
\chapter{1}

\par 1 و واقع شد بعد از وفات موسی، بنده خداوند، که خداوند یوشع بن نون، خادم موسی را خطاب کرده، گفت:
\par 2 «موسی بنده من وفات یافته است، پس الان برخیز و از این اردن عبور کن، تو و تمامی این قوم، به زمینی که من به ایشان، یعنی به بنی‌اسرائیل می‌دهم.
\par 3 هر جایی که کف پای شما گذارده شود به شما داده‌ام، چنانکه به موسی گفتم.
\par 4 از صحرا و این لبنان تانهر بزرگ یعنی نهر فرات، تمامی زمین حتیان و تادریای بزرگ به طرف مغرب آفتاب، حدود شماخواهد بود.
\par 5 هیچکس را در تمامی ایام عمرت یارای مقاومت با تو نخواهد بود. چنانکه با موسی بودم با تو خواهم بود، تو را مهمل نخواهم گذاشت و ترک نخواهم نمود.
\par 6 قوی و دلیر باش، زیرا که تو این قوم را متصرف، زمینی که برای پدران ایشان قسم خوردم که به ایشان بدهم، خواهی ساخت.
\par 7 فقط قوی و بسیار دلیر باش تابرحسب تمامی شریعتی که بنده من، موسی تو راامر کرده است متوجه شده، عمل نمایی. زنهار ازآن به طرف راست یا چپ تجاوز منما تا هر جایی که روی، کامیاب شوی.
\par 8 این کتاب تورات ازدهان تو دور نشود، بلکه روز و شب در آن تفکرکن تا برحسب هر‌آنچه در آن مکتوب است متوجه شده، عمل نمایی زیرا همچنین راه خود را فیروز خواهی ساخت، و همچنین کامیاب خواهی شد.
\par 9 آیا تو را امر نکردم؟ پس قوی ودلیر باش، مترس و هراسان مباش زیرا در هر جاکه بروی یهوه خدای تو، با توست.»
\par 10 پس یوشع روسای قوم را امر فرموده، گفت:
\par 11 «در میان لشکرگاه بگذرید و قوم را امرفرموده، بگویید: برای خود توشه حاضر کنید، زیرا که بعد از سه روز، شما از این اردن عبورکرده، داخل خواهید شد تا تصرف کنید در زمینی که یهوه خدای شما، به شما برای ملکیت می‌دهد.»
\par 12 و یوشع روبینیان و جادیان و نصف سبطمنسی را خطاب کرده، گفت:
\par 13 «بیاد آورید آن سخن را که موسی، بنده خداوند، به شما امرفرموده، گفت: یهوه، خدای شما به شما آرامی می‌دهد و این زمین را به شما می‌بخشد.
\par 14 زنان واطفال و مواشی شما در زمینی که موسی در آن طرف اردن به شما داد خواهند ماند، و اما شمامسلح شده، یعنی جمیع مردان جنگی پیش روی برادران خود عبور کنید، و ایشان را اعانت نمایید.
\par 15 تا خداوند برادران شما را مثل شما آرامی داده باشد، و ایشان نیز در زمینی که یهوه، خدای شمابه ایشان می‌دهد تصرف کرده باشند، آنگاه به زمین ملکیت خود خواهید برگشت و متصرف خواهید شد، در آن که موسی، بنده خداوند به آن طرف اردن به سوی مشرق آفتاب به شما داد.»
\par 16 ایشان در جواب یوشع گفتند: «هر‌آنچه به ما فرمودی خواهیم کرد، و هر جا ما را بفرستی، خواهیم رفت.
\par 17 چنانکه موسی را در هر چیزاطاعت نمودیم، تو را نیز اطاعت خواهیم نمود، فقط یهوه، خدای تو، با تو باشد چنانکه با موسی بود.هر کسی‌که از حکم تو رو گرداند و کلام تو را در هر چیزی که او را امر فرمایی اطاعت نکند، کشته خواهد شد، فقط قوی و دلیر باش.»
\par 18 هر کسی‌که از حکم تو رو گرداند و کلام تو را در هر چیزی که او را امر فرمایی اطاعت نکند، کشته خواهد شد، فقط قوی و دلیر باش.»
 
\chapter{2}

\par 1 و یوشع بن نون دو مرد جاسوس از شطیم به پنهانی فرستاده، گفت: «روانه شده، زمین واریحا را ببینید.» پس رفته، به خانه زن زانیه‌ای که راحاب نام داشت داخل شده، در آنجا خوابیدند.
\par 2 و ملک اریحا را خبر دادند که «اینک مردمان ازبنی‌اسرائیل امشب داخل این‌جا شدند تا زمین راجاسوسی کنند.»
\par 3 و ملک اریحا نزد راحاب فرستاده، گفت: «مردانی را که نزد تو آمده، به خانه تو داخل شده‌اند، بیرون بیاور زیرا برای جاسوسی تمامی زمین آمده‌اند.»
\par 4 و زن آن دومرد را گرفته، ایشان را پنهان کرد و گفت: «بلی آن مردان نزد من آمدند اما ندانستم از کجا بودند.
\par 5 ونزدیک به وقت بستن دروازه، آن مردان در تاریکی بیرون رفتند و نمی دانم که ایشان کجا رفتند. به زودی ایشان را تعاقب نمایید که به ایشان خواهیدرسید.»
\par 6 لیکن او ایشان را به پشت بام برده، درشاخه های کتان که برای خود بر پشت بام چیده بود، پنهان کرده بود.
\par 7 پس آن کسان، ایشان را به راه اردن تا گدارها تعاقب نمودند، و چون تعاقب کنندگان ایشان بیرون رفتند، دروازه رابستند.
\par 8 و قبل از آنکه بخوابند، او نزد ایشان به پشت بام برآمد.
\par 9 و به آن مردان گفت: «می‌دانم که یهوه این زمین را به شما داده، و ترس شما بر مامستولی شده است، و تمام ساکنان زمین به‌سبب شما گداخته شده‌اند.
\par 10 زیرا شنیده‌ایم که خداوند چگونه آب دریای قلزم را پیش روی شما خشکانید، وقتی که از مصر بیرون آمدید، وآنچه به دو پادشاه اموریان که به آن طرف اردن بودند کردید، یعنی سیهون و عوج که ایشان راهلاک ساختید.
\par 11 و چون شنیدیم دلهای ماگداخته شد، و به‌سبب شما دیگر در کسی جان نماند، زیرا که یهوه خدای شما، بالا در آسمان وپایین بر زمین خداست.
\par 12 پس الان برای من به خداوند قسم بخورید که چنانکه به شما احسان کردم، شما نیز به خاندان پدرم احسان خواهیدنمود، و نشانه امانت به من بدهید
\par 13 که پدرم ومادرم و برادرانم و خواهرانم و هر‌چه دارند زنده خواهید گذارد، و جانهای ما را از موت رستگارخواهید ساخت.»
\par 14 آن مردان به وی گفتند: «جانهای ما به عوض شما بمیرند که چون خداوند این زمین را به ما بدهد اگر این کار ما رابروز ندهید، البته به شما احسان و امانت خواهیم کرد.»
\par 15 پس ایشان را با طناب از دریچه پایین کرد، زیرا خانه او بر حصار شهر بود و او بر حصار ساکن بود.
\par 16 و ایشان را گفت: «به کوه بروید مباداتعاقب کنندگان به شما برسند و در آنجا سه روزخود را پنهان کنید، تا تعاقب کنندگان برگردند بعد از آن به راه خود بروید.»
\par 17 آن مردان به وی گفتند: «ما از این قسم تو که به ما دادی مبراخواهیم شد.
\par 18 اینک چون ما به زمین داخل شویم، این طناب ریسمان قرمز را به دریچه‌ای که ما را به آن پایین کردی ببند، و پدرت و مادرت وبرادرانت و تمامی خاندان پدرت را نزد خود به خانه جمع کن.
\par 19 و چنین خواهد شد که هر کسی‌که از در خانه تو به کوچه بیرون رود، خونش برسرش خواهد بود و ما مبرا خواهیم بود، و هر‌که نزد تو در خانه باشد، اگر کسی بر او دست بگذارد، خونش بر سر ما خواهد بود.
\par 20 و اگر این کار ما رابروز دهی، از قسم تو که به ما داده‌ای مبرا خواهیم بود.»
\par 21 او گفت: «موافق کلام شما باشد.» پس ایشان را روانه کرده، رفتند، و طناب قرمز را به دریچه بست.
\par 22 پس ایشان روانه شده، به کوه آمدند و درآنجا سه روز ماندند تا تعاقب کنندگان برگشتند، وتعاقب کنندگان تمامی راه را جستجو کردند ولی ایشان را نیافتند.
\par 23 پس آن دو مرد برگشته، از کوه به زیر آمدند و از نهر عبور نموده، نزد یوشع بن نون رسیدند، و هر‌آنچه به ایشان واقع شده بود، برای وی بیان کردند.و به یوشع گفتند: «هرآینه خداوند تمامی زمین را به‌دست ما داده است و جمیع ساکنان زمین به‌سبب ما گداخته شده‌اند.»
\par 24 و به یوشع گفتند: «هرآینه خداوند تمامی زمین را به‌دست ما داده است و جمیع ساکنان زمین به‌سبب ما گداخته شده‌اند.»
 
\chapter{3}

\par 1 بامدادان یوشع بزودی برخاسته، او و تمامی بنی‌اسرائیل از شطیم روانه شده، به اردن آمدند، و قبل از عبور کردن در آنجا منزل گرفتند.
\par 2 و بعد از سه روز روسای ایشان از میان لشکرگاه گذشتند.
\par 3 و قوم را امر کرده، گفتند: «چون تابوت عهد یهوه، خدای خود را ببینید که لاویان کهنه آن را می‌برند، آنگاه شما از جای خود روانه شده، در عقب آن بروید.
\par 4 و در میان شما و آن، به مقدار دوهزار ذراع مسافت باشد، ونزدیک آن میایید تا راهی که باید رفت بدانید، زیراکه از این راه قبل از این عبور نکرده‌اید.»
\par 5 و یوشع به قوم گفت: «خود را تقدیس نمایید زیرا فرداخداوند در میان شما کارهای عجیب خواهدکرد.»
\par 6 و یوشع کاهنان را خطاب کرده، گفت: «تابوت عهد را برداشته، پیش روی قوم بروید.» پس تابوت عهد را برداشته، پیش روی قوم روانه شدند.
\par 7 و خداوند یوشع را گفت: «امروز به بزرگ ساختن تو در نظر تمام اسرائیل شروع می‌کنم تابدانند که چنانکه با موسی بودم با تو نیز خواهم بود.
\par 8 پس تو کاهنان را که تابوت عهد رابرمی دارند امر فرموده، بگو: چون شما به کنار آب اردن برسید در اردن بایستید.»
\par 9 و یوشع بنی‌اسرائیل را گفت: «اینجا نزدیک آمده، سخنان یهوه خدای خود را بشنوید.»
\par 10 و یوشع گفت: «به این خواهید دانست که خدای زنده در میان شماست، و او کنعانیان و حتیان و حویان و فرزیان و جرجاشیان و اموریان و یبوسیان را از پیش روی شما البته بیرون خواهد کرد.
\par 11 اینک تابوت عهد خداوند تمامی زمین، پیش روی شما به اردن عبور می‌کند.
\par 12 پس الان دوازده نفر از اسباطاسرائیل، یعنی از هر سبط یک نفر را انتخاب کنید.
\par 13 و واقع خواهد شد چون کف پایهای کاهنانی که تابوت یهوه، خداوند تمامی زمین رابرمی دارند در آبهای اردن قرار گیرد که آبهای اردن، یعنی آبهایی که از بالا می‌آید شکافته شده مثل توده بر روی هم خواهد ایستاد.»
\par 14 و چون قوم از خیمه های خود روانه شدند تا از اردن عبور کنند و کاهنان تابوت عهد را پیش روی قوم می‌بردند.
\par 15 و بردارندگان تابوت به اردن رسیدند، و پایهای کاهنانی که تابوت رابرداشته بودند، به کنار آب فرو رفت (و اردن، تمام موسم حصاد، بر همه کناره هایش سیلاب می‌شود).
\par 16 واقع شد که آبهایی که از بالامی آمد، بایستاد و به مسافتی بسیار دور تا شهرآدم که به‌جانب صرتان است، بلند شد، و آبی که به سوی دریای عربه، یعنی بحرالملح می‌رفت تمام قطع شد، و قوم در مقابل اریحا عبور کردند.و کاهنانی که تابوت عهد خداوند رابرمی داشتند در میان اردن بر خشکی قایم ایستادند، و جمیع اسرائیل به خشکی عبورکردند تا تمامی قوم از اردن، بالکلیه گذشتند.
\par 17 و کاهنانی که تابوت عهد خداوند رابرمی داشتند در میان اردن بر خشکی قایم ایستادند، و جمیع اسرائیل به خشکی عبورکردند تا تمامی قوم از اردن، بالکلیه گذشتند.
 
\chapter{4}

\par 1 و واقع شد که چون تمامی قوم از اردن بالکلیه گذشتند، خداوند یوشع را خطاب کرده، گفت:
\par 2 «دوازده نفر از قوم، یعنی از هر سبطیک نفر را بگیرید.
\par 3 و ایشان را امر فرموده، بگویید: از اینجا از میان اردن از جایی که پایهای کاهنان قایم ایستاده بود، دوازده سنگ بردارید، وآنها را با خود برده، در منزلی که امشب در آن فرود می‌آیید بنهید.»
\par 4 پس یوشع آن دوازده مردرا که از بنی‌اسرائیل انتخاب کرده بود، یعنی از هرسبط یک نفر طلبید.
\par 5 و یوشع به ایشان گفت: «پیش تابوت یهوه، خدای خود به میان اردن بروید، و هر کسی از شما یک سنگ موافق شماره اسباط بنی‌اسرائیل بر دوش خود بردارد.
\par 6 تا این در میان شما علامتی باشد هنگامی که پسران شمادر زمان آینده پرسیده، گویند که مقصود شما ازاین سنگها چیست؟
\par 7 آنگاه به ایشان بگویید: که آبهای اردن از حضور تابوت عهد خداوندشکافته شد، یعنی هنگامی که آن از اردن می گذشت، آبهای اردن شکافته شد، پس این سنگها به جهت بنی‌اسرائیل برای یادگاری ابدی خواهد بود.»
\par 8 و بنی‌اسرائیل موافق آنچه یوشع امر فرموده بود کردند، و دوازده سنگ از میان اردن به طوری که خداوند به یوشع گفته بود، موافق شماره اسباط بنی‌اسرائیل برداشتند، و آنها را با خود به‌جایی که در آن منزل گرفتند برده، آنها را در آنجانهادند.
\par 9 و یوشع در وسط اردن، در جایی که پایهای کاهنانی که تابوت عهد را برداشته بودند، ایستاده بود، دوازده سنگ نصب کرد و در آنجا تاامروز هست.
\par 10 و کاهنانی که تابوت را برمی داشتند دروسط اردن ایستادند، تا هر‌آنچه خداوند یوشع راامر فرموده بود که به قوم بگوید تمام شد، به حسب آنچه موسی به یوشع امر کرده بود و قوم به تعجیل عبور کردند.
\par 11 و بعد از آنکه تمامی قوم بالکل گذشتند، واقع شد که تابوت خداوند وکاهنان به حضور قوم عبور کردند.
\par 12 و بنی روبین و بنی جاد و نصف سبط منسی مسلح شده، پیش روی بنی‌اسرائیل عبور کردند چنانکه موسی به ایشان گفته بود. 
\par 13 قریب به چهل هزار نفر مهیاشده کارزار به حضور خداوند به صحرای اریحابرای جنگ عبور کردند.
\par 14 و در آن روز خداوند، یوشع را در نظرتمامی اسرائیل بزرگ ساخت، و از او در تمام ایام عمرش می‌ترسیدند، چنانکه از موسی ترسیده بودند.
\par 15 و خداوند یوشع را خطاب کرده، گفت:
\par 16 «کاهنانی را که تابوت شهادت را برمی دارند، بفرما که از اردن برآیند.»
\par 17 پس یوشع کاهنان راامر فرموده، گفت: «از اردن برآیید.»
\par 18 و چون کاهنانی که تابوت عهد خداوند را برمی داشتند ازمیان اردن برآمدند و کف پایهای کاهنان برخشکی گذارده شد، آنگاه آب اردن بجای خودبرگشت و مثل پیش بر تمامی کناره هایش جاری شد.
\par 19 و در روز دهم از ماه اول، قوم از اردن برآمدند و در جلجال به‌جانب شرقی اریحا اردوزدند.
\par 20 و یوشع آن دوازده سنگ را که از اردن گرفته بودند، در جلجال نصب کرد.
\par 21 وبنی‌اسرائیل را خطاب کرده، گفت: «چون پسران شما در زمان آینده از پدران خود پرسیده، گویند: که این سنگها چیست؟
\par 22 آنگاه پسران خود راتعلیم داده، گویید که اسرائیل از این اردن به خشکی عبور کردند.
\par 23 زیرا یهوه، خدای شما، آب اردن را از پیش روی شما خشکانید تا شماعبور کردید، چنانکه یهوه خدای شما به بحرقلزم کرد که آن را پیش روی ما خشکانید تا ماعبور کردیم.تا تمامی قومهای زمین دست خداوند را بدانند که آن زورآور است، و از یهوه، خدای شما، همه اوقات بترسند.»
\par 24 تا تمامی قومهای زمین دست خداوند را بدانند که آن زورآور است، و از یهوه، خدای شما، همه اوقات بترسند.»
 
\chapter{5}

\par 1 و واقع شد که چون تمامی ملوک اموریانی که به آن طرف اردن به سمت مغرب بودند، و تمامی ملوک کنعانیانی که به کناره دریا بودند، شنیدند که خداوند آب اردن را پیش روی بنی‌اسرائیل خشکانیده بود تا ما عبور کردیم، دلهای ایشان گداخته شد و از ترس بنی‌اسرائیل، دیگر جان در ایشان نماند.
\par 2 در آن وقت، خداوند به یوشع گفت: «کاردها از سنگ چخماق برای خود بساز، و بنی‌اسرائیل را بار دیگر مختون ساز.»
\par 3 و یوشع کاردها ازسنگ چخماق ساخته، بنی‌اسرائیل را بر تل غلفه ختنه کرد.
\par 4 و سبب ختنه کردن یوشع این بود که تمام ذکوران قوم، یعنی تمام مردان جنگی که ازمصر بیرون آمدند به‌سر راه در صحرا مردند.
\par 5 اماتمامی قوم که بیرون آمدند مختون بودند، وتمامی قوم که در صحرا بعد از بیرون آمدن ایشان از مصر به‌سر راه مولود شدند، مختون نگشتند.
\par 6 زیرا بنی‌اسرائیل چهل سال در بیابان راه می‌رفتند، تا تمامی آن طایفه، یعنی آن مردان جنگی که از مصر بیرون آمده بودند، تمام شدند. زانرو که آواز خداوند را نشنیدند و خداوند به ایشان قسم خورده، گفت: «شما را نمی گذارم که آن زمین را ببینید که خداوند برای پدران ایشان قسم خورده بود که آن را به ما بدهد، زمینی که به شیر و شهد جاری است.»
\par 7 و اما پسران ایشان که در جای آنها برخیزانیده بود یوشع ایشان رامختون ساخت، زیرا نامختون بودند چونکه ایشان را در راه ختنه نکرده بودند.
\par 8 و واقع شد که چون از ختنه کردن تمام قوم فارغ شدند، در جایهای خود در لشکرگاه ماندندتا شفا یافتند.
\par 9 و خداوند به یوشع گفت: «امروزعار مصر را از روی شما غلطانیدم. از این سبب نام آن مکان تا امروز جلجال خوانده می‌شود.»
\par 10 و بنی‌اسرائیل در جلجال اردو زدند و عیدفصح را در شب روز چهاردهم ماه، در صحرای اریحا نگاه داشتند.
\par 11 و در فردای بعد از فصح درهمان روز، از حاصل کهنه زمین، نازکهای فطیر وخوشه های برشته شده خوردند.
\par 12 و در فردای آن روزی که از حاصل زمین خوردند، من موقوف شد و بنی‌اسرائیل دیگر من نداشتند، و در آن سال از محصول زمین کنعان می‌خوردند.
\par 13 و واقع شد چون یوشع نزد اریحا بود که چشمان خود را بالا انداخته، دید که اینک مردی با شمشیر برهنه در دست خود پیش وی ایستاده بود. و یوشع نزد وی آمده، او را گفت: «آیا تو از ماهستی یا از دشمنان ما؟»
\par 14 گفت: «نی، بلکه من سردار لشکر خداوند هستم که الان آمدم.» پس یوشع روی به زمین افتاده، سجده کرد و به وی گفت: «آقایم به بنده خود چه می‌گوید؟»سردار لشکر خداوند به یوشع گفت که «نعلین خود را از پایت بیرون کن زیرا جایی که توایستاده‌ای مقدس است.» و یوشع چنین کرد.
\par 15 سردار لشکر خداوند به یوشع گفت که «نعلین خود را از پایت بیرون کن زیرا جایی که توایستاده‌ای مقدس است.» و یوشع چنین کرد.
 
\chapter{6}

\par 1 (و اریحا به‌سبب بنی‌اسرائیل سخت بسته شد، به طوری که کسی به آن رفت و آمدنمی کرد. )
\par 2 و یهوه به یوشع گفت: «ببین اریحا وملکش و مردان جنگی را به‌دست تو تسلیم کردم.
\par 3 پس شما یعنی همه مردان جنگی شهر را طواف کنید، و یک مرتبه دور شهر بگردید، و شش روزچنین کن.
\par 4 و هفت کاهن پیش تابوت، هفت کرنای یوبیل بردارند، و در روز هفتم شهر را هفت مرتبه طواف کنید، و کاهنان کرناها را بنوازند.
\par 5 و چون بوق یوبیل کشیده شود و شما آواز کرنا رابشنوید، تمامی قوم به آواز بلند صدا کنند، وحصار شهر به زمین خواهد افتاد، و هر کس از قوم پیش روی خود برآید.»
\par 6 پس یوشع بن نون کاهنان را خوانده، به ایشان گفت: «تابوت عهد رابردارید و هفت کاهن هفت کرنای یوبیل را پیش تابوت خداوند بردارند.»
\par 7 و به قوم گفتند: «پیش بروید و شهر را طواف کنید، و مردان مسلح پیش تابوت خداوند بروند.»
\par 8 و چون یوشع این را به قوم گفت، هفت کاهن هفت کرنای یوبیل را برداشته، پیش خداوندرفتند و کرناها را نواختند و تابوت عهد خداونداز عقب ایشان روانه شد.
\par 9 و مردان مسلح پیش کاهنانی که کرناها را می‌نواختند رفتند، و ساقه لشکر از عقب تابوت روانه شدند و چون می‌رفتند، کاهنان کرناها را می‌نواختند.
\par 10 ویوشع قوم را امر فرموده، گفت: «صدا نزنید و آوازشما شنیده نشود، بلکه سخنی از دهان شما بیرون نیاید تا روزی که به شما بگویم که صدا کنید. آن وقت صدا زنید.»
\par 11 پس تابوت خداوند را به شهرطواف داد و یک مرتبه دور شهر گردش کرد. وایشان به لشکرگاه برگشتند و شب را در لشکرگاه به‌سر بردند.
\par 12 بامدادان یوشع به زودی برخاست و کاهنان تابوت خداوند را برداشتند.
\par 13 و هفت کاهن هفت کرنای یوبیل را برداشته، پیش تابوت خداوندمی رفتند، و کرناها را می‌نواختند، و مردان مسلح پیش ایشان می‌رفتند، و ساقه لشکر از عقب تابوت خداوند رفتند، و چون می‌رفتند (کاهنان )کرناها را می‌نواختند.
\par 14 پس روز دوم، شهر رایک مرتبه طواف کرده، به لشکرگاه برگشتند، وشش روز چنین کردند.
\par 15 و در روز هفتم، وقت طلوع فجر، به زودی برخاسته، شهر را به همین طور هفت مرتبه طواف کردند، جز اینکه در آن روز شهر را هفت مرتبه طواف کردند.
\par 16 و چنین شد در مرتبه هفتم، چون کاهنان کرناها را نواختند که یوشع به قوم گفت: «صدا زنید زیرا خداوند شهر را به شما داده است.
\par 17 و خود شهر و هر‌چه در آن است برای خداوند حرام خواهد شد، و راحاب فاحشه فقط، با هر‌چه با وی در خانه باشد زنده خواهد ماند، زیرارسولانی را که فرستادیم پنهان کرد.
\par 18 و اما شمازنهار خویشتن را از چیز حرام نگاه دارید، مبادابعد از آنکه آن را حرام کرده باشید، از آن چیزحرام بگیرید و لشکرگاه اسرائیل را حرام کرده، آن را مضطرب سازید.
\par 19 و تمامی نقره و طلا وظروف مسین و آهنین، وقف خداوند می‌باشد و به خزانه خداوند گذارده شود.»
\par 20 آنگاه قوم صدازدند و کرناها را نواختند. و چون قوم آواز کرنا راشنیدند و قوم به آواز بلند صدا زدند، حصار شهربه زمین افتاد. و قوم یعنی هر کس پیش روی خودبه شهر برآمد و شهر را گرفتند.
\par 21 و هر‌آنچه درشهر بود از مرد و زن و جوان و پیر و حتی گاو وگوسفند و الاغ را به دم شمشیر هلاک کردند.
\par 22 و یوشع به آن دو مرد که به‌جاسوسی زمین رفته بودند، گفت: «به خانه زن فاحشه بروید، و زن را با هر‌چه دارد از آنجا بیرون آرید چنانکه برای وی قسم خوردید.»
\par 23 پس آن دو جوان جاسوس داخل شده، راحاب و پدرش و مادرش و برادرانش را با هر‌چه داشت بیرون آوردند، بلکه تمام خویشانش را آورده، ایشان را بیرون لشکرگاه اسرائیل جا دادند.
\par 24 و شهر را با آنچه در آن بود، به آتش سوزانیدند. لیکن نقره و طلا وظروف مسین و آهنین را به خزانه خانه خداوندگذاردند.
\par 25 و یوشع، راحاب فاحشه و خاندان پدرش را با هر‌چه از آن او بود زنده نگاه داشت، واو تا امروز در میان اسرائیل ساکن است، زیرارسولان را که یوشع برای جاسوسی اریحافرستاده بود پنهان کرد.
\par 26 و در آنوقت یوشع ایشان را قسم داده، گفت: «ملعون باد به حضور خداوند کسی‌که برخاسته، این شهر اریحا را بنا کند، به نخست زاده خود بنیادش خواهد نهاد، و به پسر کوچک خوددروازه هایش را برپا خواهد نمود.»و خداوند با یوشع می‌بود و اسم اودرتمامی آن زمین شهرت یافت.
\par 27 و خداوند با یوشع می‌بود و اسم اودرتمامی آن زمین شهرت یافت.
 
\chapter{7}

\par 1 و بنی‌اسرائیل در آنچه حرام شده بود خیانت ورزیدند، زیرا عخان ابن کرمی ابن زبدی ابن زارح از سبط یهودا، از آنچه حرام شده بود گرفت، و غضب خداوند بر بنی‌اسرائیل افروخته شد.
\par 2 و یوشع از اریحا تا عای که نزد بیت آون به طرف شرقی بیت ئیل واقع است، مردان فرستاد وایشان را خطاب کرده، گفت: «بروید و زمین راجاسوسی کنید.» پس آن مردان رفته، عای راجاسوسی کردند.
\par 3 و نزد یوشع برگشته، او راگفتند: «تمامی قوم برنیایند؛ به قدر دو یا سه هزارنفر برآیند و عای را بزنند و تمامی قوم را به آنجازحمت ندهی زیرا که ایشان کم‌اند.»
\par 4 پس قریب به سه هزار نفر از قوم به آنجا رفتند و از حضورمردان عای فرار کردند.
\par 5 و مردان عای از آنها به قدر سی و شش نفر کشتند و از پیش دروازه تاشباریم ایشان را تعاقب نموده، ایشان را در نشیب زدند، و دل قوم گداخته شده، مثل آب گردید.
\par 6 و یوشع و مشایخ اسرائیل جامه خود راچاک زده، پیش تابوت خداوند تا شام رو به زمین افتادند، و خاک به‌سرهای خود پاشیدند.
\par 7 و یوشع گفت: «آه‌ای خداوند یهوه برای چه این قوم را از اردن عبور دادی تا ما را به‌دست اموریان تسلیم کرده، ما را هلاک کنی، کاش راضی شده بودیم که به آن طرف اردن بمانیم.
\par 8 آه‌ای خداوندچه بگویم بعد از آنکه اسرائیل از حضور دشمنان خود پشت داده‌اند.
\par 9 زیرا چون کنعانیان و تمامی ساکنان زمین این را بشنوند دور ما را خواهندگرفت و نام ما را از این زمین منقطع خواهند کرد، و تو به اسم بزرگ خود چه خواهی کرد؟»
\par 10 خداوند به یوشع گفت: «برخیز چرا تو به این طور به روی خود افتاده‌ای.
\par 11 اسرائیل گناه کرده، و از عهدی نیز که به ایشان امر فرمودم تجاوز نموده‌اند و از چیز حرام هم گرفته، دزدیده‌اند، بلکه انکار کرده، آن را در اسباب خودگذاشته‌اند.
\par 12 از این سبب بنی‌اسرائیل نمی توانندبه حضور دشمنان خود بایستند و از حضوردشمنان خود پشت داده‌اند، زیرا که ملعون شده‌اند، و اگر چیز حرام را از میان خود تباه نسازید، من دیگر با شما نخواهم بود.
\par 13 برخیزقوم را تقدیس نما و بگو برای فردا خویشتن راتقدیس نمایید، زیرا یهوه خدای اسرائیل چنین می‌گوید: ای اسرائیل چیزی حرام در میان توست و تا این چیز حرام را از میان خود دور نکنی، پیش روی دشمنان خود نمی توانی‌ایستاد.
\par 14 پس بامدادان، شما موافق اسباط خود نزدیک بیایید، وچنین شود که سبطی را که خداوند انتخاب کند به قبیله های خود نزدیک آیند، و قبیله‌ای را که خداوند انتخاب کند به خاندانهای خود نزدیک بیایند، و خاندانی را که خداوند انتخاب کند به مردان خود نزدیک آیند.
\par 15 و هر‌که آن چیز حرام نزد او یافت شود با هر‌چه دارد به آتش سوخته شود، زیرا که از عهد خداوند تجاوز نموده، قباحتی در میان اسرائیل به عمل آورده است.»
\par 16 پس یوشع بامدادان بزودی برخاسته، اسرائیل را به اسباط ایشان نزدیک آورد و سبطیهودا گرفته شد.
\par 17 و قبیله یهودا را نزدیک آوردو قبیله زارحیان گرفته شد. پس قبیله زارحیان رابه مردان ایشان نزدیک آورد و زبدی گرفته شد. 
\par 18 و خاندان او را به مردان ایشان نزدیک آورد وعخان بن کرمی ابن زبدی بن زارح از سبط یهوداگرفته شد.
\par 19 و یوشع به عخان گفت: «ای پسر من الان یهوه خدای اسرائیل را جلال بده و نزد اواعتراف نما و مرا خبر بده که چه کردی و از مامخفی مدار.»
\par 20 عخان در جواب یوشع گفت: «فی الواقع به یهوه خدای اسرائیل گناه کرده، وچنین و چنان به عمل آورده‌ام.
\par 21 چون در میان غنیمت ردایی فاخر شنعاری و دویست مثقال نقره و یک شمش طلا که وزنش پنجاه مثقال بوددیدم، آنها را طمع‌ورزیده، گرفتم، و اینک درمیان خیمه من در زمین است و نقره زیر آن می‌باشد.»
\par 22 آنگاه یوشع رسولان فرستاد و به خیمه دویدند، و اینک در خیمه او پنهان بود و نقره زیرآن.
\par 23 و آنها را از میان خیمه گرفته، نزد یوشع وجمیع بنی‌اسرائیل آوردند و آنها را به حضورخداوند نهادند.
\par 24 و یوشع و تمامی بنی‌اسرائیل با وی عخان پسر زارح و نقره و ردا و شمش طلا وپسرانش و دخترانش و گاوانش و حمارانش وگوسفندانش و خیمه‌اش و تمامی مایملکش راگرفته، آنها را به وادی عخور بردند.
\par 25 و یوشع گفت: «برای چه ما را مضطرب ساختی؟ خداوندامروز تو را مضطرب خواهد ساخت.» پس تمامی اسرائیل او را سنگسار کردند و آنها را به آتش سوزانیدند و ایشان را به سنگها سنگسار کردند.و توده بزرگ از سنگها بر او برپا داشتند که تا به امروز هست، و خداوند از شدت غضب خودبرگشت، بنابراین اسم آن مکان تا امروز وادی عخور نامیده شده است.
\par 26 و توده بزرگ از سنگها بر او برپا داشتند که تا به امروز هست، و خداوند از شدت غضب خودبرگشت، بنابراین اسم آن مکان تا امروز وادی عخور نامیده شده است.
 
\chapter{8}

\par 1 و خداوند به یوشع گفت: «مترس و هراسان مباش. تمامی مردان جنگی را با خود بردارو برخاسته، به عای برو. اینک ملک عای و قوم اوو شهرش و زمینش را به‌دست تو دادم.
\par 2 و به عای و ملکش به طوری که به اریحا و ملکش عمل نمودی بکن، لیکن غنیمتش را با بهایمش برای خود به تاراج گیرید و در پشت شهر کمین ساز.»
\par 3 پس یوشع و جمیع مردان جنگی برخاستندتا به عای بروند، و یوشع سی هزار نفر از مردان دلاور انتخاب کرده، ایشان را در شب فرستاد.
\par 4 وایشان را امر فرموده، گفت: «اینک شما برای شهردر کمین باشید، یعنی از پشت شهر و از شهربسیار دور مروید، و همه شما مستعد باشید.
\par 5 ومن و تمام قومی که با منند نزدیک شهر خواهیم آمد، و چون مثل دفعه اول به مقابله ما بیرون آینداز پیش ایشان فرار خواهیم کرد.
\par 6 و ما را تعاقب خواهند کرد تا ایشان را از شهر دور سازیم، زیراخواهند گفت که مثل دفعه اول از حضور ما فرارمی کنند، پس از پیش ایشان خواهیم گریخت.
\par 7 آنگاه از کمین گاه برخاسته، شهر را به تصرف آورید، زیرا یهوه، خدای شما آن را به‌دست شماخواهد داد.
\par 8 و چون شهر را گرفته باشید پس شهر را به آتش بسوزانید و موافق سخن خداوندبه عمل آورید. اینک شما را امر نمودم.»
\par 9 پس یوشع ایشان را فرستاد و به کمین گاه رفته، در میان بیت ئیل و عای به طرف غربی عای ماندند و یوشع آن شب را در میان قوم بسر برد.
\par 10 و یوشع بامدادان بزودی برخاسته، قوم راصف آرایی نمود، و او با مشایخ اسرائیل پیش روی قوم بسوی عای روانه شدند.
\par 11 و تمامی مردان جنگی که با وی بودند روانه شده، نزدیک آمدند و در مقابل شهر رسیده، به طرف شمال عای فرود آمدند، و در میان او و عای وادی‌ای بود.
\par 12 و قریب به پنج هزار نفر گرفته، ایشان را درمیان بیت ئیل و عای به طرف غربی شهر در کمین نهاد.
\par 13 پس قوم، یعنی تمامی لشکر که به طرف شمالی شهر بودند و آنانی را که به طرف غربی شهر در کمین بودند قرار دادند، و یوشع آن شب در میان وادی رفت.
\par 14 و چون ملک عای این رادید او و تمامی قومش تعجیل نموده، به زودی برخاستند، و مردان شهر به مقابله بنی‌اسرائیل برای جنگ به‌جای معین پیش عربه بیرون رفتند، و او ندانست که در پشت شهر برای وی در کمین هستند.
\par 15 و یوشع و همه اسرائیل خود را ازحضور ایشان منهزم ساخته، به راه بیابان فرارکردند.
\par 16 و تمامی قومی را که در شهر بودند ندادر‌دادند تا ایشان را تعاقب کنند. پس یوشع راتعاقب نموده، از شهر دور شدند.
\par 17 و هیچکس در عای و بیت ئیل باقی نماند که از عقب بنی‌اسرائیل بیرون نرفت، و دروازه های شهر را بازگذاشته، اسرائیل را تعاقب نمودند.
\par 18 و خداوند به یوشع گفت: «مزراقی که دردست توست بسوی عای دراز کن، زیرا آن را به‌دست تو دادم و یوشع، مزراقی را که به‌دست خود داشت به سوی شهر دراز کرد.
\par 19 و آنانی که در کمین بودند بزودی از جای خود برخاستند وچون او دست خود را دراز کرد دویدند و داخل شهر شده، آن را گرفتند و تعجیل نموده، شهر را به آتش سوزانیدند.
\par 20 و مردان عای بر عقب نگریسته، دیدند که اینک دود شهر بسوی آسمان بالا می‌رود. پس برای ایشان طاقت نماند که به این طرف و آن طرف بگریزند و قومی که به سوی صحرا می‌گریختند بر تعاقب کنندگان خودبرگشتند.
\par 21 و چون یوشع و تمامی اسرائیل دیدند که آنانی که در کمین بودند شهر را گرفته اندو دود شهر بالا می‌رود ایشان برگشته، مردان عای را شکست دادند.
\par 22 و دیگران به مقابله ایشان ازشهر بیرون آمدند، و ایشان در میان اسرائیل بودند. آنان از یک طرف و اینان از طرف دیگر وایشان را می‌کشتند به حدی که کسی از آنها باقی نماند و نجات نیافت.
\par 23 و ملک عای را زنده گرفته، او را نزد یوشع آوردند.
\par 24 و واقع شد که چون اسرائیل از کشتن همه ساکنان عای در صحرا و در بیابانی که ایشان را درآن تعاقب می‌نمودند فارغ شدند، و همه آنها از دم شمشیر افتاده، هلاک گشتند، تمامی اسرائیل به عای برگشته آن را به دم شمشیر کشتند.
\par 25 و همه آنانی که در آن روز از مرد و زن افتادند دوازده هزار نفر بودند یعنی تمامی مردمان عای.
\par 26 زیرایوشع دست خود را که با مزراق دراز کرده بود، پس نکشید تا تمامی ساکنان عای را هلاک کرد.
\par 27 لیکن بهایم و غنیمت آن شهر را اسرائیل برای خود به تاراج بردند موافق کلام خداوند که به یوشع امر فرموده بود.
\par 28 پس یوشع عای راسوزانید و آن را توده ابدی و خرابه ساخت که تاامروز باقی است.
\par 29 و ملک عای را تا وقت شام به دار کشید، و در وقت غروب آفتاب، یوشع فرمودتا لاش او را از دار پایین آورده، او را نزد دهنه دروازه شهر انداختند و توده بزرگ از سنگها بر آن برپا کردند که تا امروز باقی است.
\par 30 آنگاه یوشع مذبحی برای یهوه، خدای اسرائیل در کوه عیبال بنا کرد.
\par 31 چنانکه موسی، بنده خداوند، بنی‌اسرائیل را امر فرموده بود، به طوری که در کتاب تورات موسی مکتوب است، یعنی مذبحی از سنگهای ناتراشیده که کسی برآنها آلات آهنین بلند نکرده بود و بر آن قربانی های سلامتی برای خداوند گذرانیدند وذبایح سلامتی ذبح کردند.
\par 32 و در آنجا بر آن سنگها نسخه تورات موسی را که نوشته بود به حضور بنی‌اسرائیل مرقوم ساخت.
\par 33 و تمامی اسرائیل و مشایخ و روسا و داوران ایشان به هر دوطرف تابوت پیش لاویان کهنه که تابوت عهدخداوند را برمی داشتند ایستادند، هم غریبان وهم متوطنان؛ نصف ایشان به طرف کوه جرزیم ونصف ایشان به طرف کوه عیبال چنانکه موسی بنده خداوند امر فرموده بود، تا قوم اسرائیل رااول برکت دهند.
\par 34 و بعد از آن تمامی سخنان شریعت، هم برکت‌ها و هم لعنت‌ها را به طوری که در کتاب تورات مرقوم است، خواند.از هرچه موسی‌امر فرموده بود حرفی نبود که یوشع به حضور تمام جماعت اسرائیل با زنان و اطفال وغریبانی که در میان ایشان می‌رفتند، نخواند.
\par 35 از هرچه موسی‌امر فرموده بود حرفی نبود که یوشع به حضور تمام جماعت اسرائیل با زنان و اطفال وغریبانی که در میان ایشان می‌رفتند، نخواند.
 
\chapter{9}

\par 1 و واقع شد که تمامی ملوک حتیان و اموریان و کنعانیان و فرزیان و حویان و یبوسیان، که به آن طرف اردن در کوه و هامون و درتمامی کناره دریای بزرگ تا مقابل لبنان بودند، چون این را شنیدند،
\par 2 با هم جمع شدند، تا بایوشع و اسرائیل متفق جنگ کنند.
\par 3 و اما ساکنان جبعون چون آنچه را که یوشع به اریحا و عای کرده بود شنیدند،
\par 4 ایشان نیز به حیله رفتار نمودند و روانه شده، خویشتن را مثل ایلچیان ظاهر کرده، جوالهای کهنه بر الاغهای خود و مشکهای شراب که کهنه و پاره و بسته شده بود، گرفتند.
\par 5 و بر پایهای خود کفشهای مندرس و پینه زده و بر بدن خود رخت کهنه وتمامی نان توشه ایشان خشک و کفه زده بود.
\par 6 ونزد یوشع به اردو در جلجال آمده، به او و به مردان اسرائیل گفتند که «از زمین دور آمده‌ایم پس الان با ما عهد ببندید.»
\par 7 و مردان اسرائیل به حویان گفتند: «شاید در میان ما ساکن باشید. پس چگونه با شما عهد ببندیم؟»
\par 8 ایشان به یوشع گفتند: «ما بندگان تو هستیم.» یوشع به ایشان گفت که «شما کیانید و از کجامی آیید؟»
\par 9 به وی گفتند: «بندگانت به‌سبب اسم یهوه خدای تو از زمین بسیار دور آمده‌ایم زیرا که آوازه او و هر‌چه را که در مصر کرد، شنیدیم.
\par 10 ونیز آنچه را به دو ملک اموریان که به آن طرف اردن بودند یعنی به سیهون، ملک حشبون، و عوج، ملک باشان، که در عشتاروت بود، کرد.
\par 11 پس مشایخ ما و تمامی ساکنان زمین ما به ما گفتند که توشه‌ای به جهت راه به‌دست خود بگیرید و به استقبال ایشان رفته، ایشان را بگویید که ما بندگان شما هستیم. پس الان با ما عهد ببندید.
\par 12 این نان ما در روزی که روانه شدیم تا نزد شما بیاییم از خانه های خود آن را برای توشه راه گرم گرفتیم، والان اینک خشک و کفه زده شده است.
\par 13 و این مشکهای شراب که پر کردیم تازه بود واینک پاره شده، و این رخت و کفشهای ما از کثرت طول راه کهنه شده است.»
\par 14 آنگاه آن مردمان از توشه ایشان گرفتند و از دهان خداوند مشورت نکردند.
\par 15 و یوشع با ایشان صلح کرده، عهد بست که ایشان را زنده نگهدارد و روسای جماعت باایشان قسم خوردند.
\par 16 اما بعد از انقضای سه روز که با ایشان عهدبسته بودند، شنیدند که آنها نزدیک ایشانند و درمیان ایشان ساکنند.
\par 17 پس بنی‌اسرائیل کوچ کرده، در روز سوم به شهرهای ایشان رسیدند وشهرهای ایشان، جبعون و کفیره و بئیروت و قریه یعاریم، بود.
\par 18 و بنی‌اسرائیل ایشان را نکشتندزیرا روسای جماعت برای ایشان به یهوه، خدای اسرائیل، قسم خورده بودند، و تمامی جماعت برروسا همهمه کردند.
\par 19 و جمیع روسا به تمامی جماعت گفتند که برای ایشان به یهوه، خدای اسرائیل، قسم خوردیم پس الان نمی توانیم به ایشان ضرر برسانیم.
\par 20 این را به ایشان خواهیم کرد و ایشان را زنده نگاه خواهیم داشت مبادا به‌سبب قسمی که برای ایشان خوردیم، غضب بر مابشود.
\par 21 و روسا به ایشان گفتند: «بگذارید که زنده بمانند.» پس برای تمامی جماعت هیزم شکنان و سقایان آب شدند، چنانکه روسا به ایشان گفته بودند.
\par 22 و یوشع ایشان را خواند و بدیشان خطاب کرده، گفت: «چرا ما را فریب دادید و گفتید که مااز شما بسیار دور هستیم و حال آنکه در میان ما ساکنید.
\par 23 پس حال شما ملعونید و از شماغلامان و هیزم شکنان و سقایان آب همیشه برای خانه خدای ما خواهند بود.»
\par 24 ایشان در جواب یوشع گفتند: «زیرا که بندگان تو را یقین خبر دادندکه یهوه، خدای تو، بنده خود موسی را امر کرده بود که تمامی این زمین را به شما بدهد، و همه ساکنان زمین را از پیش روی شما هلاک کنند، وبرای جانهای خود به‌سبب شما بسیار ترسیدیم، پس این کار را کردیم.
\par 25 و الان، اینک ما در دست تو هستیم؛ به هر طوری که در نظر تو نیکو وصواب است که به ما رفتار نمایی، عمل نما.»
\par 26 پس او با ایشان به همین طور عمل نموده، ایشان را از دست بنی‌اسرائیل رهایی داد که ایشان را نکشتند.و یوشع در آن روز ایشان را مقررکرد تا هیزم شکنان و سقایان آب برای جماعت وبرای مذبح خداوند باشند، در مقامی که او اختیارکند و تا به امروز چنین هستند.
\par 27 و یوشع در آن روز ایشان را مقررکرد تا هیزم شکنان و سقایان آب برای جماعت وبرای مذبح خداوند باشند، در مقامی که او اختیارکند و تا به امروز چنین هستند. 
 
\chapter{10}

\par 1 و چون ادونی صدق، ملک اورشلیم شنید که یوشع عای را گرفته و آن را تباه کرده، و به طوری که به اریحا و ملکش عمل نموده بود به عای و ملکش نیز عمل نموده است، و ساکنان جبعون با اسرائیل صلح کرده، در میان ایشان می‌باشند،
\par 2 ایشان بسیار ترسیدند زیراجبعون، شهر بزرگ، مثل یکی از شهرهای پادشاه نشین بود، و مردانش شجاع بودند.
\par 3 پس ادونی صدق، ملک اورشلیم نزد هوهام، ملک حبرون، و فرآم، ملک یرموت، و یافیع، ملک لاخیش، و دبیر، ملک عجلون، فرستاده، گفت:
\par 4 «نزد من آمده، مرا اعانت کنید، تا جبعون را بزنیم زیرا که با یوشع و بنی‌اسرائیل صلح کرده‌اند.»
\par 5 پس پنج ملک اموریان یعنی ملک اورشلیم وملک حبرون و ملک یرموت و ملک لاخیش وملک عجلون جمع شدند، و با تمام لشکر خودبرآمدند، و در مقابل جبعون اردو زده، با آن جنگ کردند.
\par 6 پس مردان جبعون نزد یوشع به اردو درجلجال فرستاده، گفتند: «دست خود را ازبندگانت بازمدار. بزودی نزد ما بیا و ما را نجات بده، و مدد کن زیرا تمامی ملوک اموریانی که درکوهستان ساکنند، بر ما جمع شده‌اند.»
\par 7 پس یوشع با جمیع مردان جنگی و همه مردان شجاع از جلجال آمد.
\par 8 و خداوند به یوشع گفت: «ازآنها مترس زیرا ایشان را به‌دست تو دادم و کسی از ایشان پیش تو نخواهد ایستاد.»
\par 9 پس یوشع تمامی شب از جلجال کوچ کرده، ناگهان به ایشان برآمد.
\par 10 و خداوند ایشان را پیش اسرائیل منهزم ساخت، و ایشان را در جبعون به کشتارعظیمی کشت. و ایشان را به راه گردنه بیت حورون گریزانید، و تا عزیقه و مقیده ایشان راکشت.
\par 11 و چون از پیش اسرائیل فرار می‌کردندو ایشان در سرازیری بیت حورون می‌بودند، آنگاه خداوند تا عزیقه بر ایشان از آسمان سنگهای بزرگ بارانید و مردند. و آنانی که ازسنگهای تگرگ مردند، بیشتر بودند از کسانی که بنی‌اسرائیل به شمشیر کشتند.
\par 12 آنگاه یوشع در روزی که خداوند اموریان را پیش بنی‌اسرائیل تسلیم کرد، به خداوند درحضور بنی‌اسرائیل تکلم کرده، گفت: «ای آفتاب بر جبعون بایست و تو‌ای ماه بر وادی ایلون.»
\par 13 پس آفتاب ایستاد و ماه توقف نمود تا قوم ازدشمنان خود انتقام گرفتند، مگر این در کتاب یاشر مکتوب نیست که آفتاب در میان آسمان ایستاد و قریب به تمامی روز به فرو رفتن تعجیل نکرد.
\par 14 و قبل از آن و بعد از آن روزی مثل آن واقع نشده بود که خداوند آواز انسان را بشنودزیرا خداوند برای اسرائیل جنگ می‌کرد.
\par 15 پس یوشع با تمامی اسرائیل به اردو به جلجال برگشتند.
\par 16 اما آن پنج ملک فرار کرده، خود را در مغاره مقیده پنهان ساختند.
\par 17 و به یوشع خبر داده، گفتند: «که آن پنج ملک پیدا شده‌اند و در مغاره مقیده پنهانند.»
\par 18 یوشع گفت: «سنگهایی بزرگ به دهنه مغاره بغلطانید و بر آن مردمان بگمارید تاایشان را نگاهبانی کنند.
\par 19 و اما شما توقف منمایید بلکه دشمنان خود را تعاقب کنید وموخر ایشان را بکشید و مگذارید که به شهرهای خود داخل شوند، زیرا یهوه خدای شما ایشان رابه‌دست شما تسلیم نموده است.»
\par 20 و چون یوشع و بنی‌اسرائیل از کشتن ایشان به کشتاربسیار عظیمی تا نابود شدن ایشان فارغ شدند، وبقیه‌ای که از ایشان نجات یافتند، به شهرهای حصاردار درآمدند.
\par 21 آنگاه تمامی قوم نزدیوشع به اردو در مقیده به سلامتی برگشتند، وکسی زبان خود را بر احدی از بنی‌اسرائیل تیزنساخت.
\par 22 پس یوشع گفت: «دهنه مغاره را بگشایید وآن پنج ملک را از مغاره، نزد من بیرون آورید.»
\par 23 پس چنین کردند، و آن پنج ملک، یعنی ملک اورشلیم و ملک حبرون و ملک یرموت و ملک لاخیش و ملک عجلون را از مغاره نزد وی بیرون آوردند.
\par 24 و چون ملوک را نزد یوشع بیرون آوردند، یوشع تمامی مردان اسرائیل را خواند وبه‌سرداران مردان جنگی که همراه وی می‌رفتند، گفت: «نزدیک بیایید و پایهای خود را بر گردن این ملوک بگذارید.» پس نزدیک آمده، پایهای خودرا بر گردن ایشان گذاردند.
\par 25 و یوشع به ایشان گفت: «مترسید و هراسان مباشید. قوی و دلیرباشید زیرا خداوند با همه دشمنان شما که باایشان جنگ می‌کنید، چنین خواهد کرد.»
\par 26 وبعد از آن یوشع ایشان را زد و کشت و بر پنج دارکشید که تا شام بر دارها آویخته بودند.
\par 27 و دروقت غروب آفتاب، یوشع فرمود تا ایشان را ازدارها پایین آوردند، و ایشان را به مغاره‌ای که درآن پنهان بودند انداختند، و به دهنه مغاره سنگهای بزرگ که تا امروز باقی است، گذاشتند.
\par 28 و در آن روز یوشع مقیده را گرفت، و آن وملکش را به دم شمشیر زده، ایشان و همه نفوسی را که در آن بودند، هلاک کرد، و کسی را باقی نگذاشت، و به طوری که با ملک اریحا رفتارنموده بود، با ملک مقیده نیز رفتار کرد.
\par 29 و یوشع با تمامی اسرائیل از مقیده به لبنه گذشت و با لبنه جنگ کرد.
\par 30 و خداوند آن را نیزبا ملکش به‌دست اسرائیل تسلیم نمود، پس آن وهمه کسانی را که در آن بودند به دم شمشیر کشت و کسی را باقی نگذاشت، و به طوری که با ملک اریحا رفتار نموده بود با ملک آن نیز رفتار کرد.
\par 31 و یوشع با تمامی اسرائیل از لبنه به لاخیش گذشت و به مقابلش اردو زده، با آن جنگ کرد.
\par 32 و خداوند لاخیش را به‌دست اسرائیل تسلیم نمود که آن را در روز دوم تسخیر نمود. و آن وهمه کسانی را که در آن بودند به دم شمشیر کشت چنانکه به لبنه کرده بود.
\par 33 آنگاه هورام ملک جازر برای اعانت لاخیش آمد، و یوشع او و قومش را شکست داد، به حدی که کسی را برای او باقی نگذاشت.
\par 34 و یوشع با تمامی اسرائیل از لاخیش به عجلون گذشتند و به مقابلش اردو زده، با آن جنگ کردند.
\par 35 و در همان روز آن را گرفته، به دم شمشیر زدند و همه کسانی را که در آن بودند درآن روز هلاک کرد چنانکه به لاخیش کرده بود.
\par 36 و یوشع با تمامی اسرائیل از عجلون به حبرون برآمده، با آن جنگ کردند.
\par 37 و آن راگرفته، آن را با ملکش و همه شهرهایش و همه کسانی که در آن بودند به دم ششیر زدند، و موافق هر‌آنچه که به عجلون کرده بود کسی را باقی نگذاشت، بلکه آن را با همه کسانی که در آن بودند، هلاک ساخت.
\par 38 و یوشع با تمامی اسرائیل به دبیر برگشت وبا آن جنگ کرد.
\par 39 و آن را با ملکش و همه شهرهایش گرفت و ایشان را به دم شمشیر زدند، و همه کسانی را که در آن بودند، هلاک ساختند واو کسی را باقی نگذاشت و به طوری که به حبرون رفتار نموده بود به دبیر و ملکش نیز رفتار کرد، چنانکه به لبنه و ملکش نیز رفتار نموده بود.
\par 40 پس یوشع تمامی آن زمین یعنی کوهستان و جنوب و هامون و وادیها و جمیع ملوک آنها رازده، کسی را باقی نگذاشت و هر ذی نفس راهلاک کرده، چنانکه یهوه، خدای اسرائیل، امرفرموده بود.
\par 41 و یوشع ایشان را از قادش برنیع تاغزه و تمامی زمین جوشن را تا جبعون زد.
\par 42 و یوشع جمیع این ملوک و زمین ایشان را در یک وقت گرفت، زیرا که یهوه، خدای اسرائیل، برای اسرائیل جنگ می‌کرد.و یوشع با تمامی اسرائیل به اردو در جلجال مراجعت کردند.
\par 43 و یوشع با تمامی اسرائیل به اردو در جلجال مراجعت کردند.
 
\chapter{11}

\par 1 و واقع شد که چون یابین ملک حاصور این را شنید، نزد یوباب ملک مادون و نزد ملک شمرون و نزد ملک اخشاف فرستاد.
\par 2 و نزدملوکی که به طرف شمال در کوهستان، و در عربه، جنوب کنروت، و در هامون و در نافوت دور، به طرف مغرب بودند.
\par 3 و نزد کنعانیان به طرف مشرق و مغرب و اموریان و حتیان و فرزیان ویبوسیان در کوهستان، و حویان زیر حرمون درزمین مصفه.
\par 4 و آنها با تمامی لشکرهای خود که قوم بسیاری بودند و عدد ایشان مثل ریگ درکناره دریا بود با اسبان و ارابه های بسیار بیرون آمدند.
\par 5 و تمامی این ملوک جمع شده، آمدند ونزد آبهای میروم در یک جا اردو زدند تا بااسرائیل جنگ کنند.
\par 6 و خداوند به یوشع گفت: «از ایشان مترس زیرا که فردا چنین وقتی جمیع ایشان را کشته شده، به حضور اسرائیل تسلیم خواهم کرد، واسبان ایشان را پی خواهی کرد، و ارابه های ایشان را به آتش خواهی سوزانید.»
\par 7 پس یوشع باتمامی مردان جنگی به مقابله ایشان نزد آبهای میروم ناگهان آمده، بر ایشان حمله کردند.
\par 8 وخداوند ایشان را به‌دست اسرائیل تسلیم نمود، که ایشان را زدند و تا صیدون بزرگ و مسرفوت مایم و تا وادی مصفه به طرف شرقی تعاقب کرده، کشتند، به حدی که کسی را از ایشان باقی نگذاشتند.
\par 9 و یوشع به طوری که خداوند به وی گفته بود با ایشان رفتار نموده، اسبان ایشان را پی کرد و ارابه های ایشان را به آتش سوزانید.
\par 10 و یوشع در آن وقت برگشت، و حاصور راگرفته، ملکش را با شمشیر کشت، زیرا حاصورقبل از آن سر جمیع آن ممالک بود.
\par 11 و همه کسانی را که در آن بودند به دم شمشیر کشته، ایشان را بالکل هلاک کرد، و هیچ ذی حیات باقی نماند، و حاصور را به آتش سوزانید.
\par 12 و یوشع تمامی شهرهای آن ملوک و جمیع ملوک آنها راگرفت و ایشان را به دم شمشیر کشته، بالکل هلاک کرد به طوری که موسی بنده خداوند امر فرموده بود.
\par 13 لکن همه شهرهایی که بر تلهای خوداستوار بودند اسرائیل آنها را نسوزانید، سوای حاصور که یوشع آن را فقط سوزانید.
\par 14 وبنی‌اسرائیل تمامی غنیمت آن شهرها و بهایم آنها را برای خود به غارت بردند، اما همه مردم رابه دم شمشیر کشتند، به حدی که ایشان را هلاک کرده، هیچ ذی حیات را باقی نگذاشتند.
\par 15 چنانکه خداوند بنده خود موسی را امر فرموده بود، همچنین موسی به یوشع امر فرمود و به همین طور یوشع عمل نمود، و چیزی از جمیع احکامی که خداوند به موسی فرموده بود، باقی نگذاشت.
\par 16 پس یوشع تمامی آن زمین کوهستان وتمامی جنوب و تمامی زمین جوشن و هامون وعربه و کوهستان اسرائیل و هامون آن را گرفت.
\par 17 از کوه حالق که به سوی سعیر بالا می‌رود تابعل جاد که در وادی لبنان زیر کوه حرمان است، و جمیع ملوک آنها را گرفته، ایشان را زد و کشت.
\par 18 و یوشع روزهای بسیار با این ملوک جنگ کرد.
\par 19 و شهری نبود که با بنی‌اسرائیل صلح کرده باشد، جز حویانی که در جبعون ساکن بودند وهمه دیگران را در جنگ گرفتند.
\par 20 زیرا از جانب خداوند بود که دل ایشان را سخت کند تا به مقابله اسرائیل درآیند و او ایشان را بالکل هلاک سازد، و بر ایشان رحمت نشود بلکه ایشان را نابود سازدچنانکه خداوند به موسی‌امر فرموده بود.
\par 21 و در آن زمان یوشع آمده، عناقیان را ازکوهستان از جبرون و دبیر و عناب و همه کوههای یهودا و همه کوههای اسرائیل منقطع ساخت، و یوشع ایشان را با شهرهای ایشان بالکل هلاک کرد.
\par 22 کسی از عناقیان در زمین بنی‌اسرائیل باقی نماند، لیکن در غزا و جت واشدود بعضی باقی ماندند.پس یوشع تمامی زمین را برحسب آنچه خداوند به موسی گفته بود، گرفت، و یوشع آن را به بنی‌اسرائیل برحسب فرقه‌ها و اسباط ایشان به ملکیت بخشید و زمین ازجنگ آرام گرفت.
\par 23 پس یوشع تمامی زمین را برحسب آنچه خداوند به موسی گفته بود، گرفت، و یوشع آن را به بنی‌اسرائیل برحسب فرقه‌ها و اسباط ایشان به ملکیت بخشید و زمین ازجنگ آرام گرفت.
 
\chapter{12}

\par 1 و اینانند ملوک آن زمین که بنی‌اسرائیل کشتند، و زمین ایشان را به آن طرف اردن به سوی مطلع آفتاب از وادی ارنون تا کوه حرمون، و تمامی عربه شرقی را متصرف شدند.
\par 2 سیهون ملک اموریان که در حشبون ساکن بود، واز عروعیر که به کناره وادی ارنون است، و ازوسط وادی و نصف جلعاد تا وادی یبوق که سرحد بنی عمون است، حکمرانی می‌کرد.
\par 3 و ازعربه تا دریای کنروت به طرف مشرق و تا دریای عربه، یعنی بحرالملح به طرف مشرق به راه بیت یشیموت و به طرف جنوب زیر دامن فسجه.
\par 4 و سر حد عوج، ملک باشان، که از بقیه رفائیان بود و در عشتاروت و ادرعی سکونت داشت.
\par 5 ودر کوه حرمون و سلخه و تمامی باشان تا سر حدجشوریان و معکیان و بر نصف جلعاد تا سرحدسیهون، ملک حشبون حکمرانی می‌کرد.
\par 6 اینهارا موسی بنده خداوند و بنی‌اسرائیل زدند، وموسی بنده خداوند آن را به روبینیان و جادیان ونصف سبط منسی به ملکیت داد. 
\par 7 و اینانند ملوک آن زمین که یوشع وبنی‌اسرائیل ایشان را در آن طرف اردن به سمت مغرب کشت، از بعل جاد در وادی لبنان، تا کوه حالق که به سعیر بالا می‌رود، و یوشع آن را به اسباط اسرائیل برحسب فرقه های ایشان به ملکیت داد.
\par 8 در کوهستان و هامون و عربه ودشتها و صحرا و در جنوب از حتیان و اموریان وکنعانیان و فرزیان و حویان و یبوسیان.
\par 9 یکی ملک اریحا و یکی ملک عای که در پهلوی بیت ئیل است.
\par 10 و یکی ملک اورشلیم و یکی ملک حبرون.
\par 11 و یکی ملک یرموت و یکی ملک لاخیش.
\par 12 و یکی ملک عجلون و یکی ملک جازر.
\par 13 و یکی ملک دبیر و یکی ملک جادر.
\par 14 و یکی ملک حرما و یکی ملک عراد.
\par 15 و یکی ملک لبنه و یکی ملک عدلام.
\par 16 و یکی ملک مقیده و یکی ملک بیت ئیل.
\par 17 و یکی ملک تفوح و یکی ملک حافر.
\par 18 و یکی ملک عفیق و یکی ملک لشارون.
\par 19 و یکی ملک مادون و یکی ملک حاصور.
\par 20 و یکی ملک شمرون مرون و یکی ملک اکشاف.
\par 21 و یکی ملک تعناک و یکی ملک مجدو
\par 22 و یکی ملک قادش و یکی ملک یقنعام در کرمل.
\par 23 و یکی ملک دور در نافت دور و یکی ملک امتها در جلجال.پس یکی ملک ترصه وجمیع ملوک سی و یک نفر بودند.
\par 24 پس یکی ملک ترصه وجمیع ملوک سی و یک نفر بودند.
 
\chapter{13}

\par 1 و یوشع پیر و سالخورده شد، و خداوندبه وی گفت: «تو پیر و سالخورده شده‌ای و هنوز زمین بسیار برای تصرف باقی می‌ماند.
\par 2 و این است زمینی که باقی می‌ماند، تمامی بلوک فلسطینیان و جمیع جشوریان.
\par 3 ازشیحور که در مقابل مصر است تا سرحد عقرون به سمت شمال که از کنعانیان شمرده می‌شود، یعنی پنج سردار فلسطینیان از غزیان و اشدودیان و اشقلونیان و جتیان و عقرونیان و عویان.
\par 4 و ازجنوب تمامی زمین کنعانیان و مغاره‌ای که ازصیدونیان است تا افیق و تا سرحد اموریان.
\par 5 وزمین جبلیان و تمامی لبنان به سمت مطلع آفتاب از بعل جاد که زیر کوه حرمون است تا مدخل حمات.
\par 6 تمامی ساکنان کوهستان از لبنان تامصرفوت مایم که جمیع صیدونیان باشند، من ایشان را از پیش بنی‌اسرائیل بیرون خواهم کرد، لیکن تو آنها را به بنی‌اسرائیل به ملکیت به قرعه تقسیم نما چنانکه تو را امر فرموده‌ام.
\par 7 پس الان این زمین را به نه سبط و نصف سبط منسی برای ملکیت تقسیم نما.»
\par 8 با او روبینیان و جادیان ملک خود را گرفتندکه موسی در آن طرف اردن به سمت مشرق به ایشان داد، چنانکه موسی بنده خداوند به ایشان بخشیده بود.
\par 9 از عروعیر که بر کناره وادی ارنون است، و شهری که در وسط وادی است، و تمامی بیابان میدبا تا دیبون.
\par 10 و جمیع شهرهای سیهون ملک اموریان که در حشبون تا سرحد بنی عمون حکمرانی می‌کرد.
\par 11 و جلعاد و سرحدجشوریان و معکیان و تمامی کوه حرمون وتمامی باشان تا سلخه.
\par 12 و تمامی ممالک عوج در باشان که در اشتاروت و ادرعی حکمرانی می‌کرد، و او از بقیه رفائیان بود. پس موسی ایشان را شکست داد و بیرون کرد.
\par 13 اما بنی‌اسرائیل جشوریان و معکیان را بیرون نکردند، پس جشور و معکی تا امروز در میان اسرائیل ساکنند.
\par 14 لیکن به سبط لاوی هیچ ملکیت نداد، زیراهدایای آتشین یهوه خدای اسرائیل ملکیت وی است چنانکه به او گفته بود.
\par 15 و موسی به سبط بنی روبین برحسب قبیله های ایشان داد.
\par 16 و حدود ایشان از عروعیربود که به کنار وادی ارنون است و شهری که دروسط وادی است و تمامی بیابان که پهلوی میدبااست.
\par 17 حشبون و تمامی شهرهایش که در بیابان است و دیبون و باموت بعل و بیت بعل معون.
\par 18 ویهصه و قدیموت و میفاعت.
\par 19 و قریتایم و سبمه و سارت شحر که در کوه دره بود.
\par 20 و بیت فغورو دامن فسجه و بیت یشیموت.
\par 21 و تمامی شهرهای بیابان و تمامی ممالک سیهون، ملک اموریان، که در حشبون حکمرانی می‌کرد، وموسی او را با سرداران مدیان یعنی اوی و راقم وصور و حور و رابع، امرای سیهون، که در آن زمین ساکن بودند، شکست داد.
\par 22 و بلعام بن بعور فالگیر را بنی‌اسرائیل در میان کشتگان به شمشیرکشتند.
\par 23 و سرحد بنی روبین اردن و کناره‌اش بود. این ملکیت بنی روبین برحسب قبیله های ایشان بود یعنی شهرها و دهات آنها.
\par 24 و موسی به سبط جاد یعنی به بنی جادبرحسب قبیله های ایشان داد.
\par 25 و سرحد ایشان یعزیز بود و تمامی شهرهای جلعاد و نصف زمین بنی عمون تا عروعیر که در مقابل ربه است.
\par 26 واز حشبون تا رامت مصفه و بطونیم و از محنایم تاسرحد دبیر.
\par 27 و در دره بیت هارام و بیت نمره وسکوت و صافون و بقیه مملکت سیهون، ملک حشبون، و اردن و کناره آن تا انتهای دریای کنرت در آن طرف اردن به سمت مشرق.
\par 28 این است ملکیت بنی جاد برحسب قبیله های ایشان یعنی شهرها و دهات آنها.
\par 29 و موسی به نصف سبط منسی داد و برای نصف سبط بنی منسی برحسب قبیله های ایشان برقرار شد.
\par 30 و حدود ایشان از محنایم تمامی باشان یعنی تمامی ممالک عوج، ملک باشان وتمامی قریه های یائیر که در باشان است، شصت شهر بود.
\par 31 و نصف جلعاد و عشتاروت و ادرعی شهرهای مملکت عوج در باشان برای پسران ماکیر بن منسی یعنی برای نصف پسران ماکیربرحسب قبیله های ایشان بود.
\par 32 اینهاست آنچه موسی در عربات موآب درآن طرف اردن در مقابل اریحا به سمت مشرق برای ملکیت تقسیم کرد.لیکن به سبط لاوی، موسی هیچ نصیب نداد زیرا که یهوه، خدای اسرائیل، نصیب ایشان است چنانکه به ایشان گفته بود.
\par 33 لیکن به سبط لاوی، موسی هیچ نصیب نداد زیرا که یهوه، خدای اسرائیل، نصیب ایشان است چنانکه به ایشان گفته بود.
 
\chapter{14}

\par 1 و اینهاست ملکهایی که بنی‌اسرائیل درزمین کنعان گرفتند، که العازار کاهن ویوشع بن نون و روسای آبای اسباط بنی‌اسرائیل برای ایشان تقسیم کردند.
\par 2 برحسب قرعه، ملکیت ایشان شد، برای نه سبط و نصف سبط، چنانکه خداوند به‌دست موسی‌امر فرموده بود.
\par 3 زیرا که موسی ملکیت دو سبط و نصف سبط رابه آن طرف اردن داده بود، امابه لاویان هیچ ملکیت در میان ایشان نداد.
\par 4 زیرا پسران یوسف دو سبط بودند، یعنی منسی و افرایم، و به لاویان هیچ قسمت در زمین ندادند، غیر از شهرها به جهت سکونت و اطراف آنها به جهت مواشی واموال ایشان.
\par 5 چنانکه خداوند موسی را امرفرموده بود، همچنان بنی‌اسرائیل عمل نموده، زمین را تسلیم کردند.
\par 6 آنگاه بنی یهودا در جلجال نزد یوشع آمدند، و کالیب بن یفنه قنزی وی را گفت: «سخنی را که خداوند به موسی، مرد خدا، درباره من و تو وقادش برنیع گفت می‌دانی.
\par 7 من چهل ساله بودم وقتی که موسی، بنده خداوند، مرا از قادش برنیع برای جاسوسی زمین فرستاد، و برای او خبر بازآوردم چنانکه در دل من بود.
\par 8 لیکن برادرانم که همراه من رفته بودند دل قوم را گداختند، و اما من یهوه خدای خود را به تمامی دل پیروی کردم.
\par 9 ودر آن روز موسی قسم خورد و گفت: البته زمینی که پای تو بر آن گذارده شد برای تو و اولادت ملکیت ابدی خواهد بود، زیرا که یهوه خدای مرابه تمامی دل پیروی نمودی.
\par 10 و الان اینک خداوند چنانکه گفته بود این چهل و پنج سال مرازنده نگاه داشته است، از وقتی که خداوند این سخن را به موسی گفت هنگامی که اسرائیل دربیابان راه می‌رفتند، و الان، اینک من امروز هشتادو پنج ساله هستم.
\par 11 و حال امروز قوت من باقی است مثل روزی که موسی مرا فرستاد، چنانکه قوت من در آن وقت بود، همچنان قوت من الان است، خواه برای جنگ کردن و خواه برای رفتن وآمدن.
\par 12 پس الان این کوه را به من بده که در آن روز خداوند درباره‌اش گفت، زیرا تو در آن روزشنیدی که عناقیان در آنجا بودند، و شهرهایش بزرگ و حصاردار است، شاید خداوند با من خواهد بود تا ایشان را بیرون کنم، چنانکه خداوند گفته است.»
\par 13 پس یوشع او را برکت داد و حبرون را به کالیب بن یفنه به ملکیت بخشید.
\par 14 بنابراین حبرون تا امروز ملکیت کالیب بن یفنه قنزی شد، زیرا که یهوه خدای اسرائیل را به تمامی دل پیروی نموده بود.و قبل از آن نام حبرون، قریه اربع بود که او در میان عناقیان مرد بزرگ ترین بود. پس زمین از جنگ آرام گرفت.
\par 15 و قبل از آن نام حبرون، قریه اربع بود که او در میان عناقیان مرد بزرگ ترین بود. پس زمین از جنگ آرام گرفت.
 
\chapter{15}

\par 1 و قرعه به جهت سبط بنی یهودا، به حسب قبایل ایشان، به طرف جنوب به‌سر حد ادوم، یعنی صحرای صین به اقصای تیمان رسید.
\par 2 و حد جنوبی ایشان از آخر بحرالملح، از خلیجی که متوجه به سمت جنوب است، بود.
\par 3 و به طرف جنوب، فراز عکربیم بیرون آمده، به صین گذشت، و به جنوب قادش برنیع برآمده، به حصرون گذشت، و به اداربرآمده، به سوی قرقع برگشت.
\par 4 و از عصمون گذشته، به وادی مصر بیرون آمد، و انتهای این حدتا به دریا بود. این حد جنوبی شما خواهد بود.
\par 5 وحد شرقی، بحرالملح تا آخر اردن بود، و حدطرف شمال، از خلیج دریا تا آخر اردن بود.
\par 6 واین حد تا بیت حجله برآمده، به طرف شمالی بیت عربه گذشت، و این حد نزد سنگ بهن پسر روبین برآمد.
\par 7 و این حد از وادی عخور نزد دبیر برآمد، و به طرف شمال به سوی جلجال که مقابل فرازادمیم است، که در جنوب وادی است، متوجه می‌شود، و این حد نزد آبهای عین شمس گذشت، و انتهایش نزد عین روجل بود.
\par 8 و این حد ازوادی پسر هنوم به‌جانب یبوسی، به طرف جنوب که همان اورشلیم باشد، برآمد. پس این حد به سوی قله کوهی که به طرف مغرب مقابل وادی هنوم، و به طرف شمال به آخر وادی رفائیم است، گذشت.
\par 9 و این حد از قله کوه به چشمه آبهای نفتوح کشیده شد، و نزد شهرهای کوه عفرون بیرون آمد، و تا بعله که قریه یعاریم باشد، کشیده شد.
\par 10 و این حد از بعله به طرف مغرب به کوه سعیر برگشت، و به طرف شمال از جانب کوه یعاریم که کسالون باشد گذشت، و نزد بیت شمس بزیر آمده، از تمنه گذشت.
\par 11 و این حد به سوی شمال از جانب عقرون بیرون آمد، و تا شکرون کشیده شد، و از کوه بعله گذشته، نزد یبنئیل بیرون آمد، و انتهای این حد دریا بود.
\par 12 و حد غربی دریای بزرگ و کناره آن بود، این است حدودبنی یهودا از هر طرف به حسب قبایل ایشان.
\par 13 و به کالیب بن یفنه به حسب آنچه خداوند به یوشع فرموده بود، در میان بنی یهودا قسمتی داد، یعنی قریه اربع پدر عناق که حبرون باشد.
\par 14 وکالیب سه پسر عناق یعنی شیشی و اخیمان وتلمی اولاد عناق را از آنجا بیرون کرد.
\par 15 و ازآنجا به ساکنان دبیر برآمد و اسم دبیر قبل از آن قریه سفر بود.
\par 16 و کالیب گفت: «هر‌که قریه سفررا بزند و آن را بگیرد، دختر خود عکسه را به زنی به او خواهم داد.
\par 17 و عتنئیل پسر قناز برادرکالیب آن را گرفت، و دختر خود عکسه را به او به زنی داد.
\par 18 و چون او نزد وی آمد او را ترغیب کرد که از پدر خود زمینی طلب نماید، و دختر ازالاغ خود پایین آمد، و کالیب وی را گفت: «چه می‌خواهی؟»
\par 19 گفت: «مرا برکت ده. چونکه زمین جنوبی را به من داده‌ای، چشمه های آب نیزبه من بده. پس چشمه های بالا و چشمه های پایین را به او بخشید.
\par 20 این است ملک سبط بنی یهودا به حسب قبایل ایشان.
\par 21 و شهرهای انتهایی سبطبنی یهودا به سمت جنوب بر سرحد ادوم قبصئیل و عیدر و یاجور بود،
\par 22 و قینه و دیمونه وعدعده،
\par 23 و قادش و حاصور و یتنان،
\par 24 و زیف و طالم و بعلوت،
\par 25 و حاصور حدته و قریوت حصرون که حاصور باشد.
\par 26 امام و شماع ومولاده،
\par 27 و حصرجده و حشمون و بیت فالط،
\par 28 و حصر شوعال و بیرشبع و بزیوتیه،
\par 29 و بعاله و عییم و عاصم،
\par 30 و التولد و کسیل و حرمه،
\par 31 وصقلج و مدمنه و سنسنه،
\par 32 و لباوت و سلخیم و عین و رمون، جمیع این شهرها با دهات آنهابیست و نه می‌باشد.
\par 33 و در هامون اشتاول وصرعه و اشنه، 
\par 34 و زانوح و عین جنیم و تفوح وعینام،
\par 35 و یرموت و عدلام و سوکوه و عزیقه،
\par 36 و شعرایم و عدیتایم و الجدیره و جدیرتایم، چهارده شهر با دهات آنها.
\par 37 صنان و حداشاه و مجدل جاد.
\par 38 و دلعان و المصفه و یقتئیل.
\par 39 و لاخیش و بصقه وعجلون.
\par 40 و کبون و لحمان و کتلیش.
\par 41 وجدیروت و بیت داجون و نعمه و مقیده. شانزده شهر با دهات آنها.
\par 42 و لبنه و عاتر و عاشان.
\par 43 و یفتاح و اشنه و نصیب.
\par 44 و قعیله واکزیب و مریشه. نه شهر با دهات آنها.
\par 45 وعقرون و قصبه‌ها و دهات آن.
\par 46 از عقرون تادریا، همه که به اطراف اشدود بود با دهات آنها.
\par 47 و اشدود و قصبه‌ها و دهات آن. و غزا وقصبه‌ها و دهات آن تا وادی مصر، و تا دریای بزرگ و کنار آن.
\par 48 و در کوهستان شامیر و یتیر و سوکوه.
\par 49 ودنه و قریه سنه که دبیر باشد.
\par 50 و عناب و اشتموه و عانیم.
\par 51 و جوشن و حولون و جیلوه، یازده شهر با دهات آنها.
\par 52 و اراب و دومه و اشعان.
\par 53 و یانوم و بیت تفوح و افیقه.
\par 54 و حمطه و قریه اربع که حبرون باشد، و صیعور، نه شهر با دهات آنها.
\par 55 و معون و کرمل و زیف و یوطه.
\par 56 ویزرعیل و یقدعام و زانوح.
\par 57 و القاین و جبعه وتمنه، ده شهر با دهات آنها.
\par 58 و حلحول و بیت صور و جدور.
\par 59 ومعارات و بیت عنوت و التقون، شش شهر با دهات آنها.
\par 60 و قریه بعل که قریه یعاریم باشد و الربه، دوشهر با دهات آنها.
\par 61 و در بیابان بیت عربه و مدین و سکاکه.
\par 62 والنبشان و مدینه الملح و عین جدی، شش شهر بادهات آنها.و اما یبوسیان که ساکن اورشلیم بودند، بنی یهودا نتوانستند ایشان را بیرون کنند. پس یبوسیان با بنی یهودا تا امروز در اورشلیم ساکنند.
\par 63 و اما یبوسیان که ساکن اورشلیم بودند، بنی یهودا نتوانستند ایشان را بیرون کنند. پس یبوسیان با بنی یهودا تا امروز در اورشلیم ساکنند.
 
\chapter{16}

\par 1 و قرعه برای بنی یوسف به سمت مشرق، از اردن اریحا به طرف آبهای اریحا تا صحرایی که از اریحا به سوی کوه بیت ئیل بر می‌آید، بیرون آمد.
\par 2 و از بیت ئیل تالوز برآمده، به‌سرحد ارکیان تا عطاروت گذشت.
\par 3 و به سمت مغرب به‌سرحد یفلیطیان تا کناربیت حورون پایین و تا جازر پایین آمد، و انتهایش تا دریا بود.
\par 4 پس پسران یوسف، منسی و افرایم، ملک خود را گرفتند.
\par 5 و حدود بنی افرایم به حسب قبایل ایشان چنین بود که حد شرقی ملک ایشان عطاروت ادارتا بیت حورون بالا بود.
\par 6 و حد غربی ایشان به طرف شمال نزد مکمیت برآمد و حد ایشان به سمت مشرق به تانه شیلوه برگشته، به طرف مشرق یانوحه از آن گذشت.
\par 7 و از یانوحه به عطاروت و نعره پایین آمده، به اریحا رسید و به اردن منتهی شد.
\par 8 و سرحد غربی آن از تفوح تاوادی قانه رفت و آخر آن به دریا بود، این است ملک سبط بنی افرایم به حسب قبایل ایشان.
\par 9 علاوه بر شهرهایی که از میان ملک بن منسی برای بنی افرایم جدا شده بود، جمیع شهرها بادهات آنها بود.و کنعانیان را که در جازر ساکن بودند، بیرون نکردند. پس کنعانیان تا امروز درمیان افرایم ساکنند، و برای جزیه، بندگان شدند.
\par 10 و کنعانیان را که در جازر ساکن بودند، بیرون نکردند. پس کنعانیان تا امروز درمیان افرایم ساکنند، و برای جزیه، بندگان شدند.
 
\chapter{17}

\par 1 و قسمت سبط منسی این شد، زیرا که او نخست زاده یوسف بود، و اما ماکیرنخست زاده منسی که پدر جلعاد باشد، چونکه مرد جنگی بود جلعاد و باشان به او رسید.
\par 2 وبرای پسران دیگر منسی به حسب قبایل ایشان قسمتی شد، یعنی برای پسران ابیعزر، و برای پسران هالک، و برای پسران اسرئیل، و برای پسران شکیم، و برای پسران حافر، و برای پسران شمیداع. اینان اولاد ذکور منسی بن یوسف برحسب قبایل ایشان می‌باشند.
\par 3 و اما صلفحاد بن حافر بن جلعاد بن ماکیربن منسی را پسران نبود، بلکه دختران، و اینهاست نامهای دخترانش: محله و نوعه و حجله و ملکه وترصه.
\par 4 پس ایشان نزد العازار کاهن و نزد یوشع بن نون و نزد روسا آمده، گفتند که «خداوند موسی را امر فرمود که ملکی در میان برادران ما به مابدهد.» پس برحسب فرمان خداوند، ملکی درمیان برادران پدرشان به ایشان داد.
\par 5 و به منسی سوای زمین جلعاد و باشان که به آن طرف اردن واقع است، ده حصه رسید.
\par 6 زیرا که دختران منسی، ملکی در میان پسرانش یافتند، و پسران دیگر منسی، جلعاد را یافتند.
\par 7 و حد منسی از اشیر تا مکمته که مقابل شکیم است، بود، و حدش به طرف راست تا ساکنان عین تفوح رسید.
\par 8 و زمین تفوح از آن منسی بود، اماتفوح که به‌سرحد منسی واقع است از آن بنی افرایم بود.
\par 9 و حدش به وادی قانه یعنی به طرف جنوب وادی برآمد، و این شهرها از میان شهرهای منسی، ملک افرایم بود، و حد منسی به طرف شمال وادی و انتهایش به دریا بود.
\par 10 جنوب آن از آن افرایم، و شمال آن از آن منسی و دریا حد او بود، و ایشان به سوی شمال تااشیر و به سوی مشرق تا یساکار رسیدند.
\par 11 ومنسی در یساکار و در اشیر بیت‌شان وقصبه هایش، و یبلعام و قصبه هایش، و ساکنان دور و قصبه هایش، و ساکنان عین دور وقصبه هایش، و ساکنان تعناک و قصبه هایش، وساکنان مجدو و قصبه هایش، یعنی سه محال کوهستانی داشت.
\par 12 لیکن بنی منسی ساکنان آن شهرها را نتوانستند بیرون کنند، و کنعانیان جازم بودند که در آن زمین ساکن باشند.
\par 13 و واقع شدکه چون بنی‌اسرائیل قوت یافتند، از کنعانیان جزیه گرفتند، لیکن ایشان را بالکل بیرون نکردند.
\par 14 و بنی یوسف یوشع را خطاب کرده، گفتند: «چرا یک قرعه و یک حصه فقط به من برای ملکیت دادی؟ و حال آنکه من قوم بزرگ هستم، چونکه خداوند تا الان مرا برکت داده است.»
\par 15 یوشع به ایشان گفت: «اگر تو قوم بزرگ هستی به جنگل برآی و در آنجا در زمین فرزیان ورفائیان برای خود مکانی صاف کن، چونکه کوهستان افرایم برای تو تنگ است.»
\par 16 بنی یوسف گفتند: «کوهستان برای ما کفایت نمی کند، و جمیع کنعانیان که در زمین وادی ساکنند، ارابه های آهنین دارند، چه آنانی که دربیت‌شان و قصبه هایش، و چه آنانی که در وادی یزرعیل هستند.»
\par 17 پس یوشع به خاندان یوسف یعنی به افرایم و منسی خطاب کرده، گفت: «توقوم بزرگ هستی و قوت بسیار داری، برای تو یک قرعه نخواهد بود.بلکه کوهستان نیز از آن توخواهد بود، و اگر‌چه آن جنگل است آن راخواهی برید، و تمامی حدودش مال تو خواهدبود زیرا که کنعانیان را بیرون خواهی کرد، اگر‌چه ارابه های آهنین داشته، و زورآور باشند.»
\par 18 بلکه کوهستان نیز از آن توخواهد بود، و اگر‌چه آن جنگل است آن راخواهی برید، و تمامی حدودش مال تو خواهدبود زیرا که کنعانیان را بیرون خواهی کرد، اگر‌چه ارابه های آهنین داشته، و زورآور باشند.»
 
\chapter{18}

\par 1 و تمامی جماعت بنی‌اسرائیل درشیلوه جمع شده، خیمه اجتماع را درآنجا برپا داشتند، و زمین پیش روی ایشان مغلوب بود.
\par 2 و از بنی‌اسرائیل هفت سبط باقی ماندند، که هنوز ملک خود را تقسیم نکرده بودند.
\par 3 و یوشع به بنی‌اسرائیل گفت: «شما تا به کی کاهلی می‌ورزید و داخل نمی شوید تا در آن زمینی که یهوه خدای پدران شما، به شما داده است، تصرف نمایید؟
\par 4 سه نفر برای خود از هر سبطانتخاب کنید، تا ایشان را روانه نمایم، و برخاسته، از میان زمین گردش کرده، آن را برحسب ملکهای خود ثبت کنند، و نزد من خواهند برگشت.
\par 5 و آن را به هفت حصه تقسیم کنند، و یهودا به سمت جنوب به حدود خود خواهد ماند، و خاندان یوسف به سمت شمال به حدود خود خواهدماند.
\par 6 و شما زمین را به هفت حصه ثبت کرده، آن را نزد من اینجا بیاورید، و من برای شما در اینجادر حضور یهوه، خدای ما، قرعه خواهم‌انداخت.
\par 7 زیرا که لاویان در میان شما هیچ نصیب ندارند، چونکه کهانت خداوند نصیب ایشان است، و جادو روبین و نصف سبط منسی ملک خود را که موسی، بنده خداوند، به ایشان داده بود در آن طرف اردن به سمت شرقی گرفته‌اند.»
\par 8 پس آن مردان برخاسته، رفتند و یوشع آنانی را که برای ثبت کردن زمین می‌رفتند امر فرموده، گفت: «بروید و در زمین گردش کرده، آن را ثبت نمایید و نزد من برگردید، تا در اینجا در حضورخداوند در شیلوه برای شما قرعه اندازم.»
\par 9 پس آن مردان رفته، از میان زمین گذشتند و آن را به هفت حصه به حسب شهرهایش در طوماری ثبت نموده، نزد یوشع به اردو در شیلوه برگشتند.
\par 10 ویوشع به حضور خداوند در شیلوه برای ایشان قرعه انداخت، و در آنجا یوشع زمین را برای بنی‌اسرائیل برحسب فرقه های ایشان تقسیم نمود.
\par 11 و قرعه سبط بنی بنیامین برحسب قبایل ایشان برآمد، و حدود حصه ایشان در میان بنی یهودا و بنی یوسف افتاد.
\par 12 و حد ایشان به سمت شمال از اردن بود، و حد ایشان به طرف اریحا به سوی شمال برآمد، و از میان کوهستان به سوی مغرب بالا رفت، و انتهایش به صحرای بیت آون بود.
\par 13 و حد ایشان از آنجا تا لوزگذشت، یعنی به‌جانب لوز جنوبی که بیت ئیل باشد، و حد ایشان به سوی عطاروت ادار برجانب کوهی که به جنوب بیت حورون پایین است، رفت.
\par 14 و حدش کشیده شد و به‌جانب مغرب به سوی جنوب از کوهی که در مقابل بیت حورون جنوبی است گذشت، و انتهایش نزد قریه بعل بود که آن را قریت یعاریم می‌گویند و یکی ازشهرهای بنی یهوداست. این‌جانب غربی است.
\par 15 و جانب جنوبی از انتهای قریت یعاریم بود، واین حد به طرف مغرب می‌رفت و به سوی چشمه آبهای نفتوح برآمد.
\par 16 و این حد به انتهای کوهی که در مقابل دره ابن هنوم است که در جنوب وادی رفائیم باشد، برآمد، و به سوی دره هنوم به‌جانب جنوبی یبوسیان رفته، تا عین روجل رسید.
\par 17 و ازطرف شمال کشیده شده، به سوی عین شمس رفت، و به جلیلوت که در مقابل سر بالای ادمیم است برآمد، و به سنگ بوهن بن روبین به زیر آمد.
\par 18 و به‌جانب شمالی در مقابل عربه گذشته، به عربه به زیر آمد.
\par 19 و این حد به‌جانب بیت حجله به سوی شمال گذشت، و آخر این حدبه خلیج شمالی بحرالملح نزد انتهای جنوبی اردن بود. این حد جنوبی است.
\par 20 و به طرف مشرق، حد آن اردن بود و ملک بنی بنیامین به حسب حدودش به هر طرف و برحسب قبایل ایشان این بود.
\par 21 و این است شهرهای سبط بنی بنیامین برحسب قبایل ایشان اریحا و بیت حجله و عیمق قصیص.
\par 22 و بیت عربه و صمارایم و بیت ئیل.
\par 23 و عویم و فاره و عفرت.
\par 24 و کفر عمونی وعفنی و جابع، دوازده شهر با دهات آنها.
\par 25 وجبعون و رامه و بئیروت.
\par 26 و مصفه و کفیره وموصه.
\par 27 و راقم و یرفئیل و تراله.و صیله وآلف و یبوسی که اورشلیم باشد و جبعه و قریت، چهارده شهر با دهات آنها. این ملک بنی بنیامین به حسب قبایل ایشان بود.
\par 28 و صیله وآلف و یبوسی که اورشلیم باشد و جبعه و قریت، چهارده شهر با دهات آنها. این ملک بنی بنیامین به حسب قبایل ایشان بود.
 
\chapter{19}

\par 1 و قرعه دومین برای شمعون برآمد، یعنی برای سبط بنی شمعون برحسب قبایل ایشان، و ملک ایشان در میان ملک بنی یهودا بود.
\par 2 و اینها نصیب ایشان شد یعنی بئیر شبع و شبع و مولادا.
\par 3 و حصر شوعال و بالح و عاصم.
\par 4 و التولد و بتول و حرمه.
\par 5 صقلغ و بیت مرکبوت و حصر سوسه.
\par 6 و بیت لباعوت وشاروحن. سیزده شهر با دهات آنها.
\par 7 و عین ورمون و عاتر و عاشان، چهارده شهر با دهات آنها.
\par 8 تمامی دهاتی که در اطراف این شهرها تا بعلت بئیر رامه جنوبی بود. ملک سبط بنی شمعون برحسب قبایل ایشان این بود.
\par 9 و ملک بنی شمعون از میان قسمت بنی یهودا بود، زیرا قسمت بنی یهودا برای ایشان زیاد بود، پس بنی شمعون ملک خود را از میان ملک ایشان گرفتند.
\par 10 و قرعه سوم برای بنی زبولون برحسب قبایل ایشان برآمد، و حد ملک ایشان تا ساریدرسید. 
\par 11 و حد ایشان به طرف مغرب تا مرعله رفت و تا دباشه رسید و تا وادی که در مقابل یقنعام است، رسید.
\par 12 و از سارید به سمت مشرق به سوی مطلع آفتاب تا سرحد کسلوت تابور پیچید، و نزد دابره بیرون آمده، به یافیع رسید.
\par 13 و از آنجا به طرف مشرق تا جت حافر وتا عت قاصین گذشته، نزد رمون بیرون آمد و تانیعه کشیده شد.
\par 14 و این حد به طرف شمال تاحناتون آن را احاطه کرد، و آخرش نزد وادی یفتحئیل بود.
\par 15 و قطه و نهلال و شمرون و یداله و بیت لحم، دوازده شهر با دهات آنها.
\par 16 این ملک بنی زبولون برحسب قبایل ایشان بود، یعنی‌این شهرها با دهات آنها.
\par 17 و قرعه چهارم برای یساکار برآمد یعنی برای بنی یساکار برحسب قبایل ایشان.
\par 18 و حدایشان تا یزرعیل و کسلوت و شونم بود.
\par 19 وحفارایم و شیئون و اناحره.
\par 20 و ربیت و قشیون وآبص.
\par 21 و رمه و عین جنیم و عین حده و بیت فصیص.
\par 22 و این حد به تابور و شحصیمه وبیت شمس رسید، و آخر حد ایشان نزد اردن بود. یعنی شانزده شهر با دهات آنها.
\par 23 این ملک سبطبنی یساکار برحسب قبایل ایشان بود، یعنی شهرها با دهات آنها.
\par 24 و قرعه پنجم برای سبط بنی اشیر برحسب قبایل ایشان بیرون آمد.
\par 25 و حد ایشان حلقه وحلی و باطن و اکشاف.
\par 26 و الملک و عمعاد ومشال و به طرف مغرب به کرمل و شیحور لبنه رسید.
\par 27 و به سوی مشرق آفتاب به بیت داجون پیچیده، تا زبولون رسید، و به طرف شمال تاوادی یفتحئیل و بیت عامق و نعیئیل وبه طرف چپ نزد کابول بیرون آمد.
\par 28 و به حبرون ورحوب و حمون و قانه تا صیدون بزرگ.
\par 29 واین حد به سوی رامه به شهر حصاردار صور پیچید واین حد به سوی حوصه برگشت، و انتهایش نزددریا در دیار اکزیب بود.
\par 30 و عمه و عفیق ورحوب، و بیست و دو شهر با دهات آنها.
\par 31 ملک سبط بنی اشیر برحسب قبایل ایشان این بود، یعنی‌این شهرها با دهات آنها.
\par 32 و قرعه ششم برای بنی نفتالی بیرون آمد، یعنی برای بنی نفتالی برحسب قبایل ایشان.
\par 33 وحد ایشان از حالف از بلوطی که در صعنیم است و ادامی و ناقب و یبنئیل تا لقوم بود و آخرش نزداردن بود
\par 34 و حدش به سمت مغرب به سوی ازنوت تا بور پیچید، و از آنجا تا حقوق بیرون آمد، و به سمت جنوب به زبولون رسید و به سمت مغرب به اشیر رسید، و به سمت مشرق به یهودا نزد اردن.
\par 35 و شهرهای حصاردار صدیم وصیر و حمه و رقه و کناره.
\par 36 و ادامه و رامه وحاصور.
\par 37 و قادش و اذرعی و عین حاصور.
\par 38 ویرون و مجدلئیل و حوریم و بیت عناه و بیت شمس، نوزده شهر با دهات آنها.
\par 39 ملک سبطبنی نفتالی برحسب قبایل ایشان این بود، یعنی شهرها با دهات آنها.
\par 40 و قرعه هفتم برای سبط بنی دان برحسب قبایل ایشان بیرون آمد.
\par 41 و حد ملک ایشان صرعه و اشتئول و عیر شمس بود.
\par 42 و شعلبین وایلون و یتله.
\par 43 و ایلون و تمنه و عقرون.
\par 44 والتقیه و جبتون و بعله.
\par 45 و یهود و بنی برق و جت رمون.
\par 46 و میاه یرقون و رقون با سر حدی که درمقابل یافا است.
\par 47 و حد بنی دان از طرف ایشان بیرون رفت، زیر که بنی دان برآمده، با لشم جنگ کردند و آن را گرفته، به دم شمشیر زدند. ومتصرف شده، در آن سکونت گرفتند. پس لشم رادان نامیدند، موافق اسم دان که پدر ایشان بود.
\par 48 این است ملک سبط بنی دان برحسب قبایل ایشان، یعنی‌این شهرها با دهات آنها.
\par 49 و چون از تقسیم کردن زمین برحسب حدودش فارغ شدند، بنی‌اسرائیل ملکی را درمیان خود به یوشع بن نون دادند.
\par 50 برحسب فرمان خداوند شهری که او خواست، یعنی تمنه سارح را در کوهستان افرایم به او دادند، پس شهررا بنا کرده، در آن ساکن شد.این است ملکهایی که العازار کاهن با یوشع بن نون و روسای آبای اسباط بنی‌اسرائیل درشیلوه به حضور خداوند نزد در خیمه اجتماع به قرعه تقسیم کردند. پس از تقسیم نمودن زمین فارغ شدند.
\par 51 این است ملکهایی که العازار کاهن با یوشع بن نون و روسای آبای اسباط بنی‌اسرائیل درشیلوه به حضور خداوند نزد در خیمه اجتماع به قرعه تقسیم کردند. پس از تقسیم نمودن زمین فارغ شدند.
 
\chapter{20}

\par 1 و خداوند یوشع را خطاب کرده، گفت:
\par 2 «بنی‌اسرائیل را خطاب کرده، بگو: شهرهای ملجایی را که درباره آنها به واسطه موسی به شما سخن گفتم، برای خود معین سازید
\par 3 تا قاتلی که کسی را سهو و ندانسته کشته باشدبه آنها فرار کند، و آنها برای شما از ولی مقتول ملجا باشد.
\par 4 و او به یکی از این شهرها فرار کرده، و به مدخل دروازه شهر ایستاده، به گوش مشایخ شهر ماجرای خود را بیان کند، و ایشان او را نزدخود به شهر درآورده، مکانی به او بدهند تا باایشان ساکن شود.
\par 5 و اگر ولی مقتول او را تعاقب کند، قاتل را به‌دست او نسپارند، زیرا که همسایه خود را از نادانستگی کشته، و او را پیش از آن دشمن نداشته بود.
\par 6 و در آن شهر تا وقتی که به جهت محاکمه به حضور جماعت حاضر شود تاوفات رئیس کهنه که در آن ایام می‌باشد توقف نماید، و بعد از آن قاتل برگشته، به شهر و به خانه خود یعنی به شهری که از آن فرار کرده بود، داخل شود.
\par 7 پس قادش را در جلیل در کوهستان نفتالی وشکیم را در کوهستان افرایم و قریه اربع را که درحبرون باشد در کوهستان یهودا، تقدیس نمودند.
\par 8 و از آن طرف اردن به سمت مشرق اریحا باصررا در صحرا در بیابان از سبط روبین و راموت را درجلعاد از سبط جاد و جولان را در باشان از سبطمنسی تعیین نمودند.اینهاست شهرهایی که برای تمامی بنی‌اسرائیل و برای غریبی که در میان ایشان ماوا گزیند معین شده بود، تا هر‌که کسی راسهو کشته باشد به آنجا فرار کند، و به‌دست ولی مقتول کشته نشود تا وقتی که به حضور جماعت حاضر شود.
\par 9 اینهاست شهرهایی که برای تمامی بنی‌اسرائیل و برای غریبی که در میان ایشان ماوا گزیند معین شده بود، تا هر‌که کسی راسهو کشته باشد به آنجا فرار کند، و به‌دست ولی مقتول کشته نشود تا وقتی که به حضور جماعت حاضر شود.
 
\chapter{21}

\par 1 آنگاه روسای آبای لاویان نزد العازرکاهن و نزد یوشع بن نون و نزد روسای آبای اسباط بنی‌اسرائیل آمدند.
\par 2 و ایشان را درشیلوه در زمین کنعان مخاطب ساخته، گفتند که «خداوند به واسطه موسی‌امر فرموده است که شهرها برای سکونت و حوالی آنها به جهت بهایم ما، به ما داده شود.»
\par 3 پس بنی‌اسرائیل برحسب فرمان خداوند این شهرها را با حوالی آنها از ملک خود به لاویان دادند.
\par 4 و قرعه برای قبایل قهاتیان بیرون آمد، وبرای پسران هارون کاهن که از‌جمله لاویان بودند سیزده شهر از سبط یهودا، و از سبط شمعون و ازسبط بنیامین به قرعه رسید.
\par 5 و برای بقیه پسران قهات، ده شهر از قبایل سبط افرایم و از سبط دان و از نصف سبط منسی به قرعه رسید.
\par 6 و برای پسران جرشون سیزده شهر از قبایل سبط یساکار و از سبط اشیر و از سبط نفتالی و ازنصف سبط منسی در باشان به قرعه رسید.
\par 7 و برای پسران مراری برحسب قبایل ایشان دوازده شهر از سبط روبین و از سبط جاد و از سبطزبولون رسید.
\par 8 و بنی‌اسرائیل، این شهرها و حوالی آنها را به لاویان به قرعه دادند، چنانکه خداوند به واسطه موسی‌امر فرموده بود.
\par 9 و از سبط بنی یهودا و از سبط بنی شمعون این شهرها را که به نامها ذکر می‌شود، دادند.
\par 10 واینها به پسران هارون که از قبایل قهاتیان ازبنی لاوی بودند رسید، زیرا که قرعه اول از ایشان بود.
\par 11 پس قریه اربع پدر عناق که حبرون باشددر کوهستان یهودا با حوالی که در اطراف آن بود، به ایشان دادند.
\par 12 لیکن مزرعه های شهر و دهات آن را به کالیب بن یفنه برای ملکیت دادند.
\par 13 و به پسران هارون کاهن، حبرون را که شهرملجای قاتلان است با حوالی آن، و لبنه را باحوالی آن دادند.
\par 14 و یتیر را با نواحی آن واشتموع را با نواحی آن.
\par 15 و حولون را با نواحی آن و دبیر را با نواحی آن.
\par 16 و عین را با نواحی آن و یطه را با نواحی آن و بیت شمس را با نواحی آن، یعنی از این دو سبط نه شهر را.
\par 17 و از سبطبنیامین جبعون را با نواحی آن و جبع را با نواحی آن.
\par 18 عناتوت را با نواحی آن و علمون را بانواحی آن، یعنی چهار شهر دادند.
\par 19 تمامی شهرهای پسران هارون کهنه سیزده شهر با نواحی آنها بود.
\par 20 و اما قبایل بنی قهات لاویان، یعنی بقیه بنی قهات شهرهای قرعه ایشان از سبط افرایم بود.
\par 21 پس شکیم را در کوهستان افرایم که شهرملجای قاتلان است با نواحی آن و جازر را بانواحی آن به ایشان دادند.
\par 22 و قبصایم را بانواحی آن و بیت حورون را با نواحی آن، یعنی چهار شهر.
\par 23 و از سبط دان التقی را با نواحی آن و جبتون را با نواحی آن.
\par 24 و ایلون را با نواحی آن و جت رمون را با نواحی آن، یعنی چهار شهر.
\par 25 و از نصف سبط منسی تعنک را با نواحی آن وجت رمون را با نواحی آن، یعنی دو شهر دادند.
\par 26 تمامی شهرهای قبایل بقیه بنی قهات با نواحی آنها ده بود.
\par 27 و به بنی جرشون که از قبایل لاویان بودند ازنصف سبط منسی جولان را در باشان که شهرملجای قاتلان است با نواحی آن و بعشتره را بانواحی آن، یعنی دو شهر دادند.
\par 28 و از سبطیساکار قشیون را با نواحی آن و دابره را با نواحی آن.
\par 29 و یرموت را با نواحی آن و عین جنیم را بانواحی آن، یعنی چهار شهر.
\par 30 و از سبط اشیرمشال را با نواحی آن و عبدون را با نواحی آن.
\par 31 و حلقات را با نواحی آن و رحوب را با نواحی آن، یعنی چهار شهر.
\par 32 و از سبط نفتالی قادش رادر جلیل که شهر ملجای قاتلان است با نواحی آن و حموت دور را با نواحی آن و قرتان را، یعنی سه شهر دادند.
\par 33 و تمامی شهرهای جرشونیان برحسب قبایل ایشان سیزده شهر بود با نواحی آنها.
\par 34 و به قبایل بنی مراری که از لاویان باقی‌مانده بودند، از سبط زبولون یقنعام را با نواحی آن و قرته را با نواحی آن.
\par 35 و دمنه را با نواحی آن و نحلال را با نواحی، یعنی چهار شهر.
\par 36 و از سبطروبین، باصر را با نواحی آن و یهصه را با نواحی آن.
\par 37 و قدیموت را با نواحی آن و میفعه را بانواحی آن، یعنی چهار شهر.
\par 38 و از سبط جادراموت را در جلعاد که شهر ملجای قاتلان است بانواحی آن و محنایم را با نواحی آن.
\par 39 و حشبون را با نواحی آن و یعزیر را با نواحی آن؛ همه این شهرها چهار می‌باشد.
\par 40 همه اینها شهرهای بنی مراری برحسب قبایل ایشان بود، یعنی بقیه قبایل لاویان و قرعه ایشان دوازده شهر بود.
\par 41 و جمیع شهرهای لاویان در میان ملک بنی‌اسرائیل چهل و هشت شهر با نواحی آنها بود.
\par 42 این شهرها هر یکی با نواحی آن به هر طرفش بود، و برای همه این شهرها چنین بود.
\par 43 پس خداوند تمامی زمین را که برای پدران ایشان قسم خورده بود که به ایشان بدهد به اسرائیل داد، و آن را به تصرف آورده، در آن ساکن شدند.
\par 44 و خداوند ایشان را از هر طرف آرامی داد چنانکه به پدران ایشان قسم خورده بود، واحدی از دشمنان ایشان نتوانست با ایشان مقاومت نماید، زیرا که خداوند جمیع دشمنان ایشان را به‌دست ایشان سپرده بود.و از جمیع سخنان نیکویی که خداوند به خاندان اسرائیل گفته بود، سخنی به زمین نیفتاد بلکه همه واقع شد.
\par 45 و از جمیع سخنان نیکویی که خداوند به خاندان اسرائیل گفته بود، سخنی به زمین نیفتاد بلکه همه واقع شد.
 
\chapter{22}

\par 1 آنگاه یوشع روبینیان و جادیان و نصف سبط منسی را خوانده،
\par 2 به ایشان گفت: «شما هر‌چه موسی بنده خداوند به شما امرفرموده بود، نگاه داشتید، و کلام مرا از هر‌چه به شما امر فرموده‌ام، اطاعت نمودید.
\par 3 و برادران خود را در این ایام طویل تا امروز ترک نکرده، وصیتی را که یهوه، خدای شما، امر فرموده بود، نگاه داشته‌اید.
\par 4 و الان یهوه خدای شما به برادران شما آرامی داده است، چنانکه به ایشان گفته بود. پس حال به خیمه های خود و به زمین ملکیت خود که موسی بنده خداوند از آن طرف اردن به شما داده است بازگشته، بروید. 
\par 5 امابدقت متوجه شده، امر و شریعتی را که موسی بنده خداوند به شما امر فرموده است به‌جاآورید، تا یهوه، خدای خود، را محبت نموده، به تمامی طریقهای او سلوک نمایید، و اوامر او رانگاه داشته، به او بچسبید و او را به تمامی دل وتمامی جان خود عبادت نمایید.»
\par 6 پس یوشع ایشان را برکت داده، روانه نمود و به خیمه های خود رفتند.
\par 7 و به نصف سبط منسی، موسی ملک درباشان داده بود، و به نصف دیگر، یوشع در این طرف اردن به سمت مغرب در میان برادران ایشان ملک داد. و هنگامی که یوشع ایشان را به خیمه های ایشان روانه می‌کرد، ایشان را برکت داد.
\par 8 و ایشان را مخاطب ساخته، گفت: «با دولت بسیار و با مواشی بی‌شمار، با نقره و طلا و مس وآهن و لباس فراوان به خیمه های خود برگردید، وغنیمت دشمنان خود را با برادران خویش تقسیم نمایید.»
\par 9 پس بنی روبین و بنی جاد و نصف سبطمنسی از نزد بنی‌اسرائیل از شیلوه که در زمین کنعان است برگشته، روانه شدند تا به زمین جلعاد، و به زمین ملک خود که آن را به واسطه موسی برحسب فرمان خداوند به تصرف آورده بودند، بروند.
\par 10 و چون به حوالی اردن که در زمین کنعان است رسیدند، بنی روبین و بنی جاد و نصف سبطمنسی در آنجا به کنار اردن مذبحی بنا نمودند، یعنی مذبح عظیم المنظری.
\par 11 و بنی‌اسرائیل خبر این را شنیدند که اینک بنی روبین و بنی جاد ونصف سبط منسی، به مقابل زمین کنعان، درحوالی اردن، بر کناری که از آن بنی‌اسرائیل است، مذبحی بنا کرده‌اند.
\par 12 پس چون بنی‌اسرائیل این را شنیدند، تمامی جماعت بنی‌اسرائیل در شیلوه جمع شدند تا برای مقاتله ایشان برآیند.
\par 13 و بنی‌اسرائیل فینحاس بن العازار کاهن رانزد بنی روبین و بنی جاد و نصف سبط منسی به زمین جلعاد فرستادند.
\par 14 و با او ده رئیس، یعنی یک رئیس از هر خاندان آبای از جمیع اسباطاسرائیل را که هر یکی از ایشان رئیس خاندان آبای ایشان از قبایل بنی‌اسرائیل بودند.
\par 15 پس ایشان نزد بنی روبین و بنی جاد و نصف سبطمنسی به زمین جلعاد آمدند و ایشان را مخاطب ساخته، گفتند:
\par 16 «تمامی جماعت خداوند چنین می‌گویند: این چه فتنه است که به خدای اسرائیل انگیخته‌اید که امروز از متابعت خداوندبرگشته‌اید و برای خود مذبحی ساخته، امروز ازخداوند متمرد شده‌اید؟
\par 17 آیا گناه فغور برای ماکم است که تا امروز خود را از آن طاهرنساخته‌ایم، اگر‌چه وبا در جماعت خداوندعارض شد.
\par 18 شما امروز از متابعت خداوندبرگشته‌اید و واقع خواهد شد چون شما امروز ازخداوند متمرد شده‌اید که او فردا بر تمامی جماعت اسرائیل غضب خواهد نمود.
\par 19 و لیکن اگر زمین ملکیت شما نجس است، پس به زمین ملکیت خداوند که مسکن خداوند در آن ساکن است عبور نمایید، و در میان ما ملک بگیرید و ازخداوند متمرد نشوید، و از ما نیز متمرد نشوید، در این که مذبحی برای خود سوای مذبح یهوه خدای ما بنا کنید.
\par 20 آیا عخان بن زارح درباره چیز حرام خیانت نورزید؟ پس بر تمامی جماعت اسرائیل غضب آمد، و آن شخص درگناه خود تنها هلاک نشد.»
\par 21 آنگه بنی روبین و بنی جاد و نصف سبطمنسی در جواب روسای قبایل اسرائیل گفتند:
\par 22 «یهوه خدای خدایان! یهوه خدای خدایان! اومی داند و اسرائیل خواهند دانست اگر این کار ازراه تمرد یا از راه خیانت بر خداوند بوده باشد، امروز ما را خلاصی مده،
\par 23 که برای خودمذبحی ساخته‌ایم تا از متابعت خداوند برگشته، قربانی سوختنی و هدیه آردی بر آن بگذرانیم، وذبایح سلامتی بر آن بنماییم؛ خود خداوندبازخواست بنماید.
\par 24 بلکه این کار را از راه احتیاط و هوشیاری کرده‌ایم، زیرا گفتیم شاید دروقت آینده پسران شما به پسران ما بگویند شما رابا یهوه خدای اسرائیل چه علاقه است؟
\par 25 چونکه خداوند اردن را در میان ما و شما‌ای بنی روبین و بنی جاد حد گذارده است. پس شما رادر خداوند بهره‌ای نیست و پسران شما پسران مارا از ترس خداوند باز خواهند داشت.
\par 26 پس گفتیم برای ساختن مذبحی به جهت خود تدارک ببینیم، نه برای قربانی سوختنی و نه برای ذبیحه،
\par 27 بلکه تا در میان ما و شما و در میان نسلهای ما بعد از ما شاهد باشد تا عبادت خداوند را به حضور او با قربانی های سوختنی و ذبایح سلامتی خود به‌جا آوریم، تا در زمان آینده پسران شما به پسران ما نگویند که شما را در خداوند هیچ بهره‌ای نیست.
\par 28 پس گفتیم اگر در زمان آینده به ما و به نسلهای ما چنین بگویند، آنگاه ما خواهیم گفت، نمونه مذبح خداوند را ببینید که پدران ماساختند نه برای قربانی سوختنی و نه برای ذبیحه، بلکه تا در میان ما و شما شاهد باشد.
\par 29 حاشا از ما که از خداوند متمرد شده، امروز ازمتابعت خداوند برگردیم، و مذبحی برای قربانی سوختنی و هدیه آردی و ذبیحه سوای مذبح یهوه، خدای ما که پیش روی مسکن اوست، بسازیم.»
\par 30 و چون فینحاس کاهن و سروران جماعت و روسای قبایل اسرائیل که با وی بودند سخنی راکه بنی روبین و بنی جاد و بنی منسی گفته بودند، شنیدند، در نظر ایشان پسند آمد.
\par 31 و فینحاس بن العازار کاهن به بنی روبین و بنی جاد و بنی منسی گفت: «امروز دانستیم که خداوند در میان ماست، چونکه این خیانت را بر خداوند نورزیده‌اید، پس الان بنی‌اسرائیل را از دست خداوند خلاصی دادید.»
\par 32 پس فینحاس بن العازار کاهن و سروران ازنزد بنی روبین و بنی جاد از زمین جلعاد به زمین کنعان، نزد اسرائیل برگشته، این خبر به ایشان رسانیدند.
\par 33 و این کار به نظر بنی‌اسرائیل پسندآمد و بنی‌اسرائیل خدا را متبارک خواندند، ودرباره برآمدن برای مقاتله ایشان تا زمینی را که بنی روبین و بنی جاد در آن ساکن بودند خراب نمایند، دیگر سخن نگفتند.و بنی روبین وبنی جاد آن مذبح را عید نامیدند، زیرا که آن در میان ما شاهد است که یهوه خداست.
\par 34 و بنی روبین وبنی جاد آن مذبح را عید نامیدند، زیرا که آن در میان ما شاهد است که یهوه خداست.
 
\chapter{23}

\par 1 و واقع شد بعد از روزهای بسیار چون خداوند اسرائیل را از جمیع دشمنان ایشان از هر طرف آرامی داده بود، و یوشع پیر وسالخورده شده بود.
\par 2 که یوشع جمیع اسرائیل رابا مشایخ و روسا و داوران و ناظران ایشان طلبیده، به ایشان گفت: «من پیر و سالخورده شده‌ام.
\par 3 وشما هرآنچه یهوه، خدای شما به همه این طوایف به‌خاطر شما کرده است، دیده‌اید، زیرایهوه، خدای شما اوست که برای شما جنگ کرده است.
\par 4 اینک این طوایف را که باقی‌مانده‌اند ازاردن و جمیع طوایف را که مغلوب ساخته‌ام تادریای بزرگ، به سمت مغرب آفتاب برای شما به قرعه تقسیم کرده‌ام تا میراث اسباط شما باشند.
\par 5 و یهوه، خدای شما اوست که ایشان را از حضورشما رانده، ایشان را از پیش روی شما بیرون می‌کند، و شما زمین ایشان را در تصرف خواهیدآورد، چنانکه یهوه خدای شما به شما گفته است.
\par 6 پس بسیار قوی باشید و متوجه شده، هر‌چه درسفر تورات موسی مکتوب است نگاه دارید و به طرف چپ یا راست از آن تجاوز منمایید.
\par 7 تا به این طوایفی که در میان شما باقی‌مانده‌اند داخل نشوید، و نامهای خدایان ایشان را ذکر ننمایید، وقسم نخورید و آنها را عبادت منمایید و سجده نکنید.
\par 8 بلکه به یهوه، خدای خود بچسبیدچنانکه تا امروز کرده‌اید.
\par 9 زیرا خداوند طوایف بزرگ و زورآور را از پیش روی شما بیرون کرده است، و اما با شما کسی را تا امروز یارای مقاومت نبوده است.
\par 10 یک نفر از شما هزار را تعاقب خواهد نمود زیرا که یهوه، خدای شما، اوست که برای شما جنگ می‌کند، چنانکه به شما گفته است.
\par 11 پس بسیار متوجه شده، یهوه خدای خود را محبت نمایید.
\par 12 و اما اگر برگشته، با بقیه این طوایفی که در میان شما مانده‌اند بچسبید و باایشان مصاهرت نمایید، و به ایشان درآیید وایشان به شما درآیند،
\par 13 یقین بدانید که یهوه خدای شما این طوایف را از حضور شما دیگربیرون نخواهد کرد، بلکه برای شما دام و تله وبرای پهلوهای شما تازیانه و در چشمان شما خارخواهند بود، تا وقتی که از این زمین نیکو که یهوه خدای شما، به شما داده است، هلاک شوید.
\par 14 و اینک من امروز به طریق اهل تمامی زمین می‌روم، و به تمامی دل و به تمامی جان خودمی دانید که یک چیز از تمام چیزهای نیکو که یهوه، خدای شما درباره شما گفته است به زمین نیفتاده، بلکه همه‌اش واقع شده است، و یک حرف از آن به زمین نیفتاده.
\par 15 و چنین واقع خواهد شد که چنانکه همه‌چیزهای نیکو که یهوه، خدای شما به شما گفته بود برای شما واقع شده است، همچنان خداوند همه‌چیزهای بد رابر شما عارض خواهد گردانید، تا شما را از این زمین نیکو که یهوه، خدای شما به شما داده است، هلاک سازد.اگر از عهد یهوه، خدای خود، که به شما امر فرموده است، تجاوز نمایید، و رفته، خدایان دیگر را عبادت نمایید، و آنها را سجده کنید، آنگاه غضب خداوند بر شما افروخته خواهد شد، و از این زمین نیکو که به شما داده است به زودی هلاک خواهید شد.
\par 16 اگر از عهد یهوه، خدای خود، که به شما امر فرموده است، تجاوز نمایید، و رفته، خدایان دیگر را عبادت نمایید، و آنها را سجده کنید، آنگاه غضب خداوند بر شما افروخته خواهد شد، و از این زمین نیکو که به شما داده است به زودی هلاک خواهید شد.
 
\chapter{24}

\par 1 و یوشع تمامی اسباط اسرائیل را درشکیم جمع کرد، و مشایخ اسرائیل وروسا و داوران و ناظران ایشان را طلبیده، به حضور خدا حاضر شدند.
\par 2 و یوشع به تمامی قوم گفت که «یهوه خدای اسرائیل چنین می‌گوید که پدران شما، یعنی طارح پدر ابراهیم و پدر ناحور، در زمان قدیم به آن طرف نهر ساکن بودند، وخدایان غیر را عبادت می‌نمودند.
\par 3 و پدر شماابراهیم را از آن طرف نهر گرفته، در تمامی زمین کنعان گردانیدم، و ذریت او را زیاد کردم و اسحاق را به او دادم.
\par 4 و یعقوب و عیسو را به اسحاق دادم، و کوه سعیر را به عیسو دادم تا ملکیت اوبشود، و یعقوب و پسرانش به مصر فرود شدند.
\par 5 و موسی و هارون را فرستاده مصر را به آنچه دروسط آن کردم، مبتلا ساختم؛ پس شما را از آن بیرون آوردم.
\par 6 و چون پدران شما را از مصر بیرون آوردم وبه دریا رسیدید، مصریان با ارابه‌ها و سواران، پدران شما را تا بحر قلزم تعاقب نمودند.
\par 7 و چون نزد خداوند فریاد کردند، او در میان شما ومصریان تاریکی گذارد، و دریا را برایشان آورده، ایشان را پوشانید، و چشمان شما آنچه را در مصرکردم دید، پس روزهای بسیار در بیابان ساکن می‌بودید.
\par 8 پس شما را در زمین اموریانی که به آن طرف اردن ساکن بودند آوردم، و با شما جنگ کردند، و ایشان را به‌دست شما تسلیم نمودم، وزمین ایشان را در تصرف آوردید، و ایشان را ازحضور شما هلاک ساختم.
\par 9 و بالاق بن صفور ملک موآب برخاسته، با اسرائیل جنگ کرد وفرستاده، بلعام بن بعور را طلبید تا شما را لعنت کند.
\par 10 و نخواستم که بلعام را بشنوم لهذا شما رابرکت همی داد و شما را از دست او رهانیدم.
\par 11 واز اردن عبور کرده، به اریحا رسیدید، و مردان اریحا یعنی اموریان و فرزیان و کنعانیان و حتیان و جرجاشیان و حویان و یبوسیان با شما جنگ کردند، و ایشان را به‌دست شما تسلیم نمودم.
\par 12 و زنبور را پیش شما فرستاده، ایشان، یعنی دوپادشاه اموریان را از حضور شما براندم، نه به شمشیر و نه به کمان شما.
\par 13 و زمینی که در آن زحمت نکشیدید، و شهرهایی را که بنا ننمودید، به شما دادم که در آنها ساکن می‌باشید و ازتاکستانها و باغات زیتون که نکاشتید، می‌خورید.
\par 14 پس الان از یهوه بترسید، و او را به خلوص و راستی عبادت نمایید، و خدایانی را که پدران شما به آن طرف نهر و در مصر عبادت نمودند ازخود دور کرده، یهوه را عبادت نمایید.
\par 15 و اگردر نظر شما پسند نیاید که یهوه را عبادت نمایید، پس امروز برای خود اختیار کنید که را عبادت خواهید نمود، خواه خدایانی را که پدران شما که به آن طرف نهر بودند عبادت نمودند، خواه خدایان اموریانی را که شما در زمین ایشان ساکنید، و اما من و خاندان من، یهوه را عبادت خواهیم نمود.» 
\par 16 آنگاه قوم در جواب گفتند: «حاشا از ما که یهوه را ترک کرده، خدایان غیر را عبادت نماییم.
\par 17 زیرا که یهوه، خدای ما، اوست که ما و پدران مارا از زمین مصر از خانه بندگی بیرون آورد، و این آیات بزرگ را در نظر ما نمود، و ما را در تمامی راه که رفتیم و در تمامی طوایفی که از میان ایشان گذشتیم، نگاه داشت.
\par 18 و یهوه تمامی طوایف، یعنی اموریانی را که در این زمین ساکن بودند ازپیش روی ما بیرون کرد، پس ما نیز یهوه را عبادت خواهیم نمود، زیرا که او خدای ماست.»
\par 19 پس یوشع به قوم گفت: «نمی توانید یهوه راعبادت کنید زیرا که او خدای قدوس است و اوخدای غیور است که عصیان و گناهان شما رانخواهد آمرزید.
\par 20 اگر یهوه را ترک کرده، خدایان غیر را عبادت نمایید، آنگاه او خواهدبرگشت و به شما ضرر رسانیده، بعد از آنکه به شما احسان نموده است، شما را هلاک خواهدکرد.»
\par 21 قوم به یوشع گفتند: «نی بلکه یهوه راعبادت خواهیم نمود.»
\par 22 یوشع به قوم گفت: «شما برخود شاهد هستید که یهوه را برای خوداختیار نموده‌اید تا او را عبادت کنید.» گفتند: «شاهد هستیم.»
\par 23 (گفت ): «پس الان خدایان غیررا که در میان شما هستند دور کنید، و دلهای خودرا به یهوه، خدای اسرائیل، مایل سازید.»
\par 24 قوم به یوشع گفتند: «یهوه خدای خود را عبادت خواهیم نمود و آواز او را اطاعت خواهیم کرد.»
\par 25 پس در آن روز یوشع با قوم عهد بست و برای ایشان فریضه و شریعتی در شکیم قرار داد.
\par 26 ویوشع این سخنان را در کتاب تورات خدا نوشت و سنگی بزرگ گرفته، آن را در آنجا زیر درخت بلوطی که نزد قدس خداوند بود برپا داشت.
\par 27 ویوشع به تمامی قوم گفت: «اینک این سنگ برای ما شاهد است، زیرا که تمامی سخنان خداوند را که به ما گفت، شنیده است؛ پس برای شما شاهدخواهد بود، مبادا خدای خود را انکار نمایید.»
\par 28 پس یوشع، قوم یعنی هر کس را به ملک خودروانه نمود.
\par 29 و بعد از این امور واقع شد که یوشع بن نون، بنده خداوند، چون صد و ده ساله بود، مرد.
\par 30 و او را در حدود ملک خودش در تمنه سارح که در کوهستان افرایم به طرف شمال کوه جاعش است، دفن کردند.
\par 31 و اسرائیل در همه ایام یوشع و همه روزهای مشایخی که بعد از یوشع زنده ماندند وتمام عملی که خداوند برای اسرائیل کرده بوددانستند، خداوند را عبادت نمودند.واستخوانهای یوسف را که بنی‌اسرائیل از مصرآورده بودند در شکیم، در حصه زمینی که یعقوب از بنی حمور، پدر شکیم به صد قسیطه خریده بود، دفن کردند، و آن ملک بنی یوسف شد.
\par 32 واستخوانهای یوسف را که بنی‌اسرائیل از مصرآورده بودند در شکیم، در حصه زمینی که یعقوب از بنی حمور، پدر شکیم به صد قسیطه خریده بود، دفن کردند، و آن ملک بنی یوسف شد.


\end{document}