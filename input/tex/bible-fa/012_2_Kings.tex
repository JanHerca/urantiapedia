\begin{document}

\title{دوم پادشاهان}

 
\chapter{1}

\par 1 عاصی شدند.
\par 2 و اخزیا از پنجره بالاخانه خود که در سامره بود افتاده، بیمار شد. پس رسولان را روانه نموده، به ایشان گفت: «نزد بعل زبوب، خدای عقرون رفته، بپرسید که آیا از این مرض شفا خواهم یافت؟»
\par 3 و فرشته خداوند به ایلیای تشبی گفت: «برخیز و به ملاقات رسولان پادشاه سامره برآمده، به ایشان بگو که آیا از این جهت که خدایی دراسرائیل نیست، شما برای سوال نمودن از بعل زبوب، خدای عقرون می‌روید؟
\par 4 پس خداوند چنین می‌گوید: ازبستری که بر آن برآمدی، فرود نخواهی شد بلکه البته خواهی مرد.»
\par 5 و ایلیا رفت و رسولان نزد وی برگشتند و او به ایشان گفت: «چرا برگشتید؟»
\par 6 ایشان در جواب وی گفتند: «شخصی به ملاقات ما برآمده، ما را گفت: بروید و نزد پادشاهی که شما را فرستاده است، مراجعت کرده، او راگویید: خداوند چنین می‌فرماید: آیا از این جهت که خدایی در اسرائیل نیست، تو برای سوال نمودن از بعل زبوب، خدای عقرون می‌فرستی؟ بنابراین از بستری که به آن برآمدی، فرود نخواهی شد بلکه البته خواهی مرد.»
\par 7 او به ایشان گفت: «هیات شخصی که به ملاقات شما برآمد و این سخنان را به شما گفت چگونه بود؟»
\par 8 ایشان او را جواب دادند: «مرد موی دار بود و کمربند چرمی بر کمرش بسته بود.» او گفت: «ایلیای تشبی است.»
\par 9 آنگاه سردار پنجاهه را با پنجاه نفرش نزد وی فرستاد و او نزد وی آمد در حالتی که او بر قله کوه نشسته بود وبه وی عرض کرد که «ای مرد خدا، پادشاه می‌گوید به زیر آی؟»
\par 10 ایلیا در جواب سردار پنجاهه گفت: «اگر من مرد خدا هستم، آتش از آسمان نازل شده، تو را و پنجاه نفرت را بسوزاند.» پس آتش از آسمان نازل شده، او را وپنجاه نفرش را بسوخت.
\par 11 و باز سردار پنجاهه دیگر را با پنجاه نفرش نزد وی فرستاد و او وی را خطاب کرده، گفت: «ای مرد خدا، پادشاه چنین می‌فرماید که به زودی به زیر آی؟»
\par 12 ایلیا در جواب ایشان گفت: «اگر من مرد خدا هستم آتش ازآسمان نازل شده، تو را و پنجاه نفرت را بسوزاند.» پس آتش خدا از آسمان نازل شده، او را و پنجاه نفرش رابسوخت.
\par 13 پس سردار پنجاهه سوم را با پنجاه نفرش فرستاد و سردار پنجاهه سوم آمده، نزد ایلیا به زانو درآمد و از اوالتماس نموده، گفت که «ای مرد خدا، تمنا اینکه جان من و جان این پنجاه نفر بندگانت در نظر تو عزیز باشد.
\par 14 اینک آتش از آسمان نازل شده، آن دو سردار پنجاهه اول را با پنجاهه های ایشان سوزانید، اما الان جان من در نظر تو عزیز باشد.»
\par 15 و فرشته خداوند به ایلیا گفت: «همراه او به زیر آی و از او مترس.» پس برخاسته، همراه وی نزد پادشاه فرود شد.
\par 16 و وی را گفت: «خداوند چنین می‌گوید: چونکه رسولان فرستادی تا از بعل زبوب، خدای عقرون سوال نماید، آیا از این سبب بود که در اسرائیل خدایی نبود که از کلام او سوال نمایی؟ بنابراین از بستری که به آن برآمدی، فرود نخواهی شد البته خواهی مرد.»
\par 17 پس او موافق کلامی که خداوند به ایلیا گفته بود، مرد و یهورام در سال دوم یهورام بن یهوشافاط، پادشاه یهودا در جایش پادشاه شد، زیرا که او را پسری نبود.و بقیه اعمال اخزیا که کرد، آیا در کتاب تواریخ ایام پادشاهان اسرائیل مکتوب نیست؟
\par 18 و بقیه اعمال اخزیا که کرد، آیا در کتاب تواریخ ایام پادشاهان اسرائیل مکتوب نیست؟
 
\chapter{2}

\par 1 و چون خداوند اراده نمود که ایلیا را درگردباد به آسمان بالا برد، واقع شد که ایلیا والیشع از جلجال روانه شدند.
\par 2 و ایلیا به الیشع گفت: «در اینجا بمان، زیرا خداوند مرا به بیت ئیل فرستاده است.» الیشع گفت: «به حیات یهوه وحیات خودت قسم که تو را ترک نکنم.» پس به بیت ئیل رفتند.
\par 3 و پسران انبیایی که در بیت ئیل بودند، نزد الیشع بیرون آمده، وی را گفتند: «آیامی دانی که امروز خداوند آقای تو را از فوق سرتو خواهد برداشت.» او گفت: «من هم می‌دانم؛ خاموش باشید.»
\par 4 و ایلیا به او گفت: «ای الیشع در اینجا بمان زیرا خداوند مرا به اریحا فرستاده است.» او گفت: «به حیات یهوه و به حیات خودت قسم که تو راترک نکنم.» پس به اریحا آمدند.
\par 5 و پسران انبیایی که در اریحا بودند، نزد الیشع آمده، وی راگفتند: «آیا می‌دانی که امروز خداوند، آقای تو رااز فوق سر تو برمی دارد؟» او گفت: «من هم می‌دانم؛ خاموش باشید.»
\par 6 و ایلیا وی را گفت: «در اینجا بمان زیراخداوند مرا به اردن فرستاده است.» او گفت: «به حیات یهوه و به حیات خودت قسم که تو را ترک نکنم.» پس هردوی ایشان روانه شدند.
\par 7 و پنجاه نفر از پسران انبیا رفته، در مقابل ایشان از دورایستادند و ایشان نزد اردن ایستاده بودند.
\par 8 پس ایلیا ردای خویش را گرفت و آن را پیچیده، آب را زد که به این طرف و آن طرف شکافته شد وهردوی ایشان بر خشکی عبور نمودند.
\par 9 و بعد از گذشتن ایشان، ایلیا به الیشع گفت: «آنچه را که می‌خواهی برای تو بکنم، پیش ازآنکه از نزد تو برداشته شوم، بخواه.» الیشع گفت: «نصیب مضاعف روح تو بر من بشود.»
\par 10 او گفت: «چیز دشواری خواستی اما اگر حینی که از نزد توبرداشته شوم مرا ببینی، از برایت چنین خواهدشد والا نخواهد شد.»
\par 11 و چون ایشان می‌رفتندو گفتگو می‌کردند، اینک ارابه آتشین و اسبان آتشین ایشان را از یکدیگر جدا کرد و ایلیا درگردباد به آسمان صعود نمود.
\par 12 و چون الیشع این را بدید فریاد برآورد که «ای پدرم! ای پدرم! ارابه اسرائیل و سوارانش! پس او را دیگر ندیدو جامه خود را گرفته، آن را به دو حصه چاک زد.
\par 13 و ردای ایلیا را که از او افتاده بود، برداشت و برگشته به کناره اردن ایستاد.
\par 14 پس ردای ایلیارا که از او افتاده بود، گرفت و آب را زده، گفت: «یهوه خدای ایلیا کجاست؟» و چون او نیز آب رازد، به این طرف و آن طرف شکافته شد و الیشع عبور نمود.
\par 15 و چون پسران انبیا که روبروی او در اریحابودند او را دیدند، گفتند: «روح ایلیا بر الیشع می‌باشد.» و برای ملاقات وی آمده، او را رو به زمین تعظیم نمودند.
\par 16 و او را گفتند: «اینک حال با بندگانت پنجاه مرد قوی هستند، تمنا اینکه ایشان بروند و آقای تو را جستجو نمایند، شایدروح خداوند او را برداشته، به یکی از کوهها یا دریکی از دره‌ها انداخته باشد.» او گفت: «مفرستید.»
\par 17 اما به حدی بر وی ابرام نمودند که خجل شده، گفت: «بفرستید.» پس پنجاه نفر فرستادند و ایشان سه روز جستجو نمودند اما او را نیافتند.
\par 18 وچون او در اریحا توقف می‌نمود، ایشان نزد وی برگشتند و او به ایشان گفت: «آیا شما را نگفتم، که نروید.»
\par 19 و اهل شهر به الیشع گفتند: «اینک موضع شهر نیکوست چنانکه آقای ما می‌بیند، لیکن آبش ناگوار و زمینش بی‌حاصل است.»
\par 20 او گفت: «نزد من طشت نوی آورده، نمک در آن بگذارید.» پس برایش آوردند.
\par 21 و او نزد چشمه آب بیرون رفته، نمک را در آن انداخت و گفت: «خداوندچنین می‌گوید: این آب را شفا دادم که بار دیگرمرگ یا بی‌حاصلی از آن پدید نیاید.»
\par 22 پس آب تا به امروز برحسب سخنی که الیشع گفته بود، شفا یافت.
\par 23 و از آنجا به بیت ئیل برآمد و چون او به راه برمی آمد اطفال کوچک از شهر بیرون آمده، او راسخریه نموده، گفتند: «ای کچل برآی! ای کچل برآی!»
\par 24 و او به عقب برگشته، ایشان را دید وایشان را به اسم یهوه لعنت کرد و دو خرس ازجنگل بیرون آمده، چهل و دو پسر از ایشان بدرید.و از آنجا به کوه کرمل رفت و از آنجا به سامره مراجعت نمود.
\par 25 و از آنجا به کوه کرمل رفت و از آنجا به سامره مراجعت نمود.
 
\chapter{3}

\par 1 و یهورام بن اخاب در سال هجدهم یهوشافاط، پادشاه یهودا در سامره براسرائیل آغاز سلطنت نمود و دوازده سال پادشاهی کرد.
\par 2 و آنچه در نظر خداوند ناپسندبود به عمل می‌آورد، اما نه مثل پدر و مادرش زیرا که تمثال بعل را که پدرش ساخته بود، دورکرد.
\par 3 لیکن به گناهان یربعام بن نباط که اسرائیل رامرتکب گناه ساخته بود، چسبیده، از آن دوری نورزید.
\par 4 و میشع، پادشاه موآب، صاحب مواشی بودو به پادشاه اسرائیل صدهزار بره و صدهزار قوچ با پشم آنها ادا می‌نمود.
\par 5 و بعد از وفات اخاب، پادشاه موآب بر پادشاه اسرائیل عاصی شد.
\par 6 ودر آن وقت یهورام پادشاه از سامره بیرون شده، تمامی اسرائیل را سان دید.
\par 7 و رفت و نزدیهوشافاط، پادشاه یهودا فرستاده، گفت: «پادشاه موآب بر من عاصی شده است آیا همراه من برای مقاتله با موآب خواهی آمد؟» او گفت: «خواهم آمد، من چون تو هستم و قوم من چون قوم تو واسبان من چون اسبان تو.»
\par 8 او گفت: «به کدام راه برویم؟» گفت: «به راه بیابان ادوم.»
\par 9 پس پادشاه اسرائیل و پادشاه یهودا و پادشاه ادوم روانه شده، سفر هفت روزه دور زدند و به جهت لشکر و چارپایانی که همراه ایشان بود، آب نبود.
\par 10 و پادشاه اسرائیل گفت: «افسوس که خداوند این سه پادشاه را خوانده است تا ایشان رابه‌دست موآب تسلیم کند.»
\par 11 و یهوشافاطگفت: «آیا نبی خداوند در اینجا نیست تا به واسطه او از خداوند مسالت نماییم؟» و یکی ازخادمان پادشاه اسرائیل در جواب گفت: «الیشع بن شافاط که آب بر دستهای ایلیامی ریخت، اینجاست.»
\par 12 و یهوشافاط گفت: «کلام خداوند با اوست.» پس پادشاه اسرائیل و یهوشافاط و پادشاه ادوم نزد وی فرودآمدند.
\par 13 و الیشع به پادشاه اسرائیل گفت: «مرا با توچه‌کار است؟ نزد انبیای پدرت و انبیای مادرت برو.» اما پادشاه اسرائیل وی را گفت: «نی، زیراخداوند این سه پادشاه را خوانده است تا ایشان رابه‌دست موآب تسلیم نماید.»
\par 14 الیشع گفت: «به حیات یهوه صبایوت که به حضور وی ایستاده‌ام قسم که اگر من احترام یهوشافاط، پادشاه یهودا رانگاه نمی داشتم به سوی تو نظر نمی کردم و تو رانمی دیدم.
\par 15 اما الان برای من مطربی بیاورید.» وواقع شد که چون مطرب ساز زد، دست خداوندبر وی آمد.
\par 16 و او گفت: «خداوند چنین می‌گوید: این وادی را پر از خندقها بساز.
\par 17 زیراخداوند چنین می‌گوید: باد نخواهید دید و باران نخواهید دید اما این وادی از آب پر خواهد شد تاشما و مواشی شما و بهایم شما بنوشید.
\par 18 و این در نظر خداوند قلیل است بلکه موآب را نیز به‌دست شما تسلیم خواهد کرد.
\par 19 و تمامی شهرهای حصاردار و همه شهرهای بهترین را منهدم خواهید ساخت و همه درختان نیکو راقطع خواهید نمود و جمیع چشمه های آب راخواهید بست و هر قطعه زمین نیکو را با سنگهاخراب خواهید کرد.»
\par 20 و بامدادان در وقت گذرانیدن هدیه، اینک آب از راه ادوم آمد و آن زمین را از آب پر ساخت.
\par 21 و چون تمامی موآبیان شنیده بودند که پادشاهان برای مقاتله ایشان برمی آیند هر‌که به اسلاح جنگ مسلح می‌شد و هرکه بالاتر از آن بود، جمع شدند و به‌سرحد خود اقامت کردند.
\par 22 پس بامدادان چون برخاستند و آفتاب بر آن آب تابید، موآبیان از آن طرف، آب را مثل خون سرخ دیدند،
\par 23 و گفتند: «این خون است، پادشاهان البته مقاتله کرده، یکدیگر را کشته‌اند، پس حال‌ای موآبیان به غنیمت بشتابید.»
\par 24 اماچون به لشکرگاه اسرائیل رسیدند، اسرائیلیان برخاسته، موآبیان را شکست دادند که از حضورایشان منهزم شدند، و به زمین ایشان داخل شده، موآبیان را می‌کشتند.
\par 25 و شهرها را منهدم ساختند و بر هر قطعه نیکو هرکس سنگ خود راانداخته، آن را پر کردند و تمام چشمه های آب رامسدود ساختند، و تمامی درختان خوب را قطع نمودند لکن سنگهای قیرحارست را در آن واگذاشتند و فلاخن اندازان آن را احاطه کرده، زدند.
\par 26 و چون پادشاه موآب دید که جنگ بر اوسخت شد هفتصد نفر شمشیرزن گرفت که تا نزدپادشاه ادوم را بشکافند اما نتوانستند.پس پسرنخست زاده خود را که به‌جایش می‌بایست سلطنت نماید، گرفته، او را بر حصار به جهت قربانی سوختنی گذرانید. و غیظ عظیمی براسرائیل پدید آمد. پس از نزد وی روانه شده، به زمین خود مراجعت کردند.
\par 27 پس پسرنخست زاده خود را که به‌جایش می‌بایست سلطنت نماید، گرفته، او را بر حصار به جهت قربانی سوختنی گذرانید. و غیظ عظیمی براسرائیل پدید آمد. پس از نزد وی روانه شده، به زمین خود مراجعت کردند. 
 
\chapter{4}

\par 1 و زنی از زنان پسران انبیا نزد الیشع تضرع نموده، گفت: «بنده ات، شوهرم مرد و تومی دانی که بنده ات از خداوند می‌ترسید، وطلبکار او آمده است تا دو پسر مرا برای بندگی خود ببرد.»
\par 2 الیشع وی را گفت: «بگو برای تو چه کنم؟ و در خانه چه داری؟ او گفت: «کنیزت را درخانه چیزی سوای ظرفی از روغن نیست.»
\par 3 اوگفت برو و ظرفها از بیرون از تمامی همسایگان خود طلب کن، ظرفهای خالی و بسیار بخواه.
\par 4 و داخل شده، در را بر خودت و پسرانت ببندو در تمامی آن ظرفها بریز و هرچه پر شود به کناربگذار.»
\par 5 پس از نزد وی رفته، در را بر خود وپسرانش بست و ایشان ظرفها نزد وی آورده، اومی ریخت.
\par 6 و چون ظرفها را پر کرده بود به یکی از پسران خود گفت: «ظرفی دیگر نزد من بیاور.» او وی را گفت: «ظرفی دیگر نیست.» و روغن بازایستاد.
\par 7 پس رفته، آن مرد خدا را خبر داد. واو وی را گفت: «برو و روغن را بفروش و قرض خود را ادا کرده، تو و پسرانت از باقی‌مانده گذران کنید.»
\par 8 و روزی واقع شد که الیشع به شونیم رفت ودر آنجا زنی بزرگ بود که بر او ابرام نمود که طعام بخورد و هرگاه عبور می‌نمود، به آنجا به جهت نان خوردن میل می‌کرد.
\par 9 پس آن زن به شوهرخود گفت: «اینک فهمیده‌ام که این مرد مقدس خداست که همیشه از نزد ما می‌گذرد.
\par 10 پس برای وی بالاخانه‌ای کوچک بر دیوار بسازیم و بستر و خوان و کرسی و شمعدانی درآن برای وی بگذرانیم که چون نزد ما آید، در آنجا فرودآید.»
\par 11 پس روزی آنجا آمد و به آن بالاخانه فرودآمده، در آنجا خوابید.
\par 12 و به خادم خود، جیحزی گفت: «این زن شونمی را بخوان.» و چون او را خواند، او به حضور وی ایستاد.
\par 13 و او به خادم گفت: «به او بگو که اینک تمامی این زحمت را برای ما کشیده‌ای پس برای تو چه شود؟ آیا باپادشاه یا سردار لشکر کاری داری؟ او گفت: «نی، من در میان قوم خود ساکن هستم.»
\par 14 و او گفت: «پس برای این زن چه باید کرد؟» جیحزی عرض کرد: «یقین که پسری ندارد و شوهرش سالخورده است.»
\par 15 آنگاه الیشع گفت: «او را بخوان.» پس وی را خوانده، او نزد در ایستاد.
\par 16 و گفت: «دراین وقت موافق زمان حیات، پسری در آغوش خواهی گرفت.» و او گفت: «نی‌ای آقایم؛ ای مردخدا به کنیز خود دروغ مگو.»
\par 17 پس آن زن حامله شده، در آن وقت موافق زمان حیات به موجب کلامی که الیشع به او گفته بود، پسری زایید.
\par 18 و چون آن پسر بزرگ شد روزی اتفاق افتادکه نزد پدر خود نزد دروگران رفت.
\par 19 و به پدرش گفت: «آه سر من! آه سر من! و او به خادم خودگفت: «وی را نزد مادرش ببر.»
\par 20 پس او رابرداشته، نزد مادرش برد و او به زانوهایش تا ظهرنشست و مرد.
\par 21 پس مادرش بالا رفته، او را بربستر مرد خدا خوابانید و در را بر او بسته، بیرون رفت.
\par 22 و شوهر خود را آواز داده، گفت: «تمنااینکه یکی از جوانان و الاغی از الاغها بفرستی تانزد مرد خدا بشتابم و برگردم.
\par 23 او گفت: «امروز چرا نزد او بروی، نه غره ماه و نه سبت است.» گفت: «سلامتی است.»
\par 24 پس الاغ را آراسته، به خادم خود گفت: «بران و برو و تا تو را نگویم درراندن کوتاهی منما.»
\par 25 پس رفته، نزد مرد خدا به کوه کرمل رسید.
\par 26 پس حال به استقبال وی بشتاب و وی را بگو: آیا تو را سلامتی است و آیا شوهرت سالم وپسرت سالم است؟» او گفت: «سلامتی است.»
\par 27 و چون نزد مرد خدا به کوه رسید، به پایهایش چسبید. و جیحزی نزدیک آمد تا او را دور کنداما مرد خدا گفت: «او را واگذار زیرا که جانش دروی تلخ است و خداوند این را از من مخفی داشته، مرا خبر نداده است.»
\par 28 و زن گفت: «آیا پسری ازآقایم درخواست نمودم، مگر نگفتم مرا فریب مده؟»
\par 29 پس او به جیحزی گفت: «کمر خود راببند و عصای مرا به‌دستت گرفته، برو و اگر کسی را ملاقات کنی، او را تحیت مگو و اگر کسی تو راتحیت گوید، جوابش مده و عصای مرا بر روی طفل بگذار.»
\par 30 اما مادر طفل گفت: «به حیات یهوه و به حیات خودت قسم که تو را ترک نکنم.» پس او برخاسته، در عقب زن روانه شد.
\par 31 وجیحزی از ایشان پیش رفته، عصا را بر روی طفل نهاد اما نه آواز داد و نه اعتنا نمود، پس به استقبال وی برگشته، او را خبر داد و گفت که طفل بیدارنشد.
\par 32 پس الیشع به خانه داخل شده، دید که طفل مرده و بر بستر او خوابیده است.
\par 33 و چون داخل شد، در را بر هر دو بست و نزد خداوند دعا نمود.
\par 34 و برآمده بر طفل دراز شد و دهان خود را بردهان وی و چشم خود را بر چشم او و دست خودرا بر دست او گذاشته، بر وی خم گشت و گوشت پسر گرم شد.
\par 35 و برگشته، درخانه یک مرتبه این طرف و آن طرف بخرامید و برآمده، بر وی خم شد که طفل هفت مرتبه عطسه کرد، پس طفل چشمان خود را باز کرد.
\par 36 و جیحزی را آوازداده، گفت: «این زن شونمی را بخوان.» پس او راخواند و چون نزد او داخل شد، او وی را گفت: «پسر خود را بردار.»
\par 37 پس آن زن داخل شده، نزد پایهایش افتاد و رو به زمین خم شد و پسرخود را برداشته، بیرون رفت.
\par 38 و الیشع به جلجال برگشت و قحطی درزمین بود و پسران انبیا به حضور وی نشسته بودند. و او به خادم خود گفت: «دیگ بزرگ رابگذار و آش به جهت پسران انبیا بپز.»
\par 39 و کسی به صحرا رفت تا سبزیها بچیند و بوته بری یافت وخیارهای بری از آن چیده، دامن خود را پرساخت و آمده، آنها را در دیگ آش خرد کردزیرا که آنها را نشناختند.
\par 40 پس برای آن مردمان ریختند تا بخورند و چون قدری آش خوردند، صدا زده، گفتند: «ای مرد خدا مرگ در دیگ است.» و نتوانستند بخورند.
\par 41 او گفت: «آردبیاورید.» پس آن را در دیگ انداخت و گفت: «برای مردم بریز تا بخورند.» پس هیچ‌چیز مضردر دیگ نبود.
\par 42 و کسی از بعل شلیشه آمده، برای مرد خداخوراک نوبر، یعنی بیست قرص نان جو و خوشه‌ها در کیسه خود آورد. پس او گفت: «به مردم بده تا بخورند.»
\par 43 خادمش گفت: «اینقدر راچگونه پیش صد نفر بگذارم؟» او گفت: «به مردمان بده تا بخورند، زیرا خداوند چنین می‌گوید که خواهند خورد و از ایشان باقی خواهد ماند.»پس پیش ایشان گذاشت و به موجب کلام خداوند خوردند و از ایشان باقی ماند.
\par 44 پس پیش ایشان گذاشت و به موجب کلام خداوند خوردند و از ایشان باقی ماند.
 
\chapter{5}

\par 1 و نعمان، سردار لشکر پادشاه ارام، در حضور آقایش مردی بزرگ و بلند جاه بود، زیرا خداوند به وسیله او ارام را نجات داده بود، وآن مرد جبار، شجاع ولی ابرص بود.
\par 2 و فوجهای ارامیان بیرون رفته، کنیزکی کوچک از زمین اسرائیل به اسیری آوردند و او در حضور زن نعمان خدمت می‌کرد.
\par 3 و به خاتون خود گفت: «کاش که آقایم در حضور نبی‌ای که در سامره است، می‌بود که او را از برصش شفا می‌داد.»
\par 4 پس کسی درآمده، آقای خود را خبر داده، گفت: «کنیزی که از ولایت اسرائیل است، چنین وچنان می‌گوید.»
\par 5 پس پادشاه ارام گفت: «بیا برو ومکتوبی برای پادشاه اسرائیل می‌فرستم.»
\par 6 ومکتوب را نزد پادشاه اسرائیل آورد و در آن نوشته بود که «الان چون این مکتوب به حضورت برسد اینک بنده خود نعمان را نزد تو فرستادم تااو را از برصش شفا دهی.»
\par 7 اما چون پادشاه اسرائیل مکتوب را خواند لباس خود را دریده، گفت: «آیا من مرد خدا هستم که بمیرانم و زنده کنم که این شخص نزد من فرستاده است تا کسی رااز برصش شفا بخشم. پس بدانید و ببینید که اوبهانه جویی از من می‌کند.»
\par 8 اما چون الیشع، مرد خدا شنید که پادشاه اسرائیل لباس خود را دریده است، نزد پادشاه فرستاده، گفت: «لباس خود را چرا دریدی؟ اونزد من بیاید تا بداند که در اسرائیل نبی‌ای هست.»
\par 9 پس نعمان با اسبان و ارابه های خودآمده، نزد در خانه الیشع ایستاد.
\par 10 و الیشع رسولی نزد وی فرستاده، گفت: «برو و در اردن هفت مرتبه شست و شو نما. و گوشتت به توبرگشته، طاهر خواهی شد.»
\par 11 اما نعمان غضبناک شده، رفت و گفت: «اینک گفتم البته نزدمن بیرون آمده، خواهد ایستاد و اسم خدای خود، یهوه را خوانده، و دست خود را بر جای برص حرکت داده، ابرص را شفا خواهد داد.
\par 12 آیا ابانه و فرفر، نهرهای دمشق، از جمیع آبهای اسرائیل بهتر نیست؟ آیا در آنها شست وشو نکنم تا طاهر شوم؟» پس برگشته، با خشم رفت.
\par 13 اما بندگانش نزدیک آمده، او را خطاب کرده، گفتند: «ای پدر ما اگر نبی، تو را امری بزرگ گفته بود، آیا آن را بجا نمی آوردی؟ پس چندمرتبه زیاده چون تو را گفته است شست و شو کن و طاهر شو.»
\par 14 پس فرود شده، هفت مرتبه دراردن به موجب کلام مرد خدا غوطه خورد وگوشت او مثل گوشت طفل کوچک برگشته، طاهرشد.
\par 15 پس او با تمامی جمعیت خود نزد مرد خدامراجعت کرده، داخل شد و به حضور وی ایستاده، گفت: «اینک الان دانسته‌ام که در تمامی زمین جز در اسرائیل خدایی نیست و حال تمنااینکه هدیه‌ای از بنده ات قبول فرمایی.»
\par 16 اوگفت: «به حیات یهوه که در حضور وی ایستاده‌ام قسم که قبول نخواهم کرد.» و هرچند او را ابرام نمود که بپذیرد ابا نمود.
\par 17 و نعمان گفت: «اگرنه تمنا این که دو بار قاطر از خاک، به بنده ات داده شود زیرا که بعد از این، بنده ات قربانی سوختنی وذبیحه نزد خدایان غیر نخواهد گذرانید الا نزدیهوه.
\par 18 اما در این امر، خداوند بنده تو را عفوفرماید که چون آقایم به خانه رمون داخل شده، در آنجا سجده نماید و بر دست من تکیه کند و من در خانه رمون سجده نمایم، یعنی چون در خانه رمون سجده کنم، خداوند بنده تو را در این امرعفو فرماید.»
\par 19 او وی را گفت: «به سلامتی برو.»
\par 20 اماجیحزی که خادم الیشع مرد خدا بود گفت: «اینک آقایم از گرفتن از دست این نعمان ارامی آنچه راکه آورده بود، امتناع نمود. به حیات یهوه قسم که من از عقب او دویده، چیزی از او خواهم گرفت.»
\par 21 پس جیحزی از عقب نعمان شتافت و چون نعمان او را دید که از عقبش می‌دود از ارابه خودبه استقبالش فرود آمد و گفت: «آیا سلامتی است؟»
\par 22 او گفت: «سلامتی است. آقایم مرافرستاده، می‌گوید: اینک الان دو جوان از پسران انبیا از کوهستان افرایم نزد من آمده‌اند، تمنا اینکه یک وزنه نقره و دو دست لباس به ایشان بدهی.»
\par 23 نعمان گفت: «مرحمت فرموده، دو وزنه بگیر.» پس بر او ابرام نمود تا او دو وزنه نقره را در دوکیسه با دو دست لباس بست و بر دو خادم خودنهاد تا پیش او بردند.
\par 24 و چون به عوفل رسید، آنها را از دست ایشان گرفته، در خانه گذاشت و آن اشخاص را مرخص کرده، رفتند.
\par 25 و او داخل شده، به حضور آقای خودایستاد و الیشع وی را گفت: «ای جیحزی از کجامی آیی؟» گفت: «بنده ات جایی نرفته بود.»
\par 26 الیشع وی را گفت: «آیا دل من همراه تو نرفت هنگامی که آن مرد از ارابه خود به استقبال توبرگشت؟ آیا این وقت، وقت گرفتن نقره و گرفتن لباس و باغات زیتون و تاکستانها و گله هاو رمه هاو غلامان و کنیزان است؟پس برص نعمان به تو و به ذریت تو تا به ابد خواهد چسبید.» و ازحضور وی مبروص مثل برف بیرون رفت.
\par 27 پس برص نعمان به تو و به ذریت تو تا به ابد خواهد چسبید.» و ازحضور وی مبروص مثل برف بیرون رفت.
 
\chapter{6}

\par 1 و پسران انبیا به الیشع گفتند که «اینک مکانی که در حضور تو در آن ساکنیم، برای ما تنگ است.
\par 2 پس به اردن برویم و هریک چوبی از آنجا بگیریم و مکانی برای خود در آنجابسازیم تا در آن ساکن باشیم.» او گفت: «بروید.»
\par 3 و یکی از ایشان گفت: «مرحمت فرموده، همراه بندگانت بیا.» او جواب داد که «می‌آیم.»
\par 4 پس همراه ایشان روانه شد و چون به اردن رسیدند، چوبها را قطع نمودند.
\par 5 و هنگامی که یکی ازایشان تیر را می‌برید، آهن تبر در آب افتاد و اوفریاد کرده، گفت: «آه‌ای آقایم زیرا که عاریه بود.»
\par 6 پس مرد خدا گفت: «کجا افتاد؟» و چون جا را به وی نشان داد، او چوبی بریده، در آنجاانداخت و آهن را روی آب آورد.
\par 7 پس گفت: «برای خود بردار.» پس دست خود را دراز کرده، آن را گرفت. 
\par 8 و پادشاه ارام با اسرائیل جنگ می‌کرد و با بندگان خود مشورت کرده، گفت: «در فلان جااردوی من خواهد بود.»
\par 9 اما مرد خدا نزد پادشاه اسرائیل فرستاده، گفت: «با حذر باش که از فلان جا گذر نکنی زیرا که ارامیان به آنجا نزول کرده‌اند.»
\par 10 و پادشاه اسرائیل به مکانی که مردخدا او را خبر داد و وی را از آن انذار نمود، فرستاده، خود را از آنجا نه یکبار و نه دو بارمحافظت کرد.
\par 11 و دل پادشاه ارام از این امر مضطرب شد وخادمان خود را خوانده، به ایشان گفت: «آیا مراخبر نمی دهید که کدام از ما به طرف پادشاه اسرائیل است؟»
\par 12 و یکی از خادمانش گفت: «ای آقایم چنین نیست، بلکه الیشع نبی که دراسرائیل است، پادشاه اسرائیل را از سخنی که درخوابگاه خود می‌گویی، مخبر می‌سازد.»
\par 13 اوگفت: «بروید و ببینید که او کجاست، تا بفرستم واو را بگیرم.» پس او را خبر دادند که اینک دردوتان است.
\par 14 پس سواران و ارابه‌ها و لشکرعظیمی بدانجا فرستاد و ایشان وقت شب آمده، شهر را احاطه نمودند.
\par 15 و چون خادم مرد خداصبح زود برخاسته، بیرون رفت. اینک لشکری باسواران و ارابه‌ها شهر را احاطه نموده بودند. پس خادمش وی را گفت: «آه‌ای آقایم چه بکنیم؟»
\par 16 او گفت: «مترس زیرا آنانی که با مایند از آنانی که با ایشانند بیشترند.»
\par 17 و الیشع دعا کرده، گفت: «ای خداوند چشمان او را بگشا تا ببیند.» پس خداوند چشمان خادم را گشود و او دید که اینک کوههای اطراف الیشع از سواران وارابه های آتشین پر است.
\par 18 و چون ایشان نزدوی فرود شدند الیشع نزد خداوند دعا کرده، گفت: «تمنا اینکه این گروه را به کوری مبتلاسازی.» پس ایشان را به موجب کلام الیشع به کوری مبتلا ساخت.
\par 19 و الیشع، ایشان را گفت: «راه این نیست و شهر این نیست. از عقب من بیاییدو شما را به کسی‌که می‌طلبید، خواهم رسانید.» پس ایشان را به سامره آورد.
\par 20 و هنگامی که وارد سامره شدند، الیشع گفت: «ای خداوند چشمان ایشان را بگشا تاببینند.» پس خداوند چشمان ایشان را گشود ودیدند که اینک در سامره هستند.
\par 21 آنگاه پادشاه اسرائیل چون ایشان را دید، به الیشع گفت: «ای پدرم آیا بزنم؟ آیا بزنم؟»
\par 22 او گفت: «مزن آیاکسانی را که به شمشیر و کمان خود اسیر کرده‌ای، خواهی زد؟ نان و آب پیش ایشان بگذار تابخورند و بنوشند و نزد آقای خود بروند.»
\par 23 پس ضیافتی بزرگ برای ایشان برپا کرد و چون خوردند و نوشیدند، ایشان را مرخص کرد که نزدآقای خویش رفتند. و بعد از آن، فوجهای ارام دیگر به زمین اسرائیل نیامدند.
\par 24 و بعد از این، واقع شد که بنهدد، پادشاه ارام، تمام لشکر خود را جمع کرد و برآمده، سامره را محاصره نمود.
\par 25 و قحطی سخت درسامره بود و اینک آن را محاصره نموده بودند، به حدی که سر الاغی به هشتاد پاره نقره و یک ربع قاب جلغوزه، به پنج پاره نقره فروخته می‌شد.
\par 26 و چون پادشاه اسرائیل بر باره گذر می‌نمود، زنی نزد وی فریاد برآورده، گفت: «ای آقایم پادشاه، مدد کن.»
\par 27 او گفت: «اگر خداوند تو رامدد نکند، من از کجا تو را مدد کنم؟ آیا از خرمن یا از چرخشت؟»
\par 28 پس پادشاه او را گفت: «تو راچه شد؟» او عرض کرد: «این زن به من گفت: پسرخود را بده تا امروز او را بخوریم و پسر مرا فردا خواهیم خورد.
\par 29 پس پسر مرا پختیم و خوردیم و روز دیگر وی را گفتم: پسرت را بده تا او رابخوریم اما او پسر خود را پنهان کرد.»
\par 30 و چون پادشاه سخن زن را شنید، رخت خود را بدرید واو بر باره می‌گذشت و قوم دیدند که اینک در زیرلباس خود پلاس دربر داشت.
\par 31 و گفت: «خدا به من مثل این بلکه زیاده از این بکند اگر سر الیشع بن شافاط امروز بر تنش بماند.»
\par 32 و الیشع در خانه خود نشسته بود و مشایخ، همراهش نشسته بودندو پادشاه، کسی را از نزد خود فرستاد و قبل ازرسیدن قاصد نزد وی، الیشع به مشایخ گفت: «آیامی بینید که این پسر قاتل فرستاده است تا سر مرااز تن جدا کند؟ متوجه باشید وقتی که قاصدبرسد، در را ببندید و او را از در برانید، آیا صدای پایهای آقایش در عقبش نیست.»و چون اوهنوز به ایشان سخن می‌گفت، اینک قاصد نزد وی رسید و او گفت: «اینک این بلا از جانب خداونداست، چرا دیگر برای خداوند انتظار بکشم.»
\par 33 و چون اوهنوز به ایشان سخن می‌گفت، اینک قاصد نزد وی رسید و او گفت: «اینک این بلا از جانب خداونداست، چرا دیگر برای خداوند انتظار بکشم.»
 
\chapter{7}

\par 1 و الیشع گفت: «کلام خداوند را بشنوید. خداوند چنین می‌گوید که «فردا مثل این وقت یک کیل آرد نرم به یک مثقال و دو کیل جوبه یک مثقال نزد دروازه سامره فروخته می‌شود.»
\par 2 و سرداری که پادشاه بر دست وی تکیه می‌نموددر جواب مرد خدا گفت: «اینک اگر خداوندپنجره‌ها هم در آسمان بسازد، آیا این چیز واقع تواند شد؟» او گفت: «همانا تو به چشم خودخواهی دید اما از آن نخواهی خورد.»
\par 3 و چهار مرد مبروص نزد دهنه دروازه بودند و به یکدیگر گفتند: «چرا ما اینجا بنشینیم تابمیریم؟
\par 4 اگر گوییم به شهر داخل شویم هماناقحطی در شهر است و در آنجا خواهیم مرد و اگردر اینجا بمانیم، خواهیم مرد. پس حال برویم وخود را به اردوی ارامیان بیندازیم. اگر ما را زنده نگاه دارند، زنده خواهیم ماند و اگر ما را بکشند، خواهیم مرد.»
\par 5 پس وقت شام برخاستند تا به اردوی ارامیان بروند، اما چون به کنار اردوی ارامیان رسیدند اینک کسی در آنجا نبود.
\par 6 زیراخداوند صدای ارابه‌ها و صدای اسبان و صدای لشکر عظیمی را در اردوی ارامیان شنوانید و به یکدیگر گفتند: «اینک پادشاه اسرائیل، پادشاهان حتیان و پادشاهان مصریان را به ضد ما اجیر کرده است تا بر ما بیایند.»
\par 7 پس برخاسته، به وقت شام فرار کردند و خیمه‌ها و اسبان و الاغها و اردوی خود را به طوری که بود ترک کرده، از ترس جان خود گریختند.
\par 8 و آن مبروصان به کنار اردوآمده، به خیمه‌ای داخل شدند و اکل و شرب نموده، از آنجا نقره و طلا و لباس گرفته، رفتند وآنها را پنهان کردند و برگشته، به خیمه‌ای دیگرداخل شده، از آن نیز بردند؛ و رفته، پنهان کردند.
\par 9 پس به یکدیگر گفتند: «ما خوب نمی کنیم؛ امروز روز بشارت است و ما خاموش می‌مانیم واگر تا روشنایی صبح به تاخیر اندازیم، بلایی به ماخواهد رسید، پس الان بیایید برویم و به خانه پادشاه خبر دهیم.»
\par 10 پس رفته، دربانان شهر راصدا زدند و ایشان را مخبر ساخته، گفتند: «به اردوی ارامیان درآمدیم و اینک در آنجا نه کسی و نه صدای انسانی بود مگر اسبان بسته شده، والاغها بسته شده و خیمه‌ها به حالت خود.»
\par 11 پس دربانان صدا زده، خاندان پادشاه را دراندرون اطلاع دادند.
\par 12 و پادشاه در شب برخاست و به خادمان خود گفت: «به تحقیق شمارا خبر می‌دهم که ارامیان به ما چه خواهند کرد: می‌دانند که ما گرسنه هستیم پس از اردو بیرون رفته، خود را در صحرا پنهان کرده‌اند و می‌گویندچون از شهر بیرون آیند، ایشان را زنده خواهیم گرفت و به شهر داخل خواهیم شد.»
\par 13 و یکی ازخادمانش در جواب وی گفت: «پنج راس ازاسبان باقی‌مانده که در شهر باقی‌اند، بگیرند(اینک آنها مثل تمامی گروه اسرائیل که در آن باقی‌اند یا مانند تمامی گروه اسرائیل که هلاک شده‌اند، می‌باشند) و بفرستیم تا دریافت نماییم.»
\par 14 پس دو ارابه با اسبها گرفتند و پادشاه از عقب لشکر ارام فرستاده، گفت: «بروید و تحقیق کنید.»
\par 15 پس از عقب ایشان تا اردن رفتند و اینک تمامی راه از لباس و ظروفی که ارامیان از تعجیل خود انداخته بودند، پر بود. پس رسولان برگشته، پادشاه را مخبر ساختند.
\par 16 و قوم بیرون رفته، اردوی ارامیان را غارت کردند و یک کیل آرد نرم به یک مثقال و دو کیل جو به یک مثقال به موجب کلام خداوند به فروش رفت.
\par 17 و پادشاه آن سردار را که بر دست وی تکیه می‌نمود بر دروازه گماشت و خلق، او را نزددروازه پایمال کردند که مرد بر‌حسب کلامی که مرد خدا گفت هنگامی که پادشاه نزد وی فرودآمد.
\par 18 و واقع شد به نهجی که مرد خدا، پادشاه را خطاب کرده، گفته بود که فردا مثل این وقت دو کیل جو به یک مثقال و یک کیل آرد نرم به یک مثقال نزد دروازه سامره فروخته خواهد شد.
\par 19 و آن سردار در جواب مرد خدا گفته بود: اگر خداوند پنجره‌ها هم در آسمان بگشاید، آیامثل این امر واقع تواند شد؟ و او گفت: «اینک به چشمان خود خواهی دید اما از آن نخواهی خورد.»پس او را همچنین واقع شد زیرا خلق او را نزد دروازه پایمال کردند که مرد.
\par 20 پس او را همچنین واقع شد زیرا خلق او را نزد دروازه پایمال کردند که مرد.
 
\chapter{8}

\par 1 و الیشع به زنی که پسرش را زنده کرده بود، خطاب کرده، گفت: «تو و خاندانت برخاسته، بروید و در جایی که می‌توانی ساکن شوی، ساکن شو، زیرا خداوند قحطی خوانده است و هم بر زمین هفت سال واقع خواهد شد.»
\par 2 و آن زن برخاسته، موافق کلام مرد خدا، عمل نمود و با خاندان خود رفته، در زمین فلسطینیان هفت سال ماوا گزید.
\par 3 و واقع شد بعد از انقضای هفت سال که آن زن از زمین فلسطینیان مراجعت کرده، بیرون آمد تا نزد پادشاه برای خانه و زمین خود استغاثه نماید.
\par 4 و پادشاه با جیحزی، خادم مرد خدا گفتگو می‌نمود و می‌گفت: «حال تمام اعمال عظیمی را که الیشع بجا آورده است، به من بگو.»
\par 5 و هنگامی که او برای پادشاه بیان می‌کردکه چگونه مرده‌ای را زنده نمود، اینک زنی که پسرش را زنده کرده بود، نزد پادشاه به جهت خانه و زمین خود استغاثه نمود. و جیحزی گفت: «ای آقایم پادشاه! این همان زن است و پسری که الیشع زنده کرد، این است.»
\par 6 و چون پادشاه از زن پرسید، او وی را خبر داد، پس پادشاه یکی ازخواجگان خود را برایش تعیین نموده، گفت: «تمامی مایملک او وتمامی حاصل ملک او را از روزی که زمین را ترک کرده است تا الان به او ردنما.»
\par 7 و الیشع به دمشق رفت و بنهدد، پادشاه ارام، بیمار بود و به او خبر داده، گفتند که مرد خدااینجا آمده است.
\par 8 پس پادشاه به حزائیل گفت: «هدیه‌ای به‌دست خود گرفته، برای ملاقات مردخدا برو و به واسطه او از خداوند سوال نما که آیااز این مرض خود شفا خواهم یافت؟»
\par 9 و حزائیل برای ملاقات وی رفته، هدیه‌ای به‌دست خودگرفت، یعنی بار چهل شتر از تمامی نفایس دمشق. و آمده، به حضور وی ایستاد و گفت: «پسرت، بنهدد، پادشاه ارام مرا نزد تو فرستاده، می‌گوید: آیا از این مرض خود شفا خواهم یافت؟»
\par 10 و الیشع وی را گفت: «برو و او را بگو: البته شفا توانی یافت لیکن خداوند مرا اعلام نموده است که هرآینه او خواهد مرد.»
\par 11 و چشم خود را خیره ساخته، بر وی نگریست تا خجل گردید. پس مرد خدا بگریست.
\par 12 و حزائیل گفت: «آقایم چرا گریه می‌کند؟» او جواب داد: «چونکه ضرری را که تو به بنی‌اسرائیل خواهی رسانید، می‌دانم، قلعه های ایشان را آتش خواهی زد و جوانان ایشان را به شمشیر خواهی کشت، واطفال ایشان را خرد خواهی نمود و حامله های ایشان را شکم پاره خواهی کرد.»
\par 13 و حزائیل گفت: «بنده تو که سگ است، کیست که چنین عمل عظیمی بکند؟» الیشع گفت: «خداوند بر من نموده است که تو پادشاه ارام خواهی شد.»
\par 14 پس از نزد الیشع روانه شده، نزد آقای خود آمد و او وی را گفت: «الیشع تو را چه گفت؟» اوجواب داد: «به من گفت که البته شفا خواهی یافت.»
\par 15 و در فردای آن روز، لحاف را گرفته آن را در آب فرو برد و بر رویش گسترد که مرد وحزائیل در جایش پادشاه شد.
\par 16 و در سال پنجم یورام بن اخاب، پادشاه اسرائیل، وقتی که یهوشافاط هنوز پادشاه یهودابود، یهورام بن یهوشافاط، پادشاه یهودا آغازسلطنت نمود.
\par 17 و چون پادشاه شد، سی و دوساله بود و هشت سال در اورشلیم پادشاهی کرد.
\par 18 و به طریق پادشاهان اسرائیل به نحوی که خاندان اخاب عمل می‌نمودند سلوک نمود، زیراکه دختر اخاب، زن او بود و آنچه در نظر خداوندناپسند بود، به عمل می‌آورد.
\par 19 اما خداوند به‌خاطر بنده داود نخواست که یهودا را هلاک سازدچونکه وی را وعده داده بود که او را و پسرانش راهمیشه اوقات، چراغی بدهد. 
\par 20 و در ایام وی ادوم از زیر دست یهودا عاصی شده، پادشاهی برخود نصب کردند.
\par 21 و یورام با تمامی ارابه های خود به صعیر رفتند و در شب برخاسته، ادومیان را که او را احاطه نموده بودند و سرداران ارابه‌ها راشکست داد و قوم به خیمه های خود فرار کردند.
\par 22 و ادوم از زیر دست یهودا تا امروز عاصی شده‌اند و لبنه نیز در آن وقت عاصی شد.
\par 23 و بقیه وقایع یورام و آنچه کرد، آیا در کتاب تواریخ ایام پادشاهان یهودا مکتوب نیست؟
\par 24 و یورام باپدران خود خوابید و در شهر داود با پدران خوددفن شد. و پسرش اخزیا به‌جایش پادشاهی کرد.
\par 25 و در سال دوازدهم یورام بن اخاب، پادشاه اسرائیل، اخزیا ابن یهورام، پادشاه یهودا، آغاز سلطنت نمود.
\par 26 و اخزیا چون پادشاه شد، بیست و دو ساله بود و یک سال در اورشلیم پادشاهی کرد و اسم مادرش عتلیا، دختر عمری پادشاه اسرائیل بود.
\par 27 و به طریق خاندان اخاب سلوک نموده، آنچه در نظر خداوند ناپسند بود، مثل خاندان اخاب به عمل می‌آورد زیرا که دامادخاندان اخاب بود.
\par 28 و با یورام بن اخاب برای مقاتله با حزائیل پادشاه ارام به راموت جلعاد رفت و ارامیان، یورام را مجروح ساختند.و یورام پادشاه به یزرعیل مراجعت کرد تا از جراحتهایی که ارامیان به وی رسانیده بودند هنگامی که با حزائیل، پادشاه ارام جنگ می‌نمود، شفا یابد. و اخزیا ابن یهورام، پادشاه یهودا، به یزرئیل فرود آمد تا یورام بن اخاب را عیادت نماید چونکه مریض بود.
\par 29 و یورام پادشاه به یزرعیل مراجعت کرد تا از جراحتهایی که ارامیان به وی رسانیده بودند هنگامی که با حزائیل، پادشاه ارام جنگ می‌نمود، شفا یابد. و اخزیا ابن یهورام، پادشاه یهودا، به یزرئیل فرود آمد تا یورام بن اخاب را عیادت نماید چونکه مریض بود.
 
\chapter{9}

\par 1 و الیشع نبی یکی از پسران انبیا را خوانده، به او گفت: «کمر خود را ببند و این حقه روغن را به‌دست خود گرفته، به راموت جلعادبرو.
\par 2 و چون به آنجا رسیدی، ییهو ابن یهوشافاطبن نمشی را پیدا کن و داخل شده، او را از میان برادرانش برخیزان و او را به اطاق خلوت ببر.
\par 3 وحقه روغن را گرفته، به‌سرش بریز و بگو خداوندچنین می‌گوید که تو را به پادشاهی اسرائیل مسح کردم. پس در را باز کرده، فرار کن و درنگ منما.»
\par 4 پس آن جوان، یعنی آن نبی جوان به راموت جلعاد آمد.
\par 5 و چون بدانجا رسید، اینک سرداران لشکر نشسته بودند و او گفت: «ای سردار با توسخنی دارم.» ییهو گفت: «به کدام‌یک از جمیع ما؟» گفت: «به تو‌ای سردار!»
\par 6 پس او برخاسته، به خانه داخل شد و روغن را به‌سرش ریخته، وی را گفت: «یهوه، خدای اسرائیل چنین می‌گوید که تو را بر قوم خداوند، یعنی بر اسرائیل به پادشاهی مسح کردم.
\par 7 و خاندان آقای خود، اخاب راخواهی زد تا من انتقام خون بندگان خود، انبیا را وخون جمیع بندگان خداوند را از دست ایزابل بکشم.
\par 8 و تمامی خاندان اخاب هلاک خواهندشد. و از اخاب هر مرد را و هر بسته و رهاشده‌ای در اسرائیل را منقطع خواهم ساخت.
\par 9 و خاندان اخاب را مثل خاندان یربعام بن نباط و مانندخاندان بعشا ابن اخیا خواهم ساخت.
\par 10 و سگان، ایزابل را در ملک یزرعیل خواهند خورد ودفن کننده‌ای نخواهند بود.» پس در را باز کرده، بگریخت.
\par 11 و ییهو نزد بندگان آقای خویش بیرون آمدو کسی وی را گفت: «آیا سلامتی است؟ و این دیوانه برای چه نزد تو آمد؟» به ایشان گفت: «شمااین مرد و کلامش را می‌دانید.»
\par 12 گفتند: «چنین نیست. ما را اطلاع بده.» پس او گفت: «چنین وچنان به من تکلم نموده، گفت که خداوند چنین می‌فرماید: تو را به پادشاهی اسرائیل مسح کردم.»
\par 13 آنگاه ایشان تعجیل نموده، هر کدام رخت خود را گرفته، آن را زیر او به روی زینه نهادند، و کرنا را نواخته، گفتند که «ییهو پادشاه است.»
\par 14 لهذا ییهو ابن یهوشافاط بن نمشی بر یورام بشورید و یورام خود و تمامی اسرائیل، راموت جلعاد را از حزائیل، پادشاه ارام نگاه می‌داشتند.
\par 15 اما یهورام پادشاه به یزرعیل مراجعت کرده بود تا از جراحتهایی که ارامیان به او رسانیده بودند وقتی که با حزائیل، پادشاه ارام، جنگ می‌نمود، شفا یابد. پس ییهو گفت: «اگر رای شمااین است، مگذارید که کسی رها شده، از شهربیرون رود مبادا رفته، به یزرعیل خبر برساند.»
\par 16 پس ییهو به ارابه سوار شده، به یزرعیل رفت زیرا که یورام در آنجا بستری بود و اخزیا، پادشاه یهودا برای عیادت یورام فرود آمده بود.
\par 17 پس دیده یانی بر برج یزرعیل ایستاده بود، و جمعیت، ییهو را وقتی که می‌آمد، دید و گفت: «جمعیتی می‌بینم.» و یهورام گفت: «سواری گرفته، به استقبال ایشان بفرست تا بپرسد که آیاسلامتی است؟»
\par 18 پس سواری به استقبال وی رفت و گفت: «پادشاه چنین می‌فرماید که آیاسلامتی است؟» ییهو جواب داد که «تو را باسلامتی چه‌کار است؟ به عقب من برگرد.» و دیده بان خبر داده گفت که «قاصد نزد ایشان رسید، امابرنمی گردد.»
\par 19 پس سوار دیگری فرستاد و اونزد ایشان آمد و گفت: «پادشاه چنین می‌فرماید که آیا سلامتی است؟» ییهو جواب داد: «تو را باسلامتی چه‌کار است؟ به عقب من برگرد.»
\par 20 ودیده بان خبر داده، گفت که «نزد ایشان رسید امابرنمی گردد و راندن مثل راندن ییهو ابن نمشی است زیرا که به دیوانگی می‌راند.»
\par 21 و یهورام گفت: «حاضر کنید.» پس ارابه اورا حاضر کردند و یهورام، پادشاه اسرائیل واخزیا، پادشاه یهودا، هر یک بر ارابه خود بیرون رفتند و به استقبال ییهو بیرون شده، او را در ملک نابوت یزرعیلی یافتند.
\par 22 و چون یهورام، ییهو رادید گفت: «ای ییهو آیا سلامتی است؟» او جواب داد: «چه سلامتی مادامی که زناکاری مادرت ایزابل و جادوگری وی اینقدر زیاد است؟»
\par 23 آنگاه یهورام، دست خود را برگردانیده، فرارکرد و به اخزیا گفت: «ای اخزیا خیانت است.»
\par 24 و ییهو کمان خود را به قوت تمام کشیده، درمیان بازوهای یهورام زد که تیر از دلش بیرون آمدو در ارابه خود افتاد.
\par 25 و ییهو به بدقر، سردارخود گفت: «او را برداشته، در حصه ملک نابوت یزرعیلی بینداز و بیادآور که چگونه وقتی که من وتو با هم از عقب پدرش اخاب، سوار می‌بودیم، خداوند این وحی را درباره او فرمود.
\par 26 خداوندمی گوید: هرآینه خون نابوت و خون پسرانش رادیروز دیدم و خداوند می‌گوید: که در این ملک به تو مکافات خواهم رسانید. پس الان او را بردار وبه موجب کلام خداوند او را در این ملک بینداز.»
\par 27 اما چون اخزیا، پادشاه یهودا این را دید، به راه خانه بوستان فرار کرد و ییهو او را تعاقب نموده، فرمود که او را بزنید و او را نیز در ارابه‌اش به فراز جور که نزد یبلعام است (زدند) و او تامجدو فرار کرده، در آنجا مرد.
\par 28 و خادمانش اورا در ارابه به اورشلیم بردند و او را در مزارخودش در شهر داود با پدرانش دفن کردند.
\par 29 و در سال یازدهم یورام بن اخاب، اخزیا بریهودا پادشاه شد.
\par 30 و چون ییهو به یزرعیل آمد، ایزابل این راشنیده، سرمه به چشمان خود کشیده و سر خود را زینت داده، از پنجره نگریست.
\par 31 و چون ییهوبه دروازه داخل شد، او گفت: «آیا زمری را که آقای خود را کشت، سلامتی بود؟»
\par 32 و او به سوی پنجره نظر افکنده، گفت: «کیست که به طرف من باشد؟ کیست؟» پس دو سه نفر ازخواجگان به سوی او نظر کردند.
\par 33 و او گفت: «اورا بیندازید.» پس او را به زیر انداختند و قدری ازخونش بر دیوار و اسبان پاشیده شد و او را پایمال کرد.
\par 34 و داخل شده، به اکل و شرب مشغول گشت. پس گفت: «این زن ملعون را نظر کنید، و اورا دفن نمایید زیرا که دختر پادشاه است.»
\par 35 اماچون برای دفن کردنش رفتند، جز کاسه سر وپایها و کفهای دست، چیزی از او نیافتند.
\par 36 پس برگشته، وی را خبر دادند. و او گفت: «این کلام خداوند است که به واسطه بنده خود، ایلیای تشبی تکلم نموده، گفت که سگان گوشت ایزابل را در ملک یزرعیل خواهند خورد.و لاش ایزابل مثل سرگین به روی زمین، در ملک یزرعیل خواهد بود، به طوری که نخواهند گفت که این ایزابل است.»
\par 37 و لاش ایزابل مثل سرگین به روی زمین، در ملک یزرعیل خواهد بود، به طوری که نخواهند گفت که این ایزابل است.»
 
\chapter{10}

\par 1 و هفتاد پسر اخاب در سامره بودند. پس ییهو مکتوبی نوشته، به سامره نزدسروران یزرعیل که مشایخ و مربیان پسران اخاب بودند فرستاده، گفت:
\par 2 «الان چون این مکتوب به شما برسد چونکه پسران آقای شما و ارابه‌ها واسبان و شهر حصاردار و اسلحه با شما است،
\par 3 پس بهترین و نیکوترین پسران آقای خود راانتخاب کرده، او را بر کرسی پدرش بنشانید و به جهت خانه آقای خود جنگ نمایید.»
\par 4 اما ایشان به شدت ترسان شدند و گفتند: «اینک دو پادشاه نتوانستند با او مقاومت نمایند، پس ما چگونه مقاومت خواهیم کرد؟»
\par 5 پس ناظر خانه و رئیس شهر و مشایخ و مربیان را نزد ییهو فرستاده، گفتند: «ما بندگان تو هستیم و هر‌چه به ما بفرمایی بجا خواهیم آورد، کسی را پادشاه نخواهیم ساخت. آنچه در نظر تو پسند آید، به عمل آور.»
\par 6 پس مکتوبی دیگر به ایشان نوشت و گفت: «اگرشما با من هستید و سخن مرا خواهید شنید، سرهای پسران آقای خود را بگیرید و فردا مثل این وقت نزد من به یزرعیل بیایید.» و آن پادشاه زادگان که هفتاد نفر بودند، نزد بزرگان شهرکه ایشان را تربیت می‌کردند، می‌بودند.
\par 7 و چون آن مکتوب نزد ایشان رسید، پادشاه زادگان را گرفته، هر هفتاد نفر را کشتند وسرهای ایشان را در سبدها گذاشته، به یزرعیل، نزد وی فرستادند.
\par 8 و قاصدی آمده، او را خبرداد و گفت: «سرهای پسران پادشاه را آوردند.» اوگفت: «آنها را به دو توده نزد دهنه دروازه تا صبح بگذارید.»
\par 9 و بامدادان چون بیرون رفت، بایستادو به تمامی قوم گفت: «شما عادل هستید. اینک من بر آقای خود شوریده، او را کشتم. اما کیست که جمیع اینها را کشته است؟
\par 10 پس بدانید که ازکلام خداوند که خداوند درباره خاندان اخاب گفته است، حرفی به زمین نخواهد افتاد و خداوندآنچه را که به واسطه بنده خود ایلیا گفته، بجاآورده است.»
\par 11 و ییهو جمیع باقی ماندگان خاندان اخاب را که در یزرعیل بودند، کشت، وتمامی بزرگانش و اصدقایش و کاهنانش را تا از برایش کسی باقی نماند.
\par 12 پس برخاسته، و روانه شده، به سامره آمد وچون در راه به بیت عقد شبانان رسید،
\par 13 ییهو به برادران اخزیا، پادشاه یهودا دچار شده، گفت: «شما کیستید؟» گفتند: «برادران اخزیا هستیم ومی آییم تا پسران پادشاه و پسران ملکه را تحیت گوییم.»
\par 14 او گفت: «اینها را زنده بگیرید.» پس ایشان را زنده گرفتند و ایشان را که چهل و دو نفربودند، نزد چاه بیت عقد کشتند که از ایشان احدی رهایی نیافت.
\par 15 و چون از آنجا روانه شد، به یهوناداب بن رکاب که به استقبال او می‌آمد، برخورد و او راتحیت نموده، گفت که «آیا دل تو راست است، مثل دل من با دل تو؟» یهوناداب جواب داد که «راست است.» گفت: «اگر هست، دست خود را به من بده.» پس دست خود را به او داد و او وی را نزدخود به ارابه برکشید.
\par 16 و گفت: «همراه من بیا، وغیرتی که برای خداوند دارم، ببین.» و او را بر ارابه وی سوار کردند.
\par 17 و چون به سامره رسید، تمامی باقی ماندگان اخاب را که در سامره بودند، کشت به حدی که اثر او را نابود ساخت بر‌حسب کلامی که خداوند به ایلیا گفته بود.
\par 18 پس ییهو تمامی قوم را جمع کرده، به ایشان گفت: «اخاب بعل را پرستش قلیل کرد اماییهو او را پرستش کثیر خواهد نمود.
\par 19 پس الان جمیع انبیای بعل و جمیع پرستندگانش و جمیع کهنه او را نزد من بخوانید و احدی از ایشان غایب نباشد زیرا قصد ذبح عظیمی برای بعل دارم. هر که حاضر نباشد زنده نخواهد ماند.» اما ییهو این رااز راه حیله کرد تا پرستندگان بعل را هلاک سازد.
\par 20 و ییهو گفت: «محفلی مقدس برای بعل تقدیس نمایید.» و آن را اعلان کردند.
\par 21 و ییهونزد تمامی اسرائیل فرستاد و تمامی پرستندگان بعل آمدند و احدی باقی نماند که نیامد و به خانه بعل داخل شدند و خانه بعل سرتاسر پر شد. 
\par 22 وبه ناظر مخزن لباس گفت که «برای جمیع پرستندگان بعل لباس بیرون آور.» و او برای ایشان لباس بیرون آورد.
\par 23 و ییهو و یهوناداب بن رکاب به خانه بعل داخل شدند و به پرستندگان بعل گفت: «تفتیش کرده، دریافت کنید که کسی ازبندگان یهوه در اینجا با شما نباشد، مگر بندگان بعل و بس.»
\par 24 پس داخل شدند تا ذبایح وقربانی های سوختنی بگذرانند. و ییهو هشتاد نفربرای خود بیرون در گماشته بود و گفت: «اگریکنفر از اینانی که به‌دست شما سپردم رهایی یابد، خون شما به عوض جان او خواهد بود.»
\par 25 و چون از گذرانیدن قربانی سوختنی فارغ شدند، ییهو به شاطران و سرداران گفت: «داخل شده، ایشان را بکشید و کسی بیرون نیاید.» پس ایشان را به دم شمشیر کشتند و شاطران وسرداران ایشان را بیرون انداختند. پس به شهربیت بعل رفتند.
\par 26 و تماثیل را که در خانه بعل بود، بیرون آورده، آنها را سوزانیدند.
\par 27 و تمثال بعل را شکستند و خانه بعل را منهدم ساخته، آن را تا امروز مزبله ساختند.
\par 28 پس ییهو، اثر بعل رااز اسرائیل نابود ساخت.
\par 29 اما ییهو از پیروی گناهان یربعام بن نباط که اسرائیل را مرتکب گناه ساخته بود برنگشت، یعنی از گوساله های طلا که در بیت ئیل و دان بود.
\par 30 و خداوند به ییهو گفت: «چونکه نیکویی کردی و آنچه در نظر من پسندبود، بجا آوردی و موافق هر‌چه در دل من بود باخانه اخاب عمل نمودی، از این جهت پسران توتا پشت چهارم بر کرسی اسرائیل خواهندنشست.»
\par 31 اما ییهو توجه ننمود تا به تمامی دل خود در شریعت یهوه، خدای اسرائیل، سلوک نماید، و از گناهان یربعام که اسرائیل را مرتکب گناه ساخته بود، اجتناب ننمود.
\par 32 و در آن ایام، خداوند به منقطع ساختن اسرائیل شروع نمود و حزائیل، ایشان را درتمامی حدود اسرائیل می‌زد،
\par 33 یعنی از اردن به طرف طلوع آفتاب، تمامی زمین جلعاد و جادیان و روبینیان و منسیان را از عروعیر که بر وادی ارنون است و جلعاد و باشان.
\par 34 و بقیه وقایع ییهوو هر‌چه کرد و تمامی تهور او، آیا در کتاب تواریخ ایام پادشاهان اسرائیل مکتوب نیست؟
\par 35 پس ییهو با پدران خود خوابید و او را در سامره دفن کردند و پسرش یهواخاز به‌جایش پادشاه شد.و ایامی که ییهو در سامره بر اسرائیل سلطنت نمود، بیست و هشت سال بود.
\par 36 و ایامی که ییهو در سامره بر اسرائیل سلطنت نمود، بیست و هشت سال بود.
 
\chapter{11}

\par 1 و چون عتلیا، مادر اخزیا دید که پسرش مرده است، او برخاست و تمامی خانواده سلطنت را هلاک ساخت.
\par 2 اما یهوشبع دختریورام پادشاه که خواهر اخزیا بود، یوآش پسراخزیا را گرفت، و او را از میان پسران پادشاه که کشته شدند، دزدیده، او را با دایه‌اش در اطاق خوابگاه از عتلیا پنهان کرد و او کشته نشد.
\par 3 و اونزد وی در خانه خداوند شش سال مخفی ماند وعتلیا بر زمین سلطنت می‌نمود.
\par 4 و در سال هفتم، یهویاداع فرستاده، یوزباشیهای کریتیان و شاطران را طلبید و ایشان را نزد خود به خانه خداوند آورده، با ایشان عهدبست و به ایشان در خانه خداوند قسم داد و پسرپادشاه را به ایشان نشان داد.
\par 5 و ایشان را امرفرموده، گفت: «کاری که باید بکنید، این است: یک ثلث شما که در سبت داخل می‌شوید به دیده بانی خانه پادشاه مشغول باشید.
\par 6 و ثلث دیگر به دروازه سور و ثلثی به دروازه‌ای که پشت شاطران است، حاضر باشید، و خانه را دیده بانی نمایید که کسی داخل نشود.
\par 7 و دو دسته شما، یعنی جمیع آنانی که در روز سبت بیرون می‌روید، خانه خداوند را نزد پادشاه دیده بانی نمایید.
\par 8 و هر کدام سلاح خود را به‌دست گرفته، به اطراف پادشاه احاطه نمایید و هر‌که از میان صف‌ها درآید، کشته گردد. و چون پادشاه بیرون رود یا داخل شود نزد او بمانید.»
\par 9 پس یوزباشیها موافق هر‌چه یهویاداع کاهن امر فرمود، عمل نمودند، و هر کدام کسان خود راخواه از آنانی که در روز سبت داخل می‌شدند وخواه از آنانی که در روز سبت بیرون می‌رفتند، برداشته، نزد یهویاداع کاهن آمدند.
\par 10 و کاهن نیزه‌ها و سپرها را که از آن داود پادشاه و در خانه خداوند بود، به یوزباشیها داد.
\par 11 و هر یکی ازشاطران، سلاح خود را به‌دست گرفته، از طرف راست خانه تا طرف چپ خانه به پهلوی مذبح وبه پهلوی خانه، به اطراف پادشاه ایستادند.
\par 12 و اوپسر پادشاه را بیرون آورده، تاج بر سرش گذاشت، و شهادت را به او داد و او را به پادشاهی نصب کرده، مسح نمودند و دستک زده، گفتند: «پادشاه زنده بماند.»
\par 13 و چون عتلیا آواز شاطران و قوم را شنید، نزد قوم به خانه خداوند داخل شد.
\par 14 و دید که اینک پادشاه بر‌حسب عادت، نزد ستون ایستاده. و سروران و کرنانوازان نزد پادشاه بودند و تمامی قوم زمین شادی می‌کردند و کرناها را می‌نواختند. پس عتلیا لباس خود را دریده، صدا زد که خیانت! خیانت!
\par 15 و یهویاداع کاهن، یوزباشیها را که سرداران فوج بودند، امر فرموده، ایشان را گفت: «او را از میان صفها بیرون کنید و هر‌که از عقب اوبرود، به شمشیر کشته شود.» زیرا کاهن فرموده بود که در خانه خداوند کشته نگردد.
\par 16 پس او راراه دادند و از راهی که اسبان به خانه پادشاه می‌آمدند، رفت و در آنجا کشته شد.
\par 17 و یهویادع در میان خداوند و پادشاه و قوم عهد بست تا قوم خداوند باشند و همچنین درمیان پادشاه و قوم.
\par 18 و تمامی قوم زمین به خانه بعل رفته، آن را منهدم ساختند و مذبح هایش وتماثیلش را خرد درهم شکستند. و کاهن بعل، متان را روبروی مذبح‌ها کشتند و کاهن ناظران برخانه خداوند گماشت.
\par 19 و یوزباشیها و کریتیان و شاطران و تمامی قوم زمین را برداشته، ایشان پادشاه را از خانه خداوند به زیر آوردند و به راه دروازه شاطران به خانه پادشاه آمدند و او برکرسی پادشاهان بنشست.
\par 20 و تمامی قوم زمین شادی کردند و شهر آرامی یافت و عتلیا را نزدخانه پادشاه به شمشیر کشتند.و چون یوآش پادشاه شد، هفت ساله بود.
\par 21 و چون یوآش پادشاه شد، هفت ساله بود.
 
\chapter{12}

\par 1 در سال هفتم ییهو، یهوآش پادشاه شدو چهل سال در اورشلیم پادشاهی کرد.
\par 2 و یهوآش آنچه را که در نظر خداوند پسند بود، در تمام روزهایی که یهویاداع کاهن او را تعلیم می‌داد، بجا می‌آورد.
\par 3 مگر این که مکان های بلند برداشته نشد و قوم هنوز در مکان های بلند قربانی می‌گذرانیدند و بخور می‌سوزانیدند.
\par 4 و یهوآش به کاهنان گفت: «تمام نقره موقوفاتی که به خانه خداوند آورده شود، یعنی نقره رایج و نقره هر کس بر‌حسب نفوسی که برای او تقویم شده است، و هر نقره‌ای که در دل کسی بگذرد که آن را به خانه خداوند بیاورد،
\par 5 کاهنان آن را نزد خود بگیرند، هر کس از آشنای خود، وایشان خرابیهای خانه را هر جا که در آن خرابی پیدا کنند، تعمیر نمایند.»
\par 6 اما چنان واقع شد که در سال بیست و سوم یهوآش پادشاه، کاهنان، خرابیهای خانه را تعمیر نکرده بودند.
\par 7 و یهوآش پادشاه، یهویاداع کاهن و سایر کاهنان را خوانده، به ایشان گفت که «خرابیهای خانه را چرا تعمیرنکرده‌اید؟ پس الان نقره‌ای دیگر از آشنایان خودمگیرید بلکه آن را به جهت خرابیهای خانه بدهید.»
\par 8 و کاهنان راضی شدند که نه نقره از قوم بگیرند و نه خرابیهای خانه را تعمیر نمایند.
\par 9 و یهویاداع کاهن صندوقی گرفته و سوراخی در سرپوش آن کرده، آن را به پهلوی مذبح به طرف راست راهی که مردم داخل خانه خداوندمی شدند، گذاشت. و کاهنانی که مستحفظان دربودند، تمامی نقره‌ای را که به خانه خداوندمی آوردند، در آن گذاشتند.
\par 10 و چون دیدند که نقره بسیار در صندوق بود، کاتب پادشاه و رئیس کهنه برآمده، نقره‌ای را که در خانه خداوند یافت می‌شد، در کیسه هابسته، حساب آن را می‌دادند.
\par 11 و نقره‌ای را که حساب آن داده می‌شد، به‌دست کارگذارانی که برخانه خداوند گماشته بودند، می‌سپردند. و ایشان آن را به نجاران و بنایان که در خانه خداوند کارمی کردند، صرف می‌نمودند،
\par 12 و به معماران وسنگ تراشان و به جهت خریدن چوب وسنگهای تراشیده برای تعمیر خرابیهای خانه خداوند، و به جهت هر خرجی که برای تعمیرخانه لازم می‌بود.
\par 13 اما برای خانه خداوندطاسهای نقره و گلگیرها و کاسه‌ها و کرناها و هیچ ظرفی از طلا و نقره از نقدی که به خانه خداوندمی آوردند، ساخته نشد.
\par 14 زیرا که آن را به‌کارگذاران دادند تا خانه خداوند را به آن، تعمیرنمایند.
\par 15 و از کسانی که نقره را به‌دست ایشان می‌دادند تا به‌کارگذاران بسپارند، حساب نمی گرفتند، زیرا که ایشان به امانت رفتارمی نمودند.
\par 16 اما نقره قربانی های جرم و نقره قربانی های گناه را به خانه خداوند نمی آوردند، چونکه از آن کاهنان می‌بود.
\par 17 آنگاه حزائیل، پادشاه ارام برآمده، با جت جنگ نمود و آن را تسخیر کرد. پس حزائیل توجه نموده، به سوی اورشلیم برآمد.
\par 18 ویهوآش، پادشاه یهودا تمامی موقوفاتی را که پدرانش، یهوشافاط و یهورام و اخزیا، پادشاهان یهودا وقف نموده بودند و موقوفات خود وتمامی طلا را که در خزانه های خانه خداوند وخانه پادشاه یافت شد، گرفته، آن را نزد حزائیل، پادشاه ارام فرستاد و او از اورشلیم برفت.
\par 19 و بقیه وقایع یوآش و هر‌چه کرد، آیا در کتاب تواریخ ایام پادشاهان یهودا مکتوب نیست؟
\par 20 و خادمانش برخاسته، فتنه انگیختند ویوآش را در خانه ملو به راهی که به سوی سلی فرود می‌رود، کشتند.زیرا خادمانش، یوزاکاربن شمعت و یهوزاباد بن شومیر، او را زدند که مردو او را با پدرانش در شهر داود دفن کردند وپسرش امصیا در جایش سلطنت نمود.
\par 21 زیرا خادمانش، یوزاکاربن شمعت و یهوزاباد بن شومیر، او را زدند که مردو او را با پدرانش در شهر داود دفن کردند وپسرش امصیا در جایش سلطنت نمود.
 
\chapter{13}

\par 1 در سال بیست و سوم یوآش بن اخزیا، پادشاه یهودا، یهواخاز بن ییهو، براسرائیل در سامره پادشاه شده، هفده سال سلطنت نمود.
\par 2 و آنچه در نظر خداوند ناپسندبود به عمل آورد، و در‌پی گناهان یربعام بن نباطکه اسرائیل را مرتکب گناه ساخته بود، سلوک نموده، از آن اجتناب نکرد.
\par 3 پس غضب خداوندبر اسرائیل افروخته شده، ایشان را به‌دست حزائیل، پادشاه ارام و به‌دست بنهدد، پسرحزائیل، همه روزها تسلیم نمود.
\par 4 و یهواخاز نزدخداوند تضرع نمود و خداوند او را اجابت فرمودزیرا که تنگی اسرائیل را دید که چگونه پادشاه ارام، ایشان را به تنگ می‌آورد.
\par 5 و خداوندنجات‌دهنده‌ای به اسرائیل داد که ایشان از زیردست ارامیان بیرون آمدند و بنی‌اسرائیل مثل ایام سابق در خیمه های خود ساکن شدند.
\par 6 اما ازگناهان خانه یربعام که اسرائیل را مرتکب گناه ساخته بود، اجتناب ننموده، در آن سلوک کردند، و اشیره نیز در سامره ماند.
\par 7 و برای یهواخاز، ازقوم به جز پنجاه سوار و ده ارابه و ده هزار پیاده وانگذاشت زیرا که پادشاه ارام ایشان را تلف ساخته، و ایشان را پایمال کرده، مثل غبارگردانیده بود.
\par 8 و بقیه وقایع یهواخاز و هر‌چه کردو تهور او، آیا در کتاب تواریخ ایام پادشاهان اسرائیل مکتوب نیست؟
\par 9 پس یهواخاز با پدران خود خوابید و او را در سامره دفن کردند وپسرش، یوآش، در جایش سلطنت نمود.
\par 10 و در سال سی و هفتم یوآش، پادشاه یهودا، یهوآش بن یهواخاز بر اسرائیل در سامره پادشاه شد و شانزده سال سلطنت نمود.
\par 11 وآنچه در نظر خداوند ناپسند بود، به عمل آورد واز تمامی گناهان یربعام بن نباط که اسرائیل رامرتکب گناه ساخته بود اجتناب نکرده، در آنهاسلوک می‌نمود.
\par 12 و بقیه وقایع یوآش و هر‌چه کرد و تهور او که چگونه با امصیا، پادشاه یهوداجنگ کرد، آیا در کتاب تواریخ ایام پادشاهان اسرائیل مکتوب نیست؟
\par 13 و یوآش با پدران خود خوابید و یربعام بر کرسی وی نشست ویوآش با پادشاهان اسرائیل در سامره دفن شد.
\par 14 و الیشع به بیماری‌ای که از آن مرد، مریض شد و یوآش، پادشاه اسرائیل، نزد وی فرود شده، بر او بگریست و گفت: «ای پدر من! ای پدر من! ای ارابه اسرائیل و سوارانش!» 
\par 15 و الیشع وی راگفت: «کمان و تیرها را بگیر.» و برای خود کمان وتیرها گرفت.
\par 16 و به پادشاه اسرائیل گفت: «کمان را به‌دست خود بگیر.» پس آن را به‌دست خودگرفت و الیشع دست خود را بر دست پادشاه نهاد.
\par 17 و گفت: «پنجره را به سوی مشرق باز کن.» پس آن را باز کرد و الیشع گفت: «بینداز.» پس انداخت.
\par 18 و گفت: «تیرها را بگیر.» پس گرفت و به پادشاه اسرائیل گفت: «زمین را بزن.» پس سه مرتبه آن را زده، باز ایستاد.
\par 19 و مرد خدابه او خشم نموده، گفت: «می‌بایست پنج شش مرتبه زده باشی آنگاه ارامیان را شکست می‌دادی تا تلف می‌شدند، اما حال ارامیان را فقط سه مرتبه شکست خواهی داد.»
\par 20 و الیشع وفات کرد و او را دفن نمودند و دروقت تحویل سال لشکرهای موآب به زمین درآمدند.
\par 21 و واقع شد که چون مردی را دفن می‌کردند، آن لشکر را دیدند و آن مرده را در قبرالیشع انداختند، و چون آن میت به استخوانهای الیشع برخورد، زنده گشت و به پایهای خودایستاد.
\par 22 و حزائیل، پادشاه ارام، اسرائیل را درتمامی ایام یهواخاز به تنگ آورد.
\par 23 اماخداوند بر ایشان رافت و ترحم نموده، به‌خاطرعهد خود که با ابراهیم و اسحاق و یعقوب بسته بود به ایشان التفات کرد و نخواست ایشان راهلاک سازد، و ایشان را از حضور خود هنوز دورنینداخت.
\par 24 پس حزائیل، پادشاه ارام مرد و پسرش، بنهدد به‌جایش پادشاه شد.و یهوآش بن یهواخاز، شهرهایی را که حزائیل از دست پدرش، یهواخاز به جنگ گرفته بود، از دست بنهدد بن حزائیل باز پس گرفت، و یهوآش سه مرتبه او را شکست داده، شهرهای اسرائیل رااسترداد نمود.
\par 25 و یهوآش بن یهواخاز، شهرهایی را که حزائیل از دست پدرش، یهواخاز به جنگ گرفته بود، از دست بنهدد بن حزائیل باز پس گرفت، و یهوآش سه مرتبه او را شکست داده، شهرهای اسرائیل رااسترداد نمود.
 
\chapter{14}

\par 1 در سال دوم یوآش بن یهواخاز پادشاه اسرائیل، امصیا بن یوآش، پادشاه یهوداآغاز سلطنت نمود.
\par 2 و بیست و پنج ساله بود که پادشاه شد. و بیست و نه سال در اورشلیم پادشاهی کرد و اسم مادرش یهوعدان اورشلیمی بود.
\par 3 و آنچه در نظر خداوند پسند بود، به عمل آورد اما نه مثل پدرش داود بلکه موافق هر‌چه پدرش یوآش کرده بود، رفتار می‌نمود.
\par 4 لیکن مکان های بلند برداشته نشد، و قوم هنوز درمکان های بلند قربانی می‌گذرانیدند و بخورمی سوزانیدند.
\par 5 و هنگامی که سلطنت در دستش مستحکم شد، خادمان خود را که پدرش، پادشاه را کشته بودند، به قتل رسانید.
\par 6 اما پسران قاتلان را نکشت به موجب نوشته کتاب تورات موسی که خداوند امر فرموده و گفته بود پدران به جهت پسران کشته نشوند و پسران به جهت پدران مقتول نگردند، بلکه هر کس به جهت گناه خودکشته شود.
\par 7 و او ده هزار نفر از ادومیان را در وادی ملح کشت و سالع را در جنگ گرفت و آن را تا به امروزیقتئیل نامید.
\par 8 آنگاه امصیا رسولان نزد یهوآش بن یهواخازبن ییهو، پادشاه اسرائیل، فرستاده، گفت: «بیا تا بایکدیگر مقابله نماییم.»
\par 9 و یهوآش پادشاه اسرائیل نزد امصیا، پادشاه یهودا فرستاده، گفت: «شترخار لبنان نزد سرو آزاد لبنان فرستاده، گفت: دختر خود را به پسر من به زنی بده، اما حیوان وحشی‌ای که در لبنان بود، گذر کرده، شترخار راپایمال نمود.
\par 10 ادوم را البته شکست دادی و دلت تو را مغرور ساخته است پس فخر نموده، درخانه خود بمان زیرا برای چه بلا را برای خودبرمی انگیزانی تا خودت و یهودا همراهت بیفتید.»
\par 11 اما امصیا گوش نداد. پس یهوآش، پادشاه اسرائیل برآمد و او و امصیا، پادشاه یهودا دربیت شمس که در یهوداست، با یکدیگر مقابله نمودند.
\par 12 و یهودا از حضور اسرائیل منهزم شده، هر کس به خیمه خود فرار کرد.
\par 13 ویهوآش، پادشاه اسرائیل، امصیا ابن یهوآش بن اخزیا پادشاه یهودا را در بیت شمس گرفت و به اورشلیم آمده، حصار اورشلیم را از دروازه افرایم تا دروازه زاویه، یعنی چهار صد ذراع منهدم ساخت.
\par 14 و تمامی طلا و نقره و تمامی ظروفی را که در خانه خداوند و در خزانه های خانه پادشاه یافت شد، و یرغمالان گرفته، به سامره مراجعت کرد.
\par 15 و بقیه اعمالی را که یهوآش کرد و تهور او وچگونه با امصیا پادشاه یهودا جنگ کرد، آیا درکتاب تواریخ ایام پادشاهان اسرائیل مکتوب نیست؟
\par 16 و یهوآش با پدران خود خوابید و باپادشاهان اسرائیل در سامره دفن شد و پسرش یربعام در جایش پادشاه شد.
\par 17 و امصیا ابن یوآش، پادشاه یهودا، بعد ازوفات یهوآش بن یهواخاز، پادشاه اسرائیل، پانزده سال زندگانی نمود.
\par 18 و بقیه وقایع امصیا، آیا در کتاب تواریخ ایام پادشاهان یهودا مکتوب نیست؟
\par 19 و در اورشلیم بر وی فتنه انگیختند. پس او به لاکیش فرار کرد و از عقبش به لاکیش فرستاده، او را در آنجا کشتند.
\par 20 و او را بر اسبان آوردند و با پدران خود در اورشلیم در شهر داود، دفن شد.
\par 21 و تمامی قوم یهودا، عزریا را که شانزده ساله بود گرفته، او را به‌جای پدرش، امصیا، پادشاه ساختند.
\par 22 او ایلت را بنا کرد و بعداز آنکه پادشاه با پدران خود خوابیده بود، آن رابرای یهودا استرداد ساخت.
\par 23 و در سال پانزدهم امصیا بن یوآش، پادشاه یهودا، یربعام بن یهوآش، پادشاه اسرائیل، درسامره آغاز سلطنت نمود، و چهل و یک سال پادشاهی کرد.
\par 24 و آنچه در نظر خداوند ناپسندبود، به عمل آورده، از تمامی گناهان یربعام بن نباط که اسرائیل را مرتکب گناه ساخته بود، اجتناب ننمود.
\par 25 او حدود اسرائیل را از مدخل حمات تا دریای عربه استرداد نمود، موافق کلامی که یهوه، خدای اسرائیل، به واسطه بنده خود یونس بن امتای نبی که از جت حافر بود، گفته بود.
\par 26 زیرا خداوند دید که مصیبت اسرائیل بسیار تلخ بود چونکه نه محبوس و نه آزادی باقی ماند. و معاونی به جهت اسرائیل وجود نداشت.
\par 27 اما خداوند به محو ساختن نام اسرائیل از زیرآسمان تکلم ننمود لهذا ایشان را به‌دست یربعام بن یوآش نجات داد.
\par 28 و بقیه وقایع یربعام و آنچه کرد و تهور او که چگونه جنگ نمود و چگونه دمشق و حمات راکه از آن یهودا بود، برای اسرائیل استردادساخت، آیا در کتاب تواریخ ایام پادشاهان اسرائیل مکتوب نیست؟پس یربعام با پدران خود، یعنی با پادشاهان اسرائیل خوابید و پسرش زکریا در جایش سلطنت نمود.
\par 29 پس یربعام با پدران خود، یعنی با پادشاهان اسرائیل خوابید و پسرش زکریا در جایش سلطنت نمود.
 
\chapter{15}

\par 1 و در سال بیست و هفتم یربعام، پادشاه اسرائیل، عزریا ابن امصیا، پادشاه یهوداآغاز سلطنت نمود.
\par 2 و شانزده ساله بود که پادشاه شد و پنجاه و دو سال در اورشلیم پادشاهی کرد واسم مادرش یکلیای اورشلیمی بود.
\par 3 و آنچه درنظر خداوند پسند بود، موافق هر‌چه پدرش امصیا کرده بود، بجا آورد.
\par 4 لیکن مکانهای بلند برداشته نشد و قوم هنوز در مکانهای بلندقربانی می‌گذرانیدند و بخور می‌سوزانیدند.
\par 5 وخداوند، پادشاه را مبتلا ساخت که تا روزوفاتش ابرص بود و در مریضخانه‌ای ساکن ماندو یوتام پسر پادشاه بر خانه او بود و بر قوم زمین داوری می‌نمود.
\par 6 و بقیه وقایع عزریا و هرچه کرد، آیا در کتاب تواریخ ایام پادشاهان یهودا مکتوب نیست؟
\par 7 پس عزریا با پدران خود خوابید و او را با پدرانش در شهر داوددفن کردند و پسرش، یوتام در جایش پادشاه بود.
\par 8 در سال سی و هشتم عزریا، پادشاه یهودا، زکریا ابن یربعام بر اسرائیل در سامره پادشاه شد وشش ماه پادشاهی کرد.
\par 9 و آنچه در نظر خداوندناپسند بود، به نحوی که پدرانش می‌کردند، به عمل آورد و از گناهان یربعام بن نباط که اسرائیل را مرتکب گناه ساخته بود، اجتناب ننمود.
\par 10 پس شلوم بن یابیش بر او شوریده، او را در حضور قوم زد و کشت و به‌جایش سلطنت نمود.
\par 11 و بقیه وقایع زکریا اینک در کتاب تواریخ ایام پادشاهان اسرائیل مکتوب است.
\par 12 این کلام خداوند بودکه آن را به ییهو خطاب کرده، گفت: «پسران تو تا پشت چهارم برکرسی اسرائیل خواهند نشست.» پس همچنین به وقوع پیوست.
\par 13 در سال سی و نهم عزیا، پادشاه یهودا، شلوم بن یابیش پادشاه شد و یک ماه در سامره سلطنت نمود.
\par 14 و منحیم بن جادی از ترصه برآمده، به سامره داخل شد. و شلوم بن یابیش رادر سامره زده، او را کشت و به‌جاش سلطنت نمود.
\par 15 و بقیه وقایع شلوم و فتنه‌ای که کرد، اینک در کتاب تواریخ ایام پادشاهان اسرائیل مکتوب است.
\par 16 آنگاه منحیم تفصح را با هر‌چه در آن بود و حدودش را از ترصه زد، از این جهت که برای او باز نکردند، آن را زد، و تمامی زنان حامله‌اش را شکم پاره کرد.
\par 17 در سال سی و نهم عزریا، پادشاه یهودا، منحیم بن جادی، بر اسرائیل پادشاه شد و ده سال در سامره سلطنت نمود.
\par 18 و آنچه در نظرخداوند ناپسند بود، به عمل آورد و از گناهان یربعام بن نباط که اسرائیل را مرتکب گناه ساخته بود، اجتناب ننمود.
\par 19 پس فول، پادشاه آشور، بر زمین هجوم آورد و منحیم، هزار وزنه نقره به فول داد تا دست او با وی باشد و سلطنت رادر دستش استوار سازد.
\par 20 و منحیم این نقد را براسرائیل، یعنی بر جمیع متمولان گذاشت تا هریک از ایشان پنجاه مثقال نقره به پادشاه آشوربدهند. پس پادشاه آشور مراجعت نموده، درزمین اقامت ننمود.
\par 21 و بقیه وقایع منحیم و هرچه کرد، آیا در کتاب تواریخ ایام پادشاهان اسرائیل مکتوب نیست؟
\par 22 پس منحیم با پدران خود خوابید و پسرش فقحیا به‌جایش پادشاه شد.
\par 23 و در سال پنجاهم عزریا، پادشاه یهودا، فقحیا ابن منحیم بر اسرائیل در سامره پادشاه شد و دو سال سلطنت نمود.
\par 24 و آنچه در نظرخداوند ناپسند بود، به عمل آورد و از گناهان یربعام بن نباط که اسرائیل را مرتکب گناه ساخته بود، اجتناب ننمود.
\par 25 و یکی ازسردارانش، فقح بن رملیا بر او شوریده، او رابا ارجوب واریه در سامره در قصر خانه پادشاه زد و با وی پنجاه نفر از بنی جلعادبودند. پس او را کشته، به‌جایش سلطنت نمود.
\par 26 و بقیه وقایع فقحیا و هر‌چه کرد، اینک درکتاب تواریخ ایام پادشاهان اسرائیل مکتوب است.
\par 27 و در سال پنجاه و دوم عزریا، پادشاه یهودا، فقح بن رملیا بر اسرائیل، در سامره پادشاه شد وبیست سال سلطنت نمود.
\par 28 و آنچه در نظرخداوند ناپسند بود، به عمل آورد و از گناهان یربعام بن نباط که اسرائیل را مرتکب گناه ساخته بود، اجتناب ننمود.
\par 29 در ایام فقح، پادشاه اسرائیل، تغلت فلاسر، پادشاه آشور آمده، عیون و آبل بیت معکه ویانوح و قادش و حاصور و جلعاد و جلیل وتمامی زمین نفتالی را گرفته، ایشان را به آشوربه اسیری برد.
\par 30 و در سال بیستم یوتام بن عزیا، هوشع بن ایله، بر فقح بن رملیا بشورید واو را زده، کشت و در جایش سلطنت نمود.
\par 31 وبقیه وقایع فقح و هر‌چه کرد، اینک درکتاب تواریخ ایام پادشاهان اسرائیل مکتوب است.
\par 32 در سال دوم فقح بن رملیا، پادشاه اسرائیل، یوتام بن عزیا، پادشاه یهودا، آغاز سلطنت نمود.
\par 33 او بیست و پنج ساله بود که پادشاه شد وشانزده سال در اورشلیم پادشاهی کرد و اسم مادرش یروشا، دختر صادوق بود.
\par 34 و آنچه درنظر خداوند شایسته بود، موافق هر‌آنچه پدرش عزیا کرد، به عمل آورد.
\par 35 لیکن مکان های بلندبرداشته نشد و قوم در مکان های بلند هنوز قربانی می‌گذرانیدند و بخور می‌سوزانیدند، و او باب عالی خانه خداوند را بنا نمود.
\par 36 و بقیه وقایع یوتام و هر‌چه کرد، آیا در کتاب تواریخ ایام پادشاهان یهودا مکتوب نیست.
\par 37 در آن ایام خداوند شروع نموده، رصین، پادشاه ارام و فقح بن رملیا را بر یهودا فرستاد.پس یوتام با پدران خود خوابید و در شهر پدرش داود با پدران خوددفن شد و پسرش، آحاز به‌جایش سلطنت نمود.
\par 38 پس یوتام با پدران خود خوابید و در شهر پدرش داود با پدران خوددفن شد و پسرش، آحاز به‌جایش سلطنت نمود.
 
\chapter{16}

\par 1 در سال هفدهم فقح بن رملیا، آحاز بن یوتام، پادشاه یهودا آغاز سلطنت نمود. 
\par 2 و آحاز بیست ساله بود که پادشاه شد و شانزده سال در اورشلیم سلطنت نمود و آنچه در نظریهوه خدایش شایسته بود، موافق پدرش داودعمل ننمود.
\par 3 و نه فقط به راه پادشاهان اسرائیل سلوک نمود، بلکه پسر خود را نیز از آتش گذرانید، موافق رجاسات امتهایی که خداوند، ایشان را از حضور بنی‌اسرائیل اخراج نموده بود.
\par 4 و در مکان های بلند و تلها و زیر هر درخت سبزقربانی می‌گذرانید و بخور می‌سوزانید.
\par 5 آنگاه رصین، پادشاه ارام، و فقح بن رملیا، پادشاه اسرائیل، به اورشلیم برای جنگ برآمده، آحاز را محاصره نمودند اما نتوانستند غالب آیند.
\par 6 در آن وقت رصین، پادشاه ارام، ایلت رابرای ارامیان استرداد نمود و یهود را از ایلت اخراج نمود و ارامیان به ایلت داخل شده، تاامروز در آن ساکن شدند.
\par 7 و آحاز رسولان نزدتغلت فلاسر، پادشاه آشور، فرستاده، گفت: «من بنده تو و پسر تو هستم. پس برآمده، مرا از دست پادشاه ارام و از دست پادشاه اسرائیل که به ضدمن برخاسته‌اند، رهایی ده.»
\par 8 و آحاز، نقره وطلایی را که در خانه خداوند و در خزانه های خانه پادشاه یافت شد، گرفته، آن را نزد پادشاه آشور پیشکش فرستاد.
\par 9 پس پادشاه آشور، وی را اجابت نمود و پادشاه آشور به دمشق برآمده، آن را گرفت و اهل آن را به قیر به اسیری برد ورصین را به قتل رسانید.
\par 10 و آحاز پادشاه برای ملاقات تغلت فلاسر، پادشاه آشور، به دمشق رفت و مذبحی را که دردمشق بود، دید و آحاز پادشاه شبیه مذبح و شکل آن را بر‌حسب تمامی صنعتش نزد اوریای کاهن فرستاد.
\par 11 و اوریای کاهن مذبحی موافق آنچه آحاز پادشاه از دمشق فرستاده بود، بنا کرد، واوریای کاهن تا وقت آمدن آحاز پادشاه از دمشق، آن را همچنان ساخت.
\par 12 و چون پادشاه ازدمشق آمد، پادشاه مذبح را دید. و پادشاه به مذبح نزدیک آمده، برآن قربانی گذرانید.
\par 13 و قربانی سوختنی و هدیه آردی خود را سوزانید و هدیه ریختنی خویش را ریخت و خون ذبایح سلامتی خود را بر مذبح پاشید.
\par 14 و مذبح برنجین را که پیش خداوند بود، آن را از روبروی خانه، از میان مذبح خود و خانه خداوند آورده، آن را به طرف شمالی آن مذبح گذاشت.
\par 15 و آحاز پادشاه، اوریای کاهن را امر فرموده، گفت: «قربانی سوختنی صبح و هدیه آردی شام و قربانی سوختنی پادشاه و هدیه آردی او را با قربانی سوختنی تمامی قوم زمین و هدیه آردی ایشان وهدایای ریختنی‌ایشان بر مذبح بزرگ بگذران، وتمامی خون قربانی سوختنی و تمامی خون ذبایح را بر آن بپاش اما مذبح برنجین برای من باشد تامسالت نمایم.»
\par 16 پس اوریای کاهن بر وفق آنچه آحاز پادشاه امر فرموده بود، عمل نمود.
\par 17 و آحاز پادشاه، حاشیه پایه‌ها را بریده، حوض را از آنها برداشت و دریاچه را از بالای گاوان برنجینی که زیر آن بودند، فرود آورد و آن را بر سنگ فرشی گذاشت.
\par 18 و رواق سبت را که در خانه بنا کرده بودند و راهی را که پادشاه ازبیرون به آن داخل می‌شد، در خانه خداوند به‌خاطر پادشاه آشور تغییر داد.
\par 19 و بقیه اعمال آحاز که کرد، آیا در کتاب تواریخ ایام پادشاهان یهودا مکتوب نیست.پس آحاز با پدران خودخوابید و با پدران خویش در شهر داود دفن شد وپسرش حزقیا در جایش پادشاه شد.
\par 20 پس آحاز با پدران خودخوابید و با پدران خویش در شهر داود دفن شد وپسرش حزقیا در جایش پادشاه شد.
 
\chapter{17}

\par 1 در سال دوازدهم آحاز، پادشاه یهودا، هوشع بن ایلا بر اسرائیل در سامره پادشاه شد و نه سال سلطنت نمود.
\par 2 و آنچه درنظر خداوند ناپسند بود، به عمل آورد اما نه مثل پادشاهان اسرائیل که قبل از او بودند.
\par 3 و شلمناسر، پادشاه آشور، به ضد وی برآمده، هوشع، بنده او شد و برای او پیشکش آورد.
\par 4 اماپادشاه آشور در هوشع خیانت یافت زیرا که رسولان نزد سوء، پادشاه مصر فرستاده بود وپیشکش مثل هر سال نزد پادشاه آشور نفرستاده، پس پادشاه آشور او را بند نهاده، در زندان انداخت.
\par 5 و پادشاه آشور بر تمامی زمین هجوم آورده، به سامره برآمد و آن را سه سال محاصره نمود.
\par 6 ودر سال نهم هوشع، پادشاه آشور، سامره را گرفت و اسرائیل را به آشور به اسیری برد و ایشان را درحلح و خابور بر نهر جوزان و در شهرهای مادیان سکونت داد.
\par 7 و از این جهت که بنی‌اسرائیل به یهوه، خدای خود که ایشان را از زمین مصر از زیردست فرعون، پادشاه مصر بیرون آورده بود، گناه ورزیدند و از خدایان دیگر ترسیدند.
\par 8 و درفرایض امتهایی که خداوند از حضور بنی‌اسرائیل اخراج نموده بود و در فرایضی که پادشاهان اسرائیل ساخته بودند، سلوک نمودند.
\par 9 وبنی‌اسرائیل به خلاف یهوه، خدای خود کارهایی را که درست نبود، سر به عمل آوردند، و درجمیع شهرهای خود، از برجهای دیدبانان تاشهرهای حصاردار، مکان های بلند برای خودساختند.
\par 10 و تماثیل و اشیریم بر هر تل بلند وزیر هر درخت سبز برای خویشتن ساختند.
\par 11 ودر آن جایها مثل امتهایی که خداوند از حضورایشان رانده بود، در مکان های بلند بخور سوزانیدند واعمال زشت به‌جا آورده، خشم خداوند را به هیجان آوردند.
\par 12 و بتها را عبادت نمودند که درباره آنها خداوند به ایشان گفته بود، این کار را مکنید.
\par 13 و خداوند به واسطه جمیع انبیا و جمیع رائیان بر اسرائیل و بر یهودا شهادت می‌داد و می‌گفت: «از طریقهای زشت خودبازگشت نمایید و اوامر و فرایض مرا موافق تمامی شریعتی که به پدران شما امر فرمودم و به واسطه بندگان خود، انبیا نزد شما فرستادم، نگاه دارید.»
\par 14 اما ایشان اطاعت ننموده، گردنهای خود را مثل گردنهای پدران ایشان که به یهوه، خدای خود ایمان نیاوردند، سخت گردانیدند.
\par 15 و فرایض او و عهدی که با پدران ایشان بسته، وشهادات را که به ایشان داده بود، ترک نمودند، وپیروی اباطیل نموده، باطل گردیدند و امتهایی راکه به اطراف ایشان بودند و خداوند، ایشان رادرباره آنها امر فرموده بود که مثل آنها عمل منمایید، پیروی کردند.
\par 16 و تمامی اوامر یهوه خدای خود را ترک کرده، بتهای ریخته شده، یعنی دو گوساله برای خود ساختند و اشیره راساخته، به تمامی لشکر آسمان سجده کردند وبعل را عبادت نمودند.
\par 17 و پسران و دختران خودرا از آتش گذرانیدند و فالگیری و جادوگری نموده، خویشتن را فروختند تا آنچه در نظرخداوند ناپسند بود، به عمل آورده، خشم او را به هیجان بیاوردند.
\par 18 پس از این جهت غضب خداوند بر اسرائیل به شدت افروخته شده، ایشان را از حضور خود دور انداخت که جز سبط یهودافقط باقی نماند.
\par 19 اما یهودا نیز اوامر یهوه، خدای خود را نگاه نداشتند بلکه به فرایضی که اسرائیلیان ساخته بودند، سلوک نمودند.
\par 20 پس خداوند تمامی ذریت اسرائیل را ترک نموده، ایشان را ذلیل ساخت و ایشان را به‌دست تاراج کنندگان تسلیم نمود، حتی اینکه ایشان را از حضور خود دورانداخت.
\par 21 زیرا که او اسرائیل را از خاندان داود منشق ساخت و ایشان یربعام بن نباط را به پادشاهی نصب نمودند و یربعام، اسرائیل را از پیروی خداوند برگردانیده، ایشان را مرتکب گناه عظیم ساخت.
\par 22 و بنی‌اسرائیل به تمامی گناهانی که یربعام ورزیده بود سلوک نموده، از آنها اجتناب نکردند.
\par 23 تا آنکه خداوند اسرائیل را موافق آنچه به واسطه جمیع بندگان خود، انبیا گفته بود، از حضور خود دور انداخت. پس اسرائیل از زمین خود تا امروز به آشور جلای وطن شدند.
\par 24 و پادشاه آشور، مردمان از بابل و کوت وعوا و حمات و سفروایم آورده، ایشان را به‌جای بنی‌اسرائیل در شهرهای سامره سکونت داد وایشان سامره را به تصرف آورده، در شهرهایش ساکن شدند.
\par 25 و واقع شد که در ابتدای سکونت ایشان در آنجا از خداوند نترسیدند. لهذا خداوندشیران در میان ایشان فرستاد که بعضی از ایشان راکشتند.
\par 26 پس به پادشاه آشور خبر داده، گفتند: «طوایفی که کوچانیدی و ساکن شهرهای سامره گردانیدی، قاعده خدای آن زمین را نمی دانند و اوشیران در میان ایشان فرستاده است و اینک ایشان را می‌کشند از این جهت که قاعده خدای آن زمین را نمی دانند.»
\par 27 و پادشاه آشور امر فرموده، گفت: «یکی از کاهنانی را که از آنجا کوچانیدید، بفرست تا برود و در آنجا ساکن شود و ایشان راموافق قاعده خدای زمین تعلیم دهد.»
\par 28 پس یکی از کاهنانی که از سامره کوچانیده بودند، آمدو در بیت ئیل ساکن شده، ایشان را تعلیم داد که چگونه خداوند را باید بپرستند.
\par 29 اما هر امت، خدایان خود را ساختند و درخانه های مکان های بلند که سامریان ساخته بودندگذاشتند، یعنی هر امتی در شهر خود که در آن ساکن بودند.
\par 30 پس اهل بابل، سکوت بنوت را واهل کوت، نرجل را و اهل حمات، اشیما راساختند.
\par 31 و عویان، نبحز و ترتاک را ساختند واهل سفروایم، پسران خود را برای ادرملک وعنملک که خدایان سفروایم بودند، به آتش می‌سوزانیدند.
\par 32 پس یهوه را می‌پرستیدند وکاهنان برای مکان های بلند از میان خود ساختندکه برای ایشان در خانه های مکان های بلند قربانی می‌گذرانیدند.
\par 33 پس یهوه را می‌پرستیدند وخدایان خود را نیز بر وفق رسوم امتهایی که ایشان را از میان آنها کوچانیده بودند، عبادت می‌نمودند.
\par 34 ایشان تا امروز بر‌حسب عادت نخستین خود رفتار می‌نمایند و نه از یهوه می‌ترسند و نه موافق فرایض و احکام او و نه مطابق شریعت و اوامری که خداوند به پسران یعقوب که او را اسرائیل نام نهاد، امر نمود، رفتارمی کنند،
\par 35 با آنکه خداوند با ایشان عهد بسته بود و ایشان را امر فرموده، گفته بود: «از خدایان غیر مترسید و آنها را سجده منمایید و عبادت مکنید و برای آنها قربانی مگذرانید.
\par 36 بلکه ازیهوه فقط که شما را از زمین مصر به قوت عظیم وبازوی افراشته بیرون آورد، بترسید و او را سجده نمایید و برای او قربانی بگذرانید.
\par 37 و فرایض واحکام و شریعت و اوامری را که برای شما نوشته است، همیشه اوقات متوجه شده، به‌جا آورید واز خدایان غیر مترسید.
\par 38 و عهدی را که با شمابستم، فراموش مکنید و از خدایان غیر مترسید.
\par 39 زیرا اگر از یهوه، خدای خود بترسید، او شمارا از دست جمیع دشمنان شما خواهد رهانید.»
\par 40 اما ایشان نشنیدند بلکه موافق عادت نخستین خود رفتار نمودند.پس آن امتها، یهوه را می‌پرستیدند و بتهای خود را نیز عبادت می‌کردند و همچنین پسران ایشان و پسران پسران ایشان به نحوی که پدران ایشان رفتار نموده بودند تا امروز رفتار می‌نمایند.
\par 41 پس آن امتها، یهوه را می‌پرستیدند و بتهای خود را نیز عبادت می‌کردند و همچنین پسران ایشان و پسران پسران ایشان به نحوی که پدران ایشان رفتار نموده بودند تا امروز رفتار می‌نمایند.
 
\chapter{18}

\par 1 و در سال سوم هوشع بن ایله، پادشاه اسرائیل، حزقیا ابن آحاز، پادشاه یهوداآغاز سلطنت نمود.
\par 2 او بیست و پنج ساله بود که پادشاه شد و بیست و نه سال در اورشلیم سلطنت کرد و اسم مادرش ابی، دختر زکریا بود.
\par 3 و آنچه در نظر خداوند پسند بود، موافق هر‌چه پدرش داود کرده بود، به عمل آورد.
\par 4 او مکان های بلندرا برداشت و تماثیل را شکست و اشیره را قطع نمود و مار برنجین را که موسی ساخته بود، خردکرد زیرا که بنی‌اسرائیل تا آن زمان برایش بخورمی سوزانیدند. و او آن را نحشتان نامید.
\par 5 او بریهوه، خدای اسرائیل توکل نمود و بعد از او ازجمیع پادشاهان یهودا کسی مثل او نبود و نه ازآنانی که قبل از او بودند.
\par 6 و به خداوند چسپیده، از پیروی او انحراف نورزید و اوامری را که خداوند به موسی‌امر فرموده بود، نگاه داشت.
\par 7 وخداوند با او می‌بود و به هر طرفی که رو می‌نمود، فیروز می‌شد و بر پادشاه آشور عاصی شده، او راخدمت ننمود.
\par 8 او فلسطینیان را تا غزه وحدودش و از برجهای دیده بانان تا شهرهای حصاردار شکست داد.
\par 9 و در سال چهارم حزقیا پادشاه که سال هفتم هوشع بن ایله، پادشاه اسرائیل بود، شلمناسر، پادشاه آشور به سامره برآمده، آن رامحاصره کرد. 
\par 10 و در آخر سال سوم در سال ششم حزقیا آن را گرفتند، یعنی در سال نهم هوشع، پادشاه اسرائیل، سامره گرفته شد.
\par 11 وپادشاه آشور، اسرائیل را به آشور کوچانیده، ایشان را در حلح و خابور، نهر جوزان، و درشهرهای مادیان برده، سکونت داد.
\par 12 از این جهت که آواز یهوه، خدای خود را نشنیده بودندو از عهد او و هر‌چه موسی، بنده خداوند، امرفرموده بود، تجاوز نمودند و آن را اطاعت نکردندو به عمل نیاوردند.
\par 13 و در سال چهاردهم حزقیا پادشاه، سنحاریب، پادشاه آشور بر تمامی شهرهای حصاردار یهودا برآمده، آنها را تسخیر نمود.
\par 14 وحزقیا پادشاه یهودا نزد پادشاه آشور به لاکیش فرستاده، گفت: «خطا کردم. از من برگرد و آنچه راکه بر من بگذاری، ادا خواهم کرد.» پس پادشاه آشور سیصد وزنه نقره و سی وزنه طلا بر حزقیاپادشاه یهودا گذاشت.
\par 15 و حزقیا تمامی نقره‌ای را که در خانه خداوند و در خزانه های خانه پادشاه یافت شد، داد.
\par 16 در آن وقت حزقیا طلارا از درهای هیکل خداوند و از ستونهایی که حزقیا، پادشاه یهودا آنها را به طلا پوشانیده بودکنده، آن را به پادشاه آشور داد.
\par 17 و پادشاه آشور، ترتان و ربساریس وربشاقی را از لاکیش نزد حزقیای پادشاه به اورشلیم با موکب عظیم فرستاد. و ایشان برآمده، به اورشلیم رسیدند و چون برآمدند، رفتند و نزدقنات برکه فوقانی که به‌سر راه مزرعه گازر است، ایستادند.
\par 18 و چون پادشاه را خواندند، الیاقیم بن حلقیا که ناظر خانه بود و شبنای کاتب و یوآخ بن آساف وقایع نگار، نزد ایشان بیرون آمدند.
\par 19 و ربشاقی به ایشان گفت: «به حزقیا بگویید: سلطان عظیم، پادشاه آشور چنین می‌گوید: این اعتماد شما که بر آن توکل می‌نمایی، چیست؟
\par 20 تو سخن می‌گویی، اما مشورت و قوت جنگ تو، محض سخن باطل است. الان کیست که بر اوتوکل نموده‌ای که بر من عاصی شده‌ای.
\par 21 اینک حال بر عصای این نی خرد شده، یعنی بر مصرتوکل می‌نمایی که اگر کسی بر آن تکیه کند، به‌دستش فرو رفته، آن را مجروح می‌سازد. همچنان است فرعون، پادشاه مصر برای همگانی که بر وی توکل می‌نمایند.
\par 22 و اگر مرا گویید که بر یهوه، خدای خود توکل داریم، آیا او آن نیست که حزقیا مکان های بلند و مذبح های او را برداشته است و به یهودا و اورشلیم گفته که پیش این مذبح در اورشلیم سجده نمایید؟
\par 23 پس حال با آقایم، پادشاه آشور شرط ببند و من دو هزار اسب به تومی دهم. اگر از جانب خود سواران بر آنها توانی گذاشت!
\par 24 پس چگونه روی یک پاشا ازکوچکترین بندگان آقایم را خواهی برگردانید و برمصر به جهت ارابه‌ها و سواران توکل داری؟
\par 25 وآیا من الان بی‌اذن خداوند بر این مکان به جهت خرابی آن برآمده‌ام، خداوند مرا گفته است بر این زمین برآی و آن را خراب کن.»
\par 26 آنگاه الیاقیم بن حلقیا و شبنا و یوآخ به ربشاقی گفتند: «تمنا اینکه با بندگانت به زبان ارامی گفتگو نمایی که آن را می‌فهمیم و با ما به زبان یهود در گوش مردمی که بر حصارند، گفتگومنمای.»
\par 27 ربشاقی به ایشان گفت: «آیا آقایم مرانزد آقایت و تو فرستاده است تا این سخنان رابگویم؟ مگر مرا نزد مردانی که بر حصارنشسته‌اند، نفرستاده، تا ایشان با شما نجاست خود را بخوردند و بول خود را بنوشند؟»
\par 28 پس ربشاقی ایستاد و به آواز بلند به زبان یهود صدا زد و خطاب کرده، گفت: «کلام سلطان عظیم، پادشاه آشور را بشنوید.
\par 29 پادشاه چنین می‌گوید: حزقیا شما را فریب ندهد زیرا که اوشما را نمی تواند از دست وی برهاند.
\par 30 و حزقیاشما را بر یهوه مطمئن نسازد و نگوید که یهوه، البته ما را خواهد رهانید و این شهر به‌دست پادشاه آشور تسلیم نخواهد شد.
\par 31 به حزقیاگوش مدهید زیرا که پادشاه آشور چنین می‌گوید: با من صلح کنید و نزد من بیرون آیید تا هرکس ازمو خود و هرکس از انجیر خویش بخورد وهرکس از آب چشمه خود بنوشد.
\par 32 تا بیایم وشما را به زمین مانند زمین خودتان بیاورم، یعنی به زمین غله و شیره و زمین نان و تاکستانها و زمین زیتونهای نیکو و عسل تا زنده بمانید و نمیرید. پس به حزقیا گوش مدهید زیرا که شما را فریب می‌دهد و می‌گوید: یهوه ما را خواهد رهانید.
\par 33 آیا هیچکدام از خدایان امتها، هیچ وقت زمین خود را از دست پادشاه آشور رهانیده است؟
\par 34 خدایان حمات و ارفاد کجایند؟ و خدایان سفروایم و هینع و عوا کجا؟ و آیا سامره را ازدست من رهانیده‌اند؟
\par 35 از جمیع خدایان این زمینها کدامند که زمین خویش را از دست من نجات داده‌اند تا یهوه، اورشلیم را از دست من نجات دهد؟»
\par 36 اما قوم سکوت نموده، به او هیچ جواب ندادند زیرا که پادشاه امر فرموده بود و گفته بود که او را جواب ندهید.پس الیاقیم بن حلقیا که ناظر خانه بود و شبنه کاتب و یوآخ بن آساف وقایع نگار با جامه دریده نزد حزقیا آمدند وسخنان ربشاقی را به او باز‌گفتند.
\par 37 پس الیاقیم بن حلقیا که ناظر خانه بود و شبنه کاتب و یوآخ بن آساف وقایع نگار با جامه دریده نزد حزقیا آمدند وسخنان ربشاقی را به او باز‌گفتند.
 
\chapter{19}

\par 1 و واقع شد که چون حزقیای پادشاه این را شنید، لباس خود را چاک زده، وپلاس پوشیده، به خانه خداوند داخل شد.
\par 2 والیاقیم، ناظر خانه و شبنه کاتب و مشایخ کهنه راملبس به پلاس نزد اشعیا ابن آموص نبی فرستاده،
\par 3 به وی گفتند: «حزقیا چنین می‌گوید که «امروزروز تنگی و تادیب و اهانت است زیرا که پسران به فم رحم رسیده‌اند و قوت زاییدن نیست.
\par 4 شاید یهوه خدایت تمامی سخنان ربشاقی را که آقایش، پادشاه آشور، او را برای اهانت نمودن خدای حی فرستاده است، بشنود و سخنانی را که یهوه، خدایت شنیده است، توبیخ نماید. پس برای بقیه‌ای که یافت می‌شوند، تضرع نما.»
\par 5 وبندگان حزقیای پادشاه نزد اشعیا آمدند.
\par 6 و اشعیابه ایشان گفت: «به آقای خود چنین گویید که خداوند چنین می‌فرماید: از سخنانی که شنیدی که بندگان پادشاه آشور به آنها به من کفرگفته‌اند، مترس.
\par 7 همانا روحی بر او می‌فرستم که خبری شنیده، به ولایت خود خواهد برگشت و اورا در ولایت خودش به شمشیر هلاک خواهم ساخت.»
\par 8 پس ربشاقی مراجعت کرده، پادشاه آشور را یافت که با لبنه جنگ می‌کرد، زیرا شنیده بود که ازلاکیش کوچ کرده است.
\par 9 و درباره ترهاقه، پادشاه حبش، خبری شنیده بود که به جهت مقاتله با توبیرون آمده است (پس چون شنید) بار دیگرایلچیان نزد حزقیا فرستاده، گفت:
\par 10 «به حزقیا، پادشاه یهودا چنین گویید: «خدای تو که به اوتوکل می‌نمایی، تو را فریب ندهد و نگوید که اورشلیم به‌دست پادشاه آشور تسلیم نخواهدشد.
\par 11 اینک تو شنیده‌ای که پادشاهان آشور باهمه ولایتها چه کرده و چگونه آنها را بالکل هلاک ساخته‌اند، و آیا تو رهایی خواهی یافت؟
\par 12 آیا خدایان امتهایی که پدران من، ایشان راهلاک ساختند، مثل جوزان و حاران و رصف وبنی عدن که در تلسار می‌باشند، ایشان را نجات دادند؟
\par 13 پادشاه حمات کجاست؟ و پادشاه ارفاد و پادشاه شهر سفروایم و هینع و عوا؟»
\par 14 و حزقیا مکتوب را از دست ایلچیان گرفته، آن را خواند و حزقیا به خانه خداوند درآمده، آن را به حضور خداوند پهن کرد.
\par 15 و حزقیا نزدخداوند دعا نموده، گفت: «ای یهوه، خدای اسرائیل که بر کروبیان جلوس می‌نمایی، تویی که به تنهایی بر تمامی ممالک جهان خدا هستی و توآسمان و زمین را آفریده‌ای.
\par 16 ‌ای خداوند گوش خود را فرا‌گرفته، بشنو. ای خداوند چشمان خودرا گشوده، ببین و سخنان سنحاریب را که به جهت اهانت نمودن خدای حی فرستاده است، استماع نما.
\par 17 ‌ای خداوند، راست است که پادشاهان آشور امت‌ها و زمین ایشان را خراب کرده است،
\par 18 و خدایان ایشان را به آتش انداخته، زیرا که خدا نبودند، بلکه ساخته دست انسان از چوب وسنگ. پس به این سبب آنها را تباه ساختند.
\par 19 پس حال یهوه، خدای ما، ما را از دست اورهایی ده تا جمیع ممالک جهان بدانند که تو تنهاای یهوه، خدا هستی.»
\par 20 پس اشعیا ابن آموص نزد حزقیا فرستاده، گفت: «یهوه، خدای اسرائیل، چنین می‌گوید: آنچه را که درباره سنحاریب، پادشاه آشور، نزدمن دعا نمودی اجابت کردم.
\par 21 کلامی که خداوند درباره‌اش گفته، این است: آن باکره، دختر صهیون، تو را حقیر شمرده، استهزا نموده است و دختر اورشلیم سر خود را به تو جنبانیده است.
\par 22 کیست که او را اهانت کرده، کفر گفته‌ای و کیست که بر وی آواز بلند کرده، چشمان خودرا به علیین افراشته‌ای؟ مگر قدوس اسرائیل نیست؟
\par 23 به واسطه رسولانت، خداوند را اهانت کرده، گفته‌ای: به کثرت ارابه های خود بر بلندی کوهها و به اطراف لبنان برآمده‌ام و بلندترین سروهای آزادش و بهترین صنوبرهایش را قطع نموده، به بلندی اقصایش و به درختستان بوستانش داخل شده‌ام.
\par 24 و من، حفره کنده، آب غریب نوشیدم و به کف پای خود تمامی نهرهای مصر را خشک خواهم کرد.
\par 25 آیا نشنیده‌ای که من این را از زمان سلف کرده‌ام و از ایام قدیم صورت داده‌ام و الان، آن را به وقوع آورده‌ام تا توبه ظهور آمده و شهرهایی حصاردار را خراب نموده، به توده های ویران مبدل سازی؟
\par 26 از این جهت، ساکنان آنها کم قوت بوده، ترسان و خجل شدند، مثل علف صحرا و گیاه سبز و علف پشت بام و مثل غله‌ای که پیش از رسیدنش پژمرده شود، گردیدند.
\par 27 «اما من نشستن تو را و خروج و دخولت وخشمی را که بر من داری، می‌دانم.
\par 28 چونکه خشمی که بر من داری و غرور تو، به گوش من برآمده است. بنابراین مهار خود را به بینی تو ولگام خود را به لبهایت گذاشته، تو را به راهی که آمده‌ای، برخواهم گردانید.
\par 29 «و علامت، برای تو این خواهد بود که امسال غله خودرو خواهید خورد و سال دوم آنچه از آن بروید و در سال سوم بکارید و بدرویدو تاکستانها غرس نموده، میوه آنها را بخورید.
\par 30 و بقیه‌ای که از خاندان یهودا رستگار شوند، باردیگر به پایین ریشه خواهند زد و به بالا میوه خواهند‌آورد.
\par 31 زیرا که بقیه‌ای از اورشلیم ورستگاران از کوه صهیون بیرون خواهند آمد. غیرت یهوه این را بجا خواهد آورد.
\par 32 «بنابراین خداوند درباره پادشاه آشورچنین می‌گوید که «به این شهر داخل نخواهد شدو به اینجا تیر نخواهد انداخت و در مقابلش با سپرنخواهد آمد و منجنیق را در‌پیش آن بر نخواهدافراشت.
\par 33 به راهی که آمده است به همان برخواهد گشت و به این شهر داخل نخواهد شد. خداوند این را می‌گوید.
\par 34 زیرا که این شهر راحمایت کرده، به‌خاطر خود و به‌خاطر بنده خویش داود، آن را نجات خواهم داد.»
\par 35 پس فرشته خداوند در آن شب بیرون آمده، صد و هشتاد و پنج هزار نفر از اردوی آشور را زدو بامدادان چون برخاستند، اینک جمیع آنهالاشه های مرده بودند.
\par 36 و سنحاریب، پادشاه آشور کوچ کرده، روانه گردید و برگشته، درنینوی ساکن شد.و واقع شد که چون او درخانه خدای خویش، نسروک عبادت می‌کرد، پسرانش ادرملک و شرآصر او را به شمشیر زدند، و ایشان به زمین آرارات فرار کردند و پسرش آسرحدون به‌جایش سلطنت نمود.
\par 37 و واقع شد که چون او درخانه خدای خویش، نسروک عبادت می‌کرد، پسرانش ادرملک و شرآصر او را به شمشیر زدند، و ایشان به زمین آرارات فرار کردند و پسرش آسرحدون به‌جایش سلطنت نمود.
 
\chapter{20}

\par 1 در آن ایام، حزقیا بیمار و مشرف به موت شد و اشعیا ابن آموص نبی نزدوی آمده، او را گفت: «خداوند چنین می‌گوید: تدارک خانه خود را ببین زیرا که می‌میری و زنده نخواهی ماند.»
\par 2 آنگاه او روی خود را به سوی دیوار برگردانید و نزد خداوند دعا نموده، گفت:
\par 3 «ای خداوند مسالت اینکه بیاد آوری که چگونه به حضور تو به امانت و به دل کامل سلوک نموده‌ام و آنچه در نظر تو پسند بوده است، بجا آورده‌ام.» پس حزقیا زارزار بگریست.
\par 4 و واقع شد قبل از آنکه اشعیا از وسط شهربیرون رود، که کلام خداوند بر وی نازل شده، گفت:
\par 5 «برگرد و به پیشوای قوم من حزقیا بگو: خدای پدرت، داود چنین می‌گوید: دعای تو راشنیدم و اشکهای تو را دیدم. اینک تو را شفاخواهم داد و در روز سوم به خانه خداوند داخل خواهی شد. 
\par 6 و من بر روزهای تو پانزده سال خواهم افزود، و تو را و این شهر را از دست پادشاه آشور خواهم رهانید، و این شهر را به‌خاطر خود و به‌خاطر بنده خود، داود حمایت خواهم کرد.»
\par 7 و اشعیا گفت که «قرصی از انجیربگیرید.» و ایشان آن را گرفته، بر دمل گذاشتند که شفا یافت.
\par 8 و حزقیا به اشعیا گفت: «علامتی که خداوند مرا شفا خواهد بخشید و در روز سوم به خانه خداوند خواهم برآمد، چیست؟»
\par 9 و اشعیا گفت: «علامت از جانب خداوند که خداوند این کلام راکه گفته است، بجا خواهد آورد، این است: آیاسایه ده درجه پیش برود یا ده درجه برگردد؟»
\par 10 حزقیا گفت: «سهل است که سایه ده درجه پیش برود. نی، بلکه سایه ده درجه به عقب برگردد.»
\par 11 پس اشعیای نبی از خداوند استدعانمود و سایه را از درجاتی که بر ساعت آفتابی آحاز پایین رفته بود، ده درجه برگردانید.
\par 12 و در آن زمان، مرودک بلدان بن بلدان، پادشاه بابل، رسایل و هدیه نزد حزقیا فرستاد زیراشنیده بود که حزقیا بیمار شده است.
\par 13 و حزقیاایشان را اجابت نمود و تمامی خانه خزانه های خود را از نقره و طلا و عطریات و روغن معطر وخانه اسلحه خویش و هرچه را که در خزاین اویافت می‌شد، به ایشان نشان داد، و در خانه‌اش ودر تمامی مملکتش چیزی نبود که حزقیا آن را به ایشان نشان نداد.
\par 14 پس اشعیای نبی نزد حزقیای پادشاه آمده، وی را گفت: «این مردمان چه گفتند؟ و نزد تو از کجا آمدند؟» حزقیا جواب داد: «ازجای دور، یعنی از بابل آمده‌اند.»
\par 15 او گفت: «درخانه تو چه دیدند؟» حزقیا جواب داد: «هرچه درخانه من است، دیدند و چیزی در خزاین من نیست که به ایشان نشان ندادم.»
\par 16 پس اشعیا به حزقیا گفت: «کلام خداوند رابشنو:
\par 17 اینک روزها می‌آید که هرچه در خانه توست و آنچه پدرانت تا امروز ذخیره کرده‌اند، به بابل برده خواهد شد. و خداوند می‌گوید که چیزی باقی نخواهد ماند.
\par 18 و بعضی از پسرانت را که از تو پدید آیند و ایشان را تولید نمایی، خواهند گرفت و در قصر پادشاه بابل، خواجه خواهند شد.»
\par 19 حزقیا به اشعیا گفت: «کلام خداوند که گفتی نیکوست.» و دیگر گفت: «هرآینه در ایام من سلامتی و امان خواهد بود.»
\par 20 و بقیه وقایع حزقیا و تمامی تهور او وحکایت حوض و قناتی که ساخت و آب را به شهر آورد، آیا در کتاب تواریخ ایام پادشاهان یهودا مکتوب نیست؟پس حزقیا با پدران خود خوابید و پسرش، منسی به‌جایش سلطنت نمود.
\par 21 پس حزقیا با پدران خود خوابید و پسرش، منسی به‌جایش سلطنت نمود.
 
\chapter{21}

\par 1 منسی دوازده ساله بود که پادشاه شد وپنجاه و پنج سال در اورشلیم سلطنت نمود. و اسم مادرش حفصیبه بود.
\par 2 و آنچه درنظر خداوند ناپسند بود، موافق رجاسات امت هایی که خداوند، آنها را از حضوربنی‌اسرائیل اخراج کرده بود، عمل نمود.
\par 3 زیرامکانهای بلند را که پدرش، حزقیا خراب کرده بود، بار دیگر بنا کرد و مذبح‌ها برای بعل بنا نمودو اشیره را به نوعی که اخاب، پادشاه اسرائیل ساخته بود، ساخت و به تمامی لشکر آسمان سجده نموده، آنها را عبادت کرد.
\par 4 و مذبح‌ها درخانه خداوند بنا نمود که درباره‌اش خداوند گفته بود: «اسم خود را در اورشلیم خواهم گذاشت.»
\par 5 و مذبح‌ها برای تمامی لشکر آسمان در هر دوصحن خانه خداوند بنا نمود.
\par 6 و پسر خود را ازآتش گذرانید و فالگیری و افسونگری می‌کرد و بااصحاب اجنه و جادوگران مراوده می‌نمود. و درنظر خداوند شرارت بسیار ورزیده، خشم او را به هیجان آورد.
\par 7 و تمثال اشیره را که ساخته بود، درخانه‌ای که خداوند درباره‌اش به داود و پسرش، سلیمان گفته بود که «در این خانه و در اورشلیم که آن را از تمامی اسباط اسرائیل برگزیده‌ام، اسم خود را تا به ابد خواهم گذاشت برپا نمود.
\par 8 وپایهای اسرائیل را از زمینی که به پدران ایشان داده‌ام بار دیگر آواره نخواهم گردانید. به شرطی که توجه نمایند تا بر‌حسب هرآنچه به ایشان امرفرمودم و بر‌حسب تمامی شریعتی که بنده من، موسی به ایشان امر فرموده بود، رفتار نمایند.»
\par 9 اما ایشان اطاعت ننمودند زیرا که منسی، ایشان را اغوا نمود تا از امتهایی که خداوند پیش بنی‌اسرائیل هلاک کرده بود، بدتر رفتار نمودند.
\par 10 و خداوند به واسطه بندگان خود، انبیا تکلم نموده، گفت:
\par 11 «چونکه منسی، پادشاه یهودا، این رجاسات را بجا آورد و بدتر از جمیع اعمال اموریانی که قبل از او بودند عمل نمود، و به بتهای خود، یهودا را نیز مرتکب گناه ساخت،
\par 12 بنابراین یهوه، خدای اسرائیل چنین می‌گوید: اینک من بر اورشلیم و یهودا بلا خواهم رسانید که گوشهای هرکه آن را بشنود، صدا خواهد کرد.
\par 13 و بر اورشلیم، ریسمان سامره و ترازوی خانه اخاب را خواهم کشید و اورشلیم را پاک خواهم کرد، به طوری که کسی پشقاب را زدوده وواژگون ساخته، آن را پاک می‌کند.
\par 14 و بقیه میراث خود را پراکنده خواهم ساخت و ایشان رابه‌دست دشمنان ایشان تسلیم خواهم نمود، وبرای جمیع دشمنانشان یغما و غارت خواهندشد،
\par 15 چونکه آنچه در نظر من ناپسند است، به عمل آوردند و از روزی که پدران ایشان از مصربیرون آمدند تا امروز، خشم مرا به هیجان آوردند.»
\par 16 و علاوه براین، منسی خون بی‌گناهان را ازحد زیاده ریخت تا اورشلیم را سراسر پر کرد، سوای گناه او که یهودا را به آن مرتکب گناه ساخت تا آنچه در نظر خداوند ناپسند است بجاآورند.
\par 17 و بقیه وقایع منسی و هرچه کرد و گناهی که مرتکب آن شد، آیا در کتاب تواریخ ایام پادشاهان یهودا مکتوب نیست؟
\par 18 پس منسی باپدران خود خوابید و در باغ خانه خود، یعنی درباغ عزا دفن شد و پسرش، آمون، به‌جایش پادشاه شد.
\par 19 آمون بیست و دو ساله بود که پادشاه شد ودو سال در اورشلیم سلطنت نمود و اسم مادرش مشلمت، دختر حاروص، از یطبه بود.
\par 20 و آنچه در نظر خداوند ناپسند بود، موافق آنچه پدرش منسی کرد، عمل نمود.
\par 21 و به تمامی طریقی که پدرش به آن سلوک نموده بود، رفتار کرد، وبت هایی را که پدرش پرستید، عبادت کرد و آنهارا سجده نمود.
\par 22 و یهوه، خدای پدران خود راترک کرده، به طریق خداوند سلوک ننمود.
\par 23 پس خادمان آمون بر او شوریدند و پادشاه را در خانه‌اش کشتند.
\par 24 اما اهل زمین همه آنانی را که بر آمون پادشاه، شوریده بودند به قتل رسانیدند واهل زمین پسرش، یوشیا را در جایش به پادشاهی نصب کردند.
\par 25 و بقیه اعمالی که آمون بجا آورد، آیا در کتاب تواریخ ایام پادشاهان یهودا مکتوب نیست؟و در قبر خود در باغ عزا دفن شد وپسرش یوشیا به‌جایش سلطنت نمود.
\par 26 و در قبر خود در باغ عزا دفن شد وپسرش یوشیا به‌جایش سلطنت نمود.
 
\chapter{22}

\par 1 یوشیا هشت ساله بود که پادشاه شد ودر اورشلیم سی و یک سال سلطنت نمود. و اسم مادرش یدیده، دختر عدایه، ازبصقت بود.
\par 2 و آنچه را که در نظر خداوند پسندبود، به عمل آورد، و به تمامی طریق پدر خود، داود سلوک نموده، به طرف راست یا چپ انحراف نورزید.
\par 3 و در سال هجدهم یوشیا پادشاه واقع شد که پادشاه، شافان بن اصلیا بن مشلام کاتب را به خانه خداوند فرستاده، گفت:
\par 4 «نزد حلقیا رئیس کهنه برو و او نقره‌ای را که به خانه خداوند آورده می‌شود و مستحفظان در، آن را از قوم جمع می‌کنند، بشمارد.
\par 5 و آن را به‌دست سرکارانی که بر خانه خداوند گماشته شده‌اند، بسپارند تا ایشان آن را به کسانی که در خانه خداوند کار می‌کنند، به جهت تعمیر خرابیهای خانه بدهند،
\par 6 یعنی به نجاران و بنایان و معماران، و تا چوبها و سنگهای تراشیده به جهت تعمیر خانه بخرند.»
\par 7 امانقره‌ای را که به‌دست ایشان سپردند، حساب نکردند زیرا که به امانت رفتار نمودند.
\par 8 و حلقیا، رئیس کهنه، به شافان کاتب گفت: «کتاب تورات را در خانه خداوند یافته‌ام.» وحلقیا آن کتاب را به شافان داد که آن را خواند.
\par 9 وشافان کاتب نزد پادشاه برگشت و به پادشاه خبرداده، گفت: «بندگانت، نقره‌ای را که در خانه خداوند یافت شد، بیرون آوردند و آن را به‌دست سرکارانی که بر خانه خداوند گماشته بودند، سپردند.»
\par 10 و شافان کاتب، پادشاه را خبر داده، گفت: «حلقیا، کاهن، کتابی به من داده است.» پس شافان آن را به حضور پادشاه خواند.
\par 11 پس چون پادشاه سخنان سفر تورات راشنید، لباس خود را درید.
\par 12 و پادشاه، حلقیای کاهن و اخیقام بن شافان و عکبور بن میکایا وشافان کاتب و عسایا، خادم پادشاه را امر فرموده، گفت:
\par 13 «بروید و از خداوند برای من و برای قوم و برای تمامی یهودا درباره سخنانی که در این کتاب یافت می‌شود، مسالت نمایید، زیرا غضب خداوند که بر ما افروخته شده است، عظیم می‌باشد، از این جهت که پدران ما به سخنان این کتاب گوش ندادند تا موافق هرآنچه درباره مامکتوب است، عمل نمایند.»
\par 14 پس حلقیای کاهن و اخیقام و عکبور وشافان و عسایا نزد حلده نبیه، زن شلام بن تقوه بن حرحس لباس دار، رفتند و او در محله دوم اورشلیم ساکن بود. و با وی سخن‌گفتند.
\par 15 و او به ایشان گفت: «یهوه، خدای اسرائیل چنین می‌گوید: به کسی‌که شما را نزد من فرستاده است، بگویید:
\par 16 خداوند چنین می‌گوید: اینک من بلایی بر این مکان و ساکنانش خواهم رسانید، یعنی تمامی سخنان کتاب را که پادشاه یهوداخوانده است،
\par 17 چونکه مرا ترک کرده، برای خدایان دیگر بخور‌سوزانیدند تا به تمامی اعمال دستهای خود، خشم مرا به هیجان بیاورند. پس غضب من بر این مکان مشتعل شده، خاموش نخواهد شد.
\par 18 لیکن به پادشاه یهودا که شما را به جهت مسالت نمودن از خداوند فرستاده است، چنین بگویید: یهوه، خدای اسرائیل چنین می‌فرماید: درباره سخنانی که شنیده‌ای
\par 19 چونکه دل تو نرم بود و هنگامی که کلام مرادرباره این مکان و ساکنانش شنیدی که ویران ومورد لعنت خواهند شد، به حضور خداوندمتواضع شده، لباس خود را دریدی، و به حضورمن گریستی، بنابراین خداوند می‌گوید: من نیز تورا اجابت فرمودم.لهذا اینک من، تو را نزدپدرانت جمع خواهم کرد و در قبر خود به سلامتی گذارده خواهی شد و تمامی بلا را که من بر این مکان می‌رسانم، چشمانت نخواهد دید.» پس ایشان نزد پادشاه جواب آوردند.
\par 20 لهذا اینک من، تو را نزدپدرانت جمع خواهم کرد و در قبر خود به سلامتی گذارده خواهی شد و تمامی بلا را که من بر این مکان می‌رسانم، چشمانت نخواهد دید.» پس ایشان نزد پادشاه جواب آوردند.
 
\chapter{23}

\par 1 و پادشاه فرستاد که تمامی مشایخ یهودا و اورشلیم را نزد وی جمع کردند.
\par 2 و پادشاه و تمامی مردان یهودا و جمیع سکنه اورشلیم با وی و کاهنان و انبیا و تمامی قوم، چه کوچک و چه بزرگ، به خانه خداوندبرآمدند. و او تمامی سخنان کتاب عهدی را که درخانه خداوند یافت شد، در گوش ایشان خواند.
\par 3 و پادشاه نزد ستون ایستاد و به حضور خداوندعهد بست که خداوند را پیروی نموده، اوامر وشهادات و فرایض او را به تمامی دل و تمامی جان نگاه دارند و سخنان این عهد را که در این کتاب مکتوب است، استوار نمایند. پس تمامی قوم این عهد را برپا داشتند.
\par 4 و پادشاه، حلقیا، رئیس کهنه و کاهنان دسته دوم و مستحفظان در را امر فرمود که تمامی ظروف را که برای بعل و اشیره و تمامی لشکرآسمان ساخته شده بود، از هیکل خداوند بیرون آورند. و آنها را در بیرون اورشلیم در مزرعه های قدرون سوزانید و خاکستر آنها را به بیت ئیل برد.
\par 5 و کاهنان بتها را که پادشاهان یهودا تعیین نموده بودند تا در مکان های بلند شهرهای یهودا ونواحی اورشلیم بخور بسوزانند، و آنانی را که برای بعل و آفتاب و ماه و بروج و تمامی لشکرآسمان بخور می‌سوزانیدند، معزول کرد.
\par 6 واشیره را از خانه خداوند، بیرون از اورشلیم به وادی قدرون برد و آن را به کنار نهر قدرون سوزانید، و آن را مثل غبار، نرم ساخت و گرد آن را بر قبرهای عوام الناس پاشید.
\par 7 و خانه های لواط را که نزد خانه خداوند بود که زنان در آنهاخیمه‌ها به جهت اشیره می‌بافتند، خراب کرد. 
\par 8 وتمامی کاهنان را از شهرهای یهودا آورد ومکانهای بلند را که کاهنان در آنها بخورمی سوزانیدند، از جبع تا بئرشبع نجس ساخت، ومکان های بلند دروازه‌ها را که نزد دهنه دروازه یهوشع، رئیس شهر، و به طرف چپ دروازه شهربود، منهدم ساخت.
\par 9 لیکن کاهنان، مکانهای بلند، به مذبح خداوند در اورشلیم برنیامدند اما نان فطیر در میان برادران خود خوردند.
\par 10 و توفت راکه در وادی بنی هنوم بود، نجس ساخت تا کسی پسر یا دختر خود را برای مولک از آتش نگذراند.
\par 11 و اسبهایی را که پادشاهان یهودا به آفتاب داده بودند که نزد حجره نتنملک خواجه‌سرا درپیرامون خانه بودند، از مدخل خانه خداوند دورکرد و ارابه های آفتاب را به آتش سوزانید.
\par 12 ومذبح هایی را که بر پشت بام بالاخانه آحاز بود وپادشاهان یهودا آنها را ساخته بودند، ومذبح هایی را که منسی در دو صحن خانه خداوندساخته بود، پادشاه منهدم ساخت و از آنجاخراب کرده، گرد آنها را در نهر قدرون پاشید.
\par 13 و مکانهای بلند را که مقابل اورشلیم به طرف راست کوه فساد بود و سلیمان، پادشاه اسرائیل، آنها را برای اشتورت، رجاست صیدونیان و برای کموش، رجاست موآبیان، و برای ملکوم، رجاست بنی عمون، ساخته بود، پادشاه، آنها رانجس ساخت.
\par 14 و تماثیل را خرد کرد و اشیریم را قطع نمود و جایهای آنها را از استخوانهای مردم پر ساخت.
\par 15 و نیز مذبحی که در بیت ئیل بود و مکان بلندی که یربعام بن نباط که اسرائیل را مرتکب گناه ساخته، آن را بنا نموده بود، هم مذبح و هم مکان بلند را منهدم ساخت و مکان بلند راسوزانیده، آن را مثل غبار، نرم کرد و اشیره راسوزانید.
\par 16 و یوشیا ملتفت شده، قبرها را که آنجا در کوه بود، دید. پس فرستاده، استخوانها رااز آن قبرها برداشت و آنها را بر آن مذبح سوزانیده، آن را نجس ساخت، به موجب کلام خداوند که آن مرد خدایی که از این امور اخبارنموده بود، به آن ندا درداد.
\par 17 و پرسید: «این مجسمه‌ای که می‌بینم، چیست؟» مردان شهر وی را گفتند: «قبر مرد خدایی است که از یهودا آمده، به این کارهایی که تو بر مذبح بیت ئیل کرده‌ای، نداکرده بود.»
\par 18 او گفت: «آن را واگذارید و کسی استخوانهای او را حرکت ندهد.» پس استخوانهای او را با استخوانهای آن نبی که ازسامره آمده بود، واگذاشتند.
\par 19 و یوشیا تمامی خانه های مکان های بلند را نیز که در شهرهای سامره بود و پادشاهان اسرائیل آنها را ساخته، خشم (خداوند) را به هیجان آورده بودند، برداشت و با آنها موافق تمامی کارهایی که به بیت ئیل کرده بود، عمل نمود.
\par 20 و جمیع کاهنان مکان های بلند را که در آنجا بودند، بر مذبح هاکشت و استخوانهای مردم را بر آنها سوزانیده، به اورشلیم مراجعت کرد.
\par 21 و پادشاه تمامی قوم را امر فرموده، گفت که «عید فصح را به نحوی که در این کتاب عهدمکتوب است، برای خدای خود نگاه دارید.»
\par 22 به تحقیق فصحی مثل این فصح از ایام داورانی که بر اسرائیل داوری نمودند و در تمامی ایام پادشاهان اسرائیل و پادشاهان یهودا نگاه داشته نشد.
\par 23 اما در سال هجدهم، یوشیا پادشاه، این فصح را برای خداوند در اورشلیم نگاه داشتند.
\par 24 و نیز یوشیا اصحاب اجنه و جادوگران وترافیم و بتها و تمام رجاسات را که در زمین یهوداو در اورشلیم پیدا شد، نابود ساخت تا سخنان تورات را که در کتابی که حلقیای کاهن در خانه خداوند یافته بود، به‌جا آورد.
\par 25 و قبل از اوپادشاهی نبود که به تمامی دل و تمامی جان وتمامی قوت خود موافق تمامی تورات موسی به خداوند رجوع نماید، و بعد از او نیز مثل او ظاهر نشد.
\par 26 اما خداوند از حدت خشم عظیم خودبرنگشت زیرا که غضب او به‌سبب همه کارهایی که منسی خشم او را از آنها به هیجان آورده بود، بر یهودا مشتعل شد.
\par 27 و خداوند گفت: «یهودارا نیز از نظر خود دور خواهم کرد چنانکه اسرائیل را دور کردم و این شهر اورشلیم را که برگزیدم و خانه‌ای را که گفتم اسم من در آنجاخواهد بود، ترک خواهم نمود.»
\par 28 و بقیه وقایع یوشیا و هرچه کرد، آیا درکتاب تواریخ ایام پادشاهان یهودا مکتوب نیست؟
\par 29 و در ایام او، فرعون نکوه، پادشاه مصر، بر پادشاه آشور به نهر فرات برآمد و یوشیای پادشاه به مقابل او برآمد و چون (فرعون ) او رادید، وی را در مجدو کشت.
\par 30 و خادمانش او رادر ارابه نهاده، از مجدو به اورشلیم، مرده آوردندو او را در قبرش دفن کردند و اهل زمین، یهوآحازبن یوشیا را گرفتند و او را مسح نموده، به‌جای پدرش به پادشاهی نصب کردند.
\par 31 و یهوآحاز بیست و سه ساله بود که پادشاه شد و سه ماه در اورشلیم سلطنت نمود و اسم مادرش حموطل، دختر ارمیا از لبنه بود.
\par 32 و اوآنچه را که در نظر خداوند ناپسند بود، موافق هرآنچه پدرانش کرده بودند، به عمل آورد.
\par 33 وفرعون نکوه، او را در ربله، در زمین حمات، دربند نهاد تا در اورشلیم سلطنت ننماید و صد وزنه نقره و یک وزنه طلا بر زمین گذارد.
\par 34 و فرعون نکوه، الیاقیم بن یوشیا را به‌جای پدرش، یوشیا، به پادشاهی نصب کرد و اسمش را به یهویاقیم تبدیل نمود و یهوآحاز را گرفته، به مصر آمد. و اودر آنجا مرد.
\par 35 و یهویاقیم، آن نقره و طلا را به فرعون داد اما زمین را تقویم کرد تا آن مبلغ راموافق فرمان فرعون بدهند و آن نقره و طلا را ازاهل زمین، از هرکس موافق تقویم او به زور گرفت تا آن را به فرعون نکوه بدهد.
\par 36 یهویاقیم بیست و پنج ساله بود که پادشاه شد و یازده سال در اورشلیم سلطنت کرد و اسم مادرش زبیده، دختر فدایه، از رومه بود.وآنچه را که در نظر خداوند ناپسند بود موافق هرآنچه پدرانش کرده بودند، به عمل آورد.
\par 37 وآنچه را که در نظر خداوند ناپسند بود موافق هرآنچه پدرانش کرده بودند، به عمل آورد.
 
\chapter{24}

\par 1 و در ایام او، نبوکدنصر، پادشاه بابل آمد، و یهویاقیم سه سال بنده او بود. پس برگشته، از او عاصی شد.
\par 2 و خداوندفوجهای کلدانیان و فوجهای ارامیان و فوجهای موآبیان و فوجهای بنی عمون را بر او فرستاد وایشان را بر یهودا فرستاد تا آن را هلاک سازد، به موجب کلام خداوند که به واسطه بندگان خودانبیا گفته بود.
\par 3 به تحقیق، این از فرمان خداوند بریهودا واقع شد تا ایشان را به‌سبب گناهان منسی وهرچه او کرد، از نظر خود دور اندازد.
\par 4 و نیز به‌سبب خون بی‌گناهانی که او ریخته بود، زیرا که اورشلیم را از خون بی‌گناهان پر کرده بود وخداوند نخواست که او را عفو نماید.
\par 5 و بقیه وقایع یهویاقیم و هرچه کرد، آیا در کتاب تواریخ ایام پادشاهان یهودا مکتوب نیست؟
\par 6 پس یهویاقیم با پدران خود خوابید و پسرش یهویاکین به‌جایش پادشاه شد.
\par 7 و پادشاه مصر، بار دیگر ازولایت خود بیرون نیامد زیرا که پادشاه بابل هرچه را که متعلق به پادشاه مصر بود، از نهر مصر تا نهرفرات، به تصرف آورده بود.
\par 8 و یهویاکین هجده ساله بود که پادشاه شد وسه سال در اورشلیم سلطنت نمود و اسم مادرش نحوشطا دختر الناتان اورشلیمی بود.
\par 9 و آنچه راکه در نظر خداوند ناپسند بود، موافق هرآنچه پدرش کرده بود، به عمل آورد.
\par 10 در آن زمان بندگان نبوکدنصر، پادشاه بابل، بر اورشلیم برآمدند. و شهر محاصره شد.
\par 11 ونبوکدنصر، پادشاه بابل، در حینی که بندگانش آن را محاصره نموده بودند، به شهر برآمد.
\par 12 ویهویاکین، پادشاه یهودا با مادر خود و بندگانش وسردارانش و خواجه‌سرایانش نزد پادشاه بابل بیرون آمد، و پادشاه بابل در سال هشتم سلطنت خود، او را گرفت.
\par 13 و تمامی خزانه های خانه خداوند وخزانه های خانه پادشاه را از آنجا بیرون آورد وتمام ظروف طلایی را که سلیمان، پادشاه اسرائیل برای خانه خداوند ساخته بود، به موجب کلام خداوند، شکست.
\par 14 و جمیع ساکنان اورشلیم وجمیع سرداران و جمیع مردان جنگی را که ده هزار نفر بودند، اسیر ساخته، برد و جمیع صنعت گران و آهنگران را نیز، چنانکه سوای مسکینان، اهل زمین کسی باقی نماند.
\par 15 ویهویاکین را به بابل برد و مادر پادشاه و زنان پادشاه و خواجه‌سرایانش و بزرگان زمین را اسیر ساخت و ایشان را از اورشلیم به بابل برد.
\par 16 و تمامی مردان جنگی، یعنی هفت هزار نفر و یک هزار نفراز صنعت گران و آهنگران را که جمیع ایشان، قوی و جنگ آزموده بودند، پادشاه بابل، ایشان رابه بابل به اسیری برد.
\par 17 و پادشاه بابل، عموی وی، متنیا را در جای او به پادشاهی نصب کرد واسمش را به صدقیا مبدل ساخت.
\par 18 صدقیا بیست و یکساله بود که آغازسلطنت نمود و یازده سال در اورشلیم پادشاهی کرد و اسم مادرش حمیطل، دختر ارمیا از لبنه بود.
\par 19 و آنچه را که در نظر خداوند ناپسند بود، موافق هرآنچه یهویاقیم کرده بود، به عمل آورد.زیرا به‌سبب غضبی که خداوند بر اورشلیم ویهودا داشت، به حدی که آنها را از نظر خودانداخت، واقع شد که صدقیا بر پادشاه بابل عاصی شد.
\par 20 زیرا به‌سبب غضبی که خداوند بر اورشلیم ویهودا داشت، به حدی که آنها را از نظر خودانداخت، واقع شد که صدقیا بر پادشاه بابل عاصی شد.
 
\chapter{25}

\par 1 و واقع شد که نبوکدنصر، پادشاه بابل، باتمامی لشکر خود در روز دهم ماه دهم از سال نهم سلطنت خویش بر اورشلیم برآمد، ودر مقابل آن اردو زده، سنگری گرداگردش بنانمود.
\par 2 و شهر تا سال یازدهم صدقیای پادشاه، محاصره شد.
\par 3 و در روز نهم آن ماه، قحطی درشهر چنان سخت شد که برای اهل زمین نان نبود.
\par 4 پس در شهر رخنه‌ای ساختند و تمامی مردان جنگی در شب از راه دروازه‌ای که در میان دوحصار، نزد باغ پادشاه بود، فرار کردند. و کلدانیان به هر طرف در مقابل شهر بودند (و پادشاه ) به راه عربه رفت.
\par 5 و لشکر کلدانیان، پادشاه را تعاقب نموده، در بیابان اریحا به او رسیدند و تمامی لشکرش از او پراکنده شدند.
\par 6 پس پادشاه راگرفته، او را نزد پادشاه بابل به ربله آوردند و بر اوفتوی دادند.
\par 7 و پسران صدقیا را پیش رویش به قتل رسانیدند و چشمان صدقیا را کندند و او را به دو زنجیر بسته، به بابل آوردند.
\par 8 و در روز هفتم ماه پنجم از سال نوزدهم نبوکدنصر پادشاه، سلطان بابل، نبوزرادان، رئیس جلادان، خادم پادشاه بابل، به اورشلیم آمد.
\par 9 وخانه خداوند و خانه پادشاه را سوزانید و همه خانه های اورشلیم و هر خانه بزرگ را به آتش سوزانید.
\par 10 و تمامی لشکر کلدانیان که همراه رئیس جلادان بودند، حصارهای اورشلیم را به هر طرف منهدم ساختند.
\par 11 و نبوزرادان، رئیس جلادان، بقیه قوم را که در شهر باقی‌مانده بودند وخارجین را که به طرف پادشاه بابل شده بودند وبقیه جمعیت را به اسیری برد.
\par 12 اما رئیس جلادان بعضی از مسکینان زمین را برای باغبانی وفلاحی واگذاشت.
\par 13 و کلدانیان ستونهای برنجینی که در خانه خداوند بود و پایه‌ها و دریاچه برنجینی را که درخانه خداوند بود، شکستند و برنج آنها را به بابل بردند.
\par 14 و دیگها و خاک اندازها و گلگیرها وقاشقها و تمامی اسباب برنجینی را که با آنهاخدمت می‌کردند، بردند.
\par 15 و مجمرها و کاسه هایعنی طلای آنچه را که از طلا بود و نقره آنچه را که از نقره بود، رئیس جلادان برد.
\par 16 اما دو ستون و یک دریاچه و پایه هایی که سلیمان آنها را برای خانه خداوند ساخته بود، وزن برنج همه این اسباب بی‌اندازه بود.
\par 17 بلندی یک ستون هجده ذراع و تاج برنجین بر سرش و بلندی تاج سه ذراع بود و شبکه و انارهای گرداگرد روی تاج، همه ازبرنج بود و مثل اینها برای ستون دوم بر شبکه‌اش بود.
\par 18 و رئیس جلادان، سرایا، رئیس کهنه، وصفنیای کاهن دوم و سه مستحفظ در را گرفت.
\par 19 و سرداری که بر مردان جنگی گماشته شده بودو پنج نفر را از آنانی که روی پادشاه را می‌دیدند ودر شهر یافت شدند و کاتب سردار لشکر را که اهل ولایت را سان می‌دید، و شصت نفر از اهل زمین را که در شهر یافت شدند، از شهر گرفت.
\par 20 و نبوزرادان رئیس جلادان، ایشان را برداشته، به ربله، نزد پادشاه بابل برد. 
\par 21 و پادشاه بابل، ایشان را در ربله در زمین حمات زده، به قتل رسانید. پس یهودا از ولایت خود به اسیری رفتند.
\par 22 و اما قومی که در زمین یهودا باقی ماندند ونبوکدنصر، پادشاه بابل ایشان را رها کرده بود، پس جدلیا ابن اخیقام بن شافان را بر ایشان گماشت.
\par 23 و چون تمامی سرداران لشکر بامردان ایشان شنیدند که پادشاه بابل، جدلیا راحاکم قرار داده است، ایشان نزد جدلیا به مصفه آمدند، یعنی اسماعیل بن نتنیا و یوحنان بن قاری و سرایا ابن تنحومت نطوفاتی و یازنیا ابن معکاتی با کسان ایشان.
\par 24 و جدلیا برای ایشان و برای کسان ایشان قسم خورده، به ایشان گفت: «از بندگان کلدانیان مترسید. در زمین ساکن شوید وپادشاه بابل را بندگی نمایید و برای شما نیکوخواهد بود.»
\par 25 اما در ماه هفتم واقع شد که اسماعیل بن نتنیا ابن الیشمع که از ذریت پادشاه بود، به اتفاق ده نفر آمدند و جدلیا را زدند که بمردو یهودیان و کلدانیان را نیز که با او در مصفه بودند(کشتند).
\par 26 و تمامی قوم، چه خرد و چه بزرگ، و سرداران لشکرها برخاسته، به مصر رفتند زیراکه از کلدانیان ترسیدند.
\par 27 و در روز بیست و هفتم ماه دوازدهم از سال سی و هفتم اسیری یهویاکین، پادشاه یهودا، واقع شد که اویل مرودک، پادشاه بابل، درسالی که پادشاه شد، سر یهویاکین، پادشاه یهودا را از زندان برافراشت.
\par 28 و با اوسخنان دلاویز گفت و کرسی او را بالاتر ازکرسیهای سایر پادشاهانی که با او در بابل بودند، گذاشت.و لباس زندانی او را تبدیل نمود و او در تمامی روزهای عمرش همیشه درحضور وی نان می‌خورد.
\par 29 و لباس زندانی او را تبدیل نمود و او در تمامی روزهای عمرش همیشه درحضور وی نان می‌خورد.


\end{document}