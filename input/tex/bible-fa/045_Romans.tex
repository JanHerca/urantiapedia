\begin{document}

\title{روميان}


\chapter{1}

\par 1 پولس، غلام عیسی مسیح و رسول خوانده شده و جدا نموده شده برای انجیل خدا،
\par 2 که سابق وعده آن را داده بود به وساطت انبیای خود در کتب مقدسه،
\par 3 درباره پسر خود که بحسب جسم از نسل داود متولد شد،
\par 4 و بحسب روح قدوسیت پسر خدا به قوت معروف گردید ازقیامت مردگان یعنی خداوند ما عیسی مسیح،
\par 5 که به او فیض و رسالت را یافتیم برای اطاعت ایمان در جمیع امت‌ها به‌خاطر اسم او،
\par 6 که درمیان ایشان شما نیز خوانده شده عیسی مسیح هستید
\par 7 به همه که در روم محبوب خدا و خوانده شده و مقدسید، فیض و سلامتی از جانب پدر ما خدا و عیسی مسیح خداوند بر شما باد.
\par 8 اول شکر می‌کنم خدای خود را به وساطت عیسی مسیح درباره همگی شما که ایمان شما درتمام عالم شهرت یافته است؛
\par 9 زیرا خدایی که اورا به روح خود در انجیل پسرش خدمت می‌کنم، مرا شاهد است که چگونه پیوسته شما را یادمی کنم،
\par 10 و دائم در دعاهای خود مسالت می‌کنم که شاید الان آخر به اراده خدا سعادت یافته، نزد شما بیایم.
\par 11 زیرا بسیار اشتیاق دارم که شما را ببینم تا نعمتی روحانی به شما برسانم که شما استوار بگردید،
\par 12 یعنی تا در میان شماتسلی یابیم از ایمان یکدیگر، ایمان من و ایمان شما.
\par 13 لکن‌ای برادران، نمی خواهم که شمابی خبر باشید از اینکه مکرر اراده آمدن نزد شماکردم و تا به حال ممنوع شدم تا ثمری حاصل کنم در میان شما نیز چنانکه در سایر امت‌ها.
\par 14 زیرا که یونانیان و بربریان و حکما و جهلا راهم مدیونم.
\par 15 پس همچنین بقدر طاقت خودمستعدم که شما را نیز که در روم هستید بشارت دهم.
\par 16 زیرا که از انجیل مسیح عار ندارم چونکه قوت خداست، برای نجات هر کس که ایمان آورد، اول یهود و پس یونانی،
\par 17 که در آن عدالت خدا مکشوف می‌شود، از ایمان تا ایمان، چنانکه مکتوب است که عادل به ایمان زیست خواهد نمود.
\par 18 زیرا غضب خدا از آسمان مکشوف می‌شود بر هر بی‌دینی و ناراستی مردمانی که راستی را در ناراستی باز می‌دارند.
\par 19 چونکه آنچه از خدا می‌توان شناخت، در ایشان ظاهراست زیرا خدا آن را بر ایشان ظاهر کرده است.
\par 20 زیرا که چیزهای نادیده او یعنی قوت سرمدی و الوهیتش از حین آفرینش عالم بوسیله کارهای او فهمیده و دیده می‌شود تا ایشان را عذری نباشد.
\par 21 زیرا هر‌چند خدا را شناختند، ولی اورا چون خدا تمجید و شکر نکردند بلکه درخیالات خود باطل گردیده، دل بی‌فهم ایشان تاریک گشت.
\par 22 ادعای حکمت می‌کردند واحمق گردیدند.
\par 23 و جلال خدای غیرفانی را به شبیه صورت انسان فانی و طیور و بهایم وحشرات تبدیل نمودند.
\par 24 لهذا خدا نیز ایشان رادر شهوات دل خودشان به ناپاکی تسلیم فرمود تادر میان خود بدنهای خویش را خوار سازند،
\par 25 که ایشان حق خدا را به دروغ مبدل کردند وعبادت و خدمت نمودند مخلوق را به عوض خالقی که تا ابدالاباد متبارک است. آمین.
\par 26 از این سبب خدا ایشان را به هوسهای خباثت تسلیم نمود، به نوعی که زنانشان نیز عمل طبیعی را به آنچه خلاف طبیعت است تبدیل نمودند.
\par 27 و همچنین مردان هم استعمال طبیعی زنان را ترک کرده، از شهوات خود با یکدیگرسوختند. مرد با مرد مرتکب اعمال زشت شده، عقوبت سزاوار تقصیر خود را در خود یافتند.
\par 28 و چون روا نداشتند که خدا را در دانش خودنگاه دارند، خدا ایشان را به ذهن مردود واگذاشت تا کارهای ناشایسته به‌جا آورند.
\par 29 مملو از هرنوع ناراستی و شرارت و طمع و خباثت؛ پر ازحسد و قتل و جدال و مکر و بدخویی؛
\par 30 غمازان و غیبت کنندگان و دشمنان خدا و اهانت کنندگان و متکبران و لافزنان و مبدعان شر و نامطیعان والدین؛
\par 31 بی‌فهم و بی‌وفا و بی‌الفت و بی‌رحم.زیرا هر‌چند انصاف خدا را می‌دانند که کنندگان چنین کارها مستوجب موت هستند، نه فقط آنها را می‌کنند بلکه کنندگان را نیز خوش می‌دارند.
\par 32 زیرا هر‌چند انصاف خدا را می‌دانند که کنندگان چنین کارها مستوجب موت هستند، نه فقط آنها را می‌کنند بلکه کنندگان را نیز خوش می‌دارند.

\chapter{2}

\par 1 لهذا‌ای آدمی که حکم می‌کنی، هر‌که باشی عذری نداری زیرا که به آنچه بردیگری حکم می‌کنی، فتوا بر خود می‌دهی، زیراتو که حکم می‌کنی، همان کارها را به عمل می‌آوری.
\par 2 و می‌دانیم که حکم خدا بر کنندگان چنین اعمال بر‌حق است.
\par 3 پس‌ای آدمی که برکنندگان چنین اعمال حکم می‌کنی و خود همان را می‌کنی، آیا گمان می‌بری که تو از حکم خداخواهی رست؟
\par 4 یا آنکه دولت مهربانی و صبر وحلم او را ناچیز می‌شماری و نمی دانی که مهربانی خدا تو را به توبه می‌کشد؟
\par 5 و به‌سبب قساوت و دل ناتوبه‌کار خود، غضب را ذخیره می‌کنی برای خود در روز غضب و ظهور داوری عادله خدا
\par 6 که به هر کس برحسب اعمالش جزاخواهد داد:
\par 7 اما به آنانی که با صبر در اعمال نیکوطالب جلال و اکرام و بقایند، حیات جاودانی را؛
\par 8 و اما به اهل تعصب که اطاعت راستی نمی کنندبلکه مطیع ناراستی می‌باشند، خشم و غضب
\par 9 وعذاب و ضیق بر هر نفس بشری که مرتکب بدی می‌شود، اول بر یهود و پس بر یونانی؛
\par 10 لکن جلال و اکرام و سلامتی بر هر نیکوکار، نخست بریهود و بر یونانی نیز.
\par 11 زیرا نزد خدا طرفداری نیست،
\par 12 زیراآنانی که بدون شریعت گناه کنند، بی‌شریعت نیزهلاک شوند و آنانی که با شریعت گناه کنند، ازشریعت بر ایشان حکم خواهد شد.
\par 13 از آن جهت که شنوندگان شریعت در حضور خدا عادل نیستند بلکه کنندگان شریعت عادل شمرده خواهند شد.
\par 14 زیرا هرگاه امت هایی که شریعت ندارند کارهای شریعت را به طبیعت به‌جا آرند، اینان هرچند شریعت ندارند، برای خود شریعت هستند،
\par 15 چونکه از ایشان ظاهر می‌شود که عمل شریعت بر دل ایشان مکتوب است و ضمیرایشان نیز گواهی می‌دهد و افکار ایشان با یکدیگریا مذمت می‌کنند یا عذر می‌آورند،
\par 16 در روزی که خدا رازهای مردم را داوری خواهد نمود به وساطت عیسی مسیح برحسب بشارت من.
\par 17 پس اگر تو مسمی به یهود هستی و برشریعت تکیه می‌کنی و به خدا فخر می‌نمایی،
\par 18 و اراده او را می‌دانی و از شریعت تربیت یافته، چیزهای افضل را می‌گزینی،
\par 19 و یقین داری که خود هادی کوران و نور ظلمتیان
\par 20 و مودب جاهلان و معلم اطفال هستی و در شریعت صورت معرفت و راستی را داری،
\par 21 پس‌ای کسی‌که دیگران را تعلیم می‌دهی، چرا خود رانمی آموزی؟ و وعظ می‌کنی که دزدی نباید کرد، آیا خود دزدی می‌کنی؟
\par 22 و از زنا کردن نهی می‌کنی، آیا خود زانی نیستی؟ و از بتها نفرت داری، آیا خود معبدها را غارت نمی کنی؟
\par 23 و به شریعت فخر می‌کنی، آیا به تجاوز از شریعت خدا را اهانت نمی کنی؟
\par 24 زیرا که به‌سبب شمادر میان امت‌ها اسم خدا را کفر می‌گویند، چنانکه مکتوب است.
\par 25 زیرا ختنه سودمند است هرگاه به شریعت عمل نمایی. اما اگر از شریعت تجاوزنمایی، ختنه تو نامختونی گشته است.
\par 26 پس اگرنامختونی، احکام شریعت را نگاه دارد، آیانامختونی او ختنه شمرده نمی شود؟
\par 27 ونامختونی طبیعی هرگاه شریعت را به‌جا آرد، حکم خواهد کرد بر تو که با وجود کتب و ختنه ازشریعت تجاوز می‌کنی.
\par 28 زیرا آنکه در ظاهراست، یهودی نیست و آنچه در ظاهر در جسم است، ختنه نی.بلکه یهود آن است که در باطن باشد و ختنه آنکه قلبی باشد، در روح نه در حرف که مدح آن نه از انسان بلکه از خداست.
\par 29 بلکه یهود آن است که در باطن باشد و ختنه آنکه قلبی باشد، در روح نه در حرف که مدح آن نه از انسان بلکه از خداست.

\chapter{3}

\par 1 پس برتری یهود چیست؟ و یا از ختنه چه فایده؟
\par 2 بسیار از هر جهت؛ اول آنکه بدیشان کلام خدا امانت داده شده است.
\par 3 زیرا که چه بگوییم اگر بعضی ایمان نداشتند؟ آیابی ایمانی‌ایشان امانت خدا را باطل می‌سازد؟
\par 4 حاشا! بلکه خدا راستگو باشد و هر انسان دروغگو، چنانکه مکتوب است: «تا اینکه درسخنان خود مصدق شوی و در داوری خودغالب آیی.»
\par 5 لکن اگر ناراستی ما عدالت خدا را ثابت می‌کند، چه گوییم؟ آیا خدا ظالم است وقتی که غضب می‌نماید؟ بطور انسان سخن می‌گویم.
\par 6 حاشا! در این صورت خدا چگونه عالم راداوری خواهد کرد؟
\par 7 زیرا اگر به دروغ من، راستی خدا برای جلال او افزون شود، پس چرا برمن نیز چون گناهکار حکم شود؟
\par 8 و چرا نگوییم، چنانکه بعضی بر ما افترا می‌زنند و گمان می‌برندکه ما چنین می‌گوییم، بدی بکنیم تا نیکویی حاصل شود؟ که قصاص ایشان به انصاف است.
\par 9 پس چه گوییم؟ آیا برتری داریم؟ نه به هیچ وجه! زیرا پیش ادعا وارد آوردیم که یهود ویونانیان هر دو به گناه گرفتارند.
\par 10 چنانکه مکتوب است که «کسی عادل نیست، یکی هم نی.
\par 11 کسی فهیم نیست، کسی طالب خدا نیست.
\par 12 همه گمراه و جمیع باطل گردیده‌اند. نیکوکاری نیست یکی هم نی.
\par 13 گلوی ایشان گور گشاده است و به زبانهای خود فریب می‌دهند. زهر مار در زیر لب ایشان است،
\par 14 ودهان ایشان پر از لعنت و تلخی است.
\par 15 پایهای ایشان برای خون ریختن شتابان است.
\par 16 هلاکت و پریشانی در طریقهای ایشان است،
\par 17 و طریق سلامتی را ندانسته‌اند.
\par 18 خدا ترسی درچشمانشان نیست.»
\par 19 الان آگاه هستیم که آنچه شریعت می‌گوید، به اهل شریعت خطاب می‌کند تا هر دهانی بسته شود و تمام عالم زیر قصاص خدا آیند.
\par 20 ازآنجا که به اعمال شریعت هیچ بشری در حضوراو عادل شمرده نخواهد شد، چونکه از شریعت دانستن گناه است.
\par 21 لکن الحال بدون شریعت، عدالت خداظاهر شده است، چنانکه تورات و انبیا بر آن شهادت می‌دهند؛
\par 22 یعنی عدالت خدا که بوسیله ایمان به عیسی مسیح است، به همه و کل آنانی که ایمان آورند. زیرا که هیچ تفاوتی نیست،
\par 23 زیراهمه گناه کرده‌اند واز جلال خدا قاصر می‌باشند،
\par 24 و به فیض او مجان عادل شمرده می‌شوند به وساطت آن فدیه‌ای که در عیسی مسیح است.
\par 25 که خدا او را از قبل معین کرد تا کفاره باشد به واسطه ایمان به وسیله خون او تا آنکه عدالت خود را ظاهر سازد، به‌سبب فرو گذاشتن خطایای سابق در حین تحمل خدا،
\par 26 برای اظهار عدالت خود در زمان حاضر، تا او عادل شود و عادل شمارد هرکسی را که به عیسی ایمان آورد.
\par 27 پس جای فخر کجا است؟ برداشته شده است! به کدام شریعت؟ آیا به شریعت اعمال؟ نی بلکه به شریعت ایمان.
\par 28 زیرا یقین می‌دانیم که انسان بدون اعمال شریعت، محض ایمان عادل شمرده می‌شود.
\par 29 آیا او خدای یهود است فقط؟ مگر خدای امت‌ها هم نیست؟ البته خدای امت‌ها نیز است.
\par 30 زیرا واحد است خدایی که اهل ختنه را از ایمان، و نامختونان را به ایمان عادل خواهد شمرد.پس آیا شریعت را به ایمان باطل می‌سازیم؟ حاشا! بلکه شریعت رااستوار می‌داریم.
\par 31 پس آیا شریعت را به ایمان باطل می‌سازیم؟ حاشا! بلکه شریعت رااستوار می‌داریم.

\chapter{4}

\par 1 پس چه چیز را بگوییم که پدر ما ابراهیم بحسب جسم یافت؟
\par 2 زیرا اگر ابراهیم به اعمال عادل شمرده شد، جای فخر دارد اما نه درنزد خدا.
\par 3 زیرا کتاب چه می‌گوید؟ «ابراهیم به خدا ایمان آورد و آن برای او عدالت محسوب شد.»
\par 4 لکن برای کسی‌که عمل می‌کند، مزدش نه ازراه فیض بلکه از راه طلب محسوب می‌شود.
\par 5 واما کسی‌که عمل نکند، بلکه ایمان آورد به او که بی‌دینان را عادل می‌شمارد، ایمان او عدالت محسوب می‌شود.
\par 6 چنانکه داود نیز خوش حالی آن کس را ذکر می‌کند که خدا برای او عدالت محسوب می‌دارد، بدون اعمال:
\par 7 «خوشابحال کسانی که خطایای ایشان آمرزیده شد وگناهانشان مستور گردید؛
\par 8 خوشابحال کسی‌که خداوند گناه را به وی محسوب نفرماید.»
\par 9 پس آیا این خوشحالی بر اهل ختنه گفته شدیا برای نامختونان نیز؟ زیرا می‌گوییم ایمان ابراهیم به عدالت محسوب گشت.
\par 10 پس در چه حالت محسوب شد، وقتی که او در ختنه بود یادر نامختونی؟ در ختنه نی، بلکه در نامختونی؛
\par 11 و علامت ختنه را یافت تا مهر باشد بر آن عدالت ایمانی که در نامختونی داشت، تا او همه نامختونان را که ایمان آورند پدر باشد تا عدالت برای ایشان هم محسوب شود؛
\par 12 و پدر اهل ختنه نیز یعنی آنانی را که نه فقط مختونند بلکه سالک هم می‌باشند بر آثار ایمانی که پدر ماابراهیم در نامختونی داشت.
\par 13 زیرا به ابراهیم و ذریت او، وعده‌ای که اووارث جهان خواهد بود، از جهت شریعت داده نشد بلکه از عدالت ایمان.
\par 14 زیرا اگر اهل شریعت وارث باشند، ایمان عاطل شد و وعده باطل.
\par 15 زیرا که شریعت باعث غضب است، زیرا جایی که شریعت نیست تجاوز هم نیست.
\par 16 و ازاین جهت از ایمان شد تا محض فیض باشد تاوعده برای همگی ذریت استوار شود نه مختص به ذریت شرعی بلکه به ذریت ایمانی ابراهیم نیزکه پدر جمیع ما است،
\par 17 (چنانکه مکتوب است که تو را پدر امت های بسیار ساخته‌ام )، در حضورآن خدایی که به او ایمان آورد که مردگان را زنده می‌کند و ناموجودات را به وجود می‌خواند؛
\par 18 که او در ناامیدی به امید ایمان آورد تا پدرامت های بسیار شود، برحسب آنچه گفته شد که «ذریت تو چنین خواهند بود.»
\par 19 و در ایمان کم قوت نشده، نظر کرد به بدن خود که در آن وقت مرده بود، چونکه قریب به صد ساله بود و به رحم مرده ساره.
\par 20 و در وعده خدا از بی‌ایمانی شک ننمود، بلکه قوی الایمان گشته، خدا را تمجیدنمود،
\par 21 و یقین دانست که به وفای وعده خود نیزقادر است.
\par 22 و از این جهت برای او عدالت محسوب شد.
\par 23 ولکن اینکه برای وی محسوب شد، نه برای او فقط نوشته شد،
\par 24 بلکه برای ما نیزکه به ما محسوب خواهد شد، چون ایمان آوریم به او که خداوند ما عیسی را از مردگان برخیزانید،که به‌سبب گناهان ما تسلیم گردید و به‌سبب عادل شدن ما برخیزانیده شد.
\par 25 که به‌سبب گناهان ما تسلیم گردید و به‌سبب عادل شدن ما برخیزانیده شد.

\chapter{5}

\par 1 پس چونکه به ایمان عادل شمرده شدیم، نزد خدا سلامتی داریم بوساطت خداوندما عیسی مسیح،
\par 2 که به وساطت او دخول نیزیافته‌ایم بوسیله ایمان در آن فیضی که در آن پایداریم و به امید جلال خدا فخر می‌نماییم.
\par 3 ونه این تنها بلکه در مصیبتها هم فخر می‌کنیم، چونکه می‌دانیم که مصیبت صبر را پیدا می‌کند،
\par 4 و صبر امتحان را و امتحان امید را.
\par 5 و امید باعث شرمساری نمی شود زیرا که محبت خدا دردلهای ما به روح‌القدس که به ما عطا شد ریخته شده است.
\par 6 زیرا هنگامی که ما هنوز ضعیف بودیم، در زمان معین، مسیح برای بیدینان وفات یافت.
\par 7 زیرا بعید است که برای شخص عادل کسی بمیرد، هرچند در راه مرد نیکو ممکن است کسی نیز جرات کند که بمیرد.
\par 8 لکن خدا محبت خود را در ما ثابت می‌کند از اینکه هنگامی که ماهنوز گناهکار بودیم، مسیح در راه ما مرد.
\par 9 پس چقدر بیشتر الان که به خون او عادل شمرده شدیم، بوسیله او از غضب نجات خواهیم یافت.
\par 10 زیرا اگر در حالتی که دشمن بودیم، بوساطت مرگ پسرش با خدا صلح داده شدیم، پس چقدربیشتر بعد از صلح یافتن بوساطت حیات او نجات خواهیم یافت.
\par 11 و نه همین فقط بلکه در خدا هم فخر می‌کنیم بوسیله خداوند ما عیسی مسیح که بوساطت او الان صلح یافته‌ایم.
\par 12 لهذا همچنان‌که بوساطت یک آدم گناه داخل جهان گردید و به گناه موت؛ و به اینگونه موت بر همه مردم طاری گشت، از آنجا که همه گناه کردند.
\par 13 زیرا قبل از شریعت، گناه در جهان می‌بود، لکن گناه محسوب نمی شود در جایی که شریعت نیست.
\par 14 بلکه از آدم تا موسی موت تسلط می‌داشت بر آنانی نیز که بر مثال تجاوز آدم که نمونه آن آینده است، گناه نکرده بودند.
\par 15 و نه‌چنانکه خطا بود، همچنان نعمت نیز باشد. زیرااگر به خطای یک شخص بسیاری مردند، چقدرزیاده فیض خدا و آن بخششی که به فیض یک انسان، یعنی عیسی مسیح است، برای بسیاری افزون گردید.
\par 16 و نه اینکه مثل آنچه از یک گناهکار سر زد، همچنان بخشش باشد؛ زیراحکم شد از یک برای قصاص لکن نعمت ازخطایای بسیار برای عدالت رسید.
\par 17 زیرا اگر به‌سبب خطای یک نفر و بواسطه آن یک موت سلطنت کرد، چقدر بیشتر آنانی که افزونی فیض و بخشش عدالت را می‌پذیرند، در حیات سلطنت خواهند کرد بوسیله یک یعنی عیسی مسیح.
\par 18 پس همچنان‌که به یک خطا حکم شد برجمیع مردمان برای قصاص، همچنین به یک عمل صالح بخشش شد بر جمیع مردمان برای عدالت حیات.
\par 19 زیرا به همین قسمی که ازنافرمانی یک شخص بسیاری گناهکار شدند، همچنین نیز به اطاعت یک شخص بسیاری عادل خواهند گردید.
\par 20 اما شریعت در میان آمد تاخطا زیاده شود. لکن جایی که گناه زیاده گشت، فیض بینهایت افزون گردید.تا آنکه چنانکه گناه در موت سلطنت کرد، همچنین فیض نیزسلطنت نماید به عدالت برای حیات جاودانی بوساطت خداوند ما عیسی مسیح.
\par 21 تا آنکه چنانکه گناه در موت سلطنت کرد، همچنین فیض نیزسلطنت نماید به عدالت برای حیات جاودانی بوساطت خداوند ما عیسی مسیح.

\chapter{6}

\par 1 پس چه گوییم؟ آیا در گناه بمانیم تا فیض افزون گردد؟
\par 2 حاشا! مایانی که از گناه مردیم، چگونه دیگر در آن زیست کنیم؟
\par 3 یانمی دانید که جمیع ما که در مسیح عیسی تعمیدیافتیم، در موت او تعمید یافتیم؟
\par 4 پس چونکه در موت او تعمید یافتیم، با او دفن شدیم تا آنکه به همین قسمی که مسیح به جلال پدر از مردگان برخاست، ما نیز در تازگی حیات رفتار نماییم.
\par 5 زیرا اگر بر مثال موت او متحد گشتیم، هرآینه در قیامت وی نیز چنین خواهیم شد.
\par 6 زیرا این رامی دانیم که انسانیت کهنه ما با او مصلوب شد تاجسد گناه معدوم گشته، دیگر گناه را بندگی نکنیم.
\par 7 زیرا هر‌که مرد، از گناه مبرا شده است.
\par 8 پس هرگاه با مسیح مردیم، یقین می‌دانیم که با اوزیست هم خواهیم کرد.
\par 9 زیرا می‌دانیم که چون مسیح از مردگان برخاست، دیگر نمی میرد و بعداز این موت بر او تسلطی ندارد.
\par 10 زیرا به آنچه مرد یک مرتبه برای گناه مرد و به آنچه زندگی می‌کند، برای خدا زیست می‌کند.
\par 11 همچنین شما نیز خود را برای گناه مرده انگارید، اما برای خدا در مسیح عیسی زنده.
\par 12 پس گناه در جسم فانی شما حکمرانی نکند تا هوسهای آن را اطاعت نمایید،
\par 13 واعضای خود را به گناه مسپارید تا آلات ناراستی شوند، بلکه خود را از مردگان زنده شده به خداتسلیم کنید و اعضای خود را تا آلات عدالت برای خدا باشند.
\par 14 زیرا گناه بر شما سلطنت نخواهد کرد، چونکه زیر شریعت نیستید بلکه زیرفیض.
\par 15 پس چه گوییم؟ آیا گناه بکنیم از آنرو که زیر شریعت نیستیم بلکه زیر فیض؟ حاشا!
\par 16 آیانمی دانید که اگر خویشتن را به بندگی کسی تسلیم کرده، او را اطاعت نمایید، شما آنکس را که او رااطاعت می‌کنید بنده هستید، خواه گناه را برای مرگ، خواه اطاعت را برای عدالت.
\par 17 اما شکرخدا را که هرچند غلامان گناه می‌بودید، لیکن الان از دل، مطیع آن صورت تعلیم گردیده‌اید که به آن سپرده شده‌اید.
\par 18 و از گناه آزاد شده، غلامان عدالت گشته‌اید.
\par 19 بطور انسان، به‌سبب ضعف جسم شما سخن می‌گویم، زیرا همچنان‌که اعضای خود را بندگی نجاست و گناه برای گناه سپردید، همچنین الان نیز اعضای خود را به بندگی عدالت برای قدوسیت بسپارید.
\par 20 زیراهنگامی که غلامان گناه می‌بودید از عدالت آزادمی بودید.
\par 21 پس آن وقت چه ثمر داشتید از آن کارهایی که الان از آنها شرمنده‌اید که انجام آنهاموت است؟
\par 22 اما الحال چونکه از گناه آزاد شده و غلامان خدا گشته‌اید، ثمر خود را برای قدوسیت می‌آورید که عاقبت آن، حیات جاودانی است.زیرا که مزد گناه موت است، اما نعمت خدا حیات جاودانی در خداوند ماعیسی مسیح.
\par 23 زیرا که مزد گناه موت است، اما نعمت خدا حیات جاودانی در خداوند ماعیسی مسیح.

\chapter{7}

\par 1 ای برادران آیا نمی دانید (زیرا که با عارفین شریعت سخن می‌گویم ) که مادامی که انسان زنده است، شریعت بر وی حکمرانی دارد؟
\par 2 زیرا زن منکوحه برحسب شریعت به شوهرزنده بسته است، اما هرگاه شوهرش بمیرد، ازشریعت شوهرش آزاد شود.
\par 3 پس مادامی که شوهرش حیات دارد، اگر به مرد دیگر پیوندد، زانیه خوانده می‌شود. لکن هرگاه شوهرش بمیرد، از آن شریعت آزاد است که اگر به شوهری دیگرداده شود، زانیه نباشد.
\par 4 بنابراین، ای برادران من، شما نیز بوساطت جسد مسیح برای شریعت مرده شدید تا خود رابه دیگری پیوندید، یعنی با او که از مردگان برخاست، تا بجهت خدا ثمر آوریم.
\par 5 زیرا وقتی که در جسم بودیم، هوسهای گناهانی که ازشریعت بود، در اعضای ما عمل می‌کرد تا بجهت موت ثمر آوریم.
\par 6 اما الحال چون برای آن چیزی که در آن بسته بودیم مردیم، از شریعت آزاد شدیم، بحدی که در تازگی روح بندگی می‌کنیم نه در کهنگی حرف.
\par 7 پس چه گوییم؟ آیا شریعت گناه است؟ حاشا! بلکه گناه را جز به شریعت ندانستیم. زیراکه شهوت را نمی دانستم، اگر شریعت نمی گفت که طمع مورز.
\par 8 لکن گناه از حکم فرصت جسته، هر قسم طمع را در من پدید آورد، زیرا بدون شریعت گناه مرده است.
\par 9 و من از قبل بدون شریعت زنده می‌بودم؛ لکن چون حکم آمد، گناه زنده گشت و من مردم.
\par 10 و آن حکمی که برای حیات بود، همان مرا باعث موت گردید.
\par 11 زیراگناه از حکم فرصت یافته، مرا فریب داد و به آن مرا کشت.
\par 12 خلاصه شریعت مقدس است و حکم مقدس و عادل و نیکو.
\par 13 پس آیا نیکویی برای من موت گردید؟ حاشا! بلکه گناه، تا گناه بودنش ظاهر شود. بوسیله نیکویی برای من باعث مرگ شد تا آنکه گناه به‌سبب حکم بغایت خبیث شود.
\par 14 زیرا می‌دانیم که شریعت روحانی است، لکن من جسمانی و زیر گناه فروخته شده هستم،
\par 15 که آنچه می‌کنم نمی دانم زیرا آنچه می‌خواهم نمی کنم بلکه کاری را که از آن نفرت دارم بجامی آورم.
\par 16 پس هرگاه کاری را که نمی خواهم به‌جا می‌آورم، شریعت را تصدیق می‌کنم که نیکوست.
\par 17 و الحال من دیگر فاعل آن نیستم بلکه آن گناهی که در من ساکن است.
\par 18 زیرامی دانم که در من یعنی در جسدم هیچ نیکویی ساکن نیست، زیرا که اراده در من حاضر است اماصورت نیکو کردن نی.
\par 19 زیرا آن نیکویی را که می‌خواهم نمی کنم، بلکه بدی را که نمی خواهم می‌کنم.
\par 20 پس چون آنچه را نمی خواهم می‌کنم، من دیگر فاعل آن نیستم بلکه گناه که در من ساکن است.
\par 21 لهذا این شریعت را می‌یابم که وقتی که می‌خواهم نیکویی کنم بدی نزد من حاضر است.
\par 22 زیرا برحسب انسانیت باطنی به شریعت خداخشنودم.
\par 23 لکن شریعتی دیگر در اعضای خودمی بینم که با شریعت ذهن من منازعه می‌کند و مرااسیر می‌سازد به آن شریعت گناه که در اعضای من است.
\par 24 وای بر من که مرد شقی‌ای هستم! کیست که مرا از جسم این موت رهایی بخشد؟خدا راشکر می‌کنم بوساطت خداوند ما عیسی مسیح. خلاصه اینکه من به ذهن خود شریعت خدا رابندگی می‌کنم و اما به جسم خود شریعت گناه را.
\par 25 خدا راشکر می‌کنم بوساطت خداوند ما عیسی مسیح. خلاصه اینکه من به ذهن خود شریعت خدا رابندگی می‌کنم و اما به جسم خود شریعت گناه را.

\chapter{8}

\par 1 پس هیچ قصاص نیست بر آنانی که درمسیح عیسی هستند.
\par 2 زیرا که شریعت روح حیات در مسیح عیسی مرا از شریعت گناه وموت آزاد گردانید.
\par 3 زیرا آنچه از شریعت محال بود، چونکه به‌سبب جسم ضعیف بود، خدا پسرخود را در شبیه جسم گناه و برای گناه فرستاده، برگناه در جسم فتوا داد،
\par 4 تا عدالت شریعت کامل گردد در مایانی که نه بحسب جسم بلکه برحسب روح رفتار می‌کنیم.
\par 5 زیرا آنانی که برحسب جسم هستند، درچیزهای جسم تفکر می‌کنند و اما آنانی که برحسب روح هستند در چیزهای روح.
\par 6 از آن جهت که تفکر جسم موت است، لکن تفکر روح حیات و سلامتی است.
\par 7 زانرو که تفکر جسم دشمنی خدا است، چونکه شریعت خدا رااطاعت نمی کند، زیرا نمی تواند هم بکند.
\par 8 وکسانی که جسمانی هستند، نمی توانند خدا راخشنود سازند.
\par 9 لکن شما در جسم نیستید بلکه در روح، هرگاه روح خدا در شما ساکن باشد؛ وهرگاه کسی روح مسیح را ندارد وی از آن اونیست.
\par 10 و اگر مسیح در شما است، جسم به‌سبب گناه مرده است و اما روح، به‌سبب عدالت، حیات‌است.
\par 11 و اگر روح او که عیسی را ازمردگان برخیزانید در شما ساکن باشد، او که مسیح را از مردگان برخیزانید، بدنهای فانی شمارا نیز زنده خواهد ساخت به روح خود که در شماساکن است.
\par 12 بنابراین‌ای برادران، مدیون جسم نیستیم تابرحسب جسم زیست نماییم.
\par 13 زیرا اگربرحسب جسم زیست کنید، هرآینه خواهید مرد. لکن اگر افعال بدن را بوسیله روح بکشید، همانا خواهید زیست.
\par 14 زیرا همه کسانی که از روح خدا هدایت می‌شوند، ایشان پسران خدایند.
\par 15 از آنرو که روح بندگی را نیافته‌اید تا باز ترسان شوید بلکه روح پسر‌خواندگی را یافته‌اید که به آن ابا یعنی‌ای پدر ندا می‌کنیم.
\par 16 همان روح برروحهای ما شهادت می‌دهد که فرزندان خداهستیم.
\par 17 و هرگاه فرزندانیم، وارثان هم هستیم یعنی ورثه خدا و هم‌ارث با مسیح، اگر شریک مصیبتهای او هستیم تا در جلال وی نیز شریک باشیم.
\par 18 زیرا یقین می‌دانم که دردهای زمان حاضرنسبت به آن جلالی که در ما ظاهر خواهد شد هیچ است.
\par 19 زیرا که انتظار خلقت، منتظر ظهورپسران خدا می‌باشد،
\par 20 زیرا خلقت، مطیع بطالت شد، نه به اراده خود، بلکه بخاطر او که آن را مطیع گردانید،
\par 21 در امید که خود خلقت نیز از قیدفساد خلاصی خواهد یافت تا در آزادی جلال فرزندان خدا شریک شود.
\par 22 زیرا می‌دانیم که تمام خلقت تا الان با هم در آه کشیدن و درد زه می‌باشند.
\par 23 و نه این فقط، بلکه ما نیز که نوبر روح را یافته‌ایم، در خود آه می‌کشیم در انتظارپسرخواندگی یعنی خلاصی جسم خود.
\par 24 زیراکه به امید نجات یافتیم، لکن چون امید دیده شد، دیگر امید نیست، زیرا آنچه کسی بیند چرا دیگردر امید آن باشد؟
\par 25 اما اگر امید چیزی را داریم که نمی بینیم، با صبر انتظار آن می‌کشیم.
\par 26 و همچنین روح نیز ضعف ما را مددمی کند، زیرا که آنچه دعا کنیم بطوری که می‌بایدنمی دانیم، لکن خود روح برای ما شفاعت می‌کند به ناله هایی که نمی شود بیان کرد.
\par 27 و او که تفحص کننده دلهاست، فکر روح را می‌داند زیراکه او برای مقدسین برحسب اراده خدا شفاعت می‌کند.
\par 28 و می‌دانیم که بجهت آنانی که خدا رادوست می‌دارند و بحسب اراده او خوانده شده‌اند، همه‌چیزها برای خیریت (ایشان ) با هم در کار می‌باشند.
\par 29 زیرا آنانی را که از قبل شناخت، ایشان را نیز پیش معین فرمود تا به صورت پسرش متشکل شوند تا او نخست زاده ازبرادران بسیار باشد.
\par 30 و آنانی را که از قبل معین فرمود، ایشان را هم خواند و آنانی را که خواندایشان را نیز عادل گردانید و آنانی را که عادل گردانید، ایشان را نیز جلال داد.
\par 31 پس به این چیزها چه گوییم؟ هرگاه خدا باما است کیست به ضد ما؟
\par 32 او که پسر خود رادریغ نداشت، بلکه او را در راه جمیع ما تسلیم نمود، چگونه با وی همه‌چیز را به ما نخواهدبخشید؟
\par 33 کیست که بر برگزیدگان خدا مدعی شود؟ آیا خدا که عادل کننده است؟
\par 34 کیست که بر ایشان فتوا دهد؟ آیا مسیح که مرد بلکه نیزبرخاست، آنکه به‌دست راست خدا هم هست وما را نیز شفاعت می‌کند؟
\par 35 کیست که ما را ازمحبت مسیح جدا سازد؟ آیا مصیبت یا دلتنگی یاجفا یا قحط یا عریانی یا خطر یا شمشیر؟
\par 36 چنانکه مکتوب است که «بخاطر تو تمام روزکشته و مثل گوسفندان ذبحی شمرده می‌شویم.»
\par 37 بلکه در همه این امور از حد زیاده نصرت یافتیم، بوسیله او که ما را محبت نمود.
\par 38 زیرایقین می‌دانم که نه موت و نه حیات و نه فرشتگان و نه روسا و نه قدرتها و نه چیزهای حال و نه چیزهای آیندهو نه بلندی و نه پستی و نه هیچ مخلوق دیگر قدرت خواهد داشت که ما را ازمحبت خدا که در خداوند ما مسیح عیسی است جدا سازد.
\par 39 و نه بلندی و نه پستی و نه هیچ مخلوق دیگر قدرت خواهد داشت که ما را ازمحبت خدا که در خداوند ما مسیح عیسی است جدا سازد.

\chapter{9}

\par 1 در مسیح راست می‌گویم و دروغ نی و ضمیر من در روح‌القدس مرا شاهد است،
\par 2 که مرا غمی عظیم و در دلم وجع دائمی است.
\par 3 زیرا راضی هم می‌بودم که خود از مسیح محروم شوم در راه برادرانم که بحسب جسم خویشان منند،
\par 4 که ایشان اسرائیلی‌اند و پسرخواندگی وجلال و عهدها و امانت شریعت و عبادت ووعده‌ها از آن ایشان است؛
\par 5 که پدران از آن ایشانند و از ایشان مسیح بحسب جسم شد که فوق از همه است، خدای متبارک تا ابدالاباد، آمین.
\par 6 ولکن چنین نیست که کلام خدا ساقط شده باشد؛ زیرا همه که از اسرائیل‌اند، اسرائیلی نیستند،
\par 7 و نه نسل ابراهیم تمام فرزند هستند؛ بلکه نسل تو در اسحاق خوانده خواهند شد.
\par 8 یعنی فرزندان جسم، فرزندان خدا نیستند، بلکه فرزندان وعده از نسل محسوب می‌شوند.
\par 9 زیراکلام وعده این است که موافق چنین وقت خواهم آمد و ساره را پسری خواهد بود.
\par 10 و نه این فقط، بلکه رفقه نیز چون از یک شخص یعنی از پدر مااسحاق حامله شد،
\par 11 زیرا هنگامی که هنوز تولدنیافته بودند و عملی نیک یا بد نکرده، تا اراده خدابرحسب اختیار ثابت شود نه از اعمال بلکه ازدعوت کننده
\par 12 بدو گفته شد که «بزرگتر کوچکتر را بندگی خواهد نمود.»
\par 13 چنانکه مکتوب است: «یعقوب را دوست داشتم اما عیسو را دشمن.»
\par 14 پس چه گوییم؟ آیا نزد خدا بی‌انصافی است؟ حاشا!
\par 15 زیرا به موسی می‌گوید: «رحم خواهم فرمود بر هر‌که رحم کنم و رافت خواهم نمود بر هر‌که رافت نمایم.»
\par 16 لاجرم نه ازخواهش کننده و نه از شتابنده است، بلکه ازخدای رحم کننده.
\par 17 زیرا کتاب به فرعون می‌گوید: «برای همین تو را برانگیختم تا قوت خود را در تو ظاهر سازم و تا نام من در تمام جهان ندا شود.»
\par 18 بنابراین هر‌که را می‌خواهد رحم می‌کند و هر‌که را می‌خواهد سنگدل می‌سازد.
\par 19 پس مرا می‌گویی: «دیگر چرا ملامت می‌کند؟ زیرا کیست که با اراده او مقاومت نموده باشد؟»
\par 20 نی بلکه تو کیستی‌ای انسان که با خدا معارضه می‌کنی؟ آیا مصنوع به صانع می‌گوید که چرا مراچنین ساختی؟
\par 21 یا کوزه‌گر اختیار بر گل نداردکه از یک خمیره ظرفی عزیز و ظرفی ذلیل بسازد؟
\par 22 و اگر خدا چون اراده نمود که غضب خود را ظاهر سازد و قدرت خویش را بشناساند، ظروف غضب را که برای هلاکت آماده‌شده بود، به حلم بسیار متحمل گردید،
\par 23 و تا دولت جلال خود را بشناساند بر ظروف رحمتی که آنها را ازقبل برای جلال مستعد نمود،
\par 24 و آنها را نیزدعوت فرمود یعنی ما نه از یهود فقط بلکه ازامت‌ها نیز.
\par 25 چنانکه در هوشع هم می‌گوید: «آنانی را که قوم من نبودند، قوم خود خواهم خواند و او را که دوست نداشتم محبوبه خود.
\par 26 و جایی که به ایشان گفته شد که شما قوم من نیستید، در آنجا پسران خدای حی خوانده خواهند شد.»
\par 27 و اشعیا نیز در حق اسرائیل ندامی کند که «هرچند عدد بنی‌اسرائیل مانند ریگ دریا باشد، لکن بقیه نجات خواهند یافت؛
\par 28 زیراخداوند کلام خود را تمام و منقطع ساخته، برزمین به عمل خواهد آورد.»
\par 29 و چنانکه اشعیاپیش اخبار نمود که «اگر رب الجنود برای ما نسلی نمی گذارد، هرآینه مثل سدوم می‌شدیم و مانندغموره می‌گشتیم.»
\par 30 پس چه گوییم؟ امت هایی که در‌پی عدالت نرفتند، عدالت را حاصل نمودند، یعنی عدالتی که از ایمان است.
\par 31 لکن اسرائیل که در‌پی شریعت عدالت می‌رفتند، به شریعت عدالت نرسیدند.
\par 32 از چه سبب؟ از این جهت که نه از راه ایمان بلکه از راه اعمال شریعت آن را طلبیدند، زیرا که به سنگ مصادم لغزش خوردند.چنانکه مکتوب است که «اینک در صهیون سنگی مصادم و صخره لغزش می‌نهم و هر‌که براو ایمان آورد، خجل نخواهد گردید.»
\par 33 چنانکه مکتوب است که «اینک در صهیون سنگی مصادم و صخره لغزش می‌نهم و هر‌که براو ایمان آورد، خجل نخواهد گردید.»

\chapter{10}

\par 1 ای برادران خوشی دل من و دعای من نزد خدا بجهت اسرائیل برای نجات ایشان است.
\par 2 زیرا بجهت ایشان شهادت می‌دهم که برای خدا غیرت دارند لکن نه از روی معرفت.
\par 3 زیرا که چون عدالت خدا را نشناخته، می‌خواستند عدالت خود را ثابت کنند، مطیع عدالت خدا نگشتند.
\par 4 زیرا که مسیح است انجام شریعت بجهت عدالت برای هر کس که ایمان آورد.
\par 5 زیرا موسی عدالت شریعت را بیان می‌کندکه «هر‌که به این عمل کند، در این خواهدزیست.»
\par 6 لکن عدالت ایمان بدینطور سخن می‌گوید که «در خاطر خود مگو کیست که به آسمان صعود کند یعنی تا مسیح را فرود آورد،
\par 7 یا کیست که به هاویه نزول کند یعنی تا مسیح رااز مردگان برآورد.»
\par 8 لکن چه می‌گوید؟ اینکه «کلام نزد تو و در دهانت و در قلب تو است یعنی‌این کلام ایمان که به آن وعظ می‌کنیم.»
\par 9 زیرا اگربه زبان خود عیسی خداوند را اعتراف کنی و دردل خود ایمان آوری که خدا او را از مردگان برخیزانید، نجات خواهی یافت.
\par 10 چونکه به دل ایمان آورده می‌شود برای عدالت و به زبان اعتراف می‌شود بجهت نجات.
\par 11 و کتاب می‌گوید «هر‌که به او ایمان آورد خجل نخواهدشد.»
\par 12 زیرا که در یهود و یونانی تفاوتی نیست که همان خداوند، خداوند همه است و دولتمنداست برای همه که نام او را می‌خوانند.
\par 13 زیرا هرکه نام خداوند را بخواند نجات خواهد یافت.
\par 14 پس چگونه بخوانند کسی را که به او ایمان نیاورده‌اند؟ و چگونه ایمان آورند به کسی‌که خبراو را نشنیده‌اند؟ و چگونه بشنوند بدون واعظ؟
\par 15 و چگونه وعظ کنند جز اینکه فرستاده شوند؟ چنانکه مکتوب است که «چه زیبا است پایهای آنانی که به سلامتی بشارت می‌دهند و به چیزهای نیکو مژده می‌دهند.»
\par 16 لکن همه بشارت را گوش نگرفتند زیرا اشعیا می‌گوید «خداوندا کیست که اخبار ما را باور کرد؟»
\par 17 لهذا ایمان از شنیدن است و شنیدن از کلام خدا.
\par 18 لکن می‌گویم آیانشنیدند؟ البته شنیدند: «صوت ایشان در تمام جهان منتشر گردید و کلام ایشان تا اقصای ربع مسکون رسید.»
\par 19 و می‌گویم آیا اسرائیل ندانسته‌اند؟ اول موسی می‌گوید: «من شما را به غیرت می‌آورم به آن که امتی نیست و بر قوم بی‌فهم شما را خشمگین خواهم ساخت.»
\par 20 واشعیا نیز جرات کرده، می‌گوید: آنانی که طالب من نبودند مرا یافتند و به کسانی که مرا نطلبیدندظاهر گردیدم.»اما در حق اسرائیل می‌گوید: «تمام روز دستهای خود را دراز کردم به سوی قومی نامطیع و مخالف.»
\par 21 اما در حق اسرائیل می‌گوید: «تمام روز دستهای خود را دراز کردم به سوی قومی نامطیع و مخالف.»

\chapter{11}

\par 1 پس می‌گویم آیا خدا قوم خود را ردکرد؟ حاشا! زیرا که من نیز اسرائیلی ازاولاد ابراهیم از سبط بنیامین هستم.
\par 2 خدا قوم خود را که از قبل شناخته بود، رد نفرموده است. آیا نمی دانید که کتاب در الیاس چه می‌گوید، چگونه بر اسرائیل از خدا استغاثه می‌کند
\par 3 که «خداوندا انبیای تو را کشته و مذبحهای تو راکنده‌اند و من به تنهایی مانده‌ام و در قصد جان من نیز می‌باشند»؟
\par 4 لکن وحی بدو چه می‌گوید؟ اینکه «هفت هزار مرد بجهت خود نگاه داشتم که به نزد بعل زانو نزده‌اند».
\par 5 پس همچنین در زمان حاضر نیز بقیتی بحسب اختیار فیض مانده است.
\par 6 و اگر از راه فیض است دیگر از اعمال نیست وگرنه فیض دیگر فیض نیست. اما اگر از اعمال است دیگر از فیض نیست والا عمل دیگر عمل نیست.
\par 7 پس مقصود چیست؟ اینکه اسرائیل آنچه راکه می‌طلبد نیافته است، لکن برگزیدگان یافتند وباقی ماندگان سختدل گردیدند؛
\par 8 چنانکه مکتوب است که «خدا بدیشان روح خواب‌آلود دادچشمانی که نبیند و گوشهایی که نشنود تا امروز.»
\par 9 و داود می‌گوید که «مائده ایشان برای ایشان تله و دام و سنگ مصادم و عقوبت باد؛
\par 10 چشمان ایشان تار شود تا نبینند و پشت ایشان را دائم خم گردان.»
\par 11 پس می‌گویم آیا لغزش خوردند تا بیفتند؟ حاشا! بلکه از لغزش ایشان نجات به امت‌ها رسیدتا در ایشان غیرت پدید آورد.
\par 12 پس چون لغزش ایشان دولتمندی جهان گردید و نقصان ایشان دولتمندی امت‌ها، به چند مرتبه زیادترپری ایشان خواهد بود.
\par 13 زیرا به شما‌ای امت هاسخن می‌گویم پس از این‌روی که رسول امت هامی باشم خدمت خود را تمجید می‌نمایم،
\par 14 تاشاید ابنای جنس خود را به غیرت آورم و بعضی از ایشان را برهانم.
\par 15 زیرا اگر رد شدن ایشان مصالحت عالم شد، باز‌یافتن ایشان چه خواهدشد؟ جز حیات از مردگان!
\par 16 و چون نوبر مقدس است، همچنان خمیره و هرگاه ریشه مقدس است، همچنان شاخه‌ها.
\par 17 و چون بعضی از شاخه‌ها بریده شدند و تو که زیتون بری بودی در آنها پیوند گشتی و در ریشه وچربی زیتون شریک شدی،
\par 18 بر شاخه‌ها فخرمکن و اگر فخر کنی تو حامل ریشه نیستی بلکه ریشه حامل تو است.
\par 19 پس می‌گویی که «شاخه‌ها بریده شدند تا من پیوند شوم؟»
\par 20 آفرین بجهت بی‌ایمانی بریده شدند و تومحض ایمان پایدار هستی. مغرور مباش بلکه بترس!
\par 21 زیرا اگر خدا بر شاخه های طبیعی شفقت نفرمود، بر تو نیز شفقت نخواهد کرد.
\par 22 پس مهربانی و سختی خدا را ملاحظه نما؛ اماسختی بر آنانی که افتادند، اما مهربانی برتو اگر درمهربانی ثابت باشی والا تو نیز بریده خواهی شد.
\par 23 و اگر ایشان نیز در بی‌ایمانی نمانند باز پیوندخواهند شد، زیرا خدا قادر است که ایشان را باردیگر بپیوندد.
\par 24 زیرا اگر تو از زیتون طبیعی بری بریده شده، برخلاف طبع به زیتون نیکو پیوندگشتی، به چند مرتبه زیادتر آنانی که طبیعی‌اند درزیتون خویش پیوند خواهند شد.
\par 25 زیرا‌ای برادران نمی خواهم شما از این سربی خبر باشید که مبادا خود را دانا انگارید که مادامی که پری امت‌ها درنیاید، سختدلی بربعضی از اسرائیل طاری گشته است.
\par 26 وهمچنین همگی اسرائیل نجات خواهند یافت، چنانکه مکتوب است که «از صهیون نجات‌دهنده‌ای ظاهر خواهد شد و بی‌دینی را ازیعقوب خواهد برداشت؛
\par 27 و این است عهد من باایشان در زمانی که گناهانشان را بردارم.»
\par 28 نظر به انجیل بجهت شما دشمنان‌اند، لکن نظر به اختیاربه‌خاطر اجداد محبوبند.
\par 29 زیرا که در نعمتهاودعوت خدا بازگشتن نیست.
\par 30 زیرا همچنان‌که شما در سابق مطیع خدا نبودید و الان به‌سبب نافرمانی‌ایشان رحمت یافتید،
\par 31 همچنین ایشان نیز الان نافرمان شدند تا بجهت رحمتی که بر شما است بر ایشان نیز رحم شود
\par 32 زیرا خدا همه رادر نافرمانی بسته است تا بر همه رحم فرماید.
\par 33 زهی عمق دولتمندی و حکمت و علم خدا! چقدر بعید از غوررسی است احکام او وفوق از کاوش است طریقهای وی!
\par 34 زیرا کیست که رای خداوند را دانسته باشد؟ یا که مشیر اوشده؟
\par 35 یا که سبقت جسته چیزی بدو داده تا به او باز داده شود؟زیرا که از او و به او و تا او همه‌چیز است؛ و او را تا ابدالاباد جلال باد، آمین.
\par 36 زیرا که از او و به او و تا او همه‌چیز است؛ و او را تا ابدالاباد جلال باد، آمین.

\chapter{12}

\par 1 زندگی مسیحی لهذا‌ای برادران شما را به رحمتهای خدا استدعا می‌کنم که بدنهای خود راقربانی زنده مقدس پسندیده خدا بگذرانید که عبادت معقول شما است.
\par 2 و همشکل این جهان مشوید بلکه به تازگی ذهن خود صورت خود راتبدیل دهید تا شما دریافت کنید که اراده نیکوی پسندیده کامل خدا چیست.
\par 3 زیرا به آن فیضی که به من عطا شده است، هریکی از شما را می‌گویم که فکرهای بلندتر ازآنچه شایسته است مکنید بلکه به اعتدال فکرنمایید، به اندازه آن بهره ایمان که خدا به هر کس قسمت فرموده است.
\par 4 زیرا همچنان‌که در یک بدن اعضای بسیار داریم و هر عضوی را یک کارنیست،
\par 5 همچنین ما که بسیاریم، یک جسدهستیم در مسیح، اما فرد اعضای یکدیگر.
\par 6 پس چون نعمتهای مختلف داریم بحسب فیضی که به ما داده شد، خواه نبوت برحسب موافقت ایمان،
\par 7 یا خدمت در خدمت گذاری، یا معلم در تعلیم،
\par 8 یا واعظ در موعظه، یا بخشنده به سخاوت، یاپیشوا به اجتهاد، یا رحم کننده به‌سرور.
\par 9 محبت بی‌ریا باشد. از بدی نفرت کنید و به نیکویی بپیوندید.
\par 10 با محبت برادرانه یکدیگر رادوست دارید و هر یک دیگری را بیشتر از خوداکرام بنماید.
\par 11 در اجتهاد کاهلی نورزید و درروح سرگرم شده، خداوند را خدمت نمایید.
\par 12 در امید مسرور و در مصیبت صابر و در دعامواظب باشید.
\par 13 مشارکت در احتیاجات مقدسین کنید و در مهمانداری ساعی باشید.
\par 14 برکت بطلبید بر آنانی که بر شما جفا کنند؛ برکت بطلبید و لعن مکنید.
\par 15 خوشی کنید باخوشحالان و ماتم نمایید با ماتمیان.
\par 16 برای یکدیگر همان فکر داشته باشید و در چیزهای بلند فکر مکنید بلکه با ذلیلان مدارا نمایید و خودرا دانا مشمارید.
\par 17 هیچ‌کس را به عوض بدی بدی مرسانید. پیش جمیع مردم تدارک کارهای نیکو بینید.
\par 18 اگر ممکن است بقدر قوه خود باجمیع خلق به صلح بکوشید.
\par 19 ‌ای محبوبان انتقام خود را مکشید بلکه خشم را مهلت دهید، زیرا مکتوب است «خداوند می‌گوید که انتقام ازآن من است من جزا خواهم داد.»
\par 20 پس «اگردشمن تو گرسنه باشد، او را سیر کن و اگر تشنه است، سیرابش نما زیرا اگر چنین کنی اخگرهای آتش بر سرش خواهی انباشت.»مغلوب بدی مشو بلکه بدی را به نیکویی مغلوب ساز.
\par 21 مغلوب بدی مشو بلکه بدی را به نیکویی مغلوب ساز.

\chapter{13}

\par 1 هر شخص مطیع قدرتهای برتر بشود، زیرا که قدرتی جز از خدا نیست و آنهایی که هست از جانب خدا مرتب شده است.
\par 2 حتی هر‌که با قدرت مقاومت نماید، مقاومت باترتیب خدا نموده باشد و هر‌که مقاومت کند، حکم بر خود آورد.
\par 3 زیرا از حکام عمل نیکو راخوفی نیست بلکه عمل بد را. پس اگر می‌خواهی که از آن قدرت ترسان نشوی، نیکویی کن که از اوتحسین خواهی یافت.
\par 4 زیرا خادم خداست برای تو به نیکویی؛ لکن هرگاه بدی کنی، بترس چونکه شمشیر را عبث برنمی دارد، زیرا او خادم خداست و با غضب انتقام از بدکاران می‌کشد.
\par 5 لهذا لازم است که مطیع او شوی نه به‌سبب غضب فقط بلکه به‌سبب ضمیر خود نیز.
\par 6 زیرا که به این سبب باج نیز می‌دهید، چونکه خدام خدا ومواظب در همین امر هستند.
\par 7 پس حق هرکس رابه او ادا کنید: باج را به مستحق باج و جزیه را به مستحق جزیه و ترس را به مستحق ترس و عزت را به مستحق عزت.
\par 8 مدیون احدی به چیزی مشوید جز به محبت نمودن با یکدیگر، زیرا کسی‌که دیگری رامحبت نماید شریعت را به‌جا آورده باشد.
\par 9 زیراکه زنا مکن، قتل مکن، دزدی مکن، شهادت دروغ مده، طمع مورز و هر حکمی دیگر که هست، همه شامل است در این کلام که همسایه خود راچون خود محبت نما.
\par 10 محبت به همسایه خودبدی نمی کند پس محبت تکمیل شریعت است.
\par 11 و خصوص چون وقت را می‌دانید که الحال ساعت رسیده است که ما را باید از خواب بیدارشویم زیرا که الان نجات ما نزدیک تر است از آن وقتی که ایمان آوردیم.
\par 12 شب منقضی شد و روز نزدیک آمد. پس اعمال تاریکی را بیرون کرده، اسلحه نور را بپوشیم.
\par 13 و با شایستگی رفتار کنیم چنانکه در روز، نه در بزمها و سکرها وفسق و فجور و نزاع و حسد؛بلکه عیسی مسیح خداوند را بپوشید و برای شهوات جسمانی تدارک نبینید.
\par 14 بلکه عیسی مسیح خداوند را بپوشید و برای شهوات جسمانی تدارک نبینید.

\chapter{14}

\par 1 و کسی را که در ایمان ضعیف باشدبپذیرید، لکن نه برای محاجه درمباحثات.
\par 2 یکی ایمان دارد که همه‌چیز را بایدخورد اما آنکه ضعیف است بقول می‌خورد.
\par 3 پس خورنده ناخورنده را حقیر نشمارد وناخورنده بر خورنده حکم نکند زیرا خدا او راپذیرفته است.
\par 4 تو کیستی که بر بنده کسی دیگرحکم می‌کنی؟ او نزد آقای خود ثابت یا ساقطمی شود. لیکن استوار خواهد شد زیرا خدا قادراست که او را ثابت نماید.
\par 5 یکی یک روز را از دیگری بهتر می‌داند ودیگری هر روز را برابر می‌شمارد. پس هر کس درذهن خود متیقن بشود.
\par 6 آنکه روز را عزیز می‌داندبخاطر خداوند عزیزش می‌دارد و آنکه روز راعزیز نمی دارد هم برای خداوند نمی دارد؛ و هرکه می‌خورد برای خداوند می‌خورد زیرا خدا راشکر می‌گوید، و آنکه نمی خورد برای خداوندنمی خورد و خدا را شکر می‌گوید.
\par 7 زیرا احدی از ما به خود زیست نمی کند و هیچ‌کس به خودنمی میرد.
\par 8 زیرا اگر زیست کنیم برای خداوندزیست می‌کنیم و اگر بمیریم برای خداوند می میریم. پس خواه زنده باشیم، خواه بمیریم، ازآن خداوندیم.
\par 9 زیرا برای همین مسیح مرد وزنده گشت تا بر زندگان و مردگان سلطنت کند.
\par 10 لکن تو چرا بر برادر خود حکم می‌کنی؟ یا تونیز چرا برادر خود را حقیر می‌شماری؟ زانرو که همه پیش مسند مسیح حاضر خواهیم شد.
\par 11 زیرا مکتوب است «خداوند می‌گوید به حیات خودم قسم که هر زانویی نزد من خم خواهد شد وهر زبانی به خدا اقرار خواهد نمود.»
\par 12 پس هریکی از ما حساب خود را به خدا خواهد داد.
\par 13 بنابراین بر یکدیگر حکم نکنیم بلکه حکم کنید به اینکه کسی سنگی مصادم یا لغزشی در راه برادر خود ننهد.
\par 14 می‌دانم و در عیسی خداوندیقین می‌دارم که هیچ‌چیز در ذات خود نجس نیست جز برای آن کسی‌که آن را نجس پندارد؛ برای او نجس است.
\par 15 زیرا هرگاه برادرت به خوراک آزرده شود، دیگر به محبت رفتارنمی کنی. به خوراک خود هلاک مساز کسی را که مسیح در راه او بمرد.
\par 16 پس مگذارید که نیکویی شما را بد گویند.
\par 17 زیرا ملکوت خدا اکل وشرب نیست بلکه عدالت و سلامتی و خوشی درروح‌القدس.
\par 18 زیرا هر‌که در این امور خدمت مسیح را کند، پسندیده خدا و مقبول مردم است.
\par 19 پس آن اموری را که منشا سلامتی و بنای یکدیگر است پیروی نمایید.
\par 20 بجهت خوراک کار خدا را خراب مساز. البته همه‌چیز پاک است، لیکن بد است برای آن شخص که برای لغزش می‌خورد.
\par 21 گوشت نخوردن و شراب ننوشیدن و کاری نکردن که باعث ایذا یا لغزش یا ضعف برادرت باشد نیکواست.
\par 22 آیا تو ایمان داری؟ پس برای خودت درحضور خدا آن را بدار، زیرا خوشابحال کسی‌که بر خود حکم نکند در آنچه نیکو می‌شمارد.لکن آنکه شک دارد اگر بخورد ملزم می‌شود، زیرا به ایمان نمی خورد؛ و هر‌چه از ایمان نیست گناه است.
\par 23 لکن آنکه شک دارد اگر بخورد ملزم می‌شود، زیرا به ایمان نمی خورد؛ و هر‌چه از ایمان نیست گناه است.

\chapter{15}

\par 1 و ما که توانا هستیم، ضعفهای ناتوانان رامتحمل بشویم و خوشی خود را طالب نباشیم.
\par 2 هر یکی از ما همسایه خود را خوش بسازد در آنچه برای بنا نیکو است.
\par 3 زیرا مسیح نیز خوشی خود را طالب نمی بود، بلکه چنانکه مکتوب است «ملامتهای ملامت کنندگان تو برمن طاری گردید.»
\par 4 زیرا همه‌چیزهایی که از قبل مکتوب شد، برای تعلیم ما نوشته شد تا به صبر وتسلی کتاب امیدوار باشیم.
\par 5 الان خدای صبر و تسلی شما را فیض عطاکناد تا موافق مسیح عیسی با یکدیگر یکرای باشید.
\par 6 تا یکدل و یکزبان شده، خدا و پدرخداوند ما عیسی مسیح را تمجید نمایید.
\par 7 پس یکدیگر را بپذیرید، چنانکه مسیح نیز مارا پذیرفت برای جلال خدا.
\par 8 زیرا می‌گویم عیسی مسیح خادم ختنه گردید بجهت راستی خدا تا وعده های اجداد را ثابت گرداند،
\par 9 و تاامت‌ها خدا را تمجید نمایند به‌سبب رحمت اوچنانکه مکتوب است که «از این جهت تو را درمیان امت‌ها اقرار خواهم کرد و به نام تو تسبیح خواهم خواند.»
\par 10 و نیز می‌گوید «ای امت‌ها باقوم او شادمان شوید.»
\par 11 و ایض «ای جمیع امت‌ها خداوند را حمد گویید و‌ای تمامی قومهااو را مدح نمایید.»
\par 12 و اشعیا نیز می‌گوید که «ریشه یسا خواهد بود و آنکه برای حکمرانی امت‌ها مبعوث شود، امید امت‌ها بر وی خواهدبود.»
\par 13 الان خدای امید، شما را از کمال خوشی وسلامتی در ایمان پر سازد تا به قوت روح‌القدس در امید افزوده گردید.
\par 14 لکن‌ای برادران من، خود نیز درباره شمایقین می‌دانم که خود از نیکویی مملو و پر از کمال معرفت و قادر بر نصیحت نمودن یکدیگر هستید.
\par 15 لیکن‌ای برادران بسیار جسارت ورزیده، من خود نیز به شما جزئی نوشتم تا شما را یادآوری نمایم به‌سبب آن فیضی که خدا به من بخشیده است،
\par 16 تا خادم عیسی مسیح شوم برای امت هاو کهانت انجیل خدا را به‌جا آورم تا هدیه امت هامقبول افتد، مقدس شده به روح‌القدس.
\par 17 پس به مسیح عیسی در کارهای خدا فخر دارم.
\par 18 زیراجرات نمی کنم که سخنی بگویم جز در آن اموری که مسیح بواسطه من به عمل آورد، برای اطاعت امت‌ها در قول و فعل،
\par 19 به قوت آیات ومعجزات و به قوت روح خدا. بحدی که ازاورشلیم دور زده تا به الیرکون بشارت مسیح راتکمیل نمودم.
\par 20 اما حریص بودم که بشارت چنان بدهم، نه در جایی که اسم مسیح شهرت یافته بود، مبادا بر بنیاد غیری بنا نمایم.
\par 21 بلکه چنانکه مکتوب است «آنانی که خبر او را نیافتند، خواهند دید و کسانی که نشنیدند، خواهندفهمید.»
\par 22 بنابراین‌بارها از آمدن نزد شما ممنوع شدم.
\par 23 لکن چون الان مرا در این ممالک دیگرجایی نیست و سالهای بسیار است که مشتاق آمدن نزد شما بوده‌ام،
\par 24 هرگاه به اسپانیا سفرکنم، به نزد شما خواهم آمد زیرا امیدوار هستم که شما را در عبور ملاقات کنم و شما مرا به آن سوی مشایعت نمایید، بعد از آنکه از ملاقات شما اندکی سیر شوم.
\par 25 لکن الان عازم اورشلیم هستم تا مقدسین را خدمت کنم.
\par 26 زیرا که اهل مکادونیه و اخائیه مصلحت دیدند که زکاتی برای مفلسین مقدسین اورشلیم بفرستند،
\par 27 بدین رضادادند و بدرستی که مدیون ایشان هستند زیرا که چون امت‌ها از روحانیات ایشان بهره‌مندگردیدند، لازم شد که در جسمانیات نیز خدمت ایشان را بکنند.
\par 28 پس چون این را انجام دهم واین ثمر را نزد ایشان ختم کنم، از راه شما به اسپانیا خواهم آمد.
\par 29 و می‌دانم وقتی که به نزدشما آیم، در کمال برکت انجیل مسیح خواهم آمد.
\par 30 لکن‌ای برادران، از شما التماس دارم که بخاطر خداوند ما عیسی مسیح و به محبت روح (القدس )، برای من نزد خدا در دعاها جد وجهدکنید،
\par 31 تا از نافرمانان یهودیه رستگار شوم وخدمت من در اورشلیم مقبول مقدسین افتد،
\par 32 تابرحسب اراده خدا با خوشی نزد شما برسم و باشما استراحت یابم.و خدای سلامتی با همه شما باد، آمین.
\par 33 و خدای سلامتی با همه شما باد، آمین.

\chapter{16}

\par 1 و خواهر ما فیبی را که خادمه کلیسای در کنخریا است، به شما می‌سپارم
\par 2 تااو را در خداوند بطور شایسته مقدسین بپذیرید ودر هر چیزی که به شما محتاج باشد او را اعانت کنید، زیرا که او بسیاری را و خود مرا نیز معاونت می‌نمود.
\par 3 سلام برسانید به پرسکلا و اکیلا، همکاران من در مسیح عیسی
\par 4 که در راه جان من گردنهای خود را نهادند و نه من به تنهایی ممنون ایشان هستم، بلکه همه کلیساهای امت‌ها.
\par 5 کلیسا را که در خانه ایشان است و حبیب من اپینطس را که برای مسیح نوبر آسیاست سلام رسانید.
\par 6 و مریم را که برای شما زحمت بسیار کشید، سلام گویید.
\par 7 و اندرونیکوس و یونیاس خویشان مرا که با من اسیر می‌بودند سلام نمایید که مشهور در میان رسولان هستند و قبل از من در مسیح شدند.
\par 8 وامپلیاس را که در خداوند حبیب من است، سلام رسانید.
\par 9 و اوربانس که با ما در کار مسیح رفیق است و استاخیس حبیب مرا سلام نمایید.
\par 10 و اپلیس آزموده شده در مسیح را سلام برسانید و اهل خانه ارستبولس را سلام برسانید.
\par 11 و خویش من هیردیون را سلام دهیدو آنانی را از اهل خانه نرگسوس که در خداوندهستند سلام رسانید.
\par 12 طریفینا و طریفوسا را که در خداوند زحمت کشیده‌اند سلام گویید؛ وپرسیس محبوبه را که در خداوند زحمت بسیارکشید سلام دهید.
\par 13 و روفس برگزیده درخداوند و مادر او و مرا سلام بگویید.
\par 14 اسنکریطس را و فلیکون و هرماس وپطروباس و هرمیس و برادرانی که باایشانند سلام نمایید.
\par 15 فیلولکس را و جولیه و نیریاس و خواهرش و اولمپاس و همه مقدسانی که با ایشانند سلام برسانید.
\par 16 ویکدیگر را به بوسه مقدسانه سلام نمایید. وجمیع کلیساهای مسیح شما را سلام می‌فرستند.
\par 17 لکن‌ای برادران از شما استدعا می‌کنم آن کسانی را که منشا تفاریق و لغزشهای مخالف آن تعلیمی که شما یافته‌اید می‌باشند، ملاحظه کنیدو از ایشان اجتناب نمایید.
\par 18 زیرا که چنین اشخاص خداوند ما عیسی مسیح را خدمت نمی کنند بلکه شکم خود را و به الفاظ نیکو وسخنان شیرین دلهای ساده دلان را می‌فریبند.
\par 19 زیرا که اطاعت شما در جمیع مردم شهرت یافته است. پس درباره شما مسرور شدم. اماآرزوی این دارم که در نیکویی دانا و در بدی ساده دل باشید.
\par 20 و خدای سلامتی بزودی شیطان را زیرپایهای شما خواهد سایید.فیض خداوند ما عیسی مسیح با شما باد.
\par 21 تیموتاوس همکار من و لوقا و یاسون وسوسیپاطرس که خویشان منند شما را سلام می‌فرستند.
\par 22 من طرتیوس، کاتب رساله، شما رادر خداوند سلام می‌گویم.
\par 23 قایوس که مرا وتمام کلیسا را میزبان است، شما را سلام می‌فرستد. و ارسطس خزینه دار شهر و کوارطس برادر به شما سلام می‌فرستند.
\par 24 الان او را که قادر است که شما را استوار سازد، برحسب بشارت من و موعظه عیسی مسیح، مطابق کشف آن سری که از زمانهای ازلی مخفی بود،لکن درحال مکشوف شد و بوسیله کتب انبیا برحسب فرموده خدای سرمدی به جمیع امت‌ها بجهت اطاعت ایمان آشکارا گردید،
\par 25 لکن درحال مکشوف شد و بوسیله کتب انبیا برحسب فرموده خدای سرمدی به جمیع امت‌ها بجهت اطاعت ایمان آشکارا گردید،


\end{document}