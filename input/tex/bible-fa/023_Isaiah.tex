\begin{document}

\title{Isaiah}

 
\chapter{1}

\par 1 رویای اشعیا ابن آموص که آن را درباره یهودا و اورشلیم، در روزهای عزیا و یوتام و آحاز و حزقیا پادشاهان یهودا دید.
\par 2 ‌ای آسمان بشنو و‌ای زمین گوش بگیر زیراخداوند سخن می‌گوید. پسران پروردم وبرافراشتم اما ایشان بر من عصیان ورزیدند.
\par 3 گاومالک خویش را و الاغ آخور صاحب خود رامی شناسد، اما اسرائیل نمی شناسند و قوم من فهم ندارند.
\par 4 وای بر امت خطاکار و قومی که زیربار گناه می‌باشند و بر ذریت شریران و پسران مفسد. خداوند را ترک کردند و قدوس اسرائیل رااهانت نمودند و بسوی عقب منحرف شدند.
\par 5 چرا دیگر ضرب یابید و زیاده فتنه نمایید؟ تمامی سر بیمار است و تمامی دل مریض.
\par 6 ازکف پا تا به‌سر در آن تندرستی نیست بلکه جراحت و کوفتگی و زخم متعفن، که نه بخیه شده و نه بسته گشته و نه با روغن التیام شده است.
\par 7 ولایت شما ویران و شهرهای شما به آتش سوخته شده است. غریبان، زمین شما را در نظرشما می‌خورند و آن مثل واژگونی بیگانگان خراب گردیده است.
\par 8 و دختر صهیون مثل سایه بان در تاکستان و مانند کپر در بوستان خیار ومثل شهر محاصره شده، متروک است.
\par 9 اگر یهوه صبایوت بقیه اندکی برای ما وا نمی گذاشت، مثل سدوم می‌شدیم و مانند عموره می‌گشتیم.
\par 10 ‌ای حاکمان سدوم کلام خداوند را بشنویدو‌ای قوم عموره شریعت خدای ما را گوش بگیرید.
\par 11 خداوند می‌گوید از کثرت قربانی های شما مرا چه فایده است؟ از قربانی های سوختنی قوچها و پیه پرواریها سیر شده‌ام و به خون گاوان و بره‌ها و بزها رغبت ندارم.
\par 12 وقتی که می‌آیید تابه حضور من حاضر شوید، کیست که این را ازدست شما طلبیده است که دربار مرا پایمال کنید؟
\par 13 هدایای باطل دیگر میاورید. بخور نزدمن مکروه است و غره ماه و سبت و دعوت جماعت نیز. گناه را با محفل مقدس نمی توانم تحمل نمایم.
\par 14 غره‌ها و عیدهای شما را جان من نفرت دارد؛ آنها برای من بار سنگین است که ازتحمل نمودنش خسته شده‌ام.
\par 15 هنگامی که دستهای خود را دراز می‌کنید، چشمان خود را ازشما خواهم پوشانید و چون دعای بسیارمی کنید، اجابت نخواهم نمود زیرا که دستهای شما پر از خون است.
\par 16 خویشتن را شسته، طاهرنمایید و بدی اعمال خویش را از نظر من دورکرده، از شرارت دست بردارید.
\par 17 نیکوکاری رابیاموزید و انصاف را بطلبید. مظلومان را رهایی دهید. یتیمان را دادرسی کنید و بیوه‌زنان راحمایت نمایید.
\par 18 خداوند می‌گوید بیایید تا با همدیگر محاجه نماییم. اگر گناهان شما مثل ارغوان باشد مانند برف سفید خواهد شد و اگرمثل قرمز سرخ باشد، مانند پشم خواهد شد.
\par 19 اگر خواهش داشته، اطاعت نمایید، نیکویی زمین را خواهید خورد.
\par 20 اما اگر ابا نموده، تمردکنید شمشیر شما را خواهد خورد زیرا که دهان خداوند چنین می‌گوید.
\par 21 شهر امین چگونه زانیه شده است. آنکه ازانصاف مملو می‌بود و عدالت در وی سکونت می‌داشت، اما حال قاتلان.
\par 22 نقره تو به درد مبدل شده، و شراب تو از آب ممزوج گشته است.
\par 23 سروران تو متمرد شده و رفیق دزدان گردیده، هریک از ایشان رشوه را دوست می‌دارند و در‌پی هدایا می‌روند. یتیمان را دادرسی نمی نمایند ودعوی بیوه‌زنان نزد ایشان نمی رسد.
\par 24 بنابراین، خداوند یهوه صبایوت، قدیر اسرائیل می‌گویدهان من از خصمان خود استراحت خواهم یافت واز دشمنان خویش انتقام خواهم کشید.
\par 25 ودست خود را بر تو برگردانیده، درد تو را بالکل پاک خواهم کرد، و تمامی ریمت را دور خواهم ساخت.
\par 26 و داوران تو را مثل اول و مشیران تو رامثل ابتدا خواهم برگردانید و بعد از آن، به شهرعدالت و قریه امین مسمی خواهی شد.
\par 27 صهیون به انصاف فدیه داده خواهد شد وانابت کنندگانش به عدالت.
\par 28 و هلاکت عاصیان و گناهکاران با هم خواهد شد و آنانی که خداوندرا ترک نمایند، نابود خواهند گردید.
\par 29 زیراایشان از درختان بلوطی که شما خواسته بودیدخجل خواهند شد و از باغاتی که شما برگزیده بودید رسوا خواهند گردید.
\par 30 زیرا شما مثل بلوطی که برگش پژمرده و مانند باغی که آب نداشته باشد خواهید شد.و مرد زورآور پرزه کتان و عملش شعله خواهد شد و هردوی آنها باهم سوخته خواهند گردید و خاموش کننده‌ای نخواهد بود.
\par 31 و مرد زورآور پرزه کتان و عملش شعله خواهد شد و هردوی آنها باهم سوخته خواهند گردید و خاموش کننده‌ای نخواهد بود.
 
\chapter{2}

\par 1 کلامی که اشعیا ابن آموص درباره یهودا واورشلیم دید.
\par 2 و در ایام آخر واقع خواهد شد که کوه خانه خداوند برقله کوهها ثابت خواهد شد و فوق تلهابرافراشته خواهد گردید و جمیع امت‌ها بسوی آن روان خواهند شد.
\par 3 و قوم های بسیار عزیمت کرده، خواهند گفت: «بیایید تا به کوه خداوند و به خانه خدای یعقوب برآییم تا طریق های خویش را به ما تعلیم دهد و به راههای وی سلوک نماییم.» زیرا که شریعت از صهیون و کلام خداوند از اورشلیم صادر خواهد شد.
\par 4 و اوامت‌ها را داوری خواهد نمود و قوم های بسیاری را تنبیه خواهد کرد و ایشان شمشیرهای خود رابرای گاوآهن و نیزه های خویش را برای اره هاخواهند شکست و امتی بر امتی شمشیر نخواهدکشید و بار دیگر جنگ را نخواهند آموخت.
\par 5 ‌ای خاندان یعقوب بیایید تا در نور خداوند سلوک نماییم.
\par 6 زیرا قوم خود یعنی خاندان یعقوب را ترک کرده‌ای، چونکه از رسوم مشرقی مملو و مانندفلسطینیان فالگیر شده‌اند و با پسران غربا دست زده‌اند،
\par 7 و زمین ایشان از نقره و طلا پر شده وخزاین ایشان را انتهایی نیست، و زمین ایشان ازاسبان پر است و ارابه های ایشان را انتهایی نیست؛
\par 8 و زمین ایشان از بتها پر است؛ صنعت دستهای خویش را که به انگشتهای خود ساخته‌اند سجده می‌نمایند.
\par 9 و مردم خم شده و مردان پست می‌شوند. لهذا ایشان را نخواهی آمرزید.
\par 10 از ترس خداوند و از کبریای جلال وی به صخره داخل شده، خویشتن را در خاک پنهان کن.
\par 11 چشمان بلند انسان پست و تکبر مردان خم خواهد شد و در آن روز خداوند به تنهایی متعال خواهد بود.
\par 12 زیرا که برای یهوه صبایوت روزی است که بر هرچیز بلند و عالی خواهد آمد و برهرچیز مرتفع، و آنها پست خواهد شد؛
\par 13 و برهمه سروهای آزاد بلند و رفیع لبنان و بر تمامی بلوطهای باشان؛
\par 14 و بر همه کوههای عالی و برجمیع تلهای بلند؛
\par 15 و بر هربرج مرتفع و بر هرحصار منیع؛
\par 16 و برهمه کشتیهای ترشیش وبرهمه مصنوعات مرغوب؛
\par 17 و کبریای انسان خم شود و تکبر مردان پست خواهد شد. و در آن روز خداوند به تنهایی متعال خواهد بود،
\par 18 وبتها بالکل تلف خواهند شد.
\par 19 و ایشان به مغاره های صخره‌ها و حفره های خاک داخل خواهند شد، به‌سبب ترس خداوند وکبریای جلال وی هنگامی که او برخیزد تا زمین را متزلزل سازد.
\par 20 در آن روز مردمان، بتهای نقره و بتهای طلای خود را که برای عبادت خویش ساخته‌اند، نزد موش کوران و خفاشها خواهندانداخت،
\par 21 تا به مغاره های صخره‌ها و شکافهای سنگ خارا داخل شوند، به‌سبب ترس خداوند و کبریای جلال وی هنگامی که او برخیزد تا زمین را متزلزل سازد.شما از انسانی که نفس او دربینی‌اش می‌باشد دست برکشید زیرا که او به چه چیز محسوب می‌شود؟
\par 22 شما از انسانی که نفس او دربینی‌اش می‌باشد دست برکشید زیرا که او به چه چیز محسوب می‌شود؟
 
\chapter{3}

\par 1 زیرا اینک خداوند یهوه صبایوت پایه ورکن را از اورشلیم و یهودا، یعنی تمامی پایه نان و تمامی پایه آب را دور خواهد کرد،
\par 2 وشجاعان و مردان جنگی و داوران و انبیا وفالگیران و مشایخ را،
\par 3 و سرداران پنجاهه وشریفان و مشیران و صنعت گران ماهر و ساحران حاذق را.
\par 4 و اطفال را بر ایشان حاکم خواهم ساخت و کودکان بر ایشان حکمرانی خواهندنمود.
\par 5 و قوم مظلوم خواهند شد، هرکس ازدست دیگری و هرشخص از همسایه خویش. واطفال بر پیران و پستان بر شریفان تمرد خواهندنمود.
\par 6 چون شخصی به برادر خویش در خانه پدرش متمسک شده، بگوید: «تو را رخوت هست پس حاکم ما شو و این خرابی در زیر دست تو باشد»،
\par 7 در آن روز او آواز خود را بلند کرده، خواهد گفت: «من علاج کننده نتوانم شد زیرا درخانه من نه نان و نه لباس است پس مرا حاکم قوم مسازید.»
\par 8 زیرا اورشلیم خراب شده و یهودامنهدم گشته است، از آن جهت که لسان و افعال ایشان به ضد خداوند می‌باشد تا چشمان جلال اورا به ننگ‌آورند.
\par 9 سیمای رویهای ایشان به ضدایشان شاهد است و مثل سدوم گناهان خود رافاش کرده، آنها را مخفی نمی دارند. وای برجانهای ایشان زیرا که به جهت خویشتن شرارت را بعمل آورده‌اند.
\par 10 عادلان را بگویید که ایشان را سعادتمندی خواهد بود زیرا از ثمره اعمال خویش خواهند خورد.
\par 11 وای بر شریران که ایشان را بدی خواهد بود چونکه مکافات دست ایشان به ایشان کرده خواهد شد.
\par 12 و اما قوم من، کودکان بر ایشان ظلم می‌کنند و زنان بر ایشان حکمرانی می‌نمایند. ای قوم من، راهنمایان شماگمراه کنندگانند و طریق راههای شما را خراب می‌کنند.
\par 13 خداوند برای محاجه برخاسته و به جهت داوری قومها ایستاده است.
\par 14 خداوند بامشایخ قوم خود و سروران ایشان به محاکمه درخواهد آمد، زیرا شما هستید که تاکستانها راخورده‌اید و غارت فقیران در خانه های شمااست.
\par 15 خداوند یهوه صبایوت می‌گوید: «شمارا چه شده است که قوم مرا می‌کوبید و رویهای فقیران را خرد می‌نمایید؟»
\par 16 و خداوند می‌گوید: «از این جهت که دختران صهیون متکبرند و با گردن افراشته وغمزات چشم راه می‌روند و به ناز می‌خرامند و به پایهای خویش خلخالها را به صدا می‌آورند.»
\par 17 بنابراین خداوند فرق سر دختران صهیون را کل خواهد ساخت و خداوند عورت ایشان را برهنه خواهد نمود.
\par 18 و در آن روز خداوند زینت خلخالها و پیشانی بندها و هلالها را دور خواهدکرد.
\par 19 و گوشواره‌ها و دستبندها و روبندها را،
\par 20 و دستارها و زنجیرها و کمربندها و عطردانها وتعویذها را،
\par 21 و انگشترها و حلقه های بینی را،
\par 22 و رخوت نفیسه و رداها و شالها و کیسه‌ها را،
\par 23 و آینه‌ها و کتان نازک و عمامه‌ها و برقع‌ها را.
\par 24 و واقع می‌شود که به عوض عطریات، عفونت خواهد شد و به عوض کمربند، ریسمان و به عوض مویهای بافته، کلی و به عوض سینه بند، زنار پلاس و به عوض زیبایی، سوختگی خواهدبود.
\par 25 مردانت به شمشیر و شجاعانت در جنگ خواهند افتاد.و دروازه های وی ناله و ماتم خواهند کرد، و او خراب شده، بر زمین خواهدنشست.
\par 26 و دروازه های وی ناله و ماتم خواهند کرد، و او خراب شده، بر زمین خواهدنشست.
 
\chapter{4}

\par 1 و در آن روز هفت زن به یک مرد متمسک شده، خواهند گفت: «نان خود را خواهیم خورد و رخت خود را خواهیم پوشید، فقط نام توبر ما خوانده شود و عار ما را بردار.»
\par 2 در آن روز شاخه خداوند زیبا و جلیل ومیوه زمین به جهت ناجیان اسرائیل فخر و زینت خواهد بود.
\par 3 و واقع می‌شود که هرکه در صهیون باقی ماند و هر‌که در اورشلیم ترک شود مقدس خوانده خواهد شد یعنی هرکه در اورشلیم دردفتر حیات مکتوب باشد.
\par 4 هنگامی که خداوندچرک دختران صهیون را بشوید و خون اورشلیم را به روح انصاف و روح سوختگی از میانش رفع نماید،
\par 5 خداوند بر جمیع مساکن کوه صهیون وبر محفلهایش ابر و دود در روز و درخشندگی آتش مشتعل در شب خواهد آفرید، زیرا که برتمامی جلال آن پوششی خواهد بود.و در وقت روز سایه بانی به جهت سایه از گرما و به جهت ملجاء و پناهگاه از طوفان و باران خواهد بود.
\par 6 و در وقت روز سایه بانی به جهت سایه از گرما و به جهت ملجاء و پناهگاه از طوفان و باران خواهد بود.
 
\chapter{5}

\par 1 سرود محبوب خود را درباره تاکستانش برای محبوب خود بسرایم.
\par 2 و آن را کنده از سنگها پاک کرده و موبهترین در آن غرس نمود و برجی در میانش بناکرد و چرخشتی نیز در آن کند. پس منتظر می‌بودتا انگور بیاورد اما انگور بد آورد.
\par 3 پس الان‌ای ساکنان اورشلیم و مردان یهودا، در میان من وتاکستان من حکم کنید. 
\par 4 برای تاکستان من دیگرچه توان کرد که در آن نکردم؟ پس چون منتظربودم که انگور بیاورد چرا انگور بد آورد؟
\par 5 لهذاالان شما را اعلام می‌نمایم که من به تاکستان خودچه خواهم کرد. حصارش را برمی دارم و چراگاه خواهد شد؛ و دیوارش را منهدم می‌سازم وپایمال خواهد گردید.
\par 6 و آن را خراب می‌کنم که نه پازش و نه کنده خواهد شد و خار و خس در آن خواهد رویید، و ابرها را امر می‌فرمایم که بر آن باران نباراند.
\par 7 زیرا که تاکستان یهوه صبایوت خاندان اسرائیل است و مردان یهودا نهال شادمانی او می‌باشند. و برای انصاف انتظارکشید و اینک تعدی و برای عدالت و اینک فریادشد.
\par 8 وای بر آنانی که خانه را به خانه ملحق ومزرعه را به مزرعه ملصق سازند تا مکانی باقی نماند. و شما در میان زمین به تنهایی ساکن می‌شوید.
\par 9 یهوه صبایوت در گوش من گفت: «به درستی که خانه های بسیار خراب خواهد شد، وخانه های بزرگ و خوش نما غیرمسکون خواهدگردید.
\par 10 زیرا که ده جفت گاو زمین یک بت خواهد آورد و یک حومر تخم یک ایفه خواهدداد.»
\par 11 وای بر آنانی که صبح زود برمی خیزند تادر‌پی مسکرات بروند، و شب دیر می‌نشینند تاشراب ایشان را گرم نماید
\par 12 و در بزمهای ایشان عود و بربط و دف و نای و شراب می‌باشد. اما به فعل خداوند نظر نمی کنند و به عمل دستهای وی نمی نگرند.
\par 13 بنابراین قوم من به‌سبب عدم معرفت اسیر شده‌اند و شریفان ایشان گرسنه وعوام ایشان از تشنگی خشک گردیده.
\par 14 از این سبب هاویه حرص خود را زیاد کرده و دهان خویش را بی‌حد باز نموده است و جلال وجمهور و شوکت ایشان و هر‌که در ایشان شادمان باشد در آن فرو می‌رود.
\par 15 و مردم خم خواهندشد و مردان ذلیل خواهند گردید و چشمان متکبران پست خواهد شد.
\par 16 و یهوه صبایوت به انصاف متعال خواهد بود و خدای قدوس به عدالت تقدیس کرده خواهد شد.
\par 17 آنگاه بره های (غربا) در مرتع های ایشان خواهند چریدو غریبان ویرانه های پرواریهای ایشان را خواهندخورد.
\par 18 وای برآنانی که عصیان را به ریسمانهای بطالت و گناه را گویا به طناب ارابه می‌کشند.
\par 19 ومی گویند باشد که او تعجیل نموده، کار خود رابشتاباند تا آن را ببینیم. و مقصود قدوس اسرائیل نزدیک شده، بیاید تا آن را بدانیم.
\par 20 وای بر آنانی که بدی را نیکویی و نیکویی را بدی می‌نامند، که ظلمت را به‌جای نور و نور را به‌جای ظلمت می‌گذارند، و تلخی را به‌جای شیرینی و شیرینی را به‌جای تلخی می‌نهند.
\par 21 وای برآنانی که درنظر خود حکیمند، و پیش روی خود فهیم می‌نمایند.
\par 22 وای بر آنانی که برای نوشیدن شراب زورآورند، و به جهت ممزوج ساختن مسکرات مردان قوی می‌باشند.
\par 23 که شریران رابرای رشوه عادل می‌شمارند، و عدالت عادلان رااز ایشان برمی دارند.
\par 24 بنابراین به نهجی که شراره آتش کاه را می‌خورد و علف خشک در شعله می‌افتد، همچنان ریشه ایشان عفونت خواهد شد و شکوفه ایشان مثل غبار برافشانده خواهد گردید. چونکه شریعت یهوه صبایوت راترک کرده، کلام قدوس اسرائیل را خوارشمرده‌اند.
\par 25 بنابراین خشم خداوند بر قوم خودمشتعل شده و دست خود را بر ایشان دراز کرده، ایشان را مبتلا ساخته است. و کوهها بلرزیدند ولاشهای ایشان در میان کوچه‌ها مثل فضلات گردیده‌اند. با وجود این همه، غضب او برنگردیدو دست وی تا کنون دراز است.
\par 26 و علمی به جهت امت های بعید برپا خواهد کرد. و از اقصای زمین برای ایشان صفیر خواهد زد. و ایشان تعجیل نموده، بزودی خواهند آمد،
\par 27 و درمیان ایشان احدی خسته و لغزش خورنده نخواهد بودو احدی نه پینگی خواهد زد و نه خواهد خوابید. و کمربند کمر احدی از ایشان باز نشده، دوال نعلین احدی گسیخته نخواهد شد.
\par 28 که تیرهای ایشان، تیز و تمامی کمانهای ایشان، زده شده است. سمهای اسبان ایشان مثل سنگ خارا وچرخهای ایشان مثل گردباد شمرده خواهند شد.
\par 29 غرش ایشان مثل شیر ماده و مانند شیران ژیان غرش خواهند کرد و ایشان نعره خواهند زدو صید را گرفته، بسلامتی خواهند برد ورهاننده‌ای نخواهد بود.و در آن روز برایشان مثل شورش دریا شورش خواهند کرد. واگر کسی به زمین بنگرد، اینک تاریکی و تنگی است و نور در افلاک آن به ظلمت مبدل شده است.
\par 30 و در آن روز برایشان مثل شورش دریا شورش خواهند کرد. واگر کسی به زمین بنگرد، اینک تاریکی و تنگی است و نور در افلاک آن به ظلمت مبدل شده است.
 
\chapter{6}

\par 1 در سالی که عزیا پادشاه مرد، خداوند رادیدم که بر کرسی بلند و عالی نشسته بود، و هیکل از دامنهای وی پر بود.
\par 2 و سرافین بالای آن ایستاده بودند که هر یک از آنها شش بال داشت، و با دو از آنها روی خود را می‌پوشانید و بادو پایهای خود را می‌پوشانید و با دو پروازمی نمود.
\par 3 و یکی دیگری را صدا زده، می‌گفت: «قدوس قدوس قدوس یهوه صبایوت، تمامی زمین از جلال او مملو است.»
\par 4 و اساس آستانه ازآواز او که صدا می‌زد می‌لرزید و خانه از دود پرشد.
\par 5 پس گفتم: «وای بر من که هلاک شده‌ام زیراکه مرد ناپاک لب هستم و در میان قوم ناپاک لب ساکنم و چشمانم یهوه صبایوت پادشاه را دیده است.»
\par 6 آنگاه یکی از سرافین نزد من پرید و در دست خود اخگری که با انبر از روی مذبح گرفته بود، داشت.
\par 7 و آن را بر دهانم گذارده، گفت که «اینک این لبهایت را لمس کرده است و عصیانت رفع شده و گناهت کفاره گشته است.»
\par 8 آنگاه آواز خداوند را شنیدم که می‌گفت: «که را بفرستم و کیست که برای ما برود؟» گفتم: «لبیک مرا بفرست.»
\par 9 گفت: «برو و به این قوم بگو البته خواهید شنید، اما نخواهید فهمید و هرآینه خواهید نگریست اما درک نخواهید کرد.
\par 10 دل این قوم را فربه ساز و گوشهای ایشان را سنگین نما و چشمان ایشان را ببند، مبادا با چشمان خودببینند و با گوشهای خود بشنوند و با دل خود بفهمند و بازگشت نموده، شفا یابند.»
\par 11 پس من گفتم: «ای خداوند تا به کی؟» او گفت: «تا وقتی که شهرها ویران گشته، غیر مسکون باشد و خانه هابدون آدمی و زمین خراب و ویران شود.
\par 12 وخداوند مردمان را دور کند و در میان زمین خرابیهای بسیار شود.اما باز عشری در آن خواهد بود و آن نیز بار دیگر تلف خواهد گردیدمثل درخت بلوط و چنار که چون قطع می‌شودکنده آنها باقی می‌ماند، همچنان ذریت مقدس کنده آن خواهد بود.»
\par 13 اما باز عشری در آن خواهد بود و آن نیز بار دیگر تلف خواهد گردیدمثل درخت بلوط و چنار که چون قطع می‌شودکنده آنها باقی می‌ماند، همچنان ذریت مقدس کنده آن خواهد بود.»
 
\chapter{7}

\par 1 و در ایام آحاز بن یوتام بن عزیا پادشاه یهودا، واقع شد که رصین، پادشاه آرام وفقح بن رملیا پادشاه اسرائیل بر اورشلیم برآمدندتا با آن جنگ نمایند، اما نتوانستند آن را فتح نمایند.
\par 2 و به خاندان داود خبر داده، گفتند که ارام در افرایم اردو زده‌اند و دل او و دل مردمانش بلرزید به طوری که درختان جنگل از باد می‌لرزد.
\par 3 آنگاه خداوند به اشعیا گفت: «تو با پسر خودشاریاشوب به انتهای قنات برکه فوقانی به راه مزرعه گازر به استقبال آحاز بیرون شو.
\par 4 و وی رابگو: باحذر و آرام باش مترس و دلت ضعیف نشود از این دو‌دم مشعل دودافشان، یعنی ازشدت خشم رصین و ارام و پسر رملیا.
\par 5 زیرا که ارام با افرایم و پسر رملیا برای ضرر تو مشورت کرده، می‌گویند:
\par 6 بر یهودا برآییم و آن رامحاصره کرده، به جهت خویشتن تسخیر نماییم و پسر طبئیل را در آن به پادشاهی نصب کنیم.»
\par 7 خداوند یهوه چنین می‌گوید که «این بجا آورده نمی شود و واقع نخواهد گردید.
\par 8 زیرا که سرارام، دمشق و سر دمشق، رصین است و بعد ازشصت و پنج سال افرایم شکسته می‌شود به طوری که دیگر قومی نخواهد بود.
\par 9 و سر افرایم سامره و سر سامره پسر رملیا است و اگر باورنکنید هرآینه ثابت نخواهید ماند.»
\par 10 و خداوند بار دیگر آحاز را خطاب کرده، گفت:
\par 11 «آیتی به جهت خود از یهوه خدایت بطلب. آن را یا از عمق‌ها بطلب یا از اعلی علیین بالا.»
\par 12 آحاز گفت: «نمی طلبم و خداوند راامتحان نخواهم نمود.»
\par 13 گفت: «ای خاندان داود بشنوید! آیا شما راچیزی سهل است که مردمان را بیزار کنید بلکه می‌خواهید خدای مرا نیز بیزار کنید.
\par 14 بنابراین خود خداوند به شما آیتی خواهد داد: اینک باکره حامله شده، پسری خواهد زایید و نام او راعمانوئیل خواهد خواند.
\par 15 کره و عسل خواهدخورد تا آنکه ترک کردن بدی و اختیار کردن خوبی را بداند.
\par 16 زیرا قبل از آنکه پسر، ترک نمودن بدی و اختیار کردن خوبی را بداند، زمینی که شما از هر دو پادشاه آن می‌ترسید، متروک خواهد شد.
\par 17 خداوند بر تو و بر قومت و برخاندان پدرت ایامی را خواهد آورد که از ایامی که افرایم از یهودا جدا شد تا حال نیامده باشدیعنی پادشاه آشور را.»
\par 18 و در آن روز واقع خواهد شد که خداوند برای مگسهایی که به کناره های نهرهای مصرند و زنبورهایی که درزمین آشورند صفیر خواهد زد.
\par 19 و تمامی آنهابرآمده، در وادیهای ویران و شکافهای صخره وبر همه بوته های خاردار و بر همه مرتع‌ها فرود خواهند آمد.
\par 20 و در آن روز خداوند به واسطه استره‌ای که از ماورای نهر اجیر می‌شود یعنی به واسطه پادشاه آشور موی سر و موی پایها راخواهد تراشید و ریش هم سترده خواهد شد.
\par 21 و در آن روز واقع خواهد شد که شخصی یک گاو جوان و دو گوسفند زنده نگاه خواهد داشت.
\par 22 و از فراوانی شیری که می‌دهند کره خواهدخورد زیرا هرکه در میان زمین باقی ماند خوراکش کره و عسل خواهد بود.
\par 23 و در آن روز هر مکانی که هزار مو به جهت هزار پاره نقره داده می‌شد پراز خار و خس خواهد بود.
\par 24 با تیرها و کمانهامردم به آنجا خواهند آمد زیرا که تمامی زمین پراز خار و خس خواهد شد.و جمیع کوههایی که با بیل کنده می‌شد، از ترس خار و خس به آنجانخواهند آمد بلکه گاوان را به آنجا خواهندفرستاد و گوسفندان آن را پایمال خواهند کرد.
\par 25 و جمیع کوههایی که با بیل کنده می‌شد، از ترس خار و خس به آنجانخواهند آمد بلکه گاوان را به آنجا خواهندفرستاد و گوسفندان آن را پایمال خواهند کرد.
 
\chapter{8}

\par 1 و خداوند مرا گفت: «لوحی بزرگ به جهت خود بگیر و بر آن با قلم انسان برای مهیرشلال حاش بز بنویس.
\par 2 و من شهود امین یعنی اوریای کاهن و زکریا ابن یبرکیا را به جهت خودبرای شهادت می‌گیرم.»
\par 3 پس من به نبیه نزدیکی کردم و او حامله شده، پسری زایید. آنگاه خداوند به من گفت: «او رامهیر شلال حاش بز بنام،
\par 4 زیرا قبل از آنکه طفل بتواند‌ای پدرم و‌ای مادرم بگوید، اموال دمشق وغنیمت سامره را پیش پادشاه آشور به یغماخواهند برد.»
\par 5 و خداوند بار دیگر مرا باز خطاب کرده، گفت:
\par 6 «چونکه این قوم آبهای شیلوه را که به ملایمت جاری می‌شود خوار شمرده، از رصین وپسر رملیا مسرور شده‌اند،
\par 7 بنابراین اینک خداوند آبهای زورآور بسیار نهر یعنی پادشاه آشور و تمامی حشمت او را بر ایشان برخواهدآورد و او از جمیع جویهای خود برخواهد آمد واز تمامی کناره های خویش سرشار خواهد شد،
\par 8 و بر یهودا تجاوز نموده، سیلان کرده، عبورخواهد نمود تا آنکه به گردنها برسد و بالهای خودرا پهن کرده، طول و عرض ولایتت را‌ای عمانوئیل پر خواهد ساخت.»
\par 9 به هیجان آیید‌ای قومها و شکست خواهیدیافت و گوش گیرید‌ای اقصای زمین و کمر خودرا ببندید و شکست خواهید یافت. کمر خود راببندید و شکست خواهید یافت.
\par 10 با هم مشورت کنید و باطل خواهد شد و سخن گویید وبجا آورده نخواهد شد زیرا خدا با ما است.
\par 11 چونکه خداوند با دست قوی به من چنین گفت و مرا تعلیم داد که به راه این قوم سلوک ننمایم وگفت:
\par 12 «هرآنچه را که این قوم فتنه می‌نامند، شما آن را فتنه ننامید و از ترس ایشان ترسان وخائف مباشید.
\par 13 یهوه صبایوت را تقدیس نمایید و او ترس و خوف شما باشد.
\par 14 و او(برای شما) مکان مقدس خواهد بود اما برای هردو خاندان اسرائیل سنگ مصادم و صخره لغزش دهنده و برای ساکنان اورشلیم دام و تله. 
\par 15 وبسیاری از ایشان لغزش خورده، خواهند افتاد وشکسته شده و بدام افتاده، گرفتار خواهند گردید.»
\par 16 شهادت را به هم بپیچ و شریعت را درشاگردانم مختوم ساز.
\par 17 و من برای خداوند که روی خود را از خاندان یعقوب مخفی می‌سازدانتظار کشیده، امیدوار او خواهم بود.
\par 18 اینک من و پسرانی که خداوند به من داده است، از جانب یهوه صبایوت که در کوه صهیون ساکن است به جهت اسرائیل آیات و علامات هستیم.
\par 19 وچون ایشان به شما گویند که از اصحاب اجنه وجادوگرانی که جیک جیک و زمزم می‌کنند سوال کنید، (گویید) «آیا قوم از خدای خود سوال ننمایند و آیا از مردگان به جهت زندگان سوال باید نمود؟»
\par 20 به شریعت و شهادت (توجه نمایید) واگر موافق این کلام سخن نگویند، پس برای ایشان روشنایی نخواهد بود.
\par 21 و با عسرت و گرسنگی در آن خواهند گشت و هنگامی که گرسنه شوند خویشتن را مشوش خواهند ساخت و پادشاه و خدای خود را لعنت کرده، به بالاخواهند نگریست.و به زمین نظر خواهندانداخت و اینک تنگی و تاریکی و ظلمت پریشانی خواهد بود و به تاریکی غلیظ رانده خواهند شد.
\par 22 و به زمین نظر خواهندانداخت و اینک تنگی و تاریکی و ظلمت پریشانی خواهد بود و به تاریکی غلیظ رانده خواهند شد.
 
\chapter{9}

\par 1 لیکن برای او که در تنگی می‌بود، تاریکی نخواهد شد. در زمان پیشین زمین زبولون وزمین نفتالی را ذلیل ساخت، اما در زمان آخر آن را به راه دریا به آن طرف اردن در جلیل امت هامحترم خواهد گردانید.
\par 2 قومی که در تاریکی سالک می‌بودند، نور عظیمی خواهند دید و بر ساکنان زمین سایه موت نور ساطع خواهد شد.
\par 3 تو قوم را بسیار ساخته، شادی ایشان را زیادگردانیدی. به حضور تو شادی خواهند کرد مثل شادمانی وقت درو و مانند کسانی که در تقسیم نمودن غنیمت وجد می‌نمایند.
\par 4 زیرا که یوغ باراو را و عصای گردنش یعنی عصای جفا کننده وی را شکستی چنانکه در روز مدیان کردی.
\par 5 زیرا همه اسلحه مسلحان در غوغا است ورخوت ایشان به خون آغشته است اما برای سوختن و هیزم آتش خواهند بود.
\par 6 زیرا که برای ما ولدی زاییده و پسری به ما بخشیده شد وسلطنت بر دوش او خواهد بود و اسم او عجیب ومشیر و خدای قدیر و پدر سرمدی و سرورسلامتی خوانده خواهد شد.
\par 7 ترقی سلطنت وسلامتی او را بر کرسی داود و بر مملکت وی انتهانخواهد بود تا آن را به انصاف و عدالت از الان تاابدالاباد ثابت و استوار نماید. غیرت یهوه صبایوت این را بجا خواهد آورد.
\par 8 خداوند کلامی نزد یعقوب فرستاد و آن براسرائیل واقع گردید.
\par 9 و تمامی قوم خواهنددانست یعنی افرایم و ساکنان سامره که از غرور وتکبر دل خود می‌گویند.
\par 10 خشتها افتاده است امابا سنگهای تراشیده بنا خواهیم نمود؛ چوبهای افراغ در هم شکست اما سرو آزاد بجای آنهامی گذاریم.
\par 11 بنابراین خداوند دشمنان رصین رابضد او خواهد برافراشت و خصمان او را خواهدبرانگیخت.
\par 12 ارامیان را از مشرق و فلسطینیان رااز مغرب و ایشان اسرائیل را با دهان گشوده خواهند خورد. اما با این همه خشم او برگردانیده نشده و دست او هنوز دراز است.
\par 13 و این قوم بسوی زننده خودشان بازگشت ننموده و یهوه صبایوت را نطلبیده‌اند.
\par 14 بنابراین خداوند سر ودم و نخل و نی را از اسرائیل در یک روز خواهدبرید.
\par 15 مرد پیر و مرد شریف سر است و نبی‌ای که تعلیم دروغ می‌دهد، دم می‌باشد.
\par 16 زیرا که هادیان این قوم ایشان را گمراه می‌کنند و پیروان ایشان بلعیده می‌شوند.
\par 17 از این سبب خداوند از جوانان ایشان مسرور نخواهد شد و بر یتیمان و بیوه‌زنان ایشان ترحم نخواهد نمود. چونکه جمیع ایشان منافق وشریرند و هر دهانی به حماقت متکلم می‌شود بااینهمه غضب او برگردانیده نشده و دست او هنوزدراز است.
\par 18 زیرا که شرارت مثل آتش می‌سوزاند و خار و خس را می‌خورد، و دربوته های جنگل افروخته شده، دود غلیظ پیچان می‌شود.
\par 19 از غضب یهوه صبایوت زمین سوخته شده است و قوم هیزم آتش گشته‌اند و کسی بربرادر خود شفقت ندارد.
\par 20 از جانب راست می‌ربایند و گرسنه می‌مانند و از طرف چپ می‌خورند و سیر نمی شوند و هرکس گوشت بازوی خود را می‌خورد.منسی افرایم را وافرایم منسی را و هر دوی ایشان بضد یهودا متحدمی شوند. با اینهمه غضب او برگردانیده نشده ودست او هنوز دراز است.
\par 21 منسی افرایم را وافرایم منسی را و هر دوی ایشان بضد یهودا متحدمی شوند. با اینهمه غضب او برگردانیده نشده ودست او هنوز دراز است.
 
\chapter{10}

\par 1 وای بر آنانی که احکام غیر عادله راجاری می‌سازند و کاتبانی که ظلم رامرقوم می‌دارند،
\par 2 تا مسکینان را از داوری منحرف سازند و حق فقیران قوم مرا بربایند تا آنکه بیوه‌زنان غارت ایشان بشوند و یتیمان راتاراج نمایند.
\par 3 پس در روز بازخواست در حینی که خرابی از دور می‌آید، چه خواهید کرد وبسوی که برای معاونت خواهید گریخت و جلال خود را کجا خواهید انداخت؟
\par 4 غیر از آنکه زیراسیران خم شوند و زیر کشتگان بیفتند. با اینهمه غضب او برگردانیده نشده و دست او هنوز درازاست.
\par 5 وای بر آشور که عصای غضب من است. وعصایی که در دست ایشان است خشم من می‌باشد.
\par 6 او را بر امت منافق می‌فرستم و نزد قوم مغضوب خود مامور می‌دارم، تا غنیمتی بربایندو غارتی ببرند و ایشان را مثل گل کوچه‌ها پایمال سازند.
\par 7 اما او چنین گمان نمی کند و دلش بدینگونه قیاس نمی نماید، بلکه مراد دلش این است که امت های بسیار را هلاک و منقطع بسازد.
\par 8 زیرا می‌گوید آیا سرداران من جمیع پادشاه نیستند؟
\par 9 آیا کلنو مثل کرکمیش نیست و آیاحمات مثل ارفاد نی، و آیا سامره مانند دمشق نمی باشد؟
\par 10 چنانکه دست من بر ممالک بتهااستیلا یافت و بتهای تراشیده آنها از بتهای اورشلیم و سامره بیشتر بودند.
\par 11 پس آیا به نهجی که به سامره و بتهایش عمل نمودم به اورشلیم و بتهایش چنین عمل نخواهم نمود؟
\par 12 و واقع خواهد شد بعد از آنکه خداوندتمامی کار خود را با کوه صهیون و اورشلیم به انجام رسانیده باشد که من از ثمره دل مغرورپادشاه آشور و از فخر چشمان متکبر وی انتقام خواهم کشید.
\par 13 زیرا می‌گوید: «به قوت دست خود و به حکمت خویش چونکه فهیم هستم این را کردم و حدود قومها را منتقل ساختم و خزاین ایشان را غارت نمودم و مثل جبار سروران ایشان را به زیر انداختم.
\par 14 و دست من دولت قوم‌ها رامثل آشیانه‌ای گرفته است و به طوری که تخمهای متروک را جمع کنند من تمامی زمین راجمع کردم. و کسی نبود که بال را بجنباند یا دهان خود را بگشاید یا جک جک بنماید.»
\par 15 آیا تبر بر کسی‌که به آن می‌شکند فخرخواهد نمود یا اره بر کسی‌که آن را می‌کشدافتخار خواهد کرد، که گویا عصا بلند کننده خودرا بجنباند یا چوب دست آنچه را که چوب نباشدبلند نماید؟
\par 16 بنابراین خداوند یهوه صبایوت برفربهان او لاغری خواهد فرستاد و زیر جلال اوسوختنی مثل سوختن آتش افروخته خواهد شد.
\par 17 و نور اسرائیل نار و قدوس وی شعله خواهدشد، و در یکروز خار و خسش را سوزانیده، خواهد خورد.
\par 18 و شوکت جنگل و بستان او هم روح و هم بدن را تباه خواهد ساخت و مثل گداختن مریض خواهد شد.
\par 19 و بقیه درختان وجنگلش قلیل العدد خواهد بود که طفلی آنها راثبت تواند کرد.
\par 20 و در آن روز واقع خواهد شد که بقیه اسرائیل و ناجیان خاندان یعقوب بار دیگر برزننده خودشان اعتماد نخواهند نمود. بلکه برخداوند که قدوس اسرائیل است به اخلاص اعتماد خواهند نمود.
\par 21 و بقیه‌ای یعنی بقیه یعقوب بسوی خدای قادر مطلق بازگشت خواهند کرد.
\par 22 زیرا هرچند قوم تو اسرائیل مثل ریگ دریا باشند فقط از ایشان بقیتی بازگشت خواهند نمود. هلاکتی که مقدر است به عدالت مجرا خواهد شد
\par 23 زیرا خداوند یهوه صبایوت هلاکت و تقدیری در میان تمام زمین به عمل خواهد آورد.
\par 24 بنابراین خداوند یهوه صبایوت چنین می‌گوید: «ای قوم من که در صهیون ساکنیداز آشور مترسید، اگر‌چه شما را به چوب بزند وعصای خود را مثل مصریان بر شما بلند نماید.
\par 25 زیرا بعد از زمان بسیار کمی غضب تمام خواهد شد و خشم من برای هلاکت ایشان خواهد بود.»
\par 26 و یهوه صبایوت تازیانه‌ای بر وی خواهد برانگیخت چنانکه در کشتار مدیان برصخره غراب. و عصای او بر دریا خواهد بود وآن را بلند خواهد کرد به طوری که بر مصریان کرده بود.
\par 27 و در آن روز واقع خواهد شد که باراو از دوش تو و یوغ او از گردن تو رفع خواهد شدو یوغ از فربهی گسسته خواهد شد.
\par 28 او به عیات رسید و از مجرون گذشت و درمکماش اسباب خود را گذاشت.
\par 29 از معبر عبورکردند و در جبع منزل گزیدند، اهل رامه هراسان شدند و اهل جبعه شاول فرار کردند.
\par 30 ‌ای دخترجلیم به آواز خود فریاد برآور! ای لیشه و‌ای عناتوت فقیر گوش ده!
\par 31 مدمینه فراری شدند وساکنان جیبیم گریختند.
\par 32 همین امروز در نوب توقف می‌کند و دست خود را بر جبل دخترصهیون و کوه اورشلیم دراز می‌سازد.
\par 33 اینک خداوند یهوه صبایوت شاخه‌ها را با خوف قطع خواهد نمود و بلند قدان بریده خواهند شد و مرتفعان پست خواهند گردید،و بوته های جنگل به آهن بریده خواهد شد و لبنان به‌دست جباران خواهد افتاد.
\par 34 و بوته های جنگل به آهن بریده خواهد شد و لبنان به‌دست جباران خواهد افتاد.
 
\chapter{11}

\par 1 و نهالی از تنه یسی بیرون آمده، شاخه‌ای از ریشه هایش خواهدشکفت.
\par 2 و روح خداوند بر او قرار خواهدگرفت، یعنی روح حکمت و فهم و روح مشورت و قوت و روح معرفت و ترس خداوند.
\par 3 وخوشی او در ترس خداوند خواهد بود و موافق رویت چشم خود داوری نخواهد کرد و بر وفق سمع گوشهای خویش تنبیه نخواهد نمود.
\par 4 بلکه مسکینان را به عدالت داوری خواهد کرد و به جهت مظلومان زمین براستی حکم خواهد نمود. و جهان را به عصای دهان خویش زده، شریران رابه نفخه لبهای خود خواهد کشت.
\par 5 و کمربندکمرش عدالت خواهد بود و کمربند میانش امانت.
\par 6 و گرگ با بره سکونت خواهد داشت و پلنگ با بزغاله خواهد خوابید و گوساله و شیر و پرواری با هم، و طفل کوچک آنها را خواهد راند.
\par 7 و گاوبا خرس خواهد چرید و بچه های آنها با هم خواهند خوابید و شیر مثل گاو کاه خواهد خورد.
\par 8 و طفل شیرخواره بر سوراخ مار بازی خواهدکرد و طفل از شیر باز داشته شده دست خود را برخانه افعی خواهد گذاشت.
\par 9 و در تمامی کوه مقدس من ضرر و فسادی نخواهند کرد زیرا که جهان از معرفت خداوند پر خواهد بود مثل آبهایی که دریا را می‌پوشاند.
\par 10 و در آن روز واقع خواهد شد که ریشه یسی به جهت علم قوم‌ها برپا خواهد شد و امت هاآن را خواهند طلبید و سلامتی او با جلال خواهدبود.
\par 11 و در آن روز واقع خواهد گشت که خداوند بار دیگر دست خود را دراز کند تا بقیه قوم خویش را که از آشور و مصر و فتروس وحبش و عیلام و شنعار و حمات و از جزیره های دریا باقی‌مانده باشند باز آورد.
\par 12 و به جهت امت‌ها علمی برافراشته، رانده شدگان اسرائیل را جمع خواهد کرد، وپراکندگان یهودا را از چهار طرف جهان فراهم خواهد آورد.
\par 13 و حسد افرایم رفع خواهد شد ودشمنان یهودا منقطع خواهند گردید. افرایم بریهودا حسد نخواهد برد و یهودا افرایم را دشمنی نخواهد نمود.
\par 14 و به‌جانب مغرب بر دوش فلسطینیان پریده، بنی مشرق را با هم غارت خواهند نمود. و دست خود را بر ادوم و موآب دراز کرده، بنی عمون ایشان را اطاعت خواهندکرد.
\par 15 و خداوند زبانه دریای مصر را تباه ساخته، دست خود را با باد سوزان بر نهر درازخواهد کرد، و آن را با هفت نهرش خواهد زد ومردم را با کفش به آن عبور خواهد داد.و به جهت بقیه قوم او که از آشور باقی‌مانده باشند شاه راهی خواهد بود. چنانکه به جهت اسرائیل در روز بر‌آمدن ایشان از زمین مصربود.
\par 16 و به جهت بقیه قوم او که از آشور باقی‌مانده باشند شاه راهی خواهد بود. چنانکه به جهت اسرائیل در روز بر‌آمدن ایشان از زمین مصربود. 
 
\chapter{12}

\par 1 و در آن روز خواهی گفت که «ای خداوند تو را حمد می‌گویم زیرا به من غضبناک بودی اما غضبت برگردانیده شده، مراتسلی می‌دهی.
\par 2 اینک خدا نجات من است بر اوتوکل نموده، نخواهم ترسید. زیرا یاه یهوه قوت وتسبیح من است و نجات من گردیده است.»
\par 3 بنابراین با شادمانی از چشمه های نجات آب خواهید کشید.
\par 4 و در آن روز خواهید گفت: «خداوند راحمد گویید و نام او را بخوانید و اعمال او را درمیان قوم‌ها اعلام کنید و ذکر نمایید که اسم اومتعال می‌باشد.
\par 5 برای خداوند بسرایید زیراکارهای عظیم کرده است و این در تمامی زمین معروف است.‌ای ساکنه صهیون صدا رابرافراشته، بسرای زیرا قدوس اسرائیل در میان توعظیم است.»
\par 6 ‌ای ساکنه صهیون صدا رابرافراشته، بسرای زیرا قدوس اسرائیل در میان توعظیم است.»
 
\chapter{13}

\par 1 وحی درباره بابل که اشعیا ابن آموص آن را دید.
\par 2 علمی بر کوه خشک برپا کنید و آواز به ایشان بلند نمایید، با دست اشاره کنید تا به درهای نجبا داخل شوند.
\par 3 من مقدسان خود رامامور داشتم و شجاعان خویش یعنی آنانی را که در کبریای من وجد می‌نمایند به جهت غضبم دعوت نمودم.
\par 4 آواز گروهی در کوه‌ها مثل آوازخلق کثیر. آواز غوغای ممالک امت‌ها که جمع شده باشند. یهوه صبایوت لشکر را برای جنگ سان می‌بیند.
\par 5 ایشان از زمین بعید و از کرانه های آسمان می‌آیند. یعنی خداوند با اسلحه غضب خود تا تمامی جهان را ویران کند.
\par 6 ولوله کنیدزیرا که روز خداوند نزدیک است، مثل هلاکتی ازجانب قادر مطلق می‌آید.
\par 7 از این جهت همه دستها سست می‌شود و دلهای همه مردم گداخته می‌گردد.
\par 8 و ایشان متحیر شده، المها و دردهای زه بر ایشان عارض می‌شود، مثل زنی که می‌زایددرد می‌کشند. بر یکدیگر نظر حیرت می‌اندازند ورویهای ایشان رویهای شعله‌ور می‌باشد.
\par 9 اینک روز خداوند با غضب و شدت خشم و ستمکیشی می‌آید، تا جهان را ویران سازد و گناهکاران را ازمیانش هلاک نماید.
\par 10 زیرا که ستارگان آسمان وبرجهایش روشنایی خود را نخواهند داد. وآفتاب در وقت طلوع خود تاریک خواهد شد وماه روشنایی خود را نخواهد تابانید.
\par 11 و من ربع مسکون را به‌سبب گناه و شریران را به‌سبب عصیان ایشان سزا خواهم داد، و غرور متکبران راتباه خواهم ساخت و تکبر جباران را به زیرخواهم‌انداخت.
\par 12 و مردم را از زر خالص وانسان را از طلای اوفیر کمیابتر خواهم گردانید.
\par 13 بنابراین آسمان را متزلزل خواهم ساخت وزمین از جای خود متحرک خواهد شد. در حین غضب یهوه صبایوت و در روز شدت خشم او.
\par 14 و مثل آهوی رانده شده و مانند گله‌ای که کسی آن را جمع نکند خواهند بود. و هرکس به سوی قوم خود توجه خواهد نمود و هر شخص به زمین خویش فرار خواهد کرد.
\par 15 و هرکه یافت شود با نیزه زده خواهد شد و هرکه گرفته شود با شمشیر خواهد افتاد.
\par 16 اطفال ایشان نیز در نظرایشان به زمین انداخته شوند و خانه های ایشان غارت شود و زنان ایشان بی‌عصمت گردند.
\par 17 اینک من مادیان را بر ایشان خواهم برانگیخت که نقره را به حساب نمی آورند و طلا را دوست نمی دارند.
\par 18 و کمانهای ایشان جوانان را خردخواهد کرد. و بر ثمره رحم ترحم نخواهند نمودو چشمان ایشان بر اطفال شفقت نخواهد کرد.
\par 19 و بابل که جلال ممالک و زینت فخر کلدانیان است، مثل واژگون ساختن خدا سدوم و عموره راخواهد شد.
\par 20 و تا به ابد آباد نخواهد شد و نسلابعد نسل مسکون نخواهد گردید. و اعراب درآنجا خیمه نخواهند زد و شبانان گله‌ها را در آنجانخواهند خوابانید.
\par 21 بلکه وحوش صحرا درآنجا خواهند خوابید و خانه های ایشان از بومهاپر خواهد شد. شترمرغ در آنجا ساکن خواهد شدو غولان در آنجا رقص خواهند کرد،و شغالهادر قصرهای ایشان و گرگها در کوشکهای خوش نما صدا خواهند زد و زمانش نزدیک است که برسد و روزهایش طول نخواهد کشید.
\par 22 و شغالهادر قصرهای ایشان و گرگها در کوشکهای خوش نما صدا خواهند زد و زمانش نزدیک است که برسد و روزهایش طول نخواهد کشید.
 
\chapter{14}

\par 1 زیرا خداوند بر یعقوب ترحم فرموده، اسرائیل را بار دیگر خواهد برگزید وایشان را در زمینشان آرامی خواهد داد. و غربا باایشان ملحق شده، با خاندان یعقوب ملصق خواهند گردید.
\par 2 و قوم‌ها ایشان را برداشته، به مکان خودشان خواهند‌آورد. و خاندان اسرائیل ایشان را در زمین خداوند برای بندگی و کنیزی، مملوک خود خواهند ساخت. و اسیرکنندگان خود را اسیر کرده، بر ستمکاران خویش حکمرانی خواهند نمود.
\par 3 و در روزی که خداوند تو را از الم واضطرابت و بندگی سخت که بر تو می‌نهادندخلاصی بخشد واقع خواهد شد،
\par 4 که این مثل رابر پادشاه بابل زده، خواهی گفت: چگونه آن ستمکار تمام شد و آن جور پیشه چگونه فانی گردید.
\par 5 خداوند عصای شریران و چوگان حاکمان را شکست.
\par 6 آنکه قوم‌ها را به خشم باصدمه متوالی می‌زد و بر امت‌ها به غضب با جفای بیحد حکمرانی می‌نمود،
\par 7 تمامی زمین آرام شده و ساکت گردیده‌اند و به آواز بلند ترنم می‌نمایند.
\par 8 صنوبرها نیز و سروهای آزاد لبنان درباره توشادمان شده، می‌گویند: «از زمانی که توخوابیده‌ای قطع کننده‌ای بر ما برنیامده است.»
\par 9 هاویه از زیر برای تو متحرک است تا چون بیایی تو را استقبال نماید، و مردگان یعنی جمیع بزرگان زمین را برای تو بیدار می‌سازد. و جمیع پادشاهان امت‌ها را از کرسیهای ایشان برمی دارد.
\par 10 جمیع اینها تو را خطاب کرده، می‌گویند: «آیاتو نیز مثل ما ضعیف شده‌ای و مانند ماگردیده‌ای.»
\par 11 جلال تو و صدای بربطهای تو به هاویه فرود شده است. کرمها زیر تو گسترانیده شده و مورها تو را می‌پوشانند.
\par 12 ‌ای زهره دخترصبح چگونه از آسمان افتاده‌ای؟ ای که امت‌ها راذلیل می‌ساختی چگونه به زمین افکنده شده‌ای؟
\par 13 و تو در دل خود می‌گفتی: «به آسمان صعودنموده، کرسی خود را بالای ستارگان خدا خواهم افراشت. و بر کوه اجتماع در اطراف شمال جلوس خواهم نمود.
\par 14 بالای بلندیهای ابرها صعود کرده، مثل حضرت اعلی خواهم شد.»
\par 15 لکن به هاویه به اسفلهای حفره فرود خواهی شد.
\par 16 آنانی که تو را بینند بر تو چشم دوخته و درتو تامل نموده، خواهند گفت: «آیا این آن مرداست که جهان را متزلزل و ممالک را مرتعش می‌ساخت؟
\par 17 که ربع مسکون را ویران می‌نمودو شهرهایش را منهدم می‌ساخت و اسیران خودرا به خانه های ایشان رها نمی کرد؟»
\par 18 همه پادشاهان امت‌ها جمیع هر یک در خانه خود با جلال می‌خوابند.
\par 19 اما تو از قبر خودبیرون افکنده می‌شوی و مثل شاخه مکروه ومانند لباس کشتگانی که با شمشیر زده شده باشند، که به سنگهای حفره فرو می‌روند و مثل لاشه پایمال شده.
\par 20 با ایشان در دفن متحدنخواهی بود چونکه زمین خود را ویران کرده، قوم خویش را کشته‌ای. ذریت شریران تا به ابد مذکورنخواهند شد.
\par 21 برای پسرانش به‌سبب گناه پدران ایشان قتل را مهیا سازید، تا ایشان برنخیزند و در زمین تصرف ننمایند و روی ربع مسکون را از شهرها پر نسازند.
\par 22 و یهوه صبایوت می‌گوید: «من به ضدایشان خواهم برخاست.» و خداوند می‌گوید: «اسم و بقیه را و نسل و ذریت را از بابل منقطع خواهم ساخت.
\par 23 و آن را نصیب خارپشتها و خلابهای آب خواهم گردانید و آن رابا جاروب هلاکت خواهم رفت.» یهوه صبایوت می‌گوید.
\par 24 یهوه صبایوت قسم خورده، می‌گوید: «یقین به طوری که قصد نموده‌ام همچنان واقع خواهد شد. و به نهجی که تقدیر کرده‌ام همچنان بجا آورده خواهد گشت.
\par 25 و آشور را در زمین خودم خواهم شکست و او را بر کوههای خویش پایمال خواهم کرد. و یوغ او از ایشان رفع شده، بار وی از گردن ایشان برداشته خواهد شد.»
\par 26 تقدیری که بر تمامی زمین مقدر گشته، این است. و دستی که بر جمیع امت‌ها دراز شده، همین است.
\par 27 زیرا که یهوه صبایوت تقدیرنموده است، پس کیست که آن را باطل گرداند؟ ودست اوست که دراز شده است پس کیست که آن را برگرداند؟
\par 28 در سالی که آحاز پادشاه مرد این وحی نازل شد:
\par 29 ‌ای جمیع فلسطین شادی مکن از اینکه عصایی که تو را می‌زد شکسته شده است. زیرا که از ریشه مار افعی بیرون می‌آید و نتیجه او اژدهای آتشین پرنده خواهد بود.
\par 30 و نخست زادگان مسکینان خواهند چرید و فقیران در اطمینان خواهند خوابید. و ریشه تو را با قحطی خواهم کشت و باقی ماندگان تو مقتول خواهند شد.
\par 31 ‌ای دروازه ولوله نما! و‌ای شهر فریاد برآور! ای تمامی فلسطین تو گداخته خواهی شد. زیراکه از طرف شمال دود می‌آید و از صفوف وی کسی دور نخواهد افتاد.پس به رسولان امت هاچه جواب داده شود: «اینکه خداوند صهیون رابنیاد نهاده است و مسکینان قوم وی در آن پناه خواهند برد.»
\par 32 پس به رسولان امت هاچه جواب داده شود: «اینکه خداوند صهیون رابنیاد نهاده است و مسکینان قوم وی در آن پناه خواهند برد.»
 
\chapter{15}

\par 1 وحی درباره موآب: زیرا که در شبی عار موآب خراب وهلاک شده است زیرا در شبی قیر موآب خراب وهلاک شده است.
\par 2 به بتکده و دیبون به مکان های بلند به جهت گریستن برآمده‌اند. موآب برای نبوو میدبا ولوله می‌کند. بر سر هریکی از ایشان گری است و ریشهای همه تراشیده شده است.
\par 3 درکوچه های خود کمر خود را به پلاس می‌بندند وبر پشت بامها و در چهارسوهای خود هرکس ولوله می‌نماید و اشکها می‌ریزد.
\par 4 و حشبون والعاله فریاد برمی آورند. آواز ایشان تا یاهص مسموع می‌شود. بنابراین مسلحان موآب ناله می‌کنند و جان ایشان در ایشان می‌لرزد.
\par 5 دل من به جهت موآب فریاد برمی آورد. فراریانش تا به صوغر و عجلت شلشیا نعره می‌زنند زیرا که ایشان به فراز لوحیت با گریه برمی آیند. زیرا که از راه حورونایم صدای هلاکت برمی آورند.
\par 6 زیرا که آبهای نمریم خراب شده، چونکه علف خشکیده و گیاه تلف شده و هیچ‌چیز سبز باقی نمانده است.
\par 7 بنابراین دولتی را که تحصیل نموده‌اند و اندوخته های خود را بر وادی بیدها می‌برند.
\par 8 زیرا که فریادایشان حدود موآب را احاطه نموده و ولوله ایشان تا اجلایم و ولوله ایشان تا بئر ایلیم رسیده است.چونکه آبهای دیمون از خون پر شده زانرو که بر دیمون (بلایای ) زیاد خواهم آوردیعنی شیری را بر فراریان موآب و بر بقیه زمینش (خواهم گماشت ).
\par 9 چونکه آبهای دیمون از خون پر شده زانرو که بر دیمون (بلایای ) زیاد خواهم آوردیعنی شیری را بر فراریان موآب و بر بقیه زمینش (خواهم گماشت ).
 
\chapter{16}

\par 1 بره‌ها را که خراج حاکم زمین است ازسالع بسوی بیابان به کوه دختر صهیون بفرستید.
\par 2 و دختران موآب مثل مرغان آواره ومانند آشیانه ترک شده نزد معبرهای ارنون خواهند شد.
\par 3 مشورت بدهید و انصاف را بجاآورید، و سایه خود را در وقت ظهر مثل شب بگردان. رانده شدگان را پنهان کن و فراریان راتسلیم منما.
\par 4 ‌ای موآب بگذار که رانده شدگان من نزد تو ماوا گزینند. و برای ایشان از روی تاراج کننده پناه گاه باش. زیرا ظالم نابود می‌شود وتاراج کننده تمام می‌گردد و ستمکار از زمین تلف خواهد شد.
\par 5 و کرسی به رحمت استوار خواهدگشت و کسی به راستی بر آن در خیمه داودخواهد نشست که داوری کند و انصاف را بطلبد وبه جهت عدالت تعجیل نماید.
\par 6 غرور موآب و بسیاری تکبر و خیلاء و کبر وخشم او را شنیدیم و فخر او باطل است.
\par 7 بدین سبب موآب به جهت موآب ولوله می‌کند وتمامی ایشان ولوله می‌نمایند. به جهت بنیادهای قیر حارست ناله می‌کنید زیرا که بالکل مضروب می‌شود.
\par 8 زیرا که مزرعه های حشبون و موهای سبمه پژمرده شد و سروران امت‌ها تاکهایش راشکستند. آنها تا به یعزیر رسیده بود و در بیابان پراکنده می‌شد و شاخه هایش منتشر شده، از دریامی گذشت.
\par 9 بنابراین برای مو سبمه به گریه یعزیرخواهم گریست. ای حشبون و العاله شما را بااشکهای خود سیراب خواهم ساخت زیرا که برمیوه‌ها و انگورهایت گلبانگ افتاده است. 
\par 10 شادی و ابتهاج از بستانها برداشته شد و در تاکستانها ترنم و آواز شادمانی نخواهد بود وکسی شراب را در چرخشتها پایمال نمی کند. صدای شادمانی را خاموش گردانیدم.
\par 11 لهذااحشای من مثل بربط به جهت موآب صدا می‌زندو بطن من برای قیر حارس.
\par 12 و هنگامی که موآب در مکان بلند خود حاضر شده، خویشتن را خسته کند و به مکان مقدس خود برای دعابیاید کامیاب نخواهد شد.
\par 13 این است کلامی که خداوند درباره موآب از زمان قدیم گفته است.اما الان خداوند تکلم نموده، می‌گوید که بعد از سه سال مثل سالهای مزدور جلال موآب با تمامی جماعت کثیر اومحقر خواهد شد و بقیه آن بسیار کم و بی‌قوت خواهند گردید.
\par 14 اما الان خداوند تکلم نموده، می‌گوید که بعد از سه سال مثل سالهای مزدور جلال موآب با تمامی جماعت کثیر اومحقر خواهد شد و بقیه آن بسیار کم و بی‌قوت خواهند گردید.
 
\chapter{17}

\par 1 وحی درباره دمشق: اینک دمشق از میان شهرها برداشته می‌شود و توده خراب خواهد گردید.
\par 2 شهرهای عروعیر متروک می‌شود و به جهت خوابیدن گله‌ها خواهد بود و کسی آنها را نخواهد ترسانید.
\par 3 و حصار از افرایم تلف خواهد شد و سلطنت ازدمشق و از بقیه ارام. و مثل جلال بنی‌اسرائیل خواهند بود زیرا که یهوه صبایوت چنین می‌گوید.
\par 4 و در آن روز جلال یعقوب ضعیف می‌شود و فربهی جسدش به لاغری تبدیل می‌گردد.
\par 5 و چنان خواهد بود که دروگران زرع راجمع کنند و دستهای ایشان سنبله‌ها را درو کند. و خواهد بود مثل وقتی که در وادی رفایم سنبله‌ها را بچینند.
\par 6 و خوشه های چند در آن باقی ماند و مثل وقتی که زیتون را بتکانند که دو یاسه دانه بر سر شاخه بلند و چهار یا پنج دانه برشاخچه های بارور آن باقی ماند. یهوه خدای اسرائیل چنین می‌گوید.
\par 7 در آن روز انسان بسوی آفریننده خود نظرخواهد کرد و چشمانش بسوی قدوس اسرائیل خواهد نگریست.
\par 8 و بسوی مذبح هایی که به‌دستهای خود ساخته است نظر نخواهد کرد و به آنچه با انگشتهای خویش بنا نموده یعنی اشیریم و بتهای آفتاب نخواهد نگریست.
\par 9 در آن روز شهرهای حصینش مثل خرابه هایی که در جنگل یا بر کوه بلند است خواهد شد که آنها را از حضور بنی‌اسرائیل واگذاشتند و ویران خواهد شد.
\par 10 چونکه خدای نجات خود را فراموش کردی و صخره قوت خویش را به یاد نیاوردی بنابراین نهالهای دلپذیرغرس خواهی نمود و قلمه های غریب را خواهی کاشت.
\par 11 در روزی که غرس می‌نمایی آن را نموخواهی داد و در صبح مزروع خود را به شکوفه خواهی آورد اما محصولش در روز آفت مهلک وحزن علاج ناپذیر بر باد خواهد رفت.
\par 12 وای بر شورش قوم های بسیار که مثل شورش دریا شورش می‌نمایند و خروش طوایفی که مثل خروش آبهای زورآور خروش می‌کنند.
\par 13 طوایف مثل خروش آبهای بسیارمی خروشند اما ایشان را عتاب خواهد کرد و به‌جای دور خواهند گریخت و مثل کاه کوهها دربرابر باد رانده خواهند شد و مثل غبار در برابرگردباد.در وقت شام اینک خوف است و قبل از صبح نابود می‌شوند. نصیب تاراج کنندگان ما و حصه غارت نمایندگان ما همین است.
\par 14 در وقت شام اینک خوف است و قبل از صبح نابود می‌شوند. نصیب تاراج کنندگان ما و حصه غارت نمایندگان ما همین است.
 
\chapter{18}

\par 1 وای بر زمینی که در آن آواز بالها است که به آن طرف نهرهای کوش می‌باشد.
\par 2 و ایلچیان به دریا و در کشتیهای بردی بر روی آبها می‌فرستد و می‌گوید: ای رسولان تیزروبروید نزد امت بلند قد و براق، نزد قومی که ازابتدایش تا کنون مهیب بوده‌اند یعنی امت زورآورو پایمال کننده که نهرها زمین ایشان را تقسیم می‌کند.
\par 3 ‌ای تمامی ساکنان ربع مسکون و سکنه جهان، چون علمی بر کوهها بلند گردد بنگرید وچون کرنا نواخته شود بشنوید.
\par 4 زیرا خداوند به من چنین گفته است که من خواهم آرامید و ازمکان خود نظر خواهم نمود. مثل گرمای صاف برنباتات و مثل ابر شبنم دار در حرارت حصاد.
\par 5 زیرا قبل از حصاد وقتی که شکوفه تمام شود وگل به انگور رسیده، مبدل گردد او شاخه‌ها را بااره‌ها خواهد برید و نهالها را بریده دور خواهدافکند.
\par 6 و همه برای مرغان شکاری کوهها ووحوش زمین واگذاشته خواهد شد. و مرغان شکاری تابستان را بر آنها بسر خواهند برد وجمیع وحوش زمین زمستان را بر آنها خواهندگذرانید.و در آن زمان هدیه‌ای برای یهوه صبایوت از قوم بلند قد و براق و از قومی که ازابتدایش تا کنون مهیب است و از امتی زورآور وپایمال کننده که نهرها زمین ایشان را تقسیم می‌کند به مکان اسم یهوه صبایوت یعنی به کوه صهیون آورده خواهد شد.
\par 7 و در آن زمان هدیه‌ای برای یهوه صبایوت از قوم بلند قد و براق و از قومی که ازابتدایش تا کنون مهیب است و از امتی زورآور وپایمال کننده که نهرها زمین ایشان را تقسیم می‌کند به مکان اسم یهوه صبایوت یعنی به کوه صهیون آورده خواهد شد.
 
\chapter{19}

\par 1 وحی درباره مصر: اینک خداوند بر ابر تیزرو سوار شده، به مصر می‌آید و بتهای مصر از حضور وی خواهدلرزید و دلهای مصریان در اندرون ایشان گداخته خواهد شد.
\par 2 و مصریان را بر مصریان خواهم برانگیخت. برادر با برادر خود و همسایه باهمسایه خویش و شهر با شهر و کشور با کشورجنگ خواهند نمود.
\par 3 و روح مصر در اندرونش افسرده شده، مشورتش را باطل خواهم گردانید وایشان از بتها و فالگیران و صاحبان اجنه وجادوگران سوال خواهند نمود.
\par 4 و مصریان را به‌دست آقای ستم کیش تسلیم خواهم نمود وپادشاه زورآور بر ایشان حکمرانی خواهد کرد. خداوند یهوه صبایوت چنین می‌گوید.
\par 5 و آب از دریا (نیل ) کم شده، نهر خراب وخشک خواهد گردید.
\par 6 و نهرها متعفن شده، جویهای ماصور کم شده می‌خشکد و نی و بوریاپژمرده خواهد شد.
\par 7 و مرغزاری که بر کنار نیل وبر دهنه نیل است و همه مزرعه های نیل خشک ورانده شده و نابود خواهد گردید.
\par 8 و ماهی گیران ماتم می‌گیرند و همه آنانی که قلاب به نیل اندازندزاری می‌کنند و آنانی که دام بر روی آب گسترانند افسرده خواهند شد.
\par 9 و عاملان کتان شانه زده و بافندگان پارچه سفید خجل خواهندشد.
\par 10 و ارکان او ساییده و جمیع مزدوران رنجیده دل خواهند شد.
\par 11 سروران صوعن بالکل احمق می‌شوند ومشورت مشیران دانشمند فرعون وحشی می گردد. پس چگونه به فرعون می‌گویید که من پسر حکما و پسر پادشاهان قدیم می‌باشم.
\par 12 پس حکیمان تو کجایند تا ایشان تو را اطلاع دهند وبدانند که یهوه صبایوت درباره مصر چه تقدیرنموده است.
\par 13 سروران صوعن ابله شده وسروران نوف فریب خورده‌اند و آنانی که سنگ زاویه اسباط مصر هستند آن را گمراه کرده‌اند.
\par 14 و خداوند روح خیرگی در وسط آن آمیخته است که ایشان مصریان را در همه کارهای ایشان گمراه کرده‌اند مثل مستان که در قی خود افتان وخیزان راه می‌روند.
\par 15 و مصریان را کاری نخواهد ماند که سر یا دم نخل یا بوریا بکند.
\par 16 در آن روز اهل مصر مثل زنان می‌باشند و ازحرکت دست یهوه صبایوت که آن را بر مصر به حرکت می‌آورد لرزان و هراسان خواهند شد.
\par 17 و زمین یهودا باعث خوف مصر خواهد شد که هرکه ذکر آن را بشنود خواهد ترسید به‌سبب تقدیری که یهوه صبایوت بر آن مقدر نموده است.
\par 18 در آن روز پنج شهر در زمین مصر به زبان کنعان متکلم شده، برای یهوه صبایوت قسم خواهند خورد و یکی شهر هلاکت نامیده خواهدشد.
\par 19 در آن روز مذبحی برای خداوند در میان زمین مصر و ستونی نزد حدودش برای خداوندخواهد بود.
\par 20 و آن آیتی و شهادتی برای یهوه صبایوت در زمین مصر خواهد بود. زیرا که نزدخداوند به‌سبب جفاکنندگان خویش استغاثه خواهد نمود و او نجات‌دهنده و حمایت کننده‌ای برای ایشان خواهد فرستاد و ایشان راخواهد رهانید.
\par 21 و خداوند بر مصریان معروف خواهد شد و در آن روز مصریان خداوند راخواهند شناخت و با ذبایح و هدایا او را عبادت خواهند کرد و برای خداوند نذر کرده، آن را وفاخواهند نمود.
\par 22 و خداوند مصریان را خواهدزد و به زدن شفا خواهد داد زیرا چون بسوی خداوند بازگشت نمایند ایشان را اجابت نموده، شفا خواهد داد.
\par 23 در آن روز شاهراهی از مصر به آشورخواهد بود و آشوریان به مصر و مصریان به آشورخواهند رفت و مصریان با آشوریان عبادت خواهند نمود.
\par 24 در آن روز اسرائیل سوم مصر وآشور خواهد شد و آنها در میان جهان برکت خواهند بود.زیرا که یهوه صبایوت آنها رابرکت داده خواهد گفت قوم من مصر و صنعت دست من آشور و میراث من اسرائیل مبارک باشند.
\par 25 زیرا که یهوه صبایوت آنها رابرکت داده خواهد گفت قوم من مصر و صنعت دست من آشور و میراث من اسرائیل مبارک باشند.
 
\chapter{20}

\par 1 در سالی که ترتان به اشدود آمد هنگامی که سرجون پادشاه آشور او رافرستاد، پس با اشدود جنگ کرده، آن را گرفت.
\par 2 در آن وقت خداوند به واسطه اشعیا ابن آموص تکلم نموده، گفت: «برو و پلاس را از کمر خودبگشا و نعلین را از پای خود بیرون کن.» و او چنین کرده، عریان و پا برهنه راه می‌رفت.
\par 3 و خداوند گفت: «چنانکه بنده من اشعیا سه سال عریان و پا برهنه راه رفته است تا آیتی وعلامتی درباره مصر و کوش باشد،
\par 4 بهمان طورپادشاه آشور اسیران مصر و جلاء وطنان کوش رااز جوانان وپیران عریان و پابرهنه و مکشوف سرین خواهد برد تا رسوایی مصر باشد.
\par 5 و ایشان به‌سبب کوش که ملجای ایشان است و مصر که فخر ایشان باشد مضطرب و خجل خواهند شد.و ساکنان این ساحل در آن روز خواهند گفت: اینک ملجای ما که برای اعانت به آن فرار کردیم تااز دست پادشاه آشور نجات یابیم چنین شده است، پس ما چگونه نجات خواهیم یافت؟»
\par 6 و ساکنان این ساحل در آن روز خواهند گفت: اینک ملجای ما که برای اعانت به آن فرار کردیم تااز دست پادشاه آشور نجات یابیم چنین شده است، پس ما چگونه نجات خواهیم یافت؟»
 
\chapter{21}

\par 1 وحی درباره بیابان بحر: چنانکه گردباد در جنوب می‌آید، این نیز از بیابان از زمین هولناک می‌آید.
\par 2 رویای سخت برای من منکشف شده است، خیانت پیشه خیانت می‌کند و تاراج کننده تاراج می‌نماید. ای عیلام برآی و‌ای مدیان محاصره نما. تمام ناله آن را ساکت گردانیدم.
\par 3 از این جهت کمر من از شدت درد پر شده است و درد زه مثل درد زنی که می‌زاید مرادرگرفته است. پیچ و تاب می‌خورم که نمی توانم بشنوم، مدهوش می‌شوم که نمی توانم ببینم.
\par 4 دل من می‌طپید و هیبت مرا ترسانید. او شب لذت مرابرایم به خوف مبدل ساخته است.
\par 5 سفره را مهیاساخته و فرش را گسترانیده به اکل و شرب مشغول می‌باشند. ای سروران برخیزید و سپرهارا روغن بمالید.
\par 6 زیرا خداوند به من چنین گفته است: «برو ودیده بان را قرار بده تا آنچه را که بیند اعلام نماید.
\par 7 و چون فوج سواران جفت جفت و فوج الاغان وفوج شتران را بیند آنگاه به دقت تمام توجه بنماید.»
\par 8 پس او مثل شیر صدا زد که «ای آقا من دائم در روز بر محرس ایستاده‌ام و تمامی شب بردیده بانگاه خود برقرار می‌باشم.
\par 9 و اینک فوج مردان و سواران جفت جفت می‌آیند و او مزیدکرده، گفت: بابل افتاد افتاده است و تمامی تمثال های تراشیده خدایانش را بر زمین شکسته‌اند.»
\par 10 ‌ای کوفته شده من و‌ای محصول خرمن من آنچه از یهوه صبایوت خدای اسرائیل شنیدم به شما اعلام می‌نمایم.
\par 11 وحی درباره دومه: کسی از سعیر به من ندا می‌کند که «ای دیده بان از شب چه خبر؟ ای دیده بان از شب چه خبر؟»
\par 12 دیده بان می‌گوید که صبح می‌آید و شام نیز. اگر پرسیدن می‌خواهید بپرسید و بازگشت نموده، بیایید.»
\par 13 وحی درباره عرب: ای قافله های ددانیان در جنگل عرب منزل کنید.
\par 14 ‌ای ساکنان زمین تیما تشنگان را به آب استقبال کنید و فراریان را به خوراک ایشان پذیره شوید.
\par 15 زیرا که ایشان از شمشیرها فرارمی کنند. از شمشیر برهنه و کمان زه شده و ازسختی جنگ.
\par 16 زانرو که خداوند به من گفته است بعد ازیکسال موافق سالهای مزدوران تمامی شوکت قیدار تلف خواهد شد.و بقیه شماره تیراندازان و جباران بنی قیدار قلیل خواهد شدچونکه یهوه خدای اسرائیل این را گفته است. 
\par 17 و بقیه شماره تیراندازان و جباران بنی قیدار قلیل خواهد شدچونکه یهوه خدای اسرائیل این را گفته است.
 
\chapter{22}

\par 1 وحی درباره وادی رویا: الان تو را چه شد که کلیه بر بامهابرآمدی؟
\par 2 ‌ای که پر از شورشها هستی و‌ای شهرپرغوغا و‌ای قریه مفتخر. کشتگانت کشته شمشیر نیستند و در جنگ هلاک نشده‌اند.
\par 3 جمیع سرورانت با هم گریختند و بدون تیراندازان اسیر گشتند. همگانی که در تو یافت شدند با هم اسیر گردیدند و به‌جای دور فرارکردند.
\par 4 بنابراین گفتم نظر خود را از من بگردانیدزیرا که با تلخی گریه می‌کنم. برای تسلی من درباره خرابی دختر قومم الحاح مکنید.
\par 5 زیرا خداوند یهوه صبایوت روز آشفتگی وپایمالی و پریشانی‌ای در وادی رویا دارد. دیوارها را منهدم می‌سازند و صدای استغاثه تا به کوهها می‌رسد.
\par 6 و عیلام با افواج مردان و سواران ترکش را برداشته است و قیر سپر را مکشوف نموده است،
\par 7 و وادیهای بهترینت از ارابه‌ها پرشده، سواران پیش دروازه هایت صف آرایی می‌نمایند،
\par 8 و پوشش یهودا برداشته می‌شود ودر آن روز به اسلحه خانه جنگل نگاه خواهیدکرد.
\par 9 و رخنه های شهر داود را که بسیارندخواهید دید و آب برکه تحتانی را جمع خواهیدنمود.
\par 10 و خانه های اورشلیم را خواهید شمرد وخانه‌ها را به جهت حصاربندی دیوارها خراب خواهید نمود.
\par 11 و درمیان دو دیوار حوضی برای آب برکه قدیم خواهید ساخت اما به صانع آن نخواهید نگریست و به آنکه آن را از ایام پیشین ساخته است نگران نخواهید شد.
\par 12 و در آن روزخداوند یهوه صبایوت (شما را) به گریستن و ماتم کردن و کندن مو و پوشیدن پلاس خواهدخواند.
\par 13 و اینک شادمانی و خوشی و کشتن گاوان و ذبح کردن گوسفندان و خوردن گوشت ونوشیدن شراب خواهد بود که بخوریم و بنوشیم زیرا که فردا می‌میریم.
\par 14 و یهوه صبایوت درگوش من اعلام کرده است که این گناه شما تابمیرید هرگز کفاره نخواهد شد. خداوند یهوه صبایوت این را گفته است.
\par 15 خداوند یهوه صبایوت چنین می‌گوید: «برو و نزد این خزانه‌دار یعنی شبنا که ناظر خانه است داخل شو و به او بگو:
\par 16 تو را در اینجا چه‌کار است و در اینجا که را داری که در اینجا قبری برای خود کنده‌ای؟ ای کسی‌که قبر خود را درمکان بلند می‌کنی و مسکنی برای خویشتن درصخره می‌تراشی.»
\par 17 اینک‌ای مرد، خداوند البته تو را دورخواهد انداخت و البته تو را خواهد پوشانید.
\par 18 والبته تو را مثل گوی سخت خواهد پیچید و به زمین وسیع تو را خواهد افکند و در آنجا خواهی مرد و در آنجا ارابه های شوکت تو رسوایی خانه آقایت خواهد شد.
\par 19 و تو را از منصبت خواهم راند و از مکانت به زیر افکنده خواهی شد.
\par 20 و در آن روز واقع خواهد شد که بنده خویش الیاقیم بن حلقیا را دعوت خواهم نمود.
\par 21 و او را به‌جامه تو ملبس ساخته به کمربندت محکم خواهم ساخت و اقتدار تو را به‌دست اوخواهم داد و او ساکنان اورشلیم و خاندان یهودارا پدر خواهد بود.
\par 22 و کلید خانه داود را بر دوش وی خواهم نهاد و چون بگشاید احدی نخواهدبست و چون ببندد، احدی نخواهد گشاد.
\par 23 و او را در جای محکم مثل میخ خواهم دوخت و برای خاندان پدر خود کرسی جلال خواهد بود.
\par 24 وتمامی جلال خاندان پدرش را از اولاد و احفاد وهمه ظروف کوچک را از ظروف کاسه‌ها تا ظروف تنگها بر او خواهند آویخت.و یهوه صبایوت می‌گوید که در آن روز آن میخی که در مکان محکم دوخته شده متحرک خواهد گردید و قطع شده، خواهد افتاد و باری که بر آن است، تلف خواهد شد زیرا خداوند این راگفته است.
\par 25 و یهوه صبایوت می‌گوید که در آن روز آن میخی که در مکان محکم دوخته شده متحرک خواهد گردید و قطع شده، خواهد افتاد و باری که بر آن است، تلف خواهد شد زیرا خداوند این راگفته است.
 
\chapter{23}

\par 1 وحی درباره صور ای کشتیهای ترشیش ولوله نمایید زیراکه بحدی خراب شده است که نه خانه‌ای و نه مدخلی باقی‌مانده. از زمین کتیم خبر به ایشان رسیده است.
\par 2 ‌ای ساکنان ساحل که تاجران صیدون که از دریا عبور می‌کنند تو را پرساخته‌اند آرام گیرید.
\par 3 و دخل او از محصول شیحور و حصاد نیل بر آبهای بسیار می‌بود پس اوتجارت گاه امت‌ها شده است.
\par 4 ‌ای صیدون خجل شو زیرا که دریا یعنی قلعه دریا متکلم شده، می‌گوید درد زه نکشیده‌ام و نزاییده‌ام وجوانان را نپرورده‌ام و دوشیزگان را تربیت نکرده‌ام.
\par 5 چون این خبر به مصر برسد از اخبارصور بسیار دردناک خواهند شد.
\par 6 ‌ای ساکنان ساحل دریا به ترشیش بگذرید وولوله نمایید.
\par 7 آیا این شهر مفتخر شما است که قدیمی و از ایام سلف بوده است و پایهایش او رابه‌جای دور برده، تا در آنجا ماوا گزیند؟
\par 8 کیست که این قصد را درباره صور آن شهر تاج بخش که تجار وی سروران و بازرگانان او شرفای جهان بوده‌اند نموده است.
\par 9 یهوه صبایوت این قصد رانموده است تا تکبر تمامی جلال را خوار سازد وجمیع شرفای جهان را محقر نماید.
\par 10 ‌ای دختر ترشیش از زمین خود مثل نیل بگذر زیرا که دیگر هیچ بند برای تو نیست.
\par 11 اودست خود را بر دریا دراز کرده، مملکتها رامتحرک ساخته است. خداوند درباره کنعان امرفرموده است تا قلعه هایش را خراب نمایند.
\par 12 وگفته است: ای دوشیزه ستم رسیده و‌ای دخترصیدون دیگر مفتخر نخواهی شد. برخاسته، به کتیم بگذر اما در آنجا نیز راحت برای تو نخواهدبود.
\par 13 اینک زمین کلدانیان که قومی نبودند وآشور آن را به جهت صحرانشینان بنیاد نهاد. ایشان منجنیقهای خود را افراشته، قصرهای آن را منهدم و آن را به خرابی مبدل خواهند ساخت.
\par 14 ‌ای کشتیهای ترشیش ولوله نمایید زیرا که قلعه شما خراب شده است.
\par 15 و در آن روز واقع خواهد شد که صور، هفتاد سال مثل ایام یک پادشاه فراموش خواهدشد و بعد از انقضای هفتاد سال برای صور مثل سرود زانیه خواهد بود.
\par 16 ‌ای زانیه فراموش شده بربط را گرفته، در شهر گردش نما. خوش بنواز وسرودهای بسیار بخوان تا به یاد آورده شوی.
\par 17 وبعد از انقضای هفتاد سال واقع می‌شود که خداوند از صور تفقد خواهد نمود و به اجرت خویش برگشته با جمیع ممالک جهان که بر روی زمین است زنا خواهد نمود.و تجارت واجرت آن برای خداوند وقف شده ذخیره و اندوخته نخواهد شد بلکه تجارتش برای مقربان درگاه خداوند خواهد بود تا به سیری بخورند ولباس فاخر بپوشند.
\par 18 و تجارت واجرت آن برای خداوند وقف شده ذخیره و اندوخته نخواهد شد بلکه تجارتش برای مقربان درگاه خداوند خواهد بود تا به سیری بخورند ولباس فاخر بپوشند.
 
\chapter{24}

\par 1 اینک خداوند زمین را خالی و ویران می کند، و آن را واژگون ساخته، ساکنانش را پراکنده می‌سازد.
\par 2 و مثل قوم، مثل کاهن و مثل بنده، مثل آقایش و مثل کنیز، مثل خاتونش و مثل مشتری، مثل فروشنده و مثل قرض دهنده، مثل قرض گیرنده و مثل سودخوار، مثل سود دهنده خواهد بود.
\par 3 و زمین بالکل خالی و بالکل غارت خواهد شد زیرا خداوند این سخن را گفته است.
\par 4 زمین ماتم می‌کند و پژمرده می‌شود. ربع مسکون کاهیده و پژمرده می‌گردد، شریفان اهل زمین کاهیده می‌شوند.
\par 5 زمین زیرساکنانش ملوث می‌شود زیرا که از شرایع تجاوزنموده و فرایض را تبدیل کرده و عهد جاودانی راشکسته‌اند.
\par 6 بنابراین لعنت، جهان را فانی کرده است و ساکنانش سزا یافته‌اند لهذا ساکنان زمین سوخته شده‌اند و مردمان، بسیار کم باقی‌مانده‌اند.
\par 7 شیره انگور ماتم می‌گیرد و مو کاهیده می‌گردد وتمامی شاددلان آه می‌کشند.
\par 8 شادمانی دفها تلف شده، آواز عشرت کنندگان باطل و شادمانی بربطها ساکت خواهد شد.
\par 9 شراب را با سرودهانخواهند آشامید و مسکرات برای نوشندگانش تلخ خواهد شد.
\par 10 قریه خرابه منهدم می‌شود وهر خانه بسته می‌گردد که کسی داخل آن نتواندشد.
\par 11 غوغایی برای شراب در کوچه‌ها است. هرگونه شادمانی تاریک گردیده و سرور زمین رفع شده است.
\par 12 ویرانی در شهر باقی است ودروازه هایش به هلاکت خرد شده است.
\par 13 زیرا که در وسط زمین در میان قوم هایش چنین خواهدشد مثل تکانیدن زیتون و مانند خوشه هایی که بعد از چیدن انگور باقی می‌ماند.
\par 14 اینان آواز خود را بلند کرده، ترنم خواهندنمود و درباره کبریایی خداوند از دریا صداخواهند زد.
\par 15 از این جهت خداوند را در بلادمشرق و نام یهوه خدای اسرائیل را درجزیره های دریا تمجید نمایید.
\par 16 از کرانهای زمین سرودها را شنیدیم که عادلان را جلال باد. اما گفتم: وا حسرتا، وا حسرتا، وای بر من! خیانت کاران خیانت ورزیده، خیانت کاران به شدت خیانت ورزیده‌اند.
\par 17 ‌ای ساکن زمین ترس و حفره و دام بر تو است.
\par 18 و واقع خواهد شد که هرکه از آواز ترس بگریزد به حفره خواهد افتاد وهر‌که از میان حفره برآید گرفتار دام خواهد شدزیرا که روزنه های علیین باز شده و اساسهای زمین متزلزل می‌باشد.
\par 19 زمین بالکل منکسرشده. زمین تمام از هم پاشیده و زمین به شدت متحرک گشته است.
\par 20 زمین مثل مستان افتان وخیزان است و مثل سایه بان به چپ و راست متحرک و گناهش بر آن سنگین است. پس افتاده است که بار دیگر نخواهد برخاست.
\par 21 و در آن روز واقع خواهد شد که خداوندگروه شریفان را بر مکان بلند ایشان و پادشاهان زمین را بر زمین سزا خواهد داد.
\par 22 و ایشان مثل اسیران در چاه جمع خواهند شد و درزندان بسته خواهند گردید و بعد از روزهای بسیار، ایشان طلبیده خواهند شد.و ماه خجل و آفتاب رسوا خواهد گشت زیرا که یهوه صبایوت در کوه صهیون و در اورشلیم و به حضور مشایخ خویش، با جلال سلطنت خواهدنمود.
\par 23 و ماه خجل و آفتاب رسوا خواهد گشت زیرا که یهوه صبایوت در کوه صهیون و در اورشلیم و به حضور مشایخ خویش، با جلال سلطنت خواهدنمود.
 
\chapter{25}

\par 1 ای یهوه تو خدای من هستی پس تو راتسبیح می‌خوانم و نام تو را حمدمی گویم، زیرا کارهای عجیب کرده‌ای وتقدیرهای قدیم تو امانت و راستی است.
\par 2 چونکه شهری را توده و قریه حصین را خرابه گردانیده‌ای و قصر غریبان را که شهر نباشد و هرگز بنا نگردد.
\par 3 بنابراین قوم عظیم، تو را تمجید می‌نمایند وقریه امت های ستم پیشه از تو خواهند ترسید.
\par 4 چونکه برای فقیران قلعه و به جهت مسکینان درحین تنگی ایشان قلعه بودی و ملجا از طوفان وسایه از گرمی، هنگامی که نفخه ستمکاران مثل طوفان بر دیوار می‌بود.
\par 5 و غوغای غریبان را مثل گرمی در جای خشک فرود خواهی آورد و سرودستمکاران مثل گرمی از سایه ابر پست خواهدشد.
\par 6 و یهوه صبایوت در این کوه برای همه قوم هاضیافتی از لذایذ برپا خواهد نمود. یعنی ضیافتی از شرابهای کهنه از لذایذ پر مغز و از شرابهای کهنه مصفا.
\par 7 و در این کوه روپوشی را که برتمامی قوم‌ها گسترده است و ستری را که جمیع امت‌ها را می‌پوشاند تلف خواهد کرد.
\par 8 و موت را تا ابدالاباد نابود خواهد ساخت و خداوند یهوه اشکها را از هر‌چهره پاک خواهد نمود و عار قوم خویش را از روی تمامی زمین رفع خواهد کردزیرا خداوند گفته است.
\par 9 و در آن روز خواهند گفت: «اینک این خدای ما است که منتظر او بوده‌ایم و ما را نجات خواهدداد. این خداوند است که منتظر او بوده‌ایم پس از نجات او مسرور و شادمان خواهیم شد.»
\par 10 زیراکه دست خداوند بر این کوه قرار خواهد گرفت وموآب در مکان خود پایمال خواهد شد چنانکه کاه در آب مزبله پایمال می‌شود.
\par 11 و او دستهای خود را در میان آن خواهد گشاد مثل شناوری که به جهت شنا کردن دستهای خود را می‌گشاید وغرور او را با حیله های دستهایش پست خواهدگردانید.و قلعه بلند حصارهایت را خم کرده، بزیر خواهد افکند و بر زمین با غبار یکسان خواهد ساخت.
\par 12 و قلعه بلند حصارهایت را خم کرده، بزیر خواهد افکند و بر زمین با غبار یکسان خواهد ساخت.
 
\chapter{26}

\par 1 در آن روز این سرود در زمین یهوداسراییده خواهد شد؛ ما را شهری قوی است که دیوارها و حصار آن نجات است. 
\par 2 دروازه‌ها را بگشایید تا امت عادل که امانت رانگاه می‌دارند داخل شوند.
\par 3 دل ثابت را درسلامتی کامل نگاه خواهی داشت، زیرا که بر توتوکل دارد.
\par 4 بر خداوند تا به ابد توکل نمایید، چونکه در یاه یهوه صخره جاودانی است.
\par 5 زیراآنانی را که بر بلندیها ساکنند فرود می‌آورد. وشهر رفیع را به زیر می‌اندازد. آن را به زمین انداخته، با خاک یکسان می‌سازد.
\par 6 پایها آن راپایمال خواهد کرد. یعنی پایهای فقیران وقدمهای مسکینان.
\par 7 طریق عادلان استقامت است. ای تو که مستقیم هستی طریق عادلان را هموار خواهی ساخت.
\par 8 پس‌ای خداوند در طریق داوریهای توانتظار تو را کشیده‌ایم. و جان ما به اسم تو و ذکر تو مشتاق است.
\par 9 شبانگاه به‌جان خود مشتاق توهستم. و بامدادان به روح خود در اندرونم تو رامی طلبم. زیرا هنگامی که داوریهای تو بر زمین آید سکنه ربع مسکون عدالت را خواهندآموخت.
\par 10 هرچند بر شریر ترحم شود عدالت را نخواهد آموخت. در زمین راستان شرارت می‌ورزد و جلال خداوند را مشاهده نمی نماید.
\par 11 ‌ای خداوند دست تو برافراشته شده است امانمی بینند. لیکن چون غیرت تو را برای قوم ملاحظه کنند خجل خواهند شد. و آتش نیزدشمنانت را فرو خواهد برد.
\par 12 ‌ای خداوند سلامتی را برای ما تعیین خواهی نمود. زیرا که تمام کارهای ما را نیز برای ما به عمل آورده‌ای.
\par 13 ‌ای یهوه خدای ما آقایان غیر از تو بر ما استیلا داشتند. اما به تو فقط اسم تورا ذکر خواهیم کرد.
\par 14 ایشان مردند و زنده نخواهند شد. خیالها گردیدند و نخواهندبرخاست. بنابراین ایشان را سزا داده، هلاک ساختی و تمام ذکر ایشان را محو نمودی.
\par 15 قوم را افزودی‌ای خداوند قوم را مزید ساخته، خویشتن را جلال دادی. و تمامی حدود زمین راوسیع گردانیدی.
\par 16 ‌ای خداوند ایشان در حین تنگی، تو راخواهند طلبید. و چون ایشان را تادیب نمایی دعاهای خفیه خواهند ریخت.
\par 17 مثل زن حامله‌ای که نزدیک زاییدن باشد و درد او راگرفته، از آلام خود فریاد بکند همچنین ما نیز‌ای خداوند در حضور تو هستیم.
\par 18 حامله شده، دردزه ما را گرفت و باد را زاییدیم. و در زمین هیچ نجات به ظهور نیاوردیم. و ساکنان ربع مسکون نیفتادند.
\par 19 مردگان تو زنده خواهند شد و جسدهای من خواهند برخاست. ای شما که در خاک ساکنیدبیدار شده، ترنم نمایید! زیرا که شبنم تو شبنم نباتات است. و زمین مردگان خود را بیرون خواهدافکند.
\par 20 ‌ای قوم من بیایید به حجره های خویش داخل شوید و درهای خود را در عقب خویش ببندید. خویشتن را اندک لحظه‌ای پنهان کنید تاغضب بگذرد.زیرا اینک خداوند از مکان خود بیرون می‌آید تا سزای گناهان ساکنان زمین را به ایشان برساند. پس زمین خونهای خود رامکشوف خواهد ساخت و کشتگان خویش رادیگر پنهان نخواهد نمود.
\par 21 زیرا اینک خداوند از مکان خود بیرون می‌آید تا سزای گناهان ساکنان زمین را به ایشان برساند. پس زمین خونهای خود رامکشوف خواهد ساخت و کشتگان خویش رادیگر پنهان نخواهد نمود.
 
\chapter{27}

\par 1 در آن روز خداوند به شمشیر سخت عظیم محکم خود آن مار تیز رو لویاتان را و آن مار پیچیده لویاتان را سزا خواهد داد و آن اژدها را که در دریا است خواهد کشت.
\par 2 در آن روز برای آن تاکستان شراب بسرایید.
\par 3 من که یهوه هستم آن را نگاه می‌دارم و هر دقیقه آن راآبیاری می‌نمایم. شب و روز آن را نگاهبانی می‌نمایم که مبادا احدی به آن ضرر برساند.
\par 4 خشم ندارم. کاش که خس و خار با من به جنگ می‌آمدند تا بر آنها هجوم آورده، آنها را با هم می‌سوزانیدم.
\par 5 یا به قوت من متمسک می‌شد تا بامن صلح بکند و با من صلح می‌نمود.
\par 6 در ایام آینده یعقوب ریشه خواهد زد واسرائیل غنچه و شکوفه خواهد آورد. و ایشان روی ربع مسکون را از میوه پر خواهند ساخت.
\par 7 آیا او را زد بطوری که دیگران او را زدند؟ یاکشته شد بطوری که مقتولان وی کشته شدند؟
\par 8 چون او را دور ساختی به اندازه با وی معارضه نمودی. با باد سخت خویش او را در روز بادشرقی زایل ساختی.
\par 9 بنابراین گناه یعقوب از این کفاره شده و رفع گناه او تمامی نتیجه آن است. چون تمامی سنگهای مذبح را مثل سنگهای آهک نرم شده می‌گرداند آنگاه اشیریم و بتهای آفتاب دیگر برپا نخواهد شد.
\par 10 زیرا که آن شهرحصین منفرد خواهد شد و آن مسکن، مهجور ومثل بیابان واگذاشته خواهد شد. در آنجاگوساله‌ها خواهند چرید و در آن خوابیده، شاخه هایش را تلف خواهند کرد.
\par 11 چون شاخه هایش خشک شود شکسته خواهد شد. پس زنان آمده، آنها را خواهند سوزانید، زیرا که ایشان قوم بیفهم هستند. لهذا آفریننده ایشان برایشان ترحم نخواهد نمود و خالق ایشان بر ایشان شفقت نخواهد کرد.
\par 12 و در آن روز واقع خواهد شد که خداوند ازمسیل نهر (فرات ) تا وادی مصر غله را خواهدکوبید. و شما‌ای بنی‌اسرائیل یکی یکی جمع کرده خواهید شد.و در آن روز واقع خواهدشد که کرنای بزرگ نواخته خواهد شد وگم شدگان زمین آشور و رانده شدگان زمین مصرخواهند آمد. و خداوند را در کوه مقدس یعنی دراورشلیم عبادت خواهند نمود.
\par 13 و در آن روز واقع خواهدشد که کرنای بزرگ نواخته خواهد شد وگم شدگان زمین آشور و رانده شدگان زمین مصرخواهند آمد. و خداوند را در کوه مقدس یعنی دراورشلیم عبادت خواهند نمود.
 
\chapter{28}

\par 1 وای بر تاج تکبر میگساران افرایم وبرگل پژمرده زیبایی جلال وی، که بر سر وادی بارور مغلوبان شراب است.
\par 2 اینک خداوند کسی زورآور و توانا دارد که مثل تگرگ شدید و طوفان مهلک و مانند سیل آبهای زورآورسرشار، آن را به زور بر زمین خواهد انداخت.
\par 3 وتاج تکبر میگساران افرایم زیر پایها پایمال خواهد شد.
\par 4 و گل پژمرده زیبایی جلال وی که بر سر وادی بارور است مثل نوبر انجیرها قبل ازتابستان خواهد بود که چون بیننده آن را بیندوقتی که هنوز در دستش باشد آن را فرو می‌برد.
\par 5 و در آن روز یهوه صبایوت به جهت بقیه قوم خویش تاج جلال و افسر جمال خواهد بود.
\par 6 وروح انصاف برای آنانی که به داوری می‌نشینند وقوت برای آنانی که جنگ را به دروازه هابرمی گردانند (خواهد بود).
\par 7 ولکن اینان نیز از شراب گمراه شده‌اند و ازمسکرات سرگشته گردیده‌اند. هم کاهن و هم نبی از مسکرات گمراه شده‌اند و از شراب بلعیده گردیده‌اند. از مسکرات سرگشته شده‌اند و دررویا گمراه گردیده‌اند و در داوری مبهوت گشته‌اند.
\par 8 زیرا که همه سفره‌ها از قی و نجاست پر گردیده و جایی نمانده است.
\par 9 کدام را معرفت خواهد آموخت و اخبار را به که خواهد فهمانید؟ آیا نه آنانی را که از شیر بازداشته و از پستانها گرفته شده‌اند؟
\par 10 زیرا که حکم بر حکم و حکم بر حکم، قانون بر قانون وقانون بر قانون اینجا اندکی و آنجا اندکی خواهدبود.
\par 11 زیرا که با لبهای الکن و زبان غریب با این قوم تکلم خواهد نمود.
\par 12 که به ایشان گفت: «راحت همین است. پس خسته شدگان را مستریح سازید و آرامی همین است.» امانخواستند که بشنوند.
\par 13 و کلام خداوند برای ایشان حکم بر حکم و حکم برحکم، قانون برقانون و قانون بر قانون اینجا اندکی و آنجا اندکی خواهد بود تا بروند و به پشت افتاده، منکسرگردند و به دام افتاده، گرفتار شوند.
\par 14 بنابراین‌ای مردان استهزا کننده و‌ای حاکمان این قوم که دراورشلیم‌اند کلام خداوند را بشنوید.
\par 15 از آنجاکه گفته‌اید با موت عهد بسته‌ایم و با هاویه همداستان شده‌ایم، پس چون تازیانه مهلک بگذرد به ما نخواهد رسید زیرا که دروغها راملجای خود نمودیم و خویشتن را زیر مکرمستور ساختیم.
\par 16 بنابراین خداوند یهوه چنین می‌گوید: «اینک در صهیون سنگ بنیادی نهادم یعنی سنگ آزموده و سنگ زاویه‌ای گرانبها و اساس محکم پس هر‌که ایمان آورد تعجیل نخواهد نمود.
\par 17 وانصاف را ریسمان می‌گردانم و عدالت را ترازو وتگرگ ملجای دروغ را خواهد رفت و آبها ستر راخواهد برد.
\par 18 و عهد شما با موت باطل خواهدشد و میثاق شما با هاویه ثابت نخواهد ماند وچون تازیانه شدید بگذرد شما از آن پایمال خواهید شد.
\par 19 هر وقت که بگذرد شما را گرفتارخواهد ساخت زیرا که هر بامداد هم در روز و هم در شب خواهد گذشت و فهمیدن اخبار باعث هیبت محض خواهد شد.»
\par 20 زیرا که بسترکوتاه تر است از آنکه کسی بر آن دراز شود ولحاف تنگ تر است از آنکه کسی خویشتن رابپوشاند.
\par 21 زیرا خداوند چنانکه در کوه فراصیم (کرد) خواهد برخاست و چنانکه در وادی جبعون (نمود) خشمناک خواهد شد، تا کار خودیعنی کار عجیب خود را بجا آورد و عمل خویش یعنی عمل غریب خویش را به انجام رساند.
\par 22 پس الان استهزا منمایید مبادا بندهای شمامحکم گردد، زیرا هلاکت و تقدیری را که ازجانب خداوند یهوه صبایوت بر تمامی زمین می‌آید شنیده‌ام.
\par 23 گوش گیرید و آواز مرا بشنوید و متوجه شده، کلام مرا استماع نمایید.
\par 24 آیا برزگر، همه روز به جهت تخم پاشیدن شیار می‌کند و آیا همه وقت زمین خود را می‌شکافد و هموار می‌نماید؟
\par 25 آیا بعد از آنکه رویش را هموار کرد گشنیز رانمی پاشد و زیره را نمی افشاند و گندم را درشیارها و جو را در جای معین و ذرت را درحدودش نمی گذارد؟
\par 26 زیرا که خدایش او را به راستی می‌آموزد و او را تعلیم می‌دهد.
\par 27 چونکه گشنیز با گردون تیز کوبیده نمی شود و چرخ ارابه بر زیره گردانیده نمی گردد، بلکه گشنیز به عصا وزیره به چوب تکانیده می‌شود.
\par 28 گندم آردمی شود زیرا که آن را همیشه خرمن کوبی نمی کندو هرچند چرخ ارابه و اسبان خود را بر آن بگرداندآن را خرد نمی کند.این نیز از جانب یهوه صبایوت که عجیب الرای و عظیم الحکمت است صادر می‌گردد.
\par 29 این نیز از جانب یهوه صبایوت که عجیب الرای و عظیم الحکمت است صادر می‌گردد.
 
\chapter{29}

\par 1 وای بر اریئیل! وای بر اریئیل! شهری که داود در آن خیمه زد. سال بر سال مزید کنید و عیدها دور زنند.
\par 2 و من اریئیل را به تنگی خواهم‌انداخت و ماتم و نوحه گری خواهد بود و آن برای من مثل اریئیل خواهد بود.
\par 3 و بر توبه هر طرف اردو زده، تو را به باره‌ها محاصره خواهم نمود و منجنیقها بر تو خواهم افراشت.
\par 4 و به زیر افکنده شده، از زمین تکلم خواهی نمودو کلام تو از میان غبار پست خواهد گردید و آوازتو از زمین مثل آواز جن خواهد بود و زبان تو ازمیان غبار زمزم خواهد کرد.
\par 5 اما گروه دشمنانت مثل گرد نرم خواهند شد و گروه ستم کیشان مانندکاه که می‌گذرد و این بغته در لحظه‌ای واقع خواهد شد.
\par 6 و از جانب یهوه صبایوت با رعد وزلزله و صوت عظیم و گردباد و طوفان و شعله آتش سوزنده از تو پرسش خواهد شد.
\par 7 وجمعیت تمام امت هایی که با اریئیل جنگ می‌کنند یعنی تمامی آنانی که بر او و بر قلعه وی مقاتله می‌نمایند و او را بتنگ می‌آورند مثل خواب و رویای شب خواهند شد.
\par 8 و مثل شخص گرسنه که خواب می‌بیند که می‌خورد وچون بیدار شود شکم او تهی است. یا شخص تشنه که خواب می‌بیند که آب می‌نوشد و چون بیدار شود اینک ضعف دارد و جانش مشتهی می‌باشد. همچنین تمامی جماعت امت هایی که باکوه صهیون جنگ می‌کنند خواهند شد.
\par 9 درنگ کنید و متحیر باشید و تمتع برید وکور باشید. ایشان مست می‌شوند لیکن نه ازشراب و نوان می‌گردند اما نه از مسکرات.
\par 10 زیراخداوند بر شما روح خواب سنگین را عارض گردانیده، چشمان شما را بسته است. و انبیا وروسای شما یعنی رائیان را محجوب کرده است.
\par 11 و تمامی رویا برای شما مثل کلام تومارمختوم گردیده است که آن را به کسی‌که خواندن می داند داده، می‌گویند: این را بخوان و اومی گوید: نمی توانم چونکه مختوم است.
\par 12 و آن طومار را به کسی‌که خواندن نداند داده، می‌گوینداین را بخوان و او می‌گوید خواندن نمی دانم.
\par 13 و خداوند می‌گوید: «چونکه این قوم ازدهان خود به من تقرب می‌جویند و به لبهای خویش مرا تمجید می‌نمایند اما دل خود را از من دور کرده‌اند و ترس ایشان از من وصیتی است که از انسان آموخته‌اند، 
\par 14 بنابراین اینک من بار دیگربا این قوم عمل عجیب و غریب بجا خواهم آوردو حکمت حکیمان ایشان باطل و فهم فهیمان ایشان مستور خواهد شد.»
\par 15 وای بر آنانی که مشورت خود را از خداوندبسیار عمیق پنهان می‌کنند و اعمال ایشان درتاریکی می‌باشد و می‌گویند: «کیست که مارا ببیندو کیست که ما را بشناسد؟»
\par 16 ‌ای زیر و زبرکنندگان هرچیز! آیا کوزه‌گر مثل گل محسوب شود یا مصنوع درباره صانع خود گوید مرانساخته است و یا تصویر درباره مصورش گویدکه فهم ندارد؟
\par 17 آیا در اندک زمانی واقع نخواهدشد که لبنان به بوستان مبدل گردد و بوستان به جنگل محسوب شود؟
\par 18 و در آن روز کران کلام کتاب را خواهند شنید و چشمان کوران از میان ظلمت و تاریکی خواهد دید.
\par 19 و حلیمان شادمانی خود را در خداوند مزید خواهند کرد ومسکینان مردمان در قدوس اسرائیل وجدخواهند نمود.
\par 20 زیرا که ستمگران نابود واستهزاکنندگان معدوم خواهند شد و پیروان شرارت منقطع خواهند گردید.
\par 21 که انسان را به سخنی مجرم می‌سازند و برای کسی‌که در محکمه حکم می‌کند دام می‌گسترانند و عادل رابه بطالت منحرف می‌سازند.
\par 22 بنابراین خداوندکه ابراهیم را فدیه داده است درباره خاندان یعقوب چنین می‌گوید که از این به بعد یعقوب خجل نخواهد شد و رنگ چهره‌اش دیگرنخواهد پرید.
\par 23 بلکه چون فرزندان خود را که عمل دست من می‌باشند در میان خویش بیندآنگاه ایشان اسم مرا تقدیس خواهند نمود وقدوس یعقوب را تقدیس خواهند کرد و ازخدای اسرائیل خواهند ترسید.و آنانی که روح گمراهی دارند فهیم خواهند شد و متمردان تعلیم را خواهند آموخت.
\par 24 و آنانی که روح گمراهی دارند فهیم خواهند شد و متمردان تعلیم را خواهند آموخت.
 
\chapter{30}

\par 1 خداوند می‌گوید که وای بر پسران فتنه انگیز که مشورت می‌کنند لیکن نه ازمن و عهد می‌بندند لیکن نه از روح من، تا گناه را برگناه مزید نمایند.
\par 2 که برای فرود شدن به مصرعزیمت می‌کنند اما از دهان من سوال نمی نمایندو به قوت فرعون پناه می‌گیرند و به سایه مصراعتماد دارند.
\par 3 لهذا قوت فرعون خجالت واعتماد به سایه مصر رسوایی شما خواهد بود.
\par 4 زیرا که سروران او در صوعن هستند و ایلچیان وی به حانیس رسیده‌اند.
\par 5 همگی ایشان از قومی که برای ایشان فایده ندارند خجل خواهند شد که نه معاونت و نه منفعتی بلکه خجالت و رسوایی نیز برای ایشان خواهند بود.
\par 6 وحی درباره بهیموت جنوبی: از میان زمین تنگ و ضیق که از آنجا شیر ماده و اسد وافعی ومار آتشین پرنده می‌آید. توانگری خویش را برپشت الاغان و گنجهای خود را بر کوهان شتران نزد قومی که منفعت ندارند می‌برند.
\par 7 چونکه اعانت مصریان عبث و بی‌فایده است از این جهت ایشان را رهب الجلوس نامیدم.
\par 8 الان بیا و این رادر نزد ایشان بر لوحی بنویس و بر طوماری مرقوم ساز تا برای ایام آینده تا ابدالاباد بماند.
\par 9 زیرا که این قوم فتنه انگیز و پسران دروغگومی باشند. پسرانی که نمی خواهند شریعت خداوند را استماع نمایند.
\par 10 که به رائیان می‌گویند: رویت مکنید و به انبیا که برای ما به راستی نبوت ننمایید بلکه سخنان شیرین به ماگویید و به مکاید نبوت کنید.
\par 11 از راه منحرف شوید و از طریق تجاوز نمایید و قدوس اسرائیل را از نظر ما دور سازید.
\par 12 بنابراین قدوس اسرائیل چنین می‌گوید: «چونکه شما این کلام را ترک کردید و بر ظلم وفساد اعتماد کرده، بر آن تکیه نمودید،
\par 13 از این جهت این گناه برای شما مثل شکاف نزدیک به افتادن که در دیوار بلند پیش آمده باشد و خرابی آن در لحظه‌ای بغته پدید آید خواهد بود.
\par 14 وشکستگی آن مثل شکستگی کوزه کوزه‌گرخواهد بود که بی‌محابا خرد می‌شود بطوری که ازپاره هایش پاره‌ای به جهت گرفتن آتش از آتشدان یا برداشتن آب از حوض یافت نخواهد شد.»
\par 15 زیرا خداوند یهوه قدوس اسرائیل چنین می‌گوید: «به انابت و آرامی نجات می‌یافتید وقوت شما از راحت و اعتماد می‌بود، امانخواستید.
\par 16 و گفتید: نی بلکه بر اسبان فرارمی کنیم، لهذا فرار خواهید کرد و بر اسبان تیز روسوار می‌شویم لهذا تعاقب کنندگان شما تیز روخواهند شد.
\par 17 هزار نفر از نهیب یک نفر فرارخواهند کرد و شما از نهیب پنج نفر خواهیدگریخت تا مثل بیدق بر قله کوه و علم بر تلی باقی مانید.»
\par 18 و از این سبب خداوند انتظار می‌کشد تا بر شما رافت نماید و از این سبب برمی خیزد تا برشما ترحم فرماید چونکه یهوه خدای انصاف است. خوشابحال همگانی که منتظر وی باشند.
\par 19 زیرا که قوم در صهیون در اورشلیم ساکن خواهند بود و هرگز گریه نخواهی کرد و به آوازفریادت بر تو ترحم خواهد کرد، و چون بشنود تورا اجابت خواهد نمود.
\par 20 و هرچند خداوند شمارا نان ضیق و آب مصیبت بدهد اما معلمانت باردیگر مخفی نخواهند شد بلکه چشمانت معلمان تو را خواهد دید.
\par 21 و گوشهایت سخنی را ازعقب تو خواهد شنید که می‌گوید: راه این است، در آن سلوک بنما هنگامی که به طرف راست یاچپ می‌گردی.
\par 22 و پوشش بتهای ریخته نقره خویش را و ستر اصنام تراشیده طلای خود رانجس خواهید ساخت و آنها را مثل چیز نجس دور انداخته، به آن خواهی گفت: دور شو.
\par 23 و باران تخمت را که زمین خویش را به آن زرع می‌کنی و نان محصول زمینت را خواهد داد وآن پر مغز و فراوان خواهد شد و در آن زمان مواشی تو در مرتع وسیع خواهند چرید.
\par 24 وگاوان و الاغانت که زمین را شیار می‌نمایند، آذوقه نمک دار را که با غربال و اوچوم پاک شده است خواهند خورد.
\par 25 و در روز کشتار عظیم که برجها در آن خواهد افتاد نهرها و جویهای آب برهر کوه بلند و به هر تل مرتفع جاری خواهد شد.
\par 26 و در روزی که خداوند شکستگی قوم خود راببندد و ضرب جراحت ایشان را شفا دهدروشنایی ماه مثل روشنایی آفتاب و روشنایی آفتاب هفت چندان مثل روشنایی هفت روزخواهد بود.
\par 27 اینک اسم خداوند از جای دور می‌آید، درغضب خود سوزنده و در ستون غلیظ و لبهایش پر از خشم و زبانش مثل آتش سوزان است.
\par 28 ونفخه او مثل نهر سرشار تا به گردن می‌رسد تا آنکه امت‌ها را به غربال مصیبت ببیزد و دهنه ضلالت رابر چانه قوم‌ها بگذارد.
\par 29 و شما را سرودی خواهد بود مثل شب تقدیس نمودن عید وشادمانی دل مثل آنانی که روانه می‌شوند تا به آوازنی به کوه خداوند نزد صخره اسرائیل بیایند.
\par 30 وخداوند جلال آواز خود را خواهد شنوانید وفرود آوردن بازوی خود را با شدت غضب وشعله آتش سوزنده و طوفان و سیل و سنگهای تگرگ ظاهر خواهد ساخت.
\par 31 زیرا که آشور به آواز خداوند شکسته خواهد شد و او را با عصاخواهد زد.
\par 32 و هر ضرب عصای قضا که خداوندبه وی خواهد آورد با دف و بربط خواهد بود و باجنگهای پر شورش با آن مقاتله خواهد نمود.زیرا که توفت از قبل مهیا شده و برای پادشاه آماده گردیده است. آن را عمیق و وسیع ساخته است که توده‌اش آتش و هیزم بسیار است و نفخه خداوند مثل نهر کبریت آن را مشتعل خواهدساخت.
\par 33 زیرا که توفت از قبل مهیا شده و برای پادشاه آماده گردیده است. آن را عمیق و وسیع ساخته است که توده‌اش آتش و هیزم بسیار است و نفخه خداوند مثل نهر کبریت آن را مشتعل خواهدساخت.
 
\chapter{31}

\par 1 وای بر آنانی که به جهت اعانت به مصرفرود آیند و بر اسبان تکیه نمایند و برارابه‌ها زانرو که کثیرند و بر سواران زانرو که بسیارقوی‌اند توکل کنند اما بسوی قدوس اسرائیل نظرنکنند و خداوند را طلب ننمایند.
\par 2 و او نیز حکیم است و بلا را می‌آورد و کلام خود را برنخواهدگردانید و به ضد خاندان شریران و اعانت بدکاران خواهد برخاست.
\par 3 اما مصریان انسانند و نه خداو اسبان ایشان جسدند و نه روح و خداوند دست خود را دراز می‌کند و اعانت کننده را لغزان و اعانت کرده شده را افتان گردانیده هر دوی ایشان هلاک خواهند شد.
\par 4 زیرا خداوند به من چنین گفته است چنانکه شیر و شیر ژیان بر شکار خود غرش می‌نمایدهنگامی که گروه شبانان بروی جمع شوند و ازصدای ایشان نترسیده از غوغای ایشان سر فرونمی آورد همچنان یهوه صبایوت نزول می‌فرماید تا برای کوه صهیون و تل آن مقاتله نماید.
\par 5 مثل مرغان که در طیران باشند همچنان یهوه صبایوت اورشلیم را حمایت خواهد نمود وحمایت کرده، آن را رستگار خواهد ساخت و ازآن درگذشته، خلاصی خواهد داد.
\par 6 ‌ای بنی‌اسرائیل بسوی آن کس که بر وی بینهایت عصیان ورزیده‌اید بازگشت نمایید.
\par 7 زیرا که در آن روز هر کدام از ایشان بتهای نقره وبتهای طلای خود را که دستهای شما آنها را به جهت گناه خویش ساخته است ترک خواهندنمود.
\par 8 آنگاه آشور به شمشیری که از انسان نباشد خواهد افتاد و تیغی که از انسان نباشد او راهلاک خواهد ساخت و او از شمشیر خواهدگریخت و جوانانش خراج گذار خواهند شد.وصخره او از ترس زایل خواهد شد و سرورانش ازعلم مبهوت خواهند گردید. یهوه که آتش او درصهیون و کوره وی در اورشلیم است این رامی گوید.
\par 9 وصخره او از ترس زایل خواهد شد و سرورانش ازعلم مبهوت خواهند گردید. یهوه که آتش او درصهیون و کوره وی در اورشلیم است این رامی گوید.
 
\chapter{32}

\par 1 اینک پادشاهی به عدالت سلطنت خواهد نمود و سروران به انصاف حکمرانی خواهند کرد.
\par 2 و مردی مثل پناه گاهی از باد و پوششی از طوفان خواهد بود. و مانندجویهای آب در جای خشک و سایه صخره عظیم در زمین تعب آورنده خواهد بود.
\par 3 وچشمان بینندگان تار نخواهد شد و گوشهای شنوندگان اصغا خواهد کرد.
\par 4 و دل شتابندگان معرفت را خواهد فهمید و زبان الکنان کلام فصیح را به ارتجال خواهد گفت.
\par 5 و مرد لئیم بار دیگرکریم خوانده نخواهد شد و مرد خسیس نجیب گفته نخواهد گردید.
\par 6 زیرا مرد لئیم به لامت متکلم خواهد شد و دلش شرارت را بعمل خواهد آورد تا نفاق را بجا آورده، به ضد خداوندبه ضلالت سخن گوید و جان گرسنگان را تهی کند و آب تشنگان را دور نماید.
\par 7 آلات مرد لئیم نیز زشت است و تدابیر قبیح می‌نماید تا مسکینان را به سخنان دروغ هلاک نماید، هنگامی که مسکینان به انصاف سخن می‌گویند.
\par 8 اما مردکریم تدبیرهای کریمانه می‌نماید و به کرم پایدارخواهد شد.
\par 9 ‌ای زنان مطمئن برخاسته، آواز مرا بشنوید وای دختران ایمن سخن مرا گوش گیرید.
\par 10 ‌ای دختران ایمن بعد از یک سال و چند روزی مضطرب خواهید شد زانرو که چیدن انگور قطع می‌شود و جمع کردن میوه‌ها نخواهد بود.
\par 11 ‌ای زنان مطمئن بلرزید و‌ای دختران ایمن مضطرب شوید. لباس را کنده، برهنه شوید و پلاس بر میان خود ببندید.
\par 12 برای مزرعه های دلپسند وموهای بارور سینه خواهند زد.
\par 13 بر زمین قوم من خار و خس خواهد رویید بلکه بر همه خانه های شادمانی در شهر مفتخر.
\par 14 زیرا که قصر منهدم وشهر معمور متروک خواهد شد و عوفل ودیده بانگاه به بیشه‌ای سباع و محل تفرج خران وحشی و مرتع گله‌ها تا به ابد مبدل خواهد شد.
\par 15 تا زمانی که روح از اعلی علیین بر ما ریخته شود و بیابان به بوستان مبدل گردد و بوستان به جنگل محسوب شود.
\par 16 آنگاه انصاف در بیابان ساکن خواهد شد و عدالت در بوستان مقیم خواهد گردید.
\par 17 و عمل عدالت سلامتی ونتیجه عدالت آرامی و اطمینان خواهد بود تاابدالاباد.
\par 18 و قوم من در مسکن سلامتی و درمساکن مطمئن و در منزلهای آرامی ساکن خواهند شد.
\par 19 و حین فرود آمدن جنگل تگرگ خواهد بارید و شهر به درکه اسفل خواهد افتاد.خوشابحال شما که بر همه آبها تخم می‌کاریدو پایهای گاو و الاغ را رها می‌سازید.
\par 20 خوشابحال شما که بر همه آبها تخم می‌کاریدو پایهای گاو و الاغ را رها می‌سازید.
 
\chapter{33}

\par 1 وای بر تو‌ای غارتگر که غارت نشدی وای خیانت کاری که با تو خیانت نورزیدند. هنگامی که غارت را به اتمام رسانیدی غارت خواهی شد و زمانی که از خیانت نمودن دست برداشتی به تو خیانت خواهند ورزید.
\par 2 ‌ای خداوند بر ما ترحم فرما زیرا که منتظر تومی باشیم و هر بامداد بازوی ایشان باش و در زمان تنگی نیز نجات ما بشو. 
\par 3 از آواز غوغا، قوم هاگریختند و چون خویشتن را برافرازی امت هاپراکنده خواهند شد.
\par 4 و غارت شما را جمع خواهند کرد بطوری که موران جمع می‌نمایند وبر آن خواهند جهید بطوری که ملخها می‌جهند.
\par 5 خداوند متعال می‌باشد زانرو که دراعلی علیین ساکن است و صهیون را از انصاف وعدالت مملو خواهد ساخت.
\par 6 و فراوانی نجات وحکمت و معرفت استقامت اوقات تو خواهد شد. و ترس خداوند خزینه او خواهد بود.
\par 7 اینک شجاعان ایشان در بیرون فریاد می‌کنند و رسولان سلامتی زار‌زار گریه می‌نمایند.
\par 8 شاهراههاویران می‌شود و راه گذریان تلف می‌گردند. عهدرا شکسته است و شهرها را خوار نموده، به مردمان اعتنا نکرده است.
\par 9 زمین ماتم‌کنان کاهیده شده است و لبنان خجل گشته، تلف گردیده است و شارون مثل بیابان شده و باشان و کرمل برگهای خود را ریخته‌اند.
\par 10 خداوند می‌گوید که الان برمی خیزم و حال خود را برمی افرازم و اکنون متعال خواهم گردید.
\par 11 و شما از کاه حامله شده، خس خواهید زایید. و نفس شما آتشی است که شما را خواهدسوزانید.
\par 12 و قوم‌ها مثل آهک سوخته و مانندخارهای قطع شده که از آتش مشتعل گرددخواهند شد.
\par 13 ‌ای شما که دور هستید آنچه را که کرده‌ام بشنوید و‌ای شما که نزدیک می‌باشید جبروت مرا بدانید.
\par 14 گناه کارانی که در صهیون اندمی ترسند و لرزه منافقان را فرو گرفته است، (ومی گویند): کیست از ما که در آتش سوزنده ساکن خواهد شد و کیست از ما که در نارهای جاودانی ساکن خواهد گردید؟
\par 15 اما آنکه به صداقت سالک باشد و به استقامت تکلم نماید و سود ظلم را خوار شمارد و دست خویش را از گرفتن رشوه بیفشاند و گوش خود را از اصغای خون ریزی ببندد و چشمان خود را از دیدن بدیها بر هم کند،
\par 16 او در مکان های بلند ساکن خواهد شد وملجای او ملاذ صخره‌ها خواهد بود. نان او داده خواهد شد و آب او ایمن خواهد بود.
\par 17 چشمانت پادشاه را در زیباییش خواهددید و زمین بی‌پایان را خواهد نگریست.
\par 18 دل تومتذکر آن خوف خواهد شد (وخواهی گفت ): کجا است نویسنده و کجا است وزن کننده (خراج ) و کجا است شمارنده برجها.
\par 19 قوم ستم پیشه و قوم دشوار لغت را که نمی توانی شنید والکن زبان را که نمی توانی فهمید بار دیگرنخواهی دید.
\par 20 صهیون شهر جشن مقدس ما راملاحظه نما. و چشمانت اورشلیم مسکن سلامتی را خواهد دید یعنی خیمه‌ای را که منتقل نشود ومیخهایش کنده نگردد و هیچکدام از طنابهایش گسیخته نشود.
\par 21 بلکه در آنجا خداوندذوالجلال برای ما مکان جویهای آب و نهرهای وسیع خواهد بود که در آن هیچ کشتی با پاروهاداخل نخواهد شد و سفینه بزرگ از آن عبورنخواهد کرد.
\par 22 زیرا خداوند داور ما است. خداوند شریعت دهنده ما است. خداوند پادشاه ما است پس ما را نجات خواهد داد.
\par 23 ریسمانهای تو سست بود که پایه دکل خود رانتوانست محکم نگاه دارد و بادبان را نتوانست بگشاید، آنگاه غارت بسیار تقسیم شد و لنگان غنیمت را بردند.لیکن ساکن آن نخواهد گفت که بیمار هستم و گناه قومی که در آن ساکن باشندآمرزیده خواهد شد.
\par 24 لیکن ساکن آن نخواهد گفت که بیمار هستم و گناه قومی که در آن ساکن باشندآمرزیده خواهد شد.
 
\chapter{34}

\par 1 ای امت‌ها نزدیک آیید تا بشنوید! و‌ای قوم‌ها اصغا نمایید! جهان و پری آن بشنوند. ربع مسکون و هرچه از آن صادر باشد.
\par 2 زیرا که غضب خداوند بر تمامی امت‌ها و خشم وی بر جمیع لشکرهای ایشان است پس ایشان رابه هلاکت سپرده، بقتل تسلیم نموده است.
\par 3 وکشتگان ایشان دور افکنده می‌شوند و عفونت لاشهای ایشان برمی آید. و از خون ایشان کوههاگداخته می‌گردد.
\par 4 و تمامی لشکر آسمان از هم خواهند پاشید و آسمان مثل طومار پیچیده خواهد شد. و تمامی لشکر آن پژمرده خواهندگشت، بطوریکه برگ از مو بریزد و مثل میوه نارس از درخت انجیر.
\par 5 زیرا که شمشیر من درآسمان سیراب شده است. و اینک بر ادوم و برقوم مغضوب من برای داوری نازل می‌شود.
\par 6 شمشیر خداوند پر خون شده و از پیه فربه گردیده است یعنی از خون بره‌ها و بزها و از پیه گرده قوچها. زیرا خداوند را در بصره قربانی است و ذبح عظیمی در زمین ادوم.
\par 7 و گاوان وحشی باآنها خواهند افتاد و گوساله‌ها با گاوان نر. و زمین ایشان از خون سیراب شده، خاک ایشان از پیه فربه خواهد شد.
\par 8 زیرا خداوند را روز انتقام و سال عقوبت به جهت دعوی صهیون خواهد بود.
\par 9 و نهرهای آن به قیر و غبار آن به کبریت مبدل خواهد شد وزمینش قیر سوزنده خواهد گشت.
\par 10 شب و روزخاموش نشده، دودش تا به ابد خواهد برآمد. نسلا بعد نسل خراب خواهد ماند که کسی تاابدالاباد در آن گذر نکند.
\par 11 بلکه مرغ سقا وخارپشت آن را تصرف خواهند کرد و بوم وغراب در آن ساکن خواهند شد و ریسمان خرابی و شاقول ویرانی را بر آن خواهد کشید.
\par 12 و ازاشراف آن کسی در آنجا نخواهد بود که او را به پادشاهی بخوانند و جمیع روسایش نیست خواهند شد.
\par 13 و در قصرهایش خارها خواهدرویید و در قلعه هایش خسک و شتر خار و مسکن گرگ و خانه شترمرغ خواهد شد.
\par 14 و وحوش بیابان با شغال خواهند برخورد و غول به رفیق خود ندا خواهد داد و عفریت نیز در آنجا ماواگزیده، برای خود آرامگاهی خواهد یافت.
\par 15 درآنجا تیرمار آشیانه ساخته، تخم خواهد نهاد و برآن نشسته، بچه های خود را زیر سایه خود جمع خواهد کرد و در آنجا کرکسها با یکدیگر جمع خواهند شد.
\par 16 از کتاب خداوند تفتیش نموده، مطالعه کنید. یکی از اینها گم نخواهد شد و یکی جفت خود را مفقود نخواهد یافت زیرا که دهان او این را امر فرموده و روح او اینها را جمع کرده است.و او برای آنها قرعه انداخته و دست اوآن را به جهت آنها با ریسمان تقسیم نموده است. و تا ابدالاباد متصرف آن خواهند شد و نسلا بعدنسل در آن سکونت خواهند داشت.
\par 17 و او برای آنها قرعه انداخته و دست اوآن را به جهت آنها با ریسمان تقسیم نموده است. و تا ابدالاباد متصرف آن خواهند شد و نسلا بعدنسل در آن سکونت خواهند داشت.
 
\chapter{35}

\par 1 بیابان و زمین خشک شادمان خواهدشد و صحرا به وجد آمده، مثل گل سرخ خواهد شکفت.
\par 2 شکوفه بسیار نموده، باشادمانی و ترنم شادی خواهد کرد. شوکت لبنان و زیبایی کرمل و شارون به آن عطا خواهد شد. جلال یهوه و زیبایی خدای ما را مشاهده خواهندنمود.
\par 3 دستهای سست را قوی سازید و زانوهای لرزنده را محکم گردانید.
\par 4 به دلهای خائف بگویید: قوی شوید و مترسید اینک خدای شما باانتقام می‌آید. او با عقوبت الهی می‌آید و شما رانجات خواهد داد.
\par 5 آنگاه چشمان کوران باز خواهد شد وگوشهای کران مفتوح خواهد گردید.
\par 6 آنگاه لنگان مثل غزال جست و خیز خواهند نمود وزبان گنگ خواهد سرایید. زیرا که آبها در بیابان ونهرها در صحرا خواهد جوشید.
\par 7 و سراب به برکه و مکان های خشک به چشمه های آب مبدل خواهد گردید. در مسکنی که شغالها می‌خوابندعلف و بوریا و نی خواهد بود.
\par 8 و در آنجاشاهراهی و طریقی خواهد بود و به طریق مقدس نامیده خواهد شد و نجسان از آن عبور نخواهندکرد بلکه آن به جهت ایشان خواهد بود. و هرکه در آن راه سالک شود اگرچه هم جاهل باشدگمراه نخواهد گردید.
\par 9 شیری در آن نخواهد بودو حیوان درنده‌ای بر آن برنخواهد آمد و در آنجایافت نخواهد شد بلکه ناجیان بر آن سالک خواهند گشت.و فدیه شدگان خداوندبازگشت نموده، با ترنم به صهیون خواهند آمد وخوشی جاودانی بر سر ایشان خواهد بود. وشادمانی و خوشی را خواهند یافت و غم و ناله فرار خواهد کرد.
\par 10 و فدیه شدگان خداوندبازگشت نموده، با ترنم به صهیون خواهند آمد وخوشی جاودانی بر سر ایشان خواهد بود. وشادمانی و خوشی را خواهند یافت و غم و ناله فرار خواهد کرد.
 
\chapter{36}

\par 1 و در سال چهاردهم حزقیا پادشاه واقع شد که سنحاریب پادشاه آشور برتمامی شهرهای حصاردار یهودا برآمده، آنها راتسخیر نمود.
\par 2 و پادشاه آشور ربشاقی را ازلاکیش به اورشلیم نزد حزقیا پادشاه با موکب عظیم فرستاد و او نزد قنات برکه فوقانی به راه مزرعه گازر ایستاد.
\par 3 و الیاقیم بن حلقیا که ناظرخانه بود و شبنای کاتب و یوآخ بن آساف وقایع نگار نزد وی بیرون آمدند.
\par 4 و ربشاقی به ایشان گفت: «به حزقیا بگویید سلطان عظیم پادشاه آشور چنین می‌گوید: این اعتماد شما که برآن توکل می‌نمایید چیست؟
\par 5 می‌گویم که مشورت و قوت جنگ سخنان باطل است. الان کیست که بر او توکل نموده‌ای که به من عاصی شده‌ای؟
\par 6 هان بر عصای این نی خرد شده یعنی بر مصر توکل می‌نمایی که اگر کسی بر آن تکیه کند به‌دستش فرو رفته، آن را مجروح می‌سازد. همچنان است فرعون پادشاه مصر برای همگانی که بر وی توکل نمایند.
\par 7 و اگر مرا گویی که بریهوه خدای خود توکل داریم آیا او آن نیست که حزقیا مکان های بلند و مذبح های او را برداشته است و به یهودا و اورشلیم گفته که پیش این مذبح سجده نمایید؟
\par 8 پس حال با آقایم پادشاه آشورشرط ببند و من دو هزار اسب به تو می‌دهم اگر ازجانب خود سواران بر آنها توانی گذاشت.
\par 9 پس چگونه روی یک والی از کوچکترین بندگان آقایم را خواهی برگردانید و بر مصر به جهت ارابه‌ها وسواران توکل داری؟
\par 10 و آیا من الان بی‌اذن یهوه بر این زمین به جهت خرابی آن برآمده‌ام؟ یهوه مرا گفته است بر این زمین برآی و آن را خراب کن.»
\par 11 آنگاه الیقایم و شبنا و یوآخ به ربشاقی گفتند: «تمنا اینکه با بندگانت به زبان آرامی گفتگونمایی زیرا آن را می‌فهمیم و با ما به زبان یهود درگوش مردمی که بر حصارند گفتگو منمای.»
\par 12 ربشاقی گفت: «آیا آقایم مرا نزد آقایت و توفرستاده است تا این سخنان را بگویم؟ مگر مرانزد مردانی که بر حصار نشسته‌اند نفرستاده، تاایشان با شما نجاست خود را بخورند و بول خودرا بنوشند.»
\par 13 پس ربشاقی بایستاد و به آواز بلند به زبان یهود صدا زده، گفت: «سخنان سلطان عظیم پادشاه آشور را بشنوید.
\par 14 پادشاه چنین می‌گوید: حزقیا شما را فریب ندهد زیرا که شمارا نمی تواند رهانید.
\par 15 و حزقیا شما را بر یهوه مطمئن نسازد و نگوید که یهوه البته ما را خواهدرهانید و این شهر به‌دست پادشاه آشور تسلیم نخواهد شد.
\par 16 به حزقیا گوش مدهید زیرا که پادشاه آشور چنین می‌گوید: با من صلح کنید ونزد من بیرون آیید تا هرکس از مو خود و هر کس از انجیر خود بخورد و هر کس از آب چشمه خود بنوشد.
\par 17 تا بیایم و شما را به زمین مانندزمین خودتان بیاورم یعنی به زمین غله و شیره وزمین نان و تاکستانها.
\par 18 مبادا حزقیا شما را فریب دهد و گوید یهوه ما را خواهد رهانید. آیاهیچکدام از خدایان امت‌ها زمین خود را از دست پادشاه آشور رهانیده است؟
\par 19 خدایان حمات وارفاد کجایند و خدایان سفروایم کجا و آیا سامره را از دست من رهانیده‌اند؟
\par 20 از جمیع خدایان این زمینها کدامند که زمین خویش را از دست من نجات داده‌اند تا یهوه اورشلیم را از دست من نجات دهد؟»
\par 21 اما ایشان سکوت نموده، به اوهیچ جواب ندادند زیرا که پادشاه امر فرموده وگفته بود که او را جواب ندهید.پس الیاقیم بن حلقیا که ناظر خانه بود و شبنای کاتب و یوآخ بن آساف وقایع نگار با جامه دریده نزد حزقیا آمدندو سخنان ربشاقی را به او باز‌گفتند.
\par 22 پس الیاقیم بن حلقیا که ناظر خانه بود و شبنای کاتب و یوآخ بن آساف وقایع نگار با جامه دریده نزد حزقیا آمدندو سخنان ربشاقی را به او باز‌گفتند.
 
\chapter{37}

\par 1 و واقع شد که چون حزقیا پادشاه این راشنید لباس خود را چاک زده و پلاس پوشیده، به خانه خداوند داخل شد.
\par 2 و الیاقیم ناظر خانه و شبنای کاتب و مشایخ کهنه را ملبس به پلاس نزد اشعیا ابن آموص نبی فرستاد.
\par 3 و به وی گفتند: «حزقیا چنین می‌گوید که امروز روزتنگی و تادیب و اهانت است زیرا که پسران بفم رحم رسیده‌اند و قوت زاییدن نیست.
\par 4 شایدیهوه خدایت سخنان ربشاقی را که آقایش پادشاه آشور او را برای اهانت نمودن خدای حی فرستاده است بشنود و سخنانی را که یهوه خدایت شنیده است توبیخ نماید. پس برای بقیه‌ای که یافت می‌شوند تضرع نما.» 
\par 5 و بندگان حزقیا پادشاه نزد اشعیا آمدند.
\par 6 واشعیا به ایشان گفت: «به آقای خود چنین گوییدکه یهوه چنین می‌فرماید: از سخنانی که شنیدی که بندگان پادشاه آشور مرا بدانها کفر گفته اندمترس.
\par 7 همانا روحی بر او می‌فرستم که خبری شنیده، به ولایت خود خواهد برگشت و او را درولایت خودش به شمشیر هلاک خواهم ساخت.»
\par 8 پس ربشاقی مراجعت کرده، پادشاه آشور رایافت که با لبنه جنگ می‌کرد زیرا شنیده بود که ازلاکیش کوچ کرده است.
\par 9 و او درباره ترهاقه پادشاه کوش خبری شنید که به جهت مقاتله با توبیرون آمده است. پس چون این را شنید (باز)ایلچیان نزد حزقیا فرستاده، گفت:
\par 10 «به حزقیاپادشاه یهودا چنین گویید: خدای تو که به او توکل می‌نمایی تو را فریب ندهد و نگوید که اورشلیم به‌دست پادشاه آشور تسلیم نخواهد شد.
\par 11 اینک تو شنیده‌ای که پادشاهان آشور با همه ولایتها چه کرده و چگونه آنها را بالکل هلاک ساخته‌اند و آیاتو رهایی خواهی یافت؟
\par 12 و آیا خدایان امت هایی که پدران من آنها را هلاک ساختند مثل جوزان و حاران و رصف و بنی عدن که در تلسارمی باشند ایشان را نجات دادند.
\par 13 پادشاه حمات کجا است و پادشاه ارفاد و پادشاه شهر سفروایم وهینع و عوا؟»
\par 14 و حزقیا مکتوب را از دست ایلچیان گرفته، آن را خواند و حزقیا به خانه خداوند درآمده، آن را به حضور خداوند پهن کرد.
\par 15 و حزقیا نزدخداوند دعا کرده، گفت:
\par 16 «ای یهوه صبایوت خدای اسرائیل که بر کروبیان جلوس می‌نمایی! تویی که بتنهایی بر تمامی ممالک جهان خداهستی و تو آسمان و زمین را آفریده‌ای.
\par 17 ‌ای خداوند گوش خود را فرا‌گرفته، بشنو و‌ای خداوند چشمان خود را گشوده، ببین و همه سخنان سنحاریب را که به جهت اهانت نمودن خدای حی فرستاده است استماع نما!
\par 18 ‌ای خداوند راست است که پادشاهان آشور همه ممالک و زمین ایشان را خراب کرده.
\par 19 و خدایان ایشان را به آتش انداخته‌اند زیرا که خدا نبودندبلکه ساخته دست انسان از چوب و سنگ. پس به این سبب آنها را تباه ساختند.
\par 20 پس حال‌ای یهوه خدای ما ما را از دست او رهایی ده تا جمیع ممالک جهان بدانند که تو تنها یهوه هستی.»
\par 21 پس اشعیا ابن آموص نزد حزقیا فرستاده، گفت: «یهوه خدای اسرائیل چنین می‌گوید: چونکه درباره سنحاریب پادشاه آشور نزد من دعانمودی،
\par 22 کلامی که خداوند در باره‌اش گفته این است: آن باکره دختر صهیون تو را حقیر شمرده، استهزا نموده است و دختر اورشلیم سر خود را به تو جنبانیده است.
\par 23 کیست که او را اهانت کرده، کفر گفته‌ای و کیست که بر وی آواز بلند کرده، چشمان خود را به علیین افراشته‌ای؟ مگر قدوس اسرائیل نیست؟
\par 24 به واسطه بندگانت خداوند رااهانت کرده، گفته‌ای به کثرت ارابه های خود بربلندی کوهها و به اطراف لبنان برآمده‌ام وبلندترین سروهای آزادش و بهترین صنوبرهایش را قطع نموده، به بلندی اقصایش و به درختستان بوستانش داخل شده‌ام.
\par 25 و من حفره زده، آب نوشیدم و به کف پای خود تمامی نهرهای مصر راخشک خواهم کرد.
\par 26 «آیا نشنیده‌ای که من این را از زمان سلف کرده‌ام و از ایام قدیم صورت داده‌ام و الان آن رابه وقوع آورده‌ام تا تو به ظهور آمده و شهرهای حصار دار را خراب نموده به توده های ویران مبدل سازی.
\par 27 از این جهت ساکنان آنها کم قوت بوده، ترسان و خجل شدند. مثل علف صحرا وگیاه سبز و علف پشت بام و مثل مزرعه قبل از نموکردنش گردیدند.
\par 28 اما من نشستن تو را و خروج و دخولت و خشمی را که بر من داری می‌دانم.
\par 29 چونکه خشمی که به من داری و غرور تو به گوش من برآمده است. بنابراین مهار خود را به بینی تو و لگام خود را به لبهایت گذاشته، تو را به راهی که آمده‌ای برخواهم گردانید.
\par 30 و علامت برای تو این خواهد بود که امسال غله خودروخواهید خورد و سال دوم آنچه از آن بروید و درسال سوم بکارید و بدروید و تاکستانها غرس نموده، میوه آنها را بخورید.
\par 31 و بقیه‌ای که ازخاندان یهودا رستگار شوند بار دیگر به پایین ریشه خواهند زد و به بالا میوه خواهند‌آورد.
\par 32 زیرا که بقیه‌ای از اورشلیم و رستگاران از کوه صهیون بیرون خواهند آمد. غیرت یهوه صبایوت این را بجا خواهد آورد.
\par 33 بنابراین خداوند درباره پادشاه آشور چنین می‌گوید که به این شهرداخل نخواهد شد و به اینجا تیر نخواهد انداخت و در مقابلش با سپر نخواهد آمد و منجنیق درپیش او برنخواهد افراشت.
\par 34 به راهی که آمده است به همان برخواهد گشت و به این شهر داخل نخواهد شد. خداوند این را می‌گوید.
\par 35 زیرا که این شهر را حمایت کرده، به‌خاطر خود و به‌خاطر بنده خویش داود آن را نجات خواهم داد.»
\par 36 پس فرشته خداوند بیرون آمده، صد وهشتاد و پنجهزار نفر از اردوی آشور را زد وبامدادان چون برخاستند اینک جمیع آنهالاشهای مرده بودند.
\par 37 و سنحاریب پادشاه آشور کوچ کرده، روانه گردید و برگشته در نینوی ساکن شد.و واقع شد که چون او در خانه خدای خویش نسروک عبادت می‌کرد، پسرانش ادرملک و شرآصر او را به شمشیر زدند و ایشان به زمین اراراط فرار کردند و پسرش آسرحدون به‌جایش سلطنت نمود.
\par 38 و واقع شد که چون او در خانه خدای خویش نسروک عبادت می‌کرد، پسرانش ادرملک و شرآصر او را به شمشیر زدند و ایشان به زمین اراراط فرار کردند و پسرش آسرحدون به‌جایش سلطنت نمود.
 
\chapter{38}

\par 1 در آن ایام حزقیا بیمار و مشرف به موت شد و اشعیا ابن آموص نبی نزد وی آمده، او را گفت: «خداوند چنین می‌گوید: تدارک خانه خود را ببین زیرا که می‌میری و زنده نخواهی ماند.»
\par 2 آنگاه حزقیا روی خود را بسوی دیواربرگردانیده، نزد خداوند دعا نمود،
\par 3 و گفت: «ای خداوند مستدعی اینکه بیاد آوری که چگونه به حضور تو به امانت و به دل کامل سلوک نموده‌ام وآنچه در نظر تو پسند بوده است بجا آورده‌ام.» پس حزقیا زار‌زار بگریست.
\par 4 و کلام خداوند بر اشعیا نازل شده، گفت:
\par 5 «برو و به حزقیا بگو یهوه خدای پدرت داود چنین می‌گوید: دعای تو راشنیدم و اشکهایت را دیدم. اینک من بر روزهای تو پانزده سال افزودم.
\par 6 و تو را و این شهر را ازدست پادشاه آشور خواهم رهانید و این شهر راحمایت خواهم نمود.
\par 7 و علامت از جانب خداوند که خداوند این کلام را که گفته است بجاخواهد آورد این است:
\par 8 اینک سایه درجاتی که از آفتاب بر ساعت آفتابی آحاز پایین رفته است ده درجه به عقب برمی گردانم.» پس آفتاب ازدرجاتی که بر ساعت آفتابی پایین رفته بود، ده درجه برگشت.
\par 9 مکتوب حزقیا پادشاه یهودا وقتی که بیمارشد و از بیماریش شفا یافت:
\par 10 من گفتم: «اینک در فیروزی ایام خود به درهای هاویه می‌روم و ازبقیه سالهای خود محروم می‌شوم.
\par 11 گفتم: خداوند را مشاهده نمی نمایم. خداوند را در زمین زندگان نخواهم دید. من با ساکنان عالم فنا انسان را دیگر نخواهم دید.
\par 12 خانه من کنده گردید ومثل خیمه شبان از من برده شد. مثل نساج عمرخود را پیچیدم. او مرا از نورد خواهد برید. روز وشب مرا تمام خواهی کرد.
\par 13 تا صبح انتظارکشیدم. مثل شیر همچنین تمامی استخوانهایم رامی شکند. روز و شب مرا تمام خواهی کرد.
\par 14 مثل پرستوک که جیک جیک می‌کند صدامی نمایم. و مانند فاخته ناله می‌کنم و چشمانم ازنگریستن به بالا ضعیف می‌شود. ای خداوند درتنگی هستم کفیل من باش.
\par 15 «چه بگویم چونکه او به من گفته است وخود او کرده است. تمامی سالهای خود را به‌سبب تلخی جانم آهسته خواهم رفت.
\par 16 ‌ای خداوند به این چیزها مردمان زیست می‌کنند و به اینها و بس حیات روح من می‌باشد. پس مرا شفابده و مرا زنده نگاه دار.
\par 17 اینک تلخی سخت من باعث سلامتی من شد. از راه لطف جانم را از چاه هلاکت برآوردی زیرا که تمامی گناهانم را به پشت سر خود انداختی.
\par 18 زیرا که هاویه تو راحمد نمی گوید و موت تو را تسبیح نمی خواند. وآنانی که به حفره فرو می‌روند به امانت تو امیدوارنمی باشند.
\par 19 زندگانند، زندگانند که تو را حمدمی گویند، چنانکه من امروز می‌گویم. پدران به پسران راستی تو را تعلیم خواهند داد.
\par 20 خداوندبه جهت نجات من حاضر است، پس سرودهایم را در تمامی روزهای عمر خود در خانه خداوندخواهیم سرایید.»
\par 21 و اشعیا گفته بود که قرصی از انجیر بگیریدو آن را بر دمل بنهید که شفا خواهد یافت.وحزقیا گفته بود علامتی که به خانه خداوندبرخواهم آمد چیست؟
\par 22 وحزقیا گفته بود علامتی که به خانه خداوندبرخواهم آمد چیست؟
 
\chapter{39}

\par 1 در آن زمان مرودک بلدان بن بلدان پادشاه بابل مکتوبی و هدیه‌ای نزدحزقیا فرستاد زیرا شنیده بود که بیمار شده وصحت یافته است.
\par 2 و حزقیا از ایشان مسرورشده، خانه خزاین خود را از نقره و طلا و عطریات و روغن معطر و تمام خانه اسلحه خویش و هرچه را که در خزاین او یافت می‌شد به ایشان نشان دادو در خانه‌اش و در تمامی مملکتش چیزی نبودکه حزقیا آن را به ایشان نشان نداد.
\par 3 پس اشعیانبی نزد حزقیا پادشاه آمده، وی را گفت: «این مردمان چه گفتند و نزد تو از کجا آمدند؟» حزقیاگفت: «از جای دور یعنی از بابل نزد من آمدند.»
\par 4 او گفت: «در خانه تو چه دیدند؟» حزقیا گفت: «هرچه در خانه من است دیدند و چیزی درخزاین من نیست که به ایشان نشان ندادم.»
\par 5 پس اشعیا به حزقیا گفت: «کلام یهوه صبایوت را بشنو:
\par 6 اینک روزها می‌آید که هرچه در خانه تو است و آنچه پدرانت تا امروز ذخیره کرده‌اند به بابل برده خواهد شد. و خداوندمی گوید که چیزی از آنها باقی نخواهد ماند.
\par 7 وبعضی از پسرانت را که از تو پدید آیند و ایشان راتولید نمایی خواهند گرفت و در قصر پادشاه بابل خواجه‌سرا خواهند شد.»حزقیا به اشعیا گفت: «کلام خداوند که گفتی نیکو است و دیگر گفت: هرآینه در ایام من سلامتی و امان خواهد بود.»
\par 8 حزقیا به اشعیا گفت: «کلام خداوند که گفتی نیکو است و دیگر گفت: هرآینه در ایام من سلامتی و امان خواهد بود.»
 
\chapter{40}

\par 1 تسلی دهید! قوم مرا تسلی دهید! خدای شما می‌گوید:
\par 2 سخنان دلاویزبه اورشلیم گویید و اورا ندا کنید که اجتهاد اوتمام شده و گناه وی آمرزیده گردیده، و از دست خداوند برای تمامی گناهانش دو چندان یافته است.
\par 3 صدای ندا کننده‌ای در بیابان، راه خداوندرا مهیا سازید و طریقی برای خدای ما در صحراراست نمایید.
\par 4 هر دره‌ای برافراشته و هر کوه وتلی پست خواهد شد. و کجیها راست وناهمواریها هموار خواهد گردید.
\par 5 و جلال خداوند مکشوف گشته، تمامی بشر آن را با هم خواهند دید زیرا که دهان خداوند این را گفته است.
\par 6 هاتفی می‌گوید: «ندا کن.» وی گفت: «چه چیز را ندا کنم؟ تمامی بشر گیاه است و همگی زیبایی‌اش مثل گل صحرا.
\par 7 گیاه خشک و گلش پژمرده می‌شود زیرا نفخه خداوند بر آن دمیده می‌شود. البته مردمان گیاه هستند.
\par 8 گیاه خشک شد و گل پژمرده گردید، لیکن کلام خدای ما تاابدالاباد استوار خواهد ماند.»
\par 9 ‌ای صهیون که بشارت می‌دهی به کوه بلند برآی! و‌ای اورشلیم که بشارت می‌دهی آوازت را با قوت بلند کن! آن را بلند کن و مترس و به شهرهای یهودا بگو که «هان خدای شما است!»
\par 10 اینک خداوند یهوه باقوت می‌آید و بازوی وی برایش حکمرانی می‌نماید. اینک اجرت او با وی است و عقوبت وی پیش روی او می‌آید.
\par 11 او مثل شبان گله خود را خواهد چرانید و به بازوی خود بره‌ها راجمع کرده، به آغوش خویش خواهد گرفت وشیر دهندگان را به ملایمت رهبری خواهد کرد.
\par 12 کیست که آبها را به کف دست خود پیموده وافلاک را با وجب اندازه کرده و غبار زمین را درکیل گنجانیده و کوهها را به قپان و تلها را به ترازووزن نموده است؟
\par 13 کیست که روح خداوند را قانون داده یا مشیراو بوده او را تعلیم داده باشد.
\par 14 او از که مشورت خواست تا به او فهم بخشد و طریق راستی را به اوبیاموزد؟ و کیست که او را معرفت آموخت و راه فطانت را به او تعلیم داد؟ 
\par 15 اینک امت‌ها مثل قطره دلو و مانند غبار میزان شمرده می‌شوند. اینک جزیره‌ها را مثل گرد برمی دارد.
\par 16 و لبنان به جهت هیزم کافی نیست و حیواناتش برای قربانی سوختنی کفایت نمی کند.
\par 17 تمامی امت‌ها بنظروی هیچند و از عدم و بطالت نزد وی کمترمی نمایند.
\par 18 پس خدا را به که تشبیه می‌کنید و کدام شبه را با او برابر می‌توانید کرد؟
\par 19 صنعتگربت را می‌ریزد و زرگر آن را به طلا می‌پوشاند، وزنجیرهای نقره برایش می‌ریزد.
\par 20 کسی‌که استطاعت چنین هدایا نداشته باشد درختی را که نمی پوسد اختیار می‌کند و صنعتگر ماهری رامی طلبد تا بتی را که متحرک نشود برای او بسازد.
\par 21 آیا ندانسته ونشنیده‌اید و از ابتدا به شماخبر داده نشده است و از بنیاد زمین نفهمیده‌اید؟
\par 22 او است که بر کره زمین نشسته است وساکنانش مثل ملخ می‌باشند. اوست که آسمانهارا مثل پرده می‌گستراند و آنها را مثل خیمه به جهت سکونت پهن می‌کند.
\par 23 که امیران را لاشی می‌گرداند و داوران جهان را مانند بطالت می‌سازد.
\par 24 هنوز غرس نشده و کاشته نگردیده‌اند و تنه آنها هنوز در زمین ریشه نزده است، که فقط بر آنها می‌دمد و پژمرده می‌شوند وگرد باد آنها را مثل کاه می‌رباید.
\par 25 پس مرا به که تشبیه می‌کنید تا با وی مساوی باشم؟ قدوس می‌گوید:
\par 26 چشمان خودرا به علیین برافراشته ببینید. کیست که اینها راآفرید و کیست که لشکر اینها را بشماره بیرون آورده، جمیع آنها را بنام می‌خواند؟ از کثرت قوت و از عظمت توانایی وی یکی از آنها گم نخواهد شد.
\par 27 ‌ای یعقوب چرا فکر می‌کنی و‌ای اسرائیل چرا می‌گویی: «راه من از خداوند مخفی است و خدای من انصاف مرا از دست داده است.»
\par 28 آیا ندانسته و نشنیده‌ای که خدای سرمدی یهوه آفریننده اقصای زمین درمانده و خسته نمی شود و فهم او را تفحص نتوان کرد؟
\par 29 ضعیفان را قوت می‌بخشد و ناتوانان را قدرت زیاده عطا می‌نماید.
\par 30 حتی جوانان هم درمانده و خسته می‌گردند و شجاعان بکلی می‌افتند.اما آنانی که منتظر خداوند می‌باشند قوت تازه خواهند یافت و مثل عقاب پرواز خواهند کرد. خواهند دوید و خسته نخواهند شد. خواهندخرامید و درمانده نخواهند گردید.
\par 31 اما آنانی که منتظر خداوند می‌باشند قوت تازه خواهند یافت و مثل عقاب پرواز خواهند کرد. خواهند دوید و خسته نخواهند شد. خواهندخرامید و درمانده نخواهند گردید.
 
\chapter{41}

\par 1 ای جزیره‌ها به حضور من خاموش شوید! و قبیله‌ها قوت تازه بهم رسانند! نزدیک بیایند آنگاه تکلم نمایند. با هم به محاکمه نزدیک بیاییم.
\par 2 کیست که کسی را از مشرق برانگیخت که عدالت او را نزد پایهای وی می‌خواند؟ امت‌ها را به وی تسلیم می‌کند و او رابر پادشاهان مسلط می‌گرداند. و ایشان را مثل غبار به شمشیر وی و مثل کاه که پراکنده می‌گرددبه کمان وی تسلیم خواهد نمود.
\par 3 ایشان راتعاقب نموده، به راهی که با پایهای خود نرفته بودبسلامتی خواهد گذشت.
\par 4 کیست که این را عمل نموده و بجا آورده، و طبقات را از ابتدا دعوت نموده است؟ من که یهوه و اول و با آخرین می‌باشم من هستم.
\par 5 جزیره‌ها دیدند و ترسیدند واقصای زمین بلرزیدند و تقرب جسته، آمدند.
\par 6 هر کس همسایه خود را اعانت کرد و به برادرخود گفت: قوی‌دل باش.
\par 7 نجار زرگر را و آنکه باچکش صیقل می‌کند سندان زننده را تقویت می‌نماید و در باره لحیم می‌گوید: که خوب است و آن را به میخها محکم می‌سازد تا متحرک نشود.
\par 8 اما تو‌ای اسرائیل بنده من و‌ای یعقوب که تورا برگزیده‌ام و‌ای ذریت دوست من ابراهیم!
\par 9 که تو را از اقصای زمین گرفته، تو را از کرانه هایش خوانده‌ام و به تو گفته‌ام تو بنده من هستی، تو رابرگزیدم و ترک ننمودم.
\par 10 مترس زیرا که من با توهستم و مشوش مشو زیرا من خدای تو هستم. تورا تقویت خواهم نمود و البته تو را معاونت خواهم داد و تو را به‌دست راست عدالت خوددستگیری خواهم کرد.
\par 11 اینک همه آنانی که برتو خشم دارند خجل و رسوا خواهند شد و آنانی که با تو معارضه نمایند ناچیز شده، هلاک خواهند گردید.
\par 12 آنانی را که با تو مجادله نمایندجستجو کرده، نخواهی یافت و آنانی که با توجنگ کنند نیست و نابود خواهند شد.
\par 13 زیرا من که یهوه خدای تو هستم دست راست تو را گرفته، به تو می‌گویم: مترس زیرا من تو را نصرت خواهم داد.
\par 14 ‌ای کرم یعقوب و شرذمه اسرائیل مترس زیرا خداوند و قدوس اسرائیل که ولی تومی باشد می‌گوید: من تو را نصرت خواهم داد.
\par 15 اینک تو را گردون تیز نو دندانه دار خواهم ساخت و کوهها را پایمال کرده، خرد خواهی نمود و تلها را مثل کاه خواهی ساخت.
\par 16 آنها راخواهی افشاند و باد آنها را برداشته، گردباد آنها راپراکنده خواهد ساخت. لیکن تو از خداوندشادمان خواهی شد و به قدوس اسرائیل فخرخواهی نمود.
\par 17 فقیران و مسکینان آب را می‌جویند ونمی یابند و زبان ایشان از تشنگی خشک می‌شود. من که یهوه هستم ایشان را اجابت خواهم نمود. خدای اسرائیل هستم ایشان را ترک نخواهم کرد.
\par 18 بر تلهای خشک نهرها و در میان وادیهاچشمه‌ها جاری خواهم نمود. و بیابان را به برکه آب و زمین خشک را به چشمه های آب مبدل خواهم ساخت.
\par 19 در بیابان سرو آزاد و شطیم و آس و درخت زیتون را خواهم گذاشت و درصحرا صنوبر و کاج و چنار را با هم غرس خواهم نمود.
\par 20 تا ببینند و بدانند و تفکر نموده، با هم تامل نمایند که دست خداوند این را کرده وقدوس اسرائیل این را ایجاد نموده است.
\par 21 خداوند می‌گوید: دعوی خود را پیش آوریدو پادشاه یعقوب می‌گوید: براهین قوی خویش راعرضه دارید.
\par 22 آنچه را که واقع خواهد شدنزدیک آورده، برای ما اعلام نمایند. چیزهای پیشین را و کیفیت آنها را بیان کنید تا تفکر نموده، آخر آنها را بدانیم یا چیزهای آینده را به مابشنوانید.
\par 23 و چیزها را که بعد از این واقع خواهدشد بیان کنید تا بدانیم که شما خدایانید. باری نیکویی یا بدی را بجا آورید تا ملتفت شده، با هم ملاحظه نماییم.
\par 24 اینک شما ناچیز هستید وعمل شما هیچ است و هر‌که شما را اختیار کندرجس است.
\par 25 کسی را از شمال برانگیختم و او خواهدآمد و کسی را از مشرق آفتاب که اسم مرا خواهدخواند و او بر سروران مثل برگل خواهد آمد ومانند کوزه‌گری که گل را پایمال می‌کند.
\par 26 کیست که از ابتدا خبر داد تا بدانیم و از قبل تابگوییم که او راست است. البته خبر دهنده‌ای نیست و اعلام کننده‌ای نی و کسی هم نیست که سخنان شما را بشنود.
\par 27 اول به صهیون گفتم که اینک هان این چیزها (خواهد رسید) و به اورشلیم بشارت دهنده‌ای بخشیدم.
\par 28 و نگریستم و کسی یافت نشد و در میان ایشان نیز مشورت دهنده‌ای نبود که چون از ایشان سوال نمایم جواب تواندداد.اینک جمیع ایشان باطلند و اعمال ایشان هیچ است و بتهای ریخته شده ایشان باد و بطالت است.
\par 29 اینک جمیع ایشان باطلند و اعمال ایشان هیچ است و بتهای ریخته شده ایشان باد و بطالت است.
 
\chapter{42}

\par 1 اینک بنده من که او را دستگیری نمودم و برگزیده من که جانم از او خشنوداست، من روح خود را بر او می‌نهم تا انصاف رابرای امت‌ها صادر سازد.
\par 2 او فریاد نخواهد زد وآواز خود را بلند نخواهد نمود و آن را درکوچه‌ها نخواهد شنوانید.
\par 3 نی خرد شده رانخواهد شکست و فتیله ضعیف را خاموش نخواهد ساخت تا عدالت را به راستی صادرگرداند.
\par 4 او ضعیف نخواهد گردید و منکسرنخواهد شد تا انصاف را بر زمین قرار دهد وجزیره‌ها منتظر شریعت او باشند.
\par 5 خدا یهوه که آسمانها را آفرید و آنها را پهن کرد و زمین و نتایج آن را گسترانید و نفس را به قومی که در آن باشندو روح را بر آنانی که در آن سالکند می‌دهد چنین می‌گوید:
\par 6 «من که یهوه هستم تو را به عدالت خوانده‌ام و دست تو را گرفته، تو را نگاه خواهم داشت و تو را عهد قوم و نور امت‌ها خواهم گردانید.
\par 7 تا چشمان کوران را بگشایی و اسیران را از زندان و نشینندگان در ظلمت را از محبس بیرون آوری.
\par 8 من یهوه هستم و اسم من همین است. و جلال خود را به کسی دیگر و ستایش خویش را به بتهای تراشیده نخواهم داد.
\par 9 اینک وقایع نخستین واقع شد و من از چیزهای نو اعلام می‌کنم و قبل از آنکه بوجود آید شما را از آنهاخبر می‌دهم.»
\par 10 ‌ای شما که به دریا فرود می‌روید و ای آنچه در آن است! ای جزیره‌ها و ساکنان آنهاسرود نو را به خداوند و ستایش وی را از اقصای زمین بسرایید!
\par 11 صحرا و شهرهایش وقریه هایی که اهل قیدار در آنها ساکن باشند آوازخود را بلند نمایند و ساکنان سالع ترنم نموده، ازقله کوهها نعره زنند!
\par 12 برای خداوند جلال راتوصیف نمایند و تسبیح او را در جزیره هابخوانند!
\par 13 خداوند مثل جبار بیرون می‌آید ومانند مرد جنگی غیرت خویش را برمی انگیزاند. فریاد کرده، نعره خواهد زد و بر دشمنان خویش غلبه خواهد نمود.
\par 14 از زمان قدیم خاموش وساکت مانده، خودداری نمودم. الان مثل زنی که می‌زاید نعره خواهم زد و دم زده آه خواهم کشید.
\par 15 کوهها و تلها را خراب کرده، تمامی گیاه آنها راخشک خواهم ساخت و نهرها را جزایر خواهم گردانید و برکه‌ها را خشک خواهم ساخت.
\par 16 وکوران را به راهی که ندانسته‌اند رهبری نموده، ایشان را به طریق هایی که عارف نیستند هدایت خواهم نمود. ظلمت را پیش ایشان به نور و کجی را به راستی مبدل خواهم ساخت. این کارها را بجاآورده، ایشان را رها نخواهم نمود.
\par 17 آنانی که بربتهای تراشیده اعتماد دارند و به اصنام ریخته شده می‌گویند که خدایان ما شمایید به عقب برگردانیده، بسیار خجل خواهند شد.
\par 18 ‌ای کران بشنوید و‌ای کوران نظر کنید تاببینید.
\par 19 کیست که مثل بنده من کور باشد وکیست که کر باشد مثل رسول من که می‌فرستم؟
\par 20 چیزهای بسیار می‌بینی اما نگاه نمی داری. گوشها را می‌گشاید لیکن خودنمی شنود.
\par 21 خداوند را به‌خاطر عدل خودپسند آمد که شریعت خویش را تعظیم و تکریم نماید.
\par 22 لیکن اینان قوم غارت و تاراج شده‌اند وجمیع ایشان در حفره‌ها صید شده و در زندانهامخفی گردیده‌اند. ایشان غارت شده و رهاننده‌ای نیست. تاراج گشته و کسی نمی گوید که باز ده.
\par 23 کیست در میان شما که به این گوش دهد وتوجه نموده، برای زمان آینده بشنود.
\par 24 کیست که یعقوب را به تاراج و اسرائیل را به غارت تسلیم نمود؟ آیا خداوند نبود که به او گناه ورزیدیم چونکه ایشان به راههای او نخواستندسلوک نمایند و شریعت او را اطاعت ننمودند.بنابراین حدت غضب خود و شدت جنگ را برایشان ریخت و آن ایشان را از هر طرف مشتعل ساخت و ندانستند و ایشان را سوزانید اما تفکرننمودند.
\par 25 بنابراین حدت غضب خود و شدت جنگ را برایشان ریخت و آن ایشان را از هر طرف مشتعل ساخت و ندانستند و ایشان را سوزانید اما تفکرننمودند.
 
\chapter{43}

\par 1 و الان خداوند که آفریننده تو‌ای یعقوب، و صانع تو‌ای اسرائیل است چنین می‌گوید: «مترس زیرا که من تو را فدیه دادم و تو را به اسمت خواندم پس تو از آن من هستی.
\par 2 چون از آبها بگذری من با تو خواهم بود و چون از نهرها (عبورنمایی ) تو را فرونخواهند گرفت. وچون از میان آتش روی، سوخته نخواهی شد وشعله‌اش تو را نخواهد سوزانید.
\par 3 زیرا من یهوه خدای تو و قدوس اسرائیل نجات‌دهنده توهستم. مصر را فدیه تو ساختم و حبش و سبا را به عوض تو دادم.
\par 4 چونکه در نظر من گرانبها و مکرم بودی و من تو را دوست می‌داشتم پس مردمان را به عوض تو و طوایف را در عوض جان تو تسلیم خواهم نمود.
\par 5 مترس زیرا که من با توهستم و ذریت تو را از مشرق خواهم آورد و تو رااز مغرب جمع خواهم نمود.
\par 6 به شمال خواهم گفت که "بده " و به جنوب که "ممانعت مکن". پسران مرا از جای دور و دخترانم را از کرانهای زمین بیاور.
\par 7 یعنی هر‌که را به اسم من نامیده شودو او را به جهت جلال خویش آفریده و او رامصور نموده و ساخته باشم.» 
\par 8 قومی را که چشم دارند اما کور هستند و گوش دارند اما کرمی باشند بیرون آور.
\par 9 جمیع امت‌ها با هم جمع شوند و قبیله‌ها فراهم آیند پس در میان آنهاکیست که از این خبر دهد و امور اولین را به مااعلام نماید. شهود خود را بیاورند تا تصدیق شوند یا استماع نموده، اقرار بکنند که این راست است.
\par 10 یهوه می‌گوید که «شما و بنده من که او رابرگزیده‌ام شهود من می‌باشید. تا دانسته، به من ایمان آورید و بفهمید که من او هستم و پیش از من خدایی مصور نشده و بعد از من هم نخواهد شد.
\par 11 من، من یهوه هستم و غیر از من نجات‌دهنده‌ای نیست.
\par 12 من اخبار نموده و نجات داده‌ام و اعلام نموده و درمیان شما خدای غیر نبوده است. خداوند می‌گوید که شما شهود من هستید و من خدا هستم.
\par 13 و از امروز نیز من او می‌باشم وکسی‌که از دست من تواند رهانید نیست. من عمل خواهم نمود و کیست که آن را رد نماید؟»
\par 14 خداوند که ولی شما و قدوس اسرائیل است چنین می‌گوید: «بخاطر شما به بابل فرستادم و همه ایشان را مثل فراریان فرود خواهم آورد وکلدانیان را نیز در کشتیهای وجد ایشان.
\par 15 من خداوند قدوس شما هستم. آفریننده اسرائیل و پادشاه شما.»
\par 16 خداوند که راهی در دریا و طریقی درآبهای عظیم می‌سازد چنین می‌گوید:
\par 17 «آنکه ارابه‌ها و اسبها و لشکر و قوت آن را بیرون می‌آورد، ایشان با هم خواهند خوابید و نخواهندبرخاست و منطفی شده، مثل فتیله خاموش خواهند شد.
\par 18 چیزهای اولین را بیاد نیاورید ودر امور قدیم تفکر ننمایید.
\par 19 اینک من چیزنویی بوجود می‌آورم و آن الان بظهور می‌آید. آیاآن را نخواهید دانست؟ بدرستی که راهی دربیابان و نهرها در هامون قرار خواهم داد.
\par 20 حیوانات صحرا گرگان و شترمرغها مرا تمجیدخواهند نمود چونکه آب در بیابان و نهرها درصحرا بوجود می‌آورم تا قوم خود و برگزیدگان خویش را سیراب نمایم.
\par 21 این قوم را برای خودایجاد کردم تا تسبیح مرا بخوانند.
\par 22 اما تو‌ای یعقوب مرا نخواندی و تو‌ای اسرائیل از من به تنگ آمدی!
\par 23 گوسفندان قربانی های سوختنی خود را برای من نیاوردی و به ذبایح خود مراتکریم ننمودی! به هدایا بندگی بر تو ننهادم و به بخور تو را به تنگ نیاوردم.
\par 24 نی معطر را به جهت من به نقره نخریدی و به پیه ذبایح خویش مرا سیر نساختی. بلکه به گناهان خود بر من بندگی نهادی و به خطایای خویش مرا به تنگ آوردی.
\par 25 من هستم من که بخاطر خود خطایای تو رامحو ساختم و گناهان تو را بیاد نخواهم آورد.
\par 26 مرا یاد بده تا با هم محاکمه نماییم. حجت خودرا بیاور تا تصدیق شوی.
\par 27 اجداد اولین تو گناه ورزیدند و واسطه های تو به من عاصی شدند.بنابراین من سروران قدس را بی‌احترام خواهم ساخت و یعقوب را به لعنت و اسرائیل را به دشنام تسلیم خواهم نمود.
\par 28 بنابراین من سروران قدس را بی‌احترام خواهم ساخت و یعقوب را به لعنت و اسرائیل را به دشنام تسلیم خواهم نمود.
 
\chapter{44}

\par 1 اما الان‌ای بنده من یعقوب بشنو و‌ای اسرائیل که تو را برگزیده‌ام!
\par 2 خداوندکه تو را آفریده و تو را از رحم بسرشت و معاون تو می‌باشد چنین می‌گوید: ای بنده من یعقوب مترس! و‌ای یشرون که تو را برگزیده‌ام!
\par 3 اینک بر(زمین ) تشنه آب خواهم ریخت و نهرهابرخشک. روح خود را بر ذریت تو خواهم ریخت و برکت خویش را بر اولاد تو.
\par 4 و ایشان در میان سبزه‌ها، مثل درختان بید بر جویهای آب خواهندرویید.
\par 5 یکی خواهد گفت که من از آن خداوندهستم و دیگری خویشتن را به نام یعقوب خواهدخواند و دیگری بدست خود برای خداوندخواهد نوشت و خویشتن را به نام اسرائیل ملقب خواهد ساخت.
\par 6 خداوند پادشاه اسرائیل و یهوه صبایوت که ولی ایشان است چنین می‌گوید: من اول هستم ومن آخر هستم و غیر از من خدایی نیست.
\par 7 و مثل من کیست که آن را اعلان کند و بیان نماید و آن راترتیب دهد، از زمانی که قوم قدیم را برقرارنمودم. پس چیزهای آینده و آنچه را که واقع خواهد شد اعلان بنمایند.
\par 8 ترسان و هراسان مباشید. آیا از زمان قدیم تو را اخبار و اعلام ننمودم و آیا شما شهود من نیستید؟ آیا غیر از من خدایی هست؟ البته صخره‌ای نیست و احدی را نمی شناسم.»
\par 9 آنانی که بتهای تراشیده می‌سازندجمیع باطلند و چیزهایی که ایشان می‌پسندندفایده‌ای ندارد و شهود ایشان نمی بینند ونمی دانند تا خجالت بکشند.
\par 10 کیست که خدایی ساخته یا بتی ریخته باشد که نفعی ندارد؟
\par 11 اینک جمیع یاران او خجل خواهند شد وصنعتگران از انسان می‌باشند. پس جمیع ایشان جمع شده، بایستند تا با هم ترسان و خجل گردند.
\par 12 آهن را با تیشه می‌تراشد و آن را در زغال کار می‌کند و با چکش صورت می‌دهد و با قوت بازوی خویش آن را می‌سازد و نیز گرسنه شده، بی‌قوت می‌گردد و آب ننوشیده، ضعف بهم می‌رساند.
\par 13 چوب را می‌تراشد و ریسمان راکشیده، با قلم آن را نشان می‌کند و با رنده آن راصاف می‌سازد و با پرگار نشان می‌کند پس آن را به شبیه انسان و به جمال آدمی می‌سازد تا در خانه ساکن شود.
\par 14 سروهای آزاد برای خود قطع می‌کند و سندیان و بلوط را گرفته، آنها را ازدرختان جنگل برای خود اختیار می‌کند وشمشاد را غرس نموده، باران آن را نمو می‌دهد.
\par 15 پس برای شخص به جهت سوخت بکارمی آید و از آن گرفته، خود را گرم می‌کند و آن راافروخته نان می‌پزد و خدایی ساخته، آن رامی پرستد و از آن بتی ساخته، پیش آن سجده می‌کند.
\par 16 بعضی از آن را در آتش می‌سوزاند وبر بعضی گوشت پخته می‌خورد و کباب را برشته کرده، سیر می‌شود و گرم شده، می‌گوید: وه گرم شده، آتش را دیدم.
\par 17 و از بقیه آن خدایی یعنی بت خویش را می‌سازد و پیش آن سجده کرده، عبادت می‌کند و نزد آن دعا نموده، می‌گوید: مرا نجات بده چونکه تو خدای من هستی.
\par 18 ایشان نمی دانند و نمی فهمند زیرا که چشمان ایشان رابسته است تا نبینند و دل ایشان را تا تعقل ننمایند.
\par 19 و تفکر ننموده، معرفت و فطانتی ندارند تابگویند نصف آن را در آتش سوختیم و بر زغالش نیز نان پختیم و گوشت را کباب کرده، خوردیم پس آیا از بقیه آن بتی بسازیم و به تنه درخت سجده نماییم؟
\par 20 خاکستر را خوراک خودمی سازد و دل فریب خورده او را گمراه می‌کند که جان خود را نتواند رهانید و فکر نمی نماید که آیادر دست راست من دروغ نیست.
\par 21 «ای یعقوب و‌ای اسرائیل اینها را بیادآورچونکه تو بنده من هستی. تو را سرشتم‌ای اسرائیل تو بنده من هستی از من فراموش نخواهی شد.
\par 22 تقصیرهای تو را مثل ابر غلیظ و گناهانت را مانند ابر محو ساختم. پس نزد من بازگشت نمازیرا تو را فدیه کرده‌ام.
\par 23 ‌ای آسمانها ترنم نمایید زیرا که خداوند این را کرده است! و‌ای اسفلهای زمین! فریاد برآورید و‌ای کوهها وجنگلها و هر درختی که در آنها باشد بسرایید! زیرا خداوند یعقوب را فدیه کرده است وخویشتن را در اسرائیل تمجید خواهد نمود.»
\par 24 خداوند که ولی تو است و تو را از رحم سرشته است چنین می‌گوید: من یهوه هستم وهمه‌چیز را ساختم. آسمانها را به تنهایی گسترانیدم و زمین را پهن کردم و با من که بود.
\par 25 آنکه آیات کاذبان را باطل می‌سازد وجادوگران را احمق می‌گرداند. و حکیمان رابعقب برمی گرداند و علم ایشان را به جهالت تبدیل می‌کند.
\par 26 که سخنان بندگان خود را برقرار می‌دارد و مشورت رسولان خویش را به انجام می‌رساند، که درباره اورشلیم می‌گویدمعمور خواهد شد و درباره شهرهای یهودا که بناخواهد شد و خرابی های آن را برپا خواهم داشت.
\par 27 آنکه به لجه می‌گوید که خشک شو ونهرهایت را خشک خواهم ساخت.و درباره کورش می‌گوید که او شبان من است و تمامی مسرت مرا به اتمام خواهد رسانید و درباره اورشلیم می‌گوید بنا خواهد شد و درباره هیکل که بنیاد تو نهاده خواهد گشت.»
\par 28 و درباره کورش می‌گوید که او شبان من است و تمامی مسرت مرا به اتمام خواهد رسانید و درباره اورشلیم می‌گوید بنا خواهد شد و درباره هیکل که بنیاد تو نهاده خواهد گشت.»
 
\chapter{45}

\par 1 خداوند به مسیح خویش یعنی به کورش که دست راست او را گرفتم تا به حضور وی امت‌ها را مغلوب سازم و کمرهای پادشاهان را بگشایم تا درها را به حضور وی مفتوح نمایم و دروازه‌ها دیگر بسته نشود چنین می‌گوید
\par 2 که «من پیش روی تو خواهم خرامید وجایهای ناهموار را هموار خواهم ساخت. ودرهای برنجین را شکسته، پشت بندهای آهنین راخواهم برید.
\par 3 و گنجهای ظلمت و خزاین مخفی را به تو خواهم بخشید تا بدانی که من یهوه که تو رابه اسمت خوانده‌ام خدای اسرائیل می‌باشم.
\par 4 به‌خاطر بنده خود یعقوب و برگزیده خویش اسرائیل، هنگامی که مرا نشناختی تو را به اسمت خواندم و ملقب ساختم.
\par 5 من یهوه هستم ودیگری نیست و غیر از من خدایی نی. من کمر تورا بستم هنگامی که مرا نشناختی.
\par 6 تا از مشرق آفتاب و مغرب آن بدانند که سوای من احدی نیست. من یهوه هستم و دیگری نی.
\par 7 پدیدآورنده نور و آفریننده ظلمت. صانع سلامتی وآفریننده بدی. من یهوه صانع همه این چیزها هستم.
\par 8 ‌ای آسمانها از بالا ببارانید تا افلاک عدالت را فرو ریزد و زمین بشکافد تا نجات وعدالت نمو کنند و آنها را با هم برویاند زیرا که من یهوه این را آفریده‌ام.
\par 9 وای برکسی‌که با صانع خود چون سفالی با سفالهای زمین مخاصمه نماید. آیا کوزه به کوزه‌گر بگوید چه چیز راساختی؟ یا مصنوع تو درباره تو بگوید که اودست ندارد؟
\par 10 وای بر کسی‌که به پدر خودگوید: چه چیز را تولید نمودی و به زن که چه زاییدی.
\par 11 خداوند که قدوس اسرائیل و صانع آن می‌باشد چنین می‌گوید: درباره امور آینده ازمن سوال نمایید و پسران مرا و اعمال دستهای مرابه من تفویض نمایید.
\par 12 من زمین را ساختم وانسان را بر آن آفریدم. دستهای من آسمانها راگسترانید و من تمامی لشکرهای آنها را امرفرمودم.
\par 13 من او را به عدالت برانگیختم و تمامی راههایش را راست خواهم ساخت. شهر مرا بناکرده، اسیرانم را آزاد خواهد نمود، اما نه برای قیمت و نه برای هدیه. یهوه صبایوت این رامی گوید.»
\par 14 خداوند چنین می‌گوید: «حاصل مصر وتجارت حبش و اهل سبا که مردان بلند قدمی باشند نزد تو عبور نموده، از آن تو خواهندبود. و تابع تو شده در زنجیرها خواهند آمد وپیش تو خم شده و نزد تو التماس نموده، خواهندگفت: البته خدا در تو است و دیگری نیست وخدایی نی.»
\par 15 ‌ای خدای اسرائیل و نجات‌دهنده یقین خدایی هستی که خود را پنهان می‌کنی.
\par 16 جمیع ایشان خجل و رسوا خواهندشد و آنانی که بتها می‌سازند با هم به رسوایی خواهند رفت.
\par 17 اما اسرائیل به نجات جاودانی از خداوند ناجی خواهند شد و تا ابدالاباد خجل و رسوا نخواهند گردید.
\par 18 زیرا یهوه آفریننده آسمان که خدا است که زمین را سرشت و ساخت و آن را استوار نمود و آن را عبث نیافرید بلکه به جهت سکونت مصور نمود چنین می‌گوید: «من یهوه هستم و دیگری نیست.
\par 19 در خفا و درجایی از زمین تاریک تکلم ننمودم. و به ذریه یعقوب نگفتم که مرا عبث بطلبید. من یهوه به عدالت سخن می‌گویم و چیزهای راست را اعلان می‌نمایم.
\par 20 ‌ای رهاشدگان از امت‌ها جمع شده، بیایید و با هم نزدیک شوید. آنانی که چوب بتهای خود را برمی دارند و نزد خدایی که نتواند رهانیددعا می‌نمایند معرفت ندارند.
\par 21 پس اعلان نموده، ایشان را نزدیک آورید تا با یکدیگرمشورت نمایند. کیست که این را از ایام قدیم اعلان نموده و از زمان سلف اخبار کرده است؟ آیا نه من که یهوه هستم و غیر از من خدایی دیگرنیست؟ خدای عادل و نجات‌دهنده و سوای من نیست.
\par 22 ‌ای جمیع کرانه های زمین به من توجه نمایید و نجات یابید زیرا من خدا هستم و دیگری نیست.
\par 23 به ذات خود قسم خوردم و این کلام به عدالت از دهانم صادر گشته برنخواهد گشت که هر زانو پیش من خم خواهد شد و هر زبان به من قسم خواهد خورد.و مرا خواهند گفت عدالت و قوت فقط در خداوند می‌باشد» و بسوی او خواهند آمد و همگانی که به او خشمناکندخجل خواهند گردید. و تمامی ذریت اسرائیل در خداوند عادل شمرده شده، فخر خواهندکرد. 
\par 24 و مرا خواهند گفت عدالت و قوت فقط در خداوند می‌باشد» و بسوی او خواهند آمد و همگانی که به او خشمناکندخجل خواهند گردید. و تمامی ذریت اسرائیل در خداوند عادل شمرده شده، فخر خواهندکرد.
 
\chapter{46}

\par 1 بیل خم شده و نبو منحنی گردیده بتهای آنها بر حیوانات و بهایم نهاده شد. آنهایی که شما برمی داشتید حمل گشته و بارحیوانات ضعیف شده است.
\par 2 آنها جمیع منحنی و خم شده، آن بار را نمی توانند رهانیدبلکه خود آنها به اسیری می‌روند.
\par 3 ‌ای خاندان یعقوب و تمامی بقیه خاندان اسرائیل که از بطن برمن حمل شده و از رحم برداشته من بوده‌اید!
\par 4 وتا به پیری شما من همان هستم و تا به شیخوخیت من شما را خواهم برداشت. من آفریدم و من برخواهم داشت و من حمل کرده، خواهم رهانید.
\par 5 مرا با که شبیه و مساوی می‌سازید و مرا با که مقابل می‌نمایید تا مشابه شویم؟
\par 6 آنانی که طلا رااز کیسه می‌ریزند و نقره را به میزان می‌سنجند، زرگری را اجیر می‌کنند تا خدایی از آن بسازدپس سجده کرده، عبادت نیز می‌نمایند.
\par 7 آن را بردوش برداشته، می‌برند و به‌جایش می‌گذارند و اومی ایستد و از جای خود حرکت نمی تواند کرد. نزد او استغاثه هم می‌نمایند اما جواب نمی دهد وایشان را از تنگی ایشان نتواند رهانید.
\par 8 این را بیاد آورید و مردانه بکوشید. و‌ای عاصیان آن را در دل خود تفکر نمایید!
\par 9 چیزهای اول را از زمان قدیم به یاد آورید. زیرا من قادرمطلق هستم و دیگری نیست. خدا هستم و نظیرمن نی.
\par 10 آخر را از ابتدا و آنچه را که واقع نشده از قدیم بیان می‌کنم و می‌گویم که اراده من برقرارخواهد ماند و تمامی مسرت خویش را بجاخواهم آورد.
\par 11 مرغ شکاری را از مشرق و هم مشورت خویش را از جای دور می‌خوانم. من گفتم و البته بجا خواهم آورد و تقدیر نمودم والبته به وقوع خواهم رسانید.
\par 12 ‌ای سختدلان که از عدالت دور هستید مرا بشنوید.عدالت خودرا نزدیک آوردم و دور نمی باشد و نجات من تاخیر نخواهد نمود و نجات را به جهت اسرائیل که جلال من است در صهیون خواهم گذاشت.
\par 13 عدالت خودرا نزدیک آوردم و دور نمی باشد و نجات من تاخیر نخواهد نمود و نجات را به جهت اسرائیل که جلال من است در صهیون خواهم گذاشت.
 
\chapter{47}

\par 1 ای باکره دختر بابل فرود شده، بر خاک بنشین و‌ای دختر کلدانیان بر زمین بی‌کرسی بنشین زیرا تو را دیگر نازنین و لطیف نخواهند خواهند.
\par 2 دستاس را گرفته، آرد را خردکن. نقاب را برداشته، دامنت را بر کش و ساقها رابرهنه کرده، از نهرها عبور کن.
\par 3 عورت تو کشف شده، رسوایی تو ظاهر خواهد شد. من انتقام کشیده، بر احدی شفقت نخواهم نمود.
\par 4 و امانجات‌دهنده ما اسم او یهوه صبایوت و قدوس اسرائیل می‌باشد.
\par 5 ‌ای دختر کلدانیان خاموش بنشین و به ظلمت داخل شو زیرا که دیگر تو راملکه ممالک نخواهند خواند.
\par 6 بر قوم خود خشم نموده و میراث خویش را بی‌حرمت کرده، ایشان را به‌دست تو تسلیم نمودم. بر ایشان رحمت نکرده، یوغ خود را بر پیران بسیار سنگین ساختی.
\par 7 و گفتی تا به ابد ملکه خواهم بود. و این چیزها رادر دل خود جا ندادی و عاقبت آنها را به یادنیاوردی.
\par 8 پس الان‌ای که در عشرت بسر می‌بری و دراطمینان ساکن هستی این را بشنو. ای که در دل خود می‌گویی من هستم و غیر از من دیگری نیست و بیوه نخواهم شد و بی‌اولادی را نخواهم دانست.
\par 9 پس این دو چیز یعنی بی‌اولادی وبیوگی بغته در یکروز به تو عارض خواهد شد و باوجود کثرت سحر تو و افراط افسونگری زیاد توآنها بشدت بر تو استیلا خواهد یافت.
\par 10 زیرا که بر شرارت خود اعتماد نموده، گفتی کسی نیست که مرا بیند. و حکمت و علم تو، تو را گمراه ساخت و در دل خود گفتی: من هستم و غیر از من دیگری نیست.
\par 11 پس بلایی که افسون آن رانخواهی دانست بر تو عارض خواهد شد ومصیبتی که به دفع آن قادر نخواهی شد تو را فروخواهد گرفت و هلاکتی که ندانسته‌ای ناگهان برتو استیلا خواهد یافت.
\par 12 پس در افسونگری خود و کثرت سحر خویش که در آنها ازطفولیت مشقت کشیده‌ای قائم باش شاید که منفعت توانی برد و شاید که غالب توانی شد.
\par 13 از فراوانی مشورتهایت خسته شده‌ای پس تقسیم کنندگان افلاک و رصد بندان کواکب وآنانی که در غره ماهها اخبار می‌دهند بایستند وتو را از آنچه بر تو واقع شدنی است نجات دهند.
\par 14 اینک مثل کاهبن شده، آتش ایشان را خواهدسوزانید که خویشتن را از سورت زبانه آن نخواهند رهانید و آن اخگری که خود را نزد آن گرم سازند و آتشی که در برابرش بنشینندنخواهد بود.آنانی که از ایشان مشقت کشیدی برای تو چنین خواهند شد و آنانی که از طفولیت با تو تجارت می‌کردند هر کس بجای خود آواره خواهد گردید و کسی‌که تو را رهایی دهدنخواهد بود.
\par 15 آنانی که از ایشان مشقت کشیدی برای تو چنین خواهند شد و آنانی که از طفولیت با تو تجارت می‌کردند هر کس بجای خود آواره خواهد گردید و کسی‌که تو را رهایی دهدنخواهد بود.
 
\chapter{48}

\par 1 ای خاندان یعقوب که به نام اسرائیل مسمی هستید و از آب یهودا صادرشده‌اید، و به اسم یهوه قسم می‌خورید و خدای اسرائیل را ذکر می‌نمایید، اما نه به صداقت وراستی، این را بشنوید.
\par 2 زیرا که خویشتن را ازشهر مقدس می‌خوانند و بر خدای اسرائیل که اسمش یهوه صبایوت است اعتماد می‌دارند.
\par 3 چیزهای اول را از قدیم اخبار کردم و از دهان من صادر شده، آنها را اعلام نمودم بغته به عمل آوردم و واقع شد.
\par 4 چونکه دانستم که تو سخت دل هستی و گردنت بند آهنین و پیشانی تو برنجین است.
\par 5 بنابراین تو را از قدیم مخبر ساختم و قبل از وقوع تو را اعلام نمودم. مبادا بگویی که بت من آنها را بجا آورده و بت تراشیده و صنم ریخته شده من آنها را امر فرموده است.
\par 6 چونکه همه این چیزها را شنیدی آنها را ملاحظه نما. پس آیاشما اعتراف نخواهید کرد، و از این زمان چیزهای تازه را به شما اعلام نمودم و چیزهای مخفی را که ندانسته بودید.
\par 7 در این زمان و نه در ایام قدیم آنها آفریده شد و قبل از امروز آنها را نشنیده بودی مبادا بگویی اینک این چیزها را می‌دانستم.
\par 8 البته نشنیده و هر آینه ندانسته و البته گوش توقبل از این باز نشده بود. زیرا می‌دانستم که بسیارخیانتکار هستی و از رحم (مادرت ) عاصی خوانده شدی.
\par 9 به‌خاطر اسم خود غضب خویش را به تاخیر خواهم‌انداخت و به‌خاطرجلال خویش بر تو شفقت خواهم کرد تا تو رامنقطع نسازم.
\par 10 اینک تو را قال گذاشتم اما نه مثل نقره و تو را در کوره مصیبت آزمودم.
\par 11 به خاطر ذات خود، به‌خاطر ذات خود این رامی کنم زیرا که اسم من چرا باید بی‌حرمت شود وجلال خویش را به دیگری نخواهم داد.
\par 12 ‌ای یعقوب و‌ای دعوت شده من اسرائیل بشنو! من او هستم! من اول هستم و آخر هستم!
\par 13 به تحقیق دست من بنیاد زمین را نهاد و دست راست من آسمانها را گسترانید. وقتی که آنها رامی خوانم با هم برقرار می‌باشند.
\par 14 پس همگی شما جمع شده، بشنوید کیست از ایشان که اینهارا اخبار کرده باشد. خداوند او را دوست می‌دارد، پس مسرت خود را بر بابل بجا خواهد آورد وبازوی او بر کلدانیان فرود خواهد آمد.
\par 15 من تکلم نمودم و من او را خواندم و او را آوردم تا راه خود را کامران سازد.
\par 16 به من نزدیک شده، این رابشنوید. از ابتدا در خفا تکلم ننمودم و از زمانی که این واقع شد من در آنجا هستم و الان خداوندیهوه مرا و روح خود را فرستاده است.
\par 17 خداوندکه ولی تو و قدوس اسرائیل است چنین می‌گوید: «من یهوه خدای تو هستم و تو را تعلیم می‌دهم تاسود ببری و تو را به راهی که باید بروی هدایت می‌نمایم.
\par 18 کاش که به اوامر من گوش می‌دادی، آنگاه سلامتی تو مثل نهر و عدالت تو مانند امواج دریا می‌بود.
\par 19 آنگاه ذریت تو مثل ریگ و ثمره صلب تو مانند ذرات آن می‌بود و نام او از حضورمن منقطع و هلاک نمی گردید.»
\par 20 از بابل بیرون شده، از میان کلدانیان بگریزید و این را به آواز ترنم اخبار و اعلام نماییدو آن را تا اقصای زمین شایع ساخته، بگویید که خداوند بنده خود یعقوب را فدیه داده است.
\par 21 وتشنه نخواهند شد اگر‌چه ایشان را در ویرانه هارهبری نماید، زیرا که آب از صخره برای ایشان جاری خواهد ساخت و صخره را خواهدشکافت تا آبها بجوشد.و خداوند می‌گوید که برای شریران سلامتی نخواهد بود.
\par 22 و خداوند می‌گوید که برای شریران سلامتی نخواهد بود.
 
\chapter{49}

\par 1 ای جزیره‌ها از من بشنوید! و‌ای طوایف از جای دور گوش دهید! خداوند مرا از رحم دعوت کرده واز احشای مادرم اسم مرا ذکر نموده است.
\par 2 و دهان مرا مثل شمشیر تیز ساخته، مرا زیر سایه دست خودپنهان کرده است. و مرا تیر صیقلی ساخته درترکش خود مخفی نموده است.
\par 3 و مرا گفت: ای اسرائیل تو بنده من هستی که از تو خویشتن راتمجید نموده‌ام!
\par 4 اما من گفتم که عبث زحمت کشیدم و قوت خود را بی‌فایده و باطل صرف کردم لیکن حق من با خداوند و اجرت من با خدای من می‌باشد.
\par 5 و الان خداوند که مرا از رحم برای بندگی خویش سرشت تا یعقوب را نزد اوبازآورم و تا اسرائیل نزد وی جمع شوند می‌گوید(و در نظر خداوند محترم هستم و خدای من قوت من است ).
\par 6 پس می‌گوید: این چیز قلیلی است که بنده من بشوی تا اسباط یعقوب را برپاکنی و ناجیان اسرائیل را باز آوری. بلکه تو را نورامت‌ها خواهم گردانید و تا اقصای زمین نجات من خواهی بود.
\par 7 خداوند که ولی و قدوس اسرائیل می‌باشد به او که نزد مردم محقر و نزد امت ها مکروه و بنده حاکمان است چنین می‌گوید: پادشاهان دیده برپا خواهند شد و سروران سجده خواهند نمود، به‌سبب خداوند که امین است وقدوس اسرائیل که تو را برگزیده است.
\par 8 خداوند چنین می‌گوید: «در زمان رضامندی تو را اجابت نمودم و در روز نجات تو را اعانت کردم. و تو را حفظ نموده عهد قوم خواهم ساخت تا زمین را معمور سازی و نصیب های خراب شده را (به ایشان ) تقسیم نمایی.
\par 9 و به اسیران بگویی: بیرون روید و به آنانی که درظلمتند خویشتن را ظاهر سازید. و ایشان درراهها خواهند چرید و مرتعهای ایشان بر همه صحراهای کوهی خواهد بود.
\par 10 گرسنه و تشنه نخواهند بود و حرارت و آفتاب به ایشان ضررنخواهد رسانید زیرا آنکه بر ایشان ترحم داردایشان را هدایت خواهد کرد و نزد چشمه های آب ایشان را رهبری خواهد نمود.
\par 11 و تمامی کوههای خود را طریق‌ها خواهم ساخت وراههای من بلند خواهد شد.
\par 12 اینک بعضی ازجای دور خواهند آمد و بعضی از شمال و ازمغرب و بعضی از دیار سینیم.»
\par 13 ‌ای آسمانهاترنم کنید! و‌ای زمین وجد نما! و‌ای کوهها آوازشادمانی دهید! زیرا خداوند قوم خود را تسلی می‌دهد و بر مظلومان خود ترحم می‌فرماید.
\par 14 اما صهیون می‌گوید: «یهوه مرا ترک نموده و خداوند مرا فراموش کرده است.»
\par 15 آیا زن بچه شیر خواره خود را فراموش کرده بر پسر رحم خویش ترحم ننماید؟ اینان فراموش می‌کنند امامن تو را فراموش نخواهم نمود.
\par 16 اینک تو را برکف دستهای خود نقش نمودم و حصارهایت دائم در نظر من است.
\par 17 پسرانت به تعجیل خواهند آمد و آنانی که تو را خراب و ویران کردند از تو بیرون خواهند رفت.
\par 18 چشمان خودرا به هر طرف بلند کرده، ببین جمیع اینها جمع شده، نزد تو می‌آیند. خداوند می‌گوید: «به حیات خودم قسم که خود را به جمیع اینها مثل زیورملبس خواهی ساخت و مثل عروس خویشتن رابه آنها خواهی آراست.
\par 19 زیرا خرابه‌ها وویرانه های تو و زمین تو که تباه شده بود، اما الان تو از کثرت ساکنان تنگ خواهی شد و هلاک کنندگانت دور خواهند گردید.
\par 20 پسران تو که بی‌اولاد می‌بودی در سمع تو (به یکدیگر)خواهند گفت: این مکان برای من تنگ است، مراجایی بده تا ساکن شوم.
\par 21 و تو دردل خودخواهی گفت: کیست که اینها را برای من زاییده است و حال آنکه من بی‌اولاد و نازاد و جلای وطن و متروک می‌بودم. پس کیست که اینها راپرورش داد. اینک من به تنهایی ترک شده بودم پس اینها کجا بودند؟» 
\par 22 خداوند یهوه چنین می‌گوید: «اینک من دست خود را بسوی امت هادراز خواهم کرد و علم خویش را بسوی قوم هاخواهم برافراشت. و ایشان پسرانت را درآغوش خود خواهند‌آورد و دخترانت بر دوش ایشان برداشته خواهند شد.
\par 23 و پادشاهان لالاهای تو و ملکه های ایشان دایه های توخواهند بود و نزد تو رو به زمین افتاده، خاک پای تو را خواهند لیسید و تو خواهی دانست که من یهوه هستم و آنانی که منتظر من باشند خجل نخواهند گردید.
\par 24 آیا غنیمت از جبار گرفته شود یا اسیران ازمرد قاهر رهانیده گردند.
\par 25 زیرا خداوند چنین می‌گوید: «اسیران نیز از جبار گرفته خواهند شد وغنیمت از دست ستم پیشه رهانیده خواهدگردید. زیرا که من با دشمنان تو مقاومت خواهم نمود و من پسران تو را نجات خواهم داد.و به آنانی که بر تو ظلم نمایند گوشت خودشان راخواهم خورانید و به خون خود مثل شراب مست خواهند شد و تمامی بشر خواهند دانست که من یهوه نجات‌دهنده تو و ولی تو و قدیر یعقوب هستم.»
\par 26 و به آنانی که بر تو ظلم نمایند گوشت خودشان راخواهم خورانید و به خون خود مثل شراب مست خواهند شد و تمامی بشر خواهند دانست که من یهوه نجات‌دهنده تو و ولی تو و قدیر یعقوب هستم.»
 
\chapter{50}

\par 1 خداوند چنین می‌گوید: «طلاق نامه مادر شما که او را طلاق دادم کجااست؟ یا کیست از طلبکاران من که شما را به اوفروختم؟ اینک شما به‌سبب گناهان خود فروخته شدید و مادر شما به جهت تقصیرهای شما طلاق داده شد.
\par 2 چون آمدم چرا کسی نبود؟ و چون ندا کردم چرا کسی جواب نداد؟ آیا دست من به هیچ وجه کوتاه شده که نتواند نجات دهد یا درمن قدرتی نیست که رهایی دهم؟ اینک به عتاب خود دریا را خشک می‌کنم و نهرها رابیابان می‌سازم که ماهی آنها از بی‌آبی متعفن شود و از تشنگی بمیرد.
\par 3 آسمان را به ظلمت ملبس می‌سازم و پلاس را پوشش آن می‌گردانم.»
\par 4 خداوند یهوه زبان تلامیذ را به من داده است تا بدانم که چگونه خستگان را به کلام تقویت دهم. هر بامداد بیدار می‌کند. گوش مرا بیدارمی کند تا مثل تلامیذ بشنوم.
\par 5 خداوند یهوه گوش مرا گشود و مخالفت نکردم و به عقب برنگشتم.
\par 6 پشت خود را به زنندگان و رخسارخود را به موکنان دادم و روی خود را از رسوایی و آب دهان پنهان نکردم.
\par 7 چونکه خداوند یهوه مرا اعانت می‌کند پس رسوا نخواهم شد از این جهت روی خود را مثل سنگ خارا ساختم ومی دانم که خجل نخواهم گردید.
\par 8 آنکه مراتصدیق می‌کند نزدیک است. پس کیست که با من مخاصمه نماید تا با هم بایستیم و کیست که بر من دعوی نماید پس او نزدیک من بیاید.
\par 9 اینک خداوند یهوه مرا اعانت می‌کند پس کیست مراملزم سازد. همانا همگی ایشان مثل رخت مندرس شده، بید ایشان را خواهد خورد.
\par 10 کیست از شما که از خداوند می‌ترسد وآواز بنده او را می‌شنود؟ هرکه در ظلمت سالک باشد و روشنایی ندارد، او به اسم یهوه توکل نماید و به خدای خویش اعتماد بکند.هان جمیع شما که آتش می‌افروزید و کمر خود را به مشعلها می‌بندید، در روشنایی آتش خویش و درمشعلهایی که خود افروخته‌اید سالک باشید، امااین از دست من به شما خواهد رسید که در اندوه خواهید خوابید.
\par 11 هان جمیع شما که آتش می‌افروزید و کمر خود را به مشعلها می‌بندید، در روشنایی آتش خویش و درمشعلهایی که خود افروخته‌اید سالک باشید، امااین از دست من به شما خواهد رسید که در اندوه خواهید خوابید.
 
\chapter{51}

\par 1 ای پیروان عدالت و طالبان خداوند مرابشنوید! به صخره‌ای که از آن قطع گشته و به حفره چاهی که از آن کنده شده‌اید نظرکنید.
\par 2 به پدر خود ابراهیم و به ساره که شما رازایید نظر نمایید زیرا او یک نفر بود حینی که او رادعوت نمودم و او را برکت داده، کثیر گردانیدم.
\par 3 به تحقیق خداوند صهیون را تسلی داده، تمامی خرابه هایش را تسلی بخشیده است و بیابان او رامثل عدن و هامون او را مانند جنت خداوندساخته است. خوشی و شادی در آن یافت می‌شود و تسبیح و آواز ترنم.
\par 4 ‌ای قوم من به من توجه نمایید و‌ای طایفه من به من گوش دهید. زیرا که شریعت از نزد من صادرمی شود و داوری خود را برقرار می‌کنم تا قوم هارا روشنایی بشود.
\par 5 عدالت من نزدیک است ونجات من ظاهر شده، بازوی من قوم‌ها را داوری خواهد نمود و جزیره‌ها منتظر من شده، به بازوی من اعتماد خواهند کرد.
\par 6 چشمان خود را بسوی آسمان برافرازید و پایین بسوی زمین نظر کنیدزیرا که آسمان مثل دود از هم خواهد پاشید وزمین مثل جامه مندرس خواهد گردید وساکنانش همچنین خواهند مرد اما نجات من تا به ابد خواهد ماند و عدالت من زایل نخواهد گردید.
\par 7 ‌ای شما که عدالت را می‌شناسید! و‌ای قومی که شریعت من در دل شما است! مرابشنوید. از مذمت مردمان مترسید و از دشنام ایشان هراسان مشوید.
\par 8 زیرا که بید ایشان را مثل جامه خواهد زد و کرم ایشان را مثل پشم خواهدخورد اما عدالت من تا ابدالاباد و نجات من نسلابعد نسل باقی خواهد ماند.
\par 9 بیدار شو‌ای بازوی خداوند بیدار شو و خویشتن را با قوت ملبس ساز. مثل ایام قدیم و دوره های سلف بیدار شو. آیا تو آن نیستی که رهب را قطع نموده، اژدها رامجروح ساختی.
\par 10 آیا تو آن نیستی که دریا وآبهای لجه عظیم را خشک کردی و عمق های دریا را راه ساختی تا فدیه شدگان عبور نمایند؟
\par 11 و فدیه شدگان خداوند بازگشت نموده، با ترنم به صهیون خواهند آمد و خوشی جاودانی بر سرایشان خواهد بود و شادمانی و خوشی را خواهندیافت و غم و ناله فرار خواهد کرد.
\par 12 من هستم، من که شما را تسلی می‌دهم. پس تو کیستی که از انسانی که می‌میرد می‌ترسی و ازپسر آدم که مثل گیاه خواهد گردید.
\par 13 و خداوندرا که آفریننده تو است که آسمانها را گسترانید وبنیاد زمین را نهاد فراموش کرده‌ای و دائم تمامی روز از خشم ستمکار وقتی که به جهت هلاک کردن مهیا می‌شود می‌ترسی. و خشم ستمکارکجا است؟
\par 14 اسیران ذلیل بزودی رها خواهندشد و در حفره نخواهند مرد و نان ایشان کم نخواهد شد.
\par 15 زیرا من یهوه خدای تو هستم که دریا را به تلاطم می‌آورم تا امواجش نعره زنند، یهوه صبایوت اسم من است.
\par 16 و من کلام خود رادر دهان تو گذاشتم و تو را زیر سایه دست خویش پنهان کردم تا آسمانها غرس نمایم و بنیادزمینی نهم و صهیون را گویم که تو قوم من هستی.
\par 17 خویشتن را برانگیز‌ای اورشلیم! خویشتن را برانگیخته، برخیز! ای که از دست خداوندکاسه غضب او را نوشیدی و درد کاسه سرگیجی را نوشیده، آن را تا ته آشامیدی.
\par 18 از جمیع پسرانی که زاییده است یکی نیست که او رارهبری کند و از تمامی پسرانی که تربیت نموده، کسی نیست که او را دستگیری نماید.
\par 19 این دوبلا بر تو عارض خواهد شد، کیست که برای توماتم کند؟ یعنی خرابی و هلاکت و قحط وشمشیر، پس چگونه تو را تسلی دهم.
\par 20 پسران تو را ضعف گرفته، بسر همه کوچه‌ها مثل آهو دردام خوابیده‌اند. و ایشان از غضب خداوند و ازعتاب خدای تو مملو شده‌اند.
\par 21 پس‌ای زحمت کشیده این را بشنو! و‌ای مست شده اما نه ازشراب!
\par 22 خداوند تو یهوه و خدای تو که دردعوی قوم خود ایستادگی می‌کند چنین می‌گوید: اینک کاسه سرگیجی را و درد کاسه غضب خویش را از تو خواهم گرفت و آن را بار دیگرنخواهی آشامید.و آن را به‌دست آنانی که برتو ستم می‌نمایند می‌گذارم که به‌جان تومی گویند: خم شو تا از تو بگذریم و تو پشت خودرا مثل زمین و مثل کوچه به جهت راه گذریان گذاشته‌ای.
\par 23 و آن را به‌دست آنانی که برتو ستم می‌نمایند می‌گذارم که به‌جان تومی گویند: خم شو تا از تو بگذریم و تو پشت خودرا مثل زمین و مثل کوچه به جهت راه گذریان گذاشته‌ای.
 
\chapter{52}

\par 1 بیدار شو‌ای صهیون! بیدار شو و قوت خود را بپوش‌ای شهر مقدس اورشلیم! لباس زیبایی خویش را در بر کن زیرا که نامختون و ناپاک بار دیگر داخل تو نخواهد شد.
\par 2 ‌ای اورشلیم خود را از گرد بیفشان و برخاسته، بنشین! و‌ای دختر صهیون که اسیر شده‌ای بندهای گردن خود را بگشا!
\par 3 زیرا خداوند چنین می‌گوید: مفت فروخته کشتید و بی‌نقره فدیه داده خواهیدشد.
\par 4 چونکه خداوند یهوه چنین می‌گوید: که درایام سابق قوم من به مصر فرود شدند تا در آنجاساکن شوند و بعد از آن آشور بر ایشان بی‌سبب ظلم نمودند.
\par 5 اما الان خداوند می‌گوید: در اینجامرا چه‌کار است که قوم من مجان گرفتار شده‌اند. و خداوند می‌گوید: آنانی که بر ایشان تسلط دارندصیحه می‌زنند و نام من دائم هر روز اهانت می‌شود.
\par 6 بنابراین قوم من اسم مرا خواهندشناخت. و در آن روز خواهند فهمید که تکلم کننده من هستم، هان من هستم.
\par 7 چه زیبا است بر کوهها پایهای مبشر که سلامتی را ندا می‌کند و به خیرات بشارت می‌دهدو نجات را ندا می‌کند و به صهیون می‌گوید که خدای تو سلطنت می‌نماید.
\par 8 آواز دیده بانان تواست که آواز خود را بلند کرده، با هم ترنم می‌نمایند زیرا وقتی که خداوند به صهیون رجعت می‌کند ایشان معاینه خواهند دید.
\par 9 ‌ای خرابه های اورشلیم به آواز بلند با هم ترنم نمایید، زیرا خداوند قوم خود را تسلی داده، و اورشلیم را فدیه نموده است.
\par 10 خداوند ساعد قدوس خود را در نظر تمامی امت‌ها بالا زده است وجمیع کرانه های زمین نجات خدای ما را دیده‌اند.
\par 11 ‌ای شما که ظروف خداوند را برمی داریدبیکسو شوید بیکسو شوید و از اینجا بیرون رویدو چیز ناپاک را لمس منمایید و از میان آن بیرون رفته، خویشتن را طاهر سازید.
\par 12 زیرا که به تعجیل بیرون نخواهید رفت و گریزان روانه نخواهید شد، چونکه یهوه پیش روی شماخواهد خرامید و خدای اسرائیل ساقه شماخواهد بود.
\par 13 اینک بنده من با عقل رفتار خواهد کرد وعالی و رفیع و بسیار بلند خواهد شد.
\par 14 چنانکه بسیاری از تو در عجب بودند (از آن جهت که منظر او از مردمان و صورت او از بنی آدم بیشترتباه گردیده بود).همچنان بر امت های بسیارخواهد پاشید و به‌سبب او پادشاهان دهان خود را خواهند بست زیرا چیزهایی را که برای ایشان بیان نشده بود خواهند دید و آنچه را که نشنیده بودند خواهند فهمید.
\par 15 همچنان بر امت های بسیارخواهد پاشید و به‌سبب او پادشاهان دهان خود را خواهند بست زیرا چیزهایی را که برای ایشان بیان نشده بود خواهند دید و آنچه را که نشنیده بودند خواهند فهمید.
 
\chapter{53}

\par 1 کیست که خبر ما را تصدیق نموده وکیست که ساعد خداوند بر او منکشف شده باشد؟
\par 2 زیرا به حضور وی مثل نهال و مانندریشه در زمین خشک خواهد رویید. او را نه صورتی و نه جمالی می‌باشد. و چون او رامی نگریم منظری ندارد که مشتاق او باشیم.
\par 3 خوار و نزد مردمان مردود و صاحب غمها ورنج دیده و مثل کسی‌که رویها را از او بپوشانند وخوار شده که او را به حساب نیاوردیم.
\par 4 لکن او غم های ما را بر خود گرفت و دردهای ما را بر خویش حمل نمود. و ما او را از جانب خدا زحمت کشیده و مضروب و مبتلا گمان بردیم.
\par 5 و حال آنکه به‌سبب تقصیرهای مامجروح و به‌سبب گناهان ما کوفته گردید. وتادیب سلامتی ما بر وی آمد و از زخمهای او ماشفا یافتیم.
\par 6 جمیع ما مثل گوسفندان گمراه شده بودیم و هریکی از ما به راه خود برگشته بود وخداوند گناه جمیع ما را بروی نهاد.
\par 7 او مظلوم شد اما تواضع نموده، دهان خود رانگشود. مثل بره‌ای که برای ذبح می‌برند و مانندگوسفندی که نزد پشم برنده‌اش بی‌زبان است همچنان دهان خود را نگشود.
\par 8 از ظلم و ازداوری گرفته شد. و از طبقه او که تفکر نمود که اواز زمین زندگان منقطع شد و به جهت گناه قوم من مضروب گردید؟
\par 9 و قبر او را با شریران تعیین نمودند و بعد از مردنش با دولتمندان. هرچند هیچ ظلم نکرد و در دهان وی حیله‌ای نبود.
\par 10 اما خداوند را پسند آمد که او را مضروب نموده، به دردها مبتلا سازد. چون جان او راقربانی گناه ساخت. آنگاه ذریت خود را خواهددید و عمر او دراز خواهد شد و مسرت خداونددر دست او میسر خواهد بود.
\par 11 ثمره مشقت جان خویش را خواهد دید و سیر خواهد شد. وبنده عادل من به معرفت خود بسیاری را عادل خواهد گردانید زیرا که او گناهان ایشان را برخویشتن حمل خواهد نمود.بنابراین او را درمیان بزرگان نصیب خواهم داد و غنیمت را بازورآوران تقسیم خواهد نمود، به جهت اینکه جان خود را به مرگ ریخت و از خطاکاران محسوب شد و گناهان بسیاری را بر خود گرفت وبرای خطاکاران شفاعت نمود. 
\par 12 بنابراین او را درمیان بزرگان نصیب خواهم داد و غنیمت را بازورآوران تقسیم خواهد نمود، به جهت اینکه جان خود را به مرگ ریخت و از خطاکاران محسوب شد و گناهان بسیاری را بر خود گرفت وبرای خطاکاران شفاعت نمود.
 
\chapter{54}

\par 1 ای عاقره‌ای که نزاییده‌ای بسرا! ای که درد زه نکشیده‌ای به آواز بلند ترنم نماو فریاد برآور! زیرا خداوند می‌گوید: پسران زن بیکس از پسران زن منکوحه زیاده‌اند.
\par 2 مکان خیمه خود را وسیع گردان و پرده های مسکن های تو پهن بشود دریغ مدار و طنابهای خود را درازکرده، میخهایت را محکم بساز.
\par 3 زیرا که بطرف راست و چپ منتشر خواهی شد و ذریت توامت‌ها را تصرف خواهند نمود و شهرهای ویران را مسکون خواهند ساخت.
\par 4 مترس زیرا که خجل نخواهی شد و مشوش مشو زیرا که رسوانخواهی گردید. چونکه خجالت جوانی خویش را فراموش خواهی کرد و عار بیوگی خود را دیگربه یاد نخواهی آورد.
\par 5 زیرا که آفریننده تو که اسمش یهوه صبایوت است شوهر تو است. وقدوس اسرائیل که به خدای تمام جهان مسمی است ولی تو می‌باشد.
\par 6 زیرا خداوند تو را مثل زن مهجور و رنجیده دل خوانده است و مانندزوجه جوانی که ترک شده باشد. خدای تو این رامی گوید.
\par 7 زیرا تو را به اندک لحظه‌ای ترک کردم اما به رحمت های عظیم تو را جمع خواهم نمود.
\par 8 و خداوند ولی تو می‌گوید: «بجوشش غضبی خود را از تو برای لحظه‌ای پوشانیدم اما به احسان جاودانی برتو رحمت خواهم فرمود.
\par 9 زیرا که این برای من مثل آبهای نوح می‌باشد. چنانکه قسم خوردم که آبهای نوح بار دیگر بر زمین جاری نخواهد شد همچنان قسم خوردم که بر توغضب نکنم و تو را عتاب ننمایم.
\par 10 هرآینه کوهها زایل خواهد شد و تلها متحرک خواهدگردید، لیکن احسان من از تو زایل نخواهد شد وعهد سلامتی من متحرک نخواهد گردید.» خداوند که بر تو رحمت می‌کند این را می‌گوید.
\par 11 «ای رنجانیده و مضطرب شده که تسلی نیافته‌ای اینک من سنگهای تو را در سنگ سرمه نصب خواهم کرد و بنیاد تو را در یاقوت زردخواهم نهاد.
\par 12 و مناره های تو را از لعل ودروازه هایت را از سنگهای بهرمان و تمامی حدود تو را از سنگهای گران قیمت خواهم ساخت.
\par 13 و جمیع پسرانت از خداوند تعلیم خواهند یافت و پسرانت را سلامتی عظیم خواهدبود.
\par 14 در عدالت ثابت شده و از ظلم دور مانده، نخواهی ترسید و هم از آشفتگی دور خواهی ماند و به تو نزدیکی نخواهد نمود.
\par 15 همانا جمع خواهند شد اما نه به اذن من. آنانی که به ضد توجمع شوند به‌سبب تو خواهند افتاد.
\par 16 اینک من آهنگری را که زغال را به آتش دمیده، آلتی برای کار خود بیرون می‌آورد، آفریدم. و من نیز هلاک کننده را برای خراب نمودن آفریدم.هر آلتی که به ضد تو ساخته شود پیش نخواهد برد و هرزبانی را که برای محاکمه به ضد تو برخیزدتکذیب خواهی نمود. این است نصیب بندگان خداوند و عدالت ایشان از جانب من.» خداوندمی گوید.
\par 17 هر آلتی که به ضد تو ساخته شود پیش نخواهد برد و هرزبانی را که برای محاکمه به ضد تو برخیزدتکذیب خواهی نمود. این است نصیب بندگان خداوند و عدالت ایشان از جانب من.» خداوندمی گوید.
 
\chapter{55}

\par 1 «ای جمیع تشنگان نزد آبها بیایید وهمه شما که نقره ندارید بیایید بخرید وبخورید. بیایید و شراب و شیر را بی‌نقره وبی قیمت بخرید.
\par 2 چرا نقره را برای آنچه نان نیست و مشقت خویش را برای آنچه سیرنمی کند صرف می‌کنید. گوش داده، از من بشنویدو چیزهای نیکو را بخورید تا جان شما از فربهی متلذذ شود.
\par 3 گوش خود را فرا داشته، نزد من بیایید و تا جان شما زنده گردد بشنوید و من باشما عهد جاودانی یعنی رحمت های امین داود راخواهم بست.
\par 4 اینک من او را برای طوایف شاهدگردانیدم. رئیس و حاکم طوایف.
\par 5 هان امتی را که نشناخته بودی دعوت خواهی نمود. و امتی که تورا نشناخته بودند نزد تو خواهند دوید. به‌خاطریهوه که خدای تو است و قدوس اسرائیل که تو راتمجید نموده است.»
\par 6 خداوند را مادامی که یافت می‌شود بطلبید ومادامی که نزدیک است او را بخوانید.
\par 7 شریر راه خود را و گناه کار افکار خویش را ترک نماید وبسوی خداوند بازگشت کند و بر وی رحمت خواهد نمود و بسوی خدای ما که مغفرت عظیم خواهد کرد.
\par 8 زیرا خداوند می‌گوید که افکار من افکار شما نیست و طریق های شما طریق های من نی.
\par 9 زیرا چنانکه آسمان از زمین بلندتر است همچنان طریق های من از طریق های شما و افکارمن از افکار شما بلندتر می‌باشد.
\par 10 و چنانکه باران و برف از آسمان می‌بارد و به آنجابرنمی گردد بلکه زمین را سیراب کرده، آن را بارورو برومند می‌سازد و برزگر را تخم و خورنده رانان می‌بخشد،
\par 11 همچنان کلام من که از دهانم صادر گردد خواهد بود. نزد من بی‌ثمر نخواهدبرگشت بلکه آنچه را که خواستم بجا خواهدآورد و برای آنچه آن را فرستادم کامران خواهدگردید.
\par 12 زیرا که شما با شادمانی بیرون خواهیدرفت و با سلامتی هدایت خواهید شد. کوهها وتلها در حضور شما به شادی ترنم خواهند نمود وجمیع درختان صحرا دستک خواهند زد.به‌جای درخت خار صنوبر و به‌جای خس آس خواهد رویید و برای خداوند اسم و آیت جاودانی که منقطع نشود خواهد بود.
\par 13 به‌جای درخت خار صنوبر و به‌جای خس آس خواهد رویید و برای خداوند اسم و آیت جاودانی که منقطع نشود خواهد بود.
 
\chapter{56}

\par 1 و خداوند چنین می‌گوید: «انصاف رانگاه داشته، عدالت را جاری نمایید، زیرا که آمدن نجات من و منکشف شدن عدالت من نزدیک است.
\par 2 خوشابحال انسانی که این را بجا آورد و بنی آدمی که به این متمسک گردد، که سبت را نگاه داشته، آن را بی‌حرمت نکند و دست خویش را از هر عمل بد باز‌دارد.»
\par 3 پس غریبی که با خداوند مقترن شده باشد تکلم نکند و نگویدکه خداوند مرا از قوم خود جدا نموده است وخصی هم نگوید که اینک من درخت خشک هستم.
\par 4 زیرا خداوند درباره خصیهایی که سبت های مرا نگاه دارند و آنچه را که من خوش دارم اختیار نمایند و به عهد من متمسک گردند، چنین می‌گوید
\par 5 که «به ایشان در خانه خود و دراندرون دیوارهای خویش یادگاری و اسمی بهتراز پسران و دختران خواهم داد. اسمی جاودانی که منقطع نخواهد شد به ایشان خواهم بخشید.
\par 6 و غریبانی که با خداوند مقترن شده، او راخدمت نمایند و اسم خداوند را دوست داشته، بنده او بشوند. یعنی همه کسانی که سبت را نگاه داشته، آن را بی‌حرمت نسازند و به عهد من متمسک شوند.
\par 7 ایشان را به کوه قدس خودخواهم آورد و ایشان را در خانه عبادت خودشادمان خواهم ساخت و قربانی های سوختنی وذبایح ایشان بر مذبح من قبول خواهد شد، زیراخانه من به خانه عبادت برای تمامی قوم‌ها مسمی خواهد شد.»
\par 8 و خداوند یهوه که جمع کننده رانده شدگان اسرائیل است می‌گوید که «بعد از این دیگران را با ایشان جمع خواهم کرد علاوه برآنانی که از ایشان جمع شده‌اند.»
\par 9 ‌ای تمام حیوانات صحرا و‌ای جمیع حیوانات جنگل بیایید و بخورید!
\par 10 دیده بانان اوکورند، جمیع ایشان معرفت ندارند و همگی ایشان سگان گنگ‌اند که نمی توانند بانگ کنند.
\par 11 و این سگان حریصند که نمی توانندسیر بشوند و ایشان شبانند که نمی توانندبفهمند. جمیع ایشان به راه خود میل کرده، هریکی بطرف خود طالب سود خویش می‌باشد.(و می‌گویند) بیایید شراب بیاوریم و ازمسکرات مست شویم و فردا مثل امروز روزعظیم بلکه بسیار زیاده خواهد بود.
\par 12 (و می‌گویند) بیایید شراب بیاوریم و ازمسکرات مست شویم و فردا مثل امروز روزعظیم بلکه بسیار زیاده خواهد بود.
 
\chapter{57}

\par 1 مرد عادل تلف شد و کسی نیست که این را در دل خود بگذراند و مردان روف برداشته شدند و کسی فکر نمی کند که عادلان ازمعرض بلا برداشته می‌شوند.
\par 2 آنانی که به استقامت سالک می‌باشند بسلامتی داخل شده، بربسترهای خویش آرامی خواهند یافت.
\par 3 و اماشما‌ای پسران ساحره و اولاد فاسق و زانیه به اینجا نزدیک آیید!
\par 4 بر که تمسخر می‌کنید و برکه دهان خود را باز می‌کنید و زبان را درازمی نمایید؟ آیا شما اولاد عصیان و ذریت کذب نیستید
\par 5 که در میان بلوطها و زیر هر درخت سبزخویشتن را به حرارت می‌آورید و اطفال را دروادیها زیر شکاف صخره‌ها ذبح می‌نمایید؟
\par 6 در میان سنگهای ملسای وادی نصیب تواست همینها قسمت تو می‌باشد. برای آنها نیزهدیه ریختنی ریختی و هدیه آردی گذرانیدی آیا من از اینها تسلی خواهم یافت؟
\par 7 بر کوه بلند ورفیع بستر خود را گستردی و به آنجا نیز برآمده، قربانی گذرانیدی.
\par 8 و پشت درها و باهوها یادگارخود را واگذاشتی زیرا که خود را به کسی دیگرغیر از من مکشوف ساختی و برآمده، بستر خودرا پهن کردی و درمیان خود و ایشان عهد بسته، بستر ایشان را دوست داشتی جایی که آن رادیدی.
\par 9 و با روغن در حضور پادشاه رفته، عطریات خود را بسیار کردی و رسولان خود رابجای دور فرستاده، خویشتن را تا به هاویه پست گردانیدی.
\par 10 از طولانی بودن راه درمانده شدی اما نگفتی که امید نیست. تازگی قوت خود رایافتی پس از این جهت ضعف بهم نرسانیدی.
\par 11 از که ترسان و هراسان شدی که خیانت ورزیدی و مرا بیاد نیاورده، این را در دل خود جاندادی؟ آیا من از زمان قدیم نیز ساکت نماندم پس از این جهت از من نترسیدی؟
\par 12 من عدالت واعمال تو را بیان خواهم نمود که تو را منفعت نخواهد داد.
\par 13 چون فریاد برمی آوری اندوخته های خودت تو را خلاصی بدهد و لکن باد جمیع آنها را خواهد برداشت و نفس آنها راخواهد برد. اما هر‌که بر من توکل دارد مالک زمین خواهد بود و وارث جبل قدس من خواهد گردید.
\par 14 و گفته خواهد شد برافرازید! راه را برافرازید ومهیا سازید! و سنگ مصادم را از طریق قوم من بردارید!
\par 15 زیرا او که عالی و بلند است و ساکن درابدیت می‌باشد و اسم او قدوس است چنین می‌گوید: «من در مکان عالی و مقدس ساکنم و نیزبا کسی‌که روح افسرده و متواضع دارد. تا روح متواضعان را احیا نمایم و دل افسردگان را زنده سازم.
\par 16 زیرا که تا به ابد مخاصمه نخواهم نمودو همیشه خشم نخواهم کرد مبادا روحها وجانهایی که من آفریدم به حضور من ضعف به هم رسانند.
\par 17 به‌سبب گناه طمع وی غضبناک شده، او را زدم و خود را مخفی ساخته، خشم نمودم و او به راه دل خود رو گردانیده، برفت.
\par 18 طریق های او را دیدم و او را شفا خواهم داد واو را هدایت نموده، به او و به آنانی که با وی ماتم گیرند تسلی بسیار خواهم داد.»
\par 19 خداوند که آفریننده ثمره لبها است می‌گوید: «بر آنانی که دورند سلامتی باد و بر آنانی که نزدیکند سلامتی باد و من ایشان را شفا خواهم بخشید.»
\par 20 اماشریران مثل دریای متلاطم که نمی تواند آرام گیرد و آبهایش گل و لجن برمی اندازد می‌باشند.خدای من می‌گوید که شریران را سلامتی نیست.
\par 21 خدای من می‌گوید که شریران را سلامتی نیست.
 
\chapter{58}

\par 1 آواز خود را بلند کن و دریغ مدار و آوازخود را مثل کرنا بلند کرده، به قوم من تقصیر ایشان را و به خاندان یعقوب گناهان ایشان را اعلام نما.
\par 2 و ایشان هر روز مرا می‌طلبند و ازدانستن طریق های من مسرور می‌باشند. مثل امتی که عدالت را بجا آورده، حکم خدای خود را ترک ننمودند. احکام عدالت را از من سوال نموده، ازتقرب جستن به خدا مسرور می‌شوند
\par 3 (ومی گویند): چرا روزه داشتیم و ندیدی و جانهای خویش را رنجانیدیم و ندانستی. اینک شما درروز روزه خویش خوشی خود را می‌یابید و برعمله های خود ظلم می‌نمایید.
\par 4 اینک به جهت نزاع و مخاصمه روزه می‌گیرید و به لطمه شرارت می‌زنید. «امروز روزه نمی گیرید که آواز خود رادر اعلی علیین بشنوانید.
\par 5 آیا روزه‌ای که من می‌پسندم مثل این است، روزی که آدمی جان خود را برنجاند و سر خود را مثل نی خم ساخته، پلاس و خاکستر زیر خود بگستراند؟ آیا این راروزه و روز مقبول خداوند می‌خوانی؟
\par 6 «مگر روزه‌ای که من می‌پسندم این نیست که بندهای شرارت را بگشایید و گره های یوغ را بازکنید و مظلومان را آزاد سازید و هر یوغ رابشکنید؟ 
\par 7 مگر این نیست که نان خود را به گرسنگان تقسیم نمایی و فقیران رانده شده را به خانه خود بیاوری و چون برهنه را ببینی او رابپوشانی و خود را از آنانی که از گوشت تومی باشند مخفی نسازی؟
\par 8 آنگاه نور تو مثل فجرطالع خواهد شد و صحت تو بزودی خواهدرویید و عدالت تو پیش تو خواهد خرامید وجلال خداوند ساقه تو خواهد بود.
\par 9 آنگاه دعاخواهی کرد و خداوند تو را اجابت خواهد فرمودو استغاثه خواهی نمود و او خواهد گفت که اینک حاضر هستم. اگر یوغ و اشاره کردن به انگشت وگفتن ناحق را از میان خود دور کنی،
\par 10 و آرزوی جان خود را به گرسنگان ببخشی و جان ذلیلان راسیر کنی، آنگاه نور تو در تاریکی خواهددرخشید و تاریکی غلیظ تو مثل ظهر خواهد بود.
\par 11 و خداوند تو را همیشه هدایت نموده، جان تورا در مکان های خشک سیر خواهد کرد واستخوانهایت را قوی خواهد ساخت و تو مثل باغ سیرآب و مانند چشمه آب که آبش کم نشودخواهی بود.
\par 12 و کسان تو خرابه های قدیم را بناخواهند نمود و تو اساسهای دوره های بسیار رابرپا خواهی داشت و تو را عمارت کننده رخنه هاو مرمت کننده کوچه‌ها برای سکونت خواهندخواند.
\par 13 «اگر پای خود را از سبت نگاه داری وخوشی خود را در روز مقدس من بجا نیاوری وسبت را خوشی و مقدس خداوند و محترم بخوانی و آن را محترم داشته، به راههای خودرفتار ننمایی و خوشی خود را نجویی و سخنان خود را نگویی،آنگاه در خداوند متلذذخواهی شد و تو را بر مکان های بلند زمین سوارخواهم کرد. و نصیب پدرت یعقوب را به توخواهم خورانید» زیرا که دهان خداوند این راگفته است.
\par 14 آنگاه در خداوند متلذذخواهی شد و تو را بر مکان های بلند زمین سوارخواهم کرد. و نصیب پدرت یعقوب را به توخواهم خورانید» زیرا که دهان خداوند این راگفته است.
 
\chapter{59}

\par 1 هان دست خداوند کوتاه نیست تانرهاند و گوش او سنگین نی تا نشنود.
\par 2 لیکن خطایای شما در میان شما و خدای شماحایل شده است و گناهان شما روی او را از شماپوشانیده است تا نشنود.
\par 3 زیرا که دستهای شمابه خون و انگشتهای شما به شرارت آلوده شده است. لبهای شما به دروغ تکلم می‌نماید وزبانهای شما به شرارت تنطق می‌کند.
\par 4 احدی به عدالت دعوی نمی کند و هیچکس به راستی داوری نمی نماید. به بطالت توکل دارند و به دروغ تکلم می‌نمایند. به ظلم حامله شده، شرارت رامی زایند.
\par 5 از تخمهای افعی بچه برمی آورند وپرده عنکبوت می‌بافند. هرکه از تخمهای ایشان بخورد می‌میرد و آن چون شکسته گردد افعی بیرون می‌آید.
\par 6 پرده های ایشان لباس نخواهدشد و خویشتن را از اعمال خود نخواهند پوشانیدزیرا که اعمال ایشان اعمال شرارت است و عمل ظلم در دستهای ایشان است.
\par 7 پایهای ایشان برای بدی دوان و به جهت ریختن خون بی‌گناهان شتابان است. افکار ایشان افکار شرارت است ودر راههای ایشان ویرانی و خرابی است.
\par 8 طریق سلامتی را نمی دانند و در راههای ایشان انصاف نیست. جاده های کج برای خود ساخته‌اند و هر‌که در آنها سالک باشد سلامتی را نخواهد دانست.
\par 9 بنابراین انصاف از ما دور شده است و عدالت به ما نمی رسد. انتظار نور می‌کشیم و اینک ظلمت است و منتظر روشنایی هستیم اما در تاریکی غلیظ سالک می‌باشیم.
\par 10 و مثل کوران برای دیوار تلمس می‌نماییم و مانند بی‌چشمان کورانه راه می‌رویم. در وقت ظهر مثل شام لغزش می‌خوریم و در میان تندرستان مانند مردگانیم.
\par 11 جمیع ما مثل خرسها صدا می‌کنیم و مانندفاخته‌ها ناله می‌نماییم، برای انصاف انتظارمی کشیم و نیست و برای نجات و از ما دورمی شود.
\par 12 زیرا که خطایای ما به حضور تو بسیار شده و گناهان ما به ضد ما شهادت می‌دهد چونکه خطایای ما با ما است و گناهان خود را می‌دانیم.
\par 13 مرتد شده، خداوند را انکار نمودیم. از پیروی خدای خود انحراف ورزیدیم به ظلم و فتنه تکلم کردیم و به سخنان کذب حامله شده، از دل آنها راتنطق نمودیم.
\par 14 و انصاف به عقب رانده شده وعدالت از ما دور ایستاده است زیرا که راستی درکوچه‌ها افتاده است و استقامت نمی تواند داخل شود.
\par 15 و راستی مفقود شده است و هر‌که ازبدی اجتناب نماید خود را به یغما می‌سپارد. وچون خداوند این را دید در نظر او بد آمد که انصاف وجود نداشت.
\par 16 و او دید که کسی نبود وتعجب نمود که شفاعت کننده‌ای وجود نداشت از این جهت بازوی وی برای او نجات آورد وعدالت او وی را دستگیری نمود.
\par 17 پس عدالت را مثل زره پوشید و خود نجات را بر سر خویش نهاد. و جامه انتقام را به‌جای لباس در بر کرد وغیرت را مثل ردا پوشید.
\par 18 بر وفق اعمال ایشان، ایشان را جزا خواهد داد. به خصمان خود حدت خشم را و به دشمنان خویش مکافات و به جزایرپاداش را خواهد رسانید.
\par 19 و از طرف مغرب ازنام یهوه و از طلوع آفتاب از جلال وی خواهندترسید زیرا که او مثل نهر سرشاری که باد خداوندآن را براند خواهد آمد.
\par 20 و خداوند می‌گوید که نجات‌دهنده‌ای برای صهیون و برای آنانی که دریعقوب از معصیت بازگشت نمایند خواهد آمد.و خداوند می‌گوید: «اما عهد من با ایشان این است که روح من که بر تو است و کلام من که دردهان تو گذاشته‌ام از دهان تو و از دهان ذریت تو واز دهان ذریت ذریت تو دور نخواهد شد.» خداوند می‌گوید: «از الان و تا ابدالاباد.»
\par 21 و خداوند می‌گوید: «اما عهد من با ایشان این است که روح من که بر تو است و کلام من که دردهان تو گذاشته‌ام از دهان تو و از دهان ذریت تو واز دهان ذریت ذریت تو دور نخواهد شد.» خداوند می‌گوید: «از الان و تا ابدالاباد.»
 
\chapter{60}

\par 1 برخیز و درخشان شو زیرا نور تو آمده و جلال خداوند بر تو طالع گردیده است.
\par 2 زیرا اینک تاریکی جهان را و ظلمت غلیظ طوایف را خواهد پوشانید اما خداوند بر توطلوع خواهد نمود و جلال وی بر تو ظاهرخواهد شد.
\par 3 و امت‌ها بسوی نور تو و پادشاهان بسوی درخشندگی طلوع تو خواهند آمد.
\par 4 چشمان خود را به اطراف خویش برافراز و ببین که جمیع آنها جمع شده، نزد تو می‌آیند. پسرانت از دور خواهند آمد و دخترانت را در آغوش خواهند‌آورد.
\par 5 آنگاه خواهی دید و خواهی درخشید و دل تو لرزان شده، وسیع خواهد گردید، زیرا که توانگری دریا بسوی تو گردانیده خواهد شد و دولت امت‌ها نزد تو خواهد آمد.
\par 6 کثرت شتران و جمازگان مدیان و عیفه تو راخواهند پوشانید. جمیع اهل شبع خواهند آمد وطلا و بخور آورده، به تسبیح خداوند بشارت خواهند داد.
\par 7 جمیع گله های قیدار نزد تو جمع خواهند شد و قوچهای نبایوت تو را خدمت خواهند نمود. به مذبح من با پذیرایی برخواهندآمد و خانه جلال خود را زینت خواهم داد.
\par 8 اینها کیستند که مثل ابر پرواز می‌کنند ومانند کبوتران بر وزنهای خود؟
\par 9 به درستی که جزیره‌ها و کشتیهای ترشیش اول انتظار مراخواهند کشید تا پسران تو را از دور و نقره وطلای ایشان را با ایشان بیاورند، به جهت اسم یهوه خدای تو و به جهت قدوس اسرائیل زیرا که تو را زینت داده است.
\par 10 و غریبان، حصارهای تو را بنا خواهند نمود و پادشاهان ایشان تو راخدمت خواهند کرد زیرا که در غضب خود تو رازدم لیکن به لطف خویش تو را ترحم خواهم نمود.
\par 11 دروازه های تو نیز دائم باز خواهد بود وشب و روز بسته نخواهد گردید تا دولت امت‌ها رانزد تو بیاورند و پادشاهان ایشان همراه آورده شوند.
\par 12 زیرا هر امتی و مملکتی که تو را خدمت نکند تلف خواهد شد و آن امت‌ها تمام هلاک خواهند گردید.
\par 13 جلال لبنان با درختان صنوبر و کاج و چناربا هم برای تو آورده خواهند شد تا مکان مقدس مرا زینت دهند و جای پایهای خود را تمجیدخواهم نمود.
\par 14 و پسران آنانی که تو را ستم می رسانند خم شده، نزد تو خواهند آمد و جمیع آنانی که تو را اهانت می‌نمایند نزد کف پایهای توسجده خواهند نمود و تو را شهر یهوه و صهیون قدوس اسرائیل خواهند نامید.
\par 15 به عوض آنکه تو متروک و مبغوض بودی و کسی از میان تو گذرنمی کرد. من تو را فخر جاودانی و سرور دهرهای بسیار خواهم گردانید.
\par 16 و شیر امت‌ها را خواهی مکید و پستانهای پادشاهان را خواهی مکید وخواهی فهمید که من یهوه نجات‌دهنده تو هستم و من قدیر اسرائیل، ولی تو می‌باشم.
\par 17 به‌جای برنج، طلا خواهم آورد و به‌جای آهن، نقره و به‌جای چوب، برنج و به‌جای سنگ، آهن خواهم آورد و سلامتی را ناظران تو و عدالت را حاکمان تو خواهم گردانید.
\par 18 و بار دیگر ظلم در زمین توو خرابی و ویرانی در حدود تو مسموع نخواهدشد و حصارهای خود را نجات و دروازه های خویش را تسبیح خواهی نامید.
\par 19 و بار دیگرآفتاب در روز نور تو نخواهد بود و ماه بادرخشندگی برای تو نخواهد تابید زیرا که یهوه نور جاودانی تو و خدایت زیبایی تو خواهد بود.
\par 20 و بار دیگر آفتاب تو غروب نخواهد کرد و ماه تو زوال نخواهد پذیرفت زیرا که یهوه برای تونور جاودانی خواهد بود و روزهای نوحه گری توتمام خواهد شد.
\par 21 و جمیع قوم تو عادل خواهند بود و زمین را تا به ابد متصرف خواهندشد. شاخه مغروس من و عمل دست من، تاتمجید کرده شوم.صغیر هزار نفر خواهد شدو حقیر امت قوی خواهد گردید. من یهوه دروقتش تعجیل در آن خواهم نمود.
\par 22 صغیر هزار نفر خواهد شدو حقیر امت قوی خواهد گردید. من یهوه دروقتش تعجیل در آن خواهم نمود.
 
\chapter{61}

\par 1 روح خداوند یهوه بر من است زیراخداوند مرا مسح کرده است تامسکینان را بشارت دهم و مرا فرستاده تا شکسته دلان را التیام بخشم و اسیران را به رستگاری ومحبوسان را به آزادی ندا کنم.
\par 2 و تا از سال پسندیده خداوند و از یوم انتقام خدای ما ندانمایم و جمیع ماتمیان را تسلی بخشم.
\par 3 تا قراردهم برای ماتمیان صهیون و به ایشان ببخشم تاجی را به عوض خاکستر و روغن شادمانی را به عوض نوحه گری و ردای تسبیح را به‌جای روح کدورت تا ایشان درختان عدالت و مغروس خداوند به جهت تمجید وی نامیده شوند.
\par 4 و ایشان خرابه های قدیم را بنا خواهند نمودو ویرانه های سلف را بر پا خواهند داشت وشهرهای خراب شده و ویرانه های دهرهای بسیار را تعمیر خواهند نمود.
\par 5 و غریبان برپاشده، غله های شما را خواهند چرانید و بیگانگان، فلاحان و باغبانان شما خواهند بود.
\par 6 و شماکاهنان خداوند نامیده خواهید شد و شما را به خدام خدای ما مسمی خواهند نمود. دولت امت‌ها را خواهید خورد و در جلال ایشان فخرخواهید نمود.
\par 7 به عوض خجالت، نصیب مضاعف خواهید یافت و به عوض رسوایی ازنصیب خود مسرور خواهند شد بنابراین ایشان درزمین خود نصیب مضاعف خواهند یافت و شادی جاودانی برای ایشان خواهد بود.
\par 8 زیرا من که یهوه هستم عدالت را دوست می‌دارم و از غارت و ستم نفرت می‌دارم و اجرت ایشان را به راستی به ایشان خواهم داد و عهد جاودانی با ایشان خواهم بست.
\par 9 و نسل ایشان در میان امت‌ها وذریت ایشان در میان قوم‌ها معروف خواهند شد. هر‌که ایشان را بیند اعتراف خواهد نمود که ایشان ذریت مبارک خداوند می‌باشند.
\par 10 در خداوند شادی بسیار می‌کنم و جان من در خدای خود وجد می‌نماید زیرا که مرا به‌جامه نجات ملبس ساخته، ردای عدالت را به من پوشانید. چنانکه داماد خویشتن را به تاج آرایش می‌دهد و عروس، خود را به زیورها زینت می‌بخشد.زیرا چنانکه زمین، نباتات خود رامی رویاند و باغ، زرع خویش را نمو می‌دهد، همچنان خداوند یهوه عدالت و تسبیح را پیش روی تمامی امت‌ها خواهد رویانید.
\par 11 زیرا چنانکه زمین، نباتات خود رامی رویاند و باغ، زرع خویش را نمو می‌دهد، همچنان خداوند یهوه عدالت و تسبیح را پیش روی تمامی امت‌ها خواهد رویانید.
 
\chapter{62}

\par 1 به‌خاطر صهیون سکوت نخواهم کرد وبه‌خاطر اورشلیم خاموش نخواهم شدتا عدالتش مثل نور طلوع کند و نجاتش مثل چراغی که افروخته باشد.
\par 2 و امت‌ها، عدالت تورا و جمیع پادشاهان، جلال تو را مشاهده خواهند نمود. و تو به اسم جدیدی که دهان خداوند آن را قرار می‌دهد مسمی خواهی شد.
\par 3 و تو تاج جلال، در دست خداوند و افسرملوکانه، در دست خدای خود خواهی بود. 
\par 4 و تودیگر به متروک مسمی نخواهی شد و زمینت رابار دیگر خرابه نخواهند گفت، بلکه تو را حفصیبه و زمینت را بعوله خواهند نامید زیرا خداوند از تومسرور خواهد شد و زمین تو منکوحه خواهدگردید.
\par 5 زیرا چنانکه مردی جوان دوشیزه‌ای رابه نکاح خویش در می‌آورد هم چنان پسرانت تو را منکوحه خود خواهند ساخت و چنانکه داماداز عروس مبتهج می‌گردد هم چنان خدایت از تومسرور خواهد بود.
\par 6 ‌ای اورشلیم دیده بانان برحصارهای تو گماشته‌ام که هر روز و هرشب همیشه سکوت نخواهند کرد. ای متذکران خداوند خاموش مباشید!
\par 7 و او را آرامی ندهیدتا اورشلیم را استوار کرده، آن را در جهان محل تسبیح بسازد.
\par 8 خداوند به‌دست راست خود و به بازوی قوی خویش قسم خورده، گفته است که بار دیگرغله تو را ماکول دشمنانت نسازم و غریبان، شراب تو را که برایش زحمت کشیده‌ای نخواهندنوشید.
\par 9 بلکه آنانی که آن را می‌چینند آن راخورده، خداوند را تسبیح خواهند نمود و آنانی که آن را جمع می‌کنند آن را در صحنهای قدس من خواهند نوشید.
\par 10 بگذرید از دروازه هابگذرید. طریق قوم را مهیا سازید و شاهراه را بلندکرده، مرتفع سازید و سنگها را برچیده علم را به جهت قوم‌ها برپا نمایید.
\par 11 اینک خداوند تااقصای زمین اعلان کرده است، پس به دخترصهیون بگویید اینک نجات تو می‌آید. همانااجرت او همراهش و مکافات او پیش رویش می‌باشد.و ایشان را به قوم مقدس و فدیه شدگان خداوند مسمی خواهند ساخت و تو به مطلوب و شهر غیر متروک نامیده خواهی شد.
\par 12 و ایشان را به قوم مقدس و فدیه شدگان خداوند مسمی خواهند ساخت و تو به مطلوب و شهر غیر متروک نامیده خواهی شد.
 
\chapter{63}

\par 1 این کیست که از ادوم با لباس سرخ ازبصره می‌آید؟ یعنی‌این که به لباس جلیل خود ملبس است و در کثرت قوت خویش می‌خرامد؟ من که به عدالت تکلم می‌کنم و برای نجات، زورآور می‌باشم.
\par 2 چرا لباس تو سرخ است و جامه تو مثل کسی‌که چرخشت را پایمال کند؟
\par 3 من چرخشت را تنها پایمال نمودم واحدی از قوم‌ها با من نبود و ایشان را بغضب خودپایمال کردم و بحدت خشم خویش لگد کوب نمودم و خون ایشان به لباس من پاشیده شده، تمامی جامه خود را آلوده ساختم.
\par 4 زیرا که یوم انتقام در دل من بود و سال فدیه شدگانم رسیده بود.
\par 5 و نگریستم و اعانت کننده‌ای نبود و تعجب نمودم زیرا دستگیری نبود. لهذا بازوی من مرانجات داد و حدت خشم من مرا دستگیری نمود.
\par 6 و قوم‌ها را به غضب خود پایمال نموده، ایشان را از حدت خشم خویش مست ساختم. و خون ایشان را بر زمین ریختم.
\par 7 احسانهای خداوند و تسبیحات خداوند راذکر خواهم نمود برحسب هر‌آنچه خداوند برای ما عمل نموده است و به موجب کثرت احسانی که برای خاندان اسرائیل موافق رحمت‌ها و وفوررافت خود بجا آورده است.
\par 8 زیرا گفته است: ایشان قوم من و پسرانی که خیانت نخواهند کردمی باشند، پس نجات‌دهنده ایشان شده است.
\par 9 او در همه تنگیهای ایشان به تنگ آورده شد وفرشته حضور وی ایشان را نجات داد. در محبت و حلم خود ایشان را فدیه داد و در جمیع ایام قدیم، متحمل ایشان شده، ایشان را برداشت.
\par 10 اما ایشان عاصی شده، روح قدوس او رامحزون ساختند، پس برگشته، دشمن ایشان شد واو خود با ایشان جنگ نمود.
\par 11 آنگاه ایام قدیم و موسی و قوم خویش رابیاد آورد (و گفت ) کجاست آنکه ایشان را با شبان گله خود از دریا برآورد و کجا است آنکه روح قدوس خود را در میان ایشان نهاد؟
\par 12 که بازوی جلیل خود را به‌دست راست موسی خرامان ساخت و آبها را پیش روی ایشان منشق گردانیدتا اسم جاودانی برای خویش پیدا کند؟
\par 13 آنکه ایشان را در لجه‌ها مثل اسب در بیابان رهبری نمود که لغزش نخورند.
\par 14 مثل بهایمی که به وادی فرود می‌روند روح خداوند ایشان را آرامی بخشید. هم چنان قوم خود را رهبری نمودی تابرای خود اسم مجید پیدا نمایی.
\par 15 از آسمان بنگر و از مسکن قدوسیت وجلال خویش نظر افکن. غیرت جبروت تو کجااست؟ جوشش دل و رحمت های تو که به من نمودی بازداشته شده است.
\par 16 به درستی که توپدر ما هستی اگر‌چه ابراهیم ما را نشناسد واسرائیل ما را بجا نیاورد، اما تو‌ای یهوه، پدر ما وولی ما هستی و نام تو از ازل می‌باشد.
\par 17 پس‌ای خداوند ما را از طریق های خود چرا گمراه ساختی و دلهای ما را سخت گردانیدی تا از تونترسیم. به‌خاطر بندگانت و اسباط میراث خودرجعت نما.
\par 18 قوم مقدس تو اندک زمانی آن رامتصرف بودند و دشمنان ما مکان قدس تو راپایمال نمودند.و ما مثل کسانی که تو هرگز برایشان حکمرانی نکرده باشی و به نام تو نامیده نشده باشند گردیده‌ایم.
\par 19 و ما مثل کسانی که تو هرگز برایشان حکمرانی نکرده باشی و به نام تو نامیده نشده باشند گردیده‌ایم.
 
\chapter{64}

\par 1 کاش که آسمانها را منشق ساخته، نازل می شدی و کوهها از رویت تو متزلزل می‌گشت.
\par 2 مثل آتشی که خورده چوبها رامشتعل سازد و ناری که آب را به جوش آورد تانام خود را بر دشمنانت معروف سازی و امت‌ها ازرویت تو لرزان گردند.
\par 3 حینی که کارهای هولناک را که منتظر آنها نبودیم بجا آوردی. آنگاه نزول فرمودی و کوهها از رویت تو متزلزل گردید.
\par 4 زیرا که از ایام قدیم نشنیدند و استماع ننمودند و چشم خدایی را غیر از تو که برای منتظران خویش بپردازد ندید.
\par 5 تو آنانی را که شادمانند و عدالت را بجا می‌آورند و به راههای تو تو را به یاد می‌آورند ملاقات می‌کنی. اینک توغضبناک شدی و ما گناه کرده‌ایم در اینها مدت مدیدی بسر بردیم و آیا نجات توانیم یافت؟
\par 6 زیرا که جمیع ما مثل شخص نجس شده‌ایم وهمه اعمال عادله ما مانند لته ملوث می‌باشد. وهمگی ما مثل برگ، پژمرده شده، گناهان ما مثل باد، ما را می‌رباید.
\par 7 و کسی نیست که اسم تو رابخواند یا خویشتن را برانگیزاند تا به تو متمسک شود زیرا که روی خود را از ما پوشیده‌ای و ما رابه‌سبب گناهان ما گداخته‌ای.
\par 8 اما الان‌ای خداوند، تو پدر ما هستی. ما گل هستیم و تو صانع ما هستی و جمیع ما مصنوع دستهای تو می‌باشیم.
\par 9 ‌ای خداوند بشدت غضبناک مباش و گناه را تا به ابد بخاطر مدار. هان ملاحظه نما که همگی ما قوم تو هستیم.
\par 10 شهرهای مقدس تو بیابان شده. صهیون، بیابان و اورشلیم، ویرانه گردیده است.
\par 11 خانه مقدس و زیبای ما که پدران ما تو را در آن تسبیح می خواندند به آتش سوخته شده و تمامی نفایس ما به خرابی مبدل گردیده است.‌ای خداوندآیا با وجود این همه، خودداری می‌کنی وخاموش شده، ما را بشدت رنجور می‌سازی؟
\par 12 ‌ای خداوندآیا با وجود این همه، خودداری می‌کنی وخاموش شده، ما را بشدت رنجور می‌سازی؟
 
\chapter{65}

\par 1 «آنانی که مرا طلب ننمودند مرا جستندو آنانی که مرا نطلبیدند مرا یافتند. و به قومی که به اسم من نامیده نشدند گفتم لبیک لبیک.
\par 2 تمامی روز دستهای خود را بسوی قوم متمردی که موافق خیالات خود به راه ناپسندیده سلوک می‌نمودند دراز کردم.
\par 3 قومی که پیش رویم غضب مرا همیشه بهیجان می‌آورند، که درباغات قربانی می‌گذرانند و بر آجرها بخورمی سوزانند.
\par 4 که در قبرها ساکن شده، در مغاره هامنزل دارند، که گوشت خنزیر می‌خورند وخورش نجاسات در ظروف ایشان است.
\par 5 که می‌گویند: «در جای خود بایست و نزدیک من میازیرا که من از تو مقدس تر هستم.» اینان دود دربینی من می‌باشند و آتشی که تمامی روز مشتعل است.
\par 6 همانا این پیش من مکتوب است. پس ساکت نخواهم شد بلکه پاداش خواهم داد و به آغوش ایشان مکافات خواهم رسانید.
\par 7 خداوندمی گوید درباره گناهان شما و گناهان پدران شما باهم که بر کوهها بخور‌سوزانیدید و مرا بر تلهااهانت نمودید پس جزای اعمال شما را اول به آغوش شما خواهم رسانید.»
\par 8 خداوند چنین می‌گوید: «چنانکه شیره درخوشه یافت می‌شود و می‌گویند آن را فاسدمساز زیرا که برکت در آن است. همچنان به‌خاطر بندگان خود عمل خواهم نمود تا (ایشان را)بالکل هلاک نسازم.
\par 9 بلکه نسلی از یعقوب ووارثی برای کوههای خویش از یهودا به ظهورخواهم آورد. و برگزیدگانم ورثه آن و بندگانم ساکن آن خواهند شد.
\par 10 و شارون، مرتع گله‌ها ووادی عاکور، خوابگاه رمه‌ها به جهت قوم من که مرا طلبیده‌اند خواهد شد.
\par 11 و اما شما که خداوند را ترک کرده و کوه مقدس مرا فراموش نموده‌اید، و مائده‌ای به جهت بخت مهیا ساخته وشراب ممزوج به جهت اتفاق ریخته‌اید،
\par 12 پس شما را به جهت شمشیر مقدر ساختم و جمیع شما برای قتل خم خواهید شد زیرا که چون خواندم جواب ندادید و چون سخن گفتم نشنیدید و آنچه را که در نظر من ناپسند بود بعمل آوردید و آنچه را که نخواستم برگزیدید.»
\par 13 بنابراین خداوند یهوه می‌گوید: «هان بندگان من خواهند خورد اما شما گرسنه خواهید بود اینک بندگانم خواهند نوشید اما شما تشنه خواهید بود. همانا بندگانم شادی خواهند کرد اما شما خجل خواهید گردید.
\par 14 اینک بندگانم از خوشی دل، ترنم خواهند نمود اما شما از کدورت دل، فریادخواهید نمود و از شکستگی روح، ولوله خواهیدکرد.
\par 15 و نام خود را برای برگزیدگان من به‌جای لعنت، ترک خواهید نمود پس خداوند یهوه تو رابقتل خواهد رسانید و بندگان خویش را به اسم دیگر خواهد نامید.
\par 16 پس هرکه خویشتن رابروی زمین برکت دهد خویشتن را به خدای حق برکت خواهد داد و هرکه بروی زمین قسم خوردبه خدای حق قسم خواهد خورد. زیرا که تنگیهای اولین فراموش شده و از نظر من پنهان گردیده است.
\par 17 زیرا اینک من آسمانی جدید و زمینی جدید خواهم آفرید و چیزهای پیشین بیادنخواهد آمد و بخاطر نخواهد گذشت.
\par 18 بلکه ازآنچه من خواهم آفرید، شادی کنید و تا به ابدوجد نمایید زیرا اینک اورشلیم را محل وجد وقوم او را محل شادمانی خواهم آفرید.
\par 19 و ازاورشلیم وجد خواهم نمود و از قوم خود شادی خواهم کرد و آواز گریه و آواز ناله بار دیگر در اوشنیده نخواهد شد.
\par 20 و بار دیگر طفل کم روز ازآنجا نخواهد بود و نه مرد پیر که عمر خود را به اتمام نرسانیده باشد زیرا که طفل در سن صدسالگی خواهد مرد لیکن گناهکار صد ساله ملعون خواهد بود.
\par 21 و خانه‌ها بنا کرده، در آنها ساکن خواهند شد و تاکستانها غرس نموده، میوه آنها راخواهند خورد.
\par 22 بنا نخواهند کرد تا دیگران سکونت نمایند و آنچه را که غرس می‌نماینددیگران نخواهند خورد. زیرا که ایام قوم من مثل ایام درخت خواهد بود و برگزیدگان من از عمل دستهای خود تمتع خواهند برد.
\par 23 زحمت بیجانخواهند کشید و اولاد به جهت اضطراب نخواهند زایید زیرا که اولاد برکت یافتگان خداوند هستند و ذریت ایشان با ایشانند.
\par 24 وقبل از آنکه بخوانند من جواب خواهم داد. و پیش از آنکه سخن گویند من خواهم شنید.گرگ وبره با هم خواهند چرید و شیر مثل گاو کاه خواهد خورد و خوراک مار خاک خواهد بود.خداوند می‌گوید که در تمامی کوه مقدس من، ضرر نخواهند رسانید و فساد نخواهندنمود.»
\par 25 گرگ وبره با هم خواهند چرید و شیر مثل گاو کاه خواهد خورد و خوراک مار خاک خواهد بود.خداوند می‌گوید که در تمامی کوه مقدس من، ضرر نخواهند رسانید و فساد نخواهندنمود.»
 
\chapter{66}

\par 1 خداوند چنین می‌گوید: «آسمانهاکرسی من و زمین پای انداز من است، پس خانه‌ای که برای من بنا می‌کنید کجا است؟ ومکان آرام من کجا؟»
\par 2 خداوند می‌گوید: «دست من همه این چیزها را ساخت پس جمیع اینهابوجود آمد اما به این شخص که مسکین و شکسته دل و از کلام من لرزان باشد، نظر خواهم کرد.
\par 3 کسی‌که گاوی ذبح نماید مثل قاتل انسان است و کسی‌که گوسفندی ذبح کند مثل شخصی است که گردن سگ را بشکند. و آنکه هدیه‌ای بگذراندمثل کسی است که خون خنزیری را بریزد و آنکه بخور‌سوزاند مثل شخصی است که بتی را تبریک نماید و ایشان راههای خود را اختیار کرده‌اند وجان ایشان از رجاسات خودشان مسرور است.
\par 4 پس من نیز مصیبت های ایشان را اختیار خواهم کرد و ترسهای ایشان را بر ایشان عارض خواهم گردانید، زیرا چون خواندم کسی جواب نداد وچون تکلم نمودم ایشان نشنیدند بلکه آنچه را که در نظر من ناپسند بود بعمل آوردند و آنچه را که نخواستم اختیار کردند.» 
\par 5 ‌ای آنانی که از کلام خداوند می‌لرزید سخن او را بشنوید. برادران شما که از شما نفرت دارندو شما را بخاطر اسم من از خود می‌رانندمی گویند: خداوند تمجید کرده شود تا شادی شما را ببینم لیکن ایشان خجل خواهند شد.
\par 6 آواز غوغا از شهر، صدایی از هیکل، آوازخداوند است که به دشمنان خود مکافات می‌رساند.
\par 7 قبل از آنکه درد زه بکشد، زایید. پیش از آنکه درد او را فرو‌گیرد اولاد نرینه‌ای آورد.
\par 8 کیست که مثل این را شنیده و کیست که مثل این را دیده باشد؟ آیا ولایتی در یک روزمولود گردد و قومی یکدفعه زاییده شود؟ زیرا صهیون به مجرد درد زه کشیدن پسران خود را زایید.
\par 9 خداوند می‌گوید: «آیا من بفم رحم برسانم و نزایانم؟» و خدای تومی گوید: «آیا من که زایاننده هستم، رحم راببندم؟»
\par 10 ‌ای همه آنانی که اورشلیم را دوست می‌دارید با او شادی کنید و برایش وجد نمایید. وای همه آنانی که برای او ماتم می‌گیرید، با اوشادی بسیار نمایید.
\par 11 تا از پستانهای تسلیات اوبمکید و سیر شوید و بدوشید و از فراوانی جلال او محظوظ گردید.
\par 12 زیرا خداوند چنین می‌گوید: «اینک من سلامتی را مثل نهر و جلال امت‌ها را مانند نهر سرشار به او خواهم رسانید. وشما خواهید مکید و در آغوش او برداشته شده، بر زانوهایش بناز پرورده خواهید شد.»
\par 13 و مثل کسی‌که مادرش او را تسلی دهد همچنین من شمارا تسلی خواهم داد ودر اورشلیم تسلی خواهیدیافت.»
\par 14 پس چون این را بینید دل شما شادمان خواهد شد و استخوانهای شما مثل گیاه سبز وخرم خواهد گردید و دست خداوند بر بندگانش معروف خواهد شد اما بر دشمنان خود غضب خواهد نمود.
\par 15 زیرا اینک خداوند با آتش خواهد آمد و ارابه های او مثل گردباد تاغضب خود را با حدت و عتاب خویش را باشعله آتش به انجام رساند.
\par 16 زیرا خداوند باآتش و شمشیر خود بر تمامی بشر داوری خواهد نمود و مقتولان خداوند بسیار خواهندبود.
\par 17 و خداوند می‌گوید: «آنانی که از عقب یکنفر که در وسط باشد خویشتن را در باغات تقدیس و تطهیر می‌نمایند و گوشت خنزیر ورجاسات و گوشت موش می‌خورند با هم تلف خواهند شد.
\par 18 و من اعمال و خیالات ایشان راجزا خواهم داد و آمده، جمیع امت‌ها و زبانها راجمع خواهم کرد و ایشان آمده، جلال مراخواهند دید.
\par 19 و آیتی در میان ایشان برپاخواهم داشت و آنانی را که از ایشان نجات یابندنزد امت‌ها به ترشیش و فول و تیراندازان لود وتوبال و یونان و جزایر بعیده که آوازه مرانشنیده‌اند و جلال مرا ندیده‌اند خواهم فرستاد تاجلال مرا در میان امت‌ها شایع سازند.»
\par 20 وخداوند می‌گوید که ایشان جمیع برادران شما رااز تمامی امت‌ها بر اسبان و ارابه‌ها و تخت روانهاو قاطران و شتران به کوه مقدس من اورشلیم به جهت خداوند هدیه خواهند‌آورد. چنانکه بنی‌اسرائیل هدیه خود را در ظرف پاک به خانه خداوند می‌آورند.
\par 21 و خداوند می‌گویدکه از ایشان نیز کاهنان و لاویان خواهند گرفت.
\par 22 زیرا خداوند می‌گوید: «چنانکه آسمانهای جدید و زمین جدیدی که من آنها را خواهم ساخت در حضور من پایدار خواهد ماندهمچنان ذریت شما و اسم شما پایدار خواهد ماند.»و خداوند می‌گوید که از غره ماه تاغره دیگر و از سبت تا سبت دیگر تمامی بشرخواهند آمد تا به حضور من سجده نمایند.
\par 23 و خداوند می‌گوید که از غره ماه تاغره دیگر و از سبت تا سبت دیگر تمامی بشرخواهند آمد تا به حضور من سجده نمایند.

\end{document}