\begin{document}

\title{غلطيان}


\chapter{1}

\par 1 پولس، رسول نه از جانب انسان و نه بوسیله انسان بلکه به عیسی مسیح و خدای پدر که او را از مردگان برخیزانید،
\par 2 و همه برادرانی که بامن می‌باشند، به کلیساهای غلاطیه،
\par 3 فیض و سلامتی از جانب خدای پدر وخداوند ما عیسی مسیح با شما باد؛
\par 4 که خود رابرای گناهان ما داد تا ما را از این عالم حاضر شریربحسب اراده خدا و پدر ما خلاصی بخشد،
\par 5 که او را تا ابدالاباد جلال باد. آمین.
\par 6 تعجب می‌کنم که بدین زودی از آن کس که شما را به فیض مسیح خوانده است، برمی گردیدبه سوی انجیلی دیگر،
\par 7 که (انجیل ) دیگر نیست. لکن بعضی هستند که شما را مضطرب می‌سازندو می‌خواهند انجیل مسیح را تبدیل نمایند.
\par 8 بلکه هرگاه ما هم یا فرشته‌ای از آسمان، انجیلی غیر از آنکه ما به آن بشارت دادیم به شما رساند، اناتیما باد.
\par 9 چنانکه پیش گفتیم، الان هم بازمی گویم: اگر کسی انجیلی غیر از آنکه پذیرفتیدبیاورد، اناتیما باد.
\par 10 آیا الحال مردم را در رای خود می‌آورم؟ یا خدا را یا رضامندی مردم را می طلبم؟ اگر تابحال رضامندی مردم رامی خواستم، غلام مسیح نمی بودم.
\par 11 اما‌ای برادران شما را اعلام می‌کنم ازانجیلی که من بدان بشارت دادم که به طریق انسان نیست.
\par 12 زیرا که من آن را از انسان نیافتم ونیاموختم، مگر به کشف عیسی مسیح.
\par 13 زیراسرگذشت سابق مرا در دین یهود شنیده‌اید که برکلیسای خدا بینهایت جفا می‌نمودم و آن راویران می‌ساختم،
\par 14 و در دین یهود از اکثرهمسالان قوم خود سبقت می‌جستم و در تقالیداجداد خود بغایت غیور می‌بودم.
\par 15 اما چون خداکه مرا از شکم مادرم برگزید و به فیض خود مراخواند، رضا بدین داد
\par 16 که پسر خود را در من آشکار سازد تا در میان امت‌ها بدو بشارت دهم، در آنوقت با جسم و خون مشورت نکردم،
\par 17 و به اورشلیم هم نزد آنانی که قبل از من رسول بودندنرفتم، بلکه به عرب شدم و باز به دمشق مراجعت کردم.
\par 18 پس بعد از سه سال، برای ملاقات پطرس به اورشلیم رفتم و پانزده روز با وی بسر بردم.
\par 19 اما از سایر رسولان جز یعقوب برادر خداوندرا ندیدم.
\par 20 اما درباره آنچه به شما می‌نویسم، اینک در حضور خدا دروغ نمی گویم.
\par 21 بعد از آن به نواحی سوریه و قیلیقیه آمدم.
\par 22 و به کلیساهای یهودیه که در مسیح بودند صورت غیرمعروف بودم،
\par 23 جز اینکه شنیده بودند که «آنکه پیشتر بر ما جفا می‌نمود، الحال بشارت می‌دهد به همان ایمانی که قبل از این ویران می‌ساخت.»و خدا را در من تمجید نمودند.
\par 24 و خدا را در من تمجید نمودند.

\chapter{2}

\par 1 پس بعد از چهارده سال با برنابا باز به اورشلیم رفتم و تیطس را همراه خود بردم.
\par 2 ولی به الهام رفتم و انجیلی را که در میان امت هابدان موعظه می‌کنم، به ایشان عرضه داشتم، امادر خلوت به معتبرین، مبادا عبث بدوم یا دویده باشم.
\par 3 لیکن تیطس نیز که همراه من و یونانی بود، مجبور نشد که مختون شود.
\par 4 و این به‌سبب برادران کذبه بود که ایشان را خفیه درآوردند وخفیه درآمدند تا آزادی ما را که در مسیح عیسی داریم، جاسوسی کنند و تا ما را به بندگی درآوردند.
\par 5 که ایشان را یک ساعت هم به اطاعت در این امر تابع نشدیم تا راستی انجیل در شماثابت ماند.
\par 6 اما از آنانی که معتبراند که چیزی می‌باشند -هرچه بودند مرا تفاوتی نیست، خدا بر صورت انسان نگاه نمی کند - زیرا آنانی که معتبراند، به من هیچ نفع نرسانیدند.
\par 7 بلکه به خلاف آن، چون دیدند که بشارت نامختونان به من سپرده شد، چنانکه بشارت مختونان به پطرس،
\par 8 زیرا او که برای رسالت مختونان در پطرس عمل کرد، در من هم برای امت‌ها عمل کرد.
\par 9 پس چون یعقوب وکیفا و یوحنا که معتبر به ارکان بودند، آن فیضی را که به من عطا شده بود دیدند، دست رفاقت به من وبرنابا دادند تا ما به سوی امت‌ها برویم، چنانکه ایشان به سوی مختونان؛
\par 10 جز آنکه فقرا را یادبداریم و خود نیز غیور به کردن این کار بودم.
\par 11 اما چون پطرس به انطاکیه آمد، او را روبرومخالفت نمودم زیرا که مستوجب ملامت بود،
\par 12 چونکه قبل از آمدن بعضی از جانب یعقوب، باامت‌ها غذا می‌خورد؛ ولی چون آمدند، از آنانی که اهل ختنه بودند ترسیده، باز ایستاد و خویشتن را جدا ساخت.
\par 13 و سایر یهودیان هم با وی نفاق کردند، بحدی که برنابا نیز در نفاق ایشان گرفتارشد.
\par 14 ولی چون دیدم که به راستی انجیل به استقامت رفتار نمی کنند، پیش روی همه پطرس را گفتم: «اگر تو که یهود هستی، به طریق امت‌ها ونه به طریق یهود زیست می‌کنی، چون است که امت‌ها را مجبور می‌سازی که به طریق یهود رفتارکنند؟»
\par 15 ما که طبع یهود هستیم و نه گناهکاران از امت‌ها،
\par 16 اما چونکه یافتیم که هیچ‌کس ازاعمال شریعت عادل شمرده نمی شود، بلکه به ایمان به عیسی مسیح، ما هم به مسیح عیسی ایمان آوردیم تا از ایمان به مسیح و نه از اعمال شریعت عادل شمرده شویم، زیرا که از اعمال شریعت هیچ بشری عادل شمرده نخواهد شد.
\par 17 اما اگر چون عدالت در مسیح را می‌طلبم، خود هم گناهکار یافت شویم، آیا مسیح خادم گناه است؟ حاشا!
\par 18 زیرا اگر باز بنا کنم آنچه راکه خراب ساختم، هرآینه ثابت می‌کنم که خود متعدی هستم.
\par 19 زانرو که من بواسطه شریعت نسبت به شریعت مردم تا نسبت به خدا زیست کنم.
\par 20 با مسیح مصلوب شده‌ام ولی زندگی می‌کنم لیکن نه من بعد از این، بلکه مسیح در من زندگی می‌کند. و زندگانی که الحال در جسم می‌کنم، به ایمان بر پسر خدا می‌کنم که مرا محبت نمود و خود را برای من داد.فیض خدا را باطل نمی سازم، زیرا اگر عدالت به شریعت می‌بود، هرآینه مسیح عبث مرد.
\par 21 فیض خدا را باطل نمی سازم، زیرا اگر عدالت به شریعت می‌بود، هرآینه مسیح عبث مرد.

\chapter{3}

\par 1 ای غلاطیان بی‌فهم، کیست که شما راافسون کرد تا راستی را اطاعت نکنید که پیش چشمان شما عیسی مسیح مصلوب شده مبین گردید؟
\par 2 فقط این را می‌خواهم از شمابفهمم که روح را از اعمال شریعت یافته‌اید یا ازخبر ایمان؟
\par 3 آیا اینقدر بی‌فهم هستید که به روح شروع کرده، الان به جسم کامل می‌شوید؟
\par 4 آیااینقدر زحمات را عبث کشیدید اگر فی الحقیقه عبث باشد؟
\par 5 پس آنکه روح را به شما عطامی کند و قوات در میان شما به ظهور می‌آورد، آیااز اعمال شریعت یا از خبر ایمان می‌کند؟
\par 6 چنانکه ابراهیم به خدا ایمان آورد و برای اوعدالت محسوب شد.
\par 7 پس آگاهید که اهل ایمان فرزندان ابراهیم هستند.
\par 8 و کتاب چون پیش دیدکه خدا امت‌ها را از ایمان عادل خواهد شمرد به ابراهیم بشارت داد که «جمیع امت‌ها از تو برکت خواهند یافت.»
\par 9 بنابراین اهل ایمان با ابراهیم ایمان دار برکت می‌یابند.
\par 10 زیرا جمیع آنانی که از اعمال شریعت هستند، زیر لعنت می‌باشند زیرا مکتوب است: «ملعون است هر‌که ثابت نماند در تمام نوشته های کتاب شریعت تا آنها را به‌جا آرد.»
\par 11 اما واضح است که هیچ‌کس در حضور خدا ازشریعت عادل شمرده نمی شود، زیرا که «عادل به ایمان زیست خواهد نمود.»
\par 12 اما شریعت ازایمان نیست بلکه «آنکه به آنها عمل می‌کند، درآنها زیست خواهد نمود.»
\par 13 مسیح، ما را ازلعنت شریعت فدا کرد چونکه در راه ما لعنت شد، چنانکه مکتوب است «ملعون است هرکه بر دارآویخته شود.»
\par 14 تا برکت ابراهیم در مسیح عیسی بر امت‌ها آید و تا وعده روح را به وسیله ایمان حاصل کنیم.
\par 15 ‌ای برادران، به طریق انسان سخن می‌گویم، زیرا عهدی را که از انسان نیز استوار می‌شود، هیچ‌کس باطل نمی سازد و نمی افزاید.
\par 16 اماوعده‌ها به ابراهیم و به نسل او گفته شد ونمی گوید «به نسلها» که گویا درباره بسیاری باشد، بلکه درباره یکی و «به نسل تو» که مسیح است.
\par 17 و مقصود این است عهدی را که از خدا به مسیح بسته شده بود، شریعتی که چهارصد و سی سال بعد از آن نازل شد، باطل نمی سازد بطوری که وعده نیست شود.
\par 18 زیرا اگر میراث از شریعت بودی، دیگر از وعده نبودی. لیکن خدا آن را به ابراهیم از وعده داد.
\par 19 پس شریعت چیست؟ برای تقصیرها بر آن افزوده شد تا هنگام آمدن آن نسلی که وعده بدوداده شد و بوسیله فرشتگان به‌دست متوسطی مرتب گردید.
\par 20 اما متوسط از یک نیست، اماخدا یک است.
\par 21 پس آیا شریعت به خلاف وعده های خداست؟ حاشا! زیرا اگر شریعتی داده می‌شد که تواند حیات‌بخشد، هرآینه عدالت از شریعت حاصل می‌شد.
\par 22 بلکه کتاب همه‌چیز را زیر گناه بست تا وعده‌ای که از ایمان به عیسی مسیح است، ایمانداران را عطا شود.
\par 23 اما قبل از آمدن ایمان، زیر شریعت نگاه داشته بودیم و برای آن ایمانی که می‌بایست مکشوف شود، بسته شده بودیم.
\par 24 پس شریعت لالای ماشد تا به مسیح برساند تا از ایمان عادل شمرده شویم.
\par 25 لیکن چون ایمان آمد، دیگر زیر دست لالا نیستیم.
\par 26 زیرا همگی شما بوسیله ایمان در مسیح عیسی، پسران خدا می‌باشید.
\par 27 زیرا همه شما که در مسیح تعمید یافتید، مسیح را در بر‌گرفتید.
\par 28 هیچ ممکن نیست که یهود باشد یا یونانی و نه غلام و نه آزاد و نه مرد و نه زن، زیرا که همه شمادر مسیح عیسی یک می‌باشید.اما اگر شما ازآن مسیح می‌باشید، هرآینه نسل ابراهیم وبرحسب وعده، وارث هستید.
\par 29 اما اگر شما ازآن مسیح می‌باشید، هرآینه نسل ابراهیم وبرحسب وعده، وارث هستید.

\chapter{4}

\par 1 است، از غلام هیچ فرق ندارد، هرچندمالک همه باشد.
\par 2 بلکه زیردست ناظران و وکلامی باشد تا روزی که پدرش تعیین کرده باشد.
\par 3 همچنین ما نیز چون صغیر می‌بودیم، زیر اصول دنیوی غلام می‌بودیم.
\par 4 لیکن چون زمان به‌کمال رسید، خدا پسر خود را فرستاد که از زن زاییده شد و زیر شریعت متولد،
\par 5 تا آنانی را که زیرشریعت باشند فدیه کند تا آنکه پسرخواندگی رابیابیم.
\par 6 اما چونکه پسر هستید، خدا روح پسرخود را در دلهای شما فرستاد که ندا می‌کند «یاابا» یعنی «ای پدر.»
\par 7 لهذا دیگر غلام نیستی بلکه پسر، و چون پسر هستی، وارث خدا نیز بوسیله مسیح.
\par 8 لیکن در آن زمان چون خدا را نمی شناختید، آنانی را که طبیعت خدایان نبودند، بندگی می‌کردید.
\par 9 اما الحال که خدا را می‌شناسید بلکه خدا شما را می‌شناسد، چگونه باز می‌گردید به سوی آن اصول ضعیف و فقیر که دیگرمی خواهید از سر نو آنها را بندگی کنید؟
\par 10 روزهاو ماهها و فصل‌ها و سالها را نگاه می‌دارید.
\par 11 درباره شما ترس دارم که مبادا برای شما عبث زحمت کشیده باشم.
\par 12 ‌ای برادران، از شما استدعا دارم که مثل من بشوید، چنانکه من هم مثل شما شده‌ام. به من هیچ ظلم نکردید.
\par 13 اما آگاهید که به‌سبب ضعف بدنی، اول به شما بشارت دادم.
\par 14 و آن امتحان مرا که در جسم من بود، خوار نشمردید و مکروه نداشتید، بلکه مرا چون فرشته خدا و مثل مسیح عیسی پذیرفتید.
\par 15 پس کجا است آن مبارک بادی شما؟ زیرا به شما شاهدم که اگرممکن بودی، چشمان خود را بیرون آورده، به من می‌دادید.
\par 16 پس چون به شما راست می‌گویم، آیا دشمن شما شده‌ام؟
\par 17 شما را به غیرت می‌طلبند، لیکن نه به خیر، بلکه می‌خواهند در را بر روی شما ببندند تا شما ایشان را بغیرت بطلبید.
\par 18 لیکن غیرت در امر نیکو در هر زمان نیکو است، نه‌تنها چون من نزد شما حاضر باشم.
\par 19 ‌ای فرزندان من که برای شما باز درد زه دارم تاصورت مسیح در شما بسته شود.
\par 20 باری خواهش می‌کردم که الان نزد شما حاضر می‌بودم تا سخن خود را تبدیل کنم، زیرا که درباره شمامتحیر شده‌ام.
\par 21 شما که می‌خواهید زیر شریعت باشید، مرابگویید آیا شریعت را نمی شنوید؟
\par 22 زیرامکتوب است ابراهیم را دو پسر بود، یکی از کنیزو دیگری از آزاد.
\par 23 لیکن پسر کنیز، بحسب جسم تولد یافت و پسر آزاد، برحسب وعده.
\par 24 واین امور بطور مثل گفته شد زیرا که این دو زن، دوعهد می‌باشند، یکی از کوه سینا برای بندگی می‌زاید و آن هاجر است.
\par 25 زیرا که هاجر کوه سینا است در عرب، و مطابق است با اورشلیمی که موجود است، زیرا که با فرزندانش در بندگی می‌باشد.
\par 26 لیکن اورشلیم بالا آزاد است که مادرجمیع ما می‌باشد.
\par 27 زیرا مکتوب است: «ای نازاد که نزاییده‌ای، شاد باش! صدا کن و فریادبرآور‌ای تو که درد زه ندیده‌ای، زیرا که فرزندان زن بی‌کس از اولاد شوهردار بیشتراند.»
\par 28 لیکن ما‌ای برادران، چون اسحاق فرزندان وعده می‌باشیم.
\par 29 بلکه چنانکه آنوقت آنکه برحسب جسم تولد یافت، بر وی که برحسب روح بود جفامی کرد، همچنین الان نیز هست.
\par 30 لیکن کتاب چه می‌گوید؟ «کنیز و پسر او را بیرون کن زیرا پسرکنیز با پسر آزاد میراث نخواهد یافت.»خلاصه‌ای برادران، فرزندان کنیز نیستیم بلکه از زن آزادیم.
\par 31 خلاصه‌ای برادران، فرزندان کنیز نیستیم بلکه از زن آزادیم.

\chapter{5}

\par 1 پس به آن آزادی که مسیح ما را به آن آزادکرد استوار باشید و باز در یوغ بندگی گرفتار مشوید.
\par 2 اینک من پولس به شما می‌گویم که اگر مختون شوید، مسیح برای شما هیچ نفع ندارد.
\par 3 بلی باز به هرکس که مختون شود شهادت می‌دهم که مدیون است که تمامی شریعت را به‌جا آورد.
\par 4 همه شما که از شریعت عادل می‌شوید، از مسیح باطل و از فیض ساقطگشته‌اید.
\par 5 زیرا که ما بواسطه روح از ایمان مترقب امید عدالت هستیم.
\par 6 و در مسیح عیسی نه ختنه فایده دارد و نه نامختونی بلکه ایمانی که به محبت عمل می‌کند.
\par 7 خوب می‌دویدید. پس کیست که شما را ازاطاعت راستی منحرف ساخته است؟
\par 8 این ترغیب از او که شما را خوانده است نیست.
\par 9 خمیرمایه اندک تمام خمیر را مخمر می‌سازد.
\par 10 من در خداوند بر شما اعتماد دارم که هیچ رای دیگر نخواهید داشت، لیکن آنکه شما رامضطرب سازد هرکه باشد قصاص خود راخواهد یافت.
\par 11 اما‌ای برادران اگر من تا به حال به ختنه موعظه می‌کردم، چرا جفا می‌دیدم؟ زیراکه در این صورت لغزش صلیب برداشته می‌شد.
\par 12 کاش آنانی که شما را مضطرب می‌سازند خویشتن را منقطع می‌ساختند.
\par 13 زیرا که شما‌ای برادران به آزادی خوانده شده‌اید؛ اما زنهار آزادی خود را فرصت جسم مگردانید، بلکه به محبت، یکدیگر را خدمت کنید.
\par 14 زیرا که تمامی شریعت در یک کلمه کامل می‌شود یعنی در اینکه همسایه خود راچون خویشتن محبت نما.
\par 15 اما اگر همدیگر رابگزید و بخورید، باحذر باشید که مبادا ازیکدیگر هلاک شوید.
\par 16 اما می‌گویم به روح رفتار کنید پس شهوات جسم را به‌جا نخواهید آورد.
\par 17 زیرا خواهش جسم به خلاف روح است و خواهش روح به خلاف جسم و این دو با یکدیگر منازعه می‌کنندبطوری که آنچه می‌خواهید نمی کنید.
\par 18 اما اگراز روح هدایت شدید، زیر شریعت نیستید.
\par 19 و اعمال جسم آشکار است، یعنی زنا وفسق و ناپاکی و فجور،
\par 20 و بت‌پرستی وجادوگری و دشمنی و نزاع و کینه و خشم وتعصب و شقاق و بدعتها،
\par 21 و حسد و قتل ومستی و لهو و لعب و امثال اینها که شما را خبرمی دهم چنانکه قبل از این دادم، که کنندگان چنین کارها وارث ملکوت خدا نمی شوند.
\par 22 لیکن ثمره روح، محبت و خوشی وسلامتی و حلم و مهربانی و نیکویی و ایمان وتواضع و پرهیزکاری است،
\par 23 که هیچ شریعت مانع چنین کارها نیست.
\par 24 و آنانی که از آن مسیح می‌باشند، جسم را با هوسها و شهواتش مصلوب ساخته‌اند.
\par 25 اگر به روح زیست کنیم، به روح هم رفتار بکنیم.لاف‌زن مشویم تا یکدیگر را به خشم آوریم و بر یکدیگر حسد بریم.
\par 26 لاف‌زن مشویم تا یکدیگر را به خشم آوریم و بر یکدیگر حسد بریم.

\chapter{6}

\par 1 اما‌ای برادران، اگر کسی به خطایی گرفتارشود، شما که روحانی هستید چنین شخص را به روح تواضع اصلاح کنید. و خود راملاحظه کن که مبادا تو نیز در تجربه افتی.
\par 2 بارهای سنگین یکدیگر را متحمل شوید و بدین نوع شریعت مسیح را به‌جا آرید.
\par 3 زیرا اگر کسی خود را شخص گمان برد و حال آنکه چیزی نباشد، خود را می‌فریبد.
\par 4 اما هرکس عمل خودرا امتحان بکند، آنگاه فخر در خود به تنهایی خواهد داشت نه در دیگری،
\par 5 زیرا هرکس حامل بار خود خواهد شد.
\par 6 اما هرکه در کلام تعلیم یافته باشد، معلم خود را در همه‌چیزهای خوب مشارک بسازد.
\par 7 خود را فریب مدهید، خدا رااستهزاء نمی توان کرد. زیرا که آنچه آدمی بکارد، همان را درو خواهد کرد.
\par 8 زیرا هر‌که برای جسم خود کارد، از جسم فساد را درو کند و هرکه برای روح کارد از روح حیات جاودانی خواهد دروید.
\par 9 لیکن از نیکوکاری خسته نشویم زیرا که درموسم آن درو خواهیم کرد اگر ملول نشویم.
\par 10 خلاصه بقدری که فرصت داریم با جمیع مردم احسان بنماییم، علی الخصوص با اهل بیت ایمان.
\par 11 ملاحظه کنید چه حروف جلی بدست خود به شما نوشتم.
\par 12 آنانی که می‌خواهند صورتی نیکو در جسم نمایان سازند، ایشان شمارا مجبور می‌سازند که مختون شوید، محض اینکه برای صلیب مسیح جفا نبینند.
\par 13 زیراایشان نیز که مختون می‌شوند، خود شریعت رانگاه نمی دارند بلکه می‌خواهند شما مختون شوید تا در جسم شما فخر کنند.
\par 14 لیکن حاشا ازمن که فخر کنم جز از صلیب خداوند ما عیسی مسیح که بوسیله او دنیا برای من مصلوب شد و من برای دنیا.
\par 15 زیرا که در مسیح عیسی نه ختنه چیزی است و نه نامختونی بلکه خلقت تازه.
\par 16 وآنانی که بدین قانون رفتار می‌کنند، سلامتی ورحمت بر ایشان باد و بر اسرائیل خدا.بعد ازاین هیچ‌کس مرا زحمت نرساند زیرا که من در بدن خود داغهای خداوند عیسی را دارم.
\par 17 بعد ازاین هیچ‌کس مرا زحمت نرساند زیرا که من در بدن خود داغهای خداوند عیسی را دارم.



\end{document}