\begin{document}

\title{زکريا}


\chapter{1}

\par 1 در ماه هشتم از سال دوم داریوش، کلام خداوند بر زکریا ابن برکیا ابن عدوی نبی نازل شده، گفت:
\par 2 «خداوند بر پدران شما بسیارغضبناک بود.
\par 3 پس به ایشان بگو: یهوه صبایوت چنین می‌گوید بسوی من بازگشت کنید. قول یهوه صبایوت این است. و یهوه صبایوت می‌گوید: من به سوی شما رجوع خواهم نمود.
\par 4 شما مثل پدران خود مباشید که انبیا سلف ایشان را نداکرده، گفتند یهوه صبایوت چنین می‌گوید ازراههای زشت خود و از اعمال بد خویش بازگشت نمایید، اما خداوند می‌گوید که ایشان نشنیدند و به من گوش ندادند.
\par 5 پدران شما کجاهستند و آیا انبیا همیشه زنده می‌مانند؟
\par 6 لیکن کلام و فرایض من که به بندگان خود انبیاامرفرموده بودم، آیا پدران شما را در نگرفت؟ وچون ایشان بازگشت نمودند، گفتند چنانکه یهوه صبایوت قصد نمود که موافق راهها و اعمال ما به ما عمل نماید همچنان به ما عمل نموده است.»
\par 7 در روز بیست و چهارم ماه یازدهم که ماه شباط باشد، از سال دوم داریوش، کلام خداوندبر زکریا ابن برکیا ابن عدوی نبی نازل شده، گفت:
\par 8 در وقت شب دیدم که اینک مردی بر اسب سرخ سوار بود و در میان درختان آس که در وادی بود ایستاده و در عقب او اسبان سرخ و زرد و سفیدبود.
\par 9 و گفتم: «ای آقایم اینها چیستند؟» وفرشته‌ای که با من تکلم می‌نمود، مرا گفت: «من تورا نشان می‌دهم که اینها چیستند.»
\par 10 پس آن مرد که در میان درختان آس ایستاده بود، جواب داد و گفت: «اینها کسانی می‌باشند که خداوند ایشان را برای تردد نمودن در جهان فرستاده است.»
\par 11 و ایشان به فرشته خداوند که در میان درختان آس ایستاده بودجواب داده، گفتند: «ما در جهان تردد نموده‌ایم واینک تمامی جهان مستریح و آرام است.»
\par 12 وفرشته خداوند جواب داده، گفت: «ای یهوه صبایوت تا به کی بر اورشلیم و شهرهای یهودا که در این هفتاد سال غضبناک می‌بودی رحمت نخواهی نمود؟»
\par 13 و خداوند با سخنان نیکو وکلام تسلی آمیز آن فرشته‌ای را که با من تکلم می‌نمود جواب داد.
\par 14 پس فرشته‌ای که با من تکلم می‌نمود مراگفت: «ندا کرده بگو یهوه صبایوت چنین می‌گوید: در باره اورشلیم و صهیون غیرت عظیمی داشتم.
\par 15 و برامت های مطمئن سخت غضبناک شدم زیرا که اندک غضبناک می‌بودم لیکن ایشان بلا را زیاده کردند.
\par 16 بنابراین خداوندچنین می‌گوید: به اورشلیم با رحمت‌ها رجوع خواهم نمود و خانه من در آن بنا خواهد شد. قول یهوه صبایوت این است و ریسمانکاری بر اورشلیم کشیده خواهد شد.
\par 17 بار دیگر نداکرده، بگو که یهوه صبایوت چنین می‌گوید: شهرهای من بار دیگر به سعادتمندی لبریزخواهد شد و خداوند صهیون را باز تسلی خواهد داد و اورشلیم را بار دیگر خواهدبرگزید.»
\par 18 پس چشمان خود را برافراشته، نگریستم واینک چهار شاخ بود.
\par 19 و به فرشته‌ای که با من تکلم می‌نمود گفتم: «اینها چیستند؟» او مرا گفت: «اینها شاخها می‌باشند که یهودا و اسرائیل واورشلیم را پراکنده ساخته‌اند.»
\par 20 و خداوندچهار آهنگر به من نشان داد.و گفتم: «اینان برای چه‌کار می‌آیند؟» او در جواب گفت: «آنهاشاخها می‌باشند که یهودا را چنان پراکنده نموده‌اند که احدی سر خود را بلند نمی تواند کردو اینها می‌آیند تا آنها را بترسانند و شاخهای امت هایی را که شاخ خود را بر زمین یهودابرافراشته، آن را پراکنده ساخته‌اند بیرون افکنند.»
\par 21 و گفتم: «اینان برای چه‌کار می‌آیند؟» او در جواب گفت: «آنهاشاخها می‌باشند که یهودا را چنان پراکنده نموده‌اند که احدی سر خود را بلند نمی تواند کردو اینها می‌آیند تا آنها را بترسانند و شاخهای امت هایی را که شاخ خود را بر زمین یهودابرافراشته، آن را پراکنده ساخته‌اند بیرون افکنند.»

\chapter{2}

\par 1 و چشمان خود را برافراشته، نگریستم ومردی که ریسمانکاری به‌دست خودداشت دیدم.
\par 2 و گفتم که «کجا می‌روی؟» او مراگفت: «به جهت پیمودن اورشلیم تا ببینم عرضش چه و طولش چه می‌باشد.»
\par 3 و اینک فرشته‌ای که با من تکلم می‌نمودبیرون رفت و فرشته دیگر برای ملاقات وی بیرون آمده،
\par 4 وی را گفت: «بشتاب و این جوان راخطاب کرده، بگو: اورشلیم به‌سبب کثرت مردمان و بهایمی که در اندرونش خواهند بود، مثل دهات بی‌حصار مسکون خواهد شد.
\par 5 وخداوند می‌گوید که من به اطرافش دیواری آتشین خواهم بود و در اندرونش جلال خواهم بود.
\par 6 هان هان خداوند می‌گوید از زمین شمال بگریزید زیرا که شما را مثل چهار باد آسمان پراکنده ساخته‌ام. قول خداوند این است.
\par 7 هان‌ای صهیون که با دختر بابل ساکن هستی، خویشتن را رستگار ساز.
\par 8 زیرا یهوه صبایوت که مرا بعد ازجلال نزد امت هایی که شما را غارت کردندفرستاده است، چنین می‌گوید که هر‌که شما رالمس نماید مردمک چشم او را لمس نموده است.
\par 9 «زیرا اینک من دست خود را بر ایشان خواهم افشاند و ایشان غارت بندگان خودشان خواهند شد و شما خواهید دانست که یهوه صبایوت مرا فرستاده است.
\par 10 ‌ای دختر صهیون ترنم نما و شادی کن زیرا خداوند می‌گوید که اینک می‌آیم و در میان تو ساکن خواهم شد.
\par 11 ودر آن روز امت های بسیار به خداوند ملصق شده، قوم من خواهند شد و من در میان تو سکنی خواهم گرفت و خواهی دانست که یهوه صبایوت مرا نزد تو فرستاده است.
\par 12 و خداوند یهودا را درزمین مقدس برای ملک خود به تصرف خواهدآورد و اورشلیم را بار دیگر خواهد برگزید.‌ای تمامی بشر به حضور خداوند خاموش باشید زیرا که او از مسکن مقدس خود برخاسته است.»
\par 13 ‌ای تمامی بشر به حضور خداوند خاموش باشید زیرا که او از مسکن مقدس خود برخاسته است.»

\chapter{3}

\par 1 و یهوشع رئیس کهنه را به من نشان داد که به حضور فرشته خداوند ایستاده بود و شیطان به‌دست راست وی ایستاده، تا با اومخاصمه نماید.
\par 2 و خداوند به شیطان گفت: «ای شیطان خداوند تو را نهیب نماید! خداوند که اورشلیم را برگزیده است تو را نهیب نماید. آیااین نیم سوزی نیست که از میان آتش ربوده شده است؟»
\par 3 و یهوشع به لباس پلید ملبس بود و به حضور فرشته ایستاده بود.
\par 4 و آنانی را که به حضور وی ایستاده بودند خطاب کرده، گفت: «لباس پلید را از برش بیرون کنید.» و او را گفت: «ببین عصیانت را از تو بیرون کردم و لباس فاخر به تو پوشانیدم.»
\par 5 و من گفتم که عمامه طاهر برسرش بگذارند. پس عمامه طاهر بر سرش گذاردند و او را به لباس پوشانیدند و فرشته خداوند ایستاده بود.
\par 6 و فرشته خداوند یهوشع را اعلام نموده، گفت:
\par 7 «یهوه صبایوت چنین می‌فرماید: اگر به طریق های من سلوک نمایی و ودیعت مرا نگاه داری تو نیز خانه مرا داوری خواهی نمود وصحن های مرا محافظت خواهی کرد و تو را درمیان آنانی که نزد من می‌ایستند بار خواهم داد.
\par 8 پس‌ای یهوشع رئیس کهنه بشنو تو و رفقایت که به حضور تو می‌نشینند، زیرا که ایشان مردان علامت هستند. (بشنوید) زیرا که اینک من بنده خود شاخه را خواهم آورد.
\par 9 و همانا آن سنگی که به حضور یهوشع می‌گذارم، بر یک سنگ هفت چشم می‌باشد. اینک یهوه صبایوت می‌گوید که من نقش آن را رقم خواهم کرد و عصیان این زمین را در یک روز رفع خواهم نمود.و یهوه صبایوت می‌گوید که هر کدام از شما همسایه خود را زیر مو و زیر انجیر خویش دعوت خواهید نمود.»
\par 10 و یهوه صبایوت می‌گوید که هر کدام از شما همسایه خود را زیر مو و زیر انجیر خویش دعوت خواهید نمود.»

\chapter{4}

\par 1 و فرشته‌ای که با من تکلم می‌نمود، برگشته، مرا مثل شخصی که از خواب بیدار شودبیدار کرد،
\par 2 و به من گفت: «چه چیز می‌بینی؟» گفتم: «نظر کردم و اینک شمعدانی که تمامش طلاست و روغندانش بر سرش و هفت چراغش بر آن است و هر چراغ که بر سرش می‌باشد هفت لوله دارد.
\par 3 و به پهلوی آن دو درخت زیتون که یکی بطرف راست روغندان و دیگری بطرف چپش می‌باشد.»
\par 4 و من توجه نموده، فرشته‌ای را که با من تکلم می‌نمود خطاب کرده، گفتم: «ای آقایم اینها چه می‌باشد؟»
\par 5 و فرشته‌ای که با من تکلم می‌نمودمرا جواب داد و گفت: «آیا نمی دانی که اینهاچیست؟» گفتم: «نه‌ای آقایم.»
\par 6 او در جواب من گفت: «این است کلامی که خداوند به زربابل می‌گوید: نه به قدرت و نه به قوت بلکه به روح من. قول یهوه صبایوت این است.
\par 7 ‌ای کوه بزرگ تو چیستی؟ در حضورزربابل به همواری مبدل خواهی شد و سنگ سرآن را بیرون خواهد آورد و صدا خواهند زد فیض فیض بر آن باشد.»
\par 8 و کلام خداوند بر من نازل شده، گفت:
\par 9 «دستهای زربابل این خانه را بنیاد نهاد و دستهای وی آن را تمام خواهد کرد و خواهی دانست که یهوه صبایوت مرا نزد شما فرستاده است.
\par 10 زیرا کیست که روز امور کوچک را خوار شمارد زیراکه این هفت مسرور خواهند شد حینی که شاغول را در دست زربابل می‌بینند. و اینها چشمان خداوند هستند که در تمامی جهان ترددمی نمایند.»
\par 11 پس من او را خطاب کرده، گفتم: «این دودرخت زیتون که بطرف راست و بطرف چپ شمعدان هستند چه می‌باشند؟»
\par 12 و بار دیگر او را خطاب کرده، گفتم که «این دو شاخه زیتون به پهلوی دو لوله زرینی که روغن طلا را از خود می‌ریزد چیستند؟»
\par 13 او مرا جواب داده، گفت: «آیا نمی دانی که اینها چیستند؟» گفتم: «نه‌ای آقایم.»گفت: «اینها پسران روغن زیت می‌باشند که نزد مالک تمامی جهان می‌ایستند.»
\par 14 گفت: «اینها پسران روغن زیت می‌باشند که نزد مالک تمامی جهان می‌ایستند.»

\chapter{5}

\par 1 و باز چشمان خود را برافراشته، نگریستم و طوماری پران دیدم.
\par 2 و او مرا گفت: «چه چیز می‌بینی؟» گفتم: «طوماری پران می‌بینم که طولش بیست ذراع و عرضش ده ذراع می‌باشد.»
\par 3 او مرا گفت: «این است آن لعنتی که بر روی تمامی جهان بیرون می‌رود زیرا که از این طرف هردزد موافق آن منقطع خواهد شد و از آن طرف هرکه سوگند خورد موافق آن منقطع خواهد گردید.
\par 4 یهوه صبایوت می‌گوید: من آن را بیرون خواهم فرستاد و به خانه دزد و به خانه هر‌که به اسم من قسم دروغ خورد داخل خواهد شد و در میان خانه‌اش نزیل شده، آن را با چوبهایش وسنگهایش منهدم خواهد ساخت.»
\par 5 پس فرشته‌ای که با من تکلم می‌نمود بیرون آمده، مرا گفت: «چشمان خود را برافراشته ببین که اینکه بیرون می‌رود چیست؟»
\par 6 گفتم: «این چیست؟» او جواب داد: «این است آن ایفایی که بیرون می‌رود و گفت نمایش ایشان در تمامی جهان این است.»
\par 7 و اینک وزنه‌ای از سرب برداشته شد. و زنی در میان ایفا نشسته بود.
\par 8 و او گفت: «این شرارت است.» پس وی را در میان ایفا انداخت و آن سنگ سرب را بر دهنه‌اش نهاد.
\par 9 پس چشمان خود رابرافراشته، نگریستم و اینک دو زن بیرون آمدند وباد در بالهای ایشان بود و بالهای ایشان مثل بالهای لق لق بود و ایفا را به میان زمین و آسمان برداشتند.
\par 10 پس به فرشته‌ای که با من تکلم می‌نمودگفتم: «اینها ایفا را کجا می‌برند؟»او مرا جواب داد: «تا خانه‌ای در زمین شنعار برای وی بنا نمایند و چون آن مهیا شودآنگاه او در آنجا بر پایه خود بر قرار خواهد شد.»
\par 11 او مرا جواب داد: «تا خانه‌ای در زمین شنعار برای وی بنا نمایند و چون آن مهیا شودآنگاه او در آنجا بر پایه خود بر قرار خواهد شد.»

\chapter{6}

\par 1 و بار دیگر چشمان خود را برافراشته، نگریستم و اینک چهار ارابه از میان دو کوه بیرون می‌رفت و کوهها کوههای مسین بود.
\par 2 درارابه اول اسبان سرخ و در ارابه دوم اسبان سیاه،
\par 3 و در ارابه سوم اسبان سفید و در ارابه چهارم اسبان ابلق قوی بود.
\par 4 و فرشته را که با من تکلم می‌نمود خطاب کرده، گفتم: «ای آقایم اینهاچیستند؟»
\par 5 فرشته در جواب من گفت: «اینها چهار روح افلاک می‌باشند که از ایستادن به حضور مالک تمامی جهان بیرون می‌روند.
\par 6 اماآنکه اسبان سیاه را دارد، اینها بسوی زمین شمال بیرون می‌روند و اسبان سفید در عقب آنهابیرون می‌روند و ابلقها به زمین جنوب بیرون می‌روند.»
\par 7 و آن اسبان قوی بیرون رفته، آرزو دارند که بروند و در جهان گردش نمایند؛ و او گفت: «بروید و در جهان گردش نمایید.» پس در جهان گردش کردند.
\par 8 و او به من ندا در‌داد و مرا خطاب کرده، گفت: «ببین آنهایی که به زمین شمال بیرون رفته‌اند، خشم مرا در زمین شمال فرو نشانیدند.»
\par 9 و کلام خداوند به من نازل شده، گفت:
\par 10 «ازاسیران یعنی از حلدای و طوبیا و یدعیا که از بابل آمده‌اند بگیر و در همان روز بیا و به خانه یوشیاابن صفیا داخل شو.
\par 11 پس نقره و طلا بگیر وتاجی ساخته، آن را بر سر یهوشع بن یهوصادق رئیس کهنه بگذار.
\par 12 و او را خطاب کرده، بگو: یهوه صبایوت چنین می‌فرماید و می‌گوید: اینک مردی که به شاخه مسمی است و از مکان خودخواهد رویید و هیکل خداوند را بنا خواهدنمود.
\par 13 پس او هیکل خداوند را بنا خواهد نمودو جلال را متحمل خواهد شد و بر کرسی اوجلوس نموده، حکمرانی خواهد کرد وبر کرسی او کاهن خواهد بود و مشورت سلامتی در میان هر دوی ایشان خواهد بود.
\par 14 و آن تاج برای حالم و طوبیا و یدعیا و حین بن صفنیا به جهت یادگاری در هیکل خداوند خواهد بود.وآنانی که دورند خواهند آمد و در هیکل خداوند بنا خواهند نمود و خواهید دانست که یهوه صبایوت مرا نزد شما فرستاده است و اگر قول یهوه خدای خویش را بکلی اطاعت نمایید این واقع خواهد شد.»
\par 15 وآنانی که دورند خواهند آمد و در هیکل خداوند بنا خواهند نمود و خواهید دانست که یهوه صبایوت مرا نزد شما فرستاده است و اگر قول یهوه خدای خویش را بکلی اطاعت نمایید این واقع خواهد شد.»

\chapter{7}

\par 1 و در سال چهارم داریوش پادشاه واقع شدکه کلام خداوند در روز چهارم ماه نهم که ماه کسلو باشد بر زکریا نازل شد.
\par 2 و اهل بیت ئیل یعنی شراصر و رجم ملک و کسان ایشان فرستاده بودند تا از خداوند مسالت نمایند.
\par 3 و به کاهنانی که در خانه یهوه صبایوت بودند و به انبیا تکلم نموده، گفتند: «آیا در ماه پنجم می‌باید که من گریه کنم و زهد ورزم چنانکه در این سالها کردم؟»
\par 4 پس یهوه صبایوت به من نازل شده، گفت:
\par 5 «تمامی قوم زمین و کاهنان را خطاب کرده، بگو: چون در این هفتاد سال در ماه پنجم و ماه هفتم روزه داشتید و نوحه گری نمودید، آیا برای من هرگز روزه می‌داشتید؟
\par 6 و چون می‌خورید وچون می‌نوشید، آیا به جهت خود نمی خورید وبرای خود نمی نوشید؟
\par 7 آیا کلامی را که خداوندبه واسطه انبیای سلف ندا کرد، هنگامی که اورشلیم مسکون و امن می‌بود و شهرهای مجاورش و جنوب و هامون مسکون می‌بود(نمی دانید)؟»
\par 8 و کلام خداوند بر زکریا نازل شده، گفت:
\par 9 «یهوه صبایوت امر فرموده، چنین می‌گوید: براستی داوری نمایید و با یکدیگر احسان و لطف معمول دارید.
\par 10 و بر بیوه‌زنان و یتیمان و غریبان و فقیران ظلم منمایید و در دلهای خود بر یکدیگر بدی میندیشید.
\par 11 اما ایشان از گوش گرفتن ابا نمودند و سرکشی کرده، گوشهای خودرا از شنیدن سنگین ساختند.
\par 12 بلکه دلهای خویش را (مثل ) الماس سخت نمودند تاشریعت و کلامی را که یهوه صبایوت به روح خود به واسطه انبیای سلف فرستاده بود نشنوند، بنابراین خشم عظیمی از جانب یهوه صبایوت صادر شد.
\par 13 پس واقع خواهد شد چنانکه او نداکرد و ایشان نشنیدند، همچنان یهوه صبایوت می‌گوید ایشان فریاد خواهند برآورد و من نخواهم شنید.و ایشان را بر روی تمامی امت هایی که نشناخته بودند، به گردباد پراکنده خواهم ساخت و زمین در عقب ایشان چنان ویران خواهد شد که کسی در آن عبور و ترددنخواهد کرد پس زمین مرغوب را ویران ساخته‌اند.»
\par 14 و ایشان را بر روی تمامی امت هایی که نشناخته بودند، به گردباد پراکنده خواهم ساخت و زمین در عقب ایشان چنان ویران خواهد شد که کسی در آن عبور و ترددنخواهد کرد پس زمین مرغوب را ویران ساخته‌اند.»

\chapter{8}

\par 1 و کلام یهوه صبایوت بر من نازل شده، گفت:
\par 2 «یهوه صبایوت چنین می‌فرماید: برای صهیون غیرت عظیمی دارم و با غضب سخت برایش غیور هستم.
\par 3 خداوند چنین می‌گوید: به صهیون مراجعت نموده‌ام و در میان اورشلیم ساکن خواهم شد و اورشلیم به شهر حق و کوه یهوه صبایوت به کوه مقدس مسمی خواهدشد.
\par 4 یهوه صبایوت چنین می‌گوید: مردان پیر وزنان پیر باز در کوچه های اورشلیم خواهندنشست و هر یکی از ایشان به‌سبب زیادتی عمرعصای خود را در دست خود خواهد داشت.
\par 5 وکوچه های شهر از پسران و دختران که درکوچه هایش بازی می‌کنند پر خواهد شد.
\par 6 یهوه صبایوت چنین می‌گوید: اگر این امر در این‌روزها به نظر بقیه این قوم عجیب نماید آیا در نظر من عجیب خواهد نمود؟ قول یهوه صبایوت این است.
\par 7 «یهوه صبایوت چنین می‌گوید: اینک من قوم خود را از زمین مشرق و از زمین مغرب آفتاب خواهم رهانید.
\par 8 و ایشان را خواهم آورد که دراورشلیم سکونت نمایند و ایشان قوم من خواهندبود و من براستی و عدالت خدای ایشان خواهم بود.
\par 9 یهوه صبایوت چنین می‌گوید: دستهای شما قوی شود‌ای کسانی که در این ایام این کلام را از زبان انبیا شنیدید که آن در روزی که بنیادخانه یهوه صبایوت را برای بنا نمودن هیکل نهادند واقع شد.
\par 10 زیرا قبل از این ایام مزدی برای انسان نبود و نه مزدی به جهت حیوان؛ و به‌سبب دشمن برای هر‌که خروج و دخول می‌کردهیچ سلامتی نبود و من همه کسان را به ضدیکدیگر واداشتم.
\par 11 اما الان یهوه صبایوت می‌گوید: من برای بقیه این قوم مثل ایام سابق نخواهم بود.
\par 12 زیرا که زرع سلامتی خواهد بودو مو میوه خود را خواهد داد و زمین محصول خود را خواهد آورد و آسمان شبنم خویش راخواهد بخشید و من بقیه این قوم را مالک جمیع این چیزها خواهم گردانید.
\par 13 و واقع خواهد شد چنانکه شما‌ای خاندان یهودا وای خاندان اسرائیل در میان امت‌ها (مورد)لعنت شده‌اید، همچنان شما را نجات خواهم داد تا (مورد) برکت بشوید؛ پس مترسید ودستهای شما قوی باشد.
\par 14 زیرا که یهوه صبایوت چنین می‌گوید: چنانکه قصد نمودم که به شما بدی برسانم حینی که پدران شماخشم مرا به هیجان آوردند و یهوه صبایوت می‌گوید که از آن پشیمان نشدم.
\par 15 همچنین در این‌روزها رجوع نموده، قصد خواهم نمود که به اورشلیم و خاندان یهودا احسان نمایم. پس ترسان مباشید.
\par 16 و این است کارهایی که بایدبکنید: با یکدیگر راست گویید و در دروازه های خود انصاف و داوری سلامتی را اجرا دارید.
\par 17 و در دلهای خود برای یکدیگر بدی میندیشید و قسم دروغ را دوست مدارید، زیراخداوند می‌گوید از همه این کارها نفرت دارم.»
\par 18 و کلام یهوه صبایوت بر من نازل شده، گفت:
\par 19 «یهوه صبایوت چنین می‌گوید: روزه ماه چهارم و روزه ماه پنجم و روزه ماه هفتم و روزه ماه دهم برای خاندان یهودا به شادمانی و سرور وعیدهای خوش مبدل خواهد شد پس راستی وسلامتی را دوست بدارید.
\par 20 یهوه صبایوت چنین می‌گوید: بار دیگر واقع خواهد شد که قوم‌ها و ساکنان شهرهای بسیار خواهند آمد.
\par 21 و ساکنان یک شهر به شهر دیگر رفته، خواهندگفت: بیایید برویم تا از خداوند مسالت نماییم ویهوه صبایوت را بطلبیم و من نیز خواهم آمد.
\par 22 و قوم های بسیار و امت های عظیم خواهندآمد تا یهوه صبایوت را در اورشلیم بطلبند و ازخداوند مسالت نمایند.یهوه صبایوت چنین می‌گوید در آن روزها ده نفر از همه زبانهای امت‌ها به دامن شخص یهودی چنگ زده، متمسک خواهند شد و خواهند گفت همراه شما می‌آییم زیرا شنیده‌ایم که خدا با شمااست.»
\par 23 یهوه صبایوت چنین می‌گوید در آن روزها ده نفر از همه زبانهای امت‌ها به دامن شخص یهودی چنگ زده، متمسک خواهند شد و خواهند گفت همراه شما می‌آییم زیرا شنیده‌ایم که خدا با شمااست.»

\chapter{9}

\par 1 وحی کلام خداوند بر زمین حدراخ (نازل می شود) و دمشق محل آن می‌باشد، زیراکه نظر انسان و نظر تمامی اسباط اسرائیل بسوی خداوند است.
\par 2 و بر حمات نیز که مجاور آن است و بر صور و صیدون اگر‌چه بسیار دانشمندمی باشد.
\par 3 و صور برای خود ملاذی منیع ساخت و نقره را مثل غبار و طلا را مانند گل کوچه هاانباشت.
\par 4 اینک خداوند او را اخراج خواهد کردو قوتش را که در دریا می‌باشد، تلف خواهدساخت و خودش به آتش سوخته خواهد شد.
\par 5 اشقلون چون این را بیند خواهد ترسید و غزه بسیار دردناک خواهد شد و عقرون نیز زیرا که اعتماد او خجل خواهد گردید و پادشاه از غزه هلاک خواهد شد و اشقلون مسکون نخواهدگشت.
\par 6 و حرام زاده‌ای در اشدود جلوس خواهدنمود و حشمت فلسطینیان را منقطع خواهم ساخت.
\par 7 و خون او را از دهانش بیرون خواهم آورد و رجاساتش را از میان دندانهایش؛ و بقیه اونیز به جهت خدای ما خواهد بود و خودش مثل امیری در یهودا و عقرون مانند یبوسی خواهدشد.
\par 8 و من گرداگرد خانه خود به ضد لشکر اردوخواهم زد تا کسی از آن عبور و مرور نکند و ظالم بار دیگر از میان آنها گذر نخواهد کرد زیرا که حال به چشمان خود مشاهده نموده‌ام.
\par 9 ‌ای دختر صهیون بسیار وجد بنما و ای دختر اورشلیم آواز شادمانی بده! اینک پادشاه تونزد تو می‌آید. او عادل و صاحب نجات و حلیم می‌باشد و بر الاغ و بر کره بچه الاغ سوار است.
\par 10 و من ارابه را از افرایم و اسب را از اورشلیم منقطع خواهم ساخت و کمان جنگی شکسته خواهد شد و او با امت‌ها به سلامتی تکلم خواهدنمود و سلطنت او از دریا تا دریا و از نهر تا اقصای زمین خواهد بود.
\par 11 و اما من اسیران تو را نیز به واسطه خون عهد تو از چاهی که در آن آب نیست رها کردم.
\par 12 ‌ای اسیران امید، به ملاذ منیع مراجعت نمایید. امروز نیز خبر می‌دهم که به شما(نصیب ) مضاعف رد خواهم نمود.
\par 13 زیرا که یهودا را برای خود زه خواهم کرد و افرایم راتیرکمان خواهم ساخت و پسران تو را‌ای صهیون به ضد پسران تو‌ای یاوان خواهم برانگیخت و تورا مثل شمشیر جبار خواهم گردانید.
\par 14 و خداوند بالای ایشان ظاهر خواهد شد وتیر او مانند برق خواهد جست و خداوند یهوه کرنا را نواخته، بر گردبادهای جنوبی خواهدتاخت.
\par 15 یهوه صبایوت ایشان را حمایت خواهد کرد و ایشان غذا خورده، سنگهای فلاخن را پایمال خواهند کردو نوشیده، مثل از شراب نعره خواهند زد و مثل جامها و مانند گوشه های مذبح پر خواهند شد.
\par 16 و یهوه خدای ایشان ایشان را در آن روز مثل گوسفندان قوم خودخواهد رهانید زیرا که مانند جواهر تاج بر زمین اوخواهند درخشید.زیرا که حسن و زیبایی اوچه قدر عظیم است. گندم جوانان را و عصیر انگور دوشیزگان را خرم خواهد ساخت.
\par 17 زیرا که حسن و زیبایی اوچه قدر عظیم است. گندم جوانان را و عصیر انگور دوشیزگان را خرم خواهد ساخت.

\chapter{10}

\par 1 باران را در موسم باران آخر از خداوندبطلبید. از خداوند که برقها را می‌سازدو او به ایشان باران فراوان هرکس در زمینش گیاه خواهد بخشید.
\par 2 زیرا که ترافیم سخن باطل می‌گویند و فالگیران رویاهای دروغ می‌بینند وخوابهای باطل بیان می‌کنند و تسلی بیهوده می‌دهند، از این جهت مثل گوسفندان آواره می‌باشند و از نبودن شبان ذلیل می‌گردند.
\par 3 خشم من بر شبانان مشتعل شده است و به بزهای نرعقوبت خواهم رسانید زیرا که یهوه صبایوت ازگله خود یعنی از خاندان یهودا تفقد خواهد نمودو ایشان را مثل اسب جنگی جلال خود خواهدگردانید.
\par 4 از او سنگ زاویه و از او میخ و از اوکمان جنگی و از او همه ستمکاران با هم بیرون می‌آیند.
\par 5 و ایشان مثل جباران (دشمنان خود را)در گل کوچه‌ها در عرصه جنگ پایمال خواهندکرد و محاربه خواهند نمود زیرا خداوند با ایشان است و اسب‌سواران خجل خواهند گردید.
\par 6 و من خاندان یهودا را تقویت خواهم کرد وخاندان یوسف را خواهم رهانید و ایشان را به امنیت ساکن خواهم گردانید، زیرا که بر ایشان رحمت دارم و چنان خواهند بود که گویا ایشان راترک ننموده بودم زیرا یهوه خدای ایشان من هستم؛ پس ایشان را اجابت خواهم نمود.
\par 7 وبنی افرایم مثل جباران شده، دل ایشان گویا ازشراب مسرور خواهد شد و پسران ایشان چون این را بینند شادی خواهند نمود و دل ایشان در خداوند وجد خواهد کرد.
\par 8 و ایشان را صدا زده، جمع خواهم کرد زیرا که ایشان را فدیه داده‌ام وافزوده خواهند شد چنانکه در قبل افزوده شده بودند.
\par 9 و ایشان را در میان قوم‌ها خواهم کاشت ومرا در مکان های بعید بیاد خواهند‌آورد و باپسران خود زیست نموده، مراجعت خواهندکرد.
\par 10 و ایشان را از زمین مصر باز خواهم آورد واز اشور جمع خواهم نمود و به زمین جلعاد ولبنان داخل خواهم ساخت و آن گنجایش ایشان را نخواهد داشت.
\par 11 و او از دریای مصیبت عبورنموده، امواج دریا را خواهد زد و همه ژرفیهای نهر خشک خواهد شد و حشمت اشور زایل خواهد گردید و عصای مصر نیست خواهد شد.و ایشان را در خداوند قوی خواهم ساخت ودر نام او سالک خواهند شد. قول خداوند این است.
\par 12 و ایشان را در خداوند قوی خواهم ساخت ودر نام او سالک خواهند شد. قول خداوند این است.

\chapter{11}

\par 1 ای لبنان درهای خود را باز کن تا آتش، سروهای آزاد تو را بسوزاند.
\par 2 ‌ای صنوبر ولوله نما زیرا که سرو آزاد افتاده است (ودرختان ) بلند خراب شده. ای بلوطهای باشان ولوله نمایید زیرا که جنگل منیع افتاده است.
\par 3 صدای ولوله شبانان است زیرا که جلال ایشان خراب شده؛ صدای غرش شیران ژیان است زیراکه شوکت اردن ویران گردیده است.
\par 4 یهوه خدای من چنین می‌فرماید که گوسفندان ذبح را بچران
\par 5 که خریداران ایشان آنها را ذبح می‌نمایند و مجرم شمرده نمی شوند وفروشندگان ایشان می‌گویند: خداوند متبارک باد زیرا که دولتمند شده‌ایم. و شبانان آنهابرایشان شفقت ندارند.
\par 6 زیرا خداوند می‌گوید: بر ساکنان این زمین بار دیگر ترحم نخواهم نمود واینک من هر کس از مردمان را به‌دست همسایه‌اش و به‌دست پادشاهش تسلیم خواهم نمود و زمین را ویران خواهند ساخت و از دست ایشان رهایی نخواهم بخشید.
\par 7 پس من گله ذبح یعنی ضعیف ترین گله را چرانیدم و دو عصا برای خود گرفتم که یکی از آنها را نعمه نامیدم ودیگری را حبال نام نهادم و گله را چرانیدم.
\par 8 و دریک ماه سه شبان را منقطع ساختم و جان من ازایشان بیزار شد و جان ایشان نیز از من متنفرگردید.
\par 9 پس گفتم شما را نخواهم چرانید. آنکه مردنی است بمیرد و آنکه هلاک شدنی است هلاک شود و باقی ماندگان گوشت یکدیگر رابخورند.
\par 10 پس عصای خود نعمه را گرفته، آن راشکستم تا عهدی را که با تمامی قوم‌ها بسته بودم شکسته باشم.
\par 11 پس در آن روز شکسته شد و آن ضعیف ترین گله که منتظر من می‌بودند فهمیدندکه این کلام خداوند است.
\par 12 و به ایشان گفتم: اگردر نظر شما پسند آید مزد مرا بدهید والا ندهید. پس به جهت مزد من، سی پاره نقره وزن کردند.
\par 13 و خداوند مرا گفت: آن را نزد کوزه‌گر بینداز، این قیمت گران را که مرا به آن قیمت کردند. پس سی پاره نقره را گرفته، آن را در خانه خداوند نزدکوزه‌گر انداختم.
\par 14 و عصای دیگر خود حبال راشکستم تا برادری را که در میان یهودا و اسرائیل بود شکسته باشم.
\par 15 و خداوند مرا گفت: «بار دیگر آلات شبان احمق را برای خود بگیر.
\par 16 زیرا اینک من شبانی را در این زمین خواهم برانگیخت که از هالکان تفقد نخواهد نمود و گم شدگان را نخواهد طلبیدو مجروحان را معالجه نخواهد کرد و ایستادگان را نخواهد پرورد بلکه گوشت فربهان را خواهدخورد و سمهای آنها را خواهد کند.وای برشبان باطل که گله را ترک می‌نماید. شمشیر بربازویش و بر چشم راستش فرود خواهد آمد وبازویش بالکل خشک خواهد شد و چشم راستش بکلی تار خواهد گردید.»
\par 17 وای برشبان باطل که گله را ترک می‌نماید. شمشیر بربازویش و بر چشم راستش فرود خواهد آمد وبازویش بالکل خشک خواهد شد و چشم راستش بکلی تار خواهد گردید.»

\chapter{12}

\par 1 وحی کلام خداوند درباره اسرائیل. قول خداوند است که آسمانها راگسترانید و بنیاد زمین را نهاد و روح انسان را دراندرون او ساخت.
\par 2 اینک من اورشلیم را برای جمیع قوم های مجاورش کاسه سرگیجش خواهم ساخت و این بر یهودا نیز حینی که اورشلیم را محاصره می‌کنند خواهد شد.
\par 3 و درآن روز، اورشلیم را برای جمیع قوم‌ها سنگی گران بار خواهم ساخت و همه کسانی که آن را برخود بار کنند، سخت مجروح خواهند شد وجمیع امت های جهان به ضد او جمع خواهندگردید.
\par 4 خداوند می‌گوید در آن روز من همه اسبان را به حیرت و سواران آنها را به جنون مبتلاخواهم ساخت. و چشمان خود را بر خاندان یهودا باز نموده، همه اسبان قوم‌ها را به کوری مبتلا خواهم کرد.
\par 5 و سروران یهودا در دل خودخواهند گفت که ساکنان اورشلیم در خدای خودیهوه صبایوت قوت من می‌باشند.
\par 6 در آن روزسروران یهودا را مثل آتشدانی در میان هیزم و مانند شعله آتش در میان بافه‌ها خواهم گردانید وهمه قوم های مجاور خویش را از طرف راست وچپ خواهند سوزانید و اورشلیم بار دیگر درمکان خود یعنی در اورشلیم مسکون خواهد شد.
\par 7 و خداوند خیمه های یهودا را اول خواهدرهانید تا حشمت خاندان داود و حشمت ساکنان اورشلیم بر یهودا فخر ننماید.
\par 8 در آن روزخداوند ساکنان اورشلیم را حمایت خواهد نمودو ضعیف ترین ایشان در آن روز مثل داود خواهدبود و خاندان داود مانند خدا مثل فرشته خداونددر حضور ایشان خواهند بود.
\par 9 و در آن روز قصدهلاک نمودن جمیع امت هایی که به ضد اورشلیم می‌آیند، خواهم نمود.
\par 10 و بر خاندان داود و بر ساکنان اورشلیم روح فیض و تضرعات را خواهم ریخت و بر من که نیزه زده‌اند خواهند نگریست وبرای من مثل نوحه گری برای پسر یگانه خود، نوحه گری خواهند نمود و مانند کسی‌که برای نخست زاده خویش ماتم گیرد، برای من ماتم تلخ خواهندگرفت.
\par 11 در آن روز ماتم عظیمی مانند ماتم هددرمون در همواری مجدون در اورشلیم خواهد بود.
\par 12 و اهل زمین ماتم خواهند گرفت هر قبیله علیحده، قبیله خاندان داود علیحده، وزنان ایشان علیحده، قبیله خاندان ناتان علیحده، وزنان ایشان علیحده.
\par 13 قبیله خاندان لاوی علیحده، و زنان ایشان علیحده، قبیله شمعی علیحده، و زنان ایشان علیحده،و جمیع قبایلی که باقی‌مانده باشند هر قبیله علیحده، و زنان ایشان علیحده.
\par 14 و جمیع قبایلی که باقی‌مانده باشند هر قبیله علیحده، و زنان ایشان علیحده.

\chapter{13}

\par 1 در آن روز برای خاندان داود و ساکنان اورشلیم چشمه‌ای به جهت گناه ونجاست مفتوح خواهد شد.
\par 2 و یهوه صبایوت می‌گوید در آن روز نامهای بتها را از روی زمین منقطع خواهم ساخت که بار دیگر آنها رابیادنخواهند‌آورد و انبیا و روح پلید را نیز از زمین دور خواهم کرد.
\par 3 و هر‌که بار دیگر نبوت نمایدپدر و مادرش که او را تولید نموده‌اند، وی راخواهند گفت که زنده نخواهی ماند زیرا که به اسم یهوه دروغ می‌گویی. و چون نبوت نماید پدر ومادرش که او را تولید نموده‌اند، وی را عرضه تیغ خواهند ساخت.
\par 4 و در آن روز هر کدام از آن انبیاچون نبوت می‌کنند، از رویاهای خویش خجل خواهند شد و جامه پشمین به جهت فریب دادن نخواهند پوشید.
\par 5 و هر یک خواهد گفت: من نبی نیستم بلکه زرع کننده زمین می‌باشم زیرا که ازطفولیت خود به غلامی فروخته شده‌ام.
\par 6 و او راخواهند گفت: این جراحات که در دستهای تومی باشد چیست؟ و او جواب خواهد داد آنهایی است که در خانه دوستان خویش به آنها مجروح شده‌ام.
\par 7 یهوه صبایوت می‌گوید: «ای شمشیر به ضدشبان من و به ضد آن مردی که همدوش من است برخیز! شبان را بزن و گوسفندان پراکنده خواهندشد و من دست خود را بر کوچکان خواهم برگردانید.»
\par 8 و خداوند می‌گوید که «در تمامی زمین دوحصه منقطع شده خواهند مرد و حصه سوم درآن باقی خواهد ماند.و حصه سوم را از میان آتش خواهم گذرانید و ایشان را مثل قال گذاشتن نقره قال خواهم گذاشت و مثل مصفی ساختن طلا ایشان را مصفی خواهم نمود و اسم مراخواهند خواند و من ایشان را اجابت نموده، خواهم گفت که ایشان قوم من هستند و ایشان خواهند گفت که یهوه خدای ما می‌باشد.»
\par 9 و حصه سوم را از میان آتش خواهم گذرانید و ایشان را مثل قال گذاشتن نقره قال خواهم گذاشت و مثل مصفی ساختن طلا ایشان را مصفی خواهم نمود و اسم مراخواهند خواند و من ایشان را اجابت نموده، خواهم گفت که ایشان قوم من هستند و ایشان خواهند گفت که یهوه خدای ما می‌باشد.»

\chapter{14}

\par 1 اینک روز خداوند می‌آید و غنیمت تودر میانت تقسیم خواهد شد.
\par 2 و جمیع امت‌ها را به ضد اورشلیم برای جنگ جمع خواهم کرد و شهر را خواهند گرفت و خانه‌ها راتاراج خواهند نمود و زنان را بی‌عصمت خواهندکرد و نصف اهل شهر به اسیری خواهند رفت وبقیه قوم از شهر منقطع نخواهند شد.
\par 3 و خداوندبیرون آمده، با آن قوم‌ها مقاتله خواهد نمودچنانکه در روز جنگ مقاتله نمود.
\par 4 و در آن روزپایهای او بر کوه زیتون که از طرف مشرق به مقابل اورشلیم است خواهد ایستاد و کوه زیتون درمیانش از مشرق تا مغرب منشق شده، دره بسیارعظیمی خواهد شد و نصف کوه بطرف شمال ونصف دیگرش بطرف جنوب منتقل خواهدگردید.
\par 5 و بسوی دره کوههای من فرار خواهیدکرد زیرا که دره کوهها تا به آصل خواهد رسید وشما خواهید گریخت چنانکه در ایام عزیا پادشاه یهودا از زلزله فرار کردید و یهوه خدای من خواهد آمد و جمیع مقدسان همراه تو (خواهندآمد).
\par 6 و در آن روز نور (آفتاب ) نخواهد بود وکواکب درخشنده، گرفته خواهند شد.
\par 7 و آن یک روز معروف خداوند خواهد بود. نه روز و نه شب، اما در وقت شام روشنایی خواهد بود.
\par 8 ودر آن روز، آبهای زنده از اورشلیم جاری خواهدشد (که ) نصف آنهابسوی دریای شرقی و نصف دیگر آنها بسوی دریای غربی (خواهد رفت ). درتابستان و در زمستان چنین واقع خواهد شد.
\par 9 ویهوه بر تمامی زمین پادشاه خواهد بود. در آن روزیهوه واحد خواهد بود و اسم او واحد.
\par 10 وتمامی زمین از جبع تا رمون که بطرف جنوب اورشلیم است متبدل شده، مثل عربه خواهدگردید و (اورشلیم ) مرتفع شده، در مکان خود ازدروازه بنیامین تا جای دروازه اول و تا دروازه زاویه و از برج حننئیل تا چرخشت پادشاه مسکون خواهد شد.
\par 11 و در آن ساکن خواهندشد و دیگر لعنت نخواهد بود و اورشلیم به امنیت مسکون خواهد شد.
\par 12 و این بلایی خواهد بود که خداوند بر همه قوم هایی که با اورشلیم جنگ کنند وارد خواهدآورد. گوشت ایشان در حالتی که بر پایهای خودایستاده باشند کاهیده خواهد شد و چشمانشان در حدقه گداخته خواهد گردید و زبان ایشان دردهانشان کاهیده خواهد گشت.
\par 13 و در آن روزاضطراب عظیمی از جانب خداوند در میان ایشان خواهد بود و دست یکدیگر را خواهند گرفت و دست هر کس به ضد دست دیگری بلند خواهدشد.
\par 14 و یهودا نیز نزد اورشلیم جنگ خواهدنمود و دولت جمیع امت های مجاور آن از طلا ونقره و لباس از حد زیاده جمع خواهد شد.
\par 15 وبلای اسبان و قاطران و شتران و الاغها و تمامی حیواناتی که در آن اردوها باشند همچنان ماننداین بلا خواهد بود.
\par 16 و واقع خواهد شد که همه باقی ماندگان از جمیع امت هایی که به ضداورشلیم آیند، هر سال برخواهند آمد تا یهوه صبایوت پادشاه را عبادت نمایند و عید خیمه هارا نگاه دارند.
\par 17 و هرکدام از قبایل زمین که به جهت عبادت یهوه صبایوت پادشاه برنیایند، برایشان باران نخواهد شد.
\par 18 و اگر قبیله مصربرنیایند و حاضر نشوند بر ایشان نیز (باران )نخواهد شد. این است بلایی که خداوند واردخواهد آورد بر امت هایی که به جهت نگاه داشتن عید خیمه‌ها برنیایند.
\par 19 این است قصاص مصر وقصاص همه امت هایی که به جهت نگاه داشتن عید خیمه‌ها برنیایند.و در آن روز بر زنگهای اسبان «مقدس خداوند» (منقوش ) خواهد شد و دیگها در خانه خداوند مثل کاسه های پیش مذبح خواهد بود.
\par 20 و در آن روز بر زنگهای اسبان «مقدس خداوند» (منقوش ) خواهد شد و دیگها در خانه خداوند مثل کاسه های پیش مذبح خواهد بود.


\end{document}