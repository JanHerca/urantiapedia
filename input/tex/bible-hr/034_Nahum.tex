\begin{document}

\title{Nahum}


\chapter{1}

\par 1 Proročanstvo nad Ninivom. Knjiga viđenja Nahuma Elkošanina. 
\par 2 Jahve je Bog ljubomoran i osvetnik! Jahve se osvećuje, gospodar srdžbe! Jahve se osvećuje svojim protivnicima, ustrajan u gnjevu na neprijatelje. 
\par 3 Jahve je spor u gnjevu, ali silan u moći. Ne, Jahve neće pustiti krivca nekažnjena. U vihoru i oluji put je njegov, oblaci su prašina koju podižu njegovi koraci. 
\par 4 Prijeti moru i isušuje ga, presušuje sve rijeke. ...Bašan i Karmel uvenuli su, povenuli su pupoljci Libana! 
\par 5 Pred njim se gore potresaju, bregovi se ljuljaju, zemlja se pod njim provaljuje, krug zemaljski i sve što na njem stanuje. 
\par 6 Tko može izdržati pred bijesom njegovim? Tko će odoljeti pred gnjevnom srdžbom njegovom? Jarost se njegova kao vatra izlijeva i litice se pred njim kidaju. 
\par 7 Jahve je dobar onima koji se u njeg' uzdaju, on je okrilje u dan nevolje, poznaje one koji se njemu utječu 
\par 8 kada potopne vode poplave. Uništit će one koji se protiv njega podižu, progonit će svoje dušmane u najmrkliji mrak. 
\par 9 Što vi snujete protiv Jahve? On uništava do kraja; nevolja se neće dva puta podići. 
\par 10 Kao trnovita šikara i kao pijanci na pijanki, k'o suha slama bit će potpuno smlavljeni. 
\par 11 Iz tebe je potekao onaj koji snuje zlo protiv Jahve, savjetnik Belijala. 
\par 12 Jahve ovako govori: "Neka su spremni, neka mnogobrojni, bit će pokošeni, uništeni. Ako sam te ponizio, neću te odsada ponižavati. 
\par 13 A sada, razbit ću jaram koji te steže, raskidat ću tvoje okove." 
\par 14 Protiv tebe Jahve naređuje: "Neće više biti roda tvoga imena, iz hrama tvojih bogova istrijebit ću likove rezane i livene, a od tvog groba ruglo ću učiniti." 
\par 15 (2:1) Gledajte, preko gora hrli glasnik, on naviješta: "Spasenje!" Svetkuj svoje blagdane, Judo, ispuni svoje zavjete, jer Belijal više neće prolaziti po tebi, on je sasvim zatrt. 


\chapter{2}

\par 1 (2:2) Protiv tebe dolazi rušitelj. Postavi stražu na bedeme, gledaj na put, opaši bedra, saberi sve svoje snage. 
\par 2 (2:3) Da, Jahve će obnoviti vinograd Jakovljev i vinograd Izraelov. Pljačkaši ih opljačkali, mladice im potrli. 
\par 3 (2:4) Štitovi njegovih junaka crvene se, njegovi su ratnici u grimizu; ognjem blista čelik na njihovim bojnim kolima kad krenu u boj; konji im se propinju. 
\par 4 (2:5) Po ulicama bjesne bojna kola, lete preko trgova; na pogled su baklje goruće; kao munje, samo sijevaju. 
\par 5 (2:6) Pozivaju se borci odabrani, bacaju se u rovove, hrle brzo na bedeme, već je zaklon postavljen. 
\par 6 (2:7) Vrata koja gledaju na Rijeku otvaraju se, strava je u palači. 
\par 7 (2:8) Podižu, u izgnanstvo odvode Gospodaricu, robinjice njene cvile, tuguju kao golubice, u prsa se udaraju. 
\par 8 (2:9) Niniva je nabujalo jezero, oni bježe pred vodom njezinom. "Zaustavite se, stanite!" Ali se nitko ne okreće. 
\par 9 (2:10) "Grabite srebro! Grabite zlato!" Blagu kraja nema, obilje dragocjenosti! 
\par 10 (2:11) Pljačkanje, haranje, razaranje! Srce zamire, koljena klecaju, u bedrima drhtavica, svima su lica poblijedjela. 
\par 11 (2:12) Gdje je skrovište lavovima i log lavićima? Kad je lav izlazio, lavica je ostajala i lavovi mališani; plašio ih nitko nije. 
\par 12 (2:13) Lav je grabio za svoje laviće, davio je za svoje lavice; svoje spilje punio je plijenom, svoja skrovišta lovinom. 
\par 13 (2:14) "Evo me! Tebi!" - riječ je Jahve nad Vojskama. "Pretvorit ću u dim tvoja bojna kola, mač će poklati tvoje laviće. Istrijebit ću sa zemlje tvoja pljačkanja, i neće se više čuti povik tvojih glasnika." 



\chapter{3}

\par 1 Teško gradu krvničkom, pun je laži, prepun grabeža, s pljačkanjem on ne prestaje! 
\par 2 Slušajte! Pucaju bičem! Slušajte! Štropot točkova! Konji upropanj, kola poskakuju. 
\par 3 Konjanici u stremenu, mačevi sjaju, koplja sijevaju ... gomile ranjenih, snopovi mrtvih, trupla unedogled, svuda se o truplo spotiče! 
\par 4 Eto plaće za razvrat bludnice, ljupke ljubaznice, vješte čarobnice koja je zavodila narode svojim razvratom i plemena svojim čaranjima. 
\par 5 "Evo me! Tebi!" - riječ je Jahve nad Vojskama. "Na tvoje lice podignut ću skute tvoje haljine, tvoju golotinju pokazat ću narodima, tvoju sramotu kraljevstvima. 
\par 6 Bacit ću na tebe smeće, osramotit ću te, izložiti na stup sramote. 
\par 7 Svaki koji te vidi, bježat će od tebe. Reći će: 'Niniva! Kakva razvalina!' Tko je može požaliti? Gdje joj naći tješitelje?" 
\par 8 Jesi li tvrđa od Tebe Amonove koja sjedi na rukavima Rijeke? Njezino predziđe bilo je more, njezini bedemi bile su vode. 
\par 9 Njezina snaga bila je Etiopija, Egipat; nije imala granica. Narodi Puta i Libije bili su joj pomoćnici. 
\par 10 A i ona je otišla u progonstvo, morala je ići u sužanjstvo; njezina nejaka djeca bila su razmrskana po svim raskršćima; za ugledne ljude njezine bacali su ždrijeb, svi njezini velikani okovani su lancima. 
\par 11 Tako ćeš i ti biti slomljena, bit ćeš svladana; tako ćeš i ti morati tražiti utočište pred dušmaninom. 
\par 12 Tvoje utvrde sve su kao smokvino stablo s urodom mladih smokava; kad se potrese stablo, smokve padaju u usta svakome koji ih želi jesti. 
\par 13 Gledaj svoj narod: sve je žensko u domu tvome; vrata tvoje zemlje širom se otvaraju neprijatelju; oganj je sažgao tvoje prijevornice. 
\par 14 Nacrpi vode za opsadu, utvrdi svoje bedeme, gnječi blato, gazi ilovaču, uzmi kalup za opeku. 
\par 15 A ipak će te oganj sažeći i mač potamaniti. Namnoži se kao kukci, namnoži se kao skakavci; 
\par 16 [16a] umnoži svoje trgovce da ih bude više nego zvijezda na nebu, 
\par 17 [17a] tvoje posade neka bude kao skakavaca, a tvojih pisara kao kobilica. Borave po zidovima kad je hladan dan. Sunce grane: [16b] kukci razvijaju krilašca i lete, [17b] i odlaze tko zna kamo. 
\par 18 Jao! Kako su zaspali tvoji pastiri, kralju asirski? Tvoji izabrani vojnici drijemaju, narod se tvoj raspršio po gorama, nitko ga više ne može sakupiti. 
\par 19 Tvojoj rani nema lijeka! Neizlječiva je tvoja ozljeda. Svi koji to saznaju plješću tvojoj razvalini. Tko nije bez sanka i prestanka osjećao na sebi tvoju okrutnost? 




\end{document}