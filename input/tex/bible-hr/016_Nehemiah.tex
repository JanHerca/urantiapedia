\begin{document}

\title{Nehemija}


\chapter{1}

\par 1 Povijest Nehemije, sina Hakalijina. Mjeseca Kisleva, dvadesete  godine, kad sam bio u Susi, 
\par 2 dođe Hanani, jedan od moje braće, s nekim ljudima iz Judeje. Ja ih zapitah o Židovima - o Ostatku  što se spasio od sužanjstva i o Jeruzalemu. 
\par 3 Oni mi odgovoriše:  "Ostatak, oni koji su nakon sužanjstva ostali u zemlji, u velikoj  su nevolji i sramoti. Jeruzalemski je zid sav razoren, a vrata  mu ognjem spaljena." 
\par 4 Kad sam čuo te vijesti, sjedoh i zaplakah.  Tugovao sam više dana, postio i molio se pred Bogom nebeskim. 
\par 5 I rekoh: "O, Jahve, Bože nebeski, veliki i strašni Bože  koji čuvaš Savez i naklonost onima koji te ljube i drže zapovijedi  tvoje! 
\par 6 Neka uho tvoje bude pažljivo i oči tvoje otvorene da  čuješ molitvu sluge svoga. Molim ti se sada, danju i noću, za  sinove Izraelove, sluge tvoje, i ispovijedam grijehe sinova Izraelovih  koje smo učinili protiv tebe; sagriješili smo i ja i kuća oca  mojega! 
\par 7 Veoma smo zlo činili prema tebi, ne držeći naredaba  tvojih, zakona i običaja koje si ti naredio po Mojsiju, sluzi  svome. 
\par 8 Ali se opomeni riječi koje si povjerio Mojsiju, sluzi  svome: 'Ako budete nevjerni, ja ću vas rasuti među narode; 
\par 9 ali  ako se obratite meni i budete poštovali moje naredbe i držali  ih, budu li neki od vas prognani i nakraj neba, ja ću vas sakupiti  i odvesti na mjesto koje sam izabrao da ondje prebiva moje Ime.' 
\par 10 A, eto, oni su tvoje sluge i tvoj narod, koji si izbavio  svojom velikom moći i snažnom rukom svojom. 
\par 11 Ah, Gospode,  neka uho tvoje bude pažljivo na molitvu sluge tvoga, na molitvu  slugu tvojih, koji su spremni bojati se tvoga Imena. Smjerno  te molim, udijeli danas sreću sluzi svome i učini da nađe milost  pred ovim čovjekom." A ja bijah tada peharnik kraljev. 


\chapter{2}

\par 1 Mjeseca Nisana, dvadesete godine kraljevanja Artakserksova, stajalo je vino pred kraljem. Uzeh ga i ponudih kralju. Nikada  pred njim nisam bio tužan. 
\par 2 Tada mi kralj reče: "Što ti je  tužno lice? Nisi li možda bolestan? Nije drugo nego je tuga u  tvome srcu!" Ja se veoma uplaših 
\par 3 i rekoh kralju: "Neka uvijek  živi kralj! Kako mi lice ne bi bilo tužno kad je grad gdje su  grobovi mojih otaca razoren, a vrata mu ognjem spaljena?" 
\par 4 Kralj  me upita: "Što, dakle, želiš?" Zazvah Boga nebeskoga 
\par 5 i odgovorih kralju: "Ako je kralju  po volji i ako ti je mio sluga tvoj, pusti me da odem u Judeju, u grad grobova mojih otaca, da ga obnovim." 
\par 6 Kralj me upita  pred kraljicom, koja je sjedila kraj njega: "Koliko bi trajao  tvoj put? Kada ćeš se vratiti?" Pošto sam utvrdio vrijeme koje  je odgovaralo kralju, pusti me da odem. 
\par 7 Još rekoh kralju: "Ako je kralju po volji, mogao bih ponijeti  pisma upraviteljima s onu stranu Rijeke da me propuste do Judeje; 
\par 8 i pismo Asafu, nadgledniku kraljeve šume, da mi dadne drva  za gradnju vrata na tvrđi Hrama, za gradski bedem i za kuću u  kojoj ću se nastaniti." I dade mi kralj, jer dobrostiva ruka Boga moga bijaše nada  mnom. 
\par 9 I dođoh tako k upraviteljima s onu stranu Rijeke i dadoh  im kraljeva pisma. A kralj posla sa mnom časnike i konjanike. 
\par 10 Kad to ču Sanbalat, Horonac, i sluga Tobija, Amonac,  bi im vrlo mrsko što je došao čovjek da se zauzme za dobro Izraelaca. 
\par 11 Stigavši u Jeruzalem, ostadoh ondje tri dana. 
\par 12 Zatim  ustah noću, u pratnji nekoliko ljudi, nikomu ne povjerivši što  mi je Bog moj nadahnuo da učinim za Jeruzalem; a nisam imao druge  životinje osim kljuseta na kojem sam jahao. 
\par 13 Iziđoh, dakle, noću na Dolinska vrata i uputih se Zmajevskom izvoru, a zatim  prema Smetlišnim vratima: razgledao sam jeruzalemski zid gdje  je bio razoren i vrata koja su bila spaljena. 
\par 14 Nastavio sam  put prema Izvorskim vratima i Kraljevskom ribnjaku, ali nisam  našao prolaza za životinju na kojoj sam jahao. 
\par 15 Uspeo sam  se zato noću uz Potok, i dalje razgledajući zid, i ponovo sam  ušao na Dolinska vrata. Tako sam se vratio, 
\par 16 a da savjetnici  nisu primijetili kamo sam otišao i što sam učinio. Sve do sada  nisam ništa rekao Židovima: ni svećenicima, ni velikašima, ni  savjetnicima, ni drugima nadstojnicima. 
\par 17 Tada im rekoh: "Vidite u kakvoj smo nevolji: Jeruzalem  je u ruševinama, a vrata mu spaljena. Hajte, sagradimo jeruzalemski  zid da više ne budemo izloženi ruglu." 
\par 18 I objasnih im kako  je dobrostiva ruka Boga moga bila nada mnom, a saopćih im i riječi  koje mi kralj bijaše rekao. "Ustanimo", povikaše oni, "i gradimo!"  I ukrijepiše im se ruke na dobro djelo. 
\par 19 Na te vijesti počeše nam se rugati Sanbalat, Horonac, i sluga Tobija, Amonac, i Gešem, Arapin. Prezirno su nam govorili:  "Što radite ovdje? Hoćete li se pobuniti protiv kralja?" 
\par 20 Ali  im ja odgovorih ovim riječima: "Nebeski će nam Bog dati da uspijemo.  Mi, sluge njegove, ustasmo da gradimo. A vi nemate ni dijela, ni prava, ni spomena u Jeruzalemu." 


\chapter{3}

\par 1 Tada usta veliki svećenik Elijašib sa svojom braćom svećenicima  te sagradiše Ovčja vrata. Posvetiše ih, postaviše im krila i  nastaviše graditi sve do kule Meaha i do Hananelove kule. 
\par 2 Kraj  njih su gradili Jerihonci, a do njih je gradio Zakur, sin Imrijev. 
\par 3 Sinovi Hasnaini gradili su Riblja vrata, stavili dovratke, utvrdili krila, stožere i prijevornice. 
\par 4 Kraj njih je popravljao  Merimot, sin Urije, sina Hakosova; a do njega je popravljao Mešulam, sin Berekje, sina Mešezabelova; a do njega je popravljao Sadok, sin Baanin. 
\par 5 Kraj njih su popravljali Tekoanci, ali su njihovi  plemenitaši odbili da prignu šiju na službu svojim gospodarima. 
\par 6 Stara vrata popravljali su Jojada, sin Paseahov, i Mešulam, sin Besodjin. Oni su stavili dovratke, učvrstili krila, stožere  i prijevornice. 
\par 7 A kraj njih obnavljahu Melatja Gibeonjanin, Jadon Meronoćanin i ljudi iz Gibeona i Mispe, podložnici upravitelja  s onu stranu Rijeke. 
\par 8 A do njih je popravljao Uziel, Harhajin  sin, zlatar, a do njega je popravljao Hananija, jedan od pomastara:  oni su utvrdili Jeruzalem sve do Širokog zida. 
\par 9 Do njih je popravljao Refaja, sin Hurov, glavar polovice  jeruzalemskog okruga. 
\par 10 A do njega je popravljao Jedaja, sin  Harumafov, pred svojom kućom; a do njega je popravljao Hatuš, sin Hašabnejin. 
\par 11 Malkija, sin Harimov, i Hašub, sin Pahat-Moabov, popravljali su dio sve do Pećne kule. 
\par 12 A do njih je popravljao  Šalum, sin Halohešov, glavar polovice okruga, on i njegovi sinovi. 
\par 13 Dolinska vrata popravljao je Hanum i stanovnici Zanoaha:  sagradili su ih, učvrstili krila, stožere i prijevornice i postavili  tisuću lakata zida do Smetlišnih vrata. 
\par 14 Smetlišna vrata popravljao  je Malkija, sin Rekabov, glavar bethakeremskog okruga, sa svojim  sinovima: učvrstili su krila, stožere i prijevornice. 
\par 15 Izvorska vrata popravljao je Šalum, sin Kol-Hozeov, glavar  nad mispanskim okrugom: sagradio ih je, pokrio ih, utvrdio vratna  krila, stožere i prijevornice. On je popravio i zid kod ribnjaka  Šiloaha, koji se proteže od Kraljevskog vrta do stepenica što  silaze iz Davidova grada. 
\par 16 Za njim je popravljao Nehemija, sin Azbukov, glavar nad polovicom betsurskog okruga, sve do  nasuprot Davidovim grobnicama i do umjetnog ribnjaka i Vojarne. 
\par 17 Za njim su popravljali leviti: Rehum, sin Banijev; a do njega  je popravljao Hašabja, glavar nad polovicom keilskog okruga,  za svoje područje. 
\par 18 Do njih su popravljala njihova braća:  Bavaj, sin Henadadov, glavar nad polovicom keilskog kotara; 
\par 19 a  do njega Ezer, sin Ješuin, glavar Mispe, popravljao je drugi  dio, sučelice usponu prema Oružarnici na uglu. 
\par 20 Za njim je popravljao Baruk, sin Zabajev, i popravio  je drugi dio, od ugla do kućnih vrata velikog svećenika Elijašiba. 
\par 21 Za njim je popravljao Meremot, sin Urije, sina Hakosova,  drugi dio: od Elijašibova kućnog ulaza do kraja Elijašibove kuće. 
\par 22 Za njima su radili na popravcima svećenici koji su živjeli  u Okružju. 
\par 23 Za njima su pak popravljali Benjamin i Hašub sučelice  svojim kućama. Za njima je popravljao Azarja, sin Ananijina sina  Maaseje, nasuprot svojoj kući. 
\par 24 Za njima je popravljao Binuj, sin Henadadov, drugi dio - od Azarjine kuće do ugla, do zidnog  kruništa. 
\par 25 Palal, sin Uzajev, popravljao je nasuprot uglu  i kuli koja se uzdiže iznad Gornje kraljevske palače, a nalazi  se prema dvorištu Tamnice. Za njim je Pedaja, sin Parošev, popravljao 
\par 26 sve do Vodenih vrata, u smjeru istoka, i sve do pred Uzdignutu  kulu. 
\par 27 Za njima su popravljali Tekoanci drugi dio nasuprot  velikoj Uzdignutoj kuli, sve do Ofelskog zida. 
\par 28 Od Konjskih vrata popravljali su svećenici, svaki nasuprot  svojoj kući. 
\par 29 Za njima je Sadok, sin Imerov, popravljao nasuprot  svojoj kući. Za njim je popravljao Šemaja, sin Šekanijin, čuvar  Istočnih vrata. 
\par 30 Za njim su Hananija, sin Šelemjin, i Hanun, šesti sin Salafov, popravljali drugi dio. Za njima je popravljao  Mešulam, sin Berekjin, nasuprot svome stanu. 
\par 31 Za njim je Malkija, zlatar, popravljao sve do prebivališta netinaca i trgovaca,  nasuprot Nadgledničkim vratima do Gornje dvorane na zidnom kruništu. 
\par 32 A zlatari su i trgovci popravljali između Gornje dvorane  na zidnom kruništu do Ovčjih vrata. 


\chapter{4}

\par 1 (3:33) Kad je Sanbalat čuo da gradimo zid, razljutio se. Bio  je veoma srdit, ismijavao je Židove 
\par 2 (3:34) i vikao je pred svojom  braćom i samarijanskom vojskom: "Što poduzimaju ovi jadni Židovi?  Kane li možda popraviti, žrtvovati i završiti sve u jedan dan?  Zar će iz hrpe praha dozvati u život spaljeno kamenje?" 
\par 3 (3:35) Tobija, Amonac, koji je bio uz njega, reče: "Neka samo  grade! Ali popne li se lisica, srušit će im kamene zidove." 
\par 4 (3:36) Čuj, o Bože naš, kako nas preziru! Navrni njihove poruge  na njihovu glavu. Predaj ih kao plijen u zemlju ropstva. 
\par 5 (3:37) Ne  pokrivaj njihova bezakonja i grijeh njihov neka ne bude izbrisan  pred licem tvojim jer su se rugali graditeljima. 
\par 6 (3:38) Tako smo gradili zid, koji je uskoro bio završen do pola  visine. Narod je imao oduševljenja za rad. 
\par 7 (4:1) Kad su Sanbalat, Tobija, Arapi, Amonci i Ašdođani čuli da napreduje  popravljanje jeruzalemskih zidova - jer su se počele zatvarati  pukotine - veoma se ražestiše. 
\par 8 (4:2) Zakleše se svi zajedno da će  napasti Jeruzalem i da će nas smesti. 
\par 9 (4:3) Mi smo tada zazvali Boga našega i postavljali smo dnevnu  i noćnu stražu da bismo zaštitili grad. 
\par 10 (4:4) A Židovi govorahu:  "Snage su nosačima klonule, a ruševina je mnogo: nećemo nikada  stići sagraditi zida!" 
\par 11 (4:5) A naši neprijatelji rekoše: "Uvući  ćemo se među njih prije nego što doznaju i opaze nas: tada ćemo  ih poubijati i tako osujetiti pothvat!" 
\par 12 (4:6) A kad bi došli Židovi  koji žive kraj njih, po deset bi nas puta upozoravali: "Idu protiv  vas iz svih mjesta u kojima stanuju!" 
\par 13 (4:7) Postavili smo se u nizinama, iza zida i na goletima; rasporedio  sam narod po rodovima, s mačevima, kopljima i lukovima. 
\par 14 (4:8) Kad  sam vidio kako se boje, ustao sam i objavio velikašima, odličnicima  i ostalom narodu ovo: "Ne bojte se ovih ljudi! Mislite na Gospoda, velikoga i strašnoga, i borite se za svoju braću, za sinove  i kćeri svoje, za žene i kuće svoje!" 
\par 15 (4:9) Kad su naši neprijatelji čuli da smo obaviješteni i da  je Bog osujetio njihovu osnovu, mogli smo se vratiti k zidu,  svaki svome poslu. 
\par 16 (4:10) Ali je od toga dana samo polovica mojih momaka obavljala  posao, a ostali su držali koplja, štitove, lukove i oklope, a  glavari stajali iza doma Judina, 
\par 17 (4:11) koji je gradio zid. I nosači  tereta držali su oružje: jednom je rukom svaki radio svoj posao, a u drugoj mu bilo oružje. 
\par 18 (4:12) Svaki je od graditelja, dok je  radio, nosio mač pripasan uz bok. Trubač je stajao kraj mene. 
\par 19 (4:13) Rekao sam velikašima, odličnicima i ostalom narodu: "Posao  je velik i zamašan, a mi se rasuli po zidu, daleko jedni od drugih: 
\par 20 (4:14) skupite se oko nas na mjesto gdje čujete glas trube, a Bog  naš borit će se za nas." 
\par 21 (4:15) Tako smo obavljali posao od rane  zore do prvih zvijezda. Polovica je bila naoružana kopljima. 
\par 22 (4:16) U to sam vrijeme još rekao narodu: "Svaki sa svojim slugom  neka noći u Jeruzalemu: po redu ćemo noću stražariti, a danju  raditi." 
\par 23 (4:17) Ni ja, ni moja braća, ni moji momci, ni stražari  koji su me pratili nismo skidali svojih haljina, svatko je držao  pri ruci svoje oružje. 


\chapter{5}

\par 1 Velika se vika digla među ljudima i ženama protiv njihove braće  Židova. 
\par 2 Jedni su govorili: "Zalažemo svoje sinove i kćeri  da bismo mogli nabaviti pšenice te jesti i živjeti." 
\par 3 Drugi  su govorili: "Zalažemo svoja polja, vinograde svoje i kuće svoje  da bismo mogli nabaviti pšenice za vrijeme gladi." 
\par 4 Drugi su  opet govorili: "Moramo uzaimati novac na polja svoja i vinograde  da bismo mogli isplatiti kraljeve namete. 
\par 5 Tijelo je naše kao  tijelo braće naše, sinovi su naši kao i njihovi, a mi moramo  predavati u ropstvo svoje sinove i kćeri; među našim kćerima  neke su već robinje! A mi ne možemo ništa jer polja naša i vinograde  drže drugi." 
\par 6 Razljutio sam se veoma kad sam čuo njihovu viku i te riječi. 
\par 7 Pošto sam u sebi promislio, prekorio sam velikaše i odličnike  riječima: "Vi namećete teret svojoj braći!" I sazvao sam protiv  njih velik zbor. 
\par 8 I rekao sam: "Mi smo, koliko smo mogli, otkupili  svoju židovsku braću koja bijahu prodana poganima. A sada vi  prodajete svoju braću da bismo ih otkupili!" Svi su šutjeli i  nitko nije odgovorio. 
\par 9 Nastavio sam: "Nije dobro to što činite.  Ne treba li da hodite u strahu Boga našega da se tako uklonimo  ruglu neprijateljskih naroda? 
\par 10 I ja, i moja braća, i moji  momci davali smo im novaca i žita. Ali smo im dug oprostili. 
\par 11 Vratite im i vi još danas njihova polja, vinograde, maslinike  i kuće njihove i oprostite im postotak u novcu, u žitu, u vinu, u ulju, što ste im ga nametnuli." 
\par 12 A oni odgovoriše: "Vratit ćemo; nećemo od njih ništa  tražiti. Učinit ćemo kako si rekao." Tada pozvah svećenike i  naredih neka se zakunu da će učiniti kako su obećali. 
\par 13 Zatim  istresoh skute svoje odjeće govoreći: "Neka Bog ovako istrese  iz vlastite kuće i imanja svakog čovjeka koji se ne bude držao  ovog obećanja! Tako bio istresen i ispražnjen!" A sav zbor odgovori  "Amen!" hvaleći Jahvu. I narod je učinio prema ovom dogovoru. 
\par 14 I od dana kad mi je kralj naredio da budem upravitelj  u zemlji Judinoj, od dvadesete do trideset i druge godine kraljevanja  Artakserksa, za dvanaest godina ja i moja braća nismo nikada  jeli upraviteljskog kruha. 
\par 15 Ali prijašnji upravitelji, moji  prethodnici, ugnjetavahu narod: svakoga su dana od njega uzimali  četrdeset šekela srebra za kruh; i njihove su sluge ugnjetavale  narod. A ja nisam nikada tako činio, zbog straha Božjega. 
\par 16 Čak  sam se jednako držao posla oko zida i nisam kupio ni jedne njive!  Svi su moji momci bili ondje okupljeni na poslu. 
\par 17 Za mojim su stolom jeli Židovi i odličnici, njih stotinu  i pedeset na broju, osim onih koji su k nama dolazili iz okolnih  naroda. 
\par 18 Svakoga se dana o mom trošku pripremalo jedno goveče, šest biranih ovaca i peradi; svakih deset dana donosilo se obilje  vina za sve. A opet nisam nikada tražio upraviteljskog poreza  na kruh, jer je narod već bio teško opterećen. 
\par 19 Spomeni se, Bože moj, za moje dobro svega što sam učinio  ovome narodu! 


\chapter{6}

\par 1 Kad su Sanbalat, Tobija, Gešem Arapin i ostali naši neprijatelji  dočuli da sam obnovio zid i da nije u njemu ostalo pukotine -  do toga vremena nisam zapravo bio namjestio krila na vratima  - 
\par 2 poručiše mi Sanbalat i Gešem: "Dođi da se sastanemo u Kefiri, u Dolini ononskoj." Ali su mi oni zlo mislili. 
\par 3 Zato sam im  poslao glasnike s ovim odgovorom: "Zauzet sam velikim poslom  i ne mogu sići: posao bi zastao kad bih ga ostavio da dođem k  vama!" 
\par 4 Četiri su mi puta slali isti poziv i ja sam im odvraćao  isti odgovor. 
\par 5 Tada, peti put, s istom nakanom, posla mi Sanbalat  svoga slugu s otvorenim pismom. 
\par 6 U njemu je pisalo: "Čuje se  u narodima - a Gašmu potvrđuje - da se ti i Židovi spremate na  bunu; zato da i gradiš zid i da želiš postati njihovim kraljem, kako vele. 
\par 7 I da si postavio proroke da proglase tvoj uspjeh  u Jeruzalemu i da kažu: Judeja ima kralja! Sada će ti glasovi  stići kralju do ušiju: zato dođi da se posavjetujemo." 
\par 8 Ali sam mu ja odgovorio: "Ništa nije tako kao što tvrdiš;  sve je to samo izmišljotina tvoga srca." 
\par 9 Jer su nas oni htjeli  uplašiti govoreći: "Klonut će im ruke od posla i neće ga završiti  nikada." A ja sam, naprotiv, ukrijepio ruke svoje! 
\par 10 Pošao sam Šemaji, sinu Delaje, sina Mehetabelova, koji  se bijaše zatvorio u svojoj kući. On mi objavi: "Nađimo se u Domu Božjemu, usred Hekala, i zatvorimo vrata Hekala jer će doći da te ubiju. Jest, još noćas doći će da te ubiju!" 
\par 11 A ja odgovorih: "Zar da bježi čovjek kao što sam ja?  Koji čovjek, meni sličan, može ući u Hekal i ostati živ? Ne,  ja ne idem." 
\par 12 I tada razabrah: nije ga poslao Bog, nego mi  je objavio proroštvo, jer su ga Tobija i Sanbalat podmitili, 
\par 13 da bih, uplašen, učinio onako te sagriješio. To bi im poslužilo  da me ozloglase i da mi se rugaju! 
\par 14 Sjeti se, Bože moj, Tobije  i Sanbalata prema ovim njihovim djelima, a i proročice Noadje  i ostalih proroka što me htjedoše uplašiti. 
\par 15 Zid je završen dvadeset i petog Elula, za pedeset i dva  dana. 
\par 16 A kad su čuli svi naši neprijatelji i vidjeli svi pogani  oko nas, bilo je to čudo u očima njihovim, jer su shvatili da  je Bog naš učinio to djelo. 
\par 17 A onih dana mnogi su židovski velikaši često slali svoja  pisma Tobiji i mnoga su primali od Tobije. 
\par 18 Jer u Judeji bijahu  mnogi s njime zakletvom povezani: tÓa bio je u rodu sa Šekanijom, sinom Arahovim, i sinom njegovim Johananom, koji je uzeo za  ženu kćer Mešulama, sina Berekjina. 
\par 19 I veličali su preda mnom  njegova djela, a njemu prenosili moje riječi. Zato je Tobija  i slao pisma da me uplaši. 


\chapter{7}

\par 1 A kad je zid bio sagrađen i kad sam namjestio vratna krila, postavljeni su čuvari na vratima i pjevači i leviti. 
\par 2 Upravu  sam Jeruzalema povjerio Hananiju, svome bratu, i Hananiji, zapovjedniku  tvrđave, jer je ovaj bio čovjek povjerenja i bojao se Boga kao  malo tko. 
\par 3 Rekao sam im: "Jeruzalemska vrata neka se ne otvaraju  dok sunce ne ogrije; a dok ono bude još visoko, neka ih zatvore  i prebace prijevornice. Treba postaviti straže uzete između žitelja  jeruzalemskih: svakoga na njegovo mjesto, svakoga nasuprot njegovoj  kući. 
\par 4 Grad je bio prostran i velik, ali je u njemu bilo malo  stanovnika jer nije bilo sagrađenih kuća. 
\par 5 A Bog me moj nadahnuo  te sam skupio velikaše, odličnike i narod da se unesu u rodovnike.  Tada sam našao rodovnik onih koji su se prije vratili. U njemu  nađoh zapisano: 
\par 6 Evo ljudi iz pokrajine koji su došli iz sužanjstva u koje  ih bijaše odveo Nabukodonozor, babilonski kralj. Vratili su se  u Jeruzalem i Judeju, svaki u svoj grad. 
\par 7 Došli su sa Zerubabelom, Ješuom, Nehemijom, Azarjom, Raamjom, Nahamanijem, Mordokajem, Bilšanom, Misperetom, Bigvajem, Nehumom, Baanom. Broj ljudi naroda Izraelova: 
\par 8 Paroševih sinova: dvije tisuće  stotinu sedamdeset i dva; 
\par 9 sinova Šefatjinih: tri stotine sedamdeset  i dva; 
\par 10 Arahovih sinova: šest stotina pedeset i dva! 
\par 11 Pahat-Moabovih  sinova, to jest Ješuinih i Joabovih sinova: dvije tisuće osam  stotina i osamnaest; 
\par 12 sinova Elamovih: tisuću dvjesta pedeset  i četiri; 
\par 13 Zatuovih sinova: osam stotina četrdeset i pet; 
\par 14 sinova Zakajevih: sedam stotina i šezdeset; 
\par 15 Binujevih  sinova: šest stotina četrdeset i osam; 
\par 16 sinova Bebajevih:  šest stotina dvadeset i osam; 
\par 17 Azgadovih sinova: dvije tisuće  tri stotine dvadeset i dva; 
\par 18 sinova Adonikamovih: šest stotina  šezdeset i sedam; 
\par 19 Bigvajevih sinova: dvije tisuće šezdeset  i sedam; 
\par 20 sinova Adinovih: šest stotina pedeset i pet; 
\par 21 Aterovih  sinova, to jest od Ezekije: devedeset i osam; 
\par 22 sinova Hašumovih:  trista dvadeset i osam; 
\par 23 Besajevih sinova: trista dvadeset  i četiri; 
\par 24 sinova Harifovih: stotinu i dvanaest; 
\par 25 Gibeonovih  sinova: devedeset i pet; 
\par 26 ljudi iz Betlehema i Netofe: stotinu  osamdeset i osam; 
\par 27 ljudi iz Anatota: stotinu dvadeset i osam; 
\par 28 ljudi iz Bet Azmaveta: četrdeset i dva; 
\par 29 ljudi iz Kirjat  Jearima, Kefire i Beerota: sedam stotina četrdeset i tri; 
\par 30 ljudi  iz Rame i Gabe: šest stotina dvadeset i jedan; 
\par 31 ljudi iz Mikmasa:  stotinu dvadeset i dva; 
\par 32 ljudi iz Betela i Aja: stotinu dvadeset  i tri; 
\par 33 ljudi iz Neba: pedeset i dva; 
\par 34 sinova drugoga Elama:  tisuću dvjesta pedeset i četiri; 
\par 35 Harimovih sinova: trista  dvadeset; 
\par 36 ljudi iz Jerihona: trista četrdeset i pet; 
\par 37 ljudi  iz Loda, Hadida i Onona: sedam stotina dvadeset i jedan; 
\par 38 sinova  Senajinih: tri tisuće devet stotina i trideset. 
\par 39 Svećenika: sinova Jedajinih, to jest iz kuće Ješuine:  devet stotina sedamdeset i tri; 
\par 40 Imerovih sinova: tisuću pedeset  i dva; 
\par 41 sinova Fašhurovih: tisuću dvjesta četrdeset i sedam; 
\par 42 Harimovih sinova: tisuću i sedamnaest. 
\par 43 Levita: Ješuinih sinova, to jest Kadmielovih i Hodvinih:  sedamdeset i četiri. 
\par 44 Pjevača: Asafovih sinova: stotinu četrdeset i osam. 
\par 45 Vratara: sinova Šalumovih, sinova Aterovih, sinova Talmonovih, sinova Akubovih, Hatitinih sinova, sinova Šobajevih: stotinu  trideset i osam. 
\par 46 Netinaca: sinova Sihinih, sinova Hasufinih, sinova Tabaotovih, 
\par 47 sinova Kerosovih, sinova Sijajevih, sinova Fadonovih, 
\par 48 sinova  Lebaninih, sinova Hagabinih, sinova Šalmajevih, 
\par 49 sinova Hananovih, sinova Gidelovih, sinova Gaharovih, 
\par 50 sinova Reajinih, sinova  Resinovih, sinova Nekodinih, 
\par 51 sinova Gazamovih, sinova Uzinih, sinova Fasealovih, 
\par 52 sinova Besajevih, sinova Merinimovih, sinova Nefišesimovih, 
\par 53 sinova Bakbukovih, sinova Hakufinih, sinova Harhurovih, 
\par 54 sinova Baslitovih, sinova Mehidinih,  sinova Haršinih, 
\par 55 sinova Barkošovih, sinova Sisrinih, sinova  Tamahovih, 
\par 56 sinova Nasijahovih, sinova Hatifinih. 
\par 57 Sinova Salomonovih slugu: sinova Sotajevih, sinova Soferetovih, sinova Feridinih, 
\par 58 sinova Jaalinih, sinova Darkonovih, sinova  Gidelovih, 
\par 59 sinova Šefatjinih, sinova Hatilovih, sinova Pokeret-Sebajinih, sinova Amonovih. 
\par 60 Svega netinaca i sinova Salomonovih slugu  tri stotine devedeset i dva. 
\par 61 Slijedeći ljudi koji su došli iz Tel Melaha, Tel Harše, Keruba, Adona i Imera nisu mogli dokazati da su njihove obitelji  i njihov rod izraelskog podrijetla: 
\par 62 sinovi Delajini, sinovi  Tobijini, sinovi Nekodini: šest stotina četrdeset i dva. 
\par 63 A  od svećenika: sinovi Hobajini, sinovi Hakosovi, sinovi Barzilaja  - onoga koji se oženio jednom od kćeri Barzilaja Gileađanina  te uzeo njegovo ime. 
\par 64 Ovi su ljudi tražili svoj zapis u rodovnicima, ali ga nisu mogli naći: bili su isključeni iz svećeništva 
\par 65 i  namjesnik im zabrani blagovati od svetinja sve dok se ne pojavi  svećenik za Urim i Tumin. 
\par 66 Ukupno je na zboru bilo četrdeset i dvije tisuće tri  stotine i šezdeset osoba, 
\par 67 ne računajući njihove sluge i sluškinje, kojih bijaše sedam tisuća tri stotine trideset i sedam. 
\par 68 Bilo  je i dvije stotine četrdeset i pet pjevača i pjevačica, 
\par 69 (7:68) četiri  stotine trideset i pet deva i šest tisuća sedam stotina i dvadeset  magaraca. 
\par 70 (7:69) Pojedini glavari obitelji dadoše priloge za gradnju.  Namjesnik je položio u riznicu tisuću drahmi zlata, pedeset vrčeva, trideset svećeničkih haljina. 
\par 71 (7:70) Neki su od glavara obitelji  dali u poslovnu riznicu dvadeset tisuća drahmi zlata i dvije  tisuće dvije stotine mina srebra. 
\par 72 (7:71) A darova ostalog puka bilo  je do dvadeset tisuća drahmi zlata, dvije tisuće mina srebra  i šezdeset i sedam svećeničkih haljina. 
\par 73 (7:72) Svećenici, leviti, vratari, pjevači, netinci i sav Izrael  naseliše se svaki u svoj grad. A kada se približio sedmi mjesec, već su sinovi Izraelovi bili u svojim gradovima. 


\chapter{8}

\par 1 Tada se skupi sav narod kao jedan čovjek na trg koji je pred  Vodenim vratima. Rekoše književniku Ezri da donese knjigu Mojsijeva  zakona što ga je Jahve dao Izraelu. 
\par 2 I prvoga dana sedmoga  mjeseca svećenik Ezra donese Zakon pred zbor ljudi, žena i sviju  koji su bili sposobni da ga razumiju. 
\par 3 Na trgu koji je pred  Vodenim vratima počeo je čitati knjigu, od ranoga jutra do podneva, pred ljudima, ženama i pred onima koji su bili zreli. Sav je  narod pozorno slušao knjigu Zakona. 
\par 4 Književnik Ezra stajaše na drvenu besjedištu koje su podigli  za tu zgodu. Kraj njega stajahu: s desne strane Matitja, Šema, Anaja, Urija, Hilkija i Maaseja, a s lijeve strane Pedaja, Mišael, Malkija, Hašum, Hašbadana, Zaharija i Mešulam. 
\par 5 Ezra je otvorio  knjigu naočigled svemu narodu - jer je bio poviše od svega naroda  - a kad ju je otvorio, sav narod ustade. 
\par 6 Tada Ezra blagoslovi  Jahvu, Boga velikoga, a sav narod, podignutih ruku, odgovori:  "Amen! Amen!" Zatim su kleknuli i poklonili se pred Jahvom, licem  do zemlje. 
\par 7 A leviti Ješua, Bani, Šerebja, Jamin, Akub, Šabtaj, Hodija, Maaseja, Kelita, Azarja, Jozabad, Hanan i Pelaja objašnjavahu  Zakon narodu, a narod stajaše na svome mjestu. 
\par 8 I čitahu iz  knjige Božjeg zakona po odlomcima i razlagahu smisao da narod  može razumjeti što se čita. 
\par 9 Potom namjesnik Nehemija, i svećenik i književnik Ezra, i leviti koji poučavahu narod rekoše svemu narodu: "Ovo je dan  posvećen Jahvi, Bogu vašemu! Ne tugujte, ne plačite!" Jer sav  narod plakaše slušajući riječi Zakona. 
\par 10 I još im reče Nehemija:  "Pođite i jedite masna jela, i pijte slatko, i pošaljite dio  onima koji nemaju ništa pripremljeno, jer ovo je dan posvećen  našem Gospodu. Ne žalostite se: radost Jahvina vaša je jakost." 
\par 11 I leviti umirivahu sav narod govoreći: "Umirite se: ovaj  je dan svet. Ne tugujte!" 
\par 12 I ode sav narod da jede i pije, i da šalje obroke, i da slavi veliko slavlje: jer su shvatili  riječi koje su im objavljene. 
\par 13 Drugog dana skupiše se glavari obitelji svega naroda, svećenici i leviti oko književnika Ezre da prouče riječi Zakona. 
\par 14 I nađoše napisano u Zakonu što ga je Jahve naredio preko  sluge Mojsija: "Sinovi Izraelovi neka borave pod sjenicama za  svečanosti u sedmom mjesecu." 
\par 15 Čim su čuli, proglasiše u svim  svojim gradovima i u Jeruzalemu: "Idite u goru i donesite granja  maslinova i granja divlje masline, mirtovih i palmovih grana  i granja ostaloga lisnatog drveća da načinimo sjenice, kako je  propisano." 
\par 16 I ode narod i donese granja i načiniše sjenice, svaki na svom krovu i svojim dvorištima, u predvorjima Doma  Božjega, na trgu kod Vodenih vrata i na onom kod Efrajimovih  vrata. 
\par 17 Sav zbor onih koji su se vratili iz sužanjstva načini  sjenice i boravili su u njima - Izraelci nisu toga činili od  vremena Jošue, sina Nunova, sve do toga dana. I bila je veoma  velika radost. 
\par 18 Ezra je čitao knjigu Zakona Božjeg svakog dana, od prvoga  do posljednjega. Sedam se dana svetkovao blagdan, a osmoga je  dana bio svečani zbor, kako je propisano. 


\chapter{9}

\par 1 Dvadeset i četvrtoga dana toga mjeseca skupiše se Izraelci  na post, u pokorničkim vrećama i posuti prašinom. 
\par 2 Rod se Izraelov  odvojio od svih tuđinaca: pristupili su i ispovijedali svoje  grijehe i bezakonja svojih otaca. 
\par 3 Stajali su, svatko na svome  mjestu, i čitali knjigu Zakona Jahve, Boga svoga, četvrtinu dana;  za druge su četvrtine ispovijedali svoje grijehe i klanjali se  Jahvi, Bogu svome. 
\par 4 A Ješua, Bani, Kadmiel, Šebanija, Buni, Šerebja, Bani i Kenani, popevši se na poviše mjesto za levite, vapili su snažnim glasom Jahvi, Bogu svome. 
\par 5 I govorahu leviti  Ješua, Kadmiel, Bani, Hašabneja, Šerebja, Hodija, Šebanija i  Petahja:  "Ustanite, blagoslivljajte Jahvu, Boga našega! Blagoslovljen da si, Jahve, Bože naš, odvijeka dovijeka! I neka je blagoslovljeno tvoje Ime slavno, iznad svakog blagoslova i hvale uzvišeno. 
\par 6 Ti si, Jahve, Jedini! Ti si stvorio nebo, i nebesa nad nebesima, i vojsku njihovu, zemlju i sve što je na njoj, mora i što je u njima. Ti sve to oživljavaš, i vojske se nebeske tebi klanjaju. 
\par 7 Ti si, Jahve, Bog, koji si Abrama izabrao, iz Ura kaldejskoga njega izveo i dao mu ime Abraham. 
\par 8 Vjerno si srce njegovo pred sobom našao i Savez s njim sklopio da ćeš mu dati zemlju kanaansku, i hetitsku i amorejsku, i perižansku, jebusejsku i girgašansku, njemu i potomstvu njegovu. I svoja si obećanja ispunio, jer si pravedan. 
\par 9 Nevolju si otaca naših u Egiptu vidio, i vapaj si njihov čuo kraj Mora crvenoga. 
\par 10 Znacima si se i čudesima oborio na faraona i na sve sluge njegove, i na sav narod zemlje njegove; jer znao si kolika je bila protiv njih drskost njihova. Sebi si ime stekao koje do danas traje. 
\par 11 More si pred njima razdvojio: prešli su usred mora po suhu. U dubine si utopio progonitelje njihove kao kamen među vode silovite. 
\par 12 Stupom oblaka danju si ih vodio, a noću si stupom ognjenim svijetlio im po putu kojim su hodili. 
\par 13 Na goru si Sinajsku sišao i s neba im govorio; i dao si im pravedne naredbe, čvrste zakone, zapovijedi izvrsne i uredbe. 
\par 14 Ti si im objavio svoju svetu subotu, zapovijedi, naredbe i Zakon si im propisao po glasu sluge svoga Mojsija. 
\par 15 S neba si ih hranio kruhom za njihove gladi, za njihovu si žeđ iz stijene vodu izveo. Ti si im zapovjedio da pođu zaposjesti zemlju za koju si se zakleo da ćeš im dati. 
\par 16 Ali se oni i oci naši uzjoguniše, vratove ukrutiše i zapovijedi tvojih nisu slušali. 
\par 17 Poslušnost su odbili, zaboravili čudesa što si ih za njih učinio; ukrutili su vratove, a u glavu uvrtjeli da u ropstvo se svoje vrate, u Egipat. Ali ti si Bog praštanja, milosrdan i blag, na gnjev si spor, a u milosrdđu velik: i nisi ih ostavio! 
\par 18 Čak su načinili tele saliveno, 'To bog je tvoj', rekoše, 'koji te izveo iz Egipta!' I teško su hulili, 
\par 19 a ti u beskrajnom milosrđu nisi ih napuštao u pustinji: stup se oblaka nije pred njima skrivao, danju ih je putem vodio, a stup je plameni noću pred njima svijetlio putem kojim su hodili. 
\par 20 Dao si im svoga Duha dobrog da ih naučiš mudrosti, mÓane svoje nisi uskratio njihovim ustima, i u žeđi si im vode pružio. 
\par 21 Četrdeset godina krijepio si ih u pustinji: ništa im nije nedostajalo: niti im se odijelo deralo, niti su im noge oticale. 
\par 22 I dao si im kraljevstva i narode i razdijelio ih granicama: zaposjeli su zemlju Sihona, kralja hešbonskoga, i zemlju Oga, kralja bašanskoga. 
\par 23 I sinove si im umnožio kao zvijezde nebeske, i u zemlju si ih doveo za koju si rekao njihovim ocima da će ući u nju i zaposjesti je. 
\par 24 Sinovi su ušli i pokorili zemlju, a ti si pred njima svladao stanovnike zemlje, Kanaance, i predao si u ruke njihove kraljeve i narode zemlje da rade s njima što ih je volja; 
\par 25 osvojili su gradove tvrde i zemlju plodnu i naslijedili kuće pune svakog dobra, isklesane zdence, vinograde, maslinike i mnogo plodnog drveća: jeli su, sitili se i debljali i uživali u velikoj dobroti tvojoj. 
\par 26 Ali su se bunili i odvrgli tebe, i Zakon su tvoj bacili za leđa, ubijali su proroke, koji su ih obraćali da se tebi vrate, i grdno su hulili. 
\par 27 U ruke si ih tada predao osvajačima, koji su ih tlačili. A u vrijeme muke svoje tebi su vapili i ti si ih s neba uslišio i u velikoj dobroti svojoj slao si im izbavitelje, koji su ih iz ruku tlačitelja njihovih izbavljali. 
\par 28 Ali čim bi se smirili, opet su pred tobom zlo činili, a ti si ih puštao u ruke neprijatelja njihovih, koji su ih mučili. I opet su k tebi vapili i ti si ih s neba uslišio: u milosrđu svojem mnogo si ih puta izbavio. 
\par 29 Ti si ih opominjao da se vrate tvome Zakonu: ali se oni uzjoguniše, nepokorni tvojim zapovijedima; griješili su protiv naredaba tvojih, a čovjek živi kad ih obdržava. Leđa su izvlačili, šije ukrućivali i nisu slušali. 
\par 30 Mnogo si godina bio strpljiv s njima i svojim si ih Duhom opominjao po službi svojih proroka; no nisu slušali. Tada si ih predao u ruke naroda zemaljskih. 
\par 31 U velikom milosrđu svojem ti ih nisi uništio, ni ostavio ih nisi, jer si ti Bog milostiv i pun samilosti. 
\par 32 A sada, o Bože naš, veliki Bože, jaki i strašni, koji čuvaš Savez i dobrohotnost, neka ne bude pred licem tvojim neznatna sva ova nevolja koja je snašla nas, kraljeve naše i knezove, svećenike i proroke naše, očeve naše i sav narod tvoj od vremena asirskih kraljeva pa do danas. 
\par 33 Ti si pravedan u svemu što nas je snašlo, jer si ti pokazao vjernost, a mi zloću svoju. 
\par 34 Kraljevi naši i knezovi, svećenici i oci naši nisu vršili Zakona tvoga, nisu osluškivali naredaba tvojih i opomena koje si im davao. 
\par 35 Premda su bili u svom kraljevstvu, u velikim dobrima koja si im činio, u prostranoj i plodnoj zemlji koju si im dao, oni ti nisu služili i od svojih zlih djela nisu se odvraćali. 
\par 36 Mi smo danas, evo, robovi i u zemlji koju si bio dao ocima našim da uživaju njene plodove i njena dobra, evo u njoj mi robujemo. 
\par 37 Njeni obilni prihodi idu kraljevima koje si nam postavio zbog grijeha naših, i gospodare oni po volji svojoj tjelesima našim i stokom našom. Ah, u velikoj smo nevolji! 
\par 38 (10:1) I zbog svega toga obvezujemo se pismeno na vjernost." Na zapečaćenoj  ispravi stajala su imena naših knezova, levita i svećenika ... 


\chapter{10}

\par 1 (10:2) Na zapečaćenoj ispravi su bili: namjesnik Nehemija, sin  Hakalijin, i Sidkija, 
\par 2 (10:3) Seraja, Azarja, Jeremija, 
\par 3 (10:4) Pašhur, Amarja, Malkija, 
\par 4 (10:5) Hatuš, Šebanija, Maluk, 
\par 5 (10:6) Harim, Meremot, Obadja, 
\par 6 (10:7) Daniel, Gineton, Baruk, 
\par 7 (10:8) Mešulam, Abija, Mijamin, 
\par 8 (10:9) Maazja, Bilgaj, Šemaja - to su svećenici. 
\par 9 (10:10) Zatim leviti: Ješua, sin Azanijin, Binuj, od sinova Henadadovih  - Kadmiel, 
\par 10 (10:11) i braća njihova: Šekanija, Hodija, Kelita, Pelaja, Hanan, 
\par 11 (10:12) Mika, Rehob, Hašabja, 
\par 12 (10:13) Zakur, Šerebja, Šebanija, 
\par 13 (10:14) Hodija, Bani, Beninu. 
\par 14 (10:15) Glavari naroda: Paroš, Pahat Moab, Elam, Zatu, Bani, 
\par 15 (10:16) Buni, Azgad, Bebaj, 
\par 16 (10:17) Adonija, Bigvaj, Adin, 
\par 17 (10:18) Ater,  Ezekija, Azur, 
\par 18 (10:19) Hodija, Hašum, Besaj, 
\par 19 (10:20) Harif, Anatot, Nebaj, 
\par 20 (10:21) Magpijaš, Mešulam, Hazir, 
\par 21 (10:22) Mešezabel, Sadok, Jadua, 
\par 22 (10:23) Pelatja, Hanan, Anaja, 
\par 23 (10:24) Hošea, Hananija, Hašub, 
\par 24 (10:25) Haloheš, Pilha, Šobek, 
\par 25 (10:26) Rehum, Hašabna, Maaseja, 
\par 26 (10:27) Ahija, Hanan, Anan, 
\par 27 (10:28) Maluk, Harim, Baana. 
\par 28 (10:29) ... ali i ostali narod, svećenici, leviti - vratari,  pjevači, netinci - i svi koji su se prema Zakonu Božjem odvojili  od zemaljskih naroda, a i njihove žene, sinovi i kćeri, svi koji  su bili sposobni da razumiju, 
\par 29 (10:30) priključili su se svojoj braći  i glavarima te su se obvezali prisegom i zakletvom da će stupati  prema Zakonu Božjem, koji je dan po rukama Mojsija, sluge Božjega, i da će držati i vršiti sve zapovijedi Jahve, Boga našega, njegove  naredbe i zakone. 
\par 30 (10:31) I osobito: da nećemo davati svojih kćeri narodima zemaljskim  i njihovih kćeri nećemo uzimati svojim sinovima. 
\par 31 (10:32) I ako narodi zemlje donesu na prodaju robu ili kakvo  god žito u dan subotnji, mi ništa nećemo od njih kupovati u subotu  ni u drugi posvećeni dan. Svake sedme godine ostavljat ćemo zemlju da počine i otpuštati  dugove svake ruke. 
\par 32 (10:33) Uzeli smo kao obavezu: da ćemo svake godine davati trećinu  šekela za bogoslužje u Domu Boga svojega: 
\par 33 (10:34) za postavljeni  kruh, za trajne prinosnice i za svagdanje paljenice, za žrtve  subotnje, mladog mjeseca, blagdanske i za okajnice, da se pomiri  Izrael; i za svaku službu u Domu Boga našega. 
\par 34 (10:35) Mi svećenici, leviti i narod bacili smo ždrijeb za prinos drva koja treba  da određenog dana svake godine prema svojim obiteljima donosimo  u Dom Boga našega za vatru na žrtveniku Jahve, Boga našega, kako  je zapisano u Zakonu; 
\par 35 (10:36) da ćemo svake godine donositi u Dom  Jahvin prvine od plodova zemlje i prve plodove svakoga drveta 
\par 36 (10:37) i prvorođene sinove i prvine svoje stoke, kako je to pisano  u Zakonu - prvine od krupne i sitne stoke neka se odnose u Dom  Boga našega, jer su određene svećenicima koji služe u Domu Boga  našega. 
\par 37 (10:38) Povrh toga prvine svojih naćava, plodova svakog drveta, novoga vina i ulja nosit ćemo svećenicima u sobe Doma Boga našega;  a desetinu od svoje zemlje levitima, jer leviti uzimaju desetinu  u svim mjestima gdje radimo. 
\par 38 (10:39) Svećenik, sin Aronov, neka prati  levite kad skupljaju desetinu. Leviti neka donose desetinu desetine  u Dom Boga našega, u sobe riznice, 
\par 39 (10:40) jer su onamo dužni, Izraelci  i leviti donositi prinos od žita, vina i ulja. Ondje se nalaze  posude svetišta, svećenici u službi, vratari i pjevači. Nećemo više zanemarivati Doma Boga svojega. 


\chapter{11}

\par 1 Tada se nastaniše knezovi narodni u Jeruzalemu. Ostali je  narod bacao ždrijeb da od svakih deset ljudi izađe jedan koji  će stanovati u svetom gradu Jeruzalemu, dok će ostalih devet  ostati u drugim gradovima. 
\par 2 I narod je blagoslovio sve ljude  koji su dragovoljno htjeli živjeti u Jeruzalemu. 
\par 3 A evo glavara pokrajinskih koji su se nastanili u Jeruzalemu  i po gradovima Judeje. Izrael, svećenici, leviti, netinci i sinovi  Salomonovih slugu nastanili su se u svojim gradovima, svaki na  svome posjedu. 
\par 4 U Jeruzalemu se nastaniše sinovi Judini i sinovi  Benjaminovi. Od sinova Judinih: Ataja, sin Uzije, sina Zaharijina, sina Amarjina, sina Šefatjina, sina Mahalalelova, od sinova  Faresovih; 
\par 5 Maaseja, sin Baruha, sina Kol-Hozea, sina Hazaje, sina Adaje, sina Jojariba, sina Zaharije, sina Šelina. 
\par 6 Svega  je bilo Faresovih sinova u Jeruzalemu četiri stotine šezdeset  i osam ljudi sposobnih za boj. 
\par 7 Evo Benjaminovih sinova: Salu, sin Mešulama, sina Joedova, sina Pedajina, sina Kolajina, sina Maasejina, sina Itielova, sina Ješajina, 
\par 8 i braća njegova: sposobnih za boj devet stotina  dvadeset i osam. 
\par 9 Joel, sin Zikrijev, bio je njihov zapovjednik, i Juda, sin Hasenuin, drugi upravitelj grada. 
\par 10 Od svećenika: Jedaja, Jojarib, Jakin, 
\par 11 Seraja, sin  Hilkije, sina Mešulama, sina Sadoka, sina Merajota, sina Ahituba, predstojnik Doma Božjega, i 
\par 12 njihova braća koja su vršila  službu u Domu: osam stotina dvadeset i dvojica; i Adaja, sin  Jerohama, sina Pelalije, sina Amsija, sina Zaharije, sina Pašhura, sina Malkijina, 
\par 13 i njegova braća, glavari obitelji: dvjesta  četrdeset i dvojica; i Amasaj, sin Azarela, sina Ahzaja, sina  Mešilemota, sina Imerova, 
\par 14 i njihove braće, sposobnih za boj:  stotinu dvadeset i osam. Zapovjednik nad njima bio je Zabdiel, sin Hagedolimov. 
\par 15 Od levita: Šemaja, sin Hašuba, sina Azrikama, sina Hašabje, sina Bunijeva; 
\par 16 i Šabtaj i Jozabad, od glavara levitskih, za nadzor vanjskih poslova Doma Božjega; 
\par 17 i Matanija, sin  Miheja, sina Zabdijeva, sina Asafova, koji je ravnao psalmima, počinjao zahvale i molitve; i Bakbukja, drugi među svojom braćom;  i Abda, sin Šamue, sina Galala, sina Jedutunova. 
\par 18 Svega je  levita bilo u Svetom gradu: dvjesta osamdeset i četiri. 
\par 19 A vratari: Akub, Talmon i njihova braća koja su čuvala  stražu na vratima: stotinu sedamdeset i dva. 
\par 20 A ostali Izraelci, svećenici i leviti, nastaniše se u  svim gradovima Judeje, svaki na svojoj baštini i po naseljima  u njihovim poljima. 
\par 21 Netinci su stanovali u Ofelu; Siha i Gišpa bijahu na  čelu netinaca. 
\par 22 Predstojnik je levitima u Jeruzalemu bio Uzi, sin Banija, sina Hašabje, sina Matanije, sina Mihejina. On je  bio od sinova Asafovih, koji su bili pjevači za službu Doma Božjega. 
\par 23 Jer je za njih bila kraljeva zapovijed i uredba za svakodnevnu  službu. 
\par 24 Petahja, sin Mešezabelov, od sinova Zeraha, sina  Judina, bio je kraljev povjerenik za sve poslove s narodom. 
\par 25 Od  sinova Judinih nastanili su se u Kirjat Haarbi i njezinim zaseocima, u Dibonu i njegovim zaseocima, u Jekabseelu i njegovim naseljima, 
\par 26 u Jesui, u Moladi, u Bet Peletu, 
\par 27 u Hasar Šualu, u Beer  Šebi i u njenim zaseocima, 
\par 28 u Siklagu, u Mekoni i njenim zaseocima, 
\par 29 u En Rimonu, u Sori, u Jarmutu, 
\par 30 Zanoahu, Adulamu i njihovim  naseljima; u Lakišu i njegovim poljima, u Azeki i njenim zaseocima:  tako su se naselili od Beer Šebe sve do Hinomske doline. 
\par 31 Benjaminovi sinovi življahu u Gebi, Mikmasu, Aju i Betelu  i u njihovim zaseocima, 
\par 32 u Anatotu, Nobu, Ananiji, 
\par 33 Hasoru, Rami, Gitajimu, 
\par 34 Hadidu, Seboimu, u Nebalatu, 
\par 35 Lodu, Ononu  i u Dolini rukotvoraca. 
\par 36 Skupine levita nalazile su se u Judi i Benjaminu. 


\chapter{12}

\par 1 Ovo su svećenici i leviti koji su došli sa Zerubabelom, sinom  Šealtielovim, i Ješuom: Seraja, Jeremija, Ezra, 
\par 2 Amarja, Maluk, Hatuš, 
\par 3 Šekanija, Rehum, Meremot, 
\par 4 Ido, Gineton, Abija, 
\par 5 Mijamin, Maadja, Bilga, 
\par 6 Šemaja, Jojarib, Jedaja, 
\par 7 Salu, Amok, Hilkija i Jedaja. To su bili glavari svećenički i njihova  braća za Ješuina vremena. 
\par 8 A leviti: Ješua, Binuj, Kadmiel, Šerebja, Juda i Matanija  - ovaj potonji i njegova braća ravnali su hvalospjevima. 
\par 9 Bakbukja  i Uni i braća njihova izmjenjivali su se s njima u službi. 
\par 10 Ješua rodi Jojakima; Jojakim rodi Elijašiba, a Elijašib  Jojadu; 
\par 11 Jojada rodi Jonatana, a Jonatan rodi Jaduu. 
\par 12 U Jojakimovo vrijeme glavari svećeničkih obitelji bijahu:  Serajine obitelji Meraja; Jeremijine Hananja; 
\par 13 Ezrine Mešulam;  Amarjine Johanan; 
\par 14 Malukove Jonatan; Šebanijine Josip; 
\par 15 Harimove  Adna; Meremotove Helkaj; 
\par 16 Idove Zaharija; Ginetonove Mešulam; 
\par 17 Abijine Zikri; Minjaminove ...; obitelji Moadjine Piltaj; 
\par 18 Bilgine Šamua; Šemajine Jonatan; 
\par 19 Jojaribove Matenaj;  Jedajine Uzi; 
\par 20 Saluove Kelaj; Amokove Eber; 
\par 21 Hilkijine  Hašabja; Jedajine Netanel. 
\par 22 U vrijeme Elijašiba, Jojade, Johanana i Jadue bili su  popisani glavari levitskih obitelji i svećenici sve do kraljevanja  Darija Perzijanca. 
\par 23 Sinovi Levijevi: glavari obitelji bili su zabilježeni  u Knjizi ljetopisa, do vremena Johanana, sina Elijašibova. 
\par 24 Glavari levitski bili su: Hašabja, Šerebja, Ješua, Binuj, Kadmiel, a njihova braća, koja su stajala prema njima da pjevaju  naizmjenično pohvale i zahvalnice prema uredbama Davida, Božjeg  čovjeka, 
\par 25 bijahu: Matanija, Bakbukja i Obadja. A Mešulam,  Talmon i Akub, vratari, čuvali su stražu kod skladišta blizu  vrata. 
\par 26 Ti su živjeli u vrijeme Jojakima, sina Ješue, sina Josadakova, i u vrijeme upravitelja Nehemije i književnika svećenika Ezre. 
\par 27 Kad je bila posveta jeruzalemskoga zida, potražili su  levite svugdje gdje su stanovali da ih dovedu u Jeruzalem te  proslave posvetu radošću, zahvalnicama i pjesmom uz cimbale,  harfe i citre. 
\par 28 I skupiše se pjevači, sinovi Levijevi, iz kraja oko Jeruzalema, iz netofatskih sela, 
\par 29 iz Bet Hagilgala, iz Gebe i polja Azmaveta:  jer su pjevači sebi sagradili sela oko Jeruzalema. 
\par 30 Svećenici  i leviti očistili su sebe, a zatim su očistili narod, vrata i  zid. 
\par 31 Tada sam izveo judejske knezove na zid i sastavio dva  velika zbora. Prvi je išao desno niza zid, prema Smetlišnim vratima; 
\par 32 za njima su išli Hošaja i polovina judejskih knezova - 
\par 33 Azarja, Ezra i Mešulam, 
\par 34 Juda, Benjamin, Šemaja i Jeremija, 
\par 35 a  od svećeničkih sinova s trubljama: Zaharija, sin Jonatana, sina  Šemaje, sina Matanije, sina Mikaje, sina Zakura, sina Asafa, 
\par 36 s braćom njihovom Šemajom, Azarelom, Milalajem, Gilalajem, Maajem, Netanelom, Judom, Hananijem, s glazbalima Davida, Božjega  čovjeka. A Ezra, književnik, išao je pred njima. 
\par 37 Kod Izvorskih  vrata popeli su se njima nasuprot kraj stepenica Davidova grada, zidnim usponom od Davidove palače sve do Vodenih vrata na istoku. 
\par 38 Drugi zbor, a za njim ja i polovica narodnih knezova, išao je nalijevo zidom i Pećkom kulom sve do Tržnog zida, 
\par 39 pa  onda iznad Efrajimovih vrata, Starih vrata, Ribljih vrata, Hananelove  kule, kule Meaha, sve do Ovčjih vrata. Zaustavili su se kod Zatvorskih  vrata. 
\par 40 Potom su oba zbora zauzela mjesto u Domu Božjem. Tako  i ja i sa mnom polovica odličnika, 
\par 41 svećenici Elijakim, Maaseja, Minjamin, Mikaja, Elijoenaj, Zaharija, Hananija s trubama, 
\par 42 zatim  Maaseja, Šemaja, Eleazar, Uzi, Johanan, Malkija, Elam i Ezer.  Pjevači su pjevali pod ravnanjem Jizrahjinim. 
\par 43 Toga su dana  prinesene velike žrtve, ljudi su dali oduška radosti, jer ih  je Bog ispunio velikom radošću, veselile se i žene i djeca. I  radost Jeruzalema čula se nadaleko. 
\par 44 U to su vrijeme postavljeni ljudi da nadziru spremišta  prinosa, prvina, desetina i da s polja uz gradove sabiru dijelove  koje Zakon dodjeljuje svećenicima i levitima. Jer su se Judejci  radovali svećenicima i levitima koji su bili u službi. 
\par 45 Oni  su vršili službu Bogu svome i službu očišćenja - kao i pjevači  i vratari - prema odredbi Davida i njegova sina Salomona. 
\par 46 Jer  od Davidovih i Asafovih dana, od davnine, postoje pjevački glavari  i pjesme pohvalne i zahvalnice Bogu. 
\par 47 Zato je sav Izrael u  vrijeme Zerubabela i u Nehemijino vrijeme dan za danom davao  dijelove određene za pjevače i vratare. Davali su levitima posvećene  darove, a leviti su davali sinovima Aronovim. 



\chapter{13}

\par 1 U ono vrijeme čitala se narodu knjiga Mojsijeva i ondje se  našlo zapisano da Amonac i Moabac ne smiju nikada ući u zbor  Božji, 
\par 2 jer nisu sinovima Izraelovima izašli u susret s kruhom  i vodom, nego su čak najmili protiv njih Bileama da ih prokune, ali je naš Bog obratio kletvu u blagoslov. 
\par 3 Kad su čuli Zakon, isključili su iz Izraela sve strance. 
\par 4 A prije toga svećenik Elijašib, postavljen nad sobama  Doma Boga našega, bijaše svom rođaku Tobiji 
\par 5 uredio prostranu  sobu gdje su se prije ostavljali prinosi, tamjan, posuđe, desetine  žita, vina i ulja, određene za levite, pjevače i vratare, i doprinosi  za svećenike. 
\par 6 U to vrijeme nisam bio u Jeruzalemu, jer sam  trideset i druge godine babilonskog kralja Artakserksa otišao  kralju; ali poslije nekog vremena izmolio sam u kralja 
\par 7 da  se mogu vratiti u Jeruzalem. Tada doznadoh za zlo djelo što ga  učini Elijašib uredivši Tobiji sobu u predvorjima Doma Božjega. 
\par 8 To me veoma rasrdilo: izbacih iz sobe sav namještaj Tobijina  stana 
\par 9 i naredih da se sobe očiste, zatim unesoh onamo posuđe  Doma Božjega, prinose i tamjan. 
\par 10 Doznadoh i to da levitima nisu davali njihovih dijelova  i da su se i leviti i pjevači, određeni za službu, razbježali  svaki u svoje polje. 
\par 11 I prekorih odličnike i rekoh: "Zašto  je zapušten Dom Božji?" Zatim skupih levite i pjevače i vratih  ih k njihovim službama. 
\par 12 Tada je sva Judeja donosila u spremišta  desetinu žita, vina i ulja. 
\par 13 Nad spremištima postavio sam  svećenika Šelemju, književnika Sadoka i levita Pedaju, a uz njih  Hanana, sina Zakura, sina Matanijina. Njih su smatrali pouzdanima;  njihova je dužnost bila da dijele svojoj braći. 
\par 14 Zato, sjeti  se mene, Bože moj: ne prezri mojih pobožnih djela koja učinih  za Dom Boga svoga i za službu u njemu. 
\par 15 U ono sam vrijeme vidio u Judeji ljude koji gaze u tijescima  u dan subotnji; drugi su nosili snopove žita, tovarili na magarce  vino, grožđe, smokve i svakojake terete da ih u dan subotnji  unesu u Jeruzalem. I prekorih ljude što u taj dan prodaju živež. 
\par 16 A Tirci koji su živjeli u Jeruzalemu donosili su onamo ribu  i svakovrsnu robu da je prodaju Židovima u subotu. 
\par 17 Prekorih  judejske velikaše i rekoh im: "Kakvo to zlo djelo činite i skrnavite  dan subotnji? 
\par 18 Nisu li tako činili i vaši oci te je Bog naš  doveo svu ovu nesreću na nas i na ovaj grad? A zar vi želite  umnažati gnjev protiv Izraela skrnaveći subotu?" 
\par 19 I zapovjedih  još da uoči subote, kad se mrak spusti na jeruzalemska vrata, zatvore njihova krila i rekoh neka se ne otvaraju do iza subote!  Postavio sam nekoliko svojih momaka na vrata da se ne unosi nikakav  tovar u dan subotnji. 
\par 20 Jednom su ili dvaput trgovci i prodavači  svakovrsne robe proveli noć izvan Jeruzalema, 
\par 21 ali sam ih  upozorio i rekao im: "Zašto provodite noć pod zidom? Ako to ponovite, dignut ću na vas ruku!" Od toga vremena nisu više dolazili u  subotu. 
\par 22 Zapovjedio sam levitima da se očiste i da dođu čuvati  vrata, kako bi se svetkovao dan subotnji. I za ovo se spomeni  mene, Bože moj, i smiluj mi se po svome velikom milosrđu! 
\par 23 Onih sam dana vidio i Židove koji se bijahu oženili Ašdođankama, Amonkama i Moapkama. 
\par 24 Polovica njihovih sinova govorila je  ašdodski ili jezikom ovoga ili onoga naroda: više nisu znali  govoriti židovski. 
\par 25 Korio sam ih i proklinjao, neke sam i  tukao, čupao im kose i zaklinjao ih Bogom: "Ne dajite svojih  kćeri njihovim sinovima i ne uzimajte žene od njihovih kćeri  za svoje sinove, a ni za sebe! 
\par 26 Nije li u tome sagriješio  Salomon, kralj Izraelov? Među mnogim narodima nije bilo kralja  njemu ravna. Bio je drag Bogu svome i Gospod ga je postavio kraljem  nad svim Izraelom. Ali su i njega tuđinke navele na grijeh! 
\par 27 Treba  li slušati kako i vi činite veliko zlo i postajete nevjerni Bogu  našemu ženeći se tuđinkama?" 
\par 28 Jedan od sinova Jojade, sina velikog svećenika Elijašiba, bijaše zet Horonjaninu Sanbalatu. Njega sam otjerao od sebe. 
\par 29 Spomeni se, Bože moj, ovih ljudi, jer su oskvrnuli svećeništvo  i zavjet svećenički i levitski. 
\par 30 Tako sam ih očistio od svega tuđega i opet uspostavio  službe svećenika i levita dodijelivši svakome njegov posao. 
\par 31 Uredio  sam i da se nose drva u određene dane i prvine. Sjeti me se, Bože moj, za moje dobro! 




\end{document}