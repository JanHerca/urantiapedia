\begin{document}

\title{Jevrejima}


\chapter{1}

\par 1 Više puta i na više načina Bog nekoć govoraše ocima po prorocima; 
\par 2 konačno, u ove dane, progovori nama u Sinu. Njega postavi  baštinikom svega; Njega po kome sazda svjetove. 
\par 3 On, koji je  odsjaj Slave i otisak Bića njegova te sve nosi snagom riječi  svoje, pošto očisti grijehe, sjede zdesna Veličanstvu  u visinama; 
\par 4 postade toliko moćniji od anđela koliko je uzvišenije  nego oni baštinio ime. 
\par 5 Ta kome od anđela ikad reče: Ti si sin moj, danas te rodih; ili pak: Ja ću njemu biti otac, a on će meni biti sin. 
\par 6 A opet, kad uvodi Prvorođenca u svijet, govori: Nek pred njim nice padnu svi anđeli Božji. 
\par 7 Za anđele veli: Anđele čini vjetrovima, sluge svoje plamenom ognjenim, 
\par 8 ali za Sina: Prijestolje je tvoje, Bože, u vijeke vjekova, i pravedno žezlo - žezlo je tvog kraljevstva. 
\par 9 Ti ljubiš pravednost, a mrziš bezakonje, stoga Bog, Bog tvoj, tebe pomaza uljem radosti kao nikog od tvojih drugova. 
\par 10 I: Ti u početku, Gospodine, utemelji zemlju i nebo je djelo ruku tvojih. 
\par 11 Propast će, ti ćeš ostati, sve će ostarjeti kao odjeća. 
\par 12 Mijenjaš ih poput haljine, kao odjeću, i nestaju. A ti si uvijek isti - godinama tvojim nema kraja. 
\par 13 Za koga pak od anđela ikad reče: Sjedi mi zdesna dok ne položim neprijatelje tvoje za podnožje nogama tvojim! 
\par 14 Svi ti zar nisu služnički duhovi što se šalju služiti  za one koji imaju baštiniti spasenje? 


\chapter{2}

\par 1 Zato treba da mi svesrdnije prianjamo uz ono što čusmo da ne  bismo promašili. 
\par 2 Jer ako je riječ po anđelima izrečena bila  čvrsta te je svaki prijestup i neposluh primio pravednu plaću, 
\par 3 kako li ćemo mi umaći ako zanemarimo toliko spasenje? Spasenje  koje je počeo propovijedati Gospodin, koje su nam potvrdili slušatelji, 
\par 4 a suposvjedočio Bog znamenjima i čudesima, najrazličitijim  silnim djelima i darivanjima Duha Svetoga po svojoj volji. 
\par 5 Nije doista anđelima podložio budući svijet o kojem govorimo. 
\par 6 Netko negdje posvjedoči: Što je čovjek da ga se spominješ, sin čovječji te ga pohađaš. 
\par 7 Ti ga tek za malo učini manjim od anđela, slavom i časti njega ovjenča, 
\par 8 njemu pod noge sve podloži. Kad mu, doista, sve podloži, ništa ne ostavi što mu  ne bi bilo podloženo. Sad još ne vidimo da mu je sve podloženo, 
\par 9 ali Njega, za malo manjeg od anđela, Isusa, vidimo  zbog pretrpljene smrti slavom i časti ovjenčana da milošću  Božjom bude svakome na korist što je on smrt okusio. 
\par 10 Dolikovalo je doista da Onaj radi kojega je sve i po  kojemu je sve - kako bi mnoge sinove priveo k slavi - po patnjama  do savršenstva dovede Početnika njihova spasenja. 
\par 11 Ta i Posvetitelj  i posvećeni - svi su od jednoga! Zato se on i ne stidi zvati  ih braćom, 
\par 12 kad veli: Braći ću svojoj naviještat ime tvoje, hvalit ću te usred zbora. 
\par 13 I još: Ja ću se u njega uzdati, i još: Evo, ja i djeca koju mi Bog dade. 
\par 14 Pa budući da djeca imaju zajedničku krv i meso, i sam on tako postade u tome sudionikom da smrću obeskrijepi  onoga koji imaše moć smrti, to jest đavla, 
\par 15 pa oslobodi one  koji - od straha pred smrću - kroza sav život bijahu podložni  ropstvu. 
\par 16 Ta ne zauzima se dašto za anđele, nego se zauzima  za potomstvo Abrahamovo. 
\par 17 Stoga je trebalo da u svemu  postane braći sličan, da milosrdan bude i ovjerovljen Veliki  svećenik u odnosu prema Bogu kako bi okajavao grijehe naroda. 
\par 18 Doista, u čemu je iskušan trpio, može iskušavanima pomoći. 


\chapter{3}

\par 1 Stoga, braćo sveta, sudionici nebeskoga poziva, promotrite  Apostola i Velikoga svećenika naše vjere - Isusa: 
\par 2 on je ovjerovljen  kod Onoga koji ga postavi kao ono i Mojsije u svoj kući njegovoj. 
\par 3 Dostojan je doista toliko veće slave od Mojsija koliko veću  čast od kuće ima onaj tko ju je sagradio. 
\par 4 Jer svaku kuću tkogod  gradi, a sve je sagradio Bog. 
\par 5 Da, i Mojsije bijaše  ovjerovljen u svoj kući njegovoj kao služnik da posvjedoči  za ono što je imalo biti rečeno, 
\par 6 ali Krist - kao Sin, nad  kućom njegovom. Njegova smo kuća mi ako sačuvamo smjelost i ponos  nade. 
\par 7 Zato, kao što veli Duh Sveti: Danas ako glas mu čujete, 
\par 8 ne budite srca tvrda kao u Pobuni, kao u dan iskušenja u pustinji 
\par 9 gdje me kušnjom iskušavahu očevi vaši premda gledahu djela moja 
\par 10 četrdeset godina. Zato mi dodija naraštaj onaj pa rekoh: Uvijek su nestalna srca i ne proniču moje putove. 
\par 11 Tako se zakleh u svom gnjevu: Nikad neće ući u moj počinak! 
\par 12 Pazite, braćo, da ne bi u koga od vas srce bilo  opako, nevjerno, odmetnulo se od Boga živoga. 
\par 13 Pače hrabrite  jedni druge dan za danom dok još odjekuje ono Danas da  ne otvrdne tko od vas zaveden grijehom. 
\par 14 Doista, sudionici  smo Kristovi postali ako, dakako, ono prvo imanje stalnim sačuvamo 
\par 15 kad je rečeno: Danas ako glas mu čujete, ne budite srca tvrda kao u Pobuni! 
\par 16 Jer, koji su to čuli pa se pobunili? Zar ne svi  koji su pod Mojsijem izašli iz Egipta? 
\par 17 Koji li mu dodijavahu  četrdeset godina? Zar ne oni koji sagriješiše, kojih mrtva  tijela popadaše u pustinji? 
\par 18 Kojima se zakle da neće  ući u njegov počinak, ako li ne nepokornima? 
\par 19 I vidimo  da ne mogoše ući zbog nevjere. 


\chapter{4}

\par 1 Bojmo se dakle da se, dok ostaje obećanje o ulasku u njegov  Počinak, za koga od vas ne bi utvrdilo kako je zakasnio. 
\par 2 Jer  nama je naviještena blagovijest kao i njima, ali njima Riječ  poruke nije uskoristila jer se vjerom nisu pridružili onima koji  su poslušali. 
\par 3 U Počinak doista ulazimo mi koji povjerovasmo, prema onom što je rekao: Tako se zakleh u svom gnjevu: Nikad neće ući u moj počinak, premda su djela od postanka svijeta dovršena. 
\par 4 Rekao je doista  negdje o sedmom danu ovako: I počinu Bog sedmoga dana od svih  djela svojih. 
\par 5 A ovdje opet: Nikad neće ući u moj počinak. 
\par 6 Preostaje dakle da neki imaju u nj ući, a oni koji su prvi  primili blagovijest ne uđoše zbog nepokornosti. 
\par 7 Zato Bog ponovno  određuje jedan dan, Danas, u Davidu nakon toliko vremena  govoreći, kako je već rečeno: Danas ako glas mu čujete, ne budite srca tvrda. 
\par 8 Zbilja, da je Jošua njih u Počinak uveo, ne bi Bog nakon  toga govorio o drugome danu. 
\par 9 Dakle: preostaje neki subotni  počinak narodu Božjemu! 
\par 10 Zaista, tko uđe u njegov počinak, počinuo je od djela svojih kao ono i Bog od svojih. 
\par 11 Pohitimo  dakle ući u taj Počinak da nitko ne padne po uzoru na takvu nepokornost. 
\par 12 Živa je, uistinu, Riječ Božja i djelotvorna; oštrija  je od svakoga dvosjekla mača; prodire dotle da dijeli dušu i  duh, zglobove i moždinu te prosuđuje nakane i misli srca. 
\par 13 Nema  stvorenja njoj skrivena. Sve je, naprotiv, golo i razgoljeno  očima Onoga komu nam je dati račun. 
\par 14 Imajući dakle velikoga Velikog svećenika koji prodrije  kroz nebesa - Isusa, Sina Božjega - čvrsto se držimo vjere. 
\par 15 Ta nemamo takva Velikog svećenika koji ne bi mogao biti  supatnik u našim slabostima, nego poput nas iskušavana svime, osim grijehom. 
\par 16 Pristupajmo dakle smjelo Prijestolju milosti da primimo milosrđe  i milost nađemo za pomoć u pravi čas! 


\chapter{5}

\par 1 Svaki veliki svećenik, zaista, od ljudi uzet, za ljude se postavlja  u odnosu prema Bogu da prinosi darove i žrtve za grijehe. 
\par 2 On  može primjereno suosjećati s onima koji su u neznanju i zabludi  jer je i sam zaogrnut slabošću. 
\par 3 Zato mora i za narod i za  sebe prinositi okajnice. 
\par 4 I nitko sam sebi ne prisvaja tu čast, nego je prima od Boga, pozvan kao Aron. 
\par 5 Tako i Krist ne proslavi sam sebe postavši svećenik, nego  ga proslavi Onaj koji mu reče: Ti si sin moj, danas te rodih, 
\par 6 po onome što pak drugdje veli: Zauvijek ti si svećenik  po redu Melkisedekovu. 
\par 7 On je u dane svoga zemaljskog života  sa silnim vapajem i suzama prikazivao molitve i prošnje Onomu  koji ga je mogao spasiti od smrti. I bi uslišan zbog svoje predanosti: 
\par 8 premda je Sin, iz onoga što prepati, naviknu slušati 
\par 9 i, postigavši savršenstvo, posta svima koji ga slušaju začetnik  vječnoga spasenja - 
\par 10 proglašen od Boga Velikim svećenikom  po redu Melkisedekovu. 
\par 11 O tome nas čeka besjeda velika, ali teško ju je riječima  izložiti jer ste tvrdih ušiju. 
\par 12 Pa trebalo bi doista da nakon  toliko vremena već budete učitelji, a ono treba da tkogod vas  ponovno poučava početnička počela kazivanja Božjih. Takvi ste:  mlijeka vam treba, a ne tvrde hrane. 
\par 13 Doista, tko je god još  pri mlijeku, ne zna ništa o nauku pravednosti jer - nejače je. 
\par 14 A za zrele je tvrda hrana, za one koji imaju iskustvom izvježbana  čula za rasuđivanje dobra i zla. 


\chapter{6}

\par 1 Stoga mimoiđimo početnički nauk o Kristu i uzdignimo se k savršenome  ne postavljajući iznovice temelja: obraćenje od mrtvih djela  i vjera u Boga, 
\par 2 naučavanje o krštenjima i polaganje ruku,  uskrsnuće mrtvih i vječni sud. 
\par 3 To ćemo pak učiniti, dakako, ako Bog da. 
\par 4 Zaista, onima koji su jednom prosvijetljeni,  i okusili dar nebeski, i postali dionici Duha Svetoga, 
\par 5 i okusili  Lijepu riječ Božju i snage budućega svijeta, 
\par 6 pa otpali, nemoguće  je opet se obnoviti na obraćenje kad oni sami ponovno razapinju  Sina Božjega i ruglu ga izvrgavaju. 
\par 7 Jer zemlja koja se napije  kiše što na nju često pada i rađa raslinjem korisnim onima za  koje se i obrađuje, prima blagoslov od Boga; 
\par 8 ona pak koja  donosi trnje i drač, odbačena je, blizu prokletstvu a  svršetak joj je: "U oganj!" 
\par 9 A uvjereni smo, ljubljeni, sve ako tako i govorimo, da  je s vama dobro i da ste na putu spasenja. 
\par 10 Ta Bog nije nepravedan  da bi zaboravio vaše djelo i ljubav što je iskazaste njegovu  imenu posluživši i poslužujući svetima. 
\par 11 Želimo ipak da svatko  od vas sve do svršetka pokazuje tu istu gorljivost za ispunjenje  nade 
\par 12 te ne omlitavite, nego budete nasljedovatelji onih koji  po vjeri i strpljivosti baštine obećano. 
\par 13 Doista, kad je Bog Abrahamu davao obećanje, jer se nije  imao kime većim zakleti, zakle se samim sobom: 
\par 14 Uistinu,  blagosloviti, blagoslovit ću te i umnožiti, umnožit ću te. 
\par 15 I tako Abraham, strpljiv, postiže obećano. 
\par 16 Ljudi se doista  kunu onim tko je veći i zakletva im je, kao potkrepa, kraj svake  raspre. 
\par 17 Tako i Bog: htio je baštinicima obećanja obilatije  pokazati nepromjenljivost svoje odluke pa zato zajamči zakletvom 
\par 18 da bismo po dva nepromjenljiva čina - u kojima je nemoguće  da bi Bog prevario - mi pribjeglice imali snažno ohrabrenje da  se držimo ponuđene nade. 
\par 19 Ona nam je kao pouzdano i čvrsto  sidro duše što ulazi u unutrašnjost iza zavjese, 
\par 20 kamo je kao preteča za nas ušao Isus postavši zauvijek  Veliki svećenik po redu Melkisedekovu. 


\chapter{7}

\par 1 Doista, taj Melkisedek, kralj šalemski, svećenik Boga Svevišnjega  što je izašao u susret Abrahamu koji se vraćao s poraza kraljeva  i blagoslovio ga, 
\par 2 i komu Abraham odijeli desetinu  od svega; on koji u prijevodu znači najprije "kralj pravednosti", a zatim i kralj šalemski, to jest "kralj mira"; 
\par 3 on, bez oca, bez majke, bez rodoslovlja; on, kojemu dani nemaju  početka ni život kraja - sličan Sinu Božjemu, ostaje svećenik  zasvagda. 
\par 4 Pa promotrite koliki li je taj komu Abraham, rodozačetnik, dade desetinu od najboljega. 
\par 5 Istina, i oni sinovi  Levijevi, koji primaju svećeništvo imaju zakonsku zapovijed da  ubiru desetinu od naroda, to jest od svoje braće premda su i  ona izašla iz boka Abrahamova. 
\par 6 Ali on, koji nije iz njihova  rodoslovlja, ubra desetinu od Abrahama i blagoslovi njega, nosioca  obećanja! 
\par 7 A posve je neprijeporno: veći blagoslivlja manjega. 
\par 8 K tome, ovdje desetinu primaju smrtni ljudi, a ondje onaj, za kojega se svjedoči da živi. 
\par 9 I u Abrahamu se, tako reći, ubire desetina i od Levija koji inače desetinu prima 
\par 10 jer  još bijaše u boku očevu kad mu u susret iziđe Melkisedek. 
\par 11 Da se dakle savršenstvo postiglo po levitskom svećeništvu  - jer na temelju njega narod je dobio Zakon - koja bi onda bila  potreba da se po redu Melkisedekovu postavi drugi svećenik  i da se ne imenuje po redu Aronovu? 
\par 12 Doista kad se  mijenja svećeništvo, nužno se mijenja i Zakon. 
\par 13 Jer onaj o  kojemu se to veli pripadao je drugom plemenu, od kojega se nitko  nije posvetio žrtveniku. 
\par 14 Poznato je da je Gospodin naš potekao  od Jude, plemena za koje Mojsije ništa ne reče s obzirom na svećenike. 
\par 15 To je još očitije ako se drugi svećenik postavlja  po sličnosti s Melkisedekom: 
\par 16 postao je svećenikom  ne po Zakonu tjelesne uredbe, nego snagom neuništiva života. 
\par 17 Ta svjedoči se: Zauvijek ti si svećenik po redu Melkisedekovu. 
\par 18 Dokida se dakle prijašnja uredba zbog njezine nemoći  i beskorisnosti - 
\par 19 jer Zakon nije ništa priveo k savršenstvu  - a uvodi se bolja nada, po kojoj se približujemo Bogu. 
\par 20 I to se nije zbilo bez zakletve. Jer oni su bez zakletve  postali svećenicima, 
\par 21 a on sa zakletvom Onoga koji mu reče: Zakleo se Gospodin i neće se pokajati: "Zauvijek ti si svećenik". 
\par 22 Utoliko je Isus i postao jamac boljega Saveza. 
\par 23 K tomu, mnogo je bilo svećenika jer ih je smrt priječila  trajno ostati. 
\par 24 A on, jer ostaje dovijeka, ima neprolazno  svećeništvo. 
\par 25 Zato i može do kraja spasavati one koji po njemu  pristupaju k Bogu - uvijek živ da se za njih zauzima. 
\par 26 Takav nam Veliki svećenik i bijaše potreban - svet, nedužan, neokaljan, odijeljen od grešnika i uzvišeniji od nebesa - 
\par 27 koji  ne treba da kao oni veliki svećenici danomice prinosi žrtve najprije  za svoje grijehe, a onda za grijehe naroda. To on učini jednom  prinijevši samoga sebe. 
\par 28 Zakon doista postavi za velike svećenike  ljude podložne slabosti, a riječ zakletve - nakon Zakona - Sina  zauvijek usavršena. 


\chapter{8}

\par 1 A glavno u ovom izlaganju jest: takva imamo Velikog svećenika  koji sjede zdesna prijestolja Veličanstva na nebesima 
\par 2 kao bogoslužnik Svetinje i Šatora istinskoga što ga podiže  Gospodin, a ne čovjek. 
\par 3 Doista, svaki se veliki svećenik postavlja da prinosi  darove i žrtve. Odatle je potrebno da i on ima što bi prinio. 
\par 4 Svakako, da je na zemlji, ne bi bio svećenik jer postoje oni  koji po Zakonu prinose darove. 
\par 5 Oni služe slici i sjeni onoga  nebeskoga, kako je upućen Mojsije kad se spremao praviti šator:  Pazi, veli doista, načini sve po praliku koji ti je  pokazan na brdu. 
\par 6 Ovako mu pak dopalo uzvišenije bogosluženje koliko je  Posrednik boljega Saveza, koji je uzakonjen na boljim obećanjima. 
\par 7 Da je, zbilja, onaj prvi bio besprijekoran, ne bi se drugome  tražilo mjesto. 
\par 8 Doista, kudeći ih veli: Evo dolaze dani - govori Gospodin - kad ću s domom Izraelovim i s domom Judinim dovršiti novi Savez. 
\par 9 Ne Savez kakav učinih s ocima njihovim u dan kad ih uzeh za ruku da ih izvedem iz zemlje egipatske jer oni ne ustrajaše u mom Savezu pa i ja zanemarih njih - govori Gospodin. 
\par 10 Nego, ovo je Savez kojim ću se svezati s domom Izraelovim nakon ovih dana - govori Gospodin: Zakone ću svoje staviti u dušu njihovu i upisati ih u njihova srca. I bit ću Bog njihov, a oni narod moj. 
\par 11 I neće više nitko učiti sugrađanina i nitko brata svoga govoreći: "Spoznaj Gospodina", ta svi će me poznavati, malo i veliko, 
\par 12 jer ću se smilovati bezakonjima njihovim i grijeha se njihovih neću više spominjati. 
\par 13 Kad veli novi, ostari onaj prvi. Što pak stÓari  i dotrajava, blizu je nestanku. 


\chapter{9}

\par 1 I onaj prvi je, svakako, imao bogoštovne uredbe i Svetinju, ali ovosvjetsku. 
\par 2 Šator je uistinu bio uređen: prvi, u kojem  bijaše svijećnjak, stol i prinos kruhova, a zove se Svetinja; 
\par 3 iza druge pak zavjese bio je Šator zvan Svetinja nad svetinjama  - 
\par 4 u njoj zlatni kadionik i Kovčeg saveza, sav optočen zlatom, a u njemu zlatna posuda s manom i štap Aronov, koji je ono procvao, i ploče Saveza; 
\par 5 povrh njega pak kerubi Slave što osjenjuju  Pomirilište. O tom ne treba sada potanko govoriti. 
\par 6 Pošto je to tako uređeno, u prvi Šator stalno ulaze svećenici  obavljati bogoslužje, 
\par 7 a u drugi jednom godišnje samo veliki  svećenik, i to ne bez krvi koju prinosi za sebe i za nepažnje  naroda. 
\par 8 Time Duh Sveti očituje da još nije otkriven put u  Svetinju dok još postoji prvi Šator. 
\par 9 To je slika za sadašnje  vrijeme: prinose se darovi i žrtve koje ne mogu u savjesti usavršiti  bogoslužnika - 
\par 10 sve same na ićima i pićima i raznim pranjima  utemeljene tjelesne uredbe, nametnute do časa ispravka. 
\par 11 Krist se pak pojavi kao Veliki svećenik budućih dobara  pa po većem i savršenijem Šatoru - nerukotvorenu, koji nije od  ovoga stvorenja - 
\par 12 i ne po krvi jaraca i junaca, nego po svojoj  uđe jednom zauvijek u Svetinju i nađe vječno otkupljenje. 
\par 13 Doista, ako već poškropljena krv jaraca i bikova i pepeo  juničin posvećuje onečišćene, daje tjelesnu čistoću, 
\par 14 koliko  će više krv Krista - koji po Duhu vječnom samoga sebe bez mane  prinese Bogu - očistiti savjest našu od mrtvih djela, na službu  Bogu živomu! 
\par 15 A radi ovoga je Posrednik novoga Saveza: da po  smrti za otkupljenje prekršaja iz starog Saveza pozvani zadobiju  obećanu vječnu baštinu. 
\par 16 Jer gdje je posrijedi savez-oporuka, potrebno je dokazati smrt oporučitelja. 
\par 17 Oporuka je doista  valjana tek nakon smrti: nikad ne vrijedi dok oporučitelj živi. 
\par 18 Stoga ni onaj prvi Savez nije bez krvi ustanovljen. 
\par 19 Pošto  je svemu narodu priopćio svaku zapovijed zakonsku, uze Mojsije  krv junaca i jaraca s vodom i grimiznom vunom i izopom te samu  Knjigu i sav narod poškropi 
\par 20 govoreći: Ovo je krv Saveza  koji vam odredi Bog; 
\par 21 a onda krvlju slično poškropi i  Šator i sve bogoslužno posuđe. 
\par 22 I gotovo se sve po zakonu  čisti krvlju i bez prolijevanja krvi nema oproštenja. 
\par 23 Ako  se dakle time čiste slike onoga što je na nebu, potrebno je da  se samo to nebesko čisti žrtvama od tih uspješnijima. 
\par 24 Krist doista ne uđe u rukotvorenu Svetinju, protulik  one istinske, nego u samo nebo: da se sada pojavi pred licem  Božjim za nas. 
\par 25 Ne da mnogo puta prinosi samoga sebe kao što  veliki svećenik svake godine ulazi u Svetinju s tuđom krvlju; 
\par 26 inače bi bilo trebalo da trpi mnogo puta od postanka svijeta.  No sada se pojavio, jednom na svršetku vjekova, da grijeh dokine  žrtvom svojom. 
\par 27 I kao što je ljudima jednom umrijeti, a potom na sud, 
\par 28 tako i Krist: jednom se prinese da grijehe mnogih ponese, a drugi će se put - bez obzira na grijeh - ukazati onima koji  ga iščekuju sebi na spasenje. 


\chapter{10}

\par 1 Budući da Zakon ima tek sjenu budućih dobara, a ne sam lik  zbiljnosti, on uistinu žrtvama koje se - iz godine u godinu iste  - neprestano prinose ne može nikada usavršiti one što pristupaju. 
\par 2 Ta ne bi li se prestale prinositi kad bogoslužnici, jednom  očišćeni, ne bi više imali nikakve svijesti grijeha? 
\par 3 Ali po  njima se iz godine u godinu podsjeća na grijehe. 
\par 4 Jer krv bikova i jaraca nikako ne može odnijeti grijeha. 
\par 5 Zato On ulazeći u svijet veli: Žrtva i prinos ne mile ti se, nego si mi tijelo pripravio; 
\par 6 paljenice i okajnice ne sviđaju ti se. 
\par 7 Tada rekoh: "Evo dolazim!" U svitku knjige piše za mene: "Vršiti, Bože, volju tvoju!" 
\par 8 Pošto gore reče: Žrtve i prinosi, paljenice i okajnice  - koje se po Zakonu prinose - ne mile ti se i ne sviđaju, 
\par 9 veli zatim: Evo dolazim vršiti volju tvoju! Dokida  prvo da uspostavi drugo. 
\par 10 U toj smo volji posvećeni prinosom tijela Isusa Krista  jednom zauvijek. 
\par 11 I svaki je svećenik dan za danom u bogoslužju te učestalo  prinosi iste žrtve, koje nikako ne mogu odnijeti grijeha. 
\par 12 A  ovaj, pošto je prinio jednu jedincatu žrtvu za grijehe, zauvijek  sjede zdesna Bogu 
\par 13 čekajući otad dok se neprijatelji  ne podlože za podnožje nogama njegovim. 
\par 14 Jednim uistinu  prinosom zasvagda usavrši posvećene. 
\par 15 A to nam svjedoči i Duh Sveti. Pošto je doista rekao: 
\par 16 "Ovo je Savez kojim ću se svezati s njima nakon ovih dana", Gospodin govori: "Zakone ću svoje staviti u njihova srca i upisati ih u dušu njihovu. 
\par 17 I grijeha se njihovih i bezakonja njihovih neću više spominjati." 
\par 18 A gdje su grijesi oprošteni, nema više prinosa za njih. 
\par 19 Imamo dakle, braćo, slobodan ulaz u Svetinju po krvi  Isusovoj - 
\par 20 put nov i živ što nam ga On otvori kroz zavjesu, to jest svoje tijelo; 
\par 21 imamo i Velikog svećenika nad kućom  Božjom. 
\par 22 Pristupajmo stoga s istinitim srcem u punini vjere, srdaca škropljenjem očišćenih od zle savjesti i tijela oprana  čistom vodom. 
\par 23 Čuvajmo nepokolebljivu vjeru nade jer je vjeran  Onaj koji dade obećanje. 
\par 24 I pazimo jedni na druge da se potičemo  na ljubav i dobra djela 
\par 25 te ne propuštamo svojih sastanaka, kako je u nekih običaj, nego se hrabrimo, to više što više vidite  da se bliži Dan. 
\par 26 Jer ako svojevoljno griješimo pošto primismo spoznanje  istine, nema više žrtve za grijehe, 
\par 27 nego strašno isčekivanje  suda i bijesa ognja što će proždrijeti protivnike. 
\par 28 Je li  tko prekršio Zakon Mojsijev, bez milosrđa biva pogubljen na  osnovi dvojice ili trojice svjedoka. 
\par 29 Zamislite koliko  li će goru kaznu zavrijediti tko Sina Božjega pogazi, i nečistom  smatra krv Saveza kojom je posvećen, i Duha milosti pogrdi? 
\par 30 Ta poznajemo Onoga koji je rekao: Moja je odmazda, ja  ću je vratiti; i još: Sudit će Gospodin svome puku. 
\par 31 Strašno je upasti u ruke Boga živoga. 
\par 32 A spomenite se onih prvih dana kada ste, tek prosvijetljeni, izdržali veliku patničku borbu: 
\par 33 ovamo javno izvrgnuti porugama  i nevoljama, onamo postavši zajedničari onih s kojima se tako  postupalo. 
\par 34 I doista, sa sužnjevima ste suosjećali i s radošću  prihvatili otimanje dobara znajući da imate bolji, trajan posjed. 
\par 35 Ne gubite dakle pouzdanja! Pripada mu velika plaća! 
\par 36 Postojanosti  vam uistinu treba da biste vršeći volju Božju zadobili obećano. 
\par 37 Jer još malo, sasvim malo, i Onaj koji dolazi doći će i neće zakasniti 
\par 38 A pravednik će moj od vjere živjeti, ako li pak otpadne, ne mili se on duši mojoj. 
\par 39 A mi nismo od onih koji otpadaju, sebi na propast, nego od  onih koji vjeruju na spas duše. 


\chapter{11}

\par 1 A vjera je već neko imanje onoga čemu se nadamo, uvjerenost  u zbiljnosti kojih ne vidimo. 
\par 2 Zbog nje stari primiše svjedočanstvo. 
\par 3 Vjerom spoznajemo da su svjetovi uređeni riječju Božjom  tako te ovo vidljivo ne posta od nečega pojavnoga. 
\par 4 Vjerom Abel prinese Bogu bolju žrtvu nego Kain. Po njoj  primi svjedočanstvo da je pravedan - Bog nad njegovim darovima  posvjedoči - po njoj i mrtav još govori. 
\par 5 Vjerom Henok bi prenesen da ne vidi smrti te iščeznu  jer ga je prenio Bog. Doista, prije prijenosa primio je svjedočanstvo  da omilje Bogu. 
\par 6 A bez vjere nemoguće je omiljeti Bogu  jer tko mu pristupa, vjerovati mora da postoji i da je platac  onima koji ga traže. 
\par 7 Vjerom Noa, upućen u ono što još ne bijaše vidljivo, predano  sagradi korablju na spasenje svoga doma. Time osudi svijet i  postade baštinikom vjerničke pravednosti. 
\par 8 Vjerom pozvan, Abraham posluša i zaputi se u kraj koji  je imao primiti u baštinu, zaputi se ne znajući kamo ide. 
\par 9 Vjerom  se kao pridošlica naseli u obećanoj zemlji kao u tuđini, prebivajući  pod šatorima s Izakom i Jakovom, subaštinicima istog obećanja, 
\par 10 jer iščekivaše onaj utemeljeni Grad kojemu je graditelj  i tvorac Bog. 
\par 11 Vjerom i Sara unatoč svojoj dobi zadobi moć da začne  jer vjernim smatraše Onoga koji joj dade obećanje. 
\par 12 Zato od  jednoga, i to obamrla, nasta mnoštvo poput zvijezda na nebu  i pijeska nebrojena na obali morskoj. 
\par 13 U vjeri svi su oni umrli, a da nisu zadobili obećanja, već su ih samo izdaleka vidjeli i pozdravili priznavši da su  stranci i pridošlice na zemlji. 
\par 14 Doista, koji tako  govore, jasno očituju da domovinu traže. 
\par 15 Dakako, da su mislili  na onu iz koje su izišli, imali bi još prilike vratiti se u nju. 
\par 16 Ali sada oni čeznu za boljom, to jest nebeskom. Stoga se  Bog ne stidi zvati se Bogom njihovim: ta pripravio im je Grad. 
\par 17 Vjerom Abraham, kušan, prikaza Izaka. Jedinca prikazivaše  on koji je primio obećanje, 
\par 18 kome bi rečeno: Po Izaku će  ti se nazivati potomstvo! - 
\par 19 uvjeren da Bog može i od  mrtvih uskrisiti. Zato ga u predslici i ponovno zadobi. 
\par 20 Vjerom baš u pogledu budućnosti Izak blagoslovi Jakova  i Ezava. 
\par 21 Vjerom Jakov, umirući, blagoslovi oba sina Josipova  i duboko se prignu oslonjen na vrh svojega štapa. 
\par 22 Vjerom  Josip na umoru napomenu ono o izlasku sinova Izraelovih i dade  zapovijed o svojim kostima. 
\par 23 Vjerom su Mojsija netom rođena tri mjeseca krili  njegovi roditelji jer vidješe da je djetešce lijepo  i nisu se bojali kraljeve naredbe. 
\par 24 Vjerom Mojsije, već  odrastao, odbi zvati se sinom kćeri faraonove. 
\par 25 Radije  izabra biti zlostavljan zajedno s Božjim narodom, nego se časovito  okoristiti grijehom. 
\par 26 Većim je bogatstvom od blaga egipatskih  smatrao muku Kristovu jer je gledao na plaću. 
\par 27 Vjerom napusti  Egipat, ne bojeći se bijesa kraljeva, postojan kao da Nevidljivoga  vidi. 
\par 28 Vjerom je obavio pashu i škropljenje krvlju  da Zatornik ne dotakne prvenaca Izraelovih. 
\par 29 Vjerom prođoše Crvenim morem kao po suhu, što i Egipćani  pokušaše, ali se potopiše. 
\par 30 Vjerom zidine jerihonske padoše  nakon sedmodnevnoga ophoda. 
\par 31 Vjerom Rahaba, bludnica, ne propade  zajedno s nepokornicima jer s mirom primi uhode. 
\par 32 I što još da kažem? Ta ponestat će mi vremena, počnem  li raspredati o Gideonu, Baraku, Samsonu, Jiftahu, Davidu, pa  Samuelu i prorocima, 
\par 33 koji su po vjeri osvojili kraljevstva, odjelotvorili pravednost, zadobili obećano, začepili ralje lavovima, 
\par 34 pogasili žestinu ognja, umakli oštrici mača, oporavili se  od slabosti, ojačali u boju, odbili navale tuđinaca. 
\par 35 Žene  su po uskrsnuću ponovno zadobile svoje pokojne. Drugi pak, stavljeni  na muke, ne prihvatiše oslobođenja da bi ih zapalo bolje uskrsnuće. 
\par 36 Drugi su opet iskusili izrugivanja i bičeve, pa i okove i  tamnicu. 
\par 37 Kamenovani su, piljeni, poubijani oštricom mača, potucali se u runima, u kozjim kožusima, u oskudici, potlačeni, zlostavljani - 
\par 38 svijet ih ne bijaše dostojan - vrludali po  pustinjama, gorama, pećinama i pukotinama zemaljskim. 
\par 39 I svi oni po vjeri, istina, primiše svjedočanstvo, ali  ne zadobiše obećano 
\par 40 jer Bog je za nas predvidio nešto bolje  da oni bez nas ne dođu do savršenstva. 


\chapter{12}

\par 1 Zato i mi, okruženi tolikim oblakom svjedoka, odložimo svaki  teret i grijeh koji nas sapinje te postojano trčimo u borbu koja  je pred nama! 
\par 2 Uprimo pogled u Početnika i Dovršitelja vjere, Isusa, koji umjesto radosti što je stajala pred njim podnese  križ, prezrevši sramotu te sjedi zdesna prijestolja Božjega. 
\par 3 Doista pomno promotrite njega, koji podnese toliko protivljenje  grešnika protiv sebe, da - premoreni - ne klonete duhom. 
\par 4 Ta još se do krvi ne oduprijeste u borbi protiv grijeha. 
\par 5 Pa zar ste zaboravili opomenu koja vam je kao sinovima upravljena: Sine moj, ne omalovažavaj stege Gospodnje i ne kloni kad te on ukori. 
\par 6 Jer koga Gospodin ljubi, onoga i stegom odgaja, šiba sina koga voli. 
\par 7 Poradi vašega odgajanja trpite. Bog s vama postupa kao  sa sinovima: a ima li koji sin kojega otac stegom ne odgaja? 
\par 8 Pa ako niste pod stegom, na kojoj su svi imali udjela, onda  ste kopilad, a ne djeca. 
\par 9 Zatim, tjelesne smo oce imali odgojiteljima  i poštovali ih. Pa nećemo li se kudikamo više podlagati Ocu duhova  te živjeti? 
\par 10 Oni su nas doista nešto malo dana stegom odgajali  kako se njima činilo, a On - nama na korist, da postanemo sudionici  njegove svetosti. 
\par 11 Isprva se doduše čini da nijedno odgajanje nije radost, nego žalost, ali onima koji su njime uvježbani poslije donosi  mironosni plod pravednosti. 
\par 12 Zato uspravite ruke klonule  i koljena klecava, 
\par 13 poravnite staze za noge svoje da  se hromo ne iščaši, nego, štoviše, da ozdravi. 
\par 14 Nastojte oko mira sa svima! I oko posvećenja bez kojega  nitko neće vidjeti Gospodina! 
\par 15 Pripazite da se tko ne sustegne  od milosti Božje, da kakav gorki korijen ne proklija pa ne unese  zabunu i ne zarazi mnoge, 
\par 16 da tko ne postane bludnik ili svetogrdnik  kao Ezav, koji za jedan jedini obrok proda svoje prvorodstvo. 
\par 17 Ta znate da je i poslije, kad je htio baštiniti blagoslov, odbačen jer nije našao mogućnosti promjene premda ju je sa suzama  tražio. 
\par 18 Jer niste pristupili opipljivoj gori i usplamtjelu  ognju, ni mraku, tami i vihoru, 
\par 19 ni ječanju trublje i tutnjavi  riječi. - Koji su je slušali, zamoliše da im se više ne govori 
\par 20 jer nisu podnosili naredbe: Ako se ma i živinče dotakne  brda, neka se kamenuje! 
\par 21 I prizor bijaše tako strašan  da Mojsije reče: "Strah me je i dršćem!" - 
\par 22 Nego, vi  ste pristupili gori Sionu i gradu Boga živoga, Jeruzalemu nebeskom, nebrojenim tisućama anđela, svečanom skupu, 
\par 23 Crkvi prvorođenaca  zapisanih na nebu, Bogu, sucu sviju, dusima savršenih pravednika 
\par 24 i Posredniku novog Saveza - Isusu - i krvi škropljeničkoj  što snažnije govori od Abelove. 
\par 25 Pazite da ne odbijete Onoga koji vam govori! Jer ako  ne umakoše oni što su odbili onoga koji je na zemlji davao upute, kudikamo ćemo manje mi ako se okrenemo od Onoga koji ih daje  s nebesa. 
\par 26 Njegov glas tada zemlju uzdrma, sada pak obećava:  Još jednom ja ću potresti ne samo zemlju nego i  nebo. 
\par 27 Ono "još jednom" pokazuje da će, kao stvoreno, uminuti ono uzdrmano da ostane ono neuzdrmljivo. 
\par 28 Zato jer  smo primili kraljevstvo neuzdrmljivo, iskazujmo zahvalnost iz  koje služimo Bogu kako je njemu milo, s predanjem i strahopoštovanjem. 
\par 29 Jer Bog je naš oganj što proždire. 



\chapter{13}

\par 1 Bratoljublje neka je trajno! 
\par 2 Gostoljublja ne zaboravljajte:  njime neki, i ne znajući, ugostiše anđele! 
\par 3 Sjećajte se uznika  kao suuznici; zlostavljanih - ta i sami ste u tijelu! 
\par 4 Ženidba  neka bude u časti u sviju i postelja neokaljana! Jer bludnicima  će i preljubnicima suditi Bog. 
\par 5 U življenju ne budite srebroljupci, zadovoljni onim što imate! Ta on je rekao: Ne, neću te zapustiti  i neću te ostaviti. 
\par 6 Zato možemo pouzdano reći: Gospodin mi je pomoćnik, ja ne strahujem: što mi tko može? 
\par 7 Spominjite se svojih glavara koji su vam nevješćivali  riječ Božju: promatrajući kraj njihova života, nasljedujte njihovu  vjeru. 
\par 8 Isus Krist jučer i danas isti je - i uvijeke. 
\par 9 Ne  dajte se zanijeti različitim tuđim naucima! Jer bolje je srce  utvrđivati milošću nego jelima, koja nisu koristila onima što  su ih obdržavali. 
\par 10 Imamo žrtvenik s kojega nemaju pravo jesti  služitelji Šatora. 
\par 11 Jer tijela životinja, kojih krv veliki  svećenik unosi za grijeh u Svetinju, spaljuju se izvan tabora. 
\par 12 Zato i Isus, da bi vlastitom krvlju posvetio narod, trpio  je izvan vrata. 
\par 13 Stoga iziđimo k njemu izvan tabora noseći  njegovu muku 
\par 14 jer nemamo ovdje trajna grada, nego onaj budući  tražimo. 
\par 15 Po njemu dakle neprestano prinosimo Bogu žrtvu  hvalbenu, to jest plod usana što ispovijedaju ime  njegovo. 
\par 16 Dobrotvornosti i zajedništva ne zaboravljajte jer  takve su žrtve mile Bogu! 
\par 17 Poslušni budite svojim glavarima i podložni jer oni bdiju  nad vašim dušama kao oni koji će polagati račun; neka to čine  s radošću, a ne uzdišući jer vam to ne bi bilo korisno. 
\par 18 Molite za nas! Uvjereni smo doista da imamo dobru savjest  i u svemu se želimo dobro ponašati. 
\par 19 Usrdnije vas pak molim:  učinite to kako bih vam se što brže vratio. 
\par 20 A Bog mira, koji po krvi vječnoga Saveza od mrtvih izvede velikoga Pastira ovaca, Gospodina našega Isusa, 
\par 21 osposobio vas za svako dobro djelo da vršite volju njegovu, činio u nama što je njemu milo, po Isusu Kristu, komu slava u vijeke vjekova. Amen. 
\par 22 Molim vas, braćo, podnesite ovu riječ ohrabrenja: ta  samo vam ukratko napisah! 
\par 23 Znajte: naš je brat Timotej oslobođen.  Ako uskoro stigne, s njime ću vas pohoditi. 
\par 24 Pozdravite sve svoje glavare i sve svete! Pozdravljaju vas ovi iz Italije. 
\par 25 Milost sa svima vama! 




\end{document}