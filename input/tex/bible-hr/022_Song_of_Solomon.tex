\begin{document}

\title{Pjesma nad pjesmama}


\chapter{1}

\par 1 Salomonova  Pjesma nad pjesmama 
\par 2 Poljubi me poljupcem usta svojih, ljubav je tvoja slađa od vina. 
\par 3 Miris najboljih mirodija, ulje razlito ime je tvoje, zato te ljube djevojke. 
\par 4 Povuci me za sobom, bježimo! Kralj me uveo u odaje svoje. Igrat ćemo se i radovati zbog tebe, slavit ćemo ljubav tvoju više nego vino. Pravo je da te ljube. 
\par 5 Crna sam ali lijepa, kćeri jeruzalemske, kao šatori kedarski, kao zavjese Salomonove. 
\par 6 Ne gledajte što sam garava, to me sunce opalilo. Sinovi majke moje rasrdili se na mene, postavili me da čuvam vinograde; a svog vinograda, koji je u meni, nisam čuvala. 
\par 7 Reci mi, ti koga ljubi duša moja, gdje paseš, gdje se u podne odmaraš, da ne lutam, tražeći te, oko stada tvojih drugova. 
\par 8 Ako ne znaš, o najljepša među ženama, izađi i slijedi tragove stada i pasi kozliće svoje oko pastirskih koliba. 
\par 9 Usporedio bih te s konjima pod kolima faraonovim, o prijateljice moja. 
\par 10 Lijepi su obrazi tvoji među naušnicama, vrat tvoj pod ogrlicama. 
\par 11 Učinit ćemo za tebe zlatne naušnice s privjescima srebrnim. 
\par 12 - Dok se kralj odmara na svojim dušecima, (tada) nard moj miriše. 
\par 13 Dragi mi je moj stručak smirne što mi među grudima počiva. 
\par 14 Dragi mi je moj grozd ciprov u vinogradima engedskim. 
\par 15 - Gle, kako si lijepa, prijateljice moja, gle, kako si lijepa, imaš oči kao golubica. 
\par 16 - Gle, kako si lijep, dragi moj, gle, kako si mio. Zelenilo je postelja naša. 
\par 17 - Grede kuća naših cedri su, a natkrovlje čempresi. 


\chapter{2}

\par 1 - Ja sam cvijet šaronski, ljiljan u dolu. 
\par 2 - Što je ljiljan među trnjem, to je prijateljica moja među djevojkama. 
\par 3 - Što je jabuka među šumskim stablima, to je dragi moj među mladićima; bila sam željna hlada njezina i sjedoh, plodovi njeni slatki su grlu mome. 
\par 4 Uveo me u odaje vina i pokrio me zastavom ljubavi. 
\par 5 Okrijepite me kolačima, osvježite jabukama, jer sam bolna od ljubavi. 
\par 6 Njegova mi je lijeva ruka pod glavom, a desnom me grli. 
\par 7 - Kćeri jeruzalemske, zaklinjem vas srnama i košutama poljskim, ne budite, ne budite ljubav moju dok sama ne bude htjela! 
\par 8 Glas dragoga moga! Evo ga, dolazi, prelijeće brda, preskakuje brežuljke. 
\par 9 Dragi je moj kao srna, on je kao jelenče. Evo ga za našim zidom, gleda kroz prozore, zaviruje kroz rešetke. 
\par 10 Dragi moj podiže glas i govori mi: "Ustani, dragano moja, ljepoto moja, i dođi, 
\par 11 jer evo, zima je već minula, kiša je prošla i nestala. 
\par 12 Cvijeće se po zemlji ukazuje, vrijeme pjevanja dođe i glas se grličin čuje u našem kraju. 
\par 13 Smokva je izbacila prve plodove, vinograd, u cvatu, miriše. Ustani, dragano moja, ljepoto moja i dođi. 
\par 14 Golubice moja, u spiljama kamenim, u skrovištima vrletnim, daj da ti vidim lice i da ti čujem glas, jer glas je tvoj ugodan i lice je tvoje krasno." 
\par 15 Pohvatajte lisice, male lisice što oštećuju vinograde, naše vinograde u cvatu. 
\par 16 Dragi moj pripada meni, a ja njemu, on pase među ljiljanima. 
\par 17 Prije nego dan izdahne i sjene se spuste, vrati se, dragi moj: budi lagan kao srna, kao lane na gori Beteru. 


\chapter{3}

\par 1 Po ležaju svome, u noćima, tražila sam onoga koga ljubi duša moja, tražila sam ga, ali ga nisam našla. 
\par 2 Ustat ću dakle i optrčati grad, po ulicama i trgovima tražit ću onoga koga ljubi duša moja: tražila sam ga, ali ga nisam našla. 
\par 3 Sretoše me čuvari koji grad obilaze: "Vidjeste li onoga koga ljubi duša moja?" 
\par 4 Tek što pođoh od njih, nađoh onoga koga ljubi duša moja. Uhvatila sam ga i neću ga pustiti, dok ga ne uvedem u kuću majke svoje, u ložnicu roditeljke svoje. 
\par 5 Zaklinjem vas, kćeri jeruzalemske, srnama i košutama poljskim, ne budite, ne budite ljubav moju dok sama ne bude htjela! 
\par 6 Što se to diže iz pustinje kao stup dima iz kada smirne i tamjana i svih prašaka mirodijskih? 
\par 7 Gle, to je nosiljka Salomonova, oko nje šezdeset kršnih momaka između najkršnijih u Izraelu. 
\par 8 Svi su vični mačevima, za rat su izvježbani, svakome je sablja o boku zbog opasnosti noćnih. 
\par 9 Sebi je prijestolje načinio kralj Salomon od drveta libanskoga. 
\par 10 Stupove je napravio od srebra, naslon od zlata, sjedište od grimiza, unutra je sve ukrašeno ljubavlju kćeri jeruzalemskih. 
\par 11 Izađite, kćeri sionske, i vidite kralja Salomona pod dijademom kojim ga mati ovjenčala na dan svadbe njegove, na dan radosti njegova srca. 


\chapter{4}

\par 1 Kako si lijepa, prijateljice moja, kako si lijepa! Imaš oči kao golubica (kad gledaš) ispod koprene. Kosa ti je kao stado koza što izađoše na brdo Gilead. 
\par 2 Zubi su ti kao stado ovaca ostriženih kad s kupanja dolaze: idu dvije i dvije kao blizanke i nijedna nije osamljena. 
\par 3 Usne su tvoje kao trake od grimiza i riječi su tvoje dražesne, kao kriške mogranja tvoji su obrazi pod koprenom tvojom. 
\par 4 Vrat ti je kao kula Davidova, za obranu sagrađena: tisuću štitova visi na njoj, sve oklopi junački. 
\par 5 Tvoje su dvije dojke kao dva laneta, blizanca košutina, što pasu među ljiljanima. 
\par 6 Prije nego dan izdahne i sjene se spuste, poći ću na brdo smirne, na brežuljak tamjana. 
\par 7 Sva si lijepa, prijateljice moja, i nema mane na tebi. 
\par 8 Pođi sa mnom s Libana, nevjesto, pođi sa mnom s Libana. Siđi s vrha Amane, s vrha Senira i Hermona, iz lavljih spilja, s planina leopardskih. 
\par 9 Srce si mi ranila, sestro moja, nevjesto, srce si mi ranila jednim pogledom svojim, jednim samim biserom kolajne svoje. 
\par 10 Kako je slatka ljubav tvoja, sestro moja, nevjesto! Slađa je ljubav tvoja od vina, a miris ulja tvojih ugodniji od svih mirisa. 
\par 11 S usana tvojih, nevjesto, saće kapa, pod jezikom ti je med i mlijeko, a miris je haljina tvojih kao miris libanski. 
\par 12 Ti si vrt zatvoren, sestro moja, nevjesto, vrt zatvoren i zdenac zapečaćen. 
\par 13 Mladice su tvoje vrt mogranja pun biranih plodova: 
\par 14 nard i šafran, mirisna trska i cimet, sa svim stabljikama tamjanovim, smirna i aloj s najboljim mirisima. 
\par 15 Zdenac je u mom vrtu, izvor žive vode koja teče s Libana. 
\par 16 Ustani, sjevernjače, duni, južni vjetre, duni nad vrtom mojim, neka poteku njegovi mirisi. Neka dragi moj dođe u vrt svoj, neka jede najbolje plodove u njemu. 


\chapter{5}

\par 1 Došao sam u vrt svoj, o sestro moja, nevjesto, berem smirnu svoju i balzam svoj, jedem med svoj i saće svoje, pijem vino svoje i mlijeko svoje. Jedite, prijatelji, pijte i opijte se, mili moji! 
\par 2 Ja spavam, ali srce moje bdi. Odjednom glas! Dragi moj mi pokuca: "Otvori mi, sestro moja, prijateljice moja, golubice moja, savršena moja, glava mi je puna rose a kosa noćnih kapi." 
\par 3 "Svukla sam odjeću svoju, kako da je odjenem? Noge sam oprala, kako da ih okaljam?" 
\par 4 Dragi moj promoli ruku kroz otvor, a sva mi utroba uzdrhta. 
\par 5 Ustadoh da otvorim dragome svome, a iz ruke mi prokapa smirna i poteče niz prste na ručku zavora. 
\par 6 Otvorih dragome svome, ali on se već bijaše udaljio i nestao. Ostala sam bez daha kad je otišao. Tražila sam ga, ali ga nisam našla, zvala sam, ali nije se odazvao. 
\par 7 Sretoše me čuvari koji grad obilaze, tukli su me, ranili i plašt mi uzeli čuvari zidina. 
\par 8 Zaklinjem vas, kćeri jeruzalemske, ako nađete dragoga moga, što ćete mu reći? Da sam bolna od ljubavi. 
\par 9 Što je tvoj dragi bolji od drugih, o najljepša među ženama, što je tvoj dragi bolji od drugih te nas toliko zaklinješ? 
\par 10 Dragi je moj bijel i rumen, ističe se među tisućama. 
\par 11 Glava je njegova kao zlato, zlato čisto, uvojci kao palmove mladice, crne poput gavrana. 
\par 12 Oči su njegove kao golubi nad vodom potočnom; zubi mu kao mlijekom umiveni, u okvir poredani. 
\par 13 Obrazi su njegovi kao lijehe mirisnog bilja, kao cvijeće ugodno, usne su mu ljiljani iz kojih smirna teče. 
\par 14 Ruke su mu zlatno prstenje puno dragulja, prsa su njegova kao čista bjelokost pokrita safirima. 
\par 15 Noge su mu stupovi od mramora na zlatnom podnožju. Stas mu je kao Liban, vitak poput cedra. 
\par 16 Govor mu je sladak i sav je od ljupkosti. Takav je dragi moj, takav je prijatelj moj, o kćeri jeruzalemske. 


\chapter{6}

\par 1 Kamo je otišao dragi tvoj, o najljepša među ženama? Kuda je zamakao dragi tvoj, da ga tražimo s tobom? 
\par 2 Dragi je moj sišao u svoj vrt k lijehama mirisnog bilja da pase po vrtovima i da bere ljiljane. 
\par 3 Ja pripadam dragome svome, dragi moj pripada meni, on pase među ljiljanima. 
\par 4 Lijepa si, prijateljice moja, kao Tirsa, krasna si kao Jeruzalem, strašna kao vojska pod zastavama. 
\par 5 Odvrati oči svoje od mene jer me zbunjuju. Kosa je tvoja kao stado koza koje silaze s Gileada. 
\par 6 Zubi su ti kao stado ovaca ostriženih kada s kupanja dolaze: idu dvije i dvije kao blizanke i nijedna nije osamljena. 
\par 7 Kao kriške mogranja tvoji su obrazi pod koprenom tvojom. 
\par 8 Ima šezdeset kraljica, osamdeset inoča, a djevojaka ni broja se ne zna. 
\par 9 Ali je samo jedna golubica moja, savršena moja, jedina u majke, izabrana u roditeljke svoje. Vidjele su je djevojke i nazvale je blaženom, a kraljice i inoče hvale su joj izrekle. 
\par 10 Tko je ova koja dolazi kao što zora sviće, lijepa kao mjesec, sjajna kao sunce, strašna kao vojska pod zastavama? 
\par 11 Siđoh kroz nasade oraha da vidim mladice u dolinama, da pogledam pupaju li vinogradi, cvatu li mogranji. 
\par 12 Ne znam kako, tek želja moja pope me na kola naroda mog kneževskog. 
\par 13 (7:1) Vrati se, Sulamko, vrati se, vrati se da te gledamo! Što ćete vidjeti na Sulamki koja pleše u dva zbora? 


\chapter{7}

\par 1 (7:2) Kako su krasni koraci tvoji u sandalama, kćeri kneževska! Pregibi su bokova tvojih kao grivne stvorene rukom umjetnika. 
\par 2 (7:3) Pupak ti je kao okrugla čaša koja nikad nije bez pića. Trbuh ti je kao stog pšenice ograđen ljiljanima. 
\par 3 (7:4) Dvije su dojke tvoje dva laneta, blizanca košutina. 
\par 4 (7:5) Vrat je tvoj kao kula bjelokosna. Oči su tvoje kao ribnjaci u Hešbonu kod vrata batrabimskih. Nos ti je kao kula libanska što gleda prema Damasku. 
\par 5 (7:6) Glava je tvoja kao brdo Karmel, a kosa na glavi kao purpur i kralj se zapleo u njene pletenice. 
\par 6 (7:7) Kako si lijepa i kako si ljupka, o najdraža, među milinama! 
\par 7 (7:8) Stas je tvoj kao palma, grudi su tvoje grozdovi. 
\par 8 (7:9) Rekoh: popet ću se na palmu da dohvatim vrške njezine, a grudi će tvoje biti kao grozdovi na lozi, miris daha tvoga kao jabuke. 
\par 9 (7:10) Usta su tvoja kao najbolje vino. Koje odlazi ravno dragome mome kao što teče na usnama usnulih. 
\par 10 (7:11) Ja pripadam dragome svome i on je željan mene. 
\par 11 (7:12) Dođi, dragi moj, ići ćemo u polja, noćivat ćemo u selima. 
\par 12 (7:13) Jutrom ćemo ići u vinograde da vidimo pupa li loza, zameće li se grožđe, jesu li procvali mogranji. Tamo ću ti dati ljubav svoju. 
\par 13 (7:14) Mandragore šire miris, u našim kućama ima svakog voća, novoga i starog, za te sam ga čuvala, o najdraži moj! 



\chapter{8}

\par 1 O, da si mi brat, da si sisao prsa majke moje, našla bih te vani, poljubila bih te i nitko me zato ne bi prezirao. 
\par 2 Povela bih te i uvela u kuću majke svoje koja me odgojila, pojila bih te najboljim vinom i sokom od mogranja. 
\par 3 Njegova mi je lijeva ruka pod glavom, a desnom me grli. 
\par 4 Zaklinjem vas, kćeri jeruzalemske, ne budite, ne budite ljubav mojuš dok sama ne bude htjela! 
\par 5 Tko je ta što dolazi iz pustinje, naslonjena na dragoga svoga? Probudio sam te pod jabukom gdje te mati rodila, gdje te na svijet dala roditeljka tvoja. 
\par 6 Stavi me kao znak na srce, kao pečat na ruku svoju, jer ljubav je jaka kao smrt, a ljubomora tvrda kao grob. Žar je njezin žar vatre i plamena Jahvina. 
\par 7 Mnoge vode ne mogu ugasiti ljubav niti je rijeke potopiti. Da netko daje za ljubav sve što u kući ima, taj bi navukao prezir na sebe. 
\par 8 Imamo malu sestru koja još nema grudi, što ćemo činiti sa svojom sestrom kad bude riječ o njoj? 
\par 9 Ako bude poput zida, sagradit ćemo na njemu krunište od srebra; ako bude poput vrata, utvrdit ćemo ih cedrovim daskama. 
\par 10 Ja sam zid i grudi su moje kule: tako postadoh u očima njegovim kao ona što nađe smirenje. 
\par 11 Salomon ima vinograd u Baal Hamonu, dao ga je čuvarima i svaki mora donijeti za urod tisuću srebrnjaka. 
\par 12 Moj vinograd je preda mnom: tebi, Salomone, tisuća, a dvjesta onima što čuvaju plodove. 
\par 13 O ti, koja boraviš u vrtovima, drugovi slušaju glas tvoj, daj da ga i ja čujem! 
\par 14 Pohitaj, mili moj, budi kao srna i kao jelenče na gorama mirisnim! 




\end{document}