\begin{document}

\title{Matej}



\chapter{1}



\par 1 Rodoslovlje Isusa Krista, sina Davidova, sina Abrahamova. 
\par 2 Abrahamu se rodi Izak. Izaku se rodi Jakov. Jakovu se  rodi Juda i njegova braća. 
\par 3 Judi Tamara rodi Peresa i Zeraha.  Peresu se rodi Hesron. Hesronu se rodi Ram. 
\par 4 Ramu se rodi Aminadab.  Aminadabu se rodi Nahšon. Nahšonu se rodi Salma. 
\par 5 Salmi Rahaba  rodi Boaza. Boazu Ruta rodi Obeda. Obedu se rodi Jišaj. 
\par 6 Jišaju  se rodi David kralj. Davidu bivša žena Urijina rodi Salomona. 
\par 7 Salomonu se rodi  Roboam. Roboamu se rodi Abija. Abiji se rodi Asa. 
\par 8 Asi se rodi  Jozafat. Jozafatu se rodi Joram. Joramu se rodi Ahazja. 
\par 9 Ahazji  se rodi Jotam. Jotamu se rodi Ahaz. Ahazu se rodi Ezekija. 
\par 10 Ezekiji  se rodi Manaše. Manašeu se rodi Amon. Amonu se rodi Jošija. 
\par 11 Jošiji  se rodi Jehonija i njegova braća u vrijeme progonstva u Babilon. 
\par 12 Poslije progonstva u Babilon Jehoniji se rodi Šealtiel.  Šealtielu se rodi Zerubabel. 
\par 13 Zerubabelu se rodi Abiud. Abiudu  se rodi Elijakim. Elijakimu se rodi Azor. 
\par 14 Azoru se rodi Sadok.  Sadoku se rodi Akim. Akimu se rodi Elijud. 
\par 15 Elijudu se rodi  Eleazar. Eleazaru se rodi Matan. Matanu se rodi Jakov. 
\par 16 Jakovu  se rodi Josip, muž Marije, od koje se rodio Isus koji se zove  Krist. 
\par 17 U svemu dakle: od Abrahama do Davida četrnaest naraštaja;  od Davida do progonstva u Babilon četrnaest naraštaja; poslije  progonstva u Babilon do Krista četrnaest naraštaja. 
\par 18 A rođenje Isusa Krista zbilo se ovako. Njegova majka  Marija, zaručena s Josipom, prije nego se sastadoše, nađe se  trudna po Duhu Svetom. 
\par 19 A Josip, muž njezin, pravedan, ne  htjede je izvrgnuti sramoti, nego naumi da je potajice napusti. 
\par 20 Dok je on to snovao, gle, anđeo mu se Gospodnji ukaza u snu  i reče: "Josipe, sine Davidov, ne boj se uzeti k sebi Mariju, ženu svoju. Što je u njoj začeto, doista je od Duha Svetoga. 
\par 21 Rodit će sina, a ti ćeš mu nadjenuti ime Isus jer će on spasiti  narod svoj od grijeha njegovih." 
\par 22 Sve se to dogodilo da se ispuni što Gospodin reče po  proroku: 
\par 23 Evo, Djevica će začeti i roditi sina i nadjenut će mu se ime Emanuel - što znači: S nama Bog! 
\par 24 Kad se Josip probudi oda sna, učini kako mu naredi anđeo  Gospodnji: uze k sebi svoju ženu. 
\par 25 I ne upozna je dok ne rodi  sina. I nadjenu mu ime Isus. 


\chapter{2}

\par 1 Kad se Isus rodio u Betlehemu judejskome u dane Heroda kralja, gle, mudraci se s Istoka pojaviše u Jeruzalemu 
\par 2 raspitujući  se: "Gdje je taj novorođeni kralj židovski? Vidjesmo gdje izlazi  zvijezda njegova pa mu se dođosmo pokloniti." 
\par 3 Kada to doču kralj Herod, uznemiri se on i sav Jeruzalem  s njime. 
\par 4 Sazva sve glavare svećeničke i pismoznance narodne  pa ih ispitivaše gdje se Krist ima roditi. 
\par 5 Oni mu odgovoriše:  "U Betlehemu judejskome jer ovako piše prorok: 
\par 6 A ti, Betleheme, zemljo Judina! Nipošto nisi najmanji među kneževstvima Judinim jer iz tebe će izaći vladalac koji će pasti narod moj - Izraela! 
\par 7 Tada Herod potajno dozva mudrace i razazna od njih vrijeme  kad se pojavila zvijezda. 
\par 8 Zatim ih posla u Betlehem: "Pođite, reče, i pomno se raspitajte za dijete. Kad ga nađete, javite  mi da i ja pođem te mu se poklonim." 
\par 9 Oni saslušavši kralja, pođoše. I gle, zvijezda kojoj vidješe  izlazak iđaše pred njima sve dok ne stiže i zaustavi se povrh  mjesta gdje bijaše dijete. 
\par 10 Kad ugledaše zvijezdu, obradovaše  se radošću veoma velikom. 
\par 11 Uđu u kuću, ugledaju dijete s Marijom, majkom njegovom, padnu ničice i poklone mu se. Otvore zatim  svoje blago i prinesu mu darove: zlato, tamjan i smirnu. 
\par 12 Upućeni  zatim u snu da se ne vraćaju Herodu, otiđoše drugim putem u svoju  zemlju. 
\par 13 A pošto oni otiđoše, gle, anđeo se Gospodnji u snu javi  Josipu: "Ustani, reče, uzmi dijete i majku njegovu te bježi u  Egipat i ostani ondje dok ti ne reknem jer će Herod tražiti dijete  da ga pogubi. 
\par 14 On ustane, uzme noću dijete i majku njegovu  te krene u Egipat. 
\par 15 I osta ondje do Herodova skončanja - da  se ispuni što Gospodin reče po proroku: Iz Egipta dozvah Sina  svoga. 
\par 16 Vidjevši da su ga mudraci izigrali, Herod se silno rasrdi  i posla poubijati sve dječake u Betlehemu i po svoj okolici,  od dvije godine naniže - prema vremenu što ga razazna od mudraca. 
\par 17 Tada se ispuni što je rečeno po proroku Jeremiji: 
\par 18 U Rami se glas čuje, kuknjava i plač gorak: Rahela oplakuje sinove svoje i neće da se utješi jer više ih nema. 
\par 19 Nakon Herodova skončanja, gle, anđeo se Gospodnji javi  u snu Josipu u Egiptu: 
\par 20 "Ustani, reče, uzmi dijete i njegovu  majku te pođi u zemlju izraelsku jer su umrli oni koji su djetetu  o glavi radili." 
\par 21 On ustane, uzme dijete i njegovu majku te  uđe u zemlju izraelsku. 
\par 22 Ali saznavši da Arhelaj vlada Judejom  namjesto svoga oca Heroda, bojao se poći onamo pa, upućen u snu, ode u kraj galilejski. 
\par 23 Dođe i nastani se u gradu zvanu Nazaret  - da se ispuni što je rečeno po prorocima: Zvat će se Nazarećanin. 


\chapter{3}

\par 1 U one dane pojavi se Ivan Krstitelj propovijedajući u Judejskoj  pustinji: 
\par 2 "Obratite se jer približilo se kraljevstvo nebesko!" 
\par 3 Ovo je uistinu onaj o kom proreče Izaija prorok: Glas viče u pustinji: Pripravite put Gospodinu, poravnite mu staze! 
\par 4 Ivan je imao odjeću od devine dlake i kožnat pojas oko  bokova; hranom mu bijahu skakavci i divlji med. 
\par 5 Grnuo k njemu  Jeruzalem, sva Judeja i sva okolica jordanska. 
\par 6 Primali su  od njega krštenje u rijeci Jordanu ispovijedajući svoje grijehe. 
\par 7 Kad ugleda mnoge farizeje i saduceje gdje mu dolaze na  krštenje, reče im: "Leglo gujinje! Tko li vas je samo upozorio  da bježite od skore srdžbe? 
\par 8 Donosite dakle plod dostojan obraćenja. 
\par 9 I ne usudite se govoriti u sebi: 'Imamo oca Abrahama!' Jer, kažem vam, Bog iz ovoga kamenja može podići djecu Abrahamovu. 
\par 10 Već je sjekira položena na korijen stablima. Svako dakle  stablo koje ne donosi dobroga roda, siječe se i u oganj baca." 
\par 11 "Ja vas, istina, krstim vodom na obraćenje, ali onaj  koji za mnom dolazi jači je od mene. Ja nisam dostojan obuće  mu nositi. On će vas krstiti Duhom Svetim i ognjem. 
\par 12 U ruci  mu vijača, pročistit će svoje gumno i skupiti žito u svoju žitnicu, a pljevu spaliti ognjem neugasivim." 
\par 13 Tada dođe Isus iz Galileje na Jordan Ivanu da ga on krsti. 
\par 14 Ivan ga odvraćaše: "Ti mene treba da krstiš, a ti da k meni  dolaziš?" 
\par 15 Ali mu Isus odgovori: "Pusti sada! Ta dolikuje  nam da tako ispunimo svu pravednost!" Tada mu popusti. 
\par 16 Odmah  nakon krštenja izađe Isus iz vode. I gle! Otvoriše se nebesa  i ugleda Duha Božjega gdje silazi kao golub i spušta se na nj. 
\par 17 I eto glasa s neba: "Ovo je Sin moj, Ljubljeni! U njemu  mi sva milina!" 


\chapter{4}

\par 1 Duh tada odvede Isusa u pustinju da ga đavao iskuša. 
\par 2 I propostivši  četrdeset dana i četrdeset noći, napokon ogladnje. 
\par 3 Tada mu pristupi napasnik i reče: "Ako si Sin Božji, reci  da ovo kamenje postane kruhom." 
\par 4 A on odgovori: "Pisano je:  Ne živi čovjek samo o kruhu, nego o svakoj riječi što izlazi  iz Božjih usta." 
\par 5 Ðavao ga tada povede u Sveti grad, postavi ga na vrh Hrama 
\par 6 i reče mu: "Ako si Sin Božji, baci se dolje! Ta pisano je: Anđelima će svojim zapovjediti za tebe i na rukama će te nositi da se gdje nogom ne spotakneš o kamen." 
\par 7 Isus mu kaza: "Pisano je također: Ne iskušavaj Gospodina, Boga svojega!" 
\par 8 Ðavao ga onda povede na goru vrlo visoku i pokaza mu sva  kraljevstva svijeta i slavu njihovu 
\par 9 pa mu reče: "Sve ću ti  to dati ako mi se ničice pokloniš." 
\par 10 Tada mu reče Isus: "Odlazi, Sotono! Ta pisano je: Gospodinu, Bogu svom se klanjaj i njemu jedinom služi!" 
\par 11 Tada ga pusti đavao. I gle, anđeli pristupili i služili  mu. 
\par 12 A čuvši da je Ivan predan, povuče se u Galileju. 
\par 13 Ostavi  Nazaret te ode i nastani se u Kafarnaumu, uz more, na području  Zebulunovu i Naftalijevu 
\par 14 da se ispuni što je rečeno po proroku  Izaiji: 
\par 15 Zemlja Zebulunova i zemlja Naftalijeva, Put uz more, s one strane Jordana, Galileja poganska - 
\par 16 narod što je sjedio u tmini svjetlost vidje veliku; onima što mrkli kraj smrti obitavahu svjetlost jarka osvanu. 
\par 17 Otada je Isus počeo propovijedati: "Obratite se jer približilo  se kraljevstvo nebesko!" 
\par 18 Prolazeći uz Galilejsko more, ugleda dva brata, Šimuna  zvanog Petar i brata mu Andriju, gdje bacaju mrežu u more; bijahu  ribari. 
\par 19 I kaže im: "Hajdete za mnom, učinit ću vas ribarima  ljudi!" 
\par 20 Oni brzo ostave mreže i pođu za njim. 
\par 21 Pošavši odande, ugleda druga dva brata, Jakova Zebedejeva  i brata mu Ivana: u lađi su sa Zebedejem, ocem svojim, krpali  mreže. Pozva i njih. 
\par 22 Oni brzo ostave lađu i oca te pođu za  njim. 
\par 23 I obilazio je Isus svom Galilejom naučavajući po njihovim  sinagogama, propovijedajući Evanđelje o Kraljevstvu i liječeći  svaku bolest i svaku nemoć u narodu. 
\par 24 I glas se o njemu pronese svom Sirijom. I donosili su  mu sve koji bolovahu od najrazličitijih bolesti i patnja - opsjednute, mjesečare, uzete - i on ih ozdravljaše. 
\par 25 Za njim je pohrlio silan svijet iz Galileje, Dekapola, Jeruzalema, Judeje i Transjordanije. 


\chapter{5}

\par 1 Ugledavši mnoštvo, uziđe na goru. I kad sjede, pristupe mu  učenici. 
\par 2 On progovori i stane ih naučavati: 
\par 3 "Blago siromasima duhom: njihovo je kraljevstvo nebesko! 
\par 4 Blago ožalošćenima: oni će se utješiti! 
\par 5 Blago krotkima: oni će baštiniti zemlju! 
\par 6 Blago gladnima i žednima pravednosti: oni će se nasititi! 
\par 7 Blago milosrdnima: oni će zadobiti milosrđe! 
\par 8 Blago čistima srcem: oni će Boga gledati! 
\par 9 Blago mirotvorcima: oni će se sinovima Božjim zvati! 
\par 10 Blago progonjenima zbog pravednosti: njihovo je kraljevstvo nebesko!" 
\par 11 "Blago vama kad vas - zbog mene - pogrde i prognaju i  sve zlo slažu protiv vas! 
\par 12 Radujte se i kličite: velika je  plaća vaša na nebesima! Ta progonili su tako proroke prije vas!" 
\par 13 "Vi ste sol zemlje. Ali ako sol obljutavi, čime će se  ona osoliti? Nije više ni za što, nego da se baci van i da ljudi  po njoj gaze." 
\par 14 "Vi ste svjetlost svijeta. Ne može se sakriti grad što  leži na gori. 
\par 15 Niti se užiže svjetiljka da se stavi pod posudu, nego na svijećnjak da svijetli svima u kući. 
\par 16 Tako neka svijetli  vaša svjetlost pred ljudima da vide vaša dobra djela i slave  Oca vašega koji je na nebesima." 
\par 17 "Ne mislite da sam došao ukinuti Zakon ili Proroke. Nisam  došao ukinuti, nego ispuniti. 
\par 18 Zaista, kažem vam, dok ne prođe  nebo i zemlja, ne, ni jedno slovce, ni jedan potezić iz Zakona  neće proći, dok se sve ne zbude. 
\par 19 Tko dakle ukine jednu od  tih, pa i najmanjih zapovijedi i tako nauči ljude, najmanji će  biti u kraljevstvu nebeskom. A tko ih bude vršio i druge učio, taj će biti velik u kraljevstvu nebeskom." 
\par 20 "Uistinu kažem vam: ne bude li pravednost vaša veća od  pravednosti pismoznanaca i farizeja, ne, nećete ući u kraljevstvo  nebesko." 
\par 21 "Čuli ste da je rečeno starima: Ne ubij! Tko ubije, bit će podvrgnut sudu. 
\par 22 A ja vam kažem: Svaki koji se srdi  na brata svoga, bit će podvrgnut sudu. A tko bratu rekne 'Glupane!', bit će podvrgnut Vijeću. A tko reče: 'Luđače!', bit će podvrgnut  ognju paklenomu." 
\par 23 "Ako dakle prinosiš dar na žrtvenik pa se ondje sjetiš  da tvoj brat ima nešto protiv tebe, 
\par 24 ostavi dar ondje pred  žrtvenikom, idi i najprije se izmiri s bratom, a onda dođi i  prinesi dar." 
\par 25 "Nagodi se brzo s protivnikom dok si još s njim na putu, da te protivnik ne preda sucu, a sudac tamničaru, pa da te ne  bace u tamnicu. 
\par 26 Zaita, kažem ti, nećeš izići odande dok ne  isplatiš do posljednjeg novčića." 
\par 27 "Čuli ste da je rečeno: Ne čini preljuba! 
\par 28 A  ja vam kažem: Tko god s požudom pogleda ženu, već je s njome  učinio preljub u srcu. 
\par 29 Ako te desno oko sablažnjava, iskopaj  ga i baci od sebe. Ta bolje je da ti propadne jedan od udova, nego da ti cijelo tijelo bude bačeno u pakao. 
\par 30 Ako te desnica  tvoja sablažnjava, odsijeci je i baci od sebe. Ta bolje je da  ti propadne jedan od udova, nego da ti cijelo tijelo ode u pakao." 
\par 31 "Rečeno je također: Tko otpusti svoju ženu, neka joj  dade otpusnicu. 
\par 32 A ja vam kažem: Tko god otpusti svoju  ženu - osim zbog bludništva - navodi je na preljub i tko se god  otpuštenom oženi, čini preljub." 
\par 33 "Čuli ste još da je rečeno starima: Ne zaklinji se  krivo, nego izvrši Gospodinu svoje zakletve. 
\par 34 A ja vam  kažem: Ne kunite se nikako! Ni nebom jer je prijestolje  Božje. 
\par 35 Ni zemljom jer je podnožje njegovim  nogama. Ni Jeruzalemom jer grad je Kralja velikoga! 
\par 36 Ni svojom se glavom ne zaklinji jer ni jedne vlasi ne možeš  učiniti bijelom ili crnom. 
\par 37 Vaša riječ neka bude: 'Da, da, - ne, ne!' Što je više od toga, od Zloga je." 
\par 38 "Čuli ste da je rečeno: Oko za oko, zub za zub! 
\par 39 A ja vam kažem: Ne opirite se Zlomu! Naprotiv, pljusne li  te tko po desnom obrazu, okreni mu i drugi. 
\par 40 Onomu tko bi  se htio s tobom parničiti da bi se domogao tvoje donje haljine  prepusti i gornju. 
\par 41 Ako te tko prisili jednu milju, pođi s  njim dvije. 
\par 42 Tko od tebe što zaište, podaj mu! I ne okreni  se od onoga koji hoće da mu pozajmiš." 
\par 43 "Čuli ste da je rečeno: Ljubi svoga bližnjega,  a mrzi neprijatelja. 
\par 44 A ja vam kažem: Ljubite neprijatelje, molite za one koji vas progone 
\par 45 da budete sinovi svoga oca  koji je na nebesima, jer on daje da sunce njegovo izlazi nad  zlima i dobrima i da kiša pada pravednicima i nepravednicima. 
\par 46 Jer ako ljubite one koji vas ljube, kakva li vam plaća? Zar  to isto ne čine i carinici? 
\par 47 I ako pozdravljate samo braću, što osobito činite? Zar to isto ne čine i pogani?" 
\par 48 "Budite dakle savršeni kao što je savršen  Otac vaš nebeski!" 


\chapter{6}

\par 1 "Pazite da svoje pravednosti ne činite pred ljudima da vas  oni vide. Inače, nema vam plaće u vašeg Oca koji je na nebesima. 
\par 2 Kada dakle dijeliš milostinju, ne trubi pred sobom, kako to  u sinagogama i na ulicama čine licemjeri da bi ih ljudi hvalili.  Zaista, kažem vam, primili su svoju plaću. 
\par 3 Ti naprotiv, kada  daješ milostinju - neka ti ne zna ljevica što čini desnica, 
\par 4 da  tvoja milostinja bude u skrovitosti. I Otac tvoj, koji vidi u  skrovitosti, uzvratit će ti!" 
\par 5 "Tako i kad molite, ne budite kao licemjeri. Vole moliti  stojeći u sinagogama i na raskršćima ulica da se pokažu ljudima.  Zaista, kažem vam, primili su svoju plaću. 
\par 6 Ti naprotiv, kad  moliš, uđi u svoju sobu, zatvori vrata i pomoli se svomu Ocu, koji je u skrovitosti. I Otac tvoj, koji vidi u skrovitosti, uzvratit će ti." 
\par 7 "Kad molite, ne blebećite kao pogani. Misle da će s mnoštva  riječi biti uslišani. 
\par 8 Ne nalikujte na njih. Ta zna vaš Otac  što vam treba i prije negoli ga zaištete. 
\par 9 Vi, dakle, ovako  molite:  'Oče naš, koji jesi na nebesima! Sveti se ime tvoje! 
\par 10 Dođi kraljevstvo tvoje! Budi volja tvoja kako na nebu tako i na zemlji! 
\par 11 Kruh naš svagdanji daj nam danas! 
\par 12 I opusti nam duge naše kako i mi otpustismo dužnicima svojim! 
\par 13 I ne uvedi nas u napast, nego izbavi nas od Zloga!'" 
\par 14 "Doista, ako vi otpustite ljudima njihove prijestupke, otpustit će i vama Otac vaš nebeski. 
\par 15 Ako li vi ne otpustite  ljudima, ni Otac vaš neće otpustiti vaših prijestupaka." 
\par 16 "I kad postite, ne budite smrknuti kao licemjeri. Izobličuju  lica da pokažu ljudima kako poste. Zaista, kažem vam, primili  su svoju plaću. 
\par 17 Ti naprotiv, kad postiš, pomaži glavu i umij  lice 
\par 18 da ne zapaze ljudi kako postiš, nego Otac tvoj, koji  je u skrovitosti. I Otac tvoj, koji vidi u skrovitosti, uzvratit  će ti." 
\par 19 "Ne zgrćite sebi blago na zemlji, gdje ga moljac i rđa  nagrizaju i gdje ga kradljivci potkapaju i kradu. 
\par 20 Zgrćite  sebi blago na nebu, gdje ga ni moljac ni rđa ne nagrizaju i gdje  kradljivci ne potkapaju niti kradu. 
\par 21 Doista, gdje ti je blago, ondje će ti biti i srce." 
\par 22 "Oko je tijelu svjetiljka. Ako ti je dakle oko bistro, sve će tijelo tvoje biti svijetlo. 
\par 23 Ako ti je pak oko nevaljalo, sve će tijelo tvoje biti tamno. Ako je dakle svjetlost koja  je u tebi - tamna, kolika će istom tama biti?" 
\par 24 "Nitko ne može služiti dvojici gospodara. Ili će jednoga  mrziti, a drugoga ljubiti; ili će uz jednoga prianjati, a drugoga  prezirati. Ne možete služiti Bogu i bogatstvu." 
\par 25 "Zato vam kažem: Ne budite zabrinuti za život svoj: što  ćete jesti, što ćete piti; ni za tijelo svoje: u što ćete se  obući. Zar život nije vredniji od jela i tijelo od odijela?" 
\par 26 "Pogledajte ptice nebeske! Ne siju, ne žanju niti sabiru  u žitnice, pa ipak ih hrani vaš nebeski Otac. Zar niste vi vredniji  od njih? 
\par 27 A tko od vas zabrinutošću može svome stasu dodati  jedan lakat? 
\par 28 I za odijelo što ste zabrinuti? Promotrite poljske  ljiljane, kako rastu! Ne muče se niti predu. 
\par 29 A kažem vam:  ni Salomon se u svoj svojoj slavi ne zaodjenu kao jedan od njih. 
\par 30 Pa ako travu poljsku, koja danas jest a sutra se u peć baca, Bog tako odijeva, neće li još više vas, malovjerni?" 
\par 31 "Nemojte dakle zabrinuto govoriti: 'Što ćemo jesti?'  ili: 'Što ćemo piti?' ili: 'U što ćemo se obući?' 
\par 32 Ta sve  to pogani ištu. Zna Otac vaš nebeski da vam je sve to potrebno. 
\par 33 Tražite stoga najprije Kraljevstvo i pravednost njegovu,  a sve će vam se ostalo dodati. 
\par 34 Ne budite dakle zabrinuti  za sutra. Sutra će se samo brinuti za se. Dosta je svakom danu  zla njegova." 


\chapter{7}

\par 1 "Ne sudite da ne budete suđeni! 
\par 2 Jer sudom kojim sudite bit  ćete suđeni. I mjerom kojom mjerite mjerit će vam se. 
\par 3 Što  gledaš trun u oku brata svojega, a brvna u oku svome ne opažaš? 
\par 4 Ili kako možeš reći bratu svomu: 'De da ti izvadim trun iz  oka', a eto brvna u oku tvom? 
\par 5 Licemjere, izvadi najprije brvno  iz oka svoga pa ćeš onda dobro vidjeti izvaditi trun iz oka bratova!" 
\par 6 "Ne dajte svetinje psima! Niti svoga biserja bacajte pred  svinje da ga ne pogaze nogama pa se okrenu i rastrgaju vas." 
\par 7 "Ištite i dat će vam se! Tražite i naći ćete! Kucajte  i otvorit će vam se! 
\par 8 Doista, tko god ište, prima; i tko traži, nalazi; i onomu koji kuca otvorit će se. 
\par 9 Ta ima li koga među  vama da bi svojemu sinu, ako ga zaište kruha, kamen dao? 
\par 10 Ili  ako ribu zaište, zar će mu zmiju dati? 
\par 11 Ako dakle vi, iako  zli, znate dobrim darima darivati djecu svoju, koliko li će više  Otac vaš, koji je na nebesima, dobrima obdariti one koji ga zaištu!" 
\par 12 "Sve, dakle, što želite da ljudi vama čine, činite i  vi njima. To je, doista, Zakon i Proroci." 
\par 13 "Uđite na uska vrata! Jer široka su vrata i prostran  put koji vodi u propast i mnogo ih je koji njime idu. 
\par 14 O kako  su uska vrata i tijesan put koji vodi u Život i malo ih je koji  ga nalaze!" 
\par 15 "Čuvajte se lažnih proroka koji dolaze k vama u ovčjem  odijelu, a iznutra su vuci grabežljivi. 
\par 16 Po njihovim ćete  ih plodovima prepoznati. Bere li se s trnja grožđe ili s bodljike  smokve? 
\par 17 Tako svako dobro stablo rađa dobrim plodovima, a  nevaljalo stablo rađa plodovima zlim. 
\par 18 Ne može dobro stablo  donijeti zlih plodova niti nevaljalo stablo dobrih plodova. 
\par 19 Svako  stablo koje ne rađa dobrim plodom siječe se i u oganj baca. 
\par 20 Dakle:  po plodovima ćete ih njihovim prepoznati." 
\par 21 "Neće u kraljevstvo nebesko ući svaki koji mi govori:  'Gospodine, Gospodine!', nego onaj koji vrši volju Oca mojega, koji je na nebesima. 
\par 22 Mnogi će me u onaj dan pitati: 'Gospodine, Gospodine! Nismo li mi u tvoje ime prorokovali, u tvoje ime  đavle izgonili, u tvoje ime mnoga čudesa činili?' 
\par 23 Tada ću  im kazati: 'Nikad vas nisam poznavao! Nosite se od mene, vi  bezakonici!'" 
\par 24 "Stoga, tko god sluša ove moje riječi i izvršava ih,  bit će kao mudar čovjek koji sagradi kuću na stijeni. 
\par 25 Zapljušti  kiša, navale bujice, duhnu vjetrovi i sruče se na tu kuću, ali  ona ne pada. Jer - utemeljena je na stijeni." 
\par 26 "Naprotiv, tko god sluša ove moje riječi, a ne vrši ih, bit će kao lud čovjek koji sagradi kuću na pijesku. 
\par 27 Zapljušti  kiša, navale bujice, duhnu vjetrovi i sruče se na tu kuću i ona  se sruši. I bijaše to ruševina velika." 
\par 28 Kad Isus završi ove svoje besjede, mnoštvo osta zaneseno  njegovim naukom. 
\par 29 Ta učio ih kao onaj koji ima vlast, a ne  kao njihovi pismoznanci. 


\chapter{8}

\par 1 Kad je Isus sišao s gore, pohrli za njim silan svijet. 
\par 2 I  gle, pristupi neki gubavac, pokloni mu se do zemlje i reče: "Gospodine, ako hoćeš, možeš me očistiti." 
\par 3 Isus pruži ruku i dotakne  ga se govoreći: "Hoću, očisti se!" I odmah se očisti od gube. 
\par 4 Kaže mu Isus: "Pazi, nikomu ne kazuj, nego idi, pokaži  se svećeniku i prinesi dar što ga propisa Mojsije, njima  za svjedočanstvo." 
\par 5 Kad uđe u Kafarnaum, pristupi mu satnik pa ga zamoli: 
\par 6 "Gospodine, sluga mi leži kod kuće uzet, u strašnim mukama." 
\par 7 Kaže mu: "Ja ću doći izliječiti ga." 
\par 8 Odgovori satnik: "Gospodine, nisam dostojan da uđeš pod krov moj, nego samo reci riječ i  izliječen će biti sluga moj. 
\par 9 Ta i ja, premda sam čovjek pod  vlašću, imam pod sobom vojnike pa reknem jednomu: 'Idi!' - i  ode, drugomu: 'Dođi!' - i dođe, a sluzi svomu: 'Učini to' - i  učini." 
\par 10 Čuvši to, zadivi se Isus i reče onima koji su išli za  njim: "Zaista, kažem vam, ni u koga u Izraelu ne nađoh tolike  vjere. 
\par 11 A kažem vam: Mnogi će s istoka i zapada doći  i sjesti za stol s Abrahamom, Izakom i Jakovom u kraljevstvu  nebeskom, 
\par 12 a sinovi će kraljevstva biti izbačeni van u tamu.  Ondje će biti plač i škrgut zubi." 
\par 13 I reče Isus satniku: "Idi, neka ti bude kako si vjerovao!"  I ozdravi sluga u taj čas. 
\par 14 Ušavši u kuću Petrovu, Isus ugleda njegovu punicu koja  ležaše u ognjici. 
\par 15 Dotače joj se ruke i pusti je ognjica.  Ona ustade i posluživaše mu. 
\par 16 A uvečer mu doniješe mnoge opsjednute. On izagna duhove  riječju i sve bolesnike ozdravi - 
\par 17 da se ispuni što je rečeno  po Izaiji proroku: On slabosti naše uze i boli ponese. 
\par 18 Kad Isus vidje mnoštvo oko sebe, zapovjedi da se prijeđe  prijeko. 
\par 19 I pristupi jedan pismoznanac te mu reče: "Učitelju, za tobom ću kamo god ti pošao." 
\par 20 Kaže mu Isus: "Lisice imaju  jazbine i ptice nebeske gnijezda, a Sin Čovječji nema gdje bi  glavu naslonio." 
\par 21 Drugi mu od učenika reče: "Gospodine, dopusti mi da prije  odem i pokopam svoga oca." 
\par 22 Isus mu kaže: "Hajde za mnom i  pusti neka mrtvi pokapaju svoje mrtve." 
\par 23 Kad uđe u lađu, pođoše za njim njegovi učenici. 
\par 24 I  gle, žestok vihor nasta na moru tako da lađu prekrivahu valovi.  A on je spavao. 
\par 25 Oni pristupiše i probudiše ga govoreći: "Gospodine, spasi, pogibosmo!" 
\par 26 Kaže im: "Što ste plašljivi, malovjerni?"  Tada ustade i zaprijeti vjetrovima i moru te nasta velika utiha. 
\par 27 A ljudi su u čudu pitali: "Tko je taj da mu se i vjetrovi  i more pokoravaju?" 
\par 28 I kada dođe prijeko, u gadarski kraj, eto mu u susret  dvaju opsjednutih: izlazili su iz grobnica, silno goropadni,  te nitko nije mogao proći onim putem. 
\par 29 I gle, povikaše: "Što  ti imaš s nama, Sine Božji? Došao si ovamo prije vremena mučiti  nas?" 
\par 30 A podalje od njih paslo je veliko krdo svinja. 
\par 31 Zlodusi  ga zaklinjahu: "Ako nas istjeraš, pošalji nas u ovo krdo svinja." 
\par 32 On im reče: "Idite!" Oni iziđoše i uđoše u svinje. I gle, sve krdo jurnu niz obronak u more i podavi se u vodama. 
\par 33 A svinjari pobjegoše, odoše u grad te razglasiše sve, napose o opsjednutima. 
\par 34 I gle, sav grad iziđe u susret Isusu.  Kad ga ugledaše, zamole ga da ode iz njihova kraja. 


\chapter{9}

\par 1 I ušavši u lađu, preplovi i dođe u svoj grad. 
\par 2 Kad gle, doniješe  mu uzetoga koji je ležao na nosiljci. Vidjevši njihovu vjeru, reče Isus uzetomu: "Hrabro, sinko, otpuštaju ti se grijesi!" 
\par 3 A gle, neki od pismoznanaca rekoše u sebi: "Ovaj huli!" 
\par 4 Prozrevši njihove misli, Isus reče: "Zašto snujete zlo u srcima? 
\par 5 Ta što je lakše reći: 'Otpuštaju ti se grijesi' ili reći:  'Ustani i hodi'? 
\par 6 Ali da znate: vlastan je Sin Čovječji na  zemlji otpuštati grijehe!" Tada reče uzetomu: "Ustani, uzmi nosiljku  i pođi kući!" 
\par 7 I on usta te ode kući. 
\par 8 Kad mnoštvo to vidje, zaprepasti se i poda slavu Bogu  koji takvu vlast dade ljudima. 
\par 9 Odlazeći odande, ugleda Isus čovjeka zvanog Matej gdje  sjedi u carinarnici. I kaže mu: "Pođi za mnom!" On usta i pođe  za njim. 
\par 10 Dok je Isus bio u kući za stolom, gle, mnogi carinici  i grešnici dođoše za stol s njime i njegovim učenicima. 
\par 11 Vidjevši  to, farizeji stanu govoriti: "Zašto vaš učitelj jede s carinicima  i grešnicima?" 
\par 12 A on, čuvši to, reče: "Ne treba zdravima liječnika, nego  bolesnima. 
\par 13 Hajdete i proučite što znači: Milosrđe mi je  milo, a ne žrtva. Ta ne dođoh zvati pravednike, nego grešnike." 
\par 14 Tada pristupe k njemu Ivanovi učenici govoreći: "Zašto  mi i farizeji postimo, a učenici tvoji ne poste?" 
\par 15 Nato im  Isus reče: "Mogu li svatovi tugovati dok je s njima zaručnik?  Doći će već dani kad će im se ugrabiti zaručnik, i tada će postiti!" 
\par 16 "A nitko ne stavlja krpe od sirova sukna na staro odijelo, jer zakrpa vuče s odijela pa nastane još veća rupa." 
\par 17 "I ne ulijeva se novo vino u stare mješine. Inače se  mješine proderu, vino prolije, a mješine propadnu. Nego, novo  se vino ulijeva u nove mješine pa se oboje sačuva." 
\par 18 Dok im on to govoraše, gle, pristupi neki glavar, pokloni  mu se do zemlje i reče: "Kći mi, evo, umrije, ali dođi, stavi  ruku na nju, i oživjet će." 
\par 19 Isus usta te s učenicima pođe za njim. 
\par 20 I gle, neka  žena koja bolovaše dvanaest godina od krvarenja priđe odostraga  i dotaknu se skuta njegove haljine. 
\par 21 Mislila je: "Dotaknem  li se samo njegove haljine, spasit ću se." 
\par 22 A Isus se okrenu  i vidjevši je reče: "Hrabro, kćeri, vjera te tvoja spasila."  I žena bi spašena od toga časa. 
\par 23 I uđe Isus u kuću glavarovu. Ugleda svirače i bučno mnoštvo  pa 
\par 24 reče: "Odstupite! Djevojka nije umrla, nego spava." Oni  mu se podsmjehivahu. 
\par 25 A kad je svijet bio izbačen, uđe on, primi djevojku za  ruku i ona bi uskrišena. 
\par 26 I razglasi se to po svem onom kraju. 
\par 27 Kad je Isus odlazio odande, pođu za njim dva slijepca  vičući: "Smiluj nam se, Sine Davidov!" 
\par 28 A kad uđe u kuću,  pristupe mu slijepci. Isus im kaže: "Vjerujete li da mogu to  učiniti?" Kažu mu: "Da, Gospodine!" 
\par 29 Tada se dotače njihovih  očiju govoreći: "Neka vam bude po vašoj vjeri." 
\par 30 I otvoriše  im se oči. A Isus im poprijeti: "Pazite da nitko ne dozna!" 
\par 31 Ali  oni, izišavši, razniješe glas o njemu po svem onom kraju. 
\par 32 Tek što oni iziđoše, gle, doniješe mu njemaka opsjednuta. 
\par 33 Pošto izagna đavla, progovori njemak. Mnoštvo se čudom čudilo  i govorilo: "Nikada se takvo što ne vidje u Izraelu!" 
\par 34 A farizeji  govorahu: "Po poglavici đavolskome izgoni đavle." 
\par 35 I obilazio je Isus sve gradove i sela učeći po njihovim  sinagogama, propovijedajući Evanđelje o Kraljevstvu i liječeći  svaku bolest i svaku nemoć. 
\par 36 Vidjevši mnoštvo, sažali mu se nad njim jer bijahu izmučeni  i ophrvani kao ovce bez pastira. 
\par 37 Tada reče svojim  učenicima: "Žetve je mnogo, a radnika malo. 
\par 38 Molite dakle  gospodara žetve da pošalje radnike u žetvu svoju." 


\chapter{10}

\par 1 Dozva dvanaestoricu svojih učenika i dade im vlast nad nečistim  dusima: da ih izgone i da liječe svaku bolest i svaku nemoć. 
\par 2 A ovo su imena dvanaestorice apostola: prvi Šimun, zvani  Petar, i Andrija, brat njegov; i Jakov, sin Zebedejev, i Ivan  brat njegov; 
\par 3 Filip i Bartolomej; Toma i Matej carinik; Jakov  Alfejev i Tadej; 
\par 4 Šimun Kananaj i Juda Iškariotski, koji ga  izda. 
\par 5 Tu dvanaestoricu posla Isus uputivši ih: "K poganima ne  idite i ni u koji samarijski grad ne ulazite! 
\par 6 Pođite radije  k izgubljenim ovcama doma Izraelova! 
\par 7 Putom propovijedajte:  'Približilo se kraljevstvo nebesko!' 
\par 8 Bolesne liječite, mrtve  uskrisujte, gubave čistite, zloduhe izgonite! Besplatno primiste, besplatno dajte! 
\par 9 Ne stječite zlata, ni srebra, ni mjedi sebi  u pojase, 
\par 10 ni putne torbe, ni dviju haljina, ni obuće, ni  štapa. Ta vrijedan je radnik hrane svoje." 
\par 11 "U koji god grad ili selo uđete, razvidite tko je u njemu  dostojan: ondje ostanite sve dok ne odete. 
\par 12 Ulazeći u kuću, zaželite joj mir. 
\par 13 Bude li kuća dostojna, neka mir vaš siđe  na nju. Ne bude li dostojna, neka se mir vaš k vama vrati. 
\par 14 Gdje  vas ne prime i ne poslušaju riječi vaših, iziđite iz kuće ili  grada toga i prašinu otresite sa svojih nogu. 
\par 15 Zaista, kažem  vam, lakše će biti zemlji sodomskoj i gomorskoj na Dan sudnji  negoli gradu tomu." 
\par 16 "Evo, ja vas šaljem kao ovce među vukove. Budite dakle  mudri kao zmije, a bezazleni kao golubovi! 
\par 17 Čuvajte se ljudi, jer će vas predavati vijećima i po svojim će vas sinagogama  bičevati. 
\par 18 Pred upravitelje i kraljeve vodit će vas poradi  mene, za svjedočanstvo njima i poganima. 
\par 19 Kad vas predadu, ne budite zabrinuti kako ili što ćete govoriti. Dat će vam se  u onaj čas što ćete govoriti. 
\par 20 Ta ne govorite to vi, nego  Duh Oca vašega govori u vama!" 
\par 21 "Brat će brata predavati na smrt i otac dijete. Djeca  će ustajati na roditelje i ubijati ih. 
\par 22 Svi će vas zamrziti  zbog imena moga. Ali tko ustraje do svršetka, bit će spašen." 
\par 23 "Kad vas stanu progoniti u jednom gradu, bježite u drugi.  Zaista, kažem vam, nećete obići gradova izraelskih prije nego  što dođe Sin Čovječji." 
\par 24 "Nije učenik nad učiteljem niti sluga nad gospodarom  svojim. 
\par 25 Dosta je da učenik bude kao njegov učitelj i sluga  kao njegov gospodar. Ako su domaćina Beelzebulom nazvali, koliko  će više njegove ukućane?" 
\par 26 "Ne bojte ih se dakle. Ta ništa nije skriveno što se  neće otkriti ni tajno što se neće doznati. 
\par 27 Što vam govorim  u tami, recite na svjetlu; i što na uho čujete, propovijedajte  na krovovima." 
\par 28 "Ne bojte se onih koji ubijaju tijelo, ali duše ne mogu  ubiti. Bojte se više onoga koji može i dušu i tijelo pogubiti  u paklu." 
\par 29 "Ne prodaju li se dva vrapca za novčić? Pa ipak ni jedan  od njih ne pada na zemlju bez Oca vašega. 
\par 30 A vama su i vlasi  na glavi sve izbrojene. 
\par 31 Ne bojte se dakle! Vredniji ste nego  mnogo vrabaca." 
\par 32 "Tko god se, dakle, prizna mojim pred ljudima, priznat  ću se i ja njegovim pred Ocem, koji je na nebesima. 
\par 33 A tko  se odreče mene pred ljudima, odreći ću se i ja njega pred svojim  Ocem, koji je na nebesima." 
\par 34 "Ne mislite da sam došao mir donijeti na zemlju. Ne,  nisam došao donijeti mir, nego mač. 
\par 35 Ta došao sam rastaviti  čovjeka od oca njegova i kćer od majke njezine i snahu od  svekrve njezine; 
\par 36 i neprijatelji će čovjeku biti ukućani  njegovi. 
\par 37 "Tko ljubi oca ili majku više nego mene, nije mene dostojan.  Tko ljubi sina ili kćer više nego mene, nije mene dostojan. 
\par 38 Tko  ne uzme svoga križa i ne pođe za mnom, nije mene dostojan. 
\par 39 Tko  nađe život svoj, izgubit će ga, a tko izgubi svoj život poradi  mene, naći će ga." 
\par 40 "Tko vas prima, mene prima; a tko prima mene, prima onoga  koji je mene poslao. 
\par 41 Tko prima proroka jer je prorok, primit  će plaću proročku; tko prima pravednika jer je pravednik, primit  će plaću pravedničku. 
\par 42 Tko napoji jednoga od ovih najmanjih  samo čašom hladne vode zato što je moj učenik, zaista, kažem  vam, neće mu propasti plaća." 


\chapter{11}

\par 1 Pošto Isus završi upućivati dvanaestoricu učenika, ode odande  naučavati i propovijedati po njihovim gradovima. 
\par 2 A kad Ivan u tamnici doču za djela Kristova, posla svoje  učenike 
\par 3 da ga upitaju: "Jesi li ti Onaj koji ima doći ili  drugoga da čekamo?" 
\par 4 Isus im odgovori: "Pođite i javite Ivanu  što ste čuli i vidjeli: 
\par 5 Slijepi progledaju, hromi hode, gubavi se čiste, gluhi čuju, mrtvi ustaju, siromasima se  navješćuje Evanđelje. 
\par 6 I blago onom tko se ne sablazni  o mene." 
\par 7 Kad oni odoše, poče Isus govoriti mnoštvu o Ivanu: "Što  ste izišli u pustinju gledati? Trsku koju vjetar ljulja? 
\par 8 Ili  što ste izišli vidjeti? Čovjeka u mekušasto odjevena? Eno, oni  što se mekušasto nose po kraljevskim su dvorima. 
\par 9 Ili što ste  izišli? Vidjeti proroka? Da, kažem vam, i više nego proroka. 
\par 10 On je onaj o kome je pisano: Evo, ja šaljem glasnika svoga pred licem tvojim da pripravi put pred tobom. 
\par 11 Zaista, kažem vam, između rođenih od žene ne usta veći  od Ivana Krstitelja. A ipak, i najmanji u kraljevstvu nebeskom  veći je od njega! 
\par 12 A od dana Ivana Krstitelja do sada kraljevstvo  nebesko silom se probija i siloviti ga grabe. 
\par 13 Uistinu, svi  proroci i Zakon prorokovahu do Ivana. 
\par 14 Zapravo ako hoćete:  on je Ilija koji ima doći." 
\par 15 "Tko ima uši, neka čuje." 
\par 16 "A komu da prispodobim ovaj naraštaj? Nalik je djeci  što sjede na trgovima pa jedni drugima dovikuju: 
\par 17 'Zasvirasmo vam i ne zaigraste, zakukasmo i ne zaplakaste.'" 
\par 18 "Doista, dođe Ivan. Nije jeo ni pio, a govori se: 'Ðavla  ima.' 
\par 19 Dođe Sin Čovječji koji jede i pije, a govori se: 'Gle, izjelice i vinopije, prijatelja carinika i grešnika!' Ali opravda  se Mudrost djelima svojim." 
\par 20 Tada stane prekoravati gradove u kojima se dogodilo najviše  njegovih čudesa, a oni se ne obratiše: 
\par 21 "Jao tebi, Korozaine!  Jao tebi, Betsaido! Da su se u Tiru i Sidonu zbila čudesa koja  su se dogodila u vama, odavna bi se već oni u kostrijeti i pepelu  bili obratili. 
\par 22 Ali kažem vam: Tiru i Sidonu bit će na Dan  sudnji lakše negoli vama." 
\par 23 "I ti, Kafarnaume! Zar ćeš se do neba uzvisiti? Do  u Podzemlje ćeš se strovaliti! Doista, da su se u Sodomi  zbila čudesa koja su se dogodila u tebi, ostala bi ona do danas. 
\par 24 Ali kažem vam: Zemlji će sodomskoj biti na Dan sudnji lakše  nego tebi." 
\par 25 U ono vrijeme reče Isus: "Slavim te, Oče, Gospodaru neba  i zemlje, što si ovo sakrio od mudrih i umnih, a objavio malenima. 
\par 26 Da, Oče, tako se tebi svidjelo. 
\par 27 Sve je meni predao Otac  moj i nitko ne pozna Sina doli Otac niti tko pozna Oca doli Sin  i onaj kome Sin hoće objaviti." 
\par 28 "Dođite k meni svi koji ste izmoreni i opterećeni i ja  ću vas odmoriti. 
\par 29 Uzmite jaram moj na sebe, učite se od mene  jer sam krotka i ponizna srca i naći ćete spokoj dušama svojim. 
\par 30 Uistinu, jaram je moj sladak i breme moje lako." 


\chapter{12}

\par 1 U ono vrijeme prolazio je Isus subotom kroz usjeve. Učenici  su njegovi ogladnjeli te počeli trgati klasje i jesti. 
\par 2 Vidjevši  to, farizeji mu rekoše: "Gle, učenici tvoji čine što nije dopušteno  činiti subotom." 
\par 3 On im reče: "Niste li čitali što učini David  kad ogladnje on i njegovi pratioci? 
\par 4 Kako uđe u Dom Božji te  pojedoše prinesene kruhove, a to ne bijaše slobodno jesti ni  njemu ni njegovim pratiocima, nego samo svećenicima? 
\par 5 Ili zar  niste čitali u Zakonu da subotom svećenici u Hramu krše subotu, a bez krivnje su? 
\par 6 A velim vam: veće od Hrama jest ovdje! 
\par 7 I kad biste razumjeli što ono znači: Milosrđe mi je milo, a ne žrtva, ne biste osudili ove nekrive. 
\par 8 Ta Sin Čovječji  gospodar je subote!" 
\par 9 Otišavši odande, dođe u njihovu sinagogu, 
\par 10 kad gle  čovjeka s usahlom rukom. A oni upitaše Isusa da bi ga mogli optužiti:  "Je li dopušteno subotom liječiti?" 
\par 11 On im reče: "Tko to od  vas jedinu ovcu koju ima ne bi subotom prihvatio i izvadio kad  bi mu upala u jamu? 
\par 12 A koliko je čovjek vredniji od ovce!  Tako, slobodno je subotom činiti dobro!" 
\par 13 Tada reče čovjeku: "Ispruži ruku!" On je ispruži, i -  ruka mu zdrava kao i druga! 
\par 14 A farizeji iziđoše i održaše  vijećanje protiv njega, kako da ga pogube. 
\par 15 Kad Isus to dozna, ukloni se odande. Za njim je išlo  mnoštvo. On ih sve ozdravi 
\par 16 i poprijeti im da ga ne prokazuju  - 
\par 17 da se ispuni što je rečeno po Izaiji proroku: 
\par 18 Evo Sluge mojega, koga sebi izabrah: mog ljubimca, miljenika duše moje! Stavit ću Duha svoga na njega: naviještat će pravo narodima; 
\par 19 preti se neće, neće bučiti, glas mu se neće čuti po trgovima; 
\par 20 trske stučene prelomiti neće, stijenja što tek tinja neće ugasiti - sve dok do pobjede ne izvede pravo. 
\par 21 Ime njegovo nada je narodima! 
\par 22 Tada mu donesoše opsjednuta, slijepa i nijema. I ozdravi  ga te njemak progovori i progleda. 
\par 23 I sve ono mnoštvo zapanjeno  govoraše: "Da ovo nije Sin Davidov?" 
\par 24 A farizeji čuvši to  rekoše: "Ne može ovaj izgoniti đavle osim po Beelzebulu, poglavici  đavolskom." 
\par 25 A on, znajući njihove misli, reče im: "Svako kraljevstvo  u sebi razdijeljeno opustjet će i svaki grad ili kuća u sebi  razdijeljena neće opstati. 
\par 26 Ako Sotona Sotonu izgoni, u sebi  je razdijeljen. Kako će dakle opstati kraljevstvo njegovo? 
\par 27 I  ako ja po Beelzebulu izgonim đavle, po kome ih sinovi vaši izgone?  Zato će vam oni biti suci. 
\par 28 Ali ako ja po Duhu Božjem izgonim  đavle, zbilja je došlo k vama kraljevstvo Božje." 
\par 29 "Ili kako bi tko mogao ući u kuću jakoga i oplijeniti  mu pokućstvo ako prije ne sveže jakoga? Tada će mu kuću oplijeniti." 
\par 30 "Tko nije sa mnom, protiv mene je, i tko ne sabire sa  mnom, rasipa." 
\par 31 "Zato kažem vam: svaki će se grijeh i bogohulstvo oprostiti  ljudima, ali rekne li tko bogohulstvo protiv Duha, neće se oprostiti. 
\par 32 I rekne li tko riječ protiv Sina Čovječjega, oprostit će  mu se. Ali tko rekne protiv Duha Svetoga, neće mu se oprostiti  ni na ovom svijetu ni u budućem." 
\par 33 "Ili uzmite: dobro stablo i plod mu je dobar. Ili uzmite:  trulo stablo i plod mu je truo. Ta po plodu se stablo poznaje. 
\par 34 Leglo gujinje! Kako možete govoriti dobro kad ste opaki.  Ta iz obilja srca usta govore! 
\par 35 Dobar čovjek iz riznice dobre  vadi dobro, a zao čovjek iz riznice zle vadi zlo. 
\par 36 A kažem  vam: za svaku bezrazložnu riječ koju ljudi reknu dat će račun  na Dan sudnji. 
\par 37 Doista, tvoje će te riječi opravdati i tvoje  će te riječi osuditi." 
\par 38 Jednom zapodjenuše s njime razgovor neki pismoznanci  i farizeji: "Učitelju, htjeli bismo od tebe vidjeti znak." 
\par 39 A  on im odgovori: "Naraštaj opak i preljubnički znak traži, ali  mu se znak neće dati doli znak Jone proroka. 
\par 40 Doista, kao  što Jona bijaše u utrobi kitovoj tri dana i tri noći,  tako će i Sin Čovječji biti u srcu zemlje tri dana i tri noći. 
\par 41 Ninivljani će ustati na Sudu zajedno s ovim naraštajem i  osuditi ga jer se oni na propovijed Joninu obratiše, a evo, ovdje  je i više od Jone! 
\par 42 Kraljica će Juga ustati na Sudu zajedno  s ovim naraštajem i osuditi ga jer je s krajeva zemlje došla  čuti mudrost Salomonovu, a evo, ovdje je i više od Salomona!" 
\par 43 "A kad nečisti duh iziđe iz čovjeka, luta bezvodnim mjestima  tražeći spokoja, ali ne nalazi! 
\par 44 Tada rekne: 'Vratit ću se  u kuću odakle iziđoh.' I došavši, nađe je praznu, pometenu i  uređenu. 
\par 45 Tada ode i uzme sa sobom sedam drugih duhova, gorih  od sebe, te uđu i nastane se ondje. Na kraju bude s onim čovjekom  gore nego na početku. Tako će biti i s ovim opakim naraštajem." 
\par 46 Dok on još govoraše mnoštvu, eto majke i braće njegove.  Stajahu vani tražeći da s njime govore. 
\par 47 Reče mu netko: "Evo  majke tvoje i braće tvoje, vani stoje i traže da s tobom govore." 
\par 48 Tomu koji mu to javi on odgovori: "Tko je majka moja, tko  li braća moja?" 
\par 49 I pruži ruku prema učenicima: "Evo, reče, majke moje i braće moje! 
\par 50 Doista, tko god vrši volju Oca  mojega, koji je na nebesima, taj mi je brat i sestra i majka." 


\chapter{13}

\par 1 Onoga dana Isus iziđe iz kuće i sjede uz more. 
\par 2 I nagrnu  k njemu silan svijet te je morao ući u lađu: sjede, a sve ono  mnoštvo stajaše na obali. 
\par 3 I zborio im je mnogo u prispodobama: "Gle, iziđe sijač sijati. 
\par 4 I dok je sijao, nešto zrnja  pade uz put, dođoše ptice i pozobaše ga. 
\par 5 Nešto opet pade na  kamenito tlo, gdje nemaše dosta zemlje, i odmah izniknu jer nemaše  duboke zemlje. 
\par 6 A kad sunce ogranu, izgorje i jer nemaše korijena, osuši se. 
\par 7 Nešto opet pade u trnje, trnje uzraste i uguši  ga. 
\par 8 Nešto napokon pade na dobru zemlju i davaše plod: jedno  stostruk, drugo šezdesetostruk, treće tridesetostruk." 
\par 9 "Tko ima uši, neka čuje!" 
\par 10 I pristupe učenici pa ga zapitaju: "Zašto im zboriš u  prispodobama?" 
\par 11 On im odgovori: "Zato što je vama dano znati  otajstva kraljevstva nebeskoga, a njima nije dano. 
\par 12 Doista, onomu tko ima dat će se i obilovat će, a onomu tko nema oduzet  će se i ono što ima. 
\par 13 U prispodobama im zborim zato što gledajući  ne vide i slušajući ne čuju i ne razumiju." 
\par 14 "Tako se ispunja na njima proroštvo Izaijino koje govori: Slušat ćete, slušati - i nećete razumjeti; gledat ćete, gledati - i nećete vidjeti! 
\par 15 Jer usalilo se srce naroda ovoga: uši začepiše, oči zatvoriše da očima ne vide, ušima ne čuju, srcem ne razumiju te se ne obrate pa ih izliječim. 
\par 16 A blago vašim očima što vide, i ušima što slušaju. 
\par 17 Zaista, kažem vam, mnogi su proroci i pravednici željeli vidjeti što  vi gledate, ali nisu vidjeli; i čuti što vi slušate, ali nisu  čuli." 
\par 18 "Vi, dakle, poslušajte prispodobu o sijaču. 
\par 19 Svakomu  koji sluša Riječ o Kraljevstvu, a ne razumije, dolazi Zli te  otima što mu je u srcu posijano. To je onaj uz put zasijan. 
\par 20 A  zasijani na tlo kamenito - to je onaj koji čuje Riječ i odmah  je s radošću prima, 
\par 21 ali nema u sebi korijena, nego je nestalan:  kad zbog Riječi nastane nevolja ili progonstvo, odmah se pokoleba. 
\par 22 Zasijani u trnje - to je onaj koji sluša Riječ, ali briga  vremenita i zavodljivost bogatstva uguše Riječ, te ona ostane  bez ploda. 
\par 23 Zasijani na dobru zemlju - to je onaj koji Riječ  sluša i razumije, pa onda, dakako, urodi i daje: jedan stostruko, jedan šezdesetostruko, a jedan tridesetostruko." 
\par 24 Drugu im prispodobu iznese: "Kraljevstvo je nebesko kao  kad čovjek posije dobro sjeme na svojoj njivi. 
\par 25 Dok su njegovi  ljudi spavali, dođe njegov neprijatelj, posije posred žita kukolj  i ode. 
\par 26 Kad usjev uzraste i isklasa, tada se pokaza i kukolj. 
\par 27 Sluge pristupe domaćinu pa mu reknu: 'Gospodaru, nisi li  ti dobro sjeme posijao na svojoj njivi? Odakle onda kukolj?' 
\par 28 On im odgovori: 'Neprijatelj čovjek to učini.' Nato mu sluge  kažu: 'Hoćeš li, dakle, da odemo pa da ga pokupimo?' 
\par 29 A on  reče: 'Ne! Da ne biste sabirući kukolj iščupali zajedno s njim  i pšenicu. 
\par 30 Pustite nek oboje raste do žetve. U vrijeme žetve  reći ću žeteocima: Pokupite najprije kukolj i svežite ga u snopove  da se spali, a žito skupite u moju žitnicu.'" 
\par 31 I drugu im prispodobu iznese: "Kraljevstvo je nebesko  kao kad čovjek uze gorušičino zrno i posija ga na svojoj njivi. 
\par 32 Ono je doduše najmanje od svega sjemenja, ali kad uzraste, veće je od svega povrća. Razvije se u stablo te dolaze ptice  nebeske i gnijezde mu se po granama." 
\par 33 I drugu im kaza prispodobu: "Kraljevstvo je nebesko kao  kad žena uze kvasac i zamijesi ga u tri mjere brašna dok sve  ne uskisne." 
\par 34 Sve je to Isus mnoštvu zborio u prispodobama. I ništa  im nije zborio bez prispodoba - 
\par 35 da se ispuni što je rečeno  po proroku: Otvorit ću u prispodobama usta svoja, iznijet ću što je sakriveno od postanka svijeta. 
\par 36 Tada otpusti mnoštvo i uđe u kuću. Pristupe mu učenici  govoreći: "Razjasni nam prispodobu o kukolju na njivi." 
\par 37 On  odgovori: "Sijač dobroga sjemena jest Sin Čovječji. 
\par 38 Njiva  je svijet. Dobro sjeme sinovi su Kraljevstva, a kukolj sinovi  Zloga. 
\par 39 Neprijatelj koji ga posija jest đavao. Žetva je svršetak  svijeta, a žeteoci anđeli. 
\par 40 Kao što se kukolj sabire i ognjem  sažiže, tako će biti na svršetku svijeta. 
\par 41 Sin će Čovječji  poslati svoje anđele da pokupe iz njegova kraljevstva sve zavodnike  i bezakonike 
\par 42 i bace ih u peć ognjenu, gdje će biti plač i  škrgut zubi. 
\par 43 Tada će pravednici zasjati poput sunca u kraljevstvu  Oca svojega." "Tko ima uši, neka čuje!" 
\par 44 "Kraljevstvo je nebesko kao kad je blago skriveno na  njivi: čovjek ga pronađe, sakrije, sav radostan ode, proda sve  što ima i kupi tu njivu." 
\par 45 "Nadalje, kraljevstvo je nebesko kao kad trgovac traga  za lijepim biserjem: 
\par 46 pronađe jedan dragocjeni biser, ode, rasproda sve što ima i kupi ga." 
\par 47 "Nadalje, kraljevstvo je nebesko kao kad mreža bačena  u more zahvati svakovrsne ribe. 
\par 48 Kad se napuni, izvuku je  na obalu, sjednu i skupe dobre u posude, a loše izbace. 
\par 49 Tako  će biti na svršetku svijeta. Izići će anđeli, odijeliti zle od  pravednih 
\par 50 i baciti ih u peć ognjenu, gdje će biti plač i  škrgut zubi." 
\par 51 "Jeste li sve ovo razumjeli?" Odgovore mu: "Jesmo." 
\par 52 A  on će im: "Stoga svaki pismoznanac upućen u kraljevstvo nebesko  sličan je čovjeku domaćinu koji iz svoje riznice iznosi novo  i staro." 
\par 53 Kad Isus završi sve ove prispodobe, ode odande. 
\par 54 I  dođe u svoj zavičaj. Naučavaše ih u njihovoj sinagogi te zapanjeni  govorahu: "Odakle ovomu ta mudrost i te čudesne sile? 
\par 55 Nije  li ovo drvodjeljin sin? Nije li mu majka Marija, a braća Jakov, i Josip, i Šimun, i Juda? 
\par 56 I sestre mu nisu li sve među nama?  Odakle mu sve to?" 
\par 57 I sablažnjavahu se o njega. A Isus im reče: "Nije prorok bez časti doli u svom zavičaju i  u svom domu." 
\par 58 I ne učini ondje mnogo čudesa zbog njihove  nevjere. 


\chapter{14}

\par 1 U ono vrijeme doču Herod tetrarh za Isusa 
\par 2 pa reče svojim  slugama: "To je Ivan Krstitelj! On uskrsnu od mrtvih i zato čudesne  sile djeluju u njemu." 
\par 3 Herod doista bijaše uhitio Ivana te  ga svezana bacio u tamnicu zbog Herodijade, žene brata svoga  Filipa. 
\par 4 Jer Ivan mu govoraše: "Ne smiješ je imati!" 
\par 5 Htjede  ga ubiti, ali se bojao naroda jer su ga smatrali prorokom. 
\par 6 Na  Herodov rođendan zaplesa kći Herodijadina pred svima i svidje  se Herodu. 
\par 7 Zato se zakle dati joj što god zaište. 
\par 8 A ona  nagovorena od matere: "Daj mi, reče, ovdje na pladnju glavu Ivana  Krstitelja." 
\par 9 Ražalosti se kralj, ali zbog zakletve i sustolnika  zapovjedi da se dade. 
\par 10 Posla odrubiti glavu Ivanu u tamnici. 
\par 11 I doniješe glavu njegovu na pladnju, dadoše djevojci, a ona  je odnije materi. 
\par 12 A učenici njegovi dođu, uzmu njegovo tijelo i pokopaju  ga pa odu i jave Isusu. 
\par 13 Kad je Isus to čuo, povuče se odande lađom na samotno  mjesto, u osamu. Dočuo to narod pa pohrli pješice za njim iz  gradova. 
\par 14 Kad on iziđe, vidje silan svijet, sažali mu se nad  njim te izliječi njegove bolesnike. 
\par 15 Uvečer mu pristupe učenici pa mu reknu: "Pust je ovo  kraj i već je kasno. Otpusti dakle svijet: neka odu po selima  kupiti hrane." 
\par 16 A Isus im reče: "Ne treba da idu, dajte im  vi jesti." 
\par 17 Oni mu kažu: "Nemamo ovdje ništa osim pet kruhova  i dvije ribe." 
\par 18 A on će im: "Donesite mi ih ovamo." 
\par 19 I  zapovjedi da mnoštvo posjeda po travi. On uze pet kruhova i dvije ribe, pogleda na nebo, izreče  blagoslov pa razlomi i dade kruhove učenicima, a učenici mnoštvu. 
\par 20 I jeli su svi i nasitili se. Od preteklih ulomaka nakupiše  dvanaest punih košara. 
\par 21 A blagovalo je oko pet tisuća muškaraca, osim žena i djece. 
\par 22 I odmah prisili učenike da uđu u lađu i da se prebace  prijeko dok on otpusti mnoštvo. 
\par 23 A pošto otpusti mnoštvo,  uziđe na goru, nasamo, da se pomoli. Uvečer bijaše ondje sam. 
\par 24 Lađa se već mnogo stadija bila ostisla od kraja, šibana  valovima. Bijaše protivan vjetar. 
\par 25 O četvrtoj noćnoj straži  dođe on k njima hodeći po moru. 
\par 26 A učenici ugledavši ga kako  hodi po moru, prestrašeni rekoše: "Utvara!" I od straha kriknuše. 
\par 27 Isus im odmah progovori: "Hrabro samo! Ja sam! Ne bojte se!" 
\par 28 Petar prihvati i reče: "Gospodine, ako si ti, zapovjedi mi  da dođem k tebi po vodi!" 
\par 29 A on mu reče: "Dođi!" I Petar siđe  s lađe te, hodeći po vodi, pođe k Isusu. 
\par 30 Ali kad spazi vjetar, poplaši se, počne tonuti te krikne: "Gospodine, spasi me!" 
\par 31 Isus  odmah pruži ruku, dohvati ga i kaže mu: "Malovjerni, zašto si  posumnjao?" 
\par 32 Kad uđoše u lađu, utihnu vjetar. 
\par 33 A oni na  lađi poklone mu se ničice govoreći: "Uistinu, ti si Sin Božji!" 
\par 34 Pošto preploviše, dođu na kraj, u Genezaret. 
\par 35 I ljudi  ga onoga kraja prepoznaju pa razglase po svoj onoj okolici. I  donošahu mu sve bolesnike 
\par 36 te ga moljahu da se samo dotaknu  skuta njegove haljine. I koji bi se dotakli, ozdravili bi. 


\chapter{15}

\par 1 Tada pristupe Isusu farizeji i pismoznanci iz Jeruzalema govoreći: 
\par 2 "Zašto tvoji učenici prestupaju predaju starih? Ne umivaju  ruku prije jela!" 
\par 3 On im odgovori: "A zašto vi prestupate zapovijed  Božju radi svoje predaje? 
\par 4 Ta reče Bog: Poštuj oca i majku!  I: Tko prokune oca ili majku, smrću neka se kazni! 
\par 5 A  vi velite: 'Rekne li tko ocu ili majci: Pomoć koja te od mene  ide neka bude sveti dar, 
\par 6 ne treba da poštuje oca svoga ni  majku svoju.' Tako dokinuste riječ Božju radi svoje predaje. 
\par 7 Licemjeri, dobro prorokova o vama Izaija: 
\par 8 Narod me ovaj usnama časti, a srce mu je daleko od mene. 
\par 9 Uzalud me štuju naučavajući nauke - uredbe ljudske." 
\par 10 Tada dozove mnoštvo i reče: "Slušajte i razumijte! 
\par 11 Ne  onečišćuje čovjeka što ulazi u usta, nego što iz usta izlazi  - to čovjeka onečišćuje." 
\par 12 Tada pristupe k njemu učenici i kažu mu: "Znaš li da  su se farizeji sablaznili kad su čuli tu riječ?" 
\par 13 On im odgovori:  "Svaki nasad koji ne posadi Otac moj nebeski iskorijenit će se. 
\par 14 Pustite ih! Slijepi su, vođe slijepaca! A ako slijepac slijepca  vodi, obojica će u jamu pasti." 
\par 15 Petar prihvati i reče mu:  "Protumači nam tu prispodobu!" 
\par 16 A on reče: "I vi još uvijek  ne razumijete? 
\par 17 Ne shvaćate li: sve što ulazi na usta, ide  u trbuh te se izbacuje u zahod. 
\par 18 Naprotiv, što iz usta izlazi, iz srca izvire i to onečišćuje čovjeka. 
\par 19 Ta iz srca izviru  opake namisli, ubojstva, preljubi, bludništva, krađe, lažna svjedočanstva, psovke. 
\par 20 To onečišćuje čovjeka; a jesti neopranih ruku ne  onečišćuje čovjeka." 
\par 21 Isus zatim ode odande i povuče se u krajeve tirske i  sidonske. 
\par 22 I gle: žena neka, Kanaanka iz onih krajeva, iziđe  vičući: "Smiluj mi se, Gospodine, Sine Davidov! Kći mi je teško  opsjednuta!" 
\par 23 Ali on joj ne uzvrati ni riječi. Pristupe mu  na to učenici te ga moljahu: "Udovolji joj jer viče za nama." 
\par 24 On odgovori: "Poslan sam samo k izgubljenim ovcama doma Izraelova." 
\par 25 Ali ona priđe, pokloni mu se ničice i kaže: "Gospodine, pomozi  mi!" 
\par 26 On odgovori: "Ne priliči uzeti kruh djeci i baciti ga  psićima." 
\par 27 A ona će: "Da, Gospodine! Ali psići jedu od mrvica  što padaju sa stola njihovih gospodara!" 
\par 28 Tada joj Isus reče:  "O ženo! Velika je vjera tvoja! Neka ti bude kako želiš." I ozdravi  joj kći toga časa. 
\par 29 Otišavši odande, dođe Isus do Galilejskog mora, uziđe  na goru i sjede ondje. 
\par 30 Tada nagrnu k njemu silan svijet s  hromima, kljastima, slijepima, nijemima i mnogima drugima. Polože  mu ih do nogu, a on ih izliječi. 
\par 31 Gledajući kako su nijemi  progovorili, kljasti ozdravili, hromi prohodali, slijepi progledali, divilo se mnoštvo i slavilo Boga Izraelova. 
\par 32 A Isus dozva svoje učenike pa im reče: "Žao mi je naroda  jer su već tri dana uza me, a nemaju što jesti. Otpraviti ih  gladne neću da ne klonu putem." 
\par 33 Kažu mu učenici: "Odakle  nam u pustinji toliko kruha da nahranimo toliko mnoštvo?" 
\par 34 A  Isus im reče: "Koliko kruhova imate?" Oni će: "Sedam, i malo  riba." 
\par 35 Nato zapovjedi mnoštvu da posjeda po zemlji, 
\par 36 uze  sedam kruhova i ribe, zahvali, razlomi i davaše učenicima, a  učenici mnoštvu. 
\par 37 I jeli su i nasitili se. A od preteklih ulomaka nakupiše  sedam punih košara. 
\par 38 A blagovalo je četiri tisuće muškaraca, osim žena i djece. 
\par 39 Tada otpusti mnoštvo, uđe u lađu i ode u kraj magadanski. 


\chapter{16}

\par 1 Pristupe k njemu farizeji i saduceji. Iskušavajući ga, zatraže  da im pokaže kakav znak s neba. 
\par 2 On im odgovori: "Uvečer govorite:  'Bit će vedro, nebo se žari.' 
\par 3 A ujutro: 'Danas će nevrijeme, nebo se tamno zacrvenjelo.' Lice neba znadete rasuditi, a znakove  vremena ne znate. 
\par 4 Naraštaj opak i preljubnički znak traži, ali mu se znak neće dati doli znak Jonin." Tada ih ostavi i  ode. 
\par 5 Učenici dođoše prijeko, a zaboraviše ponijeti kruha. 
\par 6 A  Isus im reče: "Pazite, čuvajte se kvasca farizejskog i saducejskog!" 
\par 7 Oni zamišljeni među sobom govorahu: "Kruha ne ponijesmo." 
\par 8 Zamijetio to Isus pa reče: "Što ste zamišljeni, malovjerni, da kruha nemate? 
\par 9 Zar još ne shvaćate, ne sjećate li se onih  pet kruhova na pet tisuća i koliko košara nakupiste? 
\par 10 I onih  sedam kruhova na četiri tisuće i koliko košara nakupiste? 
\par 11 Kako  onda ne shvaćate da vam to ne rekoh o kruhu? Nego, čuvajte se  kvasca farizejskog i saducejskog." 
\par 12 Tada razumješe kako im  ne reče da se čuvaju kvasca krušnoga, nego nauka farizejskog  i saducejskoga. 
\par 13 Kad Isus dođe u krajeve Cezareje Filipove, upita učenike:  "Što govore ljudi, tko je Sin Čovječji?" 
\par 14 Oni rekoše: "Jedni  da je Ivan Krstitelj; drugi da je Ilija; treći opet da je Jeremija  ili koji od proroka." 
\par 15 Kaže im: "A vi, što vi kažete, tko  sam ja?" 
\par 16 Šimun Petar prihvati i reče: "Ti si Krist-Pomazanik, Sin Boga živoga." 
\par 17 Nato Isus reče njemu: "Blago tebi, Šimune, sine Jonin, jer ti to ne objavi tijelo i krv, nego Otac moj, koji je na nebesima. 
\par 18 A ja tebi kažem: Ti si Petar-Stijena  i na toj stijeni sagradit ću Crkvu svoju i vrata paklena neće  je nadvladati. 
\par 19 Tebi ću dati ključeve kraljevstva nebeskoga, pa što god svežeš na zemlji, bit će svezano na nebesima; a što  god odriješiš na zemlji, bit će odriješeno na nebesima." 
\par 20 Tada  zaprijeti učenicima neka nikomu ne reknu da je on Krist. 
\par 21 Otada poče Isus upućivati učenike kako treba da pođe  u Jeruzalem, da mnogo pretrpi od starješina, glavara svećeničkih  i pismoznanaca, da bude ubijen i treći dan da uskrsne. 
\par 22 Petar ga uze na stranu i poče odvraćati: "Bože sačuvaj, Gospodine! Ne, to se tebi ne smije dogoditi!" 
\par 23 Isus se okrene  i reče Petru: "Nosi se od mene, sotono! Sablazan si mi jer ti  nije na pameti što je Božje, nego što je ljudsko!" 
\par 24 Tada Isus reče svojim učenicima: "Hoće li tko za mnom, neka se odrekne samoga sebe, neka uzme svoj križ i neka ide  za mnom. 
\par 25 Tko hoće život svoj spasiti, izgubit će ga, a tko  izgubi život svoj poradi mena, naći će ga. 
\par 26 Ta što će koristiti  čovjeku ako sav svijet stekne, a životu svojemu naudi? Ili što  će čovjek dati u zamjenu za život svoj? 
\par 27 Doći će, doista,  Sin Čovječji u slavi Oca svoga s anđelima svojim i tada će  naplatiti svakomu po djelima njegovim." 
\par 28 "Zaista, kažem vam, neki od ovdje nazočnih neće okusiti  smrti dok ne vide Sina Čovječjega gdje dolazi sa svojim kraljevstvom." 


\chapter{17}

\par 1 Nakon šest dana uze Isus sa sobom Petra, Jakova i Ivana, brata  njegova, te ih povede na goru visoku, u osamu, 
\par 2 i preobrazi  se pred njima. I zasja mu lice kao sunce, a haljine mu postadoše  bijele kao svjetlost. 
\par 3 I gle: ukazaše im se Mojsije i Ilija  te razgovarahu s njime. 
\par 4 A Petar prihvati i reče Isusu: "Gospodine, dobro nam je ovdje biti. Ako hoćeš, načinit ću ovdje tri sjenice, tebi jednu, Mojsiju jednu i Iliji jednu." 
\par 5 Dok je on još govorio, gle, svijetao ih oblak zasjeni, a glas iz oblaka govoraše: "Ovo je Sin moj, Ljubljeni! U  njemu mi sva milina! Slušajte ga!" 
\par 6 Čuvši glas, učenici  padoše licem na zemlju i silno se prestrašiše. 
\par 7 Pristupi k  njima Isus, dotakne ih i reče: "Ustanite, ne bojte se!" 
\par 8 Podigoše  oči, ali ne vidješe nikoga doli Isusa sama. 
\par 9 Dok su silazili s gore, zapovjedi im Isus: "Nikomu ne  kazujte viđenje dok Sin Čovječji od mrtvih ne uskrsne." 
\par 10 Upitaše ga učenici: "Što dakle pismoznanci govore da  prije treba da dođe Ilija?" 
\par 11 On im odgovori: "Ilija  će doduše doći i sve obnoviti. 
\par 12 No velim vam: Ilija  je već došao, ali ga ne upoznaše, već učiniše s njim što im se  prohtjelo. Tako je i Sinu Čovječjemu trpjeti od njih." 
\par 13 Tada  razumješe učenici da im to reče o Ivanu Krstitelju. 
\par 14 Kada dođoše k mnoštvu, pristupi mu čovjek, padne pred  njim na koljena 
\par 15 i reče: "Gospodine, smiluj se sinu mojemu  jer je mjesečar i zlo mu je. Često doista pada u oganj i često  u vodu. 
\par 16 Dovedoh ga tvojim učenicima i ne mogoše ga izliječiti." 
\par 17 A Isus odgovori: "O rode nevjerni i opaki! Dokle mi je biti  s vama! Dokle li vas podnositi! Dovedite mi ga ovamo!" 
\par 18 I  zaprijeti Isus zloduhu te on iziđe iz njega. I ozdravi dječak  toga časa. 
\par 19 Tada učenici pristupiše nasamo k Isusu i rekoše: "Zašto  ga mi ne mogosmo izagnati?" 
\par 20 Kaže im: "Zbog vaše malovjernosti.  Zaista, kažem vam, ako imadnete vjere koliko je zrno gorušičino  te reknete ovoj gori: 'Premjesti se odavde onamo!', premjestit  će se i ništa vam neće biti nemoguće." 
\par 21 # 
\par 22 A kad su se skupili u Galileji, reče im Isus: "Sin Čovječji  ima biti predan ljudima u ruke 
\par 23 i ubit će ga, ali on će treći  dan uskrsnuti." I ožalostiše se silno. 
\par 24 Kad stigoše u Kafarnaum, pristupe Petru oni što ubiru  dvodrahme pa mu rekoše: "Učitelj vaš ne plaća dvodrahme?" 
\par 25 "Plaća", odgovori. A kad on uđe u kuću, pretekne ga Isus: "Što ti se  čini, Šimune? Kraljevi zemaljski od koga ubiru carinu ili porez?  Od svojih sinova ili od tuđih?" 
\par 26 Kad on odgovori: "Od tuđih!", reče mu Isus: "Sinovi su, dakle, oslobođeni. 
\par 27 Ali da ih ne  sablaznimo, pođi k moru, baci udicu i prvu ribu koja naiđe uzmi, otvori joj usta i naći ćeš stater. Uzmi ga pa im ga podaj za  me i za se." 


\chapter{18}

\par 1 U onaj čas pristupe učenici Isusu pa ga zapitaju: "Tko je, dakle, najveći u kraljevstvu nebeskom?" 
\par 2 On dozove dijete, postavi ga posred njih 
\par 3 i reče: "Zaista, kažem vam, ako se  ne obratite i ne postanete kao djeca, nećete ući u kraljevstvo  nebesko. 
\par 4 Tko god se dakle ponizi kao ovo dijete, taj je najveći  u kraljevstvu nebeskom. 
\par 5 I tko primi jedno ovakvo dijete u  moje ime, mene prima." 
\par 6 "Onomu, naprotiv, tko bi sablaznio jednoga od ovih najmanjih  što vjeruju u mene bilo bi bolje da mu se o vrat objesi mlinski  kamen pa da potone u dubinu morsku." 
\par 7 "Jao svijetu od sablazni! Neizbježivo dolaze sablazni, ali jao čovjeku po kom dolazi sablazan. 
\par 8 Pa ako te ruka ili  noga sablažnjava, odsijeci je i baci od sebe. Bolje ti je ući  u život kljastu ili hromu, nego s obje ruke ili s obje noge biti  bačen u oganj vječni. 
\par 9 I ako te oko sablažnjava, izvadi ga  i baci od sebe. Bolje ti je jednooku u život ući, nego s oba  oka biti bačen u pakao ognjeni." 
\par 10 "Pazite da ne prezrete ni jednoga od ovih najmanjih jer, kažem vam, anđeli njihovi na nebu uvijek gledaju lice Oca mojega, koji je na nebesima." 
\par 11 # 
\par 12 "Što vam se čini? Ako neki čovjek imadne sto ovaca i  jedna od njih zaluta, neće li on ostaviti onih devedeset i devet  u gorama i poći u potragu za zalutalom? 
\par 13 Posreći li mu se  te je nađe, zaista, kažem vam, raduje se zbog nje više nego zbog  onih devedeset i devet koje nisu zalutale. 
\par 14 Tako ni Otac vaš, koji je na nebesima, neće da propadne ni jedan od ovih malenih." 
\par 15 "Pogriješi li tvoj brat, idi i pokaraj ga nasamo. 
\par 16 Ako  te posluša, stekao si brata. Ne posluša li te, uzmi sa sobom  još jednoga ili dvojicu, neka na iskazu dvojice ili trojice  svjedoka počiva svaka tvrdnja. 
\par 17 Ako ni njih ne posluša, reci Crkvi. Ako pak ni Crkve ne posluša, neka ti bude kao poganin  i carinik." 
\par 18 "Zaista, kažem vam, što god svežete na zemlji, bit će  svezano na nebu; i što god odriješite na zemlji, bit će odriješeno  na nebu." 
\par 19 "Nadalje, kažem vam, ako dvojica od vas na zemlji jednodušno  zaištu što mu drago, dat će im Otac moj, koji je na nebesima. 
\par 20 Jer gdje su dvojica ili trojica sabrana u moje ime, tu sam  i ja među njima." 
\par 21 Tada pristupi k njemu Petar i reče: "Gospodine, koliko  puta da oprostim bratu svomu ako se ogriješi o mene? Do sedam  puta?" 
\par 22 Kaže mu Isus: "Ne kažem ti do sedam puta, nego do  sedamdeset puta sedam." 
\par 23 "Stoga je kraljevstvo nebesko kao kad kralj odluči urediti  račune sa slugama. 
\par 24 Kad započe obračunavati, dovedoše mu jednoga  koji mu dugovaše deset tisuća talenata. 
\par 25 Kako nije imao odakle  vratiti, zapovjedi gospodar da se proda on, žena mu i djeca i  sve što ima te se podmiri dug. 
\par 26 Nato sluga padne ničice preda  nj govoreći: 'Strpljenja imaj sa mnom, i sve ću ti vratiti.' 
\par 27 Gospodar se smilova tomu sluzi, otpusti ga i dug mu oprosti." 
\par 28 "A kad taj isti sluga izađe, naiđe na jednoga svoga druga  koji mu dugovaše sto denara. Uhvati ga i stane ga daviti govoreći:  'Vrati što si dužan!' 
\par 29 Drug padne preda nj i stane ga zaklinjati:  'Strpljenja imaj sa mnom i vratit ću ti.' 
\par 30 Ali on ne htjede, nego ode i baci ga u tamnicu dok mu ne vrati duga." 
\par 31 "Kad njegovi drugovi vidješe što se dogodilo, silno ražalošćeni  odoše i sve to dojaviše gospodaru. 
\par 32 Tada ga gospodar dozva  i reče mu: 'Slugo opaki, sav sam ti onaj dug oprostio jer si  me zamolio. 
\par 33 Nije li trebalo da se i ti smiluješ svome drugu, kao što sam se i ja tebi smilovao?' 
\par 34 I gospodar ga, rasrđen, preda mučiteljima dok mu ne vrati svega duga. 
\par 35 Tako će i  Otac moj nebeski učiniti s vama ako svatko od srca ne oprosti  svomu bratu." 


\chapter{19}

\par 1 Kad Isus završi ove besjede, ode iz Galileje i dođe u judejski  kraj s onu stranu Jordana. 
\par 2 Za njim je išao silan svijet. Ondje  ih izliječi. 
\par 3 Pristupe mu tada farizeji pa, da ga iskušaju, kažu: "Je  li dopušteno otpustiti ženu s kojega god razloga?" 
\par 4 On odgovori:  "Zar niste čitali: Stvoritelj od početka muško i žensko stvori  ih 
\par 5 i reče: Stoga će čovjek ostaviti oca i majku da  prione uza svoju ženu; i dvoje njih bit će jedno tijelo? 
\par 6 Tako više nisu dvoje, nego jedno tijelo. Što, dakle, Bog združi, čovjek neka ne rastavlja." 
\par 7 Kažu mu: "Zašto onda Mojsije zapovjedi dati otpusno  pismo i - otpustiti?" 
\par 8 Odgovori im: "Zbog tvrdoće srca  vašega dopusti vam Mojsije otpustiti žene, ali od početka ne  bijaše tako. 
\par 9 A ja vam kažem: Tko otpusti svoju ženu - osim  zbog bludništva - pa se oženi drugom, čini preljub." 
\par 10 Kažu mu učenici: "Ako je tako između muža i žene, bolje  je ne ženiti se." 
\par 11 A on im reče: "Ne shvaćaju toga svi, nego  samo oni kojima je dano. 
\par 12 Doista, ima za ženidbu nesposobnih  koji se takvi iz utrobe materine rodiše. Ima nesposobnih koje  ljudi onesposobiše. A ima nesposobnih koji sami sebe onesposobiše  poradi kraljevstva nebeskoga. Tko može shvatiti, neka shvati." 
\par 13 Tada mu doniješe dječicu da na njih stavi ruke i pomoli  se. A učenici im branili. 
\par 14 Nato će im Isus: "Pustite dječicu  i ne priječite im k meni jer takvih je kraljevstvo nebesko!" 
\par 15 I položi ruke na njih pa krene odande. 
\par 16 I gle, pristupi mu netko i reče: "Učitelju, koje mi je  dobro činiti da imam život vječni?" 
\par 17 A on mu reče: "Što me  pitaš o dobrome? Jedan je samo dobar! Ali ako hoćeš u život ući, čuvaj zapovijedi." 
\par 18 Upita ga: "Koje?" A Isus reče: Ne ubij! Ne čini preljuba! Ne ukradi! Ne svjedoči lažno! 
\par 19 Poštuj oca i majku! I ljubi svoga bližnjega kao sebe samoga!" 
\par 20 Kaže mu mladić: "Sve sam to čuvao. Što mi još nedostaje?" 
\par 21 Reče mu Isus: "Hoćeš li biti savršen, idi, prodaj što imaš  i podaj siromasima pa ćeš imati blago na nebu. A onda dođi i  idi za mnom." 
\par 22 Na tu riječ ode mladić žalostan jer imaše velik  imetak. 
\par 23 A Isus reče svojim učenicima: "Zaista, kažem vas, teško  će bogataš u kraljevstvo nebesko. 
\par 24 Ponovno vam velim: Lakše  je devi kroz uši iglene nego bogatašu u kraljevstvo Božje." 
\par 25 Čuvši to, učenici se silno snebivahu govoreći: "Tko se  onda može spasiti?" 
\par 26 A Isus upre u njih pogled pa im reče:  "Ljudima je to nemoguće, ali Bogu je sve moguće." 
\par 27 Tada Petar prihvati pa upita: "Evo, mi sve ostavismo  i pođosmo za tobom. Što ćemo za to dobiti?" 
\par 28 Reče im Isus:  "Zaista, kažem vam, vi koji pođoste za mnom, o preporodu, kad  Sin Čovječji sjedne na prijestolje svoje slave, i vi ćete sjediti  na dvanaest prijestolja i suditi dvanaest plemena Izraelovih. 
\par 29 I tko god ostavi kuće, ili braću, ili sestre, ili oca, ili  majku, ili ženu, ili djecu, ili polja poradi imena mojega, stostruko  će primiti i život vječni baštiniti." 
\par 30 "A mnogi prvi bit će posljednji, i posljednji prvi." 


\chapter{20}

\par 1 "Kraljevstvo je nebesko kao kad domaćin rano ujutro izađe  najmiti radnike u svoj vinograd. 
\par 2 Pogodi se s radnicima po  denar na dan i pošalje ih u svoj vinograd. 
\par 3 Izađe i o trećoj  uri i vidje druge gdje stoje na trgu besposleni 
\par 4 pa i njima  reče: 'Idite i vi u moj vinograd pa što bude pravo, dat ću vam.' 
\par 5 I oni odoše. Izađe opet o šestoj i devetoj uri te učini isto  tako. 
\par 6 A kad izađe o jedanaestoj uri, nađe druge gdje stoje  i reče im: 'Zašto ovdje stojite vazdan besposleni?' 
\par 7 Kažu mu:  'Jer nas nitko ne najmi.' Reče im: 'Idite i vi u vinograd.'" 
\par 8 "Uvečer kaže gospodar vinograda svojemu upravitelju: 'Pozovi  radnike i podaj im plaću počevši od posljednjih pa sve do prvih.' 
\par 9 Dođu tako oni od jedanaeste ure i prime po denar. 
\par 10 Pa kada  dođu oni prvi, pomisle da će primiti više, ali i oni prime po  denar. 
\par 11 A kad primiše, počeše mrmljati protiv domaćina: 
\par 12 'Ovi  posljednji jednu su uru radili i izjednačio si ih s nama, koji  smo podnosili svu tegobu dana i žegu.'" 
\par 13 "Nato on odgovori jednomu od njih: 'Prijatelju, ne činim  ti krivo. Nisi li se pogodio sa mnom po denar? 
\par 14 Uzmi svoje  pa idi. A ja hoću i ovomu posljednjemu dati kao i tebi. 
\par 15 Nije  li mi slobodno činiti sa svojim što hoću? Ili zar je oko tvoje  zlo što sam ja dobar?'" 
\par 16 "Tako će posljednji biti prvi, a prvi posljednji." 
\par 17 Dok je Isus uzlazio u Jeruzalem, uze dvanaestoricu nasamo  te im putem reče: 
\par 18 "Evo, uzlazimo u Jeruzalem i Sin Čovječji  bit će predan glavarima svećeničkim i pismoznancima. Osudit će  ga na smrt 
\par 19 i predati poganima da ga izrugaju, izbičuju i  razapnu, ali on će treći dan uskrsnuti." 
\par 20 Tada mu pristupi mati sinova Zebedejevih zajedno sa sinovima, pade ničice da od njega nešto zaište. 
\par 21 A on će joj: "Što  želiš?" Kaže mu: "Reci da ova moja dva sina u tvome kraljevstvu  sjednu uza te, jedan tebi zdesna, drugi slijeva." 
\par 22 Isus odgovori: "Ne znate što ištete. Možete li piti čašu  koju ću ja piti?" Kažu mu: "Možemo!" 
\par 23 A on im reče: "Čašu  ćete moju doduše piti, ali sjesti meni zdesna ili slijeva - to  nisam ja vlastan dati, to je onih kojima je pripravio moj Otac." 
\par 24 Kada su to čula ostala desetorica, razgnjeve se na dva  brata. 
\par 25 Zato ih Isus dozva i reče: "Znate da vladari gospoduju  svojim narodima i velikaši njihovi drže ih pod vlašću. 
\par 26 Neće  tako biti među vama! Naprotiv, tko hoće da među vama bude najveći, neka vam bude poslužitelj. 
\par 27 I tko god hoće da među vama bude  prvi, neka vam bude sluga." 
\par 28 "Tako i Sin Čovječji nije došao da bude služen, nego  da služi i život svoj dade kao otkupninu za mnoge." 
\par 29 Kad su izlazili iz Jerihona, pođe za njim silan svijet. 
\par 30 I gle, dva slijepca sjeđahu kraj puta. Čuvši da Isus prolazi, povikaše: "Gospodine, smiluj nam se, Sine Davidov!" 
\par 31 Mnoštvo  ih ušutkivalo, ali oni još jače viknuše: "Gospodine, smiluj nam  se, Sine Davidov!" 
\par 32 Isus se zaustavi, dozove ih i reče: "Što hoćete da vam  učinim?" 
\par 33 Kažu mu: "Gospodine, da nam se otvore oči." 
\par 34 Isus  se ganut dotače njihovim očiju i oni odmah progledaše. I pođoše  za njim. 


\chapter{21}

\par 1 Kad se približiše Jeruzalemu te dođoše u Betfagu, na Maslinskoj  gori, posla Isus dvojicu učenika 
\par 2 govoreći: "Pođite u selo  pred vama i odmah ćete naći privezanu magaricu i uz nju magare.  Odriješite ih i dovedite k meni. 
\par 3 Ako vam tko što rekne, recite:  'Gospodinu trebaju', i odmah će ih pustiti." 
\par 4 To se dogodi da se ispuni što je rečeno po proroku: 
\par 5 Recite kćeri Sionskoj: Evo kralj ti tvoj dolazi, krotak, jašuć na magarcu, na magaretu, mladetu magaričinu. 
\par 6 Učenici odu i učine kako im naredi Isus. 
\par 7 Dovedu magaricu  i magare te stave na njih haljine i Isus uzjaha na njih. 
\par 8 Silan  svijet prostrije svoje haljine po putu, a drugi rezahu grane  sa stabala i sterahu ih po putu. 
\par 9 Mnoštvo pak pred njim i za  njim klicaše: "Hosana Sinu Davidovu! Blagoslovljen  Onaj koji dolazi u ime Gospodnje! Hosana u visinama!" 
\par 10 Kad uđe u Jeruzalem, uskomešao se sav grad i govorio:  "Tko je ovaj?" 
\par 11 A mnoštvo odgovaraše: "To je Prorok, Isus  iz Nazareta galilejskoga." 
\par 12 Isus uđe u Hram i izagna sve koji su prodavali i kupovali  u Hramu. Mjenjačima isprevrta stolove i prodavačima golubova  klupe. 
\par 13 Kaže im: "Pisamo je: Dom će se moj zvati Dom molitve,  a vi od njega činite pećinu razbojničku." 
\par 14 U Hramu mu priđoše slijepi i hromi i on ih ozdravi. 
\par 15 A kad glavari svećenički i pismoznanci vidješe čudesa  koja učini i djecu što viču Hramom: "Hosana Sinu Davidovu!",  gnjevni 
\par 16 mu rekoše: "Čuješ li što ovi govore?" Kaže im Isus:  "Da! A niste li čitali: Iz usta djece i dojenčadi sebi si pripravio hvalu?" 
\par 17 On ih ostavi, pođe iz grada u Betaniju te ondje prenoći. 
\par 18 Ujutro se vraćao u grad i ogladnje. 
\par 19 Ugleda smokvu  kraj puta i priđe k njoj, ali ne nađe na njoj ništa osim lišća  pa joj kaže: "Ne bilo više ploda s tebe dovijeka!" I smokva umah  usahnu. 
\par 20 Vidjevši to, učenici se začude: "Kako umah smokva usahnu!" 
\par 21 Isus im odvrati: "Zaista, kažem vam, ako budete imali vjeru  i ne posumnjate, činit ćete ne samo ovo sa smokvom, nego - reknete  li i ovoj gori: 'Digni se i baci u more!', bit će tako. 
\par 22 I  sve što zaištete u molitvi vjerujući, primit ćete." 
\par 23 I uđe u Hram. Dok je naučavao, pristupiše mu glavari  svećenički i starješine narodne te ga upitaše: "Kojom vlašću  to činiš? Tko ti dade tu vlast?" 
\par 24 Isus im odgovori: "I ja ću vas jedno upitati. Ako mi  na to odgovorite, ja ću vama kazati kojom vlašću ovo činim. 
\par 25 Krst  Ivanov odakle li bijaše? Od Neba ili od ljudi?" A oni umovahu  među sobom: "Reknemo li 'Od Neba', odvratit će nam: 'Zašto mu, dakle, ne povjerovaste?' 
\par 26 A reknemo li 'Od ljudi', strah  nas je mnoštva. Ta svi Ivana smatraju prorokom." 
\par 27 Zato odgovore  Isusu: "Ne znamo." I on njima reče: "Ni ja vama neću kazati kojom  vlašću ovo činim." 
\par 28 "A što vam se čini? Čovjek neki imao dva sina. Priđe  prvomu i reče: 'Sinko, hajde danas na posao u vinograd!' 
\par 29 On  odgovori: 'Neću!' No poslije se predomisli i ode. 
\par 30 Priđe i  drugomu pa mu reče isto tako. A on odgovori: 'Evo me, gospodaru!'  i ne ode. 
\par 31 Koji od te dvojice izvrši volju očevu?" Kažu: "Onaj  prvi." Nato će im Isus: "Zaista, kažem vam, carinici i bludnice  pretekoše vas u kraljevstvo Božje! 
\par 32 Doista, Ivan dođe k vama  putom pravednosti i vi mu ne povjerovaste, a carinici mu i bludnice  povjerovaše. Vi pak, makar to vidjeste, ni kasnije se ne predomisliste  da mu povjerujete." 
\par 33 "Drugu prispodobu čujte! Bijaše neki domaćin koji posadi  vinograd, ogradi ga ogradom, iskopa u njemu tijesak i podiže  kulu pa ga iznajmi vinogradarima i otputova. 
\par 34 Kad se približilo  vrijeme plodova, posla svoje sluge vinogradarima da uzmu njegov  urod. 
\par 35 A vinogradari pograbe njegove sluge pa jednoga istukoše, drugog ubiše, a trećega kamenovaše. 
\par 36 I opet posla druge sluge, više njih nego prije, ali oni i s njima postupiše jednako." 
\par 37 "Naposljetku posla k njima sina svoga misleći: 'Poštovat  će mog sina.' 
\par 38 Ali kad vinogradari ugledaju sina, rekoše među  sobom: 'Ovo je baštinik! Hajde da ga ubijemo i imat ćemo baštinu  njegovu!' 
\par 39 I pograbe ga, izbace iz vinograda i ubiju." 
\par 40 "Kada dakle dođe gospodar vinograda, što će učiniti s  tim vinogradarima?" 
\par 41 Kažu mu: "Opake će nemilo pogubiti, a  vinograd iznajmiti drugim vinogradarima što će mu davati urod  u svoje vrijeme." 
\par 42 Kaže im Isus: "Zar nikada niste čitali u Pismima: Kamen što ga odbaciše graditelji postade kamen zaglavni. Gospodnje je to djelo - kakvo čudo u očima našim! 
\par 43 Zato će se - kažem vam - oduzeti od vas kraljevstvo Božje  i dat će se narodu koji donosi njegove plodove! ( 
\par 44 I tko padne  na taj kamen, smrskat će se, a na koga on padne, satrt će ga.)" 
\par 45 Kad su glavari svećenički i farizeji čuli te njegove  prispodobe, razumjeli su da govori o njima. 
\par 46 I tražili su  da ga uhvate, ali se pobojaše mnoštva jer ga je smatralo prorokom. 


\chapter{22}

\par 1 Isus im ponovno prozbori u prispodobama: 
\par 2 "Kraljevstvo je  nebesko kao kad neki kralj pripravi svadbu sinu svomu. 
\par 3 Posla  sluge da pozovu uzvanike na svadbu. No oni ne htjedoše doći. 
\par 4 Opet posla druge sluge govoreći: 'Recite uzvanicima: Evo,  objed sam ugotovio. Junci su moji i tovljenici poklani i sve  pripravljeno. Dođite na svadbu!'" 
\par 5 "Ali oni ne mareći odoše - jedan na svoju njivu, drugi  za svojom trgovinom. 
\par 6 Ostali uhvate njegove sluge, zlostave  ih i ubiju. 
\par 7 Nato se kralj razgnjevi, posla svoju vojsku i  pogubi one ubojice, a grad im spali." 
\par 8 "Tada kaže slugama: 'Svadba je, evo, pripravljena ali  uzvanici ne bijahu dostojni. 
\par 9 Pođite stoga na raskršća i koga  god nađete, pozovite na svadbu!'" 
\par 10 "Sluge iziđoše na putove i sabraše sve koje nađoše -  i zle i dobre. I svadbena se dvorana napuni gostiju. 
\par 11 Kad  kralj uđe pogledati goste, spazi ondje čovjeka koji ne bijaše  odjeven u svadbeno ruho. 
\par 12 Kaže mu: 'Prijatelju, kako si ovamo  ušao bez svadbenoga ruha?' A on zanijemi. 
\par 13 Tada kralj reče  poslužiteljima: 'Svežite mu ruke i noge i bacite ga van u tamu, gdje će biti plač i škrgut zubi.' 
\par 14 Doista, mnogo je zvanih, malo izabranih." 
\par 15 Tada farizeji odoše i održaše vijeće kako da Isusa uhvate  u riječi. 
\par 16 Pošalju k njemu svoje učenike s herodovcima da  ga upitaju: "Učitelju! Znamo da si istinit te po istini putu  Božjem učiš i ne mariš tko je tko jer nisi pristran. 
\par 17 Reci  nam, dakle, što ti se čini: je li dopušteno dati porez caru ili  nije?" 
\par 18 Znajući njihovu opakost, reče Isus: "Zašto me iskušavate, licemjeri? 
\par 19 Pokažite mi porezni novac!" Pružiše mu denar. 
\par 20 On ih upita: "Čija je ovo slika i natpis?" 
\par 21 Odgovore:  "Carev." Kaže im: "Podajte dakle caru carevo, a Bogu Božje." 
\par 22 Čuvši to, zadive se pa ga ostave i odu. 
\par 23 Toga dana pristupiše k njemu saduceji, koji vele da nema  uskrsnuća, i upitaše ga: 
\par 24 "Učitelju, Mojsije reče: Umre  li tko bez djece, neka se njegov brat oženi njegovom ženom te  podigne porod bratu svomu. 
\par 25 Bijaše tako u nas sedmero  braće. Prvi se oženi i umrije bez poroda ostavivši ženu svom  bratu. 
\par 26 Tako i drugi i treći, sve do sedmoga. 
\par 27 A nakon  svih umrije i žena. 
\par 28 Kojemu će dakle od te sedmorice biti  žena o uskrsnuću? Jer sva su je sedmorica imala." 
\par 29 Odgovori im Isus: "U zabludi ste jer ne razumijete Pisama  ni sile Božje. 
\par 30 Ta u uskrsnuću niti se žene niti udavaju,  nego su kao anđeli na nebu. 
\par 31 A što se tiče uskrsnuća mrtvih, zar niste čitali što vam reče Bog: 
\par 32 Ja sam Bog Abrahamov, Bog Izakov i Bog Jakovljev? Nije on Bog mrtvih, nego živih!" 
\par 33 Čuvši to, mnoštvo osta zaneseno njegovim naukom. 
\par 34 A kad su farizeji čuli kako ušutka saduceje, okupiše  se, 
\par 35 a jedan od njih, zakonoznanac, da ga iskuša, upita: 
\par 36 "Učitelju, koja ja zapovijed najveća u Zakonu?" 
\par 37 A on mu reče: "Ljubi  Gospodina Boga svojega svim srcem svojim, i svom dušom svojom, i svim umom svojim. 
\par 38 To je najveća i prva zapovijed. 
\par 39 Druga, ovoj slična: Ljubi svoga bližnjega kao sebe samoga. 
\par 40 O tim dvjema zapovijedima visi sav Zakon i Proroci." 
\par 41 Kad se farizeji skupiše, upita ih Isus: 
\par 42 "Što mislite  o Kristu? Čiji je on sin?" Kažu mu: "Davidov." 
\par 43 A on će njima:  "Kako ga onda David u Duhu naziva Gospodinom, kad veli: 
\par 44 Reče Gospod Gospodinu mojemu: 'Sjedi mi zdesna dok ne položim neprijatelje tvoje za podnožje nogama tvojim?' 
\par 45 Ako ga dakle David naziva Gospodinom, kako mu je sin?" 
\par 46 I nitko mu nije mogao odgovoriti ni riječi, niti se od toga  dana tko usudio upitati ga bilo što. 


\chapter{23}

\par 1 Tada Isus prozbori mnoštvu i svojim učenicima: 
\par 2 "Na Mojsijevu  stolicu zasjedoše pismoznanci i farizeji. 
\par 3 Činite dakle i obdržavajte  sve što vam kažu, ali se nemojte ravnati po njihovim djelima  jer govore, a ne čine. 
\par 4 Vežu i ljudima na pleća tovare teška  bremena, a sami ni da bi ih prstom makli. 
\par 5 Sva svoja djela  čine zato da ih ljudi vide. Doista, proširuju zapise svoje i  produljuju rese. 
\par 6 Vole pročelja na gozbama, prva sjedala u  sinagogama, 
\par 7 pozdrave na trgovima i da ih ljudi zovu 'Rabbi'. 
\par 8 Vi pak ne dajte se zvati 'Rabbi', jer jedan je učitelj vaš, a svi ste vi braća. 
\par 9 Ni ocem ne zovite nikoga na zemlji jer  jedan je Otac vaš - onaj na nebesima. 
\par 10 I ne dajte da vas vođama  zovu, jer jedan je vaš vođa - Krist. 
\par 11 Najveći među vama neka  vam bude poslužitelj. 
\par 12 Tko se god uzvisuje, bit će ponižen, a tko se ponizuje, bit će uzvišen." 
\par 13 "Jao vama, pismoznanci i farizeji! Licemjeri! Zaključavate  kraljevstvo nebesko pred ljudima; sami ne ulazite, a ne date  ući ni onima koji bi htjeli." 
\par 14 # 
\par 15 "Jao vama, pismoznanci i farizeji! Licemjeri! Obilazite  morem i kopnom da pridobijete jednog sljedbenika. A kad ga pridobijete, promećete ga u sina paklenoga dvaput goreg od sebe." 
\par 16 "Jao vama! Slijepe vođe! Govorite: 'Zakune li se tko  Hramom, nije ništa. Ali ako se zakune hramskim zlatom, veže ga  zakletva.' 
\par 17 Budale i slijepci! Ta što je veće: zlato ili Hram  što posvećuje zlato? 
\par 18 Nadalje: 'Zakune li se tko žrtvenikom, nije ništa. Ali ako se zakune darom što je na njemu, veže ga  zakletva.' 
\par 19 Slijepci! Ta što je veće: dar ili žrtvenik što  dar posvećuje? 
\par 20 Tko se dakle zakune žrtvenikom, kune se njime  i svime što je na njemu. 
\par 21 I tko se zakune Hramom, kune se  njime i Onim koji u njemu prebiva. 
\par 22 I tko se zakune nebom, kune se prijestoljem Božjim i Onim koji na njemu sjedi." 
\par 23 "Jao vama, pismoznanci i farizeji! Licemjeri! Namirujete  desetinu od metvice i kopra i kima, a propuštate najvažnije u  Zakonu: pravednost, milosrđe, vjernost. Ovo je trebalo činiti, a ono ne propuštati. 
\par 24 Slijepe vođe! Cijedite komarca, a gutate  devu!" 
\par 25 "Jao vama, pismoznanci i farizeji! Licemjeri! Čistite  čašu i zdjelu izvana, a iznutra su pune grabeža i pohlepe. 
\par 26 Farizeju  slijepi! Očisti najprije nutrinu čaše da joj i vanjština bude  čista." 
\par 27 Jao vama pismoznanci i farizeji! Licemjeri! Nalik ste  na obijeljene grobove. Izvana izgledaju lijepi, a iznutra su  puni mrtvačkih kostiju i svakojake nečistoće. 
\par 28 Tako i vi izvana  ljudima izgledate pravedni, a iznutra ste puni licemjerja i bezakonja." 
\par 29 "Jao vama, pismoznanci i farizeji! Licemjeri! Gradite  grobnice prorocima i kitite spomenike pravednicima 
\par 30 te govorite:  'Da smo mi živjeli u dane otaca svojih, ne bismo bili njihovi  sudionici u prolijevanju krvi proročke.' 
\par 31 Tako sami protiv  sebe svjedočite da ste sinovi ubojica proroka. 
\par 32 Dopunite samo  mjeru otaca svojih!" 
\par 33 "Zmije! Leglo gujinje! Kako ćete uteći osudi paklenoj? 
\par 34 Zato evo ja šaljem vama proroke i mudrace i pismoznance.  Jedne ćete od njih ubiti i raspeti, druge bičevati po svojim  sinagogama i progoniti od grada do grada 
\par 35 da tako na vas dođe  sva pravedna krv, prolivena na zemlji od krvi Abela pravednoga  pa do krvi Zaharije, sina Barahijina, kojega ubiste između Hrama  i žrtvenika. 
\par 36 Zaista, kažem vam, sve će to doći na ovaj naraštaj!" 
\par 37 "Jeruzaleme, Jeruzaleme, koji ubijaš proroke i kamenuješ  one što su tebi poslani! Koliko li puta htjedoh okupiti djecu  tvoju kao što kvočka okuplja piliće pod krila, i ne htjedoste. 
\par 38 Evo, napuštena vam kuća. 
\par 39 Doista, kažem vam, odsada me  nećete vidjeti dok ne reknete: Blagoslovljen Onaj koji dolazi  u ime Gospodnje!" 


\chapter{24}

\par 1 Isus iziđe iz Hrama. Putom mu pristupiše učenici pokazujući  mu hramsko zdanje. 
\par 2 A on im reče: "Ne vidite li sve ovo? Zaista, kažem vam, ne, neće se ovdje ostaviti ni kamen na kamenu nerazvaljen." 
\par 3 Dok je zatim na Maslinskoj gori sjedio, pristupiše k njemu  učenici nasamo govoreći: "Reci nam kada će to biti i koji će  biti znak tvojega Dolaska i svršetka svijeta?" 
\par 4 Isus im odgovori: "Pazite da vas tko ne zavede! 
\par 5 Mnogi  će doista doći u moje ime i govoriti: 'Ja sam Krist!' I mnoge  će zavesti." 
\par 6 "A čut ćete za ratove i za glasove o ratovima. Pazite, ne uznemirujte se. Doista treba da se to dogodi, ali  to još nije svršetak. 
\par 7 Narod će ustati protiv naroda i kraljevstvo  protiv kraljevstva; bit će gladi i potresa po raznim mjestima. 
\par 8 Ali sve je to samo početak trudova." 
\par 9 "Tada će vas predavati na muke i ubijati vas. I svi će  vas narodi zamrziti zbog imena moga. 
\par 10 Mnogi će se tada  sablazniti, izdavat će jedni druge i mrziti se među sobom. 
\par 11 Ustat će mnogi lažni proroci i mnoge zavesti. 
\par 12 Razmahat  će se bezakonje i ohladnjeti ljubav mnogih. 
\par 13 Ali tko ustraje  do svršetka, bit će spašen." 
\par 14 "I propovijedat će se ovo evanđelje Kraljevstva po svem  svijetu za svjedočanstvo svim narodima. Tada će doći svršetak." 
\par 15 "Kada dakle vidite da grozota pustoši, po proroštvu  Daniela proroka, stoluje na svetome mjestu - tko čita, neka razumije: 
\par 16 koji se tada zateknu u Judeji, neka bježe  u gore; 
\par 17 tko bude na krovu, neka ne silazi uzeti što iz kuće; 
\par 18 i tko bude u polju, neka se ne okreće natrag da uzme  haljinu!" 
\par 19 "A jao trudnicama i dojiljama u one dane!" 
\par 20 "I molite da bijeg vaš ne bude zimi ili subotom 
\par 21 jer  tada će biti velika tjeskoba kakve ne bijaše od početka svijeta  sve do sada, a neće je ni biti." 
\par 22 "I kad se ne bi skratili dani oni, nitko se ne bi spasio.  No poradi izabranih skratit će se dani oni." 
\par 23 "Ako vam tada tko rekne: 'Gle, evo Krista!' ili: 'Eno  ga!' - ne povjerujte! 
\par 24 Ustat će, doista, lažni kristi i lažni  proroci i iznijeti znamenja velika i čudesa da, bude  li moguće, zavedu i izabrane." 
\par 25 "Eto, prorekao sam vam." 
\par 26 "Reknu li vam dakle: 'Evo, u pustinji je!', ne izlazite;  'Evo ga u ložnicama!', ne vjerujte. 
\par 27 Jer kao što munja izlazi  od istoka i bljesne do zapada, tako će biti i s dolaskom Sina  Čovječjega." 
\par 28 "Gdje bude strvine, ondje će se skupljati orlovi." 
\par 29 "A odmah nakon nevolje onih dana sunce će pomrčati i mjesec neće više svijetljeti i zvijezde će s neba padati i sile će se nebeske poljuljati." 
\par 30 "I tada će se pojaviti znak Sina Čovječjega na nebu.  I tada će proplakati sva plemena zemlje. I ugledat će  Sina Čovječjega gdje dolazi na oblacima nebeskim s velikom  moći i slavom. 
\par 31 I razaslat će anđele svoje s trubljom velikom  i sabrat će mu izabranike s četiri vjetra, s jednoga kraja  neba do drugoga." 
\par 32 "A od smokve se naučite prispodobi! Kad joj grana već  omekša i lišće potjera, znate: blizu je ljeto. 
\par 33 Tako i vi  kad sve to ugledate, znajte: blizu je, na vratima!" 
\par 34 "Zaista, kažem vam, ne, neće uminuti naraštaj ovaj dok  se sve to ne zbude. 
\par 35 Nebo će i zemlja uminuti, ali riječi  moje ne, neće uminuti." 
\par 36 "A o onom danu i času nitko ne zna, pa ni anđeli nebeski, ni Sin, nego samo Otac. 
\par 37 Kao u dane Noine, tako će biti i  Dolazak Sina Čovječjega. 
\par 38 Kao što su u dane one - prije potopa  - jeli i pili, ženili se i udavali do dana kad Noa uđe u korablju 
\par 39 i ništa nisu ni slutili dok ne dođe potop i sve odnije -  tako će biti i Dolazak Sina Čovječjega. 
\par 40 Dvojica će tada biti  u polju: jedan će se uzeti, drugi ostaviti. 
\par 41 Dvije će mljeti  u mlinu: jedna će se uzeti, druga ostaviti." 
\par 42 "Bdijte dakle jer ne znate u koji dan Gospodin vaš dolazi. 
\par 43 A ovo znajte: kad bi domaćin znao o kojoj straži kradljivac  dolazi, bdio bi i ne bi dopustio potkopati kuće. 
\par 44 Zato i vi  budite pripravni jer u čas kad i ne mislite Sin Čovječji dolazi." 
\par 45 "Tko li je onaj vjerni i razumni sluga što ga gospodar  postavi nad svojim ukućanima da im izda hranu u pravo vrijeme? 
\par 46 Blago onome sluzi kojega gospodar kada dođe nađe da tako  radi! 
\par 47 Zaista, kažem vam, postavit će ga nad svim imanjem  svojim." 
\par 48 "No rekne li taj zli sluga u srcu: 'Okasnit će gospodar  moj' 
\par 49 pa stane tući sudrugove, jesti i piti s pijanicama, 
\par 50 doći će gospodar toga sluge u dan u koji mu se ne nada i  u čas u koji i ne sluti; 
\par 51 rasjeći će ga i dodijeliti mu udes  među licemjerima. Ondje će biti plač i škrgut zubi." 


\chapter{25}

\par 1 "Tada će kraljevstvo nebesko biti kao kad deset djevica uzeše  svoje svjetiljke i iziđoše u susret zaručniku. 
\par 2 Pet ih bijaše  ludih, a pet mudrih. 
\par 3 Lude uzeše svjetiljke, ali ne uzeše sa  sobom ulja. 
\par 4 Mudre pak zajedno sa svjetiljkama uzeše u posudama  ulja." 
\par 5 "Budući da je zaručnik okasnio, sve one zadrijemaše i  pozaspaše. 
\par 6 O ponoći nasta vika: 'Evo zaručnika! Iziđite mu  u susret!' 
\par 7 Tada ustadoše sve one djevice i urediše svoje svjetiljke. 
\par 8 Lude tada rekoše mudrima: 'Dajte nam od svoga ulja, gase nam  se svjetiljke!' 
\par 9 Mudre im odgovore: 'Nipošto! Ne bi doteklo  nama i vama. Pođite radije k prodavačima i kupite!'" 
\par 10 "Dok one odoše kupiti, dođe zaručnik: koje bijahu pripravne, uđoše s njim na svadbu i zatvore se vrata. 
\par 11 Poslije dođu  i ostale djevice pa stanu dozivati: 'Gospodine! Gospodine! Otvori  nam!' 
\par 12 A on im odgovori: 'Zaista kažem vam, ne poznam vas!' 
\par 13 Bdijte dakle jer ne znate dana ni časa!" 
\par 14 "Doista, kao kad ono čovjek, polazeći na put, dozva sluge  i dade im svoj imetak. 
\par 15 Jednomu dade pet talenata, drugomu  dva, a trećemu jedan - svakomu po njegovoj sposobnosti. 
\par 16 I  otputova. Onaj koji je primio pet talenata odmah ode, upotrijebi  ih i stekne drugih pet. 
\par 17 Isto tako i onaj sa dva stekne druga  dva. 
\par 18 Onaj naprotiv koji je primio jedan ode, otkopa zemlju  i sakri novac gospodarov." 
\par 19 "Nakon dugo vremena dođe gospodar tih slugu i zatraži  od njih račun. 
\par 20 Pristupi mu onaj što je primio pet talenata  i donese drugih pet govoreći: 'Gospodaru! Pet si mi talenata  predao. Evo, drugih sam pet talenata stekao!' 
\par 21 Reče mu gospodar:  'Valjaš, slugo dobri i vjerni! U malome si bio vjeran, nad mnogim  ću te postaviti! Uđi u radost gospodara svoga!'" 
\par 22 "Pristupi i onaj sa dva talenta te reče: 'Gospodaru!  Dva si mi talenta predao. Evo, druga sam dva talenta stekao!' 
\par 23 Reče mu gospodar: 'Valjaš, slugo dobri i vjerni! U malome  si bio vjeran, nad mnogim ću te postaviti! Uđi u radost gospodara  svoga.'" 
\par 24 "A pristupi i onaj koji je primio jedan talenat te reče:  'Gospodaru! Znadoh te: čovjek si strog, žanješ gdje nisi sijao  i kupiš gdje nisi vijao. 
\par 25 Pobojah se stoga, odoh i sakrih  talenat tvoj u zemlju. Evo ti tvoje!' 
\par 26 A gospodar mu reče:  'Slugo zli i lijeni! Znao si da žanjem gdje nisam sijao i kupim  gdje nisam vijao! 
\par 27 Trebalo je dakle da uložiš moj novac kod  novčara i ja bih po povratku izvadio svoje s dobitkom.'" 
\par 28 "'Uzmite stoga od njega talenat i podajte onomu koji  ih ima deset. 
\par 29 Doista, onomu koji ima još će se dati, neka  ima u izobilju, a od onoga koji nema oduzet će se i ono što ima. 
\par 30 A beskorisnoga slugu izbacite van u tamu. Ondje će biti plač  i škrgut zubi.'" 
\par 31 "Kad Sin Čovječji dođe u slavi i svi anđeli njegovi s  njime, sjest će na prijestolje slave svoje. 
\par 32 I sabrat će se  pred njim svi narodi, a on će ih jedne od drugih razlučiti kao  što pastir razlučuje ovce od jaraca. 
\par 33 Postavit će ovce sebi  zdesna, a jarce slijeva." 
\par 34 "Tada će kralj reći onima sebi zdesna: 'Dođite, blagoslovljeni  Oca mojega! Primite u baštinu Kraljevstvo pripravljeno za vas  od postanka svijeta! 
\par 35 Jer ogladnjeh i dadoste mi jesti; ožednjeh  i napojiste me; stranac bijah i primiste me; 
\par 36 gol i zaogrnuste  me; oboljeh i pohodiste me; u tamnici bijah i dođoste k meni.'" 
\par 37 "Tada će mu pravednici odgovoriti: 'Gospodine, kada te  to vidjesmo gladna i nahranismo te; ili žedna i napojismo te? 
\par 38 Kada te vidjesmo kao stranca i primismo; ili gola i zaogrnusmo  te? 
\par 39 Kada te vidjesmo bolesna ili u tamnici i dođosmo k tebi?' 
\par 40 A kralj će im odgovoriti: 'Zaista, kažem vam, što god učiniste  jednomu od ove moje najmanje braće, meni učiniste!'" 
\par 41 "Zatim će reći i onima slijeva: 'Odlazite od mene, prokleti, u oganj vječni, pripravljen đavlu i anđelima njegovim! 
\par 42 Jer  ogladnjeh i ne dadoste mi jesti; ožednjeh i ne dadoste mi piti; 
\par 43 stranac bijah i ne primiste me; gol i ne zaogrnuste me; bolestan  i u tamnici i ne pohodiste me!'" 
\par 44 "Tada će mu i oni odgovoriti: 'Gospodine, a kada te to  vidjesmo gladna, ili žedna, ili stranca, ili gola, ili bolesna, ili u tamnici, i ne poslužismo te?' 
\par 45 Tada će im on odgovoriti:  'Zaista, kažem vam, što god ne učiniste jednomu od ovih najmanjih, ni meni ne učiniste.'" 
\par 46 "I otići će ovi u muku vječnu, a pravednici u život vječni." 


\chapter{26}

\par 1 I kad Isus završi sve te besjede, reče svojim učenicima: 
\par 2 "Znate  da je za dva dana Pasha, i Sin Čovječji predaje se da se razapne." 
\par 3 Uto se sabraše glavari svećenički i starješine narodne u dvoru  velikoga svećenika imenom Kajfe 
\par 4 i zaključiše Isusa na prijevaru  uhvatiti i ubiti. 
\par 5 Jer se govorilo: "Nikako ne o Blagdanu da  ne nastane pobuna u narodu." 
\par 6 Kad je Isus bio u Betaniji, u kući Šimuna Gubavca, 
\par 7 pristupi  mu neka žena s alabastrenom posudicom skupocjene pomasti i polije  ga po glavi, dok je on bio za stolom. 
\par 8 Vidjevši to, učenici  negodovahu: "Čemu ta rasipnost? 
\par 9 Moglo se to skupo prodati  i dati siromasima." 
\par 10 Zapazio to Isus pa im reče: "Što dodijavate ženi? Dobro  djelo učini prema meni. 
\par 11 Ta siromaha svagda imate uza se,  a mene nemate svagda. 
\par 12 Izlila je tu pomast na moje tijelo  - za ukop mi to učini. 
\par 13 Zaista, kažem vam, gdje se god bude  propovijedalo ovo evanđelje, po svem svijetu, navješćivat će  se i ovo što ona učini - njoj na spomen." 
\par 14 Tada jedan od dvanaestorice, zvan Juda Iškariotski, pođe  glavarima svećeničkim 
\par 15 i reče: "Što ćete mi dati i ja ću vam  ga predati." A oni mu odmjeriše trideset srebrnjaka. 
\par 16 Otada  je tražio priliku da ga preda. 
\par 17 Prvoga dana Beskvasnih kruhova pristupiše učenici Isusu  i upitaše: "Gdje hoćeš da ti pripravimo te blaguješ pashu?" 
\par 18 On  reče: "Idite u grad tomu i tomu i recite mu: 'Učitelj veli: Vrijeme  je moje blizu, kod tebe slavim pashu sa svojim učenicima.'" 
\par 19 I  učine učenici kako im naredi Isus i priprave pashu. 
\par 20 Uvečer bijaše Isus za stolom s dvanaestoricom. 
\par 21 I  dok su blagovali, reče: "Zaista, kažem vam, jedan će me od vas  izdati." 
\par 22 Silno ožalošćeni, stanu mu jedan za drugim govoriti:  "Da nisam ja, Gospodine?" 
\par 23 On odgovori: "Onaj koji umoči sa  mnom ruku u zdjelu, taj će me izdati. 
\par 24 Sin Čovječji, istina, odlazi kako je o njemu pisano, ali jao čovjeku onomu koji predaje  Sina Čovječjega. Tomu bi čovjeku bolje bilo da se ni rodio nije." 
\par 25 A Juda, izdajnik, prihvati i reče: "Da nisam ja, učitelju?"  Reče mu: "Ti kaza." 
\par 26 I dok su blagovali, uze Isus kruh, izreče blagoslov pa  razlomi, dade svojim učenicima i reče: "Uzmite i jedite! Ovo  je tijelo moje!" 
\par 27 I uze čašu, zahvali i dade im govoreći:  "Pijte iz nje svi! 
\par 28 Ovo je krv moja, krv Saveza koja  se za mnoge prolijeva na otpuštenje grijeha. 
\par 29 A kažem vam:  ne, neću od sada piti od ovog roda trsova do onoga dana kad ću  ga - novoga - s vama piti u kraljevstvu Oca svojega." 
\par 30 Otpjevavši  hvalospjeve, zaputiše se prema Maslinskoj gori. 
\par 31 Tada im reče Isus: "Svi ćete se vi još ove noći sablazniti  o mene. Ta pisano je: Udarit će pastira i stado će se razbjeći. 
\par 32 Ali kad uskrsnem, ići ću pred vama u Galileju.' 
\par 33 Nato  će mu Petar: "Ako se i svi sablazne o tebe, ja se nikada neću!" 
\par 34 Reče mu Isus: "Zaista, kažem ti, još ove noći, prije negoli  se pijetao oglasi, triput ćeš me zatajiti!" 
\par 35 Kaže mu Petar:  "Bude li trebalo i umrijeti s tobom, ne, neću te zatajiti." Tako  rekoše i svi učenici. 
\par 36 Tada dođe Isus s njima u predio zvan Getsemani i kaže  učenicima: "Sjednite ovdje dok ja odem onamo pomoliti se." 
\par 37 I  povede sa sobom Petra i oba sina Zebedejeva. Spopade ga žalost  i tjeskoba. 
\par 38 Tada im reče: "Duša mi je nasmrt žalosna.  Ostanite ovdje i bdijte sa mnom!" 
\par 39 I ode malo dalje, pade ničice moleći: "Oče moj! Ako je  moguće, neka me mimoiđe ova čaša. Ali ne kako ja hoću, nego kako  hoćeš ti." 
\par 40 I dođe učenicima i nađe ih pozaspale pa reče Petru: "Tako, zar niste mogli jedan sat probdjeti sa mnom? 
\par 41 Bdijte i molite  da ne padnete u napast! Duh je, istina, voljan, no tijelo je  slabo." 
\par 42 Opet, po drugi put, ode i pomoli se: "Oče moj! Ako  nije moguće da me čaša mine da je ne pijem, budi volja tvoja!" 
\par 43 I ponovno dođe i nađe ih pozaspale, oči im se sklapale. 
\par 44 Opet  ih ostavi, pođe i pomoli se po treći put ponavljajući iste riječi. 
\par 45 Tada dođe učenicima i reče im: "Samo spavajte i počivajte!  Evo, približio se čas! Sin Čovječji predaje se u ruke grešničke! 
\par 46 Ustanite, hajdemo! Evo, približio se moj izdajica." 
\par 47 Dok je on još govorio, gle, dođe Juda, jedan od dvanaestorice, i s njime silna svjetina s mačevima i toljagama poslana od glavara  svećeničkih i starješina narodnih. 
\par 48 A izdajica im dao znak:  "Koga poljubim, taj je, njega uhvatite!" 
\par 49 I odmah pristupi  Isusu i reče: "Zdravo, Učitelju!" I poljubi ga. 
\par 50 A Isus mu  reče: "Prijatelju, zašto ti ovdje!" Tada pristupe, podignu ruke  na Isusa i uhvate ga. 
\par 51 I gle, jedan od onih koji bijahu s Isusom maši se rukom, trgnu mač, udari slugu velikoga svećenika i odsiječe mu uho. 
\par 52 Kaže mu tada Isus: "Vrati mač na njegovo mjesto jer svi koji  se mača laćaju od mača i ginu. 
\par 53 Ili zar misliš da ja ne mogu  zamoliti Oca svojega i eto umah uza me više od dvanaest legija  anđela? 
\par 54 No kako bi se onda ispunila Pisma da tako mora biti?" 
\par 55 U taj čas reče Isus svjetini: "Kao na razbojnika iziđoste  s mačevima i toljagama da me uhvatite? Danomice sjeđah u Hramu  naučavajući i ne uhvatiste me." 
\par 56 A sve se to dogodilo da se  ispune Pisma proročka. Tada ga svi učenici ostave i pobjegnu. 
\par 57 Nato uhvatiše Isusa i odvedoše ga velikomu svećeniku  Kajfi, kod kojega se sabraše pismoznanci i starješine. 
\par 58 A  Petar je išao za njim izdaleka do dvora velikog svećenika; i  ušavši unutra, sjedne sa stražarima da vidi svršetak. 
\par 59 A glavari svećenički i cijelo Vijeće tražili su kakvo  lažno svjedočanstvo protiv Isusa da bi ga mogli pogubiti. 
\par 60 Ali  ne nađoše premda pristupiše mnogi lažni svjedoci. Napokon pristupe  dvojica 
\par 61 i reknu: "Ovaj reče: 'Mogu razvaliti Hram Božji i  za tri ga dana sagraditi.'" 
\par 62 Usta nato veliki svećenik i reče mu: "Zar ništa ne odgovaraš?  Što to ovi protiv tebe svjedoče?" 
\par 63 Isus je šutio. Reče mu veliki svećenik: "Zaklinjem te Bogom živim: Kaži nam  jesi li ti Krist, Sin Božji?" 
\par 64 Reče mu Isus: "Ti kaza! Štoviše, kažem vam: Odsada ćete gledati Sina Čovječjega gdje sjedi  zdesna Sile i dolazi na oblacima nebeskim." 
\par 65 Nato veliki  svećenik razdrije haljine govoreći: "Pohulio je! Što nam još  trebaju svjedoci! Evo, sada ste čuli hulu! 
\par 66 Što vam se čini?"  Oni odgovoriše: "Smrt zaslužuje!" 
\par 67 Tada su mu pljuvali u lice i udarali ga, a drugi ga pljuskali 
\par 68 govoreći: "Proreci nam, Kriste, tko te udario?" 
\par 69 A Petar je sjedio vani u dvorištu. I pristupi mu jedna  sluškinja govoreći: "I ti bijaše s Isusom Galilejcem." 
\par 70 On  pred svima zanijeka: "Ne znam što govoriš." 
\par 71 Kad iziđe u predvorje, spazi ga druga i kaže nazočnima: "Ovaj bijaše s Isusom Nazarećaninom." 
\par 72 On opet zanijeka sa zakletvom: "Ne znam toga čovjeka." 
\par 73 Malo zatim nazočni pristupiše Petru i rekoše: "Doista, i ti si od njih! Ta govor te tvoj izdaje!" 
\par 74 On se tada stane  zaklinjati i preklinjati: "Ne znam toga čovjeka." I odmah se  oglasi pijetao. 
\par 75 I spomenu se Petar riječi koju mu Isus reče:  "Prije nego se pijetao oglasi, triput ćeš me zatajiti." I iziđe  te gorko zaplaka. 


\chapter{27}

\par 1 A kad objutri, svi su glavari svećenički i starješine narodne  održali vijećanje protiv Isusa da ga pogube. 
\par 2 I svezana ga  odveli i predali upravitelju Pilatu. 
\par 3 Kada Juda, njegov izdajica, vidje da je Isus osuđen, pokaja  se i vrati trideset srebrnjaka glavarima svećeničkim i starješinama 
\par 4 govoreći: "Sagriješih predavši krv nedužnu!" Odgovoriše: "Što  se to nas tiče? To je tvoja stvar!" 
\par 5 I bacivši srebrnjake u  Hram, ode te se objesi. 
\par 6 Glavari svećenički uzeše srebrnjake i rekoše: 
\par 7 "Nije  dopušteno staviti ih u hramsku riznicu jer su krvarina." Posavjetuju  se i kupe za njih lončarovu njivu za ukop stranaca. 
\par 8 Stoga  se ona njiva zove "Krvava njiva" sve do danas. 
\par 9 Tada se ispuni  što je rečeno po proroku Jeremiji: Uzeše trideset srebrnjaka  - cijenu Neprocjenjivoga kojega procijeniše sinovi Izraelovi  - 
\par 10 i dadoše ih za njivu lončarovu kako mi naredi Gospodin. 
\par 11 Dovedoše dakle Isusa pred upravitelja. Upita ga upravitelj:  "Ti li si kralj židovski?" On odgovori: "Ti kažeš." 
\par 12 I dok  su ga glavari svećenički i starješine narodne optuživale, ništa  nije odgovarao. 
\par 13 Tada mu reče Pilat: "Ne čuješ li što sve  protiv tebe svjedoče?" 
\par 14 I ne odgovori mu ni na jednu riječ  te se upravitelj silno čudio. 
\par 15 A o Blagdanu upravitelj je običavao svjetini pustiti  jednoga uznika, koga bi već htjeli. 
\par 16 Tada upravo bijaše u  njih poznati uznik zvani Baraba. 
\par 17 Kad se dakle sabraše, reče  im Pilat: "Koga hoćete da vam pustim: Barabu ili Isusa koji se  zove Krist?" 
\par 18 Znao je doista da ga predadoše iz zavisti. 
\par 19 Dok je sjedio na sudačkoj stolici, poruči nu njegova  žena: "Mani se ti onoga pravednika jer sam danas u snu mnogo  pretrpjela zbog njega." 
\par 20 Međutim, glavari svećenički i starješine nagovore svjetinu  da zaište Barabu, a Isus da se pogubi. 
\par 21 Upita ih dakle upravitelj:  "Kojega od ove dvojice hoćete da vam pustim?" A oni rekoše: "Barabu!" 
\par 22 Kaže im Pilat: "Što dakle da učinim s Isusom koji se zove  Krist?" Oni će: "Neka se razapne." 
\par 23 A on upita: "A što je  zla učinio?" Vikahu još jače: "Neka se razapne!" 
\par 24 Kad Pilat vidje da ništa ne koristi, nego da biva sve  veći metež, uzme vodu i opere ruke pred svjetinom govoreći: "Nevin  sam od krvi ove! Vi se pazite!" 
\par 25 Sav narod nato odvrati: "Krv  njegova na nas i na djecu našu!" 
\par 26 Tada im pusti Barabu, a  Isusa, izbičevana, preda da se razapne. 
\par 27 Onda vojnici upraviteljevi uvedoše Isusa u dvor upraviteljev  i skupiše oko njega cijelu četu. 
\par 28 Svukoše ga pa zaogrnuše  skrletnim plaštem. 
\par 29 Spletoše zatim vijenac od trnja i staviše  mu na glavu, a tako i trsku u desnicu. Prigibajući pred njim  koljena, izrugivahu ga: "Zdravo, kralju židovski!" 
\par 30 Onda pljujući  po njemu, uzimahu trsku i udarahu ga njome po glavi. 
\par 31 Pošto ga izrugaše, svukoše mu plašt, obukoše mu njegove  haljine pa ga odvedoše da ga razapnu. 
\par 32 Izlazeći nađu nekoga čovjeka Cirenca, imenom Šimuna,  i prisile ga da mu ponese križ. 
\par 33 I dođoše na mjesto zvano Golgota, to jest Lubanjsko mjesto, 
\par 34 dadoše mu piti vino sa žuči pomiješano.  I kad okusi, ne htjede piti. 
\par 35 A pošto ga razapeše, razdijeliše  među se haljine njegove bacivši kocku. 
\par 36 I sjedeći ondje, čuvahu ga. 
\par 37 I staviše mu ponad glave krivicu napisanu: "Ovo je Isus, kralj židovski." 
\par 38 Tada razapeše s njime dva razbojnika, jednoga zdesna, drugoga slijeva. 
\par 39 A prolaznici su ga pogrđivali mašući glavama: 
\par 40 "Ti  koji razvaljuješ Hram i za tri ga dana sagradiš, spasi sam sebe!  Ako si Sin Božji, siđi s križa!" 
\par 41 Slično i glavari svećenički  s pismoznancima i starješinama, rugajući se, govorahu: 
\par 42 "Druge  je spasio, sebe ne može spasiti! Kralj je Izraelov! Neka sada  siđe s križa pa ćemo povjerovati u nj! 
\par 43 Uzdao se u Boga!  Neka ga sad izbavi ako mu omilje! Ta govorio je: 'Sin sam  Božji!'" 
\par 44 Tako ga vrijeđahu i s njim raspeti razbojnici. 
\par 45 Od šeste ure nasta tama po svoj zemlji - do ure devete. 
\par 46 O devetoj uri povika Isus iza glasa: "Eli, Eli, lema sabahtani?"  To će reći: "Bože moj, Bože moj, zašto si me ostavio?" 
\par 47 A neki od nazočnih, čuvši to, govorahu: "Ovaj zove Iliju." 
\par 48 I odmah pritrča jedan od njih, uze spužvu, natopi je octom,  natakne je na trsku i pruži mu piti. 
\par 49 A ostali  rekoše: "Pusti da vidimo hoće li doći Ilija da ga spasi." 
\par 50 A  Isus opet povika iz glasa i ispusti duh. 
\par 51 I gle, zavjesa se hramska razdrije odozgor dodolje, nadvoje;  zemlja se potrese, pećine se raspukoše, 
\par 52 grobovi otvoriše  i tjelesa mnogih svetih preminulih uskrsnuše 
\par 53 te iziđoše iz  grobova nakon njegova uskrsnuća, uđoše u sveti grad i pokazaše  se mnogima. 
\par 54 A satnik i oni koji su s njime čuvali Isusa vidješe potres  i što se zbiva, silno se prestrašiše i rekoše: "Uistinu, Sin  Božji bijaše ovaj." 
\par 55 A bijahu ondje i izdaleka promatrahu mnoge žene što su  iz Galileje išle za Isusom poslužujući mu; 
\par 56 među njima Marija  Magdalena i Marija, Jakovljeva i Josipova majka, i majka sinova  Zebedejevih. 
\par 57 Uvečer dođe neki bogat čovjek iz Arimateje, imenom Josip, koji i sam bijaše učenik Isusov. 
\par 58 On pristupi Pilatu i zaiska  tijelo Isusovo. Tada Pilat zapovjedi da mu se dadne. 
\par 59 Josip  uze tijelo, povi ga u čisto platno 
\par 60 i položi u svoj novi grob  koji bijaše isklesao u stijeni. Dokotrlja velik kamen na grobna  vrata i otiđe. 
\par 61 A bijahu ondje Marija Magdalena i druga Marija: sjedile  su nasuprot grobu. 
\par 62 Sutradan, to jest dan nakon Priprave, sabraše se glavari  svećenički i farizeji kod Pilata 
\par 63 te mu rekoše: "Gospodaru, sjetismo se da onaj varalica još za života kaza: 'Nakon tri  dana uskrsnut ću.' 
\par 64 Zapovjedi dakle da se grob osigura sve  do trećega dana da ne bi možda došli njegovi učenici, ukrali  ga pa rekli narodu: 'Uskrsnuo je od mrtvih!' I bit će posljednja  prijevara gora od prve." 
\par 65 Reče im Pilat: "Imate stražu! Idite  i osigurajte kako znate!" 
\par 66 Nato oni odu i osiguraju grob:  zapečate kamen i postave stražu. 



\chapter{28}

\par 1 Po suboti, u osvit prvoga dana u tjednu, dođe Marija Magdalena  i druga Marija pogledati grob. 
\par 2 I gle, nastade žestok potres  jer anđeo Gospodnji siđe s neba, pristupi, otkotrlja kamen i  sjede na nj. 
\par 3 Lice mu bijaše kao munja, a odjeća bijela kao  snijeg. 
\par 4 Od straha pred njim zadrhtaše stražari i obamriješe. 
\par 5 A anđeo progovori ženama: "Vi se ne bojte! Ta znam: Isusa  Raspetoga tražite! 
\par 6 Nije ovdje! Uskrsnu kako reče. Hajde, vidite  mjesto gdje je ležao 
\par 7 pa pođite žurno i javite njegovim učenicima  da uskrsnu od mrtvih. I evo, ide pred vama u Galileju. Ondje  ćete ga vidjeti. Evo, rekoh vam." 
\par 8 One otiđoše žurno s groba te sa strahom i velikom radošću  otrčaše javiti njegovim učenicima. 
\par 9 Kad eto im Isusa u susret!  Reče im: "Zdravo!" One polete k njemu, obujme mu noge i ničice  mu se poklone. 
\par 10 Tada im Isus reče: "Ne bojte se! Idite, javite  mojoj braći da pođu u Galileju! Ondje će me vidjeti!" 
\par 11 Dok su one odlazile, gle, neki od straže dođoše u grad  i javiše glavarima svećeničkim sve što se dogodilo. 
\par 12 Oni se  sabraše sa starješinama na vijećanje, uzeše mnogo novaca i dadoše  vojnicima 
\par 13 govoreći: "Recite: 'Noću dok smo mi spavali, dođoše  njegovi učenici i ukradoše ga.' 
\par 14 Ako to dočuje upravitelj, mi ćemo ga uvjeriti i sve učiniti da vi budete bez brige." 
\par 15 Oni  uzeše novac i učiniše kako bijahu poučeni. I razglasilo se to  među Židovima - sve do danas. 
\par 16 Jedanaestorica pođoše u Galileju na goru kamo im je naredio  Isus. 
\par 17 Kad ga ugledaše, padoše ničice preda nj. A neki posumnjaše. 
\par 18 Isus im pristupi i prozbori: "Dana mi je sva vlast na nebu i na zemlji! 
\par 19 Pođite dakle  i učinite mojim učenicima sve narode krsteći ih u ime Oca i Sina  i Duha Svetoga 
\par 20 i učeći ih čuvati sve što sam vam zapovjedio!" "I evo, ja sam s vama u sve dane - do svršetka svijeta." 




\end{document}