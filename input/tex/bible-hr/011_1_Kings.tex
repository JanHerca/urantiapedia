\begin{document}

\title{1 Kraljevima}


\chapter{1}

\par 1 Kralj David bijaše ostario i odmakao u godinama; premda su  ga pokrivali mnogim pokrivačima, nije se mogao ugrijati. 
\par 2 Tada  mu rekoše njegove sluge: "Trebalo bi potražiti za gospodara mladu  djevojku koja bi dvorila kralja i služila mu: kad bude spavala  na njegovu krilu, to će ugrijati kralja gospodara." 
\par 3 Potražiše, dakle, lijepu djevojku po svoj zemlji izraelskoj; i nađoše Abišagu  Šunamku te je dovedoše kralju. 
\par 4 Djevojka je bila izvanredno  lijepa; njegovala je kralja i služila mu, ali je on ne upozna. 
\par 5 Uto se Adonija, sin Hagitin, pooholi i pomisli: "Ja ću  biti kralj!" Zato nabavi sebi kola i konjanika i pedeset ljudi  koji su išli pred njim. 
\par 6 Njegov ga otac za svoga života nikad  nije ukorio niti ga kad upitao: "Zašto tako činiš?" Bio je, osim  toga, stasit i lijep, a mati ga rodila poslije Abšaloma. 
\par 7 On  se dogovarao s Joabom, sinom Sarvijinim, i sa svećenikom Ebjatarom, pa se obojica priključiše Adoniji. 
\par 8 Ali svećenik Sadok i Jojadin  sin Benaja, prorok Natan, Šimej i Rei i junaci Davidovi ne pristadoše  uz Adoniju. 
\par 9 Jednom nakla Adonija ovaca, volova i tovljene  teladi za žrtvu kod Zoheledskog kamena, blizu izvora Rogela,  te pozva svu svoju braću, sinove kraljeve, i sve Judejce u kraljevoj  službi; 
\par 10 ali ne pozva proroka Natana, ni Benaje, ni ostalih  junaka, a ni svoga brata Salomona. 
\par 11 Tada reče Natan Bat-Šebi, majci Salomonovoj: "Zar nisi  čula da je Adonija, sin Hagitin, postao kraljem, a da David,  naš gospodar, o tome i ne zna? 
\par 12 Dođi da te savjetujem kako  bi mogla spasiti život svoj i svoga sina Salomona. 
\par 13 Hajde, otiđi kralju Davidu i reci mu: 'Zar se nisi ti, gospodaru moj  kralju, zakleo svojoj službenici govoreći: Tvoj sin Salomon kraljevat  će poslije mene, i on će sjediti na mome prijestolju! Kako sada  Adonija posta kraljem?' 
\par 14 I dok ti budeš ondje i razgovorala  se s kraljem, doći ću ja za tobom i potvrditi tvoje riječi." 
\par 15 Bat-Šeba ode kralju u odaje - a on je bio vrlo star i  Abišaga Šunamka služila mu. 
\par 16 Pokloni mu se Bat-Šeba i pade  ničice pred kraljem, a kralj upita: "Što želiš?" 
\par 17 Ona mu odgovori:  "Gospodaru, ti si se zakleo službenici svojoj Jahvom, Bogom svojim:  'Tvoj sin Salomon kraljevat će poslije mene, on će sjesti na  moje prijestolje.' 
\par 18 A sada je, evo, Adonija postao kraljem, a ti, kralju, gospodaru moj, ništa o tome i ne znaš! 
\par 19 Naklao  je on mnogo volova, tovljene teladi i ovaca za žrtvu i pozvao  je sve sinove kraljeve, svećenika Ebjatara i vojskovođu Joaba, ali slugu tvoga Salomona nije pozvao. 
\par 20 U tebe su sada, gospodaru  moj i kralju, uprte oči svega Izraela da mu ti objaviš tko će  te naslijediti na tvome prijestolju, kralju, gospodaru moj. 
\par 21 Inače, čim počine kralj, gospodar moj, kraj svojih otaca, ja i moj  sin Salomon bit ćemo krivci." 
\par 22 Dok je ona još govorila s kraljem, dođe prorok Natan. 
\par 23 Javiše kralju: "Ovdje je prorok Natan." On uđe kralju i pade  ničice pred njim. 
\par 24 Natan reče: "Gospodaru moj i kralju, jesi  li ti odredio: 'Adonija će kraljevati poslije mene i sjedit će  na mome prijestolju?' 
\par 25 Jer evo danas je sišao i naklao volova, ugojene teladi i ovaca za žrtvu i pozvao je sve sinove kraljeve, vojskovođe i svećenika Ebjatara; eno ih gdje jedu i piju s njim  i kliču: 'Živio kralj Adonija!' 
\par 26 Ali mene, tvoga slugu, svećenika  Sadoka, a ni Benaju, sina Jojadina, ni tvoga slugu Salomona nije  pozvao. 
\par 27 Zar se to dogodilo s voljom gospodara moga kralja, a da nisi obavijestio svoga vjernog sluge tko će biti nasljednik  na prijestolju gospodara moga kralja?" 
\par 28 Tada progovori David i reče: "Pozovite mi Bat-Šebu!"  Ona dođe kralju i stupi preda nj. 
\par 29 Kralj se tada zakle: "Tako  mi Jahve živoga koji me izbavio iz svih nevolja! 
\par 30 Danas ću  ti ispuniti kako sam ti se zakleo Jahvom, Bogom Izraelovim: tvoj  će sin Salomon kraljevati poslije mene, on će sjediti na mome  prijestolju!" 
\par 31 Nato se nakloni Bat-Šeba licem do zemlje, pokloni se  pred kraljem i reče: "Neka vječno živi gospodar moj kralj David!" 
\par 32 A kralj David reče: "Pozovite mi svećenika Sadoka, proroka  Natana i Benaju, sina Jojadina." I dođoše oni pred kralja, 
\par 33 a on im reče: "Uzmite sluge  svoga gospodara sa sobom, posadite moga sina Salomona na moju  mazgu i odvedite ga do Gihona. 
\par 34 Ondje neka ga svećenik Sadok  i prorok Natan pomažu za kralja nad Izraelom. Zatrubite tada  i obznanite: 'Živio kralj Salomon!' 
\par 35 Zatim se uspnite amo  s njim i neka uđe i sjedne na moje prijestolje i neka kraljuje  mjesto mene, jer moja je volja: on neka bude glava nad Izraelom  i nad Judom." 
\par 36 Benaja, sin Jojadin, reče kralju: "Amen - tako neka bude!  To je i riječ Jahve, Gospodara kraljeva! 
\par 37 Kao što je Jahve  bio s mojim gospodarem kraljem, tako neka bude i sa Salomonom!  Neka uzvisi prijestolje njegovo još više nego prijestolje kralja  Davida, gospodara moga!" 
\par 38 Svećenik Sadok, prorok Natan, Jojadin sin Benaja, Kerećani  i Pelećani siđoše i posadiše Salomona na kraljevu mazgu i odvedoše  ga na Gihon. 
\par 39 Svećenik Sadok donese iz Šatora rog s uljem  i pomaza Salomona. Tada odjeknuše trube i sav narod povika: "Živio  kralj Salomon!" 
\par 40 I sav narod pođe za njim gore i sviraše puk  u svirale i klicaše tako da se sva zemlja tresla. 
\par 41 Čuo to Adonija i svi njegovi uzvanici. Baš su bili pri  kraju gozbe. I Joab je čuo trube pa upita: "Čemu ta buka u gradu?" 
\par 42 Dok je on još govorio, stiže Jonatan, sin svećenika Ebjatara, i Adonija mu reče: "Ti si valjan čovjek, zacijelo nosiš dobru  vijest!" 
\par 43 Jonatan odgovori: "Jest, naš gospodar, kralj David, učinio  je Salomona kraljem! 
\par 44 Kralj je poslao s njim svećenika Sadoka, proroka Natana i Jojadina sina Benaju, i Kerećane i Pelećane.  Oni ga posadiše na kraljevu mazgu, 
\par 45 i svećenik Sadok i prorok  Natan pomazaše ga na Gihonu za kralja. Zatim su sišli radosno  kličući, i sav je grad uzavreo; to je buka koju ste čuli. 
\par 46 Još  više: Salomon je već sjeo na kraljevsko prijestolje 
\par 47 i došle  su sluge kraljeve čestitati našem gospodaru kralju Davidu govoreći:  'Neka Bog tvoj proslavi ime Salomonovo više od imena tvoga i  prijestolje njegovo uzvisi više od tvoga.' Kralj se tada poklonio  na svojoj postelji 
\par 48 i ovako rekao: 'Neka je blagoslovljen  Jahve, Bog Izraelov, koji mi dade danas da mogu vidjeti svojim  očima jednoga od mojih kako sjedi na mome prijestolju.'" 
\par 49 Svi uzvanici Adonijini, uplašeni, ustadoše od stola i  raziđoše se svaki svojim putem. 
\par 50 Adonija pak, u strahu od  Salomona, usta i ode te se uhvati za rogove žrtvenika. 
\par 51 Javiše  Salomonu: "Gle, Adonija se uplašio kralja Salomona i eno se drži  za rogove žrtvenika govoreći: 'Neka mi se danas kralj Salomon  zakune da neće sluge svoga mačem pogubiti.'" 
\par 52 Salomon reče nato: "Ako se pokaže poštenim čovjekom,  neće mu ni vlas s glave pasti na zemlju; a nađe li se u zlu,  poginut će." 
\par 53 Tada zapovjedi Salomon da ga odmaknu od žrtvenika;  on dođe i pade ničice pred Salomonom, koji mu reče: "Pođi svome  domu!" 


\chapter{2}

\par 1 Kad su se dani Davidovi približavali svome svršetku, zapovjedi  David svome sinu Salomonu: 
\par 2 "Sada polazim na put sviju smrtnika.  Ti budi hrabar i pokaži se čovjekom! 
\par 3 Slušaj naredbe Jahve, Boga svoga, idi njegovim stazama, drži se njegovih zakona, zapovijedi, naredaba i njegovih pouka, kako je napisano u Zakonu Mojsijevu, da bi uspio u svemu što poduzmeš i svagdje kamo se okreneš; 
\par 4 da bi Jahve ispunio svoje obećanje koje mi je dao: 'Ako sinovi  tvoji budu pazili na svome putu, vjerno hodeći preda mnom, svim  srcem svojim i svom dušom svojom, uvijek će jedan od njih sjediti  na prijestolju Izraelovu.' 
\par 5 I sam znaš što mi je učinio Joab, sin Sarvijin, kako je  učinio obojici vojskovođa Izraelovih: Abneru, sinu Nerovu, i  Amasi, sinu Jeterovu, kad ih je ubio i time prolio krv u miru  kao u ratu te omastio krvlju pojas oko bokova svojih i obuću  na nogama svojim. 
\par 6 Ti postupi po svom razboru i ne daj da mu  sijeda kosa mirno počine u Podzemlju. 
\par 7 A sinovima Barzilaja  Gileađanina vrati ljubav: neka budu među onima koji jedu za tvojim  stolom jer su mi pomogli kad sam bježao pred tvojim bratom Abšalomom. 
\par 8 Pred sobom imaš Šimeja, sina Gerina, Benjaminovca iz Bahurima, koji me užasnim kletvama proklinjao onoga dana kad sam bježao  u Mahanajim. Ali mi je on sišao u susret na Jordan i zakleh mu  se Jahvom: 'Neću te pogubiti mačem.' 
\par 9 Ali mu ti toga ne opraštaj, jer si čovjek razborit, i već ćeš znati kako treba da postupiš  te mu sijedu kosu s krvlju u Podzemlje spremiš." 
\par 10 I potom počinu David kraj otaca svojih i bi pokopan u  Davidovu gradu. 
\par 11 David je kraljevao nad Izraelom četrdeset  godina: u Hebronu je kraljevao sedam godina, u Jeruzalemu je  kraljevao trideset i tri godine. 
\par 12 Salomon sjede na prijestolje Davida, svoga oca, i njegova  se vlast veoma učvrsti. 
\par 13 Ali Adonija, sin Hagitin, dođe Bat-Šebi, majci Salomonovoj, i pade ničice pred njom. Ona ga upita: "Je li miroljubiv tvoj  dolazak?" On odgovori: "Jest, miroljubiv je." 
\par 14 I nastavi:  "Imam ti nešto reći." Ona reče: "Govori." 
\par 15 Tada će on: "Znaš  i sama da je kraljevstvo pripadalo meni i da je sav Izrael očekivao  da ću ja biti kralj. Ali mi je kraljevstvo izmaklo i pripalo  je mome bratu, jer mu ga je Jahve namijenio. 
\par 16 Ja te sada samo  jedno molim: nemoj me odbiti." Ona reče: "Govori." 
\par 17 A on nastavi:  "Reci, molim te, kralju Salomonu - jer tebe neće odbiti - neka  mi dade za ženu Abišagu Šunamku!" 
\par 18 A Bat-Šeba odgovori: "Dobro, govorit ću kralju o tebi." 
\par 19 Kada dakle uđe Bat-Šeba kralju Salomonu da govori o Adoniji, ustade kralj i pođe joj u susret, pokloni se pred njom, zatim  sjede na svoje prijestolje i zapovjedi te namjestiše sjedalicu  za kraljicu majku, i ona mu sjede s desne strane. 
\par 20 Tada mu  reče: "Nešto bih zaiskala od tebe, nemoj me odbiti." Kralj joj  odgovori: "Traži, majko, jer te neću odbiti." 
\par 21 Ona nastavi:  "Neka se dade Abišaga Šunamka tvome bratu Adoniji za ženu." 
\par 22 Kralj  Salomon odgovori i reče svojoj majci: "Zašto tražiš Abišagu Šunamku  za Adoniju? Traži odmah i kraljevstvo za njega! Jer on je moj  stariji brat, a uz njega je svećenik Ebjatar i Joab, sin Sarvijin!" 
\par 23 Tada se kralj Salomon zakle Jahvom: "Neka mi Bog učini ovo  zlo i neka mi doda drugo ako Adonija nije to izrekao danas po  cijenu svoga života! 
\par 24 Živoga mi Jahve, koji me potvrdio i  posadio na prijestolje oca moga Davida i koji mi je dao dom kako  je obećao: još danas će Adonija umrijeti." 
\par 25 I kralj Salomon posla Benaju, sina Jojadina, koji ga  udari te Adonija umrije. 
\par 26 Svećeniku Ebjataru kralj zatim naredi: "Idi u Anatot  na svoj posjed. Zaslužio si smrt, ali te neću pogubiti danas  jer si nosio Jahvin Kovčeg pred ocem mojim Davidom i podijelio  si sve patnje s mojim ocem." 
\par 27 I Salomon isključi Ebjatara  iz svećenstva Jahvina da tako ispuni Jahvinu riječ koju je izrekao  protiv doma Elijeva u Šilu. 
\par 28 Kada je glas stigao Joabu - Joab bijaše pristao uz Adoniju, premda se nije priključio Abšalomu - on uteče u Šator Jahvin  i uhvati se za rogove žrtvenika. 
\par 29 I dojaviše kralju Salomonu:  "Joab je pobjegao u Šator Jahvin, eno ga pokraj žrtvenika." Tada  Salomon poruči Joabu: "Što se držiš žrtvenika?" Joab odgovori:  "Uplašio sam se tebe i pobjegao sam pred Jahvu." Tada Salomon  naredi Benaji, sinu Jojadinu: "Idi i ubij ga!" 
\par 30 Benaja ode u Šator Jahvin i reče Joabu: "Po naredbi kraljevoj:  iziđi!" On odgovori: "Neću, želim ovdje umrijeti!" Benaja javi  kralju: "Eto što mi je rekao Joab i što mi je odgovorio." 
\par 31 Kralj mu reče: "Učini kako je rekao: ubij ga, zatim pokopaj.  Tako ćeš danas skinuti s mene i doma oca moga nevinu krv koju  je Joab prolio. 
\par 32 Jahve će učiniti da krv njegova padne na  njegovu glavu, jer je ubio dva čovjeka pravednika i bolja od  sebe; ubio ih je mačem bez znanja moga oca Davida: Abnera, sina  Nerova, vođu vojske Izraelove, i Amasu, sina Jeterova, vojvodu  judejskoga. 
\par 33 Neka njihova krv padne na glavu Joaba i njegova  potomstva dovijeka, a Davidu, njegovu potomstvu, vladalačkoj  kući i prijestolju neka od Jahve bude trajan mir." 
\par 34 I ode Benaja, sin Jojadin, obori se na Joaba i usmrti  ga. Pokopali su Joaba u njegovu domu u pustinji. 
\par 35 Mjesto njega  postavi kralj na čelo vojske Benaju, sina Jojadina, a na mjesto  Ebjatara postavi svećenika Sadoka. 
\par 36 Salomon pozva Šimeja i reče mu: "Sagradi sebi kuću u  Jeruzalemu: tu stanuj, i nikamo odatle ne izlazi. 
\par 37 Onoga dana  kad iziđeš i prijeđeš potok Kidron, znaj dobro da ćeš umrijeti.  Krv tvoja na glavu tvoju." 
\par 38 Šimej odgovori kralju: "Dobro.  Kako moj gospodar kralj kaže, tako će učiniti sluga tvoj." I  Šimej dugo življaše u Jeruzalemu. 
\par 39 Ali poslije tri godine dogodi se te Šimeju pobjegoše  dvojica slugu k Akišu, sinu Maakinu, kralju gatskom. I dojaviše  Šimeju: "Eno ti slugu u Gatu." 
\par 40 Tada usta Šimej, osedla magarca  i ode u Gat, k Akišu, da traži svoje sluge. I vratio se Šimej  i doveo svoje sluge iz Gata. 
\par 41 I javiše Salomonu: "Šimej otišao  iz Jeruzalema u Gat i vratio se." 
\par 42 Kralj pozva Šimeja i reče mu: "Nisam li ti se zakleo  Jahvom i strogo te opomenuo: 'Onoga dana kad budeš izišao i pošao  bilo kamo, znaj dobro da ćeš umrijeti!' A ti si mi tada odgovorio:  'Dobra je riječ koju sam čuo.' 
\par 43 Zašto nisi održao zakletvu  Jahvinu i zapovijed koju sam ti dao?" 
\par 44 Još reče kralj Šimeju:  "Ti znaš sve zlo koje si učinio mome ocu Davidu. Tvoje je srce  toga svjesno. Jahve neka učini da se tvoja zloća obori na tvoju  glavu. 
\par 45 A blagoslovljen je kralj Salomon, i prijestolje će  Davidovo biti čvrsto pred Jahvom dovijeka." 
\par 46 I zapovjedi kralj Benaji, sinu Jojadinu, te on iziđe  i udari Šimeja i tako Šimej umrije. Tako se učvrstilo kraljevstvo u ruci Salomonovoj. 


\chapter{3}

\par 1 Salomon se sprijatelji s faraonom, kraljem egipatskim: oženi  se kćerju faraonovom i uvede je u Davidov grad dokle ne dovrši  gradnju svoga dvora, Hrama Jahvina i zidova oko Jeruzalema. 
\par 2 Narod  je pak prinosio žrtve na uzvišicama, jer još nije bio sagrađen  do toga vremena dom imenu Jahvinu. 
\par 3 A Salomon je ljubio Jahvu:  ravnao se prema naredbama svoga oca Davida, samo je prinosio  klanice i kađenice na uzvišicama. 
\par 4 Kralj ode u Gibeon da prinese žrtvu, jer ondje bijaše  najveća uzvišica. Salomon prinese tisuću paljenica na tom žrtveniku. 
\par 5 U Gibeonu se Jahve javi Salomonu noću u snu. Bog reče: "Traži  što da ti dadem." 
\par 6 Salomon odgovori: "Veoma si naklon bio svome  sluzi Davidu, mome ocu, jer je hodio pred tobom u vjernosti,  pravednosti i poštenju srca svoga; i sačuvao si mu tu veliku  milost i dao si da jedan od njegovih sinova sjedi na njegovu  prijestolju. 
\par 7 Sada, o Jahve, Bože moj, ti si učinio kraljem  slugu svoga na mjesto moga oca Davida, a ja sam još sasvim mlad  te još ne znam vladati. 
\par 8 Tvoj je sluga usred naroda koji si  izabrao; naroda brojnog, koji se ne da izbrojiti ni popisati. 
\par 9 Podaj svome sluzi pronicavo srce da može suditi tvom narodu, razlikovati dobro od zla, jer tko bi mogao upravljati tvojim  narodom koji je tako velik!" 
\par 10 Bijaše milo Jahvi što je Salomon to zamolio. 
\par 11 Zato  mu Jahve reče: "Jer si to tražio, a nisi iskao ni duga života, ni bogatstva, ni smrti svojih neprijatelja, nego pronicavost  u prosuđivanju pravice, 
\par 12 evo ću učiniti po riječima tvojim:  dajem ti srce mudro i razumno, kakvo nije imao nitko prije tebe  niti će ga imati itko poslije tebe, 
\par 13 ali ti dajem i što nisi  tražio: bogatstvo i slavu kakve nema nitko među kraljevima. 
\par 14 I  ako budeš stupao mojim putovima i budeš se držao mojih zakona  i zapovijedi, kao što je činio tvoj otac David, umnožit ću tvoje  dane." 
\par 15 Salomon se probudi, i gle: bijaše to san. On se vrati  u Jeruzalem i stade pred Kovčeg saveza Jahvina; prinese paljenice  i žrtve pričesnice i priredi gozbu svim slugama svojim. 
\par 16 Tada dođoše dvije bludnice kralju i stadoše preda nj. 
\par 17 I reče jedna žena: "Dopusti, gospodaru moj! Ja i ova žena  u istoj kući živimo i ja sam rodila kraj nje u kući. 
\par 18 A trećega  dana poslije moga porođaja rodi i ova žena. Bile smo zajedno  i nikoga stranog s nama; samo nas dvije u kući. 
\par 19 Jedne noći  umrije sin ove žene jer bijaše legla na njega. 
\par 20 I ustade ona  usred noći, uze moga sina o boku mojem, dok je tvoja sluškinja  spavala, i stavi ga sebi u naručje, a svoga mrtvog sina stavi  kraj mene. 
\par 21 A kad ujutro ustadoh da podojim svoga sina, gle:  on mrtav! I kad sam pažljivije pogledala, razabrah: nije to moj  sin koga sam ja rodila!" 
\par 22 Tada reče druga žena: "Ne, nije tako. Moj je sin onaj  živi, a tvoj je onaj koji je mrtav!" A prva joj odvrati: "Nije  istina! Tvoj je sin onaj koji je mrtav, a moj je onaj koji živi!"  I tako se prepirahu pred kraljem. 
\par 23 A kralj onda progovori: "Ova kaže: 'Ovaj živi moj je  sin, a onaj mrtvi tvoj'; druga pak kaže: 'Nije, nego je tvoj  sin mrtav, a moj je onaj živi.' 
\par 24 Donesite mi mač!" naredi  kralj. I donesoše mač pred kralja, 
\par 25 a on reče: "Rasijecite  živo dijete nadvoje i dajte polovinu jednoj, a polovinu drugoj." 
\par 26 Tada ženu, majku živog djeteta, zabolje srce za sinom  i povika ona kralju: "Ah, gospodaru! Neka se njoj dade dijete, samo ga nemojte ubijati!" A ona druga govoraše: "Neka ne bude  ni meni ni tebi: rasijecite ga!" 
\par 27 Onda progovori kralj i reče: "Dajte dijete prvoj, nipošto  ga ne ubijajte! Ona mu je majka." 
\par 28 Sav je Izrael čuo presudu koju je izrekao kralj i poštovali  su kralja, jer su vidjeli da je u njemu božanska mudrost u izricanju  pravde. 


\chapter{4}

\par 1 Kralj Salomon bio je kralj nad svim Izraelom, 
\par 2 a evo njegovih  odličnika:  Azarja, sin Sadokov, svećenik; 
\par 3 Elihoref i Ahija, sinovi Šišini, bilježnici; Jošafat, sin Ahiludov, savjetnik; 
\par 4 Benaja, sin Jojadin, vojskovođa; Sadok i Ebjatar, svećenici. 
\par 5 Azarja, sin Natanov, bio je nad namjesnicima; Zabud, sin Natanov, prijatelj kraljev; 
\par 6 Ahišar, upravitelj dvora; Eliab, sin Joabov, zapovjednik vojske; Adoram, sin Abdin, nadstojnik za tlaku. 
\par 7 Salomon je imao po svem Izraelu dvanaest namjesnika koji  su opskrbljivali kralja i njegov dom; za svakoga je dolazio red  da po jedan mjesec u godini podmiruje to uzdržavanje. 
\par 8 Evo njihovih imena: ...sin Hurov, u gori Efrajimovoj; 
\par 9 ...sin Dekerov, u Makasu, Šaalbimu, Bet Šemešu, Elonu  do Bet Hanana; 
\par 10 ...sin Hesedov, u Arubotu; pod njim bijaše Soho i sav  kraj heferski; 
\par 11 ...sin Abinadabov, nad svim okružjem dorskim; žena mu  je bila Tafata, kći Salomonova; 
\par 12 Baana, sin Ahiludov, u Tanaku i Megidu i u svem Bet Šeanu, koji je pokraj Saretana niže Jizreela, od Bet Šeana do Abel  Mekole, i preko Jokmeama. 
\par 13 ...sin Geberov, u Ramotu Gileadskom; njegova su bila  Sela Jaira, sina Manašeova, koja su u Gileadu; imao je i područje  Argob koje leži u Bašanu, šezdeset tvrdih gradova, opasanih zidovima  i prijevornicama od tuča; 
\par 14 Ahinabad, sin Idov, u Mahanajimu; 
\par 15 Ahimaas u Naftaliju; i on se oženio jednom Salomonovom  kćeri - Bosmatom. 
\par 16 Baana, sin Hušajev, u Ašeru i na visoravnima; 
\par 17 Jošafat, sin Paruahov, u Jisakaru; 
\par 18 Šimej, sin Elin, u Benjaminu; 
\par 19 Geber, sin Urijin, u zemlji Gileadu, zemlji Sihona, kralja  amorejskoga, i Oga, kralja bašanskoga. Povrh toga bio je još jedan namjesnik u zemlji. 
\par 20 Juda i Izrael bili su mnogobrojni, bijaše ih kao pijeska  na obali morskoj. Jeli su i pili i bili sretni. 
\par 21 (5:1) Salomon je proširio svoju vlast nad svim kraljevstvima  od Rijeke sve do zemlje filistejske i do međe egipatske. Ona  su donosila svoj danak i služila Salomonu sve dane njegova života. 
\par 22 (5:2) Svakoga je dana trebalo Salomonu za hranu: trideset kora finoga  brašna i šezdeset kora običnog brašna, 
\par 23 (5:3) deset ugojenih volova, dvadeset volova s paše, stotinu ovaca, osim jelena, srna, divokoza  i ugojene peradi. 
\par 24 (5:4) Jer on je vladao nad svime onkraj Rijeke  - od Tafse do Gaze, nad svim kraljevima s onu stranu Eufrata  - i imao je mir po svim granicama naokolo. 
\par 25 (5:5) Juda i sav Izrael  živjeli su bez straha, svaki pod svojom lozom i pod svojom smokvom, od Dana sve do Beer Šebe, svega vijeka Salomonova. 
\par 26 (5:6) Salomon  je imao četrdeset tisuća konja za vuču i dvanaest tisuća za jahanje. 
\par 27 (5:7) Ti su se namjesnici brinuli o opskrbi kralja Salomona i sviju  koji su imali dijela za kraljevim stolom, svaki po mjesec dana;  i nisu dopuštali da ičega ponestane. 
\par 28 (5:8) I ječam i slamu za konje  i tegleću marvu donosili su na mjesto gdje se zadržavao, svaki  kako bi ga zapalo. 
\par 29 (5:9) Jahve je dao Salomonu mudrost i izuzetnu razboritost i  srce široko kao pijesak na obali morskoj. 
\par 30 (5:10) Mudrost je Salomonova  bila veća od mudrosti svih sinova Istoka i od sve mudrosti Egipta. 
\par 31 (5:11) Bio je mudriji od svih ljudi, od Etana Ezrahanina, od Hemana, Kalkola i Darde, sinova Maholovih; njegovo se ime pronosilo  među svim narodima unaokolo. 
\par 32 (5:12) Izrekao je tri tisuće mudrih  izreka, a njegovih je pjesama bilo tisuću i pet. 
\par 33 (5:13) Zborio je  o drveću: od cedra što je na Libanonu pa do izopa što klija na  zidu; raspravljao je o životinjama, o pticama, o gmazovima i  o ribama. 
\par 34 (5:14) Dolazili su od sviju naroda da čuju mudrost Salomonovu, od svih zemaljskih kraljeva koji su čuli glas o njegovoj mudrosti. 


\chapter{5}

\par 1 (5:15) Tirski kralj Hiram posla svoje sluge Salomonu, jer bijaše  čuo da su ga pomazali za kralja na mjesto njegova oca, a Hiram  je svagda bio prijatelj Davidov. 
\par 2 (5:16) Tada Salomon poruči Hiramu: 
\par 3 (5:17) "Ti znaš dobro da moj otac David nije mogao sagraditi Doma  imenu Jahve, svoga Boga, zbog ratova kojima su ga okružili neprijatelji  sa svih strana, sve dok ih Jahve nije položio pod stopala nogu  njegovih. 
\par 4 (5:18) Sada mi je Jahve, Bog moj, dao mir posvuda unaokolo:  nemam neprijatelja ni zlih udesa. 
\par 5 (5:19) Namjeravam, dakle, sagraditi  Dom imenu Jahve, Boga svoga, kako je Jahve rekao mome ocu Davidu:  'Tvoj sin koga ću mjesto tebe postaviti na tvoje prijestolje, on će sagraditi Dom mome Imenu.' 
\par 6 (5:20) Stoga sada zapovjedi da  mi nasijeku cedrova na Libanonu; moje će sluge biti sa slugama  tvojim, i ja ću platiti nadnicu tvojim slugama prema svemu kako  mi odrediš. Ti znaš dobro da u nas nema ljudi koji umiju sjeći  drva kao Sidonci." 
\par 7 (5:21) Kada je Hiram primio Salomonovu poruku, veoma se obradova  i reče: "Neka je blagoslovljen danas Jahve koji je dao mudra  sina Davidu, koji upravlja ovim velikim narodom." 
\par 8 (5:22) I Hiram  javi Salomonu: "Primio sam tvoju poruku. Ispunit ću u svemu tvoju  želju glede drva cedrova i drva čempresova. 
\par 9 (5:23) Moje će ih sluge  dopremiti s Libanona na more, složit ću ih u splavi i pustiti  ih morem do mjesta koje ćeš mi označiti; ondje ću ih razložiti  i ti ćeš ih uzeti. Ti ćeš pak ispuniti moju želju i dati hranu  mojoj čeljadi." 
\par 10 (5:24) Hiram je davao Salomonu drva cedrova i čempresova koliko  je htio, 
\par 11 (5:25) a Salomon je davao Hiramu dvadeset tisuća kora pšenice  za hranu ljudstvu, i dvadeset tisuća kora ulja od tiještenih  maslina. Toliko je Salomon davao Hiramu godinu za godinom. 
\par 12 (5:26) Jahve  je dao mudrost Salomonu, kako mu bijaše obećao; između Hirama  i Salomona vladao je mir te oni sklopiše savez. 
\par 13 (5:27) Tada diže kralj Salomon kulučare iz svega Izraela; kulučara  je bilo u svemu trideset tisuća ljudi. 
\par 14 (5:28) Slao ih je naizmjence  na Libanon, svakog mjeseca deset tisuća ljudi: bili su mjesec  dana na Libanonu, a dva mjeseca kod kuće. Adoniram je bio nad  svim kulučarima. 
\par 15 (5:29) Salomon je imao i sedamdeset tisuća nosača  tereta, osamdeset tisuća kamenorezaca u gori, 
\par 16 (5:30) ne računajući  glavara službeničkih koji su upravljali poslovima; njih je bilo  tri tisuće i tri stotine, a upravljali su narodom zaposlenim  na radovima. 
\par 17 (5:31) Kralj je zapovjedio da lome gromade biranog  kamena i da ih klešu za temelje Hrama. 
\par 18 (5:32) Graditelji Salomonovi  i Hiramovi, i oni iz Gibela, tesali su i pripremali drvo i klesali  za gradnju Hrama. 


\chapter{6}

\par 1 Četiri stotine i osamdesete godine poslije izlaska Izraelaca  iz zemlje egipatske, četvrte godine kraljevanja svoga nad Izraelom, mjeseca Ziva - to je drugi mjesec - počeo je Salomon graditi  Dom Jahvin. 
\par 2 Hram što ga je kralj Salomon gradio Jahvi bio  je dug šezdeset lakata, širok dvadeset, a visok dvadeset i pet  lakata. 
\par 3 Trijem pred Hekalom Hrama bio je dvadeset lakata dug, prema širini Hrama, a deset lakata širok, prema dužini Hrama. 
\par 4 Na Hramu je napravio prozore zatvorene rešetkama. 
\par 5 Uza zid  Hrama oko Hekala i Debira sagradio je prigradnju na katove, sve  unaokolo. 
\par 6 Donji kat bio je pet lakata širok, srednji šest, a treći sedam lakata, jer je zasjeke rasporedio s vanjske strane  naokolo Hrama da ih ne bi morao ugrađivati u hramske zidove. 
\par 7 Hram je građen od kamena koji je već u kamenolomu bio oklesan, tako da se za gradnje nije čuo ni čekić ni dlijeto, ni ikakvo  željezno oruđe. 
\par 8 Ulaz u donji kat bio je s desne strane Hrama, a zavojnim se stubama uspinjalo na srednji kat i sa srednjega  na treći. 
\par 9 Sagradio je tako Hram i dovršio ga; i pokrio ga cedrovim  gredama i daskama. 
\par 10 I sagradi još prigradnju oko cijeloga  Hrama; bila je pet lakata visoka, a vezana s Hramom cedrovim  gredama. 
\par 11 I riječ Jahvina stiže Salomonu: 
\par 12 "To je Dom što  ga gradiš ... Ako budeš hodio prema naredbama mojim, ako budeš  vršio naredbe moje i držao se mojih zapovijedi, tada ću ispuniti  tebi obećanje što sam ga dao tvome ocu Davidu: 
\par 13 prebivat ću  među sinovima Izraelovim i neću ostaviti naroda svoga Izraela." 
\par 14 I tako Salomon sazida Hram i dovrši ga. 
\par 15 I obloži iznutra zidove Hrama cedrovim daskama - od poda  do stropa obloži ih drvetom iznutra - a daskama čempresovim obloži  pod Hrama. 
\par 16 I načini pregradu od dvadeset lakata, od cedrovih  dasaka, s poda pod strop, i odijeli taj dio Hrama za Debir, za  Svetinju nad svetinjama. 
\par 17 A Hekal - Svetište, dio Hrama ispred  Debira - imaše četrdeset lakata. 
\par 18 A po cedrovini unutar Hrama  bijahu urezani ukrasi - pleteri od pupoljaka i cvijeća; sve je  bilo od cedrovine i nigdje se nije vidio kamen. 
\par 19 Debir je  uredio unutra u Hramu da onamo smjesti Kovčeg saveza Jahvina. 
\par 20 Debir bijaše dvadeset lakata dug, dvadeset lakata širok i  dvadeset lakata visok, a obložio ga je čistim zlatom. Napravio  je i Žrtvenik od cedrovine, 
\par 21 pred Debirom, i obložio ga čistim  zlatom. 
\par 22 I sav je Hram obložio zlatom, sav Hram i sav oltar  koji je pred Debirom obložio je zlatom. 
\par 23 U Debiru načini dva kerubina od maslinova drveta. Bili  su visoki deset lakata. 
\par 24 Jedno je krilo u kerubina bilo pet  lakata i drugo je krilo u kerubina bilo pet lakata; deset je  lakata bilo od jednoga kraja krila do drugoga. 
\par 25 I drugi je  kerubin bio od deset lakata: jednaka mjera i jednak oblik obaju  kerubina. 
\par 26 Visina jednog kerubina bila je deset lakata, tako  i drugoga. 
\par 27 Smjestio je kerubine usred nutarnje prostorije;  širili su svoja krila, tako da je krilo jednoga ticalo jedan  zid, a krilo drugoga ticalo drugi zid; u sredini prostorije krila  im se doticahu. 
\par 28 I kerubine je obložio zlatom. 
\par 29 Po svim  zidovima Hrama unaokolo, iznutra i izvana, urezao je likove kerubina, palma i rastvorenih cvjetova, 
\par 30 zlatom je pokrio i pod Hramu  iznutra i izvana. 
\par 31 A za ulaz u Debir načini dvokrilna vrata od maslinova  drveta; dovraci s pragom bijahu na pet uglova. 
\par 32 Oba krila  na vratima od maslinova drveta ukrasi likovima kerubina, palma  i rastvorenih cvjetova, i sve ih obloži zlatom; listićima zlata  oblijepi kerubine i palme. 
\par 33 Tako i za ulaz u Hekal načini  vrata od maslinova drveta, sa četverokutnim dovracima. 
\par 34 Oba  krila na vratima bijahu od čempresova drveta i oba se otvarahu  na jednu i na drugu stranu. 
\par 35 Urezao je na njima kerubine,  palme i rastvorene cvjetove i obložio zlatom sve što bijaše urezano. 
\par 36 Potom je sagradio unutrašnje predvorje od tri reda klesanog  kamena i jednoga reda tesanih greda cedrovih. 
\par 37 Temelji su  Hramu Jahvinu bili položeni četvrte godine, mjeseca Ziva; 
\par 38 a  jedanaeste godine, mjeseca Bula - to je osmi mjesec - Hram je  dovršen sa svim dijelovima i sa svim što mu pripada. Salomon  ga sagradi za sedam godina. 


\chapter{7}

\par 1 Salomon je sagradio i svoj dvor; u trinaest ga je godina potpuno  dovršio. 
\par 2 Sagradio je dvor od libanonske šume: stotinu lakata  dug, pedeset širok i trideset lakata visok, na četiri reda cedrovih  stupova, a na stupovima bijahu cedrove grede. 
\par 3 Bio je pokriven  cedrovinom iznad soba koje su počivale na stupovima. Ovih je  bilo četrdeset i pet: petnaest u svakom redu. 
\par 4 Bila su tri  reda prozora: po tri su prozora gledala jedan prema drugome. 
\par 5 Sva vrata s dovratnicima bila su četverokutna i po tri su  prozora stajala jedan prema drugome. 
\par 6 Načinio je trijem od  stupova, pedeset lakata dug i trideset širok. 
\par 7 Zatim je sagradio  prijestolni trijem gdje je sudio; i sudački trijem, obložen cedrovinom  od poda do stropa. 
\par 8 Njegovo prebivalište, u drugom dvorištu  i unutar predvorja, bilo je istoga oblika. Sagradio je i kuću, nalik na onaj trijem, faraonovoj kćeri, kojom se bijaše oženio. 
\par 9 Sve su te građevine bile od biranog kamena, sječena po  mjeri, a klesana iznutra i izvana, od temelja sve do drvenih  spojnica, a vani sve do velikog predvorja. 
\par 10 Temelji su im  bili od birana, velikog kamena: od deset i od osam lakata, 
\par 11 a  nadgradnja od birana, po mjeri klesana kamena i od cedrovine. 
\par 12 A tri su reda klesanog kamena i red cedrovih greda okruživali  veliko predvorje, a tako i unutrašnje predvorje Doma Jahvina. 
\par 13 Salomon posla po Hirama iz Tira. 
\par 14 Bio je to sin udovice  iz plemena Naftalijeva, ali mu otac bijaše iz Tira, kovač tuča.  Bio je pun vještine, umijeća i znanja da svašta izrađuje od tuča.  Dođe on kralju Salomonu i sav mu posao izradi. 
\par 15 Salio je dva stupa od tuča; jedan je stup bio visok osamnaest  lakata, a koncem mjeren unaokolo imao je dvanaest lakata, isto  tako i drugi. 
\par 16 I načini dvije glavice od tuča da se stave  povrh stupova; jedna je glavica bila visoka pet lakata i druga  je bila pet lakata visoka. 
\par 17 Načini dva opleta u obliku pletera  i lančaste žice da pokriju glavice na vrhu stupova; sedam za  jednu glavicu i sedam za drugu. 
\par 18 Onda izradi mogranje: bili  su u dva reda oko svake mreže. 
\par 19 Glavice na vrhu stupova pred  trijemom imale su oblik ljiljana, od četiri lakta. 
\par 20 Stajale  su na oba stupa kod izbočine što je bila prema lančancu. Dvije  stotine mogranja bilo je oko prve glavice i dvije stotine oko  druge. 
\par 21 Podiže stupove pred trijemom Hekala; jedan postavi  na desnu stranu i nazva ga Jahin; postavi drugi stup na lijevu  stranu i dade mu ime Boaz. 
\par 22 Na samom vrhu stupova postavi  izrađene ljiljane. I tako dovrši stupove. 
\par 23 Tada od rastaljene kovine izli more koje je od ruba do  ruba mjerilo deset lakata; bilo je okruglo naokolo, pet lakata  visoko, a u opsegu, mjereno vrpcom, imalo je trideset lakata. 
\par 24 Pod rubom mu bijahu uresi kao cvjetne čaške koje su ga optakale  sasvim: po deset na lakat optakale su more unaokolo; cvjetne  su čaške bile u dva reda i salivene s njim. 
\par 25 Počivalo je na  dvanaest volova: tri su gledala na sjever, tri na zapad, tri  na jug, a tri na istok; more je stajalo na njima i svi su stražnjim  dijelom bili okrenuti unutra. 
\par 26 Bilo je debelo pedalj, rub  mu kao rub u čaše, kao cvijet, a moglo je primiti tri tisuće  bata. 
\par 27 Načinio je deset tučanih podnožja; svako je podnožje  bilo četiri lakta dugo, četiri lakta široko, a tri lakta visoko. 
\par 28 Podnožja su bila ovako izrađena: imala su okvire, a okviri  su stajali među preponama. 
\par 29 Na okvirima među preponama bili  su lavovi, volovi i kerubini; a na samim preponama, kako iznad  lavova i volova tako i pod njima, bijahu ukrasi poput vijenaca. 
\par 30 Svako je podnožje imalo četiri tučana točka i osovine od  tuča; četiri su njihove noge imale držače; pod umivaonikom bijahu  držači sliveni s ukrasima. 
\par 31 Gore, gdje su se držači sastavljali, bio je otvor podnožja; imao je lakat i pol; otvor je bio okrugao, u obliku ukrasne posude, a na njemu su bili uklesani i ukrasi;  ali prepone bijahu četvrtaste, a ne okrugle. 
\par 32 Četiri su točka bila pod preponom. Osovine im izlazile  na podnožju; svaki točak bijaše visok lakat i pol. 
\par 33 Točkovi  su bili slični točkovima običnih kola: njihove osovine, naplaci, paoci i glavčine - sve bijaše liveno. 
\par 34 Bila su četiri držača  na četiri ugla svakog podnožja; podnožje i držači sačinjavahu  jednu cjelinu. 
\par 35 Pri vrhu podnožja bio je sve unaokolo krug  visok pol lakta; povrh podnožja bili su klinovi; prepone su s  njima sačinjavale cjelinu. 
\par 36 Po oplošjima klinova i prepona  urezao je kerube, lavove i palme, već prema veličini praznog  oplošja i vijenaca naokolo. 
\par 37 Tako načini deset podnožja: jednako  salivenih, jednake veličine i oblika. 
\par 38 I načini deset umivaonika od tuča. Svaki je umivaonik  sadržavao četrdeset bata, a svaki je umivaonik bio od četiri  lakta; na svako od deset podnožja došao je po jedan umivaonik. 
\par 39 Postavi pet podnožja na desnoj strani Hrama, a pet na lijevoj  strani Hrama; a more stavi s desne strane Hrama, prema jugoistoku. 
\par 40 Hiram načini lonce, lopate i kotliće. Dovrši on sav posao  što ga je obavljao kralju Salomonu za Dom Jahvin: 
\par 41 dva stupa, okrugle glavice što su bile navrh stupova; dva opleta da pokriju  dvije glavice što bijahu navrh stupova; 
\par 42 četiri stotine mogranja  za oba opleta; dva reda mogranja za svaki oplet da prekriju dvije  glavice navrh stupova; 
\par 43 deset podnožja i deset umivaonika  na podnožjima; 
\par 44 jedno more i dvanaest volova pod njim; 
\par 45 lonce, lopate i kotliće. Svi ti predmeti koje je Hiram načinio kralju  Salomonu za Dom Jahvin bili su od sjajnog tuča. 
\par 46 Kralj je  zapovjedio da sve to lijevaju u kalupima od gline, u Jordanskoj  dolini, između Sukota i Sartana. 
\par 47 Na koncu je Salomon odredio  da rasporede sve te predmete, a bijaše ih toliko da se nije mogla  obračunati težina tuča. 
\par 48 Salomon načini sve predmete koji  su bili u Domu Jahvinu: zlatni žrtvenik i zlatni stol na kojemu  su stajali prineseni hljebovi; 
\par 49 pet svijećnjaka s desne i  pet s lijeve strane pred Debirom, od čistoga zlata; cvjetove, svjetiljke, usekače od zlata; 
\par 50 vrčeve, noževe, kotliće, plitice  i kadionice od čistoga zlata; stožere za vrata nutarnje dvorane  - to je Svetinja nad svetinjama - i za vrata Hekala - to jest  Hrama - sve od zlata. 
\par 51 Tako bi priveden kraju sav posao što ga Salomon obavi  za Dom Jahvin. Salomon unese sve svete darove oca svoga Davida  - srebro, zlato i posuđe - i stavi ih u riznicu Doma Jahvina. 


\chapter{8}

\par 1 Tada Salomon sazva u Jeruzalem sve starješine Izraelove, sve  knezove plemenske i glavare obitelji da se prenese Kovčeg saveza  Jahvina iz grada, Davidova grada, to jest sa Siona. 
\par 2 Svi se  ljudi Izraelovi sabraše pred kraljem Salomonom na blagdan u mjesecu  Etanimu (to je sedmi mjesec). 
\par 3 I kad su došle Izraelove starješine, svećenici ponesoše Kovčeg 
\par 4 i Šator sastanka sa svim posvećenim  priborom što bješe u Šatoru. Prenosili su ih svećenici i leviti. 
\par 5 Kralj Salomon i sva zajednica Izraelova koja se sabrala oko  njega žrtvovali su pred Kovčegom toliko ovaca i goveda da se  ne mogahu prebrojiti ni procijeniti. 
\par 6 Svećenici donesoše Kovčeg  saveza Jahvina na njegovo mjesto, u Debir Doma, to jest u Svetinju  nad svetinjama, pod krila kerubina. 
\par 7 Kerubini su, naime, imali  raširena krila nad mjestom gdje stajaše Kovčeg i zaklanjahu odozgo  Kovčeg i njegove motke. 
\par 8 [8a] Motke su bile tako dugačke da su  im se krajevi vidjeli iz Svetišta nasuprot Debiru, ali se nisu  vidjele izvana. 
\par 9 U Kovčegu nije bilo ništa, osim dviju kamenih  ploča koje metnu Mojsije na Horebu, gdje Jahve sklopi Savez s  Izraelcima pošto iziđoše iz Egipta. [8b] Ondje su ostale do danas. 
\par 10 A kad su svećenici izašli iz Svetišta, oblak ispuni Dom  Jahvin, 
\par 11 i svećenici ne mogoše od oblaka nastaviti službe:  slava Jahvina ispuni Dom Božji! 
\par 12 Tada reče Salomon: "Jahve odluči prebivati u tmastu oblaku, 
\par 13 a ja ti sagradih uzvišen Dom da u njemu prebivaš zauvijek." 
\par 14 I, okrenuvši se, kralj blagoslovi sav izraelski zbor, a sav je izraelski zbor stajao. 
\par 15 Reče on: "Neka je blagoslovljen Jahve, Bog Izraelov, koji je svojom  rukom ispunio obećanje što ga na svoja usta dade ocu mome Davidu, rekavši: 
\par 16 'Od dana kad izvedoh svoj narod iz zemlje egipatske, nisam izabrao grada ni iz kojega Izraelova plemena da se u njemu  sagradi Dom gdje bi prebivalo moje Ime, nego sam izabrao Davida  da on zapovijeda mojim narodom Izraelom.' 
\par 17 Otac mi David naumi  podići Dom Imenu Jahve, Boga Izraelova, 
\par 18 ali mu Jahve reče:  'Naumio si podići Dom mojem Imenu, i dobro učini, 
\par 19 ali nećeš  ti podići toga Doma, nego sin tvoj koji izađe iz tvoga krila, on će podići Dom mojem Imenu.' 
\par 20 Jahve ispuni obećanje svoje:  naslijedio sam svoga oca Davida i sjeo na prijestolje Izraelovo, kako obeća Jahve, i podigao Dom Imenu Jahve, Boga Izraelova, 
\par 21 i odredio sam da ondje bude mjesto Kovčegu u kojem je Savez  što ga Jahve sklopi s našim ocima kad ih je izveo iz zemlje egipatske." 
\par 22 Tada Salomon stupi, u nazočnosti svega zbora Izraelova, pred žrtvenik Jahvin, raširi ruke prema nebu 
\par 23 i reče: "Jahve, Bože Izraelov! Nijedan ti bog nije sličan ni na nebesima  ni dolje na zemlji, tebi koji držiš Savez i ljubav svojim slugama  što kroče pred tobom sa svim svojim srcem. 
\par 24 Sluzi svome Davidu, mome ocu, ti si ispunio što si mu obećao. Što si obećao na svoja  usta, ispunio si svojom rukom upravo danas. 
\par 25 Sada, Jahve,  Bože Izraelov, ispuni svome sluzi, ocu mome Davidu, što si obećao  kad si rekao: 'Neće ti preda mnom nestati nasljednika koji bi  sjedio na izraelskom prijestolju, samo ako tvoji sinovi budu  čuvali svoje putove hodeći po mojem zakonu kako si ti hodio preda  mnom.' 
\par 26 Sada, dakle, Jahve, Bože Izraelov, neka se ispuni  tvoje obećanje koje si dao svome sluzi Davidu, mome ocu! 
\par 27 Ali  zar će Bog doista boraviti s ljudima na zemlji? TÓa nebesa ni  nebesa nad nebesima ne mogu ga obuhvatiti, a kamoli ovaj Dom  što sam ga sagradio! 
\par 28 Pomno počuj molitvu i vapaj svoga sluge, Jahve, Bože moj, te usliši vapaj i molitvu što je tvoj sluga  tebi upućuje! 
\par 29 Neka tvoje oči obdan i obnoć budu otvorene  nad ovim Domom, nad ovim mjestom za koje reče: 'Tu će biti moje  Ime.' Usliši molitvu koju će sluga tvoj izmoliti na ovome mjestu. 
\par 30 I usliši molitvu sluge svoga i naroda svojega izraelskog  koju bude upravljao prema ovome mjestu. Usliši s mjesta gdje  prebivaš, s nebesa; usliši i oprosti. 
\par 31 Ako tko zgriješi protiv  bližnjega i naredi mu se da se zakune, a zakletva dođe pred tvoj  žrtvenik u ovom Domu, 
\par 32 tada je ti čuj u nebu i postupaj i  sudi svojim slugama, osudi krivca okrećući njegova djela na njegovu  glavu, a nevina oslobodi postupajući s njime po nevinosti njegovoj. 
\par 33 Ako narod tvoj bude potučen od neprijatelja jer se ogriješio  o tebe, ali se ipak k tebi obrati i proslavi Ime tvoje i u ovom  se Domu pomoli, 
\par 34 onda ti čuj to s neba, oprosti grijehe svome  narodu izraelskom i dovedi ga natrag u zemlju koju si dao njihovim  očevima. 
\par 35 Kad se zatvori nebo i ne padne kiša jer su se ogriješili  o tebe, pa ti se pomole na ovome mjestu i proslave Ime tvoje  i obrate se od svojega grijeha kad ih ti poniziš, 
\par 36 tada ti  čuj na nebu i oprosti grijeh svojim slugama i svojem izraelskom  narodu, pokazujući im valjan put kojim će ići, i pusti kišu na  zemlju koju si svojem narodu dao u baštinu. 
\par 37 Kad u zemlji zavlada glad, kuga, snijet i rđa i kad navale  skakavci, gusjenice, kad neprijatelj ovoga naroda pritisne koja  od njegovih vrata, ili kad bude kakva druga nevolja ili boleština, 
\par 38 ako koji čovjek, ili sav tvoj narod, Izrael, osjeti tjeskobu  svoga srca pa upravi molitvu ili prošnju te raširi ruke prema  ovom Hramu, 
\par 39 ti čuj s neba, s mjesta gdje prebivaš, i oprosti  i postupi; vrati svakome čovjeku prema putu njegovu, jer ti poznaješ  srce njegovo - ti jedini poznaješ srce sviju - 
\par 40 da te se uvijek  boje sve dane dokle žive na zemlji što je ti dade našim očevima. 
\par 41 Pa i tuđinca, koji nije od tvojega naroda izraelskog, nego je stigao iz daleke zemlje radi Imena tvoga 
\par 42 jer je  čuo za veliko Ime tvoje, za tvoju snažnu ruku i za tvoju mišicu  podignutu - ako dođe i pomoli se u ovom Hramu, 
\par 43 usliši ga  s neba gdje prebivaš, usliši sve vapaje njegove da bi upoznali  svi zemaljski narodi Ime tvoje i bojali se tebe kao narod tvoj  Izrael i da znaju da je tvoje Ime zazvano nad ovaj Dom koji sam  sagradio. 
\par 44 Ako narod tvoj krene na neprijatelja putem kojim ga ti  uputiš i pomoli se Jahvi, okrenut k ovom gradu što si ga izabrao  i prema Domu koji sam podigao tvojem Imenu, 
\par 45 usliši mu s neba  molitvu i prošnju i učini mu pravdu. 
\par 46 Kad ti sagriješe, jer nema čovjeka koji ne griješi, a  ti ih, rasrdiv se na njih, predaš neprijateljima da ih zarobe  i odvedu kao roblje u daleku ili blizu zemlju, 
\par 47 pa ako se  pokaju srcem u zemlji u koju budu dovedeni te se obrate i počnu  te moliti za milost u zemlji svojih osvajača govoreći: 'Zgriješili  smo, bili smo zli i naopaki', 
\par 48 i tako se obrate k tebi svim  srcem i svom dušom u zemlji svoga ropstva u koju budu dovedeni  kao roblje, i pomole se okrenuti k zemlji što je ti dade njihovim  očevima, i prema gradu koji si izabrao, i prema Domu što sam  ga podigao tvom Imenu, 
\par 49 usliši s neba, gdje prebivaš, njihovu  molbu i njihove prošnje, 
\par 50 učini im pravdu i oprosti svome  narodu što je zgriješio protiv tebe, oprosti sve uvrede koje  ti je nanio, učini da mu se smiluju osvajači i da budu milostivi  prema njemu, 
\par 51 jer su oni tvoj narod i baština tvoja, njih  si izveo iz Egipta, iz užarenog kotla. 
\par 52 Neka oči tvoje budu otvorene na prošnju tvoga sluge i  na prošnju naroda tvoga Izraela da čuješ sve njihove molbe što  će ih tebi uputiti. 
\par 53 Jer ti si ih odvojio od svih naroda na  zemlji sebi za baštinu, kako si objavio po svome sluzi Mojsiju, kada si izveo oce naše iz Egipta, o Gospode, Jahve!" 
\par 54 Pošto je Salomon dovršio svu ovu molitvu i prošnju pred  Jahvom, diže se s mjesta gdje je klečao, raširenih ruku prema  nebu, pred žrtvenikom Jahvinim, 
\par 55 pa istupi te blagoslovi sav  zbor Izraelov govoreći jakim glasom: 
\par 56 "Blagoslovljen Jahve, koji je narodu svome Izraelu dao  mir u svemu kako je obećao; nije propalo nijedno od njegovih  lijepih obećanja koja je dao sluzi svome Mojsiju. 
\par 57 Neka Jahve, Bog naš, bude s nama kao što je bio s ocima našim i neka nas  ne napusti i ne odbaci. 
\par 58 Neka prikloni naša srca k sebi da  bismo hodili svim njegovim putovima i držali njegove zapovijedi, zakone i uredbe koje je dao ocima našim. 
\par 59 Bile ove moje riječi, koje sam smjerno iznio pred Jahvu, danju i noću nazočne pred  Jahvom, Bogom našim, eda bi dan za danom činio pravdu sluzi svomu  i pravicu narodu svome Izraelu, 
\par 60 ne bi li tako svi narodi  zemlje spoznali da je Jahve jedini Bog i da nema drugoga. 
\par 61 A  vaše srce neka bude potpuno odano Jahvi, Bogu našemu, držeći  se njegovih zakona i obdržavajući njegove zapovijedi kao danas!" 
\par 62 Kralj i sav Izrael s njim prinesu žrtvu Jahvi. 
\par 63 Kao  žrtvu pričesnicu, koju je prikazao Jahvi, Salomon prinese dvadeset  i dvije tisuće volova i stotinu i dvadeset tisuća ovaca; time  kralj i svi Izraelci posvete Dom Jahvin. 
\par 64 Toga dana posveti  kralj središte predvorja, koje je ispred Doma Jahvina, jer ondje  je prinio paljenice, prinosnice i pretiline pričesnica, jer je  tučani žrtvenik pred Jahvom bio premalen da primi paljenice,  prinosnice, pretiline pričesnica. 
\par 65 Tu je svečanost u ono vrijeme  Salomon slavio sedam dana, sa svim Izraelcima, zborom velikim  od Ulaza u Hamat do Potoka Egipatskog, pred Jahvom, Bogom našim. 
\par 66 Zatim je osmoga dana otpustio ljude; oni su blagosivljali  kralja i odlazili svojim kućama, veseli i zadovoljna srca zbog  svega dobra što ga je Jahve učinio svome sluzi Davidu i narodu  svome Izraelu. 


\chapter{9}

\par 1 Kad je Salomon dovršio gradnju Doma Jahvina, kraljevskog dvora  i svega što je namislio graditi, 
\par 2 javi se Jahve i drugi put  Salomonu, kao što mu se bio javio u Gibeonu. 
\par 3 Jahve mu reče: "Uslišio sam molitvu i prošnju koju si mi uputio. Posvetio  sam ovaj Dom, koji si sagradio da u njemu prebiva Ime moje dovijeka;  moje će oči i srce biti ovdje svagda. 
\par 4 A ti, ako budeš hodio  preda mnom kako je hodio tvoj otac David, u nevinosti srca i  pravednosti, postupao u svemu kako sam ti zapovjedio i ako budeš  držao moje zakone i moje naredbe, 
\par 5 ja ću učvrstiti zauvijek  tvoje kraljevsko prijestolje nad Izraelom, kako sam obećao tvome  ocu Davidu kad sam rekao: 'Nikada ti neće nestati nasljednika  na prijestolju Izraelovu.' 
\par 6 Ali ako me ostavite, vi i vaši  sinovi, ako ne budete držali mojih zapovijedi i zakona koje sam  vam dao, ako se okrenete bogovima i budete im služili i klanjali  im se, 
\par 7 tada ću istrijebiti Izraela iz zemlje koju sam mu dao;  ovaj ću Dom, koji sam posvetio svome Imenu, odbaciti od sebe, i Izrael će biti poruga i podsmijeh svim narodima. 
\par 8 Ovaj je  Dom uzvišen, ali svi koji budu uza nj prolazili bit će zaprepašteni;  zviždat će i govoriti: 'Zašto je Jahve tako učinio s ovom zemljom  i s ovim Domom?' 
\par 9 A reći će im se: 'Jer su ostavili Jahvu,  Boga svoga, koji je izveo oce njihove iz Egipta, a priklonili  se drugim bogovima, častili ih i služili im, zato je Jahve pustio  na njih sva ova zla.'" 
\par 10 Poslije dvadeset godina, za kojih je Salomon sagradio  obje zgrade, Dom Jahvin i kraljevski dvor, 
\par 11 a Hiram, kralj  Tira, dobavljao mu drvo cedrovo i čempresovo i zlata koliko je  god želio, dade tada kralj Salomon Hiramu dvadeset gradova u  zemlji galilejskoj. 
\par 12 Hiram izađe iz Tira da vidi gradove koje  mu je Salomon darovao, ali mu se nisu svidjeli. 
\par 13 I reče: "Kakvi  su to gradovi što si mi ih dao, brate?" I od tada ih zovu "zemlja  Kabul" do današnjega dana. 
\par 14 A Hiram bijaše poslao kralju stotinu  i dvadeset zlatnih talenata. 
\par 15 Ovako je bilo s rabotom koju je kralj Salomon digao da  sagradi Dom Jahvin, svoj dvor, Milo i zidove Jeruzalema, Hasor, Megido i Gezer. 
\par 16 Faraon, kralj Egipta, krenu u vojni pohod, osvoji Gezer, popali i poubija Kanaance koji su ondje živjeli, zatim dade grad u miraz svojoj kćeri, ženi Salomonovoj, 
\par 17 a  Salomon obnovi Gezer, Bet Horon Donji, 
\par 18 Baalat, Tamar u pustinji  u zemlji, 
\par 19 sve gradove-skladišta koje je Salomon imao, gradove  za bojna kola i gradove za konjicu, i sve što je Salomon želio  sagraditi u Jeruzalemu, na Libanonu i u svim zemljama koje su  mu bile podložne. 
\par 20 Svim preostalim Amorejcima, Hetitima, Perižanima, Hivijcima i Jebusejcima, koji nisu bili Izraelci, 
\par 21 sinovima  njihovim koji ostadoše iza njih u zemlji i koje Izraelci nisu  zatrli, Salomon nametnu tešku tlaku do današnjega dana. 
\par 22 Sinove  Izraelove nije Salomon pretvarao u robove, nego su mu oni bili  vojnici, dvorani, vojskovođe, tridesetnici, zapovjednici njegovih  bojnih kola i konjice. 
\par 23 A evo nadzornika koji su upravljali  Salomonovim radovima: njih pet stotina i pedeset koji su zapovijedali  puku zaposlenu na radovima. 
\par 24 Čim je faraonova kći ušla iz Davidova grada u kuću koju  joj Salomon bijaše sagradio, tada on podiže Milo. 
\par 25 Salomon je tri puta u godini prinosio paljenice i pričesnice  na žrtveniku koji je podigao Jahvi i palio je kad pred Jahvom.  Tako je dovršio Hram. 
\par 26 Kralj Salomon je sagradio brodovlje u Esjon-Geberu, koji  je kralj Elata, na obali Crvenoga mora, u zemlji edomskoj. 
\par 27 Hiram  je poslao na tim lađama svoje sluge, mornare koji su poznavali  more, sa slugama Salomonovim. 
\par 28 Oni otploviše u Ofir, uzeše  odande četiri stotine i dvadeset talenata zlata i donesoše ih  kralju Salomonu. 


\chapter{10}

\par 1 Glas koji je u Jahvinu Imenu stekao Salomon dopro je do kraljice  od Sabe; zato ona dođe da Salomona iskuša zagonetkama. 
\par 2 Došla  je u Jeruzalem s golemom pratnjom, s devama koje su nosile mirise, nebrojeno zlato i drago kamenje. Došavši k Salomonu, porazgovori  se s njim o svemu što joj bijaše na srcu. 
\par 3 Salomon joj odgovori  na sva pitanja; nije mu bilo skriveno ništa da joj ne bi umio  objasniti. 
\par 4 Kad kraljica od Sabe vidje mudrost Salomonovu, dvor koji  bijaše sagradio, 
\par 5 jela na njegovu stolu, odaje njegove i dvorane, otmjenost njegove posluge i njihova odijela, njegove peharnike  i paljenice koje je prinio u Domu Jahvinu, zastade joj dah. 
\par 6 Tada  reče kralju: "Istina je bila što sam u svojoj zemlji čula o tebi i o tvojoj  mudrosti. 
\par 7 Ali nisam htjela vjerovati što se pripovijeda dokle  god nisam došla i vidjela na svoje oči; i doista, ni pola mi  nije bilo rečeno: ti nadvisuješ u mudrosti i blagostanju slavu  o kojoj sam čula. 
\par 8 Blago tvojim ženama, blago ovim tvojim slugama  koji su neprestano pred tobom i slušaju tvoju mudrost! 
\par 9 Neka  je blagoslovljen Jahve, Bog tvoj, komu si tako omilio da te postavio  na prijestolje Izraelaca; zato što Jahve uvijek ljubi Izraela, postavio te kraljem da činiš pravo i pravicu." 
\par 10 Dade tada kralju stotinu i dvadeset zlatnih talenata, mnogo miomirisa i dragulja. Nikad više nije bilo takvih miomirisa  kakve je kraljica od Sabe dala kralju Salomonu. 
\par 11 Pa i Hiramovo  brodovlje, koje je donosilo zlato iz Ofira, dovezlo je odande  mnogo sandalovine i dragulja. 
\par 12 Kralj je od sandalovine napravio ograde za Dom Jahvin  i za kraljevski dvor, i citre i harfe za pjevače; nikada se više  nije dovezlo toliko sandalova drveta niti se vidjelo do danas. 
\par 13 Kralj Salomon dade kraljici od Sabe što je god zaželjela  i zatražila, a povrh toga kraljevski je obdari. Potom ona krenu  i sa slugama vrati se u svoju zemlju. 
\par 14 Zlato što je dolazilo Salomonu svake godine bilo je teško  šest stotina šezdeset i šest zlatnih talenata, 
\par 15 osim onoga  što je dolazilo od trgovaca i prodavača-potukača i od svih arapskih  kraljeva i upravitelja zemaljskih. 
\par 16 Kralj Salomon načini tri  stotine velikih štitova od kovanog zlata; za svaki je štit upotrijebio  šest stotina zlatnih šekela; 
\par 17 i načini trista štitića od kovanog  zlata; za svaki je štitić utrošio tri zlatne mine. Pohranio je  sve u kuću zvanu Libanonska šuma. 
\par 18 Kralj je još napravio veliko prijestolje od bjelokosti  i obložio ga čistim zlatom. 
\par 19 Prijestolje je imalo šest stepenica, straga je na njemu bila teleća glava, a s obje strane sjedala  bile su ručice, a kraj ručica stajala dva lava. 
\par 20 Dvanaest  je lavova stajalo s obje strane onih šest stepenica. Takvo što  nije bilo izrađeno ni u jednom kraljevstvu. 
\par 21 Sve posude iz kojih je pio kralj Salomon bijahu zlatne, i sve posuđe u kući zvanoj Libanonska šuma bijaše od suhoga  zlata; ništa nije bilo od srebra, jer se ono smatralo bezvrijednim  u Salomonovo vrijeme. 
\par 22 Kralj je imao taršiško brodovlje na  moru zajedno s Hiramovim brodovljem, i svake treće godine dolazilo  je taršiško brodovlje donoseći zlato, srebro i slonovu kost,  majmune i paune. 
\par 23 Tako je kralj Salomon natkrilio sve zemaljske kraljeve  bogatstvom i mudrošću. 
\par 24 Sav je svijet želio vidjeti Salomona  i čuti mudrost koju mu je Bog ulio u srce. 
\par 25 Svatko mu je donosio  dar: srebro i zlatno posuđe, haljine, oružje, miomirise, konje  i mazge, iz godine u godinu. 
\par 26 Uz to je Salomon sakupio bojnih kola i konjanika; imao  je tisuću i četiri stotine bojnih kola i dvanaest tisuća konja  i rasporedio ih je po gradovima bojnih kola i kod kralja u Jeruzalemu. 
\par 27 Salomon je učinio da u Jeruzalemu bude srebra kao kamenja, a cedrova kao divljih smokava što rastu u Šefeli. 
\par 28 Salomon  je uvozio konje iz Musrija i Koe: kraljevi nabavljači uvozili  su ih iz Koe za određenu svotu. 
\par 29 Kola se dovozila iz Egipta  po šest stotina srebrnih šekela; a konj se plaćao po stotinu  i pedeset. Tako ih preko nabavljača dobivahu svi kraljevi hetitski  i aramejski. 


\chapter{11}

\par 1 Kralj je Salomon - uz kćer faraonovu - volio mnoge žene tuđinke:  Moapke, Amonke, Edomke, Sidonke i Hetitkinje, 
\par 2 od svih naroda  za koje je Jahve rekao Izraelcima: "Nećete odlaziti k njima i  oni neće dolaziti k vama; oni će zacijelo okrenuti vaša srca  svojim bogovima." Njima se priklonio Salomon svojom ljubavlju. 
\par 3 Imao je sedam stotina kneževskih žena i tri stotine inoča.  Njegove su žene zavodile njegovo srce. 
\par 4 I kada je Salomon ostario, njegove su mu žene okrenule srce prema drugim bogovima, i srce  njegovo nije više potpuno pripadalo Jahvi kao što je pripadalo  srce njegova oca Davida. 
\par 5 Salomon je išao za Aštartom, boginjom  Sidonaca, i Milkomom, sramotom Amonaca. 
\par 6 Činio je ono što ne  bijaše pravo u očima Jahvinim i nije se sasvim pokoravao Jahvi  kao što se pokoravao njegov otac David. 
\par 7 Tako sagradi Salomon  uzvišicu Kemošu, sramoti Moaba, na gori istočno od Jeruzalema, i Milkomu, sramoti Amonaca. 
\par 8 To učini za sve svoje žene tuđinke, koje su prinosile kad i žrtve svojim bogovima. 
\par 9 Jahve se razgnjevi na Salomona jer je okrenuo srce svoje  od Jahve, Boga Izraelova, koji mu se bio dvaput javio 
\par 10 i koji  mu je baš tada zabranio štovati druge bogove, ali on nije održao  te zapovijedi. 
\par 11 Tada Jahve reče Salomonu: "Kada je tako s  tobom te ne držiš moga Saveza i naredaba koje sam ti dao, ja  ću sigurno oduzeti od tebe kraljevstvo i dat ću ga jednom od  tvojih slugu. 
\par 12 Ali neću to učiniti za tvoga života, zbog oca  tvojega Davida; uzet ću ga iz ruke tvoga sina. 
\par 13 Ipak neću  od njega uzeti svega kraljevstva: ostavit ću jedno pleme tvome  sinu, zbog sluge mojega Davida, zbog Jeruzalema koji izabrah." 
\par 14 Tada diže Jahve protivnika Salomonu: Edomca Hadada, iz  kraljevske kuće Edomaca. 
\par 15 Jer kada je David tukao Edomce i  kad je Joab, vojskovođa, otišao da pokopa ubijene i dao pogubiti  sve muškarce u Edomu - 
\par 16 Joab i sav Izrael ostadoše ondje šest  mjeseci dok nisu istrijebili sve muškarce u Edomu - 
\par 17 Hadad  je utekao u Egipat s Edomcima koji bijahu u službi njegova oca.  Hadad je bio tada mladi dječak. 
\par 18 Otišli su iz Midjana i stigli  u Paran. Poveli su sa sobom ljude iz Parana i otišli u Egipat  pred faraona, kralja Egipta, koji mu dade kuću, odredi mu hranu  i dodijeli zemlje. 
\par 19 Hadad je stekao veliku milost u faraona, koji mu dade sestru svoje žene, sestru velike kneginje Tafnese. 
\par 20 Sestra Tafnesina rodi mu sina Genubata, koga Tafnesa odgoji  u kraljevskoj palači, i Genubat je ostao u palači među faraonovom  djecom. 
\par 21 Kada je Hadad doznao u Egiptu da je David počinuo kod  svojih otaca i da je vojskovođa Joab umro, reče faraonu: "Dopusti  mi da odem u svoju zemlju!" 
\par 22 Faraon mu reče: "Što ti nedostaje  u mene te želiš poći u svoju zemlju?" A on mu odgovori: "Ništa, ali me pusti da odem!" [25b] To je bilo zlo što ga je učinio  Hadad: mrzio je Izraela i vladao je Edomom. 
\par 23 Bog je protiv Salomona podigao protivnika mu Rezona,  sina Elijadova, koji je pobjegao od svoga gospodara Hadadezera, kralja sopskoga; 
\par 24 Rezon je skupio ljude oko sebe i postao  im četovođa kada ih David ubijaše. Rezon zauze Damask, ondje  se nastani i zavlada Damaskom. 
\par 25 [25a] On je bio protivnik Izraelov  za života Salomonova. 
\par 26 Jeroboam bijaše sin Efraćanina Nebata, iz Sareda, a majka  mu bijaše udovica imenom Serva. On je bio u službi Salomonovoj  i podigao je ruku protiv kralja. 
\par 27 Evo razloga njegove pobune. Salomon je gradio Milo da  zatrpa kosinu u gradu Davida, oca svoga. 
\par 28 Taj Jeroboam bijaše  valjan čovjek; Salomon opazi kako se mladi čovjek prihvaća posla  i postavi ga nad svom rabotom kuće Josipove. 
\par 29 Tada se dogodi  te Jeroboam ode iz Jeruzalema, i na putu ga susrete prorok Ahija  iz Šila, ogrnut novim plaštem; bijahu sami njih dvojica u polju. 
\par 30 Ahija uze novi plašt koji je imao na sebi i razdrije ga na  dvanaest komada. 
\par 31 I reče Jeroboamu: "Uzmi sebi deset komada, jer ovako govori Jahve, Bog Izraelov:  'Evo ću istrgnuti kraljevstvo iz ruke Salomonove i dat ću tebi  deset plemena. 
\par 32 On će imati jedno pleme, zbog sluge mojega  Davida i Jeruzalema, grada koji sam izabrao između svih plemena  Izraelovih. 
\par 33 To je zato što me ostavio i poklonio se Aštarti, boginji Sidonaca, Kemošu, bogu moapskom, i Milkomu, bogu Amonaca, i ne hodi više mojim putovima; ne čini što je pravo u mojim  očima, niti izvršava moje zakone i naredbe kao što je činio njegov  otac David. 
\par 34 Ali mu neću iz ruke uzeti sve kraljevstvo, jer  sam ga postavio knezom za sveg njegova života, zbog sluge svoga  Davida, koga sam izabrao i koji je držao moje zapovijedi i moje  zakone. 
\par 35 Ali ću uzeti kraljevstvo iz ruke njegova sina i tebi  ću ga dati, to jest deset plemena. 
\par 36 A njegovu ću sinu ostaviti  jedno pleme da moj sluga David ima uvijek svjetiljku preda mnom  u Jeruzalemu, gradu koji sam izabrao sebi da u njemu stoluje  Ime moje. 
\par 37 Tebe ću uzeti da kraljuješ nad svim što budeš želio  i da budeš kralj nad Izraelom. 
\par 38 Ako budeš poslušao sve što  ti zapovjedim i stupao budeš putovima mojim te činio što je pravedno  u očima mojim držeći moje zakone i zapovijedi moje, kako je to  činio moj sluga David, tada ću ja biti s tobom i sagradit ću  ti trajan dom, kao što sam sagradio Davidu, i dat ću ti Izraela. 
\par 39 Ponizit ću potomke Davidove; ali ne zauvijek.'" 
\par 40 Salomon je zato tražio da ubije Jeroboama, ali on pobježe  u Egipat k Šišaku i ostade u Egiptu do smrti Salomonove. 
\par 41 Ostala povijest Salomonova, sve što je učinio i njegova  mudrost, zar nije zapisana u knjizi Povijesti Salomonove? 
\par 42 A  kraljevaše Salomon u Jeruzalemu nad svim Izraelom četrdeset godina. 
\par 43 Onda počinu Salomon kod otaca svojih i bi sahranjen u Davidovu  gradu, a njegov sin Roboam zakralji se namjesto njega. 


\chapter{12}

\par 1 Roboam ode u Šekem, jer su u Šekem došli svi Izraelci da ga  zakralje. 
\par 2 Čim to ču Nebatov sin Jeroboam - koji još bijaše  u Egiptu, kamo je pobjegao pred kraljem Salomonom - vrati se  iz Egipta, jer 
\par 3 bijahu poslali po nj i dozvali ga. Kad dođoše  Jeroboam i sav zbor Izraelov, rekoše Roboamu: 
\par 4 "Tvoj nam je otac nametnuo teški jaram. Ti nam sada olakšaj  tešku službu svoga oca, teški jaram koji metnu na nas, pa ćemo  ti služiti!" 
\par 5 A on im odgovori: "Za tri dana dođite opet k meni." I  narod ode. 
\par 6 Tada se kralj Roboam posavjetova sa starcima koji su služili  njegovu ocu Salomonu dok je bio živ i upita ih: "Što savjetujete  da odgovorim ovome narodu?" 
\par 7 Oni mu odgovoriše: "Ako danas udovoljiš tim ljudima, budeš  im blagonaklon i odgovoriš im lijepim riječima, oni će ti uvijek  ostati sluge." 
\par 8 Ali on odbaci savjet što mu ga dadoše starci i posavjetova  se s mladićima koji su odrasli s njim i bili mu u službi. 
\par 9 Upita  ih: "Što savjetujete da odgovorim ovome narodu koji mi reče:  'Olakšaj jaram što nam ga nametnu tvoj otac?'" 
\par 10 Mladići koji bijahu s njime odrasli odgovoriše mu: "Narodu  koji ti reče: 'Tvoj nam je otac nametnuo jaram, a ti nam ga olakšaj', uzvrati ovako: 'Moj je mali prst deblji od bedara moga oca! 
\par 11 Eto, moj vam je otac nametnuo teški jaram, a ja ću još otežati  vaš jaram; moj vas je otac šibao bičevima, a ja ću vas šibati  bičevima sa željeznim štipavcima.'" 
\par 12 A treći dan dođe sav narod k Roboamu, kako im bijaše  zapovjedio kralj rekavši im: "Vratite se k meni trećega dana." 
\par 13 Kralj im oštro odgovori, odbacivši savjet koji mu dadoše  stariji. 
\par 14 I reče im po savjetu mladih: "Moj je otac otežao  vaš jaram, a ja ću još dodati na nj; moj vas je otac šibao bičevima, a ja ću vas šibati bičevima sa željeznim štipavcima." 
\par 15 Kralj  dakle ne htjede poslušati naroda, jer tako upriliči Jahve da  se ispuni riječ što je preko Ahije iz Šila kaza Nebatovu sinu  Jeroboamu. 
\par 16 Kad Izraelci vidješe gdje se kralj oglušio, odgovori  mu narod: "Kakav dio mi imamo s Davidom? Mi nemamo baštine s Jišajevim sinom. U šatore, Izraele! A sad se, Davide, brini za svoj dom!" I sav Izrael ode pod svoje šatore. 
\par 17 Roboam zavlada samo nad Izraelovim sinovima koji su živjeli  po judejskim gradovima. 
\par 18 Potom kralj Roboam posla Adorama, nadstojnika za tlaku, ali ga Izraelci kamenovaše i on umrije;  a kralj se Roboam brže-bolje pope na kola i pobježe u Jeruzalem. 
\par 19 Tako se Izrael odijelio od doma Davidova sve do danas. 
\par 20 Kada su Izraelci doznali da se vratio Jeroboam, pozvaše  ga u zajednicu i postaviše ga kraljem nad svim Izraelom. Uz kuću  Davidovu nije pristajao nitko, osim samoga plemena Judina. 
\par 21 Došavši u Jeruzalem, Roboam skupi sav dom Judin i pleme  Benjaminovo, sto i osamdeset tisuća vrsnih ratnika, da udare  na dom Izraelov i da vrate kraljevstvo Roboamu, sinu Salomonovu. 
\par 22 Ali dođe Jahvina riječ Božjem čovjeku Šemaji: 
\par 23 "Kaži Salomonovu sinu Roboamu, judejskom kralju, i svem  domu Judinu i Benjaminovu i ostalom narodu: 
\par 24 Ovako veli Jahve:  'Ne idite se tući s braćom, djecom Izraelovom! Neka se svatko  vrati svojoj kući, jer je ovo poteklo od mene.'" I oni poslušaše riječ Jahvinu i vratiše se kako im reče Jahve. 
\par 25 Jeroboam utvrdi Šekem u Efrajimovoj gori i ondje se nastani.  Poslije izađe odatle i utvrdi Penuel. 
\par 26 Jeroboam reče u svom srcu: "Sad bi se kraljevstvo moglo  vratiti domu Davidovu. 
\par 27 Ako ovaj narod bude nastavio uzlaziti  u Dom Jahvin u Jeruzalemu da prinosi žrtve, srce će se naroda  vratiti svome gospodaru, Roboamu, kralju judejskome, i mene će  ubiti." 
\par 28 Pošto se kralj posavjetovao, načini dva zlatna teleta  i reče narodu: "Dosta ste uzlazili u Jeruzalem! Evo, Izraele, tvoga boga koji te izveo iz zemlje egipatske." 
\par 29 Zatim postavi  jedno tele u Betelu, a drugo smjesti u Dan. 
\par 30 To je bila prigoda  za grijeh: narod je odlazio jednome u Betel i drugome u Dan. 
\par 31 I podiže Jeroboam hram na uzvišicama i postavi iz puka  svećenike koji nisu bili sinovi Levijevi. 
\par 32 Zatim je Jeroboam  uveo blagdan u osmom mjesecu, petnaestoga dana tog mjeseca, kao  što je blagdan koji se slavi u Judeji, i uzađe k žrtveniku. Tako  je učinio u Betelu, žrtvujući teocima koje je načinio. U Betelu  je postavio i svećenike uzvišica što ih bijaše podigao. 
\par 33 I  uzađe k žrtveniku koji je načinio, petnaestoga dana osmog mjeseca, mjeseca koji je sam izabrao; i ustanovi blagdan za Izraelce  i uzađe k žrtveniku da prinese kad. 


\chapter{13}

\par 1 A neki čovjek Božji dođe na riječ Jahvinu iz Judeje u Betel  kada Jeroboam stajaše pred žrtvenikom da prinese kad. 
\par 2 I po  Jahvinoj zapovijedi povika onaj prema žrtveniku: "Žrtveniče,  žrtveniče! Ovako veli Jahve: 'Evo će se roditi u kući Davidovoj  sin po imenu Jošija. On će na tebi žrtvovati svećenike uzvišica, te koji na tebi prinose kad, i on će na tebi spaliti ljudske  kosti!'" 
\par 3 U isto im vrijeme dade znak: "Ovo je znak da je Jahve  govorio: gle, žrtvenik će se raspuknuti i prosut će se pepeo  što je na njemu." 
\par 4 Kada je kralj čuo što je čovjek Božji rekao protiv žrtvenika  u Betelu, pruži ruku odande od žrtvenika i reče: "Uhvatite ga!"  Ali se osušila ruka koju je ispružio prema čovjeku i nije je  mogao vratiti k sebi. 
\par 5 Žrtvenik se raspuknuo i pepeo se prosuo  sa žrtvenika, prema znaku što ga je dao čovjek Božji po naredbi  Jahvinoj. 
\par 6 Kralj progovori i reče čovjeku Božjem: "Umilostivi Jahvu, Boga svoga, da bih mogao vratiti ruku k sebi." Božji čovjek  umilostivi Jahvu i ruka se kraljeva vrati k njemu i bila je kao  prije. 
\par 7 Kralj onda reče čovjeku Božjem: "Hodi sa mnom kući da  se okrijepiš. I dat ću ti dar." 
\par 8 Ali čovjek Božji odgovori  kralju: "Da mi dadeš polovinu svoje kuće, ne bih pošao s tobom.  Ni jeo ni pio ne bih na ovom mjestu, 
\par 9 jer mi je ovako zapovjeđeno  riječju Jahvinom: 'Ne jedi kruha i ne pij vode, niti se vraćaj  istim putem kojim si došao.'" 
\par 10 I otišao je drugim putem, nije se vraćao putem kojim  je došao u Betel. 
\par 11 A u Betelu živio star prorok. Došli njemu njegovi sinovi  te mu pripovjedili sve što je onoga dana učinio čovjek Božji  u Betelu; i riječi što ih je onaj kazao kralju pripovjediše sinovi  ocu. 
\par 12 A on ih upita: "Kojim je putem otišao?" Sinovi pokazaše  put kojim je otišao čovjek Božji što bijaše došao iz Judeje. 
\par 13 Prorok će nato sinovima: "Osamarite mi magarca!" I osamariše  mu magarca, a on uzjaha. 
\par 14 Krenuo je za čovjekom Božjim i našao  ga gdje sjedi pod hrastom; i upita ga: "Jesi li ti čovjek Božji  koji je došao iz Judeje?" A on mu odgovori: "Jesam." 
\par 15 Prorok  mu reče: "Hodi sa mnom mome domu da štogod pojedeš." 
\par 16 Ali  on odgovori: "Ne smijem se vratiti s tobom, niti smijem jesti  kruha ni piti vode na ovome mjestu, 
\par 17 jer mi je Jahvinom riječju  naređeno ovo: 'Ne jedi ondje kruha, ne pij vode, niti se vraćaj  putem kojim si onamo pošao'." 
\par 18 Nato će mu onaj: "I ja sam  prorok kao i ti, i anđeo mi je riječju Jahvinom rekao: 'Povedi  ga sa sobom kući da jede kruha i pije vode.'" Slagao mu je. 
\par 19 Božji čovjek vrati se s njim, u njegovoj  je kući jeo kruha i pio vode. 
\par 20 Dok su sjedili za stolom, dođe riječ Jahvina proroku  koji ga je natrag doveo 
\par 21 i povika on čovjeku Božjem koji je  došao iz Judeje: "Ovako veli Jahve: zato što nisi poslušao zapovijedi  Jahvine i nisi držao naredbe koju ti je dao Jahve, Bog tvoj, 
\par 22 nego si se vratio, jeo kruha i pio vode na mjestu gdje sam  ti rekao da ne jedeš kruha i ne piješ vode, zato tijelo tvoje  neće leći u grob otaca tvojih." 
\par 23 Pošto se onaj koga bijaše doveo najeo kruha i napio vode, osedla mu magarca. 
\par 24 I ode onaj. A na putu ga zaskoči lav  i usmrti ga. I tako je mrtvo tijelo ležalo ispruženo na putu, magarac stajao kraj njega, a i lav stajaše kraj tijela. 
\par 25 Ljudi  prolazeći vidješe mrtvo tijelo ispruženo na putu i lava gdje  stoji kraj njega; i odoše i javiše to u gradu gdje je živio stari  prorok. 
\par 26 Kad je to čuo prorok koji bijaše onoga vratio s puta, reče: "To je čovjek Božji koji se usprotivio riječi Jahvinoj!  I Jahve ga je predao lavu, koji ga je napao i ubio, prema riječi  koju je Jahve rekao." 
\par 27 I reče svojim sinovima: "Osamarite  mi magarca!" I oni mu ga osamariše. 
\par 28 Ode on i nađe mrtvo tijelo  bačeno na putu i magarca i lava gdje stoje pokraj tijela: lav  nije požderao tijelo niti je rastrgao magarca. 
\par 29 Tada prorok  podiže mrtvo tijelo čovjeka Božjeg i prebaci ga na magarca; i  vrati se u grad gdje je živio da mrtvoga ožali i pokopa. 
\par 30 Položio  je mrtvo tijelo u svoju grobnicu i jecao je nad njim: "Jao, brate  moj!" 
\par 31 A kad ga je pokopao, reče svojim sinovima: "Poslije moje  smrti sahranite me u istu grobnicu gdje je pokopan čovjek Božji;  stavite moje kosti kraj njegovih. 
\par 32 Jer će se sigurno ispuniti  riječ koju je po zapovijedi Jahvinoj objavio protiv žrtvenika  u Betelu i protiv svih svetišta na uzvišicama u gradovima Samarije." 
\par 33 Ni poslije ovoga događaja ne obrati se Jeroboam za svoga  zlog puta, nego je i dalje priproste ljude postavljao za svećenike  na uzvišicama: tko je želio, davao mu je darove da postane svećenik  uzvišica. 
\par 34 Takvim je postupkom padala u grijeh kuća Jeroboamova, rušila se i nestajala s lica zemlje. 


\chapter{14}

\par 1 U ono se vrijeme razbolje Abija, sin Jeroboamov, 
\par 2 i Jeroboam  reče svojoj ženi: "Ustani i preobuci se da te ne bi prepoznali  da si žena Jeroboamova; i idi u Šilo. Ondje je prorok Ahija:  onaj koji mi je prorokovao da ću biti kraljem ovoga naroda. 
\par 3 I  ponesi deset hljebova, kolača i posudu meda i otiđi k njemu!  On će ti reći što će biti s dječakom." 
\par 4 I učini tako žena Jeroboamova: ustade, ode u Šilo i uđe  u kuću Ahijinu. A on nije više vidio, oslabile mu oči od duboke  starosti. 
\par 5 Ali mu je Jahve rekao: "Evo dolazi žena Jeroboamova  da od tebe traži savjeta za svoga sina jer je bolestan; a ti  ćeš joj reći tako i tako. Kad bude ulazila, pretvarat će se kao  da je druga." 
\par 6 Kad Ahija ču šum njenih koraka na vratima, reče joj: "Uđi, ženo Jeroboamova! Što se pretvaraš da si druga, kad imam tešku  vijest za tebe? 
\par 7 Idi, reci Jeroboamu: 'Ovako kaže Jahve, Bog  Izraelov: Podigao sam te isred naroda i učinio sam te knezom  nad mojim narodom Izraelom, 
\par 8 istrgnuo sam kraljevstvo iz kuće  Davidove i dao ga tebi. Ali ti nisi bio kao moj sluga David,  koji je držao moje zapovijedi i koji me slijedio svim srcem svojim  i činio samo ono što je pravedno u mojim očima. 
\par 9 Ti si radio  kudikamo gore od svojih prethodnika, otišao si i načinio sebi  druge bogove, salio si im likove da me dražiš, mene si bacio  za leđa. 
\par 10 Zato, evo, puštam zlo na kuću Jeroboamovu, istrijebit  ću iz obitelji Jeroboamove sve što mokri uza zid, robove i slobodnjake  u Izraelu; ja ću netragom pomesti kuću Jeroboamovu kao što se  mete nečist, da ga ništa ne ostane. 
\par 11 One koji iz Jeroboamove  obitelji umru u gradu, proždrijet će psi, a one koji umru u polju, pojest će ptice nebeske.' - Eto tako je Jahve rekao. 
\par 12 A ti  ustani i pođi svome domu: tek što nogama stupiš u grad, dječak  će umrijeti. 
\par 13 Sav će ga Izrael oplakati i pokopat će ga. On  će biti jedini iz obitelji Jeroboamove položen u grob, jer se  jedino na njemu našlo nešto što se u kući Jeroboamovoj svidjelo  Jahvi, Bogu Izraelovu. 
\par 14 Jahve će sebi postaviti kralja nad  Izraelom i taj će istrijebiti kuću Jeroboamovu. Evo dana! Što?  Čak i trenutka! 
\par 15 Jahve će udariti Izraela te će se njihati  kao trska u vodi. Iščupat će Izraela iz ove dobre zemlje koju  je dao njihovim ocima i rasijat će ih s onu stranu Rijeke, jer  su načinili sebi ašere koje srde Jahvu. 
\par 16 Odbacit će Izraela  kao smeće, zbog grijeha što ih je učinio Jeroboam i na koje je  navodio Izraela." 
\par 17 Žena Jeroboamova ustade i ode. Stigla je u Tirsu, a kad  je prelazila kućni prag, dječak bijaše mrtav. 
\par 18 Pokopali su  ga i sav ga je Izrael oplakao prema riječi koju je Jahve rekao  po sluzi svome proroku Ahiji. 
\par 19 Ostala povijest Jeroboamova, kako je ratovao i kraljevao, to je zapisano u knjizi Ljetopisa kraljeva izraelskih. 
\par 20 Jeroboamovo  kraljevanje trajalo je dvadeset i dvije godine, zatim je Jeroboam  počinuo kraj otaca svojih, a sin mu Nadab zakraljio se mjesto  njega. 
\par 21 Roboam, sin Salomonov, bio je kralj Judejaca; bijaše  mu četrdeset i jedna godina kad je postao kraljem, a sedamnaest  je godina kraljevao u Jeruzalemu, u gradu koji Jahve izabra između  svih izraelskih plemena da ondje postavi svoje Ime. Majka mu  se zvala Naama, a bila je Amonka. 
\par 22 I Juda učini zlo u očima  Jahvinim. Grijesima koje su počinili razjarili su ga više od  svega što su učinili njihovi oci. 
\par 23 Jer su i oni podigli uzvišice, stupove i ašere na svakom brežuljku i pod svakim zelenim drvetom. 
\par 24 Bilo je čak posvećenih bludnica u zemlji. Oponašao je sve  grozote naroda što ih je Jahve otjerao ispred sinova Izraelovih. 
\par 25 Pete godine Roboamova kraljevanja egipatski kralj Šišak  navali na Jeruzalem. 
\par 26 Opljačka sve blago iz Doma Jahvina i  riznicu kraljevskog dvora; sve je uzeo; uze i sve zlatne štitove  što ih bijaše napravio Salomon. 
\par 27 Namjesto njih kralj Roboam  napravi tučane štitove i povjeri ih zapovjednicima straže koja  je čuvala vrata kraljevskog dvora. 
\par 28 Kad je god kralj išao  u Jahvin Dom, stražari su ih uzimali, a poslije ih vraćali u  stražaru. 
\par 29 Ostala povijest Roboamova, sve što je učinio, zar nije  zapisano u knjizi Ljetopisa kraljeva judejskih? 
\par 30 Za sve vrijeme  bio je rat između Roboama i Jeroboama. 
\par 31 Roboam je počinuo  sa svojim ocima i bi sahranjen sa svojim ocima u Davidovu gradu.  Majka mu se zvala Naama, a bila je Amonka. Na njegovo se mjesto  zakraljio sin mu Abijam. 


\chapter{15}

\par 1 Osamnaeste godine kraljevanja Jeroboama, sina Nebatova, zakraljio  se Abijam u Judeji. 
\par 2 Tri je godine kraljevao u Jeruzalemu;  njegova se majka zvala Maaka, a bila je kći Abšalomova. 
\par 3 On  je hodio u svim grijesima što ih je njegov otac činio prije njega, i njegovo srce nije bilo potpuno odano Jahvi, Bogu svome, kao  srce njegova praoca Davida. 
\par 4 Ipak, zbog Davida, dao mu je Jahve, Bog njegov, svjetiljku u Jeruzalemu, podigavši sinove njegove  poslije njega i sačuvavši Jeruzalem. 
\par 5 Jer je David činio sve  što je pravo u očima Jahvinim i za svega svoga života nije odstupio  ni od čega što mu je zapovjedio, osim onog što je učinio Uriji  Hetitu. 
\par 6 # 
\par 7 Ostala povijest Abijamova, sve što je učinio, zar to nije  zapisano u knjizi Ljetopisa kraljeva judejskih? A bijaše rat  između Abijama i Jeroboama. 
\par 8 Potom je Abijam počinuo sa svojim  ocima. Sahraniše ga u Davidovu gradu; na njegovo se mjesto zakralji  sin mu Asa. 
\par 9 Dvadesete godine Jeroboamova kraljevanja nad Izraelom  postade Asa kraljem Judeje. 
\par 10 Kraljevao je četrdeset i jednu  godinu u Jeruzalemu; njegova se baka zvala Maaka, a bila je kći  Abšalomova. 
\par 11 Asa je činio što je pravo u očima Jahvinim, kao  i njegov praotac David. 
\par 12 Protjerao je iz zemlje posvećene  bludnice i uklonio sve idole koje njegovi oci bijahu načinili. 
\par 13 Sam je uklonio svoju baku s dostojanstva velike kneginje, jer bijaše načinila gada Ašeri. Asa je sasjekao njezina gada  i spalio ga u potoku Kidronu. 
\par 14 Ali uzvišice nisu bile uklonjene;  ipak je Asino srce bilo privrženo Jahvi svega njegova života. 
\par 15 Unio je u Dom Jahvin posvećene darove svoga oca i svoje:  srebro, zlato i posuđe. 
\par 16 Bio je rat između Ase i Baše, kralja izraelskoga, u sve  njihove dane. 
\par 17 Izraelski kralj Baša navali na Judeju i stade  utvrđivati Ramu da spriječi svako kretanje judejskom kralju Asi. 
\par 18 Asa tada uze srebra i zlata koje je preostalo u riznicama  Doma Jahvina i u riznicama kraljevskog dvora i dade ga svojim  slugama te ih posla Ben-Hadadu, sinu Tabrimonovu, sinu Hezjonovu, aramejskom kralju, koji je stolovao u Damasku, i poruči mu: 
\par 19 "Neka bude savez između mene i tebe, između moga i tvoga  oca; evo, šaljem ti na dar srebra i zlata: hajde, raskini savez  s izraelskim kraljem Bašom da bi otišao od mene." 
\par 20 Ben-Hadad posluša kralja Asu i posla svoje vojskovođe  na izraelske gradove te oni pokoriše Ijon, Dan, Abel Bet-Maaku, sav Kineret i svu zemlju Naftali. 
\par 21 A kada to Baša dozna,  presta utvrđivati Ramu i vrati se u Tirsu. 
\par 22 Kralj Asa sazva  sve Judejce, bez izuzetka, i oni odnesoše kamenje i drvo kojima  je Baša utvrđivao Ramu, i kralj Asa utvrdi time Gebu Benjaminovu  i Mispu. 
\par 23 Ostala povijest Asina, sve njegove pobjede i sve što  je učinio i gradovi koje je utvrdio, zar to nije zapisano u knjizi  Ljetopisa kraljeva judejskih? A u starosti bolovao je od nogu. 
\par 24 Asa je počinuo sa svojim ocima i sahranjen je sa svojim ocima  u gradu Davida, svoga praoca. Njegov sin Jošafat zakralji se  mjesto njega. 
\par 25 Nadab, sin Jeroboamov, postade kraljem Izraela druge  godine Asina kraljevanja Judejom i vladao je dvije godine Izraelom. 
\par 26 Činio je zlo u očima Jahvinim. Hodio je putem svoga oca i  oponašao njegov grijeh na koji je navodio Izraela. 
\par 27 Baša,  sin Ahijin, iz kuće Jisakarove, uroti se protiv njega i ubi ga  u Gibetonu, koji pripada Filistejcima i koji su opsjedali Nadab  i sav Izrael. 
\par 28 Baša ga ubi treće godine Asina kraljevanja  Judejom i zavlada mjesto njega. 
\par 29 Kad je postao kraljem, pobi  svu kuću Jeroboamovu i ne poštedi nikoga od Jeroboamovih dokle  sve ne istrijebi po riječi koju je Jahve rekao preko sluge svoga  Ahije iz Šila. 
\par 30 Zbog grijeha što ih je učinio i na koje je  naveo Izraela i zbog gnjeva kojim je raspalio Jahvu, Boga Izraelova. 
\par 31 Ostala povijest Nadabova, i sve što je učinio, zar to  nije zapisano u knjizi Ljetopisa kraljeva izraelskih? 
\par 32 Između  Ase i Izraelova kralja Baše vladao je rat u sve njihove dane. 
\par 33 Treće godine Asina kraljevanja Judejom postade Baša,  sin Ahijin, kraljem nad svim Izraelom u Tirsi i vladao je dvadeset  i četiri godine. 
\par 34 Činio je zlo u očima Jahvinim i hodio je  putem Jeroboama i njegovih grijeha kojima je zavodio Izraelce. 


\chapter{16}

\par 1 Tada bi upućena riječ Jahvina Jehuu, sinu Hananijevu, protiv  Baše: 
\par 2 "Iz praha sam te podigao i postavio knezom nad mojim  narodom Izraelom, ali si ti krenuo Jeroboamovim putem i navodiš  narod moj Izrael na grijehe te me razjaruješ njihovim grijesima; 
\par 3 zato ću netragom pomesti Bašu i kuću njegovu: učinit ću s  tvojom kućom kao i s kućom Jeroboama, sina Nebatova. 
\par 4 Tko iz  obitelji Bašine umre u gradu, pojest će ga psi, a tko im umre  u polju, pojest će ga ptice nebeske." 
\par 5 Ostala povijest Bašina, što je učinio, njegova djela,  zar sve to nije zapisano u knjizi Ljetopisa kraljeva izraelskih? 
\par 6 Baša je počinuo sa svojim ocima i sahranjen je u Tirsi. Sin  njegov Ela zakraljio se mjesto njega. 
\par 7 Ali riječ Jahvina po Jehuu proroku, sinu Hananijevu, nije  bila upravljena protiv Baše i njegove kuće samo zbog toga što  je činio zlo u očima Jahve i ljutio ga djelima svojih ruku te  bio kao i kuća Jeroboamova nego i zbog toga što je i nju istrijebio. 
\par 8 Dvadeset i šeste godine kraljevanja Ase u Judeji postade  Ela, sin Bašin, kraljem Izraela u Tirsi; vladao je svega dvije  godine. 
\par 9 Njegov dvoranin Zimri, zapovjednik polovine bojnih  kola, uroti se protiv njega. Kad je bio u Tirsi, opio se u kući  Arse, upravitelja dvora u Tirsi. 
\par 10 Tada provali Zimri, udari  na njega i ubi ga, dvadeset i sedme godine Asina kraljevanja  Judejom, te zavlada mjesto njega. 
\par 11 Čim je zavladao i sjeo  na prijestolje, poubija svu obitelj Bašinu; nije mu poštedio  ni što uza zid mokri, ni njegovih rođaka ni prijatelja. 
\par 12 Tako  Zimri iskorijeni svu kuću Bašinu po riječi koju je Jahve rekao  protiv Baše preko sluge svoga proroka Jehua, 
\par 13 zbog sviju grijeha  što su ih činili Baša i sin mu Ela i tako zavodili Izraela, srdeći  Jahvu, Boga Izraelova, svojim krivim bogovima. 
\par 14 Ostala povijest Elina, sve što je učinio, zar to nije  zapisano u knjizi Ljetopisa kraljeva izraelskih? 
\par 15 Dvadeset i sedme godine Asina kraljevanja Judejom postade  Zimri kraljem u Tirsi i vladao je sedam dana. Narod je tada opsjedao  Gibeton, koji je pripadao Filistejcima. 
\par 16 Kad je utaboreni  narod čuo da se Zimri pobunio i ubio kralja, sav Izrael istoga  dana u taboru proglasi kraljem nad Izraelom zapovjednika vojske  Omrija. 
\par 17 Zatim Omri i sav Izrael s njime odoše od Gibetona  i opsjedoše Tirsu. 
\par 18 Kad je Zimri vidio da će grad biti osvojen, uđe u utvrdu kraljevskoga dvora, zapali nad sobom kraljevski  dvor i tako pogibe. 
\par 19 To je bilo zbog grijeha koje je počinio  radeći što je zlo u očima Jahvinim i hodeći putem Jeroboama i  njegovih grijeha kojima je zavodio Izraela. 
\par 20 Ostala povijest Zimrijeva i njegova urota koju je skovao, zar sve to nije zapisano u knjizi Ljetopisa kraljeva izraelskih? 
\par 21 Tada se Izraelov narod razdijelio: polovica se odlučila  za Tibnija, sina Ginatova, da ga učini kraljem, a druga polovica  za Omrija. 
\par 22 Ali pristaše Omrijeve nadjačaše pristaše Tibnija, sina Ginatova, pa kad Tibni umrije, postade Omri kraljem. 
\par 23 Trideset i prve godine Asina kraljevanja Judejom postade  Omri kraljem Izraela za dvanaest godina. U Tirsi je kraljevao  šest godina. 
\par 24 Tada kupi od Šemera za dva talenta srebra brdo  Samariju; sagradi grad koji po imenu Šemera, vlasnika brijega, nazva Samarija. 
\par 25 Ali je Omri činio zlo u očima Jahvinim i  bio je gori od svojih prethodnika. 
\par 26 U svemu je slijedio Jeroboama, sina Nebatova, i njegove grijehe kojima je zavodio Izraela i  srdio Jahvu, Boga Izraelova, svojim lažnim bogovima. 
\par 27 Ostala povijest Omrijeva, sve što je učinio, njegovi  pothvati koje je izveo, zar to nije zapisano u knjizi Ljetopisa  kraljeva izraelskih? 
\par 28 Omri počinu sa svojim ocima i sahranjen  je u Samariji. Njegov sin Ahab postade kraljem mjesto njega. 
\par 29 Ahab, sin Omrijev, postade izraelskim kraljem u trideset  i osmoj godini Asina kraljevanja Judejom i vladao je dvadeset  i dvije godine nad Izraelom u Samariji. 
\par 30 Ahab, sin Omrijev, činio je u očima Jahvinim više zla od svih svojih prethodnika. 
\par 31 I malo mu bijaše što je hodio u grijesima Jeroboama, sina  Nebatova, nego se još oženi Izebelom, kćerju Etbaala, kralja  sidonskog, i poče služiti Baalu i klanjati mu se; 
\par 32 Baalu podiže  žrtvenik u Baalovu hramu što ga bijaše sagradio u Samariji. 
\par 33 Ahab  je podigao i Ašeru i učinio druga zlodjela i razljutio Jahvu, Boga Izraelova, više od svih kraljeva izraelskih koji bijahu  prije njega. 
\par 34 Za njegova je vremena Hiel iz Betela sagradio Jerihon;  uz žrtvu svoga prvorođenca Abirama podigao je temelje, a uz žrtvu  svoga mezimca Seguba postavio je gradska vrata, po riječi koju  je Jahve rekao po svome sluzi Jošui, sinu Nunovu. 


\chapter{17}

\par 1 Ilija Tišbijac, iz Tišbe Gileadske, reče Ahabu: "Živoga mi  Jahve, Boga Izraelova, komu služim, neće ovih godina biti ni  rose ni kiše, osim na moju zapovijed." 
\par 2 Upućena mu je riječ Jahvina ovako: 
\par 3 "Idi odavde i kreni  na istok i sakrij se na potoku Keritu, koji je nasuprot Jordanu. 
\par 4 Pit ćeš iz potoka, a gavranima sam zapovjedio da te ondje  hrane." 
\par 5 Ode on i učini po riječi Jahvinoj i nastani se na  potoku Keritu, nasuprot Jordanu. 
\par 6 Gavrani su mu jutrom donosili  kruha, a večerom mesa; iz potoka je pio. 
\par 7 Ali poslije nekog vremena presuši potok, jer nije bilo  kiše u svoj zemlji. 
\par 8 Tada Iliji dođe riječ Jahvina: 
\par 9 "Ustani, idi u Sarfatu Sidonsku i ondje ostani. Evo, ondje sam zapovjedio  jednoj udovici da te hrani." 
\par 10 Ustade on i krenu u Sarfatu. Kada je stigao do gradskih  vrata, neka je udovica onuda skupljala drva; on joj se obrati  i reče: "Donesi mi malo vode u vrču da pijem!" 
\par 11 Kad je pošla  da donese, on viknu za njom i reče joj: "Donesi mi i malo kruha  u ruci!" 
\par 12 Ona odgovori: "Živoga mi Jahve, tvoga Boga, ja nemam  pečena kruha, nemam do pregršti brašna u ćupu i malo ulja u vrču.  I evo kupim drva, pa ću otići i ono pripremiti sebi i svome sinu  da pojedemo i da umremo." 
\par 13 Ali joj Ilija reče: "Ništa se ne boj. Idi i uradi kako  si rekla; samo najprije umijesi meni kolačić, pa mi donesi; a  onda zgotovi za sebe i za svoga sina. 
\par 14 Jer ovako govori Jahve, Bog Izraelov: 'U ćupu neće brašna nestati ni vrč se s uljem neće isprazniti sve dokle Jahve ne pusti da kiša padne na zemlju.'" 
\par 15 Ode ona i učini kako je rekao Ilija; i za mnoge dane  imadoše jela, ona, on i njen sin. 
\par 16 Brašno se iz ćupa nije  potrošilo i u vrču nije nestalo ulja, po riječi koju je Jahve  rekao preko svoga sluge Ilije. 
\par 17 Poslije ovih događaja razbolio se sin domaćičin i bolest  se njegova jako pogoršala, tako te u njemu nije ostalo daha. 
\par 18 Tada ona reče Iliji: "Što ja imam s tobom, čovječe Božji?  Zar si došao k meni da me podsjetiš na moj grijeh i da mi usmrtiš  sina!" 
\par 19 On joj reče: "Daj mi svoga sina!" Tada ga uze iz njezina  naručja, odnese ga u gornju sobu gdje je stanovao i položi ga  na svoju postelju. 
\par 20 Tada zavapi Jahvi i reče: "Jahve, Bože  moj, zar zaista želiš udovicu koja me ugostila uvaliti u tugu  umorivši joj sina?" 
\par 21 Zatim se tri puta pružio nad dječakom  zazivajući Jahvu: "Jahve, Bože, učini da se u ovo dijete vrati  duša njegova!" 
\par 22 Jahve je uslišio molbu Ilijinu, u dijete se  vratila duša i ono oživje. 
\par 23 Ilija ga uze, siđe iz gornje sobe  u kuću i dade ga njegovoj materi; i reče Ilija: "Evo, tvoj sin  živi!" 
\par 24 Žena mu reče: "Sada znam da si ti čovjek Božji i da  je riječ Jahvina u tvojim ustima istinita!" 


\chapter{18}

\par 1 Prošlo je mnogo vremena i riječ Jahvina bi upravljena treće  godine Iliji: "Idi, pokaži se Ahabu, jer želim pustiti kišu na  lice zemlje." 
\par 2 I ode Ilija da se pokaže Ahabu. Kako je glad u Samariji bivala teža, 
\par 3 pozva Ahab dvorskog  upravitelja Obadiju. Taj se Obadija veoma bojao Jahve; 
\par 4 jer  kad je Izebela poubijala proroke Jahvine, on je uzeo stotinu  proroka i sakrio ih po pedeset u jednu spilju, gdje ih je hranio  kruhom i pojio vodom. 
\par 5 I reče Ahab Obadiji: "Hajde, obići ćemo  svu zemlju, sve izvore i sve potoke, možda ćemo naći trave da  sačuvamo u životu konje i mazge i da nam ne propadne stoka." 
\par 6 Podijelili su zemlju koju će pretražiti: Ahab je sam otišao  jednim putem, a Obadija je pošao sam drugim putem. 
\par 7 I kad je Obadija bio na putu, eto mu u susret Ilije; poznavši  ga, pade ničice i reče: "Jesi li to ti, gospodaru Ilija!" 
\par 8 On  mu odgovori: "Ja sam! Idi i reci svome gospodaru: 'Evo Ilije!'" 
\par 9 Odgovori mu Obadija: "Što sam sagriješio te slugu svojega  predaješ u ruke Ahabu da me ubije? 
\par 10 Živoga mi Jahve, tvoga  Boga, nema naroda ili kraljevstva kamo moj gospodar nije slao  da te traže. I kad su mu rekli: 'Nema ga!' zakleo je kraljevstvo  i narod što te nisu našli. 
\par 11 I sada mi naređuješ: 'Idi, reci  svome gospodaru: Evo Ilije!' 
\par 12 Ali kad ja odem od tebe, Duh  Jahvin odnijet će te ne znam kamo, a ja ću doći i obavijestiti  Ahaba. Pa kad te ne nađe, ubit će me! A tvoj se sluga boji Jahve  od mladosti svoje! 
\par 13 Zar nije poznato mome gospodaru što sam  učinio kad je ono Izebela poubijala proroke Jahvine? Sakrio sam  stotinu proroka, po pedeset u jednu spilju, i kruhom ih uzdržavao  i vodom. 
\par 14 I sada ti naređuješ: 'Idi, reci svome gospodaru:  Evo Ilije!' Pa on će me ubiti!" 
\par 15 Ilija mu odgovori: "Živoga mi Jahve Sebaota, komu služim, još ću mu se danas pokazati." 
\par 16 Obadija pođe u susret Ahabu i donese mu vijest, a Ahab  pođe u susret Iliji. 
\par 17 Kad Ahab ugleda Iliju, reče mu: "Jesi  li ti onaj koji upropašćuješ Izraela?" 
\par 18 Ilija odgovori: "Ne  upropašćujem ja Izraela, nego ti i tvoja obitelj, jer ste ostavili  Jahvu, a ti si sljedbenik Baala. 
\par 19 Sada sakupi sav Izrael preda  me na gori Karmelu i četiri stotine pedeset proroka Baalovih  koji jedu za stolom Izebelinim." 
\par 20 Ahab pozva sve sinove Izraelove i sakupi proroke na gori  Karmelu. 
\par 21 Ilija pristupi svemu narodu i reče: "Dokle ćete  hramati na obje strane? Ako je Jahve Bog, slijedite ga; ako je  Baal, slijedite njega." A narod mu nije ništa odgovorio. 
\par 22 Ilija  nastavi: "Ja sam još jedini ostao kao prorok Jahvin, a Baalovih  je proroka četiri stotine i pedeset. 
\par 23 Dajte nam dva junca.  Neka oni izaberu sebi jednoga, neka ga sasijeku i stave na drva, ali neka ne podmeću ognja. Ja ću spremiti drugoga junca i neću  podmetati ognja. 
\par 24 Vi zazovite ime svoga boga, a ja ću zazvati  ime Jahvino: bog koji odgovori ognjem pravi je Bog." Sav narod  odgovori: "Dobro!" 
\par 25 Potom reče Ilija prorocima Baalovim: "Izaberite sebi  jednoga junca i počnite, jer vas je mnogo više. Zazovite ime  svoga boga, ali ne stavljajte ognja." 
\par 26 Oni uzeše junca koji je njima pripao i pripremiše ga.  Zazivali su ime Baalovo od jutra do podne govoreći: "O Baale, usliši nas!" Ali nije bilo ni glasa, ni odgovora. I skakahu  i prigibahu koljena pred žrtvenikom koji su načinili. 
\par 27 U podne im se Ilija naruga i reče: "Glasnije vičite,  jer on je bog; zauzet je, ili ima posla, ili je na putu; možda  spava, pa ga treba probuditi!" 
\par 28 A oni okrenuše vikati još  glasnije i parati se noževima i sulicama, kako je u njih običaj, sve dok ih nije oblila krv. 
\par 29 Kad je prošlo podne, pali su  u bunilo i bjesnjeli sve dok nije bilo vrijeme da se prinese  žrtva; ali nije bilo nikakva glasa ni odgovora niti znaka da  ih tkogod sluša. 
\par 30 Tada Ilija reče svemu narodu: "Priđite k meni!" I sav  mu narod pristupi. On popravi žrtvenik Jahvin koji bijaše srušen. 
\par 31 Ilija uze dvanaest kamenova prema broju plemena sinova Jakova, kome je Bog rekao: "Izrael će biti ime tvoje!" 
\par 32 I sagradi  od toga kamenja žrtvenik Imenu Jahvinu i iskopa jarak oko žrtvenika, širok da bi se mogle posijati dvije mjere pšenice. 
\par 33 Složi  drva, rasiječe junca i stavi ga na drva. 
\par 34 Tada reče: "Napunite  vodom četiri vrča i izlijte na paljenicu i na drva!" Učiniše  tako. Zapovjedi im: "Ponovite", i oni ponoviše. Tada reče: "Učinite  i treći put." Oni tako i treći put. 
\par 35 Voda je tekla oko žrtvenika  i jarak se ispunio vodom. 
\par 36 Kad bijaše vrijeme da se prinese žrtva, pristupi prorok  Ilija i reče: "Jahve, Bože Abrahamov, Izakov i Izraelov, objavi  danas da si ti Bog u Izraelu, da sam ja sluga tvoj i da sam po  zapovijedi tvojoj učinio sve ovo. 
\par 37 Usliši me, Jahve; usliši  me, da bi sav ovaj narod znao da si ti, Jahve, Bog i da ćeš ti  obratiti njihova srca." 
\par 38 I oganj Jahvin pade i proguta paljenicu i drva, kamenje  i prašinu, čak i vodu u jarku isuši. 
\par 39 Sav narod se uplaši, ljudi padoše ničice i rekoše: "Jahve je Bog! Jahve je Bog!" 
\par 40 Ilija im reče: "Pohvatajte proroke Baalove da nijedan od  njih ne utekne!" I oni ih pohvataše. Ilija ih odvede do potoka  Kišona i ondje ih pobi. 
\par 41 Ilija reče Ahabu: "Idi gore, jedi i pij, jer čujem šumor  kiše." 
\par 42 Dok je Ahab otišao gore da jede i pije, Ilija se popeo  na vrh Karmela, prignuo se zemlji i sakrio lice među koljena. 
\par 43 Rekao je zatim svome momku: "Idi gore i pogledaj prema moru."  On ode gore, pogleda i reče: "Ništa nema ondje!" Ilija odgovori:  "Vrati se sedam puta." 
\par 44 Ali sedmoga puta reče momak: "Eno  se oblak, malen kao dlan čovječji, diže od mora." Tada reče Ilija:  "Idi, kaži Ahabu: 'Upregni i silazi da te kiša ne uhvati.'" 
\par 45 Odjednom se nebo zamrači od oblaka i vihora i pade jaka  kiša. Ahab se pope na kola i odveze u Jizreel. 
\par 46 Ruka je Jahvina  bila nad Ilijom te on, opasavši se, otrča pred Ahabom sve do  u blizinu Jizreela. 


\chapter{19}

\par 1 Ahab ispriča Izebeli sve što je Ilija učinio i kako je mačem  poubijao sve proroke. 
\par 2 Tada Izebela posla Iliji glasnika s  porukom: "Neka mi bogovi učine sva zla i neka nadodadu, ako sutra  u ovo doba ne učinim s tvojim životom kao što si ti učinio sa  životom svakoga od njih!" 
\par 3 On se uplaši, ustade i ode da bi spasio život. Došao je  u Beer Šebu, koja je u Judeji, i otpustio ondje svoga momka. 
\par 4 A sam ode dan hoda u pustinju; sjede ondje pod smreku, zaželje  umrijeti i reče: "Već mi je svega dosta, Jahve! Uzmi dušu moju, jer nisam bolji od otaca svojih." 
\par 5 Zatim leže i zaspa. Ali  gle, anđeo ga taknu i reče mu: "Ustani i jedi." 
\par 6 On pogleda, kad gle - kraj njegova uzglavlja na kamenu  pečen kruh i vrč vode. Jeo je i pio, pa opet legao. 
\par 7 Ali se  anđeo Jahvin javi i drugi put, dotače ga i reče: "Ustani i jedi, jer je pred tobom dalek put!" 
\par 8 Ustao je, jeo i pio. Okrijepljen  tom hranom, išao je četrdeset dana i četrdeset noći sve do Božje  gore Horeba. 
\par 9 Ondje je ušao u neku spilju i prenoćio u njoj. I gle,  eto k njemu riječi Jahvine: "Što ćeš ti ovdje, Ilija?" 
\par 10 On  odgovori: "Revnovao sam gorljivo za Jahvu, Boga nad vojskama, jer su sinovi Izraelovi napustili tvoj Savez, srušili tvoje  žrtvenike i pobili mačem tvoje proroke. Ostao sam sam, a oni  traže da i meni uzmu život." 
\par 11 Glas mu reče: "Iziđi i stani  u gori pred Jahvom. Evo Jahve upravo prolazi." Pred Jahvom je  bio silan vihor, tako snažan da je drobio brda i lomio hridi, ali Jahve nije bio u olujnom vihoru; poslije olujnog vihora  bio je potres, ali Jahve nije bio u potresu; 
\par 12 a poslije potresa  bio je oganj, ali Jahve nije bio u ognju; poslije ognja šapat  laganog i blagog lahora. 
\par 13 Kad je to čuo Ilija, zakri lice  plaštem, iziđe i stade na ulazu u pećinu. Tada mu progovori glas  i reče: "Što ćeš ovdje, Ilija?" 
\par 14 On odgovori: "Revnovao sam  veoma gorljivo za Jahvu nad vojskama, jer su sinovi Izraelovi  napustili tvoj Savez, srušili tvoje žrtvenike i mačem poubijali  tvoje proroke. Ostadoh sam, a oni traže da i meni oduzmu život." 
\par 15 Jahve mu reče: "Idi, vrati se istim putem u damaščansku  pustinju. Kad dođeš, pomaži ondje Hazaela za kralja aramskog. 
\par 16 Pomaži Jehuu, sina Nimsijeva, za kralja izraelskoga i pomaži  Elizeja, sina Šafatova, iz Abel Mehole, za proroka namjesto sebe. 
\par 17 Koji utekne od mača Hazaelova, njega će pogubiti Jehu; a  tko utekne od Jehuova mača, njega će pogubiti Elizej. 
\par 18 Ali  ću ostaviti u Izraelu sedam tisuća, sve koljena koja se nisu  savila pred Baalom i sva usta koja ga nisu cjelivala." 
\par 19 Ode on i na povratku naiđe na Elizeja, sina Šafatova, gdje ore: pred njim dvanaest jarmova, sam bijaše kod dvanaestoga.  Ilija prođe kraj njega i baci na nj svoj plašt. 
\par 20 On ostavi  volove, potrča za Ilijom i reče: "Dopusti mi da zagrlim svoga  oca i majku, pa ću poći za tobom." Ilija mu odgovori: "Idi, vrati  se, jer što sam ti učinio?" 
\par 21 On ga ostavi, uze jaram volova  i žrtvova ih. Volujskim jarmom skuha meso i dade ga ljudima da  jedu. Zatim ustade i pođe za Ilijom da ga poslužuje. 


\chapter{20}

\par 1 Ben-Hadad, kralj Arama, skupi svu vojsku svoju - s njim bijahu  trideset i dva kralja, s konjima i bojnim kolima - i ode opsjedati  Samariju i udari na nju. 
\par 2 Posla u grad glasnike izraelskom  kralju Ahabu 
\par 3 i reče mu: "Ovako veli Ben-Hadad: 'Tvoje srebro  i tvoje zlato moje je, a žene tvoje i djeca ostaju tebi.'" 
\par 4 Izraelski  kralj ovako mu odgovori: "Na tvoju zapovijed, gospodaru kralju!  Tvoj sam ja sa svime što mi pripada." 
\par 5 Ali se glasnici vratiše i rekoše: "Ovako kaže Ben-Hadad  i poručuje ti: 'Daj mi svoje srebro i zlato, svoje žene i djecu. 
\par 6 Budi siguran da ću sutra u ovo doba poslati svoje sluge i  oni će pretražiti tvoju kuću i kuće tvojih sluga i stavit će  svoju ruku na sve što im se svidi i to će odnijeti.'" 
\par 7 Izraelski kralj sazva sve starješine zemaljske i reče:  "Promislite i pogledajte! Ovaj nam sprema zlo! Traži od mene  moje žene i djecu, premda mu nisam odbio svoje srebro i zlato." 
\par 8 Starješine mu i sav narod odgovoriše: "Nemoj poslušati! Nemoj  pristati!" 
\par 9 Tada on ovako odgovori Ben-Hadadovim poslanicima:  "Recite gospodaru kralju: 'Sve što si prvi put tražio od svoga  sluge, ja ću učiniti, ali ovo drugo ne mogu.'" I poslanici odoše  i odnesoše odgovor. 
\par 10 Tada mu Ben-Hadad poruči: "Neka mi bogovi učine zlo i  neka pridaju još toliko, ako bude dosta praha Samarije da svi  oni koji me slijede dobiju po pregršt!" 
\par 11 Ali mu kralj izraelski  odgovori: "Kaže se: 'Neka se ne hvali koji se opasuje kao onaj  koji se raspasuje!'" 
\par 12 A kad je Ben-Hadad to čuo - upravo je  pio s kraljevima pod šatorima - zapovjedi svojim slugama: "Na  svoja mjesta!" I oni zauzeše svoje položaje protiv grada. 
\par 13 Tada potraži jedan prorok Ahaba, kralja Izraela, i reče:  "Ovako veli Jahve: 'Jesi li vidio ono silno mnoštvo? Ja ću ti  ga danas evo predati u ruke i ti ćeš spoznati da sam ja Jahve.'" 
\par 14 Ahab reče: "Po kome?" On odgovori: "Ovako veli Jahve: po  momcima pokrajinskih namjesnika." Ahab upita: "Tko će početi  boj?" On odgovori: "Ti!" 
\par 15 Ahab izvrši smotru momaka pokrajinskih upravitelja. Bijaše  ih dvije stotine trideset i dva. Poslije njih izvršio je smotru  sve vojske svih Izraelaca. Bijaše ih sedam tisuća. 
\par 16 Oni iziđoše  u podne, dok je Ben-Hadad pio u šatorima sa trideset i dva kralja  koji mu bijahu saveznici. 
\par 17 Momci pokrajinskih upravitelja  iziđoše prvi. Obavijestiše Ben-Hadada: "Izišli su ljudi iz Samarije." 
\par 18 On  reče: "Ako su izišli radi mira, pohvatajte ih žive; ako su izišli  u boj, opet ih uhvatite žive!" 
\par 19 Ali kad su oni - momci pokrajinskih upravitelja - izišli  iz grada, za njima je slijedila ostala vojska 
\par 20 i svaki je  udario na svog protivnika. Aramejci su bježali, a Izraelci ih  progonili. Ben-Hadad, aramejski kralj, spasio se na konju zajedno  s nekim konjanicima. 
\par 21 Tada je izišao izraelski kralj; zarobio  je konje i kola i nanio Aramejcima težak poraz. 
\par 22 Tada pristupi prorok izraelskom kralju i reče mu: "Hajdemo!  Ohrabri se i razmisli dobro što ti je činiti, jer će dogodine  aramejski kralj napasti na te." 
\par 23 Sluge su savjetovale aramejskog  kralja: Njihov bog je bog gora, i zato su bili jači od nas. Ali  ako se pobijemo s njima u ravnici, sigurno ćemo mi biti jači  od njih. 
\par 24 Učinimo dakle ovako: makni ove kraljeve i postavi  na njihovo mjesto upravitelje. 
\par 25 Zatim skupi sebi veliku vojsku  kolika je bila ona koju si izgubio, toliko konja i toliko kola.  Tada ćemo se pobiti s njima u ravnici, i sigurno ćemo ih nadvladati."  On ih posluša i učini tako. 
\par 26 Na početku godine Ben-Hadad podiže Aramejce i pođe na  Afek da vojuje s Izraelom. 
\par 27 Izraelci se podigoše i krenuše  protiv njih. I utaboriše se Izraelci pred njima kao dva mala  stada koza, dok su Aramejci prekrili zemlju. 
\par 28 Tada pristupi Božji čovjek izraelskom kralju i reče:  "Ovako veli Jahve: 'Zato što Aramejci kažu za Jahvu da je Bog  bregova i da nije Bog ravnica, ja ću predati u tvoje ruke ovo  silno mnoštvo da spoznate da sam ja Jahve'." 
\par 29 Sedam dana bijahu utaboreni jedni sučelice drugima. Sedmoga  dana zametnu se boj i Izraelci poubijaše Aramejce, stotinu tisuća  pješaka u jedan jedini dan. 
\par 30 Ostatak pobježe u Afek, u grad, ali se sruši zidina na dvadeset i sedam tisuća ljudi koji su  ostali.  Pobjegao je i Ben-Hadad. U gradu je prelazio iz jednog skrovišta  u drugo. 
\par 31 Njegove su mu sluge rekle: "Gle! Mi smo čuli da  su izraelski kraljevi milosrdni. Stavimo kostrijet oko bokova  svojih i konope oko svojih glava, pa izađimo pred kralja izraelskog:  možda će ti poštedjeti život." 
\par 32 I svezaše kostrijeti oko bokova  svojih i konopce oko svojih glava. Otišli su pred izraelskog  kralja i rekli: "Tvoj sluga Ben-Hadad kaže: 'Ostavi me na životu!'"  On odgovori: "Je li još živ? On je moj brat." 
\par 33 Ljudi su to uzeli kao dobar znak i požurili se da ga  uhvate za riječ govoreći: "Ben-Hadad tvoj je brat." Ahab odgovori:  "Idite! Dovedite ga!" Ben-Hadad dođe i on ga uze na kola. 
\par 34 Ben-Hadad  reče mu tada: "Vratit ću ti gradove koje je moj otac uzeo tvome  ocu; stajat će ti na raspolaganju četvrti u Damasku, kao što  ih je postavio moj otac u Samariji. Pod ovim me uvjetom otpusti."  Ahab sklopi s njime savez i otpusti ga. 
\par 35 Neki od proročkih sinova reče po Jahvinoj zapovijedi  svome drugu: "Udari me!" Ali čovjek ne htjede da ga tuče. 
\par 36 Tada  mu onaj reče: "Budući da nisi slušao glasa Jahvina, evo, kad  odeš od mene, lav će te razderati." Tek što se udaljio od njega, naiđe na lava koji ga razdera. 
\par 37 Prorok nađe drugoga čovjeka  i reče: "Udari me!" Čovjek ga izudara i izrani. 
\par 38 Prorok ode, postavi se kralju na put, a preko očiju navuče povez da ga ne  prepoznaju. 
\par 39 Kad je kralj prolazio, on povika: "Tvoj je sluga  bio izišao u boj, kadli iz bojnih redova jedan istupi i dovede  mi nekog čovjeka govoreći: 'Čuvaj ovoga čovjeka! Ako nestane, tvoj će život biti za njegov život, ili ćeš platiti srebrni  talenat.' 
\par 40 I dok je tvoj sluga radio ovdje-ondje, njega je  nestalo."  Tada mu reče kralj Izraela: "Eto ti presude! Sam si je izrekao!" 
\par 41 Nato onaj odmah ukloni povez s očiju i kralj izraelski vidje  da je to jedan od proroka. 
\par 42 A on reče kralju: "Ovako veli  Jahve: 'Budući da si pustio da ti iz ruke utekne čovjek koga  sam udario prokletstvom, tvoj će život biti za njegov život,  tvoj narod za njegov narod.'" 
\par 43 I kralj izraelski ode svojoj  kući, mrk i srdit, i uđe u Samariju. 


\chapter{21}

\par 1 Nakon tih događaja dogodilo se ovo: Nabot Jizreelac imao vinograd  kraj palače Ahaba, kralja samarijskog, 
\par 2 i Ahab ovako reče Nabotu:  "Ustupi mi svoj vinograd da mi bude za povrtnjak jer je blizu  moje kuće. Ja ću ti dati za nj bolji vinograd, ili, ako to želiš, dat ću ti novca koliko vrijedi." 
\par 3 Ali Nabot reče Ahabu: "Jahve  me sačuvao od toga da ti ustupim baštinu svojih otaca!" 
\par 4 Ahab se vrati kući mrk i ljutit zbog riječi koju mu je  Nabot Jizreelac rekao: "Ne dam ti baštine svojih otaca." Legao  je na postelju i okrenuo lice i nije htio okusiti hrane. 
\par 5 Dođe  mu njegova žena Izebela i reče: "Zašto si zlovoljan i ne mariš  za hranu?" 
\par 6 On joj odgovori: "Govorio sam Nabotu Jizreelcu  i rekao mu: 'Ustupi mi svoj vinograd za novac, ili, ako ti je  draže, dat ću ti drugi vinograd za taj.' Ali mi je on rekao:  'Ne dam ti svoga vinograda.'" 
\par 7 Tada mu žena Izebela reče: "Jesi  li ti onaj koji kraljuje nad Izraelom! Ustani i jedi i budi dobre  volje. Ja ću ti pribaviti vinograd Nabota Jizreelca." 
\par 8 I napisa ona pisma u ime Ahabovo i zapečati ih kraljevskim  pečatom. Pisma je poslala starješinama i glavarima Nabotovim  sugrađanima. 
\par 9 U tim je pismima napisala: "Proglasite post i  postavite Nabota na čelo naroda. 
\par 10 Postavite prema njemu dva  nitkova koji će ga optužiti: 'Proklinjao si Boga i kralja!' Tada  ga izvedite i kamenujte ga da pogine." 
\par 11 I učiniše ljudi Nabotova grada, starješine i glavari, kako im je Izebela zapovjedila i kako je pisalo u pismima koja  im je uputila. 
\par 12 Proglasiše post i Nabota postaviše na čelo  naroda. 
\par 13 Tada dođoše dva nitkova, sjedoše mu nasuprot i optužiše  Nabota pred narodom: "Nabot je proklinjao Boga i kralja." I tako  izvedoše Nabota izvan grada, zasuše ga kamenjem i on pogibe. 
\par 14 Zatim poručiše Izebeli: "Nabot je kamenovan i umro je." 
\par 15 Pošto  je Izebela čula da je Nabot kamenovan i da je umro, reče Ahabu:  "Ustani i zaposjedni vinograd što ti ga Nabot Jizreelac ne htjede  ustupiti za novac. Nabot više nije živ, on je mrtav." 
\par 16 Kada  je Ahab doznao da je Nabot mrtav, ustade i siđe u vinograd Nabota  Jizreelca da ga zaposjedne. 
\par 17 Tada bi upućena riječ Jahvina Iliji Tišbijcu: 
\par 18 "Ustani  i siđi u Samariju, u susret Ahabu, kralju izraelskom. Eno ga  u vinogradu Nabotovu u koji je sišao da ga zaposjedne. 
\par 19 Reci  mu: 'Ovako veli Jahve: Umorio si, oteo si! Zato ovako veli Jahve:  Na mjestu gdje su psi lizali Nabotovu krv, lizat će psi i tvoju.'" 
\par 20 Ahab reče Iliji: "Nađe li me, neprijatelju moj?" Ilija odgovori:  "Nađoh te, jer si se prodao da činiš što je zlo u očima Jahvinim. 
\par 21 Evo, tek što nisam navukao na te nesreću. Pomest ću tvoje  potomstvo, istrijebiti Ahabu sve što mokri uza zid, robove i  slobodnjake u Izraelu. 
\par 22 Učinit ću s tvojom kućom kao s kućom  Jeroboama, sina Nebatova, i s kućom Baše, sina Ahijina, jer si  me rasrdio i naveo Izraela na grijeh. 
\par 23 I nad Izebelom reče  Jahve: psi će proždrijeti Izebelu na Jizreelskom polju. 
\par 24 Tko  od obitelji Ahabove umre u gradu, psi će ga izjesti, a tko umre  u polju, pojest će ga ptice nebeske." 
\par 25 Doista, nitko se nije prodao tako kao Ahab da čini što  je zlo u očima Jahvinim, jer ga je zavodila njegova žena Izebela. 
\par 26 Činio je vrlo odvratna djela: išao je za idolima baš kao  što su činili Amorejci, koje je Jahve protjerao ispred Izraelaca. 
\par 27 Kad je Ahab čuo te riječi, razdrije svoje haljine i stavi  kostrijet na tijelo; i postio je, u kostrijeti je spavao i naokolo  išao tiho jecajući. 
\par 28 Tada dođe riječ Jahvina Iliji Tišbijcu: 
\par 29 "Jesi li vidio kako se Ahab preda mnom ponizio? Budući da  se tako ponizio preda mnom, neću zla pustiti za njegova života;  u vrijeme njegova sina pustit ću zlo na kuću njegovu." 



\chapter{22}

\par 1 Tri je godine vladao mir; nije bilo rata između Aramejaca  i Izraela. 
\par 2 Treće godine Jošafat, kralj judejski, posjeti kralja  izraelskoga. 
\par 3 Kralj Izraela reče svojim dvoranima: "Znate li  da je Ramot Gilead naš? A mi ne poduzimamo ništa da ga otmemo  iz ruke aramejskog kralja." 
\par 4 Zatim reče Jošafatu: "Hoćeš li  poći sa mnom na Ramot Gilead?" Jošafat odgovori kralju izraelskom:  "Ja sam kao i ti, moj narod kao i tvoj, moji konji što i tvoji." 
\par 5 Tada Jošafat reče kralju izraelskom: "De posavjetuj se  najprije s Jahvom." 
\par 6 Tada kralj izraelski sakupi oko četiri  stotine proroka i upita ih: "Mogu li zavojštiti na Ramot Gilead  ili da se okanim toga?" Oni odgovoriše: "Idi, jer će ga Jahve  predati kralju u ruke." 
\par 7 Ali Jošafat upita: "Ima li ovdje još  koji prorok Jahvin da i njega upitamo?" 
\par 8 Kralj izraelski odgovori  Jošafatu: "Ima još jedan čovjek preko koga bismo mogli upitati  Jahvu, ali ga ne podnosim jer mi ne prorokuje ništa dobro nego  samo zlo; to je Mihej, sin Jimlin." A Jošafat reče: "Neka kralj  ne govori tako!" 
\par 9 Tada kralj izraelski dozva jednoga dvoranina  i reče mu: "Brže dovedi Jimlina sina Miheja." 
\par 10 Izraelski kralj i judejski kralj Jošafat sjedili su svaki  na svome prijestolju, u svečanim haljinama pred Samarijskim vratima, a proroci proricali pred njima. 
\par 11 Kenaanin sin Sidkija napravi  sebi željezne rogove i reče: "Ovako govori Jahve: 'Njima ćeš  nabosti sve Aramejce dok ih ne uništiš'." 
\par 12 Tako su i svi drugi  proroci proricali govoreći: "Idi na Ramot Gilead i uspjet ćeš:  Jahve će ga predati kralju u ruke." 
\par 13 Glasnik koji bijaše otišao da zove Miheja reče mu: "Eno, svi proroci složno proriču dobro kralju. Govori i ti kao jedan  od njih i proreci mu uspjeh!" 
\par 14 Ali Mihej odvrati: "Živoga  mi Jahve, govorit ću ono što mi Jahve kaže!" 
\par 15 Kad dođe pred  kralja, upita ga kralj: "Miheju, da pođem u rat na Ramot Gilead  ili da se okanim toga?" On odgovori: "Pođi! Uspjet ćeš: Jahve  će ga dati u ruke kraljeve." 
\par 16 Ali mu kralj reče: "Koliko ću  te puta zaklinjati da mi kažeš samo istinu u Jahvino ime?" 
\par 17 Tada  Mihej odgovori: "Sav Izrael vidim rasut po gorama kao stado bez pastira. I Jahve veli: 'Nemaju više gospodara, neka se u miru kući vrate.'" 
\par 18 Tada izraelski kralj reče Jošafatu: "Nisam li ti rekao  da mi neće proreći dobro nego zlo!" 
\par 19 A Mihej reče: "Zato čuj  riječ Jahvinu: vidio sam Jahvu gdje sjedi na svome prijestolju, a sva mu vojska nebeska stajaše zdesna i slijeva. 
\par 20 Jahve  upita: 'Tko će zavesti Ahaba da otiđe i padne u Ramot Gileadu?'  Jedan reče ovo, drugi ono. 
\par 21 Tada uđe jedan duh i stade pred  Jahvu. 'Ja ću ga', reče,  'zavesti.' Jahve ga upita: 'Kako?' 
\par 22 On odgovori: 'Izaći ću i bit ću lažljiv duh u ustima svih  njegovih proroka.' Jahve reče: 'Ti ćeš ga zavesti. I uspjet ćeš.  Idi i učini tako!' 
\par 23 Tako je, evo, Jahve stavio lažljiva duha  u usta svih ovih tvojih proroka, ali ti Jahve navješćuje zlo." 
\par 24 Tada pristupi Kenaanin sin Sidkija i udari Miheja po  obrazu pitajući: "Zar je Jahvin duh napustio mene da bi s tobom  govorio?" 
\par 25 Mihej odgovori: "Vidjet ćeš onoga dana kad budeš  bježao iz sobe u sobu da se sakriješ." 
\par 26 Tada izraelski kralj  naredi: "Uhvati Miheja i odvedi ga gradskom zapovjedniku Amonu  i kraljeviću Joašu. 
\par 27 Reci im: Ovako veli kralj: 'Bacite ovoga  u tamnicu i držite ga na suhu kruhu i vodi dok se sretno ne vratim.'" 
\par 28 Mihej reče: "Ako se doista sretno vratiš, onda Jahve nije  govorio iz mene." I nadoda: "Čujte, svi puci!" 
\par 29 Izraelski kralj i judejski kralj Jošafat krenuše na Ramot  Gilead. 
\par 30 Izraelski kralj reče Jošafatu: "Ja ću se preobući  i onda ući u boj, ali ti ostani u svojoj odjeći!" Izraelski se  kralj preobuče i pođe u boj. 
\par 31 Aramejski kralj naredi zapovjednicima  bojnih kola: "Ne napadajte ni na maloga ni na velikoga, nego  jedino na izraelskog kralja!" 
\par 32 Kad zapovjednici bojnih kola  ugledaše Jošafata, rekoše: "To je kralj izraelski!" I krenuše  u boj prema njemu. A Jošafat povika. 
\par 33 A kad zapovjednici bojnih  kola vidješe da to nije izraelski kralj, okrenuše se od njega. 
\par 34 Jedan nasumce odape luk i ustrijeli izraelskog kralja  između nabora pojasa i oklopa. Kralj reče vozaču: "Okreni, izvedi  me iz boja jer mi nije dobro." 
\par 35 Boj je onoga dana bio sve  žešći, ali se kralj držao uspravno na bojnim kolima prema Aramejcima.  A navečer umrije. Krv se iz rane izlila u kola. 
\par 36 O zalasku  sunčevu odjeknu glas taborom: "Svaki u svoj grad i svaki u svoju  zemlju! 
\par 37 Kralj je poginuo!" Otišli su u Samariju i pokopali  kralja u Samariji. 
\par 38 Njegova su kola oprali u samarijskom ribnjaku, psi su lizali njegovu krv i bludnice se ondje kupale, po riječi  koju je rekao Jahve. 
\par 39 Ostala povijest Ahabova, sve što je učinio, o kući od  bjelokosti, o svim gradovima koje je sagradio, zar sve to nije  zapisano u knjizi Ljetopisa kraljeva izraelskih? 
\par 40 Ahab je  počinuo sa svojim ocima, a njegov sin Ahazja zakralji se mjesto  njega. 
\par 41 Jošafat, sin Asin, postade kraljem Judeje četvrte godine  kraljevanja Ahaba, kralja izraelskoga. 
\par 42 Jošafatu bijaše trideset  i pet godina kad se zakraljio; kraljevao je dvadeset i pet godina  u Jeruzalemu; mati mu se zvala Azuba, a bila je kći Šilhijeva. 
\par 43 Išao je sasvim putem oca Ase, ne skrećući s njega, nego čineći  što je pravo u očima Jahvinim. (22:44) Samo, uzvišice nisu bile uklonjene, narod je još prinosio klanice i kađenice na uzvišicama. (22:44) Samo, uzvišice nisu bile uklonjene, narod je još prinosio klanice i kađenice na uzvišicama. 
\par 44 (22:45) Jošafat  je bio u miru s izraelskim kraljem. 
\par 45 (22:46) Ostala povijest Jošafatova, pothvati koje je izveo i  kako je vojevao, zar to nije zapisano u knjizi Ljetopisa kraljeva  judejskih? 
\par 46 (22:47) Istrijebio je iz zemlje preostale bludnice, koje  su se održale iz vremena njegova oca Ase. 
\par 47 (22:48) Nije bilo kralja  u Edomu, nego je vladao namjesnik. 
\par 48 (22:49) Kralj Jošafat sagradi  taršiško brodovlje da ide u Ofir po zlato, ali nije otišlo jer  se brodovlje razbilo kod Esjon Gebera. 
\par 49 (22:50) Tada Ahazja, sin Ahabov, reče Jošafatu: "Neka moje sluge pođu s tvojim slugama na lađama."  Ali Jošafat to ne prihvati. 
\par 50 (22:51) Jošafat počinu sa svojim ocima  i sahranjen bi u gradu Davida, svoga praoca. Na njegovo se mjesto  zakraljio sin mu Joram. 
\par 51 (22:52) Ahazja, sin Ahabov, postade kraljem Izraela u Samariji  sedamnaeste godine Jošafatova kraljevanja Judejom i kraljevao  je dvije godine nad Izraelom. 
\par 52 (22:53) On je činio što je zlo u očima  Jahvinim i hodio je putem svoga oca i putem svoje majke i putem  Jeroboama, sina Nebatova, koji je navodio Izraela na grijeh. 
\par 53 (22:54) Služio je Baalu i klanjao se pred njim. Srdio je Jahvu, Boga  Izraelova, sasvim onako kako je činio njegov otac. 





\end{document}