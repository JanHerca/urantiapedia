\begin{document}

\title{Estera}


\chapter{1}

\par 1 Bilo je u vrijeme Ahasvera, onoga Ahasvera koji je vladao  nad sto dvadeset i sedam pokrajina od Indije do Etiopije. 
\par 2 U  to vrijeme, dok je kralj Ahasver sjedio na prijestolju svoga  kraljevstva u tvrđavi grada Suze, 
\par 3 treće godine svoga kraljevanja, priredi on gozbu za sve svoje knezove i službenike. Našli su  se tako pred njim zapovjednici perzijske i medijske vojske, odličnici  i pokrajinski upravitelji. 
\par 4 Punih sto i osamdeset dana pokazivaše  on bogatstvo i slavu kraljevstva svoga i veličanstveni sjaj veličine  svoje. 
\par 5 Kad je prošlo to vrijeme, priredi kralj u vrtnom trijemu  svoje palače sedmodnevnu gozbu za sav narod koji se nalazio u  tvrđavi grada Suze, od najvišega pa do najnižega. 
\par 6 Zavjese  od najfinijeg lana, vune, ljubičasta skrleta bile su pričvršćene  vrpcama od beza i crvena grimiza o srebrne prstenove na stupovima  od bijela mramora. Na podu od zelenog i bijelog mramora, sedefa  i skupocjenog kamenja, nalazile se postelje od srebra i zlata. 
\par 7 Za piće su služili zlatni pehari, sve jedan drugačiji od drugoga, a vina je bilo kraljevski obilno, kako i dolikuje kraljevskoj  moći. 
\par 8 Pilo se po nekom pravilu, ali ne prisilno, jer je kralj  bio naredio svim nadzirateljima svoga dvora da sa svakim postupaju  prema njegovoj želji. 
\par 9 I kraljica Vašti priredi gozbu za žene u kraljevskoj palači  kralja Ahasvera. 
\par 10 Sedmoga dana, kad srce kraljevo bijaše veselo  od vina, naredi Mehumanu, Bizeti, Harboni, Bigti, Abagti, Zetaru  i Karkasu, sedmorici eunuha koji su mu služili, 
\par 11 da dovedu  preda nj kraljicu Vašti s kraljevskom krunom, da bi pokazao narodu  i knezovima ljepotu njezinu. Ona je uistinu bila privlačna. 
\par 12 Ali  se kraljica Vašti ne htjede odazvati kraljevu pozivu što joj  ga prenesoše dvorani. Kralj se tada veoma razbjesni i njegova  se srdžba rasplamsa. 
\par 13 Onda zapita mudrace koji poznaju vremena.  Jer svaki se kraljev posao tako proučavao među onima koji su  poznavali zakone i pravo. 
\par 14 Najbliži su mu bili Karsena, Šetar, Admata, Taršiš, Mares, Marsena i Memukan, sedam knezova Perzije  i Medije. Oni su smjeli gledati kraljevo lice i zauzimali su  najistaknutija mjesta u kraljevstvu. 
\par 15 Upita ih: "Što treba  prema zakonu poduzeti protiv kraljice Vašti, koja se nije pokorila  zapovijedi kralja Ahasvera koju su joj saopćili dvorani?" 
\par 16 Memukan  tada odgovori pred kraljem i knezovima: "Kraljica je Vašti skrivila ne samo kralju nego i svim poglavarima  i svem narodu koji prebiva u svim pokrajinama kralja Ahasvera. 
\par 17 Jer će za držanje kraljičino doznati sve žene pa će prezirati  svoje muževe govoreći: 'Kralj je Ahasver naredio da dovedu preda  nj kraljicu Vašti, ali ona ne htjede doći.' 
\par 18 I žene će knezova  perzijskih i medijskih, pošto doznaju za kraljičino ponašanje, još danas pripovijedati svim poglavarima kraljevim, pa će biti  prkosa i prezira u izobilju. 
\par 19 Stoga, svidi li se kralju, neka  se objavi kraljevska naredba i umetne među zakone Perzije i Medije, tako da se više ne može opozvati, da se Vašti ne smije više  pojaviti pred kraljem Ahasverom, a kralj neka preda kraljevsku  čast drugoj ženi, boljoj od nje. 
\par 20 Kad se ta naredba koju će  kralj učiniti pročuje po svem kraljevstvu, koje je zaista veliko, sve će žene iskazivati poštovanje svojim muževima, od najvišega  pa do najnižega." 
\par 21 Riječ se svidje i kralju i njegovim knezovima. Stoga  on učini kako mu je savjetovao Memukan. 
\par 22 Uputi pisma u sve  kraljevske pokrajine, svakoj pokrajini pismom kojim se ona služila, a svakom narodu njegovim jezikom, da svaki muž bude gospodar  u svojoj kući. 


\chapter{2}

\par 1 Poslije tih događaja, kako mu se utiša gnjev, kralj Ahasver  sjeti se Vaštije, onoga što je ona učinila i što je bilo odlučeno  protiv nje. 
\par 2 Rekoše tada momci što služahu kralja: "Neka se  potraže za kralja mlade djevojke, djevice lijepa izgleda. 
\par 3 Kralj  neka odredi u svim pokrajinama svojega kraljevstva povjerenike  da mu sakupe sve djevice pristala izgleda u tvrđavi grada Suze, u haremu, pod upravom Hegeja, kraljeva eunuha, čuvara žena.  On će se pobrinuti za njihovu njegu. 
\par 4 Ona djevojka koja se  najviše svidi očima kraljevim neka kraljuje umjesto Vaštije." Bijaše to po volji kralju, i on tako uradi. 
\par 5 U tvrđavi grada Suze bio je neki Židov koji se zvao Mordokaj, sin Jaira, sina Šimeja, sina Kišova, iz plemena Benjaminova. 
\par 6 On je bio protjeran iz Jeruzalema među prognanicima koje je  babilonski kralj Nabukodonozor odveo zajedno s judejskim kraljem  Jekonijom. 
\par 7 On je odgajao Hadasu, to jest Esteru, kćerku strica  svoga, jer ona ne imađaše ni oca ni majke. Djevojka je bila pristala  i lijepa izgleda. Poslije smrti njezina oca i njezine majke Mordokaj  je uze k sebi kao kćerku. 
\par 8 Kako se začu za kraljevu riječ i njegovu naredbu, mnogo  se djevojaka sabra u tvrđavi grada Suze pod Hegejevim nadzorom.  Tako dovedoše i Esteru u kraljevu palaču, pod nadzor Hegeja,  čuvara žena. 
\par 9 Djevojka se svidje njegovim očima, steče ona  njegovu naklonost i on se pobrinu za njezino uljepšavanje i uzdržavanje.  Uz to joj dade sedam najvrednijih ropkinja kraljevskog dvora  i premjesti je, skupa s djevojkama, u najudobnije prostorije  harema. 
\par 10 Estera ne spomenu ni naroda ni obitelji kojoj je pripadala, jer joj Mordokaj bijaše zabranio da to učini. 
\par 11 Svakoga je  dana Mordokaj šetao pred dvorištem harema da bi doznao kako se  Estera osjeća i kako se prema njoj odnose. 
\par 12 Svaka je djevojka morala ući kralju kad je na nju, prema  uredbi za žene, došao red, to jest nakon dvanaest mjeseci. Jer  tada se završavalo razdoblje njihova uljepšavanja: šest mjeseci  uljem iz mirne, a šest mjeseci balzamom i ostalim pomastima za  žensku njegu. 
\par 13 Pa kad bi djevojka ulazila kralju, bilo joj je dopušteno  da sa sobom iz harema u kraljevsku palaču ponese sve što bi zatražila. 
\par 14 Ona bi ulazila uvečer, a ujutro bi se vraćala u drugi harem, pod nadzorom Šaašgaza, kraljeva eunuha, čuvara priležnica. Više  se ne bi vraćala kralju, osim ako bi je posebno zaželio i dozvao  je k sebi poimence. 
\par 15 Kada dođe red na Esteru, kćerku Abihajla, koji je bio  stric Mordokaja koji ju je bio pokćerio, da uđe kralju, ona ne  zatraži ništa osim onoga što joj bijaše rekao Hegej, kraljev  eunuh, čuvar žena. Ipak je pobuđivala udivljenje svih koji su  je gledali. 
\par 16 Esteru, dakle, uvedoše kralju Ahasveru, u njegovu  kraljevsku palaču, u desetom mjesecu, mjesecu Tebetu, sedme godine  njegova vladanja. 
\par 17 Kralj zavolje Esteru više od svih drugih  žena; više nego sve ostale djevice ona mu omilje i predobi ona  njegovu naklonost. I položi on na njezinu glavu kraljevsku krunu, pa mjesto Vaštije ona posta kraljicom. 
\par 18 Nakon toga priredi  kralj u čast Estere veliku gozbu za svoje knezove i službenike;  svim pokrajinama odredi odmor i razda darove kraljevski darežljivo. 
\par 19 Kad su drugi put djevojke bile sakupljene, Mordokaj sjeđaše  na vratima kraljevim. 
\par 20 Estera ne oda ni naroda ni obitelji  iz koje je potjecala, kao što joj Mordokaj bijaše naredio. Estera  se i dalje držala svih Mordokajevih uputa kao kad se nalazila  pod njegovim skrbništvom. 
\par 21 U ono vrijeme kad je Mordokaj sjedio na vratima kraljevim, Bigtan i Tereš, dva kraljeva dvoranina, čuvari praga, planuše  gnjevom i počeše snovati da podignu ruku na kralja Ahasvera. 
\par 22 Za tu njihovu namjeru sazna Mordokaj. On je dojavi kraljici  Esteri, a Estera je u Mordokajevo ime saopći kralju. 
\par 23 Sve  se izvidje i otkri se zavjera, pa obojica budu obješena o stup.  To se pred kraljem zapisa u knjizi Ljetopisa. 


\chapter{3}

\par 1 Poslije tih događaja kralj Ahasver promaknu Hamana, Hamdatina  sina, Agađanina: uzvisi ga i njegovo prijestolje postavi iznad  svih ostalih dostojanstvenika koji su bili s njim. 
\par 2 Svi službenici  kraljevi koji su se nalazili na kraljevim vratima prigibali bi  koljena i padali ničice pred Hamanom, jer je tako zapovjedio  kralj. Ali Mordokaj ne bi prignuo koljeno niti bi pao ničice. 
\par 3 Službenici kraljevi koji su se nalazili na vratima kraljevim  rekoše Mordokaju: "Zašto prestupaš kraljevu zapovijed?" 
\par 4 Iako  su mu oni to ponavljali svaki dan, on ih ne posluša. Onda oni  to dojaviše Hamanu, da vide vrijedi li Mordokajevo opravdanje.  Jer im bijaše rekao da je Židov. 
\par 5 Kad Haman utvrdi da Mordokaj  niti prigiba koljeno niti pada ničice pred njim, jako se razljuti. 
\par 6 A kad dozna kojemu narodu pripada, učini mu se premalo podići  ruke na samog Mordokaja nego naumi s njim pobiti i sve Židove  koji su živjeli u svem kraljevstvu Ahasverovu. 
\par 7 U prvom mjesecu, to jest u mjesecu Nisanu, dvanaeste godine  Ahasverova kraljevanja u nazočnosti Hamana baciše "Pur", to jest  ždrijeb, da utvrde dan i mjesec. Ždrijeb pade na trinaesti dan  dvanaestoga mjeseca, to jest na mjeseca Adara. 
\par 8 I Haman kaza kralju Ahasveru: "U svim pokrajinama tvoga  kraljevstva ima jedan narod razasut među drugim narodima i od  njih odvojen. Njegovi su zakoni drugačiji od zakona u svih ostalih  naroda. Oni se ne drže kraljevskih odredaba. Kralj ih zato ne  smije pustiti na miru. 
\par 9 Ako je kralju po volji, neka se raspiše  da se oni zatru; a ja ću izbrojiti deset tisuća srebrnih talenata  na ruke povjerenika da ih pohrane u kraljevsku riznicu." 
\par 10 Nakon toga kralj skinu pečatni prsten s ruke i preda  ga Hamanu, sinu Hamdatinu, Agađaninu, neprijatelju Židova, 
\par 11 i  kaza mu: "Neka ti bude novac i narod, pa učini s njim što bude  dobro u tvojim očima." 
\par 12 Trinaestoga dana prvoga mjeseca bijahu sazvani kraljevi  pisari, pa o onome što je naložio Haman sastaviše pisma i upraviše  ih kraljevskim namjesnicima, upraviteljima što stajahu na čelu  pojedinih pokrajina, knezovima svakoga pojedinog naroda, svakoj  pokrajini njezinim pismom i svakom narodu njegovim jezikom. Pisma  su napisana u kraljevo ime i na njima je udaren kraljev pečat. 
\par 13 Po skorotečama razaslane su svim kraljevim pokrajinama poslanice  da se svi Židovi, od dječaka do staraca, djeca i žene unište, pobiju, zatru, a njihova dobra da se zaplijene u jednom jedinom  danu, i to trinaestog dana dvanaestog mjeseca, mjeseca Adara. 
\par 14 Sadržaj ove naredbe, koja je imala postati zakonom u  svakoj pokrajini, bio je objavljen svim narodima da bi bili spremni  za taj dan. 
\par 15 Skoroteče žurno potekoše s kraljevskom naredbom.  Zakon bi objavljen i u tvrđavi Suze, pa dok su kralj i Haman  sjedili i častili se, grad je Suza bio uznemiren. 


\chapter{4}

\par 1 Mordokaj doznade za sve što se dogodilo: razdera na sebi haljine, navuče kostrijet, posu se pepelom i prođe posred grada kukajući  glasno i gorko. 
\par 2 Dođe samo do kraljevih vrata, jer s onom kostrijeti  na sebi ne mogaše kroz njih proći. 
\par 3 U svakoj je pokrajini,  svuda gdje se doznala kraljeva riječ i njegov proglas, među Židovima  zavladala žalost: postili su, plakali i jadikovali. Mnogima od  njih kostrijet i pepeo posta ležaj. 
\par 4 Esterine djevojke i njezini eunusi dođoše da je o tome  obavijeste. Kraljica se veoma uznemiri. Posla Mordokaju haljine  da bi ih obukao a skinuo sa sebe kostrijet, ali on to odbi. 
\par 5 Nato  Estera pozva Hataka, jednog od kraljevih eunuha koji joj je bio  određen za službu, pa ga posla k Mordokaju da dozna od njega  što se dogodilo i zbog čega je takav. 
\par 6 Hatak ode do Mordokaja  na gradski trg, pred vrata kraljeva. 
\par 7 Mordokaj mu pripovjedi  što mu se dogodilo i podrobno ga obavijesti o novcu koji je Haman  obećao položiti u kraljevu riznicu da bi mogao uništiti Židove. 
\par 8 Dade mu i prijepis naredbe o njihovu zatoru, koja je objavljena  u Suzi, da je pokaže Esteri te da joj javi i naloži neka ide  kralju: neka ga moli za milost i posreduje kod njega za svoj  narod. 
\par 9 Hatak se vrati i donese Esteri Mordokajevu poruku. 
\par 10 Estera  odvrati Hataku i naredi mu da saopći Mordokaju: 
\par 11 "Svi službenici  kraljevi i narod kraljevih pokrajina znaju kako svakoga onoga, bio on muškarac ili žena, koji nepozvan uđe kralju u unutrašnje  predvorje čeka jedan jedini zakon: smrtna kazna, osim ako kralj  ne pruži takvome svoje zlatno žezlo i poštedi mu život. A ja  već trideset dana nisam bila pozvana kralju." 
\par 12 Mordokaju bjehu saopćene Esterine riječi, 
\par 13 pa on poruči  Esteri: "Nemoj misliti da ćeš se zato što se nalaziš u kraljevoj  palači spasiti jedina od svih Židova: 
\par 14 jer budeš li u ovoj  prilici šutjela, doći će Židovima pomoć i spas s druge strane, a ti ćeš s kućom svoga oca propasti. Tko zna nisi li se baš  i popela do kraljevske časti zbog časa kao što je ovaj?" 
\par 15 Estera i opet poruči Mordokaju: 
\par 16 "Hajde, sakupi sve  Židove koji se nalaze u Suzi. Postite za me: tri dana i tri noći  ne jedite niti pijte. I ja ću tako postiti sa svojim djevojkama.  Tako pripremljena ući ću kralju i unatoč zakonu, pa treba li  da poginem, poginut ću." 
\par 17 Mordokaj se onda povuče i učini što mu je naredila Estera. 


\chapter{5}

\par 1 Trećega dana, pošto presta moliti, svuče molitvene haljine  i zaodjenu se slavom svojom. 
\par 2 I podigavši  zlatno žezlo, postavi ga na vrat Esteri, zagrli je i reče:  "Govori mi!" 
\par 3 Kralj joj reče: "Što  je tebi, kraljice Estero? Što želiš? Bila to i polovica kraljevstva, dobit ćeš je!" 
\par 4 Estera odgovori: "Neka kralj, ako mu je drago, dođe s Hamanom na gozbu koju sam danas priredila." 
\par 5 Kralj  odvrati: "Obavijestite odmah Hamana da bi se izvršila Esterina  želja."  Kralj dakle dođe s Hamanom na gozbu koju je Estera priredila. 
\par 6 Dok su pili vino, kralj kaza Esteri: "Što god zatražiš, dobit  ćeš. Što god zaželiš, bila to i polovica kraljevstva, bit će  ti!" 
\par 7 Estera odgovori: "Molba mi je i želja, 
\par 8 ako sam našla  milost u očima kraljevim i ako se kralju svidi dati mi što molim  i učiniti što želim, da kralj ponovo dođe s Hamanom na gozbu  koju ću sutra pripremiti za njih i tad ću postupiti po riječi  kraljevoj." 
\par 9 Toga dana Haman iziđe sretan i zadovoljna srca, ali se  rasrdi jako na Mordokaja kad vidje da na vratima kraljevim nije  ustao ni maknuo se pred njim. 
\par 10 Haman se ipak svlada. Ode kući  i posla po svoje prijatelje i po ženu Zarešu. 
\par 11 Pripovijedao  im je o sjaju svoga bogatstva, o mnoštvu svojih sinova i o svemu  onome čime ga je kralj uzveličao i čime ga je uzdignuo nad sve  svoje knezove i službenike. 
\par 12 Haman još dometnu: "I kraljica  Estera nije uz kralja pozvala nikoga osim mene na gozbu koju  je priredila. I sutra sam samo ja uz kralja njezin uzvanik. 
\par 13 Ali  me sve to ne može učiniti sretnim dokle god gledam Židova Mordokaja  kako sjedi na vratima kraljevim." 
\par 14 Reče mu Zareša, žena njegova, i svi prijatelji njegovi: "Podigni vješala visoka pedeset lakata.  Sutra ujutro zatraži od kralja neka na njih objese Mordokaja.  Poslije toga idi sretan s kraljem na gozbu." Savjet se Hamanu  učini dobar, pa on naredi da se podignu vješala. 


\chapter{6}

\par 1 Te noći kralj ne mogaše usnuti. Zato naredi da mu donesu i  čitaju knjigu znamenitih događaja, Ljetopise. 
\par 2 Tu se nađe zapisano  kako je Mordokaj prokazao Bigtanu i Tereša, dva dvoranina kraljeva, čuvare praga, koji su se spremali da podignu ruke na kralja  Ahasvera. 
\par 3 Kralj upita: "Kakva je čast i kakvo je odlikovanje  zapalo Mordokaja za sve to?" Kraljeve sluge, dvorani koji ga  slušahu, odgovoriše: "Ništa nije učinjeno za nj." 
\par 4 Kralj onda zapita: "Tko je u predvorju?" A to u vanjsko  predvorje kraljevske palače bijaše stigao Haman da traži od kralja  neka objese Mordokaja na vješalima koja su već bila podignuta  za nj. 
\par 5 Službenici kraljevi odgovoriše: "Eno se u predvorju  nalazi Haman." "Neka uđe!" - naredi kralj. 
\par 6 Kako Haman uđe, kralj ga upita: "Što treba učiniti čovjeku koga kralj hoće da  počasti?" Haman reče u sebi: "Koga ako ne mene kralj želi počastiti?" 
\par 7 Zato odgovori kralju: "Za čovjeka koga kralj želi počastiti 
\par 8 treba donijeti kraljevske haljine koje kralj sam oblači i  dovesti konja kojega kralj jaše i položiti mu na glavu kraljevsku  krunu. 
\par 9 Haljine i konja neka kralj preda jednome od najuglednijih  kneževa kraljevih da bi taj obukao onoga koga kralj želi počastiti  i na konju ga odveo na gradski trg uzvikujući pred njim: 'Tako  biva onome koga kralj hoće da počasti!'" 
\par 10 Kralj nato naredi  Hamanu: "Uzmi odmah haljine i konja, kako si rekao, pa učini  tako Mordokaju Židovu koji sjedi na kraljevim vratima i ne propusti  ništa od onoga što si rekao!" 
\par 11 Haman uze haljine i konja: obuče u haljine Mordokaja  i provede ga na konju po trgu grada vičući pred njim: "Tako biva  onome koga kralj hoće da počasti!" 
\par 12 Malo zatim Mordokaj se  vrati k vratima kraljevim, a Haman, tužan i zastrte glave, ode  žurno kući 
\par 13 te ispriča Zareši, ženi svojoj, i svima prijateljima  svojim što se dogodilo. Njegovi mu savjetnici i žena Zareša rekoše:  "Ako Mordokaj, pred kojim si počeo posrtati, pripada židovskom  rodu, nećeš ga nadjačati, nego će te on zacijelo oboriti." 
\par 14 Još su o tom razgovarali, kad eto kraljevih dvorana.  Došli su tražiti Hamana da ga žurno odvedu na gozbu koju je priredila  Estera. 


\chapter{7}

\par 1 Kralj i Haman dođoše na gozbu kraljici Esteri. 
\par 2 I toga drugoga  dana, dok se pilo vino, reče kralj Esteri: "Koja ti je molba, kraljice Estero? Bit će ti udovoljena! Koja je tvoja želja?  Ako je i pola kraljevstva, bit će ti!" 
\par 3 Kraljica Estera odgovori:  "Ako sam, kralju, našla milost u tvojim očima i ako ti je s voljom, neka mi se u ime molbe pokloni život, a u ime želje moj narod! 
\par 4 Jer smo ja i narod moj predani za zator, klanje, uništenje.  Da smo predani u roblje, šutjela bih jer nevolja ne bi bila štetna  po kralja." 
\par 5 Ali kralj Ahasver upade kraljici Esteri u riječ  pa je upita: "Tko je taj? Gdje je taj koji je namislio takvo  što učiniti?" Estera tada odgovori: "Progonitelj i neprijatelj  jest Haman, ovaj zlikovac!" 
\par 6 Haman se zaprepasti pred kraljem  i kraljicom. 
\par 7 Kralj, gnjevan, ostavi vino te ode u vrt palače.  Haman osta uz kraljicu da je moli za svoj život, jer je uvidio  da je njegova nesreća pred kraljem gotova. 
\par 8 Kralj se vrati iz vrta u dvoranu gdje se pilo vino. Dotle  Haman bijaše pao na počivaljku na kojoj se nalazila Estera. "Pokušavaš  još i nasilje nad kraljicom, i to u mome vlastitom domu?" - povika  kralj. Tek što su te riječi izletjele iz kraljevih usta, pokriše  lice Hamanu. 
\par 9 Tada kaza Harbona, jedan od dvorana koji su stajali  u službi kraljevoj: "Eno i vješala što ih je Haman pripravio  za Mordokaja koji je govorio u korist kraljevu. Nalaze se kraj  Hamanove kuće i visoka su pedeset lakata." Kralj zapovjedi: "Objesite  ga na njih!" 
\par 10 Hamana objesiše na vješala koja bijaše pripravio  Mordokaju, i kraljeva se srdžba utiša. 


\chapter{8}

\par 1 Onoga istog dana kralj Ahasver preda kraljici Esteri kuću Hamana, progonitelja Židova, a Mordokaj je stupio pred kraljevo lice, jer je Estera objasnila kralju što joj je on. 
\par 2 Kralj skinu  pečatni prsten, koji je već bio oduzeo Hamanu, i dade ga Mordokaju, a Estera postavi Mordokaja nad Hamanovom kućom. 
\par 3 Estera tada ponovo progovori kralju. Baci mu se pred noge;  rasplaka se i najvruće ga zamoli da osujeti zlo Hamana Agađanina  i naum opaki što ga bijaše zasnovao protiv Židova. 
\par 4 Kralj pruži  prema Esteri zlatno žezlo. Estera se diže, stade pred kraljem 
\par 5 i reče: "Ako je kralju po volji, ako sam našla milost pred  licem njegovim, ako je kralju pravo te ako sam mila u njegovim  očima, neka pismeno opozove sve što napisa Haman, sin Hamdatin, Agađanin, u opakoj nakani da se pobiju Židovi koji se nalaze  u svim pokrajinama kraljevstva. 
\par 6 TÓa kako bih ja mogla gledati  nesreću koja bi pogodila moj narod? Kako bih mogla gledati zator  roda svoga?" 
\par 7 Kralj Ahasver odgovori kraljici Esteri i Mordokaju Židovu:  "Eto, poklonio sam Esteri kuću Hamanovu, a njega sam dao objesiti  jer je bio digao svoju ruku na Židove, 
\par 8 a vi u ime kraljevo  napišite o Židovima što vam se sviđa i zapečatite kraljevim prstenom.  Jer neopoziv je proglas koji je u kraljevo ime napisan te kraljevim  pečatom zapečaćen." 
\par 9 Tada, dvadeset i trećeg dana trećega mjeseca, to jest  mjeseca Sivana, budu sazvani pisari kraljevi i prema svemu što  bijaše naredio Mordokaj napisa se Židovima, namjesnicima, upravljačima  i knezovima pokrajina od Indije do Etiopije, a bijaše sto dvadeset  i sedam pokrajina, svakoj pokrajini njezinim pismom, svakom narodu  njegovim jezikom, pa i Židovima njihovim pismom i njihovim jezikom. 
\par 10 On napisa pisma u ime kralja Ahasvera i zapečati ih kraljevim  prstenom pa ih razasla po skorotečama koji su jahali na konjima, pastusima iz kraljevske ergele. 
\par 11 Kralj je dopustio Židovima  po svim gradovima da se mogu sastajati, braniti svoj život te  uništiti, ubiti i zatrti svaku vojsku narodnu ili pokrajinsku  koja bi ih napala, ne štedeći ni djecu ni žene, a slobodno im  je oplijeniti njihova dobra; 
\par 12 sve istoga dana u svim pokrajinama  kraljevstva Ahasverova: trinaestog dana dvanaestoga mjeseca,  to jest mjeseca Adara. 
\par 13 Prijepis pisma, koje je imalo postati zakonom u svakoj  pokrajini, bijaše objavljen među svim narodima, kako bi Židovi  toga dana bili spremni osvetiti se svojim neprijateljima. 
\par 14 Skoroteče, konjanici na kraljevskim pastusima, krenuše odmah i pojuriše, po kraljevoj zapovijedi. Naredba je bila objavljena i u tvrđavi  Suzi. 
\par 15 Mordokaj izađe od kralja odjeven u grimiznu i lanenu  kraljevsku haljinu, s velikom zlatnom krunom i s ogrtačem od  fine tkanine i crvena skrleta. Grad je Suza klicao i veselio  se. 
\par 16 Bio je to za Židove dan svjetla, veselja, kliktanja i  slavlja. 
\par 17 U svakoj pokrajini, u svakom gradu i mjestu do kojega  je dopro kraljev ukaz i zakon, zavlada među Židovima veselje, radost, gozba i blagdan, i mnogi među pucima zemlje postadoše  Židovi jer ih je spopao strah od Židova. 


\chapter{9}

\par 1 Trinaestoga dana dvanaestog mjeseca, mjeseca Adara, kad je  morala biti izvršena odredba kraljevog ukaza, istoga dana u koji  su se neprijatelji Židova nadali zavladati nad njima dogodi se  obrnuto: Židovi zavladaše nad neprijateljima svojim. 
\par 2 Židovi se sakupiše po svojim gradovima u svim pokrajinama  kralja Ahasvera da udare na one koji su tražili njihovu propast.  I nitko se nije usuđivao da im pruži otpor, jer je sve narode  spopao strah od Židova. 
\par 3 Svi su knezovi pokrajina i namjesnici, upravljači i činovnici kraljevi štitili Židove jer ih je obuzeo  strah od Mordokaja. 
\par 4 Jer je Mordokaj postao velik na kraljevskom  dvoru, i po svim pokrajinama širio se glas da Mordokaj postaje  sve moćniji. 
\par 5 Židovi, dakle, udariše mačem po svim svojim neprijateljima, sasjekoše ih i zatrše; sa svojim mrziteljima postupiše kako  im se htjelo. 
\par 6 Samo u tvrđavi Suzi smakoše i zatrše Židovi  pet stotina ljudi; 
\par 7 pogubiše Paršandatu, Dalfona, Aspatu, 
\par 8 Poratu, Adaliju, Aridatu, 
\par 9 Parmaštu, Arisaja, Aridaja, Jezatu 
\par 10 i  deset sinova Hamana, sina Hamdatina, progonitelja Židova. Ali  se ne pojagmiše za plijenom. 
\par 11 Toga istog dana, doznavši za broj ubijenih u tvrđavi  Suzi, 
\par 12 kralj reče kraljici Esteri: "U tvrđavi Suzi Židovi  su smaknuli i uništili pet stotina ljudi i deset Hamanovih sinova.  Što su tek onda izveli u ostalim pokrajinama kraljevim? Koja  je sada molba tvoja? Bit će uslišana! Koja je tvoja želja? Bit  će ispunjena!" 
\par 13 "Ako je kralju po volji," reče Estera, "neka  se Židovima koji žive u Suzi dopusti još sutra primijeniti isti  zakon kao i danas i neka se objesi deset Hamanovih sinova." 
\par 14 Kralj naredi da se tako učini: zakon bi u Suzi proglašen  i deset Hamanovih sinova obješeno. 
\par 15 Tako se Židovi Suze sakupiše  i četrnaestoga dana mjeseca Adara pa pobiše u Suzi još tri stotine  ljudi. Ali se ni tada ne pojagmiše za plijenom. 
\par 16 Ostali Židovi, oni koji su živjeli u kraljevskim pokrajinama, sakupiše se da brane svoje živote i mir od neprijatelja: pobiše  sedamdeset i pet tisuća dušmana. Ali se ni tada ne pojagmiše  za plijenom. Bio je trinaesti dan mjeseca Adara. 
\par 17 Četrnaestoga  dana Židovi su mirovali: to bijaše dan gozbe i veselja. 
\par 18 Židovi  u Suzi koji su se sakupili trinaestoga i četrnaestoga dana mirovahu  petnaestoga dana; to je bio dan njihova veselja i gozbi. 
\par 19 Zbog  toga Židovi pripoljci, oni koji žive po neutvrđenim selima, blagdanski  svetkuju četrnaesti dan mjeseca Adara veseleći se i gozbujući  i među sobom izmjenjujući darove. 
\par 20 Mordokaj opisa te događaje i upravi pisma Židovima svih  blizih i dalekih pokrajina kralja Ahasvera. 
\par 21 Naložio im je  da četrnaesti i petnaesti dan mjeseca Adara svake godine slave 
\par 22 kao dane u kojima su Židovi postigli spokoj od svojih neprijatelja  i kao mjesec koji je bio pretvorio u radost njihovu tugu a u  blagdan njihovo žalovanje. Neka ih slave gozbom i veseljem, izmjenjujući  među sobom darove i dijeleći poklone ubogima. 
\par 23 Židovi prihvatiše  da drže ono što su već sami od sebe počeli slaviti i o čemu im  je pisao Mordokaj: 
\par 24 "Haman, sin Hamdatin, Agađanin, progonitelj  svih Židova, kako je bio naumio sve ih uništiti, baci 'Pur',  to jest ždrijeb, za njihovo smaknuće i zator; 
\par 25 ali kad je  za njegovu zamisao doznao kralj, on pismeno naredi: 'Neka se  na njegovu glavu obori opaki naum što ga bijaše zasnovao protiv  Židova i neka bude obješen, on i sinovi njegovi.'" 
\par 26 Zbog toga su ti dani nazvani Purim, prema riječi Pur.  Zato prema svem sadržaju toga pisma i prema onome što su vidjeli  i što im bijaše preneseno 
\par 27 Židovi se neopozivo obvezaše i  prihvatiše za se, za svoje potomke i za sve one koji se s njima  budu udružili da će svake godine slaviti ta dva dana prema tom  propisu i u to vrijeme. 
\par 28 Te dane valja slaviti i njih se sjećati  od pokoljenja do pokoljenja u svakoj obitelji, pokrajini i gradu;  ti dani Purima ne smiju iščeznuti ispred Židova, ni spomen na  njih biti izbrisan iz njihova roda. 
\par 29 Kraljica Estera, kći Abihailova, i Židov Mordokaj pisali  su to što snažnije da tako još jednom potkrijepe pismo o Purimu. 
\par 30 Pisma su poslali svim Židovima u sto dvadeset i sedam pokrajina  Ahasverova kraljevstva s porukom mira i vjernosti; 
\par 31 da obdržavaju  te dane Purima u njihovo određeno vrijeme, kako su to odredili  Židov Mordokaj i kraljica Estera, i da drže post i molitve, onako  kako su to oni obvezali sebe i svoje potomke. 
\par 32 Tako Esterina  naredba ozakoni ove propise Purima i to bi zapisano u knjigu. 



\chapter{10}

\par 1 Kralj Ahasver udari danak na zemlju i na otoke morske. 
\par 2 Sva  djela njegove moći i hrabrosti, a tako i izvještaj o uzdignuću  Mordokaja koga je kralj uzvisio, zapisani su u Ljetopisima kraljeva  Medije i Perzije: 
\par 3 kako je Židov Mordokaj bio prvi iza kralja  Ahasvera, velik u očima Židova, voljen od mnoštva svoje subraće  kao pobornik blagostanja svoga naroda i glasnik mira za svoj  rod. 




\end{document}