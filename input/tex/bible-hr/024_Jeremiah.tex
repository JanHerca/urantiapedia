\begin{document}

\title{Jeremija}


\chapter{1}

\par 1 Riječi Jeremije, sina Hilkijina, svećenika iz Anatota, u zemlji  Benjaminovoj. 
\par 2 Njemu dođe riječ Jahvina, u dane Jošije, sina Amonova, kralja Judina, trinaeste godine vladanja njegova: 
\par 3 zatim u  dane Jojakima, sina Jošijina, kralja Judina, do svršetka jedanaeste  godine Sidkije, sina Jošijina, kralja Judeje sve do Jeruzalema, u petom mjesecu izgnanstva. 
\par 4 Dođe mi riječ Jahvina: 
\par 5 "Prije nego što te oblikovah u majčinoj utrobi, ja te znadoh; prije nego što iz krila majčina izađe, ja te posvetih, za proroka svim narodima postavih te." 
\par 6 A ja rekoh: "Ah, Gospode Jahve, gle, ja ne umijem govoriti: dijete sam." 
\par 7 A Jahve mi odvrati: "Ne govori: 'Dijete sam!' Već idi k onima kojima te šaljem i reci sve ono što ću ti narediti. 
\par 8 Ne boj ih se: jer ja sam s tobom da te izbavim," riječ je Jahvina. 
\par 9 I tada Jahve pruži ruku, dotače se usta mojih i reče: "Evo, u usta tvoja stavljam riječi svoje. 
\par 10 Gle: postavljam te danas nad narode i kraljevstva, da istrebljuješ i rušiš, da zatireš i ništiš, da gradiš i sadiš." 
\par 11 I dođe mi riječ Jahvina: "Što vidiš, Jeremija?" A ja  ću: "Vidim granu bademovu." 
\par 12 Tada mi Jahve reče: "Dobro vidiš, jer ja bdim nad riječima svojim da ih ispunim!" 
\par 13 I dođe mi riječ Jahvina: "Što vidiš?" A ja ću: "Vidim  uzavrio lonac, a otvor mu gleda na sjever." 
\par 14 I Jahve mi reče: "Sa sjevera buknut će zlo protiv svih stanovnika ove zemlje. 
\par 15 Jer evo, ja ću sazvati sva sjeverna kraljevstva" - riječ je Jahvina. "I ona će doći: svako će od njih staviti svoje prijestolje pred ulaz vrata Jeruzalema, protiv svih zidina njegovih, i protiv svih gradova judejskih. 
\par 16 I sudit ću im za sve opačine njihove; zato što me ostaviše, zato što kadiše drugim bogovima i klanjahu se djelima ruku svojih. 
\par 17 Ti bedra svoja sad opaši, ustaj, pa ćeš im govoriti sve što ću tebi zapovjediti. Ne dršći pred njima, da ne bih morao učiniti da uzdršćeš pred njima. 
\par 18 Danas te, evo, postavljam kao grad utvrđeni, kao stup željezni, k'o zidinu brončanu protiv sve zemlje: protiv kraljeva i knezova judejskih, svećenika i naroda ove zemlje. 
\par 19 I borit će se s tobom, al' te neće nadvladati, jer ja sam s tobom da te izbavim," riječ je Jahvina. 


\chapter{2}

\par 1 I dođe mi riječ Jahvina: 
\par 2 "Idi i viči u uši Jeruzalemu: Ovako govori Jahve: Spominjem se mladosti tvoje privržene, ljubavi tvoje vjereničke: ti pođe za mnom u pustinju, po zemlji gdje se ne sije. 
\par 3 Izrael bijaše Jahvi svetinja, prvina plodova njegovih; tko god od njih jeđaše, bijaše kažnjen; zlo ga snađe" - riječ je Jahvina. 
\par 4 "Čujte riječ Jahvinu, dome Jakovljev, i svi rodovi doma Izraelova. 
\par 5 Ovako govori Jahve: 'Kakvu nepravdu nađoše oci vaši na meni te se udaljiše od mene? Za ispraznošću pođoše, te sami isprazni postadoše. 
\par 6 Ne pitahu: Gdje je Jahve koji nas izvede iz zemlje egipatske te nas vođaše kroz pustinju, po zemlji pustoj, jedva prohodnoj, po zemlji suhoj i mračnoj, po zemlji kojom nitko ne prolazi, nit' se tko nastanjuje?' 
\par 7 U zemlju vinograda i maslinika ja vas dovedoh, da se hranite plodom i dobrotom njezinom. Ali tek što uđoste, zemlju moju oskvrnuste i baštinu moju u gnusobu pretvoriste. 
\par 8 Svećenici ne govorahu: 'Gdje je Jahve?' Tumači Zakona mene ne upoznaše, pastiri otpadoše od mene, a proroci prorokovahu u ime Baalovo i iđahu za onima što im pomoći ne mogoše. 
\par 9 Zato ću još parnicu voditi s vama - riječ je Jahvina - i parbit ću se sa sinovima sinova vaših. 
\par 10 Pođite, dakle, na otoke kitimske, da vidite, ili u Kedar pošljite izvidnice te dobro promislite i provjerite je li se igda što slično zbilo. 
\par 11 Je li koji narod mijenjao bogove - oni čak i nisu bogovi! A narod moj Slavu svoju zamijeni za one što ne pomažu! 
\par 12 Zapanjite se nad tim, nebesa, zgranite se i zaprepastite," riječ je Jahvina. 
\par 13 "Jer dva zla narod moj učini: ostavi mene, Izvor vode žive, te iskopa sebi kladence, kladence ispucane što vode držati ne mogu. 
\par 14 Je li Izrael rob il' sluga u kući rođen? Zašto plijenom posta? 
\par 15 Lavovi su na nj rikali, podizali glas svoj. U pustoš pretvoriše zemlju njegovu, gradove popališe, nema im žitelja. 
\par 16 Čak i oni iz Memfisa i Tafnisa brijahu ti tjeme. 
\par 17 Nisi li to sam sebi učinio otpavši od Jahve, Boga svojega? 
\par 18 A sad, zašto krećeš u Egipat da piješ vode iz Nila? Zašto krećeš u Asiriju da piješ vode iz Rijeke? 
\par 19 Opačina te tvoja kažnjava, otpadništvo te tvoje osuđuje. Shvati i vidi kako je teško i gorko što ostavi Jahvu, Boga svojega, što više nema straha mog u tebi" - riječ je Gospoda Jahve nad Vojskama. 
\par 20 "Da, odavna ti slomi jaram svoj, raskide veze što te vezahu i reče: 'Neću da robujem.' Pa ipak, na svakom povišem humu, pod svakim drvetom zelenim lijegao si k'o bludnica. 
\par 21 A ja te zasadih kao lozu izabranu, k'o sadnicu plemenitu. Kako li mi se samo prometnu u jalov izrod, u lozu divlju! 
\par 22 Da se i lužinom opereš, napravljenom od mnogo pepela, ostat će mrlja bezakonja tvoga preda mnom" - riječ je Jahve Gospoda. 
\par 23 "Kako samo možeš reći: 'Nisam se uprljala, za baalima nisam trčala.' Pogledaj tragove svoje u Dolini, upoznaj što si učinila. Deva brza što krstari stazama svojim, 
\par 24 magarica divlja navikla na pustinju, u pohoti svojoj požudno dašće, tko da je ukroti u vrijeme gonjenja? Tko god je traži, neće se umoriti, naći će je u mjesecu njezinu. 
\par 25 Čuvaj se da ti noga ne obosi, grlo se ne osuši. A ti kažeš: 'Ne, uzalud je! Jer volim strance, i za njima ću ići.' 
\par 26 Kao što se lupež zastidi kad ga uhvate, tako će se zastidjeti sinovi - dom Izraelov, oni, kraljevi, knezovi, svećenici i proroci njihovi 
\par 27 koji govore drvetu: 'Ti si otac moj!' a kamenu: 'Ti si me rodio!' jer mi leđa okreću, a ne lice, ali u nevolji svojoj zapomažu: 'Ustani, spasi nas!' 
\par 28 Gdje su bogovi što ih ti sam načini? Nek' ustanu ako te mogu spasiti u nevolji tvojoj! Jer imaš, o Judejo, bogova koliko i gradova! Koliko Jeruzalem ima ulica, toliko Baal ima žrtvenika. 
\par 29 Zašto hoćete da se sa mnom parbite? Svi se od mene odmetnuste" - riječ je Jahvina. 
\par 30 "Zaludu sam udarao sinove vaše: vi iz toga ne uzeste pouke: mačevi vaši rastrgaše vaše proroke kao lav zatornik. 
\par 31 Kakva li ste roda? Čujte riječ Jahvinu: Zar bijah pustinja Izraelu, il' zemlja mračna? Zašto moj narod govori: 'Slobodu hoćemo, nećemo više k tebi!' 
\par 32 Zaboravlja li djevica svoj nakit il' nevjesta pojas svoj? A narod moj mene zaboravi, bezbroj je tomu već dana. 
\par 33 O, kako li dobro znaš svoj put kad tragaš za ljubavlju! Zato si i na zlo putove svoje navikla. 
\par 34 Čak su i ruke tvoje omašćene krvlju siromaha nevinih: nisi ih zatekla kako provaljuju vrata tvoja. Da, za sve njih ti ćeš odgovarati. 
\par 35 A govoriš: 'Nevina sam, gnjev se njegov odvratio od mene.' Evo me da ti sudim jer govoriš: 'Nisam zgriješila.' 
\par 36 Kako si jadna u zabludjelosti svojoj! I Egipćani će te posramiti kao što te posramiše Asirci. 
\par 37 I odavde ćeš morati otići s rukama nad glavom svojom, jer Jahve odbaci one u koje se uzdaš; ti nećeš biti sretna s njima." 


\chapter{3}

\par 1 "Ako muž otpusti ženu svoju i ona ide od njega te se uda za drugoga, ima li još pravo da se vrati njemu? Nije li ta žena sasvim oskvrnuta? A ti si bludničila s mnogim milosnicima, pa da se meni vratiš?" - riječ je Jahvina. 
\par 2 Podigni oči na goleti i pogledaj: gdje te to nisu oskvrnuli? Na putovima si ih dočekivala kao Arapin u pustinji. Ti si oskvrnula zemlju bludom i opačinom svojom, 
\par 3 zato i kiše prestadoše i kasni daždevi ne padoše. Čelo ti je kao u bludnice: ni zacrvenjela se nisi. 
\par 4 Ne dovikuješ li mi sada: 'Oče moj, ti si prijatelj mladosti moje! 
\par 5 Hoće li zauvijek plamtjeti, vječno tinjati gnjev tvoj?' Tako govoriš, a činiš i dalje zla koliko god možeš." 
\par 6 Jahve mi reče u dane kralja Jošije: "Vidje li što učini  odmetnica Izrael? Ona odlazi na svaku visoku goru i pod svako  zeleno stablo i ondje se podaje bludu. 
\par 7 A ja mišljah: 'Poslije  svega što učini vratit će se k meni.' Ali se ona ne vraća. I  to vidje sestra njena, nevjernica Judeja. 
\par 8 A vidje i kako otpustih  odmetnicu Izraela zbog svih preljuba i dadoh joj knjigu otpusnu.  Ali sestra joj, nevjernica Judeja, nimalo se ne poboja, pa i  ona okrenu u blud. 
\par 9 I svojim lakoumnim bludom obeščasti zemlju;  činila je preljub s kamenjem i drvljem. 
\par 10 I nakon svega toga  nije se vratila k meni nevjernica sestra njezina, Judeja, svim  srcem svojim, već samo prijetvorno" - riječ je Jahvina. 
\par 11 I reče mi Jahve: "Odmetnica Izrael pravednija je od Judeje  nevjernice. 
\par 12 Idi i viči prema Sjeveru ove riječi. Reci: Vrati se, odmetnice, Izraele, riječ je Jahvina. Ne gnjevi se više lice moje na vas, jer sam milostiv - riječ je Jahvina - neću se gnjeviti dovijeka. 
\par 13 Samo priznaj svoju krivnju da si se odvrgla od Jahve, Boga svojega, i odlutala k tuđincima, pod svako drvo zeleno i nisi slušala glasa mojega - riječ je Jahvina. 
\par 14 Vratite se, sinovi odmetnici - riječ je Jahvina - jer  ja sam vaš Gospodar. Uzet ću vas, po jednoga iz svakoga grada, po dvojicu od svakoga roda, da vas odvedem na Sion. 
\par 15 I dat  ću vam pastire po srcu svojemu koji će vas pasti razumno i mudro. 
\par 16 A kad se u one dane namnožite i narodite u zemlji - riječ  je Jahvina - više se neće govoriti: 'Kovčeg Jahvina Saveza' i  nitko neće na nj misliti, nitko ga se neće sjećati ni za njim  čeznuti, niti ga ponovo graditi. 
\par 17 U to će se vrijeme Jeruzalem  zvati 'Prijestolje Gospodnje'; i svi će se narodi u njemu sabrati  u ime Jahvino, i nijedan se više neće tvrdoglavo povoditi za  pokvarenim srcem svojim. 
\par 18 U one dane slagat će se dom Judin s domom Izraelovim;  i zajedno će krenuti iz Zemlje sjeverne u zemlju koju ocima vašim  dadoh u baštinu." 
\par 19 A ja rekoh: "Kako da te ubrojim među sinove i dam ti zemlju slasti, baštinu, najljepši biser među narodima! Pomislih: Ti ćeš me zvati 'Oče moj!' i nećeš se više odvratiti od mene." 
\par 20 "Ali kao što se žena iznevjeri mužu svome, tako se i vi iznevjeriste meni, dome Izraelov" - riječ je Jahvina. 
\par 21 Čuj! Po goletima plač se čuje, zapomaganje djece Izraelove, jer skrenuše s puta svojega. Jahvu, Boga svoga, zaboraviše. 
\par 22 - Vratite se, sinovi, što se odvratiste, izliječit ću odmetništva vaša! - Evo nas, dolazimo k tebi, jer ti si Jahve, Bog naš! 
\par 23 Doista, prijevarni su visovi, graja po brdima. Odista, u Jahvi, Bogu našemu, spasenje je Izraelovo! 
\par 24 Baal je proždro trud naših otaca još od mladosti naše: ovce njihove i goveda, nihove kćeri i sinove. 
\par 25 Lezimo u sramotu svoju, nek' nas pokrije ruglo naše! Jer Jahvi, Bogu svome, sagriješismo mi i oci naši, od mladosti svoje do dana današnjeg, i ne slušamo glasa Jahve, Boga svojega. 


\chapter{4}

\par 1 "Ako se, Izraele, želiš vratiti - riječ je Jahvina - k meni se vrati; ukloniš li grozote svoje, više ne moraš bježati od mene. 
\par 2 Ako se zakuneš: 'Živoga mi Jahve!' istinito, pravo i pravedno, narodi će se blagoslivljati u tebi i tobom se dičiti." 
\par 3 Jer ovako govori Jahve Judejcima i Jeruzalemcima: "Prokrčite sebi prljuše, ne sijte po trnjacima. 
\par 4 Obrežite se Jahvi, skinite obrezak sa srca svojega, Judejci i Jeruzalemci, jer će bijes moj buknuti kao vatra i gorjet će, a nikog da ugasi, zbog zlodjela i opačina što ih počiniste. 
\par 5 Objavite u Judeji, razglasite u Jeruzalemu! Trubite u rog širom zemlje, vičite punim glasom i recite: 'Svi na okup! Zavucimo se u gradove svoje utvrđene!' 
\par 6 Dižite znak prema Sionu! Bježite! Nemojte zastajati! Jer dovodim nesreću sa Sjevera, veliku propast. 
\par 7 Lav se podiže iz čestara svoga, zatornik naroda izađe iz svog mjesta, krenu da zemlju tvoju opustoši: gradove će tvoje razoriti, nestat će im žitelja. 
\par 8 Zato se u kostrijet ogrnite, kukajte, naričite, jer rasplamtjela jarost Jahvina nas nije mimoišla. 
\par 9 U dan onaj - riječ je Jahvina - klonut će srce kralju i knezovima. Svećenici će se zapanjiti, proroci umuknuti. 
\par 10 I reći će: 'Ah, Jahve, Gospodine, zaista nas teško prevari kad reče: 'Uživat ćete mir' a sad nam je mač pod grlom.' 
\par 11 U to će se vrijeme reći narodu ovom i Jeruzalemu: Vruć vjetar s pustinjskih sipina puše prema kćeri naroda moga; ali ne da hladi i da pročisti! 
\par 12 Doći će mi vjetar pun prijetnje, i ja ću im tada izreći sud! 
\par 13 Gle: diže se k'o oblačine, kola mu slična vihoru, konji brži od orlova. Jao nama! Propadosmo! 
\par 14 Operi opačinu sa srca svoga, Jeruzaleme, da bi se spasio. Dokle će se u grudima tvojim misli zločinačke gnijezditi? 
\par 15 Jer glas naviješta od Dana, s gore Efrajimove najavljuje nesreću. 
\par 16 Opomenite, razglasite po Judeji, obznanite Jeruzalemu: neprijatelji dolaze iz daleke zemlje i poklike izvikuju protiv gradova judejskih; 
\par 17 poput čuvara poljskih okružuju Jeruzalem, jer se odmetnu od mene" - riječ je Jahvina. 
\par 18 Put tvoj i djela tvoja to ti učiniše. To je tvoja nesreća! Kako je gorka, kako li pogađa u srce! 
\par 19 Utrobo moja! Utrobo moja, bolujem, srce mi se razdire! Dršće mi duša! Ne mogu šutjeti, jer čujem glas roga, poklike bojne. 
\par 20 Javljaju slom za slomom, sva je zemlja poharana, moji su šatori iznenada opustošeni, u tren oka sva skloništa moja uništena. 
\par 21 Dokle ću gledati bojne znakove, slušati pozive roga? 
\par 22 Da, bezuman je moj narod, ne poznaju me, djeca su oni nerazumna, ništa ne shvaćaju, mudri su tek za zlodjela, al' činiti dobro ne umiju. 
\par 23 Gledam zemlju: pusta je, evo, i prazna, nebesa: svjetlost im iščezla. 
\par 24 Gledam brda: gle, tresu se, a svi se humci uzdrmali. 
\par 25 Gledam: evo čovjeka nema, ptice nebeske sve su odletjele. 
\par 26 Gledam: plodno polje, evo, opustje, sve gradove razori Jahve žestinom gnjeva svoga. 
\par 27 Da, ovako govori Jahve: "Sva će zemlja biti poharana, ja ću joj zadati posljednji udarac. 
\par 28 Na to će se zemlja u crno zaviti, a nebesa, gore, potamnjeti. Jer rekoh, i neću se raskajati, odlučih i neću odustati. 
\par 29 Pred vikom 'Konjanici i strijelci!' sva se zemlja u bijeg dade: bježe u šipražje, penju se na hridi: svi su gradovi napušteni: nigdje više žive duše. 
\par 30 A ti, opustošena, što ćeš učiniti? Da se i grimizom zaodjeneš, nakitom zlatnim ukrasiš i oči ličilom izraniš, uzalud se uljepšavaš! Ljubavnici tvoji tebe preziru: traže glavu tvoju. 
\par 31 Da, jauk kao u bolesnice čujem, vrisak kao u one što prvi put rađa; čuj, to kći sionska jeca i pruža ruke: 'Jao meni! Duša mi zamire pod udarcima ubojica!' 


\chapter{5}

\par 1 Prođite ulicama jeruzalemskim, pogledajte dobro i raspitajte se, tražite po njegovim trgovima, pa ako nađete ijednoga čovjeka koji čini pravo i traži istinu, oprostit ću ovom gradu" - riječ je Jahvina. 
\par 2 Pa kad i govore: "Živoga mi Jahve!" doista se krivo zaklinju. 
\par 3 Jahve, nisu li oči tvoje upravljene k istini? Biješ ih, ali njih ne boli; zatireš ih, al' oni odbijaju pouku tvoju. Čelo im je tvrđe od litice, odbijaju da se obrate. 
\par 4 Rekoh: "Samo siromasi tako ludo postupaju, jer ne znaju puta Jahvina ni pravo Boga svojega. 
\par 5 Poći ću, dakle, velikima i njima ću govoriti, jer oni poznaju put Jahvin i pravo Boga svojega." Ali oni svi složno razbiše jaram i sve veze pokidaše. 
\par 6 I zato ih šumski lav napada, vuk pustinjski razdire, leopardi vrebaju gradove njihove, tko god iziđe iz njih bit će rastrgan. Jer su grijesi njihovi mnogobrojni, mnogostruki otpadi njihovi. 
\par 7 "Zašto da ti oprostim? Sinovi me tvoji napustiše, zaklinju se lažnim bogovima. Ja ih nasitih, oni preljub učiniše, u bludničinu kuću nagrnuše. 
\par 8 Oni su k'o ugojeni, sileni konji: ržu za ženom bližnjega svoga. 
\par 9 Pa da to ne kaznim - riječ je Jahvina - narodu takvu da se ne osvetim? 
\par 10 Popnite se na zidove! Razarajte! Uništite, ali ne posvema! Iščupajte sve čokote jer nisu Jahvini. 
\par 11 Da, podlo me izdadoše dom Izraelov i dom Judin" - riječ je Jahvina. 
\par 12 Zanijekaše Jahvu, rekoše: "Nema ga! Zlo nas neće snaći, nećemo iskusiti ni gladi ni mača! 
\par 13 [13a] A proroci su poput vjetra, govornika nema među njima!" 
\par 14 Zato ovako govori Jahve, Bog nad Vojskama: "Zato što su tako govorili, [13b] evo što će im se zbiti: U oganj ću pretvoriti svoje riječi u tvojim ustima, a narod ovaj u drvo da ga oganj proždre. 
\par 15 Evo, dovest ću na vas narod izdaleka, dome Izraelov - riječ je Jahvina. Narod nepobjediv, narod drevan, narod kojega jezik nećeš znati, ni razumjeti što govori. 
\par 16 Tobolac mu je razjapljen grob. Svi su oni po izboru junaci. 
\par 17 On će proždrijet' tvoju žetvu, tvoj kruh, sinove i kćeri tvoje, ovce i goveda tvoja, grožđe i smokve tvoje, razorit će ti gradove tvrde u koje se sada uzdaš." 
\par 18 "Ali ni tada - riječ je Jahvina - neću te posve uništiti. 
\par 19 A kad budu pitali: 'Zašto nam Jahve, Bog naš, učini sve  ovo?' ti ćeš im odgovoriti: 'Jer ste mene ostavili da biste služili  tuđim bogovima u svojoj zemlji, služit ćete tuđincu u zemlji  koja nije vaša!'" 
\par 20 "Objavite ovo domu Jakovljevu i obznanite po Judeji: 
\par 21 Čujte, dakle, ovo, narode ludi i nerazumni: oči imaju, a ne vide, uši imaju, a ne čuju! 
\par 22 Zar se mene nećete bojati - riječ je Jahvina - zar nećete drhtati preda mnom koji sam stavio pijesak moru za granicu, za vječnu među koje nikad neće prijeći: ono se biba, al' je nemoćno, valovi mu huče, ali prijeći neće. 
\par 23 No, u naroda ovog srce je prkosno, nepokorno; oni se udaljiše - to je snaga njihova! 
\par 24 Ne rekoše u srcu svome: 'Bojmo se Jahve, Boga svojega, koji nam u pravi čas šalje dažd rani i kišu kasnu i koji nam čuva tjedne određene za žetvu.' 
\par 25 Vaša bezakonja narušiše ovo, vaši vam grijesi uništiše blagostanje. 
\par 26 Da, u mome narodu ima zlikovaca: kao ptičari vrebaju iz zasjede, postavljaju zamke, hvataju ljude. 
\par 27 Kao što je krletka puna ptica, tako su njihove kuće pune grabeža; postadoše tako veliki i bogati, 
\par 28 tusti i ugojeni. Da, prevršila se mjera zla, ne brane prava, prava sirote ne sreću, ne mare za pravo sirotinje. 
\par 29 Pa da to ne kaznim - riječ je Jahvina - narodu takvu da se ne osvetim? 
\par 30 Strahote i grozote zbivaju se u ovoj zemlji: 
\par 31 proroci laž proriču, a svećenici poučavaju na svoju ruku. A mojem narodu to omilje! Al' što ćete raditi na kraju? 


\chapter{6}

\par 1 Bježite, sinovi Benjaminovi, isred Jeruzalema! U Tekoi zatrubite u rog, na Bet Hakeremu podignite bojni stijeg! Jer sa Sjevera se nadvija nesreća, propast velika. 
\par 2 Može li se Kći sionska usporedit' s nježnom tratinom? 
\par 3 K njoj dolaze pastiri sa stadima. Svud oko nje razapeše šatore, svaki pase na dijelu svome. 
\par 4 S njome zametnite sveti boj! Na noge! Navalimo usred dana! Jao nama, jer se dan naginje k zapadu, a večernje sjene duljaju! 
\par 5 Na noge! Navalimo usred noći, razrušimo dvore njene!" 
\par 6 Jer ovako zbori Jahve nad Vojskama: "Oborite stabla njena, podignite nasip oko Jeruzalema, to je grad laži, u njemu je sve samo tlačenje. 
\par 7 Kao što iz studenca izvire voda, tako iz njega opačina izvire. U njemu se čuje samo nasilje i pustošenje, preda mnom su svagda bolesti i rane. 
\par 8 Popravi se, Jeruzaleme, da mi se duša od tebe ne odvrati, da te ne pretvorim u pustoš, u zemlju nenastanjenu." 
\par 9 Ovako govori Jahve nad Vojskama: "Paljetkuj, paljetkuj kao lozu Ostatak Izraelov! Poput berača pruži ruke među čokote!" 
\par 10 Komu treba da govorim, koga da opomenem da me saslušaju? Gle, uho im je neobrezano stog ne mogu čuti. Gle, riječ Jahvina postade im porugom, nije im mila. 
\par 11 Prepun sam gnjeva Jahvina, ne mogu više izdržati! - Izlij ga, dakle, po djeci na ulici i na skupove mladića. Sve će ih obuzeti: muža i ženu, starca i čovjeka zrele dobi. 
\par 12 Njihove će kuće pripasti drugima, a tako i polja i žene im. "Da, ispružit ću ruku svoju - govori Jahve - na stanovnike ove zemlje, 
\par 13 jer od najmanjega do najvećeg svi gramze za plijenom, od proroka do svećenika svi su varalice. 
\par 14 I olako liječe ranu naroda moga, vičući: 'Mir! Mir!' Ali mira nema. 
\par 15 Nek' se postide što gnusobu počiniše, no oni više ne znaju što je stid, ne umiju se više crvenjeti. I zato će popadati s onima što padaju, srušit će se kad stanem kažnjavati" - govori Jahve. 
\par 16 Ovako govori Jahve: "Stanite na negdašnje putove, raspitajte se za iskonske staze: Koji put vodi k dobru? Njime pođite i naći ćete spokoj dušama svojim! Al' oni rekoše: 'Ne idemo!' 
\par 17 I postavih im stražare: 'Pazite na glas roga!' Al' oni rekoše: 'Nećemo paziti!' 
\par 18 Zato čujte, narodi, i vi pastiri stada njihovih! 
\par 19 Čuj, zemljo! Gle, dovodim zlo na ovaj narod, plod njihove pobune, jer oni ne slušahu riječi moje, Zakon moj odbaciše. 
\par 20 Što će mi tamjan koji dolazi iz Šebe i trska mirisna iz zemlje daleke? Vaše mi paljenice nisu drage, nisu mi po volji klanice vaše." 
\par 21 Zato ovako govori Jahve: "Evo postavljam narodu ovome prepreke o koje će se spotaći, oci i djeca zajedno, poginut će susjed zajedno s prijateljem." 
\par 22 Ovako govori Jahve: "Evo dolazi narod iz zemlje sjeverne, puk velik diže se s krajeva zemlje: 
\par 23 u ruci im luk i koplje, okrutni su, nemilosrdni; graja im buči kao more, jašu na konjima, kao jedan za boj spremni protiv tebe, Kćeri sionska. 
\par 24 Kad saznasmo novost, ruke nam klonuše, strava nas obuze, bol kao porodilju. 
\par 25 Ne izlazite u polja, ne idite na putove, jer mačevi dušmanski prijete, užas sve uokolo. 
\par 26 Kćeri mog naroda, kostrijet pripaši, pospi se pepelom, nariči k'o za jedincem tužaljku pregorku. Jer će doći nenadano na nas pustošnik. 
\par 27 Postavih te za ispitivača naroda mojeg da spoznaš i ispitaš putove njegove. 
\par 28 Svi su oni odmetnici najgori, okolo kleveću i mjed i željezo, svi su pokvareni. 
\par 29 Mijeh sopće da bi vatra proždrla olovo, zalud se ljevač trudi da ga rastopi: šljaka se ne da izlučiti." 
\par 30 "Srebro odbačeno", tako ih zovu, jer ih Jahve odbaci! 


\chapter{7}

\par 1 Ovo je riječ što dođe Jeremiji od Gospodina: 
\par 2 "Stani pred  vrata Doma Jahvina, objavi ondje ovu riječ. Reci: Čujte riječ  Jahvinu, svi Judejci koji ulazite na ova vrata da se poklonite  Jahvi. 
\par 3 Ovako govori Jahve nad Vojskama, Bog Izraelov: 'Popravite  svoje putove i djela svoja, pa ću boraviti s vama na ovome mjestu. 
\par 4 Ne uzdajte se u lažne riječi: 'Svetište Jahvino! Svetište  Jahvino! Svetište Jahvino!' 
\par 5 Ali ako zaista popravite svoje  putove i djela svoja i ako zaista budete činili što je pravo, svatko prema bližnjemu svome, 
\par 6 ako ne budete tlačili stranca, sirote i udovice i ne budete prolijevali krvi nedužne na ovome  mjestu, ako ne budete trčali za tuđim bogovima na svoju nesreću  - 
\par 7 boravit ću s vama na ovome mjestu, u zemlji koju sam dao  vašim ocima zauvijek. 
\par 8 Ali se vi uzdate u lažne i beskorisne  riječi! 
\par 9 Zar da kradete, ubijate, činite preljub, krivo se  zaklinjete, palite tamjan Baalu i trčite za tuđim bogovima kojih  ne poznajete, 
\par 10 a onda da dolazite i stojite preda mnom u Domu  ovome koji nosi moje ime i govorite: 'Spašeni smo!' i da nakon  toga opet činite nedjela i opačine? 
\par 11 Zar je Dom ovaj, koji  se zove mojim imenom, u vašim očima pećina razbojnička? Ali ja  dobro vidim" - riječ je Jahvina. 
\par 12 "Pođite, dakle, na moje mjesto koje je u Šilu, gdje nekoć  nastanih ime svoje, i pogledajte što od njega učinih zbog opačina  naroda svoga izraelskoga. 
\par 13 Kako činite sva ona ista nedjela  - riječ je Jahvina - i premda vas neumorno opominjem, vi ne slušate, a kad vas zovem, vi se ne odazivate: 
\par 14 od ovoga Doma što se  zove mojim imenom, u koje se vi uzdate, i od ovoga mjesta što  ga dadoh vama i ocima vašim učinit ću isto što sam učinio i od  Šila. 
\par 15 Odbacit ću vas od lica svojega kao što odbacih svu  braću vašu, sve potomstvo Efrajimovo." 
\par 16 "A ti ne moli milosti za narod ovaj, ne diži glasa za  njih i ne moli, ne navaljuj na me jer te neću uslišiti. 
\par 17 Ne  vidiš li što čine po gradovima judejskim i po ulicama jeruzalemskim? 
\par 18 Djeca kupe drva, oci pale vatru, žene mijese tijesto da ispeku  kolače 'kraljici neba' i lijevaju ljevanice tuđim bogovima da  me pogrde. 
\par 19 Zar mene tim pogrđuju - riječ je Jahvina - a ne  sebe na svoju sramotu?" 
\par 20 I stoga ovako govori Jahve Gospod:  "Evo, gnjev svoj i jarost svoju izlit ću na ovo mjesto, na ljude  i na stoku, na poljsko drveće i na plodove zemlje, rasplamtjet  će se i neće se ugasiti." 
\par 21 Ovako govori Jahve nad Vojskama, Bog Izraelov: "Paljenicama  dometnite još i klanice, i jedite meso. 
\par 22 Ja ništa ne rekoh  ocima vašim o paljenicama i klanicama, niti im što o tom zapovjedih  kad ih izvedoh iz zemlje egipatske. 
\par 23 Ovo im ja zapovjedih:  'Slušajte glas moj, pa ću ja biti vaš Bog, a vi ćete biti moj  narod. Idite putem kojim vam zapovjedih, da vam dobro bude.' 
\par 24 A oni ne poslušaše, uho svoje ne prignuše, već pođoše po  savjetu i okorjelosti zloga srca svojega; okrenuše mi leđa, a  ne lice. 
\par 25 Od dana kad oci vaši iziđoše iz zemlje egipatske  pa do dana današnjeg slao sam vam tolike sluge svoje, proroke, iz dana u dan, neumorno. 
\par 26 Ali me oni nisu slušali, uho svoje  nisu prignuli, nego otvrdnuše, gori od otaca svojih. 
\par 27 Možeš  im sve to reći, ali te neće poslušati; zovi ih, neće ti se odazvati. 
\par 28 Zato im reci: 'Ovo je narod koji ne sluša glasa Jahve, Boga  svojega, i ne prima opomenÄe. Nestade istine, nestade je iz usta  njihovih.'" 
\par 29 Ostriži svoju dugu kosu, baci je. Po goletima protuži tužaljkom, jer Jahve odbaci i odvrgnu rod na koji se razgnjevio. 
\par 30 "Da, sinovi Judini čine što je zlo u očima mojim" - riječ  je Jahvina. "Postaviše grozote u Dom koji se mojim zove imenom, da ga oskvrnu; 
\par 31 podigoše uzvišice tofetske u Dolini Ben Hinomu  i spaljuju vatrom svoje sinove i kćeri - što im ja nikad ne zapovjedih  niti mi to ikada na um pade. 
\par 32 Stoga evo dolaze dani - riječ  je Jahvina - kad se više neće reći Tofet ni Dolina Ben Hinom, nego Dolina klanja. U Tofetu će se pokapati mrtvi, jer drugdje  neće biti mjesta. 
\par 33 A mrtva tijela ovoga naroda bit će hrana  pticama nebeskim i zvjeradi zemaljskoj, i nitko se neće naći  da ih poplaši i otjera. 
\par 34 Uklonit ću iz gradova judejskih i  s ulica jeruzalemskih radost i veselje: jer će se zemlja ta pretvoriti  u pustinju." 


\chapter{8}

\par 1 "U ono vrijeme - riječ je Jahvina - povadit će iz grobova kosti  kraljeva judejskih, kosti knezova njezinih, kosti svećenika,  kosti proroka i kosti žitelja jeruzalemskih. 
\par 2 I razasut će  ih prema suncu, prema mjesecu i prema svoj vojsci nebeskoj, koje  ljubljahu, kojima služahu, koje slijeđahu, koje za savjet pitahu  i kojima se klanjahu. I neće ih pokupiti i sahraniti; ostat će  kao gnoj po zemlji. 
\par 3 Tada će svima onima što preostanu od tih  zlih plemena, po svim mjestima kuda ih rasprših, smrt biti milija  od života" - riječ je Jahve nad Vojskama. 
\par 4 "Reci im: Ovako govori Jahve: 'Padne li tko, neće li opet ustati, zaluta li, neće li se opet vratiti? 
\par 5 Zašto onda taj narod luta uporno i neprekidno? Čvrsto se drže laži, neće da se obrate. 
\par 6 Pazio sam i osluškivao: Ne govore kako valja. Nitko se ne kaje zbog pakosti svoje, i ne govori 'Što učinih?' Svatko je skrenuo trku svoju kao konj kad u boj nagne. 
\par 7 Čak i roda pod nebom zna svoje vrijeme, grlica, lastavica i ždral drže se vremena kad se moraju vratiti. A moj narod ne poznaje suda Jahvina. 
\par 8 Kako možete tvrditi: 'Mi smo mudri, u nas je Zakon Jahvin!' Zaista, u laž ga je pretvorila lažljiva pisaljka pisara! 
\par 9 Mudraci će biti osramoćeni, prestravljeni i uhvaćeni u zamku. Gle, oni prezreše riječ Jahvinu! A njihova mudrost - što im koristi? 
\par 10 Zato ću žene njihove dati strancima, a vaša polja osvajačima. Jer od najmanjeg do najvećega svi gramze za plijenom, od proroka do svećenika svi su varalice. 
\par 11 I olako liječe ranu naroda mojega vičući: 'Mir! Mir!' Ali mira nema. 
\par 12 Neka se postide što gnusobu počiniše, no oni više ne znaju što je stid, ne umiju se više crvenjeti. Zato će popadati s onima što padaju, srušit će se kad stanem kažnjavati" - govori Jahve. 
\par 13 "Htjedoh u berbu k njima - riječ je Jahvina - a ono ni grozda na trsu, ni smokve na smokvi; čak je i lišće uvelo. Zato ih predah onima što prolaze kraj njih. 
\par 14 'Zašto još čekamo? Na okup! Zavucimo se u tvrde gradove da ondje izginemo, jer nas Jahve, Bog naš, zatire, napaja nas vodom otrovanom, jer zgriješismo protiv Jahve. 
\par 15 Nadasmo se miru, ali dobra nema, čekasmo vrijeme ozdravljenja, al' evo užasa! 
\par 16 Iz Dana dopire njištanje konja njegovih, od rzanja njegovih pastuha dršće zemlja sva. Dolaze da proždru zemlju i što je napunja, grad i žitelje u njemu.' 
\par 17 I gle, puštam na vas otrovnice protiv kojih nema čarolija; ujedat će vas - riječ je Jahvina - 
\par 18 lijeka biti neće." Bol me spopada, srce mi iznemoglo. 
\par 19 Evo zapomažu kćeri naroda moga iz zemlje daleke: "Zar Jahve nije više na Sionu? Kralj njegov? Zašto me razjariše svojim kipovima, ništavilima tuđinskim? 
\par 20 Žetva prođe, minu ljeto, a mi nismo spašeni!" 
\par 21 Satrven sam što je kći naroda mojega satrvena, žalostan sam, stravom obuzet. 
\par 22 Zar u Gileadu nema balzama? Nema li ondje liječnika? TÓa zašto ne dolazi ozdravljenje kćeri naroda mojega? 


\chapter{9}

\par 1 (8:23) O, tko bi glavu moju pretvorio u vrelo, a oči moje u vrutak suza, danju i noću da plačem nad poginulima kćeri svoje, naroda svojega! 
\par 2 (9:1) "Da imam u pustinji obitavalište, ostavio bih narod svoj i daleko od njih otišao. Jer svi su oni preljubnici, rulja izdajnička. 
\par 3 (9:2) Kao luk napinju jezik, laž, a ne istina, prevladava na zemlji. Iz zla u zlo srljaju, mene ne poznaju" - riječ je Jahvina! 
\par 4 (9:3) "Nek se svatko čuva prijatelja, a brat bratu neka ne vjeruje, jer brat svaki nasljeduje Jakova, a svaki prijatelj raznosi klevete. 
\par 5 (9:4) Jedan drugoga varaju, istine ne govore, privikoše jezik da govori laži; izopačeni, ne mogu se više 
\par 6 (9:5) vratiti. Nasilje na nasilje! Prijevara za prijevarom! Neće da spoznaju mene" - riječ je Jahvina. 
\par 7 (9:6) Stog ovako govori Jahve nad Vojskama: "Evo, pretopit ću ih i ispitati, tÓa kako da i postupaju prema kćeri naroda moga? 
\par 8 (9:7) Jezik im je strijela ubojita, na ustima riječ prijevarna. 'Mir s tobom', pozdravljaju bližnjega, ali mu u srcu zamku spremaju. 
\par 9 (9:8) Pa da ih zbog toga ne kaznim - riječ je Jahvina - narodu takvu da se ne osvetim?" 
\par 10 (9:9) "Zaplačite, tugujte nad brdima, nad ispašama pustinjskim naričite! Jer izgorješe, nitko ne prolazi, glas stada više se ne čuje. Od ptice nebeske do stoke sve pobježe, svega nestade. 
\par 11 (9:10) Od Jeruzalema učinit ću gomilu kamenja, brlog čagaljski, gradove judejske pretvorit ću u pustoš gdje nitko ne stanuje." 
\par 12 (9:11) Tko je mudar da bi to razumio, kome su usta Jahvina govorila da objavi zašto zemlja izgorje kao pustinja i nitko njome više ne prolazi? 
\par 13 (9:12) I reče Jahve: "Jer ostaviše Zakon moj koji im dadoh i  jer ne slušahu glasa mojega, niti ga slijeđahu, 
\par 14 (9:13) nego slijeđahu  okorjelo srce svoje i baale kojima ih oci njihovi naučiše, 
\par 15 (9:14) ovako  govori Jahve nad Vojskama, Bog Izraelov: Evo, nahranit ću narod  ovaj pelinom i napojiti ga vodom zatrovanom. 
\par 16 (9:15) I raspršit ću  ih među narode kojih ne poznavahu oni ni oci njihovi. A za njima  ću poslati mač da ih zatre." Ovako govori Jahve nad Vojskama: 
\par 17 (9:16) "Pazite! Pozovite narikače! Neka dođu! Pošaljite po najvještije! Neka dođu! 
\par 18 (9:17) Neka pohite da zapjevaju tužbalicu nad nama! Da suze poteku iz očiju naših, da voda poteče s trepavica naših! 
\par 19 (9:18) Sa Siona dopire tužbalica: 'O, kako smo upropašteni, osramoćeni veoma! Jer moramo zemlju ostaviti i stanove svoje napustiti.'" 
\par 20 (9:19) I zato, o žene, čujte riječ Jahvinu, i neka uho vaše primi riječ iz usta njegovih. Učite kćeri svoje jadati, jedna drugu naricati: 
\par 21 (9:20) "Smrt se ušulja kroz prozore naše, uđe u dvore naše, djecu pokosi nasred ulice, mladiće nasred trgova. 
\par 22 (9:21) I mrtva tjelesa leže kao gnoj po oranicama, kao snoplje za žeteocem, a nikoga da ih skupi." 
\par 23 (9:22) Ovako govori Jahve: "Mudri neka se ne hvale mudrošću, ni junak neka se ne hvali hrabrošću, ni bogati neka se ne hvali bogatstvom. 
\par 24 (9:23) A tko se hvaliti hoće, neka se hvali time što ima mudrost da mene spozna. Jer ja sam Jahve koji tvori dobrotu, pravo i pravdu na zemlji, jer to mi je milo" - riječ je Jahvina. 
\par 25 (9:24) "Evo, bliže se dani" - riječ je Jahvina - "kaznit ću  sve koji su obrezani na tijelu: 
\par 26 (9:25) Egipat, Judeju, Edom, sinove  Amonove i Moab i sve one što briju zaliske i prebivaju u pustinji.  Jer svi su ti narodi neobrezani i sav dom Izraelov neobrezana  je srca!" 


\chapter{10}

\par 1 Slušajte riječ koju vam govori Jahve, dome Izraelov. 
\par 2 Ovako govori Jahve: "Ne privikavajte se putu bezbožnom i ne dršćite pred znacima nebeskim, jer pred njima dršću samo bezbošci. 
\par 3 Jer su strašila tih naroda puka ništavnost, samo drvo posječeno u šumi, djelo ruku tesarovih, 
\par 4 ukrašeno srebrom i zlatom, pričvršćeno čavlima i čekićima da se ne klima. 
\par 5 Nalik su na ptičja strašila u vrtu: ne znaju govoriti. Treba ih nositi, jer ne umiju hodati. Njih se ne bojte, jer ne mogu zla činiti, ali ni dobra učiniti ne mogu." 
\par 6 Nitko nije kao ti, Jahve, ti si velik i silno je ime tvoje. 
\par 7 Tko da se tebe ne boji, kralju naroda? Zaista, tebi to pripada, jer među svim mudracima naroda i u svim njihovim kraljevstvima tebi nema ravna! 
\par 8 Čime vatru lože, to ih zaluđuje! Zakon im isprazan - obično drvo, 
\par 9 tankolisto srebro dovezeno iz Taršiša, zlato iz Ofira, rad kipara i rukotvorina zlatara, sva djela umješna, ogrnuta ljubičastim i crvenim grimizom. 
\par 10 A Jahve je pravi Bog. Živi je on Bog i Kralj vječni. Od njegova gnjeva zemlja se trese. Narodi ne mogu podnijeti jarosti njegove. 
\par 11 Evo što ćete o kipovima reći: "Bogovi koji nisu stvorili  neba ni zemlje moraju nestati s lica zemlje i ispod neba." 
\par 12 On stvori zemlju snagom svojom, mudrošću svojom uspostavi krug zemaljski i umom svojim razape nebesa. 
\par 13 Kad mu glas zaori, huče vode na nebesima, oblake diže s kraja zemlje; stvara kiši munje, vjetar izvodi iz skrovišta njegovih. 
\par 14 Svakom čovjeku pamet stane, svaki se zlatar zastidi svoga kipa, jer svi su mu kipovi samo varka, nema u njima duha! 
\par 15 Isprazni su oni, smiješne tvorevine, propast će u dan kazne. 
\par 16 'Jakovljev dio' nije kao oni: jer je on sve stvorio, Izrael pleme je baštine njegove, Jahve nad Vojskama ime je njegovo." 
\par 17 Skupi prnje svoje sa zemlje, ti koja stanuješ u utvrdi! 
\par 18 Jer ovako govori Gospod: "Gle, ovaj put daleko ću odbaciti stanovnike ove zemlje, pritijesniti ih da me nađu." 
\par 19 "Jao meni zbog ozljede moje, rana je moja neiscjeljiva." A ja rekoh: "Ipak, bolest je moja, nosit ću je! 
\par 20 Šator je moj obÄaljen, sva užeta pokidana. Djeca me ostaviše: nema ih. Nema ga tko bi opet razapeo šator moj i podigao krila šatorska." 
\par 21 Da, pastiri pamet izgubiše: ne tražiše Jahve. Zato ih sreća ne prati i sva se stada raspršiše. 
\par 22 Čujte vijest! Primiče se, evo, buka strašna iz zemlje sjeverne, da gradove judejske pretvori u pustinju, u brlog čagalja. 
\par 23 Znam, Jahve, da put čovjeka nije u njegovoj vlasti, da čovjek koji hodi ne može upravljati korake svoje! 
\par 24 Kazni me, Jahve, ali po pravici, ne u gnjevu, da nas ne zatreš. 
\par 25 Izlij gnjev na narode koji te ne priznaju i na plemena koja ne zazivlju imena tvoga! Jer oni su proždrli Jakova, izjeli ga, opustošili naselje njegovo. 


\chapter{11}

\par 1 Riječ koju je Jahve uputio Jeremiji: 
\par 2 "Govori Judejcima i Jeruzalemcima. 
\par 3 Reci im: Ovako veli  Jahve, Bog Izraelov: 'Proklet bio čovjek koji ne posluša riječi  Saveza ovoga, 
\par 4 riječi koje sam zapovjedio ocima vašim kad sam  ih izveo iz zemlje egipatske, iz one peći ražarene, govoreći:  Poslušajte glas moj i činite sve što vam zapovjedim: tada ćete  biti narod moj, a ja vaš Bog, 
\par 5 da bih ispunio zakletvu kojom  sam se zakleo ocima vašim da ću im dati zemlju u kojoj teče mlijeko  i med - kao što je danas.'" A ja odgovorih i rekoh: "Tako je, Jahve." 
\par 6 I dalje mi reče Jahve: "Objavi riječi ove po gradovima  judejskim i po ulicama jeruzalemskim: 'Poslušajte riječi Saveza  ovoga, te ih izvršavajte. 
\par 7 Jer sam ozbiljno opominjao očeve  vaše kad sam ih izvodio iz zemlje egipatske i do danas ih neumorno  opominjem: Poslušajte glas moj! 
\par 8 Ali oni ne slušahu i ne prignuše  uha svojega, nego se povedoše za okorjelošću zloga srca svoga.  Zato dopustih da se na njima ispune sve riječi Saveza ovoga za  koji im zapovjedih da ga se pridržavaju, ali ga se oni ne pridržavahu.'" 
\par 9 I reče mi Jahve: "Zavjera je među Judejcima i Jeruzalemcima. 
\par 10 Vratiše se bezakonjima svojih otaca koji se oglušiše na moje  riječi, pa trčahu za tuđim bogovima da im služe. Dom Izraelov  i dom Judin prekršiše Savez moj koji sam sklopio s ocima njihovim." 
\par 11 Zato ovako govori Jahve: "Evo, dovest ću na njih zlo  kojemu neće umaći; vapit će k meni, ali ih ja neću slušati. 
\par 12 Onda  neka gradovi judejski i žitelji jeruzalemski vapiju k bogovima  kojima kade, ali im oni neće pomoći u vrijeme nevolje! 
\par 13 Jer imaš, o Judejo, bogova koliko i gradova! I koliko ima ulica u Jeruzalemu, toliko žrtvenika podigoste da kadite Baalu. 
\par 14 Ti, dakle, ne moli milosti za taj narod, ne diži glasa  za njih i ne moli, jer ih neću uslišiti kad me zazovu u nevolji  svojoj." 
\par 15 Što li će draga moja u Domu mome? Kuje zle osnove. Hoće li pretilina i meso posvećeno ukloniti zlo od tebe? Mogu li te stoga proglasiti čistom? 
\par 16 "Zelena maslina lijepa uzrasta", tako te Jahve nazva. A sada uz prasak veliki plamenom sažiže njeno lišće; spaljene su grane njene. 
\par 17 Jahve nad Vojskama, koji te bijaše posadio, nesreću ti  namijeni zbog zločina što ga učini dom Izraelov i dom Judin kadeći  Baalu da bi mene razgnjevili. 
\par 18 Jahve mi objavi te znam! Tada mi ti, Jahve, razotkri  njihove spletke. 
\par 19 A ja bijah kao jagnje krotko što ga vode  na klanje i ne slutih da protiv mene snuju pakosne naume. "Uništimo  drvo još snažno, iskorijenimo ga iz zemlje živih, da mu se ime  nikad više ne spominje!" 
\par 20 Ali ti, Jahve nad Vojskama, koji pravedno sudiš, koji ispituješ srca i bubrege, daj da vidim kako se njima osvećuješ, jer tebi povjerih parnicu svoju. 
\par 21 Zato Jahve nad Vojskama govori protiv ljudi u Anatotu  koji mi rade o glavi i govore: "Ne prorokuj više u ime Jahvino, da ne pogineš od ruke naše!" 
\par 22 Ovako govori Jahve nad Vojskama:  "Evo, ja ću ih kazniti. Njihovi će mladići od mača poginuti,  sinovi i kćeri pomrijet će od gladi. 
\par 23 Ni ostatka neće ostati  kad donesem nesreću ljudima u Anatotu u godini kazne njihove." 


\chapter{12}

\par 1 Prepravedan si, Jahve, da bih se mogao s tobom parbiti. Samo bih jedno s tobom raspravio: Zašto je put zlikovaca uspješan? Zašto podmuklice uživaju mir? 
\par 2 Ti si ih posadio, i oni se ukorijeniše, rastu i plod donose. Al' si bliz samo ustima njihovim, a daleko im od srca. 
\par 3 No ti, Jahve, mene poznaješ i vidiš; ispitao si srce moje, ono je s tobom. Odvedi ih kao jagnjad na klanje, sačuvaj ih za dan pokolja. 
\par 4 Dokle će zemlja tugovati, dokle će trava na svem polju  sahnuti? Zbog opačine njezinih stanovnika ugiba stoka i ptice! Jer govore: Bog ne vidi naših putova. 
\par 5 Ako s pješacima trčeći sustaješ, kako ćeš se s konjima utrkivati? Kad ni u mirnoj zemlji nemaš uzdanja, kako ćeš onda kroz guštare jordanske? 
\par 6 Jer su i braća tvoja i obitelj tvoja licemjerni prema tebi! I oni te iza leđa ocrnjuju na sva usta. Ne vjeruj im kad ti zbore umilno. 
\par 7 Ostavih dom svoj, napustih baštinu svoju; miljenicu srca svoga dadoh u ruke dušmana njenih. 
\par 8 Baština moja postade za me kao lav u šumi. Zarikao je na me, zato ga zamrzih. 
\par 9 Zar je baština moja šarena ptica oko koje odasvud druge slijeću? Hajde, skupite se, sve divlje zvijeri, dođite žderati. 
\par 10 Mnogi pastiri opustošiše moj vinograd, zgaziše nasljedstvo moje; dragu mi baštinu pretvoriše u golu pustinju, 
\par 11 pretvoriše u pustoš, žalosna je pustoš preda mnom. Sva je zemlja pusta jer nikog u srce ne dira. 
\par 12 Preko svih goleti pustinjskih nagrnuše pustošnici. Jer u Jahve je mač što proždire: s jednog kraja zemlje do drugog nema mira nijednome tijelu. 
\par 13 Sijahu pšenicu, a požeše trnje: iscrpli se bez koristi. Stide se uroda svoga sve zbog jarosti Jahvine. 
\par 14 Ovako govori Jahve: "Sve zle susjede svoje, koji su dirnuli  u baštinu što sam je dao narodu svome Izraelu, ja ću iščupati  iz zemlje njihove. Ali dom Judin iščupat ću isred njih. 
\par 15 A  kad ih iščupam, ponovo ću im se smilovati i povesti natrag, svakoga  na baštinu i zemlju njegovu. 
\par 16 Pa ako doista nauče putove naroda  mojega i stanu se zaklinjati imenom mojim - 'Živoga mi Jahve'  - kao što su učili moj narod da se zaklinje Baalom, tada će se  opet nastaniti usred naroda moga. 
\par 17 Ako pak ne poslušaju, onda  ću takav narod potpuno iščupati i zatrti" - riječ je Jahvina. 


\chapter{13}

\par 1 Ovako mi govori Jahve: "Idi i kupi sebi lanen pojas i opaši  bokove. Ali ga u vodu ne umači." 
\par 2 I kupih pojas po riječi Jahvinoj  i opasah bokove. 
\par 3 I dođe mi drugi put riječ Jahvina: 
\par 4 "Uzmi  pojas što si ga kupio i njime se opasao, digni se, idi do rijeke  Eufrata i sakrij ga ondje u pukotinu pećine." 
\par 5 I odoh i sakrih  ga kraj Eufrata, kako mi Jahve zapovjedi. Poslije mnogo dana  reče mi Jahve: 
\par 6 "Ustaj, idi na Eufrat pa izvuci odande pojas  za koji ti zapovjedih da ga ondje sakriješ." 
\par 7 Odoh na Eufrat, izvukoh i uzeh pojas s mjesta gdje ga bijah sakrio, i gle: pojas  istrunuo, ne bijaše više nizašto. 
\par 8 Tada mi dođe riječ Jahvina: 
\par 9 "Ovako govori Jahve: Tako ću uništiti silnu oholost Judeje  i Jeruzalema. 
\par 10 Narod taj opaki koji ne sluša mojih riječi, nego slijedi okorjelo srce svoje i trči za drugim bogovima da  im služi i da im se klanja, postat će kao tvoj pojas koji nije  više nizašto. 
\par 11 Jer kao što pojas prianja uz bedra čovjekova, tako sam htio da sav dom Izraelov i sav dom Judin prianja uza  me - riječ je Jahvina - da budu moj narod, moj dobar glas, moj  ponos, moja slava i čast. Ali nisu poslušali!" 
\par 12 Reci tom narodu: "Svaki se vrč puni vinom." A oni će  ti prigovoriti: "Zar možda ne znamo da se svaki vrč puni vinom?" 
\par 13 Reci im tada: "Ovako govori Jahve: evo, napunit ću pijanošću  sve stanovnike ove zemlje, kraljeve što sjede na prijestolju  Davidovu, i svećenike, i proroke, i sve Jeruzalemce. 
\par 14 I porazbijat  ću ih jednog o drugoga, očeve zajedno sa sinovima - riječ je  Jahvina. Uništit ću ih bez samilosti, bez milosrđa i bez smilovanja." 
\par 15 Poslušajte, dobro čujte, okanite se oholosti: Jahve sad govori! 
\par 16 Dajte slavu Jahvi, Bogu svojemu, prije nego što se smrkne, prije nego što se noge vaše spotaknu po planinama mračnim. Vi se nadate svjetlosti, a on će je u mrak pretvoriti, prometnuti u crnu tamu! 
\par 17 Ako ovo ne poslušate, potajno će mi duša plakati zbog oholosti vaše, suze će roniti, oko će mi suze prolijevati, jer Jahvino stado u izgnanstvo odlazi. 
\par 18 Reci kralju i kraljici-majci: "Sjednite duboko dolje, jer vijenac slave pade s vaših glava. 
\par 19 Gradovi Negeba zatvoreni su, i nikoga nema da ih otvori. Sva je Judeja izgnana, sasvim izgnana!" 
\par 20 Podigni oči, Jeruzaleme, i pogledaj one što nadiru sa Sjevera. Gdje je stado tebi povjereno, slavne ovce tvoje? 
\par 21 Što ćeš reći kada ti se nametnu kao gospodari tvoji oni koje si sam naučio da te kao ljubavnici vode. Neće li te bolovi spopasti kao porodilju? 
\par 22 Možda ćeš se tad upitati: "Zašto me to snašlo?" Zbog mnoštva bezakonja tvojih otkriše ti skute, nasilje nad tobom učiniše. 
\par 23 Može li Etiopljanin promijeniti kožu svoju? Ili leopard krzno svoje? "A vi, možete li činiti dobro, navikli da zlo činite? 
\par 24 Zato ću vas raspršiti k'o pljevu koju raznosi pustinjski vjetar. 
\par 25 To je sudba tvoja i dio tebi odmjeren - riječ je Jahvina - jer si mene zaboravio i u laž se uzdao. 
\par 26 Sam ću ti halju do lica podići da se tvoja golotinja vidi. 
\par 27 Sve preljube tvoje, tvoje vriskanje i bestidno tvoje bludničenje, na humcima, u poljima, vidio sam tvoje grozote. Jao tebi, Jeruzaleme! Još se ne očisti i dokle će to još trajati ...?" 


\chapter{14}

\par 1 Riječ Jahvina Jeremiji o velikoj suši: 
\par 2 Judeja je tugom obrvana i ginu njeni gradovi, sumorno leže na zemlji, krik Jeruzalema do neba se vije. 
\par 3 Odličnici šalju sluge po vodu: dolaze do studenaca, ali vode ne nalaze, vraćaju se praznih vrčeva, postiđeni, posramljeni, pokriše glavu svoju. 
\par 4 Zemlja je sva ispucala jer kiše nema. Ratari se postidješe, pokriše glave. 
\par 5 Pa i košuta u polju ostavlja mlado jer trave nema. 
\par 6 Divlji magarci, stojeć' na goletima, dašću kao čagalj, oči im malaksale jer nema zelenila. 
\par 7 Bezakonja naša protiv nas svjedoče, smiluj se, o Jahve, rad' imena svoga! Jer otpadosmo od tebe, tebi sagriješismo, 
\par 8 o nado Izraelova, spasitelju njegov u danima nevolje! Zašto si kao stranac u ovoj zemlji, kao putnik što se uvrati da prenoći? 
\par 9 Zašto si kao prestravljen čovjek, kao junak koji ne može pomoći? TÓa ti si među nama, o Jahve, mi se tvojim zovemo imenom - nemoj nas ostaviti! 
\par 10 Ovako govori Jahve o narodu ovome: Jest, oni vole tumarati  i ne štede svojih nogu, i zato ih Jahve ne voli. I sada se spominje  bezakonja njihova i kažnjava grijehe njihove. 
\par 11 I reče mi Jahve:  "Ne traži milosti za ovaj narod. 
\par 12 Ako će i postiti, neću uslišiti  njihovih vapaja. Ako će i prinijeti paljenicu i prinos, neće  mi omiljeti. Jer mačem, glađu i kugom ja ću ih zatrti." 
\par 13 Tada rekoh: "Ah, Jahve Gospode! Eno, proroci im govore:  'Nećete vidjeti mača, niti će vam biti gladi, nego ću vam dati  postojan mir na ovome mjestu.'" 
\par 14 A Jahve mi reče: "Laž prorokuju ti proroci u moje ime;  niti ih poslah niti im nalog kakav dadoh - ja im i nisam govorio.  Oni vam prorokuju lažna viđenja, isprazna gatanja i snove srca  svoga." 
\par 15 Zato ovako govori Jahve: "Proroci ti što u moje ime  prorokuju, a ja ih nisam poslao, te govore da neće biti ni mača  ni gladi u zemlji ovoj, sami će od mača i gladi poginuti. 
\par 16 A  narod ovaj kojemu prorokuju ležat će po ulicama jeruzalemskim, pokošen mačem i glađu, i neće biti čovjeka da ga pokopa - njih, žene njihove, sinove i kćeri njihove. Tako ću na njih izliti  zloću njihovu." 
\par 17 A ti im reci ovako: Nek' oči moje suze rone danju i noću, i neka ne prestanu, jer je strašno slomljena djevica, kći naroda moga, ranom neobično ljutom. 
\par 18 Pođem li u polje, evo mačem pobijenih! Vratim li se u grad, evo od gladi iznemoglih! Čak i proroci i svećenici lutaju po zemlji i ništa ne znaju. 
\par 19 TÓa zar si Judeju sasvim odbacio? Zar ti duši omrznu Sion? Zašto nas tako biješ te nam više nema lijeka? Nadasmo se miru, ali dobra nema, čekasmo vrijeme ozdravljenja, al' evo užasa! 
\par 20 O Jahve, bezbožnost svoju priznajemo, bezakonje otaca svojih; doista, tebi sagriješismo! 
\par 21 Ne odbaci nas, rad' imena svoga, ne sramoti prijesto Slave svoje, spomeni se i nemoj razvrći Saveza svog s nama! 
\par 22 Zar ispraznost bezbožnika dažda daje? Ili zar nebesa sama kiše? Zar ne daješ ti to, Jahve, Bože naš? Zato se u te uzdamo, jer ti sve to činiš. 


\chapter{15}

\par 1 I reče mi Jahve: "Kad bi i Mojsije i Samuel stali pred lice  moje, ne bi mi se duša obratila narodu tome. Otjeraj ih ispred  lice mojega, neka idu od mene! 
\par 2 Ako te upitaju: 'Kamo da idemo?'  odgovori im: 'Ovako govori Jahve: Tko je za smrt, u smrt! Tko je za mač, pod mač! Tko je za glad, u glad! Tko je za izgnanstvo, u izgnanstvo!' 
\par 3 Poslat ću na njih četiri nevolje - riječ je Jahvina: mač  da ih ubija, pse da ih rastrgaju, ptice nebeske i zvjerad da  ih žderu i zatiru. 
\par 4 I učinit ću ih užasom svim kraljevstvima  zemaljskim, i to zbog Manašea, sina Ezekijina, kralja judejskoga  - za sva zla što ih počini u Jeruzalemu." 
\par 5 "Tko da se smiluje tebi, Jeruzaleme, tko da te požali? Tko li će se svratit' da te zapita kako ti je? 
\par 6 Ti me odbaci - riječ je Jahvina - i leđa mi okrenu. I zato na te digoh ruku zatornicu. Dojadi mi da ti uvijek praštam! 
\par 7 Zato ću ih izvijati vijačom na vratima zemlje ove. Narod svoj ću lišit' djece i istrijebit' ga, jer se ne obraćaju sa svojih putova. 
\par 8 Bit će u njih više udovica negoli pijeska morskoga. Na majke mladih ratnika dovest ću zatornika, usred podneva, i pustit ću na njih iznenada užas i strahotu. 
\par 9 Onesvijestila se roditeljka sedmero djece, dušu ispustila. Sunce joj zađe još za dana: postiđena, osramoćena je. A što od njih ostane, pod mač ću vrći pred njihovim dušmanima" - riječ je Jahvina. 
\par 10 Jao meni, majko, što si me rodila, da svađam se i prepirem sa svom zemljom. Nikom ne uzajmih, ni od koga zajma ne uzeh, a ipak svi me proklinju. 
\par 11 Uistinu, o Jahve, nisam li ti služio za njihovo dobro, nisam li tražio milost u tebe za neprijatelja svoga, u doba nevolje, u danima tjeskobe njegove? Ti to dobro znaš! 
\par 12 Može l' se željezo slomiti, željezo sa Sjevera i mjed? 
\par 13 Tvoje bogatstvo i blago tvoje pljački ću predati. Tako ćeš platiti za sva bezakonja svoja po svoj zemlji. 
\par 14 Učinit ću te robljem neprijatelja u zemlji koju ne poznaješ, jer gnjev moj planu ognjem koji će vas sažgati, koji će protiv vas buknuti. 
\par 15 Jahve, spomeni me se i pohodi me i kazni progonitelje moje. Ne daj da propadnem zbog sporosti srdžbe tvoje! Spomeni se da tebe radi podnosim sramotu. 
\par 16 Kad mi dođoše riječi tvoje, ja sam ih gutao: riječi tvoje ushitiše i obradovaše srce moje. Jer sam se tvojim zvao imenom, o Jahve, Bože nad Vojskama. 
\par 17 Nikad sjedio nisam u društvu veseljaka da se razveselim. Pod težinom ruke tvoje samotan živim, jer ti me jarošću prože. 
\par 18 Zašto je bol moja bez prebola? Zašto je rana moja neiscjeljiva i nikako da zaraste? Ah! Hoćeš li meni biti kao potok nestalan, vodama nepouzdan? 
\par 19 Zato ovako govori Jahve: "Ako se vratiš, pustit ću te da mi opet služiš; ako odvojiš dragocjeno od bezvrijedna, bit ćeš usta moja. Oni će se okrenuti k tebi, al' ti se zato ne smiješ okrenuti k njima! 
\par 20 Učinit ću od tebe za ovaj narod zid od mjedi, neosvojiv. Borit će se protiv tebe, al' te neće nadvladati, jer ja sam s tobom, da te spasim i izbavim" - riječ je Jahvina. 
\par 21 "Izbavit ću te iz ruku zlikovaca i otkupiti te iz ruku silnika." 


\chapter{16}

\par 1 I dođe mi riječ Jahvina i reče: 
\par 2 "Ne uzimaj sebi žene; i  nemaj ni sinova ni kćeri na ovome mjestu. 
\par 3 Jer ovako govori  Jahve o kćerima i sinovima koji će se roditi na ovome mjestu  i o majkama koje će ih rađati i o očevima koji će ih imati u  ovoj zemlji: 
\par 4 Oni će umrijeti smrću prebolnom, nitko ih neće  oplakivati, niti će ih sahraniti. Pretvorit će se u gnoj za oranice, izginut će od mača i gladi, a njihova će trupla biti hrana pticama  nebeskim i zvijerima zemaljskim." 
\par 5 Da, ovako govori Jahve: "Ne smiješ ući u kuću žalosti, ne idi naricati niti ih sažaljevati. Jer ja sam tom narodu uskratio  mir svoj - riječ je Jahvina - ljubav i samilost. 
\par 6 Pomrijet  će veliko i malo u ovoj zemlji i nitko ih neće pokopati. Nitko  neće naricati nad njima, niti će zbog njih praviti ureza, niti  kose šišati. 
\par 7 Za onog u žalosti neće kruh lomiti, da ga utješe  zbog pokojnika, nitiće mu tko pružiti pehar utjehe zbog smrti  njegova oca ili majke njegove. 
\par 8 Ne ulazi u kuću slavlja da s njima sjediš i gostiš se." 
\par 9 Jer ovako govori Jahve nad Vojskama, Bog Izraelov: "Evo, učinit  ću da s ovog mjesta i pred vašim očima i u ovim danima iščeznu  poklici radosti i veselja i glasovi zaručnika i zaručnice. 
\par 10 A kad objaviš tom narodu sve ove riječi, pa te upitaju:  'Zašto nam Jahve zaprijeti svom ovom golemom nesrećom; u čemu  je zločinstvo naše i u čemu su grijesi naši što ih počinismo  protiv Jahve, Boga našega?' - 
\par 11 onda im odgovori: 'U tom što  me ostaviše oci vaši - riječ je Jahvina - i trčaše za tuđim bogovima  da im služe i da im se klanjaju, a mene ostaviše i Zakona se  moga ne držaše. 
\par 12 A vi još gore učiniste nego oci vaši, jer  evo, svaki se povodi za okorjelošću zloga srca svoga, a mene  ne sluša. 
\par 13 Zato ću vas istjerati iz ove zemlje u zemlju koja  vam je neznana, kao što bijaše i ocima vašim. Ondje ćete služiti  tuđim bogovima danju i noću: jer neću vam se više smilovati!'" 
\par 14 "Zato, evo, dolaze dani - riječ je Jahvina - kad se više  neće govoriti: 'Živoga mi Jahve koji sinove Izraelove izvede  iz Egipta', 
\par 15 nego: 'Živoga mi Jahve koji sinove Izraelove  izvede iz zemlje sjeverne i iz svih zemalja kamo ih bijaše prognao.'  Vratit ću ih u zemlju njihovu koju dadoh ocima njihovim. 
\par 16 Evo, poslat ću mnoga ribara - riječ je Jahvina - koji  će ih uloviti. A zatim ću dovesti mnoge lovce koji će ih goniti  sa svih gora, i sa svih brežuljaka, i iz svih pećinskih rasjeklina. 
\par 17 Jer moje oči prate sve njihove putove: neće mi izmaći, niti  se bezakonje njihovo može sakriti od očiju mojih. 
\par 18 Dvostruko  ću naplatiti njihovo bezakonje i grijehe njihove, jer su truplima  svojih grozota oskvrnuli moju zemlju i moju baštinu ispunili  gnusobama." 
\par 19 Jahve, snago moja i moja utvrdo, utočište moje u danima nevolje! K tebi će doći narodi s krajeva zemlje. I govorit će: Samu nam laž oci namriješe, Ništavost i Nemoć. 
\par 20 TÓa stvara li čovjek sam sebi bogove, to nikako nisu bogovi. 
\par 21 Učinit ću, evo, da osjete, da ovaj put zaista oćute moju ruku i snagu moju, i znat će da mi je ime Jahve. 


\chapter{17}

\par 1 Judin grijeh upisan je željeznom pisaljkom, urezan dijamantnim vrškom u pločicu njihova srca i u rogove njihovih žrtvenika, 
\par 2 kao spomen sinovima njihovim na žrtvenike njihove i ašere njihove oko zelenog drveća na visokim brežuljcima, 
\par 3 na brdima i usred polja. Tvoje bogatstvo i sve blago tvoje pljački ću predati. Tako ćeš platiti za grijeh svoj po svoj zemlji. 
\par 4 Morat ćeš pustiti baštinu koju sam tebi poklonio. Učinit ću te robljem neprijatelja u zemlji koju ne poznaješ. Jer gnjev moj planu ognjem koji će vječno gorjeti. 
\par 5 Ovako govori Jahve: "Proklet čovjek koji se uzdaje u čovjeka, i slabo tijelo smatra svojom mišicom, i čije se srce od Jahve odvraća. 
\par 6 Jer on je kao drač u pustinji: ne osjeća kad je sreća na domaku, tavori dane u usahloj pustinji, u zemlji slanoj, nenastanjenoj. 
\par 7 Blagoslovljen čovjek koji se uzdaje u Jahvu i kome je Jahve uzdanje. 
\par 8 Nalik je na stablo zasađeno uz vodu što korijenje pušta k potoku: ne mora se ničeg bojati kad dođe žega, na njemu uvijek zelenilo ostaje. U sušnoj godini brigu ne brine, ne prestaje donositi plod. 
\par 9 Podmuklije od svega je srce. Jedva popravljivo, tko da ga pronikne? 
\par 10 Ja, Jahve, istražujem srca i ispitujem bubrege, da bih dao svakom po njegovu vladanju, prema plodu ruku njegovih. 
\par 11 Prepelica što leži na jajima a ne leže jest onaj što nepravdom stječe bogatstvo: usred dana svojih ostavit' ga mora i na kraju ostaje lÓuda. 
\par 12 Slavan prijestol dignut od davnina, to je naše sveto mjesto. 
\par 13 O Jahve, nado Izraela, svi koji te ostave postidjet će se, koji se odmetnu od tebe bit će u prah upisani, jer ostaviše Izvor žive vode. 
\par 14 Iscijeli me, Jahve, i bit ću zdrav, spasi me, i bit ću spašen, jer ti si pjesma moja. 
\par 15 Evo ih što mi govore: Gdje je riječ Jahvina? Neka se ispuni! 
\par 16 Ta ja se nisam vrzao oko tebe za njihovu nesreću niti sam želio kobni Dan! - ti to znaš, sve što je izlazilo iz usta mojih pred tobom je. 
\par 17 Ne budi mi na užas, ti, utočište moje, u Dan nesretni. 
\par 18 Progonitelji moji nek' se postide, ne ja, njih smeti, ne mene. Na njih dovedi Dan nesretni, zatri ih dvogubim zatorom! 
\par 19 Ovako mi reče Jahve: "Idi i stani na vrata Sinova naroda  na koja ulaze i izlaze kraljevi judejski i na sva vrata jeruzalemska. 
\par 20 Reci im: Čujte riječ Jahvinu, vi, kraljevi judejski i svi  Judejci i Jeruzalemci koji prolazite kroz ova vrata. 
\par 21 Ovako  govori Jahve: 'Čuvajte se ako vam je život mio i ne nosite tereta  u dan subotnji, i ne unosite ga na vrata jeruzalemska. 
\par 22 I  ne nosite bremena iz kuće u dan subotnji, i nikakva posla ne  radite, nego svetkujte dan subotnji, kao što sam zapovjedio vašim  ocima. 
\par 23 Oni me ne poslušaše i ne prikloniše uha svojega, nego  tvrdovrato ne poslušaše i ne prihvatiše nauka. 
\par 24 A vi, ako  me poslušate - riječ je Jahvina - i ako ne budete nosili bremena  na vrata ovoga grada, nego budete svetkovali dan subotnji, ne  radeći nikakva posla, 
\par 25 tada će na vrata ovoga grada ulaziti  kraljevi koji sjede na prijestolju Davidovu, koji se voze kolima  i jašu na konjima, oni i njihovi časnici, Judejci i Jeruzalemci, i ovaj će grad ostati dovijeka. 
\par 26 I dolazit će iz gradova  judejskih, iz okolice Jeruzalema, iz zemlje Benjaminove i iz  Šefele, iz Gorja i iz Negeba, da prinesu paljenice, klanice,  prinosnice i kad i da prinesu žrtvu zahvalnicu u Domu Jahvinu. 
\par 27 Ali ako me ne poslušate te ne budete svetkovali dan subotnji, i ako budete nosili bremena ulazeći na vrata jeruzalemska u  dan subotnji, tada ću potpaliti oganj na vratima njegovim: i  plamen će proždrijeti dvore jeruzalemske, i neće se ugasiti.'" 


\chapter{18}

\par 1 Ovo je riječ koju Jahve uputi Jeremiji: 
\par 2 "Ustani i siđi  u kuću lončarovu - ondje ću ti objaviti svoje riječi." 
\par 3 Siđoh, dakle, u kuću lončarovu, upravo je radio na lončarskom kolu. 
\par 4 I kako bi se koji sud što bi ga načinio od ilovače u ruci  lončarovoj pokvario, on bi opet od nje pravio drugi - već kako  se lončaru svidjelo da napravi. 
\par 5 I dođe mi riječ Jahvina: 
\par 6 "Ne  mogu li i ja s vama činiti kao ovaj lončar, dome Izraelov? -  riječ je Jahvina. Evo, kao ilovača u ruci lončarovoj, i vi ste  u mojoj ruci, dome Izraelov. 
\par 7 Objavim li jednom kojem narodu  ili kojem kraljevstvu da ću ga iskorijeniti, uništiti i razoriti, 
\par 8 i taj se narod, protiv kojeg sam govorio, obrati od opačina  i zloća, tada ću se ja pokajati za zlo koje mu bijah namijenio. 
\par 9 Objavim li kojem narodu, ili kojem kraljevstvu, da ću ga izgraditi  i posaditi, 
\par 10 a on stane činiti što je zlo u mojim očima, ne  slušajući glasa mojega, pokajat ću se za dobro koje sam im obećao. 
\par 11 Zato sada reci Judejcima i Jeruzalemcima: 'Ovako govori Jahve:  Evo, spremam vam zlo i snujem protiv vas osnove. Vratite se,  dakle, svaki sa svoga zlog puta i popravite svoje putove i svoja  djela.' 
\par 12 Ali oni vele: 'Uzalud! Mi ćemo radije provoditi svoje  osnove i činiti svaki po okorjelosti zlog srca svojega.'" 
\par 13 Ovako govori Jahve: "Propitajte se po narodima: je li itko takvo što čuo? Toliku grozotu učini djevica Izraelova. 
\par 14 Nestaje li s gorske vrleti snijeg libanonski? Mogu li presušiti vode daleke što studene teku? 
\par 15 A narod moj mene zaboravi! Kad prinose ništavnosti, posrnuše na putovima svojim, na stazama drevnim, i hode stazama i putem neprohodnim. 
\par 16 I tako su zemlju u pustoš obratili, na vječnu porugu, da se nad njom zgraža svaki prolaznik glavom mašući. 
\par 17 Razvitlat ću ih pred neprijateljem, kao istočnjak; leđa, a ne lice, ja ću im pokazati u dan njine propasti." 
\par 18 I rekoše: "Hajdemo da se urotimo protiv Jeremije, jer  svećeniku ne može nestati Zakona, ni mudrome savjeta, ni proroku  besjede. Hajde, udarimo ga njegovim jezikom i pazimo budno na  svaku riječ njegovu." 
\par 19 Poslušaj me, o Jahve, i čuj što govore moji protivnici. 
\par 20 TÓa zar se dobro uzvraća zlim? A oni mi jamu kopaju! Sjeti se kako stajah pred licem tvojim da u tebe milost tražim za njih, da odvratim od njih jarost tvoju. 
\par 21 Zato im djecu izruči gladi, maču ih predaj! Neka im žene ostanu jalove i udovice, neka im kuga pobije muževe, a mladići neka od mača u boju poginu. 
\par 22 Neka se prolomi vapaj iz kuća njihovih kad iznenada na njih dovedeš čete pljačkaša. Jer oni iskopaše jamu da me uhvate, nogama mojim u potaji zamke namjestiše. 
\par 23 Ti, o Jahve, znadeš sav njihov naum ubilački protiv mene. Ne oprosti im bezakonja njihova, ne izbriši im grijeha pred sobom. Neka padnu pred licem tvojim, obračunaj s njima u dan gnjeva svoga! 


\chapter{19}

\par 1 Ovako mi reče Jahve: "Idi i kupi vrč glineni. Povedi sa sobom  nekoliko starješina narodnih i svećeničkih. 
\par 2 Pođi u Dolinu  Ben Hinom, na ulazu Vrata lončarskih. Ondje proglasi riječi koje  ću ti kazati. 
\par 3 Reci im: Čujte riječ Jahvinu, kraljevi judejski  i stanovnici jeruzalemski. Ovako govori Jahve nad Vojskama, Bog  Izraelov: 'Evo dovodim nesreću na ovo mjesto te će svima koji  to čuju u ušima zazujati. 
\par 4 Zato što su me ostavili i otuđili  ovo mjesto prinoseći kad tuđim bogovima kojih ne poznavahu oci  njihovi ni kraljevi judejski; i zato što su mjesto ovo natopili  krvlju nevinih; 
\par 5 i zato što pogradiše uzvišice Baalove da mu  spaljuju sinove kao žrtve, a to im ja nikada ne naredih niti  zapovjedih, niti mi je takvo što na um palo. 
\par 6 Stoga, evo, dolaze  dani - riječ je Jahvina - kad se ovo mjesto neće više zvati Tofet  ni Dolina Ben Hinom nego Dolina klanja. 
\par 7 Izjalovit ću na ovom  mjestu naum Judeje i Jeruzalema i učiniti da svi poginu od neprijateljskog  mača, od ruku onih što im rade o glavi. A trupla ću njihova dati  za hranu pticama nebeskim i zvijerima poljskim. 
\par 8 A grad ću  ovaj učiniti ruglom i užasom: svaki koji prođe mimo nj zgrozit  će se i zviždati zbog svih nesreća njegovih. 
\par 9 I dat ću im da  jedu meso sinova i kćeri svojih. Da, svaki će jesti meso bližnjega  svoga zbog preteške nevolje kojom će ih pritisnuti njihovi neprijatelji  što im rade o glavi.' 
\par 10 Razbij vrč pred očima svojih pratilaca 
\par 11 i reci im:  Ovako govori Jahve nad Vojskama: 'Tako ću razbiti ovaj narod  i ovaj grad, kao što se razbija sud lončarski, te se više ne  da popraviti. U Tofetu će se ukapati, jer inače neće biti mjesta za ukapanje. 
\par 12 Tako ću postupiti s ovim mjestom - riječ je Jahvina - i sa  stanovnicima njegovim. I učinit ću grad ovaj sličnim Tofetu. 
\par 13 I sve kuće jeruzalemske i kuće kraljeva judejskih bit će  onečišćene kao mjesto Tofet: sve kuće kojima se na krovovima  kad prinosi svoj vojsci nebeskoj i lijevaju ljevanice tuđim bogovima.'" 
\par 14 Kad se Jeremija vratio iz Tofeta, kamo ga Jahve bijaše  poslao da prorokuje, stade u predvorju Doma Jahvina te uze govoriti  svemu narodu: 
\par 15 "Ovako govori Jahve nad Vojskama, Bog Izraelov:  'Evo, dovest ću na ovaj grad i na sve njegove gradiće sve zlo  kojim sam im prijetio, jer ukrutiše vratove svoje ne slušajući  riječi mojih.'" 


\chapter{20}

\par 1 A svećenik Pašhur, sin Imerov, vrhovni nadzornik Doma Jahvina, ču kako Jeremija prorokuje te riječi. 
\par 2 I Pašhur dade Jeremiju  batinati i baci ga u klade što se nalaze kod gornjih vrata Benjaminovih, a u Domu su Jahvinu. 
\par 3 A kad sutradan Pašhur pusti Jeremiju  iz klada, reče mu Jeremija: "Jahve te više ne zove Pašhur već  'Užas odasvud'. 
\par 4 Jer ovako govori Jahve: 'Evo, predajem te  užasu, tebe i sve prijatelje tvoje, i poginut će od mača neprijatelja  svojih, svojim ćeš očima vidjeti. I svu Judeju dat ću u ruke  kralju babilonskom. On će ih odvesti u izgnanstvo u Babilon i  mačem pobiti. 
\par 5 I sve bogatstvo ovoga grada, sav njegov trudom  stečeni imetak i sve dragocjenosti te sve blago kraljeva judejskih  predat ću u ruke neprijateljima. Oni će sve opljačkati, ugrabiti  i u Babilon odnijeti.' 
\par 6 A ti ćeš se, Pašhure, sa svim svojim  ukućanima seliti u Babilon. Da, u Babilon ćeš doći i ondje umrijeti  i biti pokopan, ti i svi tvoji prijatelji kojima si laži prorokovao." 
\par 7 Ti me zavede, o Jahve, i dadoh se zavesti, nadjačao si me i svladao me. A sada sam svima na podsmijeh iz dana u dan, svatko me ismijava. 
\par 8 Jer kad god progovorim, moram vikati, naviještati moram: "Nasilje! Propast!" Doista, riječ mi Jahvina postade na ruglo i podsmijeh povazdan. 
\par 9 I rekoh u sebi: neću više na nj misliti niti ću govoriti u njegovo ime. Al' tad mi u srcu bi kao rasplamtjeli oganj, zapretan u kostima mojim: uzalud se trudih da izdržim, ne mogoh više. 
\par 10 Čuh klevete mnogih: "Užas odasvud! Prijavite! Mi ćemo ga prijaviti." Svi koji mi bijahu prijatelji čekahu moj pad. "Možda ga zavedemo, pa ćemo njim ovladati i njemu se osvetiti!" 
\par 11 Sa mnom je Jahve kao snažan junak! Zato će progonitelji moji posrnuti i neće nadvladati, postidjet će se veoma zbog poraza, zbog nezaboravne vječne sramote. 
\par 12 O Jahve nad Vojskama, koji proničeš pravednika i vidiš mu bubrege i srce, daj da vidim kako im se osvećuješ, jer tebi povjerih parnicu svoju. 
\par 13 Pjevajte Jahvi, hvalite Jahvu, jer on izbavi dušu sirote iz ruku zlikovaca. 
\par 14 Proklet bio dan kad se rodih, dan kad me rodi majka moja ne bio blagoslovljen! 
\par 15 Proklet bio čovjek koji ocu mom dojavi: "Rodio ti se sin, muškić!" i time mi oca obradova. 
\par 16 Tom čovjeku bilo kao gradovima što ih Jahve nemilice razvali; već u cik zore čuo zapomaganje i poklike bojne u podne, 
\par 17 jer me ne pogubi u majčinoj utrobi da bi majka bila moj grob, da bi joj utroba dovijeka ostala trudna! 
\par 18 O, zašto izađoh iz majčina krila? Da vidim jad i nevolju i u sramoti da dokončam dane! 


\chapter{21}

\par 1 Riječ koju Jahve uputi Jeremiji kad kralj Sidkija posla k  njemu Pašhura, sina Malkijina, i svećenika Sefaniju, sina Maasejina, s porukom: 
\par 2 "Hajde, upitaj Jahvu za nas, jer je Nabukodonozor, kralj babilonski, zavojštio na nas; možda će Jahve opet učiniti  s nama čudo, pa će se neprijatelj povući pred nama." 
\par 3 Jeremija  im reče: "Ovako recite Sidkiji: 
\par 4 Ovako govori Jahve, Bog Izraelov:  'Povući ću oružje koje je u vašim rukama, kojim se borite protiv  kralja babilonskoga i Kaldejaca što vas napadaju izvan zidina, i skupiti ga usred ovoga grada. 
\par 5 I sam ću se boriti protiv  vas ispruženom rukom i snažnom mišicom, u srdžbi i gnjevu i velikoj  jarosti. 
\par 6 I strašnom kugom udarit ću stanovnike ovoga grada, ljude i životinje, i pomrijet će. 
\par 7 Poslije toga ću - riječ  je Jahvina - Sidkiju, kralja judejskoga, i njegove sluge i narod, i sve one koji preostadoše u tom gradu nakon pošasti, mača i  gladi, predati u ruke Nabukodonozora, kralja babilonskoga, u  ruke njihovih neprijatelja i u ruke onih koji im rade o glavi;  on će ih sasjeći oštricom mača bez samilosti, bez milosrđa i  bez smilovanja.' 
\par 8 A ovom narodu reci: 'Ovako govori Jahve: Evo stavljam  pred vas put života i put smrti. 
\par 9 Tko ostane u ovom gradu,  poginut će od mača, gladi i kuge. A tko izađe, te se preda Kaldejcima  koji vas opsjedaju, spasit će život - život će mu ostati kao  plijen. 
\par 10 Jer, okrenuh lice svoje ovomu gradu na zlo, a ne  na dobro - riječ je Jahvina - i bit će izručen u ruke kralja  babilonskoga, i on će ga vatrom spaliti.'" 
\par 11 Kraljevskom domu Judeje. Čujte riječ Jahvinu, 
\par 12 dome Davidov! Ovako govori Jahve: "Svako jutro sudite pravedno, izbavite potlačene iz ruku tlačitelja, ili će moj gnjev planut' poput vatre, raspalit' se neugasivo zbog vaših opačina. 
\par 13 Evo me protiv tebe koji stanuješ na Pećini dolinskoj - riječ je Jahvina - protiv vas koji kažete: 'Tko može na nas navaliti, tko u naše nastambe prodrijeti?' 
\par 14 Al' ja ću vam platiti prema plodu djela vaših - riječ je Jahvina. - Oganj ću podmetnuti šumi njegovoj i proždrijet će sve oko nje!" 


\chapter{22}

\par 1 Ovako govori Jahve: "Siđi u palaču kralja judejskoga i objavi  ondje ovu riječ. 
\par 2 Reci: Slušaj riječ Jahvinu, kralju judejski, koji sjediš na prijestolju Davidovu, ti i tvoje sluge i tvoj  narod koji ulaze na ova vrata. 
\par 3 Ovako govori Jahve: 'Činite  pravo i pravicu, izbavite potlačene iz ruku tlačitelja! Ne činite  krivo strancu, siroti, udovici, ne tlačite ih i ne prolijevajte  krvi nedužne na ovome mjestu. 
\par 4 Jer budete li se istinski vladali  po riječi ovoj, na vrata ovog dvora ulazit će kraljevi što sjede  na prijestolju Davidovu, voze se na kolima i jašu na konjima  - oni, njihove sluge i njihov narod. 
\par 5 Ako pak ne poslušate  ovih riječi, zaklinjem se sobom - riječ je Jahvina - da ću taj  dvor pretvoriti u ruševinu!'" 
\par 6 Jer ovako govori Jahve o dvoru kralja judejskoga: "Ti si za me Gilead, vrh libanonski. Ali, uistinu, pretvorit ću te u pustinju, u grad nenastanjen. 
\par 7 Spremit ću protiv tebe zatirače, svakoga s oružjem njegovim, nek' posijeku izabrane ti cedrove i u vatru ih pobacaju." 
\par 8 Mnoštvo će naroda prolaziti mimo taj grad i pitat će jedan  drugoga: "Zašto je Jahve tako postupio s ovim velikim gradom?" 
\par 9 Odgovorit će im: "Jer su ostavili Savez Jahve, Boga svoga, klanjali se drugim bogovima i služili im." 
\par 10 Ne oplakujte mrtvoga, ne jadikujte za njim. Radije plačite za onim koji odlazi, jer se nikad više neće vratiti ni rodne grude vidjeti. 
\par 11 Jer ovako govori Jahve o Šalumu, sinu Jošijinu, kralju  judejskomu, koji kraljevaše mjesto oca svoga i morade otići iz  ovoga mjesta: "Nikad se više neće vratiti, 
\par 12 nego će umrijeti  u mjestu kamo ga izagnaše, a ovu zemlju nikad više neće vidjeti." 
\par 13 Jao onom koji kuću gradi nepravedno i gornje odaje diže bez prava; koji bližnjega tjera na tlaku i plaću mu ne isplaćuje; 
\par 14 koji kaže: "Sagradit ću sebi kuću prostranu i prozračne gornje odaje!" koji probija prozore, oblaže ih cedrovinom crveno obojenom. 
\par 15 Jesi li zato kralj što se cedrom razmećeš? Nije li ti i otac jeo i pio, ali je činio pravo i pravicu i zato mu bješe dobro. 
\par 16 Branio je pravo siromaha i jadnika, i zato mu bješe dobro. Zar ne znači to mene poznavati? - riječ je Jahvina. 
\par 17 Ali tvoje oči i srce idu samo za dobitkom, da krv nevinu prolijevaš, da nasilje činiš i krivdu. 
\par 18 Zato ovako govori Jahve o Jojakimu, sinu Jošijinu, kralju  judejskom:  "Za njim neće naricati: 'Jao, brate moj! Jao, sestro moja!' Za njim neće naricati: 'Jao, gospodaru! Jao, veličanstvo!' 
\par 19 Pokopat će ga k'o magarca, izvući ga i baciti izvan vrata Jeruzalema." 
\par 20 "Popni se na Libanon i viči, po Bašanu nek' se ori glas, s Abarima buči, jer svi su tvoji prijatelji slomljeni! 
\par 21 Lijepo sam te svjetovao u danima mirnim, al' ti mi reče: 'Neću slušati!' Tako se vladaš od mladosti: ne slušaš glasa mojega. 
\par 22 Sve će tvoje pastire vjetar popasti, a ljubavnici će tvoji u izgnanstvo. Tada ćeš se stidjet' i sramiti zbog sve pakosti svoje. 
\par 23 Ti što prebivaš na Libanonu, ti što se gnijezdiš po cedrovima, kako li ćeš stenjati kad bolovi te spopadnu, trudovi porodilje. 
\par 24 Života mi moga - riječ je Jahvina - kad bi Konija, sin  Jojakimov, kralj judejski, bio pečatnjak na mojoj desnici, ja  bih ga strgao s prsta. 
\par 25 Dat ću te u ruke onima koji ti rade  o glavi, u ruke onima pred kojima dršćeš, u ruke Nabukodonozora, kralja babilonskog, i u ruke Kaldejaca. 
\par 26 I bacit ću tebe  i majku koja te rodila u drugu zemlju gdje se niste rodili; tamo  ćete umrijeti. 
\par 27 Ali u zemlju u koju čeznu da se vrate neće  se vratiti!" 
\par 28 TÓa zar je taj čovjek Konija sud prezren, razbijen? Il' posuda što nikom se ne sviđa? Zašto bjehu protjerani on i potomstvo, prognani u zemlju koja im je posve neznana? 
\par 29 Zemljo, zemljo, zemljo, poslušaj riječ Jahvinu. 
\par 30 Ovako govori Jahve: "Upišite za ovoga čovjeka: 'Bez djece. Život mu se nije posrećio. Nitko od potomstva njegova neće sjesti na prijesto Davidov ni vladati Judejom.'" 


\chapter{23}

\par 1 "Jao pastirima koji upropašćuju i raspršuju ovce paše moje"  - riječ je Jahvina. 
\par 2 Stoga ovako govori Jahve, Bog Izraelov, protiv pastira koji pasu narod moj: "Vi ste raspršili ovce moje, rastjerali ih, niste se brinuli za njih. Zato ću se ja sada  pobrinuti za vas zbog zlodjela vaših - riječ je Jahvina. 
\par 3 I  sam ću skupiti ostatak svojih ovaca iz svih zemalja kamo sam  ih raspršio i vratiti ih na ispaše njihove: bit će plodne i množit  će se. 
\par 4 I podići ću im pastire da ih pasu te se ničega više  neće bojati ni plašiti, niti će se gubiti" - riječ je Jahvina. 
\par 5 "Evo dolaze dani - riječ je Jahvina - podići ću Davidu izdanak pravedni. On će vladati kao kralj i biti mudar i činit će pravo i pravicu u zemlji. 
\par 6 U njegove će dane Judeja biti spašena i Izrael će živjeti spokojno. I evo imena kojim će ga nazivati: 'Jahve, Pravda naša.' 
\par 7 Zato, evo, dolaze dani - riječ je Jahvina - kad se više  neće govoriti: 'Živoga mi Jahve koji sinove Izraelove izvede  iz zemlje egipatske', 
\par 8 nego: 'Živoga mi Jahve koji potomstvo  doma Izraelova izvede i dovede iz zemlje sjeverne i iz svih zemalja  kamo ih bijaše prognao, tako da obitavaju u zemlji svojoj.'" 
\par 9 Prorocima. Srce je u meni skrhano, dršću mi kosti, sličan sam pijancu, čovjeku kojim vino ovlada, pred licem Jahvinim i njegovim svetim riječima: 
\par 10 "Jer zemlja je puna preljubnika; zbog tih se ljudi zemlja u crno zavila, a ispaše u pustinji sagorješe. Njihova je trka zloba, a moć im je nepravda. 
\par 11 Da, i prorok i svećenik zlikovci su, čak i u Domu svome nađoh im pakost" - riječ je Jahvina. 
\par 12 Stog' će im se puti prometnuti u tlo klizavo: u mraku će posrtati i padati. Jer ja ću na njih svaliti nesreću u godine kazne njihove" - riječ je Jahvina. 
\par 13 "I u proroka Samarije vidjeh mnoge ludosti: prorokuju u Baalovo ime i zavode narod moj izraelski. 
\par 14 Ali u proroka jeruzalemskih vidjeh strahote: preljub, prijevarne putove, jačaju ruke zločincima, te se nitko od zločina svojih ne obraća. Svi su mi oni kao Sodoma, a žitelji kao Gomora!" 
\par 15 I zato Jahve nad Vojskama ovako govori o prorocima: "Evo, nahranit ću ih pelinom i napojiti vodom zatrovanom, jer od proroka jeruzalemskih potječe pokvara u svoj zemlji." 
\par 16 Ovako govori Jahve nad Vojskama: "Ne slušajte riječi proroka: oni vas obmanjuju, objavljuju viđenja srca svoga, a ne što dolazi iz usta Jahvinih; 
\par 17 govore onima što preziru riječ Jahvinu: 'Bit će s vama mir!' a onima što slijede glas svog srca okorjelog: 'Nikakvo vas zlo neće snaći!'" 
\par 18 TÓa tko bijaše na vijećanju Jahvinu, tko je vidio, tko  slušao riječ njegovu? Tko ju je shvatio te je može objaviti? 
\par 19 Gle, nevrijeme Jahvino: jarost provaljuje, razmahuje se vihor silan i svaljuje na glave bezbožničke. 
\par 20 Jahvin se gnjev neće stišati, dok on ne izvrši i ne ispuni naume srca svojega. U dane posljednje jasno ćete to razumjeti. 
\par 21 "Ne poslah ti proroka, a ipak trče! Ne govorih im, a ipak prorokuju! 
\par 22 Jest, da bijahu na mom vijećanju, moje bi riječi narodu mom obznanili, i kušali ih svrnuti sa zla puta njihova i od zlodjela njihovih! 
\par 23 TÓa, zar sam ja Bog samo iz blizine - riječ je Jahvina - zar iz daljine nisam više Bog? 
\par 24 Može li se tko skriti u skrovištima da ga ja ne vidim? - riječ je Jahvina. Ne ispunjam li ja nebo i zemlju? - riječ je Jahvina. 
\par 25 Čuo sam što govore proroci koji prorokuju laži u ime  moje i tvrde: 'Usnio sam! Usnio sam!' 
\par 26 Dokle će među prorocima  biti onih koji prorokuju laž i objavljuju prijevaru srca svojega? 
\par 27 Misle da će svojim snima što ih jedan drugom pripovijedaju  postići da narod moj zaboravi ime moje, kao što već oci njihovi  zaboraviše ime moje uz Baala! 
\par 28 Prorok koji je usnio san neka  samo pripovijeda svoj san, a u koga je riječ moja, neka po istini  objavljuje riječ moju!" "Što je zajedničko slami i žitu? - riječ je Jahvina. 
\par 29 Nije li riječ moja poput vatre - riječ je Jahvina - i nije li slična malju što razbija pećinu? 
\par 30 Evo me stoga protiv proroka - riječ je Jahvina - koji  jedan drugome kradu moje riječi. 
\par 31 Evo me protiv proroka -  riječ je Jahvina - koji mlate jezikom i proroštva kuju. 
\par 32 Evo  me protiv proroka - riječ je Jahvina - koji prorokuju izmišljene  snove i pripovijedajući ih zavode narod moj izmišljotinama svojim  i lažima. A ja ih nisam poslao, niti sam im što zapovjedio, niti  su narodu ovome od kakve koristi - riječ je Jahvina. 
\par 33 A ako te ovaj narod, ili prorok, ili svećenik, zapita:  'Što je breme Jahvino?' odgovori im: 'Vi ste breme Jahvino i  ja vas odbacujem' - riječ je Jahvina. 
\par 34 A reče li koji prorok ili svećenik, ili tko iz naroda:  'Breme Jahvino', kaznit ću toga čovjeka i dom njegov. 
\par 35 Ovako  morate govoriti svaki svome bližnjemu i svaki svome bratu: 'Što  je Jahve odgovorio?' ili 'Što je Jahve rekao?' 
\par 36 Ali 'Breme  Jahvino' da više niste spomenuli, jer je breme svakome riječ  njegova." Jer vi iskrivljujete riječi Boga živoga, Jahve nad  Vojskama, našega Boga! 
\par 37 Ovako reci proroku: "Što ti je Jahve  odgovorio?" ili "Što je Jahve rekao?" 
\par 38 Ali ako kažete "Breme  Jahvino", ovako govori Jahve: "Zato što se služite riječju 'Breme  Jahvino', premda sam vam poručio da je ne izgovarate, 
\par 39 ja  ću visoko podići i odbaciti od lica svojega vas i vaš grad što  ga dadoh vama i ocima vašim! 
\par 40 I svalit ću na vas vječnu sramotu  i vječnu porugu koja se neće zaboraviti." 


\chapter{24}

\par 1 Jahve mi pokaza, i gle: dvije kotarice smokava stajahu pred  Domom Jahvinim, pošto Nabukodonozor, kralj babilonski, odvede  iz Jeruzalema i izagna u Babilon Jekoniju, sina Jojakimova, kralja  judejskoga, zajedno s knezovima judejskim, kovačima i bravarima. 
\par 2 U jednoj kotarici bijahu izvrsne smokve, kakve već jesu rane  smokve; a u drugoj bijahu pokvarene smokve, tako rđave da se  ne mogahu jesti. 
\par 3 I Jahve me upita: "Jeremija, što vidiš?"  A ja odgovorih: "Smokve! Dobre su vrlo dobre, a loše su vrlo  loše - tako loše da nisu za jelo." 
\par 4 I dođe mi riječ Jahvina: 
\par 5 Ovako govori Jahve, Bog Izraelov: "Kao na ove dobre smokve, tako ću milostivo pogledati na sužnje judejske koje sam s ovoga  mjesta prognao u zemlju kaldejsku. 
\par 6 I milostivo ću svrnuti  oči na njih i vratiti ih u ovu zemlju. Ponovo ću ih podići i  neću ih više uništiti; opet ću ih posaditi i neću ih više iščupati. 
\par 7 I dat ću im srce da me poznaju da sam ja Jahve, da budu narod  moj, a ja Bog njihov, jer će se oni svim srcem svojim opet k  meni obratiti. 
\par 8 Ali kao s lošim smokvama koje su tako loše  da nisu za jelo - da, riječ je Jahvina - tako ću postupiti i  sa Sidkijom, kraljem judejskim, s njegovim knezovima i sa svim  Jeruzalemcima što preostadoše u ovoj zemlji i s onima što se  u Egiptu nastaniše. 
\par 9 Učinit ću da budu na užas svim kraljevstvima  zemaljskim, na sramotu i porugu, na ruglo i kletvu posvuda kamo  ih protjeram. 
\par 10 I poslat ću na njih mač, glad i kugu dok se  ne istrijebe sa zemlje koju dadoh njima i ocima njihovim." 


\chapter{25}

\par 1 Riječ upućena Jeremiji o svem narodu  judejskom, četvrte godine Jojakima, sina kralja judejskog - to  je prve godine Nabukodonozora, kralja babilonskog. 
\par 2 Prorok 
\par 3 Od trinaeste godine Jošije, sina Amonova, kralja judejskoga, sve do dana današnjeg, ove dvadeset i tri godine, dolazila mi  je riječ Jahvina i ja sam vam jednako govorio, ali me niste slušali. 
\par 4 I Jahve je svejednako slao k vama sve sluge svoje, proroke, ali vi niste slušali i niste prignuli uši svoje da čujete. 
\par 5 I  govorahu vam: "Vratite se svaki sa zla puta svojega i od zlih  djela svojih i ostanite u zemlji koju Jahve dade vama i ocima  vašim za sva vremena; 
\par 6 i ne idite za tuđim bogovima da im služite  i da im se klanjate; i ne gnjevite me djelima ruku svojih, pa  vam neću ništa nažao učiniti. 
\par 7 Ali me niste poslušali - riječ  je Jahvina - nego me razgnjeviste djelima ruku svojih, na svoju  nesreću!" 
\par 8 Zato ovako govori Jahve nad Vojskama: "Jer niste poslušali  mojih riječi, 
\par 9 evo, ja ću poslati i podignuti sve narode sa  sjevera - riječ je Jahvina - i slugu svoga Nabukodonozora, kralja  babilonskoga, i dovest ću ih na ovu zemlju i na njene stanovnike  i na sve okolne narode; izručit ću ih kletom uništenju i učinit  ću ih užasom i ruglom, vječnim razvalinama. 
\par 10 I ugušit ću među  njima svaki glas radosti i veselja, klicanje zaručnika i zaručnice  i klopot žrvnja i svjetlost svjetiljke. 
\par 11 Sva će se zemlja  pretvoriti u pustoš i pustinju i svi će narodi služiti kralju  babilonskom sedamdeset godina. 
\par 12 Ali kad se navrši sedamdeset  godina, kaznit ću kralja babilonskog i narod onaj - riječ je  Jahvina - za bezakonje njihovo i zemlju kaldejsku i pretvorit  ću ih u vječne razvaline. 
\par 13 Dovest ću na tu zemlju sve što  sam protiv nje rekao - sve je to napisano u ovoj knjizi, što  prorokova Jeremija za sve narode." 
\par 14 "I oni će služiti mnogim narodima i velikim kraljevima  i platit ću im po njihovim činima i po djelima ruku njihovih." 
\par 15 Ovako mi reče Jahve, Bog Izraelov: "Uzmi ovaj pehar vina  iz moje ruke i napoji njime sve narode kojima ću te poslati. 
\par 16 Neka piju dok ne zateturaju i dok se ne izbezume zbog mača  što ću ga među njih poslati." 
\par 17 I uzeh pehar iz ruke Jahvine  i napojih njime sve narode kojima me Jahve bijaše poslao: 
\par 18 Jeruzalem  i gradove judejske s njihovim kraljevima i knezovima, neka budu  razvalina, pustoš, ruglo i prokletstvo, kao što su i danas; 
\par 19 faraona, kralja egipatskoga, s njegovim slugama i knezovima i narodom  njegovim 
\par 20 i svu onu mješavinu naroda: sve kraljeve zemlje  Usa, sve kraljeve zemlje filistejske, Aškelon, Gazu, Ekron i  ono što ostade od Ašdoda; 
\par 21 Edom, Moab i sinove Amonove; 
\par 22 sve  kraljeve Tira, sve kraljeve Sidona, kraljeve otoka onkraj mora; 
\par 23 Dedan, Temu, Buz i sve one ostriženih zalizaka, 
\par 24 sve kraljeve  Arabije, sve kraljeve mješavine naroda koji obitavaju u pustinji; 
\par 25 sve kraljeve Zimrija, sve kraljeve Elama i sve kraljeve Medije; 
\par 26 sve kraljeve Sjevera, blize i daleke, jednog za drugim, i  sva kraljevstva na licu zemlje. A kralj Šešak pit će poslije  njih. 
\par 27 I reci im: "Ovako govori Jahve nad Vojskama, Bog Izraelov:  'Pijte! Opijte se! Bljujte! Padnite da se više ne dignete od  mača koji ću pustiti među vas.' 
\par 28 Ako ne bi htjeli uzeti pehar  iz tvoje ruke da piju, reci im: 'Ovako govori Jahve nad Vojskama:  Morate piti! 
\par 29 Jer, evo, ja sam počeo kažnjavati grad koji  se zove mojim imenom. A vi zar da prođete bez kazne? Ne, nećete  ostati nekažnjeni, jer ću sam dozvati mač da udari na sve stanovnike  zemlje' - riječ je Jahve nad Vojskama. 
\par 30 Ti im, dakle, prorokuj  sve riječi ove i reci im: 'Jahve reče sa visine, iz svetoga stana grmi glasom, riče iza glasa protiv pašnjaka svoga, podvikuje kao oni što grožđe gaze. Do svih stanovnika zemlje 
\par 31 dopire bojni klik - do nakraj svijeta - jer Jahve se parbi s narodima, izlazi na sud sa svakim tijelom, bezbožnike će maču izručiti - riječ je Jahvina. 
\par 32 Ovako govori Jahve nad Vojskama: Evo, nesreća već zahvaća narod za narodom; nevrijeme strašno već se prolama s krajeva zemlje.'" 
\par 33 U onaj dan bit će pobijenih Jahvinih s jednoga kraja  svijeta do drugoga. Za njima nitko neće naricati, niti će ih  tko pokupiti i sahraniti; ostat će kao gnoj po zemlji. 
\par 34 Kukajte, pastiri, i vičite, valjajte se po prašini, vodiči stada, jer vam se ispuniše dani za klanje, popadat ćete ko ovnovi izabrani. 
\par 35 Više nema utočišta pastirima, niti spasa vodičima stada. 
\par 36 Čuj, kako vapiju pastiri, kako kukaju vodiči stada, jer Jahve pustoši pašnjak njihov, 
\par 37 mirna su pasišta poharana od jarosna gnjeva Jahvina. 
\par 38 Lav ostavlja guštaru jer će zemlja njihova opustjeti od mača pustošničkog, od jarosnog gnjeva Jahvina. 


\chapter{26}

\par 1 U početku kraljevanja Jojakima, sina Jošijina, kralja Judina, dođe mi riječ Jahvina. 
\par 2 Ovako govori Jahve: "Stani u predvorju  Doma Jahvina i svim gradovima judejskim koji dolaze da se poklone  u Domu Jahvinu naviještaj sve riječi koje sam ti zapovjedio da  im kažeš. I ne izostavi ni jedne jedine. 
\par 3 Možda će poslušati  i vratiti se svaki sa zla puta svoga, pa ću se pokajati za zlo  koje naumih učiniti zbog zlodjela njihovih. 
\par 4 Reci im: 'Ovako  govori Jahve: Ako me ne poslušate da hodite po Zakonu što ga  stavih pred vas, 
\par 5 slušajući riječi slugu mojih proroka koje  vam neumorno šaljem, premda ih do sada niste slušali, 
\par 6 postupit  ću s ovim Domom kao sa Šilom i učinit ću da ovaj grad bude prokletstvo  za sve narode na zemlji.'" 
\par 7 I svećenici i proroci i sav narod slušahu Jeremiju kako  naviješta te riječi u Domu Jahvinu. 
\par 8 A kad Jeremija izreče  sve ono što mu je Gospod zapovjedio da naviješta svemu narodu, zgrabiše ga svećenici i proroci govoreći: "Platit ćeš glavom! 
\par 9 Zašto si u ime Jahvino prorokovao: 'Postupit ću s ovim Domom  kao sa Šilom i ovaj će grad biti opustošen te nitko više u njemu  neće stanovati?'" I sav se narod skupi na Jeremiju u Domu Jahvinu. 
\par 10 Čuvši to, starješine judejske dođoše iz kraljevskog dvora  u Dom Jahvin i sjedoše pred Nova vrata Doma Jahvina. 
\par 11 Tada svećenici i proroci rekoše starješinama i svemu  narodu: "Ovaj je čovjek zaslužio smrt jer je prorokovao protiv  ovoga grada, kao što ste čuli na svoje uši." 
\par 12 Tada Jeremija  reče starješinama i svemu narodu: "Jahve me posla da prorokujem  protiv ovoga Doma i ovoga grada sve ono što ste čuli. 
\par 13 Popravite, dakle, putove svoje i djela svoja i slušajte glas Jahve, Boga  svoga: i pokajat će se za zlo kojim vam se zaprijetio. 
\par 14 Ja  sam, evo, u vašim rukama. Učinite sa mnom što vam se čini dobro  i pravo. 
\par 15 Ali dobro znajte: ako me pogubite, krv nedužnu navalit  ćete na sebe, na ovaj grad i na njegove stanovnike. Jer, zaista, Jahve me posla k vama da u vaše uši govorim sve ove riječi." 
\par 16 Tada rekoše starješine i sav narod svećenicima i prorocima:  "Ovaj čovjek nipošto ne zaslužuje smrt, jer nam je govorio u  ime Jahve, Boga našega." 
\par 17 Nato ustadoše i neki od najuglednijih  u zemlji te rekoše svemu mnoštvu naroda što se ondje okupilo: 
\par 18 "Mihej Morešećanin prorokovaše u dane Ezekije, kralja judejskog, i govoraše svemu narodu judejskom: 'Ovako govori Jahve nad Vojskama: Sion će biti polje preorano, Jeruzalem ruševina, a Goru Doma ovog šuma će prekriti.' 
\par 19 Je li ga zato pogubio Ezekija, kralj judejski, i sva  Judeja? Nisu li se pobojali Jahve i nastojali da Jahvu umilostive, te se Jahve pokaja za zlo kojim im se bijaše zaprijetio? A mi, zar da na duše svoje navalimo tako velik zločin?" 
\par 20 Bijaše ondje još neki koji prorokovaše u ime Jahvino, Urija, sin Šemajin, rodom iz Kirjat Jearima. I on prorokovaše  protiv ovoga grada i zemlje ove kao i Jeremija. 
\par 21 A kad je  kralj Jojakim sa svim ratnicima i zapovjednicima čuo te riječi, tražio je da ga smakne. Čuvši to, Urija se prestraši i pobježe  u Egipat. 
\par 22 Ali kralj Jojakim posla u Egipat Elnatana, sina  Akborova, s nekoliko ljudi; 
\par 23 dovedoše oni Uriju iz Egipta  i odvedoše ga kralju Jojakimu, koji ga mačem pogubi, a truplo  njegovo baci na groblje prostoga puka. 
\par 24 Ali Ahikam, sin Šafanov, zaštiti Jeremiju te ga ne predaše u ruke narodu da ga pogube. 


\chapter{27}

\par 1 U početku kraljevanja Sidkije, sina Jošije, kralja judejskoga, uputi Jahve Jeremiji ovu riječ. 
\par 2 Ovako mi reče Jahve: "Načini  sebi užad i jaram i stavi ga sebi na vrat. 
\par 3 Zatim poruči kralju  edomskom, kralju moapskom, kralju amonskom, kralju tirskom i  kralju sidonskom, po njihovim izaslanicima koji su došli u Jeruzalem  kralju judejskom Sidkiji. 
\par 4 Naredi im da poruče svojim gospodarima:  'Ovako govori Jahve nad Vojskama, Bog Izraelov! Ovako poručite  svojim gospodarima: 
\par 5 Ja sam snagom svojom svesilnom i rukom  ispruženom stvorio zemlju, ljude i životinje na zemlji. I ja  to dajem kome hoću. 
\par 6 Sada, dakle, sve te zemlje dajem u ruke  Nabukodonozoru, kralju babilonskom, sluzi svojemu; dajem mu i  poljsko zvijerje da mu služi. 
\par 7 I svi će narodi služiti njemu  i njegovu sinu, i sinu njegova sina, dok i njegovoj zemlji ne  kucne čas te i njega ne upokore moćni narodi i veliki kraljevi. 
\par 8 Ako koji narod, ili kraljevstvo, ne htjedne služiti Nabukodonozora, kralja babilonskoga, ne hoteć' se upregnuti u jaram kralja babilonskog, taj ću narod kazniti mačem, glađu i kugom - riječ je Jahvina  - dok ga sasvim ne zatrem rukom njegovom. 
\par 9 Ne slušajte, dakle, svojih proroka, gatalaca, sanjara, zvjezdara svojih i čarobnjaka  koji vam govore: 'Ne, vi nećete služiti kralju babilonskom!' 
\par 10 Jer vam oni laž prorokuju samo da vas udalje iz vaše zemlje, da vas otjeram pa da propadnete. 
\par 11 Ali narod koji se upregne  u jaram kralja babilonskoga da mu služi ostavit ću na miru u  zemlji njegovoj - riječ je Jahvina - da je obrađuje i u njoj  živi.'" 
\par 12 Sve sam to rekao Sidkiji, kralju judejskom, govoreći:  "Upregnite se u jaram kralja babilonskoga i pokorite se njemu  i narodu njegovu da ostanete živi. 
\par 13 Zašto da poginete, ti  i narod tvoj, od mača, gladi i kuge, kao što se Jahve zaprijetio  narodu koji se ne podvrgne kralju babilonskom? 
\par 14 Ne slušajte, dakle, riječi onih proroka koji vam govore: 'Vi nećete služiti  kralju babilonskom.' Oni vam laž prorokuju. 
\par 15 'Jer nisam ih  ja poslao da vam prorokuju - riječ je Jahvina - nego vam oni  laž prorokuju u moje ime, da vas otjeram iz vaše zemlje, pa da  propadnete - vi i proroci koji vam prorokuju.'" 
\par 16 I svećenicima i svemu ovom narodu rekao sam: "Ovako govori  Jahve: 'Ne slušajte riječi svojih proroka koji vam ovako prorokuju:  Evo, posuđe Doma Jahvina bit će uskoro vraćeno iz Babilona.'  Oni vam laži prorokuju. 
\par 17 Ne slušajte ih! Pokorite se kralju  babilonskom da ostanete živi! Zašto da ovaj grad postane ruševina? 
\par 18 Ako su oni zaista proroci, te ako je u njima riječ Jahvina, neka mole Jahvu nad Vojskama da i posuđe što još ostade u Domu  Jahvinu i u dvoru kraljeva judejskih i u Jeruzalemu ne dospije  u Babilon! 
\par 19 Jer ovako govori Jahve o stupovima, moru, podnožjima  i o preostalom posuđu što još ostade u ovome gradu - 
\par 20 što  još Nabukodonozor, kralj babilonski, ne uze sa sobom onda kad  odvede u izgnanstvo iz Jeruzalema u Babilon Jekoniju, sina Jojakimova, kralja judejskoga, i sve odličnike judejske i jeruzalemske. 
\par 21 Da, ovako govori Jahve nad Vojskama, Bog Izraelov, o posuđu  koje preostade u Domu Jahvinu, u dvoru kralja judejskog, i u  Jeruzalemu: 
\par 22 'U Babilon će ih odnijeti i ondje će ostati sve  do dana kad ja odem po njih - riječ je Jahvina. I ja ću sve to  donijeti i postaviti na ovo mjesto!'" 


\chapter{28}

\par 1 Iste godine, u početku kraljevanja Sidkije, kralja judejskoga, četvrte godine, petog mjeseca, Hananija, sin Azurov, prorok  rodom iz Gibeona, reče mi u Domu Jahvinu pred svim svećenicima  i svim narodom: 
\par 2 "Ovako govori Jahve nad Vojskama, Bog Izraelov:  'Skršit ću jaram kralja babilonskoga. 
\par 3 Do dvije godine vratit  ću na ovo mjesto sve posuđe Doma Jahvina koje je Nabukodonozor, kralj babilonski, odavde uzeo i odnio u Babilon. 
\par 4 A tako i  Jekoniju, sina Jojakimova, kralja judejskoga, i sve izgnanike  judejske, koji dospješe u Babilon, također ću vratiti na ovo  mjesto - riječ je Jahvina - jer ću skršiti jaram kralja babilonskoga.'" 
\par 5 Tada prorok Jeremija odgovori proroku Hananiji pred svećenicima  i pred svim narodom koji bijahu u Domu Jahvinu. 
\par 6 Reče prorok  Jeremija: "Amen! Učinio Jahve tako! Ispunio Jahve riječi koje  si prorokovao i vratio ovamo sve posuđe iz Doma Jahvina i sve  izgnanike iz Babilona. 
\par 7 Ali čujder i ovu riječ koju ću kazati  na tvoje uši i na uši svega naroda. 
\par 8 Proroci koji su bili prije  mene i tebe, odiskona prorokovahu mnogim moćnim zemljama i velikim  kraljevstvima rat, glad, kugu. 
\par 9 Ali o proroku koji proriče  mir možeš istom kad se ispuni njegova proročka riječ znati da  ga je zaista Jahve poslao." 
\par 10 Tada prorok Hananija skide jaram s vrata proroka Jeremije  i skrši ga. 
\par 11 I reče Hananija pred svim narodom: "Ovako govori  Jahve: 'Evo, ovako ću - za dvije godine - skršiti jaram Nabukodonozora, kralja babilonskoga, s vrata svih naroda!'" Tada prorok Jeremija  ode svojim putem. 
\par 12 A kad prorok Hananija skrši jaram s vrata proroka Jeremije, dođe riječ Jahvina Jeremiji: 
\par 13 "Idi i ovako reci Hananiji:  'Ovako govori Jahve: Ti si skršio drveni jaram, ali ću ja mjesto  njega načiniti željezni.' 
\par 14 Jer ovako govori Jahve nad Vojskama, Bog Izraelov: 'Željezni ću jaram staviti oko vrata svih ovih  naroda da ih podvrgnem Nabukodonozoru, kralju babilonskom, i  služit će mu, jer ja sam njemu podložio čak i poljsku zvjerad!'" 
\par 15 I prorok Jeremija reče proroku Hananiji: "Čuj me dobro, Hananija! Tebe nije poslao Jahve, a ti si u narodu pobudio varave  nade. 
\par 16 Zato ovako govori Jahve: 'Gle, brišem te s lica zemlje!  Umrijet ćeš još ove godine, jer si propovijedao pobunu protiv  Jahve!'" 
\par 17 I umrije prorok Hananija te godine u sedmom mjesecu. 


\chapter{29}

\par 1 Evo, ovo su riječi poslanice koju prorok Jeremija iz Jeruzalema  posla Ostatku izgnanstva - starješinama, svećenicima i prorocima  i svemu preostalom narodu što ga Nabukodonozor iz Jeruzalema  bijaše odveo u Babilon, 
\par 2 pošto kralj Jekonija i kraljica-majka, dvorjanici, odličnici judejski i jeruzalemski, kovači i bravari  ostaviše Jeruzalem. 
\par 3 Poslanica je poslana po Elasi, sinu Šafanovu, i Gemarji, sinu Hilkijinu, koje Sidkija, kralj judejski, posla  u Babilon Nabukodonozoru, kralju babilonskom. Evo sadržaja: 
\par 4 "Ovako govori Jahve nad Vojskama, kralj Izraelov: 'Svima  izgnanicima koje odvedoh iz Jeruzalema u Babilon! 
\par 5 Gradite  kuće i nastanite se, sadite vrtove i uživajte urod njihov! 
\par 6 Ženite  se i rađajte sinove i kćeri! Ženite svoje sinove i udajite svoje  kćeri da i oni rađaju sinove i kćeri! Množite se da se ne smanjite! 
\par 7 Ištite mir zemlji u koju vas izagnah, molite se za nju Jahvi, jer na njezinu miru počiva i vaš mir!' 
\par 8 Ovako govori Jahve nad Vojskama, Bog Izraelov: 'Ne dajte da  vas obmanjuju vaši proroci koji su među vama, vaši gataoci! Ne  povodite se za snovima koje oni sanjaju! 
\par 9 Jer oni vam laž prorokuju  u moje ime, a ja ih nisam poslao' - riječ je Jahvina." 
\par 10 Jer ovako govori  Jahve: 'Istom kad se Babilonu ispuni onih sedamdeset godina,  ja ću vas pohoditi te vam ispuniti dobro obećanje da ću vas vratiti  na ovo mjesto. 
\par 11 Jer ja znam svoje naume koje s vama namjeravam  - riječ je Jahvina - naume mira, a ne nesreće: da vam dadnem  budućnost i nadu. 
\par 12 Tada ćete me zazivati, dolaziti k meni, moliti mi se i ja ću vas uslišati. 
\par 13 Tražit ćete me i naći  me jer ćete me tražiti svim srcem svojim. 
\par 14 I pustit ću da  me nađete - riječ je Jahvina. Izmijenit ću udes vaš i sabrati  vas iz svih naroda i sa svih mjesta kamo vas odagnah - riječ  je Jahvina. I vratit ću vas na mjesto odakle vas u izagnanstvo  odvedoh. 
\par 15 Istina, vi velite: 'Jahve nam podiže proroke u Babilonu.' 
\par 16 Ovako govori Jahve kralju koji sjedi na prijestolju Davidovu, i svemu narodu koji živi u ovome gradu - braći vašoj što ne  moradoše s vama u izgnanstvo. 
\par 17 Ovako govori Jahve nad Vojskama:  "Evo šaljem na njih mač, glad i kugu; učinit ću da budu kao pokvarene  smokve, tako loše da nisu za jelo. 
\par 18 I gonit ću ih mačem, glađu  i kugom i učinit ću ih užasom svim kraljevstvima zemaljskim,  prokletstvom, strahotom, ruglom i sramotom svim narodima kamo  ih otjeram. 
\par 19 Jer ne poslušaše riječi mojih - riječ je Jahvina  - premda sam im svejednako slao sluge svoje proroke, ali ih oni  ne poslušaše - riječ je Jahvina. 
\par 20 Ali vi, izgnanici, koje  poslah iz Jeruzalema u Babilon, poslušajte svi riječ Jahvinu!" 
\par 21 Ovako govori Jahve nad Vojskama, kralj Izraelov, o Ahabu, sinu Kolajinu, i o Sidkiji, sinu Maasejinu, koji vam laž prorokuje  u moje ime: "Evo, predajem ih u ruke Nabukodonozora, kralja babilonskoga, da ih pogubi vama na oči. 
\par 22 I njima će se kao kletvom proklinjati  svi izgnanici koji su u Babilonu: 'Neka Jahve učini s tobom kao  sa Sidkijom i Ahabom koje kralj babilonski ispeče na vatri 
\par 23 jer  u Izraelu počiniše sramotu čineći preljub sa ženama svojih bližnjih  i govoreći u moje ime lažne riječi koje im ja nisam zapovjedio.  Ja to znam, i svjedok sam tome' - riječ je Jahvina!" 
\par 24 - 
\par 25 A Šemaji ćeš Nehelamcu poručiti: "Ovako govori Jahve  nad Vojskama, Bog Izraelov: Ti si u svoje ime poslao pisma svemu  narodu koji je u Jeruzalemu, i svećeniku Sefaniji, sinu Maasejinu, i svim ostalim svećenicima: 
\par 26 'Jahve te postavi svećenikom namjesto svećenika Jojade  da paziš u Domu Jahvinu na svakog luđaka koji se gradi prorokom  i da ga baciš u klade, sa željezom oko vrata. 
\par 27 Zašto, dakle, nisi spriječio Jeremiju iz Anatota, koji se među vama gradi  prorokom? 
\par 28 TÓa on nam je poslao poruku u Babilon: Dugo će  još trajati: Gradite kuće i nastanite se! Sadite vrtove i uživajte  urod njihov!'" 
\par 29 Svećenik Sefanija pročita pismo proroku Jeremiji. 
\par 30 Tada  dođe riječ Jahvina Jeremiji: 
\par 31 "Pošalji svim izgnanicima ovu  vijest: 'Ovako govori Jahve o Šemaji Nehelamcu: Jer vam Šemaja  prorokuje te vam budi varave nade, premda ga ja nisam poslao, 
\par 32 ovako govori Jahve: Kaznit ću Šemaju Nehelamca, njega i  potomstvo njegovo: nitko mu neće preostati usred ovoga naroda  da doživi sreću koju spremam narodu svojemu - riječ je Jahvina  - jer je propovijedao pobunu protiv Jahve.'" 


\chapter{30}

\par 1 Riječ koju Jahve upravi Jeremiji: 
\par 2 Ovako govori Jahve, Bog Izraelov: "Upiši u knjigu sve  ove riječi koje ti govorim. 
\par 3 Jer evo dolaze dani - riječ je  Jahvina - i promijenit ću udes naroda svoga Izraela i Judeje"  - govori Jahve - "i vratit ću ih u zemlju koju u baštinu dadoh  ocima njihovim." 
\par 4 Evo riječi što ih Jahve reče o Izraelu i o Judeji: 
\par 5 Ovako govori Jahve: "Čujem krik užasa: strava je to, a ne mir. 
\par 6 Hajde, propitajte se i pogledajte: je li ikad muškarac rodio? A svi se muškarci za bedra hvataju kao porodilje! Zašto su sva lica izobličena i problijedjela? 
\par 7 Jao, jer velik je dan ovaj, slična mu nÓe bÄi! Vrijeme je nevolje za Jakova, al' će se izbaviti iz nje. 
\par 8 Onoga dana - riječ je Jahve nad Vojskama - slomit ću jaram  na njihovu vratu i lance ću njihove raskinuti. Više neće služiti  tuđinu, 
\par 9 već će služiti Jahvi, Bogu svojemu, i Davidu, kralju  svome, koga ću im podići. 
\par 10 Ne boj se, Jakove, slugo moja - riječ je Jahvina - ne plaši se, Izraele! Jer evo, spasit ću te izdaleka i potomstvo tvoje iz zemlje izgnanstva. Jakov će se opet smiriti, spokojno će živjeti i nitko ga neće plašiti - riječ je Jahvina - 
\par 11 jer ja sam s tobom da te izbavim. Zatrt ću narode među koje te prognah, a tebe neću sasvim uništiti; al' ću te kazniti po pravici, ne smijem te pustit' nekažnjena." 
\par 12 Uistinu, ovako govori Jahve: "Neiscjeljiva je rana tvoja, neprebolan polom tvoj. 
\par 13 Nema lijeka rani tvojoj i nikako da zaraste. 
\par 14 Zaboraviše te svi ljubavnici, više za te i ne pitaju! Jer po tebi ja udarih k'o što udara neprijatelj, kaznom krutom za bezakonje i za mnoge grijehe tvoje. 
\par 15 Zašto kukaš zbog rane svoje? Zar je neizlječiva tvoja bol? Zbog mnoštva bezakonja i grijeha silnih tvojih to ti učinih. 
\par 16 Al' i oni što te žderu bit će prožderani, u ropstvo će svi dušmani tvoji; pljačkaši tvoji bit će opljačkani, i koji te plijeniše bit će oplijenjeni. 
\par 17 [17b] Zvahu te 'Protjeranom' i 'Sionkom za koju nitko ne pita'. [17a] Al' ja ću te iscijeliti, rane ti zaliječiti" - riječ je Jahvina. 
\par 18 Ovako govori Jahve: "Evo, izmijenit ću udes šatora Jakovljevih, smilovat ću se na stanove njegove: na razvalinama njegovim bit će opet grad sazidan, i dvori će stajati na starome mjestu. 
\par 19 Iz njih će se čuti hvalospjev, i glasovi radosni. Umnožit ću ih i više im se neće smanjiti broj, ugled ću im dati i više ih neće prezirati. 
\par 20 Sinovi njihovi bit će mi kao nekoć, zajednica njina preda mnom će čvrsto stajati, a kaznit ću sve njihove ugnjetače. 
\par 21 Glavar njihov iz njih će niknuti, vladar njihov isred njih će izaći. Pustit ću ga k sebi da mi se približi - jer tko da se usudi sam preda me!" - riječ je Jahvina. 
\par 22 "I vi ćete biti moj narod, a ja vaš Bog. 
\par 23 Gle, nevrijeme Jahvino, jarost provaljuje, razmahuje se vihor silan, i svaljuje na glave bezbožničke. 
\par 24 Jahvin se gnjev neće stišati dok on ne izvrši i ne ispuni naume srca svojega. U dane posljednje jasno ćete to razumjeti. 


\chapter{31}

\par 1 "U ono vrijeme - riječ je Jahvina - bit ću Bog svim plemenima  Izraelovim i oni će biti narod moj." 
\par 2 Ovako govori Jahve: "Nađe milost u pustinji narod koji uteče maču: Izrael ide u svoje prebivalište. 
\par 3 Iz daljine mu se Jahve ukaza: Ljubavlju vječnom ljubim te, zato ti sačuvah milost. 
\par 4 Opet ću te sazdati, i bit ćeš sazdana, djevice Izraelova. Opet ćeš se resit' bubnjićima, u veselo kolo hvatati. 
\par 5 Opet ćeš saditi vinograde na brdima Samarije: koji nasade posade, oni će i trgati. 
\par 6 Jer dolazi dan te će stražari vikati na brdu efrajimskom: 'Na noge! Na Sion se popnimo, k Jahvi, Bogu svojemu!'" 
\par 7 Jer ovako govori Jahve: "Kličite od radosti Jakovu, pozdravite burno prvaka naroda! Neka se ori vaš glas! Objavite slavopojkom: Jahve spasi narod svoj, Ostatak Izraelov! 
\par 8 Evo, ja ih vodim iz zemlje sjeverne, skupljam ih s krajeva zemlje: s njima su slijepi i hromi, trudnice i rodilje: vraća se velika zajednica. 
\par 9 Evo, u suzama pođoše, utješene sad ih vraćam! Vodit ću ih kraj potočnih voda, putem ravnim kojim neće posrnuti, jer ja sam otac Izraelu, Efrajim je moj prvenac." 
\par 10 Čujte, o narodi, riječ Jahvinu, objavite je širom dalekih otoka: "Onaj što rasprši Izraela, opet ga sabire i čuva ga k'o pastir stado svoje!" 
\par 11 Jer Jahve oslobodi Jakova, izbavi ga iz ruku jačeg od njega. 
\par 12 I oni će, radosno kličući, na vis sionski da se naužiju dobara Jahvinih: žita, ulja, mladog vina, jagnjadi i teladi, duša će im biti kao vrt navodnjen, nikad više neće ginuti. 
\par 13 Djevojke će se veselit' u kolu, mlado i staro zajedno, jer ću im tugu u radost pretvoriti, utješit ću ih i razveselit' nakon žalosti. 
\par 14 Pretilinom ću im okrijepiti svećenstvo i narod svoj nasititi dobrima" - riječ je Jahvina. 
\par 15 Ovako govori Jahve: "Čuj! U Rami se kukanje čuje i gorak plač: Rahela oplakuje sinove svoje, i neće da se utješi za djecom, jer njih više nema." 
\par 16 Ovako govori Jahve: "Prestani kukati, otari suze u očima! Patnje će tvoje biti nagrađene: oni će se vratiti iz zemlje neprijateljske. 
\par 17 Ima nade za tvoje potomstvo - riječ je Jahvina - sinovi tvoji vratit će se u svoj kraj. 
\par 18 Dobro čujem Efrajimov jecaj: 'Ti me pokara, i ja se popravih kao june još neukroćeno. Obrati me, da se obratim, jer ti si, Jahve, Bog moj. 
\par 19 Odvratih se od tebe, ali se pokajah, uvijek, i sad se u slabine tučem. Stidim se i crvenim, jer nosim sramotu mladosti svoje!'" 
\par 20 "Zar mi je Efrajim sin toliko drag, dijete najmilije? Jer koliko god mu prijetim, bez prestanka živo na njega mislim i srce mi dršće za njega od nježne samilosti" - riječ je Jahvina. 
\par 21 "Postavi putokaze, podigni stupove! Sjeti se ceste, puta kojim si prošla. I vrati se, djevice Izraelova, vrati se u gradove svoje! 
\par 22 Dokle ćeš još oklijevati, kćeri odmetnice? Jer Jahve stvori nešto novo na zemlji: Žena će okružiti Muža." 
\par 23 Ovako govori Jahve nad Vojskama, Bog Izraelov: "U zemlji  Judinoj, kad promijenim njezinu sudbinu, u njezinim će se gradovima  ovako govoriti: 'Blagoslovio te Jahve, prebivalište Pravednosti, Goro sveta!'" 
\par 24 "I u njoj će se opet nastaniti Judeja sa svim svojim  gradovima, ratari i oni što idu za stadima, 
\par 25 jer ja ću okrijepiti  dušu iscrpljenu, obilno nahraniti dušu klonulu. 
\par 26 Kao ono: 'Čim se probudih, pogledah: sladak li bijaše sanak moj!'" 
\par 27 "Evo dolaze dani - riječ je Jahvina - kad ću u domu Izraelovu  i u domu Judinu posijati sjeme čovječje i sjeme životinjsko. 
\par 28 I kao što sam nekoć bdio da ih iščupam, razvalim, istrijebim, zatrem i nesreću na njih svalim, tako ću sada brižno bdjeti  da ih podignem i posadim. 
\par 29 U one dane neće se više govoriti: 'Oci jedoše kiselo grožđe, a sinovima zubi trnu.' 
\par 30 Nego će svatko umrijeti zbog vlastite krivice. I onomu  koji bude jeo kiselo grožđe zubi će trnuti." 
\par 31 "Evo dolaze dani - riječ je Jahvina - kad ću s domom  Izraelovim i s domom Judinim sklopiti Novi savez. 
\par 32 Ne Savez  kakav sam sklopio s ocima njihovim u dan kad ih uzeh za ruku  da ih izvedem iz zemlje egipatske, Savez što ga oni razvrgoše  premda sam ja gospodar njihov - riječ je Jahvina. 
\par 33 Nego, ovo  je Savez što ću ga sklopiti s domom Izraelovim poslije onih dana  - riječ je Jahvina: Zakon ću svoj staviti u dušu njihovu i upisati  ga u njihovo srce. I bit ću Bog njihov, a oni narod moj. 
\par 34 I  neće više učiti drug druga ni brat brata govoreći: 'Spoznajte  Jahvu!' nego će me svi poznavati, i malo i veliko - riječ je  Jahvina - jer ću oprostiti bezakonje njihovo i grijeha se njihovih  neću više spominjati." 
\par 35 Ovako govori Jahve, koji daje da sunce sjaje danju, a mjesec i zvijezde da svijetle noću, koji burka more da mu valovi buče - ime mu je Jahve nad Vojskama: 
\par 36 "Ako se ikad ti zakoni poremete preda mnom - riječ je Jahvina - onda će i potomstvo Izraelovo prestati da bude narod pred licem mojim zauvijek! 
\par 37 Ako se mogu izmjeriti nebesa gore, i dolje istražiti temelji zemlje, onda ću i ja odbaciti potomstvo Izraelovo zbog svega što počiniše" - riječ je Jahvina. 
\par 38 "Evo dolaze dani - riječ je Jahvina - kada će grad Jahvin  biti opet sazidan, od Kule Hananelove do Vrata ugaonih. 
\par 39 I  još će se dalje protegnuti mjerničko uže, pravo do brežuljka  Gareba, a onda okrenuti prema Goi. 
\par 40 I sva dolina trupla i  pepela, i sva polja do potoka Kidrona, do ugla Konjskih vrata  na istoku, bit će svetinja Jahvina. I neće više biti razaranja  ni prokletstva." 


\chapter{32}

\par 1 Riječ koju Jahve uputi Jeremiji desete godine Sidkije, kralja  judejskoga, to jest osamnaeste godine Nabukodonozorove. 
\par 2 U to vrijeme vojska kralja babilonskoga opsjedaše Jeruzalem, a prorok Jeremija bijaše zatvoren u tamničkom dvorištu u dvoru  judejskoga kralja. 
\par 3 Sidkija, kralj judejski, bijaše ga ondje  zatvorio, prigovoriv mu: "Zašto si prorokovao: 'Ovako govori  Jahve: Gle, grad ću ovaj predati u ruke kralju babilonskom da  ga osvoji; 
\par 4 a Sidkija, kralj judejski, neće umaći sili kaldejskoj, nego će biti predan u ruke kralja babilonskoga - usta u usta  s njim će govoriti, oči u oči njega vidjeti. 
\par 5 Sidkiju će odvesti  u Babilon i ondje će ostati dok ga ne pohodim - riječ je Jahvina!  I ako se budete borili protiv Kaldejaca, nećete uspjeti!'" 
\par 6 Tada reče Jeremija: "Dođe mi riječ Jahvina: 
\par 7 'Uskoro  će doći k tebi Hanamel, sin tvoga strica Šaluma, da ti kaže:  Kupi njivu moju u Anatotu; ti imaš rodbinsko pravo da je kupiš!' 
\par 8 Kako je Jahve navijestio, k meni dođe moj stričević Hanamel  u tamničko dvorište i reče mi: 'De kupi moju njivu u Anatotu, jer ti imaš pravo na posjed i rodbinsko pravo da je kupiš! Kupi  je!' I tada spoznah da to bijaše riječ Jahvina. 
\par 9 Kupih, dakle, tu njivu od stričevića Hanamela iz Anatota te mu izmjerih u  novcu sedamnaest šekela srebra. 
\par 10 Napišem ugovor, udarim pečat, pozovem svjedoke i izmjerim novac na tezulji. 
\par 11 Zatim uzmem  kupovni ugovor, zapečaćen prema propisu i uredbama, 
\par 12 predam  kupovni ugovor Baruhu, sinu Mahsejeva sina Nerije. Nazočni su  bili: moj stričević Hanamel, svjedoci što su potpisali kupovni  ugovor i svi Judejci koji su se našli u tamničkom dvorištu. 
\par 13 Tada  pred njima zapovjedim Baruhu: 
\par 14 'Ovako govori Jahve nad Vojskama, Bog Izraelov: Uzmi ove isprave, ovaj kupovni ugovor, zapečaćeni  i otvoreni, i stavi ih u glinenu posudu da se zadugo sačuvaju. 
\par 15 Jer ovako govori Jahve nad Vojskama, Bog Izraelov: Još će  se u ovoj zemlji kupovati i kuće, i njive, i vinogradi!'" 
\par 16 Pošto kupovni ugovor predadoh Nerijinu sinu Baruhu, pomolih  se Jahvi: 
\par 17 "O, Jahve, Gospode! Ti stvori nebo i zemlju snagom  velikom, rukom uzdignutom! Ništa tebi nije nemoguće! 
\par 18 Tisućama  iskazuješ milost, a krivnju otaca osvećuješ na djeci, potomcima  njihovim. Bože veliki i moćni, kome je ime Jahve nad Vojskama! 
\par 19 Velik si u svojim naumima, silan u svojim djelima! Oči tvoje  bde nad svim putovima ljudskim da naplatiš svakome prema putu  njegovu i prema plodu djela njegovih! 
\par 20 Ti koji si činio znamenja  i čudesa u zemlji egipatskoj i u Izraelu, i među svim ljudima  sve do danas, 
\par 21 ti si izveo svoj narod izraelski iz zemlje  egipatske znamenjima i čudesima, rukom moćnom i mišicom podignutom, strahotama velikim. 
\par 22 Zatim im dade svu ovu zemlju koju si  zakletvom obećao ocima njihovim, zemlju u kojoj teče med i mlijeko. 
\par 23 I oni je zaposjedoše; ali nisu slušali glasa tvojega niti  su hodili putem Zakona tvojega. Ništa ne učiniše od onog što  im ti naredi; zato si dozvao na njih sve ove nevolje. 
\par 24 Gle, nasipi se već primakoše gradu, i bit će osvojen, i grad će pasti  u ruke Kaldejcima koji na nj navaljuju mačem, glađu i kugom.  Čime si prijetio, evo dolazi. I sam vidiš. 
\par 25 A ti mi, Jahve  Gospode moj, reče: 'Kupi novcem njivu i pozovi svjedoke', a grad  je već predan u ruke Kaldejcima!" 
\par 26 Tada mi dođe riječ Jahvina: 
\par 27 "Gle, ja sam Jahve, Bog  svakoga tijela! Meni ništa nije nemoguće! 
\par 28 Zato - veli Jahve - grad ovaj predajem u ruke Kaldejaca  i u ruke kralja babilonskoga, koji će ga zauzeti. 
\par 29 Ući će  u nj Kaldejci koji se bore protiv ovoga grada, ognjem će ga uništiti  i spaliti ga zajedno s kućama kojima su na krovovima Baalu palili  tamjan i lijevali ljevanice tuđim bogovima, mene gnjeveći. 
\par 30 Jer  sinovi Izraelovi i sinovi Judini od mladosti čine samo zlo pred  mojim očima. Doista, sinovi Izraelovi bez prestanka me gnjeve  djelima ruku svojih - riječ je Jahvina. 
\par 31 Grad ovaj, doista, samo mi je na gnjev i srdžbu otkako je sagrađen pa do dana današnjega  te ga moram ukloniti ispred lica svojega 
\par 32 zbog svega bezakonja  što ga sinovi Izraelovi i sinovi Judini počiniše, gnjeveći me  - oni i kraljevi njihovi, knezovi i svećenici i proroci, Judejci  i Jeruzalemci. 
\par 33 Okretahu mi leđa, a ne lice svoje, iako se  neumorno trudih da ih poučim, ali me ne slušaše niti nauk moj  primiše. 
\par 34 Postaviše grozote u Dom koji se mojim zove imenom  da ga oskvrnu. 
\par 35 Baalu podigoše uzvišice u Dolini Ben Hinomu, i sinove i kćeri svoje Moleku kroz oganj provodiše - što im  ja nikad ne zapovjedih; ni na um mi ne pade da bi činili takve  gadosti niti da bih Judu pustio u takav grijeh." 
\par 36 Ipak, ovako govori Jahve, Bog Izraelov, o tom gradu za  koji vi velite da će od mača, gladi i kuge pasti u ruke kralju  babilonskom: 
\par 37 "Evo, ja ću ih sabrati iz svih zemalja u koje  ih prognah - u gnjevu i jarosti svojoj - i vratit ću ih na ovo  mjesto da ovdje spokojno žive. 
\par 38 I oni će biti narod moj, a  ja, ja ću biti Bog njihov. 
\par 39 I dat ću im srce jedno i put jedan, da bi me se bojali u sve dane, na sreću svoju i djece svoje. 
\par 40 I sklopit ću s njima Savez vječan, nikad se više neću odvratiti  od njih i uvijek ću im činiti dobro; usadit ću im u srce svoj  strah, da se nikad više ne odmetnu od mene. 
\par 41 I radovat ću  se čineći im dobro; i čvrsto ću ih zasaditi u ovoj zemlji, svim  srcem svojim, svom dušom svojom." 
\par 42 Jer ovako govori Jahve:  "Kao što sam na ovaj narod doveo svu ovu strašnu nesreću, tako  ću na njih dovesti svu sreću koju im obrekoh. 
\par 43 Da, opet će  se kupovati njive u ovoj zemlji o kojoj vi velite: 'Ova je pustinja, bez čovjeka i živinčeta, predana na milost i nemilost Kaldejcima!' 
\par 44 Njive će se za novac kupovati, pisat će se i pečatiti kupovni  ugovori, pozivat će se svjedoci u zemlji Benjaminovoj i u okolici  Jeruzalema. U gradovima Judinim i u gradovima Gorja, Šefele,  Negeba, jer ću promijeniti udes njihov" - riječ je Jahvina. 


\chapter{33}

\par 1 Dok je Jeremija bio još zatvoren u tamničkom dvorištu, i drugi  mu put dođe riječ Jahvina: 
\par 2 "Ovako govori Jahve, koji stvori  zemlju, oblikova je i učvrsti - ime mu je Jahve! 
\par 3 Zazovi me, i odazvat ću ti se i objavit ću ti velike i nedokučive tajne  o kojima ništa ne znaš. 
\par 4 Jer ovako govori Jahve, Bog Izraelov, o kućama ovoga grada i o dvorima kraljeva judejskih, porušenim  zbog nasipa i mača, 
\par 5 i o onima što zameću borbu s Kaldejcima  da napune svoje kuće tjelesima ljudi koje pobih u srdžbi i jarosti  svojoj, i odvratih lice svoje od ovoga grada zbog njihove opakosti. 
\par 6 Evo, ja ću zaliječiti njihovu ranu, ja ću ih iscijeliti  i ozdraviti i pružiti im obilje istinskoga mira. 
\par 7 Promijenit  ću udes zemlje Judine i Jeruzalema i podići ću ih da budu kao  nekoć. 
\par 8 Očistit ću ih od svakoga grijeha kojim sagriješiše  protiv mene i oprostit ću im sve krivice koje mi skriviše odmetnuv  se od mene. 
\par 9 I Jeruzalem će mi biti na radost, na hvalu i čast  pred svim narodima svijeta: kad čuju za sve dobro kojim ću ih  nadijeliti, divit će se i čuditi svoj onoj sreći i miru što ću  im ja dati." 
\par 10 Ovako govori Jahve: "Na ovome mjestu o kojemu vi velite:  'To je pustinja bez čovjeka i bez živinčeta' - u gradovima judejskim  i po opustošenim ulicama jeruzalemskim opet će se oriti 
\par 11 poklici  radosti, poklici zaručnika i zaručnice, poklici onih koji će  u Domu Jahvinu prinositi žrtve zahvalnice pjevajući: 'Hvalite  Jahvu nad Vojskama, jer je dobar Jahve - vječna je ljubav njegova!'  Jer ja ću obnoviti zemlju da bude kao nekoć" - riječ je Jahvina. 
\par 12 Ovako govori Jahve nad Vojskama: "Na ovome mjestu koje  je sada pusto, bez čovjeka i bez živinčeta, i u svim gradovima  opet će biti pašnjaci za pastire što odmaraju stada svoja. 
\par 13 U  gradovima Gorja, i u gradovima Šefele, i u gradovima Negeba,  u kraju Benjaminovu, u okolici Jeruzalema i u gradovima Judinim  opet će prolaziti ovce ispod ruke pastira koji će ih brojiti"  - riječ je Jahvina. 
\par 14 "Evo, dolaze dani - riječ je Jahvina - kad ću ispuniti  dobro obećanje što ga dadoh domu Izraelovu i domu Judinu: 
\par 15 U one dane i u vrijeme ono podići ću Davidu izdanak pravedni; on će zemljom vladati po pravu i pravici. 
\par 16 U one dane Judeja će biti spašena, Jeruzalem će živjeti spokojno. A grad će se zvati: 'Jahve, Pravda naša.' 
\par 17 Jer ovako govori Jahve: "Nikada Davidu neće nestati potomka  koji će sjediti na prijestolju doma Izraelova. 
\par 18 I nikada neće  levitima i svećenicima nestati potomaka koji će služiti preda  mnom i prinositi paljenice, kaditi prinosnice i prikazivati klanice  u sve dane." 
\par 19 I dođe riječ Jahvina Jeremiji: 
\par 20 Ovako govori Jahve:  "Ako možete razvrći savez moj s danom i savez moj s noći, tako  da ni dana ni noći više ne bude u pravo vrijeme, 
\par 21 moći će  se raskinuti i Savez moj sa slugom mojim Davidom te više neće  imati sina koji bi kraljevao na prijestolju njegovu i s levitima  i svećenicima koji mi služe. 
\par 22 Kao što se vojska nebeska ne  može izbrojiti ni izmjeriti pijesak morski, tako ću umnožiti  potomstvo sluge svojega Davida i levite i svećenike koji mi služe." 
\par 23 I dođe riječ Jahvina Jeremiji: 
\par 24 "Nisi li opazio što  ovi ljudi govore: 'Jahve je odbacio obadva plemena koja je bio  sebi izabrao?' I s prezirom poriču narod moj kao da mi više nije  narod." 
\par 25 Ovako govori Jahve: "Da ne sklopih saveza svojega s danom  i noći i da ne postavih zakone nebu i zemlji, 
\par 26 mogao bih odbaciti  potomstvo Jakova i Davida, sluge svojega, da više ne uzimam potomka  njihova za vladara nad potomstvom Abrahamovim, Izakovim i Jakovljevim, kad promijenim udes njihov i kad im se smilujem." 


\chapter{34}

\par 1 Riječ koju Jahve uputi Jeremiji kad Nabukodonozor, kralj babilonski, i sva njegova vojska, i sva kraljevstva pod njegovom vlašću, i svi narodi navališe na Jeruzalem i na sve gradove njegove. 
\par 2 Ovako govori Jahve, Bog Izraelov: "Idi i govori sa Sidkijom, kraljem judejskim, i reci mu: Ovako govori Jahve: 'Evo, predajem  ovaj grad u ruke kralja babilonskoga da ga on ognjem spali. 
\par 3 Ni  ti nećeš ruci njegovoj umaći. Da, bit ćeš uhvaćen i predat će  te u njegove ruke; oči u oči gledat ćeš kralja babilonskoga,  usta u usta on će s tobom govoriti i bit ćeš odveden u Babilon.' 
\par 4 Ali čuj riječ Jahvinu, Sidkija, kralju judejski! Ovo ti poručuje  Jahve: 'Nećeš od mača poginuti, 
\par 5 umrijet ćeš u miru! I kao  što su tvoje očeve i kraljeve tvoje prethodnike okadili, i tebe  će okaditi i naricat će za tobom: 'Jao Gospodaru!' Ja ti to govorim'  - riječ je Jahvina. 
\par 6 I prorok Jeremija poruči sve ove riječi Sidkiji, kralju  judejskom u Jeruzalemu, 
\par 7 dok je vojska kralja babilonskoga  navaljivala na Jeruzalem i na preostale gradove Judine - na Lakiš  i Azeku, jer još samo oni preostadoše od judejskih utvrđenih  gradova. 
\par 8 Riječ koju Jahve uputi Jeremiji, pošto je kralj Sidkija  sa svekolikim narodom jeruzalemskim sklopio savez da im proglasi  slobodu, 
\par 9 da svaki pusti na slobodu svoga roba Hebreja i svoju  robinju Hebrejku te da više ni u koga ne bude Hebrej, brat njegov, kao rob. 
\par 10 I svi odličnici i sav narod koji uđoše u ovaj savez  pristadoše te svaki pusti na slobodu roba svoga i svoju ropkinju  da im više ne robuju. Pristadoše, dakle, i pustiše ih. 
\par 11 A  potom se okrenuše i uzeše opet svoje robove i ropkinje koje bijahu  oslobodili pa ih prisiliše da im opet robuju. 
\par 12 Tada Jahve  uputi riječ Jeremiji govoreći: 
\par 13 Ovako govori Jahve, Bog Izraelov: "Ja sam sklopio Savez  s ocima vašim u dan kada ih izvedoh iz Egipta, iz zemlje ropstva, govoreći: 
\par 14 'Nakon sedam godina neka svaki od vas pusti na  slobodu brata svoga Hebreja koji mu se prodao i šest godina kao  rob služio.' Ali me vaši oci ne poslušaše i ne htjedoše me čuti. 
\par 15 A vi se bijaste obratili i učinili što je pravo u očima mojim, proglasivši slobodu za svakoga bližnjega svoga i preda mnom  ste sklopili savez u Domu koji se zove mojim imenom. 
\par 16 A zatim  se okrenuste i oskvrnuste ime moje, jer je svaki od vas opet  uveo svoga roba i ropkinju koje ste već bili oslobodili, i ponovo  ste ih prisilili da vam robuju." 
\par 17 Zato ovako govori Jahve: "Vi me ne poslušaste da proglasite  slobodu subratu svojemu i bližnjemu. I zato, evo, i ja proglašavam  protiv vas slobodu - riječ je Jahvina - maču, kugi i gladi, i  učinit ću vas strašilom svim kraljevstvima zemlje. 
\par 18 A s ljudima  koji razvrgoše Savez moj i ne ispuniše saveza obećana pred mojim  licem postupit ću kao s teletom što ga nadvoje rasjekoše te između  tih pola prođoše. 
\par 19 Knezove Judeje i Jeruzalema, dvorjane,  svećenike i sav narod zemlje što prođoše između pola telećih 
\par 20 predat ću u ruke dušmana koji im rade o glavi, a njihova  trupla bit će hrana pticama nebeskim i zvijerima zemaljskim. 
\par 21 Sidkiju, kralja judejskoga, i njegove knezove predat ću u  ruke dušmana koji im rade o glavi i u ruke vojske kralja babilonskoga, koja se od vas bila povukla. 
\par 22 Evo, ja ću im zapovjediti -  riječ je Jahvina - i vratit ću ih na ovaj grad, i navalit će  na nj, osvojiti ga i ognjem spaliti. A gradove judejske obratit  ću u pustinju nenastanjenu." 


\chapter{35}

\par 1 Jahve uputi riječ Jeremiji u dane Jojakima, sina Jošijina, kralja judejskoga: 
\par 2 "Idi u zajednicu Rekabovaca, govori s  njima i dovedi ih u Dom Jahvin, u jednu od dvorana, i daj im  vina." 
\par 3 Tada dovedoh Jaazaniju, sina Habasinijina sina Jeremije, njegovu braću i sve sinove njegove i sav dom Rekabovaca 
\par 4 i  dovedoh ih u Dom Jahvin, u dvoranu čovjeka Božjega Ben Johanana, sina Jigdalijina, koja je kraj dvorane kneževske, a nad dvoranom  vratara Maaseje, sina Šalumova. 
\par 5 Zatim stavih pred sinove doma  Rekabova krčage pune vina i čaše te im rekoh: "Pijte vina!" 
\par 6 Ali oni odgovoriše: "Ne pijemo vina, jer nam je otac naš  Jonadab, sin Rekabov, zapovjedio: 'Ne smijete nikada piti vina, ni vi ni sinovi vaši. 
\par 7 Niti smijete graditi kućÄa, niti sijati  sjemena ni saditi vinogradÄa, niti ih posjedovati, nego provodite  sav život pod šatorima, da dugo živite u zemlji gdje kao stranci  boravite.' 
\par 8 I mi poslušasmo glas oca Jonadaba, sina Rekabova, u svem što nam je zapovjedio: da nikad vina ne pijemo, ni mi  ni žene naše, niti sinovi naši, ni kćeri naše, 
\par 9 da ne gradimo  kuća, ni da posjedujemo vinograda ni polja zasijanih, 
\par 10 da  stanujemo pod šatorima i držimo se poslušno svega što nam zapovjedi  naš otac Jonadab. 
\par 11 Samo kada je Nabukodonozor, kralj babilonski, krenuo protiv ove zemlje, rekosmo: 'Hajdemo, pođimo u Jeruzalem  da izbjegnemo vojsku kaldejsku i vojsku aramejsku!' I tako sada  živimo u Jeruzalemu." 
\par 12 Tada dođe riječ Jahvina Jeremiji: 
\par 13 Ovako govori Jahve  nad Vojskama, kralj Izraelov: "Idi i objavi Judejcima i Jeruzalemcima:  'Zar nećete primiti nauka moga i poslušati riječi moje?' - riječ  je Jahvina. - 
\par 14 'Ispunjuju se riječi Jonadaba, sina Rekabova, koji je sinovima svojim zabranio da piju vina, i do dana današnjega  nitko ga nije pio, jer oni slušaju riječ svoga oca. A ja sam  vam jednako govorio, ali me niste slušali. 
\par 15 I slao sam bez  prestanka k vama sluge svoje, proroke, da vam propovijedaju:  'Vratite se svaki sa svoga opakog puta, popravite djela svoja  i ne trčite za tuđim bogovima da im služite, pa ćete ostati u  zemlji koju dadoh vama i ocima vašim'; ali ne prikloniste uha  svojega i ne poslušaste me. 
\par 16 Sinovi Jonadaba, sina Rekabova, držahu se zapovijedi koju im dade otac njihov. Ali mene ovaj  narod ne sluša.' 
\par 17 Zato govori Jahve nad Vojskama, Bog Izraelov:  'Evo, navući ću na sve Jeruzalemce sve one nevolje kojima sam  im zaprijetio, jer sam im govorio, a oni me ne slušahu, dozivao  ih, ali se oni ne odazivahu.'" 
\par 18 Zajednici Rekabovaca Jeremija reče: "Ovako govori Jahve  nad Vojskama, Bog Izraelov: 'Jer ste slušali zapovijedi svoga  oca Jonadaba i držali se svih naredaba i činili sve što vam je  on zapovjedio, 
\par 19 zato - ovako govori Jahve nad Vojskama, kralj  Izraelov - Jonadabu, sinu Rekabovu, nikad neće ponestati potomka  koji će stajati pred licem mojim u sve dane.'" 


\chapter{36}

\par 1 Četvrte godine Jojakima, sina Jošijina, kralja judejskoga, uputi Jahve Jeremiji ovu riječ: 
\par 2 "Uzmi svitak i zapiši na  nj sve riječi koje ti kazah o Jeruzalemu, Judeji i svim narodima, od dana kad ti počeh govoriti, od dana Jošijinih do dana današnjega. 
\par 3 Možda će čuti dom Judin o svim nesrećama što sam ih naumio  oboriti na njih te će se vratiti svaki sa svoga zlog puta, a  ja ću im oprostiti krivicu i grijeh njihov." 
\par 4 Tada Jeremija dozva Baruha, sina Nerijina, i Baruh napisa  na svitak, po kazivanju Jeremijinu, sve riječi koje mu Jahve  bijaše objavio. 
\par 5 Tada Jeremija naredi Baruhu: "Meni nije slobodno te ne mogu  poći u Dom Jahvin. 
\par 6 Idi ti te na dan posta u Domu Jahvinu čitaj  narodu riječi Jahvine iz svitka što si ga po mojem kazivanju  napisao. Pročitaj ih i svim Judejcima koji su došli iz svojih  gradova. 
\par 7 Možda će se vapaji njihovi vinuti k Jahvi i možda  će se obratiti svatko sa zloga puta svojega; jer je velik bijes  i srdžba kojima Jahve prijeti ovom narodu." 
\par 8 I Baruh, sin Nerijin, učini sve kako mu prorok Jeremija  bijaše zapovjedio da pročita riječi Jahvine u Domu Jahvinu. 
\par 9 U petoj godini Jojakima, sina Jošijina, kralja judejskoga, mjeseca devetoga, pozvaše na post pred Jahvu sav narod jeruzalemski  i sav narod što mogaše stići iz gradova judejskih u Jeruzalem. 
\par 10 Baruh  svemu narodu pročita riječi Jeremije iz svitka u Domu Jahvinu, u dvorani Gemarje, sina pisara Šafana, u gornjem predvorju pred  Novim vratima Jahvina Doma. 
\par 11 A kad Mikaj, sin Šafanova sina Gemarje, ču iz knjige  sve Jahvine riječi, 
\par 12 siđe u kraljevski dvor u sobu pisarovu, gdje upravo sjeđahu svi dostojanstvenici: pisar Elišama, Delaja, sin Šemajin, Elnatan, sin Akborov, Gemarja, sin Šafanov, Sidkija, sin Hananijin, i svi drugi dostojanstvenici. 
\par 13 Mikaj im kaza  sve riječi što ih bijaše čuo kad ih je Baruh narodu čitao iz  knjige. 
\par 14 Tada dostojanstvenici poslaše Jehudija, sina Netanijina, i Šelemju, sina Kušijeva, Baruhu da mu kažu: "Uzmi u ruke svitak  iz kojega si čitao narodu i dođi!" Tada Baruh, sin Nerijin, uze  svitak u ruke i dođe k njima. 
\par 15 Oni mi rekoše: "Hajde, sjedi  i pročitaj nam!" I Baruh im pročita. 
\par 16 Kad čuše sve one riječi, uplašeno se pogledaše i rekoše Baruhu: "Moramo sve to kazati  kralju." 
\par 17 I upitaše Baruha: "Hajde, objasni nam kako ti napisa  sve te riječi." 
\par 18 Baruh će njima: "Jeremija mi je sve te riječi  u pero kazivao, a ja sam ih crnilom u knjigu zapisao." 
\par 19 Tada  dostojanstvenici rekoše Baruhu: "Idi i sakrijte se, ti i Jeremija;  nitko da ne zna gdje ste." 
\par 20 I, pohranivši svitak u dvorani  pisara Elišame, pođoše kralju u dvorsko predvorje i sve mu ispripovjediše. 
\par 21 Kralj posla Jehudija da donese svitak: on ga donese iz  sobe pisara Elišame i pročita ga kralju i dostojanstvenicima  koji stajahu oko njega. 
\par 22 Kralj je sjedio u zimskim odajama  - bijaše to u devetom mjesecu - a pred njim stajaše ražarena  žeravnica. 
\par 23 I kako bi Jehudi pročitao tri-četiri stupca, kralj  bi ih rezao pisarskim perorezom i bacao u vatru na žeravnice  sve dok cio svitak ne bi uništen u vatri žeravnice. 
\par 24 Ni kralj  ni njegove sluge ne prestrašiše se niti razderaše haljina kad  čuše te riječi, 
\par 25 pa ipak Elnatan, Delaja i Gemarja moljahu  kralja da ne spali svitak, ali on ih ne posluša. 
\par 26 Tada kralj  zapovjedi kraljeviću Jerahmeelu i Seraji, sinu Azrielovu, i Šelemji, sinu Abdeelovu, da uhvate pisara Baruha i proroka Jeremiju.  Ali ih Jahve bijaše sakrio. 
\par 27 Pošto je dakle kralj spalio svitak s riječima što ih  Baruh bijaše zapisao po Jeremijinu kazivanju, dođe riječ Jahvina  Jeremiji: 
\par 28 "Uzmi drugi svitak i upiši u nj sve one riječi  što bijahu na prvom svitku koji je Jojakim, kralj judejski, spalio. 
\par 29 A protiv Jojakima, kralja judejskoga, ovako reci: Ovako govori  Jahve: Spalio si svitak govoreći: 'Zašto si u njemu napisao da  će doći kralj babilonski koji će opustošiti zemlju ovu i istrijebiti  i ljude i stoku?' 
\par 30 Zato ovako govori Jahve protiv Jojakima, kralja judejskoga: 'On neće imati potomka da sjedne na prijestolje  Davidovo, a njegovo mrtvo tijelo bit će bačeno na pripeku danju  i noćni mraz. 
\par 31 Kaznit ću njega, i potomstvo njegovo, i sluge  njegove zbog njihova bezakonja, i svalit ću na Jeruzalemce i  na Judejce sve zlo kojim sam im prijetio, a nisu me slušali." 
\par 32 Tada Jeremija uze drugi svitak, dade ga pisaru Baruhu, sinu Nerijinu, i on po kazivanju Jeremijinu upisa sve riječi  knjige koju je Jojakim, kralj judejski, na žeravnici spalio.  I k njima je dopisano još mnogo onakvih riječi. 


\chapter{37}

\par 1 Nakon Konije, sina Jojakimova, zakralji se Sidkija, sin Jošijin.  Nabukodonozor, kralj babilonski, postavi ga za kralja u zemlji  judejskoj. 
\par 2 Ali ni on ni sluge njegove ni narod zemlje ne slušahu  riječi što ih je Jahve govorio na usta proroka Jeremije. 
\par 3 Kralj Sidkija posla Jehukala, sina Šelemjina, i svećenika  Sefaniju, sina Maasejina, k proroku Jeremiji s porukom: "Daj, pomoli se za nas Jahvi, Bogu našemu!" 
\par 4 Jeremija u ono vrijeme  još zalažaše među narod i još ga ne bijahu bacili u tamnicu. 
\par 5 A vojska je faraonova nadirala iz Egipta: čuvši to, Kaldejci, koji opsjedahu Jeruzalem, udaljiše se od grada. 
\par 6 Tada se javi riječ Jahvina proroku Jeremiji: 
\par 7 Ovako  govori Jahve, Bog Izraelov: "Kralju judejskomu, koji vas posla  k meni da me pitate, ovako recite: 'Evo, vojska faraonova, koja  vam priteče u pomoć, vratit će se u svoju zemlju Egipat. 
\par 8 Kaldejci  će opet napasti ovaj grad, osvojiti ga i spaliti.' 
\par 9 Ovako govori  Jahve: 'Ne zanosite se mišlju: 'Kaldejci će otići od nas', jer  oni neće otići! 
\par 10 Pa i da razbijete svu vojsku kaldejsku koja  se bori s vama, tako da bi od nje ostali samo ranjenici, oni  bi, svaki iz svoga šatora, opet poustajali da požarom unište  ovaj grad.'" 
\par 11 Kad je vojska kaldejska zbog vojske faraonove morala  prekinuti opsadu Jeruzalema, 
\par 12 i Jeremija htjede otići iz Jeruzalema  da ode u zemlju Benjaminovu te ondje od rođaka dobije dio. 
\par 13 Ali  kad stiže do Benjaminovih vrata, ondje bijaše zapovjednik straže  Jirijaj, sin Hananijina sina Šelemje. On zaustavi proroka Jeremiju  povikavši: "Ti hoćeš prebjeći Kaldejcima!" Jeremija odgovori: 
\par 14 "Nije istina, ne želim prebjeći Kaldejcima!" Ali i ne slušajući  Jeremiju, Jirijaj ga uhvati i odvede dostojanstvenicima. 
\par 15 Dostojanstvenici  se razljutiše na Jeremiju te ga istukoše i zatvoriše u kuću pisara  Jonatana, koju bijahu pretvorili u tamnicu. 
\par 16 Tako Jeremija  dospje u nadsvođen podrum. Ondje Jeremija ostade mnogo vremena. 
\par 17 Tada kralj Sidkija posla po njega. I nasamo, u dvoru, kralj ga upita: "Ima li riječi od Jahve?" A na to će Jeremija:  "Dakako!" I dometne: "Bit ćeš predan u ruke kralja babilonskoga!" 
\par 18 Onda Jeremija kaza kralju Sidkiji: "Što skrivih tebi, tvojim  slugama i ovom narodu te me baciste u tamnicu? 
\par 19 Gdje su sada  vaši proroci koji vam prorekoše: 'Kralj babilonski neće udariti  na vas ni na ovu zemlju?' 
\par 20 A sada, hajde, čuj mene, gospodaru  moj i kralju, usliši molbu moju! Nemoj da me opet vrgnu u kuću  pisara Jonatana, da ondje ne umrem!" 
\par 21 Tada kralj Sidkija naredi  i Jeremiju odvedoše u tamničko dvorište te mu davahu svaki dan  pogaču kruha iz Pekarske ulice, sve dok nije ponestalo kruha  u gradu. I tako Jeremija ostade u tamničkom dvorištu. 


\chapter{38}

\par 1 Šefatja, sin Matanov, i Gedalija, sin Pašhurov, i Jukal, sin  Šelemjin, i Pašhur, sin Malkijin, čuše tada za riječi što ih  Jeremija kaza svemu narodu: 
\par 2 "Ovako govori Jahve: 'Tko ostane  u ovome gradu, poginut će od mača, gladi i kuge. A tko izađe  pred Kaldejce, spasit će život - život će mu ostati kao plijen, ostat će živ.' 
\par 3 Jer ovako govori Jahve: 'Ovaj će grad odista  pasti u ruke vojsci kralja babilonskoga i ona će ga zauzeti!'" 
\par 4 Tada dostojanstvenici rekoše kralju: "Ovoga čovjeka valja  ubiti: on zaista obeshrabruje ratnike koji su još ostali u gradu  i sav narod kad takve riječi pred njima govori. Pa taj čovjek  ne traži dobrobit ovoga naroda, nego njegovu propast." 
\par 5 A kralj  Sidkija odgovori: "Eto, on je u vašim rukama, jer kralj ionako  više nema nikakve vlasti nad vama." 
\par 6 Tada pograbiše Jeremiju  i baciše ga u čatrnju kraljevića Malkije, što je bila u tamničkom  dvorištu, i oni ga spustiše na užetima. Ali u čatrnji ne bijaše  vode, već samo glib, tako da Jeremija propade u glib. 
\par 7 Ali Kušit Ebed-Melek, dvorjanin koji bijaše u kraljevskom  dvoru, doču da su Jeremiju bacili u čatrnju dok je kralj upravo  sjedio kod Benjaminovih vrata. 
\par 8 Tada Ebed-Melek izađe iz kraljevskog  dvora te ovako reče kralju: 
\par 9 "Gospodaru, kralju moj, zlo čine  ovi ljudi kad tako postupaju s prorokom Jeremijom: bacili su  ga u čatrnju, gdje će od gladi umrijeti, jer nema kruha u gradu." 
\par 10 Nato kralj zapovjedi Kušitu Ebed-Meleku: "Povedi trojicu  ljudi te izvuci proroka Jeremiju iz čatrnje dok nije umro." 
\par 11 I  Ebed-Melek povede ljude, uđe u kraljevski dvor, pod riznicu:  uze ondje nešto iznošenih i poderanih dronjaka te ih na užetu  spusti Jeremiji u čatrnju. 
\par 12 Kušit Ebed-Melek reče Jeremiji:  "Podmetni iznošene i poderane dronjke pod pazuha i pod užad."  Jeremija učini tako. 
\par 13 Tada na užetima izvukoše Jeremiju iz  čatrnje. Otada Jeremija ostade u tamničkom dvorištu. 
\par 14 Kralj Sidkija posla po proroka Jeremiju te ga pozva da  dođe k njemu na treći ulaz što vodi u Dom Jahvin. Kralj reče  Jeremiji: "Htio bih te nešto upitati, nemoj mi ni riječi zatajiti!" 
\par 15 Jeremija odgovori Sidkiji: "Ako ti kažem, nećeš li me pogubiti?  Ako te pak posvjetujem, nećeš me poslušati!" 
\par 16 Tada se kralj  Sidkija u tajnosti zakle Jeremiji ovim riječima: "Živoga mi Jahve, koji nam daje ovaj život, neću te pogubiti i neću te predati  onima što ti rade o glavi." 
\par 17 Jeremija, dakle, reče Sidkiji:  "Ovako govori Jahve, Bog nad Vojskama, Bog Izraelov: 'Ako izađeš  pred vojskovođe kralja babilonskoga, spasit ćeš glavu i ovaj  grad neće biti uništen požarom; živjet ćete ti i tvoj dom. 
\par 18 Ako  pak ne izađeš pred vojskovođe kralja babilonskoga, ovaj će grad  pasti u ruke Kaldejaca i oni će ga spaliti, a ti se nećeš spasiti  iz ruku njihovih.'" 
\par 19 A kralj Sidkija odgovori Jeremiji: "Bojim  se Judejaca koji su prebjegli Kaldejcima: mogli bi mene predati  njima da mi se izruguju." 
\par 20 Jeremija odvrati: "Oni to neće  učiniti. Poslušaj glas Jahvin prema kojem sam ti govorio, bit  će ti dobro i spasit ćeš život svoj. 
\par 21 Ali ako ne htjedneš  iz grada, evo riječi koju mi Jahve objavi: 
\par 22 'Gle, sve žene  koje su još ostale u dvoru kralja judejskoga bit će odvedene  k vojskovođama kralja babilonskoga i govorit će: Zaveli te, svladali te vjerni prijatelji tvoji! Kad ti noge u kal propadaju, oni te napuštaju!' 
\par 23 Da, sve će žene tvoje i djecu tvoju odvesti Kaldejcima, a ni ti sam nećeš umaći rukama njihovim: dospjet ćeš u ruke  kralju babilonskom, a grad će ovaj biti spaljen." 
\par 24 Sidkija reče Jeremiji: "Nitko živ ne smije o tome što  saznati, inače ćeš umrijeti. 
\par 25 Ako, dakle, dostojanstvenici  doznaju da sam s tobom razgovarao te dođu k tebi i kažu: 'TÓa  očituj nam što kralj kaza tebi, a ti njemu; ne krij ništa pred  nama, inače ćemo te ubiti', 
\par 26 odgovori im: 'Molio sam kralja  da me više ne vrati u kuću Jonatanovu, da ondje ne umrem!'" 
\par 27 I doista, dođoše dostojanstvenici k Jeremiji te ga ispitivahu.  Ali im on odgovori upravo onako kako mu kralj bijaše naredio.  Tada ga se okaniše, jer se ništa nije pročulo o onom razgovoru. 
\par 28 Jeremiju, dakle, ostaviše u tamničkom dvorištu sve do dana  kad neprijatelj zauze Jeruzalem. Kad Jeruzalem zauzeše, on bijaše ondje. 


\chapter{39}

\par 1 Devete godine kralja Sidkije, kralja judejskoga, desetoga  mjeseca, Nabukodonozor, kralj babilonski, krenu sa svom vojskom  na Jeruzalem te ga opsjede. 
\par 2 Jedanaeste godine kralja Sidkije, četvrtoga mjeseca, dana devetoga u mjesecu, provališe u grad. 
\par 3 Uđoše sve vojskovođe kralja babilonskoga te se smjestiše  kod Srednjih vrata: Nergal Sar-Eser, knez Sin-Magira, vrhovni  zapovjednik, Nebušasban, visoki dostojanstvenik, i sve druge  vojskovođe kralja babilonskoga. 
\par 4 Kad ih vidješe Sidkija, kralj judejski, i svi ratnici  njegovi, dadoše se u bijeg na vrata između dva zida, noću iziđoše  iz grada prema Kraljevu vrtu i krenuše k dolini Arabi. 
\par 5 Ali  ih čete kaldejske gonjahu i sustigoše Sidkiju u Poljanama jerihonskim.  Uhvatiše ga, odvedoše u Riblu, u zemlju hamatsku, pred Nabukodonozora, kralja babilonskoga, koji mu izreče sud. 
\par 6 I kralj babilonski  dade u Ribli pred očima kralja Sidkije zaklati djecu njegovu.  A dade kralj babilonski pogubiti i sve odličnike judejske. 
\par 7 Sidkiji  iskopa oči, stavi ga u okove da ga odvede u Babilon. 
\par 8 Kaldejci  zapališe kraljev dvor i kuće naroda i porušiše bedeme jeruzalemske. 
\par 9 Ostatak pučanstva koje još ostade u gradu, izbjeglice što  su mu se predale i sav ostali narod, izagna u Babilon Nebuzaradan, zapovjednik tjelesne straže. 
\par 10 A od siromašnoga puka koji  nije ništa posjedovao, Nebuzaradan, zapovjednik tjelesne straže, ostavi neke u zemlji judejskoj i porazdijeli im vinograde i  polja. 
\par 11 O Jeremiji Nabukodonozor, kralj babilonski, zapovjedi  Nebuzaradanu, zapovjedniku tjelesne straže: 
\par 12 "Uzmi ga i oko  tvoje neka bdi nad njim. Ne čini mu nikakva zla, nego postupaj  s njime kako on bude želio." 
\par 13 Tada Nebuzaradan, zapovjednik tjelesne straže, Nebušazdan, visoki dostojanstvenik, Nergal Sar-Eser, vrhovni zapovjednik, i sve vojskovođe kralja babilonskoga 
\par 14 poslaše ljude da izvedu  Jeremiju iz tamničkoga dvorišta i pustiše ga na slobodu. I tako  on osta među narodom. 
\par 15 Dok je Jeremija bio zatvoren u tamničkom dvorištu, dođe  mu riječ Jahvina: 
\par 16 "Idi i reci Kušitu Ebed-Meleku: Ovako govori  Jahve nad Vojskama, Bog Izraelov: 'Evo, učinit ću da se ispune  moje riječi protiv ovoga grada, na nesreću, ne na spas njegov.  I kad se u onaj dan na tvoje oči obistine, 
\par 17 ja ću te u onaj  dan spasiti - riječ je Jahvina - i nećeš biti predan u ruke ljudima  pred kojima dršćeš, 
\par 18 jer ja ću te pouzdano spasiti te nećeš  od mača poginuti, nego ćeš dobiti život kao plijen, jer si se  u me pouzdao' - riječ je Jahvina." 


\chapter{40}

\par 1 Ovo je riječ koju Jahve uputi Jeremiji pošto ga Nebuzaradan, zapovjednik tjelesne straže, bijaše pustio iz Rame. Odvojio  ga je kad je već, u lance okovan, bio među svim jeruzalemskim  i judejskim izgnanicima koje vođahu u Babilon. 
\par 2 Odvojivši ga, dakle, zapovjednik tjelesne straže reče  mu: "Jahve, Bog tvoj, zaprijetio je nesrećom ovome mjestu. 
\par 3 Izvršio  je i učinio kako bijaše zaprijetio, jer ste griješili protiv  Jahve i niste slušali glasa njegova. Zato vas je i snašlo ovo  zlo. 
\par 4 Evo, sada driješim okove s ruku tvojih. Ako ti je po  volji da ideš sa mnom u Babilon, pođi sa mnom i oko će moje bdjeti  nad tobom. Ako ti nije volja ići sa mnom u Babilon, ti ostani.  Gle, sva je zemlja pred tobom: možeš ići kamo ti oko želi i gdje  će ti biti dobro. 
\par 5 Ako, dakle, hoćeš ostati, možeš poći Gedaliji, sinu Šafanova sina Ahikama, koga je kralj babilonski postavio  nad gradovima judejskim, i ostati kod njega usred naroda, ili  pak možeš ići kamo ti drago." Zatim mu zapovjednik tjelesne straže  dade hrane i k tomu dar te ga otpusti. 
\par 6 Tada se Jeremija otputi  u Mispu, Gedaliji, sinu Ahikamovu, te osta kod njega među narodom  koji je ostao u zemlji. 
\par 7 Svi vojni zapovjednici i njihovi ljudi uokolo saznaše  da je kralj babilonski postavio zemlji za namjesnika Ahikamova  sina Gedaliju te mu povjerio muževe, žene i djecu i siromahe  koji još ne bijahu odvedeni u babilonsko sužanjstvo. 
\par 8 I dođoše  pred Gedaliju u Mispu: Netanijin sin Jišmael, Kareahov sin Johanan;  Tanhumetov sin Seraja, Zatim sinovi Efaja Netofljanina, Makatijev  sin Jaazanija - oni i njihovi ljudi. 
\par 9 Gedalija, sin Šafanova  sina Ahikama, zakle se njima i njihovim ljudima i reče: "Ne bojte  se služiti Kaldejcima, ostanite u zemlji, budite odani babilonskom  kralju i bit će vam dobro. 
\par 10 A ja ću, evo, ostati u Mispi na  službu Kaldejcima koji dolaze k nama. Vi pak potrgajte grožđe, poberite voće i masline, pohranite u sudove i ostanite u gradovima  što ih zaposjedoste." 
\par 11 I svi Judejci što se zatekoše u Moabu, kod sinova Amonovih, i u Edomu, po svim zemljama, saznadoše da je kralj babilonski  ostavio ostatak u Judeji i da je postavio nad njim Gedaliju,  sina Šafanova sina Ahikama. 
\par 12 I onda se vratiše svi Judejci  iz svih mjesta kamo ih bijahu raspršili, vratiše se u zemlju  judejsku Gedaliji u Mispu te nabraše veoma mnogo grožđa i drugoga  voća. 
\par 13 A Johanan, sin Kareahov, i svi vojni zapovjednici pođoše  Gedaliji u Mispu 
\par 14 te mu rekoše: "A znaš li ti da je amonski  kralj Baalis poslao Jišmaela, sina Netanijina, da te ubije?"  Ali im Gedalija, sin Ahikamov, ne povjerova. 
\par 15 Tada reče Johanan, sin Kareahov, potajno Gedaliji u Mispi: "Idem da ubijem Jišmaela, sina Netanijina, tako da nitko neće doznati. Zašto da on tebe  ubije i da se opet rasprše svi Judejci što se oko tebe skupiše?  I zašto da propadne ostatak Judejaca?" 
\par 16 Ali Gedalija, sin  Ahikamov, uzvrati Johananu, sinu Kareahovu: "Nemoj toga raditi!  Jer je laž što govoriš o Jišmaelu." 


\chapter{41}

\par 1 Ali u sedmom mjesecu dođe Jišmael, sin Elišamina sina Netanije, roda kraljevskoga, sa deset ljudi i potraži Gedaliju, sina Ahikamova, u Mispi. I dok su se ondje, u Mispi, zajedno gostili, 
\par 2 diže  se Jišmael, sin Netanijin, sa svojom desetoricom i mačem smakoše  Gedaliju, sina Ahikamova. I tako ubi čovjeka koga kralj babilonski  bijaše postavio nad zemljom. 
\par 3 A i sve Judejce koji bijahu s  njim u Mispi, i Kaldejce, vojnike što se tu nađoše - Jišmael  dade pogubiti. 
\par 4 Sutradan, pošto Gedalija bi ubijen, dok još nitko nije  znao što se zbilo, 
\par 5 dođoše ljudi iz Šekema, Šila i Samarije, njih osamdeset, obrijane brade, poderanih haljina i s urezima  po tijelu, noseći u rukama prinose i tamjan da ih prinesu u Domu  Jahvinu. 
\par 6 Jišmael, sin Netanijin, iziđe im iz Mispe u susret, dok su oni, plačući, išli svojim putem. Kad ih stiže, reče im:  "Dođite Gedaliji, sinu Ahikamovu!" 
\par 7 A kad stigoše usred grada, Jišmael, sin Netanijin, i njegovi ljudi poklaše ih i baciše  u čatrnju. 
\par 8 A među njima bijaše deset ljudi koji rekoše Jišmaelu:  "Nemoj nas ubiti, jer imamo u poljima zakopanih zaliha pšenice, ječma, ulja i meda." On tada odusta i ne ubi ih s braćom njihovom. 
\par 9 A čatrnja u koju je Jišmael pobacao sva tjelesa pobijenih  ljudi, velika čatrnja, bijaše ona ista koju je kralj Asa načinio  protiv Baše, kralja izraelskoga. I sad ju je Jišmael, sin Netanijin, napunio pobijenim ljudima. 
\par 10 Tada Jišmael odvede ostatak naroda  iz Mispe, zajedno s kćerima kraljevim koje je Nebuzaradan, zapovjednik  tjelesne straže, povjerio Gedaliji, sinu Ahikamovu: u cik zore  krenu Jišmael, sin Netanijin, i zaputi se da prijeđe u zemlju  Amonovih sinova. 
\par 11 Ali kad Johanan, sin Kareahov, i svi vojni zapovjednici  koji bijahu s njim saznadoše za sva zlodjela što ih Jišmael,  sin Netanijin, bijaše počinio, 
\par 12 uzeše sve svoje vojnike te  krenuše u boj na Jišmaela, sina Netanijina. Nađoše ga uz veliku  vodu u Gibeonu. 
\par 13 Čim oni ljudi što bijahu kod Jišmaela ugledaše  Johanana, sina Kareahova, i sve vojne zapovjednike koji bijahu  s njime, obradovaše se, 
\par 14 i sav narod što ga je Jišmael odveo  iz Mispe okrenu se i potrča Johananu, sinu Kareahovu. 
\par 15 Ali  Jišmael, sin Netanijin, sa osam ljudi, pobježe od Johanana i  ode k sinovima Amonovim. 
\par 16 Tada Johanan, sin Kareahov, i svi  vojni zapovjednici koji bijahu s njim uzeše sav preostali narod  što ga Jišmael, sin Netanijin, pošto ubi Gedaliju, sina Ahikamova, bijaše doveo iz Mispe: muškarce, žene i djecu i dvorjane koje  dovede iz Gibeona. 
\par 17 Krenuše, a kod Svratišta Kimhama, koje  je kraj Betlehema, oni se odmarahu da bi mogli nastaviti put  i stići u Egipat, 
\par 18 što dalje od Kaldejaca, kojih se bojahu:  jer je Jišmael, sin Netanijin, ubio Gedaliju, sina Ahikamova, koga kralj babilonski bijaše postavio za namjesnika u zemlji. 


\chapter{42}

\par 1 Nato svi vojni zapovjednici, osobito Johanan, sin Kareahov, i Azarja, sin Hošajin, i sav narod, malo i veliko, pristupiše 
\par 2 i rekoše proroku Jeremiji: "Pomno počuj molbu našu! Zagovaraj  nas pred Jahvom, Bogom svojim, za sav ovaj ostatak, jer nas,  kako i sam vidiš, ostade još samo malo od velikoga broja koliko  nas je nekoć bilo. 
\par 3 Neka nam Jahve, Bog tvoj, objavi kuda da  krenemo i što valja da činimo." 
\par 4 Prorok im Jeremija odgovori:  "Pristajem. Pomolit ću se, kao što rekoste, Jahvi, Bogu vašemu, i javit ću vam sve što on odgovori, ni riječi vam neću zatajiti." 
\par 5 Oni pak rekoše Jeremiji: "Neka Jahve bude istinit i vjerodostojan  svjedok protiv nas ako ne postupimo sasvim po riječima koje će  nam Jahve, Bog tvoj, po tebi objaviti. 
\par 6 Bio povoljan ili nepovoljan  glas Jahve, Boga našega, komu te šaljemo, mi ćemo ga slušati  da nam dobro bude što poslušasmo glas Jahve, Boga svojega." 
\par 7 Poslije deset dana dođe riječ Jahvina Jeremiji. 
\par 8 Tada  on pozva Johanana, sina Kareahova, sve vojne zapovjednike koji  bijahu s njim i sav narod, malo i veliko, 
\par 9 te im reče: "Ovako govori Jahve, Bog Izraelov, kojemu ste me poslali  da izlijem preda nj molbu vašu: 
\par 10 'Ako budete mirno živjeli  u zemlji ovoj, podići ću vas i neću vas više razoriti; posadit  ću vas, a ne iskorijeniti. Jer se kajem za zlo koje sam vam nanio. 
\par 11 Ne bojte se kralja babilonskoga od koga strahujete. Ne bojte  ga se - riječ je Jahvina - jer ja sam s vama da vas spasim i  izbavim iz ruku njegovih. 
\par 12 I ja ću vam pribaviti milost da  vam se smiluje i pusti vas da u zemlji svojoj živite.' 
\par 13 Ako  pak kažete: 'Nećemo ostati u ovoj zemlji', ne pokoravajući se  glasu Jahve, Boga svoga, 
\par 14 ako kažete: 'Ne, u Egipat idemo, rata više da ne vidimo, glasa bojnog roga više da ne čujemo, da ne moramo biti više gladni kruha; da, onamo idemo', 
\par 15 onda  čujte riječ Jahvinu, vi koji ste Ostatak Judeje: Ovako govori  Jahve nad Vojskama, Bog Izraelov: 'Ako ste odlučili krenuti u  Egipat da ondje živite, 
\par 16 mač kojega se plašite u zemlji će  vas egipatskoj dostići; glad od koje strahujete, u Egiptu će  vam biti za petama: i ondje ćete umrijeti! 
\par 17 I svi oni koji  odluče da odu u Egipat i da se ondje nasele, poginut će od mača, gladi i kuge: nitko živ neće umaći nesreći koju ću na njih svaliti.' 
\par 18 Jer ovako govori Jahve nad Vojskama, Bog Izraelov: 'Kao što  se srdžba moja i bijes moj oboriše na Jeruzalemce, tako će se  gnjev moj izliti i na vas ako pođete u Egipat: postat ćete prokletstvo, užas, kletva i poruga, a ovoga mjesta nikad više nećete ugledati.' 
\par 19 Vama, koji ste Ostatak Judeje, Jahve poručuje da ne idete  u Egipat. Dobro znajte da sam danas bio svjedok protiv vas. 
\par 20 Jer  sami sebe obmanjujete. TÓa vi ste me poslali k Jahvi, Bogu svome  rekavši: 'Zagovaraj nas pred Jahvom, Bogom našim, i saopći nam  sve što ti on objavi, i mi ćemo to učiniti.' 
\par 21 A danas sam  vam objavio, ali vi ne slušate više glasa Jahve, Boga svojega, koji me k vama posla. 
\par 22 Znajte, dakle, dobro: od mača, gladi  i kuge poginut ćete na mjestu kamo hoćete da odete da se ondje  naselite." 


\chapter{43}

\par 1 Kad je Jeremija svemu narodu kazao sve riječi Jahve, Boga  njihova, sve one riječi radi kojih ga je Jahve, Bog njihov, k  njima poslao, 
\par 2 Azarja, sin Hošajin, i Johanan, sin Kareahov, i svi oni drski ljudi odgovoriše Jeremiji: "Laži nam govoriš.  Nije te poslao Jahve da nam govoriš: 'Ne idite u Egipat da se  ondje nastanite', 
\par 3 nego Baruh, sin Nerijin, podgovori te da  nas predaš u ruke Kaldejcima koji će nas pogubiti ili odvesti  u sužanjstvo babilonsko!" 
\par 4 I Johanan, sin Kareahov, i svi zapovjednici i sav narod  ne poslušaše glasa Jahvina da ostanu u zemlji judejskoj. 
\par 5 Nego  Johanan, sin Kareahov, i vojni zapovjednici povedoše sav ostatak  Judin, one što se vratiše iz zemalja kamo bijahu izagnani, da  se nastane u zemlji judejskoj: 
\par 6 muževe, žene i djecu i sve  kraljevske kćeri i sve ljude koje je Nebuzaradan, zapovjednik  tjelesne straže, ostavio kod Gedalije, sina Šafanova sina Ahikama, pa i proroka Jeremiju, i Baruha, sina Nerijina, 
\par 7 te se oni  iseliše u Egipat, jer ne slušahu glasa Jahvina. I tako dođoše  u Tafnis. 
\par 8 U Tafnisu dođe riječ Jahvina Jeremiji: 
\par 9 "Uzmi u ruke  velikoga kamenja i ugradi ga, pred svim Judejcima, meljtom u  pločnik što je pred ulazom u faraonov dvor. 
\par 10 I reci im: Ovako  govori Jahve nad Vojskama, kralj Izraelov: 'Evo šaljem po slugu  svojega Nabukodonozora, kralja babilonskoga. On će postaviti  prijestolje na ovo kamenje što sam ga ugradio i nad njim će razapeti  svoju nebnicu. 
\par 11 Da, doći će i udarit će na zemlju egipatsku: Tko je za smrt, u smrt! Tko za izgnanstvo, u izgnanstvo! Tko za mač, pod mač! 
\par 12 On će vatrom sažeći hramove bogova egipatskih, spalit  će i izagnati bogove, očistit će zemlju egipatsku kao što pastir  svoj plašt otrijebi od buha. I onda će, nesmetan, odavde otići. 
\par 13 Porazbijat će spomenike hrama Sunca koji je u Heliopolu,  a hramove bogova egipatskih ognjem će spaliti.'" 


\chapter{44}

\par 1 Riječ koja se javi Jeremiji za sve Judejce što življahu u  zemlji egipatskoj, što življahu u Migdolu, u Tafnisu, u Memfisu  i u zemlji Patrosu. 
\par 2 Ovako govori Jahve nad Vojskama, Bog Izraelov:  "I sami vidjeste svu nesreću koju sam svalio na Jeruzalem i na  sve gradove judejske: evo danas su to gomile ruševina, a u njima  nitko više ne živi, 
\par 3 zbog nedjela njihovih što ih učiniše da  bi mene vrijeđali, polazeći drugim bogovima kojih nisu poznavali  ni oni, ni vi, ni oci vaši, da im kade i da im služe. 
\par 4 A ja  sam vam jednako slao svoje sluge proroke da vam kažu: 'Ne činite  tih gnusoba koje su mi ogavne!' 
\par 5 Ali me oni nisu slušali, niti  su uho svoje priklonili da se okane zloće svoje i prestanu kaditi  tuđim bogovima. 
\par 6 Zato se izli gnjev moj i srdžba moja i rasplamtje  se u gradovima judejskim i po ulicama jeruzalemskim, te se pretvoriše  u pustoš i razvaline, kao što su danas. 
\par 7 Zašto sami svaljujete na se tako golemu nesreću" - govori  Jahve, Bog nad Vojskama, Bog Izraelov - "te sami do korijena  istrebljujete iz Judeje sve muško i žensko, djecu i dojenčad, te vam ni ostatka ostati neće, 
\par 8 jer me vrijeđate djelima ruku  svojih, kadeći tuđim bogovima u zemlji egipatskoj, u koju ste  došli prebivati, a bit ćete iskorijenjeni i postat ćete kletva  i ruglo među svim narodima na zemlji? 
\par 9 Jeste li zaboravili  bezakonja otaca svojih, bezakonja kraljeva judejskih, bezakonja  knezova svojih i žena njihovih i bezakonja svoja, bezakonja žena  svojih, počinjena u zemlji judejskoj i po ulicama jeruzalemskim? 
\par 10 Do dana današnjega nisu se pokajali, nisu se bojali, nisu  živjeli po Zakonu mojemu ni po odredbama mojim, koje dadoh vama  i ocima vašim." 
\par 11 Zato ovako govori Jahve nad Vojskama, Bog Izraelov: "Evo, okrećem svoje lice vama, na nesreću vašu, da zatrem svu zemlju  judejsku. 
\par 12 Odnijet ću ostatak Judeje koji je odlučio da ide  u zemlju egipatsku da se ondje stani; svi će izginuti u zemlji  egipatskoj, past će od mača, od gladi će pogibati, poginut će  svi, malo i veliko, pomrijet će od mača i gladi, i bit će prokletstvo, užas, kletva i poruga. 
\par 13 Kaznit ću sve koji budu u zemlji  egipatskoj, kao što sam kaznio Jeruzalem: mačem, glađu i kugom. 
\par 14 A od ostatka Judeje koji je došao da se stani u Egiptu, nitko  neće uteći ni preživjeti da bi se mogao vratiti u zemlju judejsku  za kojom im duše čeznu, da se u nju vrate i ondje nastane. Jer  se nitko neće vratiti, osim izbjeglica." 
\par 15 I tada svi muškarci koji su znali da im žene kade tuđim  bogovima, i sve nazočne žene, u velikom mnoštvu, i sav narod  što življaše u zemlji egipatskoj i u Patrosu odgovoriše Jeremiji: 
\par 16 "Riječi koje si u ime Jahvino nama objavio mi ne slušamo; 
\par 17 naprotiv, i dalje ćemo se držati zadane riječi: kadit ćemo  nebeskoj kraljici i lijevati ljevanice, kao što smo i mi i oci  naši, naši kraljevi i knezovi činili u gradovima judejskim i  po ulicama jeruzalemskim: tada imadosmo kruha izobila, bijasmo  sretni i ne trpjesmo nikakvih nesreća. 
\par 18 Ali otkako prestadosmo  kaditi nebeskoj kraljici i lijevati joj ljevanice, u svemu smo  oskudijevali i od mača i gladi pogibali." 
\par 19 A žene rekoše: "Kad kadimo kraljici nebeskoj i lijevamo  joj ljevanice, zar joj bez znanja svojih muževa pečemo kolače  u obliku lika njezina i lijevamo ljevanice?" 
\par 20 Tada Jeremija svemu narodu, muškarcima i ženama, i svim  ljudima koji su mu tako govorili reče: 
\par 21 "Nije li se Jahve  spomenuo i nije li ga u srce dirnuo tamjan što ste ga palili  po ulicama jeruzalemskim, vi i oci vaši, vaši kraljevi, knezovi  i puk zemaljski? 
\par 22 Jahve više nije mogao podnositi zlodjela  vaših i gnusoba koje počiniste, i zato se zemlja vaša pretvorila  u pustoš i ruševine, u prokletstvo, bez stanovnika, kao što je  i danas. 
\par 23 Zbog toga što ste, prinoseći tamjan, zgriješili  Jahvi, što Jahvina glasa ne slušaste i Jahvina se Zakona i njegovih  se naredaba i svjedočanstava ne držaste, snađe vas ova nevolja, kakva je danas." 
\par 24 Zatim reče Jeremija svemu narodu, osobito ženama: "Čujte  riječ Jahvinu, svi Judejci koji ste u zemlji egipatskoj: 
\par 25 Ovako  govori Jahve nad Vojskama, Bog Izraelov: 'Vi žene, ono što vaša  usta obećaju, to vaše ruke moraju i izvršiti. Rekoste: 'Mi ćemo  se tvrdo držati zavjeta što ih učinismo: kaditi kraljici nebeskoj  i lijevati joj ljevanice.' Držite se samo svojih zavjeta i lijevajte  revno ljevanice! 
\par 26 Ali čujte zato riječ Jahvinu, svi vi Judejci  nastanjeni u zemlji egipatskoj! Evo, zaklinjem se velikim imenom  svojim,' govori Jahve. 'U svoj zemlji egipatskoj nijedna usta  judejska neće više izustiti mojega imena; nitko neće reći: 'Živoga  mi Jahve!' 
\par 27 Evo, bdim nad njima, na nesreću, a ne na dobro  njihovo: svi ljudi judejski što su u zemlji egipatskoj poginut  će od mača i gladi do potpunog istrebljenja. 
\par 28 I bit će malo  onih koji će izbjeći maču i vratiti se iz egipatske zemlje u  zemlju judejsku. Onda će sav ostatak judejski, svi koji dođoše  u zemlju egipatsku da ondje žive, spoznati čija riječ vrijedi, moja ili njihova. 
\par 29 A ovo neka vam bude znamenje - riječ je Jahvina - da  ću vas kazniti na ovome mjestu, da biste znali da će se vama  na nesreću ispuniti prijetnje moje protiv vas.' 
\par 30 Ovako govori  Jahve: 'Gle, predat ću faraona Hofru, kralja egipatskoga, u ruke  njegovim neprijateljima i u ruke onih koji mu rade o glavi, baš  kao što sam Sidkiju, kralja Judejskoga, predao u ruke Nabukodonozora, kralja babilonskoga, neprijatelja njegova, koji mu je radio  o glavi.'" 


\chapter{45}

\par 1 Riječ koju uputi Jeremija proroku Baruhu, sinu Nerijinu, dok  je on te riječi iz usta Jeremijinih pisao u knjigu, četvrte godine  Jojakima, sina Jošijina, kralja judejskoga: 
\par 2 Ovako govori Jahve, Bog Izraelov, za tebe, Baruše: 
\par 3 "Jer si rekao: 'Jao meni jer  mi Jahve dodaje nevolju na nevolju. Sustadoh uzdišući i ne mogu  naći mira!' 
\par 4 Ovako govori Jahve: 'Evo, što sam sagradio, porušit  ću, što sam zasadio, iščupat ću - po svoj zemlji! 
\par 5 A ti tražiš  za se čudesa! Ne traži toga! Jer, gle, svalit ću zlo na sve živo  - riječ je Jahvina. A tebi ću kao plijen pokloniti život tvoj  na svim mjestima kamo dođeš.'" 


\chapter{46}

\par 1 Riječ koju Jahve uputi proroku Jeremiji protiv naroda. 
\par 2 Još o Egiptu. Protiv vojske faraona Neka, kralja egipatskoga, što bijaše  kod rijeke Eufrata, u Karkemišu, i kralj Nabukodonozor ga potuče, četvrte godine Jojakima, sina Jošijina, kralja judejskoga. 
\par 3 Pripremite štitove i oklope! Naprijed, u boj! 
\par 4 Upregnite konje! Na kola, vozači! Postavite se pod kacigama! Naperite koplja! Navucite oklope! 
\par 5 Što vidim? Zaprepašteni, uzmiču? Junaci njihovi, poraženi, u bijeg udariše glavom bez obzira! Užas odasvud - riječ je Jahvina. 
\par 6 Ni najbrži ne umače, ni najhrabriji ne uteče! Na sjeveru, na obali Eufrata, posrću i padaju. 
\par 7 Tko se to diže poput Nila, čije vode šume, k'o brzaci nabujaše? 
\par 8 To Egipat se diže poput Nila, k'o brzaci vode mu nabujaše. I govori: dići ću se, poplaviti zemlju, opustošiti gradove i pučanstvo! 
\par 9 Konji, naprijed! Poletite, kola bojna! Navalite, ratnici! Kušiti, Putijci, štitom zaštićeni, i Ludijci, što lukom strijeljate! 
\par 10 Ovo je dan Jahve nad Vojskama - dan osvete da se dušmanima svojim osveti: mač će se nažderati, nasititi, glad utoliti krvlju njihovom! Jer Gospod, Jahve nad Vojskama, ima žrtveno klanje u sjevernoj zemlji uz obalu Eufrata. 
\par 11 Popni se na Gilead, balzama potraži, djevice, kćeri egipatska! Uzalud lijekovi mnogi: nema tebi ozdravljenja! 
\par 12 Narodi čuše za tvoju sramotu, vapaji tvoji napuniše zemlju. Jer se junak o junaka spotiče i obojica padaju. 
\par 13 Riječ koju Jahve uputi proroku Jeremiji kad Nabukodonozor, kralj babilonski, dođe da udari na zemlju egipatsku. 
\par 14 Navijestite Egiptu, objavite u Migdolu, obznanite u Memfisu: "Svrstaj se! Spremi se! Jer mač već ždere sve oko tebe! 
\par 15 Što? Zar Apis pobježe? Tvoj se Bik ne odrva?" Da, Jahve ga obori! 
\par 16 On učini te mnogi posrnuše, popadaše jedan na drugoga. I gle, govore: "Na noge! Vratimo se svom narodu, rodnoj grudi svojoj, pred mačem koljačkim!" 
\par 17 Faraonu, kralju egipatskom, ime nadjenite: "Graja što pravi čas promaši." 
\par 18 "Tako, života mi moga" - govori Kralj, komu je ime Jahve nad Vojskama - "ono će doći kao Tabor posred gora, kao Karmel iznad mora. 
\par 19 Spremi izgnanički zavežljaj, udomljena kćeri egipatska, jer Memfis će biti u pustoš pretvoren, poharan i nenastanjen. 
\par 20 Egipat bijaše lijepa junica, ali ide, ide na nju obad sa Sjevera. 
\par 21 A i plaćenici egipatski što k'o gojna telad usred nje življahu, i oni se okrenuše, u bijeg udariše, ne mogu se odhrvati jer ih stiže Dan propasti, dođe vrijeme da se kazne. 
\par 22 Slušaj! K'o da zmija sikće, sa svom silom dolaze, sjekirama na nju navaljuju, baš k'o drvosječe. 
\par 23 Posjeći će šumu - riječ je Jahvina - iako je neprohodna. Više ih je nego skakavaca, broja njima nema. 
\par 24 Osramoćena je zemlja egipatska, predana je narodu Sjevera." 
\par 25 Govori Jahve nad Vojskama, kralj Izraelov: "Evo, kaznit  ću Amona Tebskoga, faraona i Egipat, i sve njegove bogove, kraljeve, faraona i sve koji se u nj uzdaju. 
\par 26 Predat ću ih u ruke onima  što im rade o glavi, u ruke Nabukodonozora, kralja babilonskoga, i u ruke slugu njegovih. A poslije će Egipat biti opet naseljen, kao u stara vremena" - riječ je Jahvina. 
\par 27 "Ne boj se, Jakove, slugo moja, ne plaši se, Izraele! Jer, evo, spasit ću te izdaleka i potomstvo tvoje iz zemlje izgnanstva. Jakov će se opet smiriti, spokojno će živjet' i nitko ga neće plašiti. 
\par 28 Ne boj se, Jakove, slugo moja - riječ je Jahvina - jer ja sam s tobom. Zatrt ću narode među koje te prognah, a tebe neću sasvim uništiti: ali ću te kaznit' po pravici, ne smijem te pustit' nekažnjena." 


\chapter{47}

\par 1 Riječ koju Jahve uputi proroku Jeremiji o Filistejcima prije  nego što faraon osvoji Gazu. 
\par 2 Ovako reče Jahve: "Evo, vode se dižu sa Sjevera i kao nabujali brzaci poplavljuju zemlju i sve što je na njoj, gradove i sve njihovo pučanstvo. I ljudi vapiju, i kukaju svi žitelji zemlje, 
\par 3 uz tutanj kopita njegove ždrebadi, uza štropot kola i tresku točkova. Oci više ne mare za djecu svoju jer su im ruke klonule 
\par 4 zbog dana što osvanu da Filistejce istrijebi, da zatre Tiru i Sidonu sve do posljednjeg pomagača. Jer Jahve istrebljuje Filistejce, i sav ostatak otočja kaftorskog. 
\par 5 Gazi će biti obrijana glava, razoren Aškelon. A ti, Ašdode, ostače Anakovaca, dokle će te tuga razdirati? 
\par 6 Jao, maču Jahvin, kad li ćeš se smiriti? Vrati se u korice, stani i počini!" 
\par 7 Ali kako da se smiri, kad Jahvina ruka njime zapovijeda: na Aškelon i na morski žal on ga isuka. 


\chapter{48}

\par 1 O Moabu. Ovako govori Jahve nad Vojskama, Bog Izraelov: "Jao brdu Nebu jer je opustošeno, postiđen je Kirjatajim i osvojen, tvrđa posramljena, razorena, 
\par 2 nema više dike moapske. U Hešbonu mu propast skovaše: 'Hajde da ga istrijebimo iz naroda!' A ti, Madmene, bit ćeš razoren, mač već ide za tobom! 
\par 3 Slušaj! Jauci se čuju iz Horonajima: 'Pohara, propast strašna!' 
\par 4 'Moab je smlavljen!' čuje se vrištanje mališa njegovih. 
\par 5 Da, uz brdo Luhit uspinju se plačući. Da, niz obronke Horonajima razliježe se jauk nad propašću. 
\par 6 'Bježite, spasavajte život, ugledajte se u pustinjsku magarad!' 
\par 7 Jer si se pouzdao u svoje utvrde, bit ćeš i ti osvojen. Kemoš odlazi u izgnanstvo sa svećenicima i knezovima svojim. 
\par 8 Pustošnik će doći u svaki grad, nijedan mu neće izmaći: Dolina će biti poharana, Visoravan opustošena," govori Jahve! 
\par 9 Stavite Moabu nadgrobni kamen, jer je do temelja srušen; njegovi su gradovi pustare, u njima nitko ne obitava. 
\par 10 Proklet bio tko nemarno obavlja poslove Jahvine! Proklet  bio tko krvlju mač svoj ne omasti! 
\par 11 Od mladosti svoje mir uživaše Moab, ležaše na droždini svojoj, nikad ga nisu pretakali iz bačve u bačvu, nikad u izgnanstvo išao nije: zato mu okus ostade svjež, miris nepromijenjen. 
\par 12 "Ali, evo, dolaze dani" - govori Jahve - "i ja ću mu  poslati tlačitelje koji će ga pretakati, isprazniti njegove bačve  i sudove njegove porazbijati. 
\par 13 I tada će se Moab postidjeti  zbog Kemoša, kao što se dom Izraelov postidio zbog Betela u koji  se uzdao." 
\par 14 Kako možete reći: "Mi smo junaci, hrabri ratnici." 
\par 15 Pustošnik Moabov navaljuje na nj; cvijet mladosti njegove u klanice silazi, riječ je Kraljeva, Jahve nad Vojskama njemu je ime. 
\par 16 Bliži se propast Moabova, nesreća njegova hiti. 
\par 17 Žalite ga, svi susjedi njegovi, i svi koji znate ime njegovo. Recite: "Kako li se slomi čvrsta palica, žezlo veličanstveno!" 
\par 18 Siđi sa slave svoje, sjedni u blato, žitelju, kćeri dibonska! Jer pustošnik Moaba navali na te, poruši sve utvrde tvoje. 
\par 19 Stani na cestu i promatraj, o žitelju Aroera! Pitaj bjegunce i preživjele, pitaj ih: "Što se to dogodi?" 
\par 20 "Moab se stidi jer je slomljen. Plačite, jecajte! Objavite na Arnonu da je Moab poharan." 
\par 21 Sud stiže nad Visoravan i nad Holon, Jahsu i Mefaot, 
\par 22 nad Dibon, Nebo, Bet Diblatajim, 
\par 23 Kirjatajim, Bet Gamul, Bet Meon, 
\par 24 Kerijot, Bosru i nad sve gradove zemlje moapske, daleke i blize. 
\par 25 "Moabu je rog odbijen, ruka mu je slomljena." 
\par 26 "Opijte ga jer se htjede uzvisiti nad Jahvu: neka se  Moab sada valja u bljuvotini svojoj te i on neka bude na podsmijeh. 
\par 27 Nije li tebi bio Izrael na podsmijeh? Jesu li ga možda zatekli  u krađi te mašeš glavom kad god o njemu govoriš?" 
\par 28 "Ostavite gradove, živite u pećinama, stanovnici Moaba! Budite kao golubovi što se gnijezde na litici onkraj razjapljena bezdana!" 
\par 29 Čuli smo za nadutost Moaba, nadutost preveliku, ponos njegov, hvastanje, uznositost, za oholost srca njegova! 
\par 30 "Poznajem ja obijest njegovu - riječ je Jahvina - laž njegovih riječi, laž djela njegovih! 
\par 31 Zato moram jaukati nad Moabom, plakati nad svim Moapcima, jecati zbog ljudi Kir Heresa. 
\par 32 Više nego nad Jazerom, plakat ću nad tobom, o lozje sibmansko, kojem se mladice pružahu preko mora, sezahu sve do Jazera. Na tvoje berbe i žetve pade sada pustošnik. 
\par 33 Iščeznu radost i veselje iz voćnjaka i zemlje moapske. Nesta vina u kacama, mastioci više grožđa ne maste, veseli zvuci više nisu veseli." 
\par 34 Urlanje Hešbona i Elalea čuje se sve do Jahasa. Viču  od Soara do Horonajima i Eglat Šelišije, jer se i vode nimrimske  pretvoriše u pustaru. 
\par 35 "U Moabu ću učiniti - riječ je Jahvina - da se ne uzlazi  na uzvišice i kadi bogovima njegovim. 
\par 36 Stoga mi srce poput frule dršće za Moabom, srce moje  poput frule dršće za ljudima Kir Heresa: jer propade stečevina  koju stekoše! 
\par 37 Sve su glave obrijane i brade podrezane; po  svim rukama urezi, oko bokova kostrijet. 
\par 38 Na svim krovovima  Moaba i na njegovim trgovima samo zapomaganje, jer smrskah Moab  kao krčag koji se nikomu ne mili" - riječ je Jahvina. 
\par 39 Kako  li je smrskan! Kako li sramotno Moab udari u bijeg! Moab postade  ruglo i strašilo svim susjedima. 
\par 40 Jer ovako govori Jahve: "Gle, poput orla lebdi, nad Moabom širi krila. 
\par 41 Gradovi su zauzeti, osvojene tvrđave. Srce moapskih junaka bit će toga dana kao srce žene u trudovima. 
\par 42 Izbrisan je Moab iz naroda jer se uzvisi nad Jahvu. 
\par 43 Strava, jama i zamka tebi, žitelju Moaba! - riječ je Jahvina. 
\par 44 Tko stravi umakne, u jamu će pasti; tko se iz jame izvuče, u zemlju će pasti. Da, to ću svaliti na Moab u danima kazne njegove" - riječ je Jahvina. 
\par 45 "U sjeni se hešbonskoj ustavljaju iscrpljeni bjegunci. Al' vatra izlazi iz Hešbona, plamen liže iz dvora sihonskog i proždire sljepoočnice Moabu i tjeme sinova nemirničkih. 
\par 46 Jao tebi, Moabe! Umišljen si, narode Kemošev! Jer sinove tvoje u izgnanstvo odvedoše, kćeri tvoje u progonstvo. 
\par 47 Ali ću promijeniti udes Moabov u budućnosti" - riječ je Jahvina. Dovde suđenje Moabu. 


\chapter{49}

\par 1 O sinovima Amonovim. Ovako govori Jahve: "Izrael nema sinova, nema nasljednika? Zašto je Milkom baštinio Gad i narod se njegov nastanio u njegovim gradovima? 
\par 2 Zato, evo, dolaze dani - riječ je Jahvina - i učinit ću da se zaore ratni krikovi u Rabi sinova Amonovih, i ona će biti humak poharani, i naseobine njene ognjem popaljene. Tada će Izrael opljačkati svoje pljačkaše" - govori Jahve. 
\par 3 "Plači, Hešbone, jer Ar je opustošen, zapomažite kćeri rapske. Opašite kostrijet, tužbalice povedite, obilazite s urezima. Jer Milkom mora u izgnanstvo sa svećenicima i knezovima. 
\par 4 Što se dičiš dolinom svojom, kćeri odmetnice, koja se uzdaš u bogatstvo svoje i govoriš: 'Tko se usuđuje ustati protiv mene?' 
\par 5 Evo, svaljujem na te stravu odasvud uokolo: bit ćete raspršeni, svak' na svoju stranu, i nitko bjegunce neće skupiti. 
\par 6 Ali uto ću opet promijeniti udes sinova Amonovih" - riječ  je Jahvina. 
\par 7 O Edomu. Ovako govori Jahve nad Vojskama: "Zar nema više mudrosti u Temanu, zar u razumnih nesta svj§eta, zar se izvjetrila mudrost njihova? 
\par 8 Bježite, gubite se i duboko se sakrijte, stanovnici Dedana, jer Ezavu propast nosim, vrijeme kazne njegove. 
\par 9 Dođu li trgači k tebi, ni pabirka neće ostaviti; dođu li kradljivci noćni, opljačkat će sve što žele. 
\par 10 Jer ja sam onaj što će Ezava pretražiti i skrovišta mu otkriti da se ne mogne sakriti. Pleme je njegovo opustošeno: nema ga više! Nitko ne kaže: 
\par 11 'Ostavi siročad svoju, ja ću je prehraniti i neka se udovice tvoje u me pouzdaju!'" 
\par 12 Jer ovako govori Jahve: "Gle, oni koji odista ne bi morali  piti čašu moraju je iskapiti, i zar upravo ti da ostaneš nekažnjen?  Ne, ti nećeš ostati nekažnjen, morat ćeš čašu ispiti! 
\par 13 Jer  samim se sobom zakleh - riječ je Jahvina: Bosra će postati ruglo  i sramota, pustinja i prokletstvo; a svi njezini gradovi bit  će vječne razvaline." 
\par 14 Jahve mi vijest uputi, glasnik bi poslan k narodima: "Skupite se! Krenite na nj, krenite! Ustajte! U boj! 
\par 15 Jer, gle, učinit ću te malim među narodima, prezrenim među ljudima. 
\par 16 Strah te tvoj zaveo, uznositost srca tvoga, ti koji živiš u pećinama kamenim i držiš se visova planinskih te viješ gnijezdo na timoru, k'o orlovi, odande ću te strovaliti" - riječ je Jahvina. 
\par 17 "Edom će postati pustoš; tko god njime prođe, zaprepastit  će se i zviždati zbog svih rana njegovih. 
\par 18 Razorit će ga kao  Sodomu i Gomoru i susjede njihove" - govori Jahve. Čovjek ondje  neće stanovati, sin čovječji neće u njem boraviti. 
\par 19 "Gle, kao lav on izlazi iz guštare jordanske na pašnjake vječno zelene. Ali ću ga učas otjerati i smjestiti ondje svog izabranika. Jer tko je meni ravan? I tko će mene na račun pozvati? I koji će mi pastir odoljeti?" 
\par 20 Zato čujte što je Jahve naumio učiniti Edomu, čujte što je nakanio protiv stanovnika Temana: i najsitniju jagnjad on će odvući, i sam njihov pašnjak zgrozit će se nad njima. 
\par 21 Od lomljave pada njina zemlja će se potresti, razlijegat će se vapaj do Crvenog mora! 
\par 22 Gle, poput orla on se diže i lebdi, nad Bosrom širi krila. U dan onaj srce će junaka edomskih biti kao srce žene u trudovima. 
\par 23 O Damasku. Smeteni su Hamat i Arpad jer zlu vijest čuše. Srce im se steže od užasa i smirit se ne može. 
\par 24 Obeshrabren je Damask, u bijeg udario, strah ga spopao, tjeskoba i bolovi obuzeli ga kao porodilju. 
\par 25 Kako? Napušten je slavni grad, grad radosti moje! 
\par 26 Zato će mladići njegovi popadati po trgovima, svi će ratnici poginuti u onaj dan - riječ je Jahve nad Vojskama. 
\par 27 "Potpalit ću vatrom zidine Damaska: plamen će proždrijeti dvor Ben-Hadadov." 
\par 28 O Kedaru i kraljevstvima hasorskim koje je potukao Nabukodonozor, kralj babilonski. Ovako govori Jahve: "Ustajte, na Kedar navalite, uništite sinove Istoka! 
\par 29 Nek' im se oduzmu šatori i stada, šatorska krila i sva im oprema! Neka im se deve odvedu, i nek' viču na njih: 'Strava odasvud!' 
\par 30 Bježite glavom bez obzira, duboko se skrijte, žitelji Hasora - riječ je Jahvina. Jer Nabukodonozor, kralj babilonski, snuje naum protiv vas, navalu smišlja: 
\par 31 'Ustajte, udarite na mirni narod što živi bez straha - riječ je Jahvina - što nema vrata ni zasuna, što u osami prebiva! 
\par 32 Deve njihove bit će plijen, mnoštvo ovaca otimačina!' I raspršit ću ih na sve strane, one ljude obrijanih zalizaka, i dovest ću odasvud na njih nesreću - riječ je Jahvina. 
\par 33 Hasor će postati brlog čagaljski i pustinja vječna. Čovjek ondje neće prebivati, neće se ondje nastaniti sin čovječji." 
\par 34 Riječ koju Jahve uputi proroku Jeremiji o Elamu, u početku  kraljevanja Sidkije, kralja judejskoga. 
\par 35 Ovako govori Jahve nad Vojskama: "Lomim, evo, luk Elamov, srž snage njegove. 
\par 36 Četiri ću vjetra dognati na Elam sa četiri kraja neba i raspršit Elamce u sva četiri vjetra, i neće biti naroda kamo neće stići bjegunci elamski. 
\par 37 Utjerat ću Elamcima strah u kosti pred njihovim dušmanima. Pustit ću na njih nesreću, oganj gnjeva svojega. Poslat ću mač za njima dok ne budu sasvim uništeni. 
\par 38 I postavit ću u Elamu prijesto svoj i zatrt ću ondje kralja i sve knezove" - riječ je Jahvina. 
\par 39 Ali ću okrenut' udes Elama" - riječ je Jahvina. 


\chapter{50}

\par 1 Riječ koju Jahve reče protiv Babilona, protiv zemlje kaldejske: 
\par 2 "Objavite narodima! Razglasite, ne tajite, recite: Zauzet je Babilon! Bel je postiđen: Marduk razbijen! Posramljeni su kipovi njegovi, razmrskani njegovi likovi. 
\par 3 Jer sa sjevera na nj se diže narod koji će mu zemlju prometnuti u pustinju; nitko više neće u njoj živjeti, i ljudi i stoka pobjeći će i otići. 
\par 4 U one dane i u vrijeme ono - riječ je Jahvina - vratit će se sinovi Izraelovi, ići će plačuć' i tražeći Jahvu, Boga svojega. 
\par 5 Pitat će za put na Sion, onamo će pogled upravljati: 'Hodite, prionimo uz Jahvu Savezom vječnim, nezaboravnim!' 
\par 6 K'o izgubljene ovce bijaše narod moj, pastiri ih zavedoše te zalutaše po brdima: moradoše s brda na brežuljke, zaboraviše gdje su im torovi. 
\par 7 Tko ih nađe, proždire ih, neprijatelji njini zborahu: 'Nismo mi krivi, jer zgriješiše Jahvi, pašnjaku pravde, Jahvi, nadi otaca svojih!' 
\par 8 Bježite iz Babilona! Izađite iz zemlje kaldejske! Budite poput ovnova pred stadom. 
\par 9 Jer evo ću dići i dovesti na Babilon mnoštvo velikih naroda; u zemlji sjevernoj svrstat će se protiv njega - odanle će biti osvojen. Strijele su im k'o u sretna junaka, prazne se ne vraćaju. 
\par 10 Kaldeja će biti oplijenjena, do mile volje nju će pljačkati" - riječ je Jahvina. 
\par 11 "Da! Radujte se samo i klikujte, vi pljačkaši moje baštine! Poskakujte k'o june na paši! Ržite kao ždrebad! 
\par 12 Mati vaša teško se osramoti, postidi se roditeljka vaša. Evo, posljednja je među narodima: pustinja, zemlja prljuša. 
\par 13 Zbog gnjeva Jahvina bit će bez življa, sva će opustjeti. Tko god prođe mimo Babilon, zgrozit će se i zviždat će nad njegovim ranama. 
\par 14 Svrstajte se protiv Babilona, opkolite ga. Strijelci, strijeljajte na nj, ne žalite strelica - Jahvi je zgriješio. 
\par 15 Sa svih strana nek' zaore poklici bojni. On se predaje! Stupovi mu padaju, bedemi se ruše: to Jahvina je osveta! Osvetite se Babilonu, vratite mu milo za drago! 
\par 16 Istrijebite Babilonu i sijača i žeteoca što srpom zamahuje u dane žetvene! Pred mačem silničkim nek' svak' se vrati svome narodu, neka bježi zemlji svojoj." 
\par 17 Izrael bijaše stado razagnano, lavovi ga raspršiše. Prvi ga kralj asirski proždrije, a onda mu Nabukodonozor, kralj babilonski, kosti polomi. 
\par 18 Zato ovako govori Jahve nad Vojskama, Bog Izraelov: "Evo, kaznit ću kralja babilonskoga i zemlju njegovu, kao što kaznih kralja asirskoga. 
\par 19 I vratit ću Izraela pašnjaku njegovu da pase po Karmelu, Bašanu, brdima efrajimskim i u Gileadu, da se sit najede. 
\par 20 U one dane, u vrijeme ono - riječ je Jahvina - tražit će grijeh Izraelov, ali ga više neće biti; tražit će opačine judejske, ali ih neće naći. Jer oprostih svima koje sačuvah." 
\par 21 "Na zemlju meratajimsku! Kreni na nju i na stanovnike Pekoda, zatri ih do temelja, uništi iza njih sve - riječ je Jahvina - izvrši sve kako ti zapovjedih!" 
\par 22 Ratna se vika čuje u zemlji, poraz strašan. 
\par 23 O, kako li je skršen, razbijen malj cijele zemlje! Kako li Babilon posta strašilo narodima! 
\par 24 Zamku ti metnuh, Babilone, ti se uhvati i ne vidje. Zatečen si i uhvaćen, jer se s Jahvom zarati! 
\par 25 Jahve otvori svoju oružnicu, izvuče oružje gnjeva svojega, jer ima posla za Jahvu nad Vojskama u zemlji kaldejskoj. 
\par 26 Udarite na nju sa svih strana, otvorite žitnice njene, slažite je kao snoplje, zatrite Babilon kletim uništenjem, da od njega ne ostane ništa. 
\par 27 Pokoljite svu junad njegovu, u klaonicu neka siđu! Jao njima, došao je njihov dan, vrijeme kazne njihove! 
\par 28 Čujder glasa onih što pobjegoše, što utekoše iz zemlje babilonske da jave na Sionu osvetu Jahve, Boga našega, osvetu Hrama njegova! 
\par 29 Sazovite na Babilon strijelce, sve što zapinju lukove, opkolite ga sa svih strana: nitko da ne uteče! Platite mu po zasluzi, vratite mu milo za drago, jer bi se oholio na Jahvu, Sveca Izraelova. 
\par 30 Zato će mu svi mladići popadati po trgovima i svi će mu ratnici  u onaj dan izginuti - riječ je Jahvina. 
\par 31 "Evo me na te, Oholice!" - riječ je Gospoda Jahve nad Vojskama - "došao je dan tvoj, vrijeme pohoda na te. 
\par 32 Oholica će posrnuti, pasti, i nitko ga neće podići. Oganj ću podmetnuti gradovima njegovim i proždrijet će sve uokolo." 
\par 33 Ovako govori Jahve nad Vojskama: "Potlačeni su sinovi Izraelovi, zajedno sa sinovima Judinim. Oni što ih zarobiše, drže ih čvrsto i neće da ih puste. 
\par 34 Ali, moćan je njihov Otkupitelj, ime mu Jahve nad Vojskama. On će obraniti parnicu njihovu - zemlji mir donijeti i smesti pučanstvo Babilona." 
\par 35 Mač na Kaldejce - riječ je Jahvina - na pučanstvo Babilona, na knezove i mudrace njegove! 
\par 36 Mač na brbljavce njegove, neka polude! Mač na njegove ratnike, neka se prestrave! 
\par 37 Mač na njegove konje i bojna kola i na svu gomilu posred njega: nek postanu kao žene! Mač na njihove riznice; nek ih opljačkaju! 
\par 38 Mač na vode njihove, neka presahnu! Jer to je zemlja idola, zaludiše ih kipovi, strašila njihova. 
\par 39 Zato će se ondje nastaniti risovi s čagljima, i nojevi će ondje obitavat. Dovijeka će ostat' mjesto bez življa, nitko ondje neće živjeti od koljena do koljena. 
\par 40 Razorit će ga kao što Bog razori Sodomu i Gomoru i susjede njihove - riječ je Jahvina. Čovjek ondje neće stanovati, sin čovječji neće u njem' boraviti. 
\par 41 Evo dolazi narod sa Sjevera, puk velik i mnogi kraljevi, i dižu se s krajeva zemlje. 
\par 42 U ruci im luk i koplje, okrutni su, nemilosrdni. Graja im buči poput mora, jašu na konjima, kao jedan za boj spremni protiv tebe, kćeri babilonska. 
\par 43 Kralj babilonski ču vijest o njima: i ruke mu klonuše, muka ga spopade, bolovi ga obuzeše kao porodilju. 
\par 44 "Gle, kao lav on izlazi iz guštare jordanske na pašnjake vječno zelene. Al' ja ću ga učas otjerati i smjestiti ondje svog izabranika. Jer, tko je meni ravan? I tko će mene na račun pozvati? I koji će mi pastir odoljeti?" 
\par 45 Zato, čujte što je Jahve naumio da učini Babilonu, čujte što je nakanio protiv zemlje kaldejske: i najsitniju jagnjad on će odvući; i sam njihov pašnjak zgrozit će se nad njima. 
\par 46 Na glas da je pao Babilon zemlja će se potresti: razlijegat će se vapaj među narodima. 


\chapter{51}

\par 1 Ovako govori Jahve: "Gle, ja podižem protiv Babilona i protiv pučanstva kaldejskog vjetar zatornički. 
\par 2 Poslat ću na Babilon vijače da ga viju i prorešetaju zemlju njegovu. Kad ga sa svih strana opkole u kobni dan, 
\par 3 nek' strijelac luka ne odlaže, nek' ne skida oklopa! Ne štedite mladića njihovih, svu mu vojsku kleto uništite! 
\par 4 Pobijeni će padati po zemlji kaldejskoj, probodeni po ulicama njegovim." 
\par 5 Ne, Izrael - Judeja - nije udovica Boga svojega, Jahve nad Vojskama, iako je zemlja njihova puna krivice protiv Sveca Izraelova. 
\par 6 Bježite iz Babilona, nek' svak' spasi život svoj, da ne izginete s njegova bezakonja, jer ovo je vrijeme Jahvine odmazde, svakom plaća po zasluzi! 
\par 7 Babilon bijaše pehar zlatni u ruci Jahvinoj, pehar koji opi svijet cijeli. Vinom tim se puci opiše, i zato se puci obezumiše. 
\par 8 Iznenada pade Babilon, razmrskan: zakukajte nad njim! Potražite balzama rani njegovoj: možda će ozdravit'! 
\par 9 Liječili smo Babilon, al' se ne izliječi. Pustimo ga, vratimo se svaki u svoj kraj! Jer do neba dopire njegova osuda i diže se pod oblake. 
\par 10 Jahve je iznio pravdu našu! Hajde da Sionu objavimo djelo Jahve, Boga našega. 
\par 11 Naoštrite strelice, napunite tobolce! Jahve potače duh kraljeva medijskih jer naumi zatrti Babilon, Jahvina je to osveta, osveta za Hram njegov. 
\par 12 Razvijte stijeg, zidine babilonske! Pojačajte straže! Postavite stražare! Razmjestite zasjede! Jer Jahve što naumi to sad izvodi, kako je rekao protiv Babilona. 
\par 13 O, ti što prebivaš na velikim vodama i bogat blagom svakojakim! Sad ti svršetak dođe, kraj tvojoj lakomosti. 
\par 14 Životom svojim zakle se Jahve nad Vojskama: "Napunit ću te ljudstvom kao skakavcima, zaorit će protiv tebe vika bojna." 
\par 15 On snagom svojom stvori zemlju, mudrošću svojom uspostavi krug zemaljski i umom svojim razape nebesa. 
\par 16 Kad mu glas zaori, huče vode na nebesima, oblake diže s kraja zemlje: stvara kiši munje, vjetar izvodi iz skrovišta njegovih. 
\par 17 Svakom čovjeku pamet stane, svaki se zlatar zastidi svoga kipa, jer svi su mu kipovi samo varka, nema u njima duha. 
\par 18 Isprazni su oni, smiješne tvorevine, propast će u dan kazne. 
\par 19 'Jakovljev dio' nije kao oni: jer on je sve stvorio, Izrael pleme je baštine njegove. Jahve nad Vojskama ime je njegovo. 
\par 20 Ti si mi bio malj, oružje ratno. Pomlatih tobom narode, razmrskah tobom kraljevstva. 
\par 21 Pomlatih tobom konja i konjanika, pomlatih tobom bojna kola i vozača. 
\par 22 Pomlatih tobom čovjeka i ženu, pomlatih tobom starca i dijete. Pomlatih tobom mladića i djevojku, pomlatih tobom pastira i stado. 
\par 23 Pomlatih tobom ratara i zapregu njegovu, pomlatih tobom namjesnike i upravljače. 
\par 24 Ali na vaše oči sada plaća Babilonu i svim Kaldejcima za  sve zlo koje učiniše Sionu - riječ je Jahvina. 
\par 25 "Evo me na te, Goro zatornice - riječ je Jahvina - zatornice svega svijeta! Zamahnut ću rukom protiv tebe, svalit ću te s litice, pretvorit ću te u goru spaljenu. 
\par 26 Iz tebe više neće klesati kamen ugaoni ni kamen temeljac, bit ćeš vječna pustinja" - riječ je Jahvina. 
\par 27 Podignite stijeg u zemlji, zatrubite u rog među narodima! Pripremite na nj narode, sazovite na nj kraljevstva - Ararat, Mini, Aškenaz! Postavite protiv njega pozivnike, nek' nasrnu konji k'o dlakavi skakavci! 
\par 28 Spremite na nj narode, kraljeve medijske, vojvode i namjesnike njihove i svu zemlju kojom vladaju. 
\par 29 Zemlja će se tresti, drhtati, kad se stanu uspinjati k Babilonu svi naumi Jahvini da pretvori zemlju babilonsku u pustinju nenastanjenu. 
\par 30 Nebojše babilonske odustaše od borbe, u utvrdama posjedaše, nesta sile njihove: postadoše kao žene. Spaljeni su domovi njihovi, polomljeni zasuni na vratima. 
\par 31 Teklič tekliča prestiže, glasnik juri za glasnikom, da jave kralju babilonskom da mu je grad sa svih strana zauzet, 
\par 32 prijelazi zaposjednuti, tvrđave ognjem popaljene, a ratnici prestrašeni. 
\par 33 Jer ovako govori Jahve, Bog Izraelov: "Kći je babilonska kao gumno u vrijeme kad se po njem gazi; još malo, i doći će joj vrijeme žetve." 
\par 34 Izjeo me, satro babilonski kralj, odgurnuo me kao prazan pladanj, k'o zmaj me on progutao, napunio trbušinu, iz mog me istjerao Edena. 
\par 35 "Nasilje i patnje moje na Babilon!" govore stanovnici Siona. "Krv moja na Kaldejce!" govori Jeruzalem. 
\par 36 Zato ovako govori Jahve: "Gle, ja ću braniti parnicu tvoju i krvavo te osvetiti. Sasušit ću more njegovo i presahnut ću izvore njegove. 
\par 37 Babilon će biti hrpa ruševina, brlog čagljima, užas i ruglo, kraj nenastanjen. 
\par 38 Svi zajedno riču k'o lavovi, zavijaju kao lavići. 
\par 39 Kad se ugriju, priredit će im pijanku, napojiti ih da se provesele, da zaspe vječnim snom, da se više ne probude - riječ je Jahvina. 
\par 40 Odvest ću ih k'o janjce na klanje, kao jarce i ovnove." 
\par 41 "Kako li je zauzet, kako osvojen taj ponos zemlje sve? Kako li Babilon posta strašilo narodima? 
\par 42 Uzdiglo se more protiv Babilona, prekrilo ga valovlje njegovo. 
\par 43 Gradovi mu pustoš postali, zemlja suha, pustara: čovjek u njoj ne stanuje, niti njom prolazi sin čovječji." 
\par 44 "Kaznit ću Bela babilonskog, iz ralja mu otet što je progutao. Neće više k njemu hrliti narodi, srušit će se babilonske zidine. 
\par 45 Izađi iz njega, narode moj! Nek' svaki spasi život svoj od jarosnoga gnjeva Jahvina! 
\par 46 Neka vam srce ne klone! Ne bojte se glasÄa što se zemljom šire, jedne godine ovakvi, druge onakvi, i što u zemlji vlada nasilje, te silnik za silnikom ustaje. 
\par 47 Jer evo dolaze dani kada ću kazniti kipove babilonske, i sva će mu se zemlja postidjeti, i svi će mu pobijeni ležat' posred grada. 
\par 48 Tada će nad Babilonom klicati nebo i zemlja i sve što je na njima, jer će sa sjevera navaliti na grad, zatornici njegovi - riječ je Jahvina! 
\par 49 I Babilon mora pasti za pobijene Izraelce, kao što su za Babilon padali pobijeni po svem svijetu. 
\par 50 Vi što umakoste maču, idite, ne ostajte ovdje. Spominjite se Jahve u zemlji dalekoj, i neka vam Jeruzalem bude na srcu: 
\par 51 'Stidjeli smo se slušajuć' sramotu, rumenilo nam prekrilo lice kad ono tuđinci nahrupiše u Svetište Doma Jahvina.' 
\par 52 Zato, evo, dolaze dani - riječ je Jahvina - kad ću kazniti kipove njegove i po svoj će mu zemlji stenjat' ranjenici. 
\par 53 Da se Babilon popne do neba, da se utvrdi na visu nedostupnu, na moju će zapovijed na nj navalit' pustošnici" - riječ je Jahvina. 
\par 54 Čujte vapaj iz Babilona, i strašan poraz iz zemlje kaldejske! 
\par 55 Jer sam Jahve pustoši Babilon, on stišava strašnu buku njegovu: bučili su vali njegovi k'o vode velike, razlijegala se huka njihova. 
\par 56 Jest, pustošnik dođe na Babilon, uhvaćeni su ratnici njegovi, lukovi im polomljeni. Zaista, Jahve je Bog osvetnik koji plaća po zasluzi! 
\par 57 "Opojit ću mu knezove i mudrace, vojvode, namjesnike i ratnike: da zaspe vječnim snom pa se više ne probude" - govori Kralj, ime mu je Jahve nad Vojskama. 
\par 58 Ovako govori Jahve nad Vojskama: "Široke zidine babilonske bit će do temelja razvaljene, a visoka trava njegova bit će ognjem spaljena. Narodi se zalud trudili, puci se za oganj mučili!" 
\par 59 Evo zapovijedi što je prorok Jeremija dade Seraji, sinu  Mahsejina sina Nerije, kad je Seraja krenuo u Babilon sa Sidkijom, kraljem judejskim, četvrte godine njegova vladanja. Seraja bijaše  veliki komornik. 
\par 60 Jeremija je u jednu knjigu bio zapisao svu  nesreću koja je morala doći na Babilon - sva ona proroštva napisana  protiv Babilona. 
\par 61 Jeremija reče Seraji: "Kad dođeš u Babilon, išti prigodu da obznaniš sve ove riječi. 
\par 62 I reci: 'Jahve, ti sam reče da ćeš zatrti ovo mjesto te u njemu ničega više  neće biti, ni čovjeka ni živinčeta, nego će postati vječna pustinja.' 
\par 63 Kad pročitaš ovu knjigu, zaveži je za kamen i baci uEufrat. 
\par 64 I reci: 'Ovako će potonuti Babilon i neće se više podići  iz nesreće koju ću na nj svaliti.'" To su riječi Jeremijine. 



\chapter{52}

\par 1 Sidkiji je bila dvadeset i jedna godina kad se zakraljio,  a kraljevao je jedanaest godina u Jeruzalemu. Materi mu bijaše  ime Hamitala, kćerka Jeremije, i bila je iz Libne. 
\par 2 Činio je  što je zlo u očima Jahvinim, sve kao što je činio Jojakin. 
\par 3 To  je zadesilo Jeruzalem zbog gnjeva Jahvina; Jahve ih napokon i  odbaci ispred lica svoga. Sidkija se pobunio protiv babilonskog kralja. 
\par 4 Devete godine  njegova kraljevanja, desetoga dana desetoga mjeseca, krenu sam  babilonski kralj Nabukodonozor sa svom svojom vojskom na Jeruzalem.  Utabori se pred gradom i opasa ga opkopom. 
\par 5 Grad osta opkoljen  sve do jedanaeste godine Sidkijina kraljevanja. 
\par 6 Devetoga dana  četvrtoga mjeseca, kad je u gradu zavladala takva glad da priprosti  puk nije imao ni kruha, 
\par 7 neprijatelj provali u grad. Tada kralj  i svi ratnici pobjegoše noću kroz vrata između dva zida nad Kraljevskim  vrtom - Kaldejci bijahu opkolili grad - i krenuše putem prema  Arabi. 
\par 8 Kaldejske čete nagnuše za njim u potjeru i sustigoše  Sidkiju na Jerihonskim poljanama, a sva se njegova vojska razbježala. 
\par 9 I Kaldejci uhvatiše kralja i odvedoše ga u Riblu, u zemlji  hamatskoj, pred kralja babilonskog, koji mu izreče presudu. 
\par 10 Pokla  Sidkijine sinove pred njegovim očima, pobi u Ribli sve Judine  knezove; 
\par 11 Sidkiji iskopa oči i okova ga verigama i odvede  u Babilon, gdje ga je držao u tamnici sve do smrti njegove. 
\par 12 Desetoga dana petoga mjeseca - devetnaeste godine kraljevanja  Nabukodonozora, kralja babilonskog - uđe u Jeruzalem Nebuzaradan, zapovjednik tjelesne straže. 
\par 13 On zapali Dom Jahvin, kraljevski  dvor i sve kuće u Jeruzalemu, osobito kuće uglednika; 
\par 14 kaldejske  čete, pod zapovjednikom tjelesne straže, razoriše zidine oko  Jeruzalema. 
\par 15 Nebuzaradan, zapovjednik tjelesne straže, odvede u sužanjstvo  ostatak naroda koji bijaše ostao u gradu, a tako i prebjege babilonskom  kralju i ostalu svjetinu. 
\par 16 Neke od malih ljudi Nebuzaradan  ostavi u zemlji kao vinogradare i ratare. 
\par 17 Kaldejci razbiše tučane stupove u Domu Jahvinu, podnožja  i mjedeno more u Domu Jahvinu, i tuč odniješe u Babilon. 
\par 18 Uzeše  i lonce, lopate, noževe, posudice i uopće sav tučani pribor koji  se upotrebljavao za bogoslužja. 
\par 19 Zapovjednik uze i umivaonice, kadionice, škropionice, lonce, svijećnjake, zdjele, žrtvene  pehare, uopće sve što bijaše od zlata i srebra, 
\par 20 dva stupa, jedno more i dvanaest tučanih volova pod morem, podnožja što  je kralj Salomon dao izraditi za Dom Jahvin. Nije moguće procijeniti  koliko je tuča bilo u svim tim predmetima. 
\par 21 Prvi stup bijaše  visok osamnaest lakata - obuhvatiti ga je mogao konop od dvanaest  lakata - bijaše četiri prsta debeo, a šupalj. 
\par 22 Imao je glavicu  od tuča, visoku pet lakata; i obvijaše je oplet i mogranji, a  sve od tuča. Takav je bio i drugi stup. 
\par 23 A devedeset i šest  šipaka slobodno je visjelo. Sve u svemu bijaše oko sto šipaka  u tom opletu. 
\par 24 Zapovjednik je straže odveo svećeničkog poglavara Seraju, drugog svećenika, Sefaniju, i tri čuvara praga. 
\par 25 Iz grada  je odveo jednog dvorjanina, vojničkog zapovjednika, sedam ljudi  iz kraljeve pratnje koji se zatekoše u gradu, pisara zapovjednika  vojske koji je novačio puk te šezdeset pučana koji se također  zatekoše u gradu. 
\par 26 Zapovjednik tjelesne straže Nebuzaradan  odvede ih pred kralja babilonskog u Riblu. 
\par 27 I kralj babilonski  zapovjedi da ih pogube u Ribli, u zemlji hamatskoj. Tako su judejski  narod odveli s njegove rodne grude. 
\par 28 Evo broja ljudstva što ga Nabukodonozor odvede u sužanjstvo:  sedme godine tri tisuće i dvadeset tri Judejca; 
\par 29 osamnaeste  godine Nabukodonozorove osamsto trideset i dvije osobe iz Jeruzalema; 
\par 30 dvadeset i treće godine Nabukodonozorove, Nebuzaradan, zapovjednik  tjelesne straže, odvede sedam stotina četrdeset i pet Judejaca.  U svemu: četiri tisuće i šest stotona osoba. 
\par 31 A trideset i sedme godine otkako je zasužnjen judejski  kralj Jojakin, dvadeset i petoga dana dvanaestoga mjeseca, babilonski  kralj Evil Merodak u prvoj godini svoje vladavine pomilova judejskoga  kralja Jojakina i pusti ga iz tamnice. 
\par 32 Ljubezno je s njim  razgovarao i stolicu mu postavio više nego drugim kraljevima  koji bijahu s njim u Babilonu. 
\par 33 Jojakin je odložio svoje tamničke  haljine i jeo s kraljem za istim stolom svega svoga vijeka. 
\par 34 Do  kraja njegova života, sve do smrti, babilonski mu je kralj trajno, iz dana u dan, davao uzdržavanje. 




\end{document}