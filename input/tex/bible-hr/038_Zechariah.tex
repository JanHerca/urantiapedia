\begin{document}

\title{Zaharija}


\chapter{1}

\par 1 Osmoga mjeseca druge godine Darijeve dođe riječ Jahvina proroku  Zahariji, sinu Berekjinu, sinu Idonovu. 
\par 2 "Jahve se teško razgnjevio  na oce vaše! 
\par 3 Zatim im reci: 'Ovako govori Jahve nad Vojskama:  Vratite se meni, i ja ću se vratiti vama' - riječ je Jahve nad  Vojskama. 
\par 4 'Ne budite poput svojih otaca koje su pozivali negdašnji  proroci.' Ovako govori Jahve nad Vojskama: 'Vratite se sa zlih  putova svojih i od zlih djela. Ali oni nisu slušali ni pazili  na mene' - govori Jahve. 
\par 5 Gdje su sad oci vaši? Zar će dovijeka  živjeti proroci? 
\par 6 Ali moje riječi i odredbe, koje sam naložio  slugama svojim prorocima, nisu li stigle vaše oce? Oni se obratiše  i priznaše: 'Jahve nad Vojskama učinio je s nama kako bijaše  namislio učiniti prema našim putovima i našim djelima.'" 
\par 7 Dvadeset i četvrtog dana, jedanaestoga mjeseca, a to je  mjesec Šebat, druge godine Darijeve, dođe riječ Jahvina proroku  Zahariji, sinu Berekjinu, sinu Idonovu. 
\par 8 Imao sam noću viđenje. Gle, na riđanu čovjek jaše među  mirtama koje imaju duboko korijenje, a iza njega konji riđi,  smeđi i bijeli. 
\par 9 Upitah: "Koji su ovi, gospodaru?" Anđeo koji  je sa mnom govorio reče mi: "Ja ću ti pokazati koji su." 
\par 10 Čovjek  koji stajaše među mirtama odgovori: "Ovo su oni koje je poslao  Jahve da obilaze zemlju." 
\par 11 Oni se obratiše anđelu Jahvinu, koji stajaše među mirtama, i kazaše: "Obišli smo zemlju, i gle:  sva zemlja počiva i miruje." 
\par 12 Tada progovori anđeo Jahvin i reče: "Jahve nad Vojskama, kada ćeš se već jednom smilovati Jeruzalemu i gradovima judejskim  na koje se već sedamdeset godina ljuto srdiš?" 
\par 13 A Jahve anđelu  koji je govorio sa mnom odgovori utješnim riječima. 
\par 14 I anđeo  koji je govorio sa mnom reče mi: "Objavi ovo: 'Ovako govori Jahve  nad Vojskama: Ljubavlju ljubomornom gorim za Jeruzalem i za Sion, 
\par 15 a velikim gnjevom plamtim na ohole narode, jer kad se ono  malo rasrdih, oni prijeđoše mjeru.' 
\par 16 Zato ovako govori Jahve:  'Vraćam se Jeruzalemu s milosrđem; opet će u njemu sagraditi  Dom moj' - riječ je Jahve nad Vojskama - 'i opet će se u Jeruzalemu  protezati uže mjeračko.' 
\par 17 I ovo poruči: 'Ovako govori Jahve  nad Vojskama: Moji će se gradovi opet prelijevati obiljem, i  Jahve će utješiti Sion, izabrati Jeruzalem.'" 
\par 18 (2:1) Podigoh oči i vidjeh. I gle: četiri roga. 
\par 19 (2:2) Upitah anđela  koji je govorio sa mnom: "Što je ovo?" On mi odgovori: "To su  rogovi koji su raznijeli Judu, Izraela i Jeruzalem." 
\par 20 (2:3) Onda  mi Jahve pokaza četiri kovača. 
\par 21 (2:4) A ja upitah: "Što su došli  ovi raditi?" On mi odgovori: "Ono su rogovi koji su raznijeli  Judu te se nitko više ne usuđuje dići glavu; a ovi su došli da  ih zastraše i da slome rogove narodima koji podizahu rog na zemlju  Judinu kako bi je raznijeli." 


\chapter{2}

\par 1 (2:5) Podigoh oči i vidjeh. I gle: čovjek, a u ruci mu mjeračko  uže. 
\par 2 (2:6) Upitah ga: "Kamo ideš?" Odgovori mi: "Da izmjerim Jeruzalem  i da vidim koliko je širok a koliko dug." 
\par 3 (2:7) I gle, anđeo koji  je govorio sa mnom stajaše nepomično, a drugi mu iziđe u susret 
\par 4 (2:8) i reče mu: "Trči, reci onome mladiću ovako: Jeruzalem će biti  kao otvoren grad radi mnoštva ljudi i stoke koji će biti u njemu. 
\par 5 (2:9) A ja ću mu - riječ je Jahvina - biti ognjen zid unaokolo i  Slava njegova sred njega." 
\par 6 (2:10) "Hej! Hej! Bježite iz zemlje sjeverne" - riječ je Jahvina - "jer u sva četiri vjetra nebeska razasuo sam vas" - riječ je Jahvina! 
\par 7 (2:11) "Hej, Sione, koji živiš kod kćeri babilonske, spasi se!" 
\par 8 (2:12) Ovako govori Jahve nad Vojskama, čija me Slava izaslala k narodima koji su vas opljačkali: "Tko vas dira, dira mi zjenicu oka. 
\par 9 (2:13) Gle, ruku dižem na njih da plijenom budu svojim robovima." Znat ćete tako da me posla Jahve nad Vojskama! 
\par 10 (2:14) "Kliči i raduj se, kćeri sionska, jer evo, dolazim usred tebe prebivat' - riječ je Jahvina. 
\par 11 (2:15) U onaj dan mnogi će narodi uz Jahvu prionuti i bit će narod moj, a ja ću prebivati usred tebe." Znat ćeš tako da me tebi posla Jahve nad Vojskama. 
\par 12 (2:16) I Judeja će biti baština Jahvina, njegov dio u Svetoj zemlji, i on će opet izabrati Jeruzalem. 
\par 13 (2:17) Tiho, ljudi svi, pred Jahvom, jer, evo, on ustaje iz svetoga Prebivališta svojega! 


\chapter{3}

\par 1 Potom mi pokaza Jošuu, velikog svećenika, koji stajaše pred  anđelom Jahvinim, i Satana, koji mu stajaše zdesna da ga tuži. 
\par 2 Anđeo Jahvin reče Satanu: "Suzbio te Jahve, Satane! Suzbio  te Jahve koji izabra Jeruzalem! Nije li on glavnja iz ognja izvučena?" 
\par 3 A Jošua bijaše obučen u prljave haljine dok stajaše pred anđelom  Jahvinim. 
\par 4 Anđeo se obrati onima koji pred njim stajahu i reče  im: "Skinite s njega te prljave haljine!" I reče mu: "Evo, skidam  s tebe tvoju krivicu i odijevam te u dragocjenu haljinu!" 
\par 5 I  nastavi: "Stavite mu čist povez oko glave!" Oni mu staviše čist  povez oko glave i odjenuše ga u dragocjene haljine u nazočnosti  anđelovoj. 
\par 6 I anđeo Jahvin upozori Jošuu: 
\par 7 "Ovako govori  Jahve nad Vojskama: 'Ako budeš mojim putovima hodio i mojih se  pridržavao naredaba, ti ćeš biti upravitelj u Domu mojemu, čuvat  ćeš moja predvorja i dat ću ti pristup među one koji ondje stoje. 
\par 8 Poslušaj, dakle, Jošua, veliki svećeniče, ti i drugovi  tvoji koji su oko tebe, jer vi ste ljudi znamenja! Evo, ja ću  dovesti Izdanak, Slugu svojega, i uklonit ću opačinu ove zemlje  u jedan dan. 
\par 9 Jer evo kamena koji stavljam pred Jošuu: na tom  je kamenu sedam očiju i u nj ću urezati natpis' - riječ je Jahve  nad Vojskama. 
\par 10 'U dan onaj' - riječ je Jahve nad Vojskama  - 'pozivat ćete jedan drugoga pod lozu i pod smokvu.'" 


\chapter{4}

\par 1 Anđeo koji je govorio sa mnom vrati se tad i probudi me kao  čovjeka koji se oda sna budi. 
\par 2 "Što vidiš?" - upita. Ja odgovorih:  "Vidim, evo, svijećnjak, sav od zlata, s posudom za ulje vrh  njega; i sedam je žižaka na svijećnjaku, sa sedam lijevaka za  sedam žižaka što su na njemu. 
\par 3 Dvije su masline kraj njega, jedna njemu zdesna, druga slijeva." 
\par 4 Obratih se anđelu koji  je govorio sa mnom i upitah ga: "Što je to, gospodaru?" 
\par 5 Anđeo  koji je govorio sa mnom odgovori mi: "Zar ne znaš što je to?"  Ja rekoh: "Ne, gospodaru." 
\par 6 [6a] On mi tad odgovori ovako: [6b]Evo riječi Jahvine Zerubabelu: "Ne silom niti snagom, već duhom mojim!" - riječ je Jahve nad Vojskama. 
\par 7 Što si ti, goro velika? Pred Zreubabelom postaješ ravnica!  On će izvući krunišni kamen uz poklike: "Hvala! Hvala za njega!" 
\par 8 Dođe mi potom riječ Jahvina: 
\par 9 Zerubabelove su ruke ovaj  Dom utemeljile, njegove će ga ruke završiti. I vi ćete znati  da me k vama poslao Jahve nad Vojskama. 
\par 10 [10a] Jer, tko je prezreo  dan skromnih početaka? Radovat će se kad vide olovni visak u  ruci Zerubabelovoj. [10b] "Ovih sedam oči su Jahvine što strijeljaju po svoj zemlji." 
\par 11 Tad progovorih i zapitah ga: "Što su one dvije masline desno  i lijevo od svijećnjaka?" 
\par 12 Progovorih opet i upitah ga: "Što  su one dvije maslinove grančice koje kroz dvije zlatne cijevi  dolijevaju ulje?" 
\par 13 On mi odgovori: "Zar ne znaš što je to?"  Odvratih: "Ne, gospodaru!" 
\par 14 On reče: "To su dva Pomazanika  koji stoje pred Gospodarem sve zemlje." 


\chapter{5}

\par 1 Podigoh opet oči i vidjeh: leti svitak knjige. 
\par 2 Anđeo me  upita: "Što vidiš?" Odgovorih: "Vidim svitak knjige gdje leti:  dužina joj je dvadeset lakata, a širina deset." 
\par 3 On mi tad  reče: "To je prokletstvo koje će zahvatiti svu zemlju; odsad, svaki koji krade bit će po njem izgnan odavde i svaki koji krivo  priseže bit će po njem odavde protjeran. 
\par 4 Ja ću ga izvesti  - riječ je Jahve nad Vojskama - da uđe u kuću lupežu i u kuću  onome koji se krivo kune mojim imenom te da boravi usred njegove  kuće i uništi je skupa s njenim drvljem i kamenjem." 
\par 5 Anđeo koji je govorio sa mnom iziđe i reče mi: "Podigni  oči i pogledaj što se to pojavljuje." 
\par 6 Ja ga upitah: "Što je  to?" On reče: "To se pojavljuje efa." I nastavi: "To je opća  pokvarenost na zemlji." 
\par 7 I gle, podiže se olovan poklopac i  jedna žena sjedi usred efe. 
\par 8 On reče: "To je zloća." I gurnu  je u efu i baci joj na otvor olovni poklopac. 
\par 9 Podigavši oči, vidjeh: dvije žene izlaze s vjetrom u krilima, a krila im bijahu  kao krila rode; one podigoše efu između zemlje i neba. 
\par 10 Upitah  tad anđela koji je govorio sa mnom: "Kamo odnose efu?" 
\par 11 On  mi odgovori: "Da joj sagrade hram u zemlji šinearskoj i da joj  pripreme postolje na koje će je postaviti." 


\chapter{6}

\par 1 I podigoh oči i vidjeh: gle, četvera bojna kola izlaze između  dviju gora; a gore bijahu od mjedi. 
\par 2 U prvim kolima bijahu  riđi konji; u drugim kolima crni konji; 
\par 3 u trećim kolima bijeli  konji, a u četvrtim kolima konji šareni. 
\par 4 Obratih se anđelu  koji je govorio sa mnom i upitah ga: "Što je to, gospodaru?" 
\par 5 Anđeo mi odgovori ovako: "Ti kreću u četiri vjetra nebeska  pošto su stajali pred Gospodarem sve zemlje. 
\par 6 Riđani kreću  u zemlju istočnu; vranci u zemlju sjevernu; bijelci kreću u zemlju  zapadnu, a šarci kreću u zemlju južnu." 
\par 7 Krepko oni stupaju, nestrpljivi da obiđu zemlju. On im reče: "Idite, obiđite zemlju!"  I oni krenuše obilaziti zemljom. 
\par 8 On me zovnu i reče mi: "Vidi, oni koji su krenuli u sjevernu zemlju umirit će gnjev moj u  zemlji sjevernoj." 
\par 9 I dođe mi riječ Jahvina: 
\par 10 "Uzmi prinose od izgnanika  - od Heldaja, Tobije i Jedaje - i pođi danas i uđi u dom Jošije, sina Sefanijina, koji je došao iz Babilona. 
\par 11 Uzmi srebra  i zlata, načini krunu i stavi na glavu Jošui, sinu Josadakovu, velikom svećeniku. 
\par 12 I reci mu: 'Ovako govori Jahve nad Vojskama:  Evo čovjeka komu je ime Izdanak; ispod njega će proklijati i  on će sazdati Svetište Jahvino. 
\par 13 On će sazdati Svetište Jahvino  i proslaviti se. On će sjediti i vladati na prijestolju. A do  njega će na prijestolju biti svećenik. Sklad savršen bit će među  njima. 
\par 14 A kruna neka ostane u Jahvinu Svetištu za spomen Heldaju, Tobiji, Jedaji i Jošiji, sinu Sefanijinu. 
\par 15 I oni koji su  daleko doći će i sazdat će Svetište Jahvino. Znat ćete tako da  me Jahve nad Vojskama k vama poslao.' To će se zbiti ako zaista poslušate glas Jahve, Boga svojega." 


\chapter{7}

\par 1 Četvrte godine kralja Darija, četvrtoga dana devetoga mjeseca, Kisleva, dođe riječ Jahvina Zahariji. 
\par 2 Betel je naime poslao  Sar-Esera i Regem-Meleka s njihovim ljudima da mole lice Jahvino 
\par 3 i da pitaju svećenike u Domu Jahve nad Vojskama i proroke:  "Hoćemo li plakati petoga mjeseca i postiti, kao što činimo već  tolike godine?" 
\par 4 Tada mi dođe riječ Jahve nad Vojskama: "Reci svemu puku  zemlje i svećenicima: 
\par 5 'Kad postite i naričete petoga i sedmoga  mjeseca već sedamdeset godina, zar meni postite? 
\par 6 A kad jedete  i pijete, zar sebi ne jedete i pijete? 
\par 7 Nisu li to propisi  koje je Jahve objavio preko negdašnjih proroka kada Jeruzalem  bijaše naseljen i miran kao i gradovi oko njega i kada bijaše  napučen Negeb i Šefela?'" 
\par 8 Riječ Jahvina dođe Zahariji: 
\par 9 "Ovako govori Jahve nad  Vojskama: 'Sudite istinito i budite dobrostivi i milosrdni jedni  drugima. 
\par 10 Ne tlačite udovu ni sirotu, ni došljaka ni uboga, i ne snujte u srcu pakosti jedan prema drugom.' 
\par 11 Ali oni  ne htjedoše poslušati, već prkosno okrenuše leđa; zatisnuše uši  da ne bi čuli; 
\par 12 otvrdnuše srcem kao kremen, da ne bi čuli  Zakon i riječi koje im je slao Jahve nad Vojskama, svojim duhom, preko drevnih proroka. I Jahve nad Vojskama silno se tad razgnjevi. 
\par 13 I zato, kao što je on zvao a oni ga ne slušaše, tako su sad  oni zvali a ja ih nisam slušao - riječ je Jahve nad Vojskama. 
\par 14 I razmeo sam ih među sve narode kojih ne poznavahu, a zemlja  iza njih bi opustošena, te nitko nije njome prolazio niti se  vraćao. Tako su zemlju blagostanja obratili u pustoš!" 


\chapter{8}

\par 1 I dođe mi riječ Jahvina: 
\par 2 "Ovako govori Jahve nad Vojskama: 'Ljubavlju ljubomornom za Sion izgaram i gnjevom velikim plamtim za nj! Vraćam se u Sion, prebivati hoću sred Jeruzalema.' 
\par 3 Ovako govori Jahve nad Vojskama: 'Jeruzalem će se zvati Gradom vjernosti i Gorom Jahve nad Vojskama, Gorom svetosti.' 
\par 4 Ovako govori Jahve nad Vojskama: 'Starci i starice opet će posjedati po trgovima jeruzalemskim, svatko sa štapom u ruci zbog starosti prevelike. 
\par 5 A gradski će se trgovi ispuniti dječacima i djevojčicama koji će se igrati na njegovim trgovima.' 
\par 6 Ovako govori Jahve nad Vojskama: 'Ako to bude čudo u očima Ostatka u dane one, zar će to biti čudo i u mojim očima' - riječ je Jahve nad Vojskama. 
\par 7 Ovako govori Jahve nad Vojskama: 'Evo spasit ću svoj narod iz zemlje istočne i iz zemlje sunčanog zapada. 
\par 8 Ja ću ih dovesti da se nastane usred Jeruzalema. I bit će mi narod a ja ću im biti Bog u vjernosti i pravdi.' 
\par 9 Ovako Govori Jahve nad Vojskama: 'Neka ojačaju ruke vama  koji ovih dana slušate riječi ove iz usta proroka koji prorokuje  od dana kada bjehu položeni temelji Domu Jahve nad Vojskama da  bi se opet sagradilo Svetište. 
\par 10 Jer, prije ovih dana ne bijaše  nadnice za čovjeka niti nadnice za živinče; niti bijaše mira  od neprijatelja onome koji je izlazio ni onome koji je dolazio;  puštao sam ljude jedne protiv drugih. 
\par 11 Ali sada, neću biti  prema Ostatku ovog naroda kao minulih dana - riječ je Jahve nad  Vojskama - 
\par 12 nego ću posijati mir: loza će roditi grožđem,  zemlja će davati usjeve, a nebo će davati rosu svoju. Sve ću  to dati Ostatku ovoga naroda. 
\par 13 I kao što bijaste prokletstvo  među narodima, dome Judin i dome Izraelov, tako ću vas spasiti  da budete blagoslovom! Ne bojte se, nek' vam jake budu ruke!' 
\par 14 Jer ovako govori Jahve nad Vojskama: 'Kao što bijah namislio  unesrećiti vas kada su me razgnjevili oci vaši - govori Jahve  nad Vojskama - i nisam se pokajao, 
\par 15 tako, promijenivši naum, u ove dane mislim usrećiti Jeruzalem i dom Judin. Ne bojte se! 
\par 16 A ovo vam je činiti: Govorite istinu jedan drugom; sudite  istinito i miroljubivo na vratima gradskim! 
\par 17 Ne snujte jedan  drugome zlo u srcu; ne ljubite lažnu kletvu. Jer sve to ja mrzim'  - riječ je Jahvina!" 
\par 18 Dođe mi riječ Jahve nad Vojskama: 
\par 19 "Ovako govori Jahve nad Vojskama: Post četvrtoga, post  petoga, post sedmoga i post desetoga mjeseca postat će za Dom  Jahvin radost, veselje i veseli blagdani. Ali ljubite istinu  i mir!" 
\par 20 "Ovako govori Jahve nad Vojskama: Još će dolaziti narodi  i stanovnici mnogih gradova. 
\par 21 Stanovnici jednoga grada ići  će u drugi govoreći: 'Hajde da idemo moliti lice Jahvino i tražiti  Jahvu nad Vojskama!' Ići ću i ja! 
\par 22 I doći će mnogi puk i moćni  će narodi tražiti Jahvu nad Vojskama u Jeruzalemu i moliti lice  Jahvino. 
\par 23 Ovako govori Jahve nad Vojskama! U one će dane deset  ljudi od naroda svih jezika hvatati jednog Židova za skut govoreći:  'Idemo s vama, jer čusmo da je s vama Bog.'" 


\chapter{9}

\par 1 Proroštvo.  Riječ Jahvina. Jahve prolazi zemljom Hadraka, Damask mu je počivalište; jer Jahvini su gradovi Arama i sva plemena Izraela. 
\par 2 Hamat također, koji s njim graniči, 
\par 3 i Tir i Sidon, tako mudar. Tir podiže tvrde bedeme, zgrnu srebra kao prašine i zlata kao blata s ulica. 
\par 4 Al' evo, Gospod će ga osvojiti, survati u more moć njegovu, a njega će progutati oganj. 
\par 5 Vidjet će to Aškelon i prestrašiti se, a Gaza sva će uzdrhtati, i Ekron, jer ga nada prevari: nestat će kralja iz Gaze, Aškelon će pust ostati, 
\par 6 u Ašdodu stanovat će kopilad! Zatrt ću ponos Filistejaca, 
\par 7 uklonit ću im krv iz usta i gnusobu iz zuba. I oni će pripasti Bogu našem i bit će kao jedna obitelj u Judeji, a Ekron će biti kao Jebusejac. 
\par 8 Uz Dom svoj utaborit ću se kao straža, protiv onih koji odlaze i dolaze; tlačitelj neće više ovud prolaziti, jer njegovu sam uvidio bijedu. 
\par 9 Klikni iz sveg grla, Kćeri sionska! Viči od radosti, Kćeri jeruzalemska! Tvoj kralj se evo tebi vraća: pravičan je i pobjedonosan, ponizan jaše na magarcu, na magaretu, mladetu magaričinu. 
\par 10 On će istrijebit' kola iz Efrajima i konje iz Jeruzalema; on će istrijebit' luk ubojni. On će navijestit' mir narodima; vlast će mu se proširit' od mora do mora i od Rijeke do rubova zemlje. 
\par 11 A i tebi, zbog krvi tvoga Saveza, vratit ću sužnje tvoje iz jama bezvodnih. 
\par 12 Vratite se u Tvrđavu, izgnanici puni nade, još danas - ja navješćujem - dvostruko ću ti uzvratiti. 
\par 13 Jer, Judu sam kao luk napeo, a Efrajimom luk naoružao: tvoje ću, o Sione, zavitlat' sinove - protiv sinova tvojih, o Javane - i učinit ću te kao mač junaka. 
\par 14 Nad njima tad će se pojaviti Jahve i kao munja letjet će mu strijela. Jahve Gospod u rog će zatrubit', hodit će na južnim vihorima. 
\par 15 Jahve nad Vojskama zakrilit će ih i oni će gaziti nogama kamenje praćaka, pit će krv kao da je vino, napojit' se kao škropilo, kao uglovi na žrtveniku. 
\par 16 Jahve Bog njihov spasit će ih u dan onaj; kao stado on će pasti narod svoj; kao drago kamenje krune oni će blistat' u zemlji njegovoj. 
\par 17 Ah, kako li će sretan, kako lijep biti! Od žita će rasti mladići, a od slatkog vina djevice. 


\chapter{10}

\par 1 Tražite od Jahve dažda u vrijeme proljetno! Jahve stvara munje i daje kišu; čovjeku kruh daje, a stoci travu. 
\par 2 Lažno bajaju kumiri, prijevaru vide gatari, obmanu govore snovi, varljivu utjehu daju, zato kao stado blude ljudi, lutaju jer nemaju pastira. 
\par 3 Moj je gnjev planuo na pastire, i ja ću kaznom pohodit jarce. Da, Jahve nad Vojskama pohodit će stado svoje, dom Judin. I učinit će da budu k'o gizdav konj u boju: 
\par 4 od njega će poteći kamen zaglavni, klin šatorski, od njega ubojit luk, od njega sve vođe. 
\par 5 Bit će zajedno kao junaci što u boju gaze kao po blatu uličnom; vojevat će, jer Jahve je s njima, i osramotit će one koji konje jašu. 
\par 6 "Ojačat ću dom Judin, spasiti dom Josipov. Opet ću ih naseliti, žao mi ih, i bit će kao da ih nisam odbacio, jer ja sam Jahve, Bog njihov - uslišat ću ih." 
\par 7 Efrajimci bit će kao junaci i radostit će im se srce kao od vina: vidjet će sinove svoje i veseliti se, u Jahvi će klicati srce njihovo. 
\par 8 "Zazviždat ću im i sabrati ih, jer ja sam ih izbavio, bit će opet brojni kao što bjehu. 
\par 9 Rasijao sam ih među narode, ali će se oni u zemljama dalekim spomenuti mene, poučit će svoje sinove, i oni će se vratiti. 
\par 10 Vratit ću ih iz zemlje egipatske, sabrat ću ih iz Asirije i dovest ih u zemlju gileadsku i na Libanon, i neće biti dosta mjesta za njih." 
\par 11 Prijeći će more egipatsko, jer on će udariti valove morske, sve dubine Nila presahnut će. Bit će oboren ponos Asirije, oduzeto žezlo Egiptu. 
\par 12 U Jahvi će biti snaga njihova, njegovim će se oni proslavit imenom - riječ je Jahvina. 


\chapter{11}

\par 1 Otvori vrata, Libanone, nek' ti oganj sažeže cedrove! 
\par 2 Kukaj, čempresu, jer pade cedar, jer su mogućnici upropašteni! Kukajte, hrastovi bašanski, jer posječena je šuma najgušća. 
\par 3 Čuj jauk pastira, opustošen je sjaj njihov! Čuj riku lavića, opustošen je ponos jordanski! 
\par 4 Ovako mi reče Jahve: 
\par 5 "Pasi ovce klanice! Kupci ih njihovi  kolju nekažnjeno, a koji ih prodaju, govore: 'Blagoslovljen bio  Jahve, obogatio sam se!' i pastiri ih njihovi ne štede. 
\par 6 Ni  ja više neću štedjeti žitelja zemlje - riječ je Jahve nad Vojskama  - nego: predajem, evo, svakoga u ruke njegova bližnjega i u ruke  kralja njegova; i oni će pustošiti zemlju, a ja neću izbavljati  iz ruku njihovih." 
\par 7 Stadoh pasti ovce klanice za trgovce ovcama  te uzeh dva štapa: jedan nazvah Naklonost, drugi Sveza. Tako  sam pasao stado. 
\par 8 I u jednom mjesecu odbacih tri pastira. Ali  mi i ovce dojadiše, omrznuh im. 
\par 9 Tad rekoh: "Neću vas više  pasti! Koja mora uginuti, nek' ugine! Koja mora nestati, nek'  nestane! A koje ostanu, neka jedna drugoj meso prožderu!" 
\par 10 Tad  uzeh svoj štap Naklonost i slomih ga da raskinem Savez svoj što  ga bijah sklopio sa svim narodima. 
\par 11 I on se raskinu onog dana, i trgovci ovcama koji su to gledali doznaše da je to bila riječ  Jahvina. 
\par 12 Rekoh im tad: "Ako vam je to dobro, dajte mi plaću;  ako nije, nemojte." Oni mi odmjeriše plaću: trideset srebrnika. 
\par 13 A Jahve mi reče: "Baci u riznicu tu lijepu cijenu kojom su  me procijenili!" Ja uzeh trideset srebrnika i bacih u riznicu  u Domu Jahvinu. 
\par 14 Onda slomih i svoj drugi štap, Svezu - da  raskinem bratstvo između Jude i Izraela. 
\par 15 I reče mi Jahve: "Uzmi još opremu bezumna pastira, 
\par 16 jer, evo, podići ću jednoga bezumnog pastira u ovoj zemlji: za izgubljene  on se neće brinuti, zalutale neće tražiti, ranjene neće vidati, iscrpljene neće nositi, nego će jesti meso od pretilih i papke  im otkidati. 
\par 17 Teško pastiru opakom koji stado ostavlja! Neka mu mač stigne ruku i desno oko! Nek' mu desnica sasvim usahne, oko desno sasvim potamni!" 


\chapter{12}

\par 1 Proroštvo. Besjeda Jahvina o Izraelu. Govori Jahve koji razape  nebesa, utemelji zemlju i stvori dah čovjeku u grudima: 
\par 2 "Evo, učinit ću Jeruzalem čašom opojnom svim narodima  uokolo - za opsade Jeruzalema. 
\par 3 U onaj dan učinit ću Jeruzalem  teškim kamenom svim narodima: svi koji ga budu dizali teško će  se izraniti, a skupit će se na nj svi narodi zemlje. 
\par 4 U onaj  dan - riječ je Jahvina - udarit ću sve konje strahom, a njine  jahače mahnitošću. Ali nad domom Judinim otvorit ću oči, a sljepilom  ću udariti sve konje narodÄa. 
\par 5 Tada će u srcu reći plemena  Judina: 'Snaga je Jeruzalemaca u Jahvi nad Vojskama, Bogu njihovu!' 
\par 6 U onaj dan učinit ću da plemena Judina budu kao žeravnica  užarena na drvlju, kao baklja upaljena na snoplju: i proždirat  će zdesna i slijeva sve narode uokolo. A Jeruzalem će i dalje  stajati na svome mjestu." 
\par 7 Jahve će najprije spasiti Judine  šatore da se ponos doma Davidova i ponos Jeruzalemaca ne izdigne  iznad Jude. 
\par 8 U onaj dan Jahve će zakriliti Jeruzalemce: najsustaliji  među njima bit će u onaj dan kao David, a dom Davidov bit će  kao božanstvo, kao Anđeo Jahvin pred njima. 
\par 9 "U onaj dan pregnut ću da uništim sve narode koji dođu  na Jeruzalem. 
\par 10 A na dom Davidov i na Jeruzalemce izlit ću  duh milosni i molitveni. I gledat će na onoga koga su proboli;  naricat će nad njim kao nad jedincem, gorko ga oplakivati kao  prvenca. 
\par 11 U onaj dan plač velik će nastati u Jeruzalemu, poput  plača hadad-rimonskog u ravnici megidonskoj. 
\par 12 I plakat će  zemlja, svaka porodica napose, i žene njihove napose; porodica  doma Davidova napose, i žene njihove napose; porodica doma Natanova  napose, i žene njihove napose; 
\par 13 porodica doma Levijeva napose, i žene njihove napose; porodica Šimejeva napose, i žene njihove  napose; 
\par 14 i sve ostale porodice, svaka porodica za sebe, i  žene njihove napose. 


\chapter{13}

\par 1 U onaj dan otvorit će se izvor domu Davidovu i Jeruzalemcima  da se operu od grijeha i nečistoće. 
\par 2 U onaj dan - riječ je  Jahve nad Vojskama - iskorijenit ću iz zemlje imena kumirÄa da  se više ne spominju; uklonit ću iz zemlje i proroke i duh nečistoće. 
\par 3 Ako netko još bude prorokovao, otac i mati koji su ga rodili  reći će mu: 'Nećeš više živjeti, jer laž govoriš u ime Jahvino!'  Otac i mati koji su ga rodili probost će ga kada bude prorokovao. 
\par 4 U onaj dan svaki će se prorok stidjeti svoga viđenja; neće  se više ogrtati plaštem od kostrijeti da bi lagali, 
\par 5 nego će  govoriti: 'Nisam ja prorok; ja sam ratar, zemlja je moje dobro  od mladosti!' 
\par 6 Ako li ga tko upita: 'Kakve su ti to rane po  tijelu?' on će odgovoriti: 'Izranjen sam kod prijateljÄa.' 
\par 7 Probudi se, maču, protiv mog pastira, protiv čovjeka, moga srodnika - riječ je Jahve nad Vojskama. Udari pastira, i ovce će se razbjeći! Okrenut ću ruku protiv slabića, 
\par 8 i u svoj će zemlji - riječ je Jahvina - dvije trećine biti istrijebljene, a trećina ostavljena. 
\par 9 Tu ću trećinu kroz oganj provesti, pročistit ću ih kao što se pročišćuje srebro, iskušat' ih kao što se srebro iskušava. I on će zazivati ime moje, a ja ću mu se odazvati; i reći ću: 'Moj je to narod!' a on će reći: 'Jahve je Bog moj!'" 



\chapter{14}

\par 1 Gle, dolazi dan Jahvin kada će se podijeliti plijen usred  tebe. 
\par 2 I sabrat ću sve narode u Jeruzalem u borbu. I zaposjest  će grad, opljačkati kuće i silovati žene. Polovina će grada otići  u izgnanstvo, ali Ostatak neće biti istrijebljen iz grada. 
\par 3 Tada  će Jahve izaći i boriti se protiv tih naroda kako on zna ratovati  u dan ratni. 
\par 4 Noge će mu, u dan onaj, stajati na Gori maslinskoj  koja je nasuprot Jeruzalemu na istoku. I raskolit će se Gora  maslinska po srijedi, između istoka i zapada, u golemu dolinu:  jedna će se polovina pomaknuti na sjever, druga na jug. 
\par 5 Dolina  Gore moje bit će ispunjena od Goe pa do Jasola i bit će zakrčena  kao što je bila zakrčena poslije potresa u dane Uzije, kralja  judejskog. Tada će doći Jahve, Bog tvoj, i svi sveci s njim. 
\par 6 U dan onaj neće više biti ni studeni ni leda. 
\par 7 Bit će to  dan čudesan - znade ga Jahve - ni dan ni noć; i u vrijeme večeribit  će svjetlo. 
\par 8 U onaj dan žive će vode poteći iz Jeruzalema,  pola k moru istočnom, pola kmoru zapadnom. Bit će tako ljeti  i zimi. 
\par 9 I Jahve će biti kralj nad svom zemljom. 
\par 10 Sva će  se zemlja pretvoriti u ravnicu, od Gebe do Rimona negepskog.  A Jeruzalem će se uzvisiti na svom mjestu; i bit će nastanjen  - od Vrata Benjaminovih do Prvih vrata, to jest do Vrata ugaonih, i od Kule Hananeelove do Kraljeva tijeska. 
\par 11 Opet će se stanovati  u njemu, i više neće biti prokletstva; Jeruzalem će živjeti u  miru. 
\par 12 A evo kojom će ranom Jahve udariti sve narode koji budu zavojštili  na Jeruzalem: meso će im se raspadati dok budu na nogama; oči  će im trunuti u dupljama, jezik gnjiti u ustima. 
\par 13 U dan onaj  nastat će među njima silan metež od Jahve: jedan će drugoga za  ruku hvatati, i ruka će se jednoga dizati na drugoga. 
\par 14 I Juda  će se boriti u Jeruzalemu. Tu će se sakupiti bogatstva svih okolnih  naroda: zlato, srebro, odjeća u velikoj množini. 
\par 15 A slična  će rana pasti na konje, mazge, deve i magarce, i na svu stoku  koja se nađe u tome taboru. 
\par 16 Tko preživi od svih naroda koji dođu na Jeruzalem, uzlazit  će godimice da se pokloni pred Kraljem, Jahvom nad Vojskama,  i da slave Blagdan sjenica. 
\par 17 Ako koje pleme zemlje ne uzađe  u Jeruzalem da se pokloni pred Kraljem, Jahvom nad Vojskama,  neće biti kiše za njega. 
\par 18 Ako li pleme egipatsko ne uzađe  i ne dođe, stići će ga isti udarac kojim će Jahve udariti narode  koji ne bi uzašli svetkovati Blagdan sjenica. 
\par 19 Takva će biti  kazna Egiptu i svim narodima koji ne budu uzašli da svetkuju  Blagdan sjenica. 
\par 20 U onaj dan stajat će na konjskim praporcima  'Jahvi posvećen'; a u Domu Jahvinu bit će lonci kao žrtvene čaše  pred žrtvenikom; 
\par 21 i svaki će lonac u Jeruzalemu i u Judeji  biti posvećen Jahvi nad Vojskama - svi koji budu htjeli žrtvovati  uzimat će ih i kuhati u njima. I u dan onaj neće više biti trgovaca  u Domu Jahve nad Vojskama. 




\end{document}