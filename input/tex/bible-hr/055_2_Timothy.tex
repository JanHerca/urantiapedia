\begin{document}

\title{2 Timoteju}


\chapter{1}

\par 1 Pavao, apostol Krista Isusa voljom Božjom, po obećanju života, života u Kristu Isusu, 
\par 2 Timoteju, ljubljenom sinu: milost, milosrđe i mir od Boga Oca i Krista Isusa, Gospodina našega. 
\par 3 Zahvalan sam Bogu, kojemu onamo od predaka čiste savjesti  služim, dok te se neprestano spominjem u svojim molitvama noć  i dan. 
\par 4 Sjetim se tvojih suza i zaželim vidjeti te da se napunim  radosti 
\par 5 imajući na pameti neprijetvornu vjeru koja je u tebi  - onu vjeru koja je najprije prebivala u tvojoj baki Loidi i  tvojoj majci Euniki, a uvjeren sam, i u tebi. 
\par 6 Poradi toga podsjećam te: raspiruj milosni dar Božji koji  je u tebi po polaganju mojih ruku. 
\par 7 Jer nije nam Bog dao duha  bojažljivosti, nego snage, ljubavi i razbora. 
\par 8 Ne stidi se  stoga svjedočanstva za Gospodina našega, ni mene, sužnja njegova.  Nego zlopati se zajedno sa mnom za evanđelje, po snazi Boga 
\par 9 koji  nas je spasio i pozvao pozivom svetim - ne po našim djelima,  nego po svojem naumu i milosti koja nam je dana u Kristu Isusu  prije vremena vjekovječnih, 
\par 10 a očitovana je sada pojavkom  Spasitelja našega Krista Isusa, koji obeskrijepi smrt i učini  da zasja život i neraspadljivost - po evanđelju 
\par 11 za koje sam  ja postavljen propovjednikom, apostolom i učiteljem. 
\par 12 Poradi toga i ovo trpim, ali se ne stidim jer znam komu  sam povjerovao i uvjeren sam da je on moćan poklad moj sačuvati  za onaj Dan. 
\par 13 Uzorom neka ti budu zdrave riječi koje si od mene čuo  u vjeri i ljubavi u Kristu Isusu. 
\par 14 Lijepi poklad čuvaj po  Duhu Svetom koji prebiva u nama. 
\par 15 Napustiše me, to znaš, svi u Aziji, među njima i Figel  i Hermogen. 
\par 16 Neka Gospodin milosrđem podari Oneziforov dom  jer me često osvježivao i nije se stidio mojih okova, 
\par 17 nego  kad je bio u Rimu, brižljivo me potražio i našao. 
\par 18 Dao mu  Gospodin naći milosrđe u Gospodina u onaj Dan! A koliko je usluga  u Efezu iskazao, to ti najbolje znaš. 


\chapter{2}

\par 1 Ti se dakle, dijete moje, jačaj milošću u Kristu Isusu 
\par 2 i  što si od mene po mnogim svjedocima čuo, to predaj vjernim ljudima  koji će biti podobni i druge poučiti. 
\par 3 S njima se zlopati kao  dobar vojnik Krista Isusa. 
\par 4 Tko vojuje, ne zapleće se u svagdanje  poslove kako bi se vojskovođi svidio. 
\par 5 I natječe li se tko, ne ovjenčava se ako se zakonito ne natječe. 
\par 6 Ratar koji se  trudi treba da prvi primi od uroda. 
\par 7 Shvati što govorim! Ta  dat će ti Gospodin razum u svemu. 
\par 8 Spominji se Isusa Krista, uskrsla od mrtvih, od potomstva  Davidova - po mojem evanđelju. 
\par 9 Za nj se ja zlopatim sve do  okova, kao zločinac. Ali riječ Božja nije okovana! 
\par 10 Stoga  sve podnosim radi izabranih, da i oni postignu spasenje, spasenje  u Kristu Isusu, zajedno s vječnom slavom. 
\par 11 Vjerodostojna je riječ: Ako s njime umrijesmo, s njime ćemo i živjeti. 
\par 12 Ako ustrajemo, s njime ćemo i kraljevati. Ako ga zaniječemo, i on će zanijekati nas. 
\par 13 Ako ne budemo vjerni, on vjeran ostaje. Ta ne može sebe zanijekati! 
\par 14 Na to podsjećaj zaklinjući pred Bogom neka ne bude rječoborstva:  ničemu ne koristi, a na propast je onima koji slušaju. 
\par 15 Uznastoj  da kao prokušan staneš pred Boga kao radnik koji se nema čega  stidjeti, koji ispravno reže riječ istine. 
\par 16 Svjetovnih se  pak praznorječja kloni: sve će više provaljivati prema bezbožnosti 
\par 17 i riječ će njihova kao rak-rana izgrizati. Od njih su Himenej  i Filet, 
\par 18 koji zastraniše od istine tvrdeći da je uskrsnuće  već bilo te nekima prevraćaju vjeru. 
\par 19 Ipak čvrsti temelj Božji stoji - pod ovim je pečatom:  Poznaje Gospodin one koji su njegovi i neka se kloni zloće  tko god imenuje ime Gospodnje. 
\par 20 Pa u velikoj kući ima posuda ne samo zlatnih i srebrnih, nego i drvenih i glinenih; i jedne su časne, druge pak nečasne. 
\par 21 Očisti li se dakle tko od toga, bit će posuda časna, posvećena, korisna Gospodaru, za svako dobro djelo prikladna. 
\par 22 A mladenačkih se strastvenosti kloni! Teži za pravednošću, vjerom, ljubavlju, mirom sa svima koji iz čista srca prizivlju  Gospodina. 
\par 23 Lude pak i neobuzdane raspre odbijaj znajući da  rađaju svađama. 
\par 24 A sluga Gospodnji treba da se ne svađa, nego  da bude nježan prema svima, sposoban poučavati, zlo podnositi, 
\par 25 da s blagošću preodgaja protivnike, ne bi li ih Bog podario  obraćenjem te spoznaju istinu 
\par 26 i ponovno budu trijezni izvan  zamke đavla koji ih drži robljem svoje volje. 


\chapter{3}

\par 1 A ovo znaj: u posljednjim danima nastat će teška vremena. 
\par 2 Ljudi  će doista biti sebeljupci, srebroljupci, preuzetnici, oholice, hulitelji, roditeljima neposlušni, nezahvalnici, bezbožnici, 
\par 3 bešćutnici, nepomirljivci, klevetnici, neobuzdanici, goropadnici, neljubitelji dobra, 
\par 4 izdajice, brzopletnici, naduti, ljubitelji  užitka više nego ljubitelji Boga. 
\par 5 Imaju obličje pobožnosti, ali snage su se njezine odrekli. I njih se kloni! 
\par 6 Od njih su doista oni što se uvlače u kuće i zarobljuju  ženice, natovarene grijesima, vodane najrazličitijim strastima: 
\par 7 one uvijek uče, a nikako ne mogu doći do spoznaje istine. 
\par 8 Kao što se Janes i Jambres suprotstaviše Mojsiju, tako se  i ovi, ljudi pokvarena uma, u vjeri neprokušani, suprotstavljaju  istini. 
\par 9 Ali neće više napredovati jer bezumlje će ovih postati  očito, kako se to i onima dogodilo. 
\par 10 A ti si pošao za nmom u poučavanju, u ponašanju, u naumu, u vjeri, u strpljivosti, u ljubavi, u postojanosti; 
\par 11 u progonstvima, u patnjama koje su me zadesile u Antiohiji, u Ikoniju, u Listri.  Kakva li sam progonstva podnio! I iz svih me izbavio Gospodin! 
\par 12 A i svi koji hoće živjeti pobožno u Kristu Isusu, bit će  progonjeni. 
\par 13 Zli pak ljudi i vračari napredovat će sve više  u zlu - kao zavodnici i zavedeni. 
\par 14 Ti, naprotiv, ostani u onome u čemu si poučen i čemu  si vjeru dao, svjestan od koga si sve poučen 
\par 15 i da od malena  poznaješ Sveta pisma koja su vrsna učiniti te mudrim tebi na  spasenje po vjeri, vjeri u Kristu Isusu. 
\par 16 Sve Pismo, bogoduho, korisno je za poučavanje, uvjeravanje, popravljanje, odgajanje u pravednosti, 
\par 17 da čovjek Božji bude  vrstan, za svako dobro djelo podoban. 



\chapter{4}

\par 1 Zaklinjem te pred Bogom i Kristom Isusom, koji će suditi žive  i mrtve, zaklinjem te pojavkom njegovim i kraljevstvom njegovim: 
\par 2 propovijedaj Riječ, uporan budi - bilo to zgodno ili nezgodno  - uvjeravaj, prijeti, zapovijedaj sa svom strpljivošću i poukom. 
\par 3 Jer doći će vrijeme kad ljudi neće podnositi zdrava nauka  nego će sebi po vlastitim požudama nagomilavati učitelje kako  im godi ušima; 
\par 4 od istine će uho odvraćati, a bajkama se priklanjati. 
\par 5 Ti, naprotiv, budi trijezan u svemu, zlopati se, djelo izvrši  blagovjesničko, služenje svoje posve ispuni! 
\par 6 Jer ja se već prinosim za žrtvu ljevanicu, prispjelo je  vrijeme moga odlaska. 
\par 7 Dobar sam boj bio, trku završio, vjeru  sačuvao. 
\par 8 Stoga, pripravljen mi je vijenac pravednosti kojim  će mi u onaj Dan uzvratiti Gospodin, pravedan sudac; ne samo  meni, nego i svima koji s ljubavlju čekaju njegov pojavak. 
\par 9 Nastoj što prije doći k meni! 
\par 10 Jer Dema me, zaljubljen  u sadašnji svijet, napustio i otišao u Solun; Krescencije u Galaciju, Tit u Dalmaciju. 
\par 11 Luka je jedini sa mnom. Marka uzmi i dovedi  sa sobom jer mi je koristan za služenje. 
\par 12 Tihika sam poslao  u Efez. 
\par 13 Kabanicu koju ostavih u Troadi kod Karpa, kada dođeš, donesi. I knjige, osobito pergamene. 
\par 14 Aleksandar kovač nanio  mi je mnogo zla. Uzvratio mu Gospodin po njegovim djelima! 
\par 15 Njega  se i ti čuvaj jer se veoma usprotivio našim riječima. 
\par 16 Za prve moje obrane nitko ne bijaše uza me, svi me napustiše.  Ne uračunalo im se! 
\par 17 Ali Gospodin je stajao uza me, on me  krijepio da se po meni potpuno razglasi Poruka te je čuju svi  narodi; i izbavljen sam iz usta lavljih. 
\par 18 Izbavit će me Gospodin  od svakoga zla djela i spasiti za svoje nebesko kraljevstvo.  Njemu slava u vijeke vjekova! Amen! 
\par 19 Pozdravi Prisku i Akvilu i Oneziforov dom! 
\par 20 Erast  osta u Korintu, a Trofima ostavih u Miletu bolesna. 
\par 21 Nastoj  doći prije zime. Pozdravlja te Eubul, Pudencije, Lino, Klaudija  i sva braća. 
\par 22 Gospodin s duhom tvojim. Milost s vama! 




\end{document}