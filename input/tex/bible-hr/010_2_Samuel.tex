\begin{document}

\title{2 Samuelova}


\chapter{1}

\par 1 Poslije Šaulove smrti David se vratio kući pobijedivši Amalečane.  Dva je dana proveo u Siklagu. 
\par 2 Trećega dana dođe neki čovjek  iz Šaulova tabora, razdrtih haljina i prahom posute glave. Došavši  k Davidu, baci se na zemlju i pokloni mu se. 
\par 3 David ga upita:  "Odakle dolaziš?" A on mu odgovori: "Umakao sam iz izraelskog  tabora." 
\par 4 A David ga upita: "Što se dogodilo? Pripovijedaj  mi!" On odvrati: "Narod je pobjegao iz boja, a mnogo je ljudi  i poginulo. Mrtvi su i Šaul i njegov sin Jonatan." 
\par 5 Nato David upita mladoga glasonošu: "Kako znaš da je poginuo  Šaul i njegov sin Jonatan?" 
\par 6 A mladi glasonoša odgovori: "Slučajno  sam došao na goru Gilbou i vidio Šaula kako se upro u svoje koplje, a bojna kola i konjanici natisnuli se za njim. 
\par 7 Šaul, obazrevši  se, ugleda mene pa me zovnu, a ja mu se odazvah: 'Evo me!' 
\par 8 I  upita me on: 'Tko si ti?' A ja mu odgovorih: 'Amalečanin sam.' 
\par 9 Tada mi on reče: 'Dođi ovamo k meni pa me ubij, jer me obuzeo  smrtni grč, a duša je još sva u meni!' 
\par 10 Pristupih k njemu  i zadadoh mu smrtni udarac, jer sam znao da neće preživjeti nakon  pada. Zatim uzeh kraljevski znak koji mu bijaše na glavi i narukvicu  koju imaše na ruci i, evo, donesoh to svome gospodaru." 
\par 11 Tada David zgrabi svoje haljine i razdrije ih, a tako  i svi ljudi koji bijahu s njim. 
\par 12 I naricali su, plakali i  postili do večera za Šaulom i za njegovim sinom Jonatanom, za  Jahvinim narodom i za domom Izraelovim što izginuše od mača. 
\par 13 Potom David upita mladoga glasonošu: "Odakle si ti?"  A on odgovori: "Ja sam sin jednoga došljaka, Amalečanina." 
\par 14 Tada  mu reče David: "Kako se nisi bojao dići ruku da ubiješ pomazanika  Jahvina?" 
\par 15 I dozva David jednoga od momaka i zapovjedi mu:  "Dođi ovamo i smakni ga!" Udari ga momak i on umrije. 
\par 16 A David  mu još doviknu: "Tvoja krv na tvoju glavu! Tvoja su usta posvjedočila  protiv tebe kad si rekao: 'Ja sam ubio pomazanika Jahvina.'" 
\par 17 Tada David zapjeva ovu tužaljku za Šaulom i za njegovim  sinom Jonatanom. 
\par 18 Zapisana je u Knjizi Pravednikovoj da je  uče sinovi Judini. David reče: 
\par 19 "Oh, kako ti slava pade, Izraele, izginuše div-junaci na tvom visu! 
\par 20 O porazu vi u Gatu ne pričajte, aškelonskim ulicama ne glasite, da se kćeri ne vesele filistejske, mlade žene da ne kliču nevjerničke. 
\par 21 O Gilbojske gore klete, rosa na vas ne padala nit vas kiša s neba prala! Vaša polja ne vraćala rod za sjeme, jer kod vas je osramoćen štit junaka! Štit Šaulov nije bio uljem mazan, 
\par 22 nego krvlju ranjenika, mašću palih! Luk Jonate nikad nije promašio, mač Šaulov nikad bezuspješan bio! 
\par 23 Šaul i Jonata, ljupki, ponositi, ni živi se ne rastaše, ni u smrti! Od orlova bjehu brži, od lavova snagom jači! 
\par 24 Za Šaulom sad plačite, Izraelke, jer je u kras i u grimiz vas odijevo! Uz to zlatan nakit on je na ruho vam pričvršćivo. 
\par 25 Usred boja poginuše div-junaci! Smrt me tvoja, Jonatane, ožalosti! 
\par 26 Žao mi je tebe, brate, Jonatane! Kako li mi drag bijaše ti veoma! Ljubav tvoja bješe meni još od ženske čudesnija. 
\par 27 Oh, kako su izginuli div-junaci, i oružje bojno kako skršeno je!" 


\chapter{2}

\par 1 Poslije toga David upita Jahvu ovako: "Treba li da pođem u  koji Judin grad?" A Jahve mu odgovori: "Pođi!" David opet upita:  "Kamo da pođem?" A odgovor bješe: "U Hebron!" 
\par 2 Tako David ode  onamo, a s njim i njegove dvije žene, Ahinoama iz Jizreela i  Abigajila, Nabalova žena iz Karmela. 
\par 3 I ljudi koji bijahu s  Davidom odoše s njim, svaki sa svojom obitelji, i nastaniše se  u gradićima Hebrona. 
\par 4 Tada dođoše ljudi iz Jude i pomazaše  ondje Davida za kralja nad domom Judinim. Tada javiše Davidu da su ljudi iz Jabeša u Gileadu pokopali  Šaula. 
\par 5 Nato David posla glasnike k Jabešanima u Gileadu i  poruči im: "Budite blagoslovljeni od Jahve što ste izvršili to  djelo ljubavi prema svome gospodaru Šaulu i što ste ga pokopali! 
\par 6 Zato neka vam Jahve iskaže svoju ljubav i dobrotu, a i ja  ću vam učiniti dobro zato što ste tako radili. 
\par 7 A sada se ohrabrite  i budite junaci, jer je Šaul, vaš gospodar, poginuo, a mene je  Judin dom pomazao za svoga kralja." 
\par 8 Ali Abner, sin Nerov, vojvoda Šaulove vojske, bijaše uzeo  Šaulova sina Išbaala i doveo ga prijeko u Mahanajim. 
\par 9 Ondje  ga je postavio za kralja nad Gileadom, nad Ašeranima, nad Jizreelom, Efrajimom, Benjaminom i nad svim Izraelom. 
\par 10 Išbaalu, sinu  Šaulovu, bijaše četrdeset godina kad je postao kraljem nad Izraelom, a kraljevao je dvije godine. Samo je Judin dom pristao uz Davida. 
\par 11 A David je kraljevao u Hebronu nad Judinim domom sedam godina  i šest mjeseci. 
\par 12 Potom iziđe Abner, Nerov sin, s ljudima Šaulova sina  Išbaala iz Mahanajima prema Gibeonu. 
\par 13 Ali i Joab, Sarvijin  sin, iziđe s Davidovim ljudima iz Hebrona i srete se s njima  kod Gibeonskog jezera. Tu se zaustaviše, ovi s jedne strane jezera, a oni s druge strane. 
\par 14 Tada Abner reče Joabu: "Neka ustanu mladići i neka se  bore pred nama!" A Joab odgovori: "Neka ustanu!" 
\par 15 I ustadoše, pa ih izbrojiše: dvanaest od Benjamina za Išbaala, Šaulova sina, i dvanaest od Davidovih ljudi. 
\par 16 I svaki dohvati svoga protivnika  za glavu i zabode mu mač u bok, tako da su svi popadali zajedno.  Zato se to mjesto prozvalo Polje bokova, a leži kod Gibeona. 
\par 17 Potom se zametnu onoga dana vrlo žestoka bitka i Davidove  čete razbiše Abnera i Izraelce. 
\par 18 A bijahu ondje tri Sarvijina  sina: Joab, Abišaj i Asahel; a Asahel bijaše brz u trku kao gazela  u polju. 
\par 19 I jurnu Asahel u potjeru za Abnerom; u stopu je  slijedio Abnera ne skrećući ni desno ni lijevo. 
\par 20 Abner se  obazre i upita: "Jesi li to ti, Asahele?" A on odgovori: "Jesam." 
\par 21 A Abner mu reče: "Okreni se nadesno ili nalijevo pa zgrabi  jednoga od tih mladića i uzmi njegovu odoru!" Ali Asahel nije  htio da skrene od njega. 
\par 22 Abner opet reče Asahelu: "Ukloni  se od mene! Zašto da te sastavim sa zemljom? Kako bih onda još  smio doći na oči tvome bratu Joabu?" 
\par 23 Ali on nikako nije htio  da se ukloni. Zato ga Abner udari stražnjim krajem koplja u trbuh  tako da mu je koplje izašlo kroz leđa van: on ondje pade i umrije  na mjestu. I ustavljao se tko god je došao na ono mjesto gdje  je pao i umro Asahel. 
\par 24 Ali Joab i Abišaj nastaviše da gone Abnera, a kad je  sunce zašlo, stigoše do brežuljka Ame, koji leži istočno od doline, na putu prema Gebi. 
\par 25 Dotle se Benjaminovi sinovi skupiše  za Abnerom, sastaviše četu i stadoše na vrh brežuljka Ame. 
\par 26 Tada  Abner viknu Joabu: "Zar će nas dovijeka proždirati mač? Ne znaš  li da će to svršiti nesrećom? Kad ćeš napokon kazati svojim ljudima  da se okane gonjenja svoje braće?" 
\par 27 A Joab odvrati: "Tako  mi živog Jahve, da ti nisi progovorio, tek bi se sutra ujutro  ovi ljudi okanili gonjenja svoje braće." 
\par 28 Nato Joab zatrubi  u rog i sva vojska stade: prestadoše goniti Izraela i ne nastaviše  boja. 
\par 29 Abner i njegovi ljudi išli su kroz Arabu cijelu onu noć;  onda prijeđoše preko Jordana, nastaviše put cijelo jutro i stigoše  napokon u Mahanajim. 
\par 30 Kad je Joab odustao od potjere za Abnerom  i skupio svu vojsku, vidješe da između Davidovih ljudi nema devetnaestorice, i uz to Asahela. 
\par 31 A Davidovi su ljudi od Benjaminovih sinova, Abnerovih vojnika, pobili tri stotine i šezdeset ljudi. 
\par 32 Asahela  ponesoše i pokopaše u grobu njegova oca u Betlehemu. A Joab i  njegovi ljudi išli su svu noć i već se bijaše zadanilo kad stigoše  u Hebron. 


\chapter{3}

\par 1 Rat između Šaulove kuće i Davidove kuće potrajao je još dugo  vremena, ali je David sve više jačao, a Šaulova kuća postajala  sve slabija. 
\par 2 Davidu se rodiše sinovi u Hebronu. Prvenac mu je bio Amnon, od Ahinoame Jizreelke; 
\par 3 drugi mu je bio Kileab, od Abigajile, žene Nabalove iz Karmela; treći Abšalom, sin Maake, kćeri gešurskoga  kralja Tolmaja; 
\par 4 četvrti Adonija, sin Hagitin; peti Šefatja, sim Abitalin; 
\par 5 šesti Jitream, od Egle, Davidove žene. Ti se  Davidu rodiše u Hebronu. 
\par 6 Dok je trajao rat između Šaulove kuće i Davidove kuće, Abner je malo-pomalo prisvajao svu vlast u Šaulovoj kući. 
\par 7 A  u kući bijaše Šaulova inoča po imenu Rispa, kći Ajina: nju Abner  uze sebi. A Išbaal upita Abnera: "Zašto si se približio inoči  moga oca?" 
\par 8 Na te Išbaalove riječi Abner se razgnjevi i reče:  "Zar sam ja pasja glava u Judi? Do danas sam samo dobro činio  domu tvoga oca Šaula, njegovoj braći i njegovim prijateljima;  nisam dopustio da padneš u Davidove ruke, a ti me danas prekoravaš  zbog obične žene! 
\par 9 Neka Abneru Bog učini ovo zlo i neka mu  doda drugo ako ne izvršim kako se Jahve zakleo Davidu: 
\par 10 da  će oduzeti kraljevstvo Šaulovoj kući i da će utvrditi Davidov  prijesto nad Izraelom i nad Judom od Dana pa do Beer Šebe!" 
\par 11 Išbaal se ne usudi odgovoriti ni riječi Abneru jer ga  se bojaše. 
\par 12 Nato Abner posla glasnike k Davidu i poruči mu: "Čija  je zemlja?" Htio je reći: "Učini savez sa mnom i moja će ti ruka  pomoći da okupiš oko sebe svega Izraela." 
\par 13 David odgovori  Abneru: "Dobro! Učinit ću savez s tobom! Ali samo jedno tražim  od tebe: ne smiješ mi doći na oči ako ne dovedeš sa sobom Mikalu, Šaulovu kćer, kad dođeš da vidiš moje lice." 
\par 14 Ujedno posla  David glasnike i k Išbaalu, Šaulovu sinu, s porukom: "Vrati mi  moju ženu Mikalu, koju sam stekao stotinom filistejskih obrezaka." 
\par 15 Išbaal posla po nju i uze je od njezina muža Paltiela, Lajiševa  sina. 
\par 16 A njezin muž pođe s njom i pratio ju je plačući sve  do Bahurima. Tada mu Abner reče: "Hajde, vrati se sada kući!"  I on se vrati. 
\par 17 Abner je već bio razgovarao s Izraelovim starješinama  i rekao im: "Već odavna želite Davida za svoga kralja. 
\par 18 Učinite  to sada, jer je Jahve rekao o Davidu ovo: 'Rukom svoga sluge  Davida izbavit ću svoj narod Izraela iz ruke filistejske i iz  ruku svih njegovih neprijatelja.'" 
\par 19 Tako je Abner govorio  i Benjaminovim sinovima, a onda je otišao u Hebron da javi Davidu  sve što se svidjelo Izraelu i domu Benjaminovu. 
\par 20 Kad je Abner došao k Davidu u Hebron, i s njim dvadeset  ljudi, David priredi gozbu Abneru i ljudima koji bijahu s njim. 
\par 21 Tada Abner reče Davidu: "Hajdemo! Ja ću skupiti svega Izraela  oko gospodara moga kralja: oni će sklopiti s tobom savez i ti  ćeš kraljevati nad svim što budeš želio." David otpusti Abnera, koji ode u miru. 
\par 22 I gle, Davidovi se ljudi s Joabom upravo vraćali sa četovanja, noseći sa sobom bogat plijen, a Abner nije više bio kod Davida  u Hebronu, jer ga David bijaše otpustio te je on otišao u miru. 
\par 23 Kad stiže Joab i sva vojska što je išla s njim, javiše Joabu  da je Abner, Nerov sin, bio došao kralju i da ga je kralj otpustio  da ode u miru. 
\par 24 Tada Joab dođe kralju i reče mu: "Što si učinio?  Abner je došao k tebi, zašto si ga otpustio da ode u miru? 
\par 25 Zar  ne znaš Abnera, Nerova sina? Došao je da te prevari, da dozna  tvoje korake, da dozna sve što činiš!" 
\par 26 Potom izađe Joab od Davida i posla glasnike za Abnerom, koji ga vratiše, od studenca Sire, a David nije znao ništa o  tome. 
\par 27 Kad se Abner vratio u Hebron, odvede ga Joab u stranu  iza vrata, kao da želi s njim nesmetano govoriti, i ondje ga  smrtno rani u slabine da se osveti za krv svoga brata Asahela. 
\par 28 Kad je David to poslije čuo, reče: "Ja i moje kraljevstvo  nevini smo pred Jahvom dovijeka za krv Abnera, sina Nerova. 
\par 29 Neka  padne na Joabovu glavu i na sav njegov očinski dom! Nikad ne  ponestalo u Joabovu domu ljudi bolesnih od gnojenja ili od gube, ljudi koji se laćaju vretena ili padaju od mača, ljudi koji  nemaju kruha!" - 
\par 30 Joab i njegov brat Abišaj ubili su Abnera  jer je on pogubio njihova brata Asahela u boju kod Gibeona. - 
\par 31 Nato David reče Joabu i svoj vojsci koja je bila s njim:  "Razderite svoje haljine, obucite kostrijet i naričite za Abnerom!"  I kralj David pođe za nosilima. 
\par 32 Kad su ukopali Abnera u Hebronu, udari kralj u glasan plač na grobu Abnerovu, a plakao je i sav  narod. 
\par 33 Tada kralj ispjeva ovu tužaljku za Abnerom: "Zar morade umrijeti Abner kako umire luda? 
\par 34 Ruke tvoje ne bijahu vezane, noge tvoje ne bijahu okovane. Pao si kao što se pada od zlikovaca!" Tada sav narod još ljuće zaplaka za njim. 
\par 35 Nato pristupi sav narod nutkajući Davida da jede dok  je još dana, ali se David zakle ovako: "Neka mi Bog učini ovo  zlo i neka mi doda drugo zlo ako okusim kruha ili što drugo prije  zalaska sunca!" 
\par 36 Sav je narod to čuo, i bilo mu je po volji, kao što je narod i sve drugo odobravao što god je kralj činio. 
\par 37 Toga dana sav narod i sav Izrael spozna da kralj nije kriv  u umorstvu Abnera, sina Nerova. 
\par 38 Nato kralj reče svojim dvoranima: "Ne znate li da je  danas pao knez i velik čovjek u Izraelu? 
\par 39 Ali ja sam sada  još slab, iako sam pomazani kralj, a ovi ljudi, Sarvijini sinovi, jači su od mene. Neka Jahve plati zločincu po njegovoj zloći!" 


\chapter{4}

\par 1 Kad je Šaulov sin Išbaal čuo da je poginuo Abner u Hebronu, klonuše mu ruke i sav se Izrael zaprepasti. 
\par 2 A Šaulov sin  Išbaal imaše dvojicu vođa svojih četa; jedan se zvao Baana, a  drugi Rekab; bili su sinovi Rimona Beeroćanina iz Benjaminova  plemena, jer se Beerot pribraja k Benjaminu. 
\par 3 A Beeroćani bijahu  pobjegli u Gitajim, gdje su ostali kao došljaci do današnjeg  dana. 
\par 4 Šaulov sin Jonatan imao je sina hroma na obje noge. Njemu  je bilo pet godina kad je iz Jizreela došao glas o Šaulovoj i  Jonatanovoj pogibiji. Njegova ga dadilja uze i pobježe, ali u  brzini bijega dijete pade i osta hromo. Ime mu bijaše Meribaal. 
\par 5 Sinovi Rimona Beeroćanina, Rekab i Baana, digoše se i  dođoše za najveće dnevne vrućine Išbaalu do kuće, a on upravo  spavaše podnevni počinak. 
\par 6 A vratarica, čisteći pšenicu, bijaše  zadrijemala te je spavala. Rekab i njegov brat Baana prošuljaše  se kraj nje. 
\par 7 Kad su ušli u kuću, on je ležao na postelji u  svojoj spavaonici. Oni ga ubiše, odsjekoše mu glavu i uzeše je  i cijelu su onu noć išli putem kroz Arabu. 
\par 8 Glavu Išbaalovu  donesoše Davidu u Hebron i rekoše kralju: "Evo glave Išbaala, sina Šaulova, tvoga neprijatelja koji ti je radio o glavi. Jahve  je danas krvavo osvetio moga gospodara i kralja na Šaulu i njegovu  rodu." 
\par 9 Ali David odvrati Rekabu i njegovu bratu Baani, sinovima  Rimona iz Beerota, i reče im: "Tako mi živog Jahve koje me izbavio  iz svake nevolje! 
\par 10 Onaj koji mi je javio da je poginuo Šaul  mislio je da mi javlja radosnu vijest, a ja sam ga uhvatio i  pogubio u Siklagu da mu platim za njegovu dobru vijest! 
\par 11 Što  ću tek učiniti sa zlikovcima koji su ubili poštena čovjeka u  njegovoj kući, na njegovoj postelji! Zar da ne tražim od vas  račun za njegovu krv i da vas ne istrijebim sa zemlje?" 
\par 12 Nato  David zapovjedi vojnicima te ih pogubiše. Potom im odsjekoše  ruke i noge i objesiše ih kod jezera u Hebronu. Išbaalovu glavu  uzeše i pokopaše u Abnerovu grobu u Hebronu. 


\chapter{5}

\par 1 Tada se sabraše sva izraelska plemena k Davidu u Hebron i rekoše:  "Evo, mi smo od tvoje kosti i od tvoga mesa. 
\par 2 Još prije, dok  je još Šaul bio kralj nad nama, ti si upravljao svim pokretima  Izraela, a Jahve ti je rekao: 'Ti ćeš pasti moj izraelski narod  i ti ćeš biti knez nad Izraelom!'" 
\par 3 Tako dođoše sve izraelske  starješine kralju u Hebron, a kralj David sklopi s njima savez  u Hebronu pred Jahvom; i pomazaše Davida za kralja nad Izraelom. 
\par 4 Trideset je godina bilo Davidu kad je postao kralj, a  kraljevao je četrdeset godina. 
\par 5 U Hebronu je kraljevao nad  Judom sedam godina i šest mjeseci, a u Jeruzalemu kraljevaše  trideset i tri godine nad svim Izraelom i nad Judom. 
\par 6 David krene s ljudima na Jeruzalem protiv Jebusejaca,  koji su živjeli u onoj zemlji. Ali oni poručiše Davidu: "Nećeš  ući ovamo! Slijepci će te i kljasti odbiti!" (To je imalo značiti:  David neće ući ovamo.) 
\par 7 Ipak David osvoji Sionsku tvrđavu,  to jest Davidov grad. 
\par 8 Onoga dana reče David: "Tko god pobije  Jebusejce i popne se kroz prorov ..." A kljaste i slijepe mrzi  David iz sve duše. (Stoga se kaže: Slijepci i kljasti neka ne  ulaze u Hram.) 
\par 9 David se nastani u tvrđavi i prozva je Davidov  grad. Tada David podiže zid unaokolo od Mila pa unutra. 
\par 10 David  je postajao sve silniji, jer Jahve, Bog nad vojskama, bijaše  s njim. 
\par 11 Tirski kralj Hiram posla k Davidu izaslanstvo i cedrova  drveta, tesara i zidara, koji sagradiše dvor Davidu. 
\par 12 Tada  David spozna da ga je Jahve potvrdio za kralja nad Izraelom i  da je vrlo uzvisio njegovo kraljevstvo radi svojega izraelskog  naroda. 
\par 13 Po dolasku iz Hebrona David uze još inoča i žena iz Jeruzalema;  i rodi se Davidu još sinova i kćeri. 
\par 14 Evo imena djece koja  mu se rodiše u Jeruzalemu: Šamua, Šobab, Natan, Salomon, 
\par 15 Jibhar, Elišua, Nefeg, Jafija, 
\par 16 Elišama, Beeljada i Elifelet. 
\par 17 Kad su Filistejci čuli da su Davida pomazali za kralja  nad Izraelom, iziđoše svi da se dočepaju Davida. Čuvši to, David  siđe u svoj zaklon. 
\par 18 Filistejci dođoše i raširiše se po Refaimskoj  dolini. 
\par 19 Tada David upita Jahvu: "Mogu li napasti Filistejce?  Hoćeš li ih predati meni u ruke?" Jahve odgovori Davidu: "Napadni!  Predat ću Filistejce tebi u ruke!" 
\par 20 Tada David dođe u Baal  Perasim i ondje ih pobi. David reče: "Jahve je preda mnom prodro  među moje neprijatelje kao što voda prodire." Stoga se ono mjesto  prozvalo Baal Perasim. 
\par 21 Ostavili su ondje svoje bogove; a  David i njegovi ljudi odnesoše ih. 
\par 22 Filistejci opet dođoše i raširiše se po Refaimskoj dolini. 
\par 23 David opet upita Jahvu, a on mu odgovori: "Ne idi pred njih, nego im zađi za leđa i navali na njih s protivne strane Bekaima. 
\par 24 Kad začuješ topot koraka po bekaimskim vrhovima, onda se  požuri, jer će tada Jahve ići pred tobom da pobije filistejsku  vojsku." 
\par 25 David učini kako mu je zapovjedio Jahve i pobi Filistejce  od Gibeona sve do ulaza u Gezer. 


\chapter{6}

\par 1 Jednoga dana David opet skupi svu izabranu momčad u Izraelu, trideset tisuća ljudi. 
\par 2 Zatim David i sva vojska što je bila  s njim krenu na put i odoše u Baalu Judinu da odande donesu Kovčeg  Božji, što nosi ime Jahve Sebaota koji stoluje nad kerubinima. 
\par 3 Kovčeg Božji metnuše na nova kola, iznijevši ga iz kuće Abinadabove, koja je stajala na brežuljku. Uza i Ahjo, Abinadabovi sinovi, pratili su kola. 
\par 4 Uza je stupao kraj Kovčega Božjeg, a Ahjo  išao pred njim. 
\par 5 David i sav dom Izraelov igrahu pred Jahvom  iz sve snage pjevajući uza zvuke citara, harfa, bubnjeva, udaraljki  i cimbala. 
\par 6 Kad su došli do Nakonova gumna, posegnu Uza rukom  za Kovčegom Božjim da ga pridrži jer ga volovi umalo ne prevrnuše. 
\par 7 Ali se Jahve razgnjevio na Uzu: Bog ga na mjestu udari za  taj prijestup, tako da je umro ondje, kraj Kovčega Božjega. 
\par 8 Davidu  bijaše žao što je Jahve onako udario Uzu, i on prozva ono mjesto  Peres Uza, kako se zove i dan-danas. 
\par 9 Toga se dana David uplaši Jahve i reče u sebi: "Kako bi  mogao doći k meni Kovčeg Jahvin?" 
\par 10 Zato David ne htjede dovesti  Kovčeg Jahvin k sebi, u Davidov grad, nego ga otpremi u kuću  Obed-Edoma iz Gata. 
\par 11 I ostade Jahvin Kovčeg u kući Obed-Edomovoj  u Gatu tri mjeseca i Jahve blagoslovi Obed-Edoma i svu njegovu  obitelj. 
\par 12 Kad su kralju javili da je Jahve blagoslovio Obed-Edomovu  obitelj i sav njegov posjed zbog Kovčega Božjeg, ode David i  ponese Kovčeg Božji iz Obed-Edomove kuće gore u Davidov grad  s velikim veseljem. 
\par 13 Tek što su nosioci Kovčega Božjeg pokročili  šest koraka, David žrtvova vola i tovna ovna. 
\par 14 David je igrao  iz sve snage pred Jahvom, a bio je ogrnut samo lanenim oplećkom. 
\par 15 Tako su David i sav Izraelov dom nosili gore Kovčeg Jahvin  kličući i trubeći u rog. 
\par 16 A kad je Kovčeg Jahvin ulazio u Davidov grad, Šaulova  je kći Mikala gledala kroz prozor i vidjela kralja Davida kako  skače i vrti se pred Jahvom i prezre ga ona u svome srcu. 
\par 17 Tada  unesoše Kovčeg Jahvin i postaviše ga usred šatora koji mu bijaše  razapeo David. Onda David prinese pred Jahvom paljenice i pričesnice. 
\par 18 Pošto je prinio paljenice i pričesnice, David blagoslovi  narod imenom Jahve Sebaota. 
\par 19 Potom razdijeli među sav narod, među sve mnoštvo Izraelovo, ljudima i ženama, svakome po jedan  kruh, komad mesa i kolač od suhoga grožđa. Zatim se raziđe sav  narod, svaki svojoj kući. 
\par 20 Kad se David vratio kući da blagoslovi svoju obitelj, Šaulova kći Mikala iziđe u susret Davidu i reče mu: "Kako se  časno danas ponio Izraelov kralj kad se otkrio pred očima sluškinja  slugu svojih kao što se otkriva prost čovjek!" 
\par 21 Ali David  odgovori Mikali: "Pred Jahvom ja igram! Tako mi živoga Jahve, koji me izabrao mjesto tvog oca i mjesto svega njegova doma  da me postavi za kneza nad Izraelom, narodom Jahvinim: pred Jahvom  ću igrati! 
\par 22 I još ću se dublje poniziti. Bit ću neznatan u  tvojim očima, ali pred sluškinjama o kojima govoriš, pred njima  ću biti u časti." 
\par 23 A Mikala, Šaulova kći, ne imade poroda  do dana svoje smrti. 


\chapter{7}

\par 1 Kad se David nastanio u svojem dvoru i kad mu je Jahve pribavio  mir od svih njegovih neprijatelja unaokolo, 
\par 2 reče kralj proroku  Natanu: "Pogledaj! Ja, evo, stojim u dvoru od cedrovine, a Kovčeg  Božji stoji pod šatorom." 
\par 3 A Natan odgovori kralju: "Idi i  čini sve što ti je na srcu jer je Jahve s tobom." 
\par 4 Ali još iste noći dođe Natanu ova Jahvina riječ: 
\par 5 "Idi i reci mome sluzi Davidu: Ovako govori Jahve: 'Zar  ćeš mi ti sagraditi kuću da u njoj prebivam? 
\par 6 Nisam nikad prebivao  u kući otkako sam izveo iz Egipta sinove Izraelove pa do današnjega  dana, nego sam bio lutalac pod šatorom i u prebivalištu. 
\par 7 Dok  sam hodio sa svim Izraelovim sinovima, jesam li ijednu riječ  rekao nekomu od Izraelovih sudaca kojima sam zapovjedio da budu  pastiri mojem narodu izraelskom i kazao: 'Zašto mi ne sagradite  kuću od cedrovine?' 
\par 8 Zato sad ovo reci mome sluzi Davidu: Ovako  govori Jahve nad vojskama: Ja sam te doveo s pašnjaka, od ovaca  i koza, da budeš knez nad mojim izraelskim narodom. 
\par 9 Bio sam  s tobom kuda si god išao, iskorijenio sam sve tvoje neprijatelje  pred tobom. Ja ću ti pribaviti veliko ime, kao što je velikaško  ime na zemlji. 
\par 10 Odredit ću prebivalište svojem izraelskom  narodu, posadit ću ga da živi na svojem mjestu i da ne luta više  naokolo, niti da ga zlikovci muče kao prije, 
\par 11 onda kad sam  odredio suce nad svojim izraelskim narodom. Ja ću mu pribaviti  mir od svih njegovih neprijatelja. Jahve će te učiniti velikim.  Jahve će ti podići dom. 
\par 12 I kad se ispune tvoji dani i ti počineš  kod svojih otaca, podići ću tvoga potomka nakon tebe, koji će  se roditi od tvoga tijela, i utvrdit ću njegovo kraljevstvo. 
\par 13 On će sagraditi dom imenu mojem, a ja ću utvrditi njegovo  prijestolje zauvijek. 
\par 14 Ja ću njemu biti otac, a on će meni  biti sin: ako učini što zlo, kaznit ću ga ljudskom šibom i udarcima  kako ih zadaju sinovi ljudski. 
\par 15 Ali svoje naklonosti neću  odvratiti od njega, kao što sam je odvratio od Šaula koga sam  uklonio ispred tebe. 
\par 16 Tvoja će kuća i tvoje kraljevstvo trajati  dovijeka preda mnom, tvoje će prijestolje čvrsto stajati zasvagda.'" 
\par 17 Natan prenese Davidu sve te riječi i cijelo viđenje. 
\par 18 Nato kralj David uđe u šator i stade pred Jahvom i pomoli  se:  "Tko sam ja, Gospode Jahve, i što je moj dom te si me doveo  dovde? 
\par 19 Pa i to je još premalo u tvojim očima, Gospode Jahve, te daješ svoja obećanja kući svoga sluge za daleku budućnost  i gledaš na me kao na ugledna čovjeka! 
\par 20 Ali što bi ti David  još mogao kazati, kad ti sam poznaješ svoga slugu, Gospode Jahve! 
\par 21 Radi svoje riječi i po svome srcu učinio si sve ovo veliko  djelo, obznanivši ove veličajnosti. 
\par 22 Zato si velik, Gospode  Jahve; nema takvoga kakav si ti i nema Boga osim tebe, po svemu  što smo ušima svojim čuli. 
\par 23 Postoji li ijedan narod na zemlji  kao tvoj izraelski narod radi kojega je Bog išao da ga izbavi  sebi za narod da tako stečeš sebi ime velikim i strašnim čudesima, izgoneći krivobožačka plemena pred svojim narodom koji si otkupio  iz Egipta? 
\par 24 Tako si učinio svoj izraelski narod svojim narodom  zauvijek, a ti si mu, Jahve, postao Bogom. 
\par 25 Zato sada, Gospode  Jahve, ispuni zauvijek obećanje koje si dao svome sluzi i njegovu  domu i učini kako si obrekao. 
\par 26 Neka se veliča tvoje ime zauvijek  i neka se govori: Jahve nad vojskama jest Bog Izraelov, a dom  sluge tvoga Davida neka stoji čvrsto pred tobom. 
\par 27 Jer si ti, Jahve nad vojskama, Bože Izraelov, objavio svome sluzi ovo:  'Ja ću ti podići dom.' Zato je tvoj sluga smogao hrabrosti da  ti se pomoli ovom molitvom. 
\par 28 Uistinu, Gospode Jahve, ti si  Bog, tvoje su riječi istinite i ti daješ ovo lijepo obećanje  svome sluzi. 
\par 29 Udostoj se sada blagosloviti dom svoga sluge  da ostane dovijeka pred tobom. Jer kad ti, Gospode Jahve, obrekneš  i blagosloviš, kuća tvoga sluge bit će blagoslovljena zasvagda." 


\chapter{8}

\par 1 Poslije toga David porazi Filistejce i pokori ih te ote Gat  s njegovim selima iz filistejskih ruku. 
\par 2 Porazi i Moapce i  izmjeri ih uzicom polegavši ih po zemlji: dvije uzice odmjeri  onih koje treba pogubiti, a jednu punu uzicu onih koje treba  ostaviti na životu. Tako Moapci postadoše Davidovi podanici koji  su mu donosili danak. 
\par 3 David je porazio i Hadadezera, Rehobova sina, sopskoga  kralja, kad je izišao da proširi svoju vlast do Rijeke. 
\par 4 David  zarobi od njega tisuću i sedam stotina konjanika i dvadeset tisuća  pješaka; ispresijecao je petne žile svim konjima od bojnih kola;  ostavio ih je samo stotinu. 
\par 5 Damaščanski su Aramejci došli  u pomoć Hadadezeru, sopskome kralju, ali je David pobio među  Aramejcima dvadeset i dvije tisuće ljudi. 
\par 6 Postavio je namjesnike  u Damaščanskom Aramu. Tako Aramejci postadoše Davidovi podanici  i moradoše mu plaćati danak. Jahve je davao pobjedu Davidu kuda  je god išao. 
\par 7 David zaplijeni zlatne štitove što ih imahu Hadadezerove  sluge i donese ih u Jeruzalem. 
\par 8 Iz Tebaha i iz Berotaja, Hadadezerovih  gradova, donese kralj David silni tuč. 
\par 9 Kad je čuo hamatski kralj Tou da je David porazio svu  Hadadezerovu vojsku, 
\par 10 poslao je svoga sina Hadorama kralju  Davidu da ga pozdravi i da mu čestita što je vojevao protiv Hadadezera  i porazio ga, jer je Hadadezer bio u ratu s Touom; Hadoram donese  srebrnih, zlatnih i tučanih predmeta. 
\par 11 I njih kralj David  posveti Jahvi sa srebrom i zlatom što ga bijaše uzeo od svih  naroda koje je pokorio: 
\par 12 od Aramaca, Moabaca, Amonaca, Filistejaca  i od Amalečana te od plijena Hadadezera, Rehobova sina, kralja  Sobe. 
\par 13 David steče novu slavu kad je na povratku porazio Edomce, u Slanoj dolini, osamnaest tisuća njih. 
\par 14 I postavi upravitelje  u Edomu, i svi Edomci postadoše podanici Davidovi. I kuda je  god David išao, Jahve mu davaše pobjedu. 
\par 15 David kraljevaše nad svim Izraelom, čineći pravo i pravicu  svemu svome narodu. 
\par 16 Joab, sin Sarvijin, zapovijedaše vojskom, a Jošafat, sin Ahiludov, bijaše ljetopisac. 
\par 17 Sadok, sin Ahitubov, i Ebjatar, sin Ahimelekov, bijahu svećenici; Seraja bijaše državni  pisar; 
\par 18 Benaja, sin Jojadin, zapovijedaše Kerećanima i Pelećanima;  Davidovi sinovi bijahu namjesnici. 


\chapter{9}

\par 1 Jednoga dana upita David: "Ima li još koji preživjeli od Šaulove  kuće da mu učinim milost zbog Jonatana?" 
\par 2 A bijaše u Šaulovoj  kući sluga po imenu Siba: njega dozvaše pred Davida i kralj ga  zapita: "Jesi li ti Siba?" A on odgovori: "Jesam, tvoj sluga!" 
\par 3 A kralj nastavi: "Zar nema više nikoga od Šaulove kuće da  mu iskažem milost kao što je Božja milost?" A Siba odgovori kralju:  "Ima još Jonatanov sin koji je hrom na obje noge." 
\par 4 Kralj ga  upita: "Gdje je on?" A Siba odgovori kralju: "Eno ga u kući Makira, sina Amielova, u Lo Debaru." 
\par 5 Tada kralj David posla po njega u kuću Makira, sina Amielova, iz Lo Debara. 
\par 6 Kad je Meribaal, sin Jonatana, sina Šaulova, došao k Davidu, pade ničice i pokloni se. A David reče: "Meribaale!" On odgovori:  "Evo tvoga sluge!" 
\par 7 A David mu reče: "Ne boj se jer ti želim  iskazati milost zbog tvoga oca Jonatana. Vratit ću ti sva polja  tvoga djeda Šaula, a ti ćeš svagda jesti kruh za mojim stolom." 
\par 8 Meribaal se pokloni i reče: "Što je tvoj sluga te iskazuješ  milost mrtvome psu kao što sam ja?" 
\par 9 Potom kralj dozva Sibu, Šaulova slugu, i reče mu: "Sve  što je pripadalo Šaulu i njegovoj kući, sve to dajem sinu tvoga  gospodara. 
\par 10 Ti ćeš mu sa svojim sinovima i sa svojim slugama  obrađivati zemlju, od nje ćeš skupljati žetvu da obitelj tvoga  gospodara ima kruha; a Meribaal, sin tvoga gospodara, jest će  svagda za mojim stolom." A Siba imaše petnaest sinova i dvadeset  slugu. 
\par 11 Siba odgovori kralju: "Tvoj će sluga učiniti sve što  je moj gospodar i kralj zapovjedio svome sluzi." Meribaal je, dakle, jeo za Davidovim stolom kao jedan između  kraljevih sinova. 
\par 12 Meribaal je imao maloga sina po imenu Mika.  A svi koji su živjeli u Sibinoj kući bijahu u službi Meribaala. 
\par 13 A Meribaal je boravio u Jeruzalemu, jer je uvijek jeo za  kraljevim stolom. Bio je hrom na obje noge. 


\chapter{10}

\par 1 Poslije toga umrije Nahaš, kralj Amonaca, a zakralji se njegov  sin Hanun mjesto njega. 
\par 2 A David reče u sebi: "Želio bih iskazati  ljubav Nahaševu sinu Hanunu, kao što je njegov otac iskazao meni."  Zato David posla svoje sluge da mu izraze sućut zbog njegova  oca. Ali kad su Davidove sluge došle u zemlju Amonaca, 
\par 3 rekoše  knezovi Amonaca svome gospodaru Hanunu: "Zar misliš da je David  poslao ljude da ti izraze sućut zato što bi htio iskazati čast  tvome ocu? Nije li možda zato David poslao svoje ljude k tebi  da razvide grad da bi doznao njegovu obranu i potom ga oborio?" 
\par 4 Tada Hanun pograbi Davidove sluge, obrija im pola brade i  skrati im haljine dopola, sve do zadnjice, i posla ih natrag. 
\par 5 Kad su to javili Davidu, posla on čovjeka pred njih, jer su  ti ljudi bili teško osramoćeni, i poruči im: "Ostanite u Jerihonu  dok vam ne naraste brada, pa se onda vratite!" 
\par 6 Tada Amonci uvidješe da su se omrazili s Davidom; zato  Amonci poslaše glasnike da za plaću unajme Aramejce iz Bet Rehoba  i Aramejce iz Sobe, dvadeset tisuća pješaka, zatim kralja Maake, tisuću ljudi, i ljude iz Toba, dvanaest tisuća vojnika. 
\par 7 Kad  je David to čuo, posla Joaba s vojskom i izabranim junacima. 
\par 8 Amonci iziđoše i svrstaše se u bojni red pred gradskim vratima, dok su Aramejci iz Sobe i iz Rehoba i ljudi iz Toba i iz Maake  stajali zasebno na polju. 
\par 9 Vidjevši postavljene bojne redove  prema sebi sprijeda i straga, probra Joab najvrsnije među Izraelcima  i svrsta ih prema Aramejcima. 
\par 10 Ostalu vojsku predade bratu  Abišaju da je svrsta prema Amoncima. 
\par 11 I reče mu: "Ako Aramejci  budu jači od mene, onda ti meni priskoči u pomoć; ako Amonci  budu jači od tebe, ja ću tebi pohrliti u pomoć. 
\par 12 Budi hrabar  i junački se držimo radi naroda i radi gradova svoga Boga; a  Jahve neka učini što je dobro u njegovim očima." 
\par 13 Tada se  Joab i vojska koja je bila s njim počeše primicati da udare na  Aramejce, ali oni pobjegoše pred njima. 
\par 14 Kad su Amonci vidjeli  da su Aramejci pobjegli, umakoše i oni ispred Abišaja i povukoše  se u grad. Tada Joab odustane od rata protiv Amonaca i vrati  se u Jeruzalem. 
\par 15 Kad su Aramejci vidjeli gdje su ih Izraelci razbili,  sabraše ponovo svoje čete. 
\par 16 Hadadezer posla glasnike i sabra  Aramejce što su s one strane rijeke. Ovi dođoše u Helam pod vodstvom  Šobaka, vojvode Hadadezerove vojske. 
\par 17 Pošto su to javili Davidu, on skupi sve Izraelce i, prešavši preko Jordana, dođe u Helam.  Aramejci se svrstaše protiv Davida i zametnuše s njime boj. 
\par 18 Ali  Aramejci udariše u bijeg ispred Izraelaca i David im pobi sedam  stotina konja od bojnih kola i četrdeset tisuća pješaka; pogubi  i njihova vojvodu Šobaka te je ondje umro. 
\par 19 A kad svi kraljevi, Hadadezerovi vazali, vidješe da ih je razbio Izrael, sklopiše  mir s Izraelom i počeše mu služiti. A Aramejci se više nisu usuđivali  pomagati Amoncima. 


\chapter{11}

\par 1 U početku slijedeće godine, u doba kad kraljevi izlaze u rat, posla David Joaba i s njim svoje ljude i svega Izraela: oni  pobiše Amonce i podsjedoše Rabu. A David osta u Jeruzalemu. 
\par 2 A jednoga dana predveče usta David sa svoje postelje i  prošeta se po krovu svoje palače. Opazi s krova ženu gdje se  kupa. Ta žena bijaše izvanredno lijepa. 
\par 3 David se propita za  tu ženu i rekoše mu: "Pa to je Bat-Šeba, kći Eliamova i žena  Urije Hetita!" 
\par 4 Nato David posla glasnika da je dovedu k njemu.  Kad je došla, leže on s njom, upravo kad se bila očistila od  svoje nečistoće. Zatim se ona vrati svojoj kući. 
\par 5 Žena zatrudnje  te poruči Davidu: "Trudna sam!" 
\par 6 Tada David posla poruku Joabu: "Pošalji k meni Uriju Hetita!"  I Joab posla Uriju k Davidu. 
\par 7 Kad je Urija došao k njemu, zapita  ga David kako je Joab, kako je vojska i kako napreduje rat. 
\par 8 Potom  David reče Uriji: "Siđi u svoju kuću i operi svoje noge!" Urija  iziđe iz kraljeva dvora, a za njim ponesoše dar s kraljeva stola. 
\par 9 Ali Urija osta da spava pred vratima kraljeva dvora sa stražarima  svoga gospodara i ne ode svojoj kući. 
\par 10 Javiše to Davidu govoreći: "Urija nije otišao svojoj  kući!" Tada David upita Uriju: "Zar nisi došao s puta? Zašto  ne ideš svojoj kući?" 
\par 11 A Urija odgovori Davidu: "Kovčeg, Izrael  i Juda borave pod šatorima, moj gospodar Joab i straža moga gospodara  borave na otvorenu polju, a ja da uđem u svoju kuću da jedem  i da pijem i da spavam sa svojom ženom? Živoga mi Jahve, i tako  mi tvoga života, zaista neću učiniti nešto takvo!" 
\par 12 Tada David  reče Uriji: "Ostani još danas ovdje, a sutra ću te otpustiti."  Tako Urija osta u Jeruzalemu onaj dan. 
\par 13 Sutradan David pozva  Uriju da jede i da pije pred njim i on ga opi. A uvečer Urija  iziđe i leže na svoju postelju sa stražama svoga gospodara, ali  svojoj kući nije otišao. 
\par 14 Ujutro David napisa pismo Joabu i posla ga po Uriji. 
\par 15 A u tom pismu pisao je ovako: "Postavite Uriju naprijed,  gdje je najžešći boj, pa uzmaknite iza njega: neka bude pogođen  i neka pogine!" 
\par 16 Zato Joab, opsjedajući grad, postavi Uriju  na mjesto gdje je znao da stoje najhrabriji ratnici. 
\par 17 Kad  su onda građani provalili van i pobili se s Joabom, pade nekoliko  od njegove vojske, od Davidovih ljudi, a pogibe i Urija Hetit. 
\par 18 Potom Joab posla čovjeka i javi Davidu sve što se dogodilo  u boju. 
\par 19 I zapovjedi glasniku ovako: "Kad pripovjediš kralju  sve što se dogodilo u boju, 
\par 20 možda će se kralj razljutiti  pa ti kazati: 'Zašto ste se primakli tako blizu gradu da navalite?  Zar niste znali da se obično izmeću strijele sa zida? 
\par 21 Tko  je ubio Abimeleka, sina Jerubaalova? Nije li jedna žena bacila  na njega mlinski kamen, ozgo sa zida, te je poginuo u Tebesu?  Zašto ste se primakli tako blizu zidu?' Ako ti tako kaže, a ti  mu reci: 'Poginuo je i tvoj sluga Urija Hetit.'" 
\par 22 Glasnik krenu na put, dođe k Davidu i pripovjedi mu sve  što mu je naložio Joab. A David planu gnjevom na Joaba i reče  glasniku: "Zašto ste se primakli tako blizu zidu? Tko je ubio  Abimeleka, sina Jerubaalova? Nije li jedna žena bacila na njega  mlinski kamen, ozgo sa zida, te je poginuo u Tebesu? Zašto ste  se primakli tako blizu zidu?" 
\par 23 Glasnik odgovori Davidu: "Ti  su ljudi silovito udarali na nas i izašli su protiv nas na otvoreno  polje. Mi smo ih potisnuli natrag do gradskih vrata, 
\par 24 ali  su strijelci sa zida stali izmetati strijele na tvoje ljude te  ih je poginulo nekoliko između kraljevih slugu; tako je poginuo  i tvoj sluga Urija Hetit." 
\par 25 Tada David reče glasniku: "Ovako reci Joabu: 'Nemoj to  uzimati toliko k srcu, jer mač proždire sad ovoga, sad onoga.  Udaraj još jače na grad i obori ga!' Tako ćeš mu vratiti srčanost!" 
\par 26 Kad je Urijina žena čula da je poginuo njezin muž Urija,  žalila je za svojim mužem. 
\par 27 A kad je prošlo vrijeme žalosti, posla David po nju i uze je u svoj dvor, i ona mu posta ženom.  I rodi mu sina. Ali djelo koje učini David bijaše zlo u očima  Jahvinim. 


\chapter{12}

\par 1 Jahve posla proroka Natana k Davidu. On uđe k njemu i reče  mu:  "U nekom gradu živjela dva čovjeka, jedan bogat, a drugi  siromašan. 
\par 2 Bogati imaše ovaca i goveda u obilju. 
\par 3 A siromah  nemaše ništa, osim jedne jedine ovčice koju bijaše kupio. Hranio  ju je i ona je rasla kraj njega i s njegovom djecom; jela je  od njegova zalogaja, pila iz njegove čaše; spavala ja na njegovu  krilu: bila mu je kao kći. 
\par 4 I dođe putnik k bogatom čovjeku, a njemu bilo žao uzeti od svojih ovaca ili goveda da zgotovi  gostu koji mu je došao. On ukrade ovčicu siromaha i zgotovi je  za svog pohodnika." 
\par 5 Tada David planu žestokim gnjevom na toga čovjeka i reče  Natanu: "Tako mi živog Jahve, smrt je zaslužio čovjek koji je  to učinio! 
\par 6 Četverostruko će naknaditi ovcu zato što je učinio  to djelo i što nije znao milosrđa!" 
\par 7 Tada Natan reče Davidu: "Ti si taj čovjek! Ovako govori  Jahve, Bog Izraelov: 'Ja sam te pomazao za kralja nad Izraelom, ja sam te izbavio iz Šaulove ruke. 
\par 8 Predao sam ti kuću tvoga  gospodara, položio sam žene tvoga gospodara na tvoje krilo, dao  sam ti dom Izraelov i dom Judin; a ako to nije dosta, dodat ću  ti još ovo ili ono. 
\par 9 Zašto si prezreo Jahvu i učinio ono što  je zlo u njegovim očima? Ubio si mačem Uriju Hetita, a njegovu  si ženu uzeo za svoju ženu. Jest, njega si ubio mačem Amonaca. 
\par 10 Zato se neće nikada više okrenuti mač od tvoga doma, jer  si me prezreo i jer si uzeo ženu Urije Hetita da ti bude žena.' 
\par 11 Ovako govori Jahve: 'Evo ja ću podići na te zlo iz tvoga  doma. Uzet ću tvoje žene ispred tvojih očiju i dat ću ih tvome  bližnjemu, koji će spavati s tvojim ženama na vidiku ovome suncu. 
\par 12 Ti si doduše radio tajno, ali ja ću ovu prijetnju izvršiti  pred svim Izraelom i pred ovim suncem!'" 
\par 13 Tada David reče Natanu: "Sagriješio sam protiv Jahve!"  A Natan odvrati Davidu: "Jahve ti oprašta tvoj grijeh: nećeš  umrijeti. 
\par 14 Ali jer si tim djelom prezreo Jahvu, neminovno  će umrijeti dijete koje ti se rodilo!" 
\par 15 Potom Natan ode svojoj kući.  A Jahve udari dijete koje je Urijina žena rodila Davidu i  ono se teško razbolje. 
\par 16 David se molitvom obrati Bogu za dijete:  postio je, vraćao se kući i ležao preko noći na goloj zemlji, pokriven vrećom. 
\par 17 A starješine njegova doma stajahu oko njega  da ga podignu sa zemlje, ali on ne htjede i ne okusi s njima  nikakva jela. 
\par 18 A sedmi dan umrije dijete. Davidovi dvorani  ne usudiše se javiti mu da je dijete umrlo. Jer mišljahu: "Dok  je dijete bilo živo, govorili smo mu, a on nas nije htio slušati.  A kako ćemo mu kazati da je dijete umrlo? Učinit će zlo!" 
\par 19 A David opazi da njegovi dvorani šapću među sobom i on  shvati da je dijete umrlo. I upita David svoje dvorane: "Je li  dijete umrlo?" A oni odgovoriše: "Umrlo je." 
\par 20 Tada David usta sa zemlje, okupa se, pomaza se i preobuče  se u druge haljine. Zatim uđe u Dom Jahvin i pokloni se. Vrativši  se potom svojoj kući, zatraži da mu dadu jela; i jeo je. 
\par 21 A  njegovi dvorani upitaše ga: "Što to radiš? Dok je dijete bilo  živo, postio si i plakao; a sada, kad je dijete umrlo, ustaješ  i jedeš!" 
\par 22 A on odgovori: "Dok je dijete bilo živo, postio  sam i plakao jer sam mislio: 'Tko zna? Jahve će se možda smilovati  na me i dijete će ostati živo!' 
\par 23 A sada, kad je umrlo, čemu  da postim? Mogu li ga vratiti? Ja ću otići k njemu, ali se ono  neće vratiti k meni!" 
\par 24 Potom David utješi svoju ženu Bat-Šebu. Dođe k njoj i  leže s njom. Ona zatrudnje i rodi sina komu nadjenu ime Salomon.  Jahve ga zamilova 
\par 25 i objavi to po proroku Natanu. Ovaj ga  nazva imenom Jedidja, po riječi Jahvinoj. 
\par 26 Joab navali na Rabu sinova Amonovih i osvoji kraljevski  grad. 
\par 27 Tada Joab posla glasnika k Davidu s porukom: "Ja sam  navalio na Rabu i osvojio grad uz vodu. 
\par 28 Sada ti saberi ostalu  vojsku, opkoli grad i osvoji ga, da ne bih ja osvojio grada i  dao mu svoje ime." 
\par 29 I skupi David svu vojsku, krenu na Rabu, navali na grad  i zauze ga. 
\par 30 Ondje skinu s Malkomove glave krunu, koja bijaše  teška jedan zlatni talenat; u njoj je bio dragi kamen, koji posta  ures na Davidovoj glavi. I vrlo bogat plijen odnese iz grada. 
\par 31 A narod koji bijaše u njemu izvede i stavi ga da radi kod  pila, željeznim pijucima i željeznim sjekirama i upotrijebi ga  za rad u ciglanama. I tako je isto činio svim gradovima sinova  Amonovih. Potom se David sa svom vojskom vrati u Jeruzalem. 


\chapter{13}

\par 1 A potom se dogodi ovo: Davidov sin Abšalom imao je lijepu  sestru po imenu Tamaru i u nju se zaljubio Davidov sin Amnon. 
\par 2 Amnon se toliko mučio da se gotovo razbolio radi svoje sestre  Tamare: jer ona bijaše djevica, pa Amnon nije vidio mogućnosti  da joj učini bilo što. 
\par 3 Ali imaše Amnon prijatelja po imenu  Jonadaba, sina Davidova brata Šimeja; a Jonadab bijaše vrlo domišljat. 
\par 4 I upita on Amnona: "Odakle to, kraljev sine, da si svako jutro  mlitav? Ne bi li mi kazao?" A Amnon mu odgovori: "Zaljubljen  sam u Tamaru, sestru svoga brata Abšaloma." 
\par 5 A Jonadab mu reče:  "Lezi u postelju i pričini se bolestan, pa kad dođe tvoj otac  da te pohodi, ti mu reci: 'Dopusti da dođe moja sestra Tamara  da mi dade jesti; ako ona pred mojim očima zgotovi jelo da to  vidim, onda ću iz njezine ruke jesti.'" 
\par 6 Amnon, dakle, leže i pričini se bolestan. Kad je došao  kralj da ga pohodi, reče Amnon kralju: "Dopusti da dođe moja  sestra Tamara da pred mojim očima zgotovi koji kolač i ja ću  se okrijepiti iz njezine ruke." 
\par 7 Tada David poruči Tamari u  palaču: "Idi u kuću svoga brata Amnona i priredi mu jelo!" 
\par 8 Tamara ode u kuću svoga brata Amnona. A on ležaše. Uze  ona brašna, umijesi ga, načini kolače pred njegovim očima te  ih ispeče. 
\par 9 Potom uze tavu i istrese je preda nj, ali Amnon  ne htjede jesti nego reče: "Otpremite sve odavde!" I svi iziđoše  od njega. 
\par 10 Tada Amnon reče Tamari: "Donesi mi jelo u spavaonicu  da se okrijepim iz tvoje ruke!" I Tamara uze kolače koje bijaše  zgotovila i donese ih svome bratu Amnonu u spavaonicu. 
\par 11 A  kad mu je pružila da jede, on je uhvati rukom i reče joj: "Dođi, sestro moja, lezi sa mnom!" 
\par 12 A ona mu reče: "Nemoj, brate  moj! Ne sramoti me jer se tako ne radi u Izraelu. Ne čini takve  sramote! 
\par 13 Kuda bih ja sa svojom sramotom? A i ti bi bio kao  bestidnik u Izraelu! Nego govori s kraljem: on me neće uskratiti  tebi!" 
\par 14 Ali je on ne htjede poslušati, nego je svlada i leže  s njom. 
\par 15 Nato je odmah zamrzi silnom mržnjom te je mržnja kojom  ju je zamrzio bila veća od ljubavi kojom ju je prije ljubio.  I reče joj Amnon: "Ustani! Odlazi!" 
\par 16 A ona mu odvrati: "Ne, brate moj! Ako me sad otjeraš, bit će to veće zlo od onoga koje  si mi učinio!" Ali je on ne htjede slušati, 
\par 17 nego dozva momka  koji ga je služio i zapovjedi mu: "Otjeraj ovu od mene, izbaci  je i zaključaj vrata za njom!" 
\par 18 (A ona je imala na sebi haljinu  s dugim rukavima, jer su se nekoć u takve haljine oblačile kraljeve  kćeri dok su bile djevojke.) Sluga je izvede van i zaključa vrata  za njom. 
\par 19 Tada Tamara uze prašine i posu se njom po glavi, razdrije  haljinu s dugim rukavima koju je imala na sebi, stavi ruku na  glavu i ode vičući glasno dok je išla. 
\par 20 A njezin je brat Abšalom  upita: "Je li možda tvoj brat Amnon bio s tobom? Ali sada, sestro  moja, šuti: brat ti je! Ne uzimaj to k srcu!" Tako je Tamara  ostala osamljena u kući svoga brata Abšaloma. 
\par 21 Kad je kralj David čuo sve što se dogodilo, vrlo se razgnjevi, ali ne htjede žalostiti svoga sina Amnona, koga je ljubio jer  mu bijaše prvorođenac. 
\par 22 A Abšalom ne reče Amnonu ni riječi, ni zle ni dobre, jer je Abšalom zamrzio Amnona što mu osramoti  sestru Tamaru. 
\par 23 A poslije dvije godine imao je Abšalom striženje ovaca  u Baal Hasoru kod Efrajima; i Abšalom pozva svu kraljevu obitelj. 
\par 24 Abšalom dođe kralju i reče mu: "Evo, tvoj sluga ima striženje  ovaca, pa neka se kralj i njegovi dvorani udostoje doći svome  sluzi." 
\par 25 Ali kralj odgovori Abšalomu: "Ne, sine, nećemo doći  svi, da ti ne budemo na teret." Abšalom ustraja, ali kralj ne  htjede ići, nego ga blagoslovi i otpusti. 
\par 26 Ali Abšalom nastavi:  "Ako ti nećeš, dopusti da bar moj brat Amnon pođe s nama." A  kralj ga upita: "Zašto da ide s tobom?" 
\par 27 Ali je Abšalom i  dalje navaljivao te David naposljetku pusti s njim Amnona i sve  kraljeve sinove. Abšalom priredi kraljevsku gozbu 
\par 28 i zapovjedi svojim slugama  ovako: "Pazite! Kad se Amnonu razveseli srce od vina i ja vam  viknem: 'Ubijte Amnona!' tada ga pogubite! Ne bojte se, jer vam  tako zapovijedam! Ohrabrite se i pokažite se junaci!" 
\par 29 I Abšalomove  sluge učiniše s Amnonom kako im zapovjedi Abšalom. Tada skočiše  svi kraljevi sinovi, pojahaše svaki svoju mazgu i pobjegoše. 
\par 30 Dok su oni još bili na putu, dođe ovakva vijest Davidu:  "Abšalom je pobio sve kraljeve sinove, nije ostao od njih ni  jedan jedini." 
\par 31 Kralj ustade, razdrije svoje haljine i baci  se na zemlju; i svi njegovi dvorani koji stajahu oko njega razdriješe  svoje haljine. 
\par 32 Ali Jonadab, sin Davidova brata Šimeja, progovori  ovako: "Neka ne govori moj gospodar da su pobili sve mladiće, kraljeve sinove, jer je poginuo samo Amnon: na Abšalomovu licu  mogla se predviđati nesreća od onoga dana kad je Amnon osramotio  njegovu sestru Tamaru. 
\par 33 Zato neka sada moj gospodar i kralj  ne misli u srcu da su svi kraljevi sinovi poginuli. Poginuo je  samo Amnon, 
\par 34 a Abšalom je pobjegao." A momak koji bijaše na straži podiže oči i ugleda mnoštvo  naroda gdje silazi cestom od Horonajima. Stražar dođe i javi  kralju: "Vidio sam ljude gdje silaze cestom od Horonajima po  gorskom obronku." 
\par 35 Tada Jonadab reče kralju: "Evo stigoše  kraljevi sinovi! Dogodilo se kako je rekao tvoj sluga." 
\par 36 Tek  što je to izrekao, a to kraljevi sinovi uđoše i zaplakaše u sav  glas; a i kralj i svi njegovi dvorani plakahu. 
\par 37 Abšalom pak  bijaše pobjegao i otišao k Talmaju, sinu Amihudovu, gešurskom  kralju. A David tugovaše za svojim sinom bez prestanka. 
\par 38 A pošto je Abšalom pobjegao i otišao u Gešur, ostao je  ondje tri godine. 
\par 39 Kralj David prestao se srditi na Abšaloma jer se utješio  zbog smrti Amnonove. 


\chapter{14}

\par 1 A Joab, sin Sarvijin, opazi da se kraljevo srce okreće k Abšalomu. 
\par 2 Zato Joab pošalje u Tekou po jednu pametnu ženu i reče joj:  "Učini se kao da si u žalosti za mrtvim, obuci žalobne haljine, nemoj se mazati uljem, nego budi kao žena koja je već dugo vremena  u žalosti za mrtvim. 
\par 3 Otići ćeš kralju i govorit ćeš mu ovako."  I Joab je nauči što će govoriti. 
\par 4 Žena iz Tekoe ode kralju, pade ničice na zemlju i pokloni  se, zatim reče: "Pomozi, kralju!" 
\par 5 Kralj je upita: "Što ti  je?" A ona odgovori: "Ah, ja sam udovica. Muž mi je umro, 
\par 6 a  tvoja je službenica imala dva sina. Oni se posvadiše u polju, a nije bilo nikoga da ih razdvoji te je jedan od njih udario  svoga brata i ubio ga. 
\par 7 I sad se podiže sav rod na tvoju službenicu  i reče: 'Predaj nam toga što je ubio svoga brata: mi ćemo ga  pogubiti za život njegova brata koga je ubio, a time ćemo zatrti  i baštinika.' Tako hoće da ugase žeravicu koja mi je ostala,  da ne ostave mome mužu ni imena ni potomstva na zemlji." 
\par 8 A  kralj reče ženi: "Idi svojoj kući, ja ću odrediti što treba za  te." 
\par 9 A žena iz Tekoe reče kralju: "Gospodaru kralju! Neka  na me i na moj očinski dom padne krivica; kralj i njegovo prijestolje  nedužni su u tome!" 
\par 10 A kralj nastavi: "Onoga koji ti se zaprijetio  dovedi k meni! Taj te neće više dirnuti!" 
\par 11 A ona reče: "Neka  se kralj udostoji spomenuti ime Jahve, svoga Boga, da krvni osvetnik  neće umnožiti zator i da neće pogubiti moga sina!" A on obeća:  "Tako mi živog Jahve, nijedna vlas neće pasti s glave tvome sinu!" 
\par 12 A žena nastavi: "Dopusti da tvoja službenica kaže jednu  riječ svome gospodaru kralju." A on odvrati: "Govori!" 
\par 13 A  žena reče: "Dakle, zašto je kralj - jer se izričući ovakvu presudu  sam priznao krivim - donio protiv naroda Božjega odluku da ne  pušta kući onoga koga je prognao? 
\par 14 Mi smo svi osuđeni na smrt, slični smo vodi koja se prolije na zemlju i više se ne može  skupiti, i Bog ne podiže mrtvaca: neka, dakle, kralj misli na  to da prognanik ne ostane izagnan daleko od njega. 
\par 15 A razlog zašto sam došla da iznesem pred svoga gospodara  kralja ovu stvar bio je taj što su me zaplašili ljudi, pa je  mislila tvoja službenica: moram govoriti s kraljem, možda će  kralj učiniti ono što mu njegova službenica kaže. 
\par 16 Jer će  kralj poslušati svoju službenicu i izbaviti je iz ruku čovjeka  koji hoće da me istrijebi zajedno s mojim sinom iz Božje baštine. 
\par 17 Zato je tvoja službenica pomislila: neka mi riječ moga gospodara  i kralja bude na umirenje. Jer moj je gospodar i kralj kao Božji  anđeo koji sluša dobro i zlo. Jahve, tvoj Bog, neka bude s tobom!" 
\par 18 Tada progovori kralj i reče ženi: "Nemoj mi sada zatajiti  ono što ću te pitati!" A žena odgovori: "Neka govori moj gospodar  kralj!" 
\par 19 Tada kralj upita: "Nisu li Joabovi prsti s tobom  u svemu tome?" A žena odgovori: "Tako bio živ, gospodaru kralju, zaista se ne može ni desno ni lijevo od svega što je kazao moj  gospodar i kralj! Jest, tvoj mi je sluga Joab zapovjedio, on  je naučio tvoju službenicu sve ove riječi. 
\par 20 Tvoj je sluga  Joab to učinio da bi svemu dao drugo lice, ali je moj gospodar  mudar kao Božji anđeo, on zna sve što se zbiva na zemlji." 
\par 21 Tada se kralj okrenu Joabu i reče mu: "Dobro, učinit  ću to. Idi i dovedi natrag mladića Abšaloma!" 
\par 22 A Joab pade  licem na zemlju, pokloni se i zahvali kralju; zatim reče Joab:  "Danas vidi tvoj sluga da je našao milost u tvojim očima, gospodaru  kralju, kad je kralj ispunio molbu svoga sluge." 
\par 23 Potom se  diže Joab, ode u Gešur i dovede Abšaloma natrag u Jeruzalem. 
\par 24 Ali kralj reče: "Neka ide u svoju kuću, a meni neka ne dolazi  na oči!" I Abšalom se povuče u svoju kuću i ne dođe kralju na  oči. 
\par 25 U svemu Izraelu ne bijaše čovjeka tako lijepa kao Abšalom  komu bi se mogle izreći tolike pohvale: od pete do glave nije  bilo na njemu mane. 
\par 26 A kad bi šišao kosu - a šišao ju je na  koncu svake godine, jer mu je bila preteška pa ju je morao šišati  - mjerio bi svoju kosu: bila bi teška dvije stotine šekela, po  kraljevskoj mjeri. 
\par 27 Abšalomu se rodiše tri sina i jedna kći  po imenu Tamara; bila je to vrlo lijepa žena. 
\par 28 Abšalom provede dvije godine u Jeruzalemu a da nije došao  kralju na oči. 
\par 29 Tada Abšalom pozva Joaba k sebi da bi ga poslao  kralju, ali Joab ne htjede doći k njemu; i posla drugi put po  njega, ali on opet ne htjede doći. 
\par 30 Tada Abšalom zapovjedi  slugama: "Znate Joabovo polje koje je pokraj mojega i na kojem  raste ječam: idite i zapalite ga!" I Abšalomove sluge zapališe  ono polje. 
\par 31 Tada se diže Joab, dođe k Abšalomu u kuću i upita  ga: "Zašto su tvoje sluge zapalile moje polje?" 
\par 32 A Abšalom  odgovori Joabu: "Ja sam poslao k tebi i poručio ti: 'Dođi ovamo, želio bih te poslati kralju s ovom porukom: Zašto sam se vratio  iz Gešura?' Bolje bi bilo za mene da sam još ondje. Zato sad  hoću da dođem kralju na oči, pa ako ima na meni kakva krivica, neka me pogubi!" 
\par 33 Joab ode kralju i javi mu te riječi. Zatim  kralj pozva Abšaloma. Dođe on pred kralja, pokloni mu se i pade  ničice pred kralja. I kralj poljubi Abšaloma. 


\chapter{15}

\par 1 Poslije toga nabavi Abšalom sebi kola i konje i pedeset ljudi  koji su trčali pred njim. 
\par 2 Abšalom je u rano jutro stajao kraj  puta koji vodi do gradskih vrata; i tko god je imao kakvu parnicu  te išao kralju na sud, Abšalom bi ga dozvao k sebi i pitao: "Iz  kojega si grada?" A kad bi ovaj odgovorio: "Tvoj je sluga iz  toga i toga Izraelova plemena", 
\par 3 tada bi mu Abšalom rekao:  "Vidiš, tvoja je stvar dobra i pravedna, ali nećeš naći nikoga  koji bi te saslušao kod kralja." 
\par 4 Abšalom bi nastavljao: "Ah, kad bi mene postavili za suca u zemlji! Svaki bi koji ima kakvu  parnicu ili sud dolazio k meni i ja bih mu pribavio pravo!" 
\par 5 A  kad bi mu se tko približio da mu se pokloni, on bi pružio ruku, privukao ga k sebi i poljubio. 
\par 6 Tako je činio Abšalom svim  Izraelcima koji su dolazili na sud kralju. Time je Abšalom predobivao  srca Izraelaca za sebe. 
\par 7 Kad su prošle četiri godine, Abšalom reče kralju: "Dopusti  da odem u Hebron i da izvršim zavjet kojim sam se zavjetovao  Jahvi. 
\par 8 Jer kad bijah u Gešuru u Aramu, tvoj se sluga zavjetovao  ovako: 'Ako me Jahve dovede natrag u Jeruzalem, iskazat ću čast  Jahvi u Hebronu.'" 
\par 9 A kralj mu odgovori: "Idi u miru!" I on  krenu na put i ode u Hebron. 
\par 10 Abšalom razasla tajne glasnike po svim Izraelovim plemenima  i poruči im: "Kad čujete zvuk roga, tada recite: Abšalom je postao  kralj u Hebronu." 
\par 11 A ode s Abšalomom dvije stotine ljudi iz  Jeruzalema; bijahu to uzvanici koji su bezazleno pošli ne znajući  što se sprema. 
\par 12 Abšalom posla i po Gilonjanina Ahitofela,  Davidova savjetnika, iz njegova grada Gilona, da pribiva prinošenju  žrtava. Urota je bila jaka, a mnoštvo Abšalomovih pristaša sve  je više raslo. 
\par 13 Tada stiže Davidu glasnik te mu javi: "Srce Izraelaca  priklonilo se Abšalomu." 
\par 14 Tada David reče svim svojim dvoranima  koji bijahu s njim u Jeruzalemu: "Ustanite! Bježimo! Inače nećemo  uteći od Abšaloma. Pohitite brzo, da on ne bude brži i ne stigne  nas, da ne obori na nas zlo i ne pobije grada oštricom mača!" 
\par 15 A kraljevi dvorani odgovoriše kralju: "Što god odluči naš  gospodar kralj, evo tvojih slugu!" 
\par 16 I kralj iziđe pješice  sa svim svojim dvorom; ipak ostavi kralj deset inoča da čuvaju  palaču. 
\par 17 I kralj ode pješice sa svim narodom i zaustavi se  kod posljednje kuće. 
\par 18 Svi njegovi dvorani stajahu uza nj.  Tada svi Kerećani, svi Pelećani, Itaj i svi Gićani koji bijahu  došli s njim iz Gata, šest stotina ljudi, prođoše pred kraljem. 
\par 19 Kralj upita Itaja Gićanina: "Zašto i ti ideš s nama?  Vrati se i ostani kod kralja! Ti si stranac, prognan iz svoje  zemlje. 
\par 20 Jučer si došao, a danas da te vodim da se potucaš  s nama kad ja idem kamo me sreća nanese. Vrati se i odvedi svoju  braću natrag sa sobom, a Jahve neka ti iskaže ljubav i vjernost!" 
\par 21 Ali Itaj odgovori kralju ovako: "Živoga mi Jahve i tako  mi živ bio moj gospodar kralj: gdje god bude moj gospodar kralj, bilo na smrt ili na život, ondje će biti i tvoj sluga!" 
\par 22 Tada David reče Itaju: "Hajde, prođi!" I Itaj iz Gata  prođe sa svim svojim ljudima i sa svom svojom pratnjom. 
\par 23 Svi plakahu iza glasa. Kralj je stajao na potoku Kidronu  i sav je narod prolazio pred njim prema pustinji. 
\par 24 Bijaše ondje i Sadok i s njim svi leviti koji su nosili  Kovčeg Božji. I oni spustiše Kovčeg Božji kraj Ebjatara dok sav  narod nije izišao iz grada. 
\par 25 Tada kralj reče Sadoku: "Odnesi  Kovčeg Božji natrag u grad. Ako nađem  milost u Jahve, on će  me dovesti natrag i dopustiti mi da opet vidim njega i njegovo  prebivalište. 
\par 26 A ako rekne ovako: 'Nisi mi po volji!' - onda  evo me, neka čini sa mnom što je dobro u njegovim očima!" 
\par 27 Još  kralj reče svećeniku Sadoku: "Hajde, ti i Ebjatar vratite se  u miru u grad, i vaša dva sina s vama, tvoj sin Ahimaas i Ebjatarov  sin Jonatan. 
\par 28 Evo, ja ću se zadržati na ravnicama pustinje  dok ne dođe od vas glas da me obavijesti." 
\par 29 Nato Sadok i Ebjatar odnesoše Kovčeg Božji natrag u Jeruzalem  i ostadoše ondje. 
\par 30 David se uspinjao na Maslinsku goru, sve plačući, pokrivene  glave i bos, i sav narod koji ga je pratio iđaše pokrivene glave  i plačući. 
\par 31 Tada javiše Davidu da je i Ahitofel među urotnicima  s Abšalomom. A David zavapi: "Obezumi Ahitofelove savjete, Jahve!" 
\par 32 Kad je David došao na vrh gore, ondje gdje se klanja  Bogu, dođe mu u susret Hušaj Arčanin, prijatelj Davidov, razdrte  haljine i glave posute prahom. 
\par 33 David mu reče: "Ako pođeš  sa mnom, bit ćeš mi na teret. 
\par 34 Ali ako se vratiš u grad i  kažeš Abšalomu: 'Bit ću tvoj sluga, gospodaru kralju; prije sam  služio tvome ocu, a sada ću služiti tebi', moći ćeš tada okretati  Ahitofelove savjete u moju korist. 
\par 35 S tobom će biti i svećenici  Sadok i Ebjatar. Sve što čuješ iz palače, javi svećenicima Sadoku  i Ebjataru. 
\par 36 S njima su ondje i dva njihova sina, Ahimaas  Sadokov i Jonatan Ebjatarov: po njima mi javljajte sve što čujete." 
\par 37 Tako se Hušaj, prijatelj Davidov, vrati u grad upravo  u času kad je Abšalom ulazio u Jeruzalem. 


\chapter{16}

\par 1 Kad je David prešao malo preko vrha, dođe mu u susret Siba, sluga Meribaalov, sa dva osamarena magarca koja su nosila dvije  stotine kruhova, sto grozdova suhog grožđa, sto voćnjača i mijeh  vina. 
\par 2 Kralj upita Sibu: "Što ćeš s tim?" A Siba odgovori:  "Magarci će poslužiti kraljevoj obitelji za jahanje, kruh i voće  momcima za jelo, a vino za piće onima koji se umore u pustinji." 
\par 3 Kralj dalje upita: "A gdje je sin tvoga gospodara?" A Siba  odgovori kralju: "Eno, ostao je u Jeruzalemu jer je mislio: 'Danas  će mi dom Izraelov vratiti kraljevstvo moga oca.'" 
\par 4 Tada kralj  reče Sibi: "Sve što posjeduje Meribaal neka je tvoje." A Siba  odgovori: "Bacam se ničice pred tobom. O, da bih i dalje bio  dostojan milosti u tvojim očima, kralju gospodaru!" 
\par 5 Kad je kralj David došao do Bahurima, izađe odande čovjek  od roda Šaulova. Zvao se Šimej, a bio je sin Gerin. Dok je izlazio, neprestano je proklinjao. 
\par 6 Bacao je kamenje na Davida i na  sve dvorane kralja Davida, premda je sva vojska sa svim junacima  okruživala kralja s desne i lijeve strane. 
\par 7 A Šimej je ovako  govorio proklinjući: "Odlazi, odlazi, krvniče, ništarijo! 
\par 8 Jahve  je okrenuo na tebe svu krv Šaulova doma, kojemu si ti oduzeo  kraljevstvo. Ujedno je Jahve predao kraljevstvo u ruke tvome  sinu Abšalomu. Evo, sad si zapao u nevolju jer si krvnik." 
\par 9 Tada Sarvijin sin Abišaj zapita kralja: "Zar da ovaj uginuli  pas proklinje moga gospodara kralja? Dopusti da odem prijeko  i da mu skinem glavu!" 
\par 10 Ali kralj odgovori: "Što hoćete od  mene, Sarvijini sinovi? Ako on proklinje te ako mu je Jahve zapovjedio:  'Proklinji Davida!' - tko ga smije pitati: 'Zašto činiš tako?'" 
\par 11 Nato David reče Abišaju i svim svojim dvoranima: "Eto, moj  sin koji je izašao od moga tijela radi mi o glavi, a kamoli neće  sada ovaj Benjaminovac! Pustite ga neka proklinje ako mu je Jahve  to zapovjedio. 
\par 12 Možda će Jahve pogledati na moju nevolju te  mi vratiti dobro za njegovu današnju psovku." 
\par 13 Zatim David sa svojim ljudima nastavi put, a Šimej iđaše  gorskom stranom usporedo s njim, neprestano psujući, bacajući  kamenje i dižući prašinu. 
\par 14 Kralj i sav narod koji ga je pratio  stigoše umorni i ondje odahnuše. 
\par 15 Abšalom je međutim sa svim narodom izraelskim ušao u  Jeruzalem; i Ahitofel bijaše s njim. 
\par 16 A kad je Hušaj Arčanin, Davidov prijatelj, došao k Abšalomu, reče Hušaj Abšalomu: "Živio  kralj! Živio kralj!" 
\par 17 A Abšalom upita Hušaja: "Je li to tvoja  vjernost prema tvome prijatelju? Zašto nisi otišao sa svojim  prijateljem?" 
\par 18 A Hušaj odgovori Abšalomu: "Ne, nego koga je  izabrao Jahve i ovaj narod i svi Izraelci, njegov ću biti i s  njim ću ostati. 
\par 19 A drugo: kome ću služiti? Zar ne njegovu  sinu? Kako sam služio tvojemu ocu, tako ću služiti tebi." 
\par 20 Potom se Abšalom obrati Ahitofelu: "Savjetuj sada: što  da činimo?" 
\par 21 Ahitofel odgovori Abšalomu: "Uđi k inočama svoga  oca, koje je ostavio da čuvaju palaču: tada će sav Izrael čuti  da si u zavadi sa svojim ocem, pa će se ohrabriti svi oni koji  su pristali uz tebe." 
\par 22 Tada razapeše za Abšaloma šator na  krovu i Abšalom uđe k inočama svoga oca na oči svemu Izraelu. 
\par 23 A savjet što bi ga dao Ahitofel u ono vrijeme vrijedio je  kao odgovor Božji; toliko je vrijedio svaki Ahitofelov savjet  i kod Davida i kod Abšaloma. 


\chapter{17}

\par 1 Nato Ahitofel reče Abšalomu: "Dopusti da izaberem dvanaest  tisuća ljudi pa da se dignem i pođem u potjeru za Davidom još  noćas. 
\par 2 Navalit ću na njega kad bude umoran i bez snage; plašit  ću ga i razbježat će se sav narod koji je s njim. Onda ću ubiti  samoga kralja. 
\par 3 A sav ću narod dovesti natrag k tebi, kao što  se mlada vraća svome mužu: ti radiš o glavi samo jednome čovjeku, a sav će narod onda biti miran." 
\par 4 Svidje se to Abšalomu i  svim starješinama Izraelovim. 
\par 5 Ali Abšalom reče: "Pozovimo još Hušaja Arčanina da čujemo  što će nam on kazati!" 
\par 6 Kad je Hušaj došao k Abšalomu, reče  mu Abšalom: "Ahitofel je svjetovao ovako. Hoćemo li učiniti kako  je on predložio? Ako ne, govori ti!" 
\par 7 A Hušaj odgovori Abšalomu: "Ovaj put savjet Ahitofelov  nije dobar." 
\par 8 I nastavi Hušaj: "Ti znaš da su tvoj otac i njegovi  ljudi junaci i da su ljuti kao medvjedica kojoj su oteli njezine  medvjediće. Tvoj je otac ratnik, neće on dopustiti da narod počiva  preko noći. 
\par 9 On se sada krije u kakvoj jami ili na kakvu drugom  mjestu. Pa ako odmah u početku koji od naših padne, proširit  će se glas o porazu u vojsci koja je pristala uz Abšaloma. 
\par 10 Tada  će i najhrabriji, u koga je srce kao u lava, izgubiti srčanost.  Jer sav Izrael zna da je tvoj otac junak i da su hrabri oni koji  ga prate. 
\par 11 Zato ja svjetujem ovo: neka se sav Izrael, od Dana  do Beer Šebe, okupi oko tebe, da ga bude kao pijeska na obali  morskoj, a ti sam da stupaš u njihovoj sredini. 
\par 12 Tada ćemo  navaliti na njega gdje se god bude nalazio, oborit ćemo se na  nj kao što rosa pada na zemlju i nećemo ostaviti živa ni njega  niti ikojega od njegovih ljudi. 
\par 13 Ako li se povuče u koji grad, sav će izraelski narod donijeti užeta pod onaj grad pa ćemo  ga povlačiti do potoka, sve dok više ni kamenčića ne bude od  njega." 
\par 14 Tada Abšalom i svi Izraelci rekoše: "Bolji je savjet  Hušaja Arčanina nego savjet Ahitofelov." Jer Jahve bijaše odlučio  da se osujeti izvrsna Ahitofelova osnova, kako bi navukao nesreću  na Abšaloma. 
\par 15 Potom Hušaj javi svećenicima Sadoku i Ebjataru: "Ahitofel  je tako i tako savjetovao Abšaloma i starješine izraelske, a  ja sam savjetovao tako i tako. 
\par 16 Zato sad brzo javite to Davidu  i poručite mu: 'Nemoj noćas noćiti na ravnicama pustinje, nego  brzo prijeđi na drugu stranu da ne bude uništen kralj i sva vojska  koja je s njim.'" 
\par 17 Jonatan i Ahimaas zadržavali se kod Rogelskog izvora;  jedna je sluškinja dolazila i donosila im vijesti, a oni su odlazili  da to jave kralju Davidu, jer se nisu smjeli odati ulazeći u  grad. 
\par 18 Ali ih opazi neki momak te javi Abšalomu. Nato obojica  žurno odoše i dođoše u kuću nekoga čovjeka u Bahurimu. U njegovu  dvorištu bijaše studenac i oni se spustiše u nj. 
\par 19 A žena uze  i razastrije pokrivač preko otvora studencu i posu po njem stučenoga  zrnja, tako da se ništa nije moglo opaziti. 
\par 20 Abšalomove sluge dođoše k toj ženi u kuću i upitaše:  "Gdje su Ahimaas i Jonatan?" A žena im odgovori: "Otišli su dalje  prema vodi." Potom su ih još tražili, ali ih ne nađoše pa se  vratiše u Jeruzalem. 
\par 21 A kad su oni otišli, ona dvojica iziđoše  iz studenca i odoše da donesu vijesti kralju Davidu. I rekoše  mu: "Ustajte i prijeđite brže preko vode, jer je tako i tako  savjetovao protiv vas Ahitofel." 
\par 22 Tada se David i sav narod  što bijaše s njim diže i prijeđe preko Jordana; u zoru nije više  bilo nijednoga koji nije prešao preko Jordana. 
\par 23 Kad je Ahitofel vidio da se nije izvršio njegov savjet, osamari svoga magarca, krenu na put i ode svojoj kući u svoj  grad. Ondje se pobrinu za svoju kuću, zatim se objesi i umrije.  Pokopaše ga u grobu njegova oca. 
\par 24 David je već bio došao u Mahanajim kad je Abšalom prešao  preko Jordana sa svim Izraelcima koji bijahu s njim. 
\par 25 Abšalom  bijaše postavio Amasu za zapovjednika nad vojskom namjesto Joaba.  A Amasa je bio sin nekoga čovjeka po imenu Jitre, Jišmaelovca, koji je ušao k Abigajili, kćeri Jišajevoj i sestri Sarvije,  Joabove majke. 
\par 26 Izrael i Abšalom udariše tabor u zemlji gileadskoj. 
\par 27 Kad je David došao u Mahanajim, tada Šobi, sin Nahašev  iz Rabe Amonske, pa Makir, sin Amielov iz Lo Debara, i Barzilaj, Gileađanin iz Rogelima, 
\par 28 donesoše postelja, pokrivača, čaša  i zemljanog suđa, uz to pšenice, ječma, brašna, pržena žita,  boba, leće, 
\par 29 meda, kiseloga mlijeka i sira kravljeg i ovčjeg  i ponudiše Davida i narod što bijaše s njim da jedu. Jer mišljahu:  "Ljudi su u pustinji trpjeli glad, umor i žeđu." 


\chapter{18}

\par 1 Potom David pobroji narod što bijaše s njim i postavi nad  njima tisućnike i stotnike. 
\par 2 Zatim podijeli vojsku na tri skupine:  jednu trećinu predade Joabu, drugu trećinu Abišaju, sinu Sarvijinu, bratu Joabovu, a treću trećinu Itaju iz Gata. Tada David reče  narodu: "I ja ću s vama u rat." 
\par 3 Ali se narod usprotivi: "Ne  smiješ ti ići. Jer ako mi i pobjegnemo, neće nitko na to obraćati  pažnju, ili ako nas i pola izgine, neće se na to obraćati pažnja;  ali ti sam vrijediš kao nas deset tisuća. Osim toga, bolje je  da budeš pripravan da nam iz grada pomogneš." 
\par 4 A kralj im odgovori:  "Učinit ću sve što vam se čini dobro." I kralj stade kod vrata  dok je vojska izlazila po stotinama i tisućama. 
\par 5 A Joabu, Abišaju  i Itaju dade zapovijed: "Čuvajte mi mladića Abšaloma!" I sav  je narod čuo da je kralj tako zapovjedio svim vojvodama za Abšaloma. 
\par 6 Tako vojska iziđe za boj spremna pred Izraela i bitka  se zametnu u Efrajimovoj šumi. 
\par 7 Izraelsku vojsku potukoše Davidovi  ljudi; i velik poraz bijaše u onaj dan: dvadeset tisuća mrtvih. 
\par 8 Boj se proširio po svemu onom kraju i više je ljudi onoga  dana progutala šuma nego mač. 
\par 9 Abšalom slučajno zapade u ruke Davidovim ljudima. Abšalom  je jahao na mazgi, a mazga naiđe pod grane velika hrasta, tako  te je Abšalomu glava zapela o grane i on osta viseći između neba  i zemlje, dok je mazga ispod njega otišla dalje. 
\par 10 Vidje to neki čovjek i javi Joabu govoreći: "Upravo sam  vidio Abšaloma gdje visi o jednom hrastu." 
\par 11 A Joab odvrati  čovjeku koji mu je to javio: "Kad si ga vidio, zašto ga na mjestu  nisi sastavio sa zemljom? Moja bi onda bila dužnost da ti dam  deset srebrnih šekela i jedan pojas!" 
\par 12 Ali čovjek odgovori  Joabu: "I kad bi mi na dlan izbrojio tisuću srebrnih šekela,  ne bih digao ruku na kraljeva sina! Čuli smo na svoje uši kako  je kralj zapovjedio tebi, Abišaju i Itaju govoreći: 'Čuvajte  mi mladića Abšaloma!' 
\par 13 Da sam podmuklo napao na njega izlažući  opasnosti svoj život - jer kralju ništa ne ostaje skriveno -  onda bi se ti držao po strani." 
\par 14 A Joab odvrati: "Neću ja ovdje dangubiti s tobom!" I  uze tri sulice u ruke i zabode ih u srce Abšalomu, koji je bio  još živ viseći o hrastu. 
\par 15 Nato priđe deset momaka, štitonoša  Joabovih, i dotukoše Abšaloma i usmrtiše. 
\par 16 Tada Joab zapovjedi da zatrube u rog, i vojska prestade  progoniti Izraela jer je Joab zaustavio vojsku. 
\par 17 Potom uzeše  Abšaloma, baciše ga u duboku jamu usred šume i navaljaše na nj  veliku gomilu kamenja. Izraelci pak pobjegoše svaki svome šatoru. 
\par 18 Abšalom bijaše još za života postavio sebi spomenik u  Kraljevoj dolini jer mišljaše: "Nemam sina koji bi sačuvao spomen  mome imenu." I nazvao je taj spomenik po svome imenu te se još  i danas zove "Abšalomov spomenik". 
\par 19 Ahimaas, Sadokov sin, reče Joabu: "Idem javiti kralju  veselu vijest da mu je Jahve pribavio pravdu izbavivši ga iz  ruku njegovih neprijatelja." 
\par 20 Ali mu Joab reče: "Ne možeš  danas biti glasnik vesele vijesti, nego ćeš to biti koji drugi  dan; danas ne možeš javiti dobru vijest jer je poginuo kraljev  sin." 
\par 21 Zatim Joab zapovjedi Etiopljaninu: "Idi javi kralju što  si vidio!" Etiopljanin se pokloni Joabu i otrča. 
\par 22 A Sadokov sim Ahimaas opet zamoli Joaba: "Dogodilo se  što mu drago, dopusti da otrčim i ja za Etiopljaninom." A Joab  upita: "Zašto bi trčao, sine moj, kad ti ta vesela vijest neće  pribaviti nagrade?" 
\par 23 A on ponovi: "Dogodilo se što mu drago, trčat ću!" A Joab mu odvrati: "Trči!" I Ahimaas otrča putem  kroz ravnicu i preteče Etiopljanina. 
\par 24 David je upravo sjedio među dvojim gradskim vratima,  a stražar se bio uspeo na krov iznad vrata. Podigavši oči, stražar  ugleda čovjeka kako trči sam. 
\par 25 Stražar povika i javi kralju, a kralj mu reče: "Ako je sam, nosi dobar glas na ustima." Čovjek  je dolazio sve bliže. 
\par 26 Uto stražar ugleda drugoga čovjeka  gdje trči. I povika stražar koji je bio nad vratima: "Evo još  jednoga čovjeka koji trči sam!" A kralj odvrati: "I taj nosi  dobar glas." 
\par 27 Stražar nastavi: "Prepoznajem trk prvoga čovjeka:  trči kao Sadokov sin Ahimaas." A kralj odvrati: "To je dobar  čovjek, dolazi s dobrim glasom." 
\par 28 Ahimaas se približi kralju i pozdravi ga: "Zdravo!" Baci  se licem na zemlju pred kraljem i nastavi: "Blagoslovljen Jahve, tvoj Bog, koji je napustio ljude što su digli ruku na moga gospodara  i kralja!" 
\par 29 A kralj upita: "Je li spašen mladić Abšalom?"  A Ahimaas odgovori: "Vidio sam veliku vrevu kad je kraljev sluga  Joab slao tvoga slugu, ali ne znam što je bilo." 
\par 30 Kralj mu  reče: "Odstupi i stani tamo!" On odstupi i stade. 
\par 31 Uto stiže Etiopljanin i progovori: "Neka moj gospodar  kralj primi veselu vijest. Jahve ti je danas pribavio pravdu  izbavivši te iz ruku svih onih koji su ustali na tebe." 
\par 32 A  kralj upita Etiopljanina: "Je li spašen mladić Abšalom?" A Etiopljanin  odgovori: "Neka neprijatelji moga gospodara i kralja i svi koji  se dižu na tebe u zloj namjeri - prođu kao taj mladić!" 
\par 33 (19:1) Kralj zadrhta, pope se u gornju odaju nad vratima i zaplaka;  jecajući govoraše ovako: "Sine Abšalome, sine moj! Sine moj Abšalome!  Oh, da sam ja umro mjesto tebe! Abšalome, sine moj, sine moj!" 


\chapter{19}

\par 1 (19:2) I javiše Joabu: "Eno kralj plače i tuguje za Abšalomom." 
\par 2 (19:3) Tako  se pobjeda u onaj dan pretvorila u žalost za svu vojsku, jer  je vojska čula u onaj dan da kralj tuguje za svojim sinom. 
\par 3 (19:4) I  toga se dana vojskom kradom vrati u grad, kao što se kradom šulja  vojska koja se osramotila bježeći iz boja. 
\par 4 (19:5) A kralj je pokrio  svoje lice i vapio iza glasa: "Sine moj Abšalome! Abšalome, sine  moj! Sine moj!" 
\par 5 (19:6) Tada Joab dođe kralju u kuću i reče mu: "Postiđuješ danas  lice svih svojih slugu koji su danas spasili život tebi, život  tvojim sinovima i tvojim kćerima, život tvojim ženama i život  inočama tvojim, 
\par 6 (19:7) jer iskazuješ ljubav onima koji te mrze, a  mržnju onima koji te ljube. Danas si pokazao da ti ništa nije  ni do vojvoda ni do vojnika, jer vidim sada da bi ti sasvim pravo  bilo kad bi Abšalom bio živ, a mi svi da smo danas poginuli. 
\par 7 (19:8) Zato sada ustani, iziđi i prijazno progovori svojim vojnicima;  jer, kunem ti se Jahvom, ako ne iziđeš, nijedan čovjek neće ostati  noćas s tobom, i to će ti biti veća nesreća od svih koje su te  snašle od tvoje mladosti pa do sada." 
\par 8 (19:9) Kralj ustade i sjede na vrata. Javiše to svemu narodu  govoreći: "Eno kralj sjedi na vratima." I sav narod dođe pred  kralja.  A Izraelci bijahu pobjegli svaki u svoj šator. 
\par 9 (19:10) I sav  se narod po svim Izraelovim plemenima prepirao govoreći: "Kralj  nas je izbavio iz ruku naših neprijatelja, on nas je izbavio  iz ruku filistejskih, a sada je morao pobjeći iz zemlje ispred  Abšaloma. 
\par 10 (19:11) A Abšalom koga smo pomazali za kralja poginuo je  u boju. Zašto se, dakle, kolebate dovesti kralja natrag?" 
\par 11 (19:12) Te riječi svega Izraela dopru do kralja u njegovu kuću.  Zato kralj David poruči svećenicima Sadoku i Ebjataru: "Recite  starješinama judejskim ovako: 'Zašto da vi budete posljednji  koji će kralja dovesti u njegovu kuću? 
\par 12 (19:13) Vi ste moja braća, vi ste od moga mesa i od mojih kosti. Zašto biste, dakle, bili  posljednji koji će dovesti kralja natrag?' 
\par 13 (19:14) Recite i Amasi:  'Nisi li ti od mojih kosti i od moga mesa? Neka mi Bog učini  zlo i neka mi doda drugo ako mi ne budeš zauvijek vojvoda nad  mojom vojskom namjesto Joaba!'" 
\par 14 (19:15) Tada se složiše svi ljudi  Judina roda kao jedan čovjek i poručiše kralju: "Vrati se sa  svim svojim ljudima!" 
\par 15 (19:16) I tako se kralj vrati i dođe do Jordana, a Judejci bijahu  stigli do Gilgala dolazeći u susret kralju da prate kralja na  prijelazu preko Jordana. 
\par 16 (19:17) Tada je pohitio i Šimej, sin Gerin, Benjaminovac iz Bahurima, i sišao s Judejcima u susret kralju  Davidu. 
\par 17 (19:18) Imao je sa sobom tisuću ljudi od Benjaminova plemena.  I Siba, sluga Šaulova doma, sa petnaest svojih sinova i dvadeset  svojih slugu, dođe do Jordana pred kralja. 
\par 18 (19:19) Dovezli su splav  da prevezu kraljevu čeljad i da učine sve što bi mu bilo drago. A Gerin sin Šimej baci se pred noge kralju kad je kralj htio  prijeći preko Jordana; 
\par 19 (19:20) i reče kralju: "Neka mi moj gospodar  ne upiše u grijeh! Ne opominji se zla što ti ga je učinio tvoj  sluga u onaj dan kad je moj gospodar i kralj izlazio iz Jeruzalema.  Neka to kralj ne uzima k srcu! 
\par 20 (19:21) Tvoj sluga uviđa da je sagriješio;  zato sam, evo, došao danas prvi iz svega Josipova doma da siđem  u susret svome gospodaru i kralju." 
\par 21 (19:22) Ali Sarvijin sin Abišaj progovori i reče: "Zar Šimej  ne zaslužuje smrt što je proklinjao pomazanika Jahvina?" 
\par 22 (19:23) A  David odgovori: "Što ja imam s vama, Sarvijini sinovi, te me  danas uvodite u napast? Zar bi danas mogao tko biti pogubljen  u Izraelu? TÓa sada znam da sam danas opet kralj nad Izraelom." 
\par 23 (19:24) Tada kralj reče Šimeju: "Nećeš poginuti!" I kralj mu se zakle. 
\par 24 (19:25) I Šaulov sin Meribaal sišao je u susret kralju. On nije  njegovao ni svojih nogu ni svojih ruku, nije uređivao svoje brade, nije prao svojih haljina od onoga dana kad je otišao kralj pa  sve do dana kad se opet vratio u miru. 
\par 25 (19:26) Kad je iz Jeruzalema  došao u susret kralju, upita ga kralj: "Zašto nisi pošao sa mnom, Meribaale?" 
\par 26 (19:27) A on odgovori: "Kralju gospodaru! Moj me sluga  prevario. Tvoj mu je sluga rekao: 'Osamari mi magaricu da je  uzjašem i pođem s kraljem!' Jer tvoj je sluga hrom. 
\par 27 (19:28) On je  oklevetao tvoga slugu pred mojim gospodarom i kraljem. Ali moj  je gospodar i kralj kao Božji anđeo: zato čini što je dobro u  tvojim očima. 
\par 28 (19:29) Jer sav moj očinski dom nije bio drugo zaslužio  nego smrt od moga gospodara kralja, a ti si ipak primio svoga  slugu među one koji jedu za tvojim stolom. Pa kako još imam pravo  tužiti se kralju?" 
\par 29 (19:30) A kralj mu odgovori: "Čemu da još duljiš svoj govor?  Određujem: ti i Siba podijelite njive!" 
\par 30 (19:31) Meribaal reče kralju: "Neka uzme i sve, kad se moj gospodar  kralj sretno vratio u svoj dom!" 
\par 31 (19:32) I Barzilaj Gileađanin dođe iz Rogelima i nastavi s kraljem  da ga isprati preko Jordana. 
\par 32 (19:33) Barzilaj bijaše vrlo star, bilo  mu je osamdeset godina. Pribavljao je kralju opskrbu dok je boravio  u Mahanajimu jer bijaše vrlo imućan čovjek. 
\par 33 (19:34) Kralj reče Barzilaju:  "Pođi sa mnom, ja ću te u tvojim starim danima uzdržavati kod  sebe u Jeruzalemu." 
\par 34 (19:35) A Barzilaj odgovori kralju: "A koliko  mi još godina života ostaje da idem s kraljem u Jeruzalem? 
\par 35 (19:36) Sada  mi je osamdeset godina; mogu li još razlikovati što je dobro  a što zlo? Može li tvojem sluzi još goditi što jede i pije? Mogu  li još slušati glas pjevača i pjevačica? Zašto bi tvoj sluga  bio još na teret mome gospodaru kralju? 
\par 36 (19:37) Tvoj će sluga još  samo prijeći preko Jordana s kraljem, ali zašto bi mi kralj dao  takvu nagradu? 
\par 37 (19:38) Dopusti svome sluzi da se vrati, da umrem  u svom gradu kod groba svoga oca i svoje majke. Ali evo tvoga  sluge Kimhama, neka ide dalje s mojim gospodarom kraljem, pa  njemu učini što je dobro u tvojim očima!" 
\par 38 (19:39) Kralj odgovori: "Neka onda Kimham ide sa mnom dalje,  a ja ću mu učiniti što bude tebi drago i što god me zamoliš sve  ću mu učiniti za tebe." 
\par 39 (19:40) Kad je sav narod prešao preko Jordana, prijeđe i kralj, poljubi Barzilaja i blagoslovi ga, potom se ovaj vrati u svoje  mjesto. 
\par 40 (19:41) Kralj nastavi put u Gilgal, a Kimham iđaše s njim. Kralja  je pratio sav narod Judin i polovina naroda Izraelova. 
\par 41 (19:42) Uto  svi Izraelci dođu pred kralja i upitaju ga: "Zašto te naša braća  Judejci ukradoše i zašto prevedoše preko Jordana našega kralja  i njegov dom i sve Davidove ljude s njim?" 
\par 42 (19:43) A Juda odgovori  Izraelu: "Kralj je meni rod. Zašto si se ražestio zbog toga?  Jesam li jeo na kraljev račun? Ili sam si što prigrabio?" 
\par 43 (19:44) Tada  Izrael odgovori Judi ovako: "Ja imam deset udjela na kralja i  prema tebi ja sam prvorođenac. Zašto si me, dakle, prezreo? Nije  li moja riječ bila prva kad je trebalo natrag dovesti moga kralja?"  Ali govor Judin bijaše tvrđi od govora Izraelova. 


\chapter{20}

\par 1 Ondje se slučajno našao opak čovjek po imenu Šeba, Bikrijev  sin, Benjaminovac. On zatrubi u rog i viknu: "Mi nemamo udjela na Davidu ni baštine na Jišajevu sinu! Svaki svome šatoru, Izraele!" 
\par 2 Tako svi Izraelci ostaviše Davida i pođoše za Bikrijevim  sinom Šebom; a Judejci prionuše uza svoga kralja i otpratiše  ga od Jordana do Jeruzalema. 
\par 3 Kad se David vratio u svoju palaču u Jeruzalem, uze deset  inoča koje je bio ostavio da čuvaju palaču i stavi ih da budu  čuvane. Brinuo im se za uzdržavanje, ali nije više išao k njima.  Tako su one živjele zatvorene do svoje smrti, kao udovice živoga  muža. 
\par 4 Potom kralj zapovjedi Amasi: "Sazovi mi Judejce do tri  dana, a i ti da budeš ovdje!" 
\par 5 Amasa ode da sazove Judejce, ali se zadrža preko vremena koje mu bijaše odredio kralj. 
\par 6 Tada  David reče Abišaju: "Sad će nam Bikrijev sin Šeba biti opasniji  nego Abšalom. Zato uzmi ljude svoga gospodara i pođi za njim  u potjeru da se ne domogne tvrdih gradova i ne izmakne nam iz  očiju!" 
\par 7 Za Abišajem krenu na put Joab, Kerećani, Pelećani  i svi junaci; oni iziđu iz Jeruzalema u potjeru za Bikrijevim  sinom Šebom. 
\par 8 Kad su bili kod velikoga kamena što je kod Gibeona, dođe Amasa prema njima. Joab imaše na sebi ratnu haljinu, a  preko nje imaše pripasan mač uz bedro, u koricama; ali mu se  mač iskliznu i pade. 
\par 9 Joab pozdravi Amasu: "Jesi li mi dobro, brate?" I desnom rukom uhvati za bradu Amasu da ga poljubi. 
\par 10 Amasa se nije obazirao na mač koji bijaše Joabu u ruci, i  on ga udari njim u trbuh i prosu mu utrobu na zemlju. Nije morao  ponoviti udarac i Amasa umrije. Joab sa svojim bratom Abišajem  nastavi potjeru za Bikrijevim sinom Šebom. 
\par 11 Jedan od Joabovih momaka osta na straži kod Amase i tu  je vikao: "Kome je mio Joab i tko je za Davida neka slijedi Joaba!" 
\par 12 A Amasa ležao u krvi nasred puta. Videći onaj čovjek gdje  se ustavlja sav narod, odvuče Amasu s puta u polje i baci preko  njega kabanicu jer je vidio gdje se zaustavlja svatko tko naiđe  blizu njega. 
\par 13 Kad je Amasa bio uklonjen s puta, svi ljudi  pođoše za Joabom da gone Bikrijeva sina Šebu. 
\par 14 Šeba je prošao kroza sva izraelska plemena sve do Abel  Bet Maake i svi Bikrani s njim. Skupiše se oni i pođoše za njim. 
\par 15 Joab dođe i opsjede ga u Abel Bet Maaki. Dade nasuti nasip  oko grada. Sva vojska koja bijaše s Joabom navali potkopavati  zid da ga obori. 
\par 16 Tada se jedna mudra žena uspe na zid i povika  iz grada: "Čujte! Čujte! Recite Joabu: 'Priđi ovamo, da govorim  s tobom!'" 
\par 17 Kad je prišao, upita žena: "Jesi li ti Joab?"  On odgovori: "Jesam." A ona će: "Poslušaj riječ sluškinje svoje!"  On odgovori: "Slušam." 
\par 18 Žena nastavi: "Nekoć se govorilo ovako:  'Treba pitati u Abelu i u Danu 
\par 19 je li svršeno s onim što su  utvrdili vjernici u Izraelu.' Ti bi htio uništiti jedan grad, i to jedan od matičnih gradova u Izraelu. Zašto zatireš baštinu  Jahvinu?" 
\par 20 Joab odgovori ovako: "Daleko, daleko bilo to od mene!  Ne želim ni zatirati ni razarati. 
\par 21 Ne radi se o tome, nego  je jedan čovjek iz Efrajimove gore, po imenu Šeba, Bikrijev sin, podigao ruku na kralja, na Davida. Predajte samo njega, pa ću  otići od grada!" Žena odgovori Joabu: "Dobro. Odmah će ti njegovu  glavu baciti preko zida!" 
\par 22 Žena se vrati u grad i progovori  svemu narodu kako joj je govorila njezina mudrost. I odsjekoše  glavu Bikrijevu sinu Šebi i baciše je Joabu. A on zapovjedi da  zatrube u rog te se raziđoše od grada, svaki u svoj kraj. A Joab  se vrati kralju u Jeruzalem. 
\par 23 Joab je bio zapovjednik nad svom vojskom. Jojadin sin  Benaja bio je zapovjednik nad Kerećanima i Pelećanima. 
\par 24 Adoram  je bio nadglednik nad radovima. Ahiludov sin Jošafat bio je pečatnik. 
\par 25 Seraja je bio državni pisar. Sadok i Ebjatar bijahu svećenici. 
\par 26 Uz to je Jairanin Ira bio zamjenik Davidov. 


\chapter{21}

\par 1 U dane Davidove vladaše jednom glad tri godine uzastopce.  David se obrati Jahvi, a Jahve mu odgovori: "Na Šaulu i njegovu  domu leži krvna krivnja jer je pogubio Gibeonce." 
\par 2 Tada kralj  sazva Gibeonce da ih pita. Ti Gibeonci nisu pripadali Izraelcima, nego su bili ostatak Amorejaca, kojima se Izraelci bijahu zakleli  zakletvom, ali je Šaul tražio da ih uništi u svojoj revnosti  za Izraelce i Judejce. 
\par 3 David, dakle, upita Gibeonce: "Što da vam učinim i čime  da vam dadem zadovoljštinu da biste blagoslovili baštinu Jahvinu?" 
\par 4 Gibeonci odgovoriše: "Ne tražimo mi ni srebra ni zlata od  Šaula i njegova doma, niti nam je stalo da se pogubi koji čovjek  u Izraelu." David će im nato: "Što reknete učinit ću za vas." 
\par 5 A oni odgovoriše kralju: "Čovjek koji nas je zatirao i  koji je smišljao da nas uništi, da nas ne bude nigdje u svemu  izraelskom području, 
\par 6 od njegovih potomaka neka nam se preda  sedam ljudi da ih objesimo pred Jahvom u Gibeonu na gori Jahvinoj."  Kralj odvrati: "Dat ću vam ih." 
\par 7 Kralj poštedje Meribaala, sina Šaulova sina Jonatana,  zbog zakletve pred Jahvom koja ih je vezala, Davida i Jonatana, sina Šaulova. 
\par 8 Tako kralj uze oba sina Rispe, Ajine kćeri, koje je rodila Šaulu, Armonija i Meribaala, i svih pet sinova  Merabe, Šaulove kćeri, koje je rodila Adrielu, sinu Barzilajevu, iz Mehole. 
\par 9 Njih dade u ruke Gibeoncima, a oni ih objesiše  na gori pred Jahvom. Tako sva sedmorica poginuše zajedno, pogubljeni  prvih dana žetve, na početku ječmene žetve. 
\par 10 Rispa, Ajina kći, uze kostrijet i prostrije je za sebe  na stijeni od početka ječmene žetve sve dok nije kiša s neba  pala na mrtva tijela, i tako nije dala nebeskim pticama da se  spuštaju na njih danju ni poljskim zvijerima noću. 
\par 11 Kad su  Davidu javili što je učinila Ajina kći Rispa, Šaulova inoča, 
\par 12 ode David i uze Šaulove kosti i kosti njegova sina Jonatana  od stanovnika Jabeša Gileadskog, koji ih bijahu potajno odnijeli  s trga u Bet Šanu, gdje su ih objesili Filistejci u onaj dan  kad su Filistejci porazili Šaula na Gilboi. 
\par 13 David prenese  odande Šaulove kosti i kosti njegova sina Jonatana pa ih združi  s kostima pogubljenih. 
\par 14 I ukopaše Šaulove kosti i kosti njegova  sina Jonatana s kostima pogubljenih u zemlji Benjaminovoj, u  Seli, u grobu Šaulova oca Kiša. Pošto izvršiše sve što je kralj  zapovjedio, Bog se smilova zemlji. 
\par 15 Jednom opet nasta rat između Filistejaca i Izraelaca.  David ode u boj sa svojim ljudima te su se borili s Filistejcima  tako da se David umorio. 
\par 16 Išibenob, jedan od Rafinih potomaka, čije je koplje bilo teško tri stotine mjedenih šekela i koji  o pripasu imaše nov mač, hvastao se tada da će ubiti Davida. 
\par 17 Ali Davidu priskoči u pomoć Sarvijin sin Abišaj; udari on  Filistejca te ga ubi. Tada se Davidovi ljudi zakleše rekavši  Davidu: "Nećeš više ići s nama u boj, da ne ugasiš svjetiljke  Izraelove!" 
\par 18 Poslije toga opet izbi rat s Filistejcima u Gobu; tada  je Hušanin Sibkaj pogubio Sipaja, jednoga od Rafinih potomaka. 
\par 19 Uz to nasta rat s Filistejcima u Gobu; tada je Jairov sin  Elhanan iz Betlehema pogubio Golijata Gitejca, koji je imao kopljaču  kao tkalačko vratilo. 
\par 20 Potom opet izbi rat u Gatu, gdje je bio neki čovjek visoka  rasta: imaše taj na svakoj ruci i nozi po šest prstiju, dakle  dvadeset i četiri; i on bijaše potomak Rafin. 
\par 21 Kad je počeo  ružiti Izraela, ubi ga Jonatan, sin Davidova brata Šimeja. 
\par 22 Ta četvorica bijahu potomci istoga Rafe iz Gata, a poginuše  od ruke Davidove i od ruku njegovih slugu. 


\chapter{22}

\par 1 David upravi Jahvi riječi ove pjesme u dan kad ga je Jahve  izbavio iz ruku svih njegovih neprijatelja i iz ruke Šaulove. 
\par 2 Pjevao je: "Jahve, hridino moja, utvrdo moja, spase moj; 
\par 3 Bože moj, pećino moja kojoj se utječem, štite moj, spasenje moje, tvrđavo moja! Ti me izbavljaš od nasilja. 
\par 4 Zazvat ću Jahvu hvale predostojna i od dušmana bit ću izbavljen. 
\par 5 Valovi smrti okružiše mene, prestraviše me bujice pogubne, 
\par 6 Užad Podzemlja sputiše me, smrtonosne zamke padoše na me: 
\par 7 u nevolji zazvah Jahvu i Bogu svome zavapih. Iz svog Doma zov mi začu, i vapaj moj mu do ušiju doprije. 
\par 8 I zemlja se potrese i uzdrhta, uzdrmaše se temelji nebesa, pokrenuše se, jer On gnjevom planu. 
\par 9 Iz nosnica mu dim se diže, iz usta mu oganj liznu, ugljevlje živo od njega plamsa. 
\par 10 On nagnu nebesa i siđe, pod nogama oblaci mu mračni. 
\par 11 Na keruba stade i poletje; na krilima vjetra zaplovi. 
\par 12 Ogrnu se mrakom kao koprenom, prekri se tamnim vodama i oblacima tmastim, 
\par 13 od bljeska pred licem njegovim užga se ugljevlje plameno. 
\par 14 Jahve s neba zagrmje, Svevišnjega glas se ori. 
\par 15 Odape strijele i dušmane rasu, izbaci munje i na zemlju ih obori. 
\par 16 Morska dna se pokazaše, i temelji svijeta postaše goli od strašne prijetnje Jahvine, od olujna daha gnjeva njegova. 
\par 17 On pruži s neba ruku i mene prihvati, iz silnih voda on me izbavi. 
\par 18 Od protivnika moćnog mene oslobodi, od dušmana mojih jačih od mene. 
\par 19 Navališe na me u dan zlosretni, ali me Jahve zaštiti, 
\par 20 na polje prostrano izvede me, spasi me jer sam mu mio. 
\par 21 Po pravednosti mojoj Jahve mi uzvrati, po čistoći ruku mojih on me nagradi, 
\par 22 jer čuvah putove Jahvine, od Boga se svoga ne udaljih. 
\par 23 Odredbe njegove sve su mi pred očima, zapovijedi njegove nisam odbacio, 
\par 24 do srži odan njemu sam bio, čuvam se grijeha svakoga. 
\par 25 Jahve mi po pravdi mojoj vrati, čistoću ruku mojih vidje. 
\par 26 S prijateljem ti si prijatelj, poštenu poštenjem uzvraćaš. 
\par 27 S čovjekom čistim ti si čist, a lukavca izigravaš, 
\par 28 jer narodu poniženu spasenje donosiš a ponižavaš oči ohole. 
\par 29 Jahve, ti moju svjetiljku užižeš, Bože, tminu moju obasjavaš: 
\par 30 s tobom udaram na čete dušmanske, s Bogom svojim preskačem zidine. 
\par 31 Savršeni su puti Gospodnji, i riječ je Jahvina ognjem kušana. on je štit svima, samo on, koji se k njemu utječu. 
\par 32 Jer tko je Bog osim Jahve? Tko li je hridina osim Boga našega? 
\par 33 Taj Bog me snagom opasuje, stere mi put besprijekoran. 
\par 34 Noge mi dade brze ko u košute i postavi me na visine sigurne, 
\par 35 ruke mi za borbu uvježba i mišice da luk mjedeni napinju. 
\par 36 Daješ mi štit svoj koji spasava, tvoja me brižljivost uzvisi. 
\par 37 Pouzdanje daješ mom koraku, i noge mi više ne posrću. 
\par 38 Pognah svoje dušmane i dostigoh, i ne vratih se dok ih ne uništih. 
\par 39 Obaram ih, ne mogu se dići, padaju, pod nogama mi leže. 
\par 40 Ti me opasa snagom za borbu, a protivnike moje meni podloži. 
\par 41 Ti dušmane moje u bijeg natjera, i rasprših one koji su me mrzili. 
\par 42 Vapiju u pomoć, nikog da pomogne, vapiju Jahvi - ne odaziva se. 
\par 43 Smrvih ih kao prah na vjetru, zgazih ih ko blato na putu. 
\par 44 Ti me §izbavÄi od bune u mom narodu, postavi me glavarom pogana, puk koji ne poznavah služi mi. 
\par 45 Svaki moj šapat pokorno on sluša. Sinovi tuđinci meni laskaju, 
\par 46 sinovi tuđinski gube srčanost izlaze dršćuć iz svojih utvrda. 
\par 47 Živio Jahve! Blagoslovljena hridina moja! Neka se uzvisi Bog, spasenje moje! 
\par 48 Bog koji mi daje osvetu i narode meni pokorava. 
\par 49 Od dušmana me mojih izbavljaš i nad protivnike me moje izdižeš, ti mene od čovjeka silnika spasavaš. 
\par 50 Zato te slavim, Jahve, među pucima i psalam pjevam tvome Imenu: 
\par 51 umnožio si pobjede kralju svojemu, pomazaniku svome milost si iskazao, Davidu i potomstvu njegovu navijeke." 


\chapter{23}

\par 1 Ovo su posljednje Davidove riječi: "Riječ Davida, sina Jišajeva, riječ čovjeka koji je bio visoko uzdignut, pomazanika Boga Jakovljeva, pjevača pjesama Izraelovih: 
\par 2 Jahvin duh govori po meni, njegova je riječ na mom jeziku. 
\par 3 Reče mi Jakovljev Bog, reče mi Izraelova hrid: Tko vlada ljudima pravedno, i tko vlada u strahu Božjemu, 
\par 4 taj je kao jutarnja svjetlost kad ograne sunce, jutro bez oblaka, na kojem se svjetluca zemaljska trava poslije kiše. 
\par 5 Da, moja kuća stoji čvrsto pred Bogom: on je učinio vječan Savez sa mnom, u svemu dobro uređen i utvrđen. Da, on će dati da napreduje sve moje spasenje i svaka moja želja. 
\par 6 Belijalovi ljudi svi su kao trnje u pustinji, jer ih nitko ne hvata rukom. 
\par 7 Nitko ih se ne dotiče, osim gvožđem i kopljačom, i potpuno se spaljuju u ognju." 
\par 8 Ovo su imena Davidovih junaka: Išbaal, Hakmonac, prvak  među trojicom; on je zavitlao svojim kopljem protiv osam stotina  i pobio ih najedanput. 
\par 9 Za njim dolazi Eleazar, sin Dodonov, Ahoašanin, jedan od trojice junaka; on je bio s Davidom kod  Pas Damina kad su se ondje skupili Filistejci za boj, a Izraelci  se povukli pred njima. 
\par 10 Ali se on čvrsto držao i udarao Filistejce  dok mu se ruka nije ukočila i ostala kao prirasla uz mač. Jahve  je dao veliku pobjedu u onaj dan, pa se vojska vratila za Eleazarom, ali samo da pokupi plijen. 
\par 11 Za njim dolazi Šama, sin Elin, Hararac; kad su se Filistejci skupili u Lehiju, bijaše polje  puno leće, a vojska je bila pobjegla ispred Filistejaca. 
\par 12 Tada  je on stao usred polja i obranio ga i potukao Filistejce. Tako  je Jahve dao veliku pobjedu. 
\par 13 Trojica između tridesetorice jednom su krenula na put  i o početku žetve došla k Davidu u Adulamsku pećinu kad jedna  filistejska četa bijaše utaborena u Refaimskoj dolini. 
\par 14 David  je tada bio u svojoj kuli, a filistejska je posada bila tada  u Betlehemu. 
\par 15 David uzdahnu: "O, kad bi me tko napojio vodom  iz betlehemskoga studenca što je kod vrata?" 
\par 16 Tada ta tri  junaka prodriješe kroz filistejski tabor i, zahvativši vode iz  betlehemskog studenca što je kod vrata, donesoše je i dadoše  Davidu. Ali je David ne htjede piti, nego je proli kao ljevanicu  Jahvi 
\par 17 govoreći: "Ne dao mi Jahve da to učinim! Zar da pijem  krv ovih ljudi? TÓa izlažući život pogibli, donijeli su vode!"  I nije htio piti. To su, eto, učinila ta tri junaka. 
\par 18 Abišaj, Joabov brat a sin Sarvijin, bio je vojvoda nad  tridesetoricom. On je zavitlao kopljem na tri stotine, pobio  ih i proslavio se među tridesetoricom. 
\par 19 On se odlikovao među  tridesetoricom i postao njihov glavar, ali nije dostigao trojice. 
\par 20 Jojadin sin Benaja, junak iz Kabseela, bogat junačkim  djelima, ubio je dva sina Ariela iz Moaba; on je jednoga snježnog  dana sišao i ubio lava usred jame. 
\par 21 Ubio je i nekog Egipćanina, čovjeka golema stasa. Egipćanin je imao koplje u ruci, a on  izišao preda nj sa štapom: istrgavši Egipćaninu koplje iz ruke, ubi ga njegovim kopljem. 
\par 22 To je učinio Jojadin sin Benaja  i proslavio se među tridesetoricom junaka. 
\par 23 Bio je najznamenitiji  među tridesetoricom, ali one prve trojice nije dostigao; David  ga postavi za zapovjednika svoje tjelesne straže. 
\par 24 Asahel, brat Joabov, bio je među tridesetoricom. Zatim:  Elhanan, sin Dodonov, iz Betlehema; 
\par 25 Šama iz Haroda; Elika  iz Haroda; 
\par 26 Heles iz Peleta; Ira, sin Ikešev, iz Tekoe; 
\par 27 Abiezer  iz Anatota; Sibekaj iz Huše; 
\par 28 Salmon iz Ahoha; Mahraj iz Netofe; 
\par 29 Heled, sin Baanin, iz Netofe; Itaj, sin Ribajev, iz Gibeje  sinova Benjaminovih; 
\par 30 Benaja iz Pireatona; Hidaj od Gaaških  potoka; 
\par 31 Abibaal iz Bet Haarabe; Azmavet iz Bahurima; 
\par 32 Eljahba  iz Šaalbona; Jašen, sin Jonatanov; 
\par 33 Šama iz Harara; Ahiam, sin Šararov, iz Arara; 
\par 34 Elifelet, sin Ahasbajev, iz Bet Maake;  Eliam, sin Ahitofelov, iz Gilona; 
\par 35 Hesraj iz Karmela; Paaraj  iz Araba; 
\par 36 Jigeal, sin Natanov, iz Sobe; Bani iz Gada; 
\par 37 Selek  Amonac; Nahraj iz Beerota, štitonoša Sarvijina sina Joaba; 
\par 38 Ira  iz Jatira; Gareb iz Jatira; 
\par 39 Urija Hetit. Svega trideset i  sedam. 



\chapter{24}

\par 1 Još je jednom srdžba Jahvina planula na Izraelce te potakla  Davida protiv njih govoreći: "Idi, izbroj Izraelce i Judejce!" 
\par 2 I kralj zapovjedi Joabu i vojvodama koji bijahu s njim: "Obiđite  sva Izraelova plemena od Dana do Beer Šebe i popišite narod da  znam koliko ima naroda." 
\par 3 Joab odgovori kralju: "Neka Jahve, tvoj Bog, dade svome narodu još sto puta ovoliko koliko ga je  sada i neka to još vidi svojim očima moj gospodar kralj, ali  zašto moj gospodar kralj ima takvu želju?" 
\par 4 Ali kraljeva riječ bijaše jača od Joabove i od riječi  vojvoda njegove vojske. Tako Joab i vojvode odoše ispred kralja  da popišu izraelski narod. 
\par 5 Prijeđoše oni preko Jordana i počeše kod Aroera i kod  grada što leži usred doline i krenuše odande prema Gaditima i  prema Jazeru. 
\par 6 Potom dođoše u Gilead i u zemlju Hetita, u Kadeš;  zatim stigoše u Dan, a iz Dana skrenuše prema Sidonu. 
\par 7 Zatim  dođoše do tvrđave Tira i u sve gradove Hivijaca i Kanaanaca i  završiše svoj put u Negebu Judinu, u Beer Šebi. 
\par 8 Prošavši svu zemlju, vratiše se poslije devet mjeseci  i dvadeset dana u Jeruzalem. 
\par 9 Joab dade kralju popis naroda:  Izraelaca bijaše osam stotina tisuća ratnika vičnih maču, a Judejaca  pet stotina tisuća ljudi. 
\par 10 Poslije toga Davida zapeče savjest što je dao brojiti  narod pa reče Jahvi: "Veoma sam sagriješio što sam to učinio!  Ali, Jahve, oprosti tu krivicu sluzi svome, jer sam vrlo ludo  radio." 
\par 11 Kad je David ujutro ustao, već je Jahvina riječ bila  došla proroku Gadu, Davidovu vidiocu: 
\par 12 "Idi i kaži Davidu:  Ovako govori Jahve: 'Troje stavljam preda te, izaberi jedno od  toga da ti učinim!'" 
\par 13 Gad tako dođe Davidu i javi mu ovo: "Hoćeš li da dođu  tri gladne godine na tvoju zemlju, ili da tri mjeseca bježiš  pred svojim neprijateljem koji će te goniti, ili da bude tri  dana kuga u tvojoj zemlji? Sada promisli i gledaj što da odgovorim  onome koji me poslao!" 
\par 14 David odgovori Gadu: "Na velikoj sam muci! Ali neka padnemo  u ruke Jahvine, jer je veliko njegovo milosrđe, a u ljudske ruke  neka ne zapadnem!" 
\par 15 David, dakle, izabra kugu. Bilo je upravo vrijeme pšenične žetve. Jahve pusti kugu na  Izraela od jutra pa do određenoga vremena; i pomor udari na narod  i pomrije sedamdeset tisuća ljudi od Dana do Beer Šebe. 
\par 16 Kad  je anđeo pružio svoju ruku na Jeruzalem da ga uništi, sažali  se Jahvi zbog toga zla, pa reče anđelu koji je ubijao narod:  "Dosta je sada! Povuci svoju ruku!" A Jahvin je anđeo bio upravo  kod gumna Araune Jebusejca. 
\par 17 Kad David vidje anđela koji je  ubijao narod, zavapi Jahvi: "Evo, ja sam sagriješio, ja sam učinio  zlo! A oni, ovce, što su skrivili? Neka tvoja ruka padne na mene  i na moju obitelj!" 
\par 18 Istoga dana dođe Gad k Davidu i reče mu: "Idi i podigni  Jahvi žrtvenik na gumnu Araune Jebusejca!" 
\par 19 I ode David po  Gadovoj riječi, kako mu je zapovjedio Jahve. 
\par 20 Kad Arauna pogleda, opazi kralja i njegove dvorane gdje idu prema njemu. Arauna  iziđe i pokloni se kralju licem do zemlje. 
\par 21 Arauna upita:  "Zašto je moj gospodar kralj došao svome sluzi?" A David odgovori:  "Da kupi od tebe ovo gumno, da sagradi žrtvenik Jahvi, kako bi  prestao pomor u narodu." 
\par 22 Arauna reče Davidu: "Neka ga uzme moj gospodar kralj  i neka žrtvuje ono što je u njegovim očima dobro! Evo goveda  za paljenicu, mlatilice, i volujske opreme za drvo! 
\par 23 Sve to  sluga moga gospodara kralja poklanja kralju!" Još reče Arauna  kralju: "Jahve, Bog tvoj, neka ti bude milostiv!" 
\par 24 Ali kralj odgovori Arauni: "Ne, nego hoću da kupim od  tebe i da platim; neću prinositi Jahvi, svome Bogu, paljenica  koje su mi poklonjene." I tako David kupi ono gumno i goveda za pedeset srebrnih  šekela. 
\par 25 Ondje David sagradi žrtvenik Jahvi i prinese paljenice  i pričesnice. Tada se Jahve smilova zemlji i presta pomor u Izraelu. 





\end{document}