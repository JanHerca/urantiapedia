\begin{document}

\title{2 Ljetopisa}


\chapter{1}

\par 1 Salomon, sin Davidov, bio se učvrstio na prijestolju. Jahve, Bog njegov, bijaše s njim i uzvisi ga veoma. 
\par 2 Salomon se tada  obrati svem Izraelu, tisućnicima, satnicima, sucima, svim knezovima  izraelskim, glavama obitelji, 
\par 3 te se on i s njim sav Zbor popeše  na uzvišicu koja bješe u Gibeonu, jer je ondje bio Šator sastanka  što ga u pustinji podiže Mojsije, sluga Božji. 
\par 4 David bijaše  prenio Kovčeg Božji iz Kirjat Jearima do mjesta koje je sam pripravio  za nj; jer je bio podigao Šator u Jeruzalemu. 
\par 5 Tučani žrtvenik  što ga napravi Besalel, sin Hurova sina Urija, bijaše ondje pred  Prebivalištem Jahvinim, kamo dođoše Salomon i zbor da mu se obrate. 
\par 6 Ondje se Salomon pred Jahvom pope na tučani žrtvenik, koji  bješe tik do Šatora sastanka, i prinese na njemu tisuću paljenica. 
\par 7 Iste se noći Bog ukaza Salomonu i reče mu: "Traži što  da ti dadem." 
\par 8 Salomon odgovori: "Veoma si naklon bio mome  ocu Davidu i zakraljio si mene na njegovo mjesto. 
\par 9 Bože Jahve, neka se ispuni sada obećanje što si ga dao mome ocu Davidu,  jer si me zakraljio nad narodom kojega ima mnogo kao zemaljske  prašine. 
\par 10 Daj mi sada mudrost i znanje da uzmognem upravljati  ovim narodom, jer tko će upravljati tolikim narodom kao što je  ovaj tvoj!" 
\par 11 Bog reče Salomonu: "Budući da ti je to u srcu, a nisi  iskao ni bogatstva, ni blaga, ni slave, ni smrti neprijatelja  i jer nisi tražio duga života nego mudrosti i znanja kako bi  upravljao mojim narodom nad kojim te zakraljih, 
\par 12 dajem ti  mudrost i znanje. Ali ti dajem i bogatstva, blaga i slave kakve  nije imao nijedan kralj što bješe prije tebe i kakve neće imati  ni oni koji dođu poslije tebe." 
\par 13 Salomon s uzvišice u Gibeonu ode u Jeruzalem, podalje  od Šatora sastanka, i kraljevaše nad Izraelom. 
\par 14 Sakupi bojnih  kola i konjanika: imao je tisuću četiri stotine kola i dvanaest  tisuća konjanika i razmjesti ih po gradovima gdje mu bijahu kola  i kod sebe u Jeruzalemu. 
\par 15 Salomon učini da srebra i zlata  bude u Jeruzalemu izobila kao kamenja, a cedrova mnogo kao dudova  u Šefeli. 
\par 16 Konji Salomonovi bili su uvezeni iz Musrija i Koe;  kraljevski dvorani kupovahu ih u Koi za srebro. 
\par 17 Dovozila  su se i prodavala jedna bojna kola iz Egipta po šest stotina  srebrnih šekela, a konji po sto i pedeset; to bješe isto tako  za sve hetitske i aramejske kraljeve koji su ih uvozili preko  njih. 


\chapter{2}

\par 1 (1:18) Salomon naumi sagraditi Dom - jedan Imenu Jahvinu, a  drugi sebi za kraljevski dvorac. 
\par 2 (2:1) Odbroji sedamdeset tisuća nosača, osamdeset tisuća kamenolomaca  u gori i tri tisuće i šest stotina poslovođa. 
\par 3 (2:2) Tada posla ovu  poruku Hiramu, tirskomu kralju: "Kao što si mome ocu Davidu slao  cedrovine da gradi dvor gdje će živjeti, tako učini i meni. 
\par 4 (2:3) Kanim  podići Dom Imenu Jahve, svojega Boga, i posvetiti mu ga da se  diže pred njim miomirisni kad, da se uvijek postavljaju kruhovi, da se prinose paljenice jutrom i večerom, subotom, na dane mlađaka  i na blagdane Jahve, Boga našega; i tako da zauvijek ostane u  Izraelu. 
\par 5 (2:4) Dom koji gradim bit će velik, jer je naš Bog najveći  među svim bozima. 
\par 6 (2:5) TÓa tko bi imao dovoljno snage da njemu  sazda Dom kad ga ni nebesa, ni nebesa nad nebesima ne mogu obuhvatiti?  I tko sam ja da mu zidam Dom, osim zato da mu se kad diže pred  lice? 
\par 7 (2:6) Pošalji mi čovjeka vična obradi zlata, srebra, tuča, željeza, grimiza, karmezina i ljubičastog baršuna, i vična umjetnosti  rezbarstva: radit će s rukotvorcima kod mene u Judi i u Jeruzalemu, s onima što mi ih ostavi moj otac David. 
\par 8 (2:7) Pošalji mi iz Libanona  cedrovine, čempresovine i sandalovine, jer znam da tvoje sluge  umiju sjeći libanonska stabla. Moje će sluge raditi s tvojima. 
\par 9 (2:8) Morat će mi pripraviti mnogo drva, jer će kuća što je mislim  graditi biti velika i veličanstvena. 
\par 10 (2:9) Drvosječama što će obarati  stabla dajem dvadeset tisuća kora pšenice, dvadeset tisuća kora  ječma, dvadeset tisuća bata vina i dvadeset tisuća bata ulja  za izdržavanje tvojih slugu." 
\par 11 (2:10) Hiram, tirski kralj, odgovori pismom što ga posla Salomonu:  "Zato što voli svoj narod, Jahve te zakraljio nad njim." 
\par 12 (2:11) Dometnu  još i ovo: "Neka je blagoslovljen Jahve, Bog Izraelov, koji je  stvorio nebesa i zemlju. On je kralju Davidu dao mudra, pametna  i umna sina koji će jedan dom graditi Jahvi, a drugi sebi da  iz njega kraljuje. 
\par 13 (2:12) Stoga ti šaljem čovjeka mudra, vješta  i razumna, Hurama Abija, 
\par 14 (2:13) sina jedne Danovke i oca Tirca.  Umije obrađivati zlato, srebro, tuč, željezo, kamen, drvo, grimiz, ljubičasti baršun, bÓez i karmezin, umije rezbariti svakovrsne  rezbarije i zamisliti svako djelo koje mu se povjeri. On će raditi  s tvojim umjetnicima i umjetnicima moga gospodara Davida, tvoga  oca. 
\par 15 (2:14) Neka, dakle, sada moj gospodar svojim slugama pošalje  pšenice, ječma, ulja i vina kako je obećao. 
\par 16 (2:15) A mi ćemo nasjeći  stabala s Libanona koliko ti god treba i dovest ćemo ti ih na  splavima morem u Jafu, a ti ih prevezi gore u Jeruzalem." 
\par 17 (2:16) Salomon pobroji sve strance koji se zatekoše u Izraelovoj  zemlji poslije popisa što ga bijaše proveo njegov otac David  i nađe ih sto pedeset tri tisuće i šest stotina. 
\par 18 (2:17) Od njih  odredi sedamdeset tisuća nosača, osamdeset tisuća tesara u planini, tri tisuće i šest stotina ljudi da upravljaju radom naroda. 


\chapter{3}

\par 1 Salomon tada poče graditi Dom Jahvi u Jeruzalemu, na Morijskoj  gori, ondje gdje je njegov otac David imao viđenje. To je mjesto  koje je pripravio David, gumno Jebusejca Ornana. 
\par 2 Salomon otpoče  gradnju drugoga mjeseca četvrte godine svojega vladanja. 
\par 3 Ovo su temelji koje je Salomon postavio za gradnju Doma  Božjega: šezdeset lakata u duljinu - po staroj mjeri lakta -  a u širinu dvadeset lakata. 
\par 4 Trijem, koji je bio pred Domom, imao je, po širini ovoga potonjega, u dužinu dvadeset lakata, a visok je bio sto i dvadeset lakata. Obložio ga je iznutra  čistim zlatom. 
\par 5 Veliku je dvoranu obložio čempresovinom, koju  je prekrio čistim zlatom i postavio palme i cvjetne vijence. 
\par 6 Optočio je potom Dvoranu blistavim draguljima; zlato je bilo  zlato parvajimsko. 
\par 7 Prekrio je njime Dvoranu: grede, pragove, zidove i vratna krila te izrezao kerubine po zidovima. 
\par 8 Potom sazda dvoranu Svetinje nad svetinjama. Bila je,  prema hramskoj širini, dvadeset lakata duga i dvadeset lakata  široka i obloži je sa šest stotina talenata suhog zlata. 
\par 9 Za  čavle je dao na mjeru pedeset zlatnih šekela. I gornje je odaje  obložio zlatom. 
\par 10 U dvorani Svetinje nad svetinjama napravi  dva kerubina, liveno djelo. I njih obloži zlatom. 
\par 11 Krila kerubina  bila su dvadeset lakata duga: jedno krilo od pet lakata dodirivaše  hramski zid, a drugo od pet lakata doticaše krilo drugoga kerubina. 
\par 12 Tako je i krilo drugoga kerubina, od pet lakata, dodirivalo  hramski zid, a drugo mu se krilo, od pet lakata, spajalo s krilom  drugoga kerubina. 
\par 13 Raširena, krila kerubina imala su dvadeset  lakata. Stajali su kerubini uspravno, lic-a okrenutih Dvorani. 
\par 14 Napravi zastor od ljubičastog baršuna, od grimiza, karmezina  i bÓeza te na njemu izveze kerubine. 
\par 15 Pred Dvoranom napravi  dva stupa dugačka trideset i pet lakata, a glavice im na vrhu  pet lakata. 
\par 16 U Debiru on splete vijence te ih postavi navrh  stupova i napravi sto mogranja koje postavi među vijence. 
\par 17 Postavi  stupove pred Hekal, jedan zdesna, drugi slijeva, te nazva Jakin  onaj zdesna, a Boaz onaj slijeva. 


\chapter{4}

\par 1 Napravi tučani žrtvenik dugačak dvadeset lakata, širok dvadeset  i visok deset. 
\par 2 Tada od rastaljene kovine izli more koje je  od ruba do ruba mjerilo deset lakata; bilo je okruglo uokolo, pet lakata visoko, a u opsegu, mjereno vrpcom, imalo je trideset  lakata. 
\par 3 Pod njim bijahu likovi volovski što ga opasivahu uokrug.  Po deset ih je bilo na jednom laktu te okruživahu more uokolo;  dva je reda bilo tih volova, salivenih s morem. 
\par 4 More je počivalo  na dvanaest volova; tri su gledala na sjever, tri na zapad, tri  na jug, tri na istok: more je stajalo na njima i svi su stražnjim  dijelom bili okrenuti unutra. 
\par 5 Bilo je debelo pedalj, rub mu  kao rub u čaše, kao cvijet, a moglo je primiti tri tisuće bata. 
\par 6 Napravi deset umivaonika i postavi ih pet zdesna, pet slijeva  da se u njima pere; u njima su prali što je trebalo za paljenice;  more je bilo namijenjeno svećenicima da se umivaju u njemu. 
\par 7 Napravi  deset zlatnih svijećnjaka prema propisu i stavi ih u Hekal, pet  s desne strane, pet s lijeve. 
\par 8 Onda napravi deset stolova i  postavi ih u Hekalu, pet zdesna, a pet slijeva. Napravi stotinu  zlatnih kotlića. 
\par 9 Onda načini trijem svećenički veliko dvorište  s vratima koja prevuče tučem. 
\par 10 More stavi s desne strane prema  jugoistoku. 
\par 11 Huram načini lonce, lopate i kotliće. Dovrši  sav posao što ga je obavljao kralju Salomonu za Dom Božji: 
\par 12 dva stupa; dvije glavice što su bile navrh stupova; dva  opleta da prekriju dvije glavice što bijahu navrh stupova; 
\par 13 četiri  stotine mogranja za oba opleta; dva reda mogranja za svaki oplet  da prekriju dvije glavice navrh stupova; 
\par 14 deset podnožja i  deset umivaonika na podnožjima; 
\par 15 jedno more i dvanaest volova pod njim; 
\par 16 lonce, lopate, viljuške i sav pribor za njih napravi  od tuča Huram Abi kralju Salomonu za Dom Jahvin. 
\par 17 Kralj odredi  da ih saliju u Jordanskoj ravnici, kod gaza Adame, između Sukota  i Serede. 
\par 18 Salomon napravi tako mnogo tih predmeta da se nije  mogla izmjeriti težina tuča. 
\par 19 Onda napravi sve predmete namijenjene  Domu Božjemu: zlatni žrtvenik i stolove na kojima bjehu prineseni  kruhovi, 
\par 20 zlatne svijećnjake sa svjetiljkama od čistoga zlata  što su se, po propisu, trebale paliti pred Debirom; 
\par 21 cvjetove, svjetiljke i usekače od zlata; bilo je to čisto zlato; 
\par 22 nožice, kotliće, mašice i kadionice od čistoga zlata; ulaz u Dom, nutarnja  vrata - Svetinje nad svetinjama - i vrata Doma - Hekala - bila  su zlatna. 


\chapter{5}

\par 1 Tako bi priveden kraju posao što ga Salomon obavi za Dom Jahvin.  Salomon unese sve svete darove oca svoga Davida - srebro, zlato  i sve posuđe - i stavi ih u riznicu Božjega Doma. 
\par 2 Tada Salomon sazva u Jeruzalem sve Izraelove starješine, knezove plemenske i glavare obiteljske, da se prenese Kovčeg  saveza Jahvina iz Davidova grada, to jest sa Siona. 
\par 3 Svi se  ljudi Izraelovi sabraše pred kraljem na blagdan što je u sedmom  mjesecu. 
\par 4 Kad se sastadoše sve Izraelove starješine, leviti  ponesoše Kovčeg 
\par 5 i Šator sastanka sa svim posvećenim priborom  što bješe u Šatoru; svećenici ih i leviti prenesoše. 
\par 6 Potom kralj Salomon i sva izraelska zajednica što se sabra  k njemu žrtvovaše pred Kovčegom toliko ovaca i goveda da se ne  mogahu ni prebrojiti ni procijeniti. 
\par 7 Svećenici donesoše Kovčeg  saveza Jahvina na njegovo mjesto, u Debir Doma, to jest u Svetinju  nad svetinjama, pod krila kerubinÄa. 
\par 8 Kerubini su imali raširena  krila nad mjestom gdje stajaše Kovčeg i zaklanjahu Kovčeg i njegove  motke. 
\par 9 Motke su bile tako dugačke da su im se krajevi vidjeli  iz Svetišta nasuprot Debiru, ali se nisu vidjele izvana i ondje  stoje do dana današnjega. 
\par 10 U Kovčegu nije bilo ništa, osim  dviju ploča koje metnu Mojsije na Horebu, gdje Jahve sklopi Savez  s Izraelcima pošto iziđoše iz Egipta. 
\par 11 Svi svećenici izađoše iz Svetišta, jer su se svi nazočni  svećenici posvetili bez obzira na redove. 
\par 12 Svi levitski pjevači, Asaf, Heman, Jedutun sa sinovima i braćom, stajahu obučeni u  bÓez, s cimbalima, harfama i citrama, istočno od žrtvenika, a  s njima sto i dvadeset svećenika koji su trubili u trube. 
\par 13 I  dok su trubili i pjevali složno kao jedan i jednoglasno hvalili  i slavili Jahvu, podižući glas uz trube, cimbale i druga glazbala, hvaleći Jahvu "jer je dobar i jer je vječna njegova ljubav", oblak ispuni Dom Jahvin. 
\par 14 Svećenici ne mogoše od oblaka nastaviti  službe: slava Jahvina ispuni Božji dom! 


\chapter{6}

\par 1 Tada reče Salomon: "Jahve odluči prebivati u tmastu oblaku, 
\par 2 a ja ti sagradih uzvišen Dom da u njemu prebivaš zauvijek." 
\par 3 I, okrenuvši se, kralj blagoslovi sav izraelski zbor,  a sav je izraelski zbor stajao. 
\par 4 Reče on: "Neka je blagoslovljen Jahve, Bog Izraelov, koji svojom rukom  ispuni obećanje što ga na svoja usta dade ocu mome Davidu, rekavši: 
\par 5 'Od dana kad izvedoh svoj narod iz egipatske zemlje nisam  izabrao grada ni iz kojeg Izraelova plemena da se u njemu sagradi  Dom gdje bi prebivalo moje Ime, niti sam izabrao ikoga da vlada  nad mojim narodom izraelskim. 
\par 6 Ali sam izabrao Jeruzalem da  u njemu obitava moje Ime i odabrao Davida da zapovijeda mojem  narodu izraelskom.' 
\par 7 Otac mi David naumi podići Dom Imenu Jahve, Boga Izraelova, 
\par 8 ali mu Jahve reče: 'Naumio si podići Dom Imenu mojem, i dobro  učini, 
\par 9 ali nećeš ti podići toga Doma, nego tvoj sin koji izađe  iz tvoga krila; on će podići Dom Imenu mojem.' 
\par 10 Jahve ispuni obećanje svoje: naslijedio sam oca Davida  i sjeo na prijestolje Izraelovo, kako obeća Jahve, podigao Dom  Imenu Jahve, Boga Izraelova, 
\par 11 i namjestio Kovčeg u kojem je  Savez što ga Jahve sklopi sa sinovima Izraelovim." 
\par 12 Tada Salomon stupi, u nazočnosti svega zbora Izraelova, pred žrtvenik Jahvin i raširi ruke. 
\par 13 Salomon je, naime, bio  napravio tučano podnožje, dugo pet lakata i široko pet lakata, a visoko tri lakta, i stavio ga nasred predvorja; stavši na  nj, kleknuo je pred svim zborom Izraelovim i, raširivši ruke  k nebu, 
\par 14 rekao: "Jahve, Bože Izraelov! Nijedan ti bog nije sličan ni na nebesima  ni na zemlji, tebi koji držiš Savez i ljubav svojim slugama što  kroče pred tobom sa svim svojim srcem. 
\par 15 Sluzi svome Davidu, mojem ocu, ispunio si što si mu obećao. Što si obećao na svoja  usta, ispunio si svojom rukom upravo danas. 
\par 16 Jahve, Bože Izraelov, sada ispuni svome sluzi, ocu mome Davidu, što si mu obećao kad  si rekao: 'Neće ti preda mnom nestati nasljednika koji bi sjedio  na izraelskom prijestolju, samo ako tvoji sinovi budu čuvali  svoje putove hodeći po mojem zakonu kako si ti hodio preda mnom.' 
\par 17 Jahve, Bože Izraelov, neka se sada dakle ispuni obećanje  koje si dao svome sluzi Davidu! 
\par 18 Ali zar će Bog doista boraviti  s ljudima na zemlji? TÓa nebesa ni nebesa nad nebesima ne mogu  ga obuhvatiti, a kamoli ovaj Dom što sam ga sagradio! 
\par 19 Pomno  počuj molitvu i vapaj svoga sluge, Jahve, Bože moj, te usliši  prošnju i molitvu što je tvoj sluga k tebi upućuje! 
\par 20 Neka  tvoje oči obdan i obnoć budu otvorene nad ovim Domom, nad ovim  mjestom za koje reče da ćeš u nj smjestiti svoje Ime. Usliši  molitvu koju će sluga tvoj izmoliti na ovome mjestu. 
\par 21 I usliši molitvu sluge svoga i naroda svojega izraelskog  koju bude upravljao prema ovome mjestu. Usliši s mjesta gdje  prebivaš, s nebesa, usliši i oprosti! 
\par 22 Ako tko zgriješi protiv bližnjega i bude mu naređeno  da se zakune i zakletva dođe pred tvoj žrtvenik u ovom Domu, 
\par 23 ti je čuj s neba, postupaj i sudi svojim slugama, osudi krivca  okrećući njegova nedjela na njegovu glavu, a nevina oslobodi  postupajući s njime po nevinosti njegovoj. 
\par 24 Ako narod tvoj poraze neprijatelji jer se ogriješio o  tebe, ali se ipak k tebi obrati i proslavi Ime tvoje i u ovom  se Domu pomoli i zavapije k tebi, 
\par 25 onda ti čuj to s neba,  oprosti grijehe svojem narodu izraelskom i dovedi ga natrag u  zemlju koju si dao njima i njihovim očevima. 
\par 26 Ako se zatvori nebo i ne padne kiša jer su se ogriješili  o tebe, pa ti se pomole na ovom mjestu i proslave Ime tvoje i  obrate se od svojega grijeha kad ih ti poniziš, 
\par 27 tada čuj  s neba i oprosti grijeh svojim slugama i svojem izraelskom narodu, pokazujući mu valjan put kojim će ići, i pusti kišu na zemlju  koju si narodu svojem dao u baštinu. 
\par 28 Kad u zemlji zavlada glad, kuga, snijet i rđa, kad navale  skakavci i gusjenice, kad neprijatelj ovoga naroda pritisne koja  od njegovih vrata ili kad udari kakva druga nevolja ili boleština, 
\par 29 počuj svaku molitvu, svaki vapaj od kojega god čovjeka ili  od cijeloga tvoga naroda izraelskog; ako svaki osjeti bol u srcu  i raširi ruke k ovom Domu, 
\par 30 usliši im molitvu i vapaj njihov  u nebu gdje boraviš i oprosti i daj svakomu po njegovim putovima, jer ti poznaješ srce njegovo; jer ti jedini prozireš srca ljudi 
\par 31 da te se boje idući tvojim putovima dokle god žive na zemlji  što je ti dade našim očevima. 
\par 32 Pa i tuđinca, koji nije od tvojega naroda izraelskog, nego je stigao iz daleke zemlje radi veličine tvoga Imena i  radi tvoje snažne ruke i podignute mišice, ako dođe i pomoli  se u ovom Domu, 
\par 33 usliši s neba, gdje prebivaš, usliši sve  vapaje njegove da bi svi zemaljski narodi upoznali Ime tvoje  i bojali te se kao narod tvoj izraelski i da znaju da je tvoje  Ime prizvano nad ovaj Dom koji sam sagradio. 
\par 34 Kad narod tvoj krene na neprijatelja putem kojim ga ti  uputiš i pomoli se tebi, okrenut gradu što si ga izabrao i prema  Domu koji sam podigao tvome Imenu, 
\par 35 usliši mu s neba molitvu  i prošnju i učini mu pravdu. 
\par 36 Kad ti sagriješe, jer nema čovjeka koji ne griješi, a  ti ih, rasrdiv se na njih, predaš neprijateljima da ih zarobe  i odvedu kao roblje u daleku ili blizu zemlju, 
\par 37 pa ako se  pokaju srcem u zemlji u koju budu dovedeni te se obrate i počnu  te moliti za milost u zemlji svojih osvajača govoreći: 'Zgriješili  smo' 
\par 38 i tako se obrate tebi svim srcem i svom dušom u zemlji  svoga ropstva u koju budu dovedeni kao roblje, i pomole se okrenuti  k zemlji što je ti dade njihovim očevima i prema gradu koji si  odabrao i prema Domu što sam ga podigao tvom Imenu, 
\par 39 usliši  s neba, gdje prebivaš, njihovu molbu i njihove prošnje, učini  im pravdu i oprosti svome narodu što ti je zgriješio. 
\par 40 Sada, Bože moj, neka tvoje oči budu otvorene i tvoje  uši pažljive na molitve na ovom mjestu! 
\par 41 Pa sada ustani, o  Bože Jahve, pođi k svojem počivalištu, ti i Kovčeg tvoje snage;  neka se obuku u spasenje tvoji svećenici, o Bože Jahve, i vjerni  tvoji neka se raduju u sreći! 
\par 42 Bože Jahve, ne odvrati lica od svog pomazanika, spomeni  se milostÄi što ih dade sluzi svome Davidu!" 


\chapter{7}

\par 1 Kad Salomon dovrši molitvu, spusti se oganj s neba i spali  paljenicu i klanice i slava Jahvina ispuni Dom. 
\par 2 Svećenici  ne mogoše ući, jer slava Jahvina bješe ispunila Dom Jahvin. 
\par 3 Svi  sinovi Izraelovi, videći gdje se oganj sa slavom Jahvinom spustio  na Dom, padoše ničice k zemlji do kamenog poda; pokloniv se,  počeše slaviti Jahvu "jer je dobar i jer je vječna njegova ljubav". 
\par 4 Potom kralj i čitav narod stadoše žrtvovati žrtve pred Jahvom. 
\par 5 Kralj Salomon prinese za žrtvu dvadeset i dvije tisuće goveda, sto i dvadeset tisuća ovaca; i tako posvetiše Dom Jahvin i kralj  i sav narod. 
\par 6 Dok su svećenici stajali na dužnostima, leviti  su na glazbalima za Jahvine pjesme, što ih učini kralj David, slavili Jahvu "jer je vječna njegova ljubav". Time je David  preko njihovih ruku hvalio Jahvu. Pred njima su svećenici trubili  u trube, dok su Izraelci stajali. 
\par 7 Salomon je posvetio i sredinu predvorja koje je pred Jahvinim  Domom, jer je ondje prinio paljenice i pretilinu od pričesnica, jer na tučani žrtvenik koji bijaše napravio Salomon nisu mogle  stati paljenice ni prinosi ni pretilina. 
\par 8 U to je doba Salomon  svetkovao blagdan sedam dana i sav Izrael s njime, vrlo velik  zbor, od Ulaza u Hamat pa do Egipatskoga potoka. 
\par 9 A osmoga  su dana svetkovali svečani zbor, jer su posvetu žrtveniku svetkovali  sedam dana i blagdan sedam dana. 
\par 10 Dvadeset trećega dana sedmoga  mjeseca posla ljude k njihovim šatorima i odoše vesela i zadovoljna  srca zbog dobra koje je Jahve učinio Davidu i Salomonu i svem  narodu izraelskom. 
\par 11 Tako je Salomon dovršio Dom Jahvin i kraljevski dvor  i izveo sve što god bješe zasnovano da izvrši u Domu Jahvinu  i u svojem dvoru. 
\par 12 Potom se Jahve ukaza Salomonu noću i reče  mu: "Uslišao sam tvoju molitvu i izabrao to mjesto da mi bude  Dom žrtve. 
\par 13 Ako zatvorim nebo da ne bude dažda, ili zapovjedim  skakavcima da popasu zemlju, ili pustim kugu na svoj narod, 
\par 14 i  ponizi se moj narod na koji je prizvano Ime moje i pomoli se  i potraži lice moje i okani se zlih putova, ja ću ga tada uslišati  s neba i oprostiti mu grijeh i izliječit ću mu zemlju. 
\par 15 Moje  će oči biti otvorene i moje uši pažljive na molitvu s ovoga mjesta. 
\par 16 Sada sam, dakle, izabrao i posvetio ovaj Dom da ovdje bude  Ime moje zauvijek i ovdje će sve dane biti moje oči i moje srce. 
\par 17 A ti, budeš li išao preda mnom kako ti je išao otac David, vršeći sve što sam ti zapovjedio i držeći se mojih uredaba i  zakona, 
\par 18 uzdržat ću tvoje kraljevsko prijestolje kako sam  obećao tvome ocu Davidu govoreći: 'Neće ti ponestati nasljednika  koji bi vladao u Izraelu.' 
\par 19 Ali ako me ostavite i napustite  uredbe i zapovijedi koje sam vam dao te otiđete i počnete služiti  tuđim bogovima i klanjati im se, 
\par 20 istjerat ću Izraelce iz  svoje zemlje koju sam im dao i odbacit ću od sebe ovaj Dom koji  sam posvetio svojem Imenu i učinit ću od njega priču i sramotu  među svim narodima. 
\par 21 Tko god prođe mimo ovaj Dom koji bijaše  preslavan zaprepastit će se od užasa i pitati: 'Zašto je Jahve  tako učinio s ovom zemljom i s ovim Domom?' 
\par 22 I odgovorit će  mu se: 'Ostavili su Jahvu, Boga svojih otaca, koji ih je izveo  iz Egipta, i okrenuli se tuđinskim bogovima, i klanjali im se, i služili im, i zato je Jahve pustio na njih sve ovo zlo.'" 


\chapter{8}

\par 1 A kad je prošlo dvadeset godina, za koliko je vremena Salomon  podigao Jahvin Dom i svoj dvor, 
\par 2 posagradio je Salomon gradove, koje je dao Salomonu Hiram, i naselio ondje Izraelove sinove. 
\par 3 Potom otiđe Salomon na Sopski Hamat i osvoji ga. 
\par 4 Sagradi  Tadmor u pustinji i svakojaka mjesta za skladišta u Hamatu. 
\par 5 Sagradi  i Gornji Bet Horon i Donji Bet Horon, tvrde gradove sa zidovima, vratima i prijevornicama; 
\par 6 i Baalat, i sve gradove u kojima  je imao skladišta, sve gradove za bojna kola i gradove za konjanike  i što je god Salomon zaželio da gradi u Jeruzalemu i na Libanonu  i po svoj zemlji svojega kraljevstva. 
\par 7 Svim preostalim Hetitima, Amorejcima, Perižanima, Hivijcima  i Jebusejcima, koji nisu bili Izraelci, 
\par 8 sinovima njihovim  koji ostadoše iza njih u zemlji i koje Izraelci nisu zatrli -  Salomon nametnu tlaku do današnjega dana. 
\par 9 Sinove Izraelove  nije Salomon pretvarao u robove za posao, nego su bili vojnici, zapovjednici njegovih štitonoša i zapovjednici bojnih kola i  konjice. 
\par 10 Bili su poglavari nad upravnicima, kojih je kralj  Salomon imao dvjesta i pedeset, i upravljali su narodom. 
\par 11 Salomon preseli i faraonovu kćer iz Davidova grada u  kuću koju joj bijaše sagradio, jer je mislio: "Neće moja žena  živjeti u dvoru izraelskoga kralja Davida, jer je svet otkako  je u nj došao Kovčeg Jahvin." 
\par 12 Tada Salomon poče prinositi paljenice Jahvi na Jahvinu  žrtveniku što ga bijaše sagradio pred trijemom, 
\par 13 i to koliko  je trebalo iz dana u dan da prinese po Mojsijevoj zapovijedi, u subote, i na mlađake, i na blagdane tri puta u godini, na  Blagdan beskvasnih kruhova, i na Blagdan sedmica, i na Blagdan  sjenica. 
\par 14 Postavio je, po uredbi oca Davida, svećeničke redove  po njihovoj službi i levitske po njihovim dužnostima da pjevaju  hvale i da služe pred svećenicima, koliko treba iz dana u dan, i vratare po njihovim redovima na svakim vratima, jer je takva  bila zapovijed Božjega čovjeka Davida. 
\par 15 Nisu odstupili od  kraljeve zapovijedi za svećenike i levite ni u čemu, ni za riznice. 
\par 16 Tako se svršio sav Salomonov posao od dana kad je bio zasnovan  Dom Jahvin pa dokle ga god nije dovršio. Tako bijaše dovršen  Dom Jahvin. 
\par 17 Tada je Salomon otišao u Esjon-Geber i u Elat na morskoj  obali u zemlji edomskoj. 
\par 18 A Hiram mu je poslao po slugama  lađe i mornare vične moru te su otišli sa Salomonovim slugama  u Ofir; uzeše odande četiri stotine i pedeset talenata zlata  i donesoše ih kralju Salomonu. 


\chapter{9}

\par 1 Uto kraljica od Sabe ču glas o Salomonu; hoteći iskušati Salomona  zagonetkama, dođe u Jeruzalem s mnogobrojnom pratnjom i s devama  koje su nosile miomirise, mnogo zlata i dragulja. Došavši k Salomonu, porazgovori se s njim o svemu što joj bijaše na srcu. 
\par 2 Salomon  joj odgovori na sva pitanja; nije bilo Salomonu sakriveno ništa  da joj ne bi umio objasniti. 
\par 3 Kad kraljica od Sabe vidje njegovu mudrost, dvor koji  bijaše sagradio, 
\par 4 jela na njegovu stolu, odaje njegove i dvorane, otmjenost njegove posluge i njihova odijela, i njegove peharnike  i njihova odijela, i njegove paljenice koje je prinosio u Jahvinu  domu, zastade joj dah. 
\par 5 Tada reče kralju: "Istina je bila što sam u svojoj zemlji čula o tebi i o tvojoj  mudrosti. 
\par 6 Ali nisam htjela vjerovati što se pripovijeda dokle  god nisam došla i vidjela na svoje oči; i doista, ni pola mi  nije bilo rečeno o tvojoj velikoj mudrosti; nadvisio si glas  koji sam slušala. 
\par 7 Blago tvojim ljudima i tvojim slugama koji  stoje pred tobom i slušaju tvoju mudrost! 
\par 8 Neka je blagoslovljen  Jahve, tvoj Bog, komu si tako omilio da te postavio na svoje  prijestolje da kraljuješ umjesto Jahve, svojega Boga, jer Bog  tvoj ljubi Izraela da bi ga održao dovijeka; i zato je postavio  tebe za kralja da činiš pravo i pravicu." 
\par 9 Dala je tada kralju sto i dvadeset zlatnih talenata i  mnogo miomirisa i dragulja. Nikad više nije bilo takvih miomirisa  kakve je kraljica od Sabe dala kralju Salomonu. 
\par 10 Hiramove  sluge, koje su sa Salomonovim slugama donosile zlata iz Ofira, dovezle su također sandalovine i dragulja. 
\par 11 Kralj je napravio i citre i harfe za pjevače: nikad se  prije nisu vidjele takve stvari u zemlji judejskoj. 
\par 12 Kralj Salomon dade kraljici od Sabe što je zaželjela  i zatražila, izuzev ono što je sama donijela kralju. Potom ona  krenu i sa slugama ode u svoju zemlju. 
\par 13 Zlato što je dolazilo Salomonu svake godine bilo je teško  šest stotina šezdeset i šest zlatnih talenata, 
\par 14 osim onoga  što je dolazilo od trgovaca i putujućih prodavača. I svi su arapski  kraljevi i zemaljski upravitelji Salomonu donosili zlato i srebro. 
\par 15 Kralj Salomon načini dvjesta štitova od kovanoga zlata; za  svaki je štit upotrijebio šest stotina šekela kovanoga zlata; 
\par 16 i načini trista štitića od kovanoga zlata; za svaki je štitić  utrošio trista zlatnih šekela. Kralj ih je pohranio u kuću zvanu  Libanonska šuma. 
\par 17 Kralj je napravio i veliko prijestolje od bjelokosti  i obložio ga čistim zlatom. 
\par 18 Prijestolje je imalo šest stepenica  i zlatno podnožje sastavljeno s prijestoljem, i ručice s obiju  strana prijestolja, a kraj ručica stajala dva lava. 
\par 19 Dvanaest  je lavova stajalo s obiju strana onih šest stepenica. Takvo što  nije bilo izrađeno ni u jednom kraljevstvu. 
\par 20 Sve posude iz kojih je pio kralj Salomon bijahu zlatne  i sve posuđe u kući zvanoj Libanonska šuma bijaše od suhoga zlata;  srebro se smatralo bezvrijednim u Salomonovo vrijeme. 
\par 21 Kraljeve  su lađe išle u Taršiš s Hiramovim slugama; svake treće godine  vraćale su se i dolazile taršiške lađe donoseći zlato i srebro, slonovu kost, majmune i paune. 
\par 22 Tako je kralj Salomon natkrilio sve zemaljske kraljeve  bogatstvom i mudrošću. 
\par 23 Svi su zemaljski kraljevi željeli  vidjeti Salomona i čuti mudrost koju mu je Bog ulio u srce. 
\par 24 Svatko  mu je donosio dar, srebrno i zlatno posuđe, haljine, oružje i  miomirise, konje i mazge, iz godine u godinu. 
\par 25 Salomon je imao četiri tisuće konjskih jasala i bojnih  kola i dvanaest tisuća konjanika, koje je rasporedio po gradovima  bojnih kola i kod kralja u Jeruzalemu. 
\par 26 Vladao je nad svim kraljevima od Rijeke do zemlje filistejske  i do egipatske međe. 
\par 27 Kralj je učinio da u Jeruzalemu bude  srebra kao kamenja, a cedrova kao divljih smokava što rastu u  Judejskoj nizini. 
\par 28 Salomon je uvozio konje iz Musrija i iz  svih zemalja. 
\par 29 Ostala djela Salomonova, od prvih do posljednjih, zapisana  su u povijesti proroka Natana, u proročkoj knjizi Šilonjanina  Ahije i u proročkoj besjedi vidioca Adona o Nebatovu sinu Jeroboamu. 
\par 30 Salomon je vladao u Jeruzalemu nad svim Izraelom četrdeset  godina. 
\par 31 Potom je počinuo kod otaca i sahranili su ga u gradu  oca mu Davida, a na njegovo se mjesto zakraljio sin mu Roboam. 


\chapter{10}

\par 1 Tada Roboam ode u Šekem, jer su u Šekem došli svi Izraelci  da ga zakralje. 
\par 2 Čim to ču Nebatov sin Jeroboam - koji je bio  u Egiptu kamo bijaše pobjegao pred kraljem Salomonom - vrati  se iz Egipta, 
\par 3 jer bijahu poslali po nj i dozvali ga. Kad dođoše  Jeroboam i sav zbor Izraelov, rekoše Roboamu: 
\par 4 "Tvoj nam je  otac nametnuo težak jaram, ti nam sada olakšaj tešku službu svoga  oca i teški jaram koji metnu na nas, pa ćemo ti služiti." 
\par 5 On  im odgovori: "Za tri dana dođite opet k meni." I narod ode. 
\par 6 Tada se kralj Roboam posavjetova sa starcima koji su služili  njegovu ocu Salomonu dok je bio živ i upita ih: "Što savjetujete  da odgovorim ovome narodu?" 
\par 7 A oni mu odgovoriše: "Ako udovoljiš  tim ljudima, budeš im blagonaklon i odgovoriš im lijepim riječima, oni će ti uvijek biti sluge." 
\par 8 Ali on odbaci savjet što mu  ga dadoše starci i posavjetova se s mladićima koji su odrasli  s njim i bili mu u službi. 
\par 9 Upita ih: "Što savjetujete da odgovorim  ovomu narodu koji mi reče: 'Olakšaj jaram što nam ga nametnu  tvoj otac!'" 
\par 10 Odgovoriše mu mladići koji bijahu s njim odrasli: "Narodu  koji ti je rekao 'Tvoj nam je otac nametnuo jaram, a ti nam ga  olakšaj', odvrati ovako: 'Moj je mali prst deblji od bedara moga  oca! 
\par 11 Dakle, moj vam je otac nametnuo težak jaram, a ja ću  još otežati vaš jaram; moj vas je otac šibao bičevima, a ja ću  vas šibati bičevima sa željeznim štipavcima.'" 
\par 12 Trećega dana dođe Jeroboam i sav narod k Roboamu, jer  im kralj bijaše naredio: "Vratite se k meni trećega dana." 
\par 13 Kralj  im oštro odgovori; odbacivši savjet starijih, 
\par 14 odvrati po  savjetu mladih: "Moj je otac otežao vaš jaram, a ja ću još dometnuti  na nj; moj vas je otac šibao bičevima, a ja ću vas šibati bičevima  sa željeznim štipavcima." 
\par 15 Kralj, dakle, ne htjede poslušati  naroda, jer tako upriliči Bog da se ispuni riječ što je preko  Šilonjanina Ahije kaza Nabatovu sinu Jeroboamu. 
\par 16 Kad Izraelci vidješe gdje se kralj oglušio, odgovori  mu narod: "Kakav dio mi imamo s Davidom? Mi nemamo baštine s Jišajevim sinom! U šatore, Izraele! Sad se, Davide, brini za svoj dom!" I sav Izrael ode pod svoje šatore. 
\par 17 Roboam zavlada samo nad Izraelovim sinovima koji su živjeli  po judejskim gradovima. 
\par 18 Potom kralj Roboam posla Adorama, nadstojnika za tlaku, ali ga Izraelci kamenovaše i on umrije;  a kralj se Roboam brže-bolje pope na kola te pobježe u Jeruzalem. 
\par 19 Tako se Izrael odijelio od doma Davidova sve do danas. 


\chapter{11}

\par 1 Došavši u Jeruzalem, skupi sav dom Judin i Benjaminov, sto  i osamdeset tisuća vrsnih ratnika, da udare na Izraela i da Roboamu  vrate kraljevstvo. 
\par 2 Ali dođe Jahvina riječ Božjem čovjeku Šemaji: 
\par 3 "Kaži Salomonovu sinu Roboamu, judejskomu kralju, i svim  Izraelcima u Judinu i Benjaminovu plemenu: 
\par 4 Ovako veli Jahve:  'Ne idite se tući s braćom! Neka se svatko vrati svojoj kući, jer je ovo poteklo od mene.'" I oni poslušaše riječ Jahvinu, vratiše se i ne udariše na  Jeroboama. 
\par 5 Roboam, stolujući u Jeruzalemu, poče dizati tvrde gradove  po Judeji. 
\par 6 Tako je sagradio Betlehem, Etam, Tekou, 
\par 7 Bet  Sur, Sokon, Adulam, 
\par 8 Gat, Maresu, Zif, 
\par 9 Adorajim, Lakiš,  Azeku, 
\par 10 Soru, Ajalon i Hebron, tvrde gradove u Judinu i Benjaminovu  plemenu. 
\par 11 Utvrdivši gradove, postavi im zapovjednike i dovuče  zalihe hrane, ulja i vina; 
\par 12 u svaki pojedini grad stavi štitova  i kopalja i utvrdi ih vrlo jako. Tako je on imao Judino i Benjaminovo  pleme. 
\par 13 Svećenici i leviti, koji su bili po svem Izraelu, pristupiše  k njemu iz svih krajeva. 
\par 14 Leviti ostaviše pašnjake i posjed  te otiđoše u Judeju i Jeruzalem, jer ih je bio odbacio Jeroboam  i njegovi sinovi da ne obavljaju svećeničku službu Jahvi, 
\par 15 a  postavio je svećenike za uzvišice, za jarce i za telad koju je  napravio. 
\par 16 Za njima su iz svih izraelskih plemena dolazili  u Jeruzalem da žrtvuju Jahvi, Bogu svojih otaca, oni koji su  srcem tražili Jahvu, Boga Izraelova. 
\par 17 Tako su utvrdili judejsko  kraljevstvo i osokolili Salomonova sina Roboama za tri godine, jer su tri godine živjeli poput Davida i Salomona. 
\par 18 Roboam je sebi uzeo za žemu Mahalatu, kćer Davidova sina  Jerimota, i Abihajilu, kćer Jišajeva sina Eliaba, 
\par 19 koja mu  rodi sinove: Jeuša, Šemarju i Zahama. 
\par 20 A poslije nje oženio  se Abšalomovom kćerju Maakom, koja mu rodi Abiju, Etaja, Zizu  i Šelomita. 
\par 21 Roboam je ljubio Abšalomovu kćer Maaku više od  svih svojih žena i inoča, iako je uzeo osamnaest žena i šezdeset  inoča i rodio dvadeset i osam sinova i šezdeset kćeri. 
\par 22 I  Roboam postavi Maakina sina Abiju za poglavara i kneza nad njegovom  braćom, jer ga je naumio postaviti za kralja. 
\par 23 I, mudro radeći, razmjesti sinove po svim judejskim i Benjaminovim krajevima, po svim tvrdim gradovima, davši im hrane izobila i poženivši  ih sa mnogo žena. 


\chapter{12}

\par 1 Kad je Roboam utvrdio kraljevstvo i ojačao, napustio je Jahvin  zakon i on i sav Izrael s njim. 
\par 2 Ali pete godine Roboamova  kraljevanja navali egipatski kralj Šišak na Jeruzalem, koji se  bijaše iznevjerio Jahvi. 
\par 3 Došao je sa tisuću i dvjesta bojnih  kola i sa šezdeset tisuća konjanika, a narodu koji je došao s  njim iz Egipta - Libijcima, Sukijcima i Etiopljanima - nije bilo  broja. 
\par 4 Osvojivši tvrde judejske gradove, dopro je do Jeruzalema. 
\par 5 Tada dođe prorok Šemaja k Roboamu i judejskim knezovima, koji  se bijahu skupili u Jeruzalemu bježeći od Šišaka, i reče im:  "Ovako veli Jahve: 'Vi ste ostavili mene, pa i ja ostavljam vas  u ruke Šišaku.'" 
\par 6 Tada se poniziše izraelski knezovi i kralj  i rekoše: "Pravedan je Jahve!" 
\par 7 Kad ih Jahve vidje gdje se  poniziše, dođe njegova riječ Šemaji: "Ponizili su se; neću ih  uništiti, nego ću im uskoro dati spasenje te se moja srdžba neće  oboriti na Jeruzalem preko Šišaka. 
\par 8 Bit će mu sluge, da vide  što znači služiti meni, a što zemaljskim kraljevstvima." 
\par 9 Tako egipatski kralj Šišak navali na Jeruzalem, opljačka  blago iz Doma Jahvina i riznicu kraljeva dvora; sve je uzeo;  uze i zlatne štitove što ih bijaše napravio Salomon. 
\par 10 Namjesto  njih kralj Roboam napravi tučane štitove i povjeri ih zapovjednicima  straže koja je čuvala vrata kraljevskoga dvora. 
\par 11 Kad je god  kralj išao u Dom Jahvin, stražari su ih uzimali, a poslije ih  vraćali u stražaru. 
\par 12 Kad se, dakle, ponizio, odvratio se od njega Jahvin gnjev  te ga nije sasvim uništio, jer i u Judeji bijaše dobra. 
\par 13 Potom  se kralj Roboam utvrdi u Jeruzalemu i stade kraljevati. Roboamu  je bila četrdeset i jedna godina kad se zakraljio, a sedamnaest  je godina kraljevao u Jeruzalemu, u gradu koji Jahve izabra između  svih izraelskih plemena da ondje postavi Ime svoje. Majka mu  se zvala Naama, a bila je Amonka. 
\par 14 Činio je zlo, jer nije  pregnuo srcem da traži Jahvu. 
\par 15 Roboamova prva i posljednja  djela - i ratovi koji su se neprestano vodili između Roboama  i Jeroboama - zapisani su u povijesti proroka Šemaje i u plemenskom  popisu vidioca Adona. 
\par 16 Potom Roboam počinu sa svojim ocima  i bi sahranjen u Davidovu gradu; na njegovo se mjesto zakraljio  sin mu Abija. 


\chapter{13}

\par 1 Osamnaeste godine Jeroboamova kraljevanja zakralji se Abija  nad Judejom. 
\par 2 Tri je godine kraljevao u Jeruzalemu. Materi  mu je bilo ime Mikaja, Urielova kći iz Gabe. Tada izbi rat između  Abije i Jeroboama. 
\par 3 Abija je izašao u boj s hrabrim ratnicima, sa četiri stotine tisuća izabranih junaka; Jeroboam je svrstao  u bojni red protiv njega osam stotina tisuća ljudi, sve biranih  junaka. 
\par 4 Abija je stao na vrh Semarajimske gore u Efrajimovu gorju  i rekao: "Čujte me, Jeroboame i sav Izraele! 
\par 5 Ne znate li da  je Jahve, Bog Izraelov, predao Davidu kraljevstvo nad Izraelom  zauvijek, njemu i njegovim sinovima, osoljenim savezom? 
\par 6 Ali  se podigao Nebatov sin Jeroboam, sluga Davidova sina Salomona, i pobunio se protiv gospodara. 
\par 7 Skupili su se oko njega ljudi  praznovi i nevaljalci i stali prkositi Salomonovu sinu Roboamu, koji je bio mlad i strašljiva srca te se nije umio hrabro braniti  od njih. 
\par 8 Pa sada mislite da se možete oprijeti Jahvinu kraljevstvu  što je u ruci Davidovih sinova jer vas je veliko mnoštvo i imate  kod sebe zlatnu telad koju vam je napravio Jeroboam da vam budu  bogovi. 
\par 9 Otjerali ste Jahvine svećenike, Aronove sinove i levite, i postavili sebi svećenike kao drugi zemaljski narodi. Tko je  god došao s juncem i sa sedam ovnova, postao je svećenik vašim  ništavim bogovima. 
\par 10 Nama je Bog Jahve, nismo ga ostavili,  a svećenici koji služe Jahvi jesu Aronovi sinovi i leviti u svojem  poslu. 
\par 11 Pale Jahvi na kad paljenice svakoga jutra i svake  večeri s mirisnim kadom, postavljaju kruhove na čist stol i upaljuju  svake večeri zlatan svijećnjak sa svijećama; jer mi držimo naredbu  Jahve, svojega Boga, a vi ste ga ostavili. 
\par 12 Zato je, evo,  nama na čelu Bog i njegovi svećenici s glasnim trubama da gromko  trube protiv vas. Izraelovi sinovi, ne udarajte na Jahvu, Boga  svojih otaca, jer nećete imati sreće!" 
\par 13 Ali Jeroboam zavede zasjedu da im dođe za leđa; tako  su Judejcima bili jedni sprijeda, a zasjeda straga. 
\par 14 Kad se  Judejci obazreše, a ono, gle, boj im bješe sprijeda i otraga.  Tada zavapiše k Jahvi, a svećenici stadoše trubiti u trube. 
\par 15 Uto  Judejci snažno povikaše, a kad su počeli vikati, Bog razbi Jeroboama  i sav Izrael pred Abijom i Judejcima. 
\par 16 Izraelovi sinovi pobjegoše  pred Judejcima i Bog ih predade njima u ruke. 
\par 17 Abija je s  narodom učinio velik pokolj među njima te je od Izraela palo  pobijenih pet stotina tisuća izabranih ljudi. 
\par 18 Tako su sinovi  Izraelovi bili poniženi u to vrijeme, a Judini su sinovi ojačali, jer su se oslonili na Jahvu, Boga svojih otaca. 
\par 19 Abija je potjerao Jeroboama i osvojio od njega gradove  Betel sa selima, Ješanu sa selima i Efron sa selima. 
\par 20 Jeroboam  se više nije oporavio za Abijina života; Jahve ga je udario tako  da je umro. 
\par 21 Abija se utvrdio i uzeo sebi četrnaest žena te  je rodio dvadeset i dva sina i šesnaest kćeri. 
\par 22 A ostali Abijini  doživljaji i njegovi pothvati i besjede zapisani su u tumačenju  proroka Adona. 


\chapter{14}

\par 1 (13:23) Potom Abija počinu kraj svojih otaca. Sahraniše  ga u Davidovu gradu; na njegovo se mjesto zakralji sin mu Asa.  Za njegovih je dana zemlja bila mirna deset godina. 
\par 2 (14:1) Asa je činio što je dobro i pravo u očima Jahve, njegova Boga. 
\par 3 (14:2) Uklonio je tuđinske žrtvenike i uzvišice, polomio stupove  i razbio ašere. 
\par 4 (14:3) Naredio je Judejcima da traže Jahvu, Boga  svojih otaca, da se drže zakona i zapovijedi. 
\par 5 (14:4) Uklonio je iz  svih judejskih gradova uzvišice i sunčane stupove, a kraljevstvo  je bilo mirno za njegova vremena. 
\par 6 (14:5) Sagradio je tvrde gradove  u Judeji, jer je zemlja bila mirna. Nitko se nije zaratio na  nj onih godina, jer mu je Jahve dao mir. 
\par 7 (14:6) Zato je Asa rekao Judejcima: "Da pogradimo ove gradove  i da ih opašemo zidom i kulama, vratima i prijevornicama; još  je zemlja pred nama naša, jer smo tražili Jahvu, svoga Boga;  tražili smo ga, i on nam je dao mir odasvud uokolo!" Tako su gradili i bili sretni. 
\par 8 (14:7) Asa je imao vojske trista  tisuća ljudi između Judejaca koji su nosili štit i koplje, a  od Benjaminova plemena dvjesta i osamdeset tisuća koji su nosili  štit i zapinjali luk. Svi su bili hrabri junaci. 
\par 9 (14:8) Izašao je na njih Etiopljanin Zerah sa tisuću tisuća vojnika  i tri stotine bojnih kola i došao do Mareše. 
\par 10 (14:9) Asa je izašao  preda nj; svrstali su se u bojni red u Sefatskoj dolini kod Mareše. 
\par 11 (14:10) Asa zavapi k Jahvi, Bogu svome: "O Jahve, tebi je ništa pomoći  silnome ili nejakome! Pomozi nam, o Jahve, Bože naš, jer se na  te oslanjamo i u tvoje smo ime izišli na ovo mnoštvo! Jahve,  ti si Bog naš, ne daj snažnu čovjeku protiv sebe!" 
\par 12 (14:11) Jahve razbi Etiopljane pred Asom i pred Judejcima te  Etiopljani pobjegoše. 
\par 13 (14:12) Asa ih je s narodom koji bijaše s njim  potjerao sve do Gerara. Etiopljani su popadali, tako da nijedan  nije ostao živ jer ih je satro Jahve i njegove čete; i one su  odnijele vrlo velik plijen. 
\par 14 (14:13) Osvojile su sve gradove oko Gerara  jer je Jahvin strah došao na njih; oplijenile su sve te gradove, jer je u njima bilo mnogo plijena. 
\par 15 (14:14) Poharale su i šatore  za stoku i zaplijenile mnoštvo sitne stoke i deva; a onda su  se vratile u Jeruzalem. 


\chapter{15}

\par 1 Tada duh Božji dođe na Odedova sina Azarju. 
\par 2 On je izišao  pred Asu i rekao mu: "Čujte me, Asa i sve Judino i Benjaminovo  pleme! Jahve je s vama jer ste vi s njime; i ako ga budete tražili, naći ćete ga; ako li ga ostavite, i on će ostaviti vas. 
\par 3 Dugo  su Izraelci bili bez pravoga Boga i bez svećenika-učitelja i  bez Zakona. 
\par 4 Kad su se u nevolji obratili Jahvi, Bogu Izraelovu, i stali ga tražiti, našli su ga. 
\par 5 U ona vremena nitko nije  mogao na miru ni izlaziti ni dolaziti, jer su veliki nemiri vladali  među svim zemaljskim stanovnicima. 
\par 6 Udarao je narod na narod, grad na grad, jer ih je Jahve smeo svakojakom nevoljom. 
\par 7 Ali  vi budite hrabri i neka vam ne klonu ruke, jer ima nagrada za  vaša djela." 
\par 8 Čuvši te riječi i proročku besjedu proroka Odeda, Asa  se ohrabri i ukloni idolske gadove iz cijele Judine i Benjaminove  zemlje i iz gradova koje je bio osvojio u Efrajimovoj gori. Obnovio  je i Jahvin žrtvenik koji je bio pred Jahvinim trijemom. 
\par 9 Onda  je skupio sve Judino i Benjaminovo pleme i došljake koji su bili  kod njih od Efrajimova, Manašeova i Šimunova plemena, jer ih  je mnogo prebjeglo k njemu od Izraelaca kad su vidjeli da je  s njim Jahve, njegov Bog. 
\par 10 I skupili su se u Jeruzalemu trećega  mjeseca petnaeste godine Asina kraljevanja. 
\par 11 Onoga su dana  prinijeli Jahvi žrtve od plijena koji su dognali, sedam stotina  goveda i sedam tisuća sitne stoke. 
\par 12 Zavjetovaše se da će tražiti  Jahvu, Boga svojih otaca, svim srcem i svom dušom. 
\par 13 A tko  god ne bi tražio Jahvu, Izraelova Boga, da se pogubi, bio malen  ili velik, čovjek ili žena. 
\par 14 Zakleli su se Jahvi iza glasa  i uz gromki poklik, uz trube i rogove. 
\par 15 Svi su se Judejci  radovali zbog te zakletve jer su se iz svega srca zakleli i od  sve su ga svoje volje tražili i našli ga. Jahve im je dao mir  odasvud uokolo. 
\par 16 I svoju mater Maaku ukloni kralj Asa s vlasti jer je  bila načinila gada Ašeri. Asa je sasjekao njezina gada, satro  ga i spalio u potoku Kidronu. 
\par 17 Ali uzvišice nisu bile uklonjene  iz Izraela. Ipak je Asino srce bilo privrženo Jahvi svega njegova  života. 
\par 18 Unio je u Dom Božji posvećene darove svoga oca i  svoje: srebro i zlato i posuđe. 
\par 19 Nije bilo rata sve do trideset i pete godine Asina kraljevanja. 


\chapter{16}

\par 1 Trideset i šeste godine Asina kraljevanja navali izraelski  kralj Baša na Judeju i stade utvrđivati Ramu da spriječi svako  kretanje judejskom kralju Asi. 
\par 2 Asa tada uze srebra i zlata  iz riznice Doma Jahvina i kraljevskoga dvora i posla aramejskome  kralju Ben-Hadadu, koji je stolovao u Damasku, i poruči mu: 
\par 3 "Neka  bude savez između mene i tebe i između moga i tvoga oca; evo, šaljem ti na dar srebra i zlata, hajde, raskini savez s izraelskim  kraljem Bašom da bi otišao od mene." 
\par 4 Ben-Hadad posluša kralja Asu i posla svoje vojskovođe  na izraelske gradove te oni pokoriše Ijon, Dan, Abel Majinu i  sve Naftalijeve gradove-skladišta. 
\par 5 A kada to Baša dozna, presta  utvrđivati Ramu i obustavi posao. 
\par 6 Tada kralj Asa sazva sve  Judejce i oni odnesoše kamenje i drvo kojima je Baša utvrđivao  Ramu, pa time utvrdiše Gebu i Mispu. 
\par 7 U to vrijeme dođe vidjelac Hanani k judejskom kralju Asi  i reče mu: "Budući da si se oslonio na aramejskoga kralja, a  nisi se oslonio na Jahvu, Boga svoga, vojska aramejskoga kralja  izmakla ti je iz ruke. 
\par 8 Nisu li Etiopljani i Libijci imali  silne čete sa vrlo mnogo bojnih kola i konjanika? Pa kad si se  oslonio na Jahvu, predao ti ih je u ruke. 
\par 9 Jer Jahve svojim  očima gleda po svoj zemlji da bi se ohrabrili oni kojima je srce  iskreno prema njemu. Ludo si u tome radio, zato će se od sada  dizati ratovi na te." 
\par 10 Tada se Asa razgnjevi na vidioca i  baci ga u tamnicu, jer se razjario na nj. U to je vrijeme Asa  potlačio i neke iz naroda. 
\par 11 I eto, Asina djela, od prvoga do posljednjeg, zapisana  su u Knjizi o judejskim i izraelskim kraljevima. 
\par 12 Razbolio  se trideset i devete godine kraljevanja, od nogu, te mu se bolest  veoma pogoršala, ali ni u bolesti nije tražio Jahvu nego liječnike. 
\par 13 Tako Asa počinu sa svojim ocima i umrije četrdeset i prve  godine svoga kraljevanja. 
\par 14 Sahranili su ga u grobnici koju  bijaše iskopao sebi u Davidovu gradu i položili ga na odar što  ga bijaše napunio miomirisima i mastima, zgotovljenima mastilačkom  vještinom, i spalili mu ih vrlo mnogo. 


\chapter{17}

\par 1 Onda se na njegovo mjesto zakralji sin mu Jošafat; on pokaza  svoju silu protiv Izraela. 
\par 2 Razmjestio je vojsku po svim utvrđenim  judejskim gradovima i postavio namjesnike po judejskoj zemlji  i po Efrajimovim gradovima, koje bijaše zauzeo otac mu Asa. 
\par 3 Jahve je bio s Jošafatom jer je hodio pravim putovima  svoga oca Davida i nije tražio baala. 
\par 4 Tražio je Boga svojih  otaca i hodio po njegovim zapovijedima, ne čineći kao Izraelovi  sinovi. 
\par 5 Zato je Jahve utvrdio kraljevstvo u njegovoj ruci, pa su svi Judejci davali Jošafatu danak, tako da je stekao veliko  bogatstvo i slavu. 
\par 6 Njegovo se srce hrabrilo na Jahvinim putovima, pa je uklonio još i uzvišice i ašere iz Judeje. 
\par 7 Treće godine kraljevanja posla knezove Ben-Hajila, Obadju, Zahariju, Netanela i Miheja da uče po judejskim gradovima. 
\par 8 I  s njima levite: Šemaju, Netaniju, Zebadju, Asahela, Šemiramota, Jonatana, Adoniju, Tobiju i Tob Adoniju; a s njima svećenike  Elišamu i Jorama. 
\par 9 Poučavali su po Judeji noseći sa sobom Knjigu  Zakona Jahvina i obilazili sve judejske gradove učeći narod. 
\par 10 Jahvin je strah spopao sva zemaljska kraljevstva oko Judeje, tako da nisu smjela zaratiti na Jošafata. 
\par 11 Sami su mu neki  Filistejci donosili darove i novčani danak, a Arapi mu dogonili  sitnu stoku: po sedam tisuća i sedam stotina ovnova te sedam  tisuća i sedam stotina jaraca. 
\par 12 Tako je Jošafat sve više napredovao  dok ne postade vrlo velik. Sazidao je u Judeji kule i gradove-skladišta. 
\par 13 Imao je mnogo zaliha u judejskim gradovima, a hrabrih  junaka u Jeruzalemu. 
\par 14 Evo njihova popisa po obiteljima: od  Judina plemena tisućnici: vojvoda Adna i s njim trista tisuća  hrabrih junaka; 
\par 15 do njega vojvoda Johanan i s njim dvjesta  i osamdeset tisuća; 
\par 16 za njim Zikrijev sin Amasja, koji se  spremno stavio u Jahvinu službu, a s njim dvjesta tisuća hrabrih  junaka. 
\par 17 Od Benjaminova plemena: hrabri junak Eliada i s njim  dvjesta tisuća ljudi naoružanih lukom i štitom; 
\par 18 za njim Jehozabad  i s njim sto i osamdeset tisuća pripravnih za boj. 
\par 19 To su oni koji su služili kralju, ne brojeći one što  ih je kralj namjestio u tvrdim gradovima po svoj Judeji. 


\chapter{18}

\par 1 Jošafat je stekao veliko bogatstvo i slavu te se sprijateljio  s Ahabom. 
\par 2 Poslije nekoliko godina došao je k Ahabu u Samariju.  Ahab nakla mnogo sitne stoke i goveda njemu i ljudima što su  bili s njim i nagovaraše ga da pođe na Ramot Gilead. 
\par 3 Izraelski  kralj Ahab upita judejskoga kralja Jošafata: "Hoćeš li poći sa  mnom na Ramot Gilead?" On odgovori: "Ja sam kao i ti, moj je  narod kao i tvoj; s tobom ćemo u rat." 
\par 4 Jošafat još reče kralju izraelskom: "De, posavjetuj se  prije s Jahvom!" 
\par 5 Tada kralj izraelski sakupi proroke, njih  četiri stotine, i upita ih: "Hoćemo li zavojštiti na Ramot Gilead  ili da se okanim toga?" Oni odgovoriše: "Idi, jer će ga Bog predati  kralju u ruke." 
\par 6 Ali Jošafat upita: "Ima li ovdje još koji  prorok Jahvin da i njega upitamo?" 
\par 7 Kralj izraelski odgovori  Jošafatu: "Ima još jedan čovjek preko koga bismo mogli upitati  Jahvu, ali ga mrzim jer mi ne proriče dobra nego uvijek samo  zlo; to je Mihej, sin Jimlin." Jošafat reče: "Neka kralj ne govori  tako!" 
\par 8 Tada kralj izraelski dozva jednoga dvoranina i reče  mu: "Brže dovedi Jimlina sina Miheja!" 
\par 9 Izraelski kralj i judejski kralj Jošafat sjedili su svaki  na svojem prijestolju, u svečanim haljinama, na gumnu pred Samarijskim  vratima, a proroci proricali pred njima. 
\par 10 Kenaanin sin Sidkija  napravi sebi željezne rogove i reče: "Ovako veli Jahve: njima  ćeš bosti Aramejce dokle ih god ne zatreš." 
\par 11 Tako su i svi  drugi proroci proricali govoreći: "Idi na Ramot Gilead, uspjet  ćeš: Jahve će ga predati kralju u ruke." 
\par 12 Glasnik koji bijaše otišao da zove Miheja reče mu: "Evo, svi proroci složno proriču dobro kralju. Govori i ti kao jedan  od njih i proreci uspjeh!" 
\par 13 Ali Mihej odvrati: "Živoga mi  Jahve, govorit ću ono što mi Bog kaže!" 
\par 14 Kad dođe pred kralja, upita ga kralj: "Miheju, da pođem  u rat na Ramot Gilead ili da se okanim toga?" On odgovori: "Idite  i uspjet ćete, jer će vam se predati u ruke!" 
\par 15 Na to mu kralj  reče: "Koliko ću te puta zaklinjati da mi kažeš samo istinu u  Jahvino ime?" 
\par 16 Tada Mihej odgovori: "Sav Izrael vidim rasut po gorama kao stado bez pastira. I Jahve veli: 'Nemaju više gospodara, neka se u miru kući vrate!'" 
\par 17 Tada izraelski kralj reče Jošafatu: "Nisam li ti rekao da mi neće proreći dobro nego zlo?" 
\par 18 A Mihej reče: "Zato čujte riječ Jahvinu. Vidio sam Jahvu gdje sjedi na  prijestolju, a sva mu vojska nebeska stajaše zdesna i slijeva. 
\par 19 Jahve upita: 'Tko će zavesti izraelskoga kralja Ahaba da  otiđe i padne u Ramot Gileadu?' Jedan reče ovo, drugi ono. 
\par 20 Tada  uđe jedan duh, stade pred Jahvu i reče: 'Ja ću ga zavesti!' Jahve  ga upita: 'Kako?' 
\par 21 On odvrati: 'Izaći ću i bit ću lažljiv  duh u ustima svih njegovih proroka.' Jahve mu reče: 'Ti ćeš ga  zavesti. I uspjet ćeš. Idi i učini tako!' 
\par 22 Tako je, evo, Jahve  stavio lažljiva duha u usta tvojim prorocima; ali ti Jahve navješćuje  zlo." 
\par 23 Tada pristupi Kenaanin sin Sidkija i udari Miheja po  obrazu pitajući: "Zar je Jahvin duh mene napustio da bi govorio  s tobom?" 
\par 24 Mihej odvrati: "Vidjet ćeš onoga dana kad budeš  bježao iz sobe u sobu da se sakriješ." 
\par 25 Tada izraelski kralj  naredi: "Uhvatite Miheja i odvedite ga gradskom zapovjedniku  Amonu i kraljeviću Joašu. 
\par 26 Recite im: 'Ovako veli kralj: Bacite  ovoga u tamnicu i držite ga na suhu kruhu i vodi dok se sretno  ne vratim.'" 
\par 27 Mihej reče: "Ako se doista sretno vratiš, onda  nije Jahve govorio iz mene!" i nadoda: "Čujte, svi puci!" 
\par 28 Izraelski kralj i judejski kralj Jošafat krenuše na Ramot  Gilead. 
\par 29 Izraelski kralj reče Jošafatu: "Ja ću se preobući  i onda ući u boj, a ti ostani u svojoj odjeći!" Preobuče se tada  izraelski kralj i oni krenuše u boj. 
\par 30 Aramejski kralj naredi  zapovjednicima bojnih kola: "Ne udarajte ni na maloga ni na velikoga  nego jedino na izraelskoga kralja!" 
\par 31 Kad zapovjednici bojnih  kola ugledaše Jošafata, rekoše: "To je izraelski kralj!" I krenuše  na nj da udare. Ali Jošafat povika za pomoć te mu Jahve pomože  i odvrati ih od njega. 
\par 32 Kad zapovjednici bojnih kola vidješe  da to nije izraelski kralj, okrenuše se od njega. 
\par 33 Jedan nasumce odape i ustrijeli izraelskoga kralja između  nabora na pojasu i oklopa. Kralj reče vozaču: "Potegni uzdu i  izvedi me iz boja jer sam ranjen." 
\par 34 Boj je onoga dana bio  sve žešći, ali se izraelski kralj držao uspravno na bojnim kolima  prema Aramejcima sve do večeri. Umro je o zalasku sunca. 


\chapter{19}

\par 1 Kad se judejski kralj Jošafat sretno vrati kući u Jeruzalem, 
\par 2 iziđe preda nj Hananijev sin vidjelac Jehu i reče kralju  Jošafatu: "Zar da pomažeš bezbožniku i da ljubiš Jahvine mrzitelje?  Zato i udara na te srdžba Jahvina. 
\par 3 Ipak se našlo nešto dobro  u tebe: uklonio si ašere iz zemlje i pregnuo svim srcem da tražiš  Jahvu!" 
\par 4 Od tada je Jošafat živio u Jeruzalemu, opet zalazio među  narod od Beer Šebe do Efrajimske gore i obraćao ga Jahvi, Bogu  njegovih otaca. 
\par 5 Postavi suce u zemlji u svim tvrdim judejskim  gradovima, u svakome gradu. 
\par 6 I reče im: "Gledajte što radite, jer ne sudite u ime čovjeka nego u ime Jahve. On je s vama dok  sudite. 
\par 7 Sada, dakle, neka bude Jahvin strah nad vama; pazite  i savjesno radite, jer u Jahve, Boga našega, nema nepravde ni  osobne pristranosti, niti on prima mita." 
\par 8 Jošafat postavi levite, svećenike i poglavare izraelskih  obitelji u Jeruzalemu da izriču Jahvine sudove i da presuđuju  u sporovima. Oni su živjeli u Jeruzalemu 
\par 9 i on im dade naputke:  "Radite u Jahvinu strahu vjerno i iskrena srca. 
\par 10 Kakav god  spor iziđe pred vas od vaše braće što žive u gradovima: bilo  da su posrijedi krvna osveta, Zakon, zapovijedi, uredbe ili običaji, valja sve da im rastumačite, kako ne bi sagriješili Jahvi i  kako se njegova srdžba ne bi oborila na vas i na vašu braću.  Tako radite pa nećete sagriješiti. 
\par 11 I evo, svećenički će poglavar  Amarja biti nad vama u svim Jahvinim poslovima, a Jišmaelov sin  Zebadja, nadstojnik Judina doma, u svim kraljevskim poslovima.  Leviti će vam služiti kao pisari. Budite jaki, i na posao! Jahve  će biti s onim tko je dobar." 


\chapter{20}

\par 1 Poslije toga Moabovi i Amonovi sinovi, a s njima i neki od  Meunjana, zaratiše na Jošafata. 
\par 2 Ali Jošafat dobi ovu vijest:  "Dolazi na te veliko mnoštvo s one strane mora, iz Edoma; i eno  ga u Haseson Tamaru, to jest u En Gediju." 
\par 3 Jošafat se uplaši i stade tražiti Jahvu te oglasi post  po svoj Judeji. 
\par 4 Skupili se Judejci da traže Jahvu: dolazili  iz svih judejskih gradova da ga traže. 
\par 5 Tada Jošafat ustade  u judejskom zboru u Jeruzalemu, u Domu Jahvinu, pred novim predvorjem 
\par 6 i reče: "Jahve, Bože otaca naših, ti si Bog na nebu i vladaš nad  svim krivobožačkim kraljevstvima. U tvojoj je ruci takva sila  i jakost da se nitko ne može održati pred tobom. 
\par 7 Ti si, o  Bože naš, istjerao stanovnike ove zemlje pred svojim izraelskim  narodom i dao je zasvagda potomstvu svoga prijatelja Abrahama; 
\par 8 i nastanili su se u njoj i sagradili u njoj Svetište tvojem  Imenu govoreći: 
\par 9 'Kad navali na nas kakvo zlo, osvetni mač  ili kuga, ili glad, te kad stanemo pred ovim Domom i pred tobom, jer je tvoje Ime u ovom Domu, i zavapimo k tebi iz svoje nevolje, usliši nas i spasi.' 
\par 10 Sada, evo, Amonovi i Moabovi sinovi, i oni iz Seirske  gore, preko kojih nisi dao Izraelu da prođe kad je dolazio iz  zemlje egipatske, nego ih je obišao i nije ih zatro - 
\par 11 sada, dakle, oni nama uzvraćaju zlom, došavši da nas otjeraju s baštine  koju si nam ti dao. 
\par 12 O Bože naš, zar im nećeš suditi? Jer  u nas nema sile prema tome velikom mnoštvu koje dolazi na nas  niti mi znamo što da radimo, nego su nam oči uprte u te." 
\par 13 Svi su Judejci stajali pred Jahvom, s malom djecom, sa  ženama i sinovima. 
\par 14 Tada siđe Jahvin duh usred zbora na Jahaziela, sina Zaharije, sina Benaje, sina Jeiela, sina Matanijina - levita  od Asafovih sinova. 
\par 15 On reče: "Pozorno slušajte, svi Judejci, Jeruzalemci i ti, kralju Jošafate! Ovako vam govori Jahve: 'Ne  bojte se i ne plašite se toga velikog mnoštva, jer ovo nije vaš  rat, nego Božji. 
\par 16 Sutra siđite na njih; oni će se penjati  uz Hasiški uspon, a vi ćete ih sresti nakraj doline prema Jeruelskoj  pustinji. 
\par 17 Ne treba da se bijete; postavite se, stojte pa  gledajte kako će vam pomoći Jahve. Oj Judo i Jeruzaleme, ne bojte  se i ne plašite se; sutra iziđite pred njih, i Jahve će biti  s vama!'" 
\par 18 Tada Jošafat pade ničice na zemlju i svi Judejci i Jeruzalemci  padoše pred Jahvom da mu se poklone. 
\par 19 Potom leviti od Kehatovih  sinova i od Korahovih sinova ustadoše i počeše hvaliti na sav  glas Jahvu, Boga Izraelova. 
\par 20 Uranivši ujutro, krenuše prema pustinji Tekoi; kad su  izlazili, stade Jošafat i reče: "Čujte me, oj Judejci i Jeruzalemci, pouzdajte se u Jahvu svoga Boga i održat ćete se; pouzdajte  se u njegove proroke i budite sretni!" 
\par 21 Potom se posavjetova  s narodom i postavi Jahvine pjevače i hvalitelje koji će u svetom  ruhu ići pred naoružanim četama i pjevati: "Slavite Jahvu jer  je vječna ljubav njegova!" 
\par 22 Kad počeše klicati i pjevati pjesmu  pohvalnicu, Jahve podiže zasjedu na Amonce, Moapce i na one iz  Seirske gore koji su došli na Judu te biše razbijeni. 
\par 23 Jer  su Amonovi sinovi i Moapci ustali na one iz Seirske gore da ih  zatru i unište; a kad su svršili s onima iz Seira, stadoše udarati  jedan na drugoga te se poklaše. 
\par 24 Kad Judejci dođoše do stražare prema pustinji i obazreše  se na mnoštvo, a ono gle, mrtva tjelesa leže po zemlji; nitko  se nije spasio. 
\par 25 Tada dođe Jošafat s narodom da pokupi plijen  i nađoše ga mnogo: svakoga blaga, odjeće i dragocjenih predmeta;  naplijenili su toliko da više nisu mogli nositi; tri su dana  pljačkali plijen jer ga je bilo mnogo. 
\par 26 Četvrti se dan sakupiše  u Dolini blagoslova: ondje su hvalili Jahvu, pa se zato ono mjesto  prozvalo Emek Beraka, Dolina blagoslova, do danas. 
\par 27 Potom  se okrenuše svi Judejci i Jeruzalemci, s Jošafatom na čelu, da  se vrate u Jeruzalem u veselju, jer ih je Jahve razveselio nad  njihovim neprijateljima. 
\par 28 Došli su u Jeruzalem s harfama,  citrama i trubama u Dom Jahvin. 
\par 29 A strah Božji ušao je u sva  zemaljska kraljevstva kad su čula da je Jahve zavojštio na Izraelove  neprijatelje. 
\par 30 Tako je počinulo Jošafatovo kraljevstvo, jer  mu je Bog dao mir odasvud uokolo. 
\par 31 Jošafat je kraljevao nad Judejcima. Bilo mu je trideset  i pet godina kad se zakraljio; kraljevao je dvadeset i pet godina  u Jeruzalemu; mati mu se zvala Azuba, a bila je kći Šilhijeva. 
\par 32 Išao je putem oca Ase ne skrećući s njega nego čineći što  je pravo u Jahvinim očima. 
\par 33 Samo, uzvišice nisu bile uklonjene, jer narod još nije bio upravio svoje srce Bogu otaca. 
\par 34 Ostala  Jošafatova djela, od prvih do posljednjih, zapisana su u povijesti  Hananijeva sina Jehua i uvrštena su u Knjigu o izraelskim kraljevima. 
\par 35 Poslije toga udružio se judejski kralj Jošafat s izraelskim  kraljem Ahazjom, koji je bezbožno radio. 
\par 36 Udružio se s njim  zato da naprave lađe i da odu u Taršiš; napravili su lađe u Esjon  Geberu. 
\par 37 Dodavahuov sin Eliezer iz Mareše prorekao je protiv  Jošafata: "Budući da si se udružio s Ahazjom, Jahve će razoriti  tvoja djela." Lađe su se razbile i nisu mogle otploviti u Taršiš. 


\chapter{21}

\par 1 Jošafat počinu kraj svojih otaca i bi sahranjen uz njih u  Davidovu gradu. Na njegovo se mjesto zakraljio sin mu Joram. 
\par 2 Joram je imao šestoricu braće, Jošafatovih sinova: Azarju, Jehiela, Zahariju, Azarju, Mihaela i Šefatju. Svi su oni bili  sinovi izraelskog kralja Jošafata. 
\par 3 Otac im je dao mnoge darove  u srebru, zlatu i dragocjenostima, s utvrđenim gradovima u Judi;  kraljevstvo je dao Joramu jer je bio prvenac. 
\par 4 Stupivši na  očevo prijestolje i utvrdiv se, Joram pobi svu braću mačem, pa  i neke izraelske knezove. 
\par 5 Joramu su bile trideset i dvije  godine kad se zakraljio, a kraljevao je osam godina u Jeruzalemu. 
\par 6 Živio je poput izraelskih kraljeva, kao i dom Ahabov, jer  mu je kći Ahabova bila žena; radio je što je zlo u Jahvinim očima. 
\par 7 Ipak Jahve ne htjede razoriti kuće Davidu zbog Saveza što  ga sklopi s njim i zato što mu obeća da će dati svjetiljku njemu  i njegovim sinovima zauvijek. 
\par 8 U njegovo se vrijeme Edomci odmetnuše ispod judejske vlasti  i postaviše sebi kralja. 
\par 9 Zato Joram pođe sa svojim vojskovođama  i sa svim bojnim kolima. Diže se noću i pobi Edomce koji bijahu  opkolili njega i zapovjednike bojnih kola. 
\par 10 Ipak su se Edomci  oslobodili ispod judejske vlasti sve do danas. U isto se doba  odmetnu i Libna da ne bude pod njegovom vlašću, jer je on ostavio  Jahvu, Boga svojih otaca. 
\par 11 Još je i uzvišice napravio po judejskim  gorama, naveo na blud Jeruzalemce i zaveo Judejce. 
\par 12 Tada mu od proroka Ilije stiže pismo: "Ovako veli Jahve, Bog tvoga oca Davida: 'Kako nisi išao putovima oca Jošafata, ni putovima judejskoga kralja Ase, 
\par 13 nego si išao putovima  izraelskih kraljeva i naveo na blud Judejce i Jeruzalemce, kao  što je učinio dom Ahabov, a uz to si poubijao vlastitu braću, svoju obitelj, koji bjehu bolji od tebe: 
\par 14 evo, Jahve će svaliti  veliku nesreću na tvoj narod, na tvoje sinove, tvoje žene, na  sve tvoje imanje. 
\par 15 Oboljet ćeš od mnogih bolesti: od bolesti  u crijevima, tako da će ti crijeva izaći od bolesti koja će trajati  dane i dane.'" 
\par 16 Jahve podiže na Jorama srdžbu Filistejaca i Arapa koji  žive kraj Etiopljana. 
\par 17 Oni napadoše Judeju i osvojiše je,  porobiše sve blago što se našlo u kraljevu dvoru, pa i njegove  sinove i njegove žene, tako da nije ostao nitko, osim najmlađega  sina, Joahaza. 
\par 18 Poslije svega toga udari ga Jahve neizlječivom  crijevnom bolešću. 
\par 19 Ona je trajala dane i dane, a kad su se  navršile dvije godine, izašla su mu crijeva s bolešću te je umro  u strašnim mukama. Narod mu nije priredio mirisna paljenja, kao  što je palio njegovim ocima. 
\par 20 Bile su mu trideset i dvije godine kad se zakraljio,  a osam je godina kraljevao u Jeruzalemu. Preminuo je, a nitko  nije požalio za njim; i sahraniše ga u Davidovu gradu, ali ne  u kraljevskoj grobnici. 


\chapter{22}

\par 1 Jeruzalemci zakraljiše na njegovo mjesto najmlađeg mu sina, Ahazju, jer sve starije bijaše poubijala četa koja je s Arapima  navalila na tabor; tako se zakraljio Ahazja, sin judejskoga kralja  Jorama. 
\par 2 Bile su mu četrdeset i dvije godine kad se zakraljio.  Kraljevao je jednu godinu u Jeruzalemu. Materi mu je bilo ime  Atalija, Omrijeva kći. 
\par 3 I on je išao putovima doma Ahabova, jer ga mati zlo svjetovaše. 
\par 4 Činio je što je zlo u Jahvinim  očima, kao dom Ahabov, jer mu baš oni bijahu savjetnici poslije  očeve smrti, na njegovu propast. 
\par 5 Po njihovu je savjetu pošao  s Joramom, sinom izraelskoga kralja Ahaba, u boj na aramejskoga  kralja Hazaela u Ramot Gilead. Ali su Aramejci porazili Jorama. 
\par 6 On se vratio da se liječi u Jizreelu od rana što mu ih zadadoše  u Rami kad se borio s aramejskim kraljem Hazaelom. Joramov sin Ahazja, judejski kralj, sišao je u Jizreel da  posjeti Ahabova sina Jorama jer se Joram razbolio. 
\par 7 Ali Bog  učini da taj posjet Joramu bude na propast Ahazji. Došavši, izišao  je s Joramom na Nimšijeva sina Jehua, koga je Jahve pomazao da  iskorijeni Ahabovu kuću. 
\par 8 Dok je izvršavao osvetu nad Ahabovom  kućom, Jehu zateče judejske knezove i sinove Ahazjine braće koji  su posluživali Ahazju i pobi ih, 
\par 9 a onda krenu u potragu za  Ahazjom. Uhvatili su ga dok se krio u Samariji, doveli ga k Jehuu, koji ga smaknu. Ukopali su ga, jer su rekli: "Sin je onoga Jošafata  koji je tražio Jahvu svim srcem." Tako ne ostade nitko od Ahazjine kuće koji bi imao snage  da bude kralj. 
\par 10 Zato Ahazjina mati Atalija, vidjevši gdje  joj sin poginu, ustade i posmica sav kraljevski rod Judina plemena. 
\par 11 Ali kraljeva kći Jošeba uze Ahazjina sina Joaša; ukravši  ga između kraljevih sinova koje su ubijali, metnu ga s dojiljom  u ložnicu. Tako ga je Jošeba, kći kralja Jorama, žena svećenika  Jojade, sakrila od Atalije, jer je bila Ahazjina sestra, te nije  bio pogubljen. 
\par 12 Bio je sakriven s njima u Domu Božjem šest  godina, sve dok je zemljom vladala Atalija. 


\chapter{23}

\par 1 Sedme se godine Jojada ojunači i poče tražiti satnike: Jerohamova  sina Azarju, Johananova sina Jišmaela, Obedova sina Azarju, Adajina  sina Maaseju, Zikrijeva sina Elišafata, i sklopi s njima savez. 
\par 2 Počeše obilaziti po Judeji i skupiše levite iz svih judejskih  gradova i obiteljske glavare u Izraelu te dođoše u Jeruzalem. 
\par 3 Sav zbor sklopi savez s kraljem u Domu Božjem. Jojada im reče:  "Gle, kraljev će sin kraljevati kao što je obećao Jahve za Davidove  sinove. 
\par 4 Evo što valja da učinite: trećina vas koji subotom  ulazite u službu, i svećenici i leviti, neka budu vratari na  pragovima; 
\par 5 trećina neka bude u kraljevskom dvoru, trećina  na Jesodskim vratima, sav narod u predvorjima Doma Jahvina. 
\par 6 Nitko  neka ne ulazi u Dom Jahvin, osim svećenika i levita koji poslužuju;  oni neka ulaze jer su posvećeni. Sav narod neka se drži Jahvine  naredbe. 
\par 7 Leviti neka okruže kralja, svaki s oružjem u ruci, i tko god pokuša ući u Dom neka bude pogubljen. Budite uz kralja  kamo god pođe ili izađe." 
\par 8 Leviti i sav judejski narod učinili su sve onako kako  je naredio svećenik Jojada. Svaki je uzeo svoje ljude koji subotom  ulaze u službu s onima koji subotom izlaze. Jer svećenik Jojada  nije otpustio redova. 
\par 9 Svećenik Jojada dade satnicima koplja, štitove i oklope kralja Davida što su bili u Božjemu Domu. 
\par 10 Postavio  je sav narod, svakoga s kopljem u ruci, od južne do sjeverne  strane Doma, prema žrtveniku i prema Domu oko kralja unaokolo. 
\par 11 Tada izvedoše kraljeva sina, staviše mu krunu na glavu, dadoše  mu Svjedočanstvo i pomazaše ga za kralja. Tada Jojada i njegovi  sinovi povikaše: "Živio kralj!" 
\par 12 Kad Atalija ču viku naroda koji se skupio i hvalio kralja, dođe k narodu u Jahvin Dom. 
\par 13 Pogleda bolje, kad gle, kralj  stoji na svojem mjestu na ulazu, a pred kraljem zapovjednici  i svirači; sav puk kliče od radosti i trubi u trube, pjevači  pjevaju uz glazbala i predvode hvalospjeve. Tad Atalija razdrije  haljine i povika: "Izdaja, izdaja!" 
\par 14 Svećenik Jojada naredi  satnicima i vojnim zapovjednicima: "Izvedite je kroz redove napolje  i tko krene za njom pogubite ga mačem!" Još je svećenik dodao:  "Nemojte je smaknuti u Jahvinu Domu!" 
\par 15 Staviše ruke na nju  i kad je kroz Konjska vrata stigla do kraljevskoga dvora, ondje  je pogubiše. 
\par 16 Tada Jojada sklopi savez između Jahve, naroda i kralja  da narod bude Jahvin. 
\par 17 Potom sav narod otiđe u Baalov hram, razori ga skupa sa žrtvenicima i polomi likove; Baalova svećenika  Matana ubiše pred žrtvenicima. 
\par 18 Zatim Jojada postavi straže kod Jahvina Doma pod nadzorom  svećenika i levita, koje David bijaše porazdijelio za službu  u Jahvinu Domu da bi Jahvi prinosili paljenice, kao što je pisano  u Mojsijevu Zakonu, s veseljem i s pjesmama, kako uredi David. 
\par 19 Postavio je i vratare na vratima Jahvina Doma da ne bi ulazio  čovjek nečist od bilo čega. 
\par 20 Uzevši satnike, odličnike i uglednike  u narodu i sav puk, izveo je kralja iz Jahvina Doma, a onda su  ušli kroz gornja vrata u kraljevski dvor i posadili kralja na  kraljevsko prijestolje. 
\par 21 Sav se puk veselio, a grad se umirio, jer su Ataliju ubili mačem. 


\chapter{24}

\par 1 Joašu je bilo sedam godina kad se zakraljio, a kraljevao je  četrdeset godina u Jeruzalemu. Materi mu je bilo ime Sibja iz  Beer Šebe. 
\par 2 Joaš je činio što je pravo u Jahvinim očima dok  je bio živ svećenik Jojada. 
\par 3 Jojada ga je oženio dvjema ženama  i on je s njima imao sinova i kćeri. 
\par 4 Poslije toga nakanio  je u srcu obnoviti Jahvin Dom. 
\par 5 Skupivši svećenike i levite, reče im: "Zađite po judejskim  gradovima i kupite od svih Izraelaca novaca da se obnovi Dom  vašega Boga, od godine do godine, a vi pohitite s tim poslom."  Ali se levitima nije htjelo. 
\par 6 Zato kralj pozva poglavara Jojadu  i reče mu: "Zašto ne tražiš od levita da donose iz Judeje i iz  Jeruzalema porez koji je odredio Jahvin sluga Mojsije i Izraelov  zbor za Šator svjedočanstva? 
\par 7 Atalija i njeni sinovi bijahu  poharali Božji Dom, i sve stvari što su bile posvećene Jahvinu  Domu upotrijebili su za baale." 
\par 8 Potom kralj zapovjedi da se napravi kovčeg i stavi izvana  na vrata Jahvina Doma. 
\par 9 Oglasiše po Judi i Jeruzalemu da se  donosi Jahvi porez što ga bijaše odredio Božji sluga Mojsije  Izraelu u pustinji. 
\par 10 Obradovaše se svi knezovi i sav narod  i počeše donositi i bacati u kovčeg dok se nije napunio. 
\par 11 Leviti bi donosili kovčeg kraljevskom nadgledništvu,  i kad bi se vidjelo da ima mnogo novaca, dolazio je kraljev tajnik  i povjerenik svećeničkog poglavara te bi ispraznili kovčeg. Onda  su ga opet odnosili i stavljali na njegovo mjesto. Tako su činili  svaki dan i sabrali mnogo novca. 
\par 12 Onda su ga kralj i Jojada  davali poslovođama nad poslom oko Jahvina Doma, a oni su za plaću  unajmljivali klesare i drvodjelce da se obnovi Jahvin Dom, pa  kovače i mjedare da se popravi Jahvin Dom. 
\par 13 Poslovođe su poslovale  i popravljanje je napredovalo pod njihovom upravom; vratili su  Božji Dom u red i obnovili ga. 
\par 14 A kad su sve svršili, donijeli  su pred kralja i Jojadu novce što su ostali; od toga su napravili  posuđe za Jahvin Dom, posuđe za posluživanje, za paljenje, plitice  i druge zlatne i srebrne predmete. Paljenice su se prinosile u Jahvinu Domu bez prestanka dok  je god živio Jojada. 
\par 15 Onda je Jojada, ostarjevši i nasitivši  se života, umro u sto i tridesetoj godini. 
\par 16 Sahranili su ga  u Davidovu gradu kod kraljeva, jer je činio dobro u Izraelu i  prema Bogu i njegovu Domu. 
\par 17 Poslije Jojadine smrti došli su Judini knezovi i poklonili  se kralju. Tada ih kralj poče slušati. 
\par 18 Judejci bijahu ostavili  Jahvu, Boga otaca, i stali služiti ašerama i likovima; došla  je Božja srdžba na Judejce i na Jeruzalem za tu krivicu. 
\par 19 Slao  im je Bog proroke da ih obrate k Jahvi, oni su ih opominjali, ali oni nisu htjeli slušati. 
\par 20 Tada Božji duh napuni Jojadina  sina, svećenika Zahariju, koji, stavši poviše naroda, reče: "Ovako  veli Bog: 'Zašto kršite Jahvine zapovijedi? Zašto nećete da budete  sretni? Kako ste vi ostavili Jahvu, i on će vas ostaviti.'" 
\par 21 Ali  su se oni pobunili protiv njega i zasuli ga kamenjem po kraljevoj  zapovijedi u predvorju Jahvina Doma. 
\par 22 Ni kralj Joaš ne sjeti  se ljubavi koju mu učini otac Jojada, nego mu ubi sina; a on  je umirući rekao: "Jahve neka vidi i osveti!" 
\par 23 Kad je prošla godina dana, diže se na nj aramejska vojska  i, navalivši na Judu i Jeruzalem, pobi sve knezove u narodu i  posla sav plijen kralju u Damask. 
\par 24 Iako je aramejska vojska  bila malena po ljudstvu, ipak joj je Jahve predao u ruke vrlo  brojnu vojsku, jer ostaviše Jahvu, Boga svojih otaca. Tako su se Aramejci na Joašu osvetili. 
\par 25 Kad su otišli  od njega, ostaviv ga u teškim bolestima, pobuniše se protiv njega  njegovi časnici jer bijaše ubio sina svećenika Jojade, pa i oni  njega ubiše na postelji te je poginuo; sahranili su ga u Davidovu  gradu, ali ga nisu ukopali u kraljevskoj grobnici. 
\par 26 Evo onih  što se urotiše protiv njega: Zabad, sin Amonke Šimeate, i Jozabad, sin Moapke Šimrite. 
\par 27 A o njegovim sinovima i o velikim proroštvima  protiv njega, o obnavljanju Doma Božjega, sve je zapisano u tumačenju  Knjige o kraljevima. Na njegovo se mjesto zakraljio sin mu Amasja. 


\chapter{25}

\par 1 Amasji je bilo dvadeset i pet godina kad se zakraljio; kraljevao  je dvadeset i devet godina u Jeruzalemu. Mati mu se zvala Joadana  i bila je iz Jeruzalema. 
\par 2 Činio je što je pravo u Jahvinim  očima, ali ne svim srcem. 
\par 3 Kad je učvrstio kraljevstvo, pogubio  je časnike koji su ubili kralja, njegova oca. 
\par 4 Ali im sinova  nije pogubio, prema onome što je napisano u knjizi Zakona Mojsijeva, gdje Jahve zapovijeda: "Neka se očevi ne pogubljuju za sinove, ni sinovi za očeve, nego svatko neka gine za svoj grijeh." 
\par 5 Potom Amasja skupi Judejce i svega Judu i Benjamina, razvrsta  ih prema obiteljima, tisućnicima i satnicima. Pošto popisa od  dvadeset godina naviše, nađe trista tisuća izabranih momaka za  vojsku, vičnih koplju i štitu. 
\par 6 Među Izraelcima najmi sto tisuća  hrabrih junaka za sto srebrnih talenata. 
\par 7 Ali k njemu dođe  čovjek Božji i reče: "Kralju, neka ne ide s tobom izraelska vojska, jer Jahve nije s Izraelcima ni s Efrajimovim sinovima, 
\par 8 nego  idi ti sam, ponesi se junački u boju; inače će te oboriti Bog  pred neprijateljem, jer Bog može pomoći i oboriti." 
\par 9 Tada Amasja  upita čovjeka Božjeg: "A što će biti od sto talenata koje sam  dao izraelskim četama?" Božji čovjek odgovori: "Jahve ima da  ti dade više od toga." 
\par 10 Tada Amasja odvoji čete koje mu bijahu  došle od Efrajima, da se vrate u svoje mjesto. Ali se vojnici  razgnjeviše na Judejce i vratiše se u svoje mjesto plamteći od  srdžbe. 
\par 11 A Amasja, ohrabriv se, povede narod, ode u Slanu dolinu  i pobi deset tisuća seirskih sinova. 
\par 12 Judini su sinovi zarobili  deset tisuća živih, odveli ih na vrh hridi te ih pobacali, tako  da se svi razmrskaše. 
\par 13 Četa koju je Amasja poslao natrag da  ne ide s njima u boj harala je po judejskim gradovima od Samarije  pa do Bet Horona i pobila u njima tri tisuće ljudi i naplijenila  silan plijen. 
\par 14 Poslije toga, kad se Amasja vratio razbivši Edomce, donio  je bogove seirskih sinova, postavio ih sebi za bogove i počeo  im se klanjati i kaditi im. 
\par 15 Tada se Jahve razgnjevi na Amasju  i posla k njemu proroka koji ga upita: "Zašto tražiš bogove toga  naroda koji nisu izbavili svoga naroda iz tvoje ruke?" 
\par 16 Dok  je on to govorio, kralj ga upita: "Jesi li postavljen kralju  za savjetnika? Prestani! Zašto da te pogube?" Tada prorok ušutje, ali nadoda: "Znam da te Bog odlučio uništiti kad to činiš a  ne slušaš mojega savjeta." 
\par 17 Tada judejski kralj Amasja smisli i poruči izraelskom  kralju Joašu, sinu Jehuova sina Joahaza: "Dođi da se ogledamo!" 
\par 18 A izraelski kralj Joaš odvrati judejskom kralju Amasji: "Libanonski  je trn jedanput poslao glasnike k libanonskom cedru i poručio:  'Daj kćer mome sinu za ženu', ali su divlje zvijeri libanonske  prošle i trn izgazile. 
\par 19 Potukao si Edomce, pa ti se srce uzobijestilo  i tražiš slavu. Radije ostani kod kuće. Zašto izazivaš zlo i  hoćeš da padneš i ti i svi Judejci s tobom?" 
\par 20 Ali Amasja ne posluša, jer tako bijaše odredio Bog, da  ih preda u ruke Joašu zato što su pristali uz edomske bogove. 
\par 21 Izađe izraelski kralj Joaš te se ogledaše u boju on i judejski  kralj Amasja u Bet Šemešu u Judeji. 
\par 22 Izraelci poraziše Judejce  i oni pobjegoše pod svoj šator. 
\par 23 Izraelski kralj Joaš uhvati  u Bet Šemešu judejskog kralja Amasju, sina Joaševa, sina Joahazova, i odvede ga u Jeruzalem; onda sruši jeruzalemski zid od Efrajimovih  vrata do Ugaonih vrata, u dužini od četiri stotine lakata. 
\par 24 Uzevši  sve zlato, srebro i posuđe što se nalazilo u Domu Božjem kod  Obed Edoma i u riznici kraljevskog dvora, povrh toga i taoce, vrati se u Samariju. 
\par 25 Judejski je kralj Amasja, Joašev sin, živio još petnaest  godina poslije smrti izraelskoga kralja Joaša, Joahazova sina. 
\par 26 Ostala Amasjina djela, od prvih do posljednjih, zapisana  su u Knjizi o judejskim i izraelskim kraljevima. 
\par 27 Otkako je  Amasja ostavio Jahvu, kovala se protiv njega urota u Jeruzalemu.  Iako je pobjegao u Lakiš, poslaše za njim u Lakiš ljude koji  ga ondje ubiše. 
\par 28 Odande su ga prenijeli na konjima i sahranili  kraj njegovih otaca u Judinu gradu. 


\chapter{26}

\par 1 Tada sav judejski narod uze Uziju, komu bijaše šesnaest godina, i zakraljiše ga namjesto njegova oca Amasje. 
\par 2 On je opet sagradio  Elat vrativši ga Judeji, pošto je kralj počinuo kod svojih otaca. 
\par 3 Uziji bijaše šesnaest godina kad se zakraljio, a kraljevao  je pedeset i dvije godine u Jeruzalemu. Mati mu se zvala Jekolija, a bila je iz Jeruzalema. 
\par 4 Činio je što je pravo u Jahvinim  očima, sasvim kao i njegov otac Amasja. 
\par 5 Tražio je Boga za  života Zaharije, koji ga je učio Božjem strahu; dokle je god  tražio Jahvu, davao mu je Bog sreću. 
\par 6 On je izišao i zavojštio na Filistejce, srušio zid Gata, zid Jabne i zid Ašdoda; sagradio je mjesta po Ašdodu i Filisteji. 
\par 7 Bog mu je pomogao protiv Filistejaca i protiv Arapa, koji  su živjeli u Gur Baalu, i protiv Meunjana. 
\par 8 Amonci su davali  danak Uziji, a njegov se glas pronio do Egipta, jer se bijaše  vrlo osilio. 
\par 9 Uzija je sagradio kule u Jeruzalemu kod Ugaonih  vrata, kod Dolinskih vrata i na uglu te ih utvrdio. 
\par 10 Sagradio  je i u pustinji kule i iskopao mnogo studenaca, jer je imao mnogo  stoke i u Šefeli i po Ravnici, ratara i vinogradara u gorama  i vrtovima, jer je volio poljodjelstvo. 
\par 11 Uzija je imao vojsku vještu boju koja je išla u rat u  četama po broju kako ih je izbrojio tajnik Jeiel i nadzornik  Maasja pod upravom Hananije, jednoga od kraljevih knezova. 
\par 12 Svega  je na broj bilo, obiteljskih glavara, hrabrih junaka, dvije tisuće  i šest stotina. 
\par 13 Pod njihovom je upravom bilo silne vojske  trista sedam tisuća i pet stotina boju vičnih ratnika da pomažu  kralju protiv neprijatelja. 
\par 14 Uzija je pripravio svoj vojsci  štitove, koplja, kacige, oklope, lukove i kamenje za praćke. 
\par 15 Napravio je u Jeruzalemu vješto smišljene bojne sprave, iznašašće  nekoga graditelja, da stoje na kulama i na kruništima, da bacaju  strijele i veliko kamenje; pronio mu se glas nadaleko jer je  uživao čudesnu pomoć sve dok se nije osilio. 
\par 16 Ali kad se osilio, uzobijestilo mu se srce dotle da se  pokvario te se iznevjerio Jahvi, svome Bogu, jer je ušao u Jahvin  Hekal i počeo prinositi kad na kadionom žrtveniku. 
\par 17 Ali je  za njim ušao svećenik Azarja i s njim osamdeset Jahvinih svećenika, čestitih ljudi. 
\par 18 Oni ustadoše na kralja Uziju govoreći: "Nije  tvoje, Uzijo, da kadiš Jahvi, nego je to dužnost svećenika, Aronovih  sinova, koji su posvećeni da kade. Izlazi iz Svetišta! Iznevjerio  si se. I ne služi ti na čast pred Bogom Jahvom!" 
\par 19 Tada se  Uzija rasrdi držeći u ruci kadionicu da kadi; kad se rasrdio  na svećenike, izbi mu guba na čelu pred svećenicima u Domu Jahvinu  kraj kadionog žrtvenika. 
\par 20 Kad ga svećenički poglavar Azarja  i svi svećenici izbližega pogledaše, a ono, gle, izbila mu guba  na čelu; brže ga otjeraše odande, a i on sam pohitje da iziđe  jer ga Jahve bijaše udario. 
\par 21 Kralj Uzija ostade gubav do smrti i stanovaše u odvojenoj  kući, jer bijaše odstranjen od Doma Jahvina; njegov je sin Jotam  bio upravitelj kraljevskoga dvora i sudio je puku zemlje. 
\par 22 Ostala  Uzijina djela, od prvih do posljednjih, opisao je Amosov sin, prorok Izaija. 
\par 23 Uzija je počinuo i sahranili su ga kraj njegovih  otaca na polju kod kraljevske grobnice, rekavši: "Gubav je."  Na njegovo se mjesto zakraljio sin mu Jotam. 


\chapter{27}

\par 1 Jotamu je bilo dvadeset i pet godina kad se zakraljio, a kraljevao  je šesnaest godina u Jeruzalemu. Materi mu je bilo ime Jeruša, Sadokova kći. 
\par 2 Činio je što je pravo u Jahvinim očima, sasvim  kao otac mu Uzija; samo što nije ušao u Jahvin Hekal. Narod je  i dalje bio pokvaren. 
\par 3 Sagradio je Gornja vrata Doma Jahvina; i na Ofelskom zidu  mnogo je gradio. 
\par 4 Podigao je i gradove po Judejskoj gori, a  u šumama dvorove i kule. 
\par 5 Vojevao je s kraljem Amonovih sinova i pobijedio ih. Amonovi  su mu sinovi dali one godine sto srebrnih talenata i deset tisuća  kora pšenice i deset tisuća kora ječma. Toliko su mu Amonovi  sinovi priložili i druge i treće godine. 
\par 6 Tako se Jotam utvrdio, jer je uredio svoj život pred Jahvom, svojim Bogom. 
\par 7 Ostala su Jotamova djela i svi njegovi ratovi i putovi  zapisani u Knjizi o izraelskim i judejskim kraljevima. 
\par 8 Bilo  mu je dvadeset i pet godina kad se zakraljio. Kraljevao je šesnaest  godina u Jeruzalemu. 
\par 9 Tada Jotam počinu kod otaca i sahraniše  ga u Davidovu gradu. Na njegovo se mjesto zakralji sin mu Ahaz. 


\chapter{28}

\par 1 Ahazu je bilo dvadeset godina kad se zakraljio, a kraljevao  je šesnaest godina u Jeruzalemu, ali nije činio što je pravo  u Jahvinim očima kao što je činio njegov otac David. 
\par 2 Živio  je poput izraelskih kraljeva, pa je i likove salio baalima. 
\par 3 Sam  je prinosio kad u dolini Hinomova sina i proveo vlastite sinove  kroz oganj, po gnusnom običaju krivobožačkih naroda što ih je  Jahve protjerao pred Izraelovim sinovima. 
\par 4 Prinosio je žrtve  i kadio po uzvišicama i brežuljcima i pod svakim zelenim drvetom. 
\par 5 Zato ga Jahve, njegov Bog, predade u ruke aramejskom kralju  te ga on potuče, zarobi mu veliko mnoštvo ljudi i odvede ih u  Damask. Još je bio predan u ruke izraelskom kralju koji ga je  hametice porazio. 
\par 6 Remalijin je sin Pekah pobio među Judejcima  sto dvadeset tisuća hrabrih junaka u jedan dan, jer su bili ostavili  Jahvu, Boga svojih otaca. 
\par 7 A junak od Efrajimova plemena Zikri  pogubio je kraljeva sina Maaseju i dvorskoga upravitelja Azrikama  i Elkanu, drugoga do kralja. 
\par 8 Izraelovi su sinovi zarobili  od svoje braće dvjesta tisuća žena, sinova i kćeri, a zadobili  su i silan plijen od njih i odnijeli ga u Samariju. 
\par 9 Ondje bijaše Jahvin prorok po imenu Oded; izašao on pred  vojsku što je išla u Samariju i rekao: "Gle, Jahve, Bog vaših  otaca, razjario se na Judejce i zato ih je predao u vaše ruke  te ste ih gnjevno pobili da je do neba doprlo. 
\par 10 A sada još  mislite podjarmiti Judejce i Jeruzalemce da vam budu robovi i  robinje; a ipak, niste li i vi puni krivice prema Jahvi, svome  Bogu? 
\par 11 Zato me poslušajte sada i vratite to roblje što ga  zarobiste od svoje braće, jer će se izliti na vas Jahvin gnjev." 
\par 12 Tada ustadoše neki između glavara Efrajimovih sinova, i to Johananov sin Azarja, Mešilemotov sin Berekja i Šalumov  sin Ezekija, Hadlajev sin Amasa, na one što su se vraćali s vojske. 
\par 13 Pa im rekoše: "Nemojte dovoditi ovamo toga roblja, jer, uz  krivicu koja je na nama pred Jahvom, vi mislite još dometnuti  na naše grijehe i na našu krivicu, kao da nije dosta velika naša  krivica i jarosni gnjev na Izraelu." 
\par 14 Tada ostaviše ratnici  roblje i plijen pred knezovima i svim zborom. 
\par 15 Onda su poimence  prozvani ljudi ustali, osokolili robove, obukli sve gole u odjeću  iz plijena; a kad su ih obukli, obuli, nahranili, napojili i  namazali, poveli su na magarcima sve iznemogle i odveli ih u  palmov grad Jerihon do njihove braće, a potom se vratili u Samariju. 
\par 16 U to je doba kralj Ahaz zamolio asirske kraljeve da mu  pomognu. 
\par 17 Edomci bijahu opet navalili i porazili Judejce te ih  odveli u roblje. 
\par 18 Filistejci se raširili po gradovima Judejske  Šefele i Negeba i, zauzevši Bet-Šemeš, Ajalon, Gederot i Soko  sa selima, Timnu sa selima i Gimzo sa selima, nastanili se ondje. 
\par 19 Jahve je počeo ponižavati Judejce zbog judejskoga kralja  Ahaza, jer je Ahaz razuzdao Judejce i teško se iznevjerio Jahvi. 
\par 20 Došao je na nj asirski kralj Tiglat-Pileser i pritijesnio  ga umjesto da ga utvrdi. 
\par 21 Ahaz bijaše opljačkao Jahvin Dom, kraljevski dvor i knezove, i sve to dao asirskom kralju, ali  mu ništa nije pomoglo. 
\par 22 Dok je bio u nevolji, postao je još  nevjerniji Jahvi; takav je bio kralj Ahaz. 
\par 23 Počeo je žrtvovati  damaščanskim bogovima koji su ga porazili, misleći: "Kad bogovi  aramejskih kraljeva njima pomažu, žrtvovat ću im da bi i meni  pomagali." Ali su oni bili na propast njemu i svem Izraelu. 
\par 24 Ahaz je pokupio posuđe iz Božjega Doma, slupao ga, zatvorio  vrata Jahvina Doma i podigao žrtvenike po svim uglovima u Jeruzalemu. 
\par 25 I u svakom je pojedinom judejskom gradu podigao uzvišice  da kadi tuđim bogovima, dražeći Jahvu, Boga otaca. 
\par 26 Ostala su njegova djela i svi njegovi putovi, od prvih  do posljednjih, zapisani u Knjizi o judejskim i izraelskim kraljevima. 
\par 27 Onda je Ahaz počinuo kod otaca. Sahranili su ga u Gradu,  u Jeruzalemu, ali ga nisu unijeli u grobnicu judejskih kraljeva.  Na njegovo se mjesto zakraljio sin mu Ezekija. 


\chapter{29}

\par 1 Ezekiji je bilo dvadeset i pet godina kad se zakraljio. Kraljevao  je dvadeset i devet godina u Jeruzalemu. Materi mu je bilo ime  Abija, kći Zaharijina. 
\par 2 Činio je što je pravo u očima Jahvinim, sasvim kao i njegov otac David. 
\par 3 Prve godine prvoga mjeseca svojega kraljevanja otvorio  je vrata Doma Jahvina i popravio ih. 
\par 4 Onda je pozvao svećenike  i levite i, sabravši ih na istočni trg, 
\par 5 rekao: "Čujte me, leviti! Sada se posvetite i posvetite Dom Jahve, Boga svojih otaca, i uklonite nečist iz Svetinje. 
\par 6 Naši su  se oci iznevjerili i radili što je zlo u očima Jahve, našega  Boga. Ostavili su ga i odvratili lice od Jahvina Prebivališta, okrenuvši mu leđa. 
\par 7 Zatvorili su trijemska vrata i potrnuli  svjetiljke; nisu kadili kadom niti su prinosili paljenice u Svetištu  Izraelova Boga. 
\par 8 Zato se Jahve rasrdio na Judejce i na Jeruzalem  te je dopustio da budu zlostavljeni i da budu na užas i ruglo, kako vidite svojim očima. 
\par 9 I očevi su nam, eto, pali od mača, a sinovi, kćeri i žene zato su nam u ropstvu. 
\par 10 Sad sam, dakle, namislio u svom srcu sklopiti Savez s Jahvom, Izraelovim Bogom, da bi se odvratio od nas njegov jarosni gnjev. 
\par 11 Moja djeco, sad se nemojte lijeniti, jer vas je izabrao Jahve da stojite  pred njim, da mu služite i da mu budete službenici i da mu kadite." 
\par 12 Tada ustadoše leviti: Amasajev sin Mahat i Azarjin sin  Joel od Kehatovih sinova; od Merarijevih sinova: Abdijev sin  Kiš i Jehalelelov sin Azarja; od Geršonovaca: Zimin sin Joah  i Joahov sin Eden; 
\par 13 od Elisafanovih sinova: Šimri i Jeiel;  od Asafovih sinova: Zaharija i Matanija; 
\par 14 od Hemanovih sinova:  Jehiel i Šimej; od Jedutunovih sinova: Šemaja i Uziel. 
\par 15 Oni  skupiše braću, posvetiše se i dođoše kako je bio zapovjedio kralj  po Jahvinim riječima da očiste Jahvin Dom. 
\par 16 Svećenici uđoše  u Jahvin Dom da ga očiste. Počeli su svu nečist što su je našli  u Jahvinu Hekalu iznositi u predvorje Jahvina Doma; leviti su  je primali iznoseći je napolje na potok Kidron. 
\par 17 Počeli su  posvećivati prvoga dana prvoga mjeseca, a osmoga su dana istoga  mjeseca ušli u Jahvin trijem; posvećivali su Jahvin Dom osam  dana; šesnaestoga su dana prvoga mjeseca završili. 
\par 18 Onda su ušli kralju Ezekiji i rekli: "Očistili smo sav  Jahvin Dom: žrtvenik za paljenice sa svim njegovim priborom,  stol za prinesene kruhove sa svim njegovim priborom, 
\par 19 a sve  posuđe koje bijaše zabacio kralj Ahaz za svojega kraljevanja  i nevjere opet smo obnovili i posvetili; eno ga pred Jahvinim  žrtvenikom." 
\par 20 Tada kralj Ezekija porani, skupi gradske knezove i ode  u Jahvin Dom. 
\par 21 Dovedoše sedam mladih junaca, sedam ovnova, sedam jaganjaca, sedam jaraca za okajnicu, za kraljevstvo i  za Svetište i za Judu; on zapovjedi Aronovim sinovima svećenicima  da ih prinesu za paljenicu na Jahvinu žrtveniku. 
\par 22 Zaklavši  goveda, svećenici uzeše krv i poškropiše žrtvenik; zaklaše ovnove  i krvlju poškropiše žrtvenik, zaklaše jaganjce i krvlju poškropiše  žrtvenik. 
\par 23 Dovedoše jarce za okajnicu pred kralja i pred zbor  te metnuše ruke na njih. 
\par 24 Onda ih zaklaše svećenici i prinesoše  kao okajnicu njihovu krv na žrtveniku da izvrše obred pomirenja  za sav Izrael, jer kralj bijaše zapovjedio da se prinese paljenica  i okajnica za sav Izrael. 
\par 25 Postavio je u Jahvinu Domu levite s cimbalima, harfama  i citrama, kako bijaše zapovjedio David, kraljev vidjelac Gad  i prorok Natan, jer je od Jahve dolazila zapovijed po njegovim  prorocima. 
\par 26 Tako su leviti stajali s Davidovim glazbalima, a svećenici s trubama. 
\par 27 Tada Ezekija zapovjedi da prinesu  paljenice na žrtveniku. Kad se stala prinositi paljenica, počela  je Jahvina pjesma uz trube i uz glazbala izraelskoga kralja Davida. 
\par 28 Sav se zbor klanjao, pjevači pjevali, a trubači trubili,  i to sve dok se nije svršila paljenica. 
\par 29 Kad se svršilo prinošenje paljenice, kralj i svi koji  bijahu s njim pobožno padoše na koljena i pokloniše se. 
\par 30 Onda  su kralj i knezovi zapovjedili levitima da hvale Jahvu riječima  Davida i vidioca Asafa; oni su počeli hvaliti s najvećim veseljem  i, pavši ničice, poklonili se. 
\par 31 Tada Ezekija progovori: "Sada ste posvetili ruke Jahvi;  pristupite i donesite klanice i zahvalnice u Dom Jahvin." Sav  je zbor donio klanice i zahvalnice. Tko je god bio spremna srca, prinio je paljenice. 
\par 32 Paljenica što ih je donio zbor bijaše na broj: sedamdeset  goveda, sto ovnova, dvjesta jaganjaca, sve za paljenice Jahvi. 
\par 33 Ostalih posvećenih darova bilo je šest stotina goveda i tri  tisuće grla sitne stoke. 
\par 34 Ali je svećenika bilo premalo, tako  da nisu mogli oderati svih paljenica; zato su im pomagala braća  leviti, dok se nije svršio posao i dok se nisu posvetili drugi  svećenici, jer su leviti bili gorljiviji srcem da se posvete  nego svećenici. 
\par 35 Bilo je i mnogo paljenica s pretilinom od  pričesnica i s ljevanicama na paljenice. Tako se opet obnovila  služba u Jahvinu Domu. 
\par 36 I Ezekija se veselio, i sav narod  s njime, što je Bog spremio narodu, jer se sve to iznenada dogodilo. 


\chapter{30}

\par 1 Potom Ezekija poruči svim Izraelcima i Judejcima pa napisa  i pisma Efrajimovu i Manašeovu plemenu da dođu u Jahvin Dom u  Jeruzalem da proslave Pashu Jahvi, Izraelovu Bogu. 
\par 2 Kralj,  vijećajući s knezovima i sa svim zborom u Jeruzalemu, odluči  da slave Pashu drugoga mjeseca. 
\par 3 Toga nisu mogli učiniti u  pravo vrijeme jer nije bilo dosta posvećenih svećenika, i narod  se ne bijaše skupio u Jeruzalemu. 
\par 4 I to je bilo prÓavo u kraljevim  očima i u očima svega zbora, 
\par 5 pa su odredili da se oglasi po  svem Izraelu od Beer Šebe pa do Dana da dođu i proslave Pashu  Jahvi, Izraelovu Bogu, u Jeruzalemu, jer je premnogi nisu svetkovali  kako je propisano. 
\par 6 Tako su otišli glasnici s pismima od kralja i njegovih  knezova po svem Izraelu i Judi te su govorili po kraljevoj zapovijedi:  "Izraelovi sinovi, obratite se Jahvi, Abrahamovu, Izakovu i Izraelovu  Bogu, pa će se i on obratiti k Ostatku koji vam je ostao od ruku  asirskih kraljeva. 
\par 7 I nemojte biti kao vaši očevi i vaša braća, koji su se iznevjerili Jahvi, Bogu svojih otaca, te ih je predao  propasti, kako i sami vidite. 
\par 8 Nemojte, dakle, biti tvrdovrati  kao vaši očevi: pružite ruku Jahvi i dođite u njegovu Svetinju  koju je posvetio zauvijek i služite Jahvi, svome Bogu, pa će  odvratiti od vas svoj žestoki gnjev. 
\par 9 Ako se obratite Jahvi, vaša će braća i vaši sinovi naći milost u onih koji su ih zarobili  pa će se vratiti u ovu zemlju; jer je Jahve, vaš Bog, milostiv  i milosrdan i neće odvratiti lica od vas ako se vi obratite njemu." 
\par 10 I tako su glasnici krenuli od grada do grada po Efrajimovoj  i Manašeovoj zemlji pa do Zebuluna, a ljudi im se podsmijavali  i rugali. 
\par 11 Ipak su se neki od Ašerova, od Manašeova i od Zebulunova  plemena ponizili i došli u Jeruzalem. 
\par 12 Na Judejce je pak sišla  Božja ruka i prožela ih jednodušnošću da čine što bijaše zapovjedio  kralj i knezovi po Jahvinoj riječi. 
\par 13 Skupilo se u Jeruzalemu mnogo naroda da slave Blagdan  beskvasnih kruhova, drugoga mjeseca; zbor je bio vrlo velik. 
\par 14 Tada su ustali i uklonili žrtvenike što su bili u Jeruzalemu, uklonili sve kadionike i bacili ih u potok Kidron. 
\par 15 Onda su stali klati Pashu četrnaestoga dana drugoga mjeseca, a svećenici i leviti postidjeli se i, posvetiv se, počeli unositi  paljenice u Jahvin Dom. 
\par 16 Stali su na svoje mjesto po pravilu, po Zakonu Mojsija, čovjeka Božjeg; svećenici su škropili krvlju  primajući je iz ruku levita. 
\par 17 Kako ih bijaše mnogo u zboru  koji se nisu posvetili, leviti su klali pashalne jaganjce za  sve koji nisu bili čisti, da bi ih posvetili Jahvi. 
\par 18 Najveći  se dio naroda, mnogi od Efrajimova i Manašeova, Jisakarova i  Zebulunova plemena, nije očistio te je jeo Pashu nepropisno.  Ali se za njih pomolio Ezekija govoreći: "Blagi Jahve neka očisti  od grijeha svakoga 
\par 19 tko je upravio srce da traži Boga Jahvu, Boga svojih otaca, ako i nije čist kako dolikuje Svetištu!" 
\par 20 Jahve je uslišio Ezekiju i oprostio narodu. 
\par 21 Tako su Izraelovi sinovi koji su se zatekli u Jeruzalemu  svetkovali Blagdan beskvasnih kruhova sedam dana s velikim veseljem, a leviti i svećenici hvalili Jahvu iz dana u dan uz glazbala  za Jahvinu slavu. 
\par 22 Ezekija je hrabrio levite koji su pokazivali  divnu privrženost Jahvi. Jeli su svečanu žrtvu sedam dana, žrtvujući  žrtve pričesnice i slaveći Jahvu, Boga svojih otaca. 
\par 23 Potom je sav zbor vijećajući odlučio da svetkuje još  sedam dana; svetkovali su još sedam dana s veseljem. 
\par 24 Judejski  kralj Ezekija darovao je zboru tisuću mladih junaca i sedam tisuća  grla sitne stoke; a knezovi darivali zboru tisuću mladih junaca  i deset tisuća grla sitne stoke; tada se posvetilo mnogo svećenika. 
\par 25 Tako se proveselio sav judejski zbor, i svećenici i leviti, i sav zbor što je bio došao iz Izraela, i došljaci koji bijahu  došli iz zemlje izraelske, i stanovnici u Judeji. 
\par 26 Bilo je  veliko veselje u Jeruzalemu, jer od vremena Davidova sina Salomona, izraelskoga kralja, nije bilo tako u Jeruzalemu. 
\par 27 Onda su  ustali svećenici i leviti te blagoslovili narod: njihov je glas  bio uslišan, a njihova je molitva doprla do Božjega svetog Prebivališta  na nebu. 


\chapter{31}

\par 1 Kad se sve to svršilo, svi Izraelovi sinovi koji su se našli  ondje zađoše po judejskim gradovima te su razbijali stupove,  sjekli ašere i obarali uzvišice i žrtvenike po svem Judinu, Benjaminovu, Efrajimovu i Manašeovu plemenu dokle god nisu završili. Onda  se svi Izraelovi sinovi vratiše svaki na svoj posjed, u svoje  gradove. 
\par 2 Ezekija je opet uredio svećeničke i levitske redove po  njihovim redovima, svakoga prema njegovoj službi, svećenike i  levite, za paljenice i za pričesnice, da služe, slave i hvale  Boga na vratima Jahvina tabora. 
\par 3 Odredio je kraljevski doprinos  od svoga imanja za paljenice, za paljenice jutarnje i večernje  i za paljenice što se prinose subotom, za mlađaka i na blagdane, kako je napisano u Zakonu Jahvinu. 
\par 4 Zapovjedio je narodu,  jeruzalemskim stanovnicima, da daju dio svećenicima i levitima  da se utvrde u Zakonu Jahvinu. 
\par 5 Kad se to razglasilo, počeli  su Izraelovi sinovi donositi najboljega žita, novog vina, ulja  i meda i svakojaka poljskog priroda i donosili su obilne desetine  od svega. 
\par 6 Izraelovi i Judini sinovi, koji su živjeli u judejskim  gradovima, također su donosili desetinu od goveda i sitne stoke  i desetinu od svetih stvari posvećenih Jahvi, njihovu Bogu; donosili  su i davali sve hrpu na hrpu. 
\par 7 Trećega su mjeseca počeli slagati  u hrpe, a sedmoga su mjeseca završili. 
\par 8 Onda je došao Ezekija  s knezovima i, ugledavši hrpe, blagosloviše Jahvu i njegov izraelski  narod. 
\par 9 Potom se Ezekija propitao kod svećenika i levita za  hrpe. 
\par 10 Odgovarajući, svećenički poglavar Azarja, od Sadokova  doma, reče: "Otkako su počeli donositi ove prinose u Dom Jahvin, jedemo i siti smo, a mnogo i pretječe, jer je Jahve blagoslovio  svoj narod te je preteklo ovo mnoštvo." 
\par 11 Tada Ezekija zapovjedi da se urede sobe u Jahvinu Domu;  kad su ih spremili, 
\par 12 počeli su onamo unositi prinose, desetine  i svetinje; nad tim je bio predstojnik levit Konanija i brat  mu Šimej, drugi do njega. 
\par 13 A Jehiel, Azazja, Nahat, Asahel, Jerimot, Jozabad, Eliel, Jismakja, Mahat i Benaja biše postavljeni  kao nadglednici uz Konaniju i brata mu Šimeja, po nalogu kralja  Ezekije i Azarje, predstojnika u Božjem Domu. 
\par 14 Kore, sin levita  Jimne, vratar Istočnih vrata, bio je nad dragovoljnim Božjim  prinosima da bi prinosio Jahvine podizanice i svetinje nad svetinjama. 
\par 15 Pod njim su bili Eden, Minjamin, Ješua, Šemaja, Amarja i  Šekanija po svećeničkim gradovima da savjesno dijele svojoj braći  po njihovim redovima, kako velikome tako i malome - 
\par 16 osim  muškaraca starijih od trideset godina popisanih u rodovnicima  - svima koji su dolazili u Dom Jahvin na svoj svakidašnji posao  da obave obredne dužnosti po svojim redovima. 
\par 17 U rodovnike  su bili popisani svećenici po obiteljima i leviti od dvadeset  godina naviše po svojim službama, po svojim redovima. 
\par 18 U rodovnike  bijahu popisana sva njihova djeca, njihove žene, njihovi sinovi  i njihove kćeri, za svekoliki zbor, jer su se iskreno posvetili  svetinjama. 
\par 19 Aronovi sinovi, svećenici na poljskim pašnjacima  svojih gradova, u svakom pojedinom gradu, bijahu poimence određeni  da daju dio svakome muškarcu među svećenicima. Sve su rodovnike  sastavili leviti. 
\par 20 Ezekija je uradio tako po svoj Judeji čineći što je dobro, pravo i vjerno pred Jahvom, svojim Bogom. 
\par 21 U svakom poslu  koji je počeo za službu Božjega Doma, i u zakonu i u zapovijedi  tražeći Boga, trudio se svim svojim srcem i uspijevao. 


\chapter{32}

\par 1 Poslije tih događaja i dokaza vjernosti dođe asirski kralj  Sanherib i, ušavši u Judeju, opkoli tvrde gradove misleći ih  osvojiti. 
\par 2 Ezekija, vidjevši gdje je došao Sanherib i kako  snuje da zavojšti na Jeruzalem, 
\par 3 posavjetova se s knezovima  i s junacima da začepi vodene izvore koji bijahu izvan grada.  Oni mu podupriješe osnovu. 
\par 4 Sabralo se mnogo naroda te su začepili  sva vrela i potok koji teče posred zemlje; govorahu: "Zašto da  asirski kraljevi nađu toliko vode kad dođu!" 
\par 5 Ezekija se osokolio, obnovio sav oboreni zid i podigao kule na njemu; izvana je sagradio  drugi zid i utvrdio Milon u Davidovu gradu; napravio je mnogo  kopalja i štitova, 
\par 6 zatim postavio vojvode nad narodom i, pozvavši  ih k sebi na trg kraj gradskih vrata, ohrabri ih ovim riječima: 
\par 7 "Budite hrabri i junaci; ne bojte se i ne plašite se asirskoga  kralja, ni svega mnoštva što je s njim, jer je s nama moćniji  nego s njim: 
\par 8 s njim je tjelesna mišica, a s nama je Jahve, Bog naš, da nam pomaže i da bije naše bojeve." Narod se uzda  u riječi judejskoga kralja Ezekije. 
\par 9 Poslije toga asirski je kralj Sanherib, dok bijaše kod  Lakiša sa svom bojnom silom, poslao sluge u Jeruzalem k judejskome  kralju Ezekiji i k svim Judejcima koji bijahu u Jeruzalemu i  poručio im: 
\par 10 "Ovako veli asirski kralj Sanherib: 'U što se  uzdate stojeći opsjednuti u Jeruzalemu? 
\par 11 Ne zavodi li vas  Ezekija da vas preda smrti od gladi i žeđi kad govori: Jahve, Bog naš, izbavit će nas iz ruke asirskoga kralja? 
\par 12 Nije li  taj Ezekija uklonio njegove uzvišice i njegove žrtvenike; i zapovjedio  Judejcima i Jeruzalemcima govoreći: Pred jednim se žrtvenikom  klanjajte i na njemu kadite! 
\par 13 Zar ne znate što sam učinio  ja i moji preci od svih zemaljskih naroda? Zar su bogovi zemaljskih  naroda mogli izbaviti svoje zemlje iz moje ruke? 
\par 14 Koji je  među svim bogovima onih naroda što su ih sasvim uništili moji  preci mogao izbaviti narod iz moje ruke, da bi mogao vaš Bog  izbaviti vas iz moje ruke?' 
\par 15 Zato nemojte da vas sada Ezekija  vara i da vas tako zavodi i ne vjerujte mu! Jer nijedan bog nikojega  naroda ili kraljevstva nije mogao izbaviti svoga naroda iz moje  ruke, ni iz ruke mojih predaka, a kamoli će vaš Bog izbaviti  vas iz moje ruke!" 
\par 16 Još su više njegove sluge napadale Boga Jahvu i njegova  slugu Ezekiju. 
\par 17 Napisao je i pismo ružeći Jahvu, Izraelova  Boga: "Kao što bogovi zemaljskih naroda nisu izbavili svojih  naroda iz moje ruke, tako neće ni Ezekijin Bog izbaviti svojega  naroda iz moje ruke." 
\par 18 I vikahu iza glasa, na judejskom jeziku, jeruzalemskom narodu koji bijaše na zidu da ga uplaše i prepadnu  kako bi osvojili grad. 
\par 19 Govorili su o jeruzalemskom Bogu kao  o bogovima zemaljskih naroda, bogovima koji su djelo čovječjih  ruku. 
\par 20 Stoga se pomoli kralj Ezekija i prorok Izaija, Amosov  sin, i zazvaše nebo u pomoć. 
\par 21 Tada Jahve posla anđela koji  uništi sve hrabre junake, zapovjednike i vojvode u vojsci asirskoga  kralja, tako da se vratio posramljen u svoju zemlju. A kad je  ušao u hram svoga boga, sasjekli su ga ondje mačem neki koji  su se rodili iz njegova krila. 
\par 22 Tako je Jahve spasio Ezekiju  i jeruzalemske stanovnike od ruke asirskoga kralja Sanheriba  i iz ruku neprijatelja, te im dao mir odasvud uokolo. 
\par 23 Mnogi  su donosili darove Jahvi u Jeruzalem i dragocjenosti judejskome  kralju Ezekiji. Poslije toga Ezekija se uzvisio u očima svih  naroda. 
\par 24 U to se vrijeme Ezekija razbolio nasmrt, ali se pomolio  Jahvi, koji mu je progovorio i učinio čudo. 
\par 25 Ali se Ezekija  nije odužio dobročinstvu koje mu je iskazano, nego se uzoholio;  stoga je došla srdžba na nj, na Judu i na Jeruzalem. 
\par 26 Ezekija  se ponizio zato što mu se bilo uzoholilo srce, i on i Jeruzalemci, pa tako nije došla na njih Jahvina srdžba za Ezekijina života. 
\par 27 Ezekija je stekao vrlo veliko bogatstvo i slavu; napravio  je riznice za srebro i zlato, za drago kamenje, za miomirise, za štitove i za svakojake dragocjene posude; 
\par 28 skladište za  prirod od žita, od novog vina i ulja, staje za svakojaku stoku, torove za stada. 
\par 29 Podigao je i gradove, imao je mnogo blaga, sitne stoke i goveda, jer mu je Bog dao vrlo veliko imanje. 
\par 30 Isti je Ezekija začepio gornji izvor Gihonske vode i  svrnuo je pravo na zapadnu stranu Davidova grada. Ezekija je  bio sretan u svakom poslu. 
\par 31 Samo kad su došli poslanici babilonskih  knezova, poslani k njemu da se propitaju za čudo koje se dogodilo  u zemlji, ostavio ga je Bog da bi ga iskušao i da bi se doznalo  sve što mu je u srcu. 
\par 32 Ostala Ezekijina djela, njegova pobožnost, zapisani su u proročkom viđenju proroka Izaije, Amosova sina, i u Knjizi o judejskim i izraelskim kraljevima. 
\par 33 Ezekija je počinuo kod svojih otaca. Sahranili su ga  na usponu kako se ide ka grobovima Davidovih sinova. Po smrti  su mu odali počast svi Judejci i Jeruzalemci. Na njegovo se mjesto  zakraljio sin mu Manaše. 


\chapter{33}

\par 1 Manašeu je bilo dvanaest godina kad se zakraljio. Pedeset  i pet godina kraljevao je u Jeruzalemu. 
\par 2 Činio je što je zlo  u Jahvinim očima, povodeći se za gnusobama naroda što ih je Jahve  protjerao pred sinovima Izraelovim. 
\par 3 Obnovio je uzvišice što  ih bijaše oborio otac mu Ezekija, podigao je žrtvenike Baalu, načinio ašere i stao se klanjati svoj nebeskoj vojsci i služiti  joj. 
\par 4 Podigao je žrtvenike i u Domu Jahvinu, za koji bijaše  rekao Jahve: "U Jeruzalemu će prebivati Ime moje zauvijek." 
\par 5 Sagradio je žrtvenike svoj nebeskoj vojsci u oba predvorja  Doma Jahvina. 
\par 6 I sinove je svoje proveo kroz oganj u dolini  Hinomova sina. Vračao je, gatao, čarao, stvorio bajače i opsjenare  i uopće učinio premnogo zla u Jahvinim očima razjarujući ga. 
\par 7 Načinio je idolski lik i posadio ga u Domu Božjem, za koji  Bog bijaše rekao Davidu i njegovu sinu Salomonu: "U ovom Domu  i u Jeruzalemu, koji sam izabrao među svim izraelskim plemenima, postavit ću Ime svoje zauvijek. 
\par 8 Neću više dati da noga Izraelaca  uzmakne iz zemlje koju sam dao u baštinu vašim očevima, samo  ako budu držali i provodili u djelo sve što sam im zapovjedio:  sav Zakon, uredbe i običaje dane preko Mojsija." 
\par 9 Ali je Manaše  zaveo Judejce i Jeruzalemce te su radili još gore nego narodi  što ih je Jahve iskorijenio pred sinovima Izraelovim. 
\par 10 Jahve  je opominjao Manašea i njegov narod, ali oni nisu poslušali. 
\par 11 Stoga je Jahve doveo na njih vojskovođe asirskoga kralja.  Uhvativši Manašea kukama, svezali su ga u dvoje mjedene verige  i odveli u Babilon. 
\par 12 Kad se našao u nevolji, počeo se moliti  za milost Jahvi, svome Bogu, ponizivši se veoma pred Bogom otaca. 
\par 13 Molio se i Bog mu se smilovao te usliša njegovu prošnju i  vrati ga u Jeruzalem u kraljevstvo. Manaše tada spozna da je  Jahve Bog. 
\par 14 Poslije toga sagradio je vanjski zid Davidovu  gradu zapadno od Gihona, od doline pa do Ribljih vrata, i opasao  zidom Ofel, izvevši ga vrlo visoko. Postavio je bojne vojvode  u svim tvrdim gradovima u Judi. 
\par 15 Osim toga, uklonio je iz Jahvina Doma tuđinske bogove, onaj idolski lik i sve žrtvenike što ih bijaše posagradio na  gori Jahvina Doma i u Jeruzalemu i sve ih baci izvan grada. 
\par 16 Zatim  opet podiže Jahvin žrtvenik i žrtvova na njemu žrtve pričesnice  i zahvalnice. Zapovjedi i Judejcima da služe Jahvi, Bogu Izraelovu. 
\par 17 Ipak je narod još žrtvovao po uzvišicama, ali samo Jahvi, svojem Bogu. 
\par 18 Ostala Manašeova djela i njegova molitva Bogu, riječi  koje su mu govorili vidioci u ime Jahve, Izraelova Boga, zapisane  su u Povijesti izraelskih kraljeva. 
\par 19 Njegova molitva i kako  je bio uslišan, svi njegovi grijesi i njegova nevjera te mjesta  na kojima je pogradio uzvišice, podigao ašere i idole prije nego  što se ponizio - sve je to zapisano u povijesti Hozajevoj. 
\par 20 Tada  Manaše počinu kraj svojih otaca. Sahranili su ga u dvoru. Na  njegovo se mjesto zakraljio sim mu Amon. 
\par 21 Dvadeset su i dvije godine bile Amonu kad se zakraljio, a kraljevao je dvije godine u Jeruzalemu. 
\par 22 Činio je što je  zlo u Jahvinim očima, kao i otac mu Manaše, jer je svim idolima  koje bijaše načinio njegov otac Manaše on prinosio žrtve i služio  im. 
\par 23 Ali se nije ponizio pred Jahvom kako se ponizio otac  mu Manaše, nego je još i umnožio svoju krivicu. 
\par 24 Tada se protiv  njega urotiše njegove sluge i ubiše ga u dvoru. 
\par 25 Ali je prosti  puk pobio sve one koji se bijahu urotili protiv kralja Amona  i na njegovo mjesto zakraljio sina mu Jošiju. 


\chapter{34}

\par 1 Jošiji je bilo osam godina kad se zakraljio. Kraljevao je  trideset i jednu godinu u Jeruzalemu. 
\par 2 Činio je što je pravo  u Jahvinim očima. U svemu je hodio putovima oca Davida, ne skrećući  ni desno ni lijevo. 
\par 3 Osme godine kraljevanja, dok još bijaše dječak, počeo  je tražiti Boga oca Davida, a dvanaeste je godine stao čistiti  Judeju i Jeruzalem od uzvišica, ašera, od rezanih i livenih likova. 
\par 4 Pred njim su oborili žrtvenike Baalu, polomio je sunčane stupove  koji bijahu na njima; izlomio je i satro ašere i rezane i livene  likove, prosuo ih po grobovima onih što su im prinosili žrtve. 
\par 5 Svećeničke je kosti spalio na njihovim žrtvenicima i tako  očistio Judeju i Jeruzalem. 
\par 6 Isto je učinio i po gradovima  Manašeova, Efrajimova i Šimunova plemena, pa do Naftalijeva,  po njihovim opustošenim mjestima unaokolo. 
\par 7 Oborio je žrtvenike  i ašere, raskovao i satro rezane likove i isjekao sve sunčane  stupove po svoj zemlji izraelskoj, a onda se vratio u Jeruzalem. 
\par 8 Osamnaeste godine kraljevanja, očistivši zemlju i Dom, posla Asalijahina sina Šafana, gradskoga upravitelja Maaseju, Johazova sina Joaha, tajnika, da poprave Dom Jahve, njegova  Boga. 
\par 9 Oni su došli k velikom svećeniku Hilkiji i predali mu  novce donesene u Božji Dom, koje bijahu sabrali leviti, čuvari  hramskog praga, iz ruke Manašeovih i Efrajimovih sinova i od  svega Izraelova Ostatka, od svega Judina i Benjaminova plemena, od jeruzalemskih stanovnika. 
\par 10 Dali su to na ruku poslovođama, postavljenim nad Domom Jahvinim, a oni su izdavali poslenicima  koji su radili u Domu Jahvinu, popravljajući što je bilo trošno  i obnavljajući Hram. 
\par 11 Dali su drvodjeljama i graditeljima  da kupuju tesanac i drvo za grede i da se pobrvnaju kuće koje  bijahu porušili judejski kraljevi. 
\par 12 Ti su ljudi savjesno obavljali  posao; nad njima su bili postavljeni Jahat i Obadja, leviti od  Merarijevih sinova, i Zaharija i Mešulam od Kehatovih sinova  da upravljaju radom. Svi su leviti bili vični glazbalima. 
\par 13 Jedni  su bili nad bremenošama i nadstojnicima svih poslenika u svakoj  službi, a drugi su od levita bili pisari, nadzornici i vratari. 
\par 14 Kad su iznosili novce donesene u Jahvin Dom, našao je  svećenik Hilkija Knjigu Zakona Jahvina, danu preko Mojsija. 
\par 15 I, progovorivši, Hilkija reče tajniku Šafanu: "Našao sam Knjigu  Zakona u Domu Jahvinu." Hilkija dade knjigu Šafanu. 
\par 16 Šafan  odnese knjigu kralju i izvjesti ga: "Tvoje sluge rade sve što  im se povjerilo. 
\par 17 Izasuvši novce što su se našli u Domu Jahvinu, dadoše ih na ruku poslovođama i poslenicima." 
\par 18 Tada tajnik  Šafan javi kralju: "Svećenik Hilkija dade mi jednu knjigu." I  Šafan je poče čitati pred kraljem. 
\par 19 Čuvši riječi Zakona, kralj  razdrije haljine svoje. 
\par 20 I naredi Hilkiji, Šafanovu sinu Ahikamu, Mikinu sinu Abdonu, tajniku Šafanu i kraljevu sluzi Asaji: 
\par 21 "Idite  i upitajte Jahvu o meni i Ostatku Izraela i Judeje zbog ove knjige  što je nađena, jer je velika Jahvina jarost što se izlila na  nas zato što naši očevi nisu čuvali Jahvine riječi, nisu vršili  što je pisano u knjizi." 
\par 22 Hilkija s kraljevim ljudima ode proročici Huldi, ženi  Šaluma, Tokhatova sina, sina Hasre, čuvara odjeće; ona je živjela  u Jeruzalemu, u novom gradu. Kad joj to kazaše, 
\par 23 ona im reče:  "Ovako veli Jahve, Bog Izraelov: 'Kažite čovjeku koji vas je  poslao k meni: 
\par 24 Ovako veli Jahve: Evo, dovest ću nesreću na  ovaj grad i na njegove stanovnike, izvršit ću sve kletve napisane  u knjizi što je pročitaše pred judejskim kraljem. 
\par 25 Jer su  me ostavili i kadili tuđim bogovima da bi me ljutili svim djelima  ruku svojih, planut će jarost moja na to mjesto i neće se ugasiti. 
\par 26 A judejskome kralju, koji vas je poslao po Jahvin savjet, recite ovo: Ovako govori Jahve, Bog Izraelov: Riječi si čuo. 
\par 27 Ali kako ti je omekšalo srce i jer si se ponizio pred Bogom  čuvši što sam objavio tome gradu i njegovim stanovnicima, i ponizivši  se preda mnom, razdro si haljine i plakao, zato sam te uslišio  - riječ je Jahvina! 
\par 28 Evo, sjedinit ću te s ocima tvojim i  s mirom ćeš leći u grob da ne vidiš svu nesreću koju ću svaliti  na ovo mjesto i njegove stanovnike.'" Oni odnesoše taj odgovor  kralju. 
\par 29 Tada posla kralj da se saberu sve judejske i jeruzalemske  starješine. 
\par 30 Kralj se potom pope u Dom Jahvin, sa svim Judejcima, Jeruzalemcima, svećenicima, levitima i sa svim narodom, od najvećega  do najmanjeg. I pročita im sve riječi Knjige Saveza što je nađena  u Domu Jahvinu. 
\par 31 Kralj, stojeći na svome mjestu, obnovi pred  Jahvom Savez da će slijediti Jahvu, držati se njegovih zapovijedi, pouka i uredaba svim srcem i svom dušom da bi izvršio sve stavke  toga Saveza što su napisane u knjizi. 
\par 32 I sve koji su se našli  u Jeruzalemu i Benjaminovu plemenu pozva da pristupe; i Jeruzalemci  prionuše uza Savez Boga, Boga svojih otaca. 
\par 33 Tada Jošija ukloni  sve gnusobe iz svih izraelskih krajeva i učini te svi koji su  se našli u Izraelu počeše služiti Jahvi, svojem Bogu. Za svega  njegova života nisu odstupili od Jahve, Boga svojih otaca. 


\chapter{35}

\par 1 Potom je Jošija svetkovao Pashu Jahvi u Jeruzalemu: klalo  se pashalno jagnje četrnaestoga dana prvoga mjeseca. 
\par 2 Postavio je svećenike na njihove službe, osokolivši ih  na službu u Jahvinu Domu. 
\par 3 Zatim je rekao levitima koji su  poučavali sve Izraelce i bili posvećeni Jahvi: "Metnite sveti  Kovčeg u Dom koji je sagradio Davidov sin Salomon, izraelski  kralj; ne smijete ga više nositi na ramenima; sada služite Jahvi, svojem Bogu, i njegovu izraelskom narodu! 
\par 4 I pripravite se  po otačkim domovima, po redovima, kako je napisao izraelski kralj  David i propisao sin mu Salomon. 
\par 5 Stojte u Svetinji po redovima  otačkih domova svoje braće, običnoga puka, i po redu levitskoga  otačkog doma. 
\par 6 I tako koljite pashalno janje te se posvetite  i pripravite svoju braću da svetkuju kako je zapovjedio Jahve  preko Mojsija." 
\par 7 Jošija je darovao običnom puku od sitne stoke jaganjaca  i jarića, sve za Pashu, svima koji su se našli ondje, na broj  trideset tisuća, i tri tisuće goveda, sve to s kraljeva imanja. 
\par 8 Njegovi su knezovi dragovoljno darovali narodu, svećenicima  i levitima, i to: Hilkija, Zaharija i Jehiel, predstojnici u  Božjem Domu, dali su svećenicima za Pashu dvije tisuće i šest  stotina jaganjaca i jarića i tri stotine goveda. 
\par 9 A Konanija, Šemaja i Netanel, njegova braća Hašabja, Jehiel i Jozabad, levitski  knezovi, darovali su levitima za Pashu pet tisuća grla sitne  stoke i pet stotina goveda. 
\par 10 A kad je bila uređena služba, stali su svećenici na svoje mjesto i leviti u svojim redovima  po kraljevoj zapovijedi. 
\par 11 Klali su Pashu, a svećenici su škropili  krvlju, dok su leviti odirali kožu. 
\par 12 Onda su pripravili paljenice da ih dadu običnom puku  po redovima otačkih domova da ih prinese Jahvi, kako je napisano  u Mojsijevoj knjizi. Tako su učinili i s govedima. 
\par 13 Pekli  su Pashu na ognju po običaju, a ostale su posvećene stvari kuhali  u loncima, kotlovima i zdjelama i brzo ih raznosili svemu običnom  puku. 
\par 14 Poslije su pripravljali Pashu sebi i svećenicima, jer  su svećenici, Aronovi sinovi, bili zaposleni prinošenjem paljenica  i pretiline do noći; zato su leviti pripravljali sebi i svećenicima, Aronovim sinovima. 
\par 15 A pjevači, Asafovi sinovi, stajali su  na svojem mjestu, kako je bio zapovjedio David, Asaf, Heman i  kraljev vidjelac Jedutun. Vratari su stajali na svakim vratima;  oni se nisu micali od službe, nego su im njihova braća leviti  pripravljala sve. 
\par 16 Tako je bila uređena sva Jahvina služba onoga dana da  se proslavi Pasha i da se prinesu paljenice na Jahvinu žrtveniku  po zapovijedi kralja Jošije. 
\par 17 Tako su Izraelovi sinovi, koji  su se našli ondje, u to doba sedam dana slavili Pashu i Blagdan  beskvasnih kruhova. 
\par 18 Pasha kao ova u Izraelu nije se slavila  od vremena proroka Samuela niti je ijedan od izraelskih kraljeva  slavio Pashu kao što ju je slavio Jošija - sa svećenicima, levitima  i sa svim Judejcima i Izraelcima, koliko ih se god našlo, i s  Jeruzalemcima. 
\par 19 Ta se Pasha svetkovala osamnaeste godine Jošijina kraljevanja. 
\par 20 Poslije svega toga, kad je Jošija uredio Dom, došao je  egipatski kralj Neko da se bije kod Karkemiša na Eufratu, a Jošija  je izišao preda nj. 
\par 21 Kralj Neko poslao je Jošiji glasnike  i poručio: "Što ja imam s tobom, judejski kralju? Ne idem ja  danas na tebe, nego na dom s kojim sam u ratu, i Bog mi je zapovjedio  da se požurim. Okani se Boga koji je sa mnom da te ne upropastim!" 
\par 22 Ali Jošija nije odvratio lica od njega, nego se ojunačio  da se bije s njim; ne poslušavši Nekovih riječi iz Božjih usta, došao je da se bije na Megidskom polju. 
\par 23 Strijelci ustrijeliše kralja Jošiju, a on reče slugama:  "Izvedite me jer sam teško ranjen." 
\par 24 Sluge ga skinuše s bojnih  kola i metnuše u druga kola koja je imao, pa ga odvezoše u Jeruzalem;  ondje je umro i bio sahranjen u grobnici otaca. Sva Judeja s  Jeruzalemom plakala je za Jošijom. 
\par 25 I Jeremija je protužio  za Jošijom. I svi pjevači i pjevačice spominju u tužbalicama  Jošiju do danas; uveli su ih u običaj u Izraelu, i eno su zapisane  u Tužbalicama. 
\par 26 Ostala Jošijina djela i njegova pobožnost, vršeni onako  kako piše u Jahvinu Zakonu, 
\par 27 svi njegovi pothvati, od prvih  do posljednjih, zapisani su u Knjizi o izraelskim i judejskim  kraljevima. 



\chapter{36}

\par 1 Tada priprosti puk uze Jošijina sina Joahaza i zakralji ga  u Jeruzalemu namjesto njegova oca. 
\par 2 Dvadeset i tri godine bile  su Joahazu kad se zakraljio. Kraljevao je tri mjeseca u Jeruzalemu. 
\par 3 Svrgao ga je egipatski kralj u Jeruzalemu i udario na zemlju  danak od sto srebrnih talenata i jedan zlatni talenat. 
\par 4 Egipatski  kralj postavi za kralja nad Judejom i nad Jeruzalemom njegova  brata Elijakima, promijenivši mu ime na Jojakim; njegova je brata  Joahaza uzeo Neko i odveo u Egipat. 
\par 5 Dvadeset je i pet godina bilo Jojakimu kad se zakraljio.  Kraljevao je jedanaest godina u Jeruzalemu; činio je što je zlo  u očima Jahve, njegova Boga. 
\par 6 Na nj je zaratio babilonski kralj  Nabukodonozor i, svezavši ga u dvoje mjedene verige, odveo ga  u Babilon. 
\par 7 Dio posuđa iz Jahvina Doma odnio je Nabukodonozor  u Babilon i metnuo ga u svoj dvorac u Babilonu. 
\par 8 Ostala Jojakimova  djela i gnusobe koje je činio i što se na njemu našlo, sve je  zapisano u Knjizi o izraelskim i judejskim kraljevima. Na njegovo  se mjesto zakraljio sin mu Jojakin. 
\par 9 Osam je godina bilo Jojakinu kad se zakraljio, a kraljevao  je tri mjeseca i deset dana u Jeruzalemu; činio je što je zlo  u Jahvinim očima. 
\par 10 O godišnjoj je mijeni poslao kralj Nabukodonozor  te su ga odveli u Babilon s dragocjenostima iz Jahvina Doma,  a nad Judom i nad Jeruzalemom zakraljio je njegova rođaka Sidkiju. 
\par 11 Dvadeset je i jedna godina bila Sidkiji kad se zakraljio, a kraljevao je jedanaest godina u Jeruzalemu. 
\par 12 Činio je što  je zlo u očima Jahve, njegova Boga; nije se ponizio pred prorokom  Jeremijom, koji mu je govorio iz Jahvinih usta, 
\par 13 nego se još  i pobunio protiv kralja Nabukodonozora, koji ga bijaše zakleo  Bogom; ostao je tvrdoglav i uporan u srcu da se ne obrati Jahvi, Bogu Izraelovu. 
\par 14 Pa i svi su svećenički poglavari i narod gomilali nevjeru  na nevjeru slijedeći gnusna djela krivobožačkih naroda, oskvrnjujući  Dom Jahvin, posvećen u Jeruzalemu. 
\par 15 Jahve, Bog njihovih otaca, slao je k njima zarana svoje glasnike, slao ih svejednako, jer  mu bijaše žao svojega naroda i svojega Prebivališta. 
\par 16 Ali  su se oni rugali Božjim glasnicima, prezirući njegove riječi  i podsmjehujući se njegovim prorocima, dok se nije podigla Jahvina  jarost na njegov narod te više nije bilo lijeka. 
\par 17 Doveo je na njih kaldejskoga kralja, koji okrenu pod  mač njihove mladiće u domu njihova Svetišta, ne štedeći ni mladića  ni djevojke, ni starca ni nemoćna.Sve mu je predao u ruke. 
\par 18 Sve  posuđe Božjega Doma, veliko i malo, blago Jahvina Doma i kraljevo  blago, blago njegovih knezova, sve je odnio u Babilon. 
\par 19 Spalili  su Božji Dom, oborili jeruzalemski zid i sve su njegove dragocjenosti  uništili. 
\par 20 One što izbjegoše maču odvede Nabukodonozor u Babilon  u sužanjstvo. Postali su robovi njemu i njegovim sinovima, dokle  nije nastalo perzijsko kraljevstvo. 
\par 21 Da bi se ispunila riječ  koju Jahve reče na Jeremijina usta: "Dokle se zemlja ne oduži  svojim subotama, počivat će za sve vrijeme u pustoši dok se ne  ispuni sedamdeset godina." 
\par 22 Ali prve godine perzijskoga kralja Kira, da bi se ispunila  riječ Jahvina objavljena na Jeremijina usta, podiže Jahve duh  perzijskoga kralja Kira te on oglasi po svemu svojem kraljevstvu  usmeno i pismeno: 
\par 23 "Ovako veli perzijski kralj Kir: 'Sva zemaljska  kraljevstva dade mi Jahve, Bog nebeski. On mi naloži da mu sagradim  Dom u Jeruzalemu, u Judeji. Tko je god među vama od svega njegova  naroda, Bog njegov bio s njim, pa neka ide onamo!'" 





\end{document}