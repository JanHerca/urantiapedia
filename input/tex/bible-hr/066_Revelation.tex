\begin{document}

\title{Otkrivenje}


\chapter{1}

\par 1 Otkrivenje Isusa Krista: njemu ga dade Bog da on pokaže slugama  svojim ono što se ima dogoditi ubrzo. I on to označi poslavši  svog anđela sluzi svomu Ivanu 
\par 2 koji posvjedoči za riječ Božju  i za svjedočanstvo Isusa Krista - za sve što vidje. 
\par 3 Blago onomu koji čita i onima što slušaju riječi ovog  proroštva te čuvaju što je u njem napisano. Jer vrijeme je blizu! 
\par 4 Ivan sedmerim crkvama u Aziji. Milost vam i mir od Onoga  koji jest i koji bijaše i koji dolazi i od sedam duhova što  su pred Prijestoljem njegovim 
\par 5 i od Isusa Krista, Svjedoka  vjernoga, Prvorođenca od mrtvih, Vladara nad kraljevima zemaljskim.  Njemu koji nas ljubi, koji nas krvlju svojom otkupi od naših  grijeha 
\par 6 te nas učini kraljevstvom, svećenicima Bogu  i Ocu svojemu: Njemu slava i vlast u vijeke vjekova! Amen! 
\par 7 Gle, dolazi s oblacima i gledat će ga svako oko, svi koji su ga proboli, i naricat će nad njim sva plemena  zemaljska. Da! Amen. 
\par 8 Ja sam Alfa i Omega, govori Gospodin Bog - Onaj koji  jest i koji bijaše i koji dolazi, Svevladar. 
\par 9 Ja, Ivan, brat vaš i suzajedničar u nevolji, kraljevstvu  i postojanosti, u Isusu: bijah na otoku zvanu Patmos radi riječi  Božje i svjedočanstva Isusova. 
\par 10 Zanijeh se u duhu u dan Gospodnji  i začuh iza sebe jak glas, kao glas trublje. 
\par 11 Govoraše: "Što  vidiš, napiši u knjigu i pošalji sedmerim crkvama: U Efez, Smirnu, Pergam, Tijatiru, Sard, Filadelfiju, Laodiceju." 
\par 12 Okrenuh  se da vidim glas koji govoraše sa mnom. I okrenuvši se, vidjeh  sedam zlatnih svijećnjaka, 
\par 13 a posred svijećnjaka netko  kao Sin Čovječji, odjeven u dugu haljinu, oko prsiju opasan  zlatnim pojasom; 
\par 14 glava mu i vlasi bijele poput  bijele vune, poput snijega, a oči mu kao plamen ognjeni; 
\par 15 noge mu nalik mjedi uglađenoj, kao u peći užarenoj, a  glas mu kao šum voda mnogih; 
\par 16 u desnici mu sedam zvijezda, iz usta mu izlazi mač dvosječan, oštar, a lice mu kao kad sunce  sjaji u svoj svojoj snazi. 
\par 17 Kad ga vidjeh, padoh mu k nogama kao mrtav. A on stavi  na me desnicu govoreći: "Ne boj se! Ja sam Prvi i Posljednji, 
\par 18 i Živi! Mrtav bijah, a evo živim u vijeke vjekova te  imam ključe Smrti, i Podzemlja. 
\par 19 Napiši dakle što si vidio:  ono što jest i što se ima dogoditi poslije. 
\par 20 Glede  tajne onih sedam zvijezda koje vidje u mojoj desnici i sedam  zlatnih svijećnjaka: sedam zvijezda anđeli su sedam crkava, sedam  svijećnjaka sedam je crkava." 


\chapter{2}

\par 1 Anđelu Crkve u Efezu napiši: "Ovo govori Onaj koji drži sedam  zvijezda u desnici, Onaj koji stupa posred sedam zlatnih svijećnjaka: 
\par 2 Znam tvoja djela, tvoj trud i postojanost tvoju i da ne možeš  podnijeti opakih. Iskušao si one koji se prave apostolima, a  nisu, i otkrio si da su lažljivci. 
\par 3 Postojan si, podnio si  za ime moje i nisi smalaksao. 
\par 4 Ali imam protiv tebe: prvu si  ljubav svoju ostavio. 
\par 5 Spomeni se dakle odakle si pao, obrati  se i čini prva djela. Inače dolazim k tebi i - uklonit ću tvoj  svijećnjak s mjesta njegova ako se ne obratiš. 
\par 6 Ali ovo imaš:  mrziš nikolaitska djela koja i ja mrzim." 
\par 7 "Tko ima uho, nek posluša što Duh govori crkvama! Pobjedniku  ću dati jesti od stabla života koje je u raju Božjem." 
\par 8 I anđelu Crkve u Smirni napiši: "Ovo govori Prvi i  Posljednji, Onaj koji bijaše mrtav i oživje: 
\par 9 Znam tvoju  nevolju i siromaštvo - ali ti si bogat! - i pogrde od onih koji  se nazivaju Židovima, a nisu, nego su sinagoga Sotonina. 
\par 10 Ne  boj se onoga što ti je trpjeti! Evo, Sotona, će neke od vas baciti  u tamnicu da budete iskušani. Bit ćete u nevolji deset  dana. Budi vjeran do smrti i dat ću ti vijenac života." 
\par 11 "Tko ima uho, nek posluša što Duh govori crkvama! Pobjedniku  neće nauditi druga smrt." 
\par 12 I anđelu Crkve u Pergamu napiši: "Ovo govori Onaj u koga  je mač dvosjek, oštar: 
\par 13 Znam gdje prebivaš - ondje gdje je  Sotonino prijestolje - a čvrsto se držiš moga imena te nisi zanijekao  moje vjere ni u one dane kad je Antipa, moj svjedok, vjerni moj, ubijen kod vas - gdje Sotona prebiva. 
\par 14 Ali imam nešto malo  protiv tebe: imaš ondje nekih što drže nauk Bileama što pouči  Balaka da stupicu stavi sinovima Izraelovim te blaguju od mesa  žrtvovana idolima i bludu se podadu. 
\par 15 Tako i ti imaš takvih  koji drže nauk nikolaitski. 
\par 16 Obrati se dakle! Inače dolazim  ubrzo k tebi da ratujem s njima mačem usta svojih." 
\par 17 "Tko ima uho, nek posluša što Duh govori crkvama! Pobjedniku  ću dati mane sakrivene i bijel ću mu kamen dati, a na kamenu  napisano ime novo koje nitko ne zna doli onaj koji ga  prima." 
\par 18 I anđelu Crkve u Tijatiri napiši: "Ovo govori Sin Božji, Onaj u koga su oči kao plamen ognjeni, a noge mu nalik  na mjed uglađenu: 
\par 19 Znam tvoja djela: tvoju ljubav, i vjeru, i služenje, i postojanost - i tvoja posljednja djela obilatija  od prvašnjih. 
\par 20 Ali imam protiv tebe: puštaš ženu Jezabelu, koja se pravi proročicom, da uči i zavodi moje sluge te se bludu  podaju i blaguju od mesa žrtvovana idolima. 
\par 21 Dadoh joj vremena  za obraćenje, ali ona neće da se obrati od bludnosti svoje. 
\par 22 Evo, bacam je na postelju, a bludne drugare njene u veliku nevolju  ako se ne odvrate od njezinih djela; 
\par 23 i djecu ću joj smrću  pobiti. I znat će sve crkve: Ja sam Onaj koji istražuje bubrege  i srca - i dat ću vam svakomu po djelima. 
\par 24 Vama  pak velim - vama drugim u Tijatiri koji ne drže ovog nauka te  ne upoznaše takozvanih dubina sotonskih: Ne stavljam na vas drugoga  bremena 
\par 25 nego - što imate, čvrsto držite dok ne dođem." 
\par 26 "Pobjedniku, onomu što do kraja bude vršio moja djela, dat ću vlast nad narodima 
\par 27 i vladat će njima  palicom gvozdenom, kao posuđe glineno satirati ih - 
\par 28 kao  što i ja to primih od Oca svoga. I dat ću mu zvijezdu Danicu. 
\par 29 Tko ima uho, nek posluša što Duh govori crkvama!" 


\chapter{3}

\par 1 I anđelu Crkve u Sardu napiši: "Ovo govori Onaj koji ima sedam  duhova Božjih i sedam zvijezda: Znam tvoja djela: imaš ime da  živiš, a mrtav si. 
\par 2 Budan budi i utvrdi ostatak koji tek što  ne umre. Doista, ne nađoh da su ti djela pred Bogom mojim savršena. 
\par 3 Spomeni se dakle: kako si primio Riječ i poslušao, tako je  i čuvaj - i obrati se. Ne budeš li dakle budan, doći ću kao tat, a nećeš znati u koji ću čas doći na te. 
\par 4 Ali imaš u Sardu  nekolicinu imena što ne okaljaše svojih haljina; oni će hoditi  sa mnom u bjelini jer su dostojni." 
\par 5 "Tako će pobjednik biti odjeven u bijele haljine i neću  izbrisati imena njegova iz knjige života i priznat ću ime njegovo  pred Ocem svojim i anđelima njegovim." 
\par 6 "Tko ima uho, nek posluša što Duh govori crkvama!" 
\par 7 I anđelu Crkve u Filadelfiji napiši: "Ovo govori Sveti, Istiniti, Onaj koji ima ključ Davidov i kad otvori, nitko neće zatvoriti; kad zatvori, nitko neće otvoriti: 
\par 8 Znam tvoja djela. Evo, otvorio sam pred tobom vrata kojih  nitko zatvoriti ne može. Doista, malena je tvoja snaga, a očuvao  si moju riječ i nisi zatajio mog imena. 
\par 9 Evo, dovest ću neke  iz sinagoge Sotonine - koji sebe zovu Židovi, a nisu, nego lažu  - evo, prisilit ću ih da dođu da ti se do nogu poklone te upoznaju  da te ja ljubim. 
\par 10 Budući da si očuvao moju riječ o  postojanosti, i ja ću očuvati tebe od časa kušnje koji ima doći  na sav svijet da se iskušaju svi pozemljari. 
\par 11 Dolazim ubrzo.  Čvrsto drži što imaš da ti nitko ne ugrabi vijenca." 
\par 12 "Pobjednika ću postaviti stupom u hramu Boga moga i odande  on više neće izići i napisat ću na njemu ime Boga svoga i ime  grada Boga svoga, novog Jeruzalema koji siđe s neba od Boga mojega, i ime moje novo." 
\par 13 "Tko ima uho, nek posluša što Duh govori crkvama!" 
\par 14 I anđelu Crkve u Laodiceji napiši: "Ovo govori Amen, Svjedok vjerni i istiniti, Početak Božjeg stvorenja: 
\par 15 Znam  tvoja djela: nisi ni studen ni vruć. O da si studen ili vruć! 
\par 16 Ali jer si mlak, ni vruć ni studen, povratit ću te iz usta. 
\par 17 Govoriš: 'Bogat sam, obogatih se, ništa mi ne treba!' A ne  znaš da si nevolja i bijeda, i ubog, i slijep, i gol. 
\par 18 Savjetujem  ti: kupi od mene zlata u vatri žežena da se obogatiš i bijele  haljine da se odjeneš da se ne vidi tvoja sramotna golotinja;  i pomasti da oči pomažeš i vidiš. 
\par 19 Ja korim i odgajam one  koje ljubim. Revan budi i obrati se! 
\par 20 Evo, na vratima stojim  i kucam; posluša li tko glas moj i otvori mi vrata, unići ću  k njemu i večerati s njim i on sa mnom." 
\par 21 "Pobjednika ću posjesti sa sobom na prijestolje svoje, kao što i ja, pobijedivši, sjedoh s Ocem svojim na prijestolje  njegovo." 
\par 22 "Tko ima uho, nek posluša što Duh govori crkvama!" 


\chapter{4}

\par 1 Nakon toga vidjeh: gle, vrata otvorena na nebu! A onaj prijašnji  glas, što ga ono začuh kao glas trublje što govoraše sa mnom, reče: "Uziđi ovamo i pokazat ću ti što se ima dogoditi  nakon ovoga!" 
\par 2 I odmah se u duhu zanijeh kad gle: prijestolje  stajaše na nebu i na prijestolje Netko sjede. 
\par 3 Taj što sjede  bijaše nalik na jaspis i sard. A uokolo prijestolja duga slična  smaragdu. 
\par 4 Uokolo prijestolja dvadeset i četiri prijestolja, a na prijestolja sjedoše dvadeset i četiri starješine, obučene  u bijele haljine, sa zlatnim vijencima na glavi. 
\par 5 Od prijestolja  izlaze munje, i glasovi, i gromovi; pred prijestoljem gori sedam  ognjenih zubalja, to jest sedam duhova Božjih, 
\par 6 a pred prijestoljem  kao neko stakleno more, nalik na prozirac. U sredini prijestolja, oko prijestolja, četiri bića, sprijeda i straga puna  očiju: 
\par 7 prvo biće slično lavu, drugo biće slično  juncu, treće biće s licem kao čovječjim, četvrto  biće slično letećem orlu. 
\par 8 Ta su četiri bića - u svakoga  po šest krila - sve naokolo i iznutra puna očiju.  Bez predaha dan i noć govore: "Svet! Svet! Svet Gospodin, Bog Svevladar, Onaj koji bijaše i koji jest i koji dolazi!" 
\par 9 I kad god bića dadu slavu i čast pohvalnicu Onomu koji  sjedi na prijestolju, Živomu u vijeke vjekova, 
\par 10 dvadeset  i četiri starješine padnu ničice pred Onim koji sjedi na prijestolju  i poklone se njemu - Živomu u vijeke vjekova. I stavljaju  svoje vijence pred prijestolje govoreći: 
\par 11 "Dostojan si, Gospodine, Bože naš, primiti slavu i čast i moć! Jer ti si sve stvorio, i tvojom voljom sve postade i bi stvoreno!" 


\chapter{5}

\par 1 I vidjeh: na desnici Onoga koji sjedi na prijestolju - knjiga, iznutra i izvana ispisana, zapečaćena sa sedam pečata! 
\par 2 I vidjeh snažna anđela gdje iza glasa proglašuje: "Tko je  dostojan otvoriti knjigu i otpečatiti pečate njezine?" 
\par 3 I nitko  - ni na nebu, ni na zemlji, ni pod zemljom - nije mogao otvoriti  knjige i pogledati u nju. 
\par 4 Briznem u plač jer se nitko ne nađe  dostojan otvoriti knjigu i pogledati u nju. 
\par 5 A jedan od starješina  reče: "Ne plači! Evo, pobijedi Lav iz plemena Judina, Korijen Davidov, on će otvoriti knjigu i sedam pečata njezinih. 
\par 6 I vidjeh: posred prijestolja i četiriju bića i posred  starješina stoji, kao zaklan, Jaganjac sa sedam rogova  i sedam očiju, to jest sedam duhova Božjih, po svoj zemlji  poslanih. 
\par 7 On pristupi te iz desnice Onoga koji sjedi na prijestolju  uzme knjigu. 
\par 8 A kad on uze knjigu, četiri bića i dvadeset i  četiri starješine padoše ničice pred Jaganjca. U svakoga bijahu  citre i zlatne posudice pune kada, to jest molitava svetačkih. 
\par 9 Pjevaju oni pjesmu novu: "Dostojan si uzeti knjigu i otvoriti pečate njezine jer si bio zaklan i otkupio, krvlju svojom, za Boga ljude iz svakoga plemena i jezika, puka i naroda; 
\par 10 učinio si ih Bogu našemu kraljevstvom i svećenicima i kraljevat će na zemlji." 
\par 11 I vidjeh, i začuh glas anđela mnogih uokolo prijestolja, i bića i starješina. Bijaše ih na mirijade mirijada i tisuće  tisuća. 
\par 12 Klicahu iza glasa: "Dostojan je zaklani Jaganjac primiti moć, i bogatstvo, i mudrost, i snagu, i čast, i slavu, i blagoslov!" 
\par 13 I začujem: sve stvorenje, i na nebu, i na zemlji, i pod  zemljom, i u moru - sve na njima i u njima govori: "Onomu koji sjedi na prijestolju i Jaganjcu blagoslov i čast, i slava i vlast u vijeke vjekova!" 
\par 14 I četiri bića ponavljahu: "Amen!" A starješine padnu  ničice i poklone se. 


\chapter{6}

\par 1 I vidjeh: kad Jaganjac otvori prvi od sedam pečata, začujem  gdje prvo od četiri bića govori glasom kao gromovnim: "Dođi!" 
\par 2 Pogledam, a ono konj bijelac i u njegova konjanika  luk. I dan mu je vijenac te kao pobjednik pođe da pobijedi. 
\par 3 Kad  Jaganjac otvori drugi pečat, začujem drugo biće gdje govori:  "Dođi!" 
\par 4 I iziđe drugi konj, riđan. I njegovu je konjaniku  dano dignuti mir sa zemlje da se ljudi među sobom pokolju. I  dan mu je mač velik. 
\par 5 Kad Jaganjac otvori treći pečat, začujem  treće biće gdje govori: "Dođi!" Pogledam, a ono konj vranac  i njegovu konjaniku u ruci tezulja. 
\par 6 Tada začujem kao neki  glas isred četiriju bića gdje govori: "Mjera pšenice za denar!  Tri mjere ječma za denar! A ulju i vinu ne udi!" 
\par 7 Kad Jaganjac  otvori četvrti pečat, začujem glas četvrtoga bića gdje govori:  "Dođi!" 
\par 8 Pogledam, a ono konj sivac; konjaniku njegovu ime  je "Smrt" i prati ga Podzemlje. Dana im je vlast nad četvrtinom  zemlje: ubijati mačem i glađu i smrću i zvijerima zemaljskim. 
\par 9 Kad Jaganjac otvori peti pečat, vidjeh pod žrtvenikom  duše zaklanih zbog riječi Božje i zbog svjedočanstva što ga imahu. 
\par 10 Vikahu iza glasa: "Ta dokle, Gospodaru sveti i istiniti!  Zar nećeš suditi i osvetiti krv našu na pozemljarima?" 
\par 11 I  svakome je od njih dana bijela haljina i rečeno im je neka se  strpe još malo vremena dok se ne ispuni broj njihovih sudrugova  u službi i braće njihove koja imaju biti pobijena kao i oni. 
\par 12 I vidjeh: kad Jaganjac otvori šesti pečat, potres velik  nasta. I sunce pocrnje kao dlakava kostrijet, sav mjesec posta  kao krv. 
\par 13 I zvijezde padoše s neba na zemlju kao što  smokva smokvice stresa kad je potrese žestok vjetar. 
\par 14 Nebo  iščeznu kao savijena knjiga, a sve se planine i otoci  pokrenuše s mjesta. 
\par 15 Kraljevi zemaljski, i velikaši,  i vojvode, i bogataši, i mogućnici, rob i slobodnjak - svi se  sakriše u spilje i pećine gorske 
\par 16 govoreći gorama  i pećinama: "Padnite na nas i sakrijte nas od lica Onoga  koji sjedi na prijestolju i od srdžbe Jaganjčeve. 
\par 17 Jer dođe  Dan onaj veliki srdžbe njihove i tko će opstati!" 


\chapter{7}

\par 1 Nakon toga vidjeh: četiri anđela stoje na četiri kraja zemlje  zadržavajući četiri vjetra zemaljska da nikakav vjetar ne puše  ni zemljom ni morem nit ikojim drvećem. 
\par 2 I vidjeh drugoga jednog  anđela gdje uzlazi od istoka sunčeva s pečatom Boga živoga. On  povika iza glasa onoj četvorici anđela kojima bi dano nauditi  zemlji i moru: 
\par 3 "Ne udite ni zemlji ni moru ni drveću dok ne  opečatimo sluge Boga našega na čelima!" 
\par 4 I začujem broj opečaćenih  - sto četrdeset i četiri tisuće opečaćenih iz svih plemena sinova  Izraelovih: 
\par 5 iz plemena Judina dvanaest tisuća opečaćenih, iz plemena Rubenova dvanaest tisuća, iz plemena Gadova dvanaest tisuća, 
\par 6 iz plemena Ašerova dvanaest tisuća, iz plemena Naftalijeva dvanest tisuća, iz plemena Manašeova dvanaest tisuća, 
\par 7 iz plemena Šimunova dvanaest tisuća, iz plemena Levijeva dvanaest tisuća, iz plemena Jisakarova dvanaest tisuća, 
\par 8 iz plemena Zebulunova dvanaest tisuća, iz plemena Josipova dvanaest tisuća, iz plemena Benjaminova dvanaest tisuća opečaćenih. 
\par 9 Nakon toga vidjeh: eno velikoga mnoštva, što ga nitko  ne mogaše izbrojiti, iz svakoga naroda, i plemena, i puka, i  jezika! Stoje pred prijestoljem i pred Jaganjcem odjeveni u bijele  haljine; palme im u rukama. 
\par 10 Viču iz glasa: "Spasenje Bogu našemu koji sjedi na prijestolju i Jaganjcu!" 
\par 11 I svi anđeli, što stajahu uokolo prijestolja i starješina  i četiriju bića, padoše pred prijestoljem ničice, na svoja lica, 
\par 12 i pokloniše se Bogu govoreći: "Amen! Blagoslov i slava, i mudrost, i zahvalnica, i čast, i moć i snaga Bogu našemu u vijeke vjekova. Amen." 
\par 13 I jedan me od starješina upita: "Ovi odjeveni u bijele  haljine, tko su i odakle dođoše?" 
\par 14 Odgovorih mu: "Gospodine  moj, ti to znaš." A on će mi: "Oni dođoše iz nevolje velike i  oprali su haljine svoje i ubijelili ih u krvi Jaganjčevoj. 
\par 15 Zato  su pred prijestoljem Božjim i služe mu dan i noć u hramu njegovu, i Onaj koji sjedi na prijestolju razapet će Šator svoj nad njima. 
\par 16 Neće više gladovati ni žeđati, neće ih više paliti sunce  nit ikakva žega 
\par 17 jer - Jaganjac koji je posred prijestolja  bit će pastir njihov i vodit će ih na izvore voda života.  I otrt će Bog svaku suzu s očiju njihovih." 


\chapter{8}

\par 1 Kad Jaganjac otvori sedmi pečat, nasta muk na nebu oko pola  sata. 
\par 2 I vidjeh: sedmorici anđela što stoje pred Bogom dano je  sedam trubalja. 
\par 3 I drugi jedan anđeo pristupi i sa zlatnom  kadionicom stane na žrtvenik. I dano mu je mnogo kada da ga s  molitvama svih svetih prinese na zlatni žrtvenik pred prijestoljem. 
\par 4 I vinu se dim kadni s molitvama svetih iz ruke anđelove pred  lice Božje. 
\par 5 Anđeo uze kadionicu, napuni je vatrom sa žrtvenika  i prosu na zemlju. I udariše gromovi, i glasovi, i munje, i potres. 
\par 6 A sedam anđela sa sedam trubalja pripremiše se da zatrube. 
\par 7 Prvi zatrubi. I nastadoše tuča i oganj, pomiješani s krvlju, i budu bačeni na zemlju. I trećina zemlje izgorje, i trećina  stabala izgorje, i sva zelena trava izgorje. 
\par 8 Drugi anđeo zatrubi.  I nešto kao gora velika, ognjem zapaljena, bačeno bi u more.  I trećina se mora pretvori u krv 
\par 9 te izginu trećina stvorenja  što u moru žive i trećina lađa propade. 
\par 10 Treći anđeo zatrubi. I pade s neba zvijezda velika -  gorjela je kao zublja - pade na trećinu rijeka i na izvore voda. 
\par 11 Zvijezdi je ime Pelin. I trećina se voda pretvori u pelin  te mnoštvo ljudi poginu od zagorčenih voda. 
\par 12 Četvrti anđeo  zatrubi. I bi udarena trećina sunca i trećina mjeseca i trećina  zvijezda te pomrčaše za trećinu. I dan izgubi trećinu svoga sjaja, a tako i noć. 
\par 13 I vidjeh i začuh orla: letio posred neba i vikao iza  glasa: "Jao! Jao! Jao pozemljarima od novih glasova trubalja  preostale trojice anđela koji će sad-na zatrubiti!" 


\chapter{9}

\par 1 Peti anđeo zatrubi. I vidjeh: zvijezda je s neba na zemlju  pala i dani su joj ključi zjala Bezdanova. 
\par 2 Ona otvori zjalo  Bezdanovo i vinu se iz zjala dim kao dim iz peći  goleme te pomrča sunce i zrak od dima iz zjala. 
\par 3 Iz dima pak  iziđoše na zemlju skakavci i dana im je moć kakvu imaju štipavci  zemaljski. 
\par 4 I zapovjeđeno im je da ne ude travi zemaljskoj  nit ikojem zelenilu nit ikojem stablu, nego samo ljudima koji  nemaju pečata Božjega na čelu. 
\par 5 I dano im je ne da ih ubijaju, nego samo da ih muče pet mjeseci, a muka njihova da bude kao  muka od uboda štipavaca. 
\par 6 U one će dane ljudi iskati smrt,  ali je neće naći; poželjet će umrijeti, ali smrt će bježati od  njih. 
\par 7 Skakavci bijahu izgledom nalik na konje za boj spremne.  Na glavama im kao neki zlatni vijenci, lica im kao u ljudi, 
\par 8 kose  kao u žena, a zubi kao u lavova. 
\par 9 Imahu oklope kao od  željeza, a šum krila njihovih kao štropot bojnih kola  s mnogo konja što u boj jure. 
\par 10 Repovi im kao u štipavaca,  sa žalcima, a u repovima im moć da ude ljudima pet mjeseci. 
\par 11 Nad  njima je kralj, anđeo Bezdana, hebrejski mu ime Abadon, grčki  Apolion - Upropastitelj. 
\par 12 Prvi Jao prođe. Evo, za njim dolaze  još dva druga Jao. 
\par 13 Šesti anđeo zatrubi. I začujem neki glas iz rogova zlatnoga  žrtvenika pred Bogom. 
\par 14 Govoraše šestom anđelu koji je držao  trublju: "Odriješi ona četiri anđela svezana na Rijeci velikoj, Eufratu." 
\par 15 I odriješena bijahu četiri anđela, spremna za  taj čas i dan i mjesec i godinu, da pobiju trećinu ljudi. 
\par 16 A  broj četa konjaničkih, kako sam čuo, bijaše dvije mirijade  mirijada. 
\par 17 Ovako u viđenju vidjeh konje i njihove jahače:  imahu oklope ognjene, plavetne i sumporne boje; glave im kao  u lavova, iz usta im sukljao oganj, dim i sumpor. 
\par 18 Od ovih  triju zala poginu trećina ljudi - od ognja, dima i sumpora što  sukljahu konjima iz usta. 
\par 19 Doista, snaga je ovim konjima u  ustima i repovima: repovi im kao u zmija, s glavama kojima ude. 
\par 20 Ipak, preostali ljudi, što ne poginuše od tih zala, ne  obratiše se od djela ruku svojih, da se više ne klanjaju  zlodusima i kumirima - ni zlatnima, ni srebrnima, ni mjedenima, ni kamenima ni drvenima koji niti vide niti čuju nit hodaju  - 
\par 21 i ne obratiše se od svojih ubojstava ni od svojih čaranja  ni od svoga bluda niti od svojih krađa. 


\chapter{10}

\par 1 I vidjeh drugoga jednog, snažnog anđela: silazio s neba ogrnut  oblakom, na glavi mu duga, lice mu kao sunce, a noge kao ognjeno  stupovlje; 
\par 2 u ruci drži otvorenu knjižicu. I zakorači desnom  nogom na more, lijevom na zemlju pa povika iza glasa kao kad  lav riče. 
\par 3 I kad povika, oglasi se sedam gromova tutnjavom. 
\par 4 A kad se oglasi sedam gromova, htjedoh pisati, ali začujem  glas s neba: "Zapečati to što prozbori sedam gromova! Toga ne  piši!" 
\par 5 I onaj anđeo što ga vidjeh gdje stoji na moru i zemlji, podiže k nebu desnicu 
\par 6 i zakle se Živim u vijeke vjekova, koji stvori nebo i sve što je na njemu, zemlju i sve  što je na njoj, more i sve što je u njemu: "Neće više biti  vremena! 
\par 7 Nego - u dane kad se oglasi sedmi anđeo, čim zatrubi, dovršit će se otajstvo Božje kao što on to navijesti slugama  svojim prorocima." 
\par 8 I glas što ga začuh s neba opet prozbori sa mnom: "Idi, uzmi otvorenu knjigu iz ruke anđela što stoji na moru i na zemlji!" 
\par 9 Pristupim k anđelu i zamolim ga da mi dade knjižicu. A on  će mi: "Uzmi je i progutaj! Zagorčit će ti utrobu,  ali će ti u ustima biti slatka kao med." 
\par 10 Uzeh  knjižicu iz ruke anđelove i progutah je. I bijaše mi u ustima  kao med slatka, ali kad je progutah, zagorči mi utrobu. 
\par 11 I  rečeno mi je: "Treba da ponovno prorokuješ proti pucima i narodima  i kraljevima mnogim!" 


\chapter{11}

\par 1 I dana mi je trska slična palici i rečeno mi je: "Ustani i  izmjeri hram Božji i žrtvenik i poklonike u njemu! 
\par 2 Vanjsko  dvorište hrama mimoiđi, ne mjeri ga jer je dano poganima: gazit  će svetim gradom četrdeset i dva mjeseca. 
\par 3 I ja ću poslati  dva svoja svjedoka da, obučeni u kostrijet, prorokuju tisuću  dvjesta i šezdeset dana. 
\par 4 Oni su dvije masline i dva  svijećnjaka što stoje pred Gospodarom zemlje. 
\par 5 Ako im  tko hoće nauditi, oganj suče iz usta njihovih i proždire njihove  neprijatelje. Doista, htjedne li im tko nauditi, tako treba da  pogine. 
\par 6 Oni imaju vlast zaključati nebo da ne pada kiša dok  prorokuju; imaju vlast pretvoriti vode u krv i udariti zemlju  kojim god zlom kad god htjednu. 
\par 7 A kada dovrše svoje svjedočanstvo, Zvijer koja izlazi  iz Bezdana zaratit će s njima, pobijediti ih i ubiti. 
\par 8 I njihova će trupla ležati na trgu grada velikoga koji se  duhovno zove Sodoma i Egipat, gdje je i Gospodin njihov raspet. 
\par 9 Ljudi iz svih puka i plemena i jezika i naroda gledat će njihova  trupla tri i pol dana i neće dopustiti da im se trupla u grob  polože. 
\par 10 Pozemljari će se radovati i veseliti zbog njihove  nesreće i darivati jedan drugoga jer su ta dva proroka zadavala  muku pozemljarima. 
\par 11 Ali nakon tri i pol dana duh životni  od Boga uđe u njih i stadoše na noge te strah velik obuze  one koji ih promatrahu. 
\par 12 I začuše glas s neba silan: "Uziđite  ovamo!" I uziđoše na oblaku u nebo na očigled svojih neprijatelja. 
\par 13 U taj čas nasta velik potres: pade desetina grada, a u potresu  poginu sedam tisuća ljudi. Preživjele spopade strah te proslaviše  Boga nebeskoga. 
\par 14 Drugi Jao prođe. Evo, treći Jao dolazi ubrzo! 
\par 15 I sedmi anđeo zatrubi. I na nebu odjeknuše silni glasovi: "Uspostavljeno je nad svijetom kraljevstvo Gospodara našega i Pomazanika njegova i kraljevat će u vijeke vjekova!" 
\par 16 Tada dvadeset i četiri starješine, što pred Bogom sjedoše  na prijestolja, 
\par 17 padoše ničice i pokloniše se Bogu govoreći. "Zahvaljujemo ti, Gospodaru, Bože, Svevladaru, koji jesi i koji bijaše, zato što uze u ruke moć svoju veliku i zakralji se! 
\par 18 Gnjevili se narodi, ali dođe srdžba tvoja i čas da se sudi mrtvima i naplata dade slugama tvojim prorocima i svetima i svima koji se boje imena tvojega, malima i velikima; i da se unište oni koji kvare zemlju." 
\par 19 I otvori se hram Božji na nebu i pokaza se Kovčeg saveza  njegova u hramu njegovu te udare munje i glasovi i gromovi i  potres i tuča velika. 


\chapter{12}

\par 1 I znamenje veliko pokaza se na nebu: Žena odjevena suncem, mjesec joj pod nogama, a na glavi vijenac od dvanaest zvijezda. 
\par 2 Trudna viče u porođajnim bolima i mukama rađanja. 
\par 3 I pokaza se drugo znamenje na nebu: gle, Zmaj velik, ognjen, sa sedam glava i deset rogova; na glavama mu sedam kruna, 
\par 4 a  rep mu povlači trećinu zvijezda nebeskih - i obori ih na zemlju.  Zmaj stade pred Ženu koja imaše roditi da joj, čim rodi, proždre  Dijete. 
\par 5 I ona porodi sina, muškića, koji će vladati  svim narodima palicom gvozdenom. I Dijete njeno bi uzeto  k Bogu i prijestolju njegovu. 
\par 6 A Žena pobježe u pustinju gdje  joj Bog pripravi sklonište da se ondje hrani tisuću dvjesta i  šezdeset dana. 
\par 7 I nasta rat na nebu: Mihael i njegovi anđeli  zarate se sa Zmajem. Zmaj uđe u rat i anđeli njegovi, 
\par 8 ali  ne nadvlada. I ne bijaše im više mjesta na nebu. 
\par 9 Zbačen je  Zmaj veliki, Stara zmija - imenom Ðavao, Sotona, zavodnik svega  svijeta. Bačen je na zemlju, a s njime su bačeni i anđeli njegovi. 
\par 10 I začujem glas na nebu silan: "Sada nasta spasenje i  snaga i kraljevstvo Boga našega i vlast Pomazanika njegova! Jer  zbačen je tužitelj braće naše koji ih je dan i noć optuživao  pred Bogom našim. 
\par 11 Ali oni ga pobijediše krvlju Jaganjčevom  i riječju svojega svjedočanstva: nisu ljubili života svoga -  sve do smrti. 
\par 12 Zato veselite se, nebesa i svi nebesnici! A  jao vama, zemljo i more, jer Ðavao siđe k vama, gnjevan veoma, znajući da ima malo vremena!" 
\par 13 Kad Zmaj vidje da je zbačen na zemlju, stade progoniti  Ženu koja rodi muškića. 
\par 14 No Ženi bijahu dana dva velika krila  orlujska da odleti u pustinju, u svoje sklonište gdje se, sklonjena  od Zmije, hrani jedno vrijeme i dva vremena i polovicu vremena. 
\par 15 I Zmija iz usta pusti za Ženom vodu poput rijeke da je rijeka  odnese. 
\par 16 Ali zemlja priteče u pomoć Ženi: otvori usta i popi  rijeku što je Zmaj pusti iz usta. 
\par 17 I razgnjevi se Zmaj na  Ženu pa ode i zarati se s ostatkom njezina potomstva, s onima  što čuvaju Božje zapovijedi i drže svjedočanstvo Isusovo. (12:18)I stade na morski žal. 


\chapter{13}

\par 1 I vidjeh: iz mora Zvijer izlazi sa deset rogova i sedam  glava; na rogovima joj deset kruna, na glavama bogohulna imena. 
\par 2 Ta Zvijer što je vidjeh bijaše nalik na leoparda, noge  joj kao medvjeđe, usta kao usta lavlja. Zmaj joj  dade svoju silu i prijestolje i vlast veliku. 
\par 3 Jedna joj glava  bijaše kao na smrt zaklana, ali joj se smrtna rana zaliječila.  Sva se zemlja, začuđena, zanijela za Zvijeri 
\par 4 i svi se pokloniše  Zmaju koji dade takvu vlast Zvijeri. Pokloniše se i Zvijeri govoreći:  "Tko je kao Zvijer! Tko bi smio ratovati s njom?" 
\par 5 I dana su  joj usta da govori drskosti i hule i dana joj je vlast  da to čini četrdeset i dva mjeseca. 
\par 6 I ona otvori usta da huli  Boga, da huli ime njegovo, njegov Šator i nebesnike. 
\par 7 I dano  joj je da se zarati sa svecima i da ih pobijedi. Dana joj  je vlast nad svakim plemenom i pukom i jezikom i narodom: 
\par 8 da joj se poklone svi pozemljari, oni kojima ime nije zapisano  u knjizi života zaklanog Jaganjca, od postanka svijeta. 
\par 9 Tko ima uho, nek posluša! 
\par 10 Je li tko za progonstvo, u progonstvo će ići! Je li tko za mač, da bude pogubljen, mačem će biti pogubljen! U tom je postojanost i vjera  svetih. 
\par 11 I vidjeh: druga jedna Zvijer uzlazi iz zemlje, ima dva  roga poput jaganjca, a govori kao Zmaj. 
\par 12 Vrši svu vlast one  prve Zvijeri, u njenoj nazočnosti. Prisiljava zemlju i sve pozemljare  da se poklone prvoj Zvijeri kojoj ono zacijeli smrtna rana. 
\par 13 Čini  znamenja velika: i oganj spušta s neba na zemlju naočigled ljudi. 
\par 14 Tako zavodi pozemljare znamenjima koja joj je dano činiti  u nazočnosti Zvijeri: svjetuje pozemljarima da načine kip Zvijeri  koja bijaše udarena mačem, ali preživje. 
\par 15 I dano joj je udahnuti  život kipu Zvijeri te kip Zvijeri progovori i poubija sve koji  se god ne klanjaju kipu Zvijeri. 
\par 16 Ona postiže da se svima  - malima i velikima, bogatima i ubogima, slobodnjacima i robovima  - udari žig na desnicu ili na čelo, 
\par 17 i da nitko ne mogne kupovati  ili prodavati osim onog koji nosi žig s imenom Zvijeri ili s  brojem imena njezina. 
\par 18 U ovome je mudrost: u koga je uma, nek odgoneta broj  Zvijeri. Broj je to jednog čovjeka, a broj mu je šest stotina  šezdeset i šest. 


\chapter{14}

\par 1 I vidjeh: gle, Jaganjac stoji na gori Sionu, a s njime sto  četrdeset i četiri tisuće - na čelima im napisano ime njegovo  i ime Oca njegova! 
\par 2 I začujem s neba glas, kao šum voda  mnogih i tutnjavu silna groma; glas taj koji začuh bijaše  kao glas citraša što sviraju na citrama. 
\par 3 Pjevali su pjesmu  novu pred prijestoljem i pred četiri bića i pred starješinama.  Nitko ne mogaše naučiti te pjesme doli one sto četrdeset i četiri  tisuće - otkupljeni sa zemlje. 
\par 4 Ti se ne okaljaše sa ženama, djevci su! Oni prate Jaganjca kamo god pođe. Otkupljeni su od  ljudi kao prvine Bogu i Jaganjcu; 
\par 5 na ustima se njihovim  laž ne nađe, neporočni su. 
\par 6 I vidjeh: drugi jedan anđeo leti posred neba s evanđeljem  vječnim da ga proglasi svim pozemljarima, svakom narodu i plemenu  i jeziku i puku. 
\par 7 Viče iza glasa: "Bojte se Boga i dajte mu  slavu jer dođe čas suda njegova! I poklonite se njemu koji  stvori nebo i zemlju i more i izvore voda!" 
\par 8 Za njim eto drugog anđela koji govori: "Pade, pade  Babilon, veliki koji vinom gnjeva i bluda svojega  opi sve narode!" 
\par 9 Za njima eto i trećeg anđela koji  vikaše iza glasa: "Tko god se klanja Zvijeri i kipu njezinu te  primi žig na čelo ili ruku, 
\par 10 pit će vino gnjeva Božjega, nerazvodnjeno, natočeno već u čaši srdžbe njegove! I bit  će udaren na muke u ognju i sumporu svetim anđelima naočigled  i naočigled Jaganjcu. 
\par 11 Dim muke njihove suklja u  vijeke vjekova. Ni danju ni noću nemaju počinka oni koji  se klanjaju Zvijeri i kipu njezinu i tko god primi žig s imenom  njezinim." 
\par 12 U tom je postojanost svetih - onih što čuvaju  zapovijedi Božje i vjeru Isusovu. 
\par 13 I začujem glas s neba: "Piši! Od sada blaženi mrtvi koji  umiru u Gospodinu! Da, govori Duh, neka otpočinu od svojih trudova!  Jer prate ih djela njihova!" 
\par 14 I vidjeh: gle, bijel oblak, a na oblak sjede Netko kao  Sin Čovječji; na glavi mu zlatan vijenac, u ruci oštar srp. 
\par 15 I drugi jedan anđeo iziđe iz hrama vičući iza glasa onomu  što sjedi na oblaku: "Mahni srpom i žanji jer dođe čas žetvi, zrela je žetva zemaljska!" 
\par 16 I onaj što sjedi na oblaku baci  srp na zemlju i zemlja bi požnjevena. 
\par 17 I drugi jedan anđeo iziđe iz hrama nebeskoga. I on imaše  oštar srp. 
\par 18 I od žrtvenika iziđe drugi anđeo - onaj koji ima  vlast nad ognjem - pa povika iza glasa onomu, s oštrim srpom:  "Mahni oštrim srpom i poberi grozdove u vinogradu zemaljskom  jer sazri grožđe!" 
\par 19 I anđeo baci srp na zemlju i obra vinograd  zemaljski, a obrano baci u veliku kacu gnjeva Božjega. 
\par 20 Gazila  se kaca izvan grada te poteče krv iz kace konjima do uzda, tisuću  i šest stotina stadija uokolo. 


\chapter{15}

\par 1 I vidjeh drugo znamenje na nebu, veliko i čudesno: sedam anđela  sa sedam zala posljednjih - s njima se navršuje gnjev Božji. 
\par 2 I vidjeh kao neko more od prozirca pomiješano s ognjem. Oni  koji pobijediše Zvijer i kip njezin i broj imena njezina stoje  u moru od prozirca s citrama Božjim u ruci. 
\par 3 Pjevaju pjesmu  Mojsija, sluge Božjega, i pjesmu Jaganjčevu: "Velika su i čudesna djela tvoja, Gospodine, Bože, Svevladaru! Pravedni su i istiniti putovi tvoji, Kralju naroda! 
\par 4 Tko da te se ne boji, Gospodine, tko da ne slavi ime tvoje! Ti si jedini svet! I zato svi će narodi doći i klanjati se pred tobom jer se očitovahu pravedna djela tvoja!" 
\par 5 Nakon toga vidjeh: otvori se hram Šatora svjedočanstva  na nebu! 
\par 6 Iziđe sedam anđela sa sedam zala iz hrama; odjeveni  bijahu u blistav bijeli lan, oko prsiju opasani zlatnim pojasom. 
\par 7 Jedno od četiri bića dade sedmorici anđela sedam zlatnih čaša, punih gnjeva Boga koji živi u vijeke vjekova. 
\par 8 I hram se  napuni dimom od Slave Božje i od njegove snage te nitko ne  mogaše ući u hram dok se ne navrši sedam zala sedmorice anđela. 


\chapter{16}

\par 1 I začujem iz hrama jak glas koji viknu sedmorici anđela: "Hajdete, izlijte sedam čaša gnjeva Božjega na zemlju!" 
\par 2 Ode prvi i izli svoju čašu na zemlju. I pojavi se čir, koban i bolan, na ljudima što nose žig Zvijerin i klanjaju se  kipu njezinu. 
\par 3 Drugi izli svoju čašu na more. I ono posta krv kao krv  mrtvačeva te izginu sve živo u moru. 
\par 4 Treći izli svoju čašu na rijeke i izvore voda. I postadoše  krv. 
\par 5 I začujem anđela voda gdje govori: "Pravedan si, Ti  koji jesi i koji bijaše, Sveti, što si tako dosudio! 
\par 6 Oni  su prolili krv svetih i proroka i stoga ih krvlju napajaš! Zavrijedili  su!" 
\par 7 I začujem žrtvenik kako govori: "Da, Gospode, Bože, Svevladaru!  Istiniti su i pravedni sudovi tvoji!" 
\par 8 Četvrti izli svoju čašu na sunce. I suncu je dano da pali  ljude ognjem. 
\par 9 I silna je žega palila ljude te su hulili ime  Boga koji ima vlast nad tim zlima, ali se ne obratiše da mu slavu  dadu. 
\par 10 Peti izli svoju čašu na prijestolje Zvijeri. I kraljevstvo  joj prekriše tmine. Ljudi su grizli jezike od muke 
\par 11 i hulili  Boga nebeskoga zbog muka i čireva, ali se ne obratiše od djela  svojih. 
\par 12 Šesti izli svoju čašu na Eufrat, rijeku veliku. I presahnu  voda te načini prolaz kraljima s istoka sunčeva. 
\par 13 I vidjeh:  iz usta Zmajevih i iz usta Zvijerinih i iz usta Lažnoga proroka  izlaze tri duha nečista, kao žabe. 
\par 14 To su dusi zloduha što  čine znamenja, a pođoše sabrati kraljeve svega svijeta na rat  za Dan veliki Boga Svevladara. 
\par 15 Evo dolazim kao tat! Blažen onaj koji bdije i čuva haljine  svoje da ne ide gol te mu se ne vidi sramota! 
\par 16 I skupiše ih na mjesto koje se hebrejski zove Harmagedon. 
\par 17 I sedmi izli svoju čašu na zrak. Uto iz hrama, s prijestolja, iziđe jak glas i viknu: "Svršeno je!" 
\par 18 I udariše munje i  glasovi i gromovi i nasta potres velik, kakva ne bijaše otkako  je ljudi - tako bijaše silan potres taj. 
\par 19 I prasnu natroje grad veliki i gradovi naroda padoše.  Spomenu se Bog Babilona velikoga da mu dade piti iz čaše vina  gnjevne srdžbe Božje. 
\par 20 I pobjegoše svi otoci, iščezoše gore, 
\par 21 a iz neba se  spusti na ljude tuča velika, poput talenta. Ljudi su hulili Boga  zbog zla tuče jer zlo njezino bijaše silno veliko. 


\chapter{17}

\par 1 I dođe jedan od sedam anđela što nose sedam čaša i prozbori  mi: "Dođi pokazat ću ti sud nad Bludnicom velikom što sjedi  nad vodama velikim, 
\par 2 s kojom su bludničili kraljevi zemlje  i pozemljari se opiše vinom bluda njezina." 
\par 3 I odnese  me u duhu u pustinju. Tu vidjeh Ženu koja sjede na skrletnu Zvijer, punu bogohulnih imena, sa sedam glava i deset rogova. 
\par 4 Žena  bijaše odjevena u grimiz i skrlet, sva u zlatu, dragom kamenju  i biserju. U ruci joj zlatna čaša puna gnusobe i nečisti bluda  njezina. 
\par 5 Na čelo joj napisano ime - tajna: "Babilon veliki, mati bludnica i gnusoba zemljinih." 
\par 6 I vidjeh: Žena je pijana  od krvi svetih i od krvi svjedoka Isusovih. Kad je vidjeh, čudom  se silnim začudih. 
\par 7 Nato će mi anđeo: "Što se čudiš? Ja ću  ti kazati tajnu te žene i Zvijeri koja je nosi, Zvijeri sa sedam  glava i deset rogova." 
\par 8 "Zvijer koju vidje bijaše i više nije; zamalo izlazi iz  Bezdana i ide u propast. I zapanjit će se pozemljari - oni kojima  ime, od postanka svijeta, nije zapisano u knjigu života - kad  vide da Zvijer bijaše i više nije, a opet je tu. 
\par 9 Tu se hoće mudre pameti! Sedam glava sedam je bregova  na kojima žena sjedi. A i sedam kraljeva: 
\par 10 pet ih već pade, jedan jest, a jedan još ne dođe: kada dođe, ostati mu je zamalo. 
\par 11 I Zvijer koja bijaše i više nije, osma je, a iz broja je  njih sedmero, i ide u propast. 
\par 12 Deset rogova što ih  vidje deset je kraljeva; oni još ne primiše kraljevstva, ali će - samo za jedan sat - primiti vlast kao kraljevi zajedno  sa Zvijeri. 
\par 13 Jedne su misli: svu svoju silu i vlast predati  Zvijeri. 
\par 14 Ratovat će protiv Jaganjca, ali će ih pobijediti  Jaganjac - i njegovi pozvanici, izabranici, vjernici - jer on  je Gospodar gospodara i Kralj kraljeva." 
\par 15 I  reče mi anđeo: "Vode što ih vidje, na kojima Bludnica sjedi,  to su puci i mnoštva i narodi i jezici. 
\par 16 I onih deset rogova  što ih vidje i Zvijer - oni će zamrziti Bludnicu, opustošiti  je i ogoliti, najesti se mesa njezina i ognjem je spaliti. 
\par 17 Jer  Bog im u srce stavi izvršiti naum njegov: da jednodušno predadu  kraljevstvo svoje Zvijeri dok se ne ispune riječi Božje. 
\par 18 Žena  koju vidje grad je veliki što kraljuje nad kraljevima zemaljskim." 


\chapter{18}

\par 1 Nakon toga vidjeh: jedan drugi anđeo silazi s neba s moći  velikom! Sva se zemlja zasvijetlila od njegova sjaja. 
\par 2 On povika  iza glasa: "Pade, pade Babilon veliki - Bludnica - i postade  prebivalištem zloduha, nastambom svih duhova nečistih, nastambom  svih ptica nečistih mrskih 
\par 3 jer se gnjevnim vinom bluda njezina  opiše narodi; s njom su bludničili svi kraljevi zemaljski, a  trgovci se zemaljski obogatiše od silna raskošja njezina." 
\par 4 Začujem i drugi glas s neba: "Iziđite iz nje, narode moj, da vas ne zadese zla njezina te ne budete suzajedničari grijeha  njezinih! 
\par 5 Jer njezini grijesi do neba dopriješe i spomenu  se Bog zločina njezinih. 
\par 6 Vratite joj milo za drago, naplatite joj dvostruko  po djelima! U čašu u koju je ona natakala natočite dvostruko! 
\par 7 Koliko se razmetala sjajem i raskoši, toliko joj zadajte muka  i jada! Jer u srcu je svome govorila: 'Na prijestolju sjedim  kao kraljica i nikad neću obudovjeti, jad me nikada zadesiti  neće!' 
\par 8 Stoga u isti će je dan zla zadesiti: smrt i  jad i glad te će sva u ognju biti spaljena. Jer silan je Gospod, Bog, Sudac njezin!" 
\par 9 I plakat će i naricati za njom kraljevi zemlje što su  s njome bludničili i raskošno živjeli kad gledali budu dim požara  njezina. 
\par 10 Prestrašeni mukama njezinim, izadaleka će stajati  i naricati: "Jao, jao, grade veliki, Babilone, grade silni! Kako  li te u tren oka stiže osuda!" 
\par 11 I trgovci zemaljski plaču nad njom i tuguju jer im trga  nitko više ne kupuje: 
\par 12 ni zlata, ni srebra, ni dragoga kamenja, ni biserja, ni tanana lana, ni grimiza, ni svile, ni skrleta:  nit ikakva mirisava drveta, nit ikakva predmeta od slonove kosti, nit ikakva predmeta od skupocjena drveta, nit od mjedi, nit  od željeza, nit od mramora; 
\par 13 ni cimeta, ni balzama, ni miomirisa, ni pomasti, ni tamjana, ni vina, ni ulja, ni bijeloga brašna, ni pšenice; ni goveda, ni ovaca, ni konja, ni kočija, ni roblja  nit ikoje žive duše. 
\par 14 "Voće za kojim ti duša žudjela pobježe od tebe, sav raskoš  i sjaj propade ti - ne, nema ga više!" 
\par 15 Trgovci što svim tim trgovahu, što ih ona obogati, izdaleka  će stajati, prestrašeni mukama njezinim, plakat će i tugovati: 
\par 16 "Jao, jao, grade veliki, odjeveni nekoć u lan tanan i grimiz  i skrlet, nakićeni zlatom i dragim kamenjem i biserjem! 
\par 17 U  tren oka opustje toliko bogatstvo!" I svi kormilari i putnici, svi mornari i moreplovci izdaleka  stoje 
\par 18 i, gledajući dim njezina požara, zapomažu: "Koji li  je grad sličan gradu ovom velikom?" 
\par 19 I posuše glavu pepelom  te plačući i tugujući viknuše: "Jao, jao, grada li velikoga!  Dragocjenostima se njegovim obogatiše svi posjednici morskih  brodova, a evo - u tren oka opustje!" 
\par 20 Veseli se nad njom, nebo, i svi sveti i apostoli i proroci  jer Bog osudivši nju, vama pravo dosudi! 
\par 21 I jedan snažan anđeo uze kamen, velik poput mlinskoga  kamena, i baci ga u more govoreći: "Tako će silovito biti strovaljen  Babilon, grad veliki, i nikada ga više biti neće!" 
\par 22 "Glas citraša i pjevača i svirača i trubljača u tebi se više neće čuti! Obrtnik vješt kojem god umijeću u tebi se više neće naći! Klopot žrvnja u tebi se više neće čuti! 
\par 23 Svjetlost svjetiljke u tebi više neće sjati! Glas zaručnika i zaručnice u tebi se više neće čuti! Jer trgovci tvoji bijahu velikaši zemlje i čaranja tvoja zavedoše sve narode; 
\par 24 i u tebi se našla krv proroka i svetaca i svih zaklanih na zemlji." 


\chapter{19}

\par 1 Nakon toga začujem kao jak glas silnoga mnoštva na nebu: "Aleluja! Spasenje i slava i moć Bogu našemu! 
\par 2 Doista, istiniti su i pravedni sudovi njegovi jer osudi veliku Bludnicu, što pokvari zemlju bludom svojim, i osveti na njoj krv slugu svojih!" 
\par 3 I ponove: "Aleluja! Dim njezin suklja u vijeke vjekova!" 
\par 4 Nato starješine, njih dvadesetčetvorica, i ona četiri  bića padoše ničice i p        okloniše se Bogu, koji sjedi na  prijestolju, govoreći: "Amen! Aleluja!" 
\par 5 I s         prijestolja  iziđe glas: "Hvalite Boga našega, sve sluge njegove, svi koji se njega bojite, i mali i veliki!" 
\par 6 I začuh kao glas silna mnoštva i kao šum voda mnogih  i kao prasak gromova silnih: "Aleluja! Zakraljeva Gospod, Bog naš Svevladar! 
\par 7 Radujmo se i kličimo i slavu mu dajmo jer dođe svadba Jaganjčeva, opremila se Zaručnica njegova! 
\par 8 Dano joj je odjenuti se u lan tanan, blistav i čist!" A lan - pravedna su djela svetih. 
\par 9 I reče mi: "Piši! Blago onima koji su pozvani na svadbenu  gozbu Jaganjčevu!" I reče mi: "Ove su riječi istinite, Božje." 
\par 10 Padoh mu pred noge da mu se poklonim. A on će mi: "Nipošto!  Sluga sam kao i ti i braća tvoja koja imaju svjedočanstvo Isusovo.  Bogu se pokloni!" Jer svjedočanstvo Isusovo duh je proročki. 
\par 11 I vidjeh: nebo otvoreno - i gle, konj bijelac,  a na nj sjeo On, zvani Vjerni i Istiniti, a sudi i vojuje  po pravdi; 
\par 12 oči mu plamen ognjeni, na glavi mu mnoge  krune; nosi napisano ime kojeg nitko ne zna doli on sam; 
\par 13 ogrnut  je ogrtačem krvlju natopljenim; ime mu: Riječ Božja. 
\par 14 Prate  ga na bijelcima Vojske nebeske, odjevene u lan tanan, bijel i  čist. 
\par 15 Iz usta mu izlazi oštar mač kojim će posjeći narode.  Vladat će njima palicom gvozdenom. On gazi u kaci  gnjevne srdžbe Boga Svevladara. 
\par 16 Na ogrtač, o boku, napisano  mu ime: "Kralj kraljeva i Gospodar gospodara." 
\par 17 I vidjeh jednog anđela: stajaše na suncu vičući iza glasa  svim pticama što nebom lete: "Ovamo! Skupite se na  veliku gozbu Božju 
\par 18 da se najedete mesa kraljeva, i mesa  vojvoda, i mesa mogućnika, i mesa konja i konjanika njihovih, i mesa svih mogućih ljudi, slobodnjaka i robova, malih i velikih!" 
\par 19 I vidjeh: Zvijer i kraljevi zemlje i vojske njihove skupiše  se u boj da se zarate s Onim što sjedi na konju i s vojskom njegovom. 
\par 20 I Zvijer bi uhvaćena, a s njom i Lažni prorok koji je u njenoj  nazočnosti činio znamenja i njima zavodio one što su primili  žig Zvijeri i klanjali se njezinu kipu. Živi su oboje bačeni  u ognjeno jezero što gori sumporom. 
\par 21 A drugi su posječeni  mačem što iziđe iz usta Onoga koji sjedi na konju i sve se ptice  nasitiše mesa njihova. 


\chapter{20}

\par 1 I vidjeh anđela: siđe s neba s ključima Bezdana i s velikim  okovima u ruci. 
\par 2 Zgrabi Zmaja, Staru zmiju, to jest Ðavla,  Sotonu, i okova ga za tisuću godina. 
\par 3 Baci ga u Bezdan koji  nad njim zatvori i zapečati da više ne zavodi narode dok se ne  navrši tisuću godina. Nakon toga ima biti odriješen za malo vremena. 
\par 4 I vidjeh prijestolja - onima što sjedoše na njih dano  je suditi - i duše pogubljenih zbog svjedočanstva Isusova i zbog  Riječi Božje i sve koji se ne pokloniše Zvijeri ni kipu njezinu  te ne primiše žiga na čela svoja ni na ruke. Oni oživješe i zakraljevaše  s Kristom tisuću godina. 
\par 5 Drugi mrtvi ne oživješe dok se ne  navrši tisuću godina. To je ono prvo uskrsnuće. 
\par 6 Blažen i svet  onaj tko je dionik toga prvog uskrsnuća! Nad njim druga smrt  nema vlasti: oni će biti svećenici Božji i Kristovi i s njime  će kraljevati tisuću godina. 
\par 7 A kad se navrši tisuću godina, Sotona će iz svoga zatvora  biti pušten: 
\par 8 izići će zavesti narode sa četiri kraja zemlje, Goga i Magoga, i skupiti ih u boj. Bit će ih kao pijeska  morskoga. 
\par 9 Skupiše se na prostrano polje zemlje i opkoliše  tabor svetih i ljubljeni grad. Ali oganj siđe s neba te ih proguta. 
\par 10 A njihov zavodnik, Ðavao, bačen bi u jezero ognjeno i sumporno, gdje se nalaze i Zvijer i Lažni prorok: ondje će se mučiti danju  i noću u vijeke vjekova. 
\par 11 I vidjeh veliko bijelo prijestolje i Onoga što sjede  na nj: pred licem njegovim pobježe zemlja i nebo; ni mjesta im  se više ne nađe. 
\par 12 I vidjeh mrtve, velike i male: stoje pred  prijestoljem, a knjige se otvoriše. I otvori se jedna druga knjiga, knjiga života. I mrtvi bijahu suđeni po onome što stoji napisano  u knjigama, po djelima svojim. 
\par 13 More predade svoje  mrtvace, a Smrt i Podzemlje svoje: i svaki bi suđen po djelima  svojim. 
\par 14 A Smrt i Podzemlje bili su bačeni u jezero ognjeno.  Jezero ognjeno - to je druga smrt: 
\par 15 tko se god ne nađe zapisan  u knjizi života, bio je bačen u jezero ognjeno. 


\chapter{21}

\par 1 I vidjeh novo nebo i novu zemlju jer - prvo nebo i  prva zemlja uminu; ni mora više nema. 
\par 2 I Sveti grad,  novi Jeruzalem, vidjeh: silazi s neba od Boga, opremljen kao  zaručnica nakićena za svoga muža. 
\par 3 I začujem jak glas s prijestolja: "Evo Šatora Božjeg s ljudima! On će prebivati s njima: oni će biti narod njegov, a on će biti Bog s njima. 
\par 4 I otrt će im svaku suzu s očiju te smrti više neće biti, ni tuge, ni jauka, ni boli više neće biti jer - prijašnje uminu." 
\par 5 Tada Onaj što sjedi na prijestolju reče: "Evo, sve činim novo!" I doda: "Napiši: Ove su riječi vjerne i istinite." 
\par 6 I još mi reče: "Svršeno je! Ja sam Alfa i Omega, Početak i Svršetak! Ja ću žednomu dati s izvora vode života zabadava. 
\par 7 To će biti baština pobjednikova. I ja ću njemu biti Bog, a on meni sin. 
\par 8 Kukavicama pak, nevjernima i okaljanima, ubojicama, bludnicima, vračarima i idolopoklonicima i svim lažljivcima udio je u jezeru što gori ognjem i sumporom. To je druga smrt." 
\par 9 I dođe jedan od sedam anđela što imaju sedam čaša punih  zala konačnih te progovori sa mnom: "Dođi, pokazat ću ti Zaručnicu, Ženu Jaganjčevu!" 
\par 10 I prenese me u duhu na goru veliku, visoku  i pokaza mi sveti grad Jeruzalem: silazi s neba od Boga, 
\par 11 sav  u slavi Božjoj, blistav poput dragoga kamena, kamena slična kristalnom  jaspisu; 
\par 12 okružen zidinama velikim i visokim, sa dvanaest  vrata: na vratima dvanaest anđela i napisana imena dvanaest  plemena Izraelovih. 
\par 13 Od istoka vrata troja, od sjevera vrata  troja, od juga vrata troja, od zapada vrata troja. 
\par 14 Gradske  su zidine imale dvanaest temelja, a na njima dvanaest imena dvanaestorice  apostola Jaganjčevih. 
\par 15 Moj subesjednik imaše mjeru, zlatnu trsku, da  izmjeri grad, vrata njegova i zidine. 
\par 16 Grad se stere  u četvorini: dužina mu jednaka širini. On izmjeri trskom grad:  dvanaest tisuća stadija - dužina mu i širina i visina jednaka. 
\par 17 Izmjeri i njegove zidine: sto četrdeset i četiri lakta po  čovjekovoj mjeri kojom je mjerio anđeo. 
\par 18 Zidine su gradske  sagrađene od jaspisa, a sam grad od čistoga zlata, slična čistu  staklu. 
\par 19 Temelji su gradskih zidina urešeni svakovrsnim dragim  kamenjem: prvi je temelj od jaspisa, drugi od safira, treći od  kalcedona, četvrti od smaragda, 
\par 20 peti od sardoniksa, šesti  od sarda, sedmi od krizolita, osmi od berila, deveti od topaza, deseti od krizopraza, jedanaesti od hijacinta, dvanaesti od  ametista. 
\par 21 Dvanaest vrata - dvanaest bisera: svaka od svoga  bisera. A gradski trg - čisto zlato, kao prozirno staklo. 
\par 22 Hrama u gradu ne vidjeh. Ta Gospod, Bog, Svevladar, hram  je njegov - i Jaganjac! 
\par 23 I gradu ne treba ni sunca ni mjeseca  da mu svijetle. Ta Slava ga Božja obasjala i svjetiljka mu Jaganjac! 
\par 24 Narodi će hoditi u svjetlosti njegovoj, a kraljevi  zemaljski u nj donositi slavu svoju. 
\par 25 Vrata mu se ne zatvaraju  obdan, a noći ondje i nema. 
\par 26 U nj će se donijeti slava i čast  naroda. 
\par 27 Ali u nj neće unići ništa nečisto i nijedan tko čini  gadost i laž, nego samo oni koji su zapisani u Jaganjčevoj knjizi  života. 



\chapter{22}

\par 1 I pokaza mi rijeku vode života, bistru kao prozirac: izvire  iz prijestolja Božjeg i Jaganjčeva. 
\par 2 Posred gradskoga  trga, s obje strane rijeke, stablo života što rodi dvanaest  puta, svakog mjeseca svoj rod. A lišće stabla za zdravlje  je narodima. 
\par 3 I neće više biti nikakva prokletstva.  I prijestolje će Božje i Jaganjčevo biti u gradu i sluge će mu  se njegove klanjati 
\par 4 i gledati lice njegovo, a ime će im njegovo  biti na čelima. 
\par 5 Noći više biti neće i neće trebati svjetla  od svjetiljke ni svjetla sunčeva: obasjavat će ih Gospod Bog  i oni će kraljevati u vijeke vjekova. 
\par 6 I reče mi: "Ove su riječi vjerne i istinite jer Gospod  Bog, nadahnitelj proroka, posla svoga anđela da on pokaže slugama  njegovim što se ima dogoditi ubrzo. 
\par 7 I evo, dolazim ubrzo!  Blago onomu koji čuva riječi proroštva ove knjige!" 
\par 8 Ja, Ivan, čuo sam i vidio sve ovo. I kad sam to vidio  i čuo, padoh pred noge anđelu koji mi to pokaza da mu se poklonim. 
\par 9 A on će mi: "Nipošto! Sluga sam kao i ti i braća tvoja proroci  i svi koji čuvaju riječi ove knjige. Bogu se pokloni!" 
\par 10 A zatim će mi: "Ne zapečati riječi proroštva ove knjige  jer - vrijeme je blizu! 
\par 11 Nepravednik neka samo i dalje čini  nepravdu! Okaljan neka se i dalje kalja! Pravednik neka i dalje  živi pravedno! Svet neka se i dalje posvećuje!" 
\par 12 "Evo, dolazim ubrzo i plaća moja sa mnom: naplatit ću  svakom po njegovu djelu!" 
\par 13 "Ja sam Alfa i Omega, Prvi i Posljednji, Početak  i Svršetak! 
\par 14 Blago onima koji peru svoje haljine: imat će  pravo na stablo života i na vrata će smjeti u grad! 
\par 15 Vani  pak ostaju psi i vračari, bludnice, ubojice i idolopoklonici  i tko god ljubi i čini laž." 
\par 16 "Ja, Isus, poslah anđela svoga posvjedočiti ovo po crkvama.  Ja sam korijen i izdanak Davidov, sjajna zvijezda Danica." 
\par 17 I Duh i Zaručnica govore: "Dođi!" I tko ovo čuje, neka  rekne: "Dođi!" Tko je žedan, neka dođe; tko hoće, neka zahvati vode života  zabadava! 
\par 18 Ja svjedočim svakomu tko sluša riječi proroštva u ovoj  knjizi: Tko ovomu što doda, Bog će njemu dodati zla napisana  u ovoj knjizi. 
\par 19 I tko oduzme od riječi proroštva u ovoj knjizi, Bog će mu oduzeti udio na stablu života i na svetom gradu -  na svemu što je napisano u ovoj knjizi. 
\par 20 Svjedok za sve ovo govori: "Da, dolazim ubrzo!" Amen! Dođi, Gospodine Isuse! 
\par 21 Milost Gospodina Isusa sa svima! 




\end{document}