\begin{document}

\title{Habakuk}


\chapter{1}

\par 1 Proroštvo koje vidje prorok Habakuk. 
\par 2 Dokle ću, Jahve, zapomagati, a da ti ne čuješ? Vikati k tebi "Nasilje!" a da ti ne spasiš? 
\par 3 Zašto mi nepravdu iznosiš pred oči, zašto gledaš ugnjetavanje? Pljačka je i nasilje preda mnom. Raspra je, razmirica bjesni! 
\par 4 Zakon je izgubio snagu, a pravda se ni načas ne pomalja. Da, zlikovac progoni pravednika, pravo je stoga izopačeno. 
\par 5 Obazrite se na narode, pogledajte, čudite se, zapanjite! Jer u vaše dane činim djelo u koje ne biste vjerovali da vam ga tko ispriča. 
\par 6 Da! Evo dižem Kaldejce, narod divlji i naprasit što nadire širom zemlje da obitavališta otme tuđa. 
\par 7 On je strašan i jezovit, od njega samog izlazi njegovo pravo i njegov ponos. 
\par 8 Konji su mu brži od leoparda, hitriji od vukova uvečer; jahači mu poskakuju, stižu izdaleka, ustremljeni k'o orlovi da plijen proždru. 
\par 9 Svi će doći rad' grabeža, lica im žegu k'o istočni vjetar, grabe roblje kao pijesak! 
\par 10 Taj se narod kraljevima ruga, podsmjehuje knezovima, poigrava se svim utvrdama, nasipa zemlju i zauzima ih. 
\par 11 Tad se k'o vjetar okrenu i ode, zlikovac komu je snaga bog postala. 
\par 12 Nisi li od davnih vremena, Jahve, Bože moj, Sveče moj? Ti koji ne umireš! Ti si, Jahve podigao ovaj narod radi pravde, postavio ga, Stijeno, da kažnjava. 
\par 13 Prečiste su tvoje oči da bi zloću gledale. Ti ne možeš motriti tlačenja. Zašto gledaš vjerolomce, šutiš kad zlikovac ništi pravednijeg od sebe? 
\par 14 Postupaš s ljudima k'o s morskim ribama, k'o s gmazovima što nemaju gospodara! 
\par 15 On ih sve lovi na udicu, izvlači ih mrežom, pređom ih skuplja i tako se raduje i likuje. 
\par 16 Stog žrtvuje mreži svojoj, pali tamjan svojoj pređi jer mu pribavljaju zalogaj slastan, hranu pretilu. 
\par 17 Valja li, dakle, da neprestano poteže mač i kolje narod nemilice? 


\chapter{2}

\par 1 Stat ću na stražu svoju, postavit se na bedem, paziti što će mi reći, kako odgovorit na moje tužbe. 
\par 2 Tada Jahve odgovori i reče: "Zapiši viđenje, ureži ga na pločice, da ga čitač lako čita." 
\par 3 Jer ovo je viđenje samo za svoje vrijeme: ispunjenju teži, ne vara; ako stiže polako, čekaj, jer odista će doći i neće zakasniti! 
\par 4 Gle: propada onaj čija duša nije pravedna, a pravednik živi od svoje vjere. 
\par 5 Bogatstvo je odista podmuklo! Ohol je i ne može počinuti tko ždrijelo razvaljuje k'o Podzemlje, tko je kao smrt nezasitan, tko sabire za se sve narode, tko kupi za se sva plemena! 
\par 6 Zar mu se neće svi podrugivati, rugalicu i zagonetku spjevat' protiv njega? Reći će: Jao onom tko množi što nije njegovo (a dokle će?) i opterećuje se zalogama! 
\par 7 Neće li naglo ustat' vjerovnici tvoji, neće li se probuditi ljuti tvoji tlačitelji? Tada ćeš im plijen biti! 
\par 8 Jer si opljačkao mnoge narode, sav ostatak naroda opljačkat će tebe, jer si prolio krv ljudsku, poharao zemlju, grad i sve mu žitelje. 
\par 9 Jao onom tko otimačinu zgrće nepravednu kući svojoj, da visoko svije gnijezdo svoje i otkloni ruku zla! 
\par 10 Nanese sramotu kući svojoj: zatirući mnoga plemena, griješiš protiv sebe. 
\par 11 Jer iz samih zidova kamen kriči, a krovna mu greda odgovara. 
\par 12 Jao onom tko grad diže krvlju i tvrđavu zasnuje na nepravdi! 
\par 13 Nije li to, gle, od Jahve nad Vojskama da se narodi za oganj trude, puci nizašto muče? 
\par 14 Jer će se zemlja napuniti znanja o slavi Jahvinoj kao što vode prekrivaju more. 
\par 15 Jao onom tko bližnjeg navodi na piće, ulijeva otrov dok on pije da bi promatrao njegovu nagost! 
\par 16 Ti si pijan od sramote, ne od slave! Pij samo i pokazuj kapicu. Dolazi ti pehar iz desnice Jahvine i sramota na slavu tvoju! 
\par 17 Nasilje nad Libanonom tebe će prestraviti, pokolj zvijeri, jer si ljudsku krv prolio, poharao zemlju, grad i njegove žitelje. 
\par 18 Čemu koristi tesan lik da ga umjetnik teše? Čemu lijevan lik, lažno proroštvo, da se tvorac njegov u nj pouzdaje oblikujuć' nijeme kipove? 
\par 19 Jao onom tko komadu drva kaže: "Probudi se!" Kamenu nijemom: "Preni se!" On da prorokuje? Optočen može biti i zlatom i srebrom, ali nikakva daha životnog nema u njemu. 
\par 20 Ali je Jahve u svojem svetom Hramu: nek' zemlja sva zašuti pred njime! 



\chapter{3}

\par 1 Molitva. Od proroka Habakuka. Na način tužbalice. 
\par 2 Jahve, čuo sam za slavu tvoju, Jahve, tvoje mi djelo ulijeva jezu! Ponovi ga u naše vrijeme! Otkrij ga u naše vrijeme! U gnjevu se svojem smilovanja sjeti! 
\par 3 Bog stiže iz Temana, a Svetac s planine Parana! Veličanstvo njegovo zastire nebesa, zemlja mu je puna slave. 
\par 4 Sjaj mu je k'o svjetlost, zrake sijevaju iz njegovih ruku, ondje mu se krije sila. 
\par 5 Kuga pred njim ide, groznica ga sustopice prati. 
\par 6 On stane, i zemlja se trese, on pogleda, i dršću narodi. Tad se raspadoše vječne planine, bregovi stari propadoše, njegove su staze od vječnosti. 
\par 7 Prestrašene vidjeh kušanske šatore, čadore što dršću u zemlji midjanskoj. 
\par 8 Jahve, planu li tvoj gnjev na rijeke ili jarost tvoja na more te jezdiš na svojim konjima, na pobjedničkim bojnim kolima? 
\par 9 Otkrivaš svoj luk i otrovnim ga strijelama sitiš. Bujicama rasijecaš tlo, 
\par 10 planine dršću kad te vide, navaljuje oblaka prolom, bezdan diže svoj glas. 
\par 11 Sunce uvis diže ruke, mjesec u obitavalištu svojem popostaje, pred blijeskom tvojih strijela, pred blistavim sjajem koplja tvoga. 
\par 12 Jarosno po zemlji koračaš, srdito gaziš narode. 
\par 13 Iziđe da spasiš narod svoj, da spasiš svog pomazanika; sori vrh kuće bezbožnikove, ogoli joj temelje do stijene. 
\par 14 Kopljima si izbo vođu ratnika njegovih, koji navališe da nas s radošću satru, kao da će potajice proždrijet' ubogoga. 
\par 15 Gaziš po moru s konjima svojim, po pučini silnih voda! 
\par 16 Čuo sam! Sva se moja utroba trese, podrhtavaju mi usne na taj zvuk, trulež prodire u kosti moje, noge klecaju poda mnom. Počinut ću kada dan tjeskobni svane narodu što nas sad napada. 
\par 17 Jer smokvino drvo neće više cvasti niti će na lozi biti ploda, maslina će uskratiti rod, polja neće donijeti hrane, ovaca će nestati iz tora, u oborima neće biti ni goveda. 
\par 18 Ali ja ću se radovati u Jahvi i kliktat ću u Bogu, svojem Spasitelju. 
\par 19 Jahve, moj Gospod, moja je snaga, on mi daje noge poput košutinih i vodi me na visine. Zborovođi. Na žičanim glazbalima. 




\end{document}