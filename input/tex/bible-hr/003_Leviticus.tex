\begin{document}

\title{Levitski zakonik}


\chapter{1}

\par 1 Zovnu Jahve Mojsija te mu iz Šatora sastanka reče: 
\par 2 "Govori Izraelcima i kaži im: 'Kad tko od vas želi prinijeti  Jahvi žrtvu od stoke, prinijet će je ili od krupne ili od sitne  stoke. 
\par 3 Ako njegov prinos za žrtvu paljenicu bude od krupne stoke, neka prinese muško bez mane. Neka ga dovede k ulazu u Šator  sastanka da pred Jahvom bude primljen. 
\par 4 Neka stavi svoju ruku  na glavu žrtve paljenice da mu za njegovo ispaštanje bude primljena. 
\par 5 Neka zatim zakolje junca pred Jahvom. A Aronovi sinovi, svećenici, neka prinesu krv. Neka njome zapljusnu sve strane žrtvenika  koji stoji pred ulazom u Šator sastanka. 
\par 6 Potom neka se žrtva  sadre i rasiječe na dijelove. 
\par 7 Neka sinovi Aronovi, svećenici, nalože vatru na žrtveniku i na vatru metnu drva. 
\par 8 Neka zatim  sinovi Aronovi, svećenici, naslažu dijelove, s glavom i lojem, na drva što su na vatri na žrtveniku. 
\par 9 Drobina i noge neka  se operu u vodi. A onda neka svećenik sve sažeže u kad na žrtveniku.  To je žrtva paljenica, žrtva paljena Jahvi na ugodan miris.' 
\par 10 Ako bi htio prinijeti za žrtvu paljenicu od sitne stoke  - od ovaca ili koza - neka prinese muško bez mane. 
\par 11 Neka ga  zakolje pred Jahvom, na žrtveniku sa sjeverne strane. Neka zatim  Aronovi sinovi, svećenici, zapljusnu žrtvenik krvlju sa svih  strana. 
\par 12 Potom neka je rasijeku na dijelove, a svećenik neka  ih, s glavom i lojem, naslaže na drva što su na vatri na žrtveniku. 
\par 13 Drobina i noge neka se operu u vodi. Onda svećenik neka sve  prinese i na žrtveniku sažeže. To je žrtva paljenica, žrtva paljena  Jahvi na ugodan miris. 
\par 14 Ako bi hto prinijeti Jahvi ptice kao žrtvu paljenicu, neka onda prinese grlicu ili golubića. 
\par 15 Neka ga svećenik  prinese k žrtveniku i, zavrnuvši mu vratom, otkine glavu i na  žrtveniku sažeže. Zatim neka mu krv iscijedi žrtveniku sa strane. 
\par 16 Neka mu gušu i perje ukloni i pobaca ih na istočnu stranu  žrtvenika, na mjesto za otpatke. 
\par 17 Neka ga raspori duž obaju  krila, ali neka ih ne rastavlja. Onda neka ga svećenik na žrtveniku  sažeže na drvima što su na vatri. To je žrtva paljenica, žrtva  paljena Jahvi na ugodan miris." 


\chapter{2}

\par 1 "Kad tko želi prinijeti Jahvi žrtvu prinosnicu, neka njegov  dar bude od najboljeg brašna; neka ga polije uljem i na nj stavi  tamjana. 
\par 2 Neka ga onda donese Aronovim sinovima, svećenicima.  Zatim neka zagrabi šaku od toga brašna i ulja i sav tamjan, pa  neka svećenik na žrtveniku to sažeže u kad za spomen-žrtvu. To  je žrtva paljena Jahvi na ugodan miris. 
\par 3 A što od žrtve prinosnice  ostane, neka pripadne Aronu i njegovim sinovima - najsvetije  od žrtava Jahvi paljenih. 
\par 4 Ako za žrtvu prinosnicu želiš prinijeti tijesta pečena  u peći, neka to budu beskvasne pogače od najboljeg brašna, zamiješene  u ulju, ili beskvasne prevrte uljem namazane. 
\par 5 Ako tvoj dar  bude žrtva prinosnica pečena na tavi, neka bude od najboljeg  brašna, neukvasana i u ulju zamiješena. 
\par 6 U komade je izlomi  i po njima ulja polij: žrtva je to prinosnica. 
\par 7 Bude li tvoja  prinosnica kuhana u kotluši, neka bude od najboljeg brašna, pripravljena  s uljem. 
\par 8 Donosi Jahvi žrtvu prinosnicu tako pripravljenu! Neka  se preda svećeniku, a on će je polagati na žrtvenik. 
\par 9 Neka  svećenik odvoji od žrtve prinosnice dio kao spomen-žrtvu, pa  neka ga sažeže u kad na žrtveniku - kao žrtvu paljenu Jahvi na  ugodan miris! 
\par 10 A što od žrtve prinosnice ostane, neka pripadne  Aronu i njegovim sinovima - najsvetije od žrtava Jahvi paljenih. 
\par 11 Nikakva žrtva prinosnica koju budeš prinosio Jahvi neka  ne bude priređivana s kvasom, jer ne smiješ u kad sažigati ni  kvasa ni meda kao žrtvu paljenicu. 
\par 12 Prinosite ih Jahvi kao  prvine plodova, ali neka se sa žrtvenika ne viju na ugodan miris. 
\par 13 Svaku svoju žrtvu prinosnicu posoli. Ne ostavljaj svoje žrtve  prinosnice bez soli Saveza sa svojim Bogom: sa svakim svojim  prinosom prinesi i sol. 
\par 14 Ako prinosiš Jahvi žrtvu prinosnicu od prvina, prinesi  tu žrtvu od prvina svojih plodova u obliku klasa pržena na vatri  ili brašna od samljevenog zrnja. 
\par 15 Dodaj još ulja i na nju  stavi tamjana. To je žrtva prinosnica. 
\par 16 Onda neka svećenik  sažeže u kad za spomen-žrtvu dio kruha i ulja i sav tamjan kao  žrtvu Jahvi paljenu." 


\chapter{3}

\par 1 "Ako tko prinosi žrtvu pričesnicu te ako prinosi goveče - žensko  ili muško - neka je bez mane što prinosi pred Jahvom. 
\par 2 Neka  stavi svoju ruku na glavu svoje žrtve i zakolje je na ulazu u  Šator sastanka. Neka zatim Aronovi sinovi, svećenici, zapljusnu  krvlju sve strane žrtvenika. 
\par 3 Od žrtve pričesne, kao žrtvu  paljenu, neka prinese loj što omotava drobinu, sav loj što je  oko drobine; 
\par 4 oba bubrega i loj što je na njima i na slabinama;  pa privjesak s jetre: neka i njega s bubrezima izvadi. 
\par 5 Zatim  neka Aronovi sinovi te dijelove sažegu na žrtveniku sa žrtvom  paljenicom koja bude na drvima na vatri. To neka je žrtva paljena  Jahvi na ugodan miris. 
\par 6 Ako tko prinosi za žrtvu pričesnicu od sitne stoke Jahvi, neka prinese bez mane, bilo muško ili žensko. 
\par 7 Ako na dar  prinosi ovcu, neka je prinese pred Jahvom. 
\par 8 Neka stavi svoju  ruku na glavu svoje žrtve i neka je zakolje pred Šatorom sastanka.  Zatim neka Aronovi sinovi zapljusnu njezinom krvlju sve strane  žrtvenika. 
\par 9 Od žrtve pričesnice neka prinesu žrtvu paljenu  Jahvi: njezin loj, cio pretili rep, otkinuvši ga tik uz hrptenjaču;  loj što omotava drobinu, sav loj što je oko drobine; 
\par 10 oba  bubrega i loj što je na njima i na slabinama; pa privjesak s  jetre: neka i njega s bubrezima izvadi. 
\par 11 Onda neka svećenik  to sažeže na žrtveniku u kad - kao hranu vatre u čast Jahvi. 
\par 12 Ako li prinosi kozu, neka je prinese pred Jahvom: 
\par 13 neka  stavi svoju ruku na glavu svoje žrtve i neka je zakolje pred  Šatorom sastanka. Neka zatim Aronovi sinovi zapljusnu njezinom  krvlju sve strane žrtvenika. 
\par 14 Onda neka od nje prinese, kao  paljenu žrtvu Jahvi, loj što omotava drobinu, sav loj što je  oko drobine; 
\par 15 oba bubrega i loj što je na njima i na slabinama;  pa privjesak s jetre; neka i njega s bubrezima izvadi. 
\par 16 Onda  neka ih svećenik sažeže na žrtveniku - žrtvu paljenu Jahvi na  ugodan miris. Sav loj pripada Jahvi. 
\par 17 Neka ovo bude zakon  za sva vremena svim vašim naraštajima u kojem god mjestu budete  boravili: nipošto ne smijete jesti ni loja ni krvi." 


\chapter{4}

\par 1 Jahve reče Mojsiju: 
\par 2 "Ovako kaži Izraelcima: 'Ako se tko  nehotice ogriješi o bilo koju Jahvinu zapovijed te učini što  je zabranjeno činiti: 
\par 3 Bude li to pomazanjem posvećeni svećenik koji pogriješi  i navuče tako krivnju na narod, onda za grijeh koji učini neka  prinese Jahvi jedno grlo krupne stoke, jednoga junca bez mane, kao žrtvu okajnicu. 
\par 4 Neka junca dovede pred Jahvu do ulaza  u Šator sastanka; neka juncu na glavu položi svoju ruku i onda  junca zakolje pred Jahvom. 
\par 5 Zatim neka pomazanjem posvećeni  svećenik uzme krvi od junca i donese je u Šator sastanka. 
\par 6 Onda  neka svećenik zamoči svoj prst u krv i tom krvlju neka sedam  puta poškropi prednju stranu zavjese Svetišta, pred Jahvom. 
\par 7 Potom  neka svećenik stavi te krvi na rogove žrtvenika za miomirisni  kad koji se dimi pred Jahvom u Šatoru sastanka. Svu ostalu krv  od junca neka izlije podno žrtvenika za žrtve paljenice što se  nalazi na ulazu u Šator sastanka. 
\par 8 Iz junca što ga prinosi kao žrtvu okajnicu neka izvadi:  loj što omotava drobinu, sav loj što je oko drobine; 
\par 9 oba bubrega  i loj što je na njima i na slabinama, privjesak s jetre; neka  i njega izvadi s bubrezima; 
\par 10 onako kako se uzima dio iz vola  žrtve pričesnice. Neka ih zatim svećenik sažeže u kad na žrtveniku  za žrtve paljenice. 
\par 11 Kožu od junca, sve meso od njega, njegovu  glavu, njegove noge, drobinu i njegovu nečist 
\par 12 - svega junca  - neka iznese na čisto mjesto izvan tabora gdje se pepeo izasiplje  i neka ga spali na vatri od drva; tu na pepelu neka se junac  spali.'" 
\par 13 "Ako sva izraelska zajednica nehotično pogriješi počinivši  štogod što je Jahve zabranio pa tako postanu krivi a ne budu  svjesni krivnje, 
\par 14 onda, kad se sazna za učinjeni prijestup, neka zajednica prinese jedno grlo krupne stoke - jednoga junca  bez mane - kao žrtvu okajnicu. Neka ga dovedu pred Šator sastanka. 
\par 15 Tu pred Jahvom neka starješine zajednice polože svoje ruke  juncu na glavu. Neka se onda junac zakolje pred Jahvom. 
\par 16 Neka  zatim pomazanjem posvećeni svećenik donese krvi od junca u Šator  sastanka; 
\par 17 neka svećenik zamoči svoj prst u krv i sedam puta  poškropi prednju stranu zavjese, pred Jahvom. 
\par 18 Neka zatim  stavi krvi na rogove žrtvenika koji se nalazi pred Jahvom u Šatoru  sastanka. Svu ostalu krv neka izlije podno žrtvenika za žrtve  paljenice što se nalazi na ulazu u Šator sastanka. 
\par 19 S junca neka skine sav loj i sažeže ga u kad na žrtveniku. 
\par 20 I s juncem neka uradi kako je uradio s onim juncem žrtve  okajnice - tako neka učini i s tim. I pošto svećenik nad članovima  zajednice izvrši obred pomirenja, bit će im oprošteno. 
\par 21 Neka  odnese junca izvan tabora i spali ga kako je spalio i prvoga  junca. To je žtrva za prijestup zajednice." 
\par 22 "Ako nehotično pogriješi glavar i učini štogod što je  Jahve, Bog njegov, zabranio i tako sagriješi, 
\par 23 onda, kad ga  obznane o prijestupu koji je počinio, neka kao svoj prinos donese  muško jare bez mane. 
\par 24 Položivši svoju ruku jaretu na glavu, neka ga zakolje na mjestu gdje se kolju pred Jahvom žrtve paljenice.  To je žrtva okajnica. 
\par 25 Svećenik neka uzme na svome prstu krvi  od žrtve okajnice pa je stavi na rogove žrtvenika za žrtve paljenice.  A svu ostalu krv neka izlije podno žrtvenika za žrtve paljenice. 
\par 26 Sav loj neka sažeže u kad na žrtveniku kao i loj sa žrtve  pričesnice. Neka tako svećenik nad glavarom izvrši obred pomirenja  za njegov grijeh, pa će mu biti oprošteno." 
\par 27 "Ako tko od običnoga puka nehotično pogriješi učinivši  štogod što je Jahve zabranio i tako sagriješi, 
\par 28 onda, kad  ga obznane o prijestupu koji je počinio, neka kao svoj prinos  za grijeh koji je počinio donese žensko jare bez mane. 
\par 29 Neka  stavi svoju ruku na glavu okajnice i zakolje žrtvu okajnicu na  mjestu za žrtve paljenice. 
\par 30 Neka svećenik uzme krvi na svome  prstu pa je stavi na rogove žrtvenika za žrtve paljenice. A svu  ostalu krv neka izlije podno žrtvenika. 
\par 31 Neka zatim izvadi  sav njezin loj kao što se vadi loj iz žrtve pričesnice; neka  ga onda svećenik sažeže u kad na žrtveniku kao ugodan miris Jahvi.  Kad svećenik izvrši nad tim čovjekom obred pomirenja, bit će  mu oprošteno. 
\par 32 Ako bi tko htio dovesti janje kao žrtvu okajnicu, neka  dovede žensko bez mane. 
\par 33 Položivši svoju ruku na glavu žrtve  okajnice, neka je zakolje kao žrtvu okajnicu na mjestu gdje se  kolju žrtve paljenice. 
\par 34 Neka onda svećenik uzme krvi od žrtve  okajnice na svome prstu pa je stavi na rogove žrtvenika za žrtve  paljenice. A svu ostalu krv neka izlije podno žrtvenika. 
\par 35 Neka  zatim izvadi sav njezin loj kao što se vadi loj iz žrtve pričesnice.  Neka to svećenik sažeže u kad povrh žrtava paljenih Jahvi u čast.  Neka tako svećenik izvrši nad tim čovjekom obred pomirenja za  grijeh koji je počinio, pa će mu biti oprošteno." 


\chapter{5}

\par 1 "Zgriješi li tko tako što čuje riječi proklinjanja a odbije  da svjedoči iako je mogao biti svjedokom jer je ili sam vidio  ili doznao pa tako nosi krivnju na sebi; 
\par 2 ili ako tko dirne  kakav nečist predmet, strv nečiste zvijeri, strv nečista živinčeta  ili strv nečista puzavca - i u neznanju postane nečist i odgovoran; 
\par 3 ili kad se tko dotakne nečistoće čovječje, bilo to što mu  drago od čega se nečistim postaje i toga ne bude svjestan, kad  dozna, biva odgovoran; 
\par 4 nadalje, kad tko nepromišljeno izusti  zakletvu na dobro ili zlo - na što se već čovjek nepromišljeno  zaklinje - i toga ne bude svjestan, onda, kad dozna, biva odgovoran; 
\par 5 ako, dakle, tko postane odgovoran u bilo čemu od toga, neka prizna počinjeni grijeh. 
\par 6 I neka prinese Jahvi kao žrtvu  naknadnicu za počinjeni grijeh jednu ženku od sitne stoke, janje  ili kozle, kao žrtvu okajnicu. Neka svećenik izvrši nad njim  obred pomirenja koji će ga osloboditi od njegova grijeha." 
\par 7 "Ako mu sredstva ne dopuštaju da pribavi glavu sitne stoke, neka Jahvi, kao naknadnicu za počinjeni grijeh, prinese dvije  grlice ili dva golubića; jedno kao žrtvu okajnicu, a drugo kao  žrtvu paljenicu. 
\par 8 Neka ih donese svećeniku, a on neka najprije  prinese ono što je određeno kao žrtva okajnica. Stisnuvši ga  za vrat, neka mu slomi šiju, ali neka glave ne otkida. 
\par 9 Neka  krvlju žrtve poškropi žrtvenik sa strane, a ostatak krvi neka  se iscijedi podno žrtvenika. To je žrtva okajnica. 
\par 10 Onda neka  drugo prinese kao žrtvu paljenicu prema propisu. Neka tako svećenik  nad tim čovjekom izvrši obred pomirenja za grijeh koji je počinio, i bit će mu oprošteno. 
\par 11 Ako mu sredstva ne dopuštaju da pribavi dvije grlice  ili dva golubića, neka Jahvi, u zadovoljštinu za počinjeni grijeh, prinese jednu desetinu efe njaboljeg brašna. Ulja u nj neka  ne ulijeva niti na nj tamjana stavlja jer je žrtva okajnica. 
\par 12 Kada to donese svećeniku, neka svećenik zagrabi punu pregršt  kao spomen-žrtvu i na žrtveniku sažeže u čast Jahvi povrh paljenih  žrtava. To je žrtva okajnica. 
\par 13 Neka tako svećenik izvrši nad  tim čovjekom obred pomirenja za grijeh koji je počinio u bilo  kojem od tih slučajeva, pa će mu biti oprošteno. Ono ostalo neka  pripadne svećeniku kao i od žrtve prinosnice." 
\par 14 Još reče Jahve Mojsiju: 
\par 15 "Ako tko počini pronevjerenje  ogriješivši se nehotično o svete stvari Jahvine, neka za naknadu, kao žrtvu naknadnicu, prinese Jahvi, iz svoga stada, ovna bez  mane, vrijedna - po tvojoj procjeni - najmanje dva šekela srebra  - prema cijeni hramskog šekela. 
\par 16 Neka nadoknadi koliko se  ogriješio o svete stvari i tome doda još petinu i neka dadne  svećeniku. Neka svećenik nad njim izvrši obred pomirenja ovnom  žrtve naknadnice, i bit će mu oprošteno. 
\par 17 Ako tko i ne znajući pogriješi i učini štogod što je  Jahve zabranio, kriv je, pa neka snosi posljedice svoje krivnje. 
\par 18 Neka svećeniku dovede za naknadnicu iz svoga stada ovna bez  mane, prema tvojoj procjeni. Neka svećenik nad tim čovjekom izvrši  obred pomirenja za pogrešku što je počinio u neznanju, i bit  će mu oprošteno. 
\par 19 To je žrtva naknadnica; on je doista bio  odgovoran Jahvi." 


\chapter{6}

\par 1 (5:20) Jahve još reče Mojsiju: 
\par 2 (5:21) "Kad se tko ogriješi i počini  pronevjeru prema Jahvi prevarivši svoga bližnjega u pologu ili  pohrani, a tako i krađom ili iskorištavanjem svoga bližnjega; 
\par 3 (5:22) ili, nađe li što je bilo izgubljeno pa slaže i krivo se zakune  u bilo kojem grijehu što ga čovjek može učiniti; 
\par 4 (5:23) ako tko  tako pogriješi i kriv postane, onda ono što je krađom oduzeo  ili što je iskorištavanjem namakao; ili polog što mu je bio povjeren;  ili izgubljenu stvar što ju je našao; 
\par 5 (5:24) ili ono za što se bio  krivo zakleo - neka u cijelosti vrati i, dodavši tome petinu, neka dadne onome kome pripada istoga dana kad spozna svoju krivnju. 
\par 6 (5:25) Neka potom svećeniku za naknadu, kao žrtvu naknadnicu Jahvi, dovede iz svog stada jednoga ovna bez mane, prema tvojoj procjeni, 
\par 7 (5:26) a svećenik neka nad tim čovjekom izvrši obred pomirenja  pred Jahvom, i bit će mu oprošteno, ma kakvo bilo nedjelo kojega  je krivac." 
\par 8 (6:1) Jahve još reče Mojsiju: 
\par 9 (6:2) "Naredi Aronu i njegovim sinovima:  'Ovakav je obred za žrtvu paljenicu: žrtva paljenica neka ostane  na žeravi na žrtveniku svu noć do jutra; i vatra neka neprestano  gori na žrtveniku. 
\par 10 (6:3) Neka svećenik stavi na se lanenu košulju, na svoje tijelo neka navuče lanene gaće; zatim neka zgrne zamašćeni  pepeo u što je vatra pretvorila žrtvu paljenicu na žrtveniku  i neka ga stavi pokraj žrtvenika. 
\par 11 (6:4) Potom neka svuče svoje ruho  i na se obuče drugo te neka odnese zamašćeni pepeo na čisto mjesto  izvan tabora. 
\par 12 (6:5) Vatra na žrtveniku mora uvijek gorjeti; ne smije  se gasiti. Neka svako jutro svećenik na nju naloži drva i onda  na nju naslaže žrtvu paljenicu. Tu neka u kad sažiže loj sa žrtava  pričesnica. 
\par 13 (6:6) Neka na žrtveniku trajno gori vatra; neka se ne  gasi.'" 
\par 14 (6:7) "Ovo je obred za žrtvu prinosnicu: neka je Aronovi sinovi  pronose u nazočnosti Jahve pred žrtvenikom. 
\par 15 (6:8) Pošto jedan od  njih zagrabi pregršt najboljeg brašna i ulja sa žrtve prinosnice  i sav tamjan što bude na njoj, pošto to sažeže na žrtveniku kao  spomen-žrtvu, ugodan miris Jahvi, 
\par 16 (6:9) neka ostatak u obliku beskvasnih  kruhova pojedu Aron i njegovi sinovi; neka ga jedu na posvećenu  mjestu - u dvorištu Šatora sastanka. 
\par 17 (6:10) Neka se ne peče s kvascem.  To je dio žrtava meni paljenih što im ga ja dajem - dio najsvetiji, jednako kao i žrtva okajnica i kao žrtva naknadnica. 
\par 18 (6:11) Svaki  muškarac Aronova potomstva može jesti taj dio od žrtava paljenih  Jahvi, i to je vječni zakon za sve vaše naraštaje: i tko ih se  god dotakne, bit će posvećen." 
\par 19 (6:12) Jahve još reče Mojsiju: 
\par 20 (6:13) "Neka Aron i njegovi sinovi  na dan svoga pomazanja prinesu Jahvi ovaj prinos: desetinu efe  najboljeg brašna kao redovitu žrtvu prinosnicu, polovinu ujutro, a polovinu uvečer. 
\par 21 (6:14) Neka bude pripravljena u tavi na ulju.  Donesi je dobro namočenu i prinesi Jahvi kao ugodan miris, kao  žrtvu prinosnicu od više komada. 
\par 22 (6:15) Neka je tako pripravi svećenik  koji od njegovih sinova bude pomazan da ga naslijedi. To je vječni  zakon. Neka se ta žrtva Jahvi sva sažeže! 
\par 23 (6:16) Svaka svećenička  žrtva prinosnica treba da bude posve spaljena; neka se od nje  ne jede." 
\par 24 (6:17) Još reče Jahve Mojsiju: 
\par 25 (6:18) "Kaži Aronu i njegovim sinovima:  'Ovo je obred žrtvovanja za grijeh: žrtva okajnica neka se zakolje  pred Jahvom na mjestu gdje se kolje žrtva paljenica - presveta  je! 
\par 26 (6:19) Svećenik koji prinosi tu žrtvu okajnicu neka od nje i  jede; neka se ona jede na posvećenu mjestu, u dvorištu Šatora  sastanka. 
\par 27 (6:20) Tko se god dotakne njezina mesa bit će posvećen;  ako krv poštrapa odijelo, poštrapani dio neka se ispere na posvećenu  mjestu. 
\par 28 (6:21) A posuda od ilovače u kojoj bude meso kuhano neka  se razbije; a ako bude kuhano u posudi od tuča, neka se istare  i vodom ispere. 
\par 29 (6:22) Svaki muški od svećeničke loze može od nje  jesti - presveta je! 
\par 30 (6:23) Ali nijedna žrtva okajnica od koje je  krv donesena u Šator sastanka za obred pomirenja u Svetištu neka  se ne jede, nego na vatri spali.'" 


\chapter{7}

\par 1 "Ovo je obred za žrtvu naknadnicu. 
\par 2 Nadasve je sveta! Neka  se žrtva naknadnica zakolje na mjestu gdje se kolju žrtve paljenice, a njezinom krvlju neka svećenik zapljusne sve strane žtrvenika. 
\par 3 Zatim neka prinese sav loj s nje: pretili rep, loj što omotava  drobinu, 
\par 4 oba bubrega i loj što je na njima i na slabinama;  pa privjesak s jetre: neka i njega izvadi s bubrezima! 
\par 5 Neka  ih svećenik sažeže na žrtveniku kao žrtvu u čast Jahvi paljenu.  To je žrtva naknadnica. 
\par 6 Svaki muški od svećeničke loze može  od nje jesti. Neka je jedu na posvećenu mjestu - presveta je! 
\par 7 Kakva je žrtva okajnica, takva je i žrtva naknadnica;  jedno je pravilo za njih: neka pripadne svećeniku koji njome  vrši obred pomirenja. 
\par 8 Tako isto koža od žrtve koju tko preda  svećeniku da bude prinesena za žrtvu paljenicu neka pripadne  svećeniku. 
\par 9 Nadalje, svaka žrtva prinosnica što bude pečena  u peći, kao i svaka što bude zgotovljena u kotluši ili na tavi, neka pripadne svećeniku koji je prinosi. 
\par 10 A svaka žrtva prinosnica, zamiješena s uljem ili nasuho, neka pripadne svim Aronovim sinovima  bez razlike!" 
\par 11 "Ovo je obred za žrtvu pričesnicu koja će se prinositi  Jahvi. 
\par 12 Ako se prinosi u zahvalu, neka se zajedno sa žrtvom  zahvalnicom prinesu i beskvasne pogače uljem zamiješene; beskvasne  prevrte uljem namazane i kolači od najboljeg brašna, zamiješeni  uljem. 
\par 13 Ovaj prinos, nadopunjen kolačima od ukvasanoga tijesta, neka se prinosi zajedno sa žrtvom pričesnicom u zahvalu. 
\par 14 Od  svake ovakve žrtve neka se prinese po jedan kolač na dar Jahvi.  To neka bude za svećenika koji zapljuskuje krv od žrtve pričesnice. 
\par 15 A meso žrtve pričesnice neka se pojede istoga dana kad bude  žrtvovana; neka se od nje ne ostavlja ništa za sutradan. 
\par 16 A bude li prinos žrtva zavjetnica ili žrtva dragovoljna, neka se jede na dan kad se žrtva prinosi. Što ostane od nje  neka se jede sutradan. 
\par 17 A što još mesa od žrtve preteče, neka  se treći dan na vatri spali." 
\par 18 "Ako bi tko jeo meso žrtve pričesnice i treći dan, žrtva  neće biti primljena niti će onome koji je prinosi biti uračunata.  To je meso kvarno, i onaj koji od toga jede neka i posljedice  krivnje snosi! 
\par 19 Meso koje se dotakne bilo čega nečista neka  se ne jede nego na vatri spali! Inače, tko je god čist može jesti  meso. 
\par 20 A tko bi nečist jeo mesa od žrtve pričesnice što je  bila Jahvi prinesena takav neka se iskorijeni iz svoga naroda. 
\par 21 Kad se tko dotakne bilo čega nečista - bila to nečist čovječja, kakva nečista životinja ili bilo kakvo nečisto stvorenje - pa  pojede mesa od žrtve pričesnice koja je prinesena Jahvi takav  neka se iskorijeni iz svoga naroda!" 
\par 22 Reče Jahve Mojsiju: 
\par 23 "Ovako kaži Izraelcima: 'Ne jedite  loja ni volujskoga, ni ovčjega, ni kozjega. 
\par 24 Loj sa životinje  koja ugine, ili koju divlje zvijeri razderu, može se upotrijebiti  za bilo što, ali ga ne smijete jesti. 
\par 25 Tko god jede loj od  životinje koja se može prinijeti Jahvi kao žrtva paljenica takav  neka se iskorijeni iz svoga naroda. 
\par 26 Ne smijete uživati krvi  ni od ptica ni od stoke ni u kojem svome prebivalištu. 
\par 27 Tko  bi god uživao bilo kakvu krv neka se iskorijeni iz svoga naroda.'" 
\par 28 Jahve još reče Mojsiju: 
\par 29 "Ovako kaži Izraelcima: 'Prinos  Jahvi od žrtve pričesnice mora donijeti onaj koji Jahvi prinosi  žrtvu pričesnicu. 
\par 30 Svojim vlastitim rukama neka prinese Jahvi  žrtvu paljenicu; neka prinese loj i grudi; grudi neka se prinesu  pred Jahvom kao žrtva prikaznica. 
\par 31 Neka svećenik sažeže loj  na žrtveniku, a grudi neka pripadnu Aronu i njegovim sinovima. 
\par 32 Desno pleće od svojih žrtava pričesnica dajte svećeniku na  dar. 
\par 33 Onome Aronovu sinu koji bude prinosio krv i loj sa žrtve  pričesnice neka u dio pripadne desno pleće. 
\par 34 Jer ja uzimam  od Izraelaca grudi od žrtava pričesnica što se prinose kao žrtva  prikaznica i pleće žrtve podizanice te ih predajem svećeniku  Aronu i njegovim sinovima. To je trajna odredba za Izraelce. 
\par 35 To je dohodak Aronov i njegovih sinova od žrtava paljenih  u čast Jahvi; dodjeljuje im se od onog dana kad se dovedu da  vrše svećeničku službu u čast Jahvi. 
\par 36 Jahve je naredio da  im se od dana kad budu pomazani to daje kao pristojbina od Izraelaca.  To je trajna odredba za njihove naraštaje'." 
\par 37 To je obred za žrtvu paljenicu, prinosnicu, okajnicu, naknadnicu, žrtvu posvetnicu i žrtvu pričesnicu 
\par 38 koji je  Jahve naredio Mojsiju na Sinajskom brdu kad je zapovjedio Izraelcima  da Jahvi u Sinajskoj pustinji prinose žrtve. 


\chapter{8}

\par 1 Jahve reče Mojsiju: 
\par 2 "Uzmi Arona, a s njim i njegove sinove;  ruho, ulje pomazanja, junca žrtve okajnice, dva ovna i košaru  beskvasnih kruhova 
\par 3 te skupi svu zajednicu na ulazu u Šator  sastanka." 
\par 4 Mojsije učini kako mu je Jahve naredio, i zajednica  se sabra na ulazu u Šator sastanka. 
\par 5 Tada Mojsije progovori  zajednici: "Ovo je Jahve zpovjedio da se učini." 
\par 6 Izvede zatim  Mojsije Arona i njegove sinove pa ih u vodi opra. 
\par 7 Obuče na  nj haljinu, opasa ga pojasom, ogrnu ga ogrtačem i stavi mu oplećak.  Zatim ga opasa tkanicom oplećka i njome pritegnu uza nj oplećak. 
\par 8 Stavi mu naprsnik, a u naprsnik metnu Urim i Tumim. 
\par 9 Na  glavu mu stavi mitru, a sprijeda na mitru postavi zlatnu pločicu  - sveti vijenac - kako je Jahve naredio Mojsiju. 
\par 10 Uzme zatim Mojsije ulje pomazanja te pomaže Prebivalište  i sve što je u njemu da ih posveti. 
\par 11 Sedam puta poškropi njime  žrtvenik te pomaže žrtvenik i sav njegov pribor, umivaonik s  njegovim stalkom da ih posveti. 
\par 12 Izlije ulja pomazanja Aronu  na glavu te ga pomaže da ga posveti. 
\par 13 Potom Mojsije dovede  Aronove sinove; na njih obuče haljine, pasovima ih opaše i poveze  im zavije, kako je Jahve Mojsiju naredio. 
\par 14 Dovede potom junce  žrtve okajnice. Aron i njegovi sinovi stave svoje ruke na glavu  juncu žrtve okajnice. 
\par 15 Zatim ga Mojsije zakolje. Onda uzme  krvi pa je svojim prstom stavi na rogove oko žrtvenika. Tako  žrtvenik očisti. Zatim izlije krv podno žrtvenika; posveti ga, izvršivši na njemu obred pomirenja. 
\par 16 Zatim Mojsije uzme sav loj što je bio oko drobine, privjesak  s jetre, oba bubrega i njihov loj, pa to sažeže u kad na žrtveniku. 
\par 17 A kožu od junca, njegovo meso i njegovu nečist spali u vatri  izvan tabora, kako je Jahve naredio Mojsiju. 
\par 18 Dovede potom  ovna za žrtvu paljenicu. Aron i njegovi sinovi stave svoje ruke  ovnu na glavu. 
\par 19 Sad ga Mojsije zakolje. Onda krvlju zapljusne  žrtvenik sa svih strana. 
\par 20 Pošto isiječe ovna na dijelove,  Mojsije sažeže u kad glavu, dijelove i loj. 
\par 21 U vodi opere  drobinu i noge pa u kad sažeže na žrtveniku svega ovna. Bila  je to žrtva paljenica na ugodan miris - žrtva u čast Jahvi paljena  - kako je Jahve naredio Mojsiju. 
\par 22 Zatim dovede drugoga ovna, ovna za žrtvu posvetnicu.  Aron i njegovi sinovi stave svoje ruke ovnu na glavu. 
\par 23 Mojsije  ga zakolje. Onda uzme krvi pa stavi na resu Aronova desnog uha, na palac njegove desne ruke i na palac njegove desne noge. 
\par 24 Potom  Mojsije dovede Aronove sinove, pa im stavi iste krvi na resu  desnog uha, na palac desne ruke i na palac desne noge. Zatim  Mojsije krvlju zapljusne žrtvenik sa svih strana. 
\par 25 Poslije  toga uzme loj, pretili rep, loj što je bio oko drobine, privjesak  s jetre, oba bubrega i njihov loj - i desno pleće; 
\par 26 a iz košare  beskvasnih kruhova, što je stajala pred Jahvom, uzme jednu beskvasnu  pogaču, jednu prevrtu s uljem i jedan kolač i postavi ih na loj  i desno pleće. 
\par 27 Sve to položi na ruke Arona i na ruke njegovih  sinova pa to prinese kao žrtvu prikaznicu pred Jahvom. 
\par 28 Potom  Mojsije opet to uzme s njihovih ruku i sažeže u kad na žrtveniku  povrh žrtve paljenice. Bila je to žrtva posvetnica na ugodan  miris, žrtva u čast Jahvi paljena. 
\par 29 Naposljetku Mojsije uzme  grudi i prinese ih kao žrtvu prikaznicu pred Jahvom. To je bila  pristojbina Mojsiju od ovna žrtve posvetnice, kako je Jahve Mojsiju  naredio. 
\par 30 Zatim Mojsije uze ulja za pomazanje i krvi što je bila  na žrtveniku te poškropi Arona i njegove haljine, a tako i njegove  sinove i njihove haljine. Tako posveti Arona i njegove haljine;  njegove sinove i njihove haljine. 
\par 31 Onda reče Mojsije Aronu i njegovim sinovima: "Skuhajte  to meso na ulazu u Šator sastanka i ondje ga blagujte s kruhom  što je u košari za žrtvu posvetnicu, kako sam naredio. Neka ga  jedu Aron i njegovi sinovi! 
\par 32 A što od mesa i kruha ostane, spalite na vatri. 
\par 33 Sedam dana ne odlazite s ulaza Šatora  sastanka - do dana kad se navrši rok vašega svećeničkog posvećenja.  Jer sedam dana treba za vaše posvećenje. 
\par 34 Kako se radilo danas, Jahve je naredio da se tako radi dalje, da se nad vama izvrši  obred pomirenja. 
\par 35 Zato ostanite na ulazu Šatora sastanka sedam  dana, danju i noću vršeći što je Jahve naredio, da ne pomrete.  Takvu sam ja zapovijed dobio." 
\par 36 Aron i njegovi sinovi učiniše sve što je Jahve naredio  preko Mojsija. 


\chapter{9}

\par 1 Osmoga dana Mojsije pozva Arona, njegove sinove i starješine  Izraelove 
\par 2 i reče Aronu: "Uzmi jedno tele za žrtvu okajnicu, jednoga ovna za žrtvu paljenicu, oboje bez mane, i dovedi ih  pred Jahvu. 
\par 3 A Izraelcima reci ovako: 'Uzmite jednoga jarca  za žrtvu okajnicu, tele i janje od godine, oboje bez mane, za  žrtvu paljenicu; 
\par 4 a junca i ovna za žrtvu pričesnicu da žrtvujete  pred Jahvom; napokon prinosnicu, s uljem zamiješenu; jer će vam  se danas Jahve ukazati.'" 
\par 5 Dovedu oni pred Šator sastanka što je Mojsije naredio;  naprijed stupi sva zajednica i stane pred Jahvu. 
\par 6 "Ovo je zapovijed", reče Mojsije, "koju je Jahve izdao. Izvršite je, da vam se pokaže  slava Jahvina." 
\par 7 Zatim Mojsije reče Aronu: "Stupi k žrtveniku, prinesi svoju žrtvu okajnicu i svoju žrtvu paljenicu i tako  izvrši obred pomirenja za se i svoj dom; onda prinesi dar naroda  i za nj izvrši obred pomirenja, kako je Jahve naredio." 
\par 8 Aron se primače žrtveniku i zakla tele žrtve za svoj vlastiti  grijeh. 
\par 9 Zatim mu Aronovi sinovi donesu krvi. On u nju zamoči  svoj prst i stavi je na rogove žrtvenika. Potom ostalu krv izli  podno žrtvenika. 
\par 10 A loj, bubrege i privjesak s jetre žrtve  okajnice sažeže u kad na žrtveniku, kako je Jahve naredio Mojsiju. 
\par 11 Meso i kožu spali na vatri izvan tabora. 
\par 12 Zakolje poslije  toga žrtvu paljenicu, od koje mu sinovi Aronovi pruže krv. On  njome zapljusne žrtvenik sa svih strana. 
\par 13 Dodaju mu i žrtvu  paljenicu, dio po dio, a tako i glavu, i on je sažeže u kad na  žrtveniku. 
\par 14 Drobinu i noge opere pa i njih na žrtveniku sažeže  u kad povrh žrtve paljenice. 
\par 15 Zatim prinese dar naroda. Uze jarca žrtve okajnice za  grijehe naroda, zakla ga i prinese kao žrtvu okajnicu, isto onako  kao i prijašnju. 
\par 16 Donese potom žrtvu paljenicu i prinese je  prema propisu. 
\par 17 Donijevši poslije toga žrtvu prinosnicu, od  nje zagrabi pregršt i sažeže na žrtveniku u kad povrh jutarnje  žrtve paljenice. 
\par 18 Napokon zakolje junca i ovna kao žrtvu pričesnicu za  narod. Aronovi mu sinovi pruže krv, a on zapljusne žrtvenik sa  svih strana; 
\par 19 dodaju mu i loj s junca i ovna, pretili rep, loj oko drobine, bubrege i privjesak s jetre. 
\par 20 Metnuvši te  masne dijelove na grudi, sažga ih u kad na žrtveniku. 
\par 21 A grudi  i desno pleće Aron prinese kao žrtvu prikaznicu pred Jahvom,  kako je Mojsije naredio. 
\par 22 Tada Aron podiže ruke spram naroda i blagoslovi ga. Pošto  prinese žrtvu okajnicu, paljenicu i pričesnicu, siđe. 
\par 23 Poslije  toga Mojsije i Aron uđoše u Šator sastanka. Kad iziđoše, blagosloviše  narod. Slava Jahvina pokaza se svemu narodu. 
\par 24 Ispred Jahve  izbi oganj i sažga žrtvu paljenicu i masne komade na žrtveniku.  A sav narod, vidjevši to, viknu od veselja i pade ničice. 


\chapter{10}

\par 1 A sinovi Aronovi Nadab i Abihu uzmu svaki svoj kadionik; stave  u nj vatre i na nju metnu tamjana da prinesu pred Jahvom neposvećenu  vatru, koju im on ne bijaše propisao. 
\par 2 Ali izbije plamen ispred  Jahve te ih proguta - poginuše oni pred Jahvom. 
\par 3 Nato će Mojsije  Aronu: "To je ono što je Jahve navijestio: Po onima koji su mi blizu svetim ću se pokazati; pred svim ću se pukom proslaviti." Aron je šutio. 
\par 4 Mojsije zovnu Mišaela i Elsafana, sinove  Aronova strica Uziela, pa im reče: "Dođite i odnesite svoju braću  ispred Svetišta u polje izvan tabora!" 
\par 5 Oni dođu i odnesu ih  u njihovim košuljama u polje izvan tabora, kako je Mojsije rekao. 
\par 6 Poslije toga Mojsije reče Aronu i njegovim sinovima, Eleazaru  i Itamaru: "Ne raščupavajte svoje kose niti razdirite svojih  haljina da ne poginete i da se On ne razljuti na svekoliku zajednicu.  Vaša braća i sav dom Izraelov neka oplakuje one koje je vatra  Jahvina sažegla. 
\par 7 Ne smijete odlaziti s ulaza u Šator sastanka  da ne pomrete, jer na vama je Jahvino ulje pomazanja." Oni učine  po riječi Mojsijevoj. 
\par 8 Jahve reče Aronu: 
\par 9 "Kad ulazite u Šator sastanka, nemojte  piti vina niti opojnoga pića, ni ti ni tvoji sinovi s tobom!  Tako nećete poginuti. To je trajan zakon za vaše naraštaje; 
\par 10 da  možete lučiti posvećeno od običnoga, čisto od nečistoga; 
\par 11 da  možete učiti Izraelce svim zakonima što ih je Jahve predao preko  Mojsija." 
\par 12 Onda Mojsije rekne Aronu i njegovim preživjelim sinovima, Eleazaru i Itamaru: "Uzimajte od žrtve prinosnice što preostaje  nakon prinesene žrtve u čast Jahvi paljene i beskvasnu je uza  žrtvenik jedite jer je vrlo sveta. 
\par 13 Blagujte je u svetom mjestu, jer to je - tako je meni naređeno - pristojbina tvoja i pristojbina  tvojih sinova od žrtava paljenih u čast Jahvi. 
\par 14 A grudi žrtve  prikaznice i pleće žrtve podizanice ti i tvoji sinovi i tvoje  kćeri s tobom jedite na bilo kojem čistom mjestu. Jer to je dodijeljeno  za pristojbinu tebi i tvojim sinovima od izraelskih žrtava pričesnica. 
\par 15 Pleće žrtve podizanice i grudi žrtve prikaznice što se donose  zajedno s lojem, na vatri paljenim - pošto budu prineseni za  žrtvu prikaznicu pred Jahvom - neka pripadnu tebi i tvojim sinovima  s tobom. To je, kako je Jahve naredio, trajan zakon." 
\par 16 Potom se Mojsije potanje raspita o jarcu žrtve okajnice.  Već je bio spaljen. On se razljuti na Eleazara i Itamara, Aronove  preživjele sinove, pa rekne: 
\par 17 "Zašto ste jeli žrtvu okajnicu  na svetome mjestu? Vrlo ja sveta! To vam je dao Jahve da uklanjate  krivnju sa zajednice vršeći nad njom obred pomirenja pred Jahvom. 
\par 18 Budući da krv žrtve nije bila unesena unutar Svetišta, morali  ste je blagovati u Svetištu, kako mi je bilo zapovjeđeno." 
\par 19 Nato  će Aron Mojsiju: "Danas su, eto, prinijeli svoju žrtvu okajnicu  i svoju žrtvu paljenicu pred Jahvom! Što bi se meni dogodilo  da sam ja danas jeo od žrtve okajnice? Bi li to bilo milo Jahvi?" 
\par 20 Kad Mojsije to ču, odobri. 


\chapter{11}

\par 1 Jahve reče Mojsiju i Aronu: 
\par 2 "Ovako kažite Izraelcima: 'Ovo  su životinje koje između svih četveronožaca na zemlji možete  jesti: 
\par 3 svaku koja ima papke, ali papke razdvojene, i koja  preživa možete jesti. 
\par 4 Ali ove, iako preživaju ili papke imaju, ne smijete jesti: devu, jer iako preživa, razdvojena papka nema  - za vas je nečista; 
\par 5 svisca, jer iako preživa, razdvojena  papka nema - za vas je nečist; 
\par 6 arnebeta, iako preživa, razdvojena  papka nema - za vas je nečist. 
\par 7 A svinja, iako ima papak, i  to papak razdvojen, ne preživa - za vas je nečista. 
\par 8 Njihova  mesa nemojte jesti niti se njihove strvine doticati - za vas  su one nečiste.'" 
\par 9 Od svih vodenih životinja ove možete jesti: sve što živi  u vodi, bilo u morima, bilo u rijekama, a ima peraje i ljuske  možete jesti. 
\par 10 A što u morima i rijekama nema peraja i ljusaka  - sve životinjice u vodi, sva živa vodena bića - neka su vam  odvratna 
\par 11 i odvratna neka vam ostanu! Mesa od njih nemojte  jesti, a njihove strvine držite za odvratnost. 
\par 12 Sve, dakle, što je u vodi a nema peraja i ljusaka neka je za vas odvratno." 
\par 13 "Od ptica neka su vam ove odvratne i neka se ne jedu  - odvratnost su: orao, orao strvinar i jastreb, 
\par 14 tetrijeb  i sokol bilo koje vrste; 
\par 15 gavran svih vrsta; 
\par 16 noj, kobac  i galeb; lastavica svake vrste; 
\par 17 sova, gnjurac, ušara, 
\par 18 labud, pelikan, droplja; 
\par 19 roda, čaplja svake vrste; pupavac i šišmiš." 
\par 20 "Svi krilati kukci što hodaju četveronoške neka su vam  odvratni! 
\par 21 Od svih tih krilatih kukaca što hodaju četveronoške  možete jesti samo one koji imaju na svojim nožicama listove za  skakutanje po zemlji. 
\par 22 Od njih možete jesti: svaku vrstu skakavaca, cvrčaka i zrikavaca. 
\par 23 A svi drugi krilati kukci na četiri  nožice neka su vam odvratni! 
\par 24 I od njih ćete se onečistiti:  tko se god dotakne njihove crkotine, neka je nečist do večeri; 
\par 25 tko god ponese crkotinu bilo koje od njih, neka opere svoju  odjeću i nečistim se smatra do večeri; 
\par 26 i životinje s nerazdvojenim  papkom što ne preživaju za vas su nečiste, i tko ih se dotakne  neka je nečist! 
\par 27 Onda, četveronožne životinje koje hodaju  na četiri šape za vas su nečiste. Tko se dotakne njihova strva, neka je nečist do večeri. 
\par 28 A onaj koji ponese njihov strv, neka opere svoju odjeću i bude nečist do večeri. Za vas su one  nečiste." 
\par 29 "Od životinja što po zemlji gmižu neka su za vas ove  nečiste: krtica, miš i svaka vrsta guštera; 
\par 30 zidni macaklin, kameleon, daždevnjak, zelembać i tinšamet. 
\par 31 Te su životinje od svih što gmižu za vas nečiste. Tko  ih se mrtvih dotakne neka je nečist do večeri. 
\par 32 A na što koja  od njih mrtva padne, neka je onečišćeno: bio to kakav drveni  predmet ili odjeća, koža ili vreća. Svaki takav predmet koji  se upotrebljava neka se zamoči u vodu i ostane nečist do večeri.  Onda će postati čist. 
\par 33 Upadne li što od njih u kakvu zemljanu  posudu, razbijte je; sve je u njoj onečišćeno. 
\par 34 A bilo kakva  hrana što se jede, ako na nju kapne voda iz te posude, bit će  onečišćena. Svaka tekućina što se pije u svakoj takvoj posudi  neka se smatra nečistom. 
\par 35 A sve na što padne bilo što od njihove crkotine neka  je nečisto; bude li to peć ili ognjište, neka se sruše: onečišćeni  su za vas i neka nečisti budu. 
\par 36 A vrelo ili čatrnja koja drži  vodu neka se smatraju čistima. Ali tko dirne strvinu životinje  neka je nečist. 
\par 37 Ako što od njihova strva padne na žitno sjemenje  što će se sijati, ono ostaje čisto; 
\par 38 ali ako se sjemenje nakvasi  vodom, a onda na nj padne što od njihove crkotine, neka je za  vas nečisto. 
\par 39 Ako ugine koja životinja što vam služi za hranu, onaj  koji dotakne njezinu strvinu neka je nečist do večeri; 
\par 40 a  koji pojede od njezine strvine neka opere svoju odjeću i bude  nečist do večeri; koji ponese njezinu strvinu neka opere svoju  odjeću i bude nečist do večeri. 
\par 41 Svaka životinja što po tlu gmiže odvratna je. Neka se  ne jede! 
\par 42 Ništa što puže na trbuhu, ništa što god ide na četiri  noge ili na više nogu - nikakve puzavce što po tlu gmižu nemojte  jesti jer su odvratni! 
\par 43 Nemojte sami sebe poganiti svim tim  puzavcima što gmižu; ne prljajte se njima, da i vi zbog njih  ne postanete nečisti. 
\par 44 Ta ja - Jahve - Bog sam vaš! Posvećujte  se, dakle, da sveti budete, jer svet sam ja! Nijednim se puzavcem  što po tlu gmiže ne prljajte! 
\par 45 Jest, ja sam Jahve; izveo sam  vas iz zemlje egipatske da vam budem Bog. Budite, dakle, sveti  jer sam svet ja!" 
\par 46 To je odredba koja se odnosi na ptice i sva živa bića  što se u vodi kreću i na sve stvorove koji po zemlji gmižu. 
\par 47 Svrha  joj je da se razlikuje nečisto od čistoga; životinja koja se  može jesti od životinje koja se ne smije jesti. 


\chapter{12}

\par 1 Jahve reče Mojsiju: 
\par 2 "Kaži Izraelcima: 'Kad koja žena zatrudni  i rodi muško čedo, neka je nečista sedam dana, kako je nečista  u vrijeme svoga mjesečnog pranja. 
\par 3 Osmoga dana neka se dijete  obreže. 
\par 4 A ona neka ostane još trideset i tri dana da se očisti  od svoje krvi; ne smije dirati ništa posvećeno niti dolaziti  u Svetište dok se ne navrši vrijeme njezina čišćenja. 
\par 5 Ako  rodi žensko čedo, neka je nečista dva tjedna, kao za svoga mjesečnog  pranja, i neka ostane još šezdeset i šest dana da se očisti od  svoje krvi. 
\par 6 A kad se navrši vrijeme njezina čišćenja - bilo  za sinčića, bilo za kćerkicu - neka donese svećeniku na ulaz  u Šator sastanka jednogodišnje janje za žrtvu paljenicu i jednoga  golubića ili grlicu za žrtvu okajnicu. 
\par 7 Neka on to prinese  pred Jahvom i nad njom izvrši obred pomirenja. Tako će ona biti  očišćena od svoga krvarenja. To je odredba koja se odnosi na  ženu kad rodi bilo muško bilo žensko čedo. 
\par 8 Ali ako ne može  da nađe dovoljno sredstava za grlo od sitnoga stada, neka onda  uzme dvije grlice ili dva golubića - jedno za žrtvu paljenicu, a drugo za žrtvu okajnicu. Neka svećenik izvrši nad njom obred  pomirenja, i ona će biti očišćena'." 


\chapter{13}

\par 1 Reče Jahve Mojsiju i Aronu: 
\par 2 "Ako se kome na koži pojavi  oteklina ili lišaj ili bjelkasta pjega što bi bila nagovještaj  gube na koži njegova tijela, neka se takav dovede svećeniku Aronu  ili kojemu od njegovih sinova svećenika. 
\par 3 Neka svećenik pregleda  zaraženo mjesto na koži njegova tijela. Ako je dlaka na zaraženom  mjestu postala bijela i učini se da je ono dublje od kože njegova  tijela, onda je to guba. Pošto ga svećenik pregleda, neka ga  proglasi nečistim. 
\par 4 Ali ako se pokaže da bjelkasta pjega na  koži njegova tijela nije dublja nego i koža, a dlaka na njoj  nije pobijeljela, neka onda svećenik bolesnika osami sedam dana. 
\par 5 Neka ga sedmoga dana opet svećenik pregleda. Ako ustanovi  svojim očima da zaraza još postoji, ali da se po koži dalje ne  širi, neka ga osami još sedam dana. 
\par 6 Sedmoga dana neka ga opet  pregleda. Bude li zaraženo mjesto manje upadno, a bolest se kožom  ne bude proširila, neka ga proglasi čistim: to je samo lišaj.  Pošto opere svoje haljine, bit će čist. 
\par 7 Ali ako se lišaj kožom  proširi, pošto je svećenik bolesnika pregledao i proglasio ga  čistim, neka se ponovo pokaže svećeniku. 
\par 8 Neka ga svećenik  pregleda. Bude li se lišaj proširio po koži, neka ga svećenik  proglasi nečistim: to je guba. 
\par 9 Ako se na čovjeku pokaže guba, neka ga dovedu svećeniku. 
\par 10 Neka ga svećenik pregleda. Ako po koži bude bjelkasta oteklina  s pobijeljelom dlakom i napetim čirom, 
\par 11 to je duboko ukorijenjena  guba po koži njegova tijela. Neka ga svećenik proglasi nečistim.  Ne treba ga osamljivati, jer je sigurno nečist. 
\par 12 Ako guba izbije po koži tako da bolesniku prekrije svu  kožu od glave do pete - sve što svećenikove oči vide - 
\par 13 neka  svećenik obavi pregled. Bude li guba prekrila sve njegovo tijelo, neka ga proglasi čistim. Budući da je sav pobijelio, čist je. 
\par 14 Ali onog dana kad se na njemu pokaže čir, bit će nečist. 
\par 15 Kad svećenik vidi taj čir, neka bolesnika proglasi nečistim:  čir je nečista stvar, to je guba. 
\par 16 Ali ako se čir promijeni  u bijelo, neka čovjek dođe k svećeniku. 
\par 17 Svećenik neka ga  pregleda. Bude li rana postala bijela, neka svećenik proglasi  bolesnika čistim - čist i jest." 
\par 18 "Kad se kome na koži napne čir i zacijeli, 
\par 19 i ondje  gdje je bio čir pojavi se bjelkasta oteklina ili mjesto izblijedi  i postane bjelkasto, ili izbije bijelocrvenkasta pjega, neka  se taj čovjek pokaže svećeniku. 
\par 20 Neka ga svećenik pregleda.  Pronađe li da je tu koža udubljenija a dlaka pobijeljela, neka  ga svećenik proglasi nečistim - to je onda guba što je izbila  u čiru. 
\par 21 Ali ako svećenik ustanovi da tu dlaka nije pobijeljela, da koža nije udubljenija nego drugdje, da mjesto tamni, neka  bolesnika osami sedam dana. 
\par 22 Proširi li mu se bolest po koži, neka ga svećenik proglasi nečistim - to je guba. 
\par 23 Ako pjega  ostane na mjestu i ne proširi se, to je ožiljak od čira. Neka  toga čovjeka svećenik proglasi čistim." 
\par 24 "Kome na koži bude opeklina, pa mjesto opekline postane  pjega bijelocrvenkasta ili bjelkasta, 
\par 25 neka to svećenik pregleda.  Ako dlaka na mjestu bude pobijeljela i učini se da je to mjesto  udubljenije od kože, onda je to guba što je u opeklini izbila.  Neka ga svećenik proglasi nečistim; to je guba. 
\par 26 Ali ako svećenik  ustanovi da dlaka nije pobijeljela, da mjesto nije udubljenije  od kože i da tamni, neka ga osami sedam dana. 
\par 27 Sedmoga dana  neka ga pregleda. Ako se pjega po koži proširi, neka ga svećenik  proglasi nečistim: to je guba. 
\par 28 Ostane li ozljeda na mjestu  i proširi se po koži, to je onda oteklina od opekline. Neka čovjeka  svećenik proglasi čistim: to je ožiljak od opekline." 
\par 29 "Ako se na glavi ili na bradi kojega čovjeka ili žene  pokaže bolest, 
\par 30 neka svećenik bolest pregleda. Ustanovi li  se da je dublje od kože i da je tu dlaka požutjela i otančala, neka bolesnika svećenik proglasi nečistim. To je šuga, to jest  guba na glavi ili na bradi. 
\par 31 Ali ako svećenik, pregledavši  oboljelo mjesto, ustanovi da nije dublje od kože, ali da tu ipak  nema crne dlake, neka svećenik odstrani šugavca sedam dana. 
\par 32 Sedmoga  dana neka ga svećenik pregleda. Ako se šuga nije proširila niti  dlaka požutjela, te ako se čini da šuga nije dublja od kože, 
\par 33 neka se bolesnik obrije - ali ošugano mjesto da ne brije!  - i neka ga svećenik odstrani od drugih sedam dana. 
\par 34 Sedmoga  dana neka opet svećenik pregleda šugavo mjesto. Ako se šuga kožom  ne bude proširila i učini se da nije dublja od kože, neka tog  bolesnika svećenik proglasi čistim. On neka opere svoju odjeću  i bude čist. 
\par 35 Proširi li se šuga po koži pošto je bio čistim  proglašen, 
\par 36 neka ga svećenik ponovo pregleda. Ako se šuga  kožom bude proširila - svećenik neka više i ne traži žute dlake  - bolesnik ja nečist. 
\par 37 Ali ako opazi da je šuga stala i da  je nikla crna dlaka, šuga je zacijeljela - on je čist. Neka ga  svećenik proglasi čistim." 
\par 38 "Ako se na koži kojeg čovjeka ili žene pokažu pjege te  ako su te pjege bijele, 
\par 39 neka ih svećenik pregleda. Ako te  pjege po koži budu tamnobijele, onda je to osip što je izbio  po koži: bolesnik je čist." 
\par 40 "Ako čovjeku opadne kosa s glave, oćelavio mu je zatiljak, ali je čist. 
\par 41 Ako mu sprijeda opadne kosa s glave, oćelavio  je na čelu, ali je čist. 
\par 42 Ali ako se po ćelavu zatiljku ili  po oćelavljelu čelu pojavi crvenkastobijela bolest, to je guba  što je izbila po njegovu ćelavom zatiljku ili oćelavljelu čelu. 
\par 43 Neka ga svećenik pregleda. Ako ustanovi da je osip na ćelavu  zatiljku ili po oćelavljelu čelu bjelkastocrvenkast - naizgled  kao i guba na koži tijela - 
\par 44 čovjek se ogubavio, nečist je.  Svećenik ga mora proglasiti nečistim - guba mu je na glavi." 
\par 45 "Onaj koji se bude ogubavio, neka nosi rasparanu odjeću;  kosa neka mu je raščupana; gornju usnu neka prekrije i viče:  "Nečist! Nečist!" 
\par 46 Sve dok na njemu bude bolest, neka nečistim  ostane, a kako je nečist, neka stanuje nasamo: neka mu je stan  izvan tabora." 
\par 47 "Kad se zaraza gube pokaže na odijelu, bilo vunenu bilo  lanenu, 
\par 48 na osnovi ili na potki od lana ili vune; ili na koži;  ili na bilo kakvu predmetu od kože; 
\par 49 pa ako mrlja na odijelu  ili koži, na osnovi ili na potki, ili na bilo kakvu predmetu  od kože, bude zelenkasta ili crvenkasta, to je guba i neka se  svećeniku pokaže. 
\par 50 Neka svećenik, pošto pregleda što je zaraženo, to stavi na osamu sedam dana. 
\par 51 Onda neka sedmoga dana zarazu  pregleda. Ako se zaraza proširi po odijelu, po osnovi ili potki, ili po koži, ili po kakvu god predmetu od kože, to je zarazna  guba. Stvar je nečista. 
\par 52 To odijelo - bilo osnova bilo potka, od vune ili lana - ili kakav kožni predmet za koji je zaraza  prionula, gubom se zarazio; neka na vatri izgori. 
\par 53 Ali ako svećenik opazi da se zaraza nije proširila na  odijelu - na osnovi ni na potki - niti na bilo kakvu kožnom predmetu, 
\par 54 onda neka naredi da se zaražena stvar opere. Neka je zatim  stavi nasamo drugih sedam dana. 
\par 55 A ako, pošto je stvar bila  oprana, svećenik opazi da se zaraženo mjesto nije promijenilo, ipak, mada se bolest nije raširila, stvar je nečista. Neka se  na vatri spali: trula je i iznutra i izvana. 
\par 56 Opazi li svećenik da se bolest smanjuje nakon pranja, neka to mjesto izreže, bilo ono na odijelu ili na koži, na osnovi  ili na potki. 
\par 57 Ako se na odijelu opet pojavi, u osnovi ili  potki, ili bilo kakvu kožnom predmetu, onda je to zaraza, i zaraženi  predmet neka u vatri izgori. 
\par 58 Ako li bolest nestane s odijela  - osnove ili potke - ili bilo kakva kožnoga predmeta pošto je  bio opran, neka se opere opet, pa neka je čist." 
\par 59 To su propisi za bolest gube na odijelu od vune ili lana  - u osnovi ili potki - ili bilo kakvu predmetu od kože da se  proglase čistim ili nečistim. 


\chapter{14}

\par 1 Jahve reče Mojsiju: 
\par 2 "Neka ovo bude obred za gubavca na  dan njegova čišćenja: neka se dovede svećeniku; 
\par 3 neka svećenik  iziđe iz tabora i obavi pregled. Ako ustanovi da je gubavac od  gube ozdravio, 
\par 4 neka naredi da se za čovjeka koji se ima čistiti  uzmu dvije ptice, čiste i žive, cedrovine, grimiznog prediva  i izopa. 
\par 5 Neka zatim svećenik naredi da se jedna ptica zakolje  nad živom vodom u zemljanoj posudi. 
\par 6 Potom neka uzme živu pticu, a onda zajedno živu pticu, cedrovinu, grimizno predivo i izop  zamoči u krv ptice što je bila zaklana povrh žive vode. 
\par 7 Sada  neka sedam puta poškropi onoga koji se od gube čisti, a onda  ga čistim proglasi. Poslije toga neka pusti živu pticu na otvorenu  polju. 
\par 8 Onaj koji se čisti neka opere svoju odjeću, obrije  sve svoje dlake i u vodi se okupa. Tako neka je čist. Poslije  toga neka uđe u tabor, ali sedam dana neka stanuje izvan svoga  šatora. 
\par 9 Sedmi dan neka obrije sve svoje dlake: kosu, bradu  i obrve; neka obrije sve ostale svoje dlake. Pošto u vodi opere  svoju odjeću i okupa se, neka je čist. 
\par 10 Osmoga dana neka uzme muško janje bez mane, jedno žensko  janje od godine dana, također bez mane, tri desetine efe najboljeg  brašna zamiješena u ulju za žrtvu prinosnicu i jedan log ulja. 
\par 11 Svećenik koji vrši čišćenje neka ih stavi pred Jahvu na ulazu  u Šator sastanka s čovjekom koji se ima čistiti. 
\par 12 Neka zatim  svećenik uzme jedno muško janje pa ga s ono ulja u logu prinese  kao žrtvu naknadnicu. Neka ih prinese pred Jahvom kao žrtvu prikaznicu. 
\par 13 Neka janje zakolje ondje gdje se kolju žrtve okajnice i žrtve  paljenice - na svetome mjestu, jer žrtva naknadnica kao i okajnica  pripada svećeniku: vrlo je sveta! 
\par 14 Potom neka svećenik uzme  krvi od žrtve naknadnice, pa neka njome namaže resicu desnoga  uha, palac desne ruke i palac desne noge onoga koji se čisti. 
\par 15 Poslije toga neka uzme log s uljem i izlije na dlan svoje  lijeve ruke. 
\par 16 Zamočivši svećenik svoj desni prst u ulje na  svojoj lijevoj ruci, neka uljem sa svoga prsta obavi škropljenje  pred Jahvom sedam puta. 
\par 17 Od ulja što mu preostane u ruci neka  svećenik, po krvi od žrtve naknadnice, pomaže resicu desnoga  uha, palac desne ruke i palac desne noge onoga koji se čisti. 
\par 18 Ostatak ulja sa svoje ruke neka svećenik metne na glavu onoga  koji se čisti. Tako će svećenik nad njim izvršiti obred pomirenja  pred Jahvom. 
\par 19 Neka svećenik poslije toga prinese žrtvu okajnicu  i nad onim koji se čisti neka obavi obred pomirenja za njegovu  nečistoću. Napokon neka zakolje žrtvu paljenicu, 
\par 20 a onda neka  svećenik žrtvu paljenicu i žrtvu prinosnicu podigne na žrtvenik.  Kad tako svećenik nad njim obavi obred pomirenja, neka je čist. 
\par 21 Ako bude siromašan te ne mogne to priskrbiti, neka uzme  samo jedno muško janje za žrtvu naknadnicu i neka se ono prinese  kao žrtva prinosnica da se nad tim čovjekom izvrši obred pomirenja.  I neka uzme samo desetinu efe najboljeg brašna zamiješena u ulju  za žrtvu prinosnicu, jedan log ulja, 
\par 22 k tome dvije grlice  ili dva golubića - prema svojim mogućnostima - jedno za žrtvu  okajnicu, a drugo za žrtvu paljenicu. 
\par 23 Osam dana nakon svoga  očišćenja neka ih donese svećeniku na ulaz u Šator sastanka pred  Jahvu. 
\par 24 Neka svećenik uzme janje za žrtvu naknadnicu i log  s uljem pa ih prinese pred Jahvom kao žrtvu prikaznicu. 
\par 25 Neka  se onda zakolje janje žrtve naknadnice, a svećenik neka uzme  njegove krvi i neka njome namaže resicu desnoga uha, palac desne  ruke i palac desne noge onoga koji se čisti. 
\par 26 Poslije toga  neka svećenik izlije ulje na dlan svoje lijeve ruke. 
\par 27 A onda  neka od ulja što mu je na dlanu lijeve ruke obavi škropljenje  sedam puta prstom svoje desne ruke pred Jahvom. 
\par 28 Od ulja iz  svoje ruke neka svećenik, po krvi žrtve naknadnice, namaže resicu  desnog uha, palac desne ruke i palac desne noge onoga koji se  čisti. 
\par 29 Ostatak ulja što bude na dlanu neka svećenik stavi  na glavu onoga koji se čisti, vršeći nad njim obred pomirenja  pred Jahvom. 
\par 30 Neka zatim prinese jednu od dviju grlica ili  jednoga od dvaju golubića - što je već mogao pribaviti - 
\par 31 kao  žrtvu okajnicu, a drugu kao žrtvu paljenicu zajedno sa žrtvom  prinosnicom. Neka tako svećenik izvrši obred pomirenja pred Jahvom  nad onim koji se čisti." 
\par 32 To je propis za onoga koji je gubom zaražen a ne može  priskrbiti sve za svoje očišćenje. 
\par 33 Jahve reče Mojsiju i Aronu: 
\par 34 "Kad uđete u kanaansku  zemlju koju ću vam dati u posjed, a ja pustim gubu na koju kuću  u zemlji što je budete zaposjeli, 
\par 35 onda onaj čija je kuća  neka dođe svećeniku i kaže: 'Čini mi se da je moja kuća zaražena  gubom.' 
\par 36 Neka svećenik naredi da se kuća isprazni prije nego  on dođe da bolest pregleda, da ne bi sve što je u kući bilo proglašeno  nečistim; poslije toga neka svećenik uđe da kuću pregleda. 
\par 37 Ako  nakon pregleda zapazi da je bolest na kućnim zidovima od zelenkastih  ili crvenkastih udubina i pričini mu se da idu dublje od površine  zida, 
\par 38 neka svećenik iziđe iz kuće na kućna vrata i neka kuću  zatvori sedam dana. 
\par 39 Sedmi dan neka svećenik opet dođe i pregleda:  ako se bolest bude proširila po zidovima kuće, 
\par 40 neka svećenik  naredi da se povadi zaraženo kamenje i baci na koje nečisto mjesto  izvan grada. 
\par 41 Zatim neka zapovjedi da se svi unutarnji zidovi  kuće ostružu i da se sastrugani prah baci na koje nečisto mjesto  izvan grada. 
\par 42 Onda neka se uzme drugo kamenje i umetne namjesto  onoga kamenja. Potom neka se uzme druga žbuka i kuća ponovo ožbuka. 
\par 43 Ako se pošast na kući opet pojavi pošto je kamenje bilo  povađeno i kuća ostrugana i opet ožbukana, 
\par 44 neka svećenik  ode da pregleda: bude li se bolest po kući proširila, to je onda  u kući zarazna guba; kuća je nečista. 
\par 45 Neka se kuća poruši, a njezino kamenje, njezina drvena građa i sva žbuka s kuće neka  se odnese izvan grada na koje nečisto mjesto. 
\par 46 Tko uđe u kuću dok je zatvorena, neka je nečist do večeri. 
\par 47 Tko u kući legne, mora oprati svoju odjeću. I tko u kući  objeduje, mora svoju odjeću oprati. 
\par 48 Ako li svećenik dođe  i vidi da se bolest po kući nije proširila pošto je kuća opet  bila ožbukana, neka svećenik kuću proglasi čistom, jer se bolest  izliječila. 
\par 49 A za očišćenje kuće neka uzme: dvije ptice, cedrovine, grimizna prediva i izopa. 
\par 50 Jednu od ptica neka zakolje nad  živom vodom u zemljanoj posudi. 
\par 51 Potom neka uzme: cedrovinu, izop, grimizno predivo i pticu živu te ih zamoči u krv ptice  zaklane i u živu vodu pa kuću poškropi sedam puta. 
\par 52 Očistivši  tako od grijeha kuću krvlju ptice, živom vodom, živom pticom, cedrovinom, izopom i grimiznim predivom, 
\par 53 neka pticu živu  pusti izvan grada na otvorenu polju. Kad tako obavi obred pomirenja  nad kućom, bit će čista." 
\par 54 To je propis za svaku vrst gube i šuge, 
\par 55 za gubu odjeće  ili kuće, 
\par 56 za otekline, lišaje ili pjege. 
\par 57 On određuje  vrijeme nečistoće i čistoće. To je zakon o gubi. 


\chapter{15}

\par 1 Jahve reče Mojsiju i Aronu: 
\par 2 "Govorite Izraelcima i kažite  im: 'Kad koji čovjek imadne izljev iz svoga tijela, njegov je  izljev nečist. 
\par 3 Evo u čemu je njegova nečistoća ako ima taj  izljev: ispusti li njegovo tijelo izljev ili ga zadrži, on je  nečist. 
\par 4 Svaka postelja na koju legne onaj koji ima izljev  neka je nečista; i svaki predmet na koji sjedne neka je nečist. 
\par 5 A svaki koji se dotakne njegove posteljine neka opere svoju  odjeću, u vodi se okupa i nečistim ostane do večeri. 
\par 6 Tko god  sjedne na predmet na kojemu je sjedio onaj koji je imao izljev  neka opere svoju odjeću, u vodi se okupa i nečistim ostane do  večeri. 
\par 7 Tko se dotakne tijela onoga koji je imao izljev neka  opere svoju odjeću, u vodi se okupa i do večeri nečistim ostane. 
\par 8 Ako onaj koji ima izljev pljune na koga tko je čist neka taj  opere svoju odjeću, u vodi se okupa i nečistim ostane do večeri. 
\par 9 Neka je nečisto i svako sjedalo na koje za vožnje sjedne onaj  koji ima izljev; 
\par 10 i tko se dotakne čega što je pod tim bolesnikom  bilo neka je nečist do večeri. Tko ponese štogod takvo neka svoju  odjeću opere, u vodi se okupa i ostane nečistim do večeri. 
\par 11 A  svaki koga se onaj koji ima izljev dotakne neopranih ruku neka  svoju odjeću opere, u vodi se okupa i ostane nečistim do večeri. 
\par 12 Zemljana posuda koje se dotakne onaj s izljevom neka se razbije, a svaki drveni sud neka se vodom ispere. 
\par 13 Kad se onaj koji ima izljev od toga izliječi, neka onda  nabroji sedam dana za svoje oćišćenje; neka opere svoju odjeću, okupa se u živoj vodi i neka je čist. 
\par 14 Osmoga pak dana neka  uzme dvije grlice ili dva golubića, dođe pred Jahvu na ulaz u  Šator sastanka pa ih svećeniku preda. 
\par 15 Neka ih svećenik prinese  jedno kao žrtvu okajnicu, a drugo kao žrtvu paljenicu. Time će  svećenik izvršiti obred pomirenja nad tim čovjekom za njegov  izljev. 
\par 16 Kad čovjek imadne sjemeni izljev, neka u vodi okupa cijelo  svoje tijelo i ostane nečistim do večeri. 
\par 17 Svaka haljina i  svaka koža na koju dospije takav sjemeni izljev neka se u vodi  opere i ostane nečistom do večeri. 
\par 18 Ako koja žena legne s kojim čovjekom i on ispusti sjeme, neka se okupaju u vodi i budu nečisti do večeri'." 
\par 19 "Kad žena imadne krvarenje, izljev krvi iz svoga tijela, neka ostane u svojoj nečistoći sedam dana; tko se god nje dotakne  neka je nečist do večeri. 
\par 20 Na što god bi legla za svoje nečistoće  neka je nečisto; na što god sjedne neka je nečisto. 
\par 21 Tko se  dotakne njezine posteljine neka opere svoju odjeću, u vodi se  okupa i do večeri ostane nečistim. 
\par 22 Tko god dotakne bilo koji  predmet na kojemu je ona sjedila neka svoju odjeću opere, u vodi  se okupa i nečist ostane do večeri. 
\par 23 A ako bi se dotakao čega  što je bilo na njezinoj postelji ili na predmetu na kojem je  ona sjedila, neka je nečist do večeri. 
\par 24 Ako koji čovjek s  njom legne, njezina nečistoća za nj prianja, pa neka je nečist  sedam dana. Svaka postelja na koju on legne neka je nečista. 
\par 25 Ako žena imadne krvarenje dulje vremena izvan svoga mjesečnog  pranja, ili ako se njezino mjesečno pranje produžuje, neka se  smatra nečistom sve vrijeme krvarenja kao da su dani njezina  mjesečnog pranja. 
\par 26 Svaka postelja na koju legne za sve vrijeme  svoga krvarenja bit će joj kao i postelja za njezina mjesečnog  pranja. I svaki predmet na koji sjedne neka postane nečistim  kao što bi bio nečist u vrijeme njezina mjesečnog pranja. 
\par 27 A  svatko tko ih se dotakne neka je nečist; neka opere svoju odjeću, okupa se u vodi i ostane nečistim do večeri. 
\par 28 Ako ozdravi od svog krvarenja, neka namiri sedam dana, a poslije toga neka je čista. 
\par 29 Osmoga dana neka uzme dvije  grlice ili dva golubića te ih donese svećeniku na ulaz u Šator  sastanka. 
\par 30 Neka jedno svećenik prinese kao žrtvu okajnicu, a drugo kao žrtvu paljenicu. Tako će svećenik obaviti pred Jahvom  obred pomirenja nad njom, za njezino nečisto krvarenje." 
\par 31 "Odvraćajte Izraelce od njihovih nečistoća, da ne bi  zbog njih pomrli oskvrnjujući moje Prebivalište koje se nalazi  među njima. 
\par 32 To je propis za čovjeka koji ima izljev; za onoga koga  čini nečistim sjemeni izljev; 
\par 33 za ženu u vrijeme nečistoće  njezina mjesečnog pranja; za svakoga - bilo muško bilo žensko  - tko imadne izljev, a tako i za čovjeka koji legne s onečišćenom  ženom." 


\chapter{16}

\par 1 Poslije smrti dvojice Aronovih sinova, koji su poginuli prinoseći  pred Jahvom neposvećenu vatru, progovori Jahve Mojsiju. 
\par 2 Jahve  reče Mojsiju: "Kaži svome bratu Aronu da ne ulazi u svako doba u Svetište  iza zavjese, pred Pomirilište koje se nalazi na Kovčegu, da ne  pogine. Jer ja ću se pojavljivati nad Pomirilištem u oblaku. 
\par 3 Neka Aron ulazi u Svetište ovako: s juncem za žrtvu okajnicu  i ovnom za žrtvu paljenicu. 
\par 4 Neka se obuče u posvećenu košulju  od lana; na svoje tijelo neka navuče gaće od lana; neka se opaše  lanenim pasom, a na glavu stavi mitru od lana. To je posvećeno  ruho koje ima obući pošto se okupa u vodi. 
\par 5 Od zajednice izraelske  neka primi dva jarca za žrtvu okajnicu i jednoga ovna za žrtvu  paljenicu. 
\par 6 Pošto Aron prinese junca za žrtvu okajnicu za svoj grijeh  i izvrši obred pomirenja za se i za svoj dom, 
\par 7 neka uzme oba  jarca i postavi ih pred Jahvu na ulaz u Šator sastanka. 
\par 8 Neka  Aron baci kocke za oba jarca te jednoga odredi kockom Jahvi,  a drugoga Azazelu. 
\par 9 Jarca na kojega je kocka pala da bude Jahvi  neka Aron prinese za žrtvu okajnicu. 
\par 10 A jarac na kojega je  kocka pala da bude Azazelu neka se smjesti živ pred Jahvu, da  se nad njim obavi obred pomirenja i otpremi Azazelu u pustinju. 
\par 11 Zatim neka Aron prinese junca za žrtvu okajnicu za svoj  grijeh; i obavi obred pomirenja za se i za svoj dom: i neka zakolje  toga junca za žrtvu okajnicu za svoj grijeh. 
\par 12 Potom neka uzme  kadionik pun užarena ugljevlja sa žrtvenika ispred Jahve i dvije  pune pregršti miomirisnoga tamjana u prah smrvljenoga. Neka to  unese iza zavjese. 
\par 13 Sad neka stavi tamjan na vatru pred Jahvom  da oblak od tamjana zastre Pomirilište što je na Svjedočanstvu.  Tako neće poginuti. 
\par 14 Poslije toga neka uzme krvi od junca  i svojim prstom poškropi istočnu stranu Pomirilišta; a ispred  Pomirilišta neka svojim prstom poškropi sedam puta tom krvlju. 
\par 15 Neka potom zakolje jarca za žrtvu okajnicu za grijeh  naroda; neka unese njegovu krv za zavjesu te s njegovom krvi  učini kako je učinio s krvlju od junca: neka njome poškropi po  Pomirilištu i pred njim. 
\par 16 Tako će obaviti obred pomirenja  nad Svetištem zbog nečistoća Izraelaca, zbog njihovih prijestupa  i svih njihovih grijeha. A tako neka učini i za Šator sastanka  što se među njima nalazi, sred njihovih nečistoća. 
\par 17 Kad on  uđe da obavi obred pomirenja u Svetištu, neka nikoga drugog ne  bude u Šatoru sastanka dok on ne iziđe. Obavivši obred pomirenja za se, za svoj dom i za svu izraelsku  zajednicu, 
\par 18 neka ode k žrtveniku koji se nalazi pred Jahvom  te nad žrtvenikom obavi obred pomirenja. Neka uzme krvi od junca  i krvi od jarca pa stavi na rogove oko žrtvenika. 
\par 19 Neka svojim  prstom poškropi žrtvenik istom krvlju sedam puta. Tako će ga  očistiti od nečistoća Izraelaca i posvetiti. 
\par 20 Kad svrši obred pomirenja Svetišta, Šatora sastanka i  žrtvenika, neka primakne jarca živoga. 
\par 21 Neka mu na glavu Aron  stavi obje svoje ruke i nad njim ispovjedi sve krivnje Izraelaca, sve njihove prijestupe i sve njihove grijehe. Položivši ih tako  jarcu na glavu, neka ga pošalje u pustinju s jednim prikladnim  čovjekom. 
\par 22 Tako će jarac na sebi odnijeti sve njihove krivnje  u pusti kraj. Otpremivši jarca u pustinju, 
\par 23 neka se Aron vrati u Šator  sastanka, sa sebe svuče lanenu odjeću u koju se bio obukao kad  je ulazio u Svetište i neka je ondje ostavi. 
\par 24 Neka potom opere  svoje tijelo vodom na posvećenu mjestu, na se obuče svoju odjeću  te iziđe da prinese svoju žrtvu paljenicu i žrtvu paljenicu naroda  i obavi obred pomirenja za se i za narod. 
\par 25 Loj sa žrtve okajnice  neka sažeže u kad na žrtveniku. 
\par 26 Onaj koji je odveo jarca Azazelu neka opere svoju odjeću, svoje tijelo u vodi okupa i poslije toga može opet doći u tabor. 
\par 27 A junca žrtve okajnice i jarca žrtve okajnice od kojih  je krv bila donesena u Svetište da se obavi obred pomirenja neka  odnesu izvan tabora pa neka na vatri spale njihove kože, njihovo  meso i njihovu nečist. 
\par 28 Tko ih bude spaljivao, neka opere  svoju odjeću, svoje tijelo okupa u vodi i poslije toga može opet  doći u tabor. 
\par 29 Ovaj zakon neka za vas trajno vrijedi. U sedmom mjesecu, deseti dan toga mjeseca, postite i ne obavljajte  nikakva posla: ni domorodac ni stranac koji među vama boravi. 
\par 30 Jer toga dana nad vama se ima izvršiti obred pomirenja da  se očistite od svih svojih grijeha te da pred Jahvom budete čisti. 
\par 31 Neka je to za vas subotnji počinak kad postite. Trajan je  to zakon. 
\par 32 Neka obred pomirenja obavi onaj svećenik koji bude pomazan  i posvećen za vršenje svećeničke službe namjesto svoga oca. Neka  se obuče u posvećeno laneno ruho; 
\par 33 on neka obavi obred pomirenja  za posvećeno Svetište, za Šator sastanka i za žrtvenik. Zatim  neka izvrši obred pomirenja nad svećenicima i nad svim narodom  zajednice. 
\par 34 Tako neka to bude za vas trajan zakon; jednom  na godinu neka se nad Izraelcima obavi obred pomirenja za sve  njihove grijehe." Mojsije je učinio kako mu je Jahve naredio. 


\chapter{17}

\par 1 Jahve reče Mojsiju: 
\par 2 "Govori Aronu, njegovim sinovima i  svima Izraelcima te im reci: 'Evo što je zapovjedio Jahve: 
\par 3 svaki  onaj od Izraelova doma koji u taboru ili izvan tabora zakolje  vola, ili ovcu, ili kozu, 
\par 4 a ne donese ih na ulaz u Šator sastanka  da se prinesu na dar Jahvi pred njegovim Prebivalištem, svaki  takav neka je odgovoran: prolio je krv i neka se odstrani iz  svoga naroda.' 
\par 5 Zato neka Izraelci svoje žrtve koje bi htjeli  klati vani u polju dovedu na ulaz u Šator sastanka, k svećeniku, i neka ih prinose kao žrtve pričesnice. 
\par 6 Neka svećenik izlije  krv po Jahvinu žrtveniku koji se nalazi na ulazu u Šator sastanka, a loj spali na ugodan miris Jahvi, 
\par 7 tako da ubuduće ne prinose  svojih žrtava klanica jarcima s kojima se odaju bludu. Neka je  ovo trajan zakon za njih i njihove naraštaje. 
\par 8 I kaži im: 'Svaki pojedinac od Izraelova doma, ili stranac  koji među vama boravi, koji prinese paljenicu ili klanicu 
\par 9 a  ne donese je na ulaz u Šator sastanka da se prinese Jahvi, taj  neka se odstrani iz svoga naroda.'" 
\par 10 "Nadalje, protiv svakoga pojedinca od Izraelova doma, a tako i protiv svakoga pridošlice među vama koji bi blagovao  bilo kakvu krv, ja ću se okrenuti, svakoga tko blaguje krv odstranit  ću iz njegova naroda. 
\par 11 Jer je život živoga bića u krvi. Tu  krv ja sam vama dao da na žrtveniku njome obavljate obred pomirenja  za svoje živote. Jer krv je ono što ispašta za život. 
\par 12 Zato  sam kazao Izraelcima: neka nitko od vas ne jede krvi; neka ni  stranac koji među vama bude ne jede krvi. 
\par 13 Tko god, Izraelac ili stranac koji među vama boravi,  uhvati u lovu kakvu zvijer ili pticu što se može jesti neka joj  prolije krv i zatrpa zemljom. 
\par 14 Jer život svakoga živog bića  jest njegova krv. Zato sam i rekao Izraelcima: ne smijete jesti  krvi ni od kakva živog bića, jer život svakoga živog bića jest  njegova krv. Tko god je bude jeo, neka se odstrani. 
\par 15 Tko bi god, Izraelac ili stranac, jeo što je uginulo  ili što su zvijeri rastrgale neka opere svoju odjeću, u vodi  se okupa i ostane nečistim do večeri. Tada će postati čist. 
\par 16 Ali  ako je ne opere i ne okupa svoga tijela, neka snosi posljedice  svoje krivnje." 


\chapter{18}

\par 1 Jahve reče Mojsiju: 
\par 2 "Govori Izraelcima i reci im: 'Ja sam  Jahve, Bog vaš. 
\par 3 Nemojte raditi kako se radi u zemlji egipatskoj, gdje ste boravili; niti radite kako se radi u zemlji kanaanskoj, kamo vas vodim; ne povodite se za njihovim običajima! 
\par 4 Vršite  moje naredbe; vršite moje zapovijedi; prema njima hodite. Ja  sam Jahve, Bog vaš. 
\par 5 Zato držite moje zakone i moje naredbe; tko ih vrši -  u njima će naći život. Ja sam Jahve! 
\par 6 Neka se nitko od vas ne približuje svojoj krvnoj rodbini  da otkriva njezinu golotinju. Ja sam Jahve! 
\par 7 Ne otkrivaj golotinje svoga oca ni golotinje svoje majke.  Majka ti je, ne otkrivaj njezine golotinje! 
\par 8 Ne otkrivaj golotinje žene svoga oca! I to je golotinja  tvoga oca! 
\par 9 Ne otkrivaj golotinje svoje sestre - kćeri svoga oca ili  kćeri svoje majke - bila rođena u kući ili izvan nje! 
\par 10 Ne otkrivaj golotinje kćeri svoga sina niti golotinje  kćeri svoje kćeri! TÓa njihova je golotinja tvoja vlastita golotinja. 
\par 11 Ne otkrivaj golotinje kćeri žene svoga oca! Jer, rođena  od tvog oca, ona ti je sestra. 
\par 12 Ne otkrivaj golotinje sestre svoga oca! Ona je krv tvoga  oca. 
\par 13 Ne otkrivaj ni golotinje sestre svoje majke! Ta i ona  je krv tvoje majke! 
\par 14 Ne otkrivaj golotinje svoga strica! To jest, nemoj se  približavati njegovoj ženi. Ta ona je tvoja strina. 
\par 15 Ne otkrivaj golotinje svoje snahe! Ona je žena tvoga  sina. Ne otkrivaj golotinje njezine. 
\par 16 Ne otkrivaj golotinje žene svoga brata! Ta to je golotinja  tvoga brata. 
\par 17 Ne otkrivaj golotinje koje žene i njezine kćeri! Nemoj  se ženiti kćerju njezina sina niti kćerju njezine kćeri te im  golotinju otkrivati. Oni su krvna rodbina. To bi bila pokvarenost. 
\par 18 Ne uzimaj sebi koju ženu u isto vrijeme kad i njezinu  sestru da je ljubomorom žalostiš otkrivajući golotinju ovoj preko  nje za njezina života! 
\par 19 Ne približuj se ni jednoj ženi kad je u nečistoći svoga  mjesečnog pranja da joj otkrivaš golotinju! 
\par 20 Ne lijegaj sa ženom bližnjega svoga; od nje bi postao  nečist. 
\par 21 Ne smiješ dopuštati da koje tvoje dijete bude žrtvovano  Moleku; ne smiješ tako obeščašćivati ime Boga svoga. Ja sam Jahve! 
\par 22 Ne lijegaj s muškarcem kako se liježe sa ženom! To bi  bila grozota. 
\par 23 Da nisi legao ni s jednom životinjom - od nje bi postao  nečist. Žena ne smije stati pred životinju da se s njom pari.  To bi bila krajnja opačina. 
\par 24 Ničim se od toga nemojte onečišćavati! Ta svim su se  tim onečišćavali narodi koje ja ispred vas tjeram. 
\par 25 I zemlja  je postala nečista. Zato ću kazniti njezinu opačinu, i zemlja  će ispljuvati svoje stanovnike. 
\par 26 Vi pak držite moje zakone  i moje naredbe: ni jedne od tih opačina nemojte počinjati - ni  vi ni stranac koji među vama boravi. 
\par 27 Sve je te zloće počinjao  svijet koji je bio u toj zemlji prije vas te je zemlja postala  nečista. 
\par 28 Neće li, ako je učinite nečistom, zemlja ispljuvati  i vas kako je ispljuvala narod koji je bio prije vas? 
\par 29 Jest, svi koji bi počinili bilo koju od tih zloća bit će odstranjeni  iz svoga naroda. 
\par 30 Zato držite moje zapovijedi; nemojte se  podavati ni jednome od onih odvratnih običaja što su se održavali  prije vas; tako se njima nećete onečistiti. Ja sam Jahve, Bog  vaš!'" 


\chapter{19}

\par 1 Jahve reče Mojsiju: 
\par 2 "Govori svoj zajednici Izraelaca i  reci im: 'Sveti budite! Jer sam svet ja, Jahve, Bog vaš! 
\par 3 Svoje se majke i svoga oca svaki bojte! Subote moje držite! Ja sam Jahve, Bog vaš! 
\par 4 Ne obraćajte se na ništavila! Ne pravite sebi lijevanih  kumira! Ja sam Jahve, Bog vaš! 
\par 5 Kad prinosite Jahvi žrtvu pričesnicu, prinesite je tako  da budete primljeni. 
\par 6 Neka se pojede na dan kad je prinosite  ili sutradan. Što preostane za prekosutra neka se spali na vatri. 
\par 7 Kad bi se jelo od toga jela treći dan, bilo bi odvratno i  žrtva ne bi bila primljena. 
\par 8 A onaj koji je ipak jede neka  snosi posljedice svoje krivnje. Budući da je oskvrnuo ono što  je Jahvi posvećeno, neka se takav odstrani iz svoga naroda. 
\par 9 Kad žetvu žanjete po svojoj zemlji, ne žanjite dokraja  svoje njive; niti pabirčite ostatke poslije svoje žetve. 
\par 10 Ne  paljetkuj svoga vinograda; ne kupi po svom vinogradu palih boba  nego ih ostavljaj sirotinji i strancu! Ja sam Jahve, Bog vaš. 
\par 11 Nemojte krasti; nemojte lagati i varati svoga bližnjega. 
\par 12 Nemojte se krivo kleti mojim imenom i tako oskvrnjivati ime  svoga Boga. Ja sam Jahve! 
\par 13 Ne iskorišćuj svoga bližnjega niti ga pljačkaj! Radnikova  zarada neka ne ostane pri tebi do jutra. 
\par 14 Nemoj psovati gluhoga  niti pred slijepca stavljaj zapreku. Svoga se Boga boj! Ja sam  Jahve! 
\par 15 Ne počinjajte nepravde u osudama! Ne budi pristran prema  neznatnome, niti popuštaj pred velikima; po pravdi sudi svome  bližnjemu! 
\par 16 Ne raznosi klevete među svojim narodom; ne izvrgavaj  pogibli krv svoga bližnjega. Ja sam Jahve! 
\par 17 Ne mrzi svoga brata u svom srcu! Dužnost ti je koriti  svoga sunarodnjaka. Tako nećeš pasti u grijeh zbog njega. 
\par 18 Ne  osvećuj se! Ne gaji srdžbe prema sinovima svoga naroda. Ljubi  bližnjega svoga kao samoga sebe. Ja sam Jahve! 
\par 19 Držite moje zapovijedi! Ne daj svome blagu da se pari s drugom vrstom. Svoga polja  ne zasijavaj dvjema vrstama sjemena. Ne stavljaj na se odjeće  od dvije vrste tkanine. 
\par 20 Ako bi tko legao s ropkinjom koja je zaručena za drugoga, a ona ne bude ni otkupljena ni oslobođena, treba ga kazniti, ali ne smrću, jer ona nije slobodna. 
\par 21 Neka on na ulazu u  Šator sastanka prinese Jahvi žrtvu naknadnicu, to jest jednoga  ovna kao žrtvu naknadnicu. 
\par 22 Neka svećenik tim ovnom žrtve  naknadnice izvrši nad tim čovjekom obred pomirenja pred Jahvom  za počinjeni grijeh. I grijeh koji je počinio bit će mu oprošten. 
\par 23 Kad uđete u zemlju i zasadite bilo kakvu voćku, smatrajte  njezine plodove za neobrezane. Tri godine neka vam budu neobrezani:  neka se ne jedu. 
\par 24 Četvrte godine neka se svi njezini plodovi  posvete na svetkovinu zahvale Jahvi. 
\par 25 Istom pete godine jedite  njezin plod i ubirite sebi njezin urod. Ja sam Jahve, Bog vaš! 
\par 26 Ništa s krvlju nemojte jesti! Ne gatajte! Ne čarajte! 
\par 27 Ne zaokružujte kose na svojim sljepoočnicama; ne šišajte  okrajka svoje brade. 
\par 28 Ne urezujte zareza na svome tijelu za  pokojnika; niti na sebi usijecajte kakvih biljega. Ja sam Jahve! 
\par 29 Ne obeščašćuj svoje kćeri dajući je za javnu bludnicu.  Tako se zemlja neće podati bludnosti niti će se napuniti pokvarenošću. 
\par 30 Držite moje subote; štujte moje Svetište. Ja sam Jahve! 
\par 31 Ne obraćajte se na zazivače duhova i vračare; ne pitajte  ih za savjet. Oni bi vas opoganili. Ja sam Jahve, Bog vaš! 
\par 32 Ustani pred sijedom glavom; poštuj lice starca; boj se  svoga Boga. Ja sam Jahve! 
\par 33 Ako se stranac nastani u vašoj zemlji, nemojte ga ugnjetavati. 
\par 34 Stranac koji s vama boravi neka vam bude kao sunarodnjak;  ljubi ga kao sebe samoga. TÓa i vi ste bili stranci u egipatskoj  zemlji. Ja sam Jahve, Bog vaš. 
\par 35 Ne počinjajte nepravde u osudama, u mjerama za duljinu, težinu i obujam. 
\par 36 Neka su vam mjerila točna; utezi jednaki;  efa prava; prav hin. Ja sam Jahve, Bog vaš, koji sam vas izveo  iz zemlje egipatske. 
\par 37 Držite sve moje zakone i sve moje naredbe; vršite ih.  Ja sam Jahve!'" 


\chapter{20}

\par 1 Jahve reče Mojsiju: 
\par 2 "Kaži Izraelcima: 'Tko god, Izraelac, ili stranac koji živi s Izraelcima, ustupi svoje čedo Moleku, mora se smaknuti; narod zemlje neka ga kamenuje. 
\par 3 Ja ću se  okrenuti protiv toga čovjeka i odstraniti ga iz njegova naroda, jer je on, ustupivši svoje čedo Moleku, okaljao moje Svetište  i obeščastio moje sveto ime. 
\par 4 A ako narod zatvori svoje oči  nad tim čovjekom kad svoje čedo ustupi Moleku te ga ne smakne, 
\par 5 ja ću se suprotstaviti tome čovjeku i njegovoj obitelji;  odstranit ću ih iz njihova naroda, njega i sve koji poslije njega  pođu za Molekom da se podaju bludu s Molekom. 
\par 6 Ako se tko obrati na zazivače duhova i vračare da se za  njima poda javnom bludu, ja ću se okrenuti protiv takva čovjeka  i odstranit ću ga iz njegova naroda. 
\par 7 Posvećujte se da budete sveti! TÓa ja sam Jahve, Bog vaš. 
\par 8 Držite moje zakone i vršite ih. Ja, Jahve, posvećujem vas'." 
\par 9 "Tko god prokune svoga oca i svoju majku, neka se smakne.  Jer je oca svoga i majku svoju prokleo, neka njegova krv padne  na nj. 
\par 10 Čovjek koji počini preljub sa ženom svoga susjeda neka  se kazni smrću - i preljubnik i preljubnica. 
\par 11 Čovjek koji bi legao sa ženom svoga oca - otkrio bi golotinju  svoga oca - neka se oboje kazne smrću, krv njihova neka padne  na njih. 
\par 12 Legne li tko sa svojom snahom, neka se oboje kazne smrću.  Učinili su rodoskvrnuće i neka krv njihova padne na njih. 
\par 13 Ako bi muškarac legao s muškarcem kao što se liježe sa  ženom, obojica bi počinila odvratno djelo. Neka se smaknu i krv  njihova neka padne na njih. 
\par 14 Čovjek koji se oženi kćerju i njezinom majkom - krajnja  je to pokvarenost! - neka se u vatri spali i on i one, da među  vama ne bude pokvarenosti. 
\par 15 Čovjek koji bi spolno općio sa životinjom ima se smaknuti.  Životinju ubijte! 
\par 16 Ako bi se žena primakla bilo kakvoj životinji da se s  njom pari, ubij i ženu i životinju. Neka se smaknu i njihova  krv neka padne na njih. 
\par 17 Čovjek koji bi se oženio svojom sestrom, kćerju svoga  oca ili kćerju svoje majke te vidio njezinu golotinju, a ona  vidjela njegovu - pogrdno je to djelo! - neka se istrijebe pred  očima naroda. Otkrio je golotinju svoje sestre, pa neka snosi  i posljedice svoje krivnje. 
\par 18 Čovjek koji bi legao sa ženom za njezina mjesečnog pranja  te otkrio njezinu golotinju - razgolio izvor njezine krvi i ona  sama otkrila izvor svoje krvi - neka se oboje odstrane iz svoga  naroda. 
\par 19 Ne otkrivaj golotinje sestre svoje majke niti sestre  svoga oca - to je otkrivanje golotinje svoga roda, neka snose  posljedice svoje krivnje. 
\par 20 Čovjek koji bi legao sa svojom  strinom otkrio bi golotinju svoga strica. Neka snose posljedice  svoga grijeha: neka umru bez poroda. 
\par 21 Čovjek koji bi se oženio  ženom svoga brata - golotinju bi svoga brata otkrio - i to je  nečisto. Neka ostanu bez poroda." 
\par 22 "Zato držite sve moje zakone, sve moje naredbe i vršite  ih da vas ne ispljune zemlja u koju vas vodim da se u njoj nastanite. 
\par 23 Nemojte živjeti po zakonima naroda koje ja ispred vas tjeram.  TÓa oni su činili sve to, i zato mi se zgadili. 
\par 24 A vama sam  ja rekao: vi ćete zaposjesti njihovu zemlju; vama ću je predati  u posjed - zemlju kojom teče mlijeko i med. Ja sam Jahve, vaš Bog, koji sam vas odvojio od tih naroda. 
\par 25 Pravite, dakle, razliku između čiste životinje i nečiste;  između čiste ptice i nečiste. Nemojte sami sebe opoganjivati  ni životinjom, ni pticom, ni bilo čim što zemljom puže: što sam  vam ja odlučio kao nečisto. 
\par 26 Budite mi dakle sveti, jer sam ja, Jahve, svet; ja sam  vas odvojio od tih naroda da budete moji. 
\par 27 Čovjek ili žena koji među vama postanu zazivači duhova  ili vračari neka se kazne smrću; neka se kamenuju i neka njihova  krv padne na njih." 


\chapter{21}

\par 1 Jahve još reče Mojsiju: "Govori svećenicima, Aronovim sinovima, i reci im: Neka se nitko ne okalja dodirom pokojnika u svome  narodu, 
\par 2 osim svoje najbliže rodbine: svoje majke, svoga oca, svoga sina, svoje kćeri i svojega brata. 
\par 3 I svojom sestrom, djevicom, koja mu je također najbliža, jer nije bila udata,  može se okaljati. 
\par 4 Ali neka se ne okalja svojom svojtom i tako  se oskvrne. 
\par 5 Neka ne briju glave; neka ne šišaju okrajke svojih brada  niti prave ureze na svome tijelu. 
\par 6 Neka budu posvećeni svome  Bogu; neka ne oskvrnjuju ime svoga Boga, jer oni prinose žrtve  u čast Jahvi paljene, hranu Boga svoga. Zato moraju biti sveti. 
\par 7 Neka se ne žene javnom bludnicom i obeščašćenom ženom;  niti se smiju ženiti onom koju je njezin muž otpustio. Jer je  svećenik posvećen svome Bogu. 
\par 8 Svetim ga drži, jer on prinosi hranu tvoga Boga. Neka  ti je svet, jer sam svet ja, Jahve, koji vas posvećujem. 
\par 9 Ako se kći kojeg svećenika oskvrne podavši se javnom bludništvu, ona oca svoga skvrne, pa se mora na vatri spaliti." 
\par 10 "A svećenik koji je najveći među svojom braćom, na čiju  je glavu bilo izliveno ulje pomazanja i koji je posvećen da nosi  svetu odjeću, neka ne ide raščupane kose niti razdire svoje odjeće. 
\par 11 Neka ne ulazi nijednom mrtvacu; ne smije se okaljati ni za  svojim ocem ni za svojom majkom. 
\par 12 Neka ne izlazi iz Svetišta, tako da ne oskvrne Svetište svoga Boga, jer na sebi nosi posvećenje  uljem pomazanja Boga svoga. Ja sam Jahve! 
\par 13 Neka za ženu uzme djevicu. 
\par 14 Udovicom, otpuštenicom, obeščašćenom i bludnicom ne smije se ženiti. Jedino djevicom  između svoga naroda neka se ženi; 
\par 15 tako neće oskvrnuti svoga  potomstva među svojim narodom, jer ja, Jahve, njega posvećujem." 
\par 16 Jahve reče Mojsiju: 
\par 17 "Reci Aronu: 'Nitko od tvojih potomaka, za njihovih naraštaja, koji imadne kakvu tjelesnu manu ne smije se primaknuti da prinosi  hranu svoga Boga. 
\par 18 Ni jedan na kome bude mane ne smije se  primaknuti: nitko koji je slijep ili sakat; nitko izobličen ili  iznakažena kojeg uda; 
\par 19 nitko tko ima slomljenu nogu ili ruku; 
\par 20 ni poguren, ni kržljav, ni bolesnih očiju, ni lišajav, ni  krastav, niti uškopljenik. 
\par 21 Dakle, ni jedan od potomaka svećenika  Arona koji imadne manu neka se ne primiče da prinosi u čast Jahvi  paljenu žrtvu; budući da ima manu, neka se ne primiče da prinosi  hranu svoga Boga. 
\par 22 Može blagovati hranu svoga Boga i od žrtava  presvetih, i svetih, 
\par 23 ali neka ne dolazi k zavjesi niti se  žrtveniku primiče jer ima manu. Neka ne skvrne mojih svetih stvari, jer sam ih ja, Jahve, posvetio.'" 
\par 24 Mojsije to kaza Aronu, njegovim sinovima i svim Izraelcima. 


\chapter{22}

\par 1 Jahve reče Mojsiju: 
\par 2 "Reci Aronu i njegovim sinovima da  sveto postupaju sa svetim prinosima Izraelaca; neka ne oskvrnjuju  moje sveto ime koje oni - ta moje je! - moraju svetiti. Ja sam  Jahve! 
\par 3 Reci im: 'Ako se ikad tko od vaših naraštaja primakne  u stanju nečistoće k svetim prinosima što ih Izraelci posvećuju  Jahvi, taj će biti uklonjen od moje nazočnosti. Ja sam Jahve!' 
\par 4 Neka nitko od Aronovih potomaka koji bude gubav ili imadne  izljev ne blaguje svetih prinosa dok ne postane čist. Onaj koji  se dotakne bilo čega što je mrtvo tijelo okaljalo ili onaj koji  iz sebe prospe sjemeni izljev; 
\par 5 onaj koji se dotakne kakva  puzavca koji ga onečisti; ili čovjeka od kojega se okalja bilo  kakvom nečistoćom - 
\par 6 onaj koji se dotakne čega takva neka je  nečist do večeri i neka ne blaguje svetih prinosa dok ne okupa  svoje tijelo u vodi. 
\par 7 Čim sunce zađe, čist je. Poslije toga  može blagovati od svetih prinosa jer mu je to hrana. 
\par 8 Neka ne jede ni strva ni što je zvjerad rastrgla. Time  bi se okaljao. 
\par 9 Neka drže moje naredbe, da ne navuku na se krivnju i zbog  nje, oskvrnuvši se, ne poginu. Ta ja, Jahve, njih posvećujem." 
\par 10 "Neka nijedan svjetovnjak ne blaguje od prinosa; ni ukućanin  ni svećenikov sluga ne smije jesti od svetoga prinosa. 
\par 11 Ali  ako svećenik steče koga novcem u svoje vlasništvo, taj to može  jesti kao onaj što se rodi u njegovoj kući; oni mogu jesti od  njegove hrane. 
\par 12 Ako se svećenikova kći uda za svjetovnjaka, ne smije blagovati od podizanih svetih prinosa. 
\par 13 Ali ako  svećenikova kći obudovi ili bude otpuštena, a nema djece pa se  vrati u očevu kuću, može se hraniti očevom hranom kao u svojoj  mladosti. Nikakav svjetovnjak ne smije što od toga jesti. 
\par 14 Bude  li tko iz neznanja jeo sveti prinos, neka ga nadoknadi svećeniku  dodajući petinu. 
\par 15 Neka ne oskvrnjuju svetih prinosa što ih Izraelci Jahvi  podižu. 
\par 16 Jedući ih, navukli bi na se krivnju koja bi ih obvezivala  na nadoknadu, jer ja, Jahve, posvetio sam te prinose." 
\par 17 Jahve reče Mojsiju: 
\par 18 "Govori Aronu, njegovim sinovima  i svim Izraelcima i reci im: 'Svaki čovjek doma Izraelova ili stranac u Izraelu koji donosi  svoj prinos kao zavjet ili kao dragovoljan dar da se prinese  Jahvi kao žrtva paljenica - da bude primljen - 
\par 19 mora prinijeti  muško bez mane, bilo to goveče, ovca ili koza. 
\par 20 Nikakvo s  manom na njemu nemojte prinositi jer vam to neće biti primljeno. 
\par 21 Ako tko prinosi Jahvi žrtvu pričesnicu da izvrši kakav  zavjet ili učini dragovoljan prinos, bilo od krupne ili sitne  stoke, ta životinja, da bude primljena, mora biti bez mane; nikakve  mane na njoj ne smije biti. 
\par 22 Nikakvu slijepu, ili hromu, ili  osakaćenu, gušavu, šugavu ili krastavu životinju, nikakvu takvu  Jahvi nemoj prinositi niti ikakvu takvu na žrtvenik kao paljenu  žrtvu Jahvi polagati. 
\par 23 Junca ili ovcu s kakvim udom protegnutim  ili prikaćenim možeš prinijeti kao dragovoljan prinos, ali kao  žrtva zavjetnica neće biti primljena. 
\par 24 Jahvi nemojte prinositi  životinje sa zgnječenim, stučenim, rastrgnutim ili odsječenim  mošnjama. To u svoj zemlji ne činite 
\par 25 niti takvo što primajte  od stranca da to prinesete kao hranu svoga Boga. S manom su jer  su osakaćene. Zato vam neće biti primljene.'" 
\par 26 Jahve reče Mojsiju: 
\par 27 "Kad se tele oteli, janje se ojanji ili se kozle okozi, sedam dana neka ostane uza svoju majku. Od osmoga dana može  biti primljeno kao paljena žrtva Jahvi. 
\par 28 Ne koljite krave  ni ovce u isti dan s njezinim mladim. 
\par 29 Kad Jahvi žrtvujete žrtvu zahvalnicu, žrtvujte je tako  da budete primljeni. 
\par 30 Neka se žrtva blaguje onoga istog dana;  od nje ništa ne ostavljajte za ujutro. Ja sam Jahve!" 
\par 31 "Moje zapovijedi držite i vršite ih. Ja sam Jahve! 
\par 32 Ne  oskvrnjujte moga svetog imena, nego neka budem proglašen svetim  među Izraelcima - ja, Jahve, koji vas posvećujem. 
\par 33 Ja koji  sam vas izbavio iz zemlje egipatske da budem vaš Bog, ja, Jahve." 


\chapter{23}

\par 1 Jahve reče Mojsiju: 
\par 2 "Kaži Izraelcima i reci im: Blagdani  Jahvini koje imate sazivati jesu sveti zborovi. Ovo su moji blagdani: 
\par 3 Šest dana neka se posao obavlja, a sedmi je dan subota  - dan potpunog odmora, dan svetoga zbora, kad ne smijete raditi  nikakva posla. Gdje god boravili, subota je Jahvina." 
\par 4 "A ovo su blagdani Jahvini - sveti zborovi - koje imate  proglasiti u njihovo određeno vrijeme: 
\par 5 U prvom mjesecu četrnaestoga dana u suton jest Pasha u  čast Jahvi; 
\par 6 petnaestoga dana toga mjeseca jest Blagdan beskvasnih  kruhova u čast Jahvi - sedam dana jedite beskvasan kruh. 
\par 7 Prvoga  dana neka vam bude sveti zbor; nikakva težačkog posla nemojte  raditi. 
\par 8 Sedam dana prinosite paljenu žrtvu u čast Jahvi, a  sedmoga dana neka opet bude sveti zbor; nikakva težačkog posla  ne radite." 
\par 9 Jahve reče Mojsiju: 
\par 10 "Kaži Izraelcima i reci im: 'Kad uđete u zemlju koju  vam dajem i u njoj žetvu požanjete, prvi snop svoje žetve donesite  svećeniku. 
\par 11 Neka ga on prinese kao žrtvu prikaznicu pred Jahvom  da budete primljeni. Sutradan po suboti neka ga svećenik prinese  kao žrtvu prikaznicu. 
\par 12 A u dan kad budete prinosili snop kao  žrtvu prikaznicu, prinesite Jahvi jednogodišnjeg janjca bez mane  kao žrtvu paljenicu. 
\par 13 Uz to žrtva prinosnica neka bude: dvije  desetine efe najboljeg brašna zamiješena u ulju, kao paljena  žrtva Jahvi na ugodan miris; a s njom ljevanica od vina neka  bude četvrt hina. 
\par 14 Prije toga dana - dok ne donesete prinose  svoga Boga - ne smijete jesti ni kruha, ni pržena zrnja, ni svježa  klasja. To je trajan zakon za vaše naraštaje gdje god vi boravili.'" 
\par 15 "A počevši od sutrašnjega dana po suboti - dana u koji  donesete snop za žrtvu prikaznicu - nabrojte punih sedam tjedana. 
\par 16 Onda na dan po sedmoj suboti, na Pedesetnicu, prinesite Jahvi  novu žrtvu. 
\par 17 Donesite iz svojih stanova po dva kruha za žrtvu  prikaznicu. Neka svaki bude od dvije desetine efe najboljeg brašna;  neka budu ispečeni ukvas, kao prvine Jahvi. 
\par 18 S kruhom prinesite  sedam jednogodišnjih janjaca bez mane, jednoga junca i dva ovna  kao žrtvu paljenicu Jahvi zajedno sa žrtvom prinosnicom i ljevanicom, žrtvom paljenom na ugodan miris Jahvi. 
\par 19 Prinesite i jednoga  jarca kao žrtvu okajnicu, a dva janjca od godine dana za žrtvu  pričesnicu. 
\par 20 Neka ih svećenik prinese pred Jahvom kao žrtvu  prikaznicu povrh kruha od prvina. Uz oba janjca, i ovo je Jahvi  sveto i neka pripadne svećeniku. 
\par 21 Toga istog dana sazovite  zbor. Neka vam to bude posvećen zbor - nikakva težačkog posla  ne radite. To je trajan zakon za vaše naraštaje gdje god vi boravili. 
\par 22 Kad budete želi žetvu sa svoje zemlje, nemoj žeti dokraja  svoje njive niti pabirčiti poslije svoje žetve. Ostavi to sirotinji  i strancu. Ja sam Jahve, Bog vaš." 
\par 23 Jahve reče Mojsiju: 
\par 24 "Govori Izraelcima i reci: 'Sedmoga  mjeseca, prvoga dana u mjesecu, neka vam je potpun odmor, proglašen  glasom trube, sveti zbor. 
\par 25 Nikakva teškog posla ne radite;  u čast Jahvi paljenu žrtvu prinesite.'" 
\par 26 Reče Jahve Mojsiju: 
\par 27 "Povrh toga, u deseti dan toga  sedmog mjeseca pada Dan pomirenja. Neka vam to bude prigoda za  sveti zbor; postite i prinesite u čast Jahvi paljenu žrtvu. 
\par 28 Toga  dana nemojte raditi nikakva posla. To je, naime, Dan pomirenja, kada će se za vas obaviti obred pomirenja pred Jahvom, Bogom  vašim. 
\par 29 Jest, tko god ne bude postio toga dana, neka se odstrani  iz svoga naroda. 
\par 30 A tko bi god radio kakav posao na taj dan, toga ću ja istrijebiti iz njegova naroda. 
\par 31 Nikakva posla  nemojte raditi. To je trajan zakon za vaše naraštaje gdje god  vi boravili. 
\par 32 Neka vam je to subotnji počinak. Postite! Navečer  devetoga dana u mjesecu - od večeri do večeri - prestanite raditi." 
\par 33 Jahve reče Mojsiju: 
\par 34 "Reci Izraelcima: 'Od petnaestoga dana toga sedmog mjeseca  neka se sedam dana drži Blagdan sjenica u čast Jahvi. 
\par 35 Prvoga  dana, u dan svetoga zbora, nikakva težačkog posla nemojte raditi. 
\par 36 Sedam dana prinosite paljenu žrtvu u čast Jahvi. Osmi dan  neka vam bude sveti zbor, kada ćete u čast Jahvi prinijeti paljenu  žrtvu. To je svečani zbor; nikakva težačkog posla nemojte obavljati.'" 
\par 37 "To su blagdani Jahvini koje imate sazvati - sveti zborovi  određeni za prinošenje žrtava u čast Jahvi; žrtava paljenica, prinosnica, žrtava klanica i ljevanica; svaku na njezin pravi  dan, 
\par 38 povrh Jahvinih subota, povrh vlastitih prinosa, povrh  svojih zavjetnih i dragovoljnih darova koje inače prinosite Jahvi." 
\par 39 "Osim toga, petnaestoga dana mjeseca sedmoga, pošto pokupite  sa zemlje plodove, svetkujte Jahvin blagdan sedam dana. Na prvi  dan i na osmi dan neka je potpun počinak. 
\par 40 Uzmite već prvoga  dana lijepih plodova, palmovih grana, grančica s lisnatih drveta  i potočne vrbovine pa se veselite u nazočnosti Jahve, Boga svoga, sedam dana. 
\par 41 Svetkujte tako blagdan u čast Jahvi sedam dana  svake godine. Neka je to trajan zakon za vaše naraštaje. Svetkujte  taj blagdan sedmoga mjeseca. 
\par 42 Sedam dana stanujte u sjenicama.  Svi Izraelovi domoroci neka proborave u sjenicama, 
\par 43 da vaši  potomci znaju kako sam ja učinio da Izraelci žive u sjenicama  kad sam ih izbavio iz zemlje egipatske. Ja sam Jahve, Bog vaš." 
\par 44 I tako Mojsije objavi Izraelcima Jahvine blagdane. 


\chapter{24}

\par 1 Jahve reče Mojsiju: 
\par 2 "Naredi Izraelcima da ti za svijećnjak donose čistoga ulja  od istupanih maslina, da se uvijek održava svjetlo. 
\par 3 Neka ga  Aron svagda sprema pred Jahvom od večeri do jutra u Šatoru sastanka, pred zavjesom Svjedočanstva. Neka je ovo trajan zakon vašim  naraštajima. 
\par 4 Neka Aron neprekidno održava svjetlila na čistome  svijećnjaku pred Jahvom." 
\par 5 "Potom uzmi najboljeg brašna i od njega ispeci dvanaest  pogača. Neka u svakoj pogači budu dvije desetine efe. 
\par 6 Onda  ih poredaj u dva reda - po šest u redu - na čistome stolu što  je pred Jahvom. 
\par 7 Na svaki red stavi čistoga tamjana. Neka to  bude hrana prinesena kao spomen - paljena žrtva Jahvi. 
\par 8 Svake  subote, bez prijekida, neka se postavljaju pred Jahvu. To neka  Izraelci vrše zbog vječnoga Saveza. 
\par 9 Neka pripadnu Aronu i  njegovim sinovima. Oni ih imaju blagovati na posvećenu mjestu.  To je njemu vrlo svet dio Jahvinih paljenih žrtava. To neka bude  trajna odredba." 
\par 10 A sin jedne Izraelke, komu otac bijaše Egipćanin, iziđe  među Izraelce i zametne u taboru svađu s nekim Izraelcem. 
\par 11 Uto  sin Izraelke pogrdi Ime i opsuje ga. Tada ga dovedu Mojsiju.  - Mati mu se zvala Šelomit, a bila je kći Dibrijeva iz plemena  Danova. - 
\par 12 Stave ga u zatvor dok im se ne očituje volja Jahvina. 
\par 13 Onda Jahve reče Mojsiju: 
\par 14 "Izvedi psovača iz tabora.  Potom svi oni koji su ga čuli neka stave svoje ruke na njegovu  glavu. A onda neka ga sva zajednica kamenuje. 
\par 15 Poslije toga ćeš ovako prozboriti Izraelcima: Tko god opsuje Boga svoga neka snosi svoju krivnju; 
\par 16 tko  izgovori hulu na ime Jahvino neka se smakne - neka ga sva zajednica  kamenuje; bio stranac ili domorodac, ako pohuli ime Jahvino,  mora mrijeti. 
\par 17 Ako čovjek zada smrtan udarac drugome, mora se smaknuti. 
\par 18 Tko usmrti živinče mora ga nadomjestiti: život za život. 
\par 19 Tko ozlijedi svoga bližnjega neka mu se učini kako je  on učinio: 
\par 20 lom za lom, oko za oko, zub za zub - rana koju  je on zadao drugome neka se zada i njemu. 
\par 21 Tko usmrti živinče mora ga nadoknaditi, ali tko ubije  čovjeka mora umrijeti. 
\par 22 Neka vam je jednak sud i strancu i  domorocu. Jer ja sam Jahve, Bog vaš." 
\par 23 Pošto je Mojsije to izložio Izraelcima, oni izvedu psovača  izvan tabora i zaspu ga kamenjem. Učine, dakle, Izraelci kako  je Jahve Mojsiju naredio. 


\chapter{25}

\par 1 Jahve reče Mojsiju na Sinajskom brdu: 
\par 2 "Govori Izraelcima  i kaži im: Kad uđete u zemlju koju vam dajem, neka ta zemlja održava  Jahvin subotni počinak. 
\par 3 Šest godina zasijavaj svoju njivu, šest godina svoj vinograd obrezuj i beri njegov plod. 
\par 4 Ali  sedme godine neka i zemlja uživa subotnji počinak, Jahvinu subotu:  svoje njive ne zasijavaj niti obrezuj svoga vinograda. 
\par 5 Što  samo od sebe uzraste na tvojoj njivi nemoj žeti niti beri grožđe  s neobrezane loze. Neka to bude zemlji godina počivanja. 
\par 6 Zemljišni  počinak neka vam priskrbi prehranu: tebi, tvome sluzi, tvojoj  sluškinji, tvome najamniku koji s tobom živi; 
\par 7 a i tvojoj stoci  i zvjeradi u tvojoj zemlji neka njezini plodovi služe za hranu." 
\par 8 "Nabroj sedam sedmica takvih godina, sedam puta sedam  godina. Sedam sedmica godina iznosit će ti četrdeset devet godina. 
\par 9 A onda zaori u trubu! U sedmome mjesecu, desetoga dana toga  mjeseca, na Dan pomirenja, zatrubite u trubu širom svoje zemlje. 
\par 10 Tu pedesetu godinu proglasite svetom! Zemljom proglasite  oslobađanje svim njezinim stanovnicima. To neka vam bude jubilej, oprosna godina. Neka se svatko vaš vrati na svoju očevinu; neka  se svatko vrati k svome rodu! 
\par 11 Ta pedesetogodišnjica neka  vam je jubilejska godina: nemojte sijati, nemojte žeti što samo  od sebe uzraste niti berite grožđe s neobrezane loze. 
\par 12 Jer  jubilej vam mora biti svet! Hranite se onim što njiva donese  od sebe. 
\par 13 Te jubilejske godine neka se svatko vrati na svoju očevinu. 
\par 14 Zato, kad prodajete imanje svome sunarodnjaku ili kupujete  od svoga sunarodnjaka, nemojte nanositi štete svome bratu! 
\par 15 Od  svoga sunarodnjaka kupuj, odbivši samo broj godina poslije jubileja, a on neka ti proda prema broju proizvodnih godina. 
\par 16 Što više  godina, više i cijenu povisi; što manje godina, neka je i cijena  manja. Jer, ono što ti on prodaje jest broj ljetina. 
\par 17 Neka  nitko od vas ne nanosi štete svome sunarodnjaku, nego se boj  Boga svoga! Jer ja sam Jahve, Bog vaš. 
\par 18 Vršite moje zakone i moje naredbe; vjerno ih provodite  u djelo pa ćete u sigurnosti živjeti na zemlji. 
\par 19 Zemlja će  davati svoj rod, jest ćete do sitosti i živjet ćete u sigurnosti. 
\par 20 Ako biste rekli: 'Čime ćemo se hraniti te sedme godine kad  ne budemo ni sijali ni brali plodova?' 
\par 21 evo, blagoslov ću  svoj pustiti na vas: šesta godina rodom će roditi za tri godine. 
\par 22 Kad budete sijali osme godine, hranit ćete se starim prihodom  sve do devete godine; dok ne dođe njezin prihod, jest ćete stari." 
\par 23 "Zemlja se ne smije prodati potpuno, jer zemlja pripada  meni, dok ste vi samo stranci i gosti kod mene. 
\par 24 Zato u svakome  kraju gdje imate zemljišne posjede morate dopustiti otkupljivanje  zemlje. 
\par 25 Ako tvoj brat zapadne u škripac te moradne prodati dio  svoje očevine, neka dođe njegov najbliži izbavitelj i otkupi  što je njegov brat prodao. 
\par 26 Ako nema koga da mu ga otkupi, a poslije i sam postane imućan te stekne sredstva da je otkupi, 
\par 27 neka prebroji godine od prodaje, isplati kupcu svotu za  preostalo vrijeme i vrati se na svoju očevinu. 
\par 28 Ako nema sredstava  da je vrati, onda prodano neka ostane u rukama kupca do jubilejske  godine. A stupivši u jubilej, neka se vrati na svoju očevinu. 
\par 29 Ako tko proda stojnu kuću u gradu zidom obzidanu, može  je otkupiti dokle se ne navrši godina poslije prodaje; otkupni  rok neka je, dakle, jedna godina. 
\par 30 Ako je ne otkupi u roku  od godine, onda kuća u gradu zidom opasana prelazi potpuno kupcu  i njegovim potomcima: ni za jubileja neka se ne vraća. 
\par 31 Ali  kuće po selima što nemaju zidova oko sebe neka se smatraju kao  posjedi u polju; mogu se otkupljivati. U jubileju kupac mora  iz njih izići. 
\par 32 Kuće koje u levitskim gradovima pripadaju  levitima mogu leviti otkupiti u svako vrijeme. 
\par 33 Ako se koji  levit ne posluži svojim pravom otkupa, onda će kuća što bude  prodana u gradu njegova vlasništva biti za jubileja vraćena.  Jer kod Izraelaca kuće u gradovima levita njihovo su vlasništvo. 
\par 34 Neograđena zemlja oko njihovih gradova ne može se prodati, jer je ona njihovo vlasništvo za sva vremena." 
\par 35 "Ako tvoj brat zapadne u škripac i ne mogne održavati  svoje odnose s tobom, primi ga; i neka s tobom živi kao stranac  ili gost. 
\par 36 Ne uzimaj od njega ni lihve ni kamata. Boga se  svoga boj, i neka tvoj brat živi s tobom! 
\par 37 Ne uzajmljuj mu  novac na kamate niti mu lihvarski davaj svoju hranu. 
\par 38 Ja,  Jahve, Bog vaš, izbavio sam vas iz zemlje egipatske da vam dadem  zemlju kanaansku i budem vaš Bog. 
\par 39 Ako li tvoj brat padne u škripac dok je s tobom u urednim  odnosima te se moradne tebi prodati, nemoj ga prisiliti da služi  kao rob; neka bude kod tebe kao najamnik ili nadničar. 
\par 40 Neka  služi kod tebe do jubilejske godine. 
\par 41 Onda neka bude slobodan  da ode od tebe - i on i njegova djeca s njim; neka ide natrag  svome rodu i opet zaposjedne svoju djedovinu. 
\par 42 TÓa oni su  moji službenici, ja sam ih izbavio iz zemlje egipatske; oni se  ne smiju prodavati kao robovi. 
\par 43 Nemoj s njim grubo postupati!  Boga se svoga boj! 
\par 44 A robove i ropkinje, budeš li ih htio  imati, možete kupiti, i muške i ženske, od naroda koji su oko  vas. 
\par 45 Možete ih kupovati i od pridošlica koji s vama borave;  od njihovih obitelji što žive s vama i rođeni su u vašoj zemlji.  Takvi mogu postati vašim vlasništvom. 
\par 46 Njih možete predati  u nasljedstvo svojoj djeci da ih zavazda naslijede u baštinu.  Prema njima možete postupati kao prema robovima. Ali prema svojoj  braći, Izraelcima, nitko ne smije grubo postupati. 
\par 47 Ako se stranac s tobom nastanjen obogati, a tvoj brat, u svojim odnosima prema njemu, zapadne u škripac te se proda  strancu koji je s tobom nastanjen ili kojemu god potomku strančeve  obitelji, 
\par 48 on ima pravo i nakon prodaje biti otkupljen. Neka  ga otkupi netko od njegove braće; 
\par 49 ili neka ga otkupi njegov  stric, njegov rođak ili bilo tko od njegove obitelji koji bude  od njegove krvi. Ili, ako ima sredstava, neka se sam otkupi. 
\par 50 Sa svojim kupcem neka proračuna vrijeme od godine kad mu  se prodao do jubilejske godine. Cijena za njegovo oslobođenje  neka bude prema broju godina. Vrijeme što ga je proveo sa svojim  vlasnikom neka se procijeni kao vrijeme jednog najamnika. 
\par 51 Ako  ostaje još mnogo godina, neka isplati za svoju otkupninu u omjeru  svoje prodajne svote. 
\par 52 A ako ostaje samo nekoliko godina do  jubilejske godine, neka izračuna pa isplati za svoj otkup prema  godinama službe. 
\par 53 Prema njemu neka bude kao prema najamniku  koji se iznajmljuje od godine na godinu. Neka se na tvoje oči  s njim ne postupa grubo. 
\par 54 Ne bude li iskupljen ovako, onda  i on i njegova djeca s njim neka odu u jubilejskoj godini. 
\par 55 Jer  Izraelci su moji službenici; oni su moji službenici koje sam  ja izveo iz zemlje egipatske, ja, Jahve, Bog vaš." 


\chapter{26}

\par 1 "Ne pravite sebi kumira; ne podižite sebi ni kipa ni spomen-stupa;  ne postavljajte u svojoj zemlji kamenja s likovima da pred njih  padate. 
\par 2 Održavajte moje subote; poštujte moje Svetište - jer  ja sam Jahve, Bog vaš." 
\par 3 "Budete li živjeli prema mojim zakonima, održavali moje  zapovijedi i u djelo ih provodili, 
\par 4 davat ću vam kiše u pravo  vrijeme te će zemlja rađati rodom a stabla po polju donositi  plodove. 
\par 5 Vršidba će vam stizati berbu, a berba stizati sjetvu.  Jest ćete kruh svoj do sitosti i u svojoj ćete zemlji živjeti  u sigurnosti. 
\par 6 Zemlji ću dati mir; tako ćete počivati a da vas nitko  ne plaši. Štetne ću životinje iz zemlje ukloniti; mač neće prolaziti  vašom zemljom. 
\par 7 U bijeg ćete nagoniti svoje neprijatelje, a  oni će padati pred vama od mača. 
\par 8 Petorica vas nagonit će u  bijeg stotinu njih, a stotina vas nagonit će u bijeg deset tisuća  njih. Da, vaši će neprijatelji padati pred vama od mača. 
\par 9 K vama ću se okrenuti te vas rodnima činiti i razmnažati.  Držat ću svoj Savez s vama. 
\par 10 Starom ćete se zalihom hraniti; štoviše, trebat će vam  zalihe ispražnjavati da mognete sasipati novo žito. 
\par 11 Među vama ću postaviti svoje Prebivalište i neću vas  odbaciti; 
\par 12 među vama ću hoditi i bit ću vam Bog, a vi ćete  mi biti narod. 
\par 13 Ja, Jahve, Bog vaš, izveo sam vas iz zemlje  egipatske da im više ne budete roblje; polomio sam palice vaših  jarmova i učinio da hodate uspravno." 
\par 14 "Ali ako me ne poslušate i u djelo ne provedete sve ove  moje zapovijedi; 
\par 15 ako odbacite moje zakone, pogazite moje  naredbe i prekršite moj Savez, ne provodeći u djelo sve moje  zapovijedi, 
\par 16 evo što ću ja učiniti vama: podvrgnut ću vas  strepnji, iznemoglosti i groznici što oči troše a život gase. 
\par 17 Sjetve ćete svoje uzalud sijati - neprijatelji vaši njima  će se hraniti. Ja ću se protiv vas okrenuti, a vaši će vas neprijatelji  ametice tući. Oni koji vas mrze gospodarit će nad vama. Bježat  ćete i onda kad vas nitko ne bude progonio. 
\par 18 Pa ako me i unatoč tome ne poslušate, ja ću vas sedmerostruko  kažnjavati za vaše grijehe. 
\par 19 Slomit ću ja vašu drsku silu.  Vaša ću nebesa učiniti poput gvožđa, a zemlju vašu poput tuča. 
\par 20 Uzalud će se trošiti vaša snaga. Zemlja vam više neće davati  svoga roda niti će stabla na zemlji donositi svojih plodova. 
\par 21 Budete li se još i dalje protivili, ne htjednete li me  poslušati, sedmerostruko ću još na vama povisiti rane za vaše  grijehe. 
\par 22 Na vas ću pustiti šumsku zvjerad da vas liši djece, blago vam podavi a vas prorijedi tako da vam putovi postanu  pusti. 
\par 23 Ako vas ni to ne popravi nego nastavite življenje koje  se meni protivi, 
\par 24 onda ću se i ja suprotstaviti vama i sam  ću vas još sedmerostruko udariti za vaše grijehe. 
\par 25 Na vas  ću dovesti mač neka se iskali osvetom za Savez. A kad se zbijete  u svoje gradove, poslat ću na vas kugu i bit ćete predani u ruke  neprijatelju. 
\par 26 Još kad vam obustavim namicanje kruha, deset  žena moći će vam peći kruh u jednoj peći i na mjeru će vam kruh  davati. Jest ćete, ali se nećete nasititi. 
\par 27 Ako me ni tada ne poslušate nego mi se dalje budete suprotstavljali, 
\par 28 i ja ću se vama suprotstaviti - sedmerostruko ću vas kazniti  za vaše grijehe. 
\par 29 Jest ćete meso od svojih sinova, jest ćete  meso od svojih kćeri. 
\par 30 Porušit ću vaše idolske uzvišice; oborit  ću vaše kadione žrtvenike, zgrnut ću vaša mrtva trupla na trupla  vaših kumira i odbacit ću vas. 
\par 31 Gradove ću vaše pretvoriti  u ruševine; svetišta ću vaša opustošiti, vaš ugodni miris neću  više mirisati. 
\par 32 Zemlju ću ja pretvoriti u zgarište tako da  će se vaši neprijatelji koji se u njoj nastane zaprepastiti nad  njom. 
\par 33 Vas ću rasijati po narodima; izvući ću protiv vas mač  iz korica tako da će vam se zemlja pretvoriti u pustaru a gradovi  u ruševine. 
\par 34 Tada će zemlja namiriti svoje subote za sve vrijeme dok  bude pusta i vi budete u zemlji svojih neprijatelja. Otpočinut  će tada zemlja i moći će namiriti svoje subote. 
\par 35 Sve dok bude  pusta, imat će počinak koji nije imala za vaših subota dok ste  vi u njoj stanovali. 
\par 36 A onima od vas koji na životu ostanu  po zemljama svojih neprijatelja, njima ću strah u srce utjerati.  U bijeg će ih nagoniti šuštaj lista što zatrepti. Bježat će kao  što se bježi od mača; padat će, iako ih nitko neće progoniti. 
\par 37 Spoticat će se jedan o drugoga kao kad se bježi ispred mača, premda ih nitko neće progoniti. Nećete se održati pred svojim  neprijateljima; 
\par 38 izginut ćete među narodima - proždrijet će  vas zemlja vaših neprijatelja. 
\par 39 A koji od vas prežive venut će u zemljama svojih neprijatelja  zbog svojih opačina; venut će i zbog opačina svojih otaca. 
\par 40 Priznat  će tada svoju opačinu i opačinu svojih otaca što su je protiv  mene počinili izdajom, što su mi se protivili. 
\par 41 I ja sam sa morao suprotstaviti njima i odvesti ih u  zemlju njihovih neprijatelja." "Onda će se napokon njihovo tvrdokorno srce poniziti; ispaštat  će oni svoju krivnju. 
\par 42 Tada ću se ja sjetiti svoga Saveza s Jakovom i svoga  Saveza s Izakom; sjetit ću se svoga Saveza s Abrahamom - zemlje  ću se sjetiti. 
\par 43 Zemlja će, ostavljena od njih, namiriti svoje subote  kad ostane pusta zbog njih. A oni će ispaštati svoju krivnju  što su odbacili moje zapovijedi; što su prezreli moje zakone. 
\par 44 Ali ni onda dok budu u zemlji svojih neprijatelja, neću ih  zabaciti niti ću ih prezreti tako da ih posve uništim i da prekršim  svoj Savez s njima. TÓa ja sam Jahve, Bog njihov. 
\par 45 Radi njih  sjetit ću se Saveza s njihovim precima koje sam izveo iz zemlje  egipatske naočigled naroda da budem njihov Bog, ja Jahve." 
\par 46 To su odredbe, uredbe i zakoni koje je Jahve uglavio  između sebe i Izraelaca po Mojsiju na Sinajskome brdu. 



\chapter{27}

\par 1 Jahve reče Mojsiju: 
\par 2 "Govori Izraelcima i reci im: 'Ako  tko zaželi podmiriti Jahvi zavjet što vrijedi koliko čovjek, 
\par 3 neka ti je mjerilo: muškarca od dvadeset do šezdeset godina  starosti procijeni pedeset šekela u srebru, prema hramskom šekelu, 
\par 4 a žensku procijeni trideset šekela. 
\par 5 A za dob od pet do  dvadeset godina neka tvoja procjena bude: za muškarca dvadeset  šekela, a za žensku deset šekela. 
\par 6 Je li dob od jednoga mjeseca  do pet godina, neka ti je procjena: za muško pet šekela u srebru, a procjena za žensko tri šekela u srebru. 
\par 7 Bude li u starosti  od šezdeset godina ili više, neka ti je procjena: za muškarca  petnaest šekela, a za žensku deset šekela. 
\par 8 Ali ako je tko  siromašan te ne može platiti svoju cijenu, neka ga dovedu pred  svećenika i neka ga svećenik procijeni. Ali neka svećenik procijeni  prema onome što zavjetovalac može dati. 
\par 9 Ako zavjetovani prinos bude od životinja koje se mogu  Jahvi prinositi, svaki takav prinos Jahvi bit će posvećena stvar. 
\par 10 Neka se ne nadomješta niti zamjenjuje za što drugo - bilo  dobro za loše, bilo loše za dobro. Ako li se napravi zamjena  jednoga živinčeta za drugo, onda će i zavjetovano i ono koje  ga je zamijenilo biti posvećena stvar. 
\par 11 Bude li zavjetovani  prinos od nečiste životinje koja se ne može Jahvi prinositi,  neka se takvo živinče dovede k svećeniku 
\par 12 pa neka ga on procijeni.  Bilo skupo, bilo jeftino, kako svećenik procijeni, neka tako  bude. 
\par 13 Zaželi li ga tko otkupiti, neka doda njegovoj procjeni  jednu petinu. 
\par 14 Ako tko posveti svoju kuću zavjetovavši je Jahvi, neka  svećenik procijeni da li je dobra ili loša. Kako svećenik prosudi, neka tako ostane. 
\par 15 Ako onaj koji je svoju kuću zavjetovao  zaželi da je otkupi, neka dometne jednu petinu svoti na koju  je procijenjena pa neka bude njegova. 
\par 16 Ako tko zavjetuje Jahvi dio zemljišta od svoga vlasništva, procijeni ga prema njegovu usjevu: za jedan homer ječmena sjemena  pedeset šekela u srebru. 
\par 17 Zavjetuje li zemljište za jubilejske  godine, neka ostane prema ovoj procjeni. 
\par 18 Ali ako zemljište  zavjetuje poslije jubilejske godine, neka svećenik proračuna  cijenu prema godinama što preostaju do jubilejske godine i prema  tome smanji procjenu. 
\par 19 Ako onaj tko je zemljište zavjetovao  zaželi da ga otkupi, neka doda jednu petinu svoti na koju je  procijenjeno pa neka mu ostane. 
\par 20 Ako zemljište ne otkupi nego  ga proda drugome, ne može se više otkupiti. 
\par 21 Kad zemljište  bude oslobođeno u jubilejskoj godini, neka se posveti Jahvi kao  zavjetovano zemljište i postane svećenikov posjed. 
\par 22 Zavjetuje li tko Jahvi kupljeno zemljište koje nije dio  njegove očevine, 
\par 23 neka mu svećenik proračuna razmjernu procjenu  do jubilejske godine. I toga istog dana neka isplati iznos kao  stvar posvećenu Jahvi. 
\par 24 U jubilejskoj godini zemljište se ima vratiti onome od  koga je kupljeno - kome pripada zemljišno vlasništvo. 
\par 25 Svaka procjena neka se vrši prema hramskom šekelu: dvadeset  gera jedan šekel. 
\par 26 Ali neka nitko ne zavjetuje prvinu od stoke. TÓa prvina  ionako pripada Jahvi - Jahvina je, pa bila od sitnoga bila od  krupnoga blaga. 
\par 27 Bude li od nečiste stoke, može se otkupiti  prema procjeni, dometnuvši petinu cijene. Ako se ne otkupi, neka  se prema procjeni proda. 
\par 28 Ali ništa od 'herema', od onog što je Jahvi izručeno, bio to čovjek ili živinče ili njegovo baštinjeno zemljište,  ništa što je tko Jahvi zavjetom posvetio, ne može se niti prodati  niti otkupiti. Svaka zavjetom posvećena stvar najveća je Jahvina  svetinja. 
\par 29 Nijedno ljudsko biće koje bude 'heremom' - prokletstvom  - udareno ne smije se otkupljivati: mora se smaknuti. 
\par 30 Svaka desetina sa zemljišta, bilo od poljskih usjeva  bilo od plodova sa stabala, pripada Jahvi; to je Jahvi posvećeno. 
\par 31 Ako bi tko htio otkupiti koji dio svoje desetine, mora tome  dodati jednu petinu cijene. 
\par 32 Svaka desetina od krupnoga i  sitnoga blaga, to jest svako deseto od svega što prolazi ispod  pastirskog štapa, neka bude posvećeno Jahvi. 
\par 33 Neka se ne gleda  je li dobro ili rđavo; i neka se ne zamjenjuje. Ako se ipak zamijeni, neka je onda i jedno i drugo posvećeno i ne smije se otkupljivati.'" 
\par 34 To su zapovijedi koje je Jahve izdao Mojsiju za Izraelce  na Sinajskome brdu. 





\end{document}